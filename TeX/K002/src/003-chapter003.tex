
\chapter{ಚಿಂತೆಯ ಚಿತೆಯಿಂದ ಪಾರಾಗಿ}

\indentsecionsintoc

\begin{itemize}
\item \textbf{ಚಿಂತಾಕ್ರೋಭ} (ಎಂದರೆ ಚಿಂತಾಕ್ರೋಶಭಯೋದ್ವೇಗವು) ನಿಮ್ಮ ರಕ್ತಸಂಚಾರ\break ಜೀರ್ಣಾಂಗ ವ್ಯೂಹ ಹಾಗೂ ನರಮಂಡಲಗಳ ಮೇಲೆ ತನ್ನ ಮಾರಕ ಪ್ರಭಾವವನ್ನುಂಟು\-ಮಾಡಿ ನಿಮ್ಮ ಆರೋಗ್ಯವನ್ನು ಹದಗೆಡಿಸೀತು. ಕಠಿಣ ಪರಿಶ್ರಮದಿಂದ ಯಾರೂ ಸತ್ತುದನ್ನು ನಾನರಿಯೆ. ಆದರೆ ಚಿಂಕ್ರೋಭದಿಂದ ಕಂಗಾಲಾಗಿ ಸತ್ತುಹೋದವರು ಹಲವು ಮಂದಿ.\hfill–ಡಾ. ಚಾರ್ಲ್ಸ್ ಮೇಯೋ

 \item ಚಿತೆಯು ಸತ್ತ ಮನುಷ್ಯನನ್ನು ಸುಟ್ಟರೆ ಚಿಂತೆಯು ಜೀವಂತ ವ್ಯಕ್ತಿಯನ್ನು ಸುಡುತ್ತದೆ.

 \general{~\hfill}–ಸುಭಾಷಿತ

 \item ನೀನೇನು ತಿನ್ನುವಿಯೋ ಅದರಿಂದ ಹೊಟ್ಟೆಯ ಹುಣ್ಣು ಬರುವುದಿಲ್ಲ; ನಿನ್ನನ್ನೇನು ತಿನ್ನುತ್ತಿದೆಯೋ ಅದರಿಂದಲೇ ಹೊಟ್ಟೆಯಲ್ಲಿ ಹುಣ್ಣಾಗುವುದು–ತಿಳಿ.

 \general{~\hfill}–ಡಾ. ಜೋಸೆಫ್ ಎಫ್. ಮೊಂಟೇಗ್​

 \item ಕ್ರೋಧಾದ್ಭವತಿ ಸಂಮೋಹಃ ಸಂಮೋಹಾತ್ ಸ್ಮೃತಿ ವಿಭ್ರಮಃ~। ಸ್ಮೃತಿಭ್ರಂಶಾದ್ಬುದ್ಧಿ\-ನಾಶೋ ಬುದ್ಧಿನಾಶಾತ್ ಪ್ರಣಶ್ಯತಿ~॥~\hfill–ಭಗವದ್ಗೀತಾ

 \item ಭಯವು ಸುಪ್ತಮನಸ್ಸಿನಲ್ಲಿ ಊಹಾತೀತ ರೀತಿಯಲ್ಲಿ ತಳಮಳವನ್ನುಂಟುಮಾಡಿ, ತನು ಮನಗಳ ಸಮತೋಲನವನ್ನೇ ಹಾಳುಗೆಡವಿ ವ್ಯಕ್ತಿಯ ಸರ್ವನಾಶಕ್ಕೆ ಕಾರಣವಾಗುತ್ತದೆ. ವಿವೇಕಯುತ ಪ್ರಜ್ಞೆಯನ್ನು ಬೆಳೆಸಿಕೊಳ್ಳುವುದರಿಂದಷ್ಟೇ ಸುಪ್ತಮನಸ್ಸಿನಲ್ಲಿ ಬೇರೂರಿರುವ ಭಯದ ದುರ್ಬೀಜವನ್ನು ಕಿತ್ತೆಸೆಯಬಹುದೇ ಹೊರತು, ಬೇರಾವ ಮಾನಸಿಕ ಒತ್ತಡ\-ದಿಂದಲೂ ಖಂಡಿತ ಸಾಧ್ಯವಿಲ್ಲ.\hfill\hbox{–ಕ್ರಿಶ್ಚಿಯನ್ ಡಿ.\ ಲಾರ್ಸನ್​}

 \item \general{\enginline{Businessmen who do not know how to fight worry die young.}}

 \general{~\hfill}–Dr. Alexis Carrel

 \item \general{\enginline{To fear disease, failure or trouble is to sow seeds in the subconscious field that will bring forth a harvest of diseased conditions, troubled thoughts, confused mental states and misdirected actions in body and mind.}}

 To remove any wrong impression from the subconscious, the opposite correct impression must be made in its place. A wrong impression cannot be removed by mental force, resistance or denial; produce the right impression and the wrong imprssion will cease to exist.\general{~\hfill}–Christian D. Larson

 \item \enginline{The secret of being miserable is to have the leisure to bother about whether you are happy or not.\general{~\hfill}–Bemard Shaw}

 \item \enginline{True peace of mind comes from accepting the worst.\general{\hfill}–Lin Yutang}

 \item \enginline{The greatest mistake physicians make is that they attempt to cure the body without attempting to cure the mind. Yet the mind and body are one and should not be treated separately.\general{~\hfill}–Plato}

\end{itemize}


\section*{ಚಿಂತೆಗೊಂದು ಬೊಂತೆ ಕಟ್ಟಿ}

\vskip -6pt\addsectiontoTOC{ಚಿಂತೆಗೊಂದು ಬೊಂತೆ ಕಟ್ಟಿ}

ವೈಜ್ಞಾನಿಕಯುಗದ ವೇಗ ಅವಸರ ಗೊಂದಲಗಳ ಮಧ್ಯದಲ್ಲಿ ಬದುಕುವ ನಾವು, ರಾಜಕೀಯದ ರಾಗದ್ವೇಷಗಳ ಜಾಲದಲ್ಲಿ ಸಿಕ್ಕಿಕೊಂಡಿದ್ದೇವೆ. ಬದುಕಿಗಾಗಿ, ಧನಾರ್ಜನೆಗಾಗಿ ಕೊರಳಿರಿ ಯುವ ತೀವ್ರ ಸ್ಪರ್ಧೆಯ ಸುಳಿಯಲ್ಲಿ ಸುತ್ತುತ್ತಿದ್ದೇವೆ. ವಿಭಿನ್ನ ಜಾತಿ ಮತ ಪಥ ಪಂಥ ಪಕ್ಷ ಪಂಗಡಗ\-ಳೊಳಗಿನ ಗುಪ್ತ ವ್ಯಕ್ತ ವಿದ್ವೇಷ ಭಾವನೆಗಳ ಪ್ರವಾಹದಲ್ಲಿ ತೇಲುತ್ತಿದ್ದೇವೆ. ಮೌಲ್ಯಗಳ ಕುಸಿತ, ಸಂಕುಚಿತ ಸ್ವಾರ್ಥತೆ ಹಾಗೂ ನೈತಿಕ ಅಧಃಪತನಗಳನ್ನು ಸರ್ವತ್ರ ಕಾಣುತ್ತಿದ್ದೇವೆ. ಇಂಥ ಪರಿಸ್ಥಿತಿಯಲ್ಲಿ ಜನರು ತಮ್ಮ ಆಂತರಿಕಶಾಂತಿಯನ್ನೂ, ಸಮತೋಲವನ್ನೂ, ಚಿತ್ತ ಸ್ವಾಸ್ಥ್ಯವನ್ನೂ ಕಾಪಾಡಿಕೊಳ್ಳುವ ವಿಧಾನವನ್ನು ತಿಳಿಯದಿದ್ದರೆ ಚಿಂಕ್ರೋಭ ಪೀಡಿತರಾಗಿ ನರ ಮಂಡಲದ\break ದೌರ್ಬಲ್ಯ ಹಾಗೂ ನಾನಾ ತೆರನಾದ ಮಾನಸಿಕ ದೈಹಿಕ ವ್ಯಾಧಿಗ್ರಸ್ತರಾಗುವುದು ಖಂಡಿತ ಎಂಬುದು ತಜ್ಞರ ಪ್ರಯೋಗ ಪರಿಶೀಲನೆಗಳಿಂದ ವ್ಯಕ್ತವಾದ ಸತ್ಯ.

ದಿನದಿನವೂ, ಚಿಂಕ್ರೋಭವನ್ನು ಹನಿಹನಿಯಾಗಿ ಪುಟ್ಟಪುಟ್ಟ ಕಂತುಗಳಲ್ಲಿ ಸಂಗ್ರಹಿಸುತ್ತ ಗಂಡಾಂತರ ದಟ್ಟೈಸಿದಾಗ ಅಥವಾ ವ್ಯಾಧಿ ತೀವ್ರವಾದ ಮೇಲೆ ವೈದ್ಯರ ಬಳಿ ಓಡುವುದಕ್ಕಿಂತ ರೋಗವೇ ಬರದಂತೆ ಕೆಲವು ತಥ್ಯಗಳನ್ನು ತಿಳಿದುಕೊಂಡು ನಿಯಮ ಪಾಲಿಸುತ್ತಲಿರುವುದೇ\break ಉಚಿತವಲ್ಲವೇ? ಮನೆಗೆ ಬೆಂಕಿ ಬಿದ್ದ ಮೇಲೆ ಬಾವಿ ತೋಡ ಹೊರಟವನ ಕತೆಯಂತೆ ಆಗ ಬಾರದು ನಮ್ಮ ಪಾಡು! ಹಾಗಾಗಿಯೇ ಹೇಳುವುದು ನಿಮ್ಮ ಚಿಂತೆಗಿಂದೇ ಬೊಂತೆ ಕಟ್ಟಿ ಎಂದು.


\section*{ಚಿಂತೆಯ ಪಾಶ}

\vskip -6pt\addsectiontoTOC{ಚಿಂತೆಯ ಪಾಶ}

ಆತ್ಮೀಯನೊಬ್ಬ ತನ್ನ ಬದುಕಿನಲ್ಲಿ ಅನುಭವಿಸಿದ್ದ ವಿಚಿತ್ರ ಬೇನೆಯ ಬಗ್ಗೆ ಒಮ್ಮೆ ವಿವರಿಸಿದ್ದ. ಕೆಲವು ದಿನಗಳ ಹಿಂದೆ ಇದ್ದಕ್ಕಿದ್ದಂತೆ ಆತನಿಗೆ ತಲೆತಿರುಗು, ಜೊತೆಗೆ ವಾಂತಿ ಪ್ರಾರಂಭವಾಯಿತು. ಎದ್ದು ನಿಂತೊಡನೆಯೇ ತಲೆ ಸುತ್ತತೊಡಗಿ ವಾಂತಿಯಾಗುತ್ತಿತ್ತು. ಚಿಕಿತ್ಸೆ ನಡೆಯಿತಾದರೂ ಬೇನೆ ವಾಸಿಯಾಗಲಿಲ್ಲ. ಆದರೆ ಮುಂದೊಂದು ದಿನ ವೈದ್ಯರಿಗೆ ಸವಾಲಾಗಿದ್ದ ಆ ಕಾಯಿಲೆ ತನ್ನಿಂದ\break ತಾನಾಗಿಯೇ ದೂರವಾಯಿತೆಂದರೆ ಆಶ್ಚರ್ಯವೆನಿಸದೆ?

ಹೌದು, ಅದು ಚಿಂಕ್ರೋಭದ ಪೀಡೆ. ಆತನ ಗೆಳೆಯ ಅಪಘಾತಕ್ಕೀಡಾಗಿ ಆಸ್ಪತ್ರೆ ಸೇರಿದ ಸುದ್ದಿ ಕೇಳುತ್ತಲೇ ಆ ಬೇನೆ ಕಾಣಿಸಿಕೊಂಡಿತ್ತು. ಆತ ಪ್ರಾಣಾಪಾಯದಿಂದ ಪಾರಾದ ವಿಚಾರ ದೃಢವಾಗುತ್ತಲೇ ಆ ಬೇನೆ ನಿವಾರಣೆಯಾಗಿತ್ತು. ಅಂದರೆ ಈ ಸ್ನೇಹಿತರಲ್ಲಿ ಅಂಥ ಅನ್ಯೋನ್ಯ ಭಾವ ಸಂಬಂಧ ಎಂದು ಅರ್ಥವೇ? ಅಲ್ಲ. ಆ ಸ್ನೇಹಿತನ ಒತ್ತಾಯಕ್ಕೆ ಕಟ್ಟುಬಿದ್ದು ಆತ ಬ್ಯಾಂಕಿನಿಂದ ತೆಗೆದ ಭಾರೀ ಸಾಲಕ್ಕೆ ಈತ ಜಾಮೀನು ಹಾಕಿದ್ದ. ಆತ ಸಾವನ್ನಪ್ಪಿದರೆ ತನ್ನ ಗತಿ ಏನೆಂಬ ಆಘಾತವೇ ಅವನಲ್ಲಿ ತಲೆ ತಿರುಗು ಹುಟ್ಟಿಸಿ ವಾಂತಿ ಬರಿಸಿತ್ತು. ಸ್ನೇಹಿತನ ಚೇತರಿಕೆ ಇವನನ್ನು ಚೇತರಿಸುವಂತೆ ಮಾಡಿತಷ್ಟೇ. ಚಿಂತೆಯ ಪಾಶದ ವೈಖರಿ ನೋಡಿದಿರಾ?

ಆದಿಕವಿ ವಾಲ್ಮೀಕಿ ಎನ್ನುವಂತೆ, ‘ಕೆರಳಿದ ಸರ್ಪವು ಬಾಲಕನನ್ನು ಕಚ್ಚಿಕೊಲ್ಲುವಂತೆ ಚಿಂತೆಯು ಮನಸ್ಸನ್ನು ಮುತ್ತಿ ಮನುಷ್ಯನನ್ನೇ ನಾಶಮಾಡುತ್ತದೆ. ಯಾರು ಶೋಕಾಕುಲನೂ, ಚಿಂತಾ\-ಕ್ರಾಂತನೂ, ನಿರುತ್ಸಾಹಿಯೂ ಆಗಿರುವನೋ ಅವನ ಎಲ್ಲ ಕಾರ್ಯಗಳೂ ಹಾಳಾಗುತ್ತವೆ. ಅವನು ಬಹಳ ವಿಪತ್ತಿಗೀಡಾಗಿ ನರಳಬೇಕಾಗುತ್ತದೆ.’


\section*{ಕಾರ್ಯಭಾರ–ತಲೆಭಾರ}

\addsectiontoTOC{ಕಾರ್ಯಭಾರ–ತಲೆಭಾರ}

ಅನಂತರಾವ್ ಒಂದು ದೊಡ್ಡ ಔದ್ಯಮಿಕ ಸಂಸ್ಥೆಯಲ್ಲಿ ಸಹಾಯಕ ಮ್ಯಾನೇಜರ್ ಆಗಿ ಕೆಲಸ ಮಾಡಿದವರು. ಅವರು ಒಳ್ಳೆಯ ಕೆಲಸಗಾರರೆಂದೂ, ಹಿರಿಯರಿಗೆ ವಿಧೇಯರೆಂದೂ ಹೆಸರು ಪಡೆದವರು. ಜನಪ್ರಿಯತೆ ಜನಾದರಣೆ ಪಡೆದ ಆ ವ್ಯಕ್ತಿಯೇ ಬಡ್ತಿಯಾಗಿ ಇದೀಗ ಮ್ಯಾನೇಜರ್ ಆಗಿ ನೇಮಕಗೊಂಡಿದ್ದಾರೆ. ಅವರ ಕೈ ಕೆಳಗೆ ಸುಮಾರು ಮುನ್ನೂರು ಮಂದಿ ಸಹಾಯಕರೂ ಸಿಬ್ಬಂದಿ ವರ್ಗದವರೂ ಇದ್ದಾರೆ. ಅವರು ಮ್ಯಾನೇಜರ್ ಸ್ಥಾನವನ್ನು ಅಲಂಕರಿಸಿ ಸುಮಾರು ಹದಿನೈದು ದಿನಗಳಾಗಿವೆಯಷ್ಟೆ. ಹೃದಯದ ಅತಿಯಾದ ಬಡಿತ, ಭಯಸಂಕಟಗಳಿಂದ ರಾತ್ರಿಯಲ್ಲಿ ಗಂಟೆಗಳ ಕಾಲ ಎಚ್ಚೆತ್ತು ಸುಮ್ಮನೆ ಕುಳಿತಿರುತ್ತಾರೆ.

ಡಾಕ್ಟರರನ್ನು ಕಂಡು ಮಾತಾನಾಡಿದಾಗ ತಿಳಿದುಬಂದ ಸಂಗತಿ ಇದು: ಮ್ಯಾನೇಜರ್ ಆಗಿ ಜವಾಬ್ದಾರಿ ಸ್ಥಾನದ ಕಾರ್ಯನಿರ್ವಹಣೆ ಅವರಿಗೆ ದೊಡ್ಡ ಭಾರವೆನಿಸಿದೆ. ಸಮಸ್ಯೆಗಳಿಗೆ ಪರಿಹಾರ ನಿಷ್ಕರ್ಷೆ ಮಾಡುವಾಗ, ಕೆಲವೊಂದು ಕಾರ್ಯನೀತಿ ನಿರ್ಧರಿಸುವಾಗ, ಸಹಾಯಕರ ತರ್ಕಕುತರ್ಕಗಳನ್ನು ಎದುರಿಸುವಾಗ, ಅವರಿಗೆ ವಿಪರೀತ ಬೇಸರ ಬಳಲಿಕೆಯಾಗುತ್ತದೆ. ಕಿರಿಯರು ತನ್ನ ತೀರ್ಮಾನಗಳನ್ನು ಒಪ್ಪುವಂತೆ ಮಾಡಿ ಅವರು ಸರಾಗವಾಗಿ ಕೆಲಸ ಮಾಡುವಂತೆ ಪ್ರೇರಿಸಲು ಸಾಧ್ಯವಾಗುತ್ತಿಲ್ಲ. ಆದರೆ ಎದೆ ಬಡಿತ ನಿದ್ರಾಹೀನತೆಗಳು ಮಾನಸಿಕ ತಳಮಳದಿಂದ ಎಂಬುದನ್ನು ಅವರು ಒಪ್ಪುತ್ತಿಲ್ಲ. ತಜ್ಞರ ಅಭಿಪ್ರಾಯದ ಪ್ರಕಾರ ಅದು ಮಾನಸಿಕ ಮೂಲವಾದ ತೊಂದರೆಯೆ.

ಅಂತೂ ತಜ್ಞರ ಸಲಹೆಯಂತೆ ಅವರು ಕೆಲ ಕಾಲ ರಜೆ ತೆಗೆದುಕೊಂಡರು. ಅದೃಷ್ಟವಶಾತ್ ಅವರು ರಜೆಯ ಮೇಲಿದ್ದಾಗ ಇನ್ನೊಬ್ಬ ದಕ್ಷ ಸಹಾಯಕ ಆಫೀಸರ್​ ಅವರ ಕೆಲಸವನ್ನು ನಿರ್ವಹಿ\-ಸಿದರು. ಮ್ಯಾನೇಜರ್ ರಜೆಯಿಂದ ಹಿಂದಿರುಗಿದ ಮೇಲೆ ಈ ನೂತನ ಆಫೀಸರ್​ ಮ್ಯಾನೇಜರಿಗೆ ಅಸಾಧ್ಯವೆನಿಸಿದ ಜವಾಬ್ದಾರಿಯ ನಿರ್ವಹಣೆಗಳನ್ನು ದಕ್ಷತೆಯಿಂದ ತಾನೇ ನೆರವೇರಿಸಿ ಅವರ ತಲೆಭಾರ ಕಡಿಮೆ ಮಾಡಿದ. ಪರಿಣಾಮವಾಗಿ ಮ್ಯಾನೇಜರ್ ಆ ಬಳಿಕ ಹೃದಯದ ಅತಿ ಬಡಿತ, ನಿದ್ರಾಹೀನತೆ ಮೊದಲಾದ ತೊಂದರೆಗಳಿಂದ ಮುಕ್ತರಾದರು. ಡಾ.\ ಅಲೆಕ್ಸಿಸ್ ಕೆರೆಲ್​ರ ಮಾತು –‘ಚಿಂತೆಯನ್ನು ದೂರಕ್ಕೆಸೆಯಲರಿಯದ ವ್ಯಾಪಾರೋದ್ಯಮಿಗಳು ಯೌವನದಲ್ಲೇ ಸಾವನ್ನಪ್ಪುತ್ತಾರೆ’–ಅದೆಷ್ಟು ಅರ್ಥಗರ್ಭಿತ!


\section*{ಮೂಕವೇದನೆ}

\addsectiontoTOC{ಮೂಕವೇದನೆ}

ಸಂಸ್ಕೃತಿ ಸಂಪ್ರದಾಯ, ನಯನಾಜೂಕುಗಳನ್ನು ನೆಚ್ಚಿಕೊಂಡ ಕುಟುಂಬದಿಂದ ಬಂದ ಒಳ್ಳೆಯ ಹುಡುಗಿ ಸುಜಾತಾ. ತುಂಬ ಓದಿಕೊಂಡವಳು ಕೂಡ. ಅವಳ ಪತಿಯೂ ಉನ್ನತ ತಾಂತ್ರಿಕ ಪರೀಕ್ಷೆಗಳನ್ನು ಪಾಸು ಮಾಡಿದ ಮೇಧಾವಿ. ಕೈ ತುಂಬ ಸಂಬಳ ಬರುವ ಉದ್ಯೋಗ \hbox{ಅವನಿಗೆ.} ಮದುವೆಯಾದ ಹೊಸದರಲ್ಲೇ ಆತ ಕುಡಿತದ ಚಟಕ್ಕೆ ಬಲಿಯಾಗಿದ್ದಾನೆ ಎಂಬುದನ್ನು ಆಕೆ ತಿಳಿದಳು. ತನ್ನ ದುರದೃಷ್ಟವನ್ನು ಹಳಿದುಕೊಂಡರೂ ಧೃತಿಗೆಡದೆ ಆತನನ್ನು ಆ ಚಟದಿಂದ ಬಿಡಿಸಲು ಹೆಚ್ಚು ಕಡಿಮೆ ಎರಡು ವರ್ಷಗಳ ಕಾಲ ಅವಳು ಸಜ್ಜನಿಕೆಯಿಂದ ಬಿಡದೆ ಯತ್ನಿಸಿದ್ದಳು. ತಾಯಿಯ ಒತ್ತಾಯಕ್ಕೆ ಮಣಿದು ಪೂಜೆ ಪ್ರಾರ್ಥನೆ ಮಾಡಿದ್ದಳು. ಹರಕೆಯನ್ನೂ ಹೊತ್ತಿದ್ದಳು. ಪ್ರಯೋಜನವಾಗಲಿಲ್ಲ. ನೆರೆಹೊರೆಯವರು ತನ್ನ ಪತಿಯನ್ನು ನಿಂದಿಸುವುದನ್ನೂ, ತಿರಸ್ಕಾರ ದೃಷ್ಟಿಯಿಂದ ನೋಡುವುದನ್ನೂ ಸಾಕಷ್ಟು ಕಂಡು ಕೇಳಿ ನೊಂದಿದ್ದಳು. ಆತ ದಿನವೂ ಕುಡಿದು ಬಂದು, ಮನೆಯಲ್ಲಿ ವ್ಯರ್ಥಾಲಾಪ ಮಾಡಿ ಅರೆಹುಚ್ಚನಂತೆ ವರ್ತಿಸಿದಾಗ ಅವಳ ಹೊಟ್ಟೆ ಸಂಕಟವನ್ನು ದೇವರಿಗೇ ಹೇಳಬೇಕು! ದಿನದಿನವೂ ಪತಿಯ ದುಃಸ್ಥಿತಿಯನ್ನು ಅಸಹಾಯಕಳಾಗಿ ನೋಡುತ್ತಲಿದ್ದ ಅವಳ ಮನಸ್ಸನ್ನು ನಿರಾಸೆಯ ಕಾರ್ಮೋಡ ಕವಿಯಿತು. ಭವಿಷ್ಯದ ಭರವಸೆಯ ಕಲ್ಪನೆ ನುಚ್ಚುನೂರಾಯಿತು. ಆಕೆ ಯಾವ ಕೆಲಸವನ್ನೂ ಉತ್ಸಾಹದಿಂದ ಮಾಡಲಾರದಾದಳು. ಕೂತರೆ ನಿಂತರೆ ಬಳಲಿಕೆ ಸುಸ್ತು, ಮನೆಗೆಲಸವೆಲ್ಲ ಅಸ್ತವ್ಯಸ್ತ. ವಿಪರೀತ ಮರೆವು. ಬೆಳಿಗ್ಗೆ ನಿದ್ದೆಯಿಂದೇಳುವಾಗಲೇ ಮೈಕೈ ನೋವು, ಬಳಲಿಕೆ! ಯಾಕೆ ಹೀಗೆ? ಚಿಂತೆ ಕೊಡುವ ಬಳುವಳಿ ಅವೆಲ್ಲ. ಎಂಥ ದುರ್ಭರ ಸನ್ನಿವೇಶವನ್ನೂ ಧೈರ್ಯಸ್ಥೈರ್ಯದಿಂದ ಸ್ವೀಕರಿಸುವುದರಿಂದಷ್ಟೇ ಮನಶ್ಶಾಂತಿ ಚಿರಸ್ಥಾಯಿಯಾಗಬಲ್ಲದು. ಆದರೆ ಆ ಧೈರ್ಯ ಸ್ಥೈರ್ಯ ಬರಬೇಕು ಎಲ್ಲಿಂದ?


\section*{ಉದ್ವೇಗದ ಉಪಟಳ}

\addsectiontoTOC{ಉದ್ವೇಗದ ಉಪಟಳ}

ಮುಂಜಾನೆಯಿಂದ ಸಂಜೆಯವರೆಗೆ ಜವುಳಿ ಅಂಗಡಿಯಲ್ಲಿ ಸೇಲ್ಸ್​ಮ್ಯಾನ್ ಆಗಿ ಕೆಲಸ ಮಾಡುತ್ತಿದ್ದ ಉತ್ಸಾಹೀ ತರುಣ ರಂಗ. ಒಂದು ದಿನ ಯಾರ ಚಾಡಿ ಮಾತು ಕೇಳಿಯೋ ಅಥವಾ ಸಂಶಯ\-ದಿಂದಲೋ, ಯಜಮಾನರು ಕೋಪೋದ್ರಿಕ್ತರಾಗಿ ‘ಆ ಮೂರ್ಖಕೆಲಸ ಮಾಡಿದವನು ನೀನೇ’ ಎಂದು ರಂಗನೆಡೆ ಕೈ ತೋರಿಸಿ ಕೂಗಿಕೊಂಡರು. ಯಥಾರ್ಥವಾಗಿ ಅದು ಆತನ ತಪ್ಪಾಗಿರಲಿಲ್ಲ. ಹಾಗಾಗಿ ತನ್ನ ಬಗ್ಗೆ ಉಂಟಾದ ತಪ್ಪು ತಿಳಿವನ್ನು ಸರಿಪಡಿಸಲು ಏನೋ ಹೇಳಹೊರಟಿದ್ದನಷ್ಟೆ! ಆಗಲೇ ಯಜಮಾನರು ‘ಬಾಯಿ ಮುಚ್ಚು, ಎದುರಾಡಬೇಡ, ಎಲ್ಲ ಗೊತ್ತು!’ ಎಂದು ಗುಡುಗಿದರು. ಬಂದ ಗಿರಾಕಿಗಳೆದುರಿಗೆ, ಸಹೋದ್ಯೋಗಿಗಳೆದುರಿಗೆ ರಂಗನ ಮುಖಕ್ಕೆ ಮಂಗಳಾರತಿ\-ಯಾಯ್ತು! ಅವನು ಈ ಅಪಮಾನವನ್ನು ಹೇಗೋ ನುಂಗಿಕೊಂಡ. ಆದರೆ ಚಡಪಡಿಸಿದ, ಗಿರಾಕಿಗಳೊಡನೆ ಅರ್ಧಕ್ಕೆ ನಿಲ್ಲಿಸಿದ ವ್ಯವಹಾರವನ್ನು ಹೇಗೋ ಮುಗಿಸಿದ. ತಲೆ ತಿರುಗ\-ಲಾರಂಭಿ\-ಸಿತು. ಕಣ್ಣೆದುರು ಕತ್ತಲೆ ಕವಿಯಿತು. ಕೂಡಲೇ ಕುಳಿತಿರದಿದ್ದರೆ ಅವನು ಕುಸಿದು ಬೀಳುತ್ತಿದ್ದ. ಸ್ವಲ್ಪ ನೀರು ಕುಡಿದು ಹೇಗೋ ಸಾವರಿಸಿಕೊಂಡ! ಆತನಿಗಾದ ಮಾನಸಿಕ ಆಘಾತದ ಪರಿಣಾಮ ಇದು.

ಮನಸ್ಸಿನಲ್ಲಿ ಮೂಡಿ, ಮಾಯವಾಗುವ ಯೋಚನಾ ತರಂಗಗಳು, ಭಾವನೆಗಳು ರಕ್ತ ಪ್ರವಾಹ ಹಾಗೂ ಹೃದಯದ ಆಳಕ್ಕೂ ಇಳಿಯಬಲ್ಲವು! ನಮ್ಮ ಆರೋಗ್ಯದ ಔನ್ನತ್ಯಕ್ಕೂ, ಅನಾರೋಗ್ಯಕ್ಕೂ ಸಂಕಟಕ್ಕೂ ಕಾರಣವಾಗಬಲ್ಲವು. ನಿಷೇಧಾತ್ಮಕ ಅಥವಾ ದುಷ್ಟ ಯೋಚನೆಗಳು ಕೆಟ್ಟ ಪರಿಣಾಮ\-ವನ್ನುಂಟು ಮಾಡಿದರೆ, ಸದಾಲೋಚನೆ, ಸದ್ಭಾವನೆಗಳು ಒಳ್ಳೆಯ ಪ್ರಭಾವವನ್ನು ಬೀರಿಯೇ ಬೀರುತ್ತವೆ ಎಂಬ ಮಾತು ಕಲ್ಪನೆಯಲ್ಲ ಅಥವಾ ಧಾರ್ಮಿಕ ಮುಖಂಡರ ಸವಕಲು ಬೋಧನೆಯಲ್ಲ, ಸಂಶೋಧನೆಗಳಿಂದ ಕಂಡುಕೊಂಡ ಸತ್ಯ!


\section*{ಚಿಂತೆಯ ಹಣ್ಣು ಹೊಟ್ಟೆಯ ಹುಣ್ಣು}

\addsectiontoTOC{ಚಿಂತೆಯ ಹಣ್ಣು ಹೊಟ್ಟೆಯ ಹುಣ್ಣು}

ಸುಮಾರು ೧೯೫೬ ರಲ್ಲಿ ರಷ್ಯಾದೇಶದ ತಜ್ಞರು ಒಂದು ಸಂಶೋಧನೆಯನ್ನು ಕೈಗೊಂಡು ಮೇಲಿನ ಮಾತನ್ನು ಪ್ರಯೋಗದ ಮೂಲಕ ಸಾಬೀತುಪಡಿಸಿದರು. ವ್ಯಕ್ತಿಯೊಬ್ಬನ ಮನಸ್ಸಿನಲ್ಲಿ ಚಿಂತೆ ಭಯ ಉದ್ವೇಗಗಳನ್ನುಂಟುಮಾಡುವ ಯೋಚನೆ, ಭಾವನೆಗಳನ್ನು ಬಿತ್ತಿದಾಗ, ರೋಗಾಣುಗಳಿಂದ ನಮ್ಮನ್ನು ರಕ್ಷಿಸುವ ರಕ್ತಪ್ರವಾಹದಲ್ಲಿನ ಸಾವಿರದ ಆರುನೂರು ಬಿಳಿರಕ್ತ ಕಣಗಳು ನಿರ್ದಿಷ್ಟಾವಧಿಯಲ್ಲಿ ನಾಶವಾದವು. ಉತ್ಸಾಹ, ಧೈರ್ಯ, ಸಂತೋಷಗಳನ್ನುಂಟು ಮಾಡುವ ಭಾವನೆಗಳನ್ನು ಹರಿಯಗೊಟ್ಟಾಗ, ಸಾವಿರದ ಐನೂರರಷ್ಟು ಬಿಳಿ ರಕ್ತಕಣಗಳು ವೃದ್ಧಿ ಯಾದವು! ಚಿಂತಾಕ್ರೋಶ ಭಯೋದ್ವೇಗಗಳಿಂದ ನಮ್ಮ ಆರೋಗ್ಯಕ್ಕೆ ನಾವೇ ಕಲ್ಲುಹಾಕಿಕೊಳ್ಳುತ್ತೇವೆ.

ಹೊಟ್ಟೆಹುಣ್ಣಿನ ರೋಗ ಆಗಾಗ ಉಲ್ಬಣಗೊಳ್ಳುವುದು ಮಾನಸಿಕವಾದ ಚಿಂತೆ ಭಯೋದ್ವೇಗಗಳ ಏರುಪೇರುಗಳಿಂದ ಎಂದು ಮೇಯೋ ಕ್ಲಿನಿಕ್​ನ ಡಾ. ಆಲ್ವಾರಿಸ್ ಹೇಳಿದರು. ಅವರ ಈ ಘೋಷಣೆಗೆ ಆಧಾರ ಆ ಕ್ಲಿನಿಕ್​ನಲ್ಲೇ ನಡೆದ ಪ್ರಯೋಗಗಳು.

ಅವರು ಹೊಟ್ಟೆಯ ನೋವಿನ ರೋಗಗಳಿಂದ ನರಳುವ ಹದಿನೈದು ಸಹಸ್ರ ರೋಗಿಗಳನ್ನು ಪರೀಕ್ಷೆಗೊಳಪಡಿಸಿದರು. ಅವರೆಲ್ಲರ ನರಳಾಟದ ಕಾರಣವನ್ನೂ ಕಂಡುಹಿಡಿದರು. ಅವರಲ್ಲಿ ಸುಮಾರು ಹನ್ನೆರಡು ಸಾವಿರ ಮಂದಿಯ ನರಳಾಟಕ್ಕೆ ಕಾರಣ ಮಾನಸಿಕ–ಮೂಲವಾದದ್ದೆಂದು ತಿಳಿಯಿತು. ಆಹಾರ–ನೀರುಗಳ ದೋಷ, ವಾತಾವರಣದ ಮಾಲಿನ್ಯ, ಸಾಂಕ್ರಾಮಿಕ ರೋಗಾಣುಗಳ ಪ್ರಸಾರ–ಇವು ಅವರ ಅನಾರೋಗ್ಯಗದ ಕಾರಣಗಳಾಗಿರಲಿಲ್ಲ. ಮನಸ್ಸಿನ ಏರುಪೇರುಗಳು – ಎಂದರೆ ಭಯ, ಚಿಂತೆ, ದ್ವೇಷ, ಮತ್ಸರ, ಅತಿಯಾದ ಸ್ವಾರ್ಥಪರತೆ, ಪರಿಸ್ಥಿತಿಯೊಂದಿಗೆ ಹೊಂದಿಕೊಳ್ಳಲಾಗದ ಚಡಪಡಿಕೆ–ಇವುಗಳು ಮಾಡಿದ ದುಷ್ಪರಿಣಾಮದ ಅನಾಹುತವೇ ಅವರ ಹೊಟ್ಟೆಯ ನೋವಿಗೆ ಕಾರಣವೆಂದು ಸಾಬೀತಾಯಿತು.

ಇಪ್ಪತ್ತು ವರ್ಷಗಳ ಕಾಲ ಅಸಂಖ್ಯ ರೋಗಿಗಳನ್ನು ಉಪಚರಿಸಿ ಅವರ ಬದುಕನ್ನು ಅಧ್ಯಯನ ಮಾಡಿ ಚಿಂತಾಕ್ರೋಶಭಯೋದ್ವೇಗಗಳಿಂದ ಉಂಟಾಗುವ ಹಾನಿಯನ್ನು ಪರಿಶೀಲಿಸಿ, ಅವುಗಳಿಂದ ಬಿಡಿಸಿಕೊಳ್ಳುವ ವಿಧಾನವನ್ನು ತಿಳಿಸಿ ಉಪಕರಿಸಿದವರು ಅಮೇರಿಕದ ಡಾ.\ ಜಾನ್ ಎ.\ ಷಿಂಡ್ಲರ್. ಅವರು ಹೇಳುವಂತೆ ನಮ್ಮ ನರಳಾಟಕ್ಕೆ ಕಾರಣವಾದ ರೋಗಗಳಲ್ಲಿ ನೂರರಲ್ಲಿ ಐವತ್ತರಷ್ಟು, ಚಿಂತಾಕ್ರೋಶಭಯೋದ್ವೇಗಗಳಿಂದಲೇ ಉಂಟಾಗುತ್ತವೆ.\footnote{ ಬ್ರಿಟನ್ನಿನ ಪ್ರಸಿದ್ಧ ಮನೋರೋಗ ಚಿಕಿತ್ಸಾ ತಜ್ಞರಾದ ಪೀಟರ್ ಬ್ಲೀತ್ ತಮ್ಮ \engfoot{Stress Disease: The Growing Plague} ಎನ್ನುವ ಗ್ರಂಥದಲ್ಲಿ ಈ ಕೆಳಗಿನ ರೋಗಗಳನ್ನು ಮನೋದೈಹಿಕ ಬೇನೆಗಳೆಂದು ಹೆಸರಿಸಿದ್ದಾರೆ: \engfoot{ Hypertension. Heart Attack, Cerebral Vascular Episode. Migraine, Allergies. Bronchial Asthma, Hay Fever. Anorexia Nervosa, Pruritis, Colitis, Constipation, Diarrhoea, Belching, Hiccough, Menstrual Disorders, Hyper Thyroidism, Diabetes, Mellitus, Rheumatoid Arthritis, Skin Disorders and Tuberculosis.}} ನಮ್ಮಲ್ಲಿ ಬಹು ಮಂದಿ ನಾನಾ ತೆರನಾದ ತೊಂದರೆ ತಾಪತ್ರಯಗಳ ಮಧ್ಯದಲ್ಲಿ ಸಿಲುಕಿಕೊಂಡಿದ್ದರೂ, ಚಿಂತೆ ಭಯೋದ್ವೇಗಗಳು ದೈಹಿಕ ಮಾನಸಿಕ ಆರೋಗ್ಯವನ್ನು ಹೇಗೆ ನಾಶಮಾಡುವುವು ಎಂಬುದನ್ನು ಅರಿಯರು. ನಮ್ಮ ಬದುಕು ಉತ್ಸಾಹ, ಧೈರ್ಯ, ಕಾರ್ಯ ಕುಶಲತೆ, ಆನಂದ–ಇವುಗಳ ಚಿಲುಮೆಯಾಗಿರಬೇಕಾದರೆ ನಾವು ಆ ಬಗ್ಗೆ ಜಾಗ್ರತರಾಗಬೇಕು.

~\\[-1cm]


\section*{ರೋಗಕ್ಕೆ ಮದ್ದು, ರೋಗಿಗೆ ಮುದ್ದು}

\addsectiontoTOC{ರೋಗಕ್ಕೆ ಮದ್ದು, ರೋಗಿಗೆ ಮುದ್ದು}

ರೋಗಿಯೊಬ್ಬ ವೈದ್ಯರ ಬಳಿ ಬಂದು ತನ್ನ ತೊಂದರೆಯನ್ನು ಹೇಳಿಕೊಂಡನೆನ್ನಿ. ವೈದ್ಯರು ಅವನನ್ನು ಪರಿಶೀಲಿಸಿ ಅದು ಮಾನಸಿಕ ಏರಿಳಿತಗಳಿಂದ ಉಂಟಾದ ತೊಂದರೆ ಎಂದು ತಿಳಿದು ಕೊಂಡರೂ ಅದನ್ನು ಹೇಳಲಾರರು. ಏಕೆಂದರೆ ‘ಇದೆಲ್ಲ ಮಾನಸಿಕತೊಂದರೆ–ಸೈಕಲಾಜಿಕಲ್​– ಔಷಧವೇನೂ ಬೇಕಿಲ್ಲ. ಸ್ವಲ್ಪ ರಿಲ್ಯಾಕ್ಸ್ ಮಾಡಿಕೊಂಡು ಶಾಂತಚಿತ್ತರಾಗಿರಿ, ಉದ್ವೇಗ ರಾಗ ದ್ವೇಷಗಳನ್ನು ಕಡಿಮೆ ಮಾಡಿ’ ಎಂದು ಸಮಾಧಾನಮಾಡಿದರೆ ರೋಗಿಗೆ ಹೇಗನ್ನಿಸಬಹುದು? ರೋಗಿ ತಪ್ಪುತಿಳಿದುಕೊಳ್ಳುತ್ತಾನೆ. ಕೋಪಿಸಿಕೊಳ್ಳಲೂಬಹುದು. ‘ಔಷಧ ಕೊಡುವುದನ್ನು ಬಿಟ್ಟು ಇವನೇನೋ ತತ್ತ್ವಜ್ಞಾನ ಹೇಳ ಹೊರಟನಲ್ಲ–ಇವನೆಂಥ ಡಾಕ್ಟರ್​!’ ಎನ್ನಿಸಬಹುದು. ಆಗ ಆತ ಇನ್ನೊಬ್ಬ ಡಾಕ್ಟರರನ್ನು ಸಮೀಪಿಸಬಹುದು. ಆದುದರಿಂದ ರೋಗವನ್ನು ಮಾತ್ರವಲ್ಲ, ರೋಗಿಯನ್ನೂ ಚಿಕಿತ್ಸೆಗೊಳಪಡಿಸುವ ಕಾರ್ಯವನ್ನು ವೈದ್ಯರು ಕೈಗೊಳ್ಳಬೇಕಾಗಿದೆ. ರೋಗಕ್ಕೆ ಮದ್ದನ್ನು ಕೊಡುವುದರ ಜೊತೆಗೆ ರೋಗಿಯನ್ನು ಮುದ್ದಿನಿಂದ ನೋಡಿದರೆ ಚಿಕಿತ್ಸೆ ಬೇಗ ಫಲಕಾರಿ ಆಗುತ್ತದೆ. ಪ್ಲೇಟೋ ಎಂದಂತೆ ವೈದ್ಯರು ದೇಹ ಮತ್ತು ಮನಸ್ಸುಗಳನ್ನು ಒಂದಾಗಿ ಪರಿಗಣಿಸಿ ನೀಡಿದ ಚಿಕಿತ್ಸೆ ಮಾತ್ರವೇ ಫಲಪ್ರದವಾಗುತ್ತದೆ.

ರೋಗಿಯ ಕೌಟುಂಬಿಕ ಹಾಗೂ ಸಾಮಾಜಿಕ ಆರ್ಥಿಕ ಪರಿಸ್ಥಿತಿ, ಆತನ ನಂಬಿಕೆ, ದೃಷ್ಟಿಕೋನ, ಬಾಲ್ಯದ ಅನುಭವಗಳು, ವಿದ್ಯಾಭ್ಯಾಸ, ಮಿತ್ರರು–ಇತ್ಯಾದಿಗಳನ್ನು ಸರಿಯಾಗಿ ತಿಳಿದುಕೊಂಡು, ಅವನಿಗೆ ಅರ್ಥವಾಗುವ ರೀತಿಯಲ್ಲಿ ಸದ್ಯದ ಪರಿಸ್ಥಿತಿಗೆ ಕಾರಣವೇನೆಂಬುದನ್ನು ಮೊದಲು ತಿಳಿಸಿ\-ಕೊಟ್ಟು ದೀರ್ಘಕಾಲದಿಂದ ಬೆಳೆಸಿಕೊಂಡು ಬಂದ ನಿಷೇಧಾತ್ಮಕವಾದ ಭಾವನಾವೈಪರೀತ್ಯಗಳನ್ನು ದೂರ ಮಾಡಲು ರಚನಾತ್ಮಕವಾದ ಭಾವನೆಗಳನ್ನು ಹೇಗೆ ರೂಢಿಸಿಕೊಳ್ಳಬೇಕೆಂಬುದನ್ನು ಹೇಳಿ ಮಾರ್ಗದರ್ಶನ ಮಾಡಿದರೆ ಅದನ್ನು ಸರಿಯಾದ ಚಿಕಿತ್ಸಾ ವಿಧಾನವೆನ್ನಬಹುದು. ಇದನ್ನು ಯುಕ್ತ ಮನೋರೋಗ ಚಿಕಿತ್ಸಾವಿಧಾನ \enginline{(Adequate Psychotherapy)} ಎನ್ನುತ್ತಾರೆ. ಆದರೆ ಈ ವಿಧಾನವನ್ನು ಅನುಸರಿಸಲು ಸಮಯವೆಷ್ಟು ಬೇಕು ಬಲ್ಲಿರಾ? ಒಬ್ಬ ರೋಗಿಯನ್ನು ಪರಿಶೀಲಿಸಿ ಚಿಕಿತ್ಸೆ ನೀಡಲು ಸುಮಾರು ಇಪ್ಪತ್ತು ಗಂಟೆಗಳ ಕಾಲ ಬೇಕು! ನಾನಾ ಅನುಕೂಲತೆಗಳಿರುವ ಅಮೇರಿಕ ದೇಶದಲ್ಲಿ ಪ್ರತಿದಿನ ಸಾಮಾನ್ಯ ಡಾಕ್ಟರ್ ಒಬ್ಬರು ಸರಾಸರಿ ಇಪ್ಪತ್ತಮೂರು ಮಂದಿ ರೋಗಿಗಳನ್ನು ನೋಡುತ್ತಾರಂತೆ!


\section*{ಮಾನಸಿಕ ಮೂಲ}

\addsectiontoTOC{ಮಾನಸಿಕ ಮೂಲ}

ಚಿಂಕ್ರೋಭದಿಂದುಂಟಾಗುವ ಹಲವು ತೆರನಾದ ಕಾಯಿಲೆಗಳಲ್ಲಿ ಕೆಲವನ್ನು ಇಲ್ಲಿ ಉಲ್ಲೇಖಿಸಲಾಗಿದೆ. ಅವುಗಳಲ್ಲಿ ಶೇಕಡಾ ಎಷ್ಟು ಮಾನಸಿಕ ಮೂಲದವುಗಳೆಂಬುದನ್ನು ತಿಳಿಸಲಾಗಿದೆ–

\begin{longtable}{@{}ll@{}}
\multicolumn{1}{c}{\textbf{ಬೇನೆಗಳು}} & \multicolumn{1}{c}{\textbf{ಮಾನಸಿಕ ಮೂಲದವು ಶೇಕಡಾ}} \\
ಕತ್ತಿನ ಹಿಂಭಾಗದ ಬೇನೆ & \multicolumn{1}{c}{೭೮} \\
ಗಂಟಲಿನಲ್ಲಿ ಗಡ್ಡೆಯ ಅನುಭವ & \multicolumn{1}{c}{೯೦} \\
ಹೊಟ್ಟೆಯ ಹುಣ್ಣಿನ ನೋವು & \multicolumn{1}{c}{೫೦} \\
ಗಾಲ್ ಬ್ಲ್ಯಾಡರ್ ನೋವು & \multicolumn{1}{c}{೫೦} \\
ವಾಯು ತೊಂದರೆ & \multicolumn{1}{c}{೯೯} \\
ತಲೆತಿರುಗು & \multicolumn{1}{c}{೮೦} \\
ತಲೆನೋವು & \multicolumn{1}{c}{೮೦} \\
ಮಲಬದ್ಧತೆ & \multicolumn{1}{c}{೭೦} \\
ಬಳಲಿಕೆ & \multicolumn{1}{c}{೯೦} \\
\end{longtable}

ಚಿಂತಾಕ್ರೋಶಭಯೋದ್ವೇಗಗಳು ನಮ್ಮ ದೇಹದಲ್ಲಿ ಮಾರಕ ರಾಸಾಯನಿಕ ಬದಲಾವಣೆ\-ಗಳನ್ನುಂಟು\-ಮಾಡುತ್ತವೆ ಎಂಬುದನ್ನು ತಿಳಿದುಕೊಳ್ಳಲು ಕಷ್ಟವೇನೂ ಇಲ್ಲ. ಆದರೆ ಹೆಚ್ಚಿನ ಜನ ಅದನ್ನು ತಿಳಿಯುವ ಗೋಜಿಗೇ ಹೋಗುವುದಿಲ್ಲ. ಅದರಿಂದೆಂಥ ಕೆಡುಕೆಂಬುದನ್ನು ಯೋಚಿಸುವು\-ದಿಲ್ಲ. ಅದರಿಂದ ತಪ್ಪಿಸಿಕೊಳ್ಳಲು ಯಾವ ವಿಧಾನವನ್ನು ಅನುಸರಿಸಬೇಕೆಂಬುದನ್ನು ತಿಳಿಯು\-ವುದಿಲ್ಲ. ಕೋಪ ಬಂದಾಗ ಹುಬ್ಬುಗಂಟಿಕ್ಕುತ್ತೇವೆ. ಕಣ್ಣು ಕೆಂಪಗಾಗುತ್ತದೆ, ಧ್ವನಿ ಗಡಸಾಗು\-ತ್ತದೆ. ಇವು ನಮ್ಮ ಶರೀರದ ಹೊರಭಾಗದಲ್ಲಿ ಕಂಡುಬರುವ ಲಕ್ಷಣವಾದರೆ ದೇಹದ ಒಳಗೆ ಎಂಥ ಪರಿಣಾಮ ಬೀರುತ್ತದೆಂಬುದನ್ನು ವೈದ್ಯರು ಹೇಳುತ್ತಾರೆ. ಮುಖ್ಯವಾಗಿ, ರಕ್ತದ ಒತ್ತಡ ಅಧಿಕವಾಗುವುದು. ಕ್ರೋಧೋನ್ಮತ್ತನಾದಾಗ ಎದೆ ಬಡಿತ ನೂರೆಂಬತ್ತರಿಂದ ಇನ್ನೂರಮೂವತ್ತರವರೆಗೆ ಏರುವುದುಂಟು. ಕ್ರೋಧದ ಆವೇಶ ನಮ್ಮನ್ನು ಬಿಟ್ಟು ಹೋಗುವವರೆಗೂ ಈ ಬಡಿತದ ತೀವ್ರತೆ ಉಳಿದಿರುತ್ತದೆ. ಇದು ಆರೋಗ್ಯದ ಮೇಲೆ ಅತ್ಯಂತ ಕೆಟ್ಟ ಪರಿಣಾಮವನ್ನುಂಟು ಮಾಡುತ್ತದೆ. ರಕ್ತದ ಒತ್ತಡವೂ ನೂರಮೂವತ್ತರಿಂದ ಇನ್ನೂರಮೂವತ್ತರವರೆಗೂ ಏರುತ್ತದೆ. ಈ ಕೋಪದ ತಾಪದಿಂದ ಮಿದುಳಿನಲ್ಲಿ ರಕ್ತಸ್ರಾವ ಉಂಟಾಗಬಹುದು. ಹೃದಯದ ರಕ್ತನಾಳ ಸಂಕುಚಿತಗೊಳ್ಳಬಹುದು. ಇದು ಹೃತ್ಕ್ರಿಯೆ ನಿಂತು ಮರಣಕ್ಕೂ ಕಾರಣವಾಗಬಹುದು. ಎಲ್ಲರೂ ತೀವ್ರತರದ ಕೋಪತಾಪಗಳನ್ನು ದೀರ್ಘಕಾಲ ಮುಂದುವರಿಸಿಕೊಂಡು ಹೋಗಲಾರರು. ಮುಂದುವರಿಸಿದರೆ ಪ್ರಾಣಾಂತಿಕ ಆಘಾತ ಉಂಟಾಗಬಹುದು. ಶರೀರವು ಅತಿ ಉದ್ವೇಗದ ಸ್ಥಿತಿಯನ್ನು ಭರಿಸಲಾರದು. ಕೆಲವರೇನೋ ಮುಗುಳ್ನಗುತ್ತ ಕೋಪ ಹಾಗೂ ತಮಗಾದ ಅಪಮಾನವನ್ನು ಹೊರಗಡೆ ತೋರಿಸಿಕೊಳ್ಳದೆ ನುಂಗಿಕೊಳ್ಳಬಹುದು. ಆದರೆ ಕೋಪದ ಅಭಿವ್ಯಕ್ತಿ ತೀವ್ರತರವಾಗಿ ವ್ಯಕ್ತವಾಗ\-ದಿದ್ದರೂ ಸೂಕ್ಷ್ಮವಾಗಿ ಅಂತರ್ಮನಸ್ಸಿನಲ್ಲಿ ತನ್ನ ಕೆಲಸ ಮಾಡುತ್ತಲಿರುತ್ತದೆ. ಮನಸ್ಸಿನ ಆಳದಲ್ಲಿ ತನಗೆ ನೋವನ್ನುಂಟುಮಾಡಿದವರ ಬಗ್ಗೆ, ತನ್ನನ್ನು ತಿರಸ್ಕರಿಸಿದವರ ಬಗ್ಗೆ ಒಂದು\break ತೆರನಾದ ಅಸಮಾಧಾನ ಹಾಗೂ ದ್ವೇಷ ಒಳಗೊಳಗೇ ಕುದಿಯುತ್ತಿರುತ್ತದೆ. ವ್ಯಕ್ತಿ ಅಸಂತುಷ್ಟನಾಗಿ ಒಳಗೊಳಗೇ ಗೊಣಗುಟ್ಟುತ್ತ ಕುದಿಯುತ್ತಿರುತ್ತಾನೆ. ಈ ಅಭ್ಯಾಸವೂ ಆರೋಗ್ಯಕ್ಕೆ ಮಾರಕ!

ಅಲ್ಪಸ್ವಲ್ಪ ನೋವು, ಒತ್ತಡಗಳನ್ನು ಬದುಕಿರುವ ಎಲ್ಲರೂ ಅನುಭವಿಸಲೇ ಬೇಕು. ಅವುಗಳನ್ನು ಎದುರಿಸುವ ಸಾಮರ್ಥ್ಯ ಹೆಚ್ಚುಕಡಿಮೆ ಸಾಮಾನ್ಯವ್ಯಕ್ತಿಯಲ್ಲಿ ಇದ್ದೇ ಇದೆ. ವಿಜ್ಞಾನದ ನೂತನ ಸಂಶೋಧನೆಗಳ ಪರಿಣಾಮವಾಗಿ ನಾವಿಂದು ಶ್ರಮ ಹಾಗೂ ಸಮಯವನ್ನು ಉಳಿಸುವ ನೂರಾರು ಸೌಕರ್ಯಗಳನ್ನು ಪಡೆದಿದ್ದೇವೆ. ಈ ವೇಗದ ಯುಗದಲ್ಲಿ ನೂರು ವರ್ಷಗಳ ಹಿಂದಿನ ಜನರ ತಾಳ್ಮೆ ಮತ್ತು ಶಾಂತಿ–ಇವುಗಳನ್ನು ಕಾಣುವುದು ಕಷ್ಟ. ಆದರೆ ಇಂದಿನವರು ಆ ಕಾಲದ ಜನರ ತಾಳ್ಮೆ ಶಾಂತಿಗಳನ್ನು ಸ್ಮರಿಸಿಕೊಳ್ಳುವುದು ಲೇಸು. ಹಾಗೆ ಸ್ಮರಿಸಿಕೊಂಡರೆ ಆತುರ ಅವಸರ ಗೊಂದಲಗಳಿಗೆ ಕೊಂಚವಾದರೂ ತಡೆ ಹಾಕಲು ಸಾಧ್ಯವಾಗಬಹುದು. ಮನುಷ್ಯನ \hbox{ ಶರೀರದಲ್ಲಿ} ಪರಿಸ್ಥಿತಿಗೆ ಹೊಂದಿಕೊಳ್ಳುವ ಚಾತುರ್ಯ ಇದ್ದೇ ಇದೆ. ವಿಮಾನ ನಿಲ್ದಾಣದ ಸಮೀಪದಲ್ಲಿರುವ ಮನೆಗಳಲ್ಲಿ ವಾಸ ಮಾಡುವಾಗ ಮೊದಲಿಗೆ ವಿಮಾನದ ಸದ್ದುಗದ್ದಲದಿಂದ ವಿಶ್ರಾಂತಿ ಪಡೆಯಲಾರದೆ ವಿಚಲಿತನಾದ ವ್ಯಕ್ತಿ ಕೆಲವೇ ತಿಂಗಳುಗಳಲ್ಲಿ ಗೊರಕೆ ಹೊಡೆದು ನಿದ್ರಿಸುವಷ್ಟು ಹೊಂದಿಕೊಳ್ಳುತ್ತಾನೆ. ನಗರವಾಸಿಗಳು ಬಸ್ಸಿಗಾಗಿ ಕ್ಯೂನಲ್ಲಿ ಗಂಟೆಗಟ್ಟಲೆ ಕಾಯಬೇಕಾಗುತ್ತದೆ. ಅಯ್ಯೋ, ಇನ್ನೂ ಬರಲಿಲ್ಲವಲ್ಲ ಎಂಬ ಚಡಪಡಿಕೆ ಪ್ರಾರಂಭವಾಗುತ್ತದೆ. ಜನದಟ್ಟಣಿಯ ರಸ್ತೆಯಲ್ಲಿ ವಾಹನವೇರಿ ಹೋಗುವಾಗ ಜನ ಕಾಲುದಾರಿಯನ್ನು ಬಿಟ್ಟು ವಾಹನಕ್ಕೆದುರಾಗಿ ಬಂದಾಗ ಕೋಪ ಸಿಡಿಯುತ್ತದೆ. ಸಹೋದ್ಯೋಗಿಗಳೊಡನೆ ಬಿಸಿಬಿಸಿ ಚರ್ಚೆ, ವಾಗ್ಯುದ್ಧ, ಸಣ್ಣ ಜಗಳ\-ವಾದಾಗಲೂ ಮನಸ್ಸು ಕಸಿವಿಸಿಗೊಳ್ಳುತ್ತದೆ. ಮನೆಯಲ್ಲಿ ಹೆಂಡತಿ ಮಕ್ಕಳ ಜೊತೆಯಲ್ಲಿ ಅಭಿಪ್ರಾಯ ವ್ಯತ್ಯಾಸದಿಂದ ಕಳವಳ, ಸಿಡಿಮಿಡಿಗೊಳ್ಳುವುದು ಸ್ವಾಭಾವಿಕ. ಮನಸ್ಸು ಯಂತ್ರದಂತೆ ಯಾವ ಭಾವೋದ್ವೇಗವಿಲ್ಲದೆ ವರ್ತಿಸಲಾರದು. ಆದರೂ ಚಿಂಕ್ರೋಭದ ದುಷ್ಪರಿಣಾಮವನ್ನರಿತಾಗ ಮನುಷ್ಯ ತನ್ನನ್ನು ತಾನು ಸೈರಿಸಿಕೊಳ್ಳುವುದನ್ನು ಕಲಿಯಬೇಕು. ಆ ನಿಟ್ಟಿನಲ್ಲಿ ಮರೆವೆಯೂ ಮನುಷ್ಯನ ಪಾಲಿಗೆ \hbox{ ವರವಾಗುತ್ತದೆ.}


\section*{ನರಕದ ಬಾಗಿಲು}

\addsectiontoTOC{ನರಕದ ಬಾಗಿಲು}

ಮುಂಜಾನೆಯಿಂದ ಬಿಸಿಲೇರಿ ಮಧ್ಯಾಹ್ನದವರೆಗೂ ಹೊಲದಲ್ಲಿ ಬೆವರಿಳಿಸಿ ದುಡಿದು ಹಸಿದ ಹೆಬ್ಬುಲಿಯಂತೆ ಮನೆಗೆ ಬಂದ ದುರುಗಪ್ಪ ಹೆಂಡತಿಯನ್ನು ದುರುಗುಟ್ಟಿಕೊಂಡೇ ಕೇಳಿದ: ‘ಊಟ ಸಿದ್ಧವಾಗಿದೆಯೆ?’ಎಂದು. ಅಸಡ್ಡೆಯಿಂದ ಅನ್ಯಮನಸ್ಕಳಾಗಿ ‘ಇನ್ನೂ ಒಂದರ್ಧ ಗಂಟೆ ಸ್ವಲ್ಪ ಕಾಯಿರಿ’ ಎಂದಳು ಆಕೆ. ‘ಆಂ! ಏನೆಂದೆ!’ ಎನ್ನುತ್ತ ಕೈಯಲ್ಲಿದ್ದ ಪಿಕಾಸಿಯಿಂದ ಅವಳ ತಲೆಗೆ ಬಲವಾಗಿ ಬಾರಿಸಿಯೇ ಬಿಟ್ಟ. ‘ಅಯ್ಯೋ! ಸತ್ತೆ’ ಎನ್ನುತ್ತ ಕೆಳಕ್ಕುರುಳಿದ ಆಕೆ ಮತ್ತೆ ಮೇಲೇಳಲೇ ಇಲ್ಲ. ಕೋಪ ಇಳಿದ ಮೇಲೆ ಅವನು ಪಟ್ಟ ಪಶ್ಚಾತ್ತಾಪಕ್ಕೆ ಅಂತ್ಯವೇ ಇರಲಿಲ್ಲ.

ತನ್ನ ಮೇಲಧಿಕಾರಿ ಸ್ವಲ್ಪ ಬಿರುಸಾಗಿ ಮಾತನಾಡಿದ್ದಕ್ಕೆ, ಶಾಮಣ್ಣ ಹಿಂದುಮುಂದು ನೋಡದೆ, ಕೆಲಸಕ್ಕೆ ರಾಜೀನಾಮೆ ಕೊಟ್ಟು ಮನೆಗೆ ನಡೆದ. ಸಿಟ್ಟಿನ ಹೊತ್ತಿನಲ್ಲಿ ಹೇಳಿಕೊಂಡ– ‘ರಾಜೀನಾಮೆ ಅವನ ಮುಖದ ಮೇಲೆ ಎಸೆದು ಬಂದೆ. ನಾನು ಯಾರಿಗೂ ಕೇರ್ ಮಾಡುವವನಲ್ಲ’ ಎಂದು. ಸಿಟ್ಟು ಇಳಿದ ಮೇಲೆ ಪಶ್ಚಾತ್ತಾಪ ಅವನನ್ನು ಹಲವು ವಿಧಗಳಲ್ಲಿ ಕಾಡಿತು. ಹೆಂಡತಿ ಮಕ್ಕಳ, ಕುಟುಂಬದ ಭರಣೆ ಪೋಷಣೆಯ ಚಿಂತೆ ಅವನನ್ನು ಪೀಡಿಸಿತು. ತಾಳ್ಮೆಗೆಟ್ಟು ಎಂಥ ಬೆಪ್ಪು ಕೆಲಸ ಮಾಡಿದೆ ಎಂದುಕೊಂಡು ಪರಿತಪಿಸಿದ.

ಸಿಟ್ಟಿನ ಅನರ್ಥಕಾರಿ ಪರಿಣಾಮಗಳು ನೂರಾರು. ಪ್ರಪಂಚದ ಎಲ್ಲ ಮಹಾತ್ಮರೂ ಸಿಟ್ಟನ್ನು ಬಿರುಸಾಗಿಯೇ ಶಪಿಸಿದ್ದಾರೆ. ಭಗವದ್ಗೀತೆಯಲ್ಲಿ ಸಿಟ್ಟನ್ನು ‘ನರಕದ ಬಾಗಿಲು’ ಎಂದು ಕರೆಯಲಾಗಿದೆ. ಕೋಪದ ಪರಿಣಾಮ ಜಗಳ, ಬೈಗಳಿಗೆ ಪ್ರತಿಬೈಗಳು, ಹಿಂಸೆಗೆ ಪ್ರತಿಹಿಂಸೆ. ಹೀಗೆ ಕೋಪಿಷ್ಠರಿಂದ ಅಧರ್ಮ ಪ್ರಸಾರ ಅಲ್ಲದೆ ಸರ್ವನಾಶ ಉಂಟಾದೀತೆಂದು ಎಚ್ಚರಿಸುತ್ತದೆ\break ಮಹಾ\-ಭಾರತ.

ಮನುಷ್ಯನ ಮಿದುಳನ್ನು ಸಿಟ್ಟು ಪ್ರವೇಶಿಸಿದಾಗ, ಅವನು ತೋರಿಸುವ ವಿಕಾರ ವರ್ತನೆಗಳನ್ನು ಪರಿಶೀಲಿಸಿದ್ದೀರಾ? ಅವು ನಿಮಗೆ ಒಳ್ಳೆಯ ಮನೋರಂಜನೆಯನ್ನು ಒದಗಿಸಿಯಾವು. ನನಗೆ ತಿಳಿದ ಒಬ್ಬಾತನ ಸಿಟ್ಟಿನ ವರ್ತನೆಯನ್ನು ಹೇಳುತ್ತೇನೆ. ತನ್ನ ಸಿಟ್ಟಿಗೆ ಕಾರಣರಾದ ವಿರೋಧಿಗಳ ‘ರಕ್ತ ಹೀರುತ್ತೇನೆ’ ಎನ್ನುವ ಅಭ್ಯಾಸ ಆತನಿಗೆ. ಆ ಮಾತನ್ನು ಅವನು ವಿವಿಧ ಭಂಗಿಗಳಲ್ಲಿ ಹೇಳುತ್ತಿದ್ದ–‘ಹೀರುತ್ತೇನೆ’, ‘ಹೀರಿಯೇ ಬಿಡುತ್ತೇನೆ’, ‘ಹೀರುತ್ತೇನೆಂದರೆ ಹೀರಿಯೇ ಬಿಡುತ್ತೇನೆ’, ‘ಫಾಶಿಯಾದರೂ ಹೀರುತ್ತೇನೆ’, ‘ಗಡೀಪಾರಾದರೂ ಹೀರುತ್ತೇನೆ’, ‘ಹೀರುವುದು, ಹೀರುವುದೇ ಅದರಲ್ಲಿ ಸ್ವಲ್ಪವೂ ಸಂದೇಹವಿಲ್ಲ!’ ಅವನ ಈ ನಾಟಕವನ್ನು ನೋಡಿ ಯಾರಾದರೂ ನಕ್ಕರೆ ಬೆಂಕಿಗೆ ತುಪ್ಪ ಹಾಕಿದಂತೆ. ಆಗ ಅವನ ಸಿಟ್ಟು ನೆತ್ತಿಗೇರಿ ಪ್ರಾಸಬದ್ಧ ಕನ್ನಡದಲ್ಲಿ ಮಾತನಾಡುವುದುಂಟು–‘ನಿನ್ನ ದಿಟ್ಟತನವನ್ನು ತಟ್ಟನೇ ನಿಲ್ಲಿಸಿ ಬಿಟ್ಟೇನು!’ ಇತ್ಯಾದಿ. ಈ ಸಿಟ್ಟುಗಾರರಲ್ಲಿ ‘ಕತ್ತರಿಸಿ ಹಾಕುವವರು’, ‘ಧ್ವಂಸ ಮಾಡುವವರು’, ‘ನೆಲಸಮ ಮಾಡುವವರು’, ‘ಸೊಂಟ ಮುರಿಯುವರು’ ಇದ್ದಾರೆ! ಸಿಟ್ಟಿನ ಸಮಯದಲ್ಲಿ ಕೊಂಕು ನುಡಿಗಳೂ, ಅಶ್ಲೀಲ ವಾಕ್ಯಗಳೂ, ಪುಂಖಾನುಪುಂಖವಾಗಿ ಹೊರಹೊಮ್ಮುವುದುಂಟು. ಸಿಟ್ಟುಗಾರರ ಇದಿರಿನಲ್ಲಿ ಅವರ ವರ್ತನೆಯನ್ನು ನೋಡಿ ನಗಲು ಪ್ರಾರಂಭಿಸಬೇಡಿ. ಅವರು ನಿಮ್ಮ ಮುಖವನ್ನು ಪರಚಿಬಿಟ್ಟಾರು –ಜೋಕೆ! ಸಿಟ್ಟನ್ನು ನೇರವಾಗಿ ಪ್ರದರ್ಶಿಸಲು ಅಸಾಧ್ಯವಾದಾಗ, ಎಷ್ಟೋ ಮಂದಿ ಅದನ್ನು ಇತರರ ಮೇಲೆ ಹರಿಯಗೊಡುತ್ತಾರೆ. ಕೆಲವರು ಸಿಟ್ಟನ್ನು ನುಂಗಿ ಸ್ತಬ್ಧರಾಗುತ್ತಾರೆ. ಇದೊಂದು ಅಪಾಯಕಾರಿ ಅಭ್ಯಾಸ.

‘ಸಿಟ್ಟು ನಿಮ್ಮ ಮಿದುಳನ್ನು ಪ್ರವೇಶಿಸುವ ಮೊದಲು, ನಿಮ್ಮ ಬುದ್ಧಿಯನ್ನು ತಲೆಬುರುಡೆಯಿಂದ ಹೊರಗೆ ಹಾಕಿ, ತಿರುಗಿ ಅದು ಒಳಗೆ ಬರದಂತೆ ತಲೆಯ ಎಲ್ಲಾ ಬಾಗಿಲುಗಳನ್ನು ಮುಚ್ಚಿಬಿಡುತ್ತದೆ’ ಎಂದು ಪ್ಲುಟಾರ್ಕ್ ಎಷ್ಟು ಚೆನ್ನಾಗಿ ಹೇಳಿದ್ದಾನೆ! ಸಿಟ್ಟಿನ ಕುರಿತು ಒಂದು ಶ್ಲೋಕ ಹೀಗೆನ್ನುತ್ತದೆ:

‘ಉತ್ತಮರ ಕೋಪ ಒಂದು ಕ್ಷಣಕಾಲವಿರುತ್ತದೆ. ಮಧ್ಯಮರದು ಎರಡು ಘಂಟೆಯ ತನಕ, ಅಧಮರು ದಿನವಿಡೀ ಕೋಪಗೊಂಡರೆ, ದುರಾತ್ಮರು ಜೀವಮಾನವೆಲ್ಲ ಕೋಪವನ್ನುಳಿಸಿ\-ಕೊಳ್ಳುತ್ತಾರೆ.’ ಅಂದರೆ, ಕೋಪವನ್ನು ದೀರ್ಘಾಯುವಾಗಲು ಬಿಡಬೇಡಿ.

ಡಾ.\ ಜಾನ್ಸನರ ಜೀವನಚರಿತ್ರೆಯ ಲೇಖಕ ಬಾಸ್ವೆಲ್​ನಿಗೆ, ಅವನ ಸಂಗಾತಿಯೊಬ್ಬ ಒಮ್ಮೆ ಅಪಮಾನಸೂಚಕ ಮಾತು ಹೇಳಿದ. ಸಿಟ್ಟಿನಿಂದ ಕಿಡಿಕಿಡಿಯಾಗಿ ಆತ ಜಾನ್ಸನರಿಗೆ ಈ ಬಗ್ಗೆ\break ದೂರಿದ. ‘ಇಂದಿಗೆ ಒಂದು ವರ್ಷದ ಬಳಿಕ ಈ ಸಂಗತಿ ಅದೆಷ್ಟು ತುಚ್ಛವೆನಿಸೀತು ಎಂಬುದನ್ನು ಯೋಚಿಸಿ ನೋಡು’ ಎಂದರು ಜಾನ್ಸನ್.

ಬಾಸ್ವೆಲ್ ಗಂಭೀರವಾಗಿ ಯೋಚಿಸಿದ, ಅವನ ಸಿಟ್ಟು ಶಾಂತವಾಯಿತು. ಈ ಘಟನೆಯನ್ನು ಪ್ರಸ್ತಾಪಿಸುತ್ತ ಅವನು ಹೀಗೆ ಬರೆದಿದ್ದಾನೆ: ‘ಡಾ.\ ಜಾನ್ಸನರ ಈ ಉಪದೇಶವನ್ನು ನಾನು ಇನ್ನೂ ಹಲವು ಸಲ ಬಳಸಿನೋಡಿದೆ; ಅದರಿಂದ ನನಗೆ ಯಾವಾಗಲೂ ಲಾಭ ದೊರೆತಿದೆ.’

ಸ್ವಾಮಿ ವಿವೇಕಾನಂದರು ರೈಲಿನಲ್ಲಿ ಕುಳಿತು ಎಲ್ಲಿಗೋ ಹೋಗುತ್ತಿದ್ದರು. ಇಬ್ಬರು ಆಂಗ್ಲರು ಅದೇ ಡಬ್ಬಿಯಲ್ಲಿ ಕುಳಿತು ಪ್ರಯಾಣ ಮಾಡುತ್ತಿದ್ದರು. ಅವರು ಆಂಗ್ಲ ಭಾಷೆಯಲ್ಲೇ\break ವಿವೇಕಾನಂದರ ಸಂನ್ಯಾಸಿ ಉಡುಗೆಯನ್ನು ಗೇಲಿ ಮಾಡುತ್ತಿದ್ದರು. ಇದನ್ನೆಲ್ಲ ಅರಿತರೂ ಸ್ವಾಮೀಜಿ ಶಾಂತಭಾವದಿಂದ ಕುಳಿತಿದ್ದರು.

\newpage

ಒಂದು ನಿಲ್ದಾಣದಲ್ಲಿ ಗಾಡಿ ನಿಂತಾಗ, ಸ್ವಾಮೀಜಿ ಟಿಕೆಟ್ ಕಲೆಕ್ಟರ್​ನೊಡನೆ ಇಂಗ್ಲೀಷಿನಲ್ಲೇ ಏನೋ ಹೇಳಿದರು. ಅವರು ಇಂಗ್ಲೀಷಿನಲ್ಲಿ ಮಾತನಾಡುವುದನ್ನು ಕಂಡು ಆಂಗ್ಲರಿಗೆ ತುಂಬಾ ನಾಚಿಕೆಯಾಯಿತು. ಗಾಡಿ ಹೊರಟಾಗ ಅವರಲ್ಲೊಬ್ಬರು ಸ್ವಾಮೀಜಿಯೊಡನೆ–‘ನಿಮಗೆ ಇಂಗ್ಲೀಷ್ ಬರುತ್ತದೆ. ಇಷ್ಟಾದರೂ ನೀವು ನಮ್ಮ ಮಾತು ಕೇಳಿ ಸಿಟ್ಟಾಗಲಿಲ್ಲ. ನೀವೇಕೆ ಸುಮ್ಮನಿದ್ದಿರಿ?’\break ಎಂದರು.

‘ಇಂತಹ ಮಾತುಗಳನ್ನು ಈ ಹಿಂದೆಯೂ ಎಷ್ಟೋ ಸಲ ಕೇಳಿರುವೆ. ಅಜ್ಞಾನಿಗಳ ಮೇಲೆ ಸಿಟ್ಟಾಗಿ ನಾನು ನನ್ನ ಶಕ್ತಿ ಹಾಗೂ ಶಾಂತಿಯನ್ನು ಅದೇಕೆ ಕೆಡಿಸಿಕೊಳ್ಳಲಿ?’ ಎಂದು ಶಾಂತಭಾವದಿಂದ ನುಡಿದರು ಸ್ವಾಮೀಜಿ.


\section*{ಬಚಾವಾಗಲು ಬುಸುಗುಟ್ಟು}

\addsectiontoTOC{ಬಚಾವಾಗಲು ಬುಸುಗುಟ್ಟು}

ಮಗುವಿನ ಕೈಯಿಂದ ಆಟದ ಸಾಮಾನನ್ನು ಸೆಳೆದುಕೊಳ್ಳಲು ಯತ್ನಿಸಿ. ಅದು ತನ್ನ ಅಸಮಾಧಾನವನ್ನು ತೋರಿಸುತ್ತದೆ. ಸಿಟ್ಟು ಒಂದು ಹುಟ್ಟುಗುಣ, ಅದೊಂದು ಸ್ವಾಭಾವಿಕ ಪ್ರತಿಕ್ರಿಯೆ. ನಮ್ಮ ಇಚ್ಛೆಗೆ ವಿರುದ್ಧವಾಗಿ ಏನಾದರೂ ಸಂಭವಿಸಿದಾಗಲೆಲ್ಲಾ ನಾವು ಸಿಟ್ಟಿಗೇಳುತ್ತೇವೆ. ಜಗತ್ತಿನಲ್ಲಿ ಎಲ್ಲವೂ ನಮ್ಮ ಇಚ್ಛಾಧೀನವಾಗಿ ನಡೆಯುವುದಿಲ್ಲ. ನಮ್ಮ ಇಚ್ಛೆಗಳ ಮೇಲೆ ನಿಯಂತ್ರಣ ಹಾಕಿಕೊಳ್ಳಲು ಕಲಿತರೆ, ಸಿಟ್ಟು ನಮ್ಮನ್ನು ತನ್ನ ಸೇವಕನನ್ನಾಗಿ ಮಾಡದು.

ಹಲವರು ಸಿಟ್ಟನ್ನು ಚಟವಾಗಿ ಬೆಳೆಸಿಕೊಳ್ಳುತ್ತಾರೆ. ಮಾತು ಮಾತಿಗೂ ಸಿಟ್ಟಿಗೇಳುತ್ತಾರೆ. ಮರದ ಮೇಲೆ ಕಾಗೆ ಕರ್ಕಶವಾಗಿ ಕೂಗಿದಾಗ ಸಿಟ್ಟಾಗು\-ತ್ತಾರೆ. ಮಗು ಕಿರಿಚಿಕೊಂಡಾಗ ಸಿಟ್ಟಾಗು\-ತ್ತಾರೆ. ಸ್ನಾನದ ಬಟ್ಟೆ ದೊರೆಯದಾಗಲೂ, ಕಾಫಿಗೆ ಸಕ್ಕರೆ ಕಡಿಮೆಯಾದಾಗಲೂ ಅವರಿಗೆ ಸಿಟ್ಟು. ‘ಸಿಟ್ಟು ಬಂದರೆ ಹುಲಿ’ ಎನ್ನಿಸಿಕೊಳ್ಳುವುದರಲ್ಲಿ ಅವರಿಗೆ ಹೆಮ್ಮೆ!

ನೀವು ಸಿಟ್ಟು ಮಾಡುವುದರ ಮೊದಲು ಈ ಕೆಳಗಿನ ಮಾತುಗಳನ್ನು ಜ್ಞಾಪಿಸಿಕೊಳ್ಳಿ–

ಸಿಟ್ಟು ನಿಮ್ಮ ದೈಹಿಕ ಮಾನಸಿಕ ಸಮತ್ವವನ್ನು ನಾಶಗೊಳಿಸುವುದು. ಅದರಿಂದಾಗಿ ಇತರ\-ರೊಡನೆ ಸುಮ್ಮನೆ ದ್ವೇಷ ಕಟ್ಟಿಕೊಳ್ಳಬೇಕಾಗುವುದು. ಸಿಡುಕಿನ ನಿಮ್ಮ ವರ್ತನೆ ಇತರರಿಗೆ ಅಸಹ್ಯವಾಗುವುದು.

ಜನರು ನಿಮ್ಮನ್ನು ಹಿಂದಿನಿಂದ ಅಪಹಾಸ್ಯ ಮಾಡುವರು. ನಿಮಗೆ ಯಾರ ಸಹಾನು\-ಭೂತಿಯೂ ದೊರಕದು. ನಿಮ್ಮ ಸುತ್ತಮುತ್ತಲಿರುವವರಿಗೆ ನಿಮ್ಮಿಂದ ತುಂಬ ತೊಂದರೆಯಾಗುವುದು.

ಸಿಟ್ಟನ್ನು ಕುರಿತು ಬಸವಣ್ಣನವರ ವಾಣಿ ನೆನಪಿರಲಿ–

‘ಮನೆಯೊಳಗಣ ಕಿಚ್ಚು ಮನೆಯನ್ನೇ ಸುಡುವಂತೆ ತನ್ನಲ್ಲಿ ಹುಟ್ಟಿದ ಕೋಪ ತನ್ನನ್ನೇ ಸುಡುವು\-ದಲ್ಲದೆ ಬಿಡದು.’

\newpage

ಆದರೆ ಸ್ವಲ್ಪವೂ ಸಿಟ್ಟು ಇಲ್ಲದೆ ವ್ಯವಹಾರ ನಡೆಯುವುದೇ? ಭಗವಾನ್ ಶ‍್ರೀರಾಮಕೃಷ್ಣರನ್ನು ಒಮ್ಮೆ ಭಕ್ತನೊಬ್ಬ ಕೇಳಿದ: ‘ಮಹಾಶಯರೆ, ದುಷ್ಟವ್ಯಕ್ತಿಗಳು ನಮಗೆ ಕೇಡು ತರಲು ಉದ್ಯುಕ್ತ\-ರಾಗಿದ್ದರೂ, ನಾವು ಸುಮ್ಮನೆ ಕೈ ಕಟ್ಟಿ ಕೂತಿರಬೇಕೆ?’ ಶ‍್ರೀರಾಮಕೃಷ್ಣರು:‘ಸಮಾಜದ ಮಧ್ಯೆ ಇರಬೇಕಾದರೆ, ದುಷ್ಟರ ಕೈಯಿಂದ ತಪ್ಪಿಸಿಕೊಳ್ಳಲು ಮನುಷ್ಯ ಸ್ವಲ್ಪ ತಮೋಗುಣವನ್ನು ಪ್ರದರ್ಶಿಸ\-ಬೇಕಾಗುತ್ತದೆ. ಆದರೆ ಮುಯ್ಯಿಗೆ ಮುಯ್ಯಿ ತೀರಿಸಲು ಹೋಗಬಾರದು.’

‘ಒಂದು ಕಥೆ ಕೇಳು–ಕೆಲವು ಗೊಲ್ಲರ ಹುಡುಗರು ಒಂದು ಹುಲ್ಲುಗಾವಲಿನಲ್ಲಿ ದನ ಕಾಯು\-ತ್ತಿದ್ದರು. ಅಲ್ಲಿ ಒಂದು ಭಯಂಕರ ವಿಷಸರ್ಪ ವಾಸವಾಗಿತ್ತು. ಆ ಹಾವಿಗೆ ಹೆದರಿ ಎಲ್ಲರೂ ಬಹಳ ಎಚ್ಚರಿಕೆಯಿಂದ ಇರುತ್ತಿದ್ದರು. ಒಂದು ದಿನ ಒಬ್ಬ ಬ್ರಹ್ಮಚಾರಿ ಆ ಹುಲ್ಲುಗಾವಲು ಮಾರ್ಗವಾಗಿ ಹೋಗುತ್ತಿದ್ದ. ಹುಡುಗರು ಓಡಿ ಹೋಗಿ ಆತನಿಗೆ ಹೇಳಿದರು: “ಮಹಾಶಯರೆ, ಈ ಮಾರ್ಗವಾಗಿ ಹೋಗಬೇಡಿ. ಅಲ್ಲಿ ಭಯಂಕರ ವಿಷಸರ್ಪವೊಂದಿದೆ.” ಬ್ರಹ್ಮಚಾರಿ ಹೇಳಿದ: ‘ಗೆಳೆಯರಾ, ಇದ್ದರೆ ಇರಲಿ. ನನಗೆ ಅದನ್ನು ಪಳಗಿಸುವ ಮಂತ್ರ ಗೊತ್ತಿದೆ, ಹಾಗಾಗಿ ಹೆದರಿಕೆ ಇಲ್ಲ.’ ಹೀಗೆಂದು ಹೇಳಿ ಆತ ಹಾಗೇ ಮುಂದುವರಿದ. ಸ್ವಲ್ಪ ದೂರ ಹೋಗುವುದರೊಳಗೇ ಆ ಹಾವು ಹೆಡೆಬಿಚ್ಚಿ ಓಡಿಬರಲಾರಂಭಿಸಿತು. ಹತ್ತಿರಕ್ಕೆ ಬಂದೊಡನೆ ಆತ ಏನೋ ಒಂದು ಮಂತ್ರ ಪಠಿಸಿದ. ಹಾವು ಎರೆಹುಳುವಿನೋಪಾದಿಯಲ್ಲಿ ಆತನ ಪಾದದ ಬಳಿ ಸುಮ್ಮನೆ ಬಿದ್ದುಕೊಂಡಿತು. ಬ್ರಹ್ಮಚಾರಿ ಹೇಳಿದ: ‘ಕೇಳಿಲ್ಲಿ, ಪರರಿಗೆ ಹಿಂಸೆಮಾಡುತ್ತ ಏಕೆ ನೀನು ಸುತ್ತಾಡುತ್ತಿದ್ದೀಯೆ? ನಿನಗೆ ಮಂತ್ರ ಕೊಡುತ್ತೇನೆ. ಅದನ್ನು ಜಪಿಸಿದರೆ ನಿನಗೆ ಭಗವಂತನಲ್ಲಿ ಭಕ್ತಿಯುಂಟಾಗುತ್ತದೆ. ಆತನ ಸಾಕ್ಷಾತ್ಕಾರ ದೊರಕುತ್ತದೆ; ಹಿಂಸಾ ಪ್ರವೃತ್ತಿ ಬಿಟ್ಟು ತೊಲಗುತ್ತದೆ.’ ಹೀಗೆಂದು ಹೇಳಿ ಮಂತ್ರೋಪದೇಶ ಮಾಡಿದ. ಮಂತ್ರ ಪಡೆದ ನಂತರ ಆ ಹಾವು ಗುರುವಿಗೆ ಪ್ರಣಾಮ ಮಾಡಿ ಆತನನ್ನು ಕೇಳಿಕೊಂಡಿತು: ‘ಪೂಜ್ಯರೆ, ಯಾವ ರೀತಿ ಸಾಧನೆಯಲ್ಲಿ ತೊಡಗಬೇಕು. ದಯವಿಟ್ಟು ತಿಳಿಸಿ.’ ಗುರು ಹೇಳಿದ: ‘ಈ ಮಂತ್ರ ಜಪಿಸು, ಯಾರಿಗೂ ಹಿಂಸೆ ಮಾಡಬೇಡ.’ ಹೊರಡುವಾಗ ಹೇಳಿದ: ‘ಮತ್ತೆ ಇನ್ನೊಮ್ಮೆ ಬಂದು ನಿನ್ನನ್ನು ನೋಡುತ್ತೇನೆ.’

‘ಹಾಗೆ ಕೆಲವು ದಿನ ಉರುಳಿದವು. ಆ ಹುಡುಗರಿಗೆ ಗೊತ್ತಾಯಿತು. ಆ ಹಾವು ಕಡಿಯುವು\-ದಿಲ್ಲ ಎಂಬುದಾಗಿ. ಒಂದು ದಿನ ಒಬ್ಬ ಹುಡುಗ ಅದರ ಹತ್ತಿರಕ್ಕೆ ಹೋಗಿ, ಬಾಲ ಹಿಡಿದು ಚೆನ್ನಾಗಿ ತಿರುಗಿಸಿ, ನೆಲಕ್ಕೆ ಅಪ್ಪಳಿಸಿ ಒಂದು ಕಡೆಗೆ ಬಿಸಾಡಿಬಿಟ್ಟ. ಅದರ ಬಾಯಿಂದ ರಕ್ತ ಸುರಿಯ ಲಾರಂಭಿಸಿತು; ಪ್ರಜ್ಞೆತಪ್ಪಿ ಬಿದ್ದುಕೊಂಡಿತು; ಚಲನವಲನವೆಲ್ಲ ನಿಂತುಹೋಯಿತು. ಆ ಹಾವು ಸತ್ತಿತೆಂದು ಭಾವಿಸಿ ಹುಡುಗರೆಲ್ಲರೂ ಹೊರಟುಹೋದರು. ಆದರೆ ಅದು ಸಾಯಲಿಲ್ಲ. ಹಗಲು ವೇಳೆ ಹುಡುಗರಿಗೆ ಹೆದರಿ ಹುತ್ತದಿಂದ ಹೊರಕ್ಕೆ ಬರುತ್ತಲೇ ಇರಲಿಲ್ಲ. ರಾತ್ರಿ ವೇಳೆ ತನ್ನ ಆಹಾರಕ್ಕಾಗಿ ಆಗಾಗ ಹೊರಬರುತ್ತಿತ್ತು. ಮಂತ್ರ ಪಡೆದಂದಿನಿಂದ ಯಾರಿಗೂ ಹಿಂಸೆ ಮಾಡುತ್ತಿರಲಿಲ್ಲ. ಕಸ ಕಡ್ಡಿ, ಎಲೆ, ಮರದಿಂದ ಬಿದ್ದ ಹಣ್ಣುಹಂಪಲು–ಇವನ್ನು ತಿಂದೇ ಪ್ರಾಣಧಾರಣೆ ಮಾಡಿಕೊಳ್ಳುತ್ತಿತ್ತು.

\newpage

‘ಸುಮಾರು ಒಂದು ವರ್ಷ ಕಳೆದ ನಂತರ, ಆ ಬ್ರಹ್ಮಚಾರಿ ಅದೇ ಮಾರ್ಗವಾಗಿ ಬಂದು, ಆ ಹಾವನ್ನು ಹುಡುಕಿದ. ಗೊಲ್ಲ ಹುಡುಗರು: ‘ಆ ಹಾವು ಸತ್ತುಹೋಯಿತು’ ಎಂದು ಹೇಳಿದರು. ಬ್ರಹ್ಮಚಾರಿ ನಂಬಲಿಲ್ಲ. ಆತನಿಗೆ ಗೊತ್ತಿತ್ತು–ಮಂತ್ರ ಸಿದ್ಧಿಯಾದ ಹೊರತು ಅದು ಸಾಯುವು\-ದಿಲ್ಲ ಎಂಬುದು. ಹಾಗೇ ಹುಡುಕಿಕೊಂಡು, ಹುಡುಕಿಕೊಂಡು ಅದು ವಾಸವಾಗಿದ್ದ ಸ್ಥಳಕ್ಕೆ ಹೋಗಿ ತಾನು ಇಟ್ಟಿದ್ದ ಹೆಸರಿನಿಂದ ಅದನ್ನು ಕೂಗಲಾರಂಭಿಸಿದ. ಗುರುವಿನ ಧ್ವನಿಯನ್ನು ಕೇಳಿ, ಅದು ಹುತ್ತದಿಂದ ಹೊರಕ್ಕೆ ಬಂದು, ಅತ್ಯಂತ ಶ್ರದ್ಧಾಭಕ್ತಿಯಿಂದ ಆತನಿಗೆ ಪ್ರಣಾಮ ಮಾಡಿತು. ಬ್ರಹ್ಮಚಾರಿ ಕೇಳಿದ: ‘ಕ್ಷೇಮವೇ?’ ಅದು ಹೇಳಿತು: ‘ಹೌದು, ಕ್ಷೇಮದಿಂದಿದ್ದೇನೆ.’ ಬ್ರಹ್ಮಚಾರಿ ಕೇಳಿದ: ‘ಆದರೆ ನೀನೇಕೆ ಇಷ್ಟು ಸೊರಗಿ ಹೋಗಿದ್ದೀಯೆ?’ ಹಾವು ಹೇಳಿತು: ‘ಪೂಜ್ಯರೆ, ತಾವು ಉಪದೇಶವಿತ್ತಿದ್ದೀರಿ, ಯಾರಿಗೂ ಹಿಂಸೆ ಮಾಡಕೂಡದು ಎಂಬುದಾಗಿ. ಅದ ಕ್ಕಾಗಿ ಎಲೆ, ಹಣ್ಣು, ಹಂಪಲು ತಿನ್ನುತ್ತಿರುವುದರಿಂದ ಶರೀರ ಬಹುಶಃ ಕೃಶವಾಗಿ ತೋರುತ್ತಿರಬಹುದು.’

‘ಬ್ರಹ್ಮಚಾರಿ ಹೇಳಿದ: ‘ಕೇವಲ ಆಹಾರಾಭಾವದಿಂದಲೆ ಈ ದುರವಸ್ಥೆ ಉಂಟಾಗದು. ನಿಶ್ಚಯವಾಗಿ ಬೇರೆ ಏನೋ ಕಾರಣವಿರಬೇಕು, ಯೋಚಿಸಿ ನೋಡು.’ ಗೊಲ್ಲರ ಹುಡುಗರು ತನ್ನನ್ನು ಚೆನ್ನಾಗಿ ನೆಲಕ್ಕೆ ಅಪ್ಪಳಿಸಿದ್ದರ ಜ್ಞಾಪಕ ಅದಕ್ಕೆ ಬಂತು. ಅದು ಹೇಳಿತು: ‘ಪೂಜ್ಯರೆ, ಈಗ ನನ್ನ ಜ್ಞಾಪಕಕ್ಕೆ ಬಂತು. ಒಂದು ದಿನ ಆ ಗೊಲ್ಲರ ಹುಡುಗರು ಚೆನ್ನಾಗಿ ತಿರುಗಿಸಿ ನನ್ನನ್ನು ನೆಲಕ್ಕೆ ಅಪ್ಪಳಿಸಿದರು. ಏನೆಂದರೂ ಅವರಿನ್ನೂ ಅರಿಯದ ಹುಡುಗರು. ನನ್ನ ಮನಸ್ಸಿನಲ್ಲಿ ಈಗ ಎಂಥ ಬದಲಾವಣೆ ಆಗಿಬಿಟ್ಟಿದೆ ಎಂಬುದು ಅವರಿಗೆ ಗೊತ್ತಾಗಲಿಲ್ಲ. ನಾನು ಯಾರನ್ನೂ ಕಚ್ಚುವುದಿಲ್ಲ. ಬೇರೆ ಯಾವ ವಿಧದಿಂದಲೂ ಹಿಂಸೆ ಮಾಡುವುದಿಲ್ಲ ಎಂಬುದು ಅವರಿಗೆ ಹೇಗೆ ಗೊತ್ತಾಗ ಬೇಕು?’ ಬ್ರಹ್ಮಚಾರಿ ಬಯ್ದು ಹೇಳಿದ: ‘ಛೆ! ನೀನು ಎಂಥಾ ತಿಳಿಗೇಡಿ! ನಿನ್ನನ್ನು ನೀನು ಹೇಗೆ ರಕ್ಷಿಸಿಕೊಳ್ಳಬೇಕು ಎಂಬುದು ನಿನಗೆ ಗೊತ್ತಿಲ್ಲವಲ್ಲ. ಕಚ್ಚಬೇಡ ಎಂದು ಹೇಳಿದೆನೇ ವಿನಾ ಬುಸುಗುಟ್ಟಬೇಡ ಎಂದು ಹೇಳಲಿಲ್ಲವಲ್ಲ. ಬುಸುಗುಟ್ಟಿ ಏಕೆ ಅವರನ್ನು ಹೆದರಿಸಬಾರದಾಗಿತ್ತು?’

‘ದುಷ್ಟರ ಕಡೆ ಬುಸುಗುಟ್ಟಬೇಕು. ಕೇಡು ಮಾಡದೆ ಇರಲೆಂದು ಅವರಿಗೆ ಭಯ ತೋರಿಸಬೇಕು.’


\section*{ಅಂತಕನ ದೂತರಿಗೆ...}

\addsectiontoTOC{ಅಂತಕನ ದೂತರಿಗೆ . . .}

ಹರೆಯ ಹದಿನೈದು. ಸುಂದರ, ಬುದ್ಧಿವಂತ, ಹುಡುಗನಾತ. ತುಂಟರ ದುಷ್ಟಕೂಟಕ್ಕೆ ಸೇರಿದವನಲ್ಲ. ಓದಿನಲ್ಲಿ ಹಿಂದೆ ಬಿದ್ದವನಲ್ಲ. ತಂದೆತಾಯಿಗಳ ಏಕಮಾತ್ರ ಪುತ್ರ. ಅವರಿಗೆ ವಿಧೇಯ. ಅವನ ಸದ್ಗುಣ ಮತ್ತು ವ್ಯಕ್ತಿತ್ವಗಳಿಂದಾಗಿ, ಮಿತ್ರರು ಅವನನ್ನು ಗೌರವದಿಂದ ನೋಡುತ್ತಲಿದ್ದರು. ಒಂದು ದಿನ ಮಿತ್ರರ ಜೊತೆ ಪ್ರವಾಸ ಹೋಗಿ ಬರುತ್ತೇನೆಂದು ಹೊರಟ. ಯುವ ವಿದ್ಯಾರ್ಥಿಗಳ ಕೂಟದಲ್ಲಿ ಆತನನ್ನು ಕಾಣದಿದ್ದಾಗ, ವಿದ್ಯಾರ್ಥಿ ಮುಖಂಡ ‘ಆತನೆಲ್ಲಿ?’ ಎಂದು ಪ್ರಶ್ನಿಸಿದ. ಉಳಿದವರೂ ‘ಆತನೆಲ್ಲಿ?’ ಎಂದು ಆಶ್ಚರ್ಯಚಕಿತರಾಗಿ ಉದ್ಗರಿಸಿದರು. ಸರಿ, ಹುಡುಕಾಟ ಪ್ರಾರಂಭ\-ವಾಯಿತು. ನದಿತೀರಕ್ಕೆ ಈಜಲು ಹೋಗಿದ್ದಾನೆಂದು ತಿಳಿಯಿತು. ಈಜಲು ಹೋದವನು ನದಿಯಿಂದ ಮೇಲೆ ಬರಲಿಲ್ಲ! ಕಷ್ಟಪಟ್ಟು ಆತನ ಶವವನ್ನು ಹೊರತರಲಾಯಿತು. ದುರ್ಮರಣದ ವಾರ್ತೆಯನ್ನು ಆತನ ತಾಯಿಗೆ ತಿಳಿಸಿದರು. ತಂದೆಗೂ ವಿಚಾರ ತಿಳಿಯಿತು. ತಂದೆ ತಲೆ ಹೊಡೆದುಕೊಂಡು ಕರುಳು ಕಿತ್ತುಬರುವಂತೆ ಅಳುತ್ತ ಕೂತರು. ನಯವಿನಯ ಸಂಪನ್ನನಾದ ತನ್ನ ಮಗನ ಒಂದೊಂದೇ ಗುಣಗಳನ್ನು ನೆನೆಸಿಕೊಂಡು ಹಲಬುತ್ತ ಹೊರಳಾಡಿದರು. ತಾಯಿ ಗಂಭೀರವಾಗಿದ್ದಳು. ಮೌನವಾಗಿ ದುಃಖವನ್ನು ನುಂಗಿಕೊಂಡಂತಿತ್ತು. ಮಗನನ್ನು ಪ್ರವಾಸಕ್ಕೆ ಕೊಂಡೊಯ್ದ ಯಾರ ಬಗೆಗೂ, ಒಂದೂ ಕಟುನುಡಿಯನ್ನಾಡಲಿಲ್ಲ. ಗಂಡನನ್ನು ತಾನೇ ಸಮಾಧಾನ\-ಪಡಿಸಿದಳು. ದೃಢಮನಸ್ಕಳಾಗಿದ್ದುಕೊಂಡು, ಫೋನುಮಾಡಿ ಬಂಧುಬಾಂಧವರಿಗೂ, ಪೋಲೀಸ\-ರಿಗೂ, ತನ್ನ ಪತಿಯ ಆಫೀಸಿನ ಅಧಿಕಾರಿಗಳಿಗೂ ವಿಚಾರ ತಿಳಿಸಿ ಶವವನ್ನು ಸ್ಮಶಾನಕ್ಕೆ ಸಾಗಿಸಲು ವ್ಯವಸ್ಥೆ ಮಾಡಿದಳು. ಶವಕ್ಕೆ ಬೆಂಕಿ ಇಟ್ಟಾಗ ಮಾತ್ರ ಆ ತಾಯಿ ಶೋಕವನ್ನು ತಡೆಯಲಾರದೇ ಕುಸಿದು ಬಿದ್ದಳು.

ಇನ್ನೊಬ್ಬಾತತ ಇಂಜಿನಿಯರಿಂಗ್ ಕಲಿಯುತ್ತಿದ್ದ. ನಕ್ಕು ನಗಿಸುವುದು ಅವನ ಸ್ವಭಾವವಾಗಿತ್ತು. ಅವನಿನ್ನೂ ಮಕ್ಕಳಾಟಿಕೆ ಬಿಡಲಿಲ್ಲವೆಂದೆಣಿಸಿ ತಾಯಿ ಆತನನ್ನು ಪ್ರಶ್ನಿಸಿದಳು. ‘ಏಕೆ ಅಷ್ಟೊಂದು ನಗುತ್ತಿ? ಏನಾಗಿದೆ ನಿನಗೆ?’ ಆತ ನಗುತ್ತಲೇ ಉತ್ತರಿಸಿದ: ‘ನಾಳೆ ಸತ್ತುಹೋದೆ ಎಂದಿಟ್ಟುಕೊ, ಯಮನು ಆ ಲೋಕದಲ್ಲಿ ಏನು ಮಾಡಿದೆ? ಎಂದು ಕೇಳಿದರೆ, ಜನರನ್ನು ಚೆನ್ನಾಗಿ\break ನಗಿಸಿದ್ದೇನೆ ಎನ್ನುತ್ತೇನೆ’ ಎಂದ. ಏನಾಶ್ಚರ್ಯ! ಮಾರನೇ ದಿನ, ಸೈಕಲ್ ಏರಿ ಪೇಟೆಯಲ್ಲಿ ಹೋಗುತ್ತಿದ್ದಾಗ, ಲಾರಿಗೆ ಢಿಕ್ಕಿ ಹೊಡೆದು, ಮೃತನಾದ! ಅಂತಕನ ದೂತರಿಗೆ ಕಿಂಚಿತ್ತೂ ದಯವಿಲ್ಲ. ನೋವೇ ಅವರ ಗುರಿಯೇ?

ತ್ರಿವೇಂಡ್ರಂನ ಶ‍್ರೀ ನಾಯರ್, ವಿಶ್ವವಿದ್ಯಾಲಯದಲ್ಲಿ ಮಲೆಯಾಳಂ ಬೋಧಿಸುವ ಪ್ರೊಫೆಸರ್, ಸಂಸ್ಕೃತ ಬಲ್ಲವರು. ಆದರೆ ದೇವರಲ್ಲಿ ವಿಶ್ವಾಸವಿಲ್ಲದವರು. ವಿವಾಹವಾಗಿ ಬಹುಕಾಲದ ನಂತರ ಅವರಿಗೊಬ್ಬ ಮಗ ಜನಿಸಿದ. ಆರನೇ ವಯಸ್ಸಿನಲ್ಲಿ ಆತ ಅಸ್ವಸ್ಥನಾದ. ಪ್ರಾಣ ಬಿಡುವುದಕ್ಕೆ ಮೊದಲು, ತಂದೆಯನ್ನು ಹತ್ತಿರ ಕರೆಯುವಂತೆ ತಾಯಿಯನ್ನು ಕೇಳಿಕೊಂಡ. ತಂದೆ ಹತ್ತಿರ ಬಂದೊಡನೆ, ಅವರನ್ನೇ ನೋಡುತ್ತ ಯಾರೂ ಈವರೆಗೆ ಹೇಳಿಕೊಡದೇ ಇದ್ದ ದೇವರ ನಾಮವನ್ನು ಹದಿನೈದು ನಿಮಿಷಗಳ ಕಾಲ, ಗಟ್ಟಿಯಾಗಿ ಉಚ್ಚರಿಸುತ್ತ ಪ್ರಾಣಬಿಟ್ಟ. ಶ‍್ರೀ ನಾಯರ್ ಚಕಿತ\-ರಾದರು, ಸ್ತಂಭಿತರಾದರು. ಈ ಘಟನೆ, ಅವರ ಬದುಕಿಗೆ ಒಂದು ತಿರುವನ್ನು ನೀಡಿತು. ಸಾವು ಅವರಿಗೆ ಪಾಠ ಕಲಿಸಿತು. ಇಂದು ಅವರು ಆಸ್ತಿಕರು.

ಹೇಳದೆ ಕೇಳದೆ ಸಾವು ಅಕಸ್ಮಾತ್ತಾಗಿ ಬಂದುಬಿಡುತ್ತದೆ. ಆಗ ಮನುಷ್ಯನ ಅಸಹಾಯಕತೆ ಪರಮಾವಧಿಗೇರುತ್ತದೆ. ಬಾಳಿನ ಭರವಸೆಯೆಲ್ಲ ಕ್ಷಣಾರ್ಧದಲ್ಲಿ ಮಾಯವಾಗುವಾಗ, ಎಂಥ ದೃಢಹೃದಯವೂ ತತ್ತರಿಸುತ್ತದೆ. ಈ ಹೊತ್ತು ಜೀವಂತವಾಗಿರುವ ವ್ಯಕ್ತಿಯೇ, ನಾಳೆ ಮಣ್ಣಿನಲ್ಲಿ ಮಣ್ಣಾಗುವುದನ್ನು ಕಣ್ಣಾರೆ ಕಾಣುವಾಗ, ಎಂಥ ಎದೆಯೂ ತಲ್ಲಣಿಸುತ್ತದೆ. ಅತ್ಯಂತ ಪ್ರೀತಿಯಿಂದ ನೋಡಿಕೊಂಡ ಜನಗಳೇ, ಮೃತವ್ಯಕ್ತಿಯನ್ನು ಕಲ್ಲು ಕೊರಡನ್ನು ತ್ಯಜಿಸುವಂತೆ ಬಿಟ್ಟು\break ಹೋಗುತ್ತಾರಲ್ಲ!


\section*{ಸಾವು ನೋವು}

\addsectiontoTOC{ಸಾವು ನೋವು}

ಏನಿದು ಈ ಸಾವು? ಭಯಾನಕವಲ್ಲವೆ?

ಹೌದು, ಅಲ್ಲ–ಎಂದರು ಋಷಿಗಳು.

ಅಜ್ಞಾನಿಗಳಿಗೆ ಭಯಾನಕ, ಜ್ಞಾನಿಗೆ ಅಲ್ಲ ಎಂದರು–ಆ ಅನುಭಾವೀ ಮನೀಷಿಗಳು. ಸಾವು ಜೀವನದ ಒಂದು ಅತ್ಯಂತ ಪ್ರಮುಖ ಘಟನೆ. ಆದರೂ ಅದನ್ನು ಕುರಿತು ಬುದ್ಧಿವಂತ, ವಿದ್ಯಾವಂತ ಜನಗಳೇ ಯೋಚಿಸುತ್ತಿಲ್ಲ. ಅದರ ಬಗ್ಗೆ ನಮ್ಮ ತಿಳಿವಳಿಕೆ ಎಷ್ಟೊಂದು ಅಲ್ಪ\break ಹಾಗೂ ಸೀಮಿತ! ಅಂಥ ಸೀಮಿತ ಜ್ಞಾನವೇ ಬುದ್ಧಿವಂತಿಕೆಯ ಲಕ್ಷಣ ಎಂಬ ಛಲ ಬೇರೆ!

ಮರಣ ಹಾಗೂ ಮರಣಾತೀತ ಅಸ್ತಿತ್ವದ ಗುಟ್ಟೆಲ್ಲ ತಮಗೆ ಗೊತ್ತಿದೆ ಎನ್ನುವ ಮತ ಪಂಡಿತರು. ‘ದೇಣಿಗೆ, ಕಾಣಿಕೆಗಳನ್ನು ಸಲ್ಲಿಸಿ, ನಿಮ್ಮ ಪರಲೋಕದ ಸ್ಥಾನವನ್ನು ರಿಸರ್ವ್ ಮಾಡಿಸಿ, ಈಗಾಗಲೆ ಭದ್ರಪಡಿಸಿಕೊಳ್ಳಿ’–ಎಂದಾರೆ ಹೊರತು, ಈ ಕುರಿತು ಚಿಂತನೆಗೆ ನಿಮ್ಮನ್ನು ಪ್ರೇರಿಸಲಾರರು.

ದಿನನಿತ್ಯವೂ ಇತರರು ಸಾಯುತ್ತಿರುವುದನ್ನು ನೋಡುವ ಮನುಷ್ಯ, ಪರಿಚಿತರು ತೀರಿ ಕೊಂಡಾಗ ‘ಅಯ್ಯೋ ಹೋದರೇ!’ ಎಂದು ಒಂದು ಕ್ಷಣ ಚಕಿತನಾಗುತ್ತಾನೆ. ಸ್ವಲ್ಪ ಸಂತಾಪವನ್ನೂ ಪಡಬಹುದು. ಬಳಿಕ ‘ಎಲ್ಲರೂ ಒಂದು ದಿನ ಹೋಗಬೇಕಲ್ಲ’ ಎಂಬ ಸಾಂತ್ವನದ ಮಾತನ್ನು ಮೆಲುಕಾಡಬಹುದು. ಆದರೆ ಸಾವು ಏನು? ಸಾವಿನ ಆಚೆಗೆ ಏನು?–ಎಂಬುದನ್ನು ಆತ ಯೋಚಿಸಲಾರ. ಸಮೀಪದ ಬಂಧುಗಳು ಸತ್ತಾಗ ತೀವ್ರತರದ ಆಘಾತ, ತಡೆಯಲಾರದ ಸಂಕಟಗಳಿಂದ ಕಂಗೆಡುತ್ತಾನೆ, ಕಣ್ಣೀರಿಡುತ್ತಾನೆ. ಕೆಲದಿನಗಳ ಬಳಿಕ ಮೆಲ್ಲನೇ ನೋವನ್ನು ಮರೆಯುತ್ತಾನೆ. ತಿರುಗಿ ತನ್ನ ದೈನಂದಿನ ಚಟುವಟಿಕೆಗಳಲ್ಲಿ ಮುಳುಗುತ್ತಾನೆ. ತನಗೇ ಸಾವು ಬಂದಾಗ ಯಾವ ಸಿದ್ಧತೆಯೂ ಇಲ್ಲದೇ, ಅಳುತ್ತ, ಗೋಗರೆಯುತ್ತ, ಭಯವಿಹ್ವಲನಾಗಿ ಸಾಯುತ್ತಾನೆ. ಹೆಚ್ಚಿನ ಜನ, ನರಿನಾಯಿಗಳು ಸತ್ತಂತೆ ಸಾಯುತ್ತಾರೆ. ‘ಸಾವು ಬಂದರೆ ಸತ್ತರಾಯಿತು. ಅದು ಪ್ರಕೃತಿಯ ನಿಯಮವಲ್ಲವೆ?’ ಎಂದು ಹಾರಿಕೆಯ ಮಾತನಾಡಿ ಬುದ್ಧಿವಂತರು ಈ ಸಮಸ್ಯೆಯನ್ನು ಅಲ್ಲಗಳೆಯುವುದುಂಟು.\footnote{\engfoot{Death is a subject that is evaded, ignored by our youth worshipping, progress-oriented society.}\hfill\engfoot{ –Kubler Ross}} ಕೆಲವು ಸಾವುಗಳು ಸಾವಿನ ಸ್ವರೂಪದ ಕ್ಷಣಿಕ ದರ್ಶನವ ನ್ನೀಯುವುದುಂಟು. ಅನಿವಾರ್ಯವಾದುದನ್ನು ಎದುರಿಸಲು, ನಿರ್ಭೀತಿಯ ಮನೋವೃತ್ತಿಯನ್ನು ಉಂಟುಮಾಡುವಂಥ ನೈಜ ಘಟನೆಗಳು, ಅನುಭವಗಳು ಅವು. ಆದರೆ ಅವುಗಳ ಅರ್ಥ ತಿಳಿಯದೆ ಮನುಷ್ಯ ವಿಹ್ವಲನಾಗುತ್ತಾನೆ.

ಏಳು ವರ್ಷ ವಯಸ್ಸಿನ ಮಗ. ಅಲ್ಪಕಾಲದ ಅಸ್ವಸ್ಥತೆಯಿಂದ ತೀರಿಕೊಂಡ. ದುಃಖವನ್ನು ಮರೆಯಲು ದಂಪತಿಗಳು ದಕ್ಷಿಣ ದೇಶಕ್ಕೆ ಬಂದರು. ಇಲ್ಲಿಯೇ ಉದ್ಯೋಗ ಮಾಡಿಕೊಂಡಿದ್ದಾರೆ. ತಾಯಿ ಮತ್ತೆ ಮತ್ತೆ ‘ದೇವರು ನನ್ನ ಮಗುವನ್ನು ಏಕೆ ಕರೆದುಕೊಂಡ?’ ಎಂಬ ಪ್ರಶ್ನೆಯನ್ನು ಕೇಳುತ್ತಲೇ ಇದ್ದಳು. ಸಮಾಧಾನದ ಮಾತುಗಳನ್ನು ಹೇಳಲು ಎಷ್ಟು ಪ್ರಯತ್ನಮಾಡಿದರೂ, ಅವಳು ತನ್ನ ಪ್ರಶ್ನೆಯನ್ನು ಉಚ್ಚರಿಸುತ್ತಲೇ ಇದ್ದಳು. ಪುತ್ರಶೋಕದಿಂದ ತಪ್ತಳಾಗಿದ್ದಳು. ಯಾರದ್ದೇ ಮಾತುಗಳನ್ನು ಕೇಳುವುದಕ್ಕೂ, ವಿಚಾರ ಮಾಡುವುದಕ್ಕೂ, ಅಸಾಧ್ಯವಾದ ಮಾನಸಿಕ ಸ್ಥಿತಿ ಅದು. ಪ್ರಾಯಕ್ಕೆ ಬಂದ ಮಗನನ್ನು ಕಳೆದುಕೊಂಡ ವಿಚಾರವಂತರೊಬ್ಬರು ‘ನಾನೊಬ್ಬನೇ ಮಗನನ್ನು ಕಳೆದುಕೊಂಡವನಲ್ಲ, ಲೋಕದಲ್ಲಿ ಪುತ್ರಶೋಕದಿಂದ ದುಃಖಿತರಾದವರೆಷ್ಟೋ ಮಂದಿ ಇಲ್ಲವೆ!’ ಎಂದು ಸಮಾಧಾನ ತಂದುಕೊಳ್ಳಬಹುದು. ವ್ಯಾವಹಾರಿಕವಾಗಿ ಇದು ಹೊಂದಾಣಿಕೆಯ\break ಮನೋಭಾವ\-ವನ್ನು ಪ್ರದರ್ಶಿಸಿದಂತಾಯಿತೇ ಹೊರತು, ಸಾವಿನ ಮೂಲ ಸ್ವರೂಪದ ಅರಿವಿನಿಂದ ಬಂದ ಸಾಂತ್ವನದ ಭಾವವಲ್ಲ.


\section*{ನೋವಿನ ನಾನಾವಿಧ}

\addsectiontoTOC{ನೋವಿನ ನಾನಾವಿಧ}

ಜೀವನದಲ್ಲಿ ಎದುರಿಸಬೇಕಾದ ತೀವ್ರ ಕಷ್ಟಕಂಟಕಗಳು ಎಲ್ಲರನ್ನೂ ಒಂದೇ ರೀತಿ ಕಂಗೆಡಿಸುತ್ತವೆ ಎನ್ನಲಾಗದು. ಕೆಲವರು ಸಣ್ಣಪುಟ್ಟ ತೊಂದರೆಗಳಿಂದಲೂ ಖಿನ್ನರಾಗಿಬಿಡುವ ಅಪ್ರ ಬುದ್ಧರು. ಕೆಲ ಜನ ಶಾಂತಚಿತ್ತರಾಗಿದ್ದುಕೊಂಡು ಎಡರು ತೊಡರುಗಳನ್ನು ಎದುರಿಸುತ್ತಾರೆ. ಮನುಷ್ಯರಲ್ಲಿ ವೈವಿಧ್ಯ, ವೈಶಿಷ್ಟ್ಯಗಳಿರುವಂತೆ ಅವರು ಎದುರಿಸುವ ಸಮಸ್ಯೆಗಳಲ್ಲೂ ವೈವಿಧ್ಯ, ವೈಶಿಷ್ಟ್ಯಗಳಿರುತ್ತವೆ. ಆದರೆ ಎಲ್ಲರನ್ನೂ ಒಂದಲ್ಲ ಒಂದು ಪ್ರಮಾಣದಲ್ಲಿ ದುಃಖಕ್ಕೀಡು ಮಾಡುವ ಸಂದರ್ಭ, ಸನ್ನಿವೇಶಗಳಿವೆ. ಪ್ರೀತಿಪಾತ್ರರ ಸಾವು ಅಂಥ ಒಂದು ಸಂದರ್ಭ. ಅಗಲುವಿಕೆಯ ಪ್ರಥಮ ವರ್ಷದಲ್ಲಿ, ಸತಿಪತಿಗಳಲ್ಲಿ ಒಬ್ಬರ ಮರಣದಿಂದುಂಟಾಗುವ ದುಃಖ ದುಮ್ಮಾನಗಳ ಪರಿಣಾಮವಾಗಿ, ಉಂಟಾಗುವ ಸಾವುಗಳು ಸಾಮಾನ್ಯವಾಗಿ ಇತರ ಅದೇ ವಯಸ್ಸಿನ ಗುಂಪಿನಲ್ಲಿ ಆಗುವ ಸಾವುಗಳಿಗಿಂತ ಹತ್ತುಪಾಲು ಹೆಚ್ಚು ಎಂದು ತಿಳಿದುಬಂದಿದೆ. ವಿವಾಹಿತರಿಗಿಂತ ವಿಚ್ಛೇದ ಹೊಂದಿದವರಲ್ಲಿ ರೋಗರುಜಿನ ಹನ್ನೆರಡುಪಟ್ಟು ಹೆಚ್ಚು ಎಂಬುದನ್ನು ಪರಿಶೀಲಿಸಿದ್ದಾರೆ. ಅತ್ಯಂತ ನಿರಾಶಾದಾಯಕ ಹಾಗೂ ಅಸಹಾಯಕ ಪರಿಸ್ಥಿತಿಯಲ್ಲಿದ್ದಾಗ ನೂರರಲ್ಲಿ ಎಂಬತ್ತರಷ್ಟು ದೈಹಿಕ ವ್ಯಾಧಿ ಉಲ್ಬಣಿಸುವುದು ಎಂದು ಪತ್ತೆಹಚ್ಚಿದ್ದಾರೆ.

ಸೃಷ್ಟಿಯಾದಂದಿನಿಂದ ಇಂದಿನವರೆಗೂ ಪ್ರತಿಯೊಬ್ಬನನ್ನೂ ಒಂದಲ್ಲ ಒಂದು ನೋವು, ಚಿಂತೆ ಮೆಟ್ಟಿಕೊಂಡೇ ಇದೆ. ಎಲ್ಲೋ ಕೆಲವರು ಜೀವನವನ್ನು ಕುರಿತು ತಮ್ಮ ದೃಷ್ಟಿಕೋನ, ಅನುಭವದ ಪಕ್ವತೆ, ಅಭ್ಯಾಸ–ತರಬೇತಿ ಮತ್ತು ಸತ್ಸಂಗ, ಭಗವಂತನಲ್ಲಿ ಪ್ರಾರ್ಥನೆ–ಇವುಗಳಿಂದ ಚಿಂತೆಯನ್ನು ಬಹುಮಟ್ಟಿಗೆ ನಿಯಂತ್ರಿಸಬಲ್ಲರು. ಉಳಿದವರು ಚಿಂತೆಯಿಂದ ಕಂಗೆಟ್ಟು ದಿಕ್ಕೆಟ್ಟು ಬಳಲಿ ಬೆಂಡಾಗುವರು. ಕೆಲವರು ಮಾದಕ ದ್ರವ್ಯಗಳನ್ನು ಸೇವಿಸಿ, ಮತ್ತುಮಂಪರು ಬರುವ ಔಷಧಗಳನ್ನು ತಿಂದು ಚಿಂತೆಯನ್ನು ಸ್ವಲ್ಪ ಹೊತ್ತು ಮರೆಯಲು ಯತ್ನಿಸಿ ಮತ್ತಷ್ಟು ದುರ್ಬಲರಾಗುವರು! ದುಃಖಿತರಾಗುವರು! ಇನ್ನು ಕೆಲವರು ಸಮಸ್ಯೆಗೆ ಪರಿಹಾರ ಆತ್ಮಹತ್ಯೆ ಎಂದು ತಿಳಿದು ಆ ಹೇಯಕೃತ್ಯವನ್ನೆಸಗಿ ಇಹಪರ ಎರಡನ್ನೂ ಕಳೆದುಕೊಳ್ಳುವರು. ಇವೆಲ್ಲ ಚಿಂಕ್ರೋಭದ ಆಘಾತವನ್ನು ಹೆಚ್ಚಿಸಬಲ್ಲವೇ ಹೊರತು ನೀಗಿಸಲಾರವು. ಹಾಗಾದರೆ ಚಿಂಕ್ರೋಭದಿಂದ ಬಿಡುಗಡೆ\-ಯಾಗಲು ದಾರಿ ಯಾವುದು?


\section*{ಅನಿವಾರ್ಯ, ಅಪರಿಹಾರ್ಯ}

\addsectiontoTOC{ಅನಿವಾರ್ಯ, ಅಪರಿಹಾರ್ಯ}

ಎಲ್ಲರ ಬದುಕಿನಲ್ಲೂ ಅನಿವಾರ್ಯವಾದ ಘಟನೆಗಳು ನಡೆದೇ ನಡೆಯುತ್ತವೆ. ಕೆಲವೊಮ್ಮೆ ನಿಮ್ಮ ನಿರೀಕ್ಷೆಗೆ ಅನುಗುಣವಾಗಿ ನಡೆದರೆ, ಇನ್ನು ಕೆಲವೊಮ್ಮೆ ಅನಿರೀಕ್ಷಿತವಾಗಿ.

ನಿವಾರಿಸಲು ಅಸಾಧ್ಯವಾದುದನ್ನು, ಪರಿಹರಿಸಲಾಗದುದನ್ನು, ಹೇಗೆ ಎದುರಿಸುವುದು\break ಎನ್ನುವು\-ದನ್ನು ಕುರಿತು ನಮ್ಮಲ್ಲಿ ಯೋಚಿಸುವ ಅಭ್ಯಾಸ ಹೆಚ್ಚಿನವರು ಇಟ್ಟುಕೊಂಡಿಲ್ಲ.

ಮನುಷ್ಯ ಆಶಾವಾದಿಯಾಗಬೇಕು, ದುರ್ಘಟನೆಗಳನ್ನು ನಿರೀಕ್ಷಿಸುತ್ತ ಕೂಡಬೇಕಿಲ್ಲ\break ಎಂಬುದು ದಿಟ. ಆದರೆ ಜೀವನದಲ್ಲಿ ಏರುಪೇರುಗಳಿಲ್ಲದೆ, ವೃದ್ಧಿಕ್ಷಯಗಳಿಲ್ಲದೆ ಎಲ್ಲವೂ ಸುಗಮವಾಗಿಬಿಡುವುದು ಎಂಬುದು, ಸಿನಿಮೀಯ ಭ್ರಾಮಕ ಕಲ್ಪನೆಯಾಗುವುದಲ್ಲವೆ?

ಬಾಲ್ಯ, ಯೌವನಾವಸ್ಥೆಗಳಲ್ಲಿ, ದೈಹಿಕಶಕ್ತಿ ಚೆನ್ನಾಗಿರುವಾಗ, ಭರವಸೆ ಬಲವಾಗಿರುವಾಗ, ಕಲ್ಪನೆಯ ಕನಸುಗಳನ್ನು ಕಾಣುತ್ತ, ಭವಿಷ್ಯದ ಚಿತ್ರಸೌಧಗಳನ್ನು ಕಟ್ಟಬಹುದು. ಸೋಲು, ಹತಾಶೆ, ವಿರೋಧ, ದುರ್ಬಲತೆ, ಅಸಹಾಯಕತೆ–ಈ ತೆರೆಗಳು ಬಂದು ಬಡಿದಪ್ಪಳಿಸಿದಾಗ ಮನುಷ್ಯ ತತ್ತರಿಸುತ್ತಾನೆ. ಕಿಂಕರ್ತವ್ಯ ಮೂಢನಾಗುತ್ತಾನೆ. ಆ ಸ್ಥಿತಿಯಲ್ಲಿ ತಾನು ಯೋಚಿಸಲೂ ಆರ, ಇತರರ ಸಲಹೆಗಳನ್ನು ಸ್ವೀಕರಿಸಿ ತನ್ನ ಮನಸ್ಸಿನ ದುಗುಡವನ್ನು ದೂರಮಾಡಲೂ ಆರ. ಆದುದರಿಂದಲೇ ಅನಿವಾರ್ಯವಾದುದನ್ನು ಎದುರಿಸಲು ಸಿದ್ಧತೆ ಬೇಕು.

ಆಗಸವನ್ನು ಕಾರ್ಮೋಡಗಳು ಮುಸುಕುತ್ತವೆ. ಕ್ಷಣಾರ್ಧದಲ್ಲಿ ಧಾರಾಕಾರವಾಗಿ ಮಳೆ ಸುರಿಯುತ್ತದೆ. ನಾವು ಮಳೆಯನ್ನು ಬೇಡವೆಂದು ತಡೆಯಬಲ್ಲೆವೇ? ಮಳೆಯಲ್ಲಿ ಸಿಕ್ಕಿ ತೋಯದೆ ಇರಬೇಕಾದರೆ, ಮನೆಯನ್ನು ಮಾಡಿನಿಂದ ಮುಚ್ಚಿರಬೇಕು. ಸಂಚರಿಸುವಾಗ ಕೈಯಲ್ಲಿ ಕೊಡೆಯನ್ನು ಹಿಡಿದುಕೊಳ್ಳಬೇಕು. ಕಾಡಿನ ದಾರಿಯಲ್ಲಿ ನಡೆದು ಹೋಗುವಾಗ, ನಾವು ಕಲ್ಲುಮುಳ್ಳುಗಳನ್ನು ತೆಗೆದು ಸ್ವಚ್ಛ ಮಾಡಲಾರೆವು. ಕಲ್ಲುಮುಳ್ಳುಗಳು ಚುಚ್ಚದಂತೆ ಪಾದರಕ್ಷೆಗಳನ್ನು ಧರಿಸಬೇಕು. ಅಂತೆಯೇ ಈ ಪ್ರಪಂಚದ ಬಾಳ್ವೆಯಲ್ಲಿ ಬರಬಹುದಾದ ಕಷ್ಟನಷ್ಟಗಳನ್ನು, ಸೋಲು ನಿರಾಸೆಗಳನ್ನು, ಸಾವು ನೋವುಗಳನ್ನು, ಬೇಡವೆಂದರೂ ಬಿಟ್ಟಿರಲು ಸಾಧ್ಯವಿಲ್ಲ. ನಮ್ಮ ಮನಸ್ಸನ್ನು ಅರಿವಿನ ಬೆಳಕಿನಿಂದ ಬೆಳಗಬೇಕು. ಸಮಸ್ಯೆಗಳ ಮೂಲವನ್ನು ಅರ್ಥೈಸಿಕೊಳ್ಳಬೇಕು. ಎಲ್ಲಕ್ಕಿಂತ ಮಿಗಿಲಾಗಿ ಜೀವನವನ್ನು ನಿಯಂತ್ರಿಸುವ ಸೂಕ್ಷ್ಮ ನಿಯಮಗಳನ್ನು ಅರಿತು, ಅರಗಿಸಿಕೊಂಡು, ದೃಢಚಿತ್ತರೂ, ಧೀರರೂ, ಶಾಂತಮನಸ್ಕರೂ ಆಗಿ ಸಮಸ್ಯೆಗಳನ್ನು ಎದುರಿಸಬೇಕು.

ಮನಸ್ಸನ್ನು ಶಾಂತವಾಗಿಟ್ಟುಕೊಳ್ಳಬೇಕೆಂದು ಬಯಸುವವರು ಬಹಳಷ್ಟು ಜಾಗರೂಕತೆಯಿಂದ ಅನಾವಶ್ಯಕ ಸಮಸ್ಯೆಗಳನ್ನು ದೂರಕ್ಕೆಸೆಯಬೇಕು. ಆವಶ್ಯಕವೂ, ಅನಿವಾರ್ಯವೂ ಆದ ಸಮಸ್ಯೆಗಳನ್ನು ನಾವು ಎದುರಿಸುವಾಗ ಗೌತಮಬುದ್ಧ ‘ಅಂಗುತ್ತರನಿಕಾಯ’ದಲ್ಲಿ ಹೇಳಿದ ಮಾತುಗಳು ತುಂಬ ಉಪಯುಕ್ತವೆನಿಸುತ್ತವೆ–

‘ಭಿಕ್ಷುಗಳೇ, ಈ ಐದು ಸಂಗತಿಗಳನ್ನು ಕುರಿತು ಗಂಡಸರೂ, ಹೆಂಗಸರೂ, ಗೃಹಸ್ಥರೂ, ಭಿಕ್ಷುಗಳೂ ಚಿಂತನೆ ಮಾಡಬೇಕು.

\begin{enumerate}
\item ವೃದ್ಧಾಪ್ಯವು ಬಂದೇ ಬರುವುದು. ಅದನ್ನು ನಾವು ನಿವಾರಿಸುವಂತಿಲ್ಲ, ತಪ್ಪಿಸಿಕೊಳ್ಳು\-ವಂತಿಲ್ಲ.

 \item ರೋಗರುಜಿನಗಳು ಒಂದಲ್ಲ ಒಂದು ದಿನ ಬರುವ ಸಂದರ್ಭವಿಲ್ಲವೆಂದು ಹೇಳಲಾರೆ. ಅದನ್ನು ನಿವಾರಿಸಲು ಪ್ರಯತ್ನಿಸಿದರೂ, ಪೂರ್ಣನಿವಾರಣೆ ಶಕ್ಯವಲ್ಲ.

 \item ಒಂದಲ್ಲ ಒಂದು ದಿನ ಮರಣ ಬಂದೇ ಬರುವುದು. ಇದನ್ನು ತಪ್ಪಿಸಿಕೊಳ್ಳು\-ವಂತಿಲ್ಲ. ‘ಮರಣ ಬಂದಾಗ ಸತ್ತರಾಯಿತು’ ಎಂದು ನಿರ್ಲಿಪ್ತರಾಗಿ ಇತರರಿಗೆ ಉಪದೇಶಿಸ\-\break ಬಹುದು. ಆದರೆ ನುಡಿದಂತೆ ನಡೆಯಲು ನಮಗೇ ಕಷ್ಟ.

 \item ನನ್ನ ಪ್ರೀತಿಪಾತ್ರವಾದ ವಸ್ತುಗಳೆಲ್ಲ ಬದಲಾವಣೆ ಹೊಂದುತ್ತವೆ, ದುರ್ಬಲವಾಗುತ್ತವೆ, ವಿಯೋಗ ಹೊಂದುತ್ತವೆ. ಇದನ್ನು ತಪ್ಪಿಸಿಕೊಳ್ಳುವಂತಿಲ್ಲ.

 \item ನನ್ನ ಬದುಕನ್ನು ರೂಪಿಸುವುದು ನನ್ನ ಯೋಚನೆ ಮತ್ತು ಭಾವನೆಗಳು. ಒಳಿತೋ ಕೆಡುಕೋ ಮಾಡಿದಂಥ ಕಾರ್ಯಗಳು. ಅವುಗಳ ಫಲಗಳನ್ನು ಉಣ್ಣುವವನು ನಾನೇ.

\end{enumerate}

ಭಿಕ್ಷುಗಳೇ,

ವೃದ್ಧಾಪ್ಯವನ್ನು ಕುರಿತು ಚಿಂತಿಸಿದರೆ ಯೌವನದ ಅಹಂಕಾರ ಸ್ಥಗಿತವಾಗದಿದ್ದರೂ, ಸ್ವಲ್ಪ ಮಟ್ಟಿಗೆ ಅಳಿಯುವುದು.

ರೋಗರುಜಿನಗಳಿಂದ ಉದ್ಭವಿಸುವ ಅಸಹಾಯಕತೆಯನ್ನು ಕುರಿತು ಚಿಂತಿಸಿದಾಗ, ತನ್ನ ಆರೋಗ್ಯದ ಬಗೆಗಿನ ಅಭಿಮಾನ ಸ್ಥಗಿತವಾಗುವುದು, ಇಲ್ಲವೇ ಸ್ವಲ್ಪ ಮಟ್ಟಿಗಾದರೂ ಕಡಿಮೆಯಾಗುವುದು.

ಮರಣವನ್ನು ಕುರಿತು ಅಧ್ಯಯನ ಅನುಧ್ಯಾನ ಮಾಡಿದಾಗ, ಜೀವನದ ಬಗೆಗಿನ ಅಭಿಮಾನ ಅಹಂಕಾರ ದೂರವಾಗುವುದು, ಇಲ್ಲವೇ ಕಡಿಮೆಯಾಗುವುದು.

ಪ್ರಿಯವಾದ ವಸ್ತುಗಳ ಬದಲಾವಣೆ, ಕ್ಷೀಣ ಹೊಂದುವಿಕೆ, ವಿಯೋಗ–ಇವುಗಳನ್ನು ಕುರಿತು ಚಿಂತಿಸಿದಾಗ ಲೋಭವು ದೂರವಾಗುವುದು ಅಥವಾ ಕಡಿಮೆಯಾಗುವುದು.

ತನ್ನ ಕರ್ಮಕ್ಕೆ ತಾನೇ ಹೊಣೆ, ತನ್ನ ಬದುಕನ್ನು ದಿನದಿನವೂ ಸೂಕ್ಷ್ಮವಾಗಿ ರೂಪಿಸತಕ್ಕ ಯೋಚನೆಗಳು, ಭಾವನೆಗಳು, ಆಸೆ ಆಕಾಂಕ್ಷೆಗಳು, ಭವಿಷ್ಯದ ಯೋಚನೆಗಳು ಇವುಗಳನ್ನು ಹುಟ್ಟುಹಾಕುವವನು ನಾನೇ ಎಂಬುದನ್ನು ಚಿಂತಿಸಿದಾಗ, ದುಷ್ಟ ಯೋಚನೆಗಳು, ದುಷ್ಟಶಬ್ದಗಳು, ದುಷ್ಟಕರ್ಮಗಳು ಸ್ಥಗಿತವಾಗುತ್ತವೆ, ಇಲ್ಲವೇ ಕಡಿಮೆಯಾಗುತ್ತವೆ.’

ಈ ಐದು ವಿಚಾರಗಳನ್ನು ಮನನ ಮಾಡಿದಾಗ ತನ್ನ ‘ಅಹಂ’, ‘ಮಮ’ಗಳನ್ನೂ, ದುರಾಸೆ ದುರಾಸಕ್ತಿಗಳನ್ನೂ, ಕಡಿಮೆ ಮಾಡಿಕೊಂಡು ದಿವ್ಯ ಪಥದಲ್ಲಿ ಮುನ್ನಡೆಯಲು ಸಮರ್ಥನಾಗುತ್ತಾನೆ ಮನುಜ. ಆಗ ಅನಿವಾರ್ಯ ಅಪರಿಹಾರ್ಯಗಳನ್ನು ಧೈರ್ಯದಿಂದ ಆತ ಎದುರಿಸುತ್ತಾನೆ.

~\\[-1.3cm]


\section*{ಅರಿವಿನ ಬೆಳಕು ಹೆಚ್ಚಿದಂತೆ}

\vskip -6pt\addsectiontoTOC{ಅರಿವಿನ ಬೆಳಕು ಹೆಚ್ಚಿದಂತೆ}

ಈ ಜಗತ್ತು ನಿಗೂಢ ಶಕ್ತಿಯಿಂದ ನಡೆಯುತ್ತಿದೆ ಎಂಬುದನ್ನು ತಿಳಿಯಲು ಕಷ್ಟವಿಲ್ಲ. ಆದರೆ ಆ ಕಡೆಗೆ ನಾವು ಗಮನವನ್ನು ಹರಿಯಗೊಡುವುದಿಲ್ಲ. ಹುಟ್ಟಿನ ಮೊದಲು ಏನು ಎಂಬುದು ನಮಗೆ ತಿಳಿಯದು! ಸಾವಿನ ಆಚೆಗೆ ಏನು ಎಂಬುದೂ ನಮಗೆ ತಿಳಿಯದು! ಜೀವಿ ಅನುಭವಿಸುವ ಸುಖದುಃಖಗಳ ಕಾರಣ ಅರ್ಥವೂ ತಿಳಿಯದು! ಜಗತ್ತಿನಲ್ಲಿ ಪ್ರತಿಯೊಂದು ಜೀವಿ ಯಾವುದೋ ಸೂಕ್ಷ್ಮಾತಿಸೂಕ್ಷ್ಮ ನಿಯಮವನ್ನನುಸರಿಸಿ ಬಂದು ಹೋಗುತ್ತಾನೆಂಬ ವಿಚಾರವೂ ತಿಳಿಯದು! ಹುಟ್ಟು ಸಾವುಗಳ ಮಧ್ಯದ ಅಂತರವೇ ಸರ್ವಸ್ವವೆಂದು ತಿಳಿದುಕೊಂಡವರು ನಾವು! ಜೀವನದಲ್ಲಿ ಬಂದೊದಗುವ ಸಂಕೀರ್ಣ ಸಮಸ್ಯೆಗಳನ್ನು ನಮ್ಮ ಸಾಮಾನ್ಯ ಅರಿವಿನಿಂದ ಪರಿಹರಿಸಲು ಸಾಧ್ಯವೆ?

ಅರಿವಿನ ಪರಿಧಿ ವಿಶಾಲವಾದಂತೆ, ಆಳವಾದಂತೆ, ಅನಿವಾರ್ಯ ಅಪರಿಹಾರ್ಯಗಳನ್ನು ಎದುರಿಸುವ ವಿಧಾನ ನಮಗೆ ಸ್ವಾಭಾವಿಕವಾಗಿ ತಿಳಿಯುತ್ತದೆ. ಅಜ್ಞಾನವೇ ಸಕಲ ದುಃಖಗಳ ಮೂಲ. ಅದು ಕಳೆದಂತೆ ಮನಸ್ಸಿನ ಕೊಳೆ ದೂರವಾಗಿ ಸ್ಥೈರ್ಯ ವೃದ್ಧಿಯಾಗುತ್ತದೆ. ನಮ್ಮಲ್ಲಿ ಹಳೆಯ ತಲೆಮಾರಿನ ಕೆಲವರಿಗೆ ತಾವು ದೇಹವಲ್ಲ, ದೇಹದಲ್ಲಿದ್ದೇವೆ ಅಷ್ಟೆ ಎನ್ನುವ ದೃಢವಿಶ್ವಾಸ ಇತ್ತು. ಇಂಥ ಜನರು ಸಾವನ್ನು ಹೇಗೆ ಎದುರಿಸಿದರು ಎಂಬುದಕ್ಕೊಂದು ನಿದರ್ಶನವನ್ನು ಕನ್ನಡದ ಹಿರಿಯ ಬರಹಗಾರರಾದ ಮಾಸ್ತಿ ಅವರು ತಮ್ಮ ‘ಧರ್ಮಸಂರಕ್ಷಣ’ ಎನ್ನುವ ಗ್ರಂಥದಲ್ಲಿ ಹೇಳಿದ್ದಾರೆ: ‘ನಾನು ಕಂಡ ಹಿರಿಯರೊಬ್ಬರ ಜೀವನದ ಕೊನೆಯ ದಿನಗಳು. ಇನ್ನೇನು ಎರಡು ದಿನಗಳಲ್ಲಿ ಜೀವ ದೇಹವನ್ನು ಬಿಡಬೇಕು. ನಾನು ಅವರ ಹತ್ತಿರ ಹೋಗಿ ಕುಳಿತಾಗ ‘ಗಜೇಂದ್ರನಿಗೆ ಇನ್ನೂ ಮೋಕ್ಷವಾಗಿಲ್ಲ’ ಎಂದರು. ದೇಹದಲ್ಲಿ ಸಿಕ್ಕಿ ಬಿದ್ದಿರುವ ಜೀವಕ್ಕೆ ಇನ್ನೂ ಬಿಡುಗಡೆ ಆಗಿಲ್ಲ ಎಂದೇ ಅದರರ್ಥ. ಈ ಅಜ್ಜನಿಗೆ ಸಾವು ಒಂದು ಭಯದ ಸಂಗತಿಯಾಗಿ ಇರಲಿಲ್ಲ. ಸಾವು ತನ್ನ ದೈವ ತನ್ನನ್ನು ಒಂದು ಕ್ಲೇಶದಿಂದ ಬಿಡಿಸುವ ವ್ಯಾಪಾರ ಎಂದು ಕಾಣುತ್ತಿತ್ತು. ಆದುದರಿಂದಲೇ ಅವರು ಸಾವಿನ ಎದುರಲ್ಲಿ ಅಷ್ಟು ಸಮಾಧಾನವಾಗಿ ಮಲಗಿದ್ದರು. ಅದರ ಬರವನ್ನು ಕಾಯುತ್ತಿದ್ದರು. ಬರಬೇಕಾಗಿ ಹಾತೊರೆಯುತ್ತಲೂ ಇದ್ದರೋ? ಅಹುದು ಎಂದರೆ ಆಶ್ಚರ್ಯವಿಲ್ಲ.’

ಈ ವೃದ್ಧರು ಅನಿವಾರ್ಯವಾದುದನ್ನು ಎದುರಿಸಿದ ಬಗೆ ನೋಡಿ.


\section*{ಮುಖಂಡರ ಮೌಢ್ಯ}

\vskip -6pt\addsectiontoTOC{ಮುಖಂಡರ ಮೌಢ್ಯ}

ನಮ್ಮ ಶಿಕ್ಷಣ ಪದ್ಧತಿಯಲ್ಲಿ, ಕೌಟುಂಬಿಕ ಹಾಗೂ ಸಾಮಾಜಿಕ ನಡವಳಿಕೆಗಳಲ್ಲಿ ಈ ಮೌಲ್ಯಗಳನ್ನು ಕುರಿತು ಕಾಳಜಿವಹಿಸುವವರೇ ಅಪರೂಪ. ನಮ್ಮ ಸಿನಿಮಾ ನಾಟಕಗಳ ಕಥನಕರ್ತೃಗಳು, ಬರಹಗಾರರು, ಶಿಕ್ಷಕರು ಈ ವಿಚಾರಗಳನ್ನು ಅರ್ಥಮಾಡಿಕೊಳ್ಳುವ ಸ್ಥಿತಿಯಲ್ಲಿಲ್ಲ. ನಮ್ಮ ಕ್ರಾಂತಿಕಾರಿ ಮುಖಂಡರುಗಳು, ರಾಜಕೀಯಸ್ಥ ಜನರು, ರಾಗದ್ವೇಷ ಕೋಪತಾಪಗಳನ್ನು ವೃದ್ಧಿಸದೆ ತಮ್ಮ ಸೇವಾಕಾರ್ಯ ಅಸಾಧ್ಯ ಎಂದು ನಂಬಿದವರು ಎಂದರೆ ತಪ್ಪಲ್ಲ. ನಮ್ಮಲ್ಲಿ ತುಳಿತಕ್ಕೊಳಗಾದವರು ತಾವು ಮೇಲೇಳುತ್ತಲಿರುವಾಗಲೇ ತಮಗಿಂತ ಆರ್ಥಿಕವಾಗಿ ಸಾಮಾಜಿಕವಾಗಿ ಮೇಲಿರುವವರನ್ನು ಖಂಡಿಸದೆ, ನಿಂದಿಸದೆ, ದ್ವೇಷಿಸದೆ, ತಮ್ಮ ಅಭ್ಯುದಯವಿಲ್ಲ ಎಂದು ನಂಬಿದಂತಿದೆ. ನಿಜವಾದ ಧಾರ್ಮಿಕರಿಗೆ ಸಾಧ್ಯವಾದರೂ, ಸಾಮಾನ್ಯ ಧಾರ್ಮಿಕರೆನ್ನಿಸಿಕೊಂಡವರು ಈ ದಿಸೆಯಲ್ಲಿ ಮಾರ್ಗದರ್ಶನ ಮಾಡುತ್ತಿರುವರೆ ಎಂದರೆ ಅವರೂ ರಾಜಕೀಯ ಪಕ್ಷ ಪಂಗಡಗಳಂತೆ ತಮ್ಮ ತಮ್ಮ ಶ್ರೇಷ್ಠತೆಯ ಪಿಡುಗಿನಿಂದ ಪೀಡಿತರಾಗಿ ತಮ್ಮ ಮೇಲ್ಮೆಯನ್ನು ಸಾರುತ್ತ, ಇತರ ಪಥ ಪಂಥಗಳನ್ನು ನಿಂದಿಸುತ್ತ, ಜನರಲ್ಲಿ ದ್ವೇಷ ದೌರ್ಮನಸ್ಯಗಳನ್ನು ಹೆಚ್ಚಿಸುವವರೇ ಆಗುತ್ತಲಿದ್ದಾರೆ. ಸಾಮಾನ್ಯ ಹಳ್ಳಿಯ ಸರಳ ವ್ಯಕ್ತಿಗಳಿಗಿಂತಲೂ ಬುದ್ಧಿವಂತ ವಿದ್ಯಾ ವಂತರೆನಿಸಿಕೊಂಡ ಜನಗಳೆ ಮಾನಸಿಕ ತಳಮಳ, ಚಿಂತೆಭಯೋದ್ವೇಗಗಳ ರೋಗಕ್ಕೆ ತುತ್ತಾಗುವುದು ಹೆಚ್ಚು ಎನ್ನುತ್ತಾರೆ ತಜ್ಞರು!

ಚಿಂಕ್ರೋಭದಿಂದ ಪಾರಾಗಿ ಸರಿಯಾಗಿ ನಮ್ಮನ್ನು ಬಾಳಿ ಬದುಕುವಂತೆ ನಮ್ಮ ಬುದ್ಧಿ ವಂತಿಕೆ, ವಿದ್ಯಾವಂತಿಕೆಗಳು ಪ್ರೇರಿಸದಿದ್ದಲ್ಲಿ ಅವುಗಳಿಗೇನು ಅರ್ಥ? ಅವುಗಳಿಂದೇನು? ಎನ್ನುವ ಪ್ರಶ್ನೆ ಸಹಜವಲ್ಲವೇ? ಸರಿಯಾಗಿ ಬಾಳಿಬದುಕುವಂತೆ ಪ್ರೇರಿಸದಿದ್ದರೂ ಹೋಗಲಿ–ದ್ವೇಷ ಅಸೂಯೆ ಹಿಂಸೆಯ ನರಕಕ್ಕೆ ತಳ್ಳುವಂತಾದರೆ?


\section*{ಇಲ್ಲಿದೆ ಪರಿಹಾರ}

\vskip -6pt\addsectiontoTOC{ಇಲ್ಲಿದೆ ಪರಿಹಾರ}

ಅಕ್ಷರಸ್ಥನಾಗಲಿ, ನಿರಕ್ಷರಿಯಾಗಲಿ, ಬಡವನಾಗಲಿ ಅಥವಾ ಶ‍್ರೀಮಂತನಾಗಿರಲಿ, ಏಕಾಂತ ಜೀವಿಯಾಗಲಿ, ಸಂಘಜೀವಿಯಾಗಲಿ, ಅಂತರ್ಮುಖಿಯಾಗಲಿ, ಉದ್ಯೋಗದಲ್ಲಿ ಅತೃಪ್ತನಾದವ\-ನಿರಲಿ, ಅಸುಖಿ ದಂಪತಿಗಳಾಗಲಿ, ಯಾವುದೇ ಜಾತಿಮತ ಧರ್ಮಕ್ಕೆ ಸೇರಿದವರಾಗಲಿ – ಎಲ್ಲರೂ ತಮ್ಮ ಜೀವನದಲ್ಲಿ ಸುಖದ ಹಾಗೂ ಅಭಿವೃದ್ಧಿಯ ಅಪೇಕ್ಷಿಗಳೇ. ಸಮಸ್ಯೆಗಳು, ಕಷ್ಟಕಂಟಕಗಳು ಯಾವುದೇ ವಿಶಿಷ್ಟ ರೀತಿಯವುಗಳಿದ್ದರೂ, ನಮ್ಮ ಅಂತರಂಗದ ಬದಲಾವಣೆಯಿಂದ ಅವುಗಳನ್ನು ಎದುರಿಸಬೇಕು, ನಮ್ಮ ದೃಷ್ಟಿಕೋನ ಬದಲಾಗಬೇಕು, ಯೋಚನಾವಿಧಾನ ಹೊಸರೂಪ ತಾಳಬೇಕು, ನಮ್ಮ ನಡತೆಯಲ್ಲಿ ಬದಲಾವಣೆ ಉಂಟಾಗಬೇಕು. ನಿಂದೆ, ದ್ವೇಷ, ಅಹಂಕಾರ, ನಿರ್ಲಕ್ಷ್ಯ, ನಿಷೇಧಾತ್ಮಕ ಯೋಚನೆ ಯೋಜನೆಗಳಿಂದ ಪರಿಸ್ಥಿತಿಯನ್ನು ಎದುರಿಸಿದರೆ ಪರಿಣಾಮ\-ವೇನಾ\-ದೀತು? ಇನ್ನಷ್ಟು ತೊಂದರೆಗೊಳಗಾಗಬೇಕಾಗುವುದಷ್ಟೆ. ಸಾಮಾಜಿಕ ಹೊಣೆಗಾರಿಕೆ ಇಲ್ಲದೆ ಕೇವಲ ಸಂಕುಚಿತ ಸ್ವಾರ್ಥಪರಾಯಣತೆ ಎಂದೆಂದೂ ಶಾಂತಿ ಸಂತೃಪ್ತಿಗಳ ಪಥವಾಗದು. ವೈಯಕ್ತಿಕ, ಕೌಟುಂಬಿಕ ಹಾಗೂ ಸಾಮಾಜಿಕವಾದ ಕಷ್ಟಗಳನ್ನು ಸರಿಯಾದ, ಯುಕ್ತವಾದ ಸದ್ಗುಣಗಳ ಬೆಳವಣಿಗೆಯಿಂದ ಪ್ರಬುದ್ಧವಾಗಿ ಎದುರಿಸಬೇಕು.\footnote{\engfoot{Suffering does not depend upon what we have, but what we are.}} ಈ ಕಾರ್ಯಕ್ಕೆ ಜೀವನವನ್ನು ನಿಯಂತ್ರಿ\-ಸುವ ಮೂಲಭೂತ ತತ್ತ್ವಗಳಲ್ಲಿ ದೃಢ ವಿಶ್ವಾಸ ಬೇಕು. ಮನಸ್ಸು, ಆತ್ಮ, ಜೀವನದ ಪರಮ ಗುರಿ ಇವುಗಳ ಸ್ವರೂಪ ಸ್ವಭಾವದ ಸ್ಪಷ್ಟ ಅರಿವು, ಕಲ್ಪನೆ ಬೇಕು. ಸದ್ಭಾವನೆಗಳನ್ನು ತಾಳ್ಮೆಯಿಂದ ರೂಢಿಸಿಕೊಳ್ಳಬೇಕು.


\section*{ಮನಸ್ಸಿನ ಅಗಾಧ ಶಕ್ತಿ}

\addsectiontoTOC{ಮನಸ್ಸಿನ ಅಗಾಧ ಶಕ್ತಿ}

ಮನಸ್ಸು ರೋಗವನ್ನು ಸೃಷ್ಟಿಸಬಲ್ಲದು, ಗುಣಪಡಿಸಬಲ್ಲದು! ತಾಳ್ಮೆ, ಪ್ರೀತಿ, ಕರುಣೆ, ದಾನ ಬುದ್ಧಿ, ನಿಃಸ್ವಾರ್ಥ ಸೇವಾಮನೋಭಾವ–ಇವೇ ಮೊದಲಾದ ರಚನಾತ್ಮಕ ಭಾವನೆಗಳು ದೇಹ ಯಂತ್ರದ ಎಲ್ಲ ಭಾಗಗಳಲ್ಲಿ ಉತ್ಸಾಹಯುತ ಆರೋಗ್ಯಕರ ಚಟುವಟಿಕೆಗಳಿಗೆ ಕಾರಣವಾಗುತ್ತವೆ. ನಮ್ಮಲ್ಲಿ ಸಂತೋಷ, ತೃಪ್ತಿ ಹಾಗೂ ಸೃಜನಶೀಲ ಪ್ರವೃತ್ತಿಗಳನ್ನು ಎಚ್ಚರಗೊಳಿಸುತ್ತವೆ. ಹಾಗೆಯೇ ನಿಷೇಧಮಯ ಭಾವನೆಗಳಿಂದ ಭಯ, ಚಿಂತೆ, ದ್ವೇಷ, ಅಸೂಯೆ, ಸೋಲಿನ ಮನೋಭಾವ ಹಾಗೂ ಸಂಕುಚಿತ ಸ್ವಾರ್ಥದ ನಾನಾ ಅವತಾರಗಳು ಸಮಸ್ತ ದೇಹ ಮತ್ತು ಅಂಗಾಂಗಗಳಿಗೆ ಕೆಡುಕನ್ನುಂಟುಮಾಡಿ ರೋಗಕ್ಕೆ ಕಾರಣವಾಗುತ್ತವೆ. ಈ ವಿಚಾರ ಸ್ಪಷ್ಟವಾಗಿ ಮನಸ್ಸಿನಲ್ಲಿ ಅಚ್ಚೊತ್ತಿ ನಿಲ್ಲುವಂತಾಗಲಿ ಎಂಬ ಉದ್ದೇಶದಿಂದ ಪುನಃ ಪುನಃ ಹೇಳುತ್ತಲಿದ್ದೇನೆ. ಮನಸ್ಸು ರೋಗವನ್ನು ಸೃಷ್ಟಿಸಬಹುದು, ಗುಣಪಡಿಸಲೂ ಬಲ್ಲದು!

ಪ್ರಸನ್ನತೆ, ಶಾಂತತೆ, ಧೈರ್ಯ, ಆತ್ಮವಿಶ್ವಾಸ, ದೃಢನಿಶ್ಚಯ–ಇವೇ ಮೊದಲಾದ ಭಾವನೆಗಳು ಶರೀರದ ಸುಸ್ಥಿತಿಗೆ ಆರೋಗ್ಯವರ್ಧಕ ಟಾನಿಕ್ಕಿಗಿಂತಲೂ ಹೆಚ್ಚಿನ ಸಹಾಯ ಮಾಡುತ್ತವೆ!

ಮನಸ್ಸೇ ದೇಹವನ್ನು ತನ್ನ ಉದ್ದೇಶ ಸಾಧನೆಗಾಗಿ ನಿರ್ಮಿಸಿಕೊಳ್ಳುತ್ತದೆ–ರೇಷ್ಮೆಹುಳು ಗೂಡು\-ಕಟ್ಟಿಕೊಳ್ಳುವಂತೆ ಎಂದರು ವಸಿಷ್ಠಮಹರ್ಷಿಗಳು.

ಮನುಷ್ಯನ ಬಂಧನಕ್ಕೂ, ಮೋಕ್ಷಕ್ಕೂ ಮನಸ್ಸೇ ಕಾರಣ ಎಂದರು ಮನೀಷಿಗಳು.

ಮನಸ್ಸಿನಲ್ಲಿ ಉದಿಸುವ ಯೋಚನೆ ಎಂಬ ದ್ರವ್ಯದಿಂದ ವಿಷವನ್ನೂ ತಯಾರಿಸಬಹುದು, ಅಮೃತವನ್ನೂ ತಯಾರಿಸಬಹುದು. ತಿಳಿದೋ, ತಿಳಿಯದೆಯೋ, ವಿಷವನ್ನು ತಯಾರಿಸುವವರೇ ಹೆಚ್ಚುಮಂದಿ. ಬುದ್ಧಿವಂತರೂ ಕೂಡ ಇದೇ ಕೆಲಸ ಮಾಡುತ್ತಿದ್ದಾರೆ. ಮನಸ್ಸು ಕೆಲಸ ಮಾಡುವ ಸೂಕ್ಷ್ಮನಿಯಮವನ್ನು ತಿಳಿದುಕೊಂಡರೆ, ಶ್ರದ್ಧೆಯಿಂದ ಶ್ರಮಿಸಿದರೆ ವಿಷವನ್ನೂ ಅಮೃತವನ್ನಾಗಿಸಬಹುದು. ಈ ಮನಸ್ಸಾದರೂ ಏನು?

ಹೆಚ್ಚಿನ ಶಾಸ್ತ್ರೀಯ ನಿರೂಪಣೆಗಳು ನಮಗೆ ಬೇಕಿಲ್ಲ. ಮನಸ್ಸೆಂದರೆ ಅಸಂಖ್ಯ ಯೋಚನೆ, ಭಾವನೆ, ಕಲ್ಪನೆ, ಸಂಕಲ್ಪ–ಇವುಗಳಿಂದ ಅಥವಾ ಇಚ್ಛಾಕ್ರಿಯಾಜ್ಞಾನಾತ್ಮಕ ಯೋಚನೆಗಳಿಂದ\break ಕೂಡಿದ, ನಮ್ಮ ವ್ಯಕ್ತಿತ್ವವನ್ನು ರೂಪಿಸುವ ಜಟಿಲವೂ, ಅತ್ಯಂತ ಸೂಕ್ಷ್ಮವೂ, ಸಂಕೀರ್ಣವೂ ಆದ ಒಂದು ವಿಶಿಷ್ಟಶಕ್ತಿ. ಈ ಶಕ್ತಿಯ ಉಪಯೋಗದಿಂದಲೇ ನಮ್ಮ ಭವಿಷ್ಯವನ್ನು ನಾವು ನಿರ್ಮಿಸಿ\-ಕೊಳ್ಳುವುದು.

ಬದುಕಿನಲ್ಲಿ ನಮ್ಮ ಎಲ್ಲ ಸಾಧನೆ–ಸಿದ್ಧಿಗಳು ಮನಸ್ಸಿನಲ್ಲಿ ಉದಿಸಿದ ಯೋಚನೆ, ಕಲ್ಪನೆ, ಸಂಕಲ್ಪ, ಭಾವನೆಗಳ ಮಾದರಿಯನ್ನನುಸರಿಸಿಯೇ ಸಾಧ್ಯವಾಗಿವೆ. ಪಶ್ಚಿಮದ ಅನುಭವಿಯೊಬ್ಬನ ವಾಣಿಯನ್ನು ಆಲಿಸಿ:

‘ಒಂದು ತಿಂಗಳ ಕಾಲ ಪ್ರತಿದಿನವೂ ದಿನದ ಬೇರೆಬೇರೆ ಹೊತ್ತಿನಲ್ಲಿ ಐದು ಬಾರಿ ಮನಸ್ಸಿನಲ್ಲಿ ಉದಿಸುವ ಯೋಚನೆ ಭಾವನೆಗಳನ್ನು ಪರಿಶೀಲಿಸಿದರೆ, ನಿಮ್ಮ ಮುಂದಿನ ಜನ್ಮದ ಜೀವನದ ಆಗುಹೋಗುಗಳನ್ನು ಹೇಗೆ ವ್ಯವಸ್ಥೆ ಮಾಡಿದ್ದೀರಿ ಎಂಬ ಅರಿವು ನಿಮಗಾಗುವುದು. ನೀವು ನಿಮ್ಮ ಈ ಸದ್ಯದ ಮನಸ್ಸಿನ ಗತಿಯ ಬಗ್ಗೆ ಪರಿಶೀಲನೆ ಮಾಡಿ ಕಂಡುಕೊಂಡ ಸಂಗತಿಗಳು ನಿಮಗೆ ಹಿಡಿಸದಿದ್ದಲ್ಲಿ, ಇಂದಿನಿಂದಲೇ ನಿಮ್ಮ ಯೋಚನೆ ಭಾವನೆಗಳನ್ನು ಬದಲಿಸಿಕೊಳ್ಳಲು ಪ್ರಾಮಾಣಿಕವಾಗಿ ಯತ್ನಿಸಿ.’

ಮೇಲೆ ಹೇಳಿದ ವಿಚಾರ ನಮ್ಮ ಪಾಲಿಗೆ ನಿಚ್ಚಳವಾದರೆ ನಮ್ಮ ಯೋಚನೆಗಳನ್ನು ನಿಯಂತ್ರಿಸಲು ನಾವು ಸ್ವಲ್ಪವಾದರೂ ಯತ್ನಿಸುತ್ತೇವೆ. ಜೀವನದಲ್ಲಿ ಒಂದು ಉನ್ನತ, ಮುಖ್ಯ ಗುರಿಯನ್ನು ಹೊಂದಲು ಸಾಧ್ಯವಾಗುವಂತೆ ನಮ್ಮ ಯೋಚನೆ, ಕಲ್ಪನೆ, ಭಾವನೆಗಳನ್ನು ರೂಪಿಸಿಕೊಳ್ಳುವ ಅಭ್ಯಾಸ ಅತ್ಯಂತ ಆವಶ್ಯಕ.

ಸ್ವಯಂಚಾಲಿತ ಯಂತ್ರಗಳ ಬಳಕೆ ಇತ್ತೀಚೆಗೆ ಸರ್ವತ್ರ ಪ್ರಸಾರವಾಗುತ್ತಿದೆ. ಕಂಪ್ಯೂಟರೀ\-ಕರಣದ ಮಾತನ್ನು ಇಂದು ಅಭಿವೃದ್ಧಿಶೀಲ ರಾಷ್ಟ್ರಗಳಲ್ಲಿ ಕೇಳಲಾಗುತ್ತಿದೆ. ನಮಗೆ ದೇವದತ್ತವಾಗಿ ಸಿಕ್ಕಿರುವ ಪರಮಾದ್ಭುತ ಯಂತ್ರವಾದ ಮನಸ್ಸಿಗೆ ಒಳ್ಳೆಯ ಯೋಚನೆ ಮತ್ತು ಒಳ್ಳೆಯ ಆದರ್ಶಗಳನ್ನು ಕ್ರಮವರಿತು ‘ಫೀಡ್​’ ಮಾಡಿದರೆ, ನಾವು ಪಡೆಯಬಹುದಾದ ಸಂತೋಷ, ಶಾಂತಿ, ಸಂತೃಪ್ತಿ, ಸದ್ಭಾವನೆಗಳನ್ನು ಅಲುಗಾಡಿಸಲು ಯಾರಿಂದಲೂ ಸಾಧ್ಯವಿಲ್ಲ–ಎಂಬುದನ್ನು ಯಾರೂ ಅಷ್ಟಾಗಿ ಗಮನಿಸಿದಂತಿಲ್ಲ. ಆಂಗ್ಲಭಾಷೆಯಲ್ಲಿ ಆರ್ಡರ್ \enginline{(Order)} ಎಂಬ ಶಬ್ದವಿದೆ. ನಮ್ಮಲ್ಲಿ ‘ಪುತ’ ಎಂಬ ಶಬ್ದವನ್ನು ಪುರಾತನರು ಉಪಯೋಗಿಸಿದರು. ಭಗವಂತನ ಸೃಷ್ಟಿಯಲ್ಲಿ ಒಂದು ಕ್ರಮ ವ್ಯವಸ್ಥೆಯನ್ನು ಕಾಣುತ್ತೇವೆ. ಕ್ರಮವಾಗಿ ಸೂರ್ಯ, ಚಂದ್ರ, ತಾರೆ, ನೀಹಾರಿಕೆಗಳು ಸುತ್ತುತ್ತಿವೆ. ಹಗಲು, ರಾತ್ರಿ, ಕಾಲಭೇದ, ಪುತುಭೇದಗಳು ಕ್ರಮವರಿತು ನಡೆ ಯುತ್ತಿವೆ. ನಮ್ಮ ಕಣ್ಣಿಗೆ ಕಾಣುವ ಜಗತ್ತಿನಲ್ಲಿ ಮಾತ್ರವಲ್ಲ, ಜಗತ್ತನ್ನು ಕಾಣುವ ನಮ್ಮ ದೇಹೇಂದ್ರಿಯ ಎನ್ನುವ ಯಂತ್ರದಲ್ಲೂ ಹೃದಯದ ಬಡಿತ, ಶ್ವಾಸೋಚ್ಛ್ವಾಸ, ರಕ್ತಪರಿಚಲನೆ, ಎಚ್ಚರ, ನಿದ್ರೆಗಳು ಒಂದು ತೆರನಾದ ಲಯಬದ್ಧತೆಯನ್ನು ಅನುಸರಿಸುತ್ತವೆ. ಇದಕ್ಕಿಂತಲೂ ಸೂಕ್ಷ್ಮವಾದ ಮನುಷ್ಯನ ಸುಖ ದುಃಖವನ್ನು ನಿಯಂತ್ರಿಸುವ ನೈತಿಕ ನಿಯಮವಿದೆ ಎನ್ನುತ್ತಾರೆ. ಎಲ್ಲೆಲ್ಲೂ ವ್ಯವಸ್ಥೆ, ಶಿಸ್ತು, ನಿಯಮಗಳು, ಯಶಸ್ಸು, ವಿಜಯಗಳ ರಹಸ್ಯವಿರುವುದು ಇಲ್ಲಿಯೇ. ಕ್ರಮ, ನಿಯಮ, ವ್ಯವಸ್ಥೆ–ನಮ್ಮ ಬದುಕನ್ನು ಆವರಿಸಿಕೊಳ್ಳಬೇಕು. ನಮ್ಮ ಮನಸ್ಸೆಂಬ ಕಂಪ್ಯೂಟರಿನ ರಚನೆಯಲ್ಲಿ ವಿಶ್ವಾಸ ಪ್ರೀತಿಯಿಂದ ಕೂಡಿದ ನಿಷ್ಪಕ್ಷಪಾತ ಪರಿಶೀಲನೆ, ನಿಃಸ್ವಾರ್ಥ ದೃಷ್ಟಿ, ಏಕಾಗ್ರತೆ, ಸ್ವಲ್ಪವಾದರೂ ಹಾಸ್ಯದೃಷ್ಟಿ ಸೇರಿಕೊಂಡಿದ್ದರೆ ಅದು ಬಹಳ ಉಪಯೋಗಿಯಾಗುವುದು. ಜಾಡ್ಯ ಮನೋಭಾವ, ಅನೈಕಾಗ್ರತೆ, ಅವ್ಯವಸ್ಥೆ, ಸಿಡುಕು, ಬಲವಾದ ಪೂರ್ವಾಗ್ರಹಗಳು ಮನಸ್ಸೆಂಬ ಯಂತ್ರವನ್ನು ಹಾಳುಮಾಡಿಬಿಡುತ್ತವೆ. ಕ್ರಮ–ನಿಯಮ, ವ್ಯವಸ್ಥೆ, ತಾಳ್ಮೆ, ಸಮತೋಲ, ಸದುದ್ದೇಶ,\break ವ್ಯಾವಹಾರಿಕತೆ–ಇವುಗಳ ಸಹಕಾರದಿಂದ ಯೋಜನೆಗಳನ್ನು ರೂಪಿಸಿಕೊಂಡರೆ, ನಮ್ಮನ್ನು\break ದುರ್ಬಲ\-ಗೊಳಿಸುವ, ದುರಂತಕ್ಕೆಳೆಸುವ ನಿಷೇಧಾತ್ಮಕ ಭಾವನೆಗಳು ಸಮೀಪ ಸುಳಿಯಲಾರವು.


\section*{ಸಾರ್ವಕಾಲಿಕ ಸರಳ ಚಿಕಿತ್ಸೆ}

\addsectiontoTOC{ಸಾರ್ವಕಾಲಿಕ ಸರಳ ಚಿಕಿತ್ಸೆ}

ಕಾಯಿಲೆಯನ್ನು ಓಡಿಸುವ ಪುರಾತನ, ಸಾರ್ವಕಾಲಿಕ ಸರಳ ಸೂತ್ರಗಳಿವು.\footnote{\engfoot{Quoted by Dr.\ Roy D. Kirkland, D. O. in `Health in your Design' in the Edger Caycee Reader.}}

\begin{enumerate}
\itemsep=1pt
\item ಜೀವಿಗಳಲ್ಲಿ ಅನುಕಂಪೆ, ದಯೆ, ಸಹಾನುಭೂತಿಗಳನ್ನು ಬೆಳೆಸಿ ಉಳಿಸಿಕೊಂಡಲ್ಲಿ ತನ್ಮೂಲಕ ಆಂತರಿಕ ಸುಖಸಂತೋಷಗಳು ಹೆಚ್ಚುತ್ತವೆ ಮಾತ್ರವಲ್ಲ, ಎಲ್ಲ ಜೀರ್ಣರಸಗಳ ನಿರಾತಂಕ ಉತ್ಪಾದನೆಯಿಂದ ಸಂಬಂಧಿತ ರೋಗರುಜಿನಗಳು ದೂರವಾಗುತ್ತವೆ.

 \item ನಿಃಸ್ವಾರ್ಥತೆ, ಕಾರ್ಯತತ್ಪರತೆ, ವಿಶಾಲ ಮನೋಭಾವಗಳನ್ನು ರೂಢಿಸಿಕೊಂಡಲ್ಲಿ ದೈಹಿಕ ದೃಢತೆ ಹಾಗೂ ಮಾನಸಿಕ ಸ್ಥಿರತೆ ಹೆಚ್ಚಿ, ಪದೇ ಪದೇ ಆರೋಗ್ಯ ಕೆಡುವ ಸಂಭವವೇ ಇರುವು\-ದಿಲ್ಲ.

 \item ಬಿಗುವನ್ನು ತೊರೆದು ನಗುವನ್ನು ಹೊರಸೂಸುತ್ತಾ, ಆತ್ಮವಿಶ್ವಾಸದ ಕೆಚ್ಚಿನಿಂದ ಮುನ್ನಡೆದಲ್ಲಿ, ಪರಸ್ಪರ ವೈಷಮ್ಯದ, ವಿರೋಧದ ಯಾವುದೇ ಗಾಯವನ್ನೂ ಬೇಗ ವಾಸಿಮಾಡಬಹುದು.

 \item ಪರೋಪಕಾರ ಗುಣ, ಉದಾರತೆ ಹಾಗೂ ಸಹಕಾರ ಮನೋಭಾವ – ಇವುಗಳನ್ನು ಮೈಗೂಡಿಸಿ\-ಕೊಂಡಲ್ಲಿ, ಆಂತರಿಕ ಸಾಮರಸ್ಯವು ಕಂಡು ಬಂದು, ದೇಹದ ಕೀಲುಗಳ ಬಿಗಿ ಕಣ್ಮರೆಯಾಗಿ, ಚಲನವಲನಗಳಲ್ಲಿ ಹೊಸತನ ಕಂಡುಬರುತ್ತದೆ.

 \item ತನ್ನಂತೆ ಇತರರು ಎಂಬ ವಿಶ್ವಭ್ರಾತೃತ್ವದ ಈ ವಿಶಾಲ ಮನೋಭಾವವನ್ನು ದೇಹದ ಕಣಕಣದಲ್ಲೂ ತುಂಬಿಕೊಂಡು, ಹೃದಯದ ಮಿಡಿತದೊಂದಿಗೆ ಪ್ರತಿಸ್ಪಂದಿಸುತ್ತಿದ್ದರೆ, ಕಳ್ಳತನ, ಕೊಲೆ, ಸುಲಿಗೆ, ಅನ್ಯಾಯ, ಅನಾಚಾರಗಳ ಯಾವ ಸಾಂಕ್ರಾಮಿಕರೋಗವೂ ಬಂದು ತಟ್ಟದು.

 \item ‘ತಾಳಿದವ ಬಾಳಿಯಾನು’ ಎಂಬುದನ್ನು ನೆಚ್ಚಿ ಸಹನೆಯನ್ನು ಹೆಚ್ಚಿಸಿಕೊಂಡು ವರ್ತಿಸಿದಲ್ಲಿ, ದೂರದೃಷ್ಟಿಯೊಂದಿಗೆ ಅಂತರ್ದಷ್ಟಿಯೂ ವೃದ್ಧಿಸಿ ಬಾಳು ಬೆಳಗುತ್ತ ಹೋದೀತು, ಜೀವನ ಜೇನಾದೀತು.

 \item ‘ಮಾತು ಕಡಿಮೆ, ಜಾಸ್ತಿ ದುಡಿಮೆ’ ಎಂಬೀ ತತ್ತ್ವದನ್ವಯ ಅನಾವಶ್ಯಕ ಮಾತುಕತೆಗಳನ್ನೂ, ಕ್ಷುಲ್ಲಕ ವ್ಯವಹಾರಗಳನ್ನೂ ನಿಲ್ಲಿಸಿ ಕಾರ್ಯೋನ್ಮುಖರಾದಲ್ಲಿ ಆಸ್ಪತ್ರೆಯ ಒಳ ರೋಗಿಯಾಗಿ ನರಳುವ ಪ್ರಮೇಯವೇ ಬಾರದು, ಬಂಗಾರದ ಬದುಕು ಕನ್ನಡಿಯೊಳಗಿನ ಗಂಟಿನಂತೆ ಎಂದೆಂದಿಗೂ ಆಗದು.

 \item ಪೂರ್ವಾಪರ ವಿಮರ್ಶೆ ಹಾಗೂ ವಿವೇಕಗಳನ್ನು ನಿಮ್ಮದಾಗಿಸಿಕೊಂಡಲ್ಲಿ, ಎಂಥ ಜಟಿಲ, ಸಂದಿಗ್ಧ ಸನ್ನಿವೇಶಗಳಲ್ಲೂ ಯಶಸ್ಸು ನಿಮ್ಮದಾಗಬಲ್ಲದು. ಎಂಥ ವಿಷಮ ಪರಿಸ್ಥಿತಿಯನ್ನೂ ಸರಿಪಡಿಸುವ ನವಶಕ್ತಿ ಸಂಜೀವಿನಿಯ ರಸಭರಿತ ಟಾನಿಕ್​ ಇದಲ್ಲದೆ ಬೇರೆ ಇರದು.

 \item ದಿನನಿತ್ಯವೂ ಮಾಡುವ ಕ್ರಮಬದ್ಧ ಪ್ರಾರ್ಥನೆಯಿಂದ ದೇಹದ ಸರ್ವತೋಮುಖ ಏಳಿಗೆ ಸುಲಭ ಸಾಧ್ಯವಾಗಿ, ಆಂತರಿಕ ಶಕ್ತಿ, ಚೈತನ್ಯ ಹಾಗೂ ಸ್ಥಿರತೆ ಹೆಚ್ಚುತ್ತದೆ. ತನ್ಮೂಲಕ ಇತರರಲ್ಲೂ ಈ ಹೊಸತನ ಹಾಗೂ ಸ್ಫೂರ್ತಿಯನ್ನು ಹೆಚ್ಚಿಸುವ ಸಾಮರ್ಥ್ಯ ಪುಟಿದೇಳುತ್ತದೆ.

\end{enumerate}

ನೀವು ಶಿವಲಿಂಗದ ಮೇಲಿರುವ ಧಾರಾಪಾತ್ರೆಯನ್ನು ನೋಡಿದ್ದೀರಾ? ಅದರಿಂದ ಹನಿಹನಿಯಾಗಿ ಬೀಳುವ ಜಲದಂತೆ ದಿನ ನಿತ್ಯವೂ ಹಟತೊಟ್ಟು ಮೇಲಿನ ಭಾವನೆಗಳನ್ನು ನಿಮ್ಮದಾಗಿ ಮಾಡಿಕೊಂಡರೆ ಮನಸ್ಸು ಪ್ರಸನ್ನವಾಗಿಯೇ ಇರಲು ಸಾಧ್ಯ. ಯಾವುದೇ ಔಷಧ, ಟಾನಿಕ್ಕುಗಳಿ ಗಿಂತ ಇದು ಹೆಚ್ಚಿನ ಆರೋಗ್ಯಕಾರಿ. ಔಷಧ ಸೇವನೆಯಿಂದ ಪ್ರತಿಕ್ರಿಯೆಯುಂಟಾಗಿ ತೊಂದರೆ ಯಾಗುವುದುಂಟು. ಸದ್ಭಾವನೆಗಳಿಂದ ಆರೋಗ್ಯಕ್ಕೆ ಕೆಡುಕು ಎಂದಿಗೂ ಆಗದು.


\section*{ಮನೋಬಲವೇ ಮಹಾಬಲವು}

\addsectiontoTOC{ಮನೋಬಲವೇ ಮಹಾಬಲವು}

ಆರೋಗ್ಯವರ್ಧಕ ಟಾನಿಕ್ಕಿಗಿಂತಲೂ ಹೆಚ್ಚಿನ ಸಹಾಯಗೈಯುವ ಈ ಸದ್ಭಾವನೆಗಳನ್ನು ರೂಢಿಸಿ\-ಕೊಳ್ಳುವ ಕ್ರಮವನ್ನು ಧರ್ಮಗ್ರಂಥಗಳು ಬೋಧಿಸುತ್ತವೆ. ನಮ್ಮ ದೇಶದಲ್ಲಿ ಅತ್ಯಂತ ಪ್ರಾಚೀನ ಕಾಲದಿಂದಲೂ ಈ ಬಗ್ಗೆ ಸಾಕಷ್ಟು ಸಂಶೋಧನೆಗಳಿಂದ ತಥ್ಯಗಳನ್ನು ಸಂಗ್ರಹಿಸಿದ್ದಾರೆ. ಅವುಗಳನ್ನು ಮನನ ಮಾಡಿ ನಮ್ಮ ಜೀವನದಲ್ಲಿ ಅಳವಡಿಸಿಕೊಳ್ಳದಿದ್ದರೆ ನಮ್ಮ ಬದುಕಿನಲ್ಲಿ ಶಾಂತಿ, ಸ್ಥೈರ್ಯಗಳನ್ನು ಪಡೆಯುವುದು ಕನಸಿನ ಮಾತಾಗಿಯೇ ಉಳಿಯುವುದು. ಕೃತಕ ವಿಧಾನಗಳಿಂದ ಕ್ಷಣಿಕ ಉತ್ಸಾಹ ಸ್ಫೂರ್ತಿಯನ್ನು ಪಡೆಯುವ ಔಷಧಗಳಾಗಲೀ, ಉತ್ತೇಜಕ ಪಾನೀಯಗಳಾಗಲೀ, ಶರೀರ ಮನಸ್ಸುಗಳನ್ನು ದುರ್ಬಲಗೊಳಿಸಿ, ವ್ಯಕ್ತಿಯನ್ನು ನಿರ್ವೀರ್ಯನನ್ನಾಗಿ ಮಾಡುವ ವಿಚಾರ ತಿಳಿದುಕೊಳ್ಳಲು ಕಷ್ಟವಿಲ್ಲ. ಕುಡಿತದ ಚಟಕ್ಕೆ ಬಲಿಯಾದವರು, ಮತ್ತು ಬರಿಸುವ ಔಷಧ ಸೇವಿಸು\-ವವರು ಆ ಬಂಧನದಲ್ಲಿ ಸಿಕ್ಕಿ ನರಳುತ್ತ, ತಾವೂ ಕೆಟ್ಟು, ಕುಟುಂಬ ಹಾಗೂ ಸಮಾಜದ ಸ್ವಾಸ್ಥ್ಯವನ್ನು ಹೇಗೆ ಕೆಡಿಸಬಲ್ಲರೆಂಬುದನ್ನು ಅಂಕಿ ಅಂಶಗಳಿಂದ ತಜ್ಞರು ವಿವರಿಸುತ್ತಾರೆ. ಯೋಗಾಸನ, ವಿಪಶ್ಯನಗಳಿಂದ ರೋಗ ನಿವಾರಣೆ, ಸ್ವಾಸ್ಥ್ಯ ರಕ್ಷಣೆಗೆ ಬಹಳಷ್ಟು ಉಪಯೋಗವಿದೆ ಎಂಬುದು ದಿಟ. ಆದರೆ ಮಾನಸಿಕ ಪಾವಿತ್ರ್ಯಕ್ಕೆ ಯೋಗಶಾಸ್ತ್ರ ವಿಶೇಷ ಪ್ರಾಧಾನ್ಯ ನೀಡಿದೆ ಎಂಬುದು ಹಲವು ಮಂದಿ ಯೋಗಾಸನ ನಿರತರಿಗೆ ತಿಳಿದಿಲ್ಲ. ಮೋಸ, ವಂಚನೆಗಳಿಂದ ವ್ಯಾಪಾರದಲ್ಲಿ ಲಾಭಗಳಿಸಿ, ದೇವಸ್ಥಾನದಲ್ಲಿ ಕಾಣಿಕೆಡಬ್ಬಿಗೆ ಹಣಹಾಕಿ ಸಂತೃಪ್ತಿ ಪಡುವ ವ್ಯಕ್ತಿ ನಿಜವಾದ ಧಾರ್ಮಿಕನಾದಾನೆ? ಮನಸ್ಸಿನಲ್ಲಿ ಕೋಪತಾಪ, ಚಿಂತೆ, ಭಯ, ದ್ವೇಷ ಅಸೂಯೆಗಳಿಗೆ ಇಂಬುಕೊಡುತ್ತ ಶೀರ್ಷಾಸನ, ಶಲಭಾಸನ, ಸರ್ವಾಂಗಾಸನ ಹಾಕಿದೊಡನೆ, ಎಷ್ಟೇ ಸೈಂಟಿಫಿಕ್ ಆಗಿ ಯೋಗಾಸನ ಮಾಡಿದರೂ ಸ್ವಾಸ್ಥ್ಯರಕ್ಷಣೆ ಸಾಧ್ಯವಾಗುವುದೆ? ಆದಂತೆ ಕಂಡರೂ ಅದು ತಾತ್ಕಾಲಿಕ.

ಅಮೇರಿಕದಲ್ಲಿ ದೀರ್ಘಕಾಲ ವೇದಾಂತ ಪ್ರಚಾರ ಮಾಡಿದ ಸ್ವಾಮಿ ಸತ್ಪ್ರಕಾಶಾನಂದಜೀ ಗ್ರಂಥ ಒಂದರಲ್ಲಿ ತಮ್ಮ ಒಂದು ಅನುಭವವನ್ನು ಹೇಳಿದ್ದಾರೆ. ಧರ್ಮ, ಆಧ್ಯಾತ್ಮಿಕ ಸಾಧನೆಯನ್ನು ಕುರಿತು ಹಲವು ಉಪನ್ಯಾಸಗಳನ್ನು ನೀಡಿದ ಬಳಿಕ ಅವರ ಉಪನ್ಯಾಸವನ್ನು ಕೇಳಿದ ಮಹಿಳೆಯೊಬ್ಬಳು ‘ಸ್ವಾಮೀಜಿ, ನಾವು ಅಮೇರಿಕನ್ನರಿಗೆ ಧರ್ಮ, ಆಧ್ಯಾತ್ಮಿಕತೆಗಿಂತಲೂ ಮಾನಸಿಕ ದಾರ್ಢ್ಯ ಮತ್ತು ಶಾಂತಿಯನ್ನು ಕಾಪಾಡಿಕೊಳ್ಳುವ ರಹಸ್ಯ ಬೇಕಾಗಿದೆ. ನೀವು ಅದನ್ನು ನಮಗೆ ತಿಳಿಸಿಕೊಟ್ಟರೆ ಅತಿ ದೊಡ್ಡ ಸೇವೆ ಸಲ್ಲಿಸಿದಂತಾಗುವುದು. ನರಮಂಡಲದ ಒತ್ತಡ, ಜಾಡ್ಯಗಳಿಂದ ನಮಗೆ ಬಿಡುಗಡೆ ಬೇಕಾಗಿದೆ’ ಎಂದಳಂತೆ. ಮಾನಸಿಕದಾರ್ಢ್ಯ ಹಾಗೂ ಶಾಂತಿಯನ್ನು ಕಾಪಾಡಿಕೊಳ್ಳಲು ಧರ್ಮದ ಹಿನ್ನೆಲೆ ಹೇಗೆ ಸಹಕಾರಿ ಎಂಬುದರ ಅರಿವು ಹೆಚ್ಚಿನ ವಿದ್ಯಾಬುದ್ಧಿ ಸಂಪನ್ನರೆನಿಸಿಕೊಂಡವರಿಗೇ ತಿಳಿಯದು!

ಒತ್ತಡ, ಜಾಡ್ಯಗಳಿಂದ ಬಿಡಿಸಿಕೊಳ್ಳಲು ವಿಶ್ರಾಮ ಭಾವನೆ ಅತ್ಯಂತ ಆವಶ್ಯಕ ಎಂಬುದು ಪ್ರಯೋಗದಿಂದ ಸಿದ್ಧವಾಗಿದೆ. ಈ ಕ್ಷೇತ್ರದಲ್ಲಿ ಪ್ರವರ್ತಕನ ಸ್ಥಾನವನ್ನು ಪಡೆದ ಹ್ಯಾನ್ಸ್​ಸಿಲೀ\break ‘ನಿಜವಾದ ವಿಶ್ರಾಮ’ ಸ್ಥಿತಿಗೇರಿದಾಗ ಎಲ್ಲ ಒತ್ತಡಗಳಿಂದ ಬಿಡಿಸಿಕೊಳ್ಳಬಹುದು ಎಂದಿದ್ದಾರೆ. ಆದರೆ ಈ ನಿಜವಾದ ವಿಶ್ರಾಮ ಭಾವನೆಯನ್ನು ಸೂಚನೆಯಿಂದ, ಸುಪ್ತ್ಯಾವಾಹನೆಯಿಂದ ಕೆಲ ಮಟ್ಟಿಗೆ ಉಂಟು ಮಾಡಬಹುದಾದರೂ, ಅದು ಸ್ಥಾಯಿಯಾಗಿ ಉಳಿಯದು. ಅದಕ್ಕೆ ಬೇರೆಯೇ ಮಾರ್ಗವನ್ನು ಕಂಡುಕೊಳ್ಳಬೇಕು. ಅದೇ ಆಧ್ಯಾತ್ಮಿಕ ಪಥ. ಜೀವನವನ್ನು ಕುರಿತ ಹಾಗೂ ಸುಖ ದುಃಖಗಳ ಮೂಲ ನಿಯಮದ ಬಗೆಗೆ ನಮ್ಮ ದೃಷ್ಟಿಕೋನ ಬದಲಾವಣೆಯಾಗದೇ ನಿಜವಾದ ವಿಶ್ರಾಮ ಭಾವನೆಯನ್ನು ಪಡೆದರೂ ಅದು ಕ್ಷಣಿಕವೆಂದು ಗಾಢಚಿಂತನೆಯಿಂದ ತಿಳಿಯದಿರದು.

ಪಶ್ಚಿಮ ಪ್ರತಿಭಾಶಾಲಿಗಳೇ ಅಲ್ಲಿನ ಸಮಾಜ ದುರಂತದತ್ತ ಧಾವಿಸುವುದರ ಕಾರಣವನ್ನು ಹೀಗೆಂದು ಹೇಳಿದ್ದಾರೆ:

‘ಜಡವಾದ ಭೌತವಾದದ ಒಲವೇ ನಮ್ಮಲ್ಲಿ ಹಿಂಸೆಯ ಪ್ರವೃತ್ತಿಯನ್ನು ಹುಟ್ಟಿಸುತ್ತದೆ. ನಮ್ಮ ಸಮಾಜವು ಕುಟುಂಬ ಅಥವಾ ವರ್ಗದ ನಿಯಂತ್ರಣದ ಮೇಲೆ ಕಟ್ಟಲ್ಪಟ್ಟಿಲ್ಲ. ಅದು ಯಶಸ್ಸಿನ ಮೇಲೆ ನಿಂತಿದೆ. ಯಶಸ್ಸು ಇಲ್ಲದಿದ್ದರೆ ಜನ ಹತಾಶರಾಗುತ್ತಾರೆ. ಹತಾಶೆಯಿಂದ ಹತ್ಯಾಕಾಂಡ–ಹಿಂಸೆ.’

ಜೀವನದಲ್ಲಿ ಸಹಕಾರಕ್ಕಿಂತ ಸ್ಪರ್ಧೆ, ಈರ್ಷ್ಯೆಗಳನ್ನು ಬದುಕುವ ದಾರಿಯೆಂದು ಇಟ್ಟು ಕೊಳ್ಳುವ ನಮ್ಮ ಒತ್ತಾಸೆಯ ಒಂದು ಪ್ರತಿಫಲ ಹಿಂಸೆ. ಸ್ಪರ್ಧೆಯ ಪ್ರಚೋದನೆಯಿಂದ ಆಕ್ರಮಣ ಮನೋವೃತ್ತಿಯನ್ನೂ, ವೈರ ಪ್ರವೃತ್ತಿಯನ್ನೂ ಹೆಚ್ಚಿಸಿಕೊಳ್ಳದೇ ಗತ್ಯಂತರವಿಲ್ಲ. ಸ್ಪರ್ಧೆ ಉಪಯುಕ್ತ ಸಂಗತಿ ನಿಜ. ಆದರೆ ಅದು ಉತ್ಪಾದನೆಯನ್ನು ಹೆಚ್ಚಿಸುವುದಕ್ಕಾಗಿ ಉಂಟಾದರೆ ಆ ಉತ್ಪಾದನೆಯನ್ನು ಹೆಚ್ಚಿಸುವುದಕ್ಕಾಗಿ ತೆರುವ ಬೆಲೆ ಯಾವುದು? ಜನರನ್ನು ಬಗ್ಗಿಸುವುದು, ಬೀಳಿಸುವುದು, ಮೆಟ್ಟಿ ತುಳಿಯುವುದು.

ಜಡವಾದದ ದೃಷ್ಟಿಕೋನ ಸ್ಪರ್ಧೆಯ ಮನೋವೃತ್ತಿಯೇ ಪ್ರಧಾನವಾಗಿರುವ ಸಮಾಜದಲ್ಲಿ ಬದುಕುವ ವ್ಯಕ್ತಿ ಯಾವೆಲ್ಲ ಉಪಾಯಗಳಿಂದ ವಿಶ್ರಾಂತಿ ಮನೋಭಾವನೆಯನ್ನು ಹೊಂದಲು ಯತ್ನಿಸಿದರೂ ಅವನ ಅಂತರಂಗದ ತುಮುಲವನ್ನು ತಡೆಯಲು ಸಾಧ್ಯವಾದೀತೇ? ಮನುಷ್ಯ ಸ್ವಭಾವದ ವೈವಿಧ್ಯ ವೈಚಿತ್ರ್ಯ, ತನ್ನ ಆಸೆ ಆಕಾಂಕ್ಷೆಗಳ ಪೂರೈಕೆಗಾಗಿ ಆತನ ಚಡಪಡಿಕೆ,\break ಹಲವು ವಸ್ತುಗಳ ಬೆನ್ನು ಹತ್ತಿ ಓಡುವ ಪ್ರವೃತ್ತಿ, ಅವನಿಗೇ ತಿಳಿಯದಂತೆ ಮನಸ್ಸು ಅವನನ್ನು ಕುಣಿಸುವ ವಿಧಾನ, ಸಂಸ್ಕಾರದ ಸೆಳೆತಕ್ಕೆ ಸಿಕ್ಕಿ ಅವನು ನಜ್ಜು ಗುಜ್ಜಾಗುವ ರೀತಿ, ಅವನ\break ಅಸಹಾಯಕತೆ – ಇವುಗಳನ್ನು ಪರಿಶೀಲಿಸಿದರೆ ಕೇವಲ ಆರ್ಥಿಕ ಹಾಗೂ ವೈಜ್ಞಾನಿಕ ಪ್ರಗತಿಗಳಿಂದ ಮನುಷ್ಯ ಒಳ್ಳೆಯವನಾಗಿಬಿಡುತ್ತಾನೆಂದು ತಿಳಿಯುವುದು ಮೂರ್ಖತನವೇ ಸರಿ. ತತ್ತ್ವಜ್ಞಾನಿ ಷೋಫನೀಯರ್ ಮನುಷ್ಯರನ್ನು ಒಂದೆಡೆ ಮುಳ್ಳು ಹಂದಿಗಳಿಗೆ ಹೋಲಿಸಿದ್ದಾರೆ. ಮುಳ್ಳು ಹಂದಿಗಳನ್ನು ಇಕ್ಕಟ್ಟಾದ ಒಂದು ಜಾಗದಲ್ಲಿ ಕೂಡಿಟ್ಟರೆ ಒಂದರ ಮುಳ್ಳು ಇನ್ನೊಂದನ್ನು ಚುಚ್ಚಿ ಅವು ನಾಶವಾಗುತ್ತವಷ್ಟೆ. ಅಂತೆಯೇ ಮನುಷ್ಯನ ಅತಿಯಾದ ಸ್ವಾರ್ಥ ಪರಸ್ಪರ ನಾಶಕ್ಕೆ ಕಾರಣವಾಗುತ್ತದೆ. ಪ್ರತಿಯೊಬ್ಬ ವ್ಯಕ್ತಿಯೂ ತನ್ನ ಸ್ವಾರ್ಥವನ್ನು ಇತರರಿಗೆ ಕೆಡುಕನ್ನು ಉಂಟುಮಾಡದಂತೆ ಮೊಟಕುಗೊಳಿಸಲು ಪ್ರೇರಿಸುವ ಸಾರ್ವತ್ರಿಕ ಸ್ಫೂರ್ತಿಯ ಹಿನ್ನೆಲೆ ಯಾವುದು? ಶ್ರದ್ಧಾಕೇಂದ್ರ ಯಾವುದು? ಇದು ಮುಖ್ಯ ಪ್ರಶ್ನೆ. ಈ ಪ್ರಶ್ನೆಗೆ ಸರಿಯಾದ ಉತ್ತರ ನಮಗೆ ಬೇಕು. ಸಮಾಜದ ಎಲ್ಲ ಸದಸ್ಯರೂ ಎಲ್ಲಿ ಅತ್ಯಂತ ಸಂಯಮದಿಂದ ನಡೆದುಕೊಳ್ಳುವರೋ ಅದೇ ಶ್ರೇಷ್ಠ ಸಮಾಜ. ಅಲ್ಲಿ ಒಗ್ಗಟ್ಟು, ಶಾಂತಿ, ಸಹಕಾರ, ಸೌಹಾರ್ದ ನೆಲೆಸಲು ಸಾಧ್ಯ. ಆದರೆ ಆಧುನಿಕ ಸಮಾಜದಲ್ಲಿ ಇದು ಸಾಧ್ಯವಾಗುತ್ತಿದೆಯೆ? ಎಲ್ಲೆಲ್ಲೂ ಗೌಜುಗೊಂದಲ, ವೈರವಿದ್ವೇಷ, ಹೊಡೆತಬಡಿತಗಳ ಚಿತ್ರ! ಕಾರಣವೇನು? ಅನಿಯಂತ್ರಿತ ಸ್ವಾರ್ಥವೇ. ಧನಸಂಗ್ರಹ ಮಾಡುವುದು, ಅಧಿಕಾರಗಳಿಸುವುದು, ಸ್ಥಾನಮಾನ ಪಡೆಯುವುದು, ಜನಪ್ರಿಯತೆಯ ದಾರಿ ಹುಡುಕುವುದು. ಆ ಸ್ವೇಚ್ಛಾ ವರ್ತನೆಗೆ ಹಂಬಲಿಸುವ ಮನೋವೃತ್ತಿ ಸ್ವಾರ್ಥವನ್ನು ಮತ್ತಷ್ಟು ಚೂಪಾಗಿಸುವ ಜಡವಾದದ ದೃಷ್ಟಿಕೋನ. ಆಂಗ್ಲ ತತ್ತ್ವಜ್ಞಾನಿ ಹಾಬ್ಸ್ ಹೇಳಿದ: ‘ಮನುಷ್ಯ ಜನ್ಮತಃ ಪಶು ಸ್ವಭಾವದವನೆ– ನೀತಿ ವಿರೋಧಿ; ಶಿಸ್ತು, ಶಿಕ್ಷಣ, ತರಬೇತಿಗಳನ್ನು ಅವನ ಮೇಲೆ ಹೇರಿದರೆ, ಅವನು ಸ್ವಲ್ಪ ಒಳ್ಳೆಯ ರೀತಿಯಿಂದ ವರ್ತಿಸುವ ಪ್ರಾಣಿಯಾಗುತ್ತಾನೆ.’ ಗ್ರೀಕ್ ತತ್ತ್ವವೇತ್ತನಾದ ಪ್ಲೇಟೋ ತನ್ನ ಸಂವಾದದಲ್ಲಿ ಹೇಳುತ್ತಾನೆ–‘ಮನುಷ್ಯ ಹುಟ್ಟು ಬಂಡುಗಾರ. ಒಬ್ಬಾತ ಯಾವುದಾದರೂ ರೀತಿಯಿಂದ ಎಣಿಸಿದಾಗಲೆಲ್ಲ ಮಾಯವಾಗುವ ಸಿದ್ಧಿಯನ್ನು ಪಡೆದುಕೊಂಡಿದ್ದಾನೆನ್ನಿ. ಆಗ ಯಾವುದೇ ಖಜಾನೆ ದರೋಡೆಯಾಗದೇ ಉಳಿದೀತೆ? ಯಾವ ಹೆಂಗಸೂ ಮಾನಭಂಗವಾಗದೆ ಉಳಿದಾಳೆ?’ ನಮ್ಮಲ್ಲಿ ಎಷ್ಟೋ ಮಂದಿ ಸಜ್ಜನರಾಗಿರುವುದು ಸಮಾಜದ ಕಟುಟೀಕೆ ಮತ್ತು ಅಪವಾದಗಳ ಭಯದಿಂದ, ಆರಕ್ಷಕರ ಏಟಿನ ಭಯದಿಂದ. ಇವೆರಡೂ ಇಲ್ಲದಲ್ಲಿ ಅವರ ಸೌಜನ್ಯ ಮಾಯವಾಗಿ ಬಿಡುವುದು. ಆ ಸೌಜನ್ಯ ಮಾಯವಾಗದೆ ಉಳಿಯುವ ಜನರ ಸಂಖ್ಯೆ ಕೈ ಬೆರಳೆಣಿಕೆಯಷ್ಟಿರಬಹುದು. ಮನೋವಿಕಾರಗಳಿಗೆ ಕಾರಣವಾಗುವಂಥ ವಾತಾವರಣವಿದ್ದರೂ ನಿಜವಾದ ಧೀರರ, ಸಂಯಮಿಗಳ ಮನಸ್ಸು ವಿಕಾರ ಹೊಂದುವುದಿಲ್ಲ. ಹಾಗೆ ವಿಕಾರವಾಗದೆ ಉಳಿಯುವಂತೆ ಅವರಿಗೆ ಪ್ರೇರಕವಾದ ಶಕ್ತಿಯೇ ಧರ್ಮ. ದೇವರುಗಳಲ್ಲಿ ನೀತಿನಿಯಮಗಳಲ್ಲಿ ಅವರ ಅಪಾರ ವಿಶ್ವಾಸ ಮತ್ತು ಸದಾಚಾರ ನಿಷ್ಠೆ.


\section*{ಧರ್ಮ ಮತ್ತು ಮನಸ್ಸು}

\addsectiontoTOC{ಧರ್ಮ ಮತ್ತು ಮನಸ್ಸು}

ನಿಜವಾದ ಧರ್ಮ ಎರಡು ಕೆಲಸಗಳನ್ನು ಮಾಡಿಕೊಡುತ್ತದೆ. ಒಂದು ಮನುಷ್ಯನ ಮನಸ್ಸಿನ ಪುನಾರಚನೆ–ಚಂಚಲವಾದ ಮನಸ್ಸನ್ನು ನಿಗ್ರಹಿಸಿ ಅದರ ಗತಿ, ವೇಗಗಳನ್ನು ಉದಾತ್ತ ಗುರಿಯೆಡೆಗೆ ತಿರುಗಿಸುತ್ತದೆ. ಇನ್ನೊಂದು, ಸಮಾಜದ ವಿವಿಧ ವ್ಯಕ್ತಿಗಳೊಳಗೆ ಬಾಂಧವ್ಯ ಮತ್ತು ಸಹಕಾರದ ಬಾಳ್ವೆಗೆ ತಕ್ಕ ವ್ಯವಸ್ಥೆ ಮಾಡುತ್ತದೆ. ಇತರರ ಸುಖ ಮತ್ತು ಆಧ್ಯಾತ್ಮಿಕ ಉನ್ನತಿಗೆ ತಡೆಯಾಗದಂತೆ ತನ್ನ ಸ್ವಾರ್ಥವನ್ನು ನಿಯಂತ್ರಿಸಿಕೊಳ್ಳುವ ನಡತೆ ಅದು.

ಅರ್ಥಕಾಮಗಳೇ ಮುಖ್ಯವಲ್ಲ. ಏಕೆಂದರೆ ಮನುಷ್ಯನು ಬರಿಯ ದೇಹವೇ ಅಲ್ಲ. ಅವನು ಜೀವಾತ್ಮ. ದೇಹ, ಇಂದ್ರಿಯ, ಮನಸ್ಸು–ಇವುಗಳನ್ನು ಬಳಸಿ ಬಾಳುವೆಯನ್ನನುಭವಿಸುತ್ತಿರುವ ಒಂದು ತತ್ತ್ವವಿದೆ. ‘ನಾನು’, ‘ನಾನು’ ಎಂದುಕೊಳ್ಳುತ್ತದೆಯಲ್ಲ–ಅದೇ ಆ ತತ್ತ್ವ. ‘ಕರ್ತೃತ್ವ ಭೋಕ್ತೃತ್ವ ಅಭಿಮಾನಿ’ ಎಂದರೆ ‘ಹೀಗೆ, ಹೀಗೆ ಮಾಡುವೆ’ ಎನ್ನುವ, ‘ಸುಖ ದುಃಖಗಳನ್ನು ಅನುಭವಿಸುವೆ’ ಎನ್ನುವ, ‘ನಾನು’, ‘ನಾನು’ ಎಂದುಕೊಳ್ಳುವ ಜೀವಾತ್ಮ. ತಾನು ಆತ್ಮ, ತನ್ನ ಎದುರಿಗೂ ಇತರ ಆತ್ಮಗಳಿವೆ ಎಂಬುದನ್ನು ತಿಳಿದು ನಾವು ನಡೆಯಬೇಕು. ನಾನು ಎಂಬುದು ದೇಹಕ್ಕಿಂತ ಹೆಚ್ಚಾಗಿ ಆತ್ಮವಾದ ಕಾರಣ ದೇಹದ ಸುಖವನ್ನು ಮಾತ್ರ ಎಣಿಸಿದರಾಗದು. ತನ್ನ ಎಲ್ಲ ಸಂಪತ್ತನ್ನೂ ಆತ್ಮದ ಕ್ಷೇಮಕ್ಕೆ ವಿರೋಧವಾಗದಂತೆ ಉಪಯೋಗಿಸಬೇಕು. ಎಂದರೆ ಇಂದ್ರಿಯ ಸುಖದ ದಾಸನಾಗಬಾರದು. ಅವುಗಳು ಎಳೆದಲ್ಲಿಗೆ ಹೋಗಬಾರದು. ಅವುಗಳ ಯಜಮಾನ ‘ನಾನು’ ಎಂಬುದನ್ನು ಮರೆಯಬಾರದು. ಇಂದ್ರಿಯನಿಗ್ರಹ, ಚಿತ್ತಶುದ್ಧಿ, ತನ್ಮೂಲಕ ಆತ್ಮಜ್ಞಾನ, ತತ್ಫಲವಾಗಿ ಸರ್ವಬಂಧ ಸರ್ವದುಃಖ ನಿವೃತ್ತಿ, ಶಾಶ್ವತ ಆನಂದಪ್ರಾಪ್ತಿ ಇವು ಧರ್ಮ ಗ್ರಂಥಗಳೂ, ಅನುಭವಿಗಳೂ ಸಾರುವ ಮುಖ್ಯ ಬೋಧನೆಗಳು. ಆದುದರಿಂದ ಧರ್ಮನಿಯಮಗಳನ್ನು ಪಾಲಿಸಿದರೆ, ಎಂದರೆ, ಧರ್ಮವನ್ನು ರಕ್ಷಿಸಿದರೆ, ದಿವ್ಯವಾದುದನ್ನು ಪಡೆಯುವ ಪಥದಲ್ಲಿ ಅದು ನಿಮ್ಮನ್ನು ರಕ್ಷಿಸುತ್ತದೆ; ಗುರಿ ಮುಟ್ಟಿಸುತ್ತದೆ. ಈ ಗುರಿ, ದಿವ್ಯವಾದ ಗುರಿಯಲ್ಲಿನ ವಿಶ್ವಾಸ, ಅದನ್ನು ಹೊಂದಲು ಮಾಡುವ ಪ್ರಾಮಾಣಿಕ ಪ್ರಯತ್ನದಿಂದ ನಮ್ಮ ಬದುಕಿನ ದಿಕ್ಕೇ ಬದಲಿಸುವುದು. ಸ್ವಲ್ಪ ತಡವಾದರೂ, ದೀರ್ಘಕಾಲದ ಪರಿಶ್ರಮ ಬೇಕಾದರೂ ಈ ಪಥದಲ್ಲಿ ಮುನ್ನಡೆಯುವ ಪ್ರತಿಯೊಬ್ಬನ ಪಾಲಿಗೂ ದಿವ್ಯ ಆನಂದ, ಶಾಂತಿ, ಸಂತೃಪ್ತಿಗಳು ಸಿಕ್ಕಿಯೇ ಸಿಕ್ಕುವುವು. ಇದು ಸ್ಥಾಯೀ ವಿಶ್ರಾಮ ಭಾವನೆಯ ಪಥ. ಈ ಗುರಿಯ ಪ್ರಜ್ಞೆ ಇಲ್ಲದ ವ್ಯಕ್ತಿಯ ಸುಖದ ಕಲ್ಪನೆ ಅಥವಾ ಸುಖದ ಅನುಭವ ಒಂದು ರೀತಿಯ ಮೈಮರೆತ. ಅಷ್ಟೇ ಅಲ್ಲ, ದುಃಖ ಹಾಗೂ ಗೊಂದಲಕ್ಕೆ ಕಾರಣ.

\vskip 3pt

ಮನೋದೈಹಿಕ ಬೇನೆ ಎಂದಾಗ ಮನಸ್ಸಿನಲ್ಲಿ ಚಿಂಕ್ರೋಭಗಳಿಂದ ಉದಿಸಿ, ದೈಹಿಕವಾಗಿ ಗೋಚರವಾಗುವ ರೋಗವೆಂದು ಅರ್ಥ. ಎಲ್ಲದಕ್ಕೂ ಮನಸ್ಸೇ ಕಾರಣವಾದುದರಿಂದ, ಮನಸ್ಸನ್ನು ಸರಿಯಾಗಿಟ್ಟುಕೊಂಡರೆ ಆ ಬೇನೆಗಳಿಗೇ ಬರ. ಮನಸ್ಸನ್ನು ಸರಿಯಾಗಿಟ್ಟುಕೊಳ್ಳಲು ಔಷಧವೇ ಕಾರಣವಾಗಬಾರದು. ಮನಸ್ಸಿನಲ್ಲಿ ಮಾಡುವ ಯೋಚನೆ, ಭಾವನೆಗಳ ಶುದ್ಧಿಯಿಂದಲೇ\break ಮನಸ್ಸನ್ನು ಸರಿಯಾಗಿಟ್ಟುಕೊಳ್ಳಬೇಕು. ಹಾಗೆ ಸರಿಯಾಗಿಟ್ಟುಕೊಳ್ಳಲು ಅನುಕೂಲವಾಗುವ ನಿಯಮಗಳೇ–ಯೋಗಶಾಸ್ತ್ರದಲ್ಲಿ ಹೇಳಿದ, ಎಲ್ಲ ಧರ್ಮಗಳೂ ಸಾರುವ ಹತ್ತು ಮೌಲ್ಯಗಳು. ಅಹಿಂಸೆ, ಸತ್ಯ, ಅಸ್ತೇಯ, ಬ್ರಹ್ಮಚರ್ಯ, ಅಪರಿಗ್ರಹ, ಶೌಚ, ಸಂತೋಷ, ತಪಸ್ಸು, ಸ್ವಾಧ್ಯಾಯ, ಈಶ್ವರ ಪ್ರಣಿಧಾನ.

\vskip 3pt


\section*{ಮನಸ್ಸಿಗೊಂದು ಟಾನಿಕ್​}

\addsectiontoTOC{ಮನಸ್ಸಿಗೊಂದು ಟಾನಿಕ್​}

ಸತ್ಯಪ್ರಿಯತೆ, ನ್ಯಾಯಪರಾಯಣತೆ, ಪರೋಪಕಾರ ಬುದ್ಧಿ, ಆತ್ಮಸಂಯಮ ಈ ಗುಣಗಳನ್ನು ಎಲ್ಲ ಧರ್ಮಬೋಧಕರೂ ಕೊಂಡಾಡುತ್ತಾರೆ. ಎಲ್ಲ ಧರ್ಮಗ್ರಂಥಗಳೂ ಬೋಧಿಸುತ್ತವೆ. ಧರ್ಮಗಳೇ ಏಕೆ? ಕೌಟುಂಬಿಕ, ಸಾಂಘಿಕ, ಸಾಮಾಜಿಕ ಹಿತಚಿಂತನೆ ಮಾಡುವ ಯಾವನೂ, ಬುದ್ಧಿ ಎನ್ನುವ ವಿವೇಚನಾಶಕ್ತಿ ಇರುವವನೂ, ಈ ಗುಣಗಳನ್ನು ಮೂಢನಂಬಿಕೆ ಎನ್ನಲಾರ. ಲಂಗುಲಗಾಮಿಲ್ಲದ ಅಪ್ರಾಮಾಣಿಕ ಸ್ವೇಚ್ಛಾಚಾರದ ಬದುಕನ್ನು ಒಪ್ಪಲಾರ. ಯಾವ ವಿಚಾರವಾದಿ ನಾಸ್ತಿಕನೂ, ವೈಜ್ಞಾನಿಕ ಮನೋವೃತ್ತಿಯವನೂ, ಮೇಲಿನ ಸದ್ಗುಣಗಳನ್ನು ವಿರೋಧಿಸ ಲಾರ. ವ್ಯಕ್ತಿಯ ಅಭ್ಯುದಯ, ಸಮಾಜದ ಕಲ್ಯಾಣ ಎಂಬ ಎರಡು ಫಲಗಳನ್ನು ನೀಡುವ, ಅನುಷ್ಠಾನಯೋಗ್ಯವಾದ ಮುಖ್ಯ ನಿಯಮಗಳು ಯೋಗಶಾಸ್ತ್ರದಲ್ಲಿ ಯಮ, ನಿಯಮಗಳೆಂದು ಕರೆಯಲ್ಪಡುವ ಈ ಹತ್ತು ಸದ್ಗುಣಗಳು ಅಥವಾ ಮೌಲ್ಯಗಳು. ತಾನು ಚೆನ್ನಾಗಿ ಬದುಕಿ, ಇತರರನ್ನೂ ಚೆನ್ನಾಗಿ ಬದುಕಗೊಡುವ, ಇತರರ ಸ್ವಾತಂತ್ರ್ಯ ಹಾಗೂ ಬೆಳವಣಿಗೆಗೆ ಮಾರಕ ವಾಗದಂತೆ ನಡೆದುಕೊಳ್ಳುವ ಸಂಸ್ಕಾರ ಈ ಗುಣಗಳ ಬೆಳವಣಿಗೆಯಿಂದ ಸಾಧ್ಯವಾಗುವುದು. ಆದ್ದರಿಂದ ಈ ಗುಣಗಳನ್ನು ಸಾಮಾಜಿಕ ಮೌಲ್ಯಗಳು ಎನ್ನಬಹುದು.

\vskip 3pt

ಇತರರು ನನ್ನನ್ನು ಹಿಂಸಿಸಿದಾಗ ನನಗೆ ಅತೀವ ದುಃಖವಾಗುತ್ತದೆ. ಯಾವ ಮಾತು, ಕೃತಿಗಳಿಂದ ನಾನು ನೋವು, ದುಃಖಗಳನ್ನು ಅನುಭವಿಸಬೇಕಾಗುವುದೊ ಅದು ಇತರರಿಗೂ ಸಹ್ಯವಲ್ಲ. ಮನಸ್ಸು, ಮಾತು, ಕೃತಿಗಳಿಂದ ಯಾರಿಗೂ ಕೇಡನ್ನು ಬಯಸದೇ ಎಲ್ಲರೆಡೆಗೂ ಶುಭ ಚಿಂತನೆಯನ್ನು ಹರಿಯಬಿಡುವ ಮನೋವೃತ್ತಿಯೇ ಅಹಿಂಸೆ. ಅಂತೆಯೇ ಸತ್ಯ, ಅಸ್ತೇಯ, ಬ್ರಹ್ಮಚರ್ಯ, ಅಪರಿಗ್ರಹ ಮುಂತಾದ ಇತರ ಗುಣಗಳು. ಬೇರೆ ಬೇರೆ ವ್ಯಕ್ತಿಗಳ ಸಾಂಘಿಕ ಜೀವನದ ಸುವ್ಯವಸ್ಥೆಗೆ ಇವುಗಳ ಪಾಲನೆ ಅತ್ಯಗತ್ಯ.

ಸತ್ಯ ಮತ್ತು ಪ್ರಾಮಾಣಿಕತೆ ನಮ್ಮ ದೇಶದ ಧರ್ಮಗ್ರಂಥಗಳಲ್ಲಿ ಬಹುವಾಗಿ ಕೊಂಡಾಡ ಲ್ಪಟ್ಟ ಮೌಲ್ಯಗಳು. ಮಹಾಭಾರತದ ಋಷಿಗೆ ಸತ್ಯವನ್ನು ಎಷ್ಟು ಹೊಗಳಿದರೂ ತೃಪ್ತಿಯಿಲ್ಲ. ಅಹಿಂಸೆಯೂ ಸತ್ಯದಲ್ಲಿ ಪ್ರತಿಷ್ಠಿತವಾಗಿದೆ ಎಂದು ಮಹಾಭಾರತದಲ್ಲಿ ಧರ್ಮವ್ಯಾಧ ಹೇಳುತ್ತಾನೆ. ಸತ್ಯದ ಅಡಿಪಾಯವಿಲ್ಲದೇ ಯಾವ ನಿಜವಾದ ಅಭಿವೃದ್ಧಿ ಸಾಧ್ಯವಿಲ್ಲ ಎಂಬುದನ್ನು ತಿಳಿಯಲು ಕಷ್ಟವಿಲ್ಲ. ವಿಜ್ಞಾನಿಯ ಅದ್ಭುತ ಶೋಧನೆಗಳು ಸಾಧ್ಯವಾದುದು ಸತ್ಯಾನ್ವೇಷಣೆಯಿಂದಲೇ, ಸತ್ಯವೇನೆಂದು ತಿಳಿಯುವ ಕುತೂಹಲದಿಂದಲೇ, ತಪೋಮಯ ನಿಷ್ಠೆ, ಏಕಾಗ್ರತೆಗಳಿಂದಲೇ. ‘ಸತ್ಯಮೇವ ಜಯತೇ ನಾನೃತಮ್​–ಸತ್ಯವೊಂದೇ ಜಯಿಸುವುದು, ಸುಳ್ಳಲ್ಲ’ ಎಂಬ ಮಹಾವಾಕ್ಯ ಶ‍್ರೀಸಾಮಾನ್ಯರ ಅನುಭವಕ್ಕೂ ನಿಲುಕುವಂಥ ವಿಚಾರ. ಮಹಾತ್ಮಾ ಗಾಂಧೀಜಿ ಸತ್ಯವೇ ದೇವರೆಂದರು. ಭಗವಾನ್ ರಾಮಕೃಷ್ಣರು ಸತ್ಯವಂತನು ದೇವರ ಮಡಿಲಲ್ಲಿದ್ದಾನೆ ಎಂದರು.

ಕಂಡದ್ದನ್ನು ಕಂಡಂತೆ ಹೇಳುವುದು ಸತ್ಯದ ಒಂದು ರೂಪ. (ಇದಕ್ಕೆ ಅಪವಾದವಿರಬಹುದು. ಕೆಲವೊಮ್ಮೆ ಕಂಡದ್ದನ್ನು ಕಂಡಂತೆ ಹೇಳಿಬಿಡುವುದು ಧರ್ಮವಾಗದು.) ಸತ್ಯದ ಇನ್ನೊಂದು ರೂಪ ಹೇಳಿದ ಮಾತನ್ನು ನಡೆಯಿಸುವುದು. ಸತ್ಯದ ಮತ್ತೊಂದು ರೂಪವೇ ಕರ್ತವ್ಯನಿಷ್ಠೆ. ಇಂಥ ಸತ್ಯನಿಷ್ಠರಾದ ಮಹಾತ್ಮರ ಮಹಿಮಾತಿಶಯವನ್ನು ಪುಣ್ಯಕತೆಗಳು ಕೊಂಡಾಡುತ್ತವೆ. ಸತ್ಯವನ್ನು ಹೇಳುವುದರಲ್ಲಿ, ಹೇಳಿದ್ದನ್ನು ನಡೆಯಿಸುವುದರಲ್ಲಿ, ಸೇವಾವೃತ್ತಿಯಲ್ಲಿ, ಅದ್ರೋಹದಿಂದ ನಡೆದುಕೊಳ್ಳುವಲ್ಲಿ ತೊಂದರೆಗೊಳಗಾದರೂ, ಕರ್ತವ್ಯವನ್ನು ನಡೆಯಿಸುವುದರಲ್ಲಿ ಮನುಷ್ಯ\break ಸ್ವಾರ್ಥವನ್ನು ಬದಿಗಿಟ್ಟು ಪರಹಿತವನ್ನು ಸಾಧಿಸುತ್ತಾನೆ. ಮಾತ್ರವಲ್ಲ, ಸತ್ಯದ ಸರಿಯಾದ ಅನುಷ್ಠಾನದಿಂದ ಪರಮಸತ್ಯಸ್ವರೂಪನಾದ ಭಗವಂತನನ್ನು ಪಡೆಯಲು ಸಾಧ್ಯ. ಆದುದರಿಂದ ಸತ್ಯ ಅಷ್ಟೊಂದು ಶ್ರೇಷ್ಠವಾದುದು.

ಜೀವನ ಉತ್ತಮವಾಗಬೇಕಾದರೆ ಸತ್ಯವು ಮೊದಲು ವ್ಯಕ್ತಿಯಲ್ಲಿ, ಆ ಬಳಿಕ ಕೌಟುಂಬಿಕ ವ್ಯವಹಾರದಲ್ಲಿ, ಶಾಲೆ ಕಾಲೇಜುಗಳಲ್ಲಿ, ಸಾಮಾಜಿಕ ಜೀವನದಲ್ಲಿ ವ್ಯಕ್ತವಾಗಿ ಆಚರಣೆಯಲ್ಲಿ ಬರಬೇಕು. ಆಗ ಆ ವ್ಯಕ್ತಿಗೂ ಶಾಂತಿ ಹಾಗೂ ಇಡಿಯ ಸಮಾಜಕ್ಕೂ ಹಿತ. ಸತ್ಯವನ್ನು ಆಚರಣೆಯಲ್ಲಿ ತರಲು ಯತ್ನಿಸುವ ವ್ಯಕ್ತಿಯ ಬೆಳವಣಿಗೆಯ ಹಂತವನ್ನು ಪರಿಶೀಲಿಸಿದರೆ ಚಿಂತಾಕ್ರೋಶ ಭಯೋದ್ವೇಗಗಳು ಅವನಲ್ಲಿ ತಮ್ಮ ಅಸ್ತಿತ್ವವನ್ನು ಕಳೆದುಕೊಳ್ಳುವ ವಿಚಾರ ಸ್ಪಷ್ಟವಾಗಿ ತಿಳಿದು\break ಬರುವುದು.

ಪ್ರಾಮಾಣಿಕ ವ್ಯಕ್ತಿ ತನ್ನ ಮಾತು ಮತ್ತು ಕೃತಿಗಳಲ್ಲಿ ಅಂತರವನ್ನು ಕಡಿಮೆ ಮಾಡಲು ಯತ್ನಿಸುತ್ತಾನೆ. ತನಗೆ ತಾನೇ ಪ್ರಾಮಾಣಿಕನಾಗಿರುತ್ತಾನೆ. ಎಂದರೆ ಮನಸ್ಸಾಕ್ಷಿಗೆ ಸರಿಯಾಗಿ ವರ್ತಿಸುತ್ತಾನೆ. ತನ್ನಲ್ಲಿರುವ ಶಕ್ತಿ, ವೈಶಿಷ್ಟ್ಯ ಹಾಗೂ ನ್ಯೂನತೆ ಪರಿಮಿತಿಗಳನ್ನು ಪ್ರಾಮಾಣಿಕವಾಗಿ ಒಪ್ಪಿಕೊಂಡು, ವೈಶಿಷ್ಟ್ಯಗಳನ್ನು ವೃದ್ಧಿಸಿಕೊಂಡು ನ್ಯೂನತೆಗಳನ್ನು ತಿದ್ದಿಕೊಳ್ಳಲು ಯತ್ನಿಸುತ್ತಾನೆ.

ತಂದೆ, ತಾಯಿ, ತಮ್ಮ, ತಂಗಿ, ಅಣ್ಣ, ಅಕ್ಕ, ಬಂಧುಗಳು ಇವರೊಡನೆ ಪ್ರಾಮಾಣಿಕನಾಗಿ ವರ್ತಿಸುತ್ತಾನೆ. ಖರ್ಚು ಮಾಡಿದ ಹಣಕ್ಕೆ ಸರಿಯಾದ ಲೆಕ್ಕವನ್ನಿಡುತ್ತಾನೆ. ಕೊಟ್ಟ ವಚನ ಪಾಲಿಸಲು ಸರ್ವಪ್ರಯತ್ನ ಮಾಡುತ್ತಾನೆ. ಅನಿವಾರ್ಯವಾಗಿ ಅದು ಸಾಧ್ಯವಾಗದಾಗ ಕ್ಷಮೆ ಕೇಳುತ್ತಾನೆ. ತನ್ನದೇನಾದರೂ ತಪ್ಪುಗಳಾಗಿದ್ದರೆ ಅದನ್ನು ಒಪ್ಪಿಕೊಳ್ಳುತ್ತಾನೆ. ಪಶ್ಚಾತ್ತಾಪ ಪಡುತ್ತಾನೆ. ಸಂಬಂಧಿಸಿದವರ ಹತ್ತಿರ ಕ್ಷಮಾಯಾಚನೆ ಮಾಡುತ್ತಾನೆ. ಮುಂದೆ ತಪ್ಪಾಗದಂತೆ ಜಾಗರೂಕತೆ ವಹಿಸುತ್ತಾನೆ. ಕುಟುಂಬದಲ್ಲಿ ಇತರರಿಗೆ ಸೇರಿದ ವಸ್ತುಗಳನ್ನು ಉಪಯೋಗಿಸಬೇಕೆಂದಾದಲ್ಲಿ ಅವರಿಂದ ಅನುಮತಿ ಪಡೆಯುತ್ತಾನೆ. ಇನ್ನೊಬ್ಬರಿಂದ ಸಾಲ ಪಡೆದರೆ ಅದನ್ನು ಸಕಾಲದಲ್ಲಿ ಹಿಂದಿರುಗಿಸುತ್ತಾನೆ. ಕುಟುಂಬದ ಎಲ್ಲ ಸದಸ್ಯರಿಗಾಗಿ ತಯಾರಿಸಿದ ತಿಂಡಿಯನ್ನು ತಾನೊಬ್ಬನೇ ಯಾರಿಗೂ ಕಾಣದಂತೆ ತಿಂದು ಮುಗಿಸುವ ಅದಮ್ಯ ಇಚ್ಛೆಯನ್ನು ನಿಗ್ರಹಿಸುತ್ತಾನೆ. ಶಾಲೆಯಲ್ಲೂ, ನೆರೆಹೊರೆಯಲ್ಲೂ ಅವನ ವರ್ತನೆಯ ಧಾಟಿ ಇದೇ ಆಗಿರುತ್ತದೆ.

ಇದರ ಪರಿಣಾಮ, ಪ್ರಾಮಾಣಿಕ ವ್ಯಕ್ತಿಯಲ್ಲಿ ಶಾಂತಿ ನೆಲೆಸುತ್ತದೆ. ಅಸಂಗತ ಭೀತಿ ದೂರವಾಗುತ್ತದೆ. ಮಾಡಬೇಕಾದುದನ್ನು ಸರಿಯಾಗಿ ಮಾಡಿದಾಗ ಉಂಟಾಗುವ ಕೃತಕೃತ್ಯತೆಯ ಭಾವನೆ ಬರುತ್ತದೆ. ಸದಾ ತೃಪ್ತಿಯ ಮನೋಭಾವ ಉಳಿದುಕೊಳ್ಳುತ್ತದೆ. ಜವಾಬ್ದಾರಿಯ ವರ್ತನೆ ವೃದ್ಧಿಯಾಗುತ್ತದೆ. ಮಾತು ಕೃತಿಗಳಲ್ಲಿ ಸಾಮಂಜಸ್ಯವಿರುತ್ತದೆ. ಸತ್ಯವೇ ಜಯಿಸುವುದು ಎಂಬುದರಲ್ಲಿ ಅವರಿಗೆ ದೃಢವಿಶ್ವಾಸವಿರುತ್ತದೆ. ದೀರ್ಘಕಾಲ ಈ ರೀತಿ ಸತ್ಯದ ಉಪಾಸನೆ ಮಾಡಿದ ವ್ಯಕ್ತಿಯ ಶೀಲವು ದೃಢತೆಯನ್ನು ಪಡೆಯುತ್ತದೆ. ಮುಚ್ಚುಮರೆಯಿಲ್ಲದ ನೇರ ಮಾತುಕತೆ\break ಆತನಲ್ಲಿ ವ್ಯಕ್ತವಾಗುತ್ತದೆ. ಯುಕ್ತಾಯುಕ್ತ ವಿವೇಚನೆ ಅವನಲ್ಲಿ ಸ್ವಾಭಾವಿಕವಾಗಿ ಅಭಿವ್ಯಕ್ತವಾಗುವುದು. ನಿರ್ಭೀತಿಯ ಮನೋವೃತ್ತಿ ಹಾಗೂ ನಿಶ್ಚಿಂತೆಯ ನಡವಳಿಕೆ ಅವನಲ್ಲಿ ಕಂಡು ಬರುವುದು. ಖಚಿತವಾದ ಜ್ಞಾನ, ಖಚಿತವಾದ ನಡೆನುಡಿ ಅವನ ವ್ಯಕ್ತಿತ್ವವನ್ನು ಬೆಳಗುವುದು. ಇತರರನ್ನು ಮೆಚ್ಚಿಸುವ ಹುಚ್ಚುಹವ್ಯಾಸ ಅವನಲ್ಲಿ ಇಲ್ಲದಿದ್ದರೂ, ಈ ವ್ಯಕ್ತಿ ಇತರರ ಮೇಲೆ ತನ್ನ ಸತ್ಪ್ರಭಾವವನ್ನು ಬೀರಿಯೇ ಬೀರುತ್ತಾನೆ. ಈತನಿಗೆ ನಾವು ಹೇಗಾದರೂ ಸಹಕಾರ ನೀಡೋಣ ಎನಿಸುತ್ತದೆ ಅವರಿಗೆ. ಎಂಥ ಸದ್ಗುಣವಂತನಿವನು! ಇವನ ಸದ್ಗುಣವನ್ನು ಹಾಡಿ ಹೊಗಳಬೇಕು ಎಂದೆನಿಸುತ್ತದೆ. ಅನುಕರಣೆಗೆ ಯೋಗ್ಯನಾದ ವ್ಯಕ್ತಿ ಈತ ಎಂದು ಗೋಚರವಾಗುತ್ತದೆ. ‘ಇವನು ನನ್ನ ಶ್ರದ್ಧೆ ಗೌರವಗಳಿಗೆ ಪಾತ್ರನು. ಇವನನ್ನು ಖಂಡಿತವಾಗಿ ನಂಬಬೇಕು. ಇವನ ಸನಿಹಕ್ಕೆ ಬಂದು ನನ್ನ ಬದುಕಿಗೊಂದು ತಿರುವು ಸಿಕ್ಕಿದೆ – ಇತ್ಯಾದಿ.’ ಹೀಗೆ ಅಹಿಂಸೆ, ಸತ್ಯ, ಅಸ್ತೇಯ, ಬ್ರಹ್ಮಚರ್ಯ, ಅಪರಿಗ್ರಹಗಳೆಂಬ ಗುಣಗಳನ್ನು ಬೆಳೆಸಿಕೊಂಡ ವ್ಯಕ್ತಿ ಸಾಮಾಜಿಕವಾಗಿ ಬರಬಹುದಾದ ಎಲ್ಲ ಒತ್ತಡ, ಒತ್ತಾಸೆ, ಆಂತರಿಕ ಏರುಪೇರು, ಅಶಾಂತಿಗಳಿಂದ ಬಿಡುಗಡೆ ಹೊಂದುತ್ತಾನೆ. ಜನರ ವಿಶೇಷ ಪ್ರೀತಿ, ವಿಶ್ವಾಸ, ಸಹಕಾರಗಳನ್ನು ಸಹಜವಾಗಿ ಪಡೆಯುತ್ತಾನೆ. ಆದ್ದರಿಂದ ಆನಂದದಿಂದಲೇ\break ಇರುತ್ತಾನೆ.

\newpage

ವ್ಯಕ್ತಿಯ ಬದುಕಿನ ಪರಿಪೂರ್ಣತೆಗೆ ನೆರವಾಗುವ ಇತರ ಐದು ಗುಣಗಳು–ಶೌಚ, ಸಂತೋಷ, ತಪಸ್ಸು, ಸ್ವಾಧ್ಯಾಯ, ಈಶ್ವರ ಪ್ರಣಿಧಾನ. ದೈಹಿಕ ಶುಚಿತ್ವಗಳ ಕಡೆಗೆ ಸರಿಯಾಗಿ ಗಮನವೀಯಬೇಕು. ದುಃಖ, ಚಿಂತೆ, ಕೋಪ, ತಾಪ, ಮದ, ಮತ್ಸರಗಳಿಂದ ವಿರಹಿತವಾದ ಪ್ರಸನ್ನವಾದ ಮನಸ್ಸು ತೃಪ್ತಿ ಯಾರಿಗೆ ಬೇಡ? ಯಾರಾದರೂ ದಿನದಿಂದ ದಿನಕ್ಕೆ ತಾನು ಹೆಚ್ಚು ಹೆಚ್ಚು ಚಂಚಲನಾಗಿ ಯಾವ ಕಾರ್ಯವನ್ನೂ ಚೆನ್ನಾಗಿ ಮಾಡದಂತಾಗಲಿ ಎಂದು ಅಶಿಸುವನೆ? ಮನಸ್ಸು, ಇಂದ್ರಿಯಗಳ ಏಕಾಗ್ರತೆ ಎಲ್ಲರಿಗೂ ಬೇಕು. ಮನಸ್ಸನ್ನು ನಿಗ್ರಹಿಸಲು ಸಹಾಯ ಮಾಡುವ ಗುರಿಯತ್ತ ನಮ್ಮ ದೃಷ್ಟಿಯನ್ನು ಹರಿಯುವಂತೆ ಮಾಡುವ ಅಧ್ಯಯನ ಬೇಕು. ಸರ್ವಶಕ್ತನಾದ ಭಗವಂತನಲ್ಲಿ ಶ್ರದ್ಧಾಭಕ್ತಿಗಳಿಂದ ಶರಣಾಗಬೇಕು.

ಈ ಮೌಲ್ಯಗಳನ್ನು ಗಮನವಿತ್ತು ಅಲ್ಪಸ್ವಲ್ಪವಾದರೂ ರೂಢಿಸಿಕೊಳ್ಳಲು ಯತ್ನಿಸಿದರೆ,\break ಚಿಂಕ್ರೋಭದಿಂದ ಉಂಟಾಗುವ ರೋಗಗಳು ಪಲಾಯನಸೂತ್ರ ಪಠಿಸುತ್ತವೆ.

ಈ ಎಲ್ಲ ಸದ್ಗುಣಗಳನ್ನೂ ಪ್ರತ್ಯೇಕವಾಗಿ ಅಭ್ಯಸಿಸದಿದ್ದರೂ ಅವುಗಳಲ್ಲಿ ಪರಿಪೂರ್ಣತೆಯನ್ನು ಪಡೆಯುವ ನೇರ ಉಪಾಯ ಒಂದಿದೆ. ಆಧ್ಯಾತ್ಮಿಕ ಪಥದಲ್ಲಿ ಬಹಳ ಮುನ್ನಡೆದ ಹಿರಿಯ ಸಾಧುಗಳೊಮ್ಮೆ ಹೀಗೆಂದರು:

‘ಜಗತ್ತನ್ನು ಮೆಚ್ಚಿಸುವ ಹುಚ್ಚು ಯತ್ನ ಬೇಡ. ಭಗವಂತನು ಎಲ್ಲವನ್ನೂ ನೋಡುತ್ತಿದ್ದಾನೆ ಎಂದು ಮನಸ್ಸಿಗೆ ದೃಢವಾದರೆ ನಾವು ಜಾಗರೂಕರಾಗುತ್ತೇವೆ. ಸದಾ ನಮ್ಮ ಸಹಾಯಕ್ಕೆ ಅವನಿದ್ದಾನೆ ಎಂಬ ದೃಢವಿಶ್ವಾಸ ಉಂಟಾದರೆ, ನಾವು ಹೃತ್ಪೂರ್ವಕವಾಗಿ ಸಾಧನೆ ಮಾಡುತ್ತೇವೆ. ಭಗವಂತ ಸರ್ವಶಕ್ತ, ದಯಾಮಯ ಎಂದು ನಂಬಿಕೆ ಉಂಟಾದರೆ ನಾವು ಎಲ್ಲ ಚಿಂತೆ, ಭೀತಿಗಳನ್ನು ದೂರಮಾಡಿ ಅವನಲ್ಲಿ ಶರಣಾಗತರಾಗುತ್ತೇವೆ. ಭಗವಂತ ತಾಯಿಯೆಂದು ಭಾಸವಾದರೆ, ಸಕಲ ಆನಂದದ ಮೂಲ ಅವನೇ ಎಂದು ಸ್ಥಿರವಾದರೆ, ಪ್ರಾಪಂಚಿಕವಾದ ಯಾವ ವಸ್ತುಗಳೂ ನಮ್ಮನ್ನು ಸೆಳೆಯಲಾರವು. ಅವನ ಕೃಪೆ, ದಯೆಯಿಂದ ಏನೂ ಸಾಧ್ಯ ಎಂದು ಮನಸ್ಸಿಗೆ ತಿಳಿದುಬಂದರೆ ನಾವು ಅವನನ್ನು ಪಡೆಯಲು ಎಲ್ಲ ರೀತಿಯಿಂದಲೂ ಪ್ರಯತ್ನಿಸುತ್ತೇವೆ.’

ಅವನನ್ನು ಪಡೆಯಲು ಶ್ರದ್ಧೆಯಿಂದ ಪ್ರಯತ್ನ ಮಾಡುತ್ತಿರುವ ವ್ಯಕ್ತಿಯಲ್ಲಿ ಸ್ವಾಭಾವಿಕವಾಗಿ ಮೇಲೆ ಹೇಳಿದ ಅಹಿಂಸೆ, ಸತ್ಯವೇ ಮೊದಲಾದ ಹತ್ತು ಮೌಲ್ಯಗಳು ವಿಶೇಷ ರೀತಿಯಿಂದ ಪ್ರಕಾಶವಾಗುವುವು.

‘ಪ್ರತಿಯೊಬ್ಬರೂ ವರ್ತಮಾನದಲ್ಲಿ ಕಾರ್ಯನಿರತರಾಗಬೇಕು. ಭೂತ, ಭವಿಷ್ಯಗಳ ಚಿಂತನೆಯನ್ನು ಬಿಡಬೇಕು. ಭಗವಂತನಲ್ಲಿ ಶರಣಾಗಬೇಕು. ವ್ಯಕ್ತಿಯೊಬ್ಬ ಎಂಥ ಪರಿಸ್ಥಿತಿಯಲ್ಲಿದ್ದರೂ ಸದ್ಯಕ್ಕೆ ಹೆದರಕೂಡದು. ಸಕಾಲದಲ್ಲಿ ಭಗವಂತ ಎಲ್ಲರ ಮೇಲೂ ಕೃಪೆ ಮಾಡುವನು. ಎಲ್ಲರೂ ಶ್ರದ್ಧೆಯಿಂದ ತಮ್ಮ ಪಾಲಿನ ಕರ್ತವ್ಯವನ್ನು ಅತ್ಯುತ್ತಮ ರೀತಿಯಲ್ಲಿ ನಿರ್ವಹಿಸುತ್ತ ಸಾಧನೆಯನ್ನು ಮುಂದುವರಿಸಬೇಕು. ಕೆಲವರು ಸ್ವಲ್ಪ ಹಿಂದೆ, ಇನ್ನು ಕೆಲವರು ಮುಂದೆ. ಕೆಲವರಲ್ಲಿ ವಿಶೇಷತೆ ಇದೆ, ಇನ್ನು ಕೆಲವರಲ್ಲಿ ಅಂಥದೇನೂ ಇಲ್ಲ. ಈ ಎಲ್ಲ ಭಿನ್ನತೆ ಇದ್ದರೂ ಎಲ್ಲರೂ ದೇವರ ಮಕ್ಕಳು. ಅವರವರ ಕೆಲಸವನ್ನು ಭಿನ್ನಭಾವ ಮರೆತು ಒಗ್ಗಟ್ಟಾಗಿ ಸರಿಯಾಗಿ ಮಾಡಬೇಕೆಂಬುದನ್ನು ಮರೆಯದಿರಿ.’

ಮೇಲಿನ ಸರಳವಾದ ಮಾತುಗಳಲ್ಲಿ ಧರ್ಮದ ಸಾರವೇ ಅಡಗಿದೆ. ಎಲ್ಲ ಸಮಸ್ಯೆಗಳಿಗೂ ಆಧ್ಯಾತ್ಮಿಕ ಉತ್ತರವನ್ನು ಕಂಡುಹಿಡಿಯುವವರೆಗೂ, ಮನುಷ್ಯನ ಮನಸ್ಸಿಗೆ ನಿಜವಾದ ಶಾಂತಿ ದೊರೆಯುವುದು ಅಸಾಧ್ಯ.


\section*{ಶ್ರದ್ಧೆಯಿಂದ ಸಿದ್ಧಿ}

\addsectiontoTOC{ಶ್ರದ್ಧೆಯಿಂದ ಸಿದ್ಧಿ}

ಭಗವಂತನಲ್ಲಿಡುವ ದೃಢಶ್ರದ್ಧೆಯು, ಆತ್ಮಶ್ರದ್ಧೆಗೂ, ಸತ್ಕರ್ಮದಿಂದಲೇ ಶ್ರೇಯಸ್ಸು, ಯಶಸ್ಸು, ವಿಜಯ ಎನ್ನುವ ಶ್ರದ್ಧೆಗೂ ಕಾರಣವಾಗುವುದು. ಶ್ರದ್ಧೆಯ ಮಹಿಮೆಯನ್ನು ಅಧ್ಯಾತ್ಮವಾದಿಯೂ ಕೊಂಡಾಡುತ್ತಾನೆ. ವಿಜ್ಞಾನಿಯೂ ಕೊಂಡಾಡುತ್ತಾನೆ.

ಶ್ರದ್ಧೆಯೇ ಜೀವನ.\footnote{\engfoot{Faith is one of the forces by which men live and the total absence of it means collapse.}\hfill\engfoot{ –William James}} ಶ್ರದ್ಧೆಯಿಂದಲೇ ಜೀವನ ಸಾರ್ಥಕ್ಯ. ಶ್ರದ್ಧೆಯೇ ಜೀವನದ ತಳಹದಿ, ಶ್ರದ್ಧೆಯೇ ಆಧ್ಯಾತ್ಮಿಕ ಸೌಧದ ಅಡಿಪಾಯ, ಶ್ರದ್ಧೆಯೇ ಆತನ ಕೃಪೆ ಪಡೆಯಲು ಇರುವ ದಿವ್ಯ ಸಾಧನ, ಶ್ರದ್ಧೆಯಿಂದ ಎಲ್ಲವೂ ಸಾಧ್ಯ. ಶ್ರದ್ಧೆ ಇದ್ದಲ್ಲಿ ಆತನಿದ್ದಾನೆ. ಆತನ ದರ್ಶನ ಪಡೆಯಲು ಶ್ರದ್ಧೆ ಬೇಕು. ಇಂತಹ ಅಮೂಲ್ಯವಾದ ಶ್ರದ್ಧೆ ಎಲ್ಲರ ಪಾಲಿಗೆ ಒಂದೇ ದಿನದಲ್ಲಿ ದೊರಕುವುದೆ?

ಅದಕ್ಕಾಗಿ ನಿದ್ದೆ ತೊರೆದು, ಎದ್ದುಬಿದ್ದು, ಆತನನ್ನು ಸ್ಮರಿಸುತ್ತ, ವಿದ್ಯೆ, ವಿನಯಗಳನ್ನುಳಿಸಿ ಕೊಂಡು, ಆ ದಾರಿಯಲ್ಲಿ ಸಾಗಿದಾಗ ಶ್ರದ್ಧೆ ಮೈಗೂಡುವುದು. ಈಜಲು ಕಲಿಯಲು ನೀರಿ\-ಗಿಳಿಯಲೇ\-ಬೇಕು. ಹಾಗೆಯೇ ಶ್ರದ್ಧೆ ಮೈಗೂಡಲು ಪ್ರಯತ್ನವೆಂಬ ಸಮುದ್ರಕ್ಕಿಳಿಯಲೇ ಬೇಕು. ಮೊದಲು ಈಜು ಕಲಿಯುವಾತ ಅಂಜುತ್ತಾನೆ ಅಥವಾ ಒಂದೆರಡು ಬಾರಿ ವಿಫಲನಾಗುತ್ತಾನೆ. ಹಾಗೆಂದು ಅವನು ಈಜುವ ಪ್ರಯತ್ನವನ್ನೇ ಬಿಟ್ಟುಬಿಟ್ಟರೆ ಹೇಗೆ? ಹಾಗೆಯೇ ಆಧ್ಯಾತ್ಮಿಕ ರಾಜ್ಯದಲ್ಲಿ ಪ್ರಯತ್ನ ಅತ್ಯಾವಶ್ಯಕ. ಒಂದೆರಡು ಬಾರಿಯೇ ಏಕೆ, ಎಷ್ಟು ಬಾರಿ ವಿಫಲರಾದರೂ ಮತ್ತೂ ಬಿಡದೆ ಪ್ರಯತ್ನ ಮುಂದುವರಿಸಬೇಕು. ಈಜು ಕಲಿಯುವ ಯತ್ನದಲ್ಲಿ ಸ್ವಲ್ಪಮಟ್ಟಿಗೆ ಯಶಸ್ವಿ ಯಾದವ ಕೊಳದಲ್ಲಿ ಈಜಬಲ್ಲ. ಆದರೆ ಅವನು ಸಮುದ್ರದಲ್ಲಿ ಈಜುವ ಸಾಹಸ ಮಾಡಲಾರ. ಈ ದಿಸೆಯಲ್ಲಿ ಅವನಿಗೆ ಇನ್ನೂ ಹೆಚ್ಚು ತರಬೇತಿ ಬೇಕು. ಹಾಗೆಯೇ ಅಲ್ಪ ಶ್ರದ್ಧಾನ್ವಿತರಿಂದ ಅಲ್ಪಸ್ವಲ್ಪ ಕಾರ್ಯ ಸಾಧ್ಯ. ಆದರೆ ಮಹತ್ಕಾರ್ಯ ಆಗದು. ಅದಕ್ಕೆ ಪೂರ್ಣ ಶ್ರದ್ಧೆ ಮೈಗೂಡಿಸಿ ಕೊಂಡವರೇ ಬೇಕು. ಆ ನಿಟ್ಟಿನಲ್ಲೇ ನಾವು ಗುರಿಸೇರುವವರೆಗೂ ವಿಶ್ರಮಿಸದೆ ಮುನ್ನಡೆಯುವ ಛಲ ಬೆಳೆಸಿಕೊಳ್ಳಬೇಕು.

ನಾನಾ ರೀತಿಯ ವಿಶ್ರಾಮಭಾವನೆಯ ತಂತ್ರಗಳನ್ನು ಬೇಕಾದರೆ ಉಪಯೋಗಿಸೋಣ. ಆದರೆ ಆಧ್ಯಾತ್ಮಿಕಪಥದಲ್ಲಿ ಮುನ್ನಡೆದಲ್ಲಿ ಮಾತ್ರ ದಿವ್ಯಶಾಂತಿ ಆನಂದಗಳನ್ನು ಪಡೆದು ಎಲ್ಲ ತರದ ಒತ್ತಡ, ಚಿಂಕ್ರೋಭಗಳಿಂದ ಪಾರಾಗಬಹುದೆಂಬುದನ್ನು ಎಂದೆಂದೂ ಮರೆಯದಿರೋಣ.


\section*{ಶ್ರದ್ಧೆಯೇ ಭಯಕ್ಕೆ ಮಾರಕ}

\addsectiontoTOC{ಶ್ರದ್ಧೆಯೇ ಭಯಕ್ಕೆ ಮಾರಕ}

ಇಪ್ಪತ್ತು ಅಡಿ ಉದ್ದ, ಒಂದು ಅಡಿ ಅಗಲ ಮತ್ತು ಎರಡು ಇಂಚು ದಪ್ಪದ ಒಂದು ಹಲಗೆಯನ್ನು ನೆಲದ ಮೇಲೆ ಇಟ್ಟು ಅದರ ಮೇಲೆ ನಡೆಯಲು ಬಾಲಕನೊಬ್ಬನಿಗೆ ಹೇಳಿ. ಅವನು ಸ್ವಲ್ಪವೂ ಗಾಬರಿಗೊಳ್ಳದೇ ಸರಾಗವಾಗಿ ನಡೆಯುತ್ತಾನಷ್ಟೆ. ಹಾಗೆ ನಡೆಯ ಹೊರಟ ಅವನ ಸಂಕಲ್ಪ ಅಥವಾ ಇಚ್ಛಾಶಕ್ತಿಗೆ ಯಾವ ಭಯದ ಕಲ್ಪನೆಯೂ ತಡೆಯನ್ನೊಡ್ಡುವುದಿಲ್ಲ. ಈಗ ಅದೇ ಹಲಗೆಯನ್ನು ಇಪ್ಪತ್ತು ಅಡಿ ಎತ್ತರದ ಎರಡು ಕಂಬಗಳ ಮೇಲೆ ಸೇತುವೆಯಂತೆ ಇಟ್ಟು ಆ ಬಾಲಕನಿಗೆ ಅದರ ಮೇಲೆ ನಡೆಯುವಂತೆ ಹೇಳಿ. ತಾನು ಹಾಗೆ ನಡೆಯುವಾಗ ಬಿದ್ದುಬಿಡಬಹುದೆಂಬ ಭಯದ ಕಲ್ಪನೆ ಅವನನ್ನು ಆವರಿಸುತ್ತದೆ. ಧೈರ್ಯ ನೀಡಿದರೂ, ಬಹುಮಾನದ ಆಸೆ ತೋರಿಸಿದರೂ, ಅವನು ಆ ಹಲಗೆಯ ಮೇಲೆ ನಡೆಯುವ ಸಾಹಸ ಮಾಡಲಾರ. ಒಂದು ವೇಳೆ ಒತ್ತಾಯಕ್ಕೆ ಮಣಿದು ಪ್ರಯತ್ನಿಸಿ ನೋಡುತ್ತೇನೆಂದು ದೃಢ ಸಂಕಲ್ಪಮಾಡಿ ನಡೆಯ ಹೊರಟರೂ, ಒಂದು ಹೆಜ್ಜೆ ಇಡುತ್ತಲೇ, ಭಯದಿಂದ ತತ್ತರಿಸಿ ಅಲ್ಲೇ ಕುಳಿತುಬಿಡುತ್ತಾನೆ. ಅವನ ಇಚ್ಛಾಶಕ್ತಿಗೂ, ಭಯದ ಕಲ್ಪನೆಗೂ ನಡೆದ ಸಂಘರ್ಷದಲ್ಲಿ ಭಯದ ಕಲ್ಪನೆಯು ಇಚ್ಛಾಶಕ್ತಿಯ ವರ್ಗದಷ್ಟು ಬಲವತ್ತರವಾಗುವುದು. ಇಚ್ಛಾಶಕ್ತಿಯು–‘ನಡೆಯಲು ಯತ್ನಿಸು, ಬಹುಮಾನ ಸಿಗುತ್ತದೆ’ ಎಂದು ಅವನನ್ನು ಪ್ರೇರಿಸುತ್ತದೆ. ಆದರೆ ಭಯದ ಕಲ್ಪನೆ–‘ನೀನು ನಡೆದರೆ ಕೆಳಗೆ ಬಿದ್ದು ಕೈಕಾಲು ಮುರಿದು\-ಕೊಳ್ಳುವುದು ಖಂಡಿತ’ ಎನ್ನುತ್ತದೆ. ಭಯದ ಕಲ್ಪನೆಯು ಇಚ್ಛಾಶಕ್ತಿಯನ್ನು ಸೋಲಿಸಿ ತಾನೇ ವಿರಾಜಿಸುತ್ತದೆ!

ತನ್ನಿಂದ ಅಷ್ಟು ಎತ್ತರದ ಹಲಗೆಯ ಮೇಲೆ ನಡೆಯಲು ಅಸಾಧ್ಯ ಎಂದು ನಂಬಿದ್ದ ಆ ಬಾಲಕನನ್ನೇ ಕ್ರಮಕ್ರಮವಾಗಿ ಹಲಗೆಯ ಎತ್ತರವನ್ನು ಏರಿಸುತ್ತ ನಡೆಯುವಂತೆ ಪ್ರೋತ್ಸಾಹಿಸಿದರೆ ಅವನ ಭಯವು ದೂರವಾಗಿ ತನ್ನಿಚ್ಛೆಯಂತೆ ಸಲೀಸಾಗಿ ಅವನು ನಡೆಯಬಲ್ಲ. ಆಗ ಭಯದ ಕಲ್ಪನೆ ಅವನನ್ನು ಬಾಧಿಸುವುದಿಲ್ಲ. ಇಚ್ಛಾಶಕ್ತಿಯನ್ನು ಭಯದ ಕಲ್ಪನೆ ತಡೆಹಿಡಿಯುವುದಿಲ್ಲ. ಆಗ ಅವು ಒಮ್ಮುಖವಾಗಿ ಹರಿಯುತ್ತವೆ. ನೆಲದ ಮೇಲೆ ಹಲಗೆಯನ್ನು ಹಾಕಿದ್ದಾಗ, ನಿರ್ಭೀತಿಯಿಂದ ಸಲೀಸಾಗಿ ನಡೆದಂತೆಯೆ ಇಪ್ಪತ್ತು ಅಡಿಗಳಷ್ಟು ಎತ್ತರದ ಆ ಹಲಗೆಯ ಮೇಲೆ ಆಗ ಅವನು ನಡೆಯಬಲ್ಲ.

ಈಗ ವಿಶ್ವಾಸ ಅಥವಾ ನಂಬಿಕೆಯ ಬಗೆಗೆ ನಾವೊಂದು ಸಿದ್ಧಾಂತವನ್ನು ಕಂಡುಕೊಳ್ಳ ಬಹುದು. ನಂಬಿಕೆ ಅಥವಾ ವಿಶ್ವಾಸ ಎಂದರೆ ವಿರೋಧ ಅಥವಾ ನಿಷೇಧಾತ್ಮಕವಲ್ಲದ ಕಲ್ಪನೆಗಳ ಸಾರ. ಪ್ರಥಮ ಬಾರಿ ಇಪ್ಪತ್ತು ಅಡಿಗಳಷ್ಟು ಎತ್ತರದ ಹಲಗೆಯ ಮೇಲೆ ನಿಂತಾಗ ಹುಡುಗನ ಮನಸ್ಸಿನಲ್ಲಿ ತಾನು ಬಿದ್ದುಬಿಡಬಹುದು ಎಂಬ ಭಯದ ಕಲ್ಪನೆ ಇತ್ತು. ನಿಯಮಿತವಾಗಿ ಮಾಡಿದ ಕ್ರಮವಾದ ಅಭ್ಯಾಸದಿಂದ, ಈ ನಿಷೇಧಾತ್ಮಕ ಭಯದ ಕಲ್ಪನೆಯು ದೂರವಾಗಿ, ‘ಸಾಧ್ಯ’ ಎನ್ನುವ ರಚನಾತ್ಮಕ ಕಲ್ಪನೆಯು ದೃಢವಾಯಿತು. ವಿಶ್ವಾಸವು ಅನುಭವದಿಂದ ದೃಢಪಡಿಸಿಕೊಂಡ ಕಲ್ಪನೆ, ಹಲವಾರು ಪ್ರಯೋಗಗಳಿಂದ ರಚನಾತ್ಮಕವಾಗಿ ಸರಿಪಡಿಸಿಕೊಂಡ ಕಲ್ಪನೆ ಎಂದಾ ಯಿತು. ವಿಶ್ವಾಸವು ಕಲ್ಪನೆಗಿಂತ ಭಿನ್ನವಾಗಿದೆ. ವಿಶ್ವಾಸವಿಲ್ಲದೆ ಹಲವಾರು ಕಲ್ಪನೆಗಳನ್ನು ನೀವು ಮಾಡಬಹುದು. ಆದರೆ ಕಲ್ಪನೆಗಳ ಬೆಂಬಲವಿಲ್ಲದೆ ವಿಶ್ವಾಸವಿರದು. ತಿದ್ದಿ ಬೆಳೆಸಿಕೊಂಡ, ಪ್ರಯೋಗ ಪರೀಕ್ಷೆಗಳಿಂದ ದೃಢಪಡಿಸಿಕೊಂಡ, ಒರೆ ಹಚ್ಚಿ ಅನುಭವದ ಮೂಲಕ ದೃಢಪಡಿಸಿ ಕೊಂಡ ಕಲ್ಪನೆಯೇ ವಿಶ್ವಾಸ. ಹುಡುಗನಿಗೆ ಅಷ್ಟು ಎತ್ತರವಿರುವ ಹಲಗೆಯ ಮೇಲೆ ನಡೆವ ವಿಶ್ವಾಸವಿದೆ ಎಂದರೆ ಏನರ್ಥ? ಭಯದ ಕಲ್ಪನೆಗಳು ಮೂಡದೆ ‘ಸಲೀಸಾಗಿ ನಡೆಯಬಲ್ಲೆ’ ಎನ್ನುವ ಕಲ್ಪನೆ ದೃಢವಾಗಿದೆ ಎಂದರ್ಥ ಅಲ್ಲವೆ?

ನಮ್ಮೆಲ್ಲರ ಒಳಮನಸ್ಸು ಭಯ, ತಿರಸ್ಕಾರ, ನಿಂದೆ ಮತ್ತು ನೋವು ಇವನ್ನು ಎಂದೆಂದೂ ವಿರೋಧಿಸುತ್ತದೆ. ಯುವಕರು ಕೆಲಸಗಳ್ಳರಾಗಲು ಕಾರಣ ಅವರು ಚಿಕ್ಕವರಾಗಿದ್ದಾಗ ಯಾವುದೋ ಪ್ರಯತ್ನದಲ್ಲಿ ಮತ್ತೆ ಮತ್ತೆ ಸೋತಾಗ ಅನುಭವಿಸಿದ ನೋವು, ತಿರಸ್ಕಾರ, ಭಯ, ‘ಇದು ನನ್ನಿಂದ ಅಸಾಧ್ಯ’–ಎನ್ನುವ ಭಾವನೆಗಳೇ. ಮಕ್ಕಳು ಮಾಡುವ ಕೆಲಸಗಳಲ್ಲಿ ಜಯ ದೊರಕುವಂತೆ ಮಾಡಿದರೆ, ಸೋತರೂ, ಸೋಲಿನಿಂದ ಕಂಗೆಡದಂತೆ ಮಾಡಿ ಜಯವನ್ನು ತಂದು ಕೊಟ್ಟರೆ, ಅವರ ವಿಶ್ವಾಸಕ್ಕೆ ಆಘಾತವಾಗುವುದಿಲ್ಲ.

ಶಿಕ್ಷಣದಲ್ಲಿ ವಿಶ್ವಾಸದ ಪಾತ್ರ ಎಷ್ಟು ಮಹತ್ವದ್ದು–ಅದನ್ನು ಮಕ್ಕಳಲ್ಲಿ ಉಂಟುಮಾಡುವುದಕ್ಕೆ ಅಧ್ಯಾಪಕರು ಎಂಥ ತಾಳ್ಮೆಯನ್ನು ಸಂಪಾದಿಸಿರಬೇಕು ಎಂಬುದು ಎಲ್ಲರೂ ಯೋಚಿಸಬೇಕಾದ ವಿಚಾರವಲ್ಲವೆ?

ಭಯ ಬಾಗಿಲು ಬಡಿಯಿತು. ಶ್ರದ್ಧೆ ಬಾಗಿಲು ತೆರೆದು ‘ಯಾರಲ್ಲಿ?’ ಎಂದು ಕೇಳಿತು. ಆದರೆ ಅಲ್ಲಿ ಯಾರೂ ಇರಲಿಲ್ಲ. ಶ್ರದ್ಧೆಯ ಸದ್ದನ್ನು ಕೇಳಿಯೇ ಭಯ ಪಲಾಯನ ಮಾಡಿತ್ತು!

ಆತ್ಮಶ್ರದ್ಧೆ, ಭಗವಂತನಲ್ಲಿ ಶ್ರದ್ಧೆ–ಇವುಗಳೇ ಯಶಸ್ಸು, ವಿಜಯಗಳ ಕೀಲಿಕೈ.


\section*{ಭಯದ ಕಾರ್ಯಾಗಾರ}

\addsectiontoTOC{ಭಯದ ಕಾರ್ಯಾಗಾರ}

ತೇಲುತ್ತಿರುವ ಹಿಮಗಡ್ಡೆಯಲ್ಲಿ ನಮಗೆ ಕಾಣಿಸುವ ಅಂಶ ಒಂದು ಭಾಗ ಮಾತ್ರ. ಉಳಿದ ಒಂಬತ್ತು ಅಂಶ ನೀರಿನಲ್ಲಿ ಮುಳುಗಿದ್ದು, ನಮ್ಮ ಸಾಮಾನ್ಯ ದೃಷ್ಟಿಗೆ ಅಗೋಚರ. ಅಂತೆಯೇ ನಮ್ಮ ಹೊರ ಮನಸ್ಸು ಒಂದು ಪಾಲೆಂದಿಟ್ಟುಕೊಂಡರೆ, ಉಳಿದ ಒಂಬತ್ತು ಪಾಲು ಸುಪ್ತ ಮನಸ್ಸು. ಇದರ ಚಟುವಟಿಕೆ, ಅಸ್ತಿತ್ವ ನಮಗೆ ಕಾಣಿಸದು. ನಮ್ಮ ವರ್ತನೆ, ಸ್ವಭಾವ, ಭಾವನೆಗಳನ್ನು ಸುಪ್ತಮನಸ್ಸು ಆಳುತ್ತಿರುತ್ತದೆ ಅಥವಾ ನಿಯಂತ್ರಿಸುತ್ತಿರುತ್ತದೆಯೆಂದು ಇಂದು ಮನೋವಿಜ್ಞಾನಿಗಳು ಹೆಚ್ಚುಹೆಚ್ಚಾಗಿ ಹೇಳುತ್ತಾರೆ. ಭಯ, ಯಾತನೆಗಳನ್ನು ಅನುಭವಿಸುವ ನಮಗೆ ಕೆಲವೊಮ್ಮೆ ಅವುಗಳ ಕಾರಣ ತಿಳಿಯುವುದಿಲ್ಲ. ಮನಸ್ಸಿನ ಆಳದ ಸ್ತರಗಳಲ್ಲಿ ಅವುಗಳ ಮೂಲ ಹುದುಗಿರುತ್ತದೆ. ಅಲ್ಲಿಂದಲೇ ಭಯ ನಮ್ಮನ್ನು ತನ್ನ ತಾಳಕ್ಕೆ ತಕ್ಕಂತೆ ಕುಣಿಸುತ್ತದೆ. ಅದೇ ಅದರ ಕಾರ್ಯಾಗಾರ.


\section*{ಭಯ ಬೇತಾಳ}

\addsectiontoTOC{ಭಯ ಬೇತಾಳ}

ಮನುಷ್ಯರಲ್ಲಿ ಹೆಚ್ಚಿನ ಜನ, ಒಂದಲ್ಲ ಒಂದು ತೆರನಾದ ಭಯದಿಂದ ಪೀಡಿತರೆಂದರೆ ತಪ್ಪಿಲ್ಲ. ಕೆಲವೊಮ್ಮೆ ಈ ಭಯ ನಮ್ಮ ಹೊರಮನಸ್ಸಿನಲ್ಲಿ ಅಷ್ಟೊಂದು ಕಾಣಿಸಿಕೊಳ್ಳದಿದ್ದರೂ, ಅಜ್ಞಾತವಾಗಿ ಒಳಗೊಳಗೇ ತನ್ನ ಕೊರೆತವನ್ನು ನಡೆಯಿಸುವುದುಂಟು. ಸಾಮಾನ್ಯ ಭಯ ಒಂದು ಪ್ರಮಾಣದಲ್ಲಿ ನಮ್ಮ ಆತ್ಮರಕ್ಷಣೆ ಹಾಗೂ ಮುಂದಾಲೋಚನೆಗೆ ಬೇಕಾದುದೇ. ಆದರೆ ಅಸಂಗತ ಭಯ ನಮ್ಮನ್ನು ಎಲ್ಲ ವಿಧಗಳಿಂದಲೂ ದುರ್ಬಲಗೊಳಿಸುವ ಒಂದು ರೋಗ. ಅದು ನಮ್ಮಲ್ಲಿ ಶಕ್ತಿ, ಸಾಮರ್ಥ್ಯ ಅಥವಾ ಬಲದ ಅಭಾವವನ್ನು ತೋರಿಸುತ್ತದೆ. ದೈಹಿಕ, ಮಾನಸಿಕ ಮತ್ತು ಆಧ್ಯಾತ್ಮಿಕ ಅನಾರೋಗ್ಯ ಅಥವಾ ಶಕ್ತಿಯ ಅಭಾವವೇ ಎಲ್ಲ ಭಯಗಳಿಗೂ ಮೂಲ. ಅಜ್ಞಾನವು ಇವೆಲ್ಲಕ್ಕೂ ಆದಿಮೂಲ.

ನಿದ್ರಿಸುತ್ತಿರುವಾಗ ಮನುಷ್ಯರ ರಕ್ತ ಹೀರುವ ಒಂದು ಜಾತಿಯ ಪಿಶಾಚಿಯನ್ನು ಆಂಗ್ಲ ಭಾಷೆಯಲ್ಲಿ ವಾಂಪಾಯರ್ ಎನ್ನುತ್ತಾರೆ. ಅಂಥ ಒಂದು ಪಿಶಾಚಿ ಇದೆ ಎಂದು ಅಲ್ಲಿನ ಜನರು ನಂಬುತ್ತಾರೆ. ಆ ಭೂತ ಅಥವಾ ಪಿಶಾಚಿ ನಿದ್ರಿಸುವಾಗ ನಮ್ಮ ರಕ್ತವನ್ನು ಹೀರಿದರೆ, ಭಯ ಎನ್ನುವ ಕಲ್ಪನೆಯ ಭೂತ ಎಚ್ಚರದಲ್ಲೂ, ಸ್ವಪ್ನದಲ್ಲೂ, ನಿದ್ರೆಯಲ್ಲೂ ರಕ್ತವನ್ನು ಹೀರಿ ನಮ್ಮನ್ನು ದುರ್ಬಲಗೊಳಿಸುತ್ತಿದೆ.

ಕೊಳೆ, ಕಶ್ಮಲಗಳು ತುಂಬಿದ ಜಾಗದಲ್ಲೇ ರೋಗಾಣುಗಳ ಉತ್ಪತ್ತಿ ಆದಂತೆ ಅಜ್ಞಾನ ತುಂಬಿದ ಮನಸ್ಸಿನಲ್ಲೇ ಅಸಂಗತ ಭಯದ ಉತ್ಪತ್ತಿ, ನಿರ್ಭೀತಿಯ ಜನನವೂ ಮನಸ್ಸಿನಲ್ಲೇ. ಕೆಲವರು ಬಾಲ್ಯದಿಂದಲೇ–ತಂದೆ ತಾಯಂದಿರಿಂದ ಅಥವಾ ಪರಿಸರದಿಂದ ಭಯದ ಭಾವನೆಗಳನ್ನು ರೂಢಿಸಿಕೊಂಡು ಬಂದಿರುತ್ತಾರೆ. ಇನ್ನು ಕೆಲವರು ನಿರ್ಭೀತಿಯ ಭಾವನೆಗಳನ್ನು ಚಿಕ್ಕಂದಿನಿಂದಲೇ ಅರಗಿಸಿಕೊಂಡಿರುತ್ತಾರೆ.

\vskip 2pt

ಭಯದ ಭಾವನೆಗಳನ್ನು ಕಿತ್ತೆಸೆದು, ನಿರ್ಭೀತಿಯ ಭಾವನೆಗಳನ್ನೂ, ಧೈರ್ಯ, ಸ್ಥೈರ್ಯ, ಶಕ್ತಿ, ಸಹನೆಯ ಭಾವನೆಗಳನ್ನೂ ನಮ್ಮ ಮನಸ್ಸಿನಲ್ಲಿ ಬಿತ್ತಿ ಬೆಳೆಸಲು ಸಾಧ್ಯ ಎನ್ನುವುದು ಬೆಳಕಿನಷ್ಟು ಸತ್ಯ. ಮನುಷ್ಯನ ಮನಸ್ಸೇ ಆತನ ಅಭಿವೃದ್ಧಿ ಅಭ್ಯುದಯಕ್ಕೂ, ದುಃಖ ದುರಂತಕ್ಕೂ ಕಾರಣ ಎನ್ನುವ ಅನುಭವವಾಣಿ ಹಳೆಯ ಕಾಲದ್ದಾದರೂ, ಅದು ಹೇಳುವ ಸತ್ಯ ನಿತ್ಯನೂತನವಾದುದು. ಎರಡಕ್ಕೆ ಎರಡು ಸೇರಿದರೆ ನಾಲ್ಕು ಎನ್ನುವ ಲೆಕ್ಕಾಚಾರ ಎಲ್ಲ ಕಾಲಕ್ಕೂ ಸತ್ಯವಷ್ಟೆ.


\section*{ಭಯ ತರುವ ಬವಣೆ}

\addsectiontoTOC{ಭಯ ತರುವ ಬವಣೆ}

ಭಯ ನಮ್ಮನ್ನು ಆವರಿಸಿದಾಗ ಎದೆ ಬಡಿತ ಹೆಚ್ಚುತ್ತದೆ. ಉಸಿರಾಡಲು ಕಷ್ಟವೆನಿಸುತ್ತದೆ. ನಡುಕ ಉಂಟಾಗುತ್ತದೆ. ‘ಏನೋ ಒಂದು ತರಹ ಆಗುತ್ತದೆ’ ಎನ್ನುತ್ತೇವೆ. ಮೈ ಬೆವರುತ್ತದೆ. ಆಗಾಗ ಶೌಚಗೃಹಕ್ಕೆ ಹೋಗೋಣವೆನಿಸುತ್ತದೆ. ಭಯ ನಮ್ಮ ವಿಚಾರಶಕ್ತಿಯನ್ನೂ ಮೊಟಕುಗೊಳಿಸುತ್ತದೆ. ಕಿಂಕರ್ತವ್ಯಮೂಢರನ್ನಾಗಿ ಮಾಡುತ್ತದೆ. ದೈಹಿಕವಾಗಿಯೂ, ಮಾನಸಿಕವಾಗಿಯೂ ನಮ್ಮನ್ನು ದುರ್ಬಲಗೊಳಿಸುತ್ತದೆ. ನಮ್ಮ ಆತ್ಮವಿಶ್ವಾಸವನ್ನು ಅಲುಗಾಡಿಸುತ್ತದೆ. ಯಾವುದೇ ಕಾರ್ಯವನ್ನು ಕೈಗೊಂಡು ಅದನ್ನು ಯಶಸ್ವಿಯಾಗಿ ಮುಗಿಸಲು ತಡೆ ಒಡ್ಡುತ್ತದೆ. ಇಚ್ಛಾಶಕ್ತಿ, ಸ್ಮೃತಿಶಕ್ತಿಗಳು ದುರ್ಬಲವಾಗಲೂ ಭಯವೇ ಕಾರಣ. ಅದೇ ಎಲ್ಲೆಲ್ಲೂ ಸೋಲಿನ ಎಡೆಗೆ ನಮ್ಮನ್ನು ತಳ್ಳುತ್ತದೆ. ಸದ್ಭಾವನೆ, ಪ್ರೀತಿ, ವಿಶ್ವಾಸ, ಸ್ನೇಹ, ಕರುಣೆಗಳನ್ನು ಇಲ್ಲವಾಗಿಸುತ್ತದೆ. ನಿದ್ರಾಹೀನತೆ, ಬುದ್ಧಿಭ್ರಮೆ, ರಕ್ತದ ಏರೊತ್ತಡವೇ ಮೊದಲಾದ ರೋಗಗಳಿಗೆ ಆಹ್ವಾನವೀಯುತ್ತದೆ.

\vskip 2pt

ರೂಢಮೂಲವಾದ ಕೆಲವೊಂದು ಭಯದಿಂದ ಮನುಷ್ಯರಲ್ಲಿ ವಿಪರೀತ ವರ್ತನೆಗಳುಂಟಾಗುತ್ತವೆ. ಭಯಗ್ರಸ್ತರಾದ ಜನ ಮಾಡದ ಪಾತಕಗಳಿಲ್ಲ. ವರ್ಷ ಒಂದರಲ್ಲಿ ಹಾವಿನಿಂದ ಕಚ್ಚಿಸಿ ಕೊಂಡು ಒಬ್ಬ ವ್ಯಕ್ತಿ ಸತ್ತರೆ, ಮನುಷ್ಯನ ಭಯಕ್ಕೆ ಒಂದು ಲಕ್ಷ ಹಾವುಗಳು ಬಲಿಯಾಗುವುವೆಂ ಬುದು ಒಬ್ಬ ತಜ್ಞನ ಅಂಬೋಣ. ನಾಗರಿಕರೆನಿಸಿಕೊಂಡ ಜನರೂ ಭಯದಿಂದ, ಅತ್ಯಂತ ಹೇಯವೂ ಹೀನವೂ ಆದ ಕೃತ್ಯಗಳನ್ನೆಸಗುತ್ತಾರೆ. ಕೊಲೆ, ಸುಲಿಗೆ ಮಾಡುತ್ತಾರೆ; ಯುದ್ಧದ ಕಿಡಿ ಯನ್ನೂ ಹೊತ್ತಿಸುತ್ತಾರೆ. ಇವೆಲ್ಲವುಗಳ ಹಿನ್ನೆಲೆಯಲ್ಲಿ, ಮನುಷ್ಯರ ಮನಸ್ಸಿನ ಆಳದಲ್ಲಿ ಹುದು ಗಿದ್ದು ಕೆಲಸ ಮಾಡುವ ಸೂಕ್ಷ್ಮಾತಿಸೂಕ್ಷ್ಮ ಕಾರಣವೇ ಭಯ.

\vskip 2pt

ಇಂದಿನ ಜಗತ್ತಿನಲ್ಲಿ ಎತ್ತ ನೋಡಿದರೂ ಭಯವೆಗ್ಗಳಿಸುತ್ತಿದೆ. ಅಮೇರಿಕಾ ಎಂದರೆ ಪ್ರಗತಿಶೀಲ ರಾಷ್ಟ್ರಗಳಿಗೆ ಭೀತಿ. ಅರಬರಿಗೆ ಯಹೂದಿಗಳ ಭೀತಿ. ಯಹೂದಿಗಳಿಗೆ ಅರಬರ ಭೀತಿ. ಬಿಳಿ ಜನರಿಗೆ ಕರಿ ಜನರ ಭೀತಿ, ಕರಿ ಜನರಿಗೆ ಬಿಳಿ ಜನರ ಭೀತಿ. ಎಲ್ಲ ಸ್ತರಗಳಲ್ಲೂ – ರಾಷ್ಟ್ರೀಯ, ಅಂತಾರಾಷ್ಟ್ರೀಯ, ಸಾಮಾಜಿಕ, ಜಾತೀಯ ಮತ್ತು ಧಾರ್ಮಿಕ, ಸಂಕುಚಿತ ಸ್ವಾರ್ಥದ ಕಾರಣದಿಂದ ಭೀತಿಯನ್ನು ಬೆಳೆಸಲಾಗುತ್ತಿದೆ.

ಭಯದಿಂದ ಸಂಶಯ, ಸಂಶಯದಿಂದ ಕೋಪ, ಕೋಪದಿಂದ ಹಿಂಸೆ, ಹಿಂಸೆಯಿಂದ ಸರ್ವ ನಾಶಕ ದುರಂತಗಳೇ ಉಂಟಾಗುವುವು. ಶತಮಾನಗಳಿಂದ ರೂಢಿಸಿಕೊಂಡು ಬಂದ ಉದಾತ್ತ ಗುಣಗಳನ್ನು ಈ ಭಯ ನಾಶಮಾಡಿ ಬಿಡುವುದು. ವಿಚಾರಶಕ್ತಿಯನ್ನೂ, ಕಾರ್ಯನಿಷ್ಠೆಯನ್ನೂ ಹಾಳುಗೆಡಹುವುದು. ಹೀಗೆ ಮನುಷ್ಯರನ್ನು ಪಶುವಿಗಿಂತಲೂ ಕೆಳಕ್ಕೆಳೆಯುವುದು ಈ ಭಯ.

ಭಯದ ಮೂಲವಿರುವುದು ಆಕಾಶದ ಯಾವುದೇ ಮೂಲೆಯಲ್ಲಲ್ಲ. ಆಕಾಶದಿಂದ ಬಾಂಬುಗಳು ಉದುರಿದರೂ, ಕಾರ್ಖಾನೆಗಳಲ್ಲಿ ಅವು ತಯಾರಾಗಿದ್ದರೂ, ಮನುಷ್ಯನ ಮನಸ್ಸಿನಲ್ಲಿರುವ ಭಯವೇ ಅದರ ಮೂಲ ಕಾರಣ. ಆದುದರಿಂದ ಅಲ್ಲಿಂದಲೇ ಅದನ್ನು ಉಚ್ಚಾಟನೆ ಮಾಡಬೇಕು. ಒಂದು ರಾಷ್ಟ್ರದ ಭೀತಿ ಅಲ್ಲಿರುವ ವ್ಯಕ್ತಿಗಳ ಒಟ್ಟು ಭೀತಿಯ ಮೊತ್ತ! ಒಬ್ಬ ನಿರ್ಭೀತಿಯ ಮಹಾವೀರ ತನ್ನ ಬದುಕಿನಲ್ಲಿ ಧೈರ್ಯ, ಪರಾಕ್ರಮ, ಉನ್ನತ ಆದರ್ಶಗಳ ಮೂಲಕ ಜನಾಂಗದ ಭೀತಿಯನ್ನೇ ದೂರ ಮಾಡಬಲ್ಲ. ಸ್ವಾಮಿ ವಿವೇಕಾನಂದರ ಜೀವನದಲ್ಲೂ, ಗಾಂಧೀಜಿಯವರ ಜೀವನದಲ್ಲೂ ನಾವು ಕಾಣುವುದು ಇದನ್ನೇ ಅಲ್ಲವೇ? ಬ್ರಿಟಿಷರ ಗುಂಡುಗಳಿಗೆ ಭಾರತೀಯರು ಬೆದರುವುದು ನಿಂತಾಗ ಅವರು ಭಾರತವನ್ನೇ ಬಿಟ್ಟು ತೊಲಗಬೇಕಾಯಿತು.


\section*{ಭಯದ ವೈವಿಧ್ಯ}

\vskip -7pt\addsectiontoTOC{ಭಯದ ವೈವಿಧ್ಯ}

ರಾಜನು ಕಾಲಕ್ಕೆ ಕಾರಣನಾಗುತ್ತಾನೆಂಬ ಪಿತಾಮಹ ಭೀಷ್ಮರ ವಚನದಂತೆ ಮುಖಂಡರು, ಅಧಿಕಾರ\-ಸ್ಥಾನದಲ್ಲಿರುವವರು, ಜನರ ಬದುಕಿನಲ್ಲಿ ಬದಲಾವಣೆಗಳನ್ನು ತರಬಲ್ಲರು. ಜನ ಮಾನಸದಲ್ಲಿ ಸಮರಸ, ಸಹಕಾರದ ಭಾವನೆಗಳನ್ನು, ನಿರ್ಭೀತಿಯ ಭಾವನೆಗಳನ್ನು ಬಿತ್ತಬಲ್ಲರು. ಈ ದೇಶಕ್ಕೆ ದಂಡೆತ್ತಿ ಬಂದ ದಾಳಿಕಾರರೂ, ವಿದ್ವಾಂಸರಾದ ಪ್ರವಾಸಿಗಳೂ, ಕೆಲವು ರಾಜರ ಆಳ್ವಿಕೆಯಲ್ಲಿ ಇಲ್ಲಿನ ಜನರ ಪರಮಶ್ರೇಷ್ಠ ಚಾರಿತ್ರ್ಯವನ್ನು ಕಂಡು ಕೊಂಡಾಡಿದ್ದಾರೆ. ಆದರೆ ಪಟ್ಟಭದ್ರ ಹಿತಾಸಕ್ತಿಯ, ದೂರದೃಷ್ಟಿ ಇಲ್ಲದ, ದೀನದಲಿತರ ಬಗೆಗೆ ನಿಜವಾದ ಕಳಕಳಿ ಇಲ್ಲದ, ನಾನಾ ತೆರನಾದ ಒತ್ತಡಗಳಿಗೊಳಗಾದ ಮುಖಂಡರಾದರೋ ಏನು ಮಾಡಬಲ್ಲರು? ಜಾತಿ ಮತ ಗುಂಪುಗಳಲ್ಲಿ ಪರಸ್ಪರ ಭೀತಿಯನ್ನುಂಟುಮಾಡಿಯೇ ತಮ್ಮ ಸ್ಥಾನಚ್ಯುತಿಯ ಭೀತಿಯಿಂದ ಪಾರಾ\-ಗಲು ಅವರು ಯತ್ನಿಸುತ್ತಾರೆ. ತಾವೇ ಭೀತರಾದವರು ನಿರ್ಭೀತಿಯ ಆದರ್ಶವನ್ನು ಇತರರಿಗೆ ತೋರಬಲ್ಲರೇ? ಜಗತ್ತನ್ನು ಸರಿಪಡಿಸಲು ನಮ್ಮ ನಿಮ್ಮಿಂದ ಸಾಧ್ಯವಿಲ್ಲ. ಪ್ರಯತ್ನಿಸಿದರೆ ಸ್ವಲ್ಪಮಟ್ಟಿಗೆ ನಮ್ಮನ್ನು ನಾವು ತಿದ್ದಿಕೊಳ್ಳಬಹುದು. ‘ನಿನ್ನನ್ನು ನೀನು ಸರಿಪಡಿಸಿಕೊಂಡರೆ ಜಗತ್ತಿನಲ್ಲಿ ಒಬ್ಬ ಮೂರ್ಖ ಕಡಿಮೆಯಾದ’ ಎಂದನಂತೆ ಕಾರ್ಲೈಲ್. ಅಂತೆಯೇ ವೈಯಕ್ತಿಕ ನೆಲೆಯಲ್ಲಿ ಭೀತಿಯನ್ನು ಹೊಡೆದೋಡಿಸಲು ಪ್ರಯತ್ನಿಸಿದ್ದಾದರೆ ಸಮಾಜ ತನ್ನಿಂದ ತಾನೇ ಭಯ ಮುಕ್ತವಾಗುತ್ತದೆ.

ವೈಯಕ್ತಿಕ ನೆಲೆಯಲ್ಲಿ ಎಲ್ಲರನ್ನೂ ಕಾಡುತ್ತಿರುವ ಭೀತಿಯ ಬಹುಮುಖಗಳು ಇಲ್ಲಿವೆ–

{\noindent\leftskip=0.6cmನಾನಾ ತೆರನಾದ ರೋಗಗಳ ಭಯ\\
 ವಿಷಜಂತುಗಳ ಭಯ\\
 ಭೂತ, ಪ್ರೇತ, ಪಿಶಾಚಿಗಳ ಭಯ\\
 ಮಂತ್ರ, ಮಾಟಗಳ ಭಯ\\
 ಮಡಿಮೈಲಿಗೆಯ ಭಯ\\
 ಬಡತನ, ದೈನ್ಯಾವಸ್ಥೆಗಳ ಭಯ\\
 ಪ್ರೀತಿಪಾತ್ರ ವ್ಯಕ್ತಿಗಳ ಅಗಲಿಕೆಯ ಭಯ\\
 ವ್ಯಾಪಾರ, ಉದ್ಯಮಗಳಲ್ಲಿ ಸೋಲಿನ ಭಯ\\
 ವಿದ್ಯಾರ್ಥಿಗಳಿಗೆ ಪರೀಕ್ಷೆಯ ಭಯ\\
 ರಹಸ್ಯವಾಗಿ ಮಾಡಿದ ತಪ್ಪು ಇತರರಿಗೆ ತಿಳಿದೀತೆಂಬ ಭಯ\\
 ನಂಬಿದವರು ಮೋಸ ಮಾಡಿದರೆ ಎಂಬ ಭಯ\\
 ವಿರೋಧಿಗಳ ಸೇಡಿನ ಭಯ\\
 ಸಾಲಗಾರನಾಗಿಬಿಡುವ ಭಯ\\
 ಸಾಲದ ಭಾರದಿಂದ ಪಾರಾಗಲು ಸಾಧ್ಯವಿಲ್ಲವೆಂಬ ಭಯ\\
 ಸಂಪತ್ತನ್ನು ಸರಕಾರವು ಕಿತ್ತುಕೊಳ್ಳಬಹುದೆಂಬ ಭಯ\\
 ದರೋಡೆಕೋರರ ಮತ್ತು ಕಳ್ಳಕಾಕರ ಭಯ\\
 ಯುದ್ಧ ಹಾಗೂ ಭೀಕರ ಅಸ್ತ್ರಗಳ ಭಯ\\
 ಪರಿಸರ ಮಾಲಿನ್ಯದ ಭಯ\\
 ಸಾವಿನ ಭಯ, ನರಕದ ಭಯ\\
 ಇತ್ಯಾದಿ ನಾನಾ ತೆರನಾದ ಭಯಗಳು\\
 ಭಯದ ಭಯಂಕರ ಪಾಶವು ಜೀವಿಯನ್ನು ಹೇಗೆ ಕಂಗೆಡಿಸುತ್ತದೆ ಎಂದು ಬಲ್ಲಿರಾ?\par}


\section*{ಬೆಚ್ಚಿದರೆ, ಬೆದರಿದರೆ}

\addsectiontoTOC{ಬೆಚ್ಚಿದರೆ, ಬೆದರಿದರೆ}

ನಾಗರಹಾವಿನ ಬಿಚ್ಚಿದ ಹೆಡೆಯನ್ನೂ, ಕ್ರೂರದೃಷ್ಟಿಯನ್ನೂ ಕಂಡ ಇಲಿಯು ಹೆದರಿಕೆಯಿಂದ ಅರ್ಧಸತ್ತಂತಾಗಿ ಚಲನಹೀನವಾಗಿಬಿಡುತ್ತದೆ!

ಹುಲಿಯ ಕ್ರೂರದೃಷ್ಟಿಗೆ ಸಿಲುಕಿ, ಬಹಳಷ್ಟು ಎತ್ತರದಲ್ಲಿದ್ದರೂ, ತಾನು ಗಟ್ಟಿಯಾಗಿ ಹಿಡಿದುಕೊಂಡ ಮರದ ಕೊಂಬೆಯನ್ನು ಬಿಟ್ಟು ಭಯದಿಂದ ನಡುಗುತ್ತ, ಕಪಿಯು ಹುಲಿಯ ಸಮೀಪ ಬಂದು ಬಿದ್ದುಬಿಡುತ್ತದೆ!

ಕುಂಟು ಸಿಂಹದ ಗರ್ಜನೆಯನ್ನು ಕೇಳಿದ ಜಿಂಕೆ ಭಯಗ್ರಸ್ತವಾಗಿ, ದಿಕ್ಕೆಟ್ಟು ಓಡಲಾರಂಭಿಸುತ್ತದೆ. ವೇಗವಾಗಿ ಓಡಿ ಆಪತ್ತಿನಿಂದ ತಪ್ಪಿಸಿಕೊಳ್ಳುವ ಸಾಮರ್ಥ್ಯ ಅದಕ್ಕಿದೆ. ಯಾವ ಕಡೆಯಿಂದ ಸಿಂಹದ ಗರ್ಜನೆ ಕೇಳಿಸುತ್ತಿದೆ ಯಾವ ದಿಕ್ಕಿನಲ್ಲಿ ಓಡಿದರೆ ಕ್ಷೇಮ ಎಂಬುದನ್ನೂ ಅದು ಯೋಚಿಸದೆ, ಭಯವಿಹ್ವಲವಾಗಿ ದಿಕ್ಕು ಕೆಟ್ಟಂತೆ ಓಡುತ್ತ ಸಿಂಹದ ಸಮೀಪವೇ ಬಂದು ಬೀಳುತ್ತದೆ!

ಇಲಿಯು ಹಾವನ್ನು ಕಂಡು ಹೆದರುವುದಾಗಲಿ, ಕಪಿಯು ಹುಲಿಯ ಕ್ರೂರ ದೃಷ್ಟಿಯನ್ನು ಕಂಡು ತತ್ತರಿಸುವುದಾಗಲಿ, ಜಿಂಕೆಯು ಸಿಂಹದ ಗರ್ಜನೆಯನ್ನು ಕೇಳಿ ಭಯಗ್ರಸ್ತವಾಗುವುದಾಗಲಿ ಸಹಜ ಪ್ರತಿಕ್ರಿಯೆಗಳೆ. ಭಯವು ಮುಂದಾಲೋಚನೆ ಮತ್ತು ಆತ್ಮರಕ್ಷಣೆಗೆ ಬೇಕಾದ ಒಂದು ಹುಟ್ಟುಗುಣವೆ. ಆದರೆ ಮಿತಿಮೀರಿದ ಭಯ ಹೇಗೆ ಘಾತುಕವಾಗುತ್ತದೆಂಬುದು ತಿಳಿಯಬೇಕಾದ ಅಂಶ.

ಕತ್ತಲಲ್ಲಿ ಏನನ್ನೋ ಅಸ್ಪಷ್ಟವಾಗಿ ಕಂಡು, ಭೂತ–ಪಿಶಾಚಿಗಳ, ಕಳ್ಳಕಾಕರ ಕಲ್ಪನೆಯ ಭಯದಿಂದ, ಕೂಗಿಕೊಳ್ಳಲೂ ಆಗದೇ ತತ್ತರಿಸುವವರಿದ್ದಾರೆ!

‘ನನಗೆ ಭೂತಪ್ರೇತಗಳ ಅಸ್ತಿತ್ವದಲ್ಲಿ ನಂಬಿಕೆ ಇಲ್ಲ. ಆದರೆ ರಾತ್ರಿಹೊತ್ತು, ಒಬ್ಬನೇ ಸಂಚಾರ ಮಾಡಲು ಭಯ–ಯಾರೋ ಹಿಂಬಾಲಿಸಿದಂತಾಗುತ್ತದೆ’ ಎಂದರಂತೆ ಮಹನೀಯರೊಬ್ಬರು!\break ‘ರಾಹುಕಾಲ ಗುಳಿಕಕಾಲ ಎಲ್ಲ ಮೂಢನಂಬಿಕೆ. ಆದರೆ ಯಾವ ಶುಭಕಾರ್ಯವನ್ನೂ ಆ ವೇಳೆಯಲ್ಲಿ ಮಾಡಲಾಗುವುದಿಲ್ಲ, ಹೆಂಗಸರು ಮಕ್ಕಳ ಮನಸ್ಸು ನೋಯಿಸಲು ಇಚ್ಛೆ ಇಲ್ಲ’ ಎಂದರಂತೆ ಇನ್ನೊಬ್ಬ ಮಹನೀಯರು!

\newpage

ನಮ್ಮ ಹಲವಾರು ಭಯದ ಭಾವನೆಗಳು ಶೈಶವದಿಂದಲೇ ನಮಗೆ ಬಳುವಳಿಯಾಗಿ ಬಂದಿರುತ್ತದೆ. ಆ ಭಾವನೆಗಳು ಮನಸ್ಸಿನ ಆಳದಲ್ಲಿ ಉಳಿದುಕೊಂಡಿರುತ್ತವೆ. ನಾವು ಅವುಗಳನ್ನು ಬಿಟ್ಟು ಬಿಡಲು ಯತ್ನಿಸಿದರೂ ಅವು ನಮ್ಮನ್ನು ಸುಲಭವಾಗಿ ಬಿಡಲಾರವು. ಅವು ಸುಮ್ಮನೆ ಕುಳಿತಿರಲೂ ಆರವು. ಅಜ್ಞಾತವಾಗಿ ಅವು ನಮ್ಮ ಮುನ್ನಡೆಯ ಪಥದಲ್ಲಿ ತಡೆ, ತೊಂದರೆಗಳನ್ನು ಒಡ್ಡುತ್ತವೆ.

\vskip 3pt

ಇನ್ನೂ ಚೆನ್ನಾಗಿ ಪರಿಣತಿ ಪಡೆಯದ ಸೈಕಲ್ ಸವಾರ ಎದುರುಗಡೆ ಬರುವ ವಾಹನಗಳನ್ನು ಕಂಡು ಗಾಬರಿಯಾಗಿ, ಸೈಕಲ್ ಎಲ್ಲಿ ಚರಂಡಿಯಲ್ಲಿ ಬೀಳುವುದೊ ಎಂಬ ಭಯಕ್ಕೊಳಗಾದಾಗ, ಅದು ಚರಂಡಿಯ ದಾರಿಯನ್ನೇ ಹಿಡಿಯುತ್ತದೆ!

\vskip 3pt

ಪರೀಕ್ಷೆ ಬರೆಯುವಾಗ ಓದಿದ್ದೆಲ್ಲ ಎಲ್ಲಿ ಮರೆತುಹೋಗುವುದೋ ಏನೋ! ಎಂದು ಭಯ ಪಡುವ ವಿದ್ಯಾರ್ಥಿಗಳು ಚೆನ್ನಾಗಿ ಓದಿದ್ದರೂ, ಪರೀಕ್ಷಾ ಸಮಯದಲ್ಲಿ ಮುಖ್ಯ ಸಂಗತಿಗಳನ್ನು ಮರೆತು ಪರೀಕ್ಷಾಭವನದಿಂದ ಬಂದ ಮೇಲೆ, ‘ಅಯ್ಯೋ, ಗೊತ್ತಿದ್ದೂ ಮರೆತುಬಿಟ್ಟೆನಲ್ಲ’ ಎಂದು ಪರಿತಾಪಪಡುವುದುಂಟು! ಪರೀಕ್ಷಾಭಯದಿಂದ ವಿದ್ಯಾರ್ಥಿಗಳು ಎಷ್ಟೊಂದು ಸಂಕಟ ಪಡುತ್ತಾರೆ! ಈ ಭಯ ಅವರನ್ನು ಹೇಗೆ ತಪ್ಪು ದಾರಿಯಲ್ಲಿ ನಡೆಯಲು ಪ್ರೇರಣೆಯನ್ನೀಯುತ್ತಿದೆ ಎಂಬುದನ್ನು ವಿವರಿಸಬೇಕಿಲ್ಲ. ಈ ಭಯವನ್ನು ದೂರಮಾಡಲು ಸಾಧ್ಯವಿಲ್ಲವೆ?


\section*{ಉದ್ವೇಗದ ಉರುಳು}

\addsectiontoTOC{ಉದ್ವೇಗದ ಉರುಳು}

ಭಯ ಉದ್ವೇಗಗಳು ಸೇರಿಕೊಂಡಾಗ ನಾವು ಕಲ್ಪಿಸಿಕೊಳ್ಳುವ ಅಪಾಯದ ಚಿತ್ರಗಳು ಕಾರ್ಯ ರೂಪಕ್ಕೆ ಬಂದುಬಿಡುತ್ತವೆ!

\vskip 3pt

ಅವರು ಒಳ್ಳೆಯ ಸಂಭಾಷಣಾಚತುರರು. ಅದು ಅವರ ಮೊದಲ ಭಾಷಣ. ಸಭೆಯಲ್ಲಿ ಭರ್ಜರಿಯಾದ ಉಪನ್ಯಾಸ ನೀಡಲು ಅವರು ಸಂಕಲ್ಪಿಸಿದ್ದರು. ಸಾಕಷ್ಟು ಸಿದ್ಧತೆಯನ್ನೂ ಮಾಡಿದ್ದರು. ನಿಲುಗನ್ನಡಿಯ ಎದುರು ನಿಂತು ಹಾವಭಾವ ಅಭಿನಯ ಮಾಡುತ್ತ ಮುಖ\-ಭಾವ\-ಪ್ರಕಾಶದ ಕಲೆಯನ್ನೂ ಅಭ್ಯಸಿಸಿದ್ದರು. ಮನೆ ಮಂದಿಯನ್ನು ಒಂದೆಡೆ ಕೂಡಿಸಿ, ಅವರೆದುರು ಭಾಷಣ ಬಿಗಿದು ತಯಾರಿ ನಡೆಸಿದ್ದರು. ಉತ್ತಮ ಪೋಷಾಕನ್ನು ಧರಿಸಿ ಮುಗುಳುನಗೆಯನ್ನು ಬೀರುತ್ತ, ಗಾಂಭೀರ್ಯದ ನಡೆಯಿಂದ ಆತ್ಮವಿಶ್ವಾಸವನ್ನು ಹೊಂದಿದವರಾಗಿಯೇ ಸಭೆಯನ್ನು ಪ್ರವೇಶಿಸಿದರು. ಎಲ್ಲರೂ ಕುತೂಹಲದಿಂದ ಅವರ ಭಾಷಣ ಕೇಳುವುದಕ್ಕಾಗಿ ಅವರನ್ನೇ ದಿಟ್ಟಿಸಿ ನೋಡುತ್ತ ಕುಳಿತರು. ತುಂಬಿದ ಸಭೆಯನ್ನು ಕಂಡಾಗ ನೂರಾರು ದೃಷ್ಟಿಗಳು ಅವರನ್ನು ತಿವಿಯುತ್ತಿರುವಂತೆ ಭಾಸವಾಗಬೇಕೆ? ಅವರ ತುಟಿ ಒಣಗತೊಡಗಿತು. ಹಾವು ನಾಲಗೆಯನ್ನು ಹೊರಚಾಚುವಂತೆ ಅವರು ಆಗಾಗ ನಾಲಗೆಯಿಂದ ತಮ್ಮ ಒಣಗಿದ ತುಟಿಗಳನ್ನು ಸವರಿಕೊಂಡರು. ಮುಖದಲ್ಲಿ ಒಂದು ತೆರನಾದ ಕಳವಳದ ಛಾಯೆ ಕಾಣಿಸಿತು. ಎದ್ದುನಿಂತು ಮಾತನಾಡಹೊರಟಾಗ ಕೃತಕ ಕೆಮ್ಮು ಅವರ ಸಂಕಟವನ್ನು ಮರೆಮಾಚಲು ಸಹಕರಿಸಿತು. ಚಳಿಗಾಲವಾಗಿದ್ದರೂ ಮೈ ಬೆವತು ಹೋಗಿತ್ತು! ಮುಕ್ಕಾಲು ಗಂಟೆ ಮಾಡಬೇಕಿದ್ದ ಭಾಷಣವನ್ನು ಹತ್ತು ನಿಮಿಷದಲ್ಲೇ ಮುಗಿಸಿ, ದೇಹಾಲಸ್ಯದ ನೆವದಿಂದ ಬೇಗನೆ ಮನೆಗೆ ಸಾಗಿದರು!

‘ಎಷ್ಟೇ ಜನರಿದ್ದರೂ ಕುಳಿತು ಸಂಭಾಷಿಸಲು ಕಷ್ಟವಿಲ್ಲ. ಆದರೆ ಸಭೆಯ ಎದುರು ನಿಂತಾಗ ಏನೋ ಒಂದು ತರಹ ಆಗಿಬಿಡುತ್ತದೆ’ ಎನ್ನುತ್ತಾರವರು!

ಮಾಡುವ ಯಾವುದೇ ಕೆಲಸ ಭಯದಿಂದಲೇ, ಉದ್ವೇಗದ ಬಿರುಸಿನಿಂದಲೇ ಅಸ್ತವ್ಯಸ್ತವಾಗುವುದು, ಹಾಳಾಗುವುದು.

ಮುಂಜಾನೆ ಹೊಲದಲ್ಲಿ ಕೆಲಸಕ್ಕಾಗಿ ಹೊರಟ ರೈತನೊಬ್ಬ ಕೋಣೆಯ ಮೂಲೆಯಲ್ಲಿ ಚೀಲ\-ದಲ್ಲಿಟ್ಟಿದ್ದ ಬಿತ್ತನ್ನು ಬುಟ್ಟಿಯಲ್ಲಿ ತುಂಬಿಸಿಕೊಂಡು ಹೊರಟ. ಚೀಲವನ್ನು ಸಮೀಪಿಸಿದಾಗ ಕಾಲುಬೆರಳಿಗೆ ಮುಳ್ಳು ಚುಚ್ಚಿದ ಅನುಭವವಾಗಿತ್ತು. ಸಂಜೆಯವರೆಗೂ ಹೊಲದಲ್ಲಿ ದುಡಿದು ಉಳಿದ ಬಿತ್ತನ್ನು ಹಿಂದೆ ತಂದು ಆ ಚೀಲದೊಳಗೇ ಸುರುವಿದ. ಆಗ ಅಲ್ಲಿದ್ದ ನಾಗರಹಾವು ‘ಬುಸ್​’ ಎಂದಿತು. ಬೆಳಗಿನ ಹೊತ್ತು ಮುಳ್ಳು ಚುಚ್ಚಿದಂತಾದ ಅನುಭವವನ್ನು ನೆನಪಿಸಿಕೊಂಡು ಆ ಹಾವೇ ಕಚ್ಚಿದ್ದಿರಬೇಕು ಎಂಬ ಭಯದಿಂದ ಅಲ್ಲೇ ಕುಸಿದುಬಿದ್ದ! ಸತ್ತೇಹೋದ! ಹಾವಿನ ವಿಷದಿಂದ ಅವನು ಸತ್ತದ್ದಲ್ಲ; ಭಯದಿಂದಲೇ ಸತ್ತ!

ಭಯದ ದುಷ್ಪರಿಣಾಮವನ್ನು ಧ್ವನಿಸುವ ಕತೆಯೊಂದಿದೆ–

ಒಮ್ಮೆ ಯಮಧರ್ಮ ಭೂಲೋಕದಿಂದ ನಾನೂರು ಜೀವಿಗಳನ್ನು ತನ್ನ ಲೋಕಕ್ಕೆ ಕರೆತರುವಂತೆ ದೂತರಿಗೆ ಆಜ್ಞೆಯನ್ನು ನೀಡಿದ. ಯಮನ ದೂತರು ವಿವಿಧ ರೋಗಾಸ್ತ್ರಗಳನ್ನು ಉಪಯೋಗಿಸಿ ನಾನೂರು ಜೀವಿಗಳನ್ನು ಕೊಲ್ಲಲು ವ್ಯವಸ್ಥೆ ಮಾಡಿದರು. ಆದರೆ ನಾನೂರು ಜೀವಿಗಳ ಬದಲು ಎಂಟುನೂರು ಜೀವಿಗಳನ್ನು ಅವರು ಕರೆತಂದಾಗ, ‘ನನ್ನ ಆಜ್ಞೆಗೆ ವಿರೋಧವಾಗಿ ಏಕೆ ವರ್ತಿಸಿದಿರಿ?’ ಎಂದು ಯಮಧರ್ಮ ದೂತರನ್ನು ಗದರಿಸಿದ. ದೂತರು ಇಂತೆಂದರು: ‘ಮಹಾ ಪ್ರಭು, ನಾವು ನಾನೂರು ಮಂದಿಯನ್ನೇ ಕೊಂದೆವು. ಉಳಿದ ನಾನೂರು ಮಂದಿ ಭಯದಿಂದಲೇ ಪ್ರಾಣಬಿಟ್ಟರು!’

ಭಯದ ಸ್ವರೂಪ, ಸ್ವಭಾವ ಮತ್ತು ಅದು ನಮ್ಮನ್ನು ದುರಂತಕ್ಕೆಳೆಯುವ ವಿಧಾನದ ಅಲ್ಪಸ್ವಲ್ಪ ಪರಿಚಯ ಈಗ ನಿಮಗಾಗಿರಬೇಕಲ್ಲವೆ?


\section*{ಭಯದ ಇತಿಮಿತಿ}

\addsectiontoTOC{ಭಯದ ಇತಿಮಿತಿ}

ಶಿಶುವು ದೊಡ್ಡ ಕರ್ಕಶ ಧ್ವನಿಯನ್ನು ಕೇಳಿ ಭಯಪಡುತ್ತದೆ. ನಡೆಯಲು ಯತ್ನಿಸುವಾಗ ಬಿದ್ದು ಬಿಟ್ಟೇನು ಎಂಬ ಭಯ ಅದನ್ನು ಆವರಿಸುತ್ತದೆ. ಭಯ ಒಂದು ಸ್ವಾಭಾವಿಕ ಹುಟ್ಟುಗುಣ. ನಮ್ಮ ಅಸ್ತಿತ್ವ ಮತ್ತು ಸ್ವಾತಂತ್ರ್ಯಕ್ಕೆ ಬರಬಹುದಾದ ಅಪಾಯದ ಸನ್ನಿವೇಶ ಮತ್ತು ಯೋಚನೆಗಳು ನಮ್ಮಲ್ಲಿ ಭಯವನ್ನುಂಟುಮಾಡುತ್ತವೆ. ಬದುಕಿನ ಸಮಸ್ಯೆಗಳನ್ನು ಎದುರಿಸಲು ಸಾಮರ್ಥ್ಯ, ದಕ್ಷತೆ ಸಾಲದು ಎನ್ನುವ ವ್ಯಕ್ತಿಯೂ ಭಯಗ್ರಸ್ತನಾಗುತ್ತಾನೆ. ಕೀಳರಿಮೆಯನ್ನು ಬೆಳೆಸಿಕೊಳ್ಳುತ್ತಾನೆ. ಆದರೆ ಅಪಾಯವನ್ನು ಎದುರಿಸಿ ದಾಟಿದ ಬಳಿಕ ನಿರ್ಭೀತಿಯ ಮನೋಭಾವ, ಆತ್ಮ ವಿಶ್ವಾಸ ವೃದ್ಧಿಯಾಗುತ್ತದೆ. ಮಗು ಏಳುವಾಗ ಬೀಳುವಾಗ ಮೊದಲು ಹೆದರಿದರೂ, ಕೊನೆಗೆ ನಡೆಯಲು ಮಾತ್ರವಲ್ಲ, ಓಡಲು ಕಲಿಯುತ್ತದೆ. ಆರೋಗ್ಯಕರವಾದ ಭೀತಿ ನಮ್ಮನ್ನು ಜಾಗರೂಕರನ್ನಾಗಿ ಮಾಡುತ್ತದೆ! ಅತಿರಂಜಿತ, ಕಾಲ್ಪನಿಕ ಹಾಗೂ ನಿಷೇಧಾತ್ಮಕ ಭಯ ಕೆಟ್ಟದ್ದು ಎಂಬುದನ್ನು ನಾವು ಅರಿತಿರಬೇಕು. ಅಪಾಯದ ಅರಿವು ಆ ಅಪಾಯವನ್ನು ದಾಟುವ ವಿಧಾನವನ್ನು ಹುಡುಕಲು ನಮ್ಮನ್ನು ಪ್ರೇರಿಸುವಂತಿರಬೇಕು. ಈ ನಿಟ್ಟಿನಲ್ಲಿ ಭಯ ನಮಗೊಂದು ಪಂಥಾಹ್ವಾನ ಅಥವಾ ಸವಾಲು. ನಮ್ಮ ಅಂತಶ್ಶಕ್ತಿಯನ್ನು ಎಚ್ಚರಿಸಲು ಒಂದು ಅವಕಾಶ ಎಂದು ತಿಳಿಯಬೇಕು.


\section*{ದೇವರ ಭಯ!}

\addsectiontoTOC{ದೇವರ ಭಯ!}

ದೇವರ ಭಯವೇ ಜ್ಞಾನದ ಆರಂಭ ಎನ್ನುವ ಗಾದೆಯ ಮಾತಿದೆ. ದೇವರನ್ನು ಕುರಿತು ನಾವೇಕೆ ಭಯಪಡಬೇಕು? ನಿಜವಾಗಿಯೂ ಭಯಪಡಬೇಕಿಲ್ಲ. ದೇವರು ಪರಮ ಪ್ರೇಮಸ್ವರೂಪ. ಅವನಷ್ಟು ಪ್ರಿಯಸ್ವಭಾವಿ ಬೇರಾರೂ ಇಲ್ಲ. ಅಂಥ ಪರಿಶುದ್ಧಪ್ರೇಮ ಇರುವಲ್ಲಿ ಭೀತಿ ನೆಲೆಸಲು ಸಾಧ್ಯವಿಲ್ಲ. ಹಾಗಾದರೆ ದೇವರ ಭಯ ಜ್ಞಾನಕ್ಕೆ ಹೇಗೆ ಕಾರಣವಾದೀತು?

ಉತ್ತರ ಹೀಗಿದೆ: ‘ಸರ್ವಜ್ಞನೂ, ಸರ್ವಶಕ್ತನೂ ಆದ ದೇವರಿದ್ದಾನೆ. ಅವನಿಗೆ ತಿಳಿಯದ್ದು ಯಾವುದೂ ಇಲ್ಲ. ಅವನನ್ನು ಮೋಸಗೊಳಿಸಲು ಯಾರಿಂದಲೂ ಸಾಧ್ಯವಿಲ್ಲ. ಆತನಿಂದಾಗಿ ಈ ವಿಶ್ವಬ್ರಹ್ಮಾಂಡವೆಲ್ಲ ಒಂದು ವ್ಯವಸ್ಥಿತ ರೀತಿಯಲ್ಲಿ ನಿಯಮಪೂರ್ವಕವಾಗಿ ನಡೆಯುತ್ತಿದೆ. ನಮ್ಮ ಜೀವನವೂ ಯಾವುದೋ ಒಂದು ಸೂಕ್ಷ್ಮನಿಯಮವನ್ನನುಸರಿಸಿ ನಡೆಯುತ್ತಿದೆ. ನಮ್ಮ ಜೀವನ ಸಾರ್ಥಕವಾಗಬೇಕಾದರೆ, ನಾವೆಲ್ಲರೂ ಕೆಲವು ನೈತಿಕ ನಿಯಮಗಳನ್ನು ಪಾಲಿಸಬೇಕು. ಆರೋಗ್ಯದ ನಿಯಮಗಳನ್ನು ಪಾಲಿಸಿ ಆರೋಗ್ಯವಂತರಾಗುವಂತೆ, ನೈತಿಕ ನಿಯಮಗಳನ್ನು ಪಾಲಿಸಿ, ಮನಃಸ್ಥೈರ್ಯವನ್ನು ಪಡೆದು ನಾವು ನಿಜವಾದ ಮಾನವರಾಗಬೇಕು. ಭಗವಂತನ ರಾಜ್ಯದ ನೈತಿಕ ನಿಯಮಗಳಾದ ‘ಕಳ ಬೇಡ, ಕೊಲ ಬೇಡ, ಹುಸಿಯ ನುಡಿಯಲು ಬೇಡ’– ಇವುಗಳನ್ನು ನಾವೆಲ್ಲರೂ ಪಾಲಿಸಬೇಕು. ಸರಿಯಾಗಿ ಪಾಲಿಸಬೇಕು. ಹಾಗೆ ಪಾಲಿಸದಿದ್ದರೆ ತೊಂದರೆಗೊಳಗಾಗುತ್ತೇವೆ. ನಿಯಮಭಂಗದಿಂದ ನಮಗೂ ನಮ್ಮ ಸುತ್ತಮುತ್ತಲ ಜನರಿಗೂ ಕೆಡುಕಾಗುತ್ತದೆ ಎಂಬ ಪ್ರಜ್ಞೆ ಇದ್ದರೆ, ಜಾಗರೂಕತೆ ಅಥವಾ ಮುನ್ನೆಚ್ಚರಿಕೆ ಎನ್ನುವ ಭಯ ಉಂಟಾಗುತ್ತದೆ. ಈ ಭಯ ಬೇಕಾದದ್ದೆ. ಇದು ನಿಜವಾಗಿಯೂ ಭಯವಲ್ಲ. ಎಚ್ಚರಿಕೆ! ಎಲ್ಲಿ ತಪ್ಪು ಮಾಡಿಬಿಡುವೆನೋ, ಸುಳ್ಳು ಹೇಳುವಂತಾಗುವುದೋ, ಕಳ್ಳತನದ ದಾರಿಹಿಡಿಯುವೆನೋ, ಅನ್ಯಾಯ ವಂಚನೆಗಳನ್ನು ಮಾಡಿಬಿಡುವೆನೊ, ಕರ್ತವ್ಯಭ್ರಷ್ಟನಾಗುವೆನೊ ಎಂಬ ಭಯ ಅಥವಾ ಎಚ್ಚರಿಕೆ ದುರ್ನಡತೆಯ ಜಾಡಿನಿಂದ ನಮ್ಮನ್ನು ರಕ್ಷಿಸುತ್ತದೆ. ಏಕೆಂದರೆ ದುರ್ನಡತೆಯ ದುರ್ಬೀಜಗಳನ್ನು ಬಿತ್ತಿದರೆ, ದುಃಖರೂಪವಾದ ಫಲಗಳನ್ನು ಪಡೆಯಲೇಬೇಕಾಗುತ್ತದಷ್ಟೆ.

\newpage

ದೇವರನ್ನು ವಿಷ್ಣುಸಹಸ್ರನಾಮದಲ್ಲಿ ‘ಭಯಕೃತ್ ಭಯನಾಶನಃ’ ಎಂದು ಸ್ತುತಿಸಿದ್ದಾರೆ. ತಪ್ಪು ಮಾಡಿದವನಿಗೆ ಭಯಂಕರನು, ಸಜ್ಜನರ ಭಯವನ್ನು ದೂರಮಾಡುವವನು ಅವನು ಎಂಬುದು ಆ ಮಾತಿನ ಅರ್ಥ. ನರಸಿಂಹನನ್ನು ಕಂಡಾಗ ಹಿರಣ್ಯಕಶಿಪು ಭಯಗ್ರಸ್ತನಾದ. ಆದರೆ ಪ್ರಹ್ಲಾದ ‘ನಾಹಂ ಬಿಭೇಮಿ’–ನಾನು ಹೆದರುತ್ತಿಲ್ಲ ಎಂದ! ಎಂಥ ತಪ್ಪು ಮಾಡಿದವನಾದರೂ, ತಪ್ಪೊಪ್ಪಿಕೊಂಡು ಪಶ್ಚಾತ್ತಾಪದಿಂದ ಕರಗಿ ಭಗವಂತನಲ್ಲಿ ಮೊರೆಯಿಟ್ಟು ಶರಣಾದರೆ, ಅವನೂ ಧರ್ಮಾತ್ಮನಾಗಿ ಪರಿವರ್ತನೆ ಹೊಂದಲು ಭಗವಂತ ಸದಾ ಸಹಾಯ ಹಸ್ತವನ್ನು ನೀಡುತ್ತಾನೆ. ಆತ ದಯೆ ತೋರಿ ‘ಎಲ್ಲರನು ಸಲಹುವನು’ ಎಂಬುದು ಸತ್ಯಸ್ಯ ಸತ್ಯ. ಆದರೆ ಪಾಪಿ ತನ್ನ ಪಾಪ ಪ್ರಜ್ಞೆಯಿಂದ ಬಿಡಿಸಿಕೊಳ್ಳಲು ಎಡೆಬಿಡದೆ ಶ್ರದ್ಧೆಯಿಂದ ಹೋರಾಡಬೇಕಾಗುವುದು. ದೇವರು ಮಾಡಿದ ನಿಯಮಗಳನ್ನು ಸರಿಯಾಗಿ ತಿಳಿದುಕೊಂಡು ಅವುಗಳನ್ನು ಚಾಚೂ ತಪ್ಪದೆ ಅನುಸರಿಸಬೇಕೆಂಬ ಅರಿವೇ ಜ್ಞಾನ. ಅಂಥ ಕ್ರಿಯೋನ್ಮುಖ ಜ್ಞಾನದಿಂದಲೆ ಜೀವನ ಸಾರ್ಥಕ್ಯ. ನಿಯಮ ಮತ್ತು ನಿಯಾಮಕರ ಸ್ವರೂಪದ ಅರಿವಿನಿಂದ ಉದಿಸುವ ಎಚ್ಚರಿಕೆಯೇ ಜ್ಞಾನಕ್ಕೆ ಸಾಧನ. ಪ್ರಕೃತಿರಹಸ್ಯಗಳನ್ನು ಭೇದಿಸಿ ಸೂಕ್ಷ್ಮಪ್ರಾಕೃತಿಕ ನಿಯಮಗಳನ್ನು ವಿಜ್ಞಾನಿ ನಮಗೆ ತಿಳಿಸಿಕೊಡುತ್ತಾನೆ. ಪ್ರಪಂಚದ ಧರ್ಮಗ್ರಂಥಗಳು, ಋಷಿಮುನಿಗಳು, ಅನುಭಾವಿಗಳು, ಮಹಾತ್ಮರು ನಮಗೆ ನೈತಿಕ, ಧಾರ್ಮಿಕ ನಿಯಮಗಳನ್ನು ತಿಳಿಸಿಕೊಡುತ್ತಾರೆ.

ನಮ್ಮ ಜನರು ದೇವರ ಬಗೆಗೂ ನಾನಾ ತೆರನಾದ ಭಯಗಳನ್ನು ಬೆಳೆಸಿಕೊಳ್ಳುತ್ತಾರೆ. ಇದಕ್ಕೆ ಕಾರಣ ದೇವರ ಬಗ್ಗೆ ಬಾಲ್ಯದಿಂದಲೇ ಅವರು ರೂಢಿಸಿಕೊಂಡ ನಾನಾ ತೆರನಾದ ಅಸಂಬದ್ಧ ಕಲ್ಪನೆಗಳು. ಸ್ತೋತ್ರ ಪಾಠ ಮಾಡುವಾಗ ಮಡಿ ಸಾಕಾಗಲಿಲ್ಲ. ಮೈಲಿಗೆಯಾಯಿತೊ ಏನೋ ಎಂದು ಹೆದರಿಕೊಂಡೇ ದೇವರನಾಮ ಪಾರಾಯಣ ಮಾಡಿ ಕೆಡುಕನ್ನು ನಿರೀಕ್ಷಿಸುವವರಿದ್ದಾರೆ! ದೇವರು ಶಾಪ ಕೊಟ್ಟು ಬಿಡುತ್ತಾನೆಂದು ಯೋಚಿಸುವವರಿದ್ದಾರೆ! ಆ ಸ್ತೋತ್ರಪಾಠ, ದೇವರ ನಾಮ, ಹೇಳತೊಡಗಿದ ಮೇಲೆಯೇ ದಾರಿಯಲ್ಲಿ ಕಾಲುಜಾರಿ ಬಿದ್ದು ಜಖಂ ಆದದ್ದು! ಆ ಸ್ತೋತ್ರ ನನಗಾಗುವುದಿಲ್ಲ! ಅದರಿಂದಲೇ ಕೆಡುಕಾದುದು ಎನ್ನುವವರಿದ್ದಾರೆ!

ಹಳ್ಳಿಗನೊಬ್ಬ ಓಡಿ ಬಂದು ಬಸ್ಸನ್ನೇರಿದ. ಬಸ್ಸಿನಲ್ಲಿ ಕುಳಿತ ಒಂದು ನಿಮಿಷದಲ್ಲೇ ತಲೆಸುತ್ತು ಬಂದು ಬಿದ್ದ. ಹತ್ತಿರವಿದ್ದ ಜನರು ಶೈತ್ಯೋಪಚಾರ ಮಾಡಿದರು. ಡಾಕ್ಟರ್ ಬಂದು ಇಂಜೆಕ್ಶನ್ ಚುಚ್ಚಿದರು. ಸ್ವಲ್ಪ ಹೊತ್ತಿನಲ್ಲೇ ಆತ ಚೇತರಿಸಿಕೊಂಡ. ಎದ್ದುಕುಳಿತು ಸುತ್ತಮುತ್ತ ನೋಡಿದ. ತನ್ನ ದುಃಸ್ಥಿತಿಗೆ ಆ ಬಸ್ಸೇ ಕಾರಣವೆಂದ. ಬಸ್ಸಿನ ಡೈವರ್ ಹಾಗೂ ಕಂಡಕ್ಟರುಗಳನ್ನು ನಿಂದಿಸಿದ. ಈ ಬಸ್ಸನ್ನೇರಿದ ಕೂಡಲೇ ತಲೆಸುತ್ತಿದ್ದು. ಬಸ್ಸಿನಲ್ಲೇ ಏನೋ ದೋಷವಿದೆ ಎಂದ. ನಿಜವಾಗಿಯೂ ಬಸ್ಸಿನ ದೋಷವೇ ಅದು? ಅಲ್ಲ. ಯಾವುದೋ ಹೋಟೆಲಿನಲ್ಲಿ ಹಳಸಿದ ಪದಾರ್ಥ ತಿಂದುಕೊಂಡು, ಬಿಸಿಲಿನಲ್ಲಿ ಓಡಿ ಬಂದುದರಿಂದ ತಲೆಸುತ್ತು ಬಂದದ್ದು. ಆದರೆ ಅವನು ಕಂಡು ಕೊಂಡ ಕಾರ್ಯಕಾರಣ ಸಂಬಂಧ ಬೇರೆ! ಬಸ್ಸು ಹತ್ತಿದ ಕೂಡಲೇ ತಲೆ ತಿರುಗಿ ಬಿದ್ದುದರಿಂದ ಬಸ್ಸೇ ಕಾರಣ ಎಂಬ ವಾದ ಅವನದು. ಹಾಗೆಯೇ ದೇವರು ಎಂದಾದರೂ ಯಾರಿಗಾದರೂ ಕೆಡುಕನ್ನು ಮಾಡುವನೆ? ಎಂದಿಗೂ ಮಾಡುವುದಿಲ್ಲ. ಆದರೆ ಭಕ್ತನೆನಿಸಿಕೊಂಡವನು ತನ್ನ ಅಜ್ಞಾನ, ಮೂರ್ಖತೆ, ಮೂಢನಂಬಿಕೆಗಳಿಂದ ತೊಂದರೆಗಳನ್ನು ಆಮಂತ್ರಿಸಿಕೊಳ್ಳುತ್ತಾನೆ.


\section*{ದೇವಪ್ಪನ ದಿಗಿಲು}

\vskip -6pt\addsectiontoTOC{ದೇವಪ್ಪನ ದಿಗಿಲು}

ದೇವಪ್ಪ ಒಳ್ಳೆಯ ಯುವಕ, ಇಲೆಕ್ಟ್ರಿಕ್ ಫಿಟ್ಟಿಂಗ್ ಕೆಲಸದಲ್ಲಿ ಬಹಳ ಬುದ್ಧಿವಂತನೆನಿಸಿ ಕೊಂಡಿದ್ದಾನೆ. ಕೆಲವೇ ಗಂಟೆಗಳ ಹೊತ್ತಿನಲ್ಲಿ ನಾಜೂಕಾಗಿ ಇಡಿಯ ಮನೆಯ ವೈಯರಿಂಗ್ ಮಾಡಿ ಮುಗಿಸಬಲ್ಲ. ಒಮ್ಮೆ ಯಾರೋ ಪರಿಚಿತ ಗೌರವಾರ್ಹ ಹಿರಿಯರು ಆತನಿಗೆ ಹೀಗೆಂದರು: ‘ನಿನಗೆ ವಂಶಗತವಾಗಿ ಬಂದ ಕ್ಷಯರೋಗ ತಗಲಬಲ್ಲುದು. ನಿನ್ನ ವಂಶಸ್ಥರು ಆ ರೋಗದಿಂದಲೇ ಸತ್ತದ್ದು.’ ಈ ಮಾತು ಮತ್ತೆ ಮತ್ತೆ ಅವನ ಮನಸ್ಸಿನಲ್ಲಿ ಕೆಲವು ದಿನಗಳವರೆಗೆ ಅನು ರಣಿತವಾಗುತ್ತಿತ್ತು. ಈಗ ಆತ ಕೆಮ್ಮು ಬಂದಾಗಲೆಲ್ಲ ಅಂಜಿಕೊಳ್ಳುತ್ತಿದ್ದಾನೆ. ಕ್ಷಯರೋಗದ ಲಕ್ಷಣಗಳ ವಿವರಣೆಯನ್ನು ಯಾವುದೋ ಪುಸ್ತಕದಲ್ಲಿ ಓದಿದಾಗ ಅವನು ಹೌಹಾರಿದ. ಆ ಲಕ್ಷಣಗಳೆಲ್ಲ ಹೆಚ್ಚು ಕಡಿಮೆ ತನ್ನಲ್ಲಿವೆ ಎಂಬ ಅರಿವಾಗಿ ಕಂಗಾಲಾದ! ‘ನಾಲಿಗೆಯ ರುಚಿ ಕೆಟ್ಟುಹೋಗಿದೆ. ದಿನವೂ ಸಂಜೆಯ ಹೊತ್ತು ಸ್ವಲ್ಪ ಜ್ವರ ಬರಲು ಆರಂಭವಾಗಿದೆ. ಯಾವ ಕೆಲಸವನ್ನೂ ಮಾಡಲು ಉತ್ಸಾಹವಿಲ್ಲ, ನನ್ನ ಕತೆ ಮುಗಿಯಿತು’ ಎಂದು ಆಗಾಗ ದೀನನಾಗಿ ಹೇಳಿಕೊಳ್ಳುತ್ತಿದ್ದ. ಮೂರು ತಿಂಗಳಲ್ಲೇ ಅವನ ಆರೋಗ್ಯ ತೀರ ಕೆಡುತ್ತ ಬಂತು. ಡಾಕ್ಟರರು ಯಾವ ರೋಗವೂ ಇಲ್ಲ ಎಂದರೂ ಅವನು ನಂಬಲಾರ. ಕೊನೆಗೆ ಎಕ್ಸ್​ರೇ ಪಟಲದಲ್ಲಿ ಅವನ ಎದೆಗೂಡಿನ, ಶ್ವಾಸಕೋಶಗಳ ಚಿತ್ರಗಳನ್ನು ತೆಗೆದು ಡಾಕ್ಟರ್ ‘ಎಲ್ಲ ಸರಿಯಾಗಿದೆ, ಏನೂ ತೊಂದರೆ ಇಲ್ಲ. ಈ ಔಷಧಿ ಸೇವಿಸು, ಒಂದು ವಾರದಲ್ಲಿ ಸರಿ ಹೋಗುವುದು’ ಎಂದರು. ಆರೋಗ್ಯವಂತನ ಶ್ವಾಸಕೋಶದ ಚಿತ್ರವನ್ನೂ, ರೋಗಿಯ ಶ್ವಾಸಕೋಶದ ಚಿತ್ರವನ್ನು ತೋರಿಸಿ, ಅವನು ಹೇಗೆ ಆರೋಗ್ಯವಂತ ಎಂಬುದನ್ನು ಸ್ಪಷ್ಟವಾಗಿ ತಿಳಿಸಿದರು. ಕೆಲವೇ ದಿನಗಳಲ್ಲಿ ಆತ ಆ ಮೊದಲಿನ ಆರೋಗ್ಯದ ಸ್ಥಿತಿಯನ್ನು ಪಡೆದುಕೊಂಡ! ಭಯ, ಸಂಶಯಗಳ ತಾಕಲಾಟದಲ್ಲಿ ಇನ್ನೂ ಮುಂದುವರಿದಿದ್ದರೆ ಯಾವ ಔಷಧವೂ ಆತನಿಗೆ ನಾಟುತ್ತಿರಲಿಲ್ಲ.


\section*{ಭಯದ ಮೂಲ, ಜಾಲ}

\vskip -6pt\addsectiontoTOC{ಭಯದ ಮೂಲ, ಜಾಲ}

ಭಯವು ಕಾರ್ಯರೂಪಕ್ಕೆ ಬರುವಂತಾಗಲು ಪ್ರೇರಣೆ ಎಲ್ಲಿಂದ? ನೀವು ಯೋಚಿಸಿಕೊಂಡ ಅಪಾಯದ ಕಲ್ಪನೆಯೊಂದಿಗೆ, ಭಯದ ಭಾವನೆಯೂ ಸೇರಿ ಸುಪ್ತಮನಸ್ಸನ್ನು ಪ್ರವೇಶಿಸುತ್ತದೆ. ಕ್ರಮೇಣ ಭಯದ ಒತ್ತಡ ತೀವ್ರವಾಗಿ, ಅಪಾಯದ ಚಿತ್ರ ಸ್ಪಷ್ಟವಾಗುತ್ತಾ ಬಂದಲ್ಲಿ, ಚಿತ್ತ ಅದನ್ನು ಕಾರ್ಯರೂಪಕ್ಕೆ ತಂದೇ ಬಿಡುತ್ತದೆ.

ದೇವಪ್ಪ ಮೊದಲು ಕ್ಷಯ ರೋಗಿಯ ದುಃಸ್ಥಿತಿಯ ಚಿತ್ರವನ್ನು ಮನಸ್ಸಿನಲ್ಲಿ ಕಲ್ಪಿಸಿಕೊಂಡ. ತಾನೂ ಅಂಥ ನರಳಾಟವನ್ನು ಅನುಭವಿಸಬೇಕಲ್ಲ ಎಂದು ಭಯ ಗಾಬರಿಗಳನ್ನು ಬೆಳೆಸಿಕೊಂಡ. ತೀವ್ರತರವಾದ ಭಾವನೆ, ದುರಂತದ ಸ್ಪಷ್ಟವಾದ ಮಾನಸಿಕ ಚಿತ್ರ, ಸುಪ್ತಮನಸ್ಸಿನಲ್ಲಿ ದೃಢವಾದ ಸಂಸ್ಕಾರವನ್ನುಂಟು ಮಾಡಿತು; ಅದು ಕಾರ್ಯರೂಪಕ್ಕೂ ಬರುವಂತಾಯಿತು!

‘ಅಯ್ಯೋ! ಸೈಕಲ್ ಚರಂಡಿಯಲ್ಲಿ ಬೀಳುತ್ತದೆ’ ಎಂದು, ಬಿದ್ದು ಅನಾಹುತವಾಗುವ ಚಿತ್ರ ದೊಂದಿಗೆ ಭಯೋದ್ವೇಗಗಳನ್ನು ಸೇರಿಸಿದ ಯುವಕನು, ಚರಂಡಿಗೆ ಎಳೆಯಲ್ಪಟ್ಟ ಹಿಂದಿನ ಉದಾಹರಣೆಯನ್ನು ಸ್ಮರಿಸಿಕೊಳ್ಳಬಹುದು. ಪರೀಕ್ಷೆಯ ಭಯದಿಂದ ಓದಿದ್ದೆಲ್ಲ ಮರೆತು\break ಹೋದೀತು ಎಂಬ ಉದ್ವೇಗಭರಿತನಾದವನ ಕತೆಯೂ ಅಂಥದೆ.

‘ಕೆಡುಕುಗಳನ್ನೋ, ದೋಷಗಳನ್ನೋ, ತೀವ್ರತರದ ಭಾವೋದ್ವೇಗದಿಂದ ಯೋಚಿಸುತ್ತಿದ್ದರೆ ಅವು ನಮ್ಮ ಸುಪ್ತಮನಸ್ಸಿನಲ್ಲಿ ಮುದ್ರಿತವಾಗಿ, ಮೆಲ್ಲನೇ ಅಂಕುರಿತ, ಪಲ್ಲವಿತ, ಪುಷ್ಪಿತವಾಗ ತೊಡಗುತ್ತವೆ. ಫಲರೂಪವಾಗಿ ಅನಾರೋಗ್ಯ, ತೊಂದರೆ, ಸಂಕಟಗಳನ್ನು ಆಮಂತ್ರಿಸಿದಂತೆ\break ಆಗುತ್ತದೆ.’

‘ಯಾವುದೇ ಸಂಸ್ಕಾರ ಸುಪ್ತಮನಸ್ಸಿನಲ್ಲಿ ಮುದ್ರಿತವಾಗಿದ್ದರೆ, ಅದು ವ್ಯಕ್ತವಾಗದೇ ಇರದು. ನಾವು ಆಶಿಸದಿರುವುದನ್ನು, ನಮಗೆ ಬೇಡವಾದುದನ್ನು ತೀವ್ರವಾಗಿ ಚಿಂತನೆ ಮಾಡುವುದೆಂದರೆ, ನಮ್ಮ ಸುಪ್ತಮನಸ್ಸಿನಲ್ಲಿ ನಮಗೆ ಬೇಡವಾದುದರ ಬೀಜವನ್ನೇ ಸರಿಯಾಗಿ ಬಿತ್ತಿದಂತಾಗುತ್ತದೆ.\break ಇದರಿಂದ ಅದೇ ಬೆಳೆದು ಬಲವಾಗಿ ಯಾವುದನ್ನು ನಾವು ದೂರಕ್ಕೆಸೆಯಬೇಕೆಂದು ಭಾವಿಸು ತ್ತೇವೋ, ಅದರಲ್ಲೇ ಮುಳುಗಿಬಿಡುವ ಪ್ರಸಂಗ ಉಂಟಾಗುತ್ತದೆ.


\section*{ಗುಂಡ ಗುಂಡುಸೂಜಿಯನ್ನು ನುಂಗಿದ!}

\addsectiontoTOC{ಗುಂಡ ಗುಂಡುಸೂಜಿಯನ್ನು ನುಂಗಿದ!}

‘ಭಯ: ಸರಳ ವಿಶ್ಲೇಷಣೆ’ ಎಂಬ ತಮ್ಮ ಒಂದು ಪುಟ್ಟ ಪುಸ್ತಕದಲ್ಲಿ ಡಾ.\ ಶಿವರಾಂ ಒಂದು ಘಟನೆ ವಿವರಿಸಿದ್ದಾರೆ–

ಸೊಟ್ಟಗಾಗಿದ್ದ ಗುಂಡುಸೂಜಿಯನ್ನು ಶ‍್ರೀಮಂತ ಯುವಕ ಗುಂಡೂರಾಯ ಆಕಸ್ಮಿಕವಾಗಿ ನುಂಗಿದ. ಅಂತಹ ಸೂಜಿಯು ಕರುಳಿನಲ್ಲಿ ಎಲ್ಲಿಯಾದರೂ ಚುಚ್ಚಿಕೊಂಡರೆ, ನಂತರ ತಾಪತ್ರಯಗಳು ಬಹಳ. ಶಸ್ತ್ರಚಿಕಿತ್ಸೆಯು ಅಗತ್ಯವಾಗಬಹುದು. ಪ್ರಾಣಾಪಾಯವೂ ಉಂಟು. ಆರು ಗಂಟೆಗಳಿಗೊಮ್ಮೆ ಎಕ್ಸ್​ರೇಗಳನ್ನು ತೆಗೆಸುತ್ತಾ, ಮಲವಿಸರ್ಜನೆಯನ್ನು ಮಾಡಿದಾಗ ಗುಂಡು ಸೂಜಿ ಬಿದ್ದಿತೇ ಎಂದು ಪರೀಕ್ಷಿಸು ಎಂದು ಎಚ್ಚರಿಕೆ ಕೊಟ್ಟೆ. ನಾಲ್ಕನೆಯ ಎಕ್ಸ್​ರೇ ಪೋಟೋದಲ್ಲಿ ಗುಂಡುಸೂಜಿಯು ಕಾಣಿಸಲಿಲ್ಲ. ಅಂದರೆ ವಿಸರ್ಜನೆಯಾಗಿರಲೇಬೇಕು. ‘ಮಲವನ್ನು ನೀರು ಹಾಕಿ ತೊಳೆದು ಗುಂಡುಸೂಜಿ ಇತ್ತೇ, ನೋಡಿದೆಯಾ?’ ಎಂದು ಕೇಳಿದೆ. ‘ಅದನ್ನೆಲ್ಲಾ ಯಾರು ಪರೀಕ್ಷೆ ಮಾಡುತ್ತಾರೆ, ಡಾಕ್ಟರ್​! ಇದ್ದರೆ ಇರುತ್ತೆ, ಹೋದರೆ ಹೋಗುತ್ತೆ. ಎಕ್ಸ್​ರೇನಲ್ಲಿ ಹೇಗೂ ಕಾಣಿಸುತ್ತಿಲ್ಲವಲ್ಲ!’ ಎಂದು ಯಾವ ಆತಂಕವೂ ಇಲ್ಲದೆ ಅವನೆಂದ.

ಕರುಳಲ್ಲಿ ಗುಂಡುಸೂಜಿಯು ಸಿಕ್ಕಿಕೊಂಡಿದ್ದರೆ, ಆಪರೇಷನ್ ಮಾಡಿ ತೆಗೆಯುತ್ತಾರೆ ಎಂದು ಅವನು ಶಾಂತನಾಗಿದ್ದ. ಹಾಗಿದ್ದುದರಿಂದಲೇ ಅವನ ಕರುಳು ಸುಗಮವಾಗಿ ಕೆಲಸಮಾಡಿ ಗುಂಡು ಸೂಜಿಯನ್ನು ಹೊರಹಾಕಿತ್ತು! ‘ಗಂಡಾಂತರ ಕಾದಿದೆ, ಅಯ್ಯೋ ಏನು ಮಾಡಲಿ?’ ಎಂದು ಕಾತರನಾಗಿದ್ದರೆ ಹೆದರಿಕೆಯ ದೆಸೆಯಿಂದಲೇ ಸೂಜಿ ಅವನ ಕರುಳನ್ನು ಚುಚ್ಚುತ್ತಿತ್ತು!


\section*{ಗಾಳಿಯೊಡನೆ ಗುದ್ದಾಟ}

\addsectiontoTOC{ಗಾಳಿಯೊಡನೆ ಗುದ್ದಾಟ}

‘ನೀವು ಕೇಳಿದರೆ ನಗುತ್ತೀರಿ.’ ಯುವಕನೊಬ್ಬ ತನ್ನ ಕಾಲ್ಪನಿಕ ಭಯದ ಕತೆಯನ್ನು ಹೇಳಲು ತೊಡಗಿದ್ದ. ‘ಆ ದಿನಗಳಲ್ಲಿ ನನ್ನ ಹೆದರಿಕೆಗೆ ಆದಿ ಅಂತ್ಯವಿರಲಿಲ್ಲ. ಬಿರುಗಾಳಿ ಬೀಸಿದಾಗ ಮನೆಯ ಮಾಡು ಹಾರಿಹೋದರೇನು ಗತಿ? ಎಂದು ಚಿಂತಿಸುತ್ತಿದ್ದೆ. ಸಿಡಿಲು ಬಡಿದಾಗ ನಾನು ಸಾಯುವುದು ನಿಶ್ಚಯವೆಂದು ಚೀರಿಕೊಳ್ಳುತ್ತಿದ್ದೆ. ಯಾವುದೋ ಪುರಾಣದ ಕತೆಗಳಲ್ಲಿ ಹೇಳಿದ ನರಕದ ಚಿತ್ರಗಳನ್ನು ನೋಡಿ, ಸತ್ತಮೇಲೆ ಎಷ್ಟು ಕಷ್ಟ, ಏನೆಲ್ಲಾ ಅನುಭವಿಸಬೇಕೊ ಎಂದು ಅಳುತ್ತಿದ್ದೆ. ತೋಟದಲ್ಲಿ ಕೆಲಸಗಾರರು ಹೇಳಿದ ಕೆಲಸ ಮಾಡದಿದ್ದಾಗ, ಅಯ್ಯೋ! ಈ ಮನುಷ್ಯರೊಂದಿಗೆ ಹೇಗೆ ವ್ಯವಹಾರ ಮಾಡುವುದೆಂದು ನಿರಾಶನಾಗುತ್ತಿದ್ದೆ. ಅವರನ್ನು ಗದರಿಸಿದರೆ, ಅವರು ಒಟ್ಟಾಗಿ ನನ್ನನ್ನು ಹೇಳಹೆಸರಿಲ್ಲದಂತೆ ಮಾಡಿಯಾರೆಂದು ಭಯಪಡುತ್ತಿದ್ದೆ. ದಾರಿಯಲ್ಲಿ ಹೋಗುವಾಗ, ಹುಡುಗಿಯರು ನನ್ನನ್ನು ನೋಡಿ ನಕ್ಕರೆ? ಎಂದು ಚಿಂತಿಸುತ್ತಿದ್ದೆ. ಯಾವ ಹುಡುಗಿಯೂ ನನ್ನನ್ನು ಮದುವೆಯಾಗಲು ಒಪ್ಪದಿದ್ದರೆ ಏನು ಗತಿ? ಎಂಬ ಚಿಂತೆ ನನಗೆ. ಮದುವೆಯಾದ ಬಳಿಕ ಮಾವನ ಮನೆಯವರೊಡನೆ ಹೇಗೆ ನಡೆದುಕೊಳ್ಳಬೇಕು ಎಂದು ಚಿಂತಿಸುತ್ತಿದ್ದೆ. ಹೆಂಡತಿಯ ಮನೆ ಹಳ್ಳಿಯಾಗಿದ್ದು, ಕಾಡುದಾರಿಯಲ್ಲಿ ಬರುವಾಗ ದುಷ್ಟ ದರೋಡೆ ಗಾರರು ಎದುರಾದರೆ ಏನಾದೀತೆಂದು ಯೋಚಿಸಿ ನಡುಗುತ್ತಿದ್ದೆ.’

ಒಮ್ಮೆ ಪೇಟೆಯಿಂದ ಬರುವಾಗ ಮಿತ್ರನೊಬ್ಬ ಒಂದು ಸಮಾಚಾರವನ್ನು ತಿಳಿಸಿದ:\break ‘ನೋಡಯ್ಯಾ, ಆ ಮೇಲು ಮನೆಯ ಪೆದ್ದನನ್ನು ಮೊನ್ನೆ ಪೇಟೆ ಕಡೆಯಿಂದ ಬರುವಾಗ ಒಂದು ನಾಗರಹಾವು ಕಡಿಯಿತಂತೆ. ಹಾವು ಕಡಿದು ಅರ್ಧಘಂಟೆಯಲ್ಲೇ ಆತ ತೀರಿಕೊಂಡನಂತೆ. ಸ್ವಲ್ಪ ಜಾಗ್ರತೆಯಾಗಿರುವುದೊಳಿತು. ಇಲ್ಲೆಲ್ಲಾ ಹಾವುಗಳು ಸುತ್ತುತ್ತಿರುವುದನ್ನು ನಾನು ನೋಡಿದ್ದೇನೆ. ಹಾವಿನ ಕುರಿತಾದ ಭಯ ನನ್ನನ್ನು ಆವರಿಸತೊಡಗಿತು. ಅದೇ ದಿನ ಪೇಟೆಗೆ ಹೋಗಿ ಒಂದು ಟಾರ್ಚು ಕೊಂಡುಕೊಂಡೆ. ಸಂಜೆಯಾದ ಬಳಿಕ ಸದಾ ಟಾರ್ಚುಧಾರಿಯಾಗಿರುತ್ತಿದ್ದೆ. ದಾರಿಯಲ್ಲಿ ಹೋಗುವಾಗ ಎಲ್ಲಿಯಾದಾರೂ “ಸರ್ರ್​” ಎಂದು ಸದ್ದಾದರೆ, ಒಂದು ಅಡಿ ಎತ್ತರಕ್ಕೆ ನೆಗೆಯುತ್ತಿದ್ದೆ. ರಾತ್ರಿ ಮಲಗಿದಾಗ ನನ್ನ ಭಯಕ್ಕೆ ಮಿತಿ ಇರಲಿಲ್ಲ. ಗಂಟೆಗಂಟೆಗೂ ಎಚ್ಚರಗೊಂಡು ಸುತ್ತಲೂ ದೀಪಹಾಕಿ ನೋಡುತ್ತಿದ್ದೆ. ರಾತ್ರಿ ಮಲಗಿದಾಗ ಹಾವು ಕಾಲನ್ನು ಕಡಿದರೆ ಕೂಡಲೇ ಕಟ್ಟುಹಾಕಬಹುದು. ತಲೆಯನ್ನು ಕಡಿದರೆ ಕಟ್ಟು ಎಲ್ಲಿ ಹಾಕುವುದು? ಏನು ಗತಿ? ಎನ್ನುವ ಭಯ ಬೇರೆ. ಆರು ತಿಂಗಳ ಕಾಲ ಟಾರ್ಚು ಹಿಡಿದುಕೊಂಡು ಅಲೆದು ಒಮ್ಮೆಯೂ ನಾನು ಹಾವನ್ನು ನೋಡಲಿಲ್ಲ. ಆಗ ನನ್ನ ಅವಸ್ಥೆ ನೋಡಿ ನನಗೆ ನಾಚಿಕೆಯಾಯಿತು. ನಗು ಬಂತು! ಸುಮ್ಮನೇ ಭಯದ ಚಿತ್ರಗಳನ್ನು ನಾನು ನನ್ನ ಮನಸ್ಸಿನಲ್ಲಿ ತುಂಬಿಕೊಳ್ಳುತ್ತಿದ್ದೇನಲ್ಲ ಎಂದು ಅರಿತೆ. ನನ್ನ ಯಾವತ್ತೂ ಭಯಗಳು ವಾಸ್ತವಿಕವಾಗಿ ಘಟಿಸಲಿಲ್ಲ–ಎಂಬುದನ್ನು ನೋಡಿದಾಗ ನಾನೆಂಥ ಹುಚ್ಚುತನ ಮಾಡಿದೆ ಎನ್ನಿಸಿತು.’

\newpage

ಅಯ್ಯೋ, ಅನ್ನದಲ್ಲಿ ಕಲ್ಲು ಎಂದೊಡನೆ ಅನ್ನದಲ್ಲಿ ಎಲ್ಲವೂ ಕಲ್ಲಲ್ಲ! ದುರ್ಘಟನೆಗಳು ನಡೆಯುತ್ತಿವೆ ಎಂದೊಡನೆ ಎಲ್ಲವೂ ದುರ್ಘಟನೆಗಳಲ್ಲ! ಈ ವಿವೇಕ, ವಿವೇಚನೆ ನಮ್ಮನ್ನು ಕಾಲ್ಪನಿಕ ಭಯದಿಂದ ಮುಕ್ತರನ್ನಾಗಿಸುತ್ತದೆ.

~\\[-1.4cm]


\section*{ತಂಗಮ್ಮನ ತಳಮಳ}

\vskip -5pt\addsectiontoTOC{ತಂಗಮ್ಮನ ತಳಮಳ}

ಎಂಥದೋ ಭಯದ ಕಲ್ಪನೆಯು ತಂಗಮ್ಮನನ್ನು ಎರಡು ವರ್ಷಗಳ ಕಾಲ ದಂಗುಬಡಿಸಿತ್ತು. ಆಕೆ ಎರಡು ಮಕ್ಕಳ ತಾಯಿ. ಮಕ್ಕಳ ರಕ್ಷಣೆಯನ್ನು ಕುರಿತು ಅವಳಿಗೆ ಅತ್ಯಧಿಕ ಚಿಂತೆ. ಮಕ್ಕಳು ಶಾಲೆಗೆ ಹೊರಡುವುದಕ್ಕೆ ಮೊದಲು ಮೂರು ಬಾರಿ ಪಠಿಸುತ್ತಾಳೆ. ‘ರಸ್ತೆಯ ಬದಿಯಲ್ಲೇ ಹೋಗಿ, ಹಾಂ! ಮೋಟಾರಡಿ ಬಿದ್ದೀರಿ ಜಾಗ್ರತೆ! ಜಾಗ್ರತೆ’ ಇತ್ಯಾದಿ. ಕ್ಷಣಕ್ಷಣವೂ ಆಕೆಗೆ ‘ಮಕ್ಕಳು ಮೋಟಾರಡಿ’ ಈ ಚಿತ್ರವೇ ಇದಿರುಬರುತ್ತದೆ. ಅಡುಗೆ ಮನೆಯಿಂದ ಉದ್ವೇಗದಿಂದ ಓಡಿ ಬಂದು ರಸ್ತೆಯಲ್ಲಿ ನಿಂತು ನೋಡುತ್ತಿರುತ್ತಾಳೆ. ಅದರ ಪ್ರತಿಫಲ–ಒಂದು ದಿನ ಸಾರಿಗೆ ಉಪ್ಪಿಲ್ಲ! ಇನ್ನೊಂದು ದಿನ ಸಾಂಬಾರಿಗೆ ಖಾರ ಹಾಕಿಲ್ಲ! ಎಲ್ಲ ಕೆಲಸಗಳೂ ಅಸ್ತವ್ಯಸ್ತ! ವಿಪರೀತ ಮರೆವು! ದಿನವೂ ಅದೇ ಕತೆ.

ಆಕೆಯ ಪತಿ ಒಮ್ಮೆ ಕೇಳಿದ: ‘ನೀನು ನಿನ್ನ ತಲೆ ಎಲ್ಲಿ ಇರಿಸಿದ್ದೀಯ! ಕತ್ತಿನ ಮೇಲಿಲ್ಲವೆಂದು ಕಾಣುತ್ತದೆ. ಇಲ್ಲವಾದರೆ ಈ ಎರಡು ವರ್ಷಗಳಲ್ಲಿ ಎಷ್ಟು ಮಕ್ಕಳು ಸಾಯಬೇಕಾಗಿತ್ತು ಮೋಟಾ ರಡಿ ಬಿದ್ದು? ಸುಮ್ಮನೆ ತಲೆ ಬಿಸಿ ಮಾಡಿಕೊಳ್ಳುತ್ತಿ ಅಷ್ಟೆ!’ ಆಗ ‘ಸಾಮಾನ್ಯವಾಗಿ ಹಾಗಾಗುವುದಿಲ್ಲ’ ಎಂದು ಆಕೆ ಯೋಚಿಸಿ ಕ್ರಮೇಣ ಸಮಾಧಾನ ತಂದುಕೊಂಡಳು.

ಒಂದೆರಡು ದಿನಗಳ ಹಿಂದೆ, ರಾತ್ರಿ ಬಸ್ಸಿನಲ್ಲಿ ಪ್ರಯಾಣ ಮಾಡುವವರನ್ನು ಕಳ್ಳರು ದರೋಡೆ ಮಾಡಿದ ವಿಚಾರ ಪತ್ರಿಕೆಯಲ್ಲಿ ಓದಿರುತ್ತೀರಿ ಅಥವಾ ಯಾರಿಂದಲೋ ಕೇಳಿರುತ್ತೀರಿ. ಈ ದಿನ ನೀವು ಅದೇ ಮಾರ್ಗವಾಗಿ ರಾತ್ರಿ ಪ್ರಯಾಣ ಮಾಡಬೇಕಾಗಿದೆ. ಅದೋ! ಕಳ್ಳರ ಚಿತ್ರ ನಿಮ್ಮ ಮನಸ್ಸಿನ ಪಟಲದಲ್ಲಿ ಕಾಣಲು ಆರಂಭವಾಗಿದೆ. ಪತ್ರಿಕೆಗಳಲ್ಲಿ ಕಣ್ಣಿಗೆ ಪಟ್ಟಿ ಕಟ್ಟಿಕೊಂಡು ಕೈಯಲ್ಲಿ ಪಿಸ್ತೂಲು ಹಿಡಿದು ದಾರಿಗರನ್ನು ಗದರಿಸುವ ಚಿತ್ರವನ್ನು ಎಂದೋ ಗಮನಿಸಿದ್ದೀರಿ. ಅವು ಈಗ ಜೀವಂತವಾಗಿ ನಿಮಗೆ ಕಾಣಿಸತೊಡಗಿವೆ! ಒಂದು ವೇಳೆ ಕಳ್ಳರು ಎದುರಾದರೆ ಕೈಯಲ್ಲಿದ್ದ ಹಣವನ್ನು ಹೇಗೆ ಸುರಕ್ಷಿತವಾಗಿಟ್ಟುಕೊಳ್ಳುವುದೆಂದು ಯೋಚಿಸುತ್ತೀರಿ. ಇಡೀ ರಾತ್ರಿ ಜಾಗರಣೆ ಮಾಡುತ್ತೀರಿ. ಎದೆ ಬಡಿತ ಹೆಚ್ಚುತ್ತಲಿರುತ್ತದೆ. ಆಗಾಗ ಇದೆಲ್ಲಾ ಭ್ರಾಂತಿ ಎಂದು ಕೊಂಡರೂ ಭೀತಿ ನಿಮ್ಮನ್ನು ಬಿಟ್ಟಿರುವುದಿಲ್ಲ. ಯಾವ ತೊಂದರೆಗಳಿಲ್ಲದೆ ಬಸ್ಸು ಹೋಗಬೇಕಾದ ಜಾಗವನ್ನು ತಲುಪಿದಾಗ ನೀವು ನಿಟ್ಟುಸಿರುಬಿಡುತ್ತೀರಿ.

ಹೀಗೆ ಹಲವರು ಕಾಲ್ಪನಿಕ ಭಯದೊಂದಿಗೆ ಗುದ್ದಾಡಿಯೇ ಕಂಗಾಲಾಗುತ್ತಾರೆ.


\section*{ಆಘಾತದ ಅನಾಹುತ}

\addsectiontoTOC{ಆಘಾತದ ಅನಾಹುತ}

ಆರೋಗ್ಯವಂತನಾದ ಯುವಕನಾತ, ಯಾವ ತರದ ದೈಹಿಕ ಕುಂದುಕೊರತೆಗಳಿರಲಿಲ್ಲ. ಸಾಮಾನ್ಯವಾಗಿ ಮಾತನಾಡುವಾಗ ಉಗ್ಗುತ್ತಿರಲಿಲ್ಲ. ಆದರೆ ಜನರ ಗುಂಪನ್ನು ಕಂಡಾಗ ಮಾತ್ರ ಮಾತು ಸಂಪೂರ್ಣವಾಗಿ ನಿಂತುಹೋಗುತ್ತಿತ್ತು. ಮೊದಮೊದಲು ಯಾರಿಗೂ ಈ ವಿಚಿತ್ರ ವರ್ತನೆಯ ಕಾರಣವನ್ನು ಕಂಡು ಹಿಡಿಯಲಾಗಲಿಲ್ಲ. ಬಳಿಕ ತಜ್ಞರು ಅವನನ್ನು ವಶ್ಯಸುಪ್ತಿಗೆ ಒಳಪಡಿಸಿ, ಅವನ ಸುಪ್ತಮನಸ್ಸಿನ ಆಳದಲ್ಲಿ ಆ ತೊಂದರೆಯ ಮೂಲವನ್ನು ಕಂಡುಕೊಂಡರು. ಮೂರು ವರ್ಷ ವಯಸ್ಸಿನ ಬಾಲಕನಾಗಿದ್ದಾಗ, ಒಮ್ಮೆ ತಾಯಿಯು ಅವನ ಕೈ ಹಿಡಿದುಕೊಂಡು ಕರೆದೊಯ್ಯು\-ತ್ತಿದ್ದಳು. ರಸ್ತೆ ತಿರುಗುವಲ್ಲಿ ಒಂದು ಮೋಟಾರ್ ಕಾರು ಇನ್ನೊಂದು ಕಾರಿಗೆ ಡಿಕ್ಕಿ ಹೊಡೆದು ಬೆಂಕಿ ಹೊತ್ತಿ\-ಕೊಂಡಿತು. ಕಾರಿನೊಳಗಿದ್ದ ಜನರನ್ನುಳಿಸಲು ಸುತ್ತಮುತ್ತಲಿನ ಜನ ಧಾವಿಸಿ\-ಬಂದರು. ಅಲ್ಲಿ ಕೂಗಾಟ ಆರ್ತನಾದ, ಗದ್ದಲ ಮತ್ತು ಗಡಿಬಿಡಿಯ ವಾತಾವರಣ ಉಂಟಾಯಿತು. ಈ ಘಟನೆ ಬಾಲಕನ ಕೋಮಲ ಮನಸ್ಸಿನಲ್ಲಿ ತೀವ್ರತರದ ಭಯದ ಆಘಾತವನ್ನು ಉಂಟುಮಾಡಿತು. ಮುಂದೆ ಜನರ ಗುಂಪನ್ನು ಕಂಡಾಗಲೆಲ್ಲ ಅವನು ಮಾತಿನ ಶಕ್ತಿಯನ್ನೇ ಕಳೆದುಕೊಂಡು ಮೂಕ\-ನಾಗು\-ತ್ತಿದ್ದ. ತಜ್ಞರು ಅವನನ್ನು ವಶ್ಯಸುಪ್ತಿಗೊಳಪಡಿಸಿ ಭೀತಿಯ ಸಂಸ್ಕಾರವನ್ನು ಮನಸ್ಸಿನಿಂದ ತೆಗೆದುಹಾಕಲು ಯತ್ನಿಸಿದರು. ಹೊಸ ಭರವಸೆಯನ್ನು ಮೂಡಿಸಿದರು. ಫುಟ್​ಬಾಲ್ ಪಂದ್ಯ, ಸರ್ಕಸ್ ಇವೇ ಮೊದಲಾದ ಸಂತೋಷವನ್ನುಂಟುಮಾಡುವ ಘಟನೆಗಳನ್ನು ವೀಕ್ಷಿಸಲು ಕರೆ\-ದೊಯ್ದರು. ಜನರ ಗುಂಪು ಇರುವಲ್ಲಿ ಭಯ ಮಾತ್ರವೇ ಅಲ್ಲ, ಆನಂದವೂ ಇದೆ ಎಂಬುದನ್ನು ಅವನ ಸುಪ್ತಮನಸ್ಸು ಸ್ವೀಕರಿಸಿತು. ಆಗ ಯುವಕ ಚೆನ್ನಾಗಿಯೆ ಮಾತನಾಡತೊಡಗಿದ.

ತರುಣಿಯೊಬ್ಬಳು ವಿಪರೀತ ಉಗ್ಗಿನ (ತೊದಲು) ಬಾಧೆಯಿಂದ ಸಂಕಟಪಡುತ್ತಿದ್ದಳು. ಈ ತೊಂದರೆಯ ಮೂಲ ಕಾರಣವನ್ನು ಮನೋವಿಜ್ಞಾನಿಗಳು ಅವಳ ಸುಪ್ತಮನಸ್ಸಿನಲ್ಲಿ ಗುರುತಿಸಿದರು. ಚಿಕ್ಕವಳಾಗಿದ್ದಾಗ ತನ್ನ ತಂದೆತಾಯಿ ಜಗಳವಾಡುತ್ತ ಪರಸ್ಪರ ದೂಷಣೆ, ದೋಷಾರೋಪಣೆ ಮಾಡುತ್ತಿರುವುದನ್ನೂ, ಒಮ್ಮೆ ತಂದೆಯು ತಾಯಿಯನ್ನು ಕ್ರೂರವಾಗಿ ಹೊಡೆದುದನ್ನೂ ನೋಡಿ ಭಯಗ್ರಸ್ತಳಾಗಿದ್ದಳು. ಆ ಭಯವು ಅವಳ ಸುಪ್ತಮನಸ್ಸಿನಲ್ಲಿ ಆಘಾತವನ್ನು ಉಂಟುಮಾಡಿಬಿಟ್ಟಿತ್ತು. ಆ ಭಯದ ಸಂಸ್ಕಾರವನ್ನು ಅಳಿಸಿದ ಮೇಲೆ ಆಕೆ ಚೆನ್ನಾಗಿ ತಡೆಯಿಲ್ಲದೆ ಮಾತನಾಡತೊಡಗಿದಳು.

ಮೇಲಿನ ಘಟನೆಯ ಅರ್ಥವ್ಯಾಪ್ತಿಯನ್ನು ಆ ಘಟನೆಗೇ ಸೀಮಿತಗೊಳಿಸದೆ ಯೋಚಿಸಿ ನೋಡಬೇಕಾಗಿದೆ. ಎಳವೆಯಲ್ಲೇ ನಮ್ಮ ದೇಶದ ಮಕ್ಕಳ ಬದುಕಿನಲ್ಲಿ ಹಿರಿಯ ಹಾಗೂ ಪರಿಸರದ ಪ್ರಭಾವದಿಂದ, ಎಂಥ ಸನ್ನಿವೇಶ, ಸಮಸ್ಯೆಗಳು ಎದುರಾಗುತ್ತವೆ ಎಂಬುದನ್ನು ಈ ಘಟನೆಯ ಬೆಳಕಿನಲ್ಲಿ ಕಲ್ಪಿಸಿಕೊಳ್ಳಬಹುದು. ಬೇರೆ ಬೇರೆ ಜಾತಿಮತ, ಕುಲಗೋತ್ರ, ಸಂಸ್ಕೃತಿ ಸಂಪ್ರದಾಯಗಳಿಗೆ ಸೇರಿದ ಕುಟುಂಬಗಳಿಂದ ಬರುತ್ತಿರುವ ನಮ್ಮ ದೇಶದ ಮಕ್ಕಳ ಬದುಕು ಎಷ್ಟೊಂದು ಸಂಕೀರ್ಣ ಸಮಸ್ಯೆಗಳಿಂದ ಕೂಡಿದೆ! ನೇರವಾಗಿ ಪ್ರಸ್ತುತ ವಿಷಯವಾದ ‘ಭಯ’ಕ್ಕೆ ಸಂಬಂಧಿಸಿರ\-ದಿದ್ದರೂ, ಮಕ್ಕಳ ಭಾವಜೀವನ ಹಾಗೂ ಶಿಕ್ಷಣಕ್ಕೆ ಸಂಬಂಧಿಸಿದ ವಿಚಾರವಾದುದರಿಂದ ಇಲ್ಲಿ ಇದನ್ನು ಕುರಿತು ಚಿಂತನೆ ನಡೆಸುವುದು ಯುಕ್ತ. ಮಗು ಬೆಳೆಯುತ್ತ ಹೋದಂತೆ, ಮನೆ, ಶಾಲೆ ಮತ್ತು ಸಮಾಜ–ಈ ಮೂರು ಕ್ಷೇತ್ರಗಳಲ್ಲಿ ಯುಕ್ತವಾಗಿಯೋ, ಅಯುಕ್ತವಾಗಿಯೋ ತನ್ನ ಸಂಬಂಧವನ್ನು ಬೆಳೆಸಿಕೊಳ್ಳುತ್ತದೆ. ಮೊದಲಿಗೆ ಮನೆಯೇ, ಬದುಕಿನಲ್ಲಿ ಅತ್ಯಂತ ಪ್ರಭಾವ ಶಾಲಿಯಾದ ಪರಿಣಾಮವನ್ನುಂಟುಮಾಡುವ ತಾಣ. ಪರಸ್ಪರ ಹೊಡೆದಾಟ, ದೋಷಾರೋಪಣೆಗಳಿಂದ ಛಿದ್ರವಾದ ಮನೆಯಲ್ಲಿ–ಪ್ರೀತಿ, ಸಹನೆ, ಅನುಕಂಪೆಯೇ ಮೊದಲಾದ ಗುಣಗಳನ್ನು ಕಂಡರಿಯದ ಮಗುವಿನ ಭವಿಷ್ಯ ಎಂತಾದೀತು? ಅದರ ಮಾನಸಿಕ ಆರೋಗ್ಯಕ್ಕೆ ಎಂಥ ಆಘಾತ ಉಂಟಾದೀತು? ಅದರ ವ್ಯಕ್ತಿತ್ವದ ಬೆಳವಣಿಗೆಗೆ ಎಂಥ ವಿಘ್ನ ಒದಗೀತು?–ಎಂಬುದನ್ನು ಕಲ್ಪಿಸಿಕೊಳ್ಳಲು ಕಷ್ಟವಿಲ್ಲ.

ಮನೆಯಲ್ಲೂ, ಶಾಲೆಯಲ್ಲೂ, ನೆರೆಕೆರೆಯಲ್ಲೂ ಬಾಲಕನೊಬ್ಬ ಅವಹೇಳನ, ತಿರಸ್ಕಾರಗಳಿಗೆ ಗುರಿಯಾಗುತ್ತಾನೆ. ದುಷ್ಟರ ಸಂಗದಲ್ಲಿ ಸಿಲುಕಿ, ನಾನಾ ರೀತಿಯ ಹಿಂಸೆಗಳನ್ನು ಅನುಭವಿಸುತ್ತಾನೆ. ಅತ್ಯಾಚಾರಕ್ಕೊಳಗಾಗುತ್ತಾನೆ. ಆಗ ಆ ಘಟನೆಗಳ ಸ್ಮೃತಿಚಿತ್ರವು ವಿಶಿಷ್ಟ ನೋವು, ಸಂಕಟಗಳ ಭಾವನೆಯೊಂದಿಗೆ ಕೂಡಿಕೊಂಡು, ಆತನ ಮನಸ್ಸಿನಲ್ಲಿ ಹುದುಗಿರುತ್ತದೆ. ತಾನು ತಿರಸ್ಕೃತ, ತನಗೆ ಯಾರ ಆದರಣೆಯೂ ಇಲ್ಲ, ತಾನು ಯಾರಿಗೂ ಬೇಕಾದವನಲ್ಲ, ಮಾಡಿದ ತಪ್ಪುಗಳೂ ಬೇಕಾದಷ್ಟಿವೆ. ಯಾವ ಒಳ್ಳೆಯ ಕಾರ್ಯವನ್ನೂ ಕೈಗೊಂಡು ಮುಂದುವರಿಯಲು ನನ್ನಿಂದ ಅಸಾಧ್ಯ–ಇಂಥ ಭಾವನೆಗಳು ಅವನ ಸರಿಯಾದ ಬೆಳವಣಿಗೆಗೆ ಬಾಧಕವಾಗುತ್ತವೆ.\break ಕೀಳರಿಮೆಯಿಂದ ಇತರರೊಂದಿಗೆ ಬೆರೆತು ತನ್ನ ನೋವನ್ನು ಹೇಳಿಕೊಳ್ಳಲೂ ಸಾಧ್ಯವಿಲ್ಲ. ಭವಿಷ್ಯ\-ದಲ್ಲೂ ನಾನಾ ತೆರನಾದ ತಪ್ಪು ವರ್ತನೆಗಳು ಮರುಕಳಿಸಬಹುದೆಂಬ ಭಯ. ಒಂದು ತೆರನಾದ ಅಸಹಾಯಕತೆ, ಆಶಾಭಂಗ, ಪಾಪಪ್ರಜ್ಞೆ, ಕೋಪ, ಸೇಡಿನ ಮನೋಭಾವ ಆತನ ಮನಸ್ಸನ್ನು ತುಂಬಿರುತ್ತವೆ. ಇಂಥವರೇ ಮುಂದೆ ಸಮಾಜಕಂಟಕರಾಗುತ್ತಾರೆ.


\section*{ಇಂಥವರೂ ಇದ್ದಾರೆ!}

\addsectiontoTOC{ಇಂಥವರೂ ಇದ್ದಾರೆ!}

ತಪ್ಪು ಮಾಡಿದಾಗ ಮಕ್ಕಳನ್ನು ಗದರಿಸಿ ಬೆದರಿಸಿ, ಹೊಡೆದುಬಡಿದು, ಬೈದು ಭಂಗಿಸಿ ಸರಿಪಡಿಸುತ್ತೇವೆಂದು ಹೊರಡುವ ತಂದೆತಾಯಂದಿರು ಕೆಲವರು. ಮಕ್ಕಳ ಅರ್ಥಹೀನ ಬೇಡಿಕೆಗಳನ್ನು ಪೂರೈಸಿ ಅವರನ್ನು ಅತಿ ಮುದ್ದಿನಿಂದ ಬೆಳೆಸುವವರು ಕೆಲವರು. ಮಕ್ಕಳನ್ನು ಶಾಲೆಗೆ ಸೇರಿಸಿದ ಬಳಿಕ ತಂದೆತಾಯಂದಿರ ಕೆಲಸ ಮುಗಿಯಿತೆಂದು ತಿಳಿಯುವ, ಮಕ್ಕಳನ್ನು ಅವರ ಪಾಡಿಗೆ ಬಿಡುವ ಜಾತಿಗೆ ಸೇರಿದ ಬಹುಸಂಖ್ಯೆಯ ಹಲವರು. ಕೆಲವೊಮ್ಮೆ ಅತಿ ಪ್ರೀತಿಯನ್ನು ತೋರಿಸುತ್ತ,\break ಬೇಕೆಂದಾಗ ಅವರಿಗೆ ಕಾಸನ್ನು ನೀಡುತ್ತ, ಇನ್ನು ಕೆಲವೊಮ್ಮೆ ಕೋಪೋದ್ರಿಕ್ತರಾಗಿ ವರ್ತಿಸತ್ತ, ಅಸ್ತವ್ಯಸ್ತವಾಗಿ ನಡೆದುಕೊಂಡು, ಮಕ್ಕಳಿಗೆ ಬಿಡಿಸಲಾಗದ ಸಮಸ್ಯೆಯಾಗುವವರು ಕೆಲವರು, ಬುದ್ಧಿವಂತರಾದ, ಜಾಣ ಮಕ್ಕಳೊಂದಿಗೆ ತಮ್ಮ ಮಗುವನ್ನು ಹೋಲಿಸಿ, ಅಪಹಾಸ್ಯ ಮಾಡಿ, ಮನನೋಯಿಸಿ, ಆತ್ಮವಿಶ್ವಾಸಕ್ಕೆ ಆಘಾತವನ್ನುಂಟುಮಾಡುವವರು ಹಲವರು. ಮಕ್ಕಳ ಎದುರಿಗೆ ಹಿರಿಯರನ್ನೂ, ಗೌರವಾರ್ಹ ವ್ಯಕ್ತಿಗಳನ್ನೂ, ಅಧ್ಯಾಪಕರನ್ನೂ ನಿಂದಿಸುತ್ತ, ಅವರ ದೋಷಗಳನ್ನು ವರ್ಣಿಸುವವರು ಹಲವರು. ತಮ್ಮ ಮಕ್ಕಳಿಂದ ಅವರ ಸಾಮರ್ಥ್ಯಕ್ಕೆ ಮೀರಿದ ಪ್ರತಿಭೆಯನ್ನು ನಿರೀಕ್ಷಿಸಿ ನಿರಾಶಾವಾದಿಗಳಾಗಿ ಬೈಯುವವರು ಹಲವರು. ಇದರೊಂದಿಗೆ ತಂದೆ ತಾಯಿಗಳು ಬೆಳೆಸಿಕೊಂಡ ಮೂಢನಂಬಿಕೆ, ಮಲತಾಯಿ ಇದ್ದರೆ ಆಕೆಯ ದೃಷ್ಟಿಕೋನ, ಮತಾಚಾರಗಳು, ಜಾತಿ ಮತಗಳಿಗೆ ಸೇರಿದ ಕೀಳರಿಮೆಗಳು, ಮೇಲರಿಮೆ, ಹಮ್ಮು ಬಿಮ್ಮುಗಳು, ರಾಜಕೀಯದ ರಾಗದ್ವೇಷಗಳು–ಇವುಗಳೆಲ್ಲ ಮಕ್ಕಳ ಮನಸ್ಸಿಗೆ ಚಿತ್ರವಿಚಿತ್ರರೂಪವನ್ನು ಕೊಡುವುದು\break ಸ್ವಾಭಾವಿಕವಷ್ಟೆ.

‘ಸುಸಂಸ್ಕೃತ ಸಜ್ಜನರ ಗೃಹಕ್ಕೆ ಸಮಾನವಾದ ಶಾಲೆ ಇನ್ನೊಂದಿಲ್ಲ. ಸತ್ಯಸಂಧರೂ, ಗುಣ\-ವಂತರೂ ಆದ ತಂದೆತಾಯಂದಿರಿಗೆ ಸಮನಾದ ಅಧ್ಯಾಪಕರು ಇನ್ನಿಲ್ಲ. ಆಧುನಿಕ ಹೈಸ್ಕೂಲುಗಳಲ್ಲಿ ಪಡೆಯುವ ಶಿಕ್ಷಣ ಹಳ್ಳಿಗರ ಮೇಲಿರುವ ಒಂದು ಭರಿಸಲಾಗದ ಹೊರೆ. ಹಳ್ಳಿಗರ ಮಕ್ಕಳು ಸಂಸ್ಕಾರ ನಿರ್ಮಾಣದ ಈ ತರಬೇತಿಯನ್ನು ಶಾಲೆಯಲ್ಲಿ ಪಡೆಯಲು ಸಾಧ್ಯವೇ ಇಲ್ಲ. ಒಳ್ಳೆಯ ಮನೆಯ ಪರಿಸರದಲ್ಲಿ ಅದನ್ನು ಅವರು ಪಡೆದರಾದರೆ ಅದರ ಪರಿಣಾಮ ಅವರ ಬದುಕಿನಿಂದ ಮಾಸುವುದಿಲ್ಲ’\footnote{\engfoot{There is no school equal to decent home and no teachers equal to honest and virtuous parents. Modern high school eduaction is a dead weight on the villagers. Their children will never be able to get it, and thank God they will never miss it if they have the training of the decent home.}} ಎಂದರು ಗಾಂಧೀಜಿ.

ಇಂಥ ಗೃಹಗಳ ಅಭಾವವನ್ನು ಸ್ವಲ್ಪಮಟ್ಟಿಗಾದರೂ ಶಾಲೆಗಳು ದೂರಮಾಡಬೇಕೆಂದು ನಾವು ಇಂದು ಸ್ವತಂತ್ರ ಭಾರತದಲ್ಲಿ ಬಯಸುವುದು ಸ್ವಾಭಾವಿಕ. ನಮ್ಮ ದೇಶದಲ್ಲಿ ಸರಕಾರ ಕೋಟಿಗಟ್ಟಲೆ ಹಣವನ್ನು ವಿದ್ಯಾಭ್ಯಾಸಕ್ಕಾಗಿ ವಿನಿಯೋಗಿಸುತ್ತಿದೆ. ರಷ್ಯಾದಲ್ಲಿ ಮಕ್ಕಳ ಮಾನಸಿಕ ಆರೋಗ್ಯದ ರಕ್ಷಣೆಗೆ ಅಲ್ಲಿನ ಶಿಕ್ಷಣ ಕೇಂದ್ರಗಳು ನೀಡುವ ಆದ್ಯತೆ, ಮಹತ್ವ ಬೆರಗುಗೊಳಿಸುವ ಮಟ್ಟದ್ದಾಗಿದೆ. ಅದರಲ್ಲೂ, ಛಿದ್ರವಾದ ಕುಟುಂಬಗಳಿಂದ ಬಂದ ಹಿಂದುಳಿದ ಮಕ್ಕಳ ಮಾನಸಿಕ ತೊಂದರೆಯನ್ನು ನಿವಾರಿಸಲು ವಿಶೇಷ ಕಾಳಜಿಯನ್ನು ಅಲ್ಲಿನ ಸರಕಾರ ತೋರಿಸಿದೆ. ಶಿಕ್ಷಣ ಸಂಸ್ಥೆಗಳು ವಿಶೇಷ ನಿಯೋಜಿತ ಕಾರ್ಯಕ್ರಮಗಳನ್ನು ಇಟ್ಟುಕೊಂಡಿವೆ. ಆದರೆ ನಮ್ಮಲ್ಲಿ ಹೊಟ್ಟೆ ಪಾಡು, ಹಣಸಂಗ್ರಹವೇ ಜೀವಿತದ ಮಹೋದ್ದೇಶವಾಗಿರುವ ಉದ್ಯೋಗಸ್ಥರ ಇಂದಿನ (ಮಾನಸಿಕ) ಪರಿಸರದಲ್ಲಿ, ಈ ಸಮಸ್ಯೆಯನ್ನು ಅರ್ಥಮಾಡಿಕೊಳ್ಳುವವರಾರು? ಅರ್ಥಮಾಡಿ ಕೊಂಡರೂ ಅದರಿಂದ ಪಾರಾಗಲು ಸಹಾಯ ಮಾಡುವವರಾರು? ಆಗಲೇ ವೈವಿಧ್ಯಪೂರ್ಣ ಪರಂಪರೆ ಮತ್ತು ಪರಿಸರಗಳ ಒತ್ತಡಗಳಿಗೆ ಒಳಗಾದ ಮಕ್ಕಳ ಮಾನಸಿಕ ಆರೋಗ್ಯವನ್ನು ಕಾಪಾಡುವ ಕೆಲಸ ಅವರ ಶಾಲಾಜೀವನದ ಪ್ರಾಥಮಿಕ ಹಂತದಲ್ಲೇ ನಡೆಯಬೇಕಷ್ಟೆ. ಮುಖ್ಯವಾಗಿ ಹಿಂದುಳಿದ ಕುಟುಂಬಗಳಿಂದ ಬರುವ ಮಕ್ಕಳ ಹಿತಚಿಂತನೆಗೆ ವಿಶೇಷ ಗಮನವೀಯ ಬೇಕಾಗಿದೆ. ಈ ಕೆಲಸಕ್ಕೆ ಬೇಕಾಗುವ ಅರ್ಹತೆ ಇದು: ಮಕ್ಕಳ ಅಭ್ಯುದಯಕ್ಕಾಗಿ ತೀವ್ರ ಹಂಬಲಿಸುವ, ಮಕ್ಕಳನ್ನು ವಿಶೇಷ ಗಮನವಿತ್ತು ನೋಡಿಕೊಳ್ಳಬಲ್ಲ, ಕರ್ತವ್ಯನಿಷ್ಠೆ, ಪ್ರೀತಿ, ತಾಳ್ಮೆಗಳ ಮೂರ್ತಿವಂತ ಅಧ್ಯಾಪಕರು. ಇವರನ್ನು ತರುವುದೆಲ್ಲಿಂದ? ಗುಲಾಮರು ಶಿಕ್ಷಣ ನೀಡಿದರೆ ಗುಲಾಮರ ಒಂದು ಪಡೆಯೇ ನಿರ್ಮಾಣವಾಗುವುದಲ್ಲವೆ?


\section*{ಹೊಸ ದಿಗಂತ}

\addsectiontoTOC{ಹೊಸ ದಿಗಂತ}

ಇದನ್ನು ಸರಿಪಡಿಸಲು ಸಾಧ್ಯವಿಲ್ಲವೆ? ಖಂಡಿತ ಸಾಧ್ಯವಿದೆ.

ಬಾಲ್ಯದಿಂದ ನಿಷ್ಕಲ್ಮಷ ಪ್ರೀತಿಯನ್ನು ಕಂಡರಿಯದ ಜೀವ ಸಂಕಟಕ್ಕೆ ಸಿಲುಕಿದೆ. ದೀರ್ಘ ಕಾಲ ತಾಳ್ಮೆಯಿಂದ, ಪ್ರೀತಿಯೆಂಬ ಅಮೃತದ ತೀರ್ಥ ನೀಡಿದರೆ ಈ ರೋಗ ವಾಸಿಯಾಗುತ್ತದೆ. ಆ ಜೀವ ನಲಿವಿನಿಂದ ಕುಣಿದಾಡುತ್ತದೆ. ಅದರ ಬಾಳಿನಲ್ಲಿ ಹೊಸಯುಗಕ್ಕೆ ಆಗ ನಾಂದಿಯಾಗುತ್ತದೆ. ಆದರೆ ಪ್ರೀತಿ ಯಾವ ಮಾರುಕಟ್ಟೆಯಲ್ಲೂ ಲಭ್ಯವಿಲ್ಲ. ಅದು ಅತ್ಯಂತ ಅಮೂಲ್ಯವಾದ ವಸ್ತು.\break ಆದರೆ ಎಲ್ಲರಲ್ಲೂ ಅದು ಇದ್ದೇ ಇದೆ. ಎಲ್ಲೆಲ್ಲೂ ದೊರೆಯುತ್ತಿದೆ. ಅದು ಎಲ್ಲರಿಗೂ ಬೇಕು. ಅದನ್ನು ಸರಿಯಾದ ರೀತಿಯಲ್ಲಿ ಸಂಗ್ರಹಿಸಿ ನೀಡಿದರೆ ಎಲ್ಲ ರೋಗಗಳೂ ದೂರವಾಗುವುವು. ಜಗತ್ತು ನಂದನವನವಾಗುವುದು. ಅದನ್ನು ಸಂಗ್ರಹಿಸಿ ನೀಡುವ ವಿಧಾನ ಮುಂದಿನ ಅಧ್ಯಾಯ\-ದಲ್ಲಿದೆ.


\section*{ಕಲ್ಪನೆಗೆ ವಿಮುಕ್ತಿ}

\addsectiontoTOC{ಕಲ್ಪನೆಗೆ ವಿಮುಕ್ತಿ}

ತೀವ್ರತರದ ಭಾವನೆಯ ಸ್ಪಂದನದೊಂದಿಗೆ ಕಲ್ಪಿಸಿಕೊಳ್ಳುವಂಥ ಚಿತ್ರವು, ಅಂಥ ಅನುಭವಗಳನ್ನೇ ನಮ್ಮ ಬಳಿಗೆ ಎಳೆಯುವ ಶಕ್ತಿಯನ್ನು ಹೊಂದಿದೆ ಎಂಬುದನ್ನು ನಾವು ತಿಳಿದಿರಬೇಕು.\footnote{\engfoot{Throughout life, it is important to emphasise that experiences are drawn to us in direct accordance with the degree of intensity of the feelings we put behind our mental images.}\hfill\engfoot{ –Herold Sherman}}

ನೀವು ದುರ್ಘಟನೆಗೆ ಹೆದರುತ್ತಿದ್ದರೆ, ಅಂಥ ದುರ್ಘಟನೆಗಳನ್ನೇ ಕಣ್ಣೆದುರು ಕಾಣಲು ಪ್ರಾರಂಭಿಸುತ್ತಿದ್ದೀರಿ. ನೀವು ಹಾವಿಗೆ ಹೆದರುತ್ತಿದ್ದರೆ, ಹಾವು ನಿಮ್ಮ ಕಣ್ಣಿಗೆ ಆಗಾಗ ಕಾಣಿಸಿಕೊಳ್ಳುತ್ತದೆ. ನೀವು ರೋಗಕ್ಕೆ ಹೆದರುತ್ತಿದ್ದರೆ, ವಿಶಿಷ್ಟ ರೋಗಕ್ಕೆ ಆಮಂತ್ರಣ ನೀಡಿದಂತಾಗುತ್ತದೆ. ನೀವು ಸಾಲಕ್ಕೆ ಹೆದರುವಿರಾದರೆ, ಆಗಾಗ ಸಾಲ ತೆಗೆದುಕೊಳ್ಳಬೇಕಾದ ಪರಿಸ್ಥಿತಿ ಬರಬಹುದು.\break ನೀವು ಭೂತಪಿಶಾಚಿಗಳಿಗೆ ಹೆದರಿದರೆ ಅವು ನಿಮಗೆ ಆಗಾಗ ದರ್ಶನ ಕೊಡಲು ಪ್ರಾರಂಭಿಸುತ್ತವೆ. ನೀವು ಅಪಮಾನಕ್ಕೆ ಹೆದರುತ್ತಿದ್ದರೆ ಅಪಮಾನಕ್ಕೀಡಾಗುವಂತಹ ಸನ್ನಿವೇಶಗಳು ನಿಮ್ಮೆಡೆಗೆ ಸೆಳೆಯಲ್ಪಡುತ್ತವೆ. ಇವೆಲ್ಲವುಗಳಿಂದ ಪಾರಾ\-ಗಲು ನೀವು ಅಂತಹ ಕಲ್ಪನೆಗಳಿಗೇ ಅವಕಾಶ ಕೊಡಬಾರದು. ನೀವು ಧೈರ್ಯವಾಗಿರುವುದೇ ಅವುಗಳಿಂದ ಪಾರಾಗುವ ರಕ್ಷಣೋಪಾಯ.

ಹಾವಿನ ಅಪಾಯದಿಂದ ನಿಮ್ಮನ್ನು ಕಾಪಾಡಿಕೊಳ್ಳುವ ರೀತಿ ಬೇರೆ, ಹಾವಿನ ಕಲ್ಪನೆಯಿಂದ ಹೆದರಿಕೊಳ್ಳುವುದು ಬೇರೆ. ದುರ್ಘಟನೆಯಿಂದ ನಿಮ್ಮನ್ನು ನೀವು ರಕ್ಷಿಸಿಕೊಳ್ಳುವುದು ಬೇರೆ. ದುರ್ಘಟನೆಗಳನ್ನು ಚಿತ್ರಿಸಿಕೊಂಡು ಹೆದರುವುದು ಬೇರೆ. ರೋಗ ಬಾರದಂತೆ ರಕ್ಷಿಸಿಕೊಳ್ಳುವುದು ಬೇರೆ. ರೋಗಗಳನ್ನು ಕಲ್ಪಿಸಿಕೊಂಡು ಭಯಗ್ರಸ್ತರಾಗುವುದು ಬೇರೆ. ಅಪಮಾನದಿಂದ ಪಾರಾ\-ಗಲು ಹೇಗೆ ನಡೆದುಕೊಳ್ಳಬೇಕೆಂದು ತಿಳಿದುಕೊಳ್ಳುವುದು ಬೇರೆ, ಅಪಮಾನಕ್ಕೆ ಭಯಪಡುವುದು ಬೇರೆ. ಭೂತ, ಪ್ರೇತಗಳ ದುಷ್ಟ ಪ್ರಭಾವದಿಂದ ಮುಕ್ತರಾಗುವುದು ಬೇರೆ, ಅವುಗಳ ಭಯಾನಕ ಅಸಹ್ಯ ಚಿತ್ರಗಳನ್ನು ಕಲ್ಪಿಸಿಕೊಳ್ಳುತ್ತ ಭಯಪಡುವುದು ಬೇರೆ.

ಭಯಪಡುವುದೆಂದರೆ ಕಿಂಕರ್ತವ್ಯಮೂಢರಾಗಿ ಚಿಂತಿಸತೊಡಗುವುದು. ನಿಮ್ಮನ್ನು ನೀವು ರಕ್ಷಿಸಿಕೊಳ್ಳುವುದೆಂದರೆ ಬುದ್ಧಿಪೂರ್ವಕ ಪ್ರಯತ್ನದಲ್ಲಿ ತೊಡಗುವುದು. ಬುದ್ಧಿಯನ್ನು ಉಪ\-ಯೋಗಿಸಿ ಮಾಡುವ ಪ್ರಯತ್ನಗಳು ಕೇಡಿನಿಂದ ನಮ್ಮನ್ನು ರಕ್ಷಿಸುತ್ತವೆ. ಭಯದ ಕಲ್ಪನೆಗಳೂ, ಚಿಂತನೆಗಳೂ ನಮ್ಮನ್ನು ಸಂಕಟದಲ್ಲಿ ಸಿಲುಕಿಸುತ್ತವೆ.


\section*{ನಿರ್ಭೀತಿಯ ಚಾಂಪಿಯನ್ ಭೀತಿಯಿಂದ ಸತ್ತುಹೋದ!}

\addsectiontoTOC{ನಿರ್ಭೀತಿಯ ಚಾಂಪಿಯನ್ ಭೀತಿಯಿಂದ ಸತ್ತುಹೋದ!}

ನಿರ್ಭೀತಿಯ ಸ್ವಭಾವದ ವೈಜ್ಞಾನಿಕ ಮನೋವೃತ್ತಿಯ ಯುವಕನೊಬ್ಬನಿಗೆ ಆತನ ಸ್ನೇಹಿತರು ಒಂದು ಪಂಥಾಹ್ವಾನ ನೀಡಿದರು. ಅಮಾವಾಸ್ಯೆಯ ದಿನ ಮಧ್ಯರಾತ್ರಿಯ ಹೊತ್ತು ನಗರದ ಹೊರವಲಯದಲ್ಲಿರುವ ಸ್ಮಶಾನವನ್ನು ಪ್ರವೇಶಿಸಿ, ಮಧ್ಯಭಾಗದ ನೆಲದಲ್ಲಿ ಒಂದು ಗೂಟವನ್ನು ನೆಟ್ಟು ಹಿಂದಿರುಗಬೇಕೆಂದು ಅವರು ಪಣನೀಡಿದರು. ಯುವಕನಾದರೋ ಹಲವಾರು ಶ್ಮಶಾನಗಳಲ್ಲಿ ಸಂಚಾರ ಮಾಡಿ ಭೂತಗಳ ಬೇಟೆ ಮಾಡಹೊರಟವನೆ–ನಿರ್ಭೀತಿಯ ಚಾಂಪಿಯನ್ ಎಂದು ಹೆಸರುಗಳಿಸಿದವನು. ಆತ ಧೈರ್ಯದಿಂದಲೇ ಮುನ್ನುಗ್ಗಿದ್ದ. ಆತನನ್ನು ಗೆಳೆಯರು ಹಿಂಬಾಲಿಸುತ್ತಾ ಸ್ಮಶಾನದ ಗೇಟಿನಿಂದ ಹೊರಗಡೆ ಅನತಿ ದೂರದಲ್ಲಿ ನಿಂತು ಕೊಂಡರು. ಆತ ಒಳಗೆ ಪ್ರವೇಶ ಮಾಡಿದ. ಗೂಟವನ್ನು ನೆಲದಲ್ಲಿಟ್ಟು ಹೊಡೆಯುವ ಸದ್ದು ಕೇಳಿಸಿತು. ಸ್ವಲ್ಪ ಹೊತ್ತಿನ ನಂತರ ಎಲ್ಲವೂ ಮೌನ. ಹೊರಗಡೆ ನಿಂತ ಗೆಳೆಯರು ಅವನ ಧೈರ್ಯವನ್ನು ಮೆಚ್ಚಿಕೊಳ್ಳತೊಡಗಿದ್ದರು. ಅವನನ್ನು ಕಾಯುತ್ತ ಕುಳಿತ ಅವರಿಗೆ ದುರಂತದ ಆಶ್ಚರ್ಯಕರ ವಾರ್ತೆ ಕಾದಿತ್ತೆಂಬುದು ಆಗ ತಿಳಿದಿರಲಿಲ್ಲ. ಕೂಗಿ ಕರೆದರೂ ಅವನಿಂದ ಮರುದನಿ ಬರಲಿಲ್ಲ. ಮೂಡಣಬಾನಿನಲ್ಲಿ ಕೆಂಪಡರ\-ತೊಡಗಿತ್ತು. ಬೆಳಗಾಗುತ್ತಲೇ ಸ್ನೇಹಿತರೆಲ್ಲ ಸ್ಮಶಾನದೊಳಗೆ ಧಾವಿಸಿದರು. ತಮ್ಮ ಗೆಳೆಯ ಮಲಗಿಕೊಂಡಂತಿರುವುದನ್ನು ಕಂಡರು. ಇನ್ನೂ ಸಮೀಪಿಸಿ ಅವನನ್ನು ಸ್ಪರ್ಶಿಸಿ ಪರಿಶೀಲಿಸಿದಾಗ ಅವನು ಮರಗಟ್ಟಿ ಶವವಾಗಿರುವುದನ್ನು ಕಂಡರು. ಅವನ ಪ್ರಾಣಪಕ್ಷಿ ದೇಹಪಂಜರದಿಂದ ಹಾರಿಹೋಗಿ ಆಗಲೇ ಗಂಟೆಗಳಾಗಿದ್ದುವು!

ಅವನ ಮರಣದ ಏಕಮಾತ್ರ ಕಾರಣವನ್ನು ಅವರು ತಿಳಿದುಕೊಂಡರು. ಅವನು ಸತ್ತದ್ದು ಭಯದಿಂದಲೇ! ಗೂಟವನ್ನು ನೆಲದಲ್ಲಿ ನೆಡುವಾಗ, ತಾನು ಹೊದೆದುಕೊಂಡ ಶಾಲಿನ ಒಂದು ತುದಿಯನ್ನು ಅದರೊಂದಿಗೇ ಅರಿವಿಲ್ಲದೇ ಸೇರಿಸಿದ್ದ. ಗೂಟವನ್ನು ನೆಲದಲ್ಲಿ ಆಳವಾಗಿ ಇಳಿಯುವಂತೆ ಬಲವಾಗಿ ಹೊಡೆದು ಬೇಗ ಬೇಗನೇ ಹಿಂದಿರುಗುವ ವೇಳೆ, ಯಾರೋ ಶಾಲಿನ ತುದಿಯನ್ನು ಎಳೆದು ಹಿಡಿದಂತಾಯಿತು! ಭೂತವಲ್ಲದೆ ಇನ್ನಾರು ಆ ನಿರ್ಜನ ಪ್ರದೇಶದಲ್ಲಿ ಆತನನ್ನು ಎಳೆಯಬಲ್ಲರು? ಭಯದಿಂದ ಅಲ್ಲೇ ಕುಸಿದು ಬಿದ್ದ!

\newpage

ಅಯ್ಯೋ! ಶಾಲಿನ ತುದಿಯನ್ನು ಹಿಡಿದೆಳೆದವರಾರೆಂದು ಹಿಂದಿರುಗಿ ನೋಡಿದ್ದರೆ....ಭಯದ ಕಾರಣವನ್ನು ಕಂಡುಹಿಡಿದಿದ್ದರೆ....ಗೂಟದಲ್ಲಿ ಶಾಲು ಸಿಕ್ಕಿಕೊಂಡುದನ್ನು ಕಂಡಿದ್ದರೆ....ಅವನು ಬದುಕಿಕೊಳ್ಳುತ್ತಿದ್ದ!

ಭಯಕ್ಕೆ ಕಾರಣವಾದ ಕೆಲವೊಂದು ಸಂಗತಿಗಳ ಸ್ವರೂಪ ಸ್ವಭಾವಗಳನ್ನು ತಿಳಿದುಕೊಂಡು, ಅದರಿಂದ ಪಾರಾಗುವ ವಿಧಾನಗಳನ್ನು ಅನುಸರಿಸಬೇಕು. ಭಯವನ್ನುಂಟುಮಾಡುವ ಕೆಲವು ಸನ್ನಿವೇಶಗಳನ್ನು ಎದುರಿಸಬೇಕು. ಕೆಲವು ಭಯಗಳಿಗೆ ಕಾರಣವನ್ನು ಕಂಡುಹಿಡಿಯಬೇಕು. ಕೆಲವೊಂದು ಸಂದರ್ಭಗಳಲ್ಲಿ, ಸಕಲಭಯತಲ್ಲಣವನ್ನು ದೂರಮಾಡುವ ಅಭಯದಾತನಾದ ಭಗವಂತನನ್ನು ಪ್ರಾರ್ಥಿಸಬೇಕು. ಆತನಲ್ಲಿ ಪೂರ್ಣಶ್ರದ್ಧೆಯಿಂದ ಶರಣಾಗಬೇಕು.


\section*{ಭೂತಗಣದೊಡನೆ ಭೀತಿ ಇಲ್ಲದೆ ಮಾತಾಡಿದರು!}

\addsectiontoTOC{ಭೂತಗಣದೊಡನೆ ಭೀತಿ ಇಲ್ಲದೆ ಮಾತಾಡಿದರು!}

ದಕ್ಷಿಣೇಶ್ವರದಲ್ಲಿ ಭಗವಾನ್ ಶ‍್ರೀರಾಮಕೃಷ್ಣರ ತಪಸ್ಸಿನ ದಿವ್ಯ ತಾಣವಾದ ಪಂಚವಟಿ ಒಂದು ನಿರ್ಜನ ಪ್ರದೇಶ. ಅದ್ವೈತ ಸಾಧನೆಯ ರಹಸ್ಯವನ್ನು ಶ‍್ರೀರಾಮಕೃಷ್ಣರಿಗೆ ತಿಳಿಸ ಬಂದ ತೋತಾಪುರಿ, ಪಂಚವಟಿಯ ವೃಕ್ಷಗಳ ತಲದಲ್ಲಿ ಧುನಿ ಹೊತ್ತಿಸಿ ರಾತ್ರಿಯೆಲ್ಲ ಗಂಭೀರಧ್ಯಾನದಲ್ಲಿ ಮುಳುಗಿರುತ್ತಿದ್ದರು. ಒಂದು ದಿನ ಮಧ್ಯರಾತ್ರಿ ದಾಟಿರಬೇಕು. ಜಗತ್ತು ಮಲಗಿ ನಿದ್ರಿಸುತ್ತಿದೆ. ಅತ್ಯಂತ ನೀರವ ವಾತಾರವಣ. ತೋತಾಪುರಿಯವರು ಧುನಿಯ ಹತ್ತಿರ ಧ್ಯಾನಕ್ಕೆ ಕುಳಿತು ಕೊಂಡಿದ್ದರು. ಆ ಪ್ರಶಾಂತ ವಾತಾವರಣದಲ್ಲಿ ಇದ್ದಕ್ಕಿದ್ದಂತೆ ಮರದ ಕೊಂಬೆಗಳು ಅಲುಗಾಡ ತೊಡಗಿದವು. ಅವರ ಗಮನ ಆ ಕಡೆ ಹರಿಯಿತು. ದೀರ್ಘಕಾಯದ ವ್ಯಕ್ತಿಯೊಬ್ಬ ಮರದಿಂದ ಇಳಿದು ತೋತಾಪುರಿಯನ್ನೇ ದಿಟ್ಟಿಸಿ ನೋಡುತ್ತ ಧುನಿಯ ಹತ್ತಿರವೇ ಬಂದು ಕುಳಿತುಕೊಂಡ. ಆಶ್ಚರ್ಯವೆನಿಸಿದರೂ, ಸ್ವಲ್ಪವೂ ಭಯ ಪಡದೇ ‘ಯಾರು ನೀವು?’ ಎಂದು ಆಗಂತುಕ ವ್ಯಕ್ತಿಯನ್ನು ತೋತಾಪುರಿ ಕೇಳಿದರು. ‘ನಾನು ಭೈರವ. ಇಲ್ಲಿ ವೃಕ್ಷಗಳ ಮೇಲೆ ವಾಸಿಸುತ್ತಿದ್ದೇನೆ. ಭವತಾರಿಣಿಯ ದೇವಾಲಯವನ್ನು ನೋಡಿಕೊಳ್ಳುತ್ತಿದ್ದೇನೆ’ ಎಂದ ಆ ವ್ಯಕ್ತಿ. ಸ್ವಲ್ಪವೂ ವಿಚಲಿತರಾಗದೆ ತೋತಾಪುರಿ ಹೇಳಿದರು: ‘ಒಳ್ಳೆಯದು. ನೀವು ಮತ್ತು ನಾನು ಬೇರೆ ಬೇರೆ ಅಲ್ಲ. ನಾವಿಬ್ಬರೂ ಪರತತ್ತ್ವದ ಆವಿರ್ಭಾವಗಳು, ಇಲ್ಲೇ ಕುಳಿತು ಧ್ಯಾನ ಮಾಡಿ.’ ಆದರೆ ಆ ಮಾತನ್ನು ಕೇಳಿ ಭೈರವ ಗಟ್ಟಿಯಾಗಿ ನಕ್ಕು ಗಾಳಿಯಲ್ಲಿ ಮಾಯವಾದ. ತೋತಾಪುರಿ ಮರುಕ್ಷಣವೇ ನಿಶ್ಚಿಂತರಾಗಿ ಧ್ಯಾನಮಗ್ನರಾದರು. ಮರುದಿನ ಶ‍್ರೀರಾಮಕೃಷ್ಣರಿಗೆ ಹಿಂದಿನ ರಾತ್ರಿ ನಡೆದ ಘಟನೆಯನ್ನು ತಿಳಿಸಿದಾಗ ಅವರು ‘ಹೌದು, ಅವನು ಅಲ್ಲಿದ್ದಾನೆ. ನಾನು ಹಲವು ಬಾರಿ ಅವನನ್ನು ನೋಡಿ ಮಾತನಾಡಿದ್ದೇನೆ. ಕೆಲವೊಮ್ಮೆ ಮುಂದೆ ನಡೆಯಬಹುದಾದ ಘಟನೆಗಳನ್ನು ಸನ್ನೆಯ ಮೂಲಕ ಅವನು ತಿಳಿಸುತ್ತಿದ್ದ’ ಎಂದರು.

ಆ ಭೂತಗಣದ ಸ್ವರೂಪ, ಸ್ವಭಾವಗಳನ್ನು ತಿಳಿದಿದ್ದ ತೋತಾಪುರಿ ಭಯಾನಕವೆನಿಸುವ ಮೇಲಿನ ಘಟನೆಯನ್ನು ನಿರ್ಭಯವಾಗಿ ಎದುರಿಸಿದ ವಿಧಾನವನ್ನು ಪರಿಶೀಲಿಸಿ.

ತಮ್ಮ ಕಲ್ಪನೆಗಳಿಂದಲೇ ಅಜ್ಞಾತವಾಗಿ ವಿಚಿತ್ರ ಭೂತಗಳನ್ನು ಸೃಷ್ಟಿಸಿಕೊಂಡು ಅವುಗಳೊಡನೆ ಗುದ್ದಾಡಿ ಬಳಲುವ ಭ್ರಾಂತರಿದ್ದಾರೆ! ಮನೋವಿಜ್ಞಾನಿಗಳು ಇಂಥವರಲ್ಲಿ ಕೆಲವರಿಗೆ ಸರಿಯಾದ ಔಷಧ ನೀಡಬಲ್ಲರು.

ನಿಜವಾದ ಭೂತಗಣಗಳನ್ನು ಕಂಡರೂ ನಾವು ಹೆದರಬೇಕಿಲ್ಲ. ಅವುಗಳು ನಮ್ಮನ್ನು ಹೆದರಿಸುವುದಿಲ್ಲ. ಬಾಲ್ಯದಲ್ಲಿ ಕೇಳಿದ ಭೂತವನ್ನು ಕುರಿತ ಭೀತಿಯ ಸಂಸ್ಕಾರಗಳು ಎಚ್ಚೆತ್ತು ನಾವೇ ಹೆದರಿಕೊಳ್ಳುವಂತೆ ಮಾಡುತ್ತವೆ. ಇನ್ನೊಂದು ದೃಷ್ಟಿಯಿಂದಲೂ ಈ ಸಮಸ್ಯೆಯನ್ನು ಅರಿತು ಕೊಳ್ಳಬೇಕು. ದೇಹವನ್ನು ತ್ಯಜಿಸಿದ ನಂತರ ಸೂಕ್ಷ್ಮಸ್ಥಿತಿಯಲ್ಲಿರುವ ಕೆಲವು ಚೇತನಗಳೇ ಭೂತ ಪ್ರೇತಗಳು. ದೇಹಧಾರಿಗಳಾದ ವ್ಯಕ್ತಿಗಳನ್ನು ಕಂಡಾಗ ನಾವು ಹೆದರುವುದಿಲ್ಲ. ದೇಹವನ್ನು ಕಳೆದುಕೊಂಡ ವ್ಯಕ್ತಿಯನ್ನು ಸೂಕ್ಷ್ಮರೂಪದಲ್ಲಿ ಕಂಡು ಹೆದರುವುದೇಕೆ? ಭಯದ ಕಾರಣ ತಿಳಿಯದೆ ಸುಮ್ಮನೆ ಭಯದ ದವಡೆಯಲ್ಲಿ ಸಿಲುಕಿ ನಜ್ಜುಗುಜ್ಜಾಗುತ್ತೇವೆ. ಏನು? ಎಂಥದು? ಏಕೆ? ಹೇಗೆ? ಎಂಬ ಪ್ರಶ್ನೆಯನ್ನು ಕೇಳ ಹೋಗುವುದಿಲ್ಲ. ವಸ್ತುಸ್ಥಿತಿಯ ಅಜ್ಞಾನವೇ ನಮ್ಮ ಭಯಕ್ಕೆ ಕಾರಣವೆಂದಾಯಿತಲ್ಲವೇ? ಕೆಲವರೇನೋ ಭೂತ ಪ್ರೇತಗಳ ಭೀತಿಯಿಂದ ನರಳುವವರನ್ನು ‘ಮೆಂಟಲ್​’, ‘ಸೈಕಲಾಜಿಕಲ್​’, ‘ಹೆಪ್ಪುಗಟ್ಟಿದ ಮೂಢನಂಬಿಕೆ’ ಎಂದು ತಿರಸ್ಕರಿಸಬಹುದು. ಭೂತ ಪ್ರೇತಗಳ ಅಸ್ತಿತ್ವವನ್ನೇ ಅಲ್ಲಗಳೆಯಬಹುದು. ಬರಿಯ ಭ್ರಾಂತಿ ಎಂದು ಬೈಯಬಹುದು. ಅಂಥವರು ವೈಜ್ಞಾನಿಕ ಮನೋಭಾವದವರೆಂದು ಖ್ಯಾತಿ ಪಡೆದರೂ ಭೀತಿಯ ಸುಳಿಯಲ್ಲಿ ಸಿಲುಕಿದ ವ್ಯಕ್ತಿಗಳಿಗೆ ನೆರವು ನೀಡಲಾರರು! ಅವರಿಗೆ ಭೀತಿಯ ಮೂಲವನ್ನು ತಿಳಿಯುವ ಸಾಮರ್ಥ್ಯವಿಲ್ಲದಿದ್ದರೂ ತಾವು ಕಂಠಪಾಠ ಮಾಡಿದ ವಿಶಿಷ್ಟ ಪ್ರತಿಕ್ರಿಯೆಯನ್ನು ಉಚ್ಚರಿಸುತ್ತ ಇನ್ನೊಬ್ಬರ ಭೀತಿ ಅರ್ಥಹೀನವೆಂದು ಸಮರ್ಥಿಸುತ್ತಾರೆ. ಅಂಥವರ ಹತ್ತಿರ ರೋಗಿಗಳು ತಮ್ಮ ಅನುಭವವನ್ನೋ, ಸಂಕಟವನ್ನೋ ಮನಬಿಚ್ಚಿ ಹೇಳಿಕೊಳ್ಳಲು ಸಾಧ್ಯವೆ? ಭಯಕ್ಕೆ ಕಾರಣವಾದ ವಸ್ತುವಿನ ಸ್ವರೂಪ, ಸ್ವಭಾವಗಳನ್ನು ಸರಿಯಾಗಿ ತಿಳಿದುಕೊಂಡಾಗ ಅಸಂಗತ, ಅರ್ಥಹೀನ ಭಯ ದೂರವಾಗುತ್ತದೆ. ಭಯವನ್ನು ದೂರ ಮಾಡಲು ಏನು ಮಾಡಬೇಕೆಂಬುದು ಸರಿಯಾಗಿ ತಿಳಿಯುತ್ತದೆ.


\section*{ಪ್ರೇತಗಳ ಪೇಚಾಟ}

\addsectiontoTOC{ಪ್ರೇತಗಳ ಪೇಚಾಟ}

ಶ‍್ರೀರಾಮಕೃಷ್ಣರು ತಾವು ಪ್ರತ್ಯಕ್ಷವಾಗಿ ಪಡೆದ ವಿಚಿತ್ರವಾದ ಒಂದು ಅನುಭವವನ್ನು ಶಿಷ್ಯರಿಗೆ ಹೇಳಿದ್ದರು. ಸ್ವಾಮಿ ಶಾರದಾನಂದರು ತಮ್ಮ ‘ಶ‍್ರೀರಾಮಕೃಷ್ಣ ಲೀಲಾಪ್ರಸಂಗ’ ಎಂಬ ಗ್ರಂಥದಲ್ಲಿ ಆ ಅನುಭವದ ವಿವರಣೆಯನ್ನಿತ್ತಿದ್ದಾರೆ. ಒಮ್ಮೆ ಪರಮಭಕ್ತೆಯಾದ ಗೋಪಾಲನ ತಾಯಿಯ ಆಮಂತ್ರಣದ ಮೇರೆಗೆ ಅವರ ಮನೆಗೆ ಶ‍್ರೀರಾಮಕೃಷ್ಣರು ತಮ್ಮ ಶಿಷ್ಯರಾದ (ಮುಂದೆ ಬ್ರಹ್ಮಾನಂದರೆಂದು ಪ್ರಸಿದ್ಧರಾದ) ರಾಖಾಲರ ಜೊತೆಗೂಡಿ ಹೋಗಿದ್ದರು. ಮಧ್ಯಾಹ್ನ ಭೋಜನವಾದ ಮೇಲೆ ಅಲ್ಲೇ ಕೊಂಚ ಹೊತ್ತು ವಿಶ್ರಮಿಸುತ್ತಿದ್ದರು. ಸ್ವಲ್ಪ ಹೊತ್ತಿನಲ್ಲೇ ಒಂದು ತೆರನಾದ ದುರ್ವಾಸನೆ ಬರತೊಡಗಿತು. ಕೋಣೆಯ ಮೂಲೆಯಲ್ಲಿ ಎರಡು ರೂಪಗಳು ಕಾಣಿಸುತ್ತಿದ್ದವು. ಅವುಗಳ ರೂಪ ವಿಕಾರವೂ, ಅಸಹ್ಯಕರವೂ ಆಗಿತ್ತು. ಅವುಗಳ ಹೊಟ್ಟೆಯಿಂದ ಕರುಳು ಹೊರಗಡೆ ಜೋತಾಡುತ್ತಿತ್ತು. ಮೆಡಿಕಲ್ ಕಾಲೇಜುಗಳಲ್ಲಿ ಇರಿಸಿದ ಎಲುಬಿನ ಗೂಡಿನಂತೆಯೇ ಅವು ಕಾಣಿಸುತ್ತಿದ್ದವು. ಅವು ಅತ್ಯಂತ ದೀನತೆಯಿಂದ ಶ‍್ರೀರಾಮಕೃಷ್ಣರನ್ನು ಬೇಡಿ ಕೊಂಡವು–‘ನೀವೇಕೆ ಇಲ್ಲಿಗೆ ಬಂದಿದ್ದೀರಿ? ದಯವಿಟ್ಟು ಇಲ್ಲಿಂದ ಹೊರಟು ಹೋಗಿ. ನಿಮ್ಮ ಸನ್ನಿಧಿಯಲ್ಲಿ ನಮ್ಮ ದುಸ್ಥಿತಿಯ ಅರಿವುಂಟಾಗಿ ನಮಗೆ ನೋವಾಗುತ್ತದೆ.’ ಒಂದೆಡೆ ಆ ಪ್ರೇತಗಳು ಈ ರೀತಿ ಬೇಡಿಕೊಳ್ಳುತ್ತಿವೆ. ಇನ್ನೊಂದೆಡೆ ರಾಖಾಲ ನಿದ್ರೆ ಮಾಡುತ್ತಿದ್ದಾನೆ. ಆದರೂ ಅವುಗಳಿಗುಂಟಾದ ವ್ಯಥೆಯನ್ನು ಕಂಡು ನಾನು ನನ್ನ ಸಣ್ಣ ಚೀಲ ಹಾಗೂ ಕರವಸ್ತ್ರ ತೆಗೆದುಕೊಂಡು ಹೊರಡಲು ಸಿದ್ಧನಾದೆ. ಆಗ ಎಚ್ಚೆತ್ತ ರಾಖಾಲ ನನ್ನನ್ನು ಪ್ರಶ್ನಿಸಿದ. ‘ನೀವು ಎಲ್ಲಿಗೆ ಹೋಗುತ್ತಿದ್ದೀರಿ?’ ‘ಅನಂತರ ತಿಳಿಸುತ್ತೇನೆ’ ಎಂದು ನಾನು, ರಾಖಾಲನ ಕೈ ಹಿಡಿದುಕೊಂಡು ಕೆಳಗಿಳಿದು ಬಂದು, ಆಗ ತಾನೇ ಊಟ ಮುಗಿಸಿದ್ದ ಗೋಪಾಲನ ತಾಯಿಯಿಂದ ಬೀಳ್ಕೊಂಡೆ. ಅಲ್ಲಿಂದ ಹೊರಟು ದೋಣಿಯನ್ನು ಸೇರಿದೆ. ಆಗ ರಾಖಾಲನಿಗೆ ತಿಳಿಸಿದೆ: ‘ಅಲ್ಲಿ ಎರಡು ಪ್ರೇತಗಳಿದ್ದವು. ಬಿಳಿಯ ಜನರು ತಿಂದು ಎಸೆದ ಮೂಳೆಗಳನ್ನು ಆ ಪ್ರೇತಗಳು ಮೂಸುವುದರ ಮೂಲಕ ಆಹಾರ ಸೇವಿಸುತ್ತವೆ. ಆ\break ಮನೆಯಲ್ಲಿ ಒಂಟಿಯಾಗಿ ವಾಸಿಸಬೇಕಾಗಿದ್ದ ಆ ವೃದ್ಧ ತಾಯಿಗೆ ಆ ವಿಚಾರವನ್ನು ನಾನು\break ತಿಳಿಸಲಿಲ್ಲ.’

ಭೂತಪ್ರೇತಗಳಿಗೆ ಅವುಗಳದೇ ಆದ ಸಮಸ್ಯೆ ಸಂಕಟಗಳಿವೆ. ನಿಮ್ಮ ಸಹಾಯವನ್ನು ಯಾಚಿ\-ಸುವು\-ದಕ್ಕಾಗಿ ಅವುಗಳು ನಿಮಗೆ ಕಾಣಿಸಿಕೊಂಡಿರಬಹುದು. ಆಗ ನೀವು ‘ದೇವರೇ ಅವುಗಳನ್ನು ಪಾರುಮಾಡು’ ಎಂದು ಹೇಳುವುದನ್ನು ಬಿಟ್ಟು ಭಯದಿಂದ ದಿಕ್ಕೆಟ್ಟರೆ ಆ ಭೂತಗಳಿಗೆಷ್ಟು ಆಶಾ ಭಂಗವಾಗಬಹುದು! ಮನುಷ್ಯರಲ್ಲೂ ದುರ್ಜನರಿರುವಂತೆ, ಪ್ರೇತಗಳಲ್ಲೂ ಕೆಟ್ಟ ಕಾಮನೆಗಳಿಂದ ಕೂಡಿದ ಪ್ರೇತಗಳಿರಬಹುದು. ಅವುಗಳಿಗೆ ಹೆದರುವುದರಿಂದ ಸಮಸ್ಯೆ ಇನ್ನೂ ಉಲ್ಬಣವಾಗುವುದು. ಅವುಗಳನ್ನು ಎದುರಿಸಬೇಕು. ಎದುರಿಸಲು ಸಾಧ್ಯ. ಎದುರಿಸಲು ಉಪಾಯಗಳೂ ಇವೆ. ಆದುದರಿಂದ ಈ ವಿಷಯವಾಗಿ ನಾವು ಭಯಪಡಬೇಕಿಲ್ಲ; ಚಿಂತಿತರಾಗಬೇಕಿಲ್ಲ.


\section*{ಓಡಬೇಡಿ, ಎದುರಿಸಿ!}

\vskip -6pt\addsectiontoTOC{ಓಡಬೇಡಿ, ಎದುರಿಸಿ!}

ಸ್ವಾಮಿ ವಿವೇಕಾನಂದರು ಪರಿವ್ರಾಜಕರಾಗಿ ಸಂಚರಿಸುತ್ತಿದ್ದ ದಿನಗಳು. ಒಮ್ಮೆ ಅವರು ಕಾಶಿಯಲ್ಲಿ ತಂಗಿದ್ದರು. ಅಲ್ಲಿನ ದುರ್ಗಾಮಾತೆಯ ದೇವಾಲಯವನ್ನು ನೋಡಿ ಹಿಂದಿರುಗುತ್ತಿದ್ದರು. ದಾರಿಯ ಒಂದು ಬದಿಯಲ್ಲಿ ಎತ್ತರವಾದ ಗೋಡೆ. ಇನ್ನೊಂದು ಬದಿಯಲ್ಲಿ ವಿಶಾಲವಾದ ಸರೋವರ. ಕಿರುದಾರಿಯಲ್ಲಿ ಅವರು ನಡೆದು ಹೋಗುತ್ತಿದ್ದಾಗ ದೊಡ್ಡ ಕೋತಿಗಳ ಗುಂಪು ಒಂದು ಅವರನ್ನು ಅಟ್ಟಿಸಿಕೊಂಡು ಬಂತು. ಜೋರಾಗಿ ಕಿರಿಚಾಡುತ್ತ ಕೋತಿಗಳು ಅವರ ಕಾಲನ್ನೇ ಹಿಡಿಯಬಂದವು. ಅವು ಹತ್ತಿರ ಬರುತ್ತಿದ್ದಂತೆಯೇ ಅವರು ಓಡತೊಡಗಿದರು. ಕೋತಿಗಳು ಅತಿ ಶೀಘ್ರವಾಗಿ ಓಡುತ್ತ ಅವರನ್ನು ಹಿಂಬಾಲಿಸಿ ಕಚ್ಚಲು ಬಂದವು. ಇನ್ನು ತಪ್ಪಿಸಿಕೊಳ್ಳಲು ಅಸಾಧ್ಯ ವೆನಿಸಿತು. ಇದ್ದಕ್ಕಿದ್ದಂತೆ ಒಬ್ಬ ವೃದ್ಧ ಸಂನ್ಯಾಸಿ ಸ್ವಾಮೀಜಿಯವರನ್ನು ಕುರಿತು ಗಟ್ಟಿಯಾಗಿ ಕೂಗಿಕೊಂಡ, ‘ಓಡಬೇಡಿ, ಎದುರಿಸಿ.’ ಆ ಮಾತನ್ನು ಕೇಳಿದೊಡನೆಯೇ ಸ್ವಾಮೀಜಿ ಹಿಂದಿರುಗಿ ಧೈರ್ಯದಿಂದ ಕೋತಿಗಳನ್ನು ಎದುರಿಸಿದರು. ಅವರು ಹಿಂದಿರುಗಿ ಧೈರ್ಯದಿಂದ ಅವುಗಳನ್ನು ದಿಟ್ಟಿಸಿ ನೋಡಿದಾಗ ಅವು ಹಿಂಜರಿಯತೊಡಗಿದವು. ಕೊನೆಗೆ ಓಡಿಹೋಗಲು ಆರಂಭಿಸಿದವು.

‘ಎಲ್ಲರ ಬದುಕಿಗೂ ಇದೊಂದು ಮಹಾಪಾಠ. ಭಯಾನಕವಾದುದನ್ನು ಎದುರಿಸಿ. ನಾವು ಭಯವಿಹ್ವಲರಾಗಿ ಓಡಿ ಹೋಗದಿದ್ದರೆ, ದೃಢನಿಷ್ಠೆಯಿಂದ ಎದುರಿಸಿದರೆ, ಕೋತಿಗಳಂತೆ ಕಷ್ಟ ಕಂಟಕಗಳೂ ದೂರಕ್ಕೋಡುವುವು. ಪ್ರಕೃತಿಯನ್ನು ಜಯಿಸಿದರೆ ಮಾತ್ರ ನಾವು ಸ್ವತಂತ್ರರಾಗುತ್ತೇವೆ. ಪ್ರಕೃತಿಗೆ ಬೆನ್ನು ತೋರಿಸಿ ಓಡುವುದರಿಂದಲ್ಲ. ಹೇಡಿಗಳು ಎಂದೂ ವಿಜಯಿಗಳಾಗುವುದಿಲ್ಲ. ಭಯ, ತೊಂದರೆ, ಅಜ್ಞಾನ–ಇವು ಪಲಾಯನ ಸೂತ್ರ ಪಠಿಸುವಂತಾಗಬೇಕಾದರೆ ನಾವು ಅವುಗಳನ್ನು ಎದುರಿಸಬೇಕು, ಹೋರಾಡಬೇಕು’\footnote{\engfoot{If you feel afraid of anything always turn round and face it. Never think of running away.}\hfill\engfoot{ –Swami Vivekananda}} ಎಂದರು ಸ್ವಾಮೀಜಿ.


\section*{ಧೈರ್ಯದ ತವರಾಗಿ!}

\addsectiontoTOC{ಧೈರ್ಯದ ತವರಾಗಿ!}

ವಿವೇಕಾನಂದರು ಇನ್ನೊಂದು ಅಪಾಯದ ಸನ್ನಿವೇಶವನ್ನು ಎದುರಿಸಿದ ಅರ್ಥಪೂರ್ಣ ಘಟನೆಯನ್ನು ಪರಿಶೀಲಿಸಿ:

ಇಂಗ್ಲೆಂಡಿಗೆ ಪ್ರಥಮಬಾರಿ ಭೇಟಿ ಇತ್ತ ಸಂದರ್ಭ. ಒಂದುದಿನ ವಿವೇಕಾನಂದರು ಸ್ನೇಹಿತ ರೊಂದಿಗೆ ಸಂಜೆಯ ಹೊತ್ತು ತಿರುಗಾಡಲು ಹೊರಟಿದ್ದರು. ಅಕಸ್ಮಾತ್ ಎಲ್ಲಿಂದಲೋ ಹುಚ್ಚು ಗೂಳಿಯೊಂದು ಅವರಿಗೆ ಎದುರಾಗಿ ಓಡಿ ಬರುತ್ತಿತ್ತು. ಸಮೀಪದಲ್ಲಿದ್ದವರೆಲ್ಲ ದಿಕ್ಕೆಟ್ಟು ಪ್ರಾಣ ಭಯದಿಂದ ಸ್ವಾಮೀಜೀಯವರನ್ನು ಅವರ ಪಾಡಿಗೆ ಬಿಟ್ಟು ಓಡತೊಡಗಿದರು. ಅತಿರಭಸದಿಂದ ಓಡಿ ಕೆಲವರು ಕುಸಿದು ಕುಳಿತರು. ಸ್ನೇಹಿತರಿಗೆ ಯಾವ ರಕ್ಷಣೆಯನ್ನೂ ನೀಡಲಾಗಲಿಲ್ಲವೆಂದು ಕೊಳ್ಳುತ್ತ ಸ್ವಾಮೀಜಿ ಕೈಕಟ್ಟಿಕೊಂಡು ಸ್ಥಿರವಾಗಿ ನಿಂತು ಓಡಿಬರುತ್ತಿರುವ ಗೂಳಿಯನ್ನು ದಿಟ್ಟಿಸಿ ನೋಡತೊಡಗಿದರು. ಗೂಳಿ ಹತ್ತಿರ ಬಂದು ಅವರನ್ನೇ ಸ್ವಲ್ಪ ಹೊತ್ತು ನೋಡುತ್ತ ಬಳಿಕ ತಲೆ ಅಲ್ಲಾಡಿಸಿ ನಿಂತ ಜಾಗದಲ್ಲೇ ಒಂದೆರಡು ಬಾರಿ ತಿರುಗಿ ಅಲ್ಲಿಂದ ಹೊರಟು ಹೋಯಿತು!

ಈ ಅಪಾಯದಿಂದ ಪಾರಾಗಿ ಹಿಂದಿರುಗುವಾಗ ಸ್ನೇಹಿತರು ಸ್ವಾಮೀಜಿಯವರನ್ನು ಪ್ರಶ್ನಿಸಿದರು: ‘ಸ್ವಾಮೀಜಿ, ಗೂಳಿಯು ನಿಮ್ಮನ್ನು ಸಮೀಪಿಸಿದಾಗ ನಿಮ್ಮ ಮನಸ್ಸಿನ ಸ್ಥಿತಿ ಹೇಗಿತ್ತು?’ ಸ್ವಾಮೀಜಿಯವರ ಉತ್ತರ: ‘ನನ್ನಷ್ಟು ಭಾರದ ವ್ಯಕ್ತಿಯನ್ನು ಒಂದು ವೇಳೆ ಅದು ಎತ್ತಿ ಎಸೆದಿದ್ದರೆ ಎಷ್ಟು ದೂರ ಹೋಗಿ ಬೀಳುತ್ತಿದ್ದೆ ಎಂದು ಲೆಕ್ಕ ಗುಣಿಸುತ್ತಿದ್ದೆ!’ ಮುಂದುವರಿದು ‘ಅಪಾಯದ ಸ್ಥಿತಿ ಹತ್ತಿರ ಬಂದಂತೆಲ್ಲ, ಮೃತ್ಯುಮುಖದಲ್ಲೂ ಕೂಡ, ನನ್ನ ಮನಸ್ಸು ಅತ್ಯಂತ ಶಾಂತವೂ, ಸ್ಥಿರವೂ, ದೃಢವೂ ನಿರ್ಭೀತವೂ ಆಗಿರುತ್ತದೆ’ ಎಂದರಂತೆ.

ಅಂಥ ಮಾನಸಿಕ ಸ್ಥಿತಿಯ ಕಾರಣವನ್ನೂ ಅವರು ಹೇಳಿದ್ದರು–‘ಭಗವಂತನ ಪಾದಸ್ಪರ್ಶವನ್ನು ಮಾಡಿದ ನನಗೆ ಭೀತಿ ಇಲ್ಲ.’ ಯಾರಿಗೂ ಅಂಜದೆ, ಯಾರಿಗೂ ಅಂಜಿಕೆಯನ್ನುಂಟು ಮಾಡದೆ ಇರುವುದು ಮಹಾತ್ಮರ ಲಕ್ಷಣ ಎಂದ ಮಹಾರಾಷ್ಟ್ರದ ಸಂತ ಶ‍್ರೀರಾಮದಾಸರ ನುಡಿಯನ್ನಿಲ್ಲಿ ಸ್ಮರಿಸಬಹುದು.

ಮಹಾತ್ಮರೆಂದರೆ ಮಹಾನ್ ಆದ ಆತ್ಮನನ್ನು ಅಥವಾ ಸರ್ವಶಕ್ತವಾದ ಭಗವಂತನನ್ನು\break ಸಾಕ್ಷಾತ್ಕರಿಸಿಕೊಂಡವರೆಂದು ಅರ್ಥ. ಈ ಸಾಕ್ಷಾತ್ಕಾರದಿಂದ ಲಭಿಸುವ ಎರಡು ಫಲಗಳೇ ಬಲ ಮತ್ತು ನಿರ್ಭೀತಿ. ನಿಜವಾದ ಭಗವದ್ ಭಕ್ತರು ಅಥವಾ ಜ್ಞಾನಿಗಳು ನಿರ್ಭೀತ ಮನೋವೃತ್ತಿ ಯವರು. ಅವರ ಸಾನ್ನಿಧ್ಯವನ್ನು ಸಮೀಪಿಸಿದ ಪ್ರಾಣಿ ಪಕ್ಷಿಗಳೂ ನಿರ್ಭೀತಿಯ ಸ್ಪಂದನವನ್ನು ಅನುಭವಿಸಬಲ್ಲವು. ಅವರ ಉಪದೇಶಗಳನ್ನು ಶ್ರದ್ಧೆ ಮತ್ತು ನಿಷ್ಠೆಗಳಿಂದ ಅನುಸರಿಸಿದರೆ ನಾವೂ ನಿರ್ಭೀತರಾಗಬಲ್ಲೆವು.


\section*{ದೈವಕೃಪೆಯ ಕಾವಲು}

\vskip -7pt\addsectiontoTOC{ದೈವಕೃಪೆಯ ಕಾವಲು}

ಕೆಲವೊಮ್ಮೆ ಅಪಾಯದ ಸನ್ನಿವೇಶ ಮೊದಲೇ ದೃಷ್ಟಿಗೋಚರವಾಗುವುದಿಲ್ಲ. ಆಗ ಅದನ್ನು ಎದುರಿಸುವ ಪ್ರಶ್ನೆಯೇ ಇಲ್ಲ. ಇಂತಹ ಘಟನೆಗಳು ಅಕಸ್ಮಾತ್ ಘಟಿಸಿ ದೃಷ್ಟಿಗೋಚರವಾದಾಗ ಅದನ್ನು ಎದುರಿಸಲು ಬೇಕಾಗುವ ಮನಸ್ಥೈರ್ಯ ಮತ್ತು ಶಾಂತಿಯನ್ನು ಕಾಪಾಡಿಕೊಳ್ಳುವುದು ಸುಲಭಸಾಧ್ಯವಲ್ಲ. ಇಂಥ ಗಂಡಾಂತರಗಳನ್ನು ಎದುರಿಸುವುದು ಹೇಗೆ? ಇಂತಹ ಸಂದರ್ಭಗಳಲ್ಲಿ ಭಯವಿಹ್ವಲರಾಗದಿರಲು ಸಾಧ್ಯವೇ? ಸ್ವಾಮಿ ವಿವೇಕಾನಂದರು ತಮ್ಮ ಪರಿವ್ರಾಜಕ ದಿನಗಳಲ್ಲಿ ನಡೆದ ಇನ್ನೊಂದು ಘಟನೆಯನ್ನು ಕುರಿತು ತಮ್ಮ ಶಿಷ್ಯರಿಗೆ ಹೇಳಿದ್ದರು. ನಿಜವಾದ ಅರ್ಥಪೂರ್ಣ ಘಟನೆಯಾದುದರಿಂದ ಅದು ಗಮನಾರ್ಹ. ಅವರ ಮಾತುಗಳಲ್ಲೇ ಅದನ್ನು ಕೇಳಿ:

‘ಆಗ ನಾನು ಹಿಮಾಲಯ ಪ್ರದೇಶದಲ್ಲಿ ಮನೆಯಿಂದ ಮನೆಗೆ ಹೋಗಿ ಭಿಕ್ಷೆ ಬೇಡಿ ಆಹಾರ ಪಡೆಯುತ್ತಿದ್ದೆ. ಹೆಚ್ಚಿನ ಸಮಯವನ್ನು ಕಠಿಣವಾದ ಆಧ್ಯಾತ್ಮಿಕ ಸಾಧನೆಗಳಲ್ಲಿ ಕಳೆಯುತ್ತಿದ್ದೆ. ಆಗ ಲಭ್ಯವಿದ್ದ ಆಹಾರ ತೀರ ಸಾಧಾರಣವಾದುದು. ಅದೂ ಸಹ ನನ್ನ ಹಸಿವನ್ನು ತಣಿಸುತ್ತಿರಲಿಲ್ಲ. ನನ್ನ ಜೀವನ ನಿಷ್ಪ್ರಯೋಜಕವೆಂದು ಒಂದು ದಿನ ಯೋಚಿಸಿದೆ. ಇಲ್ಲಿನ ಪರ್ವತವಾಸಿಗಳಾದ ಜನರು ತಾವೇ ತುಂಬಾ ಬಡವರು. ತಮ್ಮ ಮಕ್ಕಳಿಗೆ ಮತ್ತು ಸಂಸಾರವಂದಿಗರಿಗೆ ಅವರು ಆಹಾರ ಒದಗಿಸಲಾರರು. ಆದರೂ ನನಗಾಗಿ ಅವರು ಸ್ವಲ್ಪ ಉಳಿಸಲು ಪ್ರಯತ್ನಿಸುತ್ತಾರೆ. ಇಂತಿರುವಾಗ ಈ ಜೀವನದಿಂದ ಪ್ರಯೋಜನವೇನು? ಆಹಾರಕ್ಕಾಗಿ ಹೊರಗೆ ಹೋಗುವುದನ್ನು ಅಂದಿನಿಂದ ನಿಲ್ಲಿಸಿದೆ. ಆಹಾರವಿಲ್ಲದೆ ಎರಡು ದಿನಗಳು ಕಳೆದುಹೋದವು. ಬಾಯಾರಿಕೆ ಯಾದಾಗ ಬೊಗಸೆಯಿಂದ ತೊರೆಯ ನೀರನ್ನು ಕುಡಿಯುತ್ತಿದ್ದೆ. ಒಂದು ದಿನ ದಟ್ಟವಾದ ಕಾಡನ್ನು ಪ್ರವೇಶಿಸಿದೆ. ಅಲ್ಲಿ ಕಲ್ಲಿನ ಮೇಲೆ ಕುಳಿತು ಧ್ಯಾನಮಗ್ನನಾದೆ. ಕಣ್ಣು ಬಿಟ್ಟಾಗ ಎದುರಿಗೆ ಕೆಲವು ಗಜಗಳ ದೂರದಲ್ಲಿ ಒಂದು ಭಾರೀ ಹುಲಿ ಗೋಚರಿಸಿತು. ತನ್ನ ಹೊಳೆಯುವ ಕಣ್ಣುಗಳಿಂದ ಅದು ನನ್ನನ್ನು ನೋಡಿತು. ನಾನು ಯೋಚಿಸಿದೆ. ಅಂತೂ ಕೊನೆಗೆ ಶಾಂತಿ ದೊರೆಯಿತು. ಹುಲಿಗೂ ಆಹಾರ ಸಿಕ್ಕಿತು. ಈ ಪ್ರಾಣಿಗೆ ಈ ದೇಹ ಆಹಾರವಾದರೆ ಅಷ್ಟೇ ಸಾಕು. ಕಣ್ಮುಚ್ಚಿ ನಾನು ಹುಲಿ ಮೇಲೆರಗಬಹುದೆಂದು ಕೆಲವು ನಿಮಿಷ ಕಳೆದೆ. ಆದರೆ ಅದು ನನ್ನ ಮೇಲೆ ಆಕ್ರಮಣ ಮಾಡಲಿಲ್ಲ. ಆಗ ಕಣ್ಣುಬಿಟ್ಟು ಸುತ್ತಮುತ್ತ ನೋಡಿದೆ. ಹುಲಿ ಹಿಂದೆ ಸರಿದು ಕಾಡಿನಲ್ಲಿ ಹೋಗುತ್ತ ಇತ್ತು. ಆಶ್ಚರ್ಯವಾಯಿತು! ದೇವರು ನನ್ನನ್ನು ರಕ್ಷಿಸುತ್ತಿರುವ ಅರಿವಾಯಿತು! ಮಾಡಬೇಕಾದ ಕೆಲಸಗಳಿವೆ. ಅಲ್ಲಿಯವರೆಗೂ ಜಗತ್ತಿನಿಂದ ಬಿಡುಗಡೆ ಇಲ್ಲ.’

ಅಪಾಯದ ಮುನ್ಸೂಚನೆ ಇರಲಿಲ್ಲ. ಧ್ಯಾನದಿಂದೆದ್ದು ಕಣ್ತೆರೆದಾಗ ಭೀಕರವಾದ ನರಭಕ್ಷಕ ಸಮೀಪದಲ್ಲೇ ನಿಂತು ಸ್ವಾಮೀಜಿಯವರನ್ನು ದಿಟ್ಟಿಸುತ್ತಿದೆ. ಸ್ವಾಮೀಜಿಯವರು ಭಯಪಡಲಿಲ್ಲ. ತಮ್ಮ ಶರೀರವನ್ನು ಹುಲಿಗೆ ಆಹಾರವಾಗಿ ಸಮರ್ಪಿಸಲು ಸಿದ್ಧರಾಗಿ ಧ್ಯಾನಸ್ಥರಾಗಿ ಕುಳಿತಿದ್ದಾರೆ! ದೈವೀ ಸಹಾಯವನ್ನೂ ಅವರು ಯಾಚಿಸಲಿಲ್ಲ! ಆದರೂ ಹುಲಿಯಿಂದ ಯಾವ ಅಪಾಯವೂ ಉಂಟಾಗಲಿಲ್ಲ. ಅದು ಅಲ್ಲಿಂದ ಹೊರಟೇ ಹೋಯಿತು!

ಮನುಷ್ಯರ ಬದುಕಿನಲ್ಲಿ ಕೆಲವೊಮ್ಮೆ ಕೆಲವರಿಗೆ ಸಾಮಾನ್ಯ ನಿಯಮಗಳನ್ನು ಮೀರಿದ ಶಕ್ತಿ\-ಯೊಂದು ಕೆಲಸ ಮಾಡುವುದು ಸ್ಪಷ್ಟವಾಗಿ ಗೋಚರಿಸುತ್ತದೆ. ದೇವರಲ್ಲಿ, ಅತಿಮಾನುಷ ವ್ಯಾಪಾರಗಳಲ್ಲಿ, ನಂಬಿಕೆ ಇರದ ವೈಚಾರಿಕರು ಇಂಥ ಘಟನೆಗಳನ್ನು ‘ಆಕಸ್ಮಿಕ’, ‘ಚಾನ್ಸ್​’, ‘ಕಾಕತಾಳೀಯ’, ‘ಕೆಲವೊಮ್ಮೆ ಅಂಥ ಘಟನೆಗಳು ನಡೆಯುತ್ತವೆ’, ‘ಅದೇನು ಮಹಾ!’ ಎಂದು ಅಲ್ಲಗಳೆಯುವುದುಂಟು. ಆದರೆ ಅನುಭವಿಗಳು, ಅನುಭಾವಿಗಳು ಅವುಗಳಲ್ಲಿ ಅಗೋಚರ ಶಕ್ತಿಯ ಕೈವಾಡದ ಕಾರಣವನ್ನು ನಂಬುವುದಲ್ಲ, ಕಾಣುತ್ತಾರೆ!

‘ದೇವರು ನನ್ನನ್ನು ರಕ್ಷಿಸುತ್ತಿರುವ ಅರಿವಾಯಿತು!’ ಎಂದರು ಸ್ವಾಮೀಜಿ. ಶರಣರನ್ನು ಭಗವಂತ ವಿಶೇಷ ರೀತಿಯಿಂದ ಕಾಪಾಡುತ್ತಾನೆಂಬುದು ಭಕ್ತರ ಪಾಲಿಗೆ ಸತ್ಯಸ್ಯ ಸತ್ಯ. ಯಾರನ್ನೇ ಆಗಲಿ, ಎಂಥ ಪರಿಸ್ಥಿತಿಯಲ್ಲೇ ಇರಲಿ, ಭಗವಂತ ರಕ್ಷಿಸಬಲ್ಲ. ಭಗವಂತನಿಗೆ ಅಸಾಧ್ಯವಾದುದು ಯಾವುದೂ ಇಲ್ಲ. ರಕ್ಷಿಸಲಿಲ್ಲವೆಂದು ಕೆಲವೊಮ್ಮೆ ಕಂಡರೂ ಅದಕ್ಕೆ ಅರ್ಥ, ಉದ್ದೇಶಗಳಿವೆ–ಇದು ಭಗವದ್ಭಕ್ತರ ದೃಷ್ಟಿಕೋನ.

ಮೇಲಿನ ಅಸಾಮಾನ್ಯ ಘಟನೆಗಳು ಸಂತರಿಗೇ, ಮಹಾಪುರುಷರಿಗೇ ಮೀಸಲಾದ ಅನುಭವಗಳಲ್ಲ. ಎಲ್ಲ ದೇಶಗಳ, ಎಲ್ಲ ಕಾಲಗಳ ಬೇರೆ ಬೇರೆ ಧರ್ಮದ ಅನುಯಾಯಿಗಳಾದ ಸಾಮಾನ್ಯರ ಬದುಕಿನಲ್ಲೂ ಕಂಡುಬರುವ ಅನುಭವಗಳು.


\section*{ಹಿಮಕರಡಿಯ ಎದುರು}

\addsectiontoTOC{ಹಿಮಕರಡಿಯ ಎದುರು}

ಭಾರತ–ಚೀನಾ ಗಡಿ ಘರ್ಷಣೆಯ ಕಾಲದಲ್ಲಿ ರಕ್ಷಣಾ ಪಡೆಯಲ್ಲಿದ್ದ ತರುಣರೊಬ್ಬರು ತಮ್ಮ ಒಂದು ಅಪೂರ್ವ ಅನುಭವವನ್ನು ನನಗೆ ತಿಳಿಸಿದರು. ಒಂದು ದಿನ ಏಕಾಂಗಿಯಾಗಿ\hbox{ ಹಿಮಾಲಯದ} ಪರ್ವತ ಪ್ರಾಂತದಲ್ಲಿ ಸಂಜೆಯ ಹೊತ್ತು ವಾಕಿಂಗ್ ಹೊರಟಿದ್ದರು. ಸುಮಾರು ನಾಲ್ಕು ಕಿಲೋ\-ಮೀಟರ್ ನಡೆದಿರಬೇಕು. ಎದುರುಗಡೆಯಿಂದ ಒಂದು ಹಿಮಕರಡಿ ಬರುತ್ತಿರುವುದು ಅವರಿಗೆ ಗೋಚರವಾಯಿತು. ಈ ತರುಣ ಮಿತ್ರರ ಜಂಘಾಬಲವೇ ಉಡುಗಿಹೋಯಿತು. ಎದುರು ಸಿಕ್ಕಿದ ಮನುಷ್ಯರನ್ನು ಬಡಿದು ಚಚ್ಚಿ, ಸಾಯಿಸದೇ ಬಿಡದಂಥ ಪ್ರಾಣಿ ಅದು ಎಂಬುದು ಅವರಿಗೆ ತಿಳಿದ ವಿಷಯವೇ ಆಗಿತ್ತು. ಆಗ ಪಿಸ್ತೂಲನ್ನು ತೆಗೆದುಕೊಳ್ಳದೇ ಬಂದುದರ ತಪ್ಪಿನ ಅರಿವೂ ಆಯಿತು. ‘ನನ್ನ ಕತೆ ಮುಗಿಯಿತು. ಊರಿನಿಂದ, ಬಂಧುಬಾಂಧವರಿಂದ \hbox{ದೂರವಾಗಿ}~ಇಲ್ಲಿ ಕರಡಿಯ ಕೈಯಲ್ಲಿ ಸಾಯಬೇಕಾದ ದುಃಸ್ಥಿತಿ ಬಂತಲ್ಲ!’ ಎಂದು ಯೋಚಿಸಿ ದಿಗ್ಭ್ರಾಂತ\-ನಾದೆ. ಮನೆಯವರೆಲ್ಲರ ಚಿತ್ರ, ನಂಬಿದ ಮಾರುತಿದೇವರ ಚಿತ್ರ ಮನಸ್ಸಿನಲ್ಲಿ ಮೂಡಿ ಮಾಯ\-ವಾಯಿತು. ಸುಮ್ಮನೆ ಕರಡಿಯನ್ನು ದಿಟ್ಟಿಸುತ್ತ ನಿಂತುಕೊಂಡಿದ್ದೆ. ನನ್ನಿಂದ ಹತ್ತು ಹದಿನೈದು ಅಡಿಗಳ ದೂರದವರೆಗೆ ಬಂದಿದ್ದ ಕರಡಿ ಕ್ಷಣಕಾಲ ನೆಟ್ಟದೃಷ್ಟಿಯಿಂದ ನೋಡಿ ಹಿಂದಿರುಗಿತು! ಹಾಗೆ ಹಿಂದಿರುಗಲು ಇಂದ್ರಿಯಗೋಚರವಾದ ಯಾವ ಕಾರಣವೂ ಇರಲಿಲ್ಲ. ದೇವರೇ ಕಾಪಾಡಿದ ಎನ್ನುತ್ತ ಹಿಂದಿರು\-ಗಿದೆ. ನಡೆದ ಘಟನೆಯನ್ನು ಕೇಳಿ ಕ್ಯಾಂಪಿನಲ್ಲಿ ಎಲ್ಲರೂ ಆಶ್ಚರ್ಯ \hbox{ಚಕಿತರಾದರು.}

‘ಹತ್ತು ದಿನಗಳ ಬಳಿಕ ದಕ್ಷಿಣ ಭಾರತದಿಂದ ಬಂದ ತಂದೆಯ ಪತ್ರ ಕೈ ಸೇರಿತು. “ಕನಸಿನಲ್ಲಿ ಮಾರುತಿ ದೇವರು ನಿನ್ನ ಮಗನನ್ನು ಅಪಾಯದಿಂದ ಪಾರುಮಾಡಿದ್ದೇನೆ” ಎಂದಿದ್ದಾರೆ. “ನಿನ್ನ ಕ್ಷೇಮ ಸಮಾಚಾರವನ್ನು ಕೂಡಲೇ ತಿಳಿಸು” ಎಂದಿತ್ತು ಅದರಲ್ಲಿ.’

ಸರ್ವಶಕ್ತನಾದ ಭಗವಂತನ ನೆರವನ್ನು ಪಡೆದು, ಅವನೊಂದಿಗೆ ನಮಗಿರುವ ನಿತ್ಯಸಂಬಂಧವನ್ನು ನೆನೆದು ಎಂಥ ಕಷ್ಟಕಂಟಕಗಳನ್ನೂ ಎದುರಿಸಲು ಸಾಧ್ಯ–ಎಂದ ಸಂತಮಹಾತ್ಮರ, ಶ‍್ರೀ\-ಸಾಮಾನ್ಯರ, ಭಕ್ತಸಾಧಕರ, ಅಸಂಖ್ಯ ಅನುಭವಗಳು ಸಾರುತ್ತವೆ. ನಾವೆಲ್ಲರೂ ಭಗವಂತನ ನೆರವನ್ನು ಪಡೆದು ಬದುಕಿನ ಸಮಸ್ಯೆಗಳನ್ನು ಎದುರಿಸಲು ಪ್ರಯತ್ನಿಸುವುದು ಒಳಿತಲ್ಲವೇ?

ಎಲ್ಲ ಸಮಸ್ಯೆಗಳಿಗೂ ಆಧ್ಯಾತ್ಮಿಕ ಶ್ರದ್ಧೆಯಿಂದ ಉತ್ತರಗಳನ್ನು ಪಡೆಯಬಹುದು. ಆದರೆ ಆ ಶ್ರದ್ಧೆ, ಸತ್ಸಂಗ ಮತ್ತು ಅಭ್ಯಾಸಗಳಿಂದ ಸಾಧ್ಯ.


\section*{ಅಭ್ಯಾಸದಿಂದ ಭಯ ದೂರ}

\addsectiontoTOC{ಅಭ್ಯಾಸದಿಂದ ಭಯ ದೂರ}

ಎಲ್ಲ ತೆರನಾದ ದೈಹಿಕ, ಮಾನಸಿಕ ವರ್ತನೆ ಮತ್ತು ಪ್ರವೃತ್ತಿಗಳನ್ನು ತಾಳ್ಮೆಯಿಂದ ಕೂಡಿದ ನಿಯಮಿತ ಅಭ್ಯಾಸದಿಂದ ನಮ್ಮ ಸಹಜ ಸ್ವಭಾವವನ್ನಾಗಿ ಪರಿವರ್ತಿಸಿಕೊಳ್ಳಬಹುದು.

{\noindent\leftskip=0.6cm ಸ್ನಾನ ಮಾಡಿ, ಮಾಡಿ, ಜನರು ಸ್ನಾನಶೀಲರಾಗುತ್ತಾರೆ.\\
 ದಾನ ಮಾಡಿ, ಮಾಡಿ, ದಾನಶೀಲರಾಗುತ್ತಾರೆ.\\
 ಕ್ಷಮಿಸಿ, ಕ್ಷಮಿಸಿ, ಕ್ಷಮಾಶೀಲರಾಗುತ್ತಾರೆ.\\
 ಧ್ಯಾನ ಮಾಡಿ, ಮಾಡಿ, ಧ್ಯಾನಶೀಲರಾಗುತ್ತಾರೆ.\\
 ಚಿಂತಿಸಿ, ಚಿಂತಿಸಿ, ಚಿಂತಾಶೀಲರಾಗುತ್ತಾರೆ.\\
 ದಯೆಯನ್ನು ತೋರಿಸಿ, ದಯಾಶೀಲರಾಗುತ್ತಾರೆ.\\
 ಪಾಪಕೃತ್ಯಗಳನ್ನು ಆಚರಿಸಿ, ಪಾಪಶೀಲರಾಗುತ್ತಾರೆ.\\
 ಶ್ರದ್ಧೆಯನ್ನು ವೃದ್ಧಿಸಿಕೊಳ್ಳುತ್ತ, ಶ್ರದ್ಧಾಶೀಲರಾಗುತ್ತಾರೆ.\\
 ಜಯವನ್ನು ಗಳಿಸುತ್ತ, ಗಳಿಸುತ್ತ, ಜಯಶೀಲರಾಗುತ್ತಾರೆ.\\
 ಭಯಪಡುತ್ತ, ಭಯಗ್ರಸ್ತ ವ್ಯಕ್ತಿಗಳಾಗುತ್ತಾರೆ.\par}

ಯಾವ ನಡತೆಯನ್ನಾದರೂ ಕ್ರಮವರಿತು, ನಿಯಮಿತವಾಗಿ ಮಾಡುತ್ತ ಬಂದರೆ ಅದೇ ಅಭ್ಯಾಸವಾಗುವುದು. ಅಭ್ಯಾಸವಾದುದನ್ನು ಬಿಡಲಾಗುವುದಿಲ್ಲ. ಬಿಟ್ಟರೆ ಏನನ್ನೋ ಕಳೆದು ಕೊಂಡಂತಾಗುವುದು; ದಿನದ ಕೆಲಸವೆಲ್ಲ ಹಾಳಾಗುವುದು. ಅಭ್ಯಾಸಗಳೇ ಶೀಲನಿರ್ಮಾಣದ ಅಡಿ ಗಲ್ಲುಗಳು.

ಬೇರೆ ಬೇರೆ ಯೋಚನೆಗಳೂ, ಭಾವನೆಗಳೂ ಪುನರಾವರ್ತಿತವಾದರೆ ಅಥವಾ ಮನಸ್ಸಿನಲ್ಲಿ ತಿರುತಿರುಗಿ ಬರುತ್ತಲಿದ್ದರೆ ಅಂಥ ಸ್ವಭಾವ ನಮ್ಮದು ಎಂದಾಗುತ್ತದೆ.

ಚಿಂತೆ, ಸಂತೋಷ, ಸಿಟ್ಟು, ಶಾಂತತೆ–ಇವು ಹೇಗೆ ಮನಸ್ಸಿನ ಸ್ವಭಾವವೋ, ಹಾಗೆಯೇ ಭಯವೂ ಕೂಡ ಮನಸ್ಸಿನ ಒಂದು ಸ್ವಭಾವ. ಈ ಸ್ವಭಾವವನ್ನು ನಿಯಮಿತ ಅಭ್ಯಾಸದಿಂದ ಸಾವಕಾಶವಾಗಿ, ಆದರೆ ದೃಢವಾಗಿ ನಾವು ರೂಢಿಸಿಕೊಂಡಿದ್ದೇವೆ. ನಿರ್ಭೀತಿಯ ಭಾವನೆಗಳನ್ನು ನಿಯಮಿತ ರೀತಿಯಲ್ಲಿ ಮೆಲುಕಾಡುತ್ತ, ನಿಯಮಿತ ರೀತಿಯಲ್ಲಿ ಸಾವಕಾಶವಾಗಿ ಅಭ್ಯಾಸ ಮಾಡಿದರೆ, ಭಯಪಡುವ ನಮ್ಮ ಸ್ವಭಾವ ಬದಲಾಗಲೇಬೇಕು.

ಅಕ್ಷರಾಭ್ಯಾಸ ಮಾಡುವಾಗ ಮಕ್ಕಳು ನಿಧಾನವಾಗಿ ಒಂದೊಂದೇ ಅಕ್ಷರಗಳನ್ನು ತಿದ್ದುವಂತೆ, ಬದಲಾವಣೆಯನ್ನು ತಂದುಕೊಳ್ಳಬೇಕೆಂದು ಯೋಚಿಸುವ ವ್ಯಕ್ತಿ, ಶ್ರದ್ಧೆಯಿಂದ, ಮೆಲ್ಲಮೆಲ್ಲನೆ ಮುಂದುವರಿಯಬೇಕು. ತಾಳ್ಮೆಗೆಟ್ಟರೆ ಯಾವ ಕ್ಷೇತ್ರದಲ್ಲೂ ನಮ್ಮ ಪ್ರಯತ್ನವು ಯಶಸ್ವಿ ಯಾದೀತೆ? ಯೋಚಿಸಿ ನೋಡಿ.

ಹಲವಾರು ಸಣ್ಣಸಣ್ಣ ಬಿಂದುಗಳನ್ನು ಕ್ರಮಬದ್ಧವಾಗಿ ಜೋಡಿಸಿ ನಾನಾ ಆಕಾರಗಳನ್ನುಂಟು ಮಾಡಬಹುದು–ವೃತ್ತಾಕಾರ, ತ್ರಿಕೋನ, ಚತುರ್ಭುಜ, ಷಡ್ಭುಜಗಳನ್ನೂ ರಚಿಸಬಹುದು. ಅತ್ಯಂತ ಸಣ್ಣಪುಟ್ಟ ಕೆಲಸಗಳನ್ನು ಸಮರ್ಪಕವಾಗಿ ಮಾಡುವ ವಿಧಾನದಿಂದ, ನಮ್ಮ ಚಾರಿತ್ರ್ಯಕ್ಕೂ ಒಂದು ಆಕಾರ ಅಥವಾ ರೂಪು ಕೊಡುತ್ತಿರುತ್ತೇವೆ ಎಂಬುದು ಗಮನಾರ್ಹ.


\section*{ಕಷ್ಟದ ಕುಲುಮೆ}

\addsectiontoTOC{ಕಷ್ಟದ ಕುಲುಮೆ}

ಮನುಷ್ಯನ ಬದುಕಿನಲ್ಲಿ ಚಿಂತೆಯ ಕುಲುಮೆಯಲ್ಲಿ ಕಾಯಿಸುವ ಎಂಥೆಂಥ ಕಷ್ಟಸಂಕಟಗಳು, ನೋವು ತಾಪಗಳು, ಅಸಹಾಯಕಸ್ಥಿತಿಗಳು ಬರಬಹುದೆಂಬುದನ್ನು ವರ್ಣನೆ ಮಾಡಿ ಮುಗಿಸಲು ಯಾರಿಂದಲೂ ಸಾಧ್ಯವಾಗದು. ಅಂತಹ ಕೆಲವು ನೈಜ ಘಟನೆಗಳು ಇಲ್ಲಿವೆ:

ಬಾಳಿನ ಗೋಳನ್ನು ತಾಳಲಾರದೇ ಒಬ್ಬಾಕೆ ತೋಡಿಕೊಂಡಳು–

\newpage

‘ನಾನೊಬ್ಬ ತಬ್ಬಲಿ. ಚಿಕ್ಕಂದಿನಲ್ಲೇ ತಾಯಿಯನ್ನು ಕಳೆದುಕೊಂಡು ಅಣ್ಣ ಅತ್ತಿಗೆಯವರ ಅನಾದರಣೆಯಲ್ಲಿ ಬೆಳೆದೆ. ನನ್ನ ದುರದೃಷ್ಟದಿಂದ ಒಬ್ಬ ಕುಡುಕನೂ, ವಿಷಯಲಂಪಟನೂ ಆದ ಕ್ರೂರಿಯನ್ನು ಮದುವೆಯಾದೆ. ಈಗ ಮಕ್ಕಳಿಬ್ಬರು ಇದ್ದಾರೆ. ಮೊದಲನೆಯ ಮಗುವಿಗೆ ಒಂಬತ್ತು ವರ್ಷ, ಎರಡನೆಯದಕ್ಕೆ ನಾಲ್ಕು ವರ್ಷ. ಜೀವನದಲ್ಲಿ ಪಡಬಾರದ ಕಷ್ಟಗಳನ್ನೆಲ್ಲ ಅನುಭವಿಸಿ ಸಾಕಾಗಿದೆ. ಇತ್ತ ಗಂಡನ ಪ್ರೀತಿಯೂ ಇಲ್ಲ, ಅತ್ತ ಬಂಧುಬಳಗದವರಿಂದ ದೂರ\-ವಿದ್ದೇನೆ. ಎಷ್ಟೆಂದರೆ ನನಗಾಗಲೀ, ನನ್ನ ಮಕ್ಕಳಿಗಾಗಲೀ, ಹೇಳಿಕೊಳ್ಳಲಾದರೂ ಒಬ್ಬ ಗೆಳತಿಯಾಗಲಿ, ಆತ್ಮೀಯರಾಗಲಿ ಇಲ್ಲ. ಆದರೂ, ಇಷ್ಟು ವರ್ಷಗಳು ದೇವರ ಮೇಲೆ ನಂಬಿಕೆ(?)ಯಿಂದ ಕಾಲ ತಳ್ಳಿದೆವು. ಇದೀಗ ಅದೂ ಸಾಧ್ಯವಾಗುತ್ತಿಲ್ಲ. ಅವರ ಬಾಯಿಯಿಂದ “ನೀನು ಸತ್ತರೂ ಚಿಂತೆ ಇಲ್ಲ” ಎನ್ನುವ ಅರ್ಥದ ಮಾತುಗಳು ಈಗೀಗ ಹೆಚ್ಚಾಗಿ ಬರುತ್ತಿವೆ. ಮೊದಲು ಇಷ್ಟೊಂದು ಕಠಿಣ ಹೃದಯಿಗಳು ಅವರಾಗಿರಲಿಲ್ಲ. ಈಗ ಅವರು ಮಾನಸಿಕವಾಗಿ, ದೈಹಿಕವಾಗಿ, ಹಿಂಸೆಯಿಂದ ಘೋರ ನರಕದಲ್ಲಿ ತಳ್ಳಿದ್ದಾರೆ. ಅವರಿಗೆ ಹೆಣ್ಣುಗಳ ಖಯಾಲಿ ಹೆಚ್ಚಿದೆ. ಅವರನ್ನು ಮೊದಲಿನಿಂದಲೂ ನೋಡಿರುವ ನಾನು ಈಗ ಹತಾಶಳಾಗಿದ್ದೇನೆ. ಬದುಕುವ ಆಸೆ ನನಗಿಲ್ಲವಾಗಿದೆ. ಮಕ್ಕಳ ಮೋಹವೂ ನನ್ನನ್ನು ತಡೆಯುತ್ತಿಲ್ಲ. ಆದರೂ ಮುಂದೆ ಆ ಮಕ್ಕಳ ಅವಸ್ಥೆ ಏನು ಎಂದು ಒಮ್ಮೊಮ್ಮೆ ಕಂಗಾಲಾಗುತ್ತಿದ್ದೇನೆ. ಈ ಎಲ್ಲ ದ್ವಂದ್ವಗಳಿಂದ ನಾನು ಪಾರಾಗಲು ಈ ಹತ\-ಭಾಗಿನಿಗೆ ದಾರಿ ಇದೆಯೆ?’

‘ಹೆದರಬೇಡ. ಭಗವಂತನಿಗೆ ನಿನ್ನ ಕಷ್ಟ ತಿಳಿಯದೆ ಇಲ್ಲ. ಕಬ್ಬಿಣ ಕಾದ ನಂತರ ಅದನ್ನು ಚೆನ್ನಾಗಿ ಬೇಕಾದ ರೀತಿಯಲ್ಲಿ ಬಗ್ಗಿಸಬಹುದು. ದೇವರಲ್ಲಿ ಎಂದೂ ಶ್ರದ್ಧೆಯನ್ನು ಕಳೆದುಕೊಳ್ಳ ಬೇಡ. ಆದರಣೀಯರೆಂದು ಮೇಲೆಮೇಲೆ ತೋರಿಕೊಂಡು, ನಿಜವಾಗಿ ಅನಾದರವನ್ನು ತೋರುವ ಸ್ವಾರ್ಥಿ ದುರಾಸೆಯ ಈ ಜಗತ್ತು ದುಃಖಮಯ ಎಂಬುದು ನಿನಗೀಗ ಅರಿವಾಗಿರಬಹುದು. ನಿನ್ನ ಪ್ರಾರ್ಥನೆಯನ್ನು ಭಗವಂತ ಕೇಳುವುದಿಲ್ಲವೆಂದು ಎಂದೂ ತಿಳಿಯಬೇಡ. ಎರಡು ಬಾರಿ ಈಗ ಮಾಡುತ್ತಿರುವ ಪ್ರಾರ್ಥನೆಯನ್ನು ಇನ್ನೂ ಹೃತ್ಪೂರ್ವಕವಾಗಿ ಕಂಬನಿದುಂಬಿ ಮುಂದುವರಿಸು ತ್ತಿರು. ಥಟ್ಟನೆ ಉಪಯೋಗ ತಿಳಿಯದಿದ್ದರೂ ಖಂಡಿತ ಪ್ರಯೋಜನ ಆಗುವುದು. ಅವರಿವರು ನಿನಗೆ ಒಳಿತು ಮಾಡಲಾರರು. ದೇವರಲ್ಲದೆ ಜೀವಿಯನ್ನು ಉದ್ಧರಿಸಲು ಯಾರಿಗೂ ಸಾಧ್ಯವಿಲ್ಲ. ಹಿಂದಿನದನ್ನು ನೆನಸಿಕೊಳ್ಳಬೇಡ. ಮೊನ್ನೆ, ಆಚೆ ಮೊನ್ನೆ, ನೀನು ಏನು ಊಟಮಾಡಿದೆ ಎಂದು ನೆನೆಸಿಕೊಳ್ಳುವಿಯೇನು? ಅಂತೆಯೇ ಹಿಂದೆ ಆಗಿ ಹೋದ ವಿಚಾರವನ್ನು ಮತ್ತೆ ಮತ್ತೆ ಏಕೆ ನೆನಪುಮಾಡಿಕೊಳ್ಳುವಿ? ಈಗ ಕಳೆದುಹೋದ ಕ್ಷಣ ಮತ್ತೆ ಹಿಂದಿರುಗದು. ಇನ್ನು ಮೇಲೆ ಎಚ್ಚೆತ್ತು ಮುನ್ನಡಿ ಇಡು. ಮಕ್ಕಳಿಗೂ ಮುಂದೆ ಸ್ತೋತ್ರಪಾಠ, ಪ್ರಾರ್ಥನೆ ಮಾಡುವ ವಿಧಾನ ಕಲಿಸು. ಈ ವಿಚಾರವನ್ನು ಬೇರಾರಿಗೂ ತಿಳಿಸಬೇಕಿಲ್ಲ. ನಿನ್ನ ಗಂಡನಿಗಾಗಿಯೂ ಹೃತ್ಪೂರ್ವಕ ಪ್ರಾರ್ಥಿಸು. ಒಳಿತಾಗುವುದು, ಮಂಗಲವಾಗುವುದು. ಒಂದು ತಿಂಗಳ ಕಾಲ ನಿಷ್ಠೆಯಿಂದ ಹೀಗೆ ಮಾಡು. ಮುಂಜಾನೆ ಎದ್ದು ದೇವರನ್ನು ಸ್ಮರಿಸು. ನಿತ್ಯಕರ್ಮ ಮುಗಿಸಿ ದೇವರ ಕೋಣೆಯನ್ನು ಪ್ರವೇಶಿಸಿ ನಮಸ್ಕರಿಸಿ ಹೃತ್ಪೂರ್ವಕ ಪ್ರಾರ್ಥಿಸು. ದುಃಖದ ಆವೇಗ ಒತ್ತರಿಸಿಬಂದಾಗ ನಿನಗೆ ಇಷ್ಟಬಂದ ರೀತಿಯಲ್ಲೇ ದೇವರಲ್ಲಿ ಮೊರೆ ಇಡು. ಸ್ವಲ್ಪ ತಾಳು. ಎಂದೂ ದುಡುಕಬೇಡ. ಅಮೂಲ್ಯ ವಾದುದು ಜೀವನ. ಮಾನವಳಾಗಿ ಹುಟ್ಟಿಬಂದು ಜೀವಿತದ ಗುರಿಯನ್ನು ಮರೆಯಬೇಡ.

\vskip 2pt

‘ಕಾರಣವಿಲ್ಲದೇ ಕಷ್ಟ ಬರುವುದಿಲ್ಲ. ಎಂದೂ ಭಗವಂತ ಜೀವಿಗಳನ್ನು ಕಷ್ಟದ ಕೊಠಡಿಯಲ್ಲಿ ನೂಕಿ ತಮಾಷೆ ನೋಡುವುದಿಲ್ಲ. ಆತ ಸದಾ ನಿರ್ಲಿಪ್ತನಾಗಿ ಎಲ್ಲವನ್ನೂ ನೋಡುತ್ತಾನೆ. ಯಾವ ಜೀವಿ ಸಹಾಯಕ್ಕಾಗಿ ಹಾತೊರೆಯುವುದೋ ಆ ಜೀವಿಗೆ ಆ ಸಂದರ್ಭಕ್ಕೆ ತಕ್ಕಂತೆ ಎಲ್ಲ ರೀತಿಯಿಂದ ಸಹಾಯ ಮಾಡುತ್ತಾನೆ. ಹಾಗಾಗಿ ಭಗವಂತನಲ್ಲಿ ದೃಢ ಶ್ರದ್ಧೆ ಇಡಬೇಕು. ಪ್ರಾರಬ್ಧವನ್ನು ರದ್ದು ಮಾಡೆಂದು ಆತನಲ್ಲೇ ಬಿಡದೆ ಪ್ರಾರ್ಥಿಸಬೇಕು. ದಟ್ಟವಾಗಿ ಕವಿದ ಮೋಡದ ವಾತಾವರಣದಂತೆ, ಸುತ್ತಲೂ ಕವಿದ ಹೊಗೆಯ ವಾತಾವರಣವಿದ್ದಂತೆ ಈ ಸಂಸಾರ. ಇಲ್ಲಿ ಎಂದೂ ನಿಜವಾದ ಸುಖ ಸಂತೋಷ ದೊರೆಯದು. ಯಾರಿಗಾದರೂ ಹೀಗೆಂದು ಬರಿದೆ ಹೇಳಿದರೆ ಅವರು ಅದನ್ನು ಥಟ್ಟನೆ ನಂಬುವುದಿಲ್ಲ. ಹಾಗಾಗಿಯೇ ಸ್ವಂತ ಅನುಭವದಿಂದ ಸಂಸಾರವೇನೆಂದು ತಿಳಿಯಬೇಕಾಗುವುದು. ಹೀಗೆ ಮನಸ್ಸಿಗೆ ಅದರ ಅಲ್ಪತೆ ದೃಢವಾದ ಬಳಿಕ ನಾವು ಅದರ ಆಕರ್ಷಣೆಯಿಂದ ಖಂಡಿತ ಪಾರಾಗಬಹುದು.

\vskip 2pt

‘ಕಷ್ಟವೆಂಬ ಕಗ್ಗಂಟು ಅಷ್ಟೊಂದು ಜಟಿಲವಾಗಿದೆ. ಅದನ್ನು ಬಿಡಿಸಿ ಬಿಡಿಸಿ ನಿವಾರಿಸಲು ಸಾಧ್ಯವಿಲ್ಲ. ಆದರೆ ಭಗವಂತನ ನಾಮ ಮತ್ತು ಪ್ರಾರ್ಥನೆ ಎಂಬ ದಿವ್ಯಾಗ್ನಿಯಲ್ಲಿ ಆ ಕಗ್ಗಂಟು ಕರಗುವುದು. ಇದೇ ನಿಜವಾದ ಯಶಸ್ಸಿನ ಗುಟ್ಟು. ಆದರೆ ಅದಕ್ಕೆ ಕಾಲಾವಕಾಶಬೇಕು. ಸಾವಧಾನದಿಂದಲೇ ಇಷ್ಟು ವರ್ಷ ಸಂಸಾರದಲ್ಲಿ ಬೆಂದೆ, ನೊಂದೆ. ಈಗ “ಖಂಡಿತ ಶಾಂತಿ ದೊರೆಯುತ್ತದೆ–ಹೀಗೆ ಮಾಡಿದರೆ” ಎಂಬ ನೂತನ ಮಾರ್ಗ ಹಿಡಿದು ನೋಡು. ಊಟದ ನಂತರ ತೇಗಿನಲ್ಲಿ ಆಗಾಗ ಉಂಡ ಆಹಾರದ ರುಚಿ ತೋರುವಂತೆ ಮುಂದೆಯೂ ಸ್ವಲ್ಪಕಾಲ ಹಿಂದಿನ ಕಷ್ಟಗಳ ನೆನಪಾಗುವುದಾದರೂ ಹೆದರಬೇಡ. ಮುಂದೆ ಖಂಡಿತ ಒಳಿತಾಗುವುದು.’

\vskip 2pt

ಇನ್ನೊಂದು ಘಟನೆ ಇಂತಿದೆ–

\vskip 2pt

ವರ್ಷ ಮೂವತ್ತೈದು. ರಾತ್ರಿ ಮಲಗಿಕೊಂಡಿದ್ದಾಗ ಬಿರುಗಾಳಿ ಬೀಸಿ ಮನೆಯ ಎದುರಿಗಿದ್ದ ಮರ ಮುರಿದು ಬಿತ್ತು. ಗಂಡ ಹೆಂಡತಿ ಮಕ್ಕಳು ಮಲಗಿದ್ದಾರೆ. ಬೇರಾರಿಗೂ ಯಾವ ತೊಂದರೆ\-ಯಾಗದೇ ಆ ಮೂವತ್ತೈದು ವರ್ಷ ದಾಟಿದ ಐದು ಮಕ್ಕಳ ತಾಯಿಯಾದ ಹೆಂಗಸಿನ ಕಾಲಿಗೆ ಮರದ ಗೆಲ್ಲು ಬಡಿದು ವಿದ್ಯುದಾಘಾತವಾಯಿತು. ಸೊಂಟದ ಭಾಗ, ಕಾಲುಗಳು ನಿರ್ಜೀವವಾದುವು. ಔಷಧೋಪಚಾರಗಳಿಂದ ಪ್ರಯೋಜನವಾಗಲಿಲ್ಲ. ಮಲಗಿದ್ದಲ್ಲಿಂದ ಏಳುವಂತಿಲ್ಲ. ಸಂಪೂರ್ಣ ಪರಾಧೀನ ಸ್ಥಿತಿ. ಆಕೆಯೇ ಎನ್ನುವಂತೆ–

\vskip 2pt

‘.... ಈಗ ಕತ್ತಿನಿಂದ ಕೆಳಭಾಗವನ್ನೆಲ್ಲಾ ನೋವು ಆಕ್ರಮಿಸಿ ಬಹಳ ಯಾತನೆಯನ್ನು ಅನುಭವಿಸುತ್ತಿದ್ದೇನೆ. ಇನ್ನೂ ನನ್ನ ಪ್ರಾರಬ್ಧ ಮುಗಿದಿಲ್ಲವೇ? ಇತ್ತೀಚೆಗೆ ಜ್ವರವೂ ಪ್ರಾರಂಭವಾಗಿದೆ. ನಾನು ಬಹಳ ಬಳಲಿ ಹೋಗಿದ್ದೇನೆ. ದಾರಿದ್ರ್ಯದ ಬೆಂಕಿಯ ಜೊತೆಗೆ, ಯಾತನೆಯ ಬೆಂಕಿ ಸೇರಿ ದಹಿಸುತ್ತಿದೆ. ಹಗಲಿರುಳು ಸಾವಿಗಾಗಿ ಬೇಡುತ್ತಿದ್ದೇನೆ. ಬದುಕುವ ಆಸೆ ಸ್ವಲ್ಪವೂ ಇಲ್ಲ.

‘ಹಿರಿಯಕ್ಕ ವ್ರತ, ನೇಮ, ತೀರ್ಥಯಾತ್ರೆ ಮಾಡಿದಳು. ಇನ್ನೊಬ್ಬಳು ಹಿರಿಯರ ಸೇವೆಯನ್ನು ಮಾಡಿ ಪುಣ್ಯಸಂಪಾದಿಸಿಕೊಂಡಳು. ಆದರೆ ನನಗೆ ತೀರ್ಥಯಾತ್ರೆ ಹೋಗಲಿ, ತೀರ್ಥರೂಪರ ದರ್ಶನ ಭಾಗ್ಯವೇ ಇಲ್ಲ. ಕೊನೆಯದಾಗಿ ಪತಿಯ ಸೇವೆಯನ್ನಾದರೂ ಮಾಡಲು ಕಾಲುಗಳೇ ಇಲ್ಲ. ಈ ಎಲ್ಲ ಕಷ್ಟಗಳ ನಡುವೆ ನಾನೇಕೆ ಬದುಕಿ ಉಳಿಯಬೇಕೋ ತಿಳಿಯದು.’

ಜೀವನ ಇಷ್ಟೊಂದು ಘೋರ ಎಂದಾಕೆ ತಿಳಿದಿರಲಿಲ್ಲ. ಆಕೆಯ ಈ ನರಳಾಟವನ್ನು ಕಂಡ ಬಂಧುಗಳ ಮಾತಿನಂತೆ–

‘ಬಿದ್ದಲ್ಲಿಂದ ಕದಲದೇ ಆರು ವರ್ಷಗಳಾದವು. ಮೈಯಲ್ಲಿ ಎರಡು ಮೂರು ದೊಡ್ಡ ಗಾತ್ರದ ಬೆಡ್​ಸೋರ್​ಗಳಾಗಿವೆ. ಮಗ್ಗಲು ಮಗಚಲು ಆಗುವುದಿಲ್ಲ. ಇಡೀ ಮೈಯಲ್ಲಿ ಅಸಹನೀಯ ನೋವು. ಅವಳ ಗಂಡನ ಆರ್ಥಿಕ ಪರಿಸ್ಥಿತಿ ಮತ್ತಷ್ಟು ಹದಗೆಟ್ಟಿದೆ. ನಾವೆಲ್ಲ ಸಾಧ್ಯವಾದಷ್ಟು ಸಹಾಯ ಮಾಡುತ್ತಿದ್ದೇವಾದರೂ ಅದು ಏನೇನೂ ಸಾಕಾಗುತ್ತಿಲ್ಲ. ಮಕ್ಕಳಿಗೆ ಶಾಲೆಯಲ್ಲಿ ಬಹು ಮಾನವಾಗಿ ದೊರೆತ ಸಣ್ಣಪುಟ್ಟ ಸಾಮಾನುಗಳನ್ನು ಮಾರಿ ತಿನ್ನುವ ಸ್ಥಿತಿ ಎಂದು ಹೇಳಿದರೆ ಸಮಸ್ಯೆಯ ದಾರುಣತೆ ಸ್ಪಷ್ಟವಾಗಬಹುದು. ಈ ಕಷ್ಟದಲ್ಲಿಯೂ ಅವಳ ಗಂಡ ಹಗಲುರಾತ್ರಿ ದುಡಿಯುತ್ತಿದ್ದಾರೆ; ರೋಸಿ ಓಡಿ ಹೋಗಿಲ್ಲ ಎಂಬುದು ದೊಡ್ಡಸಂಗತಿ. ಆಗಿರುವ ದುರಂತ ದಾರುಣವೆ ಆದರೂ ಇನ್ನೂ ಹೆಚ್ಚಿನ ದುರಂತವಾಗದಿರಲಿ ಎನ್ನುವ ಆಸೆಯಿಂದ ನಾವು ಸಾಧ್ಯವಿದ್ದಷ್ಟು ಸಹಾಯ ಮಾಡುತ್ತಿದ್ದೇವೆ. ಆದರೆ ನಮ್ಮ ಸಹಾಯ ಅವರ ಆವಶ್ಯಕತೆಯ ಮುಂದೆ ಯಾವ ಲೆಕ್ಕಕ್ಕೂ ಬರುವಂತಿಲ್ಲ. ಮನುಷ್ಯ ಸಹಾಯದ ಪರಿಮಿತಿ ಎಷ್ಟು ಅಲ್ಪ ಎನ್ನುವುದು ಇದರಿಂದ ಸ್ಪಷ್ಟವಾಗದಿರದು.’

ಈ ಸಂಕಟದಿಂದ ಪಾರಾಗುವ ಬಗೆ ಹೇಗೆ? ದಾರಿ ತೋರಿದ ಹಿರಿಯರ ನುಡಿ ಮುತ್ತುಗಳಿವು–

‘ಕಾಲಚಕ್ರದ ಉರುಳಾಟದಲ್ಲಿ ದುಃಖ, ಸಂತಸ ಸಹಜ. ಕಷ್ಟಪರಂಪರೆಯೇ ಬಂದರೂ ಧೃತಿಗೆಡಬಾರದು. ಶ್ರದ್ಧೆಯಿಂದ ನಾಮಸ್ಮರಣೆ ಮಾಡು. ಹಿಂದಾದುಕ್ಕೆ ಚಿಂತಿಸಬೇಡ. ಭಗವಂತ ಯಾರನ್ನೂ, ಎಂದೂ, ಎಂತಹ ಸಂದರ್ಭದಲ್ಲೂ ಕೈಬಿಡುವುದಿಲ್ಲವೆಂಬ ನಿತ್ಯಸತ್ಯವನ್ನು ಮರೆಯ\-ಬೇಡ. “ತಲ್ಲಣಿಸದಿರು ಕಂಡ್ಯ ತಾಳು ಮನವೇ, ಎಲ್ಲರನು ಸಲಹುವನು ಇದಕೆ ಸಂಶಯವಿಲ್ಲ” ಎಂಬ ದಾಸವಾಣಿ ಸತ್ಯಸ್ಯ ಸತ್ಯ. ಹೆದರುವುದೇಕೆ? ನಮ್ಮ ಎಲ್ಲ ದುಃಖ, ದುಮ್ಮಾನಗಳಿಗೂ ಕಾರಣ ಉಂಟು. ಅಂತೆಯೇ ಸತ್ಕರ್ಮ ಮಾಡಿದ್ದಕ್ಕೆ ಖಂಡಿತ ಫಲ ಉಂಟು. ದೇವರು, ಕರ್ಮವು ನಿಯಮದಂತೆ ಕೆಲಸ ಮಾಡಲು ಪ್ರೇರಕ. ಜೀವನದಲ್ಲಿ ಬರುವ ನೋವನ್ನು ಭಗವಂತನ ಕಡೆ ಮುಂದುವರಿಯಲು ಕೊಟ್ಟವರವೆಂದು ಭಾವಿಸಿ ಮುಂದುವರಿ. ಬೆಂಕಿಯಲ್ಲಿ ಬೆಂದ ಬಾಳು ಹಸನಾಗುವುದು–ಪುಟಕ್ಕಿಟ್ಟ ಚಿನ್ನದಂತಾಗುವುದು. ಉಜ್ವಲ ಭವಿಷ್ಯ ಕಾದಿದೆ. ಭಯ ಬೇಡ. ಎಲ್ಲ ದುಃಖ, ಸಂಕಟಗಳಿಗೂ ಶಾಶ್ವತ ಪರಿಹಾರ ಒಂದೇ ಒಂದು–ಭಗವಂತನ ಪಾದದಲ್ಲಿ ಶರಣು ಹೊಂದುವುದು. ತಾಯಿಗೆ ಮಕ್ಕಳ ಕಷ್ಟ ಕಂಡು ದುಃಖವಾಗುವುದಿಲ್ಲವೆ? ತಾಯಿ ಕರುಣಾಮಯಿ ಅಲ್ಲವೆ? ಭಗವಂತನ ನಾಮಸ್ಮರಣೆ ಮುಂದುವರಿಸು. ಭಗವಂತನಿಗೆ ಎಲ್ಲ ವಿಚಾರ ತಿಳಿದಿದೆ. ನಿರಂತರ ಪ್ರಾರ್ಥನೆಯೊಂದೇ ದಾರಿ. ಪ್ರಾರ್ಥನೆ, ಸಚ್ಚಿಂತನೆ, ಸದ್ಗ್ರಂಥಗಳ ಅಧ್ಯಯನ, ಸತ್ಕರ್ಮಾಚರಣೆ ಮಾಡುತ್ತ ಹೋಗುವುದಲ್ಲದೇ ಮನುಷ್ಯನ ಶ್ರೇಯಸ್ಸಿಗೆ ಬೇರಾವ ದಾರಿ ಇದೆ?’

ಮೇಲಿನ ಸಾಂತ್ವನದ ವಾಣಿಯನ್ನು ನೀಡಿದವರು ದೀರ್ಘಕಾಲ ಅಧ್ಯಾತ್ಮದ ಪಥದಲ್ಲಿ\break ಮುನ್ನಡೆ\-ದವರು. ನೂರಾರು ಜನರ ಹೃದಯದಲ್ಲಿ ಉತ್ಸಾಹ ಚೈತನ್ಯ ತುಂಬಿದವರು. ನೂರಾರು ಜನರ ಪರಿಹರಿಸಲಾಗದೆಂದು ತೋರಿದ ಕಷ್ಟಕಂಟಕಗಳನ್ನು ಕಂಡು, ಮರುಗಿ, ಹೃತ್ಪೂರ್ವಕ ಪ್ರಾರ್ಥನೆಯನ್ನು ಮಾಡಿದವರು. ಪ್ರಾರ್ಥನೆಯಿಂದ ಪರಿವರ್ತನೆ ಸಾಧ್ಯ ಎಂಬ ವಿಷಯದಲ್ಲಿ ಅನುಭವದಿಂದ ವಿಶ್ವಾಸವನ್ನು ದೃಢಪಡಿಸಿಕೊಂಡವರು. ಯೋಚಿಸಿ ಮಾತನಾಡುವ ಸ್ವಭಾವವುಳ್ಳವರು. ಸ್ವಾರ್ಥ, ಸ್ವಲಾಭಚಿಂತನೆ ಇಲ್ಲದವರು.

ನಾಸ್ತಿಕರಿಗೆ, ವಿಚಾರವಾದಿಗಳಿಗೆ, ವೈಜ್ಞಾನಿಕ ಮನೋವೃತ್ತಿಯವರಿಗೆ ಮೇಲಿನ ಮಾತು ಪಥ್ಯ\-ವೆನಿಸದು. ಇಂಥ ಸಮಸ್ಯೆಗೆ ಈ ವಿಚಾರವಾದಿಗಳು ಏನೆಂದು ಸಾಂತ್ವನ ನೀಡಿಯಾರು? ಮೂಢ ನಂಬಿಕೆ, ಶೋಷಕ–ಶೋಷಣೆಗಳ ಹೊದಿಕೆಯಲ್ಲಿ ಇಂಥ ಸಮಸ್ಯೆಗಳನ್ನು ಮುಚ್ಚಿಡಲು ಸಾಧ್ಯ\-ವಿಲ್ಲ\-ವಷ್ಟೆ!


\section*{ಸಂಪತ್ತಿದೆ, ಸುಖವಿಲ್ಲ}

\addsectiontoTOC{ಸಂಪತ್ತಿದೆ, ಸುಖವಿಲ್ಲ}

ಸುಖವು ಸಂಪತ್ತನ್ನು ಅವಲಂಬಿಸಿಲ್ಲ ಎನ್ನಲು ಇಲ್ಲಿ ಕೆಲವು ನಿದರ್ಶನಗಳಿವೆ–

ಸಮಾಜದಲ್ಲಿ ಒಬ್ಬ ಗಣ್ಯ ವ್ಯಕ್ತಿಗಳಾಗಿ, ಹಲವಾರು ಸಂಘ ಸಂಸ್ಥೆಗಳಲ್ಲಿ ಪದಾಧಿಕಾರಿಗಳಾಗಿ, ಸಮಾಜಕ್ಕಾಗಿ ದುಡಿದು, ಹತ್ತಾರು ಕುಟುಂಬಗಳ ಉದ್ಧಾರಕ್ಕೆ ಕಾರಣೀಭೂತರಾದವರು ಅವರು. ಒಳ್ಳೆಯ ಕುಟುಂಬದಲ್ಲಿ ಜನಿಸಿದ ಶ‍್ರೀಯುತರು ದೈವ-ಧಾರ್ಮಿಕ ಕಾರ್ಯಗಳನ್ನು ನಡೆಯಿಸುತ್ತ ಬಂದಿರುವ ಪುಜುಜೀವಿಗಳು. ಕಷ್ಟದ ದುಡಿಮೆಯಿಂದ ಅವರು ಆರ್ಥಿಕವಾಗಿ ತಮ್ಮ ಮನೆತನವನ್ನು ಶ‍್ರೀಮಂತಿಕೆಯ ಮಟ್ಟಕ್ಕೆ ಏರಿಸಿದ್ದರು. ಆದರೆ ಒಬ್ಬ ಸಹೋದರ, ನಾಲ್ಕು ಗಂಡು, ಒಂದು ಹೆಣ್ಣು ಮಕ್ಕಳಿರುವ ಅವರ ತುಂಬ ಸಂಸಾರ ನಾಟಕೀಯವೋ ಎಂಬಂತೆ ಇತ್ತೀಚೆಗೆ ಒಡೆದು ಬೇರೆಬೇರೆಯಾಯಿತು.

ಇಂದು ಆ ಹಿರಿಯರು ಹೆಂಡತಿ ಮಕ್ಕಳಿಂದ ದೂರವಾಗಿ, ಮನೆಯಿಂದ ದೂರವಾಗಿ ಇರು\-ವಂತಾಗಿದೆ. ಇಂದು ಅವರು ತೀವ್ರವಾದ ಮಾನಸಿಕ ಅಶಾಂತಿಯಿಂದ ಬಳಲುವಂತಾಗಿದೆ. ಅವರೊಬ್ಬರೇ ಅಲ್ಲದೆ ಅವರ ಕುಟುಂಬದ ಸದಸ್ಯರೆಲ್ಲರೂ, ಎಲ್ಲ ಅನುಕೂಲವಿದ್ದೂ,\break ಒಂದಿಲ್ಲೊಂದು ಮಾನಸಿಕ ಅವಸ್ಥೆಯಿಂದ ಬಳಲುತ್ತಿದ್ದಾರೆ. ಇಂದು ತಂದೆಮಕ್ಕಳು ನ್ಯಾಯಾ\-ಲಯದ ಕಟ್ಟೆ ಹತ್ತಿ ಹೊಡೆದಾಡುವ ಪರಿಸ್ಥಿತಿ ಬಂದಿದೆ. ಇದು ಏಕೆ ಹೀಗಾಯಿತು? ಹೀಗಾಗುತ್ತಿದೆ? ಎದೆಗುದಿಯೇ ಗತಿಯೇ ಅವರಿಗೆ? ಇಂತಹ ಪರಿಸ್ಥಿತಿಯಲ್ಲಿ, ಆ ಹಿರಿಯರಿಗೆ ಮಾನಸಿಕ ಶಾಂತಿ, ಸಮಾಧಾನಗಳು ದೊರೆಯುವ ದಾರಿ ಯಾವುದು? ಸುಖ ಕೊಡದ ಅವರ ಸಂಪತ್ತಿಗೇನು ಬೆಲೆ?

ತಂದೆ ಒಳ್ಳೆಯ ಅಧ್ಯಾಪಕರೆಂದು ಹೆಸರಾಂತ ಪ್ರೊಫೆಸರ್. ನ್ಯಾಯಶಾಸ್ತ್ರ ಬೋಧಿಸುತ್ತಾರೆ. ಮೂರನೇ ತರಗತಿಯಲ್ಲಿರುವಾಗ ಅಂಗಡಿಯಿಂದ ಒಂದು ಗಾಜಿನ ಚೂರನ್ನು ಕೊಂಡುಬಂದ ನಂತೆ ಹುಡುಗ. ಅಂದಿನಿಂದ ಅವನ ತಾಯಿ ಹೇಳುತ್ತಾರೆ: ‘ಪ್ರಾರಂಭವಾಯಿತು ಗ್ರಹಚಾರ’ ಅಂತ. ಈ ಆರೇಳು ವರ್ಷಗಳಿಂದ ಅವನ ಬುದ್ಧಿ ನೆಟ್ಟಗಿಲ್ಲ. ಯಾರಿಗೂ ತಿಳಿಯದಂತೆ ಮೂರು ಮೂರು ಅಲ್ಸೇಷನ್ ನಾಯಿಗಳಿರುವ ಮನೆಯಿಂದ ವಾಚು, ಟೇಪ್ ರೆಕಾರ್ಡರ್ ಹಾರಿಸಿಕೊಂಡು ಬಂದಿದ್ದಾನೆ. ಹೇಗೆ–ಎಂಬುದು ಯಾರಿಗೂ ತಿಳಿಯದ ರಹಸ್ಯ. ಇನ್ನೂ ಕುಚೋದ್ಯದ ಸಂಗತಿ: ಯಾರ್ಯಾರ ಮನೆಯ ಹತ್ತಿರ ಸುತ್ತಮುತ್ತ ಮಲವಿಸರ್ಜನೆ ಮಾಡಿ ಬಂದುಬಿಡುತ್ತಾನೆ. ಯಾರದೋ ಮನೆಯ ಬಾಗಿಲು ತಟ್ಟುತ್ತಾನೆ. ಬಾಗಿಲು ತೆರೆದರೆ ‘ಸಾರಿ ಕ್ಷಮಿಸಿ, ತಪ್ಪಾಗಿ ಈ ಮನೆಗೆ ಬಂದೆ’ ಎನ್ನುತ್ತಾನೆ. ಮನೆಯಲ್ಲಿ ಯಾರೂ ಇಲ್ಲದಿರುವುದನ್ನು ಕಂಡರೆ, ಕಿಟಿಕಿಯ ಸರಳನ್ನು ಬಗ್ಗಿಸಿ, ಮನೆಯೊಳಗೆ ಪ್ರವೇಶಿಸಿ, ಸಾಮಾನುಗಳನ್ನು ಅಲ್ಲಿಂದ ಸಾಗಿಸಿ, ಎಲ್ಲೋ ಬಚ್ಚಿಡುತ್ತಾನೆ. ತಂದೆತಾಯಿಗಳನ್ನೇ ಅಶ್ಲೀಲ ಮಾತುಗಳಲ್ಲಿ ಬೈಯ್ಯುವುದರಲ್ಲಿ ನಿಸ್ಸೀಮ. ಸಾಲದ್ದಕ್ಕೆ ದಿನವೂ ಏನೇನು ಮಾಡಿದೆನೆಂಬ ಬಗ್ಗೆ ದಿನಚರಿಯನ್ನೂ ಬರೆದಿಡುತ್ತಾನೆ. ಇನ್ನೊಂದು ಆಶ್ಚರ್ಯವೆಂದರೆ, ತನ್ನ ಕ್ಲಾಸಿನ ಹುಡುಗಿಯೊಡನೆ ಪತ್ರವ್ಯವಹಾರ ಮಾಡಿ ಅವಳ ಸಖ್ಯವನ್ನೂ ಗಳಿಸಿದ್ದಾನಂತೆ! ಹೊಗೆಬತ್ತಿ ಸೇದಲು ಚೆನ್ನಾಗಿ ಕಲಿತಿದ್ದಾನೆ. ಇವನನ್ನು ಸೈಕಿಯಾಟ್ರಿಸ್ಟ್ ಯಾನೆ ಮನೋರೋಗತಜ್ಞರಲ್ಲಿಗೆ ಕರೆದೊಯ್ಯಲಾಗಿತ್ತು. ಅವರು ಇವನ ಮಾನಸಿಕ ಸ್ಥಿತಿಯನ್ನು \enginline{Socio- path} ಎಂದು ಹೆಸರಿಸಿ ಔಷಧವೇನನ್ನೂ ಕೊಡದೆ, ‘ಪ್ರೀತಿ ಹಾಗೂ ತಾಳ್ಮೆಯಿಂದ ಅವನನ್ನು ತಿದ್ದುತ್ತ ಹೋಗಿ. ಎಲ್ಲ ಸಕಾಲದಲ್ಲಿ ಸರಿಯಾಗುತ್ತದೆ’ ಎಂದು, ತಮಗೆ ತಿಳಿದ ಚಿಕಿತ್ಸೆಯನ್ನು ಸೂಚಿಸಿದರು. ಒಂದೆರಡು ಬಾರಿ ಇವನ ಕಾಟ ತಡೆಯಲಾರದೆ, ಮನೋರೋಗ ಚಿಕಿತ್ಸಾಲಯಕ್ಕೆ ಸೇರಿಸಲಾಯಿತು. ಅಲ್ಲಿ ರಕ್ಷಣಾಧಿಕಾರಿಗಳ ಕಣ್ಣುತಪ್ಪಿಸಿ ಓಡಲು ಯತ್ನಿಸಿ ಸಿಕ್ಕಿಕೊಂಡ. ಇನ್ನೊಮ್ಮೆ ಚಿಕಿತ್ಸಾಲಯದಲ್ಲಿದ್ದಾಗ, ಮೈಗೆ ಬೆಂಕಿ ಹಚ್ಚಿಕೊಳ್ಳುವ ವಿಫಲಯತ್ನ ನಡೆಯಿಸಿದ್ದ. ಇತ್ತೀಚೆಗೆ ತಂದೆತಾಯಿಗಳನ್ನು ಸಂಬೋಧಿಸಿ ಹೇಳುತ್ತಾನೆ: ‘ಪರೀಕ್ಷೆಯ ಫಲಿತಾಂಶ ಬರಲಿ, ಮುಗಿಸಿಬಿಡುತ್ತೇನೆ ನಿಮ್ಮನ್ನು’ ಎಂದು!

ತಜ್ಞರು ಇವನನ್ನು ಪ್ರೀತಿಯಿಂದ ನೋಡಿಕೊಳ್ಳಿ ಎನ್ನುತ್ತಾರೆ!

ಅಜ್ಜಿ ಈ ಹುಡುಗನ ಲೀಲೆಯನ್ನು ಕಂಡು, ತಾನು ಬಾಲ್ಯದಲ್ಲಿ ಕೇಳಿದ್ದ ಗಾದೆಯೊಂದನ್ನು ಆಗಾಗ ಮೆಲುಕಾಡುವುದುಂಟು:

‘ಅಯ್ಯೋ! ಪೂರ್ವಜನ್ಮದ ಇಕ್ಕಳು, ಈ ಜನ್ಮದ ಮಕ್ಕಳು; ನಾವು ಪಡೆದುಕೊಂಡು ಬಂದದ್ದು. ಏನು ಮಾಡುವುದು?’ ಪೂಜೆ ಪ್ರಾರ್ಥನೆಗಳಿಂದ, ಹರಕೆ ದಾನಗಳಿಂದ, ಒಳಿತಾಗುವುದೆಂದು ಹಲವಾರು ಮಂದಿ ಹೇಳುತ್ತಿದ್ದಾರೆ. ಒಬ್ಬೊಬ್ಬರು ಒಂದೊಂದು ಸ್ಥಳದ, ದೇವಾಲಯದ, ಸಂತರ ಮಾಹಾತ್ಮ್ಯ ಹೇಳುತ್ತಾರೆ. ತಂದೆಗೆ ಈ ವಿಚಾರದಲ್ಲಿ ಆಸಕ್ತಿ ಇಲ್ಲ. ಅವರು ವೈಜ್ಞಾನಿಕ ಮನೋಭಾವದವರು. ತಾಯಿ ಯಾವ ದೇವರಿಗೆ ಏನು ಹರಕೆ ಸಲ್ಲಿಸಲಿ ಎಂದು ಯೋಚಿಸುತ್ತ\-ಲಿದ್ದಾರೆ. ಹುಡುಗ ಮನೆಮಂದಿಗೆಲ್ಲಾ ಸದಾ ತಲೆನೋವಾಗಿ ಕಾಡುತ್ತಲೇ ಇದ್ದಾನೆ!

ಈ ಅನಿವಾರ್ಯದಿಂದ ಪಾರಾಗಲು ಸಾಧ್ಯವಿಲ್ಲವೆ? ಅನುಭವಿಸುತ್ತಲೇ ಇರಬೇಕೆ? ಪಾರು ಆಗಲು ದಾರಿ ಇಲ್ಲವೆಂದಲ್ಲ. ಮನಸ್ಸಿದ್ದರೆ ಮಾರ್ಗವಿದ್ದೇ ಇದೆ.


\section*{ಸಂಶಯಪಿಶಾಚಿ}

\addsectiontoTOC{ಸಂಶಯಪಿಶಾಚಿ}

ಸಂಶಯ ಚಿಂಕ್ರೋಭದ ಸಂಗಾತಿ. ಸಂಶಯರಾಕ್ಷಸನೊಮ್ಮೆ ಮನದೊಳಗೆ ಇಣುಕಿದನೆಂದರೆ, ಮನುಷ್ಯನ ಅವಸ್ಥೆ ಹೇಗೆ ದುರಂತವನ್ನು ಸಮೀಪಿಸುತ್ತದೆನ್ನಲು ಈ ಘಟನೆಗಳೇ ಸಾಕ್ಷಿ–

ದಿನೇಶ ನೋಡಲು ಎತ್ತರ. ಸುಂದರಕಾಯ. ‘ಒಳ್ಳೆಯ ಹೈಟ್ ಉಂಟಯ್ಯ ನಿನಗೆ’ ಎಂದು ಮಿತ್ರರು ಆತನಿಗೆ ಹೇಳುತ್ತಿರುತ್ತಾರೆ. ಪೋಲೀಸ್ ಇಲಾಖೆಯಲ್ಲಿ ಉನ್ನತ ಹುದ್ದೆಯೇ ಸಿಕ್ಕಿತ್ತು ಆತನಿಗೆ. ದಿನೇಶ ಒಮ್ಮೆ ಪತ್ನಿಯ ಜೊತೆ ತರಕಾರಿ ಕೊಂಡುಕೊಳ್ಳಲು ಮಾರುಕಟ್ಟೆಗೆ ಹೋಗಿದ್ದ. ಈತ ನಿಂತ ಜಾಗದ ಹತ್ತಿರವಿದ್ದ ಅಂಗಡಿಯಲ್ಲಿದ್ದ ತರುಣಿಯೊಬ್ಬಳು, ಅವನನ್ನು ದಿಟ್ಟಿಸಿ ನೋಡಿ ನಕ್ಕಳು. ಅವನೂ ನಕ್ಕನು. ಇದನ್ನು ಗಮನವಿಟ್ಟು ನೋಡುತ್ತಿದ್ದ ದಿನೇಶನ ಪತ್ನಿ ಶ‍್ರೀಮತಿಯ ಮನಸ್ಸನ್ನು ಸಂಶಯಪಿಶಾಚಿ ಪ್ರವೇಶಿಸಿತು. ಆಕೆಗೆ ತಲೆಬಿಸಿ, ಹೊಟ್ಟೆಸಂಕಟ ಪ್ರಾರಂಭವಾಯಿತು. ಆ ಬಿಕನಾಸಿ ತನ್ನ ಯಜಮಾನರ ಯಾರೋ ಗರ್ಲ್​ಫ್ರೆಂಡ್ ಇರಬೇಕೆಂದು ಊಹಿಸಿದಳು. ಅಂದಿನಿಂದ ಏಕಪ್ರಕಾರವಾಗಿ ಬೈಗಳ ಮಾತಿನಿಂದ ಗಂಡನನ್ನು ಹಿಂಸಿಸತೊಡಗಿದ್ದಾಳೆ. ಎಷ್ಟು ವಿಧವಾಗಿ ಸತ್ಯ ಏನೆಂಬುದನ್ನು ತಿಳಿಸಿದರೂ ನಂಬಲಾರಳು. ಕೆಲವೊಮ್ಮೆ ಯಾರ ಮನೆಗೋ ಫೋನ್ ಮಾಡಿ ಅಲ್ಲಿನ ಹೆಂಗಸರನ್ನು ಗದರಿಸುವುದುಂಟು. ‘ಎಷ್ಟು ಹಣ ತೆಗೆದು ಕೊಂಡು ನನ್ನ ಯಜಮಾನರ ಮನಸ್ಸನ್ನು ಕೆಡಿಸಿ, ಮನೆ ಹಾಳುಮಾಡಿದೆಯಲ್ಲಾ ಗಯ್ಯಾಳಿ’– ಎಂದು ಫೋನಿನಲ್ಲಿ ಗದರಿಸಿ ಬೈಯ್ಯುತ್ತಾಳೆ! ಒಮ್ಮೆ ದಿನೇಶನು ಕ್ಲಬ್ಬಿನಲ್ಲಿ ಮಹಿಳೆಯೊಬ್ಬಳ ಜೊತೆ ಮಾತನಾಡುತ್ತಿದ್ದಾಗ, ಈಕೆ ಅಲ್ಲಿಗೂ ಹೋಗಿ, ರಂಪಾಟ ಮಾಡಿ ಗುಲ್ಲೆಬ್ಬಿಸಿದಳು! ಪಾಪ! ದಿನೇಶ ಸೋತು ಸುಣ್ಣವಾಗಿ, ವಿಚ್ಛೇದನಕ್ಕೂ ಪ್ರಯತ್ನಿಸಿದನು. ಆದರೆ ಸಫಲವಾಗಲಿಲ್ಲ\enginline{– Happily married} ಆದರೆ \enginline{un-happily living} ಮುಂದುವರಿಯುತ್ತಲಿದೆ! ಇದು ಅಪರಿಹಾರ್ಯವೆ?

ಲಕ್ಷ್ಮಮ್ಮ ಮದುವೆಯಾಗದೇ ಉಳಿದರು. ಅವರಿಗೆ ಸ್ವಂತ ಮನೆ, ಪಿತ್ರಾರ್ಜಿತ ಆಸ್ತಿ ಇದೆ. ಮನೆವಾರ್ತೆಯ ಕೆಲಸ, ಪೂಜೆ, ಪಾಠ, ಓದಿನಲ್ಲಿ, ಅವರ ಸಮಯ ಕಳೆಯುತ್ತದೆ. ಅವರ ತಂಗಿ ತುಂಬ ಓದಿಕೊಂಡವರು. ಕಾಲೇಜೊಂದರಲ್ಲಿ ಅಧ್ಯಾಪಕಿ. ಅವರಿಗೂ ಮದುವೆಯಾಗಿಲ್ಲ. ಈ ಇಬ್ಬರು ಅಕ್ಕತಂಗಿಯರಿಗೂ, ನಲ್ವತ್ತು ವಯಸ್ಸು ದಾಟುತ್ತಿರುವ, ಒಳ್ಳೆಯ ಉದ್ಯೋಗದಲ್ಲಿರುವ ತಮ್ಮ ಸಹೋದರನಿಗೆ ಮದುವೆ ಮಾಡಿಸುವ ಹಂಬಲ, ವಂಶ ಉಳಿಯಬೇಕೆಂಬ ಆಸೆ. ಎಷ್ಟೆಷ್ಟೋ ಪ್ರಯತ್ನ ಮಾಡಿದರು. ಸಮೀಪದ ಬಂಧುವೊಬ್ಬ, ಎಂಬತ್ತು ವಯಸ್ಸು ದಾಟಿದ ಮುದುಕ, ಕುಹಕ, ಕುತಂತ್ರಗಳಲ್ಲಿ ನಿಸ್ಸೀಮ. ಪ್ರತಿಬಾರಿಯೂ ಇನ್ನೇನು ಕೂಡಿಬಂದು, ಮದುವೆ ಆಗಿಯೇ ಹೋಯಿತು ಎನ್ನುವಷ್ಟರಲ್ಲೇ, ಸುಳ್ಳು, ಮೋಸದ ಸಾಕ್ಷಿಗಳಿಂದ ಮದುವೆ ನಡೆಯದಂತೆ ನೋಡಿ\-ಕೊಳ್ಳುತ್ತಾನೆ. ಅಕ್ಕತಂಗಿ ಅವನ ದೌರ್ಜನ್ಯವನ್ನು ಸಹಿಸಿ ಕಂಗಾಲಾಗಿದ್ದಾರೆ. ಸಹೋದರನೇ ‘ಅಯ್ಯೋ, ನನಗೆ ಮದುವೆ ಬೇಡಮ್ಮ!’ ಎನ್ನುತ್ತಿರುತ್ತಾನೆ. ಆದರೂ ಇವರು ಆಶಾವಾದಿಗಳು. ತಮ್ಮ ಅಸಹಾಯಕತೆಗೆ ಮರುಗುತ್ತಿದ್ದಾರೆ. ಈ ತೊಂದರೆಗೆ ಅಂತ್ಯವೇ ಇಲ್ಲವೆ?

ರಾಜಣ್ಣ ಮೆಚ್ಚಿದವಳನ್ನು ಮದುವೆಯಾಗುತ್ತೇನೆಂದಾಗ ತಾಯಿತಂದೆಗಳ ವಿರೋಧ. ಪಾಪ! ಅವರ ತಲೆಯಲ್ಲಿ ಏನೇನೋ ಸಂಶಯದ ಬೀಜ ಬಿತ್ತಿದ್ದಾರೆ. ಹಾಗಾಗಿ ಅವರೆನ್ನುತ್ತಾರೆ– ‘ಗೂಂಡಾಗಳಿಂದ ಹೊಡೆಸುತ್ತೇವೆ, ಮಂತ್ರಮಾಟ ಮಾಡಿ ಕಣ್ಣು ಕುರುಡಾಗುವಂತೆ ಮಾಡುತ್ತೇವೆ. ಆ ಹುಡುಗಿಯನ್ನು ಮದುವೆಯಾದರೆ ನಿನ್ನ ಸರ್ವನಾಶವಾಗುತ್ತದೆ. ಶಾಪ ಹಾಕುತ್ತೇನೆ’– ಎನ್ನುತ್ತಾರವರು. ರಾಜಣ್ಣ ಅರ್ಧ ಹುಚ್ಚನಂತಾಗಿ ಬಿಟ್ಟಿದ್ದಾನೆ. ಯಾರೋ ‘ಪ್ರಾರ್ಥನೆ ಮಾಡಿ ಒಳ್ಳೆಯದಾಗುತ್ತೆ’ ಎಂದರಂತೆ. ಅವನಿಗೆ ದೇವರು ದಿಂಡರು, ಪೂಜೆ ಪ್ರಾರ್ಥನೆಗಳಲ್ಲಿ ನಂಬಿಕೆ ಇಲ್ಲ ಎಂದೇನೂ ಅಲ್ಲ. ಅವನು ‘ಸೈಂಟಿಫಿಕಲೀ ಓರಿಯೆಂಟೆಡ್​’ (ವೈಜ್ಞಾನಿಕ ಮನೋಭಾವದವನು) ವ್ಯಕ್ತಿ. ಈಗ ವ್ಯಥಿತನಾಗಿದ್ದಾನೆ. ಈ ಪರಿಸ್ಥಿತಿಯಿಂದ ಬಿಡುಗಡೆ ಇಲ್ಲವೆ?


\section*{ನಾನೇನು ಮಾಡಲಿ?}

\addsectiontoTOC{ನಾನೇನು ಮಾಡಲಿ?}

‘ಪ್ರಾರ್ಥನೆ ಮಾಡಲು ಪ್ರಯತ್ನಿಸಿದೆ. ಆದರೂ ಸಾಧ್ಯವಾಗುತ್ತಿಲ್ಲ. ಮನಶ್ಶಾಂತಿಯೇ ಸಿಗುತ್ತಿಲ್ಲ. ಮನಸ್ಸಿನ ಸ್ತಿಮಿತ ತಪ್ಪುತ್ತಿದೆ. ಮನಸ್ಸಿನ ಚಂಚಲತೆಯನ್ನು ನನ್ನ ಹಿಡಿತಕ್ಕೆ ತರಲು ಸಾಧ್ಯ ವಾಗುತ್ತಿಲ್ಲ. ಎಂಥ ಸಣ್ಣಪುಟ್ಟ ಚಿಲ್ಲರೆ ವಿಷಯಗಳಿಗೂ ಸಿಟ್ಟು ಬಂದುಬಿಡುತ್ತದೆ. ಯದ್ವಾ ತದ್ವಾ ಮಾತಾಡಿಬಿಡುತ್ತೇನೆ. ಅನಂತರ ನನ್ನ ತಪ್ಪು ತಿಳಿಯುತ್ತದೆ. ಇಷ್ಟಾದರೂ ತಿದ್ದಿಕೊಳ್ಳಲು ಸಾಧ್ಯವಾಗುತ್ತಿಲ್ಲ. ಏನೆಲ್ಲ ಆಸೆ ಆಕಾಂಕ್ಷೆಗಳನ್ನಿಟ್ಟುಕೊಂಡು ಏನನ್ನೂ ಸಾಧಿಸಲಾಗದೆ ಬರಿಯ ಆಸೆಯನ್ನಷ್ಟೇ ಉಳಿಸಿಕೊಂಡು ಪ್ರತಿಯೊಂದರಲ್ಲೂ ಸೋಲುತ್ತಿರುವ ನಾನು ಹುಟ್ಟಿದಗಳಿಗೆ ಕೆಟ್ಟದ್ದಿರಬೇಕು. ಆರೋಗ್ಯವೂ ಅಷ್ಟು ಚೆನ್ನಾಗಿಲ್ಲ. ಕುಳ್ಳನಾಗಿರುವುದರಿಂದ ಜನರ ಅಪ ಹಾಸ್ಯಕ್ಕೂ ಈಡಾಗಬೇಕಾಗಿದೆ. ಈ ಎಲ್ಲ ದುಃಖವನ್ನು ತಡೆದುಕೊಳ್ಳುವುದು ಹೇಗೆ? ನನಗರಿಯದಂತೆ ಅನೈತಿಕತೆಯತ್ತ ಜಾರುತ್ತಿದ್ದೇನೆ. ಒಂದಲ್ಲ ಒಂದು ದಿನ ನಾನೂ ಕೆಟ್ಟವನಾದರೆ ಆಶ್ಚರ್ಯವಿಲ್ಲ!’

ಸತ್ಸಂಗ, ಸದ್ಗ್ರಂಥಗಳ ಅಧ್ಯಯನ, ಪ್ರಾರ್ಥನೆ–ಸ್ವಲ್ಪವಾದರೂ ನಿಷ್ಠೆಯಿಂದ ಎಂದರೆ ಎಡೆ\-ಬಿಡದೆ ಕ್ರಮವಾಗಿ ಮಾಡುವ ಪ್ರಯತ್ನದಿಂದ ಖಂಡಿತವಾಗಿಯೂ ಮನಸ್ಸನ್ನು ನಿಯಂತ್ರಿಸಲು ಸಾಧ್ಯ. ಆದರೆ ಸ್ವಲ್ಪ ಕಾಲಾವಕಾಶಬೇಕು. ಧೃತಿಗೆಡದೆ ಅಭ್ಯಾಸ ಮಾಡಿದ್ದಾದರೆ ಯಶಸ್ಸು ದೊರೆಯುತ್ತದೆ. ಊಟ, ತಿಂಡಿ, ವಿಶ್ರಾಂತಿ, ಹರಟೆ, ನಿದ್ರೆ, ತಿರುಗಾಟ–ಎಲ್ಲಕ್ಕೂ ಸಮಯವಿದೆ. ಆದರೆ ಪ್ರಾರ್ಥನೆ, ನಾಮಜಪ, ಸದ್ಗ್ರಂಥಗಳ ಅಧ್ಯಯನ ನಿಷ್ಠೆಯಿಂದ ಮಾಡಲು ಆಗುವುದಿಲ್ಲ ಸಮಯವಿಲ್ಲ ಎಂದರೆ ಆ ವಿಚಾರಗಳಲ್ಲಿ ನಿನಗೆ ಶ್ರದ್ಧೆಯಿಲ್ಲ ಎಂದಾಗುತ್ತದೆ ಅಷ್ಟೆ. ಪ್ರಯತ್ನ, ಪ್ರಯತ್ನ, ಬಿಡದ ಪ್ರಯತ್ನ–ಪ್ರಾರ್ಥನೆ ಪ್ರಯತ್ನ ಮುಂದುವರಿಸು. ಪ್ರಗತಿ ಖಂಡಿತ ಸಾಧ್ಯ.

ಮಳೆಗಾಲ ಬಂತೆಂದು ಮುಂಚಿತವಾಗಿ ಬೇಕಾದ ಆಹಾರ ಸಾಮಗ್ರಿ ದಾಸ್ತಾನು ಮಾಡುತ್ತಾರೆ. ಒಂದೆರಡು ದಿನ ವಿದ್ಯುತ್ ಕೊರತೆ ಇದೆ ಎಂದಲ್ಲಿ ಅದಕ್ಕಾಗಿ ಬೇಕಾದ ಎಲ್ಲ ವ್ಯವಸ್ಥೆಯನ್ನು ಮುಂಚಿತವಾಗಿ ಮಾಡಿಕೊಳ್ಳುತ್ತಾರೆ. ತಾಯಿ ಊರಿಗೆ ಹೋದಾಗಲೂ ಸಹ ಊಟ ಮಾತ್ರ ಬೇರೆಲ್ಲಿಯಾದರೂ (ಹೋಟೆಲ್ ಇತ್ಯಾದಿ) ಮಾಡದೇ ಇರಲಾರರು. ಧೋಬಿ ಬರುವುದಿಲ್ಲವೆಂದಲ್ಲಿ ಬಟ್ಟೆ ಒಗೆದುಕೊಳ್ಳುತ್ತಾರೆ. ಮನರಂಜನೆಗಾಗಿ ಗಂಟೆಗಟ್ಟಲೆ ಕಾಲ ವ್ಯರ್ಥ ಮಾಡುತ್ತಾರೆ. ಆದರೆ ಸರ್ವಶಕ್ತನಾದ ಭಗವಂತನ ಚಿಂತನೆಗಾಗಿ ಯಾರೂ ಸಮಯವನ್ನು–ದಿನದ ಇಪ್ಪತ್ತನಾಲ್ಕು ಗಂಟೆಗಳಲ್ಲಿ ಇಪ್ಪತ್ತನಾಲ್ಕು ನಿಮಿಷಗಳನ್ನಾದರೂ ನಿಷ್ಠೆಯಿಂದ ಉಪಯೋಗಿಸಲಾರರಲ್ಲ!


\section*{ಈ ಕ್ಷಣ ಮುಂದಾಗಿ}

\addsectiontoTOC{ಈ ಕ್ಷಣ ಮುಂದಾಗಿ}

ಏಕೆ ಹೀಗಾಗುತ್ತಿದೆ? ಇದಕ್ಕೆಲ್ಲ ಏನರ್ಥ ಎಂದು ತಿಳಿಸುವುದೂ, ತಿಳಿದುಕೊಳ್ಳುವುದೂ ಕಷ್ಟವೇ. ಆದರೆ, ತಾಪತ್ರಯಗಳಿಂದ ಪಾರಾಗುವ ವಿಧಾನವನ್ನು ತಿಳಿದುಕೊಂಡು, ಮಾಡಬೇಕಾದುದನ್ನು ಮಾಡಿದರೆ, ಖಂಡಿತ ತೊಂದರೆಗಳಿಂದ ಪಾರಾಗಬಹುದು, ಹಾಗೆ ಪಾರಾದವರಿದ್ದಾರೆ. ಏಕೆ? ಹೇಗೆ? ಎಂಬುದು ತುಂಬ ಜಟಿಲವಾದ ವಿಚಾರ. ಸೂಕ್ಷ್ಮದರ್ಶಕದ ಸಹಾಯವಿಲ್ಲದೆ ಸೂಕ್ಷ್ಮ ವಸ್ತುಗಳನ್ನು ನೋಡಲಾಗದು. ದೂರದರ್ಶಕದ ಸಹಾಯವಿಲ್ಲದೆ ದೂರದ ವಸ್ತುಗಳ ಸಂಬಂಧವಾಗಿ ತಿಳಿಯಲಾಗದು. ಈ ಜಗತ್ತು ಜಟಿಲ, ಸಂಕೀರ್ಣ ಹಾಗೂ ಅತ್ಯಂತ ಸೂಕ್ಷ್ಮ ವಾದ ನಿಯಮಗಳಿಂದ ನಿಯಂತ್ರಿಸಲ್ಪಟ್ಟಿದೆ. ಇದು ಸಾಮಾನ್ಯ ದೃಷ್ಟಿಗೆ ಗೋಚರಿಸದು. ಜೀವಿ ಆರ್ಜಿಸಿದ ಕರ್ಮಕ್ಕನುಗುಣವಾಗಿ ಸುಖದುಃಖಗಳು ಉಂಟಾಗುತ್ತವೆ. ಈ ವಿಚಾರವನ್ನು ಮುಂದಿನ ಅಧ್ಯಾಯ\-ವೊಂದರಲ್ಲಿ ತಿಳಿಸಲು ಯತ್ನಿಸಲಾಗಿದೆ. ಈಗ ಉಂಟಾಗಿರುವ ತೊಂದರೆಗೆ ಕಾರಣ ಯಾವುದೆಂದು ತಿಳಿಯದಿರಬಹುದು. ಆದರೆ ಕಾರಣ ಇದ್ದೇ ಇದೆ. ಈಗ ಮಾಡಿದ ಸತ್ಕರ್ಮಕ್ಕೆ ಫಲವಾಗಿ ಸದ್ಯದ ಪರಿಸ್ಥಿತಿ ಪ್ರಾಪ್ತವಾದುದಲ್ಲ–ಎಂಬುದನ್ನು ನೆನಪಿನಲ್ಲಿಡಬೇಕು. ಇಂತಹ ವಿಚಾರ ಸಾಮಾನ್ಯರಿಗೆ ಗೊಂದಲವೆನಿಸುವುದು ಸಹಜ. ಆದರೆ ಎಲ್ಲವೂ ಸೂಕ್ಷ್ಮ ನಿಯಮಗಳಿಗನುಗುಣವಾಗಿ ನಡೆಯುವುದು.

ಸಜ್ಜನರಿಗೂ, ಸತ್ಕರ್ಮನಿರತರಿಗೂ, ದೈವಭಕ್ತರಿಗೂ ಸಂಕಟಗಳೇಕೆ ಬರುತ್ತವೆ? ಎಷ್ಟೋ ಮಂದಿಯ ದೇವರಲ್ಲಿನ ಶ್ರದ್ಧೆ ಇಂಥ ಪ್ರಶ್ನೆಗಳಿಗೆ ಉತ್ತರ ಕಂಡುಕೊಳ್ಳಲಾರದೆ, ಪಲಾಯನ ಸೂತ್ರ ಪಠಿಸುವುದು ಆಶ್ಚರ್ಯವೇನಲ್ಲ. ಸಮಸ್ಯೆಗೊಂದು ಉತ್ತರ ತಮಗೆ ತಿಳಿಯದಿದ್ದಲ್ಲಿ, ಅದಕ್ಕೆ ಉತ್ತರವೇ ಇಲ್ಲವೆಂದು ಮೊಂಡುವಾದ ಮಾಡುವ ಬುದ್ಧಿವಂತರಿಗೆ ಕೊರತೆ ಇಲ್ಲ. ನಿಜವಾಗಿಯೂ ಎಲ್ಲ ಸಮಸ್ಯೆಗಳಿಗೂ, ನಿಗೂಢಪ್ರಶ್ನೆಗಳಿಗೂ, ಉತ್ತರ ಇದ್ದೇ ಇದೆ. ಮಾನವನಿಗೆ ಕೊಟ್ಟಿರುವ ಅವಕಾಶಗಳನ್ನು ಸದುಪಯೋಗ ಪಡಿಸಿಕೊಳ್ಳುವುದರ ಕಡೆಗೆ ಎಲ್ಲರೂ ಹೆಚ್ಚು ಗಮನವೀಯಬೇಕು. ಬೆಂಕಿ ಬಿದ್ದಾಗ ಮನೆಯ ಬಳಿ ನಿಂತು, ಮನೆಯ ಎತ್ತರವೆಷ್ಟು, ಮನೆಗೆ ಹೇಗೆ ಬೆಂಕಿ ಬಿತ್ತು? ಎಂದು ಚರ್ಚಿಸುವುದು ಉಚಿತವಲ್ಲ–ಅಷ್ಟು ಮಾತ್ರವೇ ಅಲ್ಲ, ಅಂಥ ಚರ್ಚೆಯಿಂದ ದೋಷ ಉಂಟಾಗುತ್ತದೆ. ಸಣ್ಣ ಮಗುವಿಗೆ ಬೃಹತ್ತಾದ ನಕ್ಷತ್ರದ ಬಗ್ಗೆ ತಿಳಿಸಿದಾಗ ಅದನ್ನು ಅರಿಯಲಾರದು. ಸ್ವಲ್ಪ ಬೆಳವಣಿಗೆ ಉಂಟಾದರೆ ಅರ್ಥವಾಗುತ್ತದೆ. ಭಗವಂತನನ್ನು ಕಂಡ ಮಹಾತ್ಮರ ಮಾತಿನಲ್ಲಿ ವಿಶ್ವಾಸವಿಡದೆ, ನಮಗೆ ಮುನ್ನಡೆಯಲು ಬೇರಾವ ದಾರಿ ಇದೆ? ‘ಜೀವನ ಏರುಪೇರುಗಳಿಲ್ಲದೆ ಎಲ್ಲ ಸುಗಮವಾಗಬೇಕು. ಇಲ್ಲವಾದರೆ ದೇವರೇ ಇಲ್ಲ ಎನ್ನುತ್ತೇನೆ’ ಎಂದು ದೇವರನ್ನು ಹೆದರಿಸಲು ಸಾಧ್ಯವೇ? ಸಂಕಟ ಸಂಕಷ್ಟಗಳೆಲ್ಲ ದೇವರು ಕೊಡುವ ಸತ್ವಪರೀಕ್ಷೆ ಎಂದೆಣಿಸಿ ಎಲ್ಲರೂ ನಿಷ್ಠೆಯಿಂದ ಪ್ರಾರ್ಥನೆ ಮುಂದುವರಿಸಲಿ. ಶುಭವಾಗಿಯೆ ತೀರುವುದು. ಈ ನಿಟ್ಟಿನಲ್ಲಿ ಈ ಕ್ಷಣವೇ ಮುಂದಾಗಿ.


\section*{ತೀರದ ದುಃಖ}

\addsectiontoTOC{ತೀರದ ದುಃಖ}

ಜೀವನದ ಸಮಸ್ಯೆಗಳು ಎಂಬ ಶಬ್ದವನ್ನು ಇಲ್ಲಿ ಆಗಾಗ ಉಪಯೋಗಿಸಿದೆ. ಆಹಾರ, ಬಟ್ಟೆ, ವಸತಿಯ ಅಭಾವದ ಸಮಸ್ಯೆಯನ್ನು ಕುರಿತ ಅರ್ಥದಲ್ಲಿ ಆ ಮಾತನ್ನು ಹೇಳಿಲ್ಲ. ಭಾರತದಂಥ ಬಡತನದ ದವಡೆಯಲ್ಲಿ ಸಿಲುಕಿರುವ ದೇಶದಲ್ಲಿ, ಇದು ದೊಡ್ಡ ಸಮಸ್ಯೆಯೇನೋ ಹೌದು. ಇದನ್ನು ಸರಿಯಾದ ಸಾಮಾಜಿಕ, ರಾಜಕೀಯ, ಆರ್ಥಿಕ ಬದಲಾವಣೆಗಳಿಂದ ಸರಿಪಡಿಸಲು ಎಲ್ಲರೂ ಸಹಕರಿಸಿದರೆ ಸಾಧ್ಯ. ಪಶ್ಚಿಮದ ಕೆಲವು ರಾಷ್ಟ್ರಗಳಲ್ಲಿ ಈ ಹಸಿವೆ, ಬಡತನಗಳನ್ನು ಈ ರೀತಿಗಳಿಂದ ದೂರಮಾಡಿದ್ದಾರೆ.

ಆದರೆ ಬದುಕಿನ ಕೆಲವು ಪ್ರಮುಖ ಸಮಸ್ಯೆಗಳಿಗೂ, ಬಡತನಕ್ಕೂ ಸಂಬಂಧವಿಲ್ಲ; ಮಾತ್ರವಲ್ಲ, ಭೌತಿಕ ವಿಧಾನಗಳನ್ನು ಅನುಸರಿಸಿ ಅವುಗಳಿಗೆ ಪರಿಹಾರ ಕಂಡುಕೊಳ್ಳಲೂ ಸಾಧ್ಯವಿಲ್ಲ. ಈ ಸಮಸ್ಯೆಗಳು ನಮ್ಮ ಜೀವಿತಕ್ಕೆ ಅಂಟಿಕೊಂಡೆ ಇರುವ ಸಮಸ್ಯೆಗಳು. ನಮ್ಮ ಅಸ್ತಿತ್ವಕ್ಕೆ ಅಂಟಿ\-ಕೊಂಡಿರುವ ಈ ಸಮಸ್ಯೆಗಳನ್ನು ಸಮಷ್ಟಿಯಾಗಿ ‘ದುಃಖ’ ಎಂದು ಕರೆಯುತ್ತಾರೆ. ನಮ್ಮ ಸಂಪತ್ತು ಸಲಕರಣೆಗಳನ್ನು ಆಧರಿಸಿ ಈ ದುಃಖದ ಬೋಧೆಯಲ್ಲಿ ವ್ಯತ್ಯಾಸ ಎಂದಲ್ಲ–ನಮ್ಮ ವ್ಯಕ್ತಿತ್ವವನ್ನು ಹೊಂದಿಕೊಂಡೇ ಅದು ಬದಲಾಗುತ್ತದೆ. ಬಡವನಾಗಿದ್ದರೂ ಶ‍್ರೀಮಂತನಿಗಿಂತ ಹೆಚ್ಚು ತೃಪ್ತಿ ಹಾಗೂ ಶಾಂತಿಯ ಜೀವನವನ್ನು ಒಬ್ಬಾತ ನಡೆಯಿಸಬಹುದು. ಗೌತಮಬದ್ಧ ಜಗತ್ತು ದುಃಖಮಯ ಎಂದದ್ದು ಈ ಅಸ್ತಿತ್ವದ ಪರಿಮಿತಿಯಿಂದುಂಟಾದ ನರಳಾಟವನ್ನು ಕುರಿತೆ. ಬುದ್ಧನ ಜೀವನದಲ್ಲಿ ರೋಗ, ವೃದ್ಧಾಪ್ಯ ಮತ್ತು ಸಾವು–ಇವು ಗಾಢವಾದ ಪರಿಣಾಮವನ್ನು ಉಂಟುಮಾಡಿದ ಪ್ರಮುಖ ಘಟನೆಗಳು. ಕೇವಲ ಹಸಿದು ಕಂಗೆಟ್ಟ ವ್ಯಕ್ತಿಯನ್ನು ಕಂಡಿದ್ದರೆ ಆತ ತನ್ನ ರಥವನ್ನು ನಿಲ್ಲಿಸಿ ಸೇವಕರಿಗೆ ಆಹಾರವನ್ನು ತಂದುಕೊಡುವಂತೆ ಹೇಳುತ್ತಿದ್ದ. ಬಡತನದ ಸಮಸ್ಯೆಗೆ ಸಾಮಾಜಿಕ–ಆರ್ಥಿಕ–ಕೃಷಿ–ಸಂಬಂಧವಾದ ಸುಧಾರಣೆಯಿಂದ ಪರಿಹಾರ ಸೂಚಿಸುತ್ತಿದ್ದ.

ಪ್ರಾಪಂಚಿಕ ವಿಧಾನಗಳಿಂದ ಈ ಮೂಲಭೂತ ಸಮಸ್ಯೆಗಳನ್ನು ಏಕೆ ಪರಿಹರಿಸಲಾಗುವುದಿಲ್ಲ? ಎಂದು ನೀವು ಕೇಳಬಹುದು. ಭಯ, ಉದ್ವೇಗ, ಅತೃಪ್ತಿ, ಪ್ರೀತಿ, ದ್ವೇಷ–ಇವು ಮನುಷ್ಯನ ಅನುಭವಕ್ಕೆ ಸಂಬಂಧಿಸಿದ ಸಮಸ್ಯೆಗಳು. ಬಾಹ್ಯವಸ್ತುಗಳು, ಜನರು, ಪ್ರಮುಖ ಕಾರಣಗಳಲ್ಲ; ನಮ್ಮ ಪ್ರಜ್ಞೆಯ ಆಳದಲ್ಲೇ ಈ ಸಮಸ್ಯೆಗಳ ಮೂಲಕಾರಣ ಅಡಗಿದೆ. ಅಲ್ಲೇ ಅದರ ಪರಿಹಾರವನ್ನು ಕಂಡುಕೊಳ್ಳಬೇಕು. ಆಧ್ಯಾತ್ಮಿಕ ಸಾಧನೆ ಈ ಕೆಲಸ ಮಾಡಿಕೊಡುತ್ತದೆ. ಅನಿವಾರ್ಯ, ಅಪರಿಹಾರ್ಯಗಳನ್ನು ಆ ಮೂಲಕ ಎದುರಿಸಬೇಕಲ್ಲದೇ ಬೇರೆ ದಾರಿ ಇಲ್ಲ.


\section*{ಸುಖದ ಮರೀಚಿಕೆ}

\addsectiontoTOC{ಸುಖದ ಮರೀಚಿಕೆ}

ದುರ್ಯೋಧನನಿಂದ ತಿರಸ್ಕೃತನಾದ ವಿದುರ ಹಸ್ತಿನಾಪುರವನ್ನು ಬಿಟ್ಟು ತಾಪಸಿಯಾಗಿ ಕಾಡು ಮೇಡುಗಳಲ್ಲಿ ಅಲೆದಾಡುತ್ತಿದ್ದ. ಎಷ್ಟೋ ವರ್ಷಗಳ ಬಳಿಕ ಶ‍್ರೀಕೃಷ್ಣನ ಪರಮಭಕ್ತನಾದ ಉದ್ಧವನನ್ನು ಭೆಟ್ಟಿಯಾದ. ಮಹಾಭಾರತದ ಯುದ್ಧದಲ್ಲಿ ನಡೆದ ದಾರುಣ ಹತ್ಯಾಕಾಂಡದ ಕತೆಯನ್ನು ವಿವರವಾಗಿ ಕೇಳಿದ. ಶ‍್ರೀಕೃಷ್ಣನ ದೇಹತ್ಯಾಗ ಹಾಗೂ ಯಾದವರ ವಂಶನಾಶದ ಘಟನೆಯನ್ನು ಕೇಳಿದ. ವಿಧಿಯ ಅಗಮ್ಯ ವೈಚಿತ್ರ್ಯವನ್ನು ನೆನೆದು ಸ್ತಂಭಿತನಾದ. ಕೆಲಕಾಲ ಗಾಢಚಿಂತನೆಯಲ್ಲಿ ಮುಳುಗಿದ. ಮುಂದೆ ನೇರವಾಗಿ ಮೈತ್ರೇಯ ಋಷಿಯ ಆಶ್ರಮಕ್ಕೆ ಹೋಗಿ ಅವರೊಡನೆ ತನ್ನ ಮನಸ್ಸಿನ ದುಗುಡವನ್ನು ತೋಡಿಕೊಂಡ.

\vskip 2pt

‘ಮನುಷ್ಯರೆಲ್ಲರೂ ಸುಖವನ್ನು ಪಡೆಯುವುದಕ್ಕಾಗಿ ನಾನಾ ಚಟುವಟಿಕೆಗಳಲ್ಲಿ ತೊಡಗುತ್ತಾರೆ. ಆದರೆ ಅವರು ಸುಖವನ್ನು ಪಡೆಯಲು ಸಮರ್ಥರಾಗುವುದೂ ಇಲ್ಲ, ಮಾತ್ರವಲ್ಲ, ತಮ್ಮ ಚಟುವಟಿಕೆಗಳಿಂದ ಮತ್ತೆ ಮತ್ತೆ ಕಷ್ಟಗಳನ್ನೇ ಅನುಭವಿಸುತ್ತಾರೆ. ಓ ಪರಮಪೂಜ್ಯರಾದ ಋಷಿಗಳೆ, ಇಂಥ ಪರಿಸ್ಥಿತಿಯಲ್ಲಿ ತಮ್ಮ ಮಾರ್ಗದರ್ಶಕ ಉಪದೇಶವಾದರೂ ಏನು?’

\vskip 2pt

ಈ ಪ್ರಶ್ನೆಯನ್ನು ಕೇಳಿದ ವಿದುರ ಸಾಮಾನ್ಯ ವ್ಯಕ್ತಿಯ ಸಾಲಿನಲ್ಲಿ ನಿಂತವನಲ್ಲ. ಆ ಕಾಲದ ಮಹಾತತ್ತ್ವವೇತ್ತನೆಂದು ಆತ ಪರಿಗಣಿತನಾಗಿದ್ದ. ವಿದುರನ ಪ್ರಶ್ನೆಗೆ ಮೈತ್ರೇಯ ಋಷಿಗಳು ಉತ್ತರಿಸಿದ ವಿಧಾನವಾದರೂ ಯಾವುದು? ಯಾರಾದರೂ ಪ್ರಭಾವಶಾಲಿಯಾದ ವ್ಯಕ್ತಿಗಳನ್ನು ಭೇಟಿಮಾಡಲು ಮೈತ್ರೇಯರು ಹೇಳಿದರೇ? ಇಲ್ಲ. ಸಮಾಜ ಸುಧಾರಣಾ ಕಾರ್ಯಕ್ರಮಗಳನ್ನು ಕೈಗೊಳ್ಳಲು ಹೇಳಿದರೇ? ಇಲ್ಲ. ರಾಜಕೀಯ ಅಥವಾ ಅರ್ಥಶಾಸ್ತ್ರದ ಅಧ್ಯಯನ ಮಾಡಲು ಸೂಚಿಸಿದರೆ? ಅದೂ ಇಲ್ಲ. ಏಕಪ್ರಕಾರವಾಗಿ ಭಗವಂತನ ಮಹಿಮೆ, ಮಾಹಾತ್ಮ್ಯೆಗಳನ್ನು ಕುರಿತು ಮಾತನಾಡತೊಡಗಿದರು. ಎಂದರೆ, ಬದುಕಿನ ಸಮಸ್ಯೆಗಳಿಗೆ ಆಧ್ಯಾತ್ಮಿಕ ಪರಿಹಾರವನ್ನು\break ಸೂಚಿಸಿದರು.

\vskip 2pt

ಮುಚ್ಚುಮರೆ ಏಕೆ? ಕೊಳೆತ ಹುಣ್ಣುಗಳನ್ನು ಹೂವಿನಿಂದ ಮುಚ್ಚಿಡುವ ಪ್ರಯತ್ನದಿಂದೇನು? ಶರೀರ, ಮನಸ್ಸುಗಳ ಪರಿಮಿತಿಯಿಂದ, ಇತರ ವ್ಯಕ್ತಿಗಳ ಸಂಪರ್ಕದಿಂದ, ಪ್ರಕೃತಿಯ ವಿಕೋಪದಿಂದ ಉಂಟಾಗುವ ವಿವಿಧ ತಾಪತ್ರಯಗಳಲ್ಲಿ ಒಂದರಿಂದಲಾದರೂ, ಸಂಪೂರ್ಣ ಬಿಡುಗಡೆ ಜೀವಿಯ ಪಾಲಿಗೆ ಲಭ್ಯವಿಲ್ಲ! ಯಾವುದೋ ಒಂದು ಸನ್ನಿವೇಶದಲ್ಲಿ ಪರಿಹಾರ ದೊರೆತಿದೆ ಎಂದು ತೋರಿದರೂ ಅದು ಪರಿಪೂರ್ಣವಲ್ಲ–ಆಂಶಿಕ ಎಂಬುದನ್ನು ಎಲ್ಲ ವಿವೇಕಿಗಳೂ ಹೇಳುತ್ತಾರೆ. ತಲೆಯಮೇಲೆ ಹೊತ್ತುಕೊಂಡಿದ್ದ ಭಾರದ ವಸ್ತುವನ್ನು ಹೆಗಲ ಮೇಲಿಟ್ಟುಕೊಂಡಾಗ ಸ್ವಲ್ಪ ಸುಖದ ಭಾಸವಾದಂತೆ, ಪ್ರಾಪಂಚಿಕ ಪರಿಹಾರಗಳು ತಾತ್ಕಾಲಿಕ, ಅಪೂರ್ಣ.

ಹಾಗಾದರೆ, ಅನಿಶ್ಚಿತತೆ ಮತ್ತು ಕಷ್ಟಪರಂಪರೆಗಳಿಂದ ತುಂಬಿದ ಜೀವನದ ಸಮಸ್ಯೆಗಳಿಗೆ ಆಧ್ಯಾತ್ಮಿಕ ಪರಿಹಾರವಿದೆಯೆ? ಪ್ರಪಂಚ ಗೌರವಿಸುವ, ನಂಬುವ ಎಲ್ಲ ಧರ್ಮಗ್ರಂಥಗಳಲ್ಲೂ ಖಂಡಿತವಾಗಿ ಪರಿಹಾರವಿದೆ ಎನ್ನಲಾಗಿದೆ. ಸಹಸ್ರಾರು ಸಂತರು, ಮಹಾತ್ಮರು, ಋಷಿಮುನಿಗಳು, ಪ್ರವಾದಿಗಳು, ಸಿದ್ಧಪುರುಷರು ತಮ್ಮ ಜೀವನಾನುಭವಗಳಿಂದ ಆಧ್ಯಾತ್ಮಿಕ ಪರಿಹಾರವಿದೆ ಎಂದು ಘಂಟಾಘೋಷವಾಗಿ ಸಾರುತ್ತಾರೆ. ಆಧ್ಯಾತ್ಮಿಕ ಜೀವನದ ತಳಹದಿಯೆ ಈ ಉಪದೇಶಗಳಲ್ಲಿನ ಶ್ರದ್ಧೆ. ಈ ಶ್ರದ್ಧೆಯೇ, ತ್ಯಾಗ, ತಪಸ್ಸು, ಪೂಜೆ, ಧ್ಯಾನ, ಆತ್ಮ ವಿಶ್ಲೇಷಣೆಯೇ ಮೊದಲಾದ ನಿಯಮಗಳನ್ನು ಪಾಲಿಸಲು ಪ್ರೇರಕ.


\section*{ಆರ್ಥಿಕ ಅಡಚಣೆ}

\addsectiontoTOC{ಆರ್ಥಿಕ ಅಡಚಣೆ}

‘ಈ ಹಾಳು ಹಣದ ಮುಗ್ಗಟ್ಟೊಂದಿಲ್ಲದಿದ್ದರೆ ಎಲ್ಲವನ್ನೂ ಮಾಡುತ್ತಿದ್ದೆ’ ಎಂದು ಹೇಳದವರು ವಿರಳ. ಅಮೇರಿಕದ ಪತ್ರಿಕಾ ಸಂಪಾದಕರೊಬ್ಬರು ಸಾವಿರಾರು ಜನರ ಜೀವನವನ್ನು ಪರಿಶೀಲಿಸಿದ ಮೇಲೆ ನೂರಕ್ಕೆ ಎಪ್ಪತ್ತರಷ್ಟು ಚಿಂತೆ, ಆರ್ಥಿಕ ತೊಡಕುಗಳಿಂದಲೇ ಉಂಟಾಗುತ್ತದೆ ಎಂದು ಕಂಡುಕೊಂಡರು. ಪರಂಪರಾಗತ ಬಡತನದ ದೇಶವಾದ ನಮ್ಮಲ್ಲಿ ಈ ಪ್ರಮಾಣ ಭಯಾನಕ\-ವೆನಿಸು\-ವಷ್ಟು ಹೆಚ್ಚಿರುವುದು ಸ್ವಾಭಾವಿಕವೇ. ಆರ್ಥಿಕ ತೊಂದರೆ ನಮ್ಮನ್ನು ಆವರಿಸಿದಾಗ ಯಾವ ಸಮಸ್ಯೆಯನ್ನೂ ಇದಿರಿಸುವ ಧೈರ್ಯ ಮಾಯವಾಗತೊಡಗುತ್ತದೆ. ನಿರುತ್ಸಾಹ ತಲೆದೋರುತ್ತದೆ. ಬೇಸರ, ಉದ್ವಿಗ್ನತೆಗಳು ಕಾಡುತ್ತವೆ. ಇತರ ಎಲ್ಲ ಅಂಶಗಳಿಗಿಂತಲೂ ಹಣಕ್ಕೆ ಅತಿ ಪ್ರಾಧಾನ್ಯ ಕೊಡುವ ಸಮಾಜದಲ್ಲಿ, ಆರ್ಥಿಕ ದುಃಸ್ಥಿತಿಯಲ್ಲಿ ಬಾಳುವ ವ್ಯಕ್ತಿಗೆ ಮನೋ ವೇದಕ ಸನ್ನಿವೇಶಗಳನ್ನು ಎದುರಿಸಬೇಕಾಗುವುದು ಅಸ್ವಾಭಾವಿಕವಲ್ಲ.

ಹಾಗಾದರೆ ಆರ್ಥಿಕ ಸಮಸ್ಯೆಗಳಿಗೆ ಪರಿಹಾರೋಪಾಯವೇ ಇಲ್ಲವೇ? ಹಣದ ಮುಗ್ಗಟ್ಟೆಂದು ಕೈಕಟ್ಟಿ ಕುಳಿತು, ನಿರಾಶೆಯ ತೀಕ್ಷ್ಣ ರೇಖೆಗಳನ್ನು ಮುಖದಲ್ಲಿ ಮೂಡಿಸಿಕೊಂಡರೆ ಕೆಲಸ\-ವಾಗು\-ವುದೇ?

ನಿಮ್ಮ ಆರ್ಥಿಕ ತೊಡಕುಗಳನ್ನು ಬಗೆಹರಿಸುವ ಮಾಂತ್ರಿಕ ಉಪಾಯವೇನೂ ನನ್ನ ಹತ್ತಿರ ಇಲ್ಲ. ಆದರೆ ಪ್ರಾಮಾಣಿಕವಾಗಿ ಯೋಚಿಸಿದರೆ, ಯೋಚಿಸಿದಂತೆ ಕೆಲಸ ಮಾಡಿದರೆ, ಖಂಡಿತವಾಗಿಯೂ ಅದು ನಿಮ್ಮಿಂದ ಅಸಾಧ್ಯವಾದ ಸಂಗತಿಯಲ್ಲ. ಸಮಸ್ಯೆ ನಿಮ್ಮ ಆರ್ಥಿಕ ಅವ್ಯವಸ್ಥೆಯಿಂದ ಬಂದಿರಲಿ, ಅಥವಾ ನಿಮ್ಮ ಹಿರಿಯರ ಅವ್ಯವಸ್ಥೆಯಿಂದ ಉಂಟಾಗಿರಲಿ, ಅದನ್ನು ನಿವಾರಿಸುವ ಯತ್ನ ಮಾಡಬೇಕಾದವರು ನೀವೇ.

ನನ್ನ ಪರಿಚಿತರೊಬ್ಬರ ಕತೆ ಹೇಳುತ್ತೇನೆ. ಮೂರುಮಂದಿಯ ಅವರ ಸಂಸಾರವನ್ನು ಸಾಧಾರಣ ರೀತಿಯಲ್ಲಿ ನಡೆಸುವಷ್ಟು ಆದಾಯ ಅವರಿಗಿದೆ. ಸರಕಾರಿ ನೌಕರಿಯಲ್ಲಿರುವ ಅವರಿಗೆ\break ವಿಶಿಷ್ಟವೆಂದು ಹೇಳಿಕೊಳ್ಳುವಂತಹ ದೊಡ್ಡ ಖರ್ಚುಗಳೇನೂ ಇಲ್ಲವಾದರೂ, ತಿಂಗಳ ಮೊದಲನೇ ವಾರದಿಂದಲೇ ಸಾಲ ಹುಡುಕಲು ಆರಂಭಿಸುತ್ತಾರೆ. ಹಿಂದಿನ ತಿಂಗಳ ಸಾಲದ ಒಂದಂಶ ಮಾತ್ರ ಈ ತಿಂಗಳ ಸಂಬಳದಿಂದ ತೀರಿಸುತ್ತಾರೆ. ಪುನಃ ಅದೇ ವಿಷಚಕ್ರ. ಹಳೆಯ ಸಾಲ ತೀರಿಸಲು ಮತ್ತೊಂದು ಕಡೆ ಸಾಲ ಪಡೆಯಬೇಕಾಗುತ್ತದೆ. ಅನಿವಾರ್ಯವಾಗಿ ಸುಳ್ಳು ಹೇಳಬೇಕಾಗುತ್ತದೆ. ಸ್ನೇಹಿತರು ಕೊಡದಿದ್ದರೆ ಮನಸ್ತಾಪ. ಗೃಹಕೃತ್ಯದ, ದೈನಿಕ ಆವಶ್ಯಕತೆಗಳನ್ನು ಪೂರೈಸಲೂ ಒಮ್ಮೊಮ್ಮೆ ಬಹಳ ಕಷ್ಟಪಡಬೇಕಾಗುತ್ತದೆ. ಈ ಮಧ್ಯೆ ಯಾರಾದರೂ ಅತಿಥಿಗಳು ಬಂದರೆ ಅಥವಾ ಯಾರಿಗಾದರೂ ಕಾಯಿಲೆಯಾದರೆ ಅವಸ್ಥೆ ಹೇಳತೀರದು. ಕೊನೆಗೆ ಪ್ರಪಂಚವನ್ನೂ, ಹಣಕ್ಕೆ ಬೆಲೆ ಕೊಡುವ ಸಮಾಜವನ್ನೂ ಶಪಿಸುತ್ತಾರೆ!

ಈ ಮನುಷ್ಯ ಬಹಳ ಒಳ್ಳೆಯವರು, ಉದಾರಿಗಳು, ಧಾರಾಳಿಗಳು, ಸ್ವಲ್ಪ ವೈಭವದ ಜೀವನ, ಸ್ನೇಹಪರತೆ, ಮಾನವತೆ–ಇಂಥ ಗುಣಗಳ ಮೇಲೆ ಅವರಿಗೆ ವಿಶ್ವಾಸವಿದೆ. ತಮ್ಮ ಎಲ್ಲ ಆವಶ್ಯಕತೆಗಳೂ ಮುಗಿದ ಮೇಲೆಯೇ ಇದನ್ನು ಮಾಡುತ್ತೇನೆಂದರೆ ಯಾವ ಕೆಲಸವೂ ಆಗದೆಂದು ಅವರು ಸ್ಫೂರ್ತಿ ಬಂದಾಗ, ಸಾಲವಾದರೂ ಮಾಡಿ, ಫ್ಯಾನ್, ರೇಡಿಯೋ, ರೆಫ್ರಿಜರೇಟರ್ ಕೊಂಡು\-ಕೊಳ್ಳುತ್ತಾರೆ!

ಅರ್ಥಶಾಸ್ತ್ರಕ್ಕೆ ದಯಾದಾಕ್ಷಿಣ್ಯವಿಲ್ಲ. ಎಂಥ ಮಾನವೀಯ ಗುಣಗಳಿದ್ದರೂ ವೆಚ್ಚವನ್ನು ನೀವು ನಿಲ್ಲಿಸಲಾರಿರಿ. ಆದರೆ ಸ್ವಲ್ಪ ಮುಂದಾಲೋಚನೆ, ಭವಿಷ್ಯದ ಕಡೆಗೆ ದೃಷ್ಟಿ ಇವನ್ನು ನೀವು ಬೆಳೆಯಿಸಿಕೊಂಡರೆ ನಿಮ್ಮ ಕಿಸೆಯಿಂದ ಹಣವು ಅಷ್ಟು ಬೇಗನೆ ಜಾರಿ ಹೋಗದು.

ಹಣದ ಸಮಸ್ಯೆ ಎಂದರೆ ಹಣದ ಅಭಾವವೆಂದೇ ಹಲವರ ಭಾವನೆ. ಇದು ಪೂರ್ಣ ಸತ್ಯವಲ್ಲ –ದೊಡ್ಡ ದೊಡ್ಡ ಅರ್ಥಶಾಸ್ತ್ರಜ್ಞರು ಹಣದ ಅಭಾವಕ್ಕಿಂತಲೂ, ಇದ್ದ ಹಣವನ್ನು ಸರಿಯಾದ ರೀತಿಯಲ್ಲಿ ಹೊಂದಿಸಿಕೊಳ್ಳುವುದೇ ದೊಡ್ಡ ಸಮಸ್ಯೆ ಎಂದು ಕಂಡುಕೊಂಡಿದ್ದಾರೆ. ಬಹಳ ಹಣ ವಿರುವವರು ಹೆಚ್ಚು ಸಮಸ್ಯೆಗಳನ್ನು ಸೃಷ್ಟಿಸಿಕೊಂಡು ಒದ್ದಾಡುವ ಉದಾಹರಣೆಗೆ ಕೊರತೆ ಇಲ್ಲ. ಮಿತವಾದ ಆದಾಯವನ್ನೇ ಚಾತುರ್ಯದಿಂದ ವ್ಯಯಿಸಿ ಸಂತೃಪ್ತರಾಗಿರುವವರೂ ಇಲ್ಲದಿಲ್ಲ.

ಗಾಂಧೀಜಿ ಇಂಗ್ಲೆಂಡಿನಲ್ಲಿ ಕಲಿಯುತ್ತಿದ್ದಾಗ ಮೊದಲು ತಿಂಗಳಿಗೆ ೪೫ಪೌಂಡು ವೆಚ್ಚ ಮಾಡು\-ತ್ತಿದ್ದು\-ದನ್ನು ಮಿತವ್ಯಯಕ್ಕೆ ಒಗ್ಗಿ ೧೫ ಪೌಂಡಿಗೆ ಇಳಿಸಿದ್ದರು.

ಬೆಂಗಳೂರಿನಲ್ಲಿ ಅವರ ಬಿಡಾರವಿದ್ದಾಗ ‘ಧೋಬಿಗೆ ಕೊಟ್ಟ ಹಣ ಹೆಚ್ಚಾಯಿತು’ ಎಂದು ಗಾಂಧೀಜಿ ಹೇಳಿದಾಗ, ಮಹಾದೇವ ಭಾಯಿ ‘ನಾನು ಬೆಳಿಗ್ಗೆ ನಾಲ್ಕು ಗಂಟೆಗೆ ಎದ್ದು ರಾತ್ರಿ ಹನ್ನೆರಡರವರೆಗೆ ದುಡಿಯುತ್ತಿದ್ದೇನೆ’ ಎಂದರು. ಗಾಂಧೀಜಿ ‘ಮೂರುವರೆ ಗಂಟೆಗೇನೇ ಎದ್ದು ಬಟ್ಟೆ ಒಗೆದುಕೊ’ ಎಂದಿದ್ದರು.

ಗಾಂಧೀಜಿ ಯರವಾಡ ಸೆರೆಮನೆಯಲ್ಲಿದ್ದಾಗ ಸರಕಾರದವರು ಅವರ ಖರ್ಚಿಗೆ ಕಳುಹಿಸುತ್ತಿದ್ದ ಇನ್ನೂರು ರೂಪಾಯಿಗಳಲ್ಲಿ ಮೂವತ್ತೈದು ರೂಪಾಯಿಗಳನ್ನು ಮಾತ್ರ ಖರ್ಚು ಮಾಡಿ ಉಳಿದು\-ದನ್ನು ಹಿಂದೆ ಕಳುಹಿಸುತ್ತಿದ್ದರು.

ನಿಜ, ಗಾಂಧೀಜಿ ಏರಿದ ಎತ್ತರಕ್ಕೆ ನಾವು ತಲಪಲು ಸಾಧ್ಯವಿಲ್ಲ ಎನ್ನೋಣ. ಅವರದು ನೂರಕ್ಕೆ ನೂರರಷ್ಟು ಆದರ್ಶ ಜೀವನ. ಆದರೆ ನೂರರಲ್ಲಿ ಒಂದಂಶವಾದರೂ ನಮ್ಮಿಂದ ಸಾಧ್ಯ ಎನ್ನುವ ಯೋಚನೆ ನಮ್ಮ ಮಿದುಳನ್ನು ಪ್ರವೇಶಿಸಬಾರದೇ?

‘ಹಣವನ್ನು ಹೇಗೆ ಸಂಪಾದಿಸಬೇಕೆಂಬುದು ಎಲ್ಲರಿಗೂ ತಿಳಿದಿದೆ. ಆದರೆ ಅದನ್ನು ಹೇಗೆ ವೆಚ್ಚ ಮಾಡಬೇಕೆಂಬುದು ಲಕ್ಷಕ್ಕೊಬ್ಬನಿಗೂ ಸರಿಯಾಗಿ ತಿಳಿದಿಲ್ಲ’ ಎಂದು ಆಂಗ್ಲನೊಬ್ಬ ಹೇಳಿದ. ಮೊದಲ ನೋಟಕ್ಕೆ ಇದು ವಿರೋಧವಾಗಿ ಕಂಡರೂ ಅದರಲ್ಲಡಗಿರುವ ಅಮೂಲ್ಯ ಸತ್ಯದ ಪ್ರಾಮುಖ್ಯ ಗಮನಾರ್ಹವಾದುದು. ನೀವೇ ನಿಮ್ಮ ದಿನನಿತ್ಯದ ವೆಚ್ಚವನ್ನು ನಿರ್ದಾಕ್ಷಿಣ್ಯವಾಗಿ ವಿಮರ್ಶಿಸಿ ನೋಡಿ. ಹೇಗೆ ವೆಚ್ಚವಾಗುತ್ತಿದೆ? ಎಂಬುದನ್ನು ನೋಡಿ. ನಿಮಗೆ ಅನೇಕ ರಹಸ್ಯಗಳು ಪ್ರತ್ಯಕ್ಷವಾಗುತ್ತವೆ.

‘ನೋಟು ಚಿಲ್ಲರೆಯಾಗುವುದೇನೋ ಕಾಣುತ್ತದೆ. ಆದರೆ ಚಿಲ್ಲರೆ ಎಲ್ಲಿ ಹೋಗುತ್ತದೆಯೊ ನಾ ಕಾಣೆ’–ಎಂದು ನನ್ನ ಮಿತ್ರರೊಬ್ಬರು ಹೇಳುತ್ತಿದ್ದರು. ಬಹಳ ಹೊತ್ತು ಗಂಟಲ ಶಕ್ತಿಯನ್ನು ವ್ಯಯಿಸಿ ಒಂದು ತಿಂಗಳ ಮಟ್ಟಿಗಾದರೂ ದಿನನಿತ್ಯದ ಖರ್ಚಿನ ಪ್ರಾಮಾಣಿಕವಾದ ಲೆಕ್ಕವನ್ನಿಡಲು ಒತ್ತಾಯಿಸಿದೆ. ತಿಂಗಳ ಅಂತ್ಯದಲ್ಲಿ ದಿನಚರಿಯ ಸಮೀಕ್ಷೆ ನಡೆಯಿಸಿದಾಗ ಅವರ ಕಿಸೆಯ\break ತೂತುಗಳು ಎಲ್ಲಿವೆ ಎಂಬುದು ಅವರಿಗೇ ಅರ್ಥವಾಯಿತು. ಸಣ್ಣಪುಟ್ಟದ್ದೆಂದು ಅಲಕ್ಷಿಸುವ ಎಷ್ಟೋ ವೆಚ್ಚಗಳು ನಮ್ಮ ಆರ್ಥಿಕ ವ್ಯವಸ್ಥೆಗೆ ಕೊಡುವ ತೊಂದರೆ ನಮಗೇ ತಿಳಿದಿರುವುದಿಲ್ಲ.

\vskip 2pt

ನಿಮ್ಮ ಸಮಸ್ಯೆಯ ಮೂಲವೆಲ್ಲಿ ಎಂದು ತಿಳಿಯಲು ದಿನಚರಿಯ ಸಹಾಯ ಪಡೆಯಿರಿ. ಆಮೇಲೆ ಆದಾಯಕ್ಕೆ ಸರಿಯಾದ ಒಂದು ಕರಡು ಬಜೆಟ್ ತಯಾರಿಸಿ. ಅದಕ್ಕನುಗುಣವಾಗಿ ವ್ಯಯಿ\-ಸಲು ಅಭ್ಯಾಸ ಮಾಡಿ.

\vskip 2pt

ದಿನನಿತ್ಯದ ಅನಿವಾರ್ಯ ಆವಶ್ಯಕತೆಗಳ ಪೂರೈಕೆಗೆ ಹೆಚ್ಚು ಗಮನ ಕೊಡಿ. ಬಿದ್ದುಹೋದ ಬಚ್ಚಲು ಮನೆಯ ಗೋಡೆಯನ್ನು ಸರಿಪಡಿಸುವುದನ್ನು ಬಿಟ್ಟು, ಬೇಸರ ಕಳೆಯಲು ರೇಡಿಯೋ ಬೇಕೆಂದು ಅದನ್ನು ಕೊಂಡು ನಿತ್ಯವೂ ಅವ್ಯವಸ್ಥೆಗೊಳಗಾದ ನನ್ನ ಪರಿಚಿತರೊಬ್ಬರ ನೆನಪಾಗುತ್ತದೆ. ಭಾವನಾತ್ಮಕ ಹಾಗೂ ವಿಲಾಸದ ವಸ್ತುಗಳಿಗೆ ಗೌಣಸ್ಥಾನವಿರಲಿ. ಪ್ರತಿಯೊಂದು ವಸ್ತುವನ್ನು ಕೊಂಡುಕೊಳ್ಳುವಾಗ ‘ನನಗಿದು ನಿಜವಾಗಿಯೂ ಅವಶ್ಯವೇ’ ಎಂದು ಮೂರು ಬಾರಿ ಯೋಚಿಸಿ ನಿರ್ಧರಿಸಿಕೊಳ್ಳಿ. ನಿಮ್ಮ ಮಡದಿ, ಮಕ್ಕಳಿಗೆ ಸಂಯಮದ ಆರ್ಥಿಕ ಜೀವನ ನಡೆಯಿಸಲು ಕಲಿಸಿ. ಅನಿರೀಕ್ಷಿತ ಘಟನೆಗಳನ್ನು ಇದಿರಿಸಲು ವಿಮೆ ಮತ್ತು ಸೇವಿಂಗ್ಸ್ ಬ್ಯಾಂಕಿನ ಸಹಾಯ ಪಡೆಯಿರಿ. ಲಾಟರಿ, ಜೂಜು, ಚಕ್ರಬಂಧಸ್ಪರ್ಧೆಯಿಂದ ಹಣಗಳಿಸುವ ಆಕಾಂಕ್ಷೆ ಸರ್ವನಾಶಕ್ಕೆ ನಾಂದಿ ಎಂಬುದನ್ನು ಮರೆಯಬೇಡಿ.

\vskip 2pt

ಜನಪ್ರಿಯ ಮುಂದಾಳುಗಳೂ ಅರ್ಥಶಾಸ್ತ್ರಜ್ಞರೂ ಪ್ರಸಿದ್ಧಬ್ಯಾಂಕರರೂ ಆದ ಗಣ್ಯರೊಬ್ಬರು ಒಮ್ಮೆ ಯುವವಿದ್ಯಾರ್ಥಿಗಳಿಗೆ ಆರ್ಥಿಕ ಯಶಸ್ಸಿನ ಪಂಚಶೀಲಗಳನ್ನು ಹೀಗೆ ತಿಳಿಸಿದರು–

\bgroup\renewcommand\labelenumi{\protect\arabictokannada{\theenumi}.}

\begin{enumerate}
\itemsep=3pt
\item ನೀವು ಪಡೆಯಲು ಸಾಧ್ಯವಾದುದಕ್ಕಿಂತ ಹೆಚ್ಚು ಎಂದೂ ವೆಚ್ಚ ಮಾಡಬೇಡಿರಿ.

 \item ನಿಮಗೆ ಸಾಲವಿದ್ದರೆ ಸರ್ವಪ್ರಯತ್ನ ಮಾಡಿ ಎಲ್ಲಕ್ಕೂ ಮೊದಲು ಅದನ್ನು ತೀರಿಸಿಕೊಳ್ಳಿರಿ.

 \item ಯಾವಾಗಲೂ ಸಾಲದಿಂದ ದೂರವಿರಲು ಯತ್ನಿಸಿ. ‘ಹಣ ಮತ್ತೆ ಕೊಟ್ಟರೆ ಸಾಕು. ಈಗ ಇದನ್ನು ಕೊಂಡುಹೋಗಿ’ ಎನ್ನುವ ಚತುರ ವ್ಯಾಪಾರಸ್ಥರ ಮಾತಿಗೆ ಮರುಳಾಗಬೇಡಿ.

 \item ಆವಶ್ಯಕವೆಂದು ಖಚಿತವಾದುದಕ್ಕೆ ಖರ್ಚುಮಾಡಿ. ವಸ್ತುಗಳನ್ನು ಕೊಳ್ಳುವಾಗ ಅವುಗ\-ಳಿಲ್ಲದೆ ಅಥವಾ ಕಡಿಮೆ ಕ್ರಯದವುಗಳಿಂದ ಸುಧಾರಿಸುವುದು ಸಾಧ್ಯವೇ? ಎಂದು ವಿಮರ್ಶಿಸಿ ಮುಂದುವರಿಯಿರಿ.

 \item ಇದ್ದುದರಲ್ಲಿ ಸ್ವಲ್ಪವನ್ನಾದರೂ ಉಳಿಸುವ, ಉಳಿಸಿದುದನ್ನು ಬೆಳೆಸುವ ಕಲೆಯನ್ನು\break ಕಲಿಯಿರಿ.

\end{enumerate}

ಓ ಇದೇನು ಮಹಾ, ಗೊತ್ತಿದ್ದ ವಿಚಾರ ಎನ್ನುತ್ತೀರಿ ನೀವು. ವಿಚಾರ ನಿಮಗೆ ಗೊತ್ತಿರುವುದು ಮುಖ್ಯವಲ್ಲ. ಅದು ನಿಮ್ಮ ಜೀವನದಲ್ಲಿ ಎಷ್ಟರಮಟ್ಟಿಗೆ ಕಾರ್ಯರೂಪಕ್ಕೆ ಬಂದಿದೆ? ಹಾಗೆ ಕಾರ್ಯರೂಪಕ್ಕೆ ತರಲು ನೀವೆಷ್ಟು ಪ್ರಯತ್ನಿಸಿದ್ದೀರಿ? ಎನ್ನುವುದು ಗಮನಿಸತಕ್ಕ ಸಂಗತಿ. ಮೇಲಿನ ಪಂಚಶೀಲಗಳನ್ನು ನಿಮ್ಮ ಬದುಕಿನಲ್ಲಿ ರೂಢಿಸಿಕೊಳ್ಳಲು ಒಂದು ವರ್ಷದ ಯೋಜನೆಯನ್ನು ಹಾಕಿಕೊಂಡು ಸ್ವಲ್ಪ ಶ್ರಮಿಸಿದರೆ ಅದರ ಮಹತ್ವ ತಿಳಿಯುತ್ತದೆ.

ಬೇಕನ್ ಹೇಳಿದ: ‘ಹಣ ಒಳ್ಳೆಯ ಸೇವಕನೂ ಹೌದು, ದುಷ್ಟ ಯಜಮಾನನೂ ಹೌದು. ಹಣವನ್ನು ನಿಮ್ಮ ಸೇವಕನನ್ನಾಗಿ ಮಾಡಿರಿ. ಆದರೆ ನೀವೇ ಅದರ ಸೇವಕರಾಗಬೇಡಿ.’


\section*{ಗೆಲುವಿನ ಗುಟ್ಟು}

\addsectiontoTOC{ಗೆಲುವಿನ ಗುಟ್ಟು}

ಸಹೋದ್ಯೋಗಿಗಳ ಅಪಾರ ಮೆಚ್ಚುಗೆಗಳಿಸಿದ ಜೀವವಿಮಾ ನಿಗಮದ ಹಿರಿಯ ಆಫೀಸರರೊಡನೆ ಅವರ ಗೆಲವಿನ ಗುಟ್ಟೇನೆಂದೊಮ್ಮೆ ಕೇಳಿದೆ. ಚಿಂಕ್ರೋಭ ಅವರನ್ನು ಪೀಡಿಸಲಿಲ್ಲವೇ ಎಂಬುದನ್ನು ತಿಳಿಯಬೇಕಿತ್ತು ನನಗೆ. ಚಿಂಕ್ರೋಭದ ಸೋಂಕು ತಮಗೆ ತಗಲಿದ್ದರೂ, ತಾವು ಅದರ ಕಪಿಮುಷ್ಠಿಯಿಂದ ಪಾರಾದ ಬಗೆಯನ್ನು ಬಯಲು ಮಾಡುತ್ತಾ ಅವರೇ ಅಂದರು–

‘ಆಫೀಸಿಗೆ ಆಫೀಸರನ್ನಾಗಿ ನೇಮಕಗೊಂಡಾಗ, ನನ್ನ ಮನಸ್ಸಿಗೆ ಆರಂಭದಲ್ಲಿ ಸಂತೋಷವಾದರೂ, ಅನಂತರ ಒಂದು ಬಗೆಯ ಭಯ, ಸಂಶಯ (ಸಸ್ಪೆನ್ಸ್​) ಕಾಡತೊಡಗಿತು. ಸುಮಾರು ಹದಿನೇಳು ವರ್ಷಗಳಷ್ಟು ಕಾಲ ನೌಕರನಾಗಿದ್ದು, ಬೇರೆ ಬೇರೆ ಸಂಘಟನೆಗಳಲ್ಲಿ ದುಡಿಯುತ್ತಿದ್ದುದರಿಂದ ಗಳಿಸಿದ ನಿಕಟ ಸಂಪರ್ಕದಿಂದಾಗಿ ನೌಕರರ ಸ್ವಭಾವ, ದೌರ್ಬಲ್ಯ, ಸಿನಿಕತೆ, ಶಿಸ್ತುಗೇಡಿ ತನ, ಧಾರ್ಷ್ಟ್ಯ–ಇವುಗಳ ಪರಿಚಯ ನನಗಿತ್ತು. ಇಂಥವರನ್ನು ಕರ್ತವ್ಯನಿಷ್ಠೆಗೆ ಪ್ರೇರಿಸಿ, ನನ್ನ ಹುದ್ದೆಯ ಜವಾಬ್ದಾರಿಯನ್ನು ಸಮರ್ಥವಾಗಿ ನೆರವೇರಿಸುವ ಸಾಹಸ ನನ್ನಿಂದಾದೀತೇ ಎನ್ನುವ ಸಂಶಯ, ಆಗದಿದ್ದರೆ ಅನುಭವಿಸಬೇಕಾದ ಮಾನಸಿಕ ಯಾತನೆಯ ಭಯ–ಇವು ನನ್ನ ಮನಸ್ಸನ್ನು ವ್ಯಾಪಿಸಿಕೊಂಡಿದ್ದವು. ಒಂದಲ್ಲಾ ಒಂದು ಸಂದರ್ಭದಲ್ಲಿ ನೌಕರ ವರ್ಗದೊಡನೆ ಸಂಘರ್ಷವೂ ಅನಿವಾರ್ಯವಾಗಬಹುದು. ಅಂತಹ ಸಂದರ್ಭದಲ್ಲಿ, ಯಾವ ರೀತಿಯಲ್ಲಿ ಮುಂದುವರಿಯಬೇಕು ಎನ್ನುವುದು ಯಕ್ಷಪ್ರಶ್ನೆಯಾಗಿತ್ತು. ಏಕೆಂದರೆ, ಇಂತಹ ಸಂಘರ್ಷದಲ್ಲಿ ಕೇವಲ ವಾಕ್​\-ಚಾತುರ್ಯ, ತರ್ಕಶಕ್ತಿ, ವಾದದಕ್ಷತೆ ಪ್ರಯೋಜನವಾಗಲಾರದು. ವಾದದಲ್ಲಿ ಗೆದ್ದರೂ ಕೆಲಸ ಕೆಡುತ್ತದೆ. ಮಾತಿನಲ್ಲಿ ಸೋತವರು ತಮಗಾದ ಅವಮಾನವನ್ನು ಕೃತಿಯಲ್ಲಿ ವ್ಯಕ್ತಪಡಿಸದೇ ಬಿಡುವವರಲ್ಲ. ಆದ್ದರಿಂದ ಇಲ್ಲಿ ಗೆದ್ದರೂ ಗೆದ್ದಂತೆ ತೋರಿಸಿಕೊಳ್ಳದೇ, ಸೋತರೂ ಸೋತಂತೆ ಕಾಣಿಸದ ಕಲೆಯ ಸಿದ್ಧಿಯ ಅಗತ್ಯವಿತ್ತು. ಅಷ್ಟು ಮಾತ್ರವಲ್ಲ, ನೌಕರ ವರ್ಗದ ಸಿನಿಕತೆಯನ್ನು ಮೀರಿಸಬಲ್ಲ ರಚನಾತ್ಮಕ ಶಕ್ತಿಯ ಸಂಚಯ ತೀರಾ ಅಗತ್ಯವಾಗಿತ್ತು. ಸಹಜವಾಗಿ ಬುದ್ಧಿವಂತರೂ ವಿದ್ಯಾವಂತರೂ ಆದ, ಪ್ರಬಲ ಸಂಘಟನೆಯ ಬೆಂಬಲವಿರುವ, ಅನೇಕ ಬಗೆಯ ಮಾನಸಿಕ ಸಮರವನ್ನು (ಸೈಕೊಲಾಜಿಕಲ್ ವಾರ್​ಫೇರ್​) ಪರಿಣಾಮಕಾರಿಯಾಗಿ ಹೂಡುವ ಶಕ್ತಿಯುಳ್ಳ ಒಂದು ಗುಂಪಿನೊಂದಿಗೆ ಕೆಲಸ ಮಾಡುವುದು, ಮಾಡಿಸುವುದು ಸಾಕಷ್ಟು ತಲೆನೋವಿನ ವಿಷಯವಾಗಿತ್ತು. ಈ ಕೆಲಸ ಯಾವುದೇ ಗಿಮಿಕ್ಸ್​ನಿಂದ ನಡೆಯುವ ಹಾಗಿಲ್ಲ ಎನ್ನುವುದು ಮೊದಲೇ ಗೊತ್ತಿತ್ತು. ಯಾವ ತಂತ್ರವನ್ನು ಬಳಸಿದರೂ, ಅದು ತಂತ್ರವೆಂದು ಕೂಡಲೇ ಅವರಿಗೆ ತಿಳಿದುಬಿಡುತ್ತದೆ ಮತ್ತು ಅದನ್ನು ನಿಷ್ಫಲ ಮಾಡುವ ಹಟವೂ ಉಂಟಾಗುತ್ತದೆ. ಆದ್ದರಿಂದ ಹೃದಯ ಪರಿವರ್ತನೆಯಿಲ್ಲದೆ ಅನ್ಯಮಾರ್ಗವಿಲ್ಲ. ಆದರೆ ಅದು ಸುಲಭವಾಗಿ, ಶೀಘ್ರವಾಗಿ ಆಗುವ ಸಾಧನೆಯಲ್ಲ. ಆದರೂ ಈವರೆಗಿನ ಜೀವನಾನುಭವದಿಂದ ಗಳಿಸಿದ ಆತ್ಮವಿಶ್ವಾಸದಿಂದ ಪರಿಸ್ಥಿತಿಯನ್ನು ಯಶಸ್ವಿಯಾಗಿ ಎದುರಿಸಲು ಸಾಧ್ಯವಾಗಬಹುದು ಎನ್ನುವ ಸಮಾಧಾನ ತಂದುಕೊಂಡಿದ್ದೆ. ದೇವರ ದಯೆಯಿಂದ ಅದೇ ರೀತಿ ಯಶಸ್ವಿಯೂ ಆದೆ.

‘ಆ ಯಶಸ್ಸಿನ ಸಿಂಹಪಾಲು ನಾನು ಬೆಳೆಸಿಕೊಂಡ ತಾಳ್ಮೆಗೆ ಸಲ್ಲುತ್ತದೆ. ಯಾವುದೇ ಆಫೀಸಿ ನಲ್ಲಿ ಉದ್ಭವಿಸುವ ಎಷ್ಟೋ ಸಮಸ್ಯೆಗಳನ್ನು ನಿವಾರಿಸುವಲ್ಲಿ ಮೊದಲ ಪಾತ್ರ ಸಹನೆಯದ್ದೆ ಆಗಿದೆ. ಏಕೆಂದರೆ, ಎಷ್ಟೋ ಸಮಸ್ಯೆಗಳು ಕೇವಲ ಉದ್ವಿಗ್ನತೆಯ ಅಭಿವ್ಯಕ್ತಿಗಳಷ್ಟೇ. ಅಂತಹ ಅಭಿವ್ಯಕ್ತಿಗೆ ತಕ್ಕ ಅವಕಾಶ ಕೊಟ್ಟರೆ, ಅದು ಅಲ್ಲಿಗೆ ಮುಗಿದುಹೋಗುತ್ತದೆ. ಇಂಥ ಸಂದರ್ಭದಲ್ಲಿ, ಉದ್ರಿಕ್ತ ವ್ಯಕ್ತಿ ಬಳಸುವ ಶಬ್ದ, ವಾಕ್ಯ, ಹಾವಭಾವಗಳನ್ನು ಕೇವಲ ವಾಚ್ಯಾರ್ಥದಿಂದ, ಬಾಹ್ಯರೂಪದಿಂದ, ಸಿಟ್ಟು ಸಿಡುಕಿನಿಂದ ಗ್ರಹಿಸಕೂಡದು. ಹಾಗೆ ಗ್ರಹಿಸಿದರೆ ಅದನ್ನು ನಾವು ಖಂಡಿಸಬೇಕಾಗುತ್ತದೆ. ಖಂಡಿಸ ಹೊರಟರೆ, ಸಮಸ್ಯೆ ನೂರುಪಟ್ಟು ಬೆಳೆಯುತ್ತದೆ. ಇನ್ನಷ್ಟು ಕ್ಲಿಷ್ಟವಾಗುತ್ತದೆ. ಆದ್ದರಿಂದ ಅಂಥ ಸಂದರ್ಭಗಳಲ್ಲಿ ಸಹನೆಯಿಂದ ಗ್ರಹಿಸುವುದು ತೀರಾ ಅಗತ್ಯ. ಎಲ್ಲೋ ಓದಿದ ಮಾತು: ಅನೇಕ ರೋಗಿಗಳಿಗೆ ರೋಗ ನಾಶವಾಗಲು ಬೇಕಾಗಿರುವುದು ಡಾಕ್ಟರಲ್ಲ\enginline{–Audience} ಎಂದು. ಅಂದರೆ, ಆಂತರಿಕ ನೋವುಗಳನ್ನು, ನರಳಾಟವನ್ನು ತೋಡಿ ಕೊಳ್ಳುವುದಕ್ಕೆ ತಕ್ಕ ಅವಕಾಶ ಸಿಕ್ಕಿದರೆ, ಎಷ್ಟೋ ಮನುಷ್ಯರ ರೋಗಗಳು ಗುಣವಾಗುತ್ತವೆ. ಮಾನಸಿಕ ರೋಗಗಳ ವಿಷಯದಲ್ಲಂತೂ ಈ ಮಾತು ಸತ್ಯ. ನಾಲ್ಕು ಮಂದಿ ಸೇರಿ ಕೆಲಸ ಮಾಡುವಂತಹ ಆಫೀಸಿನಲ್ಲಿ ಮುಖ್ಯ ಸಮಸ್ಯೆ–ಮಾನವೀಯ ಸಂಬಂಧಗಳದ್ದು \enginline{(human relations)}. ಒಬ್ಬ ನೌಕರ ಯಾವುದೇ ಒಂದು ಸನ್ನಿವೇಶದಲ್ಲಿ ಉಗ್ರ ಅಥವಾ ವ್ಯಗ್ರ ಪ್ರತಿಕ್ರಿಯೆ ತೋರಿಸಿದಾಗ ಅನೇಕ ಬಾರಿ ಸಮಸ್ಯೆಗಳು ಉಂಟಾಗುತ್ತವೆ. ಇಂಥ ಸಂದರ್ಭದಲ್ಲಿ ತರ್ಕದಿಂದ ಸಮಸ್ಯೆಯ ಮೂಲವನ್ನು ಶೋಧಿಸುವುದು ಫಲಕಾರಿಯಾಗುವುದು ಕಷ್ಟ. ಇಂತಹದರಲ್ಲಿ ಬೇಕಾಗಿರುವುದು ವ್ಯಗ್ರತೆಯ ಉಪಶಮನ. ಅದಕ್ಕೆ ಸಹನೆಯಿಂದ, ಸಹಾನುಭೂತಿಯಿಂದ, ನೌಕರರ ಅನಿಸಿಕೆಗಳನ್ನು ಕೇಳುವುದು ಮುಖ್ಯವಾದ ಮಾರ್ಗ. ಹಾಗೆ ವ್ಯಕ್ತಪಡಿಸಿದಾಗಲೇ ಸಮಸ್ಯೆಯ ಬಗ್ಗೆ ಅವನ ಭಾವನೆಯ ತೀವ್ರತೆ ಸಾಕಷ್ಟು ಕಡಿಮೆಯಾಗುತ್ತದೆ. ಕೇಳಿದ ನಂತರ ಸಹಾನುಭೂತಿಯಿಂದ ಪರಿಶೀಲಿಸುವ ಭರವಸೆಯಿತ್ತರೆ ಅದು ಬಹುಮಟ್ಟಿಗೆ ಸಮಸ್ಯೆಯನ್ನು ಪರಿಹರಿಸುತ್ತದೆ. ಏಕೆಂದರೆ, ಮರುದಿನ ಅದು ಅವನ ಜ್ಞಾಪಕದಲ್ಲೇ ಇರುವು\-ದಿಲ್ಲ! ಹೀಗೆ ಸಹನೆ ಅನೇಕ ಸಮಸ್ಯೆಗಳನ್ನು ಬಗೆಹರಿಸಬಲ್ಲುದು. ಸಹನೆಯಿಂದ ಕೇಳಿದರೆ, ಹೇಳುವವನು ಒಂದು ರೀತಿಯ ಗೌರವವನ್ನು ಪಡೆದಂತಾಗುವುದರಿಂದ, ಅವನ ಮನಸ್ಸಿನ ವ್ಯಗ್ರತೆ ಕಡಿಮೆಯಾಗುತ್ತದೆ. ಸಮಸ್ಯೆಗಳು ಪರಿಹಾರವಾಗದಿದ್ದರೂ ಉಲ್ಬಣಗೊಳ್ಳುವ ಸಾಧ್ಯತೆ ಖಂಡಿತ ಕಡಿಮೆಯಾಗುತ್ತದೆ. ಸಹನೆಯಿಂದ ಆಗುವ ಇನ್ನೊಂದು ಪರಿಣಾಮ–ಎದುರಾಳಿಯ ಮನಸ್ಸಿನಲ್ಲಿ ಉಂಟಾಗುವ ಗೆಲುವಿನ ಭಾವ. ತನ್ನ ಅನಿಸಿಕೆಗಳನ್ನು ಈತ ತಳ್ಳಿಹಾಕಲಿಲ್ಲ, ವಿರೋಧಿಸಲಿಲ್ಲ\break ಎನ್ನುವುದು ಹೇಳಿದವನ \enginline{ego}ವನ್ನು ತುಂಬಾ ಸಾಂತ್ವನಗೊಳಿಸುತ್ತದೆ. ಇದರಿಂದ ಆತನ ಉದ್ವೇಗ ಶಮನವಾಗುತ್ತದೆ.

‘ಆದರೆ ಸಹನೆಗೂ ಮಿತಿಯಿರಬೇಕು. ಇಲ್ಲದೆ ಹೋದರೆ ಅದೇ ಒಂದು ಬಗೆಯ ದೌರ್ಬಲ್ಯವಾಗಿ ಪರಿಣಮಿಸಿ, ಸಮಸ್ಯೆಗಳನ್ನು ಸೃಷ್ಟಿಸಬಹುದು. ಏನು ಹೇಳಿದರೂ ಬಾಯಿ ಮುಚ್ಚಿಕೊಂಡು ಕೇಳುವ, ಬಡಪಾಯಿಯ ಅಸಹಾಯಕತೆಯಾಗಿ, ಸಹನೆ ಮಾರ್ಪಡಬಾರದು. ಹಾಗಾಗದಂತೆ ಇರಬೇಕಾದರೆ ಸ್ಪಷ್ಟವಾದ, ಉದ್ದೇಶಪೂರ್ವಕವಾದ ಗಂಭೀರ ದುರ್ವರ್ತನೆಗಳನ್ನೂ, ದುಶ್ಚಟಗಳನ್ನೂ, ಮಿಥ್ಯಾರೋಪಗಳನ್ನೂ ದೃಢವಾಗಿ ವಿರೋಧಿಸುವುದು ಅಗತ್ಯ. ಒಬ್ಬ ನೌಕರ ಪದೇ ಪದೇ ಅಶಿಸ್ತಿನ ವರ್ತನೆಯನ್ನು ಉದ್ದೇಶಪೂರ್ವಕವಾಗಿ ಅಥವಾ ಅಲಕ್ಷ್ಯ ಮನೋಭಾವದಿಂದ ತೋರಿಸಿದರೆ ಅದನ್ನು ವಿರೋಧಿಸಬೇಕಾಗುತ್ತದೆ. ಆಗ ಕಟ್ಟುನಿಟ್ಟಾದ ಕ್ರಮವನ್ನು ಯಶಸ್ವಿಯಾಗಿ ಕೈಗೊಳ್ಳುವ ಸಾಮರ್ಥ್ಯವನ್ನು ತೋರಿಸಿದರೆ ಮಾತ್ರ, ಸಹನೆ ಅರ್ಥಪೂರ್ಣವಾಗುತ್ತದೆ. ಸಹನೆ \enginline{positive quality} ಆಗಬೇಕಾದರೆ ಅದು ಹೀಗೆ ಶಕ್ತಿಯ ಸಂಕೇತವಾಗಬೇಕಾಗುತ್ತದೆ. ಆಗ ಮಾತ್ರ ಅದು ಪರಿಣಾಮಕಾರಿಯಾಗುತ್ತದೆ. ಆದರೆ, ಇಂಥ ಗಂಭೀರ ಕ್ರಮವನ್ನು ಕೊನೆಯ ಅಸ್ತ್ರವಾಗಿ ಮಾತ್ರ ಉಪಯೋಗಿಸಬೇಕು. ಇದು ಮುಖ್ಯ.

‘ಸಹನೆಯಂತೆಯೇ ಅತ್ಯಾವಶ್ಯಕವಾದ ಇನ್ನೊಂದು ಗುಣ ಪ್ರಾಮಾಣಿಕತೆ. ಪ್ರಾಮಾಣಿಕತೆ ಒಂದು ದೊಡ್ಡ ಶಕ್ತಿ. ಪ್ರಾಮಾಣಿಕತೆಯನ್ನು ತ್ರಿಕರಣಸಾಂಗತ್ಯ ಎಂದೂ ಕರೆಯಬಹುದು. ಅಂದರೆ ಯೋಚಿಸಿದ್ದೂ, ಹೇಳುವುದೂ, ಮಾಡುವುದೂ ಒಂದೇ ಆಗಿರುವುದು. ಉದಾಹರಣೆಗೆ ನೌಕರರಿಗೆ ಒಂದು ಭರವಸೆಯನ್ನು ಕೊಟ್ಟರೆ, ಅದನ್ನು ಕೃತಿಶಃ ನಡೆಸುವುದು ಅಥವಾ ಕನಿಷ್ಠಪಕ್ಷ ಕೃತಿಗಿಳಿಸಲು ನಿಜವಾದ ಯತ್ನ ಮಾಡುವುದು ಪ್ರಾಮಾಣಿಕತೆ; ಹಾಗೆ ಮಾಡದಿದ್ದರೆ ಅದು ಅಪ್ರಾಮಾಣಿಕತೆಯಾಗುತ್ತದೆ; ಮೋಸಗಾರಿಕೆಯಾಗುತ್ತದೆ. ಮೋಸಕ್ಕೆ ಒಳಗಾದವನು ತನಗಾದ ಅವಮಾನವನ್ನೂ, ಆಘಾತವನ್ನೂ ಮರೆಯುವುದು ಕಷ್ಟ. ಅದು ಒಂದಲ್ಲ ಒಂದು ರೂಪದಲ್ಲಿ ಒಂದಲ್ಲ ಒದು ಸಂದರ್ಭದಲ್ಲಿ, ಸ್ಫೋಟವಾಗದೇ ಇರುವುದಿಲ್ಲ. ಒಂದು ವೇಳೆ ಸ್ಫೋಟವಾಗದೆ ಹೋದರೆ, ಒಳಗೊಳಗೇ ಉರಿಯುತ್ತಿರುತ್ತದೆ. ಆದ್ದರಿಂದ ಪ್ರಾಮಾಣಿಕತೆಯಿಂದ ಮನುಷ್ಯನ ಹೃದಯವನ್ನು ಗೆಲ್ಲಲು ಸಾಧ್ಯ. ಆಡಿದ ಮಾತನ್ನು, ನೀಡಿದ ಭರವಸೆಯನ್ನು, ಕೃತಿಶಃ ನಡೆಸುವುದರಲ್ಲಿ ಯಶಸ್ವಿಯಾಗದಿದ್ದರೂ, ಪ್ರಯತ್ನ ಪ್ರಾಮಾಣಿಕವಾಗಿದ್ದರೆ, ಅದಕ್ಕೆ ಬೆಲೆ ಇದೆ, ಗೌರವ ಇದೆ, ಅದು ಸತ್​ಪರಿಣಾಮವನ್ನು ಮಾಡುತ್ತದೆ.

‘ಯಶಸ್ಸಿನ ನಿಚ್ಚಣಿಕೆಯನ್ನೇರಲು ನಮ್ಮ ನಮ್ಮ ಕೆಲಸದ ಬಗ್ಗೆ ಪರಿಪೂರ್ಣ ಜ್ಞಾನವೂ ನಮಗಿರ\-ಬೇಕಾಗುತ್ತದೆ. \enginline{Knowledge is power} ಎನ್ನುವ ಮಾತಿನಲ್ಲಿ ತಾನು ಮಾಡಬೇಕಾದ ಕೆಲಸದ ಬಗ್ಗೆ ಪೂರ್ಣವೂ, ನಿಷ್ಪಕ್ಷವೂ ಆದ ಜ್ಞಾನದ ಆವಶ್ಯಕತೆಯನ್ನೇ ಒತ್ತಿ ಹೇಳಲಾಗಿದೆ. ಅದರ ಜೊತೆಗೆ ಸಾಮಾನ್ಯ ಜ್ಞಾನ ಅಥವಾ ಅನ್ಯಕ್ಷೇತ್ರದಲ್ಲಿ ಗಳಿಸಿದ ಜ್ಞಾನ. ಅದು ಕಲೆಯೋ, ಸಾಹಿತ್ಯವೋ, ರಾಜಕೀಯವೋ, ತತ್ತ್ವಜ್ಞಾನವೋ ಆಗಿರಬಹುದು ಅಥವಾ ಭಾಷೆಗಳಿದ್ದಿರಬಹುದು. ಹೆಚ್ಚು ಭಾಷೆಗಳನ್ನು ಬಲ್ಲವನು, ಹೆಚ್ಚು ವಿಷಯಗಳನ್ನು ತಿಳಿದವನು, ತಾನು ಮಾಡುವ ಕೆಲಸವನ್ನು ಹೆಚ್ಚು ತಿಳುವಳಿಕೆಯಿಂದ ಮಾಡಲು ಸಾಧ್ಯವಾಗುತ್ತದೆ. ಅಲ್ಲದೆ ತನ್ನ ಸಹೋದ್ಯೋಗಿಗಳ ಮೇಲೆ ಉತ್ತಮ ಪ್ರಭಾವ ಬೀರಲು ಸಾಧ್ಯವಾಗುತ್ತದೆ. \enginline{Professional and General Knowledge\supskpt{\footnote{ ವಿಶೇಷ ಜ್ಞಾನ ಮತ್ತು ಪರಿಣತಿ.}}} ಇವೆರಡೂ ಕಡಿಮೆಯಾದಷ್ಟೂ ಅನ್ಯರ ಮೇಲೆ ನಮ್ಮ ಪ್ರಭಾವ ಕಡಿಮೆಯಾಗುತ್ತಾ ಹೋಗುತ್ತದೆ. ಅದರಲ್ಲೂ \enginline{Professional Knowledge} ಕಡಿಮೆಯಾದಾಗ ಮುಂದೆ ನಗೆಗೀಡಾಗುವ ಪರಿಸ್ಥಿತಿಯುಂಟಾಗಬಹುದು. ಆದ್ದರಿಂದ ಜ್ಞಾನ ಅತ್ಯಂತ ಮುಖ್ಯವಾದ ಶಕ್ತಿ. ಜ್ಞಾನದೀಪವನ್ನು ಕುಂದದಂತೆ, ನಂದದಂತೆ, ನೋಡಿಕೊಳ್ಳುವುದು ಕಾರ್ಯದಕ್ಷತೆಗೆ ಅತೀ ಅಗತ್ಯ.

‘ಹಾಸ್ಯಪ್ರಜ್ಞೆ ಮತ್ತು ಸರಸತೆ–ಇವು ಆಫೀಸು ಜೀವನದಲ್ಲಿ ಎದುರಾಗುವ ಅನೇಕ ಸಮಸ್ಯೆಗಳನ್ನು ಎದುರಿಸುವಲ್ಲಿ ತುಂಬಾ ಸಹಕಾರಿ. ಮುಖ್ಯವಾಗಿ ಮನಸ್ಸು ಉಲ್ಲಾಸದಲ್ಲಿರುವಾಗ ಎಂಥ ಸೀರಿಯಸ್ ಸಮಸ್ಯೆಯೂ ದೊಡ್ಡದಾಗುವುದಿಲ್ಲ. ಉಲ್ಲಾಸ ಇಲ್ಲದಾಗ ಸಣ್ಣಪುಟ್ಟ ವಿಷಯಗಳೂ ದೊಡ್ಡಸಮಸ್ಯೆಗಳಾಗಿ ತೋರಿ, ಆಫೀಸಿನ ಸಮತೋಲನವನ್ನು ಕೆಡಿಸುವುದುಂಟು. ಆದ್ದರಿಂದ ನೌಕರರ ಮನಸ್ಸನ್ನು ಉಲ್ಲಾಸದ ಸ್ಥಿತಿಯಲ್ಲಿರಿಸಲು ಸಾಧ್ಯವಾದ ಪ್ರಯತ್ನ ಮಾಡುವುದು ಕಾರ್ಯದಕ್ಷತೆಗೆ ತುಂಬಾ ಸಹಕಾರಿಯಾಗುತ್ತದೆ. ಈ ಕೆಲಸದಲ್ಲಿ \enginline{sense of humour} ಮತ್ತು ಸರಸತೆ ಸಾಕಷ್ಟು ಪರಿಣಾಮಕಾರಿಯಾಗುತ್ತವೆ. ಆದರೆ ಸರಸತೆ ಹರಟೆಮಲ್ಲತನವಾಗಿ ಬೆಳೆಯದಂತೆ ನಿಯಂತ್ರಿಸುವುದೂ ಅಷ್ಟೇ ಅಗತ್ಯ ಎನ್ನುವುದನ್ನು ಮರೆಯಬಾರದು. ಗಂಭೀರ ಜಗಳ ವಾಗಬಹುದಾದಂಥ ಸಂದರ್ಭವನ್ನು ಸರಿಯಾದ ಹಾಸ್ಯ ಚಟಾಕಿಗಳಿಂದ ನಿವಾರಿಸಲು ಸಾಧ್ಯ. ಹಾಗೆಯೇ ಹಾಸ್ಯಪ್ರಜ್ಞೆ ಔಚಿತ್ಯ ಮೀರಿದರೆ ಅದೇ ಗಂಭೀರ ಪ್ರಕರಣಕ್ಕೆ ಕಾರಣವಾಗಬಹುದು. ಆದರೂ ಬಹುಮಟ್ಟಿಗೆ ಹಾಸ್ಯಪ್ರಜ್ಞೆ ಸರಸತೆಗಳು ಸಮಸ್ಯೆಗಳನ್ನು ನಿವಾರಿಸುವುದಕ್ಕೆ ಸೌಹಾರ್ದವನ್ನು ಬೆಳೆಸುವುದಕ್ಕೆ ಸಹಕಾರಿಯಾಗುತ್ತವೆ. ಅವುಗಳ ಔಚಿತ್ಯಪೂರ್ಣ ಪ್ರಯೋಗ ಬದುಕಿನ ಯಶಸ್ಸಿಗೆ ಬಹುದೊಡ್ಡ ಕೊಡುಗೆಯಾಗಬಲ್ಲದು.’


\section*{ಬಾಂಧವ್ಯದ ಬೆಸುಗೆ}

\addsectiontoTOC{ಬಾಂಧವ್ಯದ ಬೆಸುಗೆ}

ವಿವಿಧತೆಯಲ್ಲಿ ಏಕತೆ ಪ್ರಕೃತಿಯ ಒಂದು ವೈಶಿಷ್ಟ್ಯ. ಮಾವಿನಮರದ ಎಲೆಗಳೆಲ್ಲ ಮಾವಿನ ಎಲೆಗಳೇ. ಆದರೆ ಲಕ್ಷ ಲಕ್ಷ ಎಲೆಗಳಲ್ಲಿ ಒಂದು ಇನ್ನೊಂದರಂತಿಲ್ಲ. ಒಂದೇ ಜಾತಿಯ ಲಕ್ಷ ಲಕ್ಷ ಪಕ್ಷಿಗಳಲ್ಲಿ ಒಂದರಂತೆ ಒಂದಿಲ್ಲ. ಮಾನವ ಕುಲ ಒಂದೇ, ಆದರೆ ವ್ಯಕ್ತಿಗಳೊಳಗಣ ಅಂತರ ಎಷ್ಟು ಸ್ಪಷ್ಟ! ಎಲ್ಲರನ್ನು ಚುಚ್ಚಿದರೂ ರಕ್ತ ಹೊರಚಿಮ್ಮುತ್ತದೆ. ಆದರೆ ಆ ರಕ್ತದಲ್ಲಿ ಸೂಕ್ಷ್ಮ ಪ್ರಭೇದಗಳು. ಅನಾರೋಗ್ಯ, ಅಸ್ವಸ್ಥತೆ ಎಂದರೆ ಒಂದೇ ಅರ್ಥ. ಆದರೆ ರೋಗಗಳಲ್ಲಿ ಸಾವಿರ ಹೆಸರಿನ ಸಾವಿರ ವಿಧಗಳಿರಬಹುದು. ಸೇವಿಸುವ ವಸ್ತು ಎನ್ನುವ ದೃಷ್ಟಿಯಿಂದ ಆಹಾರ ಒಂದೇ. ರುಚಿಯ ದೃಷ್ಟಿಯಿಂದ ಎಷ್ಟು ವೈವಿಧ್ಯ!

ಗೌತಮಬುದ್ಧ ಮನುಷ್ಯರನ್ನು ಅವರ ಆಸಕ್ತಿ, ಗಮನ, ಪರಿಸರ–ಇವುಗಳ ದೃಷ್ಟಿಯಿಂದ ನಾಲ್ಕು ವಿಧಗಳಾಗಿ ವಿಂಗಡಿಸಿದ:

\begin{enumerate}
\item ಕತ್ತಲೆಯಿಂದ ಕತ್ತಲೆಗೆ ಹೋಗುವವರು.

 \item ಬೆಳಕಿನಿಂದ ಕತ್ತಲೆಗೆ ಹೋಗುವವರು.

 \item ಕತ್ತಲೆಯಿಂದ ಬೆಳಕಿನೆಡೆಗೆ ಹೋಗುವವರು.

 \item ಬೆಳಕಿನಿಂದ ಬೆಳಕಿನೆಡೆಗೆ ಹೋಗುವವರು.

\end{enumerate}

\egroup

ಕೊಳಚೆ ಪ್ರದೇಶದಲ್ಲಿ ಹುಟ್ಟಿದ ಮಗುವನ್ನು ನೆನೆಸಿಕೊಳ್ಳಿ. ಮಗು ಹೇಗೋ ಬೆಳೆಯುತ್ತದೆ. ಶೌಚವಿಲ್ಲ, ಆಚಾರವಿಲ್ಲ, ಆರೋಗ್ಯವೂ ಚೆನ್ನಾಗಿಲ್ಲ. ವಿದ್ಯೆಯಿಲ್ಲ, ಹೊಟ್ಟೆ ಹೊರೆಯುವ ಹಂಬಲವಿದೆ. ಸತ್ಸಂಗವಿಲ್ಲ, ಸತ್ಸಂಸ್ಕಾರಗಳನ್ನು ಆ ಮಗುವಿನಲ್ಲಿ ರೂಪಿಸುವವರಿಲ್ಲ. ಆತ ಹೇಗೋ ಬೆಳೆಯುತ್ತಾನೆ. ದುಷ್ಟಕೂಟಕ್ಕೆ ಸೇರಿಕೊಳ್ಳುತ್ತಾನೆ. ಒಂದು ದಿನ ಈ ಜಗತ್ತಿನಿಂದ ಯಾವ ಉದಾತ್ತ ಭಾವನೆಗಳನ್ನೂ, ಸಂಸ್ಕಾರಗಳನ್ನೂ ಸಂಗ್ರಹಿಸದೆ ನಿರ್ಗಮಿಸುತ್ತಾನೆ. ಇವನು ಕತ್ತಲೆಯಿಂದ ಕತ್ತಲೆಗೆ ಹೋದಂತಾಯಿತು.

ಅಂಥ ಕೊಳಚೆ ಪ್ರದೇಶದಲ್ಲಿ ಜನಿಸಿದ ಮಗು ಬೆಳೆದು ದೊಡ್ಡವನಾದ. ಯಾವುದೋ ಕೆಲಸ ಕಾರ್ಯಕ್ಕಾಗಿ ನಗರಕ್ಕೆ ಹೋಗಿದ್ದ. ಅಲ್ಲಿ ಅವನ ದಕ್ಷತೆಯನ್ನು ಕಂಡು ವ್ಯಾಪಾರಿಗಳೊಬ್ಬರು ಅವನಿಗೆ ವಿದ್ಯೆಯನ್ನೂ, ತರಬೇತಿಯನ್ನೂ ಕೊಡುತ್ತಾರೆ. ಪಾಲಿಗೆ ಬಂದ ಕರ್ತವ್ಯವನ್ನು ನಿಷ್ಠೆಯಿಂದ ಮಾಡಿ ಒಳ್ಳೆಯ ಕೆಲಸಗಾರನೆನಿಸಿಕೊಳ್ಳುತ್ತಾನೆ. ಮುಂದೆ ದೊಡ್ಡ ವ್ಯಾಪಾರಿಯೂ ಆಗುತ್ತಾನೆ. ಧನ ಸಂಪಾದನೆಯ ಜೊತೆಗೆ ವಿನಯ, ಸಂಸ್ಕೃತಿಗಳನ್ನೂ ಬೆಳೆಸಿಕೊಳ್ಳುತ್ತಾನೆ. ಈತ ಕತ್ತಲೆಯಿಂದ ಬೆಳಕಿನೆಡೆಗೆ ಬಂದಂತೆ ಅಲ್ಲವೇ?

ಕೆಲವರೇನೋ ಒಳ್ಳೆಯ ಮನೆ ಪರಿಸರಗಳಲ್ಲಿ ಈ ಜಗತ್ತಿನ ಬೆಳಕನ್ನು ಕಾಣುತ್ತಾರೆ. ಯಾವ ತೆರನಾದ ಕೊರತೆ ಇಲ್ಲದೇ ಬೆಳೆಯುತ್ತಾರೆ. ವಿದ್ಯಾಬುದ್ಧಿ ಸಂಪನ್ನರ ಜೊತೆಯಲ್ಲಿ ಓಡಾಡುತ್ತಾರೆ. ಸ್ವಭಾವತಃ ಒಳ್ಳೆಯವರೇ, ಆದರೆ ದುರ್ಯೋಧನ ಕ್ಲಬ್​ನ ಮೆಂಬರಿಕೆ ಅಕಸ್ಮಾತ್ ದೊರೆತು ದುಶ್ಚಟಗಳಿಗೆ ಬಲಿಯಾಗಿ, ದುರ್ಬಲರಾಗಿ, ರೋಗಗ್ರಸ್ತರಾಗಿ ಜಗತ್ತನ್ನು ತೊರೆಯುತ್ತಾರೆ. ಇವರು ಬೆಳಕಿನಿಂದ ಕತ್ತಲೆಯೆಡೆಗೆ ಪಯಣಿಸಿದಂತೆ.

ಆಗಲೇ ಒಳ್ಳೆಯ ಕುಟುಂಬದಲ್ಲಿ ಜನಿಸಿದ್ದಾರೆ. ಶೈಶವದಿಂದಲೇ ಶುಭ ಸಂಸ್ಕಾರಗಳಿಂದ ಕೂಡಿದ್ದಾರೆ. ಸಜ್ಜನರೂ, ಸುಸಂಸ್ಕೃತರೂ, ಧರ್ಮಭೀರುಗಳಾದ ತಂದೆತಾಯಿಗಳು. ಮಹಾತ್ಮರ, ಸಿದ್ಧಪುರುಷರ ಸಹವಾಸ ಹಾಗೂ ಮಾರ್ಗದರ್ಶನ. ಆತ್ಮಸಾಕ್ಷಾತ್ಕಾರದ ಪಥದಲ್ಲಿ ಮುನ್ನಡೆದು ದಿವ್ಯಾನಂದವನ್ನು ಪಡೆಯುತ್ತಾರೆ–ಇಂಥವರು. ಇವರು ಬೆಳಕಿನಿಂದ ಬೆಳಕಿನೆಡೆಗೆ ಹೋದಂತಾಯಿತು.

ಸಂಸ್ಕೃತ ಕವಿ, ದಾರ್ಶನಿಕ ಭರ್ತೃಹರಿಯ ಮಾತು ಸಮಾಜದ ಸ್ಥಿತಿ ಗತಿಯನ್ನು ಅಳೆಯುವ ಕೀಲಿಕೈಯಂತಿದೆ. ಮನುಷ್ಯ ಸಮಾಜದಲ್ಲಿ ವಿವಿಧ ವ್ಯಕ್ತಿಗಳ ಗುಣಕರ್ಮಗಳನ್ನೂ, ವರ್ತನೆಯ ವಿಧಾನಗಳನ್ನೂ, ಶ್ಲೋಕವೊಂದರಲ್ಲಿ ತಿಳಿಸುತ್ತಾನೆ–

ತಮ್ಮ ಸುಖಸೌಕರ್ಯಗಳನ್ನು ಚಿಂತಿಸದೆ, ಬಹುಜನರ ಹಿತಕ್ಕಾಗಿ ಯಾವ ಪ್ರತಿಫಲಾಪೇಕ್ಷೆ ಇಲ್ಲದೆ ದುಡಿಯುವವರು ಕೆಲವರು. ತಮ್ಮ ವೈಯಕ್ತಿಕ ಕಷ್ಟ ಸಂಕಟಗಳನ್ನು ಗಮನಿಸದೆ ಪರೋಪ\-ಕಾರದಲ್ಲೇ ತಲ್ಲೀನರಾದ ಇವರನ್ನು ಕವಿ ‘ಸತ್ಪುರುಷರು’ ಎಂದು ಹೆಸರಿಸುತ್ತಾನೆ. ಇನ್ನು ಕೆಲವರು ತಮ್ಮ ಆಸೆ ಆಕಾಂಕ್ಷೆಗಳನ್ನು ಸರಿಯಾದ ರೀತಿಯಲ್ಲಿ ಸಾಧಿಸಿಕೊಳ್ಳುತ್ತ, ಇತರರಿಗೆ ತಮ್ಮಿಂದಾದ ಉಪಕಾರವನ್ನು ಮಾಡುವರು. ಇಂಥವರನ್ನು ಕವಿ ‘ಸಾಮಾನ್ಯ’ರೆಂದು ಕರೆಯುತ್ತಾನೆ. ತಮ್ಮ ಸ್ವಾರ್ಥಸಾಧನೆಗಾಗಿ ಇತರರ ಕೊರಳನ್ನು ಕತ್ತರಿಸಲೂ ಹಿಂದೆ ಮುಂದೆ ನೋಡದ ಪಾತಕಿಗಳನ್ನು ‘ಮಾನವರಾಕ್ಷಸ’ರೆಂದು ಹೆಸರಿಸುತ್ತಾನೆ. ಆದರೆ ಯಾವ ಸ್ವಾರ್ಥವೂ ಇಲ್ಲದೆ, ಸದಾ ಇತರರಿಗೆ ಕೇಡನ್ನು ಬಗೆಯುವವರನ್ನು ‘ಏನೆಂದು ಹೆಸರಿಸಲಿ?’ ಎಂಬುದೇ ತನಗೆ ತಿಳಿಯದೆಂದು ಕವಿ\break ಹೇಳುತ್ತಾನೆ.

ಮಾನವರಲ್ಲಿ ನಿದ್ರಾಮಾನವರಿದ್ದಾರೆ. ದುಡಿಮೆಯ ಮಾನವರಿದ್ದಾರೆ, ಚಿಂತನಶೀಲ ಮಾನವ\-ರಿದ್ದಾರೆ. ನಿದ್ರೆ ಎಲ್ಲರಿಗೂ ಬರುತ್ತದೆ. ಎಲ್ಲರಿಗೂ ಗಾಢನಿದ್ರೆಯ ಅನುಭವವಿದೆ. ಹಾಗೆಂದು ನಿದ್ರೆ ಎಲ್ಲರಿಗೂ ಸಮಾನವಲ್ಲ. ಕೆಲವರಿಗೆ ಎಷ್ಟು ನಿದ್ರೆ ಮಾಡಿದರೂ ತೃಪ್ತಿ ಇಲ್ಲ. ಕೆಲವರಿಗೆ ಆರು ಗಂಟೆ ನಿದ್ರೆ ಸಾಕು. ವಿಜ್ಞಾನಿ ನಿಕೋಲಾಯ್ ಟೆಸ್ಲಾ ಒಮ್ಮೆ ಎಪ್ಪತ್ತಾರು ಗಂಟೆಗಳ ಕಾಲ ಪ್ರಯೋಗಶಾಲೆಯಲ್ಲಿ ಎಚ್ಚೆತ್ತು ಸಂಶೋಧನೆಯಲ್ಲಿ ಮಗ್ನನಾಗಿದ್ದ ಎನ್ನುತ್ತಾರೆ. ಸ್ವಾಮಿ\break ವಿವೇಕಾನಂದರು ಪರಿವ್ರಾಜಕ ದಿನಗಳಲ್ಲಿ ಒಮ್ಮೆ ಮೂರು ದಿನಗಳ ಕಾಲ ಏಕಪ್ರಕಾರ ಮಾತನಾಡಿ, ನಿದ್ರಾಜಯಿಗಳಾಗಿ, ದಾಖಲೆ ಸ್ಥಾಪಿಸಿದ ಅತಿಮಾನುಷ ವ್ಯಾಪಾರವನ್ನು, ತಾವೇ ಶಿಷ್ಯರ ಹತ್ತಿರ ಮಾತಾಡುತ್ತ ತಿಳಿಸಿದ್ದರು.

ಜೀವನದಲ್ಲಿ ಪಾಠವನ್ನು ಕಲಿಯುವಲ್ಲೂ ವೈವಿಧ್ಯವಿದೆ. ಇತರರು ತಪ್ಪು ಮಾಡಿದ ಬಳಿಕ ದುಃಖ ಅನುಭವಿಸುವುದನ್ನು ಕಂಡು ಪಾಠ ಕಲಿಯುವವರು ಒಂದು ವರ್ಗ. ತಾವು ತಪ್ಪು ಮಾಡಿ ನೋವು ತಿಂದು ಮತ್ತೆ ಪಾಠ ಕಲಿಯುವವರು ಇನ್ನೊಂದು ವರ್ಗ. ತಪ್ಪು ಮಾಡಿ ಎಷ್ಟು ದುಃಖ ಅನುಭವಿಸಿದರೂ ಪಾಠ ಕಲಿಯದವರು ಮತ್ತೊಂದು ವರ್ಗ.

ವೈರಾಗ್ಯದಲ್ಲೂ ವೈವಿಧ್ಯವಿದೆ–ಪುರಾಣ ವೈರಾಗ್ಯ, ಶ್ಮಶಾನ ವೈರಾಗ್ಯ, ಪ್ರಸವ ವೈರಾಗ್ಯ ಇತ್ಯಾದಿ. ಆರ್ಥಿಕವಾಗಿಯೂ ಉಳ್ಳವರು, ಇಲ್ಲದವರು, ಮಧ್ಯಮವರ್ಗದವರು ಇತ್ಯಾದಿ ವೈವಿಧ್ಯಗಳು. ಜನ–ಜೀವನದಲ್ಲಿನ ವೈವಿಧ್ಯಗಳಿಗಂತೂ ಇತಿಮಿತಿಯೇ ಇಲ್ಲ!

ಚಾರಿತ್ರ್ಯ ವೈವಿಧ್ಯವೇ ವ್ಯಕ್ತಿ–ವ್ಯಕ್ತಿಯೊಳಗಿನ ಹೊಂದಾಣಿಕೆ ಕಷ್ಟಸಾಧ್ಯವಾಗುವುದಕ್ಕೆ\break ಕಾರಣ. ಜನ ವಿವಿಧ ಗುಂಪುಗಳಾಗಿ ವಿಂಗಡವಾಗುವ ಮೂಲ ಕಾರಣವು ಇಲ್ಲಿದೆ. ವಿಂಗಡವಾಗಿ\break ಪಂಗಡ\-ಗಳಾಗು\-ವುದು ದೋಷವಲ್ಲ. ಆದರೆ ದ್ವೇಷಕಾರುತ್ತ, ಪರಸ್ಪರರ ನಾಶವನ್ನು ಹಾರೈಸುವ ಪ್ರವೃತ್ತಿ ಮಾತ್ರ ಅಸಹ್ಯ.

ಮನುಷ್ಯರನ್ನು ಅವರ ಚಾರಿತ್ರ್ಯ ವೈಶಿಷ್ಟ್ಯಕ್ಕನುಗುಣವಾಗಿ ನಿಯಂತ್ರಿಸಲು ಉಪಯೋಗಿಸುವ ಉಪಾಯಗಳು ನಾಲ್ಕು ಎಂದು ಹಿತೋಪದೇಶದಲ್ಲಿ ಹೇಳಿದ್ದಾರೆ. ವಿಶ್ವಾಸದಿಂದ ಹೇಳುವ ಬುದ್ಧಿ ಮಾತು ಒಂದು ವಿಧ. ಇದನ್ನು ಸಾಮವೆಂದು ಕರೆಯುತ್ತಾರೆ. ಎಲ್ಲೋ ಕೆಲವರು ಈ ಉಪಾಯಕ್ಕೆ ಒಲಿಯುವರು. ಬಹುಮಾನದ ಭರವಸೆ ನೀಡಿ ಹೇಳುವ ಮಾತು ಹಲವರಿಗೆ ಆಪ್ಯಾಯಮಾನ ವೆನಿಸೀತು–ಅದೇ ದಾನ. ಮೂರನೆಯದೇ ಭೇದ–ಪೈಪೋಟಿಯ ಭಾವನೆಯನ್ನುಂಟುಮಾಡಿ ಕೆಲಸ ಮಾಡಿಸುವುದು. ಕೊನೆಯದೇ ದಂಡ–ಭಯದಿಂದ ಎಚ್ಚರಿಕೆಯನ್ನು ಕೊಟ್ಟು ಕೆಲಸ ಮಾಡಿಸುವುದು. ಅಪರಾಧಕ್ಕೆ ಸರಿಯಾದ ದಂಡವನ್ನು ವಿಧಿಸುವುದು ರಾಜನ ಕರ್ತವ್ಯ ಎಂಬ ಭಾವನೆ ಹಿಂದೆ ಇತ್ತು. ಆದರೆ ಇಂದು ದಂಡಿಸಲ್ಪಟ್ಟವನೂ, ಪ್ರಜಾಪ್ರಭುತ್ವದ ಸರ್ವಸಮತೆಯ ವಾದದ ಬಲದಿಂದ, ತನಗೆ ಸತ್ಪುರುಷರಿಗೆ ಲಭ್ಯವಾಗುವ ಉಪಚಾರವೇ ಸಿಗಬೇಕೆಂದು ವಾದಿಸಬಹುದು!

ಕೌಟುಂಬಿಕ ಜೀವನದಲ್ಲಾಗಲೀ, ಸಾಂಘಿಕ, ಸಾಮಾಜಿಕ ಜೀವನದಲ್ಲಾಗಲೀ, ವ್ಯಕ್ತಿ–ವ್ಯಕ್ತಿಗ\-ಳೊಳಗಿನ ಮಧುರ ಬಾಂಧವ್ಯದ ಮೂಲಸೂತ್ರ ಯಾವುದು? ಸ್ವಾರ್ಥ ತ್ಯಾಗ. ಹಕ್ಕು, ಬಾಧ್ಯತೆಗಳ ಹೋರಾಟಕ್ಕಿಂತ, ಅಹಂಕಾರವನ್ನು ಮರೆತು ತನ್ನ ಪಾಲಿನ ಕರ್ತವ್ಯವನ್ನು ಪ್ರೀತಿಯಿಂದ ಮಾಡುವ ಸ್ಥೈರ್ಯ. ಇದರ ಮೂಲದಲ್ಲಿ ಕೆಲಸ ಮಾಡುವ ತಥ್ಯ–ಭಗವಂತನಲ್ಲಿ ಶ್ರದ್ಧೆ. ಬಾಯಿಯಲ್ಲಿ ತನಗೆ ದೇವರಲ್ಲಿ ನಂಬಿಕೆ ಇದೆ ಅಥವಾ ಇಲ್ಲ ಎಂದರೂ, ಸ್ವಾರ್ಥ, ಒಣಜಂಭ, ಅಹಂಕಾರವನ್ನು ದೂರಮಾಡಿ, ತಾಳ್ಮೆಯಿಂದ ತನ್ನ ಕರ್ತವ್ಯಪಾಲನೆಯ ಮೂಲಕ, ವಿಶ್ವ ನಿಯಾಮಕನನ್ನು ತುಷ್ಟಿಗೊಳಿಸಬೇಕೆನ್ನುವ ವ್ಯಕ್ತಿ ಶಾಂತಚಿತ್ತನಾಗಿರಬಲ್ಲ. ಚಿಂಕ್ರೋಭದ ತೊಂದರೆಗಳಿಂದ ಮುಕ್ತನಾಗಬಲ್ಲ. ಲೋಕದ ವ್ಯವಹಾರದಲ್ಲಿ, ಆಸೆಗಳನ್ನು ತೃಪ್ತಿಪಡಿಸಿ ಕೊಳ್ಳುವ ಆತುರದಲ್ಲಿ, ಮನುಷ್ಯ ತಾನು ಮಾಡುತ್ತಿರುವುದು ಧರ್ಮವೇ ಎಂದು ಯೋಚಿಸುತ್ತ ಮುಂದೆ ಹೆಜ್ಜೆ ಇಡಬೇಕು.


\section*{ಎಲ್ಲರನು ಸಲಹುವನು}

\vskip -7pt\addsectiontoTOC{ಎಲ್ಲರನು ಸಲಹುವನು}

ಈ ಎಲ್ಲ ಕಷ್ಟಕೋಟಲೆಗಳ ನಡುವೆಯೂ, ಪಾರುಮಾಡೆಂದು ದೇವರೊಡನೆ ಮೊರೆ ಹೋಗುವುದರಿಂದ ಚಿಂಕ್ರೋಭದ ಕಾಟವಿರಲಿ, ಬೇರಾವ ಕಷ್ಟಕಂಟಕಗಳೇ ಇರಲಿ, ಅವೆಲ್ಲ ಕರಗಿಯೇ ಹೋಗುವುದು ಖಂಡಿತ. ಆ ಕಷ್ಟಗಳನ್ನೆಲ್ಲ ದೇವರೊಡ್ಡುವ ಸತ್ವಪರೀಕ್ಷೆ ಎಂದೇ ಎಣಿಸಿ ಎದುರಿಸಬೇಕು. ಅವುಗಳಿಂದಾಗಿಯೇ ನಮ್ಮ ವ್ಯಕ್ತಿತ್ವಕ್ಕೆ ಪುಟವಿಟ್ಟಂತಾಗುವುದು. ಸಾವಿರ ಉಳಿ ಏಟಿನಿಂದ ತಾನೇ ಒಂದು ಶಿಲ್ಪ?

ಭಗವಂತ ಭಕ್ತವತ್ಸಲ, ಕರುಣಾಸಾಗರ, ದಯಾನಿಧಿ, ಭಕ್ತನ ಮೊರೆ ತೀವ್ರವಾಗಿದ್ದರೆ, ಎಂದರೆ ನೀವು ಸರಿಯಾಗಿ \enginline{apply} ಮಾಡಿದ್ದರೆ, ಒಂದೋ ಅಭಯವನ್ನಿತ್ತು ಆತ \enginline{reply} ಕೊಡುತ್ತಾನೆ, ಇಲ್ಲವಾದರೆ ನೀವು ಬೇಡಿದ್ದನ್ನು \enginline{supply} ಮಾಡಿ ನಿಮ್ಮನ್ನು ಸಲಹುತ್ತಾನೆ. ಹಾಗಾಗಿ ಬಿಡದೆ ಆತನ ನಾಮಸ್ಮರಣೆ ಮಾಡುತ್ತಿದ್ದರೆ, ಆತನ ಬೆಂಗಾವಲು ತಪ್ಪುವುದೇ ಇಲ್ಲ. ಸ್ತುತಿ, ಭಜನೆ, ಪಾರಾಯಣಗಳಿಂದಲೂ ಆತನಲ್ಲಿ ಭಕ್ತಿಯನ್ನು, ಅಂದರೆ ದೃಢವಾದ ನಂಬಿಕೆಯನ್ನು ಬೆಳೆಸಿಕೊಳ್ಳಬೇಕು.

ನಮ್ಮಲ್ಲಿ ಹೆಚ್ಚಿನವರ ದೇವರಲ್ಲಿನ ನಂಬಿಕೆ ಮಕ್ಕಳು ಆಟಕ್ಕಾಗಿ ‘ದೇವರಾಣೆ’ ಎಂದಷ್ಟೇ ಬಲವುಳ್ಳ ನಂಬಿಕೆ! ದೇವರಲ್ಲಿ ನಿಜವಾದ ನಂಬಿಕೆ ಬಂದರೆ ಎಲ್ಲ ಸಮಸ್ಯೆಗಳೂ ಇತ್ಯರ್ಥ\-ವಾದಂತೆ. ನಂಬಿಕೆ ಇಲ್ಲದಿದ್ದಲ್ಲಿ ಅಥವಾ ಸ್ವಲ್ಪ ಮಾತ್ರವಿದ್ದಲ್ಲಿ, ಅದನ್ನು ಪಡೆದು ವೃದ್ಧಿಗೊಳಿಸಿಕೊಳ್ಳಲು ಎಲ್ಲರಿಗೂ ಸಾಧ್ಯ. ಖಂಡಿತ ಅಸಾಧ್ಯವಿಲ್ಲ. ಇದು ಒಂದೇ ದಿನದಲ್ಲಿ ಆಗುವ ಕೆಲಸವಲ್ಲ ಅಥವಾ ಪೇಟೆಗೆ ಹೋಗಿ ತರಕಾರಿ ತಂದಂತಲ್ಲ. ದೀರ್ಘಕಾಲದ ಸಾಧನೆ, ಭಜನೆ, ನಿಷ್ಠೆಯ ಪ್ರಾರ್ಥನೆ, ಸತ್ಸಂಗ, ಸದ್ಗ್ರಂಥಗಳ (ಸಂತರ, ಶ‍್ರೀಸಾಮಾನ್ಯರ ಬದುಕಿನಲ್ಲಿ ಭಗವಂತನ ಕೃಪೆ ಒದಗಿದ ರೀತಿ, ವಿಧಾನಗಳನ್ನು ಕುರಿತ) ಅಧ್ಯಯನ ಹಾಗೂ ಸ್ವಲ್ಪವಾದರೂ ಸ್ವಾನುಭವ– ಇವುಗಳಿಂದ ಶ್ರದ್ಧೆ ದೃಢವಾಗುವುದು. ಶ್ರದ್ಧೆ ದೃಢವಾಗುವವರೆಗೂ ಕಾಯುತ್ತ ಕೂಡಬೇಕಿಲ್ಲ. ಅತ್ಯಂತ ಆಪ್ತರನ್ನು ಸಮೀಪಿಸಿ ನಮ್ಮ ದುಃಖಸಂಕಟಗಳನ್ನು ಅವರೆದುರು ತೋಡಿಕೊಂಡು ಸಹಾಯ ಯಾಚಿಸುವಂತೆ, ಭಗವಂತನಲ್ಲಿ ಕಂಬನಿದುಂಬಿ, ಏಕಾಂತದಲ್ಲಿ ನಿರಂತರ ಪ್ರಾರ್ಥನೆ ಸಲ್ಲಿಸಿದರೂ ಎಷ್ಟೋ ಒಳಿತಾಗುವುದು; ಮನಃಸ್ಥೈರ್ಯ ಉದಿಸುವುದು. ಸಂಕಟ ಬಂದಾಗ ಜನ ಬೇರೆ ಬೇರೆ ದೇವಾಲಯಗಳಿಗೆ ಹೋಗಿ, ಹರಕೆ, ಕಾಣಿಕೆಗಳನ್ನು ಸಲ್ಲಿಸಿ ಪ್ರಾರ್ಥಿಸುತ್ತಾರೆ. ಇದರಿಂದ ಅಲ್ಪಸ್ವಲ್ಪ ಉಪಯೋಗವಾಗುವುದು. ಶ್ರದ್ಧೆ ಇದ್ದಲ್ಲಿ ಇನ್ನೂ ಹೆಚ್ಚಿನ ಉಪಯೋಗವಾಗುವುದು. ಸದ್ಯದ ಪರಿಸ್ಥಿತಿಗೆ ಏನೇ ಕಾರಣವಿರಲಿ, ಭಗವಂತನಿಗೆ ಅಸಾಧ್ಯವಾದುದು ಯಾವುದೂ ಇಲ್ಲ ಎಂಬ ಶ್ರದ್ಧೆ, ಅದಕ್ಕನುಗುಣವಾದ ಪ್ರಾರ್ಥನೆ, ಮನಸ್ಸಿಗೆ ಅಪಾರ ಸ್ಥೈರ್ಯ ನೀಡಬಲ್ಲುದು. ನೆನಪಿಡಿ:

ಶ್ರದ್ಧಾರಹಿತ ಪ್ರಾರ್ಥನೆ ಸ್ಟ್ಯಾಂಪ್​ರಹಿತ ಪತ್ರದಂತೆ.

ಭಕ್ತಿರಹಿತ ಪ್ರಾರ್ಥನೆ ವಿಳಾಸರಹಿತ ಪತ್ರದಂತೆ.

ಶ್ರದ್ಧಾಭಕ್ತಿಸಹಿತ ವ್ಯಾಕುಲತೆಯಿಂದ ಮಾಡಿದ ಪ್ರಾರ್ಥನೆ ಟೆಲಿಗ್ರಾಂ ಕೊಟ್ಟಂತೆ.


\section*{ಅಧ್ಯಾತ್ಮದ ಔನ್ನತ್ಯ, ಕರ್ತವ್ಯದ ಸಾರ್ಥಕ್ಯ}

\vskip -6.2pt\addsectiontoTOC{ಅಧ್ಯಾತ್ಮದ ಔನ್ನತ್ಯ, ಕರ್ತವ್ಯದ ಸಾರ್ಥಕ್ಯ}

‘ಈಸಬೇಕು ಇದ್ದು ಜೈಸಬೇಕು’ ಎಂಬ ದಾಸರ ನುಡಿಯನ್ನು ಅಕ್ಷರಶಃ ಪಾಲಿಸುತ್ತ ಆಧ್ಯಾತ್ಮಿಕವಾಗಿ ಮುನ್ನಡೆದ, ಸಂಸಾರದಲ್ಲಿದ್ದರೂ ಅದಕ್ಕಂಟಿಕೊಳ್ಳದೆ, ಧೈರ್ಯದಿಂದ ಸಾರ್ಥಕಜೀವನ ನಡೆಸಿದ ತಾಯಿಯೊಬ್ಬರು ಕೊನೆಗಾಲದಲ್ಲಿ ತಮ್ಮ ಮಗನನ್ನು ಹತ್ತಿರ ಕರೆದು ತಮ್ಮ ಬದುಕಿನಲ್ಲಿ ಕಲಿತ ಪಾಠಗಳನ್ನು ಕುರಿತು ಹೀಗೆಂದರು: ‘ಭಗವಂತನನ್ನು ನೆನೆಯದ ಮನ ಅಪವಿತ್ರ, ಆತನ ಸ್ಮರಣೆ ಮಾಡದ ದಿನ ಅಮಂಗಳ–ಎಂದು ನನ್ನ ಬಾಲ್ಯದಲ್ಲೇ ನನ್ನ ಹಿರಿಯರಿಂದ ನಾನು ಕಲಿತದ್ದನ್ನು ಅಕ್ಷರಶಃ ಪಾಲಿಸಲು ಯತ್ನಿಸಿದೆ. ದೇವರ ಕೋಣೆಯಲ್ಲಿರುವ ದೇವರು ನಾವು ಮಾಡುವ ಕೆಲಸ ಕಾರ್ಯಗಳನ್ನು ತದೇಕಚಿತ್ತದಿಂದ ನೋಡುತ್ತಾನೆ ಎಂದು ಚಿಕ್ಕಂದಿನಲ್ಲಿ ತಿಳಿದು, ನನ್ನ ಕೆಲಸದಲ್ಲಿ ನಾಜೂಕುತನ, ಚೊಕ್ಕಟತನಗಳನ್ನು ಕಲಿತೆ. ಅನಂತರ ಪತಿಯೇ ದೇವರು ಎಂದು ತಿಳಿದಾಗ, ಪತಿಗೆ ಎಳ್ಳಷ್ಟೂ ಬೇಸರವಾಗದಂತೆ ಅವರ ನೆರಳಾಗಿ ಬಾಳಿ ಬದುಕಿದೆ. ಈ ಮಧ್ಯೆ ಅವರೆಣಿಕೆಯಂತೆ ನಾನು ಸಹಕರಿಸಲಾರದಾದಾಗ ಅವರ ಕ್ಷಮೆ ಕೇಳಿದೆ. ದೇವರ ಎದುರು ಕಣ್ಣೀರಿಟ್ಟು ನನ್ನನ್ನು ತಿದ್ದುವಂತೆ ಬೇಡಿದೆ. “ಚಿನ್ನ, ಸೀರೆ, ವಾಚು, ಪಾದರಕ್ಷೆ ಇತ್ಯಾದಿ ಯಾವ ವಸ್ತುವ್ಯಾಮೋಹವೂ ನನ್ನನ್ನು ಕಾಡದೆ, ಹೇ ದೇವರೆ, ನಿನ್ನ ನಾಮದಲ್ಲಿ ರುಚಿಕೊಟ್ಟು, ನಿನ್ನ ರಾಜ್ಯದಲ್ಲಿ ಎಲ್ಲರೂ ನಿನ್ನ ಪ್ರತಿನಿಧಿಗಳೇ ಎಂದು ತಿಳಿದುಕೊಂಡು ಎಂತಹ ಪೆಟ್ಟನ್ನೂ, ಎಂತಹ ಅವಮಾನವನ್ನೂ, ತಾಳ್ಮೆಯಿಂದ, ಸಹಿಸಿಕೊಳ್ಳುವ ಶಕ್ತಿಕೊಡು ದೇವಾ!” ಎಂದು ಕಣ್ಣೀರಿಟ್ಟು ದಿನಂಪ್ರತಿ ದೇವರಲ್ಲಿ ಅಳುತ್ತಿದ್ದೆ. ಮಕ್ಕಳು ದೇವರು ಕೊಟ್ಟ ವರ ಎಂದು, ಅವರ ಆರೈಕೆ ಮಾಡುವಾಗಲೂ, “ಸಾಕ್ಷಾತ್ ದೇವರು ನೋಡುತ್ತಿದ್ದಾನೆ, ಗಂಡನಿಗೆ ಬೇಸರವಾಗದಂತೆ ಮಕ್ಕಳ ಆರೋಗ್ಯ ಕಾಪಾಡಬೇಕು” ಎಂಬ ಏಕಮೇವ ಉದ್ದೇಶದಿಂದ, ನನ್ನ ಕರ್ತವ್ಯವನ್ನು ಪರಿಪಕ್ವವಾದ ರೀತಿಯಲ್ಲಿ ಮಾಡಿದ್ದೆ. ಮಕ್ಕಳು ಅಳುವುದು, ಹಟ ಮಾಡುವುದು ಎಂದರೆ ಅವರಿಗೆ ಆಗುತ್ತಿರಲಿಲ್ಲ. ಈ ಎಲ್ಲ ಷರತ್ತುಗಳು ಭಗವಂತ ನನಗೊಡ್ಡಿದ ಸವಾಲು ಎಂದು ದೃಢವಾಗಿ ನಂಬಿ ನಾನು ಬಾಳುತ್ತಿದ್ದೆ. ಬಂದ ಸೊಸೆ ಹೊಂದಿಕೊಳ್ಳದಾದಾಗ ಪುನಃ ಕಣ್ಣೀರಿಟ್ಟು ಬೇಡಿದೆ. ಆಗ ನನಗೆ ದೊರೆತ ದೇವರ ಸಂದೇಶ ಏನು ಗೊತ್ತೆ? “ನೋಡು, ಎಲ್ಲವೂ ನಿನಗೆ ಅನುಕೂಲವಾಗಿದ್ದರೆ ನೀನು ಮಾಯೆಯತ್ತ ಪಯಣಿಸುವೆಯಲ್ಲವೆ? ಮಾತೆಯ ದರ್ಶನವಾಗಬೇಕಾಗಿರುವ ಈ ಕಾಲದಲ್ಲಿ ಮಾಯೆಯ ದಾರಿಯಿಂದ ದೂರವಿರಲು, ನಿನ್ನ ತಾಳ್ಮೆ ಇನ್ನೂ ಪ್ರಕಟಗೊಳ್ಳಲು, ವಿಧಿಯು ಸೊಸೆಗೆ ಈ ಬುದ್ಧಿ ಕೊಟ್ಟಿದೆ, ಮಗೂ ಸಹಿಸಿಕೋ” ಎಂಬುದಾಗಿ. ನೋಡು, ಅಂದಿನಿಂದ ಮೊಮ್ಮಕ್ಕಳ ಲಾಲನೆಪಾಲನೆಯ ಹೊಣೆಯೂ ನನ್ನ ಮೇಲೆ ಬಿತ್ತು. ಚಿನ್ನ ಪುಟಗೊಳ್ಳಲು ಅಕ್ಕಸಾಲಿ ಅದನ್ನು ಕಾಯಿಸಿ, ಬಡಿದು ಹೊಡೆಯುವಂತೆ ಪರಮಾತ್ಮ ನನ್ನನ್ನು ಈ ರೀತಿಯಲ್ಲಿ ಪರಿಶುದ್ಧಗೊಳಿಸುತ್ತಾನೆ ಎಂದು ನಂಬಿ, ಹಗಲು ರಾತ್ರಿ ಕಣ್ಣಿನ ಮೇಲೆ ಕಣ್ಣಿಟ್ಟು ಮೊಮ್ಮಕ್ಕಳನ್ನು ಕಂಡೆ. ಆದರೆ ಅಲ್ಲಿಯೂ ಒಂದೇ ಒಂದು ಒಳ್ಳೆಯ ಮಾತನ್ನು ಕೇಳಲಿಲ್ಲ. ಯಾವ ಕ್ಷಣದಲ್ಲಿ ಈ ಜಗತ್ತಿನ ವ್ಯಾಮೋಹಕ್ಕೆ ಬಲಿಬಿದ್ದು ನಾನು ದೇವರಿಂದ ದೂರವಾಗುತ್ತೇನೆಂಬುದು ಖಾತ್ರಿ ಯಾಯಿತೋ, ಆಗ ನಾನೇ ಕಾಡಿ ಬೇಡಿದೆ. ದೇವರ\break ಸಾನ್ನಿಧ್ಯಕ್ಕಾಗಿ ಈಗ ಹಾತೊರೆಯುತ್ತಿದ್ದೇನೆ. ನಿಮ್ಮೆಲ್ಲರನ್ನೂ ತ್ಯಜಿಸಲು ಸಿದ್ಧಳಾಗಿದ್ದೇನೆ. ಮಗೂ, ಯಾರಿಗೆ ಈ ಬಾಳು ಶಾಶ್ವತ? ನಾನೇನೂ ಓದಿ ಪರೀಕ್ಷೆ ಪಾಸು ಮಾಡಿದವಳಲ್ಲ. ಆದರೂ ಕೈ ಹಿಡಿದ ಗಂಡನಿಗೆ, ನಂಬಿದ ದೇವರಿಗೆ, ಆತ್ಮೀಯ ಗುರುಹಿರಿಯರಿಗೆ ಅತೀವ ವಿಧೇಯತೆಯಿಂದಲೇ ನನ್ನ ಬಾಳು ಉದ್ಧಾರವಾಯಿತೆನಿಸುತ್ತದೆ. ಜೀವನದಲ್ಲಿ ಬರುವ ದುಃಖದುಗುಡಗಳನ್ನೆಲ್ಲ ಭಗವಂತನ ಪರೀಕ್ಷೆ ಎಂದು ತಿಳಿದು ತಾಳ್ಮೆಯಿಂದ ನುಂಗಿಕೊಂಡು, ದೇವರೊಡನೆ ಬೇಡಿ ಕಾಡಿ, ಕಣ್ಣೀರು ಬಿಡುತ್ತಾ ಮುನ್ನಡೆದರೆ ಖಂಡಿತ ಶ್ರೇಯಸ್ಸಾಗುತ್ತದೆ. ಮಗೂ, ನೀನೂ ಆ ದಾರಿಯಲ್ಲಿ ಮುನ್ನಡೆಯಬೇಕು.’

ಹೌದು, ಮಕ್ಕಳಲ್ಲಿ ದೈವತ್ವವನ್ನು ಕಂಡು ಮಾಡುವ ಸೇವೆಯೇ ದೇವರ ಸೇವೆ.\footnote{ ‘ಕಂದನಲ್ಲಿ ಶಿವನ ಕಾಣುವ ಬಂಧನವೆ ಮುಕ್ತಿ’–ಕುವೆಂಪು} ಭಗವಂತನ ರಾಜ್ಯದಲ್ಲಿ ವಿಧೇಯತೆ, ಸೇವಾಪರಾಯಣತೆ ಹಾಗೂ ತಾಳ್ಮೆಗಳಿಗೇ ಮಾರ್ಕ್ಸ್ ಕೊಡಲಾಗುತ್ತದೆ. ಆದರೆ ನಾವು ಆ ಮಹಾ ಗುಣಗಳನ್ನು ಬೆಳೆಸಿಕೊಳ್ಳುತ್ತಿದ್ದೇವೆಯೇ? ಇಂದಿನ ನಮ್ಮ ಮಕ್ಕಳಲ್ಲಿ ಬೆಳೆಸುತ್ತಿದ್ದೇವೆಯೇ? ದೇವರೆಂದರೆ ಪಳೆಯುಳಿಕೆ, ಇಂದ್ರಿಯ ಭೋಗವೇ ಸರ್ವಸ್ವ, ಸ್ವಾರ್ಥ ಸಾಧನೆಯೇ ಪರಮಗುರಿ ಎಂದೆಲ್ಲ ನಂಬಿ ನಡೆಯುವ ಆಧುನಿಕರೇ ತುಂಬಿರುವ ನಮ್ಮ ಸಮಾಜ\-ವಾಗಲಿ, ಸರಕಾರವಾಗಲಿ ಈ ಬಗ್ಗೆ ಯೋಚಿಸುತ್ತಿದೆಯೇ? ದೇಶದ ರಕ್ಷಣೆಗೆಂದು ಕೋಟಿಗಟ್ಟಲೆ ಹಣ ಸುರಿಯುವ ಸರಕಾರ, ಜನಮನದಲ್ಲಿ ದ್ವೇಷ, ಅಂಜಿಕೆ, ಅಸೂಯೆಗಳಿಗೆ ಎಳ್ಳಷ್ಟೂ ಇಂಬು\-ಗೊಡದೆ ಪರಸ್ಪರ ಸ್ನೇಹ ಸೌಹಾರ್ದತೆಗಳನ್ನು ಉಕ್ಕಿಸಲೆಂದೇ ರಚನಾತ್ಮಕ ಧೋರಣೆ ತಳೆದರೆ ಅದೆಷ್ಟು ಚೆನ್ನ! ಆಗ ಚಿಂಕ್ರೋಭದ ಪೀಡೆ ತೊಲಗಿಸಲು ಆಟಂಬಾಂಬನ್ನೂ ಮೀರಿಸಿದ ಬಾಂಬ್ ಇದಾದೀತು!


\section*{ಪುಣ್ಯಸ್ಮೃತಿಯ ಪರಂಪರೆ}

\addsectiontoTOC{ಪುಣ್ಯಸ್ಮೃತಿಯ ಪರಂಪರೆ}

ಮಾತೃತ್ವದ ಈ ಆದರ್ಶವನ್ನು ಮೈಗೂಡಿಸಿಕೊಂಡು, ತ್ಯಾಗ ಸೇವೆಗಳಿಂದ ತಮ್ಮ ಬಾಳನ್ನು ಗಂಧದಂತೆ ತೇದ ಅಸಂಖ್ಯ ಮಹಾ ಮಹಿಳೆಯರು ಈ ನಾಡಿನಲ್ಲಿ ಬೆಳಗಿ ಬಾಳಿದ್ದಾರೆ. ಮಕ್ಕಳನ್ನು ಪ್ರಸವಿಸಿದ ಮಾತ್ರಕ್ಕೇ ಅವರು ಮಾತೃತ್ವದ ಮಹಾ ಪ್ರತಿನಿಧಿಗಳಾಗುತ್ತಾರೆನ್ನಲಾಗದು. ಅದಕ್ಕೆ ಆಧ್ಯಾತ್ಮಿಕ ಹಿನ್ನೆಲೆ ಬೇಕು. ನಿಃಸ್ವಾರ್ಥತೆಯ ಉತ್ತುಂಗ ಶಿಖರವನ್ನೇರಿ ಆಧ್ಯಾತ್ಮಿಕ ದಿವ್ಯಾನು ಭೂತಿಯ ಗುರಿಯಲ್ಲಿ ದೃಢವಿಶ್ವಾಸವಿರಿಸಿದ ಅಂಥವರ ಸದ್ಗುಣದ ಸೌರಭ ಸತ್ಯಾನ್ವೇಷಿಗೆ ಸ್ಫೂರ್ತಿ\-ದಾಯಕ. ಖ್ಯಾತ ವಿದ್ವಾಂಸರೂ, ಬರಹಗಾರರೂ ಆದ ಶ‍್ರೀ ಡಿ.ವಿ.ಜಿ. ಅವರ ‘ಭಗವದ್ಗೀತಾ ತಾತ್ಪರ್ಯ’ ಎಂಬ ಪುಸ್ತಕ ಪ್ರಸಿದ್ಧವಾಗಿದೆ. ನೂರಾರು ಪುಟಗಳಲ್ಲಿ ವಿಸ್ತೃತವಾಗಿ ವಿವರಿಸಲ್ಪಟ್ಟ ಗೀತಾ ಸಂದೇಶವು, ಒಂದೆಡೆ ಅತ್ಯಂತ ಸಾರಗರ್ಭಿತವಾಗಿ ಸೂಚಿಸಲ್ಪಟ್ಟಿದೆ. ಅದೇ ಗ್ರಂಥಕರ್ತರು ಯಾರ ಹೆಸರಿಗೆ ಗ್ರಂಥವನ್ನು ಅರ್ಪಿಸಿದರೊ ಅವರ ಬದುಕು–ಗ್ರಂಥಕರ್ತರ ಮಾತುಗಳನ್ನೇ ಕೇಳಿ: ‘ಬಾಲ್ಯದಲ್ಲಿಯೇ ಒದಗಿದ ವೈಧವ್ಯ ದುಃಖಕ್ಕೆ, ಅದು ದೈವವ್ಯವಸ್ಥೆ ಎಂದುಕೊಂಡು ತಲೆಬಾಗಿ, ಭಗವದ್ಭಕ್ತಿ, ಸದಾಚಾರ, ಶ್ರದ್ಧೆಗಳ ದೃಢಬಲದಿಂದ ಎಲ್ಲ ಕಷ್ಟಗಳನ್ನು ಸಹಿಸಿ, ವೃದ್ಧ ಶುಶ್ರೂಷೆ, ಅನಾಥ ಶಿಶು ಪೋಷಣೆ, ದೀನದುರ್ಬಲ ಪರಿಚರ್ಯೆ ಮೊದಲಾದ ಸತ್ಕಾರ್ಯಗಳಲ್ಲಿ ಆಯುಷ್ಯವನ್ನು ತೇಯ್ದು, ನಮ್ಮ ಪಾಲಿಗೆ ಪುಣ್ಯ ಸ್ಮೃತಿಯನ್ನು ಇಟ್ಟುಹೋಗಿರುವ, ನನ್ನ ತಂಗಿ ಲಕ್ಷ್ಮೀದೇವಮ್ಮ ಇವರ ಸಾಧು ಜೀವನಕ್ಕೆ ಇದು ಅಂಕಿತ.’

ಇಲ್ಲಿ ತ್ಯಾಗ ಮತ್ತು ಸೇವೆ ಒತ್ತಾಯದ ಒದ್ದಾಟವಲ್ಲ. ಗೊಣಗುತ್ತ ಮಾಡುವ ಸೇವೆಯ ಸೋಗಲ್ಲ. ಧನ್ಯತೆ, ಕೃತಕೃತ್ಯತೆಯನ್ನುಂಟುಮಾಡುವ ತಪೋನಿಷ್ಠ ಅನುಷ್ಠಾನ.

ಭಗವಂತನಲ್ಲಿ ದೃಢವಾದ ಶ್ರದ್ಧಾಭಕ್ತಿಗಳಿಲ್ಲದೇ ಇಂಥ ಆದರ್ಶವನ್ನು ಕಾರ್ಯರೂಪಕ್ಕೆ ತರಲು ಸಾಧ್ಯವೇ?

ಇರುವ ದೇವರನ್ನು ಇಲ್ಲವೆಂದು ತಿಳಿದುಕೊಳ್ಳುವುದೇ ಬುದ್ಧಿವಂತಿಕೆ, ವೈಚಾರಿಕತೆ, ವೈಜ್ಞಾನಿ ಕತೆ ಎನ್ನುವಂಥ ಸಂದಿಗ್ಧ ಮನೋವೃತ್ತಿಯ ಆಧುನಿಕ ವಿದ್ಯಾವಂತರಿಗೆ, ಮೇಲೆ ಹೇಳಿದ ಆದರ್ಶ\-ಜೀವನ ಅರ್ಥವಾದೀತೇ? ಆಪ್ಯಾಯಮಾನವೆನಿಸೀತೇ? ಅವರು ನಂಬಬಹುದು, ಬಿಡಬಹುದು. ಅವರಿಗೆ ಹಿಡಿಸಬಹುದು, ಹಿಡಿಸದಿರಬಹುದು. ಆದರೆ ಸತ್ಯವು ಸತ್ಯವೇ. ಅದನ್ನು ನಮ್ಮ ನಿಮ್ಮ ಫ್ಯಾಷನ್ನಿನಿಂದ ಬದಲಿಸಲು ಸಾಧ್ಯವಿಲ್ಲ.

ಈ ದೇಶದಲ್ಲಿ ಜಗದಾದಿಕಾರಣವನ್ನು ಅಥವಾ ದೇವರನ್ನು ವಿವಿಧ ಭಾವಗಳಿಂದ ಸಮೀಪಿಸ ಬಹುದೆಂಬುದನ್ನು, ಸ್ವಾನುಭವದಿಂದ ಕಂಡುಕೊಂಡು, ಈ ಸತ್ಯವನ್ನು ಸಂತರು ಬೋಧಿಸಿದ್ದರು. ಈ ಯುಗದಲ್ಲಿ ಶ‍್ರೀರಾಮಕೃಷ್ಣರು ತಮ್ಮ ಬದುಕಿನಲ್ಲಿ ಅದನ್ನು ಸಾಧಿಸಿ, ದಿವ್ಯ ಅನುಭೂತಿಯನ್ನು ಪಡೆದು, ಶಿಷ್ಯರಿಗೂ ಮಾರ್ಗದರ್ಶನ ಮಾಡಿ, ಆ ವಿಚಾರದ ಸತ್ಯತೆಯನ್ನು ಪುನಃ ಸಾರಿದರು. ‘ದೇವರನ್ನು ತಂದೆ ಎನ್ನುವ ಭಾವದಿಂದ ಪೂಜಿಸಬಹುದು. ತಾಯಿ ಎನ್ನುವ ಭಾವದಿಂದ ಪ್ರಾರ್ಥಿಸಿ ಸಮೀಪಿಸಬಹುದು. ಭಗವಂತನನ್ನು ತಾಯಿ ಎಂದು ಸಂಬೋಧಿಸಲಾರಂಭಿಸಿದರೆ, ಬಹುಬೇಗ ಹೃದಯದಲ್ಲಿ ಭಕ್ತಿ ಉಂಟಾಗಿಬಿಡುತ್ತದೆ’ ಎಂದರು ಶ‍್ರೀರಾಮಕೃಷ್ಣರು. ‘ಸಖನೆಂದು ನಂಬಿ ಆಧ್ಯಾತ್ಮಿಕವಾಗಿ ಮುನ್ನಡೆಯಬಹುದು. ದಾಸ್ಯಭಾವದಿಂದ ಸಾಧನೆ ಮಾಡಬಹುದು. “ಬೇರೆ ಮತಿ, ಬೇರೆ ಮತ”–ವಿಭಿನ್ನರುಚಿಯ ಜನರಿಗೆ ವಿಭಿನ್ನ ಪಥಗಳು. ಭಗವಂತನ ಸಾನಿಧ್ಯ ಸಾಧನೆಗೆ ಈ ಯುಗದ ವಿಶಿಷ್ಟ ಆದರ್ಶ ಮಾತೃಭಾವದ ಸಾಧನೆ’ ಎಂದರು ಪರಮಹಂಸರು. ಜಗದಾದಿ ಕಾರಣವನ್ನು ‘ಅಮ್ಮಾ’ ಎಂದು ಕರೆಯುವುದು, ಎಲ್ಲರಲ್ಲೂ ಆ ಮಾತೆಯನ್ನೇ ಕಾಣುವುದು, ಈ ಅನುಷ್ಠಾನದ ಗುರಿ. ಈ ವಿಧಾನ ಅಥವಾ ಭಾವ ದೃಢವಾದರೆ ಮನುಷ್ಯ ತನ್ನ ಎಲ್ಲ ಪರಿಮಿತಿಗಳಿಂದ ಪಾರಾಗಿ ಪರಮಾರ್ಥವನ್ನು ಸುಲಭವಾಗಿ ಪಡೆಯಬಲ್ಲ ಎಂಬುದನ್ನು ಪರಮಹಂಸರು ಸಾಧಿಸಿ, ಬೋಧಿಸಿ ತೋರಿಸಿದರು.


\section*{ನರಳುತ್ತಿದೆ ಲೋಕ!}

\vskip -6.5pt\addsectiontoTOC{ನರಳುತ್ತಿದೆ ಲೋಕ!}

ಆಧುನಿಕತೆಯ ಅಟ್ಟಹಾಸಕ್ಕೆ ಪ್ರಜಾಪ್ರಭುತ್ವಕ್ರಮದ ಆಡಳಿತ ಪದ್ಧತಿಯೂ ತನ್ನ ದೇಣಿಗೆಯನ್ನು ನೀಡಿದೆ. ಪ್ರಜಾಪ್ರಭುತ್ವವೆಂದರೆ ನಿರ್ದೋಷವಾದ ಆಡಳಿತ ವಿಧಾನವೆನ್ನುವಂತಿಲ್ಲವಷ್ಟೆ. ಅದರ ತವರು ಮನೆಯಲ್ಲೇ ಅದು ಸಂಕಟಗ್ರಸ್ತವಾಗಿದೆ. ಸರಿಯಾದ ಪೌರಶಿಕ್ಷಣ, ಸಿದ್ಧತೆಗಳಿಲ್ಲದೆ ಭಾರತದಲ್ಲಿ ಅದು ಹೇಗೆ ವಿಕಟ ಅಟ್ಟಹಾಸ ಮಾಡುತ್ತಿದೆ, ಕರಾಳ ರೂಪವನ್ನು ತೋರುತ್ತಿದೆ ಎಂಬುದನ್ನು ಇಂದು ವಿಚಾರವಂತರೆಲ್ಲರೂ ತಿಳಿಯತೊಡಗಿದ್ದಾರೆ. ಎಲ್ಲರೂ ಸಮಾನರು ಎನ್ನುವ ಅದರ ಮೂಲಭೂತ ನಂಬಿಕೆ ‘ಒಬ್ಬ ಮನುಷ್ಯ, ಒಂದು ಓಟು’ ಎಂಬ ವಿಧಾನದಲ್ಲಿ ವ್ಯಕ್ತವಾಗಿದೆ. ಸರ್ವರ ಅಭ್ಯುದಯಕ್ಕೂ ಸಮಾನ ಅವಕಾಶವಿರಬೇಕಾದುದು ನ್ಯಾಯವೇ. ಆದರೆ ಕಾರ್ಯತಃ ಇದು ಯಾವ ರೂಪದಲ್ಲಿ ಈಗ ಮೈದಳೆಯುತ್ತಿದೆ? ಫ್ರೆಂಚ್ ಕ್ರಾಂತಿಯ ಘೋಷಣೆಗಳಾದ ಸ್ವಾತಂತ್ರ್ಯ, ಸಮಾನತೆ, ಸಹೋದರತೆ ಇವುಗಳ ದುರುಪಯೋಗ ಹೇಗೆ ಆಗುತ್ತಲಿದೆ ಎಂಬುದನ್ನು ಕೆಲವು ಹಿರಿಯ ಚಿಂತನಶೀಲ ವ್ಯಕ್ತಿಗಳು ಹೇಳಿದ್ದುಂಟು. ಹೆಚ್ಚಿನ ಜನ ಸ್ವಾತಂತ್ರ್ಯವೆಂದರೆ ‘ನನ್ನ ಮನ ಬಂದಂತೆ ನಾನು ವರ್ತಿಸಬಲ್ಲೆ’; ಸಮಾನತೆ ಎಂದರೆ ‘ನನಗಿಂತ ನೀನೇನೂ ಹೆಚ್ಚಿನವನಲ್ಲ’; ಸಹೋದರತೆ ಎಂದರೆ ‘ನಿನ್ನ ವಸ್ತುವೂ ನನಗೆ ಬೇಕಾದಲ್ಲಿ ನನ್ನದೇ’ ಎನ್ನುವಂಥ ಮನೋಭಾವದವರು! ಆರ್ಥಿಕಸ್ಥಿತಿ, ನ್ಯಾಯವಿಧಾನಗಳಲ್ಲಿ ಬಲವಂತವಾಗಿ ಸಮಾನತೆಯನ್ನು ತರಬಹುದು. ಆದರೆ ಮನುಷ್ಯನ ಆಂತರಿಕಶಕ್ತಿ, ಪ್ರತಿಭೆ, ವಿವೇಚನಾಸಾಮರ್ಥ್ಯ, ಸ್ಥಿರವಾದ ನೈತಿಕ ನಿಷ್ಠೆ ಇವುಗಳಲ್ಲಿ ಸಮಾನತೆ ಇದೆಯೆ? ನೇರ ಮಾತುಗಳಲ್ಲಿ ಹೇಳಬೇಕೆಂದಿದ್ದರೆ ಲೋಕದ ಹೆಚ್ಚಿನ ಜನರು ಸುಸಂಸ್ಕೃತ ವೇಷದಲ್ಲಿ, ತಮ್ಮ ಮನಶ್ಚಾಂಚಲ್ಯವನ್ನು ತಾಳಲಾರದೆ ಕಾಮ ಲೋಭ ಪರಾಯಣರಾಗಿ, ಎಲ್ಲ ತರದ ಸುಖಸಾಧನಗಳನ್ನೂ, ಇಂದ್ರಿಯಭೋಗವನ್ನೂ ತಮ್ಮದೇ ಆದ ಅಸಂಸ್ಕೃತ ಹಾಗೂ ಹೊಲಸು ಮಟ್ಟಕ್ಕೆಳೆಯುತ್ತಾರೆ. ಪತ್ರಿಕಾಕರ್ತರೂ, ಚಲ ಚ್ಚಿತ್ರ ನಿರ್ಮಾಪಕರೂ, ಬರಹಗಾರರೂ, ನಾಟಕನೃತ್ಯಗಾರರೂ, ಹೋಟೇಲು ಮಾಲಿಕರೂ ಈ ಜನರಿಂದ ಹಣ ಸೆಳೆಯಲು ಸತ್ಯ, ಧರ್ಮ, ಸಭ್ಯತೆಯ ನಿಯಮಗಳನ್ನು ಗಾಳಿಗೆ ತೂರುತ್ತಾರೆ. ಅಧಿಕಾರ ದಾಹದಿಂದ ಪ್ರೇರಿತರಾದ ರಾಜಕೀಯ ಮುಖಂಡರು ಈ ಜನರ ಓಟು ಗಳಿಸಲು ಅವರು ಹೇಳಿದಂತೆ ಕುಣಿಯಬೇಕಾಗುತ್ತದೆ. ಅಧಿಕಾರ\-ದಲ್ಲಿರು\-ವವರೂ, ವಿರೋಧಪಕ್ಷಗಳವರೂ, ತಮ್ಮ ಅಭಿಪ್ರಾಯವನ್ನು ಒಪ್ಪದವರ ಮೇಲೆ\break ಅಮಾನುಷ, ಹೇಯ ಪ್ರಚಾರ ಮಾಡಿ, ಪರಸ್ಪರ ಚಾರಿತ್ರ್ಯ ಹನನ ಮಾಡಿ, ಜನರ ವಿಶ್ವಾಸ, ಸದ್ಭಾವನೆಗಳನ್ನೇ ಹಾಳು ಮಾಡುವುದಲ್ಲದೇ, ತಾವೂ ಅವನ್ನು ಕಳೆದುಕೊಳ್ಳುತ್ತಾರೆ. ಅದರಿಂದ ವ್ಯಕ್ತಿಯ ಆತ್ಮವಿಕಾಸವನ್ನೂ, ಸಮಾಜದ ಸಮಗ್ರ ಹಿತವನ್ನೂ ಕಣ್ಮುಂದೆ ಇರಿಸಿಕೊಂಡಂಥ, ಸ್ವಾರ್ಥವನ್ನು ಸುಟ್ಟು ಬಹುಜನ ಹಿತದ ಆದರ್ಶಕ್ಕಾಗಿ ಜೀವನವನ್ನೇ ಮುಡಿಪಾಗಿಟ್ಟು ಅಹೋ ರಾತ್ರಿ ಅದಕ್ಕಾಗಿ ಜೀವನ ತೇಯುವ ಮಹಾತ್ಮರ ಮಾತು ಅರಣ್ಯ ರೋದನವಾಗುತ್ತದೆ.

ನರಕದ ಬಾಗಿಲುಗಳು ಮೂರು: ಕಾಮ, ಕ್ರೋಧ ಮತ್ತು ಲೋಭ ಎಂದು ಭಗವದ್ಗೀತೆ ಸಾರುತ್ತದೆ. ವಿಜ್ಞಾನ ಮತ್ತು ಆಧುನಿಕ ಜಡವಾದದ ವಿಚಾರಗಳು ಧರ್ಮನಾಶಕ ವಿಧಾನಗಳಿಂದ ಈ ಆಸುರೀ ಪ್ರವೃತ್ತಿಯನ್ನು ಒಂದೇ ಸಮನೆ ಹೆಚ್ಚಿಸುತ್ತಲಿವೆ. ‘ವಿಜ್ಞಾನವು ಪ್ಲುಟೋನಿಯಂನಿಂದ ಶಕ್ತಿ ಸೂರೆಗೊಳ್ಳುವ ವಿಧಾನವನ್ನು ತಿಳಿಸೀತೇ ಹೊರತು, ಮನುಷ್ಯನ ಹೃದಯದಲ್ಲಿ ತಾಂಡವವಾಡುವ ದುಷ್ಟತನವನ್ನು ಹೋಗಲಾಡಿಸಲಾರದು’\footnote{\engfoot{Science can denature Plutonium, but not the wickedness of human heart.}} ಎಂಬ ಮಹಾ ವಿಜ್ಞಾನಿ ಐನ್‍ಸ್ಟೀನ್​ರ ಮಾತು ಇಲ್ಲಿ ಗಮನಾರ್ಹ.


\section*{ಶ್ರದ್ಧೆಯೇ ಸಿದ್ಧೌಷಧ}

\addsectiontoTOC{ಶ್ರದ್ಧೆಯೇ ಸಿದ್ಧೌಷಧ}

ವಿಜ್ಞಾನದ ಬಿರುಗಾಳಿ ಹತ್ತೊಂಬತ್ತನೇ ಶತಮಾನದ ಆದಿಭಾಗದಿಂದಲೇ ಭಾರತದೆಡೆ ಬೀಸ\-ತೊಡ\-ಗಿತ್ತು. ಪಶ್ಚಿಮದ ದೇಶಗಳ ವೈಜ್ಞಾನಿಕ ಮನೋವೃತ್ತಿಯ ಜನರು ಅಲ್ಲಿನ ಧಾರ್ಮಿಕರನ್ನು ಟೀಕಿಸಿದ ಜಾಡನ್ನೇ ಹಿಡಿದು ಇಲ್ಲಿಯ ವಿದ್ಯಾವಂತರೆನಿಸಿಕೊಂಡ ಜನರೂ ಆ ಮಾತುಗಳನ್ನೇ ಉಚ್ಚರಿಸತೊಡಗಿದ್ದರು. ಭಾರತದಲ್ಲಿ ಧರ್ಮವು ನಿಜವಾದ ವೈಜ್ಞಾನಿಕ ಮನೋವೃತ್ತಿಗೆ ವಿರೋಧ\-ವಾಗಿರಲಿಲ್ಲ. ಧರ್ಮದ ನಿಯಂತ್ರಣದಲ್ಲಿದ್ದು, ಧರ್ಮವಿರೋಧವಲ್ಲದ ಸುಖಸೌಕರ್ಯಗಳನ್ನು ಇಲ್ಲಿನ ಪವಿತ್ರ ಗ್ರಂಥಗಳು ವಿರೋಧಿಸಿದ್ದಿಲ್ಲ. ಧಾರ್ಮಿಕ ಅಂತರ್ಮುಖತೆಯಿಂದ ಪರಮ ಸತ್ಯ ಸಾಕ್ಷಾತ್ಕಾರಕ್ಕಾಗಿ ನಡೆಯಿಸಿದ ಅನ್ವೇಷಣೆಗಳು ವೈಜ್ಞಾನಿಕ ವಿಧಾನದಿಂದಲೇ ಮುನ್ನಡೆದಿದ್ದವು. ಜೀವ, ಜಗತ್ತು, ಜಗನ್ನಿಯಾಮಕ–ಇವುಗಳ ಸಂಬಂಧವಾಗಿ ಈ ದೇಶದಲ್ಲಿ ನಡೆದ ಅನ್ವೇಷಣೆಗಳೂ, ಸಾಧನೆಗಳೂ, ಹೊರಜಗತ್ತಿನ ಬಗೆಗೆ ಬಿಡಿಬಿಡಿಯಾಗಿ ನಡೆಯಿಸಿದ ಅಧ್ಯಯನಗಳಾಗಿರದೆ, ಅಂತರ್ದೃಷ್ಟಿ ಅಥವಾ ಅನುಭಾವದಿಂದ ಪಡೆದ ಪೂರ್ಣದೃಷ್ಟಿಯ ತಥ್ಯಗಳಾಗಿದ್ದವು. ಮೂಢ\-ನಂಬಿಕೆಗಳನ್ನುಳಿದು ಸನಾತನಧರ್ಮದ ಸರ್ವಗ್ರಾಹಿಯಾದ ಮೂಲಭೂತ ತಥ್ಯಗಳು ವಿಜ್ಞಾನದ ಸಂದೇಹಗಳನ್ನು ಎದುರಿಸಬಲ್ಲವು ಮತ್ತು ಮಾನವರಿಗೆ ಶ್ರೇಯಸ್ಸು, ಶಾಂತಿಯೆಡೆಗೆ ದಾರಿದೀಪ\-ವಾಗ\-ಬಲ್ಲವು ಎಂದು ನಮ್ಮ ಇತಿಹಾಸದ ನವೋದಯದಲ್ಲಿ ತಮ್ಮ ಅನುಪಮ ಜೀವನ, ಅನುಭವ ಮತ್ತು ಅಸ್ಖಲಿತ ವಾಣಿಗಳಿಂದ ಸ್ವಾಮಿ ವಿವೇಕಾನಂದರು ಸ್ಪಷ್ಟಪಡಿಸಿದ್ದರು. ಯಾವ ಮತ ಪಥ ಪಂಥಗಳಿಗೂ ತೊಡಕಾಗದೆ, ವ್ಯಕ್ತಿಯ ಸರ್ವಾಂಗೀಣ ಪ್ರಗತಿಗೂ ಸಮಾಜದ ಕಲ್ಯಾಣಕ್ಕೂ ದೇಶದ ಅಭ್ಯುದಯಕ್ಕೂ ಕಾರಣವಾಗಬಹುದಾದ, ಪ್ರತಿಯೊಬ್ಬನಲ್ಲೂ ಅಡಗಿರುವ ದಿವ್ಯತೆ, ಸಮಗ್ರ ವಿಶ್ವದ ಆಧ್ಯಾತ್ಮಿಕ ಏಕತೆ, ಧರ್ಮಸಮನ್ವಯ ದೃಷ್ಟಿ ಹಾಗೂ ಜೀವಶಿವ ಸೇವೆ–ಎಂಬ ತತ್ತ್ವಗಳನ್ನು ಅವರು ಪ್ರತಿಪಾದಿಸಿದರು. ಸ್ವಾತಂತ್ರ್ಯಪೂರ್ವದಲ್ಲಿ ಬಡವರ ಉದ್ಧಾರವಾಗಬೇಕೆಂಬ, ರಾಷ್ಟ್ರಕ್ಕಾಗಿ ದುಡಿದು ಮಡಿಯಬೇಕೆಂಬ, ಮಹಾಪ್ರೇರಣೆಯನ್ನು ವಿದ್ಯಾವಂತರಲ್ಲಿ ಮೂಡಿಸಿದ ದೇಶಪ್ರೇಮಿ, ಧಾರ್ಮಿಕ ಮುಖಂಡರು ಅವರಾಗಿದ್ದರು.\footnote{\engfoot{It is now a known fact that hundreds of young revolutionaries of Bengal were inspired by the message of Swami Vivekananda and cheerfully embraced sufferings and death with `Vandemataram' on their lips and Vivekananda's teaching in their heart.}

~\hfill\engfoot{ –Dr. R. C. Majumdar, \textit{Three Faces of India's Struggle for Freedom}}} ಈ ಕಾರ್ಯಕ್ಕೆ ಅವರಿಗೆ ನೆರವಾದುದು ಧರ್ಮಮೂಲದ ನಿಶ್ಚಯಜ್ಞಾನವೇ ಆಗಿತ್ತು–ಎಂಬುದನ್ನು ಯಾರೂ ಮರೆಯತಕ್ಕದ್ದಲ್ಲ. ಅನಂತರ ಧಾರ್ಮಿಕ ಹಿನ್ನೆಲೆಯಿಂದ ಪರಿಷ್ಕೃತವಾದ ಜೀವನ, ದೇಶಕ್ಕೆ ಎಂಥ ಸ್ವಾರ್ಥ ತ್ಯಾಗದ ಸ್ಫೂರ್ತಿಯನ್ನು ನೀಡಬಲ್ಲುದೆಂಬುದಕ್ಕೆ ಮಹಾತ್ಮಾಗಾಂಧೀಜಿ ಉದಾಹರಣೆಯಾದರು.

ಈ ಶತಮಾನದ ಮೊದಲಿಗೆ ಪಶ್ಚಿಮದ ವೈಜ್ಞಾನಿಕ ಪ್ರಗತಿಯಿಂದ ಮುಗ್ಧರಾಗಿ ಪಶ್ಚಿಮವನ್ನು ಅನುಕರಿಸ ಹೊರಟವರಿಗೆ ಸ್ವಾಮಿ ವಿವೇಕಾನಂದರು ಈ ಎಚ್ಚರಿಕೆಯ ಮಾತನ್ನು ಹೇಳಿದರು: ‘ಸಾಮಾಜಿಕವಾಗಿ, ಆರ್ಥಿಕವಾಗಿ ಹಿಂದುಳಿದ ನಮ್ಮ ಜನರು ಸ್ವಲ್ಪಮಟ್ಟಿನ ಸುಖಸೌಕರ್ಯಗಳನ್ನು ಪಡೆಯುವಂತಾಗಬೇಕು. ಉತ್ತಮ ಜೀವನವನ್ನು ನಡೆಯಿಸಿದ ಮೇಲೆ ತ್ಯಾಗವು ಸ್ವಾಭಾವಿಕವಾಗಿ ಬರುತ್ತದೆ. ಪ್ರಾಯಃ ಈ ವಿಚಾರಗಳಲ್ಲಿ ಕೆಲವಂಶಗಳನ್ನು ನಾವು ಅವರಿಂದ ಕಲಿಯಬಹುದು. ಆದರೆ ನಾವು ಅತ್ಯಂತ ಜಾಗರೂಕರಾಗಿರಬೇಕು. ಪಶ್ಚಿಮದ ಭಾವನೆಗಳನ್ನು ಅರ್ಥ ಮಾಡಿಕೊಂಡಿದ್ದೇವೆನ್ನುವ ಹೆಚ್ಚಿನ ಜನರು ನಮ್ಮ ಸಮಾಜಕ್ಕೆ ಒಳಿತನ್ನು ಮಾಡುವುದರಲ್ಲಿ ಅಸಮರ್ಥರಾದ ಉದಾಹರಣೆಗಳೇ ಹೆಚ್ಚು ಎಂಬುದನ್ನು ದುಃಖದಿಂದ ಹೇಳಬೇಕಾಗಿದೆ. ವಿಜ್ಞಾನದ ವಿಚಾರಗಳು, ಸಾಮಾನ್ಯ ವಸ್ತುಗಳಿಂದ ಹೆಚ್ಚಿನ ಉಪಯೋಗವನ್ನು ಪಡೆಯುವ ವಿಧಾನಗಳು, ಸಂಘಟಿತರಾಗಿ ದುಡಿಯುವ ಕಲೆ–ಇವುಗಳನ್ನು ನಾವು ಅವರಿಂದ ಕಲಿಯಬಹುದು. ಆದರೆ ಭಾರತದಲ್ಲಿ ಯಾರಾದರೂ ತಿಂದು, ಕುಡಿದು, ಕುಣಿಯವುದೆ–ಅಂದರೆ, ಇಂದ್ರಿಯ ಸಂತೃಪ್ತಿಯೇ ಆದರ್ಶವೆಂದರೆ ಆತನು ಸುಳ್ಳುಗಾರ, ಪಶ್ಚಿಮದ ವಿಜ್ಞಾನ ಮತ್ತು ನಾಗರಿಕತೆಯ ಥಳಕು ಬೆಳಕುಗಳು ಕಣ್ಣು ಕೋರೈಸುವಂತಿದ್ದರೂ ನಾನು ಸ್ಪಷ್ಟವಾಗಿ ಸಾರುತ್ತೇನೆ–ಅದು ವ್ಯರ್ಥವೆಂದು. ಅಧ್ಯಾತ್ಮವನ್ನು ತೊರೆಯಬೇಡಿ. ಅದೊಂದೇ ಉಳಿಯುವುದು. ಇತರ ವಿಚಾರಗಳು ಬೇಡವೆಂದಲ್ಲ. ರಾಜಕೀಯ, ಸಾಮಾಜಿಕ ವಿಚಾರಗಳು ಬೇಡವೆಂದು ಈ ಮಾತಿನ ಅರ್ಥವಲ್ಲ. ಪ್ರಥಮ ಸ್ಥಾನ ಅವುಗಳಿಗಲ್ಲ. ಅವು ಜೀವನದ ಶ್ರದ್ಧಾಕೇಂದ್ರಗಳಲ್ಲ. ಧರ್ಮವು ಭಾರತೀಯರ ಸರ್ವಸ್ವ. ಅದು ನಾಶವಾದಾಗ ಭರತಖಂಡ ನಾಶವಾಗುವುದು. ಪ್ರತಿಯೊಬ್ಬರ ತಲೆಯ ಮೇಲೂ ಕುಬೇರನ ಭಂಡಾರದಿಂದ ಹಣವನ್ನು ಸುರಿದರೂ, ಎಷ್ಟೇ ಸಾಮಾಜಿಕ ಸುಧಾರಣೆಗಳನ್ನು ಕೈಗೊಂಡರೂ, ಎಂಥ ರಾಜಕೀಯ ಕ್ರಾಂತಿಯನ್ನು ಮಾಡಿದರೂ ಧರ್ಮವಿಲ್ಲದೆ ಭಾರತ ಉಳಿಯದು.’

ಧರ್ಮದ ಆದರ್ಶವನ್ನು ಧ್ವನಿಸುವವರು ಭಾರತೀಯ ಧಾರ್ಮಿಕ ಮುಖಂಡರು ಮಾತ್ರ ಎಂಬುದು ತಪ್ಪು ತಿಳಿವಳಿಕೆ. ಸಮಗ್ರ ಜಗತ್ತಿನ ಇತಿಹಾಸದ ಮಾರ್ಮಿಕ ವಿಶ್ಲೇಷಣೆಯನ್ನು ಮಾಡಿದ ಆರ್ನಾಲ್ಡ್ ಟೊಯ್ನಬೀ ಇದೇ ಮಾತನ್ನು ಹೇಳಿದರು: ‘ರಾಜಕೀಯ ಮತ್ತು ಆರ್ಥಿಕ ವಿಷಯಗಳಲ್ಲಿ ಆಸಕ್ತಿ ಮಿತಿಮೀರಿ ಬೆಳೆದು, ಉಳಿದೆಲ್ಲ ಜೀವನಾದರ್ಶಗಳನ್ನೂ ಅವುಗಳಿಗೆ ಅಧೀನಗೊಳಿಸಿದುದೇ ಎಲ್ಲ ನಾಗರಿಕತೆಗಳ ಅವನತಿಗೆ ಕಾರಣವಾಯಿತು. ಮತಾಂಧತೆಯನ್ನು ತೊಡೆದುಹಾಕುವ ಪ್ರಯತ್ನದಲ್ಲಿ ಧರ್ಮಶ್ರದ್ಧೆಯನ್ನೇ ನಂದಿಸಿರುವ ಈ ಮನಃಪ್ರವೃತ್ತಿಯು ಹದಿನೇಳನೆ\break ಶತಮಾನದಿಂದ ಪ್ರಾರಂಭಗೊಂಡು, ಇಪ್ಪತ್ತನೇ ಶತಮಾನದಲ್ಲಿ ಬೇರುಬಿಟ್ಟು, ಇಂದು ಪಾಶ್ಚಾತ್ಯ ಮಹಾ ಸಮಾಜದ ಎಲ್ಲ ಭಾಗಗಳಲ್ಲೂ ದಟ್ಟವಾಗಿ ವ್ಯಾಪಿಸಿಬಿಟ್ಟಿದೆ. ಈಗೀಗ ಅದರ ಅಪಾಯದ ಬಗೆಗೆ ಅರಿವು ಮೂಡುತ್ತಿದೆ. ಪಾಶ್ಚಾತ್ಯ ಸಮಾಜದ ಆಧ್ಯಾತ್ಮಿಕ ಆರೋಗ್ಯಕ್ಕೆ ಮಾತ್ರವಲ್ಲ, ಅದರ ಭೌತಿಕ ಅಸ್ತಿತ್ವಕ್ಕೂ ಕೂಡ ಬಂದಿರುವ, ಎಲ್ಲಕ್ಕಿಂತ ಮಿಗಿಲಾದ ಅಪಾಯವೇ ಅದು. ಅಷ್ಟೇಕೆ, ಅತಿಯಾಗಿ ಪ್ರಚಾರವಾಗಿರುವ ಯಾವುದೇ ಭಯಾನಕ ರಾಜಕೀಯ, ಆರ್ಥಿಕ ರೋಗಕ್ಕಿಂತಲೂ ಮರಣಾಂತಿಕವಾದ ಆಪತ್ತು ಧರ್ಮಶ್ರದ್ಧೆಯ ಅಭಾವ ಎಂಬುದನ್ನು ಅನೇಕರು ಮನಗಾಣುತ್ತಿದ್ದಾರೆ.’


\section*{ವಿಜ್ಞಾನಿಗಳೂ ಹೇಳುತ್ತಿದ್ದಾರೆ!}

\addsectiontoTOC{ವಿಜ್ಞಾನಿಗಳೂ ಹೇಳುತ್ತಿದ್ದಾರೆ!}

ನೈತಿಕ, ಧಾರ್ಮಿಕ ಮೌಲ್ಯಗಳಿಗೆ ಸ್ಥಾನವಿಲ್ಲದಿದ್ದರೆ ಸಮಾಜದ ಅಧಃಪತನ ಖಂಡಿತ ಎಂಬುದನ್ನು ವಿಜ್ಞಾನಿಗಳೂ ಹೇಳುತ್ತಿದ್ದಾರೆ.

‘ವಿಜ್ಞಾನದ ಪ್ರತಿಯೊಂದು ಶೋಧನೆಯೂ ಮಾನವರಿಗೆ ಒಂದು ದೊಡ್ಡ ದೌರ್ಭಾಗ್ಯವಾಗಿ ಪರಿಣಮಿಸುತ್ತಿದೆ. ತಾಂತ್ರಿಕ ಪರಿಣತಿಯ ಜೊತೆಜೊತೆಗೆ ಪ್ರಜ್ಞಾನ ಎಂದರೆ ಪಡೆದ ಜ್ಞಾನ ಹಾಗೂ ಪರಿಣತಿಯ ಸದುಪಯೋಗ ಮಾಡುವ ಅರಿವು ಮೂಡದಿದ್ದರೆ ದುಃಖವೇ ಹೆಚ್ಚುತ್ತ ಹೋಗುವುದು’–ಎಂದರು ತತ್ತ್ವ ಜ್ಞಾನಿ ಬರ್ಟ್ರಾಂಡ್ ರಸ್ಸೆಲ್.

‘ನವನಾಗರಿಕತೆಯು ನಮಗೆ ಹಿಡಿಸದ್ದು. ಅದು ನಮ್ಮ, ಎಂದರೆ ಮಾನವನ ನೈಜಸ್ವಭಾವದ ಪರಿಜ್ಞಾನವಿಲ್ಲದೆ ಕಟ್ಟಿದ ಕಟ್ಟಡ. ಅದಕ್ಕೊಂದು ನಿರ್ದಿಷ್ಟ ಯೋಜನೆ ಅಥವಾ ಗುರಿ ಇಲ್ಲ. ಮಾನವರ ಸರ್ವತೋಮುಖ ಅಭ್ಯುದಯವಾಗಬೇಕೆಂಬ ಹಂಬಲದಿಂದೇನೂ ಅದು ಕೆಲಸ ಮಾಡುತ್ತಿಲ್ಲ. ಏನು ಪರಿಣಾಮವಾದೀತೆಂಬುದನ್ನು ಮೊದಲೇ ತಿಳಿದುಕೊಳ್ಳದೇ, ವೈಜ್ಞಾನಿಕ ಸಂಶೋಧನೆಗಳನ್ನು ಮಾಡುತ್ತ ಹೋಗುವುದರಿಂದೇನು ಲಾಭ? ವಿಜ್ಞಾನದ ಅಪಾರ ಸಂಪತ್ತಿನಿಂದ ನಾವು ಆರಿಸಿಕೊಳ್ಳುವ ಭಾಗವು ಮಾನವಕುಲದ ಉನ್ನತಿಯ ದೃಷ್ಟಿಯನ್ನಿಟ್ಟುಕೊಂಡು ಪಡೆದುದಲ್ಲ. ಯಾವುದು ನಮಗೆ ಹಿಡಿಸಿತೋ, ಅನುಕೂಲವಾಯಿತೋ ಅದನ್ನು ಪಡೆದೆವು. ಅವುಗಳಿಂದ ಮಾನವನ ಮೇಲಾಗುವ ಪರಿಣಾಮಗಳನ್ನು ಗಣನೆಗೆ ತಂದುಕೊಂಡದ್ದೇ ಇಲ್ಲ... ಈ ವಿಜ್ಞಾನದ ನಾಗರಿಕತೆ ಮಾನವನಿಗೆ ವಿರಾಮವನ್ನು ತಂದುಕೊಟ್ಟು, ಅಪಾರ ದೌರ್ಭಾಗ್ಯವನ್ನುಂಟುಮಾಡಿದೆ. ಮನೋದುರ್ಬಲತೆ, ಹುಚ್ಚು, ಉನ್ಮಾದಗಳೆಲ್ಲ ಪ್ರಾಯಃ, ನಮ್ಮ ತಂತ್ರೋದ್ಯಮ ನಾಗರಿಕತೆಯ ಫಲವಾಗಿ ನಾವು ತೆರಬೇಕಾದ ಬೆಲೆಯಾಗಿದೆ’ ಎಂದರು ಅಲೆಕ್ಸಿಸ್ ಕೆರಲ್.

‘ಯಾವ ನಾಗರಿಕತೆಯೇ ಆಗಲಿ, ಯಾಂತ್ರಿಕ ಅಭಿವೃದ್ಧಿ ಹಾಗೂ ತಾಂತ್ರಿಕ ಉಪಾಯಗಳನ್ನೇ ಹೊಂದಿಕೊಂಡಿದ್ದರೆ, ಅದರ ನಾಶ ಖಂಡಿತ. ಮಾನವ ಚರಿತ್ರೆಯಲ್ಲಿ ಮೊದಲ ಬಾರಿಗೆ ಕೇವಲ ಬುದ್ಧಿಶಕ್ತಿಗೂ, ನೈತಿಕ ಮೌಲ್ಯಗಳಿಗೂ, ಯಾವುದು ಬಾಳೀತು, ಯಾವುದು ನಾಶವಾದೀತು ಎಂಬಂಥ ಸಂಘರ್ಷ ಉಂಟಾಗಿದೆ’ ಎಂದರು ಲೇಕೋಮ್ ಡಿ ನೊಯ್.

‘ವಿಜ್ಞಾನ ಮತ್ತು ಅದರ ಪ್ರಯೋಗಶಾಸ್ತ್ರದ ಫಲಗಳೆಲ್ಲ ದುರುಪಯೋಗವಾಗುತ್ತ ಬಂದಿವೆ. ಅಣುಬಾಂಬಿನ ಸರ್ವನಾಶಕ ಶಕ್ತಿಯ ಭೀತಿಯೇ ಇನ್ನು ಮುಂದಿನ ಯುದ್ಧವನ್ನು ತಡೆದೀತೆಂದು ನಂಬುವುದು ಪಿಶಾಚಿಗೆ ಪೈಶಾಚಿಕ ಗುಣ ಹೆಚ್ಚಿದಷ್ಟೂ ಅದು ದೈವೀಗುಣ ಹೊಂದೀತೆಂಬ ಕಟ್ಟುಕಥೆಯಂತೆ ಮಿಥ್ಯ. ಮಾನವನ ಆರ್ಥಿಕ ಸ್ಥಿತಿಯ ಅಭಿವೃದ್ಧಿ, ಅವನ ಶೀಲದ ಬೆಳವಣಿಗೆಯನ್ನು ಅದೇ ಪ್ರಮಾಣದಲ್ಲಿ ಬೆಳೆಯಿಸುತ್ತೆಂಬುದು ಮಿಥ್ಯೆ ಎಂದು ಸಾಬೀತಾಗಿದೆ’ ಎಂದರು ಸೊರೊಕಿನ್.

ಇಂದು ಮನೋವಿಜ್ಞಾನಿಗಳಲ್ಲಿ ಕೆಲವು ಪ್ರಮುಖರು ಧರ್ಮ ಅಥವಾ ಧಾರ್ಮಿಕ ಆದರ್ಶ ಮನುಷ್ಯನ ಅಂತರಂಗದ ಜೀವನಕ್ಕೆ ಆವಶ್ಯಕ, ಅವನ ಮಾನಸಿಕ ಸಮತೋಲ ಹಾಗೂ ಆರೋಗ್ಯಕ್ಕೆ ಅಗತ್ಯ ಎನ್ನುತ್ತಿದ್ದಾರೆ.\footnote{\engfoot{The modern experiment to live without religion has failed and once we have understood this, we know what our post-modern tasks really are.}\hfill\engfoot{–E. F. Schu- macher, \textit{A Guide for the Perplexed}}}

ಮನೋರೋಗಿಗಳನ್ನು ಪರಿಶೀಲಿಸಿ, ಅಧ್ಯಯನಮಾಡಿ, ಮಾರ್ಗದರ್ಶನ ಮಾಡಿದ ಸಿ.\ ಜಿ.\ ಯೂಂಗ್ ಹೇಳಿದರು ‘ಮೂವತ್ತೈದು ವರ್ಷ ದಾಟಿದ ನನ್ನ ರೋಗಿಗಳೆಲ್ಲರೂ ಯಾವುದೇ ಧಾರ್ಮಿಕ ದೃಷ್ಟಿಕೋನವನ್ನು ಬೆಳೆಸಿಕೊಳ್ಳದವರೇ ಆಗಿದ್ದರು. ಒಂದಲ್ಲ ಒಂದು ತೆರನಾದ ಧಾರ್ಮಿಕ ದೃಷ್ಟಿಕೋನವನ್ನು ಪಡೆದಲ್ಲಿ ಅವರು ನಿರೋಗಿಗಳಾಗಬೇಕಿತ್ತು–ಇದಕ್ಕೆ ಒಬ್ಬನೂ\break ಅಪವಾದವಿರಲಿಲ್ಲ. ಬದುಕಿನಲ್ಲಿ ಎದುರಿಸಬೇಕಾದ ದುಃಖ ದುಮ್ಮಾನಗಳನ್ನೂ, ತಾಮಸಿಕ ಶಕ್ತಿಗಳನ್ನೂ, ಮನುಷ್ಯ ಏಕಾಕಿಯಾಗಿ ಎದುರಿಸಲು ಇದುವರೆಗೂ ಸಾಧ್ಯವಾಗಿಲ್ಲ ಎಂಬುದನ್ನು ಫ್ರಾಯ್ಡ್ ದುರದೃಷ್ಟವಶದಿಂದ ಗಮನಿಸಲೇ ಇಲ್ಲ. ಧರ್ಮ ನೀಡುವ ಸಹಾಯ ಮಾನವನಿಗೆ ಎಂದೆಂದೂ ಆವಶ್ಯಕವೇ. ಅದು ಅವನನ್ನು ಸಂಕಟದಿಂದ ಮೇಲಕ್ಕೆತ್ತಬಲ್ಲದು.’


\section*{ಮುಖಂಡರ ಮನೋವೃತ್ತಿ}

\addsectiontoTOC{ಮುಖಂಡರ ಮನೋವೃತ್ತಿ}

ವಿಸ್ತಾರವಾಗಿ ಈ ಮಹನೀಯರುಗಳ ವಾಕ್ಯಗಳನ್ನು ಇಲ್ಲಿ ಕೊಟ್ಟಿರುವುದರ ಆವಶ್ಯಕತೆಯನ್ನು ಕೆಲವರು ಪ್ರಶ್ನಿಸಬಹುದು. ಧಾರ್ಮಿಕ ಆದರ್ಶದಲ್ಲಿ ಆಗಲೇ ದೃಢಶ್ರದ್ಧೆಯನ್ನು ಬೆಳೆಸಿಕೊಂಡವರಿಗೆ ಇದರ ಅಗತ್ಯವಿಲ್ಲ ನಿಜ. ಇತ್ತ ಧಾರ್ಮಿಕ ನಂಬಿಕೆ ಇಲ್ಲ, ಆತ್ತ ವೈಜ್ಞಾನಿಕ ಮನೋವೃತ್ತಿ ಎಂದುಕೊಂಡು ಧರ್ಮನಿಂದೆ ಮಾಡುವ ಆಧುನಿಕ ವಿದ್ಯಾಸಂಪನ್ನರಿಗೆ, ಈ ವಿಚಾರ ಮನವರಿಕೆಯಾಗ\-ಬೇಕಾಗಿದೆ. ಇಂದು ವಿದ್ಯಾವಂತರೆನ್ನಿಸಿಕೊಂಡವರು ಚರಿತ್ರೆಯ ಭೌತಿಕ ವ್ಯಾಖ್ಯಾನವನ್ನು ಅಂಗೀ\-ಕರಿಸಿ\-ದ್ದಾರೆ. ರಾಷ್ಟ್ರದ ರಾಜಕೀಯ ಚಟುವಟಿಕೆಗಳೆಲ್ಲ ಪ್ರತ್ಯಕ್ಷವಾಗಿ, ಪರೋಕ್ಷವಾಗಿ, ಭೌತಿಕ ದೃಷ್ಟಿಕೋನಗಳಿಂದಲೇ ಪ್ರಭಾವಿತವಾಗಿವೆ. ಸ್ವಾತಂತ್ರ್ಯಪೂರ್ವದಲ್ಲಿ ತಮ್ಮ ಧರ್ಮ ಸಂಸ್ಕೃತಿಯನ್ನು ರಕ್ಷಿಸಿಕೊಳ್ಳಬೇಕು, ಉಳಿಸಿಕೊಳ್ಳಬೇಕು ಎಂಬ ಹಂಬಲದಿಂದ ನಮ್ಮ ಧುರೀಣರೂ, ವಿದ್ಯಾವಂತ ರಾಷ್ಟ್ರಪ್ರೇಮಿಗಳೂ, ತಮ್ಮ ಪ್ರಾಣ ಪಣವಾಗಿಟ್ಟು ಹೋರಾಡಿದರು. ಆದರೆ ಸ್ವಾತಂತ್ರ್ಯ ಬಂದ ಮೇಲೆ ಆದದ್ದೇನು? ಹಿಂದುಗಳ ಪ್ರತಿನಿಧಿಗಳೂ, ಮುಖಂಡರೂ, ಸ್ವಾತಂತ್ರ್ಯ ಹೋರಾಟದಲ್ಲಿ ಭಾಗವಹಿಸಿದವರಲ್ಲಿ ಪ್ರಮುಖರೂ ನಮ್ಮ ಪರಂಪರೆಯಲ್ಲಿ ಗೌರವವಿದ್ದರೂ, ದೇವರು ಧರ್ಮ ಅಧ್ಯಾತ್ಮಗಳಲ್ಲಿ, ಜನರ ದುರದೃಷ್ಟದಿಂದ, ವಿಶ್ವಾಸವಿಲ್ಲದಿದ್ದವರು ಎಂಬುದನ್ನು ದುಃಖದಿಂದ ಹೇಳಬೇಕಾಗಿದೆ. ಹಿರಿಯ ಲೇಖಕರಾದ ಮಾಸ್ತಿ ವೆಂಕಟೇಶ ಅಯ್ಯಂಗಾರರು ರಾಷ್ಟ್ರದ ಭಾಗ್ಯಸೂತ್ರವನ್ನು ಹಿಡಿದ ಅಂದಿನ ಹಿರಿಯರ ಮನೋಭಾವವನ್ನು ಅತ್ಯಂತ ನಿರ್ಲಿಪ್ತರಾಗಿ, ನಿಷ್ಪಕ್ಷಪಾತ ಭಾವನೆಯಿಂದ, ಆದರೆ, ಸ್ಪಷ್ಟವಾಗಿ ಹೀಗೆ ನಿರೂಪಣೆ ಮಾಡಿದ್ದಾರೆ:

‘ನಾವು ಸ್ವಾತಂತ್ರ್ಯವನ್ನು ಬಯಸಿದ್ದು ಏಕೆ? ಭರತವರ್ಷ ತನ್ನದು ಎಂಬ ಒಂದು ಮನೋ ಧರ್ಮವನ್ನು, ಸಂಸ್ಕೃತಿಯನ್ನು ಪಡೆದಿದೆ. ಪರಾಧೀನವಾಗಿರುವ ರಾಷ್ಟ್ರ ತನ್ನತನವನ್ನು ಬೆಳೆಸಿಕೊಳ್ಳಲಾರದು. ನಮ್ಮ ರಾಷ್ಟ್ರದ ಎಲ್ಲ ಮುಖದ ವಿಕಾಸ ಸಾಧ್ಯವಾಗಬೇಕೆಂದರೆ ನಮ್ಮರಾಷ್ಟ್ರ ತನ್ನ ಜೀವನವನ್ನು ತಾನೇ ರೂಪಿಸಿಕೊಳ್ಳಬೇಕು. ನಮ್ಮ ನಾಯಕರು ರಾಜಕೀಯ ಸ್ವಾತಂತ್ರ್ಯವನ್ನು ಅಪೇಕ್ಷಿಸಿದ್ದೇ ಇದಕ್ಕಾಗಿ. ಸ್ವಾತಂತ್ರ್ಯ ಲಭಿಸಿ ಈಗ 35 ವರ್ಷ ಆಗಿದೆ.\footnote{ ಇದು ಕೆಲವು ವರ್ಷಗಳ ಹಿಂದೆ ಪ್ರಕಟವಾದ ಲೇಖನ ಎಂಬುದನ್ನು ಗಮನಿಸಬೇಕು.} ರಾಜಕೀಯವಾಗಿ ಸ್ವತಂತ್ರವಾದ ರಾಷ್ಟ್ರ, ಸಂಸ್ಕೃತಿಯ ದೃಷ್ಟಿಯಿಂದಲೂ ಸ್ವಾತಂತ್ರ್ಯವನ್ನು ವರಿಸಿದ್ದರೆ ಎಷ್ಟೋ ಶುಭವನ್ನು ಸಾಧಿಸಿರಬಹುದಾಗಿತ್ತು. ಆದರೆ ನಮ್ಮ ದುರದೃಷ್ಟ ಇದಕ್ಕೆ ಅಡ್ಡಿಯಾಯಿತು. ರಾಷ್ಟ್ರಕ್ಕೆ ಸ್ವಾತಂತ್ರ್ಯವನ್ನು ತಂದ ಕಾಂಗ್ರೆಸ್ ಒಂದು ರಾಜಕೀಯ ಪಕ್ಷ ಆಗಿ ರಾಷ್ಟ್ರವನ್ನು ಆಳುವುದು ಬೇಡ ಎಂದು ಗಾಂಧೀಜಿ ಸೂಚಿಸಿದರು. ನಮ್ಮ ನಾಯಕರು ಈ ಮಾತನ್ನು ಕೇಳಲಿಲ್ಲ. ಕಾಂಗ್ರೆಸ್ ಅಧಿಕಾರವನ್ನು ವಹಿಸಿಕೊಂಡಿತು. ಪಂಡಿತ ಜವಾಹರಲಾಲ್ ನೆಹರೂ ಪ್ರಧಾನ ಮಂತ್ರಿ ಆದರು. ಜವಾಹರಲಾಲ್ ನೆಹರೂ ಅತ್ಯಂತ ಪುಣ್ಯವಂತ ಮನೆತನ ಒಂದರಲ್ಲಿ ಹುಟ್ಟಿದವರು. ಸ್ವಭಾವತಃ ಧ್ಯೇಯವಾದಿ! ಇಂಗ್ಲೆಂಡಿನಲ್ಲಿ ವಿದ್ಯೆ ಪಡೆದವರು; ಅಲ್ಲಿಯ ಉದಾರವಾದಿ ರಾಜಕೀಯ ಚಿಂತಕರ ವಾದಕ್ಕೆ ಮನಸೋತವರು; ಪ್ರಾಯಶಃ ಕಮ್ಯುನಿಸ್ಟ್ ಎಂದೇ ಹೇಳಬಹುದಾದ ವ್ಯವಸ್ಥೆಯನ್ನು ಒಪ್ಪಿಕೊಂಡ ತರುಣನಾಗಿ ನಮ್ಮ ರಾಷ್ಟ್ರಕ್ಕೆ ಹಿಂದಿರುಗಿದವರು. ನಮ್ಮ ಜನಾಂಗದ ಜೀವನದಲ್ಲಿರುವ ಕುಂದುಗಳನ್ನು ನೋಡಿ ಇವರ ಮನ ರೋಸಿತು. ಇವರು ಹುಟ್ಟಿನಿಂದ ಹಿಂದೂ, ಆದರೆ ಇವರಿಗೆ ಹಿಂದೂ ಧರ್ಮದಲ್ಲಿ ನಂಬಿಕೆ ಇಲ್ಲ. ಕಾಂಗ್ರೆಸ್ ನಾಯಕರಾಗಿ ಇವರು ಮಹಾತ್ಮರ ಬಲಗೈಯಾಗಿ ದುಡಿದರು. ಹೀಗಿರುತ್ತ ಮಹಾತ್ಮರ ಜೀವನದ ಪ್ರೇರಕ ಭಾವಗಳಿಗೆ ಇವರ ಮನಸ್ಸಿನಲ್ಲಿ ಎಡೆ ಇರಲಿಲ್ಲ. ಮಹಾತ್ಮರು “ಲೋಕವನ್ನು ನಡೆಸುತ್ತಿರುವ ದಿವ್ಯಶಕ್ತಿ ಒಂದು ಇದೆ, ಅದು ದೇವರು” ಎನ್ನುತ್ತಿದ್ದರು. ಜವಾಹರಲಾಲರಿಗೆ ದೇವರು ಆವಶ್ಯಕವಾಗಿರಲಿಲ್ಲ. ಮಹಾತ್ಮರು “ನಾನು ಹಿಂದೂ, ಹಿಂದೂ ಧರ್ಮ ನನಗೆ ಬೇಕಾಗಿರುವುದಕ್ಕೆ ಕಾರಣ ಅದು ಬೇರೆ ಯಾವ ಧರ್ಮವನ್ನೂ ಬೇಡ ಎನ್ನುವುದಿಲ್ಲ, ಎಲ್ಲಧರ್ಮಗಳ ಸಾರವನ್ನೂ ಒಳಗೊಂಡಿದೆ” ಎಂದು ಹೇಳಿದರು. ಜವಾಹರಲಾಲರಿಗೆ ಈ ಹಿಂದೂ ಧರ್ಮದ ಪರಿಚಯ ಇರಲಿಲ್ಲ. ಮಹಾತ್ಮರು ದಿನ ತಪ್ಪದೆ\break ಸಂಜೆಯಲ್ಲಿ ಸಭೆ ಸೇರಿಸಿ ಪ್ರಾರ್ಥನೆ ನಡೆಸುತ್ತಿದ್ದರು. ಜವಾಹರಲಾಲರು ಪ್ರಾರ್ಥನೆಯಿಂದ ಏನಾದರೂ ಪ್ರಯೋಜನ ಇದೆಯೆಂದು ತಿಳಿದಿದ್ದರೋ, ಇಲ್ಲವೋ, ಹೇಳುವಂತಿಲ್ಲ. ಮಹಾತ್ಮರು ಪ್ರಜೆಗಳ ಮನಸ್ಸನ್ನು ಅರಿತು ಅವರ ಕ್ಷೇಮಕ್ಕಾಗಿ ನಡೆಯುವ ರಾಜ್ಯವ್ಯವಸ್ಥೆಯನ್ನು “ರಾಮರಾಜ್ಯ” ಎಂದು ಕರೆದರು. ಒಮ್ಮೆ ಈ ಮಾತು ಬಂದಾಗ, ಜವಾಹರಲಾಲರು “ರಾಮರಾಜ್ಯ ಎಂದರೆ ಏನು? ನನಗೇನೂ ತಿಳಿಯದು” ಎಂದರು. ಪಾಶ್ಚಾತ್ಯ ವಿಚಾರವಂತರು ಬೌದ್ಧಮತವನ್ನು ಬುದ್ಧಿ ಒಪ್ಪುವ ಏಕೈಕಮತ ಎಂದು ಹೊಗಳಿರುವುದುಂಟು. ಜವಾಹರಲಾಲ್ ತುಂಬ ಬುದ್ಧಿವಂತರು. ಈ ಮಾತು ಅವರಿಗೆ ಯುಕ್ತವೆಂದು ಕಂಡಿರಬೇಕು. ಅವರು ಭರತವರ್ಷದ ಚರಿತ್ರೆ ಬುದ್ಧಗುರುವಿನಿಂದ ಆರಂಭ\-ವಾಯಿತು ಎಂಬಂತೆ ನಡೆದುಕೊಂಡರು. ಅದರಿಂದಲೇ ನಮ್ಮ ರಾಷ್ಟ್ರ ಸಾರನಾಥದ ಧ್ವಜಸ್ತಂಭವನ್ನು ರಾಷ್ಟ್ರದ ಲಾಂಛನವಾಗಿ ಇಟ್ಟುಕೊಂಡಿದೆ. ರಾಷ್ಟ್ರಧ್ವಜದಲ್ಲಿ ಅಶೋಕಚಕ್ರ ಕೇಂದ್ರವಾಗಿದೆ. ನಮ್ಮ ರಾಷ್ಟ್ರರಚನೆಯ ಸಭೆ ರಾಷ್ಟ್ರಕ್ಕೆ ಸ್ವಾತಂತ್ರ್ಯವನ್ನು ಅನುಗ್ರಹಿಸಿದ್ದಕ್ಕೆ ದೇವರಿಗೆ ವಂದನೆ ಸಲ್ಲಿಸಬೇಕೆಂದು ಯೋಚಿಸಿದಾಗ, ಜವಾಹರಲಾಲರು ಇದನ್ನು ಒಪ್ಪದ ಕಾರಣ ಆ ಯೋಚನೆಯನ್ನು ಕೈಬಿಟ್ಟಿತು. “ದೇವರು ಇದ್ದಾನೆ” ಎಂದು ನಂಬಿ, “ರಾಮ ದೇವರು” ಎಂದು ತಿಳಿದು, ಆ ನಂಬಿಕೆಯಲ್ಲಿ, ರಾಷ್ಟ್ರದ ಸ್ವಾತಂತ್ರ್ಯದ ಸಾಧನೆಯಲ್ಲಿ ಬಾಳನ್ನು ಸವೆಯಿಸಿ, ಸ್ವಾತಂತ್ರ್ಯವನ್ನು ನಮಗೆ ಕೊಡಿಸಿದ ಕಾರಣ, “ರಾಷ್ಟ್ರಪಿತ” ಎಂಬ ಹೆಸರನ್ನು ಜನತೆಯಿಂದ ಪಡೆದ ಮಹಾತ್ಮ ಗಾಂಧೀಜಿಯವರ ಜೀವನದ ಮೂಲಭಾವನೆಗೆ ನಮ್ಮ ರಾಷ್ಟ್ರ ರಚನೆಯ ಶಾಸನದಲ್ಲಿ ಎಡೆಯಿಲ್ಲದೆ ಹೋಯಿತು.

‘ಭರತವರ್ಷದ ಜನತೆ ಈ ಹೊತ್ತು ಕೂಡ ದೇವರು ಎಂಬ ಭಾವವನ್ನು ತೊರೆದಿಲ್ಲ. ಆ ಭಾವನೆ ಎಲ್ಲ ವೇಳೆಯಲ್ಲೂ ಒಳ್ಳೆಯದನ್ನೇ ಮಾಡಿದೆ ಎಂದು ಹೇಳುವಂತಿಲ್ಲ. ಹೀಗಿರುತ್ತ ಭಾರತ ಜನಾಂಗ ಬಹು ಸಂಖ್ಯೆಯಿಂದ ದೇವರು ಇದ್ದಾನೆ ಎನ್ನುವ ಜನಾಂಗ. ಜವಾಹರಲಾಲ್ ನೆಹರು ಈ ಜನಾಂಗದ ನಾಯಕರಾದರೂ ಇದರ ಈ ಭಾವನೆಗೆ ಪ್ರತಿನಿಧಿ ಆಗಲಿಲ್ಲ.

‘ಜವಾಹರಲಾಲರು ದೇವನೊಬ್ಬನಿರುವನೆಂದು ನಂಬದೆಯೇ ಸತ್ಯವಾಗಿ ನಡೆದುಕೊಳ್ಳಬಲ್ಲ ಧ್ಯೇಯವಾದಿ. ಅವರು ಒಟ್ಟಿನಲ್ಲಿ ಸತ್ಯವಾಗಿಯೇ ನಡೆದುಕೊಂಡರು. ಆದರೆ ಬರಿಯ ಧ್ಯೇಯವಾದ ಒಂದು ಆಡಳಿತವನ್ನು ಒಂದು ಧ್ಯೇಯದತ್ತ ನಡೆಸಲಾರದು. ಪ್ರಧಾನಿಯಾದ ಜವಾಹರ ಲಾಲರ ಜೊತೆಗೆ ಅವರಷ್ಟೇ ಶಕ್ತರಾದ ಬೇರೆ ಕೆಲವರು ಇದ್ದರು. ಆದರೆ ಇವರೆಲ್ಲರ ಮನಸ್ಸು ಒಂದು ಮುಖವಾಗಿ ಹರಿಯಲಿಲ್ಲ. ಕೆಲವು ವರ್ಷಗಳ ನಂತರ ಜವಾಹರಲಾಲರ ನೇತೃತ್ವದಲ್ಲಿ ಹಿಂದುಗಳ ಪಾಡು ಅತಂತ್ರವಾಗುತ್ತಿದೆ ಎಂದು ಹೆದರಿ ಕಾಂಗ್ರೆಸ್ಸಿನಲ್ಲಿ ದುಡಿದಿದ್ದ ನಾಯಕರು ಹಲವರು, ಹಿಂದೂ ಮಹಾಸಭೆ ಎಂದು ಒಂದು ಸಂಸ್ಥೆಯನ್ನು ನಿರ್ಮಿಸಿಕೊಂಡರು. ರಾಜಾಜಿ ಜವಾಹರಲಾಲರನ್ನು ಬಿಟ್ಟು ಬೇರೆಯೇ ಒಂದು ಪಕ್ಷವನ್ನು ಸ್ಥಾಪಿಸಿದರು. ವಲ್ಲಭಬಾಯಿ ಪಟೇಲರು ಜವಾಹರಲಾಲರನ್ನು ಬಿಟ್ಟು ಸಿಡಿಯಲಿಲ್ಲ. ಆದರೆ ಇವರಿಬ್ಬರ ಮನಸ್ಸು ಒಮ್ಮುಖವಾಗಿ ಹರಿಯಲಿಲ್ಲ. ಪರಿಸ್ಥಿತಿ ಹೀಗಿರುತ್ತ ಜವಾಹರಲಾಲ್ ನೆಹರು ೧೯೬೪ರಲ್ಲಿ ತೀರಿಕೊಂಡರು. ಆ ಬಳಿಕ ಕಾಂಗ್ರೆಸ್ ಸಂಸ್ಥೆಯೇ ಇಬ್ಭಾಗವಾಯಿತು. ಈಚೆಗೆ ಅದು ಮತ್ತೆ ಮತ್ತೆ ಒಡೆದು ಆರು ಚೂರಾಗಿ ನಿಂತಿದೆ. ಜವಾಹರಲಾಲ್ ಜೀವಿಸಿದ್ದಾಗ ಆತ ಸತ್ಯವಂತ ಎಂಬ ಕಾರಣದಿಂದ ಜೊತೆಯ ಜನ ಅವರಿಗೆ ಬೇಸರವಾಗುವಂತೆ ನಡೆಯಲು ಹಿಂದೆಗೆಯುತ್ತಿದ್ದರು. ಜವಾಹರಲಾಲ್ ತೀರಿಕೊಂಡರು; ಅವರೊಂದಿಗೆ ಈ ಭಯ ತೀರಿಕೊಂಡಿತು. ಪೂರ್ವದ ಜನ ತಪ್ಪು ಮಾಡಿದರೆ “ದೇವರು ದಂಡಿಸುತ್ತಾನೆ” ಎಂದು ಭಯಪಡುತ್ತಿದ್ದರು. ಇದ್ದುದರಲ್ಲಿ ತಕ್ಕ ಮಟ್ಟಿಗೆ ಎಚ್ಚರದಿಂದ ನಡೆಯುತ್ತಿದ್ದರು. ಆ ದೇವರಿಗೆ ಪ್ರತಿಯಾಗಿ ಪಿಳ್ಳೆ ದೇವರಾಗಿದ್ದ ಜವಾಹರಲಾಲರು ತೀರಿಕೊಂಡ ಮೇಲೆ, ನಮ್ಮ ರಾಜಕೀಯ ನಾಯಕರು ತೀರ ನಿರ್ಭಯದಿಂದ, ಒಬ್ಬೊಬ್ಬರೂ ತಮಗೆ ತೋಚಿದಂತೆ ನಡೆಯಲು ತೊಡಗಿದರು. ಇವರಲ್ಲಿ ಕೆಲವರು ಒಳ್ಳೆಯವರೆ–ಆದರೆ ಬಹು ಜನ ಅಲ್ಲ. ಒಟ್ಟಿನಲ್ಲಿ ರಾಷ್ಟ್ರದ ಜೀವನ ನೆನೆಯ ಬಾರದ ರೀತಿಯಲ್ಲಿ ಹಾಳಾಯಿತು.’

ಕ್ರೈಸ್ತ ಹಾಗೂ ಇಸ್ಲಾಂ ಬಾಂಧವಸಮುದಾಯಗಳ ಮುಖಂಡರು ತಮ್ಮ ಧರ್ಮಸಂಸ್ಕೃತಿಗಳ ಮಹಿಮಾತಿಶಯದ ಬಗ್ಗೆ ಹೆಮ್ಮೆ ತಾಳಿ, ಅವುಗಳ ರಕ್ಷಣೆಯ ಪ್ರಜ್ಞೆಯನ್ನು ಆ ಅನುಯಾಯಿಗಳಲ್ಲಿ ಉಂಟುಮಾಡಲು ಕಟಿಬದ್ಧರಾಗಿದ್ದರು. ಅದು ಸಹಜವೇ. ಆದರೆ ಭಾರತದಲ್ಲಿ ಧರ್ಮ ಸಂಸ್ಕೃತಿಯ ಮೂಲಭಾವನೆಗಳು ಉದಾರವಾಗಿದ್ದರೂ, ಪ್ರಪಂಚದ ಎಲ್ಲ ಧರ್ಮಗಳ ಬಗ್ಗೆ ಗೌರವವಿದ್ದರೂ, ‘ಸಮನ್ವಯವೇ ಸಾಧುವಾದುದು’ ಎಂದು ಬೋಧಿಸಿದ್ದ ಅಶೋಕ ಚಕ್ರವರ್ತಿಯ ನಾಡಿನಲ್ಲಿ, ಬಹುಪಾಲು ಹಿಂದುಗಳಿಗೆ ತಮ್ಮ ಧರ್ಮದ ವಿಚಾರ ಶಾಲೆಗಳಲ್ಲಿ ಬೋಧಿಸಲು ಆರ್ಥಿಕ ನೆರವನ್ನು ನೀಡಲು, ಧರ್ಮನಿರಪೇಕ್ಷ ರಾಷ್ಟ್ರವಾದ್ದರಿಂದ ಸರಕಾರ ಸಮ್ಮತಿಸಲಿಲ್ಲ! ಅಷ್ಟೇ ಅಲ್ಲ, ಧರ್ಮದ ವಿಚಾರ ಬೇಡ, ಸಮಸ್ತ ಮಾನವಜನಾಂಗಕ್ಕೆ ಹಿತಕಾರಿಯಾದ ನೈತಿಕ ಹಾಗೂ ಆಧ್ಯಾತ್ಮಿಕ ಮೌಲ್ಯಗಳನ್ನಾದರೂ ಮಕ್ಕಳಿಗೆ ಶಾಲೆಯಲ್ಲಿ ವ್ಯವಸ್ಥಿತ ರೀತಿಯಲ್ಲಿ ತಿಳಿಸಲು ಏರ್ಪಾಡಾಗಬೇಕೆಂದಾಗ ರಾಜ್ಯಾಂಗ ಘಟನೆಯಲ್ಲಿ ಅದನ್ನೂ ವಿರೋಧಿಸಲಾಯಿತೆಂದು ಶ‍್ರೀ ಹೆಚ್.\ ವಿ.\ ಕಾಮತ್ ಅವರು ಹೇಳಿದ್ದುಂಟು.\footnote{\engfoot{When to Article 28 which prohibits ‘religious instruction’ in educational institutions maintained out of State Funds, I moved an amendment that the ban should not apply to ‘Moral and Spiritual Instruction’, it was rejected, and I felt sad; it augured ill for the future... India’s political awakening was preceded by a luminous religio-spiritual renasissance which indeed provided the foundation and infrastructure for our freedom struggle. So, if our parliamentary democratic polity had continued to be vivified by Moral and Spiritual Values, all would have been well with us.}\hfill\engfoot{–H. V. Kamath, Bhavan’s Journal, August 1981.}}

\newpage


\section*{ಮೂಲಕ್ಕೇ ಕೊಡಲಿ ಏಟು!}

\addsectiontoTOC{ಮೂಲಕ್ಕೇ ಕೊಡಲಿ ಏಟು!}

ನಮ್ಮ ಧರ್ಮ ಸಂಸ್ಕೃತಿಯಲ್ಲಿ ಹುದುಗಿದ್ದ ಅಮೋಘ ಭಾವನೆಗಳನ್ನು ಶತಮಾನಗಳಿಂದ ಆ\break ಭಾವನೆಗಳ ಪರಿಚಯವೇ ಆಗಿರದಿದ್ದ ಹಿಂದುಳಿದ ಜನರಿಗೆ ತಿಳಿಸಿಕೊಡುವುದಿರಲಿ, ‘ಧರ್ಮ ಕರ್ಮ ಎಂದೇ ನಾವು ಕೆಟ್ಟೆವು; ಅದು ಮೋಸ, ಮೂಢನಂಬಿಕೆ, ಶೋಷಣೆ’ ಎಂಬಂಥ ಪ್ರಚಾರವೇ ನಮ್ಮ ಮುಖಂಡರಿಂದಲೇ ನಡೆದುದು ಎಂಥ ವಿಪರ್ಯಾಸ ಹಾಗೂ ದುರಂತ! ಧರ್ಮನಿರಪೇಕ್ಷತೆ, ವೈಜ್ಞಾನಿಕತೆ, ವೈಚಾರಿಕತೆಗಳ ಹೆಸರಲ್ಲಿ ಜನರಿಗೆ ಉನ್ನತವಾದ ನೈತಿಕ, ಆಧ್ಯಾತ್ಮಿಕ ಭಾವನೆಗಳ ಪರಿಚಯವಾಗದಂತೆ ಮಾಡಿ, ಈ ಮುಖಂಡರು ಹೇಗೆ ಒಂದು ತೆರನಾದ ಶೋಷಣೆಯ\break ಯಂತ್ರವಾದರು ಎಂಬುದನ್ನು ಕಾಣಬಹುದು. ಹಿಂದೂ ಸಮಾಜದಲ್ಲಿ ಏಕತೆಯನ್ನುಂಟು\-ಮಾಡಲು ಸುವರ್ಣ ಅವಕಾಶವಿದ್ದರೂ ಅದನ್ನು ಸದುಪಯೋಗ ಮಾಡದೆ ಸಂಕುಚಿತ ಸ್ವಾರ್ಥದಿಂದ ಅಪ್ರತ್ಯಕ್ಷವಾಗಿ ಗುಲಾಮರಿಗೆ ಸಹಜವಾದ ವಿಚ್ಛಿದ್ರಕಾರಕ ಮನೋವೃತ್ತಿಯನ್ನೇ ಬೆಳೆಸು\-ವಂತಾ\-ಯಿತು! ಇತರ ಧರ್ಮದ ಅನುಯಾಯಿಗಳು ಸ್ವಮತ ಪ್ರಚಾರ, ಪರಮತ ನಿಂದನೆ– ಖಂಡನೆಯನ್ನು ಮುಂದುವರಿಸುತ್ತ, ತಮ್ಮ ಧರ್ಮಾನುಯಾಯಿಗಳಲ್ಲಿ ರಾಜಕೀಯ ಪ್ರಜ್ಞೆಯನ್ನು ಮೂಡಿಸುತ್ತ, ಮತ ಪರಿವರ್ತನೆಯ ಮೂಲಕ ತಮ್ಮ ಗುಂಪನ್ನು ಬಲಪಡಿಸಿಕೊಳ್ಳುತ್ತ, ಸರಕಾರದಿಂದ ನಾನಾರೀತಿಯ ಸವಲತ್ತುಗಳನ್ನು ಪಡೆದುಕೊಳ್ಳುವುದರಲ್ಲಿ ಸಮರ್ಥರಾದರು. ನಮ್ಮ ಮುಖಂಡರು ದೂರದೃಷ್ಟಿಯ ಅಭಾವದಿಂದ ನಮ್ಮ ಜನರ ಶ್ರದ್ಧಾಬಿಂದುಗಳನ್ನು ದುರ್ಬಲಗೊಳಿಸುತ್ತ ನಡೆದರು, ಒಳಜಗಳಗಳನ್ನು ಅಪ್ರತ್ಯಕ್ಷವಾಗಿ ಪ್ರೋತ್ಸಾಹಿಸಿದರು. ನಾಡಿನ ಏಳ್ಗೆಗೆ ತಿಲಾಂಜಲಿ\-ಯಿತ್ತು, ಸ್ವಾರ್ಥದ ಬೇಳೆ ಬೇಯಿಸಿಕೊಳ್ಳುವುದರಲ್ಲೇ ಅವರು ಮುಳುಗಿ ಹೋದರು. ಜನಾಂಗದ ಬದುಕಿನ ಜೀವರಸದಂತಿದ್ದ ಧರ್ಮ ಸಂಸ್ಕೃತಿಯ ಮೌಲ್ಯಗಳನ್ನೆಲ್ಲ ಬಿಟ್ಟು ರಾಷ್ಟ್ರ ಜೀವನವನ್ನು ಶಿಥಿಲಗೊಳಿಸಿದರು.


\section*{ಮೂಢನಂಬಿಕೆಯೇ?}

\addsectiontoTOC{ಮೂಢನಂಬಿಕೆಯೇ?}

ವಿಜ್ಞಾನಶಾಸ್ತ್ರಗಳನ್ನು ಭಾಗಶಃ ಓದಿಕೊಂಡು ತಮ್ಮನ್ನು ವೈಜ್ಞಾನಿಕ ಮನೋವೃತ್ತಿಯವರೆಂದು ತಿಳಿದುಕೊಂಡ ಜನರೂ, ವಿಶಿಷ್ಟ ರಾಜಕೀಯ ಸಿದ್ಧಾಂತಗಳಿಗೆ ಕಟ್ಟುಬಿದ್ದ ಮುಖಂಡರೂ, ನಂಬಿಕೆಯನ್ನು–ಮುಖ್ಯವಾಗಿ, ದೇವರು, ಧರ್ಮ, ಆತ್ಮ–ಇವುಗಳಲ್ಲಿನ ನಂಬಿಕೆಯನ್ನು, ಮೂಢನಂಬಿಕೆ ಎಂದು ಅಲ್ಲಗಳೆಯಬಹುದು. ಧರ್ಮದ ದುರ್ಬಲ ಅನುಯಾಯಿಗಳನ್ನೂ, ಧರ್ಮದ ಹೆಸರಲ್ಲಿ ನಡೆದ ಅತ್ಯಾಚಾರ, ಅನಾಚಾರಗಳನ್ನೂ ಕಂಡು ಅವರು ಇಂಥ ಟೀಕೆ ಮಾಡಿದ್ದಾರೆ ಎಂದರೂ ಅದೊಂದು ಸರಿಯಾದ ವಿಮರ್ಶೆಯಾಗುವುದೇ?\footnote{\engfoot{Science deals with but a partial aspect of reality. There is no faintest reason for supporting that everything Science ignores is less real than what it accepts. We are no longer taught that the Scientific Method of approaching is the only valid method of acquiring knowledge about reality. Eminent men of Science are insisting, with what seems a strange enthusiasm, on the fact that Science gives us but a partial knowledge of reality, and we are no longer required to regard as illusory, everything that Science finds itself able to ignore.}\hfill\engfoot{\general{\hbox\bgroup}–W. N. Sullivan\general{\egroup}}

\engfoot{In matters of Philosophy and Religion, Science is not the arbiter. The final judgement on ultimate questions must always rest with Philosophy and Religion themselves.\general{~\hfill\hbox\bgroup}–W. J. Sollas\general{\egroup}}

\engfoot{Incompatibility between Science and Faith exists in the minds of those who want such an imcompatibility.}\hfill\engfoot{ –Prof.\ Lecomte}} ಏಕೆಂದರೆ, ಧರ್ಮವೃಕ್ಷವು ಅತ್ಯುನ್ನತ ಚಾರಿತ್ರ್ಯದ ಉನ್ನತ ಮಾನವರೆಂಬ ಫಲಗಳನ್ನು ನೀಡಿದ್ದು ದಿಟ. ಅಂಥ ಉನ್ನತ ಚಾರಿತ್ರ್ಯದ ಹಿನ್ನೆಲೆಯಲ್ಲಿ ಕೆಲಸ ಮಾಡಿದ ಪ್ರೇರಣೆ, ಆ ದೃಢವಾದ ನೈತಿಕ ನಿಷ್ಠೆಯ ಅಡಿಪಾಯದಲ್ಲಿರುವ ಅನುಭವ–ಇವನ್ನು ಆಳವಾಗಿ ಅಧ್ಯಯನ ಮಾಡದೆ ‘ಮೂಢನಂಬಿಕೆ’ ಎಂದು ತೆಗಳುವ ಇಂಥವರ ನಾಲಗೆ ಅವರ ಸಂಕುಚಿತತೆ ಹಾಗೂ ಒಂದು ತೆರನಾದ ಮೂಢನಂಬಿಕೆಗಳನ್ನೇ ಗಳಹುತ್ತದೆಂಬುದು ಅವರಿಗೆ ತಿಳಿಯದಿದ್ದುದು ದುರ್ದೈವ! ಇನ್ನು ವಿಜ್ಞಾನವು ಧಾರ್ಮಿಕ ಭಾವನೆಗೆ ಅಥವಾ ತತ್ತ್ವಕ್ಕೆ ಕುಠಾರಾಘಾತ ಮಾಡಿತು ಎಂಬ ವಾದವೂ ಹುರುಳಿಲ್ಲದ್ದು. ವಿಜ್ಞಾನವು ತನ್ನ ಸಂಶೋಧನೆಗಳಿಂದ ಕೆಲವೊಂದು ಮೂಢನಂಬಿಕೆಗಳನ್ನು ಕಿತ್ತೆಸೆದುದು ದಿಟ. ಆದರೆ ಪ್ರಪಂಚದ ಮಹಾಧರ್ಮಗಳು ಸಾರುವ ಮೂಲತತ್ತ್ವಗಳಾಗಲಿ, ಸಮಾಜಜೀವಿಯಾದ ಮನುಷ್ಯನು ಪರಿಪಾಲಿಸಬೇಕಾದ ನೈತಿಕ ನಿಯಮಗಳಾಗಲಿ, ವಿಜ್ಞಾನವೆಂದು ನಾವು ತಿಳಿದುಕೊಂಡಿ ರುವ ಸತ್ಯಾನ್ವೇಷಣೆಯ ಕ್ರಮದ ಪರಿಧಿಯೊಳಗೆ ಬರತಕ್ಕವಲ್ಲ. ಸತ್ಯಪ್ರಿಯತೆ, ನ್ಯಾಯ ಪರತೆ, ಪರೋಪಕಾರಬುದ್ಧಿ, ಆತ್ಮಸಂಯಮ–ಈ ಗುಣಗಳನ್ನು, ವಿವೇಚನಾಶಕ್ತಿ ಇರುವ ಯಾವ ವ್ಯಕ್ತಿಯೂ ವಿರೋಧಿಸಲಾರನಷ್ಟೆ! ಕ್ಷಣಿಕ ಸುಖಕ್ಕಾಗಿ ಮಾನವನು ಈ ಸದ್ಗುಣಗಳನ್ನು ಬಲಿ ಕೊಟ್ಟು ಎಂದೆಂದೂ ದುರ್ಬಲನಾಗಿ ಬದುಕಬಾರದೆಂದೇ ಧರ್ಮ ಬೋಧಿಸುತ್ತದೆ. ಈ ಸದ್ಗುಣಗಳ ಪರಿಪೂರ್ಣ ಬೆಳವಣಿಗೆಗೆ, ಸಾಮಾನ್ಯ ಮನಸ್ಸಿಗೆ ಗೋಚರಿಸದ ಅತೀಂದ್ರಿಯ ತತ್ತ್ವಗಳಲ್ಲಿ, ಪರಮಾತ್ಮನಲ್ಲಿ ವಿಶ್ವಾಸ ಬೇಕೇಬೇಕಾಗುತ್ತದೆ. ಈ ತತ್ತ್ವಗಳ ಯಥಾರ್ಥ ಸ್ವರೂಪವನ್ನು ಜಗದ ಮಹಾ ಅನುಭಾವಿಗಳು ತಾವು ಅನುಭವಿಸಿ, ತಮ್ಮ ಬದುಕಿನ ನಿದರ್ಶನದಿಂದಲೂ ಬೋಧಿಸಿದ್ದಾರೆ. ಆದರೆ ಭಾರತವು ಜಗತ್ತಿನ ಎಲ್ಲ ಅನುಭಾವಿಗಳ ಬೋಧನೆಯಲ್ಲೂ ಇರುವ ಮುಖ್ಯಾಂಶದಲ್ಲಿ ಕಂಡುಬರುವ ಏಕತೆಯನ್ನು ಗುರುತಿಸಿ, ಸಮನ್ವಯ ಸಂದೇಶವನ್ನು ಜಗತ್ತಿಗೆ ಮತ್ತೆ ಮತ್ತೆ ಸಾರಿದೆ. ಮನುಷ್ಯನ ಪರಿಪೂರ್ಣ ಅಭ್ಯುದಯಕ್ಕೆ, ಸಮಾಜದ ಕಲ್ಯಾಣಕ್ಕೆ, ಈ ಭಾರತೀಯ ದೃಷ್ಟಿಕೋನ ಅತ್ಯಂತ ಆವಶ್ಯಕ ಎಂಬುದನ್ನು ಇತಿಹಾಸ ತಜ್ಞ ಆರ್ನಾಲ್ಡ್ ಟೊಯ್ನಬಿ ಸಾರಿದ.\footnote{\engfoot{The Indian religions are not exclusively-minded. They are ready to allow that there may be alternative approaches to the mystery. I feel sure that in this they are right, and that this catholic-minded Indian religious spirit is the way of salvation for human beings of all religions in an age in which we have to learn to live as a single family if we are not to destroy ourselves.}

~\hfill\engfoot{–Arnold Toynbee}} ಮನುಷ್ಯನ ನಡತೆಯನ್ನು ರೂಪಿಸುವುದರಲ್ಲಿ ಪಾರಮಾರ್ಥಿಕ ಸತ್ಯಗಳೆಂದು ನಿರ್ಣೀತವಾದ ದೇವರು, ಆತ್ಮ–ಇವುಗಳಲ್ಲಿನ ಶ್ರದ್ಧೆ ಪ್ರಮುಖ ಪಾತ್ರವನ್ನು ವಹಿಸುತ್ತದೆ ಎಂಬುದನ್ನು ಯಾರೂ ಅಲ್ಲಗಳೆಯುವಂತಿಲ್ಲ.


\section*{ಪಾರಮಾರ್ಥಿಕ ಸತ್ಯ}

\addsectiontoTOC{ಪಾರಮಾರ್ಥಿಕ ಸತ್ಯ}

ತನ್ನ \enginline{Ends and Means} ಎನ್ನುವ ಪ್ರಸಿದ್ಧ ಗ್ರಂಥದಲ್ಲಿ ಆಲ್ಡಸ್ ಹಕ್ಸಲೀಯ ಆಳಚಿಂತನೆಯ ಫಲವಾದ ಮಾತಿಲ್ಲಿ ಗಮನಾರ್ಹ: ‘ಪಾರಮಾರ್ಥಿಕ ಸತ್ಯದ ಬಗೆಗಿನ ನಮ್ಮ ಶ್ರದ್ಧೆಯನ್ನನುಸರಿಸಿ, ಒಳಿತು ಕೆಡುಕುಗಳನ್ನು ಕುರಿತು ನಾವು ಒಂದು ನಿಶ್ಚಿತ ಅಭಿಪ್ರಾಯಗಳನ್ನು ಮಾಡಿಕೊಳ್ಳುತ್ತೇವೆ. ಈ ಒಳಿತು ಕೆಡುಕುಗಳ ಭಾವನೆಯನ್ನನುಸರಿಸಿ ನಮ್ಮ ವರ್ತನೆ ಅಭಿವ್ಯಕ್ತಿಗೊಳ್ಳುತ್ತದೆ. ಇದು ನಮ್ಮ ಆಂತರಿಕ ಜೀವನದಲ್ಲಿ ಮಾತ್ರವಲ್ಲ, ಸಾಮಾಜಿಕ, ಆರ್ಥಿಕ ಹಾಗೂ ರಾಜಕೀಯ ಜೀವನ ದಲ್ಲೂ ಪ್ರಕಟವಾಗುವುದು. ಸಾಮಾನ್ಯ ವ್ಯಾವಹಾರಿಕ ತಜ್ಞನಾಗಲೀ, ಬುದ್ಧಿಜೀವಿಯಾಗಲೀ, ಜೀವನದ ಕೊನೆಯ ಗುರಿ, ಪರಮಾರ್ಥ–ಇವುಗಳನ್ನು ಕುರಿತು “ನಾವೇಕೆ ತಲೆ ಬಿಸಿ ಮಾಡಿಕೊಳ್ಳಬೇಕು”, “ಒಳ್ಳೆಯ ರೀತಿಯಿಂದ ಬದುಕಿದರಾಯಿತು”, “ನಾನು ಮಾನವತಾವಾದಿ” ಎನ್ನಬಹುದು. ಆದರೆ ಯಥಾರ್ಥವಾಗಿ, ಮನುಷ್ಯನ ವರ್ತನೆಯನ್ನು ನಿಯಂತ್ರಿಸುವುದು ಸತ್ಯಸ್ಯ ಸತ್ಯವೆಂದು ಪರಿಗಣಿತವಾದ ಪರಮಾರ್ಥದಲ್ಲಿನ ಶ್ರದ್ಧೆಯೇ.\footnote{\engfoot{It is in the light of our beliefs about the ultimate nature of reality that we formulate our conceptions of right and wrong; and it is in the light of our conceptions of right and wrong that we frame our conduct, not only in the relations of private life, but also in the sphere of politics and economics. So, far from being irrelevant, our metaphysical beliefs are the finally determining factor in all our actions.}\hfill\engfoot{–Aldous Huxley}}


\section*{ನಾಗರಿಕತೆಯ ಅನಾಗರಿಕತೆ}

\addsectiontoTOC{ನಾಗರಿಕತೆಯ ಅನಾಗರಿಕತೆ}

ಇಂದು ಎಲ್ಲ ಕ್ಷೇತ್ರಗಳಲ್ಲೂ ಈ ಶ್ರದ್ಧೆಯ ಅಭಾವ ಕಂಡುಬರುತ್ತಿದೆ. ದೇವರಲ್ಲಿ, ನ್ಯಾಯ ನೀತಿಗಳಲ್ಲಿ ಶ್ರದ್ಧೆಯ ಅಭಾವ ಮಾತ್ರವಲ್ಲ. ಆತ್ಮಶ್ರದ್ಧೆಯ ಅಭಾವವೂ ಸರ್ವತ್ರ, ವಿದ್ಯಾವಂತ ರಲ್ಲೇ ಪ್ರಸಾರವಾಗುತ್ತಲಿದೆ. ರಾಷ್ಟ್ರದ ಮುಖಂಡರುಗಳಲ್ಲಿ, ಹೆಚ್ಚು ಕಡಿಮೆ ಎಲ್ಲರೂ, ವಿಜ್ಞಾನ ಮತ್ತು ತಾಂತ್ರಿಕ ಜ್ಞಾನದ ನಿರಂತರ ಅಭಿವೃದ್ಧಿಯೇ ಭಾರತದ ಪ್ರಗತಿಗೂ, ಬಲಸಂವರ್ಧನೆಗೂ ಏಕಮಾತ್ರ ಉಪಾಯ ಎಂದು ನಂಬಿದ್ದಾರೆ. ಸ್ವಾಭಾವಿಕ ಒಲವು ಮತ್ತು ರಾಜಕೀಯ ಆವಶ್ಯಕತೆಗಳ ಸೆಳೆತದಿಂದ ನಮ್ಮ ರಾಷ್ಟ್ರ ಪರಿಚಿತ ಪರಂಪರೆಯಿಂದ ದೂರ ಸರಿಯುತ್ತಿದೆ. ಎಂಬತ್ತೈದು ವರ್ಷಗಳ ಹಿಂದೆ ನಮ್ಮ ದೇಶದಲ್ಲಿ ಮೂರು ವಿಶ್ವವಿದ್ಯಾಲಯಗಳಿದ್ದರೆ, ಇಂದು ನೂರೈವತ್ತು ಇರಬಹುದು. ವಿದ್ಯಾಭ್ಯಾಸದಲ್ಲಿ ವಿಜ್ಞಾನ ಮತ್ತು ತಾಂತ್ರಿಕ ವಿಚಾರಗಳಿಗೆ ಅತಿ ಹೆಚ್ಚಿನ ಪ್ರಾಮುಖ್ಯ ಕೊಡಲಾಗಿದೆ. ವಿದ್ಯಾರ್ಥಿನಿಯರೂ ತಾಂತ್ರಿಕ ವಿಷಯಗಳಿಂದ ಆಕರ್ಷಿತರಾಗುತ್ತಿದ್ದಾರೆ. ವಿಷಯ ನಿರೂಪಣೆಯ ನಮ್ಮ ಶಿಕ್ಷಣವಿಧಾನವು, ಸ್ವಭಾವ ನಿರ್ಮಾಣ ಕಾರ್ಯದಲ್ಲಿ ಅಸಮರ್ಥವಾದುದು. ಜೀವನೋಪಾಯಕ್ಕೆ ಅನುಕೂಲವಾಗುವಂತೆ ಏನಾದರೊಂದಿಷ್ಟು ಕಲಿತರಾಯಿತು ಎಂಬುದೇ\break ಅದರ ಯಾವತ್ತೂ ಗುರಿಯಾಗಿದೆ. ಅದರಲ್ಲಿ ಸಮಗ್ರ ವ್ಯಕ್ತಿತ್ವದ ವಿಕಾಸಕ್ಕೆ ಗಮನವೇ ಇಲ್ಲ. ಜೀವನಮಟ್ಟವನ್ನು ಉತ್ತಮಗೊಳಿಸುವುದೆಂದರೆ ಸುಖಸೌಕರ್ಯಗಳನ್ನು ಹೆಚ್ಚಿಸಿಕೊಂಡು, ಕೀಳು ಕಾಮನೆಗಳನ್ನು ತೃಪ್ತಿಪಡಿಸುವುದೆಂದೇ ತಿಳಿಯಲಾಗಿದೆ. ಕಡಿಮೆ ಶ್ರಮದ, ಆದರೆ ಅತಿ ಹೆಚ್ಚಿನ ಆದಾಯವನ್ನು ಕೊಡುವ ಉದ್ಯೋಗ ಪಡೆಯುವುದೇ ಆಧುನಿಕ ತರುಣನ ಅಭೀಪ್ಸೆ! ಎಷ್ಟೋ ವೇಳೆ, ವಿದ್ಯಾರ್ಥಿಗಳು ಮುಷ್ಕರ ಹೂಡಿ ಪರೀಕ್ಷೆಗಳನ್ನು ಮುಂದೂಡಬಲ್ಲರು! ಪರೀಕ್ಷೆಗಳನ್ನು ಕಡಿಮೆ ಮಾಡಿಕೊಳ್ಳಬಲ್ಲರು! ಚಾಕು, ಚೈನುಗಳನ್ನು ತೋರಿಸಿ ಪರೀಕ್ಷಾಧಿಕಾರಿಗಳನ್ನು ಗದರಿಸ\-ಬಲ್ಲರು! ಕೆಲವೆಡೆ ಪೋಲೀಸರ ಹದ್ದುಬಸ್ತಿನಲ್ಲಿ ಪರೀಕ್ಷೆ ನಡೆಯಬೇಕಾದ ಪರಿಸ್ಥಿತಿ! ಪ್ರಶ್ನಪತ್ರಿಕೆಯನ್ನು ಮೊದಲೇ ಪಡೆಯಬಲ್ಲರು! ಲಂಚ ನೀಡಿ ರ್ಯಾಂಕ್ ಗಿಟ್ಟಿಸಬಲ್ಲರು! ಈ ರೀತಿ ವಿದ್ಯೆಯನ್ನು ಸಂಪಾದಿಸಿ ಡಿಗ್ರಿ ಪಡೆದು ಉದ್ಯೋಗಸ್ಥರಾದೊಡನೆ, ಅವರು ವಿವೇಕ, ಶಿಸ್ತು, ಸಂಯಮ, ಸಭ್ಯತೆಗಳಿಂದ ಶೋಭಿಸುವ ಸುಯೋಗ್ಯ ನಾಗರಿಕರಾಗಿಬಿಡುವರೇ? ಶಿಸ್ತು ಸಂಯಮಗಳಿಲ್ಲದ, ಔಚಿತ್ಯ ಜ್ಞಾನವಿಲ್ಲದ, ನೈತಿಕನಿಷ್ಠೆಯಿಲ್ಲದ ವಿದ್ಯೆ ತುಂಟಕುದುರೆಯಾಗಬಲ್ಲದು. ಅದು ಹಾಕಿದ್ದೇ ಹೆಜ್ಜೆ, ನಡೆದದ್ದೇ ದಾರಿ. ಪರಿಣಾಮವಾಗಿ ಅದು ಆತ್ಮಘಾತಕ ಹಾಗೂ ಸಮಾಜಘಾತುಕ ಕೃತ್ಯಗಳಿಗೆಳಸಬಲ್ಲದು. ಅಪಾರ ಮಟ್ಟದ ವಿದ್ಯಾರ್ಥಿಗಳ ಅಶಿಸ್ತು, ಅಶಾಂತಿ ಮತ್ತು ದುರ್ನಡತೆಗಳು ಮೇಲಿನ ಮಾತಿಗೆ ಸಾಕ್ಷಿಯಾಗಿವೆ. ಸಮಾಜದಲ್ಲಿ, ಆಡಳಿತದಲ್ಲಿ, ಎತ್ತರದ ಸ್ಥಾನದಲ್ಲಿ ಇರುವ ಮುಖಂಡರು ತಮ್ಮ ನೈತಿಕನಿಷ್ಠೆಯಿಂದ ಜನರಿಗೆ ಆದರ್ಶ ವ್ಯಕ್ತಿಗಳಾಗಬೇಕು. ಶ್ರೇಷ್ಠರು ಯಾವುದನ್ನು ಆಚರಿಸುತ್ತಾರೋ ಅದನ್ನು ಇತರರೂ ಅನುಸರಿಸುತ್ತಾರೆಂಬುದು ಭಗವದ್ಗೀತೆಯ ಉಕ್ತಿ. ಸರದಾರನಾಗುವವನು ‘ಸಿರದಾರ’ನೂ ಆಗಬೇಕು. ಎಂದರೆ ಮುಖಂಡನಾಗುವವನು ತ್ಯಾಗ ಸಮರ್ಪಣೆಯಲ್ಲಿ ಅಗ್ರ ಗಣ್ಯನೇ ಆಗಬೇಕು. ಆದರೆ, ಜನರ ನೈತಿಕಮಟ್ಟದ ಉನ್ನತಿಗೆ, ರಾಜಕೀಯ ಮುಖಂಡರು ಮಾರ್ಗದರ್ಶಕರಾಗಿಲ್ಲದಿರುವುದು ನಮ್ಮ ದೌರ್ಭಾಗ್ಯ. ಸಮಾಜದ ಮುಖಂಡರಾಗುವ ಹಂಬಲವಿರುವವರಿಗೆ ಅರ್ಥಶಾಸ್ತ್ರ, ರಾಜಕೀಯಶಾಸ್ತ್ರದ ಪರಿಚಯ ಬೇಕಿಲ್ಲ. ದೇಶದ ಇತಿಹಾಸದ ಪರಿಚಯವೂ ಬೇಕಿಲ್ಲ. ಯಾವ ತರಬೇತಿಯೂ ಬೇಕಿಲ್ಲ. ಸಂಕುಚಿತ ಸ್ವಾರ್ಥವೇ ಸಾಕಾಗುತ್ತದೆ! ಮುಖ್ಯವಾಗಿ, ಮತದಾರರನ್ನು ಆಕರ್ಷಿಸಿ, ಅವರ ಪ್ರತಿನಿಧಿಗಳಾಗುವ ತಂತ್ರ ಗೊತ್ತಿದ್ದರೆ ಸಾಕು! ಈ ಮತದಾರರಲ್ಲಿ ಶೇಕಡಾ ಎಪ್ಪತ್ತು ಮಂದಿ ವಿದ್ಯೆ ಇಲ್ಲದವರು, ತತ್ಕಾಲದ ಲಾಭದ ಹೊರತು ರಾಷ್ಟ್ರದ ಸಮಗ್ರಹಿತ, ಭವಿಷ್ಯ ಚಿಂತನೆ ಮಾಡುವ ಸಾಮರ್ಥ್ಯವಿಲ್ಲದವರು. ರಾಜಕೀಯಸ್ಥರು ಇದನ್ನು ಅರಿತುಕೊಂಡು ತಮ್ಮ ಸ್ವಾರ್ಥವನ್ನು ಸಾಧಿಸಿಕೊಳ್ಳುವುದರಲ್ಲಿ ಆಸಕ್ತರು. ‘ರಾಜಕಾರಣಿಗಳು ಈಗಿನ ದಿನಗಳಲ್ಲಿ ನೈತಿಕ ತತ್ತ್ವವನ್ನು ತಿರಸ್ಕರಿಸಿದಷ್ಟು ಈ ಹಿಂದೆ ಎಂದೂ ಮಾಡಿದ್ದಿಲ್ಲ. ಸಾರ್ವಜನಿಕ ಜೀವನದ ಹಾನಿಗೆ ಕಾರಣವಾದ ಈ ನಡವಳಿಕೆಗೆ ಕಾಯಕಲ್ಪ ಅಗತ್ಯ’ ಎಂದು ಹತ್ತುವರ್ಷಗಳ ಹಿಂದೆ ರಾಷ್ಟ್ರಪತಿಗಳೇ ಕರೆ ನೀಡುವಂತಾಗಿತ್ತು ಎಂದ ಮೇಲೆ ಈಗಿನ ಪರಿಸ್ಥಿತಿ ಊಹಿಸಿಕೊಳ್ಳಿ! ಇಂದು ಅನೈತಿಕತೆ, ಲಂಚಕೋರತನ, ಕಾಳಸಂತೆಗಳು ನಿರ್ಭೀತಿಯಿಂದ ಸರ್ವತ್ರ ತಾಂಡವವಾಡುತ್ತಿವೆ. ದೇಶದಲ್ಲಿ ಕೊಲೆ, ದರೋಡೆಗಳು ಹೆಚ್ಚುತ್ತಲಿವೆ. ಸ್ತ್ರೀಯರನ್ನು ಅತ್ಯಂತ ಹೀನಾಯವಾಗಿ ನಡೆಸಿಕೊಳ್ಳುವ ಘಟನೆಗಳು ಸಮಾಜದ ವಿಪರೀತ ನೈತಿಕ ಅಧಃಪತನದ ಲಕ್ಷಣವೆಂದು ಸುಪ್ರೀಂಕೋರ್ಟಿನ ಮುಖ್ಯನ್ಯಾಯಾಧೀಶರು ಹೇಳಿದ್ದರು. ಬ್ರಿಟಿಷರ ಕಾಲದಲ್ಲಿ, ದೆಹಲಿಯಲ್ಲಿದ್ದ ಪೋಲೀಸರ ಸಂಖ್ಯೆ ಕೇವಲ ಒಂದು ಸಾವಿರ ಆಗಿದ್ದರೆ, ಈಗ ಈ ಸಂಖ್ಯೆ ಮೂರು ಲಕ್ಷಕ್ಕೇರಿದೆ. ಆದರೂ ಹಗಲಿನಲ್ಲೇ ಹೆಂಗಸು ಒಬ್ಬಳೇ ತಿರುಗಲು ಹೆದರುವಂತಾಗಿದೆ!


\section*{ಈ ಗತಿ! ಅವನತಿ!}

\addsectiontoTOC{ಈ ಗತಿ! ಅವನತಿ!}

ಬ್ರಿಟಿಷರು ತಮ್ಮ ಆಳ್ವಿಕೆಯ ಕಾಲದಲ್ಲಿ ಕೈಗಾರಿಕೆಗಳನ್ನು ನಾಶ ಮಾಡಿ ದೇಶದ ಸಂಪತ್ತನ್ನು ಸಾಕಷ್ಟು ಸುಲಿಗೆ ಮಾಡಿದ್ದರು.\footnote{\engfoot{William Digby estimates (Prosperous British India, Page 33) that probably between (the battle of) Plassey (1757) and Waterloo (1815) a sum of \pounds 1000 million was transfered from Indian hoards to English Banks.}

~\hfill\engfoot{–\textit{Tarachand, History of Freedom Movement.}}} ನೂರರಲ್ಲಿ ನಲ್ವತ್ತೈದು ಮಂದಿಗೆ ಎರಡು ಹೊತ್ತಿನ ಊಟಕ್ಕೆ ಆಹಾರವಿರಲಿಲ್ಲ. ಈ ಬಡತನವನ್ನು ದೂರಮಾಡಿ, ಬಡವರಿಗೆ ವಿದ್ಯೆ ನೀಡಿ, ರಾಷ್ಟ್ರದ ಸರ್ವಾಂಗೀಣ ಉನ್ನತಿ ಸಾಧಿಸಬೇಕೆಂಬ ಹಂಬಲವೇ ಅಂದಿನ ಮುಖಂಡರ ಸಂಕಲ್ಪವಾಗಿತ್ತು. ಸ್ವಾತಂತ್ರ್ಯಾನಂತರದ ಈ ವರ್ಷಗಳಲ್ಲಿ ವಿವಿಧ ಕ್ಷೇತ್ರಗಳಲ್ಲಿನ ವಿಶಿಷ್ಟ ಪ್ರಗತಿಯನ್ನು ಅಲ್ಲಗಳೆಯುವಂತಿಲ್ಲವಾದರೂ, ಬಡತನದ ರೇಖೆ ಕೆಳಗಿಳಿದ ವರದಿ ಇಲ್ಲ.

ಪಶ್ಚಿಮದ ದೇಶಗಳಲ್ಲಿ ಸಂಪತ್ತು ಹೆಚ್ಚಿನ ಪ್ರಮಾಣದಲ್ಲಿ ಶೇಖರವಾದ ಬಳಿಕ, ಕಾರ್ಮಿಕರು ತಮ್ಮ ಹಕ್ಕಿನ ಭಾವನೆಯನ್ನು ವ್ಯಕ್ತಗೊಳಿಸಿ, ಮುಷ್ಕರಗಳ ಕ್ರಮವನ್ನು ಜಾರಿಗೆ ತಂದರು. ಆದರೆ ಭಾರತದಲ್ಲಿ ಬಡತನ ದೂರವಾಗುವುದಕ್ಕೆ ಮೊದಲೇ ಕಾರ್ಮಿಕರು ತಮ್ಮ ಹಕ್ಕುಗಳಿಗಾಗಿ ಹೋರಾಟ, ಮುಷ್ಕರದ ಕ್ರಮಗಳನ್ನು ಕೈಗೊಂಡಿದ್ದಾರೆ. ತಮ್ಮ ಬೇಡಿಕೆಗಳು ಮುಷ್ಕರ, ಸತ್ಯಾಗ್ರಹಗಳ ಮೂಲಕ ಕೈಗೂಡದಿದ್ದಾಗ, ಸಾರ್ವಜನಿಕ ವಸ್ತುಗಳನ್ನು ನಾಶ ಮಾಡುವ ಆತ್ಮಘಾತುಕ ಪ್ರವೃತ್ತಿಯೂ ಬೆಳೆ\-ಯುತ್ತ\-ಲಿದೆ. ಕಾರ್ಮಿಕರು ತಮಗೆ ಅನ್ನ ನೀಡಿದ, ನೀಡುವ ಸಂಸ್ಥೆಯ ಅಥವಾ ಕಾರ್ಖಾನೆಯ ಸಾಮಗ್ರಿ–ಯಂತ್ರ ಸಲಕರಣೆಗಳನ್ನು ಕೋಪೋದ್ರಿಕ್ತರಾಗಿ, ನಾಶಮಾಡಿಬಿಡುವುದುಂಟು.

ವಿದ್ಯಾರ್ಥಿಗಳೂ ಈ ನಾಶಕಾರ್ಯದಲ್ಲಿ ಸಾಕಷ್ಟು ಭಾಗಿಗಳಾಗುತ್ತಲಿದ್ದಾರೆ. ಯಾರೋ ಮಾಡಿದ ತಪ್ಪಿಗೆ ಕೋಟ್ಯಂತರ ಬೆಲೆಬಾಳುವ ಸಾರ್ವಜನಿಕ ವಸ್ತು ನಾಶ ಮಾಡುವ ಮನೋವೃತ್ತಿ ಬೆಳೆಯುತ್ತಿದೆ. ಹೀಗೆ ಕೋಟಿಗಟ್ಟಲೆ ರೂಪಾಯಿ ಬೆಲೆಬಾಳುವ ಸಾರ್ವಜನಿಕ ವಸ್ತುಗಳ ನಷ್ಟದ ವರದಿ ಆಗಾಗ ಪತ್ರಿಕೆಗಳಲ್ಲಿ ಬರುತ್ತಲಿದೆ. ಒಮ್ಮೆ ಬಂಗಾಳದಲ್ಲಿ ರಾಜಕೀಯ ಕಾರಣಗಳಿಂದ ವಿದ್ಯುತ್ ಇಲಾಖೆಯಲ್ಲಿ ಅನೇಕ ಅಸಮರ್ಥರನ್ನು ಸೇರಿಸಿಕೊಂಡಿದ್ದುದರ ಪರಿಣಾಮವಾಗಿ, ಅಶಿಸ್ತು, ಅದಕ್ಷತೆ ವೃದ್ಧಿಯಾಯಿತು. ಕಾಲ ಕಾಲದಲ್ಲಿ ಯಂತ್ರಗಳನ್ನು ಸರಿಪಡಿಸದೇ ಇದ್ದುದರಿಂದ ವಿದ್ಯುತ್ತಿನ ಉತ್ಪನ್ನ ವಿಪರೀತ ಕುಸಿದು, ಕಾರ್ಖಾನೆಗಳ ಉತ್ಪತ್ತಿ ವರ್ಷಕ್ಕೆ ಸಾವಿರಕೋಟಿ ರೂಪಾಯಿಗಳಷ್ಟು ಕಡಿಮೆಯಾಯಿತು! ಸಮರ್ಥತೆ, ದಕ್ಷತೆಗಳಿಗೆ ಬೆಲೆ ನೀಡದೆ, ರಾಜಕೀಯ ಕಾರಣಗಳಿಗಾಗಿ ಉದ್ಯೋಗಕ್ಕೆ ಜನರನ್ನು ಸೇರಿಸಿಕೊಳ್ಳುವ ಧೋರಣೆ ಸರ್ವತ್ರ ಕಂಡುಬರುತ್ತಲಿದೆ. ಎಲ್ಲೆಲ್ಲೂ ಲಂಚಕೋರತನ, ಕಾಳಸಂತೆಯ ವಿಧಾನ ತಾಂಡವವಾಡುತ್ತಿರುವುದು ಎಲ್ಲರಿಗೂ ತಿಳಿದ ವಿಷಯ. ಕಾಳಸಂತೆಯ ಪ್ರವರ್ತಕರನ್ನು ಸರಕಾರ ಮುಟ್ಟುಗೋಲು ಹಾಕುವ ಯೋಜನೆಯ ವಾರ್ತೆ ಕೇಳಿ ಬಂದಿದ್ದರೂ, ಲಂಚಕೋರತನವನ್ನು ಸ್ಥಗಿತಗೊಳಿಸಲು ವ್ಯಾಪಕ ಕ್ರಮ ಕೈಗೊಂಡದ್ದಿಲ್ಲ. ಸರ್ವತ್ರ ವ್ಯಾಪಿಯಾಗಿರುವ ಈ ಪಿಡುಗನ್ನು ತಡೆಯಲು ಯತ್ನಿಸದಿರುವುದು ನೈತಿಕ ಅವನತಿಯ ಹೆಗ್ಗುರುತಲ್ಲವೇ? ಇನ್ನು ಮೂವತ್ತೈದು ವರ್ಷಗಳಲ್ಲಿ ದೇಶದ ಜನಸಂಖ್ಯೆ ಇಮ್ಮಡಿಯಾಗಲಿದೆ ಎಂದು ತಜ್ಞರು ಹೇಳುತ್ತಿದ್ದಾರೆ. ಆಗ ಜೀವನದ ಹೋರಾಟ ಯಾವ ದಾರುಣ ಮಟ್ಟದ್ದಾಗಬಹು\-ದೆಂಬುದನ್ನು ಎಷ್ಟು ಮಂದಿ ವಿದ್ಯಾವಂತರು ಈ ದೇಶದಲ್ಲಿ ಯೋಚಿಸಬಲ್ಲರು? ಮಹಾತ್ಮಾ ಗಾಂಧೀಜಿ ಸ್ವಾತಂತ್ರ್ಯದ ನಂತರ ರಾಮರಾಜ್ಯದ ಕನಸನ್ನು ಕಂಡರು. ದೇಶದಲ್ಲಿ ಪ್ರಾಮಾಣಿಕ, ಶೀಲವಂತ ಜನರ ಸುಸಂಘಟಿತ ಪ್ರಯತ್ನದಿಂದ ಇದು ಸಾಧ್ಯವಾಗುವುದೆಂದು ಅವರು ನಂಬಿದ್ದರು. ಆದರೆ ಇಂದು ಶೀಲ ಶಿಲಾಧಾರಕ್ಕೆ ಕುಠಾರಾಘಾತವಾಗಿದೆಯಲ್ಲ!

ನೈತಿಕ ಅವನತಿ ಅತ್ಯಂತ ಭಯಾನಕವಾದದ್ದಾದರೂ ಯಾವ ವಿಚಾರವಂತ ದೇಶಪ್ರೇಮಿಗಳೂ, ಧಾರ್ಮಿಕ ಮುಖಂಡರೂ, ರಾಜಕೀಯಸ್ಥರೂ, ‘ಏಕೆ ಹೀಗಾಗಿದೆ?’ ಎಂಬುದನ್ನು ಗಮನಿ\-ಸಿಲ್ಲ\-ವಲ್ಲ! ಇದನ್ನು ಗಂಭೀರವಾಗಿ ಪರಿಶೀಲಿಸಿ ಪರಿಗಣನೆಗೆ ತಂದುಕೊಂಡಿಲ್ಲವಲ್ಲ! ಏನು ಮಾಡ\-ಬಹುದೆಂಬ ಬಗ್ಗೆ ತಜ್ಞರು ಒಂದೆಡೆ ಸಂಘಟಿತರಾಗಿ ಚಿಂತನಮಂಥನ ನಡೆಸಿದ ವರ್ತ ಮಾನವೂ ಇಲ್ಲವಲ್ಲ! ಎಂದಾದರೊಂದು ದಿನ ತನ್ನಿಂದ ತಾನೇ ಎಲ್ಲ ಸ್ಪಾಂಟೇನಿಯಸ್ ಆಗಿ ಸರಿಯಾಗಿ ಬಿಡುತ್ತದೆಯೇ ಹೇಗೆ?

~\\[-1.45cm]


\section*{ವಿವೇಕ ವಾಣಿ}

\vskip -6pt\addsectiontoTOC{ವಿವೇಕ ವಾಣಿ}

ಸ್ವಾಮಿ ವಿವೇಕಾನಂದರು ಕಳೆದ ಶತಮಾನದಲ್ಲೇ ಹೀಗೆಂದು ಎಚ್ಚರಿಕೆ ನೀಡಿದ್ದರು: ‘ಎಲ್ಲ ರಾಜ\-ಕೀಯ ಸಾಮಾಜಿಕ ವ್ಯವಸ್ಥೆ, ವಿಧಾನಗಳೂ, ಮೂಲತಃ ಮನುಷ್ಯನ ಒಳ್ಳೆಯತನವನ್ನು ಹೊಂದಿ\-ಕೊಂಡಿವೆ. ಪಾರ್ಲಿಮೆಂಟು ಒಳ್ಳೆಯ ಶಾಸನಗಳನ್ನು ಜಾರಿಗೆ ತಂದ ಮಾತ್ರದಿಂದಲೇ ರಾಷ್ಟ್ರ ಒಳ್ಳೆಯದಾಗಿಬಿಟ್ಟಿತು ಎನ್ನಲಾಗದು. ಆದರೆ ರಾಷ್ಟ್ರದ ಜನ ಒಳ್ಳೆಯವರಾದರೆ, ದೊಡ್ಡವ\-ರಾದರೆ, ಆ ರಾಷ್ಟ್ರ ಒಳ್ಳೆಯದು ಅಥವಾ ದೊಡ್ಡದು. ಜಗತ್ತಿನ ಎಲ್ಲ ಸಂಪತ್ತಿಗಿಂತಲೂ ಮನುಷ್ಯ ಹೆಚ್ಚು ಮೌಲ್ಯಯುತ.’

‘ಮುಂದೆ ಸಮಾಜವಾದ ಅಥವಾ ಜನರಿಂದಲೇ ಆರಿಸಿದ ಪ್ರತಿನಿಧಿಗಳು ಆಳುವ ಸ್ಥಿತಿ ಬರುವುದು ಎಂಬುದನ್ನು ಸದ್ಯ ಆಗುತ್ತಿರುವ ಬದಲಾವಣೆಯಿಂದ ಊಹಿಸಬಹುದು. ಜನರ ಜೀವ ನೋಪಾಯಕ್ಕೆ ತಕ್ಕ ಅನುಕೂಲಗಳಿರಬೇಕೆಂದು ಆಶಿಸುವುದು ಸಹಜ. ಕಡಿಮೆ ಕೆಲಸ, ದಬ್ಬಾ ಳಿಕೆಯ ಅಭಾವ, ಯುದ್ಧವಿಲ್ಲದಿರುವುದು ಮತ್ತು ಬೇಕಷ್ಟು ಆಹಾರ–ಇವನ್ನು ಅವರು ಆಶಿಸುವುದು ಸ್ವಾಭಾವಿಕ. ಧಾರ್ಮಿಕ ಆದರ್ಶ ಹಾಗೂ ಮನುಷ್ಯನ ಒಳ್ಳೆಯತನವನ್ನವಲಂಬಿಸದೆ ಯಾವುದೇ ನಾಗರಿಕತೆ ದೀರ್ಘಕಾಲ ಉಳಿದಿರಬಲ್ಲುದು ಎಂದು ಯಾರು ತಾನೇ ಭರವಸೆ ನೀಡಬಲ್ಲರು? ದೃಢವಾಗಿ ನಂಬಿ–ಧರ್ಮವು ಸಮಸ್ಯೆಯ ಮೂಲಕ್ಕೇ ಹೋಗುವುದು. ಅದು ಸರಿಯಾಗಿದ್ದರೆ ಎಲ್ಲವೂ ಸರಿಯಾಗಿದ್ದಂತೆ.’

‘ಶಾಸನ, ಸರ್ಕಾರ, ರಾಜಕೀಯ–ಇವುಗಳೆಲ್ಲ ಮಾರ್ಗವೇ ಹೊರತು ಅಂತಿಮ ಗುರಿ ಅಲ್ಲ. ಇವುಗಳಿಗೆ ಅತೀತವಾಗಿ, ಯಾವ ನಿಯಮಗಳ ಆವಶ್ಯಕತೆ ಇರದ ಒಂದು ಗುರಿ ಇದೆ. ಆ ಗುರಿ ಅಥವಾ ತಳಹದಿ ಶಾಸನಗಳಿಂದಾಗತಕ್ಕದ್ದಲ್ಲ. ನೈತಿಕ ನಿಷ್ಠೆ, ಹೃದಯಪರಿಶುದ್ಧಿಯೇ ನಿಜವಾದ ಬಲ ಎಂಬುದನ್ನು ಏಸುಕ್ರಿಸ್ತ ಕಂಡುಕೊಂಡ. ಋಷಿಗಳೂ ಸಾರಿದರು. ಆದುದರಿಂದಲೆ ಧರ್ಮವು ಸಮಸ್ಯೆಯ ಮೂಲಕ್ಕೆ ಹೋಗುವುದು, ಮನುಷ್ಯನ ಚಾರಿತ್ರ್ಯವನ್ನು ರೂಪಿಸುವುದು.’

‘ಹಿಂದೂ ಸಮಾಜವನ್ನು ಮೇಲಕ್ಕೆತ್ತಬೇಕಾದರೆ, ಧರ್ಮವನ್ನು ನಾಶಮಾಡಬೇಕಾಗಿಲ್ಲ ಎಂದು ನಾನು ಸಾರಿ ಹೇಳುತ್ತೇನೆ. ನಮ್ಮ ಸಮಾಜದಲ್ಲಿ ಈಗ ಕಾಣುತ್ತಿರುವ ನ್ಯೂನತೆಗಳಿಗೆ ಧರ್ಮವು ಕಾರಣವಲ್ಲ. ಧಾರ್ಮಿಕ ಭಾವನೆಗಳನ್ನು ಸಾಮಾಜಿಕರಲ್ಲಿ ಸರಿಯಾಗಿ ಹರಡದಿರುವುದೇ ಕಾರಣ. ನಮ್ಮ ಪುರಾತನ ಗ್ರಂಥಗಳ ಆಧಾರದಿಂದಲೇ ಶಬ್ದಶಃ ಇದನ್ನು ನಾನು ತೋರಿಸಿ ಕೊಡಬಲ್ಲೆ. ನಾವೆಲ್ಲರೂ ಬದುಕಿನ ಉದ್ದಕ್ಕೂ ಕಾರ್ಯರೂಪಕ್ಕೆ ತರಬೇಕಾಗಿರುವ ಈ ಸಂಗತಿಯನ್ನು ನಾನು ಬೋಧಿಸುತ್ತೇನೆ: ನಾವು ಮಾಡಬೇಕಾದ ಮೊದಲನೇ ಕೆಲಸವೇ ನಮ್ಮ ವೇದ, ಉಪನಿಷತ್ತು, ಪುರಾಣಗಳಲ್ಲಿರುವ ಅದ್ಭುತವಾದ ಸತ್ಯಗಳನ್ನು ಸರ್ವತ್ರ ಎಲ್ಲ ಜನರಿಗೂ ತಿಳಿಯ ಪಡಿಸುವುದು. ಭರತಖಂಡದಲ್ಲಿ ವಿಕಾಸದ ಪ್ರತಿಯೊಂದು ಮೆಟ್ಟಲನ್ನು ಹತ್ತಿ ಮೇಲೇರಬೇಕಾದರೆ ಧಾರ್ಮಿಕ ಭಾವನೆಗಳ ಪುನರುತ್ಥಾನವಾಗಬೇಕು. ಸಮಾಜವಾದ ಅಥವಾ ರಾಜಕೀಯ ಭಾವನೆಗಳ ಪ್ರವಾಹದಿಂದ ದೇಶವನ್ನು ತುಂಬುವುದಕ್ಕೆ ಮೊದಲು ಆಧ್ಯಾತ್ಮಿಕ ಭಾವನೆಗಳ ಪ್ರವಾಹವನ್ನು ಹರಿಯಿಸಬೇಕು. ದಾನಕ್ಕೆ ಹೆಸರಾಂತ ಈ ದೇಶದಲ್ಲಿ ಮೊದಲನೆಯದಾಗಿ ಆಧ್ಯಾತ್ಮಿಕ ದಾನವನ್ನು\break ಕೈಗೊಳ್ಳೋಣ.’

‘ಚೆದುರಿ ಹೋಗಿರುವ ಆಧ್ಯಾತ್ಮಿಕ ಶಕ್ತಿಯನ್ನು ಸಂಗ್ರಹಿಸುವುದೇ ಭರತ ಖಂಡದ ರಾಷ್ಟ್ರೀಯ ಏಕತೆಯ ರಹಸ್ಯ. ನಮ್ಮ ಆತ್ಮೋದ್ಧಾರ ಆಗಬೇಕೆಂದಿದ್ದರೆ ನಮ್ಮನಮ್ಮಲ್ಲೇ ಕಾದಾಡುವುದನ್ನು ನಾವು ಮರೆಯಬೇಕಾಗಿದೆ.’ ‘ಜನಸಮೂಹದ ಧಾರ್ಮಿಕ ಭಾವನೆಗಳಿಗೆ ಹಿಂಸೆ ಮಾಡದೆ ಅವರನ್ನು ಮೇಲಕ್ಕೊಯ್ಯಬೇಕು.’

\newpage

ಒಟ್ಟಿನಲ್ಲಿ, ಜನಮನದಲ್ಲಿ ಧಾರ್ಮಿಕ ಜಾಗೃತಿ ಉಂಟಾದರೆ ರಕ್ತಶುದ್ಧಿಯಾದಂತೆ; ಬೇರಿಗೆ ತಗುಲಿದ ರೋಗ ನಿವಾರಣೆಯಾದಂತೆ. ವಿದ್ಯಾವಂತರು, ತಿಳಿದವರು, ಕಷ್ಟಪಟ್ಟು ಧಾರ್ಮಿಕ ವಿಚಾರಗಳನ್ನು ತಾವು ಸರಿಯಾಗಿ ತಿಳಿದುಕೊಂಡು ಆಚರಿಸಿ ಜನರಿಗೆ ತಿಳಿಸಬೇಕು. ಮನುಷ್ಯನ ಹೃದಯಶುದ್ಧಿಯಾಗದೆ ಯಾವ ಪಕ್ಷ ಗೆದ್ದರೂ ಸ್ವಾರ್ಥದ ಗೆದ್ದಲು ಹೆಚ್ಚುತ್ತದೆ. ಪರಿಶುದ್ಧ ಹೃದಯವುಳ್ಳ ವ್ಯಕ್ತಿಗೆ ಸಮನಾರು? ಒಂದೊಂದು ಇಟ್ಟಿಗೆಯೂ ಸರಿಯಾಗಿದ್ದರೆ ಒಂದು ಕಟ್ಟಡ ಉತ್ತಮ. ಒಂದೊಂದೇ ಹನಿ ಕೂಡಿ ಹಳ್ಳ. ಒಂದೊಂದೇ ಧಾನ್ಯದ ಕಾಳು ಸೇರಿ ಒಂದು ದೊಡ್ಡ ಧಾನ್ಯದ ರಾಶಿ. ಅಂತೆಯೇ, ವ್ಯಕ್ತಿಯ ಶೀಲೋತ್ಕರ್ಷದಿಂದ ಸಮಾಜದ ಸುಸ್ಥಿತಿ, ವ್ಯಕ್ತಿಯ ಈ ಶೀಲೋತ್ಕರ್ಷಕ್ಕೆ ತಳಹದಿಯೇ ಎಲ್ಲರ ಶ್ರದ್ಧಾಕೇಂದ್ರವಾದ–ಧರ್ಮ, ದೇವರುಗಳಲ್ಲಿನ ದೃಢ ವಿಶ್ವಾಸ.


\section*{ಶ್ರದ್ಧೆಯ ಪ್ರತಿಷ್ಠಾಪನೆ}

\addsectiontoTOC{ಶ್ರದ್ಧೆಯ ಪ್ರತಿಷ್ಠಾಪನೆ}

ನಮ್ಮ ದೇಶದಲ್ಲಿ ಇಂದು ವಿದ್ಯಾವಂತರು ಅಧ್ಯಾತ್ಮ, ಧರ್ಮ, ದೇವರೆಂದರೆ ಆಸಕ್ತರಾಗುವ ಲಕ್ಷಣವಿಲ್ಲ. ಧರ್ಮದ ದುರ್ಬಲ ಅನುಯಾಯಿಗಳೇ ಒಂದು ರೀತಿಯಿಂದ ಧರ್ಮಗ್ಲಾನಿಗೆ ಕಾರಣರು ಎಂಬುದೇನೋ ಸತ್ಯ. ಇಂದು ಧರ್ಮದ ನೆಲೆಗಟ್ಟು ಶಿಥಿಲವಾಗುತ್ತಿದೆ ಎಂದು ವಿಚಾರವಂತ ಧಾರ್ಮಿಕ ಜನ ಹೇಳುತ್ತಿದ್ದಾರೆ. ಧಾರ್ಮಿಕ ಶ್ರದ್ಧೆಯ ಅಭಾವವೇ ಸಮಾಜದಲ್ಲಿ ನಾವಿಂದು ಕಾಣುವ ನೈತಿಕ ಅಧಃಪತನಕ್ಕೆ ಕಾರಣವೆಂಬುದು ಅವರ ಅಂಬೋಣ. ಧರ್ಮದ ನೆಲೆಗಟ್ಟು ಶಿಥಿಲವಾಗುತ್ತಿದೆ ಎಂದರೆ ಏನರ್ಥ? ಧರ್ಮದ ನಿಯಮಗಳು ದುರ್ಬಲವಾದುವೆಂದರ್ಥವಲ್ಲ. ನಿಯಮಗಳನ್ನು ಪಾಲಿಸುವ ಮನುಷ್ಯರು ತಮ್ಮ ದುರ್ಬಲತೆಗಳಿಂದ ಈ ನಿಯಮಗಳನ್ನು ಅನುಸರಿಸದೆ, ಆ ನಿಯಮಗಳಲ್ಲಿ ವಿಶ್ವಾಸವನ್ನು ಕಳೆದುಕೊಂಡಿದ್ದಾರೆ ಅಷ್ಟೆ. ಆದರೆ ಆ ನಿಯಮಗಳ ಅಸ್ತಿತ್ವಕ್ಕೆ ಬಾಧಕವಿಲ್ಲ. ಗುರುತ್ವಾಕರ್ಷಣ ನಿಯಮ ನ್ಯೂಟನ್ ಕಂಡು ಹಿಡಿಯುವುದಕ್ಕೆ ಮೊದಲು ಇತ್ತು. ನಾವೆಲ್ಲರೂ ಮರೆತುಬಿಟ್ಟರೂ ಇದ್ದೇ ಇರುತ್ತದೆ. ಆದರೆ ನಿಯಮವನ್ನು ತಿಳಿದುಕೊಂಡಲ್ಲಿ ಪ್ರಕೃತಿಯನ್ನು ನಿಯಂತ್ರಿಸಿ, ಅದನ್ನು ನಮ್ಮ ಸುಖ ಸೌಕರ್ಯಗಳ ಸಾಧಕಗಳನ್ನಾಗಿ ಮಾಡಿಕೊಳ್ಳಬಹುದು. ಅಂತೆಯೇ ಧರ್ಮ ನಿಯಮಗಳು ಶಾಶ್ವತ ಸತ್ಯದ ಹಿನ್ನೆಲೆಯನ್ನು ಆಧರಿಸಿವೆ. ನಾನಾ ಕಾರಣಗಳಿಂದ ಜನಾಂಗ ಆ ನಿಯಮಗಳಲ್ಲಿ ವಿಶ್ವಾಸವನ್ನು ಕಳೆದುಕೊಳ್ಳಬಹುದು. ಕ್ಷಣಿಕವಾದ ಮೋಹಕ ಪ್ರಭಾವಗಳಿಗೆ ಬಲಿಬಿದ್ದು ಶಾಶ್ವತವಾದ ಫಲಗಳ ನ್ನೀಯುವ ಧರ್ಮಾನುಷ್ಠಾನದಲ್ಲಿ ಆಸಕ್ತಿಯನ್ನು ಕಳೆದುಕೊಳ್ಳಬಹುದು. ಅದೇ ಧರ್ಮಗ್ಲಾನಿಯ ಮೂಲ. ಧರ್ಮ ನಿಯಮಗಳನ್ನು ಅನುಸರಿಸದೆ ಇರುವುದರಿಂದ ಧರ್ಮ ತತ್ತ್ವಕ್ಕೆ ಹಾನಿ ಇಲ್ಲ. ಆರೋಗ್ಯ ರಕ್ಷಣೆಯ ನಿಯಮಗಳನ್ನು ಪಾಲಿಸದಿದ್ದರೆ ಯಾರಿಗೆ ಹಾನಿ? ನಿಯಮ ಪಾಲಿಸದ ವ್ಯಕ್ತಿಗೆ ಅಲ್ಲವೇ? ಈ ಹಾನಿಯನ್ನು ದೂರಮಾಡುವ ವಿಧಾನವನ್ನು ಭಗವಾನ್ ಶ‍್ರೀಕೃಷ್ಣ ಅರ್ಜುನನ ಮೂಲಕ ಈ ಜಗತ್ತಿಗೆ ಸಾರಿದ್ದಾನೆ:

‘ಈ ನಾಶರಹಿತವಾದ ಫಲವನ್ನು ತಂದುಕೊಂಡುವ ಧರ್ಮದ ರಹಸ್ಯ ನಿಯಮವನ್ನು ನಾನು ಹಿಂದೆ ಸೂರ್ಯನಿಗೆ ಅರುಹಿದ್ದೆ. ಅವನು ಅದನ್ನು ಮನುವಿಗೆ ಹೇಳಿದ್ದ. ಮನುವು ಇಕ್ಷ್ವಾಕುವಿಗೆ ತಿಳಿಸಿದ್ದ. ಹೀಗೆ ಶ್ರೇಷ್ಠವಾದ ಈ ತತ್ತ್ವವು ಪರಂಪರೆಯಾಗಿ ರಾಜರ್ಷಿಗಳಿಗೆ ತಿಳಿದಿತ್ತು. ಆದರೆ ಅರ್ಜುನ, ಕಾಲಗತಿಯಿಂದ ಆ ಪರಂಪರೆ ವಿಚ್ಛೇದವಾಗಿ ಈ ಯೋಗವು ಕಣ್ಮರೆಯಾಯಿತು.’ ಎಂದರೆ ದುರ್ಬಲರೂ, ಇಂದ್ರಿಯ ನಿಗ್ರಹವಿಲ್ಲದವರೂ ಆದ ಜನರ ಕೈಯಲ್ಲಿ ಸಿಕ್ಕಿ ಈ ಯೋಗವು ನಷ್ಟವಾಯಿತು. ನಿಜವಾಗಿಯೂ ಯೋಗವು ನಷ್ಟವಾಗಲಿಲ್ಲ. ಯೋಗವನ್ನು ತಿಳಿದು ಕೊಂಡಿದ್ದೇವೆ ಎನ್ನುವ ಜನ ಭ್ರಷ್ಟರಾದರು, ದುಷ್ಟರಾದರು. ತಮ್ಮ ಕ್ಷುದ್ರ ಸ್ವಾರ್ಥಸಾಧನೆಗೆ ಆ ಯೋಗವನ್ನು ಬಳಸಿ ಅದಕ್ಕೆ ಕೆಟ್ಟಹೆಸರನ್ನು ತಂದರು. ಅದನ್ನು ನಷ್ಟಪ್ರಾಯವನ್ನಾಗಿ ಮಾಡಿದರು ಎಂದರ್ಥ. ಆಗ ಸ್ವಾಭಾವಿಕವಾಗಿ ಜನರು ಆ ಯೋಗದಲ್ಲಿ ಶ್ರದ್ಧೆಯನ್ನು ಕಳೆದುಕೊಂಡರು. ಈಗ ಧರ್ಮದ ಪುನರುತ್ಥಾನ ಎಂದರೆ ಧರ್ಮದ ನಿಯಮಗಳನ್ನು ನಿಷ್ಠೆಯಿಂದ ಆಚರಿಸುವಂತೆ ಪ್ರೇರಿಸುವ ಶ್ರದ್ಧೆಯ ಪುನಃಪ್ರತಿಷ್ಠಾಪನೆಯೇ ಆಗಬೇಕಾಗಿದೆ. ಈ ಶ್ರದ್ಧೆಯ ಪ್ರತಿಷ್ಠಾಪನೆಯ ಕಾರ್ಯವನ್ನು ಸಂತರು, ಮಹಾತ್ಮರು, ಆಚಾರ್ಯಪುರುಷರು, ಅವತಾರ ಪುರುಷರು ಮಾಡುತ್ತ ಬಂದಿದ್ದಾರೆ. ಭಾರತದಲ್ಲಂತೂ ಪ್ರತಿಯೊಂದು ಸಾಮಾಜಿಕ, ರಾಜಕೀಯ ಅಭ್ಯುದಯದ ಹಿನ್ನೆಲೆಯಲ್ಲೂ, ಧಾರ್ಮಿಕ ಮಹಾಪುರುಷರ ಮಾರ್ಗದರ್ಶನದ ಸ್ಫೂರ್ತಿ ಐತಿಹಾಸಿಕ ಸತ್ಯ. ಆದರೆ ಇಂದು ವಿದ್ಯಾವಂತರೆನಿಸಿಕೊಂಡವರೂ, ಅಧಿಕಾರ ಸ್ಥಾನದಲ್ಲಿದ್ದವರೂ ಇದನ್ನು ಮರೆತುದು ದುರ್ದೈವ.

\medskip


\section*{ಭಾರತದ ಆದರ್ಶ}

\addsectiontoTOC{ಭಾರತದ ಆದರ್ಶ}

ಅತ್ಯಂತ ಪ್ರಾಚೀನ ಕಾಲದಿಂದ ಭಾರತವು ಈಶ್ವರ, ಆತ್ಮ ಮೊದಲಾದ ಇಂದ್ರಿಯಾತೀತ ವಸ್ತುಗಳು ಧ್ರುವಸತ್ಯವೆಂದು ನಂಬಿ ಅವುಗಳನ್ನು ಪ್ರತ್ಯಕ್ಷೀಕರಿಸಲು ತನ್ನ ಸರ್ವಸ್ವವನ್ನೂ ವಿನಿಯೋಗಿಸಿದೆ. ಈ ರೀತಿಯ ಸಾಕ್ಷಾತ್ಕಾರ ಅಥವಾ ಅನುಭವವೇ ಪ್ರತಿಯೊಬ್ಬ ವ್ಯಕ್ತಿಯ ಹಾಗೂ ರಾಷ್ಟ್ರೀಯ ಧ್ಯೇಯದ ಚರಮಸೀಮೆ ಎಂಬ ಸಿದ್ಧಾಂತವನ್ನು ಮಾಡಿತ್ತು. ಇಂದ್ರಿಯಾತೀತ ವಿಷಯಗಳಲ್ಲಿ ಈ ತೆರನಾದ ವಿಶೇಷ ಅನುರಾಗಕ್ಕೆ ಕಾರಣವೇನು? ದೈವೀಗುಣಗಳಿಂದ ಶೋಭಿ ಸುವ ಅಧ್ಯಾತ್ಮ ಜ್ಞಾನಸಂಪನ್ನರಾದ ಮಹಾಪುರುಷರು ಆಗಾಗ್ಗೆ ಭಾರತದಲ್ಲಿ ಉದಿಸಿದುದೇ ಕಾರಣ. ಪಶ್ಚಿಮ ದೇಶಗಳನ್ನು ವಿಜ್ಞಾನದ ತವರುಮನೆ ಎಂದು ಕರೆಯಬಹುದಾದಂತೆ, ಭಾರತವನ್ನು ಧರ್ಮ ಹಾಗೂ ಅಧ್ಯಾತ್ಮದ ತವರುಮನೆ ಎಂದು ಕರೆಯುವುದರಲ್ಲಿ ಅತಿಶಯೋಕ್ತಿ ಇಲ್ಲ. ಭಾರತದಲ್ಲಿ ಆಧ್ಯಾತ್ಮಿಕ ಜೀವನದ ಭದ್ರ ಬುನಾದಿಯ ಮೇಲೆ ಪ್ರತ್ಯಕ್ಷ ಧರ್ಮಲಾಭ ಅಥವಾ ಸಾಕ್ಷಾತ್ಕಾರವೆಂಬ ಗುರಿಯನ್ನು ಇಟ್ಟುಕೊಂಡು ಸಾಮಾಜಿಕ ರೀತಿ ನಿಯಮಗಳನ್ನು ಸೃಷ್ಟಿಸಿದ್ದರು. ಸಮಾಜದ ಪ್ರತಿಯೊಬ್ಬ ವ್ಯಕ್ತಿಯೂ ತನ್ನ ಸ್ವಭಾವಸಿದ್ಧವಾದ ಗುಣವನ್ನು ಅವಲಂಬಿಸಿ, ತನ್ನ ಕರ್ತವ್ಯವನ್ನು ನಿಷ್ಠೆಯಿಂದ ನೆರವೇರಿಸಿ, ಧರ್ಮಲಾಭ ಅಥವಾ ಭಗವತ್ ಸಾಕ್ಷಾತ್ಕಾರ ಪಡೆಯ ಬಹುದು ಎಂದು ಅವರು ದೃಢವಾಗಿ ನಂಬಿದ್ದರು. ಈ ಮಹಾನ್ ಆದರ್ಶಗಳು ಕೇವಲ ಪುಸ್ತಕದಲ್ಲಿ ಉಳಿಯದೇ ಕಾರ್ಯರೂಪಕ್ಕೆ ಬಂದ ಉದಾಹರಣೆಗಳು ಇತಿಹಾಸದುದ್ದಕ್ಕೂ ಲಭ್ಯ.


\section*{ಸತ್ವ ಪರೀಕ್ಷೆ}

\addsectiontoTOC{ಸತ್ವ ಪರೀಕ್ಷೆ}

‘ನಮ್ಮಲ್ಲಿ ಒಬ್ಬರನ್ನೊಬ್ಬರು ದ್ವೇಷಿಸಲು ಬೇಕಷ್ಟು ಧರ್ಮಗಳಿವೆ; ಆದರೆ ಪ್ರೀತಿಸಲು ಇಲ್ಲ’ ಎನ್ನುವುದು ಪಾಶ್ಚಾತ್ಯ ದಾರ್ಶನಿಕನೊಬ್ಬನ ಉದ್ಗಾರ. ಅಂದರೆ ಧರ್ಮದ ಹೆಸರಿನಲ್ಲಿ ನಾವು ಕಚ್ಚಾಡುತ್ತಿದ್ದೇವೆ ಎಂದರ್ಥ.

ಎಲ್ಲ ಜೀವಿಗಳಲ್ಲೂ ಜೀವ ತುಂಬಿದ ಆ ಒಂದು ಶಕ್ತಿಯೇ ದೇವರಲ್ಲವೇ? ಶ‍್ರೀರಾಮಕೃಷ್ಣ ಪರಮಹಂಸರು ಹೇಳುತ್ತಿದ್ದಂತೆ ‘ಕೊಳದ ನೀರನ್ನು ಒಬ್ಬ “ವಾಟರ್​” ಎಂದರೆ, ಇನ್ನೊಬ್ಬ “ಪಾನೀ” ಎನ್ನುತ್ತಾನೆ. ಅದೇ ರೀತಿ ಆ ಶಕ್ತಿಯನ್ನೇ ಒಂದೊಂದು ಧರ್ಮದವರು ಒಂದೊಂದು ರೀತಿಯಲ್ಲಿ ಕಂಡು ಪೂಜಿಸುತ್ತಾ ಬಂದರಷ್ಟೆ. ಇದನ್ನು ಅರ್ಥಮಾಡಿಕೊಳ್ಳದೆ ಇಂದಿಗೂ “ನಮ್ಮ ದೇವರು ಶ್ರೇಷ್ಠ, ನಿಮ್ಮ ದೇವರು ನಿಕೃಷ್ಟ” ಎಂದು ಹೊಡೆದಾಡಿಕೊಳ್ಳುತ್ತಿದ್ದಾರಲ್ಲ! ಬಹುಶಃ ಮನುಷ್ಯನ ಅಜ್ಞಾನಕ್ಕೆ ಭಗವಂತ ನಗುತ್ತಾನೆ!’

ದೇವರು ಗುರಿಯಾದರೆ, ಧರ್ಮವು ದಾರಿ. ಧರ್ಮಾಚರಣೆಯ ಹಾದಿಯಲ್ಲಿ, ಭಕ್ತನಿಗೆ ಎದುರಾಗುವ ಕಂಟಕಗಳೇ ಚಿಂಕ್ರೋಭಗಳು. ಬಹುಶಃ ಅವು ಭಕ್ತನ ಭಕ್ತಿಭಾವದ ಯಥಾರ್ಥತೆಗೆ ದೇವರೊಡ್ಡುವ ಪರೀಕ್ಷೆ. ನಾಯಿ ಮೈಮೇಲೇರಿ ಬರುವಾಗ, ಮನೆಯೊಡೆಯನನ್ನು ಕೂಗಿ ಕರೆದರೆ, ಹೇಗೆ ಆತನೇ ಬಂದು ನಾಯಿಯನ್ನು ಹಿಂದೆ ಕರೆದು ಎದುರುಗೊಂಡು ಆದರಿಸುವನೋ, ಅದೇ ರೀತಿ ಚಿಂಕ್ರೋಭಗಳ ಪೀಡೆ ಹೆಚ್ಚುತ್ತಿರುವಾಗ ದೇವರ ಮೊರೆ ಹೋದರೆ ಸುಲಭದಲ್ಲಿ ಪಾರಾಗ ಬಹುದು. ಕತ್ತಲೆಯಲ್ಲಿದ್ದು ‘ಕತ್ತಲೆ, ಕತ್ತಲೆ’ ಎಂದು ಗೋಗರೆಯುತ್ತಿದ್ದರೆ ಕತ್ತಲೆ ಓಡುವುದೇ? ಬೆಳಕನ್ನು ತಂದಾಗ ಮಾತ್ರ ಕತ್ತಲು ಮಾಯವಾಗುವುದಲ್ಲವೇ? ಹಾಗೆಯೇ ಚಿಂಕ್ರೋಭದ ಕಾಟದಿಂದ ಕಂಗೆಟ್ಟು ಅದನ್ನೇ ಎಣಿಸಿ ಕೊರಗಿದರೆ ಎಂದಿಗೂ ಅದರ ಜಾಲದಿಂದ ಬಿಡುಗಡೆ ದೊರೆ ಯದು. ಅದರಿಂದ ಬಿಡುಗಡೆ ದೊರೆಯಬೇಕಾದರೆ ವಿವೇಕದ ಬೆಳಕು ತರುವ ಕೆಲಸವಾಗಬೇಕು. ಹೃನ್ಮನಗಳು ಬೆಳಗಬೇಕು.


\section*{ಚಿಂಕ್ರೋಭ ಅಳಿಯಲಿ, ಹೃನ್ಮನ ಅರಳಲಿ!}

\addsectiontoTOC{ಚಿಂಕ್ರೋಭ ಅಳಿಯಲಿ, ಹೃನ್ಮನ ಅರಳಲಿ!}

ಹೃದಯ ಮನಸ್ಸುಗಳನ್ನು ಒಮ್ಮಿಂದೊಮ್ಮೆಗೆ ತರಿದು ತಿಂದು ನಾಶಮಾಡುವ ಮಾರಕಶಕ್ತಿ\break ಚಿಂಕ್ರೋಭಕ್ಕಿದೆ. ಗಂಧದ ಮರದ ಬೇರಿಗೆ ಗೆದ್ದಲು ಹಿಡಿದಂತೆ, ಚಿಂಕ್ರೋಭದ ಪಾಶಕ್ಕೊಳಗಾದ ವ್ಯಕ್ತಿ ಪೂರ್ವಾಪರ ವಿವೇಕ ಹಾಗೂ ವಿಮರ್ಶಾ ಸಾಮರ್ಥ್ಯಗಳನ್ನು ಕಳೆದುಕೊಂಡು ಸರ್ವನಾಶ ದತ್ತ ಮುನ್ನುಗ್ಗುತ್ತಾನೆ. ತನ್ನಲ್ಲಿರುವ ಅಪಾರಶಕ್ತಿಯನ್ನು ಆಗ ಆತ ಮರೆತು, ತಾನು ‘ಜಡ, ಅಶಕ್ತ, ನಿರ್ವೀರ್ಯ’ ಎಂದೆಲ್ಲ ಭ್ರಮಿಸಿ, ಆಲಸ್ಯವನ್ನೇ ಆಹ್ವಾನಿಸುತ್ತಾ ವಿಷಯವಸ್ತುಗಳ \hbox\bgroup ನೆರಳಿನಲ್ಲೇ\egroup\break ಬಿದ್ದು, ಇಂದ್ರಿಯಗಳ ಗುಲಾಮನಾಗುತ್ತಾನೆ, ರಾಗದ್ವೇಷಗಳ ತವರಾಗುತ್ತಾನೆ, ಹಿಂಸೆ ಕ್ರೌರ್ಯ ಕೊಲೆ ಸುಲಿಗೆಗಳ ಕೇಂದ್ರವಾಗುತ್ತಾನೆ, ಸಿಡಿದೆದ್ದು ಧ್ವಂಸ ಮಾಡುವ ಭಸ್ಮಾಸುರನೇ ಆಗುತ್ತಾನೆ. ಧರ್ಮಭೂಮಿ ಭಾರತದಲ್ಲೂ ಇಂದು ದೇವರು ಧರ್ಮಗಳ ಹೆಸರಿನಲ್ಲಿ ರಕ್ತದ ಹೊಳೆ ಹರಿಯು\-ತ್ತಿದ್ದರೆ ಅದಕ್ಕೆ ಆಧುನಿಕತೆಯ ಸೋಗಿನಲ್ಲಿ ಚಿಂಕ್ರೋಭಕ್ಕೆ ಬಲಿಬಿದ್ದವರ ಸಂಕುಚಿತತೆ, ಸ್ವಾರ್ಥತೆ, ಮತಿಹೀನತೆಗಳೇ ಕಾರಣ.

ನಾಗರಿಕತೆಯ ಓಟ ಹಿಮ್ಮುಖವಾಗಿ ಅವನತಿಯತ್ತ ಧಾವಿಸದಿರಬೇಕಾದರೆ, ಜನರಲ್ಲಿ\break ಸೋದರತೆ, ಉದಾರತೆ, ನಿಃಸ್ವಾರ್ಥತೆಗಳು ಹೆಚ್ಚಿ ಅವರ ಹೃನ್ಮನಗಳು ಬೆಳಗಬೇಕು. ಆಗ ಚಿಂಕ್ರೋಭದ ಸಂಕೋಲೆ ಸಡಿಲಾಗಿ ಸರಿದು, ಹೃನ್ಮನಗಳು ಅರಳಿ ಸ್ನೇಹ ಸೌಹಾರ್ದತೆಗಳ, ತುಷ್ಟಿ ಪುಷ್ಟಿಗಳ, ಶಾಂತಿಸುಭಿಕ್ಷೆಗಳ ಸೌರಭ ಸೂಸತೊಡಗುತ್ತವೆ. ಸಮಾಜದ ಹಿತಕ್ಕೂ, ಸುಖಕ್ಕೂ ಈಗ ಬೇಕಾದುದು ಅದೇ.

\chapterend

\addtocontents{toc}{\protect\par\egroup}

