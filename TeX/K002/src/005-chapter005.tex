
\chapter{ಅದ್ಭುತ(ಘಟನೆ)ಗಳು ಸಾರುವ ಸತ್ಯ}

\indentsecionsintoc

\begin{itemize}
\itemsep=4pt
\item ಮೊದಲು ಜನನವನರಿಯೆ\\ಮರಣದ ಹದನ ಕಡೆಯಲಿ ತಿಳಿಯೆ\\ಮಧ್ಯದಲಿ ನಾ ನೆರೆನಿಪುಣನೆಂಬುದು ಬಳಿಕ ನಗೆಗೇಡು!\hfill–ಕನಕದಾಸ

 \item ವಿಜ್ಞಾನವು ಕಲಿಯಲು ಬೇಕಾದಷ್ಟು ವಿಚಾರಗಳು ಅತಿಮಾನಸಕ್ಷೇತ್ರದಲ್ಲಿ ಈಗಲೂ ಇವೆ. ಈವರೆಗಿನ ಅದ್ಭುತಗಳನ್ನೆಲ್ಲ ಪರಾಮರ್ಶಿಸಿ ಅವುಗಳ ಒಳಗುಟ್ಟು ಬಯಲು ಮಾಡಲು ವಿಜ್ಞಾನವಿನ್ನೂ ಹೋರಾಡಬೇಕಷ್ಟೆ.\hfill–ಜೆ. ಜಿ. ಫುಲ್ಲರ್​

 \item ಮನುಷ್ಯನ ದಬ್ಬಾಳಿಕೆಗೆ ಹೆದರಿ ಸತ್ಯವನ್ನೆಂದಿಗೂ ಅವಮಾನಿಸಬೇಡ.

 \general{~\hfill}–ರವೀಂದ್ರನಾಥ ಟಾಗೂರ್​

 \item ನಿನಗೆ ಗೊತ್ತಿಲ್ಲದ ವಿಚಾರವನ್ನು ಗೊತ್ತಿಲ್ಲ ಎಂದು ನಮ್ರನಾಗಿ ಒಪ್ಪಿಕೊ. ಅದೇ ನಿನ್ನ ಜ್ಞಾನಕಿಂಡಿಯನ್ನು ತೆರೆಯಬಲ್ಲದು.~\hfill–ಕನ್​ಫ್ಯೂಶಿಯಸ್​

 \item ಸಾವಿನ ಆಚೆಗೆ ಜೀವನವಿದೆ ಎಂಬೀ ವಿಚಾರ ಕೇವಲ ನನ್ನ ನಂಬಿಕೆ ಅಥವಾ ಒಣ ಅಭಿಮತ ಅಲ್ಲ. ಎಳ್ಳಷ್ಟೂ ಸಂಶಯವಿಲ್ಲದೆ ನಾನೀ ಮಾತನ್ನು ಈಗಲೂ ದೃಢೀಕರಿಸುತ್ತೇನೆ.

 \general{~\hfill}–ಡಾ. ಎಲಿಜಬೆತ್ ಕುಬ್ಲೆರ್ ರೋಸ್

 \item \enginline{Science has much more to learn specially in the area of paranormal which had not been scratched yet.\general{~\hfill}–J. G. Fuller}

 \item \general{\enginline{.... I am convinced that we live in eternity now but like an unborn child we are in deep sleep. Occasionally we have lucid moments when we realise that the world is not what it looks like. When we really wake up we may find eternity dimly spread out in front of us and our past clearly distinguishable behind us.}}

 \general{~\hfill}–Dr. Gustaf Stromberg

 \item \enginline{We are in this world by our actions. Just as we go out with the sum-total of our present actions upon us so do we come into it with the sum-total of our past actions upon us; that which takes us out is the very same thing that brings us in. What brings us in? Our past deeds. What takes us out? Our own deeds here and so on and on we go.\general{~\hfill}–Swami Vivekananda}

\end{itemize}


\section*{ವಿಚಿತ್ರವಾದರೂ ನಿಜ!}

\addsectiontoTOC{ವಿಚಿತ್ರವಾದರೂ ನಿಜ !}

ಸಾಮಾನ್ಯ ದೃಷ್ಟಿಗೆ ಅದ್ಭುತ ಎಂದು ಕಾಣುವುದು ಯಥಾರ್ಥವಾಗಿ ಪರಿಶೀಲಿಸಿ ನೋಡಿದಾಗ ಸತ್ಯವೆಂದೇ ಕಾಣುತ್ತದೆ. ಇದು ಜಗತ್ತಿನಾದ್ಯಂತ ನಡೆದು ಬಂದ ಕತೆಯೇ ಆಗಿದೆ. ಹಸುಗೂಸಿಗೆ ಉರಿಯುವ ಬಲ್ಬೊಂದು ಅದ್ಭುತವಾಗಿ ತೋರಬಹುದು; ಓಡುವ ಬಸ್ಸೊಂದು ಅತ್ಯದ್ಭುತವಾಗಿ ಕಾಣಬಹುದು. ಆದರೆ ಪ್ರಾಯಪ್ರಬುದ್ಧನಿಗೆ ವಿದ್ಯುತ್ತಿನಿಂದಾಗಿ ಬಲ್ಬು ಬೆಳಗುತ್ತದೆ; ಡೀಸೆಲ್ ನಿಂದಾಗಿ ಬಸ್ಸು ಓಡುತ್ತದೆ ಎಂದೆಲ್ಲ ತಿಳಿದಿರುವುದರಿಂದ ಅವೆಲ್ಲ ಅದ್ಭುತವೆನಿಸವು. ಅಂದರೆ ‘ಅದ್ಭುತ’ ಒಂದು ಸಾಪೇಕ್ಷ ಭಾವವೆಂದಾಯ್ತು. ಇಂಥ ಅದ್ಭುತಗಳೆಷ್ಟೋ ನಿತ್ಯಸತ್ಯವಾಗಿ ನಮ್ಮ ಬದುಕಿನಲ್ಲಿ ಹಾಸು ಹೊಕ್ಕಾಗಿವೆ.

ಶ‍್ರೀರಾಮಕೃಷ್ಣ ಪರಮಹಂಸರೆನ್ನುತ್ತಿದ್ದರು–‘ಕುರಿಯಂತೆ ಕೂಗುವ ಒಬ್ಬಾತನನ್ನು ಜನ ಮುತ್ತಿ\-ಹಾಕಿ ಅವನ “ಅದ್ಭುತ” ಸಾಮರ್ಥ್ಯಕ್ಕೆ ತಲೆದೂಗುತ್ತಿದ್ದರಂತೆ. ಆಗ ಒಬ್ಬ ಪ್ರಾಜ್ಞ ಸಾಕ್ಷಾತ್ ಕುರಿಯನ್ನೇ ತಂದು ಕೂಗಿಸಿದನಂತೆ. ಆದರೆ ಜನ ಅದಕ್ಕೆ ಕಿವಿಗೊಡಲಿಲ್ಲ. ಅಂದರೆ ಸತ್ಯವನ್ನು ಕೃತಕತೆಯ ತೆರೆಯ ಮರೆಯಲ್ಲಿ ಜನ ಕಾಣಬಯಸುತ್ತಾರೆಯೆ ಹೊರತು, ಆ ತೆರೆ ಸರಿಸಿ ಸತ್ಯದ ಸಾಕ್ಷಾತ್ಕಾರಕ್ಕೆ ಅಡಿಯಿಡುವುದಿಲ್ಲ.’ ಯಕ್ಷಿಣಿಗಾರನ ಇಂದ್ರಜಾಲಗಳನ್ನು ನೋಡಿದ ಪ್ರೇಕ್ಷಕ ಎಂಥ ‘ಅದ್ಭುತ’ ಎಂದು ಹುಬ್ಬೇರಿಸಬಹುದು. ಆದರೆ ಯಕ್ಷಿಣಿಗಾರನ ಸ್ನೇಹ ಬೆಳೆಸಿ ವಿಚಾರಿಸಲಾಗಿ ತಿಳಿಯುತ್ತದೆ ಅವೆಲ್ಲ ‘ಬರೆ ಮ್ಯಾಜಿಕ್​–ಕಣ್ಣುಕಟ್ಟುವ ವಿದ್ಯೆ’ ಎಂಬುದಾಗಿ! ಆದರೂ ಇಂದ್ರಜಾಲ ನೋಡಬೇಕೆಂಬ ಹಂಬಲ ಹಿಂಗುವುದಿಲ್ಲ. ಆದರೆ ಆಗ ಮಾತ್ರ ‘ಅದ್ಭುತ’ದ ಪೊರೆ ಕಳಚಿದ್ದು, ಅದು ಸಾಮಾನ್ಯವಾಗಿ ಬಿಡುತ್ತದೆ. ಈ ಬದಲಾವಣೆಗೆ ಕಾರಣ ಇಂದ್ರಜಾಲವಲ್ಲ– ಪ್ರೇಕ್ಷಕನ ದೃಷ್ಟಿ ಎಂಬುದನ್ನು ಗಮನಿಸಿ.

ಸಾಗರದ ಆಳಕ್ಕಿಳಿದು ಅರಸಿದಾಗ ಮಾತ್ರ ಮುತ್ತುರತ್ನಗಳು ದೊರೆಯುವಂತೆ, ಜನಮನಕ್ಕೆ ‘ಅದ್ಭುತ’ವೆನಿಸುವ ಸಾವು ಬದುಕಿನ ಈ ಘಟನೆಗಳ ಹಿನ್ನೆಲೆಯ ಆಳ ಅಧ್ಯಯನ ಮಾಡಿದಾಗ ಅಲ್ಲಿ ಅಡಗಿರುವ ಸತ್ಯ ಒಡೆದು ಕಾಣುವುದಂತೂ ದಿಟ. ಅದರಿಂದ ನಮ್ಮ ಬದುಕಿನ ಹಿಂದು ಮುಂದಿನ ರೂಪುರೇಖೆಗಳ ಜಾಡೇ ನಮಗೆ ತಿಳಿಯುತ್ತದೆ.


\section*{ಕಾಣುವ ಮತ್ತು ಕಾಣದ ತಥ್ಯಗಳು}

\addsectiontoTOC{ಕಾಣುವ ಮತ್ತು ಕಾಣದ ತಥ್ಯಗಳು}

ನಾವು ಮನುಷ್ಯರು ಕೇವಲ ಎಲುಬು ರಕ್ತ ಮಾಂಸ ನರವ್ಯೂಹಗಳ ಮುದ್ದೆಗಳೇ? ದೇಹದಲ್ಲಿ ಇದ್ದುಕೊಂಡು ಸಮಸ್ತ ಚಟುವಟಿಕೆಗಳನ್ನು ನಡೆಯಿಸುವ ದೇಹಕ್ಕಿಂತ ಭಿನ್ನವಾದ ಪ್ರಜ್ಞೆಯನ್ನೊಳಗೊಂಡ ಶಕ್ತಿ ಇರುವುದು ದಿಟವೇ? ದೇಹವೆನ್ನುವುದು ಆ ಶಕ್ತಿಯ ಹೊರಕವಚ ಮಾತ್ರವೇ?

ಸತ್ತವನಿಗೂ ಬದುಕಿರುವವನಿಗೂ ವ್ಯತ್ಯಾಸ ಸ್ಪಷ್ಟವಾಗಿ ಕಾಣಿಸಿದರೂ ಜೀವ–ಜೀವನವನ್ನು ಕುರಿತ ನಮ್ಮ ಅರಿವು ಈ ದಿಸೆಯಲ್ಲಿ ಆಳವೂ ತಲಸ್ಪರ್ಶಿಯೂ ಅಲ್ಲ. ನಾವು ಯಾವುದನ್ನು ‘ಜೀವಂತ’ ಎಂದು ನಿರ್ದೇಶಿಸುತ್ತೇವೊ ಅದು ಕೇವಲ ಭೌತದ್ರವ್ಯಗಳ ಮೊತ್ತವಲ್ಲ. ಸಸ್ಯವೇ ಆಗಲಿ ಪ್ರಾಣಿ ಅಥವಾ ಮನುಷ್ಯನೇ ಆಗಲಿ ಚೇತನ, ಜೀವ, ಅಥವಾ ಪ್ರಾಣ ಇದೆ\break ಎಂದಾಗ ಅದು ತನ್ನ ವೈಶಿಷ್ಟ್ಯವನ್ನು ಉಳಿಸಿಕೊಂಡು ಪರಿಸರದಿಂದ ವಾಯು ಹಾಗೂ ಆಹಾರವನ್ನು ಸಂಗ್ರಹಿಸಿ ಬೇಕಾದುದನ್ನು ಜೀರ್ಣಿಸಿಕೊಂಡು ಬೇಡವಾದುದನ್ನು ತ್ಯಜಿಸುವ ಸೂಕ್ಷ್ಮ ಕ್ರಿಯೆಗಳನ್ನು ನಿರಂತರ ನಡೆಯುವಂತೆ ಮಾಡುವುದನ್ನು ಕಾಣುತ್ತೇವೆ. ಜೀವಂತ ಮನುಷ್ಯ ಶರೀರದಲ್ಲಿ ನಡೆಯುವ ಅತ್ಯಂತ ಸಂಕೀರ್ಣ ಕ್ರಿಯೆಗಳು ಅಚ್ಚರಿಗೊಳಿಸುವಷ್ಟು ಒಗ್ಗಟ್ಟಿನಿಂದ ಅರ್ಥ ಪೂರ್ಣವಾಗಿ, ಉದ್ದೇಶಪೂರ್ವಕವಾಗಿ ನಡೆಯುತ್ತವೆ. ಉಸಿರಾಟ, ಹೃದಯದ ಬಡಿತ, ರಕ್ತಪರಿ ಚಲನೆ, ಗ್ರಂಥಿಗಳ ಚಟುವಟಿಕೆ, ಜೀರ್ಣಾಂಗಗಳ ಕಾರ್ಯ, ದೇಹಕ್ಕೆ ಬೇಕಾದುದನ್ನು ಅರಗಿಸಿಕೊಂಡು ಬೇಡವಾದ ವಸ್ತುಗಳನ್ನು ಹೊರಚೆಲ್ಲುವ ಕಾರ್ಯ, ರಿಪೇರಿ ಕೆಲಸ, ಉಷ್ಣದ ಸ್ಥಿತಿ ಸ್ಥಾಪಕತ್ವ–ಇವು ವ್ಯವಸ್ಥಿತವಾಗಿ ನಿರಂತರ ನಡೆಯುತ್ತವೆ. ಜೀವಂತವಾಗಿದ್ದು ಈಗ ಸತ್ತು ಹೋಗಿದೆ ಎಂದು ಕರೆಯಲ್ಪಡುವ ವಸ್ತುವಿನಲ್ಲಿ ವ್ಯವಸ್ಥಿತವಾಗಿ ನಡೆಯುವ ಈ ಚಟುವಟಿಕೆಗಳಿಗೆ ಕಾರಣವಾದ ಶಕ್ತಿ ಇಲ್ಲ ಅಥವಾ ಕಾಣೆಯಾಗಿದೆ–ಕರೆಂಟ್ ಫೇಲಾದಾಗ ರಭಸವಾಗಿ ವೇಗದಿಂದ ತಿರುಗುತ್ತಿರುವ ಫ್ಯಾನ್ ಸ್ತಬ್ಧವಾದಂತೆ. ಕಾಣೆಯಾದ ಆ ಶಕ್ತಿಯ ಸ್ವರೂಪವೇನು? ಎನ್ನುವುದು ಮುಖ್ಯ ಪ್ರಶ್ನೆ. ಮಿಲಿಯಗಟ್ಟಲೆ ಜೀವಕೋಶಗಳು ತಮ್ಮ ಚಟುವಟಿಕೆಗಳನ್ನು ಆ ಶಕ್ತಿಯು ಶರೀರದಲ್ಲಿರುವಾಗ ಅತ್ಯಂತ ದಕ್ಷತೆಯಿಂದ ಪರಮಕರ್ತವ್ಯನಿಷ್ಠೆಯಿಂದ ನೆರವೇರಿಸುತ್ತವೆ. ಆ ಶಕ್ತಿಯ ಅನುಪಸ್ಥಿತಿಯಲ್ಲಿ ಅವು ಸ್ತಬ್ಧ ಮಾತ್ರವಲ್ಲ ದೇಹ ಕೊಳೆಯಲು ತೊಡಗುತ್ತದೆ. ಮನುಷ್ಯನಲ್ಲಿ, ಜೀವಂತವಾಗಿದ್ದಾಗ ಈ ಎಲ್ಲ ಚಟುವಟಿಕೆಗಳು ನಿರಂತರ ನಡೆಯುವುದಷ್ಟೇ ಅಲ್ಲ. ಆತ ಇದನ್ನೆಲ್ಲ ತಿಳಿದುಕೊಳ್ಳಬಲ್ಲ. ಹಿಂದಿನ ಘಟನೆಗಳನ್ನು ನೆನಪಿಸಿಕೊಳ್ಳಬಲ್ಲ. ತನಗೆ ಬೇಕಾದುದನ್ನು ಆಯ್ಕೆ ಮಾಡಬಲ್ಲ, ಭವಿಷ್ಯದರ್ಶನ ಮಾಡಬಲ್ಲ. ಅನಿವಾರ್ಯವಾದ ತನ್ನ ಮರಣವನ್ನೂ ತಿಳಿದುಕೊಳ್ಳಬಲ್ಲ. ಹೀಗೆ ಪ್ರಜ್ಞೆ, ಪ್ರಾಣ, ಮನಸ್ಸು, ಸೂಕ್ಷ್ಮವಾದ ಇಂದ್ರಿಯಗಳಿಂದ ಕೂಡಿದ ಸ್ವನಿರ್ದೇಶನಾ ಶಕ್ತಿಯ ಜೀವತತ್ತ್ವ ಇಲ್ಲವಾದಾಗ ಸಾವು ಎನ್ನುತ್ತೇವೆ. ಅದು ವ್ಯಕ್ತವಾದಾಗ ಅದನ್ನೇ ಜೀವಂತ ಎನ್ನುತ್ತೇವೆ. ದೇಹಕ್ಕಿಂತ ಭಿನ್ನವಾದುದು ಎನ್ನುವ ಸೂಚನೆ ಸ್ಪಷ್ಟವಾಗಿ ಲಭ್ಯವಾದರೂ ಈ ಜೀವತತ್ತ್ವವನ್ನು ವಿಜ್ಞಾನಿಗಳು ಗಮನಕ್ಕೆ ತಂದುಕೊಳ್ಳುವುದಿಲ್ಲ.

ಎಲುಬು ರಕ್ತಮಾಂಸ ನರವ್ಯೂಹಗಳಿಂದ ಕೂಡಿದ ನಮ್ಮ ಈ ಸ್ಥೂಲ ಶರೀರದಲ್ಲಿ ಅತ್ಯಂತ ಸೂಕ್ಷ್ಮವಾದ ಇನ್ನೊಂದು ಶರೀರವಿದೆ ಎಂಬುದನ್ನು ನಮ್ಮ ದೇಶದಲ್ಲಿ ಪುರಾತನರು ತಿಳಿದು\-ಕೊಂಡಿದ್ದರು. ಅದನ್ನು ಅವರು ಸೂಕ್ಷ್ಮಶರೀರ ಎಂದು ಕರೆದರು. ಇದಕ್ಕೆ ಲಿಂಗ ಶರೀರ ಎಂಬ ಹೆಸರೂ ಇದೆ. ನಮ್ಮ ಸ್ಥೂಲದೇಹ ಅಳಿದರೂ ಈ ಸೂಕ್ಷ್ಮದೇಹಕ್ಕೆ ನಾಶವಿಲ್ಲ. ನಮ್ಮ ಸ್ಥೂಲ ದೇಹವೆನ್ನುವ ಅತ್ಯಂತ ಸಂಕೀರ್ಣಯಂತ್ರವನ್ನು ಅತ್ಯದ್ಭುತ ರೀತಿಯಲ್ಲಿ ನಡೆಯಿಸುವ ವಿದ್ಯುಚ್ಛಕ್ತಿಗಿಂತಲೂ ಸೂಕ್ಷ್ಮವಾದ ‘ಕರೆಂಟ್​’ನಂಥ ಶಕ್ತಿವಿಶಿಷ್ಟವಾದ ಶರೀರ ಅದು. ಧ್ಯಾನನಿರತರಾದ ಸಿದ್ಧಯೋಗಿಗಳಿಗೆ ಈ ಸೂಕ್ಷ್ಮದೇಹ ಎನ್ನುವುದು ನಂಬಿಕೆ ಅಥವಾ ಕಲ್ಪನೆಯ ವಿಷಯವಾಗಿರಲಿಲ್ಲ. ಅದು ಅವರ ಅನುಭವಕ್ಕೆ ಗೋಚರವಾದ ವಿಷಯವಾಗಿತ್ತು. ಶ‍್ರೀಸಾಮಾನ್ಯನ ಅರಿವಿಗೂ ಇಂಥ ಸಂಗತಿಗಳು ನಿಲುಕದವುಗಳಲ್ಲ. ದೇಹಾತೀತ ಅನುಭವ, ಮೈಮೇಲೆ ಬರುವುದು ಅಥವಾ ಗತಿಸಿದ ವ್ಯಕ್ತಿಯೊಬ್ಬ ಪ್ರೇತವಾಗಿ ಜೀವಂತ ವ್ಯಕ್ತಿಯಲ್ಲಿ ಆವಾಹಿತನಾಗುವಂಥ ಘಟನೆಗಳು ಎಲ್ಲ ದೇಶಗಳಲ್ಲೂ ಅತ್ಯಂತ ಪ್ರಾಚೀನಕಾಲದಲ್ಲೂ ಆಧುನಿಕ ಯುಗದಲ್ಲೂ ನಡೆದಂಥವುಗಳೇ. ಇವುಗಳಲ್ಲಿ ಮೋಸ, ಭ್ರಮೆ, ಮೂಢನಂಬಿಕೆಗಳಿಗೆ ಸಾಕಷ್ಟು ಆಸ್ಪದವಿದ್ದರೂ ಕೆಲವೊಂದು ನಿಜವಾದ ಘಟನೆಗಳಿರುವುದು ದಿಟ. ಇವುಗಳ ಮೂಲವನ್ನು ಕುರಿತು ಚಿಂತಿಸಿದಾಗ ಮನುಷ್ಯನ ಜೀವನ ದೇಹಕ್ಕೇ ಸೀಮಿತವಲ್ಲ, ದೇಹಾತೀತವಾದುದು ಏನೋ ಇದೆ ಎಂಬುದು ಖಚಿತವೆನಿಸುತ್ತದೆ.

ಮೆದುಳಿಗಿಂತ ಮನಸ್ಸಿಗೆ ಪ್ರತ್ಯೇಕವಾದ ಅಸ್ತಿತ್ವವಿದೆ ಎಂಬುದನ್ನು ಸಾಮಾನ್ಯವಾಗಿ ವಿಜ್ಞಾನವೆಂದು ನಾವು ಕರೆಯುವ ಶಾಸ್ತ್ರಗಳು ಒಪ್ಪುತ್ತಿಲ್ಲ. ಇಂಥ ಘಟನೆಗಳನ್ನು ವೈಜ್ಞಾನಿಕವಾಗಿ ಪರಿಶೀಲಿಸಿ ಸತ್ಯವೇನೆಂದು ತಿಳಿಯುವ ಪ್ರಯತ್ನ ವಿಜ್ಞಾನಿಗಳಿಂದಲೇ ನಡೆದಿದೆ. ಈ ಕೆಳಗೆ ನೀಡಲಾದ ಘಟನೆಗಳು ತಜ್ಞರು ವಿಮರ್ಶಾತ್ಮಕ ದೃಷ್ಟಿಕೋನದಿಂದ, ಜಾಗರೂಕತೆಯಿಂದ ಪರಿಶೀಲಿಸಿ ಕಂಡುಕೊಂಡ ಸತ್ಯಘಟನೆಗಳು. ಅವುಗಳನ್ನು ಓದಿ ಏಕೆ? ಹೇಗೆ? ಏನು? ಎಂಥದು? ಎಂದು ಪ್ರಶ್ನಿಸಿ.


\section*{ಇವನೊಳಗೆ ಅವನು!}

\addsectiontoTOC{ಇವನೊಳಗೆ\break ಅವನು!}

ಇದು ಹಳೆಯ ಕತೆ ಅಲ್ಲ. ಈಚೆಗೆ ನಡೆದ ಘಟನೆ. ಸಹಸ್ರಾರು ಜನ ತಮ್ಮ ಕಣ್ಣಿನಿಂದ ಕಂಡು, ಆತನಿಂದ ಉಪಕಾರವನ್ನು ಹೊಂದಿದರು. ಅವನೊಬ್ಬ ಮಹಾತ್ಮನೂ ಅಲ್ಲ, ಸಿದ್ಧ ಪುರುಷನೂ ಅಲ್ಲ, ಸಾಧುಸಂನ್ಯಾಸಿಯೂ ಅಲ್ಲ. ಬ್ರೆಜಿಲ್ಲಿನ ಹಳ್ಳಿಯ ಬಡರೈತ. ಅವನು ಮಹಾಸರ್ಜನ್ ಆಗಿ ಪರಿವರ್ತನೆಯಾದ ಘಟನೆ ವಿಚಾರವಂತರಿಗೂ, ವಿಜ್ಞಾನಿಗಳಿಗೂ ಒಂದು ಸಮಸ್ಯೆಯೇ ಆಗಿ ಪರಿಣಮಿಸಿತು. ಅವನು ಯಾವ ವೈದ್ಯಕೀಯ ವ್ಯಾಸಂಗ ಮಾಡಿದವನಲ್ಲ. ಯಾವ ತರಬೇತನ್ನೂ ಪಡೆದವನಲ್ಲ. ಅವನು ಉಪಯೋಗಿಸುತ್ತಿದ್ದ ಶಸ್ತ್ರಚಿಕಿತ್ಸೆಯ ಉಪಕರಣಗಳು ಯಾವ ಆಧುನಿಕ ಯಂತ್ರಾಗಾರದಿಂದಲೂ ಬಂದವಲ್ಲ. ಅಡುಗೆ ಮನೆಯಲ್ಲಿ ತರಕಾರಿಯನ್ನು ಕತ್ತರಿಸುವ ಸಾಮಾನ್ಯ ಚಾಕುವಿನಿಂದಲೇ ಆತ ಅಸಂಖ್ಯ ಆಪರೇಶನ್ನುಗಳನ್ನು ಮಾಡಿದ. ಕಣ್ಣಿನ ಕ್ಯಾಟರಾಕ್ಟ್ ಇರಲಿ ಅಥವಾ ಇತರ ಸಣ್ಣ ದೊಡ್ಡ ನೋವಿನ ಭಾಗಗಳೇ ಇರಲಿ, ರೋಗಿಗಳ ಸ್ಮೃತಿಯನ್ನು ತಪ್ಪಿಸದೆ,\break ನೋವನ್ನೂ ಉಂಟುಮಾಡದೆ, ಅವರು ಎಚ್ಚರದಿಂದಿರುವಾಗ, ತನ್ನ ಪುಟ್ಟ ಚಾಕುವಿನಿಂದಲೇ ಚಾಕಚಕ್ಯತೆಯಿಂದ ಶಸ್ತ್ರಚಿಕಿತ್ಸೆಯನ್ನು ನೆರವೇರಿಸುತ್ತಿದ್ದ.

ಪ್ರತಿದಿನವೂ ಸುಮಾರು ಒಂದು ಸಾವಿರದ ಐನೂರು ಮಂದಿ ಚಿಕಿತ್ಸೆಗಾಗಿ ಅವನನ್ನು ಮುತ್ತು\-ತ್ತಿದ್ದರು. ಅವನು ಆಪರೇಷನ್ ಮಾಡುತ್ತಿದ್ದ ಕೋಣೆ ಚಿಕ್ಕದು. ಅಲ್ಲಿರುವುದು ಒಂದು ಮೇಜು ಮತ್ತು ಕುರ್ಚಿ ಮಾತ್ರ. ಅಲ್ಲಿಗೆ ಬ್ರೆಜಿಲ್ಲಿನ ಎಲ್ಲ ಭಾಗಗಳಿಂದಲೂ ಜನರು ಬರುತ್ತಿದ್ದರು–ಈ ಪವಾಡಪುರುಷನಿಂದ ಸಹಾಯ ಪಡೆಯಲು, ರೋಗಮುಕ್ತರಾಗಲು. ಕುರುಡರು ಕಣ್ಣುಗಳನ್ನು ಪಡೆದರು! ಕುಂಟರು ತಮ್ಮ ಊರುಗೋಲನ್ನು ದೂರಕ್ಕೆಸೆದು ಸಹಜವಾಗಿ ನಡೆದರು! ತಜ್ಞರಿಂದಲೂ ತಿಳಿಯಲಾಗದ, ದೇಹದೊಳಗಿನ ಹುಣ್ಣುಗಳನ್ನೂ, ಗಡ್ಡೆಗಳನ್ನೂ, ಗ್ರಂಥಿಗಳನ್ನೂ ಕ್ಷಣಾರ್ಧದಲ್ಲಿ ತನ್ನ ಚಾಕುವಿನಿಂದ ಹೊರತೆಗೆದು ಬಿಡುತ್ತಿದ್ದ. ಜೊತೆಗೆ ಸರಸರನೆ ಔಷಧ ಸೂಚಿಸಿ ಚೀಟಿಯನ್ನೂ ಬರೆದುಕೊಡುತ್ತಿದ್ದ. ಆತ ಹಳ್ಳಿಯ ಜನರನ್ನು ಮೋಸಗೊಳಿಸುವ ಮಾಂತ್ರಿಕನಾಗಿರಲಿಲ್ಲ. ರಿಯೋಡಿಜೆನಿರೊ ಪ್ರದೇಶದ ಡಾಕ್ಟರರೂ, ಪೌಲೋ ವಿಶ್ವವಿದ್ಯಾಲಯ ಮತ್ತು ಮೆಡಿಕಲ್ ಎಕೆಡಮಿಯ ವೈದ್ಯರುಗಳೂ ಅವನ ಶಸ್ತ್ರಚಿಕಿತ್ಸೆಯ ವಿಧಾನವನ್ನು ಪ್ರತ್ಯಕ್ಷವಾಗಿ ಪರಿಶೀಲಿಸಿ ಮೂಕವಿಸ್ಮಿತರಾದರು. ಅವನು ಮಾಡಿದ ಎಲ್ಲ ಶಸ್ತ್ರಚಿಕಿತ್ಸೆಗಳೂ ಯಶಸ್ವಿಯಾಗಿರುವುದು ಸತ್ಯ ಎಂದು ಕಂಡುಕೊಂಡರು. ಅಮೇರಿಕದ ಪ್ರಸಿದ್ಧ ಡಾ. ಪುಹಾರಿಚ್ ತಾವೇ ಸ್ವತಃ ಅವನಿಂದ ಶಸ್ತ್ರಚಿಕಿತ್ಸೆಯನ್ನು ಮಾಡಿಸಿಕೊಂಡುದಲ್ಲದೇ, ಫೋಟೋಗಳನ್ನೂ ತೆಗೆದಿರಿಸಿಕೊಂಡು ಅವನ ಸಾಮರ್ಥ್ಯಕ್ಕೆ ಸಾಕ್ಷಿಯಾದರು. ಅವನು ಮಾಡಿದ ನೂರಾರು ಶಸ್ತ್ರಚಿಕಿತ್ಸೆಗಳು ಅವರಿಗೆಲ್ಲ ಬಿಡಿಸ\-ಲಾಗದ ಒಗಟೇ ಆಗಿತ್ತು.

ಆತನ ಹೆಸರು ಜಿ. ಅರ್ಹಿಗೊ. ಆತ ಹೆಚ್ಚೇನೂ ಓದುಬರಹ ಬಾರದ ಬಡ ಹಳ್ಳಿಗ. ಆದರೆ ಇತರರ ದುಃಖಸಂಕಟಗಳನ್ನು ಕಂಡು ಮರುಗುವ ಮನಸ್ಸು ಅವನದು. ಅದೊಂದೇ ಅವನ ಸೇವೆಗೆ ಪ್ರೇರಕಶಕ್ತಿ. ೧೯೫೫ನೇ ಇಸವಿಯಿಂದ ಪ್ರಾರಂಭಿಸಿ ಸಾಯುವವರೆಗೂ ಎಂದರೆ ೧೯೭೧ನೇ ಇಸವಿಯವರೆಗೂ, ದಿನವೂ, ತನ್ನ ಈ ಅದ್ಭುತ ಶಸ್ತ್ರಚಿಕಿತ್ಸೆಯ ಕಾಯಕವನ್ನು ಆತ ಬಿಡದೆ ನೆರವೇರಿಸುತ್ತಿದ್ದ ‘ನಜರೇತಿನ ಏಸು ಸ್ವಾಮಿಯ ಆಧ್ಯಾತ್ಮಿಕ ಕೇಂದ್ರ’ ಎಂಬುದು ಅವನ ಚಿಕಿತ್ಸಾಲಯದ ಹೆಸರು. ಪ್ರತಿದಿನ ಬೆಳಿಗ್ಗೆ 6 ಗಂಟೆಗೆ ಆತ ಅಲ್ಲಿರುತ್ತಿದ್ದ. ಆಗಲೇ ಜನ ಅವನ ಚಿಕಿತ್ಸೆಗೆ ಕಾಯುತ್ತಾ ಸಾಲು ಸಾಲಾಗಿ ನಿಂತಿರುತ್ತಿದ್ದರು. ಆತ ಪ್ರಾರ್ಥನೆ ಸಲ್ಲಿಸಿ ಅಣಿಯಾಗುವಾಗ ಹಳ್ಳಿಗಾಡಿನ ವ್ಯಕ್ತಿಯಾಗಿರುತ್ತಿರಲಿಲ್ಲ. ಡಾ. ಫ್ರಿಟ್ಸ್ ಅವರ ಪೂರ್ಣ ವ್ಯಕ್ತಿತ್ವ ಅವನಲ್ಲಿ ಆವಾಹಿತವಾಗಿರುತ್ತಿತ್ತು. ಆಗ ಅವನು ಜರ್ಮನ್ ಮಿಶ್ರಿತ ಪೋರ್ಚುಗೀಸ್ ಭಾಷೆಯಲ್ಲಿ ಸಂಭಾಷಣೆ ಮಾಡುತ್ತಿದ್ದ. ಉಳಿದ ವೇಳೆಯಲ್ಲಿ ಜರ್ಮನ್ ಭಾಷೆಯನ್ನು ಸ್ವಲ್ಪವೂ ಅರಿಯದ ಆತ ಆಗ ಆ ಭಾಷೆಯಲ್ಲಿ ನಿರಾತಂಕವಾಗಿ ಮಾತನಾಡುತ್ತಿದ್ದ!

ಸರಕಾರದಿಂದ ಸ್ವೀಕೃತವಾದ ಯಾವ ವೈದ್ಯಕೀಯ ಪದವಿ ಅಥವಾ ಅನುಮತಿ ಇಲ್ಲದೆ ರೋಗಿಗಳಿಗೆ ಔಷಧ ನೀಡುವುದನ್ನು ಕ್ರೈಸ್ತ ಸಂಪ್ರದಾಯದ ಧರ್ಮಗುರುಗಳು, ಅದೊಂದು ಕರಾಳ ಮಾಂತ್ರಿಕತೆ ಇರಬೇಕೆಂದು ಯೋಚಿಸಿ ಅವನನ್ನು ವಿರೋಧಿಸಿದರು. ಅವನು ಜೈಲುವಾಸ ಮಾಡಬೇಕಾಯಿತು. ಅನುಮತಿ ಇಲ್ಲದೆ ಜನರಿಗೆ ಶಸ್ತ್ರಚಿಕಿತ್ಸೆ ಮತ್ತು ಔಷಧೋಪಚಾರಗಳನ್ನು ಮಾಡುವುದಲ್ಲದೆ, ಬೇರಾವ ದೋಷವನ್ನೂ ಅವನಲ್ಲಿ ಕಾಣಲಾಗಲಿಲ್ಲ. ಈ ಕಾರ್ಯಕ್ರಮಗಳನ್ನು ನಿಲ್ಲಿಸುವಂತೆ ಅವನನ್ನು ಕೇಳಿಕೊಂಡರು. ಅವನೇನೋ ಅದಕ್ಕೆ ಸದಾ ಸಿದ್ಧನೇ. ಆದರೆ ಜನರ ಸಂಕಟ ಮತ್ತು ಗೋಳನ್ನು ಕಂಡಾಗ, ಅವನು ಪರವಶನಾಗಿ ತನ್ನನ್ನು ತಾನು ನಿಯಂತ್ರಿಸಿಕೊಳ್ಳಲಾರದೆ ಚಿಕಿತ್ಸೆಯನ್ನು ಮಾಡುತ್ತಿದ್ದ. ಪರಿಣಾಮವಾಗಿ ಎರಡನೇ ಬಾರಿ ಜೈಲುವಾಸ ಮಾಡಬೇಕಾಯಿತು. ಅವನಿಂದ ಉಪಕಾರವನ್ನು ಪಡೆದ ಸಹಸ್ರಾರು ಜನರು, ತಮ್ಮನ್ನು ರೋಗಮುಕ್ತರನ್ನಾಗಿ ಮಾಡಿದ, ಸಿದ್ಧಹಸ್ತನಾದ, ಅರ್ಹಿಗೋನ ದುಃಸ್ಥಿತಿಗೆ ಕರಗಿ, ಎಂಟು ಲಕ್ಷದ ನಲ್ವತ್ತು ಸಾವಿರ ಡಾಲರುಗಳ ಒಂದು ನಿಧಿಯನ್ನು ಸಂಗ್ರಹಿಸಿ ಜೈಲಿನಲ್ಲಿರುವ ಅವನಿಗೆ ಕಳುಹಿಸಿದರು. ಆದರೆ ಅವನು ಅದನ್ನು ಹಿಂದಿರುಗಿಸಿದ. ರೋಗಿಗಳಿಂದ ಒಂದು ಚಿಕ್ಕಾಸನ್ನೂ ಅವನು ಬಯಸಿದವ ನಲ್ಲ!

೧೯೭೧ನೇ ಇಸವಿ ಜನವರಿ ೧೧ನೇ ತಾರೀಖು ರಸ್ತೆಯ ಅಪಘಾತವೊಂದರಲ್ಲಿ ಆತ ಅಸು ನೀಗಿದ. ಸಹಸ್ರಸಹಸ್ರ ಸಂಖ್ಯೆಯಲ್ಲಿ ಜನರು ಆತನ ಅಂತಿಮಯಾತ್ರೆಯಲ್ಲಿ ಭಾಗವಹಿಸಿ, ಅಶ್ರುಜಲದಿಂದ ಶ್ರದ್ಧಾರ್ಘ್ಯವನ್ನು ಅರ್ಪಿಸಿದರು. ಅವನನ್ನು ಕುರಿತ, ಎಲ್ಲ ವಿಚಾರಗಳು ಸಚಿತ್ರವಾಗಿ ಅಮೇರಿಕದ ಪ್ರಖ್ಯಾತ ಆಂಗ್ಲ ಮಾಸಪತ್ರಿಕೆ ‘ರೀಡರ್ಸ್ ಡೈಜೆಸ್ಟ್​’(ಮೇ ೧೯೭೫)ನ ಪುಸ್ತಕ ವಿಭಾಗದಲ್ಲಿ ಪ್ರಕಟವಾಗಿತ್ತು. ಎಲ್ಲರೂ ಓದಬೇಕಾದ ವಿಚಾರ ಅದು.

‘ಈ ಅದ್ಭುತ ಶಕ್ತಿಯ ರಹಸ್ಯವೇನು?’ ಎಂದು ಅವನನ್ನೇ ಕೇಳಿದಾಗ, ಆತ ಹೇಳಿದ್ದಿಷ್ಟೆ– ‘ರಾತ್ರಿ ಹೊತ್ತು ವೈದ್ಯಕೀಯ ಉಡುಪನ್ನು ಧರಿಸಿದ ವ್ಯಕ್ತಿಯೊಬ್ಬನು, ಆಗಾಗ ನನಗೆ ಕಾಣಿಸಿಕೊಳ್ಳುತ್ತಿದ್ದ. ಆತ ಇತರ ಕೆಲವು ತಜ್ಞ ಡಾಕ್ಟರರೊಂದಿಗೆ ಚರ್ಚಿಸುತ್ತ, ಅಸಾಧ್ಯವೆಂದು ತೋರುವ ರೋಗಗಳನ್ನು ಗುಣಪಡಿಸಲು ಶಸ್ತ್ರಚಿಕಿತ್ಸೆಗೆ ಸಿದ್ಧನಾಗುತ್ತಿದ್ದ. ಕೊನೆಗೆ ಆ ಡಾಕ್ಟರ್ ತಾನು ಯಾರೆಂಬುದನ್ನು ತಿಳಿಸಿದ: “ನನ್ನ ಹೆಸರು ಡಾಕ್ಟರ್ ಫ್ರಿಟ್ಸ್ ಎಂದು. ಎರಡನೇ ಮಹಾಯುದ್ಧದ ಸಮಯ, ಸರ್ಜನ್ ಆಗಿ ಸೇವೆ ಸಲ್ಲಿಸುತ್ತಿರಬೇಕೆನ್ನುವಾಗಲೇ ಮಡಿದ ನನಗೆ, ಸಂಕಟಗ್ರಸ್ತರಾದ ರೋಗಿಗಳ ಸೇವೆ ಮಾಡಬೇಕೆಂಬ ತೀವ್ರ ಹಂಬಲವಿದೆ. ನನಗೆ ದೇಹವಿಲ್ಲದಿದ್ದರೂ ವ್ಯಕ್ತಿತ್ವ, ಸ್ಮೃತಿ, ದಕ್ಷತೆಗಳನ್ನು ನಾನು ಕಳೆದುಕೊಂಡಿಲ್ಲ. ಹಾಗಾಗಿ ನಿನ್ನ ದೇಹದ ಮೂಲಕ ರೋಗಿಗಳಿಗೆ ಶಸ್ತ್ರಚಿಕಿತ್ಸೆ ಮತ್ತು ಔಷಧ ಸೇವನೆಗೆ ನಿರ್ದೇಶನ ನೀಡುವೆ” ಎಂಬುದಾಗಿ.’

ಡಾ. ಫ್ರಿಟ್ಸ್​ನ ವ್ಯಕ್ತಿತ್ವ ಅರ್ಹಿಗೋನಲ್ಲಿ ಆವಾಹಿತವಾದಾಗ, ತಾನು ಯಾರು, ಏನು ಮಾಡು ತ್ತಿದ್ದೇನೆ, ಎಂಬ ಅರಿವು ಅವನಿಗಿರುತ್ತಿರಲಿಲ್ಲ. ನಡೆ ನುಡಿ, ಆಚಾರ ವಿಚಾರಗಳಲ್ಲಿ ಪೂರ್ಣ ರೀತಿಯ ಬದಲಾವಣೆ ಅರ್ಹಿಗೋನಲ್ಲಿ ಆಗ ಕಂಡು ಬರುತ್ತಿತ್ತು!

ಇದೊಂದು ಕಲ್ಪನೆಯ ಕಗ್ಗವಲ್ಲ, ನಡೆದ ಘಟನೆ. ಸಹಸ್ರಾರು ಜನರು ಪ್ರತ್ಯಕ್ಷ ಕಂಡ ಘಟನೆ, ವಿಜ್ಞಾನಿಗಳನ್ನೂ, ವಿಚಾರವಂತರನ್ನೂ, ಬೆರಗುಗೊಳಿಸಿದ್ದಲ್ಲದೆ, ಚಿಂತನಶೀಲರನ್ನಾಗಿ ಮಾಡಿದ, ಮಾಡುವ ಘಟನೆ. ಮನುಷ್ಯ ದೇಹದ ನಾಶದೊಂದಿಗೆ ಮೃತನ ವ್ಯಕ್ತಿತ್ವ, ಸಾಮರ್ಥ್ಯ, ಸಂಸ್ಕಾರ, ಅಭಿರುಚಿ ಮತ್ತು ದಕ್ಷತೆ–ಇವು ನಾಶವಾಗುವುದಿಲ್ಲವೇ? ಇಂಥ ಘಟನೆಗಳ ಬಗೆಗೆ ತಜ್ಞರ ಅರಿವು, ಅಭಿಪ್ರಾಯಗಳೇನು? ಅವರು ಮಾಡಿದ ಪರಿಶೀಲನೆ ಪರೀಕ್ಷಣೆಗಳೇನು?

ನಿಜವಾಗಿಯೂ ಸಾವಿನೊಂದಿಗೆ ಎಲ್ಲವೂ ನಾಶವೇ? ಅಥವಾ ಸಾವು ಜೀವನದ ಅಂತ್ಯವಲ್ಲ, ಒಂದು ಮಜಲು ಮಾತ್ರವೇ? ಈ ವಿಚಾರವನ್ನು ಕುರಿತ ನಿಜವಾದ ಮತ್ತು ಸ್ಪಷ್ಟವಾದ ತಿಳಿವು ನಮ್ಮ ಬದುಕಿನ ದಿಕ್ಕನ್ನೇ ಬದಲಿಸೀತಲ್ಲವೆ?


\section*{ಸಾವಿನ ಆಚೆಗೆ ಏನು?}

\addsectiontoTOC{ಸಾವಿನ ಆಚೆಗೆ ಏನು?}

ಅತ್ಯಂತ ಪ್ರಾಚೀನ ಕಾಲದಿಂದಲೂ, ಸಾವು ಸಾಧಾರಣ ಮನುಷ್ಯನನ್ನು ಚಿಂತೆಗೀಡು\-ಮಾಡಿ\-ದಂತೆ, ಚಿಂತನಶೀಲನನ್ನಾಗಿಯೂ ಮಾಡಿದೆ. ಹುಟ್ಟು ಸಾವುಗಳ ರಹಸ್ಯ ತಿಳಿಯುವ ಯತ್ನ, ಅನಾದಿಕಾಲದಿಂದಲೂ ಇದೆ. ಸಾಯುವ ಕಾಲದ ಅನುಭವಗಳೇನಿರಬಹುದು? ಸಾಯುವ ಕ್ಷಣ ಜೀವಿಯ ಸ್ಥಿತಿ ಹೇಗಿರಬಹುದು? ಪ್ರತಿಯೊಬ್ಬರ ಪಾಲಿಗೂ ಸಾವು ಅಪಾರ ದುಃಖ, ಸಂಕಟಗಳನ್ನು ತರುವುದೇ? ‘ದೇಹಧಾರಿಯಾದ ವ್ಯಕ್ತಿ’ ಎನ್ನುವ ಭಾರತೀಯರ ಮಾತಿನಲ್ಲಿ ದೇಹವನ್ನು ಧರಿಸಿದ, ದೇಹಕ್ಕೆ ಅತೀತವಾದ ಅಥವಾ ಬೇರೆಯೇ ಆದ, ಸಾವಿಲ್ಲದ ಚೈತನ್ಯ, ಶಕ್ತಿ ಎನಿಸಿದ ಜೀವಾತ್ಮ ಎಂಬ ಅರಿವು ನಮಗಾಗುವುದು. ಇದು ನಿಜವೆ? ‘ಸತ್ತು ಹೋದ’ ಎನ್ನುವ ಮಾತಿನಲ್ಲಿ ‘ಎಲ್ಲಿಗೆ ಹೋದ?’ ಎನ್ನುವ ಪ್ರಶ್ನೆಯೂ ಅಡಕವಾಗಿದೆ. ‘ಅವನ ಪ್ರಾಣಪಕ್ಷಿಯು ದೇಹ ಪಂಜರದಿಂದ ಬಿಡುಗಡೆ ಹೊಂದಿತು’ ಎನ್ನುವಂಥ ಮಾತುಗಳೆಲ್ಲ, ನಮಗೂ ದೇಹಕ್ಕೂ ಇರುವ ಸಂಬಂಧವು, ನಾವು ವಾಸಿಸುವ ಮನೆಗೂ ಅಥವಾ ಧರಿಸುವ ಬಟ್ಟೆಗೂ ಇರುವ ಸಂಬಂಧದಂತೆ ಎನ್ನುವುದನ್ನು ಸೂಚಿಸುತ್ತದೆಯಲ್ಲವೆ? ಅದನ್ನೇ ಶ‍್ರೀರಾಮಕೃಷ್ಣರೆಂದಿದ್ದರು–‘ಸಾವೆಂದರೆ ಒರೆಯಿಂದ ಖಡ್ಗ ತೆಗೆದಂತೆ, ಹೊರಗಡೆಯ ಚೀಲವನ್ನು ಬಿಟ್ಟು ಒಳಗಡೆಯ ಜೀವಾತ್ಮ ಹೊರಟು ಹೋಗುವುದೇ\break ಆಗಿದೆ.’ ಹಾಗೆ ಹೊರಟುಹೋದ ಮೇಲೇನಾಗುತ್ತದೆ?

ಮರಣಾನಂತರದ ಜೀವನ, ಜನ್ಮಾಂತರ ಮತ್ತು ಕರ್ಮವಾದ–ಇವುಗಳ ಅಡಿಪಾಯದ ಮೇಲೆ ಹಿಂದೂ, ಬೌದ್ಧ ಹಾಗೂ ಜೈನ ಧರ್ಮಗಳು ತಮ್ಮ ಸೌಧಗಳನ್ನು ಕಟ್ಟಿವೆ. ಪಶ್ಚಿಮದೇಶದ ಧರ್ಮಗಳಲ್ಲಿ–ಕರ್ಮ, ಜನ್ಮಾಂತರವಾದಗಳನ್ನವು ಒಪ್ಪದಿದ್ದರೂ–ಮರಣಾನಂತರ ಜೀವಾತ್ಮನು ಸ್ವರ್ಗ ಅಥವಾ ನರಕದಲ್ಲಿ ವಾಸಮಾಡುತ್ತಾನೆ ಎಂಬ ಕಲ್ಪನೆಗಳಿವೆ. ಭಾರತೀಯ ತತ್ತ್ವಶಾಸ್ತ್ರದ ಆಧಾರಸ್ತಂಭಗಳಾದ ಕರ್ಮವಾದ ಮತ್ತು ಜನ್ಮಾಂತರ ಸಿದ್ಧಾಂತಗಳ ಬುಡ ನಿಜವಾಗಿಯೂ ಭದ್ರ\-ವಾಗಿದೆಯೇ? ಅದು ಒಂದು ಪುರಾತನ ನಂಬಿಕೆ ಮಾತ್ರವೇ? ಅದು ಸಾರ್ವತ್ರಿಕ ಹಾಗೂ ಸಾರ್ವಕಾಲೀಕವಾದ ಒಂದು ಸೂಕ್ಷ್ಮ ನಿಯಮ ಎಂಬ ಅರಿವು, ಯಥಾರ್ಥ ಆಧುನಿಕ ಅನ್ವೇಷಣೆಗಳಿಂದ ಸಾಬೀತಾಗಬಹುದೆ? ಹಾಗೆಂದಾದರೆ ಗುರುತ್ವಾಕರ್ಷಣ ನಿಯಮವು ಸರ್ವಜನಗ್ರಾಹಿಯಾದಂತೆ, ಸಕಲ ಧರ್ಮಗಳಲ್ಲೂ ಮೂಲಭೂತ ಸಾಮ್ಯವು ಕಂಡುಬಂದು, ಒಂದು ನಿಜವಾದ ಸೌಹಾರ್ದಭಾವನೆ ಮೂಡುವ ಸಂಭವವಿದೆಯೆ? ಈ ಪ್ರಶ್ನೆಗಳಿಗೆ ಉತ್ತರವನ್ನು ಈ ಅಧ್ಯಾಯ\-ದಲ್ಲಿ ಕೊಡಲು ಯತ್ನಿಸಲಾಗಿದೆ.

ಕೆಲವು ವರ್ಷಗಳಿಂದೀಚೆಗೆ ಮನೋವಿಜ್ಞಾನಿಗಳೂ, ಮನೋರೋಗ ತಜ್ಞರೂ, ಅತೀಂದ್ರಿಯ ಅನುಭವಗಳ ಬಗೆಗೆ ಅಮೇರಿಕ ಮತ್ತು ರಷ್ಯಾ ದೇಶಗಳಲ್ಲಿ ಸಾಕಷ್ಟು ಸಂಶೋಧನೆಗಳನ್ನು ಕೈಗೊಂಡಿದ್ದಾರೆ. ಇಂಥ ತಜ್ಞರುಗಳಲ್ಲಿ ಅಂತಾರಾಷ್ಟ್ರೀಯ ಖ್ಯಾತಿಯ ಮನೋರೋಗ ಚಿಕಿತ್ಸಕಳಾದ ಅಮೇರಿಕದ ಡಾ.\ ಎಲಿಜಬೆತ್ ಕುಬ್ಲೇರ್ ರೋಸ್ ಅವರದು ದೊಡ್ಡ ಹೆಸರು. ಸಾವಿನ ಅಂಚಿನಲ್ಲಿರುವ ರೋಗಿಗಳನ್ನು ಕುರಿತ ಡಾಕ್ಟರುಗಳ, ಶುಶ್ರೂಷೆಯನ್ನು ಮಾಡುವವರ, ರೋಗಿಗಳ ಸಂಬಂಧಿಗಳ ದೃಷ್ಟಿಕೋನವನ್ನೇ ಬದಲಿಸುವ ಕ್ರಾಂತಿಯನ್ನು ಸ್ವಿಸ್ ದೇಶದ ಈ ಮಹಿಳಾ ವೈದ್ಯೆ ಮಾಡಿದ್ದಾರೆ ಎಂದರೆ ತಪ್ಪಿಲ್ಲ. ಸುಮಾರು ಹತ್ತು ವರ್ಷಗಳಿಂದ ಅವರು ಈ ವಿಚಾರವಾಗಿ ಸಂಶೋಧನೆಗಳನ್ನು ನಡೆಯಿಸಿ, ಹಲವಾರು ಗ್ರಂಥಗಳನ್ನು ಪ್ರಕಟಿಸಿದ್ದಾರೆ. ಅವರು ನೂರಾರು ಆಸ್ಪತ್ರೆಗಳಲ್ಲಿ ನಾನಾ ತೆರನಾದ ರೋಗಗಳಿಂದ ನರಳುವ, ಸಾವನ್ನು ಸಮೀಪಿಸಿದ, ಸಹಸ್ರಾರು ರೋಗಿಗಳನ್ನು ಭೇಟಿ ಮಾಡಿ, ಸಾವನ್ನಪ್ಪುವ ಮೊದಲಿನ ಅವರ ಮಾನಸಿಕ ಸ್ಥಿತಿಗತಿಗಳನ್ನು ಅಧ್ಯಯನ ಮಾಡಿದ್ದಾರೆ. ಸುಮಾರು ಇಪ್ಪತ್ತೈದು ವರ್ಷಗಳಿಂದ ಡಾಕ್ಟರ್ ಆಗಿ ಕೆಲಸ ಮಾಡುತ್ತಿರುವ ಅವರು ರೋಗಿಗಳು ಸಾವನ್ನು ಸಮೀಪಿಸಿದಾಗ ಮಾಡುವ ವಿಭಿನ್ನ ವರ್ತನೆಗಳಲ್ಲಿ ಆಸಕ್ತರಾದರು. ಕೆಲವು ರೋಗಿಗಳು ಆರ್ತನಾದ ಮಾಡುತ್ತ ಸತ್ತರೆ, ಕೆಲವರು ಸಾಯುವಾಗ ಅಪೂರ್ವ ಆನಂದ, ಶಾಂತಿಗಳನ್ನು ವ್ಯಕ್ತಪಡಿಸುತ್ತಿದ್ದರು. ಇನ್ನು ಕೆಲವರು, ಯಾವುದೋ ಅವ್ಯಕ್ತ ವ್ಯಕ್ತಿಗಳೊಂದಿಗೆ ಮಾತುಕತೆಯಾಡಿ ಸಂತಸದಿಂದ ಇತರ ರೋಗಿಗಳಿಗೆ ಅಥವಾ ಬಂಧುಗಳಿಗೆ ತಮ್ಮ ಲೋಕಾಂತರ ಪಯಣ ಯಾವ ದಿನ ಎಂಬ ವಿಚಾರ ತಿಳಿಸುತ್ತಿದ್ದರು. ಔಷಧ ಸೇವನೆಯ ವೈಪರೀತ್ಯದಿಂದ, ಭ್ರಮಾತ್ಮಕ ಕಲ್ಪನೆಗಳಿಂದ ಬಳಲುತ್ತಿರಬಹುದೆಂದು ಮೊದಲು ಅವರು ಯೋಚಿಸಿದರೂ, ಯಾವ ಔಷಧವನ್ನೂ ಸೇವಿಸದಿದ್ದ ರೋಗಿಗಳಲ್ಲೂ ಇಂಥ ವರ್ತನೆಗಳು ಕಂಡು ಬರುತ್ತಿದ್ದವು. ಕೆಲವು ರೋಗಿಗಳ ವರ್ತನೆಯಂತೂ ತೀರ ಆಶ್ಚರ್ಯಕರವೂ, ಸಾವಿನ ಆಚೆಗೆ ಏನು ಎಂಬುದನ್ನು ಸೂಚಿಸುವಂತೆಯೂ ಇತ್ತು.

\vskip 2pt

ಆಳ ಅನುಭವವಿರುವ ಮನೋರೋಗತಜ್ಞೆಯಾದ ಇವರು ಅಮೇರಿಕದ ಕಾಲೇಜಿನ\break ವಿದ್ಯಾರ್ಥಿಗಳನ್ನುದ್ದೇಶಿಸಿ ಭಾಷಣ ಮಾಡುತ್ತಿರುವಾಗ, ತಮ್ಮ ತಿಳಿವಳಿಕೆಯನ್ನು ನಿರ್ಭೀತಿಯಿಂದ ಸಾರಿ ವಿದ್ಯಾರ್ಥಿಗಳ, ತಜ್ಞರ ಮತ್ತು ಪತ್ರಿಕೆಗಳವರ ನಾನಾ ತೆರನಾದ ಪ್ರತಿಕ್ರಿಯೆಗಳ ಕೋಲಾಹಕ್ಕೆ ಕಾರಣರಾದರು. ಸಭೆಯಲ್ಲಿ ತಮ್ಮ ಆಂತರಿಕ ಅರಿವನ್ನು ವ್ಯಕ್ತಗೊಳಿಸಲು ಸಂಕಲ್ಪಿಸಿರದಿದ್ದರೂ ಮಾತೆಯೊಬ್ಬಳ ಮನಕರಗುವ ಪ್ರಶ್ನೆಗುತ್ತರವಾಗಿ ಈ ಮಾತನ್ನು ಹೇಳಿಯೇಬಿಟ್ಟರು: “ಇದು ನನ್ನ ನಂಬಿಕೆ ಅಥವಾ ಅಭಿಪ್ರಾಯವಾದ ಮಾತಲ್ಲ. ಸ್ವಲ್ಪವೂ ಸಂಶಯವಿಲ್ಲದೇ ತಿಳಿದುಕೊಂಡಿದ್ದೇನೆ –ಸಾವಿನ ಆಚೆಗೆ ಜೀವನವಿದೆ” ಎಂದು.\footnote{\engfoot{It is not a matter of belief or opinion. I know beyond a shadow of doubt that there is life after death.—}ಅಮೇರಿಕದ \engfoot{Macgall (Aug.}೧೯೭೬ )ಮಾಸಪತ್ರಿಕೆಯಿಂದ ಉದ್ಧೃತ.}

\vskip 2pt

ಈ ಹೇಳಿಕೆಯು ಉಂಟುಮಾಡಿದ ಕೋಲಾಹಲ ಕಳೆದ ಬಳಿಕ ಕೆನ್ನೆತ್ ವುಡ್​ವರ್ಡ್ ಎಂಬುವರು ಖುದ್ದಾಗಿ ಡಾ. ಕುಬ್ಲೇರ್ ರೋಸ್ ಅವರನ್ನು ಭೆಟ್ಟಿಯಾಗಿ, ಅವರಿಂದ ಈ ಬಗ್ಗೆ ಇನ್ನೂ ಹೆಚ್ಚಿನ ವಿವರಗಳನ್ನು ತಿಳಿಯಲು ಅಪೇಕ್ಷಿಸಿದರು. ಆಗ ತಮ್ಮ ಗ್ರಂಥಗಳಲ್ಲಿ ಇದುವರೆಗೂ ಪ್ರಕಟವಾಗದ ಅರ್ಥಪೂರ್ಣ ಘಟನೆಯೊಂದನ್ನು ಅವರು ವುಡ್​ವರ್ಡಿಗೆ ತಿಳಿಸಿದರಂತೆ. ಅದೇ ಇಲ್ಲಿದೆ–


\section*{ಸತ್ತು ಬದುಕಿದಳು!}

\addsectiontoTOC{ಸತ್ತು ಬದುಕಿದಳು!}

ಇದು ಅಮೇರಿಕದ ಇಂಡಿಯಾನಾ ಆಸ್ಪತ್ರೆಯಲ್ಲಿ ನಡೆದ ಘಟನೆ. ನಲ್ವತ್ತು ವರ್ಷ ವಯಸ್ಸಿನ ಹೆಂಗಸೊಬ್ಬಳು ಯಕೃತ್ ಸಂಬಂಧವಾದ ಕಾಯಿಲೆಯಿಂದ ಹಲವು ಬಾರಿ ಸಾವನ್ನು ಸಮೀಪಿಸಿ ದ್ದಳೂ ಬದುಕಿ ಉಳಿದಿದ್ದಳು. ಒಮ್ಮೆ ಆ ರೋಗ ಬಹಳಷ್ಟು ಉಲ್ಬಣಗೊಂಡು, ತುರ್ತು ಚಿಕಿತ್ಸೆಗಾಗಿ ಆಸ್ಪತ್ರೆಗೆ ಸೇರಿಸಿದರು. ಒಂದು ಮಧ್ಯಾಹ್ನ ಆಕೆ ಮೇಲುಸಿರು ಬಿಡುತ್ತ ಸಾವಿನೊಡನೆ ಸೆಣಸು ತ್ತಿದ್ದುದನ್ನು ಕಂಡ ಆಸ್ಪತ್ರೆಯ ದಾದಿಯೊಬ್ಬಳು, ಡಾಕ್ಟರುಗಳನ್ನು ಕರೆತರಲು ಹೊರಗೆ ಧಾವಿಸಿದರು. ಆ ಹೊತ್ತಿನಲ್ಲೇ ಆ ರೋಗಿಗೆ ಒಂದು ಅಪೂರ್ವವಾದ ಅನುಭವವಾಯಿತು. ಆಕೆ ದೇಹದ ಹೊರಗೆ ಬಂದು ಆ ಕೋಣೆಯ ಮೇಲ್ಭಾಗದಲ್ಲಿ, ಆಕಾಶದಲ್ಲಿ ಮೋಡವು ತೇಲುವಂತೆ ತೇಲುತ್ತ ಮಂಚದಲ್ಲಿ ಮಲಗಿದ್ದ ತನ್ನ ಶರೀರವನ್ನೇ ಕಂಡಳು! ತನ್ನ ಮುಖ ಎಷ್ಟೊಂದು ಬಾಡಿ ಸೊರಗಿದೆ ಎಂಬುದೂ ಅವಳಿಗೆ ಆಗ ಗೋಚರಿಸಿತು. ಅದೇ ಹೊತ್ತಿನಲ್ಲಿ ಆಕೆಗೆ ಒಂದು ತೆರನಾದ ಅಪೂರ್ವ ಶಾಂತಿ ಮತ್ತು ಬಂಧಮುಕ್ತಭಾವನೆಯ ಅರಿವೂ ಆಯಿತು. ಈ ಘಟನೆಯ ಇನ್ನೂ ಆಶ್ಚರ್ಯದ ಅಂಶವೆಂದರೆ ದೇಹದಿಂದ ಹೊರಗೆ, ಕೋಣೆಯ ಮೇಲ್ಭಾಗದಲ್ಲಿ ತೇಲುತ್ತಿದ್ದ ಆಕೆ, ಅಲ್ಲಿ ಗುಂಪುಗೂಡಿಕೊಂಡು ತನ್ನ ಶರೀರದಲ್ಲಿ ಪುನಃ ಪ್ರಾಣಪ್ರತಿಷ್ಠೆ ಮಾಡಲು ಹೆಣಗುತ್ತಿದ್ದ ಡಾಕ್ಟರುಗಳ ಚಟುವಟಿಕೆಗಳನ್ನೂ ಚೆನ್ನಾಗಿ ವೀಕ್ಷಿಸಬಲ್ಲವಳಾಗಿದ್ದಳು! ಅವರೆಲ್ಲಾ ಆಗಾಗ ಆಡುತ್ತಿದ್ದ ಮಾತುಗಳೂ ಅವಳಿಗೆ ಸರಿಯಾಗಿ ಕೇಳಿಸುತ್ತಿದ್ದುವು. ಆ ವೈದ್ಯರ ಗುಂಪಿನಲ್ಲಿದ್ದ ಒಬ್ಬಾತ ಹಾಸ್ಯದ ಚಟಾಕಿಯನ್ನು ಹಾರಿಸಿ, ಗಂಭೀರವಾದ ವಾತಾವರಣವನ್ನು ಸ್ವಲ್ಪ ಲಘುವಾಗಿ ಮಾಡಲು ಯತ್ನಿ ಸಿದ್ದ. ಅವನಾಡಿದ ಹಾಸ್ಯದ ಮಾತನ್ನೂ ಅವಳು ಕೇಳಿಸಿಕೊಂಡಿದ್ದಳು. ಆ ವೈದ್ಯರನ್ನು ಸಂಬೋ ಧಿಸಿ ಅಷ್ಟೊಂದು ಶ್ರಮಿಸಬೇಡಿ, ಎಲ್ಲವೂ ಸರಿಯಾಗಿದೆ ಎಂದು ಹೇಳಲು ಯತ್ನಿಸಿದಳು. ಆದರೆ ಅವಳ ಶ್ವಾಸೋಚ್ಛ್ವಾಸಕ್ರಿಯೆ, ನಾಡಿಬಡಿತ ನಿಂತಿದ್ದವು, ರಕ್ತದ ಒತ್ತಡವಿರಲಿಲ್ಲ, ಮೆದುಳಿನ ಅಲೆಗಳ ಕ್ರಿಯೆಯೂ ನಿಂತಿತ್ತು. ಅಲ್ಲಿ ಸೇರಿದ್ದ ವೈದ್ಯರೆಲ್ಲ ಆಕೆ ಸತ್ತಿದ್ದಾಳೆ ಎಂದು ಹೇಳಿಕೆಯನ್ನಿ ತ್ತರು. ಆಶ್ಚರ್ಯದ ಸಂಗತಿ ಎಂದರೆ ಸುಮಾರು ಮೂರು ಗಂಟೆಗಳ ಬಳಿಕ ಅವಳು ತನ್ನ ದೇಹಕ್ಕೆ ಹಿಂದಿರುಗಿದಳು! ಮಿದುಳಿಗೆ ಯಾವ ತರದ ಆಘಾತ, ತೊಂದರೆಗಳು ಆಗದೆ, ಆಕೆ ಮತ್ತೆ ಹದಿನೆಂಟು ತಿಂಗಳುಗಳ ಕಾಲ ಬದುಕಿದ್ದಳು.

ತಾವು ಆಕೆಯಿಂದ ಕೇಳಿ ಪರಿಶೀಲಿಸಿದ ಈ ಘಟನೆಯನ್ನು ಡಾ.\ ರೋಸ್​–ವೈದ್ಯರು, ದಾದಿ ಯರು, ವೈದ್ಯಕೀಯ ಮತ್ತು ಮನೋವಿಜ್ಞಾನದ ವಿದ್ಯಾರ್ಥಿಗಳು, ಪಾದ್ರಿಗಳು–ಇವರೆಲ್ಲರಿಗೆ ವಿವರವಾಗಿ ತಿಳಿಸಿದರು. ತಾನು ಅದೊಂದು ‘ಮತಿಭ್ರಮೆ ಎಂದು ಒಪ್ಪಲು ಸಿದ್ದಳಿಲ್ಲ’ ಎಂದಾಗ, ಅವರೆಲ್ಲರೂ ರೋಷಾವೇಶದಿಂದ ಕೂಗಾಡಿದರು. ‘ದೇಹದಲ್ಲಿ ಪ್ರಜ್ಞೆಯು ಪುನಃ ಮೂಡಿದಾಗ ಅವಳು ಹೇಳಿದ ವಿಚಾರಗಳೆಲ್ಲ ಸತ್ಯವಾಗಿರುವಾಗ, ಅದನ್ನು “ಮತಿಭ್ರಮೆ” ಎಂಬ ವಿಭಾಗಕ್ಕೆ ಹೇಗೆ ಸೇರಿಸಲಿ? ಆ ಅದ್ಭುತ ಘಟನೆಯ ನೈಜಸ್ವರೂಪವನ್ನು ಮರೆಮಾಚಲು ನಾನು ಇಚ್ಛಿಸುವುದಿಲ್ಲ’\break ಎಂದರು ರೋಸ್. ಇಂಥ ನೂರಾರು ಘಟನೆಗಳನ್ನು ಅಧ್ಯಯನ ಮಾಡಿದ ಕುಬ್ಲೇರ್ ರೋಸ್ ‘ತಾನು ಮತಾಂಧಳಲ್ಲದ ಧಾರ್ಮಿಕಳಾಗಿದ್ದೇನೆ’ ಎನ್ನುತ್ತಾರೆ. ಈ ಕ್ಷೇತ್ರದಲ್ಲಿ ಹಲವಾರು ಡಾಕ್ಟರರೂ, ಆಸಕ್ತರೂ ತಮ್ಮ ಅನುಭವಗಳನ್ನು ಸಂಗ್ರಹಿಸಿ ಪ್ರಕಟಿಸುತ್ತಲೇ ಇದ್ದಾರೆ.\footnote{ ಇಂತಹದೇ ಇನ್ನೊಂದು ಅನುಭವಕ್ಕೆ ಓದಿ:‘ನನ್ನ ಮರಣೋತ್ತರ’–ಸಂತೋಷಕುಮಾರ ಗುಲ್ವಾಡಿ, ಕಸ್ತೂರಿ,\break ಸೆಪ್ಟಂಬರ್ ೧೯೮೧}

ಇಲ್ಲಿ ಒಂದು ಮಾತನ್ನು ಮರೆಯಬಾರದು. ಅತೀಂದ್ರಿಯಾನುಭವ ಅಥವಾ ದೇಹಾತೀತ ಅನುಭವಗಳನ್ನು ಪಡೆದ, ಎಲ್ಲ ರೋಗಿಗಳಲ್ಲೂ ಕಂಡು ಬಂದ ಒಂದು ಪ್ರಮುಖ ಬದಲಾವಣೆ ಇದು: ಸಾವನ್ನು ಎದುರಿಸುವುದರಲ್ಲಿ ಅವರ ಮಾನಸಿಕ ಸ್ಥೈರ್ಯ ಮತ್ತು ಉತ್ಸಾಹ. ಎರಡನೇ ಬಾರಿ ಮರಣವನ್ನು ಸಂಧಿಸಲು ಅವರಲ್ಲಿ ಕಂಡು ಬಂದ ತವಕ ಮತ್ತು ಧೈರ್ಯ!\footnote{ ಆದರೆ ಒಂದು ವಿಚಾರವನ್ನು ಇಲ್ಲಿ ಹೇಳಬೇಕು. ಆತ್ಮಹತ್ಯೆಯಂಥ ಪಾಪಕಾರ್ಯವು ಜೀವಿಗೆ ಅತ್ಯಂತ ದುಃಖ ಹಾಗೂ ದುಃಸ್ಥಿತಿಯನ್ನುಂಟುಮಾಡುವುದು ಎಂಬುದು ಡಾ. ರೋಸ್ ಹಾಗೂ ಇತರ ತಜ್ಞರ ಅಭಿಮತ.}


\section*{ದೇಹ ಇಲ್ಲಿ, ದೇಹಿ ಅಲ್ಲಿ!}

\addsectiontoTOC{ದೇಹ ಇಲ್ಲಿ, ದೇಹಿ ಅಲ್ಲಿ!}

ನಿಜವಾದ ದೇಹಾತೀತ ಅನುಭವವನ್ನೂ, ಭ್ರಾಂತಿದರ್ಶನ ಅಥವಾ ಹೆಚ್ಚೆಂದರೆ ಒಂದು ಸ್ಪಷ್ಟ\-ವಾಗಿರ\-ಬಹುದಾದ ಸ್ವಪ್ನ ಸೃಷ್ಟಿ ಎಂದು ಹೆಸರಿಸಲು ಯತ್ನಿಸುವವರಿಲ್ಲದಿಲ್ಲ. ಮನೋವಿಜ್ಞಾನಿಗಳೂ, ಇತರ ಕೆಲವು ಬುದ್ಧಿಜೀವಿಗಳೆನಿಸಿಕೊಂಡವರೂ ಅತೀಂದ್ರಿಯ ಘಟನೆಗಳ ಬಗ್ಗೆ ಮೇಲಿನಂತೆ ವಿವರಣೆಯನ್ನು ಹಲವು ವರ್ಷಗಳಿಂದ ಕೊಡುತ್ತ ಬಂದಿದ್ದಾರೆ. ಇಂಥ ಅನುಭವಗಳನ್ನು ಒಂದು ತೆರನಾದ ಬುದ್ಧಿಭ್ರಮಣೆಯಾದ ವ್ಯಕ್ತಿಗಳೇ ಪಡೆಯಲು ಸಾಧ್ಯ ಎಂಬ ಭಾವನೆಯೂ ಇತ್ತು. ಆದರೆ ಇಂದು ಮನೋವಿಜ್ಞಾನಿಗಳಲ್ಲಿ ಕೆಲವರಾದರೂ ಈ ಭಾವನೆ ತಪ್ಪು ಎಂಬುದನ್ನು ಕಂಡುಕೊಂಡಿದ್ದಾರೆ–ಕಂಡುಕೊಳ್ಳುತ್ತಿದ್ದಾರೆ. ಅಮೇರಿಕದ ಟ್ರಾನ್ಸ್​ಪರ್ಸನಲ್ ಸೈಕಾಲಜಿ ವಿಭಾಗದ ಅನ್ವೇಷಣಾತಜ್ಞರು ತಮ್ಮ ಪತ್ರಿಕೆಯಲ್ಲಿ ಈ ಬಗ್ಗೆ ಸಾಕಷ್ಟು ಮಾಹಿತಿಗಳನ್ನು ನೀಡಿದ್ದಾರೆ.

ಮನುಷ್ಯನ ಮನಸ್ಸು ಮಿದುಳಿಗಿಂತ ಭಿನ್ನವಾದುದು ಎಂದು ಯೋಚಿಸುವ ಕಾಲ ಬಂದಿದೆ ಎನ್ನುತ್ತಾರವರು. ಮನಸ್ಸಿನ ನಿಯಂತ್ರಣದಿಂದ ನರಮಂಡಲ ಮತ್ತು ದೈಹಿಕ ಚಟುವಟಿಕೆಗಳನ್ನು ನಿಯಂತ್ರಿಸಬಹುದೆಂಬುದು ಅವರು ಒಪ್ಪಿಕೊಳ್ಳುವ ಇನ್ನೊಂದು ವಿಚಾರ. ನಾವು ಯಾವುದನ್ನು ಜೀವತತ್ತ್ವ ಅಥವಾ ಜೀವಾತ್ಮ ಎಂದು ಕರೆಯುತ್ತೇವೋ, ಅದೊಂದು ನಿರ್ದಿಷ್ಟವಾದ ಶಕ್ತಿ ಮಾತ್ರವಲ್ಲ, ದೇಹದ ನಾಶದೊಂದಿಗೆ ನಾಶವಾಗದ, ‘ಆತ್ಮ’ ಎಂದು ಹಿಂದಿನ ಕಾಲದ ಜನ ಕರೆಯುತ್ತಲಿದ್ದ ತಥ್ಯ ಎಂದು ತಿಳಿಯುವ ಕಾಲ ದೂರವಿಲ್ಲ ಎಂಬುದೂ ಅವರ ಅಭಿಪ್ರಾಯ.

ನ್ಹಾಸಾದಲ್ಲಿರುವ ಟೋಪೇಕಾದ ವೆಟರನ್ಸ್ ಆಸ್ಪತ್ರೆಯ ಸೈಕಿಯಾಟ್ರಿ ವಿಭಾಗದ ಮುಖ್ಯಸ್ಥರಾದ ಸ್ಟೂಆರ್ಟ್ ಟ್ವೆಮ್ಲೋರ ಅಭಿಪ್ರಾಯವಿದು: ‘ವಿಜ್ಞಾನಿಗಳು ಈ ದೇಹಾತೀತ ಅನುಭವವನ್ನು ಸ್ವತಃ ತಾವೇ ಪಡೆಯಬೇಕು.’ ಇದನ್ನು ಪ್ರಯೋಗಶಾಲೆಯಲ್ಲಿ, ಔಷಧಸೇವನೆ, ಸುಪ್ತಿಸೂಚನೆ, ವಿದ್ಯುತ್ ತರಂಗದ ಮೂಲಕ ಮಿದುಳಿನ ಉತ್ತೇಜನ–ಈ ವಿಧಾನಗಳಿಂದ ಪಡೆಯಬಹುದೆಂದು ಟ್ವೆಮ್ಲೋ ತಮ್ಮ ಅನೇಕ ಪ್ರಯೋಗಗಳಿಂದ ದೃಢಪಡಿಸಿಕೊಂಡು, ಆ ಅನುಭವಗಳ ಆಧಾರದಿಂದ ಹೇಳಿದ ಮಾತಿದು ಎಂಬುದು ಗಮನಾರ್ಹ. ಅವರ ಒಂದು ಅನುಭವ ಇಲ್ಲಿದೆ–

ಕೆಂಟುಕಿಯ ಆಷ್ಟನ್ ಪ್ರಯೋಗಶಾಲೆಯ ಸುಸಜ್ಜಿತ ಪಂಜರ ಒಂದರಲ್ಲಿ ಟ್ವೆಮ್ಲೋ ಕುಳಿತಿದ್ದರು. ಶ್ರವಣಾತೀತ ಶಬ್ದತರಂಗಗಳಿಂದ ಅವರ ಮಿದುಳನ್ನು ಪ್ರಚೋದಿಸಲಾಯಿತು. ಆಗಿನ ತಮ್ಮ ದೇಹಾತೀತ ಅನುಭವವನ್ನು ಅವರು ಹೀಗೆಂದು ವರ್ಣಿಸಿದರು: “ದೇಹದಿಂದ ಮನಸ್ಸು ಬೇರೆ\-ಯಾಗು\-ವುದನ್ನು ಸ್ಪಷ್ಟವಾಗಿ ನಾನು ಗಮನಿಸುತ್ತಿದ್ದೆ. ಈ ಸ್ಥಿತಿಯಲ್ಲೇ ಕಗ್ಗತ್ತಲ ಸುರಂಗವೊಂದರಲ್ಲಿ ವೇಗವಾಗಿ ಪಯಣಿಸಿದೆ. ಟೋಪೇಕಾದಲ್ಲಿರುವ ನನ್ನ ಮನೆಯಲ್ಲಿ ಕೆಲವೇ ಕ್ಷಣಗಳಲ್ಲಿ ನಾನಿದ್ದೆ. ಪ್ರಯೋಗಶಾಲೆಯಿಂದ ನನ್ನ ಮನೆ ಸುಮಾರು ನಲ್ವತ್ತು ಮೈಲಿ ದೂರದಲ್ಲಿತ್ತು ಎಂಬುದನ್ನು ನೀವು ಮರೆಯಬಾರದು. ನನ್ನ ಪತ್ನಿ ಅಡುಗೆ ಮನೆಗೆ ನೀರು ಕುಡಿಯಲು ಹೋಗುವುದನ್ನು ಕಂಡೆ. ಆಕೆ ಅಲ್ಲಿಂದ ಹಿಂದಿರುಗಿ ಮಲಗುವ ಕೋಣೆಯ ಹತ್ತಿರ ಹೋಗುತ್ತಿದ್ದಳು. ಆಗ ರಾತ್ರಿ ೯.೩೦ರ ಹೊತ್ತು. ನಾನು ಅವಳ ಸಮೀಪ ಕೆಲವು ಕ್ಷಣ ನಿಂತಿದ್ದು ನಿಲುಗನ್ನಡಿಯ ಹತ್ತಿರ ಹೋಗಿ ನನ್ನ ಪ್ರತಿಬಿಂಬ ಕಾಣಲು ಸಾಧ್ಯವೇ ಎಂದು ಯತ್ನಿಸಿದೆ. ಆದರೆ ಆ ಕನ್ನಡಿಯಲ್ಲಿ ನನಗೆ ಏನೂ ಕಾಣಲಿಲ್ಲ. ಕೆಲವೇ ಕ್ಷಣಗಳಲ್ಲಿ ಪುನಃ ನನ್ನ ಶರೀರದಲ್ಲಿ ನಾನಿರುವ ಅರಿವು ಬಂತು.

“ಈ ಅದ್ಭುತ ಅನುಭವವನ್ನು ಕುರಿತು ನನ್ನ ಹೆಂಡತಿಗೆ ನಾನು ಏನೂ ತಿಳಿಸಿರಲಿಲ್ಲ. ಅವಳಿಗೆ ದೇಹಾತೀತ ಅನುಭವಗಳ ಬಗ್ಗೆ ನಂಬಿಕೆಯೂ ಇರಲಿಲ್ಲ. ಆಮೇಲೆ ನಾನು ಊರಿಗೆ ಹೋದಾಗ ನಾನಾಗಿ ಪ್ರಸ್ತಾಪಿಸದಿದ್ದರೂ, ಅವಳಾಗಿ ಹೇಳಿದಳು: ‘ಮೊನ್ನೆ ಒಂದು ದಿನ ನೀವು ಕೆಂಟಕಿಯಲ್ಲಿ ದ್ದಾಗ ನಾನು ರಾತ್ರಿ ೯.೩೦ರ ಹೊತ್ತಿನಲ್ಲಿ ನೀರು ಕುಡಿಯಲು ಅಡಿಗೆಮನೆಗೆ ಹೋಗಿ ಮಲಗುವ ಕೋಣೆಗೆ ಹಿಂದಿರುಗುತ್ತಿದ್ದೆ. ಯಾವುದೋ ನೆರಳಿನಂಥದೊಂದು ಕ್ಷಣಕಾಲ ನನ್ನ ಸಮೀಪದಲ್ಲಿದ್ದು ಆ ಬಳಿಕ ನಿಲುಗನ್ನಡಿಯ ಎದುರು ನಿಂತು ಮಾಯವಾಯಿತು’ ಎಂದು.”

ಇಂಥ ಸಾಕ್ಷ್ಯಾಧಾರ ಹಾಗೂ ಹಲವು ಪ್ರಯೋಗಗಳ ಬಳಿಕ ‘ಆತ್ಮದ’ ಅಸ್ತಿತ್ವದ ಬಗ್ಗೆ ಟ್ವೆಮ್ಲೋ ನೇರವಾಗಿ, ನಿರ್ಭೀತಿಯಿಂದ ಮಾತಾಡತೊಡಗಿದ್ದಾರೆ. ಸಾಮಾನ್ಯವಾಗಿ ಈ ವಿಚಾರ ಮಾತಾಡುವವರ ಪಾಡು ನಗೆಗೀಡಾಗುತ್ತದೆ. ವೈಜ್ಞಾನಿಕ ಪೂರ್ವಾಗ್ರಹವಿರುವ ದೇಶದಲ್ಲಿ ಇದೊಂದು ದಿಟ್ಟ ಹೆಜ್ಜೆಯೇ ಸರಿ. ಆದರೂ ಟ್ವೆಮ್ಲೋ ಜಾಗರೂಕತೆಯಿಂದ ಹೀಗೆಂದು ಹೇಳುತ್ತಾರೆ: ‘ದೇಹಾತೀತ ಅಸ್ತಿತ್ವ ಇದೆ ಎಂಬುದಕ್ಕೆ ಇಂಬುಕೊಡುವ ಸೂಚನೆ ತರ್ಕಬದ್ಧವಾಗಿದೆ.’


\section*{ಕನಸಲ್ಲ ನನಸು}

\addsectiontoTOC{ಕನಸಲ್ಲ ನನಸು}

ಎಷ್ಟೋ ಮಂದಿ ವಿಜ್ಞಾನಿಗಳು, ಗಣಿತಜ್ಞರು, ಕೆಲವೊಮ್ಮೆ ಸ್ವಪ್ನಗಳಲ್ಲಿ ತಮ್ಮ ಸಮಸ್ಯೆಗಳಿಗೆ ಪರಿಹಾರವನ್ನು ಕಂಡ ಘಟನೆಗಳಿವೆ. ನಮ್ಮ ದೇಶದ ಮಹಾಗಣಿತಜ್ಞ ರಾಮಾನುಜನ್ ಅಂಥವರಲ್ಲಿ ಒಬ್ಬರು. ನಮ್ಮ ಪ್ರಜ್ಞೆಯು ವಿವಿಧ ಸ್ತರಗಳಲ್ಲಿ ಕೆಲಸ ಮಾಡುವುದೆಂದು ತಿಳಿಯಲು ಅವೇ ಸಾಕ್ಷಿಗಳು.\break ತಜ್ಞರಿಂದ ಪರಿಶೀಲಿಸಲ್ಪಟ್ಟ ಅಂತಹ ಇನ್ನೊಂದು ಘಟನೆ ಇದು–

ಅಮೇರಿಕಾ ದೇಶದ ಪ್ಲೋರಿಡಾ ಪ್ರಾಂತದ ವಿಂಟರ್ ಪಾರ್ಕಿನಲ್ಲಿ ವಾಸಿಸುತ್ತಿರುವ ಸುಮಾರು ಮೂವತ್ತು ವರ್ಷ ವಯಸ್ಸಿನ ಮಹಿಳೆ ಆಕೆ. ಅವಳ ಹೆಸರು ಎಲಿಜಬೆತ್ ವಿಲ್​ಸನ್. ಎಷ್ಟೋ ವೇಳೆ ಆಕೆ ಕನಸಿನಲ್ಲಿ ಅತ್ಯಂತ ಸ್ಪಷ್ಟವಾಗಿ ಅಮೇರಿಕದ ಒಂದು ನೌಕೆ ‘ಟಿಟಾನಿಕ್​’ ಮುಳುಗುವ ಭಯಾನಕ ದೃಶ್ಯವನ್ನು ಕಂಡು ಚೀತ್ಕಾರ ಮಾಡುತ್ತಾಳೆ. ತಾನು ಕಂದು ಬಣ್ಣದ ಸಿಲ್ಕ್​ಬಟ್ಟೆಗಳನ್ನು ಧರಿಸಿ ಆ ಹಡಗಿನ ನೃತ್ಯದ ಕೋಣೆಯಲ್ಲಿ ನಿಂತಂತೆಯೂ, ಪಾಲಿಶ್ ಮಾಡಿದ ಮರದ ಕಂಭಗಳು ನೆಲದಿಂದ ಮಾಡಿನ ಮುಚ್ಚಿಗೆಯವರೆಗೂ ತಾಗಿ ನಿಂತುದನ್ನೂ, ಹಡಗಿನ ಅಲುಗಾಟಕ್ಕೆ ತೂಗು ದೀಪಗಳು ಜೋತಾಡುವುದನ್ನೂ ಕಾಣುತ್ತಾಳೆ. ತಾನು ನಿಂತುಕೊಂಡಿರುವ ಕೋಣೆಯು ದೀಪದ ಬೆಳಕಿನಿಂದ ಥಳಥಳನೆ ಹೊಳೆಯುತ್ತಿದೆ. ಕೋಣೆಯಿಂದ ಹೊರಹೊರಟು ಹಡಗಿನ ಹೊರಭಾಗಕ್ಕೆ ಬಂದಾಗ ಸಮುದ್ರದ ಉಪ್ಪುನೀರಿನ ವಾಸನೆ ಹೊಡೆಯುತ್ತಿದ್ದಂತೆ, ಜನರೆಲ್ಲ ತಮ್ಮ ರಕ್ಷಣೆಗಾಗಿ ಸಣ್ಣಪುಟ್ಟ ನೌಕೆಗಳಲ್ಲಿ ಸೇರಿಕೊಂಡಿದ್ದುದನ್ನು ಕಾಣುತ್ತಿರುವಷ್ಟರಲ್ಲೇ, ಹಡಗು ವಾಲಿಕೊಂಡು ಭಯಾರ್ತರಾದ ಜನರ ಚೀರಾಟಗಳ ನಡುವೆ ಮುಳುಗುವಾಗ, ತಾನೂ ಆ ಅಗಾಧ ಕಪ್ಪು ಬಣ್ಣದ ಜಲರಾಶಿಯಲ್ಲಿ ಮುಳುಗುತ್ತಾಳೆ.

ಹಡಗು ನೀರಿನ ಆಳಪ್ರದೇಶಕ್ಕೆ ಹೋಗುತ್ತಿದ್ದಂತೆ, ಒಂದು ಕ್ಷಣ ಎಲ್ಲವೂ ಕತ್ತಲೆ. ಸ್ವಲ್ಪ ಹೊತ್ತಿನಲ್ಲೆ ಆಕೆಗೆ ಬೋಸ್ಟನ್ನಿನಲ್ಲಿ ಪಾರ್ಕಿಗೆ ಎದುರಾಗಿರುವ, ಕಂದು ಬಣ್ಣದ ಕಲ್ಲುಗಳಿಂದ ಕಟ್ಟಿದ ಒಂದು ವಾಸದ ಮನೆ ಕಾಣಿಸುತ್ತದೆ. ಅದರ ಹೆಸರು ‘ಕೆನ್ಸಿಂಗ್​ಟನ್.’ ತಾನು ಆ ಮನೆಯ ಹೊರಭಾಗದಲ್ಲಿ ನಿಂತುಕೊಂಡಿದ್ದ, ಉದ್ದನೆಯ ಮುಖದ, ಕಪ್ಪು ಕಣ್ಣುಗಳ ಯುವಕನ ಎದುರಲ್ಲಿ ನಿಂತಂತೆ ಅವಳಿಗೆ ಕಾಣಿಸುತ್ತದೆ. ಪ್ರಾಯಃ ಆತ ಆಕೆಯ ಪತಿ ಇರಬೇಕು. ಅದೇ ಸ್ವಪ್ನದ ಕೊನೆ. ಆಕೆ ಆ ಕನಸು ಕಂಡುದು ಒಂದೆರಡು ಬಾರಿ ಅಲ್ಲ, ಹಲವು ಬಾರಿ.

ವರ್ಜೀನಿಯಾ ವಿಶ್ವವಿದ್ಯಾಲಯದ ಸೈಕಿಯಾಟ್ರಿ ವಿಭಾಗದ ಅಧ್ಯಕ್ಷರೂ, ಜನ್ಮಾಂತರದ ನೆನಪುಗಳ ಸಂಬಂಧವಾದ ಸಂಶೋಧನೆಗಳಲ್ಲಿ ನಿರತರೂ ಆದ ಡಾ. ಆಯಾನ್ ಸ್ವೀವನ್​ಸನ್ ಮತ್ತು ಸಂಗಡಿಗರು ಈ ಘಟನೆಯಲ್ಲಿ ಆಸಕ್ತಿ ತಾಳಿ ಅನ್ವೇಷಣೆಗೆ ಆರಂಭಿಸಿದರು. ಫ್ಲೋರಿಡಾ ನಿವಾಸಿಯಾದ ಎಲಿಜಬೆತ್ ಎಂದೂ ಬೋಸ್ಟನ್ನಿಗೆ ಹೋಗಿರಲಿಲ್ಲ. ಆದರೆ ಈಗ ಸ್ವೀವನ್​ಸನ್ ಮತ್ತು ಸಂಗಡಿಗರು ಬೋಸ್ಟನ್ನಿಗೆ ತೆರಳಿ ಹುಡುಕಾಡತೊಡಗಿದರು. ತನ್ನ ಸ್ವಪ್ನದಲ್ಲಿ ಆಕೆ ವರ್ಣಿಸಿದ\break ಮನೆ ಕೊನೆಗೂ ಅವರಿಗೆ ಸಿಕ್ಕಿತು. ಅದರ ಹೆಸರು ಅದೇ–‘ಕೆನ್ಸಿಂಗ್​ಟನ್.’ ಕಟ್ಟಡದ ಬಣ್ಣವೂ ಅವಳು ವರ್ಣಿಸಿದಂತೆಯೇ ಇತ್ತು. ಆ ಮನೆಯಲ್ಲಿ ಎಂಬತ್ತು ವರ್ಷದ ಮುದುಕನನ್ನು ಅವರು ಭೇಟಿಯಾದರು. ವೃದ್ಧನಾಗಿದ್ದ ಅವನ ಮುಖ ಉದ್ದವಾಗಿದ್ದು, ಕಣ್ಣುಗಳು ಕಪ್ಪಾಗಿದ್ದವು. ಅವನೊಡನೆ ಮಾತುಕತೆಯಾಡುತ್ತ ಅವನ ಬದುಕಿನ ಬಗೆಗೆ ವಿಚಾರಿಸಿದಾಗ ಮತ್ತೊಂದು ಆಶ್ಚರ್ಯ ಕಾದಿತ್ತು! ತನ್ನ ಯುವತಿಯಾದ ಪತ್ನಿಯು ‘ಟೆಟಾನಿಕ್​’ ಹಡಗಿನಲ್ಲಿ ಹಿಂದಿರುಗುವಾಗ ಆ ಹಡಗಿನ ದುರಂತದಲ್ಲಿ ಕಣ್ಮರೆಯಾದ ಕತೆಯನ್ನು ಹೇಳಿ ಅವನು ನಿಟ್ಟುಸಿರುಬಿಟ್ಟ. ‘ಅವಳದು ಏನಾದರೂ ಭಾವಚಿತ್ರಗಳು ಸಿಗಬಹುದೇ?’ ಎಂದು ಸಂಶೋಧಕ ಗೆಳೆಯರು ಕೇಳಿದಾಗ, ಆತ ಒಂದು ಹಳೆಯ ಮೇಜಿನ ಡ್ರಾಯರ್ ಎಳೆದು ಒಂದು ಭಾವಚಿತ್ರದ ಪ್ರತಿಯನ್ನು ಅವರಿಗೆ ತೋರಿಸಿದ. ಇನ್ನೊಂದು ಆಶ್ಚರ್ಯ! ಈಗ ಫ್ಲೋರಿಡಾದಲ್ಲಿರುವ ಎಲಿಜಬೆತ್ ಅವಳ ಪ್ರತಿಕೃತಿಯಂತೆಯೇ ಆ ಪೋಟೋ ಇತ್ತು!

ಎಲಿಜಬೆತ್ ಅವಳ ಈಗಿನ ಮಾನಸಿಕ ಸ್ಥಿತಿಗೂ ಅವಳ ಹಿಂದಿನ ಜೀವನದ ದುರಂತಕ್ಕೂ ಸಂಬಂಧವೇನಾದರೂ ಇದೆಯೇ? ಕಂದುಗುಲಾಬಿ ಬಣ್ಣದ ರೇಷ್ಮೆಯನ್ನು ಕಂಡರೆ ಅವಳಿಗೆ ಸಹಿಸಲಾಗುತ್ತಿಲ್ಲ ಏಕೆ? ಸಮುದ್ರದ ಹೆಸರು ಕೇಳಿದರೆ ತಡೆಯಲಾರದ ಭಯ ಏಕೆ? ಸ್ವಪ್ನಗಳು ಕೆಲವೊಮ್ಮೆ ಪ್ರಜ್ಞೆಯ ಆಳದ ಸ್ತರಗಳಿಗೆ ಕಿಂಡಿಗಳೇ? ನನಸಾದ ಆಕೆಯ ಕನಸು ಜನ್ಮಾಂತರಕ್ಕೆ ಸಾಕ್ಷಿ ಆಗದೆ?


\section*{ಸತ್ತುಹೋದವಳೇ ಹುಟ್ಟಿ ಬಂದಳು}

\addsectiontoTOC{ಸತ್ತುಹೋದವಳೇ ಹುಟ್ಟಿ ಬಂದಳು}

ಅಂತೋನಿ ಕ್ರ್ಯಾನ್ಸಿ ಇಟೆಲಿಯ ಮಿಲಾನ್ ನಗರದ ಅಗರ್ಭ ಶ‍್ರೀಮಂತ. ಅವರು ತಮ್ಮ ಎಸ್ಟೇಟಿನಲ್ಲಿ ಹೆಂಡತಿ ಮತ್ತು ಹನ್ನೆರಡು ವರ್ಷ ವಯಸ್ಸಿನ ಮಗಳೊಂದಿಗೆ ಸುಖವಾಗಿದ್ದ ಕಾಲ ಅದು. ಒಂದು ದಿನ, ಚುರುಕು ಬುದ್ಧಿಯ ಚೆಲುವೆಯಾದ ಅವರ ಮಗಳನ್ನು ಯಾರೋ ದುಷ್ಟರು ಅಪಹರಿಸಿದರು. ಎಷ್ಟು ಪ್ರಯತ್ನಿಸಿದರೂ ಆಕೆಯನ್ನು ಪತ್ತೆಹಚ್ಚಲು ಸಾಧ್ಯವಾಗಲಿಲ್ಲ. ಕೊನೆ ಗೊಂದು ದಿನ, ಮಗಳ ಶವವನ್ನೇ ಇವರು ನೋಡಬೇಕಾಯಿತು! ಅಪಾರ ಸಂಪತ್ತು, ಆರೋಗ್ಯದಿಂದ ಕೂಡಿದ ತಾರುಣ್ಯ, ಸಮಾಜದಲ್ಲಿ ಮಾನ್ಯತೆ–ಇವೆಲ್ಲ ಇದ್ದರೂ, ಕ್ರ್ಯಾನ್ಸಿ ದಂಪತಿಗಳಿಗೆ ಈ ದುರ್ಘಟನೆಯಿಂದ ಜೀವನದಲ್ಲಿ ತೀವ್ರತರದ ಜುಗುಪ್ಸೆ ಬಂದೊದಗಿತು. ಧರ್ಮಭೂಮಿ ಭಾರತದಲ್ಲಿರುವ ಯಾರಾದರೂ ಸಂತರ ದರ್ಶನದಿಂದ ಮನಶ್ಶಾಂತಿ ದೊರತೀತೆಂದು ಹಂಬ ಲಿಸಿ, ಅವರು ಭಾರತಕ್ಕೆ ಬಂದರು. ಅನೇಕ ಪವಿತ್ರ ಸ್ಥಾನಗಳನ್ನು ಸಂದರ್ಶಿಸಿ, ಅನೇಕ ಸಾಧು ಸಂತರನ್ನು ಭೇಟಿಯಾದರು. ಕೊನೆಗೆ ಶ‍್ರೀ ಸತ್ಯಸಾಯಿಬಾಬಾರ ಅಭಯ, ಆಶೀರ್ವಾದ ಅವರ ಪಾಲಿಗೆ ದೊರೆಯಿತು. ತಮ್ಮ ಕಷ್ಟಗಳನ್ನು ನಿವೇದಿಸಿಕೊಂಡಾಗ ಬಾಬಾರು “ತಂದೆ, ತಾಯಿಗಳು ಮಕ್ಕಳನ್ನು ಪ್ರೀತಿಸುವುದು ಸಹಜ. ಮಗುವಿನ ಅಭ್ಯುದಯವನ್ನು ಹೆತ್ತವರು ಬಯಸುವುದು ಸ್ವಾಭಾವಿಕ. ಆದರೆ ಪ್ರಪಂಚದಲ್ಲಿ ಮನುಷ್ಯನು ಅನುಭವಿಸುವ ಸುಖದುಃಖಗಳಿಗೆ ಕಾರ್ಯ ಕಾರಣ ಸಂಬಂಧವಿದೆ. ಪುಣಾನುಬಂಧದಿಂದ ಆ ಮಗು ಹನ್ನೆರಡು ವರ್ಷಗಳ ಕಾಲ ನಿಮ್ಮ ಪ್ರೇಮದ ಆಸರೆ ಪಡೆಯಿತು. ಆದರೆ ಆ ಮಗುವಿನ ಹಿಂದಿನ ಜನ್ಮದ ಕರ್ಮದಿಂದಾಗಿ ಅದು ಅಪಮೃತ್ಯುವಿಗೆ ಸಿಲುಕಿತು. ಆ ಶಿಶುವಿನ ದೇಹ ನಾಶವಾಗಿದ್ದರೂ ಆತ್ಮ ನಾಶವಾಗಿಲ್ಲ. ನಿಮಗೆ ಇನ್ನೂ ಆ ಶಿಶುವಿನ ಮೇಲಿನ ವ್ಯಾಮೋಹ ಹೋಗಿಲ್ಲ. ದೇವರ ಅನುಗ್ರಹದಿಂದ ಆ ಶಿಶುವು ಕೆಲವೇ ವರ್ಷಗಳಲ್ಲಿ ನಿಮ್ಮ ಮಗುವಾಗಿಯೇ ಜನಿಸುತ್ತದೆ. ಚಿಂತೆ ಬೇಡ” ಎಂದರು.

\newpage

ಕ್ರ್ಯಾನ್ಸಿ ದಂಪತಿಗಳಿಗೆ ಮನಃಶಾಂತಿ ದೊರೆಯಿತು. ಅವರು ಭಾರತದಲ್ಲೇ ನೆಲಸುವ ನಿರ್ಧಾರ ಮಾಡಿದರು. ತಮ್ಮ ಸಂಪನ್ಮೂಲಗಳ ಮೇಲ್ತನಿಖೆ ನೋಡಲು ಅವರು ಈಗ ವರ್ಷಕ್ಕೊಂದೆರಡು ಬಾರಿ ಇಟೆಲಿಗೆ ಹೋಗಿ ಬರುತ್ತಾರಷ್ಟೆ. ಇತ್ತೀಚೆಗೆ ಅವರಿಗೊಂದು ಹೆಣ್ಣುಮಗು ಹುಟ್ಟಿತು. ಆ ಸಂದರ್ಭದಲ್ಲಿ ಸಂದರ್ಶನ ನೀಡಿದ ಬಾಬಾರು “ಕ್ರ್ಯಾನ್ಸೀ, ಇಟೆಲಿಯಲ್ಲಿ ನೀನು ಕಳೆದುಕೊಂಡ ಮಗುವೇ ತಿರುಗಿ ನಿನ್ನ ಮಗುವಾಗಿ ಜನಿಸಿದೆ” ಎಂದರು. ಮಗುವಿನ ನಡೆನುಡಿ ಎಲ್ಲವೂ ಹಿಂದೆ ಕಳೆದುಕೊಂಡ ಮಗುವಿನಂತೆಯೇ ಇವೆ! ಅಂದು ಸತ್ತುಹೋದವಳೇ ಇಂದು ಹುಟ್ಟಿ ಬಂದುದೇ?


\section*{ಬನಾರಸ್​ನಲ್ಲಿ ಮರಣ, ಪೆರುವಿನಲ್ಲಿ ಪುನರ್ಜನನ!}

\addsectiontoTOC{ಬನಾರಸ್​ನಲ್ಲಿ ಮರಣ, ಪೆರುವಿನಲ್ಲಿ ಪುನರ್ಜನನ !}

ದಕ್ಷಿಣ ಅಮೆರಿಕದ ಎಮೆಸಾನ್ ಪ್ರಾಂತ್ಯದ ಪೆರು ಎಂಬಲ್ಲಿನ ಕಗ್ಗಾಡುಗಳಲ್ಲಿ, ಅಯಮಾರಾ ಇಂಡಿಯನ್ ನಿವಾಸಿಗಳು ವಾಸಿಸುತ್ತಾರೆ. ಅವರು ಎಷ್ಟು ಅನಾಗರಿಕರೆಂದರೆ ಅವರಿಗೆ ಅವರ ದೇಶದ ಹೆಸರು ಕೂಡ ತಿಳಿಯದು. ರಾಷ್ಟ್ರದ ರಾಜಕೀಯ ಭಾಷೆಯಾದ ಸ್ಪೇನಿಷ್ ಕೇಳಿದವರಲ್ಲ. ಅಂಥವರ ಒಂದು ಹಳ್ಳಿಯ ಇಗರ್ಜಿ ನಡೆಸಲು ಒಬ್ಬ ಗುರು ಲಿಮಾ ಪಟ್ಟಣದಿಂದ ನೇಮಕ\-ಗೊಂಡು ಬಂದ. ಅಲ್ಲಿ ಅವನೊಬ್ಬನೇ ನಾಗರಿಕ. ಒಮ್ಮೆ ಅವನಲ್ಲಿಗೆ ಅಯಮಾರಾ ದಂಪತಿಗಳು ತಮ್ಮ ಏಳು ವರ್ಷದ ಮುದ್ದು ಮಗನೊಡನೆ ಬಂದರು. “ಹುಡುಗನಿಗೆ ಏನೋ ಭೂತಬಾಧೆಯಿದೆ; ತನ್ನ ತಂದೆ ತಾಯಿ ಬನಾರಸ್​ನಲ್ಲಿದ್ದಾರೆ, ತಾನು ಅಲ್ಲಿಗೆ ಹೋಗಬೇಕು ಎಂದು ಆಗಾಗ ಅಳುತ್ತಾನೆ” ಎಂದರು. ಪಾದ್ರಿಯು ಆ ಹುಡುಗನೊಡನೆ “ಬನಾರಸ್ ಎಲ್ಲಿದೆ?” ಮುಂತಾದ ಪ್ರಶ್ನೆಗಳನ್ನು ಕೇಳಿದಕ್ಕೆ, “ಅದು ಇಂಡಿಯಾದಲ್ಲಿದೆ” ಎಂದು, ಅಲ್ಲಿರುವ ತನ್ನ ಬ್ರಾಹ್ಮಣ ತಂದೆ ತಾಯಿಗಳ ಹೆಸರು, ಅವರಿರುವ ಬೀದಿಯ ಹೆಸರು ಎಲ್ಲ ಹೇಳಿದ. ಮುಂದುವರಿದು “ನನ್ನ ತಂದೆ ನನ್ನ ಏಳನೇ ಹುಟ್ಟು ಹಬ್ಬದಂದು ಒಂದು ಆಟಿಕೆಯ ಗಾಡಿಯನ್ನು ಕೊಡುಗೆಯಾಗಿ ಇತ್ತಿದ್ದರು. ಆಡುತ್ತಾ ಆಡುತ್ತಾ ರಸ್ತೆಗೆ ಮಾತ್ರ ಇಳಿಯಬಾರದೆಂದು ಅವರು ಎಚ್ಚರಿಸಿದ್ದರೂ, ಒಂದು ದಿನ ನಾನು ಅವರ ಮಾತು ಮೀರಿ ರಸ್ತೆಗೆ ಇಳಿದೆ. ಕೂಡಲೇ ಒಂದು ಕಾರಿನ ಅಡಿ ಬಿದ್ದು ಸತ್ತು ಹೋಗಿದ್ದೆ. ಈಗ ನನ್ನ ಆ ಮನೆಗೆ ಹೋಗಲು ಹಾತೊರೆಯುತ್ತಿದ್ದೇನೆ” ಎನ್ನುತ್ತಾ ಕಂಬನಿಗರೆದ. ಏಳನೇ ವರ್ಷದಲ್ಲಿ ಅಂದು ಸತ್ತುಹೋಗಿದ್ದ ಆತ, ಇಂದೀಗ ತಿರುಗಿ ಹುಟ್ಟಿ ಬಂದು ಏಳು ವರ್ಷಗಳಾಗುತ್ತಲೇ ಅಲ್ಲಿಗೆ ಹೋಗಬೇಕೆಂದು ಕಾಡತೊಡಗಿದ್ದಾನೆ!

ಪಾದ್ರಿಗೆ ಅತ್ಯಾಶ್ಚರ್ಯವಾಯಿತು. ಹುಡುಗನ ಹೇಳಿಕೆಯಲ್ಲಿ ಯಾವ ತರದ ಕೃತಕತೆಯೂ, ಮೋಸವೂ ಇಲ್ಲವೆಂಬುದು ಸ್ಪಷ್ಟವಾಗಿ ತೋರಿದ್ದರಿಂದ, ಅವನ ಈ ವಿವರಗಳನ್ನೆಲ್ಲ ಪರಮಾನ\break ಶಾಸ್ತ್ರದಲ್ಲಿ ತುಂಬಾ ಆಸಕ್ತಿಯಿದ್ದ, ಲಿಮಾ ನಗರದ ತನ್ನ ಮಿತ್ರ ಅಗಸ್ಟೋ ಎಂಬ ನ್ಯಾಯವಾದಿಗೆ ಬರೆದು ತಿಳಿಸಿದ. ಅಗಸ್ಟೋ ಬನಾರಸ್ ಮುನ್ಸಿಪಾಲಿಟಿಗೆ, ‘ಆ ಬ್ರಾಹ್ಮಣ ಕುಟುಂಬದ ಯಜಮಾನನ ಹೆಸರು ವಿಳಾಸಗಳನ್ನೂ ಅವನ ಪತ್ನಿಯ ಹೆಸರನ್ನೂ, ಈಗ ಬಾಳಿಕೊಂಡಿರುವ ಮತ್ತು\break ಮೃತ\-ರಾದ ಅವರ ಮಕ್ಕಳ ವಿಚಾರಗಳನ್ನೂ ದಯಮಾಡಿ ತಿಳಿಸಬೇಕು. ಆ ಕುಟುಂಬದವರು\break ಅಗತ್ಯವಾಗಿ ತಿಳಿಯಬೇಕಾದ ಒಂದು ವಿಚಾರ ನನಗೆ ತಿಳಿದಿರುವುದರಿಂದ ಆ ಬಗ್ಗೆ ಹೆಚ್ಚಿನ ವಿವರ ಕೇಳಿರುವುದಾಗಿ’ ಪತ್ರಬರೆದ. ಹಲವು ತಿಂಗಳುಗಳ ಮೇಲೆ ಅವರಿಂದ ಉತ್ತರ ಬಂತು. ಏನಾಶ್ಚರ್ಯ! ಆ ಬ್ರಾಹ್ಮಣ ಕುಟುಂಬ ಆ ಬೀದಿಯಲ್ಲೇ ಇರುವುದು ನಿಜ! ಅಷ್ಟು ಮಾತ್ರವಲ್ಲ, ಅವರ ಹಿರೇ ಹುಡುಗ ತನ್ನ ಏಳನೇ ವರ್ಷದಲ್ಲಿ ಕಾರಿನ ಅಡಿಬಿದ್ದು ಮೃತನಾದುದೇ ಮೊದಲಾದ ವಿವರಗಳೆಲ್ಲವೂ ಅದರಲ್ಲಿದ್ದುವು. ಆ ವಿವರಗಳೆಲ್ಲ ಹುಡುಗ ಹೇಳಿದ್ದಕ್ಕೆ ಸಂಪೂರ್ಣ ಹೊಂದಿಕೆಯಾಗುತ್ತಿದ್ದವು. ಈ ಕತೆಯನ್ನು ಪಾದ್ರಿ ಲಿಮಾದ ಒಂದು ಪತ್ರಿಕೆಯಲ್ಲಿ ಪ್ರಕಟಿಸಿದ.

ಜೀನ್ ಬಗ್ಲರ್ ಎಂಬುವರು ತಿರುವಣ್ಣಾಮಲೆಯ ರಮಣಾಶ್ರಮದ ಪ್ರಕಟಣೆಯಾದ\break \enginline{‘Mountain Path’} ಎಂಬ ಮಾಸಪತ್ರಿಕೆಯಲ್ಲಿ \enginline{(No.\ 3, 1966) ‘From the Ganges to the Amazon’} ಎಂಬ ಶಿರೋನಾಮೆಯಲ್ಲಿ ಈ ವಿಚಾರ ಪ್ರಕಟಿಸಿದರು. “ಮತ್ತೇನಾಯಿತೆಂದು ನನಗೆ ತಿಳಿಯದು. ಏಕೆಂದರೆ ಅಷ್ಟರವರೆಗೆ ಲಿಮಾದಲ್ಲಿದ್ದ ನಾನು ಆ ಪಟ್ಟಣ ಬಿಟ್ಟುಬಂದೆ. ಈ ಘಟನೆ ನಡೆದದ್ದು ೧೯೪೬ಲ್ಲಿ” ಎಂದೂ ಬರೆದಿದ್ದರು.

ಹೌದು, ನಿಮಗೆ ನಂಬಲು ಕಷ್ಟವಾಗಬಹುದು. ಆದರೆ ಇದು ನಡೆದ ಘಟನೆ!\footnote{ ಕೋಟ ಶ‍್ರೀ ವಾಸುದೇವ ಕಾರಂತ, ‘ಪುನರ್ಜನ್ಮ’}


\section*{ಜನನ ಮತ್ತು ಅದಕ್ಕೆ ಮೊದಲು \protect\footnote{\engfoot{Birth and before: What people say about in hypnosis. David Chamberlin (A Clincial Psychologist).} ಡೆಕ್ಕನ್ ಹೆರಾಲ್ಡ್​ನಲ್ಲಿ \engfoot{Beth Anna Krier} ಅವರು ಮೇಲಿನ ಗ್ರಂಥವನ್ನು ಕುರಿತು ಬರೆದ ವಿಮರ್ಶೆಯ ಲೇಖನದಿಂದ ಇದನ್ನು ಸಂಗ್ರಹಿಸಲಾಗಿದೆ.}}

\addsectiontoTOC{ಜನನ ಮತ್ತು ಅದಕ್ಕೆ\break ಮೊದಲು}

ಅಮೆರಿಕದ ಮನೋವಿಜ್ಞಾನಿ ಚೆಂಬರ್​ಲಿನ್ ‘ಜನ್ಮ ಮತ್ತು ಅದಕ್ಕೆ ಮೊದಲು’ ಎಂಬ ಹೆಸರಿನ ತಮ್ಮ ಗ್ರಂಥದಲ್ಲಿ ಆಗತಾನೇ ಜನಿಸಿದ ಮಕ್ಕಳು ತಮಗಾಗುತ್ತಿರುವ ಅನುಭವಗಳನ್ನೂ ಸುತ್ತುಮುತ್ತಲಿನ ಘಟನೆಗಳನ್ನೂ ತಿಳಿದುಕೊಳ್ಳಬಲ್ಲರು ಎನ್ನುತ್ತಾರೆ. ಪ್ರಸವ ಕಾಲದಲ್ಲಿ ಡಾಕ್ಟರು ಉಪಯೋಗಿಸುವ ಉಪಕರಣಗಳಿಂದಾಗುವ ತೊಂದರೆ, ತಲೆಕೆಳಗಾಗಿ ಹಿಡಿದುಕೊಂಡಾ ಗಿನ ಭಾವನೆ, ಏಟನ್ನು ಕೊಟ್ಟು ಚೆನ್ನಾಗಿ ಅಲುಗಾಡಿಸುವಾಗಿನ ಅನುಭವ, ಹೊಸದಾಗಿ ಜಗತ್ತಿನ ಬೆಳಕನ್ನು ಕಂಡ ಶಿಶುವಿನ ಬಗ್ಗೆ ತಾಯಿಯ ಅಭಿಪ್ರಾಯ ಅನಿಸಿಕೆಗಳು–ಇವೆಲ್ಲವನ್ನೂ ಮಕ್ಕಳು ಅರಿಯಬಲ್ಲರು ಎಂಬುದು ಅವರು ಪ್ರಯೋಗದಿಂದ ಕಂಡುಕೊಂಡ ಸಂಗತಿಗಳು. ದೀರ್ಘ ಕಾಲ ಈ ಸಂಬಂಧವಾದ ಸಂಶೋಧನೆಯಲ್ಲಿ ನಿರತರಾಗಿ ಸುಮಾರು ನೂರುಕೇಸುಗಳ ವಿಸ್ತೃತ ಹಾಗೂ ಆಳ ಅಧ್ಯಯನದ ಫಲವಾಗಿ ಸಂಗ್ರಹವಾದ ವಿಚಾರಗಳನ್ನು ತಮ್ಮ ಗ್ರಂಥದಲ್ಲಿ ಚರ್ಚಿಸಿದ್ದಾರೆ. ವಶ್ಯಸುಪ್ತಿಯಲ್ಲಿ ಜನನ ಕಾಲದ ಅನುಭವಗಳನ್ನು ಅತ್ಯಂತ ವಿಸ್ತಾರವಾಗಿ ಚಾಚೂ ತಪ್ಪದೆ ಎಲ್ಲ ವಿವರಗಳನ್ನೂ ಪ್ರಯೋಗಕ್ಕೊಳಪಟ್ಟ ವ್ಯಕ್ತಿಗಳು ಹೇಳಲು ಸಮರ್ಥರಾಗಿದ್ದರು.

\newpage

ಸ್ಟೂವರ್ಟ್ ಒಬ್ಬ ವಯಸ್ಕ ವ್ಯಕ್ತಿ. ವಶ್ಯಸುಪ್ತಿಗೊಳಗಾದಾಗ ತನ್ನ ಜನನ ಕಾಲದ ಅನುಭವಗಳನ್ನು ವಿಸ್ತಾರವಾಗಿ ವಿವರಿಸಿದ. ಅದರ ಸಂಕ್ಷಿಪ್ತರೂಪ ಮಾತ್ರ ಇಲ್ಲಿದೆ:

‘ನನ್ನನ್ನು ತಾಯಿಯ ಗರ್ಭದಿಂದ ಕೆಳಕ್ಕೆ ತಳ್ಳುವಂಥ ಬಹಳಷ್ಟು ಒತ್ತಡ....ಸುತ್ತಲಿಂದಲೂ ತಳ್ಳಿದ ಅನುಭವ. ತನ್ನ ಪುಟ್ಟಕಾಲುಗಳಿಂದ ಒದೆಯಲು ಧೈರ್ಯವಿರಲಿಲ್ಲ. ಹಾಗೆ ಮಾಡಿದರೆ ಮತ್ತಷ್ಟು ನೋವೇ ಆಗುತ್ತಿತ್ತು. ಆದುದರಿಂದ ಚೆನ್ನಾಗಿ ಮುದುಡಿಕೊಂಡಿದ್ದೇನೆ. ನನ್ನ ತಲೆಯ ಮೇಲೂ ಒತ್ತಡ....ಯಾರೋ ಕೆಳಕ್ಕೆ ತಳ್ಳಿದಂತಾಗುತ್ತದೆ. ಆದರೆ ನಾನು ಚಲಿಸುತ್ತಿಲ್ಲ. ಸಿಕ್ಕಿಕೊಂಡಂ\-ತಾಗಿದೆ. ನನ್ನ ಭುಜಗಳನ್ನು ಅಲುಗಾಡಿಸುವಂತಿಲ್ಲ. ನನ್ನ ದವಡೆಗೆ ನೋವಾಗುತ್ತಿದೆ. ಡಾಕ್ಟರ್ ನನ್ನ ದವಡೆಯನ್ನು ಹಿಡಿದು ಎಳೆಯುತ್ತಿದ್ದಾರೆ. ಆದರೂ ಹೊರಗಡೆ ಬರಲಾಗುತ್ತಿಲ್ಲ. “ತಳ್ಳಿ” ಎಂದು ಡಾಕ್ಟರ್ ಎಲ್ಲರಿಗೂ ಕೂಗಿ ಹೇಳುತ್ತಿದ್ದಾರೆ. ತಳ್ಳಿ ತಳ್ಳಿ....ನನ್ನ ತಾಯಿ ಅಳುತ್ತಿದ್ದಾಳೆ. ಅವಳು ಸ್ವಲವೂ ಆರಾಮ (ರಿಲೇಕ್ಸ್​) ಆಗಿಲ್ಲ. ಸೆಟೆದುಕೊಂಡು ಬಿಗಿಯಾಗಿಯೇ ಇದ್ದಾಳೆ. ನಾನೂ ಬಿಗುವಾಗಿಯೇ ಇದ್ದೇನೆ. ಡಾಕ್ಟರ್ ನಿಜವಾಗಿ ಕೋಪಿಸಿಕೊಳ್ಳುತ್ತಿದ್ದಾರೆ– ಅವರ ನಿರೀಕ್ಷೆಯಂತೆ ನೇರವಾಗಿ ನನಗೆ ಹೊರಗಡೆ ಬರಲಾಗುತ್ತಿಲ್ಲವೆಂದು. ಮಗು ಬೇಗನೆ ಉಸಿರಾಡುವಂತಾಗಬೇಕು ಎಂದು ಅವರು ಹೇಳುತ್ತಿದ್ದಾರೆ. ಇದೀಗ ನನ್ನ ಭುಜ ಹೊರಕ್ಕೆ ಬಂದಿದ್ದೆ. ಈಗ ಹಿಂದಿನಷ್ಟು ನಾನು ಸೆಟೆದುಕೊಂಡಿಲ್ಲ. ಹಿಂದಿನಷ್ಟು ಜೋರಾಗಿ ಡಾಕ್ಟರ್ ನನ್ನನ್ನು ಎಳೆಯುತ್ತಿಲ್ಲ. ಇನ್ನೂ ಬೇಗನೆ ಬರಬೇಕಿತ್ತೆಂದು ಹೇಳುತ್ತಲೇ ಇದ್ದಾರೆ. ಆದರೆ ಆಗುತ್ತಿಲ್ಲ. ತಿರುಗಿ ಸಿಕ್ಕಿಕೊಂಡಿದ್ದೇನೆ, ನಾನು ಎಕ್ಸ್​ಟೆಂಡ್ ಆಗುತ್ತಿಲ್ಲವೆಂದು ಡಾಕ್ಟರ್ ಹೇಳುತ್ತಿದ್ದಾರೆ. ಹಾಗೆಂದರೇನೆಂದು ನನಗೆ\break ಅರ್ಥವಾಗುತ್ತಿಲ್ಲ.’

೧೯೭೫ರಲ್ಲಿ ಡಾ.\ ಚೇಂಬರ್​ಲಿನ್ ವ್ಯಕ್ತಿಯ ಗರ್ಭವಾಸ ಮತ್ತು ಜನನದ ಸಮಯಕ್ಕೆ\break ಸಂಬಂಧಿಸಿದ ಸ್ಮೃತಿಯಲ್ಲಿ ಆಸಕ್ತರಾದರು. ಇದಕ್ಕೆ ಕಾರಣ ಅವರ ಆರೈಕೆಗೊಳಗಾದ ವ್ಯಕ್ತಿಯೇ. ವಶ್ಯಸುಪ್ತಿಗೊಳಗಾದಾಗ ಆತ ಸದ್ಯದ ಸಮಸ್ಯೆಗೆ ಪರಿಹಾರವಾಗಿ ಜನನ ಹಾಗೂ ಶೈಶವದ ಘಟನೆಗಳನ್ನು\break ಸ್ಪಷ್ಟವಾಗಿ ಚೇಂಬರ್​ಲಿನ್​ಗೆ ಹೇಳಿದ್ದ. ಎರಡು ತಿಂಗಳ ಬಳಿಕ ಸ್ಯಾನ್ ಫ್ರಾನ್ಸಿಸ್ಕೋ ನಗರದ ಹೆರಿಗೆಯ ತಜ್ಞ ಡಾ. ಡೇವಿಡ್ ಬಿ. ಚೀಕ್ ಸುಮಾರು ಇಪ್ಪತ್ತು ವರ್ಷಗಳಿಂದ ಈ ಸಂಶೋಧನೆಯಲ್ಲಿ ನಿರತರಾದವರೆಂದು ತಿಳಿಯಿತು. ಅವರ ಬರಹಗಳನ್ನು ಓದಿದ ಬಳಿಕ ತನ್ನ ಸಂಶೋಧನೆ ಸರಿಯಾದ ಪಥದಲ್ಲಿದೆ ಎನ್ನಿಸಿತು ಡಾ. ಚೇಂಬರ್​ಲಿನ್ ಅವರಿಗೆ.

‘ವಶ್ಯಸುಪ್ತಿಯ ಮೂಲಕ ಮನೋವ್ಯಾಧಿ ಚಿಕಿತ್ಸಾ ವಿಧಾನ’ ಎಂಬ ಗ್ರಂಥದ ಕರ್ತೃ\break ಡಾ.\ ಡೇವಿಡ್ ಬಿ. ಚೀಕ್ ‘ಅಮೇರಿಕನ್ ಸೊಸೈಟಿ ಆಫ್ ಕ್ಲಿನಿಕಲ್ ಹಿಪ್ನಾಸಿಸ್​’–ಇದರ ಅಧ್ಯಕ್ಷರು. ಅವರ ಅಭಿಪ್ರಾಯದ ಪ್ರಕಾರ ನವಜಾತ ಶಿಶುಗಳು ತಮ್ಮ ಸುತ್ತುಮುತ್ತಲ ವಾತಾವರಣವನ್ನು ಚೆನ್ನಾಗಿ ಅರಿತುಕೊಳ್ಳುತ್ತವೆ. ಈ ಸಂಗತಿ ಹಿಂದಿನ ತಜ್ಞರುಗಳಿಗೆ ತಿಳಿದಿರಲಿಲ್ಲ. ಕೆಲವೊಮ್ಮೆ ಮಕ್ಕಳನ್ನು ಹೆರಿಗೆಯಾದೊಡನೆಯೆ ತಾಯಿಗೆ ತಿಳಿಸದೆ ಯಾವ ರೀತಿಯ ಉದ್ವೇಗವನ್ನುಂಟು ಮಾಡದೆ ಅವನ್ನು ಬೇರೆಡೆಗೆ ಒಯ್ಯುತ್ತಾರಷ್ಟೆ. ಹಾಗೆ ಬೇರೆ ಕಡೆ ಸಾಗಿಸಲ್ಪಡುವ ಶಿಶುಗಳಲ್ಲಿ ದತ್ತುಸ್ವೀಕಾರಕ್ಕೆ ಸಿದ್ಧರಾದ ಶಿಶುಗಳು ತಮ್ಮ ತಾಯಂದಿರಿಗೆ ತಾವು ಬೇಡವಾಗಿರುವುದನ್ನು ಸ್ಪಷ್ಟವಾಗಿ ತಿಳಿದು ಸಂಕಟಕ್ಕೊಳಗಾಗುತ್ತವೆಂದು ಡಾ. ಚೀಕ್ ಹೇಳುತ್ತಾರೆ. ಅವರು ದೀರ್ಘಕಾಲ ಅಧ್ಯಯನದ ಫಲವಾಗಿ ಕಂಡುಕೊಂಡ ಕೆಲವು ಸಂಗತಿಗಳಿವು:

೧.\ ಹೆರಿಗೆಯಾದೊಡನೆಯೇ ನವಜಾತ ಶಿಶುಗಳು ತಾಯಂದಿರು ತಮ್ಮ ಹತ್ತಿರ ಅಥವಾ ತಮ್ಮ ಬಗ್ಗೆ ಮಾತನಾಡುವುದನ್ನು ನಿರೀಕ್ಷಿಸುತ್ತವೆ. ಮಾತೆಯ ಸಾಂತ್ವನದಾಯಿಯಾದ ವಾತ್ಸಲ್ಯದ ಮಾತು ಮತ್ತು ಸ್ಪರ್ಶ ದತ್ತು ಸ್ವೀಕಾರಕ್ಕೆಂದು ಗೊತ್ತಾದ ಮಕ್ಕಳಿಗೆ ದೊರೆಯುವುದಿಲ್ಲ. ಒಂದೋ ಆ ತಾಯಂದಿರು ಅರಿವಳಿಕೆಯ ಸ್ಥಿತಿಯಲ್ಲಿರುತ್ತಾರೆ. ಇಲ್ಲವಾದರೆ ಹುಟ್ಟಿದ ಮಗುವನ್ನು ಕೂಡಲೇ ಇತರ ಕೋಣೆಗೆ ಅಥವಾ ನರ್ಸರಿಗೆ ಕೊಂಡೊಯ್ಯುತ್ತಾರೆ. ತಾಯಿಯ ಸಾಮೀಪ್ಯವೇ ಅವುಗಳಿಗೆ ದೊರೆಯುವುದಿಲ್ಲ.

೨.\ ಹೆರಿಗೆಯಾಗುವಾಗ ಅಥವಾ ಆ ಬಳಿಕ ತಾಯಂದಿರು ನೋವಿನಿಂದ ನರಳು\-ತ್ತಿರು\-ವುದನ್ನೂ ಅಳುತ್ತಿರುವುದನ್ನೂ ಕಂಡ ಶಿಶುಗಳ ಮನಸ್ಸಿನಲ್ಲಿ ತಾವು ತಮ್ಮ ತಾಯಂದಿರಿಗೆ ಅಷ್ಟೊಂದು ಕಷ್ಟವಿತ್ತು ಬೇನೆಗೆ ಕಾರಣವಾದೆವಲ್ಲ ಎಂಬ ತಪ್ಪಿತಸ್ಥ ಮನೋಭಾವ ಉಳಿದುಬಿಡುತ್ತದೆ. ತಾಯಿ ತಮ್ಮಿಂದ ಸಂತುಷ್ಟಳು ಎಂಬುದನ್ನು ಕಂಡಾಗ ಮಗುವಿನ ಮನಸ್ಸಿಗೆ ಶಾಂತಿ ಮಾತ್ರವಲ್ಲ ತಪ್ಪಿತಸ್ಥ ಭಾವನೆಯಿಂದ ಅದು ದೂರವಾಗುವ ಸಂಭವವಿದೆ.

೩.\ ದತ್ತು ಸ್ವೀಕಾರ ಮಾಡಿರಲಿ ಇಲ್ಲದಿರಲಿ ನರ್ಸರಿಗೆ ಕೊಂಡೊಯ್ದ ಬಳಿಕ ಅಲ್ಲಿನ ಜನ ತಮ್ಮನ್ನು ಅನುಕಂಪೆಯಿಂದ ಪ್ರೀತಿ ವಿಶ್ವಾಸಗಳಿಂದ ನೋಡಿಕೊಂಡರೇ ಇಲ್ಲವೇ ಎಂಬುದನ್ನು ಶಿಶುಗಳು ತಿಳಿಯಬಲ್ಲವು. ಒಂದೆಡೆ ಒಟ್ಟಾಗಿರಿಸಿದ ಶಿಶುಗಳಲ್ಲಿ ಪರಸ್ಪರ ಯೋಚನೆಗಳ ವಿನಿಮಯ\-ವಿರುವಂತೆ ಕಂಡುಬಂದಿದೆ. ಸುಪ್ತಿ ಆವಾಹನೆಯ ಮೂಲಕ ವ್ಯಕ್ತಿಯೊಬ್ಬನನ್ನು ಶೈಶವದ ಸ್ತರಕ್ಕೆ ಕೊಂಡೊಯ್ದಾಗ ದಾದಿಯರ ನಯವಾದ ಮಾತು ಪ್ರೀತಿಯ ಮೃದುಸ್ಪರ್ಶಗಳ ನೆನಪನ್ನು ಮಾಡಿ\-ಕೊಂಡವರಿದ್ದಾರೆ.

೪.\ ಅನಾಥಾಲಯ ಅಥವಾ ವಿಶಿಷ್ಟ ಶಿಶುಸಂಗೋಪನಾ ಕೇಂದ್ರಗಳಿಂದ ದತ್ತುಸ್ವೀಕಾರ ಮಾಡಿದ ಮನೆಗಳಿಗೆ ಹೋಗುವ ಮುನ್ನ ಶಿಶುಗಳು ಬಹಳ ಅಸಮಾಧಾನ ಮತ್ತು ರೋಷವನ್ನು ಬೆಳಸಿಕೊಳ್ಳುತ್ತವೆ. ತಮ್ಮ ತಾಯಂದಿರೊಡನೆ ಆಗಲೆ ಹೊಂದಿಕೊಂಡಿದ್ದ ಆ ಶಿಶುಗಳಿಗೆ ತಿರುಗಿ ನೂತನ ವಾತಾವರಣ–ಜನಗಳೊಡನೆ ಹೊಂದಿಕೊಳ್ಳಲು ಬಹಳಷ್ಟು ತೊಂದರೆಯಾಗುವುದು.

‘ದರ್ಶನ ಮತ್ತು ಶ್ರವಣೇಂದ್ರಿಯಗಳಾಗಲಿ ನರಮಂಡಲವಾಗಲಿ ಸರಿಯಾದ ರೀತಿಯಲ್ಲಿ ಬೆಳವಣಿಗೆಯಾಗದೆ ಕಂಡುಕೇಳಿದ ಸಂಗತಿಗಳನ್ನು ಮನಸ್ಸಿನಲ್ಲಿ ಸಂಗ್ರಹಮಾಡಿಟ್ಟುಕೊಳ್ಳುವುದು ಅಸಾಧ್ಯ ಎಂಬ ಭಾವನೆ ಹಿಂದೆ ಸಾರ್ವತ್ರಿಕವಾಗಿ ತಜ್ಞರಲ್ಲಿ ದೃಢವಾಗಿತ್ತು. ಕೆಲವು ಶಿಶುಗಳಲ್ಲಿ ಜನನ ಕಾಲದ ಅನುಭವಗಳನ್ನು ತೀವ್ರವಾಗಿ ಅರಿಯುವ ಸಾಮರ್ಥ್ಯ ಒಂದು ಆಕಸ್ಮಿಕ ಎಂಬ ಭಾವನೆ ರೂಢಮೂಲವಾಗಿತ್ತು. ಆದರೆ ಇದಕ್ಕೆ ವಿರೋಧವಾದ ಸಾಕ್ಷ್ಯಾಧಾರಗಳು ಏಕಪ್ರಕಾರವಾಗಿ ಸಂಗ್ರಹವಾಗುತ್ತ ಬಂದ ಬಳಿಕ ಆ ಹೇಳಿಕೆಗಳ ಸತ್ಯಾಂಶವನ್ನು ತಿಳಿಯಲು ಯತ್ನಿಸಿ ಯಶಸ್ವಿಯಾದೆವು. ಸುಪ್ತಮನಸ್ಸಿನಲ್ಲಿ ಸ್ಪಷ್ಟವಾಗಿ ಜನನ ಸಮಯದ ಸ್ಮೃತಿ ಅಡಗಿದೆ ಎಂಬುದು ವಶ್ಯಸುಪ್ತಿಯ ಪ್ರತಿ\-ಗಮನ ವಿಧಾನದಿಂದ ಇದೀಗ ದೃಢಪಟ್ಟಿದೆ.’ ಎನ್ನುತ್ತಾರೆ ಡಾ. ಚೀಕ್.

ಹಲವು ಮಂದಿ ವಯಸ್ಕರ ಮಾನಸಿಕ ತುಮುಲ ಹಾಗೂ ವ್ಯಗ್ರತೆಗಳ ಸಂಕೀರ್ಣ ಸಮಸ್ಯೆಗಳ ಮೂಲ ಜನನಕಾಲ ಮತ್ತು ಶೈಶವದಲ್ಲಿ ಅವರುಗಳು ಎದುರಿಸಿದ ಕಷ್ಟ ಸಂಕಟಗಳ ಸ್ಮೃತಿಯೇ ಆಗಿದೆ ಎಂಬುದನ್ನು ಇದೀಗ ಮನೋವಿಜ್ಞಾನಿಗಳು ತಿಳಿದುಕೊಳ್ಳುತ್ತಿದ್ದಾರೆ. ಶಸ್ತ್ರವೈದ್ಯರು ತಮ್ಮ ಇಕ್ಕಳದಿಂದ ಶಿಶುವನ್ನು ಒರಟಾಗಿ ಹೊರಗೆಳೆಯಲು ಯತ್ನಿಸಿದ ಕಾರ್ಯದಲ್ಲಿ ವಯಸ್ಕನ ತಲೆನೋವಿನ ಮೂಲ ಅಡಗಿತ್ತು. ಹೆರಿಗೆಯ ಕಾಲದಲ್ಲಿ ತಾಯಿ ಮತ್ತು ಸುತ್ತುಮುತ್ತಲ ಜನ ತೋರಿದ ಗಾಬರಿ ದಿಗಿಲುಗಳು ವಯಸ್ಕನ ಅಸ್ತಮಾ ನರಳಾಟಕ್ಕೆ ಕಾರಣವಾದವು. ತಾಯಿಯು ಮಗುವಿಗೆ ಮೊಲೆಹಾಲನ್ನು ಕೊಡಲು ನಿರಾಕರಿಸಿದ ಘಟನೆ ಮುಂದೆ ಹುಡುಗನ ಜೀರ್ಣಾಂಗದ ನ್ಯೂನತೆಗೆ ಕಾರಣವಾಯಿತು. ಹುಟ್ಟಿನ ಸಮಯದಲ್ಲಿ ಡಾಕ್ಟರರೊಬ್ಬರು ಮಾಡಿದ ಆತಂಕಕಾರಿಯಾದ ಟೀಕೆಯಿಂದ ವಯಸ್ಕನೊಬ್ಬ ನರಮಂಡಲಗಳ ದೌರ್ಬಲ್ಯದಿಂದ ನರಳುವಂತಾಯಿತು.

ಡಾ.\ ಚೀಕ್ ಅವರ ಸಿದ್ಧಾಂತದ ಸತ್ಯತೆಯನ್ನು ಪರಿಶೀಲಿಸಲು ಡಾ. ಚೇಂಬರ್​ಲಿನ್ ಯತ್ನಿಸಿದರು. ಶಿಶುವಿನ ಮನಸ್ಸು ಜನನಕಾಲದಿಂದಲೇ ಚಟುವಟಿಕೆಯಿಂದ ಕೂಡಿರುತ್ತದೆಂಬುದು ಹಲವು ಭೌತವಾದಿಗಳಿಗೆ ನುಂಗಲಾರದ ತುತ್ತಾಗಬಹುದು. ‘ಮೆದುಳಿನ ಚಟುವಟಿಕೆಯನ್ನು ನಾವು ಆಗ ಗಮನದಲ್ಲಿಟ್ಟುಕೊಂಡಿರುವುದಿಲ್ಲ. ಮಿದುಳಿನ ಬೆಳವಣಿಗೆಯನ್ನು ಹೊಂದಿಕೊಂಡು ಈ ಸ್ಮರಣೆ ಎಂದಲ್ಲ.\footnote{\engfoot{What we were dealing with is mind. It is a non-physical aspect of every person.}\hfill\engfoot{ —Dr. Chamberlin.}} ಮನಸ್ಸಿಗೆ ಅಭೌತಿಕವಾದ ಆಯಾಮವಿದೆ’ ಎನ್ನುತ್ತಾರೆ ಚೇಂಬರ್​ಲಿನ್.


\section*{ಅಂದಿನ ಭಾಷೆ ಇಂದಿಗೆ}

\addsectiontoTOC{ಅಂದಿನ ಭಾಷೆ ಇಂದಿಗೆ}

ಲಾರೆಂಟ್ ಹೇ ತಂದೆತಾಯಿಗಳ ಏಕಮಾತ್ರ ಪುತ್ರಿ. ವಯಸ್ಸು ಇನ್ನೂ ಆರು ವರುಷ ಮಾತ್ರ. ತಂದೆತಾಯಿಗಳು ಕೆಲಸಕ್ಕಾಗಿ ಹೊರಗಡೆ ಹೋದಾಗ ಮನೆಯಲ್ಲಿ ಆಕೆ ಒಬ್ಬಳೇ ಇರುತ್ತಿದ್ದಳು. ಯಾವಾಗಲೂ ತನ್ನ ಪಾಡಿಗೆ ತಾನು ಹಾಡಿಕೊಳ್ಳುವುದು ಅವಳ ಸ್ವಭಾವ. ಎಷ್ಟೋ ವೇಳೆ\break ತಮಗೆ ಪರಿಚಿತವಿಲ್ಲದ ಶಬ್ದಗಳನ್ನು ಕೇಳಿದರೂ, ಏನೋ ಅರ್ಥಹೀನ ಶಬ್ದಗಳನ್ನು ಉಚ್ಚರಿಸು\-ತ್ತಿದ್ದಾಳೆಂದೇ ಆಕೆಯ ತಂದೆತಾಯಿಗಳು ತಿಳಿದುಕೊಂಡಿದ್ದರು. ಆ ಕಡೆಗೆ ಅಷ್ಟಾಗಿ ಅವರು ಗಮನವಿತ್ತಿರಲಿಲ್ಲ. ಅವರ ಕುಟುಂಬದ ಸ್ನೇಹಿತರಾದ ಪಾದ್ರಿಯೊಬ್ಬರು ಅವರ ಮನೆಗೆ ಬಂದಾಗ, ಅವರು ಆಶ್ಚರ್ಯದಿಂದ ಹೌಹಾರುವ ಪ್ರಸಂಗ ಒದಗಿತು. ಪಾದ್ರಿಗಳು ಬಹುಕಾಲ ಪೌರಸ್ತ್ಯ ದೇಶಗಳಲ್ಲಿದ್ದು ಆಗತಾನೇ ಹಿಂದಿರುಗಿದ್ದರು. ಅವರು ಲಾರೆಂಟ್​ಳ ತಂದೆತಾಯಿಗಳ ಆಹ್ವಾನದ ಮೇರೆಗೆ, ಒಂದು ದಿನ ಉಪಾಹಾರ ಸ್ವೀಕರಿಸುತ್ತಿದ್ದಾಗ ಮಾತುಕತೆಯಾಡುತ್ತಿದ್ದರು. ಮೂಲೆಯಲ್ಲಿ ಎಂದಿನಂತೆ ಲಾರೆಂಟ್ ಏನೇನೋ ಹಾಡುಗಳನ್ನು ಹೇಳಿಕೊಳ್ಳುತ್ತಿದ್ದಳು. ಚಕಿತರಾದ ಆ ಪಾದ್ರಿಗಳು ಮಗುವನ್ನು ಸಮೀಪಿಸಿ ಕ್ಷಣಕಾಲ ಅವಳ ಹಾಡುಗಳನ್ನು ಕೇಳಿ ‘ಲಾರೆಂಟ್ ಪರ್ಶಿಯನ್ ಭಾಷೆಯನ್ನು ಎಲ್ಲಿ ಕಲಿತಳು?’ ಎಂದು ಕೇಳಿದರು. ಲಾರೆಂಟಳ ತಂದೆತಾಯಿಗಳು ‘ಪರ್ಶಿಯನ್​!’ ಎಂದು ಉಚ್ಚರಿಸಿ, ಆಶ್ಚರ್ಯದಿಂದ ಸ್ತಂಭಿತರಾದರು. ಪಾದ್ರಿಗಳು ಆ ಬಳಿಕ ಹೇಳಿದರು: ‘ಆಕೆ ಆಡುತ್ತಿರುವುದು ಆಧುನಿಕ ಪರ್ಶಿಯನ್ ಭಾಷೆಯಲ್ಲ. ಮಹಾಕಾವ್ಯ ಯುಗದ ಪರ್ಶಿಯನ್.’

ಲಾರೆಂಟಳ ತಂದೆತಾಯಿಗಳು ಈ ಮಾತನ್ನು ನಂಬದಾದರು. ಅವರ ಕುಟುಂಬದ ಬಂಧುಗಳೆಲ್ಲ ಯುರೋಪು ದೇಶದವರು. ಪರ್ಶಿಯನ್ ಭಾಷೆಯನ್ನು ತಿಳಿದ ಯಾರೂ ತಮ್ಮ ಮಗಳ ಜೊತೆ ಇರಲಿಲ್ಲ. ಹಾಗಿರುವಾಗ ಇದು ಹೇಗೆ ಸಾಧ್ಯ ಎಂಬ ನಿಲುವು ಅವರದ್ದು. ಹಾಗಾಗಿ ಹುಡುಗಿ ಏಕಾಂತದಲ್ಲಿ ಹಾಡಿಕೊಳ್ಳುತ್ತಿದ್ದಾಗಲೇ ಟೇಪ್​ರೆಕಾರ್ಡರಿನಲ್ಲಿ ಅವಳು ಉಚ್ಚರಿಸಿದ ಪದಗಳನ್ನೂ, ಹಾಡುಗಳನ್ನೂ ಸಂಗ್ರಹಿಸಿ ಭಾಷಾತಜ್ಞರಿಗೆ ತೋರಿಸಿದರು. ಅವರ ಅಭಿಪ್ರಾಯದಂತೆ ‘ಅವಳ ಮಾತುಗಳು ಹಳೆಯ ಪರ್ಶಿಯನ್ ಭಾಷೆಯದ್ದು; ಹಾಡುಗಳು ಮಾತ್ರ ಪರ್ಶಿಯನ್ ಕಾವ್ಯಗಳದ್ದು.’ ಆ ಬಗ್ಗೆ ಹುಡುಗಿಯನ್ನೇ ಪ್ರಶ್ನಿಸಿದಾಗ ಆಕೆ ತಿಳಿಸಿದ್ದಿಷ್ಟು: ‘ದೂರದ ದೇಶದಲ್ಲಿ ಬಹಳ ಹಿಂದಿನ ಕಾಲದಲ್ಲಿ ನಾವು ಮಾತನಾಡುತ್ತಿದ್ದ ರೀತಿ ಅದು.’ ಏನಾಶ್ಚರ್ಯ!

ಎಚ್ಚರದ ಸ್ಥಿತಿಯಲ್ಲಿ ಶೈಶವದಿಂದ ಕಲಿತ ಭಾಷೆಯಲ್ಲೇ ಎಲ್ಲರೂ ಮಾತನಾಡುತ್ತಾರಷ್ಟೆ. ಆದರೆ ವಶ್ಯಸುಪ್ತಿಗೊಳಗಾದ ಕೆಲವರು ಶೈಶವ ಬಾಲ್ಯ ತಾರುಣ್ಯದಲ್ಲಾಗಲೀ, ಆ ಬಳಿಕವಾಗಲಿ ಕಂಡರಿಯದ, ಕೇಳಿ ತಿಳಿದುಕೊಳ್ಳದ ವಿದೇಶಿ ಭಾಷೆಯಲ್ಲಿ ಮಾತನಾಡಿದ ಘಟನೆಗಳಿವೆ! ಮಾತ ನಾಡುವುದೆಂದರೆ ಕೆಲವೊಂದು ಶಬ್ದಗಳನ್ನೋ, ವಾಕ್ಯಪುಂಜವನ್ನೋ ಉಚ್ಚರಿಸುವುದಷ್ಟೇ ಅಲ್ಲ. ವಶ್ಯಸುಪ್ತಿಗೊಳಗಾದಾಗ ಆ ವಿದೇಶೀ ಭಾಷೆಯಲ್ಲಿ ನಿರರ್ಗಳವಾಗಿ ಸಂಭಾಷಣೆ ಮಾಡುವ\break ಸಾಮರ್ಥ್ಯವಿರುತ್ತದೆ. ಕೇಳಿದ ಪ್ರಶ್ನೆಗೆ ಆ ಭಾಷೆಯಲ್ಲೇ ಉತ್ತರಿಸುತ್ತಾರೆ. ಯಾವುದೋ ಒಂದು ಕಾಲದಲ್ಲಿ ಒಂದು ವಿಶಿಷ್ಟ ಪ್ರದೇಶದಲ್ಲಿ ಬಳಕೆಯಲ್ಲಿರುವ ಭಾಷೆಯನ್ನೇ ಅವರು ಮಾತಾಡುತ್ತಾರೆ. ಅವರು ಹಿಂದಿನ ಜನ್ಮಗಳ ಅನುಭವವನ್ನಾಧರಿಸಿಯೇ ಅದು ಸಾಧ್ಯವಾದುದರಿಂದ ಇದೂ\break ಜನ್ಮಾಂತರದ ಘಟನೆಗಳಿಗೊಂದು ಪುರಾವೆಯನ್ನೊದಗಿಸುತ್ತದೆ. ಉದಾಹರಣೆಗೆ ಈಗ ಇಂಗ್ಲೆಂಡಿನಲ್ಲಿರುವ ಜಾನ್​ಫಾರ್ಕ್ ವಶ್ಯಸುಪ್ತಿಯಲ್ಲಿ ಮನಸ್ಸಿನಾಳಕ್ಕೆ ಮುಳುಗಿ ಕನ್ನಡ ಮಾತಾಡಲು\break ತೊಡಗಿದನೆನ್ನಿ. ಅದು ಯಾವ ಕನ್ನಡ? ಹದಿನೆಂಟನೇ ಶತಮಾನದ ದಕ್ಷಿಣಕನ್ನಡ ಜಿಲ್ಲೆಯ ಕುಂದಾಪುರದ ಕನ್ನಡ ಎಂದು ತಜ್ಞರು ಕಂಡುಹಿಡಿಯುತ್ತಾರೆನ್ನಿ. ಎಂದರೆ ಜಾನ್​ಫಾಕ್ಸ್ ಒಮ್ಮೆ ಕುಂದಾಪುರದಲ್ಲಿ ನಾಗಪ್ಪಯ್ಯ ಕಾರಂತನೋ, ವಿಟ್ಠಲ ಶೆಟ್ಟಿಯೋ ಅಥವಾ ಯಾರೋ ಒಬ್ಬ ಪ್ರಜೆಯಾಗಿದ್ದಾನೆಂದು ಅರ್ಥವಾಗುತ್ತದಷ್ಟೆ. ನಿಜ, ವಶ್ಯಸುಪ್ತಿಯಲ್ಲಿದ್ದ ವ್ಯಕ್ತಿ ತನ್ನ ಆಗಿನ ಹೆಸರು, ಜಾತಿ, ವೃತ್ತಿಗಳನ್ನೆಲ್ಲ ಹೇಳಬಲ್ಲ.

ಇಂಥ ಘಟನೆಗಳನ್ನೇ ಆಂಗ್ಲ ಭಾಷೆಯಲ್ಲಿ \enginline{Responsive Xenoglossy} ಎನ್ನುತ್ತಾರೆ.\break ಇಂಥ ಅನೇಕ ಘಟನೆಗಳನ್ನು ಸ್ಟಿವನ್​ಸನ್ ಕಲೆಹಾಕಿದ್ದಾರೆ.


\section*{ವಶ್ಯಸುಪ್ತಿಯ ಅಮೂಲ್ಯ ನೆರವು}

\vskip -6.2pt\addsectiontoTOC{ವಶ್ಯಸುಪ್ತಿಯ ಅಮೂಲ್ಯ ನೆರವು}

ಬೊಂಬಾಯಿಯಿಂದ ಪ್ರಕಟವಾಗುವ ಆಂಗ್ಲಭಾಷೆಯ ‘ಮಿರರ್​’ ಮಾಸ ಪತ್ರಿಕೆಯಲ್ಲಿ ಸೆಪ್ಟೆಂಬರ್ 1983ರ ಲೇಖನ ಕುತೂಹಲಕಾರಿಯಾಗಿದೆ. ಸುಪ್ತಿಶಾಸ್ತ್ರಜ್ಞ ಪ್ರೊ.\ ಜೆ.\ ವಿ.\ ರಾವ್ ತಮ್ಮ ಪ್ರಯೋಗಗಳನ್ನು ಕುರಿತು ಅಲ್ಲಿ ವಿವರಣೆ ನೀಡಿದ್ದಾರೆ–‘ವಶ್ಯಸುಪ್ತಿಗೊಳಪಡಿಸಿ ವ್ಯಕ್ತಿ ತನ್ನ ಹಿಂದಿನ ಅಭ್ಯಾಸಗಳನ್ನು ತೊರೆಯುವಂತೆ ಮಾಡಬಹುದು. ವಶ್ಯಸುಪ್ತಿಗೊಳಪಟ್ಟ ಗಂಡಸು ಅಥವಾ ಹೆಂಗಸನ್ನು ತಮ್ಮ ಹಿಂದಿನ ಜನ್ಮದ ನೆನಪು ಮಾಡಿಕೊಳ್ಳುವಂತೆ ಮಾಡಬಹುದು. ಕೆಲವು ವರ್ಷಗಳ ಹಿಂದೆ, ಬೊಂಬಾಯಿಯ ಅಂಧೇರಿಯಲ್ಲಿ ನಾನು ವೇದಿಕೆಯ ಮೇಲೆ ಪ್ರದರ್ಶನ ನೀಡುತ್ತಿದ್ದೆ. ಆಗ ಹದಿನೆಂಟುವರ್ಷ ವಯಸ್ಸಿನ ಕ್ಯಾಥೋಲಿಕ್ ಸಂಪ್ರದಾಯದ ತರುಣಿಯನ್ನು, ವಶ್ಯಸುಪ್ತಿ\-ಗೊಳಪಡಿಸ\-ಬೇಕಾದ ಸಂದರ್ಭ ಬಂತು. ನಾನು ಆಕೆಯನ್ನು ತನ್ನ ಹಿಂದಿನ ಜನ್ಮದ ಕಾಲಕ್ಕೆ ಹೋಗುವಂತೆ ಸೂಚಿಸಿದೆ. ಇದ್ದಕ್ಕಿದ್ದಂತೆ ಆಕೆ ಅರ್ಥವಾಗದ ಭಾಷೆಯಲ್ಲಿ ಮಾತನಾಡಿದಾಗ ನಾನು ಅವಾಕ್ಕಾದೆ. ವೇದಿಕೆಯ ಸಮೀಪ ಮೊದಲ ಸಾಲಿನಲ್ಲಿ ಕುಳಿತಿದ್ದ ಇಟೆಲಿಯ ಒಬ್ಬ ಪಾದ್ರಿಯವರು ವೇದಿಕೆಗೆ ಬಂದು–ಆಕೆ ಇಟಾಲಿಯನ್ ಭಾಷೆ ಮಾತನಾಡುತ್ತಿದ್ದಾಳೆಂದು ತಿಳಿಸಿದರು. ಮುಸಲೋನಿ ಇಟಲಿಯನ್ನು ಆಳುತ್ತಿದ್ದ ಸಂದರ್ಭದಲ್ಲಿ ನಿರ್ದಿಷ್ಟ ಹಳ್ಳಿಯೊಂದರಲ್ಲಿ ಮಾತನಾಡುತ್ತಿದ್ದ, ಆದರೆ ಈಗ ವಾಡಿಕೆಯಲ್ಲಿಲ್ಲದ, ಇಟಾಲಿಯನ್ ಭಾಷೆಯನ್ನು ಆಕೆ ಮಾತನಾಡುತ್ತಿದ್ದಾಳೆಂದು, ಆ ಪಾದ್ರಿಗಳು ತಿಳಿಸಿದರು. ವಶ್ಯಸುಪ್ತಿಗೊಳಪಟ್ಟ ತರುಣಿ ತಾನು ಈ ಹಿಂದೆ ಗಂಡಸಾಗಿದ್ದುದಾಗಿ ತಿಳಿಸಿದಳು. ಅಷ್ಟು ಮಾತ್ರವಲ್ಲ ತನ್ನ ಹಿಂದಿನ ಇಟಾಲಿಯನ್ ಪೋಷಕರ ಹೆಸರನ್ನೂ ತಿಳಿಸಿದಳು. ಕೊನೆಗೆ ರಸ್ತೆ ಅಪಘಾತ ಒಂದರಲ್ಲಿ ತಾನು ಮರಣ ಹೊಂದಿದುದಾಗಿಯೂ ತಿಳಿಸಿದಳು. ವಶ್ಯಸುಪ್ತಿಯಿಂದ ಪುನಃ ಮುಂಚಿನ ಸ್ಥಿತಿಗೆ ಬಂದಾಗ ಆ ತರುಣಿಗೆ ತಾನು ತಿಳಿಸಿದ ವಿಚಾರದ ಒಂದು ತುಣುಕೂ ಸಹ ನೆನಪಿರಲಿಲ್ಲ! ತಾನು ಎಂದೂ ಭಾರತದಿಂದ ಹೊರದೇಶಗಳಿಗೆ ಹೋಗಿರಲಿಲ್ಲವೆಂದೂ, ಇಟಾಲಿಯನ್ ಭಾಷೆಯ ಶಬ್ದದ ಉಚ್ಚಾರ ಹೇಗಿರುವುದೆಂಬುದರ ಸುಳಿವೂ ತನಗೆ ತಿಳಿಯದೆಂದೂ ಆಗ ಆಕೆ ಪ್ರಾಮಾಣಿಕವಾಗಿ ತಿಳಿಸಿದಳು. ಆಕೆ ಇಟಾಲಿಯನ್ ಭಾಷೆ ಮಾತನಾಡಿದುದು ಸತ್ಯ. ಇಟಾಲಿಯನ್ ಪಾದ್ರಿಯೊಬ್ಬರು ಇದನ್ನು ಖಚಿತಪಡಿಸಿರುವುದರಿಂದ, ಇಲ್ಲಿ ಅನುಮಾನಕ್ಕೆ ಎಡೆಯಿಲ್ಲ. ಈ ಪಾದ್ರಿಗಳು ಸಭೆಯಲ್ಲಿದ್ದುದು ನನ್ನ ಅದೃಷ್ಟ.’ ಪ್ರಸಕ್ತ ಪ್ರಸಂಗದಿಂದ ನಾವು ಏನನ್ನು ನಿರ್ಧರಿಸಬಹುದು?

‘ಇಂತಹ ಘಟನೆಗಳನ್ನು ವಿಶದವಾಗಿ ಪರಿಶೀಲಿಸಲು ಅವಕಾಶ ಒದಗಿದರೆ ಆ ಘಟನೆಗಳು ಸತ್ಯವಾಗಬಹುದೆಂದು ನೀವು ಭಾವಿಸುವಿರಾ?’ ಎಂದು ಪ್ರಶ್ನಿಸಿದಾಗ ಪ್ರೊ. ರಾವ್ ನುಡಿದರು: ‘ಇದರಲ್ಲಿ ನನಗೆ ಸ್ವಲ್ಪವೂ ಸಂಶಯವಿಲ್ಲ. ನಮ್ಮ ಜಾಗ್ರತ ಮನಸ್ಸಿನ ಆಚೆ ಸುಪ್ತ ಮನಸ್ಸಿದೆ. ಸ್ವಲ್ಪವೂ ಸಂಶಯಕ್ಕೆಡೆಕೊಡದಂತಹ, ನಮ್ಮ ಜೀವನದ ಎಲ್ಲ ನೆನಪುಗಳ ಉಗ್ರಾಣ ಇದಾಗಿದೆ. ಎಲ್ಲ–ಎಂದರೆ ಮನುಷ್ಯ ಹುಟ್ಟಿದಂದಿನಿಂದ ಕ್ಷಣಕ್ಷಣವೂ, ದಿನ ದಿನವೂ ಪಡೆದ ಎಲ್ಲ ಅನುಭವಗಳು ಜೀವನದ ಪ್ರತಿಯೊಂದು ಸಣ್ಣ ಪುಟ್ಟ ಘಟನೆಗಳು–ಇಲ್ಲಿ ದಾಖಲಾಗಿವೆ. ತಜ್ಞರು ಮಾಡುವುದಿಷ್ಟೆ–ಜಾಗ್ರತ ಮನಸ್ಸನ್ನು ದಾಟಿ ನೇರವಾಗಿ ಸುಪ್ತ ಮನಸ್ಸಿಗೆ ಸಲಹೆ ನೀಡುವುದು. ಹೆಚ್ಚು ಕಡಿಮೆ ನಾವೆಲ್ಲರೂ ವಶ್ಯಸುಪ್ತಿಗೊಳಪಟ್ಟಾಗ ನಮ್ಮ ಪೂರ್ವಜನ್ಮಸ್ಮರಣೆ ಮಾಡಿ ಕೊಳ್ಳಲು ಸಾಧ್ಯ. ಇದು ಹೇಗೆ ಸಾಧ್ಯವಾಗುತ್ತದೆ ಎಂಬುದು ಇನ್ನೂ ಗೂಢವಾದ ವಿಚಾರವಾಗಿದೆ!’

ಅವರೇ ಮುಂದುವರಿದು ಅಂತಹುದೇ ಇನ್ನೊಂದು ಘಟನೆಯನ್ನು ತಿಳಿಸಿದರು–‘ಒಮ್ಮೆ\break ಜೆಮ್​ಶೆಡ್​ಪುರದ ಪ್ರದರ್ಶನದಲ್ಲಿ ಗುಜರಾತಿನ ಹದಿಮೂರು ವರ್ಷದ ಬಾಲಕನನ್ನು ವಶ್ಯಸುಪ್ತಿ ಗೊಳಪಡಿಸಿದೆ. ಅವನ ಪೋಷಕರು ಬಹಳ ಹಿಂದಿನಿಂದ ಆ ನಗರದಲ್ಲಿ ನೆಲೆಸಿದ್ದರು. ವಶ್ಯಸುಪ್ತಿ ಗೊಳಪಡಿಸಿ ಬಾಲಕನ ಹಿಂದಿನ ಜನ್ಮದ ಬಗ್ಗೆ ಕೇಳಿದಾಗ, ತಾನು ಹಿಂದಿನ ಜನ್ಮದಲ್ಲಿ ಬೊಂಬಾಯಿಯಲ್ಲಿ ಜನಿಸಿದ್ದುದಾಗಿ ತಿಳಿಸಿದ್ದಲ್ಲದೆ, ಜುಹುನಲ್ಲಿದ್ದ ಬಂಗಲೆಯೊಂದರ ವಿಳಾಸವನ್ನೂ ನೀಡಿದ. ತಾನು ಒಂಭತ್ತನೆಯ ವಯಸ್ಸಿನವನಿದ್ದಾಗ ಸಮುದ್ರದಲ್ಲಿ ಮುಳುಗಿ ಮರಣ ಹೊಂದಿದ್ದಾಗಿ ತಿಳಿಸಿದ. ಅದೃಷ್ಟವಶಾತ್ ಬಾಲಕನ ಹಿರಿಯ ಎಂಜಿನಿಯರ್ ಸಹೋದರರೊಬ್ಬರು ಕೆಲಸದ ನಿಮಿತ್ತ\break ಬೊಂಬಾಯಿಗೆ ಹೋಗಬೇಕಾಯಿತು. ಈ ಮೇಲ್ಕಂಡ ಹೇಳಿಕೆಯಿಂದ ತುಂಬಾ ಕುತೂಹಲಗೊಂಡ ಅವರು ವಶ್ಯಸುಪ್ತಿಗೊಳಗಾದ ತಮ್ಮ ಸಹೋದರನ ಮಾತಿನಲ್ಲಿ ಏನಾದರೂ ತಥ್ಯವಿದೆಯೇ ಎಂಬು\-ದನ್ನು ತಿಳಿಯಬಯಸಿದರು. ಆದ್ದರಿಂದ ಬೊಂಬಾಯಿಗೆ ಹೋಗಿದ್ದಾಗ, ಅವರು ಜುಹುವಿಗೆ ಹೋಗಿ, ಈ ಹಿಂದೆ ತಮ್ಮ ಸಹೋದರ ತಿಳಿಸಿದ್ದ ಬಂಗಲೆಗಾಗಿ ಹುಡುಕಿದರು. ಆದರೆ ಅದು ಪತ್ತೆಯಾಗಲಿಲ್ಲ. ಆದರೂ ನಿರಾಶರಾಗದೆ ಅವರು ಸುತ್ತಮುತ್ತ ಆ ಬಗ್ಗೆ ವಿಚಾರಿಸಿದರು.\break ಅತ್ಯಾಶ್ಚರ್ಯವೆಂಬಂತೆ ಕೆಲವರು ಹಲವು ವರ್ಷಗಳ ಹಿಂದೆ ಈ ಜಾಗದಲ್ಲಿ ಬಂಗಲೆಯೊಂದು ಇದ್ದುದಾಗಿ ತಿಳಿಸಿದರು. ಅಲ್ಲದೆ ಕೆಲವು ವರ್ಷಗಳ ಹಿಂದೆ ಅದನ್ನು ಮಾರಿದಾಗ, ಅದೇ ಜಾಗದಲ್ಲಿ ಬೇರೊಂದು ಭವ್ಯ ಕಟ್ಟಡ ಕಟ್ಟಲು ಆ ಬಂಗಲೆಯನ್ನು ಉರುಳಿಸ ಲಾಯಿತು–ಎಂದು ತಿಳಿಸಿದರು. “ಆ ಬಂಗಲೆಯಲ್ಲಿದ್ದ ಕುಟುಂಬದವರು ಯಾರು? ಈಗ ಅವರು ಎಲ್ಲಿದ್ದಾರೆ, ಏನು ಕತೆ?” ಎಂಬುದರ ಬಗ್ಗೆ ಆ ಎಂಜಿನಿಯರ್ ವಿಚಾರಿಸಿದರು. ಆದರೆ ಆ ಪ್ರಶ್ನೆಗೆ ಉತ್ತರ ದೊರೆಯಲಿಲ್ಲ. ಆದಾಗ್ಯೂ ಒಬ್ಬ ವಯೋವೃದ್ಧ ಈ ರೀತಿ ತಿಳಿಸಿದ: “ಆ ಮನೆಯವರು ತಮ್ಮ ಒಂಬತ್ತನೇ ವಯಸ್ಸಿನ ಬಾಲಕ ಸಮುದ್ರದಲ್ಲಿ ಮುಳುಗಿ ಸತ್ತಾಗ ಆ ಬಂಗಲೆಯನ್ನು ಮಾರಿದರು!”

ಸರಕಾರಿ ಕಛೇರಿಗಳಲ್ಲಿ ಹಿಂದಿನ ಕಡತಗಳನ್ನು ಹುಡುಕಾಡಿ ದಾಖಲೆಗಳನ್ನು ಹಾಜರುಪಡಿಸುವಂತೆ, ವಶ್ಯಸುಪ್ತಿಯಿಂದ ಸುಪ್ತಮನಸ್ಸನ್ನು ಕೆದಕಿ ಬೆದಕಿದರೆ, ಹಿಂದಿನ ಜನ್ಮಗಳ ದಾಖಲೆಗಳೂ ಸಿಗುತ್ತವೆಂದಾಯ್ತು ಅಲ್ಲವೇ ಹಾಗಾದರೆ?


\section*{ಪೂರ್ವಜನ್ಮದ ಸ್ಮೃತಿ}

\addsectiontoTOC{ಪೂರ್ವಜನ್ಮದ ಸ್ಮೃತಿ}

ಇಂಗ್ಲೆಂಡಿನ ಮನೋರೋಗ ತಜ್ಞ ಡೇನಿಸ್ ಕೆಲ್​ಸಿ ಅವರು ಸುಮಾರು ಮುನ್ನೂರು ಮನೋ ರೋಗಿಗಳನ್ನು ವಶ್ಯಸುಪ್ತಿಗೆ ಒಳಪಡಿಸಿ ಅವರ ಮನಸ್ಸಿನ ಆಳದ ಸ್ತರಗಳಲ್ಲಿ ಅಡಗಿರುವ ನೋವು, ದುಃಖ, ನೈಚ್ಯಾನುಸಂಧಿಯ ಗ್ರಂಥಿಗಳನ್ನು \enginline{(Complexes)} ಕಂಡುಕೊಂಡು ಚಿಕಿತ್ಸೆ ನೀಡಿ ಗುಣ ಪಡಿಸಿದ್ದರು. ಮುನ್ನೂರು ಮಂದಿ ರೋಗಿಗಳೂ ಸುಪ್ತಸ್ಥಿತಿಯಲ್ಲಿ ಮನಸ್ಸಿನ ಆಳಕ್ಕೆ ಮುಳುಗಿ ತಮ್ಮ ಹಿಂದಿನ ಜನ್ಮಗಳ ಘಟನೆಗಳ ಸ್ಮೃತಿಯನ್ನು ಅತ್ಯಂತ ಸ್ಪಷ್ಟವಾಗಿ ಸ್ಮರಿಸಿಕೊಂಡು ತಮ್ಮ ತಪ್ಪುಗಳನ್ನು ವಿವರವಾಗಿ ತಿಳಿಸಲು ಸಮರ್ಥರಾಗಿದ್ದರು. ಸಾಮಾನ್ಯವಾಗಿ ಮನೋರೋಗತಜ್ಞರು ರೋಗಿಯ ಸದ್ಯದ ಮಾನಸಿಕ ವ್ಯಾಧಿಗೆ ಆತನು ಶಿಶುವಾಗಿರುವವರೆಗಿನ ಸ್ಮೃತಿಯನ್ನು ಮಾತ್ರ ಸುಪ್ತಾವಸ್ಥೆಯಲ್ಲಿ ಕೆದಕಿ ನೋಡುತ್ತಾರೆ. ಆದರೆ ಇದು ರೋಗದ ಕಾರಣವನ್ನು ಕಂಡುಹಿಡಿಯಲು ಸಾಕಾಗುವಷ್ಟು ಸಾಕ್ಷ್ಯಗಳನ್ನು ಒದಗಿಸುವುದಿಲ್ಲ. ರೋಗಿಯ ಹಿಂದಿನ ಜನ್ಮದ ಘಟನಾವಳಿಗಳ ಪರಿಚಯವೂ ಕೆಲವು ಸಂದರ್ಭಗಳಲ್ಲಿ ಅವಶ್ಯ ಎಂದು ಹೇಳುತ್ತಾರೆ ಕೆಲ್​ಸಿ.

ಕೆಲವು ವರ್ಷಗಳ ಹಿಂದೆ ವಿಜ್ಞಾನದ ವಿವಿಧ ಕ್ಷೇತ್ರಗಳಲ್ಲಿ ದುಡಿಯುವವರು ಆತ್ಮ, ದೇಹಾ ತೀತ ಅಸ್ತಿತ್ವ–ಇವುಗಳನ್ನು ಕುರಿತು ಮಾತನಾಡುವುದು ತಮ್ಮ ಗೌರವಕ್ಕೇ ಕುಂದುತರುವ ಸಂಗತಿ ಎಂದು ತಿಳಿಯುತ್ತಿದ್ದರು. ಭೌತಿಕ ನಿಯಮಗಳಿಗೆ ಅಳವಡದ ಸಂಗತಿಗಳು ಇವೆ ಎಂಬ ವಿಶ್ವಾಸ ಮೂಢನಂಬಿಕೆಯಲ್ಲದೆ ಬೇರೇನೂ ಅಲ್ಲ ಎಂಬುದು ಅವರ ನಿಲುವಾಗಿತ್ತು. ಆದರೆ ಇದೀಗ ಈ ತೆರನಾದ ವೈಜ್ಞಾನಿಕ ದೃಷ್ಟಿ ಮೆಲ್ಲನೆ ಬದಲಾವಣೆ ಹೊಂದುತ್ತಲಿದೆ. ೧೯೮೨ನೇ ಇಸವಿ ಜುಲೈ\break ತಿಂಗಳ ‘ಸೈನ್ಸ್ ಡೈಜಸ್ಟ್​’ ಪತ್ರಿಕೆಯಲ್ಲಿ ಜಾನ್ ಗ್ಲಾಡ್​ಮನ್ ಒಂದು ಕುತೂಹಲಕಾರಿಯಾದ ಲೇಖನವನ್ನು ಪ್ರಕಟಿಸಿದ್ದಾರೆ. “ಆತ್ಮಾನ್ವೇಷಣೆಯಲ್ಲಿ ವಿಜ್ಞಾನಿಗಳು” ಎಂಬುದು ಲೇಖನದ ಶೀರ್ಷಿಕೆ. ಬಹುಮಂದಿ ಪ್ರಬುದ್ಧ ವಿಜ್ಞಾನಿಗಳು ಆಶ್ಚರ್ಯವೆನಿಸುವ ಹಾಗೂ ಗೂಢವಾದ ಆಧ್ಯಾತ್ಮಿಕ ಜಗತ್ತನ್ನು ಕುರಿತು ನಿಶ್ಚಿತ ಅಭಿಪ್ರಾಯ ಉಳ್ಳವರಾಗಿದ್ದಾರೆ–ಎನ್ನುತ್ತಾರೆ ಗ್ಲಾಡ್​ಮನ್. ೧೯೬೩ನೇ ಇಸವಿಯಲ್ಲಿ ಅಂಗರಚನಾಶಾಸ್ತ್ರದ ಸಂಶೋಧನೆಗಾಗಿ ನೊಬೆಲ್ ಪಾರಿತೋಷಕವನ್ನು ಪಡೆದ ಆಸ್ಟ್ರೇಲಿಯಾ ಸಂಜಾತ ವಿಜ್ಞಾನಿ ಜಾನ್ ಈಕ್ಲೇಸ್ ‘ಮನುಷ್ಯನ ವ್ಯಕ್ತಿತ್ವ ಕೇವಲ ಭೌತದ್ರವ್ಯಗಳಿಂದಲೇ ನಿರ್ಮಿತವಲ್ಲ. ಅದು ಭೌತಿಕ ಹಾಗೂ ಅಭೌತಿಕವಾದ ಚೇತನ ಶಕ್ತಿಯನ್ನು ಒಳಗೊಂಡಿದೆ. ಸಾವಿನ ಆಚೆಗೂ ಈ ಶಕ್ತಿಯ ಅಸ್ತಿತ್ವವಿದೆ. ಪ್ರತಿಯೊಂದು ಜೀವಾತ್ಮವೂ ದೈವೀಸೃಷ್ಟಿ’ ಎಂದು ನಿರ್ಭೀತಿಯಿಂದ ಸಾರುತ್ತಾರೆ.\footnote{\engfoot{Eccles strongly defends the ancient religious belief that human being consists of a mysterious compound of physical matter and intangible spirit. This non-material self survies the death of the physical brain.}}


\section*{ಮರಳಿ ಬಂದಳು, ವರದಿ ತಂದಳು}

\addsectiontoTOC{ಮರಳಿ ಬಂದಳು, ವರದಿ ತಂದಳು}

ಮೃದುಲ ಶರ್ಮಾ ದಕ್ಷ ಗೃಹಿಣಿ. ಅವಳಿ ಎಂ.\ ಎ.\ ಪದವೀಧರೆ. ಆದರೆ ಈ ಎರಡು ಪದವಿಗಳನ್ನೂ ಒಂದೇ ಜೀವನದಲ್ಲಿ ಪಡೆದದ್ದಲ್ಲ; ಒಂದೇ ವಿಶ್ವವಿದ್ಯಾಲಯದಲ್ಲೂ ಪಡೆದದ್ದಲ್ಲ. ವಾರಾಣಸಿಯ ವಿಶ್ವವಿದ್ಯಾಲಯದಲ್ಲಿ ಹಿಂದೀ ಎಂ.\ ಎ.\ ಯನ್ನು ಪಡೆದುದು ಮೊದಲು. ನಂತರ ಮೀರತ್ ವಿಶ್ವವಿದ್ಯಾಲಯದಲ್ಲಿ ರಾಜಕೀಯ ಶಾಸ್ತ್ರವನ್ನು ಅಧ್ಯಯನ ಮಾಡಿ ಎಂ.\ ಎ.\ ಪಾಸಾದದ್ದು. ಈ ಎರಡು ಡಿಗ್ರಿಗಳ ಮಧ್ಯದ ಅಂತರ ಒಂದು ಜೀವನ ಮಾತ್ರ! ಎಂದರೆ ನೀವು ಚಕಿತರಾಗುವುದು ಸ್ವಾಭಾವಿಕ. ಮೃದುಲ ಶರ್ಮಾ ದೃಢವಾಗಿ ನಂಬಿದ್ದಾರೆ–‘ಹಿಂದಿನ ಜೀವನದಲ್ಲಿ ಮುನ್ನು ಹೆಸರಿನ ವ್ಯಕ್ತಿಯೇ ಇಂದು ತಿರುಗಿ ಮೃದುಲ ಆಗಿ ಬಂದಿದ್ದೇನೆ’ ಎಂದು.

ಹಿಂದಿನ ಜೀವನ ಡೆಹರಾಡೂನಿನಲ್ಲಿ. ತೀರಿಕೊಂಡದ್ದು ೧೯೪೫ರಲ್ಲಿ ಎಂಬುದಕ್ಕೆ ದಾಖಲೆ ಇದೆ. ಮುನ್ನು ಆಗಿದ್ದು ಅಲ್ಲಿ ತೀರಿಕೊಂಡ ನಾಲ್ಕು ವರ್ಷಗಳ ಬಳಿಕ ೧೯೪೯ರಲ್ಲಿ ಮೃದುಲ ಆಗಿ ನಾಸಿಕ್​ನಲ್ಲಿ ಜನಿಸಿದ್ದು ಎಂದು ಆಕೆ ತಿಳಿದುಕೊಂಡಿದ್ದಾಳೆ. ಈಗ ಆಕೆಯ ವಯಸ್ಸು ಇಪ್ಪತ್ತ ನಾಲ್ಕು. ಹಿಂದಿನ ಜೀವನದಲ್ಲಿ ಅವಳು ಬನಿಯಾ ಜಾತಿಯವಳಾಗಿದ್ದಳು. ಈ ಜೀವನದಲ್ಲಿ ಬ್ರಾಹ್ಮಣ ಜಾತಿಯಲ್ಲಿ ಜನಿಸಿದ್ದಾಳೆ.

ಈಕೆಯ ಹೇಳಿಕೆಯನ್ನು ಒಂದು ಭ್ರಾಮಕ ಕಲ್ಪನೆ ಎಂದು ತಳ್ಳಿ ಹಾಕಬಹುದಿತ್ತು. ವರ್ಜೀನಿಯಾ ವಿಶ್ವವಿದ್ಯಾಲಯದ ಪ್ರಾಧ್ಯಾಪಕ ಅಯಾನ್ ಸ್ಟೀವನ್​ಸನ್ ೧೯೭೪ನೇ ಮಾರ್ಚ್ ೧೮, ೧೯ರಂದು ಈ ಘಟನೆಯನ್ನು ಪರಿಶೀಲಿಸಲೇ ಭಾರತಕ್ಕೆ ಬಂದಿದ್ದರು ‘ಇದು ಪುನರ್ಜನ್ಮಕ್ಕೊಂದು ಸ್ಪಷ್ಟವಾದ ಸಾಕ್ಷ್ಯಾಧಾರ ಎನ್ನುತ್ತಾರವರು; ಪುನರ್ಜನ್ಮವಲ್ಲದೇ ಬೇರಾವ ಅರ್ಥ– ವಿವರಣೆ ಕೊಡಲು ಸಾಧ್ಯವಿಲ್ಲ’ ಎಂದು ಮನಗಂಡೇ ಅವರು ಹಿಂದಿರುಗಿದರು.

ಇದೊಂದು ಅಪೂರ್ವ ಘಟನೆ. ಹಿಂದಿನ ಜನ್ಮದ ತಾಯಿ ಸತ್ಯವತಿ ಸೆಟ್ಟಿ. ಆಗಿನ ದಿನಗಳ ಪ್ರಮುಖ ಸಾಮಾಜಿಕ ಕಾರ್ಯಕರ್ತೆ; ಸ್ವಾತಂತ್ರ್ಯ ಹೋರಾಟಗಾರರ ಜೊತೆಯಲ್ಲಿ ದುಡಿದ ಮಹಿಳೆ. ಈ ಜನ್ಮದ ತಾಯಿ ಶಾಂತಾ ಶಾಸ್ತ್ರಿ. ಇಬ್ಬರೂ ಬದುಕಿಕೊಂಡಿದ್ದಾರೆ. ಇಬ್ಬರು ತಾಯಂದಿರೂ ತಮ್ಮ ಈ ಮಗಳನ್ನು ಅತ್ಯಂತ ಪ್ರೀತಿಯಿಂದ ನೋಡಿಕೊಳ್ಳುತ್ತಾರೆ. ಮೃದುಲಾ ಇನ್ನೂ ಎರಡು ವರ್ಷದ ಕೂಸಾಗಿರುವಾಗ ಪೂರ್ವಜನ್ಮದ ಸ್ಮರಣೆಯ ಘಟನೆ ಪ್ರಾರಂಭವಾದದ್ದು. ಲಿಚಿ ಹಣ್ಣನ್ನು ತಿನ್ನಲು ಕೊಟ್ಟಾಗ ಅವಳ ಮನಸ್ಸಿನಲ್ಲಿ ಹಿಂದಿನ ಸ್ಮರಣೆ ಸ್ಫುಟವಾಗಿ ಅರಳಿತು. ಡೆಹರಾಡೂನಿನ ತನ್ನ ಮನೆಯಲ್ಲಿ ಅತ್ಯುತ್ತಮ ಜಾತಿಯ ಹಣ್ಣುಗಳು ತಿನ್ನಲು ಸಿಕ್ಕುತ್ತಿದ್ದವೆಂದು ತನ್ನ ತಾಯಿಯ ಹತ್ತಿರ ಆಕೆ ಹೇಳಿಕೊಂಡಳು. ಚಕಿತರಾದರೂ ತಂದೆತಾಯಿ ಮೊದಲು ಆ ಮಾತಿಗೆ ಗಮನ ಕೊಟ್ಟಿರಲಿಲ್ಲ. ಕೆಲವೇ ದಿನಗಳಲ್ಲಿ ಆಕೆಯ ತಂದೆ ತೀರಿಕೊಂಡರು. ಆ ಬಳಿಕ ದೈವ ಯೋಗದಿಂದಲೋ ಎಂಬಂತೆ ಶಾಂತಾ ಶಾಸ್ತ್ರಿ ಡೆಹರಾಡೂನಿನ ಒಂದು ಶಾಲೆಯಲ್ಲಿ ಅಧ್ಯಾಪಕಿಯಾಗಿ ಅಲ್ಲಿಯೇ ಮಗಳೊಂದಿಗೆ ವಾಸಿಸತೊಡಗಿದರು. ಆಗ ಪದೇ ಪದೇ ಮಗಳು ತನ್ನ ಹಿಂದಿನ ಜೀವನದ ಘಟನೆ ಹೇಳುತ್ತಿದ್ದಳು.

ಒಂದು ದಿನ ಡೆಹರಾಡೂನಿನಲ್ಲಿ ನಡೆದ ಧಾರ್ಮಿಕ ಸಭೆಗೆ ತಾಯಿಯ ಜೊತೆ ಮೃದುಲಾ ಹೋಗಿದ್ದಳು. ಆಕೆ ಆ ಸಭೆಯಲ್ಲಿ ಒಂದೆಡೆ ಕುಳಿತಿದ್ದ ವಯಸ್ಸಾದ ಮಹಿಳೆಯನ್ನು ಗುರುತಿಸಿ, ನೇರವಾಗಿ ಅವರನ್ನು ಸಮೀಪಿಸಿ ಅವರ ಮಡಿಲಲ್ಲಿ ಕುಳಿತೇ ಬಿಟ್ಟಳು. ಆ ವೃದ್ಧ ಮಹಿಳೆ ಸಹಜವಾಗಿ ಮಗುವನ್ನು ಪ್ರೀತಿಯಿಂದ ನೇವರಿಸಿದಳು. ತೀರಿಕೊಂಡಿದ್ದ ತನ್ನ ಮಗಳೇ ಈ ರೂಪ ಧರಿಸಿ ಬಂದಿದ್ದಾಳೆಂದು ಅವರಾದರೂ ಹೇಗೆ ತಿಳಿಯಲು ಸಾಧ್ಯ? ಆ ವೃದ್ಧ ಮಹಿಳೆಯ ಒಂದು ಪಕ್ಕದಲ್ಲಿ ಕುಳಿತ ಇನ್ನೊಬ್ಬ ಹೆಂಗಸನ್ನು ಕುರಿತು ‘ನೀನು ನನ್ನ ತಂಗಿ’ ಎಂದು ದೃಢವಿಶ್ವಾಸದಿಂದ ಹೇಳಿದಳು.

‘ಅದು ಹೇಗೆ ಸಾಧ್ಯ? ನಾನು ದೊಡ್ಡವಳು, ನೀನಿನ್ನೂ ಮಗು!’

‘ನಾನು ನಿನ್ನ ಹಿರಿಯಕ್ಕ ಮುನ್ನೂ.’

ಅಲ್ಲಿ ಕುಳಿತಿದ್ದ ಆ ಮಹಿಳೆಗೆ ಪರಮಾಶ್ಚರ್ಯ! ಏನಾದರೂ ಮೋಸವಿರಬಹುದೆಂಬ ಸಂಶಯದಿಂದ, ಮಗುವಿಗೆ ತಿಳಿದಿರಲು ಸಾಧ್ಯವಿರದ ಹಲವಾರು ಪ್ರಶ್ನೆಗಳನ್ನು ಕೇಳಿದಳು. ಮೃದುಲ ನೀಡಿದ ಉತ್ತರದಿಂದ ಆಕೆಯ ಸಂಶಯ ದೂರವಾಯಿತು. ಆದರೆ ತಾಯಿ ಸತ್ಯವತಿಗೆ ಮಗುವಿನ ಮಾತಿನಲ್ಲಿ ವಿಶ್ವಾಸ ಮೂಡಲಿಲ್ಲ. ಪರೀಕ್ಷಿಸಿ ನೋಡುವ ಕುತೂಹಲದಿಂದ ಮೃದುಲಳನ್ನು ತಮ್ಮ ಮನೆಗೇ ಕರೆದೊಯ್ದರು. ಪ್ರತಿಯೊಂದು ಕೋಣೆಯಲ್ಲೂ ಹುಡುಗಿ ಸಂತೋಷದಿಂದ ಓಡಾಡಿದಳು. ತನ್ನ ಪುಸ್ತಕಗಳ ಬೀರುವನ್ನೇ ತೋರಿಸಿದಳು. ಮಾಳಿಗೆಯ ಫ್ಯಾನ್ ನೋಡಿ ‘ಇದು ಹಿಂದೆ ಇರಲಿಲ್ಲ’ ಎಂದಳು. ಅದೆಲ್ಲ ನಿಜವಾಗಿತ್ತು.

‘ಈ ಭಾಗ ಹೊಸದಾಗಿ ಕಾಣಿಸುತ್ತಿದೆ’–ರಿಪೇರಿಯಾದ ಮನೆಯ ಒಂದು ನೂತನ ವಿಭಾಗವನ್ನು ನೋಡಿ ಎಂದಳು. ಆ ಹೇಳಿಕೆ ಸರಿಯಾಗಿತ್ತು. ಹಳೆಯ ಮುಖಗಳನ್ನು ಗುರುತಿಸಬಲ್ಲಳೇ ಎಂದು ತಿಳಿಯುವುದಕ್ಕಾಗಿ ಹುಡುಗಿಯನ್ನು ಹೊರಗಡೆ ಕರೆದುಕೊಂಡು ಹೋದರು. ‘ಈ ತಿಂಡಿ ಯಂಗಡಿಯಿಂದ ಯಾರೂ ಏನನ್ನೂ ಕೊಂಡುಕೊಳ್ಳಬೇಡಿ. ಅಂಗಡಿಯಾತ ನನ್ನ ಹಣದ ಚೀಲವನ್ನು ಒಮ್ಮೆ ಬಲಾತ್ಕಾರ ಮಾಡಿ ಎಳೆದುಕೊಂಡಿದ್ದ’ ಎಂದು ಅಲ್ಲಿ ಕುಳಿತ ವೃದ್ಧನನ್ನು ತೋರಿಸಿ ಹೇಳಿದಳು. ಆತ ತಾನು ಮಾಡಿದ್ದ ತಪ್ಪನ್ನು ಸ್ಮರಿಸಿ ‘ಹೌದು’ ಎಂದು ಒಪ್ಪಿಕೊಂಡ. ಈಗ ಆನಂದಸ್ವಾಮಿಯಾಗಿರುವ ಕುಶಾಲಚಂದರನ್ನು ಗುರುತಿಸಿ ‘ಸ್ವಾಮೀಜಿ, ಯಾವಾಗಿನಿಂದ ನೀವು ಕಾವಿ ಧರಿಸಲು ಆರಂಭಿಸಿದಿರಿ? ಹಿಂದೆ ನೀವು ಬಿಳಿ ಬಟ್ಟೆ ಧರಿಸುತ್ತಿದ್ದಿರಿ’ ಎಂದಳು. ಈ ಮಾತನ್ನು ಕೇಳಿ ಕುಶಾಲ್​ಚಂದರು ಸ್ತಂಭಿತರಾದರು.

ಈ ಎಲ್ಲವನ್ನೂ ವೇಷ ಮರೆಸಿಕೊಂಡು ನೋಡುತ್ತಿದ್ದ ತಂದೆ ಎಂದರು: ‘ಮಗೂ, ನೀನು ಎಲ್ಲರನ್ನೂ ಗುರುತಿಸಿದೆ. ಈ ಮನೆಯಲ್ಲಿದ್ದ ನಿನ್ನ ಸೇವಕನನ್ನು ಗುರುತಿಸಲೇ ಇಲ್ಲ.’ ತಂದೆಯನ್ನು ತಬ್ಬಿಕೊಂಡು ಮೃದುಲಾ ಹೇಳಿದಳು ‘ನೀವು ಸೇವಕರಲ್ಲ, ನನ್ನ ಅಪ್ಪ.’

ಗತಿಸಿದ ತನ್ನ ಮಗಳು ತಿರುಗಿ ದೇಹಧಾರಿಯಾಗಿ ಬಂದ ವಿಚಾರ ಮನವರಿಕೆಯಾದ ನಂತರ ಸತ್ಯವತಿಗೆ ಅಪಾರ ಸಂತೋಷ. ಅಂದಿನಿಂದ ಏಕಪ್ರಕಾರವಾಗಿ ಅವರು ಪ್ರೀತಿಯ ಹೊನಲನ್ನೇ ಹರಿಸುತ್ತಿದ್ದಾರೆ ಈ ಪುತ್ರಿಯ ಮೇಲೆ. ಮೃದುಲಳ ಮದುವೆಯ ಸಂದರ್ಭದಲ್ಲಿ ಒಂದು ಒಳ್ಳೆಯ ಬಂಗಲೆಯನ್ನೂ, ಹತ್ತು ಸಾವಿರ ರೂಪಾಯಿಗಳನ್ನೂ ಕೊಟ್ಟರು. ಜೊತೆಗೆ ಚಿನ್ನಾಭರಣ ಬೇರೆ.

‘ಸ್ಮರಣೆಯಲ್ಲದೇ ಹಿಂದಿನ ಜನ್ಮದ ಬಗ್ಗೆ ಇನ್ನೇನಾದರೂ ಸಾಕ್ಷ್ಯ ನೀಡಲು ಸಾಧ್ಯವೇ?’ ಎಂದು ಸ್ಟೀವನ್​ಸನ್ ಆಕೆಯನ್ನು ಕೇಳಿದರು. ‘ನನಗೇನೂ ತಿಳಿಯದು’ ಎಂದಳಾಕೆ. ಸ್ವೀವನ್ ಸನ್ ‘ಇದು ನಿಜವಾದ ಪುನರ್ಜನ್ಮದ ಘಟನೆ’ ಎನ್ನುತ್ತಾರೆ. ಈಗ ಅವರು ಹುಟ್ಟಿನಿಂದಲೇ ಇರುವ ಮಚ್ಚೆಗಳನ್ನು ಕುರಿತು ಅಧ್ಯಯನ ನಡೆಸುತ್ತಿದ್ದಾರೆ. ಪುನರ್ಜನ್ಮದ ಘಟನೆಗಳಲ್ಲಿ ಅಂದಿನ ಜನ್ಮದ ಮಚ್ಚೆಗಳು ಕಾಣಿಸಿಕೊಳ್ಳುತ್ತವೆ ಎನ್ನುತ್ತಾರೆ.

‘ನೀವು ಹೇಗೆ ಮರಣ ಹೊಂದಿದಿರೆಂಬುದು ನೆನಪಿದೆಯೆ?’ ಎಂದುದಕ್ಕೆ,

‘ಹೌದು, ನೆನಪಿದೆ. ತೀವ್ರ ನೋವು, ಪಾದದಿಂದ ನನ್ನನ್ನು ಯಾರೋ ಮೇಲಕ್ಕೆಳೆದಂತೆ ಸೆಳೆತ. ಸೆಳೆತ ಮೇಲು ಮೇಲಕ್ಕೇರುತ್ತ ಬಂದು ತಲೆಯಲ್ಲಿ ವಿಪರೀತ ನೋವುಂಟಾಯಿತು. ಆಮೇಲೆ ತೇಲ ತೊಡಗಿದ ಅನುಭವ. ಆಕಾಶದಲ್ಲಿ ಹಾರಾಟ ಮಾಡಿದ ಅನುಭವ. ಎರಡು ವರ್ಷದವಳಿದ್ದಾಗ ಹೆಚ್ಚಿನ ವಿಚಾರ ನೆನಪಿತ್ತು. ಈಗ ಮರವೆ ಬರುತ್ತಲಿದೆ’ ಎಂದಿದ್ದಾಳಾಕೆ.

ಮೃದುಲ ಶರ್ಮಾ ಐದು ವರ್ಷದವರಿದ್ದಾಗ ಹೃಷಿಕೇಶದ ಸ್ವಾಮಿ ಶಿವಾನಂದರನ್ನು ತನ್ನ ತಾಯಿ ಜೊತೆಯಲ್ಲಿ ಹೋಗಿ ಭೇಟಿ ಮಾಡಿದ್ದಳು. ಸ್ವಾಮಿ ಶಿವಾನಂದರು ತಮ್ಮ ಬರಹದಲ್ಲಿ ಈ ವಿಚಾರವನ್ನು ಸೂಚಿಸಿದ್ದಾರೆ. ‘ದಿ ಟೈಮ್ಸ್ ಆಫ್ ಇಂಡಿಯ’ದ ೧೯೭೬ನೇ ಇಸವಿ, ಅಕ್ಟೋಬರ್ ೩೧ರ ಭಾನುವಾರದ ಸಂಚಿಕೆಯಲ್ಲಿ ಸಚಿತ್ರ ವಿವರಣೆಯೊಂದಿಗೆ ಈ ಘಟನೆ ಪ್ರಕಟವಾಗಿತ್ತು. ಈ ವಿವರಣೆಗೆಲ್ಲ ಈ ಲೇಖನವೇ ಆಧಾರ.


\section*{ಪರಕಾಯ ಪ್ರವೇಶ?}

\addsectiontoTOC{ಪರಕಾಯ\break ಪ್ರವೇಶ?}

ಗರ್ಭವಾಸ ಮಾಡದೆ ಪರಕಾಯ ಪ್ರವೇಶದ ಮೂಲಕ ಹೊಸಜನ್ಮ ತಾಳಿದ ವಿಚಿತ್ರ ಘಟನೆ ಯೊಂದನ್ನು ಸ್ಟೀವನ್​ಸನ್ ಸ್ವತಃ ಪರಿಶೀಲಿಸಿ ವಿವರಣೆ ನೀಡಿದ್ದಾರೆ. ಮೂರೂವರೆ ವರ್ಷ ವಯಸ್ಸಿನ ಬಾಲಕನೊಬ್ಬನ ಶರೀರ ಇನ್ನೊಬ್ಬ ಜೀವಾತ್ಮದ ನೆಲೆಬೀಡಾದ ಅದ್ಭುತವೆನಿಸುವ ಘಟನೆ ಇದು. ಉತ್ತರ ಪ್ರದೇಶದ ಮುಜಪ್ಫರ್ ನಗರದ ರಸೂಲ್​ಪುರದಲ್ಲಿ ೧೯೫೪ರಲ್ಲಿ ನಡೆದ ಘಟನೆ. ಗಿರಿಧರಲಾಲ್ ಜಾಠರ ಮಗ ಜಸಬೀರ್ ಹೆಸರಿನ ಹುಡುಗ ಸಿಡುಬು ರೋಗದಿಂದ ನರಳುತ್ತಿದ್ದ. ರೋಗ ಉಲ್ಬಣಿಸಿ ಒಂದು ದಿನ ಅವನು ಸತ್ತು ಹೋದನೆಂದೇ ಮನೆಯವರೆಲ್ಲರೂ ತಿಳಿದಿದ್ದರು. ಹೋದದ್ದಂತೂ ನಿಜವೇ. ಅವನ ಶವ–ಸಂಸ್ಕಾರ ಮಾಡಲು ಗಿರಿಧರಲಾಲ್ ಹಳ್ಳಿಯ ಬಂಧುಗಳನ್ನು ಸಂಪರ್ಕಿಸಿದರು. ರಾತ್ರಿ ವೇಳೆ ಮೀರಿದುದರಿಂದ ಮಾರನೇ ದಿನ ಬೆಳಿಗ್ಗೆ ಆ ಕಾರ್ಯ ಮಾಡೋಣವೆಂದು ಅವರೆಲ್ಲ ಕೇಳಿಕೊಂಡರು. ಗಿರಿಧರಲಾಲ್ ಅವರ ಮಾತನ್ನು ಅಲ್ಲಗಳೆಯಲಿಲ್ಲ. ಎರಡು ಮೂರು ಗಂಟೆಗಳ ಬಳಿಕ ಗತಿಸಿದ್ದನೆಂದು ತಿಳಿಯಲಾದ ಜಸಬೀರನ ಶರೀರದಲ್ಲಿ ಚಲನೆ ಕಾಣಿಸಿತು. ಸ್ವಲ್ಪ ಹೊತ್ತಿನಲ್ಲೇ ಆತ ಸರಿಯಾಗಿ ಉಸಿರಾಡಿದ. ಅತ್ಯಂತ ದುರ್ಬಲನಾಗಿದ್ದ ಹುಡುಗ ಮೆಲ್ಲನೇ ಚೇತರಿಸಿಕೊಂಡ. ಆರೋಗ್ಯ ಲಾಭವಾದ ಮೇಲೆ ಮಾತನಾಡತೊಡಗಿದ. ಮಾತನಾಡಲು ಶುರುಮಾಡಿದಾಗ ಜಸಬೀರನ ಶರೀರದಲ್ಲಿ ಬೇರೊಂದು ವ್ಯಕ್ತಿತ್ವ ಪ್ರವೇಶಿಸಿರುವ ಸಂಗತಿ ಕ್ರಮೇಣ ಎಲ್ಲರಿಗೂ ಸ್ಪಷ್ಟವಾಯಿತು.

ತಾನು ವಿಹೇದಿಯ ಶಂಕರಲಾಲನ ಮಗ ಶೋಭಾರಾಮ್. ತಾನು ತನ್ನ ಹಳ್ಳಿಗೆ ಹೋಗಬೇಕಾ ಗಿದೆ ಎನ್ನತೊಡಗಿದ. ಬ್ರಾಹ್ಮಣತ್ವದ ಅಭಿಮಾನದಿಂದ ಆ ಜಾಠರ ಮನೆಯಲ್ಲಿ ಮಾಡಿದ ಅಡುಗೆಯನ್ನು ಊಟ ಮಾಡಲು ನಿರಾಕರಿಸಿದ. ಅವನ ಹಠ ಎಷ್ಟು ತೀವ್ರವಾಗಿತ್ತೆಂದರೆ ಮನೆಯ ಹತ್ತಿರವಿದ್ದ ಬ್ರಾಹ್ಮಣ ಹೆಂಗಸಿನ ಮೂಲಕ ಅಡಿಗೆ ಮಾಡಿಸಬೇಕಾಯಿತು. ಎರಡು ವರ್ಷಗಳವರೆಗೆ ಈ ಹಠ ಮುಂದುವರಿಯಿತು. ಮಧ್ಯೆ ಮಧ್ಯೆ ಅವನ ಕಣ್ಣುತಪ್ಪಿಸಿ ಅವರ ಮನೆಯಲ್ಲೇ ತಯಾರಿಸಿದ ಆಹಾರ ಪದಾರ್ಥಗಳನ್ನು ತಿನ್ನಿಸುತ್ತಿದ್ದರು. ತನ್ನ ಕಣ್ಣುತಪ್ಪಿಸಿ ಈ ರೀತಿ ತಿನ್ನಿಸುತ್ತಿದ್ದಾರೆಂದು ತಿಳಿದು ಮೊದಮೊದಲು ದುಃಖಿತನಾದರೂ ಕ್ರಮೇಣ ಅವನು ಹೊಂದಿಕೊಂಡ.

ವಿಹೇದಿಯಲ್ಲಿ ಹಿಂದಿನ ಜನ್ಮದಲ್ಲಿ ತಾನು ಶೋಭಾರಾಮನಾಗಿ ಮೃತಪಟ್ಟ ವಿಚಾರದ ವಿವರಗಳನ್ನೆಲ್ಲ ನೀಡಿದ. ಒಂದು ಮದುವೆಯ ಔತಣದಲ್ಲಿ ಭಾಗವಹಿಸಿ ಗಾಡಿಯಲ್ಲಿ ಕುಳಿತು ಹಿಂದಿರುಗುವಾಗ ತಲೆ ತಿರುಗಿ ಕೆಳಗೆ ಬಿದ್ದು ಮಿದುಳಿಗೆ ಬಲವಾದ ಏಟಾಗಿ ತೀರಿಕೊಂಡೆನೆಂದು ಹೇಳಿದ. ತನ್ನಿಂದ ಸಾಲಪಡೆದ ವ್ಯಕ್ತಿಯೊಬ್ಬ ಆ ದಿನ ತನಗೆ ವಿಷ ಉಣಿಸಿದ ಸಂಗತಿಯನ್ನೂ ತಿಳಿಸಿದ.

ಜಸಬೀರನ ತಂದೆ, ಶೋಭಾರಾಮನು ಜಸಬೀರನ ಶರೀರದಲ್ಲಿ ಆವಾಹಿತನಾದ ಸಂಗತಿಯನ್ನು ಪ್ರಸಾರಮಾಡಲು ಇಷ್ಟಪಡಲಿಲ್ಲ. ಆದರೂ ಹುಡುಗನ ವರ್ತನೆ ಮಾತುಕತೆ ಜನರ\break ಕುತೂಹಲವನ್ನು ಸಾವಕಾಶವಾಗಿ ಸೆಳೆಯಿತು. ಶೋಭಾರಾಮನಿಗಾಗಿ ತಯಾರಿಸಲ್ಪಡುತ್ತಿದ್ದ\break ಬ್ರಾಹ್ಮಣಿಯ ಅಡುಗೆಯ ವಿಚಾರ ಊರಿನ ಇತರ ಬ್ರಾಹ್ಮಣರಿಗೆ ತಿಳಿದು ಕ್ರಮೇಣ ವರ್ತಮಾನ ವಿಹೇದಿಗೂ ತಲುಪಿತು. ಸುಮಾರು ಮೂರು ವರ್ಷಗಳ ಬಳಿಕ ಘಟನೆಯ ಸತ್ಯಾಂಶವನ್ನು ಸಾರುವ ಸಾಕ್ಷಿಗಳು ದೊರಕತೊಡಗಿದವು. ರವಿದತ್ತ ಶುಕ್ಲರು ತಮ್ಮ ಪತ್ನಿಯೊಂದಿಗೆ ಜಸಬೀರನನ್ನು ನೋಡಲು ಬಂದಾಗ ಆತ ಅವರನ್ನು ನೇರವಾಗಿ ಗುರುತಿಸಿದ. ಜಸಬೀರನ ಶರೀರದಲ್ಲಿದ್ದ ಶೋಭಾರಾಮನು ಕೊಟ್ಟ ತನ್ನ ಸಾವಿನ ವಿವರಣೆಯೂ ವಿಹೇದಿಯ ಶಂಕರಲಾಲ್ ತ್ಯಾಗಿಯು ನೀಡಿದ ವಿವರಣೆಯೂ ಸರಿಯಾಗಿಯೇ ಇದ್ದವು. ಮುಂದೆ ವಿಹೇದಿಯಿಂದ ಶಂಕರಲಾಲರೂ ಕುಟುಂಬದ ಇತರ ಸದಸ್ಯರೂ ಬಂದಾಗ ಹುಡುಗ ಪ್ರತಿಯೊಬ್ಬರನ್ನೂ ಗುರುತಿಸಿದ. ಬಹಳ ಆತ್ಮೀಯತೆಯಿಂದ ಎಲ್ಲರೊಡನೆ ಮಾತನಾಡಿದ. ಕೆಲದಿನಗಳ ಬಳಿಕ ವಿಹೇದಿಯ ಹತ್ತಿರ ಹುಡುಗನನ್ನು ಕರೆತಂದು ರೈಲ್ವೇ ಸ್ಟೇಷನ್ ಹತ್ತಿರಬಿಟ್ಟಾಗ ತ್ಯಾಗಿಯ ಮನೆಯ ದಾರಿಯನ್ನು ಸುಲಭವಾಗಿ ಸರಿಯಾಗಿಯೇ ಗುರುತಿಸಿದ. ದಾರಿತಪ್ಪಿಸಿ ಬಿಟ್ಟರೂ ಕಷ್ಟವಿಲ್ಲದೇ ಗಂತವ್ಯಸ್ಥಾನಕ್ಕೆ ಹೋಗಿ\-ಬಿಡು\-ತ್ತಿದ್ದ. ತ್ಯಾಗೀ ಕುಟುಂಬದ ವಿವರಗಳನ್ನೆಲ್ಲ ಹೇಳಬಲ್ಲವನಾಗಿದ್ದ. ತನ್ನ ಹೊಸ ಜನ್ಮಸ್ಥಾನವಾದ ರಸೂಲ್​ಪುರಕ್ಕಿಂತ ವಿಹೇದೀ ಗ್ರಾಮದ ತ್ಯಾಗಿಯವರ ಮನೆಯಲ್ಲಿರಲು ಆಸಕ್ತಿ ತೋರುತ್ತಿದ್ದ. ಮನಸ್ಸಿಲ್ಲದ ಮನಸ್ಸಿನಿಂದ ಒಮ್ಮೆ ರಸೂಲ್​ಪುರಕ್ಕೆ ಹಿಂದಿರುಗುವಾಗ ಗೊಳೋ ಎಂದು ಅತ್ತೇಬಿಟ್ಟ. ಪುಟ್ಟ ಶರೀರದಲ್ಲಿದ್ದುಕೊಂಡಿದ್ದರೂ ತನ್ನನ್ನು ವಯಸ್ಕನೆಂದೇ ತಿಳಿದುಕೊಂಡು ಮಾತಾಡುತ್ತಿದ್ದ–ತನ್ನ ಹೆಂಡತಿ ಮಕ್ಕಳು ಆ ಊರಲ್ಲಿದ್ದಾರೆ ಇತ್ಯಾದಿ.

ಮಾತುಕತೆಯೆಲ್ಲ ಹಿಂದಿನ ವ್ಯಕ್ತಿತ್ವದ ತಾದಾತ್ಮ್ಯದಿಂದಲೇ, ಎಂದರೆ ತಾನು ಶೋಭಾರಾಮ್ ಎಂಬ ಭಾವದಿಂದಲೇ ನಡೆಯುತ್ತಿತ್ತು. ಜಸಬೀರನ ಭಾಷೆಯಲ್ಲಿ ಬ್ರಾಹ್ಮಣರು ಉಪಯೋಗಿಸುತ್ತಿದ್ದ ಶಬ್ದಗಳನ್ನು ಗಿರಿಧರಲಾಲ್ ಗುರುತಿಸಿದ್ದರು. ಮೊದಲು ಹುಡುಗನದೇನೋ ಹುಚ್ಚಾಟವೆಂದು ಗದರಿಸಿದರೂ ಸತ್ಯಾಂಶ ತಿಳಿದ ಮೇಲೆ ಅವನಿಗೆ ಸಾಂತ್ವನ ನೀಡಿ ಮರ್ಯಾದೆಯಿಂದಲೇ ನೋಡಿ\-ಕೊಂಡರು. ತನ್ನ ಹೆಂಡತಿ ಮಕ್ಕಳು ವಿಹೇದಿಯಲ್ಲಿದ್ದಾರೆ ಎಂದು ನಿಸ್ಸಂಕೋಚವಾಗಿ ಹುಡುಗನೊಬ್ಬ ಹೇಳುತ್ತಾನೆಂದರೆ ಯಾರಾದರೂ ಹಾಸ್ಯ ಮಾಡದೆ ಇರಲು ಸಾಧ್ಯವೆ? ಸಾಕಷ್ಟು ಹಾಸ್ಯ ಹಾಗೂ ಬೈದು ಗದರಿಸಿದ ಬಳಿಕ ಆ ರೀತಿ ಮಾತನಾಡುವುದನ್ನು ನಿಲ್ಲಿಸಿದ.

ವಿಹೇದಿಗೆ ಹೋದಾಗ ಜಸಬೀರನು ಬಾಲೇಶ್ವರನನ್ನು (ಹಿಂದಿನ ಜನ್ಮದ ಮಗ) ತನ್ನ ಜೊತೆ\-ಯಲ್ಲೇ ತಂದೆ ಮಕ್ಕಳನ್ನು ಮಲಗಿಸಿಕೊಳ್ಳುವಂತೆ ಹತ್ತಿರದಲ್ಲೇ ಮಲಗಿಸಿಕೊಳ್ಳುತ್ತಿದ್ದ. ಯಾರಾದರೂ ತನಗೆ ಬಹುಮಾನವಾಗಿ ಕೊಟ್ಟ ವಸ್ತುಗಳನ್ನು ಬಾಲೇಶ್ವರನಿಗೇ ಒಪ್ಪಿಸಿಬಿಡುತ್ತಿದ್ದ. ಬಾಲೇಶ್ವರ ಶಾಲೆಗೆ ಹೊರಟುನಿಂತಾಗ ಇವನು ಚಡಪಡಿಸುತ್ತಿದ್ದ. ಹಿಂದಿನ ಜೀವನದ ಬಾಂಧವ್ಯ ಅಷ್ಟು ಪ್ರಬಲವಾಗಿಯೇ ವ್ಯಕ್ತವಾಗುತ್ತಿತ್ತು. ತ್ಯಾಗಿಗಳ ಜೊತೆಗೇ ಇದ್ದುಬಿಡಬಹುದೆಂಬ ಭೀತಿಯಿಂದ ಜಾಠರು ಜಸಬೀರನನ್ನು ವಿಹೇದಿಗೆ ಹೋಗಲು ಬಿಡುತ್ತಿರಲಿಲ್ಲ.

ಶೋಭಾರಾಮ್ ತೀರಿಕೊಂಡು ಜಸಬೀರನ ಶರೀರವನ್ನು ಪ್ರವೇಶಿಸುವುದಕ್ಕೆ ಮಧ್ಯೆ ಇಂದ್ರಿಯಾ\-ತೀತ ಅನುಭವಗಳೇನಾದರೂ ಆಗಿದ್ದುವೇ ಎಂದು ಸ್ಟೀವನ್​ಸನ್ ಪ್ರಶ್ನಿಸಿದ್ದರು. ‘ಸಾಧುವೊಬ್ಬ ಕಾಣಿಸಿಕೊಂಡು ಜಸಬೀರನ ಶರೀರವನ್ನು ಆಶ್ರಯಿಸಿ ಇರು’ ಎಂದು ಆದೇಶ ನೀಡಿದ್ದರು ಎಂದ. ಸಂದಿಗ್ಧ ಪರಿಸ್ಥಿತಿ ಬಂದೊದಗಿದಾಗ ಕೆಲವೊಮ್ಮೆ ಈ ಸಾಧುಮಹಾತ್ಮರಿಂದ ಮಾರ್ಗದರ್ಶನ\break ಸಿಗುವುದೂ ಇದೆ ಎಂದೂ ಹೇಳಿದ್ದ.

ಸ್ಟೀವನ್​ಸನ್ ಈ ಘಟನೆಯನ್ನು ಆಮೂಲಾಗ್ರ ಪರಿಶೀಲಿಸಲು ೧೯೫೭ರಿಂದ ೧೯೭೪ರ\break ವರೆಗೂ ಹಲವು ಬಾರಿ ಅಮೇರಿಕದಿಂದ ಮುಜಪ್ಫರ್ ನಗರಕ್ಕೆ ಬಂದು ಈ ಘಟನೆಗೆ ಸಂಬಂಧಿಸಿದ ಹತ್ತಾರು ಸಾಕ್ಷ್ಯಾಧಾರಗಳನ್ನು ನಿಖರವಾಗಿ ಶೋಧನೆ ಮಾಡಿದ್ದರು. ಇತ್ತೀಚೆಗೆ ಜಸಬೀರನಿಗೆ ಜಾಠರ ಜಾತಿಯ ಹುಡುಗಿಯೊಂದಿಗೆ ವಿವಾಹವಾಯಿತು. ತನ್ನ ಈ ಜೀವನದ ತಂದೆತಾಯಿ ಸಹೋದರರೊಡನೆ ಚೆನ್ನಾಗಿ ಹೊಂದಿಕೊಂಡಿದ್ದಾನೆ. ಹಿಂದಿನ ಜೀವನದ ಘಟನೆಗಳನ್ನು ಮೆಲ್ಲನೇ ಮರೆಯುತ್ತಿದ್ದಾನೆ.


\section*{ಸಾಯಲಾರಿರಿ ನೀವು!}

\addsectiontoTOC{ಸಾಯಲಾರಿರಿ ನೀವು!}

{\parfillskip=0pt ‘ಇಂಥ ಘಟನೆಗಳು ಅಲ್ಲೊಂದು ಇಲ್ಲೊಂದು ನಡೆದ ಮಾತ್ರಕ್ಕೆ ಎಲ್ಲರಿಗೂ ಜನ್ಮಾಂತರ ಸಂಬಂಧದ ಅನುಭವಗಳಿವೆ ಎಂದಾಗುತ್ತದೆಯೇ?’ ಎಂದು ನೀವು ಕೇಳಬಹುದು. ಆದರೆ ಪ್ರಪಂಚದ ಎಲ್ಲ ಭಾಗಗಳಲ್ಲಿಯೂ ಇಂಥ ಘಟನೆಗಳನ್ನು ಸಾಕಷ್ಟು ಕಲೆಹಾಕಿದ್ದಾರೆ. ಜನ್ಮಾಂತರದಲ್ಲಿ ವಿಶ್ವಾಸವಿಲ್ಲದ ಜನಾಂಗದಲ್ಲೂ ಹಿಂದಿನ ಜನ್ಮಗಳ ಸ್ಮರಣೆಯುಳ್ಳ ವ್ಯಕ್ತಿಗಳು ಕಾಣಿಸಿ ಕೊಂಡಿದ್ದಾರೆ ಎಂಬುದನ್ನು ನೀವು ಗಮನಿಸಬೇಕು. ಅಯಾನ್ ಸ್ಟೀವನ್​ಸನ್ ಇಂತಹ ಸಹಸ್ರಾರು ಘಟನೆಗಳನ್ನು ಸಂಗ್ರಹಿಸಿ ಅಧ್ಯಯನ ನಡೆಸಿದ್ದಾರೆ. ಸಾವನ್ನು ಕುರಿತು ಒಂದು ನೂರು ವರ್ಷಗಳ ಸಂಶೋಧನೆಯಿಂದ ಸಂಗ್ರಹಿಸಿದ ಆಶ್ಚರ್ಯಕರ ತಥ್ಯಗಳನ್ನೊಳಗೊಂಡ, ಅಯಾನ್ ಕ್ಯೂರಿ ಅವರ ಗ್ರಂಥ ‘ನೀವು\par}\newpage\noindent ಸಾಯಲಾರಿರಿ’\footnote{\engfoot{Ian Currie, You cannot Die: The Incredible findings of a century of research of death (Hamlyn: Jonathan James Books).}}ಈ ನಿಟ್ಟಿನಲ್ಲಿ ಒಂದು ಮೈಲುಗಲ್ಲು. ಅವರ ದೀರ್ಘಕಾಲ ಅಧ್ಯಯನದಿಂದ ವ್ಯಕ್ತವಾದ ಸತ್ಯಗಳಿವು–

೧. ಸಾವಿನೊಂದಿಗೆ ಜೀವನ ಅಂತ್ಯವಾಗುವುದಿಲ್ಲ. ದೇಹಾತೀತ ಅಸ್ತಿತ್ವವಿದೆ.

೨. ಗತಿಸಿಹೋದ ಜೀವಾತ್ಮರು ವಿಭಿನ್ನಸ್ತರ ಅಥವಾ ಲೋಕಗಳಲ್ಲಿ ಅರಿವು ಚಟುವಟಿಕೆಗಳಿಂದ ಕೂಡಿಯೇ ಇರುತ್ತಾರೆ. ಸ್ಥೂಲ ದೇಹಧಾರಿಗಳಾದ ನಮಗದು ಗೋಚರವಾಗದು.

೩. ಆ ಲೋಕಗಳನ್ನು ಬಿಟ್ಟು ಇಳೆಯಲ್ಲಿ ಹೊಸ ಶರೀರವನ್ನು ಧರಿಸಿದಾಗ ಅಲ್ಲಿನ ಹಾಗೂ ಹಿಂದಿನ ಜೀವನದ ನೆನಪು ಮಾಸುವುದು.

೪. ಹೇಗೆಂದರೆ ಹಾಗೆ ಜೀವಾತ್ಮ ತಿರುತಿರುಗಿ ದೇಹಧಾರಿಯಾಗಿ ಬರಲಾಗುವುದಿಲ್ಲ. ಕಾರ್ಯ ಕಾರಣ ನಿಯಮವನ್ನನುಸರಿಸಿ ಹೀಗೆ ಬಂದು ಹೋಗುವ ಕೆಲಸ ಮುಂದುವರಿಯುವುದು.

ಜನ್ಮಾಂತರ ಮತ್ತು ಕರ್ಮಸಿದ್ಧಾಂತಕ್ಕೆ ಸಂಬಂಧಿಸಿ ಯಾವ ಪೂರ್ವಾಗ್ರಹವಿಲ್ಲದೆ, ಮಾನವ ಶಾಸ್ತ್ರ, ಸಮಾಜಶಾಸ್ತ್ರಗಳ ಆಳವಾದ ಪಾಂಡಿತ್ಯದಿಂದ ಪರಾಮಾನಸಶಾಸ್ತ್ರದಲ್ಲಿ ಆಸಕ್ತಿ ತಳೆದು ತೆರೆದ ಮನಸ್ಸಿನಿಂದ ಸತ್ಯಾನ್ವೇಷಣೆಗೆ ಹೊರಟವರು ಅಯಾನ್ ಕ್ಯೂರಿ ಎಂಬುದನ್ನು ಗಮನಿಸಿದಾಗ, ಈ ಮಾತುಗಳ ಮಹಿಮೆ ನಿಮಗೆ ವ್ಯಕ್ತವಾಗದಿರದು.


\section*{ಅತೀಂದ್ರಿಯ ವಿಜ್ಞಾನದ ಕೊಡುಗೆ}

\addsectiontoTOC{ಅತೀಂದ್ರಿಯ ವಿಜ್ಞಾನದ ಕೊಡುಗೆ}

ಶೀಲಾ ಒಸ್ಟ್ರೆಂಡರ್ ಮತ್ತು ಲಿನ್ ಶ್ರೋಡರ್ ಬರೆದ ಒಂದು ಉದ್ಗ್ರಂಥವನ್ನು ಲಂಡನ್ನಿನ ಎಬೇಕಸ್ ಪ್ರಕಾಶನ ಸಂಸ್ಥೆ ಕೆಲವು ವರ್ಷಗಳ ಹಿಂದೆ ಪ್ರಕಟಿಸಿತು. ಈ ಇಬ್ಬರು ಲೇಖಕಿಯರು ಎಂಟು ವರ್ಷಗಳ ಕಾಲ ರಷ್ಯದ ಅತೀಂದ್ರಿಯ ಅನ್ವೇಷಣಾತಜ್ಞರ ಸಹಕಾರದಿಂದ ಹಲವು ಲೇಖನಗಳನ್ನು ಪ್ರಕಟಿಸಿದ್ದರು. ೧೯೬೮ನೆ ಇಸವಿಯಲ್ಲಿ ಮಾಸ್ಕೋ ನಗರದಲ್ಲಿ ನಡೆದ ಪರಾಮನ ಶ್ಶಾಸ್ತ್ರದ ವಿಶೇಷ ಸಭೆಗೆ ಆಹ್ವಾನಿತರಾಗಿ ಹೋಗಿ, ಅಲ್ಲಿ ಹಲವು ಮಂದಿ ತಜ್ಞರನ್ನು ಭೇಟಿಮಾಡಿ, ಅಲ್ಲಿ ನಡೆದ, ನಡೆಯುತ್ತಿರುವ ಅಸಂಖ್ಯಾತ ಅತೀಂದ್ರಿಯ ಸಂಬಂಧವಾದ ಸಂಶೋಧನೆಗಳನ್ನು ಕಂಡು, ಕೇಳಿ, ವಿಸ್ತೃತವೂ, ಅಧಿಕಾರಯುತವೂ ಆದ ಗ್ರಂಥವನ್ನು ಬರೆದರು. ಆ ಗ್ರಂಥದ ಹೆಸರು \enginline{Psychic Discoveries Behind The Iron Curtains} ಎಂದು. ‘ಕಬ್ಬಿಣದ ಪರದೆಯ ಹಿಂದಿನ ಅತೀಂದ್ರಿಯ ವಿಜ್ಞಾನಕ್ಕೆ ಸಂಬಂಧಿಸಿದ ಶೋಧನೆಗಳು’ ಎಂಬುದು ಆ ಶಿರೋನಾಮೆಯ ಅರ್ಥ. ರಷ್ಯದ ಸರಕಾರವು ಈ ತೆರನಾದ ಸಂಶೋಧನೆಗಳಿಗಾಗಿ ಪ್ರತಿವರ್ಷವೂ ಇಪ್ಪತ್ತು ಮಿಲಿಯ ರೂಬಲುಗಳನ್ನು ನಿಗದಿಮಾಡಿದೆಯೆಂದೂ, ಅಮೇರಿಕಾ ದೇಶದ ಇಂಥ ಸಂಶೋಧನೆಗಳಿಗಿಂತ ಐವತ್ತು ವರ್ಷಗಳಷ್ಟಾದರೂ ರಷ್ಯದವರು ಮುಂದಿದ್ದಾರೆಂದೂ ಅವರು ಹೇಳುತ್ತಾರೆ. ಅಮೇರಿಕ ಈ ಬಗ್ಗೆ ತೀವ್ರವಾದ ಕ್ರಮ ಕೈಗೊಳ್ಳಬೇಕೆಂದು ಅವರು ಕಳಕಳಿಯಿಂದ ಕೇಳಿಕೊಂಡಿದ್ದಾರೆ– ತಮ್ಮ ಗ್ರಂಥದಲ್ಲಿ. ರಷ್ಯನ್ ತಜ್ಞರು ತಾವು ಕಂಡುಹಿಡಿದ ಎಲ್ಲ ರಹಸ್ಯಗಳನ್ನು ಹೊರದೇಶದಿಂದ ಬಂದವರಿಗೆ ತಿಳಿಸುವುದಿಲ್ಲ. (ಅದೇ ‘ಕಬ್ಬಿಣದ ಪರದೆ!’) ಆದರೆ ಬೇಕಷ್ಟು ಆಶ್ಚರ್ಯಕರ ಶೋಧನೆಗಳನ್ನು ಮಾಡಿದ್ದಾರೆಂಬುದು ಸತ್ಯ. ಆ ಗ್ರಂಥಕರ್ತರು ಭೆಟ್ಟಿಯಾದ ಅತೀಂದ್ರಿಯ ಅನ್ವೇಷಣಾತಜ್ಞ ಡಾ. ವೆಲಿಸಿವ್​ರ ಒಂದು ಮಾತು–‘ಇಪ್ಪತ್ತೈದು ವರ್ಷಗಳ ಹಿಂದೆಯೇ ಸೋವಿ ಯತ್ ವಿಜ್ಞಾನ “ಭಾವಪ್ರೇಷಣ\enginline{”(Telepathy)}ದಲ್ಲಿ ಯಶಸ್ವೀ ಪ್ರಯೋಗಗಳನ್ನು ಮಾಡಿತ್ತು’ ಎಂಬುದು ಸ್ಮರಣೀಯ ಮಾತ್ರವಲ್ಲ, ಎಷ್ಟು ದೀರ್ಘಕಾಲದಿಂದ ರಷ್ಯದ ವಿಜ್ಞಾನಿಗಳು ಈ ದಿಸೆಯಲ್ಲಿ ದುಡಿಯುತ್ತಿದ್ದಾರೆಂಬುದಕ್ಕೊಂದು ಜೀವಂತ ಸಾಕ್ಷಿ. ೧೯೩೦ರ ಹೊತ್ತಿಗೆ ರಷ್ಯಾದಲ್ಲಿ ರಾಕೆಟ್ ಯುಗದ ಪಿತನೆನಿಸಿದ ಕೆ. ಈ. ಟ್ಸಿವೊಲ್ಕೊವ್​ಸ್ಕಿ ಹೇಳಿದ ಮಾತುಗಳನ್ನು ಓದಿದರೆ, ವಿಜ್ಞಾನದ ವಿವಿಧ ಕ್ಷೇತ್ರಗಳಲ್ಲಿ ದುಡಿಯುತ್ತಿದ್ದ ಹಿರಿಯ ವ್ಯಕ್ತಿಗಳೂ, ಅತೀಂದ್ರಿಯ ವಿಚಾರದಲ್ಲಿ ಆಸಕ್ತಿಯನ್ನು ತಾಳಿ ಸಂಶೋಧನೆಗೆ ಹೇಗೆ ಪ್ರೋತ್ಸಾಹ ನೀಡುತ್ತಿದ್ದರು ಎಂಬ ಅರಿವಾ ದೀತು. ಅವರು ಹೇಳುತ್ತಾರೆ: ‘ದೂರದ ಗ್ರಹಗಳಿಗೆ ಪಯಣಿಸುವ ಸಂದರ್ಭದಲ್ಲಿ “ಭಾವ ಪ್ರೇಷಣೆ”ಯ ತರಬೇತಿ ಮತ್ತು ಆ ಕಲೆಯಲ್ಲಿ ನೈಪುಣ್ಯ ಆವಶ್ಯಕ. ಆ ಕಲೆಯಲ್ಲಿ ನಿಷ್ಣಾತರಾದ ಜನರು ಪೂರ್ಣ ಮಾನವಜನಾಂಗದ ವಿಕಾಸಕ್ಕೆ ನೆರವಾಗುತ್ತಾರೆ. ಅಂತರ್ ಗ್ರಹಗಳಿಗೆ ಹೋಗುವ ರಾಕೆಟ್ಟು ಬಾಹ್ಯ ಜಗತ್ತಿನ ರಹಸ್ಯವನ್ನು ತಿಳಿಸಿಕೊಟ್ಟರೆ, ಅತೀಂದ್ರಿಯಾನುಭವಗಳನ್ನು ಕುರಿತ ಅಧ್ಯಯನ ನಮ್ಮ ಮಾನಸಿಕ ಸ್ವಭಾವ ವೈಚಿತ್ರ್ಯದ ರಹಸ್ಯವನ್ನು ಬಯಲು ಮಾಡುವುದು. ಈ ರಹಸ್ಯೋದ್ಘಾಟನೆಯನ್ನು ಮಾಡುವುದೇ ಮಾನವನು ಸಾಧಿಸಬಹುದಾದ ಸರ್ವೋಚ್ಚ ಗುರಿ ಅಥವಾ ಸಿದ್ಧಿ.’

ಎಡ್ವರ್ಡ್ ಸೌಮಾವ್ ಮತ್ತು ಭಾವಪ್ರೇಷಕ ತಜ್ಞ ಕಾರ್ಸ್​ನಿಕೋಲಾೖವ್ ಅವರು ಸೋವಿಯತ್ ದೇಶದಲ್ಲಿ ಪ್ರಕಟವಾದ ಪರಾಮಾನಸಶಾಸ್ತ್ರ ಗ್ರಂಥಗಳ ಜನಪ್ರಿಯತೆಯನ್ನು ಕುರಿತು ವಿವರವಾಗಿ ತಿಳಿಸಿದ್ದಾರೆ. ಈ ಅತೀಂದ್ರಿಯಾನುಭವಗಳನ್ನು ಕುರಿತು ೧೫೨ರಷ್ಟು ಹೆಚ್ಚಿನ ಗ್ರಂಥಗಳು ಇತ್ತೀಚೆಗೆ ಪ್ರಕಟವಾಗಿವೆ! ಸೋವಿಯತ್ ಸರಕಾರ ಮುದ್ರಣ ಸಂಸ್ಥೆಗಳನ್ನು ನಿಯಂತ್ರಿಸು ತ್ತಿದೆ. ಮುದ್ರಿಸುವ ವಿಷಯಗಳಿಗೂ ಕತ್ತರಿ ಪ್ರಯೋಗ ಮಾಡುವ ಹಕ್ಕನ್ನು ಇರಿಸಿಕೊಂಡಿದೆ. ಹೀಗಿದ್ದರೂ ಈ ಜಾತಿಯ ಗ್ರಂಥಗಳು ಇಷ್ಟೊಂದು ಅಪಾರ ಪ್ರಮಾಣದಲ್ಲಿ ಪ್ರಕಟವಾಗಲು ಕಾರಣ ಅಲ್ಲಿ ಈ ನಿಟ್ಟಿನ ಸಂಶೋಧನೆಗೆ ಸಿಗುವ ಪ್ರೋತ್ಸಾಹ ಮತ್ತು ಜನರ ಆಸಕ್ತಿ ಎಂದು ಬೇರೆ ಹೇಳಬೇಕೇ? ಆ ದೇಶದಲ್ಲಿ ಎಂಥ ಅದ್ಭುತ ಶೋಧನೆಗಳು ನಡೆದಿವೆ! ಇನ್ನೂ ನಡೆಯು ತ್ತಿವೆ!

\medskip


\section*{ಎರಡು ದೇಹಗಳಿವೆಯೇ ನಮಗೆ?}

\addsectiontoTOC{ಎರಡು ದೇಹಗಳಿವೆಯೇ ನಮಗೆ?}

ಎಲುಬು ರಕ್ತ ಮಾಂಸ ನರವ್ಯೂಹಗಳಿಂದ ಕೂಡಿದ ಮನುಷ್ಯ ಶರೀರದ ಹಿನ್ನೆಲೆಯಲ್ಲಿ, ಸಾಮಾನ್ಯ ದೃಷ್ಟಿಗೆ ಗೋಚರಿಸದ, ಭಾರತದಲ್ಲಿ ಪುರಾತನ ಕಾಲದಿಂದಲೂ ಯೋಗಿಗಳಿಗೆ ಗೋಚರಿಸಿದೆ ಎಂದು ಹೇಳಲಾದ, ಅತ್ಯಂತ ಸೂಕ್ಷ್ಮವಾದ ಒಂದು ಚೈತನ್ಯಮಯ ಶರೀರವಿದೆ ಎಂಬ ಸೂಚನೆ ರಷ್ಯನ್ ತಜ್ಞರಿಗೆ ಸುಮಾರು ೧೯೩೯ರ ಹೊತ್ತಿಗೇನೇ ತಿಳಿದಿತ್ತು. ಆಗಲೇ ದಕ್ಷಿಣ ರಷ್ಯಾದ ಕ್ಯೂಬಾನ್ ಪ್ರಾಂತದ ಕ್ರಸ್ನೋದರ್ ಎಂಬಲ್ಲಿ ಸಂಶೋಧನೆ ಪ್ರಾರಂಭವಾಗಿತ್ತು. ವಿಜ್ಞಾನಿಗಳು ಮನುಷ್ಯ ದೇಹದಲ್ಲಿ ಸೂಕ್ಷ್ಮಾತಿಸೂಕ್ಷ್ಮರೂಪದಿಂದಿರುವ ಈ ಚೈತನ್ಯಶರೀರವನ್ನು ಗುರುತಿಸಲು ಅಗತ್ಯವಾದ ಕೆಲವೊಂದು ಉಪಕರಣಗಳನ್ನು ಸಜ್ಜುಗೊಳಿಸಲು ತಜ್ಞನನ್ನು ಹುಡುಕು ತ್ತಿದ್ದರು. ಅವರಿಗೆ ಸಹಾಯಕನಾಗಿ ಬಂದ ತಾಂತ್ರಿಕತಜ್ಞನೇ ಸಮ್ಯೋನ್ ಡೆವಿಡೊವಿಚ್ ಕೀರ್ಲಿ ಯನ್. ಅವನೇನೂ ವಿಜ್ಞಾನಿ ಎನಿಸಿಕೊಂಡು ಪ್ರಸಿದ್ಧಿಗೆ ಬಂದವನಲ್ಲ. ವಿದ್ಯುದ್ಯಂತ್ರಗಳನ್ನು ರಿಪೇರಿಮಾಡುವ, ಅವುಗಳಿಗೆ ಸುಧಾರಿತ ರೂಪವನ್ನು ಕೊಡುವ, ತಾಂತ್ರಿಕ ತರಬೇತನ್ನು ಪಡೆದ ಪ್ರತಿಭಾಶಾಲಿ, ಸಾಹಸಿ, ಶ್ರಮಶೀಲ ವ್ಯಕ್ತಿ. ಅವನ ಪತ್ನಿ ವ್ಯಾಲೆಂಟೈನ್ ಪತ್ರಿಕೋದ್ಯಮಿ, ಅಧ್ಯಾಪಕಿ. ಕೀರ್ಲಿಯನ್ ದಂಪತಿಗಳು ಸಸ್ಯಜೀವನದ ಸೂಕ್ಷ್ಮ ಪರಿಶೀಲನೆಗಾಗಿ ಒಂದು ಸೂಕ್ಷ್ಮ ದರ್ಶಕವನ್ನು ಹಲವು ವರ್ಷಗಳ ಮೊದಲೇ ತಯಾರಿಸಿದ್ದರು. ಇದೀಗ ಶಕ್ತಿಶಾಲಿಯಾದ ಇಲೆ ಕ್ಟ್ರಾನ್ ಮೈಕ್ರೋಸ್ಕೋಪನ್ನು ಒಳಗೊಂಡ ಒಂದು ಪರಿಷ್ಕೃತ ಉಪಕರಣವನ್ನು ಸಿದ್ಧಪಡಿಸಿ ಆತ ತನ್ನ ಹೆಸರನ್ನು ಶಾಶ್ವತಗೊಳಿಸಿದ. ಅಧಿಕ ಆವರ್ತನದ ವಿದ್ಯುತ್​ಪ್ರವಾಹ ಕ್ಷೇತ್ರದ ಸಮೀಪದಲ್ಲಿ ಆತನು ಫೋಟೋಗ್ರಫಿಯ ಕಾಗದದ ಮೇಲೆ ಕೈ ತಾಗಿಸಿದಾಗ ಅಲ್ಲಿ ಅಸ್ಪಷ್ಟವಾದ ಚುಕ್ಕೆಗಳೂ, ಚಂಚಲವಾದ ರೇಖೆಗಳೂ ಕಾಣಿಸಿಕೊಂಡವು. ಈ ಘಟನೆ ಅವನನ್ನು ಪರಿಷ್ಕೃತ ಉಪಕರಣದ ಶೋಧನೆಗೆ ಪ್ರೇರಿಸಿ ಒಬ್ಬ ಸಂಶೋಧಕನನ್ನಾಗಿ ಮಾಡಿತು.

ಹೆಚ್ಚು ಪಾರಿಭಾಷಿಕ ಶಬ್ದಗಳನ್ನು ಉಪಯೋಗಿಸದೆ ಈ ವಿಚಾರವನ್ನು ಸುಲಭವಾಗಿ ತಿಳಿದು ಕೊಳ್ಳಲು ಮುಂದೆ ಹೇಳುವ ಉದಾಹರಣೆ ಸಹಾಯಕವಾಗಬಹುದು. ಅಯಸ್ಕಾಂತವು ಕಬ್ಬಿಣದ ಸಣ್ಣ ತುಣುಕುಗಳನ್ನು ತನ್ನೆಡೆಗೆ ಸೆಳೆಯುತ್ತದೆಂಬುದು ಎಲ್ಲರೂ ತಿಳಿದ ವಿಚಾರ. ಜೀವಂತ ಮನುಷ್ಯ ಶರೀರವು ಹೊರಸೂಸುವ ವಿಕಿರಣದ ಪರಿಧಿಯನ್ನು ಅದಕ್ಕೆ ಹೋಲಿಸಬಹುದು. ಅಯಸ್ಕಾಂತದ ಸುತ್ತಲೂ ಇರುವ ಕಾಂತಶಕ್ತಿಯಂತೂ ನಮ್ಮ ಕಣ್ಣಿಗೆ ಕಾಣಿಸುವುದಿಲ್ಲ. ಆದರೆ ಸಮೀಪಕ್ಕೆ ಬಂದ ಕಬ್ಬಿಣದ ತುಣುಕುಗಳನ್ನು ತನ್ನೆಡೆಗೆ ಸೆಳೆದಾಗ ಆ ಶಕ್ತಿಯನ್ನು ಕುರಿತ ಅರಿವು–ಕಲ್ಪನೆ, ನಮ್ಮ ಮನಸ್ಸಿನಲ್ಲಿ ಮೂಡುವುದು. ಅದೇ ರೀತಿ ಮನುಷ್ಯದೇಹದಿಂದ ಹೊರ ಹೊಮ್ಮುವ ಒಂದು ವಿಶಿಷ್ಟ ಚೈತನ್ಯವು ವಿದ್ಯುತ್ತಿನ ಪ್ರವಾಹವನ್ನು ಪ್ರಭಾವಗೊಳಿಸುವುದೆಂದು ತಜ್ಞರು ಹೇಳುತ್ತಾರೆ. ದೇಹದಿಂದ ಹೊರಹೊಮ್ಮುವ ಆ ಚೈತನ್ಯ ಅಥವಾ ಪ್ರಭಾವಲಯದ ಫೋಟೋ ತೆಗೆಯಲು ನಮ್ಮ ದೇಹವನ್ನು ಅಥವಾ ಯಾವುದಾದರೊಂದು ಅಂಗವನ್ನು ಅಧಿಕ ಆವರ್ತನದ ವಿದ್ಯುತ್ ಪ್ರವಾಹ ಕ್ಷೇತ್ರದ ಸಮೀಪವಿರಿಸಬೇಕು. ಆಗ ಹೊರ ಹೊಮ್ಮುವ ಚೈತನ್ಯವು ವಿದ್ಯುತ್ ಕ್ಷೇತ್ರದ ಮೇಲೆ ತನ್ನ ಪ್ರಭಾವವನ್ನು ಬೀರುತ್ತದೆ. ಆಗ ಅವುಗಳ ಬಣ್ಣವೂ ಬದಲಿಸುತ್ತದೆ. ಕೀರ್ಲಿಯನ್ ಸಿದ್ಧಪಡಿಸಿದ ಪರಿಷ್ಕೃತ ಉಪಕರಣದಿಂದ ಅವುಗಳ ಸ್ಥಿತ್ಯಂತರ, ಗತಿವೈಚಿತ್ರ್ಯ, ವರ್ಣವೈವಿಧ್ಯಗಳನ್ನು ಕಾಣಬಹುದು. ಕೀರ್ಲಿಯನ್ ಉಪಕರಣ ವಿದ್ಯುತ್ ಅಲ್ಲದ ‘ಚೈತನ್ಯವನ್ನು’ ಒಂದು ರೀತಿ ವಿದ್ಯುತ್ತಾಗಿ ಪರಿವರ್ತಿಸಿ ನಮ್ಮ ದೃಷ್ಟಿಗೆ ನಿಲುಕುವಂತೆ ಮಾಡುತ್ತದೆ.\footnote{\engfoot{The Kirlian device apparently converts a nonelectrical phenomenon to an electrical one to make it visible.}}

ಸೂರ್ಯಗ್ರಹಣ ಕಾಲದಲ್ಲಿ ಸೂರ್ಯಮಂಡಲವನ್ನು ಚಂದ್ರಬಿಂಬ ಮರೆಮಾಡಿ ಮಸುಕು ಗೊಳಿಸುವಂತೆ ನಮ್ಮ ದೇಹವನ್ನೆಲ್ಲ ಸೂಕ್ಷ್ಮವಾಗಿ ಬೆಳಗುತ್ತಿರುವ ಆ ಅತ್ಯಂತ ಸೂಕ್ಷ್ಮವಾದ ಚೈತನ್ಯ\-ಮಯ ಶರೀರವನ್ನು ನಮ್ಮ ಸ್ಥೂಲದೇಹವು ಮರೆಯಾಗಿಸಿದೆ ಎಂಬುದನ್ನು ಸೋವಿಯತ್ ವಿಜ್ಞಾನಿಗಳು ಕೀರ್ಲಿಯನ್ ಉಪಕರಣಗಳ ಮೂಲಕ ಪ್ರತ್ಯಕ್ಷವಾಗಿ ಕಂಡರು. ನಾವು ಜೀವವಿರುವವುಗಳು ಎನ್ನುವ ಸಸ್ಯಗಳೇ ಇರಲಿ ಅಥವಾ ಮನುಷ್ಯರೇ ಇರಲಿ, ಪ್ರಾಣಿಗಳೇ ಇರಲಿ–ಈ ಎಲ್ಲ ಚೇತನ ವಸ್ತುಗಳೂ ನಮ್ಮ ಇಂದ್ರಿಯಗಳಿಗೆ ಗೋಚರವಾಗುವ, ಅಣುಪರಮಾಣುಗಳಿಂದ ರಚಿತವಾದ ಸ್ಥೂಲ ಆಕಾರ ಮತ್ತು ದೇಹಗಳಿಂದ ಕೂಡಿವೆ. ಇವುಗಳ ಹಿನ್ನೆಲೆಯಲ್ಲಿ ಚೈತನ್ಯಶಾಲಿಯಾದ ಇನ್ನೊಂದು ಸೂಕ್ಷ್ಮ ಶರೀರವಿದೆ ಎಂಬುದನ್ನು ಅವರು ಕಂಡರು. ಅದನ್ನವರು ‘ದಿ ಬಯಲೋಜಿಕಲ್ ಪ್ಲಾಸ್ಮಾಬಾಡಿ\footnote{\engfoot{All living things—plants, animals and humans—not only have a physical body made of atoms and molecules but also have a counterpart body of energy. They called it `The Biological Plasma Body.'}}ಎಂದು ಕರೆದರು. ಇದನ್ನು ಬೇರೆ ಬೇರೆ ದೇಶಗಳಲ್ಲಿ ನಾನಾ ಹೆಸರುಗಳಿಂದ ಅತೀಂದ್ರಿಯಾನುಭವಿಗಳು ನಿರ್ದೇಶಿಸುತ್ತಿದ್ದಾರೆ. ಆ ಕೆಲವು ಹೆಸರುಗಳು ಇಂತಿವೆ: \enginline{‘Energy body’, ‘Secondary body’, ‘Astral body,’ ‘Etheric body’, ‘Fluidic body’, ‘Counterpart body’, ‘Pre-physical body’.}

ಕೀರ್ಲಿಯನ್ ಉಪಕರಣದ ಮೂಲಕ ಈ ಸೂಕ್ಷ್ಮ ದೇಹದ ಅಥವಾ ಶಕ್ತಿದೇಹದ ರೋಮಾಂಚ\-ಕಾರಿ ಅದ್ಭುತ ದೃಶ್ಯವನ್ನು ವೀಕ್ಷಿಸಿದವರು ಅತೀಂದ್ರಿಯ ವಿಚಾರಗಳಲ್ಲಿ ಸಾಮಾನ್ಯ ಕುತೂಹಲಿ\-ಗಳಾದ ಜನರಲ್ಲ. ಸೋವಿಯತ್ ದೇಶದ ಪ್ರಿಸೀಡಿಯಮ್ ಆಫ್ ದಿ ಅಕಾಡೆಮಿ ಆಫ್ ಸೈನ್​ಸಸ್​ನ ಮೇಧಾವಿಗಳಾದ ಸಂಶೋಧಕ ವಿಜ್ಞಾನಿಗಳು, ಇತರ ಸಂಶೋಧನಾಲಯಗಳ ಹಿರಿಯ ವಿಜ್ಞಾನಿಗಳು, ವಿಶ್ವವಿದ್ಯಾಲಯಗಳ ನುರಿತ ಪ್ರಾಧ್ಯಾಪಕರುಗಳು. ಅವರೆಲ್ಲರ ಉದ್ಗಾರಗಳ ಧಾಟಿ ಇದು–

‘ವಿವರಿಸಲಸಾಧ್ಯವಾದುದು! ವಿದ್ಯುತ್ ಜ್ವಾಲೆಗಳಂತೆ ಜಗಜಗಿಸುತ್ತವೆ! ಜ್ವಾಲೆಯ ಅಗ್ರ ಭಾಗ ನೀಲ ಕಿತ್ತಳೆ ಬಣ್ಣಗಳಿಂದ ಹೊಳೆಯುತ್ತವೆ. ಆ ಜ್ವಾಲೆಯ ಕೆಲವು ಭಾಗಗಳು ಸ್ಥಾಯಿಯಾಗಿ ನಿರಂತರ ಬೆಳಕನ್ನು ಕಾರುತ್ತವೆ. ಇನ್ನು ಕೆಲವು ಬೆಳಕುಗಳು ಮೂಡಿ ಮಾಯವಾಗುವ ತಾರಿಕೆಗಳಂತೆ ಗೋಚರಿಸಿ ಕಣ್ಮರೆಯಾಗುತ್ತವೆ. ನಿಜವಾಗಿಯೂ ಅದ್ಭುತ, ಆಶ್ಚರ್ಯಕರ, ರಹಸ್ಯ ಪೂರ್ಣವಾದ ತೇಜೋಮಯ ಜಗತ್ತು!!’

ಈ ಉಪಕರಣಗಳ ಮೂಲಕ ಆ ಚೈತನ್ಯಮಯ ದೇಹದಿಂದ ಹೊರ ಹೊಮ್ಮುವ ಪ್ರಭೆಯ ವಿವಿಧ ವರ್ಣಗಳನ್ನು ಅಧ್ಯಯನ ಮಾಡಲಾಯಿತು. ವ್ಯಕ್ತಿಯು ಭಾವೋದ್ವೇಗಗಳಿಗೆ ಸಿಲುಕಿದಾಗ, ಸಂತಪ್ತನಾದಾಗ, ಆ ವರ್ಣಗಳಲ್ಲಿ ಸ್ಪಷ್ಟ ವ್ಯತ್ಯಾಸ ಕಂಡುಬಂತು. ಅದರ ಬಣ್ಣಗಳನ್ನು ನೋಡಿ ವ್ಯಕ್ತಿಯ ಆರೋಗ್ಯವನ್ನು ಗುರುತಿಸಬಹುದು, ಮಾತ್ರವಲ್ಲ ಮುಂದೆ ರೋಗ ಬರುವ ಸಾಧ್ಯತೆಯನ್ನೂ ಕಂಡುಕೊಳ್ಳಬಹುದು ಎಂಬುದನ್ನು ತಜ್ಞರು ಪ್ರಯೋಗಗಳ ಮೂಲಕ ಕಂಡುಕೊಂಡರು. ಒಟ್ಟಿನಲ್ಲಿ, ರಷ್ಯಾ ಸಾಧಿಸಿದ ಒಂದು ಅದ್ಭುತ ಸಂಶೋಧನೆ ಇದು ಎನ್ನುವ ವಿಚಾರದಲ್ಲಿ ಎರಡು ಅಭಿಪ್ರಾಯಗಳಿಲ್ಲ.

ದೀರ್ಘಕಾಲ ಕೀರ್ಲಿಯನ್ ಉಪಕರಣಗಳ ಮೂಲಕ ಸಂಶೋಧನೆ ಮಾಡಿದ ಸೋವಿಯತ್ ತಜ್ಞರು ಸಾವಿನ ಸಮಯದಲ್ಲೂ ಉಂಟಾಗುವ ಸೂಕ್ಷ್ಮ ಬದಲಾವಣೆಗಳ ಚಿತ್ರಗಳನ್ನು ತೆಗೆದಿದ್ದರು. ಸಸ್ಯ ಮತ್ತು ಪ್ರಾಣಿಗಳು ಸಾವಿನ ಅಂಚನ್ನು ಸಮೀಸಿದಾಗ ಚೈತನ್ಯಮಯ ಶರೀರದ ಜ್ವಾಲೆಗಳು ಸ್ಥೂಲ ಶರೀರವನ್ನು ಬಿಟ್ಟು ಆಕಾಶದಲ್ಲಿ ಮಾಯವಾಗುವುದನ್ನು ಅವರು ಕಂಡರು. ಸಂಪೂರ್ಣ ಸತ್ತುಹೋದ ಸಸ್ಯ ಅಥವಾ ಪ್ರಾಣಿಗಳಲ್ಲಿ ಈ ಚೈತನ್ಯಮಯ ಶರೀರದ ಸುಳಿವೇ ಇರಲಿಲ್ಲ!

ಒಟ್ಟಿನಲ್ಲಿ ಈ ವಿಚಾರಗಳು ಸತ್ಯ: ಅತೀಂದ್ರಿಯ ವಿಚಾರಗಳಿಗೆ ಸಂಬಂಧಿಸಿದ ಅಗಾಧ ಪ್ರಮಾಣದ ಪ್ರಯೋಗ ಪರೀಕ್ಷಣೆಗಳು ರಷ್ಯದಲ್ಲಿ ನಡೆದಿವೆ, ನಡೆಯುತ್ತಿವೆ. ಅತೀಂದ್ರಿಯ ಅನು ಭವಗಳಿಂದ ಪಡೆದ ತಥ್ಯಗಳನ್ನು ಗುಪ್ತ ಪೋಲೀಸ್ ಪಡೆ, ಸೈನ್ಯ ಇಲಾಖೆಯವರು ತಾವು ಹೇಗೆ ಉಪಯೋಗಿಸಿಕೊಳ್ಳಬಹುದೆಂಬುದರ ಬಗ್ಗೆ ಯೋಜನೆಗಳನ್ನು ಕೈಗೊಂಡಿದ್ದಾರೆ. ಸಮಾಜ ವಿರೋಧಿ ಚಟುವಟಿಕೆಗಳಲ್ಲಿ ನಿರತರಾದ ಜನರನ್ನು ಸುಧಾರಿಸಲು ಭಾವಪ್ರೇಷಣ ವಿಧಾನದ ಒಂದು ಯೋಜನೆಯನ್ನು ಕಾರ್ಯಗತ ಮಾಡಿದ್ದಾರೆಂದು ಹೇಳುತ್ತಾರೆ. ಉನ್ನತ ಸ್ಥಾನದಲ್ಲಿರುವ ಮಹಾವಿಜ್ಞಾನಿಗಳು ಅತೀಂದ್ರಿಯಾನುಭವವನ್ನು ತುಚ್ಛವಾಗಿ ಕಾಣುತ್ತಿಲ್ಲ. ಜಗತ್ತಿನ ಯಾವ ಭಾಗದಲ್ಲೇ ಆಗಲಿ, ಅತೀಂದ್ರಿಯಾನುಭವಕ್ಕೆ ಸಂಬಂಧಿಸಿದ ಯಾವುದೇ ಅನ್ವೇಷಣೆಗಳು ಅಥವಾ ಅಲೌಕಿಕ ಘಟನೆಗಳು ನಡೆದರೂ, ಆ ಘಟನೆಗಳ ವಿವರವಾದ ವಾರ್ತೆ ಅವರಿಗೆ ಸಿಗುವಂತೆ ಪ್ರತ್ಯೇಕ ವ್ಯವಸ್ಥೆ ಇದೆ. ಪಶ್ಚಿಮದ ಎಲ್ಲ ಅತೀಂದ್ರಿಯಾನುಭವಿಗಳ, ಸಂಶೋಧಕರ ವಿಚಾರವೂ ರಷ್ಯನರಿಗೆ ಗೊತ್ತಿದೆ. ತಮ್ಮ ದೇಶದಲ್ಲೇ ಇರುವ ಅನುಭಾವಿಗಳನ್ನೂ, ದೂರದರ್ಶನ ಶಕ್ತಿಯುಳ್ಳವರನ್ನೂ, ಭವಿಷ್ಯವಾದಿಗಳನ್ನೂ, ನಂಬಿಕೆಯಿಂದ ರೋಗಗಳನ್ನು ಗುಣಪಡಿಸುವವರನ್ನೂ ಸಮೀಪಿಸಿ, ಆ ಬಗ್ಗೆ ದೀರ್ಘಕಾಲ ಅಧ್ಯಯನವನ್ನು ಮಾಡಿ, ವೈಜ್ಞಾನಿಕವಾಗಿ ಆ ಘಟನೆಗಳ ಹಿನ್ನೆಲೆಯಲ್ಲಿ ಕೆಲಸ ಮಾಡುವ ನಿಯಮವನ್ನು ಕಂಡುಹಿಡಿಯುವ ಹಂಬಲ ಅವರದು. ಈ ವಿಭಾಗದಲ್ಲಿ ಕಳೆದ ಇಪ್ಪತ್ತು ವರ್ಷಗಳಿಂದ ಪರಿಶೋಧನೆ ಮಾಡಿದವರು ಡಾ. ಲೊಜನೋವ್. ಅವರು ಭಾರತದ ಯೋಗಶಾಸ್ತ್ರವನ್ನು ದೀರ್ಘಕಾಲದಿಂದ ಅಧ್ಯಯನ ಮಾಡಿದವರು.


\section*{ಭಾರತದಲ್ಲಿ ಅಸಡ್ಡೆ ಏಕೆ?}

\addsectiontoTOC{ಭಾರತದಲ್ಲಿ ಅಸಡ್ಡೆ ಏಕೆ?}

ಆಧುನಿಕ ವಿಜ್ಞಾನದ ಕ್ಷೇತ್ರದಲ್ಲಿ ಅಸಂಖ್ಯ ಶೋಧನೆಗಳನ್ನು ಕೈಗೊಂಡು ಜಗತ್ತಿನ ಮುಖವನ್ನೇ ಬದಲಿಸಿದ ರಾಷ್ಟ್ರಗಳಲ್ಲಿ ನಡೆದ, ನಡೆಯುತ್ತಿರುವ ಮನುಷ್ಯನ ಮನಸ್ಸಿನ ಆಳದ ಸ್ತರಗಳ\break ಶೋಧನೆಗಳನ್ನು ಕುರಿತು ಅಧಿಕೃತ ವಿವರಗಳನ್ನು ಮೊದಲಾಗಿ ಓದಿಕೊಂಡರೆ, ನಾವು ಭಾರತೀಯ ವಿಚಾರವಾಹಿನಿಯ ಪರಿಚಯವನ್ನು–ಮುಖ್ಯವಾಗಿ ಜನ್ಮಾಂತರ ಕರ್ಮಗಳ ಬಗೆಗೆ ನಮ್ಮ ಪೂರ್ವ\-ಜರು ಅನ್ವೇಷಿಸಿ ಕಂಡುಕೊಂಡ ಸಂಗತಿಗಳನ್ನು ತಿಳಿಯಲು ಮಾನಸಿಕ ಸಿದ್ಧತೆಯನ್ನು ಮಾಡಿ\-ಕೊಂಡಂತಾ\-ಗುವುದು. ಏಕೆಂದರೆ, ಭಾರತದಲ್ಲಿ ಇಂದು ಆಂಗ್ಲ ವಿದ್ಯಾಭ್ಯಾಸ ಮಾಡಿದ ಯಾವ ವ್ಯಕ್ತಿಗಾದರೂ ಈ ಹಳೆಯ ಸಿದ್ಧಾಂತಗಳು ‘ಮೂಢನಂಬಿಕೆಯ ಕಂತೆ, ಪಾಮರರನ್ನು ವಂಚಿಸಲು ಮಾಡಿದ ಸಂಚು’–ಎನ್ನುವ ಭಾವನೆಯು ರೂಢಮೂಲವಾಗುತ್ತಿದೆ. ಪ್ರಾಯಃ ನೂರರಲ್ಲಿ ಒಬ್ಬಿಬ್ಬರು ಇವುಗಳನ್ನು ಅಧ್ಯಯನ ಮಾಡಿ ವಿಮರ್ಶೆಯ ಒರೆಗಲ್ಲಿನಲ್ಲಿ ಪರಿಶೀಲಿಸಿ ತಮ್ಮ ವೈಯಕ್ತಿಕ ನಂಬಿಕೆಯನ್ನು ದೃಢಪಡಿಸಿಕೊಂಡಿರಬಹುದು. ಆದರೆ ಆಧುನಿಕ ವಿಜ್ಞಾನದ ಬೆಳಕಿನಲ್ಲಿ ಪ್ರಾಕ್ತನ ಸಿದ್ಧಾಂತಗಳ ಸತ್ವ ಮತ್ತು ಮಹತ್ವವನ್ನು ತಿಳಿದು, ಅದನ್ನು ವೈಯಕ್ತಿಕ ಜೀವನದಲ್ಲಿ ಅಳವಡಿಸಿಕೊಂಡರೆ, ಬದುಕಿಗೊಂದು ಸಾರ್ಥಕತೆ ಹೇಗೆ ಬಂದೀತು ಎಂಬುದನ್ನು ತಿಳಿಸಿಕೊಡುವ ಪ್ರಾಮಾಣಿಕ ಪ್ರಯತ್ನ ದುರ್ಲಭ ಎನ್ನುವಷ್ಟು ವಿರಳ ಎಂದರೆ ತಪ್ಪಿಲ್ಲ. ಇತ್ತ ಭೌತಿಕ ವಿಜ್ಞಾನದ ಪ್ರಗತಿಯ ಕತೆಯನ್ನು ಸರ್ವತ್ರ ಪ್ರಸಾರಮಾಡುತ್ತಿರುವ ವೇಳೆ, ಏಕೋ ಭಾರತೀಯ ವಿದ್ವಾಂಸರು, ವಿಚಾರವಂತರು, ಇಂಥ ವಿಷಯಗಳನ್ನು ವಿವೇಚಿಸುವುದು ತಮ್ಮ ಗೌರವಕ್ಕೆ ಕುಂದು ತರುವಂಥ ಸಂಗತಿಯೆಂದು ನಾಚುವಂತೆ ಕಾಣುತ್ತದೆ! ಇಂದ್ರಿಯಾತೀತವಾದ ಯಾವುದೇ ಜ್ಞಾನವನ್ನು ಖಂಡಿಸುವುದೇ ವೈಜ್ಞಾನಿಕ ಮನೋವೃತ್ತಿಯ ಲಕ್ಷಣವೆನ್ನುವಂತಾಗಿದೆ! ಇದಕ್ಕೆ ಕಾರಣ ಬಾಲಿಶ ಪರಾನುಕರಣೆಯೇ. ಹಿಂದೆ ಸ್ವಾತಂತ್ರ್ಯಕ್ಕಾಗಿ ಹೋರಾಡಿದ ಹಿರಿಯರೂ, ಶ‍್ರೀಸಾಮಾನ್ಯರೂ ಈ ದೇಶವು ಸಹಸ್ರಾರು ವರ್ಷಗಳಿಂದ ನೆಚ್ಚಿಕೊಂಡು ಬಂದ ಧರ್ಮ ಸಂಸ್ಕೃತಿಯ ಅಭಿಮಾನಿಗಳಾಗಿದ್ದರು. ಭಾರತವು ಜಗತ್ತಿಗೆ ಕೊಡುವ ಕಾಣಿಕೆಯೊಂದಿದೆ ಎಂಬ ದೃಢನಂಬಿಕೆ ಅವರಲ್ಲಿತ್ತು. ವಿಜ್ಞಾನದ ಅನ್ವೇಷಣೆಗಳಿಂದ ಉದ್ಭವಿಸಿದ ಸಂದೇಹಗಳನ್ನು ಎದುರಿಸಿ ನಿಲ್ಲುವ ಸಾಮರ್ಥ್ಯ ಸನಾತನ ತತ್ತ್ವಗಳಿವೆ ಎಂದವರು ತಿಳಿದುಕೊಂಡಿದ್ದರು. ಆದರೆ ನಂತರದ ಪೀಳಿಗೆ, ಈ ಧಾರ್ಮಿಕ, ತಾತ್ವಿಕ ಆದರ್ಶಗಳನ್ನು ಮರೆಯತೊಡಗಿರುವುದು ಸತ್ಯ. ಇಂದು ಭಾರತೀಯರ ಯೋಗ, ತತ್ತ್ವಚಿಂತನೆ, ಸಂಗೀತ ಮತ್ತು ಕಲೆ–ಇವು ಇತರ ದೇಶದ ಜನರ ಮನಸ್ಸನ್ನು ಏಕೆ ಸೆಳೆಯುತ್ತಿವೆ? ಕೇವಲ ಹೊಸತನದ ಕ್ಷಣಿಕ ಕುತೂಹಲ, ಉದ್ವೇಗಕ್ಕಾಗಿಯೇ ಅಲ್ಲ. ಈ ದೇಶದ ಪುರಾತನ ಪುಷಿಗಳು ಮನಸ್ಸಿನ ಆಳಕ್ಕೆ ಮುಳುಗಿ, ಜೀವನದ ಅರ್ಥ ಉದ್ದೇಶಗಳನ್ನು ಕಂಡುಕೊಂಡರು. ಜನತೆಗೆ ಕೆಲವೊಂದು ಶಾಶ್ವತ ಮೌಲ್ಯಗಳನ್ನು ನೀಡಿದರು. ವಿಜ್ಞಾನದ ಪ್ರಗತಿಯ ದೇಶಗಳಲ್ಲಿ ಮನಸ್ಸಿನ ಆಳಕ್ಕೆ ಮುಳುಗಿ ಅದರ ರಹಸ್ಯವನ್ನು ಕಂಡು ಹಿಡಿಯುವ ಪ್ರಯತ್ನ ನಡೆದಿದೆ. ಅದರ ಪ್ರಾಥಮಿಕ ಪರಿಚಯವಾದವರಿಗೆ ಭಾರತದ ಅನ್ವೇಷಣಾ ವಿಧಾನ ತಿಳಿಯುವ ಹಂಬಲವುಂಟಾಗುವುದು ಸ್ವಾಭಾವಿಕ. ಇಂಥ ಪರಿಸ್ಥಿತಿಯಲ್ಲಿ ಭಾರತೀಯರು ತಮ್ಮ ದೇಶದಲ್ಲೇ ‘ಪರದೇಶಿ’ಗಳಾಗಬಾರದು ಎಂದು ರಾಷ್ಟ್ರದ ಪುನರ್ನಿರ್ಮಾಣಕ್ಕಾಗಿ ದುಡಿದು, ಮಡಿದ, ಎಲ್ಲ ಹಿರಿಯರೂ ಸಾರಿ ಹೇಳಿದ್ದಾರೆ. ದೀರ್ಘಕಾಲದ ದಾಸ್ಯದಿಂದ, ಅಜ್ಞಾನದಿಂದ, ನೈಚ್ಯಾನುಸಂಧಾನದಿಂದ ಮತ್ತು ಪರಾನುಕರಣೆಯಿಂದ ತನ್ನ\break ಪರಂಪರೆಗೆ ನಾಚುವ ರೋಗ ಭಾರತೀಯನನ್ನು ಬಿಟ್ಟಿಲ್ಲ. ಆದುದರಿಂದ ವಿದೇಶಗಳಲ್ಲಿ ನಡೆದ ನಂಬಲರ್ಹವಾದ ಅಧಿಕೃತವಾದ ಸಾಕ್ಷ್ಯಾಧಾರಗಳಿಂದ ಕೂಡಿದ ಸಂಗತಿಗಳನ್ನು ಮೊದಲು\break ಮುಂದಿಟ್ಟು, ಅನಂತರವೇ ಭಾರತೀಯ ವಿಚಾರ ವಿಧಾನದ ಪರಿಚಯ ಮಾಡಿಕೊಡಲು\break ಹೊರಟಿದ್ದೇನೆ.


\section*{ಮರೆತುಹೋದ ಸತ್ಯ?}

\addsectiontoTOC{ಮರೆತುಹೋದ ಸತ್ಯ?}

ಅನಾದಿಕಾಲದಿಂದಲೂ ಜಗತ್ತಿನಾದ್ಯಂತ ಅತಿ ಮಾನುಷವೂ, ಅದ್ಭುತವೂ ಆದ ಅತೀಂದ್ರಿಯ ಅನುಭವಗಳನ್ನೂ ಘಟನೆಗಳನ್ನೂ ಜನ ನಂಬುತ್ತ ಬಂದಿದ್ದಾರೆ. ಭಾರತದಲ್ಲಿ ಅತ್ಯಂತ ಪ್ರಾಚೀನ ಕಾಲದಲ್ಲೇ ಇಂಥ ಘಟನೆಗಳ ಮೂಲವನ್ನು ಅನ್ವೇಷಿಸಲು ಧ್ಯಾನಾಭ್ಯಾಸ ಯೋಗಾಭ್ಯಾಸಗಳ ಮೂಲಕ ಯತ್ನಿಸುತ್ತಿದ್ದುದು ಕಂಡುಬರುತ್ತದೆ. ಪ್ರಾಯಶಃ ಈ ಅನ್ವೇಷಣೆಯೇ ತೀವ್ರವಾಗಿ ಮುಂದಿನ ಶತಮಾನಗಳಲ್ಲಿ ಈಶ್ವರ, ಆತ್ಮ, ಪರಕಾಲ, ಜನ್ಮಾಂತರ, ಲೋಕಾಂತರ, ಕರ್ಮ, ಧರ್ಮ, ಭಕ್ತಿ ಮತ್ತು ಮುಕ್ತಿ–ಇವುಗಳನ್ನು ಕುರಿತ ವಿಶಿಷ್ಟ ಸಿದ್ಧಾಂತಗಳಲ್ಲಿ ಪರ್ಯವಸಾನ\-\break ವಾಯಿತು ಎಂದು ಊಹಿಸಬಹುದು. ಕಂಗೊಳಿಸುವ ಆಧುನಿಕ ವೈಜ್ಞಾನಿಕ ಅನ್ವೇಷಣೆಗಳ ಬೆಳಕಿನ ಈಗಿನ ದಿನಗಳಲ್ಲೂ, ಇಂಥ ಘಟನೆಗಳ ಬಗೆಗೆ ಆಗೀಗ ವಿಶ್ವಸನೀಯವಾದ ವರದಿಗಳು ಪ್ರಕಟವಾಗುವುದನ್ನು ನೋಡುತ್ತೇವೆ. ಕೆಲವೊಮ್ಮೆ ಇಂಥ ಘಟನೆಗಳು ಸತ್ಯದೂರವಾಗಿದ್ದರೂ, ಸತ್ಯವನ್ನು ಅನುಕರಿಸುವ ತೋರಿಕೆಗಳಾಗಿರುವುದರಲ್ಲಿ ಸಂದೇಹವಿಲ್ಲ. ಸಾಮಾನ್ಯವಾಗಿ ತಮ್ಮದು ವೈಜ್ಞಾನಿಕ ಮನೋವೃತ್ತಿ ಎಂದುಕೊಳ್ಳುವ ಜನರು ಸಹಜವಾಗಿ ಇಂಥ ಘಟನೆಗಳನ್ನು ಸಂಶಯ ದೃಷ್ಟಿಯಿಂದ ನೋಡುತ್ತ ಬಂದಿದ್ದಾರೆ. ಕಾರಣವಿಷ್ಟೆ: ವಿಜ್ಞಾನದ ಶೋಧನೆಗಳು ನಾವು ದಿನ ನಿತ್ಯವೂ ಪಂಚೇಂದ್ರಿಯಗಳ ಮೂಲಕ ಅನುಭವಿಸುವ ಈ ಜಗತ್ತಿನ ಘಟನಾವಳಿಗಳನ್ನು\break ಕುರಿತಾದದ್ದು. ಆದರೆ ಅಮಾನುಷ ಅಥವಾ ಪವಾಡದ ಜಾತಿಗೆ ಸೇರಿದ ಅದ್ಭುತವೆನಿಸುವ ಘಟನೆಗಳು, ಸರ್ವದಾ ಎಲ್ಲರ ಪರಿಶೀಲನೆ ಪರೀಕ್ಷಣೆಗಳಿಗೆ ಸಿಗುವುದಿಲ್ಲ, ನಿಲುಕುವುದೂ ಇಲ್ಲ. ಪ್ರಯೋಗ ಪರೀಕ್ಷಣೆ\-ಗೊಳ\-ಪಡಲಾರದ ಎಷ್ಟು ಘಟನೆಗಳಿದ್ದರೂ ಸ್ವಾಭಾವಿಕವಾಗಿ ವೈಜ್ಞಾನಿಕ\break ಪರಂಪರೆಯಿಂದ ವಿಜ್ಞಾನಿಗಳು ಅದನ್ನು ಗಮನಿಸಲು ಇಷ್ಟಪಡುವುದಿಲ್ಲ. ಗತಾನುಗತಿಕವಾಗಿ ಇಂಥ ಘಟನೆಗಳು ವಿಜ್ಞಾನಿಯ ಸಂಶಯದೃಷ್ಟಿಗೆ ಮತ್ತು ಕಟುವಿಮರ್ಶೆಗೆ ತುತ್ತಾಗಿದ್ದುದರಿಂದ ಒಂದು ವೇಳೆ ಪ್ರತ್ಯಕ್ಷವಾಗಿ ಅಂಥ ಘಟನೆಗಳನ್ನು ಕಂಡು, ಅನುಭವಿಸಿ, ಅವುಗಳಲ್ಲಿ ನಂಬಿಕೆ\-ಯುಂಟಾದ, ವೈಜ್ಞಾನಿಕ ಕ್ಷೇತ್ರಗಳಲ್ಲಿ ದುಡಿಯುವ ವ್ಯಕ್ತಿಗಳೂ ಈ ಬಗ್ಗೆ ತಮ್ಮ ದೃಢನಿಲುಮೆಯನ್ನು ವ್ಯಕ್ತಪಡಿಸಲು ಹಿಂಜರಿಯುತ್ತಾರೆ! ‘ಮೂಢನಂಬಿಕೆಯನ್ನು ಬೆಂಬಿಡದ ವಿಜ್ಞಾನಿಯಾತ’ ಎಂಬ ಅಪಹಾಸ್ಯಕ್ಕೆ ತುತ್ತಾಗಬೇಕೆಂಬ ಭೀತಿಯೇ ಅವರನ್ನು ಇಂಥ ಬಿಕ್ಕಟ್ಟಿನ ಪರಿಸ್ಥಿತಿಯಲ್ಲಿ ಬೀಳುವಂತೆ ಮಾಡುತ್ತದೆ! ಯಥಾರ್ಥ ಸಂಶೋಧಕರೂ, ವಿಜ್ಞಾನಿಗಳೂ ಅತಿಮಾನುಷ ಘಟನೆಗಳನ್ನು ಕಂಡುಹಿಡಿಯಲು ಯತ್ನಿಸಿ ಸಫಲರಾದರೂ, ಈ ಘಟನೆಗಳ ಬಗೆಗೆ ದೀರ್ಘಕಾಲದಿಂದ ವಿದ್ಯಾವಂತ ಸಾರ್ವಜನಿಕರ ತಿರಸ್ಕಾರ ಭಾವನೆಯನ್ನು ದಾಟಿ ನಿಲ್ಲಲು ಸಮರ್ಥರಾಗುತ್ತಿಲ್ಲ. ಅವರು ತಮ್ಮ ಅಭಿಪ್ರಾಯಗಳನ್ನು ನಿರ್ಭಯವಾಗಿ ವ್ಯಕ್ತಪಡಿಸಿದರೂ, ವಿರೋಧಿಗಳ ಸಂಖ್ಯಾ ಬಲದೆದುರಲ್ಲಿ ಅವರ ದುರ್ಬಲ ದನಿಯು ಅರಣ್ಯರೋದನವಾಗುವುದು. ತಮ್ಮ ಮತೀಯ ಸಾಧನೆ, ಬೋಧನೆಗಳಲ್ಲಿ ಇಂಥ ವಿಚಾರಗಳಿಗೆ ಸ್ಥಾನವೀಯದಿದ್ದ ಆಯಾಯ ಧರ್ಮದ\break ಮುಖಂಡರುಗಳು ಇಂಥ ಘಟನೆಗಳನ್ನು ಸರ್ವಪ್ರಕಾರವಾಗಿ ವಿರೋಧಿಸುತ್ತಾರೆ. ತಮ್ಮ ಸಿದ್ಧಾಂತದ ವಿಚಾರಲಹರಿಗೆ, ಸಾಮಾಜಿಕ ಆರ್ಥಿಕ ದೃಷ್ಟಿಕೋನಕ್ಕೆ, ವಿರೋಧವೆಂದು ತೋರುವುದನ್ನು\break ರಾಜಕೀಯಸ್ಥರೂ ಅಧಿಕಾರಾರೂಢರೂ ಮೂಢನಂಬಿಕೆಯೆಂದು ಅಪಹಾಸ್ಯ ಮಾಡುವುದು\break ಸ್ವಾಭಾ\-ವಿಕ. ಹೀಗೆ ಸತ್ಯವು ಪೂರ್ವಾಗ್ರಹದೂಷಿತರ ಗೊಂದಲದ ಮಡುವಿನಲ್ಲಿ ಮರೆಯಾದಂತೆ ಕಂಡರೂ ಯಥಾರ್ಥ ಅನ್ವೇಷಕರಿಗೆ ತನ್ನ ನೈಜಸ್ವರೂಪವನ್ನು ತೋರದಿರದು.

ಪ್ರಯಾಣದ ಅನುಕೂಲತೆಗಳೂ, ಜ್ಞಾನಪ್ರಸಾರದ ವಿಧಾನಗಳೂ, ಬಹಳಷ್ಟು ಸೀಮಿತವಾಗಿದ್ದ ಕಾಲದಲ್ಲಿ ಕೆಲವೊಂದು ಭಾವನೆಗಳು ಜನಾಂಗದಲ್ಲಿ ಪ್ರಸಾರವಾಗಿ ರೂಢಮೂಲವಾಗಲು ಅನೇಕ ಶತಮಾನಗಳು ಬೇಕಾಗುತ್ತಿದ್ದವು. ಮನುಷ್ಯನ ದೇಹವು ನಶಿಸಿ, ಬಿದ್ದು ಹೋದರೂ, ಆತ್ಮಕ್ಕೆ ಚ್ಯುತಿ ಇಲ್ಲ ಎನ್ನುವ ವಿಷಯವನ್ನು ತಿಳಿಯಲು ಕೆಲವು ಜನಾಂಗಗಳಿಗೆ ಸಹಸ್ರಾರು ವರ್ಷಗಳೇ ಬೇಕಾದವು. ದೇಹವಲ್ಲದೇ ಆತ್ಮನಿರುವನು ಎಂಬುದನ್ನು ಯುಕ್ತಿಪೂರ್ವಕ ಅರಿಯುವುದಕ್ಕೆ ಇನ್ನಷ್ಟು ಸಮಯ ಬೇಕಾಯಿತು. ಜೀವಾತ್ಮನು ದೇಹದೊಂದಿಗೆ ತಾತ್ಕಾಲಿಕ ಸಂಬಂಧವನ್ನು ಹೊಂದಿದ್ದಾನೆ ಎಂಬ ಭಾವನೆ ಬಂದ ಮೇಲೆ ಆ ಜೀವಾತ್ಮ ಬರುವುದೆಲ್ಲಿಂದ, ಮುಂದೆ ಹೋಗುವುದೆಲ್ಲಿಗೆ, ಅವನ ಗಮನ, ಆಗಮನಗಳ ಕ್ರಮಗಳೇನು? ಎನ್ನುವ ವಿಚಾರವಾಗಿ ಚಿಂತನೆ ಪ್ರಾರಂಭವಾಯಿತು. ಈ ಬಗ್ಗೆ ಒಂದು ನಿಷ್ಕೃಷ್ಟ ಸಿದ್ಧಾಂತಕ್ಕೆ ಬರಲು ಸಾಕಷ್ಟು ಕಾಲಾವಕಾಶ ದೊಂದಿಗೆ ಬೇಕಷ್ಟು ವಿರೋಧವನ್ನೂ ವಿಚಾರವಂತರು ಎದುರಿಸಬೇಕಾಯಿತು. ಆದರೂ ವಿರೋಧ ಪರಿಹಾರ ಮಾಡಿ ಸತ್ಯವನ್ನು ಯುಕ್ತಿಯುಕ್ತವಾಗಿ ಸರ್ವಜನ ಪರಿಗ್ರಾಹಿಯಾಗುವಂತೆ ಸ್ಥಾಪಿಸಲು ನಮ್ಮ ಪೂರ್ವಜರು ಹಿಂದಾಗಲಿಲ್ಲ.

ಯೋಗಶಾಸ್ತ್ರದಂಥ ಒಂದು ಗ್ರಂಥವನ್ನು ಅಧ್ಯಯನ ಮಾಡಿದರೆ ಆ ಗ್ರಂಥದ ಉದಯಕ್ಕೆ ಮೊದಲು ಎಷ್ಟೊಂದು ಚಿಂತನ–ಮಂಥನ, ಪರಿಶೀಲನೆ, ಪ್ರಯೋಗ, ಸಾಧನೆ, ಅನುಷ್ಠಾನಗಳು ನಡೆದಿರಬೇಕು! ಆ ವಿಷಯಗಳಲ್ಲಿ ತಜ್ಞರಾದವರು ತಮ್ಮ ಅನುಭವಗಳನ್ನೂ, ಸಿದ್ಧಾಂತಗಳನ್ನೂ ಹೇಗೆ ತಮ್ಮ ಮುಂದಿನ ಪೀಳಿಗೆಗೆ ತಿಳಿಸಿರಬಹುದು–ಎನ್ನುವ ಬಗ್ಗೆ ಒಂದು ಕಲ್ಪನೆ ನಮ್ಮ ಮನಸ್ಸಿಗೆ ಬಂದೀತು. ಮ್ಯಾಕ್ಸ್​ಮುಲ್ಲರ್ ಅವರ ಅಭಿಪ್ರಾಯದಂತೆ ಯೋಗಶಾಸ್ತ್ರಕ್ಕೆ ಸುಮಾರು ಆರು ಸಹಸ್ರ ವರ್ಷಗಳ ಪ್ರಾಚೀನತೆ ಇದೆ. ಇತರ ಕೆಲವು ಮೂಲಗಳ ಪ್ರಕಾರ ಇನ್ನೂ ಹೆಚ್ಚಿನ ಪ್ರಾಚೀನತೆ ಇದೆ. ಗೌತಮನು ಬುದ್ಧನಾಗುವುದಕ್ಕೆ ಮೊದಲು ಅರಣ್ಯಗಳಲ್ಲಿ ಅಲೆಯುತ್ತಿರುವಾಗ ಅಸಂಖ್ಯ ಸಾಧಕ ತಪಸ್ವಿಗಳನ್ನು ಕಂಡ ವಿಚಾರವಿದೆ. ಯೋಗಸೂತ್ರಕಾರರಾದ ಪತಂಜಲಿ ಮಹಾ ಮುನಿ ಕ್ರಿ.ಪೂ. ಎರಡನೇ ಶತಮಾನಕ್ಕೆ ಸೇರಿದವರೆಂದು ಮ್ಯಾಕ್ಸ್​ಮುಲ್ಲರ್ ಅವರ ಅಭಿಪ್ರಾಯ. ಮನುಷ್ಯನ ಮನಸ್ಸಿನ ಸ್ಥಿತಿಗತಿಗಳನ್ನೂ, ಅದು ಏರಬಹುದಾದ ಎತ್ತರವನ್ನೂ ಅತ್ಯಂತ ಸ್ಪಷ್ಟವಾಗಿ ತರ್ಕಬದ್ಧವಾಗಿ ಪ್ರಯೋಗ–ಅನುಭವಪುರಸ್ಸರವಾಗಿ ಬರೆದ ಒಂದು ಅನುಷ್ಠಾನ ಗ್ರಂಥ ಅದು. ಅಣು–ಪರಮಾಣು–ರಾಕೆಟ್ಟುಗಳ ಯುಗ ಆರಂಭವಾಗುವುದಕ್ಕೆ ಮೊದಲು ಮುನ್ನೂರು ವರ್ಷಗಳ ವೈಜ್ಞಾನಿಕ ಪ್ರಗತಿಯ ಹಿನ್ನೆಲೆ ಇದ್ದಂತೆ, ಈಶ್ವರ, ಆತ್ಮ, ಪರಕಾಲ, ಜನ್ಮಾಂತರ ರಹಸ್ಯ, ಕರ್ಮವಾದ ಮೊದಲಾದ ವಿಚಾರಗಳ ಬಗ್ಗೆ ನಿರ್ಧಾರಾತ್ಮಕ ಸಿದ್ಧಾಂತಗಳನ್ನು ಪ್ರಕಟಿಸುವುದಕ್ಕೆ ಮೊದಲು, ಸಾಕಷ್ಟು ಅಧ್ಯಯನ ಅನ್ವೇಷಣೆಗಳು ಈ ದಿಸೆಯಲ್ಲಿ ನಡೆದಿರಬೇಕಲ್ಲವೇ?


\section*{ಸತ್ಯಾನ್ವೇಷಣೆಯ ಮುಕ್ತಮಾರ್ಗ}

\addsectiontoTOC{ಸತ್ಯಾನ್ವೇ\-ಷಣೆಯ ಮುಕ್ತಮಾರ್ಗ}

ಚಾರುವಾಕರ ಹೊರತಾಗಿ ಭಾರತೀಯ ತತ್ತ್ವಶಾಸ್ತ್ರಜ್ಞರು, ಸತ್ಯಾನ್ವೇಷಣೆಗಾಗಿ ಒಂದು ಸರ್ವ ಸಮ್ಮತವಾದ ಕ್ರಮವನ್ನು ಗೊತ್ತುಪಡಿಸಿಕೊಂಡಿದ್ದರು. ಪ್ರತ್ಯಕ್ಷ, ಅನುಮಾನ ಮತ್ತು ಶಬ್ದ ಪ್ರಮಾಣ ಎಂಬವು ಸತ್ಯಾನ್ವೇಷಣೆಯ ಆ ವಿಧಾನದ ಸೋಪಾನಗಳು. ವೀಕ್ಷಣೆ, ಊಹನೆ, ಪ್ರಯೋಗ–ಪರಿಶೀಲನೆಗಳನ್ನೊಳಗೊಂಡ ವೈಜ್ಞಾನಿಕ ವಿಧಾನದಂತಲ್ಲವೇ ಇದು? ಆದರೆ ಅವುಗಳಲ್ಲಿ ಒಂದು ವ್ಯತ್ಯಾಸವಿದೆ. ಇಲ್ಲಿ ಅನ್ವೇಷಿಸ ಹೊರಟ ತಥ್ಯಗಳು ಬಾಹ್ಯ ಜಗತ್ತನ್ನು ಕುರಿತಾದುವಲ್ಲ – ಅಂತರ್ಜಗ\-ತ್ತಿಗೆ ಸಂಬಂಧಿಸಿದವು. ಮನಸ್ಸಿನ ಆಳ–ಅಂತರಾಳದಲ್ಲಿ ಏನಿದೆ? ಸುಖ, ದುಃಖಗಳೇನು? ಬದುಕಿನ ಅರ್ಥ, ಉದ್ದೇಶಗಳೇನು? ಸಾವು ಏನು? ಸಾವಿನ ಆಚೆಗೆ ಏನು? ವ್ಯಕ್ತಿ ವ್ಯಕ್ತಿಗಳಲ್ಲಿ ಕಂಡು ಬರುವ ವೈಶಿಷ್ಟ್ಯ, ವೈವಿಧ್ಯಗಳಿಗೇನು ಮೂಲಕಾರಣ? ಇತ್ಯಾದಿ ಪ್ರಶ್ನೆಗಳ ಉತ್ತರ, ಅವುಗಳನ್ನು ಕುರಿತು ನಿಯಮ–ಅವರ ಅನ್ವೇಷಣೆಯ ವಸ್ತು ಅಥವಾ ಗುರಿ ಆಗಿತ್ತು.

ಸಾಮಾನ್ಯವಾಗಿ ನಾವು ಜ್ಞಾನವನ್ನು ಪಡೆಯುವ ವಿಧಾನ ಇಂದ್ರಿಯಗಳ ಮೂಲಕ. ಇದನ್ನು ಪ್ರತ್ಯಕ್ಷವೆನ್ನುತ್ತಾರೆ. ಅಕ್ಷವೆಂದರೆ ಇಂದ್ರಿಯಗಳು. ಕಣ್ಣು, ಕಿವಿ, ಮೂಗು, ನಾಲಗೆ, ಚರ್ಮ– ಇವುಗಳು ಬಾಹ್ಯವಸ್ತುಗಳ ಸಂಪರ್ಕಕ್ಕೆ ಬಂದಾಗ ಉಂಟಾಗುವ ಜ್ಞಾನವೇ ಪ್ರತ್ಯಕ್ಷಪ್ರಮಾಣ ಲಬ್ಧ ಜ್ಞಾನ. ಆಗಸದಲ್ಲಿ ಸೂರ್ಯೋದಯವಾದುದನ್ನು ಕಣ್ಣಿನಿಂದ ಗ್ರಹಿಸುತ್ತೇವೆ. ಯಾರಾದರೂ ಇನ್ನೂ ಸೂರ್ಯೋದಯವಾಗಿಲ್ಲ ಎಂದರೆ ಅವನನ್ನು ಸುಳ್ಳುಗಾರ ಎನ್ನಬೇಕಾಗುತ್ತದೆ. ಹಾಗೆಯೇ ಯಾರ ವಿಷಯವನ್ನೋ ಕೇಳಿದಾಗ, ಯಾವುದೋ ಸುಗಂಧ, ದುರ್ಗಂಧವನ್ನು ಮೂಸಿದಾಗ, ಸಿಹಿ, ಖಾರ ಪದಾರ್ಥಗಳ ರುಚಿ ಸವಿದಾಗ, ಶೀತೋಷ್ಣಗಳ ಅನುಭವವಾದಾಗ–ಅವುಗಳ ಅಸ್ತಿತ್ವದ ವಿಚಾರದಲ್ಲಿ ಸಂಶಯರಹಿತರಾಗುತ್ತೇವೆ. ಈ ಪ್ರತ್ಯಕ್ಷ ಪ್ರಮಾಣವನ್ನು ಯಾರೂ ವಿರೋಧಿಸು\-ವಂತಿಲ್ಲ. ಇದೇ ರೀತಿ ದೂರದರ್ಶನ ಹಾಗೂ ಹಿಂದಿನ ಜನ್ಮಗಳ ನೆನಪು–ಇವುಗಳ ಅನುಭವ ಯಾರ ಪಾಲಿಗೆ ಪ್ರತ್ಯಕ್ಷವಾಗುವುವೋ ಅಂಥವರ ಮಾತುಗಳು ಪುರಾವೆಗಳ ಮೂಲಕ ಸಮರ್ಥಿಸಲ್ಪಟ್ಟಾಗ, ಆ ತಥ್ಯಗಳ ಬಗೆಗೆ ನಾವು ಸಂದೇಹರಹಿತರಾಗುವೆವು. ಇದು ಪ್ರತ್ಯಕ್ಷದ ವಿಚಾರ.

ಪ್ರಪಂಚದ ಧಾರ್ಮಿಕರೆಲ್ಲರೂ, ಸರ್ವಶಕ್ತನೂ, ಸರ್ವಜ್ಞನೂ, ಸರ್ವವ್ಯಾಪಿಯೂ, ದಯಾ\-ಮಯನೂ ಆದ ಭಗವಂತನಲ್ಲಿ ವಿಶ್ವಾಸವಿಟ್ಟಿದ್ದಾರೆ. ಪೂಜೆ, ಪ್ರಾರ್ಥನೆ ಮತ್ತು ಮತೀಯ ವಿಧಿವಿಧಾನಗಳಲ್ಲಿ ಸಾಕಷ್ಟು ವ್ಯತ್ಯಾಸಗಳಿರಬಹುದಾದರೂ ಈ ಚರಾಚರ ಜಗತ್ತಿನ ಸೃಷ್ಟಿಸ್ಥಿತಿ ಲಯಕರ್ತೃವಾದ ದೇವರಲ್ಲಿ ಅಥವಾ ವಿಶ್ವನಿಯಾಮಕನಲ್ಲಿ ಒಂದಲ್ಲ ಒಂದು ತೆರನಾದ ನಂಬಿಕೆ ಯನ್ನಿಟ್ಟಿದ್ದಾರೆ. ಪವಿತ್ರ ಗ್ರಂಥಗಳು ಆತನ ಮಹಿಮೆಯನ್ನು ಕೊಂಡಾಡುತ್ತವೆ. ಆದರೆ ನ್ಯಾಯ\-ಪರನೂ, ಪರಮ ಕಾರುಣಿಕನೂ ಆದ ಆತನ ಸೃಷ್ಟಿಯ ವೈಚಿತ್ರ್ಯಗಳು ಕೆಲವೊಮ್ಮೆ ಆಸ್ತಿಕರನ್ನೇ ದಂಗುಬಡಿಸುತ್ತವೆ. ಕೆಲವು ವ್ಯಕ್ತಿಗಳ ಬದುಕಿನಲ್ಲಿ ಹುಟ್ಟುತ್ತಲೇ ಸುಖಸಂತೋಷಗಳು ಹಾಸು ಹೊಕ್ಕಾಗಿರುತ್ತವೆ. ಅವರಿಗೆ ದೃಢವಾದ ಆರೋಗ್ಯವಾದ ಶರೀರ, ಶಕ್ತಿಶಾಲಿಯಾದ ಮನಸ್ಸು ಬುದ್ಧಿಗಳು, ಅನುಕೂಲವಾದ ವಾತಾವರಣ–ಇವು ದೊರೆತು ಅವರು ಅಭಿವೃದ್ಧಿಯೆಡೆಗೆ ಆತಂಕವಿಲ್ಲದೇ ಮುನ್ನಡೆಯುತ್ತಾರೆ. ಇತರ ಕೆಲವರು ಆಜನ್ಮದುಃಖಿಗಳು, ಕೆಲವರಿಗೆ, ಅಂಗವಿಕಲತೆ, ಕೆಲವರಿಗೆ ಬುದ್ಧಿಮಾಂದ್ಯ–ಅವರೆಲ್ಲ ಹೇಗೋ ತಮ್ಮ ದುಃಖಮಯ ಜೀವನವನ್ನು ಕಳೆಯುವರು. ದುಃಖ ತಡೆಯದೆ ಆತ್ಮಹತ್ಯೆಯ ಪಾಪಕಾರ್ಯವನ್ನೂ ಎಸಗುವರು. ಈ ಜೀವನದಲ್ಲಿ ಕಷ್ಟಸಂಕಟಗಳನ್ನು ಅನುಭವಿಸಿದರೆ ಮುಂದೆ ಸುಖಶಾಂತಿಗಳು ಲಭಿಸುತ್ತವೆ ಎನ್ನುವ ಕೆಲವು ಮತಗಳ ಬೋಧನೆಯು, ಸಮಸ್ಯೆಯನ್ನು ಸ್ವಲ್ಪವೂ ಪರಿಹರಿಸುವುದಿಲ್ಲ. ಕಷ್ಟ, ಸಂಕಟ, ನರಳಾಟಗಳ ಕಾರಣವನ್ನು ತಿಳಿಸು\-ವುದೂ ಇಲ್ಲ. ಇಲ್ಲಿಯೇ ಯುಕ್ತಿಯು ತನ್ನ ಕೆಲಸವನ್ನು ಪ್ರಾರಂಭಿಸುವುದು. ತನ್ನ ಅನುಭವ ಮತ್ತು ಅರಿವಿನ ಪರಿಧಿಗೆ ನಿಲುಕವ ವಿಚಾರಗಳ ಆಧಾರದಿಂದ ಸಾಮಾನ್ಯವಾಗಿ ಅರಿಯಲಾರದ ಸೂಕ್ಷ್ಮವಾದ ಕಾರಣಗಳನ್ನು ಅನ್ವೇಷಿಸಲು ತವಕಿಸುವುದು– ಇದೇ ಅನುಮಾನ.

‘ಬೀಜವಿಲ್ಲದೇ ವೃಕ್ಷವಿಲ್ಲ’ ಎಂದಂತೆ ಕಾರಣವಿಲ್ಲದೇ ಕಾರ್ಯವಿಲ್ಲ ಎಂಬುದು ವಿಜ್ಞಾನದ ಒಂದು ಅಭೇದ್ಯ ನಿಯಮವಾಗಿದೆ. ಕಾರಣ ಮತ್ತು ಪರಿಣಾಮ ಬೇರೆ ಬೇರೆ ಎನ್ನುವುದು ಸಾಮಾನ್ಯ ದೃಷ್ಟಿ. ಕಾರಣ ಯಾವಾಗಲೂ ಸೂಕ್ಷ್ಮ; ಪರಿಣಾಮ ಸ್ಥೂಲ. ಹಲವು ವೇಳೆ ಕಾರಣ ನಮ್ಮ ದೃಷ್ಟಿಗೆ ಗೋಚರಿಸುವುದಿಲ್ಲ. ಪರಿಣಾಮ ಮಾತ್ರ ತಿಳಿಯುತ್ತದೆ. ಈ ಕಾರ್ಯಕಾರಣ ಸಂಬಂಧವನ್ನು ಸರಿಯಾಗಿ ತಿಳಿಯಲು ಸಾಮಾನ್ಯ ಪರಿಶೀಲಕನಿಗೆ ಸಾಧ್ಯವಾಗದು. ‘ಹಸುರಾಗಿದ್ದ ಮಾವಿನಕಾಯಿ ಈಗೇಕೆ ಕೆಂಪು ಅಥವಾ ಹಳದಿ ಬಣ್ಣ ತಾಳಿದೆ?’ ಎಂದು ಮಗು ಕೇಳಿದಾಗ ‘ಅದು ಕಾಯಿಯಾಗಿದ್ದಾಗ ಹಸುರಾಗಿರುತ್ತದೆ, ಹಣ್ಣಾದಾಗ ಬೇರೆ ಬೇರೆ ಬಣ್ಣಗಳನ್ನು ಪಡೆಯುತ್ತದೆ’ ಎಂದು ಹೇಳಿದರೆ ಅದೊಂದು ವಿವರಣೆಯಾಯಿತೇ ಹೊರತು ಕಾರಣ ಪರಿಶೋಧನೆಯಾಗಲಿಲ್ಲ. ‘ಕಾಯಿಯಲ್ಲಿದ್ದ ಪಿಷ್ಟಾಂಶವು ಸಕ್ಕರೆಯಾಗಿ ಪರಿಣಾಮ ಹೊಂದುವಾಗ ರಾಸಾಯನಿಕ ಬದಲಾವಣೆಗಳಿಂದ ಹಾಗಾಯಿತು’ ಎಂದರೆ ಕಾರಣವನ್ನು ಕುರಿತು ಸ್ವಲ್ಪ ಹೇಳಿದಂತಾಯಿತು. ಎಲ್ಲ ವೈಜ್ಞಾನಿಕ ಪರಿಶೀಲಕರೂ ವಸ್ತುವಿನಲ್ಲೇ ಅದರ ಸದ್ಯದ ಸ್ಥಿತಿಗೆ ಕಾರಣವನ್ನು ಕಂಡು ಹಿಡಿಯುತ್ತಾರೆ. ಕಾರಣವೆಂದರೆ ಪರಿಣಾಮಕ್ಕೆ ಮೊದಲಿನ ಅವ್ಯಕ್ತಸ್ಥಿತಿ; ಪರಿಣಾಮವು ಕಾರಣದ ವ್ಯಕ್ತಸ್ಥಿತಿ ಎಂಬುದನ್ನು ಅರಿತಾಗ ಕಾರ್ಯ–ಕಾರಣಗಳ ಅವಿಭಾಜ್ಯ ಸಂಬಂಧದ ಪರಿಚಯ ನಮಗಾಗುವುದು. ಅತಿ ವಿಶಾಲವಾಗಿ ಕಂಗೊಳಿಸುವ ಆಲದ ಮರದ ಸೂಕ್ಷ್ಮವಾದ ಬೀಜದಲ್ಲಿ ಹುದುಗಿಕೊಂಡಿದೆ.\break ಯಾವುದು ಆಗಲೇ ಸೂಕ್ಷ್ಮವಾದ ಬೀಜದಲ್ಲಿ ಹುದುಗಿಕೊಂಡಿತ್ತೋ, ಅದು ಮಣ್ಣು, ನೀರು, ಗಾಳಿ ಮತ್ತು ಸೂರ್ಯಕಿರಣಗಳಿಂದ ತನಗೆ ಬೇಕಾದುದನ್ನು ಪಡೆದು ಬಲಿಯಿತು. ಎತ್ತರಕ್ಕೆ ಬೆಳೆಯಿತು. ಸೂಕ್ಷ್ಮವಾಗಿದ್ದುದು ಸ್ಥೂಲವಾಗಿ ವ್ಯಕ್ತವಾಯಿತು. ಮಾವು, ಬೇವು, ಹಲಸು, ಗೇರು–ಎಲ್ಲ ತೆರನಾದ ವಿವಿಧ ವೃಕ್ಷಗಳೂ ತಮ್ಮ ತಮ್ಮ ಮೂಲಕಾರಣಗಳಾದ ಬೀಜಗಳಿಂದ ಸಮಾನವಾಗಿ ಮಣ್ಣು, ಗೊಬ್ಬರ, ನೀರು, ಗಾಳಿ, ಸೂರ್ಯಕಿರಣಗಳನ್ನು ಪಡೆದು, ತಮ್ಮ ಬೆಳವಣಿಗೆಯ ನಿಯಮಕ್ಕನುಗುಣವಾಗಿ ಬೆಳೆದು, ತಮ್ಮ ವೈಶಿಷ್ಟ್ಯಗಳನ್ನೇ ವ್ಯಕ್ತಗೊಳಿಸಿದವು. ಬೇವನ್ನು ನೆಟ್ಟು\break ಮಾವನ್ನು ಪಡೆಯಲು ಸಾಧ್ಯವಿಲ್ಲವಷ್ಟೆ. ಎಂಥ ಪ್ರಬಲವಾದ ಸನ್ನಿವೇಶ ಅಥವಾ ವಾತಾವರಣವನ್ನು ನಿರ್ಮಿಸಿದರೂ, ಸೂಕ್ಷ್ಮವಾದ ಕಾರಣವು ತನ್ನ ವೈಶಿಷ್ಟ್ಯದೊಂದಿಗೆ ವ್ಯಕ್ತವಾಗುವುದನ್ನು ತಡೆಯಲು ಸಹಜವಾಗಿ ಸಾಧ್ಯವಾಗದು. ಆದುದರಿಂದ ಯಾವುದನ್ನು ನಾವು ಪರಿಣಾಮದಲ್ಲಿ ಕಾಣುತ್ತೇವೋ, ಅದು ಕಾರಣದಲ್ಲಿ ಸೂಕ್ಷ್ಮವಾಗಿ ಅಡಗಿಕೊಂಡಿದೆ–ಎಂಬುದು ಸ್ಪಷ್ಟವಾಗುವುದು. ಇದು ಅತ್ಯಂತ ಯುಕ್ತಿಯುಕ್ತವೂ, ವೈಜ್ಞಾನಿಕ ಅನ್ವೇಷಣಾ ಪದ್ಧತಿಗೆ ಸಮ್ಮತವೂ ಆದ ವಿಚಾರ.

ಹುಡುಗನೊಬ್ಬ ಇನ್ನೂ ಸರಿಯಾಗಿ ಮಾತು ಕಲಿಯುವುದಕ್ಕೆ ಮೊದಲು ಇತರರು ಹಾಡುತ್ತಿರು\-ವಾಗಲೇ ‘ಅದು ಅಂತಿಂಥ ರಾಗ’ ಎಂದು ಸ್ಪಷ್ಟವಾಗಿ ಹೇಳುತ್ತಿದ್ದ. ಸುಮಾರು ಇನ್ನೂರು ರಾಗಗಳನ್ನು ಗುರುತಿಸಬಲ್ಲವನಾಗಿದ್ದ. ಅವನಲ್ಲಿ ಕಂಡುಬಂದ ಈ ಸಿದ್ಧಿಗೆ ಕಾರಣವೆಲ್ಲಿರಬೇಕು? ಯಾವುದೇ ವಿಷಯ ಅಥವಾ ಕಲೆಯಲ್ಲಿ ನೈಪುಣ್ಯ ಸಾಧ್ಯವಾಗುವುದು ಪ್ರಜ್ಞಾಪೂರ್ವಕ ಪ್ರಯತ್ನದಿಂದ ಅಥವಾ ಅಭ್ಯಾಸ ಬಲದಿಂದ. ಆದರೆ ಈ ಜೀವನದಲ್ಲಿ ಅಂಥ ಅಭ್ಯಾಸ ಮಾಡಲು ಅವನಿಗೆ ಅವಕಾಶವಿಲ್ಲ ಎಂದರೆ ಆ ಪರಿಣತಿ ಅವನಿಗೆ ಲಭ್ಯವಾಗಿರುವುದಾದರೂ ಹೇಗೆ? ಸದ್ಯದ ಅವನ ಈ ಸಿದ್ಧಿಗೆ ಕಾರಣವನ್ನು ಅವನ ಮನಸ್ಸಿನ ಆಳದ ಅವ್ಯಕ್ತ ಸ್ಥಿತಿಯಲ್ಲೇ ಹುಡುಕಬೇಕಲ್ಲವೇ? ಮನುಷ್ಯನ ವಿಶಿಷ್ಟ ಸಿದ್ಧಿಗಳಿಗೆ, ಅಭಿರುಚಿ, ಸಾಮರ್ಥ್ಯಗಳಿಗೆ, ಅವನ ಹಿಂದಿನ ಜೀವನದ ಅಭ್ಯಾಸ ಅಧ್ಯಯನಗಳು ಕಾರಣವೆಂದಾದರೆ, ಅವನ ಸುಖದುಃಖಗಳಿಗೂ, ಅವನು ಬುದ್ಧಿ ಪೂರ್ವಕವಾಗಿ ಮಾಡಿದ ಸತ್ಕರ್ಮ ದುಷ್ಕರ್ಮಗಳಿಗೂ ಸಂಬಂಧವಿದೆ ಎಂದರೆ ತರ್ಕ ವಿರೋಧವಲ್ಲ. ಸತ್ಯವನ್ನು ಕುರಿತು ತಾರ್ಕಿಕವಾಗಿ ನಾವು ನಮ್ಮ ನಿಲುಮೆಯನ್ನು ದೃಢಪಡಿಸುವ ವಿಧಾನ ಇದು. ಈ ವಿಚಾರ ವಿಧಾನವನ್ನು ಮುಂದೆ ನಾವು ವೈಜ್ಞಾನಿಕ ವಿಶ್ಲೇಷಣೆಗೆ ಗುರಿಪಡಿಸೋಣ. ಪ್ರಪಂಚದ ಎಲ್ಲ ದೇಶಗಳಲ್ಲೂ, ಎಲ್ಲ ಕಾಲಗಳಲ್ಲೂ, ಪರಮಾರ್ಥವನ್ನು ತಿಳಿಯಲು ತೀವ್ರ ಹಂಬಲದಿಂದ, ಆಜೀವನ ತಪಸ್ಸು ಸಾಧನೆ ಅನ್ವೇಷಣೆಗಳನ್ನು ಮಾಡಿ, ಯಥಾರ್ಥವಾದ ಅತೀಂದ್ರಿಯಾನುಭೂತಿಯನ್ನು ಪಡೆದ ಮಹಾತ್ಮರು ಆಗಿ ಹೋಗಿದ್ದಾರೆ. ಅಂಥವರು ಜಾಹಿರಾತು ಪ್ರಸಿದ್ಧಿಗಳನ್ನು ಪಡೆಯದಿದ್ದರೂ, ಇಂದೂ ಇಲ್ಲದಿಲ್ಲ. ಸತ್ಯಸಾಕ್ಷಾತ್ಕಾರಕ್ಕಾಗಿ ನಿರಂತರ ಶ್ರಮಿಸಿದವರು ಅವರು. ಲೌಕಿಕ ಆಸೆ ಆಕಾಂಕ್ಷೆಗಳನ್ನು ತೊರೆದವರು ಅವರು. ದಯೆ ಅನುಕಂಪೆಗಳೇ, ಮನುಷ್ಯಕುಲದ ಮಂಗಲ ಚಿಂತನೆಯೆ, ಅವರ ಎಲ್ಲ ತೆರನಾದ ಚಟುವಟಿಕೆಗಳಿಗೆ ಪ್ರೇರಣೆ. ಅವರ ವಿರಕ್ತಿ, ಭಕ್ತಿ, ನೈರ್ಮಲ್ಯ, ತ್ಯಾಗ, ವಿವೇಕ, ಸಮರ್ಪಣೆಗಳು ಯಥಾರ್ಥವಾಗಿರಲು ಕಾರಣ ಅವರು ದೃಢವಾಗಿ ನಿಂತ ನೆಲ–ಅವರಿಗೆ ಆಧಾರವಾದ ಆಧ್ಯಾತ್ಮಿಕ ಅನುಭೂತಿಗಳು. ಅವರ ಉಪದೇಶಗಳು ಪಾಂಡಿತ್ಯಪೂರ್ಣ ಪ್ರವಚನಗಳಾಗದಿರಬಹುದು. ಆದರೆ ಅವು ಅವರ ‘ಜೀವನ’ ಎನ್ನುವ ಪುಸ್ತಕದ ಅನುಭವದ ಹಾಳೆಗಳು. ಪ್ರಪಂಚದ ಪ್ರತಿಯೊಂದು ಧರ್ಮದ ಮೂಲಕ್ಕೆ ಹೋದರೂ, ಆ ಧರ್ಮವು ಅಂಥ ವ್ಯಕ್ತಿಗಳ ವಜ್ರೋಪಮವಾದ ಅನುಭವಗಳ ಮೇಲೆ ನಿಂತಿದೆ ಎಂಬುದು ಅರ್ಥವಾಗುವುದು. ಇಂದ್ರಿಯಗಳು ಸಾಮಾನ್ಯವಾಗಿ ನಮಗೆ ತೋರಿಸಿಕೊಡುವುದಕ್ಕಿಂತ ಹಿರಿದಾದ ಸತ್ಯವನ್ನು ತಾವು ಕಂಡೆವೆಂತಲೂ, ಅದನ್ನು ಪ್ರತಿಯೊಬ್ಬರೂ, ತಾವು ಕಂಡುಕೊಂಡ ದಾರಿಯಲ್ಲಿ ಪ್ರಾಮಾಣಿಕತೆಯಿಂದ ನಡೆದರೆ ಪಡೆಯಬಹುದೆಂದೂ ಹೇಳುತ್ತಾರೆ. ಅವರ ಮಾತಿಗೆ ಆಪ್ತವಾಕ್ಯ ಅಥವಾ ಶಬ್ದಪ್ರಮಾಣ ಎನ್ನುತ್ತಾರೆ. ನಮ್ಮ ದೇಶದಲ್ಲಿ ಪುಷಿಗಳಿಗೆ ಗೋಚರಿಸಿದ ಈ ಸತ್ಯಗಳನ್ನು ಶ್ರುತಿ ಎಂದು ಹೇಳುತ್ತಾರೆ. ಎಲ್ಲ ದೇಶಗಳಿಗೂ, ಕಾಲಗಳಿಗೂ, ಅನ್ವಯಿಸುವಂಥ ಕೆಲವು ಪ್ರಮುಖ ಆಧ್ಯಾತ್ಮಿಕ ಸತ್ಯಗಳನ್ನು ಅಲ್ಲಿ ಸಂಗ್ರಹಿಸಲಾಗಿದೆ. ಆ ಸತ್ಯಗಳನ್ನು ಪ್ರತಿಯೊಬ್ಬರೂ ತಮ್ಮ ಬದುಕಿನಲ್ಲಿ ಸಾಕ್ಷಾತ್ಕರಿಸಿಕೊಳ್ಳಬಹುದು. ಆ ಸತ್ಯದ ಸಾಕ್ಷಾತ್ಕಾರದಿಂದ ಮನುಷ್ಯನು ಆತ್ಯಂತಿಕ ದುಃಖನಿವೃತ್ತಿಯನ್ನೂ, ಶಾಶ್ವತ ಆನಂದ ಸ್ವಾತಂತ್ರ್ಯಗಳನ್ನೂ ಪಡೆಯುತ್ತಾನೆ. ಜೀವನದ ಕೊನೆಯ ಗುರಿಯೇ ಅದು ಎಂದು ಅವರು ಹೇಳುತ್ತಾರೆ.

\medskip


\section*{ಅದ್ಭುತಗಳ ಯಥಾರ್ಥತೆ}

\addsectiontoTOC{ಅದ್ಭುತಗಳ ಯಥಾರ್ಥತೆ}

ಪರಮ ಸತ್ಯಸಾಕ್ಷಾತ್ಕಾರದ ಸಾಧನೆಯಲ್ಲಿ ಆಪ್ತವಾಕ್ಯ ಅಥವಾ ಶಬ್ದ ಪ್ರಮಾಣ, ಯುಕ್ತಿ ಅಥವಾ ತರ್ಕದ ಮೂಲಕ ವಿಚಾರದ ಒರೆಗಲ್ಲಿನಲ್ಲಿ ವಿಮರ್ಶಿಸುವ ವಿಧಾನ ಮತ್ತು ಸ್ವಂತ ಬದುಕಿನಲ್ಲಿ ಆ ಅನುಭವ–ಈ ಮೂರು ವಿಧಾನಗಳೂ ಸೇರಿಕೊಂಡಿವೆ. ಯುಕ್ತಿತರ್ಕಗಳನ್ನೂ, ಸ್ವಾನುಭವವನ್ನೂ ಬಿಟ್ಟು ಕೇವಲ ಶಬ್ದಪ್ರಮಾಣವನ್ನೇ ಅಧ್ಯಯನ ಮಾಡುವ ವ್ಯಕ್ತಿ, ತಾನು ಹೇಳುವುದೇ ಸತ್ಯ, ತನ್ನ ಧರ್ಮವೇ ಸತ್ಯ ಎನ್ನುತ್ತಾ, ತನ್ನ ನಂಬಿಕೆಯ ಪವಿತ್ರಗ್ರಂಥಗಳಿಗೆ ಅರ್ಥವಾಖ್ಯಾನ ನೀಡುವ ಯಂತ್ರವಾಗುವ, ಮತಾಂಧನಾಗುವ ಸಂಭವವಿದೆ. ಯುಕ್ತಿ ಮತ್ತು ಆಪ್ತವಾಕ್ಯಗಳನ್ನು ಗಮನಿಸದೆ, ತನ್ನ ಅನುಭವವೇ ಮುಖ್ಯ ಎನ್ನುವ ಸಾಧಕನು ತನ್ನ ಸುಪ್ತ ಮನಸ್ಸಿನ ಭ್ರಮಾತ್ಮಕ ದೃಶ್ಯಗಳನ್ನೇ ‘ಮಹಾ ಅನುಭವಗಳು’ ಎಂದುಕೊಳ್ಳಬಹುದು. ಪರಿಮಿತ ಪರಿಧಿಯೊಳಗೇ ಸಂಚರಿಸುವ ಮತ್ತು ಇಂದ್ರಿಯಗಳು ನಮಗೆ ತಂದುಕೊಡುವ ಜ್ಞಾನವನ್ನು ಆಧರಿಸಿ ನಿಂತ ತರ್ಕ, ಯುಕ್ತಿಗಳನ್ನೆ ಹೊಂದಿಕೊಂಡ ವ್ಯಕ್ತಿ, ತನ್ನ ಸುಪ್ತ ಆಸೆ, ಆಕಾಂಕ್ಷೆ, ಅನಿಸಿಕೆಗಳೇ ಸತ್ಯವೆಂದು ತೋರಲು ತನ್ನ ಬುದ್ಧಿಬಲವನ್ನು ವ್ಯಯಿಸಬಹುದು. ಆದರೆ ಆಪ್ತವಾಕ್ಯ, ತರ್ಕ ಮತ್ತು ಅನುಭವ–ಈ ಮೂರು ಒಂದೇ ಸಿದ್ಧಾಂತವನ್ನು ಸಂಗತವೆಂದು ತೋರಿಸಿಕೊಟ್ಟಾಗ ಅದು ಯಥಾರ್ಥ ಸತ್ಯ ಎಂದು ಸಿದ್ಧವಾಗುವುದು. ಜನ್ಮಜನ್ಮಾಂತರದ ಯಥಾರ್ಥತೆಯ ನಿರ್ಧಾರಕ್ಕೂ ಇದನ್ನೇ ಅನ್ವಯಿಸಬಹುದು. ನಮ್ಮ ಶಾಸ್ತ್ರಗ್ರಂಥಗಳ ಆ ಬೋಧನೆಯನ್ನು ವೈಜ್ಞಾನಿಕ ವಿಚಾರವಿಮರ್ಶೆಯ ನಿಕಷಕ್ಕೆ ಒಡ್ಡಿ, ಅನೇಕ ಅನುಭವಗಳ ಸಾಕ್ಷಿಯ ನೆಲೆಯಲ್ಲಿ ಪರೀಕ್ಷಿಸಿದಾಗ ನಮಗಾಗುವುದು ಅದೇ ಸತ್ಯದ ಸಾಕ್ಷಾತ್ಕಾರ: ‘ಪುನರಪಿ ಜನನಂ, ಪುನರಪಿ ಮರಣಂ.’ ಆದುದರಿಂದಲೇ ಒಕ್ಕೊರಳ ಮೊರೆ: ‘ಕೃಪಯಾವಾರೇ, ಪಾಹಿ ಮುರಾರೇ.’


\section*{ವಿಶ್ವನಿಯಮ}

\addsectiontoTOC{ವಿಶ್ವನಿಯಮ}

ಹಲವಾರು ಜೀವನಗಳಲ್ಲಿ ನಡೆಸಿದ ಹೋರಾಟದ ಮೂಲಕ ಜೀವಾತ್ಮ ಮೆಲ್ಲನೆ ವಿಕಾಸದ ಮೇಲಿನ ಮಜಲುಗಳನ್ನು ತಲುಪುತ್ತಾನೆ ಎನ್ನುವುದು ಈ ಕರ್ಮಸಿದ್ಧಾಂತದ ಬೋಧನೆ. ಡಾರ್ವಿನ್​ನ ವಿಕಾಸವಾದವು ಅರ್ಥಪೂರ್ಣವೆಂದು ತೋರುವವರಿಗೆ ಜನ್ಮಾಂತರವಾದದಲ್ಲಿ ಕಂಡುಬರುವ\break ವಿಕಾಸ ವಿಧಾನದ ವಿಚಾರವನ್ನು ಅರ್ಥೈಸಿಕೊಳ್ಳಲು ಕಷ್ಟವಾಗದು. ವಿಕಾಸವಾದದ ಪ್ರಾಥಮಿಕ ಸಿದ್ಧಾಂತವನ್ನು ಎರಡು ಪ್ರಮುಖ ಅಂಶಗಳಲ್ಲಿ ಕರ್ಮ–ಜನ್ಮಾಂತರವಾದ ವಿಸ್ತೃತ ಗೊಳಿಸುವುದು.

೧.\ ವಿಕಾಸವೆನ್ನುವುದು ಕೇವಲ ಜೀವಿಯ ಹೊರ ಆಕಾರಕ್ಕೇ ಸೀಮಿತವಾದುದಲ್ಲ. ಅದು ಜೀವಿಯ ಪ್ರಜ್ಞೆಯ ವಿಕಾಸವನ್ನೂ ಒಳಗೊಂಡಿದೆ.

೨.\ ಆಕೃತಿಯಲ್ಲಿ ಮಾತ್ರ ಪ್ರಗತಿಯಲ್ಲ. ಮನಸ್ಸು ಮತ್ತು ಆಧ್ಯಾತ್ಮಿಕ ವಿಷಯಗಳಿಗೆ\break ಸಂಬಂಧಿಸಿ ಬದಲಾವಣೆ ಪ್ರಗತಿಯ ದಿಕ್ಕಿನಲ್ಲಿ. ಇದು ಈ ಸಿದ್ಧಾಂತ ತಿಳಿಸಿಕೊಡುವ ವಿಕಾಸ ವಿಧಾನ.

ಭೌತವಿಜ್ಞಾನದ ನಿಯಮಗಳು ಕರಾರುವಾಕ್ಕಾಗಿ ಹೇಗೆ ವರ್ತಿಸುವುವೋ ಅಂತೆಯೇ ಕಾರ್ಯ ಕಾರಣ ಅಥವಾ ಕ್ರಿಯೆ ಪ್ರತಿಕ್ರಿಯೆಯ ಈ ಕರ್ಮನಿಯಮವೂ ಕೆಲಸಮಾಡುವುದು. ಎಂದರೆ ಹಿಂದು ಹಿಂದಿನ ಜೀವನಗಳಲ್ಲಿ ಘಟಿತವಾದ ಚಾರಿತ್ರ್ಯದ ಒಳಿತು ಕೆಡಕು ನ್ಯೂನಾತಿರೇಕಗಳನ್ನು ಅನುಸರಿಸಿ ವ್ಯಕ್ತಿಯೊಬ್ಬ ಪರಿಸರ ಮತ್ತು ವಂಶಪರಂಪರೆಯ ಕೊಡುಗೆಗಳೊಂದಿಗೆ ಶರೀರಧಾರಿಯಾಗಿ ಜಗತ್ತನ್ನು ಪ್ರವೇಶಿಸುತ್ತಾನೆ. ತಾನೇ ರೂಪಿಸಿ ಆಯ್ಕೆಮಾಡಿಕೊಂಡ ಜಗತ್ತಿನಲ್ಲಿ ಮನುಷ್ಯ ಬರಬೇಕಾಗುತ್ತದೆ.

ಗೀನಾ ಸೆರ್ಮಿನಾರಾ ಹೇಳುತ್ತಾರೆ: ‘ನನಗೆ ಗೊತ್ತಿರುವ ಪ್ರೊಟೆಸ್ಟೆಂಟ್ ಧರ್ಮಗುರು\-ಗಳೊ\-ಬ್ಬರು ಉದಾರಮನಸ್ಕರು. ಆಧುನಿಕ ವಿಚಾರ ವಿಮರ್ಶಾವಿಧಾನಗಳಲ್ಲಿ ಆಸಕ್ತರು. ಕರ್ಮ ಜನ್ಮಾಂತರವಾದದ ಬಗೆಗೆ ಸಾಕಷ್ಟು ಅಧ್ಯಯನ ಮಾಡಿದವರು. ‘ನೀವು ನಿಮ್ಮ ಧಾರ್ಮಿಕ ವೇದಿಕೆಯಲ್ಲಿ–ನೀಡುವ ಪ್ರವಚನದಲ್ಲಿ ಈ ಜನ್ಮಾಂತರ ಮತ್ತು ಕರ್ಮಸಿದ್ಧಾಂತವನ್ನು ಕುರಿತು ಎಂದಾದರೂ ಬೋಧಿಸುವ ಧೈರ್ಯಮಾಡಿದ್ದುಂಟೇ?’ ಎಂದು ನಾನು ಪ್ರಶ್ನಿಸಿದೆ. ‘ಹೌದು. ಕೆಲವು ಬಾರಿ ಮಾತ್ರ. ಅತ್ಯಂತ ಜಾಗರೂಕತೆಯಿಂದ. ಕಳೆದ ಬಾರಿ ಜನ್ಮಾಂತರವಾದವನ್ನು ಕುರಿತು ಹೇಳಿ ನಮ್ಮ ಕ್ರೈಸ್ತಸಂಪ್ರದಾಯದ ನರಕಸಿದ್ಧಾಂತದೊಂದಿಗೆ ಹೋಲಿಸಿದೆ. ಯಾವುದನ್ನು ನಂಬಬೇಕು ಎಂಬುದು ಕೇಳುಗರಿಗೆ ಬಿಟ್ಟದ್ದು’ ಎಂದು ಎಂದರವರು.

ಇದೊಂದು ಹೊಸ ದಿಟ್ಟ ಹೆಜ್ಜೆಯ ಪ್ರತೀಕವೆಂದರೆ ತಪ್ಪಲ್ಲ. ಇಂಥ ಮಾತನ್ನು ಹೇಳಲು ಧೈರ್ಯಮಾಡಿದ ಕ್ರೈಸ್ತಧರ್ಮಗುರುಗಳು ಆ ಧರ್ಮದ ಮತಪಂಡಿತರ ಮತ್ತು ಅಧಿಕಾರಿಗಳ ತೀವ್ರ ಅವಗಣನೆಗೆ ಗುರಿಯಾಗುವುದು ಸಹಜ, ಮಾತ್ರವಲ್ಲ ಹೆಚ್ಚಿನ ಚರ್ಚುಗಳಲ್ಲಿ ಅವರು ತಮ್ಮ ಸ್ಥಾನವನ್ನೇ ಕಳೆದುಕೊಳ್ಳಬೇಕಾಗುವುದು. ಕೆಲವು ವರ್ಷಗಳ ಹಿಂದೆ ಇದು ‘ಧರ್ಮ ದ್ರೋಹ’ ಎಂದು ಪರಿಗಣಿತವಾಗಿ ಜೈಲುವಾಸ ಅಥವಾ ಶಿರಚ್ಛೇದನ ಶಿಕ್ಷೆಗೆ ಅರ್ಹವಾದ ಅಪರಾಧ\-ವೆನಿಸಿ\-ಬಿಡುತ್ತಿತ್ತು.

ಪಶ್ಚಿಮದ ಸಂಸ್ಕೃತಿ ಸಮಾಜ ಉಳಿಯಬೇಕಾದರೆ ಕರ್ಮ ಸಿದ್ಧಾಂತದ ಅರಿವನ್ನು ಅಲ್ಲಿನ ಜನಮಾನಸದಲ್ಲಿ ಪ್ರಸಾರಗೊಳಿಸಬೇಕೆಂದು ಪ್ರಸಿದ್ಧ ಆಂಗ್ಲ ಲೇಖಕ ಪಾಲ್ ಬ್ರಂಟನ್ ಒಮ್ಮೆ ಹೇಳಿದ್ದುಂಟು. ಅತಿ ವಿಷಯಾಸಕ್ತಿ ಹಾಗೂ ಲಂಪಟತೆಯ ಮೂಗುದಾರ ಬಿಗಿಗೊಳಿಸಲು ವ್ಯಕ್ತಿ ಯೊಬ್ಬ ಸಂಯಮ, ಸ್ವನಿಯಂತ್ರಣ ಶಕ್ತಿಯನ್ನು ಗಳಿಸಿಕೊಳ್ಳಲು ಈ ಸಿದ್ಧಾಂತ ಸಹಾಯಕ ಎಂಬುದು ಆ ಮಾತಿನ ಇಂಗಿತ. ಚರ್ಚ್ ಹಾಗೂ ಕ್ರೈಸ್ತಧರ್ಮಾಧಿಕಾರಿಗಳಲ್ಲಿ ಹೆಚ್ಚು ಕಡಿಮೆ ಎಲ್ಲರೂ ಇದು ಪೌರಸ್ತ್ಯ ಮೂಢನಂಬಿಕೆ ಎಂದು ನಾನಾ ರೀತಿಯಲ್ಲಿ ವಿರೋಧಿಸಿದರೂ ಒಂದು ಗಣನೆಯ ಪ್ರಕಾರ ಅಮೇರಿಕಾದೇಶದ ನಾಲ್ವರಲ್ಲಿ ಒಬ್ಬನು ಈ ಸಿದ್ಧಾಂತವನ್ನು ಒಪ್ಪುತ್ತಾನೆ. ಬ್ರಿಟನ್ನಿನ ಸಂಡೇ ಟೆಲಿಗ್ರಾಫ್ ಸಂಪ್ರದಾಯ ಸಂರಕ್ಷಣಶೀಲ ಮನೋಭಾವದ ಪತ್ರಿಕೆ. ಅದು ನಡೆಸಿದ ಪರಿಶೀಲನೆಯಂತೆ ಬ್ರಿಟನ್ನಿನಲ್ಲಿ ಈ ಹತ್ತುವರ್ಷಗಳಲ್ಲಿ ಜನಸಾಮಾನ್ಯರ ನಂಬಿಕೆಯ ಪ್ರಮಾಣ (ಕರ್ಮ ಮತ್ತು ಜನ್ಮಾಂತರ ಸಿದ್ಧಾಂತಗಳಲ್ಲಿ) ಶೇಕಡಾ ಹದಿನೆಂಟರಿಂದ ಇಪ್ಪ ತ್ತೆಂಟಕ್ಕೆ ಏರಿದೆ. ೧೯೮೭ರಲ್ಲಿ ಒಬ್ಬ ಮನೋವಿಜ್ಞಾನಿ ಡಾ. ರೋಜರ್ ವೂಲ್ಗರ್ ‘ಇತರ ಜೀವನಗಳು ಇತರ ವ್ಯಕ್ತಿತ್ವಗಳು’\footnote{\engfoot{Other Lives, Other Selves: A Jungian Psychotherapist Discovers Past Lives, Roger.}\hfill\engfoot{ —J. Woolger. Bantam Books.}} ಎಂಬ ಗ್ರಂಥವನ್ನು ಬರೆದರು. ಅವರು ಯೂಂಗ್ ಮನೋ ವಿಜ್ಞಾನದ ಸಂಪ್ರದಾಯದಲ್ಲಿ ತರಬೇತಿ ಪಡೆದ ತಜ್ಞರು. ‘ನಮ್ಮ ಸುಪ್ತ ಮನಸ್ಸು ಯಾವುದನ್ನೂ ಮರೆಯುವುದಿಲ್ಲ’ ಎನ್ನುವ ಅವರು ಎಷ್ಟೋ ಮಂದಿ ಮನೋರೋಗಿಗಳನ್ನು ಜನ್ಮಾಂತರ ಸ್ಮರಣೆಯ ಪ್ರಯೋಗದಿಂದ ತಾವೇ ಗುಣಪಡಿಸಿದ ದಾಖಲೆಗಳನ್ನು ಪ್ರಕಟಿಸಿದ್ದಾರೆ. ಜನ್ಮಾಂತರ ಸ್ಮರಣೆಯ ಮೂಲಕ ನಡೆಯಬಹುದಾದ ಚಿಕಿತ್ಸಾ ಕೇಂದ್ರಗಳು ಇನ್ನು ಹತ್ತು ವರ್ಷಗಳಲ್ಲಿ ಜಗತ್ತಿನ ಬೇರೆ ಬೇರೆ ಭಾಗಗಳಲ್ಲಿ ಕಾಣಿಸಿಕೊಳ್ಳಬಹುದು ಎನ್ನುತ್ತಾರವರು. ಆಧುನಿಕ ವೈಜ್ಞಾನಿಕ ಪದ್ಧತಿಯ ಮೂಲಕ ಈ ಪುರಾತನ ತತ್ತ್ವವು ಒಂದು ಯಥಾರ್ಥ ವಿಶ್ವನಿಯಮವೆಂದು ನಿಶ್ಚಿತವಾದರೆ ಇಂದಿನ ಜಗತ್ತಿನಲ್ಲಿ ತಾಂಡವವಾಡುವ ಹಿಂಸೆ ದ್ವೇಷ ಸ್ವಚ್ಛಂದ ಪ್ರವೃತ್ತಿಗಳು ಕೊನೆಗೊಂಡು ಸಮಸ್ತ ಮಾನವ ಜನಾಂಗ ಸತ್ಯಪಥದಲ್ಲಿ ಮುನ್ನಡೆಯಲು ಬೇಕಾದ ಪ್ರೇರಣೆ ಪಡೆಯಲು ಸಾಧ್ಯವಾಗಬಹುದು.

\chapterend

\addtocontents{toc}{\protect\par\egroup}

