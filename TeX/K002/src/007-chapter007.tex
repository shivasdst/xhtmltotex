
\chapter{ಪ್ರಾರ್ಥನೆಯಿಂದ ಪರಿವರ್ತನೆ}

\begin{itemize}
\item ದೇವರಲ್ಲಿ ಅಚಲ ಶ್ರದ್ಧೆ ನಿಜವಾಗಿಯೂ ಅದ್ಭುತಗಳನ್ನೇ ಮಾಡಬಲ್ಲದು. ಅಂಥ ಶ್ರದ್ಧಾ ವಂತನೆ ಸರ್ವಶಕ್ತನು. ಆ ಶ್ರದ್ಧೆ ಇಲ್ಲದಾತನೇ ನಿಶ್ಶಕ್ತನು. ಖಂಡಿತವಾಗಿಯೂ ಶ್ರದ್ಧೆಯೇ ಜೀವನ; ಅಶ್ರದ್ಧೆಯೇ ಮರಣ.\\
\begin{flushright}
–ಶ‍್ರೀರಾಮಕೃಷ್ಣ ಪರಮಹಂಸ
\end{flushright}

 \item ಒಮ್ಮೆ ನಮ್ಮೊಳಗಿರುವ ಆತ್ಮದ ಮಹಿಮೆಯ ಅರಿವಾಯಿತು ಎಂದರೆ ಅತ್ಯಂತ ದುರ್ಬಲನಿಗೆ ಮೊದಲು ಧೈರ್ಯ ಸ್ಥೈರ್ಯ ಮೂಡಿಬರುತ್ತದೆ; ಅತ್ಯಂತ ಅಧಮನಿಗೆ ಮೊದಲು ಕೆಚ್ಚು ಪುಟಿದೇಳುತ್ತದೆ; ಅತ್ಯಂತ ದುಷ್ಟನಿಗೆ ಮೊದಲು ಉಜ್ವಲ ಭವಿಷ್ಯದ ರೂಪರೇಖೆ ಕಂಡು ಬರುತ್ತದೆ.\\
\begin{flushright}
–ಸ್ವಾಮಿ ವಿವೇಕಾನಂದ
\end{flushright}

 \item ನಮ್ಮ ಹೃದಯಾಂತರಾಳದ ದೈವತ್ವವನ್ನು ಜಾಗ್ರತಗೊಳಿಸುವುದೇ ಪ್ರಾರ್ಥನೆಯ ಉದ್ದೇಶ. ಪ್ರಾರ್ಥನೆಯ ಅದ್ಭುತ ಪ್ರಭಾವವನ್ನು ಉಂಡವನು ಅನ್ನವಿಲ್ಲದೆ ಅನೇಕ ದಿನ ಬದುಕ ಬಹುದು. ಆದರೆ ಪ್ರಾರ್ಥನೆ ಇಲ್ಲದೆ ಅರೆಗಳಿಗೆಯೂ ಬದುಕಿರಲಾರ. ಪ್ರಾರ್ಥನೆಯೇ ಆತನ ಬಾಳಿನ ಉಸಿರು.\\
\begin{flushright}
–ಮಹಾತ್ಮಾಗಾಂಧೀಜಿ
\end{flushright}

 \item ಕೇಳು, ನಿನಗದು ಖಂಡಿತ ಕೊಡಲ್ಪಡುತ್ತದೆ; ಹುಡುಕು, ನಿನಗದು ದೊರೆತೇ ದೊರೆಯುತ್ತದೆ; (ಬಾಗಿಲು) ತಟ್ಟು, ನಿನಗದು ನಿಜವಾಗಿಯೂ ತೆರೆಯಲ್ಪಡುತ್ತದೆ.\\
\begin{flushright}
–ಬೈಬಲ್​
\end{flushright}

 \item ಯಾರು ನನಗಾಗಿ ವ್ಯಾಕುಲತೆಯಿಂದ ಜರ್ಜರಿತರಾಗಿದ್ದಾರೋ ಅವರ ಹೃದಯದಲ್ಲೇ ನಾನಿರುವೆ.\\
\begin{flushright}
–ಕುರಾನ್​
\end{flushright}

 \item \enginline{Man’s need for prayer is as great as his need for bread. As food is necessary for the body, prayer is necessary for the soul. I have not a shadow of doubt that the strife and quarrels with which our atmosphere is so full today are due to the absence of the spirit of true prayer. True prayer never goes unanswered. When the mind is full of prayerful thoughts, everything in the world seems good and agreeable. Prayer is essential for progress of life.}\\
\begin{flushright}
\general{\enginline{–M. K. Gandhi}}
\end{flushright}

\end{itemize}


\section{ದೇವರಲ್ಲಿ ಮೊರೆ}

ಪ್ರಾರ್ಥನೆಯ ನಿಜವಾದ ಅಂತರಾರ್ಥವೇ ಇದು. ಸರ್ವಶಕ್ತನಾದ ಭಗವಂತನಲ್ಲಿ ಅಚಲ ನಂಬಿಕೆಯನ್ನಿಟ್ಟು ನಮ್ಮ ಅಂತರಾಳವನ್ನು ಬಯಲು ಮಾಡುವ ಮಾಧ್ಯಮವೇ ಪ್ರಾರ್ಥನೆ. ಸ್ನೇಹಿತನಿಗೆ ಫೋನು ಮಾಡಿ ಅರ್ಜೆಂಟ್ ವಿಚಾರ ತಿಳಿಸುವಂತೆ, ದೇವರಲ್ಲಿ ಪ್ರಾರ್ಥನೆ ಮಾಡಿ ನಮ್ಮ ಅಂತರಂಗವನ್ನು ಬಯಲು ಮಾಡುತ್ತೇವೆ. ಅದು ದೇವರೊಡನೆ ನಡೆಸುವ ಪವಿತ್ರ ಸಂಭಾಷಣೆ.

ಈ ಜಗತ್ತನ್ನು ನಡೆಯಿಸುವ ಪರಮಾದ್ಭುತ ಶಕ್ತಿಯೊಂದಿದೆ, ಇದ್ದೇ ಇದೆ. ಇಂದಿನ ವಿಜ್ಞಾನದ ವ್ಯಾಪ್ತಿಯಲ್ಲಿ ಅದನ್ನರಿಯಲೂ, ಅಳೆಯಲೂ ಯಾವ ವಿಧಿ ವಿಧಾನಗಳಾಗಲಿ, ಸಾಧನ ಸಲಕರಣೆ ಗಳಾಗಲಿ ಇಲ್ಲವಷ್ಟೆ. ಅದು ಅಚ್ಯುತವಾದುದು, ಅನಂತವಾದುದು. ಸೂರ್ಯ ಚಂದ್ರ ನಕ್ಷತ್ರ ನೀಹಾರಿಕೆಗಳ ನಡುವೆ ಸಾಮರಸ್ಯವನ್ನೇರ್ಪಡಿಸಿ, ಪ್ರಕೃತಿಯಲ್ಲಿ ಕ್ರಮಬದ್ಧವಾಗಿ ಪುತುಭೇದ ಗಳನ್ನುಂಟುಮಾಡಿ, ವಾತಾವರಣದ ಗಾಳಿಯಲ್ಲಿ ಪ್ರಾಣವಾಯು ಆಮ್ಲಜನಕವನ್ನು ಮಿಳಿತ ಗೊಳಿಸಿ, ಬದುಕಿಗೆ ಅತ್ಯಾವಶ್ಯಕವಾದ ನೀರು ಗಾಳಿ ಬೆಳಕುಗಳ ಹೊನಲನ್ನೆ ಉದಾರವಾಗಿ ಕರುಣಿಸಿ ಕಾಪಾಡುವ ಈ ಎಲ್ಲ ಅದ್ಭುತ ವ್ಯವಸ್ಥೆಗಳ ಸೂತ್ರಧಾರನನ್ನೇ ದೇವರು ಎನ್ನುವುದು. ಪ್ರತಿ ಯೊಂದು ಜೀವದಲ್ಲಿ ಜೈವಿಕ ಚಟುವಟಿಕೆಗಳನ್ನೆಲ್ಲ ನಡೆಯಿಸುವುದು ಆ ಶಕ್ತಿಯೇ. ಈ ಶಕ್ತಿ ಯನ್ನು ನಿರಾಕಾರಿಯೆಂದೂ, ಸಾಕಾರಿಯೆಂದೂ ಪೂಜಿಸಿ ಗೌರವಿಸಬಹುದು; ಮಣಿದು ಮೊರೆಯ ಬಹುದು. ಆದರೆ ಒಂದು ಮಾತು. ನಮ್ಮ ದೈವಭಕ್ತಿಯನ್ನವಲಂಬಿಸಿಯೇ ನಮ್ಮ ಜೈವಿಕ ಕ್ರಿಯೆ ಗಳು ನಡೆಯುವುದಲ್ಲ. ಆತ ದಯಾಮಯ, ಕರುಣಾಕರ. ತೀರಾ ನಾಸ್ತಿಕನ ಹೃದಯವೂ ಆಸ್ತಿಕನ ಹೃದಯದಂತೆಯೇ ರಕ್ತವನ್ನು ಪಂಪಿಸುವಂತೆ ಮಾಡಿದ್ದಾನೆ. ಕಟುಕನ ಜೈವಿಕ ಕ್ರಿಯೆಗಳೂ ಕರುಣಾಳುವಿನಲ್ಲಿರುವಂತೆಯೇ ಇರುವಂತೆ ನೋಡುತ್ತಾನೆ. ಆದರೆ ಮನಃಸ್ವಾಸ್ಥ್ಯವಾಗಲಿ, ನೆಮ್ಮದಿಯಾಗಲಿ, ಆತನ ಕೃಪೆಗೇ ನೇರವಾಗಿ ಅನುಗುಣವಾಗಿರುತ್ತವೆ.

ದೇವರು ಶಾಂತಿ ಸಮಾಧಾನದ ಬೀಡು, ಆನಂದದ ತವರು, ಸಂತೋಷದ ಸಾಗರ. ದೇವರನ್ನು ನಂಬಿ ಆತನಲ್ಲಿ ತನ್ನನ್ನು ಸಮರ್ಪಿಸಿಕೊಂಡು ಶರಣಾಗುವ ವ್ಯಕ್ತಿಗೆ ಆನಂದದ ನಿತ್ಯೋತ್ಸವ ತಪ್ಪಿದ್ದಲ್ಲ. ಅಂಥ ವ್ಯಕ್ತಿ ತಾನು ನಂಬಿದ ತನ್ನ ಇಷ್ಟದೇವರಲ್ಲಿ ದಿನದ ಅನುಕ್ಷಣವೂ ದೈನ್ಯದಿಂದ ಬೇಡಿ ಬಾಡುತ್ತಿರುತ್ತಾನೆ. ಅಂಥವನ ಮೊರೆಯೆ ಎಂಥ ಪರಿವರ್ತನೆಯನ್ನೂ ಮಾಡಬಲ್ಲದು. ಆ ದೇವರು ಸರ್ವಶಕ್ತನಲ್ಲವೆ? ಆತನ ಕೃಪೆಗೆ ಇತಿಮಿತಿ ಉಂಟೆ? ಆತನೊಲಿದರೆ ‘ಕೊರಡೂ ಕೊನರುವುದು’; ‘ಬರಡೂ ಹಯನಹುದು’!


\section{ದೈವಿಕ ಆವಾಹನೆ}

ಸರಳಾಂತಃಕರಣದ ಪ್ರಾರ್ಥನೆಯಿಂದ ದೈವೀಕೃಪೆಯನ್ನು ಪಡೆದು ಮನಸ್ಸಿನ ರಜಸ್ತಮೋ ಗುಣಗಳನ್ನು ದಾಟಬಹುದೆಂದು ಸಂತರೆಲ್ಲರೂ ಹೇಳುತ್ತಾರೆ. ದೈವೀಕೃಪೆ ಕಲ್ಪನೆಯಲ್ಲವೆಂದೂ, ನಿಷ್ಠಾವಂತ ಸಾಧಕನಿಗದು ಕ್ಷಣಕ್ಷಣವೂ ವೇದ್ಯವಾಗುತ್ತದೆಂದೂ ಅವರೆಲ್ಲರ ಅನುಭವದ ನುಡಿ. ಬಲಶಾಲಿಗಳಿರಲಿ, ಬುದ್ಧಿವಂತರಿರಲಿ, ಅಷ್ಟೇಕೆ, ಕೆಲಸಕ್ಕೆ ಬಾರದವರೆಂದು ತಿಳಿಯುವ ಯಾರೇ ಇರಲಿ, ದೈವೀಕೃಪೆಯಿಂದ ಉನ್ನತ ಸ್ಥಿತಿಗೆ ಏರಿದ ಉದಾಹರಣೆಗಳು ಆಧ್ಯಾತ್ಮಿಕ ಜಗತ್ತಿನಲ್ಲಿ ಧಾರಾಳವಾಗಿ ದೊರೆಯುತ್ತವೆ. ಜಡಸೃಷ್ಟಿಯ ಘಟನೆಗಳನ್ನು ಗಣಿತಶಾಸ್ತ್ರದ ನಿಯಮಗಳಿಂದ ನಿಶ್ಚಿತವಾಗಿ ವಿವರಿಸಬಹುದು. ಕಾರ್ಯ-ಕಾರಣದ ಆಧಾರದ ಮೇಲೆ ಅವು ನಿಂತಿರುವುದರಿಂದ ಅವುಗಳು ಹೀಗೆಯೇ ನಡೆಯುತ್ತವೆ ಎಂದು ಹೇಳಬಹುದು. ಆದರೆ ದೈವೀಕೃಪೆಯ ಆಳನಿರಾಳ ವನ್ನು ಅಳೆಯಲಾರೆವು. ಸಾರ್ವತ್ರಿಕವಾಗಿ ಸಮಾನವಾಗಿ ಅದು ಉಂಟಾಗಬೇಕೆಂದಿಲ್ಲ. ‘ದೈವೀ ಕೃಪೆಯ ಬಗ್ಗೆ ಸಂಶಯವೆ ಇಲ್ಲ. ಸಾಧಕನು ನಿರ್ವಂಚನೆಯಿಂದ ಕಳಕಳಿಯಿಂದ ಸಾಧನೆ ಮಾಡಿದರೆ ಅವನು ಭಗವಂತನನ್ನು ಹೊಂದುವನೆಂಬುದು ಪೂರ್ಣಸತ್ಯ. ಆದರೆ ಆತ ಕೂಡಲೇ ಅದನ್ನು ಪಡೆಯುವನೆಂದು ಇದರ ಅರ್ಥವಲ್ಲ’ ಎಂದು ಶ‍್ರೀ ಅರವಿಂದರು ಹೇಳಿದರು. ಅಂದರೆ ತಾಳ್ಮೆಯಿಂದ ಸಾಧನೆಯ ಏಣಿ ಏರಿ ಹೋದರೆ ಮಾತ್ರ ಕೃಪೆಯ ಮಾಳಿಗೆ ಸಿಗಬಲ್ಲದೆಂದರ್ಥ.

ಶ್ರದ್ಧೆ, ಭಗವಂತನಲ್ಲಿ ನಿರ್ಭರತೆ, ಸಮರ್ಪಣೆ, ಅಹಂಕಾರತ್ಯಾಗ–ಇವುಗಳೆಲ್ಲ ಭಗವಂತನ ವಿಶೇಷ ಕೃಪಾಲಾಭಕ್ಕೆ ಆವಶ್ಯಕ. ಎಷ್ಟೋ ವೇಳೆ ‘ದೇವರಲ್ಲಿ ಸಂಪೂರ್ಣ ಶರಣಾಗತನಾಗಿದ್ದಾನೆ’ ಎನ್ನುವ ಮಾತು ಕಾರ್ಯತಃ ಆಲಸ್ಯ, ಜಡತೆ, ದುರ್ಬಲತೆ ಮತ್ತು ಪಾಶವಿಕ ಪ್ರವೃತ್ತಿಗಳ ಸೆಳೆತ–ಇವುಗಳಲ್ಲಿ ಪರ್ಯವಸಾನವಾಗುವ ಸಂಭವವಿದೆ. ಕೃಪೆಯನ್ನು ಪಡೆಯುವುದಕ್ಕೆ ವಿಘ್ನ ಗಳಾದ ಅವುಗಳನ್ನು ನಿರಂತರ ಹೋರಾಟ, ತೀವ್ರ ಅಭೀಪ್ಸೆಗಳಿಂದ ಮೀರಿ ಹೋಗದಿದ್ದರೆ ಕೃಪಾ ಲಾಭ ಅಸಾಧ್ಯ. ಶ‍್ರೀರಾಮಕೃಷ್ಣರೆಂದರು: ‘ಭಗವಂತನ ಕೃಪೆಯ ಗಾಳಿ ಬೀಸುತ್ತಲೇ ಇರುವುದು. ನಾವು ನಮ್ಮ ಪ್ರಯತ್ನದ ಹಡಗಿನ ಹಾಯಿಯನ್ನೂ ಬಿಡಿಸಿದರೆ ಅದರ ಉಪಯೋಗವನ್ನು ಪಡೆಯಬಹುದು. ಕೆಲಸಗಳ್ಳರೂ, ಆಲಸಿಗಳೂ ಅದರ ಪ್ರಯೋಜನವನ್ನು ಹೊಂದಲಾರರು. ಆದರೆ ದಕ್ಷ ಜನರು, ಉದ್ಯಮಶೀಲರು ದೈವೀಕೃಪೆಯ ಲಾಭ ಪಡೆಯುತ್ತಾರೆ. ನಮ್ಮ ಮನಸ್ಸು ಐಹಿಕ ವಾಸನೆಗಳಿಂದ ಮಲಿನವಾಗಿದ್ದರೆ, ನಮ್ಮ ವಿಚಾರಗಳು ದುಷ್ಟವಾಗಿದ್ದರೆ, ನಮಗೆ ಈಶ ಕೃಪೆಯ ಪ್ರಯೋಜನವಾಗಲಾರದು’ ಎಂದು. ‘ಯಾರು ಹೃತ್ಪೂರ್ವಕ ತನ್ನೆಡೆಗೆ ಬರಲು ಹೋರಾಡುತ್ತಿರುವನೊ ಅವನ ಮೇಲೆ ದೇವರು ಅಪಾರ ಕರುಣೆ ಬೀರುವನು. ನೀನು ಸೋಮಾರಿ ಯಾಗಿ ಏನನ್ನೂ ಮಾಡದೆ ಇದ್ದರೆ ಅವನು ಎಂದಿಗೂ ನಿನ್ನೆಡೆಗೆ ಬರುವುದಿಲ್ಲವೆಂದು ನಿನಗೆ ತಿಳಿಯುವುದು’ ಎಂದು ಸ್ವಾಮಿ ವಿವೇಕಾನಂದರು ಹೇಳುತ್ತಿದ್ದರು. ಆದ್ದರಿಂದ ಪ್ರಾರ್ಥನೆಯ ಮೂಲಕ ಆ ದೇವರ ಆವಾಹನೆ, ಕೃಪೆಯ ಗಾಳಿಗೆ ಹಾಯಿಬಿಡಿಸಿದಂತೆ.


\section{ಪ್ರಾರ್ಥನೆಯ ಪರಿ}

ದೇವರೆಡೆಗಿನ ಯಾತ್ರೆಯನ್ನೇ ಸಾಧನೆ ಎನ್ನುವುದು. ಒಂದೇ ಗುರಿಯೆಡೆಗೆ ಸಾಗುವ ವಿವಿಧ ಮಾರ್ಗಗಳಂತೆ ಈ ಸಾಧನೆಯಲ್ಲೂ ವೈವಿಧ್ಯಗಳಿವೆ. ಪ್ರಾರ್ಥನೆ ಅವುಗಳಲ್ಲೊಂದಷ್ಟೆ. ಇದು ಆಂತರ್ಯದ ಪರಿಶುದ್ಧತೆಗೆ ಸುಲಭವಾದ ಮಾರ್ಗ. ಜಪ, ಧ್ಯಾನ, ಭಜನೆ, ಸಾಷ್ಟಾಂಗನಮನ, ಸ್ತೋತ್ರಪಾರಾಯಣ, ವ್ರತ ನಿಯಮ, ಪರಿಪಾಲನೆ, ಪೂಜೆ ಪುರಸ್ಕಾರಗಳೆಲ್ಲವೂ ಈ ಮಾರ್ಗದ ಕವಲುಗಳಷ್ಟೆ.

ಶಾಲೆಯನ್ನು ಸೇರಿ ಅಧ್ಯಾಪಕರ ಸಮರ್ಥ ಮಾರ್ಗದರ್ಶನದಲ್ಲಿ ಏಕಾಗ್ರತೆಯಿಂದ ಪಾಠ ಕಲಿತಲ್ಲಿ ಉಚ್ಚಶ್ರೇಣಿಯಲ್ಲಿ ತೇರ್ಗಡೆಯಾಗುವುದು ಹೇಗೆ ಸುಲಭವೋ ಅದೇ ರೀತಿ, ಸಾಧನೆ ಯಲ್ಲಿಯೂ ಗುರುಹಿರಿಯರ, ವೇದಶಾಸ್ತ್ರವೇ ಮೊದಲಾದ ಸದ್ಗ್ರಂಥಗಳ ಮಾರ್ಗದರ್ಶನದಲ್ಲಿ, ಸೂಕ್ತ ಮಂತ್ರೋಪದೇಶ ಪಡೆದು ಪ್ರಾರ್ಥನೆ ಮಾಡುವುದು ಬಹಳ ಸುಲಭ. ಯಶಸ್ಸಿನ ಶಿಖರ ಏರಲು, ನಿತ್ಯವೂ ಶಾಲೆಗೆ ಹೋಗಿ ನಿಷ್ಠೆಯಿಂದ ಅಭ್ಯಸಿಸುವುದು ಅಗತ್ಯವಾದಂತೆ, ಕ್ರಮ ಬದ್ಧತೆಯನ್ನು ಕಾಯ್ದುಕೊಂಡು ನಿತ್ಯವೂ ಈ ಪ್ರಾರ್ಥನೆ ಮಾಡುವುದು ಅವಶ್ಯ.

ಇಷ್ಟದೇವರ ಚಿತ್ರವನ್ನು ಎದುರಿಗಿಟ್ಟುಕೊಂಡೆ ಪ್ರಾರ್ಥನೆ ಮಾಡುವುದು ರೂಢಿ. ಹೃದಯ ದಲ್ಲೇ ದೇವರ ರೂಪವನ್ನು ಕಾಣಬಲ್ಲವರಿಗೆ ಆ ಚಿತ್ರ ಬೇಕಾಗಿಲ್ಲ ನಿಜ. ಅದೇನೇ ಇರಲಿ ಪೂರ್ಣ ಭಕ್ತಿ, ಶ್ರದ್ಧೆ, ವ್ಯಾಕುಲತೆಗಳಷ್ಟೇ ಪ್ರಾರ್ಥನೆಯಲ್ಲಿ ಪ್ರಧಾನ. ಪ್ರಾರ್ಥನೆಯ ಪಥದಲ್ಲಿ ಮುನ್ನಡೆಯಬಯಸುವವರು ಜಪ ಧ್ಯಾನದ ಮೂಲಕ ಸತತ ಇಷ್ಟನಾಮದ ಸ್ಮರಣೆಗೈಯ್ಯುತ್ತಾ, ಸ್ನಾನ ಪೂಜೆ ಭಜನೆ ಮೊದಲಾದವುಗಳಲ್ಲಿ ನಿಯಮ ಪರಿಪಾಲಿಸಿ, ರಾಮರಕ್ಷಾಸ್ತೋತ್ರ, ಲಲಿತಾ ಸಹಸ್ರನಾಮ, ವಿಷ್ಣುಸಹಸ್ರನಾಮವೇ ಮೊದಲಾದ ಸ್ತೋತ್ರಗಳ ಪಾರಾಯಣವನ್ನು ನಿಷ್ಠೆಯಿಂದ ಮಾಡಿ, ಉಪವಾಸವೇ ಮೊದಲಾದ ವ್ರತ ಕೈಗೊಂಡು, ಸಾಷ್ಟಾಂಗ ನಮನಗಳಿಂದ ಆ ದೇವರಿ ಗಾಗಿ ತನುಮನ ಕುಗ್ಗಿಸಿ ಮಣಿದು ಮೊರೆದಲ್ಲಿ, ಆಂತರ್ಯ ಪರಿಶುದ್ಧವಾಗುತ್ತಾ ಬಂದು ಕ್ರಮೇಣ ಇಷ್ಟದರುಶನ ಲಾಭ ದೊರೆಯುತ್ತದೆ. ಆದರೆ ಒಂದೆರಡು ದಿನಗಳಲ್ಲಾಗುವ ಕೆಲಸ ಇದಲ್ಲ. ಬಿಡದ ಛಲ ನಮ್ಮಲ್ಲಿದ್ದರೆ ಒಂದಲ್ಲ ಒಂದು ದಿನ ಈ ದಾರಿ ಆ ಗುರಿಗೊಯ್ಯುವು ದಂತೂ ಖಂಡಿತ.


\section{ಕತ್ತಲು ಬತ್ತಲು}

ಸೈಕಲ್ಲಿನ ಕ್ಯಾರಿಯರಿನಲ್ಲಿ ೧೦೦ ಕಿಲೋಗ್ರಾಂ ತೂಕದ ಕಬ್ಬಿಣದ ಗಟ್ಟಿಯಿದೆ ಎಂದೆಣಿಸಿ ಕೊಳ್ಳಿ. ನೀವು ಆ ಸೈಕಲನ್ನೇರಿ ಹೊರಟರೆ ಏನಾದೀತು ಹೇಳಿ. ನೀವು ಆ ತೂಕ ತಡೆದುಕೊಂಡಲ್ಲಿ, ಮುನ್ನಡೆಯುವುದಿರಲಿ, ಬ್ಯಾಲೆನ್ಸ್ ಆದರೂ ಸಿಕ್ಕೀತು. ಇಲ್ಲವಾದಲ್ಲಿ ಹಿಂಭಾರವಾಗಿ ಸೈಕಲ್ ಮೇಲೇಳಬಹುದಲ್ಲವೆ? ಆಗ ಮುನ್ನಡೆ ಎಂದಾದರೂ ಸಾಧ್ಯವೆ? ಹಾಗೆಯೇ ಆಗಿದೆ ನಮ್ಮಲ್ಲಿ ಬಹುಮಂದಿಯ ಸ್ಥಿತಿ.

ಪ್ರಾರ್ಥನೆ ಮಾಡುತ್ತಾ ಹೋದರೂ ಫಲ ಸಿಗದಾಯಿತೆಂಬವರಿಗೆ ಇಲ್ಲಿದೆ ಉತ್ತರ. ಮನುಷ್ಯನ ಕುಕರ್ಮಗಳ ಹೊರೆ ಹೆಚ್ಚಿದ್ದಲ್ಲಿ ಈ ಸೈಕಲ್ಲಿನ ಅವಸ್ಥೆಯೇ ಅವನದಾಗುತ್ತದೆ. ಪ್ರಯತ್ನ ಮುಂದುವರಿದರೂ ಮುನ್ನಡೆ ಕಾಣದಾಗುತ್ತದೆ. ಹಾಗಾದರೆ ಇದರಿಂದ ಪಾರಾಗುವುದು ಹೇಗೆ? ನೋಡಿ, ಸೈಕಲ್ಲಿನಲ್ಲಿರುವ ಕಬ್ಬಿಣದ ಗಟ್ಟಿಯಂತಲ್ಲ ಈ ಸಂಚಿತ ಕರ್ಮದ ಹೊರೆ. ಬೆಳಕು ಬಿದ್ದಾಗ ಕರಗುವ ಮಂಜಿನಂತೆ, ನಿಷ್ಠೆಯಿಂದ ಪ್ರಾರ್ಥನೆ ಮಾಡುತ್ತಾ ಹೋದಂತೆ ಈ ಹೊರೆ ಕರಗುತ್ತ ಬರುತ್ತದೆ. ಹೊರೆಯ ಭಾರ ಕಡಿಮೆಯಾದಾಗ ಸೈಕಲ್ ಮುನ್ನಡೆಯಬಹುದಾದಂತೆ, ನಿಮ್ಮಲ್ಲಿನ ಕುಕರ್ಮದ ಕತ್ತಲು ಬತ್ತುತ್ತ ಬಂದಂತೆ ಸಾಧನೆಯಲ್ಲಿನ ಮುನ್ನಡೆ ಸ್ವಯಂವೇದ್ಯ ವಾಗುತ್ತದೆ. ಅಷ್ಟರವರೆಗೂ ದೇವರೊಡನೆ ಹೃದಯ ಮನಸ್ಸುಗಳನ್ನು ಬಿಚ್ಚಿ ಬೇಡಿ, ದೇವರೊಡನೆ ಹೀಗೆ ಬತ್ತಲಾದರೆ ಬೇಗ ಕರಗುವುದು ಆಂತರ್ಯದ ಕತ್ತಲು.


\section{ವ್ಯಾಕುಲತೆಯೇ ಜೀವಾಳ}

ಅಹಿರ್ಬುಧ್ನ್ಯಸಂಹಿತೆ(೩೭..೨೮)ಯಲ್ಲಿ ಪ್ರಾರ್ಥನಾಶೀಲ ವ್ಯಕ್ತಿ ರೂಢಿಸಿಕೊಳ್ಳಬೇಕಾದ ನಿಯಮಗಳ ಉಲ್ಲೇಖವಿದೆ:

ಆನುಕೂಲ್ಯಸ್ಯ ಸಂಕಲ್ಪಃ ಪ್ರತಿಕೂಲಸ್ಯ ವರ್ಜನಂ~। ರಕ್ಷಿಷ್ಯತೀತಿ ವಿಶ್ವಾಸೋ ಗೋಪ್ತೃತ್ವವರಣಂ ತಥಾ~। ಆತ್ಮನಿಕ್ಷೇಪಕಾರ್ಪಣ್ಯೇ ಷಡ್ವಿಧಾ ಶರಣಾಗತಿಃ~॥

ಎಂದರೆ, ಭಗವದ್ ಭಜನೆಗೆ ಅನುಕೂಲವಾದ ವಿಷಯಗಳನ್ನು ವ್ರತದಂತೆ ಪರಿಗ್ರಹಿಸುವುದು, ಭಕ್ತಿಗೆ ಪ್ರತಿಕೂಲವಾದ ವಿಷಯಗಳನ್ನು ತ್ಯಜಿಸುವುದು, ಭಗವಂತ ಖಂಡಿತ ನನ್ನನ್ನು ರಕ್ಷಿಸುವ ನೆಂಬ ದೃಢವಾದ ವಿಶ್ವಾಸ, ಅವನೇ ನಿಜವಾದ ರಕ್ಷಕನೆಂಬ ಪೂರ್ಣನಂಬಿಕೆ; ಆತನ ಚರಣ ಗಳಲ್ಲಿ, ಆತ್ಮಸಮರ್ಪಣ ಮತ್ತು ತನ್ನ ಸಂಕಟ ಸಂಕಷ್ಟಗಳನ್ನು ಆತನಲ್ಲಿ ತೋಡಿಕೊಳ್ಳುವುದು– ಇವು ಶರಣಾಗತಿಯ ಲಕ್ಷಣಗಳು. ಪ್ರಾರ್ಥನೆ ಫಲಕಾರಿಯಾಗಲು ಇವೆಲ್ಲ ಮೊಳೆತು ಬೆಳೆಯ ಬೇಕಾದುದು ತೀರ ಆವಶ್ಯಕ.

ಹೌದು, ದೇವರಲ್ಲಿ ನೀವಿಡುವ ಅಚಲವಾದ ನಂಬಿಕೆಯೆ ಭಕ್ತಿಯ ಮೂಲ. ನೈಜಭಕ್ತಿಯಿಂದ ವಿಧೇಯತೆ ಮೈದೋರಿ ಇಷ್ಟದೇವರಲ್ಲಿ ಸಮರ್ಪಣಭಾವ ನಿಮ್ಮಲ್ಲಿ ಚಿಗುರೊಡೆಯುತ್ತದೆ. ಅದೇ ಶ್ರದ್ಧೆ, ಶ್ರದ್ಧೆಯಿಂದ ಶರಣಾಗತಿಭಾವ ತೀವ್ರವಾದಾಗಲೆ ವ್ಯಾಕುಲತೆಯ ಉಗಮ. ಇದೇ ಪ್ರಾರ್ಥನೆಯ ಜೀವಾಳ.

ಪರಿಶುದ್ಧತೆಗಾಗಿ ಪ್ರಾರ್ಥನೆಯ ಕ್ರಮ ಇಂದು ನಿನ್ನೆಯದಲ್ಲ, ವೈದಿಕಯುಗದಿಂದಲೂ ಪ್ರಚಲಿತವಾದದ್ದೆ. ‘ಸರ್ವೇ ಜನಾಃ ಸುಖಿನೋ ಭವಂತು’, ಬೃಹದಾರಣ್ಯಕೋಪನಿಷತ್ತಿನ ಅಭ್ಯಾರೋಹ ಮಂತ್ರ ‘ಅಸತೋ ಮಾ ಸದ್ಗಮಯ, ತಮಸೋ ಮಾ ಜ್ಯೋತಿರ್ಗಮಯ, ಮೃತ್ಯೋರ್ಮಾ ಅಮೃತಂ ಗಮಯ’ ಅಷ್ಟೇಕೆ ಧೀಶಕ್ತಿ ಬೆಳಗುವ ಗಾಯತ್ರಿ ಮಂತ್ರ ಇವೆಲ್ಲ ಅದಕ್ಕೆ ನಿದರ್ಶನಗಳು. ಆದರೆ ಬರಬರುತ್ತ ವಿಚಾರವಾದದ ಅಲೆ ಎದ್ದು ಪ್ರಾರ್ಥನಾಧರ್ಮಕ್ಕೆ ಗ್ರಹಣ ಹಿಡಿದಂತಾಯಿತು. ಪ್ರಾರ್ಥನೆ ಬೇರೆಲ್ಲ ರಾಜಮಾರ್ಗಗಳನ್ನು ಬಿಟ್ಟು ಇಷ್ಟಾರ್ಥ ಸಿದ್ಧಿ ಗಾಗಿ ಜನ ಹರಕೆ ರೂಪದ ಪ್ರಾರ್ಥನೆಯನ್ನೆ ನೆಚ್ಚುವ ಕಾಲ ಬಂತು. ದುರದೃಷ್ಟವೆಂದರೆ, ಆಚಾರ್ಯಪುರುಷರೂ ಆ ನಿಟ್ಟಿನಲ್ಲಿ ನೆರವಾಗಲಿಲ್ಲ. ಕ್ರೈಸ್ತ, ಇಸ್ಲಾಂ ಧರ್ಮಗಳಲ್ಲಿ ಪ್ರಾರ್ಥನೆಗೆ ವಿಶೇಷ ಮಹತ್ತ್ವವಿದೆ. ಹಿಂದೂಧರ್ಮದಲ್ಲಿ ಪ್ರಾರ್ಥನೆಯ ಹಿರಿಮೆಗರಿಮೆಗಳ ಪುನಃಪ್ರಕಾಶಕ್ಕೆ ಶ‍್ರೀರಾಮಕೃಷ್ಣರ ಅವತಾರವಾಯಿತು. ಅವರು ಪ್ರಾರ್ಥನಾ ಧರ್ಮದ ಪುನರುಜ್ಜೀವನಗೈದರು.

ಶ‍್ರೀರಾಮಕೃಷ್ಣರು ಸಾಕ್ಷಾತ್ಕಾರಕ್ಕಾಗಿ ತಮ್ಮ ಹೃದಯದಲ್ಲಿ ಉಂಟಾದ ತೀವ್ರ ವ್ಯಾಕುಲತೆ ಯನ್ನು ತಮ್ಮ ಪ್ರಿಯ ಶಿಷ್ಯರಲ್ಲಿ ಆಗಾಗ ಪ್ರಸ್ತಾಪಿಸುತ್ತಿದ್ದರು: ‘ಮಾತೆಯ ದರ್ಶನವಾಗಲಿಲ್ಲ ವೆಂದು ನನ್ನ ಹೃದಯದಲ್ಲಿ ಸಹಿಸಲಸಾಧ್ಯವಾದ ವ್ಯಥೆ ತುಂಬಿತ್ತು. ಒದ್ದೆ ಬಟ್ಟೆಯನ್ನು ಹಿಂಡಿ ದಂತೆ ನನ್ನ ಹೃದಯ ಮನಸ್ಸುಗಳನ್ನು ಯಾರೋ ಹಿಂಡಿದಂತಾಗುತ್ತಿತ್ತು. ಸಂಜೆಯಾಯಿತು. ಕಾಳೀ ಮಂದಿರದ ಗಂಟೆ ಬಾರಿಸತೊಡಗಿತೆಂದರೆ ಆ ತಾಯಿಯೊಡನೆ “ಅಮ್ಮಾ, ಇನ್ನೂ ಒಂದು ದಿನ ಕಳೆಯಿತು, ಎಲ್ಲಿರುವೆ ನೀನು? ಯಾಕೆ ಈ ಕಂದನ ಬಳಿ ಸಾರುವುದಿಲ್ಲ? ಬೇಗ ಬಾರಮ್ಮಾ, ದಯೆ ತೋರಮ್ಮಾ.... ” ಎಂದೆಲ್ಲ ಮೊರೆಯಿಡುತ್ತಿದ್ದೆ. ವ್ಯಥೆ ತಡೆಯಲಾರದೆ ನೆಲದ ಮೇಲೆ ಹೊರಳಾಡಿ ಗೋಗರೆಯುತ್ತಿದ್ದೆ. ನನಗೆ ಕ್ಷಣಕ್ಷಣವೂ ಅಮೂಲ್ಯವಾಗಿತ್ತು. ಅಯ್ಯೋ, ತಾಯಿಯ ದರ್ಶನವಾಗದೆ ಹೋದರೆ, ಜೀವನ ವ್ಯರ್ಥವಾಯಿತಲ್ಲಾ.... ಎಂದು ಪರಿತಪಿಸುತ್ತಿದ್ದೆ....’ ಅದೆಂಥ ವ್ಯಾಕುಲತೆ! ಅವರೇ ಹೇಳುವಂತೆ ‘ತೀವ್ರ ವ್ಯಾಕುಲತೆ ಉಂಟಾಯಿತು ಎಂದರೆ ಅರುಣೋದಯವಾದ ಹಾಗೆ; ಸೂರ್ಯೋದಯಕ್ಕೆ ಇನ್ನು ತಡವಿಲ್ಲ.’


\section{ದೀನರಾಗಿ ಬೇಡಿ}

ಕಲ್ಕತ್ತದ ಬಾಗ್​ಬಜಾರಿನ ಬಲರಾಮ ಬೋಸ್ ಶ‍್ರೀಮಂತ ಕುಟುಂಬಕ್ಕೆ ಸೇರಿದವನು. ದಾನ ಧರ್ಮಕ್ಕೆ ಹೆಸರಾದ ಮನೆತನ ಅವನದು. ತಾರುಣ್ಯದಿಂದಲೇ ಧಾರ್ಮಿಕ ಮನೋವೃತ್ತಿ ಅವನಲ್ಲಿ ಬೆಳೆದುಬಂದಿತ್ತು. ತನ್ನ ಸಮಯವನ್ನು ಶಾಸ್ತ್ರ ಗ್ರಂಥಗಳ ಅಧ್ಯಯನ, ಪ್ರಾರ್ಥನೆ, ಧ್ಯಾನಗಳಲ್ಲಿ ಕಳೆಯುತ್ತಿದ್ದ. ಕೇಶವಚಂದ್ರ ಸೇನರ ಪತ್ರಿಕೆಯಲ್ಲಿ ಪರಮಹಂಸರನ್ನು ಕುರಿತು ಓದಿ ಅವರನ್ನು ನೋಡಲು ಒಮ್ಮೆ ದಕ್ಷಿಣೇಶ್ವರಕ್ಕೆ ಬಂದು, ತುಂಬ ಜನ ಸಂದಣಿಯಿದ್ದುದರಿಂದ ಕೋಣೆಯ ಮೂಲೆಯೊಂದರಲ್ಲಿ ಕುಳಿತ. ಬಂದವರೆಲ್ಲ ಹೊರಟು ಹೋದ ಮೇಲೆ ಶ‍್ರೀರಾಮಕೃಷ್ಣರು ಅವನನ್ನು ಕರೆದು ‘ಏನಾದರೂ ಪ್ರಶ್ನೆಗಳಿವೆಯೆ?’ ಎಂದರು. ‘ಮಹಾಶಯರೇ. ನಿಜವಾಗಿಯೂ ದೇವರಿದ್ದಾನೆಯೆ?’ ‘ಹೌದು, ನಿಜವಾಗಿಯೂ’ ಪರಮಹಂಸರೆಂದರು. ‘ಯಾರಾದರೂ ಅವನನ್ನು ಸಾಕ್ಷಾತ್ಕರಿಸಬಹುದೆ?’ ಬಲರಾಮ ಕೇಳಿದ. ‘ಯಾರು ಅವನನ್ನು ಅತಿ ಸಮೀಪನು, ಪ್ರಿಯನು ಎಂದು ತಿಳಿಯುತ್ತಾನೋ, ಅಂಥ ಭಕ್ತನಿಗೆ ಅವನು ನಿಜಕ್ಕೂ ಮೈದೋರುತ್ತಾನೆ. ಎಂದೋ ಒಮ್ಮೊಮ್ಮೆ ಅವನನ್ನು ಪ್ರಾರ್ಥಿಸಿ ಅವನಿಂದ ಮರುದನಿ ಬರಲಿಲ್ಲವೆಂದು ತಿಳಿದು ಅವನು ಇಲ್ಲ ವೆಂದು ತೀರ್ಮಾನಿಸಬಾರದು.’ ಆಗ ಅವನೆಂದ: ‘ಮಹಾಶಯರೆ, ನಾನು ಅಷ್ಟೊಂದು ಪ್ರಾರ್ಥಿಸಿ ದರೂ ನನಗೇತಕ್ಕೆ ಕಾಣಿಸನು?’ ಶ‍್ರೀರಾಮಕೃಷ್ಣರು ನಗುತ್ತ ‘ನಿಜಕ್ಕೂ ನೀನು ಆತನನ್ನು ನಿನ್ನ ಸ್ವಂತ ಮಕ್ಕಳಿಗಿಂತ ಹೆಚ್ಚಾಗಿ ಪ್ರೀತಿಸುತ್ತಿದ್ದೀಯಾ?’ ಎಂದು ಕೇಳಿದರು. ‘ಇಲ್ಲ’ ಎಂದ ಬಲ ರಾಮ. ಒಂದು ಕ್ಷಣ ತಡೆದು, ‘ಅಷ್ಟೊಂದು ತೀವ್ರ ಹಂಬಲದಿಂದ ಚಿಂತಿಸಿಲ್ಲ’ ಎಂದ.

ಪರಮಹಂಸರು ದೃಢ ಸ್ವರದಲ್ಲಿ ಹೇಳಿದರು: ‘ದೇವರನ್ನು ನಿನ್ನ ಸ್ವಂತ ಜೀವಕ್ಕಿಂತ ಹೆಚ್ಚಾಗಿ ಪ್ರೀತಿಸಿ ಪ್ರಾರ್ಥಿಸು. ಅವನಷ್ಟು ಭಕ್ತರನ್ನು ಪ್ರೀತಿಸುವವರು ಇನ್ನಾರೂ ಇಲ್ಲವೆಂದು ನಾನು ಹೇಳುತ್ತೇನೆ. ಖಂಡಿತವಾಗಿಯೂ ಅಂಥವರಿಗೆ ಅವನು ಕಾಣಿಸಿಕೊಳ್ಳದೆ ಇರಲಾರ. ಅವನನ್ನು ಹುಡುಕುವ ಮೊದಲೇ ಕಾಣಸಿಕೊಳ್ಳುವನು. ದೇವರಿಗಿಂತ ಆಪ್ತನೂ, ಪ್ರಿಯಸ್ವಭಾವಿಯೂ ಇನ್ನಾರೂ ಇಲ್ಲ. ನೀನು ಅವನಿಗಾಗಿ ದೀನನಾಗಿ ಬೇಡಿ ಕಾಡಬೇಕಷ್ಟೆ.’ ಪರಮಹಂಸರ ನುಡಿಗಳು ಬಲರಾಮನ ಅಂತರಂಗದ ಆಳವನ್ನು ಸ್ಪರ್ಶಿಸಿದವು. ಅವನ ಪ್ರಾರ್ಥನೆಯು ತೀವ್ರವಾಯಿತು. ಆಧ್ಯಾತ್ಮಿಕ ಜೀವನದಲ್ಲಿ ಅವನು ದ್ರುತಗತಿಯಿಂದ ಮುಂದುವರಿಯತೊಡಗಿದ.

ಸ್ವಾಮಿ ಬ್ರಹ್ಮಾನಂದರು ಹೇಳುತ್ತಿದ್ದರು: ‘ಅತ್ಯಂತ ವ್ಯಾಕುಲನಾಗಿ ದೇವರನ್ನು ಪ್ರಾರ್ಥಿಸು. ಸರಳವಾಗಿ ಮತ್ತು ನಿಷ್ಕಪಟ ಭಾವದಿಂದ ನಿನಗೆ ಆತನೇ ಬೇಕೆಂದು ಹೇಳು. ಸಂದೇಹಪಡಬೇಡ. ಆತ ನಿಜವಾಗಿಯೂ ಇದ್ದಾನೆ. ಯಾರು ದುರ್ಬಲರೋ, ಆದರೆ ವಿನಮ್ರರೋ, ಅವರು ನಿಜ ವಾಗಿಯೂ ಆತನ ಕೃಪೆಯನ್ನು ಬೇಗನೆ ಪಡೆಯುವರು. ನೀನು ಶ್ರದ್ಧಾಭಕ್ತಿಗಳಿಂದ ಅವನನ್ನು ಸಮೀಪಿಸಿದರೆ ಅವನು ಖಂಡಿತವಾಗಿಯೂ ಕಾಣಿಸಿಕೊಳ್ಳುವನು. ನೀನು ಎಷ್ಟೋ ತಪ್ಪುಗಳನ್ನು ಮಾಡಿದ್ದೀಯೆಂದಾಗಲಿ, ಭಗವಂತನನ್ನು ಬಹಳಕಾಲದಿಂದ ಕರೆದಿಲ್ಲವೆಂದಾಗಲಿ ಚಿಂತಿಸಿ ಸಂಕೋಚಪಡಬೇಡ. ಅವನು ಪರಮ ದಯಾಮಯ. ಅವನು ನಿನ್ನ ತಪ್ಪುಗಳನ್ನೆಲ್ಲ ಎಣಿಸುತ್ತ ಕೂಡುವುದಿಲ್ಲ. ಮಗುವಿನ ಸರಳತೆಯಿಂದೊಡಗೂಡಿ ಅವನನ್ನು ಸಮೀಪಿಸು. ಅವನು ನಿನ್ನನ್ನು ಹೇಗೆ ತನ್ನೆಡೆಗೆ ಬರಮಾಡಿಕೊಳ್ಳುವನೆಂಬುದನ್ನು ನೋಡು. ಮಗುವಿನಂಥ ಶ್ರದ್ಧೆ ಇಲ್ಲದೆ ಆತನನ್ನು ಯಾರೂ ಪಡೆಯಲಾರರು.’


\section{ಕರೆದರೆ ಓ ಎನ್ನನೇ?}

ಶ‍್ರೀರಾಮಕೃಷ್ಣ ಪರಮಹಂಸರ ಪದತಲದಲ್ಲಿ ಕುಳಿತು ಶ್ರದ್ಧೆಯಿಂದ ಅವರ ವಚನಾಮೃತ ವನ್ನು ಆಲಿಸಿ ಅಂತಃಕರಣಪೂರ್ವಕವಾಗಿ ಅವರ ಸೇವೆಯನ್ನು ಮಾಡಿದವರು ಸ್ವಾಮಿ ಅದ್ಭುತಾ ನಂದರು. ಸಾಮಾನ್ಯ ಸೇವಕ ಬಾಲಕನಾಗಿದ್ದ ಅವರು ತಪಸ್ಸು, ತ್ಯಾಗ, ಸಮರ್ಪಣೆ, ಗುರುಭಕ್ತಿ ಯಲ್ಲಿ ಯಾರಿಗೂ ದ್ವಿತೀಯರೆನಿಸದೆ ತಮ್ಮ ಆಧ್ಯಾತ್ಮಿಕ ಜೀವನದಿಂದ ಅನ್ವರ್ಥನಾಮವನ್ನೇ ಪಡೆದಿದ್ದರು. ಶ‍್ರೀರಾಮಕೃಷ್ಣರ ದೇಹತ್ಯಾಗದ ಬಳಿಕ ಅವರು ತಮ್ಮ ಸಮಯವನ್ನು ತಪಸ್ಸಿನಲ್ಲೇ ಕಳೆದರು. ಅವರು ಓದುಬರಹ ಬಲ್ಲವರಲ್ಲ. ಆದರೆ ಶಾಸ್ತ್ರದ ಮರ್ಮಾಥRವನ್ನೂ, ಸಾಧನೆಯ ರಹಸ್ಯವನ್ನೂ ಸ್ಪಷ್ಟವಾದ ಮಾತುಗಳಲ್ಲಿ ಹೇಳಬಲ್ಲವರಾಗಿದ್ದರು. ಸಿದ್ಧಪುರುಷರಾದ ಅವರ ವಾಣಿ ಎಂಥ ಸಂಶಯವಾದಿಯ ಮನಸ್ಸಿನಲ್ಲೂ ಭಗವದ್ಭಕ್ತಿಯನ್ನು ಹುಟ್ಟಿಸುತ್ತಿತ್ತು. ಅದ್ಭುತಾ ನಂದರು ತಮ್ಮ ಕೊನೆಗಾಲವನ್ನು ಪ್ರಸಿದ್ಧ ತೀರ್ಥಸ್ಥಾನವಾದ ಕಾಶಿಯಲ್ಲಿ ಕಳೆಯುತ್ತಿದ್ದರು.

ಒಂದು ದಿನ ಮಹಡಿಯೊಂದರಲ್ಲಿ ಕುಳಿತು ಭಕ್ತರೊಂದಿಗೆ ಭಗವದ್ವಿಚಾರವಾಗಿ ಮಾತನಾಡು ತ್ತಿದ್ದ ಅದ್ಭುತಾನಂದರು ಆಕಸ್ಮಾತ್ ಮಾತು ನಿಲ್ಲಿಸಿ ಆಕಾಶವನ್ನು ನೋಡುತ್ತ “ಓಹ್, ಅಲ್ಲಿಂದಲೇ ನಮಸ್ಕಾರವೆ? ಬಂದು ಹೋಗು ಎಂದು ಹೇಳಿದ್ದೆ, ಆಗಲಿ ಪರವಾಗಿಲ್ಲ’ ಎಂದು ಮತ್ತೆ ಮಾತನ್ನು ಮುಂದುವರಿಸಿದರು. ಭಕ್ತರು ಅವರ ಈ ಅಪೂರ್ವವಾದ ನಡವಳಿಕೆಯನ್ನು ಕಂಡು, ‘ಏಕೆ ಮಹಾರಾಜ್​? ಯಾರ ಹತ್ತಿರ ಮಾತನಾಡಿದಿರಿ?’ ಎಂದು ಚಕಿತರಾಗಿ ಕೇಳಿದರು. ‘ಅದೇ, ಆ ಕಲ್ಕತ್ತೆಯ ಭಕ್ತನ ಕರೆಗೆ ಓಗೊಟ್ಟೆನಷ್ಟೆ.... ’ ಎನ್ನುತ್ತಾ ಅವರು ಅದರ ಹಿನ್ನಲೆ ಯನ್ನು ಬಯಲು ಮಾಡಿದರು.

ಕೆಲವು ದಿನಗಳ ಹಿಂದೆ ಕಲ್ಕತ್ತೆಯಿಂದ ಭಕ್ತನೊಬ್ಬ ಯಾವುದೋ ಕಾರ್ಯಾರ್ಥವಾಗಿ ಕಾಶಿಗೆ ಬಂದಿದ್ದ. ಅದ್ಭುತಾನಂದರನ್ನು ಕಂಡು ಪ್ರಣಾಮಗಳನ್ನು ಸಲ್ಲಿಸಿ ಊರಿಗೆ ಹಿಂದಿರುಗುವುದಕ್ಕೆ ಮೊದಲು ಅವರನ್ನು ಇನ್ನೊಮ್ಮೆ ಭೇಟಿಯಾಗುವೆನೆಂದು ಹೇಳಿದ್ದ. ಕಾರಣಾಂತರಗಳಿಂದ ಬೇಗನೆ ಹಿಂದಿರುಗಬೇಕಾಯಿತು. ಅವಸರದಲ್ಲಿ ಅವರನ್ನು ಭೇಟಿಮಾಡಲು ಮರೆತ. ಆದರೆ ರೈಲು ಹೊರಡುವುದಕ್ಕೆ ಮೊದಲು ಅದ್ಭುತಾನಂದರನ್ನು ಮನಸ್ಸಿನಲ್ಲಿಯೇ ನಮಿಸಿ ಅವರೊಡನೆ ಕ್ಷಮಾ ಯಾಚನೆ ಮಾಡಿದ. ರೈಲ್ವೇಸ್ಟೇಷನ್ನಿನಿಂದ ನಾಲ್ಕೈದು ಮೈಲು ದೂರದಲ್ಲಿರುವ ಅದ್ಭುತಾನಂದರು ಭಕ್ತಪರಿವೃತರಾಗಿ ಮಾತನಾಡುತ್ತಿದ್ದಾಗಲೇ ಅವನ ಪ್ರಣಾಮ, ಪ್ರಾರ್ಥನೆಗಳನ್ನು ಸ್ವೀಕರಿಸಿ ಮೇಲಿನಂತೆ ಉದ್ಗರಿಸಿದ್ದರು! ಹೃದಯಾಂತರಾಳದ ಕರೆ ಯಾವ ಅಡೆತಡೆ ಇಲ್ಲದೆ ಎಲ್ಲವನ್ನೂ ತೂರಿಕೊಂಡು ಹೋಗಬಲ್ಲದೆನ್ನಲು ಬೇರೆ ಸಾಕ್ಷಿ ಬೇಕೆ?

ವಿಭೂತಿಬಾಬು ಒಂದು ದಿನ ಕಾಶಿಯಲ್ಲಿ ತಮ್ಮ ಹಿರಿಯರೊಡನೆ ಮಾತನಾಡುತ್ತ ಕಾಶಿಯ ವಿಶ್ವನಾಥದೇವಾಲಯದಲ್ಲಿ ‘ಶಿವನ ಸಾನ್ನಿಧ್ಯವೆ, ಮತ್ತೇನು? ಎಲ್ಲ ಬರೇ ನಂಬಿಕೆಯೇ ಭ್ರಾಂತಿ!’ ಎಂದು ಗೇಲಿ ಮಾಡಿದರು. ಸಂಜೆಯಹೊತ್ತು ಅವರು ಅದ್ಭುತಾನಂದರನ್ನು ನೋಡಲು ಬರು ತ್ತಿರುವಾಗಲೇ ಅದ್ಭುತಾನಂದರ ಧ್ವನಿಮೊಳಗಿತು, ‘ಎಂಥ ಮೂರ್ಖತೆ! ವಿಶ್ವನಾಥನ ಸಾನ್ನಿಧ್ಯ ವನ್ನು ತಿಳಿಯಲು ನೀನು ಏನು ತಪಸ್ಸು, ಸಾಧನೆಗಳನ್ನು ಮಾಡಿರುವೆ? ಅದಕ್ಕೆ ಬೇಕಾದ ಪವಿತ್ರತೆ ವ್ಯಾಕುಲತೆಗಳು ನಿನ್ನಲ್ಲಿವೆಯೆ?’ ಎಂದು ದೃಢಸ್ವರದಲ್ಲಿ ಹೇಳಿದರು. ವಿಭೂತಿಬಾಬು ಚಕಿತರಾಗಿ ತಮ್ಮ ಸಂಶಯದ ಮಾತುಗಳಿಗಾಗಿ ಪಶ್ಚಾತ್ತಾಪಪಟ್ಟು ವಿನಮ್ರಭಾವದಿಂದ ಅದ್ಭುತಾನಂದರಿಗೆ ಪ್ರಣಾಮ ಮಾಡಿದಾಗ ಅವರು ಏನೂ ತಿಳಿಯದವರಂತೆ ನಕ್ಕು ‘ವಿಶ್ವನಾಥಮಂದಿರಕ್ಕೆ ಹೋಗಿ ಪೂಜೆ ಸಲ್ಲಿಸಿ ತೀರ್ಥಪ್ರಸಾದಗಳನ್ನು ತೆಗೆದುಕೊಂಡು ಬಾ’ ಎಂದು ಹೇಳಿದರು.

ಭಗವಂತನ ಪಾದಸ್ಪರ್ಶವನ್ನು ಮಾಡಿದ ಸಂತರೊಬ್ಬರಿಗೆ ದೂರದ ಭಕ್ತನ ಪ್ರಣಾಮ ಪ್ರಾರ್ಥನೆಗಳು, ಸಂಶಯ ಅಶ್ರದ್ಧೆಗಳು ಅರ್ಥವಾದರೆ ಸರ್ವಜ್ಞನೂ, ಸರ್ವಶಕ್ತನೂ, ಸರ್ವ ವ್ಯಾಪಿಯೂ ಆದ ದೇವರಿಗೆ ನಮ್ಮ ಪ್ರಾರ್ಥನೆ ಕೇಳಿಸದೆ? ಭಕ್ತಿಯ ಕರೆಗೆ ಆತ ಓ ಎನ್ನನೆ?

‘ಖಂಡಿತವಾಗಿಯೂ ಕೇಳಿಸುತ್ತದೆ. ಖಂಡಿತವಾಗಿಯೂ ಆತ ಓಗೊಡುತ್ತಾನೆ’ ಎಂದರು ಅದ್ಭುತಾನಂದರು.

ಆ ನಿಟ್ಟಿನಲ್ಲಿ ಅದ್ಭುತಾನಂದರೇ ಒಂದು ದಿನ ಭಕ್ತನೊಬ್ಬನಿಗೆ ಪುರೀಕ್ಷೇತ್ರದಲ್ಲಿ ತಮ್ಮ ಅನುಭವವನ್ನು ಹೇಳಿದರು: ‘ದೇವಾಲಯದಲ್ಲಿ ಜಗನ್ನಾಥನ ಸಾನ್ನಿಧ್ಯವಿದೆ. ಅವರವರ ಸಾಧನೆ, ಭಜನೆ ಮತ್ತು ಶ್ರದ್ಧೆಗಳಿಗನುಗುಣವಾಗಿ ಭಗವಂತನು ಅಲ್ಲಿ ಕೃಪೆದೋರುತ್ತಾನೆ. ನಾನೂ ಪ್ರಾರ್ಥಿ ಸಿದೆ, “ಶ‍್ರೀಕೃಷ್ಣಚೈತನ್ಯರು ನಿನ್ನ ಯಾವ ದಿವ್ಯ ಮಂಗಲರೂಪವನ್ನು ಕಂಡು ಹಲವು ವರ್ಷ ಗಳವರೆಗೆ ಆ ಸ್ಮರಣೆ ಮಾತ್ರದಿಂದ ಪ್ರೇಮಾಶ್ರುವನ್ನು ಸುರಿಸಿದರೋ, ಅದನ್ನು ಕೃಪೆಯಿಟ್ಟು ನನಗೂ ತೋರು” ಎಂದು. ಭಗವಂತ ನನ್ನ ಪ್ರಾರ್ಥನೆಗೆ ಓಗೊಟ್ಟ. “ಆತನಿಗೆ ನಮ್ಮ ಕರೆ ಕೇಳಿಸಿಯೇ ಕೇಳುತ್ತದೆ.”

‘ನಾವು ಇತರರನ್ನು ಮೋಸಗೊಳಿಸಬಹುದು, ಆದರೆ ಭಗವಂತನನ್ನು ಮೋಸಗೊಳಿಸ ಬಲ್ಲೆವೆ?’ ಎಂದು ಸ್ವಾಮಿ ಅದ್ಭುತಾನಂದರು ಹೇಳುತ್ತ, ‘ಪ್ರಾಮಾಣಿಕತೆ, ನಿಃಸ್ವಾರ್ಥತೆ ಮತ್ತು ಸಹೃದಯತೆಗಳು ಅಧ್ಯಾತ್ಮರಾಜ್ಯವನ್ನು ಪ್ರವೇಶಿಸುವುದಕ್ಕೆ ಅರ್ಹತೆಯನ್ನು ಒದಗಿಸುತ್ತವೆ’ ಎನ್ನುತ್ತಿದ್ದರು. ‘ಸ್ವಾರ್ಥದ ಯಾವ ಉದ್ದೇಶಗಳಿಲ್ಲದೆ ಪರಹಿತಕ್ಕಾಗಿ ಪರಿಶ್ರಮಪಡುವವರು ಭಗ ವಂತನ ಕೃಪೆಗೆ ಪಾತ್ರರಾಗುವರು’ ಎಂದೂ ಅವರು ಆಗಾಗ ಹೇಳುತ್ತಿದ್ದರು.

ಶ‍್ರೀರಾಮಕೃಷ್ಣರೆನ್ನುತ್ತಿದ್ದರು: ‘ಆತ ಇರುವೆಯ ಕಾಲಿನ ಸಪ್ಪಳವನ್ನೂ ಕೇಳಬಲ್ಲ ಎಂದ ಮೇಲೆ ನಿಮ್ಮ ಕರೆ ಆತನಿಗೆ ಕೇಳಿಸದಿರುವುದೆ?’ ಕೇಳಿಸಿಯೇ ಕೇಳಿಸುತ್ತದೆ ವ್ಯಾಕುಲತೆಯಿಂದ ಮೊರೆದಾಗ.


\section{ದೃಷ್ಟಿ, ಕೃಪೆಯ ವೃಷ್ಟಿ}

ಚಿತ್ತರಂಜನ ಮೊಹಂತಿ ಪಶ್ಚಿಮ ಬಂಗಾಳದ ಮಿಡ್ನಾಪುರಜಿಲ್ಲೆಯ ಬಲಿಟಕ್ ಗ್ರಾಮದವನು. ಆತ ಬಿ. ಎ. (ಆನರ್ಸ್​) ಪದವೀಧರನಾಗಿ ವಾಸುದೇವಪುರ ಎಂಬಲ್ಲಿ ಅಧ್ಯಾಪಕನಾಗಿದ್ದ. ೧೯೬೮ ರಲ್ಲಿ ಅವನಿಗೆ ಸನ್ನಿಪಾತದಿಂದ ಕಣ್ಣು ಕುರುಡಾಗಿ ಅಧ್ಯಾಪಕ ವೃತ್ತಿ ಬಿಡುವಂತಾಯಿತು. ಮಾತ್ರ ವಲ್ಲ, ಚಿಕಿತ್ಸೆಗಾಗಿ, ಕುಟುಂಬರಕ್ಷಣೆಗಾಗಿ, ಇದ್ದ ಸ್ವಲ್ಪ ಆಸ್ತಿಯನ್ನೆಲ್ಲ ಮಾರುವಂತಾಯಿತು. ಬೇರೆ ಉಪಾಯ ಕಾಣದೆ ಪರಿಚಿತರ ನಡುವೆ ಇರಲು ಇಚ್ಛಿಸದೆ, ಭಿಕ್ಷೆ ಬೇಡಿಯಾದರೂ ಜೀವನ ನಿರ್ವಹಿಸುವುದೆಂದು ಆತ ದೂರದ ಒರಿಸ್ಸಾದ ಬಲಸೋರ್ ಜಿಲ್ಲೆಯ ಒಂದು ಊರು ಸೇರಿದ.

೧೯೬೯ರ ಶ್ರಾವಣಮಾಸದಲ್ಲಿ ಬಲಸೋರಿನ ಶ‍್ರೀಚಂದನೇಶ್ವರ ದೇವಾಲಯದಲ್ಲಿ ಆರು ದಿನ ಎಡೆಬಿಡದೆ ದೇವರನ್ನು ಪ್ರಾರ್ಥಿಸಿದ. ಕೊನೆಯದಿನ ರಾತ್ರಿ ಶಿವನು ಸ್ವಪ್ನದಲ್ಲಿ ಕಾಣಿಸಿಕೊಂಡು ಹೇಳಿದ: ‘ನೀನು ನಿನ್ನ ಹಿಂದಿನ ಜನ್ಮದಲ್ಲಿ ತಂದೆತಾಯಿಗಳ ಏಕಮಾತ್ರಪುತ್ರನಾಗಿದ್ದೆ. ಒಂದು ದಿನ ನಿನ್ನ ತಾಯಿ ಹರಗೌರೀ ಪೂಜೆಗೆ ಹೂವು, ಆರತಿ, ನೈವೇದ್ಯ ಸಿದ್ಧಮಾಡಿ ಇಡುತ್ತಿದ್ದಾಗ ನೀನು ಎಲ್ಲೋ ನೋಡುತ್ತ ಅವನ್ನು ತುಳಿದುಬಿಟ್ಟೆ. ನಿನ್ನ ತಾಯಿ ಕುಪಿತಳಾಗಿ, ‘ಇಷ್ಟು ಬೆಳಕಿರು ವಲ್ಲಿ ಇವನ್ನು ಕಾಣಲಾರದವ ನೀನು ಕುರುಡನಾಗು!’ ಎಂದಳು. ನಿನ್ನ ಪೂರ್ವಜನ್ಮದ ಆ ತಂದೆ ತಾಯಿ ಈಗ ಬಲಸೋರ್ ಜಿಲ್ಲೆಯ ಬಂಪದಾ ಎಂಬಲ್ಲಿ ಗಣನಾಥ ಬೇಹಾರ ಮತ್ತು ಲಕ್ಷ್ಮೀ ದೇವಿ ಎಂಬ ಹೆಸರಿನಿಂದ ಇದ್ದಾರೆ. ನೀನು ಶಿವರಾತ್ರಿ ದಿನ ಬಂಪಾದಾಕ್ಕೆ ಸಮೀಪವಿರುವ ಜಾರೀಶ್ವರ ದೇವಸ್ಥಾನದ ಕೆರೆಯಲ್ಲಿ ಮಿಂದು, ಆ ಕೆರೆಯ ನೀರನ್ನೇ ತೆಗೆದುಕೊಂಡು ಗಣನಾಥ ಬೇಹಾರ್ ಮನೆಯಲ್ಲಿ ಹರಗೌರೀ ಪೂಜೆ ನಡೆಸು. ನಂತರ ಗಣನಾಥ ಬೇಹಾರ ಮತ್ತು ಲಕ್ಷ್ಮೀ ದೇವೀ ನಿನ್ನನ್ನು ಮುಟ್ಟಲಿ. ಒಡನೆಯೇ ನಿನಗೆ ದೃಷ್ಟಿ ಬರುತ್ತದೆ.”

ಶಿವನ ಆದೇಶದಂತೆ ನಡೆಯಲು ಕುರುಡನಾದ ಮೊಹಂತಿಗೆ ಇತರರ ಸಹಾಯ ಅಗತ್ಯವಿತ್ತು. ಅವನು ತನ್ನ ಸ್ವಪ್ನವನ್ನು ಇತರರಿಗೆ ಹೇಳಿ ಅವರಿಂದ ಸಹಾಯ ಪಡೆದ. ಹೀಗಾಗಿ ಅದು ಅನೇಕರಿಗೆ ತಿಳಿಯಿತು. ಶಿವರಾತ್ರಿಯ ದಿನ ಶಿವನ ಆದೇಶದಂತೆ ಅಲ್ಲಿ ಮಿಂದು, ನೀರು ತಂದು, ಭಕ್ತಿಶ್ರದ್ಧೆ ಗಳಿಂದ ಪೂಜೆ ಮಾಡಿದ. ಅಂದು ಅವನ ಪೂಜೆ ನೋಡಲು ಸಾವಿರಾರು ಜನರು ಸೇರಿದ್ದರು. ಎಲ್ಲರೂ ನೋಡುತ್ತಿದ್ದಂತೆಯೇ ಬೇಹಾರ ದಂಪತಿಗಳು ಅವನ್ನು ಮುಟ್ಟಿ ಹರಸಿದಾಗ ಅವನಿಗೆ ದೃಷ್ಟಿ ಬಂತು. ಭಗವಂತನ ಕೃಪೆಗೆ ಪಾರವಿದೆಯೇ?

ಭಕ್ತಿಪರವಶರಾದ ಅಲ್ಲಿಯ ಜನ ಎರಡು ದಿನ ಹರಗೌರಿಯರ ಜಾತ್ರೆ ನಡೆಸಿದರು. ಮಾತ್ರ ವಲ್ಲ, ಅಲ್ಲಿ ಹರಗೌರಿಯರ ದೇವಸ್ಥಾನ ನಿರ್ಮಿಸಲು ನಿರ್ಧರಿಸಿದರು. ಗಣನಾಥ ಬೇಹಾರ ಮತ್ತು ಆತನ ಪತ್ನಿ ಚಂದನೇಶ್ವರ ದೇವಾಲಯಕ್ಕೂ ಹೋಗಿ ವಿಶೇಷ ಸೇವೆ ಸಲ್ಲಿಸಿದರು. ಗಣನಾಥ ಬೇಹಾರ ಒರಿಸ್ಸಾ ಸರಕಾರದ ಪಿ.ಡಬ್ಲ್ಯು.ಡಿ. ವಿಭಾಗದಲ್ಲಿ ನೌಕರ. ಅವನಿಂದಾಗಿ ಈ ವಿಚಾರ ‘ರಾಷ್ಟ್ರದೀಪ’ ಎಂಬ ಒರಿಯಾ ವಾರಪತ್ರಿಕೆಯಲ್ಲಿ ೨೧-೩-೧೯೭೦ರಲ್ಲಿ ಪ್ರಥಮ ಬಾರಿ ಪ್ರಕಟ ವಾಯಿತು. ಅನಂತರ ಇಂಗ್ಲಿಷಿನ \enginline{‘Truth’} ಪತ್ರಿಕೆಯಲ್ಲೂ ಬೆಳಕು ಕಂಡಿತು. ಶ‍್ರೀಯುತ ಕೋಟ ವಾಸುದೇವ ಕಾರಂತರ ‘ದೇವರನ್ನು ನಂಬುವ’ ಗ್ರಂಥದಲ್ಲಿ ಉಲ್ಲೇಖಿತವಾಗಿರುವ ಆ ವಿಚಾರ ವನ್ನೇ ನಾನಿಲ್ಲಿ ಹೇಳಿದ್ದು. ಅದರಲ್ಲಿ ಇಂತಹ ಇನ್ನೂ ಅನೇಕ ಘಟನೆಗಳ ಪ್ರಸ್ತಾಪವೂ ಇದೆ.

ನಾವೂ ಯಾವಾಗಲೂ ಶುಭವನ್ನೇ ಆಡಬೇಕು, ಶುಭವನ್ನೇ ಕೇಳಬೇಕು, ಶುಭವನ್ನೇ ನೋಡ ಬೇಕು ಎನ್ನುತ್ತದೆ ಶಾಸ್ತ್ರ. ತಾಯಿಯ ಶಾಪ ಎಂಥ ವಿಪತ್ತನ್ನು ಉಂಟುಮಾಡಿತು! ಆದರೆ ಮೊಹಂತಿಗೆ ದೇವರಲ್ಲಿ ಭಕ್ತಿಯಿದ್ದುದರಿಂದ ದೃಷ್ಟಿಸೃಷ್ಟಿಯಾಯಿತು; ಕೃಪೆಯ ವೃಷ್ಟಿಯಾಯಿತು; ಮುಂದಿನ ಭವಿಷ್ಯ ಕತ್ತಲಾಗದೆ ಬೆಳಕಾಯಿತು.

ಒಮ್ಮೆ ಈ ಘಟನೆಯನ್ನು ಉಪನ್ಯಾಸ ಒಂದರಲ್ಲಿ ಹೇಳಿದಾಗ ಸಭೆಯಲ್ಲಿದ್ದ ಉನ್ನತ ಸರಕಾರೀ ಹುದ್ದೆಯಲ್ಲಿದ್ದ ಹರಿಜನ ಬಂಧುವೊಬ್ಬರು ಸಭೆ ಮುಗಿದ ಬಳಿಕ ನನ್ನನ್ನು ಕಂಡು, ಅದೇ ರೀತಿಯ ತಮ್ಮ ಅಪೂರ್ವ ಅನುಭವವನ್ನು ತೋಡಿಕೊಂಡರು. ವೈದ್ಯರುಗಳೆಲ್ಲ ಗುಣ ವಾಗದು ಎಂದು ಕೈಬಿಟ್ಟ ಮೇಲೆ, ಶ್ರದ್ಧೆಭಕ್ತಿಗಳಿಂದ ಎರಡು ವರ್ಷಗಳ ಕಾಲ ಮಾಡಿದ ತೀವ್ರ ಪ್ರಾರ್ಥನೆಯಿಂದಾಗಿ ಅವರ ದೃಷ್ಟಿದೋಷ ದೂರವಾಗಿತ್ತು. ಅವರು ಶ್ರದ್ಧೆ ಮತ್ತು ತಾಳ್ಮೆಯಿಂದ ಪ್ರಾರ್ಥನೆ ಮಾಡಿದ ವಿಧಾನ ಅತ್ಯಂತ ಸ್ಫೂರ್ತಿದಾಯಕವಾಗಿತ್ತು!


\section{ನಡೆ ಮುಂದೆ, ನಡೆ ಮುಂದೆ}

ಒಂದು ದಿನ ಪರಮಹಂಸರ ಬಳಿಗೆ ದೇವರಲ್ಲಿ ಶ್ರದ್ಧೆ ಕಳೆದುಕೊಂಡ ವಿದ್ಯಾವಂತನೊಬ್ಬ ಬಂದ. ಅವನು ಅವರೊಡನೆ ಹೀಗೆಂದು ಹೇಳಿಕೊಂಡ: ‘ನನಗೆ ಈಗ ಐವತ್ತೈದು ವರ್ಷ. ನಾನು ಹದಿನಾಲ್ಕು ವರ್ಷಗಳಿಂದ ದೇವರನ್ನು ಹುಡುಕುವುದರಲ್ಲಿ ನಿರತನಾದೆ. ನನ್ನ ಗುರುಗಳ ಬುದ್ಧಿ ವಾದವನ್ನು ಅನುಸರಿಸಿದೆ. ಎಲ್ಲ ತೀರ್ಥಯಾತ್ರೆಗಳನ್ನೂ ಮಾಡಿದೆ. ಎಷ್ಟೋ ಜನ ಸಾಧು ಸಂತರನ್ನು ನೋಡಿಯಾಯಿತು. ಆದರೂ ಏನೂ ಲಭಿಸಲಿಲ್ಲ. ನಾನು ಮನಶ್ಶಾಂತಿಯನ್ನು ಕಳೆದು ಕೊಂಡಿದ್ದೇನೆ. ಅದನ್ನು ಪಡೆಯಲು ಮನುಷ್ಯ ಪ್ರಯತ್ನವೆಲ್ಲವನ್ನೂ ಮಾಡಿ ವಿಫಲನಾಗಿದ್ದೇನೆ. ಈಗ ನನಗೆ ದೇವರ ಇರುವಿಕೆಯಲ್ಲಿ ನಂಬಿಕೆ ಇಲ್ಲ. ಆದುದರಿಂದ ಇನ್ನು ಅವನನ್ನು ಪ್ರಾರ್ಥಿಸ ಲಾರೆ. ನನಗೇನಾದರೂ ದಾರಿ ಇದೆಯೆ?’

ಪರಮಹಂಸರು ನಗುತ್ತ ಉತ್ತರಿಸಿದರು: ‘ಹೇ ಭಗವಾನ್, ನೀನು ಇರುವುದು ಹೌದಾದರೆ ನನಗೆ ಹಾಗೆಂದು ಕೃಪೆಯಿಟ್ಟು ತಿಳಿಸಿಕೊಡು, ನನ್ನ ಈ ವ್ಯಥೆಯನ್ನು ದೂರಗೊಳಿಸು–ಹೀಗೆಂದು ವ್ಯಾಕುಲತೆಯಿಂದ ಬಿಡದೆ ನೀನು ಪ್ರಾರ್ಥಿಸಬಲ್ಲೆಯಾ? ಫಲಕ್ಕಾಗಿ ಕಾತರಿಸದೆ ನಿಷ್ಠೆಯಿಂದ ಮುನ್ನಡೆಯಬಲ್ಲೆಯಾ? ಹಾಗಾದಲ್ಲಿ ನೀನು ಯಶಸ್ವಿಯಾಗುವಿ.’

ಆತ ಅವರ ಉಪದೇಶವನ್ನು ಪರೀಕ್ಷಿಸಲು ದೃಢನಿಶ್ಚಯ ಮಾಡಿದ. ಒಂದು ವರ್ಷದ ಬಳಿಕ ಅವನು ಪರಮಹಂಸರನ್ನು ಕಂಡು ಗದ್ಗದಿತನಾಗಿ ಅವರಿಗೆ ಪ್ರಣಾಮ ಮಾಡುತ್ತ, ‘ಮಹಾ ಶಯರೇ, ತಾವು ನನ್ನನ್ನು ಬದುಕಿಸಿದಿರಿ, ದಾರಿ ತೋರಿದಿರಿ’ ಎಂದ. ಆಗ ಅವನ ಮಾನಸಿಕ ವ್ಯಗ್ರತೆ ದೂರವಾಗಿತ್ತಲ್ಲದೆ, ಅವನಿಗೆ ಭಗವಂತನಲ್ಲಿ ದೃಢನಂಬಿಕೆಯೂ ಉಂಟಾಗಿತ್ತು.

ದೃಢಭಕ್ತಿಯಿಂದ, ಅಚಲಶ್ರದ್ಧೆಯಿಂದ ಬಿಡದೆ ಪ್ರಯತ್ನಸುವುದರಿಂದಷ್ಟೆ ಸಾಧನೆ ಸಫಲ ವಾಗಬಲ್ಲದು. ಈ ನಿಟ್ಟಿನಲ್ಲಿ ಪಟ್ಟ ಶ್ರಮ ಪರಿವರ್ತನೆಯ ಪರ್ವವನ್ನೆ ಉಂಟುಮಾಡಬಹು ದಾದರೂ, ನೆಟ್ಟ ಸಸಿ ಫಲ ಬರುವ ತನಕ ತಾಳಿದಂತೆ ಇಲ್ಲೂ ತಾಳಿ ಬಾಳಬೇಕು. ಆದರೆ ಎಂದೂ ವಿಚಲಿತನಾಗದೆ ನಡೆ ಮುಂದೆ, ನಡೆ ಮುಂದೆ, ನುಗ್ಗಿ ನಡೆ ಮುಂದೆ.


\section{ನಾಮಸ್ಮರಣೆ ಉಸಿರಾಗಲಿ!}

ರಸಿಕಲಾಲ ದಕ್ಷಿಣೇಶ್ವರದಲ್ಲಿ ಜಾಡಮಾಲಿ.ಶ‍್ರೀರಾಮಕೃಷ್ಣರ ಬಳಿ ನೂರಾರು ಮಂದಿ ಬಂದು ಹೋಗುವುದನ್ನು ನೋಡಿ ನಿಟ್ಟುಸಿರು ಬಿಡುತ್ತಿದ್ದ. ‘ಅವರೆಲ್ಲರೂ ಪರಮಹಂಸರ ದೈವೀವಾಣಿ ಯಲ್ಲಿ ಪಾಲುದಾರರಾಗಿದ್ದಾರೆ. ಆದರೆ ನಾನೋ ದೀನ, ಹೀನ ಸೇವಕ. ನನ್ನ ಜಾತಿಯೇ ನನಗೆ ಅವರ ಸಾನ್ನಿಧ್ಯವನ್ನು ಪಡೆಯುವುದಕ್ಕೆ ಅಡ್ಡಿಯಾಗಿ ನಿಂತಿದೆ’ ಎಂದು ಖಿನ್ನನಾಗುತ್ತಿದ್ದ. ಆದರೆ ಆತನ ಹೃದಯದಲ್ಲಿ ಅವರ ಬಗ್ಗೆ ಗೌರವಾದರಗಳು ಹೆಚ್ಚಿದಂತೆ, ಅವರ ಶ‍್ರೀಮುಖದಿಂದ ಭಗವದ್ವಿಚಾರವನ್ನು ಕೇಳಿ ಕೃಪೆಯನ್ನು ಪಡೆಯಬೇಕೆಂಬ ವ್ಯಾಕುಲತೆಯೂ ಹೆಚ್ಚುತ್ತಿತ್ತು. ಒಂದು ದಿನ ಶ‍್ರೀರಾಮಕೃಷ್ಣರು ಪಂಚವಟಿಯಿಂದ ತಮ್ಮ ಕೋಣೆಯ ಕಡೆಗೆ ಏಕಾಂಗಿಯಾಗಿ ಬರುತ್ತಿ ದ್ದಾಗ, ರಸಿಕ ಮನಸ್ಸಿನ ದುಗುಡವನ್ನು ತಡೆಯಲಾರದೆ ಅವರನ್ನು ಸಮೀಪಿಸಿ ಸಾಷ್ಟಾಂಗ ಪ್ರಣಾಮ ಮಾಡುತ್ತ ಗದ್ಗದಿತನಾಗಿ, ‘ಮಹಾಶಯರೇ, ನನಗೇನು ಗತಿ?’ ಎಂದು ಕೇಳಿದ. ಶ‍್ರೀರಾಮಕೃಷ್ಣರು ‘ಯಾರು? ರಸಿಕ! ಏಳು, ಏಳು’ ಎನ್ನುತ್ತ ದಿವ್ಯಭಾವಾಪನ್ನರಾದರು. ರಸಿಕ ತನ್ನ ಪ್ರೇಮಾಶ್ರುವಿನಿಂದ ಅವರ ಚರಣವನ್ನು ತೋಯಿಸಿದ. ಪರಮಹಂಸರು ಅವನ ಸರಳ ಶ್ರದ್ಧೆಯಿಂದ ಮನಕರಗಿದವರಾಗಿ, ಅವನಿಗೆ ಅಭಯವಚನವೀಯುತ್ತ ‘ನೀನೇನೂ ಚಿಂತಿಸ ಬೇಕಿಲ್ಲ. ನಿನ್ನೆಲ್ಲ ಭಾರವನ್ನೂ ನಾನೇ ವಹಿಸಿದ್ದೇನೆ. ಎಂದಿನಂತೆ ನಿನ್ನ ಕರ್ತವ್ಯಗಳನ್ನು ಮಾಡು. ಬೆಳಿಗ್ಗೆ ಮತ್ತು ಸಂಜೆಯ ಹೊತ್ತು ವ್ಯಾಕುಲನಾಗಿ ಭಗವಂತನನ್ನು ಬಿಡದೆ ಸ್ಮರಿಸು, ಪ್ರಾರ್ಥಿಸು’ ಎಂದರು. ಅವರ ಆಶೀರ್ವಚನದಿಂದ ಅವನ ಬದುಕಿನ ದಿಕ್ಕೇ ಬದಲಿಸಿತು. ಆತ ಅವರ ಆದೇಶ ವನ್ನು ಶಿರಸಾ ವಹಿಸಿದ. ನಾಮಸ್ಮರಣೆ ಆತನ ಬಾಳಿನ ಉಸಿರಾಯಿತು. ಶ‍್ರೀರಾಮಕೃಷ್ಣರ ದೇಹ ತ್ಯಾಗದ ಕೆಲ ವರ್ಷಗಳ ಬಳಿಕ ಅವನ ಅವಸಾನ ಸಮೀಪಿಸಿತು. ಜ್ವರದಿಂದ ನರಳುತ್ತಿದ್ದ ಅವನು ಒಂದು ದಿನ ಮಧ್ಯಾಹ್ನ ತನ್ನ ಹೆಂಡತಿಯನ್ನು ಕರೆದು ತನ್ನನ್ನು ತುಲಸೀಕಟ್ಟೆಯ ಹತ್ತಿರ ಕೊಂಡೊಯ್ಯಬೇಕೆಂದೂ, ತನ್ನ ಕೊನೆಯ ಗಳಿಗೆ ಸಮೀಪಿಸುತ್ತಿದೆಯೆಂದೂ ಹೇಳಿದ. ಆಕೆ ಅತ್ಯಂತ ದುಃಖದಿಂದ ಆತನ ಆಜ್ಞೆಯನ್ನು ಪಾಲಿಸಿದಳು. ಅಲ್ಲಿ ಕುಳಿತು ಪ್ರಾರ್ಥಿಸುತ್ತ ಆತನು ಚಕಿತನಾಗಿ ಯಾರೋ ದಿವ್ಯಪುರುಷರನ್ನು ಕಂಡಂತೆ ‘ಅಪ್ಪಾ, ಆಹಾ! ನೀನು ಬಂದಿದ್ದೀಯೆ. ನಿನ್ನ ಮಾತಿನಂತೆ ಬಂದಿದ್ದೀಯೆ’ ಎಂದು ಅಪೂರ್ವ ಆನಂದ ಆಶ್ಚರ್ಯಗಳಿಂದ ಕೂಗುತ್ತ ಅಸು ನೀಗಿದ.

ರಸಿಕ ಸರಳಾಂತಃಕರಣದಿಂದ ಕೂಗಿ ಕೇಳಿದ–‘ಭಗವಾನ್, ನನಗೇನು ಗತಿ?’ ಎಂಬುದಾಗಿ. ಭಗವಂತನಿಂದ ಆತನಿಗೆ ದೊರೆತ ಉತ್ತರ ‘ಸದ್ಗತಿ’ ಎಂದಿರಬೇಕಲ್ಲವೆ?


\section{ಕರೆಗೆ ಕರಗಿದ, ದಾರಿ ತೋರಿದ}

ಇದು ೧೯೧೮ರಲ್ಲಿ ನಡೆದ ಘಟನೆ. ಶ‍್ರೀರಾಮಕೃಷ್ಣರ ಆಧ್ಯಾತ್ಮಿಕ ಶಿಶುವೆನಿಸಿದ ಸ್ವಾಮಿ ಬ್ರಹ್ಮಾನಂದರು ಕಲ್ಕತ್ತೆಯ ಬಾಗ್​ಬಜಾರಿನ ಬಲರಾಮ ಬಸುವಿನ ಮನೆಯಲ್ಲಿ ತಂಗಿದ್ದರು. ಮಧ್ಯಾಹ್ನ ಊಟ ಮುಗಿಸಿ ಅವರು ವಿಶ್ರಾಂತಿ ತೆಗೆದುಕೊಳ್ಳುತ್ತಿದ್ದ ಸಮಯ. ಸೇವಕ ಸಾಧು ಗಳೊಬ್ಬರು ಅಲ್ಲಿಯೇ ಹೊರಗಡೆ ಬೆಂಚಿನ ಮೇಲೆ ಕುಳಿತುಕೊಂಡಿದ್ದಾಗ, ಸುಮಾರು ಹದಿನೈದು ವರ್ಷ ವಯಸ್ಸಿನ ಹುಡುಗಿಯೊಬ್ಬಳು ತನ್ನ ಸಹೋದರನೊಡನೆ ಅಲ್ಲಿಗೆ ಬಂದಳು. ಸ್ವಾಮಿ ಬ್ರಹ್ಮಾನಂದರನ್ನು ಕಾಣಬೇಕೆಂಬುದಾಗಿ ಕೇಳಿಕೊಂಡಳು. ಸೇವಕ ಸಾಧುಗಳು ‘ಈಗ ಸಾಧ್ಯ ವಿಲ್ಲಮ್ಮಾ’ ಎಂದಾಗ ಅವಳು ತೀರ ದುಃಖಿತಳಾದಳು. ಸ್ವಾಮಿ ಶಾರದಾನಂದರು ತನ್ನನ್ನು ಅಲ್ಲಿಗೆ ಕಳುಹಿಸಿದ್ದಾರೆಂದೂ, ತಾನು ಬ್ರಹ್ಮಾನಂದರಿಗೆ ನಮಸ್ಕರಿಸಿ ಹಿಂದಿರುಗುವೆನೆಂದೂ, ಆಕೆ ಕಳಕಳಿ ಯಿಂದ ಬಿನ್ನವಿಸಿಕೊಂಡ ಮೇಲೆ ಬ್ರಹ್ಮಾನಂದರ ಅನುಮತಿ ದೊರೆಯಿತು. ಬ್ರಹ್ಮಾನಂದರ ದರ್ಶನವನ್ನು ಪಡೆದು ಅವರಿಗೆ ವಂದಿಸಿ ಅಳುತ್ತ, ಆ ಕೋಣೆಯ ಗೋಡೆಯ ಮೇಲಿದ್ದ ಶ‍್ರೀರಾಮ ಕೃಷ್ಣರ ಭಾವಚಿತ್ರವನ್ನು ತೋರಿಸಿ, ‘ಅವರು ನಿಮ್ಮನ್ನು ಕಾಣುವಂತೆ ನನಗೆ ಹೇಳಿದರು’ ಎಂದಳು. ಬ್ರಹ್ಮಾನಂದರು ವಿವರವನ್ನು ಕೇಳಿದಾಗ ಅವಳು ತನ್ನ ಕರುಣೆಯ ಕತೆಯನ್ನು ಹೇಳಿಕೊಂಡಳು.

ಆಕೆಗೆ ಹದಿನಾಲ್ಕನೆ ವಯಸ್ಸಿನಲ್ಲೇ ಮದುವೆಯಾಗಿತ್ತು. ಮದುವೆಯಾದ ಎರಡು ವಾರ ಗಳಲ್ಲೇ ಆಕೆಯು ಪತಿವಿಯೋಗದ ದುರ್ಘಟನೆಯನ್ನು ಅನುಭವಿಸಬೇಕಾಯಿತು. ಭವಿಷ್ಯ ಅತ್ಯಂತ ಅಂಧಕಾರಮಯವಾಯಿತು. ದುಃಖವನ್ನು ತಾಳಲಾರದೆ ಅಳುತ್ತ ಪ್ರತಿನಿತ್ಯ ಎಲ್ಲ ಸಮಯದಲ್ಲೂ ಆಕೆ ಭಗವಂತನನ್ನು ಪ್ರಾರ್ಥಿಸತೊಡಗಿದಳು: ‘ಹೇ ಭಗವಾನ್ ನನ್ನ ಗತಿ ಏನು? ನಾನು ಎಂಥ ಅಸಹಾಯ ಅವಸ್ಥೆಯಲ್ಲಿ ಸಿಕ್ಕಿಕೊಂಡಿದ್ದೇನೆ. ದಯವಿಟ್ಟು ನನಗೆ ದಾರಿ ತೋರು’ ಎಂದು. ಒಂದು ವರ್ಷದ ಮೇಲೆ ಒಂದು ದಿನ ಆಕೆಯ ಸ್ವಪ್ನದಲ್ಲಿ ಶ‍್ರೀರಾಮಕೃಷ್ಣರು ಕಾಣಿಸಿ ಕೊಂಡು ‘ಅಳಬೇಡ, ಬಾಗ್​ಬಜಾರಿನಲ್ಲಿ ನನ್ನ ಮಗು ರಾಖಾಲನಿದ್ದಾನೆ. ಅವನನ್ನು ನೋಡು. ಅವನು ನಿನಗೆ ಸಹಾಯ ಮಾಡುತ್ತಾನೆ’ ಎಂದರು. ಆಕೆಗೆ ರಾಮಕೃಷ್ಣರ ವಿಚಾರವಾಗಲಿ, ಬ್ರಹ್ಮಾ ನಂದರ ವಿಷಯವಾಗಲಿ ತಿಳಿದಿರಲಿಲ್ಲ. ತಾನಿರುವ ಸ್ಥಳದಿಂದ ಅಲ್ಲಿಗೆ ಹೇಗೆ ಹೋಗುವು ದೆಂಬುದೇ ಆಕೆಗೆ ಸಮಸ್ಯೆಯಾಗಿತ್ತು. ತನ್ನ ಪತಿಯ ಮನೆಯಿಂದ ತವರಿಗೆ ಹಿಂದಿರುಗಿ ತನ್ನ ತಾಯಿಯ ಹತ್ತಿರ ನಡೆದ ಘಟನೆಯನ್ನು ಹೇಳಿದಳು. ತಾಯಿಗೆ ಶ‍್ರೀರಾಮಕೃಷ್ಣರ ವಿಚಾರ ತಿಳಿದಿತ್ತು. ಎಲ್ಲ ಸಂಗತಿಗಳನ್ನು ತಿಳಿದ ಮೇಲೆ, ತನ್ನ ಸಹೋದರನೊಂದಿಗೆ ಅವಳು ಬಾಗ್ ಬಜಾರಿಗೆ ಬಂದು, ಸ್ವಾಮಿ ಶಾರದಾನಂದರನ್ನು ಕಂಡ ಬಳಿಕ, ಅವರ ನಿರ್ದೇಶಾನುಸಾರ ಬಲರಾಮನ ಮನೆಯಲ್ಲಿ ಸ್ವಾಮಿ ಬ್ರಹ್ಮಾನಂದರ ದರ್ಶನ ಸಾಧ್ಯವೆಂದು ತಿಳಿದು ಬಂದಿದ್ದಳು.

ಸ್ವಾಮಿ ಬ್ರಹ್ಮಾನಂದರು ಆಕೆಗೆ ಮಂತ್ರದೀಕ್ಷೆಯನ್ನಿತ್ತು ಹರಸಿದರು. ಆಧ್ಯಾತ್ಮಿಕ ಜೀವನಕ್ಕೆ ಮಾರ್ಗದರ್ಶನ ಮಾಡಿ ಸಾಧನೆಗೆ ವ್ಯವಸ್ಥೆ ಮಾಡಿಸಿದರು. ಸಾಧನೆಯಿಂದ ಆಧ್ಯಾತ್ಮಿಕ ಜೀವನ ದಲ್ಲಿ ಉಚ್ಚಸ್ಥಿತಿಯನ್ನು ಹೊಂದಿ, ಬಹುಮಂದಿ ಸಾಧಿಕೆಯರಿಗೆ ಮಾರ್ಗದರ್ಶನ ಮಾಡಿದಳು ಈಕೆ. ಆಕೆಯ ಹೃತ್ಪೂರ್ವಕ ಕರೆಗೆ ಕರಗಿ ತನ್ನೆಡೆಗೇ ಬರುವಂತೆ ದಾರಿ ತೋರಿದ ಭಗವಂತ.


\section{ಕಾರಣವಾರಿಯಲ್ಲ, ಕೃಪಾವಾರಿ}

ಕಾಳೀಪದ ಘೋಷ್ ಕುಡಿದು ಉನ್ಮತ್ತನಾಗಿ ಒಂದು ಚಿಕ್ಕಾಸನ್ನೂ ಮನೆಗೆ ತರುತ್ತಿರಲಿಲ್ಲ. ಆತನ ಪತ್ನಿ ಪರಮಸಾಧ್ವಿ. ತನ್ನ ಪತಿಯನ್ನು ಆ ದುಷ್ಟಚಟ ಮತ್ತು ದುಷ್ಟರ ಸಹವಾಸದಿಂದ ತಪ್ಪಿಸಬೇಕೆಂಬುದು ಆಕೆಯ ತೀವ್ರ ಹಂಬಲ. ಒಂದು ದಿನ ದಕ್ಷಿಣೇಶ್ವರಕ್ಕೆ ಬಂದು ಪರಮ ಹಂಸರನ್ನು ಕಂಡು ಏನಾದರೂ ಔಷಧ, ಯಂತ್ರಮಂತ್ರಗಳಿಂದಾದರೂ ತನ್ನ ಪತಿಯನ್ನು ಸರಿ ದಾರಿಗೆ ತರಬೇಕೆಂದು ಬೇಡಿಕೊಂಡಳು. ಪರಮಹಂಸರೂ ತನಗೇನೂ ತಿಳಿಯದೆಂದೂ, ವಾದ್ಯ ಕುಟೀರದಲ್ಲಿರುವ ಶ‍್ರೀಶಾರದಾದೇವಿಯನ್ನು ಕೇಳಬೇಕೆಂದೂ ಹೇಳಿದಾಗ, ಆಕೆ ಅವರನ್ನು ಸಮೀ ಪಿಸಿ, ತನ್ನ ಅಳಲನ್ನು ಅವರಲ್ಲಿ ತೋಡಿಕೊಂಡಳು. ಶ‍್ರೀಮಾತೆ ಆಕೆಯನ್ನು ಪರಮಹಂಸರಲ್ಲಿಗೇ ತಿರುಗಿ ಕಳುಹಿಸಿ ಅವರನ್ನೇ ಬೇಡಿಕೊಳ್ಳುವಂತೆ ಹೇಳಿದರು. ಹೀಗೆ ಮೂರು ಬಾರಿ ಅಲ್ಲಿಂದ ಇಲ್ಲಿಗೆ ಆಕೆ ವ್ಯಥಿತ ಹೃದಯದಿಂದ ಸುತ್ತಾಡುವುದನ್ನು ಕಂಡು ಶ‍್ರೀಮಾತೆಯ ಹೃದಯ ಕರುಣೆ ಯಿಂದ ತುಂಬಿತು. ಆಕೆಯನ್ನು ತಮ್ಮ ಹತ್ತಿರ ಕರೆದು ತಾವು ಪೂಜಿಸಿದ ಬಿಲ್ವಪತ್ರೆಯಲ್ಲಿ ದೇವರ ನಾಮವೊಂದನ್ನು ಬರೆದು ಅವಳಿಗೆ ಕೊಟ್ಟು ‘ಪ್ರತಿನಿತ್ಯವೂ ಮಂತ್ರಜಪವನ್ನು ಶ್ರದ್ಧೆಯಿಂದ ಮಾಡುತ್ತ ಪ್ರಾರ್ಥಿಸು’ ಎಂದಳು. ಪೂರ್ಣಶ್ರದ್ಧೆಯಿಂದ ಅವಳು ಹನ್ನೆರಡು ವರ್ಷಗಳ ಕಾಲ ತನ್ನ ಪತಿಗಾಗಿ ಶುಭಕಾಮನೆಯಿಂದ ಭಗವಂತನನ್ನು ವ್ಯಾಕುಲತೆಯಿಂದ ಪ್ರಾರ್ಥಿಸಿದಳು. ಹನ್ನೆರಡು ವರ್ಷಗಳ ಬಳಿಕವೆ ಒಂದು ದಿನ ಕಾಳೀಪದ ಘೋಷ್ ಪರಮಹಂಸರನ್ನು ನೋಡಲು ಹೋದ. ಪರಮಹಂಸರು ಅವನನ್ನು ಅದೇ ಮೊದಲ ಬಾರಿಗೆ ಕಂಡದ್ದು. ನೋಡಿದೊಡನೆಯೇ ಉದ್ಗರಿಸಿದರು: ‘ಆಹಾ ಈ ಮೂರ್ಖ, ಹೆಂಡತಿಯನ್ನು ಹನ್ನೆರಡು ವರ್ಷಗಳಿಂದ ಸತಾಯಿಸಿ ಈಗ ಇಲ್ಲಿಗೆ ಬಂದಿದ್ದಾನೆ’ ಎಂದು. ಕಾಳೀಪದ ಚಕಿತನಾಗಿ ನಿಂತು ಅವರನ್ನೇ ನೋಡುತ್ತಿದ್ದ. ಪರಮಹಂಸರು ‘ಏನು ಬೇಕು ಬಾಬು ನಿನಗೆ?’ ಎಂದು ಪ್ರಶ್ನಿಸಿದರು. ಪ್ರಾಯಃ ಮದ್ಯದ ಅಮಲಿ ನಲ್ಲಿದ್ದ ಕಾಳೀಪದ ನಿರ್ಲಜ್ಜನಾಗಿ, ‘ಸ್ವಲ್ಪ ಮದ್ಯ ಕೊಡಿಸುವಿರಾ ಸ್ವಾಮಿ?’ ಎಂದೇ ಕೇಳಿದ. ಪರಮಹಂಸರು ನಗುತ್ತ, ‘ಅದೇನೋ ಕೊಡಬಲ್ಲೆ, ಆದರೆ ಇಲ್ಲಿಯ ಮದ್ಯದ ಅಮಲು ಬಹಳ ಜಾಸ್ತಿ, ಅದನ್ನು ನಿನಗೆ ಸಹಿಸಿಕೊಳ್ಳಲು ಸಾಧ್ಯವೇ?’ ಎಂದರು. ಅವರ ಮಾತಿನ ಮರ್ಮವನ್ನು ತಿಳಿಯದೆ, ನಿಜವಾದ ಮದ್ಯವನ್ನೇ ಕೊಡುವರೆಂದು ತಿಳಿದು ಸಂತೋಷದಿಂದ, ‘ಮಹಾಶಯರೆ, ಹಾಗಾದರೆ ವಿಲಾಯಿತಿ ಮದ್ಯವನ್ನೇ ಕೊಡಿ, ಸ್ವಲ್ಪ ಗಂಟಲು ಒದ್ದೆಮಾಡಿಕೊಳ್ಳುತ್ತೇನೆ’ ಎಂದ ಕಾಳೀಪದ. ಪರಮಹಂಸರು ‘ಇದು ವಿಲಾಯಿತಿ ಮದ್ಯವಲ್ಲ. ಇದು ದೇಶೀಯ ಕಾರಣವಾರಿ. ಎಲ್ಲರೂ ಅದರ ಅಮಲನ್ನು ಸಹಿಸಲಾರರು. ಒಂದು ಸಲ ಇದರ ರುಚಿ ಹಿಡಿದರೆ ಮತ್ತೆ ವಿಲಾ ಯತಿಯದಾಗಲಿ, ಇನ್ನಾವ ಮದ್ಯವಾಗಲಿ ರುಚಿಸುವುದಿಲ್ಲ. ನೀನು ಇಲ್ಲಿನ ಮದ್ಯವನ್ನು ಸ್ವೀಕರಿ ಸಲು ಸಿದ್ಧನೊ?’ ಪರಮಹಂಸರ ಪ್ರಶ್ನೆಗೆ ಕಾಳೀಪದ ಏನು ಯೋಚಿಸಿದನೊ! ತಟ್ಟನೆ ಉತ್ತರ ವಿತ್ತ: ‘ಮಹಾಶಯರೆ, ದಯವಿಟ್ಟು ಕೊಡಿ, ಜೀವನವೆಲ್ಲ ಈ ಅಮಲಿನ ಗುಂಗಿನಲ್ಲೇ ಕಳೆಯು ತ್ತೇನೆ.’ ಪರಮಹಂಸರು ಹತ್ತಿರ ಕರೆದು ಸ್ಪರ್ಶಿಸಿದರು. ಸ್ಪರ್ಶಮಾತ್ರದಿಂದ ಕಾಳೀಪದನಲ್ಲಿ ಅದ್ಭುತ ಬದಲಾವಣೆ ಉಂಟಾಯಿತು. ಅವನು ಆ ದಿವ್ಯ ಸ್ಫೂರ್ತಿಯ ಸ್ಪರ್ಶದಿಂದ ಪ್ರಭಾವಿತ ನಾಗಿ ಅಪೂರ್ವ ಭಾವಾಂತರವನ್ನು ಹೊಂದಿದನು. ರೋಮಾಂಚಿತನಾದ ಆತನ ಕಣ್ಣುಗಳಿಂದ ಪ್ರೇಮಾಶ್ರು ಸುರಿಯತೊಡಗಿತು. ಅವನನ್ನು ಸಮಾಧಾನಗೊಳಿಸಲು ಅಲ್ಲಿ ಸೇರಿದವರಿಗೆ ಸಾಧ್ಯ ವಾಗಲಿಲ್ಲ. ಭಗವತ್ಪ್ರೇಮದ ಅಮಲು ಅವನಿಗೆ ಹಿಡಿಯಿತು. ಅವನ ಬದುಕಿನ ದಿಕ್ಕೇ ಬದಲಾ ಯಿತು; ಉನ್ನತ ಆಧ್ಯಾತ್ಮಿಕಸ್ಥಿತಿ ಪ್ರಾಪ್ತವಾಯಿತು.

ತೀವ್ರ ಭಕ್ತಿ ಶ್ರದ್ಧೆ ವ್ಯಾಕುಲತೆಗಳು ಮುಪ್ಪುರಿಗೊಂಡ ಪ್ರಾರ್ಥನೆ ಫಲಿಸುವುದು ಖಂಡಿತ. ಸಾಧ್ವೀ ಪತ್ನಿಯ ಅಂತಹ ಪ್ರಾರ್ಥನೆಯಿಂದ ಕಾಳೀಪದ ಉದ್ಧಾರವಾದ.


\section{ಆಂತರ್ಯದ ಅಳಲು}

ಕರ್ನಾಟಕದ ಸಂತಶ್ರೇಷ್ಠರಲ್ಲೊಬ್ಬರಾಗಿ ಪರಿವರ್ತಿತರಾದ ಜಗನ್ನಾಥದಾಸರ ಮೊದಲಿನ ಹೆಸರು ಶ‍್ರೀನಿವಾಸ. ಮಹಾವಿದ್ವಾಂಸರಾದ ಅವರು ವಿನಮ್ರ ಭಕ್ತರಾಗಿ ಪರಿವರ್ತನೆಗೊಂಡ ಕತೆ ರೋಚಕವಾಗಿದೆ. ಅವರು ಜನಿಸಿದುದು ಕ್ರಿ. ಶ. ೧೬೪೯ರಲ್ಲಿ ಉತ್ತರ ಕರ್ನಾಟಕದ ಹಳ್ಳಿಯಲ್ಲಿ. ಸಂಸ್ಕೃತ ಭಾಷೆ ವೇದಾಂತ ಶಾಸ್ತ್ರಗಳಲ್ಲಿ ಅದ್ವಿತೀಯ ಪಾಂಡಿತ್ಯವನ್ನು ಪಡೆದು ಯೌವನದಲ್ಲೇ ಅವರು ಕೀರ್ತಿಶಾಲಿಗಳಾಗಿದ್ದರು. ಇವರೆಡೆಗೆ ಪಾಠಕ್ಕಾಗಿ ಶಿಷ್ಯರು ಹೆಚ್ಚಿನ ಸಂಖ್ಯೆಯಲ್ಲಿ ಬರ ತೊಡಗಿದರು. ಪಾಂಡಿತ್ಯದ ಬಲದಿಂದ ಅಹಂಕಾರವು ಹುಲುಸಾಗಿ ಬೆಳೆಯಿತು. ಪರನಿಂದೆ ಆತ್ಮಪ್ರಶಂಸೆಗಳ ಕಾರ್ಯವೂ ಬೆಳೆಯಿತು. ಅದರಿಂದ ಅವರು ಭಕ್ತರನ್ನು ಶಾಸ್ತ್ರಜ್ಞಾನ ಮತ್ತು ಶಾಸ್ತ್ರೋಕ್ತ ಸಾಧನಾಹೀನರೆಂದು ಹಳಿದು ಹೀಯಾಳಿಸಲು ಹಿಂಜರಿಯುತ್ತಿರಲಿಲ್ಲ. ಆ ಕಾಲದ ಭಾಗವತಾಗ್ರೇಸರರಾದ ವಿಜಯದಾಸರನ್ನೂ ಅಲ್ಪಭಾವದಿಂದ ಕಂಡು, ಅಲ್ಪಮಾತುಗಳಿಂದ ಅವಹೇಳನಮಾಡಿದರು. ಶರಣರನ್ನು ಕೆಣಕುವ ಸಾಹಸ ‘ಹಾವಿನ ಹೆಡೆಯನ್ನು ಕೊಂಡು ಕೆನ್ನೆಯ ತುರಿಸಿಕೊಂಡಂತೆ, ಒಡಲಲ್ಲಿ ಸುಣ್ಣದಕಲ್ಲು ಕಟ್ಟಿಕೊಂಡು ಮಡುವಿಗೆ ಬಿದ್ದಂತೆ’ ಎಂದು ಬಸವಣ್ಣನವರು ಹೇಳಿದ್ದರಲ್ಲವೇ? ಭಾಗವತೋತ್ತಮರೂ, ಮಹಾನುಭಾವರೂ ಆದ ಶ‍್ರೀವಿಜಯದಾಸರನ್ನು ನಿಂದಿಸಿದ ಫಲವು ಶ‍್ರೀನಿವಾಸಪ್ಪನವರಿಗೆ ಬಹು ಬೇಗನೇ ಲಭಿಸಿತು. ಅವರು ತೀವ್ರ ಕ್ಷಯರೋಗದಿಂದ ನರಳಿದರು. ಅನ್ನಾಹಾರ ಸೇವನೆಯಿಲ್ಲದೆ ಅತ್ಯಂತ ದುರ್ಬಲ ರಾಗಿ ದೀನ, ಹೀನಸ್ಥಿತಿಯಲ್ಲಿ ಬಿದ್ದರು. ಅತ್ಯಂತ ಅಸಹಾಯಕ ಸ್ಥಿತಿಯಲ್ಲಿ, ದುರಂತ ಸನ್ನಿವೇಶ ದಲ್ಲಿ ಸಿಕ್ಕಿಕೊಂಡು ಆರ್ತರಾದರು. ಬೇರಾವ ದಾರಿಕಾಣದೆ ಹತ್ತಿರದ ಗುಡಿಯಲ್ಲಿ ಆಂಜನೇಯ ಸ್ವಾಮಿಗೆ ನಲವತ್ತೆಂಟು ದಿನಗಳ ಕಾಲ ವಿಶೇಷ ರೀತಿಯಿಂದ ಪೂಜೆ, ಪ್ರಾರ್ಥನೆಗಳನ್ನು ಸಲ್ಲಿಸುವ ವ್ಯವಸ್ಥೆ ಮಾಡಿದರು. ಪೂಜಾಂತ್ಯದ ಒಂದು ದಿನ ಕನಸಿನಲ್ಲಿ ಮಾರುತಿ ಕಾಣಿಸಿ, ‘ವಿಜಯ ದಾಸರನ್ನು ನಿಂದೆ ಮಾಡಿದುದರಿಂದ ನಿನಗೆ ಈ ಭೀಕರ ಕ್ಷಯರೋಗವು ಪ್ರಾಪ್ತವಾಗಿದೆ. ಈಗಲಾ ದರೂ ಅವರನ್ನು ಮೊರೆಹೊಕ್ಕು ಅವರ ಅನುಗ್ರಹ ಸಂಪಾದಿಸಿದರೆ ನಿನ್ನ ರೋಗ ಪರಿಹಾರವಾಗಿ ಪೂರ್ಣ ಆಯುಸ್ಸು ಪ್ರಾಪ್ತವಾಗುವುದು’ ಎಂಬ ವಚನವಿತ್ತ.

ಶ‍್ರೀನಿವಾಸಪ್ಪ ಪಶ್ಚಾತ್ತಾಪದ ಕುಲುಮೆಯಲ್ಲಿ ಕರಗಿದ. ಪಾಂಡಿತ್ಯದ ಸೊಕ್ಕು ಅಡಗಿ ಆತ ಅತ್ಯಂತ ನಮ್ರನಾದ. ಅಶಕ್ತನಾಗಿದ್ದುದರಿಂದ ಪಲ್ಲಕ್ಕಿಯಲ್ಲಿ ಕುಳಿತು ವಿಜಯದಾಸರೆಡೆಗೆ ಹೋಗಿ ಕಂಬನಿದುಂಬಿ, ‘ಪಾಂಡಿತ್ಯದ ಒಣಹೆಮ್ಮೆಯಿಂದ ಭ್ರಾಂತನಾಗಿ, ತಮ್ಮ ಮಾಹಾತ್ಮ್ಯ ವನ್ನು ತಿಳಿಯದೆ ನಿಂದೆ ಮಾಡಿದ ಪಾಪಿ ನಾನು. ಅಪರಾಧ ಕ್ಷಮಿಸಿ ಉದ್ಧರಿಸಬೇಕು’ ಎಂದು ಸಾಷ್ಟಾಂಗವೆರಗಿ ಆಂತರ್ಯದ ಆಳಲನ್ನು ತೋಡಿಕೊಂಡ. ದಯಾಳುಗಳಾದ ವಿಜಯದಾಸರು ಅವರನ್ನು ಕ್ಷಮಿಸಿ, ತಮ್ಮ ಶಿಷ್ಯರಾದ ಗೋಪಾಲದಾಸರಲ್ಲಿಗೆ ಅವರನ್ನು ಕಳುಹಿಸಿ ಅವರಿಂದ ಮಾರ್ಗದರ್ಶನವನ್ನು ಪಡೆಯುವಂತೆ ನಿರ್ದೇಶಿಸಿದರು. ಗೋಪಾಲದಾಸರು ಶ‍್ರೀನಿವಾಸರಿಂದ ಎಲ್ಲ ವೃತ್ತಾಂತಗಳನ್ನು ತಿಳಿದುಕೊಂಡು, ಅವರಿಗೆ ಮಂತ್ರೋಪದೇಶವನ್ನು ಮಾಡಿ ಆಹಾರವನ್ನು ಅಭಿಮಂತ್ರಿಸಿ ಕೊಟ್ಟರು. ಕ್ಷಯರೋಗ ಪೀಡಿತರಾಗಿ ಒಂದಗುಳು ಅನ್ನವನ್ನೂ ತಿನ್ನಲಾರದ ಶ‍್ರೀನಿವಾಸಪ್ಪನವರು, ಅವರು ಮಂತ್ರಿಸಿ ಕೊಟ್ಟ ಎರಡು ಜೋಳದ ರೊಟ್ಟಿಗಳನ್ನೂ ಭಕ್ಷಿಸಿ ಜೀರ್ಣಿಸಿಕೊಂಡರು. ಮುಂದೆ ಕೆಲವು ದಿನಗಳಲ್ಲೇ ಶ‍್ರೀನಿವಾಸಪ್ಪನವರ ಕ್ಷಯರೋಗವು ಸಂಪೂರ್ಣ ಪರಿಹಾರವಾಯಿತು. ಶ‍್ರೀಗುರುವಿನಲ್ಲಿ ಶರಣಾಗತರಾಗಿ, ದಾಸತ್ವಕ್ಕೆ ಅಂಕಿತಪ್ರದಾನ ಮಾಡಬೇಕೆಂದಾಗ ಗೋಪಾಲದಾಸರು, ‘ನೀನು ಪಂಢರಪುರಕ್ಕೆ ಹೋಗಿ ಶ‍್ರೀ ಪಾಂಡುರಂಗ ವಿಠಲನ ಸ್ಮರಣೆ ಮಾಡುತ್ತ ಚಂದ್ರಭಾಗಾ ನದಿಯಲ್ಲಿ ಸ್ನಾನಮಾಡುತ್ತಿರುವ ಕಾಲದಲ್ಲಿ, ವಿಠಲನ ಅನುಗ್ರಹದಿಂದ ನಿನ್ನ ಹೆಸರು ಕೆತ್ತಲ್ಪಟ್ಟ ಶಿಲಾಫಲಕವೊಂದು ನಿನ್ನ ತಲೆಯ ಮೇಲೆ ಬಂದು ಕೂಡುತ್ತದೆ. ಆಗಲೇ ಅದನ್ನು ಸ್ವೀಕರಿಸಿ ಭಗವಂತನನ್ನು ಸ್ತುತಿಸು’ ಎಂದರು. ಶ‍್ರೀ ಗೋಪಾಲ ದಾಸರ ವಚನದಂತೆ ಅಲ್ಲಿ ಆ ಅದ್ಭುತವು ನಡೆಯಿತು. ಅವರಿಗೆ ‘ಜಗನ್ನಾಥ ವಿಟ್ಠಲ’ ಎಂಬ ಅಂಕಿತ ದೊರೆಯಿತು. ಶ‍್ರೀನಿವಾಸಪ್ಪನವರು ಅಂದಿನಿಂದ ಜಗನ್ನಾಥದಾಸರಾದರು. ಭಗವಂತನ ಅನಂತ ಮಹಿಮೆಗಳನ್ನು ಕೀರ್ತಿಸಿ, ಸಕಲ ಪುಣ್ಯಕ್ಷೇತ್ರಗಳನ್ನು ಸಂದರ್ಶಿಸಿ, ಮಹಾ ಭಗವದ್ಭಕ್ತ ರೆಂದು ವಿಖ್ಯಾತರಾದರು.

ಭಕ್ತಿಯ ಅಸಾಧಾರಣ ಶಕ್ತಿಯನ್ನೂ ಮಹಿಮೆಯನ್ನೂ ತಿಳಿಸುವ ಮಹಾಕಾವ್ಯವನ್ನು ಜಗನ್ನಾಥ ದಾಸರು ಹೇಳಿದ ಮನನೀಯ ಸಾಲುಗಳು ಇಂತಿವೆ:

“ಮಲಗಿ ಪರಮಾದರದಿ ಪಾಡಲು ಕುಳಿತು ಕೇಳುವನ್, ಕುಳಿತು ಪಾಡಲು ನಿಲುವನ್, ನಿಂತರೆ ನಲಿವನ್, ನಲಿದರೆ ಒಲಿವನ್​”–“ಸುಲಭನೊ ಹರಿ ತನ್ನವರ ಬಿಟ್ಟರೆಗಳಿಗೆ ನಿಲಲರಿಯನ್.” ಇಂಥವನನ್ನು ಅರಿಯದೇ ಪಾಮರರು ಜಗದಲ್ಲಿ ನರಳುವರು.


\section{ಸಂಸಾರಿಗಳಿಗೂ ಸಾಧ್ಯ}

ಗೋಂದಾವಳಿ ಕ್ಷೇತ್ರದಲ್ಲಿ ವಾಸವಾಗಿದ್ದ ಶ‍್ರೀ ಬ್ರಹ್ಮಚೈತನ್ಯ ಮಹಾರಾಜರಿಗೆ ಒಂದು ದಿನ ಸಂಸಾರಿಗಳಾದ ಭಕ್ತಜನರು ಒಂದು ಪ್ರಶ್ನೆಯನ್ನು ಕೇಳಿದರಂತೆ: ‘ನಾವು ಸಂಸಾರಿಕರು. ನಮಗೆ ಹೆಂಡತಿ ಮಕ್ಕಳು, ಹೊಲಮನೆ, ದನಕರುಗಳು–ಮುಂತಾಗಿ ಅನೇಕ ಉಪಾಧಿಗಳಿವೆ. ತಾವು ಹೇಳುವಂತೆ ಶ‍್ರೀರಾಮನಾಮಸ್ಮರಣೆಯನ್ನು ಸತತವಾಗಿ ಮಾಡುವುದಾಗಲೀ, ಶ‍್ರೀರಾಮನನ್ನು ಏಕಾಗ್ರತೆಯಿಂದ ಧ್ಯಾನಿಸುವುದಾಗಲಿ ಕೊನೆಗೆ ಸ್ವಲ್ಪಕಾಲ ಒಂದೇ ಚಿತ್ತದಿಂದ ಜಪ ಮಾಡುವು ದಾಗಲಿ ನಮಗೆ ಸಾಧ್ಯವಿಲ್ಲವಾಗಿದೆ. ಇಂಥ ಸ್ಥಿತಿಯಲ್ಲಿ ನಾವು ಭಗವಂತನ ಸಾಕ್ಷಾತ್ಕಾರವನ್ನು ಪಡೆಯುವುದು ಹೇಗೆ? ನಮ್ಮಂಥವರಿಗೆ ಪರಮಾರ್ಥ ಪ್ರಾಪ್ತಿಗೆ ಮಾರ್ಗವಾವುದು? ದಯವಿಟ್ಟು ತಿಳಿಸಿಕೊಡಿ.’

ಅದಕ್ಕೆ ಶ‍್ರೀ ಮಹಾರಾಜರು ಉತ್ತರಿಸಿದರು: ‘ನೀವು ಹೆಂಡತಿ ಮಕ್ಕಳನ್ನು ಕಂಡಾಗ ಶ‍್ರೀರಾಮನು ಈ ಹೆಂಡತಿ ಮಕ್ಕಳನ್ನು ನನಗೆ ಪಾಲನೆ ಮಾಡುವುದಕ್ಕಾಗಿ ಕೊಟ್ಟಿದ್ದಾನೆ. ನಾನು ಅವರ ಬಗ್ಗೆ ಏನೇನು ಮಾಡಬೇಕಾದ ಕರ್ತವ್ಯವಿದೆಯೋ ಅದೆಲ್ಲವನ್ನೂ ಮಾಡಿ ಶ‍್ರೀರಾಮನಿಗೆ ಒಪ್ಪಿಸಬೇಕು. “ಈ ಬಗ್ಗೆ ಏನು ಮಾಡಿದಿರಿ?” ಎಂದು ಪರಮಾತ್ಮನು ಕೇಳುತ್ತಾನೆ. ನಾನವನಿಗೆ ಉತ್ತರ ಹೇಳಬೇಕು–ಎಂಬುದನ್ನು ಮತ್ತೆ ಮತ್ತೆ ಮನಸ್ಸಿಗೆ ತಂದುಕೊಳ್ಳಿ. ಹೀಗೆಯೇ ಮನೆ ಹೊಲ ದನಗಳೇ–ಮುಂತಾದ ಎಲ್ಲದರ ಬಗ್ಗೆಯೂ ನಾನು ಶ‍್ರೀರಾಮನಿಗೆ ಉತ್ತರ ಹೇಳಬೇಕು ಎಂದು ಮತ್ತೆ ಮತ್ತೆ ನೆನಪಿಸಿಕೊಳ್ಳಿ. ಇದರಿಂದ ನಿಮಗೆ ನಿಮ್ಮ ಸಂಸಾರದ ಯಾವುದೇ ವಸ್ತು ವನ್ನು ನೋಡಿದಾಗ “ಇದು ಶ‍್ರೀರಾಮನದು, ನನಗೆ ಅವನು ಇದನ್ನು ನೋಡಿಕೊಳ್ಳಲು ಕೊಟ್ಟಿ ದ್ದಾನೆ. ನಾನು ನನ್ನ ಕರ್ತವ್ಯವನ್ನು ಆ ಬಗ್ಗೆ ಮಾಡಿ ಅವನಿಗೆ ಒಪ್ಪಿಸಬೇಕು” ಎಂಬ ಭಾವವು ಹುಟ್ಟುವುದು. ಕೆಲವು ದಿನ ಪ್ರಯತ್ನಪೂರ್ವಕವಾಗಿ ಈ ಭಾವನೆಯನ್ನು ರೂಢಿಸಿಕೊಂಡರೆ, ಕಾಲಾಂತರದಲ್ಲಿ ಅದು ರೂಢಿಯಾಗಿ, ಮುಂದೆ ಸಹಜವಾಗಿಯೇ ಆ ಭಾವನೆಯು ನಿಮ್ಮ ಅಂತಃಕರಣದಲ್ಲಿ ಮೂಡುವುದು. ಇದರಿಂದ ಮುಂದೆ ‘ನನ್ನದು’ ಎಂಬ ಅಭಿಮಾನವು ಹೊರಟು ಹೋಗಿ, ಒಳಗೂ ಹೊರಗೂ ರಾಮಭಾವನೆಯು ವ್ಯಾಪಿಸಿಬಿಡುವುದು. ಈ ಸಾಧನೆಯನ್ನು ಬಿಡದೆ ಆರು ತಿಂಗಳ ಕಾಲ ಮಾಡಿ ನೋಡಿರಿ. ಶ‍್ರೀರಾಮನ ಸಾಕ್ಷಾತ್ಕಾರವಾಗದಿದ್ದರೆ ಆಗ ನನ್ನಲ್ಲಿ ಬಂದು ಕೇಳಿರಿ’ ಎಂದರಂತೆ.

ಪ್ರಾರ್ಥನೆ ಬಲವಾದಂತೆ ಸಂಸಾರದ ಬಂಧನಗಳೂ ಸಡಿಲಾಗಿ, ದೇವರ ಕಡೆಗೆ ಓಟ ತೀವ್ರ ವಾಗುತ್ತದೆ. ದೇವರು ನೋಡುವುದು ಭಕ್ತನ ಮನಸ್ಸು ಹೃದಯಗಳನ್ನೇ ಹೊರತು ಆತ ಸಂನ್ಯಾಸಿಯೋ, ಸಂಸಾರಿಯೋ ಎಂಬುದನ್ನಲ್ಲ.


\section{ರಾಮನ ಹಿಡಿ, ಕಾಮನ ಹೊಡಿ}

ಗುರೂಪದೇಶವನ್ನು ಪಡೆದು ಅಧ್ಯಾತ್ಮದ ದಾರಿಯಲ್ಲಿ ತೀವ್ರವಾಗಿ ಮುನ್ನಡೆಯಬೇಕೆಂಬ ಹಂಬಲದ ವ್ಯಕ್ತಿಗಳೂ, ಸಾಧನೆಗೆ ಬಾಧಕಗಳಾದ ನಾನಾ ತರದ ವಿಘ್ನಗಳನ್ನು ಎದುರಿಸಿ ಹತಾಶ ರಾಗುವುದುಂಟು. ಒಳ್ಳೆಯ ಸಂಕಲ್ಪ ಶ್ರದ್ಧೆಗಳಿಂದ ಸಾಧನೆಯನ್ನು ಪ್ರಾರಂಭಿಸಿದ ಸಾಧಕನು, ಎಷ್ಟೋ ವೇಳೆ ಕೆಲವೊಂದು ಬಾಹ್ಯಾಚರಣೆಗಳಿಂದಲೇ ತೃಪ್ತನಾಗಿ ದಾರಿ ತಪ್ಪುವ ಪ್ರಸಂಗಗಳೂ ಇವೆ. ಗುರಿಯ ಬಗ್ಗೆ ಸ್ಪಷ್ಟವಾದ ತಿಳಿವಳಿಕೆ, ಸಾಧನೆಯ ಕ್ರಮಗಳ ಪರಿಚಯ, ನಿಜವಾದ ಸಾಧಕರ ಹಿತನುಡಿಯ ಪ್ರೋತ್ಸಾಹ ಮತ್ತು ವಿಷಯವಸ್ತುಗಳಿಂದ ದೂರವಿರುವ ಅವಕಾಶ–ಇವುಗಳನ್ನು ಹೊಂದಿದ್ದೂ, ಹಳೆಯ ಕುಸಂಸ್ಕಾರಗಳಿಂದ ಮನಸ್ಸು ಚಂಚಲವಾಗುವುದುಂಟು. ಮುಖ್ಯವಾಗಿ ಸಾಧಕನ ಅತಿ ದೊಡ್ಡ ಶತ್ರು ಸ್ತ್ರೀಪುರುಷರ ದೈಹಿಕ ಆಕರ್ಷಣೆಗೆ ಕಾರಣವಾದ ಕಾಮ. ಇಂದ್ರಿಯ ಮನಸ್ಸು ಬುದ್ಧಿಗಳೇ ಅದರ ವಾಸಸ್ಥಾನ. ಅದನ್ನು ಅಲ್ಲಿ ನಿಂತಿರಲು ಬಿಡಕೂಡದೆಂಬುದು ಭಗವದ್ಗೀತೆಯ ಬೋಧನೆ. ಇಂದ್ರಿಯಗಳನ್ನು ನಿಗ್ರಹಿಸಲು ಕಲಿಯುವುದೇ ಮೊದಲಿನ ಹೆಜ್ಜೆ. ಇದಕ್ಕೆ ವಿವೇಕ ವೈರಾಗ್ಯಗಳೊಂದಿಗೆ ಮಿತಾಹಾರ ವಿಹಾರ ನಿದ್ರೆಗಳು ಆವಶ್ಯಕವಾದರೂ ಅವುಗಳೇ ಸರ್ವಸ್ವವಲ್ಲ. ಶ್ರದ್ಧೆ ವ್ಯಾಕುಲತೆಗಳಿಂದ ಮಾಡುವ ಪ್ರಾರ್ಥನೆ, ನಾಮಸ್ಮರಣೆ ಮತ್ತು ಧ್ಯಾನ ಇವು ಅತ್ಯಂತ ಆವಶ್ಯಕ. ಯೋಗಾಸನ, ಹಠಯೋಗದ ಕ್ರಿಯೆಗಳು, ಔಷಧ ಸೇವನೆ, ಪ್ರಾಣಾ ಯಾಮದ ಸಹಕಾರವಿಲ್ಲದೆ ಬ್ರಹ್ಮಚರ್ಯ ಅಸಾಧ್ಯವೆಂದು ತಿಳಿದವರುಂಟು. ಶ‍್ರೀರಾಮಕೃಷ್ಣರು ಈ ವಿಷಯದಲ್ಲಿ ತೋರಿದ ಬೆಳಕು ಮತ್ತು ಅವರ ಉಪದೇಶವನ್ನನುಸರಿಸಿ ಪರಿಶುದ್ಧರಾದ ಸಾಧಕರ ಅನುಭವದ ಮಾತುಗಳು ಗಮನಾರ್ಹ.

ಶ‍್ರೀರಾಮಕೃಷ್ಣರ ಶಿಷ್ಯಾಗ್ರಣಿಗಳಲ್ಲಿ ಒಬ್ಬರಾದ ಸ್ವಾಮಿ ಯೋಗಾನಂದರು ತಮ್ಮ ಹದಿನೈ ದನೇ ವರ್ಷದ ಹೊತ್ತಿಗೇ ದಕ್ಷಿಣೇಶ್ವರಕ್ಕೆ ಹೋಗಿಬರುತ್ತಿದ್ದರು. ಒಂದು ದಿನ ದೇವಾಲಯದ ವಠಾರದಲ್ಲಿ ಹಠಯೋಗಿಯೊಬ್ಬ ವಿವಿಧ ಆಸನಗಳನ್ನೂ, ಯೌಗಿಕ ಕ್ರಿಯೆಗಳನ್ನೂ ತೋರಿಸಿ ಯುವಕರನ್ನು ಆಕರ್ಷಿಸುತ್ತಿದ್ದ. ಯೋಗಾನಂದರೂ ಅದನ್ನು ಕುತೂಹಲದಿಂದ ವೀಕ್ಷಿಸುತ್ತಿ ದ್ದರು. ಕಾಮತೃಷೆಯನ್ನು ಹೋಗಲಾಡಿಸಲು ಪರಮಹಂಸರು ಏನಾದರೊಂದು ಆಸನವನ್ನು ಹೇಳಿಕೊಡಬಹುದು ಅಥವಾ ಅಳಲೇಕಾಯಿಯೇ ಮೊದಲಾದ ಔಷಧಗಳನ್ನು ತಿನ್ನಲು ಹೇಳಬಹು ದೆಂದು ಅವರು ನಂಬಿಕೊಂಡಿದ್ದರು. ಆದರೆ ಪರಮಹಂಸರು ಅದೇನನ್ನೂ ಹೇಳದೇ, ‘ಚೆನ್ನಾಗಿ ಹರಿನಾಮ ಸ್ಮರಣೆಯನ್ನು ಮಾಡು. ಆಗ ಕಾಮಾಸಕ್ತಿಯು ದೂರವಾಗುವುದು’ ಎಂದರು. ಯೋಗಾನಂದರಿಗೆ ಈ ಮಾತು ಹೆಚ್ಚಿನ ಆಸಕ್ತಿಯನ್ನು ಹುಟ್ಟಿಸಲಿಲ್ಲ. ‘ಹರಿನಾಮ ಸ್ಮರಣೆ ಮಾಡಿದರೆ ಕಾಮಾಸಕ್ತಿ ಹೋಗುವುದಾದರೆ ಎಷ್ಟೋ ಜನರು ಜಪಮಾಲೆ ತಿರುಗಿಸುತ್ತ ನಾಮ ಸ್ಮರಣೆ ಮಾಡುತ್ತಾರಲ್ಲ? ಅವರು ಈ ಬಾಧೆಯಿಂದ ಮುಕ್ತರಾಗಿದ್ದಾರೆಯೇ?’ ಎಂಬುದು ಅವರ ಸಂಶಯವಾಗಿತ್ತು. ಇನ್ನೊಮ್ಮೆ ದಕ್ಷಿಣೇಶ್ವರಕ್ಕೆ ಬಂದು ಹಠಯೋಗಿಯ ಪ್ರದರ್ಶನಗಳನ್ನು ನೋಡುತ್ತ ನಿಂತಾಗ, ಪರಮಹಂಸರು ಅಲ್ಲಿಗೆ ಹೋಗಿ ಅವರನ್ನು ಕೈ ಹಿಡಿದು ತಮ್ಮ ಕೋಣೆಗೆ ಕರೆದುಕೊಂಡು ಬಂದು, ‘ಹಠಯೋಗದ ಕ್ರಿಯೆಗಳನ್ನು ಅಭ್ಯಾಸ ಮಾಡಿದರೆ ಮನಸ್ಸು ದೇಹದ ಮೇಲೆಯೇ ನಿಲ್ಲುವುದು, ದೇವರೆಡೆಗೆ ಹೋಗದು. ಹಾಗಾಗಿ ವ್ಯಾಕುಲತೆಯಿಂದ ಭಗವಂತನನ್ನು ಪ್ರಾರ್ಥಿಸು. ರಾಮನಿದ್ದಲ್ಲಿ ಕಾಮವಿರದು. ಬಿಡದೆ ರಾಮನ ನೆನೆ’ ಎಂದರು. ಯೋಗಾನಂದರಿಗೆ ಆಗ ಕೊಂಚ ಆಸಕ್ತಿ ಹುಟ್ಟಿತು. ಪರಮಹಂಸರ ಮಾತನ್ನು ಪರೀಕ್ಷಿಸುವೆನೆಂದುಕೊಂಡರು. ಏಕಾಗ್ರತೆಯಿಂದ ಹರಿ ನಾಮಸ್ಮರಣೆಯನ್ನು ಮಾಡುತ್ತ ಮಾಡುತ್ತ, ಪರಮಹಂಸರು ಹೇಳಿದ ಮಾತಿನ ಯಥಾರ್ಥತೆಯನ್ನು ತಿಳಿದುಕೊಂಡರು. ಪಾಶವಿಕ ಪ್ರವೃತ್ತಿಗಳಿಂದ ಸಂಪೂರ್ಣ ಮುಕ್ತ ರಾದರು. ಹಠಯೋಗದ ಸಣ್ಣಪುಟ್ಟ ಕ್ರಿಯೆಗಳನ್ನಾಗಲಿ, ಆರೋಗ್ಯ ರಕ್ಷಣೆಗೆ ಅನುಕೂಲವಾದ ಆಸನ ವ್ಯಾಯಾಮಗಳನ್ನಾಗಲಿ, ಮಾಡಬಾರದೆಂದಲ್ಲ. ಅವುಗಳ ಪ್ರಭಾವ ನಮ್ಮ ಶರೀರ ಮನಸ್ಸಿನ ಮೇಲೆ ಖಂಡಿತವಾಗಿಯೂ ಇದೆ. ಆದರೆ ಭಗವತ್ಸಾಕ್ಷಾತ್ಕಾರದ ತೀವ್ರ ಹಂಬಲವೊಂದೇ ಮನಸ್ಸಿನ ಕೊಳೆಗಳನ್ನೆಲ್ಲ ನಾಶಮಾಡಬಲ್ಲುದು ಎಂಬುದು ತಿಳಿಯಬೇಕಾದ ವಿಚಾರ. ಸಾಕ್ಷಾ ತ್ಕಾರದ ಹಂಬಲವಿಲ್ಲದ ಬಾಹ್ಯ ಸಾಧನೆಗಳು, ಇಂದ್ರಿಯ ನಿಗ್ರಹದ ದಾರಿಯಲ್ಲಿ ನಮ್ಮನ್ನು ಬಹುದೂರ ಕೊಂಡೊಯ್ಯಲಾರವು.

ಭಗವಂತನಿಗಾಗಿ ತೀವ್ರ ವ್ಯಾಕುಲತೆ ಮತ್ತು ಪ್ರೇಮ ಇವುಗಳು ಹೃದಯದಲ್ಲಿ ಉತ್ಪನ್ನವಾಗ ದಿದ್ದರೆ ಪಾಶವಿಕ ಪ್ರವೃತ್ತಿಗಳು ತೊಲಗಲಾರವೆಂಬುದು ಸಾಧಕರೆಲ್ಲರ ಅನುಭವ. ತೀವ್ರ ವ್ಯಾಕುಲತೆ ಪ್ರೇಮಗಳಾದರೋ ಹೃತ್ಪೂರ್ವಕ ಆಂತರಿಕ ಪ್ರಾರ್ಥನೆಯಿಂದಲೇ ಸಾಧ್ಯ. ಪ್ರಾರ್ಥನೆ ಎಂದರೆ ಕೇವಲ ಕೆಲವು ಶಬ್ದಗಳನ್ನು ಮಾತ್ರ ಉಚ್ಚರಿಸುವುದಲ್ಲ, ಅದು ತ್ರಿಕರಣಪೂರ್ವಕವಾದ ಅಭೀಪ್ಸೆ, ಶ್ರದ್ಧೆ ಮತ್ತು ಭಕ್ತಿಗಳಿಂದ ಮಾಡಿದ ದೀರ್ಘಕಾಲದ ಸ್ಮರಣೆಯ ಫಲಿತಾಂಶ.


\section{ಪ್ರಯತ್ನದಿಂದ ಫಲ}

ಎಲ್ಲ ಪರಿಸ್ಥಿತಿಗಳಲ್ಲೂ ಭಗವಂತನ ಸಹಾಯವನ್ನು ಕೋರುತ್ತ ನಾವು ಮಾಡಬೇಕಾದ ಕರ್ತವ್ಯಗಳನ್ನು ಪ್ರಾಮಾಣಿಕ ಬುದ್ಧಿಯಿಂದ, ನಿಷ್ಠೆಯಿಂದ ನೆರವೇರಿಸಬೇಕೆಂಬುದು ಸಂತರ ವಚನ. ‘ಭಗವಂತನು ಸರ್ವಶಕ್ತ, ಅವನಿಗೆ ನಮ್ಮ ಪ್ರಾರ್ಥನೆ ಸಲ್ಲಿಸಿದರೆ ಎಲ್ಲವೂ ಸಾಧ್ಯ’ ಎನ್ನುತ್ತಾ ದಿನವೆಲ್ಲ ಮನ ಬಂದಂತೆ ವರ್ತಿಸಿ, ಒಂದೆರಡು ಗಂಟೆಗಳ ಕಾಲ ಭಾವೋದ್ವೇಗದಿಂದ ಗಟ್ಟಿಯಾಗಿ ಭಜನೆಮಾಡಿ,ಧನ್ಯನಾದೆನೆಂದು ತಿಳಿದರೆ ಆದೀತೇ? ಅಂಟುಜಾಡ್ಯದ ಉಪಟಳದಿಂದ ದೂರವಾಗಬೇಕೆಂದು ಬಾಹ್ಯಶೌಚ ವಿಧಾನಗಳನ್ನೆಲ್ಲ ಅಲ್ಲಗಳೆದು, ಆಹಾರ, ಪಥ್ಯ, ಪಾನೀಯ ಇವುಗಳಲ್ಲಿ ನಿರ್ಲಕ್ಷ್ಯದಿಂದಿದ್ದು, ಒಂದೆರಡು ಗಂಟೆಗಳ ಕಾಲ ಉಚ್ಚಕಂಠದಿಂದ ಹರಿನಾಮ ಸ್ಮರಣೆ ಮಾಡಿದರೆ ನಿರೀಕ್ಷಿಸಿದ ಫಲ ದೊರತೀತೇ? ಭಗವದ್ಭಕ್ತನು ದಕ್ಷನಾಗಿರಬೇಕೆಂಬುದು ಭಗವದ್ಗೀತೆಯ ಮಾತು. ಯಾವುದಾದರೊಂದು ಕಾರ್ಯವು ಯಶಸ್ವಿಯಾಗಬೇಕಾದರೆ ಐದು ಕಾರಣಗಳಿರಬೇಕೆಂದು ಶಾಸ್ತ್ರ ಹೇಳುವುದು: ಉಪಯುಕ್ತವಾದ ದೇಶ, ಉದ್ಯಮಶೀಲನಾದ ಕರ್ತೃ, ಇಂದ್ರಿಯಪಟುತ್ವ, ಪುನಃ ಪುನಃ ಮಾಡುವ ಪ್ರಯತ್ನ, ಮತ್ತು ದೈವ. ಒಂದು ಕೈಯಲ್ಲಿ ದೈವ, ಇನ್ನೊಂದರಲ್ಲಿ ಪುರುಷಕಾರವನ್ನು ದೃಢವಾಗಿ ಹಿಡಿದವರೇ ಗಂತವ್ಯವನ್ನು ಸೇರುತ್ತಾರೆ. ಹಾಗಿಲ್ಲದೆ ಇದ್ದರೆ ದೇವರು ನಮಗೆ ಪುರುಷಕಾರವನ್ನೂ, ಸಮರ್ಪಣಭಾವವನ್ನೂ ಕೊಟ್ಟಿರುವು ದೇತಕ್ಕೆ? ಯಾರು ತಮಗೆ ತಾವೇ ಸಹಾಯಮಾಡಿಕೊಳ್ಳಬಲ್ಲರೋ, ಅವರಿಗೆ ದೇವರು ಸಹಾಯ ಮಾಡುತ್ತಾನೆ ಎನ್ನುವ ಮಾತಿದೆ. ಮನುಷ್ಯರೆಲ್ಲರಿಗೂ ದೇವರು ಒಂದಿಷ್ಟು ಬುದ್ಧಿಸಾಮರ್ಥ್ಯ ವನ್ನೂ, ಕಾರ್ಯಶಕ್ತಿಯನ್ನೂ ಕರುಣಿಸಿದ್ದಾನೆ. ಪ್ರತಿಯೊಬ್ಬರೂ ಜೀವನದಲ್ಲಿ ನಿರ್ವಹಿಸಬೇಕಾದ ಕರ್ತವ್ಯದ ಹೊರೆಯನ್ನು ದೇವರ ಹೆಗಲಿಗೆ ಹಿಂದಿರುಗಿಸಿದ್ದೇವೆಂದುಕೊಂಡು ಸೋಮಾರಿಯಾಗಿ ಕಾಲ ಕಳೆದರೆ ದೈವೇಚ್ಛೆಗೆ ಅಧೀನವಾಗಿ ನಡೆದಂತಾಗುವುದಿಲ್ಲ! ತನ್ನ ಭಾರವನ್ನು ತಾನು ಸರಿಯಾಗಿ ಹೊರುವಂತೆ ಮಾಡಲು ದೇವರನ್ನು ಪ್ರಾರ್ಥಿಸುತ್ತ, ಮಿಕ್ಕದ್ದನ್ನು ದೈವಕ್ಕೆ ಒಪ್ಪಿಸಬೇಕೆಂಬುದು ಸಂತರ ವಚನ. ಉನ್ನತ ಸ್ಥಿತಿಯನ್ನು ಹೊಂದಿದ ಸಾಧಕನ ಸಮರ್ಪಣಾಭಾವದ ವಿಚಾರ ಬೇರೆ. ಪೂರ್ವಕಾಲದಲ್ಲಿ ಭಾರತದ ಪುಷಿಗಳು ಮನೋವಿಜ್ಞಾನ, ಶರೀರವಿಜ್ಞಾನ, ಜ್ಯೋತಿಷ್ಯ, ರಾಜನೀತಿ –ಇವೇ ಮೊದಲಾದ ವಿಷಯಗಳಲ್ಲಿ ವಿಶೇಷ ತಿಳಿವಳಿಕೆಯನ್ನು ಸಂಪಾದಿಸಿದ್ದರು. ಉತ್ಸಾಹ, ಉದ್ಯಮಶೀಲತೆ ಇಲ್ಲದೆ ಅವರು ಅದೃಷ್ಟವನ್ನೇ ನೆಚ್ಚಿ ಕುಳಿತಿದ್ದರೆ? ಅವರಲ್ಲಿ ಎಂಥ ನಿರ್ಭೀತ ಉದ್ಯಮಶೀಲತೆ, ಕುತೂಹಲ ಮತ್ತು ಆವಿಷ್ಕಾರ ಮಾಡುವ ಬುದ್ಧಿ ಇತ್ತು! ಅಗ್ಗದ ವೈರಾಗ್ಯ ಮತ್ತು ಭಕ್ತಿಗಳ ಹೆಸರಿನಲ್ಲಿ ಎಷ್ಟೋ ವೇಳೆ ನಮ್ಮನ್ನು ನಾವು ಮೋಸಮಾಡಿಕೊಳ್ಳುವ ಪ್ರಸಂಗ ಬರುವುದುಂಟು. ಆದುದರಿಂದ ವಿವೇಕ, ವಿಚಾರ, ಆತ್ಮವಿಮರ್ಶೆಗಳಿಂದ ಕೂಡಿದ ಪ್ರಾರ್ಥನೆ ಮಾತ್ರ ಫಲಕಾರಿಯಾಗಬಲ್ಲುದು ಎಂಬುದು ಸ್ಪಷ್ಟವಲ್ಲವೆ?

ಹುಡುಗಿಯೊಬ್ಬಳ ಪುಟ್ಟ ಸಹೋದರನು ಹಕ್ಕಿಗಳನ್ನು ಹಿಡಿಯಲು ಗೂಡಿನಂಥ ಒಂದು ಸಾಧನವನ್ನು ತಯಾರಿಸಿದ್ದ. ಸಣ್ಣ ಹಕ್ಕಿಗಳನ್ನು ಹಿಡಿದು ಹಿಂಸಿಸುವುದೇ ಆತನಿಗೆ ಬಹಳ ಮೋಜಿನ ಆಟವಾಗಿತ್ತು. ಆದರೆ ಆಕೆ ಅದನ್ನು ಕಂಡು ತುಂಬ ಸಂಕಟಪಡುತ್ತಿದ್ದಳು. ಅವಳ ಪಾಲಿಗೆ ಅವನ ಆಟ ಅತ್ಯಂತ ಕ್ರೂರ ಮತ್ತು ಹೇಯವಾದ ಕಾರ್ಯವಾಗಿತ್ತು. ಅವನನ್ನು ಆ ಆಟದಿಂದ ತಪ್ಪಿಸ ಬೇಕೆಂದು ಆಕೆ ಯೋಚಿಸುತ್ತಿದ್ದಳು. ಕೆಲವು ದಿನಗಳ ಬಳಿಕ ಆ ಕಾರ್ಯದಲ್ಲಿ ಯಶಸ್ಸು ದೊರೆತು ದನ್ನು ಕಂಡು ಅಚ್ಚರಿಗೊಂಡ ಆಕೆ ತಾಯಿಯೊಡನೆ ಹೇಳಿದಳು:

‘ಅಮ್ಮಾ, ಮೊಟ್ಟ ಮೊದಲಿಗೆ ನನ್ನ ಸಹೋದರನು ಒಳ್ಳೆಯವನಾಗಲಿ ಎಂದು ದೇವರನ್ನು ಬೇಡಿದೆ; ಆ ಬಳಿಕ ಯಾವ ಹಕ್ಕಿಯೂ ಅವನ ಪಂಜರದಲ್ಲಿ ಬೀಳದಿರಲಿ ಎಂದು ಪ್ರಾರ್ಥಿಸಿದೆ; ಆ ಬಳಿಕ ಹಕ್ಕಿಗಳನ್ನು ಹಿಡಿಯಲು ಮಾಡಿದ ಆ ಎಲ್ಲ ಪಂಜರಗಳನ್ನು ತುಳಿದು ನಾಶಮಾಡಿದೆ.’

ಪ್ರಾರ್ಥನೆಯೊಂದಿಗೆ ಸ್ವಪ್ರಯತ್ನವನ್ನೂ ಸೇರಿಸಿ, ನಮ್ಮ ಹಳೆಯ ಹವ್ಯಾಸಗಳನ್ನು ಮುರಿದು, ನೂತನ ಜೀವನಾದರ್ಶಗಳನ್ನು ಅನುಸರಿಸುವಂತಾಗಬೇಕಲ್ಲವೆ?


\section{ಸಂತರು ನೀಡಿದ ಸಾಂತ್ವನ}

ಬಾಳಿನ ಸಂಕಷ್ಟಗಳಿಂದ ಕಂಗಾಲಾದ ಒಬ್ಬ ಭಕ್ತರ ಬವಣೆಗೆ ಕರಗಿದ ಸಂತರೊಬ್ಬರು ನೀಡಿದ ಸಾಂತ್ವನದ ನುಡಿಗಳಿಲ್ಲಿವೆ. ಅವುಗಳಲ್ಲಿ ಪ್ರಾರ್ಥನೆಯ ಸತ್ತ್ವವೆಲ್ಲ ಅಡಗಿರುವುದನ್ನು ಗಮನಿಸಿ:

ಭಗವಂತ ಸರ್ವಶಕ್ತ, ಭಗವಂತ ದಯಾಮಯ. ಭಗವಂತನಲ್ಲಿ ಶರಣಾಗತರಾದವರು ಯಾವುದಕ್ಕೂ ಹೆದರಬೇಕಾಗಿಲ್ಲ. ಈಗ ಬಂದಿರುವ ಕಷ್ಟಗಳನ್ನು ದೇವರ ದಯೆಯ ದಾನವೆಂದು ಭಾವಿಸು. ಭಗವನ್ನಾಮದಲ್ಲಿ ಶ್ರದ್ಧೆಯಿಡು. ಈಗ ಒದಗಿರುವ ಕಷ್ಟಗಳು ಜೀವನಪೂರ್ತಿ ಇರ ಲಾರವು. ಸೇತುವೆಯ ಕೆಳಗೆ ಹರಿಯುವ ನೀರಿನಂತೆ ಕಷ್ಟಗಳೂ ಸಹ ನಿವಾರಣೆಯಾಗುತ್ತವೆ. ತಾಳ್ಮೆಗೆಡಬೇಡ. ‘ತಲ್ಲಣಿಸದಿರು ಕಂಡ್ಯ ತಾಳು ಮನವೆ.’ ನಮ್ಮ ಜೀವನದಲ್ಲಿ ಬಂದೊದಗುವ ಎಲ್ಲ ಕಷ್ಟನಷ್ಟಗಳಿಗೂ ಅರ್ಥವುಂಟು. ಕಾರ್ಯಕಾರಣ ಸಂಬಂಧ ಉಂಟು. ಹಿಂದಾದುದಕ್ಕೆ ಚಿಂತೆ ಬೇಡ. ದೇವರು ಎಲ್ಲವನ್ನೂ ನೋಡುತ್ತಿದ್ದಾನೆ. ನಿನ್ನ ಎಲ್ಲ ಪ್ರಾರ್ಥನೆಯನ್ನೂ ಕೇಳು ತ್ತಿದ್ದಾನೆ. ಪ್ರಾರ್ಥಿಸಿದ್ದು ದೊರೆಯಲಿಲ್ಲವೆಂದು ದೇವರನ್ನು ದೂರಬೇಡ; ಪ್ರಾರ್ಥನೆಗೆ ಫಲ ಉಂಟು. ಶ್ರದ್ಧಾಭಕ್ತಿಯೆಂಬ ಎತ್ತುಗಳಿಂದ,ಕಾಯವೆಂಬ ಜಮೀನಿನಲ್ಲಿ, ಮಂತ್ರವೆಂಬ ನೇಗಿಲಿ ನಿಂದ ಉಳುಮೆ ಮಾಡಿದಲ್ಲಿ, ಭಗವಂತನ ಕೃಪೆಯೆಂಬ ಗಂಗೆಯ ಸಹಾಯದಿಂದ ಮುಂದೆ ಫಲ ಸಿಗುವುದರಲ್ಲಿ ಸ್ವಲ್ಪವೂ ಸಂದೇಹವಿಲ್ಲ. \enginline{Electric wire (}ಇಲೆಕ್ಟ್ರಿಕ್ ವೈರ್​)ನ ಮೂಲಕ ಹರಿಯುವ ವಿದ್ಯುತ್ ಕಣ್ಣಿಗೆ ಅಗೋಚರ. ಆದರೆ ಬಲ್ಬಿನ ಸಹಾಯದಿಂದ ವಿದ್ಯುತ್ತಿನ ಇರುವಿಕೆ ಖಚಿತವಾಗುತ್ತದೆ. ಅಂತೆಯೇ ಶ್ರದ್ಧಾಭಕ್ತಿ ಎಂಬ \enginline{wire (}ವೈರ್​)ನ ಮೂಲಕ ಭಗವಂತನ ಕೃಪೆ ಯೆಂಬ \enginline{electricity (}ವಿದ್ಯುತ್​) ಹರಿಯುತ್ತಿದೆ. ಮಂತ್ರವೆಂಬ ಬಲ್ಬಿನ ಮೂಲಕ ಪ್ರಕಾಶ ದೊರೆಯುವುದು. ಅಂತೆಯೇ ಹೆಚ್ಚು ನಾಮೋಚ್ಚಾರಣೆ ಮಾಡು. ಅಗ್ನಿಯಲ್ಲಿ ಬೆಂದಾಗ ಚಿನ್ನ ಪರಿಶುದ್ಧವಾಗುವುದು. ಅಂತೆಯೇ ಕಷ್ಟಗಳಲ್ಲಿ ನೊಂದು ಬೆಂದಾಗ ಮನಸ್ಸಿಗೆ ಅರಿವಾಗುವುದು, ಈ ಜಗತ್ತೆಲ್ಲ ದುಃಖಮಯ; ಭಗವಂತನ ಪಾದಾರವಿಂದವಲ್ಲದೆ ಬೇರೆಲ್ಲೂ ನಮಗೆ ಸ್ಥಳವಿಲ್ಲ; ಭಗವಂತನಲ್ಲಿ ಶರಣಾಗದೆ ಬೇರೆ ದಾರಿಯಿಲ್ಲ. ಪ್ರಾರ್ಥನೆ ಮಾಡು. ನಿನಗೆ ಪ್ರತಿಬಾರಿ ಕಷ್ಟ ಬಂದಾಗಲೂ ದೇವರ ಸ್ಮರಣೆ ಮಾಡು. ಭಗವಂತ ದಾರಿ ತೋರುತ್ತಾನೆ. ನೋಡು, ಜೋಗದಲ್ಲಿ ತಯಾರಾದ ವಿದ್ಯುತ್​ರಾಜ್ಯದ ಹಲವು ಮುಖ್ಯ ಕೇಂದ್ರಗಳ ಸಹಾಯದಿಂದ ಎಲ್ಲೆಡೆಗೂ ತಲುಪು ವಂತೆ, ದೇವರಿಂದ ಆದೇಶ ಪಡೆದ ಮಹಾತ್ಮರ ಮೂಲಕ ಧಾರ್ಮಿಕ ಜಾಗೃತಿ ಉಂಟಾಗುತ್ತದೆ.

ನಿಮ್ಮ ಮನೆಗೆ \enginline{water-pipe connection (}ವಾಟರ್ ಪೈಪ್ ಸಂಪರ್ಕ) ಇದೆಯಲ್ಲವೆ? ಇಲ್ಲಿ ಮುಖ್ಯ \enginline{pipe}ಮೂಲಕ ಭಗವಂತನ ಕೃಪೆ ಹರಿಯುತ್ತಿದೆ. ಹೇಗೆ ಎಲ್ಲ ಮನೆಗಳಿಗೂ ಏಕಕಾಲದಲ್ಲಿ ನೀರು ಸರಬರಾಜಾಗಬಲ್ಲದೋ, ಅಂತೆಯೇ ಭಗವಂತ ಏಕಕಾಲದಲ್ಲಿ ಎಲ್ಲರ ಪ್ರಾರ್ಥನೆ ಕೇಳಬಲ್ಲ. ಆದರೆ ಮನೆಯಲ್ಲಿರುವವರು ನಲ್ಲಿಯನ್ನು ತಿರುಗಿಸಿದಾಗ ಮಾತ್ರ ನೀರು ಬರುವು ದಲ್ಲವೇ? ಅದೂ ನಲ್ಲಿಯ ಅಡಿ ಪಾತ್ರೆ ಇಟ್ಟಾಗ ಮಾತ್ರ ನೀರು ತುಂಬುವುದಲ್ಲವೇ? ಹಾಗೆಯೇ ನಾವು ವಿಧೇಯತೆಯಿಂದ ನಡೆದಾಗ ಮಾತ್ರ ಕೃಪೆ ಸಾಧ್ಯ. ನಾಮಸ್ಮರಣೆಯನ್ನಂತೂ ಬಿಡದೆ ಮಾಡು. ಪ್ರತಿಬಾರಿ ಮಂತ್ರ ಉಚ್ಚರಿಸಿದಾಗಲೂ ಭಗವಂತನೆಡೆಗೆ ನೀನು ಒಂದು ಹೆಜ್ಜೆ ಇಡು ತ್ತಿರುವಂತೆ ಭಾವಿಸು. ಭಗವಂತನ ಕೃಪಾಬಿಂದು ನಿನ್ನ ಒಡಲನ್ನು ಸೇರಿದಂತೆ ಭಾವಿಸು. ಸುತ್ತ ಮುತ್ತಲೂ ಗಾಳಿಯಿದೆ. ಆದರೆ ಅದು ಕಣ್ಣಿಗೆ ಕಾಣಲಾರದು. ಮರಗಳ ಅಲುಗುವಿಕೆಯಿಂದ ನಮಗೆ ಗಾಳಿಯ ಇರುವಿಕೆಯ ಅರಿವಾಗುವುದು. ಅಂತೆಯೇ ಮಹಾತ್ಮರ ಜೀವನದಿಂದ ಭಗ ವಂತನ ಇರುವಿಕೆಯ ಅರಿವಾಗುವುದು. ಭಗವಂತ ಇರುವೆ ಕಾಲಿನ ಸಪ್ಪಳವನ್ನೂ ಕೇಳಬಲ್ಲ. ನಿನ್ನ ಪ್ರಾರ್ಥನೆ ಕೇಳದೇ ಇರುವನೇ? ಅಗೋಚರವಾದ ದಿವ್ಯಶಕ್ತಿಯಿಂದ ಈ ವಿಶ್ವಬ್ರಹ್ಮಾಂಡ ನಿಯಂತ್ರಿಸಲ್ಪಟ್ಟಿದೆ. ಅಂತಹ ದಯಾಮಯ ಭಗವಂತನನ್ನು ಸದಾ ಪ್ರಾರ್ಥಿಸು. ನಿನಗೆ ಶಾಂತಿ ದೊರೆಯುವುದು. ನಿನ್ನ ಎಲ್ಲ ಕಷ್ಟಗಳೂ ದೂರವಾಗುವುವು. ಖಂಡಿತ ಮಂಗಳವಾಗುವುದು. ಹಾಲಿನಲ್ಲಿ ಕರಗಿದ ಸಕ್ಕರೆ ಕಣ್ಣಿಗೆ ಕಾಣದು. ಆದರೆ ರುಚಿ ನೋಡಿದಾಗ ಸಕ್ಕರೆಯ ಸಿಹಿ ಅನುಭವ ವಾಗುವುದು. ಅಂತೆಯೇ ಭಗವಂತನ ನಾಮ ಹಾಗೂ ಭಗವಂತ ಬೇರೆಯಲ್ಲ. ನಾಮವನ್ನು ಜಪಿಸುತ್ತ ಹೋದಂತೆ ಭಗವಂತನ ದರ್ಶನ ನಮಗೆ ದೊರೆಯುವುದು. ಪುಟ್ಟ ಬಾಲಕನ ಸೊಂಟಕ್ಕೆ ಹಗ್ಗ ಕಟ್ಟಿ ಬಾವಿಯಲ್ಲಿ ಈಜು ಕಲಿಸುತ್ತಿರುವಂತೆ, ಭಗವಂತ ಭಕ್ತರ ಸೊಂಟಕ್ಕೆ ಕೃಪೆಯ ಹಗ್ಗ ಕಟ್ಟಿ ಈ ಸಂಸಾರಸಾಗರವನ್ನು ಯಶಸ್ವಿಯಾಗಿ ದಾಟಲು ಕಲಿಸುತ್ತಿದ್ದಾನೆ. ಬಾಲಕ ಮುಳುಗುವಂತಾದಾಗ ಹಗ್ಗ ಮೇಲೆಳೆಯುವಂತೆ, ಕಷ್ಟಗಳ ಬಂದಾಗ ಭಗವಂತನೂ ಕೈಬಿಡನು. ಹೇಗೆ ಬಸ್ಸೊಂದರಲ್ಲಿ ಎಷ್ಟೇ \enginline{rush(}ರಷ್​) ಇದ್ದರೂ, ಡ್ರೈವರನಿಗಾಗಿ ಸ್ವಲ್ಪ ಜಾಗ ಇಟ್ಟಿರು ವರೋ ಹಾಗೆಯೇ, ದಿನ ನಿತ್ಯದ ಕೆಲಸಕಾರ್ಯ ಎಷ್ಟೇ ಇದ್ದರೂ, ಧ್ಯಾನ ಪ್ರಾರ್ಥನೆಗಳಿಗಾಗಿ ದಿನವೂ ಸ್ವಲ್ಪ ಸಮಯ ಮೀಸಲಾಗಿಡು. ಸ್ತೋತ್ರಪಾಠ ಮಾಡು. ಆಡಿಕೊಳ್ಳುವವರು ಈ ಪ್ರಪಂಚದಲ್ಲಿ ಇದ್ದೇ ಇರುವರು. ಆ ಬಗ್ಗೆ ಚಿಂತಿಸದಿರು. ಭಗವಂತನ ಚಿಂತನೆ ಮಾಡು, ಶುಭವಾಗುವುದು.


\section{ಹೇಗೆ ಪ್ರಾರ್ಥಿಸಬೇಕು}

ನಮಗೆ ಪರಮಾಪ್ತನಾದ ದೇವರೊಡನೆ ಹೃನ್ಮನಗಳನ್ನು ಬಯಲು ಮಾಡುತ್ತಾ ವ್ಯಾಕುಲತೆ ಯಿಂದ ಶರಣಾಗಿ ಬೇಡಿದರೆ ಆತನ ಕೃಪೆಯ ಕಾವು ತಟ್ಟುವುದು ನಮ್ಮ ಗಮನಕ್ಕೇ ಬರುತ್ತದೆ. ಆ ನಿಟ್ಟಿನಲ್ಲಿ ಭಕ್ತನ ಮೊರೆಯ ಪರಿ ಇಲ್ಲಿದೆ:

‘ಹೇ ತಾಯೇ, ನನಗೆ ಹೆಚ್ಚು ತಿಳಿವಳಿಕೆ ಇಲ್ಲ. ಅಜ್ಞಾನಿ ನಾನು. ಈ ಮಾಯಾ ಸಮುದ್ರದಲ್ಲಿ ಆಡುತ್ತ ನಾನು ಕಾಲ ಕಳೆದೆ. ನನಗೆ ಈಗೀಗ ನನ್ನ ತಪ್ಪಿನ ಅರಿವಾಗುತ್ತಲಿದೆ. ನೀನು ಎಲ್ಲೆಲ್ಲೂ ಇದ್ದೀಯೆ. ನನ್ನ ತಪ್ಪುಗಳನ್ನು ಕ್ಷಮಿಸಿ ಸರಿಯಾದ ದಾರಿಯಲ್ಲಿ ನೀನು ಮಾತ್ರ ಕರೆದೊಯ್ಯಬಲ್ಲೆ. ಎಲ್ಲ ತರದ ಕಷ್ಟ, ಸಂಕಟಗಳನ್ನು ಎದುರಿಸುವ ಶಕ್ತಿಯನ್ನೂ, ನೋವು, ಅವಸಾದಗಳನ್ನು ಸಹಿಸುವ ಸ್ಥೈರ್ಯವನ್ನೂ ದಯಪಾಲಿಸು. ನಾನು ನೇರವಾಗಿ ಮುಚ್ಚುಮರೆ ಇಲ್ಲದೆ ಹೇಳುತ್ತೇನೆ. ನನ್ನಿಂದ ಅನೇಕ ತಪ್ಪುಗಳಾಗಿವೆ. ಆದರೆ ಹೇ ತಾಯೇ, ದಯವಿಟ್ಟು ನನ್ನನ್ನು ಕ್ಷಮಿಸು, ನನ್ನಲ್ಲಿ ಆಧ್ಯಾತ್ಮಿಕ ಜಾಗೃತಿಯನ್ನುಂಟುಮಾಡು. ನಾನು ನಿನ್ನ ಕಂದ. ಈಗ ನಿನ್ನ ಮಡಿಲೇ ಬೇಕು ನನಗೆ. ಹೇ ತಾಯೇ, ಇದು ಸತ್ಯ. ನನಗೆ ನೀನೇ ಬೇಕು. ಈ ಸಂಸಾರದಲ್ಲಿರುವುದಕ್ಕೆ ಹಣದ ಅಗತ್ಯ ನನಗೇನೋ ಇದೆ. ಆದರೆ ಹೇ ತಾಯೇ. ನನಗೆ ಮುಂದೆ ಒಳ್ಳೆಯ ಜನ್ಮಗಳನ್ನು ಪಡೆಯುವು ದಕ್ಕಾಗಿ, ಸತ್ಸಂಸ್ಕಾರಗಳನ್ನು ಸಂಗ್ರಹಿಸುವುದಕ್ಕಾಗಿ ಆಧ್ಯಾತ್ಮಿಕ ಸಂಪತ್ತು ಬೇಕು. ಹೇ ತಾಯೇ, ಕತ್ತಲೆಯಿಂದ ಬೆಳಕಿನೆಡೆಗೆ ನನ್ನನ್ನು ಕರೆದುಕೊಂಡು ಹೋಗು. ನನ್ನ ಜೀವನವನ್ನು ಉನ್ನತಿ ಪಥ ದಲ್ಲಿ ಮುನ್ನಡೆಯಿಸು. ನಿನ್ನ ಮಾಹಾತ್ಮ್ಯವನ್ನೂ, ಲೀಲೆಯನ್ನೂ ಅರ್ಥಮಾಡಿಕೊಳ್ಳುವ ಬುದ್ಧಿ ಶಕ್ತಿಯನ್ನು ನನಗೆ ನೀಡು. ನಿನ್ನ ದಿವ್ಯ ಮಂಗಲರೂಪವನ್ನು ತೋರಿಸಿ. ಪವಿತ್ರತೆಯ ಪಥದಲ್ಲಿ ನಡೆಯಿಸು. ನಾನು ಸ್ಮರಿಸಿದಾಗಲೆಲ್ಲ ನನ್ನ ಬಳಿ ಸಾರಿ ನನ್ನನ್ನು ಕಾಪಾಡು. ಹೇ ಮಾತೆ! ಪ್ರೀತಿ, ವಾತ್ಸಲ್ಯದಿಂದ ತಿದ್ದಿ ಸರಿಪಡಿಸು. ಒಂದು ಆಧ್ಯಾತ್ಮಿಕ ಕ್ರಾಂತಿಯನ್ನೇ ನನ್ನ ಬದುಕಿನಲ್ಲಿ ಉಂಟು ಮಾಡು ತಾಯೇ! ನಾನು ಎಂದೆಂದಿಗೂ ನಿನ್ನವನು–ನಿನ್ನವನೇ! ನನ್ನ ಮೇಲೆ ಕೃಪೆದೋರು, ಕೈಹಿಡಿದು ಕಾಪಾಡು!’

ಈ ರೀತಿಯ ಪ್ರಾರ್ಥನೆಯ ಧಾಟಿಯಿಂದ ವ್ಯಾಕುಲತೆ ತೀವ್ರವಾಗಿ ಕಣ್ಣೀರಿನ ಕಟ್ಟೆ ಒಡೆಯು ತ್ತದೆ. ಆ ಕಣ್ಣೀರಿನ ಕಡಲಿನಲ್ಲಿ ತೇಲಿ ಬರುತ್ತದೆ ಕೃಪೆಯ ಹಡಗು. ಆ ಕೃಪೆಯಿಂದ ಆಂತರ್ಯದ ಆನಂದದೊಂದಿಗೆ ಬಾಹ್ಯ ಸುಖಸೌಕರ್ಯಗಳೆಲ್ಲ ಬಂದೊದಗುತ್ತವೆ, ಗುಲಾಬಿಹೂವಿನ ಸೊಬಗಿನೊಂದಿಗೆ ಪರಿಮಳವು ತಾನೇತಾನಾಗಿ ಬರುವಂತೆ.


\section{ದಾಸನಲಿ ವಾಸ}

ಆಧ್ಯಾತ್ಮಿಕ ಜೀವನದಲ್ಲಿ ನಿರತನಾದ ಸಾಧಕ ಪ್ರಾರ್ಥಿಸುವುದು ಕೇವಲ ಕಾಮ, ಕ್ರೋಧ, ಲೋಭ, ಮೋಹ, ಮದ, ಮಾತ್ಸರ್ಯಗಳೆಂಬ ಮನೋವಿಕಾರಗಳನ್ನು ಕುಗ್ಗಿಸಿಕೊಳ್ಳುವುದಕ್ಕಾಗಿ ಮತ್ತು ಆಧ್ಯಾತ್ಮಿಕವಾಗಿ ಪ್ರಗತಿ ಹೊಂದುವುದಕ್ಕಾಗಿ. ಅವನ್ನು ಬಿಟ್ಟು ನಶ್ವರವಾದ ಪ್ರಾಪಂಚಿಕ ವಸ್ತುಗಳನ್ನು ಸಾಧಕ ಎಂದೂ ಕೇಳಬಯಸುವುದಿಲ್ಲ. ಅವನಿಗೆ ಬೇಕಾಗಿರುವುದು ಒಂದೇ ಒಂದು: ಆಂತರಿಕ ಬೆಳಕು, ಶಾಂತಿ. ಸಾಧಕ ಬಯಸುವುದು ಭಗವಂತ ಬಂದು ತನ್ನ ಹೃದಯದಲ್ಲಿ ನೆಲೆಸಿ ತನ್ನನ್ನು ಮುನ್ನಡೆಸಬೇಕೆಂದು; ದಾಸನಲಿ ವಾಸಮಾಡಬೇಕೆಂದು. ಆ ನಿಟ್ಟಿನಲ್ಲಿ ತುಲಸೀದಾಸರ ಪ್ರಾರ್ಥನೆಯ ಸೊಲ್ಲಿದು:

ನಾ\enginline{s}ನ್ಯಾ ಸ್ಪೃಹಾ ರಘುಪತೇ ಹೃದಯೇಸ್ಮದಿಯೇ ಸತ್ಯಂ ವದಾಮಿ ಚ ಭವಾನ್ ಅಖಿಲಾಂತರಾತ್ಮಾ~। ಭಕ್ತಿಂ ಪ್ರಯಚ್ಛ ರಘುಪುಂಗವ ನಿರ್ಭರಾಂ ಮೇ ಕಾಮಾದಿದೋಷರಹಿತಂ ಕುರು ಮಾನಸಂ ಚ~॥

ಎಂದರೆ, ‘ಅಖಿಲಾಂತರಾತ್ಮನಾದ ರಘುಪತಿಯೇ, ನಿನ್ನಲ್ಲಿ ನಾನು ಸತ್ಯವನ್ನೇ ಹೇಳುತ್ತೇನೆ. ನನಗೆ ಬೇರಾವ ಬಯಕೆಗಳಿಲ್ಲ; ನಿನ್ನಲ್ಲಿ ನಿರ್ಭರವಾದ ಭಕ್ತಿಯನ್ನೂ, ಅನುರಕ್ತಿಯನ್ನೂ ಕೊಡು. ನನ್ನ ಮನಸ್ಸನ್ನು ಕಾಮಾದಿ ದೋಷಗಳಿಂದ ಬಿಡುಗಡೆ ಮಾಡು. ನನ್ನೆದೆಯ ಗುಡಿಯಲ್ಲಿ ವಾಸಮಾಡು.’


\section{ದಿವ್ಯ ಆಸ್ವಾದನೆ}

ಆ ರೀತಿಯ ದಾಸೋಹ ಕೈಗೊಂಡು ಆಧ್ಯಾತ್ಮಿಕ ಉನ್ನತಿಯನ್ನು ಸಾಧಿಸಬಹುದೆಂಬುದಕ್ಕೆ ಉಜ್ವಲ ಉದಾಹರಣೆ ಬ್ರದರ್ ಲಾರೆನ್ಸ್​ನದು.

ಲಾರೆನ್ಸ್ ಸಾಮಾನ್ಯರಲ್ಲಿ ಜನಿಸಿದವನು. ಸಾಮಾನ್ಯರಲ್ಲೇ ಬೆಳೆದವನು. ಕೆಲಸ, ಕಾರ್ಯಗಳಲ್ಲಿ ಅಂಥ ದಕ್ಷನೇನೂ ಅಲ್ಲ. ಸಾಮಾನ್ಯನೇ. ಆದರೆ ಆತನಲ್ಲಿ ಒಂದು ಅಸಾಮಾನ್ಯತೆ ಇತ್ತು. ಭಗವಂತನಲ್ಲಿ ಅವನಿಗೆ ಅಪಾರ ವಿಶ್ವಾಸ. ಈ ವಿಶ್ವಾಸ ಮತ್ತು ತೀವ್ರ ದೈವೀಪಿಪಾಸೆಗಳು ಅವನನ್ನು ಸಂತಶ್ರೇಷ್ಠನನ್ನಾಗಿ ಮಾಡಿದವು. ಅವನ ಹದಿನೆಂಟನೇ ವಯಸ್ಸಿನಲ್ಲಿ ಭಗವಂತನ ಮಹಿಮೆಯ ಉಜ್ವಲತೆ ಅವನ ಮನಸ್ಸನ್ನು ಬೆಳಗಿತು. ಆ ಘಟನೆಯಾದರೋ ತೀರ ಸಾಮಾನ್ಯ. ಚಳಿಗಾಲದಲ್ಲಿ ಎಲೆಗಳನ್ನು ಕಳೆದುಕೊಂಡು ಬೋಳಾಗಿ ನಿಂತ ಮರವನ್ನು ನೋಡುತ್ತ ಇನ್ನು ಕೆಲವೇ ದಿನಗಳಲ್ಲಿ ಆ ಮರವು ಭಗವದಿಚ್ಛೆಯಿಂದ ಚಿಗುರೊಡೆದು ಫಲಭರಿತವಾಗುವುದಲ್ಲ! ಎಂಬ ಯೋಚನೆ ಬಂದುದೇ ತಡ, ಅವನ ಮನಸ್ಸು ದೇವರ ಮಹಿಮೆಯೆಡೆಗೆ ಸರಿಯತೊಡಗಿತು. ದೇವರ ದಾರಿಯಲ್ಲಿ ಮುಂದುವರಿಯಲು ಅವನು ಮೊದಮೊದಲು ಗ್ರಂಥಗಳನ್ನೇನೋ ಓದಲು ಪ್ರಾರಂಭಿಸಿದ. ಪಾಂಡಿತ್ಯದ ಮೋಹವಿಲ್ಲದ ಅವನಿಗೆ ವಿವಿಧ ವಿಚಾರ ತರ್ಕ ಸಿದ್ಧಾಂತಗಳಿಂದ ತುಂಬಿದ ಧಾರ್ಮಿಕ ಗ್ರಂಥಗಳು ಸ್ಫೂರ್ತಿಯನ್ನು ಕೊಡಲಿಲ್ಲ. ಬದಲಾಗಿ ಅವು ಅವನ ಮನಸ್ಸಿನ ಗೊಂದಲ ಸಂಶಯಗಳನ್ನೇ ಹೆಚ್ಚಿಸಿದವು. ಪುಸ್ತಕಗಳನ್ನೆಲ್ಲ ದೂರವಿಟ್ಟು, ‘ಅವನ ಪ್ರೀತಿಗಾಗಿ, ಅವನನ್ನು ಬಿಟ್ಟು ಉಳಿದುದೆಲ್ಲವನ್ನೂ ತ್ಯಜಿಸಿ, ಜಗತ್ತಿನಲ್ಲಿ ಭಗವಂತ ಮತ್ತು ನಾನು ಇಬ್ಬರೇ ಇದ್ದೇವೆ ಎನ್ನುವಂತೆ ಜೀವನ ಮಾಡತೊಡಗಿದೆ’ ಎಂದಾತ ಹೇಳುತ್ತಾನೆ. ಇದು ಬಹಳ ಸುಲಭ ವೆಂದು ತೋರಬಹುದು. ಆದರೆ ಲಾರೆನ್ಸ್ ಎಚ್ಚರಿಕೆ ಕೊಡುತ್ತಾನೆ: ‘ನಿಮಗೆ ನಾನು ನಿಜವನ್ನು ಹೇಳುತ್ತೇನೆ. ಮೊದಲ ಹತ್ತು ವರ್ಷಗಳು ಬಹಳ ಕಷ್ಟಪಡಬೇಕಾಯಿತು. ಎಷ್ಟೋ ಬಾರಿ ಕೆಳಗೆ ಬಿದ್ದೆ. ತಿರುಗಿ ಮೈಕೊಡಹಿ ಮೇಲಕ್ಕೆದ್ದೆ, ನರಳಾಡಿದೆ. ಎಲ್ಲ ಜೀವಿಗಳೂ, ವಿಚಾರಶಕ್ತಿಯೂ, ಸ್ವಯಂ ಭಗವಂತನೇ, ನನಗೆ ವಿರೋಧವೋ ಎಂಬಂತೆ ಕಂಡು ಬಂದಿತು!’ ಆದರೆ ಲಾರೆನ್ಸ್ ದೃಢನಿಶ್ಚಯ ಮಾಡಿದ್ದ: ‘ಏನೇ ಬರಲಿ, ಎಂಥ ಕಠಿಣ ದಿನಗಳೇ ಜೀವನದಲ್ಲಿ ನನ್ನ ಪಾಲಿಗಿರಲಿ, ಭಗವಂತನ ಪ್ರೀತಿಗಾಗಿಯೇ ನಾನು ಎಲ್ಲ ಕೆಲಸಗಳನ್ನೂ ಮಾಡುತ್ತೇನೆ.’ ‘ಸ್ವಕರ್ಮಣಾ ತಮಭ್ಯರ್ಚ್ಯ....’ ಎನ್ನುವ ಗೀತೋಕ್ತಿಗೆ ಆತನ ಜೀವನ-ಸಾಧನೆ ಒಂದು ಜ್ವಲಂತ ಉದಾಹರಣೆ. ಭಗವಂತನಿಗಾಗಿ, ಆತನ ಪ್ರೀತಿಗಾಗಿ, ಮಾಡುವ ಎಲ್ಲ ಕೆಲಸಗಳಲ್ಲೂ ಆಸಕ್ತರಾಗಬೇಕು. ಕೆಲಸ ಯಾವುದೇ ಇರಲಿ, ಒಂದು ಹುಲ್ಲು ಕಡ್ಡಿಯನ್ನು ನೆಲದ ಮೇಲಿನಿಂದ ಎತ್ತುವುದೇ ಆಗಲಿ, ಎಲ್ಲವೂ ಅವನಿಗಾಗಿ. ದಿನ ದಿನವೂ ಎಲ್ಲ ಕೆಲಸಗಳನ್ನೂ ಭಗವಂತನಿಗಾಗಿ ಮಾಡುತ್ತಾ, ಆತನ ಮಹಿಮೆಯನ್ನು ನೆನೆನೆನೆದು ವ್ಯಾಕುಲತೆಯಿಂದ ಪ್ರಾರ್ಥಿಸುತ್ತಿದ್ದ. ‘ಈ ರೀತಿಯಾಗಿ ತೊಂದರೆ, ಆತಂಕಗಳಲ್ಲೇ ನನ್ನ ಜೀವನವು ಕಳೆದು ಹೋಗುವುದೆಂಬ ವ್ಯಥೆಯಲ್ಲಿದ್ದಾಗಲೇ, ಒಂದು ದಿನ ಥಟ್ಟನೇ ನನ್ನ ಆತ್ಮವು ವಿವರಿಸಲಾಗದ ಅಪೂರ್ವ ಆನಂದಪೂರಿತ ಆಂತರಿಕ ಶಾಂತಿಯನ್ನು ಅನುಭವಿಸತೊಡಗಿತು. ಅಂದಿನಿಂದ ಏಕಪ್ರಕಾರವಾಗಿ ನಾನು ವಿನಮ್ರನಾಗಿ ಪರಿಶುದ್ಧಪ್ರೇಮ, ಶ್ರದ್ಧೆಗಳಿಂದ ದೇವರ ಸಾನ್ನಿಧ್ಯದಲ್ಲೇ ಸದಾ ನಲಿದಾಡುತ್ತಿದ್ದೇನೆ’ ಎಂದಾತ ಭಗವಂತನ ಸಾನ್ನಿಧ್ಯ ಸುಖವನ್ನು ವರ್ಣಿಸಿದ್ದಾನೆ.

ನಿಕೊಲಸ್ ಹರ್ಮನ್ ಇದು ಲಾರೆನ್ಸ್​ನ ಮೊದಲ ಹೆಸರು. ಮೊದಲು ಅವನು ಸೇನೆಯಲ್ಲಿ ಕೆಲಸಕ್ಕೆ ಸೇರಿಕೊಂಡಿದ್ದ. ಕೆಲಸ ಕಾರ್ಯಗಳಲ್ಲಿ ದಕ್ಷನಲ್ಲದ ಆತ ಅಧಿಕಾರಿಗಳ ಕೆಂಗಣ್ಣಿಗೆ ಗುರಿ ಯಾಗಿದ್ದ. ಮನಸ್ಸಿಗೆ ಆ ಕೆಲಸ ಹಿಡಿಸದೆ ಕೊನೆಗೆ ಆತ ಆ ಕೆಲಸವನ್ನು ಬಿಟ್ಟು ಪ್ಯಾರಿಸ್ಸಿನ ಒಂದು ಕ್ರೈಸ್ತ ಸಂನ್ಯಾಸಿಮಠವನ್ನು ಸೇರಿಕೊಂಡ. ದೇವರೊಡನೆ ತನ್ನನ್ನು ಸರಿಪಡಿಸೆಂದು ಬೇಡಲಾಗುವು ದೆಂದು ಯೋಚಿಸಿ ಅವನು ಅಲ್ಲಿಗೆ ಬಂದಿದ್ದ. ದೈಹಿಕವಾಗಿ ಅವನು ಸ್ವಸ್ಥನಲ್ಲ, ಕುಂಟ. ಅವನ ಕೆಲಸವಾದರೋ ಪಾಕಶಾಲೆಯಲ್ಲಿ. ಅದೇನೂ ಅವನು ಮೆಚ್ಚಿಕೊಂಡಿದ್ದ ಕೆಲಸವಾಗಿರಲಿಲ್ಲ. ಅಡಿಗೆ ಮನೆಯ ಕೆಲಸದ ಒತ್ತಡ ಅವಸರ ಗೊಂದಲಗಳು ಅವನಿಗೆ ಹಿಡಿಸಿರಲಿಲ್ಲ. ಅವನ ಯೋಗ್ಯತೆಗೆ ಮತ್ತು ಶಕ್ತಿಗೆ ಮೀರಿದ ಕಾರ್ಯಗಳನ್ನೂ ಕೆಲವೊಮ್ಮೆ ಅಲ್ಲಿ ನಿರ್ವಹಿಸಬೇಕಾಗು ತ್ತಿತ್ತು. ಆದರೆ ಅವನು ಎಲ್ಲ ಕೆಲಸಗಳನ್ನೂ ದೇವರಿಗಾಗಿ ಮಾಡಿದ. ಒಂದೆರಡು ವರ್ಷಗಳಲ್ಲ, ಅರವತ್ತು ವರ್ಷಗಳ ಕಾಲ! ಎಲ್ಲ ಕೆಲಸಕಾರ್ಯಗಳ ಮೊದಲು ಶ್ರದ್ಧೆಯಿಂದ ಪ್ರಾರ್ಥಿಸುತ್ತಿದ್ದ: ‘ಹೇ ಭಗವಾನ್, ನೀನು ನನ್ನ ಅತ್ಯಂತ ಸಮೀಪದಲ್ಲಿದ್ದೀಯೆ. ನಿನ್ನ ಇಚ್ಛೆಗನುಗುಣವಾಗಿಯೇ ನಾನು ಈ ಎಲ್ಲ ಕೆಲಸಗಳನ್ನೂ ಮಾಡಬೇಕಾಗಿದೆ. ನಿನ್ನ ಸಹಾಯವಿಲ್ಲದೇ ಇವೆಲ್ಲ ಸಾಧ್ಯವಿಲ್ಲ. ತಂದೆ, ನಿನ್ನ ಸಾನ್ನಿಧ್ಯದಿಂದ ನನ್ನನ್ನು ವಿಚ್ಯುತಿಗೊಳಿಸಬೇಡ ಎಂಬುದೆ ನಿನ್ನಲ್ಲಿ ನನ್ನ ಬೇಡಿಕೆ.’ ಭಗವಂತನಿಂದ ಸಹಾಯವನ್ನು ಪಡೆಯುವ ರಹಸ್ಯವನ್ನೂ ಅವನು ಈ ಮಾತುಗಳಿಂದ ಹೇಳು ತ್ತಾನೆ: ‘ಭಗವಂತನೊಡನೆ ಏಕಾಂತಭಾವದಿಂದ ಸರಳವಾಗಿ ವ್ಯವಹರಿಸಬೇಕು. ನಿಷ್ಕಪಟಿಯಾಗಿ, ಸಹಜ ಸರಳ ಭಾಷೆಯಲ್ಲಿ ಆತನ ಹತ್ತಿರ ಮಾತನಾಡಬೇಕು. ಯಾವುದಾದರೊಂದು ಒಳ್ಳೆಯ ಕೆಲಸ ಮಾಡುವ ಅವಕಾಶ ಸಿಕ್ಕಿದಾಗ, ಅದು ಅತ್ಯುತ್ತಮ ರೀತಿಯಲ್ಲಿ ಕೊನೆಗೊಳ್ಳುವಂತೆ ಆತನನ್ನು ಪ್ರಾರ್ಥಿಸಿಕೊಳ್ಳಬೇಕು.’ ಲಾರೆನ್ಸ್ ಪ್ರಾರ್ಥನೆಯಿಂದ ಎಂಥ ಉನ್ನತ ಮಾನಸಿಕ ಸ್ಥಿತಿ ಯನ್ನು ಪಡೆದಿದ್ದನೆಂಬುದು ಅವನ ಮಾತಿನಿಂದಲೇ ತಿಳಿಯುತ್ತದೆ! ‘ನನಗಾದರೋ ಪ್ರಾರ್ಥನೆಯ ಸಮಯ ಮತ್ತು ಕೆಲಸಕಾರ್ಯಗಳ ಸಮಯ ಬೇರೆಯಾಗಿರಲಿಲ್ಲ. ಹಲವು ಮಂದಿ ಹಲವಾರು ವಸ್ತುಗಳನ್ನು ಏಕಕಾಲದಲ್ಲಿ ಕೇಳುತ್ತಿರುವಾಗ, ಪಾಕಶಾಲೆಯ ಸದ್ದು ಗದ್ದಲಗಳ ನಡುವೆ ಕೂಡ ತೀವ್ರ ಏಕಾಗ್ರಚಿತ್ತನಾಗಿ, ಏಕಾಂತದಲ್ಲಿ ಪ್ರಾರ್ಥಿಸುವಾಗಿನಷ್ಟೇ ಸಮಚಿತ್ತತೆ ಶಾಂತಿಗಳಿಂದ ಭಗವತ್ಸಾನ್ನಿಧ್ಯದ ಅನುಭವ ನನಗಾಗುತ್ತಿತ್ತು.’

ಲಾರೆನ್ಸ್ ಭಕ್ತನೂ ಅಹುದು, ಕರ್ಮಯೋಗಿಯೂ ಅಹುದು. ‘ಒಂದು ದೃಢನಿಶ್ಚಯಮಾಡು; ಹೃತ್ಪೂರ್ವಕ ತ್ಯಾಗವನ್ನು ಮಾಡು’ ಎಂದು ಆತ ಹೇಳಿದಾಗ ಎಲ್ಲವನ್ನೂ ಬಿಟ್ಟು ಕಾಡನ್ನು ಸೇರು ಎಂದು ಹೇಳುತ್ತಿಲ್ಲ. ದೇವರಿಗಾಗಿ ಎಲ್ಲ ಕೆಲಸಗಳನ್ನೂ ಮಾಡಬೇಕೆಂದೇ ಹೇಳುತ್ತಾನೆ. ‘ದೇವರಿ ಗಾಗಿ ನಾವು ಸಣ್ಣಪುಟ್ಟ ಕೆಲಸಗಳನ್ನು ಮಾಡಲೂ ಹಿಂಜರಿಯಬಾರದು. ಕೆಲಸದ ಮಹತ್ತ್ವವನ್ನು ದೇವರು ಕಾಣುವುದಿಲ್ಲ. ಆದರೆ ಅದನ್ನು ಪ್ರೀತಿಯಿಂದ ಮಾಡಿದನೆ ಎಂದು ಕಾಣುತ್ತಾನೆ. ನಮ್ಮ ಅಧ್ಯಾತ್ಮಿಕ ಪ್ರಗತಿಯು ಕೆಲಸದ ಬದಲಾವಣೆಯನ್ನಾಗಲಿ, ಸ್ಥಾನ ಬದಲಾವಣೆಯನ್ನಾಗಲಿ ಹೊಂದಿಕೊಂಡಿಲ್ಲ. ನಿಂತ ಸ್ಥಾನದಲ್ಲೇ ನಿಂತು ಎಲ್ಲ ಕೆಲಸಗಳನ್ನೂ ಭಗವಂತನ ಪ್ರೀತಿಗಾಗಿ ಮಾಡುವುದೇ ಸೂಕ್ತ ಉಪಾಯ.’

ಸತ್ಯಸಾಕ್ಷಾತ್ಕಾರದ ಅನುಭವವನ್ನು ಹೊಂದಿದ ವ್ಯಕ್ತಿಯಷ್ಟು ಅಧಿಕಾರವಾಣಿಯಿಂದ ಓದು ಬರಹಬಲ್ಲ ಪಂಡಿತನು ಮಾತನಾಡಲಾರ. ಭಗವಂತನ ದಾರಿಯಲ್ಲಿ ನಡೆದ ಸಜ್ಜನರ ವಾಣಿ ನೇರವಾಗಿ ಮನುಷ್ಯರ ಹೃದಯವನ್ನು ಪ್ರವೇಶಿಸುತ್ತದೆ. ‘ಹಲವಾರು ವರ್ಷಗಳವರೆಗೆ ತರ್ಕ ವಿತರ್ಕಗಳನ್ನು ಮಾಡುತ್ತ, ಬೆಳಕನ್ನು ಕಾಣದೇ ಕಾಲಕಳೆಯಬಹುದು. ಆದರೆ ಸರಳ, ವಿನಮ್ರ ಹೃದಯಿಗಳ ಅಂತರಾಳದಲ್ಲಿ ಭಗವಂತ ಕೃಪೆದೋರಿ ಅತ್ಯಂತ ಸೂಕ್ಷ್ಮವಾದ ಆಧ್ಯಾತ್ಮಿಕ ಸತ್ಯ ಗಳನ್ನೂ, ತನ್ನ ಮಹಿಮೆಯನ್ನೂ ಮಿಂಚಿನಂತೆ ಬೆಳಗುವನು.’ ಲಾರೆನ್ಸ್ ಅತ್ಯಂತ ವಿನಮ್ರನಾಗಿ, ಆದರೆ ದೃಢತೆ ಯಿಂದ ಹೇಳುತ್ತಾನೆ: ‘ಈಗ ನಾನು ನಂಬುವ ಪ್ರಶ್ನೆಯೇ ಇಲ್ಲ. ಕಾರಣ ನಾನೇ ಆತನನ್ನು ಸದಾ ಸರ್ವದಾ ಕಾಣುತ್ತಿದ್ದೇನೆ. ಆತನ ದಿವ್ಯ ಆಸ್ವಾದನೆಯನ್ನು ಅನುಭವಿಸುತ್ತಿದ್ದೇನೆ.’ ‘ಬಾಗಿಲನ್ನು ತಟ್ಟು, ನಿರಂತರ ಆ ದಿಸೆಯಲ್ಲಿ ಹೋರಾಡು, ಅವನು ಖಂಡಿತವಾಗಿಯೂ ಬಾಗಿ ಲನ್ನು ತೆರೆಯುವನೆಂದು ನಾನು ಹೇಳುತ್ತೇನೆ’ ಎಂದು ಸಾಧಕರನ್ನು ಅವನು ಹುರಿದುಂಬಿಸುತ್ತಾನೆ.

ಆತನ ಕಳಕಳಿಯ ಕರೆಯನ್ನು ಯಾರು ತಾನೇ ತಿರಸ್ಕರಿಸಬಲ್ಲರು?


\section{ಧ್ಯಾನಕ್ಕೆ ಪ್ರೇರಕ, ಪೂರಕ}

ತೈಲಧಾರೆಯಂತೆ ಮನಸ್ಸು ಏಕಮುಖವಾಗಿ ದೇವರೆಡೆಗೆ ಹರಿಯುವುದೇ ಧ್ಯಾನ. ಜಪದಲ್ಲಿ ಏಕನಿಷ್ಠೆ, ಏಕಾಗ್ರತೆಗಳು ಏಕೀಭವಿಸಿದಾಗ ಅದು ಧ್ಯಾನದ ಮಟ್ಟಕ್ಕೇರುತ್ತದೆ. ಹೃದಯವು ಧ್ಯಾನಕ್ಕೆ ಪ್ರಶಸ್ತವಾದ ಸ್ಥಾನ ಎಂಬುದು ಧ್ಯಾನಸಿದ್ಧರ ಅಭಿಮತ. ಜ್ಞಾನಮಾರ್ಗಾವಲಂಬಿಗಳು ಭ್ರೂಮಧ್ಯದಲ್ಲೂ, ಭಕ್ತರು ಹೃದಯದಲ್ಲೂ ಧ್ಯಾನಿಸುತ್ತಾರೆ ಎನ್ನುವುದುಂಟು. ಧ್ಯಾನಾಭ್ಯಾಸ ರತರಿಗೆ ಪ್ರಾರಂಭದಲ್ಲಿ ಹೃದಯವೇ ಪ್ರಶಸ್ತವಾದ ತಾಣ ಎಂಬುದರಲ್ಲಿ ಸಂಶಯವಿಲ್ಲ. ಆದರೆ ಆ ಹೃದಯ ಯಾವುದು?

ಸರಿಯಾಗಿ ಪರಿಶೀಲಿಸಿದರೆ ನಮಗೆಲ್ಲರಿಗೂ ತ್ರಿವಿಧ ಹೃದಯಗಳಿವೆ ಎಂಬ ಸಂಗತಿ ಗೊತ್ತಾ ಗುವುದು. ಶರೀರಾದ್ಯಂತ ರಕ್ತವನ್ನು ಪಂಪಿಸುವ ಕೆಲಸ ಮಾಡುತ್ತಿರುವ ‘ಲಬ್, ಡಬ್​’ ಎನ್ನುವ ಹೃದಯದ ಪರಿಚಯ ಎಲ್ಲರಿಗೂ ಇದೆ. ಇದು ತನ್ನ ಕೆಲಸವನ್ನು ನಿಷ್ಠೆಯಿಂದ ನೆರವೇರಿಸದೆ ನಮ್ಮ ಯಾವ ವ್ಯವಹಾರ ಚಟುವಟಿಕೆಗಳೂ ನಡೆಯಲಾರವು ಎಂಬುದು ದಿಟ. ‘ಹೃದಯಾಂತ ರಾಳದಲ್ಲಿ ಈ ಮಾತನ್ನು ಹೇಳುತ್ತೇನೆ, ಆತನಿಗೆ ಹೃದಯಶುದ್ಧಿ ಇದೆ, ನಿಮ್ಮ ಮಾತುಗಳು ಹೃದಯದಿಂದ ಬರುತ್ತವೆ; ಅವು ಬುದ್ಧಿಯ ಕಸರತ್ತಲ್ಲ’ ಎಂದೆಲ್ಲ ಹೇಳುವಾಗ ನಾವು ಉದ್ದೇಶಿಸಿ ರುವುದು ಭಾವನೆಗಳ ನೆಲೆಯಾದ ಹೃದಯವನ್ನು. ಇದು ಎರಡನೆಯ ಬಗೆಯದು. ಪ್ರೀತಿ, ಭಕ್ತಿ, ನಿಃಸ್ವಾರ್ಥತೆ, ಪರದುಃಖ ಕಾತರತೆ, ಸೇವಾಮನೋಭಾವ, ನಿರಹಂಕಾರ ಇವೆಲ್ಲ ಹೃದಯ ವಂತಿಕೆಯ ಲಕ್ಷಣಗಳು. ನಮಗಿರುವ ಇನ್ನೊಂದು ಹೃದಯವೇ ಆಧ್ಯಾತ್ಮಿಕ ಹೃದಯ. ಇದನ್ನು ಅನಾಹತ ಚಕ್ರ\footnote{ ಸ್ಥೂಲ, ಸೂಕ್ಷ್ಮ, ಕಾರಣ–ಈ ಮೂರು ಶರೀರಗಳ ಸಂಧಿಸ್ಥಾನವನ್ನು ತಂತ್ರಶಾಸ್ತ್ರದಲ್ಲಿ ಚಕ್ರ ಎಂದು ಕರೆಯುತ್ತಾರೆ.} ಎಂದೂ ಕರೆಯುತ್ತಾರೆ.

ಹೃದಯದಲ್ಲಿ ಧ್ಯಾನಮಾಡಿ ಎಂದಾಗ ಈ ಆಧ್ಯಾತ್ಮಿಕ ಹೃದಯದಲ್ಲಿ ಮನಸ್ಸನ್ನು ನಿಲ್ಲಿಸಿ ಎಂದರ್ಥ. ನಮ್ಮಲ್ಲಿ ಹೆಚ್ಚಿನವರ ಮಾನಸಿಕ ಶಕ್ತಿಯೆಲ್ಲ ದೈಹಿಕ, ಜೈವಿಕ ಬಯಕೆಗಳ ತೃಪ್ತಿಗಾಗಿ ಮತ್ತು ‘ಅಹಂ’ನ ರಕ್ಷಣೆಗಾಗಿಯೇ ವ್ಯಯವಾಗುತ್ತದೆ. ಆಧ್ಯಾತ್ಮಿಕ ಜಾಗೃತಿಯಾಗಬೇಕಾದರೆ ಮೂಲಾಧಾರದಲ್ಲಿ ಸುಪ್ತವಾಗಿರುವ ಕುಂಡಲಿನೀ ಶಕ್ತಿ ಜಾಗ್ರತವಾಗಿ, ಮನಸ್ಸು ಈ ಬಂಧನ ಗಳಿಂದ ಬಿಡಿಸಿಕೊಂಡು ಮೇಲೇರಬೇಕು. ಅಂದರೆ ನಾವು ಧ್ಯಾನ ಮಾಡುವಾಗ ಅನಾಹತ ಚಕ್ರಕ್ಕೆ ಮನಸ್ಸನ್ನೇರಿಸಬೇಕು. ಆದರೆ ಆ ಚಕ್ರದ ಸ್ಥಾನವನ್ನು ಸ್ಪಷ್ಟವಾಗಿ ತಿಳಿದುಕೊಳ್ಳದೆ ಮನಸ್ಸನ್ನಲ್ಲಿಗೆ ಏರಿಸುವುದಾದರೂ ಹೇಗೆ? ರಮಣಮಹರ್ಷಿಗಳೆನ್ನುವಂತೆ ‘ನಾನು’ವಿನ ಮೂಲವನ್ನು ಶೋಧಿ ಸುತ್ತಾ ಹೋದರೆ, ಈ ಚಕ್ರವನ್ನು ಅಥವಾ ಆಧ್ಯಾತ್ಮಿಕ ಹೃದಯವನ್ನು, ತಲುಪಬಹುದು.

ಆದರೆ ಆ ವಿಶ್ಲೇಷಣಾ ವಿಧಾನವಾದರೋ ಸುಲಭ ಸಾಧ್ಯವಲ್ಲ. ನಾನು ನಿಜವಾಗಿಯೂ ಯಾರು? ‘ನಾನು,’ ‘ನಾನು’ ಎಂದುಕೊಳ್ಳುತ್ತೇನಲ್ಲ ಇದು ಎಲ್ಲಿಂದ ಪ್ರಾರಂಭವಾಗುತ್ತದೆ? ಎಚ್ಚರದ ಅವಸ್ಥೆಯಲ್ಲಿರುವಾಗ ನಾನಾ ಜನರ ಅಥವಾ ವಿಚಾರಗಳ ಸಂಪರ್ಕಕ್ಕೆ ಬರುವ ನಾನು, ಗಾಢನಿದ್ರೆಯ ಸಮಯದಲ್ಲಿ ಎಲ್ಲಿಗೆ ಹೋಗುತ್ತೇನೆ? ಪುನಃ ಎಚ್ಚರವಾದಾಗ ಎಲ್ಲಿಂದ ಬರು ತ್ತೇನೆ?...ಹೀಗೆ ‘ನಾನು’ ಇರುವ ಆಗ ಮತ್ತು ಹುಟ್ಟನ್ನು ಕುರಿತು ವಿಶ್ಲೇಷಣೆ ಮಾಡುತ್ತ ಹೋದಂತೆ, ಮತ್ತೆ ಮುಂದುವರಿಯಲಾಗದ ಒಂದು ಹಂತಕ್ಕೆ ಬಂದು ನಿಲ್ಲುವಂತಾಗುತ್ತದೆ. ಅಲ್ಲಿಯೇ ಇರುವುದು ಈ ಆಧ್ಯಾತ್ಮಿಕ ಹೃದಯ. ಆದರೆ ಈ ತರದ ಶೋಧನೆಯ ತೀವ್ರತೆಯನ್ನು ದೀರ್ಘಕಾಲ ಉಳಿಸಿಕೊಳ್ಳುವುದು ಹೆಚ್ಚಿನವರಿಗೆ ಕಷ್ಟ. ಹಾಗಾಗಿ ಅದಕ್ಕಿಂತ ಸುಲಭ ಉಪಾಯ ಪ್ರಾರ್ಥನೆ.

ಪ್ರಾರ್ಥನೆಯು ಧ್ಯಾನಕ್ಕೆ ಪ್ರೇರಕವೂ ಹೌದು, ಪೋಷಕವೂ ಹೌದು. ಪ್ರಾರ್ಥನೆಯ ಅಭ್ಯಾಸ ದಿಂದ ಧ್ಯಾನ ಶಕ್ತಿಶಾಲಿಯಾಗುತ್ತದೆ. ಧ್ಯಾನಕ್ಕೆ ಕುಳಿತುಕೊಳ್ಳುವುದಕ್ಕೆ ಮೊದಲ ಹತ್ತು, ಹದಿನೈದು ನಿಮಿಷಗಳ ಕಾಲ ಒಂದೇ ಮನಸ್ಸಿನಿಂದ ಪ್ರಾರ್ಥಿಸಬೇಕು. ಕಾರ್ಮೋಡಗಳು ದಟ್ಟೈಸಿದಾಗ ಹನಿಯೊಡೆದು ಮಳೆ ಸುರಿಯುವಂತೆ, ಪ್ರಾರ್ಥನೆ ತೀವ್ರವಾದಾಗ ಮನಸ್ಸು ಮೇಲೇರಿ ಭಾವನಾ ತ್ಮಕ ಹೃದಯವನ್ನು ದಾಟಿ, ಆಧ್ಯಾತ್ಮಿಕ ಹೃದಯವನ್ನು ತಲುಪುವಾಗ ಕಣ್ತುಂಬಿ ಬಂದು ಕಂಬನಿ ಹರಿಯತೊಡಗುತ್ತದೆ. ಅದೇ ವ್ಯಾಕುಲತೆಯ ಉಗಮ. ಅದರಿಂದಲೇ ಧ್ಯಾನದ ಸಿದ್ಧಿ.


\section{ಹೃದಯ ಅರಳಲು, ಬುದ್ಧಿ ಬೆಳಗಲು}

ಪ್ರತಿಯೊಬ್ಬರಲ್ಲೂ ಸುಪ್ತವಾಗಿರುವ ಈ ಆಧ್ಯಾತ್ಮಿಕ ಶಕ್ತಿಕೇಂದ್ರವನ್ನು ಎಚ್ಚರಗೊಳಿಸಲು ಸಾಧ್ಯ. ಅದಕ್ಕೆ ಬೇಕಾಗಿರುವುದು ಮಾನಸಿಕ ಪಾವಿತ್ರ್ಯ. ಬ್ರಹ್ಮಚರ್ಯದಲ್ಲಿ ದೃಢಪ್ರತಿಷ್ಠೆ ಮತ್ತು ದೇವರಿಗಾಗಿ ತೀವ್ರ ಹಂಬಲ. ಪತಿ ಪತ್ನಿಯರ ಆದರ್ಶ ಈ ಮುಖವಾಗಿಯೇ ಇದ್ದರೆ ಗೃಹಸ್ಥ ರಿಗೂ ಇದು ಅಸಾಧ್ಯದ ಮಾತೇನಲ್ಲ. ದಿನವೂ ಕ್ರಮಬದ್ಧವಾಗಿ ತೀವ್ರ ಪ್ರಾರ್ಥನೆಯೊಂದಿಗೆ ಎರಡು ಮೂರು ಗಂಟೆಗಳ ಕಾಲವಾದರೂ, ಧ್ಯಾನ ಮಾಡುತ್ತಾ ಮುನ್ನಡೆದರೆ, ಫಲ ಸಿಗಲು ಕೆಲವು ವರ್ಷಗಳಾದರೂ ಬೇಕು. ಆ ಕ್ರಮವನ್ನು ಅನುಸರಿಸದೆ ಪುರಸೊತ್ತಿದ್ದಾಗ ಕ್ಷಣಕಾಲ ಕಣ್ಮುಚ್ಚಿ ಕೂತು, ‘ಬಕಧ್ಯಾನ’ ಮಾಡಿ, ‘ಅಯ್ಯೋ, ಧ್ಯಾನಮಾಡಿ ನೋಡಿದೆ, ಏನೂ ಆಗಲಿಲ್ಲ’ ಎನ್ನುವವ ರಿಗೆ ಕೊರತೆಯಿಲ್ಲ. ದಾರಿ ಸರಿಯಾದರೆ ಮಾತ್ರ ಗುರಿ ಸೇರಬಹುದೆಂಬುದನ್ನು ಅವರು ತಿಳಿದು ಕೊಳ್ಳಬೇಕು.

ಧ್ಯಾನ = ಏಕಾಗ್ರತೆ ಎಂದು ಸಮೀಕರಿಸಿ, ಆ ಬಗ್ಗೆ ವಿಶೇಷ ಪ್ರಚಾರಗಳು ಇಂದು ನಡೆಯು ತ್ತಿವೆ. ಆದರೆ ಇದು ಆಧ್ಯಾತ್ಮಿಕ ಸಾಧನೆ ಆಗಲಾರದು. ಧ್ಯಾನದಲ್ಲಿ ವ್ಯಾಕುಲತೆಯ ಪ್ರಾರ್ಥನೆ ಯಿಂದ ಪುಷ್ಟವಾದ ಮನಸ್ಸು ಅತ್ಯಂತ ಜಾಗರೂಕವಾಗಿದ್ದು, ಅತಿಹೆಚ್ಚಿನ ಚುರುಕುತನದಿಂದ ಕೂಡಿರುತ್ತದೆ. ಅದು ವಿಶ್ರಾಮ-ಶಿಥಿಲೀಕರಣ-ನಿದ್ರಾಸ್ಥಿತಿಯ ವಿಧಾನವಲ್ಲವೇ ಅಲ್ಲ. ದೇವರಿ ಗಾಗಿ ವ್ಯಾಕುಲತೆ, ಅಭೀಪ್ಸೆ ಅಥವಾ ಹಂಬಲಗಳಿಲ್ಲದ ಧ್ಯಾನ ವಿಧಾನ, ಪೆಟ್ರೋಲ್ ಇಲ್ಲದೆ ಕಾರನ್ನು ಓಡಿಸಲು ಮಾಡುವ ಪ್ರಯತ್ನದಂತೆಯೇ ಸರಿ. ದೇವರಿಗಾಗಿ ಹಂಬಲಿಸುತ್ತಾ ತೀವ್ರ ವ್ಯಾಕುಲತೆಯಿಂದ ಮುನ್ನಡೆದರೆ, ಸರೋವರದ ನೀರಿನಡಿ ಇದುವರೆಗೆ ಹುದುಗಿದ್ದ ಕಮಲ, ಭಾನುವಿನುದಯಕ್ಕೆ ತಲೆ ಎತ್ತಿ ಅರಳಿ ಸೌರಭ ಸೂಸುವಂತೆ, ಹೃದಯಕಮಲವೂ ಅರಳಿ, ಭಗವದಭಿಮುಖವಾಗಿ ದೈವೀಗುಣಗಳಿಂದ ಶೋಭಿಸತೊಡಗುವುದು. ಸಂಶಯರಹಿತವಾದ ಈ ದಿವ್ಯ ಅನುಭೂತಿಯಿಂದ ಸಾಧಕನಲ್ಲಿ ಅಪೂರ್ವ ಸ್ಥೈರ್ಯ, ಇನ್ನೂ ಮುನ್ನಡೆಯಲು ಸ್ಫೂರ್ತಿ ಹಾಗೂ ಮಾರ್ಗದರ್ಶನ ಲಭಿಸುವುದು. ಇದೇ ಬುದ್ಧಿಯ ಬೆಳಗುವಿಕೆ. ಗಾಯತ್ರೀ ಮಂತ್ರದ ಅಂತರಾರ್ಥವೂ ಇದೇ.


\section{ಅಂತಃಶಕ್ತಿಯ ಆಗರ}

ಪರಿವರ್ತನೆಯ ಪ್ರವರ್ತಕವಾಗಬಲ್ಲ ಅಂತಃಶಕ್ತಿಯ ಅಕ್ಷಯ ಆಗರವೇ ಪ್ರಾರ್ಥನೆಯಲ್ಲಿದೆ. ಗುಂಡಿ ಒತ್ತಿದೊಡನೆ ಚಿಮ್ಮುವ ಚಿಲುಮೆಯಂತೆ, ಪ್ರಾರ್ಥನೆಯ ಪರಿಪಾಠದಿಂದ ಆ ಶಕ್ತಿಮೂಲ ಕರಗತವಾಗುತ್ತದೆಂಬುದನ್ನು ತಿಳಿದು ಬಿತ್ತರಿಸಿದ ನೋಬೆಲ್ ಪ್ರಶಸ್ತಿ ವಿಜೇತ ವಿಜ್ಞಾನಿ, ವೈದ್ಯ ಡಾ. ಅಲೆಕ್ಸಿಸ್ ಕೆರೆಲರ ಮಾತೇ ಅದಕ್ಕೆ ಸಾಕ್ಷಿ:

‘ಅತಿಹೆಚ್ಚಿನ ಪ್ರಭಾವಶಾಲಿಯಾದ ಶಕ್ತಿಯನ್ನು ಪ್ರಾರ್ಥನೆಯಿಂದ ವ್ಯಕ್ತಿಯೊಬ್ಬ ಸೃಜಿಸಬಲ್ಲ. ಭೂಮ್ಯಾಕರ್ಷಣ ಶಕ್ತಿ ಎಷ್ಟು ಸತ್ಯವೋ, ಪ್ರಾರ್ಥನೆಯಿಂದ ಬಲಸಂವರ್ಧನೆ ಸಾಧ್ಯ ಎಂಬುದೂ ಅಷ್ಟೇ ಸತ್ಯ. ಇತರ ಎಲ್ಲ ತೆರನಾದ ಚಿಕಿತ್ಸಾವಿಧಾನಗಳು ಫಲಪ್ರದವಾಗದಿದ್ದಾಗ ಕೇವಲ ಶ್ರದ್ಧಾನ್ವಿತ ಹೃತ್ಪೂರ್ವಕ ಪ್ರಾರ್ಥನೆಯಿಂದ ರೋಗಕ್ಲೇಶಗಳಿಂದ ಮುಕ್ತರಾದ ವ್ಯಕ್ತಿಗಳನ್ನು ವೈದ್ಯನಾಗಿ ನಾನು ಕಂಡುಕೊಂಡಿದ್ದೇನೆ. ಈ ಪ್ರಾರ್ಥನಾ ವಿಧಾನ ರೇಡಿಯಂನಂತೆ ಅದ್ಭುತ ಶಕ್ತಿ ವಿಕಿರಣಶೀಲದ್ದು. ತಾನೇತಾನಾಗಿ ಶಕ್ತಿಯನ್ನು ಜಾಗ್ರತಗೊಳಿಸಿ ವೃದ್ಧಿಸುವಂಥದ್ದು. ನಾವು ಪ್ರಾರ್ಥಿಸುವಾಗಲೆಲ್ಲ ಈ ವಿಶ್ವಬ್ರಹ್ಮಾಂಡವನ್ನು ನಡೆಯಿಸುವ, ಎಂದೆಂದಿಗೂ ಬತ್ತಿ ಬರಿದಾ ಗದ, ದಿವ್ಯ ಶಕ್ತಿಯ ಸಂಪರ್ಕ ಲಭ್ಯವಾಗುತ್ತದೆ. ನಮ್ಮ ಕಳಕಳಿಯ ಬೇಡಿಕೆಯು ಎಷ್ಟೋ ವಿಧದ ನಮ್ಮ ದೌರ್ಬಲ್ಯಗಳನ್ನು ದೂರ ಮಾಡುತ್ತದೆ; ತನ್ಮೂಲಕ ನಾವು ಬಲಿಷ್ಠರಾಗಿ ತಲೆ ಎತ್ತುತ್ತೇವೆ. ಭಗವಂತನನ್ನು ಹೃತ್ಪೂರ್ವಕವಾಗಿ ಪ್ರಾರ್ಥಿಸಿದಾಗಲೆಲ್ಲ ನಮ್ಮ ತನುಮನಗಳಲ್ಲಿ ಶುಭ ಪರಿ ಣಾಮ ಆಗಿಯೇ ತೀರುತ್ತದೆ.’\footnote{\engfoot{Prayer is the most powerful form of energy one can generate. It is a force as real as terrestrial gravity. As a physician, I have seen men, after all other therapy had failed, lifted out of disease and melancholy by the serene effort of prayer... Prayer, like radium, is a source of luminous, self-generating energy.... When we pray we link ourselves with the inexhaustible motive force which spins the universe. Even in asking, our human deficiencies are filled and we arise strengthened and repaired... whenever we address God in fervent prayer, we change both soul and body for the better.}\hfill\engfoot{ –Dr. Alexis Carrel}}

ಶಕ್ತಿಯ ಬರವನ್ನೆದುರಿಸಲು ಬೇರೆ ಬೇರೆ ಶಕ್ತಿಮೂಲಗಳನ್ನು ಸೂರೆಗೈಯಲು ಹವಣಿಸು ತ್ತಿರುವ ವಿಜ್ಞಾನಿಗಳ ದೃಷ್ಟಿ ಈ ಕಡೆಗೆ ಹರಿದಿದ್ದಲ್ಲಿ ಮನುಕುಲವೇ ಉದ್ಧಾರವಾಗುತ್ತಿತ್ತು!


\section{ನನ್ನನಳಿಸು, ನಿನ್ನ ಮೆರೆಸು}

ವ್ಯಾವಹಾರಿಕ ಜಗತ್ತಿನಲ್ಲಿ ಬಗ್ಗಿದವನಿಗೆ ಗುದ್ದು ಜಾಸ್ತಿಯಾದರೆ, ಅಧ್ಯಾತ್ಮಲೋಕದಲ್ಲಿ ಬಗ್ಗಿದವನಿಗೆ ಮುದ್ದು ಜಾಸ್ತಿ; ಅಂದರೆ ದೇವರ ವಿಶೇಷ ಕೃಪೆ ಲಭ್ಯ. ಆದರೆ ನಮ್ಮ ಅಂತಸ್ತು, ಕೀರ್ತಿ, ಸ್ಥಾನಮಾನ, ಆಸ್ತಿಪಾಸ್ತಿಗಳಿಂದ ನಾವು ಬೆಳೆಸಿಕೊಂಡಂಥ ‘ಅಹಂ’ ಎನ್ನುವುದು, ನಮ್ಮನ್ನು ಮಣಿದು ಮೊರೆಯಲು ಬಿಡದೆ ಸೆಟೆದು ನಿಲ್ಲಿಸುತ್ತದೆ. ‘ನಾನೇ, ನನ್ನಿಂದ, ನನಗಾ ಗಿಯೇ’ ಎಂಬೆಲ್ಲ ಭ್ರಮೆಯಿಂದ ನಮ್ಮನ್ನದು ಮತ್ತನನ್ನಾಗಿಸಿ,ಮಾಯೆಯ ಮೋಡಿಗೆ ಬಲಿಬೀಳು ವಂತೆ ಮಾಡುತ್ತದೆ. ಈ ಅಹಂಕಾರದಿಂದಲೇ ಆಗುತ್ತದೆ ನಮ್ಮ ಮುನ್ನಡೆಗೆ ಸಂಚಕಾರ. ಮಾಯೆಯ ದಾರಿಯಲ್ಲಿ ಮುನ್ನಡೆಯುವವನಿಗೆ ದೇವರು ಎಂದಾದರೂ ದೊರೆತಾನೇ? ಸೂರ್ಯೋದಯ ವೀಕ್ಷಣೆಗಾಗಿ ಪಶ್ಚಿಮ ದಿಕ್ಕಿಗೆ ಹೆಜ್ಜೆ ಹಾಕಿದಂತಲ್ಲವೇ ಇದು? ಪರಮಹಂಸರು ಹೇಳುತ್ತಿದ್ದರು: ‘ಮಳೆಯ ನೀರು ಹೇಗೆ ದಿಣ್ಣೆಯ ಮೇಲೆ ಸಂಗ್ರಹವಾಗದೊ, ಹಾಗೆಯೇ ಅಹಂಕಾರದಿಂದ ಬೀಗಿದವರಿಗೆ ಕೃಪೆ ದುರ್ಲಭ’ ಎಂದು. ಆದ್ದರಿಂದ ಈ ‘ನಾನು’ ಹೋದರೆ ಮಾತ್ರ ಅಧ್ಯಾತ್ಮ ರಾಜ್ಯದಲ್ಲಿ ಮುನ್ನಡೆ ಸಾಧ್ಯ. ‘ನಾನು’ವಿನ ನಾಶಕ್ಕೆ ಪ್ರಾರ್ಥನೆಯೇ ಪ್ರಬಲ ಅಸ್ತ್ರ. ‘ನನ್ನನಳಿಸು, ನಿನ್ನ ಮೆರೆಸು! ಬಂದು ನೆಲೆಸು, ಹೃದಯಪದ್ಮದಲದಲಿ!’ ಎಂಬುದೇ ಸಾಧಕನ ತಾರಕಮಂತ್ರ. ಅದೇ ಪ್ರಾರ್ಥನೆಯ ಪರಾಕಾಷ್ಠೆ.

ಪ್ರಾರ್ಥನೆಯು ತೀವ್ರವಾದಂತೆ ಅದರ ವೇಗದಲ್ಲಿ ‘ನಾನು ನನ್ನದು’ ಮೆಲ್ಲನೇ ಲೀನವಾಗು ತ್ತವೆ. ಸ್ಥೂಲಸೂಕ್ಷ್ಮಕಾರಣಗಳನ್ನು ದಾಟಿ ಮನಸ್ಸು ಭಗವಂತನನ್ನೇ ಸ್ಪರ್ಶಿಸುತ್ತದೆ. ಹೃದಯ ದಲ್ಲೇ ಸರ್ವಶಕ್ತನಾದ ಆತನ ಸಾನ್ನಿಧ್ಯದ ಅರಿವು ನಮಗಾಗುತ್ತದೆ. ಸರ್ವಶಕ್ತನೇ ನಮ್ಮ ಹೃದಯಾಂತರಾಳದಲ್ಲಿ ನೆಲೆಸಿದ ಅರಿವು ನಮಗಾಗುತ್ತಿರುವಾಗ ನಮ್ಮನ್ನು ನಾವು ಅಶಕ್ತರೆಂದು ಕೊಳ್ಳಬಲ್ಲೆವೆ? ಭಯಭೀತರಾಗಬಲ್ಲೆವೆ?

