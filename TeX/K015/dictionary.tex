\sethyphenation{kannada}{
ಅಂಗ-ವಾ-ಗ-ಬೇಕು
ಅಂಗು-ಲ-ವಾ-ದರೂ
ಅಂಗು-ಲವೂ
ಅಂಜಿಕೆ
ಅಂಜಿ-ಕೆಯ
ಅಂಜು-ವು-ದಿಲ್ಲ
ಅಂತರ
ಅಂತ-ರಂ-ಗದ
ಅಂತ-ರಂ-ಗ-ದಲ್ಲಿ
ಅಂತ-ರಾ-ತ್ಮ-ನಲ್ಲಿ
ಅಂತ-ರಾ-ತ್ಮ-ನಿಂದ
ಅಂತ-ರಾ-ತ್ಮ-ವನ್ನು
ಅಂತ-ರಾಳ
ಅಂತ-ರಾ-ಳಕ್ಕೆ
ಅಂತ-ರಾ-ಳ-ದಲ್ಲಿ
ಅಂತ-ರ್ಗತ
ಅಂತ-ರ್ಗ-ತ-ವಾ-ಗಿ-ರುವ
ಅಂತ-ರ್ಜ್ಯೋತಿ
ಅಂತಹ
ಅಂತ-ಹ-ವರ
ಅಂತ-ಹ-ವ-ರನ್ನು
ಅಂದರೆ
ಅಂದಿನ
ಅಗ-ಣಿತ
ಅಗ-ತೆ-ಮಾಡಿ
ಅಗೋ-ಚರ
ಅಗ್ರ
ಅಜ್ಞಾನ-ದಲ್ಲಿ
ಅಜ್ಞಾನ-ದಿಂದ
ಅಜ್ಞಾನಿ
ಅಜ್ಞಾನಿ-ಗ-ಳ-ನ್ನಾಗಿ
ಅಡಗಿ
ಅಡಗಿತ್ತು
ಅಡಗಿದೆ
ಅಡಗಿ-ರುವ
ಅಡಗಿ-ಸ-ಬ-ಹುದು
ಅಡಗಿ-ಸ-ಬೇ-ಕಾ-ದರೆ
ಅಡಗಿ-ಸು-ವು-ದಕ್ಕೆ
ಅಡಿ-ಗೆ-ಮಾ-ಡು-ವುದು
ಅಡಿ-ಗೆ-ಯನ್ನು
ಅಡಿ-ಗೆ-ಯ-ವನು
ಅಡಿ-ಯಾ-ಳಾಗು
ಅಣಿ-ಮಾ-ಡಿ-ರು-ವೆನು
ಅತಿ
ಅತಿ-ಯಾದ
ಅತ್ಯಂತ
ಅತ್ಯ-ಗ-ತ್ಯ-ವಾದ
ಅತ್ಯ-ದ್ಭುತ
ಅತ್ಯ-ದ್ಭು-ತ-ವಾದ
ಅತ್ಯ-ಮೋಘ
ಅತ್ಯಲ್ಪ
ಅತ್ಯಾ-ವ-ಶ್ಯಕ
ಅತ್ಯಾ-ವ-ಶ್ಯ-ಕ-ವಾಗಿ
ಅತ್ಯಾ-ವ-ಶ್ಯ-ಕವೊ
ಅತ್ಯು-ತ್ತಮ
ಅತ್ಯು-ತ್ತ-ಮ-ವಾದ
ಅಥವಾ
ಅದ
ಅದ-ಕ್ಕಿಂತ
ಅದಕ್ಕೆ
ಅದ-ಕ್ಕೋ-ಸ್ಕ-ರವೆ
ಅದನ್ನು
ಅದರ
ಅದ-ರಂ-ತೆಯೆ
ಅದ-ರಂ-ತೆಯೇ
ಅದ-ರಲ್ಲಿ
ಅದ-ರಿಂದ
ಅದ-ಲ್ಲದೆ
ಅದು
ಅದೃ-ಷ್ಟಕ್ಕೆ
ಅದೃ-ಷ್ಟ-ವಿ-ದ್ದರೆ
ಅದೃ-ಷ್ಟ-ವಿ-ಲ್ಲದೆ
ಅದೆಲ್ಲ
ಅದೇ
ಅದೇನೂ
ಅದೇನೋ
ಅದೊಂದೆ
ಅದ್ಭುತ
ಅದ್ಭು-ತ-ವಾದ
ಅದ್ಯೈವ
ಅಧಃ-ಪ-ತನ
ಅಧಃ-ಪಾ-ತ-ಳಕ್ಕೆ
ಅಧ-ರ್ಮಿ-ಗ-ಳಾ-ಗಿ-ರು-ವರು
ಅಧಿ-ಕಾರ
ಅಧಿ-ಕಾ-ರದ
ಅಧಿ-ಕಾ-ರ-ವಿದೆ
ಅಧೀ-ನಕ್ಕೆ
ಅಧೋ-ಗ-ತಿಗೆ
ಅಧ್ಯ-ಕ್ಷರು
ಅಧ್ಯ-ಯ-ನಕ್ಕೆ
ಅಧ್ಯ-ಯ-ನ-ಮಾ-ಡ-ಬೇಕು
ಅಧ್ಯಾ-ಪಕ
ಅನಂತ
ಅನಂ-ತ-ಜ್ಞಾ-ನ-ನಿ-ಧಿ-ಯಾದ
ಅನಂ-ತರ
ಅನಂ-ತ-ಶಕ್ತಿ
ಅನಂ-ತ-ಸಾ-ಗರ
ಅನ-ರ್ಹ-ರ-ನ್ನಾಗಿ
ಅನ-ವ-ರ-ತವೂ
ಅನಾ-ದ-ರ-ಣೀಯ
ಅನಾ-ದಿ-ಯಿಂ-ದಲೂ
ಅನಾ-ವ-ರ-ಣ-ಕ್ರ-ಮ-ದಿಂದ
ಅನಾ-ವ-ರ-ಣ-ಗೊ-ಳಿ-ಸು-ತ್ತಾನೆ
ಅನಾ-ಸ-ಕ್ತಿ-ಯನ್ನು
ಅನು
ಅನು-ಕಂಪ
ಅನು-ಕಂ-ಪ-ವನ್ನು
ಅನು-ಗು-ಣ-ವಾಗಿ
ಅನು-ದಿ-ನದ
ಅನು-ಪ-ಮಳು
ಅನು-ಭ-ವಿಸಿ
ಅನು-ಭ-ವಿ-ಸುತ್ತ
ಅನು-ಭ-ವಿ-ಸು-ತ್ತಿ-ರು-ವರು
ಅನು-ಭ-ವಿ-ಸು-ವ-ವ-ರಗೆ
ಅನು-ಭ-ವಿ-ಸೋಣ
ಅನು-ರ-ಣಿತ
ಅನು-ರ-ಣಿ-ತ-ವಾ-ಗ-ಬೇಕು
ಅನು-ವಾ-ಗಿ-ರುವ
ಅನು-ಷ್ಠಾನ
ಅನು-ಷ್ಠಾ-ನಕ್ಕೆ
ಅನು-ಷ್ಠಾ-ನ-ಯೋ-ಗ್ಯ-ವಾದ
ಅನು-ಸ-ರಿಸಿ
ಅನು-ಸ-ರಿಸು
ಅನು-ಸ-ರಿ-ಸು-ವರು
ಅನು-ಸ-ರಿ-ಸು-ವು-ದಕ್ಕೆ
ಅನು-ಸಾ-ರ-ವಾಗಿ
ಅನೇಕ
ಅನ್ನಕ್ಕೆ
ಅನ್ನ-ವಿಟ್ಟು
ಅಪ-ರೂಪ
ಅಪ-ಹ-ರಿ-ಸಲು
ಅಪಾಯ
ಅಪ್ರ
ಅಪ್ರ-ತಿಮ
ಅಪ್ರ-ಯೋ-ಜ-ಕನೇ
ಅಪ್ರ-ಯೋ-ಜ-ಕ-ರಾ-ಗ-ಬ-ಲ್ಲೆವು
ಅಭಾವ
ಅಭಿಃ
ಅಭಿ-ಪ್ರಾ-ಯ-ಪ-ಟ್ಟರೆ
ಅಭಿ-ಪ್ರಾ-ಯ-ವಲ್ಲ
ಅಭಿ-ವೃದ್ಧಿ
ಅಭಿ-ವೃ-ದ್ಧಿಗೆ
ಅಭಿ-ವೃ-ದ್ಧಿ-ಪ-ಡಿ-ಸಿ-ದರು
ಅಭ್ಯ-ಸಿ-ಸ-ಬೇಕು
ಅಭ್ಯಾಸ
ಅಭ್ಯಾ-ಸದ
ಅಭ್ಯಾ-ಸ-ದಿಂದ
ಅಭ್ಯಾ-ಸ-ವನ್ನು
ಅಭ್ಯಾ-ಸ-ವಾ-ದರೆ
ಅಭ್ಯಾ-ಸವೇ
ಅಮೇ
ಅಮೇ-ರಿಕಾ
ಅಮೋ-ಘ-ವಾದ
ಅಯ್ಯೋ
ಅರ-ಗಿಳಿ
ಅರ-ಣ್ಯ-ಗಳಲ್ಲಿ
ಅರ-ಣ್ಯ-ದಲ್ಲಿ
ಅರಿ-ತು-ಕೊ-ಳ್ಳು-ವಿರಿ
ಅರಿವು
ಅರಿವೆ
ಅರ್ಥ
ಅರ್ಥ-ಜ್ಞಾನ
ಅರ್ಥ-ಮಾ-ಡಿ-ಕೊಂ-ಡರು
ಅರ್ಥ-ಮಾ-ಡಿ-ಕೊಂಡು
ಅರ್ಥ-ವನ್ನು
ಅರ್ಪಿ-ಸಲು
ಅಲೆ-ಕ್ಸಾಂ-ಡ-ರನ
ಅಲೆಯ
ಅಲೆ-ಯಾ-ಗಲಿ
ಅಲೆ-ಯಾ-ಗಿರ
ಅಲ್ಪಾ-ತ್ಮ-ನನ್ನು
ಅಲ್ಲ
ಅಲ್ಲ-ಗ-ಳೆ-ದಿ-ರು-ವ-ರೊ-ಅಲ್ಲಿ
ಅಲ್ಲ-ಗ-ಳೆ-ಯ-ಬ-ಲ್ಲಿರಾ
ಅಲ್ಲದೆ
ಅಲ್ಲವೆ
ಅಲ್ಲಿ
ಅಲ್ಲಿಗೆ
ಅಲ್ಲಿತ್ತು
ಅಲ್ಲಿದೆ
ಅಲ್ಲಿ-ಯ-ವ-ರೆಗೂ
ಅಲ್ಲೇ
ಅಳು-ತ್ತೇವೆ
ಅಳು-ವು-ದಕ್ಕೆ
ಅವ-ಕಾಶ
ಅವ-ಕಾ-ಶ-ಕೊ-ಡ-ಬೇಡಿ
ಅವ-ಕಾ-ಶ-ವನ್ನು
ಅವ-ಕಾ-ಶ-ವಿ-ರ-ಬೇಕು
ಅವ-ಕಾ-ಶ-ವಿ-ಲ್ಲ-ದ-ವ-ರಿಗೆ
ಅವ-ಕಾ-ಶವೇ
ಅವನ
ಅವ-ನ-ತಿಗೆ
ಅವ-ನ-ತಿಯ
ಅವ-ನನ್ನು
ಅವ-ನಲ್ಲಿ
ಅವ-ನಿಗೆ
ಅವ-ನಿ-ಗೋ-ಸ್ಕರ
ಅವನು
ಅವನೆ
ಅವನ್ನು
ಅವರ
ಅವ-ರನ್ನು
ಅವ-ರಲ್ಲಿ
ಅವ-ರ-ಲ್ಲಿದ್ದ
ಅವ-ರ-ಲ್ಲಿ-ರುವ
ಅವ-ರಿಂದ
ಅವ-ರಿಗೂ
ಅವ-ರಿಗೆ
ಅವರು
ಅವರೇ
ಅವ-ರೊಂ-ದಿಗೆ
ಅವಳು
ಅವ-ಶ್ಯ-ವಾ-ಗಿದೆ
ಅವ-ಶ್ಯ-ವಾ-ಗಿ-ರು-ವುದು
ಅವ-ಶ್ಯ-ವಿದೆ
ಅವ-ಶ್ಯ-ವಿ-ದ್ದರೆ
ಅವಿ-ತಂ-ತಿ-ರುವ
ಅವಿ-ತು-ಕೊಂ-ಡಿ-ದೆಯೋ
ಅವಿ-ತು-ಕೊಂ-ಡಿ-ರುವ
ಅವು
ಅವು-ಗ-ಳಂತೆ
ಅವು-ಗಳನ್ನು
ಅವು-ಗ-ಳಿ-ಗಾಗಿ
ಅವು-ಗಳು
ಅವು-ಗ-ಳೆಲ್ಲ
ಅವು-ಗ-ಳೊಂ-ದಿಗೆ
ಅವೆ-ಲ್ಲವೂ
ಅಶ-ಕ್ತಿಯೇ
ಅಷ್ಟು
ಅಷ್ಟೆ
ಅಷ್ಟೇ
ಅಸ-ಮಾ-ಧಾ-ನ-ವಾ-ಗಿ-ದೆಯೆ
ಅಸ-ಮಾ-ನ-ವಾದ
ಅಸ-ಹಾ-ಯತೆ
ಅಸಾ-ಧ್ಯ-ವಾದ
ಅಸೂ-ಯೆ-ಗಳ
ಅಸ್ತಿ-ತ್ವಕ್ಕೆ
ಅಹ-ಮಸ್ಮಿ
ಅಹ-ಲ್ಯ-ಬಾಯಿ
ಆ
ಆಂಗ್ಲ-ಭಾಷೆ
ಆಕೆಯ
ಆಗ
ಆಗ-ಬೇ-ಕೆಂದು
ಆಗಲಿ
ಆಗಲು
ಆಗಲೆ
ಆಗಲೇ
ಆಗಿರ
ಆಗಿ-ರು-ವಿರಿ
ಆಗಿಲ್ಲ
ಆಗು
ಆಗು-ತ್ತಾನೆ
ಆಗು-ತ್ತಾರೆ
ಆಗು-ತ್ತಿರ
ಆಗು-ತ್ತಿ-ವೆಯೊ
ಆಗು-ವಂ-ತೆಯೂ
ಆಗು-ವರು
ಆಗು-ವು-ದಿಲ್ಲ
ಆಗು-ವುದು
ಆಗು-ವುವು
ಆಚ-ರ-ಣೆಗೆ
ಆಜನ್ಮ
ಆಜ್ಞೆ-ಯಂತೆ
ಆಡಿ
ಆಡಿ-ದ-ವ-ರಲ್ಲ
ಆಡು-ವಾಗ
ಆಣ-ತಿ-ಯನ್ನು
ಆತ
ಆತಂ-ಕ-ಗಳನ್ನು
ಆತಂ-ಕ-ವನ್ನು
ಆತಂ-ಕ-ವಾ-ದು-ದನ್ನು
ಆತಂ-ಕವೇ
ಆತನ
ಆತ್ಮ
ಆತ್ಮದ
ಆತ್ಮನ
ಆತ್ಮ-ನಲ್ಲಿ
ಆತ್ಮ-ನೆಂದು
ಆತ್ಮ-ರ-ಕ್ಷಣೆ
ಆತ್ಮ-ವೆಂದು
ಆತ್ಮ-ಶಕ್ತಿ
ಆತ್ಮ-ಶ-ಕ್ತಿ-ಯನ್ನು
ಆತ್ಮ-ಶ-ಕ್ತಿ-ಯನ್ನೂ
ಆತ್ಮ-ಶ್ರದ್ಧೆ
ಆತ್ಮ-ಶ್ರ-ದ್ಧೆಯ
ಆತ್ಮ-ಶ್ರ-ದ್ಧೆ-ಯನ್ನು
ಆತ್ಮೋ-ದ್ಧಾ-ರ-ಕ-ವಾದ
ಆತ್ಮೋ-ದ್ಧಾ-ರ-ವನ್ನು
ಆದ
ಆದ-ಕಾ-ರಣ
ಆದ-ಕಾರಣವೇ
ಆದರೂ
ಆದರೆ
ಆದರ್ಶ
ಆದ-ರ್ಶ-ಗಳ
ಆದ-ರ್ಶ-ಗಳನ್ನು
ಆದ-ರ್ಶ-ಗಳು
ಆದ-ರ್ಶದ
ಆದ-ರ್ಶ-ದಿಂದ
ಆದ-ರ್ಶ-ವ-ನ್ನಾಗಿ
ಆದ-ರ್ಶ-ವನ್ನು
ಆದ-ರ್ಶ-ವಾಗಿ
ಆದ-ರ್ಶ-ವೆಲ್ಲ
ಆದ-ರ್ಶ-ಶೀ-ಲ-ರನ್ನು
ಆದಿ-ಶ-ಕ್ತಿಯ
ಆದು-ದಲ್ಲ
ಆಧ-ನಿ-ಕ-ರಂತೆ
ಆಧು-ನಿಕ
ಆಧ್ಯಾ-ತ್ಮಿಕ
ಆಧ್ಯಾ-ತ್ಮಿ-ಕ-ವಾ-ಗಲಿ
ಆಪಾ-ದ-ಮ-ಸ್ತ-ಕವೂ
ಆರಾ-ಧನೆ
ಆರು
ಆರ್ಯಾ-ವರ್ತ
ಆಲದ
ಆಲೋ
ಆಲೋ-ಚನಾ
ಆಲೋ-ಚ-ನಾ-ಶ-ಕ್ತಿಯ
ಆಲೋ-ಚನೆ
ಆಲೋ-ಚ-ನೆ-ಗಳ
ಆಲೋ-ಚ-ನೆ-ಗಳನ್ನು
ಆಲೋ-ಚ-ನೆ-ಗಳು
ಆಲೋ-ಚ-ನೆ-ಯನ್ನು
ಆಲೋ-ಚ-ನೆ-ಯನ್ನೇ
ಆಲೋ-ಚ-ನೆ-ಯಲ್ಲಿ
ಆಲೋ-ಚಿ-ಸ-ಬೇಕು
ಆಲೋ-ಚಿಸಿ
ಆಲೋ-ಚಿ-ಸಿ-ದಂತೆ
ಆಲೋ-ಚಿ-ಸಿ-ದರೆ
ಆಲೋ-ಚಿ-ಸು-ತ್ತಾರೋ
ಆಲೋ-ಚಿ-ಸು-ತ್ತಿ-ರು-ವನೋ
ಆಲೋ-ಚಿ-ಸು-ತ್ತಿ-ರು-ವಿರಿ
ಆಳಕ್ಕೆ
ಆವ-ರಿಸಿ
ಆವ-ಶ್ಯಕ
ಆವ-ಶ್ಯ-ಕತೆ
ಆವ-ಶ್ಯ-ಕ-ತೆ-ಗಳು
ಆವ-ಶ್ಯ-ಕ-ತೆಗೆ
ಆವ-ಶ್ಯ-ಕ-ತೆ-ಯನ್ನು
ಆವ-ಶ್ಯ-ಕ-ವಾದ
ಆವಿ-ರ್ಭಾ-ವ-ನೆ-ಯನ್ನು
ಆವಿ-ಷ್ಕಾರ
ಆವಿ-ಷ್ಕಾ-ರ-ಗಳು
ಆಶ್ರಮ
ಆಸೆ
ಆಸೆ-ಗಳನ್ನು
ಆಸೆ-ಯನ್ನು
ಆಸ್ತಿ
ಆಸ್ತಿ-ಯಾ-ಗ-ಬೇಕು
ಆಹಾ-ರ-ದಿಂದ
ಆಹಾ-ರ-ವನ್ನು
ಇಂತಹ
ಇಂತ-ಹುದು
ಇಂದಿನ
ಇಂದಿ-ನ-ವರೆ-ವಿಗೂ
ಇಂದು
ಇಂದ್ರಿ-ಯ-ಗ್ರ-ಹಣ
ಇಂದ್ರಿ-ಯ-ವನ್ನು
ಇಚ್ಛಾ
ಇಚ್ಛಾ-ಶಕ್ತಿ
ಇಚ್ಛಾ-ಶ-ಕ್ತಿ-ಯನ್ನು
ಇಚ್ಛಾ-ಶ-ಕ್ತಿಯು
ಇಚ್ಛಿ-ಸು-ತ್ತೇ-ವೆಯೋ
ಇಚ್ಛೆ-ಗಳ
ಇಚ್ಛೆ-ಯಂತೆ
ಇಚ್ಛೆ-ಯನ್ನು
ಇಚ್ಛೆ-ಯಿ-ಲ್ಲದೆ
ಇಚ್ಛೆಯು
ಇಚ್ಛೆಯೆ
ಇಟ್ಟ
ಇಟ್ಟು
ಇಟ್ಟು-ಕೊಂಡು
ಇಟ್ಟು-ಕೊ-ಳ್ಳ-ಬ-ಹುದು
ಇಟ್ಟು-ಕೊ-ಳ್ಳ-ಬೇ-ಕಾ-ಗಿತ್ತು
ಇಟ್ಟು-ಕೊ-ಳ್ಳೋಣ
ಇಡ-ಬೇಕು
ಇಡಿ
ಇಡೀ
ಇಡುವ
ಇತರ
ಇತ-ರರೂ
ಇತಿ-ಹಾಸ
ಇತಿ-ಹಾ-ಸ-ದಲ್ಲೆಲ್ಲ
ಇತ್ತು
ಇತ್ಯಾ-ದಿ-ಗಳಲ್ಲಿ
ಇತ್ಯಾ-ದಿ-ಯಾಗಿ
ಇದ-ಕ್ಕಿಂತ
ಇದಕ್ಕೆ
ಇದನ್ನು
ಇದ-ನ್ನೆಲ್ಲ
ಇದರ
ಇದ-ರಂ-ತೆಯೆ
ಇದ-ರಂ-ತೆಯೇ
ಇದ-ರಲ್ಲಿ
ಇದ-ರ-ಲ್ಲಿ-ರುವ
ಇದ-ರಿಂದ
ಇದಾ-ದ-ಮೇಲೆ
ಇದು
ಇದು-ವರೆ-ವಿಗೂ
ಇದೆ
ಇದೆಯೆ
ಇದೇ
ಇದೊಂದು
ಇದೊಂದೆ
ಇದೊಂದೇ
ಇದ್ದ
ಇದ್ದದ್ದು
ಇದ್ದರು
ಇದ್ದರೂ
ಇದ್ದರೆ
ಇದ್ದಿ-ದ್ದರೆ
ಇನ್ನು
ಇನ್ನೂ
ಇನ್ನೇನು
ಇನ್ನೊ-ಬ್ಬ-ನನ್ನು
ಇನ್ನೊ-ಬ್ಬ-ರಿಗೆ
ಇಪ್ಪತ್ತು
ಇಬ್ಬರು
ಇಬ್ಬರೂ
ಇರ-ಬ-ಲ್ಲಿರಾ
ಇರ-ಬ-ಹುದು
ಇರ-ಬೇಕು
ಇರಲಿ
ಇರ-ಲಿಲ್ಲ
ಇರು
ಇರು-ತ್ತವೆ
ಇರುಳು
ಇರುವ
ಇರು-ವರು
ಇರು-ವ-ವ-ನನ್ನೂ
ಇರು-ವ-ವ-ರನ್ನು
ಇರು-ವ-ವ-ರಿಗೆ
ಇರು-ವಷ್ಟೇ
ಇರು-ವಿರಾ
ಇರು-ವು-ದಕ್ಕೆ
ಇರು-ವುದನ್ನು
ಇರು-ವುದು
ಇರು-ವು-ದೆಂದು
ಇರು-ವು-ದೆಲ್ಲ
ಇರು-ವುವು
ಇರುವೆ
ಇರು-ವೆವು
ಇಲ್ಲ
ಇಲ್ಲದ
ಇಲ್ಲದೆ
ಇಲ್ಲದೇ
ಇಲ್ಲವೆ
ಇಲ್ಲಿ
ಇಲ್ಲಿದೆ
ಇಲ್ಲಿ-ರು-ವಳು
ಇಳಿ-ದಿದೆ
ಇಳಿದು
ಇವನ್ನು
ಇವ-ರಲ್ಲಿ
ಇವು
ಇವು-ಗಳ
ಇವು-ಗಳನ್ನು
ಇವು-ಗಳನ್ನೆಲ್ಲ
ಇವು-ಗ-ಳೊಂ-ದಿಗೆ
ಇವೆ
ಇವೇ
ಇಷ್ಟು
ಇಷ್ಟೇ
ಇಷ್ಟೊಂದು
ಈ
ಈಗ
ಈಗಿನ
ಈಗಿ-ರುವ
ಈಡೇ-ರ-ಬೇಕಾ
ಈಡೇ-ರಿ-ಸಿ-ಕೊ-ಳ್ಳ-ಬ-ಲ್ಲ-ವ-ರಾ-ಗ-ಬೇಕು
ಈಶ್ವರ
ಈಶ್ವ-ರ-ನಿಂದೆ
ಉಂಟಾ-ದರೆ
ಉಂಟು-ಮಾ-ಡ-ಬ-ಹುದು
ಉಂಟು-ಮಾ-ಡಿ-ಕೊ-ಳ್ಳು-ತ್ತದೆ
ಉಂಟು-ಮಾ-ಡು-ತ್ತದೆ
ಉಕ್ಕಿ-ನಂ-ತಹ
ಉಗ್ರ-ವಾಗಿ
ಉಚಿ-ತ-ವಾಗಿ
ಉಚ್ಚ-ರಿ-ಸಲಿ
ಉಚ್ಚ-ರಿ-ಸುವ
ಉಟ್ಟು
ಉಡುಪು
ಉತ್ತಮ
ಉತ್ತ-ಮ-ಗೊಂಡ
ಉತ್ತ-ಮ-ಗೊ-ಳಿ-ಸ-ಬೇಕು
ಉತ್ತ-ಮ-ಗೊ-ಳಿ-ಸುವ
ಉತ್ತ-ಮ-ಗೊ-ಳಿ-ಸು-ವುದು
ಉತ್ತ-ಮ-ರಾ-ಗಿಯೆ
ಉತ್ತ-ಮ-ರಾ-ಗು-ವರು
ಉತ್ತ-ಮ-ವಾಗಿ
ಉತ್ತ-ಮ-ವಾದ
ಉತ್ತ-ಮ-ವಾ-ದು-ದನ್ನು
ಉತ್ತ-ಮೋ-ತ್ತ-ಮ-ನನ್ನೂ
ಉತ್ತ-ರವೇ
ಉತ್ಪತ್ತಿ
ಉತ್ಸಾಹ
ಉತ್ಸಾ-ಹ-ಪೂ-ರಿ-ತ-ರ-ನ್ನಾಗಿ
ಉದ-ಯಿ-ಸುವ
ಉದಾತ್ತ
ಉದಾ-ರ-ಹೃ-ದ-ಯ-ರಾ-ಗಿ-ದ್ದರು
ಉದಾ-ಹ-ರ-ಣೆ-ಯೊಂದು
ಉದಿ-ಸಿ-ದಾಗ
ಉದ್ದೀ-ಪ-ನ-ಗೊ-ಳಿಸ
ಉದ್ದೀ-ಪ-ನ-ಗೊ-ಳಿ-ಸುವ
ಉದ್ದೇಶ
ಉದ್ದೇ-ಶ-ದಿಂದ
ಉದ್ದೇ-ಶ-ವನ್ನು
ಉದ್ಧಾರ
ಉದ್ಯ-ಮ-ದಲ್ಲಿ
ಉನ್ಮ-ತ್ತ-ರಾಗಿ
ಉಪ-ದೇಶ
ಉಪ-ದೇ-ಶ-ಗಳು
ಉಪ-ದೇ-ಶದ
ಉಪ-ದೇ-ಶ-ಮಾಡಿ
ಉಪ-ದೇ-ಶಿ-ಸಿ-ದನು
ಉಪ-ನಿ-ಷ-ತ್ತಿನ
ಉಪ-ನಿ-ಷತ್ತು
ಉಪ-ನಿ-ಷ-ತ್ತು-ಗಳನ್ನು
ಉಪ-ನ್ಯಾ-ಸಕ
ಉಪ-ನ್ಯಾ-ಸ-ಕ-ನಾ-ಗಿ-ರು-ವನು
ಉಪ-ಯು-ಕ್ತ-ತೆ-ಯನ್ನು
ಉಪ-ಯೋ-ಗಿ-ಸ-ಬೇಕು
ಉಪ-ಯೋ-ಗಿಸಿ
ಉಪ-ಯೋ-ಗಿ-ಸಿ-ಕೊ-ಳ್ಳು-ವುದು
ಉಪ-ಯೋ-ಗಿಸು
ಉಪ-ಯೋ-ಗಿ-ಸು-ವು-ದಿಲ್ಲ
ಉಪ-ವಾ-ಸ-ದಿಂದ
ಉಪೇಕ್ಷೆ
ಉರಿ-ಯು-ತ್ತಿದ್ದ
ಉಳಿದ
ಉಳಿ-ದ-ವ-ರಲ್ಲ
ಉಳಿ-ದ-ವ-ರಿ-ಗಿಂತ
ಉಳಿ-ದ-ವರು
ಉಳಿ-ದು-ದ-ಕ್ಕೆಲ್ಲ
ಉಳಿ-ದುದು
ಊಟ
ಊಟ-ಮಾ-ಡು-ವಾಗ
ಊರಿಗೆ
ಊರಿ-ನಿಂದ
ಊರು-ಗಳಲ್ಲಿ
ಎಂಟು-ಭಾ-ಗ-ದಷ್ಟು
ಎಂತಹ
ಎಂತ-ಹ-ವ-ನ-ನ್ನಾ-ದರೂ
ಎಂದರೆ
ಎಂದಿಗೂ
ಎಂದಿ-ರ-ಬೇಕು
ಎಂದು
ಎಂದೂ
ಎಂದೆಂ-ದಿಗೂ
ಎಂದೋ
ಎಂಬ
ಎಂಬು
ಎಂಬು-ದನ್ನು
ಎಂಬು-ದ-ರಲ್ಲಿ
ಎಂಬುದು
ಎಕ-ರೆ-ಗ-ಳಷ್ಟು
ಎಡೆ-ಬಿ-ಡದೆ
ಎತ್ತ-ರ-ದಷ್ಟು
ಎತ್ತಿ
ಎದು-ರಿಗೆ
ಎದು-ರಿ-ಸ-ಬೇಕು
ಎದುರು
ಎದೆ
ಎದೆ-ಗಾ-ರಿಕೆ
ಎದೆ-ಗೆ-ಡ-ಬೇಡಿ
ಎದೆ-ಹಾ-ಲಿ-ನೊಂ-ದಿಗೆ
ಎನಿ-ಸಿದ
ಎನ್ನ-ಬೇಡಿ
ಎನ್ನು-ತ್ತಾರೆ
ಎನ್ನು-ತ್ತೇವೆ
ಎನ್ನುವ
ಎನ್ನು-ವನು
ಎನ್ನು-ವು-ದಕ್ಕೆ
ಎನ್ನು-ವುದನ್ನು
ಎನ್ನು-ವುದು
ಎನ್ನು-ವೆವೊ
ಎಬ್ಬಿಸಿ
ಎರ-ಡ-ನೆ-ಯ-ದಾ-ಗಿ-ರ-ಬೇಕು
ಎರ-ಡನೇ
ಎರ-ಡೆನೇ
ಎಲ್ಲ
ಎಲ್ಲ-ಕ್ಕಿಂತ
ಎಲ್ಲ-ರನ್ನೂ
ಎಲ್ಲ-ರ-ಲ್ಲಿಯೂ
ಎಲ್ಲ-ರಿಗೂ
ಎಲ್ಲರೂ
ಎಲ್ಲ-ವನ್ನೂ
ಎಲ್ಲವೂ
ಎಲ್ಲಾ
ಎಲ್ಲಿ
ಎಲ್ಲಿಂದ
ಎಲ್ಲಿ-ಯ-ವ-ರೆಗೂ
ಎಲ್ಲಿ-ಯಾ-ದರೂ
ಎಲ್ಲಿ-ರಲಿ
ಎಲ್ಲೋ
ಎಷ್ಟು
ಎಷ್ಟೋ
ಏಕ-ಕಾ-ಲ-ದ-ಲ್ಲಿಯೇ
ಏಕಾ-ಕಿ-ಯಾ-ಗಿ-ರಲಿ
ಏಕಾಗ್ರ
ಏಕಾ-ಗ್ರ-ಗೊ-ಳಿಸಿ
ಏಕಾ-ಗ್ರ-ಗೊ-ಳಿ-ಸು-ವನೊ
ಏಕಾ-ಗ್ರ-ಚಿ-ತ್ತ-ರಾದ
ಏಕಾ-ಗ್ರತೆ
ಏಕಾ-ಗ್ರ-ತೆಗೆ
ಏಕಾ-ಗ್ರ-ತೆ-ಗೊ-ಳಿಸಿ
ಏಕಾ-ಗ್ರ-ತೆಯ
ಏಕಾ-ಗ್ರ-ತೆ-ಯನ್ನು
ಏಕಾ-ಗ್ರ-ತೆ-ಯಲ್ಲಿ
ಏಕಾ-ಗ್ರ-ತೆ-ಯಿಂದ
ಏಕಾ-ಗ್ರ-ಮಾ-ಡಲು
ಏಕಾ-ಗ್ರ-ಮಾಡಿ
ಏಕೆಂ-ದರೆ
ಏತಕ್ಕೆ
ಏನನ್ನು
ಏನನ್ನೂ
ಏನನ್ನೋ
ಏನಾ-ಗಿರು
ಏನಾ-ಗಿ-ರು-ವನೋ
ಏನಾ-ಗು-ತ್ತಿದೆ
ಏನಾ-ದರೂ
ಏನು
ಏನೂ
ಏನೆಂ-ದರೆ
ಏನೆಂ-ಬು-ದನ್ನು
ಏರಿಸು
ಏಳಿ
ಏಳಿ-ಗೆಯ
ಏಳು
ಏಳು-ತ್ತಿ-ರುವ
ಏಳುವ
ಐದು
ಐರಿಷ್
ಐರ್ಲೆಂ-ಡಿ-ನ-ವನು
ಐರ್ಲೆಂ-ಡಿ-ನಿಂದ
ಐವತ್ತು
ಐಶ್ವ-ರ್ಯ-ಕ್ಕಿಂತ
ಒಂದ-ಕ್ಕೊಂದು
ಒಂದನೇ
ಒಂದ-ರಿಂ-ದಲೆ
ಒಂದು
ಒಂದು-ಗೂ-ಡಿಸಿ
ಒಂದೆ
ಒಂದೇ
ಒಂದೊಂದು
ಒಗ್ಗ-ಟ್ಟಾ-ಗ-ಲಾ-ರೆವು
ಒಟ್ಟಿಗೆ
ಒಟ್ಟು
ಒಟ್ಟು-ಗೂ-ಡಿ-ಸಿ-ದರೆ
ಒಟ್ಟು-ಸ್ವ-ಭಾವ
ಒಣ-ಗಿ-ಸ-ಲಾ-ರದು
ಒಣ-ಪಾಂ-ಡಿ-ತ್ಯದ
ಒಣ-ಮೂಳೆ
ಒತ್ತಾ-ಯ-ಪ-ಡಿ-ಸು-ತ್ತದೆ
ಒತ್ತಿ
ಒದ-ಗಿ-ಸ-ಬ-ಹುದು
ಒದ-ಗಿ-ಸಿ-ಕೊ-ಡ-ಲಾ-ರದೋ
ಒದ-ಗಿ-ಸು-ತ್ತದೆ
ಒಪ್ಪು-ವಂ-ತಹ
ಒಪ್ಪೊ-ತ್ತಿನ
ಒಬ್ಬ
ಒಬ್ಬನ
ಒಬ್ಬ-ನನ್ನು
ಒಬ್ಬ-ನಿಗೂ
ಒಬ್ಬ-ನಿಗೆ
ಒಬ್ಬನು
ಒಬ್ಬಳು
ಒಯ್ಯು-ತ್ತದೆ
ಒರೆ-ಯಿಂದ
ಒಳ
ಒಳ-ಗಿದೆ
ಒಳ-ಗಿ-ನಿಂದ
ಒಳ-ಪಟ್ಟ
ಒಳ್ಳೆಯ
ಒಳ್ಳೆ-ಯ-ದನ್ನು
ಒಳ್ಳೆ-ಯ-ದಾಗು
ಒಳ್ಳೆ-ಯ-ದಾ-ಯಿತು
ಒಳ್ಳೆ-ಯದು
ಓ
ಓಂ
ಓಡಾ-ಡು-ವು-ದಲ್ಲ
ಓದ-ಬ-ಹುದು
ಓದಿ
ಓದಿ-ದಾಗ
ಓದು-ತ್ತಾರೆ
ಓದು-ತ್ತಿ-ರು-ವೆನು
ಓದು-ವಂತೆ
ಓದು-ವು-ದಕ್ಕೆ
ಓದು-ವುದನ್ನು
ಓದು-ವುದೇ
ಔದ್ಯೋ-ಗಿಕ
ಕಂಠ-ಪಾ-ಠ-ಮಾಡಿ
ಕಂಠ-ಪಾ-ಠ-ಮಾ-ಡಿದ
ಕಂಡಾಗ
ಕಂಡು-ಬಂದ
ಕಂಡು-ಹಿ-ಡಿ-ದದ್ದು
ಕಂಡು-ಹಿ-ಡಿ-ದನು
ಕಂಡು-ಹಿ-ಡಿ-ದಿರು
ಕಂಡು-ಹಿ-ಡಿ-ದಿ-ರು-ವಿರಾ
ಕಂಡು-ಹಿ-ಡಿ-ಯು-ತ್ತಾನೆ
ಕಂಪ-ವನ್ನು
ಕಚೇ-ರಿ-ಗಳಲ್ಲಿ
ಕಟ್ಟಿ
ಕಟ್ಟಿ-ಗೆಯ
ಕಟ್ಟು-ಕ-ಥೆ-ಗಳು
ಕಟ್ಟು-ತ್ತಿ-ದ್ದನು
ಕಠಿ-ನ-ವಾದ
ಕಠೋರ
ಕಡ-ಲನ್ನು
ಕಡ-ಲಿನ
ಕಡಿಮೆ
ಕಡಿ-ಮೆ-ಯ-ಲ್ಲಿದೆ
ಕಡು
ಕಡೆ
ಕಡೆಗೆ
ಕಡೆ-ಯ-ಲ್ಲಿಯೂ
ಕಡ್ಡಿ-ಯನ್ನು
ಕಣ್ಣನ್ನು
ಕಣ್ಣ-ಮೇಲೆ
ಕಣ್ಣಿಗೆ
ಕಣ್ಣಿ-ನಲ್ಲಿ
ಕಣ್ಣೀ-ರನ್ನೂ
ಕಣ್ಣೀರು
ಕಣ್ಣು
ಕಣ್ಣು-ಗಳನ್ನು
ಕತ್ತ-ರಿ-ಸ-ಬೇ-ಕಾ-ಯಿತು
ಕತ್ತ-ರಿ-ಸ-ಲಾ-ರದು
ಕತ್ತಲೆ
ಕತ್ತ-ಲೆ-ಯಿಂದ
ಕತ್ತಿ-ಯನ್ನೂ
ಕತ್ತೆ-ಯನ್ನು
ಕಥೆ-ಗಳನ್ನು
ಕನಿ-ಕ-ರ-ವನ್ನು
ಕನಿ-ಷ್ಟ-ರಾ-ಗಿ-ರು-ವ-ರೆಂ-ಬು-ದನ್ನು
ಕನ್ನ-ಡಿಯ
ಕನ್ನ-ಡಿ-ಯಷ್ಟು
ಕಪಿ-ಮು-ಷ್ಟಿ-ಯಲ್ಲಿ
ಕಬ್ಬಿ-ಣದ
ಕಬ್ಬಿ-ಣ-ದಂ-ತಹ
ಕರ-ಗು-ವುದೋ
ಕರು-ಣಾಳು
ಕರು-ಣೆ-ಯನ್ನು
ಕರೆ-ದನು
ಕರೆದು
ಕರೆ-ದೊ-ಯ್ಯು-ವುದು
ಕರೆ-ಯ-ಲ್ಪ-ಡುವ
ಕರೆ-ಯಿತು
ಕರೆ-ಯು-ತ್ತಿದ್ದೆ
ಕರೆ-ಯು-ತ್ತೇನೆ
ಕರೆ-ಯು-ವುದು
ಕರೆ-ಯು-ವೆವೊ
ಕರೆ-ಯೊಂದೆ
ಕರೆವ
ಕರ್ತವ್ಯ
ಕರ್ಮ-ಗಳನ್ನು
ಕರ್ಮ-ಫಲ
ಕರ್ಮ-ಬ-ಲೆ-ಯನ್ನು
ಕರ್ಮ-ವೆಲ್ಲ
ಕಲಿ
ಕಲಿ-ತಿ-ರು-ವೆವು
ಕಲಿ-ತು-ಕೊ-ಳ್ಳ-ಬೇ-ಕಾ-ಗಿದೆ
ಕಲಿ-ತು-ಕೊ-ಳ್ಳ-ಲಿಲ್ಲ
ಕಲಿ-ತು-ಕೊ-ಳ್ಳು-ವುದು
ಕಲಿ-ಯ-ಬ-ಹುದು
ಕಲಿ-ಯು-ತ್ತಾರೆ
ಕಲಿ-ಯು-ವಂತೆ
ಕಲಿ-ಯು-ವು-ದಿಲ್ಲ
ಕಲಿ-ಸ-ಬೇಕು
ಕಲಿ-ಸ-ಲಾ-ರೆವು
ಕಲಿ-ಸ-ಹೋ-ಗು-ವರು
ಕಲಿ-ಸಿತು
ಕಲಿ-ಸಿ-ತು-ಎಂದು
ಕಲಿ-ಸು-ತ್ತಿ-ರು-ವೆನು
ಕಲಿ-ಸು-ತ್ತಿ-ರು-ವೆವು
ಕಲೆ
ಕಲೆ-ಗಳು
ಕಲೆತು
ಕಲೆ-ಯಲ್ಲಿ
ಕಳೆ-ದು-ಕೊಂಡ
ಕಳೆ-ಯದೆ
ಕಳೆ-ಯಿತು
ಕವಚ
ಕಷ್ಟ
ಕಷ್ಟ-ಕಾ-ಲಕ್ಕೆ
ಕಷ್ಟ-ಗಳನ್ನು
ಕಷ್ಟ-ಗಳನ್ನೆಲ್ಲ
ಕಷ್ಟ-ಗಳು
ಕಷ್ಟ-ದ-ಲ್ಲಿ-ರು-ವೆ-ನೆಂದೂ
ಕಷ್ಟ-ದಿಂದ
ಕಷ್ಟ-ಪ-ಟ್ಟಿ-ರು-ವರು
ಕಷ್ಟ-ಪ್ರಾ-ಪ್ತ-ವಾ-ದರೂ
ಕಷ್ಟ-ವನ್ನು
ಕಷ್ಟ-ವಾ-ಗಿದೆ
ಕಷ್ಟ-ವಿ-ರ-ಬ-ಹುದು
ಕಸಿದು
ಕಸು-ಬ-ನ್ನಾ-ದರೂ
ಕಹ-ಳೆ-ಗಳು
ಕಹ-ಳೆ-ಯನ್ನು
ಕಾಡಿಗೆ
ಕಾಡು-ತ್ತಿ-ರುವ
ಕಾಣದ
ಕಾಣ-ಬ-ಹುದು
ಕಾಣಿ-ಸ-ದಂತೆ
ಕಾಣು-ತ್ತದೆ
ಕಾಣು-ತ್ತಾ-ರೆಯೋ
ಕಾಣು-ವಂತೆ
ಕಾಣು-ವರು
ಕಾಣು-ವು-ದಿ-ಲ್ಲವೋ
ಕಾಣು-ವುದೆ
ಕಾತ-ರಿ-ಸ-ಬೇಕು
ಕಾದು
ಕಾಪಾ-ಡುವ
ಕಾಮ-ನಿ-ಗ್ರ-ಹವೇ
ಕಾಯೇನ
ಕಾರಣ
ಕಾರ-ಣವೆ
ಕಾರ-ಣ-ವೇನು
ಕಾರಿ-ಗ-ಳಾ-ಗು-ವುದು
ಕಾರ್ಖಾ-ನೆ-ಗ-ಳ-ನ್ನಾಗಿ
ಕಾರ್ಯ
ಕಾರ್ಯ-ಕ್ಷೇ-ತ್ರ-ಗಳನ್ನು
ಕಾರ್ಯ-ಗಳನ್ನು
ಕಾರ್ಯ-ರಂ-ಗ-ದಲ್ಲಿ
ಕಾರ್ಯ-ರೂ-ಪಕ್ಕೆ
ಕಾರ್ಯೋ-ನ್ಮು-ಖ-ರಾ-ಗ-ಬೇ-ಕೆಂ-ಬು-ದನ್ನು
ಕಾಲ
ಕಾಲ-ಕಾ-ಲಕ್ಕೆ
ಕಾಲಕ್ಕೆ
ಕಾಲದ
ಕಾಲ-ದಲ್ಲಿ
ಕಾಲ-ಮೇಲೆ
ಕಾಲು
ಕಾಳಿ-ನಲ್ಲಿ
ಕಾವ್ಯ
ಕಾವ್ಯ-ದಲ್ಲಿ
ಕಿಡಿ-ಗಳು
ಕಿರ-ಣವೂ
ಕಿರ-ಣ-ವೊಂದು
ಕಿವಿ
ಕೀರ್ತನೆ
ಕೀರ್ತಿ
ಕೀರ್ತಿಯ
ಕೀರ್ತಿ-ಯನ್ನು
ಕುಂದು-ಗಳಲ್ಲಿ
ಕುಂದೂ
ಕುಗ್ಗದೆ
ಕುಣಿ-ದಾ-ಡು-ವುದು
ಕುತೂ
ಕುತೂ-ಹ-ಲ-ವನ್ನು
ಕುದು-ರೆ-ಯಾ-ಗು-ವುದು
ಕುರಿ-ತದ್ದು
ಕುರಿತು
ಕುರು-ಡಾ-ಗುವ
ಕುಳಿ-ತರೆ
ಕುಳಿ-ತಿ-ರುವ
ಕುಳಿ-ತು-ಕೊಂ-ಡಿ-ರ-ಬ-ಹುದು
ಕುಳಿ-ತು-ಕೊ-ಳ್ಳುವ
ಕುಳಿ-ತು-ಕೊ-ಳ್ಳು-ವೆನು
ಕೂಡ
ಕೂಡಾ
ಕೂಡಿ-ಡು-ವು-ದಕ್ಕೆ
ಕೂಡಿದ
ಕೂಡಿ-ದ-ವ-ರಾಗಿ
ಕೂಡಿ-ದ್ದರೆ
ಕೂಡಿ-ದ್ದಾ-ಗಿ-ರ-ಬೇಕು
ಕೂಡಿ-ರು-ವುದು
ಕೂಳಿ-ಗಾಗಿ
ಕೃತ-ಕ-ಗೋಳ
ಕೃಷಿ
ಕೆಚ್ಚು
ಕೆಚ್ಚೆ-ದೆಯ
ಕೆಟ್ಟ
ಕೆಟ್ಟದ್ದು
ಕೆಡು-ವುದು
ಕೆಲ-ವ-ರನ್ನು
ಕೆಲ-ವ-ರಿಗೆ
ಕೆಲ-ವರು
ಕೆಲವು
ಕೆಲ-ವು-ವೇಳೆ
ಕೆಲಸ
ಕೆಲ-ಸಕ್ಕೂ
ಕೆಲ-ಸಕ್ಕೆ
ಕೆಲ-ಸ-ಗಳನ್ನು
ಕೆಲ-ಸದ
ಕೆಲ-ಸ-ದಲ್ಲಿ
ಕೆಲ-ಸ-ದಷ್ಟೆ
ಕೆಲ-ಸ-ಮಾ-ಡದೆ
ಕೆಲ-ಸ-ಮಾಡು
ಕೆಲ-ಸ-ಮಾ-ಡುವ
ಕೆಲ-ಸ-ವನ್ನು
ಕೆಲ-ಸವೂ
ಕೆಳಗಿ
ಕೆಳಗೆ
ಕೇಂದ್ರ
ಕೇಂದ್ರ-ಗಳಲ್ಲಿ
ಕೇಂದ್ರ-ವಾ-ಗ-ಬೇಕು
ಕೇಂದ್ರೀ-ಕೃ-ತ-ಮಾಡಿ
ಕೇಡು
ಕೇಳ-ಬ-ಹುದು
ಕೇಳ-ಬೇಕು
ಕೇಳ-ಬೇಡಿ
ಕೇಳಿ
ಕೇಳಿದ
ಕೇಳು-ತ್ತಾರೆ
ಕೇಳು-ತ್ತಿ-ದ್ದರೆ
ಕೇಳು-ವಂತೆ
ಕೇಳು-ವುದನ್ನು
ಕೇಳು-ವು-ದ-ರಿಂದ
ಕೇಳು-ವು-ದಿ-ಲ್ಲವೆ
ಕೇಳು-ವುದು
ಕೇವಲ
ಕೈ
ಕೈಗಳನ್ನು
ಕೈಗಾ-ರಿ-ಕೆ-ಯನ್ನು
ಕೈಗೆ
ಕೈಯಲ್ಲಿ
ಕೈಯೊಂದು
ಕೈಲಾ-ದಷ್ಚು
ಕೊಂಚ
ಕೊಂಡ
ಕೊಂಡರು
ಕೊಂಡಿತ್ತು
ಕೊಂಡಿ-ತ್ತೇನು
ಕೊಂಡಿ-ದ್ದರೆ
ಕೊಂಡಿ-ರು-ವರೋ
ಕೊಟ್ಟ
ಕೊಟ್ಟರೆ
ಕೊಟ್ಟಿತು
ಕೊಟ್ಟಿ-ರುವೆ
ಕೊಟ್ಟಿಲ್ಲ
ಕೊಟ್ಟು
ಕೊಡದೆ
ಕೊಡ-ಬೇ-ಕಾ-ಗಿರ
ಕೊಡ-ಬೇಕು
ಕೊಡ-ಬೇ-ಕೆಂದು
ಕೊಡ-ಲಾ-ರರಿ
ಕೊಡಲು
ಕೊಡಿ
ಕೊಡು-ತ್ತದೆ
ಕೊಡು-ತ್ತಾನೆ
ಕೊಡು-ತ್ತಿ-ದ್ದರು
ಕೊಡುವ
ಕೊಡು-ವ-ತ-ನಕ
ಕೊಡುವು
ಕೊಡು-ವು-ದಕ್ಕೆ
ಕೊಡು-ವು-ದಿಲ್ಲ
ಕೊಡು-ವುದು
ಕೊನೆ-ಗಾ-ಣು-ತ್ತದೆ
ಕೊನೆ-ಗಾ-ಣು-ವುದು
ಕೊನೆಗೆ
ಕೊನೆಯ
ಕೊನೆ-ಯಲ್ಲಿ
ಕೊರ-ತೆ-ಯನ್ನು
ಕೊಲ್ಲ-ಬ-ಲ್ಲ-ರು-ಎಂ-ದಿಗೂ
ಕೊಲ್ಲು-ತ್ತೇನೆ
ಕೊಳಲು
ಕೊಳ್ಳು-ವರು
ಕೊಳ್ಳು-ವಿರಿ
ಕೊಳ್ಳೋಣ
ಕೋಟಲೆ
ಕೋಟಿ
ಕೋಣೆ-ಯಲ್ಲಿ
ಕೋಮಿ-ನ-ಲ್ಲಿಯೂ
ಕೋಮು-ವಾರು
ಕೋರ್ಟು
ಕೋಲು
ಕೋಶ-ಗಳೇ
ಕ್ರಮ
ಕ್ರಮ-ಕ್ಕಿಂತ
ಕ್ರಮೇಣ
ಕ್ರಿಯೋ-ತ್ತೇ-ಜಕ
ಕ್ರಿಯೋ-ತ್ತೇ-ಜನ
ಕ್ರೈಸ್ತರ
ಕ್ಷಣ-ದಲ್ಲಿ
ಕ್ಷಣ-ಮಾ-ತ್ರ-ದಲ್ಲಿ
ಕ್ಷಣವೂ
ಕ್ಷೇತ್ರ-ದ-ಲ್ಲಿಯೂ
ಖಂಡ-ದಲ್ಲಿ
ಖಂಡಿ-ತ-ವಾ-ಗಿಯೂ
ಖಗೋ-ಳ-ಶಾಸ್ತ್ರ
ಖಗೋ-ಳ-ಶಾ-ಸ್ತ್ರಜ್ಞ
ಖಡ್ಗ
ಖೊರಾ-ನನ್ನು
ಖೊರಾನು
ಗಂಡು
ಗಂಭೀ-ರ-ವಾದ
ಗಚ್ಛತು
ಗಣಿ
ಗಣ್ಯ
ಗತ-ಕಾ-ಲದ
ಗತಿ-ಕೆ-ಟ್ಟ-ವರು
ಗತಿ-ಯಿಲ್ಲ
ಗಮನ
ಗಮ-ನ-ಕೊ-ಡದೆ
ಗಮ-ನ-ದಲ್ಲಿ
ಗಮ-ನಿಸಿ
ಗರಿ-ವಿ-ಲ್ಲ-ದಂತೆ
ಗಲೂ
ಗಳ
ಗಳನ್ನು
ಗಳಲ್ಲ
ಗಳಾದ
ಗಳಿಂದ
ಗಳಿಗೂ
ಗಳಿ-ಗೆಲ್ಲ
ಗಳಿ-ಸ-ಬೇಕು
ಗಳಿ-ಸಿ-ದವು
ಗಳಿ-ಸಿ-ದ್ದರೆ
ಗಳು
ಗಳೆಲ್ಲ
ಗಹ-ನ-ವಾದ
ಗಾಢ-ವಾಗಿ
ಗಾಢಾಂ-ಧ-ಕಾ-ರದ
ಗಾದರೂ
ಗಾರ್ಗಿ
ಗಾಳಿ
ಗಿಂತಲೂ
ಗಿಡ
ಗಿಡ-ವನ್ನು
ಗೀತೆ-ಯನ್ನು
ಗೀರಿ
ಗುಂಡಿ-ಯಲ್ಲಿ
ಗುಂಪಿಗೆ
ಗುಂಪು
ಗುಡಿ-ಸ-ಲಿ-ನಲ್ಲಿ
ಗುಣ-ಗಳು
ಗುಣ-ಗಳೂ
ಗುಣ-ವನ್ನು
ಗುಣ-ವಾ-ಚ-ಕ-ವನ್ನು
ಗುಣ-ವು-ಳ್ಳ-ವನೋ
ಗುಣವೇ
ಗುತ್ತದೆ
ಗುಮಾ-ಸ್ತ-ಗಿರಿ
ಗುರಿ
ಗುರಿ-ಯನ್ನು
ಗುರಿ-ಯೆ-ಡೆಗೆ
ಗುರು
ಗುರು-ಗಳ
ಗುರು-ಗಳು
ಗುರು-ಗೃ-ಹ-ವಾಸ
ಗುರು-ತನ್ನು
ಗುರು-ತ್ವಾ-ಕ-ರ್ಷಣ
ಗುರು-ತ್ವಾ-ಕ-ರ್ಷ-ಣ-ತ-ತ್ತ್ವ-ವನ್ನು
ಗುರು-ದೇ-ವ-ನಿಂದ
ಗುರು-ವನ್ನು
ಗುರು-ವಿಗೆ
ಗುರು-ವಿನ
ಗುರು-ವಿ-ನಲ್ಲಿ
ಗುರು-ವಿ-ನಿಂದ
ಗುರು-ವಿ-ನೊಂ-ದಿಗೆ
ಗುರುವೇ
ಗುಲಾ-ಬಿಯ
ಗುವ
ಗೂಡನ್ನು
ಗೂಢ-ತಮ
ಗೃಹಿ-ಣಿ-ಯ-ರಾಗಿ
ಗೊಂದ-ಲ-ದಿಂದ
ಗೊತ್ತಾಗು
ಗೊತ್ತಾ-ಗು-ವಂತೆ
ಗೊತ್ತಾ-ಗು-ವು-ದಿಲ್ಲ
ಗೊತ್ತಿದೆ
ಗೊತ್ತಿ-ದ್ದರೆ
ಗೊತ್ತಿ-ರಲಿ
ಗೊತ್ತಿ-ರುವ
ಗೊತ್ತಿ-ರು-ವುದೇ
ಗೊತ್ತಿಲ್ಲ
ಗೊತ್ತಿ-ಲ್ಲ-ದ-ವ-ರಿಗೆ
ಗೊಬ್ಬರ
ಗೊಳಿಸಿ
ಗೊಳಿ-ಸು-ವುದೇ
ಗೋಳಾ-ಡ-ಬೇ-ಕಾ-ಗಿಲ್ಲ
ಗೋಳಾ-ಡಿ-ದರೆ
ಗೋಳಾ-ಡು-ವೆವು
ಗೋಳಿ-ಟ್ಟರೆ
ಗೋಸುಗ
ಗೌಣ
ಗೌರವ
ಗೌರ-ವ-ದಿಂದ
ಗೌರ-ವ-ದೃ-ಷ್ಟಿ-ಯಿಂದ
ಗೌರ-ವ-ವನ್ನು
ಗ್ರಂಥ
ಗ್ರಂಥ-ಗ-ಳ-ಲ್ಲಿಯೂ
ಗ್ರಹಿ-ಸು-ವಂತೆ
ಗ್ರಹಿ-ಸು-ವು-ದ-ಕ್ಕಿಂತ
ಗ್ರೀಕರು
ಗ್ರೀಸ್
ಘನ-ಗ-ರ್ಜ-ನೆ-ಯಿಂದ
ಘರ್ಷಣೆ
ಘೋರ-ದುಃ-ಖ-ಭಾವ
ಚಂಚ-ಲ-ಗೊ-ಳಿ-ಸಲು
ಚಂಚ-ಲ-ಗೊ-ಳಿ-ಸು-ತ್ತವೆ
ಚಂಚ-ಲ-ವಾ-ಗಿದೆ
ಚಂಡಾ-ಟದ
ಚಂಡಾ-ಲ-ನ-ವ-ರೆಗೆ
ಚಂದ್ರನ
ಚಕ್ರ-ವರ್ತಿ
ಚಕ್ರ-ವ-ರ್ತಿ-ಯಾದ
ಚಟು-ವ-ಟಿ-ಕೆ-ಯಿಂದ
ಚನೆಯೂ
ಚಪ್ಪಾಳೆ
ಚರಿತ್ರೆ
ಚರ್ಚಿ-ನಲ್ಲಿ
ಚಲ-ನೆಯೂ
ಚಲಿ-ಸು-ತ್ತವೆ
ಚಳು-ವ-ಳಿ-ಯಿಂ-ದಲೂ
ಚಾಕರಿ
ಚಾರಿ-ತ್ರ-ದ-ವ-ರಾ-ಗು-ವರು
ಚಾರಿ-ತ್ರ-ಶುದ್ಧಿ
ಚಾಳಿ
ಚಿಂತಿ-ಸ-ಬೇ-ಕೆಂದು
ಚಿಂತಿ-ಸಿ-ದರೆ
ಚಿಂದಿ-ಗಂಟು
ಚಿಕ್ಕ
ಚಿತ್ತ
ಚಿತ್ತ-ಭಿ-ತ್ತಿಯ
ಚಿತ್ತ-ವನ್ನು
ಚಿತ್ರ
ಚೂರಿ
ಚೆನ್ನಾಗಿ
ಚೆಲ್ಲಾ-ಪಿ-ಲ್ಲಿ-ಯಾಗಿ
ಚೇತ-ನ-ವನ್ನು
ಚೇತ-ನ-ವಿಲ್ಲ
ಜಗ-ತ್ತನ್ನು
ಜಗ-ತ್ತಿ-ಗೆಲ್ಲ
ಜಗತ್ತು
ಜಗ-ನ್ಮಯಿ
ಜಗ್ಗದ
ಜಟಿಲ
ಜನ
ಜನ-ನಕ್ಕೆ
ಜನ-ನಾ-ರಂ-ಭ-ದಿಂ-ದಲೇ
ಜನರ
ಜನ-ರಂ-ಜ-ಕ-ವಾ-ಗುವ
ಜನ-ರ-ನ್ನಾಗಿ
ಜನ-ರನ್ನು
ಜನ-ರಲ್ಲಿ
ಜನ-ರಿಗೆ
ಜನರು
ಜನ-ರೆಲ್ಲ
ಜನ-ಸಾ-ಮಾ-ನ್ಯರ
ಜನ-ಸಾ-ಮಾ-ನ್ಯ-ರನ್ನು
ಜನ-ಸಾ-ಮಾ-ನ್ಯ-ರಲ್ಲಿ
ಜನ-ಸಾ-ಮಾ-ನ್ಯ-ರಿಗೆ
ಜನಾಂಗ
ಜನಾಂ-ಗಕ್ಕೆ
ಜನಾಂ-ಗ-ಗಳಲ್ಲಿ
ಜನಾಂ-ಗ-ಗ-ಳೆಲ್ಲ
ಜನಾಂ-ಗದ
ಜನಾಂ-ಗ-ದಂತೆ
ಜನಾ-ದ-ರ-ಣೀ-ಯ-ವ-ನ್ನಾಗಿ
ಜನಿ-ಸಿದ
ಜನಿ-ಸು-ವರು
ಜನ್ಮ-ಗಳು
ಜನ್ಮ-ತಾಳಿ
ಜನ್ಮ-ವನ್ನು
ಜನ್ಮ-ವೆ-ತ್ತಿ-ದನು
ಜನ್ಮ-ವೆ-ತ್ತಿ-ದ್ದರು
ಜಪ
ಜಯ
ಜಯ-ವನ್ನು
ಜವಾ-ಬ್ದಾರಿ
ಜಾಗೃ-ತ-ವಾ-ಗು-ತ್ತವೆ
ಜಾಗ್ರ-ತ-ಗೊ-ಳಿಸ
ಜಾಗ್ರ-ತ-ಗೊ-ಳಿ-ಸ-ಬೇ-ಕಾ-ಗಿದೆ
ಜಾಗ್ರ-ತ-ಗೊ-ಳಿ-ಸ-ಬೇಕು
ಜಾಗ್ರ-ತ-ಗೊ-ಳಿ-ಸುವ
ಜಾಗ್ರ-ತ-ನ-ನ್ನಾಗಿ
ಜಾಗ್ರ-ತ-ನಾಗು
ಜಾಗ್ರ-ತ-ರಾಗಿ
ಜಾಗ್ರ-ತ-ರಾಗು
ಜಾಗ್ರ-ತ-ರಾ-ಗು-ವರು
ಜಾಗ್ರ-ತ-ವಾ-ಗು-ವಂ-ತೆ-ಮಾಡಿ
ಜಾಜ್ವ-ಲ್ಯ-ಮಾ-ನ-ವಾಗಿ
ಜಾಜ್ವ-ಲ್ಯ-ಮಾ-ನ-ವಾದ
ಜಾಡ್ಯ
ಜಾತಿ
ಜಾತಿ-ಗಳ
ಜಾತಿಯ
ಜಾರು-ತ್ತಿ-ದೆಯೋ
ಜೀರ್ಣಿ-ಸಿ-ಕೊಂಡು
ಜೀವ
ಜೀವ-ದಿಂದ
ಜೀವನ
ಜೀವ-ನದ
ಜೀವ-ನ-ದಲ್ಲಿ
ಜೀವ-ನ-ದ-ಲ್ಲಿ-ರುವ
ಜೀವ-ನ-ವನ್ನು
ಜೀವ-ನ-ವಿ-ಲ್ಲದೆ
ಜೀವ-ನೋ-ಪಾ-ಯದ
ಜೀವ-ಮಾನ
ಜೀವವೂ
ಜೀವ-ಶ್ಶ-ವ-ದಂತೆ
ಜೀವಾ-ಣು-ವಿ-ನಲ್ಲಿ
ಜೀವಾ-ಣು-ವಿ-ನಿಂದ
ಜೀವಿ-ಗಳ
ಜೀವಿ-ಗ-ಳ-ಲ್ಲಿಯೂ
ಜೀವಿ-ಯನ್ನು
ಜೀವಿ-ಯ-ಲ್ಲಿಯೂ
ಜೀವಿ-ಸಿ-ರು-ವುದು
ಜೊತೆಗೆ
ಜೊತೆ-ಯ-ಲ್ಲಿಯೇ
ಜೋಡನ್ನು
ಜೋಡಿ-ಸಿ-ದಾಗ
ಜೋಪ-ಡಿ-ಗಳಲ್ಲಿ
ಜೋಪಾನ
ಜೋಲು
ಜ್ಞಾನ
ಜ್ಞಾನಕ್ಕೆ
ಜ್ಞಾನ-ಜ್ಯೋ-ತಿಯ
ಜ್ಞಾನ-ಜ್ಯೋ-ತಿ-ಯನ್ನು
ಜ್ಞಾನದ
ಜ್ಞಾನ-ಭಂ-ಡಾ-ರಕ್ಕೆ
ಜ್ಞಾನ-ಮಯ
ಜ್ಞಾನ-ವನ್ನು
ಜ್ಞಾನವೂ
ಜ್ಞಾನ-ವೆಲ್ಲ
ಜ್ಞಾನವೋ
ಜ್ಞಾನಾ-ಕಾಂಕ್ಷೆ
ಜ್ಞಾನಾ-ಕಾಂ-ಕ್ಷೆಯ
ಜ್ಞಾನಾ-ರ್ಜನೆ
ಜ್ಞಾನಾ-ರ್ಜ-ನೆಗೆ
ಜ್ಞಾನಾ-ರ್ಜ-ನೆಯೂ
ಜ್ಞಾನಿ
ಜ್ಞಾನಿ-ಯಲ್ಲ
ಜ್ಞಾನೋ
ಜ್ಞಾಪ-ಕ-ದ-ಲ್ಲಿ-ಡ-ಬೇ-ಕಾದ
ಜ್ಞಾಪ-ಕ-ದ-ಲ್ಲಿಡಿ
ಜ್ಞಾಪಿ-ಸಿ-ಕೊಳ್ಳಿ
ಜ್ಯೋತಿ
ಜ್ಯೋತಿ-ಯನ್ನು
ಝಾಂಸಿಯ
ಟೋಪಿಯೂ
ಡಮರು
ಡಾಲ-ರನ್ನು
ಡೆಪ್ಯೂಟಿ
ತಂದ
ತಂದರೆ
ತಂದಿ-ಟ್ಟಿ-ರು-ವು-ದಕ್ಕೆ
ತಂದಿ-ದ್ದರೆ
ತಂದಿವೆ
ತಂದು
ತಂದೆ
ತಂದೆ-ಗಳ
ತಂದೆ-ತಾ-ಯಿ-ಗಳು
ತಂದೆಯ
ತಕ್ಕಂತೆ
ತಕ್ಷ-ಣವೇ
ತಟ್ಟು-ವುದು
ತಡೆ-ದಿ-ರುವ
ತಡೆ-ಯ-ಲಾ-ಗ-ದಂ-ತಹ
ತಡೆ-ಯು-ವು-ದ-ರಲ್ಲಿ
ತಡೆ-ಯು-ವುದು
ತತ್ತ್ವ-ಗಳನ್ನು
ತತ್ವ-ವೇ-ತ್ತ-ನಲ್ಲ
ತನಕ
ತನ-ಗಾ-ಗಲಿ
ತನ-ಗಿಷ್ಟ
ತನಗೆ
ತನ್ನ
ತನ್ನನ್ನು
ತನ್ನಲ್ಲಿ
ತನ್ನಿ
ತಪ್ಪನ್ನು
ತಪ್ಪಾಗಿ
ತಪ್ಪಿ-ಗಾಗಿ
ತಪ್ಪಿ-ಸಿ-ಕೊಂಡು
ತಪ್ಪು
ತಪ್ಪು-ಗಳನ್ನು
ತಪ್ಪು-ಗ-ಳಿ-ಲ್ಲದೆ
ತಮಗೆ
ತಮ-ಟೆ-ಗಳು
ತಮ-ಸ್ಸಿನ
ತಮ್ಮ
ತಮ್ಮಂತೆ
ತಮ್ಮ-ದೆಂಬ
ತಮ್ಮದೇ
ತಮ್ಮ-ಲ್ಲಿದ್ದ
ತಯಾ-ರಾ-ಗು-ವು-ದಿ-ಲ್ಲವೆ
ತಯಾ-ರು-ಮಾ-ಡು-ವುದು
ತರಂ-ಗ-ದಿಂ-ದಲೇ
ತರ-ತ-ಮ-ದಲ್ಲಿ
ತರ-ಬ-ಲ್ಲರು
ತರ-ಬೇಕು
ತರ-ಬೇ-ಕು-ಇವು
ತರ-ಬೇ-ತಿನ
ತರ-ಬೇ-ತಿಯ
ತರ-ಬೇ-ತಿ-ಯಿಂದ
ತರ-ಬೇತು
ತರುಣ
ತರು-ವಂತೆ
ತರು-ವರು
ತಲೆಗೆ
ತಲೆ-ಯಲ್ಲಿ
ತಲೆ-ಯೆತ್ತಿ
ತಳ-ಹದಿ
ತಳ-ಹ-ದಿಯ
ತಾನು
ತಾನೇ
ತಾಯಿ
ತಾಯಿಯ
ತಾಯಿ-ಯ-ರಿಂದ
ತಾರ-ಸ್ವ-ರ-ವನ್ನು
ತಾರೆ
ತಾಳು
ತಾವು
ತಾವೆ
ತಾವೇ
ತಿಂಗಳಲ್ಲಿ
ತಿಕ್ಕು-ವನು
ತಿಕ್ಕು-ವ-ವನು
ತಿದ್ದ-ಬೇ-ಕಾ-ಯಿತು
ತಿದ್ದಿ-ರು-ತ್ತವೆ
ತಿದ್ದು-ವು-ದ-ರಲ್ಲಿ
ತಿದ್ದು-ವುದು
ತಿರು-ಗಿ-ಸಿ-ದರು
ತಿರು-ಗಿ-ಸಿ-ದರೆ
ತಿರುಳು
ತಿಳಿ-ದರೆ
ತಿಳಿ-ದಿ-ರಲಿ
ತಿಳಿದು
ತಿಳಿ-ದು-ಕೊಂ-ಡಿ-ದ್ದೆನೋ
ತಿಳಿ-ದು-ಕೊಂ-ಡಿ-ದ್ದೆವೊ
ತಿಳಿ-ದು-ಕೊಳ್ಳ
ತಿಳಿ-ದು-ಕೊ-ಳ್ಳ-ಬೇಕು
ತಿಳಿ-ದು-ಕೊ-ಳ್ಳ-ಬೇ-ಕೆಂದು
ತಿಳಿ-ದು-ಕೊ-ಳ್ಳಲಿ
ತಿಳಿ-ದು-ಕೊಳ್ಳಿ
ತಿಳಿ-ದು-ಕೊಳ್ಳು
ತಿಳಿ-ದು-ಕೊ-ಳ್ಳು-ವಂತೆ
ತಿಳಿ-ದು-ಕೊ-ಳ್ಳು-ವಷ್ಟು
ತಿಳಿ-ದು-ಕೊ-ಳ್ಳುವು
ತಿಳಿ-ದು-ಕೊ-ಳ್ಳು-ವು-ದಕ್ಕೆ
ತಿಳಿ-ದೆನು
ತಿಳಿ-ಯದೆ
ತಿಳಿ-ಯ-ಬೇ-ಕಾ-ಗಿಲ್ಲ
ತಿಳಿಯು
ತಿಳಿ-ಯು-ತ್ತಾನೆ
ತಿಳಿ-ಯು-ವಂತೆ
ತಿಳಿ-ಸ-ಬ-ಹುದು
ತಿಳಿಸಿ
ತೀರದ
ತುಂಬ-ಬೇಕು
ತುಂಬಾ
ತುಂಬಿ
ತುಂಬಿ-ಕೊಂಡು
ತುಂಬಿ-ಸಿ-ಕೊ-ಳ್ಳು-ತ್ತಾನೆ
ತುಂಬು-ವಷ್ಟು
ತುಂಬು-ವುದು
ತುತ್ತಾಗಿ
ತುತ್ತಾ-ದ-ವ-ರಿಗೆ
ತುತ್ತೂರಿ
ತುಳು
ತೂರ್ಯ-ವಾ-ಣಿ-ಯಿಂದ
ತೃಣ-ಮಾ-ತ್ರ-ವಾ-ಗು-ವುದು
ತೃಪ್ತಿ-ಕರ
ತೃಪ್ತಿ-ಪ-ಡಿ-ಸ-ಲಾ-ರಿರಿ
ತೆಗೆ-ದಂತೆ
ತೆಗೆ-ದು-ಕೊಂ-ಡಿತು
ತೆಗೆ-ದು-ಕೊಂಡು
ತೆಗೆ-ದು-ಕೊ-ಳ್ಳು-ತ್ತಿ-ರ-ಲಿಲ್ಲ
ತೆಗೆ-ಯಿರಿ
ತೆರೆ
ತೆರೆ-ದಿರಿ
ತೆರೆದು
ತೆರೆ-ಯ-ಬೇಕು
ತೆರೆ-ಯಿರಿ
ತೆರೆ-ಯು-ವುದು
ತೆರೆ-ಯೋಣ
ತೆಳ್ಳ-ಗಾ-ಗು-ವಂತೆ
ತೊಂಭ-ತ್ತ-ರಷ್ಟು
ತೊಡೆ-ಯ-ಮೇಲೆ
ತೊರೆ-ದಂತೆ
ತೊರೆ-ಯಲು
ತೋರದೆ
ತೋರ-ಬ-ಹುದು
ತೋರ-ಬೇಕು
ತೋರಿ
ತೋರಿ-ಕೆಯ
ತೋರಿತು
ತೋರಿದ
ತೋರಿ-ದುದ
ತೋರಿ-ಸ-ಕೂ-ಡದು
ತೋರಿ-ಸ-ಲ್ಪ-ಟ್ಟಿದೆ
ತೋರು
ತೋರು-ತ್ತದೆ
ತೋರು-ತ್ತಾನೆ
ತೋರುವ
ತೋರು-ವನು
ತೋರು-ವರು
ತೋರು-ವಿರಾ
ತೋರು-ವುದು
ತೋರು-ವುದೇ
ತೋರು-ವುದೊ
ತೋರೆಂದು
ತೌರೂರು
ತ್ತದೆ
ತ್ತವೆ
ತ್ತಾನೆ
ತ್ತಾರೆ
ತ್ತಾರೆಯೋ
ತ್ತಾರೋ
ತ್ತಿತ್ತು
ತ್ತಿದೆಯೇ
ತ್ತಿದ್ದನು
ತ್ತಿರು-ವರು
ತ್ತಿರು-ವುದನ್ನು
ತ್ತೀರಿ
ತ್ತೇನೆ
ತ್ತೇವೆ
ತ್ಯಾಗವೋ
ತ್ಯಾಗಿ-ಗಳ
ತ್ಯಾಗಿ-ಗ-ಳಾ-ಗಿ-ದ್ದರು
ತ್ಯಾಗಿ-ಗ-ಳಾದ
ದಕ್ಕಾಗಿ
ದಕ್ಕೆ
ದನ್ನು
ದಬ್ಬಾ-ಳಿಕೆ
ದಬ್ಬಾ-ಳಿ-ಕೆಗೆ
ದಯಕ್ಕೆ
ದರಿ-ದ್ರರ
ದರೂ
ದರೆ
ದರ್ಪ-ಗಳ
ದರ್ಪದ
ದರ್ಶ-ನ-ವನ್ನು
ದಲಿ-ತ-ರಿಗೆ
ದಲ್ಲಿ
ದಲ್ಲಿತ್ತು
ದಲ್ಲಿಯೂ
ದಲ್ಲಿ-ರುವ
ದಲ್ಲೆಲ್ಲ
ದಲ್ಲೇ
ದಷ್ಟು
ದಹಿ-ಸ-ಲಾ-ರದು
ದಾಗಿ
ದಾಟಿ-ದನು
ದಾಟು-ವಂ-ತಹ
ದಾನ
ದಾನ-ಮಾ-ಡ-ಬೇ-ಕಾ-ದರೆ
ದಾನ-ಮಾ-ಡ-ಬೇಕು
ದಾನ-ಮಾಡಿ
ದಾನ-ಮಾ-ಡು-ವುದು
ದಾರಿ
ದಾರಿ-ಯನ್ನು
ದಾರ್ಢ್ಯ
ದಿಂದಲೇ
ದಿಕ್ತ-ಟ-ಗಳು
ದಿಗಂ-ಬ-ರ-ನಾಗಿ
ದಿಟ್ಟ-ತ-ನ-ದಿಂದ
ದಿನ-ದಿಂ-ದಲೇ
ದಿಲ್ಲ
ದೀನ
ದೀನ-ನಾದ
ದೀನ-ನೆಂದೂ
ದೀನರ
ದೀನ-ರನ್ನು
ದೀನ-ರಿಗೆ
ದೀಪ
ದೀಪ-ದಂತೆ
ದೀರಾಃ
ದೀರ್ಘ-ಪ್ರ-ಯತ್ನ
ದುಃಖ
ದುಃಖದ
ದುಃಖ-ವನ್ನು
ದುಃಖವೇ
ದುಃಖಿ-ಗ-ಳಾ-ಗಿ-ರು-ವರೊ
ದುಃಖಿ-ಗ-ಳಿಗೆ
ದುಃಸ್ಥಿತಿ
ದುರ್ಬ-ಲತೆ
ದುರ್ಬ-ಲ-ತೆಯೇ
ದುರ್ಬ-ಲ-ರ-ನ್ನಾಗಿ
ದುರ್ಬ-ಲ-ರಿಗೆ
ದುರ್ಬ-ಲ-ರೆಂದು
ದುರ್ಬ-ಲರೇ
ದುರ್ಬ-ಲ-ವಾದ
ದೂರ
ದೂರ-ದ-ರ್ಶಕ
ದೂರ-ಬೇಡಿ
ದೂರು-ತ್ತಿ-ದ್ದರೆ
ದೃಢ
ದೃಶ್ಯ-ಗಳ
ದೃಶ್ಯ-ಗಳನ್ನು
ದೃಶ್ಯವೇ
ದೃಷ್ಟಿ-ಯಿಂದ
ದೆಲ್ಲೆಲ್ಲ
ದೇದೀ-ಪ್ಯ-ಮಾ-ನ-ವಾದ
ದೇವ-ತೆ-ಗ-ಳಿ-ಗಾ-ಗಲಿ
ದೇವ-ತೆ-ಗಳು
ದೇವ-ತೆ-ಯನ್ನೂ
ದೇವ-ದೂ-ತ-ರು-ಗ-ಳಿಗೆ
ದೇವರ
ದೇವ-ರಂತೆ
ದೇವ-ರನ್ನು
ದೇವ-ರ-ಮೇಲೆ
ದೇವ-ರಲ್ಲಿ
ದೇವ-ರಾ-ಗಲಿ
ದೇವ-ರಿಗೆ
ದೇವ-ರಿ-ಗೆಲ್ಲಾ
ದೇವರು
ದೇವ-ಸ್ಥಾನ
ದೇಶ
ದೇಶಕ್ಕೆ
ದೇಶ-ಗಳ
ದೇಶದ
ದೇಶ-ದಲ್ಲಿ
ದೇಶ-ದ-ಲ್ಲಿ-ದ್ದಾಗ
ದೇಶ-ದ-ಲ್ಲಿ-ರುವ
ದೇಶ-ದ-ಲ್ಲಿ-ರು-ವ-ವರೆ-ಲ್ಲ-ರನ್ನೂ
ದೇಶ-ಭ-ಕ್ತ-ರಾ-ಗ-ಬೇ-ಕೆಂದು
ದೇಶ-ಭಾ-ಷೆ-ಯಲ್ಲಿ
ದೇಶ-ವನ್ನು
ದೇಶ-ವಿ-ರು-ವುದು
ದೇಶವೂ
ದೇಹ
ದೇಹದ
ದೇಹ-ವನ್ನು
ದೇಹ-ಸ್ಥಿ-ತಿ-ಯಲ್ಲಿ
ದೊಡ್ಡ
ದೊಡ್ಡ-ದಾ-ಗಿ-ಲ್ಲದ
ದೊರ-ಕಿದ
ದೊರು-ಕುವ
ದೊರು-ಕು-ವಂತೆ
ದೊರೆ-ತರೆ
ದೊರೆ-ಯು-ವು-ದಿ-ಲ್ಲವೆ
ದೋಷ-ಗಳನ್ನು
ದೋಷ-ಗ-ಳಿ-ದ್ದರೆ
ದೋಷ-ಗಳು
ದೋಷದ
ದೋಷಾ-ರೋ-ಪಣೆ
ದೌರ್ಬ-ಲ್ಯ-ಗ-ಳೆಲ್ಲ
ದ್ದರೆ-ಪ್ರ-ಪಂ-ಚ-ದಲ್ಲಿ
ದ್ರವ್ಯ
ದ್ರವ್ಯ-ಗಳ
ದ್ರವ್ಯ-ಗಳನ್ನು
ದ್ರವ್ಯ-ವನ್ನು
ದ್ರವ್ಯ-ವೆಲ್ಲ
ದ್ರವ್ಯಾ-ರ್ಜ-ನೆಯ
ದ್ರುಪದ್
ದ್ವೇಷ
ದ್ವೇಷ-ಗಳನ್ನು
ಧನ್ಯರು
ಧರ್ಮ
ಧರ್ಮಕ್ಕೆ
ಧರ್ಮ-ಗಳ
ಧರ್ಮ-ಗಳು
ಧರ್ಮ-ಗಳೂ
ಧರ್ಮ-ಗ-ಳೆ-ಲ್ಲ-ವನ್ನೂ
ಧರ್ಮ-ಗ್ರಂಥ
ಧರ್ಮದ
ಧರ್ಮ-ದಲ್ಲಿ
ಧರ್ಮ-ದ-ಲ್ಲಿ-ರುವ
ಧರ್ಮ-ದಿಂದ
ಧರ್ಮ-ವನ್ನು
ಧರ್ಮವೂ
ಧರ್ಮವೇ
ಧರ್ಮಾನು
ಧಾರ್ಮಿಕ
ಧಾರ್ಮಿ-ಕ-ನಾ-ಗಿ-ರು-ವುದು
ಧಾರ್ಮಿ-ಕ-ರ-ನ್ನಾಗಿ
ಧೀರ
ಧೀರ-ರ-ನ್ನಾಗಿ
ಧೀರರು
ಧೀರಾ-ತ್ಮ-ರಾಗಿ
ಧೈರ್ಯ
ಧೈರ್ಯ-ದಿಂದ
ಧೈರ್ಯ-ವನ್ನು
ಧೈರ್ಯ-ವಾಗಿ
ಧ್ಯಾನ
ಧ್ಯಾನಕ್ಕೆ
ಧ್ಯಾನಾ-ಭ್ಯಾ-ಸವೂ
ಧ್ಯೇಯ
ಧ್ಯೇಯ-ದಿಂದ
ಧ್ವಂಸ-ಮಾಡು
ಧ್ವನಿ
ಧ್ವನಿ-ಮಾಡು
ಧ್ವನಿ-ಯನ್ನು
ಧ್ವನಿ-ಯಿಂದ
ಧ್ವನಿಯೆ
ನ
ನಂತರ
ನಂಬ-ಬೇಕು
ನಂಬ-ಬೇ-ಕೆಂದು
ನಂಬಿ
ನಂಬಿಕೆ
ನಂಬಿ-ಕೆ-ಯನ್ನು
ನಂಬಿ-ಕೆ-ಯಲ್ಲಿ
ನಂಬಿ-ಕೆ-ಯಿ-ಲ್ಲವೊ
ನಂಬಿ-ಕೆ-ಯಿ-ಲ್ಲವೋ
ನಂಬಿ-ದನು
ನಂಬಿ-ದರೆ
ನಕ್ಕು
ನಕ್ಷೆ
ನಕ್ಷೆ-ಯನ್ನು
ನಗು-ತ್ತಾನೆ
ನಡ-ತೆ-ಗಳಿಂದ
ನಡ-ತೆಯ
ನಡು-ಗು-ತ್ತಿತ್ತು
ನಡೆ-ದಾಗ
ನಡೆ-ಯ-ಬೇಕು
ನಡೆ-ಯು-ತ್ತಾನೆ
ನನಗೆ
ನನ್ನ
ನನ್ನಂ-ತಹ
ನನ್ನ-ದಾ-ಗಿತ್ತು
ನನ್ನದೂ
ನನ್ನನ್ನು
ನಮ
ನಮ-ಗಾಗಿ
ನಮ-ಗಿಂದು
ನಮಗೂ
ನಮಗೆ
ನಮ-ಗೆ-ಲ್ಲ-ರಿಗೂ
ನಮ-ಸ್ಕಾರ
ನಮ್ಮ
ನಮ್ಮಂ-ತೆಯೇ
ನಮ್ಮದು
ನಮ್ಮದೇ
ನಮ್ಮನ್ನು
ನಮ್ಮಲ್ಲಿ
ನಮ್ಮ-ಲ್ಲಿದೆ
ನಮ್ಮ-ಲ್ಲಿ-ರುವ
ನಮ್ಮ-ಲ್ಲಿ-ರು-ವುದು
ನಮ್ಮ-ಲ್ಲಿ-ರು-ವು-ದೆಲ್ಲ
ನಮ್ಮಿಂದ
ನಮ್ಮಿಂ-ದಲೇ
ನರ
ನರ-ಳು-ತ್ತಿ-ರುವ
ನರ-ಳು-ತ್ತಿ-ರು-ವರು
ನರ-ಳು-ವಾಗ
ನರಿ-ಯಾ-ಗು-ತ್ತಾನೆ
ನರು-ಳು-ತ್ತಿ-ರು-ವರೋ
ನಲಿ-ಯು-ವಾಗ
ನಲ್ಲಿ
ನಲ್ಲಿ-ರುವ
ನವ
ನವ-ಚೇ-ತನ
ನವ-ಚೇ-ತ-ನ-ವನ್ನು
ನವ-ರನ್ನು
ನಶಿ-ಸಿ-ದರೂ
ನಷ್ಟ-ವಾ-ದರೂ
ನಾಗ-ರಿ-ಕ-ತೆಯ
ನಾಗಿ-ರು-ವುದು
ನಾಡಿ-ಯಲ್ಲಿ
ನಾಡಿ-ಯ-ಲ್ಲಿಯೂ
ನಾಡು-ತ್ತಿ-ರು-ವನು
ನಾದರೋ
ನಾನಾ-ತ-ರಹ
ನಾನು
ನಾನೂ
ನಾನೇ
ನಾಯ-ಕನ
ನಾಯ-ಕನು
ನಾರಿ-ಯನ್ನು
ನಾರಿ-ಯರ
ನಾರಿ-ಯ-ರಷ್ಟೇ
ನಾರಿ-ಯರು
ನಾರೀ-ರ-ತ್ನ-ಗಳನ್ನು
ನಾರೀ-ರ-ತ್ನ-ಗಳು
ನಾವು
ನಾವೇ
ನಾವೇನು
ನಾಶ-ಮಾ-ಡು-ತ್ತಿ-ರ-ಲಿಲ್ಲ
ನಾಶ-ವಾ-ಗದ
ನಾಶ-ವಾ-ಗ-ದಂತೆ
ನಾಶ-ವಾ-ಗುವ
ನಾಶ-ವಾ-ಗು-ವು-ದಿಲ್ಲ
ನಾಸ್ತಿ-ಕ-ನೆಂದು
ನಾಸ್ತಿ-ಕ-ರೆಂದು
ನಿಂತ
ನಿಂತರೂ
ನಿಂತಿದೆ
ನಿಂತು
ನಿಂತು-ಕೊಂ-ಡಾಗ
ನಿಂದ
ನಿಂದಂತು
ನಿಂದಿ-ಸಲಿ
ನಿಂದೆ
ನಿಂದೆ-ಯ-ಲ್ಲವೆ
ನಿಃಸ್ವಾ-ರ್ಥ-ತೆ-ಗಳಿಂದ
ನಿಗೂ
ನಿಗ್ರ-ಹಿ-ಸ-ಬ-ಹುದು
ನಿಗ್ರ-ಹಿ-ಸು-ತ್ತೇನೆ
ನಿಗ್ರ-ಹಿ-ಸು-ವುದೇ
ನಿಜ
ನಿಜ-ವಾಗಿ
ನಿಜ-ವಾ-ಗಿಯೂ
ನಿಜ-ವಾದ
ನಿಜ-ವೆಂದು
ನಿನ-ಗೇನು
ನಿನಾ-ದ-ವನ್ನು
ನಿನ್ನ
ನಿನ್ನಂ-ತಹ
ನಿನ್ನನ್ನು
ನಿಪು-ಣರು
ನಿಮ
ನಿಮ-ಗಿಂತ
ನಿಮಗೂ
ನಿಮಗೆ
ನಿಮ್ಮ
ನಿಮ್ಮದೂ
ನಿಮ್ಮನ್ನು
ನಿಮ್ಮಲ್ಲಿ
ನಿಮ್ಮಿಂದ
ನಿಮ್ಮೊಂ-ದಿಗೆ
ನಿಯಮ
ನಿಯ-ಮ-ಗಳಿಂದ
ನಿಯ-ಮ-ಗಳು
ನಿಯ-ಮ-ವಿದೆ
ನಿರಂ-ತರ
ನಿರಂ-ತ-ರವೂ
ನಿರ-ತ-ನಾದ
ನಿರ-ತ-ರಾಗಿ
ನಿರಾಂ-ತ-ಕ-ವಾಗಿ
ನಿರಾ-ಕ-ರಿಸಿ
ನಿರಾ-ಕ-ರಿ-ಸು-ತ್ತಾನೆ
ನಿರಾ-ಶ-ರಾಗ
ನಿರಾ-ಶ-ರಾ-ದಾಗ
ನಿರ್ಗ-ತಿ-ಕ-ರಾಗಿ
ನಿರ್ಜ-ನಾ-ರ-ಣ್ಯ-ಗಳಿಂದ
ನಿರ್ದಯ
ನಿರ್ದಿಷ್ಟ
ನಿರ್ಧ-ರಿ-ಸ-ಲ್ಪ-ಡು-ತ್ತದೆ
ನಿರ್ಬ-ಲತೆ
ನಿರ್ಬ-ಲ-ನ-ನ್ನಾಗಿ
ನಿರ್ಬ-ಲ-ರ-ನ್ನಾಗಿ
ನಿರ್ಬ-ಲ-ರಾ-ಗಿ-ರು-ವು-ದ-ರಿಂದ
ನಿರ್ಬ-ಲ-ರಾ-ಗಿ-ರು-ವೆವು
ನಿರ್ಭಾ-ಗ್ಯರು
ನಿರ್ಮಾಣ
ನಿರ್ಲ-ಕ್ಷಿ-ಸಿದ್ದು
ನಿಲು-ಕದ
ನಿಲ್ಲ-ಬ-ಲ್ಲನೋ
ನಿಲ್ಲ-ಬೇಡ
ನಿಲ್ಲಿ
ನಿಲ್ಲಿ-ಸ-ಬೇಕು
ನಿಲ್ಲಿ-ಸು-ವಂತೆ
ನಿಲ್ಲು-ವುದನ್ನು
ನಿಲ್ಲು-ವುದು
ನಿವಾ-ರಿ-ಸ-ಬ-ಹುದು
ನಿವಾ-ರಿ-ಸ-ಲಾ-ಗದ
ನಿಷೇಧ
ನಿಷೇ-ಧ-ಮಯ
ನಿಷೇ-ಧ-ಮ-ಯ-ವಾದ
ನಿಷೇ-ಧಾತ್ಮ
ನಿಷ್ಕ-ಪಟ
ನಿಷ್ಪ-ಲ-ವಾ-ದಂ-ತೆಯೆ
ನಿಷ್ಫ-ಲ-ವಾ-ಗು-ವುದು
ನಿಹಾ-ರಿ-ಕೆ-ಗಳು
ನೀಗಿ-ಕೊಂಡ
ನೀಚ
ನೀಡ-ಬೇಕು
ನೀಡಿದ
ನೀಡಿದ್ದು
ನೀಡುವ
ನೀಡು-ವುದು
ನೀತಿ
ನೀತಿ-ನಿ-ಪುಣಾ
ನೀನು
ನೀರಿನ
ನೀರು
ನೀರು-ಗುಳ್ಳೆ
ನೀವು
ನೀವೆ
ನೀವೆಲ್ಲ
ನೀವೇ
ನುಡಿಯು
ನುಡಿ-ಸು-ವು-ದ-ರಿಂದ
ನೂರಾರು
ನೂರು
ನೂರ್ಮಡಿ
ನೂಲನ್ನು
ನೆಚ್ಚಿ-ಕೊಂ-ಡಿರು
ನೆಚ್ಚಿಗೆ
ನೆಚ್ಚು-ಗೆ-ಡ-ಬೇ-ಕಾ-ಗಿಲ್ಲ
ನೆನ-ಪಿ-ನಲ್ಲಿ
ನೇತಾ-ಡುವ
ನೈಜ-ಸ್ಥಿತಿ
ನೈಜ-ಸ್ವ-ಭಾ-ವ-ವನ್ನು
ನೈಷ್ಠಿಕ
ನೋಡ-ಬ-ಲ್ಲರೋ
ನೋಡ-ಬ-ಹುದು
ನೋಡ-ಬೇಕು
ನೋಡ-ಬೇ-ಕೆಂದು
ನೋಡಲು
ನೋಡಿ
ನೋಡಿ-ಕೊ-ಳ್ಳ-ಬೇಕು
ನೋಡಿ-ಕೊಳ್ಳಿ
ನೋಡಿದ
ನೋಡಿ-ದನು
ನೋಡಿ-ದರೆ
ನೋಡಿ-ದಾಗ
ನೋಡಿ-ರು-ವೆನು
ನೋಡು
ನೋಡು-ತ್ತಾ-ರೆಯೋ
ನೋಡು-ತ್ತಿದ್ದೆ
ನೋಡು-ತ್ತಿ-ರು-ತ್ತಾನೆ
ನೋಡು-ತ್ತೀರೊ
ನೋಡು-ವಿರಿ
ನೋಡು-ವು-ದಕ್ಕೆ
ನೋಡು-ವು-ದಿ-ಲ್ಲವೊ
ನೋಪಾ-ಯಕ್ಕೆ
ನ್ಯಾ
ನ್ಯಾಯ-ವಾದ
ನ್ಯೂಟನ್
ನ್ಯೂಟ-ನ್ನ-ನಿಗೆ
ನ್ಯೂನ-ತೆ-ಗ-ಳೊಂ-ದಿಗೆ
ನ್ಯೂಯಾ-ರ್ಕಿ-ನ-ಲ್ಲಿ-ದ್ದಾಗ
ಪಂಗ-ಡಕ್ಕೆ
ಪಂಗ-ಡ-ಗಳನ್ನು
ಪಂಗ-ಡ-ದ-ವ-ರಿಗೆ
ಪಂಗ-ಡ-ದ-ವರು
ಪಂಗ-ಡ-ವರು
ಪಂಡಿ-ತನಿ
ಪಂಡಿ-ತ-ರಿಗೆ
ಪಟ್ಟ-ಣ-ಗಳಲ್ಲಿ
ಪಡದೆ
ಪಡಿ-ಸು-ವಂ-ತಹ
ಪಡು-ತ್ತೇನೆ
ಪಡೆದ
ಪಡೆ-ದನು
ಪಡೆ-ದರೂ
ಪಡೆದು
ಪಡೆ-ದೂ-ಅಂ-ಥ-ವರ
ಪಡೆದೇ
ಪಡೆ-ಯದೆ
ಪಡೆ-ಯ-ಬ-ಹುದು
ಪಡೆ-ಯ-ಬೇ-ಕಾ-ದರೆ
ಪಡೆ-ಯ-ಬೇಕು
ಪಡೆ-ಯು-ತ್ತದೆ
ಪಡೆ-ಯು-ತ್ತಾರೆ
ಪಡೆ-ಯು-ತ್ತಿ-ರು-ವೆವು
ಪಡೆ-ಯು-ತ್ತೇವೆ
ಪಡೆ-ಯು-ವನು
ಪಡೆ-ಯು-ವರು
ಪಡೆ-ಯು-ವು-ದಿಲ್ಲ
ಪದ
ಪದಂ
ಪದ-ವ-ನ್ನಾ-ದರೂ
ಪದ-ವನ್ನು
ಪದವಿ
ಪದೇ
ಪಯ-ಣ-ದಲ್ಲಿ
ಪರಂ-ಪರೆ
ಪರ-ದೇ-ಶ-ದ-ವರ
ಪರ-ದೇ-ಶೀ-ಯರ
ಪರ-ಭಾ-ಷೆ-ಯಲ್ಲಿ
ಪರಮ
ಪರ-ಮ-ಗುರಿ
ಪರ-ಮ-ಪ-ವಿ-ತ್ರ-ವಾದ
ಪರ-ಮಾ-ತ್ಮ-ನನ್ನು
ಪರ-ಮಾ-ತ್ಮ-ಸ್ಪ-ರ್ಶ-ದಿಂದ
ಪರ-ಸ್ಪರ
ಪರಾ-ಕ್ರ-ಮ-ಗಳನ್ನು
ಪರಾ-ಯ-ಣೆ-ಯ-ರಾದ
ಪರಿ-ಚಿ-ತ-ವಾದ
ಪರಿ-ಣಾಮ
ಪರಿ-ಣಾ-ಮ-ಗೊ-ಳಿಸಿ
ಪರಿ-ಣಾ-ಮ-ವಾಗಿ
ಪರಿ-ಣಾ-ಮ-ವಾ-ಗಿಯೇ
ಪರಿ-ಣಾ-ಮವೇ
ಪರಿ-ಪಾ-ಲಿ-ಸು-ವುದು
ಪರಿ-ಪು-ಷ್ಠಿ-ಗೊ-ಳಿಸಿ
ಪರಿ-ಪೂ-ರ್ಣತೆ
ಪರಿ-ಪೂ-ರ್ಣ-ತೆ-ಯನ್ನು
ಪರಿ-ವ-ರ್ತನೆ
ಪರಿ-ಶುದ್ಧ
ಪರಿ-ಶು-ದ್ಧತೆ
ಪರಿ-ಶು-ದ್ಧ-ನಾ-ಗಿರ
ಪರಿ-ಶು-ದ್ಧ-ನಾ-ಗಿ-ರ-ಬೇಕು
ಪರಿ-ಹ-ರಿ-ಸ-ಲಾ-ಗು-ವು-ದಿಲ್ಲ
ಪರಿ-ಹ-ರಿ-ಸು-ವುದು
ಪರಿ-ಹಾರ
ಪರಿ-ಹಾ-ರ-ವನ್ನು
ಪರಿ-ಹಾ-ರ-ವಾಗು
ಪರೀ-ಕ್ಷಿ-ಸ-ಬೇ-ಕೆಂದು
ಪರೀ-ಕ್ಷಿ-ಸಿ-ದನು
ಪರೀ-ಕ್ಷಿ-ಸು-ತ್ತಿ-ರು-ವಾಗ
ಪರೀಕ್ಷೆ
ಪರೀ-ಕ್ಷೆ-ಮಾ-ಡಿದ
ಪರ್ಣ-ಶಾ-ಲೆಗೆ
ಪರ್ವ-ತೋ-ಪಮ
ಪಲಾ-ಯ-ನ-ವಾ-ಗು-ವುದು
ಪಳ-ಗಿದ
ಪವಿತ್ರ
ಪವಿ-ತ್ರ-ತಮ
ಪವಿ-ತ್ರತಾ
ಪವಿ-ತ್ರತೆ
ಪವಿ-ತ್ರ-ತೆ-ಗಿಂತ
ಪವಿ-ತ್ರ-ವಾ-ದದ್ದು
ಪವಿ-ತ್ರ-ವಾ-ದುದು
ಪವಿ-ತ್ರಾ-ತ್ಮ-ರಾದ
ಪಾಠ-ಶಾ-ಲೆ-ಯನ್ನು
ಪಾತಿ-ವ್ರತ್ಯ
ಪಾತಿ-ವ್ರ-ತ್ಯ-ವೆಂದರೆ
ಪಾತ್ರ
ಪಾದ-ದ-ಡಿ-ಯಲ್ಲಿ
ಪಾಪ
ಪಾಪ-ಕಾ-ರ್ಯ-ಗಳನ್ನೂ
ಪಾಪಕ್ಕೂ
ಪಾಪ-ಗಳನ್ನು
ಪಾಪ-ದೂ-ರ-ನಾ-ಗಿ-ರ-ಬೇಕು
ಪಾಪವೆ
ಪಾರ-ದ-ರ್ಶ-ಕ-ವ-ನ್ನಾಗಿ
ಪಾರ-ಮಾ-ರ್ಥಿ-ಕ-ವಾ-ಗಲಿ
ಪಾರಾ-ಯಣ
ಪಾಲಿ
ಪಾಲಿಗೆ
ಪಾಲಿ-ನಷ್ಟು
ಪಾಲಿಸಿ
ಪಾಲಿ-ಸು-ವು-ದೊಂದೇ
ಪಾಲು
ಪಾವ-ನ-ಮಾಡಿ
ಪಾಶ-ದಲ್ಲಿ
ಪಾಶ್ಚಾತ್ಯ
ಪೀಠ-ಗಳಲ್ಲಿ
ಪೀಪಾಯಿ
ಪೀಪಾ-ಯಿ-ಯಾಗಿ
ಪೀಪಾ-ಯಿ-ಯೊ-ಳಗೆ
ಪುಟ-ಗಳು
ಪುಣ್ಯ
ಪುಣ್ಯ-ವಂ-ತರು
ಪುತ್ರ-ರಿಗೆ
ಪುನಃ
ಪುನ-ರಾ-ವ-ರ್ತ-ನೆ-ಯಾದ
ಪುನ-ರು-ಜ್ಜೀ-ವ-ನ-ಗೊ-ಳಿ-ಸ-ಬೇಕು
ಪುನೀ-ತ-ರ-ನ್ನಾಗಿ
ಪುರಾ-ಣ-ಗಳು
ಪುರುಷ
ಪುರು-ಷ-ನ-ನ್ನಾಗಿ
ಪುರು-ಷನೂ
ಪುರು-ಷ-ರಿಗೂ
ಪುರು-ಷರು
ಪುರು-ಷ-ಸಿಂ-ಹ-ರ-ನ್ನಾಗಿ
ಪುರು-ಷ-ಸಿಂ-ಹ-ರನ್ನು
ಪುಷಿ-ಗಳ
ಪುಷ್ಟಿ-ಗೊ-ಳಿ-ಸ-ಬೇಕು
ಪುಷ್ಟಿ-ಗೊ-ಳಿಸು
ಪುಸ್ತ-ಕ-ಗಳನ್ನು
ಪುಸ್ತ-ಕ-ಗಳಿಂದ
ಪುಸ್ತ-ಕ-ದೊಂ-ದಿಗೆ
ಪುಸ್ತ-ಕ-ಭಂ-ಡಾರ
ಪುಸ್ತ-ಕ-ಭಂ-ಡಾ-ರ-ಗಳೇ
ಪುಸ್ತ-ಕ-ಭಂ-ಡಾ-ರ-ವನ್ನೇ
ಪುಸ್ತ-ಕ-ವ-ನ್ನೆಲ್ಲ
ಪುಸ್ತ-ಕವೂ
ಪೂಜಿ-ಸಲಿ
ಪೂಜಿ-ಸ-ಲ್ಪ-ಡು-ತ್ತಿರು
ಪೂಜಿಸಿ
ಪೂಜಿ-ಸು-ವೆನು
ಪೂಜೆ
ಪೂಜೆಯ
ಪೂಜೆ-ಯಂತೆ
ಪೂಜೆ-ಯನ್ನು
ಪೂರ್ವಿ-ಕರ
ಪೂರ್ವಿ-ಕರು
ಪೆಟ್ಟನ್ನು
ಪೆಟ್ಟಿನ
ಪೆಟ್ಟಿ-ನಂತೆ
ಪೆಟ್ಟು
ಪೋಷಿ-ಸದೆ
ಪೋಷಿ-ಸ-ಬ-ಹುದು
ಪೋಷಿಸಿ
ಪೌರೋ-ಹಿತ್ಯ
ಪ್ಯಾಟನು
ಪ್ಯಾಟ್
ಪ್ರಕಾ-ಶ-ಕ-ರಿಗೆ
ಪ್ರಕಾ-ಶ-ಕರು
ಪ್ರಕಾ-ಶಕ್ಕೆ
ಪ್ರಕೃತಿ
ಪ್ರಕೃ-ತಿಯ
ಪ್ರಕೃ-ತಿ-ಯೆಲ್ಲ
ಪ್ರಖ್ಯಾ-ತ-ನಾದ
ಪ್ರಖ್ಯಾ-ತ-ರಾದ
ಪ್ರಖ್ಯಾ-ತ-ಳಾ-ಗಿ-ದ್ದಳು
ಪ್ರಚಂಡ
ಪ್ರಚಂ-ಡ-ವಾದ
ಪ್ರಚೋ-ದಿ-ಸು-ವು-ದಲ್ಲ
ಪ್ರತಿ
ಪ್ರತಿ-ದಿ-ನವೂ
ಪ್ರತಿ-ನಿ-ಧಿ-ಯಾ-ಗಿ-ರು-ವರು
ಪ್ರತಿ-ಫಲ
ಪ್ರತಿ-ಫ-ಲಕ್ಕೆ
ಪ್ರತಿ-ಫ-ಲ-ವನ್ನು
ಪ್ರತಿಭೆ
ಪ್ರತಿ-ಯೊಂ-ದ-ರ-ಲ್ಲಿಯೂ
ಪ್ರತಿ-ಯೊಂದು
ಪ್ರತಿ-ಯೊಬ್ಬ
ಪ್ರತಿ-ಯೊ-ಬ್ಬ-ನ-ಲ್ಲಿಯೂ
ಪ್ರತಿ-ಯೊ-ಬ್ಬರ
ಪ್ರತಿ-ಯೊ-ಬ್ಬ-ರನ್ನೂ
ಪ್ರತಿ-ಯೊ-ಬ್ಬರೂ
ಪ್ರತಿ-ಶ್ಲೋ-ಕವೂ
ಪ್ರತ್ಯಕ್ಷ
ಪ್ರಥಮ
ಪ್ರಪಂಚ
ಪ್ರಪಂ-ಚಕ್ಕೆ
ಪ್ರಪಂ-ಚದ
ಪ್ರಪಂ-ಚ-ದಲ್ಲಿ
ಪ್ರಪಂ-ಚ-ದ-ಲ್ಲಿ-ರುವ
ಪ್ರಪಂ-ಚ-ದಲ್ಲೆಲ್ಲ
ಪ್ರಪಂ-ಚ-ವನ್ನೇ
ಪ್ರಪಂ-ಚವೇ
ಪ್ರಬಲ
ಪ್ರಬ-ಲ-ವಾ-ದಷ್ಟೂ
ಪ್ರಭಾ-ವ-ವನ್ನು
ಪ್ರಭಾ-ವ-ವ-ನ್ನೆಲ್ಲ
ಪ್ರಭು-ವಾ-ಗು-ವು-ದಿಲ್ಲ
ಪ್ರಯತ್ನ
ಪ್ರಯ-ತ್ನ-ದಿಂದ
ಪ್ರಯ-ತ್ನ-ಪ-ಟ್ಟರೂ
ಪ್ರಯ-ತ್ನ-ವೆಲ್ಲ
ಪ್ರಯತ್ನಿ
ಪ್ರಯ-ತ್ನಿ-ಸ-ಬೇಕು
ಪ್ರಯ-ತ್ನಿ-ಸುವ
ಪ್ರಯ-ತ್ನಿ-ಸು-ವರು
ಪ್ರಯ-ತ್ನಿ-ಸು-ವುದೇ
ಪ್ರಯೋ-ಗ-ಶಾ-ಲೆ-ಯಲ್ಲಿ
ಪ್ರಯೋ-ಜನ
ಪ್ರಯೋ-ಜ-ನ-ಕಾರಿ
ಪ್ರಯೋ-ಜ-ನ-ವಿ-ದೆಯೆ
ಪ್ರಯೋ-ಜ-ನ-ವಿಲ್ಲ
ಪ್ರವಾಹ
ಪ್ರವಿ-ಚ-ಲಂತಿ
ಪ್ರವೃತ್ತಿ
ಪ್ರವೇ-ಶ-ಮಾಡಿ
ಪ್ರವೇಶಿ
ಪ್ರವೇ-ಶಿಸು
ಪ್ರಶ್ನಿ-ಸು-ವಳು
ಪ್ರಶ್ನೆ-ಯನ್ನು
ಪ್ರಸಂ-ಗ-ಗಳಲ್ಲಿ
ಪ್ರಸಿದ್ಧ
ಪ್ರಾಚೀನ
ಪ್ರಾಣ
ಪ್ರಾಣ-ವನ್ನು
ಪ್ರಾಣಿ
ಪ್ರಾಣಿ-ಗಳನ್ನು
ಪ್ರಾಣಿ-ಗ-ಳಿಗೆ
ಪ್ರಾಣಿಗೂ
ಪ್ರಾಣಿಯು
ಪ್ರಾಣಿಯೆ
ಪ್ರಾಧಾನ್ಯ
ಪ್ರಾಧ್ಯಾನ್ಯ
ಪ್ರಾಮು-ಖ್ಯತೆ
ಪ್ರಾಯ-ಶ್ಚಿತ್ತ
ಪ್ರಾರ್ಥಿ
ಪ್ರಾರ್ಥಿಸಿ
ಪ್ರಾರ್ಥಿ-ಸಿದೆ
ಪ್ರಾರ್ಥಿ-ಸುವ
ಪ್ರಾರ್ಥಿ-ಸೋಣ
ಪ್ರೀತಿ
ಪ್ರೀತಿಸಿ
ಪ್ರೀತಿ-ಸು-ತ್ತಿ-ದ್ದರು
ಪ್ರೀತಿ-ಸು-ತ್ತೇನೆ
ಪ್ರೇಮ
ಪ್ರೇರೇ-ಪ-ಣೆ-ಗಳು
ಪ್ರೇರೇ-ಪ-ಣೆ-ಯನ್ನು
ಪ್ರೇರೇ-ಪಿ-ಸಿ-ದಂತೆ
ಪ್ರೇರೇ-ಪಿಸು
ಪ್ರೇರೇ-ಪಿ-ಸು-ತ್ತವೆ
ಪ್ರೋತ್ಸಾ-ಹಿಸಿ
ಪ್ರೋತ್ಸಾ-ಹಿ-ಸಿ-ದರೆ
ಫಲ
ಬಂಡೆ-ಯ-ಮೇಲೆ
ಬಂತು
ಬಂದ
ಬಂದಂತೆ
ಬಂದವು
ಬಂದಾಗ
ಬಂದಿ-ಗ-ಳಾ-ಗು-ವೆವು
ಬಂದಿತು
ಬಂದಿದೆ
ಬಂದಿ-ದೆಯೇ
ಬಂದಿದ್ದು
ಬಂದಿ-ರು-ವೆವು
ಬಂದು
ಬಂದೇ
ಬಂದೊ-ಡ-ನೆಯೇ
ಬಂಧಿಸಿ
ಬಗೆ-ಹರಿ
ಬಗೆ-ಹ-ರಿಸಿ
ಬಟ್ಟೆ
ಬಟ್ಟೆ-ಯನ್ನು
ಬಡ-ಜ-ನರ
ಬಡ-ತನ
ಬಡ-ವರ
ಬಡ-ವ-ರಿಗೆ
ಬದ-ಲಾಯಿ
ಬದ-ಲಾ-ಯಿ-ಸ-ಬೇಕು
ಬದ-ಲಾ-ಯಿ-ಸಲು
ಬದ-ಲಾ-ಯಿ-ಸಿತು
ಬದ-ಲಾ-ಯಿ-ಸಿದೆ
ಬದ-ಲಾ-ಯಿ-ಸಿರು
ಬದಲು
ಬದು-ಕಿ-ರುವ
ಬದು-ಕಿ-ರು-ವ-ತ-ನಕ
ಬದು-ಕಿ-ರು-ವು-ದಕ್ಕೆ
ಬದ್ಧ-ರೆಂದು
ಬಯಕೆ
ಬಯ-ಕೆ-ಯನ್ನು
ಬಯ-ಸಿ-ದರೂ
ಬಯಸು
ಬಯ-ಸುವ
ಬಯ-ಸು-ವುದು
ಬರ-ದಂತೆ
ಬರದೆ
ಬರದೇ
ಬರ-ಬ-ಹು-ದಾದ
ಬರ-ಬೇ-ಕಾ-ದರೆ
ಬರ-ಬೇಕು
ಬರ-ಲಾ-ರದು
ಬರಲಿ
ಬರ-ಲಿ-ರುವ
ಬರ-ಲಿಲ್ಲ
ಬರು
ಬರುವ
ಬರು-ವಂತೆ
ಬರು-ವ-ತ-ನಕ
ಬರುವು
ಬರು-ವು-ದಿಲ್ಲ
ಬರು-ವುದು
ಬರು-ವು-ದು-ಎ-ನ್ನು-ವುದೇ
ಬರು-ವುದೊ
ಬರು-ವುವು
ಬರೆದು
ಬರೆ-ಯು-ವಂತೆ
ಬಲ-ಕಾರಿ
ಬಲ-ತ್ಕಾರ
ಬಲ-ಪ-ಡಿ-ಸು-ವುದು
ಬಲ-ವಾದ
ಬಲ-ವಾ-ದರೆ
ಬಲ-ಶಾ-ಲಿ-ಗ-ಳ-ನ್ನಾಗಿ
ಬಲಾ-ಢ್ಯ-ರ-ನ್ನಾಗಿ
ಬಲಾ-ಢ್ಯ-ರಾ-ಗ-ಬೇಕು
ಬಲಾ-ತ್ಕ-ರಿ-ಸು-ವು-ದಕ್ಕೆ
ಬಲಾ-ತ್ಕಾ-ರಕ್ಕೆ
ಬಲಾ-ತ್ಕಾ-ರ-ಪ-ಡಿ-ಸು-ವು-ದಕ್ಕೆ
ಬಲಿ-ಕೊ-ಡಲು
ಬಲೆ-ಯನ್ನು
ಬಲ್ಲ
ಬಳಿ-ದು-ಕೊಂಡು
ಬಹಳ
ಬಹಿ-ಷ್ಕಾ-ರ-ವಿ-ಲ್ಲದೆ
ಬಹುದು
ಬಹು-ಪಾಲು
ಬಹುಶಃ
ಬಾ
ಬಾಗಿ-ಲನ್ನು
ಬಾಗಿ-ಲು-ಗ-ಳೆ-ಲ್ಲ-ವನ್ನೂ
ಬಾಗು-ವೆನು
ಬಾಧಿ-ಸು-ತ್ತಿ-ರುವ
ಬಾಧ್ಯ-ತೆ-ಗಳನ್ನು
ಬಾರ-ದ-ವರು
ಬಾರ-ದ-ವ-ರೆಂದು
ಬಾಲ-ರನ್ನು
ಬಾಲ್ಯ-ದಿಂ-ದಲೂ
ಬಾಲ್ಯ-ದಿಂ-ದಲೆ
ಬಾಳ
ಬಾಳನ್ನು
ಬಾಹು-ಬಲ
ಬಾಹು-ವಿನ
ಬಾಹ್ಯ
ಬಾಹ್ಯ-ಸಂ-ಪತ್ತು
ಬಿಡದೆ
ಬಿಡ-ಬೇಕು
ಬಿಡು-ವುದು
ಬಿಡೆಂದು
ಬಿರು-ಗಾ-ಳಿಯ
ಬಿಸಿ-ರಕ್ತ
ಬೀಗದ
ಬೀಜ
ಬೀಜಕ್ಕೆ
ಬೀರ-ಬೇಕು
ಬೀರಿ-ದರೆ
ಬೀರು
ಬೀರು-ತ್ತವೆ
ಬೀರು-ವುವು
ಬೀಳ-ಬೇಕು
ಬೀಳುವ
ಬೀಳು-ವ-ವರೆಲ್ಲ
ಬೀಸಿ
ಬೀಸು-ತ್ತಿ-ರು-ವಾಗ
ಬುದ್ಧ-ದೇ-ವನೆ
ಬುದ್ಧನ
ಬುದ್ಧಿ
ಬುದ್ಧಿಗೂ
ಬುದ್ಧಿಯ
ಬುದ್ಧಿ-ಯನ್ನು
ಬುದ್ಧಿ-ಯ-ಲ್ಲೇ-ನಿದೆ
ಬುದ್ಧಿ-ವಂ-ತನೂ
ಬುದ್ಧಿ-ವಂ-ತ-ರಾದ
ಬುದ್ಧಿ-ವಂ-ತಿಕೆ
ಬುದ್ಧಿ-ವಂ-ತಿ-ಕೆ-ಯ-ನ್ನೆಲ್ಲ
ಬುದ್ಧಿ-ವಾದ
ಬುದ್ಧಿ-ವಾ-ದ-ವನ್ನು
ಬುದ್ಧಿ-ಶಕ್ತಿ
ಬುದ್ಧಿ-ಶ-ಕ್ತಿಯ
ಬುದ್ಧಿ-ಶ-ಕ್ತಿಯು
ಬೂದಿ
ಬೃಂದಾ-ವನ
ಬೆಂಕಿ
ಬೆಂಕಿಯ
ಬೆಂಕಿ-ಯಂತೆ
ಬೆಟ್ಟ
ಬೆಟ್ಟದ
ಬೆಟ್ಟ-ದಷ್ಟು
ಬೆಲೆ
ಬೆಲೆ-ಯನ್ನೂ
ಬೆಳ-ಕನ್ನು
ಬೆಳ-ಕಿನ
ಬೆಳ-ಕಿಲ್ಲ
ಬೆಳಕು
ಬೆಳಕೆ
ಬೆಳ-ಗು-ತ್ತಾರೆ
ಬೆಳ-ಗು-ತ್ತಿ-ದೆಯೋ
ಬೆಳ-ವ-ಣಿ-ಗೆಗೂ
ಬೆಳ-ವ-ಣಿ-ಗೆಗೆ
ಬೆಳೆ-ದಿದೆ
ಬೆಳೆದು
ಬೆಳೆಯ
ಬೆಳೆ-ಯ-ಬೇ-ಕೆಂದು
ಬೆಳೆಯು
ಬೆಳೆ-ಯುವ
ಬೆಳೆ-ಯು-ವಂತೆ
ಬೆಳೆ-ಯು-ವು-ದಕ್ಕೆ
ಬೆಳೆ-ಯು-ವುದು
ಬೆಳೆ-ಸು-ವು-ದಕ್ಕೆ
ಬೆಸ್ತ-ನಾ-ಗು-ವನು
ಬೆಸ್ತನು
ಬೆಸ್ತ-ರಲ್ಲಿ
ಬೆಸ್ತರು
ಬೇಕಾ
ಬೇಕಾ-ಗಿತ್ತು
ಬೇಕಾ-ಗಿದೆ
ಬೇಕಾ-ಗಿ-ರು-ವುದು
ಬೇಕಾ-ಗಿಲ್ಲ
ಬೇಕಾದ
ಬೇಕಾ-ದರೂ
ಬೇಕಾ-ದಷ್ಟು
ಬೇಕು
ಬೇಕೆಂ-ತಲೇ
ಬೇಕೊ
ಬೇಗ
ಬೇಡಿ
ಬೇಡಿ-ಕೆ-ಯನ್ನು
ಬೇರೆ
ಬೇರೆ-ಯ-ವರ
ಬೇಲಿ-ಯನ್ನು
ಬೈಬಲ್ಲು
ಬೋಧ-ಕ-ರಿಂದ
ಬೋಧಿ
ಬೋಧಿಸ
ಬೋಧಿ-ಸ-ಕೂ-ಡದು
ಬೋಧಿ-ಸ-ಬೇಕು
ಬೋಧಿ-ಸಲು
ಬೋಧಿಸಿ
ಬೋಧಿ-ಸಿ-ಕೊ-ಳ್ಳು-ವುದು
ಬೋಧಿ-ಸಿ-ದ್ದರೆ
ಬೋಧಿ-ಸು-ತ್ತಿ-ರುವ
ಬೋಧಿ-ಸು-ವು-ದ-ಕ್ಕಿಂತ
ಬೋಧಿ-ಸು-ವು-ದಕ್ಕೆ
ಬೋಧಿ-ಸು-ವುದೇ
ಬೌದ್ಧರ
ಬೌದ್ಧಿಕ
ಬ್ಬರನ್ನು
ಬ್ರಹ್ಮ
ಬ್ರಹ್ಮ-ಚರ್ಯ
ಬ್ರಹ್ಮ-ಚ-ರ್ಯದ
ಬ್ರಹ್ಮ-ಚ-ರ್ಯ-ನಿ-ಯಮ
ಬ್ರಹ್ಮ-ಚ-ರ್ಯ-ವನ್ನು
ಬ್ರಹ್ಮ-ಚ-ರ್ಯ-ವಿ-ಲ್ಲದೆ
ಬ್ರಹ್ಮ-ಚಾರಿ
ಬ್ರಹ್ಮ-ಚಾ-ರಿ-ಗ-ಳಾ-ಗಿ-ದ್ದರು
ಬ್ರಹ್ಮ-ಚಾ-ರಿಣಿ
ಬ್ರಹ್ಮ-ಚಾ-ರಿ-ಣಿ-ಗ-ಳಾ-ಗ-ಬೇ-ಕಾ-ಗಿದೆ
ಬ್ರಹ್ಮ-ಚಾ-ರಿ-ಣಿ-ಯ-ರಾ-ಗಿದ್ದು
ಬ್ರಹ್ಮ-ಚಾ-ರಿ-ಣಿ-ಯರು
ಬ್ರಹ್ಮ-ಚಾ-ರಿ-ಯಾ-ಗಿ-ರು-ವು-ದು-ಇದೇ
ಬ್ರಹ್ಮ-ಜ್ಞಾ-ನ-ವನ್ನು
ಬ್ರಹ್ಮನ
ಬ್ರಹ್ಮ-ಮ-ಯ-ವೆಂ-ಬುದು
ಬ್ರಾಹ್ಮಣ
ಭಂಗ
ಭಕ್ತ
ಭಕ್ತಿ
ಭಕ್ತಿ-ಗಳು
ಭಗ-ವಂತ
ಭಗ-ವಂ-ತನ
ಭಗ-ವಾನ್
ಭರತ
ಭರ-ತ-ಖಂ-ಡದ
ಭರ-ತ-ಖಂ-ಡ-ದಲ್ಲಿ
ಭರ-ವ-ಸೆ-ಯಿತ್ತು
ಭರ್ತೃ-ಹರಿ
ಭಾಗ
ಭಾಗಕ್ಕೆ
ಭಾಗ-ಗಳು
ಭಾಗ-ದಿಂದ
ಭಾಗ-ವನ್ನು
ಭಾಗ್ಯ
ಭಾರ-ತ-ದಲ್ಲಿ
ಭಾರ-ತಾ-ವ-ನಿಯ
ಭಾರ-ತೀಯ
ಭಾರ-ತೀ-ಯರ
ಭಾರ-ತೀ-ಯ-ರಿ-ಗಾಗಿ
ಭಾರ-ತೀ-ಯ-ರಿಗೆ
ಭಾವ
ಭಾವಕ್ಕೂ
ಭಾವದ
ಭಾವನೆ
ಭಾವ-ನೆ-ಗಳನ್ನು
ಭಾವ-ನೆ-ಗಳನ್ನೆಲ್ಲ
ಭಾವ-ನೆ-ಗಳೂ
ಭಾವ-ನೆ-ಯಂತೆ
ಭಾವ-ನೆ-ಯನ್ನು
ಭಾವ-ನೆಯೇ
ಭಾವ-ವನ್ನು
ಭಾವ-ಶುದ್ಧಿ
ಭಾವ-ಸಾ-ಧನೆ
ಭಾವಿ-ಸ-ಬ-ಲ್ಲಿರಾ
ಭಾವಿ-ಸಿ-ದ್ದರು
ಭಾವಿ-ಸು-ವು-ದ-ರಿಂದ
ಭಾಷಾಂ-ತ-ರ-ಮಾಡಿ
ಭಾಷಾ-ಶಾ-ಸ್ತ್ರ-ಗಳು
ಭಾಷೆ
ಭಾಷೆಯ
ಭಾಷೆ-ಯನ್ನು
ಭಾಷೆ-ಯಲ್ಲಿ
ಭಿನ್ನ-ವಾ-ದು-ವು-ಗ-ಳಲ್ಲ
ಭೂಗೋಳ
ಭೂತವೂ
ಭೂಪ-ಟ-ಗ-ಳೊಂ-ದಿಗೆ
ಭೂಮಿಯ
ಭೂಮಿ-ಯನ್ನು
ಭೇದ-ಭಾವ
ಭೇದ-ಭಾ-ವಕ್ಕೆ
ಭೇದಿಸಿ
ಭೋಗದ
ಭ್ರಾಂತಿ-ಯಿಂದ
ಮಂತಾ-ದು-ವು-ಗಳನ್ನು
ಮಂತ್ರ-ಗಳನ್ನು
ಮಂತ್ರ-ದಿಂದ
ಮಂದಿ
ಮಕ್ಕ-ಳಂ-ತೆಯೇ
ಮಕ್ಕ-ಳನ್ನು
ಮಕ್ಕ-ಳಿಗೆ
ಮಕ್ಕಳು
ಮಕ್ಕ-ಳು-ಗಳಿಂದ
ಮಕ್ಕಳೂ
ಮಗು
ಮಗು-ವಿ-ಗಾ-ದರೂ
ಮಗು-ವಿಗೆ
ಮಗು-ವಿನ
ಮಗುವೂ
ಮಟ್ಟಕ್ಕೆ
ಮಟ್ಟಿಗೆ
ಮಠ-ಗಳಲ್ಲಿ
ಮತ-ಭ್ರಾಂತಿ
ಮತ್ತಷ್ಟು
ಮತ್ತಾವ
ಮತ್ತು
ಮತ್ತು-ಅ-ದರ
ಮತ್ತು-ಇ-ನ್ನೊ-ಬ್ಬ-ರನ್ನು
ಮತ್ತೆ
ಮತ್ತೆಲ್ಲೊ
ಮತ್ತೇ-ನನ್ನೂ
ಮತ್ತೇನೂ
ಮತ್ತೊಂದು
ಮತ್ತೊಬ್ಬ
ಮತ್ತೊ-ಬ್ಬ-ನಿಗೂ
ಮತ್ತೊ-ಬ್ಬನು
ಮತ್ತೊ-ಬ್ಬರ
ಮತ್ತೊ-ಬ್ಬ-ರಿಂದ
ಮತ್ತೊ-ಬ್ಬ-ರಿಗೆ
ಮತ್ತೊಮ್ಮೆ
ಮದು-ವೆ-ಯಾ-ಗು-ವಂತೆ
ಮಧ್ಯ-ದ-ಲ್ಲಿ-ರುವ
ಮಧ್ಯ-ದಿಂ-ದಲೇ
ಮನದ
ಮನ-ಮು-ಟ್ಟು-ವಂತೆ
ಮನ-ಶ್ಶಾ-ಸ್ತ್ರೀಯ
ಮನ-ಷ್ಯ-ನಿಗೂ
ಮನಸಾ
ಮನ-ಸ್ಸನ್ನು
ಮನ-ಸ್ಸಾ-ಗಲೀ
ಮನಸ್ಸಿ
ಮನ-ಸ್ಸಿಗೆ
ಮನ-ಸ್ಸಿನ
ಮನ-ಸ್ಸಿ-ನಲ್ಲಿ
ಮನ-ಸ್ಸಿ-ನ-ಲ್ಲಿತ್ತು
ಮನ-ಸ್ಸಿ-ನ-ಲ್ಲಿದೆ
ಮನ-ಸ್ಸಿ-ನ-ಲ್ಲಿದ್ದ
ಮನ-ಸ್ಸಿ-ನ-ಲ್ಲಿ-ದ್ದರೆ
ಮನ-ಸ್ಸಿ-ನ-ಲ್ಲಿ-ರುವ
ಮನಸ್ಸು
ಮನ-ಸ್ಸು-ಗಳನ್ನು
ಮನು
ಮನುಷ್ಯ
ಮನು-ಷ್ಯ-ತ್ವ-ದ-ವ-ರೆಗೆ
ಮನು-ಷ್ಯನ
ಮನು-ಷ್ಯ-ನನ್ನು
ಮನು-ಷ್ಯ-ನಲ್ಲಿ
ಮನು-ಷ್ಯ-ನ-ಲ್ಲಿದೆ
ಮನು-ಷ್ಯ-ನಾ-ಗಲೀ
ಮನು-ಷ್ಯ-ನಿಗೂ
ಮನು-ಷ್ಯನು
ಮನು-ಷ್ಯ-ರನ್ನು
ಮನು-ಷ್ಯ-ರಿಂದ
ಮನು-ಷ್ಯರು
ಮನೆ
ಮನೆಗೂ
ಮನೆಗೆ
ಮನೆ-ಯಲ್ಲಿ
ಮನೆ-ಯಿಂದ
ಮನೋ
ಮನೋ-ವಾ-ಕ್ಕಾ-ಯ-ವಾಗಿ
ಮಮತೆ
ಮರ
ಮರ-ಣ-ಕ್ಕಿಂತ
ಮರ-ಣ-ಮಸ್ತು
ಮರು
ಮರು-ಗು-ತ್ತಿ-ದ್ದಿತೋ
ಮರು-ಗು-ವುದು
ಮರು-ಮಾ-ತಿ-ಲ್ಲದೆ
ಮರೆ-ತಿ-ರು-ವರು
ಮರೆ-ತಿ-ರು-ವಿರಾ
ಮರೆತು
ಮರೆ-ಯ-ದಂತೆ
ಮರೆ-ಯು-ವನು
ಮರೆ-ಯು-ವುದು
ಮಲಗಿ
ಮಸೀದಿ
ಮಹತ್
ಮಹ-ತ್ಕಾ-ರ್ಯ-ಸಾ-ಧ-ನೆಗೆ
ಮಹ-ನೀ-ಯರ
ಮಹ-ಮ್ಮ-ದನ
ಮಹ-ಮ್ಮ-ದನು
ಮಹಾ
ಮಹಾ-ಕಾ-ರ್ಯ-ಗಳು
ಮಹಾ-ಕಾ-ರ್ಯ-ಸಾ-ಧ-ನೆಗೆ
ಮಹಾ-ಗಣಿ
ಮಹಾ-ತ್ಮನು
ಮಹಾ-ತ್ಮನೂ
ಮಹಾ-ತ್ಮ-ನೆಂದು
ಮಹಾ-ತ್ಮರ
ಮಹಾ-ತ್ಮ-ರಾ-ಗಿರು
ಮಹಾ-ತ್ಮರು
ಮಹಾ-ತ್ಮೆ-ಯಿಂದ
ಮಹಾ-ಪಾಪ
ಮಹಾ-ಪು-ರುಷ
ಮಹಾ-ಪು-ರು-ಷನ
ಮಹಾ-ಪು-ರು-ಷರು
ಮಹಾ-ಪು-ಷಿ-ಗ-ಳಾ-ಗು-ತ್ತಿ-ದ್ದವು
ಮಹಾ-ಮ-ಹಿ-ಮ-ಳಾದ
ಮಹಾ-ಮು-ನಿ-ಗ-ಳಾ-ಗು-ತ್ತಿ-ದ್ದವು
ಮಹಾ-ವಾ-ಣಿ-ಗಳು
ಮಹಾ-ವೀರ
ಮಹಾ-ವೀ-ರ-ನನ್ನು
ಮಹಾ-ವ್ಯ-ಕ್ತಿ-ಗ-ಳಾ-ಗ-ಬೇಕು
ಮಹಾ-ವ್ಯ-ಕ್ತಿ-ಗ-ಳಾ-ದರು
ಮಹಾ-ವ್ಯ-ಕ್ತಿ-ಗಳಿಂದ
ಮಹಾ-ವ್ಯ-ಕ್ತಿ-ಗಳು
ಮಹಾ-ಶಕ್ತಿ
ಮಹಾ-ಶ-ಕ್ತಿಯ
ಮಹಾ-ಶ-ಕ್ತಿ-ಯನ್ನು
ಮಹಾ-ಶ್ರದ್ಧೆ
ಮಹಾ-ಸ-ತ್ಯ-ಗಳು
ಮಹಾ-ಸ-ತ್ಯ-ವನ್ನು
ಮಹಿ-ಮೆ-ಯನ್ನು
ಮಹಿಳೆ
ಮಹೋ-ಪ-ದೇಶ
ಮಾಂಸ-ಖಂಡ
ಮಾಂಸ-ಖಂ-ಡ-ಗಳು
ಮಾಡ
ಮಾಡ-ಬಲ್ಲ
ಮಾಡ-ಬ-ಲ್ಲದು
ಮಾಡ-ಬ-ಲ್ಲಿರಿ
ಮಾಡ-ಬ-ಲ್ಲೆವು
ಮಾಡ-ಬ-ಹುದು
ಮಾಡ-ಬೇ-ಕಾದ
ಮಾಡ-ಬೇ-ಕಾ-ದರೂ
ಮಾಡ-ಬೇ-ಕಾ-ದರೆ
ಮಾಡ-ಬೇಕು
ಮಾಡ-ಬೇ-ಕೆಂದು
ಮಾಡ-ಲಾ-ರದೋ
ಮಾಡ-ಲಾ-ರವು
ಮಾಡ-ಲಾ-ರಿರಿ
ಮಾಡ-ಲಾ-ರೆವು
ಮಾಡಲು
ಮಾಡಿ
ಮಾಡಿ-ಕೊಂಡು
ಮಾಡಿ-ಕೊ-ಳ್ಳದೆ
ಮಾಡಿ-ಕೊ-ಳ್ಳ-ಬೇಕು
ಮಾಡಿ-ಕೊ-ಳ್ಳಲು
ಮಾಡಿ-ಕೊಳ್ಳಿ
ಮಾಡಿ-ಕೊಳ್ಳು
ಮಾಡಿ-ಕೊ-ಳ್ಳು-ವುದನ್ನು
ಮಾಡಿತೊ
ಮಾಡಿದ
ಮಾಡಿ-ದನು
ಮಾಡಿ-ದರು
ಮಾಡಿ-ದರೆ
ಮಾಡಿ-ದ-ವನು
ಮಾಡಿ-ದುದು
ಮಾಡಿ-ದೆಯೇ
ಮಾಡಿರು
ಮಾಡಿ-ರು-ವರೋ
ಮಾಡಿ-ರು-ವುದು
ಮಾಡಿ-ರು-ವೆನು
ಮಾಡು
ಮಾಡುತ್ತಾ
ಮಾಡು-ತ್ತಿ-ದ್ದರು
ಮಾಡು-ತ್ತಿ-ದ್ದರೆ
ಮಾಡು-ತ್ತಿ-ರಲಿ
ಮಾಡು-ತ್ತಿ-ರು-ತ್ತವೆ
ಮಾಡು-ತ್ತಿ-ರುವ
ಮಾಡು-ತ್ತಿ-ರು-ವನು
ಮಾಡು-ತ್ತಿ-ರು-ವೆವು
ಮಾಡು-ತ್ತೀರಿ
ಮಾಡು-ತ್ತೇವೆ
ಮಾಡುವ
ಮಾಡು-ವಂತೆ
ಮಾಡು-ವನು
ಮಾಡು-ವರು
ಮಾಡು-ವ-ವರು
ಮಾಡು-ವ-ವ-ರೆಗೆ
ಮಾಡು-ವಾಗ
ಮಾಡು-ವು-ದ-ಕ್ಕಾ-ಗಲಿ
ಮಾಡು-ವು-ದ-ಕ್ಕಿಂತ
ಮಾಡು-ವು-ದಕ್ಕೆ
ಮಾಡು-ವು-ದರ
ಮಾಡು-ವು-ದಿಲ್ಲ
ಮಾಡು-ವುದು
ಮಾಡು-ವು-ದೊಂದೆ
ಮಾಡು-ವುದೋ
ಮಾತ
ಮಾತ-ನಾ-ಡದೆ
ಮಾತ-ನಾ-ಡ-ಬ-ಹುದು
ಮಾತ-ನಾ-ಡಿ-ದನು
ಮಾತ-ನಾ-ಡು-ತ್ತಿ-ದ್ದನೊ
ಮಾತ-ನಾ-ಡು-ವನು
ಮಾತ-ನಾ-ಡು-ವುದು
ಮಾತ-ನಾ-ಡು-ವೆವು
ಮಾತನ್ನು
ಮಾತಿಗೆ
ಮಾತಿ-ನಲ್ಲಿ
ಮಾತು
ಮಾತು-ಗಳನ್ನು
ಮಾತು-ಗಳು
ಮಾತೃ-ಭಾ-ಷೆಯ
ಮಾತೃ-ಭೂ-ಮಿಯ
ಮಾತೆ-ಯರ
ಮಾತೆ-ಯರು
ಮಾತ್ರ
ಮಾತ್ರಕ್ಕೆ
ಮಾತ್ರ-ವಲ್ಲ
ಮಾನವ
ಮಾನ-ವನ
ಮಾನ-ವ-ನಿಗೆ
ಮಾನ-ವನು
ಮಾನ-ವ-ರಿಗೂ
ಮಾನ-ವ-ರಿಗೆ
ಮಾನ-ಸಿಕ
ಮಾನ-ಸಿ-ಕ-ಶಕ್ತಿ
ಮಾಯ
ಮಾಯ-ವಾ-ಗಿ-ದೆಯೋ
ಮಾಯ-ವಾ-ಗು-ತ್ತಿ-ದ್ದವು
ಮಾಯ-ವಾ-ಗು-ವುದನ್ನು
ಮಾಯಾ
ಮಾಯಾ-ವಾ-ಗು-ವುದೇ
ಮಾರ-ಕೂ-ಡ-ದೆಂದು
ಮಾರ್ಗ
ಮಾರ್ಗ-ವನ್ನು
ಮಾರ್ಗ-ವಿರು
ಮಾರ್ಗವೇ
ಮಾಹಿ-ತಿ-ಯನ್ನು
ಮೀನನ್ನು
ಮೀರಾ-ಬಾ-ಯಿ-ಯಂ-ತಹ
ಮೀರಿದ
ಮುಂಜಾ-ದ-ರ್ಭೆ-ಯನ್ನು
ಮುಂತಾದ
ಮುಂತಾ-ದು-ವು-ಗಳನ್ನು
ಮುಂದಾ-ಳು-ಗಳು
ಮುಂದು-ವ-ರಿ-ದಿ-ರು-ವುದು
ಮುಂದು-ವ-ರಿ-ಯ-ಲಾ-ರೆವು
ಮುಂದು-ವ-ರಿ-ಯು-ವಂತೆ
ಮುಂದು-ವ-ರಿ-ಯು-ವುದು
ಮುಂದೆ
ಮುಕ್ಕಾಲು
ಮುಖ-ದಿಂದ
ಮುಖ್ಯ
ಮುಖ್ಯ-ಕಾ-ರಣ
ಮುಖ್ಯ-ವಾದ
ಮುಖ್ಯ-ವಾ-ದುದೆ
ಮುಗಿ-ಯಿ-ತೇನು
ಮುಗಿ-ಲಿ-ನಂತೆ
ಮುಚ್ಚ-ಲ್ಪ-ಟ್ಟಿ-ರು-ತ್ತದೆ
ಮುಟ್ಟ-ಬೇಕು
ಮುತ್ತಿ-ಕೊಂ-ಡಿ-ರು-ವುದೊ
ಮುದ್ದೆ-ಯಂ-ತಿ-ರುವ
ಮುಳುಗಿ
ಮುಸು-ಕನ್ನು
ಮುಸುಕು
ಮೂಡಿವೆ
ಮೂಡು-ವುವು
ಮೂಢ
ಮೂಢ-ನಂ-ಬಿ-ಕೆ-ಗಳನ್ನು
ಮೂಢ-ನಿಂದ
ಮೂಢ-ರಿ-ಗಿಂತ
ಮೂರನೇ
ಮೂರು
ಮೂರೆ-ಳೆಯ
ಮೂರ್ಖರು
ಮೂಲ
ಮೂಲಕ
ಮೂಲ-ಕವೇ
ಮೂಲ-ಮಂ-ತ್ರ-ವಾ-ಗಲಿ
ಮೂಲವೇ
ಮೂಲೆ-ಯಲ್ಲಿ
ಮೂವ-ತ್ತೆ-ರಡು
ಮೃಗಕ್ಕೂ
ಮೃದಂಗ
ಮೃದು
ಮೃದು-ವಾದ
ಮೆಟ್ಟ-ಲನ್ನು
ಮೆದು-ಳನ್ನು
ಮೆದು-ಳಿಗೆ
ಮೆದು-ಳಿ-ನಲ್ಲಿ
ಮೆದುಳು
ಮೇಧಾವಿ
ಮೇಲಕ್ಕೆ
ಮೇಲಾ-ಗುವ
ಮೇಲಿನ
ಮೇಲಿ-ರುವ
ಮೇಲು
ಮೇಲೆ
ಮೇಲೆಬ್ಬಿ
ಮೇಲೆಯೆ
ಮೇಲ್ಪಂಕ್ತಿ
ಮೇಲ್ಪಂ-ಕ್ತಿ-ಯಲ್ಲಿ
ಮೇಲ್ಮೆ-ಗಾಗಿ
ಮೈತ್ರೇಯಿ
ಮೈಸೂರು
ಮೊತ್ತ
ಮೊತ್ತ-ದಿಂದ
ಮೊತ್ತವೇ
ಮೊದಲ
ಮೊದ-ಲನೆ
ಮೊದ-ಲ-ನೆಯ
ಮೊದ-ಲ-ನೆ-ಯ-ದಾಗಿ
ಮೊದ-ಲನೇ
ಮೊದ-ಲಾ-ಯಿ-ತೆಂದು
ಮೊದಲು
ಮೊದ-ಲು-ಮಾ-ಡಿ-ದನು
ಮೊಬ್ಬಾ-ಗಿದೆ
ಮೋಕ್ಷಕ್ಕೆ
ಮೋಕ್ಷದ
ಮೌಢ್ಯ
ಮೌಢ್ಯದ
ಮ್ಯಾಜಿ-ಸ್ಟ್ರೇಟ್
ಯಂತಹ
ಯಂತೆ
ಯಂತ್ರ-ಗ-ಳಂತೆ
ಯಂತ್ರದ
ಯಂತ್ರ-ವಾ-ಗು-ತ್ತಾನೆ
ಯತ್ನಿ-ಸ-ಬೇಡಿ
ಯತ್ನಿ-ಸಿ-ದರೆ
ಯಥೇಷ್ಟಂ
ಯದಾಗಿ
ಯದಿ
ಯನ್ನಾಗಿ
ಯನ್ನು
ಯರ
ಯಲ್ಲದೆ
ಯಲ್ಲಿ
ಯಾಗಿ-ರ-ಬ-ಹುದು
ಯಾಜ್ಞ-ವ-ಲ್ಕ್ಯ-ರನ್ನು
ಯಾತ್ಪಥಃ
ಯಾದ
ಯಾದ-ವ-ಗಿರಿ
ಯಾಯಿ-ಗಳು
ಯಾರ
ಯಾರನ್ನು
ಯಾರಲ್ಲಿ
ಯಾರಾ
ಯಾರಾ-ದರೂ
ಯಾರಿಗೆ
ಯಾರು
ಯಾರೂ
ಯಾರೊ
ಯಾರೊಂ-ದಿಗೆ
ಯಾರೊ-ಬ್ಬರ
ಯಾರೊ-ಬ್ಬರೂ
ಯಾವ
ಯಾವನು
ಯಾವಾ
ಯಾವಾ-ಗಲೂ
ಯಾವು-ದಕ್ಕೂ
ಯಾವುದನ್ನು
ಯಾವು-ದನ್ನೂ
ಯಾವು-ದರ
ಯಾವು-ದ-ರಿಂ-ದಲೂ
ಯಾವು-ದಾದ
ಯಾವು-ದಾ-ದರು
ಯಾವು-ದಾ-ದ-ರೊಂದು
ಯಾವುದು
ಯಾವುದೇ
ಯಾವುದೋ
ಯಿಂದ
ಯಿಲ್ಲದ
ಯುಕ್ತಿ-ಯ-ಲ್ಲೇ-ನಿದೆ
ಯುಗಾಂ-ತ-ರೇವಾ
ಯುತ್ತಾನೋ
ಯುದ್ಧ-ರಂ-ಗದ
ಯುವ-ಕರು
ಯೂರೋ-ಪ್ದೇ-ಶದ
ಯೊಂದಿಗೆ
ಯೊಬ್ಬರೂ
ಯೋಗದ
ಯೋಗ-ಶಾ-ಸ್ತ್ರ-ವನ್ನು
ಯೋಗಿ-ಯ-ವ-ರೆಗೆ
ಯೋಗ್ಯ-ತೆ-ಯನ್ನು
ಯೋಗ್ಯ-ರಾ-ಗಿ-ರು-ವರು
ಯೋಗ್ಯ-ರಾ-ದಂ-ತಹ
ಯೋಗ್ಯ-ರೀ-ತಿ-ಯಲ್ಲಿ
ಯೋಗ್ಯರು
ಯೋಗ್ಯ-ವಾ-ಗ-ಬೇಕು
ಯೋಚಿ-ಸು-ತ್ತಿ-ದ್ದರೆ
ಯೋಜ-ಕ-ನೆಂದೂ
ರಕ್ತ-ಗ-ತ-ವಾ-ಗದೆ
ರಕ್ತ-ಗ-ತ-ವಾಗಿ
ರಕ್ತ-ಗ-ತ-ವಾ-ಗಿ-ದೆಯೆ
ರಕ್ಷಿಸಿ
ರಜೋ-ಗುಣ
ರಣ-ಭೇ-ರಿಯ
ರತ್ನ-ಗಳನ್ನು
ರನ್ನು
ರಲ್ಲಿಯೂ
ರವಾ-ನಿ-ಸುವ
ರಸಾ-ಯನ
ರಸೋ-ದ್ದೀ-ಪಕ
ರಹಸ್ಯ
ರಹ-ಸ್ಯ-ಗಳನ್ನು
ರಹ-ಸ್ಯದ
ರಹ-ಸ್ಯ-ವನ್ನು
ರಾಗಿ-ರು-ವಿರಾ
ರಾಗಿ-ರು-ವೆವು
ರಾಗು-ವಂ-ತಹ
ರಾಜ
ರಾಜ-ಕೀಯ
ರಾಜ್ಯ-ವಾ-ಗಿದೆ
ರಾತ್ರಿ
ರಾಮ
ರಾಶಿ-ಮಾ-ಡಿ-ದರೆ
ರಾಸಾ-ಯ-ನಿಕ
ರಿಂದ
ರಿಕಾ
ರಿಗೆ
ರೀತಿ-ನೀ-ತಿ-ಗಳು
ರೀತಿಯ
ರೀತಿ-ಯ-ಲ್ಲಾ-ದರೂ
ರೀತಿ-ಯಲ್ಲಿ
ರೀತಿ-ಯ-ಲ್ಲಿಯೂ
ರೂಢಿಸಿ
ರೂಢಿ-ಸಿ-ದರೆ
ರೂಢಿ-ಸು-ವು-ದೊಂದೇ
ರೂಪಿಸಿ
ರೆಂದು
ರೇಷ್ಮೆಯ
ರೊಂದು
ರೋಗ
ರೋಗ-ದಿಂದ
ಲಕ್ಷಾಂ-ತರ
ಲಕ್ಷೀ-ಬಾಯಿ
ಲಕ್ಷ್ಮಿಯು
ಲಕ್ಷ್ಮೀ-ಸ್ಸ-ಮಾ-ವಿ-ಷತು
ಲಾಡಿ-ಸ-ಬ-ಲ್ಲದೆ
ಲಾಭ
ಲಿಲ್ಲ
ಲೀಲಾ
ಲೀಲೆಯ
ಲೆಕ್ಕ-ವಿ-ಲ್ಲ-ದಷ್ಟು
ಲೌಕಿಕ
ಲೌಕಿ-ಕ-ವಾ-ಗಲಿ
ಲ್ಲುಂಟಾ-ಗು-ತ್ತದೆ
ಳಾಗಿ
ವಂತ-ಹ-ವನೇ
ವಂತೆ
ವಂತೆಯೂ
ವಂಶಕ್ಕೆ
ವಂಶ-ಜ-ರೊಂ-ದಿಗೆ
ವಂಶ-ಪಾ-ರಂ-ಪ-ರ್ಯ-ವಾಗಿ
ವಂಶಾ-ನು-ಗ-ತ-ರಾದ
ವಂಶಾ-ನು-ಗ-ತ-ವಾಗಿ
ವಕೀಲ
ವಜಾ
ವಜ್ರೋ-ಪಮ
ವನ್ನಾ-ದರೂ
ವನ್ನು
ವನ್ನೇ
ವರು
ವರು-ಷ-ಗಳ
ವರು-ಷ-ಗಳಿಂದ
ವರು-ಷ-ಗ-ಳಿಂ-ದಲೂ
ವರೆಗೆ
ವರೊ
ವರ್ಗದ
ವರ್ಗ-ದ-ವರು
ವರ್ಣ-ದ-ವ-ರನ್ನು
ವರ್ಷ
ವರ್ಷ-ಗಳ
ವರ್ಷ-ಗ-ಳಾ-ದ-ಮೇಲೆ
ವಳು
ವವರು
ವಶ-ದ-ಲ್ಲಿ-ರು-ವನು
ವಶ-ಮಾ-ಡಿ-ಕೊಂ-ಡು-ದುದು
ವಸ್ತು
ವಸ್ತು-ಗಳ
ವಸ್ತು-ಗಳು
ವಸ್ತು-ವನ್ನು
ವಹಿ-ಸಿ-ಕೊ-ಳ್ಳ-ಬೇಕು
ವಾ
ವಾಕ್ಕಾ-ಯ-ವಾಗಿ
ವಾಕ್ಯ-ರ-ಚನೆ
ವಾಗಲಿ
ವಾಗಿ
ವಾಗಿತ್ತು
ವಾಗಿದೆ
ವಾಗಿ-ರಲಿ
ವಾಗಿರಿ
ವಾಗು-ತ್ತವೆ
ವಾಗು-ವುದು
ವಾಚಾ
ವಾಣಿಯ
ವಾಣಿ-ಯನ್ನು
ವಾತಾ-ವ-ರಣ
ವಾದ
ವಾದ-ದ-ಲ್ಲಿಯೂ
ವಾದರೂ
ವಾದುದೇ
ವಾದುವು
ವಾದ್ಯ-ಗಳ
ವಾಸ-ವಾ-ಗಿದ್ದ
ವಿಕಾ-ಸಕ್ಕೆ
ವಿಗ್ರಹ
ವಿಗ್ರ-ಹಾ-ರಾ-ಧ-ಕ-ರ-ನ್ನಾಗಿ
ವಿಚಾ-ರ-ಗಳನ್ನು
ವಿಚಾ-ರ-ಮಾ-ಡ-ಬ-ಹುದು
ವಿಚಾ-ರ-ವನ್ನು
ವಿಚಾ-ರ-ವಾಗಿ
ವಿಜ್ಞಾನ
ವಿಜ್ಞಾ-ನ-ಶಾಸ್ತ್ರ
ವಿಜ್ಞಾ-ನ-ಶಾ-ಸ್ತ್ರ-ಗಳ
ವಿಜ್ಞಾ-ನಿ-ಗಳು
ವಿದೆ
ವಿದ್ಯಾ
ವಿದ್ಯಾ-ದಾನ
ವಿದ್ಯಾ-ದಾ-ನದ
ವಿದ್ಯಾ-ಭ್ಯಾಸ
ವಿದ್ಯಾ-ಭ್ಯಾ-ಸ-ತತ್ತ್ವ
ವಿದ್ಯಾ-ಭ್ಯಾ-ಸದ
ವಿದ್ಯಾ-ಭ್ಯಾ-ಸ-ದಿಂದ
ವಿದ್ಯಾ-ಭ್ಯಾ-ಸ-ವನ್ನು
ವಿದ್ಯಾ-ಭ್ಯಾ-ಸ-ವಲ್ಲ
ವಿದ್ಯಾ-ಭ್ಯಾ-ಸ-ವಾಗಿ
ವಿದ್ಯಾ-ಭ್ಯಾ-ಸ-ವಿಲ್ಲ
ವಿದ್ಯಾ-ಭ್ಯಾ-ಸ-ವೆಂದರೆ
ವಿದ್ಯಾ-ಭ್ಯಾ-ಸವೇ
ವಿದ್ಯಾರ್ಥಿ
ವಿದ್ಯಾ-ರ್ಥಿ-ಗಳಲ್ಲಿ
ವಿದ್ಯಾ-ರ್ಥಿಯ
ವಿದ್ಯಾ-ರ್ಥಿ-ಯಾ-ಗು-ವನು
ವಿದ್ಯಾ-ವಂ-ತರ
ವಿದ್ಯಾ-ವಂ-ತ-ರಾ-ದೆ-ವೆಂದು
ವಿದ್ಯಾ-ವಂ-ತರು
ವಿದ್ಯಾ-ವಿ-ಷ-ಯ-ಗಳು
ವಿದ್ಯಾ-ವ್ಯಾ-ಸಂ-ಗದ
ವಿದ್ಯಾ-ವ್ಯಾ-ಸಂ-ಗ-ದಿಂದ
ವಿದ್ಯಾ-ಶು-ಲ್ಕ-ವನ್ನೂ
ವಿದ್ಯುತ್
ವಿದ್ಯೆ
ವಿದ್ಯೆ-ಗಳಲ್ಲಿ
ವಿದ್ಯೆ-ಯನ್ನು
ವಿದ್ಯೆ-ಯಿಂದ
ವಿದ್ಯೆ-ಯಿಲ್ಲ
ವಿದ್ರಾ-ವಕ
ವಿಧದ
ವಿಧ-ವಾದ
ವಿಧಾನ
ವಿನ
ವಿನಯ
ವಿಪ-ರೀತ
ವಿಭ-ಜನೆ
ವಿಮ-ರ್ಶೆ-ಮಾ-ಡದೆ
ವಿಮು-ಕ್ತ-ನ-ನ್ನಾಗಿ
ವಿಮೋ-ಚನೆ
ವಿರಾ
ವಿರಿ
ವಿರು-ದ್ಧ-ವಾ-ದುವು
ವಿರೋ-ಧ-ವಾಗಿ
ವಿರೋ-ಧ-ವಾದ
ವಿರೋ-ಧಿ-ಗ-ಳಾ-ದರೂ
ವಿರೋ-ಧಿಸಿ
ವಿರೋ-ಧಿ-ಸುವ
ವಿಲ್ಲ
ವಿವಾಹ
ವಿವೇಕಾನಂದ
ವಿಶಾ-ಲ-ವಾ-ಗು-ವುದೋ
ವಿಶ್ವ
ವಿಶ್ವದ
ವಿಶ್ವ-ದಲ್ಲಿ
ವಿಶ್ವ-ಧರ್ಮ
ವಿಶ್ವ-ವಿ-ದ್ಯಾ-ನಿ-ಲ-ಯದ
ವಿಶ್ವಾ-ಸ-ಘಾ-ತ-ಕ-ರೆಂದು
ವಿಷ-ದಂತೆ
ವಿಷಯ
ವಿಷ-ಯ-ಗಳ
ವಿಷ-ಯ-ಗಳನ್ನು
ವಿಷ-ಯ-ಗಳು
ವಿಷ-ಯ-ಗ-ಳೇನೋ
ವಿಷ-ಯದ
ವಿಷ-ಯ-ದಲ್ಲಿ
ವಿಷ-ಯ-ವನ್ನು
ವಿಷ-ಯ-ವಾ-ಗಲಿ
ವಿಷ-ಯವೇ
ವಿಷ-ಯ-ಸಂ-ಗ್ರಹ
ವಿಷ-ಯ-ಸಂ-ಗ್ರ-ಹ-ವಲ್ಲ
ವಿಸ್ತಾ-ರ-ವಾಗಿ
ವಿಸ್ತಾ-ರ-ವಾದ
ವಿಸ್ಮ-ಯ-ಗೊ-ಳಿ-ಸುವ
ವಿಸ್ಮಿ-ತ-ನಾಗಿ
ವಿಹಾ-ರ-ಗ-ಳಿಗೆ
ವೀರ
ವೀರ-ಧ್ವ-ನಿ-ಯನ್ನು
ವೀರ-ನಿಗೆ
ವೀರ-ಯೋ-ಧನ
ವೀರ-ರ-ಸ-ವನ್ನು
ವೀರ-ರಾಗಿ
ವುದಕ್ಕೆ
ವುದನ್ನು
ವುದ-ರಲ್ಲೇ
ವುದಿಲ್ಲ
ವುದಿ-ಲ್ಲವೆ
ವುದು
ವುದು-ಇ-ವು-ಗ-ಳನ್ನೇ
ವುದೇ
ವುದೋ
ವುವು
ವೃದ್ಧ
ವೆಂದು
ವೆವೊ
ವೆವೋ
ವೇಗ
ವೇದ
ವೇದ-ಪಾ-ರಂ-ಗ-ತ-ರಿಂದ
ವೇದ-ಶ್ಲೋ-ಕ-ಗಳ
ವೇದಾಂ-ತ-ತ-ತ್ತ್ವ-ಗಳು
ವೇದಾಂ-ತವು
ವೇದಾ-ಧ್ಯ-ಯ-ನಕ್ಕೆ
ವೇದಿ-ಕೆಯ
ವೇಳೆ
ವೈಮ-ನಸ್ಸು
ವೈರಾಗ್ಯ
ವೈರಾ-ಗ್ಯದ
ವೈಶಿ-ಷ್ಟ್ಯ-ಗಳನ್ನು
ವೈಶಿ-ಷ್ಟ್ಯ-ವನ್ನೂ
ವ್ಯಕ್ತ
ವ್ಯಕ್ತ-ಗೊ-ಳಿಸಿ
ವ್ಯಕ್ತ-ಗೊ-ಳಿಸು
ವ್ಯಕ್ತ-ಗೊ-ಳಿ-ಸುವ
ವ್ಯಕ್ತ-ಪ-ಡಿ-ಸು-ತ್ತಾನೆ
ವ್ಯಕ್ತ-ಪ-ಡಿ-ಸುವ
ವ್ಯಕ್ತ-ವಾ-ಗು-ತ್ತದೆ
ವ್ಯಕ್ತ-ವಾ-ಗುವ
ವ್ಯಕ್ತ-ವಾ-ಗು-ವಂತೆ
ವ್ಯಕ್ತ-ವಾ-ಗು-ವು-ದಕ್ಕೆ
ವ್ಯಕ್ತ-ವಾ-ಗು-ವುದನ್ನು
ವ್ಯಕ್ತ-ವಾ-ಗು-ವುದು
ವ್ಯಕ್ತ-ವಾ-ಯಿತು
ವ್ಯಕ್ತಿ
ವ್ಯಕ್ತಿ-ತ್ವ-ವನ್ನು
ವ್ಯಕ್ತಿಯ
ವ್ಯತ್ಯಾಸ
ವ್ಯತ್ಯಾ-ಸಕ್ಕೆ
ವ್ಯತ್ಯಾ-ಸ-ವಾ-ಗಿತ್ತು
ವ್ಯತ್ಯಾ-ಸವೆ
ವ್ಯತ್ಯಾ-ಸವೇ
ವ್ಯಯ-ಮಾ-ಡು-ತ್ತಿದೆ
ವ್ಯರ್ಥ-ಮಾ-ಡು-ತ್ತಿ-ರು-ವನು
ವ್ಯವ-ಹಾರ
ವ್ಯಾಪಾರ
ವ್ಯಾಸಂ-ಗ-ಮಾ-ಡು-ತ್ತಿ-ರುವ
ವ್ರತ
ವ್ರತ-ವನ್ನು
ಶಂಕಿ-ಸ-ಲಿಲ್ಲ
ಶಕ್ತ-ವಾ-ದುದು
ಶಕ್ತಿ
ಶಕ್ತಿ-ಗಳಿಂದ
ಶಕ್ತಿ-ಚಿ-ಹ್ನೆಗೆ
ಶಕ್ತಿ-ಪೂ-ಜ-ಕರು
ಶಕ್ತಿ-ಪೂಜೆ
ಶಕ್ತಿ-ಪೂ-ರಿತ
ಶಕ್ತಿಯ
ಶಕ್ತಿ-ಯನ್ನು
ಶಕ್ತಿ-ಯನ್ನೂ
ಶಕ್ತಿ-ಯ-ನ್ನೆಲ್ಲ
ಶಕ್ತಿ-ಯ-ನ್ನೆಲ್ಲಾ
ಶಕ್ತಿ-ಯ-ಲ್ಲೆಲ್ಲ
ಶಕ್ತಿ-ಯಿಂದ
ಶಕ್ತಿ-ಯು-ತ-ರಾಗಿ
ಶಕ್ತಿಯೂ
ಶಕ್ತಿಯೆ
ಶಕ್ತಿ-ಯೆಲ್ಲ
ಶಕ್ತಿಯೇ
ಶಕ್ತಿ-ಯೊಂದೇ
ಶತ-ಮಾ-ನ-ಗ-ಳಿಂ-ದಲೂ
ಶತ-ಶ-ತ-ಮಾ-ನ-ಗಳ
ಶಬ್ದ-ಗಳ
ಶಬ್ದ-ಜಾ-ಲ-ಗಳ
ಶಬ್ದ-ವೊಂದೆ
ಶಬ್ದ-ಸಾ-ಧ-ನ-ವಿದ್ಯೆ
ಶಬ್ದಾ-ಡಂ-ಬ-ರ-ದಲ್ಲೇ
ಶಮ-ನ-ಮಾಡಿ
ಶರಾ-ಣ-ಗ-ತ-ನಾ-ಗು-ವೆನು
ಶಸ್ತ್ರ-ಚಿ-ಕಿ-ತ್ಸ-ಕ-ನ-ನ್ನಾಗಿ
ಶಸ್ತ್ರ-ಚಿ-ಕಿ-ಸ್ಸೆಯ
ಶಾಂತ-ವಾ-ದಾಗ
ಶಾಂತಿ
ಶಾಂತಿ-ಮಯ
ಶಾರೀ-ರಕ
ಶಾರೀ-ರ-ಕ-ವಾ-ಗಲಿ
ಶಾಲೆಗೆ
ಶಾಶ್ವ-ತ-ವಾ-ಗ-ಲಾ-ರದು
ಶಾಸ್ತ್ರ
ಶಾಸ್ತ್ರ-ಗಳ
ಶಾಸ್ತ್ರ-ಗಳನ್ನು
ಶಾಸ್ತ್ರ-ಗಳಲ್ಲಿ
ಶಾಸ್ತ್ರ-ಗಳೂ
ಶಾಸ್ತ್ರಜ್ಞ
ಶಾಸ್ತ್ರದ
ಶಿಕ್ಷಣ
ಶಿಕ್ಷ-ಣಕ್ಕೆ
ಶಿಕ್ಷ-ಣದ
ಶಿಕ್ಷ-ಣ-ವನ್ನು
ಶಿಕ್ಷ-ಣ-ವಾ-ದರೂ
ಶಿಕ್ಷ-ಣವೂ
ಶಿಕ್ಷ-ಣ-ವೊಂದೇ
ಶಿರೋ-ಮ-ಣಿ-ಯಾಗಿ
ಶಿಲುಬೆ
ಶಿಶು-ರ-ಕ್ಷಣೆ
ಶಿಶು-ವಿ-ಹಾ-ರ-ಗಳು
ಶಿಷ್ಯ
ಶಿಷ್ಯನ
ಶಿಷ್ಯ-ನನ್ನು
ಶಿಷ್ಯ-ನಲ್ಲಿ
ಶಿಷ್ಯ-ನಿಗೆ
ಶಿಷ್ಯನು
ಶಿಷ್ಯ-ರನ್ನು
ಶಿಷ್ಯ-ರಲ್ಲಿ
ಶಿಷ್ಯ-ರಿಂದ
ಶಿಷ್ಯ-ರಿಗೆ
ಶಿಷ್ಯರು
ಶಿಸ್ತು
ಶೀಘ್ರ-ದ-ಲ್ಲಿಯೇ
ಶೀಲ
ಶೀಲಕ್ಕೆ
ಶೀಲದ
ಶೀಲ-ದ-ಮೇಲೆ
ಶೀಲ-ದಲ್ಲಿ
ಶೀಲ-ಪೋ-ಷಣೆ
ಶೀಲ-ಪೋ-ಷ-ಣೆಗೆ
ಶೀಲ-ವನ್ನು
ಶೀಲ-ವಾ-ಗಲಿ
ಶೀಲವೂ
ಶೀಲ-ವೆ-ನ್ನು-ವುದು
ಶೀಲ-ಶ-ಕ್ತಿ-ಯನ್ನು
ಶೀಲ-ಸಂ-ಪ-ನ್ನ-ರಾಗಿ
ಶೀಲ-ಸಂ-ಪ-ನ್ನ-ರಾದ
ಶುದ್ಧ
ಶುದ್ಧಾ-ತ್ಮ-ರಾಗಿ
ಶೂರ-ತ್ವ-ವನ್ನು
ಶೇಕಡ
ಶೇಖರಿ
ಶೇಖ-ರಿ-ಸ-ಲ್ಪ-ಟ್ಟಿ-ರುವ
ಶೇಖ-ರಿ-ಸಿದ
ಶೋಚ-ನೀಯ
ಶೌರ್ಯ
ಶ್ರದ್ಧೆ
ಶ್ರದ್ಧೆಯ
ಶ್ರದ್ಧೆ-ಯನ್ನು
ಶ್ರದ್ಧೆ-ಯನ್ನೂ
ಶ್ರದ್ಧೆ-ಯಲ್ಲ
ಶ್ರದ್ಧೆ-ಯಿ-ದ್ದರೆ
ಶ್ರದ್ಧೆಯೇ
ಶ್ರಮದ
ಶ್ರೀಕೃಷ್ಣ
ಶ್ರೀಕೃ-ಷ್ಣ-ಗೋ-ಪಿ-ಯರ
ಶ್ರೀಕೃ-ಷ್ಣನ
ಶ್ರೀಕೃ-ಷ್ಣ-ನನ್ನು
ಶ್ರೀಮಂ-ತ-ರಿಗೆ
ಶ್ರೀಮಂ-ತರು
ಶ್ರೀರಾ-ಮ-ಕೃಷ್ಣ
ಶ್ರೀರಾ-ಮ-ಕೃ-ಷ್ಣರು
ಶ್ರೀರಾ-ಮ-ಚಂದ್ರ
ಶ್ರೀರಾ-ಮ-ಚಂ-ದ್ರನ
ಶ್ರೀರಾ-ಮನ
ಶ್ರೀರಾ-ಮರ
ಶ್ರೇಯ-ಸ್ಸಿನ
ಶ್ರೇಷ್ಠ
ಶ್ರೇಷ್ಠ-ವಾ-ಗಿಲ್ಲ
ಸಂಗೀತ
ಸಂಗೀ-ತ-ದಲ್ಲಿ
ಸಂಗೀ-ತ-ವನ್ನು
ಸಂಗ್ರ-ಹವೇ
ಸಂಗ್ರ-ಹಿ-ಸು-ತ್ತೇನೆ
ಸಂಘ-ಮಿತ್ರ
ಸಂಘ-ರ್ಷಣೆ
ಸಂಚ-ರಿಸಿ
ಸಂಚಾರ
ಸಂದೇಶ
ಸಂದೇ-ಶ-ಕರು
ಸಂದೇ-ಹ-ವಿಲ್ಲ
ಸಂನ್ಯಾಸಿ
ಸಂನ್ಯಾ-ಸಿ-ಗಳು
ಸಂನ್ಯಾ-ಸಿ-ಯೊಂ-ದಿಗೆ
ಸಂಪಾ-ದಿಸಿ
ಸಂಪಾ-ದಿ-ಸು-ವು-ದಕ್ಕೆ
ಸಂಪೂರ್ಣ
ಸಂಪೂ-ರ್ಣ-ವಾಗಿ
ಸಂಬಂಧ
ಸಂಬಂ-ಧ-ದಂತೆ
ಸಂಬಂ-ಧ-ಪಟ್ಟ
ಸಂಬಂ-ಧ-ವನ್ನು
ಸಂಭವ
ಸಂಭ-ವ-ವಿ-ರು-ವುದು
ಸಂಭೋಗ
ಸಂಸಾ-ರ-ಧ-ರ್ಮ-ಗಳು
ಸಂಸ್ಕಾರ
ಸಂಸ್ಕಾ-ರ-ಗಳ
ಸಂಸ್ಕಾ-ರ-ಗಳನ್ನು
ಸಂಸ್ಕಾ-ರ-ಗಳಿಂದ
ಸಂಸ್ಕಾ-ರ-ಗ-ಳಿ-ದ್ದರೆ
ಸಂಸ್ಕಾ-ರ-ಗಳು
ಸಂಸ್ಕಾ-ರ-ಳೊಂ-ದಿಗೆ
ಸಂಸ್ಕಾ-ರ-ವನ್ನು
ಸಂಸ್ಕೃತ
ಸಂಸ್ಕೃ-ತ-ಪಂ-ಡಿ-ತ-ರ-ನ್ನಾಗಿ
ಸಂಸ್ಕೃ-ತ-ಭಾ-ಷೆಯ
ಸಂಸ್ಕೃ-ತವೂ
ಸಂಸ್ಕೃತಿ
ಸಂಸ್ಕೃ-ತಿಯ
ಸಂಸ್ಕೃ-ತಿ-ಯನ್ನು
ಸಂಸ್ಥೆ-ಗ-ಳಿಗೆ
ಸಚ-ರಾ-ಚರ
ಸಚೇ-ತನ
ಸಚೇ-ತ-ನ-ವಾದ
ಸಜೀ-ವ-ವಾ-ಗಿವೆ
ಸಣ್ಣ
ಸತತ
ಸತ್ಕರ್ಮ
ಸತ್ಯ
ಸತ್ಯ-ಗಳನ್ನು
ಸತ್ಯ-ಗಳು
ಸತ್ಯದ
ಸತ್ಯ-ಪ-ಥ-ದಿಂದ
ಸತ್ಯ-ವನ್ನು
ಸತ್ಯ-ವಾ-ಗ-ಲಾ-ರದು
ಸತ್ಯ-ವಾಗಿ
ಸತ್ಯ-ವಾ-ಗಿ-ದ್ದವು
ಸತ್ಯ-ವಾದ
ಸತ್ಯ-ವೆಲ್ಲ
ಸತ್ಯ-ಸಾ-ಕ್ಷಾ-ತ್ಕಾ-ರ-ಕ್ಕೋ-ಸುಗ
ಸದಾ
ಸದ್ಯಕ್ಕೆ
ಸನಾ-ತನ
ಸನ್ನಿ-ವೇ-ಶ-ಗಳು
ಸನ್ನಿ-ವೇ-ಶ-ದ-ಲ್ಲಿಯೂ
ಸನ್ನಿ-ಹಿತ
ಸಬಂ-ಧ-ಪಟ್ಟ
ಸಭೆ-ಯಲ್ಲಿ
ಸಮ-ನಾದ
ಸಮ-ನ್ವಯ
ಸಮ-ಭಾಗ
ಸಮಯ
ಸಮ-ಯ-ದಲ್ಲಿ
ಸಮ-ರ್ಥಿಸಿ
ಸಮಸ್ಯೆ
ಸಮ-ಸ್ಯೆ-ಗಳನ್ನು
ಸಮ-ಸ್ಯೆ-ಗ-ಳಿವೆ
ಸಮ-ಸ್ಯೆ-ಗ-ಳೆಲ್ಲ
ಸಮಾಜ
ಸಮಾ-ಜ-ದಲ್ಲಿ
ಸಮಾ-ಧಾ-ನ-ವುಂಟಾ
ಸಮಿ-ತ್ತನ್ನು
ಸಮೀ-ಪಕ್ಕೆ
ಸರಳ
ಸರಿ-ಯದೆ
ಸರಿ-ಯ-ಬಾ-ರದು
ಸರಿ-ಯಲ್ಲ
ಸರಿ-ಯಾ-ಗಿಲ್ಲ
ಸರಿ-ಯಾದ
ಸರಿ-ಯು-ವು-ದಿ-ಲ್ಲವೋ
ಸರಿ-ಯೆಂದು
ಸರಿ-ಸು-ತ್ತಿ-ರುವ
ಸರೋ-ವ-ರಕ್ಕೆ
ಸರೋ-ವ-ರದ
ಸರ್ವ
ಸರ್ವಜ್ಞ
ಸರ್ವ-ಜ್ಞನು
ಸರ್ವ-ತೋ-ಮು-ಖ-ರಾಗಿ
ಸರ್ವ-ನಾ-ಶದ
ಸರ್ವ-ಮಾನ್ಯ
ಸರ್ವ-ವ್ಯಾಪಿ
ಸರ್ವ-ಶ-ಕ್ತನು
ಸರ್ವ-ಶ-ಕ್ತಿ-ಯಿಂದ
ಸರ್ವಸ್ವ
ಸರ್ವ-ಸ್ವವೇ
ಸಲ
ಸಲ-ಕ-ರ-ಣೆ-ಗಳನ್ನು
ಸಲಹೆ
ಸಲ-ಹೆ-ಯನ್ನು
ಸಶ-ಸ್ತ್ರ-ರಾಗಿ
ಸಹ-ಜ-ಪ್ರೇ-ಮ-ದಿಂದ
ಸಹ-ಜ-ಸ್ಥಿ-ತಿ-ಯನ್ನು
ಸಹ-ನ-ಶೀ-ಲ-ಳಾಗಿ
ಸಹ-ಸ್ರಾರು
ಸಹಾ-ನು-ಭೂತಿ
ಸಹಾ-ನು-ಭೂ-ತಿ-ಯಿ-ಲ್ಲದೆ
ಸಹಾಯ
ಸಹಾ-ಯ-ಕ-ವಾ-ಗ-ಬೇಕು
ಸಹಾ-ಯ-ಕ್ಕಾಗಿ
ಸಹಾ-ಯ-ಕ್ಕಿಂತ
ಸಹಾ-ಯಕ್ಕೆ
ಸಹಾ-ಯ-ದಿಂದ
ಸಹಾ-ಯ-ಮಾ-ಡುವ
ಸಹಾ-ಯ-ವಾ-ಯಿತು
ಸಹಿ-ಷ್ಣುತೆ
ಸಹಿ-ಷ್ಣು-ತೆ-ಯಲ್ಲ
ಸಹೋ-ದ-ರ-ನಂತೆ
ಸಹೋ-ದ-ರರು
ಸಹೋ-ದ-ರರೆ
ಸಹೋ-ದ-ರರೇ
ಸಾಕಾ-ದ-ಷ್ಟನ್ನು
ಸಾಕಾ-ದಷ್ಟು
ಸಾಕ್ಷಾ-ತ್ಕಾರ
ಸಾಕ್ಷಾ-ತ್ಕಾ-ರವೇ
ಸಾಗರ
ಸಾಗ-ರವೇ
ಸಾಧ-ನೆ-ಗಾಗಿ
ಸಾಧ-ನೆಗೆ
ಸಾಧ-ನೆ-ಮಾಡಿ
ಸಾಧ-ನೆ-ಯನ್ನು
ಸಾಧಾ-ರಣ
ಸಾಧಿ-ಸ-ಬ-ಹುದು
ಸಾಧಿ-ಸ-ಬೇಕು
ಸಾಧಿ-ಸ-ಲಾ-ರದು
ಸಾಧಿ-ಸು-ವಿರಿ
ಸಾಧಿ-ಸು-ವು-ದಕ್ಕೆ
ಸಾಧ್ಯ-ವಾ-ಗಿ-ರು-ವ-ವನೇ
ಸಾಧ್ಯ-ವಾ-ದರೆ
ಸಾಧ್ಯ-ವಾ-ದಷ್ಟು
ಸಾಮಾನೂ
ಸಾಮಾ-ನ್ಯ-ರಿಗೆ
ಸಾಯಲು
ಸಾಯು-ವ-ವ-ನಂತೆ
ಸಾರ
ಸಾರವೇ
ಸಾರಿ
ಸಾರಿ-ದರೂ
ಸಾರು-ತ್ತದೆ
ಸಾರು-ತ್ತಿದೆ
ಸಾರು-ವುದು
ಸಾವಿರ
ಸಾವಿ-ರಾರು
ಸಾವು
ಸಾಸುವೆ
ಸಾಹಿತ್ಯ
ಸಾಹಿ-ತ್ಯ-ದಲ್ಲಿ
ಸಿಂಧೂ-ನ-ದಿಯ
ಸಿಂಹ-ವಾ-ಗು-ವು-ದಕ್ಕೆ
ಸಿಂಹ-ವಾ-ಣಿ-ಯಿಂದ
ಸಿಂಹ-ಸ-ದೃಶ
ಸಿಕ್ಕಿ-ಕೊಂಡು
ಸಿಕ್ಕಿತು
ಸಿಕ್ಕಿ-ದುದ
ಸಿಕ್ಕು-ವು-ದಿಲ್ಲ
ಸಿತು
ಸಿದ
ಸಿದೆ
ಸಿದ್ಧ
ಸಿದ್ಧ-ರಾಗಿ
ಸಿದ್ಧ-ವಾ-ಗಿ-ರು-ವುದು
ಸಿದ್ಧಾಂತ
ಸಿದ್ಧಾಂ-ತ-ಗಳನ್ನು
ಸಿದ್ಧಾಂ-ತ-ಗ-ಳಿಗೆ
ಸಿದ್ಧಾಂ-ತ-ದಲ್ಲಿ
ಸಿದ್ಧಾಂ-ತ-ವನ್ನು
ಸಿದ್ಧಾಂ-ತ-ವೆಂದು
ಸಿರು-ವಳು
ಸೀತಾ-ದೇವಿ
ಸೀತಾ-ದೇ-ವಿಯ
ಸೀತೆ
ಸೀತೆಯ
ಸುಖ-ಕ್ಕಿಂತ
ಸುಖ-ದಲ್ಲಿ
ಸುಖ-ದುಃ-ಖ-ಗಳ
ಸುಖ-ದುಃ-ಖ-ಗಳು
ಸುಖ-ದುಃ-ಖ-ಗ-ಳೆ-ರ-ಡಕ್ಕೂ
ಸುಖ-ನಿ-ದ್ರೆಗೆ
ಸುತ್ತ
ಸುತ್ತಲೂ
ಸುತ್ತಿ-ಗೆಯ
ಸುತ್ತೇನೆ
ಸುದ್ದಿ
ಸುಧಾ-ರಣೆ
ಸುಧಾ-ರ-ಣೆ-ಯನ್ನು
ಸುಧಾ-ರಿ-ಸು-ವು-ದಕ್ಕೆ
ಸುಪ್ತ-ವಾ-ಗಿತ್ತು
ಸುಪ್ತ-ವಾ-ಗಿ-ರು-ವುದು
ಸುಪ್ತ-ವಾದ
ಸುಪ್ತಾ-ವ-ಸ್ಥೆ-ಯಲ್ಲಿ
ಸುಪ್ಪ-ತ್ತಿ-ಗೆಯ
ಸುಪ್ರೀ-ತ-ರಾ-ಗು-ವರು
ಸುಮ್ಮನೆ
ಸುರಿ-ಮ-ಳೆ-ಯನ್ನು
ಸುರಿ-ಸದೆ
ಸುಲಭ
ಸುಲ-ಭ-ವಾಗಿ
ಸುಲ-ಭ-ವಾದ
ಸುಳಿ-ಯು-ತ್ತಿತ್ತು
ಸುಳಿವೇ
ಸುಳ್ಳನ್ನು
ಸುವ
ಸುವು-ದಕ್ಕೆ
ಸುಸಂ-ಸ್ಕೃ-ತ-ರಾದ
ಸೂಕ್ತ
ಸೂಚಕ
ಸೂಚನೆ
ಸೂಚ-ನೆ-ಯನ್ನು
ಸೂಚಿಸು
ಸೂಚಿ-ಸುವ
ಸೆಳೆಯು
ಸೇಬಿನ
ಸೇಬಿ-ನ-ಲ್ಲಿ-ರ-ಲಿಲ್ಲ
ಸೇರಿ-ದವ
ಸೇರಿವೆ
ಸೇರಿ-ಸು-ವುದು
ಸೇರು-ವ-ವ-ರೆಗೂ
ಸೇವಾ-ಧ-ರ್ಮ-ವನ್ನೇ
ಸೇವೆ
ಸೇವೆ-ಗ-ಲ್ಲದೆ
ಸೇವೆಯ
ಸೇವೆ-ಯನ್ನು
ಸೇವೆಯೇ
ಸೊಂಟಕ್ಕೆ
ಸೋಮಾ-ರಿ-ಗಳು
ಸೋಽಹಂ
ಸೌಕರ್ಯ
ಸೌಹಾ-ರ್ದ-ವನ್ನು
ಸ್ತುತಿ-ಗಿಂತ
ಸ್ತುವಂತು
ಸ್ತ್ರೀ
ಸ್ತ್ರೀಯನ್ನು
ಸ್ತ್ರೀಯರ
ಸ್ತ್ರೀಯ-ರನ್ನು
ಸ್ತ್ರೀಯ-ರಲ್ಲಿ
ಸ್ತ್ರೀಯ-ರಿಗೂ
ಸ್ತ್ರೀಯ-ರಿಗೆ
ಸ್ತ್ರೀಯರು
ಸ್ತ್ರೀಯೂ
ಸ್ತ್ರೀರ-ತ್ನ-ಗಳ
ಸ್ಥಳ-ದಲ್ಲಿ
ಸ್ಥಳ-ದಿಂದ
ಸ್ಥಾನಕ್ಕೆ
ಸ್ಥಾನ-ದಲ್ಲಿ
ಸ್ಥಿತಿ
ಸ್ಥಿತಿಗೆ
ಸ್ಥಿತಿ-ಯನ್ನು
ಸ್ಥಿತಿ-ಯ-ಲ್ಲಿರು
ಸ್ಥಿತಿ-ಯಿಂದ
ಸ್ಥಿರತೆ
ಸ್ಥಿರ-ವಾಗಿ
ಸ್ಥಿರ-ವಾ-ಗಿದೆ
ಸ್ಧಾನ-ದಲ್ಲಿ
ಸ್ನೇಹಿ-ತರೆ
ಸ್ನೇಹಿ-ತರೇ
ಸ್ಪಂದ-ನ-ದೊಂ-ದಿಗೆ
ಸ್ಪಷ್ಟ-ವಾಗಿ
ಸ್ಪಷ್ಟ-ವಾದ
ಸ್ಪೂರ್ತಿ
ಸ್ಫಟಿ-ಕ-ವಾ-ಗು-ವುದು
ಸ್ಫೂರ್ತಿ
ಸ್ಫೂರ್ತಿ-ಯನ್ನು
ಸ್ಮೃತ-ಪು-ಣ್ಯ-ರಾದ
ಸ್ಮೃತಿ-ಗಳನ್ನು
ಸ್ವಂತ
ಸ್ವತಂ-ತ್ರ-ವಾಗಿ
ಸ್ವತಂ-ತ್ರ-ವಾ-ಗಿರಿ
ಸ್ವತಂ-ತ್ರ-ವಾ-ಗಿ-ರು-ವುದು
ಸ್ವತಃ
ಸ್ವಭಾವ
ಸ್ವಭಾ-ವಕ್ಕೆ
ಸ್ವಭಾ-ವ-ಗಳ
ಸ್ವಭಾ-ವ-ಗ-ಳಿವೆ
ಸ್ವಭಾ-ವತಃ
ಸ್ವಭಾ-ವ-ವನ್ನು
ಸ್ವಭಾ-ವವೇ
ಸ್ವಯಂ-ಪ್ರ-ಕಾ-ಶ-ಮಾ-ನ-ವಾದ
ಸ್ವಲ್ಪ
ಸ್ವಲ್ಪವೂ
ಸ್ವಷ್ಟ-ವಾದ
ಸ್ವಾತಂತ್ರ್ಯ
ಸ್ವಾತಂ-ತ್ರ್ಯಕ್ಕೆ
ಸ್ವಾತಂ-ತ್ರ್ಯವೇ
ಸ್ವಾಧೀ-ನಕ್ಕೆ
ಸ್ವಾಭಾ-ವಿ-ಕ-ವಾ-ದುದು
ಸ್ವಾಮಿ
ಸ್ವಾರ್ಥ
ಸ್ವಾರ್ಥ-ತೆಯ
ಸ್ವಾರ್ಥ-ರ-ನ್ನಾಗಿ
ಸ್ವಾರ್ಥಾ-ಭಿ-ಲಾ-ಷೆ-ಗಳ
ಸ್ವಾರ್ಥಾ-ಭಿ-ಲಾ-ಷೆ-ಯಾ-ಗಲಿ
ಸ್ವೀಕ-ರಿಸ
ಸ್ವೀಕ-ರಿಸಿ
ಸ್ವೀಕ-ರಿ-ಸು-ವು-ದಕ್ಕೆ
ಸ್ವೀಕ-ರಿ-ಸು-ವೆನು
ಸ್ವೀಕ-ರಿ-ಸೋಣ
ಸ್ವೀಕಾರ
ಸ್ವೀಕಾ-ರಕ್ಕೂ
ಹಕ್ಕನ್ನು
ಹಕ್ಕು
ಹಕ್ಕು-ಗಳೂ
ಹಗಲು
ಹಣ-ವನ್ನು
ಹಣ್ಣು
ಹತ್ತಿರ
ಹತ್ತಿಸಿ
ಹತ್ತು-ಪಾಲು
ಹತ್ತು-ವು-ದಕ್ಕೆ
ಹತ್ತು-ವು-ದಿಲ್ಲ
ಹದ-ಗೊ-ಳಿಸಿ
ಹನಿ
ಹನು-ಮಂ-ತನ
ಹನು-ಮಂ-ತನು
ಹನು-ಮಾನ್
ಹನ್ನೆ-ರಡು
ಹರ
ಹರ-ಕು-ಬಟ್ಟೆ
ಹರ-ಡ-ಬೇ-ಕಾ-ಗಿತ್ತು
ಹರ-ಡ-ಬೇಕು
ಹರಡಿ
ಹರಡು
ಹರ-ಡುವ
ಹರ-ಡು-ವು-ದಕ್ಕೆ
ಹರ-ಡು-ವು-ದರ
ಹರಿದು
ಹರಿ-ಯು-ತ್ತಿ-ದ್ದರೆ
ಹಲ-ಕೆ-ಲ-ವ-ರಲ್ಲಿ
ಹಲ-ವನ್ನು
ಹಲವು
ಹಳೆಯ
ಹಳ್ಳಿಗೆ
ಹಳ್ಳಿ-ಯ-ಲ್ಲಿಯೂ
ಹಳ್ಳಿ-ಯಿಂದ
ಹಸಿ-ವಿ-ನಿಂದ
ಹಾಕಿ-ಕೊ-ಳ್ಳ-ವಾಗ
ಹಾಗೆ
ಹಾಗೆಯೇ
ಹಾಡಿ
ಹಾಡು-ವಾಗ
ಹಾಳು-ಮಾ-ಡ-ಬೇಡಿ
ಹಾಳು-ಮಾ-ಡು-ವನು
ಹಾಳು-ಮಾ-ಡು-ವುದು
ಹಿಂದಿನ
ಹಿಂದಿ-ನಂತೆ
ಹಿಂದು-ಗಳು
ಹಿಂದು-ಳಿದ
ಹಿಂದೂ
ಹಿಂದೂ-ವಿ-ನೊಂ-ದಿಗೆ
ಹಿಂದೆ
ಹಿಂಸಿ-ಸು-ವಂತೆ
ಹಿಗ್ಗು-ತ್ತೀರಿ
ಹಿಡಿದು
ಹಿಡಿ-ದು-ಕೊಂಡು
ಹಿಡಿ-ದು-ದನ್ನು
ಹಿಡಿ-ಯು-ತ್ತಿ-ರುವ
ಹಿತ-ಕ-ರ-ವಲ್ಲ
ಹಿತ-ಕ್ಕೋ-ಸುಗ
ಹಿತ-ರ-ಕ್ಷಣೆ
ಹಿತ-ರ-ಕ್ಷ-ಣೆ-ಗಾಗಿ
ಹಿತ-ರ-ಕ್ಷ-ಣೆ-ಗೋ-ಸು-ಗ-ವಾಗಿ
ಹಿತ-ಸಾ-ಧ-ನೆ-ಯಾ-ಗು-ವುದು
ಹಿನ್ನೆಲೆ
ಹೀಗಿ-ದ್ದರೆ
ಹೀಗೆ
ಹೀಗೆಯೇ
ಹೀನ
ಹೀನ-ನೆಂದು
ಹೀನ-ಸಂ-ಸ್ಕಾರ
ಹೀನಾ-ಯ-ವಾದ
ಹೀರಲಿ
ಹೀರಿ-ಕೊಂ-ಡಿ-ರು-ವುದು
ಹೀರು-ವುದು
ಹುಚ್ಚ-ರಾ-ಗು-ವಂತೆ
ಹುಚ್ಚು-ಹು-ಚ್ಚಾಗಿ
ಹುಟ್ಟದ
ಹುಟ್ಟಿ
ಹುಟ್ಟಿ-ನಿಂ-ದಲೂ
ಹುಟ್ಟು-ಗು-ಲಾಮ
ಹುಟ್ಟು-ವುದು
ಹುಡು-ಕಿ-ಕೊಂಡು
ಹುಡು-ಗ-ನಾ-ದಾ-ಗಿ-ನಿಂ-ದಲೂ
ಹುಡು-ಗ-ನಿಗೂ
ಹುಡು-ಗರು
ಹುದು-ಗಿದೆ
ಹುಳು-ವಿ-ನಂತೆ
ಹೃತ್ಪೂ-ರ್ವಕ
ಹೃದಯ
ಹೃದ-ಯದ
ಹೃದ-ಯ-ದಲ್ಲಿ
ಹೃದ-ಯ-ದ-ಲ್ಲಿತ್ತು
ಹೃದ-ಯ-ದಿಂದ
ಹೆಂಗ-ಸರ
ಹೆಂಗ-ಸಿನ
ಹೆಂಡತಿ
ಹೆಚ್ಚಾಗಿ
ಹೆಚ್ಚಿ
ಹೆಚ್ಚಿ-ದರೆ
ಹೆಚ್ಚಿ-ದಷ್ಟು
ಹೆಚ್ಚಿ-ದಷ್ಟೂ
ಹೆಚ್ಚಿನ
ಹೆಚ್ಚಿ-ಸ-ಬ-ಲ್ಲ-ದು-ಇ-ದ-ಕ್ಕಿಂತ
ಹೆಚ್ಚು
ಹೆಚ್ಚು-ವುದು
ಹೆಚ್ಚು-ವುದೋ
ಹೆಜ್ಜೆ
ಹೆಜ್ಜೆ-ಯಲ್ಲಿ
ಹೆಣೆ-ದು-ಕೊಂ-ಡಿ-ರು-ವೆವು
ಹೆಣೆ-ದು-ಕೊ-ಳ್ಳು-ವೆವು
ಹೆಣ್ಣು
ಹೆರುವ
ಹೆಸರು
ಹೇಗೆ
ಹೇಗೆಂ-ಬು-ದನ್ನು
ಹೇಡಿ-ತ-ನ-ಗ-ಳೆಲ್ಲ
ಹೇಳ-ಬ-ಹುದು
ಹೇಳ-ಬೇಕು
ಹೇಳ-ಲಿಲ್ಲ
ಹೇಳಿ
ಹೇಳಿ-ಕೊ-ಡು-ತ್ತಿ-ದ್ದನು
ಹೇಳಿದ
ಹೇಳಿ-ದಂ-ತಹ
ಹೇಳಿ-ದಂತೆ
ಹೇಳಿ-ದಾಗ
ಹೇಳಿ-ದು-ದನ್ನು
ಹೇಳಿ-ದ್ದರು
ಹೇಳು
ಹೇಳು-ತ್ತಿ-ದ್ದಂತೆ
ಹೇಳು-ತ್ತಿ-ರು-ವುದನ್ನು
ಹೇಳು-ತ್ತಿ-ರು-ವುದು
ಹೇಳು-ತ್ತೇನೆ
ಹೇಳು-ವನು
ಹೇಳು-ವು-ದಕ್ಕೆ
ಹೇಳು-ವು-ದರ
ಹೇಳು-ವು-ದಾ-ದರೆ
ಹೇಳು-ವು-ದಿಲ್ಲ
ಹೇಳು-ವುದು
ಹೊಂದಿದ
ಹೊಂದಿ-ದ್ದರೆ
ಹೊಂದುವ
ಹೊಂದು-ವುವು
ಹೊಗ-ಳಲಿ
ಹೊಟ್ಟೆ-ತುಂಬ
ಹೊಡೆ-ದಂತೆ
ಹೊಡೆ-ದರೆ
ಹೊಡೆ-ಯಿರಿ
ಹೊಡೆ-ಯು-ವುದು
ಹೊರ-ಗಿನ
ಹೊರ-ಗಿ-ನಿಂದ
ಹೊರಗೆ
ಹೊರ-ಗೆ-ಡ-ಹು-ವುದು
ಹೊರೆ-ಗೆ-ಡ-ಹು-ವುವು
ಹೊಲ-ದಲ್ಲಿ
ಹೊಲಿ-ಯು-ವುದು
ಹೊಳೆ-ಯು-ವುದು
ಹೊಸ
ಹೊಸದೆ
ಹೋಗ
ಹೋಗದೆ
ಹೋಗ-ಬೇಕು
ಹೋಗ-ಲಾ-ಗದ
ಹೋಗ-ಲಾ-ಡಿ-ಸಲೂ
ಹೋಗಲಿ
ಹೋಗಲು
ಹೋಗಿ
ಹೋಗಿರು
ಹೋಗಿ-ರು-ವರು
ಹೋಗು
ಹೋಗು-ತ್ತಿ-ರುವ
ಹೋಗು-ತ್ತಿ-ರು-ವರು
ಹೋಗುವ
ಹೋಗು-ವರು
ಹೋಗು-ವಿರಿ
ಹೋಗು-ವು-ದ-ಕ್ಕಿಂತ
ಹೋಗು-ವುದು
ಹೋದಂ-ತೆಲ್ಲ
ಹೋದರು
ಹೋರಾಟ
ಹೋರಾ-ಟಕ್ಕೆ
ಹೋರಾ-ಟ-ಗಳಿಂದ
ಹೋರಾ-ಟ-ದಲ್ಲೆ
ಹೋರಾ-ಡು-ತ್ತಾ-ನೆಯೋ
ಹೋಲಿ-ಸ-ಬ-ಹುದು
ಹೋಲಿಸಿ
ಹೋಲಿ-ಸಿ-ನೋಡಿ
}
