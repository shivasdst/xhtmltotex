
\chapter{ವಿದ್ಯಾಭ್ಯಾಸತತ್ತ್ವ}

\section{ಜ್ಞಾನ ಮಾನವನ ಸ್ವಭಾವ}

ಮನುಷ್ಯನಲ್ಲಿ ಆಗಲೇ ಅಂತರ್ಗತವಾಗಿರುವ ಪರಿಪೂರ್ಣತೆಯನ್ನು ವ್ಯಕ್ತ ಗೊಳಿಸುವುದೇ ವಿದ್ಯಾಭ್ಯಾಸ. ಜ್ಞಾನ ಮಾನವನಿಗೆ ಸ್ವಾಭಾವಿಕವಾದುದು. ಯಾವ ಜ್ಞಾನವೂ ಹೊರಗಿನಿಂದ ಬರುವುದಿಲ್ಲ. ಎಲ್ಲವೂ ಆಗಲೇ ಅಂತರ್ಗತ ವಾಗಿದೆ. ನಾವು ಯಾವುದನ್ನು ಮನುಷ್ಯ ತಿಳಿಯುತ್ತಾನೆ ಎನ್ನುವೆವೊ, ಅದು ನಿರ್ದಿಷ್ಟ ಮನಶ್ಶಾಸ್ತ್ರೀಯ ಭಾಷೆಯಲ್ಲಿ, “ಕಂಡುಹಿಡಿಯುತ್ತಾನೆ” ಅಥವಾ “ಅನಾವರಣಗೊಳಿಸುತ್ತಾನೆ” ಎಂದಿರಬೇಕು. ಮನುಷ್ಯ ಯಾವುದನ್ನು ಕಲಿ ಯುತ್ತಾನೋ, ಅದು ಅನಂತಜ್ಞಾನನಿಧಿಯಾದ ತನ್ನ ಆತ್ಮನ ಮೇಲೆ ಇದ್ದ ಮುಸುಕನ್ನು ತೆರೆದು ನಿಜವಾಗಿಯೂ ತನ್ನಲ್ಲಿ ತಾನೇ ಕಂಡುಹಿಡಿದದ್ದು. ನಾವು ನ್ಯೂಟನ್ ಗುರುತ್ವಾಕರ್ಷಣತತ್ತ್ವವನ್ನು ಕಂಡುಹಿಡಿದನು ಎಂದು ಹೇಳು ತ್ತೇವೆ. ಅವನಿಗೋಸ್ಕರ ಅದು ಯಾವುದಾದರೊಂದು ಮೂಲೆಯಲ್ಲಿ ಕಾದು ಕೊಂಡಿತ್ತೇನು? ಅದು ಅವನ ಮನಸ್ಸಿನಲ್ಲಿತ್ತು. ಅದಕ್ಕೆ ಸರಿಯಾದ ಸಮಯ ಬಂದಾಗ ಅದನ್ನು ಕಂಡುಹಿಡಿದನು. ಪ್ರಪಂಚಕ್ಕೆ ಬಂದ ಜ್ಞಾನವೆಲ್ಲ ಮನಸ್ಸಿ ನಿಂದ ಬಂದಿತು. ವಿಶ್ವದ ಅನಂತ ಪುಸ್ತಕಭಂಡಾರ ನಿಮ್ಮ ಮನಸ್ಸಿನಲ್ಲಿದೆ. ಬಾಹ್ಯ ಪ್ರಪಂಚ ಒಂದು ಸಲಹೆ ಮಾತ್ರ. ಅದು ನಿಮ್ಮ ಮನಸ್ಸಿನ ಅಧ್ಯಯನಕ್ಕೆ ಒಂದು ಅವಕಾಶವನ್ನು ಒದಗಿಸುತ್ತದೆ. ಬೀಳುವ ಸೇಬಿನ ಹಣ್ಣು ನ್ಯೂಟನ್ನನಿಗೆ ಒಂದು ಸೂಚನೆಯನ್ನು ಕೊಟ್ಟಿತು. ಅವನು ತನ್ನ ಮನಸ್ಸನ್ನು ಪರೀಕ್ಷಿಸಿದನು. ತನ್ನ ಮನಸ್ಸಿನಲ್ಲಿದ್ದ ಹಿಂದಿನ ತನ್ನ ಭಾವನೆಗಳನ್ನೆಲ್ಲ ಪುನಃ ಜೋಡಿಸಿದಾಗ ಕಂಡುಬಂದ ಹೊಸ ಆಲೋಚನೆಯನ್ನೇ ಗುರುತ್ವಾಕರ್ಷಣ ಸಿದ್ಧಾಂತವೆಂದು ಕರೆದನು. ಅದೇನೂ ಸೇಬಿನಲ್ಲಿರಲಿಲ್ಲ, ಅಥವಾ ಭೂಮಿಯ ಅಂತರಾಳ ದಲ್ಲಿಯೂ ಇರಲಿಲ್ಲ.


\section{ಶಿಕ್ಷಣ ಕ್ರಮ}

ಜ್ಞಾನವೆಲ್ಲ ಅದು ಲೌಕಿಕವಾಗಲಿ ಪಾರಮಾರ್ಥಿಕವಾಗಲಿ, ಮನುಷ್ಯನ ಮನಸ್ಸಿನಲ್ಲಿ ಇರುವುದು. ಅನೇಕ ವೇಳೆ ಇದನ್ನು ಇನ್ನೂ ಕಂಡುಹಿಡಿದಿರು ವುದಿಲ್ಲ. ಅದರ ಮೇಲೆ ಮುಸುಕು ಮುಚ್ಚಲ್ಪಟ್ಟಿರುತ್ತದೆ. ನಾವು ಕ್ರಮೇಣ ಈ ಮುಸುಕನ್ನು ತೆಗೆದಂತೆ, “ಶಿಕ್ಷಣವನ್ನು ಪಡೆಯುತ್ತಿರುವೆವು” ಎನ್ನುತ್ತೇವೆ. ಜ್ಞಾನಾರ್ಜನೆ ಇಂತಹ ಅನಾವರಣಕ್ರಮದಿಂದ ಆಗುವುದು. ಯಾರಲ್ಲಿ ಈ ತೆರೆ ಕ್ರಮೇಣ ಜಾರುತ್ತಿದೆಯೋ ಅವನು ಜ್ಞಾನಿ; ಯಾರಲ್ಲಿ ಇದು ಇನ್ನೂ ಗಾಢವಾಗಿ ಮುತ್ತಿಕೊಂಡಿರುವುದೊ ಅವನು ಅಜ್ಞಾನಿ. ಯಾರಲ್ಲಿ ಇದು ಸಂಪೂರ್ಣವಾಗಿ ಮಾಯವಾಗಿದೆಯೋ ಅವನೆ ಸರ್ವಜ್ಞ. ಕಟ್ಟಿಗೆಯ ಚೂರಿ ನಲ್ಲಿ ಇರುವ ಬೆಂಕಿಯಂತೆ ಜ್ಞಾನ ಮನಸ್ಸಿನಲ್ಲಿ ಹುದುಗಿದೆ. ಸೂಚನೆ ಅದನ್ನು ವ್ಯಕ್ತಪಡಿಸುವ ಘರ್ಷಣೆ. ಎಲ್ಲಾ ಜ್ಞಾನವೂ ಎಲ್ಲ ಶಕ್ತಿಯೂ ಆಗಲೆ ಒಳಗಿದೆ. ಯಾವುದನ್ನು ನಾವು ಶಕ್ತಿ, ಪ್ರಕೃತಿಯ ರಹಸ್ಯ, ಎಂದು ಕರೆಯುವೆವೊ ಅದು ಆಗಲೇ ನಮ್ಮಲ್ಲಿದೆ. ಜ್ಞಾನವೆಲ್ಲ ಅಂತರಾತ್ಮನಿಂದ ಬರುವುದು. ತನ್ನಲ್ಲಿ ಅನಾದಿಯಿಂದಲೂ ಅಡಗಿರುವ ಜ್ಞಾನವನ್ನು ಮಾನವನು ವ್ಯಕ್ತಪಡಿಸುತ್ತಾನೆ. ಕಂಡುಹಿಡಿಯುತ್ತಾನೆ.


\section{ಮಗು ಸ್ವಭಾವತಃ ಕಲಿತುಕೊಳ್ಳುವುದು}

ಯಾರೊಬ್ಬರೂ ಮತ್ತೊಬ್ಬರಿಂದ ಕಲಿತುಕೊಳ್ಳಲಿಲ್ಲ. ಪ್ರತಿಯೊಬ್ಬರೂ ತಮಗೆ ತಾವೆ ಕಲಿತುಕೊಳ್ಳಬೇಕಾಗಿದೆ. ಬಾಹ್ಯ ಗುರು ಕೇವಲ ಒಂದು ಸಲಹೆಯನ್ನು ಕೊಡುತ್ತಾನೆ. ಅದು ಅಂತರಂಗದಲ್ಲಿ ಇರುವ ಗುರುವನ್ನು ಜಾಗ್ರತನನ್ನಾಗಿ ಮಾಡಿ ವಿಷಯವನ್ನು ತಿಳಿದುಕೊಳ್ಳುವಂತೆ ಪ್ರೇರೇಪಿಸು ವುದು. ಆಗ ನಮ್ಮ ಇಂದ್ರಿಯಗ್ರಹಣ ಮತ್ತು ಆಲೋಚನಾ ಶಕ್ತಿಗಳಿಂದ ಅದು ನಮಗೆ ಸ್ಪಷ್ಟವಾಗಿ, ನಮ್ಮ ಅಂತರಾಳದಲ್ಲಿ ನಾವು ಅದನ್ನು ತಿಳಿಯು ತ್ತೇವೆ. ಅನೇಕ ಎಕರೆಗಳಷ್ಟು ವಿಸ್ತಾರವಾದ ಆಲದ ಮರ ಸಾಸುವೆ ಕಾಳಿನಲ್ಲಿ ಒಂದನೇ ಎಂಟುಭಾಗದಷ್ಟು ಕೂಡಾ ದೊಡ್ಡದಾಗಿಲ್ಲದ ಅದರ ಬೀಜ ದಲ್ಲಿತ್ತು. ಇಷ್ಟೊಂದು ಶಕ್ತಿಯೂ ಕೂಡ ಅಲ್ಲಿ ಅಡಗಿತ್ತು. ಅತ್ಯದ್ಭುತವಾದ ಬುದ್ಧಿಶಕ್ತಿ ಜೀವಾಣುವಿನಲ್ಲಿ ಸುಪ್ತವಾಗಿರುವುದು ನಮಗೆ ಗೊತ್ತಿದೆ. ಇದು ವಿಪರೀತ ಭಾವನೆಯಂತೆ ನಮಗೆ ಕಾಣಬಹುದು. ಆದರೂ ಇದು ನಿಜ. ಪ್ರತಿಯೊಬ್ಬರೂ ಕೂಡ ಒಂದೊಂದು ಜೀವಾಣುವಿನಿಂದ ಬಂದಿರುವೆವು. ನಮ್ಮಲ್ಲಿ ಈಗ ಇರುವ ಶಕ್ತಿಯೆಲ್ಲ ಅಲ್ಲಿ ಸುಪ್ತವಾಗಿತ್ತು. ಅವು ಆಹಾರದಿಂದ ಬಂದವು ಎಂದು ನೀವು ಹೇಳುವುದಕ್ಕೆ ಆಗುವುದಿಲ್ಲ. ಏಕೆಂದರೆ ನೀವು ಬೆಟ್ಟದಷ್ಟು ಆಹಾರವನ್ನು ರಾಶಿಮಾಡಿದರೆ ಅದರಿಂದ ಯಾವ ಶಕ್ತಿ ಬರು ತ್ತದೆ? ಶಕ್ತಿ ಅಲ್ಲಿ ಇತ್ತು. ಇದ್ದದ್ದು ಸುಪ್ತಾವಸ್ಥೆಯಲ್ಲಿ ನಿಜ. ಆದರೂ ಅದು ಅಲ್ಲಿತ್ತು. ಅದರಂತೆಯೇ, ಅವನಿಗೆ ತಿಳಿದಿರಲಿ ತಿಳಿಯದೆ ಇರಲಿ, ಮಾನವನ ಅಂತರಾತ್ಮನಲ್ಲಿ ಅನಂತಶಕ್ತಿ ಅಡಗಿದೆ. ಅದರ ಅರಿವು ಉಂಟಾದರೆ ಅದು ವ್ಯಕ್ತವಾಗುತ್ತದೆ.

ಅನೇಕ ಜನರಲ್ಲಿ ಅಂತರ್ಜ್ಯೋತಿ ಮೊಬ್ಬಾಗಿದೆ. ಅದು ಒಂದು ಕಬ್ಬಿಣದ ಪೀಪಾಯಿಯೊಳಗೆ ಇಟ್ಟ ದೀಪದಂತೆ ಇದೆ. ಒಂದು ಕಿರಣವೂ ಹೊರಗೆ ಬರುವುದಿಲ್ಲ. ಕ್ರಮೇಣ ಮನಸ್ಸಿನ ಪರಿಶುದ್ಧತೆ ಮತ್ತು ನಿಃಸ್ವಾರ್ಥತೆಗಳಿಂದ ಬೆಳಕನ್ನು ಹೊರಗೆ ಬರದಂತೆ ತಡೆದಿರುವ ಮಧ್ಯದಲ್ಲಿರುವ ವಸ್ತುವನ್ನು ತೆಳ್ಳಗಾಗುವಂತೆ ಮಾಡಿ ಅದನ್ನು ಕನ್ನಡಿಯಷ್ಟು ಪಾರದರ್ಶಕವನ್ನಾಗಿ ಮಾಡ ಬಹುದು. ಶ್ರೀರಾಮಕೃಷ್ಣರು ಕನ್ನಡಿಯ ಪೀಪಾಯಿಯಾಗಿ ಪರಿವರ್ತನೆ ಹೊಂದಿದ ಕಬ್ಬಿಣದ ಪೀಪಾಯಿ. ಅದರ ಮೂಲಕ ಬೆಳಕಿನ ಸಹಜಸ್ಥಿತಿಯನ್ನು ನೋಡಬಹುದು.


\section{ಸ್ವಭಾವಕ್ಕೆ ಅನುಗುಣವಾಗಿ ಬೆಳೆಯುವುದಕ್ಕೆ ಸಹಾಯ ಮಾಡಿ}

ನೀವು ಒಂದು ಗಿಡವನ್ನು ಬೆಳೆಸುವುದಕ್ಕೆ ನೀಡುವ ಸಹಾಯಕ್ಕಿಂತ ಹೆಚ್ಚಾಗಿ ಮಗುವಿಗೆ ಶಿಕ್ಷಣವನ್ನು ಕೊಡಲಾರರಿ. ಗಿಡ ತನ್ನ ಸ್ವಭಾವವನ್ನು ವ್ಯಕ್ತಗೊಳಿಸು ವುದು. ಮಗುವೂ ಕೂಡ ತನಗೆ ತಾನೇ ಬೋಧಿಸಿಕೊಳ್ಳುವುದು. ಆದರೆ ಅದು ತನ್ನ ಸ್ವಭಾವಕ್ಕೆ ಅನುಗುಣವಾಗಿ ಮುಂದುವರಿಯುವಂತೆ ನೀವು ಸಹಾಯ ಮಾಡಬಹುದು. ಅದಕ್ಕೆ ಖಂಡಿತವಾಗಿಯೂ ಹೀಗೆಯೇ ಬೆಳೆಯಬೇಕೆಂದು ಬಲಾತ್ಕರಿಸುವುದಕ್ಕೆ ಆಗುವುದಿಲ್ಲ. ಬೆಳವಣಿಗೆಗೆ ಆತಂಕವಾದುದನ್ನು ಮಾತ್ರ ನಿವಾರಿಸಬಹುದು. ಆಗ ಜ್ಞಾನ ಅದರ ಸ್ವಭಾವಕ್ಕೆ ಅನುಸಾರವಾಗಿ ಬಂದೇ ಬರುವುದು. ಭೂಮಿಯನ್ನು ಸ್ವಲ್ಪ ಅಗತೆಮಾಡಿ. ಇದರಿಂದ ಬೀಜ ಸುಲಭ ವಾಗಿ ಬೆಳೆಯುವಂತೆ ಆಗಲಿ. ಸುತ್ತಲೂ ಒಂದು ಬೇಲಿಯನ್ನು ಕಟ್ಟಿ. ಯಾವುದರಿಂದಲೂ ಅದು ನಾಶವಾಗದಂತೆ ನೋಡಿಕೊಳ್ಳಿ. ಬೆಳೆಯುವ ಬೀಜಕ್ಕೆ ಆವಶ್ಯಕವಾದ ಗೊಬ್ಬರ, ನೀರು ಗಾಳಿ ಮುಂತಾದುವುಗಳನ್ನು ನೀವು ಒದಗಿಸಬಹುದು. ನಿಮ್ಮ ಕೆಲಸ ಅಲ್ಲಿಗೆ ಕೊನೆಗಾಣುವುದು. ತನಗೆ ಬೇಕಾದ ವಸ್ತುವನ್ನು ತಾನೇ ತನ್ನ ಸ್ವಭಾವಕ್ಕೆ ಅನುಗುಣವಾಗಿ ಹೀರುವುದು. ತನ್ನ ಸ್ವಭಾವಕ್ಕೆ ಅನುಸಾರವಾಗಿ ಆಹಾರವನ್ನು ಜೀರ್ಣಿಸಿಕೊಂಡು ಬೆಳೆಯುವುದು. ಇದರಂತೆಯೇ ಮಗುವಿನ ವಿದ್ಯಾಭ್ಯಾಸ ಕೂಡಾ. ಮಗು ತಾನೇ ಶಿಕ್ಷಣವನ್ನು ಪಡೆಯುತ್ತದೆ. ಗುರು ತಾನು ಕಲಿಸುತ್ತಿರುವೆನು ಎಂದು ಭಾವಿಸುವುದರಿಂದ ಎಲ್ಲವನ್ನೂ ಹಾಳುಮಾಡುವನು. ಎಲ್ಲಾ ಜ್ಞಾನವೂ ಮನುಷ್ಯನಲ್ಲಿದೆ. ಇದನ್ನು ಜಾಗ್ರತಗೊಳಿಸಬೇಕಾಗಿದೆ ಅಷ್ಟೇ. ಇಷ್ಟೇ ಗುರುವಿನ ಕೆಲಸ. ಸ್ವಂತ ಬುದ್ಧಿಶಕ್ತಿಯ ಸಹಾಯದಿಂದ ಕೈ ಕಾಲು ಕಿವಿ ಕಣ್ಣುಗಳನ್ನು ಸರಿಯಾದ ರೀತಿಯಲ್ಲಿ ಉಪಯೋಗಿಸಿಕೊಳ್ಳುವುದು ಹೇಗೆಂಬುದನ್ನು ಮಕ್ಕಳು ಕಲಿಯುವಂತೆ ಮಾಡಬಹುದು.


\section{ಸ್ವಾತಂತ್ರ್ಯಕ್ಕೆ ಅವಕಾಶವಿರಬೇಕು}

ಹೊಡೆದರೆ ಕುದುರೆಯಾಗುವುದು ಎಂಬ ಸಲಹೆಯನ್ನು ಅನುಸರಿಸಿ ಒಬ್ಬ ವ್ಯಕ್ತಿ ತನ್ನ ಕತ್ತೆಯನ್ನು ಹೊಡೆದಂತೆ ಇದೆ ನಾವು ಮಕ್ಕಳಿಗೆ ಶಿಕ್ಷಣ ನೀಡುವ ವಿಧಾನ. ಇದನ್ನು ವಜಾ ಮಾಡಬೇಕು. ತಾಯಿ ತಂದೆಗಳ ಹೆಚ್ಚು ಬಲಾತ್ಕಾರಕ್ಕೆ ಒಳಪಟ್ಟ ನಮ್ಮ ಮಕ್ಕಳಿಗೆ ಬೆಳವಣಿಗೆಗೆ ಸಾಕಾದಷ್ಟು ಅವಕಾಶ ಸಿಕ್ಕುವುದಿಲ್ಲ. ಪ್ರತಿಯೊಬ್ಬನಲ್ಲಿಯೂ ಅನಂತ ಸ್ವಭಾವಗಳಿವೆ. ಅವುಗಳೆಲ್ಲ ಈಡೇರಬೇಕಾ ದರೆ ಅವಕಾಶ ಬೇಕು. ಉಗ್ರವಾಗಿ ಸುಧಾರಣೆ ಮಾಡುವ ಪ್ರಯತ್ನವೆಲ್ಲ ಸುಧಾರಣೆಯನ್ನು ತಡೆಯುವುದರಲ್ಲಿ ಕೊನೆಗಾಣುತ್ತದೆ. ಸಿಂಹವಾಗುವುದಕ್ಕೆ ನೀವು ಒಬ್ಬನಿಗೆ ಅವಕಾಶ ಕೊಡದೆ ಇದ್ದರೆ ಅವನು ನರಿಯಾಗುತ್ತಾನೆ.


\section{ಸ್ಪಷ್ಟವಾದ ಆದರ್ಶಗಳು}

ಸ್ಪಷ್ಟವಾದ ಆದರ್ಶಗಳನ್ನು ನಾವು ಕೊಡಬೇಕು. ನಿಷೇಧಾತ್ಮ ಸೂಚಕ ವಾದ ಆಲೋಚನೆ ಯಾವಾಗಲೂ ಮನುಷ್ಯನನ್ನು ನಿರ್ಬಲನನ್ನಾಗಿ ಮಾಡು ವುದು. ತಂದೆತಾಯಿಗಳು ತಮ್ಮ ಮಕ್ಕಳನ್ನು ಯಾವಾಗಲೂ ಓದುವಂತೆ ಬರೆಯುವಂತೆ ಬಲತ್ಕಾರ ಮಾಡಿ, ನಿಮ್ಮ ತಲೆಗೆ ಏನೂ ಹತ್ತುವುದಿಲ್ಲ; ಶುದ್ಧ ಮೂರ್ಖರು ಇತ್ಯಾದಿಯಾಗಿ ಅವರನ್ನು ದೂರುತ್ತಿದ್ದರೆ ಅನೇಕ ವೇಳೆ ಅಂತಹ ಮಕ್ಕಳು ನಿಜವಾಗಿಯೂ ಹಾಗೆಯೇ ಆಗುತ್ತಾರೆ ಎಂಬುದನ್ನು ನೀವು ನೋಡು ವುದಿಲ್ಲವೆ? ನೀವು ಅವರಿಗೆ ಒಳ್ಳೆಯ ಮಾತನ್ನು ಆಡಿ ಪ್ರೋತ್ಸಾಹಿಸಿದರೆ ಕಾಲ ಕ್ರಮೇಣ ಅವರು ಉತ್ತಮರಾಗಿಯೆ ಆಗುವರು. ನೀವು ಅವರಿಗೆ ಸ್ಪಷ್ಟವಾದ ಆದರ್ಶಗಳನ್ನು ಕೊಟ್ಟರೆ, ಅವರು ವೀರರಾಗಿ ತಮ್ಮ ಕಾಲಮೇಲೆ ತಾವು ನಿಲ್ಲುವುದನ್ನು ಕಲಿಯುತ್ತಾರೆ. ಭಾಷೆಯಲ್ಲಿ ಮತ್ತು ಸಾಹಿತ್ಯದಲ್ಲಿ, ಕಾವ್ಯದಲ್ಲಿ ಮತ್ತು ಕಲೆಯಲ್ಲಿ ಪ್ರತಿಯೊಂದರಲ್ಲಿಯೂ ಮನುಷ್ಯರು ತಮ್ಮ ಕಾರ್ಯ ಮತ್ತು ಆಲೋಚನೆಯಲ್ಲಿ ಮಾಡುತ್ತಿರುವ ತಪ್ಪನ್ನು ನಾವು ಎತ್ತಿ ತೋರಿಸಕೂಡದು. ಆದರೆ ಅದನ್ನು ಹೇಗೆ ಮತ್ತಷ್ಟು ಉತ್ತಮವಾಗಿ ಮಾಡಬಹುದು ಎಂಬುದನ್ನು ತೋರಬೇಕು. ಶಿಷ್ಯನ ಆವಶ್ಯಕತೆಗೆ ತಕ್ಕಂತೆ ನಾವು ಶಿಕ್ಷಣ ಬದಲಾಯಿಸಬೇಕು. ಹಿಂದಿನ ಜನ್ಮಗಳು ನಮ್ಮ ಸ್ವಭಾವವನ್ನು ತಿದ್ದಿರುತ್ತವೆ. ಆದಕಾರಣ ವಿದ್ಯಾರ್ಥಿಯ ಸ್ವಭಾವಕ್ಕೆ ತಕ್ಕಂತೆ ಶಿಕ್ಷಣವನ್ನು ಕೊಡಿ. ಪ್ರತಿಯೊಬ್ಬರನ್ನೂ ಅವರು ಇರುವ ಸ್ಥಳದಿಂದ ಕರೆದು ಮುಂದುವರಿಯುವಂತೆ ಮಾಡಿ. ನಾವು ಯಾರನ್ನು ಕೆಲಸಕ್ಕೆ ಬಾರದವರೆಂದು ತಿಳಿದುಕೊಂಡಿದ್ದೆವೊ ಅಂತಹವರನ್ನು ಕೂಡ ಪ್ರೋತ್ಸಾಹಿಸಿ ಅವರ ಜೀವನವನ್ನು ಶ್ರೀರಾಮಕೃಷ್ಣರು ಬದಲಾಯಿಸಿರು ವುದನ್ನು ನಾನು ನೋಡಿರುವೆನು. ಯಾರೊಬ್ಬರ ವೈಶಿಷ್ಟ್ಯವನ್ನೂ ಅವರು ನಾಶಮಾಡುತ್ತಿರಲಿಲ್ಲ. ಅತ್ಯಂತ ನೀಚ ಮಾನವರಿಗೂ ಭರವಸೆಯಿತ್ತು ಪ್ರೋತ್ಸಾಹಿಸಿ ಉದ್ಧಾರ ಮಾಡಿದರು.


\section{ಸ್ವಾತಂತ್ರ್ಯವೇ ಬೆಳವಣಿಗೆಗೆ ಪ್ರಥಮ ಆವಶ್ಯಕತೆ}

ಸ್ವಾತಂತ್ರ್ಯವೇ ಬೆಳವಣಿಗೆಗೆ ಮೊದಲನೇ ಆವಶ್ಯಕತೆ. ನಿಮ್ಮಲ್ಲಿ ಯಾರಾ ದರೂ, ಈ ಹೆಂಗಸಿನ ಅಥವಾ ಆ ಮಗುವಿನ ಮೋಕ್ಷಕ್ಕೆ ನಾನು ಪ್ರಯತ್ನ ಪಡುತ್ತೇನೆ ಎಂದು ಅಭಿಪ್ರಾಯಪಟ್ಟರೆ ಅದು ತಪ್ಪು. ಸಾವಿರ ಸಾರಿ ತಪ್ಪು. ದೂರ ನಿಲ್ಲಿ. ಅವರು ತಮ್ಮ ಸಮಸ್ಯೆಗಳನ್ನು ತಾವೇ ಬಗೆಹರಿಸಿ ಕೊಳ್ಳುವರು. ಸರ್ವಜ್ಞ ಎಂದು ತಿಳಿದುಕೊಳ್ಳುವುದಕ್ಕೆ ನೀವು ಯಾರು? ದೇವರಮೇಲೆ ನಿಮಗೆ ಅಧಿಕಾರವಿದೆ ಎಂದು ತಿಳಿದುಕೊಳ್ಳುವುದಕ್ಕೆ ನಿಮ್ಮ ಎದೆಗಾರಿಕೆ ಎಷ್ಟು. ಪ್ರತಿಯೊಂದು ಜೀವವೂ ಕೂಡ ಬ್ರಹ್ಮಮಯವೆಂಬುದು ನಿಮಗೆ ತಿಳಿಯದೆ? ಎಲ್ಲರನ್ನೂ ದೇವರಂತೆ ನೋಡಿ. ನೀವು ಸೇವೆ ಮಾತ್ರ ಮಾಡಬಲ್ಲಿರಿ. ನಿಮಗೆ ಅದೃಷ್ಟವಿದ್ದರೆ ದೇವರ ಮಕ್ಕಳಿಗೆ ಸೇವೆ ಮಾಡಿ. ದೇವರು ನಿಮಗೆ ಅವನ ಯಾವ ಮಗುವಿಗಾದರೂ ಸೇವೆ ಮಾಡಲು ಅವಕಾಶ ಕೊಟ್ಟರೆ ನೀವೇ ಪುಣ್ಯವಂತರು. ಮತ್ತೊಬ್ಬರಿಗೆ ಈ ಅದೃಷ್ಟವಿಲ್ಲದೆ ಅದು ನಿಮಗೆ ಸಿಕ್ಕಿದುದ ರಿಂದ ನೀವೆ ಧನ್ಯರು. ಸೇವೆಯನ್ನು ಪೂಜೆ ಎಂಬ ದೃಷ್ಟಿಯಿಂದ ಮಾತ್ರ ಮಾಡಿ.

