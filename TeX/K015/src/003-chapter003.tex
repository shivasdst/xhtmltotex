
\chapter{ಶಿಕ್ಷಣ ಕ್ರಮ}

\section{ಏಕಾಗ್ರತೆ}

ಜ್ಞಾನವನ್ನು ಸಂಪಾದಿಸುವುದಕ್ಕೆ ಏಕಾಗ್ರತೆ ಎಂಬ ಒಂದೇ ಮಾರ್ಗವಿರು ವುದು. ಶಿಕ್ಷಣದ ಸಾರವೇ ಮಾನಸಿಕ ಏಕಾಗ್ರತೆ. ಅತ್ಯಂತ ಮೂಢನಿಂದ ಹಿಡಿದು ಅತ್ಯುತ್ತಮ ಯೋಗಿಯವರೆಗೆ ಪ್ರತಿಯೊಬ್ಬರೂ ಜ್ಞಾನಾರ್ಜನೆಗೆ ಒಂದೇ ಮಾರ್ಗವನ್ನು ಉಪಯೋಗಿಸಬೇಕು. ತನ್ನ ಪ್ರಯೋಗಶಾಲೆಯಲ್ಲಿ ಕೆಲಸಮಾಡುವ ರಸಾಯನ ಶಾಸ್ತ್ರಜ್ಞ ತನ್ನ ಮಾನಸಿಕ ಶಕ್ತಿಯನ್ನೆಲ್ಲ ಏಕಾಗ್ರ ಗೊಳಿಸಿ ಅದನ್ನು ಕೇಂದ್ರೀಕೃತಮಾಡಿ ವಸ್ತುಗಳ ಮೇಲೆ ಬೀರಿದರೆ ವಸ್ತುಗಳು ವಿಭಜನೆ ಹೊಂದುವುವು. ಅವನಿಗೆ ಹೀಗೆ ಜ್ಞಾನ ಬರುವುದು. ಖಗೋಳಶಾಸ್ತ್ರಜ್ಞ ತನ್ನ ಮಾನಸಿಕ ಶಕ್ತಿಯನ್ನೆಲ್ಲಾ ಏಕಾಗ್ರಗೊಳಿಸಿ, ಅದನ್ನು ಕೇಂದ್ರೀಕೃತಮಾಡಿ ದೂರದರ್ಶಕ ಯಂತ್ರದ ಮೂಲಕ ವಸ್ತುಗಳ ಮೇಲೆ ಬೀರಿದರೆ, ತಾರೆ ನಿಹಾರಿಕೆಗಳು ಮುಂದೆ ಬಂದು ತಮ್ಮ ರಹಸ್ಯವನ್ನು ಹೊರೆಗೆಡಹುವುವು. ವೇದಿಕೆಯ ಮೇಲಿರುವ ಅಧ್ಯಾಪಕ, ಪುಸ್ತಕದೊಂದಿಗೆ ಕುಳಿತಿರುವ ವಿದ್ಯಾರ್ಥಿ, ಜ್ಞಾನಾರ್ಜನೆ ಮಾಡುತ್ತಿರುವ ಪ್ರತಿಯೊಬ್ಬರೂ ಎಲ್ಲಾ ಪ್ರಸಂಗಗಳಲ್ಲಿ ಅನು ಸರಿಸುತ್ತಿರುವ ಮಾರ್ಗ ಇದೊಂದೆ.


\section{ಏಕಾಗ್ರತೆಯ ಶಕ್ತಿ}

ಏಕಾಗ್ರತೆಯ ಶಕ್ತಿ ಹೆಚ್ಚಿದಷ್ಟು ನಮ್ಮ ಜ್ಞಾನಾರ್ಜನೆಯೂ ಹೆಚ್ಚುವುದು. ಅತ್ಯಂತ ದೀನನಾದ ಜೋಡನ್ನು ತಿಕ್ಕುವವನು ಕೆಲಸದ ಮೇಲೆ ಮನಸ್ಸನ್ನು ಎಷ್ಟು ಏಕಾಗ್ರಗೊಳಿಸುವನೊ ಅಷ್ಟು ಚೆನ್ನಾಗಿ ಜೋಡನ್ನು ತಿಕ್ಕುವನು. ಏಕಾಗ್ರತೆಯಿಂದ ಕೂಡಿದ ಅಡಿಗೆಯವನು ಅಡಿಗೆಯನ್ನು ಮತ್ತಷ್ಟು ಚೆನ್ನಾಗಿ ಮಾಡುವನು. ದ್ರವ್ಯಾರ್ಜನೆಯ ವಿಷಯವಾಗಲಿ, ದೇವರ ಪೂಜೆಯ ವಿಷಯ ವಾಗಲಿ, ಅಥವಾ ನೀವು ಏನು ಕೆಲಸ ಮಾಡಬೇಕಾದರೂ, ಏಕಾಗ್ರತೆ ಹೆಚ್ಚಿ ದಷ್ಟು ಕೆಲಸ ಚೆನ್ನಾಗಿ ಆಗುವುದು. ಈ ಕರೆಯೊಂದೆ, ಈ ಶಬ್ದವೊಂದೆ, ಪ್ರಕೃತಿಯ ಬಾಗಿಲನ್ನು ತೆರೆದು ಜ್ಞಾನಜ್ಯೋತಿಯನ್ನು ಹೊರಗೆಡಹುವುದು.


\section{ತರತಮದಲ್ಲಿ ಮಾತ್ರ ವ್ಯತ್ಯಾಸ}

ಸಾಧಾರಣ ಮನುಷ್ಯನು ಶೇಕಡ ತೊಂಭತ್ತರಷ್ಟು ಮಾನಸಿಕ ಶಕ್ತಿಯನ್ನು ವ್ಯರ್ಥಮಾಡುತ್ತಿರುವನು. ಆದಕಾರಣವೇ ಅವನು ಪದೇ ಪದೇ ತಪ್ಪು ಮಾಡುತ್ತಿರುವನು. ಪಳಗಿದ ಮನುಷ್ಯನಾಗಲೀ ಅಥವಾ ಮನಸ್ಸಾಗಲೀ ಎಂದಿಗೂ ತಪ್ಪು ಮಾಡುವುದಿಲ್ಲ. ಮನಷ್ಯನಿಗೂ ಪ್ರಾಣಿಗೂ ಇರುವ ಮುಖ್ಯ ವ್ಯತ್ಯಾಸವೆ ಅವರಲ್ಲಿರುವ ಏಕಾಗ್ರತೆಯ ಶಕ್ತಿಯ ಅಂತರ. ಪ್ರಾಣಿಗಳಿಗೆ ಬಹಳ ಸ್ವಲ್ಪ ಏಕಾಗ್ರತೆಯ ಶಕ್ತಿ ಇದೆ. ಯಾರು ಪ್ರಾಣಿಗಳನ್ನು ತರಬೇತು ಮಾಡಿರುವರೋ ಅವರು ತುಂಬಾ ಕಷ್ಟಪಟ್ಟಿರುವರು. ಏಕೆಂದರೆ ಅವರು ಹೇಳಿದುದನ್ನು ಪ್ರಾಣಿ ಪದೇ ಪದೇ ಮರೆಯುವುದು. ಮನುಷ್ಯನಿಗೂ ಮೃಗಕ್ಕೂ ವ್ಯತ್ಯಾಸ ಇಲ್ಲಿದೆ. ಈ ಏಕಾಗ್ರತೆಯಲ್ಲಿ ಇರುವ ವ್ಯತ್ಯಾಸವೇ ಒಬ್ಬನಿಗೂ ಮತ್ತೊಬ್ಬನಿಗೂ ಇರುವ ಮುಖ್ಯವಾದ ಭೇದಭಾವಕ್ಕೆ ಕಾರಣ. ಅತ್ಯಂತ ಕೆಳಗೆ ಇರುವವನನ್ನೂ ಉತ್ತಮೋತ್ತಮನನ್ನೂ ಹೋಲಿಸಿನೋಡಿ. ಏಕಾಗ್ರತೆಯ ಹೆಚ್ಚು ಕಡಿಮೆಯಲ್ಲಿದೆ ವ್ಯತ್ಯಾಸ.


\section{ಪರಿಣಾಮ}

ಯಾವ ಉದ್ಯಮದಲ್ಲಿ ಆದರೂ ಜಯ ಏಕಾಗ್ರತೆಯ ಮೇಲೆ ನಿಂತಿದೆ. ಕಲೆಯಲ್ಲಿ, ಸಂಗೀತದಲ್ಲಿ, ಇನ್ನೂ ಇತರ ವಿದ್ಯೆಗಳಲ್ಲಿ ಮಹತ್ ಪರಿಣಾಮ ಕಾರಿಗಳಾಗುವುದು ಏಕಾಗ್ರತೆಯ ಪ್ರತಿಫಲ. ಮನಸ್ಸನ್ನು ಏಕಾಗ್ರತೆಗೊಳಿಸಿ ನಮ್ಮ ಮೇಲೆಯೆ ತಿರುಗಿಸಿದರೆ ನಮ್ಮಲ್ಲಿರುವುದೆಲ್ಲ ನಮ್ಮ ಅಡಿಯಾಳಾಗು ತ್ತದೆ, ನಮ್ಮ ಪ್ರಭುವಾಗುವುದಿಲ್ಲ. ಗ್ರೀಕರು ತಮ್ಮ ಏಕಾಗ್ರತೆಯನ್ನು ಬಾಹ್ಯ ಪ್ರಪಂಚದ ಮೇಲೆ ತಿರುಗಿಸಿದರು. ಕಲೆ ಸಾಹಿತ್ಯ ಇತ್ಯಾದಿಗಳಲ್ಲಿ ಪಡೆದ ಪರಿಪೂರ್ಣತೆ ಇದರ ಪರಿಣಾಮ. ಹಿಂದುಗಳು ಅಂತರಂಗದ ಮೇಲೆ ಆತ್ಮ ನಲ್ಲಿರುವ ಅಗೋಚರ ವಿಷಯಗಳ ಮೇಲೆ ಮನಸ್ಸನ್ನು ಏಕಾಗ್ರಮಾಡಿ ಯೋಗಶಾಸ್ತ್ರವನ್ನು ಅಭಿವೃದ್ಧಿಪಡಿಸಿದರು. ನಾವು ಆ ರಹಸ್ಯದ ಬಾಗಿಲನ್ನು ತಟ್ಟುವುದು ಹೇಗೆ, ಅದಕ್ಕೆ ಸಾಕಾದಷ್ಟು ಪೆಟ್ಟನ್ನು ಕೊಡುವುದು ಹೇಗೆ ಎಂಬುದು ನಮಗೆ ಗೊತ್ತಿದ್ದರೆ, ಪ್ರಪಂಚ ತನ್ನ ರಹಸ್ಯವನ್ನು ನಮಗೆ ಕೊಡಲು ಸಿದ್ಧವಾಗಿರುವುದು. ಪೆಟ್ಟಿನ ಶಕ್ತಿ ಮತ್ತು ವೇಗ ಏಕಾಗ್ರತೆಯಿಂದ ಬರುವುದು.


\section{ಜ್ಞಾನದ ಬೀಗದ ಕೈ}

ಜ್ಞಾನಭಂಡಾರಕ್ಕೆ ಏಕಾಗ್ರತೆಯ ಶಕ್ತಿಯೊಂದೇ ಬೀಗದ ಕೈ. ನಮ್ಮ ಈಗಿನ ದೇಹಸ್ಥಿತಿಯಲ್ಲಿ ಮನಸ್ಸು ತುಂಬಾ ಚಂಚಲವಾಗಿದೆ; ನೂರಾರು ವಿಷಯಗಳ ಮೇಲೆ ತನ್ನ ಶಕ್ತಿಯನ್ನು ವ್ಯಯಮಾಡುತ್ತಿದೆ. ಯಾವುದಾದರು ಒಂದು ವಿಷಯದ ಮೇಲೆ ಚಿಂತಿಸಬೇಕೆಂದು ಮನಸ್ಸನ್ನು ಏಕಾಗ್ರಮಾಡಲು ಪ್ರಯತ್ನಿ ಸಿದ ತಕ್ಷಣವೇ ಸಾವಿರಾರು ಆಲೋಚನಾ ಪ್ರೇರೇಪಣೆಗಳು ಪ್ರವೇಶಮಾಡಿ, ಅದನ್ನು ಚಂಚಲಗೊಳಿಸುತ್ತವೆ. ಅದನ್ನು ತಡೆಯುವುದು ಹೇಗೆ, ಮನಸ್ಸನ್ನು ನಮ್ಮ ಸ್ವಾಧೀನಕ್ಕೆ ಹೇಗೆ ತೆಗೆದುಕೊಂಡು ಬರುವುದು–ಎನ್ನುವುದೇ ರಾಜ ಯೋಗದ ಮುಖ್ಯ ಉದ್ದೇಶ. ಧ್ಯಾನಾಭ್ಯಾಸವೂ ಏಕಾಗ್ರತೆಗೆ ದಾರಿ.


\section{ನನ್ನ ಪಾಲಿಗೆ ಶಿಕ್ಷಣದ ಗುರಿ ಏಕಾಗ್ರತೆ, ವಿಷಯಸಂಗ್ರಹವಲ್ಲ}

ನಾನು ಮತ್ತೊಮ್ಮೆ ವಿದ್ಯಾಭ್ಯಾಸವನ್ನು ಮಾಡಬೇಕಾದರೆ ವಿಷಯಗಳನ್ನು ಓದುವುದೇ ಇಲ್ಲ. ಏಕಾಗ್ರತೆಯ ಶಕ್ತಿ ಮತ್ತು ಅನಾಸಕ್ತಿಯನ್ನು ಅಭಿವೃದ್ಧಿ ಗೊಳಿಸಿ, ನಂತರ ನನ್ನ ಇಚ್ಛೆಯಂತೆ ಯಾವ ಕುಂದೂ ಇಲ್ಲದ ಯಂತ್ರದ ಮೂಲಕ ವಿಷಯಗಳನ್ನು ಸಂಗ್ರಹಿಸುತ್ತೇನೆ.


\section{ಏಕಾಗ್ರತೆಗೆ ಬ್ರಹ್ಮಚರ್ಯ ಅತ್ಯಾವಶ್ಯಕ}

ಹನ್ನೆರಡು ವರುಷಗಳ ತನಕ ಯಾರು ಬ್ರಹ್ಮಚರ್ಯವನ್ನು ಅನುಸರಿಸು ತ್ತಾರೋ ಅವರಿಗೆ ಅಪ್ರತಿಮ ಶಕ್ತಿ ಬರುವುದು. ಶುದ್ಧ ಬ್ರಹ್ಮಚರ್ಯ ಮಹಾ ಬುದ್ಧಿ ಶಕ್ತಿಯನ್ನೂ ಮತ್ತು ಆತ್ಮಶಕ್ತಿಯನ್ನೂ ಕೊಡುತ್ತದೆ. ಕಾಮನಿಗ್ರಹವೇ ಅತ್ಯುತ್ತಮವಾದ ಪ್ರತಿಫಲಕ್ಕೆ ದಾರಿ. ಸಂಭೋಗ ಶಕ್ತಿಯನ್ನು ಆತ್ಮಶಕ್ತಿ ಯನ್ನಾಗಿ ಪರಿಣಾಮಗೊಳಿಸಿ. ಈ ಶಕ್ತಿ ಪ್ರಬಲವಾದಷ್ಟೂ ಇದರಿಂದ ಹೆಚ್ಚು ಫಲ ಪಡೆಯಬಹುದು. ಅತ್ಯಂತ ಪ್ರಚಂಡವಾದ ನೀರಿನ ತರಂಗದಿಂದಲೇ ವಿದ್ಯುತ್ ಶಕ್ತಿಯನ್ನು ಉತ್ಪತ್ತಿ ಮಾಡಬಹುದು. ಬ್ರಹ್ಮಚರ್ಯದ ಅಭಾವ ದಿಂದಲೇ ನಮ್ಮ ದೇಶದಲ್ಲಿ ಎಲ್ಲವೂ ಇನ್ನೇನು ನಾಶವಾಗುವ ಸ್ಥಿತಿಯಲ್ಲಿರು ವುದು. ನೈಷ್ಠಿಕ ಬ್ರಹ್ಮಚರ್ಯದ ಮೂಲಕ ಎಲ್ಲಾ ವಿದ್ಯೆಯನ್ನು ಅತ್ಯಲ್ಪ ಕಾಲ ದಲ್ಲಿ ಕಲಿಯಬಹುದು. ಒಂದೇ ಸಲ ಕೇಳಿದ ಅಥವಾ ನೋಡಿದ ವಸ್ತುವನ್ನು ಎಂದಿಗೂ ಮರೆಯದಂತೆ ನೆನಪಿನಲ್ಲಿ ಇಟ್ಟುಕೊಳ್ಳಬಹುದು. ಪರಿಶುದ್ಧ ಮೆದುಳಿಗೆ ಪ್ರಚಂಡ ಮಾನಸಿಕ ಶಕ್ತಿ ಮತ್ತು ಅತ್ಯದ್ಭುತ ಇಚ್ಛಾಶಕ್ತಿ ಇದೆ. ಬ್ರಹ್ಮಚರ್ಯವಿಲ್ಲದೆ ಆತ್ಮ ಶಕ್ತಿ ಇಲ್ಲ. ಬ್ರಹ್ಮಚರ್ಯ ಮಾನವ ಜನಾಂಗದ ಮೇಲೆ ಅತ್ಯದ್ಭುತ ಶಕ್ತಿಯನ್ನು ಕೊಡುವುದು. ಜನಾಂಗದ ಧಾರ್ಮಿಕ ಮುಂದಾಳುಗಳು ದೊಡ್ಡ ಬ್ರಹ್ಮಚಾರಿಗಳಾಗಿದ್ದರು. ಅವರಿಗೆ ಶಕ್ತಿಯನ್ನು ನೀಡಿದ್ದು ಇದು. ಶುದ್ಧ ಬ್ರಹ್ಮಚರ್ಯವನ್ನು ಅನುಷ್ಠಾನಕ್ಕೆ ತರುವಂತೆ ಪ್ರತಿಯೊಬ್ಬ ಹುಡುಗನಿಗೂ ಶಿಕ್ಷಣವನ್ನು ಕೊಡಬೇಕು. ಆಗಲೆ ಶ್ರದ್ಧೆ ಮತ್ತು ನಂಬಿಕೆ ಬರುವುವು. ಯಾವಾಗಲೂ ಎಂತಹ ಸನ್ನಿವೇಶದಲ್ಲಿಯೂ ಮನೋ ವಾಕ್ಕಾಯವಾಗಿ ಇಂದ್ರಿಯವನ್ನು ನಿಗ್ರಹಿಸುವುದೇ ಬ್ರಹ್ಮಚರ್ಯ. ಹೀನ ಆಲೋಚನೆ ಹೀನ ಕೆಲಸದಷ್ಟೆ ಕೆಟ್ಟದ್ದು. ಬ್ರಹ್ಮಚಾರಿ ಮನೋವಾಕ್ಕಾಯವಾಗಿ ಪರಿಶುದ್ಧನಾಗಿರಬೇಕು.


\section{ಎಲ್ಲಾ ಬೆಳವಣಿಗೆಗೂ ಶ್ರದ್ಧೆಯೇ ತಳಹದಿ}

ನಿಜವಾದ ಶ್ರದ್ಧೆಯ ಭಾವನೆಯನ್ನು ನಮಗಾಗಿ ಪುನಃ ತರಬೇಕು. ಆತ್ಮ ಶ್ರದ್ಧೆಯನ್ನು ಪುನಃ ಜಾಗ್ರತಗೊಳಿಸಬೇಕು. ಆಗ ಮಾತ್ರ ನಮ್ಮ ದೇಶವನ್ನು ಬಾಧಿಸುತ್ತಿರುವ ಸಮಸ್ಯೆಗಳೆಲ್ಲ ನಮ್ಮಿಂದಲೇ ಕ್ರಮೇಣ ಪರಿಹಾರವಾಗು ವುವು. ನಮಗೆ ಬೇಕಾಗಿರುವುದು ಇಂತಹ ಶ್ರದ್ಧೆ. ಒಬ್ಬನಿಗೂ ಮತ್ತೊಬ್ಬ ನಿಗೂ ಇರುವ ವ್ಯತ್ಯಾಸಕ್ಕೆ ಕಾರಣ ಅವರಲ್ಲಿರುವ ಶ್ರದ್ಧೆಯ ಹೆಚ್ಚು ಕಡಿಮೆ ಯಲ್ಲದೆ ಮತ್ತೇನೂ ಅಲ್ಲ. ಯಾವುದು ಒಬ್ಬನನ್ನು ಮಹಾ ಪುರುಷನನ್ನಾಗಿ ಮಾಡುವುದೋ ಅದೇ ಶ್ರದ್ಧೆ. ಯಾರು ತಾವು ದುರ್ಬಲರೆಂದು ತಿಳಿಯು ತ್ತಾರೆಯೋ ಅವರು ದುರ್ಬಲರೇ ಆಗುವರು. ಇದು ನಿಜವೆಂದು ನನ್ನ ಗುರು ಹೇಳಿದ್ದರು. ಈ ಶ್ರದ್ಧೆ ನಿಮಗೆ ಬರಬೇಕು. ಪಾಶ್ಚಾತ್ಯ ಜನಾಂಗಗಳಲ್ಲಿ ಎಂತಹ ಬಾಹ್ಯಸಂಪತ್ತು ಮತ್ತು ಶಕ್ತಿ ವ್ಯಕ್ತವಾಗುವುದನ್ನು ನೀವು ನೋಡುತ್ತೀರೊ ಅವೆಲ್ಲವೂ ಈ ಶ್ರದ್ಧೆಯ ಪ್ರತಿಫಲ. ಏಕೆಂದರೆ ಅವರಿಗೆ ತಮ್ಮ ಬಾಹುಬಲ ದಲ್ಲಿ ನಂಬಿಕೆ ಇದೆ. ನೀವು ಆತ್ಮನಲ್ಲಿ ನಂಬಿದರೆ ಮತ್ತಷ್ಟು ಹೆಚ್ಚಾಗಿ ಕೆಲಸ ಮಾಡಬಹುದು.


\section{ಒಬ್ಬನು ತಾನು ಆಲೋಚಿಸಿದಂತೆ ಆಗುತ್ತಾನೆ}

ಈ ಒಂದು ವಿಷಯವನ್ನು ನೀವು ತಿಳಿದುಕೊಳ್ಳಬೇಕೆಂದು ನಾನು ಪ್ರಾರ್ಥಿ ಸುತ್ತೇನೆ. ಯಾರು ಹಗಲು ರಾತ್ರಿ ತಾವು ಯಾವ ಕೆಲಸಕ್ಕೂ ಬಾರದವರೆಂದು ಆಲೋಚಿಸುತ್ತಾರೋ ಅಂತಹ ಮನುಷ್ಯರಿಂದ ಏನೂ ಪ್ರಯೋಜನವಿಲ್ಲ. ಯಾರು ಹಗಲು ರಾತ್ರಿ ತಾನು ಕಷ್ಟದಲ್ಲಿರುವೆನೆಂದೂ ದೀನನೆಂದೂ ಅಪ್ರ ಯೋಜಕನೆಂದೂ ಆಲೋಚಿಸುತ್ತಿರುವನೋ ಅವನು ಅಪ್ರಯೋಜಕನೇ ಆಗು ತ್ತಾನೆ. ನೀವು “ಅಹಮಸ್ಮಿ” “ಅಹಮಸ್ಮಿ” ಎಂದರೆ ಅದರಂತೆಯೇ ಆಗು ತ್ತೀರಿ. ನೀವು ಜ್ಞಾಪಕದಲ್ಲಿಡಬೇಕಾದ ಮುಖ್ಯ ವಿಷಯವೇ ಇದು: ನಾವು ಭಗವಂತನ ಮಕ್ಕಳು, ಅನಂತ ಜ್ಞಾನಜ್ಯೋತಿಯ ಕಿಡಿಗಳು. ನಾವು ಹೇಗೆ ಅಪ್ರಯೋಜಕರಾಗಬಲ್ಲೆವು? ನಾವೇ ಸರ್ವಸ್ವ. ಎಲ್ಲವನ್ನೂ ಮಾಡಲು ಸಿದ್ಧ ರಾಗಿರುವೆವು. ನಾವು ಎಲ್ಲವನ್ನೂ ಮಾಡಬಲ್ಲೆವು. ಈ ಆತ್ಮಶ್ರದ್ಧೆ ನಮ್ಮ ಪೂರ್ವಿಕರ ಹೃದಯದಲ್ಲಿತ್ತು. ನಾಗರಿಕತೆಯ ಪಯಣದಲ್ಲಿ ಅವರನ್ನು ಅಗ್ರ ಸ್ಥಾನಕ್ಕೆ ತಂದ ಕ್ರಿಯೋತ್ತೇಜಕ ಶಕ್ತಿಯೆ ಅವರಲ್ಲಿದ್ದ ಆತ್ಮಶ್ರದ್ಧೆ. ಇಂದು ನಾವು ಅಧಃಪತನ ಹೊಂದಿದ್ದರೆ, ನಮ್ಮಲ್ಲಿ ದೋಷಗಳಿದ್ದರೆ, ನಮ್ಮ ಜನರು ತಮ್ಮಲ್ಲಿದ್ದ ಆತ್ಮಶ್ರದ್ಧೆಯನ್ನು ನೀಗಿಕೊಂಡ ದಿನದಿಂದಲೇ ಆ ಅಧಃಪತನ ಮೊದಲಾಯಿತೆಂದು ನಮಗೆ ಕಾಣುತ್ತದೆ.

ಆತ್ಮಶ್ರದ್ಧೆಯ ಸಿದ್ಧಾಂತವನ್ನು ಅಥವಾ ನಿಷ್ಕಪಟ ನಂಬಿಕೆಯನ್ನು ಬೋಧಿಸುವುದೇ ನನ್ನ ಜೀವನದ ಧ್ಯೇಯ. ಈ ಶ್ರದ್ಧೆಯೇ ಜನಾಂಗದ ಅತ್ಯಂತ ಮುಖ್ಯವಾದ ವಿಷಯ ಎಂಬುದನ್ನು ನಾನು ನಿಮಗೆ ಒತ್ತಿ ಹೇಳುತ್ತೇನೆ. ಮೊದಲು ನಿಮ್ಮಲ್ಲಿ ಶ್ರದ್ಧೆ ಇರಲಿ. ಒಬ್ಬನು ಒಂದು ಸಣ್ಣ ನೀರುಗುಳ್ಳೆ ಯಾಗಿರಬಹುದು. ಮತ್ತೊಬ್ಬನು ಬೆಟ್ಟದ ಎತ್ತರದಷ್ಟು ದೊಡ್ಡ ಅಲೆಯಾಗಿರ ಬಹುದು. ಆದರೂ ಈ ನೀರುಗುಳ್ಳೆ ಮತ್ತು ಅಲೆಯ ಹಿಂದೆ ಅನಂತಸಾಗರ ವಿದೆ ಎಂಬುದನ್ನು ತಿಳಿದುಕೊಳ್ಳಿ. ಈ ಅನಂತ ಸಾಗರವೇ ನಮಗೂ ಮತ್ತು ನಿಮಗೂ ಹಿನ್ನೆಲೆ. ನನ್ನದೂ ಕೂಡ ಅನಂತ ಜೀವನದ ಸಾಗರ, ಅನಂತ ಶಕ್ತಿ, ಅನಂತ ಪವಿತ್ರತೆ, ನಿಮ್ಮದೂ ಕೂಡ ಅದರಂತೆಯೆ. ಆದಕಾರಣ ನನ್ನ ಸಹೋದರರೇ, ಈ ಆತ್ಮೋದ್ಧಾರಕವಾದ ಪುನೀತರನ್ನಾಗಿ ಮಾಡುವ ಪವಿತ್ರ ಸಿದ್ಧಾಂತವನ್ನು ನಿಮ್ಮ ಮಕ್ಕಳಿಗೆ ಜನನಾರಂಭದಿಂದಲೇ ಬೋಧಿಸಿ.

