
\chapter{ಪುರುಷಸಿಂಹರನ್ನು ಮಾಡುವ ವಿದ್ಯಾಭ್ಯಾಸದ ಆವಶ್ಯಕತೆ}

\section{ವಿದ್ಯಾಭ್ಯಾಸದ ಪ್ರಾಮುಖ್ಯತೆ}

ಯೂರೋಪ್​ದೇಶದ ಅನೇಕ ಪಟ್ಟಣಗಳಲ್ಲಿ ಸಂಚಾರ ಮಾಡುವಾಗ ಸಾಧಾರಣ ಬಡವರಿಗೆ ದೊರುಕುವ ಜೀವನ ಸೌಕರ್ಯ ಮತ್ತು ವಿದ್ಯಾಭ್ಯಾಸವನ್ನು ನೋಡಿದಾಗ, ನಮ್ಮ ದೇಶದ ಬಡಜನರ ದುಃಸ್ಥಿತಿ ನನಗೆ ವ್ಯಕ್ತವಾಯಿತು. ಆಗ ನಾನು ಕಣ್ಣೀರು ಕರೆಯುತ್ತಿದ್ದೆ. ಈ ವ್ಯತ್ಯಾಸಕ್ಕೆ ಕಾರಣವೇನು! ನನಗೆ ದೊರಕಿದ ಉತ್ತರವೇ ವಿದ್ಯಾಭ್ಯಾಸ. ವಿದ್ಯಾಭ್ಯಾಸ ಮತ್ತು ಆತ್ಮಶ್ರದ್ಧೆಯ ಮೂಲಕ ಅವರಲ್ಲಿ ಸುಪ್ತವಾದ ಬ್ರಹ್ಮ ವ್ಯಕ್ತವಾಗುವುದು.

ನಾನು ನ್ಯೂಯಾರ್ಕಿನಲ್ಲಿದ್ದಾಗ ಐರ್ಲೆಂಡಿನಿಂದ ಬರುವ ಜನರನ್ನು ನೋಡುತ್ತಿದ್ದೆ. ದಬ್ಬಾಳಿಕೆಗೆ ತುತ್ತಾಗಿ, ಜೋಲು ಮುಖದಿಂದ, ಮನೆಯಲ್ಲಿ ತಮ್ಮದೆಂಬ ಒಂದು ಸಾಮಾನೂ ಇಲ್ಲದೆ ನಿರ್ಗತಿಕರಾಗಿ, ತಲೆಗೆ ಒಂದು ಟೋಪಿಯೂ ಇಲ್ಲದೇ ಇದ್ದರು. ಅವರ ಜೀವನದ ಸರ್ವಸ್ವವೇ ಒಂದು ಕೋಲು ಮತ್ತುಅದರ ಕೊನೆಯಲ್ಲಿ ನೇತಾಡುವ ಒಂದು ಚಿಂದಿಗಂಟು. ಕಾಲು ನಡುಗುತ್ತಿತ್ತು. ಕಣ್ಣಿನಲ್ಲಿ ಅಂಜಿಕೆ ಸುಳಿಯುತ್ತಿತ್ತು. ಆರು ತಿಂಗಳಲ್ಲಿ ದೃಶ್ಯವೇ ಬದಲಾಯಿಸಿತು. ಈಗ ಅದೇ ಮನುಷ್ಯನು ಧೈರ್ಯದಿಂದ ನಡೆಯುತ್ತಾನೆ. ಅವನ ಉಡುಪು ಬದಲಾಯಿಸಿದೆ. ಅವನ ಕಣ್ಣಿನಲ್ಲಿ, ಇಡುವ ಹೆಜ್ಜೆಯಲ್ಲಿ, ಅಂಜಿಕೆಯ ಸುಳಿವೇ ಇಲ್ಲ. ಇದಕ್ಕೆ ಕಾರಣವೇನು? ಐರ್ಲೆಂಡಿನವನು ತನ್ನ ದೇಶದಲ್ಲಿದ್ದಾಗ ಸುತ್ತಲೂ ಅನಾದರಣೀಯ ವಾತಾವರಣ ಆವರಿಸಿ ಕೊಂಡಿತ್ತು. ಪ್ರಕೃತಿಯೆಲ್ಲ ಒಂದು ಧ್ವನಿಯಿಂದ, “ಪ್ಯಾಟ್, ನಿನಗೇನು ಗತಿಯಿಲ್ಲ, ನೀನು ಹುಟ್ಟುಗುಲಾಮ, ಹಾಗೆಯೇ ಇರುವೆ” ಎಂದು ಹೇಳು ತ್ತಿತ್ತು. ಹುಟ್ಟಿನಿಂದಲೂ ಹಾಗೆ ಹೇಳುತ್ತಿರುವುದನ್ನು ಕೇಳಿ, ಪ್ಯಾಟನು ಅದರಲ್ಲಿ ನಂಬಿದನು. ತಾನು ಅತ್ಯಂತ ಹೀನನೆಂದು ಭ್ರಾಂತಿಯಿಂದ ತಿಳಿದುಕೊಳ್ಳುವು ದಕ್ಕೆ ಮೊದಲುಮಾಡಿದನು. ಆದರೆ ಅಮೇರಿಕಾ ದೇಶಕ್ಕೆ ಬಂದೊಡನೆಯೇ, “ಪ್ಯಾಟ್, ನಮ್ಮಂತೆಯೇ ನೀನು ಒಬ್ಬ ಮನುಷ್ಯ. ಇದನ್ನೆಲ್ಲ ಮಾಡಿದವನು ಮನುಷ್ಯ. ನಿನ್ನಂತಹ ಮತ್ತು ನನ್ನಂತಹ ಮನುಷ್ಯರು ಏನನ್ನು ಬೇಕಾದರೂ ಸಾಧಿಸಬಹುದು. ಧೈರ್ಯ ತಾಳು.” ಎಂಬ ಧ್ವನಿ ಸುತ್ತಲೂ ಅನುರಣಿತ ವಾಗಿತ್ತು. ಪ್ಯಾಟನು ತಲೆಯೆತ್ತಿ ನೋಡಿದನು. ಅದು ನಿಜವಾಗಿಯೂ ಸತ್ಯ ವಾಗಿತ್ತು. ಪ್ರಕೃತಿ “ಏಳು, ಜಾಗ್ರತನಾಗು! ಗುರಿಯನ್ನು ಸೇರುವವರೆಗೂ ನಿಲ್ಲಬೇಡ” ಎಂದು ಹೇಳುತ್ತಿದ್ದಂತೆ ತೋರಿತು.


\section{ನಮ್ಮದು ನಿಷೇಧಮಯ ವಿದ್ಯಾಭ್ಯಾಸ}

ಅವರ ದೇಶದಲ್ಲಿರುವ ಐರಿಷ್ ಜನಾಂಗದಂತೆ ನಮ್ಮ ಹುಡುಗರು ನಿಷೇಧಮಯವಾದ ವಿದ್ಯಾಭ್ಯಾಸವನ್ನು ಪಡೆಯುವರು. ಇದರಲ್ಲಿ ಕೆಲವು ಒಳ್ಳೆಯ ವಿಷಯಗಳೇನೋ ಇವೆ. ಆದರೆ ಇದರಲ್ಲಿರುವ ನ್ಯೂನತೆಗಳೊಂದಿಗೆ ಹೋಲಿಸಿ ನೋಡಿದರೆ ಒಳ್ಳೆಯದು ತೃಣಮಾತ್ರವಾಗುವುದು. ಮೊದಲನೆಯ ದಾಗಿ ಇದು ಪುರುಷಸಿಂಹರನ್ನಾಗಿ ಮಾಡುವ ವಿದ್ಯಾಭ್ಯಾಸವಲ್ಲ. ಇದು ಸಂಪೂರ್ಣ ನಿಷೇಧ ಪ್ರಾಧಾನ್ಯ ವಿದ್ಯಾಭ್ಯಾಸ. ನಿಷೇಧ ಪ್ರಾಧಾನ್ಯ ವಿದ್ಯಾಭ್ಯಾಸ ಅಥವಾ ಅದರ ತಳಹದಿಯ ಮೇಲೆ ನಿಂತ ಎಂತಹ ಶಿಕ್ಷಣವಾದರೂ ಮರಣಕ್ಕಿಂತ ಕೇಡು.

ನಾವು ಕೆಲಸಕ್ಕೆ ಬಾರದವರು ಎಂಬುದನ್ನು ಮಾತ್ರ ಕಲಿತಿರುವೆವು. ನಮ್ಮ ದೇಶದಲ್ಲಿ ಮಹಾವ್ಯಕ್ತಿಗಳು ಜನ್ಮವೆತ್ತಿದ್ದರು ಎಂಬುದನ್ನು ತಿಳಿಯುವಂತೆ ಮಾಡಿರುವುದು ಅಪರೂಪ. ಸ್ವಷ್ಟವಾದ ಯಾವುದನ್ನೂ ನಮಗೆ ಹೇಳಿ ಕೊಟ್ಟಿಲ್ಲ. ನಾವು ಹೇಗೆ ಕಾರ್ಯೋನ್ಮುಖರಾಗಬೇಕೆಂಬುದನ್ನು ಕೂಡ ನಾವು ಕಲಿಯುವುದಿಲ್ಲ. ಇದರ ಪರಿಣಾಮವಾಗಿಯೇ ಐವತ್ತು ವರುಷಗಳ ಇಂತಹ ವಿದ್ಯಾಭ್ಯಾಸದಿಂದ ಒಬ್ಬ ಸ್ವಂತ ಬುದ್ಧಿವಂತನೂ ಆಗಿಲ್ಲ. ಪ್ರತಿಯೊಬ್ಬ ಸ್ವಂತ ಬುದ್ಧಿವಂತನೂ ಕೂಡ ಮತ್ತೆಲ್ಲೊ ವಿದ್ಯಾಭ್ಯಾಸವನ್ನು ಪಡೆದನು, ಈ ದೇಶದಲ್ಲಿ ಅಲ್ಲ. ತಮಗೆ ಪ್ರಾಯಶ್ಚಿತ್ತ ಮಾಡಿಕೊಳ್ಳಲು ಹಳೆಯ ವಿದ್ಯಾ ಸಂಸ್ಥೆಗಳಿಗೆ ಹೋಗಿರುವರು.


\section{ವಿದ್ಯೆ ಕೇವಲ ವಿಷಯಸಂಗ್ರಹವಲ್ಲ}

ವಿದ್ಯಾಭ್ಯಾಸವೆಂದರೆ ನಿಮ್ಮ ಮೆದುಳಿನಲ್ಲಿ ಶೇಖರಿಸಿದ ವಿಷಯಸಂಗ್ರಹ ರಕ್ತಗತವಾಗದೆ ಬದುಕಿರುವ ತನಕ ಚೆಲ್ಲಾಪಿಲ್ಲಿಯಾಗಿ ಓಡಾಡುವುದಲ್ಲ. ಜೀವನವನ್ನು ಉತ್ತಮಗೊಳಿಸುವ ಪುರುಷಸಿಂಹರನ್ನು ಮಾಡುವ ವಿದ್ಯೆ ಬೇಕು. ಐದು ವಿಷಯಗಳನ್ನು ನೀವು ಅರ್ಥಮಾಡಿಕೊಂಡು, ಅದನ್ನು ನಿಮ್ಮ ಜೀವನದಲ್ಲಿ ಕಾರ್ಯರೂಪಕ್ಕೆ ತಂದು, ಅದರಿಂದ ಶೀಲವನ್ನು ರೂಪಿಸಿ ಕೊಂಡಿದ್ದರೆ, ಒಂದು ಪುಸ್ತಕಭಂಡಾರವನ್ನೇ ಕಂಠಪಾಠಮಾಡಿದ ಪಂಡಿತನಿ ಗಿಂತಲೂ ನೀವು ಮೇಲು. ಕೇವಲ ವಿಷಯ ಸಂಗ್ರಹವೇ ವಿದ್ಯಾಭ್ಯಾಸವಾಗಿ ದ್ದರೆ—ಪ್ರಪಂಚದಲ್ಲಿ ಪುಸ್ತಕಭಂಡಾರಗಳೇ ಮಹಾಮುನಿಗಳಾಗುತ್ತಿದ್ದವು, ವಿಶ್ವ ಕೋಶಗಳೇ ಮಹಾಪುಷಿಗಳಾಗುತ್ತಿದ್ದವು.

ಪರಭಾಷೆಯಲ್ಲಿ ಮತ್ತೊಬ್ಬರ ಭಾವನೆಗಳನ್ನು ಕಂಠಪಾಠಮಾಡಿ, ಅದ ರಿಂದ ನಿಮ್ಮ ಮೆದುಳನ್ನು ತುಂಬಿಕೊಂಡು ವಿಶ್ವವಿದ್ಯಾನಿಲಯದ ಪದವಿ ಗಳಿಸಿದ್ದರೆ ವಿದ್ಯಾವಂತರಾದೆವೆಂದು ಹಿಗ್ಗುತ್ತೀರಿ. ಇದು ವಿದ್ಯಾಭ್ಯಾಸವೇ? ನಿಮ್ಮ ವಿದ್ಯಾಭ್ಯಾಸದ ಗುರಿ ಏನು? ಗುಮಾಸ್ತಗಿರಿ, ಇಲ್ಲವೆ ಒಬ್ಬ ವಕೀಲ ನಾಗಿರುವುದು. ಅಥವಾ ಹೆಚ್ಚು ಎಂದರೆ ಒಬ್ಬ ಡೆಪ್ಯೂಟಿ ಮ್ಯಾಜಿಸ್ಟ್ರೇಟ್ ಆಗುವುದು! ಇದು ಮತ್ತೊಂದು ವಿಧದ ಗುಮಾಸ್ತಗಿರಿ. ಇದೇ ಅಲ್ಲವೆ! ಇದರಿಂದ ನಿಮಗೆ ಅಥವಾ ನಿಮಗಿಂತ ಹೆಚ್ಚಾಗಿ ನಿಮ್ಮ ದೇಶಕ್ಕೆ ಯಾವ ಹಿತಸಾಧನೆಯಾಗುವುದು? ಕಣ್ಣು ತೆರೆಯಿರಿ! ನೋಡಿ! ಅನ್ನಕ್ಕೆ ತೌರೂರು ಎನಿಸಿದ ಭರತಖಂಡದಲ್ಲಿ ಒಪ್ಪೊತ್ತಿನ ಕೂಳಿಗಾಗಿ ಏಳುತ್ತಿರುವ ಹೃದಯ ವಿದ್ರಾವಕ ಧ್ವನಿಯನ್ನು ಕೇಳಿ. ನಿಮ್ಮ ವಿದ್ಯಾಭ್ಯಾಸ ಈ ಕೊರತೆಯನ್ನು ಹೋಗ ಲಾಡಿಸಬಲ್ಲದೆ? ಯಾವ ವಿದ್ಯೆ ಅನುದಿನದ ಜೀವನದ ಹೋರಾಟಕ್ಕೆ ಬೇಕಾ ಗುವ ಸಲಕರಣೆಗಳನ್ನು ಸಾಧಾರಣ ಜನರಿಗೆ ಒದಗಿಸಿಕೊಡಲಾರದೋ, ಯಾರಲ್ಲಿ ಶೀಲಶಕ್ತಿಯನ್ನು ವ್ಯಕ್ತವಾಗುವಂತೆ ಮಾಡಲಾರದೋ, ಯಾರಿಗೆ ಸಿಂಹಸದೃಶ ಧೈರ್ಯವನ್ನು ಒದಗಿಸಿಕೊಡಲಾರದೋ, ಅಂತಹ ವಿದ್ಯೆಯಿಂದ ಏನಾದರೂ ಪ್ರಯೋಜನವಿದೆಯೆ?


\section{ನಮಗೆ ಏನು ಬೇಕು?}

ಯಾವ ವಿದ್ಯಾವ್ಯಾಸಂಗದಿಂದ, ಒಳ್ಳೆಯ ಶೀಲ ನಮ್ಮಲ್ಲಿ ಬೆಳೆಯು ವುದೋ, ಮಾನಸಿಕಶಕ್ತಿ ಹೆಚ್ಚುವುದೋ, ಬುದ್ಧಿ ವಿಶಾಲವಾಗುವುದೋ, ಯಾವುದರ ಸಹಾಯದಿಂದ ವ್ಯಕ್ತಿ ಸ್ವತಃ ತನ್ನ ಕಾಲ ಮೇಲೆ ನಿಲ್ಲಬಲ್ಲನೋ, ಅಂತಹ ವಿದ್ಯೆ ನಮಗೆ ಬೇಕಾಗಿರುವುದು. ಪರದೇಶದವರ ಅಧೀನಕ್ಕೆ ಒಳ ಪಡದೆ ನಮ್ಮದೇ ಆದ ಹಲವು ಶಾಸ್ತ್ರಗಳನ್ನು ಓದಿ, ಜೊತೆಗೆ ಆಂಗ್ಲಭಾಷೆ ಮತ್ತು ಪಾಶ್ಚಾತ್ಯ ವಿಜ್ಞಾನಶಾಸ್ತ್ರ ಇವನ್ನು ಅಧ್ಯಯನಮಾಡಬೇಕು. ನಮ್ಮ ಕೈಗಾರಿಕೆಯನ್ನು ಅಭಿವೃದ್ಧಿ ಪಡಿಸುವಂತಹ ಔದ್ಯೋಗಿಕ ಮತ್ತು ಅದಕ್ಕೆ ಸಂಬಂಧಪಟ್ಟ ಇತರ ವಿದ್ಯಾವಿಷಯಗಳು ಬೇಕು. ಏಕೆಂದರೆ ಜನರು ಚಾಕರಿ ಯನ್ನು ಹುಡುಕಿಕೊಂಡು ಹೋಗುವುದಕ್ಕಿಂತ, ತಮಗೆ ಸಾಕಾದಷ್ಟನ್ನು ತಾವೇ ಸಂಪಾದಿಸಿ, ಕಷ್ಟಕಾಲಕ್ಕೆ ಕೊಂಚ ಹಣವನ್ನು ಕೂಡಿಡುವುದಕ್ಕೆ ವಿದ್ಯಾಭ್ಯಾಸ ಸಹಾಯಕವಾಗಬೇಕು.


\section{ಪುರುಷಸಿಂಹರನ್ನು ಮಾಡುವ ವಿದ್ಯೆ}

ಎಲ್ಲಾ ವಿದ್ಯಾವ್ಯಾಸಂಗದ ಎಲ್ಲಾ ತರಬೇತಿನ ಗುರಿ ಪುರುಷಸಿಂಹರನ್ನು ಮಾಡುವುದು. ಎಲ್ಲಾ ತರಬೇತಿಯ ಪರಮಗುರಿ ಮನುಷ್ಯನನ್ನು ಬೆಳೆಯು ವಂತೆ ಮಾಡುವುದು. ಯಾವ ತರಬೇತಿಯಿಂದ ನಮ್ಮ ಇಚ್ಛಾ ಶಕ್ತಿಯ ಪ್ರವಾಹ ಮತ್ತು ಅದನ್ನು ವ್ಯಕ್ತಗೊಳಿಸುವ ವಿಧಾನ ನಮ್ಮ ಸ್ವಾಧೀನಕ್ಕೆ ಬರುವುದೊ ಅದೇ ವಿದ್ಯಾಭ್ಯಾಸ. ನಮ್ಮ ದೇಶಕ್ಕೆ ಇಂದು ಅವಶ್ಯವಾಗಿರುವುದು ಕಬ್ಬಿಣದಂತಹ ಮಾಂಸಖಂಡ, ಉಕ್ಕಿನಂತಹ ನರ, ಯಾವುದಕ್ಕೂ ಜಗ್ಗದ ಪ್ರಚಂಡ ಇಚ್ಛಾಶಕ್ತಿ. ಪ್ರಪಂಚದ ಗೂಢತಮ ರಹಸ್ಯಗಳನ್ನು ಭೇದಿಸಿ, ಅವಶ್ಯವಿದ್ದರೆ ಕಡಲಿನ ಆಳಕ್ಕೆ ಹೋಗಲು ಸಿದ್ಧರಾಗಿ ತಮ್ಮ ಬಯಕೆಯನ್ನು ಮನಸ್ಸಿಗೆ ಸೂಕ್ತ ತೋರಿದ ರೀತಿಯಲ್ಲಿ ಈಡೇರಿಸಿಕೊಳ್ಳಬಲ್ಲವರಾಗಬೇಕು. ನಮಗೆ ಇಂದು ಬೇಕಾಗಿರುವುದು ಪುರುಷಸಿಂಹರನ್ನು ಮಾಡುವ ವಿದ್ಯೆ; ಪುರುಷಸಿಂಹರನ್ನು ಮಾಡುವ ಸಿದ್ಧಾಂತ: ಸರ್ವತೋಮುಖರಾಗಿ ಮನುಷ್ಯ ರನ್ನು ಧೀರರನ್ನಾಗಿ ಮಾಡುವ ವಿದ್ಯೆ.

