
\chapter{ಗುರು ಶಿಷ್ಯರು}

\section{ಗುರುವಿನ ಪ್ರತ್ಯಕ್ಷ ಜೀವನದ ಪ್ರಾಮುಖ್ಯತೆ}

ನನ್ನ ವಿದ್ಯಾಭ್ಯಾಸ ಭಾವನೆಯೇ ಗುರುಗೃಹವಾಸ. ಗುರುವಿನ ಆದರ್ಶ ಜೀವನವಿಲ್ಲದೆ ವಿದ್ಯಾಭ್ಯಾಸವಿಲ್ಲ. ಬಾಲ್ಯದಿಂದಲೆ ಒಬ್ಬನು ಯಾರ ಶೀಲ ಜಾಜ್ವಲ್ಯಮಾನವಾಗಿ ಬೆಳಗುತ್ತಿದೆಯೋ ಅಂತಹವರ ಹತ್ತಿರ ಇರಬೇಕು. ಪರಮಪವಿತ್ರವಾದ ಉಪದೇಶದ ಸಚೇತನ ಉದಾಹರಣೆಯೊಂದು ಅವನ ಮುಂದೆ ಇರಬೇಕು. ನಮ್ಮ ದೇಶದಲ್ಲಿ ವಿದ್ಯಾದಾನ ಮಾಡುವವರು ಯಾವಾ ಗಲೂ ತ್ಯಾಗಿಗಳಾಗಿದ್ದರು. ವಿದ್ಯಾದಾನದ ಜವಾಬ್ದಾರಿ ಪುನಃ ತ್ಯಾಗಿಗಳ ಮೇಲೆ ಬೀಳಬೇಕು.


\section{ಹಿಂದಿನ ವಿದ್ಯಾಭ್ಯಾಸ ಕ್ರಮ}

ಭರತಖಂಡದಲ್ಲಿ ಹಿಂದಿನ ವಿದ್ಯಾಭ್ಯಾಸ ಕ್ರಮ ಇಂದಿನ ವಿದ್ಯಾಭ್ಯಾಸದ ಕ್ರಮಕ್ಕಿಂತ ಬಹಳ ವ್ಯತ್ಯಾಸವಾಗಿತ್ತು. ಶಿಷ್ಯರು ಏನನ್ನೂ ಕೊಡಬೇಕಾಗಿರ ಲಿಲ್ಲ. ಜ್ಞಾನ ಬಹಳ ಪವಿತ್ರವಾದದ್ದು. ಅದನ್ನು ಯಾರೂ ಮಾರಕೂಡದೆಂದು ಭಾವಿಸಿದ್ದರು. ಉಚಿತವಾಗಿ ಯಾವ ಬೆಲೆಯನ್ನೂ ಪಡೆಯದೆ ಜ್ಞಾನವನ್ನು ದಾನಮಾ[ಡಬೇಕು. ಗುರುಗಳು ಶಿಷ್ಯರಿಂದ ಯಾವ ವಿಧದ ವಿದ್ಯಾಶುಲ್ಕವನ್ನೂ ತೆಗೆದುಕೊಳ್ಳುತ್ತಿರಲಿಲ್ಲ. ಅದು ಮಾತ್ರವಲ್ಲ ಮುಕ್ಕಾಲು ಪಾಲು ಶಿಷ್ಯರಿಗೆ ಊಟ ಮತ್ತು ಬಟ್ಟೆಯನ್ನು ಕೊಡುತ್ತಿದ್ದರು. ಇಂತಹ ಗುರುಗಳ ಜೀವ ನೋಪಾಯಕ್ಕೆ ಶ್ರೀಮಂತರು ಅವರಿಗೆ ದಾನ ಮಾಡುತ್ತಿದ್ದರು. ಇದರಿಂದ ಗುರುಗಳು ಶಿಷ್ಯರನ್ನು ಉಚಿತವಾಗಿ ಇಟ್ಟುಕೊಳ್ಳಬೇಕಾಗಿತ್ತು. ಹಿಂದಿನ ಕಾಲ ದಲ್ಲಿ ಶಿಷ್ಯನು ಸಮಿತ್ತನ್ನು ತೆಗೆದುಕೊಂಡು ಗುರುವಿನ ಪರ್ಣಶಾಲೆಗೆ ಹೋಗು ತ್ತಿದ್ದನು. ಗುರು ಅವನ ಯೋಗ್ಯತೆಯನ್ನು ಪರೀಕ್ಷೆಮಾಡಿದ ಮೇಲೆ ಮೂರೆಳೆಯ ಮುಂಜಾದರ್ಭೆಯನ್ನು ಅವನ ಸೊಂಟಕ್ಕೆ ಕಟ್ಟುತ್ತಿದ್ದನು. ಇದು ದೇಹ, ಮಾತು, ಮನಸ್ಸುಗಳನ್ನು ನಿಗ್ರಹಿಸುತ್ತೇನೆ ಎಂಬ ವ್ರತವನ್ನು ಸೂಚಿಸು ತ್ತದೆ. ಅನಂತರ ಅವನಿಗೆ ವೇದ ಹೇಳಿಕೊಡುತ್ತಿದ್ದನು.


\section{ಶಿಷ್ಯನ ಗುಣಗಳು}

ಗುರು ಮತ್ತು ಶಿಷ್ಯರಲ್ಲಿ ಕೆಲವು ಗುಣಗಳು ಆವಶ್ಯಕ. ಶಿಷ್ಯನಿಗೆ ಆವಶ್ಯಕವಾದ ಗುಣಗಳು ಭಾವಶುದ್ಧಿ, ನಿಜವಾದ ಜ್ಞಾನಾಕಾಂಕ್ಷೆ ಮತ್ತು ದೀರ್ಘಪ್ರಯತ್ನ. ಕಾಯೇನ ವಾಚಾ ಮನಸಾ ಪ್ರಯತ್ನ ಅತ್ಯಂತ ಆವಶ್ಯಕ. ಜ್ಞಾನಾಕಾಂಕ್ಷೆಯ ವಿಷಯದಲ್ಲಿ ಒಂದು ಹಳೆಯ ನಿಯಮವಿದೆ. ಅದೇ ನಮಗೆ ಏನು ಬೇಕೊ ಅದನ್ನು ಪಡೆದೇ ಪಡೆಯುತ್ತೇವೆ ಎಂಬುದು. ನಾವು ಏನನ್ನು ಮನಸಾ ಇಚ್ಛಿಸುತ್ತೇವೆಯೋ ಅದಲ್ಲದೆ ಮತ್ತೇನನ್ನೂ ಪಡೆಯುವುದಿಲ್ಲ. ಜೀವನದಲ್ಲಿ ಅತ್ಯುತ್ತಮವಾದ ಬಯಕೆ ನಮ್ಮಲ್ಲಿ ಉದಯಿಸುವ ತನಕ, ಜಯವನ್ನು ಹೊಂದುವ ತನಕ, ನಮ್ಮ ಹೀನ ಸಂಸ್ಕಾರಳೊಂದಿಗೆ ಸತತ ಹೋರಾಟ ನಡೆಯಬೇಕು. ಸ್ವಲ್ಪವೂ ಕುಗ್ಗದೆ ಅವುಗಳನ್ನು ಎದುರಿಸಬೇಕು. ಇಂತಹ ದೃಢ ಪ್ರಯತ್ನದಿಂದ ಯಾವ ಶಿಷ್ಯನು ಹೋರಾಡುತ್ತಾನೆಯೋ ಅವನು ಕೊನೆಯ ಜಯವನ್ನು ಸತ್ಯವಾಗಿ ಪಡೆಯುವನು.


\section{ಗುರುವಿನ ಮೂರು ಗುಣಗಳು}

ಗುರುವಿಗೆ ಶಾಸ್ತ್ರ ಸಾರ ಗೊತ್ತಿರುವುದೇ ಎಂಬುದನ್ನು ನೋಡಬೇಕು. ಜನರೆಲ್ಲ ಬೈಬಲ್ಲು ವೇದ ಖೊರಾನನ್ನು ಓದುತ್ತಾರೆ. ಆದರೆ ಅದೆಲ್ಲ ಧರ್ಮದ ಒಣಮೂಳೆ. ಇವು ಕೇವಲ ಪದ, ವಾಕ್ಯರಚನೆ, ಶಬ್ದಸಾಧನವಿದ್ಯೆ \eng{(etimo- logy)} ಮತ್ತು ಭಾಷಾಶಾಸ್ತ್ರಗಳು ಅಷ್ಟೆ. ಕೇವಲ ಶಬ್ದಾಡಂಬರದಲ್ಲೇ ನಿರತನಾದ ಗುರು ಅರ್ಥವನ್ನು ಮರೆಯುವನು. ನಿಜವಾದ ಗುರುವಿನಲ್ಲಿ ಶಾಸ್ತ್ರದ ಅರ್ಥಜ್ಞಾನ ಇರಬೇಕು.

ಎರಡೆನೇ ಗುಣವೇ ಗುರು ಪಾಪದೂರನಾಗಿರಬೇಕು. ನಾವು ಗುರುವಿನ ಶೀಲ ಅಥವಾ ಗುಣವನ್ನು ಏತಕ್ಕೆ ನೋಡಬೇಕು ಎಂಬ ಪ್ರಶ್ನೆಯನ್ನು ಅನೇಕ ವೇಳೆ ಕೇಳುತ್ತಾರೆ. ಇದು ಸರಿಯಲ್ಲ. ತನಗಾಗಲಿ ಅಥವಾ ಇನ್ನೊಬ್ಬರಿಗೆ ಉಪದೇಶ ಮಾಡುವುದಕ್ಕಾಗಲಿ ಸತ್ಯವನ್ನು ಪಡೆಯಬೇಕಾದರೆ ಅತ್ಯಂತ ಮುಖ್ಯವಾದುದೆ ಜೀವನದ ಪರಿಶುದ್ಧತೆ. ಗುರು ಅತ್ಯಂತ ಪರಿಶುದ್ಧನಾಗಿರ ಬೇಕು. ನಂತರ ಅವನ ಮಾತಿಗೆ ಬೆಲೆ ಬರುವುದು. ಗುರುವಿನ ಕೆಲಸ ನಿಜವಾಗಿ ಏನನ್ನೋ ದಾನಮಾಡುವುದು, ಆಗಲೆ ಶಿಷ್ಯನಲ್ಲಿ ಇರುವ ಬೌದ್ಧಿಕ ಅಥವಾ ಇತರ ವೈಶಿಷ್ಟ್ಯಗಳನ್ನು ಪ್ರಚೋದಿಸುವುದಲ್ಲ, ನಿಜವಾದ ಮತ್ತು ತೃಪ್ತಿಕರ ವಾದ ಯಾವುದೋ ಒಂದು ಮಹಾಶಕ್ತಿ ಗುರುವಿನಿಂದ ಶಿಷ್ಯನನ್ನು ಪ್ರವೇಶಿಸು ವುದು. ಆದಕಾರಣವೇ ಗುರು ಪರಿಶುದ್ಧನಾಗಿರಬೇಕು.

ಮೂರನೇ ಗುಣವೇ ಅವನ ಉದ್ದೇಶವನ್ನು ಕುರಿತದ್ದು. ದ್ರವ್ಯ ಕೀರ್ತಿ ಮುಂತಾದ ಸ್ವಾರ್ಥಾಭಿಲಾಷೆಗಳ ಉದ್ದೇಶದಿಂದ ಬೋಧಿಸಕೂಡದು. ಗುರು ವಿನ ಕೆಲಸ ಸಹಜಪ್ರೇಮದಿಂದ ಕೂಡಿದ್ದಾಗಿರಬೇಕು. ಆಧ್ಯಾತ್ಮಿಕ ಶಕ್ತಿಯನ್ನು ನಾವು ದಾನಮಾಡಬೇಕಾದರೆ ಇರುವ ಒಂದು ಮಾರ್ಗವೇ ಪ್ರೀತಿ. ಲಾಭ ಅಥವಾ ಕೀರ್ತಿಯ ಯಾವ ವಿಧವಾದ ಸ್ವಾರ್ಥಾಭಿಲಾಷೆಯಾಗಲಿ ತಕ್ಷಣವೇ ರವಾನಿಸುವ ಮಾರ್ಗವನ್ನು ಹಾಳುಮಾಡುವುದು.


\section{ಗುರುವಿನಲ್ಲಿ ಶ್ರದ್ಧೆ}

ಗುರುವಿನೊಂದಿಗೆ ಇರುವ ನಮ್ಮ ಸಂಬಂಧ ಪೂರ್ವಿಕರು ಮತ್ತು ಅವರ ವಂಶಜರೊಂದಿಗೆ ಇರುವ ಪರಸ್ಪರ ಸಂಬಂಧದಂತೆ. ಗುರುವಿನ ಮಾತಿನಲ್ಲಿ ಶ್ರದ್ಧೆ, ವಿನಯ, ಅವರು ಹೇಳಿದಂತೆ ಕೇಳುವುದು, ಅವರಿಗೆ ಗೌರವ ತೋರು ವುದು ಇವು ನಮ್ಮಲ್ಲಿ ಇಲ್ಲದೆ ಇದ್ದರೆ, ನಾವು ಯಾವ ರೀತಿಯಲ್ಲಿಯೂ ಮುಂದುವರಿಯಲಾರೆವು. ಇಂತಹ ಪರಸ್ಪರ ಸಂಬಂಧವನ್ನು ಯಾವ ದೇಶ ದಲ್ಲಿ ಅಲ್ಲಗಳೆದಿರುವರೊ,ಅಲ್ಲಿ ಗುರು ಕೇವಲ ಉಪನ್ಯಾಸಕನಾಗಿರುವನು. ಉಪನ್ಯಾಸಕ ತನ್ನ ಐದು ಡಾಲರನ್ನು ಎದುರು ನೋಡುತ್ತಿರುತ್ತಾನೆ. ಶಿಷ್ಯ ನಾದರೋ ಗುರುವಿನ ಮಾತುಗಳನ್ನು ತನ್ನ ತಲೆಯಲ್ಲಿ ತುಂಬಿಸಿಕೊಳ್ಳುತ್ತಾನೆ. ಇದಾದಮೇಲೆ ಇಬ್ಬರೂ ತಮ್ಮ ದಾರಿಯನ್ನು ತಾವು ಹಿಡಿದುಕೊಂಡು ಹೋಗುವರು. ಆದರೆ ವ್ಯಕ್ತಿಯ ಮೇಲೆ ಅತಿಯಾದ ಶ್ರದ್ಧೆಯಿದ್ದರೆ ಅದು ನಮ್ಮನ್ನು ನಿರ್ಬಲರನ್ನಾಗಿ ಮಾಡಿ ಒಂದು ರೀತಿಯ ವಿಗ್ರಹಾರಾಧಕರನ್ನಾಗಿ ಮಾಡುವ ಒಂದು ಸಂಭವ ಇದೆ. ದೇವರಂತೆ ನಿಮ್ಮ ಗುರುವನ್ನು ಪೂಜಿಸಿ, ಆದರೆ ಸ್ವಲ್ಪವೂ ವಿಮರ್ಶೆಮಾಡದೆ ಅವನು ಹೇಳಿದ ಮಾತನ್ನು ಕೇಳಬೇಡಿ. ನಿಮ್ಮ ಕೈಲಾದಷ್ಚು ಅವನನ್ನು ಪ್ರೀತಿಸಿ; ಆದರೆ ಸ್ವತಂತ್ರವಾಗಿ ಆಲೋಚಿಸಿ.


\section{ಶಿಷ್ಯನ ಮೇಲೆ ಸಹಾನುಭೂತಿ}

ಗುರು ಶಿಷ್ಯನ ಶೀಲದಮೇಲೆ ತನ್ನ ಶಕ್ತಿಯ ಪ್ರಭಾವವನ್ನೆಲ್ಲ ಬೀರಬೇಕು. ನಿಜವಾದ ಸಹಾನುಭೂತಿಯಿಲ್ಲದೆ ನಾವು ಎಂದಿಗೂ ಚೆನ್ನಾಗಿ ಕಲಿಸಲಾರೆವು. ಯಾರ ಶ್ರದ್ಧೆಯನ್ನೂ ಚಂಚಲಗೊಳಿಸಲು ಯತ್ನಿಸಬೇಡಿ. ಸಾಧ್ಯವಾದರೆ ಅವನಿಗೆ ಏನಾದರೂ ಉತ್ತಮವಾದುದನ್ನು ಕೊಡಿ. ಆದರೆ ಅವನಲ್ಲಿ ಆಗಲೇ ಇರುವುದನ್ನು ಹಾಳುಮಾಡಬೇಡಿ. ಒಂದು ಕ್ಷಣದಲ್ಲಿ ತನ್ನನ್ನು ಸಹಸ್ರಾರು ಜನರನ್ನಾಗಿ ಬದಲಾಯಿಸಲು ಸಾಧ್ಯವಾಗಿರುವವನೇ ನಿಜವಾದ ಗುರು. ತಕ್ಷಣವೇ ಶಿಷ್ಯನ ಮನಸ್ಸಿನ ಮಟ್ಟಕ್ಕೆ ಇಳಿದು ಬಂದು, ತನ್ನ ಆತ್ಮಶಕ್ತಿಯನ್ನು ಅವನಿಗೆ ದಾನಮಾಡಿ, ಶಿಷ್ಯನ ಮನಸ್ಸಿನ ಮೂಲಕ ನೋಡಿ ತಿಳಿದುಕೊಳ್ಳು ವಂತಹವನೇ ನಿಜವಾದ ಗುರು. ಅಂತಹ ಗುರುವೇ ನಿಜವಾಗಿಯೂ ಬೋಧಿಸ ಬಲ್ಲ, ಉಳಿದವರಲ್ಲ.

