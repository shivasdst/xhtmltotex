
\chapter{ಸ್ತ್ರೀ ಶಿಕ್ಷಣ}

\section{ಪ್ರಾಚೀನ ಭಾರತದಲ್ಲಿ}

ವೇದಾಂತವು ಎಲ್ಲಾ ಜೀವಿಗಳಲ್ಲಿಯೂ ಒಂದೇ ಆತ್ಮ ಇರುವುದೆಂದು ಸಾರಿದರೂ, ಈ ದೇಶದಲ್ಲಿ ಪುರುಷರಿಗೂ ಸ್ತ್ರೀಯರಿಗೂ ಇಷ್ಟು ಭೇದಭಾವ ಗಳನ್ನು ತಂದಿಟ್ಟಿರುವುದಕ್ಕೆ ಕಾರಣವೇನು ಎಂಬುದನ್ನು ತಿಳಿದುಕೊಳ್ಳುವುದಕ್ಕೆ ಕಷ್ಟವಾಗಿದೆ. ಸ್ಮೃತಿಗಳನ್ನು ಬರೆದು, ಕಠೋರ ನಿಯಮಗಳಿಂದ ಅವರನ್ನು ಬಂಧಿಸಿ, ಸ್ತ್ರೀಯನ್ನು ಕೇವಲ ಮಕ್ಕಳು ಹೆರುವ ಕಾರ್ಖಾನೆಗಳನ್ನಾಗಿ ಮಾಡಿರು ವರು. ವೇದಾಧ್ಯಯನಕ್ಕೆ, ಉಳಿದ ವರ್ಣದವರನ್ನು ಅನರ್ಹರನ್ನಾಗಿ ಮಾಡಿದ ಅವನತಿಯ ಕಾಲದಲ್ಲಿ ಸ್ತ್ರೀಯರ ಹಕ್ಕು ಬಾಧ್ಯತೆಗಳನ್ನು ಕೂಡಾ ಕಸಿದು ಕೊಂಡರು. ವೇದ ಮತ್ತು ಉಪನಿಷತ್ತಿನ ಕಾಲದಲ್ಲಿ ಗಾರ್ಗಿ, ಮೈತ್ರೇಯಿ ಯಂತಹ ಸ್ಮೃತಪುಣ್ಯರಾದ ನಾರೀರತ್ನಗಳು ಪುಷಿಗಳ ಸ್ಥಾನದಲ್ಲಿ ಇದ್ದರು. ವೇದಪಾರಂಗತರಿಂದ ಕೂಡಿದ ಒಂದು ಸಾವಿರ ಬ್ರಾಹ್ಮಣ ಸಭೆಯಲ್ಲಿ ಯಾಜ್ಞವಲ್ಕ್ಯರನ್ನು ಬ್ರಹ್ಮನ ವಿಚಾರವಾಗಿ ಗಾರ್ಗಿ ಕಡು ದಿಟ್ಟತನದಿಂದ ಪ್ರಶ್ನಿಸುವಳು.


\section{ನಿಜವಾದ ಶಕ್ತಿಪೂಜೆ}

ಜನಾಂಗಗಳೆಲ್ಲ ಸ್ತ್ರೀಯರಿಗೆ ಯೋಗ್ಯರೀತಿಯಲ್ಲಿ ಗೌರವ ತೋರಿದುದ ರಿಂದ ಕೀರ್ತಿಯನ್ನು ಗಳಿಸಿದವು. ಯಾವ ಜನಾಂಗ ಮತ್ತು ದೇಶ ನಾರಿಯನ್ನು ಗೌರವದೃಷ್ಟಿಯಿಂದ ನೋಡುವುದಿಲ್ಲವೊ, ಆ ದೇಶ ಶ್ರೇಷ್ಠವಾಗಿಲ್ಲ, ಮತ್ತು ಮುಂದೆ ಆಗುವಂತೆಯೂ ಇಲ್ಲ. ಯಾರು ದೇವರನ್ನು ವಿಶ್ವದಲ್ಲಿ ಸರ್ವವ್ಯಾಪಿ ಯಾದ ಶಕ್ತಿ ಎಂದು ತಿಳಿದು, ಆ ಮಹಾಶಕ್ತಿಯ ಆವಿರ್ಭಾವನೆಯನ್ನು ಸ್ತ್ರೀಯರಲ್ಲಿ ನೋಡುತ್ತಾರೆಯೋ ಅವರೇ ನಿಜವಾದ ಶಕ್ತಿಪೂಜಕರು. ಅಮೇ ರಿಕಾ ದೇಶದಲ್ಲಿ ಪುರುಷರು ಸ್ತ್ರೀಯರನ್ನು ಸಾಧ್ಯವಾದಷ್ಟು ಮಟ್ಟಿಗೆ ಚೆನ್ನಾಗಿ ಕಾಣುವರು. ಅದಕ್ಕೋಸ್ಕರವೆ ಅವರು ಅಷ್ಟು ಮುಂದುವರಿದಿರುವುದು, ಸ್ವತಂತ್ರವಾಗಿರುವುದು ಮತ್ತು ಚಟುವಟಿಕೆಯಿಂದ ಕೂಡಿರುವುದು. ನಮ್ಮ ಜನಾಂಗ ಏತಕ್ಕೆ ಇಷ್ಟು ಅಧೋಗತಿಗೆ ಇಳಿದಿದೆ ಎನ್ನುವುದಕ್ಕೆ ಮುಖ್ಯ ಕಾರಣವೆ ಇಂತಹ ಸಚೇತನವಾದ ಶಕ್ತಿಚಿಹ್ನೆಗೆ ಗೌರವ ತೋರದೆ ಇರುವುದು. “ಸ್ತ್ರೀಯರನ್ನು ಎಲ್ಲಿ ಗೌರವದಿಂದ ಕಾಣುತ್ತಾರೆಯೋ ಅಲ್ಲಿ ದೇವತೆಗಳು ಸುಪ್ರೀತರಾಗುವರು. ಯಾರು ಅವರನ್ನು ಗೌರವದಿಂದ ಕಾಣುವುದಿಲ್ಲವೋ, ಅವರು ಮಾಡುವ ಕರ್ಮವೆಲ್ಲ, ಪ್ರಯತ್ನವೆಲ್ಲ ನಿಷ್ಫಲವಾಗುವುದು.” ಎಲ್ಲಿ ಅವರು ದುಃಖಿಗಳಾಗಿರುವರೊ ಅಂತಹ ವಂಶಕ್ಕೆ ಮತ್ತು ದೇಶಕ್ಕೆ ಏಳಿಗೆಯ ನೆಚ್ಚಿಗೆ ಇಲ್ಲ.


\section{ವಿದ್ಯಾಭ್ಯಾಸ ಅವರ ಸಮಸ್ಯೆಗಳನ್ನು ಪರಿಹರಿಸುವುದು}

ಸ್ತ್ರೀಯರಿಗೆ ನಾನಾತರಹ ಜಟಿಲ ಸಮಸ್ಯೆಗಳಿವೆ. ಆದರೆ ವಿದ್ಯಾಭ್ಯಾಸ ಎಂಬ ಅಮೋಘವಾದ ಮಂತ್ರದಿಂದ ನಿವಾರಿಸಲಾಗದ ಸಮಸ್ಯೆ ಇಲ್ಲ. ನಮ್ಮ ಮನು ಹೇಳುವನು: “ಹೆಣ್ಣು ಮಕ್ಕಳನ್ನು ಗಂಡು ಮಕ್ಕಳಂತೆಯೇ ರಕ್ಷಿಸಿ ಅವರಿಗೆ ವಿದ್ಯಾಭ್ಯಾಸವನ್ನು ಕೊಡಬೇಕು. ಗಂಡು ಮಕ್ಕಳು ಮೂವತ್ತೆರಡು ವರುಷಗಳ ವರೆಗೆ ಬ್ರಹ್ಮಚರ್ಯವನ್ನು ಪಾಲಿಸಿ ನಂತರ ಮದುವೆಯಾಗುವಂತೆ ಹೆಣ್ಣು ಮಕ್ಕಳೂ ಕೂಡ ಬ್ರಹ್ಮಚಾರಿಣಿಯರಾಗಿದ್ದು ಅವರ ತಂದೆ ತಾಯಿಯರಿಂದ ವಿದ್ಯಾಭ್ಯಾಸವನ್ನು ಪಡೆಯಬೇಕು. ಆದರೆ ನಾವೇನು ಮಾಡುತ್ತಿರುವೆವು? ಅವರಿಗೆ ಯಾವಾಗಲೂ ಅಸಹಾಯತೆ ಮತ್ತುಇನ್ನೊಬ್ಬರನ್ನು ನೆಚ್ಚಿಕೊಂಡಿರು ವುದು–ಇವುಗಳನ್ನೇ ಕಲಿಸುತ್ತಿರುವೆವು. ಆದಕಾರಣ ಸ್ವಲ್ಪ ಅಪಾಯ ಅಥವಾ ಕಷ್ಟಪ್ರಾಪ್ತವಾದರೂ ಕಣ್ಣು ಕುರುಡಾಗುವ ತನಕ ಅಳುವುದಕ್ಕೆ ಮಾತ್ರ ಅವರು ಯೋಗ್ಯರಾಗಿರುವರು. ತಮ್ಮ ಕಷ್ಟಗಳನ್ನು ತಮ್ಮದೇ ರೀತಿಯಲ್ಲಿ ಬಗೆಹರಿ ಸುವ ಸ್ಧಾನದಲ್ಲಿ ಅವರನ್ನು ಬಿಡಬೇಕು. ಇದನ್ನು ಸಾಧಿಸುವುದಕ್ಕೆ ನಮ್ಮ ಭಾರತೀಯ ನಾರಿಯರು ಇತರ ದೇಶದ ನಾರಿಯರಷ್ಟೇ ಯೋಗ್ಯರು.


\section{ಧರ್ಮ ಅದರ ಕೇಂದ್ರವಾಗಬೇಕು}

ಸ್ತ್ರೀ ಶಿಕ್ಷಣ ಧರ್ಮವನ್ನು ಕೇಂದ್ರ ಮಾಡಿಕೊಂಡು ಹರಡಬೇಕು. ಉಳಿದ ಎಲ್ಲಾ ಶಿಕ್ಷಣವೂ ಧರ್ಮಕ್ಕೆ ಎರಡನೆಯದಾಗಿರಬೇಕು. ಧಾರ್ಮಿಕ ಶಿಸ್ತು, ಶೀಲಪೋಷಣೆ, ಬ್ರಹ್ಮಚರ್ಯನಿಯಮ ಇವನ್ನು ನೋಡಿಕೊಳ್ಳಬೇಕು. ನಮ್ಮ ಹಿಂದೂ ನಾರಿಯರು ಪಾತಿವ್ರತ್ಯವೆಂದರೆ ಏನೆಂಬುದನ್ನು ಸುಲಭವಾಗಿ ತಿಳಿದು ಕೊಳ್ಳುವರು. ಏಕೆಂದರೆ ಇದು ಅವರಿಗೆ ವಂಶಾನುಗತವಾಗಿ ಬಂದಿದ್ದು, ಮೊದಲನೆಯದಾಗಿ ಎಲ್ಲಕ್ಕಿಂತ ಹೆಚ್ಚಾಗಿ ಶೀಲವನ್ನು ಉದ್ದೀಪನಗೊಳಿಸ ಬೇಕು. ಅವರು ಇದರಿಂದ ಉತ್ತಮ ಚಾರಿತ್ರದವರಾಗುವರು. ಇಂತಹ ಶಕ್ತಿಯಿಂದ, ಅವರು ಜೀವನದ ಯಾವುದೇ ಕಾರ್ಯರಂಗದಲ್ಲಿ ಆಗಲಿ, ವಿವಾಹ ವಾಗಿರಲಿ, ಅಥವಾ ಏಕಾಕಿಯಾಗಿರಲಿ, ತಮ್ಮ ಪಾತಿವ್ರತ್ಯ ಧ್ಯೇಯದಿಂದ ಒಂದು ಅಂಗುಲವಾದರೂ ಹಿಂದೆ ಸರಿಯದೆ ಇರುವುದಕ್ಕೆ ತಮ್ಮ ಪ್ರಾಣ ವನ್ನಾದರೂ ಬಲಿಕೊಡಲು ಸ್ವಲ್ಪವೂ ಅಂಜುವುದಿಲ್ಲ.


\section{ಸೀತಾದೇವಿಯ ಆದರ್ಶ}

ಭಾರತೀಯ ನಾರಿಯರು ಸೀತೆಯ ಶೀಲದ ಮೇಲ್ಪಂಕ್ತಿಯಲ್ಲಿ ಬೆಳೆಯ ಬೇಕು. ಸೀತಾದೇವಿ ಅನುಪಮಳು. ನಿಜವಾದ ಭಾರತೀಯ ನಾರಿಯರ ಮೇಲ್ಪಂಕ್ತಿ ಅವಳು. ಹಿಂದೂ ಸ್ತ್ರೀರತ್ನಗಳ ಆದರ್ಶವೆಲ್ಲ ಸೀತಾದೇವಿ ಎಂಬ ಒಂದು ಜೀವದಿಂದ ಬೆಳೆದಿದೆ. ಸಾವಿರಾರು ವರುಷಗಳಿಂದಲೂ ಆರ್ಯಾವರ್ತ ದೆಲ್ಲೆಲ್ಲ ಪುರುಷರು ಸ್ತ್ರೀಯರು ಮತ್ತು ಮಕ್ಕಳುಗಳಿಂದ ಪೂಜಿಸಲ್ಪಡುತ್ತಿರು ವಳು. ಇಲ್ಲಿರುವಳು ಮಹಾಮಹಿಮಳಾದ ಸೀತೆ. ಪವಿತ್ರತೆಗಿಂತ ಪವಿತ್ರತಮ ಳಾಗಿ, ಸಹನಶೀಲಳಾಗಿ ಕಷ್ಟಗಳನ್ನೆಲ್ಲ ಅನುಭವಿಸುತ್ತ ಎಂದಿಗೂ ಈ ಮಹಾ ತಾಯಿ ಇಲ್ಲಿರುವಳು. ಬಂದ ಎಲ್ಲ ಕಷ್ಟವನ್ನು ಮರು ಮಾತನಾಡದೆ ಅನುಭವಿಸಿ, ಅನವರತವೂ ಪವಿತ್ರತಾ ಶಿರೋಮಣಿಯಾಗಿ, ಭಾರತೀಯರ ಆದರ್ಶವಾಗಿ, ನಮ್ಮ ಜನಾಂಗದ ಜೀವನದ ಅಂತರಾಳಕ್ಕೆ ಅವಳು ಪ್ರವೇಶಿ ಸಿರುವಳು. ನಮ್ಮ ಸ್ತ್ರೀಯರನ್ನು ಆಧನಿಕರಂತೆ ಮಾಡುವಾಗ ಸೀತಾದೇವಿಯ ದರ್ಶನವನ್ನು ತೊರೆಯಲು ಯತ್ನಿಸಿದರೆ, ಅದು ನಮಗೆ ಪ್ರತಿದಿನವೂ ತೋರು ವಂತೆ ನಿಷ್ಪಲವಾದಂತೆಯೆ.


\section{ವೈರಾಗ್ಯ ಶಿಕ್ಷಣ}

ಆಧುನಿಕ ಕಾಲದ ಆವಶ್ಯಕತೆಯನ್ನು ಪರೀಕ್ಷಿಸುತ್ತಿರುವಾಗ ಕೆಲವರಿಗೆ ವೈರಾಗ್ಯದ ಆದರ್ಶಗಳ ಶಿಕ್ಷಣವನ್ನು ಅತ್ಯಾವಶ್ಯಕವಾಗಿ ಕೊಡಬೇಕೆಂದು ತೋರುತ್ತದೆ. ವಂಶಪಾರಂಪರ್ಯವಾಗಿ ತಮ್ಮ ರಕ್ತಗತವಾಗಿ ಬಂದ ಪಾತಿವ್ರತ್ಯ ಶಕ್ತಿಯ ಮೇಲೆ ನಿಂತು ಆಜನ್ಮ ಬ್ರಹ್ಮಚಾರಿಣಿ ವ್ರತವನ್ನು ಅವರು ಸ್ವೀಕರಿಸ ಬಹುದು. ನಮ್ಮ ಮಾತೃಭೂಮಿಯ ಹಿತರಕ್ಷಣೆಗಾಗಿ ಆಕೆಯ ಕೆಲವು ಮಕ್ಕಳು ಪವಿತ್ರಾತ್ಮರಾದ ಬ್ರಹ್ಮಚಾರಿ ಮತ್ತು ಬ್ರಹ್ಮಚಾರಿಣಿಗಳಾಗಬೇಕಾಗಿದೆ. ಸ್ತ್ರೀಯರಲ್ಲಿ ಒಬ್ಬಳು ಬ್ರಹ್ಮಜ್ಞಾನವನ್ನು ಪಡೆದರೂ ಆಕೆಯ ಜೀವನದ ಮಹಾತ್ಮೆಯಿಂದ ಸಹಸ್ರಾರು ಮಂದಿ ಸ್ತ್ರೀಯರು ಸ್ಪೂರ್ತಿ ಪಡೆದು, ಸತ್ಯ ಸಾಧನೆಗಾಗಿ ಜಾಗ್ರತರಾಗುವರು. ಇದರಿಂದ ದೇಶ ಮತ್ತು ಜನಾಂಗ ಉತ್ತಮ ವಾಗುವುದು.


\section{ಲೌಕಿಕ ಶಿಕ್ಷಣ}

ವಿದ್ಯೆ ಮತ್ತು ಶೀಲಸಂಪನ್ನರಾದ ಬ್ರಹ್ಮಚಾರಿಣಿಯರು ಶಿಕ್ಷಣದ ಕೆಲಸ ವನ್ನು ವಹಿಸಿಕೊಳ್ಳಬೇಕು. ಊರುಗಳಲ್ಲಿ ಮತ್ತು ಕೇಂದ್ರಗಳಲ್ಲಿ ವಿದ್ಯಾಭ್ಯಾಸ ಹರಡುವುದಕ್ಕೆ ಅವರು ಪ್ರಯತ್ನಿಸಬೇಕು. ಇಂತಹ ಶೀಲಸಂಪನ್ನರಾದ ಬೋಧಕರಿಂದ ದೇಶದಲ್ಲಿ ನಿಜವಾಗಿಯೂ ಸ್ತ್ರೀಯರ ವಿದ್ಯಾಭ್ಯಾಸ ಹರಡು ವುದು. ಇತಿಹಾಸ ಮತ್ತು ಪುರಾಣಗಳು, ಮನೆ ಕೆಲಸ ಮತ್ತು ಇತರ ಕಲೆಗಳು, ಸಂಸಾರಧರ್ಮಗಳು ಮತ್ತು ಶೀಲಸಂಪನ್ನರಾಗಿ ಮಾಡುವ ನಿಯಮಗಳು– ಇವುಗಳನ್ನು ಬೋಧಿಸಬೇಕು. ಹೊಲಿಯುವುದು, ಅಡಿಗೆಮಾಡುವುದು, ಮನೆ ಕೆಲಸದ ರೀತಿನೀತಿಗಳು, ಶಿಶುರಕ್ಷಣೆ ಮಂತಾದುವುಗಳನ್ನು ಕಲಿಸಬೇಕು. ಜಪ, ಪೂಜೆ, ಧ್ಯಾನ ಇವು ಶಿಕ್ಷಣದ ಅತ್ಯಗತ್ಯವಾದ ಅಂಗವಾಗಬೇಕು.


\section{ಆತ್ಮರಕ್ಷಣೆ}

ಇವುಗಳೊಂದಿಗೆ ಶೌರ್ಯ ಮತ್ತು ಪರಾಕ್ರಮಗಳನ್ನು ಗಳಿಸಬೇಕು. ಸದ್ಯಕ್ಕೆ ಆತ್ಮರಕ್ಷಣೆ ಮಾಡಿಕೊಳ್ಳುವುದನ್ನು ಕಲಿತುಕೊಳ್ಳುವುದು ಅವಶ್ಯವಾಗಿದೆ. ಝಾಂಸಿಯ ಲಕ್ಷೀಬಾಯಿ ಎಷ್ಟು ಪ್ರಖ್ಯಾತಳಾಗಿದ್ದಳು! ಭರತಖಂಡದ ಹಿತರಕ್ಷಣೆಗೋಸುಗವಾಗಿ ಗತಕಾಲದ ಸಂಘಮಿತ್ರ, ಲೀಲಾ, ಅಹಲ್ಯಬಾಯಿ ಮತ್ತು ಮೀರಾಬಾಯಿಯಂತಹ ಪರಂಪರೆ ನಿರಾಂತಕವಾಗಿ ಹರಿದು ಬರು ವಂತೆ ಮಾಡಬಲ್ಲ, ಶುದ್ಧಾತ್ಮರಾಗಿ ಧೀರಾತ್ಮರಾಗಿ ಪರಮಾತ್ಮಸ್ಪರ್ಶದಿಂದ ಬರುವ ಶಕ್ತಿಯಿಂದ ಕೂಡಿದವರಾಗಿ ವೀರ ಪುತ್ರರಿಗೆ ಜನ್ಮವನ್ನು ಕೊಡಲು ಯೋಗ್ಯರಾದಂತಹ ಕೆಚ್ಚೆದೆಯ ಧೀರ ನಾರೀರತ್ನಗಳನ್ನು ನಾವು ತರಬೇತು ಮಾಡಬೇಕು. ಮುಂದೆ ಅವರು ಆದರ್ಶ ಗೃಹಿಣಿಯರಾಗಿ ಬೆಳೆಯುವಂತೆ ನೋಡಿಕೊಳ್ಳಬೇಕು. ಅಂತಹ ಮಾತೆಯರ ಮಕ್ಕಳು ಸಮಾಜದಲ್ಲಿ ಗಣ್ಯ ರಾಗುವಂತಹ ಶೀಲದಲ್ಲಿ ಇನ್ನೂ ಉತ್ತಮರಾಗುವರು. ಸುಸಂಸ್ಕೃತರಾದ ಭಕ್ತ ಪರಾಯಣೆಯರಾದ ಮಾತೆಯರು ಇರುವ ಮನೆಯಲ್ಲಿ ಮಹಾಪುರುಷರು ಜನಿಸುವರು.

ಸ್ತ್ರೀಯರನ್ನು ಉತ್ತಮ ಸ್ಥಿತಿಗೆ ತಂದರೆ, ಅವರ ಮಕ್ಕಳು ತಮ್ಮ ಸತ್ಕರ್ಮ ಗಳಿಂದ ದೇಶದ ಕೀರ್ತಿಯನ್ನು ಬೆಳಗುತ್ತಾರೆ. ಆಗ ದೇಶದಲ್ಲಿ ಸಂಸ್ಕೃತಿ ಜ್ಞಾನ ಶಕ್ತಿ ಭಕ್ತಿಗಳು ಜಾಗೃತವಾಗುತ್ತವೆ.

