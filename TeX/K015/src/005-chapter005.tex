
\chapter{ಶೀಲಪೋಷಣೆಗೆ ವಿದ್ಯಾಭ್ಯಾಸ}

\section{ಆಲೋಚನಾಶಕ್ತಿಯ ಪ್ರಾಧ್ಯಾನ್ಯ}

ಯಾರ ಶೀಲವಾಗಲಿ ಅದು ಅವನಲ್ಲಿ ಇರುವ ಸ್ವಭಾವಗಳ ಮೊತ್ತ, ಅವನ ಮಾನಸಿಕ ಇಚ್ಛೆಗಳ ಮೊತ್ತ. ಅವನ ಆತ್ಮದ ಮುಂದೆ ಸುಖದುಃಖಗಳು ಹೋದಂತೆಲ್ಲ ಅದರ ಮೇಲೆ ಅನೇಕ ದೃಶ್ಯಗಳನ್ನು ಬೀರುತ್ತವೆ. ಈ ಶೇಖರಿ ಸಿದ ದೃಶ್ಯಗಳ ಒಟ್ಟು ಪರಿಣಾಮವೇ ಮನುಷ್ಯನ ಶೀಲವೆನ್ನುವುದು. ನಮ್ಮ ಆಲೋಚನೆಗಳು ಪ್ರೇರೇಪಿಸಿದಂತೆ ನಾವು ಇರುವೆವು. ಪ್ರತಿಯೊಂದು ಆಲೋ ಚನೆಯೂ, ಕಬ್ಬಿಣದ ಮುದ್ದೆಯಂತಿರುವ ನಮ್ಮ ದೇಹದ ಮೇಲೆ ಕೊಡುವ ಸುತ್ತಿಗೆಯ ಪೆಟ್ಟಿನಂತೆ. ಇದರಿಂದ ನಾವು ಏನು ಆಗಬೇಕೆಂದು ಬಯಸು ವೆವೋ ಅದನ್ನು ತಯಾರುಮಾಡುವುದು. ಮಾತುಗಳು ಗೌಣ. ಆಲೋಚನೆ ಸಜೀವವಾಗಿವೆ. ಅವು ಬಹಳ ದೂರ ಚಲಿಸುತ್ತವೆ. ಆದಕಾರಣವೇ ನೀವು ಏನನ್ನು ಆಲೋಚನೆ ಮಾಡುತ್ತೀರಿ ಎಂಬುದನ್ನು ಗಮನದಲ್ಲಿ ಇಡಿ.


\section{ಸುಖದುಃಖಗಳ ಪಾತ್ರ}

ನಮ್ಮ ಶೀಲವನ್ನು ತಿದ್ದುವುದರಲ್ಲಿ ಸುಖದುಃಖಗಳೆರಡಕ್ಕೂ ಸಮಭಾಗ ವಿದೆ. ಕೆಲವುವೇಳೆ ದುಃಖ ಸುಖಕ್ಕಿಂತ ದೊಡ್ಡ ಗುರು. ಪ್ರಪಂಚದಲ್ಲಿ ಜನಿಸಿದ ಅನೇಕ ಮಹನೀಯರ ಜೀವನವನ್ನು ಓದಿದಾಗ, ಅವರಲ್ಲಿ ಬಹುಪಾಲು ಜನರಿಗೆ ದುಃಖವೇ ಸುಖಕ್ಕಿಂತ ಹೆಚ್ಚು ಕಲಿಸಿತು; ಐಶ್ವರ್ಯಕ್ಕಿಂತ ಬಡತನ ಅವರಿಗೆ ಹೆಚ್ಚು ಕಲಿಸಿತು–ಎಂದು ಧೈರ್ಯವಾಗಿ ಹೇಳಬಹುದು. ಸ್ತುತಿಗಿಂತ ನಿಂದೆ ಅವರ ಅಂತರಂಗದಲ್ಲಿ ಉರಿಯುತ್ತಿದ್ದ ಜ್ಯೋತಿಯನ್ನು ಮೇಲೆಬ್ಬಿ ಸಿತು. ಭೋಗದ ತೊಡೆಯಮೇಲೆ ಬೆಳೆದು, ಗುಲಾಬಿಯ ಸುಪ್ಪತ್ತಿಗೆಯ ಮೇಲೆ ಮಲಗಿ ಒಂದು ಹನಿ ಕಣ್ಣೀರನ್ನೂ ಸುರಿಸದೆ ಯಾರು ಮಹಾತ್ಮರಾಗಿರು ವರು? ಹೃದಯದಲ್ಲಿ ಮಮತೆ ಉದಿಸಿದಾಗ, ಸುತ್ತಲೂ ದುಃಖದ ಗಾಳಿ ಬೀಸುತ್ತಿರುವಾಗ, ಇನ್ನೇನು ಮುಂದೆ ಬೆಳಕು ಕಾಣುವಂತೆ ಇಲ್ಲ ಎಂದು ನಿರಾಶರಾದಾಗ, ಧೈರ್ಯ ಮತ್ತು ನೆಚ್ಚಿಗೆ ಸಂಪೂರ್ಣ ತೊರೆದಂತೆ ಕಂಡಾಗ –ಆಗಲೇ, ಇಂತಹ ಆಧ್ಯಾತ್ಮಿಕ ಪ್ರಚಂಡ ಬಿರುಗಾಳಿಯ ಮಧ್ಯದಿಂದಲೇ ಅಂತರ್ಜ್ಯೋತಿ ಹೊಳೆಯುವುದು.


\section{ಕರ್ಮಫಲ}

ಚಿತ್ತವನ್ನು ಒಂದು ಸರೋವರಕ್ಕೆ ಹೋಲಿಸಬಹುದು. ಅಲ್ಲಿ ಏಳುವ ಎಷ್ಟು ಚಿಕ್ಕ ಅಲೆಯಾಗಲಿ, ಅದು ಶಾಂತವಾದಾಗ ಸಂಪೂರ್ಣ ನಾಶವಾಗುವುದಿಲ್ಲ. ಅಲ್ಲಿ ತನ್ನ ಗುರುತನ್ನು ಬಿಡುವುದು. ಮುಂದೆ ಇದೇ ಮತ್ತೊಮ್ಮೆ ಹಿಂದಿನಂತೆ ವ್ಯಕ್ತವಾಗುವ ಸಂಭವವಿರುವುದು. ನಾವು ಮಾಡುವ ಪ್ರತಿಯೊಂದು ಕೆಲಸವೂ, ನಮ್ಮ ದೇಹದ ಪ್ರತಿಯೊಂದು ಚಲನೆಯೂ ನಮ್ಮ ಚಿತ್ತಭಿತ್ತಿಯ ಮೇಲೆ ತನ್ನ ಸಂಸ್ಕಾರವನ್ನು ಬಿಡುವುದು. ಇಂತಹ ಸಂಸ್ಕಾರಗಳು ಚಿತ್ತ ಸರೋವರದ ಮೇಲೆ ಕಾಣಿಸದಂತೆ ಇದ್ದರೂ, ಸರೋವರದ ಕೆಳಗೆ ನಮ ಗರಿವಿಲ್ಲದಂತೆ ಕೆಲಸ ಮಾಡುತ್ತಿರುತ್ತವೆ. ನಾವು ಪ್ರತಿ ಕ್ಷಣವೂ ಏನಾಗಿರು ವೆವೊ ಅದು ಈ ಸಂಸ್ಕಾರಗಳ ಒಟ್ಟು ಮೊತ್ತದಿಂದ ನಿರ್ಧರಿಸಲ್ಪಡುತ್ತದೆ. ಪ್ರತಿಯೊಬ್ಬ ವ್ಯಕ್ತಿಯ ಶೀಲವೂ ಈ ಸಂಸ್ಕಾರಗಳ ಒಟ್ಟು ಮೊತ್ತದಿಂದ ನಿರ್ಧರಿಸಲ್ಪಡುತ್ತದೆ. ಒಳ್ಳೆಯ ಸಂಸ್ಕಾರಗಳಿದ್ದರೆ ಶೀಲ ಒಳ್ಳೆಯದಾಗು ವುದು. ಕೆಟ್ಟ ಸಂಸ್ಕಾರಗಳಿದ್ದರೆ ಶೀಲ ಕೆಡುವುದು. ಒಬ್ಬ ಮನುಷ್ಯ ಸದಾ ಕೆಟ್ಟ ಮಾತನ್ನು ಕೇಳುತ್ತಿದ್ದರೆ, ಕೆಟ್ಟ ಆಲೋಚನೆಯನ್ನು ಮಾಡುತ್ತಿದ್ದರೆ ಅವನ ಮನಸ್ಸು ಕೆಟ್ಟ ಸಂಸ್ಕಾರಗಳಿಂದ ತುಂಬುವುದು. ಈ ಸಂಸ್ಕಾರಗಳು ಅವನ ಅರಿವೆ ಇಲ್ಲದೆ ಅವನು ಮಾಡುವ ಕೆಲಸ ಮತ್ತು ಆಲೋಚನೆಗಳ ಮೇಲೆ ತಮ್ಮ ಪ್ರಭಾವವನ್ನು ಬೀರುವುವು. ನಿಜವಾಗಿಯೂ ಇಂತಹ ಹೀನಸಂಸ್ಕಾರ ಗಳು ಯಾವಾಗಲೂ ಕೆಲಸ ಮಾಡುತ್ತಾ ಇರುತ್ತವೆ. ಅವನಲ್ಲಿ ಇರುವ ಇಂತಹ ಸಂಸ್ಕಾರಗಳ ಮೊತ್ತವೇ ಕೆಟ್ಟ ಕರ್ಮಗಳನ್ನು ಮಾಡುವುದಕ್ಕೆ ಬಲವಾದ ಪ್ರೇರೇಪಣೆಯನ್ನು ಉಂಟುಮಾಡುತ್ತದೆ. ತನ್ನ ಸಂಸ್ಕಾರಗಳ ಕೈಯಲ್ಲಿ ಅವನು ಒಂದು ಯಂತ್ರವಾಗುತ್ತಾನೆ.


\section{ಶೀಲ ನಿರ್ಮಾಣ}

ಇದರಂತೆಯೆ ಒಬ್ಬ ಮನುಷ್ಯನು ಒಳ್ಳೆಯ ಆಲೋಚನೆಗಳನ್ನು ಮಾಡಿ, ಒಳ್ಳೆಯ ಕೆಲಸಗಳನ್ನು ಮಾಡಿದರೆ, ಇಂತಹ ಸಂಸ್ಕಾರಗಳ ಒಟ್ಟು ಪರಿಣಾಮ ಉತ್ತಮವಾಗಿ ಅವುಗಳು ಅವನ ಇಚ್ಛೆಯಿಲ್ಲದೆ ಒಳ್ಳೆಯದನ್ನು ಮಾಡುವಂತೆ ಪ್ರೇರೇಪಿಸುತ್ತವೆ. ಒಬ್ಬನು ಬೇಕಾದಷ್ಟು ಒಳ್ಳೆಯ ಕೆಲಸಗಳನ್ನು ಮಾಡಿದರೆ, ಒಳ್ಳೆಯದನ್ನು ಚಿಂತಿಸಿದರೆ ತಡೆಯಲಾಗದಂತಹ ಸತ್ಕರ್ಮ ಪ್ರವೃತ್ತಿ ಅವನ ಲ್ಲುಂಟಾಗುತ್ತದೆ. ಅವನು ತಪ್ಪು ಮಾಡಬೇಕೆಂದು ಬಯಸಿದರೂ, ಅವನ ಮನಸ್ಸು ಅವಕಾಶ ಕೊಡುವುದಿಲ್ಲ. ಅವನು ಸಂಪೂರ್ಣ ಒಳ್ಳೆಯ ಸಂಸ್ಕಾರ ಗಳ ವಶದಲ್ಲಿರುವನು. ಹೀಗಿದ್ದರೆ ಒಬ್ಬನ ಒಳ್ಳೆಯ ಶೀಲ ಸ್ಥಿರವಾಗಿದೆ ಎಂದು ಅರ್ಥ. ಒಬ್ಬನ ಶೀಲವನ್ನು ನೀವು ನಿಜವಾಗಿಯೂ ಪರೀಕ್ಷಿಸಬೇಕೆಂದು ಇದ್ದರೆ ಅವನು ಮಾಡುವ ಅತ್ಯಂತ ಸಾಧಾರಣ ಕೆಲಸವನ್ನು ನೋಡಿ. ಒಬ್ಬ ಮಹಾಪುರುಷನ ನಿಜವಾದ ಶೀಲವನ್ನು ತೋರುವ ಸನ್ನಿವೇಶಗಳು ಅವು. ದೊಡ್ಡ ಸನ್ನಿವೇಶಗಳು ಎಂತಹವನನ್ನಾದರೂ ಸ್ವಲ್ಪ ದೊಡ್ಡ ಸ್ಥಿತಿಗೆ ಏರಿಸು ತ್ತವೆ. ಯಾವನು ಎಲ್ಲಿರಲಿ, ಎಲ್ಲಾ ಕಡೆಯಲ್ಲಿಯೂ ಒಂದೇ ಸಮನಾದ ಗುಣವುಳ್ಳವನೋ ಅವನೆ ನಿಜವಾದ ಮಹಾಪುರುಷ.


\section{ಒಳ್ಳೆಯ ಮತ್ತು ಕೆಟ್ಟ ಅಭ್ಯಾಸ}

ಇಂತಹ ಅನೇಕ ಸಂಸ್ಕಾರಗಳು ಮನಸ್ಸಿನಲ್ಲಿದ್ದರೆ ಅವು ಕಲೆತು ಅಭ್ಯಾಸ ವಾಗುತ್ತವೆ. “ಅಭ್ಯಾಸವೇ ಮಾನವನ ಎರಡನೇ ಸ್ವಭಾವ” ಎನ್ನುತ್ತಾರೆ. ಇದೇ ಅವನ ಮೊದಲ ಸ್ವಭಾವ, ಅಲ್ಲದೆ ಅವನ ಒಟ್ಟುಸ್ವಭಾವ ಕೂಡ. ಈಗ ನಾವು ಇರುವುದೆಲ್ಲ ಅಭ್ಯಾಸದ ಪರಿಣಾಮ. ಇದರಿಂದ ನಮಗೆ ಸಮಾಧಾನವುಂಟಾ ಗುತ್ತದೆ. ಏಕೆಂದರೆ ಇದು ಕೇವಲ ಅಭ್ಯಾಸವಾದರೆ, ಯಾವುದೇ ಸಮಯದಲ್ಲಿ ಅದನ್ನು ಉಂಟುಮಾಡಬಹುದು, ಅಥವಾ ಹೋಗಲಾಡಿಸಲೂ ಬಹುದು. ಕೆಟ್ಟ ಅಭ್ಯಾಸವನ್ನು ಅಡಗಿಸಬೇಕಾದರೆ ಅದಕ್ಕೆ ವಿರೋಧವಾದ ಒಳ್ಳೆಯ ಅಭ್ಯಾಸ ವನ್ನು ರೂಢಿಸುವುದೊಂದೇ ದಾರಿ. ಎಲ್ಲಾ ಕೆಟ್ಟ ಅಭ್ಯಾಸವನ್ನು ಒಳ್ಳೆಯ ಅಭ್ಯಾಸದಿಂದ ನಿಗ್ರಹಿಸಬಹುದು. ನಿರಂತರವೂ ಒಳ್ಳೆಯ ಕೆಲಸಗಳನ್ನು ಮಾಡಿ, ಒಳ್ಳೆಯ ಆಲೋಚನೆಗಳನ್ನು ಮಾಡಿ. ಹೀನ ಸಂಸ್ಕಾರಗಳನ್ನು ಅಡಗಿ ಸುವುದಕ್ಕೆ ಇದೊಂದೇ ಮಾರ್ಗ. ಯಾರನ್ನು ಕೂಡ ಗತಿಕೆಟ್ಟವರು ಎನ್ನಬೇಡಿ. ಏಕೆಂದರೆ ಅವರು ಒಂದು ಶೀಲಕ್ಕೆ ಅಂದರೆ ನಡತೆಯ ಒಂದು ಗುಂಪಿಗೆ ಪ್ರತಿನಿಧಿಯಾಗಿರುವರು. ಇವುಗಳನ್ನು ನಾವು ಉತ್ತಮವಾದ ನಡತೆಗಳಿಂದ ನಿಗ್ರಹಿಸಬಹುದು. ಪುನರಾವರ್ತನೆಯಾದ ಅಭ್ಯಾಸವೇ ನಮ್ಮ ಶೀಲವನ್ನು ತಿದ್ದುವುದು.


\section{ನಮ್ಮ ಅದೃಷ್ಟಕ್ಕೆ ನಾವೇ ಕಾರಣ}

ಎಲ್ಲಾ ತೋರಿಕೆಯ ಪಾಪಕ್ಕೂ ಕಾರಣ ನಮ್ಮಲ್ಲಿರುವುದು, ಯಾವ ಹೊರಗಿನ ದೇವತೆಯನ್ನೂ ದೂರಬೇಡಿ. ನೆಚ್ಚುಗೆಡಬೇಕಾಗಿಲ್ಲ, ನಿರಾಶರಾಗ ಬೇಕಾಗಿಲ್ಲ ಅಥವಾ ಯಾರಾದರೂ ಬಂದು ಸಹಾಯಮಾಡುವ ತನಕ, ನಾವು ತಪ್ಪಿಸಿಕೊಂಡು ಹೋಗಲಾಗದ ಒಂದು ಸ್ಥಳದಲ್ಲಿ ಇರುವೆವು ಎಂದು ತಿಳಿಯಬೇಕಾಗಿಲ್ಲ. ನಾವು ಒಂದು ರೇಷ್ಮೆಯ ಹುಳುವಿನಂತೆ; ನಾವೇ ನಮ್ಮಿಂದ ನೂಲನ್ನು ಮಾಡಿ ಗೂಡನ್ನು ಹೆಣೆದುಕೊಳ್ಳುವೆವು. ಕೆಲವು ಕಾಲದ ಮೇಲೆ ಅದರಲ್ಲಿ ಬಂದಿಗಳಾಗುವೆವು. ಈ ಕರ್ಮಬಲೆಯನ್ನು ನಮ್ಮ ಸುತ್ತಲೂ ನಾವೇ ಹೆಣೆದುಕೊಂಡಿರುವೆವು. ಅಜ್ಞಾನದಿಂದ ನಾವು ಬದ್ಧರೆಂದು ತಿಳಿದು ಸಹಾಯಕ್ಕಾಗಿ ಗೋಳಾಡುವೆವು. ಆದರೆ ಸಹಾಯ ಹೊರಗಿನಿಂದ ಬರುವು ದಿಲ್ಲ. ಅದು ನಮ್ಮಿಂದಲೇ ಬರುವುದು. ಪ್ರಪಂಚದಲ್ಲಿರುವ ದೇವರಿಗೆಲ್ಲಾ ಪ್ರಾರ್ಥಿಸಿ, ನಾನೂ ಅನೇಕ ವರ್ಷ ಪ್ರಾರ್ಥಿಸಿದೆ. ಕೊನೆಗೆ ನನಗೆ ಸಹಾಯ ಸಿಕ್ಕಿತು. ಆದರೆ ಸಹಾಯ ಒಳಗಿನಿಂದ ಬಂದಿತು. ನಾನು ಏನನ್ನು ತಪ್ಪಾಗಿ ತಿಳಿದುಕೊಂಡಿದ್ದೆನೋ ಅದನ್ನು ತಿದ್ದಬೇಕಾಯಿತು. ನನ್ನ ಸುತ್ತಲೂ ಬೀಸಿ ಕೊಂಡ ಬಲೆಯನ್ನು ನಾನೇ ಕತ್ತರಿಸಬೇಕಾಯಿತು. ನನ್ನ ಜೀವನದಲ್ಲಿ ನಾನು ಎಷ್ಟೋ ತಪ್ಪುಗಳನ್ನು ಮಾಡಿರುವೆನು. ಆದರೆ ಅದನ್ನು ಗಮನಿಸಿ. ಆ ತಪ್ಪುಗಳಿಲ್ಲದೆ ಇದ್ದಿದ್ದರೆ ನಾನು ಇಂದು ಏನಾಗಿರುವನೋ ಅದು ಆಗುತ್ತಿರ ಲಿಲ್ಲ. ಅಂದರೆ ನೀವು ಮನೆಗೆ ಹೋಗಿ ಬೇಕೆಂತಲೇ ತಪ್ಪನ್ನು ಮಾಡಿ ಎಂದು ನಾನು ನಿಮಗೆ ಹೇಳುವುದಿಲ್ಲ. ಈ ರೀತಿಯಲ್ಲಿ ನನ್ನನ್ನು ತಪ್ಪು ತಿಳಿದುಕೊಳ್ಳ ಬೇಡಿ. ಆದರೆ ಹಿಂದೆ ಮಾಡಿದ ತಪ್ಪಿಗಾಗಿ ಗೋಳಾಡಬೇಕಾಗಿಲ್ಲ.


\section{ಅಜ್ಞಾನದಿಂದ ತಪ್ಪು}

ನಾವು ನಿರ್ಬಲರಾಗಿರುವುದರಿಂದ ತಪ್ಪು ಮಾಡುತ್ತೇವೆ. ನಾವು ಅಜ್ಞಾನಿ ಗಳಾದ ಕಾರಣ ನಿರ್ಬಲರಾಗಿರುವೆವು. ಯಾರು ನಮ್ಮನ್ನು ಅಜ್ಞಾನಿಗಳನ್ನಾಗಿ ಮಾಡುವರು? ನಾವೇ ಕೈಗಳನ್ನು ಕಣ್ಣಮೇಲೆ ಇಟ್ಟುಕೊಂಡು ಕತ್ತಲೆ ಎಂದು ಅಳುತ್ತೇವೆ. ಕೈಗಳನ್ನು ತೆಗೆಯಿರಿ. ಬೆಳಕು ಅಲ್ಲೇ ಇರುವುದು. ಆತ್ಮನ ಸ್ವಯಂಪ್ರಕಾಶಮಾನವಾದ ಜ್ಯೋತಿ ಯಾವಾಗಲೂ ನಮ್ಮ ಪಾಲಿಗೆ ಇರು ವುದು. ಆಧುನಿಕ ವಿಜ್ಞಾನಿಗಳು ಹೇಳುತ್ತಿರುವುದು ಕೇಳುವುದಿಲ್ಲವೆ? ವಿಕಾಸಕ್ಕೆ ಕಾರಣವೇನು? ಆಸೆ. ಪ್ರಾಣಿಯು ಏನನ್ನೋ ಮಾಡಬೇಕೆಂದು ಬಯಸುವುದು. ಆದರೆ ವಾತಾವರಣ ಅದಕ್ಕೆ ಸರಿಯಾಗಿಲ್ಲ. ಆದಕಾರಣವೇ ಒಂದು ಹೊಸ ದೇಹವನ್ನು ಉಂಟುಮಾಡಿಕೊಳ್ಳುತ್ತದೆ. ಹೊಸ ದೇಹವನ್ನು ಯಾರು ಮಾಡಿ ಕೊಳ್ಳುವರು? ಪ್ರಾಣಿಯೆ, ಅದರ ಇಚ್ಛೆಯೆ. ನಿಮ್ಮ ಇಚ್ಛಾಶಕ್ತಿಯನ್ನು ಉಪಯೋಗಿಸಿ. ಅದು ನಿಮ್ಮನ್ನು ಮೇಲಕ್ಕೆ ಒಯ್ಯುತ್ತದೆ. ಇಚ್ಛೆಯು ಸರ್ವ ಶಕ್ತವಾದುದು. ಅದು ಸರ್ವಶಕ್ತಿಯಿಂದ ಕೂಡಿದ್ದರೆ ನಾವು ಏತಕ್ಕೆ ಏನನ್ನು ಬೇಕಾದರೂ ಮಾಡಲಾರೆವು ಎಂದು ನೀವು ಕೇಳಬಹುದು. ಆದರೆ ನೀವು ಅಲ್ಪಾತ್ಮನನ್ನು ಮಾತ್ರ ಆಲೋಚಿಸುತ್ತಿರುವಿರಿ. ಜೀವಾಣುವಿನಿಂದ ಹಿಡಿದು ಮನುಷ್ಯತ್ವದವರೆಗೆ ನಿಮ್ಮ ಸ್ಥಿತಿಯನ್ನು ಅಲ್ಲಗಳೆಯಬಲ್ಲಿರಾ? ಇಷ್ಟು ಮೇಲಕ್ಕೆ ಬರುವತನಕ ಸಹಾಯ ಮಾಡಿದುದು ಮತ್ತಷ್ಟು ಮೇಲಕ್ಕೆ ಹೋಗು ವಂತೆಯೂ ಸಹಾಯ ಮಾಡಬಲ್ಲದು. ನಿಮಗೆ ಬೇಕಾಗಿರುವುದು ಶೀಲ; ಇಚ್ಛೆಯನ್ನು ಬಲಪಡಿಸುವುದು.


\section{ನಿಮ್ಮ ಶೀಲವನ್ನು ರೂಢಿಸಿ}

ನೀವು ಮನೆಗೆ ಹೋಗಿ ಹರಕುಬಟ್ಟೆ ಉಟ್ಟು ಬೂದಿ ಬಳಿದುಕೊಂಡು, ಹಿಂದೆ ಎಂದೋ ಕೆಲವು ವೇಳೆ ತಪ್ಪು ಮಾಡಿದ ಮಾತ್ರಕ್ಕೆ ಬದುಕಿರುವತನಕ ಗೋಳಿಟ್ಟರೆ, ಇದು ನಿಮಗೆ ಸಹಾಯ ಮಾಡುವುದಿಲ್ಲ; ಆದರೆ ಮತ್ತಷ್ಟು ದುರ್ಬಲರನ್ನಾಗಿ ಮಾಡುವುದು. ಸಾವಿರಾರು ವರುಷಗಳಿಂದ ಕೋಣೆಯಲ್ಲಿ ಕತ್ತಲೆ ಇದ್ದರೆ, ನೀವು ಬಂದು ಗೋಳಾಡಿದರೆ ಕತ್ತಲೆ ಮಾಯಾವಾಗುವುದೇ? ಒಂದು ಬೆಂಕಿಯ ಕಡ್ಡಿಯನ್ನು ಗೀರಿ, ತಕ್ಷಣವೇ ಬೆಳಕು ಬರುವುದು. ಬದುಕಿರುವ ತನಕ ಅಯ್ಯೋ! ನಾನು ಪಾಪ ಮಾಡಿರುವೆನು, ಅನೇಕ ತಪ್ಪು ಗಳನ್ನು ಮಾಡಿರುವೆನು ಎಂದು ಯೋಚಿಸುತ್ತಿದ್ದರೆ ನಿಮಗೆ ಇದರಿಂದ ಯಾವ ಪ್ರಯೋಜನ? ಅದನ್ನು ಹೇಳುವುದಕ್ಕೆ ಯಾವ ಭೂತವೂ ಬೇಕಾಗಿಲ್ಲ. ಜ್ಞಾನವನ್ನು ತನ್ನಿ. ತಕ್ಷಣವೇ ಪಾಪ ಪಲಾಯನವಾಗುವುದು. ಮೊದಲು ನಿಮ್ಮ ಶೀಲವನ್ನು ಪರಿಪುಷ್ಠಿಗೊಳಿಸಿ. ಸ್ವಯಂಪ್ರಕಾಶಮಾನವಾದ ನಿಮ್ಮ ಆತ್ಮದ ನೈಜಸ್ವಭಾವವನ್ನು ವ್ಯಕ್ತಗೊಳಿಸಿ; ಜೊತೆಗೆ ನಮ್ಮ ಕಣ್ಣಿಗೆ ಬೀಳುವವರೆಲ್ಲ ರಲ್ಲಿಯೂ ಅದು ಜಾಗ್ರತವಾಗುವಂತೆಮಾಡಿ.

