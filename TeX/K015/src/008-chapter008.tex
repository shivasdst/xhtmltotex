
\chapter{ಜನಸಾಮಾನ್ಯರ ಶಿಕ್ಷಣ}

\section{ಜನಾಂಗದ ಮಹಾಪಾಪ}

ಭರತಖಂಡದಲ್ಲಿ ದರಿದ್ರರ ದೀನರ ಸ್ಥಿತಿಯನ್ನು ಆಲೋಚಿಸಿದರೆ ನನ್ನ ಹೃದಯ ಮರುಗುವುದು. ಪ್ರತಿದಿನವೂ ಅಧಃಪಾತಳಕ್ಕೆ ಹೋಗುತ್ತಿರುವರು. ನಿರ್ದಯ ಸಮಾಜ ಅವರ ಮೇಲೆ ಕರೆವ ಪೆಟ್ಟಿನ ಸುರಿಮಳೆಯನ್ನು ಅವರು ಅನುಭವಿಸುತ್ತಿರುವರು. ಆದರೆ ಪೆಟ್ಟು ಎಲ್ಲಿಂದ ಬರುವುದು ಎನ್ನುವುದು ಅವರಿಗೆ ಗೊತ್ತಿಲ್ಲ. ನಾವು ಕೂಡಾ ಮನುಷ್ಯರು ಎನ್ನುವುದನ್ನು ಅವರು ಮರೆತಿರುವರು. ಎಲ್ಲಿಯವರೆಗೂ ಲಕ್ಷಾಂತರ ಜನರು ಅಜ್ಞಾನದಲ್ಲಿ ಇರು ವರೊ, ಹಸಿವಿನಿಂದ ನರುಳುತ್ತಿರುವರೋ, ಅಲ್ಲಿಯವರೆಗೂ ಯಾರು ಅವರ ಕಷ್ಟದಿಂದ ತಮ್ಮ ವಿದ್ಯಾಭ್ಯಾಸವನ್ನು ಪಡೆದೂ,ಅಂಥವರ ಕಡೆಗೆ ಸ್ವಲ್ಪವೂ ಗಮನಕೊಡದೆ ಇರುವವರನ್ನು ವಿಶ್ವಾಸಘಾತಕರೆಂದು ಕರೆಯುತ್ತೇನೆ. ನಮ್ಮ ದೇಶದ ದೊಡ್ಡ ಪಾಪವೆ ಜನಸಾಮಾನ್ಯರನ್ನು ನಿರ್ಲಕ್ಷಿಸಿದ್ದು. ಅದೇ ನಮ್ಮ ಅವನತಿಗೆ ಕಾರಣ. ಭರತಖಂಡದಲ್ಲಿ ಜನಸಾಮಾನ್ಯರಿಗೆ ಪುನಃ ಒಳ್ಳೆಯ ಶಿಕ್ಷಣ ಕೊಟ್ಟು, ಹೊಟ್ಟೆತುಂಬ ಅನ್ನವಿಟ್ಟು, ಅವರನ್ನು ಚೆನ್ನಾಗಿ ಕಾಪಾಡುವ ವರೆಗೆ ಎಂತಹ ರಾಜಕೀಯ ಚಳುವಳಿಯಿಂದಲೂ ಪ್ರಯೋಜನವಿಲ್ಲ.


\section{ಜನಸಾಮಾನ್ಯರ ಶಿಕ್ಷಣವೊಂದೇ ಪರಿಹಾರ}

ಜನಸಾಮಾನ್ಯರಲ್ಲಿ ವಿದ್ಯೆ ಮತ್ತು ಬುದ್ಧಿವಂತಿಕೆ ಎಷ್ಟು ಹೆಚ್ಚಿದರೆ ಅಷ್ಟು ದೇಶವೂ ಮುಂದುವರಿಯುವುದು. ಭರತಖಂಡದ ಅವನತಿಗೆ ಮುಖ್ಯಕಾರಣ ದೇಶದ ವಿದ್ಯಾಭ್ಯಾಸ ಮತ್ತು ಬುದ್ಧಿವಂತಿಕೆಯನ್ನೆಲ್ಲ ಎಲ್ಲೋ ಕೆಲವರು ತಮ್ಮ ವಶಮಾಡಿಕೊಂಡುದುದು. ನಾವು ಪುನಃ ಅಭಿವೃದ್ಧಿಗೆ ಬರಬೇಕಾದರೆ, ಜನ ಸಾಮಾನ್ಯರಿಗೆ ವಿದ್ಯೆಯನ್ನು ಹರಡುವುದರ ಮೂಲಕ ಸಾಧಿಸಬೇಕು. ಹಿಂದುಳಿದ ನಮ್ಮ ಪಂಗಡದವರಿಗೆ ಮಾಡಬೇಕಾದ ಒಂದು ಸೇವೆಯೇ, ಅವರು ಕಳೆದುಕೊಂಡ ತಮ್ಮ ವ್ಯಕ್ತಿತ್ವವನ್ನು ಮತ್ತೊಮ್ಮೆ ಪುಷ್ಟಿಗೊಳಿಸು ವುದಕ್ಕೆ ಶಿಕ್ಷಣ ಕೊಡುವುದು. ಅವರಿಗೆ ವಿಚಾರಗಳನ್ನು ಕೊಡಬೇಕು. ಅವರ ಸುತ್ತಲೂ ಪ್ರಪಂಚದಲ್ಲಿ ಏನಾಗುತ್ತಿದೆ ಎನ್ನುವುದನ್ನು ನೋಡುವುದಕ್ಕೆ ಅವರ ಕಣ್ಣನ್ನು ತೆರೆಯಬೇಕು. ಅವರು ತಮ್ಮ ಆತ್ಮೋದ್ಧಾರವನ್ನು ತಾವೇ ಮಾಡಿ ಕೊಳ್ಳುವರು. ಪ್ರತಿಯೊಂದು ದೇಶವೂ ಪ್ರತಿಯೊಬ್ಬ ಪುರುಷನೂ ಮತ್ತು ಸ್ತ್ರೀಯೂ ತಮ್ಮ ಆತ್ಮೋದ್ಧಾರವನ್ನು ತಾವೇ ಮಾಡಿಕೊಳ್ಳಬೇಕು. ಅವರಿಗೆ ವಿಚಾರಗಳನ್ನು ಕೊಡಿ, ಅವರಿಗೆ ಬೇಕಾದ ಸಹಾಯ ಅದೊಂದೆ. ನಂತರ ಉಳಿದುದು ಅದರ ಪರಿಣಾಮವಾಗಿ ಬಂದೇ ಬರುವುದು. ರಾಸಾಯನಿಕ ದ್ರವ್ಯಗಳನ್ನು ಒಟ್ಟಿಗೆ ಸೇರಿಸುವುದು ಮಾತ್ರ ನಮ್ಮ ಕೆಲಸ. ಪ್ರಕೃತಿ ನಿಯಮ ವನ್ನು ಅನುಸರಿಸಿ ಅದು ಸ್ಫಟಿಕವಾಗುವುದು.


\section{ಉದಾತ್ತ ಆಧ್ಯಾತ್ಮಿಕ ಸತ್ಯಗಳು ಅವರಿಗೆ ದೊರುಕುವಂತೆ ಮಾಡಿ}

ಮೊದಲನೆಯದಾಗಿ ನನ್ನ ಆಲೋಚನೆ ಏನೆಂದರೆ ನಮ್ಮ ಶಾಸ್ತ್ರಗಳಲ್ಲಿ ಮತ್ತು ಎಲ್ಲೋ ಹಲಕೆಲವರಲ್ಲಿ ಶೇಖರಿಸಲ್ಪಟ್ಟಿರುವ, ಮಠಗಳಲ್ಲಿ ಮತ್ತು ಅರಣ್ಯಗಳಲ್ಲಿ ಅವಿತಂತಿರುವ ಮಹಾ ಆಧ್ಯಾತ್ಮಿಕ ಶ್ರೇಷ್ಠ ರತ್ನಗಳನ್ನು ಹೊರಗೆ ತರಬೇಕು.ಇವು ಯಾರಲ್ಲಿ ಅವಿತುಕೊಂಡಿದೆಯೋ ಅವರಿಂದ ಮಾತ್ರವಲ್ಲ, ಅದಕ್ಕಿಂತ ಹೆಚ್ಚಾಗಿ ಶತಶತಮಾನಗಳ ಸಂಸ್ಕೃತಭಾಷೆಯ ಶಬ್ದಜಾಲಗಳ ಕಪಿಮುಷ್ಟಿಯಲ್ಲಿ ಅವಿತುಕೊಂಡಿರುವ ಅವುಗಳು ಹೊರಗೆ ಬರಬೇಕು. ಒಂದು ಮಾತಿನಲ್ಲಿ ಹೇಳುವುದಾದರೆ ಅವುಗಳನ್ನು ಜನಾದರಣೀಯವನ್ನಾಗಿ ಮಾಡಬೇಕು. ಈ ಭಾವನೆಗಳನ್ನು ಪ್ರಕಾಶಕ್ಕೆ ತರಬೇಕು. ಭರತಖಂಡದಲ್ಲಿ ಪ್ರತಿಯೊಬ್ಬರ (ಅವರಿಗೆ ಸಂಸ್ಕೃತ ಗೊತ್ತಿರಲಿ ಇಲ್ಲದೆ ಇರಲಿ) ಸರ್ವಮಾನ್ಯ ಆಸ್ತಿಯಾಗಬೇಕು ಇವು. ಇದನ್ನು ಸಾಧಿಸುವುದಕ್ಕೆ ಇರುವ ದೊಡ್ಡ ಆತಂಕವೇ ನಮ್ಮ ಪ್ರಸಿದ್ಧ ಸಂಸ್ಕೃತ ಭಾಷೆ. ಸಾಧ್ಯವಾದರೆ ದೇಶದಲ್ಲಿರುವವರೆಲ್ಲರನ್ನೂ ಒಳ್ಳೆಯ ಸಂಸ್ಕೃತಪಂಡಿತರನ್ನಾಗಿ ಮಾಡುವವರೆಗೆ ಈ ಆತಂಕವನ್ನು ನಾವು ಪರಿಹರಿಸಲಾಗುವುದಿಲ್ಲ. ಈ ಭಾಷೆಯನ್ನು ನಾನು ಹುಡುಗನಾದಾಗಿನಿಂದಲೂ ಓದುತ್ತಿರುವೆನು. ಆದರೂ ಪ್ರತಿಯೊಂದು ಹೊಸ ಪುಸ್ತಕವೂ ನನಗೆ ಹೊಸದೆ ಎಂದು ಹೇಳಿದಾಗ ಈ ಭಾಷೆ ಎಷ್ಟು ಕಷ್ಟ ಎಂಬುದು ನಿಮಗೆ ಗೊತ್ತಾಗು ವುದು. ಇದನ್ನು ಚೆನ್ನಾಗಿ ಓದುವುದಕ್ಕೆ ಅವಕಾಶವಿಲ್ಲದವರಿಗೆ ಮತ್ತೆ ಎಷ್ಟು ಕಷ್ಟವಿರಬಹುದು! ಆದಕಾರಣವೇ ಈ ಭಾವನೆಗಳನ್ನು ಮಾತೃಭಾಷೆಯ ಮೂಲಕವೇ ಹೇಳಬೇಕು. ಜನಸಾಮಾನ್ಯರಿಗೆ ದೇಶಭಾಷೆಯಲ್ಲಿ ತಿಳಿಸಿ, ಅವರಿಗೆ ಭಾವನೆಗಳನ್ನು ಕೊಡಿ. ಅವರು ಮಾಹಿತಿಯನ್ನು ಪಡೆಯುತ್ತಾರೆ. ಆದರೆ ಇದಕ್ಕಿಂತ ಹೆಚ್ಚಿನ ಅವಶ್ಯವಿದೆ. ಅವರಿಗೆ ಸಂಸ್ಕೃತಿಯನ್ನು ಕೊಡಿ. ನೀವು ಅದನ್ನು ಕೊಡುವ ತನಕ ಉತ್ತಮಗೊಂಡ ಜನಸಾಮಾನ್ಯರ ಸ್ಥಿತಿ ಶಾಶ್ವತವಾಗಲಾರದು.


\section{ಸಂಸ್ಕೃತ ಶಿಕ್ಷಣ}

ಏಕಕಾಲದಲ್ಲಿಯೇ ಸಂಸ್ಕೃತ ಶಿಕ್ಷಣವೂ ಕೂಡ ಜೊತೆಯಲ್ಲಿಯೇ ಹೋಗ ಬೇಕು. ಸಂಸ್ಕೃತ ಶಬ್ದಗಳ ಧ್ವನಿಯೆ ಜನಾಂಗಕ್ಕೆ ಒಂದು ಗೌರವವನ್ನು, ಚೇತನವನ್ನು ನೀಡುವುದು. ಸಾಧಾರಣ ಜನರು ಸಂಸ್ಕೃತ ಭಾಷೆ ಓದುವುದನ್ನು ನಿಲ್ಲಿಸುವಂತೆ ಮಾಡಿ ಭಗವಾನ್ ಬುದ್ಧದೇವನೆ ಒಂದು ತಪ್ಪನ್ನು ಮಾಡಿದನು. ಅವನಿಗೆ ತಕ್ಷಣವೇ ಪ್ರತಿಫಲ ಬೇಕಾಗಿತ್ತು. ಆದಕಾರಣ ಅಂದಿನ ಪಾಲಿ ಭಾಷೆಯಲ್ಲಿ ಭಾಷಾಂತರಮಾಡಿ ಉಪದೇಶಿಸಿದನು. ಅದೇನೋ ಬಹಳ ಒಳ್ಳೆಯದಾಯಿತು. ಅವನು ಜನರ ಭಾಷೆಯಲ್ಲಿ ಮಾತನಾಡಿದನು. ಜನರು ಅದನ್ನು ಅರ್ಥಮಾಡಿಕೊಂಡರು. ಇದರಿಂದ ವಿಷಯಗಳು ಬೇಗ ಮತ್ತು ಎಲ್ಲಾ ಕಡೆಯಲ್ಲಿಯೂ ಹರಡುವುದಕ್ಕೆ ಸಹಾಯವಾಯಿತು. ಆದರೆ ಇದರ ಜೊತೆಗೆ ಸಂಸ್ಕೃತವೂ ಹರಡಬೇಕಾಗಿತ್ತು. ಜ್ಞಾನ ಬಂತು, ಗೌರವ ಬರಲಿಲ್ಲ. ಅದನ್ನು ನೀವು ಕೊಡುವತನಕ ಸಂಸ್ಕೃತ ಭಾಷೆಯ ಉಪಯುಕ್ತತೆಯನ್ನು ಪಡೆದು ಸುಲಭವಾಗಿ ಉಳಿದವರಿಗಿಂತ ಮೇಲಾಗುವ ಮತ್ತೊಂದು ಜಾತಿ ಹುಟ್ಟುವುದು.


\section{ದೇಶವಿರುವುದು ಗುಡಿಸಲಿನಲ್ಲಿ}

ದೇಶ ಜೀವಿಸಿರುವುದು ಗುಡಿಸಲಿನಲ್ಲಿ ಎನ್ನುವುದನ್ನು ಜ್ಞಾಪಕದಲ್ಲಿಡಿ. ಸದ್ಯಕ್ಕೆ ನಿಮ್ಮ ಕರ್ತವ್ಯ ದೇಶದ ಒಂದು ಭಾಗದಿಂದ ಮತ್ತೊಂದು ಭಾಗಕ್ಕೆ, ಹಳ್ಳಿಯಿಂದ ಹಳ್ಳಿಗೆ ಹೋಗಿ ಜನರಿಗೆ ಸುಮ್ಮನೆ ಕುಳಿತರೆ ಪ್ರಯೋಜನವಿಲ್ಲ ವೆಂದು ತಿಳಿಸಿ ಅವರ ನೈಜಸ್ಥಿತಿ ಅವರಿಗೆ ಗೊತ್ತಾಗುವಂತೆ ಮಾಡುವುದು. “ಓ ನನ್ನ ಸಹೋದರರೆ ನೀವೆಲ್ಲ ಏಳಿ, ಜಾಗ್ರತರಾಗಿ!” ಎಂದು ಹೇಳಿ. ತಮ್ಮ ಸ್ಥಿತಿಯನ್ನು ಹೇಗೆ ಉತ್ತಮಗೊಳಿಸಬೇಕು ಎಂಬ ಬುದ್ಧಿವಾದವನ್ನು ಹೇಳಿ, ಅತಿ ಗಹನವಾದ ಶಾಸ್ತ್ರಗಳ ಸತ್ಯವನ್ನು ಸುಲಭವಾಗಿ ಜನರಂಜಕವಾಗುವ ರೀತಿಯಲ್ಲಿ ಅವರಿಗೆ ಹೇಳಿ ತಿಳಿದುಕೊಳ್ಳುವಂತೆ ಮಾಡಿ. ಧರ್ಮಕ್ಕೆ ಬ್ರಾಹ್ಮಣ ರಿಗೆ ಇರುವಷ್ಟೇ ಅಧಿಕಾರ ಅವರಿಗೂ ಇದೆ ಎಂಬುದನ್ನು ಮನಮುಟ್ಟುವಂತೆ ಹೇಳಿ. ಇಂತಹ ಜಾಜ್ವಲ್ಯಮಾನವಾದ ಮಂತ್ರಗಳನ್ನು ಚಂಡಾಲನವರೆಗೆ ಎಲ್ಲರಿಗೂ ಉಪದೇಶಮಾಡಿ. ಜೊತೆಗೆ ಜೀವನದ ಆವಶ್ಯಕತೆಗಳು, ವ್ಯಾಪಾರ, ವ್ಯವಹಾರ, ಕೃಷಿ ಮುಂತಾದ ವಿಷಯಗಳ ಮೇಲೆ ಸುಲಭವಾದ ಭಾಷೆಯಲ್ಲಿ ಅವರಿಗೆ ಸಲಹೆ ಕೊಡಿ.


\section{ಜೀವನದ ಎಲ್ಲಾ ಕಾರ್ಯಕ್ಷೇತ್ರಗಳನ್ನು ಪಾವನಮಾಡಿ}

ಜಾತಿ, ರಾಜ, ಪರದೇಶೀಯರ ಶತಶತಮಾನಗಳ, ಸಹಸ್ರಾರು ವರ್ಷಗಳ ದಬ್ಬಾಳಿಕೆ ಅವರಲ್ಲಿರುವ ಶಕ್ತಿಯನ್ನೆಲ್ಲ ಹೀರಿಕೊಂಡಿರುವುದು. ಶಕ್ತಿಯನ್ನು ಪಡೆಯಬೇಕಾದರೆ ಮೊದಲನೆಯ ಕೆಲಸ ಉಪನಿಷತ್ತುಗಳನ್ನು ಸಮರ್ಥಿಸಿ, “ನಾನು ಆತ್ಮ” ಎಂಬುದನ್ನು ನಂಬಬೇಕು. “ಖಡ್ಗ ನನ್ನನ್ನು ಕತ್ತರಿಸಲಾರದು. ಬೆಂಕಿ ನನ್ನನ್ನು ದಹಿಸಲಾರದು. ಗಾಳಿ ನನ್ನನ್ನು ಒಣಗಿಸಲಾರದು. ಸರ್ವಶಕ್ತನು ನಾನು. ಸರ್ವಜ್ಞನು ನಾನು” ಎಂದು ನಂಬಿ. ನಿರ್ಜನಾರಣ್ಯಗಳಿಂದ ಈ ವೇದಾಂತತತ್ತ್ವಗಳು ಹೊರಗೆ ಬರಬೇಕು. ಕೋರ್ಟು ಕಚೇರಿಗಳಲ್ಲಿ, ಉಪದೇಶ ಪೀಠಗಳಲ್ಲಿ ದೀನರ ಜೋಪಡಿಗಳಲ್ಲಿ, ಮೀನನ್ನು ಹಿಡಿಯುತ್ತಿರುವ ಬೆಸ್ತರಲ್ಲಿ, ವ್ಯಾಸಂಗಮಾಡುತ್ತಿರುವ ವಿದ್ಯಾರ್ಥಿಗಳಲ್ಲಿ ಈ ಉಪದೇಶಗಳು ಅನುಷ್ಠಾನ ಯೋಗ್ಯವಾಗಬೇಕು. ಅವರು ಎಲ್ಲಿಯಾದರೂ ಇರಲಿ, ಏನು ಕಸುಬನ್ನಾದರೂ ಮಾಡುತ್ತಿರಲಿ, ಇದು ಪ್ರತಿಯೊಬ್ಬ ಪುರುಷ ಸ್ತ್ರೀ ಬಾಲರನ್ನು ಕರೆಯುವುದು. ಬೆಸ್ತರು ಮತ್ತು ಉಳಿದವರು ಈ ವಿಷಯಗಳನ್ನು ಹೇಗೆ ಅನುಷ್ಠಾನಕ್ಕೆ ತರಬಲ್ಲರು? ಇದಕ್ಕೆ ದಾರಿ ತೋರಿಸಲ್ಪಟ್ಟಿದೆ. ಬೆಸ್ತನು ತಾನು ಆತ್ಮವೆಂದು ತಿಳಿದರೆ ಅವನು ಒಳ್ಳೆಯ ಬೆಸ್ತನಾಗುವನು. ವಿದ್ಯಾರ್ಥಿ ತಾನು ಆತ್ಮನೆಂದು ತಿಳಿದರೆ ಆತ ಉತ್ತಮ ವಿದ್ಯಾರ್ಥಿಯಾಗುವನು.


\section{ವಿದ್ಯಾಭ್ಯಾಸ ಪ್ರತಿಯೊಂದು ಮನೆಗೂ ಮುಟ್ಟಬೇಕು}

ಭರತಖಂಡದಲ್ಲಿ ಇರುವ ದೋಷದ ಮೂಲವೇ ದರಿದ್ರರ ಶೋಚನೀಯ ಸ್ಥಿತಿ. ಪ್ರತಿಯೊಂದು ಹಳ್ಳಿಯಲ್ಲಿಯೂ ನೀವು ಒಂದು ಪಾಠಶಾಲೆಯನ್ನು ತೆರೆದಿರಿ ಎಂದು ಇಟ್ಟುಕೊಳ್ಳೋಣ. ಆದರೂ ಇದರಿಂದ ಏನೂ ಪ್ರಯೋಜನ ವಿಲ್ಲ. ಶಾಲೆಗೆ ಹೋಗುವ ಬದಲು ಬಡವರ ಮಕ್ಕಳು ಹೊಲದಲ್ಲಿ ತಮ್ಮ ತಂದೆಯ ಸಹಾಯಕ್ಕೆ ಹೋಗುವರು, ಇಲ್ಲವೆ ಏನಾದರೂ ಸಂಪಾದಿಸುವುದಕ್ಕೆ ಪ್ರಯತ್ನಿಸುವರು. ಭರತಖಂಡದಲ್ಲಿ ಇರುವ ಬಡತನ ಇಂತಹುದು. ಈಗ ಬೆಟ್ಟ ಮಹಮ್ಮದನ ಸಮೀಪಕ್ಕೆ ಹೋಗದೆ ಇದ್ದರೆ ಮಹಮ್ಮದನು ಬೆಟ್ಟದ ಸಮೀಪಕ್ಕೆ ಬರಬೇಕು. ಬಡವರ ಮಕ್ಕಳು ಶಿಕ್ಷಣಕ್ಕೆ ಬರದೆ ಇದ್ದರೆ ಶಿಕ್ಷಣ ಅವರ ಹತ್ತಿರ ಹೋಗಬೇಕು. ಹಳ್ಳಿಯಿಂದ ಹಳ್ಳಿಗೆ ಧರ್ಮವನ್ನು ಬೋಧಿ ಸುತ್ತ ಹೋಗುತ್ತಿರುವ ಸಾವಿರಾರು ಮಂದಿ ಏಕಾಗ್ರಚಿತ್ತರಾದ, ತ್ಯಾಗಿಗಳಾದ ಸಂನ್ಯಾಸಿಗಳು ಇರುವರು. ಅವರಲ್ಲಿ ಕೆಲವರನ್ನು ಲೌಕಿಕ ವಿಚಾರಗಳನ್ನು ಕೂಡ ಬೋಧಿಸುವುದಕ್ಕೆ ಒಟ್ಟುಗೂಡಿಸಿದರೆ, ಅವರು ಊರಿನಿಂದ ಊರಿಗೆ ಮನೆ ಯಿಂದ ಮನೆಗೆ ಧಾರ್ಮಿಕ ವಿಚಾರವನ್ನು ಮಾತ್ರವಲ್ಲ, ಲೌಕಿಕ ವಿಷಯಗಳನ್ನು ಕೂಡ ಕಲಿಸಹೋಗುವರು. ಇವರಲ್ಲಿ ಇಬ್ಬರು ಒಂದು ಹಳ್ಳಿಗೆ ಮಾಯಾ ದೀಪ, ಕೃತಕಗೋಳ ಭೂಪಟಗಳೊಂದಿಗೆ ಹೋದರು ಎಂದು ಇಟ್ಟು ಕೊಳ್ಳೋಣ. ಅವರು ಏನೂ ಗೊತ್ತಿಲ್ಲದವರಿಗೆ ಖಗೋಳಶಾಸ್ತ್ರ ಭೂಗೋಳ ಶಾಸ್ತ್ರ ಮುಂತಾದುವುಗಳನ್ನು ಎಷ್ಟೋ ತಿಳಿಸಬಹುದು! ಬೇರೆ ಬೇರೆ ದೇಶಕ್ಕೆ ಸಬಂಧಪಟ್ಟ ಕಥೆಗಳನ್ನು ಹೇಳುವುದರ ಮೂಲಕ ತಮ್ಮ ಇಡೀ ಜೀವಮಾನ ದಲ್ಲೇ ಪುಸ್ತಕಗಳಿಂದ ಗ್ರಹಿಸುವುದಕ್ಕಿಂತ ನೂರು ಪಾಲಿನಷ್ಟು ಹೆಚ್ಚು ಸುದ್ದಿ ಗಳನ್ನು ಕೇಳುವುದರಿಂದ ಅವರು ಗ್ರಹಿಸುವಂತೆ ಮಾಡಬಹುದು. ಆಧುನಿಕ ವಿಜ್ಞಾನಶಾಸ್ತ್ರಗಳ ಸಹಾಯದಿಂದ ಅವರ ಜ್ಞಾನವನ್ನು ಹತ್ತಿಸಿ. ಚರಿತ್ರೆ, ಭೂಗೋಳ, ಸಾಹಿತ್ಯ, ವಿಜ್ಞಾನ ಮತ್ತು ಅವುಗಳೊಂದಿಗೆ ಅತ್ಯಂತ ಗಹನವಾದ ಧಾರ್ಮಿಕ ಸತ್ಯಗಳನ್ನು ಕೂಡ ಬೋಧಿಸಿ.

ಜೀವನೋಪಾಯದ ಹೋರಾಟದಲ್ಲೆ ನಿರತರಾಗಿ ಅವರಿಗೆ ಜ್ಞಾನೋ ದಯಕ್ಕೆ ಅವಕಾಶವೇ ಇರಲಿಲ್ಲ. ಇಂದಿನವರೆವಿಗೂ ಅವರು ಯಂತ್ರಗಳಂತೆ ಕೆಲಸ ಮಾಡಿದರು. ಬುದ್ಧಿವಂತರಾದ ವಿದ್ಯಾವಂತರ ಗುಂಪು ಅವರ ಶ್ರಮದ ಬಹುಪಾಲು ಪ್ರತಿಫಲವನ್ನು ತೆಗೆದುಕೊಂಡಿತು. ಆದರೆ ಈಗ ಕಾಲ ಬದಲಾಯಿ ಸಿದೆ. ಹಿಂದುಳಿದ ಪಂಗಡದವರು ಕ್ರಮೇಣ ಈ ವಿಷಯದಲ್ಲಿ ಜಾಗ್ರತರಾಗು ತ್ತಿರುವರು. ಮೇಲಿನ ವರ್ಗದವರು ಇನ್ನು ಎಷ್ಟು ಪ್ರಯತ್ನಪಟ್ಟರೂ ಕೆಳಗಿ ನವರನ್ನು ಅಡಗಿಸುವುದಕ್ಕೆ ಆಗುವುದಿಲ್ಲ. ಮೇಲಿನ ವರ್ಗದ ಹಿತರಕ್ಷಣೆ ಈಗ ಕೆಳಗೆ ಇರುವವರಿಗೆ ನ್ಯಾಯವಾದ ಹಕ್ಕನ್ನು ಸಂಪಾದಿಸಿ ಕೊಡುವುದಕ್ಕೆ ಸಹಾಯ ಮಾಡುವುದರ ಮೇಲೆ ನಿಂತಿದೆ. ಆದಕಾರಣವೇ ಜನಸಾಮಾನ್ಯರಲ್ಲಿ ವಿದ್ಯೆ ಹರಡುವ ಕೆಲಸದಲ್ಲಿ ನಿರತರಾಗಿ ಎಂದು ನಾನು ನಿಮಗೆ ಹೇಳುವುದು. “ನೀವು ನಮ್ಮ ಸಹೋದರರು. ನಮ್ಮ ದೇಶದಲ್ಲಿ ಒಂದು ಭಾಗ” ಎಂಬು ದನ್ನು ಅವರ ಮನಮುಟ್ಟುವಂತೆ ಮಾಡಿ. ಅವರಿಗೆ ನಿಮ್ಮಿಂದ ಸಹಾನುಭೂತಿ ದೊರೆತರೆ ಕೆಲಸ ಮಾಡುವುದಕ್ಕೆ ಉತ್ಸಾಹ ನೂರ್ಮಡಿ ಹೆಚ್ಚುವುದು.


\section{ಮಹಾಕಾರ್ಯಸಾಧನೆಗೆ ಅನುಕಂಪ ಆವಶ್ಯಕ}

ಮಹತ್ಕಾರ್ಯಸಾಧನೆಗೆ ಮೂರು ವಿಷಯಗಳು ಅತ್ಯಾವಶ್ಯಕ. ಮೊದಲನೆ ಯದಾಗಿ ಹೃತ್ಪೂರ್ವಕ ಅವುಗಳಿಗಾಗಿ ಕಾತರಿಸಬೇಕು. ಬುದ್ಧಿಯಲ್ಲೇನಿದೆ? ಯುಕ್ತಿಯಲ್ಲೇನಿದೆ? ಕೆಲವು ಹೆಜ್ಜೆ ಹೋಗುವುದು, ಅಲ್ಲಿ ನಿಲ್ಲುವುದು. ಆದರೆ ಸ್ಫೂರ್ತಿ ಬರುವುದು ಹೃದಯದಿಂದ, ಪ್ರೇಮ ಅತ್ಯಂತ ಅಸಾಧ್ಯವಾದ ಬಾಗಿಲನ್ನು ಕೂಡ ತೆರೆಯುವುದು. ಆದಕಾರಣವೇ ದೇಶಭಕ್ತರಾಗಬೇಕೆಂದು ಬಯಸುವ ನನ್ನ ಸಹೋದರರೇ, ಮೊದಲು ಅನುಕಂಪವನ್ನು ತೋರಿ. ಅನು ಕಂಪವನ್ನು ತೋರುವಿರಾ! ದೇವರ, ಪುಷಿಗಳ, ವಂಶಾನುಗತರಾದ ಲಕ್ಷಾಂತರ ಜನರು ಮೂಢರಿಗಿಂತ ಕನಿಷ್ಟರಾಗಿರುವರೆಂಬುದನ್ನು ನೀವು ಭಾವಿಸಬಲ್ಲಿರಾ? ಲಕ್ಷಾಂತರ ಜನರು ಇಂದು ಉಪವಾಸದಿಂದ ನರಳುತ್ತಿರುವರು, ಅನೇಕ ಶತಮಾನಗಳಿಂದಲೂ ನರಳುತ್ತಿರುವರು, ಎಂಬುದು ನಿಮಗೆ ತೋರುವುದೇ? ಭಾರತಾವನಿಯ ಮೇಲೆ ಮೌಢ್ಯ ಎನ್ನುವುದು ಗಾಢಾಂಧಕಾರದ ಮುಗಿಲಿನಂತೆ ನಿಮಗೆ ಕಾಣುವುದೆ? ಇದರಿಂದ ನಿಮಗೆ ಅಸಮಾಧಾನವಾಗಿದೆಯೆ? ಇದರಿಂದ ನಿಮ್ಮ ಸುಖನಿದ್ರೆಗೆ ಭಂಗ ಬಂದಿದೆಯೇ? ಇದು ನಿಮ್ಮ ರಕ್ತಗತವಾಗಿದೆಯೆ? ಈ ಭಾವ ನಾಡಿಯಲ್ಲಿ ಸಂಚರಿಸಿ ಹೃದಯದ ಸ್ಪಂದನದೊಂದಿಗೆ ಧ್ವನಿಮಾಡು ತ್ತಿದೆಯೇ? ಇದು ನಿಮ್ಮನ್ನು ಹುಚ್ಚರಾಗುವಂತೆ ಮಾಡಿದೆಯೇ? ಸರ್ವನಾಶದ ಘೋರದುಃಖಭಾವ ಒಂದರಿಂದಲೆ ನೀವು ಉನ್ಮತ್ತರಾಗಿ ಇರುವಿರಾ? ನಿಮ್ಮ ಹೆಸರು, ನಿಮ್ಮ ಕೀರ್ತಿ, ನಿಮ್ಮ ಹೆಂಡತಿ, ನಿಮ್ಮ ಮಕ್ಕಳು, ನಿಮ್ಮ ಆಸ್ತಿ, ನಿಮ್ಮ ಸ್ವಂತ ದೇಹವನ್ನು ಕೂಡ ಮರೆತಿರುವಿರಾ? ನೀವು ಇದನ್ನು ಮಾಡಿರು ವಿರಾ? ಇದೇ ಮೊದಲನೇ ಹೆಜ್ಜೆ.


\section{ಪರಿಹಾರ}

ನೀವು ಕನಿಕರವನ್ನು ತೋರಬಹುದು. ನಿಮ್ಮ ಶಕ್ತಿಯನ್ನು ಕೇವಲ ಬೆಲೆ ಯಿಲ್ಲದ ಮಾತಿನಲ್ಲಿ ಕಳೆಯದೆ ಅವರ ದುಃಖವನ್ನು ಶಮನಮಾಡಿ, ಅವರನ್ನು ಈ ಜೀವಶ್ಶವದಂತೆ ಇರುವ ಸ್ಥಿತಿಯಿಂದ ಸುಧಾರಿಸುವುದಕ್ಕೆ ಯಾವುದಾದ ರೊಂದು ದಾರಿಯನ್ನು, ಅನುಷ್ಠಾನಯೋಗ್ಯವಾದ ಪರಿಹಾರವನ್ನು ನೀವು ಕಂಡುಹಿಡಿದಿರುವಿರಾ? ಆದರೂ ಇದೇ ಅಲ್ಲ. ಎಲ್ಲ ಪರ್ವತೋಪಮ ಆತಂಕಗಳನ್ನು ದಾಟುವಂತಹ ವಜ್ರೋಪಮ ಇಚ್ಛಾಶಕ್ತಿಯು ನಿಮ್ಮಲ್ಲಿ ಇದೆಯೆ? ಇಡೀ ಪ್ರಪಂಚವೇ ನಿಮ್ಮನ್ನು ವಿರೋಧಿಸಿ ಸಶಸ್ತ್ರರಾಗಿ ನಿಂತರೂ, ಯಾವುದು ನಿಮಗೆ ಸರಿಯೆಂದು ತೋರುವುದೊ ಅದನ್ನು ಮಾಡಲು ಸಿದ್ಧ ರಾಗಿರುವಿರಾ? ನಿಮ್ಮ ಹೆಂಡತಿ ಮಕ್ಕಳು ನಿಮಗೆ ವಿರೋಧಿಗಳಾದರೂ ನಿಮ್ಮ ದ್ರವ್ಯವೆಲ್ಲ ನಷ್ಟವಾದರೂ, ನಿಮ್ಮ ಕೀರ್ತಿ ನಶಿಸಿದರೂ, ಆಸ್ತಿ ಮಾಯ ವಾದರೂ, ನೀವು ಹಿಡಿದುದನ್ನು ಬಿಡದೆ ಇರಬಲ್ಲಿರಾ? ಭರ್ತೃಹರಿ ಹೇಳು ವಂತೆ:

\begin{myquote}
ನಿಂದಂತು ನೀತಿನಿಪುಣಾ ಯದಿ ವಾ ಸ್ತುವಂತು\\ಲಕ್ಷ್ಮೀಸ್ಸಮಾವಿಷತು ಗಚ್ಛತು ವಾ ಯಥೇಷ್ಟಂ\\ಅದ್ಯೈವ ಮರಣಮಸ್ತು ಯುಗಾಂತರೇವಾ ನ್ಯಾ-\\ಯಾತ್ಪಥಃ ಪ್ರವಿಚಲಂತಿ ಪದಂ ನ ದೀರಾಃ
\end{myquote}

“ನೀತಿ ನಿಪುಣರು ನಿಂದಿಸಲಿ ಅಥವಾ ಹೊಗಳಲಿ, ಲಕ್ಷ್ಮಿಯು ತನಗಿಷ್ಟ ಬಂದಂತೆ ಬರಲಿ ಅಥವಾ ಹೋಗಲಿ, ಸಾವು ಈಗ ಬರಲಿ, ಅಥವಾ ಇನ್ನೂ ನೂರು ವರ್ಷಗಳಾದಮೇಲೆ ಬರಲಿ, ಸತ್ಯಪಥದಿಂದ ಯಾರು ಒಂದು ಅಂಗುಲವೂ ಹಿಂದೆ ಸರಿಯುವುದಿಲ್ಲವೋ ಅವರೇ ಧೀರರು”. ಇಂತಹ ಸ್ಥಿರತೆ ಇದೆಯೆ? ಈ ಮೂರು ಗುಣಗಳೂ ನಿಮ್ಮಲ್ಲಿ ಇದ್ದರೆ ನಿಮ್ಮಲ್ಲಿ ಪ್ರತಿ ಯೊಬ್ಬರೂ ಕೂಡ ಅದ್ಭುತ ಕಾರ್ಯಗಳನ್ನು ಸಾಧಿಸುವಿರಿ.


\section{ಕೆಲಸ ಪೂಜೆಯಂತೆ}

“ಕರುಣಾಳು ಬಾ ಬೆಳಕೆ, ದಾರಿ ತೋರೆ”ಂದು ಪ್ರಾರ್ಥಿಸುವ, ಕತ್ತಲೆಯಿಂದ ಬೆಳಕಿನ ಕಿರಣವೊಂದು ಬರುವುದು; ನಮಗೆ ದಾರಿ ತೋರುವುದು. ನೀಡಿದ ಕೈಯೊಂದು ಬರುವುದು. ಹಗಲು ರಾತ್ರಿ ನಮ್ಮಲ್ಲಿ ಪ್ರತಿಯೊಬ್ಬರೂ ಕೂಡ ದಬ್ಬಾಳಿಕೆಗೆ ತುತ್ತಾಗಿ, ಬಡತನ ಪೌರೋಹಿತ್ಯ ಮತ್ತು ದರ್ಪಗಳ ಕೋಟಲೆ ಯಲ್ಲಿ ನರಳುತ್ತಿರುವ ಲಕ್ಷಾಂತರ ಭಾರತೀಯರಿಗಾಗಿ ಪ್ರಾರ್ಥಿಸೋಣ. ಪಂಡಿತರಿಗೆ ಶ್ರೀಮಂತರಿಗೆ ಬೋಧಿಸುವುದಕ್ಕಿಂತ ಅವರಿಗೆ ಬೋಧಿಸಲು ನನಗೆ ಆಸೆ. ನಾನು ಜ್ಞಾನಿಯಲ್ಲ, ತತ್ವವೇತ್ತನಲ್ಲ, ಮಹಾತ್ಮನೂ ಅಲ್ಲ, ನಾನು ದೀನ. ದೀನರನ್ನು ಪ್ರೀತಿಸುತ್ತೇನೆ. ಬಡತನ ಮತ್ತು ಮೌಢ್ಯದ ಪಾಶದಲ್ಲಿ ಎಂದೆಂದಿಗೂ ಸಿಕ್ಕಿಕೊಂಡು ನರಳುತ್ತಿರುವ ಇಪ್ಪತ್ತು ಕೋಟಿ ಭಾರತೀಯರಿಗೆ ಯಾರು ಸಹಾನುಭೂತಿ ತೋರುವರು? ದೀನರಿಗೆ ಯಾರ ಎದೆ ಕರಗುವುದೋ, ಅವನನ್ನು ಮಹಾತ್ಮನೆಂದು ಕರೆಯುತ್ತೇನೆ. ಅವರಿಗೆ ಕರುಣೆಯನ್ನು ತೋರು ವವರು ಯಾರು ಇರುವರು? ಅವರಿಗೆ ವಿದ್ಯೆಯಿಲ್ಲ, ಜ್ಞಾನದ ಬೆಳಕಿಲ್ಲ. ಅವರಿಗೆ ಜ್ಞಾನವನ್ನು ಯಾರು ತರುವರು? ಅವರಿಗೆ ಶಿಕ್ಷಣವನ್ನು ಕೊಡುವು ದಕ್ಕಾಗಿ ಯಾರು ಮನೆಯಿಂದ ಮನೆಗೆ ಹೋಗುವರು? ಆ ನಿರ್ಭಾಗ್ಯರು ನಿಮ್ಮ ದೇವರಾಗಲಿ. ಅವರನ್ನು ಕುರಿತು ಆಲೋಚಿಸಿ, ಅವರ ಮೇಲ್ಮೆಗಾಗಿ ಪ್ರಾರ್ಥಿಸಿ. ಭಗವಂತ ನಿಮಗೆ ದಾರಿ ತೋರುವನು.

