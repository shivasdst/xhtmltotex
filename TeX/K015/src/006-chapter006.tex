
\chapter{ಧಾರ್ಮಿಕ ವಿದ್ಯಾಭ್ಯಾಸ}

\section{ಮಹಾತ್ಮರ ಆರಾಧನೆ}

ಧರ್ಮವೇ ವಿದ್ಯಾಭ್ಯಾಸದ ತಿರುಳು. ಧರ್ಮ ಎಂದರೆ ನನ್ನ ಅಥವಾ ಬೇರೆಯವರ ಸ್ವಂತ ಅಭಿಪ್ರಾಯವಲ್ಲ. ಸತ್ಯವಾದ ಸನಾತನ ತತ್ತ್ವಗಳನ್ನು ಜನರ ಮುಂದೆ ಇಡಬೇಕು. ಮೊದಲನೆಯದಾಗಿ ಮಹಾತ್ಮರ ಪೂಜೆಯನ್ನು ಆಚರಣೆಗೆ ತರಬೇಕು. ಯಾರು ಸನಾತನ ತತ್ತ್ವಗಳನ್ನು ಸಾಕ್ಷಾತ್ಕಾರ ಮಾಡಿ ಕೊಂಡಿರುವರೋ ಅಂತಹ ಶ್ರೀರಾಮಚಂದ್ರ, ಶ್ರೀಕೃಷ್ಣ, ಹನುಮಾನ್, ಶ್ರೀರಾಮಕೃಷ್ಣರು ಮುಂತಾದ ಆದರ್ಶಶೀಲರನ್ನು ಜನರ ಮುಂದೆ ಇಡಬೇಕು. ಶ್ರೀಕೃಷ್ಣನ ಜೀವನದಲ್ಲಿ ಬೃಂದಾವನ ಲೀಲೆಯ ಭಾಗವನ್ನು ಸದ್ಯಕ್ಕೆ ಮರೆತು ಸಿಂಹವಾಣಿಯಿಂದ ಗೀತೆಯನ್ನು ಬೋಧಿಸುತ್ತಿರುವ ಶ್ರೀಕೃಷ್ಣನನ್ನು ದೇಶ ದಲ್ಲೆಲ್ಲ ಹರಡಿ. ಪ್ರತಿದಿನವೂ ಜಗನ್ಮಯಿ ಆದಿಶಕ್ತಿಯ ಪೂಜೆಯನ್ನು ಆಚರಣೆಗೆ ತನ್ನಿ. ನಮಗೆ ಇಂದು ಬೇಕಾಗಿರುವುದು, ಆಪಾದಮಸ್ತಕವೂ, ಪ್ರತಿಯೊಂದು ನಾಡಿಯಲ್ಲಿಯೂ, ಪ್ರಚಂಡ ರಜೋಗುಣ ತುಂಬಿ ತುಳು ಕಾಡುತ್ತಿರುವ ನಾಯಕನ ಆದರ್ಶ. ಸತ್ಯಸಾಕ್ಷಾತ್ಕಾರಕ್ಕೋಸುಗ ಸಾಧನೆಮಾಡಿ ಸಾಯಲು ಅನುವಾಗಿರುವ ನಾಯಕನು ಬೇಕು. ಯಾರ ಕವಚ ತ್ಯಾಗವೋ, ಯಾರ ಖಡ್ಗ ಜ್ಞಾನವೋ ಅಂತಹ ಧೀರ ನಾಯಕನು ಬೇಕು. ಯುದ್ಧರಂಗದ ವೀರಯೋಧನ ಕೆಚ್ಚು ನಮಗಿಂದು ಬೇಕಾಗಿರುವುದು.


\section{ಸೇವೆಯ ಆದರ್ಶ}

ಹನುಮಂತನ ಶೀಲವನ್ನು ನಿಮ್ಮ ಆದರ್ಶವನ್ನಾಗಿ ಮಾಡಿಕೊಳ್ಳಿ. ರಾಮ ಚಂದ್ರನ ಆಜ್ಞೆಯಂತೆ ಅವನು ಕಡಲನ್ನು ದಾಟಿದನು! ಅದ್ಭುತ ಮೇಧಾವಿ. ಇಂತಹ ಉತ್ತಮ ಸೇವೆಯ ಆದರ್ಶದ ಮೇಲೆ ನಿಮ್ಮ ಜೀವನವನ್ನು ಕಟ್ಟಿ. ಈ ಆದರ್ಶದ ಮೂಲಕ ಎಲ್ಲಾ ಭಾವನೆಗಳೂ ಕ್ರಮೇಣ ತಾವೇ ಜೀವನದಲ್ಲಿ ಮೂಡುವುವು. ಮರುಮಾತಿಲ್ಲದೆ ಗುರು ಆಣತಿಯನ್ನು ಪರಿಪಾಲಿಸುವುದು, ನೈಷ್ಠಿಕ ಬ್ರಹ್ಮಚಾರಿಯಾಗಿರುವುದು–ಇದೇ ಶ್ರೇಯಸ್ಸಿನ ರಹಸ್ಯ. ಒಂದು ಕಡೆ ಹನುಮಂತನು ಸೇವಾಧರ್ಮವನ್ನೇ ತೋರುತ್ತಾನೆ. ಮತ್ತೊಂದು ಕಡೆ ಪ್ರಪಂಚವನ್ನೇ ವಿಸ್ಮಯಗೊಳಿಸುವ ಅದ್ಭುತವಾದ ಧೈರ್ಯವನ್ನು ಬೀರು ತ್ತಾನೆ. ಶ್ರೀರಾಮಚಂದ್ರನ ಹಿತಕ್ಕೋಸುಗ ತನ್ನ ಪ್ರಾಣವನ್ನು ಕೂಡ ಅರ್ಪಿಸಲು ಸ್ವಲ್ಪವೂ ಶಂಕಿಸಲಿಲ್ಲ. ಶ್ರೀರಾಮರ ಸೇವೆಗಲ್ಲದೆ ಉಳಿದುದಕ್ಕೆಲ್ಲ ಅಸಮಾನವಾದ ಉಪೇಕ್ಷೆ. ಶ್ರೀರಾಮನ ಆಣತಿಯನ್ನು ಪಾಲಿಸುವುದೊಂದೇ ಅವನ ಜೀವನದ ವ್ರತ. ಅಂತಹ ಹೃತ್ಪೂರ್ವಕ ಭಕ್ತಿ ಬೇಕಾಗಿದೆ.


\section{ವೀರ ರಸೋದ್ದೀಪಕ ತಾರಸ್ವರವನ್ನು ಎಬ್ಬಿಸಿ}

ಈಗಿನ ಕಾಲಕ್ಕೆ ಶ್ರೀಕೃಷ್ಣಗೋಪಿಯರ ಭಾವಸಾಧನೆ ಹಿತಕರವಲ್ಲ. ಕೊಳಲು ಮುಂತಾದುವುಗಳನ್ನು ನುಡಿಸುವುದರಿಂದ ದೇಶಕ್ಕೆ ನವಚೇತನ ಬರಲಾರದು. ಮೃದಂಗ ಹೊಡೆಯುವುದು ಚಪ್ಪಾಳೆ ತಟ್ಟುವುದು ಕೀರ್ತನೆ ಗಳನ್ನು ಹಾಡಿ ಹುಚ್ಚುಹುಚ್ಚಾಗಿ ಕುಣಿದಾಡುವುದು, ಇವು ಇಡೀ ಜನಾಂಗ ವನ್ನೇ ಅಧೋಗತಿಗೆ ತಂದಿವೆ. ಯಾವ ಸಾಧನೆಗೆ ಮೊದಲು ಚಾರಿತ್ರಶುದ್ಧಿ ಅತ್ಯಾವಶ್ಯಕವೊ ಅಂತಹ ಅತ್ಯುತ್ತಮವಾದ ಸಾಧನೆಯನ್ನು ಒಂದೇ ಸಾರಿ ಅನುಸರಿಸುವುದಕ್ಕೆ ಹೋಗಿ ತಮಸ್ಸಿನ ಗುಂಡಿಯಲ್ಲಿ ಮುಳುಗಿ ಹೋಗಿರು ವರು. ದೇಶದಲ್ಲಿ ತಮಟೆಗಳು ತಯಾರಾಗುವುದಿಲ್ಲವೆ? ಭರತಖಂಡದಲ್ಲಿ ತುತ್ತೂರಿ ಕಹಳೆಗಳು ದೊರೆಯುವುದಿಲ್ಲವೆ? ನಿಮ್ಮ ಹುಡುಗರು ಇಂತಹ ವಾದ್ಯಗಳ ವೀರಧ್ವನಿಯನ್ನು ಕೇಳುವಂತೆ ಮಾಡಿ. ಬಾಲ್ಯದಿಂದಲೂ ಮೃದು ವಾದ ಸಂಗೀತವನ್ನು ಕೇಳಿ ಕೇಳಿ ದೇಶ ಹೆಂಗಸರ ರಾಜ್ಯವಾಗಿದೆ. ಡಮರು ಮತ್ತು ಕಹಳೆಯನ್ನು ಹೊಡೆಯಿರಿ. ವೀರರಸವನ್ನು ಉದ್ದೀಪನಗೊಳಿಸುವ ರಣಭೇರಿಯ ನಿನಾದವನ್ನು ಮಾಡಿ. ಮಹಾವೀರ! ಮಹಾವೀರ! ಹರ! ಹರ! ಓಂ! ಓಂ! ಎಂದು ಉಚ್ಚರಿಸುವ ಧ್ವನಿಯಿಂದ ದಿಕ್​ತಟಗಳು ಅನುರಣಿತವಾಗಬೇಕು. ಮನುಷ್ಯನ ಮೃದುವಾದ ಸ್ವಭಾವವನ್ನು ಮಾತ್ರ ಜಾಗ್ರತಗೊಳಿಸುವ ಸಂಗೀತವನ್ನು ಸದ್ಯಕ್ಕೆ ನಿಲ್ಲಿಸಬೇಕು. ದ್ರುಪದ್ ಸಂಗೀತ ಕೇಳುವುದನ್ನು ಜನರು ಅಭ್ಯಸಿಸಬೇಕು.

ಗಂಭೀರವಾದ ವೇದಶ್ಲೋಕಗಳ ಘನಗರ್ಜನೆಯಿಂದ ದೇಶಕ್ಕೆ ನವಚೇತನ ವನ್ನು ತುಂಬಬೇಕು. ಜೀವನದ ಪ್ರತಿಯೊಂದು ಕಾರ್ಯ ಕ್ಷೇತ್ರದಲ್ಲಿಯೂ ವೀರನಿಗೆ ಒಪ್ಪುವಂತಹ ಶೂರತ್ವವನ್ನು ಪುನರುಜ್ಜೀವನಗೊಳಿಸಬೇಕು. ಇಂತಹ ಆದರ್ಶವನ್ನು ಇಟ್ಟುಕೊಂಡು ನೀವು ಶೀಲವನ್ನು ರೂಢಿಸಿದರೆ, ನಂತರ ಸಾವಿರಾರು ಜನ ನಿಮ್ಮನ್ನು ಅನುಸರಿಸುವರು. ಆದರೆ ಜೋಪಾನ ವಾಗಿರಿ! ಆದರ್ಶದಿಂದ ಒಂದು ಅಂಗುಲವೂ ಹಿಂದೆ ಸರಿಯಬಾರದು. ಎಂದಿಗೂ ಎದೆಗೆಡಬೇಡಿ. ಊಟಮಾಡುವಾಗ ಬಟ್ಟೆ ಹಾಕಿಕೊಳ್ಳವಾಗ ಅಥವಾ ಹಾಡುವಾಗ ಅಥವಾ ಆಡುವಾಗ ಸುಖದಲ್ಲಿ ನಲಿಯುವಾಗ ಅಥವಾ ರೋಗದಿಂದ ನರಳುವಾಗ ಯಾವಾಗಲೂ ಅತ್ಯುತ್ತಮವಾದ ಧಾರ್ಮಿಕ ಶಕ್ತಿ ಯನ್ನು ವ್ಯಕ್ತಗೊಳಿಸಿ. ನಿರ್ಬಲತೆ ಎಂದಿಗೂ ನಿಮ್ಮ ಮನಸ್ಸನ್ನು ಅಪಹರಿಸಲು ಅವಕಾಶಕೊಡಬೇಡಿ. ಮಹಾವೀರನನ್ನು ಜ್ಞಾಪಿಸಿಕೊಳ್ಳಿ. ದೌರ್ಬಲ್ಯಗಳೆಲ್ಲ ಹೇಡಿತನಗಳೆಲ್ಲ ಕ್ಷಣಮಾತ್ರದಲ್ಲಿ ಮಾಯವಾಗುವುದನ್ನು ನೀವು ನೋಡುವಿರಿ.


\section{ನವ ಧರ್ಮ}

ಹಿಂದಿನ ಧರ್ಮ, ಯಾರಿಗೆ ದೇವರಲ್ಲಿ ನಂಬಿಕೆಯಿಲ್ಲವೊ, ಅವರನ್ನು ನಾಸ್ತಿಕರೆಂದು ಕರೆಯಿತು. ಯಾರಿಗೆ ತನ್ನಲ್ಲಿ ನಂಬಿಕೆಯಿಲ್ಲವೋ ಅವನು ನಾಸ್ತಿಕನೆಂದು ಹೊಸ ಧರ್ಮ ಸಾರುತ್ತದೆ. ಆದರೆ ಇದು ಸ್ವಾರ್ಥ ಶ್ರದ್ಧೆಯಲ್ಲ. ಎಲ್ಲರಲ್ಲಿಯೂ ನಂಬಿಕೆ; ಏಕೆಂದರೆ ನೀವೇ ಎಲ್ಲವೂ ಆಗಿರುವಿರಿ. ನಿಮ ಗೋಸುಗ ಪ್ರೀತಿ ಎಂದರೆ ಪ್ರಾಣಿಗಳಿಗೆ ಮತ್ತು ಇರುವ ಸಚರಾಚರ ವಸ್ತು ಗಳಿಗೂ ಪ್ರೀತಿ. ಏಕೆಂದರೆ ನೀವೆಲ್ಲ ಒಂದೆ. ಮಹಾಶ್ರದ್ಧೆ ಜಗತ್ತನ್ನು ಉತ್ತಮಗೊಳಿಸುವುದು. ಆತ್ಮಶ್ರದ್ಧೆಯ ಆದರ್ಶ ನಮಗೆಲ್ಲರಿಗೂ ಅತ್ಯಂತ ಪ್ರಯೋಜನಕಾರಿ. ಈ ಆತ್ಮಶ್ರದ್ಧೆಯನ್ನು ವಿಸ್ತಾರವಾಗಿ ಬೋಧಿಸಿದ್ದರೆ, ಇದನ್ನು ಆಚರಣೆಗೆ ತಂದಿದ್ದರೆ, ನಮ್ಮಲ್ಲಿರುವ ಬಹುಪಾಲು ದೋಷಗಳು ಕಷ್ಟಗಳು ಮಾಯವಾಗುತ್ತಿದ್ದವು ಎಂಬುದರಲ್ಲಿ ಸಂದೇಹವಿಲ್ಲ. ಮಾನವ ಜನಾಂಗದ ಇತಿಹಾಸದಲ್ಲೆಲ್ಲ ಪ್ರಖ್ಯಾತರಾದ ಮಹನೀಯರ ಮತ್ತು ಮಹಿಳೆ ಯರ ಜೀವನದಲ್ಲಿರುವ ಕ್ರಿಯೋತ್ತೇಜನ ಶಕ್ತಿಯಲ್ಲೆಲ್ಲ ಅತ್ಯಂತ ಪ್ರಬಲ ವಾದುದೇ ಆತ್ಮಶ್ರದ್ಧೆ. ತಾವು ಮಹಾವ್ಯಕ್ತಿಗಳಾಗಬೇಕು ಎಂಬ ಆಲೋಚನೆ ಯೊಂದಿಗೆ ಅವರು ಜನ್ಮತಾಳಿ ಮಹಾವ್ಯಕ್ತಿಗಳಾದರು.


\section{ಶಕ್ತಿ}

ಅನಂತ ಶಕ್ತಿಯೇ ಧರ್ಮ; ಶಕ್ತಿಯೇ ಪುಣ್ಯ; ಅಶಕ್ತಿಯೇ ಪಾಪ. ಎಲ್ಲಾ ಪಾಪಗಳನ್ನು ಎಲ್ಲಾ ದೋಷಗಳನ್ನು ಒಂದು ಮಾತಿನಲ್ಲಿ ಅಡಗಿಸಬಹುದು; ಅದೇ ದುರ್ಬಲತೆ. ಮಾಡುವ ಎಲ್ಲಾ ಪಾಪಕಾರ್ಯಗಳನ್ನೂ ಪ್ರೇರೇಪಿಸು ವುದೇ ದುರ್ಬಲತೆ. ಎಲ್ಲಾ ಸ್ವಾರ್ಥತೆಯ ಮೂಲ ದುರ್ಬಲತೆ. ಒಬ್ಬನು ಇನ್ನೊಬ್ಬನನ್ನು ಹಿಂಸಿಸುವಂತೆ ಮಾಡುವುದು ದುರ್ಬಲತೆ. ಎಲ್ಲರೂ ತಾವು ಯಾರು ಎಂಬುದನ್ನು ತಿಳಿದುಕೊಳ್ಳಲಿ. ಹಗಲು, ಇರುಳು ತಮ್ಮ ನೈಜಸ್ಥಿತಿ ಯಾದ “ಸೋಽಹಂ” ಎಂಬುದನ್ನು ಉಚ್ಚರಿಸಲಿ. ನಾನೇ ಅವನು ಎಂಬ ಶಕ್ತಿಯ ಭಾವನೆಯನ್ನು ತಾಯಿಯ ಎದೆಹಾಲಿನೊಂದಿಗೆ ಹೀರಲಿ. ಮೊದಲು ಇದನ್ನು ಕೇಳಬೇಕು. ಇದನ್ನು ಕುರಿತು ಆಲೋಚಿಸಬೇಕು. ಅಂತಹ ಆಲೋಚನೆ ಯಿಂದ ಜಗತ್ತು ಇದುವರೆವಿಗೂ ಕಾಣದ ಮಹಾಕಾರ್ಯಗಳು ಆಗುವುವು.


\section{ಸತ್ಯ}

ಧೈರ್ಯವಾಗಿ ಸತ್ಯವನ್ನು ಹೇಳಿ. ಸತ್ಯವೆಲ್ಲ ಸನಾತನ. ಇಲ್ಲಿ ಜೀವಿಗಳ ಸ್ವಭಾವವೇ ಸತ್ಯ. ಇಲ್ಲಿದೆ ಸತ್ಯದ ಪರೀಕ್ಷೆ. ಶಾರೀರಕವಾಗಲಿ, ಮಾನಸಿಕ ವಾಗಲಿ, ಆಧ್ಯಾತ್ಮಿಕವಾಗಲಿ ಯಾವುದು ನಿಮ್ಮನ್ನು ದುರ್ಬಲರನ್ನಾಗಿ ಮಾಡು ವುದೋ ಅದನ್ನು ವಿಷದಂತೆ ನಿರಾಕರಿಸಿ. ಅದರಲ್ಲಿ ಚೇತನವಿಲ್ಲ. ಅದು ಸತ್ಯವಾಗಲಾರದು. ಸತ್ಯ ಬಲಕಾರಿ, ಸತ್ಯ ಪವಿತ್ರವಾದುದು. ಸತ್ಯ ಜ್ಞಾನಮಯ. ಸತ್ಯ ಬಲಾಢ್ಯರನ್ನಾಗಿ ಮಾಡಬೇಕು, ಜ್ಞಾನಜ್ಯೋತಿಯನ್ನು ನೀಡಬೇಕು, ಉತ್ಸಾಹಪೂರಿತರನ್ನಾಗಿ ಮಾಡಬೇಕು. ದೇದೀಪ್ಯಮಾನವಾದ ಶಕ್ತಿಪೂರಿತ ವಾದ ಸ್ಪಷ್ಟವಾದ ಉಪನಿಷತ್ತಿನ ಸಿದ್ಧಾಂತಗಳಿಗೆ ಹೋಗಿ, ಆ ಸಿದ್ಧಾಂತ ಗಳನ್ನು ಸ್ವೀಕರಿಸಿ. ಮಹಾಸತ್ಯಗಳು ಪ್ರಪಂಚದಲ್ಲೆಲ್ಲ ಅತ್ಯಂತ ಸರಳ ವಾದುವು. ಉಪನಿಷತ್ತಿನ ಸತ್ಯಗಳು ನಿಮ್ಮ ಮುಂದೆ ಇವೆ. ಅವನ್ನು ಸ್ವೀಕರಿಸಿ. ಅವುಗಳಂತೆ ನಿಮ್ಮ ಬಾಳನ್ನು ಹದಗೊಳಿಸಿ. ಭರತಖಂಡದ ವಿಮೋಚನೆ ಶೀಘ್ರದಲ್ಲಿಯೇ ಆಗುವುದು.


\section{ಶಾರೀರಕ ದಾರ್ಢ್ಯ}

ಶಾರೀರಕ ದುರ್ಬಲತೆಯೇ ನಮ್ಮ ದುಃಖದ ಮೂರನೇ ಒಂದು ಪಾಲಿ ಗಾದರೂ ಕಾರಣ. ನಾವು ಸೋಮಾರಿಗಳು, ಒಗ್ಗಟ್ಟಾಗಲಾರೆವು. ಅರಗಿಳಿ ಯಂತೆ ನಾವು ಅನೇಕ ವಿಷಯಗಳನ್ನು ಮಾತನಾಡುವೆವು. ಆದರೆ ಅವುಗಳನ್ನು ಎಂದಿಗೂ ಮಾಡುವುದಿಲ್ಲ. ಕೆಲಸಮಾಡದೆ ಮಾತನಾಡುವುದು ನಮ್ಮ ಬಾಳ ಚಾಳಿ. ಇದಕ್ಕೆ ಕಾರಣವೇನು? ಶಾರೀರಕ ದುರ್ಬಲತೆ. ಇಂತಹ ದುರ್ಬಲವಾದ ಮೆದುಳು ಏನನ್ನೂ ಸಾಧಿಸಲಾರದು. ನಾವು ಅದನ್ನು ಪುಷ್ಟಿಗೊಳಿಸಬೇಕು. ಮೊದಲನೆಯದಾಗಿ ನಮ್ಮ ಯುವಕರು ಬಲಾಢ್ಯರಾಗಬೇಕು. ಧರ್ಮ ನಂತರ ಬರುವುದು. ನನ್ನ ತರುಣ ಸ್ನೇಹಿತರೆ, ಶಕ್ತಿಯುತರಾಗಿ; ಅದೇ ನಿಮಗೆ ನನ್ನ ಬುದ್ಧಿವಾದ. ಗೀತೆಯನ್ನು ಪಾರಾಯಣ ಮಾಡುವುದಕ್ಕಿಂತ ಚಂಡಾಟದ ಮೂಲಕ ನೀವು ಮೋಕ್ಷದ ಸಮೀಪಕ್ಕೆ ಹೋಗುವಿರಿ. ನಿಮ್ಮ ಬಾಹುವಿನ ಮಾಂಸಖಂಡಗಳು ಸ್ವಲ್ಪ ಬಲವಾದರೆ ನೀವು ಗೀತೆಯನ್ನು ಚೆನ್ನಾಗಿ ತಿಳಿದು ಕೊಳ್ಳುವಿರಿ. ನಿಮ್ಮಲ್ಲಿ ಸ್ವಲ್ಪ ಬಿಸಿರಕ್ತ ಹರಿಯುತ್ತಿದ್ದರೆ, ಆಗ ಶ್ರೀಕೃಷ್ಣನ ಅದ್ಭುತ ಪ್ರತಿಭೆ ಮತ್ತು ಮಹಾಶಕ್ತಿಯನ್ನು ಚೆನ್ನಾಗಿ ಅರ್ಥ ಮಾಡಿಕೊಳ್ಳು ವಿರಿ. ನಿಮ್ಮ ದೇಹ ಕಾಲಮೇಲೆ ಸ್ಥಿರವಾಗಿ ನಿಂತುಕೊಂಡಾಗ, ನಾವು ಕೂಡ ಮನುಷ್ಯರು ಎಂಬ ಭಾವನೆ ನಿಮಗೆ ಬಂದಾಗ ಉಪನಿಷತ್ತುಗಳನ್ನು, ಆತ್ಮನ ಮಹಿಮೆಯನ್ನು ಚೆನ್ನಾಗಿ ಅರಿತುಕೊಳ್ಳುವಿರಿ.


\section{ಧೈರ್ಯ}

ಉಪನಿಷತ್ತಿನ ಪ್ರತಿಶ್ಲೋಕವೂ ಶಕ್ತಿಯನ್ನು ಸಾರುತ್ತಿದೆ. ಪ್ರಪಂಚದಲ್ಲಿ ಇದೊಂದೇ ಸಾಹಿತ್ಯ “ಅಭಿಃ” ಎಂಬ ಪದವನ್ನು ಪುನಃ ಪುನಃ ಉಪಯೋಗಿಸು ವುದು. ಪ್ರಪಂಚದಲ್ಲಿ ಮತ್ತಾವ ಧರ್ಮ ಗ್ರಂಥಗಳಲ್ಲಿಯೂ ಮಾನವರಿಗೆ ಆಗಲಿ, ದೇವತೆಗಳಿಗಾಗಲಿ ಈ ಗುಣವಾಚಕವನ್ನು ಉಪಯೋಗಿಸುವುದಿಲ್ಲ. ಗತಕಾಲದ ಪ್ರಖ್ಯಾತನಾದ ಪಾಶ್ಚಾತ್ಯ ದೇಶಗಳ ಚಕ್ರವರ್ತಿಯಾದ ಅಲೆಕ್ಸಾಂಡರನ ಚಿತ್ರ ನನ್ನ ಮನಸ್ಸಿಗೆ ಬರುವುದು. ಸಿಂಧೂನದಿಯ ತೀರದ ಅರಣ್ಯದಲ್ಲಿ ವಾಸವಾಗಿದ್ದ ಒಬ್ಬ ಸಂನ್ಯಾಸಿಯೊಂದಿಗೆ ಚಕ್ರವರ್ತಿ ಮಾತ ನಾಡುತ್ತಿರುವನು. ಅವನು ಯಾರೊಂದಿಗೆ ಮಾತನಾಡುತ್ತಿದ್ದನೊ ಆ ವೃದ್ಧ ಬಹುಶಃ ದಿಗಂಬರನಾಗಿ ಒಂದು ಬಂಡೆಯಮೇಲೆ ಕುಳಿತುಕೊಂಡಿರಬಹುದು. ಅವನ ಜ್ಞಾನಕ್ಕೆ ಚಕ್ರವರ್ತಿ ವಿಸ್ಮಿತನಾಗಿ, ಅವನಿಗೆ ಗ್ರೀಸ್ ದೇಶಕ್ಕೆ ಬಂದು ಬಿಡೆಂದು ಕೀರ್ತಿ ಮತ್ತು ದ್ರವ್ಯಗಳ ಆಸೆಯನ್ನು ತೋರುತ್ತಾನೆ. ಈ ಸಂನ್ಯಾಸಿ ಅವನ ದ್ರವ್ಯವನ್ನು ನೋಡಿ ನಗುತ್ತಾನೆ; ಅವನು ತೋರುವ ಆಸೆಗಳನ್ನು ನೋಡಿ ನಗುತ್ತಾನೆ. ಅವನ ಬೇಡಿಕೆಯನ್ನು ನಿರಾಕರಿಸುತ್ತಾನೆ. ನಂತರ ಚಕ್ರವರ್ತಿ ತನ್ನ ಅಧಿಕಾರದ ದರ್ಪದ ಮೇಲೆ ನಿಂತು, “ನೀನು ಬರದೇ ಇದ್ದರೆ ನಿನ್ನನ್ನು ಕೊಲ್ಲುತ್ತೇನೆ” ಎನ್ನುವನು. ಸಂನ್ಯಾಸಿ ನಕ್ಕು ಹೇಳುವನು: “ನೀನು ಈಗ ಹೇಳಿದಂತಹ ಸುಳ್ಳನ್ನು ನಿನ್ನ ಜೀವನದಲ್ಲಿ ಎಂದೂ ಹೇಳಲಿಲ್ಲ. ಯಾರು ನನ್ನನ್ನು ಕೊಲ್ಲಬಲ್ಲರು!ಎಂದಿಗೂ ಹುಟ್ಟದ ಎಂದಿಗೂ ನಾಶವಾಗದ ಆತ್ಮ ನಾನು.” ಇದು ಶಕ್ತಿ.


\section{ಶಕ್ತಿಯ ಗಣಿ ಉಪನಿಷತ್ತು}

ನಮ್ಮನ್ನು ನಿರ್ಬಲರನ್ನಾಗಿ ಮಾಡುವುದಕ್ಕೆ ಸಾವಿರಾರು ಜನರು ಇರುವರು, ಕಟ್ಟುಕಥೆಗಳು ಲೆಕ್ಕವಿಲ್ಲದಷ್ಟು ಇವೆ. ಆದಕಾರಣ ನನ್ನ ಸ್ನೇಹಿತರೇ, ನಿಮ್ಮ ಸಹೋದರನಂತೆ, ನಿಮ್ಮೊಂದಿಗೆ ಹುಟ್ಟಿ ಸಾಯುವವನಂತೆ ನಿಮಗೆ ಹೇಳು ತ್ತೇನೆ: ನಮಗೆ ಶಕ್ತಿ ಬೇಕು, ಶಕ್ತಿ ಬೇಕು, ಯಾವಾಗಲೂ ಶಕ್ತಿ ಬೇಕು. ಉಪನಿಷತ್ತು ಶಕ್ತಿಯ ಒಂದು ಮಹಾಗಣಿ. ಜಗತ್ತಿಗೆಲ್ಲ ನವಚೇತನವನ್ನು ತುಂಬುವಷ್ಟು ಶಕ್ತಿ ಅಲ್ಲಿದೆ. ಅದರಿಂದ ಪ್ರಪಂಚವನ್ನೇ ಜಾಗ್ರತಗೊಳಿಸ ಬಹುದು; ಶಕ್ತಿಯಿಂದ ಪೋಷಿಸಬಹುದು. ಎಲ್ಲಾ ಜನಾಂಗದ ಎಲ್ಲಾ ಜಾತಿಯ ಎಲ್ಲಾ ಧರ್ಮದಲ್ಲಿರುವ ದುರ್ಬಲರಿಗೆ ದುಃಖಿಗಳಿಗೆ ದಬ್ಬಾಳಿಕೆಗೆ ತುತ್ತಾದವರಿಗೆ ತೂರ್ಯವಾಣಿಯಿಂದ “ಸ್ವತಂತ್ರವಾಗಿರಿ, ನಿಮ್ಮ ಕಾಲಮೇಲೆ ನೀವು ನಿಲ್ಲಿ” ಎಂದು ಸಾರುವುದು. ಶಾರೀರಕ ಸ್ವಾತಂತ್ರ್ಯ ಮಾನಸಿಕ ಸ್ವಾತಂತ್ರ್ಯ ಆಧ್ಯಾತ್ಮಿಕ ಸ್ವಾತಂತ್ರ್ಯ ಇವೇ ಉಪನಿಷತ್ತಿನ ಅತ್ಯಮೋಘ ಸಂದೇಶ.


\section{ಸಾಕ್ಷಾತ್ಕಾರವೇ ಧರ್ಮ}

ಯಾವ ಶಾಸ್ತ್ರಗಳೂ ಕೂಡ ನಮ್ಮನ್ನು ಧಾರ್ಮಿಕರನ್ನಾಗಿ ಮಾಡಲಾರವು. ಪ್ರಪಂಚದಲ್ಲಿರುವ ಪುಸ್ತಕವನ್ನೆಲ್ಲ ನಾವು ಓದಬಹುದು. ಆದರೂ ಧರ್ಮಕ್ಕೆ ಅಥವಾ ದೇವರಿಗೆ ಸಂಬಂಧಪಟ್ಟ ಒಂದು ಪದವನ್ನಾದರೂ ನಾವು ಅರ್ಥ ಮಾಡಿಕೊಳ್ಳದೆ ಇರಬಹುದು. ನಾವು ಬದುಕಿರುವತನಕ ಮಾತನಾಡಬಹುದು; ವಿಚಾರಮಾಡಬಹುದು. ಆದರೆ ನಾವೇ ಪ್ರತ್ಯಕ್ಷ ಅನುಭವಿಸುವವರಗೆ ಸತ್ಯದ ಒಂದು ನುಡಿಯು ಕೂಡಾ ನಮಗೆ ಗೊತ್ತಾಗುವುದಿಲ್ಲ. ಒಬ್ಬನಿಗೆ ಕೆಲವು ಶಸ್ತ್ರಚಿಕಿಸ್ಸೆಯ ಪುಸ್ತಕಗಳನ್ನು ಕೊಟ್ಟ ಮಾತ್ರಕ್ಕೆ ಅವನನ್ನು ಶಸ್ತ್ರಚಿಕಿತ್ಸಕನನ್ನಾಗಿ ಮಾಡಲಾರಿರಿ. ಒಂದು ದೇಶವನ್ನು ನೋಡಬೇಕೆಂದು ನನಗೆ ಇರುವ ಕುತೂ ಹಲವನ್ನು ಒಂದು ನಕ್ಷೆಯನ್ನು ತೋರಿ ತೃಪ್ತಿಪಡಿಸಲಾರಿರಿ. ಇನ್ನೂ ಹೆಚ್ಚು ವಿಷಯಗಳನ್ನು ತಿಳಿದುಕೊಳ್ಳಬೇಕು ಎನ್ನುವ ಕುತೂಹಲವನ್ನು ಮಾತ್ರ ನಕ್ಷೆ ಹೆಚ್ಚಿಸಬಲ್ಲದು.ಇದಕ್ಕಿಂತ ಹೆಚ್ಚು ಅದರಿಂದ ಪ್ರಯೋಜನವಿಲ್ಲ. ದೇವಸ್ಥಾನ ಮಸೀದಿ ಶಾಸ್ತ್ರ ಮತ್ತು ವಿಗ್ರಹ ಇವು ಧಾರ್ಮಿಕ ಶಿಶುವಿಹಾರಗಳು ಮಾತ್ರ. ಮೇಲಿನ ಮೆಟ್ಟಲನ್ನು ಹತ್ತುವುದಕ್ಕೆ ಮಕ್ಕಳನ್ನು ಬಲಶಾಲಿಗಳನ್ನಾಗಿ ಮಾಡು ವುದು ಮಾತ್ರ ಇವುಗಳ ಕೆಲಸ. ಧರ್ಮ ಸಿದ್ಧಾಂತದಲ್ಲಿ ಇಲ್ಲ, ಮೂಢ ನಂಬಿಕೆಯಲ್ಲಿ ಇಲ್ಲ, ಅಥವಾ ಒಣಪಾಂಡಿತ್ಯದ ವಾದದಲ್ಲಿಯೂ ಇಲ್ಲ. ಧಾರ್ಮಿಕನಾಗಿರುವುದು ಮತ್ತು ಆಗಲು ಪ್ರಯತ್ನಿಸುವುದೇ ಧರ್ಮ. ಸತ್ಯ ಸಾಕ್ಷಾತ್ಕಾರವೇ ಧರ್ಮ.


\section{ಭಾವವನ್ನು ಪೋಷಿಸಿ}

ಪ್ರಪಂಚ ಇದುವರೆವಿಗೂ ಕಾಣದ ದೊಡ್ಡ ವಿದ್ಯಾವಂತರು ನಾವು ಆಗಿರ ಬಹುದು. ಆದರೂ ನಾವು ದೇವರ ಸಮೀಪಕ್ಕೆ ಬರದೇ ಇರಬಹುದು. ಇದಕ್ಕೆ ವಿರೋಧವಾಗಿ ಬುದ್ಧಿ ಹೆಚ್ಚಿದಷ್ಟೂ ಅಧರ್ಮಿಗಳಾಗಿರುವರು. ಪಾಶ್ಚಾತ್ಯ ಸಂಸ್ಕೃತಿಯ ಕುಂದುಗಳಲ್ಲಿ ಇದೊಂದು. ಭಾವವನ್ನು ಪೋಷಿಸದೆ ಕೇವಲ ಬುದ್ಧಿಯ ಶಿಕ್ಷಣದ ಕಡೆಗೆ ಗಮನ ಕೊಡುವುದು, ಮನುಷ್ಯರನ್ನು ಹತ್ತುಪಾಲು ಸ್ವಾರ್ಥರನ್ನಾಗಿ ಮಾತ್ರ ಮಾಡುವುದು. ಬುದ್ಧಿಗೂ ಭಾವಕ್ಕೂ ಸಂಘರ್ಷಣೆ ನಡೆದಾಗ, ಭಾವವನ್ನು ಅನುಸರಿಸಿ. ಬುದ್ಧಿಯ ಕೈಗೆ ನಿಲುಕದ ಪರಮ ಗುರಿಯೆಡೆಗೆ ಕರೆದೊಯ್ಯುವುದು ಭಾವ. ಅದೇ ಬುದ್ಧಿಯನ್ನು ಮೀರಿದ ಸ್ಫೂರ್ತಿಯನ್ನು ನೀಡುವುದು. ಯಾವಾಗಲೂ ಭಾವವನ್ನು ಪೋಷಿಸಿ ಭಾವದ ಮೂಲಕ ದೇವರು ಮಾತನಾಡುವನು.


\section{ಮತಭ್ರಾಂತಿ ಒಂದು ರೋಗ}

ಮಾನವನಿಗೆ ಪರಿಚಿತವಾದ ಅತ್ಯುತ್ತಮ ಪ್ರೀತಿ ಧರ್ಮದಿಂದ ಬಂದಿದೆ. ಪ್ರಪಂಚ ಕೇಳಿದ ಶಾಂತಿಮಯ ಮಹೋಪದೇಶ ಧಾರ್ಮಿಕ ಪ್ರಪಂಚ ದಲ್ಲಿರುವ ಮಹಾವ್ಯಕ್ತಿಗಳಿಂದ ಬಂದಿದೆ. ಅದೇ ಕಾಲದಲ್ಲಿ ಪ್ರಪಂಚಕ್ಕೆ ಗೊತ್ತಿರುವ ಅತ್ಯಂತ ಹೀನಾಯವಾದ ದೋಷಾರೋಪಣೆ ಧಾರ್ಮಿಕ ವ್ಯಕ್ತಿ ಗಳಿಂದ ಬಂದಿದೆ. ಪ್ರತಿಯೊಂದು ಧರ್ಮವೂ ಕೂಡಾ ತನ್ನ ಸಿದ್ಧಾಂತಗಳನ್ನು ತಂದು ಇವು ಮಾತ್ರ ಸತ್ಯ ಎಂದು ಒತ್ತಾಯಪಡಿಸುತ್ತದೆ. ಈ ಧರ್ಮಾನು ಯಾಯಿಗಳು ತಮ್ಮಂತೆ ಇತರರೂ ಕೂಡ ನಂಬಬೇಕೆಂದು ಅವರನ್ನು ಬಲಾತ್ಕಾರಪಡಿಸುವುದಕ್ಕೆ ತಮ್ಮ ಕತ್ತಿಯನ್ನೂ ಕೂಡಾ ಒರೆಯಿಂದ ಸೆಳೆಯು ತ್ತಾರೆ. ಇದು ಧರ್ಮದಿಂದ ಆದುದಲ್ಲ. ಇದು ಮತಭ್ರಾಂತಿ ಎಂದು ಕರೆಯಲ್ಪಡುವ ಮಾನವನ ಮನಸ್ಸಿನಲ್ಲಿರುವ ಒಂದು ಜಾಡ್ಯ. ಆದರೂ ಕೂಡ ಇಂತಹ ಧರ್ಮಗಳ ಮತ್ತು ಜಾತಿಗಳ ಪರಸ್ಪರ ವೈಮನಸ್ಸು ಮತ್ತು ಹೋರಾಟಗಳಿಂದ, ದ್ವೇಷ ಅಸೂಯೆಗಳ ಗೊಂದಲದಿಂದ ಕಾಲಕಾಲಕ್ಕೆ ಶಾಂತಿ ಮತ್ತು ಸೌಹಾರ್ದವನ್ನು ಸೂಚಿಸುವ ಮಹಾವಾಣಿಗಳು ಮೂಡಿವೆ.


\section{ಶ್ರೀರಾಮಕೃಷ್ಣರು ಸಮನ್ವಯ ವಾಣಿಯ ಸಂದೇಶಕರು}

ಯಾರು ಪ್ರತಿಯೊಂದು ಕೋಮಿನಲ್ಲಿಯೂ ಒಂದೇ ಶಕ್ತಿ ಕೆಲಸಮಾಡು ತ್ತಿರುವುದನ್ನು ನೋಡಬಲ್ಲರೋ, ಒಂದೇ ದೇವರನ್ನು ನೋಡಬಲ್ಲರೋ, ಪ್ರತಿಯೊಂದು ಜೀವಿಯಲ್ಲಿಯೂ ಪರಮಾತ್ಮನನ್ನು ನೋಡಬಲ್ಲರೋ, ಯಾರ ಹೃದಯ ದೀನರಿಗೆ, ದುರ್ಬಲರಿಗೆ, ದಲಿತರಿಗೆ ಮರುಗುತ್ತಿದ್ದಿತೋ, ಮತ್ತು ಅದೇ ಕಾಲದಲ್ಲಿ ಯಾರ ಅದ್ಭುತ ಪ್ರಚಂಡ ಬುದ್ಧಿಶಕ್ತಿಯು ಭರತ ಖಂಡದಲ್ಲಿ ಅಲ್ಲದೆ ಹೊರಗೆ ಕೂಡ ಪರಸ್ಪರ ವಿರೋಧಿಸುವ ಪಂಗಡಗಳನ್ನು ಒಂದುಗೂಡಿಸಿ ಅತ್ಯದ್ಭುತವಾದ ಸಮನ್ವಯ ವಾಣಿಯನ್ನು ಸಾರಿ ವಿಶ್ವಧರ್ಮ ಅಸ್ತಿತ್ವಕ್ಕೆ ಬರುವಂತೆ ಮಾಡಿತೊ ಅಂತಹ ವ್ಯಕ್ತಿಯ ಜನನಕ್ಕೆ ಕಾಲ ಸನ್ನಿಹಿತ ವಾಗಿತ್ತು. ಅಂತಹ ಮಹಾತ್ಮನು ಜನ್ಮವೆತ್ತಿದನು. ಅವರ ಪಾದದಡಿಯಲ್ಲಿ ಕುಳಿತುಕೊಳ್ಳುವ ಭಾಗ್ಯ ನನ್ನದಾಗಿತ್ತು. ಪ್ರಪಂಚದಲ್ಲಿರುವ ಧರ್ಮಗಳು ಒಂದಕ್ಕೊಂದು ಪರಸ್ಪರ ಭಿನ್ನವಾದುವುಗಳಲ್ಲ ಅಥವಾ ವಿರುದ್ಧವಾದುವು ಗಳಲ್ಲ ಎಂಬ ಮಹಾಸತ್ಯವನ್ನು ನನ್ನ ಗುರುದೇವನಿಂದ ತಿಳಿದೆನು. ಅವುಗಳೆಲ್ಲ ಒಂದು ಸನಾತನ ಧರ್ಮದ ಹಲವು ಭಾಗಗಳು. ಶ್ರೀರಾಮಕೃಷ್ಣರು ಯಾರೊ ಬ್ಬರನ್ನು ವಿರೋಧಿಸಿ ಒಂದು ಕಠಿನವಾದ ಮಾತನ್ನು ಕೂಡ ಆಡಿದವರಲ್ಲ. ಪ್ರತಿಯೊಂದು ಪಂಗಡವರು ಕೂಡ ಅವರು ತಮ್ಮ ಪಂಗಡಕ್ಕೆ ಸೇರಿದವ ರೆಂದು ತಿಳಿದುಕೊಳ್ಳುವಷ್ಟು ಮಟ್ಟಿಗೆ ಉದಾರಹೃದಯರಾಗಿದ್ದರು. ಅವರು ಎಲ್ಲರನ್ನೂ ಪ್ರೀತಿಸುತ್ತಿದ್ದರು. ಅವರಿಗೆ ಎಲ್ಲಾ ಧರ್ಮಗಳೂ ಸತ್ಯವಾಗಿದ್ದವು. ಕೋಮುವಾರು ದ್ವೇಷಗಳನ್ನು ಮತ್ತು ಮೂಢನಂಬಿಕೆಗಳನ್ನು ಧ್ವಂಸಮಾಡು ವುದರಲ್ಲೇ ಅವರ ಜೀವನ ಕಳೆಯಿತು.


\section{ಸಹಿಷ್ಣುತೆಯಲ್ಲ, ಸ್ವೀಕಾರ}

ಬಹಿಷ್ಕಾರವಿಲ್ಲದೆ ಸ್ವೀಕಾರ ನಮ್ಮ ಮೂಲಮಂತ್ರವಾಗಲಿ, ಸಹಿಷ್ಣುತೆ ಮಾತ್ರವಲ್ಲ. ಅನೇಕ ವೇಳೆ ಸಹಿಷ್ಣುತೆ ಎಂಬುದು ಈಶ್ವರನಿಂದೆ. ಸಹಿಷ್ಣುತೆ ಎಂದರೆ ನೀನು ತಪ್ಪು ಎಂಬುದು ನನಗೆ ಗೊತ್ತಿದೆ. ಆದರೂ ನೀನು ಬದುಕಿರುವುದಕ್ಕೆ ಅವಕಾಶ ಕೊಟ್ಟಿರುವೆ. ಇದು ಈಶ್ವರ ನಿಂದೆಯಲ್ಲವೆ? ಹಿಂದೆ ಇದ್ದ ಧರ್ಮಗಳೆಲ್ಲವನ್ನೂ ಕೂಡಾ ನಾನು ಸ್ವೀಕರಿಸುವೆನು; ಮತ್ತು ಅವುಗಳನ್ನು ಪೂಜಿಸುವೆನು. ಅವರು ಯಾವ ರೀತಿಯಲ್ಲಾದರೂ ದೇವರನ್ನು ಪೂಜಿಸಲಿ, ನಾನು ಅವರೊಂದಿಗೆ ದೇವರನ್ನು ಪೂಜಿಸುವೆನು. ಕ್ರೈಸ್ತರ ಚರ್ಚಿನಲ್ಲಿ ಶಿಲುಬೆ ಎದುರಿಗೆ ಬಾಗುವೆನು. ಬೌದ್ಧರ ವಿಹಾರಗಳಿಗೆ ಹೋಗಿ ಅಲ್ಲಿ ಬುದ್ಧನ ಮತ್ತು ಆತನ ಧರ್ಮದಲ್ಲಿ ಶರಾಣಗತನಾಗುವೆನು. ಕಾಡಿಗೆ ಹೋಗಿ ಪ್ರತಿಯೊಂದು ಜೀವಿಯನ್ನು ಕೂಡಾ ವಿಮುಕ್ತನನ್ನಾಗಿ ಮಾಡುವ ಅಂತರಾತ್ಮವನ್ನು ನೋಡಲು ಪ್ರಯತ್ನಿಸುವ ಹಿಂದೂವಿನೊಂದಿಗೆ ಧ್ಯಾನಕ್ಕೆ ಕುಳಿತುಕೊಳ್ಳುವೆನು.


\section{ನಿರಂತರ ಆವಿಷ್ಕಾರ}

ನಾನು ಇವುಗಳನ್ನೆಲ್ಲ ಮಾಡುವುದೊಂದೆ ಅಲ್ಲ. ಮುಂದೆ ಬರಬಹುದಾದ ಎಲ್ಲಾ ಧರ್ಮಗಳ ಸ್ವೀಕಾರಕ್ಕೂ ನನ್ನ ಮನಸ್ಸನ್ನು ಅಣಿಮಾಡಿರುವೆನು. ಭಗವಂತನ ಕಾವ್ಯ ಮುಗಿಯಿತೇನು? ಅಥವಾ ಎಡೆಬಿಡದೆ ಇನ್ನೂ ಸಾಕ್ಷಾತ್ಕಾರ ಗಳು ಆಗುತ್ತಿವೆಯೊ? ಈ ಆಧ್ಯಾತ್ಮಿಕ ಆವಿಷ್ಕಾರಗಳು ಒಂದು ಅದ್ಭುತವಾದ ಗ್ರಂಥ. ಬೈಬಲ್ಲು, ವೇದ, ಖೊರಾನು ಮತ್ತು ಇನ್ನೂ ಇತರ ಧರ್ಮಗ್ರಂಥ ಗಳೆಲ್ಲ ಕೆಲವು ಪುಟಗಳು ಮಾತ್ರ. ಇನ್ನೂ ಮುಂದೆ ವ್ಯಕ್ತವಾಗುವುದಕ್ಕೆ ಅಗಣಿತ ಪುಟಗಳು ಇರುವುವು. ಹಿಂದೆ ಇದ್ದ ಎಲ್ಲವನ್ನೂ ಸ್ವೀಕರಿಸೋಣ. ಈಗ ಇರುವ ಬೆಳಕನ್ನು ಅನುಭವಿಸೋಣ. ಮುಂದೆ ಬರಲಿರುವ ಎಲ್ಲವನ್ನೂ ಸ್ವೀಕರಿಸುವುದಕ್ಕೆ ಮನದ ಬಾಗಿಲುಗಳೆಲ್ಲವನ್ನೂ ತೆರೆಯೋಣ. ಹಿಂದಿನ ದೇವದೂತರುಗಳಿಗೆ, ಈಗಿರುವ ಮತ್ತು ಮುಂದೆ ಬರಲಿರುವ ಮಹಾತ್ಮರು ಗಳಿಗೆಲ್ಲ ಅನಂತ ನಮಸ್ಕಾರ.

