
\chapter{Saṅgam Literature And Vaidika Mārga}\label{chapter6}

\Authorline{K. Vidyuta}


\section*{Abstract}

The Saṅgam age is the period in the history of ancient Tamil Nadu (the present Tamilnadu, Kerala, parts of Andhra Pradesh, parts of Karnataka and northern Sri Lanka) spanning from 4th C.B.C to 2nd C.A.D.

According to Tamil legends there were three Saṅgam periods, \textit{viz., Thalai Saṅgam, Iḍai Saṅgam} and \textit{Kaḍai Saṅgam}. However the historians refer to only the Third Saṅgam or \textit{Kaḍai Saṅgam} period as the “Saṅgam period.” Most of the available Saṅgam Literature are from third Saṅgam period. This collection contains 2381 poems in Tamil composed by 473 poets, some 102 of whom remain anonymous.

The famous Saṅgam Literature texts are \textit{Agattiyam, Tolkāppiyam, Patineṇmēlkaṇakku} consisting of \textit{Eṭṭuttokai} and \textit{Pattuppāṭṭu} and the \textit{Patineṇkīḻkaṇakku}. \textit{Agattiyam} (which is lost now) belongs to the First Saṅgam period and the \textit{Tolkāppiyam} is from the Middle Saṅgam period and the rest are from the Last Saṅgam period.

These texts reflect the ideas of the Vedas in them. The Saṅgam texts talk of the Vedas and its limbs (Vedāṅgas), the concept of \textit{Varṇāśrama dharma}, the practice of \textit{agnihotra} and others \textit{yāgas}, the concept of re-birth, religious ideas, the concept of \textit{karma} and also the concept of \textit{mokṣa}.

This paper will attempt to establish the fact that the Saṅgam literature contains many Vedic principles.


\section*{Introduction}

“\textit{Teṉmoḻi tēṉmoḻi, vaḍamoḻi vāḍāmoḻi}”– “the language of the South is honey; the language of the North never fades” is a famous saying. Both the languages – Sanskrit and Tamiḻ – are the basis of our Indian culture. It is well known that Tamiḻ is one of the longest surviving classical languages in the world. The earliest period of available Tamiḻ literature, the Saṅgam literature, is dated from 300 BCE – 300 CE. The earliest epigraphic records found on rock edicts date from around the 3rd century BCE.

The Saṅgam period marked the golden period of Tamiḻ literature, which refers to the prevalent Saṅgam legends claiming literary academies lasting thousands of years, giving the name to the corpus of literature. These poems were later collected into various anthologies and edited with colophons added by anthologists and annotators around 1000 CE.

Sanskrit is one of the most ancient languages of the world. As the language was made perfect and refined by the grammarians, it was named \textit{Samskṛta}. This language is considered to be the mother of all Indian languages, except Tamiḻ, by scholars of impartial views. But they do reflect the influence of Sanskrit since all these languages flourished together.

Tamiḻ and Sanskrit were considered as the two eyes of the same face and hence were studied with equal interest by our ancestors in Tamilnadu. In fact the Vedic culture, its rituals and practice were in vogue during the Saṅgam age and they got incorporated in the works of the poets of those times.

Here we shall discuss this in four sections – I. Saṅgam Literature,\\ II. \textit{Vaidika mārga}, III. \textit{Vaidika mārga} reflected in the Saṅgam literature and VI. Saṅgam culture in contemporary Tamilnadu.

\newpage

\subsection*{I Saṅgam Literature}

Tamilnadu housed the three Saṅgams during ancient times. They were the \textit{Talai Saṅgam, Iḍai Saṅgam} and \textit{Kaḍai Saṅgam}. During the \textit{Talai Saṅgam} period there existed many grammatical and literature works as per the information gleaned from the \textit{Tolkāppiyam}; but these are not available today. Even among the \textit{Iḍai Saṅgam} works only \textit{Tolkāppiyam}, a grammatical work, authored by Tolkāppiyar. The works of \textit{Kaḍai Saṅgam,} which are available now, are all literature works – \textit{Pattuppāṭṭu, Eṭṭuttogai} and \textit{Patiṉeṇkīḻkaṇakku}.

\subsubsection*{(1)\textit{ Tolkāppiyam}:}

\vskip -7pt

The author of this text is Tolkāppiyar. He is said to have belonged to no later than the 2nd Cent. B.C. This text contains about 1610 verses or \textit{sūtras}. The text is divided into three sections \textit{viz., Eḻuttadhigāram} with 480 verses, \textit{Colladhigāram} with 465 verses and \textit{Poruḷadhigāram} with the remaining 665 verses.

The varieties of letters, number, its quality, origin and the changes they undergo when combining with other letters, is widely discussed in the first chapter \textit{Eḻuttadhigāram}. The Sanskrit equivalent to this is the \textit{śikṣā śāstra}.

The various types of words, their formation, combination with other words to form compounds and then into sentences is explained in the \textit{Colladhigāram}. This is termed as \textit{Vyākaraṇa} in Sanskrit.

The method for writing poetry the usage of different \textit{rasas} like \textit{Śṛṅgāra, Karuṇa, Vīra}, etc., the usage of Figures of speech like similes and metaphors the usage of colloquial words and so on are expressed in the \textit{Poruḷadhigāram}. These fall under the class of \textit{Alaṅkāra śāstra} in Sanskrit literature.

Those works dealing predominantly with \textit{Śṛṅgāra rasa} were classified as “\textit{Agam}” and the rest as “\textit{Puṟam}”. \textit{Śṛṅgāra rasa} is said to be predominant only in plays involving \textit{Gāndharva} style of marriage. The “\textit{Agam}” section is further divided into five sub-sections termed \textit{“Tiṇai”: Kuruñji, Mullai, Marutham, Neytal} and \textit{ Pālai}.

If one peruses the \textit{Tolkāppiyam}, one might come to know that many such grammatical texts existed during those times but none other being available now is a great loss to Tamilnadu.


\subsubsection*{(2) \textit{Pattuppāṭṭu:}}

\vskip -7pt

\textit{Tirumurugāṟṟppaḍai, Porunarāṟṟuppaḍai, Ciṟupāṇāṟṟuppaḍai, Perumpāṇā\-ṟṟuppaḍai, Mullaipāṭṭu, Maduraikāñci, Neḍunalvāḍai, Kuruñjippāṭṭu, Paṭṭi\-ṉappālai} and \textit{Malaipaḍu-kaḍām} (also known as \textit{Kūttarāṟṟuppaḍai}) – are the \textit{Pattuppāṭṭu}.

Of the 10 songs, five are sung in the form of \textit{Āṟṟuppaḍai}. This is a form of poetry where a poet suggests another poet, who he meets on the way, to go to a certain nobleman and sing his poetry and to get rewarded. Of these the \textit{Tirumurugāṟuppaḍai} is an exception.

\textbf{(i) \textit{Tirumurugāṟṟppaḍai:}} Sung by Nakkīrar, this poem on the six temples of Lord Muruga, collectively called “\textit{Aṟupaḍaivīḍu}”, contains 317 lines.

\textbf{(ii) \textit{Porunarāṟṟuppaḍai:}} Praises king Karikāla Cōḻaṉ, for his charity, bravery, and his prosperous reign, in 248 lines.

\textbf{(iii) \textit{Ciṟupāṇāṟṟuppaḍai:}} Describes the capitals of the three Tamil monarchs and about the seven philanthropists – Bōgaṉ, Pāri, Kāri, Āy, Athigaṉ, Naḷḷi and Ōṟi, in 269 lines.

\textbf{(iv) \textit{Perumpāṇāṟṟuppaḍai:}} Explains the five \textit{tiṇai} regions of Kāñci city, its lifestyle and the trade in 500 lines.

\textbf{(v) \textit{Mullaipāṭṭu:}} Expresses the state of mind of a wife whose husband has gone to war in 103 lines.

\textbf{(vi) \textit{Maduraikāñc:}} In 782 lines, this work expounds the speciality of Madurai the five \textit{tiṇais} and the glory of the ancestors of king Neḍuñceḻiya Pāṇḍiyaṉ.

\textbf{(vii) \textit{Neḍunalvāḍai:}} In 188 lines, describes the long period of wait by the queen of Pāṇḍiyaṉ Neḍuñceḻiyaṉ who has gone for war and the setting in of the North-easterly winds.

\textbf{(viii) \textit{Kuruñjippāṭṭu:}} Comprising of 261 lines, this portrays the love between the hero and heroine culminating in a secret affair.

\textbf{(ix) \textit{Paṭṭiṉappālai:}} In 301 lines, this poem recounts the words of the hero, the beauty of Kāveripūmpaṭṭiṉam, Karikālperuvaḷattāṉ's bravery, his just rule, and so on.

\textbf{(x) \textit{Malaipaḍukaḍām:}} Containing 583 lines, this text also called as \textit{Kūttarāṟṟuppaḍai}, outlines the greatness of the philanthropist Naṉṉaṉ and the beauty of the forest, mountain and groves possessed by him.


\subsubsection*{(3) \textit{Eṭṭuttogai:}}

\vskip -7pt

The \textit{Eṭṭuttogai} texts are: \textit{Naṟṟiṇai, Kuṟunthogai, Aiṅkuṟunūṟu, Paripāḍal, Kalittogai, Aganāṉūṟu, Padiṟṟuppattu} and \textit{Puṟanāṉūṟu}. These texts are compilations of song sung by different poets over a period of time in the \textit{Agam} and \textit{Puṟam} genres. Among the eight texts, the last two are of \textit{Puṟam} genre while the rest are of \textit{Agam} type. Excepting \textit{Paripāḍal and Kalittogai}, the others are set in \textit{agavarpā} metre; \textit{Paripāḍal} is in \textit{paripāḍal} metre and \textit{Kalittogai} in \textit{kalippā} metre.


\subsubsection*{(4) \textit{Patiṉeṇkīḻkaṇakku:}}

\vskip -7pt

The eighteen poems collectively termed so are:

\textbf{(i) \textit{Tirukkuṟaḷ:}} This work contains 1330 couplets or \textit{kuṟaḷs}. It talks about the glory of rain, the greatness of recluses, the power of righteousness, the virtue of house-holders, the virtues of ascetics, the ways of the material world and so on.

\textbf{(ii) \textit{Nālaḍiyār:}} 400 songs in 4 lines, the work shares ideas similar to the \textit{Tirukkuṟaḷ,} dealing with the triple principles discussed there.

\textbf{(iii) \textit{Paḻamoḻi:}} With 400 verses, the text compares the words of Tiruvaḷḷuvar and attaches a simile to it and presents it as verses pronouncing morals.

\textbf{(iv) \textit{Tirikaḍugam:}} Each of the100 verses of this work mentions three ideas like the \textit{Tirikaḍugam} mixture (\textit{Sukku, tippili} and \textit{miḷagu).}

\textbf{(v) \textit{Nāṉmaṇikkaḍigai:}} Each verse of this text (100 verses) incorporates four ideas akin to a four-beaded necklace. They also reflect \textit{advaitic} concepts like the need for clarity of thought and the reason for \textit{mokṣa}.

\textbf{(vi) \textit{Ciṟupañcamūlam:}} Just as the concoction of the five roots cures the \textit{sthūlaśarīra}, the 105 verses of this work is a medicine to the \textit{sūkṣmaśarīra}.

\textbf{(vii) \textit{Elādi:}} In 81 verses, the text expresses six concepts similar to the six-ingredient mixture by the name \textit{Elādi} – \textit{ēlam, lavaṅgam, ciṟunāvaṟpu, miḷagu, tippili} and \textit{sukku}.

\textbf{(viii) \textit{Ācārakkovai:}} In 100 verses, it recommends people to not lie to elders, the king or to other people and also describes some Vedic doctrines.

\textbf{(ix) \textit{Mudumoḷikkāñci:}} Comprises of 100 proverbs which are mostly one-liners.

\textbf{(x) \textit{Kaḷavaḻināṟpathu:}} In 40 verse deals with the defeat of Kaṇaikkālrumporai at the hands of Cōḻa and Cēraṉ.

\textbf{(xi) \textit{Iṉitunāṟpathu:}} All the 40 verses of this text teach good values using the word “\textit{iṉitu}” in them.

\textbf{(xii) \textit{Iṉṉānāṟpathu:}} In 40 verses, using the word “\textit{iṉṉā}” in all the verses, it teaches what should be avoided in life.

\textbf{(xiii) \textit{Tiṇaimālainūṟṟaimpathu:}} In 150 verses, the five types of the \textit{agattiṇais} are described.

\textbf{(xiv) \textit{Aintiṇaieḻupathu:}} Dedicates 14 verses for each of the five \textit{agattiṇais}.

\textbf{(xv) \textit{Kainnilai:}} 12 verses each have been ascribed to each of the five \textit{tiṇais}.

\textbf{(xvi) \textit{Aintiṇaiyaimpathu:}} Containing 50 verses, this explains the \textit{agattiṇais}.

\textbf{(xvii) \textit{Tiṇaimoḻiaimpathu:}} It has attributed 10 verses each to the five \textit{tiṇais.}

\textbf{(xviii) \textit{Kārnāṟpathu:}} The rainy season and the separation of the husband and wife due to war are all recorded here in 40 verses.

Among these eighteen works, the first twelve mostly deal with \textit{Puṟattiṇai} and sparsely with \textit{Agattiṇai}. Whereas, the later six works are based on \textit{Agattiṇai} alone. Some of these 18 are debated to be of a later period.


\subsection*{II. Vaidika Mārga}

\vskip -7pt

The Vedas are basis for all the Sanskrit Literature. The Vedas not only contain philosophical and spiritual information but range widely from arts, sciences and so on. Therefore scholars declare that one can find information regarding anything in them, for it will have been mentioned in the Vedas.

\subsubsection*{(1) Veda-s:}

\vskip -7pt

Vedas are said to have been thought through oral tradition and therefore Tiruvaḷḷuvar calls it the \textit{eḻudā kiḷavi}. Vedas are four in number – \textit{Ṛgveda, Yajurveda, Sāmaveda} and \textit{Atharvaveda} is known by the words Murañjiyūr Muḍināgarāyar says: “\textit{Nālvēda neṟitiriyiṉum}” (\textit{Puṟa}., 2) and Neṭṭimaiyār says: “\textit{Nālvēdattu}” (\textit{Puṟa}., 15).

The \textit{Ṛgveda} is in verse form and each verse is called \textit{ṛk}; \textit{Yajurvedic mantras} are in prose form; the \textit{mantras} of \textit{Sāmaveda} are in lyrical form and so are sung and the \textit{mantras} of \textit{Atharvaveda} are mostly like that of the \textit{Ṛgveda}.

The part of the Vedas that prays to the deities is called \textit{Mantra} and the other part that explains these \textit{mantras} is called \textit{Brāhmaṇas}. Each Veda contains a compilation of \textit{mantras} propounded by various \textit{ṛṣis} and this section is called the \textit{Samhitās}.

The next part explaining the virtue of Vānaprastha is called the \textit{Āraṅyaka}. The final part revealing the means to attain \textit{mokṣa} through the teachings of a preceptor is the \textit{Upaniṣad}. The \textit{Brāhmaṇas} and \textit{Samhitās} are collectively called as \textit{Karmakāṇḍa} and the other two together are called as \textit{Jñānakāṇḍa}.

There are many Upaniṣads and among them ten are important. \textit{Īśāvāsya, Kaṭha} and \textit{Muṇḍaka} are rendered in verse form. Others are mostly in prose form. As the \textit{Īśāvāsyopaniṣad} forms the end part of the \textit{Śuklayajurveda Samhitā} it is also known as \textit{Mantropaniṣad}.


\subsubsection*{(2) Vedāṅgas:}

\vskip -7pt

In order to understand the meaning of the \textit{mantras}, the grammar involved, etc., one needed separate texts to guide them. Thus, the Vedāṅgas were formed. They are six in number \textit{viz.,} – \textit{Śikṣā} (Phonetics), \textit{Vyākaraṇa} (Grammar), \textit{Chandas} (Prosody), \textit{Nirukta} (Etymology), \textit{Jyotiṣa} (Astronomy, Astrology) and \textit{Kalpasūtras} (Ritual instructions).

\paragraph*{(i) \textit{Śikṣā:}}

\vskip -7.3pt

As the Vedas were taught orally and pronunciation played a major role, the text on phonetics, phonology or pronunciation was needed. It was named \textit{Śikṣā}. \textit{Prātiśākhyā} is the other name for Vaidika \textit{Śikṣā}. For each \textit{śākhā} of the Vedas, there must have been one \textit{Śikṣā} as know from the term \textit{Prātiśākhyā} but only some of them are available at present.

\textit{Śākhā} means a branch. Each Veda has been practised by various \textit{ṛṣis} and their family. Only after such \textit{adhyayana} the \textit{mantras} have been compiled together. These have been classified in various \textit{śākhās} depending on the difference in the pronunciation and the compilation. Patañjali’s \textit{Mahābhāṣya} and similar texts record that the \textit{Ṛgveda} had 21 \textit{śākhās, Yajurveda} had 101 \textit{śākhās, Sāmaveda} had 1000 \textit{śākhās} and \textit{Atharvaveda} had 9 \textit{śākhās}. Unfortunately, only some of them exist today.


\paragraph*{(ii) \textit{Vyākaraṇa:}}

\vskip -7.3pt

The study of the words thus pronounced, which contained suffixes and prefixes, was necessary. Thus \textit{Vyākaraṇa} was formed. According to our ancestors there once existed eight \textit{Vyākaraṇa} texts as Pāṇiṇī’s has been considered as the ninth and it seems the other texts perished with the advent of Pāṇiṇī’s work. From the \textit{Yajurveda} we come to know that a grammar text written by Indra existed and from the \textit{Mahābhāṣya} it is evident that Bṛhaspati had written a text called \textit{Saptapārāyaṇam}.

The Vedic language differs vastly from the language of Pāṇiṇīyan times. The former is called \textit{Vaidika} and the later as \textit{Bhāṣā} by Pāṇiṇī and \textit{Laukikam} by Mahābhāṣyakāra. The grammatical work \textit{Aṣṭādhyāyī} of Pāṇiṇī has been written to suit both the types of language.


\paragraph*{(iii) \textit{Chandas:}}

\vskip -7.3pt

Since the \textit{Samhitā-s} of the Vedas were in verse form, a study of the various rules involved in writing poetry was required so \textit{Chandas śāstra} was written. \textit{Taittiriya Samhitā, Sāṅkhyāyana Śrautasūtra, Sāmavedanidānasūtra} and \textit{Kātyāyana Sarvānugrahamaṇi} deal sparsely with the \textit{Chandas} of the Vedas. The ancient texts that discussed the \textit{Chandas} in detail are not available now. Piṅgala’s \textit{Chandas śāstra} which enlists the rules for poetry writing is also considered as a Vedāṅga text.


\paragraph*{(iv) \textit{Nirukta:}}

\vskip -7.3pt

Unless the real import of the \textit{mantra-s} was understood, one will not be able to experience and render them whole-heartedly. The text that helps in discerning the meaning of the Vedic verses is the \textit{Nirukta}.

Yāska’s \textit{Nirukta} is the only available text now, though the existence of some other texts is known from this. Vedic \textit{mantras} had varied readings during early times and so were deciphered differently and some of those readings could not be rightly deciphered even from Yāska’s work.


\paragraph*{(v) \textit{Jyotiṣa:}}

\vskip -7.3pt

The \textit{karma-s} mentioned in the Vedas are bound by a time-frame. To keep up with time, astronomical texts were needed. The science that deals with it is known as \textit{Jyotiṣa}. Only some of the \textit{Jyotiṣa} texts are accessible now.


\paragraph*{(vi) \textit{Kalpa:}}

\vskip -7.3pt

The methodology of performing the \textit{yajña-s}\index{yajna@\textit{yajña}} and other Vedic rituals have been listed in the texts called \textit{Kalpasūtra-s}. They are divided into four – \textit{Śrauta} dealing with Vedic rituals\textit{, Śulba} pertains to the mathematics involved in building \textit{citis} or altars\textit{, Gṛhya} discusses the rules for domestic rituals and \textit{Dharmasūtra} outlines the customs, duties and laws for living. Based on the \textit{Dharmasūtra-s, Smṛti} texts like the \textit{Manusmṛti} were written.


\subsubsection*{(3) \textit{Varṇa-s} and \textit{Āśramadharma:}}

\vskip -7pt

The people were divided into four varņa-s as: Brāhmaṇa-s, Kṣatriya-s, Vaiśya-s and Śūdra-s. The four \textit{āśrama-s} prescribed for the people are \textit{brahmacaryā, gṛhastha, vānaprastha} and \textit{sannyāsa}.


\subsubsection*{(4) Occupation of the people:}

\vskip -7pt

The \textbf{six} activities prescribed for a Brāhmaṇa are to learn the Vedas, teach it to others, perform \textit{yajña}, perform rituals for others, give and accept charity. Kṣatriya-s are ordained \textbf{five} activities – learning the Vedas, performing \textit{yajña-s}, giving charity, protecting the citizens and punishing wrong-doers. Vaiśya-s also have to learn the Veda-s, perform \textit{yajña}, give charity, protect the cattle, involve in agriculture and trade. A Śūdra must engage himself in agriculture, carpentry, sculpting and dancing.

The \textit{Upanayana} is compulsory for the first three \textit{varṇa-s} and so they are called \textit{dvijāḥ} (or) twice-born. They have to recite the \textit{mantra} thrice, everyday, as prescribed in the scriptures. By such practice the Brāhmaṇas attain greatness of mind, the Kṣatriyas gain strength of body and the Vaiśyas procure immense wealth; for only through such gains can they help others to prosper.


\subsubsection*{(5) Rebirth:}

\vskip -7pt

Yama is the one who separates the \textit{jīva} from the body at the time of one’s death. The means for \textit{jīva} to reach the \textit{pitṛloka} is provided by the last rites performed by one’s progeny. Moreover, the food needed for the \textit{pitṛs} in \textit{pitṛloka} is the \textit{piṇḍa} offered during \textit{śrāddha} (last rites).

When a person performs a \textit{homa} to please a deity and makes an offering into the fire, Lord Agni passes it on to the concerned deity. The gods convey their acceptance of the peoples’ prayers in the form of rain.


\subsubsection*{(6) Deities of Worship:}

\vskip -7pt

The deities mentioned in the Vedas and discussed in detail by the \textit{Rāmāyaṇa}, the \textit{Mahābhārata} and the Purāṇas are Brahmā, Viṣṇu, Śiva, Durgā, Balarāma and Skanda.


\subsubsection*{(7) \textit{Karma} and \textit{Mokṣa:}}

\vskip -7pt

The pleasure and pain that one experiences in this life depends upon the amount of good and bad \textit{karmas} performed by him in his previous lives. Therefore, if one has to live a happy life later, he has to perform good \textit{karmas} in this birth. The greatest enemy for realising God is one’s desire. So, one must strive to overcome desire, for when desire perishes one is elevated to the state of \textit{mokṣa}.


\subsection*{III. Vaidika Mārga Reflected In Saṅgam Literature}

\vskip -7pt

This section will pinpoint the references to Vedic practices and principles that are imbibed in the Saṅgam literature texts as follows:

\subsubsection*{(1) Vedas and Vedāṅgas:}

\vskip -7pt

Nakkīrar in his \textit{Tirumurugāṟṟppaḍai} (vv.~179–82) records that the twice-born (\textit{irupiṟap pāḷar}) learnt the Vedas comprising of six \textit{aṅgas} in 48 years:

\begin{quote}
\textit{“aṟunāḷ kiraṭṭi yiḷamai nalliyāṇḍu}\\\textit{āṟiṉiṟ kaḻippiya vaṟaṉavil koḷkai}\\\textit{mūṉṟuvagaik kuṟitta muttīc celvattu}\\\textit{irupiṟap pāḷar moḻutaṟintu nuvala}.”
\end{quote}

Only after the \textit{upanayana} is one fit to learn the Vedas. The Dharmaśāstras say that the time limit for learning one Veda is twelve years\endnote{\textit{Cf. Gautama Dharmasūtra}, I. 1. 6, 9; I. 3. 51-2.}.

Mūlaṅkiḻār in the 166th verse of the \textit{Puṟanāṉūṟu} registers the fact that there are four Vedas, 6 \textit{aṅgas} and that they originated from Lord Śiva:

\begin{quote}
\textit{“naṉṟāynta nīṇimircaḍai}\\\textit{mutumudalvaṉ vāi pōkādu}\\\textit{oṉṟu purinta vīriraṇḍiṉ}\\\textit{āṟuṇarnta vorumutunūl}.”
\end{quote}

Kaḍuvaṉiḷaveyiṉaṉār’s \textit{Paripāḍal} states that, “the 21 worlds and the people therein are protected by Lord Viṣṇu as per the \textit{māyāvāimoḻi}”. In the adage \textit{Māyāvāmoḻi}, the prefix \textit{māyā} refers to the eternity of the Vedas.

Pāṇḍiyaṉ Eḻuthi Neḍuṅkaṇṇaṉār by the term “\textit{eḻuthākkaṟpu}” in his work the \textit{Kuṟunthogai} refers to the unwritten Vedas. The same view is shared by the texts \textit{Padiṟṟuppattu} (vv.~64, 70 and 74) and the \textit{Puṟanāṉūṟu} (v.~361).

The fact that \textit{antaṇars} (Brāhmaṇas) were the ones who learnt the Vedas and practised them is recorded by Māṅguḍi Maruthaṉār in his \textit{Maduraikāñci} (vv.~468–76). Kuṉṟambhūthaṉār in the \textit{Paripāḍal} (v.~9) states that, those who explained the Vedic \textit{mantras} in detail were termed as “\textit{vāimoḻi pulavar}”. Otalānthaiyār in the \textit{Aiṅkuṟunūṟu} (v.~387) establishes that the Vedas talk about Dharma – “\textit{aṟampuriyarumaṟai}”.

The parrots that inhabited the place where Brāhmaṇas lived, also recited the Vedas, says Uruttiraṅkaṇṇaṉār in the \textit{Perumpāṇāṟṟuppaḍai} (vv.~300–01) as:

\begin{quote}
\textit{“vaḷaivāyk kiḷḷai maṟaiviḷi payiṟṟum}\\\textit{ maṟaikāp pāḷaruṟaipati}.”
\end{quote}

The \textit{yajña-s} and \textit{yajñaśālā-s} mentioned in the Vedas were performed by kings like Mudukuḍumipperuvaḻuti Cōḻa and Karikāl Peruvaḷanttāṉ, as noted by the \textit{Puṟanāṉūṟu} (15. 17–20; 224. 8) as:

\begin{quote}
\textit{“naṟpaṉuva ṉālvēttu}\\\textit{aruñcīrttip peruṅkaṇṇuṟai}\\\textit{neiymmali yāvuti poṅkap paṉmāṇ}\\\textit{vīyāc ciṟappiṉ vēḷvi muṟṟi}.”
\end{quote}

\begin{quote}
\textit{“eruvai nukarcci yūpa neḍunthūṇ}\\\textit{vēda vēḷvit thoḻiṉmuḍit tadūvam}.”
\end{quote}


\subsubsection*{(2) \textit{Varṇas} and \textit{Āśramadharma:}}

\vskip -7pt

There existed four \textit{varṇas} and each had a number of occupations assigned to them. Tolkāppiyar has mentioned this in the \textit{Poruḷadhikāram} (v.~71–2, 78, 81):

\begin{quote}
\textit{“nūlē... antaṇark kuriya}”\\\textit{“paḍaiyum... arasark kuriya”}\\\textit{“vaisikaṉ peṟumē vāṇika vāḻkkai”}\\\textit{“vēḷāṇ māntark kuḻuthū ṇallatu}\\\textit{illeṉa moḻip piṟvakai nikaḻcci}.”
\end{quote}

The \textit{Gautama Dharmasūtra} and other Dharmasūtras also mention the various \textit{varṇas} and also the activities assigned to them. The \textit{Rāmāyaṇa} (I. 6. 13) and the \textit{Mahābhārata} (XIV. 102. 81) how much ever possible a Brāhmaṇa must curtail himself from accepting charity.

\textit{“Satyam vada}” – is a Vedic rule and this has become a mandatory rule for the Brāhmaṇas as is evident from the words of Māṅguḍi Marutaṉār in his \textit{Maduraikāñci} – “\textit{Poiyyaṟiyā naṉmāntar}”. Brāhmaṇas always followed Dharma is denoted by the words: “\textit{aṟampuri antaṇar}” in the \textit{Padiṟṟuppattu} (v.~24). Tiruvaḷḷuvar calls them “\textit{aṟu toḻilōr}” as they observe six activities. Moreover, the \textit{Nāṉmaṇikkaḍigai} (v.~33) says that, to be born as a Brāhmaṇa was a great boon: “\textit{antaṇari nalla piṟappillai yeṉceyiṉun}...”

\paragraph*{(i) Brahmacārin:}

\vskip -7pt

A Brahmacārin wore a \textit{yajñopavīta} (sacred thread) made three strands of thread. He learnt the Vedas in 48 years mentions Nakkīrar in his \textit{Tirumurugāṟṟppaḍai} (vv.~179, 184). Eṉāthi Neḍuṅkaṇṇaṉār in his \textit{Kuṟuntogai} (v.~156) declares that a Brahmacāri bears a Palāśa tree's branch in hand, a small \textit{kamaṇḍala} and partakes a restricted diet:

\begin{quote}
\textit{“cembu murukki ṉaṉṉār kaḷaintu}\\\textit{taṇḍoḍu piḍitta tāḻkamaṇ ḍalattup}\\\textit{paḍiva vuṇḍip pārppaṉa magane}.”
\end{quote}


\paragraph*{(ii) Gṛhastha:}

\vskip -7pt

The marriage ritual is of eight types – Brāhmam, Prājāpatyam, Ārṣam, Daivam, Gāndharvam, Āsuram, Rākṣasam and Paiśācam – as listed in the \textit{Gautama Dharmasūtra} (I. 4. 4–11) and such works. In the \textit{kaḷaviyal} section of the \textit{Tolkāppiyam} (15. 1. 14) it is said that the first four types of marriage belong to the \textit{Peruntiṇai}; Gāndharvam is equal to \textit{kaḷavu;} Āsuram, Rākṣasam and Paiśācam belong to the \textit{Kaikiḷaitiṇai}.

Sage Vyāsa gives an interesting information in the \textit{Mahābhārata} (I.~94.13), that if there is no one to give the bride in \textit{kanyādāna} then the bride could do that ritual by herself:

\begin{quote}
\textit{“ātmano bandhuḥ ātmaiva gatirātmaiva cātmanaḥ}\dev{।}\\\textit{ātmanaivātmano dānam karttumarhasi dharmataḥ \dev{।।}}”
\end{quote}

The same dictum is presented by Tolkāppiyar (\textit{Kaṟpu}. 2) as:

\begin{quote}
\textit{“koḍuppō riṉṟiyuṅ karaṇa muṇḍe}”
\end{quote}

If two persons walk seven steps together then they are considered to be friends forever\endnote{\textit{Cf. Kumārasambhava} of Kālidāsa, V. 38.}, records Muḍattāmakkaṇṇiyār in the \textit{Porunarāṟṟuppaḍai} (v.~166) as: \textit{“kāli ṉeḻaḍip piṉceṉṟu.}”

Likewise, the \textit{Mahābhārata} (III. 260. 35) also mentions the ritual as:

\begin{quote}
\textit{“satām sāptapadam mitram āhuḥ santaḥ kulositāḥ \dev{।}}”
\end{quote}

The \textit{Tirikaḍugam} (v.~35) states that a Gṛhastha is due to three types of \textit{ṛṇa} (debts), \textit{viz., ṛṣi ṛṇa, deva ṛṇa} and \textit{pitṛ ṛṇa:} \textit{“mūṉṟu kaḍaṉkaḻittu pārppāṉum...}”

The \textit{Taittirīya Samhitā} (VI. 3. 10) records that the \textit{ṛṣi ṛṇa} is repaid by practicing the Vedas; \textit{deva ṛṇa} by performance of \textit{homas} and \textit{pitṛ ṛṇa} by begetting a progeny:

\begin{quote}
\textit{“jāyamāno vai bramhaṇo tribhiḥ ṛṇavā}\\\textit{jāyate bramhacaryeṇa ṛṣibhyaḥ yajñena}\\\textit{devebhyaḥ prajayā pitṛbhyaḥ \dev{।।}}”
\end{quote}

The \textit{Kalittogai} (v.~2) and the \textit{Perumpāṇāṟṟuppaḍai} (v.~315–16) refer to the \\Deva debt:

\begin{quote}
\textit{“kēlvi yantaṇa raruṅkaḍa ṉiṟutta\\ vēlvit thūṇ tasaii}”
\end{quote}

The details of \textit{pitṛ} debt is mentioned in the \textit{Puṟanāṉūṟu} (v.~9) as:

\begin{quote}
\textit{“teṉpula vāḻnark karukaḍa ṉiṟukkum}\\\textit{poṉpōṟ pudalvarp peṟāvu tīrum}”
\end{quote}

In the above verse the words: “\textit{teṉpula vāḻnarkku}” denotes the \textit{pitṛ-} s, for they occupy the southern direction. This is affirmed by the \textit{Taittirīya Samhitā} (IV. 1. 1) \textit{mantra} – “\textit{dakṣiṇā pitaraḥ}\dev{।}”


\paragraph*{(iii) Vānaprastha:}

\vskip -7pt

The Vānaprastha, according to Salliyaṅkumaraṉār in \textit{Naṟṟiṉai} (v.~141), possessed long hair and performed penance on the mountains without moving:

\begin{quote}
\textit{“nīṭiya saḍaiyō ṭāḍā mēṉik}\\\textit{kuṉṟuṟai tavasiyar.}”
\end{quote}


\paragraph*{(iv) Sannyāsin:}

\vskip -7pt

Tolkāppiyar records that a person who has renounced his family is a Sannyāsin (\textit{Puṟa}. 17): “\textit{aruloḍu puṇarnta vakaṟci yāṉum.}”

\textit{Kaṭhopaniṣad} (VI. 14) establishes that, one who overcomes desire is eligible for \textit{Jīvanmukti:}

\begin{quote}
\textit{“yadā sarve pramucyante kāmā ye’sya hṛdi śritāḥ}\dev{।}\\\textit{atha martyo amṛto bhavatyatra brahma samaśnute \dev{।।}}”
\end{quote}

The \textit{Tolkāppiyam} (\textit{Puṟa}. 17), the \textit{Porunarāṟṟuppaḍai} (91. 2), \textit{Maduraikāñci} (v.~463–74) and \textit{Tirukkuṟaḷ} (\textit{Kuṟaḷ}. 370) also concur with the same idea and state:

\begin{quote}
\textit{“ārā viyaṟkai yavānīppi ṉannilayē}\\\textit{pērā viyaṟkai tarum.}”
\end{quote}



\subsubsection*{(3) \textit{Agnihotra} and \textit{Yajña:}}

\vskip -7pt

Here some of the Vedic rituals and sacrifices as articulated in the Saṅgam literature will be discussed:

The Brāhmaṇas performed \textit{agnihotra} with the use of Gārhapatya, Āhavanīya and Dakṣināgni, twice a day to repay the Deva debt. This fact is recorded in the \textit{Puṟanāṉūṟu} (122. 2–3), \textit{Paṭṭiṉappālai} (v.~200), \textit{Kalittogai} (v.~119), \textit{Kuruñjippāṭṭu} (v.~225) and so on.

\textit{Padiṟṟuppattu} (vv.~70, 74 and 21) mentions the sacrifices performed to please the Devas and gods. The Brāhmaṇas performed the \textit{yajña-s} as prescribed in the \textit{mantra-s}, is evident from texts like \textit{Tirumurugāṟṟppaḍai} (vv.~94–6), \textit{Kalittogai} (v.~36) and \textit{Ciṟupañcamūlam}.

The \textit{Perumpāṇāṟṟuppaḍai} (vv.~315–16) and \textit{Aganāṉūṟu} (v.~220) record that, during \textit{yajñas}, sacrificial posts (\textit{yūpa-s}) were constructed and they were fully covered using twisted cloth ropes:

\begin{quote}
\textit{“.................. vēḷvi}\\\textit{kayiṟṟai yātta kāṇḍaku vaṉappiṉ}\\\textit{aruṅkaḍi neḍunthūṇ pōla.}”
\end{quote}

Many kings of the Saṅgam period also performed \textit{yajña} and it has been described by the poets of that time in their works. The details are as follows:

\paragraph*{(i) Kauṉiyaṉ:}

\vskip -7pt

This king was born in a family which has studied the Vedas, protested against the non-Vedic religions and performed seven \textit{Bhāgayajñas}, seven \textit{Haviryajña-s} and seven \textit{Somasamsthai-s}. Just like his forefathers, Kauṉiyaṉ, along with his wife performed various sacrifices and entertained guests. All these details is available in the \textit{Puṟanāṉūṟu} (v.~166). The details of the \textit{Bhāgayajña-s, Haviryajña-s} and \textit{Somasamsthai} are vividly explained in the Dharmasūtra texts.


\paragraph*{(ii) Mudukuḍumipperuvaḻuti:}

\vskip -7pt

As prescribed in the four Vedas, he collected all the ingredients for the \textit{haviṣ} and with lots of ghee performed the \textit{yajña-s}\index{yajna@\textit{yajña}}. To propagate his victory he erected victory pillars or \textit{yūpas} in many places, records \textit{Puṟanāṉūṟu} (v.~15. 17–22):

\textit{“.................. puraiyil}\\\textit{naṟ paṉuval nāl vēdattu}\\\textit{aruñ cīrttip peruṅ kaṇṇuṟai}\\\textit{neim mali āvuti poṅkap, paṉmāṇ}\\\textit{vīyac ciṟappiṉ vēḷvi muṟṟi,}\\\textit{yūpam naṭṭa viyaṉkaḷam palakol}?”

From the above poem we come to know that the victory pillars were also referred to as \textit{yūpa}. Kālidāsa too, in his \textit{Raghuvamsa} (VI. 38ab) uses the word \textit{yūpa} in the same context, when describing Kārtavīryārjuṇa:

\begin{quote}
\textit{“saṅgrāma-nirviṣṭasahasrabāhur-aṣṭadaśadvīpanikhātayūpaḥ}” 
\end{quote}

Through the words of Māṅguḍi Marutaṉār, in his \textit{Maduraikāñci} (vv.~759–863), it is evident that Peruvaḻuti performed all the sacrifices selflessly and attained \textit{cittaśuddhi} (purity of mind), which is the important factor leading to \textit{ātma darśaṇa} (Self-realisation). The \textit{Bhagavad Gītā} (II. 51) explains that, one who attains \textit{cittaśuddhi} is relieved from re-births and attains the Feet of the \textit{Paramātmā:}

\begin{quote}
\textit{“karmajam buddhiyuktā hi phalam tyaktvā manīṣiṇaḥ}\dev{।}\\\textit{janmabandhavinirmuktāḥ padam gacchantyanāmayam}\dev{।।}”
\end{quote}


\paragraph*{(iii) Karikāl Peruvaḷattāṉ:}

\vskip -7pt

\textit{Puṟanāṉūṟu} (v.~224) reveals that Peruvaḷattāṉ, with the help of the Brāhmaṇas, well-versed in Dharma; with Ṛtviks, who were experts in performing the sacrifice and with his chaste wife, performed the \textit{Garuḍa śyena yajña} and won laurels:

\begin{quote}
\textit{“aṟam aṟak kaṇḍa neṟimāṇ avaiyattu,}\\\textit{muṟainaṟku aṟiyunar muṉṉuṟap pukaḻnta}\\\textit{tūviyaṟ koḷkait tugaḷaṟu magaḷiroḍu,}\\\textit{paruti uruviṉ palpaḍaip puricai...}”
\end{quote}


\paragraph*{(iv) Selvakkaḍuṅkovāḻiyātaṉ:}

\vskip -7pt

This king pleased the gods through \textit{yajña-s} performed by well-versed Brāhmaṇas, says \textit{Padiṟṟuppattu} (vv.~64, 70) and that the King rewarded them with lots of ornaments.


\paragraph*{(v) Others:}

\vskip -7pt

Perunaṟkiḷḷi is said to have performed the \textit{Rājasūya yāga} and Nallaṅkiḷḷi performed many \textit{yajña-s}. This is seen from the verses 363 and 400 of the \textit{Puṟanāṉūṟu} respectively:

\begin{quote}
“\textit{kāḍupati yāgap pōgit, tattam}\\\textit{nāḍu piṟarkoḷac cenṟumāyn taṉarē}”
\end{quote}

\begin{quote}
\textit{“kēlvi malinta vēlvit thūṇattu}”
\end{quote}



\subsubsection*{(4) Birth and Rebirth:}

\vskip -7pt

All living beings are endowed with a \textit{ātmā} and body; the body is called the \textit{śarīra} and is of three types, \textit{viz., sthūlaśarīra} (mortal body), \textit{śūkṣmaśarīra} (subtle body) and the \textit{kāraṇaśarīra} (causal body). Till the \textit{ātmā} attains \textit{mukti,} it has only one \textit{śūkṣma śarīra}, but the \textit{sthūla śarīra} varies based on the world in which the \textit{ātmā} exists. Therefore the \textit{ātmā} will leave the mortal body one day. This is clearly established in the \textit{Puṟanāṉūṟu}, the \textit{Aganāṉūṟu, Perumpāṇāṟṟuppaḍai} and \textit{Porunarāṟṟuppaḍai}.

For the \textit{ātmā} that has left the body to be reborn, the \textit{piṇḍa-s} offered by the male progeny during the last rites is the food. In case of no male progeny, the wife will have to offer the \textit{piṇḍas}, states the \textit{Puṟanāṉūṟu} (v.~234):

\begin{quote}
\textit{“piḍiyaḍi yaṉṉa ciṟuvaḻi meḻukit}\\\textit{taṉṉamar kādali puṉmēl vaitta}\\\textit{iṉciṟu piṇḍam yāṅkuṇ ḍaṉaṉkol}”
\end{quote}

The \textit{Mahābhārata} (VII. 173. 54) also vouches for the same:

\begin{quote}
\textit{“icchanti pitaraḥ putrān svārthahetorghaṭotkaca}\dev{।}\\\textit{iha lokāt pare loke tārayiṣyanti ye hitāḥ}\dev{।।}”
\end{quote}

\textit{Nāṉmaṇikkaḍigai} (v.~15) and other texts record that everyone will reap the consequence of their actions of this birth in their next birth.

\textit{Īśāvāsyopaniṣad, Bhagavad Gītā} (II. 47), \textit{Puṟanāṉūṟu} (v.\ 132, 184), \textit{Aganāṉūṟu \\} (v.~54) and \textit{Padiṟṟuppattu} (v.~38) confirms the fact that doing good deeds without expecting repayment is considered superior.

The practice of ‘Sati’ – entering the funeral pyre of the husband – after the death of the husband (or) to continue living as a widow, were both prevalent for the women as per the verses of the \textit{Puṟanāṉūṟu} (vv.~246 and 234).

\textit{Puṟanāṉūṟu} (v.~93) and the \textit{Bhagavad Gītā} (II. 37) state that a warrior who dies in battle attains heaven:

\begin{quote}
“\textit{nīḷkaḻaṉ maṟavar celvaḻic celka}” and \\ “\textit{hato vā prapyasi svargam}\dev{।।}”
\end{quote}

Indra is the Lord of the Devas. He is said to have performed a hundered \textit{yajñas} according to the \textit{Tirumurugāṟṟppaḍai} (v.~155) and \textit{Paripāḍal} (v.~9. 8). \textit{Tirukkuṟaḷ} (\textit{ku}. 25) recounts that Indra incurred the curse of the sages as he dishonoured them:

\begin{quote}
\textit{“aintavittā ṉāṟṟa lakalvisumpu ṉārkōmāṉ}\\\textit{indiranē cāluṅ kari}”
\end{quote}

The \textit{Paripāḍal} (v.~8) lists the 33 Devas to be – 12 Ādityas, 2 Aśvinikumāras, 8 Vasūs and 11 Rudras. \textit{Pitṛs} are also a type of Deva. The fact that to repay them is to produce a male child and they occupy the southern direction, have already been discussed. The \textit{Maṇusmṛti} defines them to be very calm, pure and always Brahmacārin-s:

\begin{quote}
\textit{“akrarodhanāḥ caucaparāḥ satatam brahmacāriṇaḥ \dev{।।}}”
\end{quote}

The \textit{Mahābhārata} (XIII. 86. 8) and the \textit{Kalittogai} (v.~129) explain that all the things on earth get absorbed into the \textit{paramātmā} at the time of dissolution (\textit{pralaya}).


\subsubsection*{(5) Deities of worship:}

\vskip -7pt

The deities worshipped during Saṅgam age are Brahmā, Viṣṇu, Śiva and Muruga. Balarāma and Koṟṟavai (Durgā) were also worshipped.

\paragraph*{(i) Brahmā:}

\vskip -7.5pt

He is the creator of this Universe and is said to have been born from the navel-lotus of Viṣṇu. He has four faces and his \textit{vāhana} is the Swan. All the above information is available in texts like \textit{Kalittogai} (v.~2), \textit{Perumpāṇāṟṟuppaḍai} (vv.~402–04), \textit{Tirumurugāṟṟppaḍai} (v.~164–65), \textit{Paripāḍal} (v.~8. 3) and \textit{Mahābhārata} (III. 273).


\paragraph*{(ii) Viṣṇu:}

\vskip -7.5pt

\textit{Tolkāppiyam} (\textit{Aga}. 5) mentions Viṣṇu (or) Tirumāl as the presiding deity of the \textit{Mullaitiṇai: “māyōṉ mēya kāḍuṟai yulagamum}”

The other Tamiḻ texts portray Him as the bearer of \textit{śaṅkha} and \textit{cakra} and Goddess Lakṣmī in His bosom; blue-hued lord; lotus-eyed; Garuḍa as His \textit{vāhana}; the measurer of the three worlds; the destroyer of Prahlāda’s father; slayer of the demon Keśi, etc.


\paragraph*{(iii) Śiva:}

\vskip -7.5pt

\textit{Aganāṉūṟu} (v.~360. 6) considers Lord Śiva and Viṣṇu as the two main deities:

\begin{quote}
\textit{“veruvaru kaḍuṅtiṟa liruperun deivattu}”
\end{quote}

The five epics and the various texts of Saṅgam literature depict Lord Śiva as the one who bears the Ganges and the moon on His head; the three-eyed; bearer of the \textit{maḻu} (axe); Vṛṣabha is His \textit{vāhana}; as \textit{Nīlakaṇṭha} (blue-throated). Moreover, He is said to have been born under the Tiruvādirai star; seated under the Banyan; Umā is His consort; creator of the five Elements and destroyer of the worlds at the proper time.


\paragraph*{(iv) Muruga:}

\vskip -7.5pt

\textit{Tolkāppiyam} (\textit{Aga}. 5) designates Lord Muruga as the presiding deity of the \textit{Kuruñjittiṇai:} \textit{“sēyōṉ mēya maivarai ulagamum.}”

Nakkīrar advises one to worship Lord Muruga in his six ‘\textit{paḍaivīḍu} ’ for attaining \textit{mokṣa}. This advice establishes that Muruga was worshipped as a warrior god.

It is noteworthy that Lord Kṛṣṇa in the \textit{Bhagavad Gītā} (X. 24) when talking about \textit{ātmavibhūti} mentions Skanda as a warrior: \textit{“senānīnām aham skandaḥ}\dev{।}”


\paragraph*{(v) Balarāma:}

\vskip -7.5pt

The \textit{Puṟanāṉūṟu} (v.~56, 58), \textit{Kalittogai} (v.~104. 7–8), \textit{Paripāḍal} (v.~2. 20–3), \textit{Kārnāṟpatu} (v.~19) and \textit{Paḻamoḻi} (v.~37) describe Balarāma as fair-complexioned; clothed in blue; elder to Kṛṣṇa; having Palm tree as his banner and plough as his weapon.


\paragraph*{(vi) Koṟṟavai:}

\vskip -7.5pt

Goddess Durgā and Koṟṟavai are the same. To gain victory in war and for accomplishment of all works, \textit{Tirumurugāṟṟppaḍai} (v.~258) and \textit{Neḍunalvāḍai} (v.~166–68), proclaims that people worshipped this deity. This is similar to the instances of Dharmaputra (\textit{Mahābhārata}, IV. 8) and Arjuna (\textit{Mahābhārata}, VI. 23) propitiating Goddess Durga before entering into Virāṭa and before the beginning of the war respectively.


\subsubsection*{(6) \textit{Karma} or \textit{Üḻ:}}

\vskip -7.5pt

The consequences of \textit{karma-s} or deeds of previous birth are the basis for the pleasures or pains one experience in this life. \textit{Tolkāppiyam} (\textit{Kiḷavi}. 57) denotes it by the terms – \textit{ūḻ, deivam, viṉai, pāl}, etc.

Tiruvaḷḷuvar (\textit{Kuṟaḷ}. 380, 619) tries to firmly establish the fact that though our deeds reflect upon our well-being, one’s effort to do Dharma is the real factor; he has dedicated a chapter in his text for \textit{ūḻ}. He further adds that the effect of one’s deeds may continue for the seven births or would effect immediately in this birth itself.

The \textit{Mahābhārata} (XII. 34. 43) and the \textit{Puṟanāṉūṟu} (v.~34) indicates that some \textit{pāpa}-s have possible expiation whereas some do not.


\subsubsection*{(7) Liberation:}

\vskip -7.5pt

\textit{Mukti} or Liberation is of two types, \textit{viz., videhamukti} and \textit{jīvanmukti}. The liberation attained after the \textit{jīva} leaves this body is termed as \textit{videhamukti}. To be in \textit{Brahmaloka} after liberation is called \textit{aparāmukti} and to become One with the Brahman is called \textit{parāmukti}. Tiruvaḷḷuvar denotes \textit{aparāmukti} by the term \textit{nilam} in \textit{Ku.} 3 and \textit{parāmukti} by the phrase “\textit{ciṟappeṉṉuñ cemporuḷ}” in \textit{Ku.} 358.

Saṅkarācārya in his \textit{Bṛhadāraṇyakopaniṣad bhāṣya} (IV. 4. 23) splits the word ‘\textit{bramhaloka}’ in two ways as – ‘\textit{brahmaṇo lokaḥ}’ and ‘\textit{brahmaiva lokaḥ}’. Of these the former represents \textit{aparāmukti} and the later \textit{parāmukti}.

Lord Kṛṣṇa in the \textit{Bhagavad Gītā} (II. 58, 71) and Tiruvaḷḷuvar (\textit{ku.} 6, 352) prescribe that discipline towards God is the right path to \textit{mukti}, for then, ignorance is destroyed and pure knowledge is born.

\textit{Nāṉmaṇikkaḍigai} (v.~100) suggests, love towards all beings, as one of the ways to attain liberation. Nakkīrar in his \textit{Tirumurugāṟṟppaḍai} (v.~62–4) denotes \textit{mukti} by the term ‘\textit{celavu}’. In the similar sense, \textit{Kaṭhopaniṣad} (III. 11) uses the word ‘\textit{gatiḥ}’.

While \textit{Kaṭhopaniṣad} (III. 8) uses the term ‘\textit{tatpadam}’ for \textit{mukti}, the \textit{Perumpāṇāṟṟuppaḍai} (v.~466–67) uses the word ‘\textit{annilai}’ in the same effect. The attainment of \textit{mokṣa} occurs when one overcomes desires as depicted in \textit{Tirukkuṟaḷ} (v.~370), \textit{Maduraikāñci} (vv.~468–71), \textit{Kaṭhopaniṣad} (VI. 14) and \textit{Mahābhārata} (XIV. 47. 8), has already been discussed before.


\subsection*{IV. Saṅgam Culture In Contemporary Tamilnadu}

\vskip -7.5pt

From the above study of the Saṅgam literature we find several instances of evidence of co-evolution of Vedic and Saṅgam civilisation existing together and building upon each other’s contributions to humanity at large. The paper has focussed and provided some of that foundational evidence in the form of textual articulations in Saṅgam literature that is so very resonant with Vedic cultural expressions and strongly indicate that the Indic civilisation embraced diversity of local traditions though deeply rooted in the profound pan-Indic ethos.

The question that stares at us in the present day scenario of Tamilnadu is that – whether these are still in vogue, if so – their validity; if not – their need and if needed – how to resolve it?

To my knowledge, with the minimum awareness of two decades of politico-cultural chaotic situation in Tamilnadu, it seems to me that the resurrection of these past values of Indian life is needed. Taking into consideration, the British rule which turned the basic structure of Indian culture topsy-turvy and the Communist party movement and Justice party movement, which gave room to the Dravidian political movement in Tamilnadu, appear to have undermined the values of our culture. The Dravidian movement has developed into many factions and is still having its effects on the society. However, all these have cast an effect at the superficial level without harming the roots of our culture.

The \textit{Varṇa} system exists still, though not exactly in the way elucidated in the scriptures. We have now different \textit{varṇas} like teachers, doctors, lawyers, businessmen, soldiers (army, navy and air force), agriculturists and politicians, thereby proving that the earlier \textit{varṇa} system which was based on one’s vocation still exists. One has to keep in mind that \textit{varṇa} and \textit{jāti} are two different terms often translated as ‘caste’. The \textit{Āśramadharma} is also in another form as in the life of student, life of a married man, the third and fourth of Vānaprastha and Sannyāsa are now spent in old age homes or parents staying away from their children.

The practice of worshipping in the temples or performing \textit{pūja} and \textit{homa}, for the welfare of the mankind, still persists. It seems to have had a period of oblivion during the Independence movement because of other crucial problems. In Tamilnadu, though people have questioned their validity, they had not lost faith in them, for many congregations carrying out mass \textit{pūjas} and \textit{homas} exist even today. Temple worship is also on the rise in comparison to its practice in the last few decades. More temples are also being built, in addition to renovating and rejuvenating the dilapidated old structures. In all these, both the people and the government get involved.

With regard to the philosophical concepts of belief in \textit{karma} theory and belief in rebirth, it is still in us, lying dormant. When talking to elders, I realised that the fourth generation anterior to me, though soaked in Indianness, was also open to the world at large.

While it imbibed, valuable principles from other systems, the next two generations caught in the fever of freedom movement and later the overdose of freedom, seemed to have tested their own cultural values. They have been carried away by the forces of the time, yet the generations immediately older to me have started adapting to some changes and going through the history of the previous generations.

In the process, my generation is in an advantageous position of learning from errors of the previous generations. Going through the entire analysis of the pros and cons of the generations, India in a wider perspective, Tamilnadu specifically, are none the worse for all the ups and down that have happened and are happening.


\section*{Conclusion}

\vskip -5pt

When talking about a country, we cannot take into consideration merely the elites, the freaks and the changelings. Whatever their percentage maybe, the mediocre mass is the backbone of the country, for they still remain Indian, even to this day.

So, whatever the political or the sociological or industrial or economical or cultural renovations and reformations that have taken place in India and in Tamilnadu, however much the chaos may appear to be, the undercurrent of the ancient Indian principles and ethos have not yet dried up.


\section*{Bibliography}

\begin{thebibliography}{99}
\bibitem{chap7-key01} \textit{Bhagavadgītā} (1949), Gorakhpur: Gita press.

 \bibitem{chap7-key02} \textit{Brahmasūtra} with \textit{Śaṅkarabhāṣya} (1985), Delhi: Motilal Banarsidas.

 \bibitem{chap7-key03} Dikshitar, V.R.R. (1930) “\textit{Studies in Tamil Literature and History”} Madras: M.L.J Press.

 \bibitem{chap7-key04} \textit{Īśādi Aṣṭottaraśatopaniṣad}, ed. by V.L.S. Pansikar (1991), Varanasi: Chowkamba Vidyabhavan.

 \bibitem{chap7-key05} \textit{Kaṭhopaniṣad} (1952), Chennai: R.K. Math.

 \bibitem{chap7-key06} \textit{Kumārasambhava} of Kālidāsa with com. of Mallinātha, ed. by M.R. Kale (1981), Delhi: Motilal Banarsidas.

 \bibitem{chap7-key07} \textit{Mahābhārata (} 6 vols.) (1957), Gorakhpur: Gita press.

 \bibitem{chap7-key08} \textit{Manusmṛti} with com. of Kullūkabhaṭṭa (1946), Bombay: Nirnaya Sagar Press.

 \bibitem{chap7-key09} Neelamalur, M. (1994) “\textit{Tolkāppiyam}” with Eng tr. and critical Studies, Chennai: Educational Publishers,.

 \bibitem{chap7-key10} \textit{Puṟanāṉūṟu}, with the com. of S. Duraisawami Pillai (1947) Chennai.

 \bibitem{chap7-key11} \textit{Raghuvasma} with Mallinātha’s com., ed. by Haragovinda Sastri (2007) Varanasi: Chowkhamba Sanskrit Samsthan.

 \bibitem{chap7-key12} \textit{Ṛgveda}, with com. of Sāyaṇācārya (1972) Pune: Vaidika Samsodhan Mandal.

 \bibitem{chap7-key13} Swaminatha Iyer, V. “\textit{Pattupāṭṭu”}.

 \bibitem{chap7-key14} \textit{Taittirīya Āraṇyaka} (1967) Pune: A.S.S. No. 36.

 \bibitem{chap7-key15} \textit{Taittirīya Upaniṣad} (1965) Chennai: R.K. Math.

 \bibitem{chap7-key16} \textit{Tirukkuṟaḷ}, ed. by Raghava Iyengar, 1910.

 \bibitem{chap7-key17} \textit{Vālmīki Rāmāyaṇa} (1958) Madras: M.L.J. Press.

 \bibitem{chap7-key18} Varadarajan, M. (1983) “\textit{Tamiḻ Ilakkiyiya Varalāṟu”} Chennai: Sahitya Academy.

 \end{thebibliography}

\theendnotes

