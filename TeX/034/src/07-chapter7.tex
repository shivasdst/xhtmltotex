
\chapter{SAÌGAM LITERATURE AND VAIDIKA MÄRGA}

\Authorline{K. Vidyuta, Research Scholar, K. S. R. Institute, Chennai}


\section*{ABSTRACT:}

The Saìgam age\index{Saigam age@\textit{Saìgam age}} is the period in the history of ancient Tamil Nadu (the present Tamilnadu, Kerala, parts of Andhra Pradesh, parts of Karnataka and northern Sri Lanka) spanning from 4th C.B.C to 2nd C.A.D.

According to Tamil legends there were three Saìgam periods, \textit{viz., Thalai Saìgam, Iòai Saìgam} and \textit{Kaòai Saìgam}. However the historians refer to only the Third Saìgam or \textit{Kaòai Saìgam} period as the “Saìgam period”. Most of the available Saìgam Literature are from third Saìgam period. This collection contains 2381 poems in Tamil composed by 473 poets, some 102 of whom remain anonymous.

The famous Saìgam Literature texts are \textit{Agattiyam\index{Agattiyam@\textit{Agattiyam}}, Tolkäppiyam\index{Tolkappiyam@\textit{Tolkäppiyam}}, Patineëmēlkaëakku} consisting of \textit{Eööuttokai} and \textit{Pattuppäööu} and the \textit{Patineëkéøkaëakku}. \textit{Agattiyam} (which is lost now) belongs to the First Saìgam period and the \textit{Tolkäppiyam} is from the Middle Saìgam period and the rest are from the Last Saìgam period.

These texts reflect the ideas of the Vedas in them. The Saìgam texts talk of the Vedas and its limbs (Vedäìgas), the concept of \textit{Varëäçrama\index{Vareacrama@\textit{Varëäçrama}} dharma}, the practice of \textit{agnihotra} and others \textit{yägas}, the concept of re-birth, religious ideas, the concept of \textit{karma} and also the concept of \textit{Mokña}\index{Mokna@\textit{Mokña}}.

This paper will attempt to establish the fact that the Saìgam literature contains many Vedic principles.

\begin{center}
\textbf{FULL PAPER}
\end{center}


\section*{INTRODUCTION:}

“\textit{Teṉmoøi tēṉmoøi, vaòamoøi väòämoøi}”– “the language of the South is honey; the language of the North never fades” is a famous saying. Both the languages – Sanskrit and Tamiø – are the basis of our Indian culture. It is well known that Tamiø is one of the longest surviving classical languages in the world. The earliest period of available Tamiø literature, the Saìgam literature, is dated from 300 BCE – 300 CE. The earliest epigraphic records\index{epigraphic records} found on rock edicts date from around the 3rd century BCE.

The Saìgam period marked the golden period of Tamiø literature, which refers to the prevalent Saìgam legends claiming literary academies lasting thousands of years, giving the name to the corpus of literature. These poems were later collected into various anthologies\index{anthologies} and edited with colophons added by anthologists and annotators around 1000 CE.

Sanskrit is one of the most ancient languages of the world. As the language was made perfect and refined by the grammarians, it was named \textit{Samskåta}. This language is considered to be the mother of all Indian languages, except Tamiø, by scholars of impartial views. But they do reflect the influence of Sanskrit since all these languages flourished together.

Tamiø and Sanskrit were considered as the two eyes of the same face and hence were studied with equal interest by our ancestors in Tamilnadu. In fact the Vedic culture, its rituals and practice were in vogue during the Saìgam age and they got incorporated in the works of the poets of those times.

Here we shall discuss this in four sections – I. Saìgam Literature, II. \textit{Vaidika MÄRGA}, III. \textit{Vaidika MÄRGA} reflected in the Saìgam literature and VI. Saìgam culture in contemporary Tamilnadu.

\subsection*{I. SAÌGAM LITERATURE}

Tamilnadu housed the three Saìgams during ancient times. They were the \textit{Talai Saìgam, Iòai Saìgam} and \textit{Kaòai Saìgam}. During the \textit{Talai Saìgam} period there existed many grammatical and literature works as per the information gleaned from the \textit{Tolkäppiyam}; but these are not available today. Even among the \textit{Iòai Saìgam} works only \textit{Tolkäppiyam}, a grammatical work, authored by Tolkäppiyar\index{Tolkappiyar@\textit{Tolkäppiyar}}. The works of \textit{Kaòai Saìgam,} which are available now, are all literature works – \textit{Pattuppäööu, Eööuttogai} and \textit{Patiṉeëkéøkaëakku}.

\subsubsection*{(1) \textit{Tolkäppiyam}:}

The author of this text is Tolkäppiyar. He is said to have belonged to no later than the 2nd Cent. B.C. This text contains about 1610 verses or \textit{sūtras}. The text is divided into three sections \textit{viz., Eøuttadhigäram} with 480 verses, \textit{Colladhigäram} with 465 verses and \textit{Poruÿadhigäram} with the remaining 665 verses.

The varieties of letters, number, its quality, origin and the changes they undergo when combining with other letters, is widely discussed in the first chapter \textit{Eøuttadhigäram}. The Sanskrit equivalent to this is the \textit{śikñä çästra}.

The various types of words, their formation, combination with other words to form compounds and then into sentences is explained in the \textit{Colladhigäram}. This is termed as \textit{Vyäkaraëa}\index{Vyakaraea@\textit{Vyäkaraëa}} in Sanskrit.

The method for writing poetry the usage of different \textit{rasas} like \textit{Çåìgära, Karuëa, Véra}, etc., the usage of Figures of speech like similes and metaphors the usage of colloquial words and so on are expressed in the \textit{Poruÿadhigäram}. These fall under the class of \textit{Alaìkära çästra}\index{Alaikara castra@\textit{Alaìkära çästra}} in Sanskrit literature.

Those works dealing predominantly with \textit{Çåìgära rasa} were classified as “\textit{Agam}” and the rest as “\textit{Puṟam}”. \textit{Çåìgära rasa} is said to be predominant only in plays involving \textit{Gändharva} style of marriage. The “\textit{Agam}” section is further divided into five sub-sections termed \textit{“Tiëai}”: \textit{Kuruïji, Mullai, Marutham, Neytal} and \textit{Pälai}.

If one peruses the \textit{Tolkäppiyam}, one might come to know that many such grammatical texts existed during those times but none other being available now is a great loss to Tamilnadu.


\subsubsection*{(2) \textit{Pattuppäööu}:}

\textit{Tirumurugäṟṟppaòai, Porunaräṟṟuppaòai, Ciṟupäëäṟṟuppaòai, Perumpäëäṟṟuppaòai, Mullaipäööu, Maduraikäïci, Neòunalväòai, Kuruïjippäööu, Paööiṉappälai} and Malaipaòu-kaòäm (also known as \textit{Küttaräṟṟuppaòai}) – are the \textit{Pattuppäööu}.\index{Pattuppaoou@\textit{Pattuppäööu}}

Of the 10 songs, five are sung in the form of \textit{äṟṟuppaòai}. This is a form of poetry where a poet suggests another poet, who he meets on the way, to go to a certain noble man and sing his poetry and to get rewarded. Of these the \textit{Tirumurugäṟuppaòai} is an exception.

\textbf{(i) \textit{Tirumurugä}ṟṟ\textit{ppaòai}:} Sung by Nakkérar, this poem on the six temples of Lord Muruga, collectively called “\textit{Aṟupaòaivéòu}”, contains 317 lines.

\textbf{(ii) \textit{Porunaräṟṟuppaòai}:} Praises king Karikäla Cōøaṉ, for his charity, bravery, and his prosperous reign, in 248 lines.

\textbf{(iii) \textit{Ciṟupäëäṟṟuppaòai}:} Describes the capitals of the three Tamil monarchs and about the seven philanthropists – Bōgaṉ, Päri, Käri, äy, Athigaṉ, Naÿÿi and Ōṟi, in 269 lines.

\textbf{(iv) \textit{Perumpäëäṟṟuppaòai}:} Explains the five \textit{tiëai} regions of Käïci city, its lifestyle and the trade in 500 lines.

\textbf{(v) \textit{Mullaipäööu}:} Expresses the state of mind of a wife whose husband has gone to war in 103 lines.

\textbf{(vi) \textit{Maduraikäïci}:} In 782 lines, this work expounds the speciality of Madurai the five \textit{tiëais} and the glory of the ancestors of king Neḍuïceøiya Päëòiyaṉ.

\textbf{(vii) \textit{Neòunalväòai}:} In 188 lines, describes the long period of wait by the queen of Päëòiyaṉ Neḍuïceøiyaṉ who has gone for war and the setting in of the North-easterly winds.

\textbf{(viii) \textit{Kuruïjippäööu}:} Comprising of 261 lines, this portrays the love between the hero and heroine culminating in a secret affair.

\textbf{(ix) \textit{Paööiṉappälai}:} In 301 lines, this poem recounts the words of the hero, the beauty of Käveripümpaööiṉam, Karikälperuvaÿattäṉ's bravery, his just rule, and so on.

\textbf{(x) \textit{Malaipaòukaòäm}:} Containing 583 lines, this text also called as \textit{Küttaräṟṟuppaòai,} outlines the greatness of the philanthropist Naṉṉaṉ and the beauty of the forest, mountain and groves possessed by him.


\subsubsection*{(3) \textit{Eööuttogai}:}

The \textit{Eööuttogai}\index{Eoouttogai@\textit{Eööuttogai}} texts are: \textit{Naṟṟiëai, Kuṟunthogai, Aiìkuṟunüṟu, Paripäòal, Kalittogai, Aganäṉüṟu, Padiṟṟuppattu} and \textit{Puṟanäṉüṟu}. These texts are compilations of songs sung by different poets over a period of time in the \textit{Agam} and \textit{Puṟam} genres. Among the eight texts, the last two are of \textit{Puṟam} genre while the rest are of \textit{Agam} type. Excepting \textit{Paripäòal and Kalittogai,} the others are set in \textit{agavarpä} metre; \textit{Paripäòal} is in \textit{paripäòal} metre and \textit{Kalittogai} in \textit{kalippä} metre.


\subsubsection*{(4) \textit{Patiṉeëkéøkaëakku}:}

The eighteen poems collectively termed so are:

\textbf{(i)\textit{Tirukkuṟaÿ}:}\index{Tirukkuray@\textit{Tirukkuṟaÿ}} This work contains 1330 couplets or \textit{kuṟaÿs}. It talks about the glory of rain, the greatness of recluses, the power of righteousness, the virtue of house-holders, the virtues of ascetics, the ways of the material world and so on.

\textbf{(ii) \textit{Nälaòiyär}:} 400 songs in 4 lines, the work shares ideas similar to the \textit{Tirukkuṟaÿ,} dealing with the triple principles discussed there.

\textbf{(iii) \textit{Paøamoøi}:} With 400 verses, the text compares the words of Tiruvaÿÿuvar and attaches a simile to it and presents it as verses pronouncing morals.

\textbf{(iv) \textit{Tirikaòugam}:} Each of the100 verses of this work mentions three ideas like the \textit{Tirikaòugam} mixture (\textit{Sukku, tippili} and \textit{miÿagu)}.

\textbf{(v) \textit{Näṉmaëikkaòigai}:} Each verse of this text (100 verses) incorporates four ideas akin to a four-beaded necklace. They also reflect \textit{advaitic} concepts like the need for clarity of thought and the reason for \textit{mokña}.

\textbf{(vi) \textit{Ciṟupaïcamülam}:} Just as the concoction of the five roots cures the \textit{sthülaçaréra}, the 105 verses of this work is a medicine to the \textit{sükñmaçaréra}.

\textbf{(vii) \textit{Elädi}:} In 81 verses, the text expresses six concepts similar to the six-ingredient mixture by the name \textit{Elädi} – \textit{ēlam, lavaìgam, ciṟunävaṟpu, miÿagu, tippili} and \textit{sukku}.

\textbf{(viii) \textit{Äcärakkovai}:} In 100 verses, it recommends people to not lie to elders, the king or to other people and also describes some Vedic doctrines.

\textbf{(ix) \textit{Mudumoÿikkäïci}:} Comprises of 100 proverbs which are mostly one-liners.

\textbf{(x) \textit{Kaÿavaøinäṟpathu}:} In 40 verse deals with the defeat of Kaëaikkälrumporai at the hands of Cōøa and Cēraṉ.

\textbf{(xi) \textit{Iṉitunäṟpathu}:} All the 40 verses of this text teach good values using the word “\textit{iṉitu}” in them.

\textbf{(xii) \textit{Iṉṉänäṟpathu}:} In 40 verses, using the word “\textit{iṉṉä}” in all the verses, it teaches what should be avoided in life.

\textbf{(xiii) \textit{Tiëaimälainüṟṟaimpathu}:} In 150 verses, the five types of the \textit{agattiëais} are described.

\textbf{(xiv) \textit{Aintiëaieøupathu}:} Dedicates 14 verses for each of the five \textit{agattiëais}.

\textbf{(xv) \textit{Kainnilai}:} 12 verses each have been ascribed to each of the five \textit{tiëais}.

\textbf{(xvi) \textit{Aintiëaiyaimpathu}:} Containing 50 verses, this explains the \textit{agattiëais}.

\textbf{(xvii) \textit{Tiëaimoøiaimpathu}:} It has attributed 10 verses each to the five \textit{tiëais.}

\textbf{(xviii) \textit{Kärnäṟpathu}:} The rainy season and the separation of the husband and wife due to war are all recorded here in 40 verses.

Among these eighteen works, the first twelve mostly deal with \textit{Puṟattiëai} and sparsely with \textit{Agattiëai}. Whereas, the later six works are based on \textit{Agattiëai} alone. Some of these 18 are debated to be of a later period.


\subsection*{II. VAIDIKA MÄRGA}

The Vedas are basis for all the Sanskrit Literature. The Vedas not only contain philosophical and spiritual information but range widely from arts, sciences and so on. Therefore scholars declare that one can find information regarding anything in them, for it will have been mentioned in the Vedas.

\subsubsection*{(1) Vedas:}

Vedas are said to have been thought through oral tradition and therefore Tiruvaÿÿuvar calls it the \textit{eøudä kiÿavi}. Vedas are four in number – \textit{Ågveda\index{Agveda@\textit{Ågveda}}, Yajurveda, Sämaveda} and \textit{Atharvaveda} is known by the words Muraïjiyür Muòinägaräyar says: “\textit{Nälvēda neṟitiriyiṉum}” (\textit{Puṟa.,} 2) and Neööimaiyär says: “\textit{Nälvēdattu}” (\textit{Puṟa.,} 15).

The \textit{Ågveda} is in verse form and each verse is called \textit{åk}; \textit{Yajurvedic mantras} are in prose form; the \textit{mantras} of \textit{Sämaveda} are in lyrical form and so are sung and the \textit{mantras} of \textit{Atharvaveda} are mostly like that of the \textit{Ågveda}.

The part of the Vedas that prays to the deities is called \textit{Mantra} and the other part that explains these \textit{mantras} is called \textit{Brähmaëa-s}. Each Veda contains a compilation of \textit{mantra-s} propounded by various \textit{åñi-s} and this section is called the \textit{Samhitä-s}\index{Samhita@\textit{Samhitä}}.

The next part explaining the virtue of Vänaprastha is called the \textit{äraìyaka}. The final part revealing the means to attain \textit{mokña} through the teachings of a preceptor is the \textit{Upaniñad}. The \textit{Brähmaëa-s} and \textit{Samhitä-s} are collectively called as \textit{Karmakäëòa} and the other two together are called as \textit{Jïänakäëòa}.

There are many Upaniñad-s\index{Upaninad@\textit{Upaniñad}} and among them ten are important. \textit{éçäväsya, Kaöha} and \textit{Muëòaka} are rendered in verse form. Others are mostly in prose form. As the \textit{éçäväsyopaniñad} forms the end part of the \textit{çuklayajurveda Samhitä} it is also known as \textit{Mantropaniñad}.


\subsubsection*{(2) Vedäìgas:}

In order to understand the meaning of the \textit{mantras}, the grammar involved, etc., one needed separate texts to guide them. Thus, the Vedäìga-s\index{Vedaiga@\textit{Vedäìga}} were formed. They are six in number \textit{viz.,} – \textit{çikñä} (Phonetics), \textit{Vyäkaraëa} (Grammar), \textit{Chandas} (Prosody), \textit{Nirukta} (Etymology), \textit{Jyotiña} (Astronomy, Astrology) and \textit{Kalpasütras} (Ritual instructions).

\paragraph*{(i) \textit{ Çikñä}:}

As the Vedas were taught orally and pronunciation played a major role, the text on phonetics, phonology or pronunciation was needed. It was named \textit{Çikñä}. \textit{Prätiçäkhyä} is the other name for Vaidika \textit{Çikñä}. For each \textit{çäkhä} of the Vedas, there must have been one \textit{Çikñä} as know from the term \textit{Prätiçäkhyä} but only some of them are available at present.

\textit{Çäkhä} means a branch. Each Veda has been practised by various \textit{åñis} and their family. Only after such \textit{adhyayana} the \textit{mantras} have been compiled together. These have been classified in various \textit{çäkhäs} depending on the difference in the pronunciation and the compilation. Pataïjali’s \textit{Mahäbhäñya} and similar texts record that the \textit{Ågveda} had 21 \textit{çäkhäs, Yajurveda} had 101 \textit{çäkhäs, Sämaveda} had 1000 \textit{çäkhäs} and \textit{Atharvaveda} had 9 \textit{çäkhäs}. Unfortunately, only some of them exist today.


\paragraph*{(ii) \textit{Vyäkaraëa}:}

The study of the words thus pronounced, which contained suffixes and prefixes, was necessary. Thus \textit{Vyäkaraëa} was formed. According to our ancestors there once existed eight \textit{Vyäkaraëa} texts as Päëiëé’s has been considered as the ninth and it seems the other texts perished with the advent of Päëiëé’s work. From the \textit{Yajurveda} we come to know that a grammar text written by Indra existed and from the \textit{Mahäbhäñya}\index{Mahabhanya@\textit{Mahäbhäñya}} it is evident that Båhaspati had written a text called \textit{Saptapäräyaëam}.

The Vedic language differs vastly from the language of Päëiëéyan times. The former is called \textit{Vaidika} and the later as \textit{Bhäñä} by Päëiëé and \textit{Laukikam} by Mahäbhäñyakära. The grammatical work \textit{Añöädhyäyé} of Päëiëé has been written to suit both the types of language.


\paragraph*{(iii) \textit{Chandas}:}

Since the \textit{Samhitäs} of the Vedas were in verse form, a study of the various rules involved in writing poetry was required so \textit{Chandas çästra} was written. \textit{Taittiriya Samhitä, Säìkhyäyana çrautasütra, Sämavedanidänasütra} and \textit{Kätyäyana Sarvänugrahamaëi} deal sparsely with the \textit{Chandas} of the Vedas. The ancient texts that discussed the \textit{Chandas} in detail are not available now. Piìgala’s \textit{Chandas çästra} which enlists the rules for poetry writing is also considered as a Vedäìga text.


\paragraph*{(iv) \textit{Nirukta}:}

Unless the real import of the \textit{mantras} was understood, one will not be able to experience and render them whole-heartedly. The text that helps in discerning the meaning of the Vedic verses is the \textit{Nirukta}.

Yäska’s \textit{Nirukta} is the only available text now, though the existence of some other texts is known from this. Vedic \textit{mantras} had varied readings during early times and so were deciphered differently and some of those readings could not be rightly deciphered even from Yäska’s work.


\paragraph*{(v) \textit{Jyotiña}:}

The \textit{karmas} mentioned in the Vedas are bound by a time-frame. To keep up with time, astronomical texts were needed. The science that deals with it is known as \textit{Jyotiña}. Only some of the \textit{Jyotiña} texts are accessible now.


\paragraph*{(vi) \textit{Kalpa}:}

The methodology of performing the \textit{yajïa-s} and other Vedic rituals have been listed in the texts called \textit{Kalpasütra-s}. They are divided into four – \textit{Çrauta} dealing with Vedic rituals, \textit{Çulba} pertains to the mathematics involved in building \textit{citi-s} or altars, \textit{ Gåhya} discusses the rules for domestic rituals and \textit{Dharmasütra} outlines the customs, duties and laws for living. Based on the \textit{Dharmasütra-s, Småti} texts like the \textit{Manusmåti} were written.


\subsubsection*{(3) \textit{Varëas} and \textit{äçramadharma}:}

The people were divided into four castes as: Brähmaëas, Kñatriyas, Vaiçyas and Çüdras. The four \textit{äçramas} prescribed for the people are \textit{brahmacaryä, gåhastha, vänaprastha} and \textit{sannyäsa}.


\subsubsection*{(4) Occupation of the people:}

The \textbf{six} activities are prescribed for a Brähmaëa are to learn the Vedas, teach it to others, perform \textit{yajïas}, perform rituals for others, give and accept charity. Kñatriyas are ordained \textbf{five} activities – learning the Vedas, performing \textit{yajïas}, giving charity, protecting the citizens and punishing wrong-doers. Vaiçyas also have to learn the Vedas, perform \textit{yajïa}, give charity, protect the cattle, involve in agriculture and trade. A çüdra must engage himself in agriculture, carpentry, sculpting and dancing.

The \textit{Upanayana} is compulsory for the first three \textit{varëas} and so they are called \textit{dvijaù} (or) twice-born. They have to recite the \textit{mantra} thrice, everyday, as prescribed in the scriptures. By such practise the Brähmaëas attain greatness of mind, the Kñatriyas gains strength of body and the Vaiçyas procure immense wealth; for only through such gains can they help others to prosper.


\subsubsection*{(5) Rebirth:}

Yama is the one who separates the \textit{jéva} from the body at the time of one’s death. The means for \textit{jéva} to reach the \textit{pitåloka} is provided by the last rites performed by one’s progeny. Moreover, the food needed for the \textit{pitås} in \textit{pitåloka} is the \textit{piëòa} offered during \textit{çräddha} (last rites).

When a person performs a \textit{homa} to please a deity and makes an offering into the fire, Lord Agni passes it on to the concerned deity. The gods convey their acceptance of the peoples’ prayers in the form of rain.


\subsubsection*{(6) Deities of Worship:}

The deities mentioned in the Vedas and discussed in detail by the \textit{Rämäyaëa}, the \textit{Mahäbhärata} and the Puräëas are Brahmä, Viñëu, Çiiva, Durgä, Balaräma and Skanda.


\subsubsection*{(7) \textit{Karma} and \textit{Mokña}:}

The pleasure and pain that one experiences in this life depends upon the amount of good and bad \textit{karma-s} performed by him in his previous lives. Therefore, if one has to live a happy life later, he has to perform good \textit{karma-s} in this birth. The greatest enemy for realising God is one’s desire. So, one must strive to overcome desire, for when desire perishes one is elevated to the state of \textit{mokña}.


\subsection*{III. VAIDIKA MÄRGA REFLECTED IN SAÌGAM LITERATURE}

This section will pinpoint the references to Vedic practices and principles that are imbibed in the Saìgam literature texts as follows:

\subsubsection*{(1) Vedas and Vedäìgas:}

Nakkérar in his \textit{Tirumurugäṟṟppaòai} (vv. 179-82) records that the twice-born (\textit{irupiṟap päÿar}) learnt the Vedas comprising of six \textit{aìgas} in 48 years:

\begin{verse}
\textit{“aṟunäÿ kiraööi yiÿamai nalliyäëòu}\\\textit{äṟiṉiṟ kaøippiya vaṟaṉavil koÿkai}\\\textit{müṉṟuvagaik kuṟitta muttéc celvattu}\\\textit{irupiṟap päÿar moøutaṟintu nuvala.”}
\end{verse}

Only after the \textit{upanayana} is one fit to learn the Vedas. The Dharmaçästra-s say that the time limit for learning one Veda is twelve years\endnote{\textit{Cf. Gautama Dharmasütra, I. 1. 6, 9; I. 3. 51-2.}}.

Mülaìkiøär in the 166th verse of the \textit{Puṟanäṉüṟu} registers the fact that there are four Vedas, 6 \textit{aìgas} and that they originated from Lord Çiva:

\begin{verse}
\textit{“naṉṟäynta néëimircaòai}\\\textit{mutumudalvaṉ väi pōkädu}\\\textit{oṉṟu purinta vériraëòiṉ}\\\textit{äṟuëarnta vorumutunül.”}
\end{verse}

Kaòuvaṉiÿaveyiṉaṉär’s \textit{Paripäòal} states that, “the 21 worlds and the people therein are protected by Lord Viñëu as per the \textit{mäyäväimoøi}”. In the adage \textit{Mäyävämoøi}, the prefix \textit{mäyä} refers to the eternity of the Vedas.

Päëòiyaṉ Eøuthi Neòuìkaëëaṉär by the term “\textit{eøuthäkkaṟpu}” in his work the \textit{Kuṟunthogai} refers to the unwritten Vedas. The same view is shared by the texts \textit{Padiṟṟuppattu} (vv. 64, 70 and 74) and the \textit{Puṟanäṉüṟu} (v. 361).

The fact that \textit{antaëar-s}\index{antaear@\textit{antaëar}} (Brähmaëas) were the ones who learnt the Vedas and practised them is recorded by Mäìguòi Maruthaṉär in his \textit{Maduraikäïci} (vv. 468-76). Kuṉṟambhüthaṉär in the \textit{Paripäòal} (v. 9) states that, those who explained the Vedic \textit{mantras} in detail were termed as “\textit{väimoøi pulavar}”. Otalänthaiyär in the \textit{Aiìkuṟunüṟu} (v. 387) establishes that the Vedas talk about Dharma – “\textit{aṟampuriyarumaṟai}”.

The parrots that inhabited the place where Brähmaëas lived, also recited the Vedas, says Uruttiraìkaëëaṉär in the \textit{Perumpäëäṟṟuppaòai} (vv. 300-01) as:

\begin{verse}
\textit{“vaÿaiväyk kiÿÿai maṟaiviÿi payiṟṟum}\\\textit{ maṟaikäp päÿaruṟaipati.”}
\end{verse}

The \textit{yajïa-s} and \textit{yajïaçälä-s} mentioned in the Vedas were performed by kings like Mudukuòumipperuvaøuti Cōøa\index{Cooa@\textit{Cōøa}} and Karikäl Peruvaÿanttäṉ, as noted by the \textit{Puṟanäṉüṟu} (15. 17-20; 224. 8) as:

\begin{verse}
\textit{“naṟpaṉuva ṉälvēttu}\\\textit{ aruïcérttip peruìkaëëuṟai}\\\textit{neiymmali yävuti poìkap paṉmäë}\\\textit{véyäc ciṟappiṉ vēÿvi muṟṟi.”}
\end{verse}

\begin{verse}
\textit{“eruvai nukarcci yüpa neòunthüë}\\\textit{vēda vēÿvit thoøiṉmuòit tadüvam.”}
\end{verse}


\subsubsection*{(2) \textit{Varëas} and \textit{Äçramadharma}:}

There existed four \textit{varëa-s}\index{varea@\textit{varëa}} and each had a number of occupations assigned to them. Tolkäppiyar has mentioned this in the \textit{Poruÿadhikäram} (v. 71-2, 78, 81):

\begin{verse}
\textit{“nülē... antaëark kuriya}”\\\textit{“paòaiyum... arasark kuriya”}\\\textit{“vaisikaṉ peṟumē väëika väøkkai”}\\\textit{“vēÿäë mäntark kuøuthü ëallatu\\ illeṉa moøip piṟvakai nikaøcci.”}
\end{verse}

The \textit{Gautama Dharmasütra} and other Dharmasütra-s also mention the various \textit{varëas} and also the activities assigned to them. The \textit{Rämäyaëa}\index{Ramayaea@\textit{Rämäyaëa}} (I. 6. 13) and the \textit{Mahäbhärata} (XIV. 102. 81) how much ever possible a Brähmaëa must curtail himself from accepting charity.

\textit{“Satyam vada}” – is a Vedic rule and this has become a mandatory rule for the Brähmaëas as is evident from the words of Mäìguòi Marutaṉär in his \textit{Maduraikäïci} – “\textit{Poiyyaṟiyä naṉmäntar}”. Brähmaëas always followed Dharma is denoted by the words: “\textit{aṟampuri antaëar}” in the \textit{Padiṟṟuppattu} (v. 24). Tiruvaÿÿuvar calls them “\textit{aṟu toøilōr}” as they observe six activities. Moreover, the \textit{Näṉmaëikkaòigai} (v. 33) says that, to be born as a Brähmaëa was a great boon: \textit{“antaëari nalla piṟappillai yeṉceyiṉun...”}

\paragraph*{(i) Brahmacäri:}

A Brahmacäri wore a \textit{yajïopavéta} (sacred thread\index{sacred thread}) made of nine threads and three strands. He learnt the Vedas in 48 years mentions Nakkérar in his \textit{Tirumurugäṟṟppaòai} (vv. 179, 184). Eṉäthi Neòuìkaëëaṉär in his \textit{Kuṟuntogai} (v. 156) declares that a Brahmacäri bears a Paläça tree's branch in hand, a small \textit{kamaëòala} and partakes of a restricted diet:

\begin{verse}
\textit{“cembu murukki ṉaṉṉär kaÿaintu}\\\textit{ taëòoòu piòitta täøkamaë òalattup\\ paòiva vuëòip pärppaṉa magane.”}
\end{verse}


\paragraph*{(ii) Gåhastha:}

The marriage ritual\index{marriage ritual} is of eight types – Brähmam, Präjäpatyam, ärñam, Daivam, Gändharvam, äsuram, Räkñasam and Paiçäcam – as listed in the \textit{Gautama Dharmasütra} (I. 4. 4-11) and such works. In the \textit{kaÿaviyal} section of the \textit{Tolkäppiyam} (15. 1. 14) it is said that the first four types of marriage belong to the \textit{Peruntiëai}; Gändharvam is equal to \textit{kaÿavu;} äsuram, Räkñasam and Paiçäcam belong to the \textit{Kaikiÿaitiëai}.

Sage Vyäsa gives an interesting information in the \textit{Mahäbhärata} (I. 94.13), that if there is no one to give the bride in \textit{kanyädäna} then the bride could do that ritual by herself:

\begin{verse}
\textit{“ätmano bandhuù ätmaiva gatirätmaiva cätmanaù|}\\\textit{ätmanaivätmano dänam karttumarhasi dharmataù ||”}
\end{verse}

The same dictum is presented by Tolkäppiyar (\textit{Kaṟpu}. 2) as:

\begin{verse}
\textit{“koòuppō riṉṟiyuì karaëa muëòe”}
\end{verse}

If two persons walk seven steps together then they are considered to be friends forever\endnote{\textit{Cf. Kumärasambhava of Kälidäsa, V. 38.}}, records Muòattämakkaëëiyär in the \textit{Porunaräṟṟuppaòai} (v. 166) as: \textit{“käli ṉeøaòip piṉceṉṟu.”}

Likewise, the \textit{Mahäbhärata} (III. 260. 35) also mentions the ritual as:

\begin{verse}
\textit{“satäm säptapadam mitram ähuù santaù kulositäù |”}
\end{verse}

The \textit{Tirikaòugam} (v. 35) states that a Gåhastha is due to three types of \textit{åëa} (debts), \textit{viz., åñi åëa, deva åëa} and \textit{pitå åëa}: \textit{“müṉṟu kaòaṉkaøittu pärppäṉum...”}

The \textit{Taittiréya Samhitä} (VI. 3. 10) records that the \textit{åñi åëa} is repaid by practicing the Vedas; \textit{deva åëa} by performance of \textit{homas} and \textit{pitå åëa} by begetting a progeny:

\begin{verse}
\textit{“jäyamäno vai bramhaëo tribhiù åëavä}\\\textit{jäyate bramhacaryeëa åñibhyaù yajïena}\\\textit{ devebhyaù prajayä pitåbhyaù ||”}
\end{verse}

The \textit{Kalittogai} (v. 2) and the \textit{Perumpäëäṟṟuppaòai} (v. 315-16) refer to the \\Deva debt:

\begin{verse}
\textit{“kēlvi yantaëa raruìkaòa ṉiṟutta vēlvit thüë tasaii”}
\end{verse}

The details of \textit{pitå} debt is mentioned in the \textit{Puṟanäṉüṟu} (v. 9) as:

\begin{verse}
\textit{“teṉpula väønark karukaòa ṉiṟukkum\\ poṉpōṟ pudalvarp peṟävu térum”}
\end{verse}

In the above verse the words: “\textit{teṉpula väønarkku}” denotes the \textit{pitå-}s, for they occupy the southern direction. This is affirmed by the \textit{Taittiréya Samhitä} (IV. 1. 1) \textit{mantra} – “\textit{dakñiëä pitaraù} |”


\paragraph*{(iii) Vänaprastha:}

The Vänaprastha, according to Salliyaìkumaraṉär in \textit{Naṟṟiṉai} (v. 141), possessed long hair and performed penance on the mountains without moving:

\begin{verse}
\textit{“néöiya saòaiyō öäòä mēṉik\\ kuṉṟuṟai tavasiyar.”}
\end{verse}


\paragraph*{(iv) Sannyäsi:}

Tolkäppiyar records that a person who has renounced his family is a Sannyäsi (\textit{Puṟa}. 17): \textit{“aruloòu puërnta vakaṟci yäṉum.”}

\textit{Kaöhopaniñad} (VI. 14) establishes that, one who overcomes desire is eligible for \textit{Jévanmukti}:

\begin{verse}
\textit{“yadä sarve pramucyante kämä ye’sya hådi çritäù |}\\\textit{atha martyo amåto bhavatyatra brahma samaçnute ||”}
\end{verse}

The \textit{Tolkäppiyam} (\textit{Puṟa}. 17), the \textit{Porunaräṟṟuppaòai} (91. 2), \textit{Maduraikäïci} (v. 463-74) and \textit{Tirukkuṟaÿ} (\textit{Kuṟaÿ}. 370) also concur with the same idea and state:

\begin{verse}
\textit{“ärä viyaṟkai yavänéppi ṉannilayē}\\\textit{pērä viyaṟkai tarum.”}
\end{verse}



\subsubsection*{(3) \textit{Agnihotra} and \textit{Yajïas}:}

Here some of the Vedic rituals and sacrifices as articulated in the Saìgam literature will be discussed:

The Brähmaëas performed \textit{agnihotra} with the use of Gärhapatya, ähavanéya and Dakñinägni, twice a day to repay the Deva debt. This fact is recorded in the \textit{Puṟanäṉüṟu} (122. 2-3), \textit{Paööiṉappälai} (v. 200), \textit{Kalittogai} (v. 119), \textit{Kuruïjippäööu} (v. 225) and so on.

\textit{Padiṟṟuppattu} (vv. 70, 74 and 21) mentions the sacrifices performed to please the Devas and gods. The Brähmaëas performed the \textit{yajïas} as prescribed in the \textit{mantras}, is evident from texts like \textit{Tirumurugäṟṟppaòai} (vv. 94-6), \textit{Kalittogai} (v. 36) and \textit{Ciṟupaïcamülam}.

The \textit{Perumpäëäṟṟuppaòai} (vv. 315-16) and \textit{Aganäṉüṟu} (v. 220) record that, during \textit{yajïas}, sacrificial posts (\textit{yüpas}) were constructed and they were fully covered using twisted cloth ropes:

\begin{verse}
\textit{“.................. vēÿvi}\\\textit{kayiṟṟai yätta käëòaku vaṉappiṉ}\\\textit{aruìkaòi neòunthüë pōla.”}
\end{verse}

Many kings of the Saìgam period\index{Saigam period@\textit{Saìgam period}} also performed \textit{yajïas} and it has been described by the poets of that time in their works. The details are as follow:

\paragraph*{(i) Kauṉiyaṉ:}

This king was born in a family which has studied the Vedas, protested against the non-Vedic religions and performed seven \textit{Bhägayajïas}, seven \textit{Haviryajïas} and seven \textit{Somasamsthais}. Just like his forefathers, Kauṉiyaṉ, along with his wife performed various sacrifices and entertained guests. All these details is available in the \textit{Puṟanäṉüṟu} (v. 166). The details of the \textit{Bhägayajïas, Haviryajïas} and \textit{Somasamsthai} are vividly explained in the Dharmasütra texts.


\paragraph*{(ii) Mudukuòumipperuvaøuti:}

As prescribed in the four Vedas, he collected all the ingredients for the \textit{haviñ} and with lots of ghee performed the \textit{yajïas}. To propagate his victory he erected victory pillars or \textit{yüpas} in many places, records \textit{Puṟanäṉüṟu} (v. 15. 17-22):

\begin{verse}
\textit{“.................. puraiyil}\\\textit{naṟ paṉuval näl vēdattu}\\\textit{aruï cérttip peruì kaëëuṟai}\\\textit{neim mali ävuti poìkap, paṉmäë}\\\textit{véyac ciṟappiṉ vēÿvi muṟṟi,}\\\textit{yüpam naööa viyaṉkaÿam palakol?”}
\end{verse}

From the above poem we come to know that the victory pillars were also referred to as \textit{yüpa}\index{yupa@\textit{yüpa}}. Kälidäsa too, in his \textit{Raghuvamsa} (VI. 38ab) uses the word \textit{yüpa} in the same context, when describing Kärtavéryärjuëa:

\begin{verse}
\textit{“saìgräma-nirviñöasahasrabähur-añöadaçadvépanikhätayüpaù”}
\end{verse}

Through the words of Mäìguòi Marutaṉär, in his \textit{Maduraikäïci} (vv. 759-863), it is evident that Peruvaøuti performed all the sacrifices selflessly and attained \textit{cittaçuddhi} (purity of mind), which is the important factor leading to \textit{ätma darçaëa} (Self-realisation). The \textit{Bhagavad Gétä}\index{Bhagavad Geta@\textit{Bhagavad Gétä}} (II. 51) explains that, one who attains \textit{cittaçuddhi} is relieved from re-births and attains the Feet of the \textit{Paramätmä}:

\begin{verse}
\textit{“karmajam buddhiyuktä hi phalam tyaktvä manéñiëaù |}\\\textit{janmabandhavinirmuktäù padam gacchantyanämayam ||”}
\end{verse}


\paragraph*{(iii) Karikäl Peruvaÿattäṉ:}

\textit{Puṟanäṉüṟu} (v. 224) reveals that Peruvaÿattäṉ, with the help of the Brähmaëas, well-versed in Dharma; with åtvik-s, who were experts in performing the sacrifice and with his chaste wife, performed the \textit{Garuòa çyena yajïa} and won laurels:

\begin{verse}
\textit{“aṟam aṟak kaëòa neṟimäë avaiyattu,}\\\textit{muṟainaṟku aṟiyunar muṉṉuṟap pukaønta}\\\textit{tüviyaṟ koÿkait tugaÿaṟu magaÿiroòu,}\\\textit{paruti uruviṉ palpaòaip puricai...”}
\end{verse}


\paragraph*{(iv) Selvakkaòuìkoväøiyätaṉ:}

This king pleased the gods through \textit{yajïas} performed by well-versed Brähmaëas, says \textit{Padiṟṟuppattu} (vv. 64, 70) and that the King rewarded them with lots of ornaments.


\paragraph*{(v) Others:}

Perunaṟkiÿÿi is said to have performed the \textit{Räjasüya yäga} and Nallaìkiÿÿi performed many \textit{yajïas}. This is seen from the verses 363 and 400 of the \textit{Puṟanäṉüṟu} respectively:

\begin{verse}
\textit{“käòupati yägap pōgit, tattam}\\\textit{näòu piṟarkoÿac cenṟumäyn taṉarē”}\\\textit{“kēlvi malinta vēlvit thüëattu”}
\end{verse}



\subsubsection*{(4) Birth and Rebirth:\index{Rebirth}}

All living beings are endowed with an \textit{ätmä} and body; the body is called the \textit{çaréra} and is of three types, \textit{viz., sthülaçaréra} (mortal body), \textit{çükñmaçaréra} (subtle body) and the \textit{käraëaçaréra} (causal body). Till the \textit{ätmä} attains \textit{mukti,} it has only one \textit{çükñma çaréra}, but the \textit{sthüla çaréra} varies based on the world in which the \textit{ätmä} exists. Therefore the \textit{ätmä} will leave the mortal body one day. This is clearly established in the \textit{Puṟanäṉüṟu}, the \textit{Aganäṉüṟu, Perumpäëäṟṟuppaòai} and \textit{Porunaräṟṟuppaòai}.

For the \textit{ätmä} that has left the body to be reborn, the \textit{piëòas} offered by the male progeny during the last rites is the food. In case of no male progeny, the wife will have to offer the \textit{piëòas}, states the \textit{Puṟanäṉüṟu} (v. 234):

\begin{verse}
\textit{“piòiyaòi yaṉṉa ciṟuvaøi meøukit}\\\textit{taṉṉamar kädali puṉmēl vaitta}\\\textit{iṉciṟu piëòam yäìkuë òaṉaṉkol”}
\end{verse}

The \textit{Mahäbhärata} (VII. 173. 54) also vouches for the same:

\begin{verse}
\textit{“icchanti pitaraù puträn svärthahetorghaöotkaca |}\\\textit{iha lokät pare loke tärayiñyanti ye hitäù ||”}
\end{verse}

\textit{Näṉmaëikkaòigai} (v. 15) and other texts record that everyone will reap the consequence of their actions of this birth in their next birth.

\textit{Éçäväsyopaniñad, Bhagavad Gétä} (II. 47), \textit{Puṟanäṉüṟu} (v. 132, 184), \textit{Aganäṉüṟu}(v. 54) and \textit{Padiṟṟuppattu} (v. 38) confirms the fact that doing good deeds without expecting repayment is considered superior.

The practise of ‘Sati’\index{Sati} – entering the funeral pyre of the husband – after the death of the husband (or) to continue living as a widow, were both prevalent for the women as per the verses of the \textit{Puṟanäṉüṟu} (vv. 246 and 234).

\textit{Puṟanäṉüṟu} (v. 93) and the \textit{Bhagavad Gétä} (II. 37) state that a warrior who dies in battle attains heaven:

\begin{verse}
\textit{“néÿkaøaṉ maṟavar celvaøic celka”} and \\\textit{“hato vä prapyasi svargam ||”}
\end{verse}

Indra is the Lord of the Devas. He is said to have performed a hundred \textit{yajïas} according to the \textit{Tirumurugäṟṟppaòai} (v. 155) and \textit{Paripäòal} (v. 9. 8). \textit{Tirukkuṟaÿ} (\textit{ku}. 25) recounts that Indra incurred the curse of the sages as he dishonoured them:

\begin{verse}
\textit{“aintavittä ṉäṟṟa lakalvisumpu ṉärkōmäṉ indiranē cäluì kari”}
\end{verse}

The \textit{Paripäòal} (v. 8) lists the 33 Devas to be – 12 äditya-s, 2 Açvinikumära-s, 8 Vasü-s and 11 Rudra-s. \textit{Pitå-s} are also a type of Deva. The fact that to repay them is to produce a male child and they occupy the southern direction, have already been discussed. The \textit{Maëusmåti} defines them to be very calm, pure and always Brahmacäri-s:

\begin{verse}
\textit{“akrarodhanäù caucaparäù satatam brahmacäriëaù ||”}
\end{verse}

The \textit{Mahäbhärata} (XIII. 86. 8) and the \textit{Kalittogai} (v. 129) explain that all the things on earth gets absorbed into the \textit{paramätmä} at the time of dissolution (\textit{pralaya}).


\subsubsection*{(5) Deities of worship:}

The deities worshipped during Saìgam age\index{Saigam age@\textit{Saìgam age}} are Brahmä, Viñëu, çiva and Muruga. Balaräma and Koṟṟavai (Durgä) were also worshipped.

\paragraph*{(i) Brahmä:}

He is the creator of this Universe and is said to have been born from the navel-lotus of Viñëu. He has four faces and his \textit{vähana} is the Swan. All the above information is available in texts like \textit{Kalittogai} (v. 2), \textit{Perumpäëäṟṟuppaòai} (vv. 402-04), \textit{Tirumurugäṟṟppaòai} (v. 164-65), \textit{Paripäòal} (v. 8. 3) and \textit{Mahäbhärata} (III. 273).


\paragraph*{(ii) Viñëu:}

\textit{Tolkäppiyam} (\textit{Aga}. 5) mentions Viñëu (or) Tirumäl\index{Tirumal@\textit{Tirumäl}} as the presiding deity of the \textit{Mullaitiëai}: \textit{ “mäyōṉ mēya käòuṟai yulagamum”}

The other Tamiø texts portray Him as the bearer of \textit{çaìkha} and \textit{cakra} and Goddess Lakñmé in His bosom; blue-hued lord; lotus-eyed; Garuòa as His \textit{vähana}; the measurer of the three worlds; the destroyer of Prahläda’s father; slayer of the demon Keçi, etc.


\paragraph*{(iii) Çiva:}

\textit{Aganäṉüṟu}\index{Agananuru@\textit{Aganäṉüṟu}} (v. 360. 6) considers Lord çiva and Viñëu as the two main deities:

\begin{verse}
\textit{“veruvaru kaòuìtiṟa liruperun deivattu”}
\end{verse}

The five epics and the various texts of Saìgam literature depict Lord çiva as the one who bears the Ganges and the moon on His head; the three-eyed; bearer of the \textit{maøu} (axe); Våñabha is His \textit{vähana}; as \textit{Nélakaëöha} (blue-throated). Moreover, He is said to have been born under the Tiruvädirai star; seated under the Banyan; Umä is His consort; creator of the five Elements and destroyer of the worlds at the proper time.


\paragraph*{(iv) Muruga:}

\textit{Tolkäppiyam} (\textit{Aga}. 5) designates Lord Muruga\index{Murukan} as the presiding deity of the \textit{Kuruïjittiëai}: \textit{“sēyōṉ mēya maivarai ulagamum.”}

Nakkérar\index{Nakkerar@\textit{Nakkérar}} advises one to worship Lord Muruga in his six ‘\textit{paòaivéòu}’ for attaining \textit{mokña}. This advice establishes that Muruga was worshipped as a warrior god.

It is noteworthy that Lord Kåñëa in the \textit{Bhagavad Gétä} (X. 24) when talking about \textit{ätmavibhüti} mentions Skanda as a warrior: \textit{“senänénäm aham skandaù |”}


\paragraph*{(v) Balaräma:}

The \textit{Puṟanäṉüṟu} (v. 56, 58), \textit{Kalittogai} (v. 104. 7-8), \textit{Paripäòal} (v. 2. 20-3), \textit{Kärnäṟpatu} (v. 19) and \textit{Paøamoøi} (v. 37) describe Balaräma as fair-complexioned; clothed in blue; elder to Kåñëa; having Palm tree as his banner and plough as his weapon.


\paragraph*{(vi) Koṟṟavai:}

Goddess Durgä and Koṟṟavai\index{Korravai@\textit{Koṟṟavai}} are the same. To gain victory in war and for accomplishment of all works, \textit{Tirumurugäṟṟppaòai} (v. 258) and \textit{Neòunalväòai} (v. 166-68), proclaims that people worshipped this deity. This is similar to the instances of Dharmaputra (\textit{Mahäbhärata}, IV. 8) and Arjuna (\textit{Mahäbhärata,} VI. 23) propitiating Goddess Durga before entering into Viräöa and before the beginning of the war respectively.


\subsubsection*{(6) \textit{Karma} or \textit{üø}:}

The consequences of \textit{karmas} or deeds of previous birth are the basis for the pleasures or pains one experience in this life. \textit{Tolkäppiyam} (\textit{Kiÿavi}. 57) denotes it by the terms – \textit{üø, deivam, viṉai, päl}, etc.

Tiruvaÿÿuvar (\textit{Kuṟaÿ}. 380, 619) tries to firmly establish the fact that though our deeds reflect upon our well-being, one’s effort to do Dharma is the real factor; he has dedicated a chapter in his text for \textit{üø}. He further adds that the effect of one’s deeds may continue for the seven births or would effect immediately in this birth itself.

The \textit{Mahäbhärata} (XII. 34. 43) and the \textit{Puṟanäṉüṟu} (v. 34) indicates that some \textit{päpa}s have expiation whereas some do not.


\subsubsection*{(7) Liberation:}

\textit{Mukti} or Liberation is of two types, \textit{viz., videhamukti} and \textit{jévanmukti}. The liberation attained after the \textit{jéva} leaves this body is termed as \textit{videhamukti}. To be in \textit{Brahmaloka} after liberation is called \textit{aparämukti} and to become One with the Brahman is called \textit{parämukti}. Tiruvaÿÿuvar\index{Tiruvayyuvar@\textit{Tiruvaÿÿuvar}} denotes \textit{aparämukti} by the term \textit{nilam} in \textit{Ku.} 3 and \textit{parämukti} by the phrase “\textit{ciṟappeṉṉuï cemporuÿ}” in \textit{Ku.} 358.

Saìkaräcärya\index{Sankara} in his \textit{Båhadäraëyakopaniñad bhäñya} (IV. 4. 23) splits the word ‘\textit{brahmaloka}’ in two ways as – ‘\textit{brahmaëo lokaù}’ and ‘\textit{brahmaiva lokaù}’. Of these the former represents \textit{aparämukti} and the later \textit{parämukti}.

Lord Kåñëa in the \textit{Bhagavad Gétä} (II. 58, 71) and Tiruvaÿÿuvar (\textit{ku.} 6, 352) prescribe that discipline towards God is the right path to \textit{mukti}, for then, ignorance is destroyed and pure knowledge is born.

\textit{Näṉmaëikkaòigai} (v. 100) suggests, love towards all beings, as one of the ways to attain liberation. Nakkérar in his \textit{Tirumurugäṟṟppaòai} (v. 62-4) denotes \textit{mukti} by the term \textit{‘celavu}’. In the similar sense, \textit{Kaöhopaniñad} (III. 11) uses the word \textit{‘gatiù}’.

While \textit{Kaöhopaniñad} (III. 8) uses the term ‘\textit{tatpadam}’ for \textit{mukti}, the \textit{Perumpäëäṟṟuppaòai} (v. 466-67) uses the word \textit{‘annilai}’ in the same effect. The attainment of \textit{mokña} occurs when one overcomes desires as depicted in \textit{Tirukkuṟaÿ} \\(v. 370), \textit{Maduraikäïci} (vv. 468-71), \textit{Kaöhopaniñad} (VI. 14) and \textit{Mahäbhärata} (XIV. 47. 8), has already been discussed before.


\subsection*{IV. SAÌGAM CULTURE IN CONTEMPORARY TAMILNADU}

From the above study of the Saìgam\index{Saigam@\textit{Saìgam}} literature we find several instances of evidence of co-evolution of Vedic and Saìgam civilisation existing together and building upon each other’s contributions to humanity at large. The paper has focussed and provided some of that foundational evidence in the form of textual articulations in Saìgam literature that is so very resonant with Vedic cultural expressions and strongly indicate that the Indic civilisation embraced diversity of local traditions though deeply rooted in the profound pan-Indic ethos.

The question that stares at us in the present day scenario of Tamilnadu is that – whether these are still in vogue, if so – their validity; if not – their need and if needed – how to resolve it?

To my knowledge, with the minimum awareness of two decades of politico-cultural chaotic situation in Tamilnadu, it seems to me that the resurrection of these past values of Indian life is needed. Taking into consideration, the British rule which turned the basic structure of Indian culture topsy-turvy and the Communist party movement and Justice Party\index{Justice Party} movement, which gave room to the Dravidian political movement in Tamilnadu, appear to have undermined the values of our culture. The Dravidian movement has developed into many factions and is still having its effects on the society. However, all these have cast an effect at the superficial level without harming the roots of our culture.

The \textit{Varëa} system exists still, though not exactly in the way elucidated in the scriptures. We have now different \textit{varëas} like teachers, doctors, lawyers, businessmen, soldiers (army, navy and air force), agriculturists and politicians, thereby proving that the earlier \textit{varëa} system which was based on ones vocation still exists. One has to keep in mind that \textit{varëa} and \textit{jäti} are two different terms often translated as ‘caste’. The \textit{äçramadharma} is also in another form as in the life of student, life of a married man, the third and fourth of Vänaprastha and Sannyäsa are now spent in old age homes or parents staying away from their children.

The practice of worshipping in the temples or performing \textit{püja} and \textit{homa}, for the welfare of the mankind, still persists. It seems to have had a period of oblivion during the Independence movement because of other crucial problems. In Tamilnadu, though people have questioned their validity, they had not lost faith in them, for many congregations carrying out mass \textit{püja-s} and \textit{homa-s} exist even today. Temple worship is also on the rise in comparison to its practice in the last few decades. More temples are also being built, in addition to renovating and rejuvenating the dilapidated old structures. In all these, both the people and the government get involved.

With regard to the philosophical concepts of belief in \textit{karma} theory and belief in rebirth, it is still in us, lying dormant. When talking to elders, I realised that the fourth generation anterior to me, though soaked in Indianness, was also open to the world at large.

While it imbibed, valuable principles from other systems, the next two generations caught in the fever of freedom movement and later the overdose of freedom, seemed to have tested their own cultural values. They have been carried away by the forces of the time, yet the generations immediately older to me have started adapting to some changes and going through the history of the previous generations.

In the process, my generation is in an advantageous position of learning from errors of the previous generations. Going through the entire analysis of the pros and cons of the generations, India in a wider perspective, Tamilnadu specifically, are none the worse for all the ups and down that have happened and are happening.


\section*{CONCLUSION:}

When talking about a country, we cannot take into consideration merely the elites, the freaks and the changelings. Whatever their percentage maybe, the mediocre mass is the backbone of the country, for they still remain Indian, even to this day.

So, whatever the political or the sociological or industrial or economical or cultural renovations and reformations that have taken place in India and in Tamilnadu, however much the chaos may appear to be, the undercurrent of the ancient Indian principles and ethos have not yet dried up.


\section*{Bibliography}

\begin{thebibliography}{99}
\bibitem{chap7-key01} \textit{Bhagavadgétä} (1949). Gorakhpur: Gita press.

 \bibitem{chap7-key02} \textit{Brahmasütra} with \textit{çaìkarabhäñya} (1985). Delhi: Motilal Banarsidas.

 \bibitem{chap7-key03} Dikshitar, V.R.R. (1930) \textit{Studies in Tamil Literature and History.} Madras: M.L.J Press.

 \bibitem{chap7-key04} Iyengar, Raghava (1910) \textit{Tirukkuṟaÿ}.

 \bibitem{chap7-key05} Iyer, Swaminatha V. \textit{Pattupäööu}.

 \bibitem{chap7-key06} Kale, M.R. ed. (1981) \textit{Kumärasambhava of Kälidäsa with com. of Mallinätha}. Delhi: Motilal Banarsidas.

 \bibitem{chap7-key07} \textit{Kaöhopaniñad} (1952). Chennai: R.K. Math.

 \bibitem{chap7-key08} \textit{Mahäbhärata (}6 vols.) (1957). Gorakhpur: Gita press.

 \bibitem{chap7-key09} \textit{Manusmåti} with com. of Kullükabhaööa (1946). Bombay: Nirnaya Sagar Press.

 \bibitem{chap7-key10} Neelamalur, M. (1994) Tolkäppiyam with English translation and Critical Studies. Chennai: Educational Publishers.

 \bibitem{chap7-key11} Pansikar, V.L.S. Ed. (1991) \textit{Īçädi Añöottaraçatopaniñad.} Varanasi: Chowkamba Vidyabhavan.

 \bibitem{chap7-key12} Pillai, S.Duraiswami (1947) \textit{Puṟanäṉüṟu}. Chennai.

 \bibitem{chap7-key13} \textit{Ṛgveda with commentary of Säyaëäcärya} (1972) Pune: Vaidika Samsodhan Mandal

 \bibitem{chap7-key14} Sastri, Haragovinda Ed. (2007) \textit{Raghuvasma with Mallinätha’s commentary.} Varanasi: Chowkhamba Sanskrit Samsthan.

 \bibitem{chap7-key15} Varadarajan, M. (1983) \textit{Tamiø Ilakkiyiya Varaläṟu}. Chennai: Sahitya Academy

 \bibitem{chap7-key16} \textit{Taittiréya äraëyaka} (1967) Pune: A.S.S. No. 36.

 \bibitem{chap7-key17} \textit{Taittiréya Upaniñad} (1965) Chennai: R.K. Math.

 \bibitem{chap7-key18} \textit{Välméki Rämäyaëa} (1958) Madras: M.L.J. Press.

 \end{thebibliography}

