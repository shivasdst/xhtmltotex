{\fontsize{8}{10}\selectfont
\begin{landscape}
{\renewcommand{\arraystretch}{1.25}
\begin{longtable}[l]{|p{3.25cm}|p{4.85cm}|p{3.9cm}|p{3cm}|}
\multicolumn{4}{c}{\textbf{Table 3}}\\
\multicolumn{4}{@{}>{\centering}p{15.5cm}@{}}{Source: All the references below are from \url{http://www.sanskritdocuments.org}, while those of the Rig Veda are from \url{http://www.sacred-texts.com/hin/rvsan/index.htm}.}\\
\multicolumn{4}{@{}>{\centering}p{15.5cm}@{}}{The verses of Mahabharata and Ramayana are respectively from \url{http://bombay.indology.info/mahabharata/welcome.html} and \url{http://bombay.indology.info/ramayana/text/UD/Ram01.txt}}\\
\hline
\multicolumn{1}{|c|}{\bf Source} & \multicolumn{1}{c|}{\bf Ārya} & \multicolumn{1}{c|}{\bf Drāviḍa} & \multicolumn{1}{c|}{\bf Comments}\\
\hline
\endfirsthead
\hline
\endhead
\hline
\endfoot
\hline
\endlastfoot
{\bf Rig Veda} & {\bf Book 1 Hymn 51}\hfill\break \devinlineapp{वि {\devinlineapp{\bfseries जानीह्यार्यान}} ये च {\devinlineapp{\bfseries दस्यवो}} बर्हिष्मते रन्धया शासदव्रतान~।} & & \multirow{8}{3cm}{Interplay between Ārya and Dasyu as opposing forces, no reference to any linguistic or geographical association to either term}\\
  & {\bf Book 1 Hymn 59}\hfill\break \devinlineapp{तं तवा देवासो.अजनयन्त देवं वैश्वानर {\bfseries जयोतिरिदार्याय}~॥} & &\\
 &  {\bf Book 1 Hymn 103}\hfill\break \devinlineapp{विद्वान वज्रिन दस्यवे {\bfseries हेतिमस्यार्यं} सहो वर्धया दयुम्नमिन्द्र~॥}  && \\
  & {\bf Book 1 Hymn 117}\hfill\break \devinlineapp{अभि {\bfseries दस्युं} बकुरेणा धमन्तोरु {\bfseries जयोतिश्चक्रथुरार्याय}~॥}  && \\
  & {\bf Book 1 Hymn 130}\hfill\break \devinlineapp{इन्द्रः समत्सु {\bfseries यजमानमार्यं} परावद विश्वेषु शतमूतिराजिषु सवर्मीळ्हेष्वाजिषु~।}  && \\
  & {\bf Book 1 Hymn  156}\hfill\break \devinlineapp{वेधा अजिन्वत तरिषधस्थ {\bfseries आर्यं} रतस्य भागे यजमानमाभजत~॥} && \\
 & {\bf Book 2 Hymn 11}\hfill\break \devinlineapp{अपाव्र्णो{\bfseries र्ज्योतिरार्याय} नि सव्यतः सादि दस्युरिन्द्र~॥} &&\\
  & \devinlineapp{सनेम ये त ऊतिभिस्तरन्तो विश्वा सप्र्ध {\bfseries आर्येण दस्यून}~।}  && \\
  & {\bf Book 3 Hymn 34}\hfill\break \devinlineapp{हिरण्ययमुत भोगं ससान हत्वी दस्यून {\bfseries परार्यंव}र्णमावत~॥}  && \\
  & {\bf Book 4 Hymn 26}\hfill\break \devinlineapp{अहम भूमिम अददाम {\bfseries आर्यायाहं} वर्ष्टिं दाशुषे मर्त्याय~।}  && \\
  & {\bf Book 4 Hymn 30}\hfill\break \devinlineapp{उत तया सद्य {\bfseries आर्या} सरयोर इन्द्र पारतः~।}  &&\\
  & {\bf Book 5 Hymn 34}\hfill\break \devinlineapp{इन्द्रो विश्वस्य दमिता विभीषणो यथावशं नयति दासम {\bfseries आर्यः}~॥}  && \\
  & {\bf Book 6 Hymn 18}\hfill\break \devinlineapp{तवं ह नु तयददमायो दस्यून्रेकः {\bfseries कर्ष्टीरवनोरार्याय}~।} &&\\
  & {\bf Book 6 Hymn 22}\hfill\break \devinlineapp{यया {\bfseries दासान्यार्याणि} वर्त्रा करो वज्रिन सुतुका नाहुषाणि~॥} && \\
  & {\bf Book 6 Hymn 25}\hfill\break \devinlineapp{आभिर्विश्वा अभियुजो {\bfseries विषूचीरार्याय} विशो.अव तारीर्दासीः~॥}  && \\
  & {\bf Book 6 Hymn 33}\hfill\break \devinlineapp{तवं तानिन्द्रोभयानमित्रान दासा {\bfseries वर्त्राण्यार्या} च शूर~।}  && \\
  & {\bf Book 6 Hymn 60}\hfill\break \devinlineapp{हतो {\bfseries वर्त्राण्यार्या} हतो दासानि सत्पती~।} && \\
  & {\bf Book 7 Hymn 5}\hfill\break \devinlineapp{तवं दस्यून्रोकसो अग्न आज उरु जयोतिर्जनय{\bfseries न्नार्याय}~॥} && \\
  & {\bf Book 7 Hymn 18}\hfill\break \devinlineapp{आ यो.अनयत सधमा {\bfseries आर्यस्य} गव्या तर्त्सुभ्यो अजगन युधा नर्न~॥}  && \\
  & {\bf Book 7 Hymn 33}\hfill\break \devinlineapp{तरयः कर्ण्वन्ति भुवनेषु रेतस्तिस्रः परजा {\bfseries आर्या} जयोतिरग्राः~।}  && \\
 & {\bf Book 7 Hymn 39}\hfill\break \devinlineapp{{\bfseries आर्य}मणमदितिं विष्णुमेषां सरस्वती मरुतो मादयन्ताम~॥}  && \\
  & {\bf Book 7 Hymn 83}\hfill\break \devinlineapp{दासा च वर्त्रा {\bfseries हतमार्याणि} च सुदासमिन्द्रावरुणावसावतम~॥}  && \\
  & {\bf Book 8 Hymn 24}\hfill\break \devinlineapp{य रक्षादंहसो मुचद यो {\bfseries वार्यात} सप्त सिन्धुषु~।} && \\
  & {\bf Book 8 Hymn 51}\hfill\break \devinlineapp{यस्यायं विश्व {\bfseries आर्यो दासः} शेवधिपा अरिः~।}  && \\
  & {\bf Book 10 Hymn 11}\hfill\break \devinlineapp{यदी विशो वर्णते {\bfseries दस्ममार्याग्निं} होतारमध धीरजायत~॥}  && \\
  & {\bf Book 10 Hymn 38}\hfill\break \devinlineapp{यो नो {\bfseries दास आर्यो} वा पुरुष्टुतादेव इन्द्र युधये चिकेतति~।}  && \\
  & {\bf Book 10 Hymn 43}\hfill\break \devinlineapp{परैषामनीकं शवसा दविद्युतद विदत्स्वर्मनवे {\bfseries जयोतिरार्यम}~॥}  && \\
  & {\bf Book 10 Hymn 49}\hfill\break \devinlineapp{अहं शुष्णस्य शनथिता वधर्यमंन यो रर {\bfseries आर्यं} नाम {\bfseries दस्यवे}~॥} && \\
  & {\bf Book 10 Hymn 65}\hfill\break \devinlineapp{सूर्यं दिवि रोहयन्तः सुदानव {\bfseries आर्याव्रता} विस्र्जन्तो अधि कषमि~॥}  && \\
  & {\bf Book 10 Hymn 69}\hfill\break \devinlineapp{समज्र्या पर्वत्या वसूनि दासा {\bfseries वर्त्राण्यार्या} जिगेथ~।} && \\
  & {\bf Book 10 Hymn 83}\hfill\break \devinlineapp{साह्याम {\bfseries दासमार्यं} तवया युजा सहस्क्र्तेनसहसा सहस्वता~॥}  &&\\
  & {\bf Book 10 Hymn 86}\hfill\break \devinlineapp{अयमेमि विचाकशद विचिन्वन {\bfseries दासमार्यम}~।}  && \\
  & {\bf Book 10 Hymn 102}\hfill\break \devinlineapp{दासस्यवा {\bfseries मघवन्नार्यस्य} वा सनुतर्यवया वधम~॥} && \\
  & {\bf Book 10 Hymn 138}\hfill\break \devinlineapp{वि सूर्यो मध्ये अमुचद रथं दिवो विदद दासय {\bfseries परतिमानमार्यः}~।}  && \\
\cline{1-3}
{\bf Sāma Veda} & \devinlineapp{४ ७ ३ १९०१}a  \devinlineapp{यस्यायं विश्व {\bfseries आर्यो दासः} शेवधिपा अरिः~।} & & \\
        & \devinlineapp{४ ७ ३ १९०१}c  \devinlineapp{तिरश्चिदर्ये रुशमे पवीरवि तुभ्येत्सो अज्यते रयिः~॥ १६०९} &&\\
\hline%raghu
{\bf Mahābhārata} & 01110023c\newline \devinlineapp{{\bfseries आर्या} सत्यवती भीष्मस्ते च राजपुरोहिताः} &
02028048a \newline \devinlineapp{पाण्ड्यांश्च {\bfseries द्रविडां}श्चैव सहितांश्चोड्रकेरलैः} & \multirow{10}{3cm}{Ārya used as a term of address for a man or woman with cultured refinement and noble qualities, Drāviḍa used in referring to a group of people belonging to the southern Indian region, mostly never used in isolation but while referring to a group of kingdoms from the south.}\\
& 01143005a\newline  \devinlineapp{{\bfseries आर्ये} जानासि यद्दुःखमिह स्त्रीणामनङ्गजम्} & 
02031012a \newline \devinlineapp{{\bfseries द्रविडाः} सिंहलाश्चैव राजा काश्मीरकस्तथा} &\\
& 01169008c\newline  \devinlineapp{{\bfseries आर्यस्त्वेष} पिता तस्य पितुस्तव महात्मनः} &
03048018c \newline \devinlineapp{सवङ्गाङ्गान्सपौण्ड्रोड्रान्सचोल\-{\bfseries द्रविडा}न्धकान्} &\\
& 01194009a\newline  \devinlineapp{{\bfseries आर्य}वृत्तश्च पाञ्चाल्यो न स राजा धनप्रियः} &
03118004a \newline \devinlineapp{ततो विपाप्मा {\bfseries द्रविडेषु} राज;न्समुद्रमासाद्य च लोकपुण्यम्} &\\
&&&\\
& 02049001a\newline  \devinlineapp{{\bfseries आर्यास्तु} ये वै राजानः सत्यसंधा महाव्रताः} &
05138025a \newline \devinlineapp{पुरोगमाश्च ते सन्तु {\bfseries द्रविडाः} सह कुन्तलैः} &\\
& 02069005a\newline \devinlineapp{{\bfseries आर्या} पृथा राजपुत्री नारण्यं गन्तुमर्हति} &
05158020c \newline \devinlineapp{शाल्वैः समत्स्यैः कुरुमध्यदेशै;र्म्लेच्छैः पुलिन्दै{\bfseries र्द्रविडा}न्ध्रकाञ्च्यैः} &\\
&&&\\
&  03013081c\newline \devinlineapp{{\bfseries आर्या}माश्वासयामास भ्रातॄंश्चापि वृकोदरः} &
06010057a \newline \devinlineapp{{\bfseries द्रविडाः} केरलाः प्राच्या भूषिका वनवासिनः} &\\
&&&\\
& 03013083a\newline \devinlineapp{{\bfseries आर्या}मङ्केन वामेन राजानं दक्षिणेन च} &
08004046a \newline \devinlineapp{मालवा मद्रकाश्चैव {\bfseries द्रविडा}श्चोग्रविक्रमाः} &\\
& 03031029a \newline \devinlineapp{{\bfseries आर्य}कर्मणि युञ्जानः पापे वा पुनरीश्वरः} &
08008014c \newline \devinlineapp{सात्यकिश्चेकितानश्च {\bfseries द्रविडैः} सैनिकैः सह} &\\
& 03031038a \newline \devinlineapp{{\bfseries आर्या}ञ्शीलवतो दृष्ट्वा ह्रीमतो वृत्तिकर्शितान्} &
08033004a \newline \devinlineapp{{\bfseries द्रविडा}न्ध्रनिषादास्तु पुनः सात्यकिचोदिताः} & \\
& 03031040a \newline \devinlineapp{{\bfseries आर्य}शास्त्रातिगे क्रूरे लुब्धे धर्मापचायिनि} &
14084011a \newline \devinlineapp{तत्रापि {\bfseries द्रविडै}रन्ध्रै रौद्रैर्माहिषकैरपि } &\\
& 03035014c \newline \devinlineapp{{\bfseries आर्य}स्य मन्ये मरणाद्गरीयो; यद्धर्ममुत्क्रम्य महीं प्रशिष्यात्} &&\\
& 03217009e \newline \devinlineapp{{\bfseries आर्या} पलाला वै मित्रा सप्तैताः शिशुमातरः} &&\\
& 03219040c \newline \devinlineapp{{\bfseries आर्या} माता कुमारस्य पृथक्कामार्थमिज्यते}&&\\
& 03264049a \newline \devinlineapp{{\bfseries आर्याः} खादत मां शीघ्रं न मे लोभोऽस्ति जीविते} &&\\
& 03281048a\newline \devinlineapp{{\bfseries आर्य}जुष्टमिदं वृत्तमिति विज्ञाय शाश्वतम्} &&\\
& 05033025a \newline \devinlineapp{{\bfseries आर्य}कर्मणि रज्यन्ते भूतिकर्माणि कुर्वते} &&\\
& 05037023c \newline \devinlineapp{वक्ता हितानामनुरक्त {\bfseries आर्यः}; शक्तिज्ञ आत्मेव हि सोऽनुकम्प्यः} &&\\
& 05050046a \newline \devinlineapp{{\bfseries आर्य}व्रतं तु जानन्तः संगरान्न बिभित्सवः} &&\\
& 05056039c \newline \devinlineapp{{\bfseries आर्या}न्धृतिमतः शूरानग्निकल्पान्प्रबाधितुम्} &&\\
& 05070056a \newline \devinlineapp{ये ह्येव वीरा ह्रीमन्त {\bfseries आर्याः} करुणवेदिनः} &&\\
& 05082025c \newline \devinlineapp{{\bfseries आर्याः} कुलीना ह्रीमन्तो ब्राह्मीं वृत्तिमनुष्ठिताः}  &&\\
& 05093034a \newline \devinlineapp{शुक्ला वदान्या ह्रीमन्त {\bfseries आर्याः} पुण्याभिजातयः} &&\\
& 05094043c \newline \devinlineapp{{\bfseries आर्यां} मतिं समास्थाय शाम्य भारत पाण्डवैः} &&\\
& 05167010c \newline \devinlineapp{{\bfseries आर्यवृत्तौ} महेष्वासौ स्नेहपाशसितावुभौ} &&\\
&&&\\
& 06010012c \newline \devinlineapp{{\bfseries आर्या} म्लेच्छाश्च कौरव्य तैर्मिश्राः पुरुषा विभो} &&\\
& 06068031c \newline \devinlineapp{{\bfseries आर्यां} युद्धे मतिं कृत्वा भीष्ममेवाभिदुद्रुवुः} &&\\
& 06082030c \newline \devinlineapp{{\bfseries आर्यां} युद्धे मतिं कृत्वा न त्यजन्ति स्म संयुगम्} &&\\
&&&\\
& 07009028a \newline \devinlineapp{{\bfseries आर्य}व्रतममोघेषुं ह्रीमन्तमपराजितम्} &&\\
& 07021002a \newline \devinlineapp{{\bfseries आर्यां} युद्धे मतिं कृत्वा क्षत्रियाणां यशस्करीम्} &&\\
& 07077013c \newline \devinlineapp{{\bfseries आर्यां} युद्धे मतिं कृत्वा जहि पार्थाविचारयन्} &&\\
& 07086011c \newline \devinlineapp{{\bfseries आर्यां} युद्धे मतिं कृत्वा यावद्धन्मि जयद्रथम्} &&\\
& 07100018c \newline \devinlineapp{{\bfseries आर्यां} युद्धे मतिं कृत्वा युद्धायैवोपतस्थिरे} &&\\
& 07118010a \newline \devinlineapp{{\bfseries आर्येण} सुकरं ह्याहुरार्यकर्म धनंजय} &&\\
& 07164009c \newline \devinlineapp{{\bfseries आर्यं} युद्धमकुर्वन्त परस्परजिगीषवः} &&\\
& 07166034a \newline \devinlineapp{{\bfseries आर्येण} तु न वक्तव्या कदाचित्स्तुतिरात्मनः} &&\\
&&&\\
& 08043046c \newline \devinlineapp{{\bfseries आर्यां} युद्धे मतिं कृत्वा प्रत्येहि रथयूथपम्} &&\\
& 11015013a \newline \devinlineapp{{\bfseries आर्ये} पौत्राः क्व ते सर्वे सौभद्रसहिता गताः} &&\\
& 11020014a \newline \devinlineapp{{\bfseries आर्यामार्य} सुभद्रां त्वमिमांश्च त्रिदशोपमान्} &&\\
& 12063008a \newline \devinlineapp{यः स्याद्दान्तः सोमप {\bfseries आर्यशीलः}; सानुक्रोशः सर्वसहो निराशीः} &&\\
& 12156018a \newline \devinlineapp{{\bfseries आर्यता} नाम भूतानां यः करोति प्रयत्नतः} &&\\
& 12221033a \newline \devinlineapp{दातारः संगृहीतार {\bfseries आर्याः} करुणवेदिनः} &&\\
& 12272024c \newline \devinlineapp{{\bfseries आर्यां} युद्धे मतिं कृत्वा जहि शत्रुं सुरेश्वर} &&\\
& 12312015c \newline \devinlineapp{{\bfseries आर्यावर्त}मिमं देशमाजगाम महामुनिः} &&\\
& 12345008a \newline \devinlineapp{{\bfseries आर्य} सूर्यरथं वोढुं गतोऽसौ मासचारिकः} &&\\
& 13048038c \newline \devinlineapp{{\bfseries आर्यरूप}मिवानार्यं कथं विद्यामहे नृप} &&\\
& 13048044a \newline \devinlineapp{{\bfseries आर्यरूप}समाचारं चरन्तं कृतके पथि} &&\\
& 14060025c \newline \devinlineapp{{\bfseries आर्ये} क्व दारकाः सर्वे द्रष्टुमिच्छामि तानहम्} &&\\
& 14072025a \newline \devinlineapp{{\bfseries आर्याश्च} पृथिवीपालाः प्रहृष्टनरवाहनाः} &&\\
& 14068012a \newline \devinlineapp{{\bfseries आर्यां} च पश्य पाञ्चालीं सात्वतीं च तपस्विनीम्}  &&\\
\hline
{\bf Rāmāyaṇa} & 1001015c \devinlineapp{{\bfseries आर्यः} सर्वसमश्चैव सदैकप्रियदर्शनः}&& \multirow{5}{3cm}{Ārya used as a term of address for a man or woman with cultured refinement and noble qualities}\\
&   2020020c \devinlineapp{{\bfseries आर्य}पुत्राः करिष्यन्ति वनवासं गते त्वयि}&&\\
&  2024002a \devinlineapp{{\bfseries आर्य}पुत्र पिता माता भ्राता पुत्रस्तथा स्नुषा}&&\\
&  2031008c \devinlineapp{{\bfseries आर्यो} ह्वयति वो राजा गम्यतां तत्र माचिरम्}&&\\
&  2034027c \devinlineapp{{\bfseries आर्ये} किमवमन्येयं स्त्रीणां भर्ता हि दैवतम्}&&\\
&  2064007a \devinlineapp{{\bfseries आर्या} च धर्मनिरता धर्मज्ञा धर्मदर्शिनी}&&\\
&  2066006a \devinlineapp{{\bfseries आर्य}कस्ते सुकुशलो युधाजिन्मातुलस्तव}&&\\
&  2066028a \devinlineapp{{\bfseries आर्ये} किमब्रवीद्राजा पिता मे सत्यविक्रमः}&&\\
&  2069013a \devinlineapp{{\bfseries आर्ये} कस्मादजानन्तं गर्हसे मामकिल्बिषम्}&&\\
&  2093012c \devinlineapp{{\bfseries आर्यं} द्रक्ष्यामि संहृष्टो महर्षिमिव राघवम्}&&\\
&  2093038c \devinlineapp{{\bfseries आर्ये}त्येवाभिसंक्रुश्य व्याहर्तुं नाशकत्ततः}&&\\
&  2097005a \devinlineapp{{\bfseries आर्यं} तातः परित्यज्य कृत्वा कर्म सुदुष्करम्}&&\\
&  2100002c \devinlineapp{प्राकृतस्य नरस्येव {\bfseries आर्य} बुद्धेस्तपस्विनः}&&\\
&  2103013c \devinlineapp{{\bfseries आर्यं} प्रत्युपवेक्ष्यामि यावन्मे न प्रसीदति}&&\\
&  2103025c \devinlineapp{{\bfseries आर्यं} परमधर्मज्ञमभिजानामि राघवम्}&&\\
&  2107006c \devinlineapp{{\bfseries आर्य}मार्गं प्रपन्नस्य नानुमन्येत कः पुमान्}&&\\
& &&\\
&  3017010c \devinlineapp{{\bfseries आर्यस्य} त्वं विशालाक्षि भार्या भव यवीयसी}&&\\
&  3041009a \devinlineapp{{\bfseries आर्यपुत्रा}भिरामोऽसौ मृगो हरति मे मनः}&&\\
&  3057007a \devinlineapp{{\bfseries आर्येणेव} परिक्रुष्टं हा सीते लक्ष्मणेति च}&&\\
&  &&\\
&  4018031a \devinlineapp{{\bfseries आर्येण} मम मान्धात्रा व्यसनं घोरमीप्सितम्}&&\\
&  4019027c \devinlineapp{सुप्तेव पुनरुत्थाय {\bfseries आर्य}पुत्रेति क्रोशती} &&\\
&  4054007c \devinlineapp{{\bfseries आर्यः} को विश्वसेज्जातु तत्कुलीनो जिजीविषुः}&&\\
&  &&\\
&  5057003a \devinlineapp{{\bfseries आर्यायाः} सदृशं शीलं सीतायाः प्लवगर्षभाः}&&\\
&  5060024c \devinlineapp{{\bfseries आर्यकं} प्राहरत्तत्र बाहुभ्यां कुपितोऽङ्गदः}&&\\
&  5061014a \devinlineapp{{\bfseries आर्य} लक्ष्मण संप्राह वीरो दधिमुखः कपिः}&&\\
&  &&\\
&  6023005a \devinlineapp{{\bfseries आर्येण} किं नु कैकेय्याः कृतं रामेण विप्रियम्}&&\\
&  6061020c \devinlineapp{{\bfseries आर्य} संदर्शितः स्नेहो यथा वायुसुते परः}&&\\
&  6098004a \devinlineapp{{\bfseries आर्य}पुत्रेति वादिन्यो हा नाथेति च सर्वशः}&&\\
&  6114008c \devinlineapp{{\bfseries आर्यस्य} पादुके गृह्य यथासि पुनरागतः}&&\\
&  6115014a \devinlineapp{{\bfseries आर्य}पादौ गृहीत्वा तु शिरसा धर्मकोविदः}&&\\
&  &&\\
&  7051008a \devinlineapp{{\bfseries आर्य}स्याज्ञां पुरस्कृत्य विसृज्य जनकात्मजाम्}&&\\
&  7054012a \devinlineapp{{\bfseries आर्येण} हि पुरा शून्या अयोध्या रक्षिता पुरी}&&\\
&  7054012c \devinlineapp{संतापं हृदये कृत्वा {\bfseries आर्य}स्यागमनं प्रति} & &\\
\hline
{\bf Garuda Purāṇa} & \devinlineapp{अभिमानः सहिष्णुश्च मधुश्रीरृषयः स्मृताः~। {\bfseries आर्याः} प्रभूता भाव्याश्च लेखाश्च पृथुकास्तथा~॥ १,८७.२४~॥}&& \multirow{4}{3cm}{Ārya used as a term of address for a man or woman with cultured refinement}\\
&  &&\\
&  \devinlineapp{सर्वादिमध्यान्तगलौ म्नौ भ्यौ ज्रौ स्तौ त्रिका गणाः~। {\bfseries आर्या} चतुष्कलाद्यन्तसर्वमध्ये चतुर्गणाः~॥ १,२०७.२~॥}&&\\
&  &&\\
&  \devinlineapp{{\bfseries आर्यालक्ष्म} त्वष्ट गणाः सदा जो विषमे न हि~। षष्ठे जो न्लौ वापि भवेत्पदं षष्ठे द्वितीयलात्~॥ १,२०८.१~॥}&&\\
& &&\\
&  \devinlineapp{ग्मध्ये द्वितुर्यौ जौ चपला मुखपूर्वादिचापला~। द्वितीयार्धे सजघना आर्याजातेश्च लक्षणम्~॥ १,२०८.३~॥}&&\\
&  &&\\
&  \devinlineapp{{\bfseries आर्या} प्रथमार्धलक्ष्म गीतिः स्याच्चेद्दलद्वये~। उपगीतिर्द्वितीयार्धादुद्गीतिर्व्यत्ययाद्भवेत्~॥ १,२०८.४~॥}&&\\
&  &&\\
&  \devinlineapp{{\bfseries आर्यागीतिश्चा}न्तगुरुर्गोतिजातेश्च लक्षणम्~। षट्कला विषमे चेत्स्युः समेऽष्टौ न निरन्तराः~। समा पराश्रिता न स्याद्वैतालीये रलौ गुरुः~॥ १,२०८.५~॥}&&\\
&  &&\\
&  \devinlineapp{इति श्रीगारुडे महापुराणे पूर्वखण्डे प्रथमांशाख्ये आचारकाण्डेछन्दः शास्त्रे आर्यावृत्तादिछन्दोलक्षणनिरूपणं नामाष्टोत्तरद्विशततमोऽध्यायः} & &\\
\hline
{\bf Brahmapurāṇa} & \devinlineapp{१५४०२५१ पश्यत्सु लोकपालेषु {\bfseries आर्ये} तत्र प्रवादिनि~।}\newline \devinlineapp{१५४०२५२ अग्नौ शुद्धिगतां सीतां किं तु राजा निरङ्कुशः~॥ १५४.२५।} &&\\
\hline
{\bf Skandapurāṇa} & SP0020181  \devinlineapp{शैलादिदैत्यसम्मर्दो देव्याश्च शतरूपता~।}\newline SP0020182  \devinlineapp{{\bfseries आर्या}वरप्रदानं च शैलादिस्तव एव च~॥} && 
Ārya used as a term of address for a man or woman with cultured refinement and noble qualities\\
\hline
{\bf Nātyaśāstra} &  \devinlineapp{॥ नाट्यशास्त्रम् अध्याय ६~॥}& \devinlineapp{॥ नाट्यशास्त्रम् अध्याय १३~॥} & \multirow{13}{3cm}{Ārya used as a term of address for a man or woman with cultured refinement and noble qualities, Drāviḍa used in referring to a group of people belonging to the southern Indian region, mostly never used in isolation but while referring to a group of kingdoms from the south and other parts of the country.}\\
&  &&\\
&  \devinlineapp{अपि चात्र सूत्रार्थानुविद्धे आर्ये भवतः~।}& \devinlineapp{कोसलाग्ग्स्तोशलाश्चैव कलिङ्गा यवना\hfil\break खसाः~।}&\\
&  \devinlineapp{ऋतुमाल्यालङ्कारैः प्रियजनगान्धर्वकाव्यसेवाभिः~।}& \devinlineapp{{\bfseries द्रविडा}न्ध्रमहाराष्ट्रा वैष्णा वै वानवासजाः~॥ ३९॥}&\\
&  \devinlineapp{उपवनगमनविहारैः शृङ्गाररसः समुद्भवति~॥ ४७॥} & \devinlineapp{॥ नाट्यशास्त्रम् अध्याय २१~॥}&\\
&  \devinlineapp{अथ हास्यो नाम हासस्थायिभावात्मकः~।\hfil\break स च विकृतपरवेषालङ्कारधार्ष्ट्यलौल्यकुहकासत्प्रला\-पव्यङ्गदर्शन-}& \devinlineapp{किरातबर्बरान्ध्राश्च {\bfseries द्रविडाः} काशिकोसलाः~।}&\\
&  \devinlineapp{दोषोदाहरणादिभिर्विभावैरुत्पद्यते~। तस्योष्ठनासाकपोल\-स्पन्दनदृष्टिव्याकोशाकुञ्चनस्वेदा\-स्यरागपार्श्वग्रहणादिभिरनुभावैरभिनयः\-प्रयोक्तव्यः~।}& \devinlineapp{पुलिन्दा दाक्षिणात्याश्च प्रायेण त्वसिताः स्मृताः~॥ ११०॥}&\\
&  \devinlineapp{व्यभिचारिणश्चास्यावहित्थालस्यतन्द्रानि\-द्रास्वप्नप्रबोधादयः~।}&&\\
&  \devinlineapp{द्विविधश्चायमात्मस्थः परस्थश्च~। यदा स्वयं हसतितदाऽत्मस्थः~।}&&\\
&  \devinlineapp{यदा तु परं हासयति तदा परस्थः~।}&&\\
&  \devinlineapp{अत्रानुवंश्ये आर्ये भवतः~।}&&\\
&  \devinlineapp{विपरितालङ्कारैर्विकृताचराभिधानवेषैश्च~।}&&\\
&  \devinlineapp{विकृतैरर्थविशेषैर्हसतीति रसः स्मृतो हास्यः~॥ ४९॥}&&\\
&  &&\\
&  \devinlineapp{भावाश्चास्यासम्मोहोत्साहावेगामर्षचपल\-तौग्र्यगर्वस्वेदवेपथुरोमाञ्च-गद्गदादयः~।}&&\\
&  \devinlineapp{अत्राह - यदभिहितं रक्षोदानवादीनां रौद्रो रसः।}&&\\
&  \devinlineapp{किमन्येषां नास्ति~। उच्यते अस्त्यन्येषामपि रौद्रो रसः~। किन्त्वधिकारोऽत्र गृह्यते~। ते हि स्वभावत एव रौद्रः~। कस्मात्~।}&&\\
&  \devinlineapp{बहुबाहवो बहुमुखाः प्रोद्धूतविकीर्णपिङ्गलशिरोजाः~।}&&\\
&  \devinlineapp{रक्तोद्वृत्तविलोचना भीमासितरूपिणश्चैव~।}&&\\
&  \devinlineapp{यच्च किञ्चित्समारम्भते स्वभावचेष्टितं वागङ्गादिकंतत्सर्वं रौद्रमेवैषाम्~। शृङ्गारश्च तैः प्रायशः प्रसभ्यं सेव्यते~। तेषांचानुकारिणो ये पुरुषस्तेषामपि सङ्ग्रामसम्प्रहारकृतो रौद्रोरसोऽनुमन्तव्यः~।}&&\\
&  \devinlineapp{अत्रानुवंश्ये {\bfseries आर्ये} भवतः -}&&\\
&   \devinlineapp{युद्धप्रहारघातनविकृतच्छेदनविदारणैश्चैव~।}&&\\
&   \devinlineapp{संङ्ग्रामसम्भ्रमाद्यैरेभिः सञ्जायते रौद्रः~॥ ६४~॥}&&\\
&  &&\\
&  \devinlineapp{अथ बीभत्सो नाम जुगुप्सास्थायिभावा\-त्मकः~। सचाहृद्या\-प्रियाचोष्यानिष्टश्रवणदर्शन\-कीर्तनादिभिर्विभावैरुत्पद्यते। तस्यसर्वाङ्गसंहारमुखविकूणनोल्लेखननिष्ठी\-वनोद्वेजनादिभिरनुभावैरभिनयःप्रयोक्तव्यः~। भावाश्चा\-स्यापस्मारोद्वेगावेगमोहव्याधिम\-रणादयः~।}&&\\
&  \devinlineapp{अत्रानुवंश्ये {\bfseries आर्ये} भवतः -} &&\\
&  \devinlineapp{अनभिमतदर्शनेन च गन्धरसस्पर्शशब्द\-दोषैश्च~।}&&\\
&  \devinlineapp{उद्वेजनैश्च बहुभिर्बीभत्सरसः समुद्भवति~॥ ७३॥}&&\\
&  &&\\
&  \devinlineapp{अथाद्भुतो नाम विस्मयस्थायिभावात्मकः~। स च दिव्यजनदर्शनेप्सितमनोरथावाप्त्युपवनदेवकुलादिगमन\-सभाविमानमायेन्द्रजालसम्भावनादि\-भिर्विभावैरुत्पद्यते~। तस्य नयनविस्तारानिमेषप्रेक्षणरोमाञ्चाश्रुस्वेदहर्ष\-साधुवाददानप्रबन्धहाहाकारबाहुवदनचेलाङ्गुलिभ्रमणादि\-भिरनुभावैरभि\-नयःप्रयोक्तव्यः~।}&&\\
&  \devinlineapp{भावाश्चास्यस्तम्भाश्रुस्वेदगद्गदरोमाञ्चावेगसम्भ्रमजडता\-प्रलयादयः~।}&&\\
&  \devinlineapp{अत्रानुवंश्ये {\bfseries आर्ये} भवतः  -}&&\\
&  \devinlineapp{यत्त्वातिशयार्थयुक्तं वाक्यं शिल्पं च कर्मरूपं वा~।}&&\\
&  \devinlineapp{तत्सर्वमद्भुतरसे विभावरूपं हि विज्ञेयम्~॥ ७५॥}&&\\
&  &&\\
&  \devinlineapp{॥ नाट्यशास्त्रम् अध्याय ७~॥}&&\\
&  &&\\
&  \devinlineapp{अत्रानुवंश्ये {\bfseries आर्ये} भवतः} -&&\\
&  \devinlineapp{इष्टजनविप्रयोगाद्दारिद्र्याद्व्याधितस्तथा\newline दुःखात्~।}&&\\
&  \devinlineapp{ऋद्धिं परस्य दृष्ट्वा निर्वेदो नाम सम्भवति~॥ २९॥}&&\\
&  &&\\
&  \devinlineapp{॥ नाट्यशास्त्रम् अध्याय १४~॥}&&\\
&  &&\\
&  \devinlineapp{मात्रागणो गुरुश्चैव लघुनी चैव लक्षिते~।}&&\\
&  \devinlineapp{{\bfseries आर्याणां} तु चतुर्मात्राप्रस्तारः परिकल्पितः~॥ ११८॥}&&\\
&  &&\\
&  \devinlineapp{॥ नाट्यशास्त्रम् अध्याय १५~॥}&&\\
&  \devinlineapp{पथ्या च विपुला चैव चषला मुखतोऽपरा~।}&&\\
&  \devinlineapp{जघने चपला चैव {\bfseries आर्याः} पञ्च प्रकीर्तिताः~॥ १९६॥}&&\\
&  &&\\
&  \devinlineapp{मुखचपला यथा} -&&\\
&  \devinlineapp{{\bfseries आर्यामुखे} तु चपला तथापि चार्या न मे यतः सा किम्~।}&&\\
&  \devinlineapp{दक्षा गृहकृत्येषु तथा दुःखे भवति दुःखार्ता~॥ २१८॥}&&\\
&  &&\\
&  \devinlineapp{पञ्चपञ्चाशदाद्या तु त्रिंशदाद्या तथैव च~।}&&\\
&  \devinlineapp{{\bfseries आर्या} त्वक्षरपिण्डेन विज्ञेयात्र प्रयोक्तृभिः~॥ २२३॥}&&\\
&  &&\\
&  \devinlineapp{विकल्पगणनां कृत्वा संख्यापिण्डेन निर्दिशेत्~।}&&\\
&  \devinlineapp{{\bfseries आर्यागीति}रथार्यैव केवलं त्वष्टभिर्गणैः~॥ २२६॥}&&\\
&  &&\\
&  \devinlineapp{॥ नाट्यशास्त्रम् अध्याय १७ काकुस्व\-रव्यञ्जनः~॥}&&\\
&  &&\\
&  \devinlineapp{{\bfseries आर्येति} ब्राह्मणं ब्रूयान्महाराजेति पार्थिवम्~।}&&\\
&  \devinlineapp{उपाध्यायेति चाचार्यं वृद्धं तातेति चैव हि~॥ ६८॥}&&\\
&  &&\\
&  \devinlineapp{सर्वस्त्रीभिः पतिर्वाच्य {\bfseries आर्यपुत्रे}ति यौवने~।}&&\\
&  \devinlineapp{अन्यदा पुनरार्येति महाराजेति भूपतिः~॥ ८२॥}&&\\
&  &&\\
&  \devinlineapp{{\bfseries आर्येति} पूर्वजो भ्राता वाच्यः पुत्र इवानुजः~।}&&\\
&  \devinlineapp{योषिद्भिरथ काम्येति राजपुत्रेति योधनैः~॥ ८३॥} & &\\ 
\hline
{\bf Manusmṛiti} & \devinlineapp{७।२११अ[२१५ंअ]/ {\bfseries आर्यता} पुरुषज्ञानं शौर्यं करुणवेदिता~।}& \devinlineapp{ं१०।२२अ/ झल्लो मल्लश्च राजन्याद् व्रात्यात्निच्छिविरेव च~।} \%\devinlineapp{[ं।व्रात्यात्लिच्छविरेव च]} & \multirow{7}{3cm}{Ārya along with its forms Āryata and its antonym anaryam used as a term of address for a man or woman with cultured refinement and noble qualities.}\\
&  \devinlineapp{ं७।२११च्[२१५ंच्]/ स्थौललक्ष्यं च सततमुदासीनगुणौदयः~॥} B\devinlineapp{छ्.}S\devinlineapp{छ्॥}& \devinlineapp{ं१०।२२च्/ नटश्च करणश्चैव खसो {\bfseries द्रविड} एव च~॥} B\devinlineapp{छ्.}S\devinlineapp{छ्॥} &\\
&  \devinlineapp{१०।५७अ/ वर्णापेतमविज्ञातं नरं कलुषयोनिजम्~।}&&\\
&  \devinlineapp{ं१०।५७च्/ आर्यरूपमिवानार्यं कर्मभिः स्वैर्विभावयेत्~॥} B\devinlineapp{छ्.}S\devinlineapp{छ्॥}&&\\
&  &&\\
& \devinlineapp{ं१०।६८अ/ तावुभावप्यसंस्कार्याविति धर्मो व्यवस्थितः~।}&& \multirow{6}{3cm}{Drāviḍa used in referring to a group of people belonging to the southern Indian region and in cojunction with other kingdoms}\\
&  \devinlineapp{ं१०।६८च्/ वैगुण्याज् जन्मनः पूर्व उत्तरः प्रतिलोमतः~॥} B\devinlineapp{छ्.}S\devinlineapp{छ्॥} \%\devinlineapp{[ं।जन्मतः ]}&&\\
&  \devinlineapp{ं१०।६९अ/ सुबीजं चैव सुक्षेत्रे जातं सम्पद्यते यथा~।}&&\\
&  \devinlineapp{ं१०।६९च्/ तथाऽर्याज् जात {\bfseries आर्यायां} सर्वं संस्कारमर्हति~॥} B\devinlineapp{छ्.}S\devinlineapp{छ्॥} &&\\
\hline
{\bf Kautilya Arthaśāstra} & 01.14.10 && \multirow{5}{3cm}{Ārya used as a term of address for a man or woman with cultured refinement and noble qualities}\\
&  \devinlineapp{यथा चण्डाल उदपानश्चण्डालानां एव उपभोग्यो नान्येषां, एवं अयं राजा नीचो नीचानां एव उपभोग्यो न त्वद्विधानां {\bfseries आर्याणां}, असौ राजा पुरुषविशेषज्ञः, तत्र गम्यताम् इति मानिवर्गं उपजापयेत्}&&\\
&  01.17.35 &&\\
&  \devinlineapp{यौवन उत्सेकात् परस्त्रीषु मनः कुर्वाणं {\bfseries आर्या}व्यञ्जनाभिः स्त्रीभिरमेध्याभिः शून्यागारेषु रात्रावुद्वेजयेयुः}&&\\
&  02.15.43 &&\\
&  \devinlineapp{तण्डुलानां प्रस्थः चतुर्भागः सूपः सूपषोडशो लवणस्यांशः चतुर्भागः सर्पिषः तैलस्य वा एकं {\bfseries आर्य}भक्तं पुंसः}&&\\
& 
02.25.03 &&\\
&  \devinlineapp{ग्रामाद् अनिर्णयणं असम्पातं च सुरायाः, प्रमादभयात् कर्मसु ञ्जिर्दिष्टानां, मर्यादातिक्रमभयाद् {\bfseries आर्याणां}, उत्साहभयाच्च तीष्क्णानाम्}&&\\
&  03.13.01&&\\
&  \devinlineapp{उदरदासवर्जं {\bfseries आर्यप्राणं} अप्राप्तव्यवहारं शूद्रं विक्रयाधानं नयतः स्वजनस्य द्वादशपणो दण्डः, वैश्यं द्विगुणः, क्षत्रियं त्रिगुणः, ब्राह्मणं चतुर्गुणः}&&\\
&  03.13.05 &&\\
&  \devinlineapp{अथवाऽऽर्यं आधाय कुलबन्धन {\bfseries आर्याणां} आपदि, निष्क्रयं चाधिगम्य बालं साहाय्यदातारं वा पूर्वं निष्क्रीणीरन्}&&\\
&  03.13.13 &&\\
&  \devinlineapp{आत्मविक्रयिणः प्रजां {\bfseries आर्यां} विद्यात्}&&\\
&  03.13.19 &&\\
&  \devinlineapp{{\bfseries आर्यप्राणो} ध्वजाहृतः कर्मकालानुरूपेण मूल्यार्धेन वा विमुच्येत}&&\\
&  04.9.24 &&\\
&  \devinlineapp{परिगृहीतां दासीं आहितिकां वा संरुद्धिकां अधिचरतः पूर्वः साहसदण्डः, चोरड़ामरिकभार्यां मध्यमः, संरुद्धिकां {\bfseries आर्यां} उत्तमः}&&\\
&  04.9.26 &&\\
&  \devinlineapp{तद् एवाक्षणगृहीतायां {\bfseries आर्यायां} विद्यात्, दास्यां पूर्वः साहसदण्डः}&&\\
&  05.3.17 &&\\
&  \devinlineapp{चतुष्पदद्विपदपरिचारकपारिकर्मिकाउपस्था\-यिकपालकविष्टिबन्धकाः षष्टिवेतनाः, {\bfseries आर्ययुक्ता}रोहकमाणवकशैलखनकाः सर्व उपस्थायिनश्च}&&\\
&  09.2.18 &&\\
&  \devinlineapp{{\bfseries आर्याधिष्ठितं} अमित्रबलं अटवीबलात्श्रेयः} & &\\
\hline
{\bf Abhijñānaśākuntalam} & \devinlineapp{सूत्रधारः -- ( नेपथ्याभिमुखमवलोक्य )}&& \multirow{5}{3cm}{Ārya used as a term of address for a man or woman with cultured refinement and noble qualities}\\
&  \devinlineapp{{\bfseries आर्ये} यदि नेपथ्यविधानमवसि अमितस्तावदागम्यताम्~।}&&\\
&  \devinlineapp{( प्रविश्य )}&&\\
&  \devinlineapp{नटी - {\bfseries आर्यपुत्र} इयमस्मि~।}&&\\
&  \devinlineapp{सूत्रधारः -}&&\\
&  - \devinlineapp{{\bfseries आर्य} अभिरूपभूयिष्ठा परिषदियम्~। अद्य खलु कालिदासग्रथितवस्तुना नवेनाभिज्ञानशकुंतलाख्येन}&&\\
&  \devinlineapp{नाटकेनोपस्थातव्यमस्माभिः~।\newline तत्प्रतिपात्रमाधीयतां यत्नः~।}&&\\
&  \devinlineapp{नटी} -&&\\
&  - \devinlineapp{सुविहितप्रयोगतयायार्यस्य न किमपि परिहास्यते~।}&&\\
&  \devinlineapp{सूत्रधारः -- {\bfseries आर्ये} कथयामि ते भूतार्थम्~।}&&\\
&  \devinlineapp{आ परितोषाद्विदुषां न साधु मन्ये प्रयोगविज्ञानम्~।}&&\\
&  \devinlineapp{बलवदपि शिक्षितानामात्मन्यप्रत्ययं चेतः॥२॥}&&\\
& &&\\
&   \devinlineapp{नटी} -&&\\
&  - \devinlineapp{{\bfseries आर्य} एवमेतत्~। अनन्तरकरणीयमार्य आज्ञापयतु~।}&&\\
&  \devinlineapp{सूत्रधारः -- {\bfseries आर्य} साधु गीतम्~। अहो रागबद्धचित्तवृत्तिरालिखित इव सर्वतो रङ्गः~। तदिदानीं कतमत्प्रकरणमाश्रित्यैनमाराधयामः~।}&&\\
&  \devinlineapp{नटी} -&&\\
&  - \devinlineapp{नन्वार्यमिश्रैः प्रथममेवाज्ञप्तमभिज्ञान\-शकुंतलं नामापूर्वं नाटकं प्रयोगे अधिक्रियतामिति~।}&&\\
&  \devinlineapp{सूत्रधारः} -&&\\
&  - \devinlineapp{{\bfseries आर्ये} सम्यगनुबोधितोऽस्मि~। अस्मिन्क्षणे विस्मृतं खलुमया तत्~। कुतः~।}&&\\
&  &&\\
&  \devinlineapp{अनसूया} -&&\\
&  - \devinlineapp{{\bfseries आर्य} न खलु किमप्यत्याहितम्~। इयं नौ प्रियसखीमधुकरेणाभिभूयमाना कातरीभूता~।}&&\\
&  &&\\
&  \devinlineapp{{\bfseries आर्यस्य} मधुरालापजनितो विश्रम्भो मां मन्त्रयते कतम {\bfseries आर्येण}राजर्षिवंशोऽलंक्रियते कतमो वा विरहपर्युत्सुकजनः कुतो देशःकिन्निमित्तं वा सुकुमारतरोऽपि तपोवनगमनपरिश्रमस्यात्मापदमुपनीतः~।}&&\\
&  &&\\
&  \devinlineapp{राजाः} -&&\\
&  - \devinlineapp{वयमपि तावद्भवत्योः सखीगतं किमपि पृच्छामः~।}&&\\
&  \devinlineapp{सख्यौ -- {\bfseries आर्य} अनुग्रह इवेयमभ्यर्थना~।}&&\\
&  \devinlineapp{राजाः} -&&\\
&  - \devinlineapp{भगवान्काश्यपः शाश्वते ब्रह्मणि स्थित इति प्रकाशः~।}&&\\
&  &&\\
&  \devinlineapp{अनसूया} -&&\\
&  - \devinlineapp{शृणोत्व् आर्यः~। गौतमीतीरे पुरा किल तस्य राजर्षेरुग्रे तपसि वर्तमानस्य किमपि जातशङ्कैर्देवैर्मेनकानामाप्सराः प्रेषिता नियमविघ्नकारिणी~।}&&\\
&  &&\\
&   \devinlineapp{प्रियंवदा} -&&\\
&  - \devinlineapp{{\bfseries आर्य} धर्मचरणेऽपि परवशोऽयं जनः~। गुरोः पुनरस्या अनुरूपवरप्रदाने संकल्पः~।}&&\\
&  &&\\
&  \devinlineapp{शकुंतला} -&&\\
&  - \devinlineapp{इमामसम्बद्धप्रलापिनीं प्रियंवदां {\bfseries आर्यायै} गौतम्यै निवेदयिष्यामि~।}&&\\
&  &&\\
&  \devinlineapp{प्रियंवदा} -&&\\
&  - \devinlineapp{तेन हि नार्हत्यङ्गुलीयकमङ्गुलीवियोगम्~। {\bfseries आर्यस्य} वचनेनानृणेदानीमेषा~।}&&\\
&  &&\\
&  \devinlineapp{सख्यौ} -&&\\
&  - \devinlineapp{{\bfseries आर्य} अनेनारण्यकवृत्तान्तेन पर्याकुलाः स्मः~। अनुजानीहिन उटजगमनाय~।}&&\\
&  &&\\
&  \devinlineapp{सख्यौ- {\bfseries आर्य} असम्भावितातिथिसत्कारं भूयोऽपि प्रेक्षणनिमित्तंलज्जामहे आर्यं विज्ञापयितुम्~।}&&\\
&  &&\\
&  \devinlineapp{सख्यौ -- इत इत {\bfseries आर्या} गौतमी~।}&&\\
&  \devinlineapp{गौतमी -- ( शकुंतलामुपेत्य )}&&\\
&  \devinlineapp{जाते अपि लघुसंतापानि तेऽङ्गानि~।}&&\\
&  \devinlineapp{शकुंतला -- आर्ये अस्ति मे विशेषः~।}&&\\
&  \devinlineapp{हला प्रियंवदे {\bfseries आर्यपुत्र}दर्शनोत्सुकाया अप्याश्रमपदं परित्यजन्त्यादुःखेन मे चरणौ पुरतः प्रवर्तेते~।}&&\\
&  &&\\
&  \devinlineapp{गौतमी} -&&\\
&  - \devinlineapp{{\bfseries आर्य} किमपि वक्तुकामास्मि~। न मे वचनावसरोऽस्ति~। कथमिति~।}&&\\
&  &&\\
&  \devinlineapp{{\bfseries आर्यस्य} परिणय एव संदेहः~। कुत इदानीं मे दूराधिरोहिण्याशा~।}&&\\
&  \devinlineapp{शार्ङ्गरवः-- मा तावत्~।}&&\\
&  &&\\
&  \devinlineapp{{\bfseries आर्यपुत्र}} –&&\\
&  \devinlineapp{( इत्यर्धोक्ते )}&&\\
&  \devinlineapp{संशयित इदानीं परिणये नैष समुदाचारः~।}&&\\
&  &&\\
&  \devinlineapp{सनुमती -- नास्ति संदेहः~। महाप्रभावो राजर्षिः~।}&&\\
&  \devinlineapp{प्रथमा} -&&\\
&  - \devinlineapp{{\bfseries आर्य} कति दिवसान्यावयोर्मित्रावसुना राष्ट्रियेण भट्टिनीपादमूलं प्रेषितयोः~। अत्र च नौ प्रमदवनस्य पालनकर्मसमर्पितम्~। तदागन्तुकतयाऽश्रुतपूर्व आवाभ्यामेष वृत्तान्तः~।}&&\\
&  
\devinlineapp{कञ्चुकी -- भवतु~। न पुनरेवं प्रवर्तितव्यम्~।}&&\\
&  \devinlineapp{उभे} -&&\\
&  - \devinlineapp{{\bfseries आर्य} कौतूहलं नौ~। यद्यनेन जनेन श्रोतव्यं कथयत्वार्यःकिन्निमित्तं भर्त्रा वसन्तोत्सवः प्रतिषिद्धः~।}&&\\
&  &&\\
&  \devinlineapp{चतुरिका} -&&\\
&  - \devinlineapp{{\bfseries आर्य} माधव्य अवलम्बस्व चित्रफलकं यावदागच्छामि।}&&\\
&  &&\\
&  \devinlineapp{तापसी -- ( उभौ निर्वर्ण्य )आश्चर्यमाश्चर्यम्~।}&&\\
&  \devinlineapp{राजाः -- {\bfseries आर्ये} किमिव~।}&&\\
&  \devinlineapp{तापसी} -&&\\
&  - \devinlineapp{अस्य बालकस्य तेऽपि संवादिन्याकृतिरिति विस्मिताऽस्मि~।}&&\\
&  &&\\
&  \devinlineapp{हृदय समाश्वसिहि समाश्वसिहि~।}&&\\
&  \devinlineapp{परित्यक्तमत्सरेणानुकम्पितास्मि दैवेन~।} \devinlineapp{{\bfseries आर्यपुत्रः} खल्वेषः~।}&&\\
&  \devinlineapp{राजाः -- प्रिये~।}&&\\
&  &&\\
&  \devinlineapp{( इति यथोक्तमनुतिष्ठति )}&&\\
&  \devinlineapp{शकुंतला -- ( नाममुद्रां दृष्ट्वा )}&&\\
&  \devinlineapp{{\bfseries आर्यपुत्र} इदं तदङ्गुलीयकम्~।}&&\\
&  \devinlineapp{राजाः} -&&\\
&  - \devinlineapp{अस्मादङ्गुलीयोपलम्भात्खलु स्मृति\-रुपलब्धा~।}&&\\
&  \devinlineapp{शकुंतला} -&&\\
&  - \devinlineapp{विषमं कृतमनेन यत्तदाऽऽर्यपुत्रस्य प्रत्ययकाले दुर्लभमासीत्~।}&&\\
&  \devinlineapp{राजाः} -&&\\
&  - \devinlineapp{तेन हि ऋतुसमवायचिह्नं प्रतिपद्यतां लता कुसुमम्~।}&&\\
&  \devinlineapp{शकुंतला} -&&\\
&  - \devinlineapp{नास्य विश्वसिमि~। {\bfseries आर्यपुत्र} एवैतद्धारयतु~।} && \\
\hline
{\bf Kumārasambh-avam} & \devinlineapp{{\bfseries सर्गः १-८}}\newline \devinlineapp{{\bfseries आर्या}प्यरुन्धती तत्र व्यापारं कर्तुं अर्हति~।}\newline \devinlineapp{प्रायेणैवंविधे कार्ये पुरन्ध्रीणां प्रगल्भता~॥ ६।३२॥} & &
\multirow{2}{3cm}{Ārya used as a term of address for a man or woman with cultured refinement and noble qualities}\\
\cline{1-3}
{\bf Raghuvamśam} & \devinlineapp{{\bfseries सर्ग ६}}\newline \devinlineapp{तथागतायां परिहासपूर्वं सख्यां सखी वेत्रभृदाबभाषे~।}\newline \devinlineapp{आर्ये व्रजामोऽन्यत इत्यथैनां वधूरसूयाकुटिलं ददर्श~॥ ६-८२॥} & & \\
\hline
\end{longtable}}
\end{landscape}
}
