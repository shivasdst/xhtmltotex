
\chapter*{Our Contributors \namesinorder{(in alphabetical order of last names)}}\label{contributors}

\section*{Vrinda Acharya}

Gold medallist in M.Com and faculty in management, finance and corporate laws, \textbf{Vrinda Acharya} finally discovered that her calling was Carnatic music. After tutelage from very renowned masters, she has dedicated herself to Carnatic music and has released many CDs, conducts classes and workshops in India and abroad, presented papers at several national and international music and Vedic conferences and won various prizes, awards and titles. She is also a Post- Graduate (M.A) in Sanskrit and is working towards obtaining a Ph.D. Being a good writer, she has been continuously contributing articles pertaining to Indian Culture, Music and Philosophy to various reputed magazines and websites. 

\section*{Yamuna Harshavardhana}

\textbf{Yamuna Harshavardhana} graduated as a Chemical Engineer but worked as one only for a short period. The interest she had in our tradition even while she was a teenager, grew upon her as she grew up. She moved into teaching Indian culture to school students through stories, drama and songs. From there she took off as a Special Educator for children with learning difficulties and is still continuing as one. She writes blogs on our Itihasas as well as on education. Born in Chennai, she now lives in Singapore.

\newpage

\section*{Ravindra Joshi}

\textbf{Ravindra Joshi} is an Engineering research and simulation professional with over 20 years of experience in global MNCs. He has an M.S. in Mechanical Engineering and an M.S. in Engineering Management from Drexel University, Philadelphia. He is Board Member of WAVES (World Association of Vedic Studies) and is also the co-founder of the online Medha Journal http://medhajournal.com. Long time resident in the US, Ravi is also actively involved in the local Hindu temple and is a teacher, mentor and founder of the decade old Sanatana Dharma\break school. He has also presented talks and papers at many conferences, including WAVES, HMEC, and the first Swadeshi Indology conference held in 2016 at IIT Chennai.

\section*{Megh Kalyanasundaram}

\textbf{Megh Kalyanasundaram} was most recently employed as Market Leader at a Fortune 40 firm in China. Alumnus of a globally-top-ranked Indian management school, his specialization has included a study of research methods. His professional experience includes an eight-plus-year stint in China, during which he served a term on the Board of a 1000+ member Indian Association, in addition to full-time Business Development/ Consultative Sales/ Marketing engagements in Education and Technology. Other pursuits include singing and composing music, compering and planning Indic-centric events. Perceptions of Bharat in Global/ World/ Transnational History narratives and Indology are areas of growing interest. He currently blogs @ \url{https://meghk.wordpress.com.}

\section*{Vidyuta Karthikeyan}

\vskip -2pt

\textbf{Vidyuta Karthikeyan} did her PG in Sanskrit from Karnataka State Open University, M.Phil from University of Madras and is currently pursuing Ph.D at KSRI, Chennai. She has made paper presentations at various national and international seminars, including the 16th World Sanskrit Conference at Bangkok on “Amnesty vis-à-vis Sanskrit literature and inscriptions.” She has written articles for journals and has assisted in the publications of the KSR Institute.

\section*{Jayaraman Mahadevan}

\textbf{Jayaraman Mahadevan} is currently serving as Director, Research Department, Krishnamacharya Yoga Mandiram, Scientific Industrial Research Organisation, Chennai. He has a PhD in Sanskrit from the Department of Sanskrit, University of Madras. His thesis was titled The Doctrine of Tantrayukti – A Study. He has presented 20 papers in various National and International conferences and has given talks in Universities, Colleges and Institutions of National Eminence. He has also written books, organised seminars, funded projects and been the guide for students pursuing their PhD. His areas of interest are Yoga, Tantrayukti, Vedanta, Sanskrit Poetic Literature and Manuscript Studies.

\section*{Shanthi Narayanan}

\textbf{Shanthi Narayanan} is passionate about understanding the self through different modalities including symbols work using tarot cards, Transactional Analysis and Coaching. This has helped her to explore and deepen her roots in Vishistadvaita philosophy. By travelling to Divya Desams and by listening to Ubaya Vedantis she is in the process of building her knowledge of the richness of this inclusive, comprehensive philosophy. Her prime focus is understanding the Divya Prabhandam and Vyakhyanams written in Tamil by the acharyas of Vishistadvaita philosophy. She has volunteered at Teach for India, mentored young adults and has set up HR processes for a political party. A thorough HR professional, she has worked with Honeywell, Qualcomm and Software AG especially in process and organization capability improvement.

\section*{Manogna H Sastry}

\textbf{Manogna H Sastry} is Chief Operations Officer as well as being Mentor and Research Associate at Centre for Fundamental Research and Creative Education, Bengaluru \url{http://www.cfrce.com/manognahshastry.htm.} Manogna has an M.S. from the Indian Institute of Astrophysics Bengaluru where her thesis was Inflationary Cosmology. She is a keen student of Sanskrit literature and Indic studies and is a passionate environmentalist and gardener.

\section*{Akshay Shankar}

\vskip -2pt

\textbf{Akshay Shankar} is a voracious reader interested in multiple fields that include ancient Indian history, culture, archaeology, literature, religious studies, philosophy etc. An enthusiastic young Indologist interested in how modern citizens perceive their history; he publishes articles online - on blogs and portals.

\section*{Sundari Siddhartha}

\textbf{Sundari Siddhartha} has a Doctorate in Sanskrit from Delhi University. With 43 years of teaching experience in the Indraprastha College for Women and the Delhi University, she retired as a Senior Reader of Sanskrit in 2003, post which she is serving at the International Headquarters of the Theosophical Society at Adyar, Chennai. She has a working knowledge of Sanskrit, Tamil, Hindi and English, and is currently doing proof-reading in the Editorial department of the society. From 1979 to 2015 she has participated and presented several papers at national and international conferences of Sanskrit, Philosophy, Tamil and Theosophy. Music and dance are also her interests.

\section*{Shrinivas Tilak}

\vskip -2pt

\textbf{Shrinivas Tilak} (PhD history of religions, McGill University, Montreal) is an independent researcher based in Toronto, Canada. He has taught India related courses at Concordia University, Montreal and at McGill University, Montreal, Canada. Dr.\ Tilak’s publications include \textit{Religion and aging in the Indian tradition} (Albany, NY: State University of New York Press, 1989), \textit{Understanding karma in light of Paul Ricoeur’s philosophical anthropology and hermeneutics} (Charleston, SC: BookSurge Publications, 2007), and \textit{Reawakening to a secular Hindu nation: M.S. Golwalkar’s vision of a dharmasāpekşa Hindurāşţra} (Charleston, SC: BookSurge Publications, 2008).

\section*{Bimal Trivedi}

\vskip -2pt

\textbf{Bimal Trivedi}, Team Leader / Senior Numismatist of Mintage World Ltd. has a huge data collection of 45,000 Indian coins, 25,000 world coins, other stamps and currency notes. He set up www.mintageworld.com, an online museum of numismatics and philately. In the course of coin and stamp collection, he learnt reading and writing Brahmi, deciphering coins minted with ancient script Kharoshthi, and also Arabic and Persian legend. He has a Masters in Numismatics \& Archeology, an Advanced Diploma in Business Administration and a BA in Hindi Literature.

