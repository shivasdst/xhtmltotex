
\chapter{Reaffirming Āgama Nigama Samanvaya: Tirumūlar’s Contribution To Dharma}

\Authorline{Dr Shrinivas Tilak}


\section*{Abstract}

Tirumūlar\index{Tirumular@\textit{Tirumūlar}}, one of the sixty-three Nāyanmār\index{Nayanmar@\textit{Nāyanmār}} saints and author of the foundational work of Śaiva Siddhānta\index{Saiva Siddhanta@\textit{Śaiva Siddhānta}}, the \textit{Tirumantiram}\index{Tirumantiram@\textit{Tirumantiram}}, deployed the strategy of \textit{samanvaya} to accommodate the followers of other spiritualities within the fold of [Siva] Dharma while retaining and respecting differences from them. This paper argues that Tirumūlar’s strategizing of \textit{samanvaya}\index{samanvaya@\textit{samanvaya}} can also be used to reaffirm harmony between Āgama and Nigama (the two foundational sources of Hindu Dharma) that modern Western theologians and scholars of Hinduism insist does not exist because they consider Agama and Nigama to be two rival theological systems and as such incapable of existing in harmony under the canopy of Hinduism (their appellation for Hindu Dharma). The paper is presented in the traditional debating format also pursued by Tirumūlar:

‘Pūrvapakṣa-Uttarapakṣa-Siddhānta’ (\textit{Arusamaya Pinakkam}) where the \textit{pūrvapakṣin} stands for non-Hindu scholars of Hinduism (with particular focus on Professor Natalia Lidova\index{Natalia Lidova}) as well as Drāviḍa nationalists\index{Dravida nationalists@\textit{Drāviḍa nationalists}} who employ the Orientalist perspective to divide and undermine Hindu Dharma. Refutation of their arguments (the Uttarapakṣa) is based on the accommodative vision (\textit{samanvaya}) and the relevant data to be found in the \textit{Tirumantiram}. The Siddhānta establishes Tirumūlar’s sterling contribution to Dharma to be skillful adjudication of the claims of the two jockeying parties--Āgama and Nigama. He reaffirms their harmony by juxtaposing them in such a way as to highlight their similarities; yet preserving their unique identities and differences. The paper concludes with discussion of the legacy Tirumūlar left to the followers of Dharma.


\section*{Introduction}

Tirumūlar, one of the sixty-three Nāyanmār saints and author of the foundational work of Śaiva Siddhānta, the \textit{Tirumantiram} (hereafter TM), deployed the strategy of \textit{samanvaya} (it may be translated as the ‘Middle Path’) to accommodate the followers of other spiritualities within the fold of [Śiva] Dharma (while retaining and respecting differences from them)(TM 1558).’\endnote{Tirumūlar refers to them as the ‘group of six’ without identifying them. Standard works on Siddhānta Philosophy (\textit{Sivagnana Siddhiar} for instance) begin with a brief statement and criticism of the tenets of other Indic systems of thought which are grouped into four divisions, each of which comprises six schools of philosophical opinion. In the first group are placed the Lokayata, the four major schools of Buddhism at the time, and Jainism. This group of six is called ‘Purapuram’ (Pillai 1929: 3). It is likely that Tirumūlar had this group in mind.} This paper argues that Tirumūlar’s strategizing of \textit{samanvaya} can also be used to reaffirm harmony between Āgama and Nigama (the two foundational sources of Hindu Dharma) that modern Western theologians and scholars of Hinduism insist does not exist because they consider Āgama and Nigama to be two rival theological systems and as such incapable of existing in harmony under the canopy of Hinduism (their appellation for Hindu Dharma). The paper is presented in the traditional debating format also pursued by Tirumūlar: ‘Pūrvapakṣa-Uttarapakṣa-Siddhānta’ (\textit{Arusamaya Pinakkam}). Judith Martin, who wrote her doctoral dissertation on the \textit{Tirumantiram}, has observed that in section eighteen of Tantra Five of TM, Tirumūlar distinguishes two basic approaches to Dharma, that which is preoccupied with its external form (\textit{puracamaya}) and that which is concerned with its internal form (\textit{utcamaya})(see Martin 1983: 165). It is possible to argue that Tirumūlar identifies and refutes the former as the \textit{pūrvapakṣin} because by directing attention outward it precludes access to the One [Śiva] who dwells within the body.

Pūrvapakṣa (‘the first side;’ ‘\textit{Parapakka}’ in Tamil) is the technical term for the preliminary position, or prima facie view in a metaphysical or philosophical argument put forth in the form of an objection by the \textit{pūrvapakṣin} (opponent; real or imagined). The Uttarapakṣa (\textit{Nirākaram}; also known as ‘\textit{Svapakka}’ in Tamil) refutes the argument put forth by the \textit{pūrvapakṣin}. Siddhānta (‘established end’) is the technical term for the demonstrated and definitive conclusion of the debate. In this paper, the \textit{pūrvapakṣin} connotes the scholars of Hinduism and Draviḍa \textit{nāstika}-s and nationalists that employ the Orientalist perspective to divide and undermine Dharma by rejecting Nigama. Refutation of their arguments (the Uttarapakṣa) is based on the accommodative vision and the relevant data to be found in Tirumūlar’s \textit{Tirumantiram}. The Siddhānta establishes Tirumūlar’s contribution to Dharma in reaffirming the harmony of Āgama and Nigama. The paper concludes with a discussion of the legacy Tirumūlar left to the followers of Dharma.


\section*{Āgama Nigama: two founding sources of\hfill \break Dharma}

Hindus have traditionally understood Dharma to be that by which material prosperity (\textit{abhyudaya}) in this world and welfare in the other world (\textit{nihśreyas}) is assured (\textit{Mīmāmsāsūtra-}s 1:1.1–2). It is recognized that grammatically, \textit{abhyudaya}\index{abhyudaya@\textit{abhyudaya}} and \textit{nihśreyas}\index{nihsreyas@\textit{nihśreyas}} occur as a co-ordinate compound (\textit{dvandva samāsa}); not as a dual--Rāmalakṣmaṇau. This suggests that Dharma encompasses welfare both in this and the other world and that these two goals are not mutually exclusive. Indic cultural, social, and spiritual heritage therefore is rooted in Dharma (known as \textit{Āgamanigamātmaka}), i.e. comprising the twin streams of Āgama\index{Agama@\textit{Āgama}} and Nigama\index{Nigama}. Āgama Dharma therefore refers to the stream that is rooted in the worship of an enshrined image (\textit{mandirāntargata mūrti-pūjā}’)(see Apte). According to Tirumūlar, Āgama-s were revealed from the fifth face of Śiva (in Sanskrit as well as in Tamil) to proclaim Dharma as expounded by Śiva (hence known as Siva Dharma\index{Siva Dharma} in the TM) and which later became available in eighteen languages, Āgama-s expound but one truth of Vedānta-Siddhānta (TM 3, 61, 65, 1429).

Broadly speaking, Nigama refers to the entire Vedic corpus (\textit{sāhitya}; see \textit{Bhāgavata Purāṇa} Canto 1, Chapter 1.3). Nigama Dharma denotes the religio-cultural stream that is rooted in the institution of \textit{yajña} (\textit{yajña-samsthā}). Dharma is deemed to be beginning-less (\textit{anādi}) and without divine or human authorship (\textit{apauruṣeya}\index{apauruseya@\textit{apauruṣeya}}). Speaking of the greatness of the Veda-s (Nigama), Tirumūlar declares that the Veda-s proclaimed the central core of Dharma—that there is no Dharma; barring what the Veda-s say. Because the wise sages refrained from contending the truth of Dharma they were able to attain realization by chanting the \textit{mantra}-s. Brahmā spoke of the Veda-s but only in order for Śiva to reveal them. Brahmā also spoke of the \textit{yajña-s} but only in order for Śiva to reveal them (TM 51–52). The Dharmaśāstra-s\index{Dharmasastra@\textit{Dharmaśāstra}} proposed the stages of life model of an ideal lifestyle in pursuit of Dharma based on Vedic teachings and values which became a standard feature in the Smṛti works such as the \textit{Manusmṛti} \index{Manusmrti@\textit{Manusmṛti}} written in Sanskrit (Nigama). In these works the human lifespan is normatively divided in four overlapping stages of twenty-five years duration each on the basis of differences of pedigree (\textit{varṇa}), age (\textit{vaya})\textit{,}and gender (\textit{linga}).


\section*{Pūrvapakṣa: Orientalist Indology questions Āgama Nigama harmony}

Influenced by the writings of Christian missionaries like Bishop Robert Caldwell\index{Bishop Robert Caldwell} (1819–1891) and G. U. Pope\index{G. U. Pope} (1820–1907), colonial British authorities and administrators in the nineteenth century India were led to believe that hierarchical stratification and inequality were so fundamental and rampant in Hinduism (their distorted appellation for Dharma) that Hindu culture, religion, and society must be understood and explained in terms of higher and lower social strata, forms, and levels that existed in opposition to one another (Bhat 2017, Haran 2015). Beliefs and practices of social groups with higher social status were deemed to be distinct from (and superior to) those of lower-status groups (based on Bahadur et al 2011). Once this basic dichotomy was introduced and actively pursued, it was only a question of time that the groups of Hindus claiming a higher social status for themselves were encouraged to identify themselves as such. They were designated as followers of ‘Brahmanical and Sanskritic’ Hinduism now identified with the Nigama wing. In contrast to this ‘Vedic Brahmanism’ was said to have emerged the peninsular, village-based, vernacular forms of ‘Hinduisms’ that were placed under the Āgama wing. European and American scholars of India and Hinduism internalized this dichotomous view of Hindu Dharma (hereafter Dharma) and produced hundreds of monographs on different aspects of ‘Hinduism’ that deviated considerably from the Hindu self perception of precepts and practices outlined in Dharma.

Sanskritist Natalia Lidova of the Institute of World Literature, Russian Academy of Sciences, Moscow, is one such Orientalist Indologist who presupposes this line of analysis and interpretation in her \textit{Drama and Ritual in Early Hinduism} (1994) whose central thesis is that the \textit{Nāṭyaśātra}\index{Nātyasatra@\textit{Nāṭyaśātra}} of Bharata Muni, a treatise on the theatre in ancient India, reflects the oldest premises of the Āgamic ideology, which came to \textit{replace} (emphasis added) the Vedic Nigama (Lidova 1994: 98,116). In this work Professor Lidova (hereafter Lidova) sustains the artificial division of Dharma into Nigamic/Vedic ‘Religion’ and Āgamic ‘Hinduism’ by dividing Dharma sociologically into (1) Aryan, Vedic, North Indian, higher Sanskritic, Brahminical and (2) Dravidian, Āgamic, South Indian, and Tamilian as two functional correlates of a caste-based hierarchical social structure\index{caste-based hierarchical social structure}. Lidova also claims that the unpopularity of the Vedic ritualism centered on the Śrauta and Soma rites is attested to by the rise of Buddhism and Jainism. Subsequently, the lower strata of the Brahmins who were themselves not part of the elite that performed the Śrauta rites, borrowed elements of the non-Vedic \textit{pūjā}-cult (Lidova 1994: 118).

Lidova argues that despite the similarity of (and interactions between) numerous common components, there existed fundamental differences between \textit{pūjā} and \textit{yajña} in terms of sacrificial structure, symbolism, and theological background (Lidova 1994: 40; Fitzgerald 1996). The ritual preliminaries (\textit{pūrvaranga}\index{purvaranga@\textit{pūrvaranga}}) carried out before the staging of Sanskrit drama, argues Lidova, constituted a break from Vedic \textit{yajña}; they owe more to Āgama-style (Dravidian) \textit{pūjā} (Lidova 1994).\textit{ Pūjā}, which was a non-Aryan and Dravidian rite, became the basis of the Hindu ritual-mythological system, largely at the expense of \textit{yajña} (Lidova 1994: 98). Historical and linguistic information testifies to the non-Aryan origin of the\textit{ pūjā}, which Lidova assumes to be an intrinsic Dravidian rite. While Brāhmaṇa ritual texts describe Vedic ritualism\index{Vedic ritualism} to the last detail, they display utter indifference to the \textit{pūjā} (Lidova 1994: 98). This is because there was a world of difference between the Vedic liturgical practice and Hindu worship (\textit{pūjā}). Vedism and Brahmanism knew no templar edifices, which were obligatory for Hindus who essentially follow the Āgama-s (Lidova 1994: 99).

\subsection*{\textit{Drāviḍa nationalism\index{Dravida nationalism@\textit{Drāviḍa nationalism}} rejects Dharma to exalt Tamil ‘secular’ identity }}

\vskip -8pt

Dharma in traditional Tamil Nadu functioned like a federation that allowed local communities to develop and maintain their own set of \textit{grāma devatā}-s\index{grama devata@\textit{grāma devatā}}, āgama-s, rituals, and customs. The Tamil Hindu society was organized around its local temples with spirituality, social welfare, and entertainment practices interwoven together. This was one of the reasons why British missionaries found it hard to pursue their conversion agenda. Bishop Caldwell, an evangelist for the Society for the Propagation of the Gospel, sought to break the synergistic duo of Āgama and Nigama emanating from Dharma into Vedic Aryan practices and Dravidian rituals. To perpetuate this division, Caldwell proposed the existence of the Dravidian race in his A Comparative Grammar of the Dravidian Or South-Indian Languages (1875). The term ‘draviḍa’ in Sanskrit refers to the communities located in the region south-of the Vindhya-range of mountains, which Caldwell adopted from a seventh century Sanskrit-text and modified it to ‘Dravidian’ as a ‘race.’ Caldwell therefore may be called ‘the founder’ of the Dravidian ‘racial’ identity, which ultimately created discord between what he called ‘indigenous’ Dravidians of the south and ‘outsider’ Sanskrit loving Aryan/Brahmins from the north. This ingenious manipulation stigmatized the Brahminical, Sanskritist ‘northerners’ as the colonizers and Caldwell and other missionaries as ‘Saviors’ of the ‘colonized’ Tamil ‘southerners.’ Caldwell also instigated ‘Dravidians’ to disown all that is/was Sanskrit-based and re-discover themselves through Biblical categories\index{Biblical categories} (see Bhat 2017, Haran 2015).

Chandra Mallampalli, an Indian Christian scholar, too observes that champions of Dravidianism and non-Brahminism drew upon the cultural and linguistic resources provided by missionaries such as Robert Caldwell and G.U. Pope in order to foster a distinct ‘Dravida’ identity\index{Dravida’ identity} and nationality (Mallampalli 2004: 108). Peter van der Veer, an expert on South-Asia, similarly observes that Aryan invasion and subjugation of ‘indigenous’ Dravidians was Caldwell’s invention who was resentful of Brahmins because they stood as barriers to Caldwell’s mission of conversion\index{mission of conversion} (see Bhat 2017). After independence of India, the Dravidian movement\index{Dravidian movement} picked this very issue successfully persuading some Tamils to shun Vedic Aryan/Brahmanical practices and ‘reclaim’ an indigenous Dravidian, ‘secular’ identity. Pongal, for example, has been rebranded and ‘secularized’ as Tamil Day and an earnest attempt is being made to disconnect Pongal festivities from the rural Hindu temples where they had always belonged (Haran 2015).


\section*{Uttarapakṣa}

\vskip -6pt

Indologists like Lidova and Dravidian nationalists\index{Dravidian nationalists} dismiss the extant harmony between Āgama and Nigama as \textit{syncretic} i.e. random, corrupting, and superficial on the basis of Western theological disputes concerning syncretism which came to be regarded as a betrayal of principles or as an attempt to secure unity at the expense of truth (see Berling 1980: 4). Then they proceed to divide Āgama and Nigama generating in the process the following four secondary divides: Ārya versus Drāviḍa, Siddhānta versus Vedānta, Tamil versus Sanskrit, and \textit{pūjā}\index{puja@\textit{pūjā}} versus \textit{yajña.}\index{yajna@\textit{yajña}} This paper argues that using the spirit of accommodation (\textit{samanvaya}\index{samanvaya@\textit{samanvaya}}) displayed in Tirumular’s \textit{Tirumantiram}, the disharmony and distance introduced by Indologists and Orientalists such as Natalia Lidova between Āgama and Nigama can be removed and their harmony re-affirmed.

\subsection*{\textit{Tradition of Āgama Nigama samanvaya}}

\vskip -6pt

A popular imagery describes Āgama and Nigama as two wings of the bird of Dharma. In order to fly, the functioning of both must be harmonized. From the Āgama perspective, the tradition explains the relation between the two in terms of the maxim of the ‘seed and sprout’ (\# 139 \textit{Bījānkuranyāya}) as explained in \textit{Nyāyāvali} (1980): Āgama is the root (or seed) of the tree of knowledge with its branches and fruit (Nigama; \textit{Vedavṛkṣa}). The relation of mutual causation subsists between the seed (Āgama) and sprout (Nigama), seed being the cause of sprout, which in turn, is the cause of the seed. From the Nigama perspective, Kullūkabhaṭṭa in his commentary on the \textit{Manusmṛti} (\textit{Mānvārthamuktāvali}; 2:1) cites \textit{Harita Dharmaśāstra} to the effect that Dharma is legitimated through Śruti, which is twofold: Vaidika (i.e. Nigama) and Tāntrika\index{Tantrika@\textit{Tāntrika}} (i.e. Āgama).


\subsection*{\textit{Tirumūlar and Tirumantiram}}

\vskip -6pt

Not much is known about the real persona of Tirumūlar, the author of the TM. Scholars date Tirumūlar between late sixth and early seventh centuries CE though the tradition says that he lived for three thousand years and composed one verse each year, thus totaling three thousand verses of the TM. There are legendary accounts of his being a \textit{yogin} and a mystic who resided in the Himalayas on Mount Kailāsa. One day, the \textit{yogin} desired to see his friend, Sage Agastya\index{Agastya}, in his \textit{āśrama} located in the Pothia hills. So he left Kailāsa and traveled southwards. While walking along the bank of the Kāverī River, he saw a herd of cows shedding tears over the dead body of the cowherd. Feeling sorry for the cows, the \textit{yogin} entered the body of the cowherd by his yogic powers after safely depositing his own body in the trunk of a tree. The cows were rejoiced. The dead cowherd was known as Mūlan from the village of Sathanur in Thanjavur District. The \textit{yogin}, who had reanimated Mūlan’s body, therefore came to be known as Tirumūlar (Tiru is a suffix that is attached to the name of one worthy of great honor). The next day, Tirumūlar could not find his body where he had left it. He was convinced that Śiva wanted Tirumūlar to compose in verse form a work on Śaiva philosophy in Tamil, containing the essence of all Śaiva Āgama-s. Tirumūlar therefore went in the state of \textit{samādhi}, which lasted for three thousand years. But, every year, he would come out of \textit{samādhi} to compose one verse. This collection of three thousand verses is known as the \textit{Tirumantiram}\index{Tirumantiram@\textit{Tirumantiram}} (see Swami Sivananda 1999).

Tirumūlar himself has narrated the account of his life and mission in Tantra (chapter) One of TM in the manner of an autobiography (TM 73–94), which many modern Hindus will find ‘legendary.’ But what is more important to ponder is the theme of \textit{samanvaya} that pervades the narrative: A Sanskrit knowing \textit{Siddha yogin} and an ‘\textit{ārya}’ living in the north (Himalayas) comes south to the ‘Drāviḍa’ land at the wish of Śiva and composes a poem in Tamil whose central message is that Āgama and Nigama are one (it is important to remember in this context that neither Ārya nor Drāviḍa are racial terms; see below). The theme of harmony continues to operate at another level in that the nine \textit{tantra}-s (chapters) deal with how to live a spiritual life in the midst of the worldly one. Tamil Śaiva tradition therefore treats the TM mainly as a Śaiva Siddhānta work that transcends monism and pluralism. Finally, the fact that the \textit{Tirumantiram}is the only \textit{Tirumurai,}which is deemed to be both a\textit{ cāttira} (\textit{śāstra}) and a\textit{ tottira} (\textit{stotra}) in Tamil Śaiva tradition, implies that philosophy and devotion are harmonized in this work.\endnote{According to the Śaiva Tradition of Tamilnadu, there are twelve \textit{Tirumurais,} i.e., sacred Tamil Śaiva texts and Tirumūlar’s TMconstitutes the 10th \textit{Tirumurai}. All the \textit{Tirumurais} are called \textit{tottira}-s (\textit{stotra}-s – devotional literature), which constitute the \textit{bhakti} literature of Tamil Śaivism. The philosophical literature of Tamil Śaivism is called \textit{cattira}-s (\textit{śāstra}-s – philosophical treatises).}


\subsection*{\textit{Samanvaya: a strategy of reconciliation}}

In the Indic thought world, the strategy of \textit{samanvaya} (reconciliation = \textit{camātāna} in Tamil) proceeds on the assumption that there is an underlying similarity, connection or relation between two or more entities, ideas, concepts or even traditions that on surface appear distant, distinct, or contradictory. The awareness of such a potential connection and similarity is what justifies any attempt to reconcile differing entities or ideas. \textit{Samanvaya} as a strategy of harmonization in this sense first occurs in the \textit{Brahmasūtra}, a text in four chapters wherein the author Bādarāyaṇa summarized, systematized, and reconciled the teachings of the Upaniṣads identifying the consistent themes among a variety of doctrines running through them. The first four aphorisms of the first chapter form a special unit of this work where a summary of the content of the entire work is provided in four aphorisms. \textit{Samanvaya} is formally defined as \textit{samyak anvaya}, i.e. setting up or establishing a proper sequence or order between two or more entities or ideas by removing any distortions or deviations that threaten or impede a harmonious relationship or among them.

The self and its proper identification with others constitute the foundation for feelings of solidarity. The movement from ‘I’ to ‘All’ that is predicated upon the key statement (\textit{mahāvākya}) of Chāndogya upaniṣad ‘that thou art’ (\textit{6:8.7; tat tvam asi}) suggests the continuum of the Āgama thought with its Vedic heritage. Tirumūlar avows that ‘\textit{tat tvam asi} (You-being-I)’ of Vedānta is the same as ‘\textit{Thom-tat-asi} (I-Śiva-becoming)’of Siddhānta-Vedānta (TM 2571). The \textit{Śvetāśvatara upaniṣad}, which describes the ultimate reality as Śiva, suggests another point of linkage between Nigama and Āgama. Pippalāda, composer of the \textit{Praśna upaniṣad}, also authored the \textit{Vāstuśāstra upaniṣad}\index{Vastusastra upanisad@\textit{Vāstuśāstra upaniṣad}}, an important Āgama text on temple architecture and building, in which he refers to the icon maker (\textit{sthāpaka}) as \textit{śilpahotṛ} and \textit{vāstuhotṛ} (Ghose 1997: 142). There is thus a clear desire to move away from narrow self-interest to the recognition of the feeling of sameness (\textit{samarasatā}; see below)of belonging to a larger group or community. A feeling of reciprocity and interdependence also constitutes the basis for solidarity with others. Social and cultural affiliations are a further expansion of the self with others. Finally, a more universal category is reached in the declaration ‘the entire world is a family’ (TM 407, 3041)where altruism encompassing all human beings finds expression.

\newpage


\subsection*{\textit{Samanvaya and syncretism}}

The role of \textit{samanvaya}is comparable to the Western category of syncretism \textit{as it was originally understood.} The \textit{Oxford English Dictionary} defines syncretism as attempted union or reconciliation of diverse or opposite tenets or practices, especially in philosophy or religion. Syncretism derives from the ancient Greek prefix \textit{syn} ‘with,’ and \textit{krasis}, ‘mixture.’ Plutarch\index{Plutarch} refers to the practice of the Cretans, who, though they often quarreled among themselves, made up their differences and united when outside enemies attacked. This practice was called ‘syncretism’ (Shaw and Stewart 1994: 3). Thus the concept of syncretism began its history with positive connotations, referring to a strategically practical and morally justified form of allegiance and reconciliation. Syncretism is not arbitrary or irrational, but serves a definite purpose (Berling 1980: 6–7). \textit{Samanvaya}, the Indic equivalent of syncretism, is similarly fundamental to the dynamics of Indic cultural and social interaction through time. It is not an indiscriminate or aimless combination of practices and ideas as Indologists and Orientalists argue. \textit{Samanvaya} appears ‘perfidious’ to the Indologists/Orientalists only because they view it against the backdrop of the credal and exclusivistic style of the Orientalizing culture that dominates Semitic and Western history.


\subsection*{\textit{Samarasatā: the state of harmony resulting from samanvaya}}

\textit{Sama}(literally same or equal) is the goal in both Āgama and Nigama streams and in philosophies and spiritualities inspired by them. In the \textit{Bhagavad Gītā\index{Bhagavad Gita@\textit{Bhagavad Gītā}} yoga} is defined as that which promotes evenness (\textit{samatva}) and Kṛṣṇa describes himself to be the same and equal (\textit{samoham}) toward all beings (BG 2:48, 9:29). Despite the great range of diversity operating at the phenomenal level, all beings share a common essence or life force (\textit{rasa}) although the specific nature of this essence is variously conceived. Brian K Smith refers to the \textit{Kauśītakī Brāhmaṇa} (2:7), which states that the essence of life flows out of one entity into all others vitalizing each of them. The result is a kind of a complex chain of being in which all entities, including water, plants, and humans are bound together (see Smith 1994: 209–210). The \textit{Tamil Lexicon} gives the meaning of \textit{camaraca}(\textit{samarasa}) as equality, harmony, and identity (\textit{oṛṛumai}). A second understanding of this term is impartiality (\textit{naṭunilaimai}). The \textit{Tamil-English Dictionary} similarly explains \textit{camaraca} to mean reconciliation (\textit{camātāna}) and equality (Manninezhath 1993: 153–154). That the Tamil inspired ideal of \textit{samarasatā} has pan-Indian applicability is evident from the fact that over the centuries it became thoroughly internalized in the languages, cultures, and societies prevalent in all parts of India.

Orientalist scholars studying ‘Hinduism’ must take note that Tirumūlar did not set out systematically to integrate \textit{all}pronouncements in the Upaniṣad-s on \textit{brahman}. The result would have been a hybrid monster. He applied the strategy of harmonization (\textit{samanvaya}) selectively to reduce the major sources of disagreement between Āgama and Nigama without collapsing one totally into the other. Tirumūlar’s basic contention was that divergent precepts and practices espoused in the\break Āgama and Nigama were to be judged not in relation to one another but in terms of the overall \textit{telos} or ethos of Dharma which included both. That is, the relative merits of the differing paths of Āgama and Nigama were not to be determined a priori but only in terms of their actual effectiveness in leading aspirants to true knowledge of Dharma and Śiva.


\subsection*{\textit{Instituting samanvaya through dāna}}

Tirumūlar avers that the ‘Middle Path’ (the path of \textit{sananvaya}) is the path of wisdom and of justice on which Brahmā (who gave the four Veda-s), Viṣṇu, and Śiva walk. He himself walked the same path in the noble fellowship of aspirants to attain the state of being-with-Śiva (TM 320–322). The Āgama-s, composed in Tamil and in Sanskrit, provide guidance for the way of life based on Śuddha Śaiva Siddhānta (TM 1422). To Tirumūlar, the Veda-s and the Āgama-s are both true and both are the word of Siva. The first is a general treatise and the second a special one. His strategy here is to institute harmony by validating the Āgama-s as a later ‘special revelation’ of the Veda-s that make explicit for the devotee of Siva the obscure liberating content of the Veda-s. Understood aright (i.e. as taught by Tirumūlar), Āgama-s re-present the Veda-s more clearly in and for a different (i.e. the Tamil) context. This may be illustrated with reference to the act of giving (\textit{dāna}\index{dana@\textit{dāna}}), which is recognized as a sacred duty in both Āgama and Nigama traditions.

\newpage


\subsection*{\textit{Dāna: a practice common to Nigama and Āgama}}

In the ṛgveda, the \textit{sūkta}-s known as (\textit{dānastuti}) praise the act of giving; (5: 27,31,34,38 for instance). The \textit{Bhikṣusūkta} (10:117) praises those who offer food to the hungry (\#1–2) but condemns those who do not (\#4), and warns of the futility of hoarding wealth (\#8–9). The \textit{Dajsuba sūkta} (1:187, 17: 117) promise various merits to those who give to the needy, the poor, and the hungry. The \textit{Annasūkta} (hymn in praise of food; ṛg 1:187; 17:117) assures that those who give food to the hungry acquire great merit and high social rank. Here, the transaction initiated by giving (\textit{dāna}) operates as a mechanism of maintaining the social harmony that is liable to be constantly threatened. By engaging in \textit{dāna} as part of the performance of \textit{yajña}, the rich and powerful can atone or redeem themselves from social obligation. By giving away a portion of their claim of ownership on part of their assets, the rich and powerful can retain the moral right to enjoy the balance. In the post-Vedic Purāṇa-s giving is particularly singled out as the most meritorious act, especially in the present Kaliyuga (the ‘Dark Age’ of Kali). In the Āgama-s \textit{dāna} is similarly understood but it implies more than giving in order to protect one’s assets. One gives in the spirit of love to one’s chosen deity (\textit{iṣṭadevatā}) sharing one’s resources with others, be it wealth, food or other assets. It may involve giving to philanthropic causes – providing rest-houses, planting trees, digging wells. It is important to remember that while the explanatory mechanism behind the act of giving differs: \textit{yajña} in the Nigama tradition and \textit{bhakti} in the Āgama tradition; the act of giving as an expression of Dharma is endorsed in both traditions. Orientalist Indologists\index{Orientalist Indologists} and Drāviḍa nationalist both ignore this fact stressing only difference and disharmony between the Āgama and Nigama traditions.


\subsection*{\textit{Sivayanama: giving for Dharma in the name of Śiva}}

With particular reference to the Śaivāgama tradition, Tirumūlar insists that \textit{dāna} implies giving in the spirit of love for Śiva. Tirumūlar concurred with the stages of life model prescribed in the Dharmaśāsastra-s but went further in arguing that Sanyāsāśrama,\break the final stage, did not involve total withdrawal of the \textit{sanyāsin}from the society or from his social responsibilities. The TM, accordingly, is concerned to shows how progress in the spiritual life (to be carried out in the third and fourth stage of life) can also promote values that are consonant (and continuous) with the everyday, material life (to be pursued in the second stage of life). \textit{Sanyāsa} primarily involves renouncing egotistic desires. This can be illustrated with reference to the \textit{mantra} ‘Sivayanama,’ that occurs multiple times in the Tantra nine of TM (TM 2698–2721). This five-lettered (\textit{pancākṣara}) \textit{mantra} is the abode of Śiva (as ‘Lord of Veda-s’) in his subtle form and when reversed (i.e. Namasivaya), the same \textit{mantra}is abode of Śiva’s manifest form. According to T. N. Ganapathy:

\begin{myquote}
Sivayanama has a social significance in addition to its surface meaning. \textit{Nama} means spirit of renunciation. Śiva means bliss; and \textit{aya} means outcome or result. The \textit{mantra} Sivayanama therefore means ‘the result of sacrifice is bliss.’ Tirumūlar felt bliss in this provision of sacrifice and construed it as an opportunity to serve (Ganapathy 1993: 190).
\end{myquote}

It is necessary to elaborate Ganapathy’s very insightful, though terse, interpretation. Spirit of renunciation (\textit{tyāga}) is an integral element of \textit{yajña} (it is preferable to use this original term in place of ‘sacrifice’ used by Ganapathy because its Christian meaning could be misleading) and means willingly renouncing one’s right of ownership on something of value and offering it to the needy as part of the performance of \textit{yajña}.

Thus, in devising the socially meaningful mantra Sivayanama, Tirumūlar brought together the best from Nigama and Āgama traditions; i.e. the notions of \textit{tyāga} and \textit{bhakti} respectively. In this context it is important to remember that etymologically \textit{bhakti} comes from the Sanskrit verbal root \textit{bhaj}, which means \textit{sevā} (service) in addition to devotion to one’s deity (\textit{bhaj sevāyām}; Chari 1997: 153). Thus, besides serving one’s chosen deity, \textit{bhakti} also implies serving one’s fellow beings. To implement this truth in action, Tirumūlar proposed a fourfold program: (1) love and worship Śiva; (2) love others in the same manner; (3) feed those who are hungry; and (4) if one is unable to do any of the above, Tirumūlar suggests speaking good, kind words to others (as a last resort; TM 252). When a crow comes across a source of food, he calls others to share it with him (TM 250). If a simple crow can do this unto his brethren, asks Tirumūlar, why not the humans? His statement ‘one the family; one the God’ (TM 2401) implies: Let one accept God or deny, let one belong to any family or group, let one speak any language, but as \textit{jīva-}s, they all belong to one single category—human.\endnote{This line of interpretation was suggested by Ganapathy (1993: 190).}


\subsection*{\textit{Reaffirming Vedānta Siddhānta samanvaya}}

\vskip -8pt

The transformation of the Śaiva Siddhānta philosophical school from a Sanskrit-based, pan-India network to a Tamil-based regional center confirms, for Karen Prentiss (who concurs with Lidova’s thesis), the growing rift between these two schools of Śaiva philosophy. The tension between them, she writes, was ultimately crystallized into the Tamil titles (transliterated from Sanskrit) given to the philosophical authors—\textit{Cantanācārya}-s, where \textit{cantanam} means progeny or succession, and \textit{ācārya} means leader and to the \textit{bhakti} poets—\textit{Camayācārya}-s, where \textit{camaya} means ‘religion.’ Prentiss claims that the Tamil version of Śaiva Siddhānta not only acquired its own canon, language, and \textit{guru} lineage; it also acquired its own \textit{religion} [emphasis added], all of which distinguished it from the Sanskrit Śaiva Siddhānta (Prentiss 1996: 240). Prentiss then cites a well-known verse that for her illustrates both the superior status of the Tamil Śaiva Siddhānta texts in Tamil culture and the pure knowledge they contain:

\begin{myquote}
The Vedas are the cow; the true agamas are its milk; the Tamil sung by the four is its ghee; the essence of the book in Tamil written by Meykantar\index{Meykantar} of the famous Venney is the taste of the ghee of great knowledge. In this simile, the essences get progressively more subtle and pure leading up to the ‘ghee’ that is the Tamil hymns and the sweet ‘taste of the ghee’ that is the Tamil Saiva Siddhanta (Prentiss 1982: 257).
\end{myquote}

Against Prentiss it must be pointed out that for Hindus (whether southerners or northerners) the ‘sweet tasting subtle Siddhānta ghee’ is \textit{samarasa} with the ‘Veda cow,’ i.e. shares the same essence (\textit{rasa}) as the cow. Tirumūlar discusses the process of \textit{samskāra} i.e. churning out ‘Siddhānta ghee’ from the ‘Veda cow’ in Tantra eight, section five of TM. The component ‘\textit{anta}’ of Siddhānta carries the connotation of goal or conclusion, as does the English word ‘end.’ To become truly Śiva is the quintessence of the teaching of Vedānta-Siddhānta; the four other \textit{anta}-s (end products) being intermediary. Tirumūlar lists them as: \textit{nādānta, bodhānta, yogānta}, and \textit{kālānta} (conclusion derived from meditation on primal sound, knowledge, yoga, and duration of time respectively; TM 2370–2404). Anticipating the designs of the Orientalist scholars of Hinduism like Lidova and Prentiss who, centuries later, would sunder Āgama from Nigama cutting off thereby the goal from the source and the means to that goal, Tirumūlar would ask this rhetorical question: how can the [Veda/Nigama] cow be considered apart or separate from the [Siddhanta] ghee which is its end product? Each is implicated in the other.


\subsection*{\textit{Reaffirming Ārya Draviḍa samanvaya}}

The absolute and race based dichotomy of Ārya/ Drāviḍa typically entertained by the Orientalist and Drāviḍa nationalist scholars and politicians disappears upon closer scrutiny. Sanskritist and Indologist Madhav Deshpande\index{Madhav Deshpande} draws attention to an ancient Jaina text--the \textit{Paṇṇavaṇāsutta}, according to whichthere are two kinds of Aryans: (a) ‘\textit{iddhipattāriya,}’ i.e. Aryans who have attained an exalted status and (b) ‘\textit{aniddhipattāriya}’, i.e. Aryans who have not reached an exalted status. The ‘Exalted Aryans’ are the following: (1) \textit{Arahanta;} (2) \textit{Cakkavaṭṭi;} (3) \textit{Baladeva;} (4) \textit{Vasudeva}; (5) \textit{Cāraṇa;} and (6) \textit{Vijjāhara}. It is important to note that the Brahmins have not been included in this Jaina conception of ‘Exalted Aryans.’ The ‘Non-exalted Aryans’ are subdivided into nine different categories: (1) by region (\textit{kṣetrārya}); (2) by birth (\textit{jātyārya}); (3) by clan (\textit{kulārya}); (4) by function (\textit{karmārya}); (5) by profession (\textit{śilpārya}); (6) by language (\textit{bhāṣārya}); (7) by wisdom (\textit{jñānārya}); (8) by realization (\textit{darśanārya}); and (9) by conduct (\textit{caritrārya})(see Deshpande 1993: 10). This classification suggests that (1) whoever Ārya is or whatever his/her level is; one is so (or can become Ārya) by taking to appropriate profession or learning the proper language and so on. The dichotomy between the Ārya and the non-Ārya/ Drāviḍa is therefore neither racial nor absolute and that one’s status as Ārya is not by birth alone and (2) the socio-cultural-ethical process of Aryanization is dynamic and interactive one. The Buddha endorsed and encouraged the ideal of the Ārya way of life through such expressions as \textit{ariya dhamma, ariya māgga} (the Ārya path) in the Buddhist canonical texts (see Deshpande 1993: 5–10).

The \textit{Yogavāsiṣṭha}\index{Yogavasistha@\textit{Yogavāsiṣṭha}} confirms such an understanding by defining Ārya as one who first diligently performs prescribed duties and only then strives for desired goals while desisting from proscribed deeds. Such an Ārya practicing \textit{yoga} can successfully cultivate \textit{āryatā}(noble character) whereby even the most ignorant individual feels motivated to attain spiritual liberation. Anyone who refuses to emulate Śrī Rāma revolts against nature (\textit{prakṛti}) and against oneself (\textit{ātman}). One that outwardly displays all the social graces while living only for self inwardly is doomed to remain in ignorance (\textit{Yogavāsiṣṭha} 6: 54–55). Drāviḍa, similarly, does not denote a race but the land south of the Vindhya Mountains. In the \textit{Bhāgavata Purāṇa} the term Draviḍa \textit{deśa} occurs on six occasions. Balarāma, who is on pilgrimage, bathes in the sacred Kāverī River that flows through the Draviḍa \textit{deśa}. The text goes on to declare that those who drink its water will become devotees of Vāsudeva (Canto 11, chapter 5: 38–40).


\subsection*{\textit{Reaffirming Sanskrit Tamil samanvaya}}

The title \textit{Tirumantiram} itself is made of ‘Tiru’ (Tamil equivalent of Sanskrit ‘\textit{śrī}’) and ‘\textit{mantiram}’ (derived from Sanskrit ‘\textit{mantra}’) and wherein the \textit{mantra} ‘Sivayanama’ occupies a central position. Section three of the ‘Payiram’ of TM is called ‘Ākamac Ciṟappu’ or ‘The Greatness of the Āgama-s,’(TM 57- 66), which narrates how the Āgama-s were revealed alike in both Sanskrit (\textit{Ariyam} is the term actually used) and Tamil. The ancients who used this \textit{Ariyam} language to communicate among themselves were called the Ārya-s. Professor Sivachariyar notes that the \textit{Candrajnāna Āgama,} too, states that Śiva, manifesting as Dakṣiṇamūrti\index{Daksinamurti@\textit{Dakṣiṇamūrti}}, revealed the Āgama-s and Veda-s to the sages on the summit of the Mahendra Mountains just as He had revealed them to Ananteśvara on Mount Kailāsa in the previous age (\textit{kalpa}). The sages recorded this transmission in Sanskrit using the Grantha alphabet, as instructed by Śiva. The Grantha script\index{Grantha script} is akin to Tamil and one can see the similarity in letters, such as \textit{u, o, ka, ta, tha, na, pa, va, ya} and \textit{ra}, etc. between the two scripts (Sivachariyar 2016). These accounts suggest that Ārya-s never migrated into India from outside as Orientalist scholars and some Drāviḍa nationalists would have us believe. Pamban Kumaragurudāsa Swāmigal, who received instruction on ‘\textit{Dahara Upāsanā}’ directly from Murugan, categorically stated:

\begin{myquote}
Both Sanskrit and Tamil are essential for any devotee if he wants the grace of the Lord. He who abhors Tamil or Sanskrit incurs the displeasure of Lord Murugan (Sivachariyar 2016).
\end{myquote}

Carlos Mena, who wrote his doctoral dissertation on the use of ‘hermeneutics’ in Tirumūlar’s thought (2009), invites us to consider how Tirumūlar skillfully re-presents Patañjali’s\index{Patanjali@\textit{Patañjali}} eight-fold path of \textit{yoga} as preserved in Sanskrit that he now embodies (or imbeds) in Tamil, which shows the flexibility and openness of both Tamil and Sanskrit. The TM is a great example of the harmony between two classical linguistic and literary spheres and how Tirumūlar instituted it. Despite the use of numerous Tamilized Sanskrit words, a definite Tamil aesthetic and ethical flavor runs throughout the work (Mena 2009: 73). Not surprisingly, as late as the end of nineteenth century, Tamil scholars wrote without any spirit of antagonism between Tamil and Sanskrit. Unfortunately, antagonism erupted, laments Professor Sivachariyar, in full force in the twentieth century, which was exploited by a succession of Drāviḍa nationalist and political movements for their own agenda rooted in the Aryan invasion theory--The Justice Party, The Drāviḍa Munnetra Kaļagam (DMK) and numerous splinter groups it spawned (Sivachariyar 2016).


\subsection*{\textit{Reaffirming yajña pūjā samanvaya}}

Śiva, Viṣṇu, and Devī, the typical recipients of \textit{pūjā} (whether performed in the north or south) do not belong exclusively to the Āgama or Nigama stream. Vedic Rudra, Yajña, and Vāc reappear in the Āgama texts as Śiva, Viṣṇu, and Devī and as Bhairava, Narasimha, and Caṇdikā respectively in the Purāṇa-s. The standard recessions of the Rāmāyaṇa and Mahābhārata are in Sanskrit and as such belong to the Nigama stream. Yet, they are also available in Tamil and other regional languages enjoying high status among the Hindus living in the villages. Most of the epic stories reappear in vernacular languages in which high Sanskritic deities (Rāma or Kṛṣṇa) mingle with local deities or heroes (Fuller 1992: 26). Same is the case with the Āgama-s which are available both in Sanskrit and Tamil. Tirumūlar felt that rigid theism was responsible for unnecessary controversy and hostility among the followers of different \textit{camaya}-s. Unlike the conservative Brahmins who upheld the primacy of the Vedic \textit{yajñasamsthā} and recitation of Sanskrit \textit{mantra}-s, Tirumūlar also allowed \textit{pūjā} that could be performed by women and members of all castes (see Martin 1983: 93).

The repertoire of \textit{pūjā} includes elements drawn from both the Āgama and Nigama streams. When \textit{pūjā} is offered in the temple, \textit{mantra}-s are chanted first by the Brāhmaṇa who represents the Nigama tradition followed by other \textit{mantra}-s drawn from the various Āgama texts. A village temple usually includes a range of \textit{devatā}-s (gods and goddesses) representing both the Āgama and Nigama traditions. The village guardian deities (\textit{grāmadevatā}-s\index{gramadevata@\textit{grāmadevatā}}) generally are the first to receive obeisance (since they are strategically located at the entrance and around the temple) from the devotees as they arrive to participate for celebrations at the temple. Tirumūlar nevertheless warns that one cannot truly reach Śiva and ‘know’ him until one had rid of the worldly desires (TM 1848, 1850). This is because the path of knowledge is the supreme form of \textit{pūjā} that is offered within one’s body which serves as the temple (TM 1849). To feed a sage (\textit{jñātṛ}) is more meritorious than offering \textit{pūjā} to a deity at a place of pilgrimage (TM 1851). Since one’s fellow devotees are like movable temples; serving and offering them sustenance is equivalent to offering \textit{pūjā} to one’s chosen deity (TM1857; Tirumūlar calls this Maheśvara\textit{ pūjā}).

\textit{Samanvaya} between \textit{yajña}and \textit{pūjā} was sought through the introduction of \textit{bhakti}, which was primarily designed to wean the Tamil population away from the non-theistic paths of Mahāvīra and Gautama Buddha by incorporating some of the more popular elements from these paths. Thus, according to Tirumūlar, a better form of \textit{pūjā} refrains from killing any living being and moves the arena of the \textit{pūjā} from the public space to one’s heart where the self resides (TM 197). This form of \textit{pūjā} also challenged the atheistic stance of Mīmāmsaka-s (Martin 1983: 101). Furthermore, Tantra One of TM lists actions to be avoided and virtues to be cultivated as part of Dharma (‘\textit{aram}’) yet the incentive for regulating one's conduct is not restricted to prescribed Nigamic \textit{vidhi}-s (injunctions) as the Mīmāmsaka-s insist. It also includes meditation on the transitory nature of earthly existence. From the Āgamic side Tirumūlar allows various daily observances (\textit{caryā}) and ritual practices (\textit{kriyā}), including the fire-ritual (\textit{āhuti}) performed by those who know the Veda-s, as stepping stones to the attainment of truth (\textit{unmai} in Tamil) provided they are performed with \textit{patti} (\textit{bhakti}).


\section*{Siddhānta}

\subsection*{\textit{Samarasatā: Tirumūlar’s sterling contribution to Dharma}}

The dominant theme that emerges from the [imagined] debate between Professor Natalia Lidova, the \textit{pūrvapakṣin}, and Tirumūlar, the \textit{uttarapakṣin}, is that \textit{samarasatā}, which emerges as Siddhānta, is a meeting ground for Indic spiritualities. Both the Vedānta and [Śaiva] Siddhānta traditions share it without compromising, however, the distinctness of their respective traditions. Yet, as Isaac Tambyah points out, it is not a colourless eclecticism claiming that Āgama and Nigama are the same; neither is it an attempt at synthesizing the two streams. Indic tradition recognizes six \textit{rasa}-s, which obviously do not taste the same. An attitude of \textit{samarasatā} would consider them impartially or with indifference—whether the tester recognizes taste to be bitter, sweet, or sour. Even so, the mind of \textit{samarasatā} will evaluate impartially the ‘six creeds’ that Tirumūlar refutes while acknowledging differences of outlook among them. Whereas the Vedānta says, ‘I am the supreme one, ’Siddhānta says ‘I shall become the supreme one.’ Tirumūlar observes that in the \textit{Turīya} state\index{Turiya state@\textit{Turīya state}} both realize the self’(TM 2372). If each were to hold its teaching to be complementary to the other, then the attitude of \textit{samarasatā} will emerge. \textit{Samarasatā} thus does not denote a doctrine so much as a \textit{stage} in spiritual development that can recognize harmony between them. This is arrived at by affirming the relation of \textit{jīva} to ‘Ultimate reality’ to be neither oneness (as claimed by Vedānta) nor two-ness as claimed by Siddhānta (see Tambyah 1925: xxxi-xxxvi).

In her effort to perpetuate a conceptual dichotomy between Āgama and Nigama, Lidova overlooks the essential teaching of the seminal vision of \textit{samarasatā} in Hindu thought systems. If she were to focus on it she could realize that the differences between Āgama and Nigama that their respective followers conceptualize are peripheral; not central (Manninezhath 1993: 227). It was on the foundation of \textit{samanvaya} that Dharma evolved into an umbrella term encompassing a body of indigenous cultural, social, and spiritual values as represented in the two wings of Āgama and Nigama where no single authoritative sacred text, deity, or ritual specialist enjoys exclusivity. Such polysemism is the great strength of the Indic intellectual thought world. Attending seasonal festivals at Mathura in the north or at Madurai in the south did not (and does not today) constitute a transgression of the prevailing Śaiva or Vaiṣṇava identity, whatever an Indologist such as Lidova or any Drāviḍa nationalist may assert today.

Thanks to Tirumūlar’s mediation, neither Āgama nor Nigama remain or function like an island (to the chagrin of the Draviḍa nationalists). In fact, he securely locates the Āgama component of Dharma, which he personally nurtured and nursed, in India’s ‘south’ (at the bidding of Śiva), within the ambit of Dharma.\endnote{See TM 2755. With suitable modifications, it may be restated as follows: South is the ‘Holy Land’ [of Dharma]. At the Land's end is Kanya Kumari; and then the Kaveri and other sacred waters. The South has nine ‘\textit{tīrtha}-s’ and seven sacred hills. In that land are born the Veda- Āgama-s. Thus blessed, the South is the ‘Holy Land’[of Dharma] indeed.} Thanks to Tirumūlar, Hindus need not conform to one particular interpretation of Dharma to the exclusion of the other as Lidova insists they do. Following the pluralistic outlook he taught, Hindus are able to willingly abide by the moral injunctions of their respective \textit{camaya}-s (TM 247). In this Hindus (again thanks to Tirumūlar) differ noticeably from the followers of religions that concern themselves with directly opposing each other’s creeds and their followers. These features of Dharma nullify Lidova’s verdict that Āgama and Nigama are two incompatible and rival communities that cannot exist harmoniously with each other. TM suggests otherwise; that in the Indic context Āgama and Nigama express one and same Dharma but from a particular angle of vision or style.


\subsection*{\textit{Tirumūlar’s legacy}}

Tāyumāŋavar (1705–1744 CE) was a renowned philosopher and poet saint of Tamil Nadu. He was also a scholar of Sanskrit and served as minister to the King of Thiruchirapalli. He may justly be considered a worthy recipient of Tirumūlar’s legacy. Tāyumāŋavar's vision of Śaiva Siddhānta and Vedānta is predicated on the concept of \textit{samarasa} with the help of which, like Tirumūlar, he brought forth, anew, the ecumenical and inclusive elements that are present in the Agama and Nigama traditions themselves but that were eclipsed in the later centuries.

The Rajas of Thanjavur, who belonged to Shivaji’s Bhonsle dynasty (18th century), were renowned for establishing \textit{chatra}-s (centers of hospitality) that were strategically located along the road to pilgrim centers, which took care of the hungry and the sick. With the advent of colonialism, \textit{chatra}-s were deprived of this important role. Raja Serfoji II (1777–1832 CE), the last ruler, was another legatee of Tirumūlar (particularly in the field of Siddha medicine); who established a dozen \textit{chhatra}-s. When the British acquired military and political control over his kingdom, Serfoji wrote to the British Resident in 1801 imploring the Resident to ensure that whatever else might befall Thanjavur, the tradition of hospitality was not curtailed or done away with. But the British dismissed such institutions as wasteful use of resources and the \textit{raja}-s were warned against directing funds to their maintenance (see Bajaj and Srinivas 1996: 16–20).

Ramalingam Swamigal (also known as Vellalar; 1823–1874) founded the ‘Association of the Equal, Pure, and True Path’ functioning in a harmonious way (\textit{Camaraca Cutta Caŋmārkka Cańkam} = \textit{Samarasa Śuddha Sanmārga Sangha}). His oral discourses were compiled into a text called ‘\textit{The Conduct of Compassion toward Living Beings}’ (\textit{Cīvakāruṇya Oļukkam}). The core message of the text is that the highest level of ethical behavior, that which also bestows Siva’s grace (\textit{aruļ}) on one and thereby ultimate liberation, should be based on compassion toward all beings. Like Tirumūlar, Ramalingam stressed that feeding the hungry was the greatest act of compassion (see Raman 2013: 2–3).\endnote{Srilata Raman argues that Ramalingam’s message of compassion toward all beings was also influenced by his encounter with ‘missionary Christianity’ (Raman 2013: 25).}

Coming to our times in the twenty-first century, it is heartening to note that a trust has been set up to help rural youth in their economic, educational, and social uplift. It also holds the ‘Tirumūlar International Day’ in Sathanur village, Thanjavur District, Tamil Nadu to keep Tirumūlar’s vision and mission alive (Tirumūlar Arakattalai; see \url{www.tirumular.com}). These three instances, albeit brief, do register that Tirumūlar’s legacy has found secure roots in the Tamil land. However, the task of extending and ‘routinizing’ Tirumūlar’s vision of \textit{samarasatā} across India remains. Perhaps the ‘Samajik Samarasata Manch,’ an organization that was established in 1983 might rise to the occasion. Founded on the ethic of voluntary service, the ‘Manch’ seeks to create and deploy a strategy of reconciliation supplemented with pedagogy and strategy rooted in India’s history, culture, values, and world outlook. It encourages utilization of local human and natural resources for the economic, social, and cultural advancement of the disadvantaged classes in India (see \url{www.samajiksamrastamanch.org/contact.php}).

To conclude, this paper has discussed Tirumūlar’s sterling contribution to Dharma--skillful adjudication of the claims made by adherents of --Āgama and Nigama by juxtaposing them in such a way as to highlight their similarities without interfering with their respective identities and differences. He treads the proverbial ‘Middle Path’ resisting the temptation to (a) force a new synthetic unity or (b) drive the two entities to their total alienation from each other. Mediating between these two independent positions, Tirumūlar showed how to bridge or reconcile them in meaningful ways \textit{without} reducing them to a single unifying structure. His legacy appeals to all Indians to move forward positively with Dharma regardless of the attempts in the past to sow disharmony between its two streams—Āgama and Nigama.


\section*{Bibliography}

\begin{thebibliography}{99}
\bibitem{chap11-key01} Apte, Prabhakar. \textit{Twin Streams of Hinduism: Agama and Nigama}.

 \bibitem{chap11-key02} Bahadur, Rakesh et al. (2011). \textit{Report on Equality and Inclusion Progress and development of Scheduled Castes and Tribes in Independent India}. \url{http://bharatkalyan97.blogspot.ca/2011/08/equality-and-inclusion-progress-and.html Accessed 18 August 2017}.

 \bibitem{chap11-key03} Bajaj, A K and M D Srinivas. (1996). “Annam Bahu Kurvita: The Indian tradition of growing and sharing food.” \textit{Manushi}, nos 92–93 (January-April 1996). pp16–20.

 \bibitem{chap11-key04} Berling, Judith. (1980). \textit{The Syncretic Religion of Lin Chao-en}. Columbia University Press. New York.

 \bibitem{chap11-key05} Fitzgerald, Tim. (1996). “Review: Natalia Lidova, Drama and Ritual in Early Hinduism.” \textit{Asian Folklore Studies} vol. 55, No. 1 (1996),\break pp182–184.

 \bibitem{chap11-key06} \textit{Bhāgavata Purāṇa} (1970). See Tagare.

 \bibitem{chap11-key07} Bhat, Rajnath. (2017). “Arya-Dravid Racial Categories and Colonial India.” \url{https//www.academia.edu/12010232/Arya-Dravid_Racial_Division. Accessed 5 August 2017}. \textit{Brahmasutra-}s. See Date (1973).

 \bibitem{chap11-key08} Caldwell, Robert Rev. (1875). A \textit{Comparative Grammar of the Dravidian or South-Indian Languages}. 2nd ed. Trubner \& Co.London. \url{http://archive.org/stream/comparative gramm00caldrich#page/n5/mode/2up Accessed 11 August, 2017}.

 \bibitem{chap11-key09} Chari, Srinivasa S.M. (1997). \textit{Philosophy and theistic mysticism of the Āļvārs}. Motilal Banarasidass. Delhi.

 \bibitem{chap11-key10} Davis, Richard H. (1992), "Aghorasiva's Background." \textit{Journal of Oriental Research} ("Dr. S. S. Janaki Felicitation Volume."(1992). pp367–378.

 \bibitem{chap11-key11} Date, V. H. (1973). \textit{Vedānta explained; Śamkara’s commentary on the Brahma-sūtras} by V. H. Date. 2nd ed. Munshiram Manoharlal Publishers. New Delhi.

 \bibitem{chap11-key12} Deshpande, Madhav. (1993). \textit{Sanskrit and Prakrit: Sociolinguistic Issues}. Motilal Banarasidass, Delhi.

 \bibitem{chap11-key13} Fuller, C.J. (1992). \textit{The Camphor Flame: popular Hinduism and society}. Princeton University Press. Princeton, N.J.

 \bibitem{chap11-key14} Ganapathy, T. N. (1993). \textit{The Philosophy of the Tamil Siddhas}. Indian Council of Philosophical Research. New Delhi.

 \bibitem{chap11-key15} Ghose, Rajeshwari. (1997). \textit{The Lord of Ayur: The Tyagaraja Cult in Tamilnadu. A Study in Conflict and Accommodation.} Motilal Banarasidass. Delhi.

 \bibitem{chap11-key16} Haran, B.R. (2015). “Christian Inculturation in Tamil Nadu.” Indiafacts.org/Christian-inculturation-hinduism-religious-prostitution/ Accessed 18 August 2017.

 \bibitem{chap11-key17} Kullūkabhaṭṭa. (1946). \textit{Manusmṛti with the commentary by Kullūkabhaṭa}. Satyabhamabai Pandurang. Bombay. Lidova, Natalia. (1994). \textit{Drama and Ritual in Early Hinduism}. Performing Arts; Series 4. General Editor, Farley P. Richmond. Motilal Banarsidass Publishers. Delhi.

 \bibitem{chap11-key18} Mallampalli, Chandra. (2004). \textit{Christians and Public Life in Colonial South India, 1863–1937: Contending with marginality}. Routeledge Curzon. London.

 \bibitem{chap11-key19} \textit{Manavarthamuktavali}. See Kullūkabhaṭta (1946).

 \bibitem{chap11-key20} Manninezhath, Thomas. (1993). \textit{Harmony of Religions: Vedānta Siddhānta Samarasam of Tāyumānavar}, Motilal Banarasidass. Delhi.

 \bibitem{chap11-key21} Martin, Judith G. (1983). “The Function of Mythic Figures in the Tirumantiram.” A Thesis Submitted to the School of Graduate Studies in Partial Fulfillment of the Requirements for the Degree Doctor of Philosophy, McMaster University, Hamilton, ON Canada. Accessed 30 June, 2017.

 \bibitem{chap11-key22} Mena, Carlos Ney. (2009). “The Hermeneutics of the Tirumantiram.” A dissertation submitted in partial satisfaction of the requirements for the degree of Doctor of Philosophy in South Asian Studies, University of California, Berkeley. Accessed 27 June, 2017.

 \bibitem{chap11-key23} \textit{Mīmāmsāsūtra-}s. See Sandal (1974).

 \bibitem{chap11-key24} \textit{Nyāyāvali: Sanskrit Maxims and Proverbs}. (1980). Nag Publishers. New Delhi.

 \bibitem{chap11-key25} Pansikar, Wasudev Laxman Shastri. (n.d.). \textit{The Yoga-Vasistha of Valmiki with Vasistha Maharamayana-Tatparya Prakasa in Sanskrit}. Nirnaya-Sagar Press. Bombay.

 \bibitem{chap11-key26} Pillai, Subramania K. (1929). \textit{The Metaphysics of the Saiva Siddhanta System}. The South India Saiva Siddhanta Works Publishing Society. Tinnevelly.

 \bibitem{chap11-key27} Prentiss, Karen Pechilis. (1996). “A Tamil Lineage for Saiva Siddhanta Philosophy.” \textit{History of Religions} v. 36 n.3 (1996). pp231–257. Accessed 3 July, 2017.

 \bibitem{chap11-key28} Raman, Srilata. (2013). “The Space In Between: Ramalinga Swamigal (1823–1874), Hunger, and Religion In Colonial India.” \textit{History of Religions}. Vol 53, No 1 (2013). Pp1–27. Accessed 12 August 2017.

 \bibitem{chap11-key29} Sandal, Mohan Lal. (1974). \textit{The Mīmāmsā sūtras of Jaimini}. Translated by Mohan Lal Sandal. Allahabad, Panini Office, 1923–25. AMS Press, New York.

 \bibitem{chap11-key30} Sivaraman, K. (1983). “The Role of the Emergence of Śaiva Siddhānta: Saivāgama in the Philosophical Interpretation.” \textit{Traditions in Contact and Change}, edited by Peter Slater and Donald Wiebe. Wilfrid Laurier University Press. Waterloo.

 \bibitem{chap11-key31} Shaw, Rosalind and Charles Stewart. (1994). “Introduction: problematizing syncretism.” \textit{Syncretism/Anti-Syncretism: The Politics of Religious Synthesis} edited by Charles Stewart and Rosalind Shaw. Routledge. London.

 \bibitem{chap11-key32} Sivachariyar, Sabharatnam S.P. (2016). “Sanskrit-Tamil Harmony.” \textit{Hinduism Today} (Web Edition April/May/June 2016). Accessd 7 August 2017.

 \bibitem{chap11-key33} Smith, Brian K. (1994). \textit{Classifying the Universe: The Ancient Indian Varṇa System and the Origins of Caste.} Oxford University Press. Oxford.

 \bibitem{chap11-key34} Srimad-Bhagavad-gita. See Swami Swarupananda (1972).

 \bibitem{chap11-key35} Swami, Sivananda (1999). \textit{Sixty-Three Nayanar Saints}. World Wide Web (WWW) Edition. The Divine Life Trust Society. Accessed 28 July, 2017.

 \bibitem{chap11-key36} Swami Swarupananda. (1972). Srimad-Bhagavad-gita; with text, word-for-word translation. 11th rev.ed. Advaita Ashrama. Calcutta.

 \bibitem{chap11-key37} Swaminathan, Santanan. 2014. “\textit{Origin of Tamil and Sanskrit}.” Research article No.1409; Dated 13th November 2014. Accessed 15 June, 2017.

 \bibitem{chap11-key38} Tagare, G. V. (1970). \textit{The Bhāgavata-purāṇa}. V. 12–14. Translated and annotated by G. V. Tagare. Motilal Banarasidass. Delhi.

 \bibitem{chap11-key39} Tambyah, Isaac (1925). \textit{Psalms of a Saiva Saint.}Luzac \& Co. London.

 \bibitem{chap11-key40} Tirumantiram: Fountain head of Saiva Siddhanta. English Translation of the Tamil Spiritual Classic by Saint Tirumūlar. \url{https//www.himalayanacademy.com/view/tirumantiram Accessed 5 June, 2017}.

 \bibitem{chap11-key41} \textit{Yoga Vasistha} (n.d.). See Pansikar (n.d.).

 \end{thebibliography}

\theendnotes

