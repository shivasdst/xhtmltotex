
\chapter{Reaffirming Āgama Nigama Samanvaya: Tirumūlar’s Contribution To Dharma}

\Authorline{Dr Shrinivas Tilak}


\section*{Abstract}

Tirumūlar\index{Tirumular@\textit{Tirumūlar}}, one of the sixty-three Nāyanmār\index{Nayanmar@\textit{Nāyanmār}} saints and author of the foundational work of Śaiva Siddhānta\index{Saiva Siddhanta@\textit{Śaiva Siddhānta}}, the \textit{Tirumantiram}\index{Tirumantiram@\textit{Tirumantiram}}, deployed the strategy of \textit{samanvaya} to accommodate the followers of other spiritualities within the fold of [Siva] Dharma while retaining and respecting differences from them. This paper argues that Tirumūlar’s strategizing of \textit{samanvaya}\index{samanvaya@\textit{samanvaya}} can also be used to reaffirm harmony between Āgama and Nigama (the two foundational sources of Hindu Dharma) that modern Western theologians and scholars of Hinduism insist does not exist because they consider Agama and Nigama to be two rival theological systems and as such incapable of existing in harmony under the canopy of Hinduism (their appellation for Hindu Dharma). The paper is presented in the traditional debating format also pursued by Tirumūlar:

‘Pūrvapakşa-Uttarapakşa-Siddhānta’ (\textit{Arusamaya Pinakkam}) where the \textit{pūrvapakşin} stands for non-Hindu scholars of Hinduism (with particular focus on Professor Natalia Lidova\index{Natalia Lidova}) as well as Drāviḍa nationalists\index{Dravida nationalists@\textit{Drāviḍa nationalists}} who employ the Orientalist perspective to divide and undermine Hindu Dharma. Refutation of their arguments (the Uttarapakşa) is based on the accommodative vision (\textit{samanvaya}) and the relevant data to be found in the \textit{Tirumantiram}. The Siddhānta establishes Tirumūlar’s sterling contribution to Dharma to be skillful adjudication of the claims of the two jockeying parties--Āgama and Nigama. He reaffirms their harmony by juxtaposing them in such a way as to highlight their similarities; yet preserving their unique identities and differences. The paper concludes with discussion of the legacy Tirumūlar left to the followers of Dharma.


\section*{Introduction}

Tirumūlar, one of the sixty-three Nāyanmār saints and author of the foundational work of Śaiva Siddhānta, the \textit{Tirumantiram} (hereafter TM), deployed the strategy of \textit{samanvaya} (it may be translated as the ‘Middle Path’) to accommodate the followers of other spiritualities within the fold of [Śiva] Dharma (while retaining and respecting differences from them)(TM 1558).’\endnote{(1) Tirumūlar refers to them as the ‘group of six’ without identifying them. Standard works on Siddhānta Philosophy (\textit{Sivagnana Siddhiar} for instance) begin with a brief statement and criticism of the tenets of other Indic systems of thought which are grouped into four divisions, each of which comprises six schools of philosophical opinion. In the first group are placed the Lokayata, the four major schools of Buddhism at the time, and Jainism. This group of six is called ‘Purapuram’ (Pillai 1929: 3). It is likely that Tirumūlar had this group in mind.} This paper argues that Tirumūlar’s strategizing of \textit{samanvaya} can also be used to reaffirm harmony between Āgama and Nigama (the two foundational sources of Hindu Dharma) that modern Western theologians and scholars of Hinduism insist does not exist because they consider Āgama and Nigama to be two rival theological systems and as such incapable of existing in harmony under the canopy of Hinduism (their appellation for Hindu Dharma). The paper is presented in the traditional debating format also pursued by Tirumūlar: ‘Pūrvapakşa-Uttarapakşa-Siddhānta’ (\textit{Arusamaya Pinakkam}). Judith Martin, who wrote her doctoral dissertation on the \textit{Tirumantiram}, has observed that in section eighteen of Tantra Five of TM, Tirumūlar distinguishes two basic approaches to Dharma, that which is preoccupied with its external form (\textit{puracamaya}) and that which is concerned with its internal form (\textit{utcamaya})(see Martin 1983: 165). It is possible to argue that Tirumūlar identifies and refutes the former as the \textit{pūrvapakşin} because by directing attention outward it precludes access to the One [Śiva] who dwells within the body.

Pūrvapakşa ('the first side;' ‘\textit{Parapakka}’ in Tamil) is the technical term for the preliminary position, or prima facie view in a metaphysical or philosophical argument put forth in the form of an objection by the \textit{pūrvapakşin} (opponent; real or imagined). The Uttarapakşa (\textit{Nirākaram}; also known as ‘\textit{Svapakka}’ in Tamil) refutes the argument put forth by the \textit{pūrvapakşin}. Siddhānta ('established end') is the technical term for the demonstrated and definitive conclusion of the debate. In this paper, the \textit{pūrvapakşin} connotes the scholars of Hinduism and Draviḍa \textit{nāstika}-s and nationalists that employ the Orientalist perspective to divide and undermine Dharma by rejecting Nigama. Refutation of their arguments (the Uttarapakşa) is based on the accommodative vision and the relevant data to be found in Tirumūlar’s \textit{Tirumantiram}. The Siddhānta establishes Tirumūlar’s contribution to Dharma in reaffirming the harmony of Āgama and Nigama. The paper concludes with a discussion of the legacy Tirumūlar left to the followers of Dharma.


\section*{Āgama Nigama: two founding sources of Dharma}

Hindus have traditionally understood Dharma to be that by which material prosperity (\textit{abhyudaya}) in this world and welfare in the other world (\textit{nihśreyas}) is assured (\textit{Mīmāmsāsūtra-}s 1:1.1-2). It is recognized that grammatically, \textit{abhyudaya}\index{abhyudaya@\textit{abhyudaya}} and \textit{nihśreyas}\index{nihsreyas@\textit{nihśreyas}} occur as a co-ordinate compound (\textit{dvandva samāsa}); not as a dual--Rāmalakşmaņau. This suggests that Dharma encompasses welfare both in this and the other world and that these two goals are not mutually exclusive. Indic cultural, social, and spiritual heritage therefore is rooted in Dharma (known as \textit{Āgamanigamātmaka}), i.e. comprising the twin streams of Āgama\index{Agama@\textit{Āgama}} and Nigama\index{Nigama}. Āgama Dharma therefore refers to the stream that is rooted in the worship of an enshrined image (\textit{mandirāntargata mūrti-pūjā}’)(see Apte). According to Tirumūlar, Āgama-s were revealed from the fifth face of Śiva (in Sanskrit as well as in Tamil) to proclaim Dharma as expounded by Śiva (hence known as Siva Dharma\index{Siva Dharma} in the TM) and which later became available in eighteen languages, Āgama-s expound but one truth of Vedānta-Siddhānta (TM 3, 61, 65, 1429).

Broadly speaking, Nigama refers to the entire Vedic corpus (\textit{sāhitya}; see \textit{Bhāgavata Purāņa} Canto 1, Chapter 1.3). Nigama Dharma denotes the religio-cultural stream that is rooted in the institution of \textit{yajña} (\textit{yajña-samsthā}). Dharma is deemed to be beginning-less (\textit{anādi}) and without divine or human authorship (\textit{apauruşeya}\index{apauruseya@\textit{apauruşeya}}). Speaking of the greatness of the Veda-s (Nigama), Tirumūlar declares that the Veda-s proclaimed the central core of Dharma—that there is no Dharma; barring what the Veda-s say. Because the wise sages refrained from contending the truth of Dharma they were able to attain realization by chanting the \textit{mantra}-s. Brahmā spoke of the Veda-s but only in order for Śiva to reveal them. Brahmā also spoke of the \textit{yajña-s} but only in order for Śiva to reveal them (TM 51-52). The Dharmaśāstra-s\index{Dharmasastra@\textit{Dharmaśāstra}} proposed the stages of life model of an ideal lifestyle in pursuit of Dharma based on Vedic teachings and values which became a standard feature in the Smŗti works such as the \textit{Manusmŗti} \index{Manusmrti@\textit{Manusmŗti}} written in Sanskrit (Nigama). In these works the human lifespan is normatively divided in four overlapping stages of twenty-five years duration each on the basis of differences of pedigree (\textit{varņa}), age (\textit{vaya})\textit{,}and gender (\textit{linga}).


\section*{Pūrvapakşa: Orientalist Indology questions Āgama Nigama harmony}

Influenced by the writings of Christian missionaries like Bishop Robert Caldwell\index{Bishop Robert Caldwell} (1819-1891) and G. U. Pope\index{G. U. Pope} (1820-1907), colonial British authorities and administrators in the nineteenth century India were led to believe that hierarchical stratification and inequality were so fundamental and rampant in Hinduism (their distorted appellation for Dharma) that Hindu culture, religion, and society must be understood and explained in terms of higher and lower social strata, forms, and levels that existed in opposition to one another (Bhat 2017, Haran 2015). Beliefs and practices of social groups with higher social status were deemed to be distinct from (and superior to) those of lower-status groups (based on Bahadur et al 2011). Once this basic dichotomy was introduced and actively pursued, it was only a question of time that the groups of Hindus claiming a higher social status for themselves were encouraged to identify themselves as such. They were designated as followers of ‘Brahmanical and Sanskritic’ Hinduism now identified with the Nigama wing. In contrast to this ‘Vedic Brahmanism’ was said to have emerged the peninsular, village-based, vernacular forms of ‘Hinduisms’ that were placed under the Āgama wing. European and American scholars of India and Hinduism internalized this dichotomous view of Hindu Dharma (hereafter Dharma) and produced hundreds of monographs on different aspects of ‘Hinduism’ that deviated considerably from the Hindu self perception of precepts and practices outlined in Dharma.

Sanskritist Natalia Lidova of the Institute of World Literature, Russian Academy of Sciences, Moscow, is one such Orientalist Indologist who presupposes this line of analysis and interpretation in her \textit{Drama and Ritual in Early Hinduism} (1994) whose central thesis is that the \textit{Nāţyaśātra}\index{Nātyasatra@\textit{Nāţyaśātra}} of Bharata Muni, a treatise on the theatre in ancient India, reflects the oldest premises of the Āgamic ideology, which came to \textit{replace} (emphasis added) the Vedic Nigama (Lidova 1994: 98,116). In this work Professor Lidova (hereafter Lidova) sustains the artificial division of Dharma into Nigamic/Vedic ‘Religion’ and Āgamic ‘Hinduism’ by dividing Dharma sociologically into (1) Aryan, Vedic, North Indian, higher Sanskritic, Brahminical and (2) Dravidian, Āgamic, South Indian, and Tamilian as two functional correlates of a caste-based hierarchical social structure\index{caste-based hierarchical social structure}. Lidova also claims that the unpopularity of the Vedic ritualism centered on the Śrauta and Soma rites is attested to by the rise of Buddhism and Jainism. Subsequently, the lower strata of the Brahmins who were themselves not part of the elite that performed the Śrauta rites, borrowed elements of the non-Vedic \textit{pūjā}-cult (Lidova 1994: 118).

Lidova argues that despite the similarity of (and interactions between) numerous common components, there existed fundamental differences between \textit{pūjā} and \textit{yajña} in terms of sacrificial structure, symbolism, and theological background (Lidova 1994: 40; Fitzgerald 1996). The ritual preliminaries (\textit{pūrvaranga}\index{purvaranga@\textit{pūrvaranga}}) carried out before the staging of Sanskrit drama, argues Lidova, constituted a break from Vedic \textit{yajña}; they owe more to Āgama-style (Dravidian) \textit{pūjā} (Lidova 1994).\textit{ Pūjā}, which was a non-Aryan and Dravidian rite, became the basis of the Hindu ritual-mythological system, largely at the expense of \textit{yajña} (Lidova 1994: 98). Historical and linguistic information testifies to the non-Aryan origin of the\textit{ pūjā}, which Lidova assumes to be an intrinsic Dravidian rite. While Brāhmaņa ritual texts describe Vedic ritualism\index{Vedic ritualism} to the last detail, they display utter indifference to the \textit{pūjā} (Lidova 1994: 98). This is because there was a world of difference between the Vedic liturgical practice and Hindu worship (\textit{pūjā}). Vedism and Brahmanism knew no templar edifices, which were obligatory for Hindus who essentially follow the Āgama-s (Lidova 1994: 99).

\subsection*{\textit{Drāviḍa nationalism\index{Dravida nationalism@\textit{Drāviḍa nationalism}} rejects Dharma to exalt Tamil ‘secular’ identity }}

Dharma in traditional Tamil Nadu functioned like a federation that allowed local communities to develop and maintain their own set of \textit{grāma devatā}-s\index{grama devata@\textit{grāma devatā}}, āgama-s, rituals, and customs. The Tamil Hindu society was organized around its local temples with spirituality, social welfare, and entertainment practices interwoven together. This was one of the reasons why British missionaries found it hard to pursue their conversion agenda. Bishop Caldwell, an evangelist for the Society for the Propagation of the Gospel, sought to break the synergistic duo of Āgama and Nigama emanating from Dharma into Vedic Aryan practices and Dravidian rituals. To perpetuate this division, Caldwell proposed the existence of the Dravidian race in his A Comparative Grammar of the Dravidian Or South-Indian Languages (1875). The term ‘draviḍa’ in Sanskrit refers to the communities located in the region south-of the Vindhya-range of mountains, which Caldwell adopted from a seventh century Sanskrit-text and modified it to ‘Dravidian’ as a ‘race.’ Caldwell therefore may be called ‘the founder’ of the Dravidian ‘racial’ identity, which ultimately created discord between what he called ‘indigenous’ Dravidians of the south and ‘outsider’ Sanskrit loving Aryan/Brahmins from the north. This ingenious manipulation stigmatized the Brahminical, Sanskritist ‘northerners’ as the colonizers and Caldwell and other missionaries as ‘Saviors’ of the ‘colonized’ Tamil ‘southerners.’ Caldwell also instigated ‘Dravidians’ to disown all that is/was Sanskrit-based and re-discover themselves through Biblical categories\index{Biblical categories} (see Bhat 2017, Haran 2015).

Chandra Mallampalli, an Indian Christian scholar, too observes that champions of Dravidianism and non-Brahminism drew upon the cultural and linguistic resources provided by missionaries such as Robert Caldwell and G.U. Pope in order to foster a distinct ‘Dravida’ identity\index{Dravida’ identity} and nationality (Mallampalli 2004: 108). Peter van der Veer, an expert on South-Asia, similarly observes that Aryan invasion and subjugation of ‘indigenous’ Dravidians was Caldwell’s invention who was resentful of Brahmins because they stood as barriers to Caldwell’s mission of conversion\index{mission of conversion} (see Bhat 2017). After independence of India, the Dravidian movement\index{Dravidian movement} picked this very issue successfully persuading some Tamils to shun Vedic Aryan/Brahmanical practices and ‘reclaim’ an indigenous Dravidian, ‘secular’ identity. Pongal, for example, has been rebranded and ‘secularized’ as Tamil Day and an earnest attempt is being made to disconnect Pongal festivities from the rural Hindu temples where they had always belonged (Haran 2015).


\section*{Uttarapakşa}

Indologists like Lidova and Dravidian nationalists\index{Dravidian nationalists} dismiss the extant harmony between Āgama and Nigama as \textit{syncretic} i.e. random, corrupting, and superficial on the basis of Western theological disputes concerning syncretism which came to be regarded as a betrayal of principles or as an attempt to secure unity at the expense of truth (see Berling 1980: 4). Then they proceed to divide Āgama and Nigama generating in the process the following four secondary divides: Ārya versus Drāviḍa, Siddhānta versus Vedānta, Tamil versus Sanskrit, and \textit{pūjā}\index{puja@\textit{pūjā}} versus \textit{yajña.}\index{yajna@\textit{yajña}} This paper argues that using the spirit of accommodation (\textit{samanvaya}\index{samanvaya@\textit{samanvaya}}) displayed in Tirumular’s \textit{Tirumantiram}, the disharmony and distance introduced by Indologists and Orientalists such as Natalia Lidova between Āgama and Nigama can be removed and their harmony re-affirmed.

\subsection*{\textit{Tradition of Āgama Nigama samanvaya}}

A popular imagery describes Āgama and Nigama as two wings of the bird of Dharma. In order to fly, the functioning of both must be harmonized. From the Āgama perspective, the tradition explains the relation between the two in terms of the maxim of the ‘seed and sprout’ (\# 139 \textit{Bījānkuranyāya}) as explained in \textit{Nyāyāvali} (1980): Āgama is the root (or seed) of the tree of knowledge with its branches and fruit (Nigama; \textit{Vedavŗkşa}). The relation of mutual causation subsists between the seed (Āgama) and sprout (Nigama), seed being the cause of sprout, which in turn, is the cause of the seed. From the Nigama perspective, Kullūkabhaţţa in his commentary on the \textit{Manusmŗti} (\textit{Mānvārthamuktāvali}; 2:1) cites \textit{Harita Dharmaśāstra} to the effect that Dharma is legitimated through Śruti, which is twofold: Vaidika (i.e. Nigama) and Tāntrika\index{Tantrika@\textit{Tāntrika}} (i.e. Āgama).


\subsection*{\textit{Tirumūlar and Tirumantiram}}

Not much is known about the real persona of Tirumūlar, the author of the TM. Scholars date Tirumūlar between late sixth and early seventh centuries CE though the tradition says that he lived for three thousand years and composed one verse each year, thus totaling three thousand verses of the TM. There are legendary accounts of his being a \textit{yogin} and a mystic who resided in the Himalayas on Mount Kailāsa. One day, the \textit{yogin} desired to see his friend, Sage Agastya\index{Agastya}, in his \textit{āśrama} located in the Pothia hills. So he left Kailāsa and traveled southwards. While walking along the bank of the Kāverī River, he saw a herd of cows shedding tears over the dead body of the cowherd. Feeling sorry for the cows, the \textit{yogin} entered the body of the cowherd by his yogic powers after safely depositing his own body in the trunk of a tree. The cows were rejoiced. The dead cowherd was known as Mūlan from the village of Sathanur in Thanjavur District. The \textit{yogin}, who had reanimated Mūlan’s body, therefore came to be known as Tirumūlar (Tiru is a suffix that is attached to the name of one worthy of great honor). The next day, Tirumūlar could not find his body where he had left it. He was convinced that Śiva wanted Tirumūlar to compose in verse form a work on Śaiva philosophy in Tamil, containing the essence of all Śaiva Āgama-s. Tirumūlar therefore went in the state of \textit{samādhi}, which lasted for three thousand years. But, every year, he would come out of \textit{samādhi} to compose one verse. This collection of three thousand verses is known as the \textit{Tirumantiram}\index{Tirumantiram@\textit{Tirumantiram}} (see Swami Sivananda 1999).

Tirumūlar himself has narrated the account of his life and mission in Tantra (chapter) One of TM in the manner of an autobiography (TM 73-94), which many modern Hindus will find ‘legendary.’ But what is more important to ponder is the theme of \textit{samanvaya} that pervades the narrative: A Sanskrit knowing \textit{Siddha yogin} and an ‘\textit{ārya}’ living in the north (Himalayas) comes south to the ‘Drāviḍa’ land at the wish of Śiva and composes a poem in Tamil whose central message is that Āgama and Nigama are one (it is important to remember in this context that neither Ārya nor Drāviḍa are racial terms; see below). The theme of harmony continues to operate at another level in that the nine \textit{tantra}-s (chapters) deal with how to live a spiritual life in the midst of the worldly one. Tamil Śaiva tradition therefore treats the TM mainly as a Śaiva Siddhānta work that transcends monism and pluralism. Finally, the fact that the \textit{Tirumantiram}is the only \textit{Tirumurai,}which is deemed to be both a\textit{ cāttira} (\textit{śāstra}) and a\textit{ tottira} (\textit{stotra}) in Tamil Śaiva tradition, implies that philosophy and devotion are harmonized in this work.\endnote{(2) According to the Śaiva Tradition of Tamilnadu, there are twelve \textit{Tirumurais,} i.e., sacred Tamil Śaiva texts and Tirumūlar’s TMconstitutes the 10th \textit{Tirumurai}. All the \textit{Tirumurais} are called \textit{tottira}-s (\textit{stotra}-s – devotional literature), which constitute the \textit{bhakti} literature of Tamil Śaivism. The philosophical literature of Tamil Śaivism is called \textit{cattira}-s (\textit{śāstra}-s – philosophical treatises).}


\subsection*{\textit{Samanvaya: a strategy of reconciliation}}

In the Indic thought world, the strategy of \textit{samanvaya} (reconciliation = \textit{camātāna} in Tamil) proceeds on the assumption that there is an underlying similarity, connection or relation between two or more entities, ideas, concepts or even traditions that on surface appear distant, distinct, or contradictory. The awareness of such a potential connection and similarity is what justifies any attempt to reconcile differing entities or ideas. \textit{Samanvaya} as a strategy of harmonization in this sense first occurs in the \textit{Brahmasūtra}, a text in four chapters wherein the author Bādarāyaņa summarized, systematized, and reconciled the teachings of the Upanişads identifying the consistent themes among a variety of doctrines running through them. The first four aphorisms of the first chapter form a special unit of this work where a summary of the content of the entire work is provided in four aphorisms. \textit{Samanvaya} is formally defined as \textit{samyak anvaya}, i.e. setting up or establishing a proper sequence or order between two or more entities or ideas by removing any distortions or deviations that threaten or impede a harmonious relationship or among them.

The self and its proper identification with others constitute the foundation for feelings of solidarity. The movement from ‘I’ to ‘All’ that is predicated upon the key statement (\textit{mahāvākya}) of Chāndogya upanişad ‘that thou art’ (\textit{6:8.7; tat tvam asi}) suggests the continuum of the Āgama thought with its Vedic heritage. Tirumūlar avows that ‘\textit{tat tvam asi} (You-being-I)’ of Vedānta is the same as ‘\textit{Thom-tat-asi} (I-Śiva-becoming)’of Siddhānta-Vedānta (TM 2571). The \textit{Śvetāśvatara upanişad}, which describes the ultimate reality as Śiva, suggests another point of linkage between Nigama and Āgama. Pippalāda, composer of the \textit{Praśna upanişad}, also authored the \textit{Vāstuśāstra upanişad}\index{Vastusastra upanisad@\textit{Vāstuśāstra upanişad}}, an important Āgama text on temple architecture and building, in which he refers to the icon maker (\textit{sthāpaka}) as \textit{śilpahotŗ} and \textit{vāstuhotŗ} (Ghose 1997: 142). There is thus a clear desire to move away from narrow self-interest to the recognition of the feeling of sameness (\textit{samarasatā}; see below)of belonging to a larger group or community. A feeling of reciprocity and interdependence also constitutes the basis for solidarity with others. Social and cultural affiliations are a further expansion of the self with others. Finally, a more universal category is reached in the declaration ‘the entire world is a family’ (TM 407, 3041)where altruism encompassing all human beings finds expression.


\subsection*{\textit{Samanvaya and syncretism}}

The role of \textit{samanvaya}is comparable to the Western category of syncretism \textit{as it was originally understood.} The \textit{Oxford English Dictionary} defines syncretism as attempted union or reconciliation of diverse or opposite tenets or practices, especially in philosophy or religion. Syncretism derives from the ancient Greek prefix \textit{syn} ‘with,’ and \textit{krasis}, ‘mixture.’ Plutarch\index{Plutarch} refers to the practice of the Cretans, who, though they often quarreled among themselves, made up their differences and united when outside enemies attacked. This practice was called ‘syncretism’ (Shaw and Stewart 1994: 3). Thus the concept of syncretism began its history with positive connotations, referring to a strategically practical and morally justified form of allegiance and reconciliation. Syncretism is not arbitrary or irrational, but serves a definite purpose (Berling 1980: 6-7). \textit{Samanvaya}, the Indic equivalent of syncretism, is similarly fundamental to the dynamics of Indic cultural and social interaction through time. It is not an indiscriminate or aimless combination of practices and ideas as Indologists and Orientalists argue. \textit{Samanvaya} appears ‘perfidious’ to the Indologists/Orientalists only because they view it against the backdrop of the credal and exclusivistic style of the Orientalizing culture that dominates Semitic and Western history.


\subsection*{\textit{Samarasatā: the state of harmony resulting from samanvaya}}

\textit{Sama}(literally same or equal) is the goal in both Āgama and Nigama streams and in philosophies and spiritualities inspired by them. In the \textit{Bhagavad Gītā\index{Bhagavad Gita@\textit{Bhagavad Gītā}} yoga} is defined as that which promotes evenness (\textit{samatva}) and Kŗşņa describes himself to be the same and equal (\textit{samoham}) toward all beings (BG 2:48, 9:29). Despite the great range of diversity operating at the phenomenal level, all beings share a common essence or life force (\textit{rasa}) although the specific nature of this essence is variously conceived. Brian K Smith refers to the \textit{Kauśītakī Brāhmaņa} (2:7), which states that the essence of life flows out of one entity into all others vitalizing each of them. The result is a kind of a complex chain of being in which all entities, including water, plants, and humans are bound together (see Smith 1994: 209-210). The \textit{Tamil Lexicon} gives the meaning of \textit{camaraca}(\textit{samarasa}) as equality, harmony, and identity (\textit{oŗŗumai}). A second understanding of this term is impartiality (\textit{naţunilaimai}). The \textit{Tamil-English Dictionary} similarly explains \textit{camaraca} to mean reconciliation (\textit{camātāna}) and equality (Manninezhath 1993: 153-154). That the Tamil inspired ideal of \textit{samarasatā} has pan-Indian applicability is evident from the fact that over the centuries it became thoroughly internalized in the languages, cultures, and societies prevalent in all parts of India.

Orientalist scholars studying ‘Hinduism’ must take note that Tirumūlar did not set out systematically to integrate \textit{all}pronouncements in the Upanişad-s on \textit{brahman}. The result would have been a hybrid monster. He applied the strategy of harmonization (\textit{samanvaya}) selectively to reduce the major sources of disagreement between Āgama and Nigama without collapsing one totally into the other. Tirumūlar’s basic contention was that divergent precepts and practices espoused in the Āgama and Nigama were to be judged not in relation to one another but in terms of the overall \textit{telos} or ethos of Dharma which included both. That is, the relative merits of the differing paths of Āgama and Nigama were not to be determined a priori but only in terms of their actual effectiveness in leading aspirants to true knowledge of Dharma and Śiva.


\subsection*{\textit{Instituting samanvaya through dāna}}

Tirumūlar avers that the ‘Middle Path’ (the path of \textit{sananvaya}) is the path of wisdom and of justice on which Brahmā (who gave the four Veda-s), Vişņu, and Śiva walk. He himself walked the same path in the noble fellowship of aspirants to attain the state of being-with-Śiva (TM 320-322). The Āgama-s, composed in Tamil and in Sanskrit, provide guidance for the way of life based on Śuddha Śaiva Siddhānta (TM 1422). To Tirumūlar, the Veda-s and the Āgama-s are both true and both are the word of Siva. The first is a general treatise and the second a special one. His strategy here is to institute harmony by validating the Āgama-s as a later 'special revelation' of the Veda-s that make explicit for the devotee of Siva the obscure liberating content of the Veda-s. Understood aright (i.e. as taught by Tirumūlar), Āgama-s re-present the Veda-s more clearly in and for a different (i.e. the Tamil) context. This may be illustrated with reference to the act of giving (\textit{dāna}\index{dana@\textit{dāna}}), which is recognized as a sacred duty in both Āgama and Nigama traditions.


\section*{\textit{Dāna: a practice common to Nigama and Āgama}}

In the Ŗgveda, the \textit{sūkta}-s known as (\textit{dānastuti}) praise the act of giving; (5: 27,31,34,38 for instance). The \textit{Bhikşusūkta} (10:117) praises those who offer food to the hungry (\# 1-2) but condemns those who do not (\# 4), and warns of the futility of hoarding wealth (\# 8-9). The \textit{Dajsuba sūkta} (1:187, 17: 117) promise various merits to those who give to the needy, the poor, and the hungry. The \textit{Annasūkta} (hymn in praise of food; Ŗg 1:187; 17:117) assures that those who give food to the hungry acquire great merit and high social rank. Here, the transaction initiated by giving (\textit{dāna}) operates as a mechanism of maintaining the social harmony that is liable to be constantly threatened. By engaging in \textit{dāna} as part of the performance of \textit{yajña}, the rich and powerful can atone or redeem themselves from social obligation. By giving away a portion of their claim of ownership on part of their assets, the rich and powerful can retain the moral right to enjoy the balance. In the post-Vedic Purāņa-s giving is particularly singled out as the most meritorious act, especially in the present Kaliyuga (the ‘Dark Age’ of Kali). In the Āgama-s \textit{dāna} is similarly understood but it implies more than giving in order to protect one’s assets. One gives in the spirit of love to one’s chosen deity (\textit{işţadevatā}) sharing one’s resources with others, be it wealth, food or other assets. It may involve giving to philanthropic causes – providing rest-houses, planting trees, digging wells. It is important to remember that while the explanatory mechanism behind the act of giving differs: \textit{yajña} in the Nigama tradition and \textit{bhakti} in the Āgama tradition; the act of giving as an expression of Dharma is endorsed in both traditions. Orientalist Indologists\index{Orientalist Indologists} and Drāviḍa nationalist both ignore this fact stressing only difference and disharmony between the Āgama and Nigama traditions.


\section*{\textit{Sivayanama: giving for Dharma in the name of Śiva}}

With particular reference to the Śaivāgama tradition, Tirumūlar insists that \textit{dāna} implies giving in the spirit of love for Śiva. Tirumūlar concurred with the stages of life model prescribed in the Dharmaśāsastra-s but went further in arguing that Sanyāsāśrama, the final stage, did not involve total withdrawal of the \textit{sanyāsin}from the society or from his social responsibilities. The TM, accordingly, is concerned to shows how progress in the spiritual life (to be carried out in the third and fourth stage of life) can also promote values that are consonant (and continuous) with the everyday, material life (to be pursued in the second stage of life). \textit{Sanyāsa} primarily involves renouncing egotistic desires. This can be illustrated with reference to the \textit{mantra} ‘Sivayanama,’ that occurs multiple times in the Tantra nine of TM (TM 2698-2721). This five-lettered (\textit{pancākşara}) \textit{mantra} is the abode of Śiva (as ‘Lord of Veda-s’) in his subtle form and when reversed (i.e. Namasivaya), the same \textit{mantra}is abode of Śiva’s manifest form. According to T. N. Ganapathy:

\begin{myquote}
Sivayanama has a social significance in addition to its surface meaning. \textit{Nama} means spirit of renunciation. Śiva means bliss; and \textit{aya} means outcome or result. The \textit{mantra} Sivayanama therefore means ‘the result of sacrifice is bliss.’ Tirumūlar felt bliss in this provision of sacrifice and construed it as an opportunity to serve (Ganapathy 1993: 190).
\end{myquote}

It is necessary to elaborate Ganapathy’s very insightful, though terse, interpretation. Spirit of renunciation (\textit{tyāga}) is an integral element of \textit{yajña} (it is preferable to use this original term in place of ‘sacrifice’ used by Ganapathy because its Christian meaning could be misleading) and means willingly renouncing one’s right of ownership on something of value and offering it to the needy as part of the performance of \textit{yajña}.

Thus, in devising the socially meaningful mantra Sivayanama, Tirumūlar brought together the best from Nigama and Āgama traditions; i.e. the notions of \textit{tyāga} and \textit{bhakti} respectively. In this context it is important to remember that etymologically \textit{bhakti} comes from the Sanskrit verbal root \textit{bhaj}, which means \textit{sevā} (service) in addition to devotion to one’s deity (\textit{bhaj sevāyām}; Chari 1997: 153). Thus, besides serving one’s chosen deity, \textit{bhakti} also implies serving one’s fellow beings. To implement this truth in action, Tirumūlar proposed a fourfold program: (1) love and worship Śiva; (2) love others in the same manner; (3) feed those who are hungry; and (4) if one is unable to do any of the above, Tirumūlar suggests speaking good, kind words to others (as a last resort; TM 252). When a crow comes across a source of food, he calls others to share it with him (TM 250). If a simple crow can do this unto his brethren, asks Tirumūlar, why not the humans? His statement ‘one the family; one the God’ (TM 2401) implies: Let one accept God or deny, let one belong to any family or group, let one speak any language, but as \textit{jīva-}s, they all belong to one single category—human.\endnote{}


\section*{\textit{Reaffirming Vedānta Siddhānta samanvaya }}

The transformation of the Śaiva Siddhānta philosophical school from a Sanskrit-based, pan-India network to a Tamil-based regional center confirms, for Karen Prentiss (who concurs with Lidova’s thesis), the growing rift between these two schools of Śaiva philosophy. The tension between them, she writes, was ultimately crystallized into the Tamil titles (transliterated from Sanskrit) given to the philosophical authors—\textit{Cantanācārya}-s, where \textit{cantanam} means progeny or succession, and \textit{ācārya} means leader and to the \textit{bhakti} poets—\textit{Camayācārya}-s, where \textit{camaya} means ‘religion.’ Prentiss claims that the Tamil version of Śaiva Siddhānta not only acquired its own canon, language, and \textit{guru} lineage; it also acquired its own \textit{religion} [emphasis added], all of which distinguished it from the Sanskrit Śaiva Siddhānta (Prentiss 1996: 240). Prentiss then cites a well-known verse that for her illustrates both the superior status of the Tamil Śaiva Siddhānta texts in Tamil culture and the pure knowledge they contain:

\begin{myquote}
The Vedas are the cow; the true agamas are its milk; the Tamil sung by the four is its ghee; the essence of the book in Tamil written by Meykantar\index{Meykantar} of the famous Venney is the taste of the ghee of great knowledge. In this simile, the essences get progressively more subtle and pure leading up to the ‘ghee’ that is the Tamil hymns and the sweet ‘taste of the ghee’ that is the Tamil Saiva Siddhanta (Prentiss 1982: 257).
\end{myquote}

Against Prentiss it must be pointed out that for Hindus (whether southerners or northerners) the ‘sweet tasting subtle Siddhānta ghee’ is \textit{samarasa} with the ‘Veda cow,’ i.e. shares the same essence (\textit{rasa}) as the cow. Tirumūlar discusses the process of \textit{samskāra} i.e. churning out ‘Siddhānta ghee’ from the ‘Veda cow’ in Tantra eight, section five of TM. The component ‘\textit{anta}’ of Siddhānta carries the connotation of goal or conclusion, as does the English word ‘end.’ To become truly Śiva is the quintessence of the teaching of Vedānta-Siddhānta; the four other \textit{anta}-s (end products) being intermediary. Tirumūlar lists them as: \textit{nādānta, bodhānta, yogānta}, and \textit{kālānta} (conclusion derived from meditation on primal sound, knowledge, yoga, and duration of time respectively; TM 2370-2404). Anticipating the designs of the Orientalist scholars of Hinduism like Lidova and Prentiss who, centuries later, would sunder Āgama from Nigama cutting off thereby the goal from the source and the means to that goal, Tirumūlar would ask this rhetorical question: how can the [Veda/Nigama] cow be considered apart or separate from the [Siddhanta] ghee which is its end product? Each is implicated in the other.


\section*{\textit{Reaffirming Ārya Draviḍa samanvaya}}

The absolute and race based dichotomy of Ārya/ Drāviḍa typically entertained by the Orientalist and Drāviḍa nationalist scholars and politicians disappears upon closer scrutiny. Sanskritist and Indologist Madhav Deshpande\index{Madhav Deshpande} draws attention to an ancient Jaina text--the \textit{Paņņavaņāsutta}, according to whichthere are two kinds of Aryans: (a) ‘\textit{iddhipattāriya,}’ i.e. Aryans who have attained an exalted status and (b) ‘\textit{aniddhipattāriya}’, i.e. Aryans who have not reached an exalted status. The ‘Exalted Aryans’ are the following: (1) \textit{Arahanta;} (2) \textit{Cakkavaţţi;} (3) \textit{Baladeva;} (4) \textit{Vasudeva}; (5) \textit{Cāraņa;} and (6) \textit{Vijjāhara}. It is important to note that the Brahmins have not been included in this Jaina conception of ‘Exalted Aryans.’ The ‘Non-exalted Aryans’ are subdivided into nine different categories: (1) by region (\textit{kşetrārya}); (2) by birth (\textit{jātyārya}); (3) by clan (\textit{kulārya}); (4) by function (\textit{karmārya}); (5) by profession (\textit{śilpārya}); (6) by language (\textit{bhāşārya}); (7) by wisdom (\textit{jñānārya}); (8) by realization (\textit{darśanārya}); and (9) by conduct (\textit{caritrārya})(see Deshpande 1993: 10). This classification suggests that (1) whoever Ārya is or whatever his/her level is; one is so (or can become Ārya) by taking to appropriate profession or learning the proper language and so on. The dichotomy between the Ārya and the non-Ārya/ Drāviḍa is therefore neither racial nor absolute and that one’s status as Ārya is not by birth alone and (2) the socio-cultural-ethical process of Aryanization is dynamic and interactive one. The Buddha endorsed and encouraged the ideal of the Ārya way of life through such expressions as \textit{ariya dhamma, ariya māgga} (the Ārya path) in the Buddhist canonical texts (see Deshpande 1993: 5-10).

The \textit{Yogavāsişţha}\index{Yogavasistha@\textit{Yogavāsişţha}} confirms such an understanding by defining Ārya as one who first diligently performs prescribed duties and only then strives for desired goals while desisting from proscribed deeds. Such an Ārya practicing \textit{yoga} can successfully cultivate \textit{āryatā}(noble character) whereby even the most ignorant individual feels motivated to attain spiritual liberation. Anyone who refuses to emulate Śrī Rāma revolts against nature (\textit{prakŗti}) and against oneself (\textit{ātman}). One that outwardly displays all the social graces while living only for self inwardly is doomed to remain in ignorance (\textit{Yogavāsişţha} 6: 54-55). Drāviḍa, similarly, does not denote a race but the land south of the Vindhya Mountains. In the \textit{Bhāgavata Purāņa} the term Draviḍa \textit{deśa} occurs on six occasions. Balarāma, who is on pilgrimage, bathes in the sacred Kāverī River that flows through the Draviḍa \textit{deśa}. The text goes on to declare that those who drink its water will become devotees of Vāsudeva (Canto 11, chapter 5: 38-40).


\section*{\textit{Reaffirming Sanskrit Tamil samanvaya}}



\section*{}



\section*{}



\section*{}



\section*{}



\section*{}



\section*{}



\section*{}


