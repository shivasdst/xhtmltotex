
\chapter{Umāpati Śivācārya, Śaiva Siddhānta and the Present Times}\label{chapter9}

\Authorline{Sundari Siddhartha}


\section*{Introduction}

Śaiva Siddhānta\index{Saiva Siddhanta@\textit{Śaiva Siddhānta}} is one of the foremost contributions of South India to the religion, theology and philosophy of the world thought. Umāpati of the 14th century is one of the foremost contributors to the cause of Śaiva Siddhānta. The specialty of his contribution is the copiousness and bilinguality of his works. He wrote both in Tamizh and Sanskrit. The Tamizh works (big and small) are more than thirty-five; whereas the available Sanskrit ones comprise three major and two minor works. He has the advantage of the exposure to all the available Sanskrit works. Hence, the depth and expanse of his presentation of both philosophy and rituals of Śaiva Siddhānta, has been effective and influential.

Śaiva Siddhānta continued to flourish and spread after Umāpati’s times too in rituals and worship to Lord Siva. Over time, \textit{āgama}-s\index{agama@\textit{āgama}} and Vedic \textit{vidhi}-s (methods) borrowed from each other and merged increasingly. The priests were well-versed in spiritual needs of the people. They became all-powerful and autocratic too. Then perhaps the predictable happened.

A leader (E. V. Ramasamy\index{Ramasamy, E.V.} Naicker aka Periyār) appeared in the early 20th century. Calling for self-respect and rational thinking, he started a movement. It was anti-Brahmin, anti-Sanskrit, anti- superstition, and anti-caste. Evidently the non-Brahmin Dravidian class was up in arms. As a result, Sanskrit gave way to Tamizh in temple-worship. The post of the priest was thrown open to all qualified men. Spirituality, god and religion were viewed in a new light. Was this attack on spirituality justified? Is the problem solved? If not, what should be the future action? That and more are discussed in this paper.


\section*{Preamble}

\subsection*{Umāpati and Śaivism}

\vskip -4pt

Umāpati Sivācārya holds a special place in the spiritual history of Tamil Nadu. When he appeared on the scene in the 14th century, many spiritual systems were flourishing in Tamil Nadu, besides Śaiva Siddhānta viz. Advaita Vedānta\index{Advaita Vedanta} of Śankarācārya\index{Sankara@\textit{Śankarā}}, Viśiṣṭādvaita of Rāmānuja\index{Ramanuja@\textit{Rāmānuja}} and others. Umāpati happened to belong to a family of \textit{Dikshitar}-s\index{Dikshitar@\textit{Dikshitar}} and hence had a very scholastic upbringing in the Temple town of Cidambaram\index{Cidambaram}, mastering both Sanskrit and Tamil languages. Advaita Vedanta writings were mostly in Sanskrit but Rāmānuja’s follower Vedānta Deśikar used Tamil to a great extent. Śaiva Siddhānta was initially in Tamil but later Sanskrit also came to be used. Āgama-s were proclaimed as the source of Śaiva Siddhānta. They were considered on par with the Veda-s. Āgama-s were in Sanskrit and were later translated into Tamil. Umapati being well-versed in Sanskrit and Tamil had access to relatively more texts in topics related to Śaiva Siddhānta and other philosophical writings.


\subsection*{What is Śaiva Siddhānta?}

\vskip -4pt

Śaivism has flourished in South India from very early times. Many learned scholars and enlightened men and women were involved in the writings on Śaivism and propagating it; the religious and devotional aspect of it was practiced with great fervor by the people. Many devotional songs were composed and became very popular. This was augmented by the elaborate ritual and worship in temples dedicated to Siva and other gods, built with great architectural skill.

Gradually the philosophical and theological elements in the devotional songs were identified, the principles and thoughts therein were clearly enunciated and Śaiva Siddhānta\index{Saiva Siddhanta@\textit{Śaiva Siddhānta}} emerged as an important school of thought. More and more philosophical works, commentaries on them, and new works inspired by the prevalent ones, were written.


\subsection*{Devotional Hymns of Śaivism}

\vskip -4pt

The sacred canons of Śaivism in Tamil Nadu are the devotional songs or hymns of Appar\index{Appar} (aka Tirunāvukkarasar\index{Tirunavukkarasar@\textit{Tirunāvukkarasar}}), Sambandar\index{Sambandar} and Sundarar\index{Sundarar}. The songs of the first seven books are known as Tevāram. The eighth book is Tiruvācakam\index{Tiruvacakam@\textit{Tiruvācakam}} of Mānikkavāsagar\index{Manikkavasagar@\textit{Māṇikkavāsagar}}. These four teachers – Appar, Sundarar, Sambandar, Manikkavācagar -- are referred to as the \textit{Samayācārya}-s\index{Samayacarya@\textit{Samayācārya}}, who exemplified the four spiritual ways – a) \textit{dāsa mārga}\index{dasa marga@\textit{dāsa mārga}} i.e. cary\textit{ā}; b) \textit{satputra mārga}\index{satputra marga@\textit{satputra mārga}} i.e. \textit{kriyā}; c) \textit{sakhā mārga}\index{sakha marga@\textit{sakhā mārga}} i.e. \textit{yoga};\\ d) \textit{sat-mārga} i.e\textit{. jñāna}. The ninth book is a collection of songs. The tenth book is the famous and philosophical Tirumantiram\index{Tirumantiram} of Tirumūlar\index{Tirumular@\textit{Tirumūlar}} (6th century CE). The eleventh is again a collection of songs. The twelfth book is Cekkilar’s\index{Cekkilar} Periya Pur\textit{āṇ}am\index{Periya Puranam@\textit{Periya Purāṇam}} (12th century), which contains the history of all Śaiva saints, the \textit{nāyanmār}-s\index{Nayanmar@\textit{Nāyanmār}}. The language in these texts is Tamil. These are called the devotional Tirumurai-s\index{Tirumurai}. They are not just hymns but can be considered as the philosophical enumerations of the Śaiva Siddhānta.


\subsection*{Śāstra Works}

\vskip -4pt

Besides these Tirumurai-s, there are innumerable works in Sanskrit and Tamil, on both the religious and philosophical teachings of the \textit{āgama}-s. Meykkandār of the 13th century was the first ācārya to give an organized form to those philosophical principles. Śivajñānabodham was his small and short, but deep and profound work. Aruḷnandi\index{Arulnandi@\textit{Aruḷnandi}} (14th century) wrote a scholastic commentary on it - Sivajñānassiddhi\index{Sivajnanassiddhi@\textit{Sivajñānasiddhi}}. Aruḷnandi’s disciple Maraijñāna Sambandar accepted Umāpati as his student. These four ācārya were referred to as the Santānācārya-s. Thus Śaivism acquired its theological form and got the name Śaiva Siddhānta.


\section*{Umāpati Sivācārya\index{Umapati Sivacarya@\textit{Umāpati Sivācārya}}}

With Śaiva Siddhānta thus well-postulated, what was the special role of Umāpati? What and where was the significance of his contribution?

Umapati reiterated all the principles of Śaiva Siddhānta in a logical and scientific way. He used both Sanskrit and Tamil, so that his message could reach even those who knew only one of the two languages. As mentioned before, copiousness of the works and bilingualism\index{bilingualism} highlighted him as unique.

\subsection*{Early Life}

Early in his career, the Chola\index{Chola} king was much impressed by Umapati’s Śaiva Siddhanta. As a mark of his appreciation, the king gave him a pearl palanquin, a day-torch light and a drum-band to accompany him wherever he went. Once in the month of \textit{Vaikāsi}, he was returning after worshipping the Dancing Lord, Siva. It just happened that at that very time, the great philosopher of the Śaiva Siddhānta – Maraijñāna Sambandar\index{Maraijnana Sambandar@\textit{Maraijñāna Sambandar}}, was teaching his students in the vicinity. One of the students pointed out to the teacher that Umāpati was passing by. Hearing that, Maraijñāna Sambandar remarked with disdain – \textit{“patta marattil pagal-kurudan selgindran”}-(Janaki 1996;278–279) (On the dead wood goes the day-blind). Umāpati heard the words. They made him think. (Actually, Umāpati was already looking for a philosophical guide and teacher). He fell at the feet of Maraijñāna Sambandar. Knowing about Umāpati’s scholarship and maturity, Maraijñāna Sambandar accepted him as his disciple.

Umāpati proved a very ardent as also devoted pupil. His marvelous devotion to his teacher was proved in an incident, but ironically that led to Umāpati being exiled from his community. Once the teacher (Maraijñāna Sambandar) and his pupils were walking through the weavers’ street. The teacher stopped and took from one weaver; the starch solution (\textit{kanji}) kept for the clothes and drank it. While he was drinking it, some of it dripped through his elbow. The devoted disciple Umāpati drank the \textit{kanji} dripping through the master’s elbow. Everyone was surprised to see such devotion in Umāpati for his guru. But priests of Thillai\index{Thillai} took exception to it. They excommunicated him from the priestly community. From then on he lived outside Thillai in a place called Korravan Kudi.

In that period, Umāpati performed a few miracles, which were direct orders from the Supreme One. A letter was received by King Cheramān Perumāl\index{Cheraman Perumal@\textit{Cheramān Perumāl}} for the benefit of the devotee Pāna Bhadirar. Another letter was from the Dancing Lord asking his devotee, Umāpati to initiate Perran Sāmban and liberate him. The devotee followed the Lord’s words and gave \textit{mukti }to Perran Sāmban. Later in the presence of Perran Sāmban’s\index{Perran Samban’s@\textit{Perran Sāmban’s}} wife and the king, Umāpati gave mukti to a \textit{Malli} plant also. As already mentioned, once in the flag-hoisting festival of the temple, when the flag was not rising, Umāpati sang the \textit{Kodikkavi} in praise of the Lord and made the flag rise up the pole. The poem consists of five verses; each verse overflowing with the philosophic devotion of Umāpati to Lord Siva. Each verse ends with the refrain -----‘…. for this I hoist aloft the holy flag.’ These mantras have reference to the passage of the Yogi from the grosser to the more subtle level of emancipation\index{emancipation}. Verse 4 says –

\begin{quote}
\textit{‘Añjeḻutthu meṭṭeḻutthu māneḻutthu nāleḻutthu. }\\\textit{Piñjeḻutthu melaipperuveḻutthu – neñjeḻutti}\\\textit{Peśumeḻutthudāne peśāveḻutthinaiyum}\\\textit{Kuśamarkaṭṭa koḍi kaṭṭināne. \textless \url{www.saivasiddhanta.in}\textgreater  }
\end{quote}

Verse 3 brings out God’s Advaita\index{Advaita} relation to man.—‘When seen too close, he does appear as Ananya. His grace to get, I hoist aloft the holy flag.’(\textit{kuraikku marunalhakkoḍi kattināne}.)


\subsection*{Relevance of Umāpati}

Umāpati Śivācārya was the fourth and last of the Santānācārya-s. Many later Tamizh works speak of his life – Pulavar-purānam, Śaiva Santanācārya r-Purānam, Pārthavana māhātmya and Rajendrapura māhātmya and others. Umāpati was a philosopher in the Śaivite strain. He was born at Cidambaram\index{Cidambaram} to Natarāja Dikshitar. He belonged to the Vedic tradition of the priests of Thillai in Tamil Nadu. He was a learned scholar with complete command over both the languages – Sanskrit and Tamizh. Born in the Vedic tradition, he took Śaivite initiation from Maraijñāna Sambandar. He became renowned in the Śaiva Siddhānta and Agamic tradition. We determine his date as 14th century, on the basis of the date of composition given in one of his work – Saṅkalpa nirākaraṇa as 1313 A.D.

His relevance is seen today also. With globalization setting in, there are more people trying to know the significance of the rituals, āgamic practices and temple worship. One should be ready with answers for the next generation. Propagation and practice of anything need maintenance and periodical refreshing, for them to thrive in the changing times.


\subsection*{From Individual to Ādinam (Institution)}

Umāpati’s disciple was Arulnamaccivāyar. His student was Siddar Sivappirakāsar. His follower, Namaccivāya Mūrtikal was the founder of the Tiruvāvāduturai Ādinam. Another disciple of Maraijñana Sambandar\index{Maraijnana Sambandar@\textit{Maraijñana Sambandar}} was Maccuccettiyār\index{Maccuccettiyar@\textit{Maccuccettiyār}}. The eighth generation of his disciples was one Guru jñānasambandar. He founded the Dharmapuram Ādinam.The Śaiva Siddhānta\index{Saiva Siddhanta@\textit{Śaiva Siddhānta}} tradition gets institutionalised at this stage of development. These two ādinam-s or institutions mark the advent of the revival which continues to this day.


\subsection*{Umāpati's Contribution}

As any subject or object grows, it expands, small changes and twists come in and after a long period one sees that things have gone hither thither. Śaiva Siddhānta was no exception. Umāpati observed this. He was an ardent Siva-bhakta and a scholar of the highest order. Right from childhood he was with its theology and grew up listening to the temple worship and rituals. He was so much a part of the system that one could say that he was the contribution of Siva for Śaiva Siddhānta. In this context, his Sanskrit work \textit{Kuñcitānghristava} serves as an apt hand book for knowing the details of Naṭarāja\index{Nataraja@Naṭarāja} and the temple city of Cidambaram\index{Cidambaram}. The long \textit{sragdharā}\index{sragdhara@\textit{sragdharā}} metre is conducive to convey poetic and literary \textit{alamkāra}-s. The \textit{stotra}-s\index{stotra@\textit{stotra}} to Siva bring out lofty religious and philosophic concepts. He himself speaks about the depth and richness of his offering to Siva. (verse 304) -----

\begin{myquote}
\textit{Sadvṛttām sragdharāńgām vipulaguṇavarālankṛtim mañjupāka—}\\\textit{Stāyi- śayyā-rasādyairvibudhaparivṛdhāhlādirītim suvṛttim.}\\\textit{Vedāntārthaprabodhrīm suguṇapadagatim yosmadīyam stutim tām}\\\textit{Angīkurvan sukham me pradiśati guruḥ tam kuñcitāńghrim bhajeham.}

~\hfill (Janaki:1996 168)
\end{myquote}

\newpage


\section*{Political situation in 14th century}

\vskip -4pt

The political situation in Tamil Nadu at the time of Umāpati was one of chaos and unrest. The prosperous Chola dynasty came to an end in the absence of a successor. The Pandya dynasty came to power. But internal fight amongst the princes enabled Allauddin Khilji’s\index{Allauddin Khilji} Commander to take over their kingdom. The local kings and customs were wiped off. This situation affected the religious atmosphere too. The difference between Sanskritic Veda-followers and Tamil-dominant – Āgama-followers became visible. The kings of the later period played safe and engaged and employed scholars from the North. With the Bhakti movement\index{Bhakti movement}, Śaivism, Śankara’s Advaita Vedanta, Rāmānuja’s Vaiṣnavism, Madhva’s Dvaita, etc., there were many schools of thought emerging in the country..

\subsection*{Role of Umāpati}

\vskip -4pt

Umāpati’s task was clear-cut. He had to retain and harmonize Sanskrit and Tamil texts\index{harmonize Sanskrit and Tamil texts} as also the Vedic and Āgamic tenets. He spoke about jñāna (knowledge) being more important than karma (rituals). This helped to decrease the difference between the followers of the Āgamas and Vedas, especially in Cidambaram, the place which was the ultimate ‘Śaiva shrine’.

Not only to philosophy and theology, his contribution to the worship and ritual aspect was also astounding. Umāpati was a scholar of both Sanskrit and Tamil, well-versed in Vedas and Agamas, steeped in Tamil stotra literature. Over and above all this, he was a devotee of the first order. Being of the Dikshitar family, he had the duty to worship and perform Pūjā to Lord Nataraja at Cidambaram. As an employee in the temple engaged in worship, he came into contact with the public and could know and understand the vast group of devotees who came to the temple. He wanted to do something for them, to educate and elevate them. So he used his knowledge of Sanskrit and Tamil and wrote the glory of Cidambaram. In his work, the \textit{Koyil-purāṇam}, he discusses the importance of Siva’s dance at Cidambaram\index{Cidambaram}. This way, the people’s devotion was not only deep but also graced with understanding. He also wanted the people to visit the various holy centres and temples and be benefitted. So he took up the study of the \textit{Tirumurai}-s\index{Tirumurai@\textit{Tirumurai}} and made a list of the holy centres (\textit{Tiruppadi-kovai}). He also listed the hymns for the help of the devotees.

His love and devotion to the devotees of Siva, extended to those who had done service to his God. As a result, Cekkilar\index{Cekkilar}, the biographer of all the Śaiva saints, became an icon for him. Consequently, he wrote a biography of Cekkilar and also of other saints (\textit{Tiruttondar-purana-saram}), inspite of many others having written about them. His vision was ‘the furtherance of Śaiva Siddhānta’. So he gives briefly the biographies of the sixty-three saints and nine groups of saints, mentioned in the Periya-puranam.

Umāpati was conscious of breaking new ground. He could see new vistas and in turn left new direction for posterity. Thus, the following declaration is quite appropriate to quote here. “Whatever is old cannot be deemed good (on account of its antiquity alone), and whatever book comes out today, cannot be judged ill because of its newness.” This was not self-praise or arrogance. He establishes Śaiva Siddhānta in contra-distinction to Vedanta\index{Vedanta} and he wanted the \textit{ācārya}-s to accept it.


\section*{The Ritual Aspect of Śaiva Siddhānta and Present Times}

The rituals associated with the Śaiva Siddhānta are vibrant and in use all over Tamil Nadu. While the philosophy and theories are restricted to the scholars, the devotion, worship and rituals are widely prevalent amongst the common people. The priests are the viaducts, who connect the people to their Lord Siva, through the \textit{āgama}-s, \textit{mantra}-s and the extensive structure of worship.

As the influence of the Western culture permeated South India including Tamil Nadu, many facets of our culture (good and bad), which were taken for granted, came into focus. Things started being viewed critically, socially, politically and with a keen eye and in the years close to Indian Independence, with a lot of anxiety to break with the old and blend with the modern. “Spirituality in the contemporary Tamil Nadu” also came under the scanner of the society. Gradually discontent and anger built up, and in no time, Tamil Nadu witnessed a continuous and elongated attack on the spirituality of one section of the society. Reasons for the discontent snowballed, protestors got more number, more support. Spirituality stood its ground where it could and gave in where it could not.

What are these spiritual rituals and what was objectionable in them? From the 5th century, ritual worship was in existence. In the hoary past, two major religious currents existed side by side, one Vedic and the other Āgamic. Their influence on each other goes back to great antiquity so much so that they cannot be easily segregated in the customs of the people. The \textit{āgama}-s\index{agama@\textit{āgama}} state that the knowledge of Siva leads to liberation, but Vedas lead only to worldly prosperity. The Vedas on the other hand, hold only Vedic ritual as dharma. But with time, each made inroads into the other. And the two began to merge. Vedic mantras have become a common addition to many āgamic rituals. \textit{Vedic Anukramaņi}-s\index{Vedic Anukramani@\textit{Vedic Anukramaņi}} made sure that Vedic texts were intact. These mixed temple rituals continued over the centuries. Meanwhile caste problem came to the forefront.

\subsection*{Attack on Spirituality\index{Attack on Spirituality}}

The Brahmins were intellectually higher and well-versed in Sanskrit. Things came to a head and virtually waited for the spark to ignite the protest. It came in the form of a Tamilian from Erode in Tamil Nadu. E.V. Ramasamy (later known as \textit{Tanthai Periyar}\index{Tanthai Periyar@\textit{Tanthai Periyar}} – Father-the Great) was from a wealthy family. At a young age, he had witnessed incidents of caste and gender discrimination. His contention was that the Dravidian non-brahmins were naïve; they gave donations to the upper class but were not given due respect in return. 

Periyar seems to be right, when he says – ‘since the worship is done mostly in Sanskrit, and temple-worship had been accepted by all sections of the Society, the Sanskrit-knowing, ritual-familiar priests had become powerful'. This Tamilian Social Activist-cum-politician, questions the subjugation of the Dravidian race. He joined the Justice Party of India, became its Head and changed its name to \textit{Dravida Kazhagam}\index{Dravida Kazhagam@\textit{Dravida Kazhagam}} (party). Though heading the ‘Self Respect’ movement, his major effort was with reference to the eradication of CASTE. He opposed the exploitation and marginalization of the non-brahmin Dravidian people of South India by the Indo-Aryans. He felt that people used religion only as a mask to deceive innocent people.

These efforts greatly affected the Temple authorities and the priests. Irony was that though the Brahmins empathized, they could not do much, being a part of the system. Periyar joined Congress and left it within 6 years, only because of this. It was not easy to fight caste nor the caste-Hindus.

It is interesting to observe that at one stage, there were three people against untouchability – Ramaswamy (Periyar), Dr.\ B. R. Ambedkar\index{Ambedkar, B.R.} and Mahatma Gandhi\index{Gandhi, Mahatma}. Dr.\ B. R. Ambedkar was concentrating on the \textit{Dalit}-s or \textit{Śudra}-s, within the fold of Hinduism; Mahatma Gandhi could not take a strong stand against his party (Congress) which had quite a number of brahmins in it. (But Ramasamy, though a Hindu was fighting for the Dravidians – the non-brahmins, and he was not giving up.) He blamed ideology\index{blamed ideology} for social evils\index{social evils}; social practices could and in fact have been revised and most ills have been eradicated. Blaming Hindu principles for these ills continues, confusing the ordinary public and political ideologies. This has led to a lot of exploitation by politicians by playing up on the people’s sentiments and insecurities, fuelling corruption and poor governance.

In a way, he was fighting against the very people (their ignorance), whom he was defending. They had, over the years, accepted the fact that they were not high class. Rationalism\index{Rationalism} had no stand against Tradition. He had to awaken their Self-respect first. The maximum impact that the brahmins had on the others was in the field of Spirituality. God, Religion, and The Powerful Unknown were some of the things, the majority of the people felt drawn to. And there was no dearth of such people in South India. The irony was that initially the Siva-worship in South India was for all and by all. But somewhere at some time caste came in. Even Umāpati Śivācārya\index{Umapati Sivacarya@\textit{Umāpati Śivācārya}} was of the upper caste, but the guru that he chose as his teacher, was not. Even though his knowledge and spiritual caliber were very high, there was objection from the priests of Cidambaram. It is evident that over the years, at some point, Spiritualism had slid down to the level of worldly inequalities. The ritual and worship aspect of Spirituality became the cause of an irrational imbalance in the society. E. V. Ramasamy\index{Ramasamy, E.V.} (Periyār) fought it. His message to the Brahmin community – “In the name of God, religion and \textit{sāstra-s}, you have duped us. …. Give room for rationalism and humanism.”\textless Wikipedia on Periyār E.V.Ramasamy\textgreater 

Fuel was added to fire when in 1937, Hindi was introduced into schools. Hindi was identified with Sanskrit, which in turn was identified with the Indo-Aryans from the North. To counteract Hindi, Tamizh greatness was showcased. But the fight against caste continued. Copies of the Constitution and pictures of Hindu God Rāma were burnt, forceful entry into the temple was attempted and the government was blamed for not doing enough. Periyār’s efforts bore fruit with the introduction of ‘Temple Entry Authorization\index{Temple Entry Authorization} and Indemnity Act 1939’. But even till today, the struggle is on.

Periyār’s disciple, Annādurai\index{Annadurai@\textit{Annādurai}}, was different. He wanted to share in the government and continue the fight. His definition of Theism was ----“Only one race, only one God.” He did not fight against the spiritual values of society. He said – “I don’t break coconuts. I don’t break \textit{pillaiyar}-s either.” \textless Wikipedia on C.N.Annādurai\textgreater 

In 1970, his successor M. Karunānidhi\index{Karunanidhi, M.@Karunānidhi, M.} got the law enacted for all qualified people to become priests. The hereditary priesthood too was abolished by the Supreme Court. Thus Karunānidhi fulfilled the last wish of Periyār\index{Periyar@\textit{Periyār}} to see the appointment of \textit{Arcaka}-s from all castes.

M. G. Rāmachandran\index{Ramachandran, M.G.@\textit{Rāmachandran, M. G.}} and Jayalalithā\index{Jayalalitha@\textit{Jayalalithā}} too had no anti-spirituality stand. \textit{Veda Agama} schools for all were established supported by both DMK and AIADMK. In December 2015, SC upheld the validity of the Act.


\subsection*{Change – misplaced}

\vskip -7pt

The socio-cultural scenario changed. Gradually other demands too were met – reservations, noon meals in schools (started initially by K. Kāmarāj\index{K. Kamaraj@\textit{K. Kāmarāj}}), self-respect marriages without dowry or priests, widow-remarriage, women in army and police etc. (Corruption, bad governance etc. have crept in. But then every system has its ups and downs).

Awareness increased, Dravidian movements gained prominence. Congress lost ground. Brahmins left politics. Tamil political identity was established. Dravidian parties took over the socio-politico scene of Tamil Nadu.

But Periyār’s momentum was missing. Periyār\index{Periyar@\textit{Periyār}} was an idealist to the core. (He rightfully deserved the UNESCO award\index{UNESCO award} that was given to him in 1970). But the successors were more practical and self-centered. We see the waning interest of the Drāvida political parties in taking steps towards this social reform in the State. True that Dravidian Rationality\index{Dravidian Rationality} had pushed religion to the periphery of the Tamil psyche, but it could not achieve complete revolution towards rationalism. Political empowerment beyond caste – yes. But religious system stayed. Why?

It stayed because faith does not completely vanish. The people of Tamil Nadu have always been very religious, spiritual and respected ritualism as a part of their lives. One should understand and respect the intensity of religious emotion within an individual and in the Tamil collective. We may well ask—“Why discard belief?” One can rise in the social and political hierarchy without losing the crucial, emotional and spiritual anchor that is religion.

At the same time one has to concede that religious ethics is very slowly revisiting orthodox\index{orthodoxy} ritualistic practices handed down by tradition: socially regressive or exploitative practices such as sati, child marriage\index{child marriage} etc. are not followed and are slowly disappearing; widow-remarriage is finding increasing acceptance now, but what about other issues that are wrongly connected with religion and spirituality?


\section*{Was The Attack On Spirituality Right Or Wrong?}

It is not easy to justify or condemn anything outright. Nothing is fully right and nothing is completely wrong. The attack on spiritualism in Tamil Nadu was mainly spearheaded by E. V. Ramasamy (Periyār) in the last century, nearly 80 years back. Long before him rational thinking prompted various types of legislative actions. Rājā of Panagal (also a Justice Party\index{Justice Party} member) became the Chief Minister of Madras Presidency (1921 to 1928). He was a well-educated non-brahmin having good relations with the British rulers. He put in place many relevant reforms. One of them was the Reservation policy based on caste.

\begin{longtable}{@{}|l|l|@{}}
\hline
Non-Brahmins & 44\% \\
\hline
Brahmins & 16\% \\
\hline
Muslims & 16\% \\
\hline
Anglo-Indians and Christians & 16\% \\
\hline
Scheduled Class & 8\% \\
\hline
\end{longtable}

(It is now 69\% for non-Brahmins.)

The Justice Party of India was founded mainly to help the non-brahmins against the brahmins’ repression. Rājā of Panagal went about it in a legal way. But such social problems need time. Caste, records say, was a problem as early as 400 AD. Thus one can see that drastic measures were needed. And Periyār had the passion, the finance and also the leadership qualities. But he turned out to be too idealistic and his ideological attacks on traditional institutions may have been too harsh and detrimental to cultural development and the self-esteem of the Indian society. He did not join the government and was honest. We have to judge the attack taking all these facts into consideration.

\subsection*{Opinion of the Brahmins (\textit{Pūrva pakṣa})}

Rituals, customs traditions etc., especially at the spiritual level have to be cultivated and conserved very carefully; agamas and scriptures have to be studied seriously. So we were managing the thousands and thousands of temples all over Tamil Nadu. God has created varņa vyavasthā according to men’s skill (\textit{guņa-karma-vibhāgaśaḥ}). Brahmins are more inclined to studies. We are docile, not given to violence or fight. Members of our community are hard-hit by the Reservation policies\index{Reservation policies}. But if this can mitigate their problems and stop their protests—we should wait patiently.


\subsection*{Opinion of the Attackers (\textit{Uttara pakṣa})}

“Nowhere in history has imbalance or inequality remained unchallenged for long.”

Mahātmā Phule, B. R. Ambedkar\index{Ambedkar, B.R.}, Buddhists and others in India, Sri Lanka and other places have fought not only for the Hindu untouchables\index{untouchable} but also for the \textit{Dalit-}s (the depressed class, the \textit{Pancama}-s). In Tamil Nadu, many of the Dravidians are relatively wealthier and higher intellectually. Even then protest is needed to get back the rights.


\subsection*{Overall Opinion}

It is true that intellectuals are not easily provoked. But then their apathy and indifference is exasperating. They blame everything on Fate or Karma and leave it at that. Moreover, once a group has accepted subjugation as a way of life, it does not question the subjugator and follows him blindly. Periyār\index{Periyar@\textit{Periyār}} pointed out a blind custom in our ritual. “West is sending messages to the planets, while the Tamil Society in India are sending rice and cereals to their dead forefathers through the Brahmins. Who questions it?”\textless Wikipedia on Periyar E.V.Rāmāsamy\textgreater  Perhaps every civilized society in the world considers it important to commemorate their dead forefathers but Periyar made it a point to denigrate the Brahmin priests and Vedic rituals.

Right or wrong, the attack was inevitable. The time had come. When Justice Party was formed in 1920, there were already four non-Brahmin associations there. Also an association ‘South Indian Liberal Foundation’\index{South Indian Liberal Foundation} was renamed as Justice Party with the other four non-brahmin groups joining in, indicating the growing mobilization of the backward classes.

Moreover, when Rājā of Panagal became the Chief Minister of the Madras Presidency\index{Madras Presidency} (1921), he introduced many bills besides the one on ‘Reservation on the basis of castes.’ One of them was the ‘Hindu Religious Endowment Bill’\index{Hindu Religious Endowment Bill} (1926). Trusts were formed and given sizable power over temples. This meant intervention into religious affairs\index{intervention into religious affairs}, leading to State-Regulation of Hindu temples.

The Caste system is the root cause of many ills. Some ills are cruel and inhuman. Of course, this attack is not changing their lot in one day. Yet, it is a step in the right direction for the upper class of non-Brahmins.


\section*{Future Action}

\begin{itemize}
\item Reservation known as affirmative action\index{affirmative action} in countries such as the U.S.A., is a necessity at least for some time. But it has created more castes and has become a card for politicians to play to the vote banks. Reservations should be “economics-based”. Poverty line should be the yardstick. Economically backward Brahmins too should be included.

 \item Amendment of the Constitution is needed. Every citizen of India is first and foremost an Indian. Caste, religion etc. should not interfere in the implementation of the common law of the land.

 \item A fundamental point to note. In no way can this attack be called an attack on Spirituality. Burning of Hindu God Rāma’s\index{Rama@\textit{Rāma}} picture was only a way of protest against inequality, discrimination and sub-human behavior of man to another man in the name of God. On the other hand, demand for Temple-entry, for appointment as priests, for having worship in Tamil instead of Sanskrit etc., cannot be considered ‘Anti-Spirituality’. They are not denying Spirituality. Spirituality should be nurtured like a garden, with the weeds that inevitably grow around it, removed at every stage.

 \item This is just the tip of the iceberg. Superstitions, irrational rituals, absence of the scientific temper\index{scientific temper} continue to force the ignorant to follow the herd mentality, without any attempt to overhaul the system. Worship and customs have to be separated and rationalized.

\end{itemize}

Future action is clear, but implementation is far from easy.

\begin{itemize}
\item Spirituality\index{Spirituality} stays. It cannot go just because it is sullied by the undesirable add-ons like superstitions etc. One cannot throw out the baby with the bath water. The importance of Spirituality has to be accepted by all – theists, atheists, and agnostics. They can differ, they can agree to disagree, but spirituality continues for those who need it, accept it.

 \item Quality control mechanism has to be put in place. Substandard systems cannot stand the test of time.

 \item Sincere conviction, constant vigil, coupled with a sense of justice and compassion for humanity can justify and protect this vast treasure house that makes Tamil Nadu proud of its rich cultural past and identity.

\end{itemize}

\textbf{Everything said and done – “Spirituality has to give way to Humanity; and Humanity has to acknowledge the Diversity.”}

\newpage


\section*{Bibliography}

\begin{thebibliography}{99}
\bibitem{chap10-key01} Dasgupta, Surendranath (2000) \textit{A History of Indian Philosophy Volume 5}. Delhi. Motilal Banarsidas.

 \bibitem{chap10-key02} Janaki, S.S. Ed. (1996) Sri \textit{Umāpati Śivācārya- His Life, Works and Contribution to Śaivism} (Golden Jubilee Publication). Chennai. The Kuppuswami Sastri Research Institute.

 \bibitem{chap10-key03} Janaki. S.S. Ed. (2015) \textit{Siva Temple and Temple Rituals}. Chennai. The Kuppuswami Sastri Research Institute.

 \bibitem{chap10-key04} Mudaliar, A. Shanmugha (1972) Siva \textit{Agamas and their Relationship to Vedas}. Madras.

 \bibitem{chap10-key05} Raja, C. Kunhan ( 1960 ) \textit{Some Fundamental Problems in Indian Philosophy}. Delhi. Motilal Banarsidass.

 \bibitem{chap10-key06} Sundaram, C. S. (1999) \textit{Contribution of Tamil Nadu to Sanskrit}. Chennai. Institute of Asian Studies.

 \bibitem{chap10-key07} Siddhalingaiah, T. B (1979) \textit{Origin and Development of Śaiva Siddhanta upto 14th century.} Madurai. Madurai Kamaraj University.

 \bibitem{chap10-key08} (2001) \textit{Śaiva Rituals and Philosophy}. Chennai. The Kuppuswami Sastri Research Institute.

 \end{thebibliography}

