
\chapter{Rediscovering the Lost Bridges: \textit{Tantrayukti - Tandiravutti}, An Ancient Pan-Indian, Trans-lingual Text-construction Manual}\label{chapter3}

\Authorline{Jayaraman Mahadevan}


\section*{Abstract}

Tamil literature, through millennia of its existence, has evolved methods or conventions of presentation and interpretations of literary texts and topics. \textit{Tantrayukti}-s\index{Tantrayukti@\textit{Tantrayukti}} denoted as \textit{Tandiravutti-s\index{Tandiravutti@\textit{Tandiravutti}} or Utti-s\index{Utti@\textit{Utti}}} in Tamil is one such set of text construction devices. The doctrine of \textit{Tantrayukti} is called \textit{Tandiravuttis} or merely \textit{Utti}-s in Tamil literature. \textit{Utti} is defined by the Tamil Lexicon (1982) as \textit{“Devices adopted in standard literary works”}. \textit{Utti-s} in Tamil are given as list of 32 devices either at the end or at the beginning of the text. In the Tamil literary tradition the history of utilization of this doctrine for text construction could be traced from 1st century B.C.E approximately (\textit{Tolkāppiyam}\index{Tolkappiyam@\textit{Tolkāppiyam}}) to 18th century C.E. (\textit{Cuvāminātham} (18 or 19th century CE) i.e for about 1900 years. It is interesting to note that the same \textit{Tantrayukti} doctrine is present in Sanskrit textual tradition also. The presence of the doctrine in Sanskrit literature can be traced from 3rd century B.C.E. approx. (\textit{Arthaśāstra}\index{Arthasastra@\textit{Arthaśāstra}}) to 12th century C.E (\textit{Vāmakeśvarītantra}) i.e. for over 1400 years. Such an important doctrine in ancient Indian literary tradition has relevance and practical utility.

\begin{enumerate}[{\rm a)}]
\itemsep=0pt
\item The knowledge of the \textit{Tantrayukti} doctrine would help in systematically understanding the structure and the contents of ancient Indian texts cutting across languages.

 \item The comparative studies of the texts between Indian literary traditions can be more structured and systematic with the knowledge of the \textit{Tantrayukti} doctrine that is found in various literary traditions.
 
 \item The doctrine of \textit{Tantrayukti} can be used as a guidebook for construction and interpretation of texts by authors and scholars in current topics of the contemporary era. It is impossible to realize the various promising outcomes from this doctrine unless the source of the doctrine is thoroughly studied and data emerging from it is consolidated and organized into a usable document in current context. It is unfortunate that not much has been done regarding this vital element of Tamil-Sanskrit connection. This paper does not propose to solve all the issues related to \textit{Tantrayukti} doctrine. The limited objective of this paper is to re-introduce the doctrine to the world of scholars, emphasize its importance, indicate the problems and underline the need to strategically invest time and resources in recognizing and utilizing this piece of Pan-Indian Trans-lingual textual methodology document. Once this is done, it would help to set the ‘insider’s norms\index{insider’s norms} and standards’ to understand, evaluate Indian literature of the past and also serve as a golden standard for future linguistic creations.

\end{enumerate}


\section*{Introduction}

There are about a lakh and half manuscripts in public repositories and in private collections in Tamilnadu and Kerala. Of them 12, 250 manuscripts are related to science. Of these 3,500 are independent science; only 230 of them have seen the light of print. Dr.K.V.Sharma who headed the team that conducted the manuscripts survey, the result of which is given below states -

“It would mean that scholars and historians of India have all along been wallowing in 7\% of texts as the whole and sole of science texts produced in the land... ”\hfill (Sastri \& Sarma 2002: Introduction)

This is the state of awareness about Ancient Indian scientific texts. Less known is the fact that there was \textbf{a methodology or a manual} in place, called \textit{Tantrayukti}\index{Tantrayukti@\textit{Tantrayukti}}, for construction of scientific and theoretical texts in India. It is more interesting to note that the same \textit{Tantra Yukti} doctrine has been used as a methodology manual since the beginning of the Tamil literary tradition also. In Tamil tradition \textit{Tantrayuktis} are called as \textit{Tandiravuttis}\index{Tandiravutti@\textit{Tandiravutti}} or merely \textit{Uttis}. This paper, in three sections endeavors to introduce the \textit{Tantrayuktis, TandiravUttis} and their utility in the current context of effectively presenting the insider view with regard to the indigenous literature and the problems related to the rediscovering the lost bridges between these two identical tradition of textual methodology.


\section*{Section 1: \textit{Tantrayukti}}

\subsection*{1. Section Introduction }

Prof. W. K. Lele\index{Lele, W. K.} introduces the doctrine of \textit{Tantrayukti} in the following words -

“Had the ancient Indians conceived a form of scientific composition? Had they developed a method of treatment of the scientific subject in an orderly manner? Did they expound all the aspects of the given subject or did they confine their discussion to a few of them only? Did they reproduce the views of the past and/or the contemporary thinkers? What was their mode of making cross references? What style did they resort to, to establish their new thoughts and theories? Did the idea of rendering the subject matter intelligible as well as enjoyable ever strike them? Had the ancient Indian intellectuals devised any methodology of writing scientific works?”\hfill (Lele 1981:2).

Prof. Lele raised these questions having in mind the doctrine of \textit{Tantrayukti}-s as the answer to all the above series of queries. Scholars have rendered \textit{Tantrayukti} in various terms,

\begin{enumerate}[{\rm i)}]
\itemsep=0pt
\item ‘Methodology in Sanskrit texts on Science’ (Sharma 2006:30)

 \item ‘Forms of Scientific argument’ (Vidyabhushana 1921:24) 

 \item ‘Plan of a treatise’ (Shamashastry 1909:459)

 \item ‘Method of treatment, maxims for the interpretation of textual topics’ (Solomon 1978:73) 

 \item ‘Formal elements which gave form to a scientific work’ (Oberhammer 1968:600)

 \item Methodology of theoretico-scientific treatises in Sanskrit (Lele 1981: cover page)

 \item ‘Methodology and technique, which enable one to compose and interpret scientific treatises correctly and intelligently’. – (Muthuswami 1974:i)

 \item An expedient in the writing of science – (Mittal 2000:23)

\end{enumerate}

The above statements would suffice to adequately introduce \textit{Tantra\-yukti}.


\subsection*{2. Etymology of Tantrayukti}

\vskip -6pt

\textit{Tantrayukti} is a compound of two words in Sanskrit namely \textit{Tantra} and \textit{Yukti}.

\subsubsection*{Tantra}

\vskip -5pt

\textit{Tantra} has a wide range of meanings. One definition of the term is

\vskip -6pt

\begin{verse}
\textit{tanoti vipulānarthān tattvamantrasamanvitān \dev{।}}\\\textit{ trāṇañca kurute yasmāt Tantramityabhidhīyate \dev{॥}}

~\hfill (Lele 1981:19)
\end{verse}

\textit{Tantra can be termed as that which discusses and details subjects and concepts and also that which protects.}

Further -

\begin{verse}
\textit{tatrĀyurvedaḥ śākhā vidyā sūtraṃ jñānaṃ śāstraṃ lakṣaṇaṃ Tantramityanarthāntaram\dev{।}}

~\hfill (\textit{Carakasaṃhitā}, siddhisthāna 12.29-30)
\end{verse}

\textit{Tantra is synonymously used with Āyurveda, a branch of Veda, education, aphorism, knowledge, śāstra and definition.}

But there are certain other viewpoints in presenting the meaning of the term \textit{Tantra} in the compound \textit{Tantrayukti}. Some authors restrict the meaning of the term \textit{Tantra} only to \textit{Āyurveda}.

Aruṇadatta, the author of \textit{Sarvāṅgasundarī}, feels, \textit{Tantra} as that by which the body is protected i.e. \textit{Āyurveda} –\index{Ayurveda@\textit{Āyurveda}}

\begin{verse}
\textit{tantryate dhāryate śarīramaneneti Tantram Āyurvedaḥ}

~\hfill (\textit{Āyurvedadīpikā, siddhisthānam} 12.40)
\end{verse}

Another derivation has been provided to the same end

\begin{verse}
\textit{tantri kuṭumbadhāraṇe tasmin Tantramiti rūpam\dev{।}}\\\textit{kuṭumbaṃ śarīram\dev{।} taddhārayati hitopadeśāhitanivāraṇadvārā\dev{।}}

~\hfill (Muthuswami 1974:i)
\end{verse}

This alternative bases its derivation on one of the roots from which the word \textit{Tantra} is derived i.e. \textit{tantri kuṭumbadhāraṇe}. That which protects the body (\textit{kuṭumba}) by the way of advising what is good and warding away what is not desirable is \textit{Tantra}. This derivation also invariably points to the meaning \textit{Āyurveda}.

But scholars who have commented upon \textit{Tantrayukti} feel that it is not proper to limit the meaning of the term \textit{Tantra} only to \textit{Āyurveda}. Muthuswami opines that such delimitation of the meaning is done by some commentators who –

\begin{myquote}
“Interpret the term in a roundabout way to bring \textit{Āyurveda} alone in its fold”.\hfill (Muthuswami 1974: iii)
\end{myquote}

This is in spite of his admission of the long-standing association \textit{Tantrayukti} with \textit{Āyurveda} texts. This is clear from the following statement 

“The unique style and technique of exposition which developed in \textit{Āyurveda}, as in other branches of study,... are called \textit{Āyurveda Tantrayukti}-s... \textit{Tantrayukti} is as ancient as the literature of \textit{Āyurveda} itself.”\hfill (Muthuswami 1974: iii)

The reason adduced by him to support the view that \textit{Tantra} denotes any śāstraic text is that, in the texts like Amarakośa, Medinīkośa and texts authored by Vāgbhaṭa the term \textit{Tantra} is used to denote simply a science and not \textit{Āyurveda} alone. Further he is of the opinion that other texts like \textit{Arthaśāstra}\index{Arthasastra@\textit{Arthaśāstra}} of Kauṭilya, which is a text on polity, have made use of \textit{Tantrayukti}s. Thus though etymologically the definition can be derived in such a way as to denote \textit{Āyurveda}, from the point of view of its application it is beyond doubt that \textit{Tantra} denotes any branch of science. Thus etymological and conventional usages point to the fact that \textit{Tantra} is used to denote a systematic work of literature.


\subsubsection*{Yukti}

\textit{Yukti}\index{Yukti@\textit{Yukti}}, in the context of \textit{Tantrayukti} is defined as follows -

\begin{verse}
\textit{yujyante saṅkalpyante saṃbadhyante parasparamarthāḥ}\\\textit{samyaktayā prāArthaśāstraraṇike‘bhimate’rthe}\\\textit{virodhavyāghātādidoṣajātamapāsyaanayeti Yuktiḥ\dev{।}}
\end{verse}

\begin{myquote}
\textit{“… that which removes blemishes like impropriety, contradiction, etc., from the intended meaning and thoroughly joins the meanings together.”}

~\hfill (Sharma 1949:1 )
\end{myquote}

Thus the compound \textit{Tantra-Yukti} denotes those devices that aid the composition of a text in a systematic manner to convey intended ideas clearly.


\subsection*{3. The \textit{Tantrayukti} List }

\textit{Tantrayukti-s} are given as a list in ancient texts. Some texts define and illustrate their usage in the text while others merely produce the list. The oldest available \textit{Tantrayukti} list (32* devices) of \textit{Arthaśāstra} is as follows -

\textit{Adhikaraṇa} (Topic), \textit{vidhānam} (statement of contents), \textit{yogaḥ} (employment of sentences), \textit{padārthaḥ} (meaning of the word), \textit{hetvarthaḥ} (reason), \textit{uddeśaḥ} (mention), \textit{nirdeśaḥ} (explanation), \textit{upadeśaḥ} (advice), \textit{apa\-deśaḥ} (reference), \textit{atideśaḥ} (application), \textit{pradeśaḥ} (indication), \textit{upamānam} (analogy), \textit{arthāpattiḥ} (implication), \textit{saṃśayaḥ} (doubt), \textit{prasaṅgaḥ} (situation), \textit{viparyayaḥ} (contrary), \textit{vākyaśeṣaḥ} (completion of a\break sentence), \textit{anumatam} (agreement), \textit{vyakhyānam} (emphasis), \textit{nirvacanam} (derivation), \textit{nidarśanam} (illustration), \textit{apavargaḥ} (exception), \textit{svasaṃjñā} (technical term), \textit{pūrvapakṣaḥ} (prima facie view), \textit{uttarapakṣaḥ} (correct view), \textit{ekāntaḥ} (invariable rule), \textit{anāgatāvekṣaṇam} (reference to a future statement), \textit{atikrantāvekṣaṇam} (reference to a past statement), \textit{niyogaḥ} (restriction), \textit{vikalpaḥ} (option), \textit{samuccayaḥ} (combination), \textit{ūhyam} (what is understood)\hfill (\textit{Arthaśāstra Adhikaraṇa} 15)

As can be seen, the above list contains 32 devices. In the text, the list is followed by a discussion on the doctrine of \textit{Tantrayukti}s. The author of the \textit{Arthaśāstra} mentions the \textit{Tantrayukti}-s, provides the definitions and shows the place of application of those \textit{Yukti}-s in his treatise. This is the earliest and yet complete treatment of \textit{Tantrayukti}s. Scholars feel that though \textit{Arthaśāstra} is the earliest reference regarding \textit{Tantrayukti} it cannot be deemed as the text that originated the doctrine.

Satishchandra Vidyabhushana states that

“The \textit{Tantrayukti} which literally signifies ‘scientific arguments’ was compiled possibly in the 6th century BCE. (i.e. even before \textit{Arthaśāstra}) to systematize debates in pariṣads or learned councils.”

~\hfill (Vidyabhushana 1921;24)

Prof. W.K. Lele is of the opinion that since there are indications of \textit{Tantrayukti} in Aṣṭādhyāyī of Pāṇini\index{Panini@\textit{Pāṇini}} (Lele 1981:5), his era being 5th century B.C.E. Approximately the doctrine should have evolved in the post-Pāṇini era. Thus we conjecture that the form in which the \textit{Tantrayukti}s are found in the \textit{Arthaśāstra} is the result of continuous rumination and evolution over centuries, the literary evidence for which is not traceable, right from the post-Pāṇinian era till the appearance of \textit{Arthaśāstra} in the 3rd century B.C.E.


\subsection*{4. Components Of The Doctrine Of \textit{Tantrayukti}-s}

\textit{Tantrayukti} doctrine consists of three basic elements viz.,

\begin{enumerate}[{\rm a.}]
\itemsep=0pt
\item \textit{Tantrayukti}- \textit{Tantrayukti}-s are the tools that help an author present his ideas in the form of a systematic text.

 \item \textit{Tantra}guṇa - \textit{Tantra}guṇa-s are the characteristics of a good treatise.

 \item \textit{Tantra}doṣa- \textit{Tantra}doṣa-s are the flaws that impair a systematic treatise.

\end{enumerate}

Initially, \textit{Tantrayukti}-s were the whole and sole of the doctrine. Later, scholars added \textit{Tantraguṇa-s} and \textit{Tantradoṣa}-s to the doctrine to widen its scope of applicability.

\newpage


\subsection*{5. The Prominence of The Doctrine Of \textit{Tantrayukti}-s}

Gerhard Oberhammer\index{Oberhammer, Gerhard} opines that \textit{Tantrayukti} doctrine is parallel to the Vāda doctrine on the basis of which the entire Nyāya tradition stands. To quote him:

“Indians tried to analyze the formal elements which gave form to a scientific work in the same way that they tried to analyze the elements of debate; for practical usage they collected these elements in lists and explained them.”\hfill (Oberhammer 1968: 600).

Vātsyāyana, the commentator of the Nyāyasūtras-s quotes \textit{Tantrayukti}-s (\textit{nyāyasūtra-bhāśyam} 1.1.4). This establishes the fact that it is an independent doctrine developed by ancient Indians which was respected and referred to by the Nyāya tradition. This goes to show the position the doctrine of \textit{Tantrayukti}-s occupied in ancient Indian literature.


\subsection*{6. Salient Features of \textit{Tantrayukti} Doctrine}

Three salient features of the \textit{Tantrayukti}\index{Tantrayukti@\textit{Tantrayukti}} doctrine that could be observed are -

\begin{enumerate}[{\rm 1.}]
\itemsep=0pt
\item Held the sway for over 1,700 years as the methodology of composition of texts of all scientific and theoretical treatises in Sanskrit textual tradition.

 \item Touches upon all fundamental aspects in the construction of a compact treatise.

 \item Can be adopted and adjusted according to the requirements of a treatise. (Customizable)

\end{enumerate}

Let us consider above points in detail to gain a better insight into the doctrine.

\subsubsection*{1. Held sway for over 1,700 years in Sanskrit Textual Tradition}

\textit{Tantrayukti} was compiled possibly as early as in the 6th century B.C.E. Texts belonging to various periods and disciplines have made use of these \textit{Yuktis}. A chronological presentation is attempted below:

\newpage

\textbf{(i) \textit{Arthaśāstra}}

It is \textit{Arthaśāstra}\index{Arthasastra@\textit{Arthaśāstra}} that first gave a full-fledged treatment of \textit{Tantrayukti}. It is a known fact that \textit{Arthaśāstra} is an ancient Indian work on polity and statecraft. The last \textit{adhikaraṇa} of \textit{Kauṭilya Arthaśāstra} has\break been styled \textit{Tantrayukti}, which defines and illustrates thirty-two \textit{Tantrayukti-s}. There are divergent views about the date of \textit{Arthaśāstra,} the pendulum swinging between fifth century B.C.E and seventh century C.E. Some scholars say that the text was composed during the reign of Candragupta Maurya, i.e., between 321 B.C.E and 296 B.C.E (Lele 1981:8-9); but generally 5th century BCE is accepted as the period of composition of the text.

\vskip 2pt

\textbf{(ii) \textit{Nyāyasūtrabhāṣya}}

Vātsyāyana, the commentator of \textit{Nyāyasūtra}, is also familiar with \textit{Tantrayukti-s}. He quotes a \textit{Tantrayukti} namely \textit{anumata} while discussing the fourth sūtra of the first \textit{āhnika} in the first chapter of \textit{Gautama’s} Nyāyasūtra\index{Nyayasutra@\textit{Nyāyasūtra}}. The date of \textit{Nyāyasūtrabhāṣya} is generally accepted to be 4th century B.C.E

\vskip 2pt

\textbf{(iii) \textit{Carakasaṃhitā}}

\textit{Carakasaṃhitā}\index{Carakasamhita@\textit{Carakasaṃhitā}} comes next in the order of chronology. In the verses 41 – 45 of the twelfth chapter of the Siddhisthāna, thirty-six \textit{Tantrayukti-s} are enumerated. The sequence of enumeration of \textit{Tantrayukti-s} in \textit{Carakasaṃhitā} differs from that of \textit{Arthaśāstra}. Nomenclatures of some of the \textit{Yukti-s} are also not similar. \textit{Caraka} flourished around First Century B.C.E

\vskip 2pt

\textbf{(iv) \textit{Suśrutasaṃhitā}}

\textit{Suśrutasaṃhitā}\index{Susrutasamhita@\textit{Suśrutasaṃhitā}} is a renowned work on ancient Indian surgery. It has been written in the form of questions and answers between Dhanvantari \textit{and} Suśruta. The period of composition of \textit{Suśrutasaṃhitā} is 4th century C.E. The author has in the sixty-fifth chapter listed thirty-two \textit{Tantrayukti-s}. Though the number of \textit{Yukti-s} is same as that of \textit{Arthaśāstra}, the order of enumeration is different.

\vskip 2pt

\textbf{(v) \textit{Aṣṭāṅgasaṅgraha}}

It is a text on \textit{Āyurveda}\index{Ayurveda@\textit{Āyurveda}} written by Vāgbhaṭa. In the 50th chapter of the \textit{Uttarasthāna} of this work are mentioned thirty-six \textit{Tantrayukti-s}. \textit{Vāgbhaṭa} is said have existed in the period between 3rd and 4th century CE. \textit{Aṣṭāṅgahṛdaya}, another work by the same author, also mentions \textit{Tantrayukti-s}.

\vskip 2pt

\textbf{(vi) \textit{Viṣṇudharmottarapurāṇa}}

In this purāṇa among so many topics, thirty-two \textit{Tantrayukti}s are also defined but not illustrated. These \textit{Tantrayukti-s} appear in the sixth chapter of the third khaṇda. The work is dated between 4th and 5th century C.E.

\vskip 2pt

\textbf{(vii) \textit{Yuktidīpikā}}

It is a rare commentary on \textit{Sāṅkhyakārikā}\index{Sankhyakarika@\textit{Sāṅkhyakārikā}} of \textit{Īśvarakṛṣṇa}. It is approximately dated around 6th century C.E. Ram Chandra Pandya (Larson \& Bhattacharya, 1987: 228), who has critically edited this text, tentatively names one ‘Rājā’ as the author of the work. In the introduction to the text, the author mentions eight devices and names them variously as \textit{Tantrasampat, Tantraguṇa and Tantrayukti}.

\vskip 2pt

\textbf{(viii) \textit{Tantrayukti}vicāra}

It is an independent text on \textit{Tantrayukti-s}. It was written by \textit{Nīlamegha Bhiṣak}. His definitions and illustrations follow the \textit{Carakasaṃhitā}. He has enlisted thirty-six \textit{Tantrayukti-s} in his treatise. He composed his work during 9th century C.E.

There is another independent text called \textit{Tantrayuktiḥ}. The author of the text is unknown. So is the exact date of the work. This text also defines the \textit{Tantrayukti-s} and it belongs to \textit{Āyurveda} tradition. In many places the definitions of this text differ from the previous one. 

\vskip 2pt

(ix) \textit{Īśvarapratyabhijñāvivṛtivimarśinī}, \textit{Svacchandatantra} and \textit{Vama-keśva\-rimata-vivaraṇa}

These are the three texts of \textit{Tantraśāstra}\index{Tantrasastra@\textit{Tantraśāstra}} that have made use of \textit{Tantrayukti-s}.

\textit{Īśvarapratyabhijñāvivṛtivimarśinī} is a text on Kashmir \textit{Śaivism}. Abhinavagupta\index{Abhinavagupta} wrote the text. \textit{Utpaladeva} wrote a text \textit{Ïśvarapratyabhijñā} to preach \textit{Śaiva Siddhānta}. The text comprises of 190 couplets. The same author supplied the gloss and commentary to the text. \textit{Abhinavagupta} first explained the couplets. The text was called \textit{Īśvarapratyabhijñāvimarśinī}. Later he explained the gloss and commentary too, which is the present text. This text cites two instances where the \textit{Tantrayukti Anāgatāvekṣaṇa} is cited (Shastri 1942:147).

\textit{Svacchandatantra} or the \textit{Tantra} of the autonomous is a text on \textit{Śaivasiddhānta}. It is in the form of a dialogue between \textit{Svacchandabhairava (Śiva)} and \textit{Bhairavī (Śakti)}. It is mainly concerned with rituals of initiation and the desiderative practices of a \textit{Sādhaka}. Rājanaka Kṣemarāja (1000-1050 C.E) a disciple of \textit{Abhinavagupta}, has written a commentary on the text. The commentary is to prove that this school of practice is non-dualistic. In three instances we find the utilization of the \textit{Tantrayukti} in this text. (\textit{Svacchandatantra} 1.52, 1.65, 2.130)

\textit{Vamakeśvara-tantra} is one among the most frequently cited and commented upon sources in contemporary South Indian Śrīvidyā tradition (\textit{Śāktatantra}). It is also one among the most important text on the Śrīkula tradition of \textit{Śrīvidyopāsanā}. It is considered the oldest Sanskrit source text in the tradition. Jayaratha, the 12th century Kasmiri scholar, best known as the author of the \textit{Vivaraṇa} of \textit{Abhinavagupta}’s \textit{Tantrāloka}. He calls the source text (\textit{Vamakeśvara-tantra}) as \textit{Vamakeśvaramata}. Accordingly he has styled his commentary as \textit{Vamakeśvarimata-vivaraṇa} (commentary on the \textit{Vamakeśvaramata}). While commenting upon the 58th verse of the first Paṭala and 37th verse of the 4th Paṭala, Jayaratha uitilizes the \textit{Tantrayukti - Anāgatāvekṣaṇa}.

Thus right from 5th century B.C.E to 12th century C.E. century (i.e. for 1700 years) we find references of \textit{Tantrayukti-s}. A doctrine that was in vogue for such a long period of time fell into disuse and was consequently forgotten.


\subsubsection*{2. Holistic treatment of all aspects regarding ther composition of a systematic treatise}

The following verse from \textit{Carakasaṁhitā} succinctly presents the role of \textit{Tantrayukti}s -

\begin{verse}
\textit{yathābujavanasyārkaḥ pradīpo veśmano yathā\dev{।}}\\\textit{prabodhaprakāśārthāḥ tathā Tantrasya yuktayaḥ\dev{॥} }

~\hfill \textit{(Carakasaṁhitā Siddhisthāna 12. 46)}
\end{verse}

\textit{Just as the sun causes the bed of lotuses to bloom or just as the lamp lights up a house, so also the Tantrayukti-s shed light on the meanings of the texts.}

\newpage

To the above quoted verse from the \textit{Carakasaṁhitā}, Cakrapāṇidatta, the author of a commentary on \textit{Carakasaṁhitā} called \textit{Āyurvedadīpikā} makes certain insightful observations regarding the two examples cited by \textit{Caraka} while describing the functions of the \textit{Tantrayukti} doctrine. They are as follows

\begin{enumerate}
\item 
 \textit{“yathāmbujavanasyārkaḥ”}(Just as the sun causes the bed of lotuses to bloom). To this example Cakrapāṇidatta adds the following comment.

\begin{quote}
yathāmbujavanasya saṅkucitasya vistārako'rkastathā tantre saṅkucitārthapradeśasya vistārakās \textit{Tantra} yuktayaḥ, prabodhanāt vistārakā bhavanti ityarkadṛṣṭāntena darśayati
\end{quote}

\begin{myquote}
\textit{As the sun causes the closed petals of the bed of lotuses to spread and bloom, so does the Tantrayukti elucidate and dilate those sections of the śāstra which seem to be cryptic.}\hfill \textit{(Āyurvedadīpikā Siddhisthāna 12. 46)}
\end{myquote}


 \item 
 “pradīpo veśmano yathā”( just as the lamp lights up a house) is the second example. To this \textit{Cakrapāṇidatta} in his work \textit{Āyurveda}dīpikā states that -

\begin{quote}
\textit{dīpadṛṣṭāntena tu, yathā dīpaḥ santameva tamasā pihitaṃ prakāśayati tathā hetvarthādikāsTantrayuktayaḥ santa\-marthaṃ gūḍhaṃ prakāśayantīti darśayati}
\end{quote}

\begin{myquote}
\textit{From the example of the lamp it can be surmised that as the lamp brings to light an object that is very much present but enveloped by darkeness, so does the Tantrayukti-s act to bring out the present but unmanifest meanings in the statements of the treatise.}

~\hfill \textit{(Āyurvedadīpikā Siddhisthāna 12. 46)}
\end{myquote}


\end{enumerate}

The functions of the Tantrayukti envisaged by Caraka as could be understood from the above verses and scholarly comments of \textit{Cakrapāṇidatta} are as follows

\begin{enumerate}[{\rm a.}]
\itemsep=0pt
\item Tantrayukti-s dilate the meanings of the scientific texts, provide newer hidden meanings.

 \item Tantrayukti doctrine is a set of universal codes for construction of any scientific treatise.

 \item Tantrayukti, the structural elements of scientific treatises, aid proper comprehension of the scientific concepts.

\end{enumerate}

Scholars have variously rendered the characteristics of the doctrine. Various functions of \textit{Tantrayukti}-s according to Esther Solomon are

\begin{enumerate}[{\rm a.}]
\itemsep=0pt
\item Interpretation of ideas

 \item Interpreting arrangement of textual words and connections

 \item Description of specific peculiarities of style.\hfill (Solomon 1978:73)

\end{enumerate}

According to N.E. Muthuswami, \textit{Tantrayukti}

\begin{enumerate}[{\rm i)}]
\itemsep=0pt
\item is a method of treatment of scientific subjects

 \item also guarantees the holistic presentation of various aspects of a scientific text construction.

 \item is a tool to fine-tune the diction and style of scientific works.

~\hfill (Muthuswami 1976:i)

\end{enumerate}

On a general perusal of the doctrine it can be understood that the doctrine helps in shaping any given text in the following ways. Among the other functions, \textit{Tantrayuktis} assist -

\begin{itemize}
\item In defining the basic structure of a work (Eg: \textit{prayojana, vidhāna, uddeśa, nirdeśa})

 \item In stating theories and rules (Eg: \textit{niyoga, apavarga, vikalpa, upadeśa, svasaṁjñā})

 \item In explaining Various concepts (Eg: \textit{nirvacana, pūrvapakṣa, anumata, uttarapakṣa, dr̥ṣṭānta})

 \item In fine-tuning diction and style of expression in a treatise (Eg: \textit{vākyaśeṣa, arthāpatti, samuccaya, atikrāntāvekṣaṇa, anāgatāvekṣaṇa})

\end{itemize}

Thus it could be seen that \textit{Tantrayukti}-doctrine touches upon almost all aspects required for a systematic and compact treatise. Let us discuss these with appropriate illustrations.

\paragraph*{a. \textit{Yukti}-s that assist to define the basic structure of a work}

\textit{Yukti-s} such as Prayojana – Objective of the treatise, \textit{Adhikaraṇa} - Topic(s), \textit{vidhāna} - arrangement (of the topics), \textit{uddeśa} - general pattern of enumeration aid the author to format a template based on which the whole text could be constructed. The structure of the text is hereby determined. It will be the foundation on which the superstructure of the treatise will stand. Let us consider the utilization of \textit{vidhāna}, one among the above \textit{Yukti-s} employed in \textit{kāvyamīmāmsā} of Rājaśekhara.

\textit{Kauṭilya} defines the \textit{Tantrayukti Vidhāna} as

\begin{verse}
\textit{śāstrasya prakaraṇānupūrvī vidhānam} –
\end{verse}

\begin{myquote}
\textit{The statement of the order of enumeration of topics of the treatise is arrangement}.\hfill (Arthaśāstra adhikaraṇa 15)
\end{myquote}

Rājaśekhara\index{Rajasekhara@\textit{Rājaśekhara}} shows the place of application of this \textit{Yukti} in his text. He named the first chapter as Śāstrasaṅgrahaḥ. In that, he enlists the topics as –

\begin{quote}
\textit{śāstrasaṅgrahaḥ, śāstranirdeśaḥ, kāvyapuruṣotpattiḥ, śiṣyapra\-tibhā,…deśakālaḥ bhuvanakośaḥ iti kavirahasyaṃ prathamādhi\-karaṇam.}

~\hfill \textit{(kāvyamīmāmsā prathamodhyāyaḥ)}
\end{quote}

Thus by enlisting the topic under discussion with the use of the \textit{Yukti} - \textit{vidhāna} the reader gets a clear picture of the contents of the work.


\paragraph*{b. \textit{Yukti}-s for stating theories and rules}

Any treatise, scientific or literary, would state certain principles, theories and rules on the basis of research, observation and contemplation. \textit{Tantrayukti-s} take into account this aspect and provide various devices that would help to codify those observations. Some of them are as follows - \textit{Niyoga} - invariable rule, \textit{Apavarga} – exceptions, \textit{Vikalpa} - optional rule, \textit{Upadeśa}- directives, prescriptions, advice (of do’s and don’ts), \textit{svasaṃjñā}– technical term. Let us take up the example of the \textit{Tantrayukti Apavarga} (Exceptions). The definition of this Tantrayukti is -

\begin{verse}
\textit{abhivyāpyākarṣaṇamapavargaḥ} -
\end{verse}

\begin{myquote}
\textit{The restriction of a pervasive rule is exception}.\hfill (Suśrutasaṃhitā 6.65.18)
\end{myquote}

Let us see how Suśruta illustrates it in his text. The place of application of the \textit{Yukti-}s is in the context of prescribing medication for poisonous bites. Suśruta says –

\begin{verse}
\textit{asvedyā viṣopasṛṣṭā anyatra kīṭaviṣāditi}

~\hfill (Suśrutasaṃhitā 6.65.18)
\end{verse}

The rule is – Fomentation should not be applied to persons suffering from poisoning. And the exception – But it should be applied to those suffering from insect poisoning. This is a sample of how these \textit{Yukti-s} help to present the rules and observations.


\paragraph*{c. Explanation of Various concepts}

The mere statement of rule or observation or principles might be abstract. It should be accompanied by proper explanation. \textit{Tantrayukti} doctrine had provision to help the author to explain his theory in unambiguous terms. \textit{Yukti-s} such as \textit{Nirvacana} - etymology of terms, \textit{pūrvapakṣa} - Objections (provisional view) \textit{Anumata}- others’ opinion on the topic or the rule, \textit{Uttarapakṣa}- answers (final view) \textit{Dṛṣṭānta} - Use of Analogy, illustrations and examples, instances, are a few that assist explanation.

Let us consider the \textit{Yukti Nirvacana}. It is defined in \textit{Arthaśāstra} as

\begin{quote}
\textit{saṃjñayoktaktasya tadarthena saha yojanam}.
\end{quote}

\begin{myquote}
\textit{The process of connecting the word that has been merely mentioned with its meaning is Nirvacana.}\hfill (\textit{Arthaśāstra adhikaraṇa} 15)
\end{myquote}

Let us consider an example for this cited in the \textit{Tantrayukti} chapter of \textit{Arthaśāstra} (Arthaśāstra adhikaraṇa 15). The term \textit{Vyasana} forms title of the \textit{Vyasanādhikaraṇam} (\textit{Arthaśāstra} 8.2). The \textit{Nirvacana} of the word \textit{vyasana} is provided there by \textit{Kauṭilya} –

\begin{verse}
\textit{vyasyatyenaṃ śreyasya iti vyasanam} –
\end{verse}

\begin{myquote}
\textit{that which deprives a person from his well being is Vyasana.}

~\hfill (\textit{Arthaśāstra} 8.2)
\end{myquote}


\paragraph*{d. \textit{Yukti-}s for fine tuning diction and style of expression in a treatise}

Sometimes an author, anxious to explain a concept, might end up being too verbose and consequently making the concept unintelligible, defeating the very purpose of the treatise. A crisp presentation of concepts is essential. In this process, avoiding repetitions also plays an important place. Further, intelligent use of language generates interest in the mind of readers. Diction plays a vital part even in a scientific and theoretical work even though the ideas presented are about an abstract concept. Below are some \textit{Tantrayukti-s} that would assist an author to that end. \textit{vākyaśeṣa} - completion of a sentence; \textit{arthāpatti} - implication; \textit{Samuccaya} - collection of ideas; \textit{atikrāntāvekṣaṇa} - reference to a past statement; anāgatāvekṣaṇa - reference to a future event.

Let us consider as an example \textit{Atikrāntāvekṣaṇa/atītāvekṣā}. It is defined by \textit{Cakrapāṇidatta, the commentator of Carakasaṃhitā} as -

\begin{verse}
\textit{yadatītamevocyate}\hfill \textit{(siddhisthānam 12.41-44)}
\end{verse}

Referring that which has been stated earlier is \textit{atikrāntāvekṣaṇm/ atītāvekṣā} (there by avoiding restating things). An example from \textit{Carasamhitā} has been pulled out by \textit{Cakrapāṇidatta} for this \textit{Yukti. Caraka} states -

\begin{verse}
\textit{trayodaśavidhaḥ svedaḥ svedādhyāye nidarśitaḥ}

~\hfill \textit{Cikitsāsthāna} (6.3.138).
\end{verse}

By this \textit{Caraka} refers to the 13 types of sweating that has already been stated in 14th Chapter of the \textit{Sutrasthāna}, thereby by employing the \textit{Tantrayukti} called\textit{ atītāvekṣā}.

Thus with these illustrations above it could be seen that the doctrine of \textit{Tantrayukti}s serves as a systematic and complete text construction manual.


\subsubsection*{3. Adaptable to the nature of the \textit{Text}}

The following words of Caraka are very significant.

\begin{verse}
\textit{tantre samāsavyāsokte bhavantyetā hi kṛtsnaśaḥ \dev{।}}\\\textit{ekadeśena dṛśyante samāsābhihite tathā \dev{॥} }

~\hfill (\textit{Carakasaṃhitā} siddhisthāna 12.45)
\end{verse}

\textit{“All these (Tantrayukti-s\index{Tantrayukti@\textit{Tantrayukti}}) occur in a scientific work in brief and in detail. But only some of them occur in a work written in brief.”}

The following scholarly observation drives home the same point -

“It is not as if every item in the above list (of \textit{Tantrayukti-s}) should have to be applied in the case of every work, nor in the same sequence. It only means that these are the methods of presentation of ideas in a work and shall have to be made use of appropriately as required in a context.”\hfill (Sharma 2006: 31-32)

Thus it is evident that the author can determine, depending upon the volume of the text, the number of \textit{Yukti-s} to be employed. This plastic nature of \textit{Tantrayukti-s} may be stated as one of the reasons for it being accepted as a standard and a reference manual for over millennium and half. Further, the presence of these \textit{Tantrayukti-s} seems to be one of the strongest reasons, for making Indian literature rich with scientific/systematic treatises, of which, as mentioned in the beginning; only 7\% have seen the light of print. Thus we have seen the significance and position of the doctrine of \textit{Tantrayukti-}s in the Sanskrit textual tradition.



\section*{Section 2: \textit{Tandiravutti-s}\index{Tandiravutti@\textit{Tandiravutti}}}

\subsection*{1. Section Introduction}

As has been stated in the introduction the application of \textit{Tantrayukti}-doctrine was not limited only to Sanskrit treatises. Ancient Tamil literary tradition also utilized it to a great extent. \textit{Tandiravuttis or Uttis} were part of methods or conventions of presentation and interpretations of literary topics in Tamil literature. Such conventions, in general, were called \textit{Nūnmarapu}\index{Nunmarapu@\textit{Nūnmarapu}}. These conventions were assiduously adhered to while composing texts. All the major authors of grammar texts in Tamil literature stated these canons. The \textit{Nūnmarapu}\index{Nunmarapu@\textit{Nūnmarapu}} normally appeared in the beginning of the work. Some texts mention it at the end also. The other elements of the \textit{Nūnmarapu} includes the ten types of beauties, ten types of errors, types of texts, seven types of opinions etc. \textit{Tantrayukti} has been an integral part of \textit{Nūnmarapu} from the earliest periods.

Generally the term \textit{Utti} is considered as a word borrowed from Sanskrit \textit{Yukti} (\textit{vadamozhicitaivu}). But some Scholars have attempted to define the term from Tamil root. Civalinganar (Note: translation of the Tamil quote has been given below) says “that the term \textit{Utti} is derived from the Tamil root ‘uy’ to apply or to think. That which becomes applied becomes Uytti (\textit{uykkappaṭuvatu uyttiyāyiṛṛu})\hfill (Bhagavati 1981:180).

Various scholars in their expositions have defined these \textit{Utti}-s. T.V. Gopala Iyer presents the following definition –

\begin{verse}
\textit{nūṛcaitikaḻaic cevvvamai collutal Utti}
\end{verse}

\begin{myquote}
\textit{(Presenting the intended concepts of a text clearly/systematically is Utti).}

~\hfill (Gopala Iyer 2005:228)
\end{myquote}


\subsection*{2. Tamil texts}

Tamil literature also has a long history of utilization of the \textit{Tandiravuttis} similar to that of the Sanskrit tradition. Based on current knowledge, it extended from 1st century CE to 18th or 19th century CE. Though the reference in Tamil texts on \textit{Tantrayukti}s begin later when compared to \textit{Arthaśātra}, it extend well beyond its utilization in the Sanskrit texts references to which dry up around the 12th century in Sanskrit literature. Let us briefly consider the works that have been known to have utilized the \textit{Tandiravuttis}.

(i) \textit{Tolkāppiyam,} the oldest available Tamil work, deals with \textit{Tantrayukti-s} in the \textit{Marapiyal} chapter of \textit{Porulatikāram} in \textit{Sūtra} number 665. \textit{Tolkāppianar} also enlists 32 \textit{Uttigal (Yukti-s)}, a la \textit{Arthaśāstra} of Kauṭilya. But V.R. Ramachandra Dikshtar (Dikshtar 1930:82) opines that only 22 \textit{Uttigal}\index{Uttigal@\textit{Uttigal}} of \textit{Tolkāppiyam} match with that of \textit{Arthaśāstra}. The date of \textit{Tolkāppiam} has been tentatively placed at around second or first century BCE.

\newpage

(ii) \textit{Nannūl}\index{Nannul@\textit{Nannūl}} is another grammar text, which is second only to \textit{Tolkāppiyam} in the order of prominence in \textit{Tamil} literature. This work is ascribed to Sage Pavananti. This text too mentions 32 \textit{Tantrayukti-s}. The order of enumeration and treatment of \textit{Tandiravutti-s } differ from that of \textit{Tolkāppiyam.} The date of Nannūl has been fixed at 12th century C.E.

There are other texts in Tamil tradition that mention or make use of \textit{Tantrayukti}s are as follows, (iii) \textit{Yapperungalakkārigai} (11th century C.E), (iv) \textit{Māranalaṅkāram} (1540-65 C.E), (v) \textit{Ilakkaṇaviḻakkam} (17th century) and (vi) \textit{Cuvaminātham} (18th or 19th Century CE). The above are a few known texts that have utilized the doctrine of \textit{Tandiravuttis}.


\subsection*{3. The \textit{Tolkāppiyam} list of \textit{Tandiravuttis}}

As was presented earlier with regard to the Sanskrit \textit{Tantrayukti}-s from Arthaśāstra, the following is the list of \textit{Tandiravuttis} from \textit{Tolkāppiyam\index{Tolkappiyam@\textit{Tolkāppiyam}} (Marapiyal Porulatikāram, Sūtra 665)} with translation of the definitions of the \textit{Uttis} from Tamil lexicons, dictionaries and commentators:

\begin{enumerate}
\itemsep=3pt
\item \textit{nutaliyatu aṟital} - A literary device that consists in stating ones theme before dealing with it in detail. (Gopala Iyer, Vol 16 2005:308)

 \item \textit{atikāramuṟaiye} - The logical order of subjects in a book. (Gopala Iyer, Vol 1 2005:73)

 \item \textit{tokuttukkūṟal} - A literary device that assists in summarizing /abridgement in a statement of what is to be stated in detail. (Gopala Iyer, Vol 16 2005:297)

 \item \textit{vakuttu meyniṟuttal -} Defining concepts in detail that were stated in sets. (Gopala Iyer, Vol 16 2005:362)

 \item \textit{moḻinta poruḷoṭoṉṟavavvayiṉ moḻiyātataṉai muṭṭiṉṟi muṭittal} -\\ a) When the text yields scope for several meanings, deciding upon one meaning in consonance with the earlier portion of the text. b) When the author does not explicitly state the meaning of a particular term, deciding upon a meaning with the help of a secondary text. c) Adding an apt but unstated view that fits into the topic under discussion. (Gopala Iyer, Vol 16 2005:362)

 \item \textit{vārātataṉai vantu muṭittal} - A literary device by which the sense of Sūtra insufficiently expressed is rendered complete. (Gopala Iyer, Vol 16 2005:371)

 \item \textit{vantatu koṇṭu varātatu uṇarttal} – A literary device which consists in applying to an earlier statement an implication drawn from a later explicit statement. (Tamil Lexicon, Vol. 6: 3490)

 \item \textit{muntu muṭintatu talai taṭumāṟṟe} - Establishing an idea by reversing the order of a list of concepts stated already. (Gopala Iyer, Vol 6 2005:3490)

 \item \textit{oppakkūṟal} - Defining a concept in such a way that allows the applicability of the definition to its parts too. (Gopala Iyer, Vol 16 2005:359)

 \item \textit{oru talai moḻital} - Stating that a defintion metioned elsewhere in the text should be applied to the present context also. (Gopala Iyer, Vol 16 2005:244)

 \item \textit{taṉ koḷ kūṟal} - Emphasizing one’s own view, though there may be many views, regarding a concept. (Gopala Iyer, Vol 16 2005:290)

 \item \textit{muṟai piṟaḻāmai} - Describing concepts in the order of enumeration committed earlier and in consonance with the earlier stated view. (Gopala Iyer, Vol 16 2005:329)

 \item \textit{piṟar uṭaṉ paṭṭatu tāṉ uṭaṉ paṭutal} -Accepting the view of other authors views as they are. (Gopala Iyer, Vol 16 2005:222)

 \item \textit{iṟantatu kāttal} - 1) If an idea is not clearly expressed in one occasion explaining the same later in some other context. 2) Refuting the view stated earlier in the text in a later occasion. 3) Not contradicting a rule stated earlier if it had to be repeated elsewhere. 4) Removing archaic language, meaning and conventions from the text. (Vellaivaranar 1994:202,221)

 \item \textit{etiratu poṟṟal} - Updation of the revision of an earlier concept. (Gopala Iyer, Vol 16 2005:363)

 \item \textit{moḻivām eṉṟal} - Stating that a concept would be treated later in the text. (Gopala Iyer, Vol 16 2005:266)

 \item \textit{kūṟiṟṟeṉṟal} - Pointing out that in a previous instance in the text a specific definition has been stated. (Tamil Lexicon Vol5.~1982:1858)

 \item \textit{tāṉ kuṟiyiṭutal} - Following one’s own terminology in one’s work. (Gopala Iyer, Vol 16 2005:243)

 \item \textit{orutalaiyaṉmai} - A literary device that assists in stating the possibility of an alternative interpretation to a present concept that is being defined. (Gopala Iyer, Vol 16 2005:355)

 \item \textit{muṭintatu kāṭṭal} - Rather than explaining all the aspects of a concept at hand just referring to an earlier author’s view regarding the concept. (Gopala Iyer, Vol 16 2005:212-213)

 \item \textit{āṇai kūṟal} - Stating a rule through injunction without adducing a reason. (Gopala Iyer, Vol 16 2005:323)

 \item \textit{pal poruṭku eṟpiṉ nallatu koṭal} - Fixing the best meaning among them that suits the context If a word phrase or sentence seems to yield multiple meanings. (Tamil Lexicon Vol5. 1982:2078)

 \item \textit{tokuttu moḻiyāṉ vakuttatu koṭal} - The literary device of stating by a general term what is described in detail under various heads. (Gopala Iyer, Vol 16 2005:350)

 \item \textit{maṟu talai citaittu taṉ tuṇipuraittal} -Refuting the views of others with proper proofs and establishing one’s own view. (Tamil Lexicon Vol5. 1982:2721)

 \item \textit{piṟaṉ koḷ kūṟal} - Quoting the opinion of others. (Gopala Iyer, Vol 16 2005:340)

 \item \textit{ aṟiyātu uṭampaṭal} - Accepting a concept stated by others, which is unknown to oneself. (Vellaivaranar 1994:229)

 \item \textit{poruḷ iṭaimiṭutal -} Showing that a definition spelled out pertains to a particular word with a certain meaning. (Gopala Iyer, Vol 16 2005:237) 

 \item \textit{etirporuḷuṇarttal -} Keeping in view the future changes in mind while defining a particular concept. (Gopala Iyer, Vol 16 2005:284)

 \item \textit{colliṉ eccam colliyāṅku uṇarttal} - Indicating certain ideas through implication rather than explicitly stating them. (Gopala Iyer, Vol 16 2005:287)

 \item \textit{tantu puṇarntu uraittal} - Bringing an idea expressed earlier or later to a place where it fits the context. (Tamil Lexicon Vol3. 1982:1685)

 \item \textit{ñāpakam kūṟal} - An art where by the full content of a sűtra is merely indicated in general terms. (Tamil Lexicon Vol1. 1982:433)

 \item \textit{uyttukkoṇṭuṇarttal} - A literary method consisting in the use in an exposition of such expression as would stimulate thought or further enquiry. (Vellaivaranar 1994:232)

\end{enumerate}

At the end of the enumeration of the \textit{Utti}-s, \textit{Tolkāppiyam} states that similar to that of the 32 \textit{Utti}-s many more can also be added and they should be accommodated (with in the 32) and appropriately treated.

~\hfill (\textit{Tolkāppiyam Sutra 665}).


\subsection*{4. Tamil texts and More}

This is a brief resume of \textit{Tandiravutti-s}\index{Tandiravutti@\textit{Tandiravutti}} utilized on the Tamil literature. The commentators of \textit{Tolkāppiyam} and the other texts that have been mentioned above appropriately illustrated as to how these \textit{Utti-}s have been employed in the text. This and other features of the \textit{Utti-s} are not elaborated further in this paper, as the pattern is same as Sanskrit \textit{Tantrayukti-}s. One can infer it from the detailed treatment accorded to the Sanskrit \textit{Tantrayukti-}s in section one.

It is interesting to note another observation. Scholars are of the view that \textit{Petakopadeśa} and \textit{Nettipakarana}, two Pāli texts on textual and exegetical methodology are Buddhist treatment upon the whole of \textit{Tantrayukti-s}\index{Tantrayukti@\textit{Tantrayukti}}. (Wardner 1998:319).


\section*{Section 3: Contemporary Utility and Problems}

\subsection*{1. Utility}

It is needless to state that such an important doctrine of the Indic literary tradition has relevance and practical utility. Three aspects can be highlighted as the utility and the high potential of the doctrine in literature study and research pertaining to India.

\begin{enumerate}[{\rm a)}]
\itemsep=1pt
\item The knowledge of the \textit{Tantrayukti} doctrine would help in \textbf{systematically understanding the structure and the contents of ancient Indic texts} cutting across languages. With its hoary history of utilization \textit{Tantrayukti} also can set the norms for interpreting and understanding Indic texts.

 \item The \textbf{comparative studies of the texts between Indian literary traditions can be more structured} and systematic with the knowledge of the \textit{Tantrayukti} doctrine.

 \item The doctrine of \textit{Tantrayukti} can be used as a \textbf{guidebook for construction and interpretation of texts systematically by authors and scholars in current topics} in our contemporary times.

\end{enumerate}


\subsection*{2. Problems with the Doctrine}

It is impossible to realize the various promising outcomes from this doctrine unless the source of the doctrine is thoroughly studied and data emerging form it is consolidated and organized into a useable document. The complete picture has to emerge. It is unfortunate that not much has been done regarding this vital element of Tamil-Sanskrit connection either in Sanskrit academia or in Tamil academia.

In the area of research methods studies in Sanskrit, Lele (Lele 1981), Keshav Chandra Das in his work Elements of Research Methodology in Sanskrit, (Das 1992), Sharma (2006) have discussed \textit{Tantrayukti}. But even in the works of these scholars though \textit{Tantrayukti} from Sanskrit texts are mentioned, insights from Tamil \textit{Tantrayukti-}s have been left out and not even acknowledged. This doctrine should ideally have been part of research methodology papers in at least Sanskrit PG research centers, but unfortunately it is not so. As has been shown earlier almost all foundational texts of \textit{Āyurveda}\index{Ayurveda@\textit{Āyurveda}} beginning from \textit{Carakasaṃhitā} have \textit{Tantrayuktis.} At least scholars in the realm of \textit{Āyurveda} could have taken the lead in establishing the importance of Tantrayukti-s. It might not be wise to harbor high expectation from a field where even teaching of Sanskrit language is limited only to the first year for Bachelors degree. (Regular Courses available in Āyurveda, \url{http://ayush.gov.in})

Among Tamil scholars, awareness in this regard is very limited. Even those who may have been aware, apparently did not pursue it seriously, probably due to negative bias against influence of Sanskrit in Tamil literature. Parallels in the doctrine have been noted by Tamil Scholars and historians of Tamilnadu in like P.S. Subrahmanya Sastri (Sastri 1946), Raghavaiyengar (Raghavaiyengar 1941) and Ramachandra Dikshitar (Dishitar, 1930) in their works. A study of these works reveals that the parallelisms noted by these scholars have just been curiosity studies and have not led to the emergence into solid documents that might have heralded a \textbf{unified methodological document of ancient Indic literature traditions.}

Further, the author of the current paper (Jayaraman 2009) has also made his doctoral study on this doctrine. In the doctoral study the three efforts of comparing the Tantrayuktis of the \textit{Tolkāppiyam} and \textit{Arthaśāstra} Sanskrit traditions (by P.S. Subrahmanya Sastri, Raghavaiyengar and Ramachandra Dikshitar) were analyzed based on the definitions and it was found twenty Tamil and Sanskrit yukti-s in the doctoral study match exactly with each other whereas twelve \textit{yukti-s} are unique to \textit{Tolkāppiyam} and \textit{Arthaśāstra}.

Another notable study regarding this Trans-Tamil-Sanskrit-text construction doctrine is by Jean-Luc Chevillard\index{Chevillard, Jean-Luc} (Chevillard 2009:71-132).

Chevillard’s statement in an article regarding the amount of work that needs to be on this doctrine is worth noting. He states –

\begin{myquote}
“It is only by painstaking comparison of individual glosses for Tantrayuktis (Tantrayuktis) and TUs (Tandiravuttis), making extended use of…many other works … that one can hope to finally rediscover the lost bridges that have linked Tamil and Sanskrit technical literatures for a long time.”\hfill (Chevillard 2009:117)
\end{myquote}


\subsection*{3. The Task at Hand}

Upon reflection, the following can be identified as problems that remain unresolved regarding this valuable doctrine –

\begin{enumerate}
\item At the outset, a thorough and detailed survey of texts of Sanskrit and Tamil traditions to ascertain the degree of utilization of the doctrine is yet to be done.

 \item The list of \textit{Tantrayukti-}s in Sanskrit treatises varies hugely. Their interpretation, nomenclature and illustration of utilization also change from text to text.

 \item Similar is the case of Tamil tradition. The list of \textit{Utti-s} used in one text does not agree with the other texts. One comes across entirely new set of \textit{Utti-s}. Similarly, even in the splitting of the Tamil terms and interpretation of \textit{Utti-s} there is no agreement between two commentators of the same text like \textit{Tolkāppiyam}.

 \item Then comes the arduous task of comparing and consolidating the doctrine as it appears in both Tamil and Sanskrit traditions and other Indic literary traditions.

 \item Tracing the origins of the doctrine is also an unresolved issue.

\end{enumerate}

All these fundamental issues are yet to be addressed regarding this very crucial document.


\section*{Conclusion}

\begin{verse}
\textit{ekasminnapi yasyeha śāstre labdhāspadā matiḥ|}\\\textit{sa śāstramanyadapyāśu yuktijñatvāt prabudhyate||}

~\hfill \textit{(Carakasaṃhitā siddhisthāna 12.47-48)}
\end{verse}

\begin{myquote}
\textit{(Based on the knowledge of the Yukti-s) The one whose mind has gained foothold in one śāstra (discipline of knowledge), can quickly grasp other śāstra-s as well as he is a yuktijña (the one who aware of the Yukti-s or methodology)}
\end{myquote}

Had this statement of Caraka been taken seriously, long back \textit{Tantra\-yukti-}s should have been an integral part of research methodology paper/ coursework for doctoral studies cutting across disciplines and it could also have been utilized as a tool for evaluation of doctoral thesis. As the saying goes, better late than never, at least now, this desideratum has to be addressed.

Further, in the very vital task of defense of Indic tradition, texts therein and their interpretations, it is of fundamental significance to identify and fortify the cornerstones of pan Indian textual methodologies like the doctrine of \textit{Tantrayukti}\index{Tantrayukti@\textit{Tantrayukti}} on which the superstructure rests. Towards this end, effort has been made in this paper to draw the attention of the stakeholders, academia and the intelligentsia towards this Tandiravutti\index{Tandiravutti@\textit{Tandiravutti}}/Tantrayukti doctrine. After all, Swadeshi Indology requires a Swadeshi methodology.


\section*{Bibliography}

\begin{thebibliography}{99}
\bibitem{chap4-key01} \textit{Arthaśāstra.} See Shamashastry (1915).

 \bibitem{chap4-key02} \textit{Āyurvedadīpikā.} \url{http://niimh.nic.in/ebooks/ecaraka/?mod=read}, Accessed on September 3, 2017: E-book: Searchable: e-Book on ‘Carakasaṁhitā’ with Āyurvēdadīpikā commentary of Cakrapāṇidatta, Developed by Central Council for Research in Ayurveda and Siddha (CCRAS), New Delhi.

 \bibitem{chap4-key03} Bhagavati, Dr.\ K. (1981) \textit{Tolkappiya Uraivalam.} Chennai. Porulatikaram, Marapiyal, International Institute of Tamil Studies.

 \bibitem{chap4-key04} \textit{Carakasaṃhitā}. See \textit{Āyurvedadīpikā (2017).}

 \bibitem{chap4-key05} Chevillard, Jean-Luc, 2009b, “The Metagrammatical Vocabulary inside the Lists of 32 \textit{Tantrayukti}-s and its Adaptation to Tamil: Towards a Sanskrit-Tamil Dictionary”, pp.~71--132, in Wilden, Eva (Ed.), Between Preservation and Recreation: Proceedings of a workshop in honour of T.V. Gopal Iyer, Collection Indologie – 109, IFP/EFEO, Pondicherry.

 \bibitem{chap4-key06} Dalal, M.C \& Sastri, R.A (1934) \textit{Kāvyamīmāmsā Rājaśekhara.} Baroda. Oriental Institute.

 \bibitem{chap4-key07} Das, Keshav Chandra (1992) \textit{Elements of Research Methodology in Sanskrit.} New Delhi. Chaukhamba Sanskrit Sansthan.

 \bibitem{chap4-key08} Dishitar, V.R. Ramchandra (1930) “\textit{Tantrayukti}”. \textit{The Journal of Oriental Research}. Kuppuswami Shastri Research Institute, Chennai, Vol.~4, pp.~82--91.

 \bibitem{chap4-key09} Gopala Iyer, T.V (2005) \textit{Tamiz ilakkaṇa peragarāti, Volumes 1--16}. Chennai. Tamizh man Pathippagam (Publishers).

 \bibitem{chap4-key10} Iyer, V. Duraiswamy (1935) \textit{Tolkāppiya Porulatikāram, Part 2}. Chennai. Sadhu Printers.

 \bibitem{chap4-key11} Jayaraman, M. (2009) \textit{The Doctrine of Tantrayukti} – \textit{A Study}. Ph.D. Thesis Submitted to the University of Madras.

 \bibitem{chap4-key12} Kaul, Madhusudan (1921-35) \textit{Svacchanda Tantra with the Udyota of Kṣemarāja}, \textit{7 Volumes}. Srinagar. Government Press.

 \bibitem{chap4-key13} Kāvyamīmāmsā. See (Dalal \& Sastri) (1934)

 \bibitem{chap4-key14} Lele, W.K (1981) \textit{The Doctrine of Tantrayukti}-s. Varanasi. Chaukhamba Surabharati Prakashan.

 \bibitem{chap4-key15} Mittal, Surendra Nath, (2000) \textit{Kautilya Arthashastra Revisited}. New Delhi. Centre for Civilizational Studies, PHISPC.

 \bibitem{chap4-key16} Muraleemadhavan, P.C. \& Sundareswaran, K. (2006) \textit{Sanskrit in Technological Age}. New Delhi. New Bharatiya Book Corporation.

 \bibitem{chap4-key17} Muthuswami, N.E. (1974). \textit{Tantrayuktivicāra}. Trivandrum. Publication Division, Government Ayurveda College.

 \bibitem{chap4-key18} Narayan Ram, Acharya Kavya Tirtha (1945) \textit{Suśrutasaṃhitā.} Bombay. Nirnaya Sagar Press.

 \bibitem{chap4-key19} \textit{Nyāyasūtrabhāśyam.} See Tailanga (1896).

 \bibitem{chap4-key20} Oberhammer, Gerhard (1968) “Notes on \textit{Tantrayukti}-s”\textit{, The Adyar Library Bulletin}, Vol.~31, pp.~600--611

 \bibitem{chap4-key21} Pandeya, Ram Chandra (1967) \textit{Yuktidīpikā}. Delhi. Motilal Banarsidass.

 \bibitem{chap4-key22} Regular Courses available in Ayurveda, \url{http://ayush.gov.in/sites/default/files/6729652177-Regular%20Courses%20available%20in%20Ayurveda_0.pdf}, Accessed on September 2, 2017.

 \bibitem{chap4-key23} Raghavaiyengar, Ra. (1952) \textit{Tamiḻ varalāṟu}. Annamalai Nagar. University Annamalai University. (pp.297-324)

 \bibitem{chap4-key24} Sastri, Vempaṭi Kuṭumba \& Sarma, K. V. (2002) \textit{Science texts in Sanskrit in the Manuscripts repositories of Kerala and Tamilnadu. } New Delhi\textit{.} Rashtriya Sanskrit Sansthan.

 \bibitem{chap4-key25} Sastri, M.K (1945) \textit{Vamakeśvarīmatavivaraṇam}. Research Department, Jammu and Kashmir State, Srinagar.

 \bibitem{chap4-key26} Sastri, P.S. Subrahmanya (1946), An Enquiry into the Relationship of Sanskrit and Tamil, University of Travancore, Trivandrum.

 \bibitem{chap4-key27} Shamashastrary, R. (1915) \textit{Arthaśāstra. } Bangalore\textit{.} Government Press.

 \bibitem{chap4-key28} Sharma, Shankara (1949) \textit{Tantrayukti.}\index{Tantrayukti@\textit{Tantrayukti}} Kottayam. Vaidyasarathi Press.

 \bibitem{chap4-key29} Sharma, K.V (2006) \textit{Science of Ancient India: Certain Novel Facets In Their Study}, Dr.\ K.V. Sharma p.~31--32 See (Muraleemadhavan, P.C. \& Sundareswaran), K. (2006)

 \bibitem{chap4-key30} Shastri, Mukunda Ram (1942) \textit{The Īśvarapratyabhijñāvivṛtivimarśinī of Abhinavagupta}. Bombay. Nirnayasagar.

 \bibitem{chap4-key31} Sengupta, Kaviraja Narendranath, Sengupta, Kaviaraja Balaichandra (1933) \textit{Carakasaṃhitā carakacaturānana śrīmacCakrapāṇidattapraṇītayā Āyurvedadīpikākhyaṭīkayā mahāmahopādhyāya śrīgaṅgādharakaviratnavirājaviracitayā jalpakalpatarusamākhyayā ṭīkayā ca samalaṅkṛtā}. Kolkata. C.K. Sen and Company.

 \bibitem{chap4-key32} Solomon, Esther (1978) Indian \textit{Dialectics: Methods of Philosophical Discussions.} Ahmedabad. B.J. Institute of Learning and research.

 \bibitem{chap4-key33} \textit{Svacchandatantra}. See Kaul (1921--35).

 \bibitem{chap4-key34} \textit{Suśrutasaṃhitā}. See Narayan Ram (1945).

 \bibitem{chap4-key35} Tailanga, Gangadhara Shastri (1896) \textit{Nyayasutras with Vatsyayana Bhashya.} Benares. E.J. Lazarus \& company.

 \bibitem{chap4-key36} \textit{Tamil Lexicon, Volumes I to VI} (1982) Madras. Published Under the Authority of University of Madras.

 \bibitem{chap4-key37} \textit{Tolkāppiyam.} See Iyer (1935).

 \bibitem{chap4-key38} \textit{Vamakeśvarīmatavivaraṇam}. See Sastri (1945)

 \bibitem{chap4-key39} Vellaivaranar (1994) \textit{Tolkāppiyam marapiyal uraivaḻam.} Madurai\textit{.} s Publication Division, Madurai Kamaraj University.

 \bibitem{chap4-key40} Vidyabhushana, Dr.\ Satis Chandra (1921) \textit{A History of Indian Logic.} Calcutta\textit{.} Motilal Banarsidass.

 \bibitem{chap4-key41} Warder, A.K. (1998) \textit{Indian Buddhism}. Delhi. Motilal Banarsidass.

 \end{thebibliography}

