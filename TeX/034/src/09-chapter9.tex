
\chapter{The Nature of Philosophy of Rāmānuja – Emancipation of Identity And not from Social Order; Its Relevance to Current Context}\label{chapter8}

\Authorline{Shanthi Narayanan}


\section*{Abstract}

This paper analyses how the great savant Sri Ramanujacarya is portrayed as a social reformer rather than spiritualleader by modern writers to suit their political purposes, a thousand years later.


\section*{Introduction}

April 2017 was a special month for \textit{SriVaiṣņava}-s the world over. Rāmānuja’s birth millennium was celebrated with gusto and piety in India by his followers. This occasion has created a resurgence of interest into the philosophy Rāmānuja\index{Ramanuja@\textit{Rāmānuja}} propounded, his life and teaching. There have been healthy discussions as to how relevant his life and teachings are even in today’s context. Researchers have been primarily interested in understanding Rāmānuja’s Vedāntic philosophy and comparative studies with other philosophies. His 1000th birth year saw a rise in discussions on the social influence of his philosophy. This has led to the epithet ‘social reformer’\index{Social Reformer} being used to refer to Rāmānuja, more so in recent times.

Rāmānuja’s aim was to help the masses realize their true identity as \textit{ātman}-s and their relationship with their creator, \textit{Brahman}. His life, work and teachings were grounded in this truth. He removed all bases of differences among the masses through this truth and thus helped them look beyond the inequality that existed in society then. The power of his philosophy, as showcased through his life, is such that even in the current day scenario, he attracts followers across all sections of society.

This has inspired even an atheistic political party (\textit{Dravida Munnetra Kazhagam}\index{Dravida Munnetra Kazhagam@\textit{Dravida Munnetra Kazhagam}}) to claim him as a role-model who worked towards removing caste barriers (Yamunan\index{Yamunan, Shrutisagar} 2015:1). To qualify and define Rāmānuja’s deeds simply as a social reformer would misrepresent the focus and intentions with which Rāmānuja expounded \textit{Viśiṣṭādvaita} philosophy. This paper focuses the risk of referring to Rāmānuja as a social reformer to the exclusion of his primary identity as a patron saint who sought to re-establish the primacy of Veda-s in social and religious context. The risk involves dwarfing a giant such as Rāmānuja into someone who was primarily concerned with social emancipation during his times. Contrary to such views that are being propagated, his message was timeless and sought to establish faith in the \textit{Brahmānanda}\index{Brahmānanda@\textit{Brahmananda}} that alone as the way to attain salvation.

The paper seeks to portray how Rāmānuja along with \textit{Āḷvār}\index{Alvar@\textit{Āḷvār}}-s sought to establish Bhakti towards \textit{Brahman }as the lens through which to view this world and attain salvation. His efforts were directed towards respecting and following what Veda-s espoused, which included adherence to religious rituals, social norms such as the \textit{Varņāśrama}\index{Varnasrama@\textit{Varņāśrama}}. The inclusivity\index{inclusivity} shown by Rāmānuja towards all sections of society, was in line with the above norms mentioned.


\section*{Social Reform In Modern Dravidian Terms}

The brand of social reform advocated and initiated by Dravidian political parties in the last century was influenced with the history of caste divisions, political aspirations of each of them and their inter-relationship between these components in the society. The nature of these phenomena was very different from the Bhakti movement\index{Bhakti movement} strengthened by Rāmānuja. When a staunch atheist such as the DMK president calls Rāmānuja a social reformer\index{Social Reformer}, there arise inconsistencies. Studying the social impact of Rāmānuja’s philosophy and work is different from calling him a social reformer since the latter assumes an intention and focus on improving society and its people. In the following section, the paper delves into the history of the Dravidian movement and the brand of social reform it believed in and propagated in society. In the second section of the paper, we look into the nature of movement set forth by Rāmānuja’s intentions and actions and the legacy that he inherited. In the concluding section, we look into the veracity of the inconsistencies mentioned above, having looked into the nature of the two movements.

\subsection*{The Origin of the Dravidian Identity\index{Dravidian Identity} in Southern India politics\index{Southern India politics}}

In his book, \textit{Dravida Iyakka Varalaru}, R Muthukumar\index{Muthukumar, R.} records the events that led to the Dravidian movement in Tamilnadu since 1893.The author worked with \textit{Kalki}\index{Kalki@\textit{Kalki}}, a popular Tamil magazine in the beginning of his career and is a political analyst and writer. He traces the Dravidian political identity back to the period of stabilization of the Imperial Rule\index{Imperial Rule} in the Indian subcontinent. The British were looking at institutionalizing its rule over the diverse regions of India. They saw potential in the learned Brahmin class of Indian society in recognition of their grasp of subjects and ideas, their learning agility and sharp intellect. Muthukumar attributes the desirability of Brahmins to their educated status in society and their interest and ability in mastering the English language. He mentions that the Brahmins deprived the non-Brahmin class\index{non-Brahmin class} from government jobs. If the latter managed to enter the government workforce, he reports that Brahmins monopolised senior positions and harassed the incumbents belonging to lower castes owing to their superior position in hierarchy (Muthukumar 2010: Part 1, 15–18).

Eugene F Irschick\index{ Irschik, Eugene F.} in \textit{Politics and Social Conflict in South India} identifies three constituent units of south Indian society - the Brahmins, the Non-Brahmins and the untouchables\index{untouchable}. He says that the southern society was largely influenced by religious beliefs and traditional customs. The Brahmin was considered the regulator of religious and social life. The Non-Brahmins were dependent on the Brahmin for conducting religious ceremonies. There was a sense of dependence and competition between the two classes. Therefore, he says, the conflict that developed at the beginning of the twentieth century was “the articulation of a pre-existing”\index{social rivalry} (Irschick 1969:11–12).

At the time of the \textit{Dravida} identity being primed for attracting the masses belonging to the non-Brahmin across South India, Annie Besant\index{Annie Besant} arrived on the scene with her vociferous support for ‘Home-Rule’\index{Home-Rule} and conviction in the glories of the Indian past and Hinduism. This brought her into direct conflict with the non-Brahmin groups such as the ‘Madras Social Reform Association’\index{Madras Social Reform Association}. Annie Besant was seen as sensitive to the problems that arose from the caste divisions but reticent to initiate reforms on that front.

Muthukumar delineates that such growing support for Brahmins due to Ms. Besant’s intervention spurred the Dravidian movement into action. The non-Brahmins organized themselves into ‘The Madras Non-Brahmin Association’ under the leadership of two lawyers, P Subramaniam and M Purushotama Naidu (Muthukumar 2010: Part 1, 19). The aim of this group was to look into the issues faced by people not belonging to the Brahmin class and help them improve their lot. This pioneering association had lofty goals such as assisting non-Brahmins in securing government posts, helping the meritorious in the class in continuing higher education, encouraging the youth to travel abroad to learn vocational skill\index{vocational skill}s so that they can return as potential entrepreneurs to their homeland. The initial character of the movement was that of improvement of social and financial standing of its target group – the non-Brahmin class.

The movement saw other groups mushrooming around the same period. Significant among them was the ‘\textit{Dravida} Association’\index{Dravida Association@\textit{Dravida Association}} formed in 1912 managed by C Natesa Mudaliar, the secretary of the association. They adopted the name ‘\textit{Dravida}’ from an earlier group that called itself -\textit{Dravida Jana Sabha}\index{Dravida Jana Sabha@\textit{Dravida Jana Sabha}}. It was the first time that the word came into popular use to denote the people of non-Brahmin class belonging to the Kerala, Karnataka, Andhra and Tamilnadu regions. This was an important tipping point in the history of the movement. The movement gained an identity as the custodians of ‘\textit{Dravida}’ (Muthukumar 2010: Part 1- 24).


\subsection*{E V Ramasamy\index{Ramasamy, E.V.} and Anti-Brahmin divide\index{Anti-Brahmin divide}}

Natesa Mudaliar was instrumental in bringing two influential figures in Madras into the fray – Dr. T M Nair and P Tyagaraja Chetti. The Justice Party\index{Justice Party}, begun under the leadership of Dr. Nair, aimed at promoting the political interests of the Non-Brahmin class. The Justice Party and the Madras Presidency Association\index{Madras Presidency Association} (M.P.A) worked together on various issues related to improving the presence and say of the Non-Brahmin lot in political matters and elections.

In 1917, Edwin Montagu\index{Montagu, Edwin}, Secretary of State for India proposed political reforms to increase representation of Indians in the government. This was not satisfactory to the Justice Party since it did not accept their demand for representation of Non-Brahmins as a sub group. In this context, Dr. Nair spoke out against Brahmins and derided their customs and traditions in a series of conferences held to discuss the reforms. He informed the gathering how \textit{Ādi-Drāviḍa-s}\index{Adi-Dravida@\textit{Ādi-Drāviḍa}} were the original inhabitants of South India and how Aryans (the present day Brahmin’s ancestors) intruded into the South of India to dominate Dravidians on account of their skin colour and race using Veda-s to establish their superiority. He said that the customs and traditions such as the \textit{Varņāśrama}\index{Varnasrama@\textit{Varņāśrama}} continue to be used by Brahmins to dominate \textit{Śūdra}-s\index{Sudra@\textit{Śudra}} in all spheres of life and keep them away from education and equality and rights of any nature. He urged the Non-Brahmin classes to understand that such narrative was meant to keep them away from their rights. He called for action on their part to rise up in society, to claim equality and their just rights (Muthukumar 2010: Part 1, 55–56).

In 1920, the M.P.A was dissolved and some of the Non-Brahmin leaders like E V Ramasamy\index{Ramasamy, E.V.} Naicker joined the Congress after being persuaded by C Rajagopalachari\index{Rajagopalachari, C.}, an influential \textit{Sri Vaiṣņava }who was working towards building a pro-Gandhi Congress party in the South.

E V Ramasamy (EVR) had on some occasions supported Rajagopalachari such as during support of prohibition and promotion of Khadi\index{Khadi} among South Indian dwellers.

E V R was an accomplished orator and his passionate speeches on abolishing untouchability\index{untouchability} captured the imagination of people. He urged people to see that a God who allowed such discrimination in the name of castes, was not worth believing in or being revered.

\newpage

In 1924, E V R rose to popularity after leading the \textit{Vaikom} agitation\index{Vaikom agitation@\textit{Vaikom agitation}}. It was a conflict that arose over the right of untouchables to use certain roads outside a temple in \textit{Vaikom} in Travancore state, to the west of Madras (Irschick 1969:284–285). He shaped the Dravidian social movement’s focus of recognizing the Brahmins as the real threat rather than the British who ruled India then. He called his fellow men to establish equality with Brahmins before they sought equality with foreigners. Emboldened, E V R asked the Congress to recognize the communal representation of non-Brahmins in public services and representative bodies. The Congress, assembled for the \textit{Conjeevaram} conference\index{Conjeevaram conference@\textit{Conjeevaram conference}} disallowed the resolutions submitted by E V R. Since then, E V R associated Congress with Brahmins and vowed to remove them from the political scene (Muthukumar 2010: Part 1, 272).

In 1926, he sought to encourage the lower castes to claim dignity and act with self-respect to come out of their subservient past. The Self Respect movement was born. In this context, Muthukumar quotes from E V R’s speech in his \textit{Dravida Iyakka Varalaru}\index{Dravida Iyakka Varalaru@\textit{Dravida Iyakka Varalaru}}. It translates into –“God, Religion, Śāstra-s, Purāņa-s, Itihāsa-s, Veda-s, \textit{Mokṣa, Naraka}, Ghost, Spirit, \textit{Jyotiṣa, Mantra}-s,\textit{ Pooja, Yagña}, Customs, Festivals are all fiction created by mind and thus have to be totally abolished”, says E V R (Muthukumar 2010: Part 1, 147).The Self Respect movement stabilized and one of its core values was no-God belief, abolishing superstitious practices and \textit{samadharma}\index{samadharma@\textit{samadharma}} in society (equality with Brahmins).


\subsection*{Anna and his philosophy}

In 1935, under the mentorship of E V R, the Justice party started a new magazine, \textit{Drāviḍa Nādu}\index{Dravida Nadu@\textit{Drāviḍa Nādu}} with C N Annadurai\index{Annadurai, C.N.} as the editor. Annadurai’s skill in prose and poetry attracted readers to this magazine. Keeping in with the spirit of E V R’s self-respect movement, Annadurai focused on writing critically about Hinduism and its sacred texts. He continued EVR’s legacy when he portrayed \textit{Rāmāyaņa}\index{Ramayana@\textit{Rāmāyaņa}} as a tool of Aryan dominance\index{tool of Aryan dominance}. He wrote ‘\textit{Kambarasam’ ‘Thee Paravattum’} and ‘\textit{Kamban Kaaviyam’} directed as fiery criticism against \textit{Rāmāyana}. EVR depicted Rāma\index{arya@\textit{Rāma}} as hypocritical and frail and Rāvana\index{Ravana@\textit{Rāvana}}, ‘the true Dravidian’ as possessing all kingly qualities and therefore the real hero of the epic. \textit{Rāmāyana }was considered derived as an ideological instrument in North Indian Sanskrit Literature used by Brahmins to prove their supremacy over Dravidians – the non-Brahmin class of people (Gangatharan\index{Gangatharan, A.} 2009: 82).

EVR staunchly opposed Anna entering politics and taking the help of the medium of cinema to reach people (Muthukumar 2010: Part 1, 213). He felt that it would dilute the nature and focus of the social movement that \textit{Dravida Kazhagam}\index{Dravida Kazhagam@\textit{Dravida Kazhagam}} believed in. Anna reiterated his belief that people ought to be liberated from superstitious beliefs through the use of scientific reasoning. He also stated that he was against brahminical worship of idols and was known for breaking Hindu Idols in solidarity with \textit{Dravida Kazhagam’s} policy (Ramanathan 2015). However, Anna modified his stance on belief in God when he proclaimed - “One God, One race”. Anna retained the core policies of \textit{Dravida Kazhagam} when he started \textit{Dravida Munnetra Kazhagam}\index{Dravida Munnetra Kazhagam@\textit{Dravida Munnetra Kazhagam}} (DMK) and continued to revere EVR as his mentor.


\subsection*{DMK and Aryan-Dravidian divide theory\index{Aryan-Dravidian theory}}

M Karunanidhi\index{Karunanidhi, M.@Karunānidhi, M.} succeeded Anna as the chief of DMK when the latter passed away two years after coming to power politically. Muthukumar\index{Muthukumar, R.} reports that EVR was not for merging social reform and political aspirations. However, over time, he realized the power that politics lent to his social movement. Hence he supported Karunanidhi during the many tribulations and achievements that DMK experienced since he hoped Karunanidhi would strengthen the self-respect movement. As time went by, Karunanidhi’s focus shifted from social reform to managing to secure DMK’s position in terms of its relationship with the Central Government and control dissent within the party from time to time. Muthukumar conclusively writes that out of all the Dravidian parties, DMK under Karunanidhi’s leadership alone managed to translate some of the core principles of DK into action. They include, he says, encouraging the common people to use scientific reasoning as part of the self-respect movement, opposing superstitious beliefs, encouraging the \textit{tamizhan} identity\index{tamizhan identity@\textit{tamizhan identity}} (being indigenous to \textit{Dravida Nadu} as opposed to the invading Brahmins of Aryan lineage), and giving atheism a political identity (Muthukumar 2010: Part 2, 334).

\newpage


\subsection*{Change of stance}

DMK under Karunanidhi’s patronage continued its predecessor Anna’s legacy by toning down anti-God rhetoric of \textit{Dravida Kazhagam} and focusing on securing its political power through the support of non-Brahmin communities. DMK executed what EVR and Anna only envisaged – wiping out the collective memory of the Brahmin community in Tamilnadu (Viswanath\index{Viswanath, Rohit} 2015). This was accomplished by institutionalising their theory of Aryan-Dravidian divide.

Over time, DMK realised that their atheist beliefs were not popular with the majority in Tamilnadu. The majority of the society found succour in following their beliefs in traditions and customs of religion. Recognizing the trend, DMK is trying to rebrand itself as the flag bearer of caste equality. DMK’s arch rival, AIADMK stole a march over the former in garnering the support of people of Tamilnadu through charismatic leadership and leveraging the religious fervour\index{religious fervour} of people. To add to the mix, BJP threw a gauntlet to southern states when it swept the polls in 2014. Currently, BJP is seen as making an effort to embrace all castes, growing beyond its image of being a pro-Brahmin party. This has necessitated the need for drastic steps in resuscitating the DMK brand. To gain people’s favour, DMK seems to have realized that it needs to distance itself from the anti-Brahmin Dravidian narrative without alienating the bulk of non-Brahmin castes in Tamilnadu (Ramanathan\index{Ramanathan, S.} 2015).

The master communicator and strategist that M Karunanidhi was, he decided to build this new brand of DMK by associating himself with Rāmānuja, the patron saint of Brahmin community. Rāmānuja is seen by all castes as a saint who strived for everyone’s salvation, beyond all differences in castes, gender, etc. Karunanidhi penned the dialogues and script for the teleserial on Rāmānuja. The former categorically stated that he accepted the ideas of Rāmānuja because he transcended religion and caste and wanted all communities to be treated equally. He also told THE HINDU, “Rāmānuja showed through his life that the oppressed and backward communities were not those to be hated and side-lined” (Yamunan\index{Yamunan, Shrutisagar} 2015).

The teleserial deflects the attention of the viewers from Rāmānuja as the custodian of Vedic culture, customs and traditions and proponent of the Bhakti movement\index{Bhakti movement} to Rāmānuja as a social reformer who, moved by the \textit{adharma} in society, decided to defy the then contemporary social norms to show the path to salvation to downtrodden people.

\vskip -20pt


\subsection*{Marxists and their views}

\vskip -5pt

It is not the DMK chief alone who extrapolates Rāmānuja’s attitude to caste inequality. According to Marxist historians, the Bhakti movement was a progressive one, as it propagated revolutionary thought in those days – that everyone was entitled to the benefits of knowledge and benediction of the Supreme – in defiance of prevailing orthodoxy\index{orthodoxy}. However, some of them stretch this thought too far to suit their world view. A Marxist writer from the former Soviet Union says that, “under conditions prevailing in India at the time, this idea had a profound democratic meaning, as it sanctified the struggle against feudal and caste divisions.” This is reading too far into Marxism, since there was no revolt against feudalism.

Rāmānuja formulated his doctrines within the bounds of feudalism\index{feudalism}. He proclaimed that Bhakti transcended all caste distinctions, and he stood for equality in the worship of god. According to him all worshippers of Viṣṇu were equal, because all people – high or low were manifestations of \textit{Brahman}. Fighting against tyranny was not the exhortation of the \textit{Vaiṣņava} saints. The Bhakti movement took the total surrender or \textit{prapatti}\index{prapatti@\textit{prapatti}} as the main principle and people derived the consolation it could give to the irremediable suffering, they endured in the days when the movement took its roots. The \textit{Viśiṣṭādvaita} tradition\index{Visistadvaita tradition@\textit{Viśiṣṭādvaita tradition}} is rooted in the final search towards freedom from the cycle of birth and death and the attainment of merger with the ultimate; hence doing one’s duty was considered as one’s avocation, whatever that may be, irrespective of the fruits of labour. Their focus was on attainment of the ultimate through Bhakti than on reforming the social mores of their time (Seshadri 1996: 298).


\section*{Rāmānuja And Āḷvār-s – Nature of their\hfill \break contribution}

\subsection*{\textit{Āḷvār}-s – The Agents of Change}

\vskip -6pt

Thus far, we have seen how the Dravidian party, DMK inherited its policies from Dravida Kazhagam\index{Dravida Kazhagam}. Dravida Kazhagam came into being when caste divide in the society had reached its zenith. The era of Indian independence coincided with this context and many leaders emerged who gave vent of this oppressive climate. One of them was E V Ramasamy\index{Ramasamy, E.V.}. Due to his early experiences in life of caste divide\index{caste divide} and separation, he was against the Brahmin class and the importance given to this section of society in social, religious and political context. He also propagated the self-respect movement\index{self-respect movement} in 1925 which demanded equality for all classes in the society and encouraged backward classes to command self-respect in a caste-based hierarchical society.

Long before such clamour for equality of classes in society, in the early medieval period, \textit{Āḷvār}-s, Tamil poet-saints expressed \textit{samadharma}\index{samadharma@\textit{samadharma}} through their life and works. They espoused Bhakti for the supreme lord, Vishnu in their songs of longing, ecstasy and service. Tradition places them roughly between 4200 and 2700 B. C. Historical, linguistic and literary research however, assigns to them a period from the fifth or the sixth to the ninth century A.D(Chari 1997:10). There were a total of twelve āḷvār-s, one of them being a woman. They came from different classes of the society. Their compositions comprising 4000 hymns are collectively referred to as \textit{Nālāyira Divya Prabandham}\index{Nalayira Divya Prabandham@\textit{Nālāyira Divya Prabandham}}. For the first time in the history of \textit{Sanātana Dharma}\index{Sanatana Dharma@\textit{Sanātana Dharma}}, Veda-s became accessible to all four \textit{varnas}, men and women since \textit{Divya Prabandham }was rendered in Tamil. Veda-s, Upaniṣad-s\index{Upanisad@\textit{Upaniṣad}} and Purāna-s\index{Purana@\textit{Purāna}} originally rendered in Sanskrit were the monopoly of only the learned men till then.

The \textit{Āḷvār}-s expressed through their songs that devotion or\textit{ Bhakti} for the Supreme \textit{Brahman} is the only equalizer in all worlds. \textit{Bhakti }was the only qualification mandated in their hymns to attain salvation.

\begin{enumerate}[{\rm 1.}]
\itemsep=0pt
\item Periya Āḷvār\index{Periya Aḷvar@\textit{Periya Āḷvār}} was called so, not because of his Brahmin identity but for his devotional outpouring in concern for the wellbeing of the all-powerful Lord. In \textit{Periya Āḷvār Tirumoḷi}\index{Tirumoli@\textit{Tirumoḷi}} (1–3) he welcomes people who seek Nārāyana as the last and only refuge and warns people who are slaves to senses not to join his group that chants praises of the Supreme Lord Nārāyana. The āḷvār emphasizes how important the association with other devotees / \textit{bhāgavatas}\index{bhagavatas@\textit{bhāgavatas}} is for progress in this path. Emancipation was not divorced from belief in God. In \textit{Periya Āḷvār Thirumoḷi}(4–4), he berates people, calling them wicked and ungrateful for not thinking of the first cause, Nārāyaṇa, even once in their life. He also details the behaviour one ought to possess to attain the one goal of life: salvation or Mokṣa\index{Moksa@\textit{Mokṣa}}. The āḷvār says people ought to receive guests with honour, perform dedicated temple service and pursue Vedic studies all their life. In this context, it is important to note that a person is called ‘\textit{vaidika}’ irrespective of his class in society, provided he understands and carries out his duties in line with what the Veda-s prescribe.

 \item Āṇḍāl\index{Andal@\textit{Āņḍāl}}, the only female āḷvār in this sect, occupies a unique position. She was known to be six years old when she composed \textit{Tiruppāvai}\index{Tiruppavai@\textit{Tiruppāvai}} and \textit{Nācciyār Thirumoḷi}\index{Nacciyar Thirumoli@\textit{Nācciyār Thirumoḷi}}. Her compositions are hailed by \textit{Sri Vaiṣņava}ācārya-s as containing the secrets of creation, the goal of life on earth and how to attain it. She reveals in \textit{Thirupāvai} how all are welcome to take the name of Nārāyaṇa and how intricately he is connected and interwoven with this world and nature. She is an example of how age is never a factor in recognizing words of wisdom from anybody, in this philosophy and tradition.

 \item Tirupānāḷvār\index{Tirupanaḷvar@\textit{Tirupānāḷvār}} was not allowed to enter the temple precincts in accordance with the religious custom then prevalent. Once when he was singing in ecstasy about the lord, he became the subject of ire of the temple priest. The Lord appears in the priest’s dream and asks him to bring the āḷvār with all respect and honour for an audience with Him. The priest carries the āḷvār on his shoulders to visit the Lord. The focus in this episode, as recorded in the early texts of Sri Vaishnavism, has been on the status accorded to the devotee of the Lord and not the caste divisions\index{caste divisions}.

 \item Kulaśekara Āḷvār\index{Kulasekara Alvar@\textit{Kulaśekara Āḷvār}} is known to have lamented that he would rather be a worker in the temple who can enter the sanctum sanctorum at will, than a king who does not possess the same privilege. He emphasizes through such thought that being a devotee is the highest status one can attain and that earthly titles do not hold much meaning.

 \item Tirumaṅgai Āḷvār\index{Tirumangai Alvar@\textit{Tirumańgai Āḷvār}} in his \textit{Periya Thirumoḷi}(1–9) – \textit{Kulam Tharum, Selvam Thandhidum }explains how one wastes a precious life by sinking in material wants and desires. He further shows the way out, saying the sacred \textit{mantra} of Nārāyana is the only solution for the ones caught in the cycle of birth and death. Taking the name of Narayana raises the identity to that of a devotee. This alone brings forth a good life, of wealth and family and dispels all travails faced by a person on earth.

 \item Thondaradi-podi Āḷvār\index{Thondaradi-podi Alvar@\textit{Thondaradi-podi Āḷvār}}, though being a Brahmin, falls in grace due to circumstances in his life. He showcases through his life that a so-called degenerate may be ostracized by society but not in the eyes of God. Any service rendered by a \textit{thondan }(servant of lord) is accepted with the same equanimity by the Lord. He further says that being servant of a thondan is more pleasing to the Lord than to Himself. And so he calls for focus on \textit{Bhakti} than on worldly divisions and distractions.

 \item Tirumaḷisai Āḷvār\index{Tirumalisai Alvar@\textit{Tirumaḷisai Āḷvār}}, in the nature of a philosopher-researcher learnt deeply about all the contemporary religions existing then: Jainism, Buddhism and Saivism. He chose to follow Saivism till he was won over by Periyāḷvār through his argument proving the supremacy of Viṣṇu in comparison to Brahma and Rudra. He preaches \textit{samadharma}\index{samadharma@\textit{samadharma}} by explaining how the Supreme has manifested in everything existing in the universe –the five \textit{Bhūtham-s} or elements:\index{elements} earth, water, fire, air and ether. The import of the statement being that if one understood that He is ‘what is’, that is, everything, there are no divisions to be seen in the material world we live in.

 \item 
 Nammāḷvār\index{Nammalvar@\textit{Nammāḷvār}} was regarded as a yogi even at the time of his birth. He belonged to a farming community. Nammāḷvār possessed great knowledge of the Veda-s, Upaniṣad-s, Itihāsa-s, Purāņa-s and Āgama-s. He is known to have obtained such knowledge through the grace of God. He was known to have experienced the vision of God and then spontaneously burst out with “hymns of sublime beauty and profundity” (Chari 1997:19). These are known as \textit{Tiruvaimoḷi}\index{Tiruvaimoli@\textit{Tiruvaimoḷi}} or Divine Utterances\index{Divine Utterances}. This work is a well-knit philosophical work presenting the five doctrines of \textit{Viśiṣṭādvaita} philosophy\index{Visistadvaita philosophy@\textit{Viśiṣṭādvaita philosophy}}.

\begin{itemize}
\item Brahman to be attained (\textit{prāpya})

 \item The \textit{jīva} or aspirant (\textit{prāpta})

 \item The \textit{sādhana}\index{sadhana@\textit{sādhana}} or means of attainment (\textit{prāptyupāya})

 \item The goal to be attained (\textit{phala})

 \item The obstacles in the way of attainment (\textit{prāpti-virodhi})
\end{itemize}

\end{enumerate}

The \textit{Tiruvaimoḷi}\index{Tiruvaimoli@\textit{Tiruvaimoḷi}} is known as Veda-\textit{sāram} or as that containing essence of the Veda-s.Nammāḷvār is also credited with rendering the Veda-s in Tamil (\textit{vedam tamiḷśeyda meyyan}). (Chari\index{Chari, S.M.S.} 1997:22)

In \textit{Kesavan Thamar} (\textit{Tiruvaimoḷi} 2-7-2), the āḷvār calls out with conviction that ‘Narayana’ is the master of all the worlds as extolled by the Veda-s. Āḷvār reveals that He is the cause, effect and the act of all in the universe.

Through \textit{Uyarvara Uyar Nalam} (\textit{Tiruvaimoḷi} 1-1-1), the āḷvār shares gems of knowledge such as the nature of the Supreme, how he is the manifest and unmanifested and how each human being can offer worship in their own way and that God accepts them all on even keel.

In \textit{Uṇum Soru} (6-7-1) Āḷvār calls out in longing for Lord Krishna by “My food, My drink and the betel I chew”. He sees him in everything around him – his mynah, his parrots, his ball, toys and flower boxes. This is an example of the way āḷvār\textit{-s} ‘experienced’ the bliss of reunion with God.

In \textit{Payilum Sudaroli Murthyyai} (3-7-9), the āḷvār says with conviction that a \textit{Chandala}’s (belonging to a low caste) servant’s servant would be his master, if the former were to be a devotee of Nārāyana.

In \textit{Ondrum Devum} (4–10), he shares the truth about our existence. He says of human beings, “Running tirelessly, taking numerous births, worshipping lesser gods, you have tried so many paths to truth.”

Finally in \textit{Suzh-Visumbu} (10-9-9), Āḷvār describes the destined journey to \textit{Vaikuņṭha} that a liberated soul is gifted by the Lord. Not only does he describe each step of the journey but he also talks about the beings who meet him at different \textit{lokā}-s (universes). He assures the aspirant on earth that reaching the world beyond (\textit{Vaikuņṭha}) is every being’s birth right.

There is a pattern in the songs of all the āḷvār-s. All of them:

\begin{itemize}
\item Talk about the real nature of the Supreme \textit{Brahman }and the Universe, the limitless auspicious qualities of the \textit{Brahman}.

 \item Reveal the true identity of the human beings on earth as \textit{ātman}-s.

 \item Advise about what one’s focus in life should be (\textit{bhakti} to the Lord) and one’s goal / aim in life (attain salvation)

 \item Extol the virtue of being of service to the Lord and more importantly his devotees.

 \item They assure one of God’s limitless \textit{karuņai }(kindness) in helping one cross the \textit{saṃsāra} to the world beyond, which is beyond the cycle of birth and death.

\end{itemize}

It is important to note that the works of Āḷvār\index{Alvar@\textit{Āḷvār}} were primarily concerned with \textit{Brahman} and how to reach him. They do not concern themselves with or delve into societal workings of caste divide or other reforms.

The miracles or reformist actions (such as Tirupānāḷvār entering the shrine) are all executed in the firm conviction of the relationship of \textit{Sesha-Seshi} (the Lord and the servant / dependent). Thus, belief in God and whom the Veda-s point to as the Supreme (Viṣṇu) was a prior condition to preaching \textit{samadharma}\index{samadharma@\textit{samadharma}}. The Āḷvār-s extolled the virtues of the Supreme and the destiny of the atman to attain the creator beyond the material realm. Through such conviction, the āḷvār-s seek to encourage the \textit{jīva}-s in the earthly realm to look beyond the world of differences to the world of oneness.

This is in sharp contrast to what DMK and other proponents of class equality propose. By focusing on caste inequality, one emphasizes the differences. By seeking to look at oneself as \textit{ātman}, an eternal entity that is dependent on the Supreme, the differences vanish. The \textit{Sri Vaiṣņava} sect has been nurtured in such inclusiveness since time immemorial. To blame the class hypocrisy and oppression experienced by the fourth varņa on everything to do with Brahmins including the Vedāntic philosophy\index{Vedantic philosophy@\textit{Vedāntic philosophy}} would therefore be incorrect.

In Bhagavad Gītā\index{Bhagavad Gita@\textit{Bhagavad Gītā}} (Chapter 4; verse 13), Lord Krishna remarks that he created the four orders of society (\textit{Brāhmaņa}-s, \textit{Kṣatriya}-s, \textit{Vaiśya}-s and \textit{Śūdra}-s) according to the \textit{Guņas}\index{Gunas@\textit{Guņas}} (nature) of each of them and apportioning corresponding duties accordingly. He does not mention that \textit{Varņāśrama}\index{Varnasrama@\textit{Varņāśrama}} follows by birth. It is to be deduced that it follows the actions one undertakes.

Tirupānāḷvār respects the custom in his time and does not focus on changing that, however, oppressive it may have been. Instead he is carried by a \textit{brāhmaņa}, thus emphasizing that a devotee rises above material-world imposed divisions. This is the most effective way to emancipate the masses than focusing on the divides and keeping the people fixed in such identity.


\subsection*{Rāmānuja: Custodian of Vedas}

\textit{Āḷvār}-s reiterated the true purpose of one’s life on earth and how to attain the goal. But in the process, they ended up experiencing the bliss so often that the masses could not access their outpourings in a practical and easier way. \textit{Ācārya}-s who came onto earth much later, synthesized the Sanskrit texts on Vedas and \textit{Āḷvār-s’ }\textit{Divya Prabandham}\index{Divya Prabandham@\textit{Divya Prabandham}} to deliver their purport to the masses.

In this world of duality, where contradictions abound, Rāmānuja \textit{ācārya} was unique and effective in that, he was able to resolve the duality to come up with the essence of Veda-s. He synthesized Sankara’s view of \textit{Abheda Śruti}\index{Abheda Sruti@\textit{Abheda Śruti}} and Madhva’s view of \textit{Bheda Śruti}\index{Bheda Sruti@\textit{Bheda Śruti}} to bring understanding in the form of \textit{Ghaṭaka Śruti}\index{Ghataka Sruti@\textit{Ghaṭaka Śruti}}. \textit{Āḷvār}-s had made Veda-s accessible by rendering them in Tamil. Rāmānuja emphasized the importance of understanding both Sanskrit Veda literature and Tamil \textit{Divya Prabandham}\index{Divya Prabandham@\textit{Divya Prabandham}} to reach the purport of Veda-s. One of his primary goals in life was to synthesize \textit{Bodhāyana Vṛtti}\index{Bodhayana Vrtti@\textit{Bodhāyana Vṛtti}} which was a commentary on \textit{Brahma Sutra}-s (on Brahmakāņḍam) written by Veda Vyāsa\index{Veda Vyasa@\textit{Veda Vyāsa}}. As someone who was adept at resolving seeming contradictions, he utilized both his \textit{Sri Bhāṣyam}\index{Sri Bhasyam@\textit{Sri Bhāṣyam}} (commentary on \textit{Bodhāyana Vṛtti}) and \textit{Āḷvār}-s’\textit{Divya Prabandham} to instruct his followers on the essence of \textit{Viśiṣṭādvaita} philosophy (Swamy 1988: 4–5).

\vskip -20pt


\subsection*{Importance of Sanskrit And Tamil}

\vskip -6pt

M Karunanidhi’s\index{Karunanidhi, M.@Karunānidhi, M.} endorsing Rāmānuja as a role model for his view of Tamilnadu, where Tamil and not Sanskrit is welcomed, is inconsistent with the saint’s view of \textit{Bhārata} as one culture, one people. He viewed Sanskrit and Tamil equally important for their role in accessing \textit{jñana} about the ultimate truth. He was not ready to sacrifice one for the other.

Like his predecessor Yamunācārya\index{Yamunacarya@\textit{Yamunācārya}}, Rāmānuja too endorsed the ‘\textit{pancarātra āgama}’\index{pancaratra agama@\textit{pancarātra āgama}} system which was not considered authentic by some Vedic scholars. Rāmānuja firmly believed that the pancarātra system was an approved Vedic system. It incidentally also approves of admission of \textit{śūdra}-s\index{sudra@\textit{śūdra}} and women into the sect. Rāmānuja’s inclusiveness stemmed from the belief that all devotees are equal in their quest for knowledge and Bhakti towards the Supreme. (Sawai\index{Sawai, Yoshitsugu} 1993: 62).

Rāmānuja was cognizant of the fact that the people of \textit{Bhārata} should know the truth of Veda-s as espoused by the \textit{Viśiṣṭādvaita} philosophy\index{Visistadvaita philosophy@\textit{Viśiṣṭādvaita philosophy}}. Therefore, all of his writings were in Sanskrit, with no reference to pancarātra āgamas and \textit{Āḷvār}-s’ Tamil Veda to show that the roots were Vedic rather than tāntric. He wanted to convince Brahmins in all parts of India that his interpretation of the Sanskrit Veda-s was correct, and to do so, he quoted in support only those scriptures that had common acceptance among Brahmins.

At the same time, Rāmānuja commented on Nammāḷvār’s\index{Nammalvar@\textit{Nammāḷvār}} verses in conversations with his disciples and commissioned one of his disciples, Thirukurugai Piran Pillai\index{Thirukurugai Piran Pillai} to produce the first written commentary. He also arranged for the Tamil Veda to be recited regularly at the temples of Vishnu across India. This popular practice continues till this day. Rāmānuja considered \textit{Āḷvār}-s’ Tamil Veda as nectar meant to be relished in\textit{ Bhakti} while institutionalising the \textit{viśiṣṭādvaita} philosophy in Sanskrit for propagating jñāna about the Supreme among the intellectuals. As S Radhakrishnan\index{Radhakrishnan, S.} said, “Rāmānuja\index{Ramanuja@\textit{Rāmānuja}} tried his best to reconcile the demands of religious feelings with the claims of logical thinking”.

This notion of dual Vedānta highlights how much importance Rāmānuja lent to the fusing of the two languages and two cultural traditions\index{cultural traditions} in his theology. There were always two planes on which he was devoting his works – one was for the intellectuals or the educated elite and another purely devotional, for the masses who could also attain salvation without concerning themselves with the intricacies of abstract philosophy (Seshadri\index{Seshadri, Kandadai} 1996:295–297).


\subsection*{Adherence to Rituals}

The Veda-s informs the spiritual aspirant as to the nature of reality, of God, of the means of attaining him. There are two branches in Veda-s: \textit{Brahmakāņḍa}\index{Brahmakanda@\textit{Brahmakāņḍa}} and \textit{Karmakāņḍa}\index{Karmakanda@\textit{Karmakāņḍa}}. The \textit{Karmakāņḍa} informs the seeker about the way life ought to be lived on the material realm – the duties, rituals and their outcomes. \textit{Brahmakāņḍa} talks about the nature of Supreme Lord who is the source of everything. This branch of knowledge is also called as Upaniṣad-s. Lord Krishna gives the essence of the Upaniṣad-s\index{Upanisad@\textit{Upaniṣad}} to seeker Arjuna in Bhagavad Gītā\index{Bhagavad Gita@\textit{Bhagavad Gītā}}.

\newpage

Rāmānuja emphasized the study of ritual that is part of ‘\textit{Karma Kāņḍa’}. After that and only after that, the study of \textit{Brahman} should follow unlike in the case of Śankara\index{Sankara@\textit{Śankara}} for whom \textit{jñāna }and not \textit{Karma Kāņḍa} is necessary for salvation. In Rāmānuja’s view, \textit{Karma} and \textit{bhakti} are closely interlinked, hence the rituals as prescribed have to be performed as per one’s caste and position. Rāmānuja travelled all the way to Kashmir to obtain an old text called \textit{Bodhāyana Vṛtti}\index{Bodhayana Vrtti@\textit{Bodhāyana Vṛtti}}. It is from this document that Rāmānuja derives his argument for the importance of ritual. Rāmānuja meticulously prepared the manuals of worship in the important temples, detailing the rituals for every ceremony (Seshadri\index{Seshadri, Kandadai} 1996: 295).


\subsection*{Rāmānuja’s stance on \textit{Varņāśrama}}

Dravidian political philosophy is known to assert that Brahmins deserve criticism and condemnation for their perceived role in perpetuating caste discrimination through their belief and practice of \textit{Varņāśrama}\index{Varnasrama@\textit{Varņāśrama}}. Rāmānuja, who followed the \textit{Guru Paramparā}\index{Guru Parampara@\textit{Guru Paramparā}} of the Gītā Ācārya or Krishna, was a firm believer in the authenticity of Varņāśrama Dharma and its necessity in society.

\textit{Sri Rāmānuja Vaibhavam} is one of the texts of ‘\textit{Guru Paramparai}’ written by Vadivaḷagu Nambi Dasar\index{Vadivalagu Nambi Dasar@\textit{Vadivaḷagu Nambi Dasar}} 300 to 400 years ago. R. Kannan Swamy, the commentator of this text, explains that Rāmānuja and his followers discuss the real import of “Tattvam Asi” on an occasion. The followers share their understanding of Rāmānuja’s teaching – that he broke the misconception of understanding this statement as \textit{Jīva} and \textit{Brahman} being different as believed in some philosophies. He did so by emphasizing how the \textit{Varņāśrama Dharma} helps the \textit{jīva }in reuniting with the Brahman. The duties ascribed by the \textit{Varņa} (which are decided by the \textit{karma }and qualities of the \textit{jīva} taking birth), if properly rendered in the proper manner as detailed in the Veda-s, then the \textit{jīva} nullifies the \textit{karma} which keeps him away from the \textit{Brahman}. Such a \textit{Jīva} which is able to discard its \textit{karma}, merges with the \textit{Brahman} (Swamy 1988: 360–361).

“Rāmānuja holds that duties related to various \textit{dharma}-s and \textit{āśrama}-s should be discharged even by a wise man because \textit{karma} contributes to \textit{vidyā} which means meditation on the selfless devotion to God.” Rāmānuja’s recognition of \textit{varņāśrama} duties illustrates that he recognised social life as an organic whole in which member is tied with another in an indissoluble tie. Though an individual is predominantly responsible for his own good or bad actions, he shares the collective responsibility for the good of others and of society as well. It is always due to our actions that we enjoy and suffer. But we also suffer or enjoy in consequence of the actions of others. The sense of collective responsibility has been emphasized by the law of karma which granted equality to all beings, as caste was determined by karma. (Singh 2001: 496–497)

In an incident recorded about the saint’s life, Periya Nambi (Rāmānuja’s first \textit{Ācārya}) performed the \textit{brahma medha} funeral rite, traditionally reserved only for Brahmins, for a fellow devotee who did not belong to the Brahmin class, Māraneri Nambi\index{Maraneri Nambi@\textit{Māraneri Nambi}}. The people of the town were aghast at the gross violation of the \textit{Varņāśrama Dharma} and complained to Rāmānuja about this breach of principles. When Rāmānuja challenged Periya Nambi\index{Periya Nambi} to explain how he could violate such a tradition in such a gross manner, Periya Nambi gave a convincing and moving reply about the greatness of \textit{bhāgavatas}\index{bhagavatas@\textit{bhāgavatas}} (devotees) irrespective of caste to which they belonged. He explained that Māraneri Nambi was a true Sri Vaiṣņava by this belief and actions and therefore his service to such a \textit{bhāgavata} was to fulfil his last wishes of being cremated by a Sri Vaiṣņava. This explanation satisfied Rāmānuja. This incident was set as an example by Rāmānuja to show that he respected\textit{ Varņāśrama} sincerely enough to question a breach even when it was done by his own \textit{Ācārya}. But he was equally open-minded and compassionate to see the context in which the violation took place. When it concerned Bhāgavatas (God’s own people, His devotees), Rāmānuja showed by example, that compassion rules over traditional norms. The key point to understand here, is his emphasis on the pre-condition of relationship with God for any such actions to be undertaken (Rāmānuja.org 1996).

M. Karunanidhi, in his depiction of Rāmānuja in the tele serial, misses this key point. If one were to understand this point, we would also understand that Rāmānuja’s concern went higher than social emancipation. It concerned itself with cosmic wellbeing. The way towards that, as Rāmānuja proved, was through exactness and discipline in adhering to performing one’s \textit{karma}, including the rituals and traditions prescribed by the Veda-s.


\subsection*{Preserving the Narrative}

\begin{myquote}
“Hindu orthodoxy\index{Hindu orthodoxy} would not contemplate any acceptance of alien thought and whatever was deemed to be worthy of discussion was all indigenous thought: whether it be Buddhism, Jainism or atheism.”(Seshadri\index{Seshadri, Kandadai} 1996:292)
\end{myquote}

Rāmānuja was known to have given lot of importance to understanding the author’s opinion and use the knowledge gained as-is to retain authenticity of the original text on which he wrote commentaries. He never preached his own interpretations or explanations and always quoted only those from any of the \textit{Purvācārya Grantham}-s. \textit{Guru Paramparā}\index{Guru Parampara@\textit{Guru Paramparā}} (one of the texts of \textit{Sri Vaiṣņava} traditions\index{Sri Vaisnava traditions@\textit{Sri Vaiṣņava traditions}}) has a record of an incident in Rāmānuja’s life that is relevant to this context. Rāmānuja was being instructed on \textit{Tiruvaimoḷi }by one of his ācārya-s, Thirumālai Āndān. In the course of events, Rāmānuja gives a different interpretation to one of the verses of the \textit{Āḷvār}, after hearing the one given by Āndān. The Āndān objects to changing the traditional explanation given by the previous Guru, Yāmunācārya\index{Yamunacarya@\textit{Yāmunācārya}}. He says that his Guru had never interpreted that verse in the way Rāmānuja explained. To that, Rāmānuja exclaims that he would never dream of changing the traditions laid down by the Gurus of \textit{Viśiṣṭādvaita} philosophy\index{Visistadvaita philosophy@\textit{Viśiṣṭādvaita philosophy}}. He affirms that he is only reflecting what Yāmunācārya himself would have approved of, as he submits that he was to Yāmunācārya what Ekalavya was to Drona. This incident shows that Rāmānuja was cognizant of the tradition and was not of a rebellious mindset. His focus was on re-establishing the ancient tradition of following Veda-s in understanding oneself, God and the means of attaining Him (Swamy 1988:263–270).

This truth is misrepresented when the DMK Chief portrays Rāmānuja preaching \textit{Āḷvār}-s’ \textit{Prabandham} to the people of the fourth\textit{ varņa} with the objective of bringing them into \textit{Sri Vaiṣņava} fold. He is also shown complaining to his mother that his sacred thread (worn across the trunk of the body) was hindering his ability to interact and play with other kids in the street and therefore asking if he could remove the same. Rāmānuja followed strict adherence to the traditions laid down in Guru Parampara. He was only focused on sharing the knowledge with those who \textit{chose} to be devotees of Nārāyaṇa, not on converting them to his philosophy. He was against any rebellion (such as removing the sacred thread\index{sacred thread}) as depicted in the tele serial.

\newpage


\subsection*{Cosmic Focus or Social Concern}

The source of the epithet ‘social reformer’ for Rāmānuja mainly stems from the gopuram incident at Thirukośitiyur. So much so, that the working President of DMK, Mr. M K Stalin\index{Stalin, M.K.}, a staunch atheist, visited the shrine of Rāmānuja in the gopuram, from where he is known to have shared the most important \textit{rahasyam} of \textit{Sri Vaiṣņava tradition. (Ramanathan 2015)}.

In the early days after assuming the sannyāsī order, Rāmānuja was tested rigorously by his \textit{Ācārya} before imparting the \textit{Thirumantra rahasyam}\index{Thirumantra rahasyam@\textit{Thirumantra rahasyam}} to him. It is known that Rāmānuja visited Thirukoshitiyur (where his \textit{Ācārya} was residing) from Srirangam\index{Srirangam} 18 times before the latter relented and imparted the greatest gift in Sri Vaiṣņava tradition – \textit{Thirumantra rahasyam}. This \textit{mantra} is key to understanding the way to attain salvation. Keeping in with the custom then, the \textit{Ācārya} cautioned Rāmānuja against imparting the gem of a \textit{mantra} to all and sundry without testing the readiness, mental fitness, preparedness and eagerness of the disciple. Rāmānuja is known to have gone to the temple gopuram (which could mean a large hall or open hall) and preached the \textit{rahasyas}\index{rahasyas@\textit{rahasyas}} and \textit{rahasyārthas} (meaning of the \textit{rahasya}) to \textit{Sri Vaiṣņavas }present. \textit{Arāyirappadi Guru Paramparā Vaibhavam} by Pinbaḷagiya Perumāḷ Jeeyar\index{Pinbalagiya Perumal Jeeyar@\textit{Pinbaḷagiya Perumāḷ Jeeyar}}, published by Puththoor Swamy, (The oldest account of \textit{Guru Parampara},1968 print) says

\begin{myquote}
“... \textit{therkāzvār thiruvolakkaththilanegam srivaiśnavarkaḹukku ap}\\\textit{parama rahasya arththatthai arulinār}”(Rāmānuja.org 1996).
\end{myquote}

It is important to note here that Rāmānuja shared the \textit{rahasya} with \textit{Sri Vaiṣņavas}, contrary to popular perception that he shared the secret with all and sundry. He was cognizant of the tradition of sharing it with only those with belief in Nārāyana and who wanted to reach his feet and thus attain salvation. However, due to his compassion and practical nature, he saw that restricting the \textit{mantra}\index{mantra@\textit{mantra}} to the few who could adhere to strict rules prescribed, would keep the masses of devotees of Nārāyana from reaching Him, hence, his decision to share the \textit{mantra }with the devotees at large.

Not understanding this nuance would lead one to view Rāmānuja as more concerned with breaking caste divisions than the cosmic nature of his quest.

\newpage

“Social and economic factors did not worry them since their concerns were with fundamental and universal truths. Questions of a social nature were too trivial for the philosophers, as they appeared too transitory in the context of cosmic questions, to be dealt with seriously.” (Seshadri 1996: 292)

Rāmānuja’s compassion towards the downtrodden was in the context of facilitating every \textit{jīva} to attain their birth right, which was reaching \textit{Vaikuņṭha} as asserted by Nammāḷvār\index{Nammalvar@\textit{Nammāḷvār}}. It did not matter to him who comprised Nārāyana’s devotees (people across castes). It did matter to him that those who believed in Narayana alone (as per \textit{viśiṣṭādvaita} philosophy) and who are ensnared in the infatuation induced by the duality of this material realm should be enabled to reach Him. Being practical in his thinking and action, Rāmānuja saw that in \textit{Kali Yuga}\index{Kali Yuga@\textit{Kali Yuga}}, to demand exaction from masses, would not yield results. This would mean keeping many away from the sole purpose of taking birth, which, according to the \textit{Viśiṣṭādvaita} philosophy, was to break free of the cycle of birth and death and reach \textit{Vaikuņṭha }which is beyond the laws of \textit{Karma}.


\subsection*{Risks involved in viewing Rāmānuja as a Social\hfill \break Reformer\index{Social Reformer}}

Rāmānuja is a saint who commands respect and devotion even to this day because of his unique ability to marry the emotion of Bhakti with logical thinking. He doused this combination with liberal doses of inclusiveness. Being the influential leader that he was, he was successful in building this mindset in his disciples. The 74 \textit{maṭha ācārya-s} designated by him, continue to maintain the hierarchy of relationships to disciples, spread across the world to this day.

\vskip 4pt

Aravindan Neelakandan\index{Neelakandan, Aravindan} in \textit{Swarajya} talks about a trend of observing some Sri Vaiṣņavas leaning left in their quest for being viewed as secular and logical. One of the reasons could possibly be the inclusive mindset nurtured in the \textit{viśiṣṭādvaita} tradition. Such Marxist-Vaiṣņava trade their own legacy of socially relevant spirituality in favour of realizing socialist utopia (Neelakandan 2017).

\vskip 4pt

The epithet of social reformer employed by DMK and some Marxists\index{Marxists} to Rāmānuja, may drive many more to the other end of theism without understanding the secular nature and inclusiveness prevalent in their own ancient tradition. Rāmānuja’s popularity among masses may be conveniently deflected towards strengthening self-serving political ends by confining the focus to the social impact of his teachings and work. Therefore, there is a need to represent Rāmānuja, his life and teachings in the right perspective of his contribution in reviving Vedanta and his role in the \textit{Bhakti} movement. To talk about his social impact without understanding where they flowed from – \textit{viśiṣṭādvaita} philosophy\index{Visistadvaita philosophy@\textit{Viśiṣṭādvaita philosophy}}, would be doing a disservice to the saint.


\section*{Conclusion}

We have looked at the nature of the Dravidian political movement that nurtured and shaped DMK’s philosophy. The seed of the Dravidian rule was sown when the Non-Brahmin communities organized itself and achieved political power for the first time in history. The non-Brahmin communities who felt sidelined during the Imperial Rule found expression through the self-respect movement. The trading and farming communities were influential and shaped opinion that the cause of suffering since the beginning was the exclusivity enjoyed by the Brahmin class. The \textit{Dravida Kazhagam}\index{Dravida Kazhagam@\textit{Dravida Kazhagam}} did not have any issue accepting the Imperial Rule\index{Imperial Rule} and rued the day India got Independence from foreign rule. They were rooting for a separate identity as a Dravidian state and clamoured for division from the Indian subcontinent at the time of Independence.

The real reason for the polarisation between the two communities (Brahmin and Non-Brahmin) in Tamilnadu is the belief in Aryan-Dravidian divide. Eugene Irschick\index{ Irschik, Eugene F.} explains how this theory developed - “In the nineteenth century, a number of European and Indian scholars who had begun the study the origins of Tamil, posited the idea that Non-Brahmins were Dravidians and the original civilizers of the region and that the Brahmins were the Aryan invaders from the north. These scholars believed that the Dravidians had been conquered and their institutions supplanted by an imposed Sanskritic Aryan religion and a caste system, by which non-Brahmins had for centuries been kept in an inferior position. Linguistically also, there was a strong tradition for a division between Brahmins and non-Brahmins, especially in the Tamil area. The Brahmins were regarded as the guardians of the northern Sanskrit; the non-Brahmins, or so they themselves believed were the creators of southern Tamil and Tamil culture.” (Irschick 1969:12).

We have seen that the DMK chief M Karunanidhi inherited and successfully propagated a legacy during 50 years of rule that was atheistic in outlook, pro-Tamil, anti-Brahmin and institutionalized the Aryan-Dravidian theory\index{Aryan-Dravidian theory} as a fact to the masses. Rāmānuja, a Brahmin, was a product of the so-called Aryan civilization in Tamilnadu who staunchly held a belief in \textit{Varnashrama} and was a key proponent of Bhakti movement\index{Bhakti movement} in India. He was the antithesis of everything DMK professed to stand by. M Karunanidhi, in recent times, acknowledged his reverence for this saint and said that he believed in Rāmānuja’s ideas since they were beyond caste considerations.

Rāmānuja amalgamated seeming contradictions during his times to influence the masses towards the Bhakti movement. According to him, there is no distinction between high and low, touchable and untouchable in the need of the quest for divine grace. He was the first ever to introduce the free meal scheme to all people at the closing time of the Vishnu temples in South India, irrespective of caste or creed. This shining role-model of inclusiveness\index{inclusiveness} prompted Vivekananda\index{Vivekananda, Swami} to declare that India herself should have the heart of Rāmānuja. Such was the grace of the upholder of mankind that Rāmānuja commands an enthusiastic following till this date, 1000 years since his birth.

One can see that this inclusiveness which Rāmānuja exemplified could have prompted DMK to admire him for looking beyond differences in people and communities. However, it is important to understand the motivation and nature of Dravidian movement that the latter stands for, when analysing the title of social reformer awarded to Rāmānuja. M Karunanidhi viewed the saint through the lens of his own legacy and philosophy. As observed in the beginning of this paper, the Dravidian political party has deep rooted belief in atheism and anti-Brahmin Dravidian divide. This makes the DMK chief’s admiration for Rāmānuja as a crusader against caste inequality\index{caste inequality} untenable.

Rāmānuja stood for everything that M Karunanidhi and his predecessors have sought to criticise. As seen in the above sections, \textit{Āḷvār}-s and \textit{Ācārya}-s of \textit{Sri Vaiṣņava} traditions\index{Sri Vaisnava traditions@\textit{Sri Vaiṣņava traditions}} contributed immensely to Tamil literature. At the same time, Rāmānuja and his predecessors have maintained a strong tradition of honouring Sanskrit texts and used it often in their debates and commentaries. They saw all people and languages as part of the Supreme and did not view them as Aryans or Dravidians. Rāmānuja and his \textit{Ācārya}-s have reiterated their belief in the role that \textit{Varnashrama} plays in aiding one in carrying out their karma. The Veda-s that DMK does not believe in, was the mainstay of Rāmānuja’s victory over other contemporary religions in his time. In fact, Rāmānuja’s primary focus was in re-establishing the glory and importance of Veda-s and rituals in the society. While he was inclusive in outlook, he was not a rebel and contained his magnanimity within the confines of the structures laid out in the Veda-s and espoused by his elders. Yes, he did look beyond caste, linguistic and gender differences, as long as the people in question, were devotees of Narayana. To him, the context of acknowledging Nārāyana as the Supreme lord and Bhakti towards Him, was the basis for any activity in the outer world. Without this context, the identity of Rāmānuja would be difficult to construct.

The impact of Rāmānuja’s philosophy on the society was a by-product of his all-pervading aim of establishing the identity of \textit{ātman} and guiding it towards the \textit{Brahman} through \textit{prapatti }or surrender. Such a cosmic level concern would be dwarfed if one were to confine the saint to the identity of a social reformer. In the current reality of fragmented nations and individuals, there is an urgent need to look inwards, to differentiate the nature of reality and illusion to evolve as better beings and nurture the planet. Rāmānuja’s teachings on \textit{Viśiṣṭādvaita} philosophy are as relevant today as they were a thousand years ago. He came as a light onto darkness when Veda-s were facing attack from false interpretations and new religions. In current times, he is once again instrumental in guiding masses towards realizing their identity and transforming violence in thought and action into peace through his inclusive philosophy.


\section*{Bibliography}

\begin{thebibliography}{99}
\bibitem{chap9-key01} Chari, S.M.S. (1997). \textit{Philosophy and Theistic Mysticism of the Āḷvārs.} Delhi. Motilal Banarsidass.

 \bibitem{chap9-key02} Gangatharan, A. (2009). “\textit{Reading of the Ramayana in the modern Tamil context”.} \textit{Journal of Tamil Studies}, June 2009. Pp 82.

 \bibitem{chap9-key03} Govindacharya, Alkondaville (1906). \textit{The Life of Rāmānujachary. }Madras\textit{.} S. Murthy \& Co.

 \bibitem{chap9-key04} Irschick, Eugene F. (1969). P \textit{olitics and social conflict in South India: The non-Brahman Movement and Tamil Separatism. }Bombay\textit{. }Oxford University Press.

 \bibitem{chap9-key05} Muthukumar, R. (2010). \textit{Dravida Iyakka Varalu – Part 1.} Chennai. Kizhakku Pathippagam.

 \bibitem{chap9-key06} Muthukumar, R. (2010) \textit{Dravida Iyakka Varalu – Part 2. }Chennai. Kizhakku Pathippagam.

 \bibitem{chap9-key07} Neelakandan, Aravindan. (Last modified on 1 May, 2017). “\textit{Thousand Years On, India Is Still Discovering The Man That Was Sri Rāmānuja}”. \url{https://swarajyamag.com/ideas/thousand-years-on-india-is-still-discovering-the-man-that-was-sri-Rāmānuja}. Accessed on 18 July, 2017.

 \bibitem{chap9-key08} Ramanathan, S. (Last modified on 15 October, 2015). “A historic debate is afoot in the Dravidian movement: Is DMK trying to project a ‘pro-Hindu’ image?”. \url{http://www.thenewsminute.com/article/historic-debate-afoot-dravidian-movement-dmk-trying-project-%E2%80%98pro-hindu%E2%80%99-image-35172}. Accessed on 20 July, 2017.

 \bibitem{chap9-key09} Sawai, Yoshitsugu (1993) “\textit{Rāmānuja theory of Karman}”. Journal of Indian Philosophy. 21, Pp 62.

 \bibitem{chap9-key10} Seshadri, Kandadai (1996) \textit{“Rāmānuja: Social Influence of his Life and Teaching”. Economic and Political Weekly}. Vol. 31, No. 5 (Feb. 3, 1996), Pp. 298.

 \bibitem{chap9-key11} Singh, Abha (2001) “\textit{Social Philosophy of Rāmānuja: Its Modern Relevance}”. Indian Philosophical Quarterly XXVIII No. 4. Pp. 496–497.

 \bibitem{chap9-key12} Swamy, Kannan R. (1988) \textit{Sri Rāmānuja Vaibhavam}. Madras. Amman Achakam.

 \bibitem{chap9-key13} Viswanath, Rohit. (Last modified on 14 April, 2015). “From Rāmānuja to Periyar: Re-integrating Brahmins into Dravidian narrative”. \url{http://www.firstpost.com/india/Rāmānuja-periyar-re-integrating-brahmins-dravidian-narrative-2194535.html}. Accessed on 14 August 2017.

 \bibitem{chap9-key14} Yamunan, Sruthisagar. (Last modified on 2 April, 2016). “Rāmānuja broke caste barriers: Karunanidhi”. \url{http://www.thehindu.com/news/national/tamil-nadu/vaishnavite-philosophersaint-Rāmānuja-broke-caste-barriers-says-karunanidhi/article7068531.ece}. Accessed on 20 July, 2017.

 \bibitem{chap9-key15} “The “real” Rāmānuja (excerpts from Sri Bhasya)”. Last modified on 12 March, 1996. \url{http://www.Rāmānuja.org/sv/bhakti/archives/mar96/0077.html}. Accessed on 16 July, 2017.

 \end{thebibliography}

