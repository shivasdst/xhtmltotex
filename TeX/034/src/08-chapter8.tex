
\chapter{Integration of Tamil Isai with the Vedic and Sanskritic Tradition}

\Authorline{K. Vrinda Acharya}


\section*{Abstract}

Tamil Nadu is undeniably an integral part of India not just geographically or politically, but what it has contributed to the totality of Sanātana Dharma is immense and phenomenal. The cultural, spiritual and social identity of Tamil Nadu is essentially pan-Indian and it occupies a very important place in the overall cultural fabric of India. Unfortunately, the Colonial, Dravidian, Separatist and Secular narratives have had serious adverse effects on the ethos of Tamil Nadu and on its naturally beautiful integration with pan-Hindu culture.

This paper provides evidence that Tamil Isai,\index{Tamil Isai} though one of the most ancient, rich and highly evolved centuries ago, has always been an integral part of pan-Indian or pan-Hindu culture and that its harmony with Vedic and Sanskritic tradition is very evident in all its varied dimensions. Though it is unquestionably one of the vital components (and not the only one as claimed) of South Indian Music (or Carnatic Music as it is popularly known), its constant synergy with the musical traditions of other states/languages in shaping the same into a system of music as we know it today, can in no way be undermined or altogether denied.

\newpage

This paper first tries to understand what Tamil Isai is and then examines the various facets of its integration with the Vedic and Sanskritic tradition, viz., origin of music, theory of music, musical literature, melody and rhythm, musical instruments and finally its collaboration with musical cultures of the other states. Sage Agastya,\index{Agastya} who is considered by the Tamils as the founding father of Tamil language and all elements of its culture, including music and dance, was sent south by none other than Lord Śiva. With respect to musical notes and scales, the concepts are all the same, although the ancient Tamils followed a regional nomenclature.With respect to ‘\textit{Sangītaśāstra}’\index{Sangitasastra@\textit{Sangītaśāstra}} or theory of music, a lot of parity is evident. There are umpteen numbers of illustrations in the entire gamut of Tamil musical literature which are enough to prove its essentially Indic nature. All this leads us to the conclusive congruence between the culture of Tamil Nadu and that of the rest of India.


\section*{Introduction}

Tamil Nadu is undeniably an integral part of India not just geographically or politically, but what it has contributed to the totality of Sanātana Dharma is immense and phenomenal. It has been a producer and repository of all facets of knowledge, be it science, literature, arts, culture or philosophy. Also, history has proven time and again that the cultural, spiritual and social identity of Tamil Nadu is essentially pan-Indian and that it occupies a very important place in the overall cultural fabric of India.

\begin{myquote}
“From the very beginning of historical times, Tamilnadu was the land of Vedic traditions in every field of life. The Vedic concepts, gods, customs and manners ruled Tamil life: be it the kings, merchants, cultivators, Brahmins, hunters, hill tribes, fishermen, chiefs and soldiers, cowherds, artists, musicians, dancers. They followed the Vedic ideology, worshipped Vedic Gods, and propitiated their ancestors, are shown abundantly in the data available..... In the field of music and dance, including the ‘pan’ system of music they followed the northern convention. They produced a masterly \textit{Nāṭaka kāvya} the Silappadikāram\index{Silappadikaram@\textit{Silappadikāram}} based on Bharata’s \textit{Nāṭya sāstra}....”\hfill (Nagaswamy\index{Nagaswamy, R} 2017:1,2)
\end{myquote}

\vskip 5pt

\begin{myquote}
“The Vedas have been the perennial spring of Indian and the whole of South East Asian civilization, for the past 3500 years in all most all fields of human culture. In the fields of History, Art, Architecture, Music, Dance, Administration, Judiciary, Law, Social life and so on.”

~\hfill (\url{http://tamilartsacademy.com/})
\end{myquote}

Unfortunately, it is also one of those states in India that has been subjected to maximum attack by various divisive forces, both external and internal, on cultural, social and spiritual fronts. Rajiv Malhotra\index{Rajiv Malhotra} (2011:61) rightly observes that the Colonial administrators and evangelists were able to divide and rule the peoples of the Indian subcontinent, based on imaginary histories and radical myths – to the extent of inventing an entire race called ‘Dravidians’. These Colonial, Dravidianist\index{Dravidianist}, Separatist and Secular narratives have had serious adverse effects on the ethos of Tamil Nadu and on its naturally beautiful integration with pan-Hindu culture.

Hindu art forms and other aspects of the Vedic Traditions are being targeted by missionary scholars for Christian infiltration\index{Christian infiltration} and appropriation. The Music of Tamil Nadu, or Tamil Isai as it is well known, is no exception to this. There have been constant attempts to sever the relationship of Tamil music with Indian music at large, to demean or totally negate its harmony with Sanskrit, to accuse its sacred dimension as being oppressive and also to project it as an entirely secular art form.

Talking of the work of the evangelist Bishop Robert Caldwell\index{Bishop Robert Caldwell}, Malhotra points out that -

\begin{myquote}
“...it established the theological foundation for Dravidian separatism from Hinduism, backed by the Church. It was accompanied by Christian usurpation of many of the classical art-forms of South India. The concept of disassociating Tamils from mainstream Hindu spirituality provided Caldwell an ethical rationale for Christian proselytization\index{Christian proselytization}...”

~\hfill (Malhotra 2011:64)
\end{myquote}

The purpose of this paper is to prove beyond doubt that Tamil Isai, though one of the most ancient, rich and highly evolved centuries ago, was as integral part of pan-Indian or pan-Hindu culture and that its harmony with Vedic and Sanskritic tradition is very evident in all its varied dimensions. It has been primarily sacred, bhakti being the dominant theme. It has its roots in the Vedas just as any other musical tradition of India and it has always been an integral and inseparable component of Indian Classical Music. Though it is unquestionably one of the vital components (and not the only one as claimed) of South Indian Music (or Karnatic Music\index{Karnatic Music} as it is popularly known), its constant synergy with the musical traditions of other states/languages in shaping the same into a system of music as we know it today, can in no way be undermined or altogether denied. However, the paper does not intend to either conclude that the Tamil tradition has always been only a borrower, or to show in any way that it is insignificant or less great than the Sanskrit tradition. Also, it does not aim to arrive at any rigid or strict chronology, or bring about any competitive comparisons between the two. Furthermore, taking sides with either of the two factions involved in a series of conflicting incidents during the Tamil Isai movement is beyond the scope of this paper.

This paper intends to bring to light some very significant historical facts as well as examine and analyse certain traditions and practices which establish the above thesis beyond doubt. The following scheme is followed. First, it tries to understand what Tamil Isai is and then examines the various facets of its integration with the Vedic and Sanskritic tradition, viz., origin of music, theory of music, musical literature, melody and rhythm, musical instruments and finally its collaboration with musical cultures of the other states. Thus, this paper hopes to successfully prove that the music of Tamil Nadu typically revolves around Indic spirituality.


\section*{What is Tamil Isai:\index{Tamil Isai}}

Before we go on any further with the paper, it is important to understand what exactly Tamil Isai is. Prima facie, it can be understood either as music that originated in Tamil Nadu or music that is sung in the language of Tamil. The ancient Tamils possessed a highly developed culture. The \textit{Muttamiḷ} (threefold Tamil) consisted of the divisions \tamil{இயல்}(\textit{Iyal} – literature), \tamil{இசை}(\textit{Isai} – music) and \tamil{னாடகம்} (\textit{Nāṭakam} – drama). We also find terms like ‘\textit{Paṇṇisai}’\index{Pannisai@\textit{Paṇṇisai}} and ‘\textit{Ezhisai}’\index{Ezhisai@\textit{Ezhisai}} in ancient Tamil literature, which we will see a little later in this paper. Other than these concepts, neither any clear-cut definition of ‘Tamil Isai’ (as a recognized genre of music)seems to be available in any musical or literary text, nor does it seem to have been defined by any prominent composer or poet in the history of music. Rather it is a term that was given a lot of impetus during the Tamil Isai movement (\textit{Tamiḷ Isai Iyakkam)} which started off during the 1930s and 1940s as a result of frustration at what was perceived to be the denigration of Tamil songs in Carnatic music concerts in Tamil Nadu.

\vskip 4pt

Certain misleading interpretations like the one mentioned below deserve special attention.

\begin{myquote}
“Two overlapping terms refer to classical music in South India. By far the better known of the two is \textit{Karṇāṭaka Saṅgīta} (or Karnatak music), which is for many synonymous with South Indian classical music. The other term, \textit{Tamiḷ Isai} is relatively unknown outside the state of Tamil Nadu, where the majority of residents speak Tamil as their mother tongue. These two terms do not necessarily refer to two decisively separate musical systems, but rather point to different modes of historical interpretation and competing ideologies based on language and caste. The contrastive use of \textit{saṅgīta} (Sanskrit) and \textit{isai} (Tamil), which have both been translated into English as “music”, is an eloquent testimony to the different linguistic and caste orientations. Schematically put, \textit{Karṇāṭaka Saṅgīta} refers to the culture of classical music based on compositions in Telugu and Sanskrit and performed and patronized primarily by members of the Brahman caste, whereas \textit{Tamiḷ Isai}, music in the Tamil language and/or a musical tradition nurtured by Tamils, has been advanced mostly by\break non-Brahmans.”\hfill (Terada 2008: 203)
\end{myquote}

\vskip 4pt

No doubt, Tamils are extremely proud of their culture, language and arts and striving for popularizing Tamil songs in Carnatic concerts is highly commendable and exemplary. But, that is just one part of the story. The political and divisive forces operating at that time took advantage of this movement; misrepresented it and attempts were made to create a ‘separate identity’ for Tamil Isai distinct from Carnatic Music. In fact, the concept of Tamil Isai was used more as a means of hatred against other languages, particularly Telugu and Sanskrit; and as weapon of accusation against the Brahmans and Vedas. This outlook of Tamil Isai can be clearly seen in the interpretation given above.

\vskip 4pt

A more realistic analysis of the Tamil Isai movement can be seen in the passage given below.

\vskip 4pt

\begin{myquote}
“The Tamil Isai Iyakkam\index{Tamil Isai Iyakkam} (Tamil Music Movement) developed as an auxiliary element of the Tani Tamil movement, which owed its genesis to the course of non-Brahmin agitation in the Madras presidency from around the closing decades of the 19th century. A key element in the movement was the celebration of Tamil language, which was identified as the principal vehicle for articulating a distinct identity and setting it apart from the trajectory of mainstream Indian nationalism.”
\end{myquote}



~\hfill (Subramaniam 2004: 66)

Besides, since each state, ethnicity or language in India has its own musical tradition and musical compositions, whether or not Tamil Isai deserves this separate and special identity isdebatable. Moreover, if language must become the main criterion in identifying a form of music, then Telugu Music, Kannada Music, Sanskrit Music, Malayalam Music, Hindi Music and so on also need to be given their exclusive identities, which in turn will prove detrimental to the interest of Indian Classical music (particularly South Indian Music if we consider Hindusthani music\index{Hindusthani music} to have got a separate entity due to the Persian influence post Muslim invasion).

Nevertheless, let’s say we subscribe to the concept of Tamil Isai as a very ancient form of music that was highly evolved several centuries ago, it is still reasonable, legitimate and appropriate to consider it as a part of Indian music as a whole, an integral and inseparable component at that. This in any way need not and should not be taken as undermining or negating the greatness of Tamil Isai and its contribution to Indian music at large.

This integration of Tamil Music with the Vedic and Sanskritic Tradition can be understood under the following heads.

\subsection*{I. Origin of music:}

Indian music is very old. Its literature dates from the period prior to the beginning of the common era. References to music are contained in the \textit{Veda}-s\textit{, Upaniṣad}-s, \textit{Rāmāyaṇa,\index{Ramayana@\textit{Rāmāyaṇa}} Mahābhārata\index{Mahabharata@\textit{Mahābhārata}}} and \textit{Purāṇas\index{Puranas@\textit{Purāṇas}}}. Though several theories can be attributed to the origin of music, scholars affirm that the origin of Indian music can be traced to the \textit{Sāma Veda}\index{Sama Veda@\textit{Sāma Veda}}, which in turn has its roots in ‘\textit{Aum}’ or ‘\textit{Praṇava}’.

\begin{myquote}
“It is the ‘\textit{Aum}’ from which originated the \dev{आर्चिक} (usage of one note), \dev{गाथिक} (two notes), \dev{सामिक} (three notes), \dev{स्वरान्तर} (four notes), \dev{औडव} (five notes), \dev{षाडव} (six notes), and finally \dev{सम्पूर्ण} (seven notes) or the full-fledged scheme of solfa syllables or \textit{Saptasvara}-s.
\end{myquote}

\newpage

\begin{myquote}
\end{myquote}

\begin{longtable}{@{}|c|c|c|c|@{}}
\hline
\dev{आर्चिक} & \dev{सामिक} & \dev{औडव} & \dev{सम्पूर्ण} \\
\hline
 &  &  & \dev{प} \\
\hline
 &  & \dev{ध} & \dev{ध} \\
\hline
 & \dev{नि} & \dev{नि} & \dev{नि} \\
\hline
\dev{स} & \dev{स} & \dev{स} & \dev{स} \\
\hline
 & \dev{रि} & \dev{रि} & \dev{रि} \\
\hline
 &  & \dev{ग} & \dev{ग} \\
\hline
 &  &  & \dev{म} \\
\hline
\end{longtable}

~\hfill Acharya (2011: 13,14)

The earliest practical application of these \textit{Saptasvara}-s\index{Saptasvara@\textit{Saptasvara}} can be seen in the \textit{Mantra}-s of \textit{Sāma Veda}. Here is an excerpt of a hymn from \textit{Sāma Veda} that shows how the seven \textit{svara}-s are employed.

\begin{verse}
 \dev{“स ग रि स नि द प, स,\kern 0.01pt, प, स, रि ग रि स रि ग रि स रि ग रि स स, प, स, रि ग रि स रि ग रि स}\\\dev{ओहि......................ई.. पा.. हि.. श.............................................. प स्तु व}\\\dev{ज्र........................................ }\\\dev{रि ग रि स स,\kern 0.01pt, म ग रि स, रि ग रि स रि ग रि स रि ग रि स स,\kern 0.01pt, नि रि स,\kern 0.01pt, स,\kern 0.01pt, रि ग रि स रि ग रि स}\\\dev{................. नै... दयू.......... र्न................................................. प्र.. दि न....... स...व}\\\dev{................................}\\\dev{स, नि, स,\kern 0.01pt, स,\kern 0.01pt, स,\kern 0.01pt, म ग रि,\kern 0.01pt,\kern 0.01pt, स,\kern 0.01pt,\kern 0.01pt, नि,\kern 0.01pt,\kern 0.01pt, द,\kern 0.01pt,\kern 0.01pt, प,\kern 0.01pt,\kern 0.01pt, स,\kern 0.01pt,\kern 0.01pt,}\\\dev{...................... हा.....ओहि......................................... "}

~\hfill (Acharya 2011: 13)
\end{verse}

In the history of world music, the Indian sages and musicians were the first to deeply study the intricacies of music and arrive at these \textit{Saptasvara}-s several centuries ago, which remain unaltered and intact till date. There can be no svara or note that is outside this septet. This undoubtedly is the basis for any music, including Tamil music.

It is important here to note the phenomenal contribution and role of Sage Agastya, who is considered by the Tamils as the founding father of Tamil language and all elements of its culture, including music and dance. But it also equally significant and imperative to place on record a fact narrated by the Purāṇic account that Agastya\index{Agastya} was sent south by none other than Lord Śiva, as per the advice of Viṣvakarman, the Hindu God of Architecture. This is sufficient evidence to show that Tamil culture and music was not separate from the Vedic culture even at the time of its inception and there was no north-south divide in the basic cultural and ideological identity of India.


\subsection*{II. Theory of Music (Musical treatises):}

With respect to ‘\textit{Sangītaśāstra}’ or theory of music, a lot of parity is evident.

\begin{enumerate}[{\rm 1.}]
\itemsep=0pt
\item The Sangam poems are composed in accordance with the rules prescribed by \textit{Nāṭyaśāstra}.

 \item The famous text ‘\textit{Silappadikāram}’\index{Silappadikaram@\textit{Silappadikāram}} is a ‘\textit{Nāṭakakāvyam}’ (dramatic composition) based on \textit{Nātyaśāstra}\index{Natyasastra@\textit{Nātyaśāstra}}.

 \item The division of poetry into \textit{Aham} and \textit{Puram} based on \textit{Śriṅgāra} and \textit{Tāṇdava (Āviddham) - puram} of Nātya Śāstra.

 \item Eight \textit{rasa}-s described by Bharata in \textit{Nāṭyaśāstra }are also described in the Tamil text ‘\textit{Tolkāppiyam}’\index{Tolkappiyam@\textit{Tolkāppiyam}} with Tamil names.

 \item Sage Agastya, who is the chief architect of Tamil culture, is mentioned along with Ātreya, Vaśiṣṭha, Pulastya, Pānḍyarasa, Gautama, and others as masters of music and dance in the \textit{Nāṭyaśāstra}.

 \item The rules relating to qualifications of singers and poets, making of instruments, construction of theatre and so on that appear in the chapter called ‘\textit{Arangeṭṭrukādai}’ of the text ‘\textit{Silappadikāram}’bear exact semblance to those mentioned in the \textit{Nāṭyaśāstra}.

\end{enumerate}

Here it is important to understand the concept of ‘\textit{Mārga}’ and ‘\textit{Deśī}’ traditions in Indian music/arts. Mārga is classical, antique, pure, traditional and strictly adhering to textual guidelines; whereas Deśī refers to regional variations that suit the taste and preferences of the local connoisseurs.

\begin{verse}
\dev{“मार्गो देशीति तद् द्वेधा तत्र मार्गः स उच्यते~।}\\\dev{यो मार्गितो विरिञ्च्याद्यैः प्रयुक्तो भरतादिभिः~॥}\\\dev{देवस्य पुरतः शम्भोः नियताभ्युदयप्रदः~।}\\\dev{देशे देशे जनानां यद्रुच्या हृदयरञ्जकम्~॥}\\\dev{गीतं च वादनं नृत्तं तद्देशीत्यभिधीयते~।}\\(Verse 22,23,24 Chapter 1 of Saṅgītaratnākara of Śārṅgadeva)”

~\hfill (Sastri 1943: 14,15)
\end{verse}

Ancient texts like the \textit{Nāṭyaśāstra} in addition to laying down clear-cut principles for the practice and performance of various aspects of dance, drama and music, have not failed to recognize and appreciate the presence of regional variations in language, styles and presentation.

\begin{myquote}
“Pravṛtti (Chapter 14, 36-54) Producers of drama tell us that there are four kinds of pravṛtti-s, viz. Āvantī, Dākṣiṇātyā, Pāñcālī and Oḍramāgadhī. What is Pravṛtti? The answer is: that which tells us, that in this earth there are various countries with different dresses, different languages and different customs. True, there are various countries. But how can you classify them into only four groups? The answer is: because of features which are common (to countries grouped together). Though there are these differences, looking in to the vṛtti (style of production), the four-fold division has been accepted by people (lokānumata). Different regions prefer different styles like the Bhāratī, Sāttvatī, Kaiśikī and Ārabhaṭī. Because of this, four pravṛtti-s (regional styles) have been devised. For example, in the southern region, a production is mostly in Kaiśikī style, with plenty of dance and music and clever, charming and graceful gestures. The countries within the mountains, Mahendra, Malaya, Sahya, Mekhala and Kālapañjara are known as Dakṣiṇāpatha (Deccan). Kosala, Tosala, Kaliṇga, Mosula, Drāmīḍa, Āndhra, Vaiṇya and Vānavāsaka are the countries where the production must be in Dākṣiṇātyā pravṛtti (i.e. Kaiśikī vṛtti)...... Experts should employ pravṛtti-s and vṛtti-s as common to different regions, but if the audience and the place and the occasion demand it, they may be intermixed”\hfill (Rangacharya 1996:113,114)
\end{myquote}

As it can be seen from the above, the Deccan region was commonly referred to as \textit{Dakṣiṇāpatha} or \textit{Dakṣiṇātya} with \textit{Drāmīḍa} (or \textit{draviḍa} or \textit{dramiḷa}) being a part of it. Shulman\index{Shulman, David} (2016: 13) also points out that classical Sanskrit uses \textit{draviḍa} both to refer to Tamil speakers specifically and, at times, to indicate south Indians generally.

In quotations like the one above as well as the ones that follow, it is very clear that the Mārga or Sanskrit tradition has only acknowledged and appreciated the tradition of the South and always allowed it to blossom, and has never been detrimental to it.

\begin{myquote}
\dev{तत्र दक्षिणात्यास्तावत् बहु नृत्तगीतवाद्या कैशिकी प्रायाः चतुरमधुरललितांगाभिनयाश्च...... दक्षिणस्य समुद्रस्य तथा विंध्यस्य चांतरे~॥} - “The Southern provinces of India delight in various kinds of dances, vocal music and instrumental music and they are charecterised by a graceful style and their abhinayas are clever and polished. This beautiful land of fine art extends from the Vindhya mountatins to the Sea in the South.”\hfill (Sambamoorthy 2005:15)
\end{myquote}

\begin{myquote}
“The \textit{Tevāram} stands as the finest and the earliest example of Deśī Sańgīta\index{Desi Sangita@\textit{Deśī Sańgīta}}. Deśī sańgīta is the name given to the music that naturally developed in the different provinces of India as opposed to Mārgī sańgīta\index{Margi sangita@\textit{Mārgī sańgīta}} which was developed by smṛtikāra-s or a group of lakshaņakāra-s.”

~\hfill (Sambamoorthy (6) 2006:88)\index{Sambamoorthy, P.}
\end{myquote}

These references make it clear that in Indian tradition though each variation or style was valued for its specialty, exclusive features and richness, there was never a strict or rigid demarcation between Mārga and Deśī as well as between the various Deśī genres. Over the centuries ‘Deśī sangīta’ is known to have assimilated the good traits of the old ‘Mārga sangīta’. This implies that there were no clear-cut divisions between classical and popular, national and regional, northern and southern, etc., with one identity constantly merging with the other. Never in history was there any conflict between the two.

Besides, over the centuries, a huge majority of texts on musicology are in Sanskrit and we can see their unconditional acceptance and compliance in Tamil Nadu. Some of the most important Sanskrit texts on musicology are as under.

\begin{enumerate}[{\rm 1.}]
\itemsep=0pt
\item \textit{Dattilam} – Dattila – 4th century BCE to 2nd century CE

 \item \textit{Nāṭyaśāstra} – Bharata – 2nd century BCE to 2nd century CE

 \item \textit{Bṛhaddeśi} – Mataṅga – 6th century CE

 \item \textit{Saṅgītamakaranda} – Nārada – 11th century CE

 \item \textit{Saṅgītasudhākara} – Haripāladeva – 12th century CE

 \item \textit{Mānasollāsa} – Cālukya Someśvara – 12th century CE

 \item \textit{Saṅgītaratnākara} – Śarṅgadeva – 13th century CE

 \item \textit{Saṅgītasāra }– Vidyāraṇya – 14th century CE

 \item \textit{Svarameḷakalānidhi} – Rāmāmātya – 16th century CE

 \item \textit{Saṅgītapārijāta} – Ahobala – 17th century CE

 \item \textit{Caturdanḍīprakāśikā} – Venkaṭamakhi - 17th century CE

 \item \textit{Sangrahacūḍāmaṇi} – Govindāchārya – 18th century CE

 \item \textit{Saṅgītasārāmṛta} – Tuḷajāji – 18th century CE

~\hfill (Sampatkumaracharya, Ramaratnam 2000: 251-275)

\end{enumerate}


\subsection*{III. Musical literature:}

Music of the ancient Tamils, like that of any other part of India, has been largely sacred in character. Arts, particularly music, have always been considered as the easiest and most pleasant path to reach God. In Sanātana Dharma, the concept of God as ‘\textit{Nādabrahma}’ (Embodiment of musical sound) is highly significant. Thus, ‘\textit{Bhakti}’ has been the central theme in most musical compositions; and Tamil compositions are no exception to this. The Bhakti movement\index{Bhakti movement} having its earliest expression in Tamil Nadu; and music being a part of the daily worship ritual in temples of Tamil Nadu reinforce this fact beyond question. This does not mean that the entertainment aspect of music was altogether neglected. The performances of ‘Sarva Vādyam’, music and dance dramas on sacred themes in temples drawing large crowds stands as a testimony to the fact that there was no precise demarcation between sacred and secular in ancient Indian thought. Malhotra (2011:112) rightly mentions that performing arts were positioned as spiritual practices for the general public, including those who lacked the qualifications and aptitudes to directly access the scriptures. 

The most ancient form of Tamil musical literature available to us today are the \textit{Sangam} poems dating back to 2nd-3rd centuries A.D. We then come across the \textit{Tevāram} and \textit{Tiruvāchakam} by the \textit{Śaiva} saints Tirujñāna Sambandar, Tirunāvukkarasar, Sundarar and Mānikkavāsagar which are said to have been composed between 7th and 9th centuries A.D. Then come the \textit{Vaiṣhṇava} saints, the \textit{Āḷvars }around the 11th century with their \textit{Divya Prabandham}. The \textit{Tevāram}-sand \textit{Divya Prabandham}-s\index{Divya Prabandham@\textit{Divya Prabandham}} constitute the cream of sacred music in Tamil. \textit{Tiruppugaḷ} by Aruṇagirināthar were composed around the 15th century and then there were many songs composed by a galaxy of Tamil composers like Muttu Tānḍavar, Mārimutta Piḷḷai, Ūttukkāḍu Venkaṭa Kavi, Gopālakṛṣṇa Bhāratiyār among others.

\begin{myquote}
“…Saint Jñānasambandar who was the greatest contributor to Tamil Music and devotional literature was a \textit{Chaturvedi} who was performing daily Vedic rites. Saint Appar was an agriculturist, who has rendered several passages from Veda, especially \textit{Śrī Rudram} into delightful Tamil. Nammaḻvār's \textit{Thiruvāymoḻi} is so replete with Vedic passages. His poems are called "Vedas rendered in Tamil”.

~\hfill (\url{http://tamilartsacademy.com/})
\end{myquote}

\begin{myquote}
“Sambandar was born in a Vedic Brahmin family, deeply learned in Vedas and Vedāngas, and also the 10 Mahāpurānas had sung Śiva in more than 4000 songs. He was the greatest Vedic Brahmin to have enriched particularly music with the result the whole of Vedic lore was available in Tamil. He lived in the first half of 7th century. The greatest of Tamil poets, Sambandar has given a complete picture of Vedic learning and practices in Tamilnadu. His contemporary was Saint Appar, a Vellālar by birth, mastered Vedas and poured forth over 3000 most beautiful songs in Tamil furnishing remarkable information on Saivite practices and at the same time even providing delightful translations for some Vedic and Sanskrit passages. The contributions of Vaiṣhṇava Ālvārs rendering Vedic traditions to Tamil and the integration of the epics Rāmāyana and Mahābhārata especially Krishna’s sports deserve notice. The \textit{Tevārams} and \textit{Prabhandams} raised the Tamil songs into virtual Vedas, called the Tamil Vedas. From around 600 to 1000 we may call the epoch as Tamil Vedic age. And it is the same age that saw several Vedic colonies and colleges (like \textit{Ghatikasthānas} and \textit{Vidyāsthānas}) emerging.”

~\hfill (Nagaswamy 2017: x.xi)
\end{myquote}

There are umpteen numbers of illustrations in the entire gamut of Tamil musical literature which are enough to prove its essentially~Indic nature. The Tamil poems speaking of \textit{dharmaśāstric} ideas, \textit{Rāmāyaṇa} and \textit{Mahābhārata}, the virtue of the path of the four Vedas, and the \textit{Brāhmaṇa}-s performing Vedic sacrifices, point to the irrefutable fact, that the Tamil country followed the \textit{Vaidikadharma}, from its earliest known times. These can be seen under the following heads.

\subsubsection*{A. Reference to Vedas, Vedic rites and Brāhmaṇs:}

\begin{enumerate}[{\rm 1.}]
\itemsep=0pt
\item 
 Perum Devanār who is believed to be the first \textit{Sangam} poet by some scholars and also the one who authored the Tamil version of \textit{Mahābhārata},\index{Mahabharata@\textit{Mahābhārata}} in his song of benediction (poem number 1), says:

\begin{myquote}
\tamil{“கறை மிடறு அணியலும் அணிந்தன்று அக்கறை}\\\tamil{ மறை நவில் அந்தணர் நுவலவும் படுமே.....}\\\tamil{பதினெண் கணமும் ஏத்தவும் படுமே”}
\end{myquote}

\begin{myquote}
The poison stuck in his throat is like a mark (but he wears it as if it is an ornament). Brahmans learned in the Vedas praise it. The eighteen \textit{gaṇa}-s too always praise it.”

~\hfill (\url{http://kauniyansri.blogspot.in/2014/10/literature-light-and-delight-brahmins_20.html})
\end{myquote}

 The description of the Lord taking poison and the mention of the eighteen \textit{gaṇa}-s of Lord Śiva (patinenn kanam) shows that the \textit{Sangam} poets knew the Purānic lore. Also, the mention of Vedas and the Brahmans is noteworthy.

 \item 
 Poem 2 of the \textit{Puranānūru}\supskpt{\endnote{\textit{Puranānūru}, like other \textit{Sangam} works, was unearthed and gathered from different sources by Dr. U.V. Swaminatha Iyer and edited and published by him first in 1894. He published the third edition in 1934. The volume contains massive amount of detailed information and excellent notes and explanations. Almost every subsequent writer and editor has made use of this work.}}\index{Purananuru@\textit{Puranānūru}} anthology in the \textit{Sangam} poems (considered as poem 1 by most scholars) is sung by poet Murañjiyūr Muḍināgarāyar in praise of the Chera king Perumcotru Udiyan Ceralādan.

\begin{myquote}
\tamil{“பால் புளிப்பினும் பகல் இருளினும்}\\\tamil{நால் வேத நெறி திரியினும்}\\\tamil{திரியாச் சுற்றமொடு முழுது சேண் விளங்கி}\\\tamil{நடுக்கின்றி நிலியரோ அத்தை”}

~\hfill (\url{http://kauniyansri.blogspot.in/2014/10/literature-light-and-delight-brahmins_20.html})
\end{myquote}

\begin{myquote}
“Then the poem goes on to praise the retinue of the king, as “faithful to him and accompanied by him in every endeavour, even if the day turned into a dark night, the milk lost its flavour and turned sour, and the path of four Vedas changed from its course of righteousness. This expression that if even the path of four Vedas changed its course of righteousness illustrates how the path of the four Vedas were venerated and looked upon. It is a poetic expression to say it was held in great esteem.”\hfill (Nagaswamy 2017: 5)
\end{myquote}


 \item 
 In the same poem,

\begin{myquote}
\tamil{“.....அடுக்கத்துச்} \\\tamil{சிறுதலைநவ்விப்பெருங்கண்மாப்பிணை}\\\tamil{அந்திஅந்தணர்அருங்கடன்இறுக்கும்}\\\tamil{முத்தீவிளக்கில்துஞ்சும்}\\\tamil{பொற்கோட்டுஇமயமும்பொதிகையும்போன்றே”}

~\hfill (\url{http://kauniyansri.blogspot.in/2014/10/literature-light-and-delight-brahmins_20.html})
\end{myquote}

\begin{myquote}
“And lastly, the poem says that under his protection the Brāhmaṇas performed regularly the sandhi daily junctions of time and offerings in the three alters – āhavanīya, dakṣiṇāgni and gārhapatya, as prescribed in the Vedas, from the Himālayas to the Podigai\supskpt{\endnote{Podigai hills are dear to ancient Tamils as the seat of Sage Agastya from whom Tamil language originated.}} hills without any fear. The tender and long eyed antelopes freely came and took shelter near their sacrificial fires as none would harm them. The king gave full protection to the Brāhmaṇas to perform offerings in the three fires.”\hfill (Nagaswamy 2017: 5)
\end{myquote}


 \item Kothandaraman (2015: 23,24) observes that “It is worth noting that the Sangam poems were the earliest to refer to the connection of \textit{Śiva} with Vedas. From amongst them, though the initial prayers of \textit{Kaḷithohai} and \textit{Paripāḍal} are considered to belong to later periods, there is no doubt in the antiquity of the references in \textit{Puram} 166. It says: Vedas are very ancient poems, dealing with only virtues and have been well researched. They are non-separable from the Grand Old God with long matted hair. They are documents of antiquity in four parts and are understood with the help of six guiding books. Oh! King, you belong to the lineage of the wise and famous people who, in order to counter the falsehood contrary to Vedas, had preached the truth in a convincing manner and had performed the twenty-one types of Yagnas perfectly.”

 \item Poem no. 360 of Puram anthology which is a composition of~the~great poetess Avvaiyar has a description of the Rājasūya Yāga\index{Rajasuya Yaga@\textit{Rājasūya Yāga}}~performed by the Chola king Perunar Kiḷḷi, where he invited the Chera and the Pandya kings. Nagaswamy\index{Nagaswamy, R} (2017: 6) gives a translation of this poem – “You kings, you have made this whole world the \textit{devaloka}, dividing it into parts and made them the property of Brāhmaṇas, by placing gold and flowers in their hands and pouring water as an act of gift. Having made gifts to them, you have also made limitless gifts to others (chieftains and soldiers) who were celebrating great victory as a result of their valour in battles…”

 \item The Śaiva saints preached that devotion to \textit{Śiva} can co-exist with the performance of Vedic yagnas. Kothandaraman (1995: 92,93,94,95) points out that Appar, in many of his songs, praises \textit{Śiva} by associating him with Vedas; Sundarar calls \textit{Śiva} as the Lord of Vedas; Māṇikkavāsagar\index{Manikkavasagar@\textit{Māṇikkavāsagar}} also presents \textit{Śiva} as the chief of Vedas; Sambandar’s\index{Sambandar} role was the maximum in the growth of \textit{Vedic Śaivism}. Sekkiḷār says that Sambandar was born for the resurgence of \textit{Śaiva} sect of the Vedic faith. Sambandhar, on the occasion of~his \textit{Upanayanam} (initiation into Vedic studies) ceremony, says that ‘\textit{Namas Śivāya}’ mantra is the inherent truth if the four Vedas and he prescribes it as the quintessence of \textit{Vedas} and the simplest way to attain the benefit of learning the whole Vedas for those who cannot study them fully.

 \item 
 Tirugñānasambandar, in a \textit{Tevāram} in praise of the Chola princess Mangaiyarkkarasi, is seen acclaiming the four Vedas.

\begin{myquote}
\tamil{மங்கையர்க்கரசிவளவர்கோன்பாவைவரிவளைக்கைமடமானி}\\\tamil{பங்கயச்செல்விபாண்டிமாதேவிபணிசெய்துநாடொறும்பரவ}\\\tamil{பொங்கழலுருவன்பூதநாயகன்நால்வேதமும்பொருளுக்குமருளி}\\\tamil{அங்கயற்கண்ணி தன்னொடும்அமர்ந்தஆலவாயாவதுமிதுவே”}
\end{myquote}

\begin{myquote}
This is Ālavāy, where dwells the flame-formed lord of hosts, giver of the four Vedas and their meaning, with the fair fish-eyed maid. Here, reigning like the goddess of good fortune, Mangaiyarkkarasi the Chola king’s daughter, braceleted chaste Pāṇdiyan queen, daily serves and praises God.”\hfill (Kingsbury \& Phillips 1921: 30,31)
\end{myquote}


\end{enumerate}


\subsubsection*{B. References to Vedic Gods}

\begin{enumerate}[{\rm 1.}]
\itemsep=0pt
\item Among the ancient Tamils, it was poetess Ammaiyar (around 6th century CE), one of the three women amongst the 63 Nāyanmārs,\index{Nayanmar@\textit{Nāyanmār}} who for the first time opposed atheism and other principles of the non-Vedic people and established Ṣaivam with Vedic base. Kothandaraman (2015: 59, 61)has shown that the God worshipped by Ammaiyar was firmly established in Vedas. She praises Him highly addressing as ‘\textit{Vedhiyan}’ (belonging to Vedas), ‘\textit{Vedaporuḷan}’ (the substance of Vedas) and ‘\textit{Vedathirku ādiyan}’ (Root element of Vedas). Moreover, she uses the term ‘\textit{Vānor Perumān}’ meaning ‘Great Lord’. The same concept is found in \textit{Atharva Veda} (9-7-7) where the name ‘\textit{Mahādeva}’ is attributed to \textit{Rudra}. (also appears in \textit{Śatapatha Brāhmana} 6.1.3.7 and \textit{Kauśītakī Brāhmana} 6.1.91) The word ‘Śiva’ refers to red colour in Tamil and one who does good in Sanskrit. Tirumūlar\index{Tirumular@\textit{Tirumūlar}} must have referred only to this word when he says, “He is referred to both by the Tamil and Sanskrit words. (Tirumandiram 66 – Āgamacirappu 10)

 \item 
 In a poem by Māṇickavāsagar, one can see a clear reference to the Vedic Gods Indra, Viṣṇu and Brahma.

\begin{myquote}
\tamil{“கொள்ளேன் புரந்தரன் மாலயன் வாழ்வு குடி கெடினு}\\\tamil{நள்ளே னினதடி யாரொடல்லானர கம்புகினு}\\\tamil{மெள்ளேன் றிருவருளாலே யிருக்கப்பெறினிறைவா}\\\tamil{வுள்ளேன் பிறதெய்வமுன்னையல்லாதெங்களுத்தமனே"}
\end{myquote}

\begin{myquote}
Indra or Vishnu or Brahma, their divine bliss crave not I; I seek the love of Thy saints, Though my house perish thereby. To the worst hell I will go, So but Thy grace be with me. Best of all, how could my heart think of a god beside thee?”\hfill (Kingsbury \& Phillips 1921: 88,89)
\end{myquote}


\end{enumerate}


\subsubsection*{C. Vedic concepts and ideologies}

\begin{enumerate}[{\rm 1.}]
\itemsep=0pt
\item It is said that the Sangam poems glorified the \textit{Karpu }(chaste) form of marriage prescribed in Vedic system. Kaṇṇaki,\index{Kannaki@\textit{Kaṇṇaki}} the heroine was married as per the Vedic rites.\\ (\url{http://tamilartsacademy.com/})

 \item Nagaswamy(2017: ix) affirms that “The great Thiruvalluvar in his \textit{Thirukkural} declared that Antanār-s (Vedāntin-s) are those who recognised greatness of every human soul.”

 \item Poem 2 of \textit{Puranānūru} anthology sung by poet Murañjiyūr Muḍināgarāyar is a classic example. According to Nagaswamy (2017: 3,4) the first part of the poem states that this king has the qualities of the five great basic elements or \textit{pañca-mahā-bhūta}-s, the earth, water, air, fire and ether. These five elements are detailed in the \textit{Dharmaśāstra} of Manu in chapter 7 verses 4 \& 5. Hence, the concept of the \textit{pañca-bhūta}-s constituting the qualities of kings is a part of the Vedic \textit{Dharmaśāstra}.

 \item 
 ‘The Vaiṣṇavite musical literature which is in the form of the\break ‘\textit{Nālāyira Divya Prabandham}’ (the four thousand divine verses) is referred to as ‘Drāviḍaveda’ or ‘Dramiḍopanishad’ indicating that they are nothing but the four Vedas in Tamil; and Śrī Rāmānujāchārya is hailed as ‘\textit{Ubhayavedāntapravartaka}’ (the propagator of both the Sanskrit and Tamil schools of Vedas)\supskpt{\endnote{It is interesting to note that the 108 \textit{Divyadeṣa}-s considered as sacred by the Vaiṣṇavite tradition are spread all over the country including the northern states of Gujarat, Uttar Pradesh and Uttarakhand (also one in Nepal). Equally intriguing is the fact that the ritual of procession of the deity (\textit{Utsava}) in these temples always features the recital of the \textit{Divya Prabandham} in the fore, followed by the idol, which is lastly followed by the chanting of the Vedas. Similarly, all Tamil Śaivaite hymns speak of Śiva as living in the Himalayas. Thus, we only see harmony throughout and no friction or disagreement whatsoever.}}. One of the several references to this concept of ‘Tamil Veda’ can be seen in a conversation between the Lord and the great Vaiṣṇava saint Nammāzhvār, which is described by Vedanta Dishika in his Pādukāsahasra 2nd shloka of 2nd chapter Samākhyāpaddhati.

\begin{verse}
\dev{“द्रमिडोपनिषन्निवेशशून्यानपि लक्ष्मीरमणाय रोचयिष्यन्~।}\\\dev{ध्रुवमाविशति स्म पादुकात्माशठकोपः स्वयमेव माननीयः॥”}

~\hfill (\url{http://www.prapatti.com/slokas/sanskrit/paadukaasahasram/samaakhyaa.pdf})
\end{verse}


 \item 
 Nammāzhvār, who composed the \textit{Tiruvāymozhi}, in one of his poems (called \textit{Pāśuram}-s) affirms the \textit{Apauruṣeyatva} or divine origin of the \textit{Pāśuram}-s akin to that of the Vedas.

\begin{myquote}
\tamil{“என்னெஞ்ஜத்துள் இருனந்து இங்கிரும் தமிழ்னூல் இவைமொழிந்த வன்னெஞ்ஜத்த இரணியனை மார்விடந்தவாட்டாத்தான்”}
\end{myquote}

 “The Lord of Thiruvāṭṭār (Narasimha), who destroyed tough chested Hiraṇyakaṣipu, now having stayed in my heart, poured out these Tamil śāstras.”

 \item 
 The Puruṣasūkta proclaims that the ‘birthless’ (meaning Almighty) emerges through many forms like Descent (Vibhava), Deity (Arca) and Delegate (Amṣa) (\textbf{\dev{अजायमानो बहुधा विजायते।}}) The same idea is expounded in the following Pāśuram of Nammāzhvār.

 \tamil{“பிரப்பில் பல்பிரவி பெருமானை மரப்பொன்னென்னி எண்ணம் மகிழ்வேனே”.}

\end{enumerate}


\subsubsection*{D. Elements of Ramayana and Mahabharata}

\begin{enumerate}[{\rm 1.}]
\itemsep=0pt
\item 
 In a \textit{Tiruppugaḷ} by Aruṇagirināthar–‘\tamil{தாக்கமருக்கொரு}’ we come across an elaborate description of the Ramayana.

\begin{myquote}
\tamil{முக்கறை மட்டைமஹாபல காரணி}\\\tamil{சூர்ப்பநகைப்படு மூளியுதாசனி}\\\tamil{மூர்க்க குலத்தி விபீஷணர் சோதரி....... முழுமோடி}\\\tamil{மூத்தவரக்கனி ராவண னோடியல்}\\\tamil{பேற்றிவிடக்கமலாலய சீதையை}\\\tamil{மோட்டன் வளைத்தொரு தேர்மிசையே கொடு...... முகிலேபோய்}\\\tamil{மாக்கன சித்திர கோபுர நீள்படை}\\\tamil{வீட்டிலிருத்திய நாளவன் வேரற }\\\tamil{மார்க்கமுடித்த விலாளிகள் நாயகன்....... மருகோனே}\\\tamil{வாச்சிய மத்தள பேரிகை போல்மறை}\\\tamil{வாழ்த்த மலர்க்கழுநீர் தரு நீள்சுனை}\\\tamil{வாய்த்த திருத்தணி மாமலை மேவிய...... பெருமாளே}
\end{myquote}

\begin{myquote}
“She got her nose cut off. She was stupid and cruel; yet had a lot of muscle-power. She was the root cause (for Ravana’s downfall). Her name was Soorpanagai\index{Soorpanagai}. She deserves to be disrespected. She belonged to an obstinate breed. Although she was the sister of the good Vibheeshana, she was a consummate temptress. She went to Ravana and described Sita’s beauty in such a tempting way that Sita, who was none other than Lakshmi, was kidnapped cunningly by that Rakshasa. He took her in his chariot which flew off towards the clouds to Lanka, which has tall and picturesque buildings and armies. There in Ashokavanam, he imprisoned her. On that day, it was resolved that his dynasty would be rooted out, by expert archers, whose leader was Rama and You are his nephew! Sounding like many percussion instruments as drums and trumpet is the chanting of Vedas in your shrine. There is a perennial fountain yielding red water lilies daily at Tiruttani and You reside on its hill, Oh great one!”

~\hfill (\url{http://kaumaram.com/d_loads/kdc/pdfs.php?kdc=kdcb798pd})
\end{myquote}


 \item 
 Poem 2 \textit{Puranānūru} anthologypraises the king as the participant in the Mahābhārata war between the hundred Kaurava-s and the five Pānḍava-s and fed both the armies in the field with sumptuous feast, when the Kaurava-s were uttery routed.

\begin{myquote}
\tamil{“அலங்குளைப்புரவிஐவரொடுசினைஇ}\\\tamil{நிலம்தலைக்கொண்டபொலம்பூண்தும்பை}\\\tamil{ஈரைம்பதின்மரும்பொருதுகளத்தொழிய}\\\tamil{பெரும்சோற்றுமிகுபதம்வரையாதுகொடுத்தோய்”}

~\hfill (\url{http://kauniyansri.blogspot.in/2014/10/literature-light-and-delight-brahmins_20.html})
\end{myquote}

\begin{myquote}
“It might be a myth or even a poetic exaggeration, but it still makes clear, that this Tamil king did feel he was a part of the country and did not stand in isolation. The reference to \textit{Mahābhārata} war and the king’s participation would show that the epic was a part of Tamil ethos at the very beginning of Tamil history, and is an identity and not isolation. The Tamil territorial division and linguistic difference did not make them followers of an independent culture, but remained one with the rest of the country. This illustrates a feeling of oneness.”

~\hfill (Nagaswamy 2017: 4,5)
\end{myquote}


\end{enumerate}


\subsubsection*{E. Sanskrit words}

It is repeatedly argued that Tamil is the oldest language having a grammar and structure entirely different from that of any other Indian language and having never been subject to any influence of Sanskrit whatsoever\endnote{As a part of the agenda to find Christian roots for the so-called ‘Tamil religion’ which was proclaimed as separate from the rest of India, Tamil language was being mapped on to a non-Sanskrit (rather anti-Sanskrit) framework. Thanks to the strategies of the Caldwell, who proposed that Sanskrit was imposed on the Dravidians by the cunning Aryan Brahmins and thus there is need for the complete removal of Sanskrit words from Tamil (Malhotra 2011: 62).}. In reality, Sanskrit has been copiously used as a medium of music by various composers from Tamil Nadu (Sanskrit musical compositions) and also there is a marked influence of Sanskrit on many Tamil compositions.

\begin{myquote}
“A glimpse of a literary form like the ‘\textit{Tiruppugaḷ}’ will confirm beyond question that no language in India can ever escape the shadow of the gigantic tree called ‘Sanskrit’. Aruṇagirināthar’s style was distinct from the ancient ‘\textit{Śendamiḷ}’ and also the ‘\textit{Iḷakkaṇa tamiḷ}’ of Kambar or Tiruvaḷḷuvar. He has finely blended the divinity of Sanskrit with the felicity of Tamil. The below two examples show the Sanskritised stlye of the composer.
\end{myquote}

\begin{myquote}
 1. \textit{Nādabindukalādi namo nama Vedamantrasvarūpa namo nama Jñanapanḍitaswāmi namo nama vegukoḍi}\\
 2. \textit{Kumara gurupara muruga śaravaṇa guha ṣaṇmukha kari piragāna Kuḷaga śivasuta śivayanamavena kurava naruḷ guru maṇiye enru}
\end{myquote}

\begin{myquote}
Many a times, though not pure Sanskrit words, a lot of Tamilised Sanskrit words can be seen. For instance, in the expression “\textit{asurar kiḷaivāṭṭi migavāḷa amarar siraimīṭṭa perumāḷe}” words like \textit{asurar }and\textit{ amarar}. In addition, we can see innumerable\textit{Tadbhava rūpa}-s of Sanskrit words. \textit{Sāmi (svāmi), koḍi (koṭi), mandiram (mantram), tattuva (tattva), sattiya (satya), vākkiyam (vākyam) mugam (mukham), tudi (stuti)} and so on.”

~\hfill (Acharya 2017: 96,97)
\end{myquote}

\begin{myquote}
“Venkaṭakavi’s Tamil is very erudite, but Sanskritised to a great extent. In his kriti ‘\textit{Tyāgarāja Parameśa}’ in Cakravāka Rāga, the phrase “\textit{Tyāgarāja parameśa śaraṇāgata varuṇālaya karuṇālaya kamalālaya taṭam ahalā ārūr}” and in ‘\textit{Śen śiva jaṭādhara}’ in Sindhubhairavi Rāga, the expression “\textit{Śen śiva jaṭādhara śambho śankara tripurabhayankara naṭanadhurandhara nīrajākṣa nirupamakara atiśaya}” but for a couple of Tamil words rest all is Sanskrit.”

~\hfill (Acharya 2017: 73)
\end{myquote}

\begin{myquote}
“In a composition of Aruṇācala kavi in Madhyamāvati Rāga, one can see a lot of Sanskrit words.
\end{myquote}

\begin{myquote}
\textit{Śri rāmacandranukku jayamangaḷam nalla divyamukhacandranukku śubha\break mangaḷam}\\\textit{Mārābhirāmanukku mannu parandhāmanukku īrāru nāmanukku ravikulasomanukku}
\end{myquote}

\begin{myquote}
A lot of Sanskrit words, \textit{Tadbhava rūpa}-s and \textit{kriyārūpāntara} verbs can be seen in a composition of Muttu Tānḍavar in Āndoḷikā Rāga.
\end{myquote}

\begin{myquote}
\textit{Sevikkavenḍumayya cidambaram}\\\textit{Sevikkavenḍum cidambaramūrtiyam devādidevan tirusannidhiyai kanḍu}\\\textit{Singāramāna śivagangaiyil muzhugi śivakāmi sannidhi munbhagave vandu}\\\textit{Pāngāgave pradakṣiṇam seydu bhaktargaḷ siddhargaḷ paṇiviḍaiyor tozha}

~\hfill (Acharya 2017: 111)
\end{myquote}



\subsection*{IV. Melody and Rhythm:}

Any talk or essay about Tamil Isai is incomplete without a mention of the ‘\textit{Paṇ}’(\tamil{பண்}). \textit{Paṇ} is the melodic mode used by the Tamil people in their music during the ancient times, particularly in the singing of \textit{Tevaram}-s\index{Tevaram@\textit{Tevaram}}. In short, a \textit{Paṇ} is a \textit{Rāga}. It is said that

\begin{myquote}
“Through the process of modal shift of tonic, they (the ancient tamils) derived 7 scales which have their parallels in the murchanas of the shadja grāma. \textit{Paṇ} (\tamil{பண்}) was a heptatonic scale or sampurna raga; (paṇ also meant a song); \textit{Paṇṇiyattiram} (\tamil{பண்ணியத்திரம்}) denoted a shādava raga or hexatonic scale. \textit{Tiram} (\tamil{திரம்}) denoted an audava raga or pentatonic scale and \textit{Tirattiram} (\tamil{திரத்திரம்}), a swarantara raga\endnote{It is said the Tamil Music is mostly heptatonic and hence known as ‘\textit{Ezhisai}’(\tamil{ஏழிசை}) and so was the music of the \textit{Sāmaveda}.}….The seven notes had their names kuraḷ, tuttam, kaikkilai, uḷai, iḷi, vilari, tāram. These correspond to the notes shadja, rishabha, gandhara, madhyama, panchama, dhaivata and nishada.”

~\hfill (Sambamoorthy(6) 2006:86)
\end{myquote}

\begin{myquote}
“The musical notes were classified into \textit{Inai, Kilai, Natpu} and \textit{Pagai} (\tamil{இணை, கிளை, னட்பு, பகை}) in ancient tamil music. This reminds one of the parallel classification of the notes into vadi, samvadi, anuvadi and vivadi. The three octaves mandra, Madhya and tara were referred to, by the names: \textit{melivu} (\tamil{மெலிவு}) \textit{samam} (\tamil{ஸமன்}) and \textit{valivu} (\tamil{வலிவு})”

~\hfill (Sambamoorthy(6) 2006:98)
\end{myquote}

Thus, we see that, with respect to musical notes and scales, the concepts are all the same, although the ancient Tamils followed a regional nomenclature.

Some \textit{Paṇ}-s and their corresponding ragas are as follows.

\begin{longtable}{@{}|l|l|@{}}
\hline
\multicolumn{1}{|c|}{\textit{Paṇ}} & \multicolumn{1}{c|}{\textit{Rāga}} \\
\hline
Pancamam & Āhiri \\
\hline
Kausikam & Bhairavi \\
\hline
Naṭṭapāḍai & Gambhīra Nāṭa \\
\hline
Megharāgakurinji & Nīlāmbari \\
\hline
Takkesi & Kāmbhoji \\
\hline
Paḷampanjuram & Śankarābharaṇam \\
\hline
Sādāri & Pantuvarāḷi \\
\hline
Sevvāḷi & Yadukulakāmbhoji \\
\hline
Gāndhārapanchamam & Kedāragouḷa \\
\hline
Viyāḷakkurinji & Saurāṣṭra \\
\hline
\end{longtable}

The following references with regard to melody (rāga) and rhythm (tāḷa) confirm undoubtedly that the ancient Tamil music was essentially sacred, devotional and deeply connected with as well as inspired by the Vedic culture.

\begin{myquote}
“Sundaramoorthy Nayanar (9th century) in his (Tevaram) hymns pertaining to the shrine in Tiruvarur describes God as \tamil{“ஏழிசையாய் இசைப்பயணாய்...”} You are manifest through the seven notes; you are the resultant form of music…. Arunagirinathar in his Tiruppugazh song \tamil{காதிமொதி }has described God as dwelling in Raga – \tamil{ராகத்துரைவோனே}. Thyagaraja referes to Rama as Ragarasika i.e., one who delights in Raga.”
\end{myquote}

\begin{myquote}
“A Tamil writer Tunga Munivar (\tamil{துங்க முனிவர்}) emphasizes the intricate nature of the tāla system in the following stanza:
\end{myquote}

\begin{myquote}
\tamil{“தென்றல் வடிவுஞ் சிவனார் திருவடிவு மன்றல் வடிவு மதன்வடிவுங் – குன்றாத வெயினிசை வடிவும் வேதவடிவுங் காணில் ஆயதாளம் காணலாம்”}
\end{myquote}

\begin{myquote}
Meaning: “If one can see the form of the southern breeze, the form of Śiva, the form of scent, the form of Manmatha (Cupid), the form of the flute tone and the form of the Vedas, one can see the subtlety of the tāla”

~\hfill (Sambamoorthy (1) 2005: 14,15)
\end{myquote}


\subsection*{V. Musical instruments:}

The ancient Tamils used a wide range and variety of musical instruments. The ‘\textit{Tirumurai}’ dated between 6th to 11th century lists over 70 varieties of instruments used in ancient Tamil music. Nevertheless, as Sambamoorthy\index{Sambamoorthy, P.} (2006: 86) has pointed out, the three principal musical instruments of Tamils namely \textit{yazh, kuzhal} and \textit{maddalam} (\tamil{யாழ், குழல், மத்தளம்}) had their parallels in the celebrated vādya trayam; \textit{vīṇa, veṇu} and \textit{mṛdangam}. Poems of the Sangam literature contain numerous mentions of the various musical instruments such as the \textit{Seerkazhi}, a stringed instrument of the vīṇa type and various percussion instruments such as \textit{murasu} or \textit{muzham}.

In the realm of musical instruments, the following analogies can be seen.

\begin{enumerate}[{\rm 1.}]
\itemsep=0pt
\item \textbf{Classification of musical instruments:} Bharata’s \textit{Nāṭyaśāstra} classifies musical instruments or \textit{Vādya}-s into ‘\textit{Tata}’ (Stringed instruments), ‘\textit{Suṣira}’(Wind instruments), ‘\textit{Avanaddha’ }(Percussion instruments made of animal skin) and ‘\textit{Ghana}’(Solid instruments made of metal). (Sambamoorthy (1) 2005:111) The same classification can be seen in \textit{Silappadikāram} which says musical instruments are of four types namely ‘\textit{Tolkaruvi}’ (skin instruments viz. percussion), ‘\textit{Tulaikaruvi}’ (holed instruments viz. wind instruments), ‘\textit{Narambu\-karuvi}’ (stringed instruments) and ‘\textit{Kancakaruvi}’ (gongs and cymbals).

 \item \textbf{Musical ensemble\index{Musical ensemble} or orchestra:} ‘\textit{Kutapa}’ (\dev{कुतप})is an orchestra or \textit{Vādyavṛnda}. According to the number of performers in these bands, \textit{Kutapa}-s were classified into \textit{Uttama, Madhyama} and \textit{Kaniṣṭaka}. Then there were ‘\textit{Tata kutapa}-s’ (band of stringed instruments), ‘\textit{Avanaddha kutapa}-s’(band of drums), etc based on the type of instruments used. There were also singers in the \textit{Kutapa}-s. ‘\textit{Meḷam}’ (\tamil{மெளம்}) the Tamil word corresponding to \textit{Kutapa} in Sanskrit. Meḷam denoted a group of musical instruments played together. It provided a collective music. There are six varieties of melam namely \textit{Periya Meḷam}(Nagaswaram group), \textit{Chinna Meḷam} (Bharatanātyam group), \textit{Sangīta Meḷam} (Classical music orchestra), \textit{Naiyāndi Meḷam} (Folk band), \textit{Urumi Meḷam} (Rustic band) and \textit{Muzhavu Meḷam} (band of musical instruments including the muzhavu (\tamil{முழவு}) drum). (Sambamoorthy (2) 1998:118,120,121)

 \item \textbf{Yazh and Veena:} The vīṇā is the chief instrument of Indian classical music and its origin can be traced back to the Vedic period where it was said to be the main accompaniment for recitation of Vedic \textit{mantra}-s. The word vīṇā occurs in the Vedic literature and in also the smṛti of Yāgñavalkya. \textbf{(\dev{वीणावदनतत्त्वज्ञःश्रुतिजातिविशारदः। तालज्ञश्चाप्रयासेनमोक्षमार्गंनियच्छति~॥}).}

\end{enumerate}

The corresponding stringed instrument used by ancient Tamils, which resembles the vīṇā is the ‘\textit{Yāzh}’. It was used as a primary instrument and also as an accompaniment to vocal music. But it is worth observing that the name of this instrument is inspired by Sanskrit and its structure by the vīṇā.

\begin{myquote}
“Since the tip of the danḍi or the stem of this instrument was carved into the head of the weird animal \textit{yāli} (\textit{vyāla} in Sanskrit), it was named \textit{yāzhi} or \textit{yāzh}….The head-piece of yāzh at the tip of the danḍi is retained even now as the head piece in the veena. The yāzh instrument is spoken of in glorious terms in the Tamil epic Silappadikaram.”

~\hfill (Sambamoorthy (2) 1998:159)
\end{myquote}

It is also interesting to note that during the time of Māṇickavāsagar, the vīṇā and the yāzh were both in vogue. This is proved from the statement ‘\textit{Innisai vīṇaiyar yāzhinar orupāl}’ (\tamil{இன்னிசைவீணையர்யாழினர்ஒருபால்})in the \textit{Tirupalliyezhuchi} of \textit{Tiruvembavai}. Appar also\break refers to the vīṇā in his hymn ‘\textit{Māsil Vīṇaiyum}’ (\tamil{மாசில்விணையும்}).In \tamil{தனித்திரு விருத்தம்}, Appar says in stanza 7, that when the world comes to an end and is overrun with the waters of the ocean, Śiva sits on the water surface, and delights in playing on the vīṇā \tamil{“எம்மிரைனல்வீணை வாசிக்குமே.”}


\subsection*{VI. Integration with other states}

While some narratives are aimed at creating a separate and isolated identity for Tamil Isai disconnecting it from the rest of the country, there are certain others which are aimed at claiming absolute ascendency and altogether negating the role and contribution of any other vernacular or tradition in the making if what we see today as an evolved system of music. The Wikipedia page on Ancient Tamil Music categorically calls it “the historical predecessor of the Carnatic music during the Sangam period.” Such exclusivist ideology is also reflected in the quotation below.

\begin{myquote}
"The south Indian music system, which was indeed Tamil Pannisai, was erroneously named, for the first time, Karnataka sangeetham in the 12th century by a western-Chalukya king, Someswara Bhuloka Mamalla, in his ‘Mānasoullasam', a monumental work that dealt with all the subjects under the sun, including music. In no other language in India, there existed at that time Sāhitya-s (musical compositions) as they did in Tamil. Though most of the music manuals written from the 9th century onwards were in Sanskrit, the source materials for them — like the varieties of ‘rāga-s' (pann) they had mentioned in their works — were all associated with the Tamil literary works, like ‘Silappadikāram', ‘Thevaram', and ‘Nālāyira Divya Prabandham'.."\hfill (Arunachalam)
\end{myquote}

Firstly, it is important to understand that the word ‘\textit{Karnāṭaka}’ in the nomenclature is quiet ancient and does not have anything to do with the present state of Karnataka, which was in fact formed only in 1956 (and named later in 1973) with the passage of the States Reorganisation Act, 1956. Scholars like Sambamoorthy and Kuppuswamy give evidential information in this regard.

\newpage

\begin{myquote}
“…There was a time when a single system of music prevailed throughout the length and breadth of India. The division into North Indian (Hindusthani) and South Indian (Karnatic) systems came later on and became more pronounced during the reign of the Moghul Emperors in Delhi. We come across the terms: \textit{Karnatic Music} and \textit{Hindusthani music} in the work, \textit{Sańgīta Sudhākara} of Haripala, written between 1309-1312 A.D:-
\end{myquote}

\begin{myquote}
\dev{तथापिद्विविधंज्ञेयंदक्षिणोत्तरभेदतः।}\\\dev{कर्णाटकंदक्षिणेस्याधिंदुस्थानियथोत्तरे॥}

~\hfill (This shloka is quoted in the \textit{Abhinava rāga mañjarī})
\end{myquote}

\begin{myquote}
…. The first work to mention the word ‘Karnataka’ is the \textit{Brhaddesi}. It mentions a desi raga by name Karnata (verse 375) Sarngadeva speaks of Karnataka music and dance material. He speaks of Karnata bangala, Karnata gaula. Nanyadeva in his \textit{Bharata Vartika} mentions the word Karnataka. The term \textit{Karnataka} is referred to in the \textit{Bhagavatam} and in \textit{Bhoja Champu}.
\end{myquote}

\begin{myquote}
In Tamil, Karnataka means tradition, purity, sampradayam and suddham. People who are wedded to ancient ways of living are even now referred to as \textit{Karnataka manushyas}.”\hfill (Sambamoorthy 2005: 8,9)
\end{myquote}

\begin{myquote}
“Scholars affirm that at one time (ancient) we had but one music system, which in course of time branched off into two known today as the Northern (Hindustāni) and Southern (Carnātic) schools of Indian music. The number of works on music in Sanskrit dominate the field. A handful of the ‘ancient works’ are extant. Most of these available today came to be written after the tenth century A.D. A majority belong to the fifteenth and later centuries and their authors hail from South India. Many of the dictates enumerated in the works survive in the Tamiḷakam. In as much as these works belong to a period prior to the Muslim infiltration into the South (where the Muslim influence is not pronounced) they are relatively indigenous. The favourable environment perhaps helped the consolidation of the system to earn the stamp ‘Carnātic’. The ‘Carnātic system’ was adopted in Andhra, Karnataka, Tamil Nadu and Kerala.
\end{myquote}

\begin{myquote}
Catura Kallināta in his commentary on Sangīta-Ratnākaram describes Karnātaka-deśa as lying between the Kāveri and Kṛṣṇā rivers and hence the name ‘Carnātaka-Sangīta’. Philosophical out-look would not permit the compartmentalisation of music exclusively as Northern or Southern, but what is applicable is that, at a given time a particular music-system is given preference in a particular area.”\hfill (Kuppuswamy 1992:iii)
\end{myquote}

Hence, the term ‘Karnataka’ is used in the dual sense of ‘old’ and ‘the southern part of the Indian Peninsula’. In fact, the British used the word ‘\textit{Carnātic}’ or ‘\textit{Karnātic}’ to describe peninsular India, south of the Krishna river and that’s how ‘\textit{Karnāṭaka-Sangīta}’ came to be called as ‘Carnatic Music’. 

Secondly, it would be utterly wrong to say that “In no other language in India, there existed at that time Sāhityas as they did in Tamil”. References to musical compositions are contained in the \textit{Veda}-s, \textit{Upanishad}-s, \textit{Rāmāyaṇa, Mahābhārata} and the \textit{Purāṇa}-s. Also, we have enough references in Bharata’s \textit{Nāṭyaśāstra} and many other ancient texts on musicology talking of different types of musical compositions. Bharata has actually devoted one entire chapter (Chapter 32 – \textit{Dhruvādhyāya}) to explain countless varieties of Dhruva songs in over 500 shlokas. Even the famous ‘\textit{Abhijñānashākuntala}’ of the great poet like Kālidāsa has references to songs called \textit{Gīti}-s. Then came the concept of a ‘\textit{Prabandha}’ – a well defined and structured musical composition, of which the ‘\textit{Gītagovinda}’ of Jayadeva is a classic example. Sambamoorthy (2) 1998: 37,38), observes that formerly Sanskrit was used for the sāhitya of musical compositions. Later on Bhānḍīra (a kind of Prākrit) was used. Still later, compositions came to be written extensively in the vernaculars and bhānḍīra fell into disuse. According to Viśveśvara, who has written a treatise on the grammar of this language “Bhānḍīra is best suited for music, having had its origin in the melodious medley of the lyrical notes that arose when Śrī Kṛṣṇa danced with the flute in his hand, in the company of the Gopī-s of different countries and tongues. This language is thus a creation from \textit{Kāmbhoji, Māgadhi, Gauḍi, Mahārāṣṭri, Kāḷingi} and \textit{Gairvāni}, with a sense to its potency for rich musical effect.” Bhānḍīra received an impetus for development from the time of Bhoja and Someśvara. Some lakṣaṇa gīta-s and some of the gīta-s of Purandara Dasa are in Bhānḍīra.

Thirdly, if the Sanskrit music manuals written after 9th century have their source materials associated with Tamil literary works, why is there an absence of any mention or acknowledgement of the source? Also, what about the Sanskrit music manuals written prior to 9th century which talk about \textit{grāma}-s, \textit{mūrcana}-s, \textit{jāti}-s and \textit{rāga}-s?

Thus, we see that the language or music of Tamil do not deserve the to be called the ‘one’ and the ‘only’. As Shulman (2016:8)rightly observes, “Nor was Tamil ever the sole or, for that matter, even the clearly predominant language of the south Indian civilization that I am referring to. It shared pride of place with other languages such as Sanskrit, the various Prakrits, Telugu, Kannada and Malayalam.”

Nagaswamy (2017: xiii, xiv, xv) has elucidated the contribution of the great Tamil savant and one of the greatest Tamil literary luminary, Dr. U.V.Swāminātha Iyer(1855-1942), who is said to have prepared a list of about 400 great musicians of the 19th and early 20th centuryirrespective of caste or creed of which more than 250 musicians were Vedantic Brāhmaṇs also called Smārta-s (Aiyar-s, Bhāgavata-s, Śāstri-s).Some of the most famous poet composers Subrahmanya Bhārati, Gopālakṛṣṇa Bhārati, Tyāgayyar, Aruṇāchala Kavirāyar, Girirājakavi, Vīrabhadrayyā, Kavi Kunjaram Ayyar and others, feature in the list. The Aiyar-s were scholars who had mastered the Vedas upto the end of \textit{Gaṇa}-s (\textit{gaṇāntam adhyayanam}). The Bhāgavata-s were also well-versed in the Vedas but specialized in singing and dancing especially the Kṛṣṇa Līla of Bhāgavatam and Rāmāyaṇam, and contributed to the art of Bhajans. The Śāstri-s were also Smārta-s, of the Vedic learning but seems to be immigrants from Telugu region. There were others like Gopaṇṇa who were Kannaḍiga immigrants from Mysore and Bangalore who amounted to almost 300 musicians of that time and were Smārta Brahmaṇs who made the Tamil country reverberate with classical music. The present classical music of Tamil Nadu is based mainly on two important Sanskrit treatises, one written by Śārṅgadeva who hailed from western India and another by Venkaṭamakhin who was from Karnataka.

These facts should prove sufficiently the enormous role of Vedic Smārta Brāhmaṇs particularly migrating from other regions, in the field of music and dance of Tamil Nadu and also the perennial relationship of Tamil Nadu with other South Indian states in this respect.


\section*{Conclusion:}

Thus, it is manifest at every point that the Vedas are indeed the bedrock of Tamil Isai and that Tamil musical literature is profuse with tenets of Vedas and Upanishads. All this eventually leads us to the conclusive congruence between the culture of Tamil Nadu and that of the rest of India.

\begin{myquote}
“The Chera, Chola and Pāṇḍya kings performing Vedic sacrifices frequently are proof enough, if required, to the depth to which Vedic ideals were part of earliest strata of Tamil society. In birth rites and death rites, sacrifices reflect the Vedic tradition. The judicial system, administrative system, poetic conventions, arts, music, dance, etc fully reflect the Vedic traditions. As established in the earliest recorded poems like \textit{Puranānūru}\index{Purananuru@\textit{Puranānūru}}, there is no evidence to prove isolated independent Tamil culture from the Sangam age. Therefore there is continuity from that age for the past two thousand years. Tamil developed and grew in strength as a result of this fusion can be seen century after century. That is how every other region of India also developed. The so called Dravidian linguistic theories are not based on chronological analysis of facts available but are only wishful speculations.”\hfill (Nagaswamy 2017: x)
\end{myquote}

We are also reminded of the saying by the celebrated Tamil poet Subramanya Bhārati -,\tamil{“செப்பும் மொழி பதினெட்டுடையாளெனில் சின்தனை ஒன்றுடையாள்!!”} This idea of cultural and spiritual unity of India should help us save ourselves from the disaster brought about by the deadly mixture of colonial racial categorization of the Indian population and the deconstruction of India’s religions and spirituality.


\section*{Bibliography}

\begin{thebibliography}{99}
\bibitem{chap8-key01} Acharya, K.Vrinda (2011) “Karnatak Classical Music: for Entertainment, Refinement and Enlightenment”\textit{Foundation for Indian Scientific Heritage - Decoding Veda Vidya and Traditional Resources for Rediscovering Indian Scientific Heritage} \url{www.kaumaaram.com}

 \bibitem{chap8-key02} Acharya, K. Vrinda (2017) \textit{Karnāṭaka Saṅgītadalli Trimūrtipūrva Racanegaḷu (Kannada)} Bengaluru: Karnataka Sangeetha Nrutya Academy.

 \bibitem{chap8-key03} Arunachalam, Mu.\textit{Tamizh Isai Iḷakkiya Varalāru. (Tamil)}

 \bibitem{chap8-key04} Ayyangar, R. Rangaramanuja (1972) \textit{History of South Indian (Carnatic) Music: from Vedic} \textit{times to present}. (Self Publication)

 \bibitem{chap8-key05} Kingsbury, F. \& Phillips, G. E. (1921) \textit{Hymns of the Tamil Śaivite Saints}. London: Oxford University Press.

 \bibitem{chap8-key06} Kothandaraman, S. (Translated by Ponnuswamy, S.) (2015) \textit{Vedas and Saivam}. Chennai: Giri Trading Agency Pvt. Ltd.

 \bibitem{chap8-key07} Kuppuswamy, T.V. (1992) \textit{Carnatic Music and the Tamils}. Delhi. Kalinga Publications.

 \bibitem{chap8-key08} Malhotra, Rajiv (2011) \textit{Breaking India}. New Delhi: Amaryllis.

 \bibitem{chap8-key09} Nagaswamy, R. (2017) \textit{Tamil Nadu - The Land of Vedas}. Chennai: Tamil Arts Academy.

 \bibitem{chap8-key10} Rangacharya, Adya (1996) \textit{The Nāṭyaśāstra English Translation with critical notes}. New Delhi: Munshiram Manoharlal Publishers Pvt. Ltd.

 \bibitem{chap8-key11} Sambamoorthy, P. (2005, 1998, 2008, 2010, 2006) \textit{South Indian Music} (Books 1,2,3,4,6). Chennai: The Indian Music Publishing House.

 \bibitem{chap8-key12} Sampatkumaracharya, V.S \& Ramaratnam, V. (2000) \textit{Karnāṭaka Saṅgīta Dīpike} (Kannada) Mysuru: D.V.K. Murthy

 \bibitem{chap8-key13} Sastri, S.Subrahmanya (1943) \textit{Saṅgītaratnākara of Śārṅgadeva (Vol 1).} Madras: The Adayar Library.

 \bibitem{chap8-key14} Shulman, David (2016) – \textit{Tamil: A Biography}.Cambridge, Massachusetts: The Belknap Press of Harvard University.

 \bibitem{chap8-key15} Subramaniam, Lakshmi (2004) “Contesting the Classical: The Tamil Isai Iyakkam and the Politics of Custodianship”\textit{Asian Journal of Social Sciences} 32:1. pp 66-90.

 \bibitem{chap8-key16} Terada, Yoshitaka (2008) “Tamil Isai as a challenge to Brahminical Music culture in South India”\textit{Music and Society in South Asia: Perspectives from Japan}. \textit{Senri Ethnological Studies }71. pp203-226.

 \bibitem{chap8-key17} “Tamil Nadu The Land of the Vedas by Dr. R. Nagaswamy (New book)” \url{http://tamilartsacademy.com/}. Accessed on 3 Aug, 2017 “Literature – Light and Delight. Brahmins in Sangam Literature-5 Purananooru” (last modified on 20 Oct, 2014) \url{http://kauniyansri.blogspot.in/2014/10/literature-light-and-delight-brahmins_20.html}. Accessed on 12 Aug, 2017

 \bibitem{chap8-key18} “Vedanta Desika’s Sri Ranganaatha Paaduka Sahasram in Sanskrit” \url{http://www.prapatti.com/slokas/sanskrit/paadukaasahasram/samaakhyaa.pdf.} Accessed on 20 Aug, 2017

 \end{thebibliography}

\theendnotes

