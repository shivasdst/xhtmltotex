
\chapter*{}

The I took the Hereculean task of translating adverbation this sacred ancient text.

I cannot really imageine why I undertook this project. First and formost, I profusely thank my guide Smt. Mangala Nagraj who has my mentor for the last fifteen years. Her knowledge of the Kannada language has enabled me to compile this text in English.

My source of inspiration perhaps may have been Major Ramanujam who is an exponent of the Vedas and the Upanishads but he does not know Kannada. Smt. Mangala has been an ardent devotee of her Guru and holds him in high esteem and extreme reverence and often quotes his words while teaching us in class.

This work has given me immense pleasure and an opportunity ton test my skill and discipline.

I will be failing in my duty if I do not extend my gratitude to our senior student Smt. Sharadamma an eighty year old lady who handed over this 1927 edition of, “Upanishad Sara Ratnavali” written by Mysore Shivananda Subramanya.

I hope this effort will benefit the spiritual seekers and the larger public in general.

I have put in my best efforts in the translation with my limited knowledge of Kannada and the surpictures

\begin{flushright}
\textbf{Nalini Krishnamurthy}
\end{flushright}

\chapter*{}

\begin{center}
\dev{ॐ}\\ Namah: Sri Sachidananda Bhargava
\end{center}

This is a composition written by Sri Madhvashankara Bhagwatpada\-charya.

It is Paneha Stotra along with Upanishad Sara Ratnavali By Mysore Shivananda Subramanya Written in Kannada.

The owner of a small shop Sriyutha \dev{॥} Sri Bandikeri Ramaya has been of immense help in publishing this sacred book and is printed by Mysore K. M. Narasimhaya's Press. This Grantha is Sarvatantra

\begin{center}
1927\\ All Rights Registerd.
\end{center}

\chapter*{Avatarnikey}

Srimad Khanda Sachidananda is a disciple of Brahma Swarupa Pothluri Veera Brahmananda Aproksha anubhavi Vakula Bharna Paradeshi.

For the welfare of the Andhra Pradesh Public, this well-known scholar has done yoeman service by this Grantah.

The Mumukshus of Karnataka ardently requested this grantha ‘Upanishad Sara Ratnavali’ need to be translated in Kannada language.

This Grantha is a dialogue between Sri Rama and his devotee Sri Hanuman.

\chapter*{[SARVAM KHALVIDAM BRAHMA]}

Sloka:\dev{॥}

\begin{verse}
 Atmanatma Vivekena Punasamsara Nivrith \dev{।}\\
 Tadirna Janma Kotyapi Bandhachledho Na Sidhiyeti \dev{॥}
\end{verse}

Those who do not have the viveka or a sense of discexament even if they take birtha even a crore times will not attain moksha. Therefore, “Anthatatmana Manjanadristva \dev{।} Brahamantimudha girigahavrishu” as per the above, one must analize what kind of a body it is and the atma which resides in it.

‘Aaya Matmabrahma \dev{।} atma Narayana Para’. As per this Shruti, Atma itself is Parabrahma, and the one who experiences this atma, even if he goes to vanavasa and stands upside down, and does Tapas and however much he may undergo the hardships as also foregoing food and water he will not achieve Mukti.

\begin{center}
(Agnanam Bhavanarthaya Pratima Parikalpika).
\end{center}

The one who is ignosant needs to do idol worship, those who are Atma-Gnanis do not need this idol worship.

\begin{verse}
 “Deho Devalaya Proktojivo Devasnatana \dev{।}\\
 Tyajeda Gnana Nirmalyam Soham Bhavene Pujeyeth \dev{॥}”
\end{verse}

The one who realises that this body itself is a devalaya or a temple and the atma which dwells within is the Parabrahma and who understands this Brahmatma and becomes one with it, they will be free from the cycle of rebirth, become Nitya Muktas and get Moksha. However, those who undertake physical and mental hardships are incapable of attaining Mukti.

\newpage

Hey! Priya Sahodaras, all of you please read through this whole Grantha. Please try to comprehend the various Upanidic Vakyas and I request that only Brahmdaivata Gnana is all in all the sum and substance, I sincerely pray to all of you to understand.

\vskip 2pt

Thus, the author of this Grantha's request, the Mumukshus will not delay in reading this sacred treatise. This grantha is concise and is small but the contents are amogha or wonderful and its usefulness to all the readers is of the highest order and can be understand even to a non-vedic or the common man even to the Mahatmas will have an appeal to this book.

\vskip 2pt

This Grantha contains (44) subjects. Everyone of them without fail has a beauty of its own. I will show this in the next Vishayasuchaka Patrika.

\vskip 2pt

I offer my gratitude to Sri Bandikeri Karannara's children Rajeyri and also the owner of a small shop Ramaiah who contributed financially thus enabling the publication, and to the Granthakartaru and to his family the blessings of Sri Sachitananda swarupa Parmeshwara to shower his good wishes to them.

\vskip 2pt

I humbly apologise if there have been any lapses in this grantha and request the mahapurushas who read this to overlook them.

\vskip 5pt

\textbf{An announcement:–}

\vskip 2pt

I would like to make it clear, that in this grantha, the notes and meanings written in Kannada if by any chance if any one tries to pilfer the inner meanings of Srimad Bhagwat Gita, the Vedanta Granthas' notes and This will facicilate you to see the subjects you need very soon.

\vskip 2pt

In this Grantha, I have included Sri Shankaracharya's shlokas: viz, (1) Prathasmarana stotra (2) Manusha Panchaka (3) Ekashloki (4) Kaupina Panchaka (5) Sadhana Panchaka.

\vskip 2pt

For all the above shlokas, notes in Kannada have been adequately provided. These will immensely help the Mumukshus in their quest for Atma Sandhana which will prove very useful. The meanigns of this stotras will comfilement those stated in the various Upanishads.

\vskip 2pt

This utmost sacred grantha is authored to benefit humanity I offer my heartfel gratitude to Sriyuta Sadvi Charparragiddu and explanations, they will be subjected to losses. We have registered all copy rights with the Government, and if anyone claims the contents or the quotations as their own will not be permitted even extracts of the book. Even for printing they need to take the permission of the Samiti.

\chapter*{\dev{रामाष्टकमः~॥}}

\begin{verse}
 \dev{भजेविश्ष सुन्दरम समस्त पापा खण्डनम~।\\
 स्वभक्तचित्ता रन्जनम सध्यैवरामद्वयम~॥}
\end{verse}

\begin{verse}
 \dev{जटाकलापशोभितं समस्त पापा नाशकम~।\\
 स्वभक्तभीती भन्जनम भजेहारा ममद्वयम~॥}
\end{verse}

\begin{verse}
 \dev{निजास्वरूपा भोधकम कृपाकरम भयापहम~।\\
 सममशिवम निरन्जनम भजेहरामामद्वयम~॥}
\end{verse}

\begin{verse}
 \dev{सप्रपन्चकल्पीसम थ्हानामरूपवास्तवम~।\\
 नीराकृतीम नीरामयम भजेहराममद्वयम~॥}
\end{verse}

\begin{verse}
 \dev{निष्प्रपन्चनिरवीकल्प निर्मलम नीरामयम~।\\
 चीदेकरूपासन्थतम भजेहरामाममद्वयम~॥}
\end{verse}

\begin{verse}
 \dev{भवाथभी पोतारूपकम हैशेषकल्पतम~।\\
 गुणाकरम कृपाकरम भजेहरामाममद्वयम~॥}
\end{verse}

\begin{verse}
 \dev{महाक्क्यभोदकै वीराजामानावाक्पधै\\
 परभघ्म व्यापकम भजेहारामाममद्वयम~।\\
 शिवाप्रदम सुखप्रदं भवछीधम भ्रमापदम~॥}
\end{verse}

\begin{verse}
 \dev{विराजामानदैशिकम भजेहारामाममद्वयम \\
 रामाष्टकम पठीथम सुखरम सुपुण्यम \\
 व्यासेनभाषीतमिदँ षृणुतेमनुष्यः~।\\
 विध्याम श्रीः विपुलसैख्यमनन्तकीवीतम\\
 सः प्राप्यदेहविलये लभतेचमोक्षम~॥}
\end{verse}

\begin{center}
\dev{इति श्री व्यासामुनीद्र विरचीतम\\ रामाष्टकम सम्पूर्णमः}
\end{center}

\chapter*{SRI PARMATHINE NAMAH:\\ Tatva Masya Astakam}

\begin{verse}
 NamaRupaDayo Yatra Kalpi Shashitha Vebrahmath.\\
 Vyo Minn Vanagaram Yat Brahma Tatvamasi Dhruvam.
\end{verse}

\begin{verse}
 Dra astu Darshonaya Mabhavo Yatra Drishythey.\\
 Yadey Deva Kevalam Shudam Brahma Tatvamasi Dhruvam.
\end{verse}

\begin{verse}
 Antha Purnam Bhi Purnam Svayamadvyamkriyam.\\
 Yadanantham Parenjyoti Brahma Tatva Masidhruvam
\end{verse}

\begin{verse}
 Yadgrihangraha Kalinan Bhaskam Tatvi Lakshanan.\\
 BhaMatram Perithaha Purnam Bhrama Tatvamasi \break Dhruvam.
\end{verse}

\begin{verse}
 Bhamnabhasitam Roopam Yatha chakshu Thstamanah.\\
 Yenabhathi Midamveti Brahma Tatvamasi Dhruvam
\end{verse}

\begin{verse}
 Sakridoibhatam Puratasarve Smatsu Vibhatiyathi:\\
 Yadbhana Manu Bhatidam Brahma Tatvam masi dhruvam
\end{verse}

\begin{verse}
 Darshanam Sravanam Gnanam Yatra Namasya Vidyathey\\
 Sukshma Mapya Padeshyam Yat Brahma Tatvamasi \break Dhruvam
\end{verse}

\begin{verse}
 Nishkalam Nishprikyam Shantam Nirvadyam Niranjanam\\
 Shuddha VignanaManandam Brahma Tatvamasi dhruvam.
\end{verse}

\begin{center}
Ethi Sri Madacharya Virachitam Tatvamasi dhruvam. \\ Sampurnam
\end{center}

\chapter*{Atma Purana}

This grantha has been written in Kannada completely. 250 people who came forward with donation, the publication of the book was undertaken. The donation was between Rs.1/- to Rs.15/- that was the decision taken. The rule first they had to deposit Rs.10/- The first half of the book was completed then the remaining Rs.5/- should be given.

When the second part was printed and completed and bound then it will be distributed to the Donars. Those who send the amount through mobile with their address, by December of this year, then the book will be supplied.

\begin{center}
Grantha Karta\\ Shivananda Subramania (Vedanta Subbiaha)\\ Lakshmi Vilasa, Agrahara\\ Mysore.
\end{center}

