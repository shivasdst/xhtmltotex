
\chapter*{Sri Ganapathy Sharada Gurubhyo Nama Upanishad Sara Ratnavali}

\begin{center}
\textbf{Sri Ramachandra Stuti}
\end{center}

Shloka:–

\begin{verse}
 Sri Ramam Tri jagathgurum Suruvaram Sita Mano Nayaham \dev{।}\\
 Shyamgam Shashikoti kanthi Vadanam Chanchatkala Kaustubham \dev{॥}\\
 Saumyam SatvaGunotham Susuryoterey Vasanta Prathum Trataram
\end{verse}

\begin{center}
Sakalartha Siddhi Sahitam Vandey
\end{center}

\begin{center}
\dev{रघुणाम पतीम~॥}
\end{center}

\begin{center}
\textbf{Sri Seetha Devi Stuti}
\end{center}

Shloka:–

\begin{verse}
 \dev{माण्क्या मन्दीरा} Padaravindam\\
 Ramarkasamphula Mukharavindam\\
 Bhakta Pradhantritvakara Vindaam.\\
 Devim Bhajey Raghava Vallabham Taam-
\end{verse}

\begin{center}
\textbf{Anjeneya Stuti}
\end{center}

\begin{verse}
 anjeneya Mati Patalananam \dev{।}\\
 Kanchanadevi Kamaneeya Vigraham\\
 Parijatha Tharumula Vaaineum\\
 Bhavayani Pawana Nandanam
\end{verse}

Sri MadGuruSwamy has reflected in his mind.

Srimadnantha Nigamantha Vedya SatyaGnanananda Swarupa, who is Nitya Nirmala Sri Ramachandra Parabrahma has elaborately described the dialogue between Sri Ram and Hanumantha. This conversation is in the form of Vedasara Upanishad's.

Tatparya and madguruswamy combined all the above in a very easy way for the common man to comprehend; it is called “Upanishadth Sara Ratnavali”

It is our humble request that everyone's life should be enlightened the hearts of all.

