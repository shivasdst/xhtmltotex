
Upanishad's Sara Praramba

Sri Ramanjeneya Samvada

Shruti:–

Ayodhya Nagarey – Ramyey – Ratna Mandapa~। Madhyamey~। Sita Bharata Soumitri Shatrugnadhey – Samnvitam~॥ Sanakathey – Murni Garye – Vasista Dhyey Kadabhi~। Anyey Bhargavthyshvapi Sthumana – Mahrinisham~॥ Dhevikriya – Sahastrataam Sakshitam Nirvikaranam~। Swarupadhyama – Neeritam – Somadhi – Yerimay – Harim Bhaktya – Shushrushaya – Ramam Struva – Prapachha Maruti~॥

Tatparya:– In the Srimanaha Bhumandala which lights the Ayodhyanagre, on the throne carved lifth Navaratna golden Simhasana are seated Shri Sita Lakshmana Bharata Shatrugna along with Sri Ramachandra.

Respectful and revered Rishis like Sanaka, Sanandana, Vasista, Shuka, Parashara, Vyasa Maha Munis are all praising the Lord in various stutis. Sri Ramachandra is in deep Samadhi, who is the remover of mans' Papas and protects the Dehindra Anthakarnas, Tadvathis he is a witness in the form of Nirvikara Parabrahma Sarvashwarya is in a state of Svaswarupasandana Parmananda within himself.

Just when he was awakening from the restful Samadhi state, Parama Bhagvatha Shiromani Sri Anjeneya, doing the Padaseva of the Lard and ultiring his Stuties tells the following.

Shruti:–

(Rama – Tvam Parvinathmasi – Sachidhananda – Vigraha:~। Ida Neem – Tvam – Raghushresta – Pranamami – Muhuru – Muhur~॥ Tvadrupam –Gnathu Michhainu – Tatpatho Rama – Muktiye~। Aanya – sena – yonaham Muchhieyam – BhavaBandanath~॥ Kriyatha Kripeya – Vadamey – Ram – Yena Muktho – Bhavamyaham)

(Muktiko Upanishad)

Tatparya:– Hey, Ramachandraswami who is Sachidananda swarupa, who is the Parmatma, at every step I am doing Namaskaram.

I (Hanuman) dasa, please let me know God's Neejaswarupa, who can have a darshan of you, as also Neeja mukti – an easy path. Keeping your Kalaksha on me, I would like to have a full explanation. As Hanuman with immense Bhakti Bhaya pleaded the ever merciful Sri Ramachandra with a smile on his face states the following.

Pancha Vidha Muktis

Hey, Anjaneya the broad shouldered, Parakramashali, you are extremely Buddhishali. You have a question wherein the Devatas do not understand. Please listen with utmost Shraddha. There are 5 types (Pancha Vidha) Muktis are there:–

\item Salokya

 \item Saamipya

 \item Sarupya

 \item Sayujya

 \item Vidha Kaivalya

Shruti:–

Durachara Ratovapi Mannama – Bhajankapel Salokya – Mukti Manpnothi – Natu – Lokantharadikam~॥

Tatparya:– Hey, Anjaneya, the most Durachari, if he does sankeerthana of my name continuously – Viz.

‘Sri Rama Sri Rama Sri Rama, he will attain Salokya Mukti.’

He will not attain Karmanirmitha Nashvara, Swarga etc.

Salokya means, Sri Hariloka – to live there

Shruti:–

(Kashyanthi – Brahmanalesri – Mrithau – Mathura Mafenuyath.) Punaravrithi – Rahitam – Muktim, – Prapnothi – Manava~॥ Yatra Kutrapiva – Kashyam – Maraney – Samheshvara:~। Janthodradakshma – Karnethu – matharam – Samupadishith Nirdhutha Shesha – Papaugha – Matsya Rupyam – Bhajatya yam.

(Muktiko Upanishad)

Tatparya:– Hey Anjiniyaney. In Kashi Kshetra in the Pradesh of Brahmanda marnam (Pondvaru?) “Sri Rama that is the Taraka Mantra of mine whoever gets the above Upadesa that human without Punarvrithi – will attain mukti. In Kashikshetra for Panchakrosha can be obtained in any place, if a man gets marnam he will straightaway reach Kashi Vishveshwara – Paramashiva. This is so because he whispers in the right ear the Taraka Mantra ‘Sri Rama’ – Upadesa Due to this all the Papas are destroyed and he will attain my SARUPYA Mukti. Sarupya Mukti means, to adorn oneself the Rupa of Sri Hari and enjoy oneself with him”

Shruti:–

 (Sadachara – Rato –Bhutatwa – Dujo – Nitya – Manushyadin:~। Mayyi – Sarvatmake – Bhavo – Matsya Neeteyam – Bhajatye yam~॥)

(Muktiko Upanishad)

Tatparya:– Hey anjaneya~। A man will attain Samipya Mukti – if he practises according to the Veda Shastra Satkarma, Sadachara and one who does not have any desire other than ME and has Sarvasvarupa Bhakti and dhyana on ME will get Samipya Mukti – which means he will be near Sri Hari.

Shruti:–

(Gurupadista Maagona. Dhya Yanma Drupa – Manyam~। Matsya yujyam – Duja – Samyagbha Jeth – Brahmasahitavath~॥)

(Muktiko Upanishad)

Tatparya:– Hey Anjeneya! A person who listens to the Sadguru's sermon \& takes the path of Dyana on ME will defenitely attain SAYUJYA Mukti.

Sayujya means, with immense Bhakti that is, an insect who lives in a Brahmara nest, is afraid of the bee and keeps thinking of the Brahmara all the time and begins to feel that he is also a Brahmara. It means to become Aikya (one) with Sri Hari.

Shruti:–

(Naijya – Sayujya – Mukti Syath – Brahmanda kau – Shiva Chaturidhata – Sa Mukti – Mardulipasanaya Bhaveth~॥)

(Muktiko Upanishad)

Tatparya:– Hey anjaneya! Out of the 4 Muktis – Sayujya Mukti is consulered as the most Shresta – It makes one get Brahmananda. It is Mangalahara Sri Rama says those who understands my Sahara Rupa with utmost Bhakti \& Upasana as per three level of Parapathe through Salokya, Samipya, Sarupya \& Sayujya variations of Mukti will achieve them.

Shruti:–

Kaivalya – Muktiraikavya – Parmarathika – Rupiney~। Gnanam – Labhdva – Chiradeva – Mamakam – Dhama – Yasasyi 

(Muktiko Upanishad)

Tatparya:– Hey anjeneya! (Out of all the Muktis the most Shresta Mukti is — Videha Kaivalya Mukti. This is my real Swarupa. This Mukti will real one to MOKSHA. This can be attained only to those Brahmanatma ************* Gnana, for ordinary people ***** will not get.)

Therefore, anjeneya, only if you ***** this ****** ******* Gnana can you attain *** Kaivalya Mukti.

Shruti:–

Astotara Shathopanishadeem Vidhirada Ditya Shravana Manana Nidhyadhasanami Nairanthrainena Kritva~। Prarabdha Shkaya Dehatraya Bhangam Prapyopadhi Vininirmukta Ghanta kashvat Paripurnatha Methi Videha Mukti Sayaiva Kaivalya Mukti rithii~। Eti yeva Dheydanta Shravanadi Kritvatena Saka Kaivalyam Labhantey~॥ Atha Sarveysham Kaivalya Mukti Gnana Margeyonikta Nakarma Sankhya Yogapi Sanadibhi Rethyupanishad~॥

(Muktiko Upanishad)

Tatparya:– Hey anjeneya! Those who are keen or desire to get Moksha he need to study the (108) from a Sadguru. If should be done as a Yathra Vidhi method along with the extensive meaning. Then he should always be studying Vedanta, through Shravana manana Nidhiyasana. Then he will secure Brahmaadvaitya Gnana along with Brahmatmana Abheda Aikyanisandana. Then Prarabdha will be destroyed as also the body. Ghantaipadhi gets destroyed. That Ghantakasha Mahakashadnol has to be learnt so that he becomes one and his Atma (Jeevin) Shudha Nirguna, Nirakara Paripurna in Paribrahma. This oneness with the Parabrahma Swarupa is called Vidheha Mukti, as also Kaivalya Mukti.

Shruti:–

Gnanadevatu Kaivalyam – Prapyathey – Yena – Muchhaithey~॥, Tamevam – Viditwati Mrithyu Meti~॥, Namyai Pantha – Vidyateyanaya~॥, Gnanatwa Devam – Muchhaithey Sarvapashye;~॥, Vidyamrita Hridiyam Vishnu – Vishnobhdya Hridiyam – Shiva Yatham Taram – Napashyami Thatha Mey Srusti Rayushee~। Yathantharam Nabhey Dasyu – Shya Wakeshavoastha~॥.

I do my Namaskaram in Nishithi way to the Skandopa Vishnuiujaa as also to Shiva and Swarupa vishnu. Shiva's Hridaya is Vishnus. Vishnu's Hridaya is Shivas. I do not see any bheda between the two, but at the same time Iam extremely happy, and also pray for Dhirgayu. There is absolutely no difference between Sivakeshava. One who understands this and the two of them Harihara I who meditates on this is Dhomya. This is absolute Truth Saumya and Anjeneya please listen with Shradha:–

There is no Bedha between Hari Hara

Sri Ramanjeneyara Samvada

[Shruti~॥ Rama Yeva Param Brahma~॥.] According to Rahasyo Upanishad. [Shruti~॥ Shiva Era Svayam – Saksha Dayam – Brahma~॥] According to Mahopanishad, [Shruti~॥ Yohorai – Nrisimho – Devo – Bhargavwi~। Yascha – Brahma Yashcha – Vishnu:– Yascha – Maheshwara:~॥.] According to Narasimha Tapinya Upanishad [Shruti~॥ Yohavai – Rudra – SsaBhargava~। Yascha Brahma – Yascha Vishnu: Yascha Maheshwara~॥.] According to Atharva Shikho Upanishad [Shriti~॥ Tyam Yagna – Stavam – Vashastakara – Stva Mindra – Stvam – Rudra Stvam Vishnu – Stvam Brahma – Stvam – Prajapati~॥.] According to Taitreya Upanishad.

Hey Anjeneya! The debate on Vishnu is me. Parma Shiva both of them are similar and are one Parabrahma Swarupas. Therefore Vishnu and Parma Shiva both are me – I am Parma Shiva. There is absolutely no difference between us Saumya. 

[Shruti~॥. Shivaya – Vishnu Rupaya – Shivarupaya – Vishnuvey~। Shivasya – Hridayam – Vishnu – Vishmayascha Hridiyam–Shiva~। Yatham Taram – Napashyami Thatha Mey – Svasti Rayushi~। Yathantaram – Nabhedasya – Shichi Vakesha Vayo Statha~॥.] According to Skandoupanishad. I bow down and Namaskara to Vishnurupa Shiva and Shivarupa Vishnu. Shiva's Hridaya is Vishnu and Vishnu Hridaya is Shiva. I am Happy and contented to note that I do not see any difference between Shiva and Vishnu and let there be Dhirgeyu also. There is absolutely no difference between ShivaKeshava. The individual who meditates on HariHara both of us and consider them as one are really great. This is the truth Sawmya! and Hey Anjeneya listen with absolute attention to what I shall narrate in the coming chapters.

Brahma Swarupa Is of 2 Types

Shruti:–

(Dvey – Vidhey Veditavai – Parachaivapara Cha~। Atha Paraya – Ya – Tadakshara – MadiGamyate~॥)

Tatparya:– As per the Mundakaupanishad, I have 2 types of Swarupas. (1) Sakara. (2) Nirakara. In the Nirakara Rupa, Nirguna \& Paraventh is called. Apara and Saguna is known in Sakara Swarupa.

Shruti:–

(Yatadvai Satyakama Paranchaparancha Brahma – Yadomkara~। Tasmadvidura – Yeytinya Vayutaney – Naikatara – Manveti~॥)

According to Mundoka Upanishad

Shruti:–

(Yeta Dalambanam – Gna Tva – Yo Yadhichhati – Tasyatath~॥).

According to Khato Upanishad

Tatparya:– Hey Anjeneya! One who pursues Sakara Nirakara Swarupa any of the Vidva and does Dhyana on them will ultimately get the one he has chosen \& becomes one with us.

Shruti:–

(Thuriyam – Nirakaram – Brahma~॥)

Mahanarayana Upanishad

Tatparya:– Hey Anjeneya! My real Swarupa is Nirakara.

Shruti:–

(Yatha Dadrishya – Magrayya – Magodra – Mavarna – Machakshu – Shyo Tram – Tadavani – Padam – Nityam – Vibhoom – Sarvagatham – SusuSukshyam Tadaivam – Yadlohutha – Yonim Paripashyanti – Dhira~॥).

As per Mundoka Upanishad, my real Swarupa is not visible to the eye. You cannot guess this Swarupa either through Golsa, Varna eyes ears or through the touch of hands.

This Swarupa of mine is the origin of the Panchalohutas, it is there always is Vidhiparicheda, Sarvaparipurna Sarvavyapi can never be destroyed. That kind of my real Swarupa is being seen by Brahma Gnanis.

Shruti:–

(Mayinthu – Prakritim Vidyath Mayinantu – Maheshwaram~। Tasya Vayasa Bhuthaistu –Vyaptham~॥).

Tatparya:– According Swethatva Rupa Nishitha ways, Hey Anjeneya I am that Nirakara Parabrahma and I being Shabali – RupaNama Kriya, I am known as Sakar Brahma Maheshwara, Bhagavantha Saguna, Brahma and I keep performing the unverse's creation (Sristi) Sthithi Samhara.

Shruti:–

(Yathorna Nathisthantwacharey – Dyagney Kshudra – Visvulinga – Pachyrantheva – Mey Vasma – Datma – Sarvey Prana Sarvey – Loka – Sarvey – Devasarvani Bhutani – Vyocha Ranthi~॥).

According to Brahmadaranyako Upanishad

Just like the agni which has a raideance (Prajvala) and the sparks come out of it or like the weaver risech which takes out threads from its wasts Garbhah and again draws the threads within its Similarly, I Bhagawanta create the chara chara Prapancha and again I drew within me.

Shruti:–

(Upasakanam Karyatham – Brahmano – Rupakalpana~॥)

According to the Ramatapinya Upanishad

Rupa Nama Kriyas \& Nirakara Parabrahma myself being Maya Shabali take the form of Ishwararupa. Those Uparkara Bhaktas who are to be protected, I have taken the Rama Avtara from Dashratha Maharaja my real rupa is Nirakara Parabrahma. Due to my Rupa Nama imagined by Maya, this Avatara is Mithyarupa.

Shruti:–

(Tadeyjithi – Thannaijiti – Taddurey – Tadvadantike~। Tadantarasya – Sarwasya – Tadu Sarvasya – Bhayatha~॥).

Ishyavasa Upanishad

Tatparya:– Hey Anjeneya! Though, I am only Nirakara Parabrahma yet for those who are Tatva Vichara Shinyas, I have to be Maya Upadhi “Chalanaswarupa” that is, I consist of Deha \& Indriyas, hence I am the creator sthuthi Samaharaka, I need to do. But for those ‘Tatva Vicharavam Gnani's’ Iam, “Achalaswarupa” that is without body sense organs I exist as a Nirakara Chaitanya Swarupa. For those Pamaras Brahmagnis who through Shravana \& Manana, I remain faraway yet I appear very near to them.

For Sakala Jeev Jantus I remain in their inner bodies as a sole witness Gnanaswarupa within their Atmas. I also fill up this Maya Prapaneha.

Shruti:–

(Tasmatsya Karma Nityam – Nityam – Nirakaramithi~॥).

Mahanarayana Upanishad

My Swarupa as a Sakara Kriyadata is Anitya Swarupa Without Rupa Nama, Kriyas my Nirakara Swarupa is Nitya. Truly this is my real Swarupa.

Shruti:–

(Yatra – Namya Tatpasyati – Namya chabyanoti – Namya Dvijanati – Sabhuma~। Yo Vai – Bhuma Tada Vritam~॥ Atha – Yatranya – Tapashya Chabainoti – Anya Dvijanati~। – nya Tadlabdham – Atha – Yadlabdam – Tanmatyasam~॥).

Chando Upanishad

Shruti:–

(Sampurna – Manyat – Nyuna Manyat Sthanam~॥).

Bruhadaranyako Upanishad

Tatparya:– Hey Anjeneya! My Nya Swarupa Nirakara Swarupa is known as Bhuma i.e., Akhanda, Paripurna Amrita i.e., which cannot be destroyed (Nasha Sakara Swarupa is Alpa khanda Mithya swarupa ‘Mathya’ i.e., it can be destroyed, it is known as such.)

Shruti:–

(Yo Vai – Bhuma Tat Sukham Nalpey – Sukhamasti.~। Bhumaiva Sukham~। Bhumatva – Vijigna Sitavya – Sthithi~॥).

Chandogya Upanishad

According to the above Upanishad, Hey Anjeneya! “Bhumiyo” i.e., Akhanda Paripurna Nirakara Swarupa that is Niritishaya Sukha Swarupa. At “Alpa” i.e., KhandaSwarupa Sakara Niribhaya. There is no Sukha.

Since the Nirakara is Akhanda Paripurna Swarupa. Those who have Moksha. Apekshya, one should surely \& certainly accept the Nirakara Swarupa Saumyaney!

Meaning of Rama Shabda

Hey intelligent Anjeneya! Just listen with utmost Shradha. (Ramatiti Rama) I reside very happily in the hearts of Gnanis. Due to this I have got the name, Rama

Shruti:–

(Aadhyo – Ra – Tadpatharthayath – Makara – Sthvanpadarthava~। Tayosam Yojanamasityarathey – Tatva Vidho – Vidu~॥)

Ramarasyo Upanishad

Tatparya:– There are lakhs of meaning for the ‘RA’ – It is Nirakara Parabrahma – meaning. As regards ‘Ma’ is too has several infinite meaning. Therefore, ‘Rama’ means – The Nirakara Parabrahma remains as a witness in all the Jeeva Janthus. It also means that atmaswarupa in Brahmatmara Abhedhekya. Therefore, Hey Saumyeney! it means Dehasakshi – Atma is known as Rama.

Shruti:–

(Ramanthey – YoginoNante Nityanandey – Chidatmani~। Ithi – Ramapadenasau – Parabrahma Bhi Dheeyate~। Chinmaisya – Dirti Tiyasya – Nishkalasya – Sharirena~॥).

Ramatapinya Upanishad

Tatparya:– Hey Anjeneyane! my true Swarupa is meant only for those who are Satyagnasi Nitya Nirmala Nirakara Parabrahma Swarupa and are always, residing in my Atmaswarupa in ecstesy, I am known as ‘Rama’

My real Swarupa has been mentioned in the Vedanta. Therefore, one who has studied the Vedanta as length will understand my true Swarupa become Sakshakara and attain moksha.

What is Vedanta? Sri Ramachandra began explaining in detail. Hey Anjeneyane! listen with Shradha.

During the Pralayakala while I was lying on the Pepal leaf as Srimannasayana.

Shruti:–

(Nishyashava Sabhutha – Mey Visho Jrata – Veda – Suvistara~॥).

Tatparya:– From my Nishiryasa, the 4 Vedas have been imagined–kalpise RgVeda, Yajur, Sama \& Athavarna Vedas. In these Karma, Bhakti, Gnanas have emerged. I have revealed the above to the Loka (people).

Shruti:–

(Smriti Mameivagna~॥). Therefore, one should understand that from the Vedas are known as Aagna Vakyas. These Vedas are also known as “Shruti”.

Rg Veda – 31 branches.

Yajur Veda – 109 branches.

Sama Veda – 1000 branches.

Athavarna Veda – 50 branches.

Every branch has an Upanishad out of which 108 Upanishad are important or Shresta. If one at least receives the teaching from a guru with Parma Bhakti will attain Sayujya Padvi.

These are the following Upanishad

\item Aitreya Kaushiki, NadaBindu Atma Prabodha, Nirvana, Mudgala Aksha Malika Tripurasara, Soubhagya, Bhaichha – are the 10 main Upanishad in Rg Veda.

 \item Ishavasya – Brihadaranya, Jabala, Hamsa, Parmahamsa, Subala, Mantrika, Neeralamba, Trishiki Brahmana, mundala Brahman Advaitaraka, Paiangala, Bhikshuka, Turiyaathita, Achyatma, Tarasara, Yagnavalkya Shatyani, Muktiko Upanishad.

Shukla Yajurveda: 19 of them are important Kattavalli, Taitreya, Brahma, Kaivalya, Shweta, Shivatara, Garba, Narayana, Amritabindu, Amritanada, Kalagni Rudra, Kshuraka, Sarvasera Shukresya, Tejobindu, Dhyanabindu BrahmaVidhya, Yogatatva, Dakshinamurthy, Skanda Shararika, Yogashika, Ekakshara Akshi, Avadhutha, Khanta, RudraHridaya, Yogakundalini, PanchaBrahma, Pramagnihotra, Varaha, Kalisantarna, Saraswathi.

Krishna Yajurveda: It contains 32 Upanishads. The important ones are:– Kena, Chandaogya, Aaruni, Maitryani, Maitreyi, Vajrasuchi, Yogachudamani, Vasudeva, Maha, Samnyasa, Auyakta, Kundika, Savitri, Rudraksha, Jabala, Darshana, Jabali.

Samaveda: It contains 16. They are Prashney; Mundaka Mandikya Atharvashira, Atharvashikha, Hajjabala, Narasimhatapani, Naradaparivrajika Sharbha, Mahanarayana, RamaRahasya, RamapimShandalya Parihamsa Parivrajika, Annapurna Prasyu, Atma, Pashupata, Paribrahma Tripuratapana, Devi, Pavana, Bhasma Jabala, Ganapati, Mahavakya, Gopalapana, Krishna, Haigriva, Dattatreya, Garuda.

Athavarna Veda: Mandukya is the one to be tutored by a Guru and without hesitation will attain Moksha.

Shruti:–

Tileshu, Toilyavadveda Vedanta Supratistatha~॥

Tatparya:– Just as there is oil in Til (Yallu), so also in the Vedas the last Gnanakhanda the Gnana very firmly states the Vakyas are also known as Vedantas \& Upanishads.

Meaning of the Word Upanishad

Hey Anjeneyane! The word Upanishad has originated from the Sanskrit language (Shedhal) – It is a Dhatu.

Shadhal – ***sharuna – Gatya lasada neenu. Shadhal in Sanskrit Bhasha means destruction; going treuble or finishes off something. Hence, the Dhatu from Shadhal – Upanishad – it means the seed of Agnana, the tree of Samsara is to be destroyed. Aikya gnana should be sermonsed so that there is no birth \& death.

Sloka:–

[Upneya – Mahatmanam – Brahmopasti – Duyum – Tataha: Nihantya Vidyam – Tajjancha – Tasima Dupanishanma Tam~॥.]

It should be truthfully Vowed and Brahmavinta – what it is should be explained and this chase the Agnana away and teach Brahma Aikya is Upanishad.

Therefore, the substance or Sara of the Upanishads are collected and I shall teach about them. Please be very attentive.

Shruti:–

[Adhyaropapa – Vadata – Tasva Rupam – Nishcha Yekartam – Shashyate~। Tasmatsada – Vichareya Jugjeva Parmatmanah~॥.]

Tatparya:– In the Vedanta Vicharna Adyaropayukti and Apvadayayukti. These are 2 Yuktis Adyaropa Bhandavan Apavada serves as Moksha. If one reflects and contemplates on these aspects one gets to know the true Swarupa and becomes free. Due to this reason one has to always think about the Jagat, Jeevana, Brahma through the means of the afore said 2 Yuktis.

Adyaropa Yukti

To see a rope and assume that it is a shake the illusion is known as ‘Adyaropa’. In the same manner to see the Mayashakti and think that it is Shuddah Nirguna Parabrahma. Just like the rope the Jagat Jeeveshwaras is “Adyaropa” Reflect about this aspect.

Brahma Lakshana

Shruti:–

Ekameyva – Dvithiyam – Brahma

Chandogya Upanishad

Tatparya:– Parabrahma is one who is not Vijatheya, Swagata Bhedaga. Among the Hindus, in the caste System, there are so many diversions, if may not be considered so. This is because in one manusya Jathi there is Brahmana, Kshatriya Vaishya Shudras etc., these show divisions \& is known as “Sajitiya Bheda”.

Vijitaya Bheda: This means the differences between one Jati and the other.

This concludes this differences between Man, animals, Devatas as also between Man – JeevJantus, trees all are different from each other.

Swagata Bheda: The differents in the same item. For example, in a human body there is the head, Hands, feet eyes looks to be different within the while – body.

Here, only Parabrahma is one who does not have any of the mentioned bedhas. He is Nirakara, Nirvikara Parabrahma. That is – Brahma Vishnu Maheshwara, Devatas, Brahmana, Kshitray, Vasya, Shudra, Manavas, Pashu, Pakshi, Krimi, Keeta, trees, creepers as also the nose, eyes, ears, face are all not there in Parabrahma.

Shruti:–

[Satyam – Gnanam – Mamarthan Brahma].

Taitri Upanishad

Tatparya:– Satya and Gnana are Mayas Paribrahma.

Shruti:–

[Ashabdha – Masparsha – Marupa – Mavyam – Tatha – Rasam – Nitya – Magandha – Vachha – Yath~। Anadya Nantam – Matala: – Param – Druvam –Tadeva – Shishya – Malam Niramayam~॥.]

Paincolo Upanishad

Tatparya:– Parabrahma does not have an qualities like Shabda, Sparsha, Rupa, Rasa Gandha etc.,. It is industsable. He is there as Adhimadhyantas – This is the truth. Maya is Ignorance. Druvamadadu is Nirmala and Nirmaya.

Shruti:–

[Ajata – Mabhuta – Maprathri – astati – Maprana – Mamukha – Mashrotra – Mavagamso – Tejaska – Machakshuka – Magotra – Mashira – Mapari Pada – Masnigdha – Malohita – Mapra Meya – Maharsva – Madriga – Maninvanalpa – Mapara – Manirdeshyam]

(Subaropanishad)

Tatparya:– Parabrahma has no creation or Uthpati. He does want Prathiste or he is not air idol. Parabrahma does not have Prana, Vayu, face, ears, mouth, Mind, Biddhi, eyes, Name any of the above.

The Awayawas of KulaGotra, head, hands \& legs etc., he does not have. Neither does have any kind of completion like farwiss or black none of the Panchavasnas. He does not have any common murdane gnana. He does not possess any physical features like short, tall, slim fat etc., We cannot give him any description.

Shruti:–

[Paripurna – Mandhyanta – Maprameya – Mavi Vikrayam~। Sadvanam Chidvanam – Nitya – Mananda – Ghana Mavyayam~॥. Pratya Geka – Rasam – Purnam – Manantham – Sarvatho mukham~। Aheya – Manupadeya – Manadheya – Manashryam~॥ Nirgunam – NishKriyam Sukshmam – Nirvikalpamam – Niranjanam~। Anurupya swarupam – Yanmana Vacha Magothram~॥ Satsya – Meidham – Svata sidham Shudham Buddha Mavidrisham~। Ekamevadvitiyam – Brahma – Nehavana – Sti – Kuncham~॥]

Aadyatmopanishad

Tatparya:– That Parabrahma is Paripurna. He has neither a beginning nor end. He is Aprameya, Nirvikara Chidrupa, Nitya. Ananda and is industsable. He is Pratyagatma. Akhanda Gnanarupa, eternal and lastly, Sarvaparipurna. Sarvokrista Niradhara, Shuddha Nirguna Nirvyapara as also Ati Sukshma. Nirvikalpa, Niranjanam. He cannot describe with any specifies. It is how one imagines or views, like it is hard imagine how to describe him. He is Sadrupa and Chidrupa. He is by himself Parishudda. He is Kevala Gnanakara. There is nothing either in the cosmos, Universe which which can be comparable to him. There is other like him. It is Tanganya Anuvinasta.

Shruti:–

[Chaityanam – Brahma].

[Niralambopanisat].

Tatparya:– Nirvishesha Shudda Gnana is Parabrahma.

Shruti:–

[Vignana – Manandam – Brahma].

Brihadaranya Upanishad.

Tatparya:– [TataPavitram – Parameshwarakhyam Advaitya Rupam – Vimalambaradham]

Paingola Upanishad.

Tatparya:– That Parabrahma is Parma Pavana. Servotkrisha. There is no second item to it. He is the only one illuminating, like the Akasha he has no Rupa. Anjaneya, seeing Sri Ramachandra Parabrahmanam and said “You are saying that Parabrahma does not have Nama and Rupa”. Does not “Parabrahma” itself a Nama or Name?

Sri Rama was happy to note that Anjenaya with his Sukshma Buddhi has posed this question.

Shruti:–

[Ritumatma – Parambrahma – Satya Mityadika – Budhai~। Kalpika – Vaivahartham – Yasya – Sangna – Mahatmanaha~॥]

Maha Upanishad.

Tatparya:– The shabda ‘Parabrahma’ is not just a word which can be used casually like for any worldly material or item. It cannot be said how the above word has come through any Manovakya.

For the purpose of the common man or say in the eyes of the Loka one needs a vigraha and assume that there is a Deva. But Parabrahma is not so it was there before creation and it will be there even after the Pralaya or destruction. He has been existing without any Nama or Rupa and can never be destroyed.

In this manner, the Lord sermovises to his ardent disciple. Hence, Parabrahma is just for an understanding and just a dialects in a language.

Shruti:–

[Sristey Pura – Namarupa – Viverchitam – Jagath].

Tatparya:– Before Sristi, this Jagat had no Namarupa.

Shruti:–

[Sadeva – Saumya – Idamagra – Aaseeth~॥]

Chandhogya Upanishad.

Tatparya:– Before Sristi, this Universe did not have Namarupa there was only Nirvishesha Chinmatsa Parabrahma was present.

Shruti:–

[Thasmee – Maru, – Shuktika – Sthanu – Spatika Dau – Jala – Raupya – Purusha – Rekhadivath~। Lotita – Shukla – Krishnamayi – Gunasamya – Nirvachya – Moolakriti – Raaseeth].

Paingola Upanishad.

Tatparya:– In Marumariechi, Jala, in Shukti Rajata, in a wooden pillar man in the Spatika state. Man lived even when there was no different Varnas, he lived in an illusory state (Branthi). There was a Maya – Shudda Nirguna, Nirakara, Nirvikara Parabrahma in the Moola Prakruti.

Through this Prakruti rose 3 types of Varnas Viz – Red, Black \& White from this 3 Rupas came into existence, it became 3 Shaktis.

White Varna – Maya Shakti.

Red Varna – Avidya Shakti.

Black Varna – Avarna Vikshepa

Shruti:–

[Satvodriktha Varna – Sakti Rasith~। Tath – Pratibimbam – Yath – Iadishvara – Chaitanya Maseeth~॥. Sa Swadheena – Maya – Sarvagna~। Sristi – Sthi – Layana – Madikartha – Jagadam Kurutha. Rupa Bhaviti~॥.]

Tatparya:– Maya Shakti was Prominent is Satva Guna. In that the Pratibimba which is reflected as Pratibimba was known as Ishwara. This Ishwara took under him and Brahma, Vishnu, Maheshwara rupas came about. Thus, Sristi, Sthi Samhara were taken over by the above 3.

Shruti:–

[Brahmano – Ravyata – Avyakartha – Mahat – Mahatoahankara – Ahankarath – Panchatmanatrani – Panchtanmatmana Trubhyo – Panchmaha – bhutani – Panchamahabhuteybhyo Khilam Jagat].

Trishikhi Brahmano Upanishad

Tatparya:– This is the way the universe has evowed. From Brahma – Maya from Maya – Avyatha from this Ahankara Avarna Vikshepa shaktis – from this Shabda sparsha, Rupa Rasa Ghandas these are Panchatanmatra – from these Sukshma like Akasha, Vayu Teja Jala Prithvi – are the Panchabhutas. From all the above processes the Universe has been created. The Universe which has come into being though Maya which gets exhausted \& may also be destroyed. This repeats itself i.e., Maya Shakti and again the Prapancha. This recurs again and again several times.

Apavada Yukti

Apavada Yukti has a complicated meaning i.e., without reason or cause there is no Karya or the cause itself may look like Karya – this is ‘Apavada Yukti’. This can be inferred in the following example, clay or mud with which one can mould a variety of pots, but when the pots break, it assumes its original for viz clay or mud. Hence it means the clay was there before and is also present in the pot form. The 2 are inseparable.

Shruti:–

[Brahmadhyam – Sthavarantacha Pashyanti Gnanacha Kshusla~। Tameka – Meva – Pashyanti – Parishubram – Vibhum – Dvija~॥ Yasri – Bhava – Praliyenthey – Leenashta Vyaktam Tam – Yayyu.~। Pashyanti – Vyaktam – Bhuyo – Jayantey Buddada Iva~॥.]

Tatparya:– Hey Anjeneya! Just as the rivers merge with the Sagara and becomes one with its. Similarly, Brahmadis Sthavara Sakala Chara–chara Jagat which arises from the Parabrahma merges in Him during the Pralayakala. Just as waves arise from the Sagara, so also Sristi once again takes place. This keeps recurring or one can conclude that the Universe the Jagat are not separate from Brahma. The Gnanis look at the Universe as Brahman. Just as the waves each of them look different Yet when its recides there is no difference. Therefore, we feel that there diff Nama Rupas but when its merges its looses its differences. But the ONE which ever remains constant \& permanent only Brahman.

Vedanta Vichara

Sloka:–

NamaShastra – Patheylokey – NamaDevancha – Pujanam~। Atma Gnanam Vina – Partha – Sarvakarma – Nirarthakam Achara Kriyanthey – Kotidanameha – Girikanchanan~। AtmaIatvam Vina – Nasti – Muktinarasti – Guruvina~॥ Koti Yagnam Kritam – Yena – Kotigohaya Damakam~। Gajadanam – Sahashrancha – Mukti Gnanam Vina – Nach~॥ (Garbha Gita)

Tatparya:– Hey Anjeneyaney! A man who has not attained Atma Gnana is of no use. He may be one Pandit in all the Shastras, and may worship all the Devatas like Brahma Vishnu, Maheshwara becomes a waste if he does not know Atma Gnana. He may have gone though all the Varnashramas without any negligence. He may have done Dana as high as the Meru mountain, He is unsuccessful if he fails in Atma Gnana.

If he performs Koti Yagnas, Cows horses elephants in terms of Koti if he does dana, he will attain Mukti. Without the Guru's kataksha, he will not receive Atma Gnana. Without Atma Gnana there is no Mukti. If he knows the lakshanas of the body. If he cannot identify that which is Atma he will not attain Atma Gnana. I shall describe the lakshanas or signs of a Sharira (body) Please listen athentively.

Sharira Lakshana

Shruti:–

[Sankyam – Dehavicharancha~॥]

Tatparya:– Through Sankya Vicharana one can identify a Deha and Dehi.

The Sharira or body consists of Gnanindras, Karmendriyas, Pancha – tanmatras, Pancha Pranas, Antakarna Chatustas in all 24 tatvas in a body. The dehi (Atma) which resides in such a body, this is the 25th Tatva.

\item Shotrendriya means the ears which can perceive different Shabdas – Sounds.

 \item Tvagindriya – The skin of the body. It can feel, sheeta, ushna, Mridu \& Kathina – 4 of them.

 \item Chakshurindra – means eyes. It can perceive or see (1) White (2) Red (3) Black (4) Green (5) Tall Long (6) Thick or fat (7) Dumra (8) Mixed color (9) Round (10) Square (11) Obise or Tundu (12) Tall (13) Fat (14) Lean or Thin. These 14 rupas can be discerned by the eyes.

 \item Jeevendriya: The tongue. It can identify the following tastes salt, sour, sweet, bitter, hot or spicy \& rough to the tongue. These 6 rasas can be tasted.

 \item Granendriya: Nose. This can smell, Sughand, Durghanda.

The above 5 Indriya's identifies different aspects of in a body and is known as “Gnanendriyas”.

\item Vagindriya: Mouth, this helps one to talk.

 \item Panindriya: Hands.

 \item Padendriya: Legs – one can walk, sit, stand, jump, run etc.

 \item Ghuhindriya: Men and Womans' Marma sthana – Excretion, encreates Shukla Shovi, Sambhogi.

 \item Payurindra: Guda or Kidney. Mala Visargeney.

The above 5 Indriyas each of them do different Karyas and are known as “Karmendriyas”.

\item Mana: Mind. This can discern, Sankalpa \& Vikalpa. I.e. it can think about an item and then forget about it. These are Sankalpa and Vikalpa.

 \item Budhi: Intellect. It decides which is to be done or not to be done.

 \item Chitta:: This tempts the Bodily indiriyas and see that the respective indiriyas do their work properly.

 \item Ahankara: Ego. To have Ahambhava ‘Me Mine’. Hence, it does both Punyas and Papas.

The above 4 Indiriyas are known as – Antara Indiriyas.

The body's external Gnanendriyas, Karnendriyas it enters within them to conduct (Vyapara).

\item Prana Vayu: This is located in the Hridaya – Heart and controls inhalation and exhalation through the nostrils and helps in Digestion.

 \item Apana Vayu: This is within the kidneys and extends them for excretion and contracts thereafter.

 \item Udana Vayu: This is situated in the throat, it helps to Swallow Saliva as also food and though it one Vomits also. It helps the rasas of the food to reach the various parts.

 \item Vyana Vayu: This does the work of separating the rasas, fibres etc or ratuer the waste. It also makes the body feel the sparsha or touch.

 \item Samana Vayu: This is located in the navel – Nabhi. It helps the body to know hunger, thirst and converts the food into blood, and sees that they are evenly distributed.

Due to the above 5 Vayus a man remains to be alive. Hence, known as “Pancha Prana”.

\item Sound: Shabda Akasha Tanmatra. May be through the ether the sound cuters the ears and one can hear.

 \item Sparsha: Touch. This is part of Vayu Tanmatra. As the Vayu is throughout in the skin it helps the body to feel in the touch hot cold.

 \item Rupa: This is Agni Tanmatra. The eyes have this agni tanmatra and thus can discern 10 rupas.

 \item Rasa: This is Jala Tanmatra – Water. The tongue can feel the water on it \& so it can identify the 6 Rasas.

 \item Ghanda: This is Earth. Tanmatra through the nose Suganda \& Durganda.

These 24 Tatvas cannot help one to know what is Atma.

As the Atma is Gnana Swarupa and is known as “Chittu”. Thus, these 24 tatvas consists a body and on the whole are divided into 3 Vidhas.

Sharira Traya

Sthula Sharira means the following aspects.

Sthula, Sukshma, the body is of 3 Vidhas Legs, Hands, Head, Eyes, Nose, these are Avayas and due to this, this body is made up of Mamsa and therefore, when one touches the body, it is hard, soft, fat, wide, tall, lean etc.

Sukshma Sharira: This consists of Gnanendriyas, Karmendriyas, Prana, Manasu, Budhi – they are 17 in nos. This is also known as “Linga Sharira”

If the eyes cannot see, or the hands cannot move or grasp, the inner part of the Sukshma Sharira. Agnana is the cause of Sharira (Shreyithey Shariram) (Dahyathey – Deha). As things digest in the body it is Deha.

(1) If there is no food for the body it will get emaciated, or illness may set in, or this body will within away due to old age. Another point is when the Prana goes away from the body, this so called Sthula Sharira is fils to be taken for cremation. Another reason for the feebleness of the body is Dukha or sorrow which the individual is subjected to.

(2) The Sukshma Sharira to gets a hit if due to Raga Dvesha, anger – In an extreme form – These are the Vikasas of Sukshma Sharira. The mind becomes weak and may cause Psychosomatic problems.

(3) Agnana – Ignorance – Ego ‘Me Mine I’ abhimana becomes too much, the body may deteriorate. Sukshma Sharira, gnanendriyas loses their rupa and it has repercussion the body. This kind of Sukshma Sharira is not visible like the Sthula Sharira made up of Mamsa which is visible to the eyes. Sukshma Sharira can only be felt but invisible. In order to know the existence of the Sukshma Sharira is ratha difficult. At sleep, death or epilepsy, it results in the Sthula Sharira having the eyes, ears, Indiriyas are unable to function when these Indiriyas mat function so too the body cannot operate properly. Hence, one can conclude that the Sukshma Sharira controls the Sthula Sharira to a very large extent. It is vice–versa to the Sukshma Sharira without a body cannot function. Just has wood gets burnt \& turns into ash, so also the Sukshma Sharira by staying within the body makes it function. Just as the wood loses its rupa so also when the Sukshma Sharira goes off the Sthula Sharira becomes inert or loses its capacity. Both are interconnected.

Further, Sri Rama continues saying, Hey Saumya! There may be physically 2 eyes, but can see only when there is Sukshma Sharira which makes it to see. Similarly, all the organs of eyes, ears taste \& so on.

In conclusion, one can say that the Sukshma Sharira.

Generally it is said that where there is smoke there is fire, so also, the karyarupa of Sthula and Sukshma are the reason for the body \& mind co–ordination. The Agnanarupa Sharira, when it states “Iam Ignorant”, that ignorance can be seen in reality. That ignorance or how man identifies with the body is “Karma Sharira”. This is the cause of Birth and death, because of Sthula \& Sukshma Sharira have risen.

[(Va Agnanada Viveka) Aviweka abhimana. Abhimana Dragadeya. Ragadayam – Karmaney – Karnana – Sharira – Parigraha~॥]

From Agnana Aviveka, from aviveka “Me Mine” – ego Abhimana from abhimana – kama krodha etc. from kama krodha, one does a lot of karmas, hence from this the cycle of birth and death. Purvakarmavasha, Several lives take place creating sorrow. Therefore, karnasharira gets destroyed there is no Mukti.

Shruti:–

[Thata – Vidvat Brahma Gnanagnina karma Bandam – Nirheth].

According to Paingo Upanishad from Brahma gnanagni rises Agnana Sakala karma's should be completely destroyed, then alone one attains Mukti.

Anjeneya seeing Sri Ramachandra says, ‘Swami’/ what is Gnana? What is Agnana? Then, Sri Ramachandra begins to tell him.

Gnanagna Swarupa

Shruti:–

[Chaitanyam Vina – Kinchanna Sthi – Sakshatkara Anubhava Gnanam / Nanatma – Bheda kaljitam – Gnanamajgnanam~॥.]

Tatparya:– The explanation for the above shruti is rather difficult, as the Swami gives in detail a long list of the ways a man looks at the Prapancha. a man thinks the various nama rupa Prapancha as Drishya, though his physical body and its various sense organs etc as Jada and the whole Universe as gnanamaya and he is Driganta. He thinks that the Tarangas which arises in the Samudra is different from the water in the ocean, So also, he imagines the Prapancha is not different from himself, Even so just as one imagines a rope as a Sarpa, so also the Prapancha is Asatya.

An individual, who thinks that he is Atmaswarupa, Gnanananda, Parabrahma and removes all the above mentioned doubts and as a Nishkama as Atma Dyanapara Nirvikalpa, without any worries, forgetting all the Sarvadrishya. Kevala or just he is Sakshibhuta with absolute Ananda or Bliss, this is known as real Gnana.

The Lord goes on to describe what is Agnana. Instead of thinking about himself as Atmaswarupa \& Parabrahma. He thinks otherwise, like the body is the one which belongs to him becomes egoistic and imagines that the Chara–Chara Universe is different from him, he lands himself into the vicious circle of Kama Krodha. Irshya (Jealousy) hatred and other Durgunas. He experiences Jiva Himsey Vyabhichari and commits Papa Karmas, and gets attached to the family, which leads to Tapatrayas and sorrow and imagines or rather has the illusion that there is no birth and death. – This is Agnana and so again and again he goes though the cycle of birth and deaths Saumyeney.

Shruti:–

[Atmava – Arey – Drastavya – Shryotavyo – mamtavoyo – Nidhidyasitavya~॥.]

Brihadaranya Upanishad.

Tatparya:– One has to identify the Lakshanas of an Atma and keep doing Shravana, Manana and Nididhyan, he may remove all the doubts \& thus may attain Mukti. Therefore, now I shall explain how to identify Atma.

Signs of Atma – (Lakshanas)

Shruti:–

[Ashiram – Sharireshu – Anvaste – Shiva Vastitam~॥.]

Keno Upanishad.

Hey Anjeneya! In this body, the Atma without a body, indestructible resides in the body.

Shruti:–

[Athaya – Masharira – Mrita:]

Brihadaranya Upanishad.

Tatparya:– The Atma which resides in this body cannot be destroyed and is bodiless.

Shruti:–

[Ayatma – Brahma~॥]

Brihadaranya Upanishad.

Tatparya:– The Atma which resides in this body is parabrahma.

Shruti:–

[Anoranarya – Mahato – Mahiyarya – Atmasya – Jantornihitho – Guhayam~॥]

Tatparya:– The Atma is ever present in all Jivajantus right from a tiny Jantu, or a gigantic elephant or those Jantus which are larger than the Meru Parvata.

Shruti:–

[Yadey Veha – Tadmrita – Tadnivaha Mrityo – Samrityu – Mapnoti – Ya – Iha – Naneva – Pashyate~॥.]

Kato Upanishad.

Tatparya:– The atma which is in the body in this loka becomes Parabrahma in Parloka. That same Parabrahma remains as an Atma Rupa in the body in this Loka.

Just as one Akasha becomes Ghatopadi and becomes Ghatakasha. If there is no Upadhi is known as Mahakasha. So also the Paripurna Parabrahma because he reside in this body he is known from Dvoupadi as “Atma”. Those who separate Atma and Parabrahma are subjected to Janana Marana.

Shruti:–

[Yeko – Vashi – sarvabhuthantratma Yekam – Chizam – Ya: karoti~॥.]

Kato Upanishad.

Tatparya:– The seed of the Pepal when sown grows into a huge tree, with branches, trunk, raw fruits and then they become ripe, the leaves and thus the tree takes its huge form, and the tiny seed is not all visible, though it was the cause of the whole effect and its presence is there in all the parts of the tree.

Similarly, the lone Parabrahma's. Atma, is there in the entire creation in all the chara–chara prapancha.

Shruti:–

[Yekodeva – Svarva Bhutesh – Ghuda – Sarva Vyapi – Sarvabhutatmaratma~। Karmadakshya – Sarvabhutadi – Vasa Srakshi – Cheta – Kevalo – Nirgunastiva].

Shweteshwara Upanishad.

Tatparya:– Just as each one of the pearl in a necklace are fixed in the thread so also Parabrahma. Atma is there in all the Jiva Jantus, it is there in all forms and is an important witness for all their activities.

Shruti:–

[Agniryathaiko – Bhuvanam –Pravisto – Rupam Rupam – Pratirupo – Babhuva~। Yekastho – Sarvabhutantaratma Rupam Rupam – Pratirupo – Bavisva~॥.]

Katto Upanishad.

Tatparya:– If a lamp is burning with one wick, and by the same lamp other lamps are lit, the one original lamp is everywhere in all the other lamps. So also, the only One Atma is present in all the bodies, but it appears to be several atmas which is not so.

Shruti:–

[Vayuyarthey ko – Bhuvanam – Pravisto – Rupam Rupam – Pratirupa – Babhuva~। Yeka Sthata – Sarvabhutantar – atma – Rupam Rupam Pratirupo – Bahistva].

Katto Upanishad.

Tatparya:– One Vayu is there throughout the entire Universe in various forms like Pranavayu, or like a hurricane, it may look different forms but all of it is one and the same. So also, one Atma from the start to the end Brahma parianta is there in all the bodies and it seems to appear as many Atmas.

Shruti:–

[Tadetat – Brahma – Purvamanapara Manantara – Mabhyam~॥.]

Brihadaranya Upanishad.

Tatparya:– In this Universe, front, back, right, left, inside, outside, below and above – everywhere there is Paripurna Atma.

Shruti:–

[Sava – Yesh – Mahanaja – Atmajaro Maro Mrito – Bhayo – Brahma Bhayam Hi–Vai. Brahma Bhavati~॥.]

Brihadaranya Upanishad.

Tatparya:– The one body goes though childhood, youth and old age. The Atma has no Uttpati who does not have childhood, Yavana Kaumara and old age. It has no death, does not get destroyed, has no fear, He is Parabrahma.

Shruti:–

[Najayetey – Mriyatena – Vipashvi – Nayam~। Kuthishvinna – Bababhuva – Kashchith~। Ajo – Nityo – Sharvathoyam – Purano – Nahanyate – Honyamano – Sharirey~॥.]

Khatto Upanishad.

Tatparya:– This Atma has no birth or death, and He is Sarvasakshiyu. He – Atma is not born to one and get killed by another. He has no reincarnation. He is Nitya Shashvata – permanent. He has no Adimadyanta. The body gets destroyed, but the Atma never ever gets destroyed.

Shruti:–

[Hantha Chenmanythey – Hanthum Hathashenvonarythe – Hathavu Ubhavtau – Navijanitho – Nayam Hanti – Nahanyetey~॥.]

Khato Upanishad.

Tatparya:– This Parabrahma has no birth and does not get killed by anyone. Neither does he kill anyone since he has no physical form does not have hands \& legs. Since, he does not have Rupa, Nama, he too cannot be killed by anyone. Only those who does not understand the Subtle aspects of the Parabrahma think of as having birth and death.

Chitta Lakshana of Atma

Shruti:–

[Katama – Atmeyti – Yo Yam – Vignanamaya~॥.]

Brihadaranya Upanishad.

Tatparya:– What is Atma? In all the Jeevajantus there is an Atma residing in their hearts –. “Aham Aham” means me mine, my eyes, nose, ears, my wealth my body, my wife, children thinking of all these physical universe, all these aspects is known as Gnanaswarupa Atma or ‘Chitta’. Chitta means – Gnana Swarupa.

Shruti:–

[Yena – Rupam – Rasam – Ghandam Shabdartha – Sparsham – Cha – Maithurna~। Yethai Ney Va – Vichanathi – Kimatra Parishishyate – Yeta dvai – Tat~॥.]

Tatparya:– What is Atma? That which exists in the body, though the eyes makes it perceive all the rupas, through the nose all the ghandas, taste through the tongue, all the sounds through the ears, and lastly the sense of touch through the skin. Through Upasthe all the Bhogas or pleasures of the material world. This GnanaSwarupa Vastu is called Atma.

Shruti:–

[Yatsarveshu – Bhutesh – Thrisha Sarveybhyo – Bhutebhyo anthroyam~। Sarvani – bhutani – Navdu~॥.]

Brihadaranya Upanishad.

Tatparya:– That which is Sarvavyapi in the Sarvabhutas and that which permits to be there, that Gnanaswarupa vastu is – Atma. That is which is behind the Panchabhutas which is Agnana is converted into Gnanaswarupa and that which comprehends everything is known as Atma or Chitta.

Shruti:–

[Yaha: Praney – Trista – Prana dantaro – Yam – Prano – Naveda~। Yovachitrista –Vachanto – Royam Vagmam Veda~। Yaschakshu – Trista – Chakshu Ranta Royam Chakshu nov veda – Thyodi~॥.]

Brihadaranya Upanishad.

Tatparya:– Just as a lamp in a temple lights up the temple, similarly, in the body the Gnanajyoti atma which resides within the body, it covers the entire Dehaindriyas and that atma's light envelops the Prana vayu and this vayu though the nostrils leads to inhalation and exhalation. Atma is different from the Pranavayu, the Gnana swarupa Atma is not aware of the loss of pranavayu.

The atma Prakasha is enveloped in Vak and it permits one to talk but the Atma is different from Vak (to talk). Atma which is Gnanaswarupa is not aware of the Agnana of Vagindriyas. As the atma is there in the eyes, it is ablets see but atma is different from the eyes – The eyes are also without Gnana. Similarly, in all the sense organs like ears, tongue or Sparsha, though Atma is in all the Indriyas, yet it is tatally from all of them.

Gnana swarupa Prakasha which is present in Manas (Mind). The mind's Sankalpa Vikalpa is active with these, but Atma is different from the mind too. The mind does not possess Gnana and hence the Atma which is Gnanaswarupa is different. Since, Atma is there in Sparsha, it can feel the touch of objects as also if also hot cold etc., but this is also different from the Gnanaswarupa Atma. Similarly, atma is present in Buddhi and it takes decisions as to how the body should react. Atma is different from Budhi or Intellect.

Atma Prakasha is present in Shukla therefore, this creates Pindodth utpathi Atma is different from this too.

Shruti:–

[Yeneydom – Sarvam – Vijanathi Tam – Kena vijaniyath~। Vignatamatey Kena – Vijaniyath~॥.]

Brihadaranya Upanishad.

Tatparya:– As the atma resides in the form of Gnanaswarupa among the entire humanity, which makes them understand – the loka of humans and therefore they can enjoy the Universe.

To really understand the Gnanaswarupa of the Atma there is yet another gnanaswarupa vastu – Asatya which means Parabrahma, Apart from there is still another shresta vastu Abhiprayavu.

Lakshanas of Sath – Atma

Shruti:–

[॥ Swapnantham – Jagritantam – Joubhav – Ye – Nanu Patshyati~। Mahantam – Vibhunatmanam – Matvadhiro Nashochati~॥.]

Khato Upanishad.

Tatparya:– In this Loka, everybody have an Atma in their respective bodies which helps them to perceive in the wakeful and dream state are able to comprehend the various Namarupas in the Jagat. During his state of sleep and when awakes he says that he has had a good sleep in the night. During his sleep he is not all aware of the external world.

At sleep he undergoes three states of Jagrat, Swapna, Suchupthi and the industructable Atma is there only as a Witness. Therefore, the individual is aware of these states. Thus ‘Sath’ is the, one which is not destroyed in the Kalaratri. Since, in the above 3 states, since Atma is Nasha Rahita. Those aspects which an individual cannot express does so during the swapna and Sushupti conditions. There is total darkness and all the sense organs are resting he is able to dream through his inner world. Once he wakes up all that he saw in his dream vanishes. Here they make it clear as to what is appearance and Reality that is to say that the universe he live in is Maya – Asath.

Atma's Ananda Lakshana

Shruti:–

[Tadeytat – Preya: Putratat – Preyo Vithat – Preyo – Atmanamova – Priya – Mumapasita~॥.]

Brihadaranya Upanishad.

Tatparya:– Hey Anjaneyane! In this loka everybody would like to have a chance of Ananda or Bliss. Dukha is apriya. That which has Ananda leads one to enjoy that pleasure. That which is sorrow leads one to undergo Dukha! Dhana, wife, children gives one ananda and hence everyone goes for it, but it is maya. If a house is on fire, he rushes to save his wife and children while trying to do so, if he himself is in danger, he lets go everything to save himself he gets busy. This goes to show that the atmaswarupa of himself becomes most important. It leads to the conclusion that this becomes Parmanandasrupa.

Shruti:–

[Yeshosya – Parma – Ananda~। Yetsyayi – Va – Nandasyaryani – Bhutani – Matra Mupa – Jevanti~॥.]

Brihadaranya Upanishad.

Tatparya:– This Anananda Swarupa Atma is present in the body which helps all the jeeva jantus to be alive.

Atma's Nitya Lakshana

Shruti:–

[॥ Ashabda – Masparsha – Marupa – Mavyam – Tatha. Rasam – Nitya – Maganthavachha – Yath~। Anadyanantam. Mahataha: Param Druvam Nichayatam – Mrityu Mukhath – Promuchyethey].

Khato Upanishad.

Tatparya:– Atma has no shabda (sound). The shabda Guna of the Akasha is also not Atma. Atma has no touch. The sparsha Guna of Vayu is also not Atma. Atma has no rupa. The rupaguna of Agni is not Atma. The Rasa guna of water is not atma. Atma has no Ghanda (smell). The Ghandaguna of Bhumi is not atma. The Atma is Nitya; unlike the body growth, Kshaya does not Occur to Atma. Another quality of Atma is consistent. Adi Madhya Antha is not there for Atma. Atma does not possess any physical attributes like height weight, tall short etc. unlike the Sharira, Atma is not Sthula. He cannot be measured. Atma is different from Budhi. He is only Gnana without any Namarupa. Atma is neither Anu (atom) or Mahat.

In conclusion one can say to realise and understand the most importent characteristic in the Loka i.e. Atma. One is fortunate if he comprehends this mysterious and difficult aspect of one's life – Atma, he will be liberated from Janma Mrityu and he will attain the Nityananda Atma – Sakshatkara.

Atma's Nirmala Lakshana

Shruti:–

[॥ Suryo – Yatha – Sarvalokasya – Chakshurna lipyathyte – Chakshu Shai – Bharahya Dosheai~। Yekasthata – Sarva Bhuthantaratma – Nalipyate – Lokadhukeya Bhahya~॥.]

Khato Upanishad.

Tatparya:– The most visible form of God or the cosmic energy is Surya. This wonderful phenomeno irrespective of any shines on all the plant kingdom possessing perfume or others which have Devighanda, he will shine or its rays will fall on the whole cosmos or the Bhumi. Similarly, the great Atma which resides in the hearts of all beings irrespective of their actions like Papa–Punya and is not attached to any of it.

Atma has neither Papa Punya or doshas and hence he is Nirmala.

Atma's Svayam Prakasha Lakshana

Shruti:–

[॥ Namatra – Suryabhati – Nachandra tarakam~। Nema Vidyuto – Bhanti – Kutoyamagni:~। Tameva – Bhanti – Manubhati sarvam~। Tasya Bhasa – Sarvamidam – Vibhati~॥.]

Khato Upanishad.

Tatparya:– The Teja of Surya Chandra Agni, Nakshatra, it is Svayamprakasha due to Atma's Satta, they are all radiant. The above mentioned phenomenas cannot render Prakasha to the Atma. If logs of food and other items are fed to the Agni it will burn them into ashes. Surya Chandra etc. shine due to the satta of the Atma. Therefore, there is no vastu which can lend Prakasha to the Atma. Thus, the Atma by itself in a very easy \& powerful way is Gnana – Prakasha.

Shruti:–

[॥ Aastamita – Aditya –Yagneya Valkya – Chandra Mashya Sthamitey Shantegnow – Shantayam – Vachi~। kim Jyoti – Revayam – Purushati:~। Ityatma Vasya – Jyotirabhavathi~॥.]

Tatparya:– When there is neither Surya nor Chandra, there is darkness, then you will not be aware of your body and this Papancha is in darkness. The one thing which you thought Jyoti's Prakash that is none other than – Svayam Jyoti or Atma. If there is no Atma in the body, you will not know what is light, even when Surya Chandra are lending their radiance not the cosmos. You are aware of all the natural phenomena, because of the Atma \& within your body. Thus, due to the Atma Prakasha, one is able to see the Surya Chandra otherwise it is not possible.

Therefore, Atma is also known as Param Jyoti, or Svaprakasha.

Japa Tapa is Futile

Shruti:–

[॥ Vihaya – Shastra Jalani – Yatsa tyam – Tadu Pasyetham~। Anantakarma – Shauchaucha – Japayagna – Sthathai Vach – Tirtha Yatra Bhigamanam Yavat Tatvam – Navindati~॥.]

Paingolo Upanishad.

Tatparya:– If one studies various Shastras are under an illusion. One should understand Tatvam Vivekam, and that Atmam is Satya, Parabrahma. One should always to research on the Nature of Atma. Bath, Sandya, Japa, Tapa, Homa, Nema, Yaga, Thirtha Yatra etc. If one does all this Karma it is of no use.

Shruti:–

[॥ Aditya – Chaturo – Verda – Sarvashastranai Nikesha:~। Brahma Tatva Najanati – Darvipakarasam Yatha~॥.]

Tatparya:– When one prepares a tasty with a required ingredients and if one dips a wooden ladle, that ladle is ignorant of the taste or the ingrediants which went into it. So also if one becomes a Pandit in all the Shastras and that too under the tutelage of a Guru, it is futile if he is not aware of Tatvagnana, experience Atmas presence, everything one perform is a waste.

Shruti:–

[॥ Santajya – Hridghushanam Devmanyam Prayantiye~। Te – Ratna – MabhiVanchitanati – Haststha – Kaustubha~॥.]

Tatparya:– The Atma which resides in the heart. Instead of this if one is in search of a God elsewhere is like looking for a kaustubhamani, if one is knows only the ratna in the Bhuloka he just becomes a common man.

Shruti:–

[BahuShastra – Katha – Kantha – Ramanthena Vrithva – Kim~। Anvestavyam Prayatnena – Maruthey – Jyotirantharam~॥.]

Tatparya:– Hey Vayuputra! There is absolutely no use studying the Shastras. This will not lead one to Mukti. One should see and meditate on the Parabrahma Atma from the inner eye – Antaranga Dristi have a darshana of it then alone one can attain Mukti.

Shruti:–

[॥ Naivamatma – Pravachana – Shateynapi – Nalabhyathey Nabhushrutena Nabudhigno Nashrutena – Namodhaya Nadevyayirna yagney – Nratopibhirugrey Nrasankye – Nrayagyai – Nranairatnana Mupalambhathey~॥.]

Subalopa Upanishad.

Tatparya:– This Atma Anubhava is not possible with the following actions of a being, useless hearing or rather unnecessary talk, the cleverness of the Budhi, Yagas Devatadhyana Tapas Sankhya – Hata Yoger, Varnashrama Dharma even if the above is done with utmost devotion, one will gain the ultimate.

Shruti:–

[॥ Amritena – Tripthasya – Kim prayojanam~। Yevam Svatmanam – Gnatavama – Vedyayi – Prayojanam Kim bhavati~॥.]

Tatparya:– To consume amritapana and Jalapana one will not get any result. Similarly, through Vedanta there is no fruit or outcome.

Shruti:–

[॥ Athanta malino – Deho – Dehi – Chatyanta – Nirmala~। Ubhiyorantaram Gnatva – Kasyashavicham Vidhi yatey~॥.]

Tatparya:– Our parents who gives birth to this body is Ashudha. The Atma which resides in the body is Nirmala. Therefore, these two tatvas which a Gnani tries to study has to do Shanchanam. This means, the body requires both, excretion which are supposed to be Ashuddha. Whereas, the Atma does need all the above actions. The Atma is pure by itself. It is only an illusion.

Shruti:–

[॥ Sva Sva rupasya – Vignanath Nadhikam – Tirtha Mucchaitey~॥.]

Tatparya:– The Atmanubhava gnana, one must immense and have a *** * ***, one may get liberation.

Shruti:–

[॥ Uthista – Jagrata – Prapya – Vara – Nibhodhat~। Chakshurarya – Dhasa – Nishita – Duratya – durgam Padam – Tatkvayo – Vadanti~॥.]

Kato Upanishad.

Tatparya:– Hey! the people who are slumbering in ignorance, please wake up and try to reflect. With the grace of the guru and try to contemplate on Atmanatma Viveka and other Vedanta principles and experience Atma Gnana. Without the grace of the Guru, it is like treading on a sharp edged sward which is extremely difficult. This body is made up of Panchakoshas and since the atma dwells within the body, and if one is not aware of what those Panchakoshas are, one is unable to understand the subtler aspects.

Panchakosha Viveka

Shruti:–

[॥ Atmashyarirey – Nihito – Guhayam~॥.]

Adhyatma Upanishad.

Tatparya:– One will not be able to understand the true nature of Atmaswarupa, without the knowledge of the Panchakoshas as the Atma resides within the Sharira. Therefore, I shall explain the lakshanas of the Panchakoshas. Please listen attentively says Sri Rama to Hanuman.

Hey Anjeneya! This body consists of Panchakoshas. They are as follows:–

\item Annamaya

 \item Pranamaya

 \item Manomaya

 \item Vignanamaya

 \item Ananandamaya

These are the 5 types of sheaths which covers the body and makes us unable to see or feel the Atma.

Just as one cannot see the sword when it is in its sheath, or the hard part of the mango or the pulp inside until you open up or the *** which cannot be seen when it is covered, so also one cannot have a glimpse of the Atma as long as we are unable to go on removing one sheath after another.

These koshas are like the clouds which eclipse the radiant rays of the Sun, but once the clouds clear, the bright sunshine appears; So also with Agni when it is enveloped by smoke, once the ignition penetrates deepa, the fire shows up.

Similarly, these layers of koshas or sheaths cover the Atma. Until we go deeper into each layer of the koshas till we reach the ananda or bliss we will be unable to realies the real self or Atma. Just as the sheath of the sword or the hard seed of the mango fruits though they are one and the same, Yet they cannot be viewed till we remove the peel and the pulp. With Tatva Viveka the Atma and the Pancha koshas appear to be different they are not because it has coverd the Atma.

When this explanation was rendered by Sri Ramachandra, Anjeneya poses a question to the Lord as to how these Koshas have come to be there. Further, Sri Ramachandra proceeds in detail.

Hey Anjeneya! Anonyadhyas – It means that the dharma of this vastu is also in the dharma of that Vastu. As the same ‘dharma’ is present in both vastus is called ‘Aadhysa’. Hence, this Adhyasa is present in the Panchakoshas as also in the Atma.

Shruti:–

[॥ Annaresnyai Va – Bhutvanna – Raseynathi Vriddhim – Prapyanna – Rasamaya – Pritviyam – Yadviliyethey – Sonna Mayakosha~॥.]

Painglopa Upanishad.

Tatparya:– The food one consumes that rasa converts into the relationship between female and male, they unite and the foetus remains in the womb of the mother, it becomes a Pindarupa \& from there it takes the form of flesh, nerves, bones, hair, skin the 7 Dhatus and after a tirm of 9 to 10 months, the head, eyes develop and lastly it becomes a Shishu and there is the birth of the infant. It begins its journey as a human being by consuming various types of food and there is growth and as it gets older and olders the body weakens and returns to dust. This first stage of growth is called Annamayakosha.

Then the question arises as to how this gets fused with Atma. The human starts expressing as follows:– I am a Brahmana, a human, a man, I am different from stree, I am a Brahmachari a householder, then a Vanaprasta, Sanyasi – all the differences. I am youth I am aged, – Vayo bhedhas, I am born I will die, I will grow, I am thin etc. – this is Shadvikaras. I belong to some Gotra like Bharadwaja, Kashyapa, then begins the Gotrabhedas.

I am Rama I am Govinda – Nama bhedas I am fair, I am dark – Varna bhedas I am tall, I am short – Rupa bhedas I am Shaivite, I am Vaisnavaite I am a Smartha – Mata bhedas or Sectaran – I am a Pauranika, I am Dikshita I am an Avadhani – Vesha bhedas.

All the above can be classified under Annamayya Kosha – which is Sthula belonging to the body's external appearance and the atma which is only a witness has no dharma perhaps. The human has an illusion that the body is real and even feels that this is atmaswarupa. To some extent the atma's satyagnana lakshanas or signs appear and hence a connection to the Annamaikosha.

Shruti:–

[॥ Karmendriyai – Saha – Pranadi – Panchakam – Pranamaikosha].

Paingo Upanishad.

Tatparya:– Vak (speach) Pani, Pada, upastha, Guda (Kidney) – the Karmendriyas – Prana, Apana, Udana, Vyana are the Pancha Pranas – Pranamaya Kosha. If the connection or link between the above and Atma does not take place, one will suffer from hunger, thirst, I am strong, I am doing Karma, I can talk or Vachalika, I am motion on the earth, I excrete I am a bhogi, I have all the links I maybe a trans gender. These all belong to Karmendriyas. Dharma all this is Pranamaya kosha. This too does not belong to the Atmaswarupa one feels his prana is good If he says that, he starts feeling that the signs of Satyagnana is in the Prana and hence concludes that the Atma is in the Pranamaya kosha.

Shruti:–

[॥ Gnanendriyayisaha – Manahi Manomaya Kosha~॥.]

Paingolopanishad.

Tatparya:– Ears, the body skin, eyes are Gnanendriyas along with the Mind is ‘Manomaya Kosha’. How does this Kosha \& Atma get merged? I am Sankalpa Vikalpa I am Dukhi I have Moha I am a Kami I am an enemy (Dveshi) I am anger (Raga) Task. I see I touch I taste I can smell. I am deaf I am blind I am dumb. I have no tongue. These are the Vikaras. Gnanendriyas sankalpa vikalpa. Raga dvesha kama krodha Mind all belongs to Manomaya Kosha are only Dharma, other than this kosha – there is Sakshi – Swarupa called Atma does not have even an iota of the above emotions or the mind. It appears to be so. As one says, my mind is good hence it looks to be like Atma but it is not So. So there is a Koota between Manomaya \& Atma.

Shruti:–

[॥ Gnanendriyaisaha – Budhi Vignanamaya Kosha~॥.]

Paingolopanishad.

Tatparya:– Gnanendriyas and Buddhi is ‘Vignanamaya Kosha’. As for the koota between Vignanamaya kosha \& Atma is as follows:– I am a decider of the Karyas I have Uhamohas. I am a Chatura I am Hindu Vishayvam. I am a known of 2 subjects. I am a Shrotriya, I am a Virakta I am a Pandit. I am entitled to the heaven or to the Devaloka. The Gnanendriyas, Nishchaya Vrithi Buddhi is ‘Vignanamaya Kosha’. These are also Dharmas but other than this there is Sakshi Swarupa Atma does not possess any of the above qualities or gunas. But the human being feels because he has Budhi and the indriyas and knowledgathe. This does not mean that the Gnanananda is an illusion (Branthi). So there is a koota between the two.

Shruti:–

[॥ Yetat – Koshatrayam – Linga Sharira~॥.]

Painglopanishad.

Tatparya:– Pranmaya, Manomaya, Vignanamaya all the above three koshas form the ‘Sukshma Sharira’ are “Lingasharira”. After death, this Lingasharira immediately takes the Jiva to Swarga lokas and resides there. Once again it enters the mother's womb takes the form of Sukshma sharira keeps taking births and deaths. There is no Mukti until the Lingasharira is destroyed. Due to the ignorance which is hard to be destroyed and hence the Lingasharira is unable to disappear. Therefore, firstly, the Karanasharira known as ignorance or Agnana has to get destroyed, when that is done then the Prarabdha also gets destroyed then the Lingasharira too will be destroyed and one will attain Mukti.

Shruti:–

[॥ Swarupa gnana – Mananda maya – Kosha~॥.]

Paingolopanishad.

Tatparya:– Ignorance Priyamoda, Pramoda the 3 types of happiness “Anandamaya Kosha” – This is Karanasharira – whatever one likes as soon as one sees it feels happy. “Priya” – that means one likes it as soon as the desire or object is obtained one feels happy or ‘Moda’ Once it gets pleasure after enjoying it is called ‘Pramoda’.

This Anandamaya kosha \& its convection with the Atma takes place the following way – I am the enjoyer or the experiencer, I am joyous. I am happy I am a Satvika, I am Rajas and Tamas I am ignorant I am dumb, I am dissatisfied, I am Aviveki, I am under illusion, these are the Dharmas, ignorance, Priyamoda, Pramoda are all features of ananandamaya kosha's dharma, but it is not satyagnana Atma not an iota of it. But due to ignorance one thinks that ananandamaya kosha – and its Dharma appears to be like Atma due to one's illusion and hence there seems to be a connection between anandamaya kosha and Atma.

Hey Anjeneya! Thus the Dharma of Panchakoshas and the Atma's Satyagnananda its signs appears to be present in the koshas.

This Panchakosha is entirely different from the Atma.

It means, one says, my cow any calf, my wife, my child, my daughter – in – law, all this means he is different from all of them. Similarly, my body, my prana, my mind, my buddhi, my ignorance and with the relationship of the Panchakoshas, he remains different from the Atma swarupa. He also thinks somebody else's cow or calf dies or is not good or his wife may be very good or foolish or ugly all the above get does not attainded to him, so also the Annamayakosha, caste, creed kula, gotra, rupa, nama etc the vikaras are Pranamayakosha's relationship like hunger, thirst – the vikaras belongs to Manomayakosha – sankalpa, vikalpa vikaras belongs to Vignanamayakosha Kartatatvadikaras belongs to Anandamayakoshas Bhoktadis also yet all this is different from the sakshi swarupa is not aware.

That means he is Nirvikara, and his opinion's that he is Gnanandaswarupa. But, the cow calf wife are all external factors and hence different from the animals and his wife he may understand it that way. Yet he thinks that the Panchakoshas and the atma gets merged and becomes one. Now one must understand that the two viz atma \& the Panchakoshas are different from each other.

॥ Shloka:–~॥

 [Bahya – Dristi Netram – Antardristi – Buddhi:~। 
 Bahya dristaya – Gnathumashkya Aapidhadarasta:~। 
 Antharadristya – Gnantum – Shakyanthey~॥.]

As per the shloka, external sight or Dristi are the eyes, the antharadristi is Buddhi. The external dristi sight – viz eyes, Ashkya – doubtful aspects such doubtful aspects can be comprehendend by the Antharadristi. If Just like the water and heat gets mixed, if one touches and thinks that water and heat are different. Similarly arts or drawings on the wall, the two are merged, but with one's buddhi it will be two different entities. Similarly the Panchakoshas and Atma are merged, but if one views it though buddhi they are different from each other.

Hey Anjeneyaney! So for through various examples, I have explained the difference between Pancha Koshas and the Atma. But truely, when one mistakes a rope to be a Sarpa or snake, this is really one rope and are not two. So, Saumyaney! Panchakosha and Atma are not different Yet, my legs, my body my Prana my Mind my Buddhi are all appears to be different. The clay and the pot are one and the same, as with the clay different items can be made and given different namarupa. But though Sukshma buddhi everything is same.

The Panchakoshas and Atma in the different Vyavaharas, they are one and the same – This is known as “Panchakoshvyatirekta gnana”. What is the Phala from this Gnana – Vichara – Atma Avasthastha remains as Anisyata. The Vikaras of the Panchakoshas is blatant but it is not different from the Atma. Despite this we keep saying I ate, I experienced I drank water, these activities do not cease. Due to our illusion we think that the two are different. Again the author gives an example of gold and the various ornaments that are made from it. Yes, the ornaments do look different the basic or fundamental root for all of it is Gold.

Shruti:–

[॥ Atmano Namyakurehana kinchana – Atmey Veda Mamavritam Sarvam – Nehananasti – Kurehina~॥.]

Brihadaranya Upanishad.

There are no Padarthada's as other than Atma swarupa, So also just as rope and snake are unsvit the Panchakoshas and Sarva prapancha also atmaswarupa though appears to be different, yet it is the same. Though we are under the illusion and imagination, this is Asatya. One who is atmaswarupa Satyagnananda, Nitya Nirmala, Nirvikara, Shuddha Nirguna Sarvaparipurna, only Sakshi Parabrahma, and who is under this spell and is in Bliss is a Mukti.

Jeeva Lakshana

Shruti:–

[॥ Guham Pravistanv – Parmey – Pararthyi – Chayatapan – BrahmVidho Vadanti~॥.]

Kato Upanishad.

Tatparya:– when there is a reflection of hot weather there is a shadow. So also Parabrahma Atma when it is reflected on the Jeeva both of them reside in the physical body.

Shruti:–

[॥ Ekadha – Bahudhachayyiva – Drishyetey – Jala Chandravat~॥.]

Tripurita Upanishad.

Tatparya:– Just as one Chandra when seen in a Jalaghatta (Lake) when the moon is reflected several Chandras get reflected in the water. So also Parabrahma Atman is only one, it is reflected in different Jantus and other bodies and in the Antakarnas it appears to be as though there are several Parabrahmas.

Shruti:–

[॥ Rupamrupam Pratiruvo Bhabnva~॥.]

Khato Upanishad

Tatparya:– The Anthakarna reflection Andakhara is there in all the Jeevas.

Shruti:–

[॥ Anthakarna – Pratibimba – chaitanyam – Yath Tadevavastha – Traya – Bhagvati~॥.]

Paingola Upanishad.

Tatparya:– As the Anthakarna moves around externally, Yeta as also in the mind within it is known as Buddhi. Due to this reflection of Anthakarna there is Jeeva. This Jeeva in Jagratha, Svapna, Sushupthi states or conditions, it remains within all these states and hence he experiences all the Vishayas. But the Atma does not participate in the Triyas except as a Sakshibhutha. The mirror is Jadapadartha: The reflection within it is kalpitha it is just Jadapadartha. Similarly, the Anthakna also is Jadapadartha and the reflection within it is also *****. Therefore, Jeeva is (Chetana). It means, gnanaswarupa, Nitya.

Shruti:–

[॥ Tasyabhas – Sarvamidam Vibhathi. Khato Upanishad].

As stated in Khato Upanishad because of the relationship with Agni, the Kastas take the form of agni and gets burnt. So also Antakarna with its reflection Gnanaswarupa atmas trivya becomes one and this relationship appears as atma (Chaitanyatatva). That is my mind, my body – this gnana is known as Chaitanya, jeeva etc. If there was no relationship between the atma the anthakarna and the reflection within it would have been known as Jadapadartha padartha. Therefore (Atmayendriya – Manoyuktam Bhokteythyavbu – Marshina) – Khato Upanishad.

Yogi atma, antakarna, reflection all the three samudaya is known as Jeeva (Bhokta). But if one question is a Yathortha then Antakarna reflection becomes a Jeeva.

Shruti:–

[॥ Sachaviveka – Prakruthisangatya – Tatrunvi Hrithey~। Namayoni – Shatangatwa – Sheteysan Vasanava Shathi~॥. Vimokshath – Saneharthyeva – Matsya – kuldvayam Yathah:].

Trishikha brahmano Upanishad.

Shruti:–

[॥ Karmendriyani – Gnanendriyani – Tatdvisharya – Pranaa – Samsratya – Kamakarmanvita – Vidyabhutha – Vestito – Jeevo – Dehantaram – Prapya – Lokantaram Ghatachi~। Prakarma – Phalapakena Aavartantara – Keetavat – Vrishanti – Neyyi vagacchati~॥.]

Paingola Upanishad.

Tatparya:– The moon in the sky when seen in the water remains still but when there are waves then the reflection two stirs or moves. Similarly Nirvikara Parabrahma's atma is reflection, the Jeevas too become Nirvikara, but that Jeeva gets moha with the Maya and becomes ignorant (Agnana), sankalpa, vikalpa, kritvadi vikaras takes shape and like the padarasa it wavers and begins to think me, my body, my wife, and becomes Dehiabhimana which leads to Ahankara, Mamkara, Kama, Krodha, Lobha moha, mada Matsya all these Gunas and through the Sthulasharira begins to do several karmas and enjoys its pleasures and after the death, gets merged with the Sukshmasharira becomes sukshmarupi and reaches heavens and hells, sukha–dukhas are experienced again for the sake of karmeshta, and with the Purvakarma, begins to take birth in defferent Yonis again gets into the cycle of births and deaths. Gets into a rub from which he is too weak minded to come out of it and swings from one Ghatha to the other and like a baby fish gets into the Whirlpool and there is no escape for him and wriggling like a worm gets into Janana – Marana and is immersed in Dukhasagara gets drowned and unable to escape is moving around.

Shruti:–

[॥ Sakarma – Paripakutho – Bhaunam Janma – Namanthey nrinam – Moksheyachha Jayethey~। Tada – Sadguru – Matritya – Chirakala seveya – Bhandat. Moksham – Kashith Prayashathi~॥.]

Paingola Upanishad.

Tatparya:– Thus, the jeeva takes several births and due to his Punya Vishesta takes human birth with the destra of attaining moksha takes shelta in a Sadguru and serves him with the chaturavidha shushreya. Listens to the Vedanta and from the atmanatma becomes a viveki and realises that he is a reflection of the Parabrahma atma and his Jeevaswarupa is just an illusion. The Nirvikara Parabrahma, atmaswarupa, he himself is he understands or decides, tries to become Nirvikara and gets rid of Ahankara, Mamakara (Pragnanaghama Yevai Tebhyo – Bhutebhyo – Sasmuthaya Tanyevamu – Vinashyati). The above shloka was recited by Yagnavalya to his wife Maitreyi in the form of Shruti.

Once the Prarabdha karmas are exhausted, the body is also destroyed.

Brahmaveda – Brahmeyi Va–Bhavati – as per the above shruti Ghathakasha Mahakasha becomes one. He becomes Aikya with the Akhanda Sachitananda Nityanirmala Paripurna Parabrahma and gets Nityamukta.

Shruti:–

[॥ Yeko – Bahunam – Yo Vidhadati – Kaa – Maan~॥.]

Tatparya:– Hey Anjeneyaney! The Parabrahma Atma alone is Jeeva–Sakshi for all humans and remaining within them fulfills all there desires. Thata is, for one who thinks that he is Brahma, and realises that there is no difference between the two Brahmas.

He becomes Aikya with the Paramatma. The person who does not realise this oneness, remains ignorant.

Aikya Bhodey

Shruti:–

[॥ Avyakta – Leshagnanaachadita – Parmarthika – Jeevasya~। Tatvamasyadi – Vakyani – Brahmanyat Kyatam Jago~॥.]

Tatparya:– Hey Anjeneya! In this Loka all Manushas \& all the jeevaru are under an illusion or Agnana. They feel it is an avyakta shakti which is pervading Mankind do not understand that we are a reflection of the Parabrahma and are unable to see as it seems to be hidden. If the jeevas understand the Mahavakyas of Brahma Aikyam and “Tatvamasi”, and contemplate on them, they will be rid of Jananamarana cycle and will attain Brahma Aikya Mukti.

‘Tatvamasi’ – Means – You Yourself will become Parabrahma.

Shruti:–

[॥ Para Jivopaadi – Maya Vidya – Vihaya Tatvampada – Lakshyam – Pratyagabhinnam – Brahma – Tatvamsiti~॥.]

Paingopa Upanishad.

Tatparya:– What is ‘Parmatma’? Maya Pratibimba Adhista, Sakshi or is a witness Chaitanya is “Parmatma” “Pratyagatma” is that antakarna Pratibimba.

From one Akasha Ghato upadhi rises Ghato akasha, from Matha upadi rise Matho akasha, from Griha upadi arises Griha akasha from Gopurapadi rises Gupurakasha. By these changes, Parabrahma atma alone resides in all the bodies of Jeev–Jantus is known as Pratyagatma. Since he also dwells within the bodies of Brahmarudradis he is known as ‘Parmatma’. Mahakasha is Ghatakasha and Ghatakasha as mahakasha, Pratyagatma all these are Parmatma and also Parmatma is Pratyagatma known as Samarasya Kaivam. For both of these Satyagnan Lakshanas are there, there is also no difference between the two.

The Meghajala has the reflection of the sky as also in the Ghatajala as this is only imaginary, there cannot be Ayakya between them.

Similarly, the Ishwara which is reflected in the Maya and it reflects within the Antakarana also but there is no Ayakya as it is Mithyaswarupa. So, Anjeneya, Jeeva–Swarupa is also imaginary.

Atmaswarupa, is True–swarupa, you give up the idea of Jeeva it is just Branthi. You are Pratyagatma as also Pratyaswarupa, and you are also Parmatma and hence attain Samarasya akyabhavam.

Shruti:–

[॥ Maya Vidya – Vihayaiva – oupadhi – Parajeevyo~। Akhanda – Sachitanandam – Param – Brahma – Vilakshitey~॥.]

Adhiyatmo Upanishad.

Tatparya:– Hey Anjeneyaney! Maya's reflection (Pratibimba) of Ishwara Anthakarna's Pratibimba Jeeva. Janana Marana, Sukha Dukha etc. all the Vikaras, it is not only upadhis but also that Pratyagatma Parmatma have no Vikaras. They are Akhanda Sachitananda. So yo are being a Jeeva is Just Branthi. You decide that you are Pratyagatma swarupa and you an also Parmatma and thus attain Akyabhava.

Shruti:–

[॥ Tatsytam – Satma Tatvamasi – Shwetaketo~॥.]

Chandogya Upanishad.

Tatparya:– You are atmaswarupa Brahma That, Parabrahma atmaswarupa is You, This is the Truth.

Shruti:–

[॥ Tatvamasi – Tvantadasi – Tvam Brahmasmi – Aham Brahmasmi – Tyanu – Sandanam – Kusyeth~॥.]

Painola Upanishad.

Tatparya:– You are the Atmaswarupa Brahma that ParaBrahma is You. Therefore, let apart all your thoughts and doubts and therefore you keep doing Atmanu – Sandhana that you are Atmaswarupa brahma.

Shruti:–

[॥ Yatparam – Brahma – Sarvatma – Vishwasayatanam – Mahat~। Sukshma Tsukshma – Taram – Nityam – Tatvameva Tatvameva – Tath~॥.]

Kaivalyo Upanishad.

Tatparya:– In this world, the basis is Sukshma, atisukshma, Nitya Sarvatma rupa that Parbrahma is you. You are that.

Shruti:–

[॥ Jagratsyapna Sushupthyadi – Prapancham – Yatkripashatey~। Tat – Brahma Niti Gnanatva – Sarvabandhyai: – Pramuchatey~॥.]

Kaivalyo Upanishad.

Tatparya:– Sakshiswarup Atma ParBrahma is the one present in wakefulness, Svahujati and dream states. To attain Mukti from birth and death cycle, one must understand that he himself is Atmaswarupa ParaBrahma.

Shruti:–

[॥ Tat – Brahma – Sa – Atma~॥.]

Taitreya Upanishad.

Tatparya:– That Brahma Antakarna Sakshi Atmaswarupa.

Shruti:–

[॥ Pragnana Meva tat – Brahma – Satya – Pragnana Lakshanam~। Yevam – Brahma – Parignanadeva – Marbyo Mrito – Bhaviti~॥.]

Maho Upanishad.

Tatparya:– Dehindriyadis, Antaknas, their Vritis, Avastatrayas of all these what is understood is only Gnanaswarupa that is Atma. The one who comprehends this is a NityaMukta Parabrahmaswarupa.

Shruti:–

[॥ Yadahaivesha – Yetasminna Drishey Anatmayai Niruktey – Nilyaney abhayam – Pratistam – Vindatey~॥ Athaso abyam – Gatho – Bhavati~॥ Yadhai Vaisha – Yetasmi – Nudaramantaram – kurutey Atha – Tasya – Bhayam – Bhavati~॥.]

Taitreya Upanishad.

Tatparya:– One with utmost concentration and atma dhyana dwells, on the one who is adrishya – genderless, Avajanma – who cannot be seen, Mayatita ParBrahma, Atmasvarupa he will be “Abhay” – It means the following – he will be holding the hands of the Parabrahma – Symbolic – becomes a Nitya Mukta. The one who thinks that Atma is different and Brahma is different that one will get into the hassle of birth \& death, drawn himself in the SamsaraSagara will be sorrowful.

Shruti:–

[॥Ya Yevam – Vedham – Brahm: – asmiti – Sa – Idam – Sarvam – Bhavati~॥.]

Brhadaryanyako Upanishad.

Tatparya:– The one who thinks himself as atmasvarupa he will be atmasvarup.

Shruti:–

[॥ Ayanatma – Brahma.].

Mandukyo Upanishad.

Tatparya:– The antakarana in the body for which the atma is Sakshi is Brahma.

Shruti:–

[॥ Tvameyaham – Na Bhedosti – Purnatvath Parmatmanaha~॥.]

Mandala Brahmano Upanishad.

Tatparya:– Hey Anjeneyaney! You are me and I am You. Both of us are Sarva Paripurna Parabrahma – Atma – swarupas. Both of us are Yeka Parabrahma. Truely, there is no difference between the two of us.

Omkara Niranaya

Shruti:–

[॥ Sarva Veda – Yat padamananthi~। – Tapamsi Sarvanicha – Yadvadanti~। Yadanchantho – Brahmacharyam Charanti~। Tathey Padam – Sangrahena. Bravivyo – Mityai – Mityetat~॥.]

Khato Upanishad.

Tatparya:– The Vedas, Rg Yajur etc. all the Upanishads all these granthas explains the Parabrahma, so also all the Tapas on the Parabrahma and the students who go to the Gurukula to study the scripttnes, I shall collectively tell you that is, Omkara.

Shruti:–

[॥ Astangachha – Chatistadam – Tristhanam – Pancha daivitam~। omkaram Yo – Na janati – Nabhaveth – Brahamanastu – Sa:~॥.]

Dyhanambi Upanishad.

Tatparya:– 8 organs, 4 Padas, 3 Sthanas, 5 Devatas all this is Omkara. The Maya Pratibimba is the Mahakarana for Ishwaratatva. The Ishwaratatva Sthula and Sukshma are the reasons for the 3 bodies, Maya Pratibimba mahakaranas Sharira Collectively these 4 Shariras is Ishwara Sambhandas. The Antakarana Pratibimba in the Jiva Tatva is known as Mahakarana. Jivana, Sthula Sukshma are the causes of 3 Shariras, Antakarana Pratibimba Mahakarna Sharira all in all these 4 Shariras are Jiva sambhandas. These Jivasharas Shariras 8 of them are the 8 Angas.

\end{tabular}{}
These are known as Tri Sthanas.

\end{tabular}{}
These are Chatustadas.

Vivera of Pancha Devatey

(1) Brahma – vishnu – Rudra – Maheshwara – Sadashiva. These 5 are Panchabhuta Adidevatas.

(2) Bhumi – Jala – Agni – Vayu – Akasha. These 5 elements are Panchabhutas.

(3) Na – Ma – Shya – Va – Ya. These are 5 Panchaksharas.

(4) Prana – Apana – Vyana – Udana – Samana. These 5 are Panchapranas.

These are known as Panchadevatas.

Hey Anjeneya! All the above mentioned are collectively is “Omkara”. All the above mentioned are in the body or Deha thus the body itself is Omkara Saumyeney.

Sixteen Kalas of Omkara Arts

The 16 Omkara is divided. This is because while in Jagrath it is awake, Swapna in Jagrith, Sushupti in Jagrith, Turiya in Jagrith. – In Swapna – Jagrith, Svapna in Swapna, Sushupti in Svapna, Turiya in Swapna. Jagrith in Sushupti, Svapna in Sushupti, Sushupti in Sushupti, Sushupti in Turiya: – In Turiya Jagrith, Svapna in Turiya. Sushupti in Turiya – Turiya in Turiya. There are 16 Omkara Vidhas.

(1) Jagrith in Jagrith that is:– It means one forgets, when one is hearing some Sangati, or in Vedanta Shravana or while listening with attention any important subjects or topics and then forgetting all of it.

(2) Svapna in Jagrith that is:– While one is reading some Grantha, or listening to it with discussion on it, these subjects part of it is in memory, and part of it is forgotten, or is unable to narrate the same to someone.

(3) Sushupti in Jagrith means:– While someone is lecturing on Vedanta Shastra, one sits there to listen but there is no focus or concentration and the mind is elsewhere and so fully forgets what he has heard.

(4) Turiya in Jagrith means:– He does Shravana of Vedanta from a Guru, he attains Tatvagnana. He sits in solitude and does Chintana of all that the Guru has taught him.

(5) Jagrith in Svapna:– When one wakes up from a dream and forgets everything he has dreamt.

(6) Svapna in Svapna:– When one wakes up from the dream, he remembers some of it and does not remember everything he dreamt.

(7) Sushupti in Swapna means:– When he wakes up from dream state he has forgatten the entire dream.

(8) Turiya in Swapna means:– After dreaming he does Vichara on the (Kilu?) he is neither happy or unhappy and feels that all these are Mano Vikasas and decides to contemplate on Brahma Vichara.

(9) Jagrith in Sushupti means:– After a sound sleep he gets up and tells others that he slept very well.

(10) Svapna in Sushupti means:– After Sukha Nidra and is unable to recall that he had a peaceful sleep.

(11) Sushupti in Sushupti means:– After a sound sleep and does not know about it.

(12) Turiya in Sushupti means:– After a sound sleep and gets up, it is not an Agnana aspect, but feels and thinks that it is a witness (Sakshi) for Satyagnananda atma is mine.

(13) Jagrith in Turiya means:– While he is in Brahmanista, he is not aware of the mundane aspects like hunger and thirst which one is aware in Jagratavasta he is in a PrapanchaVriti.

(14) Svapna in Turiya means:– Knows or unknown when one is in Brahmanista hunger Shrishayas, Prapanchavriti.

(15) Sushupti in Turiya means:– Without any Vikalpas, he is stable in Brahmanista.

(16) Turiya in Turiya means:– The above 15 different states, without any of it he is in Akhanda Sachitananda Paripurna state Apart from himself there is nothing different.

This 16th section is the highest state. There is nothing more than this special experience. These various 16 sections or sub–divisions is the Omkara, Prakriti or ParaparaBrahma.

Shruti:–

[॥ Omkara Prabhavam – Sarvam – Trailokyam – Sa – Characharam~॥.]

Dhyanachando Upanishad.

Shruti:–

[॥ Om Mithye tadakshada – Midamsarvam – Sarva – Momkara – Yeva~॥.]

Mandukyo Upanishad.

Tatparya:– Hey Anjeneyaney! It is only from Omkara, 3 lokas, Sa characharas, everything isborn from it. Therefore, Omkara is Sarvam. Please understand your Deha also is Omkara Swarupa.

Hamsa Bhava – Swarupa

Shruti:–

[॥ Odvarvam Prana – Muma – Yatya Panam – Pratya Gasyati~। Madhye Vaman – Masinam – Vishvey Deva – Upasatey~॥.]

Khato Upanishad.

Tatparya:– Hey Anjeneya! In all the Jeevajantus is the presence of Omkara that is in all the Shariras. In such a Sharira Pranavayu is present in the Rechakarupa. Odwarkamukha is all encompassing and is moving around. In the form of Adhomukha is the Apanavayu. Between that the two in the centre lies the Atma Brahma, Rudra and others Devatas are always present. This atma is ‘Hamsa’. From this Pranapanas keep moving around.

Shruti:–

[॥ Pranapana – Vashodevo – Hridhshyo Dhravam Prada vati~। Vamadakshina – Margona – Sanchatatva. Na drishyatey~॥ Akshipta – Bhiya dandena – Yathochhalti. Kandu – Khaa:~। Pranapana – Samakshipta – Sthadva jejjivo – Navishrameth~॥ Apanatkarsheti – Prano – Panaha – Pranascha Karshti~। Khagarajju – Vadityetad – Yojanati – Sayogavit~॥ Hakarena – Bahiryati – Sakarena – vastunaha~। Hamsahamsey – Tyomunumam – Mantram – Jeevo – Japati – Sarvada~॥.]

Dhyana Bindu Upanishad.

Tatparya:– Hey anjeneya! The children while playing with a ball, when they throw it goes up and down bouncing. Similarly, Pranapanavayus both of them it remains in the central part of the body in Shadadhara goes up and down everyday (2,16,000) times though the Rechaka rupa through the nostrils 12, 8 upto 8 counts comes out and in. The breath is in Ida named as ChandraNadi, Pingala – as SuryaNadi through this path, breathing takes place several times. Just as when a bird is tied with a rope and the movements it makes, the rope two moves around. Similarly, these Rechaka Puraka Pranapanas placed in the centre part of the body pulls it up and down.

This Jevan through its path Hamsa Keelaka “Soham” – the mantra keeps chanting on its own. Atma individual who knows this secret know the Rajayoga others will niot know about it.

Shruti:–

[॥ Gnanatsvarupam – Parmam –Hamsamantram – Samucchareth~। Prananam – Dehamadhyetu – Sthitho Hamsadachyutha:~। Hamsa Yeva – Param Satyam Shaktikam~। Hamsa Yeva – Param Vakyam – Hamsa Yevatu – Vydikam~। Hamsa – Yeva Parorudraha: – Hamsa – Yeva – Paratparai~॥.]

Brahmvidyo Upanishad.

Tatparya:– The atma resides in the centre of the Jeeva jantus in all of them of in the form of Hamsarupa.

Shruti:–

[॥ Soha – Mavapo – Virajo – Nirmukta~॥]

in the form of Taitriyo Upanishad and also

Shruti:–

[॥ Yosa – Vasau – Purusha – Ssahvismi~॥]

as also according to the Isavasa Upanishad is “Soham – Soham”. It means – I am Brahma, Brahma is me and in this way one always does the Japa on its own. Without much effort. In the bodies of Brahma, Vishnu, Maheshwara and other devatas, this Hamsa keeps Sanchrutidey going in circles and thats how they are alive. Otherwise they all would have been Prethas! Therefore, this is the Shakti of Brahma Vishnu Shiva is understood – This Hamsa is Satya Swarupa: He is Parmatma. His voice is ‘Soham’ – This mantra is the most Shresta. The mystery of this is Sohambhava – this sohambhava leads to BrahmaAkya and so also the cause of it. This is the Gopya of Sadguru. This is the sum and substance of the 4 Mahavakyas. This is Ajapa Gayitri. This is the Mauna NIrguna. Atma Dhyana. This is Atmarama Taraka. The one who understands this Hamsa Stotra is the real true Brahmana. These people are Brahmagnanis. Just as one traps the naughty monkeys who jump around, they tie them up with a rope \& bring them. Similarly, many Vishyas (subjects) are not concentrated as the mind is like a monkey skipping up from one thought to the other, this has to be brought under control by Hamsa Mantra and prevent it from roaming around remain sernienly in the mantra. This is the only way to bring the chanchala mind in a strong grip.

Shruti:–

[॥ Naskham – Vapanam – Kritva – Bahisyutram – Tya – Chedubdha:~। BrahmaSutra – Midam – Gneyam – Sutram – Nama – Paramdam~। Tatsatram Dharayedyogi – Savipro – Veda Paragaha~॥.]

Bramo Upanishad.

Tatparya:– One will not accrue any benefit by having a tikki on the head or donning a Janwara on the body. The thread – janwara made by man and wearing it does not make one Brahmin.

It on the other hand, that which is created by Deva Peepali, is Brahmaparantharu the Hamsa Sutra encompasses all the Jeevas, this is the true. ‘Janavara’ “Brahma Sutra” or “Gnanayagnopavita” The one who understands the above mentioned maneins is the True Brahmana.

Shruti:–

[॥ Hamsahamseti – Yo – Bruyath~। Hamso Nama – Harishya va:~॥ Guru Vaktrathu – Labhyte – Pratyaksha – Sarvathomukham~॥.]

Bramo Vidyo Upanishad.

Tatparya:– Hey Anjeneya! This Hamsa rupa which is responsible for Brahmaikya, this Sanchara Rahasya one who understands that Dhyana Gnanas, Hari Haras, \& devatas, he becomes equal to them. This, Paramparana Hamsa Marga with the Guru's katakshya is the only thing which is frenfull. Any other method however, hard one may attempt does not yield success.

Sanaka, Sananda, Narada, Vasista, Parashara, Vyasa, Shuka Shaubnika, Rubhu, Nidhaga Jada Bharatha Janaka Maharaja were all the extraordinary class of people who understood the above mentioned Margas which helped them to attain Brahmasvarupa and thus received Moksha. This is the true TarakaYoga. Some people do not understand this and think that the Panchakshari, Astakshari Japas is adequate, Bhuchari Kheychari, these are Panchamudra and chitta is got. Dashavida Nada – hearing of this and say this is Tarakayoga which is not right.

Shruti:–

[॥ Garba janma – Jaramarana – Samsara – Mahadbha yath~। Santarayati – Tasmatharakam.~॥.]

[Anvaya Tarako Upanishad.]

Tatparya:– ‘Taraka’ is obtained by one who has crossed Janana marana Maha Dukha sagara reaches the level of Moksha ‘I am Brahma’ ‘Brahma’ is me – it is Brahmah Aikya. This “Sohambhava Dhyana” releases one from the cycle of Janana marana and enables one to get Moksha.

Panchakshari and other Mantras, keep in one's concentration between the eyebrows Dashavidanas to realise that Brahma is one, the above mentioned aspects are the Lakshanas (Signs) – which makes one to conclude that ‘I am Brahma’ ‘Brahma is me’ – this Aikya Gnana. Yet, the above said ideas are not Tarakayogas. Therefore, the true path is Hamsa Marga, and this is Tarakayoga. So Sri Ramachandra tells his beloved Bhakta Anjeneya that the word ‘Soham’ – Hamsa Swarupa. Sanchara Rahasya has lectured him in the true form (Pratyaksha rupa).

Manasa Lakshana

Shruti:–

[॥ Chittam – Sanjayetey – Janma – Jara Marana – Karanam~॥.]

Muktiko Upanishad.

Tatparya:– Hey anjeneya! The mind is the cause of Janana Marana.

Shruti:–

[॥ Maneva – Manushyanam – Karanam – Banda – Moksha Yoho:~। Bandaya – Vishayasaktam – Mukteyai – Nirvishyam – Smayitam~॥.]

Shatyani Upanishad.

Tatparya:– The mind is the cause of Bandha and Moksha. If the mind gets attached to various aspects – or Vishayas, it causes Bandana. If he is detached on the other hand then he attains Moksha.

Shruti:–

[॥ Atsarvam – Jagat – Chitta – Godharam~॥.]

Mandala Braahmana Upanishad.

Tatparya:– The manas sees the Prapancha and therefore there is Prapancha.

Shruti:–

[॥ Yatha – Rasashaye – Phenam – Mathanadeva – Jayatey~। Manonirmatha deva Vikalpa Bahavastha~॥.]

Trishiki Brahmano Upanishad.

Tatparya:– While churning milk curd etc. foam arises, similarly, when the mind or Manas is shaken different kind of the universe is seen, there is an illusion or Brahmey.

Shruti:–

[॥ Sahastarankura – Shakatma – Phala Pallava – Shaalinaha:~। Asya – Samsara – Vrikshasya – Mano moola – Midam – Sthiam~। Sankalpa – Yeva – Tanyanyey – Sankalpo Pashamena. Tat Shosha – Yashid – Yatha – Shoshometi Samsara Padapaha:~॥.]

Muktiko Upanishad.

Tatparya:– The manasa or mind is the root of the tree of Janana marana Swarupa Samsara~। Sankalpa means various thoughts occurring in the mind. Therefore, if the thoughts gets destroyed and the mind also gets destroyed and the tree of samsara by itself will also get destroyed or Naasha.

Shruti:–

[॥ Upaya – Yekayevasti – Manasva Syanigrahey manasebyadayey – Nasho Mano – nasho – Mano Nasho – Mahodaya:~॥.]

Muktiko Upanishad.

Tatparya:– The only method to destroy the mind is to see that the thoughts are not allowed to surface.

Shruti:–

[॥ Manashumnani – Bhavey – Dvyitam – Nmney – Vopa labhyatey~। Yada – Yatyoyanmani – Bhava – stada – Tatparam – padam~॥.]

Paingolopa Upanishad.

Tatparya:– Hey Anjeneya! Manas (mind) is known as Maha Maye. Due to this questions like I, You, Jeevanu, Devan, Jagat Bhanda, Moksha all these initial natural phenomena like all the Daivata bhedas arises. This mind is Nitya Mukta Parabrahma and it makes one forget that and one becomes aware that it is body or deha. Which is an illusion.

This mind, Buddhi, Chitta, Ahankar, Kama, Krodha, Lobha all these qualities make the loka painful or tranbles. If this Chandala mind sublimates and Channelises towards Brahmanista, then he does not think of the world or Prapancha. Dvatya Branthi is not there. Then immediately everything becomes Brahma–anubhava.

Shruti:–

[॥ Sushaptan – Sukhamatrayam – Bheda: Ka – Naivalokita~। Chitta Moolo – Vikalpoyam – Chittabhavey – Nakashchan Aatvitam – Samadehi – Pratyagrupey Parmatmuni~॥.]

Tatparya:– Hey Anjeneya! During sleep the manasu is in the darkness of Agnana. Just as a seed is sown in the soil or bhumi and it sprouts up, so also in the Jagratavasta, the mind is up and one becomes conscious of it and is not aware that during sleep that he is Brahma. This does not place during NIdra as the mind is an Agnanavada because it is drowned in the darkness in a forsaken well \& hence blind. Therefore, Jeevan is unable to realise Moksha or Mukti. In Brahma Nistey, the mind is in Gnana agni, and just as a seed sown or falls in the fire, it is unable to sprout and thus the seed gets destroyed, this mind also gets destroyed. Therefore, the gnani who does Brahmanistayam that the illusion of Dvaitya gets destroyed. Thus the Prapancha is not there, that he is only a witness, Svayam Prakasha Brahma. Hey Saumyeney, for this reason between Nidra and Samadhi there is a lot of difference Bheda. Then Anjeneya addresses, Swami Ramachandra in Brahmanistey, the mind is obscure, if that is the case then the Gnani who is aware of the Brahmanista, why does he return to the worldly life, setting aside the Nista becomes awake – Nana Vyaparigalani and teaches others of Atmagnana. Is it not the mind which does all these things. How do they come to know that the Manus or mind is destroyed?

To Anjeneya's Sukshma Buddhi, Sri Ramachandra replies. Tum Baa Santoshi. Hey Anjeneya! You are very intelligent. People who question \& clear their doubts, become Muktidatas others do not get Mukti. Please question me further.

Shruti:–

[॥ Dvividha – Shvita Nashosti – Sarupo – Rupa Yevacha~। Jeevanmuktan Sarupsyath – Arupo – Dehmuktya:~॥.]

Muktiko Upanishad.

Tatparya:– Hey Anjeneya! There are 2 types Sarupa Mano Nasha, Arupa Manonacha. For a jevana mukta, Sarupa Mano nasha and in Vidhaha mukti Arupa Mano Nasha takes place.

Shruti:–

[॥ Chitta Nashabhidhanam – Hi – Yadatey – Vidhatey – Punaha~। Maitradibhi – grunai yuskta: – Shanti Meti – Nisamshaya~॥ Bhuyo – Janma – Vinirmuktam – Jeevanmuktasya – Tanmyanaha: Sarupo sau – Mano Nasha – Jivan Muktasya – Vidyathey~॥.]

Muktiko Upanishad.

Tatparya:– The manovikaras are those like Ahankara, mamkara, Sankalpa Vikalpas. When all these get destroyed the manas attains Nirvikara, Kartva Shunya. The Jeevas Maitri and all the Sadgunas, and there is Parmashanti – This is called Manonasha. Just as the Clothes which are of no use when it gets burnt in the Agni, so also the Brahmanista gnani even when he is aware of the manas, he keeps dowa all the actions, the manas gets burnt in the gnanagnagni, it remains dormant and he is engaged in Nirvara duties. Therefore, that Sarupa is Manonasha. For such gnanis there is no rebirth.

Shruti:–

[॥ Prapanchasya – Manaha: Kalpita – tvat – Tato Bheda – Bhavath~। Kada – Chitbhahirgatepi – MithyaTva bhavat Sakridvi bhata – Sadanandanubhavyaika Gocharo – Brahma vitha deva Bhavati~॥.]

Mandala Brahmano Upanishad.

Tatparya:– Charastri – Prostitute? do not know. Though she is having pleasure with her man, yet her mind is on someone else and that remains there as if it is Gnapti? (do not know the meaning).

In Brahmanistey and for the mind on it the Gnanis who are Brahmanandas again they awake from the Nista and feel the mind and though they conducting the mundane aspects of life, yet they know that the mind is imaginary and appears to be like a dream they are aware it is Mithya and hence come to form decision and realise that he is Brahma and get into Tadekadhyana. Hey Saumya! thogh they are seeing the Jagat, there is no Dosha.

Shruti:–

[॥ Arupastu – Manonasho – Vaidehi – Muktiko – Bhavet~॥.]

Muktiko Upanishad.

Tatparya:– Arupa – Manonasha means – A burnt cloth, if one touches it that slips out of the hand in pieces \& fly off in the air. Without Rupa the mind also fully gets destroyed. This is known as Vidheya Mukti. It means – once the Ghata is beaten the Ghatakasha gets merged in Mahakosha and becomes one, So also when Prarabdha gets destroyed, the gnana's Manusu and the 24 Tatvas, the 3 Dehas also lose their rupa and fully get Nasha and he becomes Brahmaikya. This is known as arupa Manonasha.

Shruti:–

[॥ Sarvam – Brahmeti – Sankalpat Sudridha – Numchaitey – Manaha.~। Krishoham – Dhukha – Baddoham – Hastapadadivanaham~। Iti Bhavanu [Bhavna] Rupena – Viyarharena – Bhaddyatey~॥.]

Tatparya:– Everything is Brahma. That Brahma is Atma, the Atmasvarupa is myself Brahma. Apart from me there is nothing else. I am all in all. This is a form decision. If he becomes Brahmakara then the mind by itself gets destroyed. I am Krisha, I am unhappy, I am a body with the limbs, I am Jeevanu if such a Vyavahara takes place then the mind vanishes. If it does not get destroyed then Janana Marana cycle is unavoidable.

Shruti:–

[॥ Svatmanaiva – Sada – Sthithva – Mano – Nashyati Yogina:~॥.]

Adhatmyo Upanishad.

Tatparya:– The body, Prapancha etc. for the Jeeva is “Neti Neti”. I am not I am not he repeats now and then (Pragnanam Brahma) in the form of a Shruti the Shuddha understanding of the above aspects is known as Brahma, that awareness that he is Brahma, he gives up all the desires, he is only Shuddha self alone Svanubhava Nistey he achieves, then the manusu will get destroyed by itself.

Shruti:–

[॥ Yachitam – Tenishya – Prana Mayati – Prana – Stheja Sayukta:~। Sohatmana – Yatha – Sankalpitam – Lokam – Nayati~॥.]

Prasno Upanishad.

Tatparya:– Whatever thoughts and feelings one is thinking about at the time of death, accordingly, he will get gnana in the next birth. Therefore, while one is alive it is better to always think that he himself is Brahma, if he is in the Sate of Brahma Bhava, and from this firm thought, he will be Brahma Swarupa towards the end.

Shruti:–

[॥ Nahi pramanam – Jantu Namukta Rakshana – Jeevitam~। Charmashvara – Vethayam – Yatkritam Tat – Sada – Kuru~॥.]

Tatparya:– In this loka all the people is unaware what happens the next moment whether one will be alive or dead Many a time death happens unknowing, unexpectedly suddenly death comes and the Prana is gone. Therefore, when this ucchavasa and Nishvasha one must be aware that it may be the last breaths and then else keep uttering ‘Aham Brahma’.

Brahma Sakshatkara

Shruti:–

[॥ Ahampteti – Yo bhavo – Dehakshada – Vanmatmne~। Abyasoyam – Nirastvyo Vidusha Brahmanistaya~॥.]

Aadhyatmo Upanishad.

Tatparya:– The Ignorance that I am the body my eyes, nose and the pride or abhimana of the body then most importantly one should akandon Brahma Gnana. When others point finger's as do Dushana and other Bhushana Kula Gotra, Jati, Neeti, Sukha – dukha all belongs to the body and not to the Atma Svarupa which you do not possess at all. It means – You being the Atma Swarupa you are not the body, not a woman or man, eyes, nose ears all belong to the body. You being the Atmasvarupa have no Avayavas, only Nirakara Saumyaney!

Shruti:–

[॥ Gnanatva – Svam – Pratyagatmanam Buddhi – Tadvavrithi – Sakshinam~। Sohmilyey Va – Tadvritya – Svanya Tratatma – Matim Tyeyatey~॥.]

Aadhyatma Upanishad.

Tatparya:– The body, Manas, Buddhi and its nature, the Jeevas Gnana knows that is sarvasakshi or it is called AtmaBrahma, that it is shuddha as also Shuddha Chaitanya, as Pragnana, Shuddha Gnapti, as Kutastvam, it is Achala, all the above Gnanavey that is you and be immersed in these thoughts. You are not the worldly Prapancha, this idea is firmly established, and do not be carried away by illusions.

Shruti:–

[॥ Dristim – Brahmaneyam – Kritva – Pashyet – Brahmanayam – Jagat~। Sadristi – Parmodara – Nana – Sagrava – Lokini~॥ Drastru – Darshana Drishyanam – Viramo – Yatrava – Bhavet~॥ Dristi – Tatryva – Kartavya – Nana sagrav – Lokine~॥.]

Tejobindu Upanishad.

Tatparya:– Hey Anjeneya! Make the Dristi gnanamaya, that Brahma should not be seen as Jagatrupti but as a Sakshi in the Nasagra – It means – Do not keep your dristi between the eye brows. (1) The Jeevana sees the Prapancham as “Drastru”. (2) The Jeeva Sambandha sight or seen as “Darshana”. (3) All that is seen by the Jeevana is known as “Drishya”.

This, Drastru, Darshana, Drishyas all the 3 is known as Triputi. This Triputi is not Atma. From this triputi the gnanaswarupa Parabrahma took place, you immense in that Parabrahma you must be yourself, one must not set the Dristi at the tip of the nose. It means – The “Pragnana” is that which tries to understand the body and the Prapancha – It means – Shuddhat Aravey is Brahma. You immense yourself in that gnana and remain there that is the Bhava.

Shruti:–

[॥ Yasya – Vagninasishuddhey – Samyagnipteycha – Sarvada~। Savyai – Sarvamapnoti – Vedantatopa – Gatam – Phalam~॥.]

Narada Parivrajako Upanishad.

Tatparya:– One whose mind is not Chanchala, and the Vak (speech) is Mauna, the Indriyas, it is Svechey and remains there, that which is Parishuddah he will experience the Atma anubhava as laid down in the Vedanta.

Saying thus, Sri Ramachandra becomes merciful seeing his ardent bhakta, Anjeneya and addresses him Leave aside all your doubts, and you are atmasvarupa Brahma. You have a darshan of the Nija – swarupa. Sits in Chinna mudra and through his Sukshma (subtle) bhuddi Anjeneya with firm Vairagya, for gets all the Sarva Sankalpas and he by himself and tries to understands the Brahma Nisteyam only as a Sakshi – witness. After a while he attains Prapanchagnapti and keeps his on the lotus feet of Sri Ramachandra and sheds anandabhasya and with his tears does abhishekha of the Lord's feet. He got up puts his hands up \& does Danda and addresses Swami.

Ramachandra! You as a Sadguru, as Bhagvanta an God I prostrate and do Namaskara and God's Kripakataksha please see my true swarupa, I have become Pavana, and starts praising the Lord and the ever merciful Sri Ramachandra lays his hands on Anjeneya's head and makes him sit near him. Hey Anjeneya, what I have taught you earlier, the Hamsadhyana, is known as Tarakayoga, Savikalpasamadhi, as also Sanchara samadhi. This Bhava is also known as Sahaja mansikayoga, Nirvikalpa Samadhi, Brahma Sakshatkara. While you are in the Prapancha Vyavahara be in. Hamsadhyana, Savikalpasamadi and, in Ekanta or while you are alone, Sahajamanaska known as Nirvikalpa Samadhi and enjoy the Ananda.

Gnanastangas

Shruti:–

[॥ Deheyindriyeshu – Vairagyam – yama – Ityuchaitey – Bhuddhai:~। Anurakti: – Parettvey – Sattatam – Niyama – Smilyaha:~॥ Sarvavastu – Nyudasina – Bhava – Masanamnvitamam~। Jagat – Sarva – Midam – Mithya – Pratiti – Prana Samyamai:~॥ anthasya – Antarmukhey – Bhavaha: Pratyaharastu – Sathama~॥. Chittasya – Nishchalibhavo – Dharana – Dharanam – Vidhu:~॥ Soham – Chinmatra Meyveti – Chintanam – Dhyanamucchaitey Dhanasya – Vismriti: – Samyak – Samadhi – Rabhidiyatey~॥.]

Trishikhi Brahmano Upanishad.

Tatparya:– Hey Anjeneya! Vairagya in Dehindriyas means – Not to allow these body organs to act as they like, one must be alert, this is “Yama”.

\item When one is always joyful in the Paribrahma is “Niyama”.

 \item When one is not decisions of objects is “Asana”.

 \item The entire universe or Jagat is just a dream and a Mithya and who is firm about is “Pranayama”.

 \item Not to allow the Chitta to see the external Vyavaharas or activities and turning oneself inwards is called “Pratyahara”.

 \item To keep the Chitta steady stable without allowing it to waver is “Dharana”.

 \item Satyagnana, Nandamaya, Sarvasakshi and realises that I am. Parabrahma is known as “Sohombhavey” “dhyana”.

 \item All the thoughts, all Drishyas, to forget them that Dhyana which makes one feel as only a Sakshibhuta is known as “Samadhi”.

 \item These are the 8 Angas. These will bad one to (Sadhyomukti). It means – You will gets Mukti at birth itself. The 8 Astangas of Yoga mores in order and helps one to attain mukti towards the end of Life and the above mentioned points are the Sadhanas.


Gnani Pujay

Shruti:–

[॥ Tasya Nishchinta Dhyanam – Sarvakarmini nisakarana – Mavahanam~। Nishcha Yagna namasanam~। unmani Bhava Padyam~। Sadamanaskanargham~। Sada deeptara Mritvrithi snanam~। Sarvatma Bhavanaghanda:~। Drikasva Rupa Vasthanamakshata:~। Chidapti: Pushpa:~। Chidagniswarupam Dhupaha:~। Chidaditya swarupam Deepaha~। Paripurnachandramrita sijayi Kikaranam Naividyam~। Nishchala Tvam Pradakshanam~। Sohombhavo Namaskara Maunam Sthuti~। Sarvasantosho Visarganamiti~॥.]

Mandala Brahmano Upanishad.

Tatparya:– Dhyana is being Nishchintha

\item Sarva Karmagalam to abstain from it Avahana.

 \item Brahma is Satya, Jagat is Mithya, this knowledge is Asana.

 \item Unmantra bhavey Padya.

 \item Sada to be in Anuskabhava is Argyam.

 \item Sarvada to be Gnanakara is Snanam.

 \item To feel and think that everything is Brahma is Ghanda.

 \item To have one's Dristi always on Brahma is Akshatey.

 \item Atmadarshanam to have this intense desire is Gnapti – this is Puspa or flower.

 \item To be firmly rooted in Atmanistey is Gnanagni – this is also Dhoopa.

 \item “Neti Neti” means – I am not the body I am neither Manus nor Jeevan. The whole Prapancha is not me, to understand all this the Sarvasakshi is Atma this is Tatva Viveka – Gnanaswarupa and this Gnanaswarupa is deepa.

 \item Iam Brahma, Brahma is me, the Sarva Paripurna. Paribrahma to be Aikya with it is Naividya.

 \item Without temptation–Chapala and to be Nissam Kalpa – this is Pradikshina.

 \item Sohombhava is Namaskara.

 \item Nirvasana maunavey – is Sthuti.

 \item Sarva Santosha is Visarganey.

 \item These are all Gnaniyam the Shodaupachara Atmapuja.


Samadhi Lakshanas

Shruti:–

[॥ Ahameva – Param brahma – Brahmamiti – Samsthiti~। Samadhi – Sthuvigneya – Sarvavriti – Vivichreta:~॥.]

Trishikhi Brahmanoka Upanishad.

Tatparya:– Hey Anjeneya Atmaswarupa Parabrahma is myself, this aspect should be firmly established and make the mani Vritis shunya and become Aikyabhava without any desires forgetting all the Drishyas, and automatically, one is just Sakshibhuta is Samadhi.

Shruti:–

[॥ Dhyathai – Dhyney – Vihaya – Nivatsthita – Deepavat~। Dhyeyaika – Gocharam – Chittam – Samadhirbhaveti].

Paingolopo Upanishad.

Tatparya:– I am Brahma, Brahma is me, that is if one does ध्यानाद्द on this aspect, Dhyanarupavrithi, forget the two, and by not diverting oneself elsewhere, be steady and not swayed like that wind. Destroy all the Sankalpas, focus the mind on Brahma and feel that he is Brahmakara Nista is Samadhi.

Shruti:–

[॥ Sampragnata – Samadhi – Syasyat – Dhyana bhyasa Prakarastha Prashanta – Vrithikam – Chittam – Parmananda – Deepkam~॥.]

Muktiko Upanishad.

Tatparya:– I am Brahma. Brahma is me to sit alone and preventing the chitta from diverting its attention elsewhere and immense in Dhyna – This is Sampragnanata Samadhi or Sarvikalpa Samadhi.

Shruti:–

[॥ Asampragata – Namayam – Samadhi – Yorginam – Priya~। Prabhashunyam – Mamushyanam – Buddhi Shunyam – Chidatatmakam~॥.]

Muktiko Upanishad.

Tatparya:– I am Brahma, Brahma is me. Forgetting this dhyana, Manobuddhi becomes Shunya, all the thoughts should vanish as also the Dehadiprapancha, and all the Drishyas should be forgotten. Becomes fully immersed and be only Sakshi Asampragnata Samadhi, Nirvikalpa Samadhi.

Shruti:–

[॥ Agnana Hridayagranthey – Nirshesha – Vilaya. Sthatha~। Samadhinavikalpena – Yada – Advaityatma Darshanam~॥.]

Adyatmo Upanishad.

Tatparya:– Through Nirvikalpa Samadhi and Nistabhyasa, that which is Agnanamaya of all the heart's Granti gets destroyed – If means – Sankalpa Vikalpas are exhausted and the only aspect that remains is Advaya Brahma–Atma hence Sakshatkara happens.

Shruti:–

[॥ Samadhi – Nista Tameyatma – Nirvikalpa – Bhavanagh~॥.]

Adhyotmo Upanishad.

Tatparya:– Hey Anjeneya! By continuously practising Sarvada Nirvikalpa Samadi Nistabyhasa and keep trying Nirvikalpakara.

Shruti:–

[॥ Dharma Megh – Mimam – Prahur – Samadhim – Yogavittama~। Varshthyava – Yatha – Dharmavritadhara – Sahasrisha~॥ Amuna – Vasana – Jaley – Nishayam Pravilapithey~॥.]

Adhyotmo Upanishad.

Tatparya:– Hey Anjeneya! The elders have mentioned this Nirvikalpa Samadhi Dharmamegha. From this, Svanubhavananda amrita, is infinitely varshusi and the Deha Vasana, Shastravasana, and Lokavasana the heat of all the above cools off.

Vasanatraya

Shruti:–

[॥ Dhrida – Bhavanaya – Purvapara – Vicharanam~। Yadadanam – Padarthasya – Vasanasa – Prakrita~॥.]

Muktiko Upanishad.

Tatparya:– This Samsara anubhava has become firmly established since ages and Mindyavado, Hindyavado, this takes a turn without Viveka and one starts behaving this amitya Samsara anubhava, in which one strongly behaves, and he sets his mind on it and keeps remembering it always – This Gnaptiyam is called “Vasana”.

Shruti:–

[॥ Vasana – Dvidha – Prokta – Shuddha cha – Malina – Thatha –~। Malina – Janm Hethneyat – Shuddha – Janm Vinasini~॥.]

Muktiko Upanishad.

Tatparya:– Hey Anjeneya! Vasanas are the remnants of Purva–Janma's Samskars. There are 2 types of Vasanas. Shuddha and Malina. The Malina (Impure) Vasanas are cause of Jannana Marana. Whereas, Shuddha Vasanas destroys the Births \& Deaths.

Shruti:–

[॥ Agnana – Ghanakara – Ghana – ahankara – Shalini~। PurvaJanm kari – Prokta – Malina – Vasana – Buddhaiyi~॥.]

Muktiko Upanishad.

Tatparya:– Agnana Swarupa always makes ahankara – a Swabhava and this Malina Vasanas causes birth and death.

Shruti:–

[॥ Loka – Vasanaya – Janto – Shastra Vasan – Yapicha~। Deha Vasanaya – Gnanam – Yathava~। Neiva Jayathey~॥.]

Muktiko Upanishad.

Tatparya:– There are 3 types of Vasanas – (1) MalinaVasana Loka Vasana (2) Shastra Vasana (3) Dehavasana. These 3 Vasanas create agnana will not happen.

\item The good deeds done in the Loka and he receives praises for it and this desire is “Loka Vasanas”

 \item The Pandit who is very proud of his knowledge of the Shastras, and creates differences between Shiva and Vishnu. This creates enemities between the two as Shaivites and Vaishnavites \& a conflict between the two as to which is superior. This is known as “Shastra Vasanas”

 \item An individual who has become old, he has become bald, suffering from diseases and since he does not want death takes resort to herbal medicines, sidda medicines and restore his health as he is so much in love (abhiman) with his body – This is “Deha Vasana”

With the above mentioned Vasanas, the individual is unable to attain Atma Gnana. If one is absorbed in atma Dyhana and experiences atmanubhava known as Nista, Nirvakalpa Samadhi Nistey should be consistently practise and this Tria Vasanas will not only be destroyed but it not be the same to others.

Shruti:–

[॥ Lokanuvartanam – Thyaktva – Thyktva – Dehamvartiam~। Shastranaha: Vartanam – Tyktava – Svadhyasapa – Nayam kuru~॥.]

Adhyatmo Upanishad.

Tatparya:– Lokavasana, Shastravasana, Dehavasana all three Vasanas keeps coming now and then like I am Jeevam. Sukhi, Dukhi, Brahmana, Shudra all these illusions one should destroy.

Shruti:–

[॥ Nama Yoni – Shatangatva – Sheteysau – Vasana Nashaka~॥.]

Trishiki Brahma Upanishad.

Tatparya:– Due to the Tri – Vasanas the Jeeva takes birth in different Yonis.

Shruti:–

[॥ Janmantara – Shatabhyasta – Mithya – Samsara Vasana~। Sachira abhyasa – Yogena – Vina – Sakshiyatey – Kvichitu~॥.]

Muktiko Upanishad.

Tatparya:– From several births one is addicted to the illusion, the Samsara Vasana, Jeeva Brahmaikya gnana Yoga without practice will not be destroyed.

Shruti:–

[॥ Gnanano – Nashamambhyeti – Mano agnanasyati – Shrinkala~। Tavami Shiva – Vetala valagninti – Hridi – Vasanaha: Yekatatva – Dridabhyasa – Dyavanna Vijetam – Manaha~॥.]

Muktiko Upanishad.

Tatparya:– A gnani's mind will by itself get destroyed. An agnani's mind gets fettered in Chains. As long as one gets victory over the mind by Brahma Nista abhayasa, such a one during the nights will not be troubled by Vetalas or devils who roam around the planet. So, in an agnanis heart the Betala of Vasanas, and the Pishatas who move around in the planets, he will be entangled in Janma, Jaramarna and sorrows.

Shruti:–

[॥ Sarvatra – Sarva – Tasarva – Brahma Matrava Lokanam Sadbhava – Vasana – Daradrayath – Vasanalaya – Mashnutey~॥.]

Adhyatko Upanishad.

Tatparya:– The Shuddha Vasana is that as when one sees with the eyes, hears with the ears, and touches with his hands all these he considers as Brahma and he firmly believes in this and so this is Shuddha Vasana. From this all the Malina Vasanas are destroyed.

Shruti:–

[॥ Asanga – Vyavahartva – Dbavabhavana – Vasanath~। Sharira Nasha – Nashadarshitva – Dvasana Napravartatey~॥.]

Muktiko Upanishad.

Tatparya:– Ghatodhgana Akasha that Ghata and its Vikaras does not attached. Asanga Atma – swarupa of mine without getting attached to the Body's organs and without getting its Vikaras, just being Asanga Sakshimatra and being atmaswarupa I am Brahma and Brahma is me and thus get the Aparoksha Gnananubhava.

\item To understand that the Samsara is mithya and firmly behave in Vairagya.

 \item The body is destroyable this aspect one must understand.

 \item These will help to destroy the Tria Vasanas will be destroyed and by no other means these Vasanas can be destroyed.


The Lakshana of Vairagya Bhoda Uparatis

Shruti:–

[॥ Vasananudayo – Bhogyai – Vairagya – Tadavadhi~॥.]

Adhatmo Upanishad.

Tatparya:– Vairagya means to be Nirasha – disappointed. One will attain Vairagya only when one gets detached and knows that is all an illusion like flowers, ghanda, wife, children, wealth, house etc.

Shruti:–

[॥ Aham – Bhavodayabhavo – Bhodasya – Parmavadhi~॥.]

Adyatmo Upanishad.

Tatparya:– Bhodha means, the entire universe is mithya, Atmasvarupa who is myself is only the truth – that is Brahma which is understood by Talva Vichara and thus enjoys Ananando or Bliss. Here, there is no Ahankara, Mamkara, Deha abhimani, such an individual will attain Bhodha.

Shruti:–

[॥ Lenavrithey – Rasutpathi – Maryadopa – Ratestu – Sa~॥.]

Adhyatmo Upanishad.

Tatparya:– When the Sankalpas and Vikalpas are destroyed, and the Sankalpas do not rise up again, and one is always immersed in Atmasandanam (Research). This is called Uparati.

Shruti:–

[॥ Sthita Pragno – Yatirayam – Yasya daAnanda – Mashnutey~॥.]

Adhyatmo Upanishad.

Tatparya:– Hey Anjeneya! This Vairagya, Bhoda, Uparatis etc, whose mind is not Chanchala, \& who has Pragney will experience Brahmananda.

Bhanda Muktas

Shruti:–

[॥ Anadya Vidya – Vasanaya – Jatahamiti – Sankalpo – Bhandah~। Pitra Matra Sahodara Darapatya – Graha Rama – Kshetradi – Mamata – Samsara. Varna – Sankalpa – Bhandaha~। Kartitvadya – Hankara – Sankalpo – Bhandaha~। Animadya Shestaitvarya – Aashasidda. Sankalpa, Bhandaha~। Deva manushyadiupasanekama Sankalpo Bhandaha~। – dharma Upasanakama – Sankalpo Bhandaha. Yamadi – Astanga – Yogabyasadi – Sankalpo Bhandaha~। Varnashrama – Dharma – Karma – Sankalpo Bhandanaha~। AgnaBhaya – SamshayatmaGuna – Sankalpo – Bhandaha~। Yagna Vrata – Tapodana Vidhi Vidhasana Sankalpo – Bhandaha~। Kevala – Moksha – Peksha – Sankalpo – Bhandaha~। Sankalpa – Matra – Samabhavo – Bhandaha~॥.] [Nityaa Nitya – Vastu – Vicharada Nitya – Samsara Sukhadukha Vishaya – Samasta – Kshetra Mamatadi – Samasta Bhandaha Sankalpakshayo – Moksha)~॥.]

Neeralambo Upanishad.

Tatparya:– When one does not understand that Satyagnananda. ParaBrahma, as from Adikala he is agnanamaya Jeevan and this Sankalpa is Bandhana. Without understanding that he himself is Sarva Sakshi Brahma. He thinks parents, brothers, wife and children, House, garden etc. are all Bandhanas as a Samsari. He is unable to realise that he is Shuddha Nirguna Brahma. I am me and he is filled with Ahankara and Mamakara, I am the Yajamana Me and my wife and keeps nourishing these concepts and this Sankalpa is Bandhana. That he is that ParaBrahma who is herfit of birth and death. So He is scared of death and attains Kayasiddi and to bring sayadam and Animadyasta Siddhi, this Sankalpa is Bhandana. He is unable to comprehend that he is the Sarudhara ParaBrahma, he does Dhyana on Hari and other Devatas, in order to fulfill his Istarthas (desires) is Sankalpa Bandhan. He or the individual is Nitya Mukta, Nirvikara and he is Advaya Brahma. He forgets all this and as Sakshibhuta experience anananda, and feels thats apart from him is Brahma, even though he does the Astanga, and concentrates between his eyebrow and adorns himself with the Panchamudras and also the Dashavidha Nadas and obsense the Chittakala Mandala, and thinks that all these are ParaBrahma – this Sankalpa is Bandhana.

He is unable to imagine that the ParaBrahma does not possess – Namarupa, KulaGotra, Varnashrama (Caste divisions) all these acharas this Sankalpa is Bandhana. 

Just as the ParaBrahma does not have Tapatrayas, this he not understand that he is that Para Brahma. He is filled with fear of Agna, doubts all this Sankalpa is Bandhana ParaBrahma is Karma Shunya which the individual but he is ignorant about this and gets himself in Japa – Tapa, Vrata, Yagna, Dana in the style of Karmakanda without fail he performs all those activities – this is Sankalpa Bandhana.

He feels that the Deha's indriyas are different, and fails to understand that. Just like ParaBrahma – he is Nitya Mukta Asharira (without body). In order to attain Moksha after death. This too is Bandhana.

Nityaa Nitya Vastu, he gets Viveka that Brahma is Satya, Jagat is Mithya. That wife and family is Amitya, Property, wealth he goes beyond all these and that joy and sorrow belongs only to the body and these do not belong to him as he is only Sakshibhuta and Atmaswarupa ParaBrahma and in this fashion destroys all the Sankalpas and immenses himself in the Brahmananda sagara – that is Moksha.

Shruti:–

[॥ Yasya – Sankalpanashasyat – Tasya – Muktihi: Kasasthita~॥.]

MandalaBrahmano Upanishad.

Tatparya:– An individual who destroys all the Sankalpas and does not think about it for such a being Moksha is in his hands, that is it becomes easy from him.

Shruti:–

[॥ Tassmat – Bhava Bhavan – Paritajya – Parmatma Dhyanena – Mukto Bhavati~॥.]

Mandala Bramano Upanishad.

Tatparya:– Bhavas – Bhavas – It means that one understands one thing and forgets the other. These forgetfulness becomes Shunya. One who is always immersed in Atma Dhyana is a Mukta.

Shruti:–

[॥ Dvepadey – BandhMokshaya – Nirvameti – Mameticha~। Mameti – Bhaddhyatey – Tantuhuhu – Nirvameti – Vimuchhetey~॥.]

Paingolopo Upanishad.

Tatparya:– This is mine, I want it this is Mamkara. Since Gnanananda Sarva–Paripurna, ParaBrahma is myself, these bodily Prapancha is not all myself nor they are mine, then the aspect of Mamakara does not bother. The above two Bhavas are the causes and first steps to attain Mukti. Mamkara is only Bandha without it is Moksha.

Shruti:–

[॥ Aham Brahmeti – Niyatam – Moksha Heturma mahatmanam~॥.]

Painglopa Upanishad.

Tatparya:– I am that Atmaswarupa ParaBrahma that Parabrahma is me. To do Aikya sandandhan is the cause of Moksha.

Shruti:–

[॥ NahamBrahmeti – Jaanati – Tasya – Muktirna Jayatey~॥.]

Paingolopo Upanishad.

Tatparya:– If I feel and think that I cannot be Brahma then I will not attain Mukti.

Shruti:–

[॥ Bandhohi – Vasanabandho – Mokshasyetal Vasanakshaya~॥.]

Muktiko Upanishad.

Tatparya:– Loka Vasana, Shastra Vasana, Deha Vasana all these Vasanas causes Bandha. To be Nirvasana will help in Mukti.

Shruti:–

[॥ Nahamdukhi – Nameydeham – Iti – Bhavanu Rupenya – Vai Vaharaney – Muchhaitey Naham – Mamsam – Nachasthini – Dehadhanya – Parosymaham~। Iti Nishchitavanantaha – Ksheenavidyo – Vimuchaitey~॥.]

Mahopa Upanishad.

Tatparya:– I am not Dukhi, I do not have a Deha. I am just a Sarvasakshi as I am apart from the body. So how can I have Dukha? All the dukhas are restricted to the body only. I am Gnananda Parabrahma Swarupa, if my thinking is so I will get Mukti. The body which is Composed of blood, flesh, bones, I am not all this, there is no Dehatraya I am Kshana, only a Sakti Atma, that is who I am. I am that atmaswarupa Brahma, Iam Brahma if one is definitely sure of this aspect, then alone I can attain Mukti.

Shruti:–

[॥ Yada – Panchavasistanthey – Gnanani – Manasa – Saha~। Buddhiyascha Navijestatey – Thahamah – Parmam Gatim~॥.]

Khatho Upanishad.

Tatparya:– One should focus on the Atma by shutting of the organs of hearing seeing without any external aspects. All the thoughts must be forgotten and with ulter steadiness absorbed in the Self. The elders call this as Moksha.

Shruti:–

[॥ Chitaitya – kalito – Bhandaha = Tanmakanth Muktiruchaityey~। Chidachaitya – kilatmeyti – Sarvasiddhanta Sangraha~॥.]

Mahopa Upanishad.

Tatparya:– Bandhaha is that when one thinks that the body which is Anatma instead of the Gnanaswarupa atma – It means: – Without Viveka the anatma is attached to the atma's Dharma, Satyagnananda's signs to the body. The body is only the Dharma Awrita Jada Dukhas signs and think that all these belong to the Atma. One does not do Adhyasam, and this anadi Samsara is Bandha. It says, leave this Adhyasas it is known as Moksha.

Shruti:–

[॥ Sarva Bhootastya – Matmanam – Sarvabhutani – Chatmani~। Samprashya – Brahma – Parmam – Yati – Nanenyana – Heytuna~॥.]

Kavalyo Upanishad.

Tatparya:– The Atma – Sarvabhuta is everywhere and so also the Sarvabhutas is present in the atma, that Atmaswarupa Brahma is me and Brahma is Atma without any Bheda (difference) between the two is the one eligible for Mukti and others are not fit for this kind of Mukti.

Shruti:–

[॥ Bhutesh – Bhutesh – Vinchintya – Dheeraha~। Pretyasma – Nloka – Damrita – Bhavanti~॥.]

Khatho Upanishad.

Tatparya:– The individual who comprenends that all the Vastus (objects) of the union that Atmaswarupa is himself that kind of Mahatma will attain Moksha in this Life itself.

Shruti:–

[॥ Yada – Sarvey – Pramuchhyantey – Kamayesya – Hridirishithaha~। Atha – Marthyomrito – Bhavatyatra – Brahma – Samushnetey~॥.]

Khato Upanishad.

Tatparya:– When all the desires fully get destroyed and there are no desires left one will attain Mukti.

Lakshanas of Jivanmukti

Shruti:–

[॥ Kriya – Nashabdha Veth – Chintanasho – Smadvasana – kshya~। Vasanapa – ksheyomoksha – Sajivanmukti – Rishyatey~॥.]

Adhyatmo Upanishad.

Tatparya:– Hey Anjeneya! If all the Vyaparas get destroyed there will be no chintas (worries). If there are no worries vasanatrayas too get destroyed as also the Vasanas destriction is “Jeevan Mukti”.

Shruti:–

[॥ Brahmanyeva – Veclimatma – Nirvikaro – Vinikriya:~। Brahmatmano – Shuddityo – Rekhabhavava – Gahini~। Nirvakalpacha – Chinmatra – Vrithihi – Pragneti – Khatyetey~। Sasarvada – Bhavedhya – Sa – Jivanmukta – Ishyatey~। Dehendriye – Shvahambhava – Idam Bhava – Sta Dhanyetey Yasya – Nobhavata: Kvapi – Sa – Jivanmukta Ishyatey~॥ Napratya grabrahmano – Bhudham – Kadapi – Brahma Sargayo~। Pragnayayo – Vijanati – Sa – Jivan Mukta – Ishyatey.~॥ Sadubhihi – Pujya – Manesmi – Pedyamanopi – Durjanye~। Samubhavo – Bhavedyasya – Sa – Jivanmukta – Ishyatey~॥.]

Adhyotma Upanishad.

Tatparya:– Hey Anjeneya! One becomes Nirvikara once he is absorbed in the Brahma that Brahma is Atma, and Atma is Brahma to understand this Tatva vichara Samarasa Aikyabhavam, once this takes place all the Sankalpas are destroyed, one forgets all of it and that he is only a Sakshi this Vrithi is called “Pragney”. One who has this Pragney is a Jivan Mukti. The feeling of Ahankara that the body organs is me is due to Agnana – If this Agnana disappears he is a Jivan Mukta. There is no difference between Atma and Brahma at the same time, Brahma is Atma, Atma is Brahma, the one who understands this through Gnana he is a Jivan Mukta. The one who handles everything with utmost equamunity like one who is praised by the Sadhus and is criticized the Durjanas, he is not upset by any of these.

Shruti:–

[॥ Purushya – Dhuka Sukhadi – Lakshana – Chittadharma:~। Klesharupatvat Bandho – Bhavati~। Tamnirodhanam – Jivanmukti~॥.]

Muktiko Upanishad.

Tatparya:– One who is very proud that he is a Yajamana and everything that is happening is all done by me that he is the doer and feels Kritva, and thus and experiences the fruits that accrue from it and thus he is Bhoktatva, and the mind keeps doing all types of actions which leads to dhuka sukha experiences. As this is the main cause for Bandha, I shall not perform any Karyas and shall not experience anything I will try hard to be just a witness. I will not get attracted to any external objects as I will definitely be aware of real Gnana. Then everything will turn to nought (Shunya) Kritva and Bhokta. The realisation that he is only a sakshibhuta, Chinmatra Brahma this is real Jivan Mukti.

Shruti:–

[॥ Gnanamrita – Trapta – Yogino – Nakinchith – Kartavyamasti~। Tadastichenna Sa – Tatvavidhbhavati~॥.]

Paingolopo Upanishad.

Tatparya:– The one who has wholly drunk the Devamrita, he does not feel hunger and thirst and other Vikaras and thus is very contented and happy. His desire for appealing dishes will not be there, he will not wander here and there, and will sleep where ever he is and will have good sleep and enjoy the Gnanamrita and is fully contented gnani and comes to the conclusion neither he is a doer or an enjoyer. Thus, Ahankara becomes Shunya and will not indulge in any external action, will be steady and where ever he may be he always in the Dhyana of Brahmananda sleep. Such an individual is a Jivana Mukta. Vyapara Shunya is – Jivan Mukti.

Shruti:–

[॥ Prarabdha Karma – Paryantam – Mahi – Nirmo kavat – Vyavahariti~।. Chandra vachharitey – Dehi – Samyukta – Shyachini Ketanaha~॥.]

Paingolopa Upanishad.

Tatparya:– Until the Prarabdha karmas are destroyed till then the body will not be destroyed. Hence, a gnani will be in a body form till the past karmas are experienced and get exhausted. Just as a germinated seed is plucked off, it will not have further growth, Similarly, a gnani will see to it that he has rebirth. This can also be compared with the full moon, without any kalanka (defeet or bad sear). So also a being without any ego or ahankara that he is the does and feels the world a reality, once thus illusion gets removed, becomes a Shunya, he lets go off the Sankalpa, like a potter who does not get attached to the worms in the clay he is using. Similarly, a human being living in the Samsara without being attached to it is independent of the Manasendriyas, is a friend of all, very compassionate, hardworking, without anger, jealousy fear etc. The objects he likes and he gets it is happy and when his desires are thwarted is angry or sorrowful; He is beyond all these emotions, sentiments and feeling, he is devoid of all of them. Whatever, is due to him as a result of Purvakarmas, he is content with it. whatever he is unable to attain he is not sorry for it and he neither does Dushana or Bhushana and treats them equally as krishna says in B.Gita “Sthithapragna”. He is always in the world of Brahatma – aikya and aborted in the Saudanam. Such a being is a Jivan Mukta.

Shruti:–

[॥ Sarvaloka – Sthutipatra – Sarvadesha – Sancharasheela:~। Parmatma Gaganey – Bindum – Nikshpya – Shuddhadvyatajadya – Sahajamanask – Yoginidra Akhandananda – Padyanuvritya Jivanmukto – Bhavati~॥.]

Mandala Brahmano Upanishad.

Tatparya:– A gnani who is praised by the Loka, but with great aplomb and devoid of fear, goes off wandering in the wild always absorbed in the knowledge that Sachitananda Brahma is himself he stealies his Buddhi on this aspect and everything is himself and achieves Advaitya Brahma Sakshatkara, Sahaja Manaska and Like Maha Vishnu is in Yoganidra always in Akhanda ananda Vrithi – Thus he is a Jivan Mukta.

Shruti:–

[॥ Thirthey – Shvapgriheva – Tanum Vihaya – Yati – Kaivalyam.~। Pranavatrya – Yati – Kaivalyam~।].

Paingolopo Upanishad.

Tatparya:– The Gnani as soon as the Prarabdha karmas get destroyed, he discards his body and becomes one with the Brahma. kashi, Rameshwara may be Punya Kshetras, Punyathirthas, or it may Chandala's home wherever a Gnani is he will become Aikya with the Brahma. Thus that Gnani wherever he may be shed his body, there is no dosha. For a Brahmignani, it does not matter whether it is a Brahamana agrahara, or any other agrahara without any differentiation between them. Everything is equal and it is Abedha.

Shruti:–

[॥ Yatra Yatra – Mritognani – Yenavakena Mrityena~। Yatha – Sarvagatham – Vymom – Tatra Tatraya – Layam Gathah~॥ Ghatakasha Mivatmanam – Vilyamveti –Thatyatyaha~। Sagachhati – Niralambam – GnanaLokam – Samatathaha~॥.]

Painglopo Upanishad.

Tatparya:– Whether it is a upper caste home or a lower caste one, anywhere it may be, if he is suffering in pain or may be free from any disease. There is no dosha whether he does this with awareness or has forgetten these aspects and death occurs there is no Dosha. The kula gotras, all diseases as also the sorrows affects only the body, and not to him as he is fully aware that he is Brahma and is in complete Bliss Ananda. Therefore as soon as he dies he enters the Ghatakasha and Mahakasha, Satya Sarva Paripurna becomes one with Brahma or Aikya.

Shruti:–

[॥ Daghasya – Dahanamnasti – Pakvasya – Pachanam Yatha~। Gnanagni Dhagdha Dehasya – Nachashraddhadam – Nachakriya~॥.]

Paingolopo Upanishad.

Tatparya:– He thinks he is Brahma or rather is aware of it. The illusion that the body is only for Gnana agni. What is burnt in the fire turns into ash and this cannot again be burnt. Similarly, from the gnanagni, one is under the illusion that the body is himself. But for the burnt gnani after death, his Sthula Sharira is again taken to the Smashana this should not be done. Not only that Tilatarpana, Pindapradhana Shraddha and other rituals should not be formed. Instead his body should be interred in the ground.

Shruti:–

[॥ Tatroko – Mohaha – Kashyoka – Vikalpa – Manupashataha~। Tarti – Shokanatmavit – Bhidyatey – Hridaya Granthi – Shchidyantey – Sarva Samshaya~। Khshyanteychasya – Karmani – Yasmrya – Dristey – Parvarey~॥.]

Khato Upanishad.

Tatparya:– Everything is himself Brahma that is himself apart from him there are no objects or Padartha, he does not get into the illusion, he is always absorbed thinking that he is all in all, when he is looking within that Gnani he does not experience sorrow or Moha. He has crossed sea birth–death and sorrow. He will attain Nityananda. For him all the granthis or glands and all doubts, as also all the activities gets destroyed and he becomes BrahmaSwarupa.

Shruti:–

[॥ Jivaha – Pancha Vishankaha – Sva Kalpita chaturviram – Shati – Tatvam – Parityajya~। Shadviamsha – Parmatmahamiti – Nishchayat – Jivan mukto – Bhavati~॥.]

Mandala Brahmano Upanishad.

Tatparya:– The 25th Tatva that the Jivatma what he has imagined as Dehindrayas. The 24th Tatva is that he is not he that illusion, to set aside. The 26th Tatva is that he himself is Pari Brahma and who is very certain about is a Jivan Mukta.

Shruti:–

[॥ Nakarta – Naiva Bhoktacha – Nacha Bhojita Thatha~॥.]

Varaho Upanishad.

Tatparya:– I am not the doer, I do nothing, I am not a Bhokta, I will not eat anything. I shall not experience like others, I am only a Sakshi, non–does non–Bhokta and he is certain about it. Kartatatva (the doer aspect) does not have the Ahankara about it and that is known as a Gnani. The Gnani, who does not have the Ahankara, he will not do any Karya. Suppose he does do it there is no Dosha as he has no ego about it. Such a Gnani Balonmata.

Shruti:–

[॥ Ya – Sa – Mastarta – Jaleshu – Vyavarhya – Api – Shitalaha~। Pararthey Shviva – Purnatma – Sa jivana Mukta uchhaitey~॥.]

Varaha Upanishad.

Tatparya:– Though a Gnani will be working like the world around him, yet it is like “others karyas”. It means – that in others actions he is not a Abhinivesha Even in his own actions there is no Abhinivesha. As he continues his actions in the Jagat it is just a dream and everything is Mithya and so there is no question of any illusion about it without any Nirasha, he is at peace and is a Brahmasvarupa.

Shruti:–

[॥ Sarvasthatesu – Gnanagnyan – Dhyana – Dhyauv – Lakshya – Drishya. Drishyetaha – Oohapohadi – Parityjya – Jivanmukto – Bhavati~॥.]

Mandala Brahma Upanishad.

Tatparya:– The 3 stages of the mind viz – Jagrat, Swapna, Sushupti and all the objects Gnangneyas, Dhynadhyas Lakshya Lakshya, Drishya drishyas, Ohamohas all this is just Nithya. Without any illusion about these aspects. There is no Ahankara, as a Sarvasakshi and that he himself is Brahma and always looks inwards, just as a Sakshi and who remains as a Brahmakara is a Jivan Mukta.

Videha Mukti Lakshanas

Shruti:–

[॥ Jivan Mukta – Padam – Tykatva – Svadhey – Kalasaskritey – Vishati – Adeha Muktitvam – Pavno – Spandatamiva~॥ Ashabdha Masparsha – Maru pamavyayam – Thatharasam – Nityamghandvachha – Yat~। Anadyanantam – Mahataha Param – Druvam – Tadev – Shishyatyamalam – Niramayam~॥.]

Paingolopo Upanishad.

Tatparya:– A Gnani when his Prarabdha gets over Dehatraya too gets destroyed, the jivan Branthi too is erased, he gives up the Jivan Muktiyam and without a Deha Vidha mukti is attained. Then there will be no shabdha Sparsha Ruparasa and the sensation of smell and Namarupa, actionless with all these being destroyed. He gets Sarvashresta Niramaya Parabrahma wherein he is fully absorbed and becomes one with the Lord.

Shruti:–

[॥ Yatha – Jalejalam – Kshiptam – Ksheramkshirey – Gritam grhitam – Avishesho – Bhavetdvat Jivatma – Parmatmano~॥ Yathodakey – Toya – Manupravitam – Thatatma rupa – Nirupadhi – Samsthithaha~॥.]

Paingolopo Upanishad.

Tatparya:– Water in water, milk in Milk, Ghee into ghee gets all blended and becomes Aikya. So also, this body as soon as it is deceased, the Jeevopadhi is destroyed, the Atma gets merged in Brahma and becomes one with him – This is Videha Mukti.

Shruti:–

[॥ Brahma – Brahmey thatha – Yanti – Yevadu Brahmana sthatha~। Atraivathey – Layamyanti – Leenaschya vyakta – Shalino~। Leenascha Vrikta shalina – Iti~॥.]

Mantriko Upanishad.

Tatparya:– Brahma is Atma: Atma is Brahma: When these two facts are understood i.e. there is no Bheda between them, and he does Brahma anusandhana by the Gnani, after death, he is Shuddha Nirguna, Nirakara, Paripurna Brahma he merges with him, and does not differentiate between him and Brahma. This is known as Videhamukti or Videha Kaivalya.

Shruti:–

[॥ Yatha – Nadya – Syandamana – Syamudrey – Astamghachhati – Namarupey Vihaya~। Thatha – Vidvarvanama – Rupadvimukta – Paratparam – Purusha – Mupyati.]

Mundko Upanishad.

Tatparya:– The flowing rivers like Ganga and others, ultimately flows into the Ocean or Samudra, it loses its individual identity and now it is the ocean. So also the knowledge of Brahmagnana when he dies or is in a Mrityu state, the Sthula Sukshma Karana Shariras gets destroyed, each of the body's namarupa is merged with the Shuddha Nirakara Brahma is Aikya and so he assumes the ParaBrahma rupa.

Shruti:–

[॥ Yeto Kamo – Nishkama – atma Kamaha Natasya – Prana – Utkramanti Brahmey – Varsa – Brahmapyeti~॥.]

Brihidaranya Upanishad.

Another one.

Shruti:–

[॥ Sokamo – Nishkamo – Aptakama – Atmakamo – Natasya Prana – Utkramtyatrayam va – SamavaliYantey~॥ Brahmey Varsa – Brahmpyate~॥ Ashariro – Nirvindriyo – Aprano – Hai manasachhidananandanatra Sa Svaradhabvati~॥.]

Tapaniyopanishad.

Tatparya:– Brahma is Atma, Atma is Brahma he comes to know the Abheydyaka Gnana. The one who is immersed in Atmadhyanam that is Brahma Gnana as soon as he dies, Deha, Indriyas, Prana mind he is without any of these, he is only concerned of the Nirakara Sachinanda ParaBrahma, he becomes one or Aikya with the Lord that is Brahma. Hey Saumyeney!

Shruti:–

[॥ Yevam – Tri putyam – Nirastayam Nista Ranga – Samudrava – Nivatstitha – Deepavadachala – Sampurna – Bhava – Bhava – Vihina – Kaivalyajyotirbhavati~॥.]

Mandala Brahmano Shad.

Tatparya:– The Drashutra Darshana, Drishya Triputiyam Parabrahma Atma is himself. The Gnani who does Atma Sandanavam, after his death he becomes like a wave in the ocean, and like a lamp where there is no breeze he is without movement and is Achala, and he is Paripurna without any bhavas or feelings becomes a zero or shunya, shuddha Nirakara Sachitananda Kaivalya Jyoti Swarupa and he just merges with the Paribrahma and he himself turns into Parabrahma.

Shruti:–

[॥ Yathodakam – Shuddhey – Shuddha Mastikam – Tadrigeva – Bhavati~। Yevam – Muney virjanata – Atma Bhavati Goutama~॥.]

Khato Upanishad.

Tatparya:– The Atmagnani who understands that Atma is Parabrahma. That Atman is Nirmalat and when this becomes one with the clear Nirmal water, so also when one merges with the Parabrahma and becomes one with Him.

Shruti:–

[॥ Yatra – Nanyatpashyati – Nanyachhainoti – Nanyadvijanate – Sabhuma~॥ Yovai Bhuma – Tadamyatam. Aathyatranya – Tatpashyatyanya – Chharnoti Anyadvijanati – Tadlpam~॥ Atha – Yadlapam – Tanmatyatriyam~॥.]

Chandogyo Upanishad.

Tatparya:– Ishwara and the Bhakta, both look at each other, and talk and seem to be happy. But Vaikunta Kailasa and other places are meant for bodily satisfaction, they are not a place for Moksha.

Shruti:–

[॥ Brahma – Vishmuchha – Rudrachha Sarvava – Bhoothajataya:~। Nashamevanu – Dhavanti – Salilaniva – Badabam~॥.]

Maho Upanishad.

Tatparya:– The above mentioned places are not for Nitya Mukti but

Shruti:–

[॥ Tatra – Komoha – Kashyoka: – Yakatva Manu Pashyata~॥.]

This is according to Ishyavasa Upanishad.

Shruti:–

[॥ Yatrahi – Sarva – Matmaivabhut~। Tatkena – Kampashyth~॥.]

According to Brihadaranya Upanishad.

Shruti:–

[॥ Nahavai – Sasharirasya – Satha – Priyapriyayo – Rapahati – Rasti~। Ashariramva – Vasantam – Napriyaprey – Sparishita:~॥.]

Tatparya:– In the manner of Shrutivakya Ishwara and his Bhakta, they are two of them, and are bodyless. This Atma seeks abhabeda with Brahma and becomes Aikya is Nityamukti as also known as Vidhaya Mukti as also Kaivalya.

Gnana Japa Tapadis

Shruti:–

[॥ AHimsatu – Tapoyagno – Vagmnaha – Kayabhihi~॥.]

Shatyani Upanishad.

Tatparya:– The Manovakyas, becomes Trikarana shuddhi and not to do Jeevahimsey that is Tapas Yagnas Krishna Pashuvam to kill the animal, and put them in the Agni Kunda \& cook then and eat that meat, it doesn't become a Yagna Man does these Jeevahimsa and then conduct Yagas, such humans truely do not become Gnanabrahmas. They are known as Karma Brahmanas.

Shruti:–

[॥ BrahmaSatyam – Jaganmithyeti – Aparoksha Gnanagnina – Brahmadaivasya – Siddha – Sankalpa Beeja. Santapa Rupam. Thatha~॥.]

Niralambo Upanishad.

Tatparya:– Brahma is Satya. Jagat is Mithya (unreal). Without understanding that he is Brahma himself this is Aproksha Gnana. From this Gnanagno the anitya Brahma, Vishnu, Maheshwara and others and consider Svarga, Kailasa these words, is like roasting Ashabeejas and perishing them and this is known as Tapas.

Shruti:–

[॥ Atmanam – Chedvijaniyat – Ayama Smite – Purushaha~॥ Kimichharva – Kasya Kamaya – Sharirmanusam – Jverith~॥.]

Shatyaniyo Upanishad.

Tatparya:– The Gnani who is aware that he is atmaswarupa Brahma, and to fulfil the desire, he does Japas and how can he subject the body for Japasyapeeda? That means – He does not desire for anything. Unlike an ignorant man he does not think there is another Daiva apart from himself which is just an illusion and begins to do Japa Tapas etc. and fasts giving up food and water and the body has to undergo these tortures.

Shruti:–

[॥ Asti – Brahmeti – Vignanam – ParokshaGnana – Muchhaitey~। Aham Brahmeti – Vignana – Maparoksha – Gnana Muchhaitey~॥.]

Tejo Bindiopa Upanishad.

Tatparya:– Paroksha Gnana is that when one thinks that there is Brahma apart from himself and does the Upasana. From this Aparoksha Gnana is attained in the Janmantara. Again when he thinks himself as Brahma and becomes Aikya (One with the Lord) this is AprokshaGnana. Therefore, he will obtain Mukti in this life itself.

Shruti:–

[॥ Indriyadvara – Sangrahai – Grahdai – Ratma devata~। Svayam – Bhaktya – Samaradya – Gnatum – Soyam – Mahamakhaha~॥.]

Tatparya:– One has to control the Indriyas and see to it that it does not go astray and does the Vikara karyas, and does Atmanisteyam is “Mahayaga”. If one sacrifices a black goat in the agnikunda and consumes the meat is not a Yaga.

Shruti:–

[॥ Pranendriya – Dyantakarana – Gunodey – Parataram~। Sachindanandmaya – Sarva Sakshikam – Nitya mukta– Brahmasthanam – Parampadam~॥.]

Niralambo Upanishad.

Tatparya:– That is Sachitananda Sarva Sakshi Atma Brahma, when the Prana, and the senses, Antakarana is thought that there is something beyond all this which is mentioned above. That kind of Atmanista Parampada, Vaikunta Kailasa it is known.

Shruti:–

[॥ Aastamsamsara – Vishaya – Janana – Samsarga. Yevanarka:~। Satsyasarga – Svargaha~॥.]

Niralambo Upanishad.

Tatparya:– To get entangled with subjects of the Samsara Vyavaharas is Narka (Hell). There is no other bigger Naraka than this in the Loka. Heaven or Svarga is that when one is manolayvam in the Brahma. There is no bigger Svarga in the Indraloka than this.

Shruti:–

[॥ Sachitanandanubhava – Svarupam Gnanatva Anandarupo – Yasthiti~। Sa Yeva Sukham – Astasyamsaravishaya – Sankalpayeva – Dukham~॥.]

Niralambo Upanishad.

Tatparya:– When one understands that the Atma is Sachitananda Brahma, he forgets all else and gets immersed in Atmanubhava this is real Sukha – happiness Mahadukha or unhappiness is that when one is engaged in wife and family and Samsara aspects.

Shruti:–

[॥ Dehohamiti – Sankalpo – Mahapapa – Mitisbhutam~॥.]

Tejo Bindo Upanishad.

Tatparya:– Mahapapa is that when one identifies himself with the body with a lot of Ahankara instead of understand that the Dehatraya Sakshi Atma is himself. There is no other Papa.

Shruti:–

[॥ Dehohamiti – Yadgnanam – Tadevakam – Smrityatam~॥.]

Tejobindo Upanishad.

Tatparya:– Hell is that when one assumes that the body is himself with Ahankara.

Shruti:–

[॥ Nivritti – Parma – Tripti – Ranandonupama – Smritihaha~॥.]

Adhyatmo Upanishad.

Tatparya:– All the Vrithis, becomes Shunya and when one does Brahmanusamdhanvamt to be focused on this is Parmavrithi and Parmananda.

Bramana Lakshana

Shruti:–

[॥ Gnanadanda – Gnanayagnopa – Vithnaha~। Shikha – Gnanamayi – Yasya – upavithincha – Tanmayam~। Brahmanyam – Sakalam – Tasyeti~॥.]

Shatyaniyopo Upanishad.

Tatparya:– I am Brahma, Brahma is me. This awareness is Yekadanda or Tridanda. This knowledge is Shikhat Yagnopavita, those who wear this sacred thread are Brahmanas. Made with cotton thread this Janawara. Those who do not have a tuft at the back of the head are not Brahmanas.

Shruti:–

[॥ Nachamarno – Naraktaasya – Samam sasya – Nachasthina~। Najatiratmano – Jathivairyavahakara – Prakalpita~॥.]

Niralambopa Upanishad.

Tatparya:– The Atma of Parabrahma does not have any caste distinctions. But is there caste distinction for the Sharira? If one thinks about it, everybody's body's skin, blood, flesh, bones are all same and nobody's body is made of gold. Therefore, everyone is of the same Jati. Hey Saumyene! Brahmanas, Shudras, Madigas there is definitely no Jati Bhedas. Ignorant people have imagined these differences in caste in the Loka Vyavahara. Truely, everyone is one Jati, everybody have the same Atma. Therefore, Atma also does not have any differences Hey Saumyaney!

Shruti:–

[॥ Tato – Brahmana – Samahito – Bhutatva – Tatvampadykai – Mevasada – Kuryat~॥.]

Paingolopo Upanishad.

Tatparya:– The one with a serene Shanti Samadhana, Sadaguna person who is always contemplating on Atma Swarupa is himself Parabrahma. That Parabrahma is myself without any Abhedha and does Dyhana on Atma, he is the Brahmana and others cannot be Brahmanas.

Shruti:–

[॥ Ya – Yevam – Vedaham – Brahma Sthi~। Sa – Idam – Sarvam – Bhavati~। Na Tasyaha – Nadeyvachya – Nabhutya Ishatey~। Atma Haishyam – Sabhavati – Athoyonyam – Devata – Moopasathey – Anyo~। Savanyoha – Masmati~। Na – Saveda – Yatha – Pashurevam – Sadeyvanamiti~॥.]

Shweteshwara Upanishad.

Tatparya:– One who has realised that he is Brahma and Brahma is myself and is eternally in this Atma Gnana Dhyana, seeing him others think that he is Brahmagnani and hence all the Devatas are happy, and keep hovering around him and he is well respected. There are others who do dhyana other than atmadhyana and do dhyana on other devatas, they are not capable of realising Parabrahma. One who thinks Brahma is different from Atma such a Bhedabuddhi person can be compared to an animal (Pashu).

Shruti:–

[॥ Jatam – Mritamidam – Deham – Matapitra – Malatmakam~। Sukhadukha Laya Veelyam – Sprastva – Snanam – Vidhiyatey~। (1) Dhatubandham – Maha Rogam – Papamandira – Madruvam~। Vikarakara – Vistrinam – Sprastiva – Snanam – Vidhiyatey~। (2) Navadvara – Malaprasravam – Sadakaley – Svabhavajam~। Durghandam – Durm lo petam Sprastatva – Snanam – Vidhiyatey (3)~॥.]

Maitreyo Upanishad.

Tatparya:– This body which is subjected to birth and death is born out of the union of the parents and the mother's mala. The body is limited to a bit of happiness and a bit of sorrow. This body is not a pleasant object as it smells of rot and one feels for abhorrence. Atma is Parishuddha (Holy and sacred) and it does not have any dosha. No Jati Bhedha (1) One has to take a bath if one touches a body having flesh, bones, skin, nerves etc. the 7 minerals or Dhatus which disappear and is subjected to dreadful diseases having done mahapapas (sins) and it has a Vikara (distorted) form, even if one touches such a body one has to have a bath. Atma does not have any doshas without Jati Bhedas.

(2) There is always bad smell in the 9 randras are filled with dirt like excretion. This Chandala body is possessed by everyone. Not even one has a golden body, or a scented body. Therefore, there are no caste distinctions. Thus, if one touches such a body one need to have a bath. Atma is Parishuddha and is present in everyone – he is the only one, and hence he has no Jati Bhedha or Kula Gotra etc. (3)

[॥ Antyajati – Dvrijatishya – Yeka Yeva – Sahodharaha~। Yekyoni – Prasuthatvat – Yekashakhena – Jayatey~॥.]

Tatparya:– In the Loka Brahmanas and other lower castes are born in the same manner like they go through the wastes excreted by thin mother, so they are Sahodaras or brethren. They emerge from the same branch or the same process, and thus there is no Bhedabhavas between them. (मनोर्जतायाः मनुजाः मनुष्यद्दाः) all humans are manubrahmaeem they are born and so they are Ekajatisahodaras.

Shruti:–

[॥ Sarvadharma – Paritijya – Nirmo – Nirahankaro – Bhutva – Brahmesham – Sharanu Mupagamya – Tatvamasi – Sarvam – Khalvidam Brahma – Nehananastikinchi Netyadi~। Mahavakyarathanubhav – Gnanadhbrahmayivaha – Masmiti – Nishtya – Nirvikalpasamadvina – Svatantro – Yathishichrati~। Sasyanyase –Samuktasyavo \_ Pujya – Sasyayogi – Saparahamsa – Sovadhuta – Sabrahmana – Iticha~॥.]

Niralambopnishad.

Tatparya:– What is anitya Samsaras' sorrowful subjects, Putra, Mitra, Kalatra aspects for all the above with a lot of Ahankara says “Iam the boss” – Yajamana. This is this is Mamakara. All the day to day mundane activities of the world should keep shedding at intervals and do Dhyana on the Brahma Mahavakyakartha understand its go deep into the Tatvas and using the Viveka, think that Brahma is me and I am that Brahma and one should be certain about it and forget all the rest, and with Nirvikalpa, and with a peaceful concentrated mind try to attain AtmaSakshatkara – he is a Sanyasi, a Sage, a Mukta, a Pujya – sacred a Rajayogi Parmahamsa, an Avadhoota Brahmana. Wearing saffron attire or moving around semi – naked avadhuta wearing layers of thread or the Janvara one does not become a Brahmana.

Akhanda Gnana

Shruti:–

[॥ Chidahasteti – Chinmatramidam – Chinmayameveeha~। Chitvam – Chideha Methecha – Lokashbiditi – Bhavaya~॥.]

Yagnavalkya Upanishad.

Tatparya:– Hey Anjeneya! It is here the Gnanaswarupa Parabrahma is in this Prapancha or world. That is Chinmaya Brahma; You and me both are that Brahmaswarupa. In this loka whether one belongs to a upper or lower strata of society is irrelevant, because both emerge from the excretions of the mother. So right from birth they are Sohodaras belonging to the same branch or category No Jata bhedas. (Manojarta – Maneya – Manushyaha:). All the humans are Manubrahmneem – Right from birth they are Ekajatisahodaras.

Hey Anjeneya! the 14 lokas all the Characharas all of it is Brahma.

Shruti:–

[॥ Natu – Tadvitiya – Masti~। Tatonyat – Vibhaktam – Yatpashyeth~॥.]

Brihadaranya Upanishad.

Tatparya:– Apart from the Parabrahma there is absolutely nothing else. Everything is Brahma. Is there is anything else apart from this Brahma?

Shruti:–

[॥ Yatrahi – Dvyaitamiva – Bhavati~। Taditara – Itaram – Pashyati~। Taditara – Itaram – Jhigrati – Ityadi~। Sarva – Matmaivabhut – Tatkena – Kampasyediti~॥.]

Brihadaranya Upanishad.

Tatparya:– There is only one Parabrahma but he exposes in different Nama rupas and pervades the Characharas as a dwallyo it appears so. One talks see something smell this and that, hear this and the other touch one thing or the other, sees.

The conclusion is with whom one talked or saw or touched or heard are all Brahma and nothing else there is no other object that is the opinion or idea.

Shruti:–

[॥ Upandanam – Prapanchasya – Brahmanonya – Nna – Vidyatey – Tasmatsarva – Prapanchoyam – Brahmey Vasti – Nachatarat~। Brahmana – Sa Sarvabhutani – Jayantey – Parmatmanaha~। Tasmadetani – Brahmeyi Va Bhavantiti. Vichintiya~॥ Yatha Taranga – Kallolai – Jarha Mevam – Sfuratyalam~। Ghata Namno – Yatha – Pritvi – Patanamnaritantavaha~। Jagannamna – Chidabati – Sarvam – Brahmai Va – Kevalam~॥.]

Yogashikho Upanishad.

Tatparya:– Hey Anjeneya! Listen attentively. The clay and the potter, and his wheel, because of the above three pots can be made. Without a cause or a reason no work can take place. The pot is made with clay. The pot, the clay is known as Upadanakarana. Without a potter a pot cannot be made, therefore the potter is a Ninutta karana for the pot. For that potter a wheel is required without which there will be no pot. So, the wheel becomes Sahakari Karana. Without the above mentioned 3 reasons, a pot, cannot be manufactured.

Similarly, the Jagat requires the Upasana Karana, Nimmitta Karana and Sahakari Karanas. If you think deeply about it is not so.

(Sristey – Pura – Namarupa – Vivarjitan Jagat~। Sadeva Saumya – Yidamagra – Aseeth, Atmeyai Va – Yidamagra – Aseet)

Brihadaranya Upanishad.

According to this Upanishad, before creation, this Prapancha there was no Namarupa for it there was only Brahma there was no Nimitta Karana, and Sahakara Karana were not there. Just as due to the Upadana Karana, the clay was converted into a pot, it appears so. the thread when woven converts into cloth it appears so. The foam in the Samudra converts into waves it appears to be so. The Upadana Karana is Brahma and takes different forms like Deva, man, animal, worms, insects, trees, creepers etc. the Jagat appears to be There is no other objects.

Shruti:–

[॥ Sa Yesha – Brahma – Yesha Indra – Yesha Prajapati – Retey – Sarveydeva – Imanicha – Panch – Mahabhutani – Pritvi – Vayurakasha – Apajyotim – Shityetanivaa nicha – Kshudra mitramiva Beejanitrani – Chodbhyaijanichashvagava – Purusha hastino Yatkincheydam – Pranijam gamancha patricha – Yachhastavasam Sarvam tat~॥.]

Aitreyo Upanishad.

Tatparya:– Brahma, Vishnu, Maheshwara, Indra, Chandra, Surya these are the primary devas Akasha etc. Panchabhutas, all the seeds, Jarajiyas that is – all the mammals born from the mother's sac like humans horses, cows, elephants etc. all the four legged creatures. Those born out of hatching of eggs – like birds, fish, reptiles. Then the Sveda Jagas that is: born by sweat like lice, bugs and other worms and insects – Udbhijagas that is – which grows from below the ground, trees, creepers, mountains. ALL EVERYTHING IS BRAHMA.

Shruti:–

[॥ Atmeyda – Mamamritamidam – Brahmeydam – Sarvam~॥.]

Bhridaranya Upanishad.

Tatparya:– Sarva – all is Brahma. Brahma is the Atma and thus everything is Atma.

Shruti:–

[॥ Purusha – Yevedam – Vishvam~। Yasmat pasam – Naparmasti~। Teneydam. Purnam – Purushona – Sarvam~॥.]

Mundko Upanishad.

Tatparya:– This cosmos everything is Brahma he is neither different mosa since of belonging, as he is everything. The entire Vishwa is filled with him.

Shruti:–

[॥ Nehananasti – Kinchina~। Purushanna Param – Kinchit~॥.]

Khato Upanishad.

Tatparya:– This Jagat is not different from Brahma. Apart from Brahma there is absolutely no vastu.

Shruti:–

[॥ Purusha – Yevadam – Sarvam – Yadbhutam – Yachha Bhavyam~। Tadeyvagni – Sthadatitya – Stadvayu – Statu – Chandra Maha~। Tadevashuksam – Tadbrahma – Tadapa – Statpra japatihi~॥ Bhokta – Bhogyam – Preritaramcha – Matva – Sarvam – Proktam – Trividham – Brahma Mitatat~॥.]

Shweteshwara Upanishad.

Tatparya:– In This cosmos everything is Brahma. So far in this creation all that is born or in future all that is born is Brahma. Agni, Surya, Chandra, Vayu, Shukra all are Brahma. Jala and Prajapati is also Brahma. Jivan and from this Jivana all that is experienced the objects and for this Jivan according to Karmanusara he reaps the fruits or Phala and this is also is due to a deva all of it is Brahma.

Shruti:–

[॥ Yekamevadvitiyam – Brahma~। Sarvam – Khalvidam – Brahma – Tajjalaniti Idam – Sarvam – Yadaya – matma~। Atmeyai – Vedam – Sarvam~॥.]

Chandogyo Upanishad.

Tatparya:– Tajjalaniti – It means Tajjim is born in the Jagat and is born in the Brahma Tallam? and merges with the Brahma. Tadniti – from Brahma Chesti?

Shruti:–

[॥ Sarvam – Nareyanahah~॥ Purusha – Yevadam – Sarvam~। Yadbhutam – Yachha – bhavyam~॥.]

Subalopanishad.

Tatparya:– Sarvam is Narayana ParaBrahma everything which has been born and in the future what will be born everything is Brahma.

Shruti:–

[॥ Sarvam – Chinmatra meyvahi – Brahmaiva Sarvam – Chinmatram~। Brahma matram – Jagetrayam~॥.]

Tejobindo Upanishad.

Tatparya:– Everything is Chinmatra, Brahma all the 3 lokas are Brahma.

Shruti:–

[॥ Sabrahma – Sashivey – Seindra – Sokshara: – Parma Svarat~॥.]

Tatparya:– Brahma, Vishnu, Rudra, Indra, Jiva, Ishwara and others are all Brahma.

Shruti:–

[॥ Deva Yeko – Narayanaha~। Nadvitiyo. Sti – Kashith~। Narayana Yevadam – Sarvam~॥.]

Narayano Upanishad.

Tatparya:– Narayana Parabrahma are one and the same, Apart from this there is nothing else. All the Jagat is ParaBrahma.

Shruti:–

[॥ Chaitanyam Vina – Kinchinastita~॥.]

Niralambo Upanishad.

Tatparya:– Chaitanya is Parabrahma, there is no other object.

Shruti:–

[॥ Sarvanai Vaitani – Praganasya – Nama – Dheyani – Bhavanti~॥.]

Aitreyo Upanishad.

Tatparya:– Pragnana is Brahma and is also known as Atma. Atma assumes different Namarupas, and is present in all. Jeevajantus, and other tiny Jevas – for all this the same Atma is there. Truely, there is no difference between Jagat and Atma one and the same. There is absolutely no other Padartha other than the Atma. You are the atmaswarupa you are everything. It means – Aravu means – one can see an object. Maravu means – one cannot see the object. When one is in the wakeful state, one sees the Jagat in various names and forms, this is known as Aravu. Again, when you are asleep, one cannot see anything, you are unaware of the external world – this is known as Marevu. Both aravu and maravu have Sankalpas and Vikalpas. The mind keeps undergoing both these conditions. Due the mind which is capable of viewing the entire Jagat seems to be having Namarupas. During sleep, the mind is calm serene and may be lying low that is resting. This same Jagat looks different – formless and nameless. Therefore, [Manu Yeka – Jagatsyarvam – Jagatasyavaram – Chitta – Gocharam – Prapanchasya – Manaha Kalpitatvat]

Mandala Upanishad.

The mind itself appears to be the entire Jagat. Just as the waves in the ocean is not different from the ocean, so also the Atmasvarupa which is perceived by the mind, those psychic scenes is no different from the nanarupas it is no different Hey~। Saumyeney~।.

Though the waves come and go from the shores and looks to be they are new waves, Yet the Ocean itself is neither born or perish, it has been there and will be there. Similarly, the mind present in the body, those mind created scenes looks to be that the Prapancha also is born again and again and perish. But the atmasvarupa which is present in every being is eternal \& the same berefit of any Vikaras, but always as a Sakshi as a Nityananda Parabrahma.

Shruti:–

[॥ Yasmirya – Sarvanibhutani – Atmai Vabhut – Vyanataha~। Tatra – Ko – Mohaha – Kashyoka – Yekatvamanu – Pathyataha~॥.]

Ishavyasa Upanishad.

Tatparya:– The one who comprehends that he is himself Atmasvarupa, he is Sarva. Such an individual will experience Shoka and Moha with equanimity.

Shruti:–

[॥ Tataha Purnam – Svamatmanam – Pashyaidekatmana – Sthitham~। Svayam Brahma – Svayam Vishnu – Svayamindra – Svayam shivaha~। Svayam Vishwamidam – Sarvam Svasmadanyam – Nakinchana~॥ Satma – Nyarupita – Shesha Bhasavastu – Nirasataha~। Svameva – Parabrahma – Purnaha – Madvaya Makriyam~॥.]

Adhayatmo Upanishad.

Tatparya:– You understand that you are Sarvaparipurna Parabrahma! (If you think so). Brahma, Vishnu, Maheshwara, Indra, Chandra, Surya all these which have been in existence, then you will be Atmasvarupa. (Ahameva – Sakolo – Vikalasmi). In this manner, there is nothing apart from you even an atom of it. So you are all in all, feel Parabrahma. Seeing a rope, one thinks it to be a snake which is an illusion or wrong assumption. So also, a Jeeva thinks he is a Jeevanta birth and death, Namarupa, KulaGotra he assumes all the above instead of realization that he is Atmasvarupa. When one closely examines the rope mistaken for a snake and is relieved that it is just a rope and not a venomous snake. So also the (Netti Neti) – the Vedanta Vakye when analysis this Dehaindriyas Prapancha is all an illusion that you are not, and that you are only Sarvadista Sakshi, Nirvikara Sarvaparipurna Brahma then you are in total Bliss.

Manu Bhodhey

Shruti:–

[॥ Santyakta – Vasana – Naumnadri Tenastuyatmam – Padam~॥.]

Muktiko Upanishad.

Tatparya:– Hey Anjeneyaney! There is no Shresta title than for the BrahmaGnani where all the Vasanas becomes Shunya and he becomes Maun or silence. Mouna means just not to keep the mouth shut. It means to rid of all the Prapancha Dharmasambandas. Perish all the sankalpas. This will gain him the status of Brahmasvarupa where is bliss or ananda. This mauna or silence is known as Dhyana.

Shruti:–

[॥ Balyena – Thista seth~। Balyancha – Nirvidya – Pandityaina – Tistaseth~। Balyam – Pandityaincha – Maunancha – Nirvidya – Atha – Brahmanaha~।].

Brihadaranya Upanishad.

Tatparya:– According to this upanishad ‘Balya’ means Shravana. Panditya means – Mannana. ‘Mauna’ means. Nididhyana. One must give up all the 3 mentioned above. With Tushnum Bhava become shuddha and realize only the BrahmaSwarupa.

Shruti:–

[॥ Hastam – Hastena – Sampidya – Dantaidrarta – Vichurńyeha~। Anganyangai – Rivakramya – Jayedadau – Svakam – Manaha~॥.]

Mahopa Upanishad.

Tatparya:– Hastam Hastena = To cut the Karmakanda with Gnanakanda. Dantaidranthra = To churn into a powder the Sankalpa and Vikalpa teeth, Anganaingai = To denigrate (Khanda) Yogastangas with Gnanastangas. With all the above make or direct it towards Brahma and if one practises this constantly, one will be successful become Victorious over the mind.

Shruti:–

[॥ Deshaddey shantara – Praptau – Samvido – Madhyameva – Yat~। Nimeysheńa Chidakasam – Tadividhi – Munipungava~॥.]

Maho Upanishad.

Tatparya:– Hey Anjeneya! The Chitta remains in one Sankalpa and while going to another Sankalpa, that Chitta the coming and goind of the Sankalpa \& Vikalpa remains in the centre and is aware of it. That awareness is known as Chidakasha, Pragnana, Pratyagatma, Parabrahma. That awareness would be you understand that and forget the rest become absorbed and enjoy the Bliss.

Shruti:–

[॥ (1) Ghatakasha – Mahakasha – Mivatmanam – Parmatmani~। Vilapyakhanda Bhavena – Tushniambhava – Sadamuney~॥ (2) Chidatmani – Sadanandey – Deharudha – Mahamdhiyam~। Nweshya – Lingamutsijya – Kevalobhava – Sarvada~॥.]

Adyatmo Upanishad.

Tatparya:– (1) Just as, Ghatakasha and Mahakasha become one so also Atma is Parabrahma, Brahma is Atma and hence it takes form of Aikya Bhava, and you being the Atmasvarupa and feel that you are Brahma and be certain about it, and forget all the desires and just be like a statue by just being only Gnapti and just be in the silence.

Leave aside the ignorance of that you are the body, and understand that you are Pratyagatma, Parabrahma and be completely certain about it and remove all the illusions and forget everything by being just Chinmatra in silence.

Shruti:–

[॥ Hridiyam – Nirmalam – Kritva – Chintayitva – Pyanamyam~। Ahamava – Param – Sarvamiti – Pashyetparam – Sukham.]

Paingolopo Upanishad.

Tatparya:– Make the heart clean or Nirmala. The highest Sukha is when one understands that he himself is Brahma, and that he is all in all and be quiet in Mauna.

Shruti:–

[॥ Svarupanu – Sandanavyatirikta – Anyashastra Byahsai – Rustrakunkuma – Bhasavat – Vyarthat.]

Samnyaso Upanishad.

Tatparya:– One must first of all understand that he is Brahma, Saswarupa and make a Sandhanam (certainty?) Leave aside the Ananda. Spend your time studying the Vedas and scriptures without wasting time, it is like a camel adorning a kumkum flower!

Shruti:–

[॥ Tasmatasyavram – Parityjya – Tatvanisto bhavanagha~॥.]

Turyatitavadhutopa Upanishad.

Tatparya:– Leave all the Sankalpas, just be a Sakshi.

Shruti:–

[॥ Vasanastavam – Parityajya – Antashyanta – Samasteho – Bhava – Chinmatravasaneha~। Tampyatha – Parityajya – Mano buddhi saminvitam – Sheshasthua – Samadhano – Mayatvam – Bhavamarutey~। Ashabdam sparshram – Ityadi~॥.]

Muktiko Upanishad.

Tatparya:– Hey Anjeneya! Keep giving up Vasanatriya, Moksha, give up desires. The Jagat is Mithya. Parabrahma is the Truth So turn to Dhyana on Brahma, later on give up on the Dhyana and Mano and Buddhi keep them still and stable in order to concentrate that Brahma is me forget all the thoughts be silent and absorbed.

Maya Lakshana

Shruti:–

[॥ Adimadhyavasaneshi – Yaama – Samaya~॥.]

Tatparya:– What is not there in Adimadyas that is Maya – It means:–

Shruti:–

[॥ Sadeva–Saumya – Tidamagramaseet – Brahmanorvyaktaha~॥.]

Tatparya:– Before the creation of the cosmos there was only Brahma. Then there was no Maya. Therefore, during that is at the very first there was no Maya. Now, the creation seems to be in the Madhya or in the centre phase of creation. This appearance is like the rope and serpent illusion and so it is not the truth. Even now in the central phase of the creation it is not there, that is the opinion again in the Brahmanista or towards the end of creation i.e. during the Pralaya, the whole cosmos is submerged and everything becomes Shunya. Therefore, during the Anta or end stage of creation again there is no Maya.

(Maayamaya – Midamklilam – Jagat). This Jagat therefore appears to be Maya. “Jayethey – Ghachhatiti – Jagat”. Like the Indrajala, there is Multifarious Namarupas just appear and again does not seem to be there So it is untruth and this is the Jagat. Therefore,

Shruti:–

[॥ Idam – Prapancham – Namanysa Nothpannam – Nosthitham~॥ Maya – Karyadikam – Nasti – Mayanasti – Bhayamneha Vandhyakumara – Vachaney – Bheetheshyeth – astvidam – Jagat~॥]

Tejobindu Upanishad.

Tatparya:– According to this upanishad apart from yourself there is nothing. This Prapancha was not at all created. As the objects seen in the Indrajala is not the truth, so also the Maya. What is seen through the Maya the namarupas Jagat is also not true. The barren son who comes to kill us if it is the truth then the Maya and the Mayakarya Jagat should also be there. Therefore, Maya and Mayakarya Jagat also is not the truth. The Jagat seems to be a Branthi (illusion). If there is no illusion then there is no Jagat. Therefore, do not get units this illusion! You are the Brahma. If you are alone your ganya? there is nothing else understand this truth mercilessly.

Shruti:–

[॥ Yetasmaki – Me – Vendrajala – Maparam Yagarbhavassy – Sthitam~। Retheyscheytati – Hasta mastaka – Padapro adhbuta – Nanakaram~। Paryayanena – Shishuteva – Yauvanna – Jara – Veshairaneykairavytam. Pashyeti – Shrunoti – Jaghragrate – Yatha – Ghachhabthatha – Ghachhati~॥.]

Tatparya:– Hey Anjeneya! What can I say about this peculiar Maya? The Sperm of the Purusha's Gharbha the cells gets into the woman's womb and then it takes the form of a Pinda. Then it takes the shape of a body with legs and arms, eyes, ears, nose etc. Then it is ready and takes birth. From the woman's womb it is born as a Shishu then as days go by, he turns into a Youth, Kaumara, Vardhikyadi, has several Mayovikaras. The Sthula Sharira keeps noticing the external world, by eating, hearing, smelling, smiling, crying, playing, learns to talk, thus it all seems wonders of wonders for the parents as the child unfolds itself with various facts: Then all of a sudden just as it came it will fly off. Aha! It becomes as though it is lifeless, a Jadapadartha and one is astonished as it had taken a namarupa but now there is no such thing everything becomes Shunya. There is no other knowledge as the Indrajalavidya. Due to this Indrajala, this Mayabranthi Jagat seems to be. If one does not have such an illusion for them there is no Jagat.

Shruti:–

[॥ Nasadaseet Nosadaseet – Tadaneem Tama – Aseet~॥.]

Tatparya:– The mysterious Maya is an illusion like serpent and the rope, the dead serpent is Asatya “Anivarchineeya”. It means – The idea of this and that one should not perceive – It means, the rope which is far away should not be assumed as a serpent. One must go near it examine it then you realise its not a snake and so you cannot conclude that it was a dead one another example is presented here. Like a son of a barren woman it can never happen. From after the rope does look like a serpent. so it may not be true. [It means if it is dead it is not there, the untruth is it is not there]. This is not and being dead is apposed to each other and it will not stand any concept. For one object there cannot be truth and untruth. Therefore, the serpent seen in the rope it cannot be assumed. as ‘Sadasatta’. At first, the rope as a serpent, and that it is a dead one, it may appear so. After examining, it may not appear to be dead. Therefore, what is dead, that is in reality this object is (Annavarchirya). This can be concluded as an illusion or Branthi. This rope servant example can be compared with Brahma and Maya – it is the workable Jagat which may not be real.

Shruti:–

[॥ Brahma matra – Masannahi – Neha Nanasti – Kinchana – Athonyat – Sakalam – Mrisha – Gnatey Dvaitam – Navidyate~॥.]

Tatparya:– According to all these shrutis not only Brahma is one but also that there is no Maya. Maya is also not the कार्यवादा जगत~।. This Branthirupa Maya is the untrue Maya by saying it so (Yekamevadvi Tiyam – Brahma) Truely, 8 Brahma is only one and to say that there is no other thing is Advaitya. There is no harm in saying is Mahavakya. There is no Dvaitya. But Nanavidha Namarupa, Kriyayoga, it is seen by the eyes which cannot deceive, so how can one say that there is no Jagat? A doubt arises just as the rope and serpent. So one can assume that it is Mithya, even after understanding this aspect (Asat – Mrigatrishnavat – ghandarva Nagaram – Yatha). It appears to be like a Marujala. But the water found in a desert nobody drinks, Even if one is extremely thirsty there is no experience that they have drunk this water. But in this Jagat is there no experience of one quenching the think? Therefore, how can one say that the water of the Jagat is like the desert one. If we doubt, then in a dream state when one sees objects and incidents and in the wakeful state one realises that all that was just Mithya or illusion. Similarly, the Jagat may also be Asatya. Thus, in one state the Jagat, its Asatya and in another state it does not appear so. All that is seen in the Jagrata avasta and the experiences, all this is not seen in the Nidra Avasta. Thus, the Jagat is Asatya and only Brahma is Satya. The Jagat was never born. (Maya Matra – Midam Dvaitam – Advaitam – Parmarthartha~॥) as stated in the Mandukyo Upanishad. According to this Maya is illusion the Jagat is always not true.

Paripurna Bhodey

Shruti:–

[॥ Brahmai Veda – Mamritam – Purustath Brahma Pashashabdhah Brahma – Dakshinascharena~। Adhashobeha dvaram cha – Prasritam – Brahmai Vedam – Vishwamidam – Varistam~॥.]

Mundko Upanishad.

Tatparya:– Hey Anjeneya! Everything is Brahma other than there is nothing else. In front, back, right–side, left–side, up, down everywhere there is Paripurna Parabrahma is the only one. The Branthi rupa this Jagat is not there. Please understand that you are Paribrahma.

Shruti:–

[॥ Deha Iti – Shankavoto –Jarda – Samadhatum – Bahyadrishya – Paribrahmam Vaditi~॥.]

Shruti:–

[॥ Natu – Dehadi – Satyatva – Bhodanaya – Vipachitam~॥.]

Adhyotma Upanishad.

Tatparya:– Dehabhimana, one thinks so, he will not be only subjected to Prarabdha. On the other hand one who is aware that there is no Prapancha he will not have Prarabdha.

Shruti:–

[॥ Purnamadha – Purnamidam – Purnatpurna – Mudachytey~। Purna – Sya – Purnamadaya – Purnameva – Shishyatey~॥.]

Brihadaranya Upanishad.

Tatparya:– Here there everywhere omnipotent paripurna. In the Loka, Chandra, Surya all of nature are supposed to be Paripurna. Apart from all of this Paripurna is superior. In this Loka, all the objects which are paripurna, there is something more above this, that is Sarvaparipurna Brahma is covered all over. You have the feeling or Bhava is that you are Brahma. (Yadrishytam – Tannashyam) It means – The objects which are sighted by you has no rupa and are liable to be Destroyed. (Drishyam – Sarvamanatmeti – Drigeva – Parmatma – Thatha). Whatever is seen by your eyes due to your illusion or lack of discernment, all of it is Asatya. The Paripurna Parabrahma is you yourself who is Sarvasakshi which is indestructible. Apart from you who is Brahma himself there is nothing else. See within yourself and be in bliss. This is as told by Sri Ramachandra to his ardent Bhakta Anjeneya. Anjeneya became ecstatic and was highly elated he became highly reverential and after a while stood up as he had attained Prapanchagnapti, and did Sastanga Namaskara to Sri Rama and then stood up.

Shruti:–

[॥ Dehabudhyasmi – Dasoham – Jiva Bhuddhya – Tvadamshaka.~॥ Atma Bhuddhya – Tvamevaha miti – Mey Nischita matihi~॥.]

Tatparya:– Anjeneya addresses Sri Ramachandra I had the knowledge that I am the body and was busy doing your Padasevey. (Kshupti pasey?) Cycle of birth and death. The rest of my life I understand that I am Sarva Jiva Samista Svarupa, and you who is Sarveshwara and I being an Amsha of you Deha Triya Vilakshana Pratyagatma Parabrahmavey is me, with this knowledge that I am you, you are me saying. Anjeneya relates further what is Svamubhava.

Svanubhavananda

Shruti:–

[॥ Kvagatam – Kenavanitam – Katraleena – Midam Jagat~। Adhunaiva – Mayadristam – Nastikim – Mahadadhubatam~॥ Heyam – Kimupadeyam – Kimanyatey – Kimvilakshanam.~। Akhandananda – Peeyusha – PurnaBrahma – Maharnavey~॥ Na Kinchidatra – Pashyami – Nashruńomi – Naveydyamaiham~।. Svatmanaiva – Sadananda – Rupenesmi – Svalakshańaha~॥ Aasangoha – Masangoha – Mahilingoha – Mahamhari~। Prashanthoha – Mananthoham – Paripurna – Shvirantana~॥ Akartah – Mabhoktaha – Mavikaroha – Mavyaha~॥ Shuddha – Bhodhasva rupoham – Kevaloham – Sadashivaha~॥.]

Aadhyatmo Upanishad.

Tatparya:– Before creation where was the Namarupa Jagat which the eyes can see? How did it come? Again during the Pralaya where does this Jagat Merge? That means – before creation there was only Brahma was there therefore, there was no Jagat and now what is being seen is only an illusion, that seems to be an opinion. This is visible during wakefulness but during sleep one cannot see even a tiny ******. This is really very surprising. There is no such other thing than this aspect. Even what cannot be seen is also Maya's delusion. The mind or manasu itself is a Branthirupa. If there is no delusion there is no Jagat and everything is Paripurna Brahma. Therefore, what does not one want? What is Anya or option? What is Vilakshana? What is Vidhi or fate? What is Nisheda or forbidden? What is Shuddha or pure? What is impure? All the above is a figment of imagination and untrue or Asatya. Paripurna Brahma is me which is (Nandanistey?) I will not listen to anything other than myself. I will not see anything and I will not understand. I will enjoy the bliss or Ananda that there is Paripurna Brahma within me.

Shruti:–

[॥ Mayyeva – Sakalam – Jatam – MayaSarvam – Pratistitam~। Mayasarvam – Layam yati – Tad – Brahmadvaya – Masmyahavi~॥.]

Kaivalyo Upanishad

Tatparya:– Due to winds the waves keep coming to the shore, it seems to look as if everytime new waves are created in the Samudra. Similarly, in the Gnananda Sagara, Swarupi, the Maya wind envelops me and the Prapancha waves keeps being born and gets destroyed. But me being an Atmaswarupa will never ever get destroyed – advitiya Paripurna Parabrahma I will always remain.

Shruti:–

[॥ Oobhey – Vidvarta – Puńyapapey – Vidhuya – Niranjanaha – Samya – Mupyeti~॥.]

Tatparya:– Apart from myself there is nothing else. I have no Vidhinishadhas. Where do I get my punyas or good deeds? Wherefrom the Papas or sins arise? It means – Truely, there is a three fold Manovakyas there is an opinion that I will not Commit Papas or Puńyas. I am an Asanga I have no identity. I am neither male or female or transgender, I have no body at all, I am bodiless, immovable, I am Hari, I am Prashanta, I have no end, I am Paripurna Parabrahma. I am “Akarta” – and “Abhokta”. It means I do not perform actions and do not experience anything. I have neither birth nor death, I cannot be destroyed. I am pure Bhodaswarupa, I am only an Atma. I am a Nityamangalasvarupa all the above description are given by Anjeneya to Sri Ramachandra. The Lord was extremely happy about. He made Anjeneya sit at his side and said the following.

Shruti:–

[॥ Pramado – Brahma Nistayam – Nakartavyaha – Kadachauna~। Pramado – Mrityurityahu – Rvidyayam – Brahma Vedana (1) Yathaprikistam – Shyavalam – Kshanamatram – Nathistati~। Avrińoti – Thatha Maya – Pragnamvapi – Parajmkhim~॥.]

Adhyatmopa Upanishad.

Tatparya:– (2) According to the above upanishad Hey Anjeneya! Ordinarily, everyone does not have Aavasanakala? Mrityu does not happen – If forgetting Brahmanista is death However, much you clean a pond or a lake covered with moss it will again come and cover up the water body. Similarly, when one however much he tries to become a Brahma, but if he is not concentrating on the practice of Brahmanista, immediately the maya will envelop him if he is not alert. Then, he will be subjected and caught in the cycle of birth and death. So, the Lord's advises him not to forget even for a moment the practice of Brahmanistey.

Shruti:–

[॥ Tadvishnoha – Parmam – Padam – Sadapashyatiti – Surayaha~। Diviva – Chakshu ratatam – Tadvi praso – Viparyavo Jagrivam – Sasamindantey – Vishnoyarth – Parmam Vadam~॥.]

Om satya Upanishad. This is Rkvakya.

Tatparya:– The Sarvavyapaka Sachitananda Parabrahma, if he is Brahma he understands his Swarupa. He is enveloped in the entire cosmos like the sky. He is also present in the heart of all Jivas as only as a Sakshi, and it looks that his Netra is overseeing as if he has a rupa and reigns supreme over all the Dehindriyas and the Prapancha. Those Brahma Gnanis who are berfit of Ahankaras and mamkaras always as an atmaswarupa and always repeats “Aham Brahmasi”. He will be engaged in Atmanusandhanam and behaves like an Abhedadristi.

Now this Grantha is complete the lines below are Tikakara Mangala.

॥ Mangalam Shri, Ramachandragey~। Mangalam Sri Ramachandragey~। Mangalam Sri Bharata Sita Lakshmana adirigey~। Mangalam Ida Nalisiruvargey~। Mangalam Ida Vachirisvarigey~। Mangalagli Niruta Shankara Charanayuggaligey~॥.

इति श्री मतपरामद्दाँस परिव्राजाकार्चाया श्री लक्ष्मणानन्द पूज्य पादा शिष्य श्री शिवानन्दा सुब्रामण्यप्रणीता करनाटका टीका समेता उपनिषद सारा रत्नावळी समाप्तः ॐ ततसत

॥ श्री गणेषोभ्यो नमः~॥

मनीषा पँचकम्

(1) श्लोक~॥

 कदाचि छन्कराचार्यः काशीम्प्रती पुरीयायौ~।चाँडालारूपीणीम द्रष्टवा गच्चगच्चीती चाभ्रवीत~॥

Tatparya:– Sri Shiva as a Parma Karunanidhi, as a Bhagvanta, came in the form of a Chandala (low caste) while Sri Shankaracharya was walking around with his disciples in the streets of the holy city of Kashi. Shankaracharya seeing the Chandala addressed this “Hey Chandala move away from me.” Hearing this that Bhagavanta replied as follows.

(2) Shloka:–

 अन्नामया दन्नमया मथना चैत्नयमेवा चैत्न्यात~।यतीवरा दूरीकरतुरम वान्छासी कीःब्रूद्दी गछगछीती~॥

Tatparya:– Hey Sanyasi!

Shruti:–

[॥ Sava – Yesha – Purshonna – Rasarniya – Iti~॥.]

By constantly chanting Soham and doing pranam or namaskara. The gist is to bow down to the Atma.

For this entire prayer, the Phala shruti or what one can benefit from reciting this sacred stangas every morning.

Shloka:–

 श्लोकत्रय मीदम पुणयम लोकात्रया विभूषकम~।प्रातःकाले पठेदयस्तु सगछत्परमम पदम~॥

Tatparya:– The 3 shlokas in this prayer serves as a decoration to all the 3 Lokas. It means the joy and happiness which one attains in the 3 lokas, it is above all this. It is the highest. Such a stuti one who recites in the early morning, understanding its finer and inner nuances the seeker will definitely attain Mukti.

Iti Srimach Shankaracharya Bhagwat Pada Acharya Virichitam Pratasmarana Trayam Samaptam~॥ OM TAT SAT~॥

There is a self illuminating light, that which has neither a begining or an end, the ideal Brahma that is Purushottamma.

Shloka:–

 [ पुरुसँझे – शरीरेस्मी – देहे – तीष्ठतीती – पुरषा–क्षे त्रक्ष्नीः~।तसमात–सर्वक्षे त्रक्ष – सक्षी मात्रो – आत्मा उत्तम्मा – परमः इति पुरुषोत्तमा~॥

पुरु – शरीरा – It means the Chidabasa who resides in the body.

पुरुषोतम्मा – The atma which is present in every creatures in the immense. Just as a witness. Therefore I bow my head with utmost reverence to Chaitanya form of Atma – the Parabrahma Atma in all the bodies and is everywhere. He has no name form, or any action. Therefore: –

Shruti:–

[सोहम भावो – नमस्कारः]

Mandala Brahmano Upanishad

According to the above maha vakya or sentence, that Parabrahma Atma the Seeker should always be thinking as himself.

