\sethyphenation{kannada}{
ಅಂಕದ
ಅಂಕೆ-ಗೊ-ಳ-ಗಾಗಿ
ಅಂಗ
ಅಂಗಡಿ
ಅಂಗ-ಡಿಗೆ
ಅಂಗ-ಲ-ಕ್ಷ-ಣ-ಗಳ
ಅಂಗ-ಲಾಚಿ
ಅಂಗಳ
ಅಂಗ-ಳದಿ
ಅಂಗ-ಸಾ-ಧನೆ
ಅಂಗ-ಸಾ-ಧ-ನೆ-ಯನ್ನು
ಅಂಗಾಂಗ
ಅಂಗಾಂ-ಗ-ಗಳನ್ನು
ಅಂಗಾಂ-ಗ-ಗಳಲ್ಲಿ
ಅಂಗಾಂ-ಗ-ಗಳು
ಅಂಗಾತ
ಅಂಗೀ-ಕ-ರಿಸಿ
ಅಂಗೀ-ಕ-ರಿ-ಸಿ-ದರೆ
ಅಂಗು-ಲವೂ
ಅಂಗೈ
ಅಂಚನ್ನು
ಅಂಚಿಗೆ
ಅಂಚಿನ
ಅಂಚಿ-ನಲ್ಲಿ
ಅಂಜ-ದ-ಳು-ಕದೆ
ಅಂಜಿ-ಕೆ-ಯ-ದೆ-ನ-ಗೆಲ್ಲಿ
ಅಂಜಿ-ಕೆ-ಯನ್ನೂ
ಅಂಟಿ-ಕೊಂ-ಡ-ರೇನು
ಅಂಟಿ-ಕೊಂ-ಡಿತು
ಅಂಟಿ-ಕೊಂಡು
ಅಂಟಿ-ಸಿ-ಕೊಂ-ಡಿ-ದ್ದರೂ
ಅಂಟು-ರೋ-ಗವೇ
ಅಂತ
ಅಂತಂ-ದು-ಕೊಂ-ಡೆಯಾ
ಅಂತಃ-ಕ-ರ-ಣ-ವನ್ನು
ಅಂತರ
ಅಂತ-ರಂಗ
ಅಂತ-ರಂ-ಗಕ್ಕೆ
ಅಂತ-ರಂ-ಗದ
ಅಂತ-ರಂ-ಗ-ದಲ್ಲಿ
ಅಂತ-ರಂ-ಗ-ದಲ್ಲೇ
ಅಂತ-ರಂ-ಗ-ದಿಂದ
ಅಂತ-ರಂ-ಗ-ವ-ಲ-ಯಕ್ಕೆ
ಅಂತ-ರ-ವಿದ್ದೇ
ಅಂತ-ರಾ-ತ್ಮ-ವನ್ನು
ಅಂತ-ರಾ-ರ್ಥ-ವನ್ನು
ಅಂತ-ರಾ-ಳ-ದಲ್ಲಿ
ಅಂತ-ರಾ-ಳ-ದೊ-ಳಗೆ
ಅಂತ-ರಾ-ಳ-ವನ್ನು
ಅಂತ-ರಾ-ಳ-ವನ್ನೇ
ಅಂತ-ರ್ಗ-ತ-ನಾ-ಗಿಯೂ
ಅಂತ-ರ್ದೃಷ್ಟಿ
ಅಂತ-ರ್ದೃ-ಷ್ಟಿ-ಯನ್ನೇ
ಅಂತ-ರ್ದೃ-ಷ್ಟಿ-ಯಿಂದ
ಅಂತ-ರ್ಮು-ಖ-ವಾ-ಗಿ-ರು-ವುದನ್ನು
ಅಂತ-ರ್ಮು-ಖಿ-ಯಾ-ಗಿ-ರು-ತ್ತಿ-ದ್ದು-ದನ್ನು
ಅಂತ-ರ್ಯಾ-ಮಿ-ಯಾಗಿದ್ದಾನೆ
ಅಂತ-ರ್ಯಾ-ಮಿ-ಯಾಗಿಯೂ
ಅಂತ-ರ್ಯಾ-ಮಿ-ಯಾದ
ಅಂತಲೂ
ಅಂತಲೆ
ಅಂತಲೇ
ಅಂತ-ಶ್ಶಕ್ತಿ-ಯನ್ನು
ಅಂತ-ಶ್ಶಕ್ತಿ-ಯನ್ನೂ
ಅಂತ-ಶ್ಶಕ್ತಿಯು
ಅಂತ-ಶ್ಶುದ್ಧಿ-ಗಾಗಿ
ಅಂತಸ್ತು
ಅಂತ-ಸ್ಸತ್ತ್ವ
ಅಂತ-ಸ್ಸತ್ತ್ವ-ದಿಂ-ದೊ-ಡ-ಗೂ-ಡಿದ
ಅಂತ-ಸ್ಸ-ತ್ತ್ವವು
ಅಂತ-ಸ್ಸ-ತ್ತ್ವವೂ
ಅಂತ-ಸ್ಸತ್ತ್ವ-ವೆಂ-ಬುದು
ಅಂತ-ಸ್ಸತ್ವ
ಅಂತ-ಸ್ಸತ್ವ-ಇ-ವು-ಗಳನ್ನೆಲ್ಲ
ಅಂತ-ಸ್ಸ-ತ್ವಕ್ಕೆ
ಅಂತ-ಸ್ಸತ್ವ-ದಿಂದ
ಅಂತ-ಸ್ಸತ್ವ-ವನ್ನು
ಅಂತ-ಸ್ಸ-ತ್ವವೂ
ಅಂತಹ
ಅಂತ-ಹ-ವರೆಲ್ಲ
ಅಂತಿಮ
ಅಂತಿ-ಮಾ-ವ-ಸ್ಥೆ-ಯಾದ
ಅಂತೂ
ಅಂತೆ
ಅಂತೆಯೇ
ಅಂತ್ಯ
ಅಂತ್ಯಕ್ಕೆ
ಅಂತ್ಯ-ಕ್ರಿಯೆ
ಅಂತ್ಯಜ
ಅಂತ್ಯ-ಜರ
ಅಂತ್ಯ-ದಲ್ಲಿ
ಅಂತ್ಯ-ದ-ವರೆ-ಗಿನ
ಅಂಥ
ಅಂಥ-ದನ್ನು
ಅಂಥ-ದ-ರಲ್ಲಿ
ಅಂಥದೇ
ಅಂಥ-ದೇ-ನಾ-ದರೂ
ಅಂಥ-ದೇನು
ಅಂಥಲ್ಲಿ
ಅಂಥ-ಲ್ಲಿಗೆ
ಅಂಥವನ
ಅಂಥವ-ನಿಗೆ
ಅಂಥವನು
ಅಂಥವನ್ನು
ಅಂಥವರ
ಅಂಥವ-ರನ್ನು
ಅಂಥವ-ರಿಗೆ
ಅಂಥವರು
ಅಂಥವು-ಗಳನ್ನೆಲ್ಲ
ಅಂಥಾ
ಅಂಥಾ-ದ್ದೇ-ನನ್ನು
ಅಂದದ್ದೇ
ಅಂದ-ಮಾ-ತ್ರಕ್ಕೆ
ಅಂದ-ಮೇಲೆ
ಅಂದರೂ
ಅಂದರೆ
ಅಂದಿ
ಅಂದಿ-ಗಾ-ಗು-ವಷ್ಟು
ಅಂದಿಗೆ
ಅಂದಿನ
ಅಂದಿ-ನಿಂದ
ಅಂದು
ಅಂದು-ಕೊಂಡೆ
ಅಂದು-ಕೊ-ಳ್ಳು-ತ್ತಾರೆ
ಅಂದೂ
ಅಂದೇ
ಅಂಧ-ವಿ-ಶ್ವಾಸ
ಅಂಧವೇ
ಅಂಧ-ಶ್ರದ್ಧೆ
ಅಂಬಾ
ಅಂಬಿ-ಗ-ರನ್ನೇ
ಅಂಬಿ-ಗ-ರಿಂದ
ಅಂಬಿ-ಗ-ರಿಂ-ದಲೂ
ಅಂಬಿ-ಗ-ರಿಗೆ
ಅಂಬಿ-ಗ-ರಿಗೋ
ಅಂಬಿ-ಗರು
ಅಂಬು
ಅಂಶ
ಅಂಶ-ಗಳ
ಅಂಶ-ಗ-ಳ-ನ್ನಾ-ದರೂ
ಅಂಶ-ಗಳನ್ನು
ಅಂಶ-ಗಳನ್ನೂ
ಅಂಶ-ಗಳು
ಅಂಶ-ಗ-ಳು-ಹ-ಲ-ವಾರು
ಅಂಶ-ವನ್ನು
ಅಂಶ-ವನ್ನೇ
ಅಂಶ-ವಿದೆ
ಅಂಶವೂ
ಅಂಶ-ವೆಂದರೆ
ಅಂಶ-ವೇನೆಂದರೆ
ಅಂಶ-ವೊಂ-ದನ್ನು
ಅಕ-ಳಂಕ
ಅಕ-ಸ್ಮಾ-ತ್ತಾಗಿ
ಅಕಾರಣ-ವಾಗಿ
ಅಕಾ-ಲ-ದಲ್ಲೆಲ್ಲ
ಅಕೃ-ತ್ಯ-ಗಳು
ಅಕೃ-ತ್ರಿಮ
ಅಕ್ಕ
ಅಕ್ಕಂ-ದಿ-ರನ್ನು
ಅಕ್ಕಂ-ದಿ-ರಿಗೆ
ಅಕ್ಕಂ-ದಿರು
ಅಕ್ಕ-ಪಕ್ಕ-ದಲ್ಲಿ
ಅಕ್ಕ-ರೆ-ಯಿಂದ
ಅಕ್ಕಿ
ಅಕ್ಕಿ-ಬೇಳೆ-ಕಾಳು-ಗಳು
ಅಕ್ಬರ್
ಅಕ್ಬ-ರ್ಲೋನಿ
ಅಕ್ರಮ
ಅಕ್ಷ-ರ-ಗಳನ್ನು
ಅಕ್ಷ-ರ-ಗಳಲ್ಲಿ
ಅಕ್ಷ-ರಶಃ
ಅಕ್ಷ-ರಾ-ಭ್ಯಾಸ
ಅಖಂಡ
ಅಖಂ-ಡ-ರಾ-ಜ್ಯ-ವನ್ನು
ಅಖಂಡಾ
ಅಖಂ-ಡಾ-ನಂದ
ಅಖಂಡಾನಂದ-ಅರು
ಅಖಂಡಾನಂದರ
ಅಖಂಡಾನಂದ-ರಂತೂ
ಅಖಂಡಾನಂದ-ರನ್ನು
ಅಖಂಡಾನಂದ-ರನ್ನೂ
ಅಖಂಡಾನಂದ-ರಿ-ಗಂತೂ
ಅಖಂಡಾನಂದ-ರಿಗೂ
ಅಖಂಡಾನಂದ-ರಿಗೆ
ಅಖಂಡಾನಂದ-ರಿ-ಗೊಂದು
ಅಖಂಡಾನಂದ-ರಿ-ದ್ದಾರೆ
ಅಖಂಡಾನಂದರು
ಅಖಂಡಾನಂದ-ರೆ-ದುರು
ಅಖಂಡಾನಂದ-ರೊಂ-ದಿಗೆ
ಅಗಣ್ಯ
ಅಗತ್ಯ
ಅಗತ್ಯ-ಕ್ಕಿಂತ
ಅಗತ್ಯ-ವಾಗಿ
ಅಗತ್ಯ-ವಾದ
ಅಗತ್ಯ-ವಾ-ದರೂ
ಅಗತ್ಯ-ವಿ-ದೆಯೆ
ಅಗತ್ಯ-ವಿ-ದ್ದಷ್ಟು
ಅಗತ್ಯ-ವಿ-ರು-ವ-ವ-ರಿ-ಗೆಲ್ಲ
ಅಗತ್ಯವೂ
ಅಗತ್ಯವೇ
ಅಗತ್ಯ-ವೇ-ನಿದೆ
ಅಗತ್ಯ-ವೇನೂ
ಅಗ-ಲಿದ
ಅಗ-ಲು-ವಿ-ಕೆಯ
ಅಗಳಿ
ಅಗ-ಳಿ-ಹಾ-ಕಿತ್ತು
ಅಗಾ-ಧ-ವಾ-ಗಿದೆ
ಅಗಾ-ಧ-ವಾದ
ಅಗಾ-ಧ-ವಾ-ದುದು
ಅಗು-ಳನ್ನು
ಅಗೆದು
ಅಗೋ
ಅಗೋ-ಚರ
ಅಗ್ಗದ
ಅಗ್ನಿ-ಕಾ-ರ್ಯಾದಿ
ಅಗ್ನಿ-ಕುಂ-ಡ-ದಲ್ಲಿ
ಅಗ್ನಿಗೆ
ಅಗ್ನಿ-ಜ್ವಾಲೆ
ಅಗ್ನಿ-ತೇ-ಜ-ದಿಂದ
ಅಗ್ನಿ-ಪರೀಕ್ಷೆ-ಯ
ಅಗ್ನಿ-ಪರೀಕ್ಷೆ--ಯಲ್ಲಿ
ಅಗ್ನಿ-ಪರೀಕ್ಷೆ-ಯೇ
ಅಗ್ನಿ-ಮಾಂ-ದ್ಯ-ದಿಂ-ದಾ-ಗಿ-ಎಂ-ದರೆ
ಅಗ್ನಿ-ಯಲ್ಲಿ
ಅಗ್ನಿಯು
ಅಗ್ನಿ-ಸ್ಪರ್ಶ
ಅಗ್ನಿ-ಹಂಸ
ಅಗ್ರ-ಗಣ್ಯ
ಅಗ್ರ-ಗ-ಣ್ಯ-ನೆಂದು
ಅಗ್ರ-ಭಾ-ಗ-ವನ್ನು
ಅಗ್ರ-ಸ್ಥಾನ
ಅಚ-ಲ-ವಾ-ದದ್ದು
ಅಚಿಂ-ತ್ಯ-ನಾದ
ಅಚ್ಚರಿ
ಅಚ್ಚರಿ-ಆ-ನಂದ
ಅಚ್ಚರಿ-ಕು-ತೂ-ಹ-ಲ-ಗಳಿಂದ
ಅಚ್ಚರಿ-ಗೊ-ಳ್ಳು-ತ್ತಿದ್ದ
ಅಚ್ಚರಿಯ
ಅಚ್ಚರಿ-ಯಿಂದ
ಅಚ್ಚರಿ-ಯಿಲ್ಲ
ಅಚ್ಚರಿಯೂ
ಅಚ್ಚರಿ-ಯೆ-ನಿ-ಸ-ಬ-ಹುದು
ಅಚ್ಚರಿ-ಯೇ-ನಿಲ್ಲ
ಅಚ್ಚ-ಳಿ-ಯದ
ಅಚ್ಚು-ಕ-ಟ್ಟಾಗಿ
ಅಚ್ಚು-ಮೆ-ಚ್ಚಿನ
ಅಚ್ಚು-ಮೆಚ್ಚು
ಅಚ್ಚೊ-ತ್ತಿ-ದ್ದವು
ಅಜೀರ್ಣ
ಅಜೀ-ರ್ಣ-ದಿಂ-ದಾಗಿ
ಅಜೇ-ಯ-ರ-ನ್ನಾಗಿ
ಅಜ್ಜಿ
ಅಜ್ಜಿಯ
ಅಜ್ಜಿಯೂ
ಅಜ್ಞಾ-ತ-ವಾಗಿ
ಅಜ್ಞಾನ
ಅಜ್ಞಾನ-ಗಳೂ
ಅಜ್ಞಾನದ
ಅಜ್ಞಾನ-ದ-ಲ್ಲಿ-ರಲು
ಅಜ್ಞಾನ-ದಿಂ-ದಾಗಿ
ಅಜ್ಞಾನಿ-ಗಳು
ಅಜ್ಞೇ-ಯ-ತಾ-ವಾದ
ಅಜ್ಞೇ-ಯ-ತಾ-ವಾ-ದ-ಕ್ಕಿಂ-ತಲೂ
ಅಜ್ಞೇ-ಯ-ತಾ-ವಾ-ದ-ದಲ್ಲೇ
ಅಜ್ಮೀ-ರಕ್ಕೆ
ಅಟ್ಟಿ-ದರು
ಅಟ್ಟಿ-ಬಿಟ್ಟ
ಅಟ್ಟಿ-ಸಿ-ಕೊಂಡು
ಅಡ-ಕ-ವಾ-ಗಿ-ರ-ಬ-ಹು-ದಾದ
ಅಡಗಿ-ಕೊಂ-ಡಿತ್ತು
ಅಡಗಿ-ಕೊಂ-ಡಿ-ರು-ತ್ತಾರೆ
ಅಡಗಿದೆ
ಅಡಗಿ-ದ್ದುದು
ಅಡಗಿ-ರು-ವುದನ್ನು
ಅಡಗಿಲ್ಲ
ಅಡಗಿ-ಸಿ-ಕೊಂಡು
ಅಡಗಿ-ಸಿ-ಟ್ಟಿದ್ದ
ಅಡಗಿ-ಸಿಟ್ಟು
ಅಡಗಿ-ಸಿ-ಡುವ
ಅಡಗಿ-ಸು-ವುದು
ಅಡ-ಚ-ಣೆ-ಗಳನ್ನೂ
ಅಡ-ಚ-ಣೆ-ಗ-ಳಾ-ಗಿ-ದ್ದುವು
ಅಡಿ
ಅಡಿ-ಕೆ-ಯನ್ನೂ
ಅಡಿ-ಗ-ಲ್ಲು-ಗಳು
ಅಡಿಗೆ
ಅಡಿ-ಗೆಗೆ
ಅಡಿ-ಗೆ-ಮ-ನೆ-ಯೊ-ಳಗೆ
ಅಡಿ-ಗೆಯ
ಅಡಿ-ಗೆ-ಯನ್ನು
ಅಡಿ-ಯಿ-ಡಲು
ಅಡುಗೆ
ಅಡು-ಗೆ-ಯನ್ನು
ಅಡು-ಗೆ-ಯನ್ನೇ
ಅಡೆ-ತ-ಡೆ-ಯನ್ನೂ
ಅಡ್ಡ
ಅಡ್ಡ-ಗಟ್ಟಿ
ಅಡ್ಡ-ದಾರಿ-ಗಿಳಿ-ಯ-ಲಿಲ್ಲ
ಅಡ್ಡ-ದಾರಿ-ಗೆ-ಳೆ-ಯು-ವಂ-ತಹ
ಅಡ್ಡ-ವಾ-ಗಿ-ರುವ
ಅಡ್ಡಾ-ಡುತ್ತ
ಅಡ್ಡಾ-ಡು-ತ್ತಿದ್ದು
ಅಡ್ಡಾ-ಡು-ತ್ತಿ-ರ-ಲಿ-ಅದೇ
ಅಡ್ಡಾ-ಡು-ತ್ತಿ-ರು-ವುದು
ಅಡ್ಡಾ-ಡು-ವಾಗ
ಅಡ್ಡಿ
ಅಡ್ಡಿ-ಯನ್ನು
ಅಡ್ಡಿ-ಯಾ-ಗುವು
ಅಣ-ಕಿ-ಸುತ್ತ
ಅಣ-ಕಿ-ಸು-ತ್ತಿದೆ
ಅಣಿ-ಗೊ-ಳಿ-ಸಿ-ದರು
ಅಣಿ-ಗೊ-ಳಿ-ಸು-ತ್ತಿ-ದ್ದರು
ಅಣಿ-ಗೊ-ಳಿ-ಸು-ತ್ತಿ-ದ್ದುದು
ಅಣಿಮಾ
ಅಣಿ-ಮಾದಿ
ಅಣು
ಅಣು-ಕು-ಚೇ-ಷ್ಟೆ-ಗಳನ್ನು
ಅಣ್ಣ
ಅಣ್ಣ-ತ-ಮ್ಮಂ-ದಿರೇ
ಅಣ್ಣನ
ಅಣ್ಣ-ನಿಂದ
ಅತಿ
ಅತಿ-ಕ್ರ-ಮಿಸಿ
ಅತಿ-ಕ್ರ-ಮಿ-ಸಿದ
ಅತಿಥಿ
ಅತಿ-ಥಿ-ಗ-ಳಾಗಿ
ಅತಿ-ಥಿ-ಗ-ಳಾ-ಗಿದ್ದು
ಅತಿ-ಥಿ-ಗ-ಳಿಗೆ
ಅತಿ-ಥಿ-ಗಳೂ
ಅತಿ-ಥಿ-ಯಂತೆ
ಅತಿ-ಥಿ-ಯಾಗಿ
ಅತಿ-ದೊಡ್ಡ
ಅತಿ-ಧಾ-ರಾ-ಳ-ತ-ನ-ವನ್ನು
ಅತಿ-ಮಾನವ
ಅತಿ-ಮಾ-ನುಷ
ಅತಿ-ಮುಖ್ಯ
ಅತಿ-ಮು-ಖ್ಯ-ನೆಂದು
ಅತಿ-ಯಾ-ಗಿ-ರ-ಬೇಕು
ಅತಿ-ಯಾದ
ಅತಿ-ರೇ-ಕಕ್ಕೆ
ಅತಿ-ರೇ-ಕ-ಗಳನ್ನು
ಅತಿ-ರೇ-ಕ-ವನ್ನು
ಅತಿ-ರೇ-ಕ-ವಾ-ಗು-ವುದು
ಅತಿ-ರೇ-ಕ-ವಾ-ದರೂ
ಅತಿ-ರೇ-ಕ-ವಾ-ದರೆ
ಅತಿ-ಶ-ಯ-ವಾಗಿ
ಅತಿ-ಶ-ಯ-ವಾ-ದದ್ದು
ಅತಿ-ಶ-ಯೋ-ಕ್ತಿ-ಯಲ್ಲ
ಅತಿ-ಶ-ಯೋ-ಕ್ತಿ-ಯಾ-ಗಲಿ
ಅತಿ-ಸ-ಹ-ಜ-ವಾಗಿ
ಅತಿ-ಸಾರ
ಅತಿ-ಸಾ-ರವೇ
ಅತೀಂ-ದ್ರಿಯ
ಅತೀಂ-ದ್ರಿ-ಯ-ವಾದ
ಅತೀ-ತ-ನಾ-ಗಿಯೂ
ಅತೀ-ತ-ನಾ-ದ-ವ-ನನ್ನು
ಅತೀ-ತನೂ
ಅತೀ-ತ-ರಾ-ದ-ವರು
ಅತೀ-ತ-ವಾ-ಗಿ-ದೆಯೋ
ಅತೀ-ತ-ವಾದ
ಅತೀವ
ಅತ್ತ
ಅತ್ತದ್ದೇ
ಅತ್ತಲೇ
ಅತ್ತಿತ್ತ
ಅತ್ತಿ-ದ್ದೇನೆ
ಅತ್ತಿ-ರ-ಲಿಲ್ಲ
ಅತ್ತು
ಅತ್ತು-ಕ-ರೆದು
ಅತ್ತು-ಕ-ರೆಯು
ಅತ್ತು-ಕ-ರೆ-ಯು-ತ್ತಿ-ದ್ದಾರೆ
ಅತ್ತು-ಬಿಟ್ಟ
ಅತ್ತು-ಬಿ-ಟ್ಟರು
ಅತ್ಯಂತ
ಅತ್ಯ-ಪೂರ್ವ
ಅತ್ಯ-ಮೂ-ಲ್ಯ-ವಾದ
ಅತ್ಯಲ್ಪ
ಅತ್ಯಾ-ದ-ರ-ದಿಂದ
ಅತ್ಯಾ-ನಂ-ದದ
ಅತ್ಯಾ-ನಂ-ದ-ದಿಂದ
ಅತ್ಯಾ-ನಂ-ದ-ವಾ-ಗಿ-ಬಿ-ಟ್ಟಿತು
ಅತ್ಯಾ-ನಂ-ದ-ವಾ-ಯಿತು
ಅತ್ಯಾ-ವ-ಶ್ಯಕ
ಅತ್ಯಾ-ವ-ಶ್ಯ-ಕ-ವಾ-ಗಿತ್ತು
ಅತ್ಯಾ-ವ-ಶ್ಯ-ಕ-ವಾದ
ಅತ್ಯಾ-ಶ್ಚರ್ಯ
ಅತ್ಯಾ-ಶ್ಚ-ರ್ಯ-ಗೊಂ-ಡರು
ಅತ್ಯಾ-ಶ್ಚ-ರ್ಯದ
ಅತ್ಯಾ-ಶ್ಚ-ರ್ಯ-ದಿಂದ
ಅತ್ಯಾ-ಶ್ಚ-ರ್ಯ-ವಾ-ಯಿತು
ಅತ್ಯಾ-ಸ-ಕ್ತಿ-ಯಿಂದ
ಅತ್ಯು-ತ್ತಮ
ಅತ್ಯು-ತ್ಸಾ-ಹ-ದಿಂದ
ಅತ್ಯು-ನ್ನತ
ಅತ್ಯು-ನ್ನ-ತ-ವಾದ
ಅತ್ಯು-ಪ-ಯುಕ್ತ
ಅಥವಾ
ಅದ
ಅದ-ಕ್ಕ-ನು-ಗು-ಣ-ವಾ-ಗಿಯೇ
ಅದ-ಕ್ಕಾಗಿ
ಅದ-ಕ್ಕಿಂತ
ಅದ-ಕ್ಕಿಂ-ತಲೂ
ಅದ-ಕ್ಕು-ತ್ತ-ರ-ವಾಗಿ
ಅದಕ್ಕೂ
ಅದಕ್ಕೆ
ಅದ-ಕ್ಕೆಲ್ಲ
ಅದಕ್ಕೇ
ಅದ-ಕ್ಕೇ-ನರ್ಥ
ಅದ-ಕ್ಕೊಂದು
ಅದ-ಕ್ಕೊ-ಪ್ಪ-ಲಿಲ್ಲ
ಅದ-ಕ್ಕೊಪ್ಪಿ
ಅದ-ಕ್ಕೊ-ಪ್ಪಿ-ದರು
ಅದದು
ಅದ-ನ್ನ-ವನು
ಅದನ್ನು
ಅದ-ನ್ನೆ-ದು-ರಿ-ಸಲು
ಅದ-ನ್ನೆಲ್ಲ
ಅದನ್ನೇ
ಅದ-ನ್ನೇ-ಅ-ಲ್ಲಿ-ಯ-ವ-ರೆಗೆ
ಅದ-ನ್ನೇನು
ಅದ-ನ್ನೊಂದು
ಅದಮ್ಯ
ಅದ-ಮ್ಯ-ವೆ-ನ್ನ-ಬೇಕು
ಅದರ
ಅದ-ರಂತೆ
ಅದ-ರಂ-ತೆಯೇ
ಅದ-ರತ್ತ
ಅದ-ರ-ದರ
ಅದ-ರಲ್ಲಿ
ಅದ-ರ-ಲ್ಲಿದ್ದ
ಅದ-ರಲ್ಲೂ
ಅದ-ರಲ್ಲೇ
ಅದ-ರ-ಲ್ಲೇನೂ
ಅದ-ರ-ಲ್ಲೊಂದು
ಅದ-ರಿಂದ
ಅದ-ರಿಂ-ದಲೂ
ಅದ-ರಿಂ-ದಾ-ಗಿಯೇ
ಅದ-ರಿಂ-ದೇ-ನಂತೆ
ಅದ-ರಿಂ-ದೇನೂ
ಅದರೆ
ಅದ-ರೊಂ-ದಿಗೆ
ಅದ-ರೊ-ಳಕ್ಕೆ
ಅದ-ರೊ-ಳಗೆ
ಅದ-ರೊ-ಳ-ಗೆ-ಸೆದು
ಅದ-ರೊ-ಳಗೇ
ಅದಲ್ಲ
ಅದ-ಷ್ಟ-ರಲ್ಲೇ
ಅದಾ-ಗಲೇ
ಅದಾ-ವು-ದನ್ನೂ
ಅದಿನ್ನೂ
ಅದಿ-ರು-ವುದು
ಅದೀಗ
ಅದು
ಅದುಮಿ
ಅದು-ವ-ರೆಗೂ
ಅದೂ
ಅದೃ-ಶ್ಯ-ರಾಗಿ
ಅದೃ-ಶ್ಯ-ರಾ-ದರು
ಅದೃ-ಶ್ಯ-ವಾ-ಯಿತು
ಅದೃಷ್ಟ
ಅದೃ-ಷ್ಟಕ್ಕೆ
ಅದೃ-ಷ್ಟ-ವ-ಶಾತ್
ಅದೃ-ಷ್ಟ-ವಾ-ದಿ-ಯಾಗ
ಅದೃ-ಷ್ಟ-ವಿ-ಶೇಷ
ಅದೃ-ಷ್ಟ-ಶಾ-ಲಿ-ಗ-ಳಿಗೆ
ಅದೆ
ಅದೆಂ-ತಹ
ಅದೆಂಥ
ಅದೆಂದು
ಅದೆಲ್ಲ
ಅದೆ-ಲ್ಲ-ವನ್ನೂ
ಅದೆ-ಲ್ಲವೂ
ಅದೆಷ್ಟು
ಅದೆಷ್ಟೊ
ಅದೆಷ್ಟೋ
ಅದೇ
ಅದೇಕೆ
ಅದೇ-ಕೆಂ-ಬು-ದನ್ನು
ಅದೇಕೋ
ಅದೇ-ನದು
ಅದೇನು
ಅದೇನೂ
ಅದೇ-ನೆಂ-ದರೆ
ಅದೇನೋ
ಅದೊಂದು
ಅದೊಂದೂ
ಅದೊಂದೇ
ಅದೋ
ಅದ್ದ-ರಿಂದ
ಅದ್ಬು-ತ-ವಾ-ಗಿದೆ
ಅದ್ಭುತ
ಅದ್ಭು-ತ-ಅ-ಪೂರ್ವ
ಅದ್ಭು-ತ-ಕಾ-ರ್ಯ-ಗಳಲ್ಲಿ
ಅದ್ಭು-ತ-ಗಳನ್ನು
ಅದ್ಭು-ತ-ಗಳು
ಅದ್ಭು-ತತೆ
ಅದ್ಭು-ತ-ವನ್ನು
ಅದ್ಭು-ತ-ವ-ಲ್ಲವೆ
ಅದ್ಭು-ತ-ವಾ-ಗಿದೆ
ಅದ್ಭು-ತ-ವಾದ
ಅದ್ಭು-ತ-ವಾ-ದದ್ದು
ಅದ್ಭು-ತ-ವಾ-ದದ್ದೇ
ಅದ್ಭು-ತವೇ
ಅದ್ಭು-ತ-ವೇನು
ಅದ್ಭು-ತಾ-ನಂದ
ಅದ್ಭು-ತಾ-ನಂ-ದ-ರನ್ನೂ
ಅದ್ವಿ-ತೀಯ
ಅದ್ವೈತ
ಅದ್ವೈ-ತಈ
ಅದ್ವೈ-ತ-ಗೋ-ಸ್ವಾ-ಮಿಯ
ಅದ್ವೈ-ತ-ಜ್ಞಾನ
ಅದ್ವೈ-ತ-ಜ್ಞಾ-ನದ
ಅದ್ವೈ-ತ-ಜ್ಞಾ-ನ-ವೆಂ-ಬುದು
ಅದ್ವೈ-ತ-ತ-ತ್ತ್ವದ
ಅದ್ವೈ-ತ-ತ-ತ್ತ್ವ-ವನ್ನು
ಅದ್ವೈ-ತದ
ಅದ್ವೈ-ತ-ವಾ-ದ-ವನ್ನು
ಅದ್ವೈ-ತ-ವೇ-ದಾಂ-ತದ
ಅದ್ವೈ-ತಾ-ನಂದ
ಅದ್ವೈ-ತಾ-ನಂ-ದರು
ಅದ್ವೈ-ತಾ-ನಂ-ದ-ರು-ಹೀಗೆ
ಅದ್ವೈತಿ
ಅಧರ್ಮ
ಅಧ-ರ್ಮ-ವನ್ನು
ಅಧ-ರ್ಮವು
ಅಧಿಕ
ಅಧಿ-ಕ-ವಾಗಿ
ಅಧಿ-ಕ-ವಾ-ಗಿ-ರು-ವ-ವ-ರನ್ನು
ಅಧಿ-ಕ-ವಾ-ಗುತ್ತ
ಅಧಿ-ಕ-ವಾ-ಗು-ತ್ತ-ದೆಯೇ
ಅಧಿ-ಕ-ವಾ-ಗು-ತ್ತಿತ್ತು
ಅಧಿ-ಕಾರ
ಅಧಿ-ಕಾ-ರ-ಮದ
ಅಧಿ-ಕಾ-ರ-ವನ್ನು
ಅಧಿ-ಕಾ-ರ-ವಾಣಿ
ಅಧಿ-ಕಾ-ರ-ವಾ-ಣಿಯ
ಅಧಿ-ಕಾ-ರ-ವಾ-ಣಿ-ಯಿಂದ
ಅಧಿ-ಕಾ-ರ-ವಿದೆ
ಅಧಿ-ಕಾ-ರ-ವಿಲ್ಲ
ಅಧಿ-ಕಾರಿ
ಅಧಿ-ಕಾ-ರಿ-ಗಳ
ಅಧಿ-ಕಾ-ರಿ-ಗಳೂ
ಅಧಿ-ಕಾ-ರಿಯ
ಅಧಿ-ಕಾ-ರಿ-ಯನ್ನು
ಅಧಿ-ಕಾ-ರಿ-ಯಾ-ಗ-ಬೇ-ಕಾ-ದ-ವನು
ಅಧಿ-ಕಾ-ರಿ-ಯಾದ
ಅಧಿ-ಕಾ-ರಿ-ಯೊಂ-ದಿಗೆ
ಅಧಿ-ಕೃತ
ಅಧಿ-ಕೃ-ತ-ನಾ-ಗಿ-ರ-ಬೇಕು
ಅಧಿ-ದೇ-ವ-ತೆಯೇ
ಅಧಿ-ಷ್ಠಾತೃ
ಅಧೀ-ನ-ದ-ಲ್ಲಿತ್ತು
ಅಧೀ-ನ-ದಲ್ಲೇ
ಅಧೈ-ರ್ಯ-ಕ್ಕೊ-ಳ-ಗಾ-ಗು-ತ್ತಿ-ದ್ದರು
ಅಧೈ-ರ್ಯ-ಗೊಂಡು
ಅಧೋ-ಗ-ತಿ-ಗಿಳಿ-ದಿವೆ
ಅಧ್ಯಕ್ಷ
ಅಧ್ಯ-ಕ್ಷ-ಮ-ಹೋ-ದಯ
ಅಧ್ಯ-ಕ್ಷ-ರ-ನ್ನಾಗಿ
ಅಧ್ಯ-ಕ್ಷರು
ಅಧ್ಯ-ಯನ
ಅಧ್ಯ-ಯ-ನ-ಕ್ಕಾಗಿ
ಅಧ್ಯ-ಯ-ನ-ಕ್ಕಾ-ಗಿಯೇ
ಅಧ್ಯ-ಯ-ನದ
ಅಧ್ಯ-ಯ-ನ-ವನ್ನು
ಅಧ್ಯ-ಯ-ನ-ವ-ನ್ನೆಲ್ಲ
ಅಧ್ಯ-ಯ-ನವು
ಅಧ್ಯ-ಯಿಸಿ
ಅಧ್ಯ-ಯಿ-ಸಿ-ದಂತೆ
ಅಧ್ಯ-ಯಿ-ಸಿ-ದರು
ಅಧ್ಯ-ಯಿ-ಸಿ-ದ-ವನು
ಅಧ್ಯ-ಯಿ-ಸಿ-ದ-ವರೂ
ಅಧ್ಯ-ಯಿ-ಸಿ-ದ-ವರೆಲ್ಲ
ಅಧ್ಯಾತ್ಮ
ಅಧ್ಯಾ-ತ್ಮಕ್ಕೆ
ಅಧ್ಯಾ-ತ್ಮದ
ಅಧ್ಯಾ-ತ್ಮ-ಭಾವ
ಅಧ್ಯಾ-ತ್ಮ-ವೆ-ನ್ನು-ವು-ದೆಲ್ಲ
ಅಧ್ಯಾ-ತ್ಮವೋ
ಅಧ್ಯಾ-ತ್ಮ-ಶೀಲ
ಅಧ್ಯಾ-ತ್ಮಾ-ನುಭ
ಅಧ್ಯಾ-ಪ-ಕನ
ಅಧ್ಯಾ-ಪ-ಕರ
ಅಧ್ಯಾ-ಪ-ಕ-ರಿಗೆ
ಅಧ್ಯಾ-ಪ-ಕರು
ಅಧ್ಯಾ-ಪ-ಕ-ರೆಲ್ಲ
ಅನ
ಅನಂತ
ಅನಂ-ತ-ತೆಯ
ಅನಂ-ತ-ತೆ-ಯನ್ನು
ಅನಂ-ತ-ದೆ-ಡೆಗೆ
ಅನಂ-ತ-ನಾದ
ಅನಂ-ತರ
ಅನಂ-ತ-ರದ
ಅನಂ-ತ-ರವೂ
ಅನಂ-ತ-ರ-ವೂ-ಶ್ರೀ-ರಾ-ಮ-ಕೃ-ಷ್ಣರು
ಅನಂ-ತ-ರವೇ
ಅನಂ-ತ-ವಾದ
ಅನಂ-ತ-ವಾ-ದದ್ದು
ಅನಂ-ತಾ-ತ್ಮ-ದಲ್ಲಿ
ಅನಂ-ತಾ-ನಂತ
ಅನ-ಕೂ-ಲಸ್ಥ
ಅನ-ಕ್ಷ-ರ-ತೆಯ
ಅನ-ಕ್ಷ-ರ-ಸ್ಥ-ನಾ-ಗಿಯೇ
ಅನ-ಗತ್ಯ
ಅನ-ತಿ-ದೂ-ರದ
ಅನ-ನು-ಕೂ-ಲ-ತೆ-ಗಳ
ಅನ-ನು-ಕೂ-ಲ-ತೆಯೋ
ಅನನ್ಯ
ಅನ-ಪೇ-ಕ್ಷಿತ
ಅನರ್ಘ್ಯ
ಅನ-ರ್ಥ-ಗ-ಳಿಗೆ
ಅನ-ವ-ರತ
ಅನಾ-ಚಾರ
ಅನಾ-ಥ-ರ-ಕ್ಷ-ಕ-ನಾದ
ಅನಾ-ಥರೇ
ಅನಾ-ದರ
ಅನಾ-ದಿ-ಕಾ-ಲದ
ಅನಾ-ದಿ-ಕಾ-ಲ-ದಿಂದ
ಅನಾ-ದಿ-ಯಾದ
ಅನಾ-ದಿ-ಯಿಂದ
ಅನಾ-ರೋಗ್ಯ
ಅನಾ-ರೋ-ಗ್ಯ-ನಿ-ಶ್ಶಕ್ತಿ-ಗಳ
ಅನಾ-ರೋ-ಗ್ಯ-ದಿಂದ
ಅನಾ-ರೋ-ಗ್ಯ-ಪೀ-ಡಿ-ತ-ರಾ-ಗಿ-ದ್ದ-ರಿಂದ
ಅನಾ-ರೋ-ಗ್ಯ-ವುಂ-ಟಾ-ಯಿತು
ಅನಾ-ಹು-ತದ
ಅನಿ-ಕೇ-ತ-ನ-ರು-ಎಂ-ದರೆ
ಅನಿ-ತ್ಯ-ವಾದ
ಅನಿ-ರಾ-ಕ-ರಣಂ
ಅನಿ-ರಾ-ಕ-ರ-ಣ-ಮಸ್ತು
ಅನಿ-ರೀ-ಕ್ಷಿತ
ಅನಿ-ರೀ-ಕ್ಷಿ-ತ-ವಾಗಿ
ಅನಿ-ರೀ-ಕ್ಷಿ-ತ-ವಾ-ದ-ದ್ದೇ-ನಲ್ಲ
ಅನಿರ್
ಅನಿ-ರ್ವ-ಚ-ನೀಯ
ಅನಿ-ರ್ವ-ಚ-ನೀ-ಯ-ವಾದ
ಅನಿ-ಲದ
ಅನಿ-ವಾರ್ಯ
ಅನಿ-ಶ್ಚಿ-ತ-ತೆ-ಗಳ
ಅನಿಷ್ಟ
ಅನಿ-ಷ್ಟ-ಗ-ಳಿಗೂ
ಅನಿ-ಸಿಕೆ
ಅನಿ-ಸಿ-ಕೆ-ಗಳನ್ನು
ಅನಿ-ಸಿ-ಕೆ-ಗ-ಳಿಂ-ದಾಗಿ
ಅನಿ-ಸಿ-ಕೆ-ಗಳೇ
ಅನಿ-ಸಿ-ತು-ಸಾ-ಧಾ-ರಣ
ಅನಿ-ಸು-ತ್ತ-ದೆ-ನಾನು
ಅನು
ಅನು-ಕಂಪ
ಅನು-ಕಂಪೆ
ಅನು-ಕಂ-ಪೆ-ಇ-ವನ್ನು
ಅನು-ಕಂ-ಪೆಯ
ಅನು-ಕಂ-ಪೆಯೇ
ಅನು-ಕ-ರಣೆ
ಅನು-ಕ-ರಿ-ಸು-ತ್ತಿ-ದ್ದರು
ಅನು-ಕ-ರಿ-ಸು-ವಂತೆ
ಅನು-ಕೂಲ
ಅನು-ಕೂ-ಲ-ಕರ
ಅನು-ಕೂ-ಲ-ಕ-ರ-ವಾ-ಗಿತ್ತು
ಅನು-ಕೂ-ಲ-ಕ-ರ-ವಾದ
ಅನು-ಕೂ-ಲ-ಗಳನ್ನು
ಅನು-ಕೂ-ಲ-ಗ-ಳಿ-ದ್ದು-ವು-ಇದು
ಅನು-ಕೂ-ಲ-ತೆ-ಗಳನ್ನು
ಅನು-ಕೂ-ಲ-ತೆ-ಗಳನ್ನೆಲ್ಲ
ಅನು-ಕೂ-ಲ-ತೆಗೆ
ಅನು-ಕೂ-ಲ-ತೆ-ಯಿದೆ
ಅನು-ಕೂ-ಲ-ತೆಯೂ
ಅನು-ಕೂ-ಲ-ವಾ-ಗಿತ್ತು
ಅನು-ಕೂ-ಲ-ವಾ-ಗು-ತ್ತ-ದೆಯೆ
ಅನು-ಕೂ-ಲ-ವಾ-ದಂತೆ
ಅನು-ಕೂ-ಲ-ವಾ-ದೀತು
ಅನು-ಕೂ-ಲ-ವಾ-ದ್ದ-ರಿಂದ
ಅನು-ಕೂ-ಲ-ವಾ-ಯಿತು
ಅನು-ಕೂ-ಲ-ವಿ-ದ್ದಂತೆ
ಅನು-ಕೂ-ಲ-ವಿಲ್ಲ
ಅನು-ಕೂ-ಲವೂ
ಅನು-ಕೂ-ಲಸ್ಥ
ಅನು-ಕೂ-ಲಿ-ಸಿ-ಕೊ-ಟ್ಟರು
ಅನು-ಕೂ-ಲಿ-ಸು-ವಂ-ತಿತ್ತು
ಅನು-ಗು-ಣ-ವಾಗಿ
ಅನು-ಗು-ಣ-ವಾ-ಗಿದೆ
ಅನು-ಗು-ಣ-ವಾ-ಗಿ-ದ್ದುವು
ಅನು-ಗ್ರಹ
ಅನು-ಗ್ರ-ಹ-ವಾ-ಗಿದೆ
ಅನು-ಗ್ರ-ಹಿಸಿ
ಅನು-ಗ್ರ-ಹಿಸು
ಅನು-ಗ್ರ-ಹಿ-ಸು-ತ್ತಾಳೆ
ಅನು-ಗ್ರ-ಹಿ-ಸು-ವಂತೆ
ಅನು-ಚ-ರ-ರಿಗೆ
ಅನುಜ್ಞೆ
ಅನು-ದಿ-ನವೂ
ಅನು-ದ್ದೇ-ಶ-ಪೂ-ರ್ವ-ಕ-ವಾ-ಗಿಯೇ
ಅನು-ಪಮ
ಅನು-ಬಂ-ಧ-ಗಳು
ಅನು-ಭವ
ಅನು-ಭ-ವಕ್ಕೆ
ಅನು-ಭ-ವ-ಗ-ಮ್ಯ-ವಾದ
ಅನು-ಭ-ವ-ಗಳ
ಅನು-ಭ-ವ-ಗ-ಳ-ನ್ನಂತೂ
ಅನು-ಭ-ವ-ಗ-ಳ-ನ್ನಾ-ಗಲಿ
ಅನು-ಭ-ವ-ಗಳನ್ನು
ಅನು-ಭ-ವ-ಗಳನ್ನೆಲ್ಲ
ಅನು-ಭ-ವ-ಗ-ಳನ್ನೇ
ಅನು-ಭ-ವ-ಗಳಲ್ಲಿ
ಅನು-ಭ-ವ-ಗ-ಳಾ-ಗು-ತ್ತಿ-ದ್ದುವು
ಅನು-ಭ-ವ-ಗ-ಳಾ-ವುವೂ
ಅನು-ಭ-ವ-ಗ-ಳಿಂ-ದೆಲ್ಲ
ಅನು-ಭ-ವ-ಗ-ಳಿಗೂ
ಅನು-ಭ-ವ-ಗಳು
ಅನು-ಭ-ವ-ಗಳೂ
ಅನು-ಭ-ವ-ಗ-ಳೆಲ್ಲ
ಅನು-ಭ-ವ-ಗ-ಳೆಷ್ಟು
ಅನು-ಭ-ವ-ಗಳೇ
ಅನು-ಭ-ವದ
ಅನು-ಭ-ವ-ದಿಂದ
ಅನು-ಭ-ವ-ದಿಂ-ದಾಗಿ
ಅನು-ಭ-ವ-ಪೂರ್ಣ
ಅನು-ಭ-ವ-ವನ್ನು
ಅನು-ಭ-ವ-ವಾ-ಗ-ಬ-ಹುದು
ಅನು-ಭ-ವ-ವಾ-ಗಿತ್ತು
ಅನು-ಭ-ವ-ವಾ-ಗಿತ್ತೋ
ಅನು-ಭ-ವ-ವಾ-ಗಿ-ರ-ಬೇಕು
ಅನು-ಭ-ವ-ವಾ-ಗು-ತ್ತ-ದೆಯೋ
ಅನು-ಭ-ವ-ವಾ-ಗು-ತ್ತಿತ್ತು
ಅನು-ಭ-ವ-ವಾ-ಗು-ವುದನ್ನು
ಅನು-ಭ-ವ-ವಾದ
ಅನು-ಭ-ವ-ವಾ-ದಂ-ತಾ-ಯಿತು
ಅನು-ಭ-ವ-ವಾ-ದ-ನಂ-ತರ
ಅನು-ಭ-ವ-ವಾ-ದರೂ
ಅನು-ಭ-ವ-ವಾ-ದರೆ
ಅನು-ಭ-ವ-ವಾ-ದಾಗ
ಅನು-ಭ-ವ-ವಾ-ದು-ದನ್ನು
ಅನು-ಭ-ವ-ವಾ-ಯಿತು
ಅನು-ಭ-ವವು
ಅನು-ಭ-ವವೇ
ಅನು-ಭ-ವ-ವೇನೂ
ಅನು-ಭ-ವ-ವೊಂದು
ಅನು-ಭ-ವ-ಸಿದ್ಧ
ಅನು-ಭ-ವಾ-ಮೃ-ತ-ವನ್ನು
ಅನು-ಭ-ವಿಸ
ಅನು-ಭ-ವಿ-ಸ-ದ-ವ-ರಿಗೆ
ಅನು-ಭ-ವಿ-ಸ-ಬ-ಹುದು
ಅನು-ಭ-ವಿ-ಸ-ಬ-ಹುದೇ
ಅನು-ಭ-ವಿ-ಸ-ಬೇ-ಕೆಂಬ
ಅನು-ಭ-ವಿ-ಸ-ಲಾ-ರಂ-ಭಿ-ಸಿದ
ಅನು-ಭ-ವಿ-ಸಲು
ಅನು-ಭ-ವಿಸಿ
ಅನು-ಭ-ವಿ-ಸಿತು
ಅನು-ಭ-ವಿ-ಸಿದ
ಅನು-ಭ-ವಿ-ಸಿ-ದನೋ
ಅನು-ಭ-ವಿ-ಸಿ-ದ-ವನೇ
ಅನು-ಭ-ವಿ-ಸಿ-ದುದು
ಅನು-ಭ-ವಿ-ಸಿದೆ
ಅನು-ಭ-ವಿ-ಸಿ-ದೆನೋ
ಅನು-ಭ-ವಿ-ಸಿದ್ದ
ಅನು-ಭ-ವಿ-ಸಿದ್ದು
ಅನು-ಭ-ವಿ-ಸಿ-ರ-ಲಿ-ಲ್ಲ-ಅಂ-ತಹ
ಅನು-ಭ-ವಿ-ಸುತ್ತ
ಅನು-ಭ-ವಿ-ಸು-ತ್ತದೆ
ಅನು-ಭ-ವಿ-ಸು-ತ್ತಿತ್ತು
ಅನು-ಭ-ವಿ-ಸು-ತ್ತಿದ್ದ
ಅನು-ಭ-ವಿ-ಸು-ತ್ತಿ-ದ್ದಂ-ತಹ
ಅನು-ಭ-ವಿ-ಸು-ತ್ತಿ-ದ್ದಾನೆ
ಅನು-ಭ-ವಿ-ಸು-ತ್ತಿ-ದ್ದುದು
ಅನು-ಭ-ವಿ-ಸು-ತ್ತಿ-ರುವ
ಅನು-ಭ-ವಿ-ಸು-ತ್ತಿ-ರು-ವ-ವನೂ
ಅನು-ಭ-ವಿ-ಸುವ
ಅನು-ಭ-ವಿ-ಸು-ವಂ-ತಾ-ಗು-ತ್ತದೆ
ಅನು-ಭ-ವಿ-ಸು-ವಂತೆ
ಅನು-ಭ-ವಿ-ಸು-ವ-ವನು
ಅನು-ಭ-ವಿ-ಸು-ವ-ವರು
ಅನು-ಭ-ವಿ-ಸು-ವಾಗ
ಅನು-ಭ-ವಿ-ಸು-ವುದು
ಅನು-ಭ-ವುಂ-ಟಾ-ದಾಗ
ಅನು-ಮತಿ
ಅನು-ಮ-ತಿ-ಪ್ರೋ-ತ್ಸಾಹ
ಅನು-ಮ-ತಿ-ಪತ್ರ
ಅನು-ಮ-ತಿ-ಪ-ತ್ರ-ಗಳು
ಅನು-ಮ-ತಿಯ
ಅನು-ಮ-ತಿ-ಯನ್ನು
ಅನು-ಮ-ತಿ-ಯಿ-ತ್ತರು
ಅನು-ಮಾನ
ಅನು-ಮಾ-ನ-ವುಂ-ಟಾ-ಯಿತು
ಅನು-ಮಾ-ನಾ-ಸ್ಪ-ದ-ವಾದ
ಅನು-ಮೋ-ದಿ-ಸುತ್ತ
ಅನು-ಮೋ-ದಿ-ಸು-ತ್ತಿ-ದ್ದರು
ಅನು-ಮೋ-ದಿ-ಸು-ವು-ದಿಲ್ಲ
ಅನು-ಯಾ-ಯಿ-ಗಳು
ಅನು-ಯಾ-ಯಿ-ಯಾ-ಗಿದ್ದ
ಅನು-ಯಾ-ಯಿಯೇ
ಅನು-ರ-ಕ್ತಿ-ಯುಂ-ಟಾ-ಗು-ತ್ತದೆ
ಅನು-ರ-ಕ್ತಿ-ಯುಂ-ಟಾ-ದಂ-ತೆಲ್ಲ
ಅನು-ವಾ-ದಿ-ಸುವ
ಅನು-ಷ್ಠಾನ
ಅನು-ಷ್ಠಾ-ನಕ್ಕೆ
ಅನು-ಷ್ಠಾ-ನಾ-ತ್ಮ-ಕ-ವಾ-ದದ್ದು
ಅನು-ಸ-ರಣೆ
ಅನು-ಸ-ರಿ-ಸ-ಬ-ಹು-ದಾ-ದಂ-ತಹ
ಅನು-ಸ-ರಿ-ಸ-ಬೇಕು
ಅನು-ಸ-ರಿಸಿ
ಅನು-ಸ-ರಿ-ಸಿ-ಕೊಂಡು
ಅನು-ಸ-ರಿ-ಸುತ್ತ
ಅನು-ಸ-ರಿ-ಸು-ತ್ತಿದ್ದ
ಅನು-ಸ-ರಿ-ಸು-ವಂ-ತಾ-ಗು-ತ್ತದೆ
ಅನು-ಸಾರ
ಅನು-ಸಾ-ರ-ವಾಗಿ
ಅನು-ಸಾ-ರ-ವಾದ
ಅನೇಕ
ಅನೇ-ಕರ
ಅನೇ-ಕ-ರನ್ನು
ಅನೇ-ಕ-ರಿ-ದ್ದರು
ಅನೇ-ಕರು
ಅನೇ-ಕಾ-ನೇಕ
ಅನೈ-ತಿಕ
ಅನೈ-ತಿ-ಕತೆ
ಅನ್ನ
ಅನ್ನ-ಬ-ಟ್ಟೆ-ಗ-ಳಿ-ಗಾಗಿ
ಅನ್ನ-ಬ-ಟ್ಟೆಗೆ
ಅನ್ನ-ಕ್ಕಾಗಿ
ಅನ್ನದ
ಅನ್ನ-ಪೂರ್ಣೆ
ಅನ್ನ-ಪೂ-ರ್ಣೆ-ಯರು
ಅನ್ನ-ಬ-ಟ್ಟೆಗೂ
ಅನ್ನ-ಬ-ಹು-ದಾ-ಗಿತ್ತು
ಅನ್ನ-ಲಾ-ರಂ-ಭಿ-ಸಿ-ದ-ರು-ನೋ-ಡಿ-ದಿರಾ
ಅನ್ನ-ವಿ-ಲ್ಲದೆ
ಅನ್ನಾ-ಹಾ-ರ-ಗ-ಳಿ-ಲ್ಲ-ದ್ದ-ರಿಂದ
ಅನ್ನಿ-ಸ-ತೊ-ಡ-ಗಿ-ದೆ-ಭ-ಗ-ವಂ-ತ-ನನ್ನು
ಅನ್ನಿ-ಸಿತು
ಅನ್ನಿ-ಸಿ-ತು-ಅಲ್ಲ
ಅನ್ನಿ-ಸಿ-ದಾಗ
ಅನ್ನಿ-ಸಿ-ಬಿ-ಟ್ಟಿತು
ಅನ್ನಿ-ಸು-ತ್ತ-ದೆ-ಆಹ್
ಅನ್ನಿ-ಸು-ತ್ತಿತ್ತು
ಅನ್ನಿ-ಸು-ತ್ತಿ-ತ್ತುಈ
ಅನ್ನಿ-ಸು-ತ್ತಿ-ತ್ತು-ಶ್ರೀ-ರಾ-ಮ-ಕೃ-ಷ್ಣರ
ಅನ್ನಿ-ಸು-ತ್ತಿ-ರ-ಲಿಲ್ಲ
ಅನ್ನಿ-ಸು-ವುದು
ಅನ್ನುತ್ತಿ
ಅನ್ಯಥಾ
ಅನ್ಯಾಯ
ಅನ್ಯಾ-ಯ-ಗಳನ್ನು
ಅನ್ಯಾ-ಯದ
ಅನ್ಯಾ-ಯ-ಮಾ-ರ್ಗ-ಗಳಿಂದ
ಅನ್ಯಾ-ಯ-ವಾ-ಗ-ಬ-ಹುದು
ಅನ್ಯಾ-ಯ-ವಾಗಿ
ಅನ್ಯಾ-ಶ್ರ-ಯದಿ
ಅನ್ವ-ಯ-ವಾ-ಗು-ತ್ತದೆ
ಅನ್ವ-ಯ-ವಾ-ಗು-ವು-ದಿಲ್ಲ
ಅನ್ವ-ಯಿ-ಸುವ
ಅನ್ವ-ರ್ಥ-ವಾ-ಗಿದೆ
ಅಪ
ಅಪಕ್ವ
ಅಪ-ಖ್ಯಾತಿ
ಅಪ-ಘಾತ
ಅಪ-ಘಾ-ತ-ದಿಂ-ದಾಗಿ
ಅಪ-ಪ್ರ-ಚಾರ
ಅಪ-ಪ್ರ-ಯೋಗ
ಅಪ-ಮಾನ
ಅಪ-ರಂಜಿ
ಅಪ-ರಂ-ಜಿ-ಯಂತೆ
ಅಪ-ರಾಧ
ಅಪ-ರಾ-ಧ-ಗಳನ್ನು
ಅಪ-ರಾಧಿ
ಅಪ-ರಾಹ್ನ
ಅಪರಿ
ಅಪ-ರಿ-ಗ್ರಹ
ಅಪ-ರಿ-ಚಿತ
ಅಪ-ರಿ-ಚಿ-ತ-ಇ-ವನು
ಅಪ-ರಿ-ಚಿ-ತ-ನೆಂ-ಬಂತೆ
ಅಪ-ರಿ-ಮಿತ
ಅಪ-ರೂಪ
ಅಪ-ರೂ-ಪದ್ದು
ಅಪ-ಲಾಪ
ಅಪ-ವಾದ
ಅಪ-ವಾ-ದ-ವಾ-ಗಿ-ರ-ಲಿಲ್ಲ
ಅಪ-ಹಾಸ್ಯ
ಅಪ-ಹಾ-ಸ್ಯದ
ಅಪಾಯ
ಅಪಾ-ಯ-ಕರ
ಅಪಾ-ಯಕ್ಕೆ
ಅಪಾ-ಯ-ಗಳಿಂದ
ಅಪಾ-ಯದ
ಅಪಾ-ಯ-ದಿಂದ
ಅಪಾ-ಯ-ವನ್ನು
ಅಪಾ-ಯವೇ
ಅಪಾರ
ಅಪಾ-ರ-ವಾದ
ಅಪಾ-ರ-ವಾ-ದದ್ದು
ಅಪಾರ್ಥ
ಅಪಾ-ರ್ಥ-ಮಾಡಿ
ಅಪೂರ್ವ
ಅಪೂ-ರ್ವ-ಜೋಡಿ
ಅಪೂ-ರ್ವ-ವಾದ
ಅಪೂ-ರ್ವ-ವಾ-ದದ್ದು
ಅಪೇಕ್ಷೆ
ಅಪೇ-ಕ್ಷೆಯ
ಅಪೇ-ಕ್ಷೆ-ಯಂತೆ
ಅಪ್ಪ
ಅಪ್ಪಟ
ಅಪ್ಪಾ
ಅಪ್ಪಿ-ಕೊಂ-ಡಿತು
ಅಪ್ರ-ತಿಮ
ಅಪ್ರ-ತಿ-ಹ-ತ-ವಾದ
ಅಪ್ರ-ತೀ-ಕಾ-ರ-ಪೂ-ರ್ವಕಂ
ಅಪ್ರಾಪ್ಯ
ಅಪ್ರಿಯ
ಅಫ-ಘಾ-ನಿ-ಸ್ತಾ-ನ-ದ-ವ-ರೆಗೆ
ಅಫೀ-ಮಿನ
ಅಬ್ಬ
ಅಬ್ಬ-ರಿ-ಸಿ-ದರು
ಅಭಯ
ಅಭ-ಯ-ದಂತೆ
ಅಭಾವ
ಅಭಾ-ವ-ದಿಂ-ದಲ್ಲ
ಅಭಾ-ವ-ವೆ-ನ್ನು-ವುದು
ಅಭಿ
ಅಭಿ-ನಂ-ದ-ನೆ-ಗಳನ್ನು
ಅಭಿ-ನಂ-ದಿ-ಸಿ-ದರು
ಅಭಿ-ಪ್ರಾಯ
ಅಭಿ-ಪ್ರಾ-ಯ-ಕೊ-ಟ್ಟರೂ
ಅಭಿ-ಪ್ರಾ-ಯ-ಗಳನ್ನು
ಅಭಿ-ಪ್ರಾ-ಯ-ಗಳನ್ನೆಲ್ಲ
ಅಭಿ-ಪ್ರಾ-ಯ-ಗಳು
ಅಭಿ-ಪ್ರಾ-ಯ-ದಿಂ-ದಾಗಿ
ಅಭಿ-ಪ್ರಾ-ಯ-ಭೇ-ದ-ಗಳಿಂದ
ಅಭಿ-ಪ್ರಾ-ಯ-ಭೇ-ದವೇ
ಅಭಿ-ಪ್ರಾ-ಯ-ಭೇ-ದ-ವೇ-ರ್ಪಟ್ಟು
ಅಭಿ-ಪ್ರಾ-ಯ-ಮೂ-ಡಿತು
ಅಭಿ-ಪ್ರಾ-ಯ-ವನ್ನು
ಅಭಿ-ಪ್ರಾ-ಯ-ವನ್ನೂ
ಅಭಿ-ಪ್ರಾ-ಯ-ವಿಲ್ಲ
ಅಭಿ-ಪ್ರಾ-ಯ-ವುಂ-ಟಾ-ಯಿತು
ಅಭಿ-ಪ್ರಾ-ಯವೂ
ಅಭಿ-ಪ್ರಾ-ಯ-ವೇ-ನಾ-ಗಿ-ತ್ತೆಂ-ದರೆ
ಅಭಿ-ಪ್ರಾ-ಯ-ವೇ-ನೆಂ-ಬುದು
ಅಭಿ-ಮತ
ಅಭಿ-ಮ-ತ-ವಾ-ಗಿತ್ತು
ಅಭಿ-ಮಾನ
ಅಭಿ-ಮಾ-ನಕ್ಕೂ
ಅಭಿ-ಮಾ-ನಕ್ಕೆ
ಅಭಿ-ಮಾ-ನ-ವನ್ನು
ಅಭಿ-ಮಾ-ನ-ವಿ-ಟ್ಟಿ-ದ್ದಾರೆ
ಅಭಿ-ಮಾ-ನಿ-ಯಾದ
ಅಭಿ-ರುಚಿ
ಅಭಿ-ರು-ಚಿ-ಗಳು
ಅಭಿ-ಲಾಷೆ
ಅಭಿ-ಲಾ-ಷೆಗೆ
ಅಭಿ-ವೃದ್ಧಿ
ಅಭಿ-ವೃ-ದ್ಧಿ-ಗಾಗಿ
ಅಭಿ-ವ್ಯ-ಕ್ತ-ವಾ-ದಷ್ಟು
ಅಭೀ-ರ-ಭೀಃ
ಅಭೀ-ಷ್ಟ-ಗಳನ್ನು
ಅಭೂ-ತ-ಪೂರ್ವ
ಅಭೇದಾ
ಅಭೇ-ದಾ-ನಂದ
ಅಭೇ-ದಾ-ನಂ-ದರ
ಅಭೇ-ದಾ-ನಂ-ದ-ರಿಗೆ
ಅಭೇ-ದಾ-ನಂ-ದರು
ಅಭೇ-ದ್ಯ-ವಾದ
ಅಭ್ಯಂ-ತ-ರ-ವಿಲ್ಲ
ಅಭ್ಯಂ-ತ-ರ-ವಿ-ಲ್ಲ-ವಂತೆ
ಅಭ್ಯ-ಸಿ-ಸಿದ
ಅಭ್ಯಾ-ಗ-ತರ
ಅಭ್ಯಾಸ
ಅಭ್ಯಾ-ಸದ
ಅಭ್ಯಾ-ಸ-ದಲ್ಲಿ
ಅಭ್ಯಾ-ಸ-ದಿಂದ
ಅಭ್ಯಾ-ಸ-ಮಾಡಿ
ಅಭ್ಯಾ-ಸ-ಮಾಡು
ಅಭ್ಯಾ-ಸ-ವ-ಲ್ಲವೆ
ಅಭ್ಯಾ-ಸ-ವಿ-ರ-ಲಿಲ್ಲ
ಅಭ್ಯಾ-ಸವೋ
ಅಭ್ಯು-ತ್ಥಾ-ನ-ಮ-ಧ-ರ್ಮಸ್ಯ
ಅಮಂ-ಗಳ
ಅಮರ
ಅಮ-ರ-ಗೊ-ಳಿ-ಸು-ವು-ದ-ಕ್ಕೋ-ಸ್ಕರ
ಅಮ-ರ-ಗ್ರಂ-ಥ-ಗಳಲ್ಲಿ
ಅಮ-ರ-ತೆ-ಗೇ-ರುವ
ಅಮಲು
ಅಮಾ-ವಾ-ಸ್ಯೆಯ
ಅಮಿತ
ಅಮೀನ
ಅಮೂಲ್ಯ
ಅಮೃ-ತ-ತ್ವದ
ಅಮೃ-ತ-ತ್ವ-ದೆ-ಡೆಗೆ
ಅಮೃ-ತ-ವ-ನ್ನುಂ-ಡ-ವ-ನ್ನ-ಲ್ಲವೆ
ಅಮೃ-ತ-ಸಾ-ಗರ
ಅಮೆ-ರಿ-ಕದ
ಅಮೆ-ರಿ-ಕ-ದಲ್ಲಿ
ಅಮೆ-ರಿ-ಕೆಗೆ
ಅಮೆ-ರಿ-ಕೆಯ
ಅಮೆ-ರಿ-ಕೆ-ಯ-ಲ್ಲಿ-ದ್ದಾಗ
ಅಮೇಲೆ
ಅಮ್ಮ
ಅಮ್ಮಂ-ದಿರೂ
ಅಮ್ಮಾ
ಅಯ-ಸ್ಕಾಂ-ತೀಯ
ಅಯೋ-ಗ್ಯರ
ಅಯೋಧ್ಯೆ
ಅಯೋ-ಧ್ಯೆಗೆ
ಅಯೋ-ಧ್ಯೆಯ
ಅಯೋ-ಧ್ಯೆ-ಯನ್ನು
ಅಯ್ಯಯ್ಯೋ
ಅಯ್ಯೋ
ಅರ
ಅರ-ಗಿಸಿ
ಅರ-ಗಿ-ಸಿ-ಕೊಂಡ
ಅರ-ಗಿ-ಸಿ-ಕೊಂ-ಡ-ದ್ದ-ರಿಂ-ದಲೇ
ಅರ-ಗಿ-ಸಿ-ಕೊ-ಳ್ಳ-ಬಲ್ಲ
ಅರ-ಗಿ-ಸಿ-ಕೊಳ್ಳು
ಅರ-ಣ್ಯ-ಮ-ಯ-ವಾ-ಗಿತ್ತು
ಅರ-ಣ್ಯ-ವಾ-ಗಲಿ
ಅರ-ಣ್ಯ-ವೊಂ-ದ-ರಲ್ಲಿ
ಅರ-ಬರು
ಅರ-ಮನೆ
ಅರ-ಮ-ನೆ-ಗಳು
ಅರ-ಮ-ನೆಗೆ
ಅರ-ಳಲು
ಅರಳಿ
ಅರ-ಳಿ-ನಿಂ-ತಾಗ
ಅರ-ಳಿ-ಸಿ-ಕೊಂಡು
ಅರ-ಳಿ-ಸಿದ
ಅರ-ಳು-ತ್ತಿ-ರುವ
ಅರ-ವತ್ತು
ಅರ-ಸಿ-ನ-ಕುಂ-ಕುಮ
ಅರ-ಸುತ್ತ
ಅರ-ಸು-ತ್ತಿ-ರುವ
ಅರಿ-ತದ್ದು
ಅರಿ-ತದ್ದೇ
ಅರಿ-ತರು
ಅರಿ-ತ-ವನೂ
ಅರಿ-ತ-ವರು
ಅರಿ-ತಾಗ
ಅರಿತಿ
ಅರಿ-ತಿದ್ದ
ಅರಿ-ತಿ-ದ್ದರು
ಅರಿ-ತಿ-ದ್ದ-ವರು
ಅರಿ-ತಿ-ರ-ಬೇ-ಕಾ-ಗು-ತ್ತದೆ
ಅರಿತು
ಅರಿ-ತುಕೊ
ಅರಿ-ತು-ಕೊಂಡ
ಅರಿ-ತು-ಕೊಂ-ಡಿಲ್ಲ
ಅರಿ-ತು-ಕೊಂಡು
ಅರಿ-ತು-ಕೊಳ್ಳ
ಅರಿ-ತು-ಕೊ-ಳ್ಳ-ಬ-ಲ್ಲೆ-ವೇನು
ಅರಿ-ತು-ಕೊ-ಳ್ಳ-ಬೇ-ಕಾದ
ಅರಿ-ತು-ಕೊ-ಳ್ಳ-ಬೇಕು
ಅರಿ-ತು-ಕೊ-ಳ್ಳ-ಲಾ-ಗದು
ಅರಿ-ತು-ಕೊ-ಳ್ಳ-ಲಾ-ಗದೆ
ಅರಿ-ತು-ಕೊ-ಳ್ಳ-ಲಿಲ್ಲ
ಅರಿ-ತು-ಕೊ-ಳ್ಳಲು
ಅರಿ-ತು-ಕೊ-ಳ್ಳ-ಲೇ-ಬೇಕು
ಅರಿ-ತು-ಕೊಳ್ಳು
ಅರಿ-ತು-ಕೊ-ಳ್ಳುವ
ಅರಿ-ತು-ಕೊ-ಳ್ಳು-ವಂ-ತಾ-ಯಿತು
ಅರಿ-ತು-ಕೊ-ಳ್ಳು-ವಂತೆ
ಅರಿ-ತು-ಕೊ-ಳ್ಳು-ವಲ್ಲಿ
ಅರಿ-ತು-ಕೊ-ಳ್ಳು-ವುದು
ಅರಿ-ತೆ-ಜ-ಗ-ನ್ಮಾತೆ
ಅರಿ-ತೆ-ಯಾ-ದರೆ
ಅರಿ-ಯ-ದವ
ಅರಿ-ಯ-ದ-ವರು
ಅರಿ-ಯ-ದಿ-ದ್ದರೆ
ಅರಿ-ಯ-ದಿ-ರು-ವಷ್ಟು
ಅರಿ-ಯ-ಬ-ಲ್ಲರು
ಅರಿ-ಯ-ಬ-ಲ್ಲ-ವ-ರಿಲ್ಲ
ಅರಿ-ಯ-ಬೇ-ಕಾ-ದದ್ದು
ಅರಿ-ಯಲು
ಅರಿ-ಯು-ತ್ತಿ-ದ್ದರು
ಅರಿ-ಯು-ವಂತೆ
ಅರಿ-ಯು-ವಲ್ಲಿ
ಅರಿ-ವನ್ನೇ
ಅರಿ-ವಾಗ
ಅರಿ-ವಾ-ಗ-ದಂತೆ
ಅರಿ-ವಾಗಿ
ಅರಿ-ವಾ-ಗು-ತ್ತದೆ
ಅರಿ-ವಾ-ದಾಗ
ಅರಿ-ವಾ-ದಾ-ಗಲೇ
ಅರಿ-ವಾ-ಯಿತು
ಅರಿ-ವಾ-ಯಿ-ತು-ಸ್ವಾ-ಮೀಜಿ
ಅರಿ-ವಿಗೆ
ಅರಿ-ವಿಗೇ
ಅರಿ-ವಿ-ದ್ದರೂ
ಅರಿ-ವಿನ
ಅರಿ-ವಿ-ನಿಂದ
ಅರಿವು
ಅರಿ-ವುಂ-ಟಾ-ಯಿತು
ಅರಿವೂ
ಅರಿವೇ
ಅರಿಸಿ
ಅರು
ಅರು-ಣ-ಕಿ-ರ-ಣ-ಧಾರಿ
ಅರು-ಣೋ-ದ-ಯ-ಕಾಲ
ಅರು-ಣೋ-ದ-ಯದ
ಅರು-ಣೋ-ದ-ಯ-ವಾದ
ಅರೆ
ಅರೆ-ಗಣ್ಣು
ಅರೆ-ಗ-ತ್ತಲು
ಅರೆ-ಜ್ಞಾ-ನ-ದಿಂದ
ಅರೆ-ಪ್ರ-ಜ್ಞಾ-ವ-ಸ್ಥೆ-ಯನ್ನು
ಅರೆ-ಪ್ರ-ಜ್ಞಾ-ವ-ಸ್ಥೆ-ಯ-ಲ್ಲಿ-ದ್ದಾಗ
ಅರೆ-ಬಿ-ರಿ-ದುವು
ಅರೆ-ಮ-ನ-ಸ್ಸಿನ
ಅರೆವು
ಅರೆ-ಸ-ಹ-ಜಾ-ವ-ಸ್ಥೆ-ಗಿ-ಳಿದು
ಅರೇ-ಬಿಕ್
ಅರ್ಘ್ಯ
ಅರ್ಚ-ಕ-ಕಾ-ರ್ಯದ
ಅರ್ಚ-ಕ-ನಾಗಿ
ಅರ್ಚ-ಕ-ನಾದ
ಅರ್ಚಿಸಿ
ಅರ್ಜಿ
ಅರ್ಜಿ-ಗಳನ್ನು
ಅರ್ಜಿ-ಯನ್ನು
ಅರ್ಜುನ
ಅರ್ಜು-ನನ
ಅರ್ಜು-ನ-ಸಂ-ನ್ಯಾ-ಸಿ-ಯಂತೆ
ಅರ್ಥ
ಅರ್ಥ-ಭಾ-ವ-ಗಳು
ಅರ್ಥ-ಮ-ಹತ್ವ
ಅರ್ಥ-ಗ-ರ್ಭಿತ
ಅರ್ಥದ
ಅರ್ಥ-ದ-ಲ್ಲಲ್ಲ
ಅರ್ಥ-ದಲ್ಲಿ
ಅರ್ಥ-ದಿಂದ
ಅರ್ಥ-ಪ-ಡಿ-ಸಲು
ಅರ್ಥ-ಪ-ಡಿ-ಸುವ
ಅರ್ಥ-ಪೂರ್ಣ
ಅರ್ಥ-ಪೂ-ರ್ಣ-ವಾ-ದದ್ದು
ಅರ್ಥ-ಮಾಡಿ
ಅರ್ಥ-ಮಾ-ಡಿ-ಕೊಂ-ಡದ್ದು
ಅರ್ಥ-ಮಾ-ಡಿ-ಕೊಂ-ಡರು
ಅರ್ಥ-ಮಾ-ಡಿ-ಕೊಂ-ಡರೂ
ಅರ್ಥ-ಮಾ-ಡಿ-ಕೊಂ-ಡಲ್ಲಿ
ಅರ್ಥ-ಮಾ-ಡಿ-ಕೊಂ-ಡ-ವ-ನೆಂ-ದರೆ
ಅರ್ಥ-ಮಾ-ಡಿ-ಕೊಂ-ಡಾರು
ಅರ್ಥ-ಮಾ-ಡಿ-ಕೊಂ-ಡಿದ್ದ
ಅರ್ಥ-ಮಾ-ಡಿ-ಕೊಂಡು
ಅರ್ಥ-ಮಾ-ಡಿ-ಕೊಳ್ಳ
ಅರ್ಥ-ಮಾ-ಡಿ-ಕೊ-ಳ್ಳ-ಬ-ಹುದು
ಅರ್ಥ-ಮಾ-ಡಿ-ಕೊ-ಳ್ಳ-ಬೇಕು
ಅರ್ಥ-ಮಾ-ಡಿ-ಕೊ-ಳ್ಳ-ಬೇ-ಕೆಂದೇ
ಅರ್ಥ-ಮಾ-ಡಿ-ಕೊ-ಳ್ಳ-ಲಾ-ಗದೆ
ಅರ್ಥ-ಮಾ-ಡಿ-ಕೊ-ಳ್ಳ-ಲಾ-ದರೂ
ಅರ್ಥ-ಮಾ-ಡಿ-ಕೊ-ಳ್ಳ-ಲಾ-ರದೆ
ಅರ್ಥ-ಮಾ-ಡಿ-ಕೊ-ಳ್ಳಲಿ
ಅರ್ಥ-ಮಾ-ಡಿ-ಕೊ-ಳ್ಳ-ಲಿ-ಲ್ಲ-ವೆಂದು
ಅರ್ಥ-ಮಾ-ಡಿ-ಕೊ-ಳ್ಳಲು
ಅರ್ಥ-ಮಾ-ಡಿ-ಕೊ-ಳ್ಳು-ತ್ತಿ-ದ್ದರು
ಅರ್ಥ-ಮಾ-ಡಿ-ಕೊ-ಳ್ಳು-ತ್ತಿ-ದ್ದರೆ
ಅರ್ಥ-ಮಾ-ಡಿ-ಕೊ-ಳ್ಳು-ತ್ತಿ-ದ್ದಾನೆ
ಅರ್ಥ-ಮಾ-ಡಿ-ಕೊ-ಳ್ಳು-ತ್ತಿ-ದ್ದುದೇ
ಅರ್ಥ-ಮಾ-ಡಿ-ಕೊ-ಳ್ಳು-ವಷ್ಟು
ಅರ್ಥ-ರ-ಹಿತ
ಅರ್ಥ-ವನ್ನು
ಅರ್ಥ-ವಲ್ಲ
ಅರ್ಥ-ವಾ-ಗ-ತೊ-ಡ-ಗಿ-ತು-ನಿ-ಜ-ವಾದ
ಅರ್ಥ-ವಾ-ಗ-ತೊ-ಡ-ಗಿತ್ತು
ಅರ್ಥ-ವಾ-ಗ-ತೊ-ಡ-ಗಿದೆ
ಅರ್ಥ-ವಾ-ಗ-ದಿ-ದ್ದುದು
ಅರ್ಥ-ವಾ-ಗದೆ
ಅರ್ಥ-ವಾ-ಗ-ಬೇಕು
ಅರ್ಥ-ವಾ-ಗ-ಬೇ-ಕು-ಹಾ-ಗಿದೆ
ಅರ್ಥ-ವಾ-ಗ-ಲಿಲ್ಲ
ಅರ್ಥ-ವಾ-ಗ-ಲಿ-ಲ್ಲ-ವೆನ್ನಿ
ಅರ್ಥ-ವಾ-ಗಲೇ
ಅರ್ಥ-ವಾ-ಗಿ-ರ-ಲಾ-ರದು
ಅರ್ಥ-ವಾಗು
ಅರ್ಥ-ವಾ-ಗು-ತ್ತದೆ
ಅರ್ಥ-ವಾ-ಗು-ತ್ತ-ದೆಯೋ
ಅರ್ಥ-ವಾ-ಗು-ತ್ತಿತ್ತು
ಅರ್ಥ-ವಾ-ಗು-ತ್ತಿ-ದೆ-ಜ-ಗ-ತ್ತಿನ
ಅರ್ಥ-ವಾ-ಗು-ತ್ತಿ-ರ-ಲಿಲ್ಲ
ಅರ್ಥ-ವಾ-ಗು-ತ್ತಿಲ್ಲ
ಅರ್ಥ-ವಾ-ಗು-ವಂ-ತಿಲ್ಲ
ಅರ್ಥ-ವಾ-ಗು-ವಂತೆ
ಅರ್ಥ-ವಾ-ಗು-ವು-ದಿಲ್ಲ
ಅರ್ಥ-ವಾ-ಗು-ವುದು
ಅರ್ಥ-ವಾ-ದಂ-ತೆಲ್ಲ
ಅರ್ಥ-ವಾ-ದದ್ದು
ಅರ್ಥ-ವಾ-ದಾಗ
ಅರ್ಥ-ವಾ-ಯಿತು
ಅರ್ಥ-ವಾ-ಯಿ-ತು-ಸಾ-ಧ-ಕ-ರೆ-ನ್ನಿ-ಸಿ-ಕೊಂ-ಡ-ವರು
ಅರ್ಥ-ವಾ-ಯಿತೇ
ಅರ್ಥ-ವಾ-ಯಿತೋ
ಅರ್ಥ-ವಿದೆ
ಅರ್ಥ-ವಿ-ಹೀನ
ಅರ್ಥವೇ
ಅರ್ಥ-ವೇ-ನಿದೆ
ಅರ್ಥ-ವೇ-ನೆಂ-ದ-ರೆ-ಭ-ಗ-ವಂತ
ಅರ್ಥ-ವ್ಯ-ತ್ಯಾ-ಸ-ವಿ-ದೆ-ಸಾಧು
ಅರ್ಥಾತ್
ಅರ್ಥಿ-ಕ-ವಾಗಿ
ಅರ್ಥೈ-ಸ-ಬ-ಹು-ದಾ-ಗಿತ್ತು
ಅರ್ಥೈ-ಸ-ಬ-ಹುದು
ಅರ್ಥೈಸಿ
ಅರ್ಥೈ-ಸಿ-ದರು
ಅರ್ಧ
ಅರ್ಧಂ-ಬರ್ಧ
ಅರ್ಧ-ಕ್ಕಿಂ-ತಲೂ
ಅರ್ಧಕ್ಕೆ
ಅರ್ಧ-ಗಂ-ಟೆ-ಯಾ-ಗು-ವು-ದ-ಕ್ಕಿಲ್ಲ
ಅರ್ಧ-ದಾರಿ
ಅರ್ಧ-ಪ್ರಜ್ಞೆ
ಅರ್ಧ-ಬಾ-ಹ್ಯ-ಪ್ರಜ್ಞೆ
ಅರ್ಧ-ಬಾಹ್ಯಾ
ಅರ್ಧ-ಬಾ-ಹ್ಯಾ-ವ-ಸ್ಥೆ-ಯ-ಲ್ಲಿದ್ದ
ಅರ್ಧ-ಭಾ-ವಾ-ವ-ಸ್ಥೆಗೆ
ಅರ್ಧ-ಭಾ-ವಾ-ವ-ಸ್ಥೆ-ಯಲ್ಲಿ
ಅರ್ಧ-ರಾತ್ರಿ
ಅರ್ಧ-ವೃ-ತ್ತಾ-ಕಾ-ರ-ದಲ್ಲಿ
ಅರ್ಧ-ಹಾ-ಸ್ಯ-ವಾಗಿ
ಅರ್ಪಿ-ಸ-ಲಾ-ಯಿತು
ಅರ್ಪಿ-ಸಿ-ಕೊಂ-ಡು-ಬಿ-ಟ್ಟಿ-ದ್ದರು
ಅರ್ಪಿ-ಸಿ-ಕೊ-ಳ್ಳ-ಬೇಕು
ಅರ್ಪಿ-ಸಿದ
ಅರ್ಪಿ-ಸಿ-ದರು
ಅರ್ಪಿ-ಸು-ತ್ತಿ-ದ್ದರು
ಅರ್ಹ-ತೆ-ಯನ್ನು
ಅರ್ಹ-ನ-ನ್ನಾಗಿ
ಅಲಂ-ಕ-ರಿಸಿ
ಅಲಂ-ಕ-ರಿ-ಸಿ-ದರು
ಅಲ-ಕ-ನಂದಾ
ಅಲಕ್ಷ್ಯ
ಅಲ-ಕ್ಷ್ಯ-ದಿಂ-ದಿ-ರು-ವಂತೆ
ಅಲ-ಹಾ-ಬಾ-ದಿಗೆ
ಅಲ-ಹಾ-ಬಾ-ದಿನ
ಅಲ-ಹಾ-ಬಾ-ದಿ-ನಲ್ಲಿ
ಅಲ-ಹಾ-ಬಾ-ದಿ-ನ-ಲ್ಲಿದ್ದ
ಅಲು-ಗಾ-ಡದೆ
ಅಲು-ಗಾ-ಡಲೂ
ಅಲು-ಗಾ-ಡಿಸಿ
ಅಲು-ಗಾ-ಡಿ-ಸಿ-ದಳು
ಅಲು-ಗಾ-ಡಿ-ಸಿ-ಬಿ-ಟ್ಟರು
ಅಲು-ಗಾ-ಡಿ-ಸಿ-ಬಿ-ಡು-ತ್ತಾನೆ
ಅಲು-ಗಾ-ಡಿ-ಸಿ-ಬಿ-ಡು-ವಂ-ಥದು
ಅಲೆ
ಅಲೆ-ಗಳು
ಅಲೆ-ದರು
ಅಲೆ-ದ-ಲೆದು
ಅಲೆ-ದಾ-ಟ-ವೇನೂ
ಅಲೆ-ದಾ-ಡ-ಬೇ-ಕಾ-ಯಿತು
ಅಲೆ-ದಾಡಿ
ಅಲೆ-ದಾ-ಡಿ-ಕೊಂ-ಡಿರು
ಅಲೆ-ದಾ-ಡಿದ
ಅಲೆ-ದಾ-ಡಿ-ದರೆ
ಅಲೆ-ದಾ-ಡಿ-ದ-ವ-ರಲ್ಲ
ಅಲೆ-ದಾ-ಡುತ್ತ
ಅಲೆ-ದಾ-ಡು-ತ್ತಿ-ದ್ದಾರೆ
ಅಲೆ-ದಾ-ಡು-ತ್ತಿ-ದ್ದೀಯ
ಅಲೆ-ದಾ-ಡು-ತ್ತಿ-ರು-ವ-ವ-ರೆಗೆ
ಅಲೆ-ದಾ-ಡುವ
ಅಲೆ-ದಾ-ಡು-ವಂ-ತಾ-ಯಿತು
ಅಲೆ-ದಾ-ಡು-ವುದನ್ನು
ಅಲೆ-ದಾ-ಡು-ವು-ದ-ರಿಂದ
ಅಲೆದು
ಅಲೆ-ಯ-ಬೇಕು
ಅಲೆ-ಯ-ಬೇಡ
ಅಲೆ-ಯು-ತ್ತಲೇ
ಅಲೆಯೇ
ಅಲೌ-ಕಿಕ
ಅಲೌ-ಕಿ-ಕತೆ
ಅಲೌ-ಕಿ-ಕ-ತೆ-ಯಿ-ರು-ತ್ತಿತ್ತು
ಅಲೌ-ಕಿ-ಕ-ವಾದ
ಅಲೌ-ಕಿ-ಕ-ವಾ-ದದ್ದು
ಅಲ್ಪ
ಅಲ್ಪ-ಬು-ದ್ಧಿಯ
ಅಲ್ಪ-ಬು-ದ್ಧಿ-ಯ-ವ-ನಲ್ಲ
ಅಲ್ಪ-ಬು-ದ್ಧಿ-ಯ-ವ-ರೇ-ನಾ-ದರೂ
ಅಲ್ಪ-ವಾ-ದರೂ
ಅಲ್ಪ-ಸ್ವಲ್ಪ
ಅಲ್ಲ
ಅಲ್ಲ-ಗ-ಳೆದು
ಅಲ್ಲ-ಗ-ಳೆ-ಯುತ್ತಿ
ಅಲ್ಲದೆ
ಅಲ್ಲಲ್ಲಿ
ಅಲ್ಲ-ಲ್ಲಿಗೆ
ಅಲ್ಲಲ್ಲೇ
ಅಲ್ಲ-ವಲ್ಲ
ಅಲ್ಲವೆ
ಅಲ್ಲವೇ
ಅಲ್ಲವೋ
ಅಲ್ಲಾ
ಅಲ್ಲಾ-ಡಿಸಿ
ಅಲ್ಲಾ-ಡಿ-ಹೋ-ದರು
ಅಲ್ಲಿ
ಅಲ್ಲಿ-ಇಲ್ಲಿ
ಅಲ್ಲಿಂದ
ಅಲ್ಲಿಂ-ದಲೂ
ಅಲ್ಲಿಂ-ದಲೇ
ಅಲ್ಲಿಂ-ದೆದ್ದು
ಅಲ್ಲಿಗೆ
ಅಲ್ಲಿ-ಗೆಲ್ಲ
ಅಲ್ಲಿ-ಗೆಲ್ಲಾ
ಅಲ್ಲಿಗೇ
ಅಲ್ಲಿತ್ತು
ಅಲ್ಲಿದ್ದ
ಅಲ್ಲಿ-ದ್ದರು
ಅಲ್ಲಿ-ದ್ದ-ವ-ನೊಬ್ಬ
ಅಲ್ಲಿ-ದ್ದ-ವ-ರ-ಲ್ಲೆಲ್ಲ
ಅಲ್ಲಿ-ದ್ದ-ವರೆಲ್ಲ
ಅಲ್ಲಿ-ದ್ದ-ವರೆ-ಲ್ಲರ
ಅಲ್ಲಿ-ದ್ದಾನೆ
ಅಲ್ಲಿ-ದ್ದುದು
ಅಲ್ಲಿ-ದ್ದು-ದೆಲ್ಲ
ಅಲ್ಲಿನ
ಅಲ್ಲಿನ್ನು
ಅಲ್ಲಿ-ಯ-ವ-ರೆಗೂ
ಅಲ್ಲಿ-ಯ-ವ-ರೆಗೆ
ಅಲ್ಲಿಯೂ
ಅಲ್ಲಿಯೇ
ಅಲ್ಲಿ-ರ-ಲಿಲ್ಲ
ಅಲ್ಲಿ-ರಲು
ಅಲ್ಲಿ-ರು-ವ-ವರೆ-ಲ್ಲರ
ಅಲ್ಲಿ-ರು-ವುದು
ಅಲ್ಲಿ-ರು-ವು-ದೆಲ್ಲ
ಅಲ್ಲಿ-ಲ್ಲ-ದಿ-ದ್ದರೆ
ಅಲ್ಲೀಗ
ಅಲ್ಲೆಲ್ಲ
ಅಲ್ಲೇ
ಅಲ್ಲೇ-ನಪ್ಪ
ಅಲ್ಲೇ-ನಿ-ರು-ತ್ತದೆ
ಅಲ್ಲೊಂದು
ಅಲ್ಲೊದು
ಅಲ್ಲೊಬ್ಬ
ಅಲ್ಲೋಲ
ಅಲ್ಲೋ-ಲ-ಕ-ಲ್ಲೋಲ
ಅಲ್ಲೋ-ಲ-ಕ-ಲ್ಲೋ-ಲವೇ
ಅಳ-ತೊ-ಡ-ಗಿದ
ಅಳ-ದಿ-ರು-ವ-ವನು
ಅಳ-ಲನ್ನು
ಅಳ-ಲನ್ನೂ
ಅಳ-ಲಾ-ರಂ-ಭಿ-ಸಿದ
ಅಳ-ಲಿ-ಗ-ಳು-ಕದೆ
ಅಳಲೂ
ಅಳ-ವ-ಡಿ-ಸಿ-ಕೊಂಡು
ಅಳ-ವ-ಡಿ-ಸಿ-ಕೊ-ಳ್ಳು-ವುದು
ಅಳಿ-ದರೂ
ಅಳಿಯ
ಅಳಿ-ಯು-ತ್ತದೆ
ಅಳಿ-ಸಿ-ಕೊಂಡು
ಅಳಿ-ಸಿಯೇ
ಅಳಿ-ಸಿ-ಹೋ-ಗಿ-ಬಿ-ಟ್ಟಿತ್ತು
ಅಳಿ-ಸಿ-ಹೋ-ಗಿ-ರ-ಲಿಲ್ಲ
ಅಳುಕಿ
ಅಳುತ್ತ
ಅಳು-ತ್ತ-ಳು-ತ್ತಲೇ
ಅಳುತ್ತಾ
ಅಳು-ತ್ತಿದ್ದೀ
ಅಳು-ವ-ವರು
ಅಳೆ-ದು-ನೋ-ಡಲು
ಅಳೆ-ಯ-ಲಾ-ರದು
ಅಳೆ-ಯಲು
ಅಳೆ-ಯು-ತ್ತಿ-ದ್ದ-ವ-ರಲ್ಲ
ಅಳೆ-ಯು-ತ್ತೀಯಾ
ಅಳೆ-ಯು-ವಂ-ತಿ-ದ್ದರೆ
ಅವ
ಅವ-ಕಾಶ
ಅವ-ಕಾ-ಶ-ವನ್ನೂ
ಅವ-ಕಾ-ಶ-ವಾಗ
ಅವ-ಕಾ-ಶ-ವಾ-ಗ-ದಂತೆ
ಅವ-ಕಾ-ಶ-ವಾ-ಗ-ಲಿಲ್ಲ
ಅವ-ಕಾ-ಶ-ವಾ-ಗು-ವು-ದಿಲ್ಲ
ಅವ-ಕಾ-ಶ-ವಾ-ದಾಗ
ಅವ-ಕಾ-ಶ-ವಾ-ಯಿತು
ಅವ-ಕಾ-ಶ-ವಿದೆ
ಅವ-ಕಾ-ಶ-ವಿ-ಲ್ಲ-ದ್ದ-ರಿಂದ
ಅವ-ಕಾ-ಶ-ವೆ-ಲ್ಲಿದೆ
ಅವ-ಕಾ-ಶ-ವೆ-ಲ್ಲಿ-ರು-ತ್ತಿತ್ತು
ಅವ-ಕಾ-ಶವೇ
ಅವಕ್ಕೆ
ಅವಗೆ
ಅವ-ಡು-ಗಚ್ಚಿ
ಅವ-ತ-ರ-ಣದ
ಅವ-ತ-ರಿಸಿ
ಅವ-ತ-ರಿ-ಸಿದ
ಅವ-ತ-ರಿ-ಸಿ-ದಂ-ತಿ-ರುವ
ಅವ-ತ-ರಿ-ಸಿ-ದ್ದಾನೆ
ಅವ-ತ-ರಿ-ಸಿ-ಬಂದ
ಅವ-ತ-ರಿ-ಸಿರು
ಅವ-ತ-ರಿ-ಸಿ-ರು-ವುದು
ಅವ-ತ-ರಿ-ಸು-ತ್ತಾನೆ
ಅವ-ತಾರ
ಅವ-ತಾ-ರ-ಗಿ-ವ-ತಾರ
ಅವ-ತಾ-ರ-ಗ-ಳಾದ
ಅವ-ತಾ-ರ-ಗಳು
ಅವ-ತಾ-ರ-ಗಳೂ
ಅವ-ತಾ-ರ-ಗ-ಳೆಲ್ಲ
ಅವ-ತಾ-ರ-ಗಳೇ
ಅವ-ತಾ-ರ-ತ-ತ್ತ್ವದ
ಅವ-ತಾ-ರ-ತ-ತ್ತ್ವ-ವನ್ನು
ಅವ-ತಾ-ರ-ತ-ತ್ತ್ವ-ವನ್ನೂ
ಅವ-ತಾ-ರ-ತ್ವ-ವನ್ನು
ಅವ-ತಾ-ರದ
ಅವ-ತಾ-ರ-ನಾದ
ಅವ-ತಾ-ರ-ಪು-ರುಷ
ಅವ-ತಾ-ರ-ಪು-ರು-ಷನ
ಅವ-ತಾ-ರ-ಪು-ರು-ಷ-ನಲ್ಲ
ಅವ-ತಾ-ರ-ಪು-ರು-ಷನು
ಅವ-ತಾ-ರ-ಪು-ರು-ಷರ
ಅವ-ತಾ-ರ-ಪು-ರು-ಷ-ರಾ-ಗಿ-ರಲು
ಅವ-ತಾ-ರ-ಪು-ರು-ಷ-ರಿಗೂ
ಅವ-ತಾ-ರ-ಪು-ರು-ಷರು
ಅವ-ತಾ-ರ-ಪು-ರು-ಷರೂ
ಅವ-ತಾ-ರ-ಪು-ರು-ಷ-ರೆಂದು
ಅವ-ತಾ-ರ-ಪು-ರು-ಷ-ರೆಂಬ
ಅವ-ತಾ-ರ-ಪು-ರು-ಷರೋ
ಅವ-ತಾ-ರರೇ
ಅವ-ತಾ-ರ-ವನ್ನು
ಅವ-ತಾ-ರ-ವ-ರಿ-ಷ್ಠ-ನೆಂದು
ಅವ-ತಾ-ರ-ವೆಂದು
ಅವ-ತಾ-ರ-ವೆಂದೂ
ಅವ-ತಾ-ರ-ವೆಂ-ಬುದು
ಅವ-ತಾ-ರ-ವೆತ್ತಿ
ಅವ-ತಾ-ರ-ವೆ-ತ್ತಿ-ದ-ವರು
ಅವ-ತಾ-ರವೇ
ಅವ-ತಾ-ರ-ಸ-ಮಾ-ಪ್ತಿಯ
ಅವ-ತಾರಾ
ಅವ-ತಾ-ರೋ-ದ್ದೇ-ಶದ
ಅವ-ತಾ-ರೋ-ದ್ದೇ-ಶ-ವನ್ನು
ಅವ-ತಾ-ರೋ-ದ್ದೇ-ಶ-ವೆಂ-ಬುದು
ಅವಧಿ
ಅವ-ಧಿಯ
ಅವ-ಧಿ-ಯಲ್ಲಿ
ಅವ-ಧಿ-ಯಲ್ಲೇ
ಅವನ
ಅವ-ನಂ-ತಹ
ಅವ-ನಂತೂ
ಅವ-ನಂ-ತೆಯೇ
ಅವ-ನತಿ
ಅವ-ನ-ತಿ-ಗಿಳಿ-ಯುವ
ಅವ-ನ-ತಿಗೆ
ಅವ-ನ-ತಿ-ಯಾಗಿ
ಅವ-ನತ್ತ
ಅವ-ನ-ದನ್ನು
ಅವ-ನ-ದಲ್ಲ
ಅವ-ನ-ದಾ-ಯಿತು
ಅವ-ನದು
ಅವ-ನ-ದು-ಬ್ರಾ-ಹ್ಮ-ಸ-ಮಾ-ಜದ
ಅವ-ನದೇ
ಅವ-ನ-ನಿ-ಗೀಗ
ಅವ-ನ-ನ್ನೀಗ
ಅವ-ನನ್ನು
ಅವ-ನ-ನ್ನು-ರಾ-ಮ-ಲಾಲ
ಅವ-ನ-ನ್ನೆಂ-ದಿಗೂ
ಅವ-ನನ್ನೇ
ಅವ-ನ-ನ್ನೊಂದು
ಅವ-ನ-ಲ್ಲದೆ
ಅವ-ನಲ್ಲಿ
ಅವ-ನ-ಲ್ಲಿಗೆ
ಅವ-ನ-ಲ್ಲಿತ್ತು
ಅವ-ನ-ಲ್ಲಿ-ದ್ದಂ-ತಹ
ಅವ-ನ-ಲ್ಲಿದ್ದೇ
ಅವ-ನ-ಲ್ಲುಂ-ಟಾ-ಗಿತ್ತು
ಅವ-ನ-ಲ್ಲುಂ-ಟಾ-ಗಿ-ಬಿ-ಟ್ಟಿತ್ತು
ಅವ-ನ-ಲ್ಲುಂ-ಟಾದ
ಅವ-ನ-ಲ್ಲು-ದಿ-ಸಿತು
ಅವ-ನಲ್ಲೂ
ಅವ-ನ-ಲ್ಲೇನೂ
ಅವ-ನ-ಲ್ಲೊಂದು
ಅವ-ನ-ಲ್ಲೊಬ್ಬ
ಅವ-ನ-ವನ
ಅವ-ನ-ಷ್ಟಕ್ಕೆ
ಅವ-ನಷ್ಟೇ
ಅವ-ನಾ-ಗಲೇ
ಅವ-ನಾ-ಡಿದ
ಅವನಿ
ಅವ-ನಿಂದ
ಅವ-ನಿಂ-ದಲೇ
ಅವ-ನಿಂ-ದಾಗಿ
ಅವ-ನಿ-ಗದು
ಅವ-ನಿ-ಗ-ನ್ನಿ-ಸ-ತೊ-ಡ-ಗಿತು
ಅವ-ನಿ-ಗ-ನ್ನಿ-ಸ-ತೊ-ಡ-ಗಿ-ತುಈ
ಅವ-ನಿ-ಗ-ನ್ನಿ-ಸಿತು
ಅವ-ನಿ-ಗಾಗ
ಅವ-ನಿ-ಗಾಗಿ
ಅವ-ನಿ-ಗಾದ
ಅವ-ನಿ-ಗಿಂತ
ಅವ-ನಿ-ಗಿತ್ತು
ಅವ-ನಿ-ಗಿದ್ದ
ಅವ-ನಿ-ಗಿನ್ನೂ
ಅವ-ನಿ-ಗಿ-ರ-ಲಿಲ್ಲ
ಅವ-ನಿ-ಗಿ-ರು-ತ್ತದೆ
ಅವ-ನಿ-ಗೀಗ
ಅವ-ನಿ-ಗುಂ-ಟಾ-ಗು-ತ್ತಿ-ದ್ದುದು
ಅವ-ನಿಗೂ
ಅವ-ನಿಗೆ
ಅವ-ನಿ-ಗೆಂ-ದಿತು
ಅವ-ನಿ-ಗೆಂದೂ
ಅವ-ನಿ-ಗೆಷ್ಟು
ಅವ-ನಿಗೇ
ಅವ-ನಿ-ಗೇನು
ಅವ-ನಿ-ಗೇನೋ
ಅವ-ನಿ-ಗೊಂದು
ಅವ-ನಿಗೋ
ಅವ-ನಿ-ಗೋ-ಸ್ಕರ
ಅವ-ನಿಚ್ಛೆ
ಅವ-ನಿ-ಟ್ಟಿದ್ದ
ಅವ-ನಿ-ದ್ದರೇ
ಅವ-ನಿನ್ನು
ಅವ-ನಿನ್ನೂ
ಅವ-ನಿ-ರು-ವುದು
ಅವ-ನಿ-ರು-ವುದೇ
ಅವ-ನಿ-ಲ್ಲಿಗೆ
ಅವ-ನೀಗ
ಅವನು
ಅವನೂ
ಅವನೆ
ಅವ-ನೆಂ-ತಹ
ಅವ-ನೆಂದ
ಅವ-ನೆಂ-ದರೆ
ಅವ-ನೆಂದೂ
ಅವ-ನೆ-ಡೆಗೆ
ಅವ-ನೆ-ದೆ-ಯೊ-ಳ-ಗಿಂದ
ಅವ-ನೆನ್ನು
ಅವ-ನೆ-ಷ್ಟಾ-ದರೂ
ಅವ-ನೆಷ್ಟೇ
ಅವ-ನೆಷ್ಟೋ
ಅವನೇ
ಅವ-ನೇಕೆ
ಅವ-ನೇ-ನಾ-ದರೂ
ಅವ-ನೇನು
ಅವ-ನೇನೂ
ಅವ-ನೇನೋ
ಅವ-ನೊಂ-ದಿಗೆ
ಅವ-ನೊಂದು
ಅವ-ನೊ-ಡನೆ
ಅವ-ನೊ-ಡ-ಲೊ-ಳ-ಗಿನ
ಅವ-ನೊಬ್ಬ
ಅವ-ನೊ-ಬ್ಬನ
ಅವ-ನೊ-ಬ್ಬನೇ
ಅವ-ನೊಮ್ಮೆ
ಅವ-ನೊ-ಳಕ್ಕೆ
ಅವ-ನೊ-ಳ-ಗಿ-ತ್ತು-ಉ-ಕ್ಕಿ-ನಂಥ
ಅವ-ನೊ-ಳ-ಗಿನ
ಅವ-ನೊ-ಳಗೆ
ಅವನ್ನು
ಅವನ್ನೂ
ಅವ-ನ್ನೆಲ್ಲ
ಅವ-ಮ-ರ್ಯಾ-ದೆ-ಯನ್ನು
ಅವ-ಮಾನ
ಅವ-ಮಾ-ನ-ಕ-ರ-ವಾ-ಗಿದೆ
ಅವ-ಯ-ವ-ಗಳ
ಅವರ
ಅವ-ರಂತೂ
ಅವ-ರಂತೆ
ಅವ-ರಂ-ತೆಯೇ
ಅವ-ರಂಥ
ಅವ-ರ-ಣ-ವಿ-ಲ್ಲದೆ
ಅವ-ರತ್ತ
ಅವ-ರದು
ಅವ-ರ-ದೊಂದೇ
ಅವ-ರ-ದ್ದೆಲ್ಲ
ಅವ-ರದ್ದೇ
ಅವ-ರ-ನ್ನಿನ್ನೂ
ಅವ-ರನ್ನು
ಅವ-ರ-ನ್ನು-ಸ್ವಾ-ಸ್ಥ್ಯಕ್ಕೆ
ಅವ-ರನ್ನೂ
ಅವ-ರ-ನ್ನೆಲ್ಲ
ಅವ-ರನ್ನೇ
ಅವ-ರಲ್ಲಿ
ಅವ-ರ-ಲ್ಲಿದ್ದ
ಅವ-ರ-ಲ್ಲಿ-ದ್ದುದು
ಅವ-ರ-ಲ್ಲಿ-ರು-ತ್ತದೆ
ಅವ-ರ-ಲ್ಲುಂ-ಟಾ-ಗಿತ್ತು
ಅವ-ರ-ಲ್ಲುಂ-ಟಾ-ಗಿ-ಬಿ-ಟ್ಟಿತ್ತು
ಅವ-ರಲ್ಲೂ
ಅವ-ರ-ಲ್ಲೆದ್ದು
ಅವ-ರ-ಲ್ಲೆಲ್ಲ
ಅವ-ರಲ್ಲೇ
ಅವ-ರ-ಲ್ಲೇನೋ
ಅವ-ರ-ಲ್ಲೊಂದು
ಅವ-ರ-ಲ್ಲೊಬ್ಬ
ಅವ-ರ-ಲ್ಲೊ-ಬ್ಬನ
ಅವ-ರ-ಲ್ಲೊ-ಬ್ಬರು
ಅವ-ರ-ವರ
ಅವ-ರ-ವರು
ಅವ-ರ-ಷ್ಟಕ್ಕೆ
ಅವ-ರಷ್ಟೂ
ಅವ-ರಾ-ಡಿದ
ಅವ-ರಾ-ಡುವ
ಅವ-ರಾ-ದರೂ
ಅವ-ರಾರೂ
ಅವರಿ
ಅವ-ರಿಂದ
ಅವ-ರಿಂ-ದಲೂ
ಅವ-ರಿಂ-ದಲೇ
ಅವ-ರಿ-ಗ-ದರ
ಅವ-ರಿ-ಗ-ರಿ-ವಿ-ಲ್ಲ-ದಂ-ತೆಯೇ
ಅವ-ರಿ-ಗಾಗಿ
ಅವ-ರಿ-ಗಾ-ಗಿಯೇ
ಅವ-ರಿ-ಗಾಗು
ಅವ-ರಿ-ಗಾ-ಗು-ತ್ತಿದ್ದ
ಅವ-ರಿ-ಗಾದ
ಅವ-ರಿ-ಗಿ-ತ್ತೇನು
ಅವ-ರಿ-ಗಿದ್ದ
ಅವ-ರಿ-ಗಿನ್ನೂ
ಅವ-ರಿ-ಗಿ-ರುವ
ಅವ-ರಿ-ಗಿ-ರು-ವು-ದಿಲ್ಲ
ಅವ-ರಿ-ಗಿ-ರು-ವುದು
ಅವ-ರಿ-ಗೀಗ
ಅವ-ರಿ-ಗುಂ-ಟಾ-ಗು-ತ್ತಿತ್ತು
ಅವ-ರಿಗೂ
ಅವ-ರಿಗೆ
ಅವ-ರಿ-ಗೆಲ್ಲ
ಅವ-ರಿ-ಗೆಲ್ಲಿ
ಅವ-ರಿಗೇ
ಅವ-ರಿ-ಗೇನು
ಅವ-ರಿ-ಗೇನೋ
ಅವ-ರಿ-ಗೊಂದು
ಅವ-ರಿ-ದನ್ನು
ಅವ-ರಿ-ದ್ದ-ಲ್ಲಿಗೆ
ಅವ-ರಿ-ದ್ದುದು
ಅವ-ರಿನ್ನು
ಅವ-ರಿನ್ನೂ
ಅವ-ರಿ-ಬ್ಬರ
ಅವ-ರಿ-ಬ್ಬ-ರದು
ಅವ-ರಿ-ಬ್ಬ-ರಲ್ಲಿ
ಅವ-ರಿ-ಬ್ಬ-ರಲ್ಲೂ
ಅವ-ರಿ-ಬ್ಬರೂ
ಅವ-ರಿ-ರ್ವರ
ಅವ-ರಿ-ವ-ರಿ-ಗಿ-ರಲಿ
ಅವ-ರಿ-ಸಿ-ಕೊಂ-ಡಿತು
ಅವ-ರೀಗ
ಅವರು
ಅವ-ರು-ಇ-ರಲಿ
ಅವ-ರು-ಎಂಥಾ
ಅವ-ರು-ಗಳು
ಅವರೂ
ಅವ-ರೆಂ-ದರು
ಅವ-ರೆಂ-ದರೆ
ಅವ-ರೆಂದೂ
ಅವರೆ-ಡೆಗೆ
ಅವರೆ-ದು-ರಿಗೆ
ಅವರೆ-ದುರು
ಅವರೆ-ನ್ನು-ತ್ತಾರೆ
ಅವರೆ-ನ್ನು-ತ್ತಿ-ದ್ದರು
ಅವರೆಲ್ಲ
ಅವರೆ-ಲ್ಲರ
ಅವರೆ-ಲ್ಲ-ರನ್ನೂ
ಅವರೆ-ಲ್ಲ-ರಿ-ಗಿಂ-ತಲೂ
ಅವರೆ-ಲ್ಲರೂ
ಅವರೆ-ಲ್ಲ-ರೊ-ಳಗೂ
ಅವ-ರೆಷ್ಟು
ಅವ-ರೆಷ್ಟೇ
ಅವರೇ
ಅವ-ರೇನೂ
ಅವ-ರೇನೋ
ಅವರೊ
ಅವ-ರೊಂ-ದಿ-ಗಿದ್ದು
ಅವ-ರೊಂ-ದಿ-ಗಿನ
ಅವ-ರೊಂ-ದಿಗೆ
ಅವ-ರೊಂ-ದಿ-ಗೆಲ್ಲ
ಅವ-ರೊಂ-ದಿಗೇ
ಅವ-ರೊ-ಡನೆ
ಅವ-ರೊಬ್ಬ
ಅವ-ರೊ-ಬ್ಬ-ರಿಗೇ
ಅವ-ರೊಮ್ಮೆ
ಅವ-ರೊ-ಳಗೆ
ಅವರೋ
ಅವ-ರ್ಣ-ನೀಯ
ಅವ-ಲಂ-ಬಿಸಿ
ಅವ-ಲಂ-ಬಿ-ಸಿ-ಕೊಂ-ಡಿ-ರು-ತ್ತದೆ
ಅವ-ಲಂ-ಬಿ-ಸಿ-ದ-ವ-ರಲ್ಲಿ
ಅವ-ಲಂ-ಬಿ-ಸಿದೆ
ಅವಳ
ಅವ-ಳದು
ಅವ-ಳನ್ನು
ಅವ-ಳ-ನ್ನೆಂದೂ
ಅವ-ಳಲ್ಲಿ
ಅವ-ಳ-ಲ್ಲಿತ್ತು
ಅವ-ಳಿ-ಗಾ-ಗಲೇ
ಅವ-ಳಿ-ಗಿನ್ನೂ
ಅವ-ಳಿಗೂ
ಅವ-ಳಿಗೆ
ಅವ-ಳಿ-ಗೇನೂ
ಅವಳು
ಅವ-ಳೆದೆ
ಅವ-ಳೆ-ನ್ನು-ತ್ತಿ-ದ್ದಳು
ಅವ-ಳೆಲ್ಲಿ
ಅವಳೇ
ಅವ-ಳೊಂ-ದಿಗೆ
ಅವ-ಳೊಂದು
ಅವ-ಳೊ-ಬ್ಬ-ಳಿಗೇ
ಅವ-ಶೇಷ
ಅವ-ಶೇ-ಷ-ಕ್ಕಾಗಿ
ಅವ-ಶೇ-ಷಕ್ಕೆ
ಅವ-ಶೇ-ಷ-ಗಳನ್ನು
ಅವ-ಶೇ-ಷ-ಗ-ಳಿಗೆ
ಅವ-ಶೇ-ಷ-ಗಳು
ಅವ-ಶೇ-ಷ-ಗ-ಳು-ಇವು
ಅವ-ಶೇ-ಷದ
ಅವ-ಶೇ-ಷ-ವನ್ನು
ಅವ-ಶ್ಯ-ಕ-ತೆ-ಯನ್ನು
ಅವ-ಸ-ರದ
ಅವಸ್ಥೆ
ಅವ-ಸ್ಥೆ-ಯಂ-ತೆಯೇ
ಅವ-ಸ್ಥೆ-ಯನ್ನು
ಅವ-ಸ್ಥೆ-ಯಲ್ಲಿ
ಅವ-ಹೇ-ಳ-ನ-ಕರ
ಅವಾ-ಕ್ಕಾಗಿ
ಅವಾ-ಕ್ಕಾ-ಗಿ-ಬಿಟ್ಟ
ಅವಾ-ಙ್ಮಾ-ನ-ಸ-ಗೋ-ಚರ
ಅವಿ-ಚಲ
ಅವಿ-ಚ್ಛಿನ್ನ
ಅವಿ-ಧೇ-ಯ-ತೆ-ಯನ್ನು
ಅವಿ-ಧೇ-ಯ-ರಾಗಿ
ಅವಿನಾ
ಅವಿ-ಭಕ್ತ
ಅವಿ-ಭಾಜ್ಯ
ಅವಿ-ರತ
ಅವಿ-ರ-ತ-ವಾಗಿ
ಅವಿ-ವಾ-ಹಿ-ತ-ನಾಗಿ
ಅವಿ-ವಾ-ಹಿ-ತ-ನಾದ
ಅವಿ-ವೇ-ಕಕ್ಕೆ
ಅವಿ-ವೇ-ಕದ
ಅವಿ-ವೇ-ಕ-ದಿಂದ
ಅವಿ-ವೇ-ಕಿ-ಯಾ-ದ-ವನು
ಅವಿ-ಶ್ವಾಸ
ಅವು
ಅವು-ಅ-ಣಿಮಾ
ಅವು-ಗಳ
ಅವು-ಗಳನ್ನು
ಅವು-ಗಳನ್ನೆಲ್ಲ
ಅವು-ಗಳಲ್ಲಿ
ಅವು-ಗ-ಳ-ಲ್ಲಿನ
ಅವು-ಗ-ಳಲ್ಲೂ
ಅವು-ಗ-ಳ-ಲ್ಲೆಲ್ಲ
ಅವು-ಗ-ಳ-ಲ್ಲೊಂ-ದನ್ನು
ಅವು-ಗಳಿಂದ
ಅವು-ಗ-ಳಿ-ಗ-ನು-ಸಾ-ರ-ವಾಗಿ
ಅವು-ಗ-ಳಿಗೆ
ಅವು-ಗ-ಳಿ-ಗೆಲ್ಲ
ಅವು-ಗ-ಳೆಲ್ಲ
ಅವು-ಗ-ಳೆ-ಲ್ಲವೂ
ಅವೆಲ್ಲ
ಅವೆ-ಲ್ಲ-ವನ್ನೂ
ಅವೆ-ಲ್ಲವು
ಅವೆ-ಲ್ಲವೂ
ಅವೆ-ಲ್ಲ-ವೂ-ಇ-ದ-ರೊ-ಳ-ಗಿ-ನಿಂ-ದಲೇ
ಅವ್ಯಕ್ತ
ಅವ್ಯ-ಕ್ತ-ದಲ್ಲಿ
ಅಶ-ಕ್ತ-ನಾ-ದರೆ
ಅಶ-ನಾ-ದಿ-ಗ-ಳಿಗೂ
ಅಶ-ನಾ-ರ್ಥಕ್ಕೆ
ಅಶಾಂ-ತ-ನಾ-ಗ-ಲಿಲ್ಲ
ಅಶಾಂ-ತಿಯ
ಅಶಿ-ಕ್ಷಿ-ತ-ಅ-ನೀ-ತಿ-ವಂ-ತ-ನಾ-ಗಿ-ದ್ದರೆ
ಅಶುದ್ಧ
ಅಶು-ದ್ಧ-ವಾದ
ಅಶ್ರು-ಧಾ-ರೆ-ಯನ್ನು
ಅಶ್ರು-ನ-ಯ-ನ-ರಾಗಿ
ಅಶ್ರು-ಭ-ರಿತ
ಅಶ್ರು-ಭ-ರಿ-ತ-ನಾಗಿ
ಅಶ್ಲೀಲ
ಅಶ್ಲೀ-ಲ-ವೆಂದು
ಅಶ್ವತ್ಥ
ಅಶ್ವ-ತ್ಥದ
ಅಶ್ವ-ತ್ಥ-ವೃ-ಕ್ಷದ
ಅಷ್ಟ-ಕ್ಕಷ್ಟೇ
ಅಷ್ಟಕ್ಕೆ
ಅಷ್ಟಕ್ಕೇ
ಅಷ್ಟನ್ನು
ಅಷ್ಟನ್ನೂ
ಅಷ್ಟ-ರ-ಮ-ಟ್ಟಿಗೆ
ಅಷ್ಟ-ರಲ್ಲಿ
ಅಷ್ಟ-ರಲ್ಲೇ
ಅಷ್ಟ-ರಿಂ-ದಲೇ
ಅಷ್ಟ-ರೊ-ಳಗೇ
ಅಷ್ಟ-ಲ್ಲದೆ
ಅಷ್ಟಷ್ಟೂ
ಅಷ್ಟ-ಸಿ-ದ್ಧಿ-ಗಳು
ಅಷ್ಟಾಗಿ
ಅಷ್ಟಾ-ವ-ಕ್ರ-ಸಂ-ಹಿ-ತೆಯೇ
ಅಷ್ಟಿ-ಷ್ಟಲ್ಲ
ಅಷ್ಟು
ಅಷ್ಟು-ದ್ದದ
ಅಷ್ಟು-ರ-ಸ-ವ-ತ್ತಾಗಿ
ಅಷ್ಟೂ
ಅಷ್ಟೆ
ಅಷ್ಟೆ-ಅ-ದ-ರಿಂದ
ಅಷ್ಟೆ-ತ್ತ-ರದ
ಅಷ್ಟೆಲ್ಲ
ಅಷ್ಟೇ
ಅಷ್ಟೇಕೆ
ಅಷ್ಟೇನೂ
ಅಷ್ಟೊಂ-ದಾಗಿ
ಅಷ್ಟೊಂದು
ಅಷ್ಟೊಂ-ದೇಕೆ
ಅಷ್ಟೊ-ತ್ತಿ-ಗಾ-ಗಲೇ
ಅಷ್ಟೊ-ತ್ತಿಗೆ
ಅಸಂ-ಖ್ಯಾತ
ಅಸಂ-ಖ್ಯೇ-ಯಾಃ
ಅಸಂ-ಬ-ದ್ಧ-ತೆ-ಗಳನ್ನು
ಅಸಂ-ಬ-ದ್ಧ-ತೆ-ಗಳನ್ನೂ
ಅಸಂ-ಬ-ದ್ಧದ
ಅಸಂ-ಭ-ವ-ನೀಯ
ಅಸಡ್ಡೆ
ಅಸ-ಡ್ಡೆಯ
ಅಸ-ತ್ಯ-ದಿಂದ
ಅಸ-ಮಗ್ರ
ಅಸ-ಮ-ರ್ಥ-ನ-ನ್ನಾ-ಗಿ-ಸಿದ
ಅಸ-ಮ-ರ್ಥ-ರಾ-ಗಿ-ದ್ದಾರೆ
ಅಸ-ಮ-ರ್ಥ-ರಾ-ಗಿ-ದ್ದಾರೋ
ಅಸ-ಮ-ರ್ಥ-ವಾ-ದರೆ
ಅಸ-ಮಾ-ಧಾ-ನಕ್ಕೆ
ಅಸ-ಮಾ-ಧಾ-ನ-ಗೊಂಡ
ಅಸ-ಮಾ-ಧಾ-ನ-ಗೊಂಡು
ಅಸ-ಮಾ-ಧಾ-ನ-ದಿಂದ
ಅಸ-ಮಾ-ಧಾ-ನ-ವಾ-ಯಿತು
ಅಸ-ಹಜ
ಅಸ-ಹ-ಜ-ಅ-ನಿಷ್ಟ
ಅಸ-ಹ-ಜ-ವಾ-ದ-ದ್ದೇನೂ
ಅಸ-ಹ-ನೀಯ
ಅಸ-ಹ-ನೀ-ಯ-ವಾ-ಗ-ತೊ-ಡ-ಗಿತ್ತು
ಅಸ-ಹ-ನೀ-ಯ-ವಾಗಿ
ಅಸ-ಹ-ನೀ-ಯ-ವಾ-ಗು-ತ್ತದೆ
ಅಸ-ಹ-ನೀ-ಯ-ವಾ-ದವು
ಅಸ-ಹ-ನೀ-ಯ-ವಾ-ಯಿತು
ಅಸ-ಹನೆ
ಅಸ-ಹ-ನೆ-ಗೊಂಡು
ಅಸ-ಹಾ-ಯಕ
ಅಸ-ಹಾ-ಯ-ಕ-ನಾದ
ಅಸ-ಹಾ-ಯ-ಕ-ರಾಗಿ
ಅಸ-ಹಾ-ಯ-ಕ-ರಾ-ಗಿದ್ದ
ಅಸ-ಹಾ-ಯ-ಕ-ರಾ-ದರು
ಅಸ-ಹಾ-ಯ-ಕ-ರಿಗೆ
ಅಸಾ-ಧ-ರಾಣ
ಅಸಾ-ಧಾ-ರಣ
ಅಸಾಧ್ಯ
ಅಸಾ-ಧ್ಯ-ವಾ-ಗಿತ್ತು
ಅಸಾ-ಮ-ರ್ಥ್ಯ-ವನ್ನು
ಅಸಾ-ಮ-ರ್ಥ್ಯ-ವನ್ನೇ
ಅಸಾ-ಮಾನ್ಯ
ಅಸಿ-ಸ್ಟೆಂಟ್
ಅಸ್ತಿತ್ವ
ಅಸ್ತಿ-ತ್ವ-ನಾ-ಸ್ತಿ-ತ್ವದ
ಅಸ್ತಿ-ತ್ವಕ್ಕೆ
ಅಸ್ತಿ-ತ್ವಕ್ಕೇ
ಅಸ್ತಿ-ತ್ವದ
ಅಸ್ತಿ-ತ್ವ-ದಲ್ಲಿ
ಅಸ್ತಿ-ತ್ವ-ದ-ಲ್ಲಿಯೇ
ಅಸ್ತಿ-ತ್ವ-ವನ್ನು
ಅಸ್ತಿ-ತ್ವ-ವನ್ನೂ
ಅಸ್ತಿ-ತ್ವ-ವನ್ನೇ
ಅಸ್ತಿ-ತ್ವ-ವೆಂ-ಬುದೇ
ಅಸ್ತು
ಅಸ್ಥಿ-ಭ-ಸ್ಮ-ಗಳನ್ನು
ಅಸ್ಥಿ-ಭ-ಸ್ಮದ
ಅಸ್ಥಿಗೆ
ಅಸ್ಥಿ-ಯನ್ನು
ಅಸ್ಪ-ಷ್ಟ-ವಾಗಿ
ಅಸ್ವ-ಸ್ಥ-ತೆಯ
ಅಹಂ
ಅಹಂ-ಕಾರ
ಅಹಂ-ಕಾ-ರದ
ಅಹಂ-ಕಾ-ರ-ದಿಂದ
ಅಹಂ-ಕಾ-ರ-ವನ್ನು
ಅಹಂ-ಕಾ-ರ-ವ-ಶ-ನಾಗಿ
ಅಹಂ-ಕಾ-ರಿ-ಯಾ-ಗಿ-ಬಿ-ಟ್ಟಿ-ದ್ದ-ನೆಂದು
ಅಹಂ-ಬುದ್ಧಿ
ಅಹ-ಮದ್
ಅಹಿಂಸಾ
ಅಹು-ದ-ಹು-ದೆಂದು
ಅಹೈ-ತುಕ
ಆ
ಆಂಗ್ಲ
ಆಂಗ್ಲಈ
ಆಂಗ್ಲರೂ
ಆಂಗ್ಲ-ವ್ಯ-ಕ್ತಿಗೆ
ಆಂಟ್
ಆಂಟ್ಪು-ರಕ್ಕೆ
ಆಂಟ್ಪು-ರ-ದಲ್ಲಿ
ಆಂತ-ರಿಕ
ಆಂತ-ರ್ಯದ
ಆಂತ-ರ್ಯ-ದಲ್ಲಿ
ಆಂದೋ-ಲನ
ಆಂದೋ-ಲ-ನ-ಗಳು
ಆಕ-ರ್ಷ-ಕ-ವಾಗಿ
ಆಕ-ರ್ಷಣಾ
ಆಕ-ರ್ಷಣೆ
ಆಕ-ರ್ಷ-ಣೆಗೆ
ಆಕ-ರ್ಷ-ಣೆ-ಯನ್ನು
ಆಕ-ರ್ಷ-ಣೆ-ಯಿಂದ
ಆಕ-ರ್ಷ-ಣೆ-ಯಿತ್ತು
ಆಕ-ರ್ಷ-ಣೆ-ಯೆಂ-ದರೆ
ಆಕ-ರ್ಷಿತ
ಆಕ-ರ್ಷಿ-ತ-ನಾ-ಗಿ-ದ್ದಾನೆ
ಆಕ-ರ್ಷಿ-ತ-ನಾ-ಗಿ-ಬಿಟ್ಟ
ಆಕ-ರ್ಷಿ-ತ-ನಾದ
ಆಕ-ರ್ಷಿ-ತ-ನಾದೆ
ಆಕ-ರ್ಷಿ-ತ-ರಾಗಿ
ಆಕ-ರ್ಷಿ-ತ-ರಾ-ಗಿ-ದ್ದರು
ಆಕ-ರ್ಷಿ-ತ-ರಾ-ಗಿ-ದ್ದರೋ
ಆಕ-ರ್ಷಿ-ತ-ರಾ-ಗಿ-ಬಿ-ಟ್ಟಿ-ದ್ದರು
ಆಕ-ರ್ಷಿ-ತ-ರಾ-ದರು
ಆಕ-ರ್ಷಿ-ತ-ವಾ-ಗಿ-ಬಿ-ಟ್ಟಿತ್ತು
ಆಕ-ರ್ಷಿ-ತ-ವಾ-ಗು-ತ್ತಿತ್ತು
ಆಕ-ರ್ಷಿ-ತ-ವಾ-ಗು-ವುದು
ಆಕ-ರ್ಷಿಸಿ
ಆಕ-ರ್ಷಿ-ಸಿ-ಕೊಂಡು
ಆಕ-ರ್ಷಿ-ಸಿದ
ಆಕ-ರ್ಷಿ-ಸಿ-ಬಿ-ಟ್ಟಿದ್ದ
ಆಕ-ಸ್ಮಿಕ
ಆಕ-ಸ್ಮಿ-ಕ-ವಾಗಿ
ಆಕ-ಸ್ಮಿ-ಕ-ವೊಂದು
ಆಕಾಂಕ್ಷೆ
ಆಕಾಂ-ಕ್ಷೆ-ಗಳು
ಆಕಾರ
ಆಕಾ-ರ-ಗಾ-ತ್ರ-ದಲ್ಲಿ
ಆಕಾ-ರ-ರ-ಹಿತ
ಆಕಾ-ರ-ವ-ನ್ನಾ-ದರೂ
ಆಕಾ-ರವೇ
ಆಕಾಶ
ಆಕಾ-ಶ-ಗೋ-ಪುರ
ಆಕಾ-ಶ-ದಂತೆ
ಆಕಾ-ಶ-ದ-ಲ್ಲಿ-ರುವ
ಆಕಾ-ಶ-ಮಾ-ರ್ಗ-ವಾಗಿ
ಆಕೃತಿ
ಆಕೃ-ತಿ-ಗಳನ್ನು
ಆಕೃ-ತಿ-ಗಳು
ಆಕೆ
ಆಕೆಯ
ಆಕೆ-ಯನ್ನು
ಆಕ್ರ-ಮ-ಣ-ದಿಂ-ದಾಗಿ
ಆಕ್ರ-ಮ-ಣ-ವನ್ನು
ಆಕ್ರ-ಮ-ಣ-ಶೀಲ
ಆಕ್ರ-ಮಿ-ಸಿ-ಬಿ-ಟ್ಟಿದೆ
ಆಕ್ರೋ-ಶದ
ಆಕ್ರೋ-ಶವೂ
ಆಕ್ಷೇ-ಪ-ಣೀಯ
ಆಕ್ಷೇ-ಪ-ಣೆ-ಯೆ-ತ್ತ-ಬ-ಹು-ದು-ಸಂ-ನ್ಯಾ-ಸಿಯ
ಆಕ್ಷೇ-ಪಿ-ಸಿ-ದ್ದರು
ಆಕ್ಷೇ-ಪಿ-ಸು-ತ್ತಿದ್ದ
ಆಖಾ-ಡ-ವನ್ನು
ಆಗ
ಆಗಂ-ತು-ಕ-ನಿಗೆ
ಆಗಂತೂ
ಆಗ-ತಾನೇ
ಆಗ-ದಿ-ರು-ವು-ದೇ-ನಿದೆ
ಆಗ-ದೆ-ಯಿ-ದ್ದರೂ
ಆಗ-ಬ-ಹು-ದಾ-ಗಿತ್ತು
ಆಗ-ಬ-ಹುದು
ಆಗ-ಬೇ-ಕಾ-ಗಿತ್ತು
ಆಗ-ಬೇ-ಕಾ-ಗಿದೆ
ಆಗ-ಬೇ-ಕಾ-ದ-ದ್ದಿಲ್ಲ
ಆಗ-ಬೇಕು
ಆಗ-ಮನ
ಆಗ-ಮ-ನದ
ಆಗ-ಮ-ನ-ದಿಂ-ದಾಗಿ
ಆಗ-ಮ-ನ-ದೊಂ-ದಿಗೆ
ಆಗ-ಮ-ನ-ವನ್ನೇ
ಆಗ-ಮಿಸಿ
ಆಗ-ಮಿ-ಸಿದ
ಆಗ-ಮಿ-ಸಿ-ದ್ದಾನೆ
ಆಗಲಿ
ಆಗ-ಲಿ-ಇ-ವೆಲ್ಲ
ಆಗ-ಲಿ-ಕ್ಕಿದೆ
ಆಗ-ಲಿಲ್ಲ
ಆಗ-ಲಿ-ಲ್ಲ-ವಲ್ಲ
ಆಗ-ಲಿ-ಲ್ಲ-ವೆಂದರೆ
ಆಗ-ಲಿ-ಲ್ಲ-ವೆಂ-ಬಂತೆ
ಆಗಲು
ಆಗಲೂ
ಆಗಲೇ
ಆಗಲೋ
ಆಗಷ್ಟೇ
ಆಗಸ
ಆಗ-ಸ-ದಲ್ಲಿ
ಆಗ-ಸ-ದೆ-ತ್ತ-ರಕ್ಕೆ
ಆಗ-ಸವೇ
ಆಗ-ಸ್ಟ್
ಆಗಾಗ
ಆಗಾ-ಗಲೇ
ಆಗಿ
ಆಗಿತ್ತು
ಆಗಿದೆ
ಆಗಿ-ದೆ-ಯಮ್ಮ
ಆಗಿದ್ದ
ಆಗಿ-ದ್ದರೂ
ಆಗಿ-ದ್ದರೆ
ಆಗಿ-ದ್ದಾನೆ
ಆಗಿ-ದ್ದಿತು
ಆಗಿ-ದ್ದೀಯೆ
ಆಗಿದ್ದು
ಆಗಿ-ದ್ದೇನೆ
ಆಗಿನ
ಆಗಿ-ನಿಂದ
ಆಗಿ-ನಿಂ-ದಲೂ
ಆಗಿನ್ನೂ
ಆಗಿ-ಬಿ-ಟ್ಟರೆ
ಆಗಿ-ಬಿ-ಟ್ಟ-ರೇನು
ಆಗಿ-ಬಿಟ್ಟಿ
ಆಗಿ-ಬಿ-ಟ್ಟಿ-ತು-ಇ-ನ್ನೇನೂ
ಆಗಿ-ಬಿ-ಟ್ಟಿತ್ತು
ಆಗಿ-ಬಿ-ಟ್ಟಿದೆ
ಆಗಿ-ಬಿ-ಟ್ಟಿ-ದೆ-ಯಲ್ಲ
ಆಗಿ-ಬಿ-ಟ್ಟಿದ್ದ
ಆಗಿ-ಬಿ-ಟ್ಟಿ-ರು-ವುದನ್ನು
ಆಗಿ-ಬಿಟ್ಟೆ
ಆಗಿ-ಬಿ-ಡ-ಬೇಕು
ಆಗಿ-ಬಿ-ಡು-ತ್ತದೆ
ಆಗಿ-ಬಿ-ಡು-ತ್ತಾನೆ
ಆಗಿ-ಬಿ-ಡು-ತ್ತಿದ್ದ
ಆಗಿಯೂ
ಆಗಿರ
ಆಗಿ-ರ-ಬ-ಹುದು
ಆಗಿ-ರ-ಬ-ಹುದೆ
ಆಗಿ-ರ-ಬೇ-ಕಾ-ಗಿತ್ತು
ಆಗಿ-ರ-ಬೇಕು
ಆಗಿ-ರ-ಬೇಕೆ
ಆಗಿ-ರ-ಲಿಲ್ಲ
ಆಗಿ-ರಲು
ಆಗಿ-ರುವ
ಆಗಿ-ರು-ವುದನ್ನು
ಆಗಿವೆ
ಆಗಿ-ಹೋ-ಯಿತು
ಆಗು-ತ್ತದೆ
ಆಗು-ತ್ತಿ-ದ್ದು-ದ-ರಿಂದ
ಆಗು-ತ್ತಿ-ದ್ದುವು
ಆಗು-ತ್ತಿ-ರ-ಲಿಲ್ಲ
ಆಗು-ತ್ತಿ-ರ-ಲಿ-ಲ್ಲ-ವೆಂ-ದಲ್ಲ
ಆಗು-ತ್ತಿಲ್ಲ
ಆಗುವ
ಆಗು-ವಂ-ತಿಲ್ಲ
ಆಗು-ವು-ದಿ-ದ್ದರೆ
ಆಗು-ವು-ದಿಲ್ಲ
ಆಗು-ವು-ದಿ-ಲ್ಲ-ಎಂದು
ಆಗು-ವು-ದೆಂಬ
ಆಗು-ವು-ದೇನು
ಆಗು-ಹೋ-ಗು-ಗಳನ್ನು
ಆಗೆಲ್ಲ
ಆಗೊಂದು
ಆಗೊಮ್ಮೆ
ಆಗ್ತೀ-ನಪ್ಪಾ
ಆಗ್ರಾ
ಆಗ್ರಾಕ್ಕೆ
ಆಗ್ರಾದ
ಆಗ್ರಾ-ದಿಂದ
ಆಘಾತ
ಆಘಾ-ತ-ಗೊಂಡ
ಆಘಾ-ತ-ವಾ-ಗ-ಬಾ-ರ-ದೆಂದು
ಆಘಾ-ತ-ವಾ-ಗು-ತ್ತಿದ್ದ
ಆಘಾ-ತ-ವಾ-ಯಿತು
ಆಘ್ರಾ-ಣಿ-ಸಿ-ದ್ದನ್ನು
ಆಚ-ರ-ಣೆಗೆ
ಆಚ-ರ-ಣೆ-ಯ-ಲ್ಲಿದ್ದ
ಆಚ-ರಿ-ಸಿದ
ಆಚ-ರಿಸು
ಆಚ-ರಿ-ಸು-ತ್ತಿ-ದ್ದರು
ಆಚ-ರಿ-ಸು-ತ್ತಿ-ದ್ದಳು
ಆಚ-ರಿ-ಸು-ವು-ದು-ಇದು
ಆಚಾ-ರ-ವಿ-ಚಾ-ರ-ಗ-ಳೆ-ಲ್ಲ-ದ-ರಲ್ಲೂ
ಆಚಾ-ರ-ಗಳ
ಆಚಾ-ರ-ಗಳು
ಆಚಾರ್ಯ
ಆಚಾ-ರ್ಯ-ಪು-ರು-ಷರೂ
ಆಚಾ-ರ್ಯರು
ಆಚಾ-ರ್ಯ-ರೊಂ-ದಿಗೆ
ಆಚೆ
ಆಚೆಗೆ
ಆಜನ್ಮ
ಆಜೀವ
ಆಜ್ಞಾ-ಪನೆ
ಆಜ್ಞಾ-ಪಿ-ಸಿ-ದರು
ಆಜ್ಞಾ-ರೂ-ಪದ
ಆಜ್ಞೆ
ಆಜ್ಞೆ-ಯನ್ನು
ಆಜ್ಞೆ-ಯಾ-ಗಿತ್ತು
ಆಜ್ಯೋತಿ
ಆಟ
ಆಟ-ಓ-ಡಾ-ಟ-ಗಳು
ಆಟ-ಕ್ಕಿ-ಳಿ-ದರೆ
ಆಟ-ಗ-ಳ-ನ್ನಾ-ಡ-ಬೇ-ಕೆಂಬ
ಆಟ-ಗಳಲ್ಲಿ
ಆಟದ
ಆಟ-ದಲ್ಲಿ
ಆಟ-ದಲ್ಲೂ
ಆಟ-ದೊಳು
ಆಟ-ವನ್ನು
ಆಟ-ವನ್ನೂ
ಆಟ-ವ-ಲ್ಲವೆ
ಆಟ-ವಾ-ಡಿ-ಕೊಂ-ಡಿರು
ಆಟ-ವಾ-ಡು-ತ್ತ-ಲಿ-ರುವೆ
ಆಟ-ವಾ-ಡು-ತ್ತಾನೆ
ಆಟ-ವಾ-ಡು-ತ್ತಿದ್ದ
ಆಟ-ವಾ-ಡು-ತ್ತಿ-ದ್ದ-ವನು
ಆಟ-ವಾ-ಡು-ತ್ತಿ-ದ್ದಾ-ನೆಯೋ
ಆಟ-ವಾದ
ಆಡದೆ
ಆಡ-ಬ-ಹುದು
ಆಡ-ಲಾ-ರದೆ
ಆಡ-ಲಿಲ್ಲ
ಆಡಲು
ಆಡಿ-ಕೊಂ-ಡಿ-ದ್ದೆವು
ಆಡಿದ
ಆಡಿ-ರ-ಬ-ಹುದು
ಆಡಿ-ಸು-ತ್ತಿ-ದ್ದಾನೆ
ಆಡು
ಆಡುತ್ತ
ಆಡು-ತ್ತಾರೆ
ಆಡು-ತ್ತಿದ್ದ
ಆಡು-ತ್ತಿ-ದ್ದರು
ಆಡು-ತ್ತಿ-ದ್ದಾಗ
ಆಡು-ತ್ತೇನೆ
ಆಡುವ
ಆಡು-ವಾ-ಗಲೂ
ಆಡು-ವುದನ್ನು
ಆಡು-ವು-ದಿಲ್ಲ
ಆಡು-ವು-ದೆಂ-ದರೆ
ಆಣ-ತಿ-ಯಂತೆ
ಆತ
ಆತಂಕ
ಆತಂ-ಕ-ಗೊಂ-ಡರು
ಆತಂ-ಕ-ಗೊಂ-ಡಿ-ದ್ದಾರೆ
ಆತನ
ಆತ-ನನ್ನು
ಆತ-ನಲ್ಲಿ
ಆತ-ನಿಂದ
ಆತ-ನಿಗೂ
ಆತ-ನಿಗೆ
ಆತನೂ
ಆತ-ನೊಂ-ದಿ-ಗಿನ
ಆತ-ನೊಂ-ದಿಗೆ
ಆತಿ
ಆತಿ-ಥೇ-ಯ-ನಾದ
ಆತಿ-ಥೇ-ಯ-ರಾದ
ಆತಿಥ್ಯ
ಆತಿ-ಥ್ಯ-ವನ್ನು
ಆತಿ-ಥ್ಯ-ವನ್ನೂ
ಆತ್ಮ
ಆತ್ಮಕ್ಕೆ
ಆತ್ಮಗೆ
ಆತ್ಮ-ಗೌ-ರ-ವದ
ಆತ್ಮ-ಗೌ-ರ-ವ-ವನ್ನು
ಆತ್ಮ-ಜಾ-ಗೃ-ತಿಯ
ಆತ್ಮ-ಜಾ-ಗೃ-ತಿ-ಯುಂ-ಟಾ-ಗಲಿ
ಆತ್ಮ-ಜ್ಞಾ-ನದ
ಆತ್ಮ-ಜ್ಯೋತಿ
ಆತ್ಮ-ತ-ತ್ತ್ವದ
ಆತ್ಮದ
ಆತ್ಮ-ದಲ್ಲಿ
ಆತ್ಮ-ದೊ-ಳಿ-ಳಿದು
ಆತ್ಮನ
ಆತ್ಮ-ನನ್ನು
ಆತ್ಮ-ನಿ-ರ-ತ-ನಾದ
ಆತ್ಮನು
ಆತ್ಮನೇ
ಆತ್ಮ-ನೊ-ಬ್ಬನೇ
ಆತ್ಮ-ಭಾ-ವ-ದಲ್ಲಿ
ಆತ್ಮ-ರ-ಕ್ಷ-ಣೆಗೆ
ಆತ್ಮ-ರಾ-ಜ್ಯ-ವನ್ನು
ಆತ್ಮ-ವಂ-ಚ-ನೆ-ಯಾ-ಗು-ತ್ತ-ದೆ-ಇದು
ಆತ್ಮ-ವನ್ನು
ಆತ್ಮ-ವಾಗಿ
ಆತ್ಮ-ವಾ-ರಿಗೆ
ಆತ್ಮ-ವಿ-ಕಾ-ಸದ
ಆತ್ಮ-ವಿ-ದ್ಯೆ-ಯನ್ನು
ಆತ್ಮ-ವಿ-ಶ್ವಾಸ
ಆತ್ಮ-ವಿ-ಶ್ವಾ-ಸಕ್ಕೆ
ಆತ್ಮ-ವಿ-ಶ್ವಾ-ಸ-ವನ್ನು
ಆತ್ಮ-ವಿ-ಶ್ವಾ-ಸ-ವಿತ್ತು
ಆತ್ಮ-ವಿ-ಶ್ವಾ-ಸ-ವಿದೆ
ಆತ್ಮ-ವಿ-ಶ್ವಾ-ಸವು
ಆತ್ಮವು
ಆತ್ಮ-ವೆಂದ
ಆತ್ಮ-ವೆ-ಲ್ಲಿಯು
ಆತ್ಮವೇ
ಆತ್ಮ-ವೊಂ-ದಾ-ಗಿ-ರು-ವುದು
ಆತ್ಮ-ವೊಂ-ದಿ-ದೆ-ಯೆಂ-ಬು-ದನ್ನು
ಆತ್ಮ-ಶಕ್ತಿ
ಆತ್ಮ-ಶ-ಕ್ತಿಯ
ಆತ್ಮ-ಸಂ-ಯಮ
ಆತ್ಮ-ಸಂ-ಯ-ಮದ
ಆತ್ಮ-ಸಂ-ಯ-ಮ-ಪೂರ್ಣ
ಆತ್ಮ-ಸಂ-ಯ-ಮ-ವನ್ನು
ಆತ್ಮ-ಸಂ-ಯ-ಮ-ವಾ-ಗಲಿ
ಆತ್ಮ-ಸಾ-ಕ್ಷಾ-ತ್ಕಾರ
ಆತ್ಮ-ಸಾ-ಕ್ಷಾ-ತ್ಕಾ-ರಕ್ಕೆ
ಆತ್ಮ-ಸಾ-ಕ್ಷಾ-ತ್ಕಾ-ರದ
ಆತ್ಮ-ಸಾ-ಕ್ಷಾ-ತ್ಕಾ-ರವು
ಆತ್ಮ-ಸ್ವ-ರೂಪಿ
ಆತ್ಮ-ಹತ್ಯೆ
ಆತ್ಮಾ-ನಂ-ದ-ವನ್ನು
ಆತ್ಮಾ-ನು-ಭ-ವದ
ಆತ್ಮಾವ
ಆತ್ಮಾ-ವ-ಲಂ-ಬನ
ಆತ್ಮೀಯ
ಆತ್ಮೀ-ಯತೆ
ಆತ್ಮೀ-ಯ-ತೆ-ಸ್ನೇಹ
ಆತ್ಮೀ-ಯ-ತೆ-ಯಿಂದ
ಆತ್ಮೀ-ಯನೂ
ಆತ್ಮೀ-ಯನೋ
ಆತ್ಮೀ-ಯರು
ಆತ್ಮೀ-ಯ-ವಾ-ಗಲು
ಆತ್ಮೀ-ಯ-ವಾಗಿ
ಆತ್ಮೀ-ಯ-ವಾ-ಗು-ತ್ತಿತ್ತು
ಆತ್ಮೋ-ದ್ಧಾರ
ಆತ್ಯಂ-ತಿಕ
ಆದ
ಆದಂ-ತಾ-ಯಿತು
ಆದ-ದ್ದಾ-ಯಿತು
ಆದ-ಮೇಲೂ
ಆದರ
ಆದ-ರ-ದಿಂದ
ಆದ-ರ-ಪೂ-ರ್ವಕ
ಆದ-ರವೂ
ಆದ-ರಿ-ನ್ನೇನು
ಆದ-ರಿ-ಸುವ
ಆದ-ರೀಗ
ಆದರೂ
ಆದರೆ
ಆದ-ರೇನು
ಆದ-ರೊಂದು
ಆದ-ರೊಂದೇ
ಆದರ್ಶ
ಆದ-ರ್ಶ-ಕ್ಕ-ನು-ಗು-ಣ-ವಾಗಿ
ಆದ-ರ್ಶ-ಕ್ಕಾಗಿ
ಆದ-ರ್ಶ-ಗಳ
ಆದ-ರ್ಶ-ಗಳನ್ನು
ಆದ-ರ್ಶ-ಗ-ಳಿಗೂ
ಆದ-ರ್ಶ-ಗಳು
ಆದ-ರ್ಶದ
ಆದ-ರ್ಶ-ದಿಂದ
ಆದ-ರ್ಶ-ದೆ-ಡೆಗೆ
ಆದ-ರ್ಶ-ಪ-ರಿ-ಪಾ-ಲ-ನೆ-ಗಾಗಿ
ಆದ-ರ್ಶ-ಪ್ರಾಯ
ಆದ-ರ್ಶ-ಪ್ರಾ-ಯ-ವಾ-ಗ-ಬಲ್ಲ
ಆದ-ರ್ಶ-ಮಾ-ನ-ವ-ನನ್ನು
ಆದ-ರ್ಶ-ವ-ನ್ನಿ-ಟ್ಟು-ಕೊಂಡು
ಆದ-ರ್ಶ-ವನ್ನು
ಆದ-ರ್ಶ-ವನ್ನೇ
ಆದ-ರ್ಶ-ವಾದಿ
ಆದ-ರ್ಶವೂ
ಆದ-ರ್ಶವೇ
ಆದ-ರ್ಶವ್ನು
ಆದ-ರ್ಶ-ಶಾ-ಸ್ತ್ರದ
ಆದಾಗ
ಆದಾಯ
ಆದಾ-ಯ-ದಲ್ಲಿ
ಆದಿ-ಅಂ-ತ್ಯ-ಗಳ
ಆದೆಲ್ಲ
ಆದೇ
ಆದೇ-ವ-ಲಾ-ಯದ
ಆದೇಶ
ಆದೇ-ಶ-ಗಳನ್ನು
ಆದೇ-ಶ-ದಂತೆ
ಆದೇ-ಶ-ವನ್ನು
ಆದೇ-ಶ-ವಾ-ಗಿತ್ತು
ಆದೇ-ಶಿ-ಸಿ-ದರು
ಆದೇ-ಶಿ-ಸಿ-ದ-ರು-ಸಾ-ಧು-ಸಂ-ನ್ಯಾ-ಸಿ-ಗಳು
ಆದೊ-ಡೇ-ನಂ-ತಾ-ತ್ಮ-ವೆಂ-ಬುದು
ಆದ್ದ
ಆದ್ದ-ರಿಂದ
ಆದ್ದ-ರಿಂ-ದಲೇ
ಆದ್ಯ
ಆದ್ಯಾ-ಶಕ್ತಿ
ಆಧ-ರಿ-ಸಿದ
ಆಧಾ-ತ್ಮಿಕ
ಆಧಾರ
ಆಧಾ-ರ-ಗಳ
ಆಧಾ-ರ-ಗಳನ್ನು
ಆಧಾ-ರದ
ಆಧಾ-ರ-ಭೂ-ತ-ವಾದ
ಆಧಾ-ರವೇ
ಆಧಾ-ರ-ಸ್ತಂಭ
ಆಧಿ-ಕ್ಯ-ದಿಂದ
ಆಧಿ-ಪ-ತ್ಯ-ವನ್ನು
ಆಧು-ನಿಕ
ಆಧ್ಯಾ
ಆಧ್ಯಾತ್ಮ
ಆಧ್ಯಾ-ತ್ಮ-ಜೀ-ವ-ನದ
ಆಧ್ಯಾ-ತ್ಮ-ಜ್ಯೋ-ತಿ-ಯಿಂದ
ಆಧ್ಯಾ-ತ್ಮ-ಶೀಲ
ಆಧ್ಯಾ-ತ್ಮ-ಸಾ-ಧ-ಕರು
ಆಧ್ಯಾ-ತ್ಮಿಕ
ಆಧ್ಯಾ-ತ್ಮಿ-ಕತೆ
ಆಧ್ಯಾ-ತ್ಮಿ-ಕ-ತೆಗೂ
ಆಧ್ಯಾ-ತ್ಮಿ-ಕ-ತೆಯ
ಆಧ್ಯಾ-ತ್ಮಿ-ಕ-ತೆ-ಯನ್ನು
ಆಧ್ಯಾ-ತ್ಮಿ-ಕ-ತೆ-ಯಲ್ಲಿ
ಆಧ್ಯಾ-ತ್ಮಿ-ಕ-ತೆ-ಯಿರು
ಆಧ್ಯಾ-ತ್ಮಿ-ಕ-ತೆ-ಯೆಂದು
ಆಧ್ಯಾ-ತ್ಮಿ-ಕ-ತೆ-ಯೆಂ-ಬುದು
ಆಧ್ಯಾ-ತ್ಮಿ-ಕರು
ಆಧ್ಯಾ-ತ್ಮಿ-ಕ-ವಾಗಿ
ಆಧ್ಯಾ-ತ್ಮಿ-ಕ-ವಾ-ಗಿಯೂ
ಆಧ್ಯಾ-ತ್ಮಿ-ಕ-ಶ-ಕ್ತಿಯು
ಆಧ್ಯಾ-ತ್ಮಿಕಾ
ಆಧ್ಯಾ-ತ್ಮಿ-ಕಾ-ನು-ಭ-ವಾ-ಯಿತು
ಆನಂದ
ಆನಂ-ದಕ್ಕೆ
ಆನಂ-ದದ
ಆನಂ-ದ-ದಲ್ಲಿ
ಆನಂ-ದ-ದಿಂದ
ಆನಂ-ದ-ದಿಂ-ದಿ-ದ್ದು-ಬಿಡು
ಆನಂ-ದ-ದಿಂ-ದಿದ್ದೆ
ಆನಂ-ದ-ದೆ-ಡೆಗೆ
ಆನಂ-ದನ
ಆನಂ-ದ-ನಾ-ರಾ-ಯಣ
ಆನಂ-ದ-ಪ-ಡುತ್ತ
ಆನಂ-ದ-ಪ-ರ-ವ-ಶ-ನಾಗಿ
ಆನಂ-ದ-ಬಾಷ್ಟ
ಆನಂ-ದ-ಭ-ರಿತ
ಆನಂ-ದ-ಭ-ರಿ-ತ-ನಾಗಿ
ಆನಂ-ದ-ಭ-ರಿ-ತ-ರಾಗಿ
ಆನಂ-ದ-ಮ-ಯ-ವಾದ
ಆನಂ-ದ-ಮ-ಯಿ-ಯಾದ
ಆನಂ-ದ-ಮೂರ್ತಿ
ಆನಂ-ದ-ಮೋ-ಹನ
ಆನಂ-ದ-ವ-ನ್ನ-ನು-ಭ-ವಿ-ಸು-ತ್ತಿ-ದ್ದರು
ಆನಂ-ದ-ವ-ನ್ನಾ-ಗಲಿ
ಆನಂ-ದ-ವನ್ನು
ಆನಂ-ದ-ವಾ-ಯಿ-ತಷ್ಟೇ
ಆನಂ-ದವೂ
ಆನಂ-ದ-ವೆಂದರೆ
ಆನಂ-ದ-ವೆಂಬ
ಆನಂ-ದ-ವೆಲ್ಲಿ
ಆನಂ-ದ-ಸ-ಮಾ-ಧಿಯ
ಆನಂದಾ
ಆನಂ-ದಾ-ತಿ-ಶ-ಯ-ದಿಂದ
ಆನಂ-ದಾ-ನು-ಭ-ವದ
ಆನಂ-ದಾ-ನು-ಭ-ವ-ದಿಂದ
ಆನಂ-ದಾ-ನು-ಭ-ವ-ವನ್ನು
ಆನಂ-ದಾ-ಮೃ-ತ-ದಲ್ಲಿ
ಆನಂ-ದಾಶ್ರು
ಆನಂ-ದಿ-ತ-ರಾದ
ಆನಂ-ದಿ-ಸ-ಬೇ-ಕಾ-ದದ್ದು
ಆನಂ-ದಿ-ಸ-ಬೇ-ಕಾ-ದರೆ
ಆನಂ-ದಿ-ಸಲು
ಆನಂ-ದಿ-ಸಿ-ದರು
ಆನಂ-ದಿ-ಸು-ತ್ತಿ-ದ್ದರು
ಆನಂ-ದಿ-ಸು-ತ್ತೇನೆ
ಆನಂ-ದಿ-ಸು-ವು-ದ-ಕ್ಕಾ-ಗಿಯೇ
ಆನಂ-ದೋ-ತ್ಸಾ-ಹ-ಗಳು
ಆನಂ-ದೋ-ತ್ಸಾ-ಹದ
ಆನಂ-ದೋ-ತ್ಸಾ-ಹ-ಭ-ರಿ-ತ-ನಾಗಿ
ಆನಂ-ದೋ-ತ್ಸಾ-ಹ-ಭ-ರಿ-ತ-ರಾಗಿ
ಆನಂ-ದೋ-ದ್ರೇ-ಕ-ದಿಂದ
ಆನಂ-ದೋ-ದ್ವೇ-ಗ-ದಿಂದ
ಆನಂ-ದೋ-ನ್ಮ-ತ್ತ-ನಾಗಿ
ಆನೆ
ಆನೆ-ಯನ್ನು
ಆಪ-ತ್ತಿಗೆ
ಆಪ್ತ
ಆಪ್ತನೂ
ಆಪ್ತ-ರಾ-ಗಿಯೇ
ಆಪ್ತರೇ
ಆಪ್ತ-ವ-ರ್ಗ-ದಲ್ಲಿ
ಆಫೀ-ಸಿಗೆ
ಆಫೀ-ಸಿನ
ಆಫೀ-ಸಿ-ನಲ್ಲಿ
ಆಫೀ-ಸಿ-ನಿಂದ
ಆಫೀಸು
ಆಮಂ-ತ್ರ-ಣ-ವಿ-ತ್ತಿ-ದ್ದರು
ಆಮಂ-ತ್ರ-ಣವೂ
ಆಮಂ-ತ್ರಿಸು
ಆಮ-ಶಂಕೆ
ಆಮೂ-ಲಾ-ಗ್ರ-ವಾಗಿ
ಆಮೇಲೂ
ಆಮೇಲೆ
ಆಮೋ-ದ-ಪ್ರ-ಮೋ-ದ-ಗಳಲ್ಲಿ
ಆಮೋ-ದ-ಪ್ರ-ಮೋ-ದ-ಗಳಲ್ಲಿ
ಆಯಾ
ಆಯಾಸ
ಆಯಾ-ಸ-ಗೊಂ-ಡಿದ್ದ
ಆಯಾ-ಸ-ಗೊಂ-ಡಿ-ದ್ದಾರೆ
ಆಯಾ-ಸ-ಗೊಂಡು
ಆಯಾ-ಸ-ವಾ-ಯಿತು
ಆಯಾ-ಸ-ವೆಲ್ಲ
ಆಯಿತು
ಆಯಿ-ತು-ಮುಂದೆ
ಆಯಿ-ತೆ-ನ್ನ-ಬೇಕು
ಆಯು-ರ್ವೇದ
ಆಯುಷ್ಯ
ಆಯ್ದು
ಆಯ್ದು-ಕೊಂಡ
ಆಯ್ದು-ಕೊಂ-ಡಿದ್ದ
ಆರ
ಆರಂ-ಭ-ದಲ್ಲೇ
ಆರಂ-ಭ-ವಾ-ಗಿ-ಬಿ-ಟ್ಟಿದೆ
ಆರಂ-ಭ-ವಾ-ಗಿ-ಬಿ-ಡು-ತ್ತಿ-ತ್ತು-ಮ-ಠ-ದಲ್ಲೇ
ಆರಂ-ಭ-ವಾ-ಗು-ತ್ತದೆ
ಆರಂ-ಭ-ವಾ-ಯಿತು
ಆರಂ-ಭಿ-ಸಿದ
ಆರಂ-ಭಿ-ಸಿ-ದ್ದನ್ನು
ಆರಂ-ಭಿ-ಸು-ತ್ತಿದ್ದೆ
ಆರತಿ
ಆರ-ತಿಯ
ಆರ-ತಿ-ಯಾ-ಗು-ತ್ತಿ-ದ್ದರೆ
ಆರದೆ
ಆರ-ನೆಯ
ಆರಾ-ಧ-ಕರು
ಆರಾ-ಧ-ಕರೂ
ಆರಾ-ಧ-ನೆ-ಯನ್ನೂ
ಆರಾ-ಧ-ನೆ-ಯನ್ನೇ
ಆರಾ-ಧಿ-ಸಲಿ
ಆರಾ-ಧಿ-ಸ-ಲ್ಪ-ಡು-ತ್ತಿ-ದ್ದಾ-ನೆಯೋ
ಆರಾ-ಧಿಸಿ
ಆರಾ-ಧಿ-ಸಿ-ದಾಗ
ಆರಾ-ಧಿ-ಸು-ವುದ
ಆರಾಧ್ಯ
ಆರಾ-ಧ್ಯ-ದೇ-ವ-ತೆಯೇ
ಆರಾ-ಧ್ಯ-ದೈ-ವ-ಇ-ರು-ವಳೆ
ಆರಾ-ಧ್ಯ-ದೈ-ವ-ರಾದ
ಆರಾ-ಮ-ವಾಗಿ
ಆರಿಗೆ
ಆರಿಲ್ಲ
ಆರಿಸಿ
ಆರಿ-ಸಿ-ಕೊಂಡ
ಆರಿ-ಸಿ-ಕೊಂ-ಡದ್ದು
ಆರಿ-ಸಿ-ಕೊಂ-ಡರೂ
ಆರಿ-ಸಿ-ಕೊಂ-ಡಿದ್ದ
ಆರಿ-ಸಿ-ಕೊಂಡು
ಆರಿ-ಸಿ-ಬಿ-ಟ್ಟರು
ಆರಿ-ಸಿ-ಬಿ-ಟ್ಟಿತು
ಆರು
ಆರೇಳು
ಆರೈಕೆ
ಆರೈ-ಕೆಯ
ಆರೈ-ಕೆ-ಯಲ್ಲಿ
ಆರೈ-ಕೆ-ಯಿಂದ
ಆರೋಗ್ಯ
ಆರೋ-ಗ್ಯ-ಕ-ರ-ವಾ-ದ-ದ್ದೆಂದು
ಆರೋ-ಗ್ಯಕ್ಕೆ
ಆರೋ-ಗ್ಯ-ದಿಂ-ದಿ-ದ್ದಾ-ನಷ್ಟೆ
ಆರೋ-ಗ್ಯ-ದಿಂ-ದಿ-ದ್ದುದು
ಆರೋ-ಗ್ಯ-ವಂತ
ಆರೋ-ಗ್ಯವೂ
ಆರೋ-ಗ್ಯ-ಶಾ-ಲಿ-ಯಾದ
ಆರೋ-ಪಿ-ಸು-ತ್ತಾರೆ
ಆರ್ಥಿಕ
ಆಲದ
ಆಲಸ್ಯ
ಆಲ-ಸ್ಯ-ವನ್ನು
ಆಲಿ
ಆಲಿಂ-ಗಿ-ಸಿ-ಕೊಂ-ಡು-ಬಿಟ್ಟ
ಆಲಿ-ಸಿದ
ಆಲಿ-ಸಿ-ದರು
ಆಲಿ-ಸಿ-ದಾಗ
ಆಲಿ-ಸುತ್ತ
ಆಲಿ-ಸು-ತ್ತಿದ್ದ
ಆಲಿ-ಸು-ತ್ತಿ-ದ್ದರು
ಆಲಿ-ಸು-ತ್ತಿ-ದ್ದಳು
ಆಲೋ
ಆಲೋ-ಚನಾ
ಆಲೋ-ಚ-ನಾ-ಪ-ಥ-ವನ್ನು
ಆಲೋ-ಚ-ನಾ-ಶಕ್ತಿ
ಆಲೋ-ಚ-ನಾ-ಶ-ಕ್ತಿ-ಯೆಂ-ಬುದು
ಆಲೋ-ಚನೆ
ಆಲೋ-ಚ-ನೆ-ನಂ-ಬಿ-ಕೆ-ಗ-ಳಿಗೆ
ಆಲೋ-ಚ-ನೆ-ಗಳ
ಆಲೋ-ಚ-ನೆ-ಗಳನ್ನು
ಆಲೋ-ಚ-ನೆ-ಗಳನ್ನೆಲ್ಲ
ಆಲೋ-ಚ-ನೆ-ಗ-ಳನ್ನೇ
ಆಲೋ-ಚ-ನೆ-ಗ-ಳಲ್ಲೇ
ಆಲೋ-ಚ-ನೆ-ಗಳು
ಆಲೋ-ಚ-ನೆ-ಗಳೇ
ಆಲೋ-ಚ-ನೆಗೆ
ಆಲೋ-ಚ-ನೆ-ಯನ್ನು
ಆಲೋ-ಚ-ನೆ-ಯನ್ನೂ
ಆಲೋ-ಚ-ನೆ-ಯನ್ನೇ
ಆಲೋ-ಚ-ನೆ-ಯಲ್ಲಿ
ಆಲೋ-ಚ-ನೆ-ಯ-ಲ್ಲಿದ್ದ
ಆಲೋ-ಚ-ನೆಯು
ಆಲೋ-ಚ-ನೆಯೇ
ಆಲೋಚಿ
ಆಲೋ-ಚಿ-ಸ-ತೊ-ಡ-ಗಿದ
ಆಲೋ-ಚಿ-ಸ-ತೊ-ಡ-ಗಿ-ದ-ನಾ-ನೆಂ-ದು-ಕೊಂ-ಡಿ-ದ್ದಂತೆ
ಆಲೋ-ಚಿ-ಸ-ತೊ-ಡ-ಗಿ-ದ-ನಿ-ರಾ-ಕಾ-ರ-ವಾದ
ಆಲೋ-ಚಿ-ಸ-ಬ-ಹುದು
ಆಲೋ-ಚಿ-ಸ-ಬೇ-ಕೆಂದು
ಆಲೋ-ಚಿಸಿ
ಆಲೋ-ಚಿ-ಸಿ-ವಿ-ಮ-ರ್ಶಿಸಿ
ಆಲೋ-ಚಿ-ಸಿದ
ಆಲೋ-ಚಿ-ಸಿ-ದಂ-ತೆಲ್ಲ
ಆಲೋ-ಚಿ-ಸಿ-ದ-ಬ-ಹುಶಃ
ಆಲೋ-ಚಿ-ಸಿ-ದರು
ಆಲೋ-ಚಿ-ಸಿ-ದಾಗ
ಆಲೋ-ಚಿ-ಸಿದೆ
ಆಲೋ-ಚಿ-ಸಿ-ದ್ದೀ-ರೇನು
ಆಲೋ-ಚಿ-ಸುತ್ತ
ಆಲೋ-ಚಿ-ಸು-ತ್ತಿ-ದೆ-ಭ-ಗ-ವಂ-ತ-ನೆಂ-ಬ-ವನು
ಆಲೋ-ಚಿ-ಸು-ತ್ತಿ-ದ್ದಾರೆ
ಆಲೋ-ಚಿ-ಸು-ತ್ತಿ-ದ್ದು-ದ-ಲ್ಲದೆ
ಆಲೋ-ಚಿ-ಸು-ತ್ತಿದ್ದೆ
ಆಲೋ-ಚಿ-ಸುವ
ಆಲೋ-ಚಿ-ಸು-ವಂ-ತಾ-ಗು-ತ್ತಿತ್ತು
ಆಲ್ಮೋ-ರಕ್ಕೆ
ಆಲ್ಮೋ-ರದ
ಆಲ್ಮೋ-ರ-ದಲ್ಲಿ
ಆಲ್ಮೋ-ರ-ದಲ್ಲೇ
ಆಲ್ಮೋ-ರ-ದ-ವ-ರೆಗೂ
ಆಲ್ಮೋ-ರಾ-ದ-ಲ್ಲಿದ್ದ
ಆಳ
ಆಳ-ವೈ-ಶಾ-ಲ್ಯ-ಗಳನ್ನು
ಆಳ-ಕ್ಕಿ-ಳಿದು
ಆಳಕ್ಕೆ
ಆಳ-ತೊ-ಡ-ಗಿದೆ
ಆಳ-ದಿಂ-ದಲೇ
ಆಳ-ವನ್ನು
ಆಳ-ವಾಗಿ
ಆಳ-ವಾ-ಗುತ್ತ
ಆಳ-ವಾದ
ಆಳ-ವಾ-ದದ್ದು
ಆಳ-ವಿ-ರ-ಲಿಲ್ಲ
ಆಳು
ಆಳು-ಕಾಳು-ಗ-ಳಿ-ದ್ದರು
ಆಳು-ಕಾಳು-ಗಳು
ಆಳ್ವಿ-ಕೆಯ
ಆವ-ರಣ
ಆವ-ರ-ಣ-ಗ-ಳೆಲ್ಲ
ಆವ-ರ-ಣ-ವನ್ನು
ಆವ-ರ-ಣ-ವನ್ನೂ
ಆವ-ರಿಸಿ
ಆವ-ರಿ-ಸಿ-ಕೊಂ-ಡು-ಬಿ-ಟ್ಟಿತ್ತು
ಆವ-ರಿ-ಸಿತು
ಆವ-ರಿ-ಸಿದ
ಆವ-ರಿ-ಸಿ-ಬಿ-ಟ್ಟಿತ್ತು
ಆವ-ರಿ-ಸಿ-ಬಿ-ಟ್ಟಿದೆ
ಆವ-ರಿ-ಸಿ-ಬಿ-ಡು-ತ್ತಿತ್ತು
ಆವ-ರಿ-ಸಿ-ಬಿ-ಡು-ತ್ತಿ-ದ್ದುವು
ಆವ-ಶ್ಯ-ಕತೆ
ಆವ-ಶ್ಯ-ಕ-ತೆ-ಗಳನ್ನು
ಆವ-ಶ್ಯ-ಕ-ತೆ-ಗಳನ್ನೆಲ್ಲ
ಆವ-ಶ್ಯ-ಕ-ತೆಗೆ
ಆವ-ಶ್ಯ-ಕ-ತೆ-ಯಾ-ದರೂ
ಆವ-ಶ್ಯ-ಕ-ತೆ-ಯಿತ್ತು
ಆವ-ಶ್ಯ-ಕ-ತೆ-ಯಿದೆ
ಆವ-ಶ್ಯ-ಕ-ತೆಯೂ
ಆವ-ಶ್ಯ-ಕ-ತೆಯೇ
ಆವ-ಶ್ಯ-ಕ-ತೆ-ಯೇ-ನಿದೆ
ಆವ-ಶ್ಯ-ಕ-ತೆ-ಯೇನೂ
ಆವ-ಶ್ಯ-ಕ-ವಾದ
ಆವ-ಶ್ಯ-ಕ-ವಾ-ದದ್ದು
ಆವಾ-ಸ-ಸ್ಥಾನ
ಆವಾ-ಸ-ಸ್ಥಾ-ನ-ವಾ-ಗಿ-ಬಿ-ಟ್ಟಿತ್ತು
ಆವಾ-ಸ-ಸ್ಥಾ-ನ-ವೆಂ-ಬುದು
ಆವಾ-ಹನೆ
ಆವಿರ್
ಆವಿ-ರ್ಭಾವ
ಆವಿ-ರ್ಭಾ-ವ-ಗಳೇ
ಆವಿ-ರ್ಭಾ-ವ-ವಾ-ಗು-ತ್ತಿ-ರು-ವಂತೆ
ಆವಿ-ಷ್ಕ-ರಿ-ಸಿ-ದರು
ಆವು-ದ-ನೀ-ತೆರ
ಆವೃ-ತ್ತಿಯ
ಆವೇಶ
ಆವೇ-ಶ-ಭ-ರಿತ
ಆವೇ-ಶ-ಭ-ರಿ-ತ-ರಾಗಿ
ಆವೇ-ಶ-ವುಂ-ಟಾ-ಯಿತು
ಆಶಯ
ಆಶ-ಯಕ್ಕೆ
ಆಶ-ಯ-ದಿಂ-ದ-ಸ್ವಾ-ಮೀಜಿ
ಆಶಾ-ಪಾ-ಶ-ಗಳು
ಆಶಾ-ಪೂರ್ಣ
ಆಶಿಸಿ
ಆಶೀ-ರ್ವ-ದಿಸಿ
ಆಶೀ-ರ್ವ-ದಿ-ಸಿ-ದಳು
ಆಶೀ-ರ್ವ-ದಿ-ಸುತ್ತ
ಆಶೀ-ರ್ವಾದ
ಆಶೀ-ರ್ವಾ-ದ-ರೂ-ಪದ
ಆಶೀ-ರ್ವಾ-ದ-ವನ್ನು
ಆಶೋ-ತ್ತ-ರ-ಗಳ
ಆಶ್ಚರ್ಯ
ಆಶ್ಚ-ರ್ಯ-ಆ-ನಂದ
ಆಶ್ಚ-ರ್ಯ-ಕರ
ಆಶ್ಚ-ರ್ಯ-ಕ-ರ-ನಾ-ಗಿರ
ಆಶ್ಚ-ರ್ಯ-ಕ-ರ-ವಾಗಿ
ಆಶ್ಚ-ರ್ಯ-ಕ-ರ-ವಾದ
ಆಶ್ಚ-ರ್ಯ-ಕು-ತೂ-ಹ-ಲ-ಗಳಿಂದ
ಆಶ್ಚ-ರ್ಯಕ್ಕೆ
ಆಶ್ಚ-ರ್ಯ-ಗೊಂ-ಡರು
ಆಶ್ಚ-ರ್ಯ-ಚ-ಕಿ-ತ-ನಾದ
ಆಶ್ಚ-ರ್ಯ-ಚಿ-ಕಿ-ತ-ರಾ-ದರು
ಆಶ್ಚ-ರ್ಯದ
ಆಶ್ಚ-ರ್ಯ-ದಿಂದ
ಆಶ್ಚ-ರ್ಯ-ಪ-ಟ್ಟರು
ಆಶ್ಚ-ರ್ಯ-ಪಟ್ಟೆ
ಆಶ್ಚ-ರ್ಯ-ಪ-ಡುತ್ತ
ಆಶ್ಚ-ರ್ಯ-ವನ್ನು
ಆಶ್ಚ-ರ್ಯ-ವಾ-ಗ-ದಿ-ರ-ಲಿಲ್ಲ
ಆಶ್ಚ-ರ್ಯ-ವಾ-ಗಿತ್ತು
ಆಶ್ಚ-ರ್ಯ-ವಾ-ಗಿ-ರ-ಬೇಕು
ಆಶ್ಚ-ರ್ಯ-ವಾ-ಗು-ವುದು
ಆಶ್ಚ-ರ್ಯ-ವಾ-ಯಿತು
ಆಶ್ಚ-ರ್ಯ-ವಿ-ರ-ಲಿಲ್ಲ
ಆಶ್ಚ-ರ್ಯ-ವಿಲ್ಲ
ಆಶ್ಚ-ರ್ಯವೂ
ಆಶ್ಚ-ರ್ಯ-ವೇ-ನಿದೆ
ಆಶ್ಚ-ರ್ಯ-ವೇ-ನಿಲ್ಲ
ಆಶ್ಚ-ರ್ಯ-ವೇನೂ
ಆಶ್ಚ-ರ್ಯವೋ
ಆಶ್ಚ-ರ್ಯಾ-ಘಾ-ತ-ಗೊಂಡು
ಆಶ್ಚ-ರ್ಯಾ-ಘಾ-ತ-ಗೊಂಡೆ
ಆಶ್ಚರ್ಯೋ
ಆಶ್ರಮ
ಆಶ್ರ-ಮಕ್ಕೆ
ಆಶ್ರ-ಮ-ಗಳು
ಆಶ್ರ-ಮ-ದಲ್ಲಿ
ಆಶ್ರ-ಮ-ವಾ-ಸಿ-ಗಳ
ಆಶ್ರಯ
ಆಶ್ರ-ಯಕ್ಕೆ
ಆಶ್ರ-ಯ-ದಲ್ಲಿ
ಆಶ್ರ-ಯ-ವ-ನ್ನ-ರ-ಸುತ್ತ
ಆಶ್ರ-ಯ-ಸ್ಥಾ-ನ-ವಾಗಿ
ಆಶ್ರ-ಯಿ-ಸಿ-ಕೊಂ-ಡಿ-ದ್ದರು
ಆಶ್ರ-ಯಿ-ಸಿದ
ಆಶ್ರ-ಯಿ-ಸಿ-ದರು
ಆಷಾ-ಢ-ಭೂತಿ
ಆಷಾ-ಢ-ಭೂ-ತಿ-ಗ-ಳೆಂದು
ಆಸಕ್ತಿ
ಆಸ-ಕ್ತಿ-ಮ-ಮ-ಕಾರ
ಆಸ-ಕ್ತಿ-ಯನ್ನು
ಆಸ-ಕ್ತಿ-ಯಿಂದ
ಆಸ-ಕ್ತಿ-ಯಿಲ್ಲ
ಆಸ-ಕ್ತಿ-ಯೇ-ನಿ-ಲ್ಲ-ದಿ-ದ್ದರೂ
ಆಸ-ನ-ಗಳನ್ನು
ಆಸ-ನದ
ಆಸ-ನ-ದಿಂ-ದೆದ್ದು
ಆಸ-ನ-ವನ್ನು
ಆಸೀ-ನ-ರಾ-ಗಿ-ದ್ದಾರೆ
ಆಸೆ
ಆಸೆ-ಆ-ಕಾಂ-ಕ್ಷೆ-ಗಳನ್ನು
ಆಸೆ-ಭ-ಯ-ಗಳನ್ನು
ಆಸೆ-ಗಳನ್ನು
ಆಸೆ-ಗಳಿಂದ
ಆಸೆ-ಗ-ಳಿಗೆ
ಆಸೆ-ಗಳೇ
ಆಸೆ-ಪಟ್ಟು
ಆಸೆಯ
ಆಸೆ-ಯಾಗಿ
ಆಸೆ-ಯಾ-ಗಿ-ಬಿ-ಟ್ಟಿತು
ಆಸೆ-ಯಾಗು
ಆಸೆ-ಯಿಂದ
ಆಸ್ತಿ
ಆಸ್ತಿ-ಪಾಸ್ತಿ
ಆಸ್ತಿ-ಮನೆ
ಆಸ್ತಿ-ಮ-ನೆ-ಯನ್ನೇ
ಆಸ್ತಿಕ
ಆಸ್ತಿ-ಕ-ತೆಯ
ಆಸ್ತಿ-ಗಾಗಿ
ಆಸ್ತಿ-ಪಾ-ಸ್ತಿಯ
ಆಸ್ತಿಯ
ಆಸ್ತಿ-ಯನ್ನೂ
ಆಸ್ತಿ-ಯೆಲ್ಲ
ಆಸ್ಥಾ-ನಿ-ಕ-ರನ್ನು
ಆಸ್ಥೆ
ಆಸ್ಪದ
ಆಸ್ಪ-ದ-ವಿ-ಲ್ಲದ
ಆಸ್ವಾ-ದಿ-ಸಲೂ
ಆಹಾ
ಆಹಾರ
ಆಹಾ-ರ-ಪಾ-ನೀ-ಯ-ಗಳೇ
ಆಹಾ-ರ-ಬ-ಟ್ಟೆ-ಗಳೇ
ಆಹಾ-ರ-ವಿ-ಶ್ರಾಂ-ತಿ-ಗ-ಳಿ-ಲ್ಲ-ದಿ-ದ್ದುದು
ಆಹಾ-ರ-ಕ್ಕಾಗಿ
ಆಹಾ-ರದ
ಆಹಾ-ರ-ದಲ್ಲಿ
ಆಹಾ-ರ-ವನ್ನು
ಆಹಾ-ರ-ವನ್ನೂ
ಆಹಾ-ರ-ವನ್ನೇ
ಆಹಾ-ರ-ಸೇ-ವ-ನೆ-ಯಿಂದ
ಆಹಾ-ರಾ-ದಿ-ಗಳ
ಆಹುತಿ
ಆಹು-ತಿ-ಕೊಟ್ಟು
ಆಹು-ತಿ-ಯನ್ನು
ಆಹು-ತಿ-ಯಾಗಿ
ಆಹ್
ಆಹ್ವಾ-ನ-ಕ್ಕೊಪ್ಪಿ
ಆಹ್ವಾ-ನದ
ಆಹ್ವಾ-ನ-ವನ್ನು
ಆಹ್ವಾ-ನಿ-ತ-ರಿಗೆ
ಆಹ್ವಾ-ನಿ-ತ-ರಿ-ಗೆಲ್ಲ
ಆಹ್ವಾ-ನಿ-ತ-ರೆ-ದುರು
ಆಹ್ವಾ-ನಿ-ಸ-ಲಾ-ಯಿತು
ಆಹ್ವಾ-ನಿಸಿ
ಆಹ್ವಾ-ನಿ-ಸಿದ
ಆಹ್ವಾ-ನಿ-ಸಿ-ದರು
ಆಹ್ವಾ-ನಿ-ಸಿದ್ದ
ಆ್ಹ
ಇಂಗಿತ
ಇಂಗಿ-ತ-ಇಚ್ಛೆ
ಇಂಗಿ-ತ-ವನ್ನು
ಇಂಗ್ಲಿ-ಷನ್ನು
ಇಂಗ್ಲಿ-ಷಿನ
ಇಂಗ್ಲಿ-ಷಿ-ನಲ್ಲಿ
ಇಂಗ್ಲಿ-ಷಿ-ನಲ್ಲೇ
ಇಂಗ್ಲಿಷ್
ಇಂಗ್ಲಿ-ಷ್-ಸಂ-ಸ್ಕೃತ
ಇಂಗ್ಲೆಂ-ಡಿಗೆ
ಇಂಗ್ಲೆಂ-ಡಿನ
ಇಂಚ-ರ-ಗಳಿಂದ
ಇಂತಹ
ಇಂತ-ಹ-ಘೋರ
ಇಂತ-ಹದೇ
ಇಂತ-ಹ-ವ-ರಿಗೆ
ಇಂತ-ಹ-ವರು
ಇಂತಿಂಥ
ಇಂತಿಷ್ಟು
ಇಂತು
ಇಂಥ
ಇಂಥದ
ಇಂಥ-ದನ್ನು
ಇಂಥದು
ಇಂಥದೇ
ಇಂಥ-ವ-ನನ್ನು
ಇಂಥ-ವ-ನಾ-ಡುವ
ಇಂಥ-ವ-ನಿಗೆ
ಇಂಥ-ವನು
ಇಂಥವು
ಇಂಥಾ
ಇಂದಲ್ಲ
ಇಂದಿ-ಗಿಷ್ಟೇ
ಇಂದಿಗೂ
ಇಂದಿನ
ಇಂದು
ಇಂದು-ವಂ-ದ್ಯ-ನಾದ
ಇಂದೇ
ಇಂದೇಕೆ
ಇಂದ್ರಿಯ
ಇಂದ್ರಿ-ಯ-ಗಳ
ಇಂದ್ರಿ-ಯ-ಗಳನ್ನು
ಇಂದ್ರಿ-ಯ-ಗ-ಳಿಗೂ
ಇಂದ್ರಿ-ಯ-ಗೋ-ಚ-ರ-ವಾದ
ಇಂದ್ರಿ-ಯ-ಗ್ರಾ-ಹ್ಯ-ವ-ಲ್ಲದ
ಇಂದ್ರಿ-ಯ-ಜೀ-ವ-ನಕ್ಕೇ
ಇಂದ್ರಿ-ಯ-ಜೀ-ವ-ನದ
ಇಂದ್ರಿ-ಯ-ನಿ-ಗ್ರಹ
ಇಂದ್ರಿ-ಯ-ಭೋ-ಗವೇ
ಇಂದ್ರಿ-ಯ-ರಾ-ಜ್ಯ-ವನ್ನು
ಇಂದ್ರಿ-ಯ-ಸು-ಖ-ಗಳನ್ನು
ಇಂದ್ರಿ-ಯ-ಸು-ಖ-ಗ-ಳಿಗೆ
ಇಂದ್ರಿ-ಯ-ಸು-ಖ-ವೆಂಬ
ಇಂದ್ರಿ-ಯಾ-ನು-ಭ-ವ-ಗ-ಳಿಗೆ
ಇಕ್ಕೆ-ಲ-ಗ-ಳ-ಲ್ಲಿನ
ಇಕ್ಕೆ-ಲ-ಗ-ಳಲ್ಲೂ
ಇಗೊ
ಇಗೋ
ಇಚ್ಛಾ
ಇಚ್ಛಾ-ನು-ಸಾರ
ಇಚ್ಛಾ-ಮ-ರ-ಣಿ-ಯಾ-ವಾಗ
ಇಚ್ಛಾ-ಮಾತ್ರ
ಇಚ್ಛಾ-ಮಾ-ತ್ರ-ದಿಂದ
ಇಚ್ಛಾ-ಶಕ್ತಿ
ಇಚ್ಛಿ-ಸಿದ
ಇಚ್ಛಿ-ಸಿ-ದರು
ಇಚ್ಛಿ-ಸಿ-ದರೆ
ಇಚ್ಛಿ-ಸಿ-ದೊ-ಡನೆ
ಇಚ್ಛಿ-ಸಿ-ದ್ದಂತೆ
ಇಚ್ಛಿ-ಸಿ-ದ್ದರು
ಇಚ್ಛಿ-ಸು-ತ್ತಾರೆ
ಇಚ್ಛೆ
ಇಚ್ಛೆ-ಅ-ವಳ
ಇಚ್ಛೆಗೆ
ಇಚ್ಛೆ-ಯಂತೆ
ಇಚ್ಛೆ-ಯಂ-ತೆಯೇ
ಇಚ್ಛೆ-ಯನ್ನು
ಇಚ್ಛೆ-ಯನ್ನೂ
ಇಚ್ಛೆ-ಯ-ಲ್ಲದೆ
ಇಚ್ಛೆ-ಯಾ-ಗಿ-ತ್ತೆಂ-ಬು-ದ-ರಲ್ಲಿ
ಇಚ್ಛೆ-ಯಾ-ಗಿ-ತ್ತೇನೋ
ಇಚ್ಛೆ-ಯಾ-ಗು-ತ್ತಿತ್ತು
ಇಚ್ಛೆ-ಯಾ-ಗು-ತ್ತಿದೆ
ಇಚ್ಛೆ-ಯಾ-ದ್ದ-ರಿಂದ
ಇಚ್ಛೆ-ಯಿಂದ
ಇಚ್ಛೆ-ಯಿಂ-ದಲ್ಲ
ಇಚ್ಛೆ-ಯಿತ್ತು
ಇಚ್ಛೆ-ಯಿ-ದ್ದರೆ
ಇಚ್ಛೆಯೂ
ಇಚ್ಛೆಯೇ
ಇಚ್ಛೆ-ಯೊಂ-ದಿ-ದೆ-ಯೆಲ್ಲ
ಇಟ್ಟಂ-ತಹ
ಇಟ್ಟಿದ್ದ
ಇಟ್ಟಿ-ರುತ್ತಿ
ಇಟ್ಟಿ-ರು-ವುದು
ಇಟ್ಟು
ಇಟ್ಟುಕೊ
ಇಟ್ಟು-ಕೊಂ-ಡರೂ
ಇಟ್ಟು-ಕೊಂ-ಡಿ-ರು-ತ್ತಿದ್ದೆ
ಇಟ್ಟು-ಕೊಂ-ಡಿಲ್ಲ
ಇಟ್ಟು-ಕೊಂಡು
ಇಟ್ಟು-ಕೊ-ಳ್ಳ-ಲಾ-ರ-ದ-ವನು
ಇಟ್ಟು-ಕೊ-ಳ್ಳಲಿ
ಇಟ್ಟು-ಕೊ-ಳ್ಳಲು
ಇಟ್ಟು-ಕೊ-ಳ್ಳು-ತ್ತಿದ್ದ
ಇಟ್ಟು-ಕೊ-ಳ್ಳು-ವ-ವ-ನಲ್ಲ
ಇಡಲು
ಇಡಾ-ಪಿಂ-ಗಳಾ
ಇಡಿ
ಇಡಿಯ
ಇಡೀ
ಇಡು-ತ್ತಿದ್ದ
ಇಡು-ವು-ದೆಂಬ
ಇತರ
ಇತ-ರರ
ಇತ-ರ-ರಂ-ತಲ್ಲ
ಇತ-ರ-ರಂ-ತ-ಲ್ಲದೆ
ಇತ-ರ-ರದು
ಇತ-ರ-ರನ್ನು
ಇತ-ರ-ರನ್ನೂ
ಇತ-ರ-ರಾ-ದರೂ
ಇತ-ರ-ರಿಂದ
ಇತ-ರ-ರಿಗೂ
ಇತ-ರ-ರಿಗೆ
ಇತ-ರರು
ಇತ-ರರೂ
ಇತ-ರ-ರೆಲ್ಲ
ಇತ-ರ-ರೆ-ಲ್ಲರ
ಇತ-ರ-ರೊಂ-ದಿಗೆ
ಇತ-ರೆ-ಡೆಗೆ
ಇತ-ರೆಲ್ಲ
ಇತ-ರೆ-ಲ್ಲ-ರಿ-ಗಿಂತ
ಇತಿ-ಮಿತಿ
ಇತಿ-ಮಿ-ತಿ-ಗ-ಳನ್ನೇ
ಇತಿ-ಮಿ-ತಿ-ಗ-ಳಿ-ರು-ತ್ತವೆ
ಇತಿ-ಹಾಸ
ಇತಿ-ಹಾ-ಸ-ಕಾ-ರರೂ
ಇತಿ-ಹಾ-ಸ-ಗ್ರಂ-ಥ-ವನ್ನು
ಇತಿ-ಹಾ-ಸದ
ಇತಿ-ಹಾ-ಸ-ದಲ್ಲಿ
ಇತಿ-ಹಾ-ಸ-ದಲ್ಲೇ
ಇತಿ-ಹಾ-ಸ-ದ-ಲ್ಲೊಂದು
ಇತಿ-ಹಾ-ಸ-ವನ್ನು
ಇತಿ-ಹಾ-ಸ-ವನ್ನೂ
ಇತಿ-ಹಾ-ಸವೂ
ಇತ್ತ
ಇತ್ತ-ಕಡೆ
ಇತ್ತಾ-ದರೂ
ಇತ್ತು
ಇತ್ಯರ್ಥ
ಇತ್ಯ-ರ್ಥಕ್ಕೆ
ಇತ್ಯಾದಿ
ಇದ-ಕ್ಕಾಗಿ
ಇದ-ಕ್ಕಿಂತ
ಇದಕ್ಕೂ
ಇದಕ್ಕೆ
ಇದ-ಕ್ಕೆಲ್ಲ
ಇದ-ಕ್ಕೇನೂ
ಇದ-ಕ್ಕೊಂದು
ಇದ-ಕ್ಕೊಪ್ಪಿ
ಇದ-ನ-ರಿತೆ
ಇದ-ನ್ನ-ರಿತು
ಇದನ್ನು
ಇದನ್ನೂ
ಇದ-ನ್ನೆಲ್ಲ
ಇದನ್ನೇ
ಇದರ
ಇದ-ರಲ್ಲಿ
ಇದ-ರ-ಲ್ಲೇ-ನಾ-ದರೂ
ಇದ-ರಿಂದ
ಇದ-ರಿಂ-ದಾಗಿ
ಇದ-ರಿಂ-ದಾ-ಚೆಗೆ
ಇದ-ರಿಂ-ದೆಲ್ಲ
ಇದ-ರೊಂ-ದಿಗೆ
ಇದ-ರೊ-ಳ-ಗಿ-ನಿಂ-ದಲೇ
ಇದ-ಲ್ಲದೆ
ಇದಾಗಿ
ಇದಾದ
ಇದಾ-ದ-ಮೇ-ಲೆಯೇ
ಇದಾ-ವು-ದಕ್ಕೂ
ಇದಾ-ವುದೂ
ಇದಿ-ರಲ್ಲಿ
ಇದಿ-ರಲ್ಲೇ
ಇದಿ-ರಾಗಿ
ಇದಿ-ರಾ-ಗಿ-ರುವ
ಇದಿರು
ಇದಿ-ರು-ನೋ-ಡು-ತ್ತಿ-ದ್ದರು
ಇದಿ-ರು-ನೋ-ಡು-ತ್ತಿ-ದ್ದಾರೆ
ಇದೀಗ
ಇದು
ಇದು-ಕ-ಷ್ಟ-ಗಳನ್ನು
ಇದು-ನಾವು
ಇದು-ವ-ರೆಗೂ
ಇದುವೆ
ಇದು-ಶ್ರೀ-ರಾ-ಮ-ಕೃ-ಷ್ಣರು
ಇದೂ
ಇದೆ
ಇದೆಂ-ತಹ
ಇದೆಂಥ
ಇದೆ-ಎಂಬ
ಇದೆ-ಮೊ-ದ-ಲಿನ
ಇದೆ-ಯಲ್ಲ
ಇದೆ-ಯ-ಲ್ಲವೆ
ಇದೆಯೆ
ಇದೆ-ಯೆಂದು
ಇದೆ-ಯೆಂದೇ
ಇದೆ-ಯೆ-ನ್ನು-ವಾಗ
ಇದೆ-ಯೇನು
ಇದೆಯೋ
ಇದೆಲ್ಲ
ಇದೆ-ಲ್ಲ-ಕ್ಕಿಂತ
ಇದೆ-ಲ್ಲ-ದರ
ಇದೆ-ಲ್ಲ-ದ-ರಿಂ-ದಾಗಿ
ಇದೆಷ್ಟು
ಇದೇ
ಇದೇ-ಕಿರ
ಇದೇ-ನಪ್ಪ
ಇದೇ-ನಾ-ಶ್ಚರ್ಯ
ಇದೇ-ನಿದು
ಇದೇನು
ಇದೇ-ನೆ-ನ್ನು-ತ್ತಿ-ದ್ದೀರಿ
ಇದೇನೋ
ಇದೇ-ಸಾಂ-ತ-ದಿಂದ
ಇದೊಂದು
ಇದೊಂದೂ
ಇದೊಂ-ದೆ-ರಡು
ಇದೊಂದೇ
ಇದೋ
ಇದ್ದ
ಇದ್ದಂ-ತಿತ್ತು
ಇದ್ದಂ-ತಿದೆ
ಇದ್ದಂ-ತಿ-ರ-ಲಿಲ್ಲ
ಇದ್ದಂ-ತಿಲ್ಲ
ಇದ್ದಂ-ತಿ-ಲ್ಲ-ವಲ್ಲಾ
ಇದ್ದಂತೆ
ಇದ್ದಕ್ಕಿ
ಇದ್ದ-ಕ್ಕಿ-ದಂತೆ
ಇದ್ದ-ಕ್ಕಿ-ದ್ದಂತೆ
ಇದ್ದ-ಕ್ಕಿ-ದ್ದ-ಹಾಗೆ
ಇದ್ದದ್ದು
ಇದ್ದದ್ದೇ
ಇದ್ದ-ನಾ-ದರೂ
ಇದ್ದ-ರಾ-ದರೂ
ಇದ್ದರು
ಇದ್ದ-ರು-ಭ-ಕ್ತಾ-ನು-ಕಂಪೆ
ಇದ್ದರೂ
ಇದ್ದರೆ
ಇದ್ದ-ವನೇ
ಇದ್ದ-ವ-ರಿಗೆ
ಇದ್ದ-ವ-ರಿ-ಗೆಲ್ಲ
ಇದ್ದಾಗ
ಇದ್ದಾ-ನಲ್ಲ
ಇದ್ದಾನೆ
ಇದ್ದಾರೆ
ಇದ್ದಿ-ತಾ-ದರೂ
ಇದ್ದಿತು
ಇದ್ದಿ-ದ್ದರೆ
ಇದ್ದಿ-ರ-ಬ-ಹುದು
ಇದ್ದಿ-ರ-ಬೇಕು
ಇದ್ದೀಯ
ಇದ್ದೀ-ಯಲ್ಲ
ಇದ್ದು
ಇದ್ದು-ಕೊಂಡು
ಇದ್ದು-ದ-ರಲ್ಲಿ
ಇದ್ದು-ದ-ರಿಂದ
ಇದ್ದು-ದೆಂ-ದರೆ
ಇದ್ದು-ದೆಲ್ಲ
ಇದ್ದು-ಬಿಟ್ಟ
ಇದ್ದು-ಬಿ-ಟ್ಟಾನು
ಇದ್ದು-ಬಿಟ್ಟೆ
ಇದ್ದು-ಬಿ-ಡ-ಬೇ-ಕೆ-ನ್ನುವ
ಇದ್ದು-ಬಿ-ಡು-ತ್ತಿದ್ದ
ಇದ್ದು-ಬಿ-ಡು-ತ್ತಿ-ದ್ದರು
ಇದ್ದು-ಬಿ-ಡು-ವುದೂ
ಇದ್ದುವು
ಇದ್ದು-ವೆ-ನ್ನ-ಬ-ಹುದು
ಇದ್ದೂ
ಇದ್ದೆ
ಇದ್ದೆನೋ
ಇದ್ದೆ-ನೋ-ಡೋಣ
ಇದ್ದೇ
ಇದ್ದೇನೆ
ಇದ್ದೇವೆ
ಇದ್ಯಾವ
ಇನಿತೆ
ಇನ್ನ-ವನು
ಇನ್ನ-ವರು
ಇನ್ನಷ್ಟು
ಇನ್ನಾ-ರದೂ
ಇನ್ನಾ-ರಲ್ಲೂ
ಇನ್ನಾರೂ
ಇನ್ನಾವ
ಇನ್ನಾ-ವುದನ್ನು
ಇನ್ನಾ-ವುದೂ
ಇನ್ನಾ-ವುದೇ
ಇನ್ನಿ-ತರ
ಇನ್ನಿ-ತ-ರರ
ಇನ್ನಿ-ತ-ರರು
ಇನ್ನಿ-ತ-ರರೂ
ಇನ್ನಿ-ಬ್ಬರು
ಇನ್ನಿಲ್ಲ
ಇನ್ನಿ-ಲ್ಲ-ದಷ್ಟು
ಇನ್ನು
ಇನ್ನು-ಮುಂದೆ
ಇನ್ನು-ಮೇಲೆ
ಇನ್ನು-ಳಿದ
ಇನ್ನೂ
ಇನ್ನೂ-ಬ್ಬಳು
ಇನ್ನೆಂ-ತಹ
ಇನ್ನೆಂ-ದಿಗೂ
ಇನ್ನೆ-ರಡೇ
ಇನ್ನೆಲ್ಲಿ
ಇನ್ನೆ-ಲ್ಲಿಗೆ
ಇನ್ನೆ-ಲ್ಲಿಯೂ
ಇನ್ನೆಲ್ಲೂ
ಇನ್ನೆ-ಲ್ಲೆಲ್ಲಿ
ಇನ್ನೆ-ಷ್ಟೆಷ್ಟು
ಇನ್ನೆ-ಷ್ಟೆಷ್ಟೋ
ಇನ್ನೆಷ್ಟೋ
ಇನ್ನೇ-ನಾ-ದರೂ
ಇನ್ನೇ-ನಿದೆ
ಇನ್ನೇನು
ಇನ್ನೇನೂ
ಇನ್ನೇನೇ
ಇನ್ನೇ-ನೇನು
ಇನ್ನೇನೋ
ಇನ್ನೇ-ನ್ಮಾ-ಡತ್ತೆ
ಇನ್ನೈದು
ಇನ್ನೊಂ-ದ-ರತ್ತ
ಇನ್ನೊಂ-ದಿಲ್ಲ
ಇನ್ನೊಂ-ದಿ-ಲ್ಲ-ಇದು
ಇನ್ನೊಂದು
ಇನ್ನೊಂ-ದು-ಕ-ರುಣೆ
ಇನ್ನೊಂ-ದು-ಧ್ಯಾ-ನಸ್ಥ
ಇನ್ನೊಂದೂ
ಇನ್ನೊಂ-ದೆ-ಡೆಗೆ
ಇನ್ನೊಂ-ದೆ-ರ-ಡನ್ನು
ಇನ್ನೊಬ್ಬ
ಇನ್ನೊ-ಬ್ಬ-ನಿಂದ
ಇನ್ನೊ-ಬ್ಬನು
ಇನ್ನೊ-ಬ್ಬ-ನೆಂದ
ಇನ್ನೊ-ಬ್ಬರ
ಇನ್ನೊ-ಬ್ಬ-ರಲ್ಲಿ
ಇನ್ನೊ-ಬ್ಬ-ರಿಲ್ಲ
ಇನ್ನೊ-ಬ್ಬರು
ಇನ್ನೊಮ್ಮೆ
ಇನ್ಫ್ಲೂ-ಯೆಂಜಾ
ಇನ್ಯಾರ
ಇನ್ಯಾವ
ಇಪ್ಪ
ಇಪ್ಪ-ತ್ತ-ನಾಲ್ಕು
ಇಪ್ಪತ್ತು
ಇಪ್ಪ-ತ್ತೆಂಟು
ಇಪ್ಪ-ತ್ತೈ-ದ-ನೆಯ
ಇಪ್ಪ-ತ್ನಾಲ್ಕು
ಇಬ್ಬಂ-ದಿಯ
ಇಬ್ಬರ
ಇಬ್ಬ-ರನ್ನೂ
ಇಬ್ಬ-ರಿಗೂ
ಇಬ್ಬರು
ಇಬ್ಬರೂ
ಇಬ್ಭಾಗ
ಇಮಾನಿ
ಇಮ್ಮ-ಡಿ-ಯಾ-ದುವು
ಇರ
ಇರ-ತೊ-ಡ-ಗಿ-ದರು
ಇರ-ಬ-ಲ್ಲರು
ಇರ-ಬ-ಹುದು
ಇರ-ಬ-ಹು-ದೆಂದು
ಇರ-ಬ-ಹು-ದೇನೋ
ಇರ-ಬಾ-ರದು
ಇರ-ಬೇ-ಕಂದು
ಇರ-ಬೇ-ಕ-ಲ್ಲವೆ
ಇರ-ಬೇ-ಕಾ-ಗಿ-ತ್ತಾ-ದ್ದ-ರಿಂದ
ಇರ-ಬೇ-ಕಾ-ಗಿತ್ತು
ಇರ-ಬೇ-ಕಾ-ಗಿದೆ
ಇರ-ಬೇ-ಕಾ-ಗು-ತ್ತದೆ
ಇರ-ಬೇ-ಕಾದ
ಇರ-ಬೇ-ಕಾ-ದದ್ದು
ಇರ-ಬೇ-ಕಾ-ದ-ವನು
ಇರ-ಬೇ-ಕಾ-ಯಿತು
ಇರ-ಬೇಕು
ಇರ-ಬೇ-ಕು-ಇ-ಪ್ಪ-ತ್ತ-ನಾಲ್ಕು
ಇರ-ಲಾ-ರದು
ಇರ-ಲಾ-ರರು
ಇರಲಿ
ಇರ-ಲಿಲ್ಲ
ಇರ-ಲಿ-ಲ್ಲ-ವೆಂ-ದಲ್ಲ
ಇರ-ಲಿ-ಲ್ಲ-ವೆಂ-ದೇನೂ
ಇರಲು
ಇರಲೇ
ಇರ-ಲೇ-ಬೇ-ಕಾದ
ಇರ-ಲೇ-ಬೇಕು
ಇರಿದು
ಇರಿ-ಸ-ಲಾ-ಗಿತ್ತು
ಇರಿ-ಸಿ-ಕೊಂಡು
ಇರು
ಇರು-ತಿ-ರುವೆ
ಇರು-ತ್ತದೆ
ಇರು-ತ್ತಾನೆ
ಇರು-ತ್ತಿತ್ತು
ಇರು-ತ್ತಿದ್ದ
ಇರು-ತ್ತಿ-ದ್ದರು
ಇರು-ತ್ತಿ-ದ್ದ-ವನು
ಇರು-ತ್ತಿ-ದ್ದ-ವರು
ಇರು-ತ್ತಿ-ದ್ದುದು
ಇರು-ತ್ತಿ-ದ್ದುವು
ಇರು-ತ್ತಿದ್ದೆ
ಇರು-ತ್ತಿ-ರ-ಲಿಲ್ಲ
ಇರು-ಳಿನಾ
ಇರುಳೂ
ಇರುವ
ಇರು-ವಂ-ತಹ
ಇರು-ವಂ-ತಾ-ಗಿ-ಬಿ-ಟ್ಟಿತ್ತು
ಇರು-ವಂ-ತಾ-ಗು-ತ್ತದೆ
ಇರು-ವಂ-ತಾ-ದರೆ
ಇರು-ವಂತೆ
ಇರು-ವಂ-ಥದು
ಇರು-ವಂ-ಥದೋ
ಇರು-ವ-ರೆಂದೇ
ಇರು-ವ-ಳಾ-ದರೆ
ಇರು-ವವ
ಇರು-ವ-ವ-ನನ್ನು
ಇರು-ವ-ವ-ನೊ-ಬ್ಬನೇ
ಇರು-ವ-ವ-ರೆಗೂ
ಇರು-ವು-ದಕ್ಕೆ
ಇರು-ವುದನ್ನು
ಇರು-ವು-ದ-ರಿಂದ
ಇರು-ವು-ದಾ-ದರೆ
ಇರು-ವು-ದಾ-ದಲ್ಲಿ
ಇರು-ವು-ದಿಲ್ಲ
ಇರು-ವುದು
ಇರು-ವು-ದೆಲ್ಲ
ಇರು-ವು-ದೆಲ್ಲಿ
ಇರು-ವುದೇ
ಇರು-ವುದೊ
ಇರು-ವು-ದೊಂದೇ
ಇರುವೆ
ಇರು-ವೆಯ
ಇರುಸು
ಇಲಿ-ಗಳ
ಇಲಿ-ಗ-ಳು-ಇ-ವೆಲ್ಲ
ಇಲ್ಲ
ಇಲ್ಲ-ಎಂ-ದು-ಬಿಟ್ಟ
ಇಲ್ಲದ
ಇಲ್ಲ-ದಂ-ತಹ
ಇಲ್ಲ-ದಂ-ತಾ-ಯಿತು
ಇಲ್ಲ-ದ-ವ-ನಾ-ಗಿ-ಬಿ-ಡು-ತ್ತಾನೆ
ಇಲ್ಲ-ದ-ವ-ರಂತೆ
ಇಲ್ಲ-ದ-ಹಾಗೆ
ಇಲ್ಲ-ದಿ-ದ್ದ-ರಾ-ಯಿತು
ಇಲ್ಲ-ದಿ-ದ್ದರೂ
ಇಲ್ಲ-ದಿ-ದ್ದರೆ
ಇಲ್ಲ-ದಿ-ರಲಿ
ಇಲ್ಲ-ದಿ-ರ-ಲಿಲ್ಲ
ಇಲ್ಲ-ದಿ-ರು-ವಾಗ
ಇಲ್ಲದು
ಇಲ್ಲದೆ
ಇಲ್ಲ-ದೆ-ಹೋ-ದರೆ
ಇಲ್ಲ-ವಲ್ಲ
ಇಲ್ಲ-ವಲ್ಲಾ
ಇಲ್ಲ-ವಾಗಿ
ಇಲ್ಲ-ವಾ-ಗಿ-ಬಿ-ಟ್ಟಿದೆ
ಇಲ್ಲ-ವಾ-ದರೆ
ಇಲ್ಲ-ವಾ-ದ್ದ-ರಿಂದ
ಇಲ್ಲ-ವಾ-ಯಿತು
ಇಲ್ಲವೆ
ಇಲ್ಲ-ವೆಂ-ದಲ್ಲ
ಇಲ್ಲ-ವೆಂ-ದಾ-ಳೆಯೆ
ಇಲ್ಲ-ವೆಂ-ಬಂತೆ
ಇಲ್ಲ-ವೆಂಬು
ಇಲ್ಲ-ವೆಂ-ಬು-ದನ್ನು
ಇಲ್ಲ-ವೆನ್ನ
ಇಲ್ಲವೇ
ಇಲ್ಲ-ವೇನೋ
ಇಲ್ಲವೋ
ಇಲ್ಲಿ
ಇಲ್ಲಿಂದ
ಇಲ್ಲಿ-ಗಿಂತ
ಇಲ್ಲಿಗೆ
ಇಲ್ಲಿಗೇ
ಇಲ್ಲಿ-ಗೇಕೆ
ಇಲ್ಲಿ-ಗೊಮ್ಮೆ
ಇಲ್ಲಿ-ದ್ದಾರೆ
ಇಲ್ಲಿ-ದ್ದೀಯ
ಇಲ್ಲಿನ
ಇಲ್ಲಿಯ
ಇಲ್ಲಿ-ಯ-ವ-ರೆಗೂ
ಇಲ್ಲಿ-ಯ-ವ-ರೆಗೆ
ಇಲ್ಲಿಯೂ
ಇಲ್ಲಿಯೇ
ಇಲ್ಲಿ-ರಲು
ಇಲ್ಲಿ-ರು-ವುದು
ಇಲ್ಲಿ-ರು-ವುದೇ
ಇಲ್ಲಿ-ಲ್ಲದ
ಇಲ್ಲೀಗ
ಇಲ್ಲೆಲ್ಲ
ಇಲ್ಲೇ
ಇಲ್ಲೇ-ತ-ಮ್ಮ-ತಮ್ಮ
ಇಲ್ಲೊಂದು
ಇಲ್ಲೊಬ್ಬ
ಇಳಿ
ಇಳಿ-ದರೆ
ಇಳಿ-ದಾಗ
ಇಳಿದು
ಇಳಿ-ದು-ಕೊಂ-ಡರು
ಇಳಿ-ದು-ಕೊಂಡು
ಇಳಿ-ದು-ಬಂ-ದಾಗ
ಇಳಿ-ದು-ಬಂದು
ಇಳಿ-ದು-ಬ-ರ-ಬ-ಲ್ಲರು
ಇಳಿ-ದು-ಬ-ರುವ
ಇಳಿ-ದು-ಹೋಗಿ
ಇಳಿ-ದು-ಹೋ-ದ-ದ್ದನ್ನು
ಇಳಿ-ಬಿ-ದ್ದಿತ್ತು
ಇಳಿ-ಯಲು
ಇಳಿ-ಯಲೇ
ಇಳಿ-ಯಿರಿ
ಇಳಿ-ಯುತ್ತ
ಇಳಿ-ಯು-ತ್ತವೆ
ಇಳಿ-ಯು-ತ್ತಿದೆ
ಇಳಿ-ಯುವ
ಇಳಿ-ಯು-ವುದನ್ನು
ಇಳಿಸಿ
ಇಳೆಗೆ
ಇವ-ಕ್ಕೆಲ್ಲ
ಇವತ್ತು
ಇವನ
ಇವ-ನಂ-ತಹ
ಇವ-ನನ್ನು
ಇವ-ನ-ಲ್ಲೊಂದು
ಇವ-ನಾರೋ
ಇವ-ನಿಂದ
ಇವ-ನಿಗೆ
ಇವ-ನಿನ್ನು
ಇವ-ನೀಗ
ಇವನು
ಇವನೂ
ಇವ-ನೆಯೇ
ಇವ-ನೆಲ್ಲೋ
ಇವನೇ
ಇವ-ನ್ನ-ರಿ-ಯದೆ
ಇವ-ನ್ನೆಲ್ಲ
ಇವರ
ಇವ-ರದ್ದು
ಇವ-ರನ್ನು
ಇವ-ರ-ನ್ನೆಲ್ಲ
ಇವ-ರ-ಲ್ಲದೆ
ಇವ-ರ-ಲ್ಲ-ನೇ-ಕರು
ಇವ-ರಲ್ಲಿ
ಇವ-ರಾಗಿ
ಇವ-ರಿಂದ
ಇವ-ರಿಂ-ದಲೇ
ಇವ-ರಿ-ಗಾಗಿ
ಇವ-ರಿ-ಗಿಂತ
ಇವ-ರಿಗೆ
ಇವ-ರಿ-ಗೆಲ್ಲ
ಇವ-ರಿಗೇ
ಇವ-ರಿ-ಗೇನು
ಇವ-ರಿ-ಬ್ಬರ
ಇವ-ರಿ-ಬ್ಬ-ರಿಗೂ
ಇವ-ರಿ-ಬ್ಬ-ರಿಗೆ
ಇವ-ರಿ-ಬ್ಬರು
ಇವ-ರಿ-ಬ್ಬರೂ
ಇವರು
ಇವ-ರು-ಗಳನ್ನು
ಇವ-ರು-ಗಳನ್ನೂ
ಇವ-ರು-ಗಳು
ಇವರೂ
ಇವರೆಲ್ಲ
ಇವರೆ-ಲ್ಲರ
ಇವರೆ-ಲ್ಲ-ರಿಂ-ದಲೂ
ಇವರೆ-ಲ್ಲ-ರಿಗೂ
ಇವರೇ
ಇವ-ರೇ-ನಾ-ದರೂ
ಇವ-ರೇನು
ಇವ-ರೇನೋ
ಇವ-ರೊಂ-ದಿಗೆ
ಇವ-ರೊಂ-ದಿಗೇ
ಇವ-ರೊಬ್ಬ
ಇವ-ರೊ-ಬ್ಬರು
ಇವ-ರೊ-ಬ್ಬರೇ
ಇವ-ರೊ-ಳ-ಗಿನ
ಇವ-ಲ್ಲದೆ
ಇವಳ
ಇವಳು
ಇವಿ-ಷ್ಟನ್ನು
ಇವಿಷ್ಟೂ
ಇವು
ಇವು-ಗಳ
ಇವು-ಗಳನ್ನು
ಇವು-ಗಳನ್ನೆಲ್ಲ
ಇವು-ಗಳಲ್ಲಿ
ಇವು-ಗ-ಳಲ್ಲೇ
ಇವು-ಗಳಿಂದ
ಇವು-ಗಳಿಂದಾ
ಇವು-ಗ-ಳಿಂ-ದಾಗಿ
ಇವು-ಗ-ಳಿಂ-ದೆಲ್ಲ
ಇವು-ಗ-ಳಿಗೆ
ಇವು-ಗ-ಳೆಲ್ಲ
ಇವೆ-ಯಲ್ಲ
ಇವೆ-ರ-ಡರ
ಇವೆ-ರಡು
ಇವೆ-ರಡೂ
ಇವೆಲ್ಲ
ಇವೆ-ಲ್ಲ-ಕ್ಕಿಂತ
ಇವೆ-ಲ್ಲಕ್ಕೂ
ಇವೆ-ಲ್ಲ-ದರ
ಇವೆ-ಲ್ಲ-ವನ್ನೂ
ಇವೆ-ಲ್ಲ-ವು-ಗಳ
ಇವೆ-ಲ್ಲ-ವು-ಗ-ಳಿಂ-ದಾಗಿ
ಇವೆ-ಲ್ಲವೂ
ಇವೊತ್ತು
ಇಷ್ಟ
ಇಷ್ಟಕ್ಕೆ
ಇಷ್ಟ-ಕ್ಕೆಲ್ಲ
ಇಷ್ಟಕ್ಕೇ
ಇಷ್ಟ-ದೇ-ವ-ತೆ-ಯಾದ
ಇಷ್ಟ-ಪ-ಟ್ಟಿದ್ದೆ
ಇಷ್ಟ-ಪಡ
ಇಷ್ಟ-ಪ-ಡದೆ
ಇಷ್ಟ-ಪ-ಡ-ಲಿಲ್ಲ
ಇಷ್ಟ-ಪಡು
ಇಷ್ಟ-ಪ-ಡು-ತ್ತಿದ್ದ
ಇಷ್ಟ-ಪ-ಡು-ತ್ತಿ-ರ-ಲಿಲ್ಲ
ಇಷ್ಟ-ಪ-ಡು-ತ್ತೀರಿ
ಇಷ್ಟ-ಪ-ಡು-ವು-ದಿಲ್ಲ
ಇಷ್ಟ-ಪ-ಡು-ವು-ದಿ-ಲ್ಲವೆ
ಇಷ್ಟ-ಬಂ-ದಂತೆ
ಇಷ್ಟ-ಮಂತ್ರ
ಇಷ್ಟ-ಮಿ-ತ್ರ-ರು-ಇ-ಷ್ಟಕ್ಕೆ
ಇಷ್ಟರ
ಇಷ್ಟ-ರ-ಮೇಲೂ
ಇಷ್ಟ-ರಲ್ಲೇ
ಇಷ್ಟ-ಲ್ಲದೆ
ಇಷ್ಟ-ವಾ-ಗದೆ
ಇಷ್ಟ-ವಾ-ಗ-ಲಿಲ್ಲ
ಇಷ್ಟ-ವಾ-ಗು-ತ್ತ-ದೆಯೋ
ಇಷ್ಟ-ವಾ-ಯಿತು
ಇಷ್ಟ-ವಿ-ರ-ಲಿಲ್ಲ
ಇಷ್ಟ-ವಿಲ್ಲ
ಇಷ್ಟವೇ
ಇಷ್ಟ-ಸ-ರಿ-ಯಾ-ಗಿಯೇ
ಇಷ್ಟಾ-ದರೂ
ಇಷ್ಟಿ-ದ್ದರೆ
ಇಷ್ಟಿದ್ದು
ಇಷ್ಟು
ಇಷ್ಟು-ಕಾಲ
ಇಷ್ಟೆಲ್ಲ
ಇಷ್ಟೇ
ಇಷ್ಟೊಂದು
ಇಸವಿ
ಇಸ-ವಿಯ
ಇಸ್ಲಾಂ
ಇಹ-ಜೀ-ವನ
ಇಹ-ಜೀ-ವ-ನದ
ಇಹ-ಜೀ-ವ-ನ-ವನ್ನೇ
ಇಹ-ಲೋಕ
ಇಹ-ಲೋ-ಕ-ಲೀ-ಲೆಯ
ಇಹವು
ಈ
ಈಕೆ
ಈಕ್ಷಿ-ಸುತ್ತ
ಈಗ
ಈಗಂತೂ
ಈಗ-ಲಾ-ದರೂ
ಈಗಲೂ
ಈಗಲೇ
ಈಗಲೋ
ಈಗಷ್ಟೇ
ಈಗಾ-ಗಲೇ
ಈಗಿಂ-ದಲೇ
ಈಗಿನ
ಈಗೀಗ
ಈಗೇಕೆ
ಈಗೇನು
ಈಗೇನೋ
ಈಗೊಂದು
ಈಗೊಮ್ಮೆ
ಈಚೀ-ಚೆ-ಗಂತೂ
ಈಚೀ-ಚೆಗೆ
ಈಚೆಗೆ
ಈಜಾ-ಡಿ-ಕೊಂಡು
ಈಜು
ಈಜು-ವು-ದ-ರಲ್ಲಿ
ಈಡಾ-ಡಿದ
ಈಡಾ-ಡು-ತ್ತಾನೆ
ಈಡಾ-ಡು-ತ್ತಿದ್ದ
ಈಡಾದ
ಈಡಾ-ದರೂ
ಈಡೇ-ರ-ಬ-ಹು-ದೆಂಬ
ಈಡೇ-ರಿ-ಸ-ಬೇ-ಕಾದ
ಈಡೇ-ರಿ-ಸ-ಲಾ-ರರೆ
ಈಡೇ-ರಿ-ಸಲು
ಈಡೇ-ರಿ-ಸಿ-ಕೊ-ಡ-ಬಲ್ಲ
ಈಡೇ-ರಿ-ಸಿ-ಕೊ-ಡಲು
ಈಡೇ-ರಿ-ಸಿದ
ಈಡೇ-ರಿ-ಸಿ-ದ್ದಾಳೆ
ಈಡೇ-ರಿ-ಸು-ವಂತೆ
ಈಡೇ-ರು-ವು-ದಿಲ್ಲ
ಈತ
ಈತನ
ಈತ-ನ-ದಾ-ಗಿತ್ತು
ಈತನೇ
ಈರುಳ್ಳಿ
ಈವ-ರೆಗೂ
ಈವ-ರೆಗೆ
ಈವ್
ಈಶಾ-ನ-ಚಂದ್ರ
ಈಶಾ-ನ-ನೀಗ
ಈಶಿತ್ವ
ಈಶ್ವ-ರ-ಕೋ-ಟಿ-ಗ-ಳಿಗೆ
ಈಶ್ವ-ರ-ಕೋ-ಟಿ-ಗಳು
ಈಶ್ವ-ರ-ಕೋ-ಟಿಗೆ
ಈಶ್ವ-ರ-ಚಂದ್ರ
ಈಶ್ವ-ರ-ನಿಗೂ
ಈಸಂನ್ಯಾಸಿ
ಉಂಟಾ
ಉಂಟಾ-ಗ-ಲಾ-ರದು
ಉಂಟಾ-ಗಿತ್ತು
ಉಂಟಾ-ಗಿದೆ
ಉಂಟಾ-ಗಿದ್ದ
ಉಂಟಾ-ಗಿ-ಬಿ-ಟ್ಟಿತು
ಉಂಟಾ-ಗಿ-ಬಿ-ಟ್ಟಿದೆ
ಉಂಟಾ-ಗಿ-ರುವ
ಉಂಟಾ-ಗುವ
ಉಂಟಾದ
ಉಂಟಾ-ದರೂ
ಉಂಟಾ-ದರೆ
ಉಂಟಾ-ದೀತೆ
ಉಂಟಾ-ಯಿತು
ಉಂಟು
ಉಂಟು-ಅ-ವನು
ಉಂಟು-ಮಾ-ಡ-ಬೇ-ಕಾ-ಗಿ-ರುವ
ಉಂಟು-ಮಾ-ಡ-ಲಾ-ಗದ
ಉಂಟು-ಮಾಡಿ
ಉಂಟು-ಮಾ-ಡಿ-ದರು
ಉಂಟು-ಮಾ-ಡು-ವಂತೆ
ಉಂಟೆ
ಉಕ್ಕಿ
ಉಕ್ಕಿ-ಬಂತು
ಉಕ್ಕಿ-ಬಂ-ದಂ-ತಿತ್ತು
ಉಕ್ಕಿ-ಬ-ರು-ತ್ತಿತ್ತು
ಉಕ್ಕಿ-ಬ-ರುವ
ಉಕ್ಕು-ತ್ತಿತ್ತು
ಉಕ್ಕು-ತ್ತಿ-ರು-ವುದನ್ನು
ಉಕ್ಕೇ-ರಿ-ದರೂ
ಉಗ-ಮದ
ಉಗ-ಮ-ವಾ-ಯಿತು
ಉಗ-ಮ-ಸ್ಥಾ-ನ-ಗ-ಳಾದ
ಉಗ-ಮ-ಸ್ಥಾ-ನ-ವಾ-ಗಿತ್ತು
ಉಗು-ರಿ-ನಿಂದ
ಉಗು-ಳಿ-ಬಿಟ್ಟ
ಉಗು-ಳುತ್ತ
ಉಗ್ರ-ವಾಗಿ
ಉಗ್ರ-ವಾ-ಗಿದೆ
ಉಗ್ರ-ವಾ-ಗು-ತ್ತಿದೆ
ಉಚ್ಚ
ಉಚ್ಚ-ಕಂ-ಠ-ದಿಂದ
ಉಚ್ಚ-ತರ
ಉಚ್ಚ-ರಿಸ
ಉಚ್ಚ-ರಿ-ಸ-ಬಾ-ರ-ದಂಥ
ಉಚ್ಚ-ರಿ-ಸ-ಬೇ-ಕಾ-ಯಿತು
ಉಚ್ಚ-ರಿ-ಸ-ಬೇಕು
ಉಚ್ಚ-ರಿ-ಸಿ-ದ-ರಾ-ಯಿತು
ಉಚ್ಚ-ರಿ-ಸುತ್ತ
ಉಚ್ಚ-ರಿ-ಸು-ತ್ತಿ-ದ್ದಂತೆ
ಉಚ್ಚ-ರಿ-ಸು-ತ್ತಿ-ದ್ದಂ-ತೆಯೇ
ಉಚ್ಚ-ರಿ-ಸುವ
ಉಚ್ಚ-ರಿ-ಸು-ವಂತೆ
ಉಚ್ಚ-ರಿ-ಸು-ವಾಗ
ಉಚ್ಛ್ರಾ-ಯಕ್ಕೆ
ಉಜಿರ್
ಉಜ್ಜಿ
ಉಜ್ಜ್ವಲ
ಉಜ್ಜ್ವ-ಲಿ-ಸು-ತ್ತಿ-ರುವ
ಉಜ್ವಲ
ಉಜ್ವ-ಲ-ಪ್ರ-ಖರ
ಉಜ್ವ-ಲ-ಗೊ-ಳ್ಲ-ತೊ-ಡ-ಗಿತು
ಉಜ್ವ-ಲ-ತೆ-ಯನ್ನು
ಉಜ್ವ-ಲ-ವಾಗಿ
ಉಜ್ವ-ಲ-ವಾ-ಗಿದೆ
ಉಜ್ವ-ಲ-ವಾದ
ಉಟ್ಟ-ಬ-ಟ್ಟೆ-ಯನ್ನು
ಉಡಾ-ಫೆ-ಯಿಂದ
ಉಡಾ-ಳರು
ಉಡಿ-ಗೆ-ತೊ-ಡಿಗೆ
ಉಡಿ-ಗೆ-ಯಲ್ಲಿ
ಉಡಿ-ಗೆ-ಯಾದ
ಉಡಿ-ಸಿದ್ದ
ಉಡಿ-ಸಿ-ದ್ದರು
ಉಡಿ-ಸು-ತ್ತಾರೆ
ಉಡುಗೆ
ಉಡು-ಗೆ-ಯನ್ನೆ
ಉಡು-ತ್ತಿ-ದ್ದರು
ಉಣ-ಬ-ಡಿ-ಸ-ಬೇ-ಕೆಂಬ
ಉಣ್ಣಲಿ
ಉಣ್ಣು-ತ್ತಿದ್ದೆ
ಉಣ್ಣು-ತ್ತಿ-ರು-ವ-ವನು
ಉತ್ಕಂ-ಠಿ-ತ-ರಾಗಿ
ಉತ್ಕಟ
ಉತ್ಕ-ಟತೆ
ಉತ್ಕ-ಟ-ತೆ-ಯನ್ನು
ಉತ್ಕ-ಟೇಚ್ಛೆ
ಉತ್ತಮ
ಉತ್ತ-ಮ-ಗೊ-ಳ್ಳು-ವು-ದರ
ಉತ್ತ-ಮ-ವಾ-ಗಿತ್ತು
ಉತ್ತ-ಮ-ವಾದ
ಉತ್ತರ
ಉತ್ತ-ರ-ಕ್ಕಾಗಿ
ಉತ್ತ-ರಕ್ಕೆ
ಉತ್ತ-ರ-ಕ್ರಿ-ಯೆ-ಗಳನ್ನೆಲ್ಲ
ಉತ್ತ-ರ-ಗಳನ್ನು
ಉತ್ತ-ರ-ಗಳಿಂದ
ಉತ್ತ-ರ-ದಿಂದ
ಉತ್ತ-ರ-ಭಾ-ರ-ತದ
ಉತ್ತ-ರ-ಭಾ-ರ-ತ-ದಲ್ಲಿ
ಉತ್ತ-ರ-ವನ್ನು
ಉತ್ತ-ರ-ವನ್ನೇ
ಉತ್ತ-ರ-ವಾಗಿ
ಉತ್ತ-ರ-ವಿ-ತ್ತರು
ಉತ್ತ-ರ-ವಿಲ್ಲ
ಉತ್ತ-ರ-ವಿಷ್ಟೆ
ಉತ್ತ-ರ-ವಿ-ಷ್ಟೆಈ
ಉತ್ತ-ರ-ವಿ-ಷ್ಟೆ-ಸಾ-ವಿ-ರಾರು
ಉತ್ತ-ರವೂ
ಉತ್ತ-ರ-ವೆಂ-ಥದು
ಉತ್ತ-ರವೇ
ಉತ್ತ-ರ-ವೇನು
ಉತ್ತ-ರವೋ
ಉತ್ತ-ರ-ಹೋಗು
ಉತ್ತ-ರಾ-ಧಿ-ಕಾರಿ
ಉತ್ತ-ರಿ-ಸಲು
ಉತ್ತ-ರಿಸಿ
ಉತ್ತ-ರಿ-ಸಿದ
ಉತ್ತ-ರಿ-ಸಿ-ದ-ರ-ಲ್ಲದೆ
ಉತ್ತ-ರಿ-ಸಿ-ದರು
ಉತ್ತ-ರಿ-ಸಿ-ದಾಗ
ಉತ್ತ-ರಿ-ಸಿಯಾ
ಉತ್ತ-ರಿ-ಸಿ-ಯಾನು
ಉತ್ತ-ರಿ-ಸು-ತ್ತಾನೆ
ಉತ್ತ-ರಿ-ಸು-ತ್ತಾರೆ
ಉತ್ತ-ರಿ-ಸು-ತ್ತಾ-ರೆಂದು
ಉತ್ತ-ರಿ-ಸು-ತ್ತಿ-ದ್ದರು
ಉತ್ತೀ-ರ್ಣ-ನಾ-ಗ-ಬೇಕು
ಉತ್ತೀ-ರ್ಣ-ನಾಗಿ
ಉತ್ತೀ-ರ್ಣ-ನಾದ
ಉತ್ತೀ-ರ್ಣನೂ
ಉತ್ತೀ-ರ್ಣ-ರಾ-ಗ-ಲೇ-ಬೇಕು
ಉತ್ತೀ-ರ್ಣ-ರಾ-ದ-ರೆಂದು
ಉತ್ತೀ-ರ್ಣ-ವಾ-ಯಿತೋ
ಉತ್ತುಂಗ
ಉತ್ತೇ-ಜ-ನ-ಕರ
ಉತ್ತೇ-ಜಿ-ಸುವ
ಉತ್ಪನ್ನ
ಉತ್ಪ-ನ್ನ-ವಾ-ಗು-ತ್ತದೆ
ಉತ್ಪ-ನ್ನ-ವಾ-ದಾಗ
ಉತ್ಪ-ನ್ನ-ವಾ-ಯಿತು
ಉತ್ಪಾ-ದ-ನೆ-ಗಾಗಿ
ಉತ್ಪ್ರೇ-ಕ್ಷೆ-ಯ-ಲ್ಲ-ವೆಂಬ
ಉತ್ಸವ
ಉತ್ಸ-ವಕ್ಕೆ
ಉತ್ಸ-ವ-ಗಳ
ಉತ್ಸ-ವ-ಗಳನ್ನು
ಉತ್ಸ-ವ-ಗಳಲ್ಲಿ
ಉತ್ಸ-ವದ
ಉತ್ಸಾಹ
ಉತ್ಸಾ-ಹಕ್ಕೆ
ಉತ್ಸಾ-ಹದ
ಉತ್ಸಾ-ಹ-ದಲ್ಲಿ
ಉತ್ಸಾ-ಹ-ದಿಂದ
ಉತ್ಸಾ-ಹ-ಪೂ-ಣನ್
ಉತ್ಸಾ-ಹ-ಭ-ರಿ-ತರ
ಉತ್ಸಾ-ಹ-ವನ್ನು
ಉತ್ಸಾ-ಹ-ವುಂ-ಟಾ-ಗಿದೆ
ಉತ್ಸಾ-ಹ-ವೇನೂ
ಉತ್ಸಾಹೀ
ಉತ್ಸು-ಕ-ರಾಗಿ
ಉತ್ಸು-ಕ-ರಾ-ಗಿ-ದ್ದಂತೆ
ಉತ್ಸು-ಕ-ರಾ-ದ-ದ್ದಾ-ಗಲಿ
ಉದ-ಯಿ-ಸ-ದೆ-ಹೋ-ದರೆ
ಉದ-ಯಿ-ಸಲು
ಉದ-ರ-ನಿ-ಮಿ-ತ್ತ-ವಾಗಿ
ಉದಾ
ಉದಾತ್ತ
ಉದಾ-ತ್ತ-ಗಂ-ಭೀ-ರ-ಭಾ-ವ-ವನ್ನು
ಉದಾ-ತ್ತ-ಪ-ರಿ-ಶುದ್ಧ
ಉದಾ-ತ್ತ-ವಾಗಿ
ಉದಾ-ತ್ತ-ವಾದ
ಉದಾ-ತ್ತವೂ
ಉದಾ-ತ್ತೀ-ಕ-ರಿಸಿ
ಉದಾ-ತ್ತೀ-ಕ-ರಿ-ಸುವ
ಉದಾರ
ಉದಾ-ರ-ದೃ-ಷ್ಟಿ-ಯ-ವ-ನಾ-ದರೂ
ಉದಾ-ರವೂ
ಉದಾರಿ
ಉದಾ-ರಿ-ಗಳೇ
ಉದಾ-ಸೀನ
ಉದಾ-ಸೀ-ನ-ನಾಗಿ
ಉದಾ-ಹ-ರಣೆ
ಉದಾ-ಹ-ರ-ಣೆ-ಗಳೂ
ಉದಾ-ಹ-ರ-ಣೆಗೆ
ಉದಾ-ಹ-ರ-ಣೆ-ಯ-ನ್ನಿತ್ತು
ಉದಾ-ಹ-ರ-ಣೆ-ಯನ್ನು
ಉದಾ-ಹ-ರ-ಣೆ-ಯಷ್ಟೆ
ಉದಾ-ಹ-ರ-ಣೆ-ಯಾದ
ಉದಾ-ಹ-ರಿಸಿ
ಉದಾ-ಹ-ರಿ-ಸುತ್ತ
ಉದಾ-ಹ-ರಿ-ಸು-ವು-ದಿತ್ತು
ಉದಿ-ಸಿತು
ಉದಿ-ಸಿತ್ತು
ಉದಿ-ಸಿದ
ಉದಿ-ಸಿ-ದಾಗ
ಉದಿ-ಸಿ-ದುವು
ಉದಿಸು
ಉದಿ-ಸು-ತ್ತವೆ
ಉದಿ-ಸುವ
ಉದು-ರಿ-ಸಿ-ಬಿ-ಡು-ತ್ತಾರೆ
ಉದ್ಗರಿ
ಉದ್ಗ-ರಿಸಿ
ಉದ್ಗ-ರಿ-ಸಿದ
ಉದ್ಗ-ರಿ-ಸಿ-ದರು
ಉದ್ಗ-ರಿ-ಸಿ-ದುದು
ಉದ್ಗ-ರಿ-ಸಿ-ದ್ದರು
ಉದ್ಗ-ರಿಸು
ಉದ್ಗ-ರಿ-ಸುತ್ತ
ಉದ್ಗ-ರಿ-ಸು-ತ್ತಾನೆ
ಉದ್ಗ-ರಿ-ಸು-ತ್ತಾರೆ
ಉದ್ಗ-ರಿ-ಸು-ತ್ತಿದ್ದ
ಉದ್ಗಾರ
ಉದ್ಗಾ-ರ-ದಲ್ಲಿ
ಉದ್ಗಾ-ರ-ವನ್ನು
ಉದ್ಗಾ-ರವೂ
ಉದ್ಘೋ-ಷಿಸು
ಉದ್ಘೋ-ಷಿ-ಸುವ
ಉದ್ದ
ಉದ್ದಂಡ
ಉದ್ದಕ್ಕೂ
ಉದ್ದ-ಗ-ಲ-ವನ್ನು
ಉದ್ದ-ನೆಯ
ಉದ್ದವೋ
ಉದ್ದೀ-ಪ-ನ-ಗೊಂ-ಡಿತು
ಉದ್ದೀ-ಪ-ನ-ಗೊ-ಳಿ-ಸುವ
ಉದ್ದು-ದ್ದದ
ಉದ್ದೇಶ
ಉದ್ದೇ-ಶ-ಕ್ಕಾಗಿ
ಉದ್ದೇ-ಶ-ಗ-ಳಿವೆ
ಉದ್ದೇ-ಶದ
ಉದ್ದೇ-ಶ-ದಿಂದ
ಉದ್ದೇ-ಶ-ದಿಂ-ದಲೇ
ಉದ್ದೇ-ಶ-ಪೂ-ರ್ವ-ಕ-ವಾಗಿ
ಉದ್ದೇ-ಶ-ವ-ನ್ನ-ರಿ-ಯದೆ
ಉದ್ದೇ-ಶ-ವ-ನ್ನಿ-ಟ್ಟು-ಕೊಂ-ಡಿ-ದ್ದರು
ಉದ್ದೇ-ಶ-ವ-ನ್ನಿ-ಟ್ಟು-ಕೊಂಡು
ಉದ್ದೇ-ಶ-ವನ್ನು
ಉದ್ದೇ-ಶ-ವಾಗಿ
ಉದ್ದೇ-ಶ-ವಾದ
ಉದ್ದೇ-ಶ-ವಾ-ದರೂ
ಉದ್ದೇ-ಶ-ವಾ-ವುದೂ
ಉದ್ದೇ-ಶ-ವಿ-ಟ್ಟು-ಕೊಂ-ಡಿ-ದ್ದರೂ
ಉದ್ದೇ-ಶ-ವೇನೋ
ಉದ್ದೇ-ಶಿಸಿ
ಉದ್ಧ-ರಿಸಿ
ಉದ್ಧ-ರಿ-ಸಿದ
ಉದ್ಧ-ರಿ-ಸಿ-ದರು
ಉದ್ಧ-ರಿ-ಸು-ತ್ತಾರೆ
ಉದ್ಧಾರ
ಉದ್ಧಾ-ರ-ಕ್ಕಾಗಿ
ಉದ್ಧಾ-ರ-ವಾ-ಗ-ಬೇ-ಕಾ-ದರೆ
ಉದ್ಭ-ವ-ವಾ-ಗು-ತ್ತಿತ್ತು
ಉದ್ಭ-ವಿ-ಸಿತು
ಉದ್ಭ-ವಿ-ಸಿದೆ
ಉದ್ಭ-ವಿಸು
ಉದ್ಭ-ವಿ-ಸು-ತ್ತಿತ್ತು
ಉದ್ಭ-ವಿ-ಸುವ
ಉದ್ಯಾನ
ಉದ್ಯಾ-ನ-ಗ-ಳಿಗೋ
ಉದ್ಯಾ-ನ-ಗಳು
ಉದ್ಯಾ-ನ-ಗೃಹ
ಉದ್ಯಾ-ನ-ಗೃ-ಹಕ್ಕೆ
ಉದ್ಯಾ-ನ-ಗೃ-ಹದ
ಉದ್ಯಾ-ನ-ಗೃ-ಹ-ದಲ್ಲಿ
ಉದ್ಯಾ-ನ-ಗೃ-ಹ-ದ-ಲ್ಲಿ-ಟ್ಟರೆ
ಉದ್ಯಾ-ನ-ಗೃ-ಹ-ದಲ್ಲೇ
ಉದ್ಯಾ-ನ-ಗೃ-ಹ-ವನ್ನು
ಉದ್ಯಾ-ನ-ಗೃ-ಹ-ವಾ-ದರೋ
ಉದ್ಯಾ-ನ-ಗೃ-ಹ-ವಾ-ಸವೇ
ಉದ್ಯಾ-ನ-ಗೃ-ಹ-ವೊಂ-ದನ್ನು
ಉದ್ಯಾ-ನದ
ಉದ್ಯಾ-ನ-ದಲ್ಲಿ
ಉದ್ಯಾ-ನ-ದಲ್ಲೇ
ಉದ್ಯಾ-ನ-ದಿಂದ
ಉದ್ಯಾ-ನ-ವ-ನಕ್ಕೆ
ಉದ್ಯಾ-ನ-ವ-ನಕ್ಕೋ
ಉದ್ಯಾ-ನ-ವ-ನದ
ಉದ್ಯಾ-ನ-ವ-ನ-ದಲ್ಲಿ
ಉದ್ಯಾ-ನ-ವನ್ನು
ಉದ್ಯೋಗ
ಉದ್ಯೋ-ಗದ
ಉದ್ಯೋ-ಗ-ವನ್ನೂ
ಉದ್ಯೋ-ಗ-ವನ್ನೇ
ಉದ್ಯೋ-ಗವೂ
ಉದ್ಯೋ-ಗ-ಸ್ಥ-ನಾಗಿ
ಉದ್ಯೋ-ಗಾ-ನ್ವೇ-ಷ-ಣೆ-ಯಲ್ಲಿ
ಉದ್ರೇ-ಕ-ಕಾರೀ
ಉದ್ವಿ-ಗ್ನ-ರಾ-ದರು
ಉದ್ವೇ-ಗದ
ಉನ್ನತ
ಉನ್ನ-ತ-ಮ-ಟ್ಟದ
ಉನ್ನ-ತ-ವಾಗಿ
ಉನ್ನ-ತ-ವಾದ
ಉನ್ನ-ತ-ವಾ-ದುದೋ
ಉನ್ನ-ತಾ-ಧಿ-ಕಾರಿ
ಉನ್ನತಿ
ಉನ್ನ-ತಿ-ಗಾಗಿ
ಉನ್ನ-ತಿ-ಗೇ-ರಿ-ದುವು
ಉನ್ನ-ತಿಯ
ಉನ್ನ-ತಿ-ಯನ್ನು
ಉನ್ನ-ತಿ-ಯಾ-ಗುವು
ಉನ್ನ-ತಿ-ಯಾ-ಗು-ವು-ದ-ರಲ್ಲಿ
ಉನ್ನು
ಉನ್ಮ-ತ್ತತೆ
ಉಪ
ಉಪ-ಕ-ರ-ಣ-ಗಳನ್ನೆಲ್ಲ
ಉಪ-ಕ-ರ-ಣವೇ
ಉಪ-ಕಾ-ರ-ಕ್ಕಾಗಿ
ಉಪ-ಕಾ-ರ-ವನ್ನು
ಉಪ-ಚ-ರಿ-ಸ-ತೊ-ಡ-ಗಿ-ದರು
ಉಪ-ಚ-ರಿ-ಸಲು
ಉಪ-ಚ-ರಿಸಿ
ಉಪ-ಚ-ರಿ-ಸಿದ
ಉಪ-ಚ-ರಿ-ಸುವ
ಉಪ-ದೇಶ
ಉಪ-ದೇ-ಶ-ಗಳ
ಉಪ-ದೇ-ಶ-ಗಳನ್ನೆಲ್ಲ
ಉಪ-ದೇ-ಶ-ಗಳು
ಉಪ-ದೇ-ಶದ
ಉಪ-ದೇ-ಶ-ವ-ನ್ನಾ-ಗಲಿ
ಉಪ-ದೇ-ಶ-ವಾಗಿ
ಉಪ-ದೇ-ಶಿ-ಸಿದ
ಉಪ-ದೇ-ಶಿ-ಸಿ-ದ-ವರೋ
ಉಪ-ನಿ-ಷ-ತ್ತಿನ
ಉಪ-ನಿ-ಷ-ತ್ತಿ-ನಲ್ಲಿ
ಉಪ-ನಿ-ಷತ್ತು
ಉಪ-ನಿ-ಷ-ತ್ತು-ಪು-ರಾ-ಣ-ಗ-ಳೇನೋ
ಉಪ-ನಿ-ಷ-ತ್ತು-ಗಳ
ಉಪ-ನಿ-ಷ-ತ್ತು-ಗಳನ್ನು
ಉಪ-ನಿ-ಷ-ತ್ತು-ಗಳನ್ನೆಲ್ಲ
ಉಪ-ನಿ-ಷ-ತ್ತು-ಗಳಲ್ಲಿ
ಉಪ-ನಿ-ಷತ್ಸು
ಉಪ-ನಿ-ಷ-ದ್ವಾಕ್ಯ
ಉಪ-ನಿ-ಷ-ದ್ವಾ-ಕ್ಯ-ಗಳ
ಉಪ-ನಿ-ಷ-ದ್ವಾ-ಕ್ಯ-ಗ-ಳಿಗೆ
ಉಪ-ನ್ಯಾ-ಸ-ಕರು
ಉಪ-ನ್ಯಾ-ಸ-ಗಳ
ಉಪ-ನ್ಯಾ-ಸ-ಗಳನ್ನು
ಉಪ-ನ್ಯಾ-ಸ-ದಲ್ಲಿ
ಉಪ-ಮಾ-ನ-ವನ್ನು
ಉಪ-ಯುಕ್ತ
ಉಪ-ಯು-ಕ್ತ-ತೆ-ಯನ್ನೂ
ಉಪ-ಯು-ಕ್ತ-ವಾ-ಗಿತ್ತು
ಉಪ-ಯೋ-ಗ-ವಾಗ
ಉಪ-ಯೋ-ಗ-ವಾ-ಗ-ಲಿಲ್ಲ
ಉಪ-ಯೋ-ಗಿ-ಸ-ಕೊ-ಳ್ಳ-ಬೇ-ಕಾ-ಗಿತ್ತು
ಉಪ-ಯೋ-ಗಿ-ಸ-ಹೋ-ಗದೆ
ಉಪ-ಯೋ-ಗಿ-ಸಿ-ಕೊಳ್ಳ
ಉಪ-ಯೋ-ಗಿ-ಸಿ-ದರೆ
ಉಪ-ಯೋ-ಗಿಸು
ಉಪ-ಯೋ-ಗಿ-ಸುತ್ತ
ಉಪ-ಯೋ-ಗಿ-ಸು-ತ್ತಿದ್ದ
ಉಪ-ಯೋ-ಗಿ-ಸುವ
ಉಪ-ರತಿ
ಉಪ-ವಾಸ
ಉಪ-ವಾ-ಸ-ವ-ನ-ವಾ-ಸ-ಗಳನ್ನು
ಉಪ-ವಾ-ಸ-ವಿ-ದ್ದದ್ದೂ
ಉಪ-ವಾ-ಸ-ವಿದ್ದು
ಉಪ-ವಾ-ಸ-ವಿ-ದ್ದು-ಕೊಂಡೇ
ಉಪ-ವಾ-ಸ-ವಿರ
ಉಪ-ವಾ-ಸವೇ
ಉಪ-ಶ-ಮ-ನ-ವಾ-ದಂ-ತಿತ್ತು
ಉಪ-ಶ-ಮ-ನ-ವಾ-ದು-ದನ್ನು
ಉಪಾ-ಧಿ-ಗ-ಳಿಗೆ
ಉಪಾ-ಧ್ಯಾಯ
ಉಪಾ-ಧ್ಯಾ-ಯರ
ಉಪಾ-ಧ್ಯಾ-ಯ-ರನ್ನು
ಉಪಾ-ಧ್ಯಾ-ಯ-ರ-ನ್ನೆಲ್ಲ
ಉಪಾ-ಧ್ಯಾ-ಯ-ರಿಂದ
ಉಪಾ-ಧ್ಯಾ-ಯರು
ಉಪಾ-ಧ್ಯಾ-ಯ-ರೊ-ಬ್ಬರು
ಉಪಾಯ
ಉಪಾ-ಯ-ಗಳನ್ನು
ಉಪಾ-ಯ-ಗಳೂ
ಉಪಾ-ಯ-ಗಾ-ಣದೆ
ಉಪಾ-ಯ-ವನ್ನು
ಉಪಾ-ಯ-ವನ್ನೇ
ಉಪಾ-ಯ-ವಾಗಿ
ಉಪಾ-ಯ-ವೇನು
ಉಪಾ-ಸ-ಕರು
ಉಪಾ-ಸನೆ
ಉಪ್ಪು
ಉಪ್ಪೇ
ಉಬ್ಬಿ-ಕೊಂ-ಡಂ-ತಿದ್ದ
ಉಬ್ಬಿ-ಹೋ-ಗಿರು
ಉಬ್ಬು-ಗಣ್ಣು
ಉಭ-ಯ-ಸಂ-ಕ-ಟಕ್ಕೆ
ಉಭ-ಯ-ಸಂ-ಕ-ಟ-ದಿಂದ
ಉಯ್ಯಾ-ಲೆಯ
ಉರಿ-ಬಿ-ಸಿ-ಲಿ-ನಲ್ಲಿ
ಉರಿ-ಯಿ-ಟ್ಟಂ-ತಾ-ಯಿತು
ಉರಿ-ಯು-ತ್ತಿದೆ
ಉರಿ-ಯು-ತ್ತಿ-ದ್ದರೆ
ಉರಿ-ಯು-ತ್ತಿ-ರುವ
ಉರಿ-ಯುವ
ಉರಿ-ಯೂತ
ಉರಿ-ಯೂ-ತದ
ಉರಿಸಿ
ಉರಿ-ಸಿ-ಕೊಂಡು
ಉರಿ-ಸು-ತ್ತಿ-ದ್ದರು
ಉರು-ಳಿ-ದುವು
ಉರು-ಳಿ-ಬೀ-ಳ-ಬೇಕು
ಉರು-ಳು-ತ್ತಿವೆ
ಉರ್ದು
ಉಲಿ
ಉಲಿಯೈ
ಉಲ್ಬ-ಣ-ವಾ-ಗು-ತ್ತಲೇ
ಉಲ್ಬ-ಣಿಸಿ
ಉಲ್ಬ-ಣಿ-ಸಿತು
ಉಲ್ಬ-ಣಿ-ಸು-ತ್ತಿತ್ತು
ಉಲ್ಬ-ಣಿ-ಸು-ತ್ತಿದ್ದ
ಉಲ್ಲಂ-ಘ-ನೆ-ಯಾ-ಗು-ತ್ತದೆ
ಉಲ್ಲ-ಸಿ-ತ-ರಾ-ದರು
ಉಲ್ಲಾಸ
ಉಲ್ಲಾ-ಸ-ಕ-ರ-ವಾ-ಗಿತ್ತು
ಉಲ್ಲಾ-ಸ-ದಿಂದ
ಉಲ್ಲಾ-ಸ-ಭಾ-ವ-ದಿಂದ
ಉಲ್ಲಾ-ಸ-ಮಯ
ಉಲ್ಲಾ-ಸವೂ
ಉಲ್ಲಾಸಿ
ಉಲ್ಲೇ-ಖಿ-ಸ-ಬ-ಹುದು
ಉಲ್ಲೇ-ಖಿ-ಸಿ-ದ್ದಾರೆ
ಉಳಿದ
ಉಳಿ-ದಂತೆ
ಉಳಿ-ದ-ದ್ದನ್ನು
ಉಳಿ-ದರು
ಉಳಿ-ದರೂ
ಉಳಿ-ದ-ವರ
ಉಳಿ-ದ-ವ-ರಂ-ತೆಯೇ
ಉಳಿ-ದ-ವ-ರಲ್ಲೂ
ಉಳಿ-ದ-ವ-ರಿಗೆ
ಉಳಿ-ದ-ವರು
ಉಳಿ-ದ-ವರೂ
ಉಳಿ-ದ-ವರೆಲ್ಲ
ಉಳಿ-ದ-ವರೆ-ಲ್ಲರೂ
ಉಳಿ-ದಿದೆ
ಉಳಿ-ದಿ-ದ್ದರು
ಉಳಿ-ದಿ-ದ್ದರೆ
ಉಳಿ-ದಿ-ದ್ದಾರೆ
ಉಳಿ-ದಿ-ರುವ
ಉಳಿ-ದಿ-ರು-ವುದನ್ನು
ಉಳಿದು
ಉಳಿ-ದು-ಕೊಂಡ
ಉಳಿ-ದು-ಕೊಂ-ಡರು
ಉಳಿ-ದು-ಕೊಂ-ಡಿತ್ತು
ಉಳಿ-ದು-ಕೊಂ-ಡಿದ್ದ
ಉಳಿ-ದು-ಕೊಂ-ಡಿ-ದ್ದರು
ಉಳಿ-ದು-ಕೊಂ-ಡಿ-ದ್ದ-ವರು
ಉಳಿ-ದು-ಕೊಂ-ಡಿದ್ದು
ಉಳಿ-ದು-ಕೊಂ-ಡಿವೆ
ಉಳಿ-ದು-ಕೊಂಡು
ಉಳಿ-ದು-ಕೊಂ-ಡು-ಬಿ-ಟ್ಟಿತು
ಉಳಿ-ದು-ಕೊ-ಳ್ಳ-ಬೇ-ಕಾಗಿ
ಉಳಿ-ದು-ಕೊ-ಳ್ಳಲಿ
ಉಳಿ-ದು-ಕೊ-ಳ್ಳ-ಲಿಲ್ಲ
ಉಳಿ-ದು-ಕೊ-ಳ್ಳಲು
ಉಳಿ-ದು-ಕೊ-ಳ್ಳು-ವಂತೆ
ಉಳಿ-ದು-ಕೊ-ಳ್ಳು-ವುದು
ಉಳಿ-ದು-ದನ್ನು
ಉಳಿ-ದು-ದೆಲ್ಲ
ಉಳಿ-ದೆಲ್ಲ
ಉಳಿ-ದೆ-ಲ್ಲರ
ಉಳಿದೇ
ಉಳಿ-ಯ-ಗೊ-ಡು-ವು-ದಿಲ್ಲ
ಉಳಿ-ಯ-ಲಿಲ್ಲ
ಉಳಿ-ಯಲೂ
ಉಳಿ-ಯಿತು
ಉಳಿ-ಯು-ತ್ತದೆ
ಉಳಿ-ಯು-ತ್ತಿತ್ತು
ಉಳಿ-ಯು-ತ್ತೇನೆ
ಉಳಿ-ಯುವ
ಉಳಿ-ಯು-ವಂತೆ
ಉಳಿ-ಯು-ವು-ದಿಲ್ಲ
ಉಳಿ-ಯು-ವು-ದಿ-ಲ್ಲವೋ
ಉಳಿ-ಯು-ವುದು
ಉಳಿ-ಯು-ವುದೂ
ಉಳಿಸಿ
ಉಳಿ-ಸಿ-ಕೊಂ-ಡರು
ಉಳಿ-ಸಿ-ಕೊಂ-ಡಿದ್ದ
ಉಳಿ-ಸಿ-ಕೊಂಡು
ಉಳಿ-ಸಿದ
ಉಳಿ-ಸಿ-ದ-ವನು
ಉಳಿ-ಸಿದ್ದ
ಉಳಿ-ಸು-ವಂತೆ
ಉಷ್ಣ-ದಿಂ-ದಾಗಿ
ಉಸಿ-ರಾ-ಟದ
ಉಸಿ-ರಾ-ಟ-ವಿಲ್ಲ
ಉಸಿ-ರಾ-ಟವೂ
ಉಸಿ-ರಾ-ಡಲು
ಉಸಿ-ರಾ-ಡು-ತ್ತೀಯೆ
ಉಸಿ-ರಾ-ಡು-ವ-ವರ
ಉಸಿ-ರಿ-ನಲ್ಲೇ
ಉಸಿರು
ಉಸ್ತಾ-ದರು
ಊಟ
ಊಟ-ತಿಂ-ಡಿ-ಗಳ
ಊಟ-ಬ-ಟ್ಟೆಗೂ
ಊಟ-ಕ್ಕಾ-ಗು-ವಷ್ಟು
ಊಟಕ್ಕೂ
ಊಟಕ್ಕೆ
ಊಟ-ಕ್ಕೇ-ಳು-ವಂತೆ
ಊಟದ
ಊಟ-ಮಾ-ಡು-ವುದು
ಊಟ-ವಾದ
ಊಟ-ವಿ-ಲ್ಲದೆ
ಊಟವೂ
ಊಟೋ-ಪ-ಚಾ-ರಕ್ಕೂ
ಊಟೋ-ಪ-ಚಾ-ರ-ಗಳ
ಊಟೋ-ಪ-ಚಾ-ರ-ಗಳನ್ನು
ಊರ
ಊರ-ಲ್ಲೆಲ್ಲ
ಊರಾದ
ಊರಿಂ-ದೂ-ರಿಗೆ
ಊರಿಗೆ
ಊರಿ-ಗೊಂದು
ಊರಿನ
ಊರಿ-ನಲ್ಲಿ
ಊರು-ಗ-ಳ-ಲ್ಲಿ-ರು-ವ-ವರು
ಊಹಿ-ಸ-ಬ-ಹುದು
ಊಹಿ-ಸಲು
ಊಹಿಸಿ
ಊಹಿ-ಸಿದ
ಊಹಿ-ಸಿ-ದರು
ಊಹಿ-ಸಿ-ಯಾರು
ಊಹಿ-ಸಿಯೂ
ಊಹಿ-ಸಿ-ರ-ಲಿಲ್ಲ
ಊಹಿಸು
ಊಹಿ-ಸು-ವು-ದಕ್ಕೂ
ಊಹೆಗೆ
ಋಣಿ
ಎ
ಎಂಎದು
ಎಂಜಲು
ಎಂಟ-ನೆಯ
ಎಂಟು
ಎಂಟ್ರೆ-ನ್ಸ್
ಎಂತಹ
ಎಂಥ
ಎಂಥದು
ಎಂಥದೋ
ಎಂಥ-ವ-ರನ್ನೂ
ಎಂಥ-ವ-ರನ್ನೇ
ಎಂಥ-ವರಿ
ಎಂಥ-ವರೂ
ಎಂಥಾ
ಎಂದ
ಎಂದಂ-ತಾ-ಯಿತು
ಎಂದಂತೆ
ಎಂದನೋ
ಎಂದ-ಮಾ-ತ್ರಕ್ಕೆ
ಎಂದ-ಮೇಲೆ
ಎಂದ-ರಾ-ಗದು
ಎಂದ-ರಿತ
ಎಂದ-ರಿತು
ಎಂದರು
ಎಂದರೂ
ಎಂದರೆ
ಎಂದ-ರೇ-ನರ್ಥ
ಎಂದರ್ಥ
ಎಂದಲ್ಲ
ಎಂದಳು
ಎಂದ-ವರು
ಎಂದಷ್ಟೇ
ಎಂದಾಗ
ಎಂದಾ-ಗಲಿ
ಎಂದಾ-ದರೂ
ಎಂದಾ-ಯಿತು
ಎಂದಾರು
ಎಂದಿ
ಎಂದಿ-ಗಾ-ದರೂ
ಎಂದಿಗೂ
ಎಂದಿಗೆ
ಎಂದಿ-ಟ್ಟು-ಕೊಂ-ಡಿ-ದ್ದರೂ
ಎಂದಿದ್ದ
ಎಂದಿ-ದ್ದರು
ಎಂದಿ-ದ್ದರೂ
ಎಂದಿ-ದ್ದ-ರೆ-ಅದು
ಎಂದಿನ
ಎಂದಿ-ನಂ-ತಾದ
ಎಂದಿ-ನಂ-ತಾ-ದಾಗ
ಎಂದಿ-ನಂ-ತಿ-ರ-ಲಿಲ್ಲ
ಎಂದಿ-ನಂತೆ
ಎಂದಿ-ನಂ-ತೆಯೇ
ಎಂದು
ಎಂದು-ಕೊಂಡ
ಎಂದು-ಕೊಂ-ಡರು
ಎಂದು-ಕೊಂಡಿ
ಎಂದು-ಕೊಂ-ಡಿರಾ
ಎಂದು-ಕೊಂ-ಡಿ-ರೇನು
ಎಂದು-ಕೊಂಡು
ಎಂದು-ಕೊಂಡೆ
ಎಂದು-ಕೊ-ಳ್ಳ-ಬ-ಹುದು
ಎಂದು-ಕೊ-ಳ್ಳುತ್ತ
ಎಂದು-ತ್ತ-ರಿ-ಸಿದ
ಎಂದು-ತ್ತ-ರಿ-ಸಿ-ದರು
ಎಂದು-ತ್ತ-ರಿ-ಸು-ತ್ತಾನೆ
ಎಂದು-ದ್ಗ-ರಿಸಿ
ಎಂದು-ದ್ಗ-ರಿ-ಸಿದ
ಎಂದು-ದ್ಗ-ರಿ-ಸಿ-ದರು
ಎಂದು-ದ್ಗ-ರಿ-ಸುತ್ತ
ಎಂದು-ದ್ಗ-ರಿ-ಸು-ತ್ತಾನೆ
ಎಂದು-ಬಿಟ್ಟ
ಎಂದು-ಬಿ-ಟ್ಟರು
ಎಂದು-ಬಿ-ಟ್ಟರೆ
ಎಂದು-ಬಿಟ್ಟೆ
ಎಂದು-ಬಿ-ಡು-ತ್ತಿ-ದ್ದರು
ಎಂದು-ಬಿ-ಡು-ವುದೆ
ಎಂದೂ
ಎಂದೆ
ಎಂದೆಂ
ಎಂದೆಂ-ದಿಗೂ
ಎಂದೆಂದೂ
ಎಂದೆ-ನಿ-ಸಿತು
ಎಂದೆ-ನಿ-ಸು-ತ್ತಿತ್ತು
ಎಂದೆ-ನ್ನುವ
ಎಂದೆ-ನ್ನು-ವುದು
ಎಂದೆ-ಯಲ್ಲ
ಎಂದೆಲ್ಲ
ಎಂದೇ
ಎಂದೇಕೆ
ಎಂದೇನೂ
ಎಂದೇನೋ
ಎಂದೊಬ್ಬ
ಎಂದೋ
ಎಂಬ
ಎಂಬಂ-ತಾ-ಗಿದೆ
ಎಂಬಂ-ತಿತ್ತು
ಎಂಬಂತೆ
ಎಂಬಂಥ
ಎಂಬತ್ತು
ಎಂಬರು
ಎಂಬ-ರ್ಥದ
ಎಂಬ-ರ್ಥ-ದಲ್ಲಿ
ಎಂಬ-ರ್ಥವೂ
ಎಂಬಲ್ಲಿ
ಎಂಬ-ಲ್ಲಿಗೆ
ಎಂಬ-ಲ್ಲಿದ್ದ
ಎಂಬ-ಲ್ಲಿನ
ಎಂಬ-ವನ
ಎಂಬ-ವ-ನನ್ನು
ಎಂಬ-ವ-ನಿಗೆ
ಎಂಬ-ವನು
ಎಂಬ-ವರ
ಎಂಬ-ವ-ರನ್ನು
ಎಂಬ-ವರು
ಎಂಬ-ವಳು
ಎಂಬವು
ಎಂಬ-ವು-ಗ-ಳೆಲ್ಲ
ಎಂಬಷ್ಟು
ಎಂಬಷ್ಟೇ
ಎಂಬಾತ
ಎಂಬಾ-ತನ
ಎಂಬಿ-ತ್ಯಾ-ದಿ-ಯಾಗಿ
ಎಂಬಿ-ಬ್ಬರು
ಎಂಬುದ
ಎಂಬು-ದಕ್ಕೆ
ಎಂಬು-ದನು
ಎಂಬು-ದನ್ನು
ಎಂಬು-ದನ್ನೂ
ಎಂಬು-ದ-ನ್ನೆಲ್ಲ
ಎಂಬು-ದನ್ನೇ
ಎಂಬು-ದರ
ಎಂಬು-ದ-ರಲ್ಲಿ
ಎಂಬು-ದಾ-ಗಿತ್ತು
ಎಂಬು-ದಾ-ದರೂ
ಎಂಬುದು
ಎಂಬುದೂ
ಎಂಬುದೇ
ಎಂಬು-ದೇ-ನಾ-ದರೂ
ಎಂಬು-ದೊಂದು
ಎಂಬು-ವನ
ಎಂಬು-ವನು
ಎಂಬು-ವರ
ಎಂಬು-ವರು
ಎಂಬೊಬ್ಬ
ಎಂಹ
ಎಕ-ರೆ-ಗ-ಟ್ಟಲೆ
ಎಚ್ಚ-ತ್ತಿ-ದ್ದರೂ
ಎಚ್ಚ-ತ್ತು-ಕೊಂಡ
ಎಚ್ಚರ
ಎಚ್ಚ-ರ-ಗೊಂಡ
ಎಚ್ಚ-ರ-ಗೊ-ಳಲು
ಎಚ್ಚ-ರ-ಗೊ-ಳಿ-ಸಲು
ಎಚ್ಚ-ರ-ಗೊ-ಳ್ಳು-ತ್ತಿದ್ದೆ
ಎಚ್ಚ-ರ-ದಿಂದ
ಎಚ್ಚ-ರ-ದಿಂ-ದಿ-ರು-ವಂತೆ
ಎಚ್ಚ-ರ-ವ-ಹಿಸಿ
ಎಚ್ಚ-ರ-ವ-ಹಿ-ಸಿ-ದರು
ಎಚ್ಚ-ರ-ವಾ-ಗಿದ್ದ
ಎಚ್ಚ-ರ-ವಾ-ಗಿಯೇ
ಎಚ್ಚ-ರ-ವಾ-ಯಿತು
ಎಚ್ಚ-ರಿಕೆ
ಎಚ್ಚ-ರಿ-ಕೆಯ
ಎಚ್ಚ-ರಿ-ಕೆ-ಯನ್ನು
ಎಚ್ಚ-ರಿ-ಕೆ-ಯಿಂದ
ಎಚ್ಚ-ರಿ-ಕೆ-ಯಿಂ-ದ-ವ-ರನು
ಎಚ್ಚ-ರಿ-ಕೆ-ಯಿ-ರಲಿ
ಎಚ್ಚ-ರಿ-ದಿಂದ
ಎಚ್ಚ-ರಿ-ಸಿ-ದರು
ಎಟಾವಾ
ಎಟು-ಕದ
ಎಟು-ಕ-ದಂತೆ
ಎಟು-ಕ-ಬಲ್ಲ
ಎಡ-ತಾ-ಕಿದ
ಎಡದ
ಎಡ-ಭಾ-ಗ-ದಲ್ಲಿ
ಎಡ-ವ-ಟ್ಟಿಲ್ಲ
ಎಡೆ-ಗೊ-ಡ-ಲಾರೆ
ಎಡೆ-ಬಿ-ಡ-ದಿ-ರದ
ಎಡೆ-ಬಿ-ಡದೆ
ಎಡ್ವ-ರ್ಡ್
ಎಣಿ-ಸಿ-ದರೆ
ಎಣ್ಣೆ
ಎತ್ತ-ಬ-ಲ್ಲೆಯಾ
ಎತ್ತರ
ಎತ್ತ-ರಕ್ಕೂ
ಎತ್ತ-ರಕ್ಕೆ
ಎತ್ತ-ರದ
ಎತ್ತ-ರ-ದಲ್ಲಿ
ಎತ್ತ-ರ-ದ-ಲ್ಲೇನೋ
ಎತ್ತ-ರ-ವಾದ
ಎತ್ತ-ರೆ-ತ್ತ-ರಕ್ಕೆ
ಎತ್ತಲು
ಎತ್ತಿ
ಎತ್ತಿ-ತೋ-ರಿಸಿ
ಎತ್ತಿ-ತೋ-ರಿ-ಸಿದ
ಎತ್ತಿ-ತೋ-ರಿ-ಸು-ತ್ತಿದ್ದ
ಎತ್ತಿ-ನ-ಗಾಡಿ
ಎತ್ತಿ-ನ-ಗಾ-ಡಿ-ಯಲ್ಲೇ
ಎತ್ತಿ-ರಿ-ಸಿ-ದರು
ಎತ್ತಿ-ಹಿ-ಡಿ-ಯ-ಲಾಗಿದೆ
ಎತ್ತಿ-ಹಿ-ಡಿ-ಯುತ್ತ
ಎತ್ತಿ-ಹಿ-ಡಿ-ಯು-ತ್ತಿದ್ದ
ಎತ್ತು-ತ್ತಿ-ದ್ದರು
ಎತ್ತು-ತ್ತಿ-ದ್ದೆವು
ಎತ್ತುವ
ಎತ್ತು-ವಾಗ
ಎತ್ತೆತ್ತಿ
ಎದುರಾ
ಎದು-ರಾ-ಗಲಿ
ಎದು-ರಾಗಿ
ಎದು-ರಾ-ಗಿ-ದ್ದು-ದ-ರಿಂ-ದಲೇ
ಎದು-ರಾ-ಗಿ-ರುವ
ಎದು-ರಾ-ಗು-ತ್ತವೆ
ಎದು-ರಾ-ಗು-ತ್ತಿತ್ತು
ಎದು-ರಾ-ಗುವ
ಎದು-ರಾ-ಡು-ತ್ತಾ-ರಲ್ಲ
ಎದು-ರಾ-ಳಿಯ
ಎದು-ರಿ-ಗಿ-ದ್ದರೂ
ಎದು-ರಿಗೆ
ಎದು-ರಿಗೇ
ಎದು-ರಿ-ನಲ್ಲಿ
ಎದು-ರಿ-ನಿಂದ
ಎದು-ರಿ-ಸ-ಬೇ-ಕಾ-ಗಿತ್ತು
ಎದು-ರಿ-ಸ-ಬೇ-ಕಾ-ಗಿದೆ
ಎದು-ರಿ-ಸ-ಬೇ-ಕಾ-ಗು-ತ್ತದೆ
ಎದು-ರಿ-ಸ-ಬೇ-ಕಾದ
ಎದು-ರಿ-ಸ-ಬೇ-ಕಾ-ಯಿತು
ಎದು-ರಿ-ಸ-ಬೇಕು
ಎದು-ರಿಸಿ
ಎದು-ರಿ-ಸಿದ
ಎದು-ರಿ-ಸಿಯೇ
ಎದು-ರಿಸು
ಎದುರು
ಎದು-ರು-ಗಡೆ
ಎದು-ರು-ಗ-ಡೆ-ಯಿಂದ
ಎದು-ರು-ನೋ-ಡುತ್ತ
ಎದು-ರು-ವಾ-ದಿಯ
ಎದು-ರು-ಸಾ-ಲಿ-ನಲ್ಲೇ
ಎದೆ
ಎದೆ-ಗಾ-ರಿಕೆ
ಎದೆ-ಗಾ-ರಿ-ಕೆ-ಯನ್ನು
ಎದೆ-ಗಾ-ರಿ-ಕೆ-ಯಾ-ವುದೂ
ಎದೆ-ಗುಂ-ದ-ಲಿಲ್ಲ
ಎದೆ-ತಟ್ಟಿ
ಎದೆ-ಬಿ-ರಿದು
ಎದೆ-ಮುಟ್ಟಿ
ಎದೆ-ಯನ್ನು
ಎದೆ-ಯನ್ನೂ
ಎದೆ-ಯಲಿ
ಎದೆ-ಯಲ್ಲಿ
ಎದೆ-ಯೊ-ಳ-ಗೆಲ್ಲ
ಎದೆ-ಯೊಳೇ
ಎದ್ದ-ಮೇಲೆ
ಎದ್ದರು
ಎದ್ದರೆ
ಎದ್ದಿತು
ಎದ್ದಿ-ತು-ಅಲ್ಲ
ಎದ್ದಿ-ತು-ಶ್ರೀ-ರಾ-ಮ-ಕೃ-ಷ್ಣರು
ಎದ್ದಿ-ರು-ತ್ತಿ-ದ್ದು-ದ-ರಿಂದ
ಎದ್ದು
ಎದ್ದು-ಕಾ-ಣು-ತ್ತದೆ
ಎದ್ದು-ಕಾ-ಣು-ತ್ತಿತ್ತು
ಎದ್ದು-ಕಾ-ಣು-ತ್ತಿದ್ದ
ಎದ್ದು-ತೋ-ರು-ತ್ತಿ-ದ್ದರೂ
ಎದ್ದು-ನಿಂತ
ಎದ್ದು-ನಿಂತು
ಎದ್ದು-ಬಂದ
ಎದ್ದು-ಬಂದು
ಎದ್ದು-ಬಿಟ್ಟ
ಎದ್ದು-ಹೋದ
ಎದ್ದೆದ್ದು
ಎದ್ದೇಳಿ
ಎದ್ದೇ-ಳು-ತ್ತಿತ್ತು
ಎದ್ದೇ-ಳುವ
ಎದ್ದೇ-ಳು-ವು-ದಿ-ರಲಿ
ಎದ್ದೊ-ಡ-ನೆಯೇ
ಎನಿ-ಸಿ-ದ್ದನ್ನು
ಎನೋ
ಎನ್ನ
ಎನ್ನ-ತೊ-ಡ-ಗಿ-ದಾಗ
ಎನ್ನದೆ
ಎನ್ನ-ಬ-ಹುದು
ಎನ್ನ-ಬೇಕು
ಎನ್ನ-ಬೇ-ಕು-ಅ-ಷ್ಟರ
ಎನ್ನಲೂ
ಎನ್ನಿಂ-ದೆಲ್ಲ
ಎನ್ನಿ-ಸ-ತೊ-ಡ-ಗಿತು
ಎನ್ನಿಸಿ
ಎನ್ನಿ-ಸಿತು
ಎನ್ನಿ-ಸು-ತ್ತದೆ
ಎನ್ನಿ-ಸು-ತ್ತಿತ್ತು
ಎನ್ನು
ಎನ್ನು-ಎ-ನ್ನುವ
ಎನ್ನುತ
ಎನ್ನುತ್ತ
ಎನ್ನು-ತ್ತವೆ
ಎನ್ನು-ತ್ತಾ-ನ-ವನು
ಎನ್ನು-ತ್ತಾನೆ
ಎನ್ನು-ತ್ತಾ-ರಲ್ಲ
ಎನ್ನು-ತ್ತಾರೆ
ಎನ್ನುತ್ತಿ
ಎನ್ನು-ತ್ತಿದ್ದ
ಎನ್ನು-ತ್ತಿ-ದ್ದರು
ಎನ್ನು-ತ್ತಿ-ದ್ದಾನೆ
ಎನ್ನು-ತ್ತಿ-ದ್ದಾರೆ
ಎನ್ನು-ತ್ತಿ-ರು-ವುದು
ಎನ್ನು-ತ್ತೀಯೆ
ಎನ್ನುವ
ಎನ್ನು-ವಂ-ತಹ
ಎನ್ನು-ವಂ-ತಿತ್ತು
ಎನ್ನು-ವಂ-ತಿದೆ
ಎನ್ನು-ವಂತೆ
ಎನ್ನು-ವಂ-ಥ-ದೇ-ನಾ-ದರೂ
ಎನ್ನು-ವ-ವನ
ಎನ್ನು-ವ-ವನು
ಎನ್ನು-ವ-ವರು
ಎನ್ನು-ವ-ವ-ರೆಗೆ
ಎನ್ನು-ವ-ವ-ಳೊ-ಬ್ಬಳು
ಎನ್ನು-ವಷ್ಟು
ಎನ್ನು-ವಾಗ
ಎನ್ನುವು
ಎನ್ನು-ವುದನ್ನು
ಎನ್ನು-ವು-ದ-ನ್ನೆಲ್ಲ
ಎನ್ನು-ವು-ದನ್ನೇ
ಎನ್ನು-ವು-ದ-ನ್ನೇನೋ
ಎನ್ನು-ವು-ದರ
ಎನ್ನು-ವು-ದ-ರೊ-ಳ-ಗಾಗಿ
ಎನ್ನು-ವು-ದ-ಲ್ಲದೆ
ಎನ್ನು-ವು-ದಾ-ದರೆ
ಎನ್ನು-ವು-ದಾ-ದಲ್ಲಿ
ಎನ್ನು-ವು-ದಿಲ್ಲ
ಎನ್ನು-ವುದು
ಎನ್ನು-ವು-ದೆಲ್ಲ
ಎನ್ನು-ವುದೇ
ಎನ್ನು-ವು-ದೇನೂ
ಎನ್ನು-ವು-ದೇನೋ
ಎನ್ನು-ವು-ದೊಂದು
ಎನ್ಸೈ-ಕ್ಲೋ-ಪೀ-ಡಿಯ
ಎಫ್ಎ
ಎಬ್ಬಿ-ಸ-ಬೇ-ಕಾ-ಗು-ತ್ತಿತ್ತು
ಎಬ್ಬಿ-ಸಲು
ಎಬ್ಬಿಸಿ
ಎಬ್ಬಿ-ಸಿ-ಬಿ-ಟ್ಟಿ-ದ್ದುವು
ಎಬ್ಬಿ-ಸಿ-ಬಿ-ಡು-ತ್ತಿದ್ದ
ಎಬ್ಬಿ-ಸು-ತ್ತಿ-ದ್ದರು
ಎಬ್ಬಿ-ಸು-ವು-ದಾ-ದರೂ
ಎರ-ಗು-ತ್ತಾರೆ
ಎರ-ಡನೆ
ಎರ-ಡ-ನೆಯ
ಎರ-ಡ-ನೆ-ಯ-ದಾಗಿ
ಎರ-ಡ-ನೆ-ಯದು
ಎರ-ಡ-ನೆ-ಯ-ದೆಂ-ದರೆ
ಎರ-ಡನೇ
ಎರ-ಡನ್ನೂ
ಎರ-ಡ-ರಲ್ಲಿ
ಎರ-ಡ-ರಷ್ಟು
ಎರ-ಡ-ರಿಂ-ದಲೂ
ಎರ-ಡರ್ಥ
ಎರ-ಡ-ರ್ಥ-ಒಂ-ದ-ನೆ-ಯ-ದಾಗಿ
ಎರ-ಡಾಗಿ
ಎರಡು
ಎರ-ಡು
ಎರ-ಡು-ಮೂರು
ಎರ-ಡು-ಮೂರು
ಎರ-ಡು-ವೈ-ಷ್ಣವ
ಎರಡೂ
ಎರ-ಡೆ-ರಡು
ಎರಡೇ
ಎರಡೋ
ಎರ-ವಲು
ಎಲ
ಎಲಾ
ಎಲೆ
ಎಲೆಗೆ
ಎಲೆ-ಯಂತೆ
ಎಲ್ಲ
ಎಲ್ಲಕ್ಕೂ
ಎಲ್ಲ-ದ-ರಿಂ-ದಲೂ
ಎಲ್ಲರ
ಎಲ್ಲ-ರಂತೆ
ಎಲ್ಲ-ರನ್ನು
ಎಲ್ಲ-ರನ್ನೂ
ಎಲ್ಲ-ರಿಂ-ದಲೂ
ಎಲ್ಲ-ರಿ-ಗಿಂತ
ಎಲ್ಲ-ರಿಗೂ
ಎಲ್ಲರೂ
ಎಲ್ಲ-ರೆ-ದು-ರಿಗೂ
ಎಲ್ಲ-ರೆ-ದು-ರಿಗೇ
ಎಲ್ಲ-ರೆ-ದು-ರಿ-ನಲ್ಲೂ
ಎಲ್ಲ-ರೆ-ದುರು
ಎಲ್ಲ-ರೊ-ಳಗೂ
ಎಲ್ಲ-ವನ್ನೂ
ಎಲ್ಲವು
ಎಲ್ಲ-ವು-ಗಳ
ಎಲ್ಲವೂ
ಎಲ್ಲ-ಸ-ರಿ-ಹೋ-ಗು-ತ್ತದೆ
ಎಲ್ಲಾದ
ಎಲ್ಲಿ
ಎಲ್ಲಿಂದ
ಎಲ್ಲಿಂ-ದಲೋ
ಎಲ್ಲಿಗೆ
ಎಲ್ಲಿಗೇ
ಎಲ್ಲಿಗೋ
ಎಲ್ಲಿದೆ
ಎಲ್ಲಿ-ಯ-ವ-ರೆಗೆ
ಎಲ್ಲಿ-ರಿಗೂ
ಎಲ್ಲಿ-ರು-ತ್ತೇನೆ
ಎಲ್ಲಿ-ರು-ವಿರಿ
ಎಲ್ಲಿ-ಲ್ಲದ
ಎಲ್ಲೂ
ಎಲ್ಲೆ
ಎಲ್ಲೆಂ-ದ-ರ-ಲ್ಲಿಗೆ
ಎಲ್ಲೆಡೆ
ಎಲ್ಲೆ-ಡೆ-ಗ-ಳಲ್ಲೂ
ಎಲ್ಲೆ-ಡೆಗೂ
ಎಲ್ಲೆಲ್ಲಿ
ಎಲ್ಲೆಲ್ಲೂ
ಎಲ್ಲೇ
ಎಲ್ಲೋ
ಎಳೆ-ದಂ-ತಾಗಿ
ಎಳೆದು
ಎಳೆ-ದು-ಕೊಂಡು
ಎಳೆ-ದು-ಕೊಂ-ಡು-ಬಿಟ್ಟ
ಎಳೆ-ದು-ಬಿ-ಡು-ವಂತೆ
ಎಳೆ-ದೆ-ಳೆದು
ಎಳೆಯ
ಎಳೆ-ಯಲಿ
ಎಳೆ-ಯಲು
ಎಳೆ-ಯುತ್ತ
ಎಳೆ-ಯು-ತ್ತಾರೆ
ಎಳೆ-ಯು-ತ್ತಿದ್ದ
ಎಳ್ಳು-ತುಪ್ಪ
ಎವೆ-ಯಿ-ಕ್ಕದೆ
ಎಷ್ಟರ
ಎಷ್ಟ-ರ-ಮ-ಟ್ಟಿ-ಗಿ-ತ್ತೆಂ-ದರೆ
ಎಷ್ಟ-ರ-ಮ-ಟ್ಟಿಗೆ
ಎಷ್ಟ-ರ-ಮ-ಟ್ಟಿ-ಗೆಂ-ದರೆ
ಎಷ್ಟಾ-ದರೂ
ಎಷ್ಟು
ಎಷ್ಟು-ಹೊತ್ತು
ಎಷ್ಟೆಂದು
ಎಷ್ಟೆಷ್ಟು
ಎಷ್ಟೆಷ್ಟೋ
ಎಷ್ಟೇ
ಎಷ್ಟೊ
ಎಷ್ಟೊಂದು
ಎಷ್ಟೋ
ಎಷ್ಟೋ-ಸಲ
ಎಸಗ
ಎಸ-ಗ-ಲಿ-ದ್ದಾನೆ
ಎಸೆ-ದರು
ಎಸೆದು
ಎಸೆ-ದು-ಕೊಂಡು
ಎಸೆ-ದು-ಬಿ-ಟ್ಟ-ನಾ-ದರೂ
ಎಸೆ-ದು-ಬಿ-ಡೋಣ
ಎಸೆ-ಯು-ವಲ್ಲಿ
ಎಸ್ರಾಜ್
ಏ
ಏ
ಏಕ-ಕಾ-ಲ-ದಲ್ಲಿ
ಏಕ-ತೆ-ಯನ್ನು
ಏಕ-ದೇ-ವತಾ
ಏಕ-ನಿ-ಷ್ಠೆಯ
ಏಕ-ಪ್ರ-ಕಾರ
ಏಕ-ಮಾತ್ರ
ಏಕ-ಮೇ-ವಾ-ದ್ವಿ-ತೀ-ಯ-ವಾದ
ಏಕಾಂ-ಗಿ-ಯಾಗಿ
ಏಕಾಂ-ಗಿ-ಯಾ-ಗಿಯೆ
ಏಕಾಂತ
ಏಕಾಂ-ತಕ್ಕೆ
ಏಕಾಂ-ತತೆ
ಏಕಾಂ-ತ-ದಲ್ಲಿ
ಏಕಾಂ-ತ-ದ-ಲ್ಲಿ-ದ್ದು-ಕೊಂಡು
ಏಕಾಂ-ತ-ವಾ-ಸ-ದ-ಲ್ಲಿ-ದ್ದು-ಕೊಂಡು
ಏಕಾ-ಗ್ರ-ಗೊಂಡ
ಏಕಾ-ಗ್ರ-ಗೊ-ಳಿ-ಸಲು
ಏಕಾ-ಗ್ರ-ಗೊ-ಳಿಸಿ
ಏಕಾ-ಗ್ರ-ಗೊ-ಳಿ-ಸಿ-ಕೊಂಡು
ಏಕಾ-ಗ್ರ-ಗೊ-ಳಿಸು
ಏಕಾ-ಗ್ರ-ಗೊ-ಳಿ-ಸು-ವಂತೆ
ಏಕಾ-ಗ್ರ-ತೆ-ಯಿಂದ
ಏಕಾ-ಗ್ರ-ವಾ-ಗದೆ
ಏಕಾ-ದರೂ
ಏಕಿ-ರ-ಬ-ಹುದು
ಏಕೆ
ಏಕೆಂ
ಏಕೆಂ-ದರೆ
ಏಕೆಂದು
ಏಕೈಕ
ಏಕೋ
ಏಟು
ಏತಕ್ಕೆ
ಏತ-ಕ್ಕೋ-ಸ್ಕರ
ಏನ-ಡಿಗೆ
ಏನದು
ಏನ-ನ್ನಾ-ದರೂ
ಏನನ್ನು
ಏನನ್ನೂ
ಏನನ್ನೋ
ಏನಪ್ಪ
ಏನಪ್ಪಾ
ಏನಯ್ಯ
ಏನರ್ಥ
ಏನಾ
ಏನಾ-ಗ-ಬೇಕು
ಏನಾ-ಗಿತ್ತು
ಏನಾ-ಗಿ-ರ-ಬ-ಹು-ದೆಂ-ಬು-ದನ್ನು
ಏನಾ-ಗಿ-ಹೋ-ಯಿತು
ಏನಾ-ಗು-ತ್ತದೋ
ಏನಾ-ಗು-ತ್ತಿತ್ತೋ
ಏನಾ-ಗುತ್ತೆ
ಏನಾದ
ಏನಾ-ದ-ರಾ-ಗಲಿ
ಏನಾ-ದರೂ
ಏನಾ-ದ-ರೊಂದು
ಏನಾಯಿ
ಏನಾ-ಯಿ-ತಪ್ಪ
ಏನಾ-ಯಿತು
ಏನಾ-ಯಿ-ತೆಂ-ಬುದು
ಏನಾ-ಯಿ-ತೆಂ-ಬುದೂ
ಏನಾ-ಶ್ಚರ್ಯ
ಏನಿತ್ತೋ
ಏನಿದು
ಏನಿದೆ
ಏನಿ-ದೆಲ್ಲ
ಏನಿ-ರ-ಬ-ಹುದು
ಏನಿಲ್ಲ
ಏನಿ-ಲ್ಲವೋ
ಏನೀಗ
ಏನು
ಏನು-ತಾನೆ
ಏನುದು
ಏನೂ
ಏನೆಂ-ದರೆ
ಏನೆಂದು
ಏನೆಂ-ದು-ಕೊಂ-ಡಾರು
ಏನೆಂ-ಬು-ದಾ-ದರೂ
ಏನೆಂ-ಬುದು
ಏನೆ-ನ್ನು-ತ್ತೀಯಾ
ಏನೆ-ನ್ನುವೆ
ಏನೇ
ಏನೇ-ನನ್ನು
ಏನೇ-ನಿ-ದೆಯೋ
ಏನೇನು
ಏನೇನೂ
ಏನೇನೋ
ಏನೋ
ಏನ್
ಏನ್ರಪ್ಪಾ
ಏಪ್ರಿಲ್
ಏಪ್ರಿ-ಲ್ನಲ್ಲಿ
ಏಯ್
ಏರ-ದಂತೆ
ಏರ-ಬ-ಲ್ಲರು
ಏರ-ಬೇ-ಕಾ-ದರೆ
ಏರಿ-ಕೊಂಡು
ಏರಿ-ರು-ವ-ವರು
ಏರಿ-ಸಿ-ಕೊ-ಳ್ಳು-ವಲ್ಲಿ
ಏರಿ-ಸುವ
ಏರಿ-ಹೋ-ಗು-ತ್ತಿದ್ದ
ಏರಿ-ಹೋ-ಯಿತು
ಏರುತ್ತ
ಏರುವ
ಏರು-ವು-ದಕ್ಕೂ
ಏರ್ಪ-ಟ್ಟಿತು
ಏರ್ಪ-ಟ್ಟಿತ್ತು
ಏರ್ಪ-ಡಿ-ಸ-ಲಾ-ಯಿತು
ಏರ್ಪ-ಡಿಸಿ
ಏರ್ಪ-ಡಿ-ಸಿದ
ಏರ್ಪ-ಡಿ-ಸಿ-ದ್ದರು
ಏರ್ಪ-ಡಿ-ಸಿ-ದ್ದಾಗ
ಏರ್ಪ-ಡಿ-ಸು-ವು-ದ-ಕ್ಕಾಗಿ
ಏರ್ಪಾಡು
ಏಳ-ದಿ-ರ-ಲಾರ
ಏಳನೇ
ಏಳ-ಬ-ಹುದು
ಏಳ-ಬ-ಹು-ದು-ಸ್ವಾ-ಮೀ-ಜಿಗೇ
ಏಳಿ
ಏಳು
ಏಳು-ತ್ತಲೇ
ಏಳು-ತ್ತಿತ್ತು
ಏಳು-ತ್ತಿ-ದ್ದುವು
ಏಳುವ
ಏಳು-ವಾಗ
ಏಳು-ವು-ದಿಲ್ಲ
ಏಳು-ಸಾ-ವಿರ
ಏಳೆಂಟು
ಏಷ್ಯಾಗೆ
ಏಸು-ಕ್ರಿಸ್ತ
ಏಸು-ಕ್ರಿ-ಸ್ತನ
ಏಸು-ಕ್ರಿ-ಸ್ತರೇ
ಏಸು-ವಿನ
ಐಎ-ಎಸ್
ಐಕ್ಯ-ಹೊಂ-ದು-ವಂ-ತಹ
ಐತಿ-ಹಾ-ಸಿಕ
ಐದ-ರಿಂದ
ಐದಾರು
ಐದು
ಐದೇ
ಐನೂರು
ಐರೋಪ್ಯ
ಐವ-ತ್ತರ
ಐಶ್ವರ್ಯ
ಐಶ್ವ-ರ್ಯವೇ
ಐಹಿಕ
ಒಂಟಿ-ಯಾ-ಗಿಯೆ
ಒಂಟಿ-ಯಾ-ಗಿಯೇ
ಒಂಟೆ-ಗಳ
ಒಂದಂತೂ
ಒಂದಂಶ
ಒಂದಂ-ಶ-ವನ್ನು
ಒಂದಂ-ಶ-ವಾ-ದರೂ
ಒಂದಂ-ಶವು
ಒಂದ-ಕ್ಕೊಂದು
ಒಂದ-ನೆ-ಯ-ದಾಗಿ
ಒಂದನೇ
ಒಂದನ್ನು
ಒಂದ-ರಂದು
ಒಂದ-ರೆಡು
ಒಂದಲ್ಲ
ಒಂದಷ್ಟು
ಒಂದಾ-ಗಿ-ಬಿ-ಟ್ಟಿ-ರು-ವಂತೆ
ಒಂದಾದ
ಒಂದಾ-ದ-ಮೇ-ಲೊಂದು
ಒಂದಾ-ನೊಂದು
ಒಂದಿಂಚು
ಒಂದಿ-ದ್ದಲ್ಲಿ
ಒಂದಿ-ಬ್ಬರ
ಒಂದಿ-ಬ್ಬ-ರನ್ನು
ಒಂದಿ-ಬ್ಬರು
ಒಂದಿ-ರ-ಲಾರು
ಒಂದಿ-ಷ್ಟನ್ನು
ಒಂದಿ-ಷ್ಟಾ-ದರೂ
ಒಂದಿಷ್ಟು
ಒಂದು
ಒಂದು-ಕ್ಷಣ
ಒಂದು-ಗೂ-ಡಿಸಿ
ಒಂದು-ಗೂ-ಡಿ-ಸು-ತ್ತಿ-ರು-ವಂತೆ
ಒಂದು-ತುಂ-ಟ-ತನ
ಒಂದು-ತುಂ-ಡನ್ನು
ಒಂದು-ಧ್ಯಾ-ನ-ಸ್ಥ-ನಾದ
ಒಂದು-ಭ-ಗ-ವ-ನ್ನಾ-ಮ-ದಲ್ಲಿ
ಒಂದೂ-ವರೆ
ಒಂದೆ
ಒಂದೆಡೆ
ಒಂದೆ-ಡೆ-ಯಲ್ಲೇ
ಒಂದೆ-ಡೆ-ಯಿಂದ
ಒಂದೆ-ಯಾ-ಗಿ-ರು-ವಾ-ತನು
ಒಂದೆ-ರಡು
ಒಂದೇ
ಒಂದೇ-ಪ್ರೇಮ
ಒಂದೊಂ-ದಾಗಿ
ಒಂದೊಂದು
ಒಂದೊಂದೇ
ಒಂದೋ
ಒಂದೋ-ನನ್ನ
ಒಂಬತ್ತು
ಒಂಬ-ತ್ತು-ಹತ್ತು
ಒಕ್ಕಣೆ
ಒಗ-ಟಾಗಿ
ಒಗಟು
ಒಗೆದು
ಒಗ್ಗ-ಟ್ಟಾ-ಗಿ-ಡು-ವುದು
ಒಗ್ಗ-ಟ್ಟಾ-ಗಿ-ರ-ದಿ-ದ್ದರೆ
ಒಗ್ಗ-ಟ್ಟಾ-ಗಿ-ರ-ಬೇ-ಕೆಂಬ
ಒಗ್ಗ-ಟ್ಟಾ-ಗಿ-ರು-ವುದು
ಒಗ್ಗ-ಟ್ಟಿ-ನಿಂ-ದಿ-ರ-ಬೇ-ಕಾ-ದರೆ
ಒಗ್ಗಿ-ಕೊಂ-ಡ-ವನು
ಒಗ್ಗಿ-ಹೋ-ಗು-ತ್ತದೆ
ಒಗ್ಗು-ವು-ದಿಲ್ಲ
ಒಗ್ಗು-ವು-ದಿ-ಲ್ಲ-ವೆಂದು
ಒಗ್ಗು-ವು-ದಿ-ಲ್ಲವೋ
ಒಗ್ಗೂ-ಡಿ-ಸಿ-ಕೊಂಡು
ಒಟು-ಟ-ಗೂ-ಡಿಸಿ
ಒಟ್ಟ-ಗೂ-ಡಿಸಿ
ಒಟ್ಟಾಗಿ
ಒಟ್ಟಾ-ಗಿಯೇ
ಒಟ್ಟಾ-ಗಿ-ರ-ಬೇಕು
ಒಟ್ಟಾ-ಗಿ-ರಿ-ಸ-ಬೇಕು
ಒಟ್ಟಾ-ಗಿ-ರು-ವಂತೆ
ಒಟ್ಟಿಗೆ
ಒಟ್ಟಿ-ನಲ್ಲಿ
ಒಟ್ಟು
ಒಟ್ಟು-ಗೂ-ಡಿ-ಸಿ-ಕೊಂಡು
ಒಟ್ಟೊ-ಟ್ಟಿಗೆ
ಒಡ-ಕು-ಗ-ಳುಂ-ಟಾ-ಗು-ತ್ತಿ-ದ್ದುವು
ಒಡ-ನಾ-ಡಿ-ಗಳಿಂದ
ಒಡ-ನಾ-ಡಿ-ಗಳು
ಒಡ-ನಾ-ಡಿ-ದಿರಿ
ಒಡ-ನಾ-ಡುತ್ತ
ಒಡ-ನೆಯೇ
ಒಡ-ಮೂ-ಡಿದ
ಒಡ-ಮೂ-ಡಿ-ರು-ವುದನ್ನು
ಒಡ-ಹು-ಟ್ಟಿದ
ಒಡ-ಹು-ಟ್ಟಿ-ದ-ವರ
ಒಡೆದು
ಒಡೆ-ದು-ಕೊಂ-ಡ-ರೇ-ನಪ್ಪ
ಒಡೆ-ಯ-ನಾದ
ಒಡೆ-ಯಲು
ಒಡ್ಡಿ-ಕೊಂ-ಡರೂ
ಒಡ್ಡಿದ
ಒಣ
ಒಣ-ಗಲು
ಒಣ-ಗಿದ
ಒಣ-ಗಿ-ಹೋ-ಗಲಿ
ಒಣ-ಜ್ಞಾನಿ
ಒಣ-ಹುಲ್ಲು
ಒತ್ತಡ
ಒತ್ತ-ಡಕ್ಕೆ
ಒತ್ತ-ಡ-ಗಳಲ್ಲಿ
ಒತ್ತ-ಡ-ದಿಂದ
ಒತ್ತ-ಡ-ದಿಂ-ದಲೋ
ಒತ್ತ-ಡ-ದಿಂ-ದಾಗಿ
ಒತ್ತನ್ನು
ಒತ್ತ-ರಿ-ಸಿ-ಬಂದು
ಒತ್ತಾಯ
ಒತ್ತಾ-ಯಕ್ಕೆ
ಒತ್ತಾ-ಯದ
ಒತ್ತಾ-ಯ-ದಿಂದ
ಒತ್ತಾ-ಯ-ಪ-ಡಿ-ಸ-ಲಾ-ರಂ-ಭಿಸಿ
ಒತ್ತಾ-ಯ-ಪ-ಡಿಸಿ
ಒತ್ತಾ-ಯ-ಪ-ಡಿ-ಸಿ-ದರು
ಒತ್ತಾ-ಯ-ವನ್ನು
ಒತ್ತಾಯಿ
ಒತ್ತಾ-ಯಿಸ
ಒತ್ತಾ-ಯಿ-ಸ-ಲಾ-ರಂ-ಭಿ-ಸಿದ
ಒತ್ತಾ-ಯಿಸಿ
ಒತ್ತಾ-ಯಿ-ಸಿ-ದ್ದರೂ
ಒತ್ತಿ
ಒತ್ತಿ-ದಷ್ಟೂ
ಒತ್ತಿ-ಬಂದ
ಒತ್ತಿ-ಹೇ-ಳಿ-ದ್ದಾರೆ
ಒತ್ತಿ-ಹೇ-ಳು-ತ್ತಿ-ದ್ದೇ-ನೆ-ನನ್ನ
ಒತ್ತುತ್ತ
ಒದ-ಗ-ಲಿಲ್ಲ
ಒದ-ಗಿ-ದರು
ಒದ-ಗಿ-ದ್ದನ್ನು
ಒದ-ಗಿ-ಬಂತು
ಒದ-ಗಿ-ಬಂದ
ಒದ-ಗಿ-ಬಂ-ದಂ-ತಹ
ಒದ-ಗಿ-ಬಂ-ದರು
ಒದ-ಗಿ-ಬಂ-ದರೂ
ಒದ-ಗಿ-ಬಂ-ದೀತು
ಒದ-ಗಿ-ರುವ
ಒದ-ಗಿ-ಸಿ-ಕೊ-ಡು-ತ್ತಾರೆ
ಒದ-ಗಿ-ಸಿ-ಕೊ-ಡು-ತ್ತಿದ್ದ
ಒದ-ಗಿ-ಸಿ-ಕೊ-ಡು-ವಲ್ಲಿ
ಒದೆ-ಯಲಿ
ಒದ್ದಾಗ
ಒದ್ದಾ-ಟ-ವನ್ನು
ಒದ್ದಾ-ಡ-ಬೇ-ಕಾ-ಗು-ತ್ತದೆ
ಒದ್ದಾ-ಡಿತು
ಒದ್ದಾ-ಡು-ತ್ತಾರೆ
ಒದ್ದಾ-ಡು-ತ್ತಿ-ದ್ದೇನೆ
ಒದ್ದಾ-ಡು-ತ್ತಿ-ರು-ತ್ತದೆ
ಒದ್ದಾ-ಡು-ತ್ತಿ-ರು-ತ್ತಾ-ರೆ-ನೀ-ರು-ಹಾ-ವು
ಒದ್ದಾ-ಡು-ತ್ತಿ-ರು-ವುದ
ಒದ್ದೆ
ಒಪ್ಪ-ಕೊ-ಳ್ಳು-ತ್ತಿ-ರ-ಲಿಲ್ಲ
ಒಪ್ಪ-ದಿ-ದ್ದರೂ
ಒಪ್ಪ-ದಿ-ರು-ವು-ದ-ರಿಂದ
ಒಪ್ಪದೆ
ಒಪ್ಪ-ಲಿಲ್ಲ
ಒಪ್ಪಲೇ
ಒಪ್ಪ-ಲೇ-ಬೇ-ಕಾ-ಗಿತ್ತು
ಒಪ್ಪಿ
ಒಪ್ಪಿ-ಕೊಂಡ
ಒಪ್ಪಿ-ಕೊಂ-ಡರು
ಒಪ್ಪಿ-ಕೊಂ-ಡಿದ್ದು
ಒಪ್ಪಿ-ಕೊಂಡು
ಒಪ್ಪಿ-ಕೊಂ-ಡು-ಬಿಟ್ಟ
ಒಪ್ಪಿ-ಕೊಂ-ಡು-ಬಿ-ಡ-ಬೇಕೋ
ಒಪ್ಪಿ-ಕೊ-ಳ್ಲ-ದೆ-ಯಿ-ದ್ದ-ಮೇಲೆ
ಒಪ್ಪಿ-ಕೊ-ಳ್ಳ-ದಿ-ದ್ದರೆ
ಒಪ್ಪಿ-ಕೊ-ಳ್ಳದೆ
ಒಪ್ಪಿ-ಕೊ-ಳ್ಳ-ಬ-ಹು-ದಾ-ಗಿತ್ತು
ಒಪ್ಪಿ-ಕೊ-ಳ್ಳ-ಬೇ-ಕಾ-ಗಿಲ್ಲ
ಒಪ್ಪಿ-ಕೊ-ಳ್ಳ-ಬೇ-ಕಾ-ಯಿತು
ಒಪ್ಪಿ-ಕೊ-ಳ್ಳ-ಲಾ-ಗು-ವು-ದಿಲ್ಲ
ಒಪ್ಪಿ-ಕೊ-ಳ್ಳ-ಲಿಲ್ಲ
ಒಪ್ಪಿ-ಕೊ-ಳ್ಳಲೇ
ಒಪ್ಪಿ-ಕೊ-ಳ್ಳ-ಲೇ-ಬೇ-ಕಾ-ಯಿತು
ಒಪ್ಪಿ-ಕೊ-ಳ್ಳ-ವು-ದಷ್ಚೇ
ಒಪ್ಪಿ-ಕೊಳ್ಳು
ಒಪ್ಪಿ-ಕೊ-ಳ್ಳು-ತ್ತದೆ
ಒಪ್ಪಿ-ಕೊ-ಳ್ಳು-ತ್ತಾನೆ
ಒಪ್ಪಿ-ಕೊ-ಳ್ಳು-ತ್ತಾರೆ
ಒಪ್ಪಿ-ಕೊ-ಳ್ಳು-ತ್ತಿದ್ದ
ಒಪ್ಪಿ-ಕೊ-ಳ್ಳು-ತ್ತಿ-ದ್ದಾ-ನಲ್ಲ
ಒಪ್ಪಿ-ಕೊ-ಳ್ಳು-ತ್ತಿ-ದ್ದಾರೆ
ಒಪ್ಪಿ-ಕೊ-ಳ್ಳು-ತ್ತಿ-ರ-ಲಿಲ್ಲ
ಒಪ್ಪಿ-ಕೊ-ಳ್ಳುವ
ಒಪ್ಪಿ-ಕೊ-ಳ್ಳು-ವಂ-ತಾ-ಗಲು
ಒಪ್ಪಿ-ಕೊ-ಳ್ಳು-ವ-ವ-ನಲ್ಲ
ಒಪ್ಪಿ-ಕೊ-ಳ್ಳು-ವ-ವ-ರೆಗೆ
ಒಪ್ಪಿ-ಕೊ-ಳ್ಳು-ವು-ದಿ-ಲ್ಲ-ವಲ್ಲ
ಒಪ್ಪಿ-ಕೊ-ಳ್ಳು-ವುದು
ಒಪ್ಪಿಗೆ
ಒಪ್ಪಿ-ಗೆ-ಯಾ-ಗ-ಲಿಲ್ಲ
ಒಪ್ಪಿ-ಗೆ-ಯಾ-ಗಿಲ್ಲ
ಒಪ್ಪಿ-ಗೆ-ಯಾಗು
ಒಪ್ಪಿ-ಗೆ-ಯಾ-ಗು-ತ್ತಿ-ರ-ಲಿಲ್ಲ
ಒಪ್ಪಿ-ಗೆ-ಯಾ-ಗು-ವಂತೆ
ಒಪ್ಪಿ-ಗೆ-ಯಾ-ಗು-ವು-ದ-ಲ್ಲದೆ
ಒಪ್ಪಿ-ಗೆ-ಯಾ-ದವು
ಒಪ್ಪಿ-ಗೆ-ಯಾ-ಯಿತು
ಒಪ್ಪಿತು
ಒಪ್ಪಿ-ದರು
ಒಪ್ಪಿ-ದ-ವನೇ
ಒಪ್ಪಿದ್ದು
ಒಪ್ಪಿ-ಯಾರು
ಒಪ್ಪಿ-ಸಿ-ಕೊ-ಟ್ಟ-ಮೇಲೆ
ಒಪ್ಪಿ-ಸಿ-ಕೊ-ಟ್ಟರು
ಒಪ್ಪಿ-ಸಿದ
ಒಪ್ಪಿ-ಸಿ-ದರು
ಒಪ್ಪಿ-ಸಿ-ದ್ದ-ರಿಂ-ದಲೇ
ಒಪ್ಪಿ-ಸಿ-ಬಿಟ್ಟ
ಒಪ್ಪಿ-ಸಿ-ಬಿ-ಟ್ಟರು
ಒಪ್ಪಿ-ಸಿ-ರು-ವುದು
ಒಪ್ಪಿ-ಸುವ
ಒಪ್ಪು-ತ್ತ-ದೆಯೆ
ಒಪ್ಪು-ತ್ತಾನೆ
ಒಪ್ಪು-ತ್ತಾರೆ
ಒಪ್ಪು-ತ್ತಿ-ದ್ದರು
ಒಪ್ಪು-ತ್ತಿ-ರ-ಲಿಲ್ಲ
ಒಪ್ಪು-ತ್ತಿಲ್ಲ
ಒಪ್ಪು-ವ-ವನೇ
ಒಪ್ಪು-ವು-ದಾ-ದರೆ
ಒಪ್ಪು-ವೆ-ಯ-ಲ್ಲವೆ
ಒಬ್ಬ
ಒಬ್ಬಂ-ಟಿ-ಗ-ನಾ-ಗಿ-ದ್ದಾ-ನಲ್ಲ
ಒಬ್ಬನ
ಒಬ್ಬ-ನನ್ನು
ಒಬ್ಬ-ನಲ್ಲಿ
ಒಬ್ಬ-ನಾಗಿ
ಒಬ್ಬ-ನಿ-ಗಾಗಿ
ಒಬ್ಬನು
ಒಬ್ಬನೂ
ಒಬ್ಬನೇ
ಒಬ್ಬರ
ಒಬ್ಬ-ರ-ನ್ನಾ-ದರೂ
ಒಬ್ಬ-ರನ್ನು
ಒಬ್ಬ-ರಲ್ಲ
ಒಬ್ಬ-ರಾ-ದರೂ
ಒಬ್ಬ-ರಿಂ
ಒಬ್ಬ-ರಿ-ಗಿಂತ
ಒಬ್ಬ-ರಿಗೆ
ಒಬ್ಬರು
ಒಬ್ಬರೇ
ಒಬ್ಬ-ಳನ್ನು
ಒಬ್ಬ-ಳಿ-ದ್ದಳು
ಒಬ್ಬಳು
ಒಬ್ಬಳೇ
ಒಬ್ಬಾ-ತನ
ಒಬ್ಬಿ-ಬ್ಬ-ರನ್ನು
ಒಬ್ಬಿ-ಬ್ಬರು
ಒಬ್ಬೊ-ಬ್ಬನ
ಒಬ್ಬೊ-ಬ್ಬನೂ
ಒಬ್ಬೊ-ಬ್ಬ-ರಾಗಿ
ಒಬ್ಬೊ-ಬ್ಬರು
ಒಮ್ಮ-ತ-ದಿಂದ
ಒಮ್ಮ-ತ-ದಿಂ-ದಿ-ದ್ದು-ಕೊಂಡು
ಒಮ್ಮೆ
ಒಮ್ಮೆಗೇ
ಒಮ್ಮೆ-ಯಂತೂ
ಒಮ್ಮೆ-ಯಾ-ದರೂ
ಒಮ್ಮೊಮ್ಮೆ
ಒಯ್ದಿ-ದ್ದರು
ಒಯ್ಯ-ಲಿ-ರುವ
ಒಯ್ಯುವ
ಒಯ್ಯು-ವಂತೆ
ಒರಗಿ
ಒರ-ಗಿ-ಕೊಂ-ಡರು
ಒರ-ಗಿ-ಕೊಂ-ಡರೆ
ಒರ-ಗಿ-ಕೊಂ-ಡಿ-ರು-ತ್ತಿದ್ದ
ಒರ-ಗಿ-ಕೊಂಡು
ಒರ-ಗಿಸಿ
ಒರಟು
ಒರ-ಟು-ಮಾ-ತು-ಗಳನ್ನು
ಒರ-ಟೊ-ರ-ಟಾಗಿ
ಒರೆ-ಗ-ಲ್ಲಿಗೆ
ಒರೆ-ಸಿ-ಕೊ-ಳ್ಳುತ್ತ
ಒಲ-ವಿ-ದ್ದ-ವನು
ಒಲವು
ಒಲವೂ
ಒಲಿದ
ಒಲಿ-ಯನು
ಒಲಿ-ಸಿ-ಕೊ-ಳ್ಳಲು
ಒಲೆ
ಒಲೆ-ಗುಂ-ಡು-ಕಲ್ಲು
ಒಲ್ಲದ
ಒಳ
ಒಳಕ್ಕೆ
ಒಳ-ಗಡೆ
ಒಳ-ಗಾ-ಗದೆ
ಒಳ-ಗಾ-ಗ-ಬಾ-ರದು
ಒಳ-ಗಾ-ಗಲೇ
ಒಳ-ಗಾ-ಗ-ಲೇ-ಬಾ-ರದು
ಒಳ-ಗಾ-ಗ-ಲೇ-ಬಾ-ರ-ದೆಂದು
ಒಳ-ಗಾಗಿ
ಒಳ-ಗಾ-ಗಿ-ಬಿ-ಡು-ತ್ತಾನೋ
ಒಳ-ಗಾ-ಗಿ-ರು-ವು-ದ-ರಿಂ-ದಲೇ
ಒಳ-ಗಾ-ಗು-ತ್ತಾರೆ
ಒಳ-ಗಾ-ಗು-ತ್ತಿದ್ದ
ಒಳ-ಗಾ-ಗು-ವ-ವ-ನಲ್ಲ
ಒಳ-ಗಾ-ದರು
ಒಳ-ಗಾ-ದರೂ
ಒಳ-ಗಿಂದ
ಒಳ-ಗಿ-ರುವ
ಒಳಗೂ
ಒಳಗೆ
ಒಳ-ಗೆಲ್ಲ
ಒಳಗೇ
ಒಳ-ಗೇ-ನಿ-ದೆ-ಯೆಂದು
ಒಳ-ಗೊಂಡ
ಒಳ-ಗೊಂ-ದೆ-ಣಿಸಿ
ಒಳ-ಗೊ-ಳಗೇ
ಒಳ-ಜ-ಗ-ಳ-ಗ-ಳೆಲ್ಲ
ಒಳ-ಪ-ಟ್ಟದ್ದು
ಒಳ-ಪ-ಟ್ಟಿರು
ಒಳ-ಪ-ಡಿ-ಸಿದ
ಒಳ-ಪ-ಡಿ-ಸಿ-ದ-ಹೊ-ರ-ತು-ನಂ-ಬ-ಲಾರೆ
ಒಳ-ವ-ಲ-ಯ-ದೊ-ಳಗೆ
ಒಳ-ಶುಂಠಿ
ಒಳ-ಹೊಕ್ಕು
ಒಳ-ಹೊ-ರ-ಗನ್ನು
ಒಳ-ಹೊ-ರ-ಗು-ಗಳನ್ನೂ
ಒಳ-ಹೊ-ರ-ಗೆಲ್ಲ
ಒಳಿ-ತನ್ನು
ಒಳಿ-ತಿನ
ಒಳ್ಳೆ
ಒಳ್ಳೆಯ
ಒಳ್ಳೆ-ಯ-ದಕ್ಕೆ
ಒಳ್ಳೆ-ಯ-ದಲ್ಲ
ಒಳ್ಳೆ-ಯ-ದ-ಲ್ಲವೆ
ಒಳ್ಳೆ-ಯ-ದಾ-ಗಲಿ
ಒಳ್ಳೆ-ಯ-ದಾ-ಯಿತು
ಒಳ್ಳೆ-ಯದು
ಒಳ್ಳೆ-ಯದೆ
ಒಳ್ಳೆ-ಯದೇ
ಒಳ್ಳೆ-ಯ-ದೇನೂ
ಒಳ್ಳೆ-ಯ-ವ-ನಾ-ದರೆ
ಒಳ್ಳೆ-ಯ-ವನು
ಒಳ್ಳೆ-ಯ-ವ-ರಾ-ದರೂ
ಒಳ್ಳೆ-ಯವೇ
ಒಳ್ಳೇ
ಒಳ್ಳೇ-ಯ-ವರೇ
ಒಳ್ಳೊ-ಳ್ಳೆಯ
ಓ
ಓಂ
ಓಂಕಾರ
ಓಂಕಾ-ರ-ವನ್ನು
ಓಂಕಾ-ರ-ವ-ನ್ನು-ಚ್ಚ-ರಿ-ಸುತ್ತ
ಓಗೊಟ್ಟು
ಓಗೊ-ಡ-ಲಿಲ್ಲ
ಓಗೊ-ಡು-ತ್ತಿದ್ದ
ಓಜೋ-ಮ-ಯ-ವಾದ
ಓಟ
ಓಟ-ಕಿ-ತ್ತುವು
ಓಟದ
ಓಡ-ದಾ-ದರು
ಓಡ-ಲಾ-ರಂ-ಭಿ-ಸಿ-ದರು
ಓಡಾ
ಓಡಾ-ಡ-ಬೇ-ಕಾ-ಗಿತ್ತು
ಓಡಾಡಿ
ಓಡಾ-ಡಿ-ಕೊಂ-ಡಿ-ದ್ದರೆ
ಓಡಾ-ಡು-ತ್ತಿದ್ದ
ಓಡಾ-ಡು-ತ್ತಿ-ರುವ
ಓಡಾ-ಡು-ತ್ತಿ-ರು-ವುದನ್ನು
ಓಡಾ-ಡುವ
ಓಡಾ-ಡು-ವುದು
ಓಡಿ
ಓಡಿದ
ಓಡಿ-ದಂ-ತೆಲ್ಲ
ಓಡಿ-ದರು
ಓಡಿದೆ
ಓಡಿ-ಬಂದ
ಓಡಿ-ಬಂ-ದರು
ಓಡಿ-ಬಂದು
ಓಡಿ-ಸಲು
ಓಡಿ-ಸಲೂ
ಓಡಿಸಿ
ಓಡಿ-ಸಿ-ಕೊಂಡು
ಓಡಿ-ಸೋ-ಣ-ವೆಂದರೆ
ಓಡಿ-ಹೋ-ಗ-ದಿ-ರ-ಲೆಂದು
ಓಡಿ-ಹೋ-ಗ-ಬೇಕು
ಓಡಿ-ಹೋ-ಗಲು
ಓಡಿ-ಹೋಗಿ
ಓಡಿ-ಹೋ-ಗಿ-ಬಿ-ಟ್ಟೆ-ನಲ್ಲ
ಓಡಿ-ಹೋಗು
ಓಡಿ-ಹೋ-ಗುವ
ಓಡು-ತ್ತಾನೆ
ಓಡು-ತ್ತಿತ್ತು
ಓಡು-ತ್ತಿದ್ದ
ಓಡು-ತ್ತಿ-ದ್ದ-ವರು
ಓಡು-ತ್ತಿ-ದ್ದಾರೆ
ಓಡು-ತ್ತಿ-ದ್ದುದು
ಓಡು-ತ್ತಿ-ರುವ
ಓಡುವ
ಓಡೋಡಿ
ಓದ
ಓದ-ಬೇ-ಕಾದ
ಓದ-ರಿ-ಯ-ದ-ವರ
ಓದಲು
ಓದಲೂ
ಓದ-ಲೇ-ಬೇ-ಕಾ-ಗಿ-ರ-ಲಿಲ್ಲ
ಓದಿ
ಓದಿ-ಓದಿ
ಓದಿ-ಕೊಂ-ಡಿ-ದ್ದಾನೆ
ಓದಿ-ಕೊಂಡು
ಓದಿ-ಕೊ-ಳ್ಳು-ತ್ತಿ-ದ್ದಾಗ
ಓದಿದ
ಓದಿ-ದರೆ
ಓದಿ-ದಾ-ಗಲೂ
ಓದಿ-ದ್ದರು
ಓದಿ-ದ್ದೀರೋ
ಓದಿದ್ದು
ಓದಿದ್ದೇ
ಓದಿ-ದ್ದೇನೆ
ಓದಿನ
ಓದಿ-ನೋಡಿ
ಓದಿ-ಬ-ರೆ-ಯುವ
ಓದಿ-ಬಿ-ಟ್ಟರು
ಓದಿ-ಬಿ-ಟ್ಟರೆ
ಓದಿ-ಬಿ-ಟ್ಟಿ-ದ್ದ-ನೆಂ-ದರೆ
ಓದಿ-ಬಿ-ಟ್ಟಿರಾ
ಓದಿಯೇ
ಓದಿ-ರ-ಲಿಲ್ಲ
ಓದಿ-ರು-ವು-ದ-ನ್ನೆಲ್ಲ
ಓದಿಲ್ಲ
ಓದಿ-ಲ್ಲವೆ
ಓದಿಸಿ
ಓದಿ-ಸು-ತ್ತಿ-ದ್ದರು
ಓದು
ಓದು-ಗೀ-ದಿನ
ಓದುತ್ತ
ಓದು-ತ್ತಿದ್ದ
ಓದು-ತ್ತಿ-ದ್ದರೆ
ಓದು-ತ್ತಿ-ದ್ದಾ-ಗಲೇ
ಓದು-ತ್ತಿ-ದ್ದಾನೆ
ಓದು-ತ್ತಿ-ರ-ಲಿಲ್ಲ
ಓದು-ತ್ತಿ-ರುವ
ಓದು-ತ್ತಿ-ರು-ವಂ-ತೆಯೇ
ಓದು-ತ್ತಿ-ರು-ವಾ-ಗಲೇ
ಓದು-ತ್ತೇನೆ
ಓದು-ತ್ತೇ-ನೆ-ನ-ರೇಂ-ದ್ರ-ನೆಂದ
ಓದು-ಬ-ರಹ
ಓದು-ಬ-ರ-ಹದ
ಓದುವ
ಓದು-ವಾಗ
ಓದು-ವು-ದಕ್ಕೆ
ಓದು-ವುದನ್ನು
ಓದು-ವು-ದ-ರಿಂದ
ಓದು-ವು-ದಿಲ್ಲ
ಓದು-ವುದೇ
ಓರ-ಗೆ-ಯ-ವ-ರಿ-ಗೆಲ್ಲ
ಓಲಾ-ಡುವ
ಓಳ್ಳೇ
ಓಹೋ
ಓಹ್
ಔತಣ
ಔತ-ಣಕ್ಕೆ
ಔದಾರ್ಯ
ಔದಾ-ರ್ಯ-ಪೌ-ರು-ಷ-ಗಳನ್ನೂ
ಔದಾ-ರ್ಯ-ಸೌ-ಜ-ನ್ಯ-ಗಳು
ಔದಾ-ರ್ಯದ
ಔದಾ-ರ್ಯ-ದೊಂ-ದಿಗೆ
ಔದಾ-ರ್ಯ-ವನ್ನು
ಔದಾ-ಸೀನ್ಯ
ಔದಾ-ಸೀ-ನ್ಯ-ವನ್ನು
ಔದಾ-ಸೀ-ನ್ಯ-ವೇನು
ಔಪ-ಚಾ-ರಿಕ
ಔಷಧ
ಔಷ-ಧ-ಆ-ಹಾ-ರ-ಗಳನ್ನು
ಔಷ-ಧ-ಗ-ಳಾ-ವೂವೂ
ಔಷ-ಧ-ವನ್ನು
ಔಷ-ಧಿ-ಪ-ಥ್ಯ-ಗಳನ್ನು
ಔಷ-ಧಿ-ಯನ್ನು
ಔಷ-ಧಿ-ಯೇ-ನಾ-ದರೂ
ಔಷ-ಧೋ-ಪ-ಚಾರ
ಔಷ-ಧೋ-ಪ-ಚಾ-ರ-ಗಳ
ಕ
ಕಂಕ-ಣ-ಬದ್ಧ
ಕಂಕು-ಮ-ವನ್ನು
ಕಂಕು-ಳ-ಲ್ಲಿ-ಟ್ಟು-ಕೊಂಡು
ಕಂಖಲ್
ಕಂಗಳಲ್ಲಿ
ಕಂಗ-ಳಿಂ
ಕಂಗಳಿಂದ
ಕಂಗ-ಳಿಗೆ
ಕಂಗಳು
ಕಂಗಾ-ಲದ
ಕಂಗಾ-ಲಾ-ಗದೆ
ಕಂಗಾ-ಲಾಗಿ
ಕಂಗಾ-ಲಾ-ಗು-ವುದನ್ನು
ಕಂಗಾ-ಲಾ-ಗು-ವುದು
ಕಂಗಾ-ಲಾದ
ಕಂಗಾ-ಲಾ-ದರು
ಕಂಗೊ-ಳಿ-ಸ-ಲಾ-ರಂ-ಭಿ-ಸಿದೆ
ಕಂಗೊ-ಳಿ-ಸು-ತ್ತಿ-ರು-ವುದನ್ನು
ಕಂಗೊ-ಳಿ-ಸುವ
ಕಂಟ-ಕ-ಗಳು
ಕಂಠ
ಕಂಠ-ಗಳಿಂದ
ಕಂಠ-ದಂತೆ
ಕಂಠ-ದಿಂದ
ಕಂಠ-ಮಾ-ಧು-ರ್ಯ-ವನ್ನು
ಕಂಠ-ಮಾ-ಧು-ರ್ಯ-ವಿದ್ದ
ಕಂಠ-ವನ್ನು
ಕಂಠವೂ
ಕಂಠಶ್ರೀ
ಕಂಠ-ಶ್ರೀ-ಯಿಂದ
ಕಂಠಸ್ಥ
ಕಂಠ-ಸ್ವರ
ಕಂಠ-ಸ್ವ-ರ-ವನ್ನು
ಕಂಠ-ಸ್ವ-ರವೇ
ಕಂಡ
ಕಂಡಂತೆ
ಕಂಡಂ-ತೆಯೇ
ಕಂಡ-ಕೂ-ಡಲೇ
ಕಂಡ-ದ್ದನ್ನೇ
ಕಂಡದ್ದು
ಕಂಡ-ದ್ದೇನು
ಕಂಡ-ಮೇ-ಲಂತೂ
ಕಂಡ-ಮೇಲೆ
ಕಂಡ-ರಪ್ಪ
ಕಂಡ-ರಾ-ಗದು
ಕಂಡ-ರಿ-ಯದ
ಕಂಡ-ರಿ-ಯರು
ಕಂಡರು
ಕಂಡರೂ
ಕಂಡರೆ
ಕಂಡರೇ
ಕಂಡರೋ
ಕಂಡ-ವ-ನಲ್ಲ
ಕಂಡ-ವನು
ಕಂಡ-ವನೇ
ಕಂಡ-ವ-ರಲ್ಲ
ಕಂಡ-ವ-ರಲ್ಲಿ
ಕಂಡ-ವ-ರಾರು
ಕಂಡ-ವ-ರಿ-ಗೆಲ್ಲ
ಕಂಡ-ವ-ರಿ-ದ್ದಾರೆ
ಕಂಡ-ವ-ರಿ-ರ-ಬ-ಹು-ದ-ಲ್ಲವೆ
ಕಂಡ-ವರು
ಕಂಡಾಗ
ಕಂಡಾ-ಗ-ಲಂತೂ
ಕಂಡಿ
ಕಂಡಿತು
ಕಂಡಿ-ತು-ಅದೂ
ಕಂಡಿದ್ದ
ಕಂಡಿ-ದ್ದರು
ಕಂಡಿ-ದ್ದಾನೆ
ಕಂಡಿ-ದ್ದೀರಾ
ಕಂಡಿದ್ದೆ
ಕಂಡಿ-ದ್ದೇನೆ
ಕಂಡಿ-ದ್ದೇ-ನೆ-ಅ-ವರೇ
ಕಂಡಿ-ದ್ದೇವೆ
ಕಂಡಿ-ರ-ಬ-ಹುದು
ಕಂಡಿ-ರ-ಬೇಕು
ಕಂಡಿ-ರ-ಲಾ-ರರು
ಕಂಡಿ-ರ-ಲಿ-ಕ್ಕಿಲ್ಲ
ಕಂಡಿ-ರ-ಲಿಲ್ಲ
ಕಂಡಿ-ರು-ತ್ತಾರೆ
ಕಂಡಿ-ರುವ
ಕಂಡಿಲ್ಲ
ಕಂಡು
ಕಂಡು-ಕೇ-ಳಿದ
ಕಂಡು-ಕೊಂಡ
ಕಂಡು-ಕೊಂ-ಡದ್ದು
ಕಂಡು-ಕೊಂ-ಡರು
ಕಂಡು-ಕೊಂ-ಡಳು
ಕಂಡು-ಕೊಂ-ಡಿದ್ದ
ಕಂಡು-ಕೊಂ-ಡಿ-ದ್ದರು
ಕಂಡು-ಕೊಂ-ಡಿ-ದ್ದಾರೆ
ಕಂಡು-ಕೊಂಡು
ಕಂಡು-ಕೊಂಡೆ
ಕಂಡು-ಕೊ-ಳ್ಳ-ಬ-ಹು-ದಾ-ಗಿದೆ
ಕಂಡು-ಕೊ-ಳ್ಳ-ಬ-ಹುದು
ಕಂಡು-ಕೊ-ಳ್ಳ-ಲಾ-ರಈ
ಕಂಡು-ಕೊ-ಳ್ಳಲು
ಕಂಡು-ಕೊ-ಳ್ಳ-ಲೇ-ಬೇಕು
ಕಂಡು-ಕೊಳ್ಳಿ
ಕಂಡು-ಕೊಳ್ಳು
ಕಂಡು-ಕೊ-ಳ್ಳು-ತ್ತಾನೋ
ಕಂಡು-ಕೊ-ಳ್ಳು-ತ್ತಾರೆ
ಕಂಡು-ಕೊ-ಳ್ಳುತ್ತಿ
ಕಂಡು-ಕೊ-ಳ್ಳು-ತ್ತಿದ್ದ
ಕಂಡು-ಕೊ-ಳ್ಳುವ
ಕಂಡು-ಕೊ-ಳ್ಳು-ವುದು
ಕಂಡು-ಬಂ-ದರು
ಕಂಡು-ಬಂ-ದರೂ
ಕಂಡು-ಬಂ-ದರೆ
ಕಂಡು-ಬಂ-ದಾಗ
ಕಂಡು-ಬಂ-ದಿತು
ಕಂಡು-ಬಂ-ದಿ-ರ-ಬೇಕು
ಕಂಡು-ಬಂದು
ಕಂಡು-ಬಂ-ದುವು
ಕಂಡು-ಬ-ರ-ಲಿಲ್ಲ
ಕಂಡು-ಬ-ರು-ತ್ತದೆ
ಕಂಡು-ಬ-ರು-ತ್ತಿತ್ತು
ಕಂಡು-ಬ-ರು-ತ್ತಿದೆ
ಕಂಡು-ಬ-ರು-ತ್ತಿದ್ದ
ಕಂಡು-ಬ-ರು-ತ್ತಿ-ದ್ದರು
ಕಂಡು-ಬ-ರು-ತ್ತಿ-ದ್ದಾರೆ
ಕಂಡು-ಬ-ರು-ತ್ತಿ-ರು-ವಾಗ
ಕಂಡು-ಬ-ರು-ತ್ತಿವೆ
ಕಂಡು-ಬ-ರುವ
ಕಂಡು-ಬ-ರು-ವಂ-ತಹ
ಕಂಡು-ಬ-ರು-ವಂ-ಥ-ವು-ಗಳು
ಕಂಡು-ಬ-ರು-ವು-ದಿಲ್ಲ
ಕಂಡು-ಹಿ-ಡಿ-ದಿಲ್ಲ
ಕಂಡೆ
ಕಂಡೆನೋ
ಕಂಡೇ
ಕಂಡೊ-ಡನೆ
ಕಂಡೊ-ಡ-ನೆಯೇ
ಕಂತೆ
ಕಂತೆ-ಪು-ರಾಣ
ಕಂತೆ-ಯನ್ನು
ಕಂತೆಯೇ
ಕಂದ-ಕ-ವುಂ-ಟಾ-ಗ-ತೊಡ
ಕಂದ-ನೆಂ-ಬಂತೆ
ಕಂದಾ-ಚಾ-ರ-ಗಳನ್ನೂ
ಕಂದಾ-ಚಾ-ರ-ಗಳಿಂದ
ಕಂದಾ-ಚಾ-ರ-ಗಳು
ಕಂಪ-ನ-ವುಂ-ಟಾ-ಯಿತು
ಕಂಬನಿ
ಕಂಬ-ನಿ-ಗ-ರೆದು
ಕಂಬ-ನಿ-ದುಂಬಿ
ಕಂಬ-ನಿ-ದುಂ-ಬಿತು
ಕಂಬಳಿ
ಕಂಬ-ಳಿ-ಗಳ
ಕಂಬಿ-ಗ-ಳಿಗೆ
ಕಂಬಿ-ಗಳೋ
ಕಕ್ಕಾ-ಬಿ-ಕ್ಕಿ-ಯಾ-ಯಿತು
ಕಕ್ಷಿ-ದಾರ
ಕಕ್ಷಿ-ದಾ-ರ-ನಿಗೆ
ಕಕ್ಷಿ-ದಾ-ರ-ರನ್ನು
ಕಕ್ಷಿ-ದಾ-ರ-ರಿ-ಗೆಲ್ಲ
ಕಕ್ಷಿ-ದಾ-ರರು
ಕಕ್ಷಿ-ದಾ-ರ-ರೆಲ್ಲ
ಕಗ್ಗ
ಕಗ್ಗವ
ಕಚ್ಚಲು
ಕಚ್ಚಲೇ
ಕಛೇರಿ
ಕಛೇ-ರಿ-ಗಳನ್ನು
ಕಛೇ-ರಿಗೆ
ಕಛೇ-ರಿ-ಯಲ್ಲಿ
ಕಛೇ-ರಿ-ಯಿಂದ
ಕಟಿ-ತ-ಟ-ಕೌ-ಪೀ-ನ-ವಂ-ತಮ್
ಕಟು-ತರ
ಕಟು-ವಾಗಿ
ಕಟು-ವಾದ
ಕಟು-ಶ-ಬ್ದ-ಗಳನ್ನು
ಕಟ್ಟ-ಕ-ಡೆಗೆ
ಕಟ್ಟಡ
ಕಟ್ಟ-ಡಕ್ಕೆ
ಕಟ್ಟ-ಡದ
ಕಟ್ಟ-ಡ-ದಲ್ಲಿ
ಕಟ್ಟ-ಡ-ವನ್ನು
ಕಟ್ಟ-ಡ-ವ-ನ್ನೇಕೆ
ಕಟ್ಟ-ಡ-ವ-ನ್ನೊಮ್ಮೆ
ಕಟ್ಟ-ಡ-ವಲ್ಲ
ಕಟ್ಟ-ದಿ-ದ್ದರೆ
ಕಟ್ಟನೆ
ಕಟ್ಟ-ಬೇ-ಕಾ-ಗು-ತ್ತದೆ
ಕಟ್ಟ-ಬೇಕು
ಕಟ್ಟ-ಲಾ-ಗಿತ್ತು
ಕಟ್ಟ-ಲಾ-ಯಿತು
ಕಟ್ಟಲು
ಕಟ್ಟಲೂ
ಕಟ್ಟಾ
ಕಟ್ಟಿ
ಕಟ್ಟಿ-ಕೊಂ-ಡಿ-ರ-ಬೇ-ಕಾ-ಗಿತ್ತು
ಕಟ್ಟಿ-ಕೊಂಡು
ಕಟ್ಟಿ-ಕೊ-ಳ್ಳು-ತ್ತಿದ್ದೆ
ಕಟ್ಟಿ-ಕೊ-ಳ್ಳೋ-ಣವೆ
ಕಟ್ಟಿ-ಗೆಯ
ಕಟ್ಟಿದ
ಕಟ್ಟಿ-ದಂ-ತಾ-ಯಿತು
ಕಟ್ಟಿ-ದರು
ಕಟ್ಟಿ-ದರೆ
ಕಟ್ಟಿದೆ
ಕಟ್ಟಿದ್ದ
ಕಟ್ಟಿ-ಬಿ-ಗಿ-ದಿಹ
ಕಟ್ಟಿಯೇ
ಕಟ್ಟಿ-ಸ-ಬೇಕು
ಕಟ್ಟಿ-ಸಲು
ಕಟ್ಟಿ-ಸಿ-ಕೊ-ಟ್ಟರು
ಕಟ್ಟಿ-ಸಿ-ಕೊ-ಡಲು
ಕಟ್ಟಿ-ಸಿ-ದ-ವಳು
ಕಟ್ಟಿ-ಸಿದ್ದ
ಕಟ್ಟಿ-ಸಿ-ದ್ದಾರೆ
ಕಟ್ಟಿ-ಹಾ-ಕ-ಲಾ-ರಳು
ಕಟ್ಟಿ-ಹಾ-ಕಿ-ಬಿ-ಡು-ತ್ತಿತ್ತು
ಕಟ್ಟಿ-ಹೋ-ಯಿತು
ಕಟ್ಟು
ಕಟ್ಟು-ಕತೆ
ಕಟ್ಟು-ಕ-ತೆಯೇ
ಕಟ್ಟು-ಕ-ಥೆ-ಗಳೆ
ಕಟ್ಟು-ಗಳು
ಕಟ್ಟು-ಗ-ಳೆಲ್ಲ
ಕಟ್ಟು-ತ್ತದೆ
ಕಟ್ಟು-ತ್ತಿತ್ತು
ಕಟ್ಟು-ನಿ-ಟ್ಟನ
ಕಟ್ಟು-ನಿ-ಟ್ಟಾಗಿ
ಕಟ್ಟು-ನಿ-ಟ್ಟಾದ
ಕಟ್ಟು-ನಿ-ಟ್ಟಿನ
ಕಟ್ಟು-ಪಾಡು
ಕಟ್ಟು-ಪಾ-ಡು-ಗಳನ್ನು
ಕಟ್ಟು-ಬಿದ್ದು
ಕಟ್ಟು-ಮ-ಸ್ತಾಗಿ
ಕಟ್ಟು-ಮ-ಸ್ತಾ-ಗಿ-ರು-ವುದನ್ನು
ಕಟ್ಟು-ಮ-ಸ್ತಾದ
ಕಟ್ಟು-ವಷ್ಟು
ಕಠಿಣ
ಕಠೋಪ
ಕಠೋರ
ಕಡಮೆ
ಕಡ-ಮೆ-ಯಾ-ಗಲೇ
ಕಡ-ಮೆ-ಯೇ-ನಲ್ಲ
ಕಡಿ
ಕಡಿ-ದಾದ
ಕಡಿದು
ಕಡಿ-ದು-ಕೊಂಡು
ಕಡಿ-ದು-ಕೊಳ್ಳ
ಕಡಿಮೆ
ಕಡಿ-ಮೆ-ಯ-ದೇ-ನಲ್ಲ
ಕಡಿ-ಮೆ-ಯಾ-ಗ-ಬ-ಹು-ದೆಂದು
ಕಡಿ-ಮೆ-ಯಾ-ಗ-ಲಿಲ್ಲ
ಕಡಿ-ಮೆ-ಯಾ-ಗಿತ್ತು
ಕಡಿ-ಮೆ-ಯಾ-ಗಿ-ರ-ಲಿಲ್ಲ
ಕಡಿ-ಮೆ-ಯಾ-ಗುತ್ತ
ಕಡಿ-ಮೆ-ಯಾ-ಗುವ
ಕಡಿ-ಮೆ-ಯಾ-ಗು-ವಂ-ಥ-ದ-ವಲ್ಲ
ಕಡಿ-ಮೆ-ಯಾ-ಯಿತು
ಕಡಿ-ಮೆ-ಯೆಂ-ದರೆ
ಕಡಿ-ಮೆ-ಯೇ-ನಿ-ರ-ಲಿಲ್ಲ
ಕಡಿ-ವಾಣ
ಕಡು-ಬ-ಡ-ತನ
ಕಡು-ವಿ-ರೋ-ಧಿ-ಯಾ-ಗಿದ್ದ
ಕಡೆ
ಕಡೆ-ಗ-ಣಿ-ಸ-ಬಾ-ರದು
ಕಡೆ-ಗ-ಣಿ-ಸ-ಲಾರೆ
ಕಡೆ-ಗ-ಣಿ-ಸಲು
ಕಡೆ-ಗ-ಣಿ-ಸಿದ್ದು
ಕಡೆ-ಗ-ಣಿ-ಸಿ-ಬಿ-ಟ್ಟಿ-ದ್ದ-ರುಈ
ಕಡೆ-ಗ-ಣಿ-ಸು-ವ-ವ-ನಲ್ಲ
ಕಡೆ-ಗ-ಣಿ-ಸು-ವು-ದಾ-ಗಲಿ
ಕಡೆ-ಗಷ್ಟೇ
ಕಡೆ-ಗಿನ
ಕಡೆಗೂ
ಕಡೆಗೆ
ಕಡೆಗೇ
ಕಡೆ-ಗೊಂದು
ಕಡೆ-ಗೊಬ್ಬ
ಕಡೆ-ಗೊಮ್ಮೆ
ಕಡೆಯ
ಕಡೆ-ಯ-ಪಕ್ಷ
ಕಡೆ-ಯಲ್ಲೂ
ಕಡೆ-ಯ-ವ-ನಾದ
ಕಡೆ-ಯ-ವ-ರನ್ನೂ
ಕಡೆ-ಯ-ವ-ರಿಂ-ದಲೂ
ಕಡೆ-ಯ-ವರು
ಕಡೆ-ಯಿಂದ
ಕಡೆ-ಯೊಳು
ಕಡೇ
ಕಡ್ಡಿ
ಕಡ್ಡಿ-ಕೊಂ-ಬೆ-ಗಳನ್ನು
ಕಡ್ಡಿ-ಗಳನ್ನು
ಕಡ್ಡಿ-ಗ-ಳಿಗೆ
ಕಡ್ಡಿ-ಯನ್ನು
ಕಡ್ಡಿ-ಯ-ನ್ನು-ಅ-ದರ
ಕಣಪ್ಪಾ
ಕಣಮ್ಮಾ
ಕಣಯ್ಯಾ
ಕಣವೇ
ಕಣಿ-ವೆ-ಗಳು
ಕಣಿ-ವೆ-ಯಲ್ಲಿ
ಕಣಿ-ವೆ-ಯೊಂ-ದರ
ಕಣೊ
ಕಣ್ಕಣ್ಣು
ಕಣ್ಣ-ಗ-ಲಿ-ಸಿ-ಕೊಂಡು
ಕಣ್ಣನ್ನೇ
ಕಣ್ಣ-ಮುಂದೆ
ಕಣ್ಣ-ರ-ಳಿ-ಸಿ-ಕೊಂಡು
ಕಣ್ಣಲ್ಲಿ
ಕಣ್ಣಾ-ಮು-ಚ್ಚಾಲೆ
ಕಣ್ಣಾರೆ
ಕಣ್ಣಿ
ಕಣ್ಣಿಂದ
ಕಣ್ಣಿಗೂ
ಕಣ್ಣಿಗೆ
ಕಣ್ಣಿ-ಗೇಕೆ
ಕಣ್ಣಿ-ಟ್ಟಿದ್ದ
ಕಣ್ಣಿ-ಟ್ಟಿ-ರ-ಬೇ-ಕಾ-ಯಿತು
ಕಣ್ಣಿಟ್ಟು
ಕಣ್ಣಿ-ದೆಯೆ
ಕಣ್ಣಿ-ನಿಂದ
ಕಣ್ಣಿಯೆ
ಕಣ್ಣಿ-ರುವ
ಕಣ್ಣಿ-ಲ್ಲದ
ಕಣ್ಣೀರ
ಕಣ್ಣೀ-ರಿಗೂ
ಕಣ್ಣೀ-ರಿಗೆ
ಕಣ್ಣೀ-ರಿ-ಳಿ-ಸುತ್ತ
ಕಣ್ಣೀರು
ಕಣ್ಣೀರೇ
ಕಣ್ಣೀರ್
ಕಣ್ಣೀ-ರ್ಗ-ರೆ-ದರು
ಕಣ್ಣೀ-ರ್ಗ-ರೆದು
ಕಣ್ಣು
ಕಣ್ಣು-ಮೂ-ಗು-ಗಳನ್ನು
ಕಣ್ಣು-ಕ-ತ್ತಲೆ
ಕಣ್ಣು-ಕೆಂ-ಪಗೆ
ಕಣ್ಣು-ಗಳ
ಕಣ್ಣು-ಗಳನ್ನು
ಕಣ್ಣು-ಗಳಲ್ಲಿ
ಕಣ್ಣು-ಗಳಿಂದ
ಕಣ್ಣು-ಗ-ಳಿಗೆ
ಕಣ್ಣು-ಗ-ಳಿವೆ
ಕಣ್ಣು-ಗಳು
ಕಣ್ಣು-ಗಳೂ
ಕಣ್ಣು-ಗ-ಳೆ-ರಡೂ
ಕಣ್ಣು-ಗಳೇ
ಕಣ್ಣು-ಜ್ಜಿ-ಕೊಂಡು
ಕಣ್ಣು-ತೆ-ರೆ-ಸಿ-ದರು
ಕಣ್ಣು-ಬಿ-ಡು-ವು-ದಕ್ಕೂ
ಕಣ್ಣು-ಮು-ಚ್ಚಿ-ಕೊಂಡು
ಕಣ್ಣು-ಮು-ಚ್ಚಿ-ಕೊಂ-ಡು-ಬಿ-ಡು-ತ್ತೇವೆ
ಕಣ್ಣು-ಮು-ಚ್ಚಿ-ಕೊ-ಳ್ಳುವ
ಕಣ್ಣುಳ್ಳ
ಕಣ್ಣೆ-ತ್ತಿಯೂ
ಕಣ್ಣೆ-ದು-ರಿಗೆ
ಕಣ್ಣೆ-ದು-ರಿಗೇ
ಕಣ್ಣೆ-ದು-ರಿ-ನಲ್ಲೇ
ಕಣ್ಣೆ-ವೆ-ಯಿಕ್ಕು
ಕಣ್ಣೊ-ರೆ-ಸುವ
ಕಣ್ಣೊ-ಳಗೆ
ಕಣ್ತೆ-ರೆದು
ಕಣ್ದೆ-ರೆದು
ಕಣ್ದೆ-ರೆ-ಸಿದ
ಕಣ್ದೆ-ರೆ-ಸು-ವಂತೆ
ಕಣ್ಮನ
ಕಣ್ಮರೆ
ಕಣ್ಮ-ರೆ-ಯಾ-ಗಿ-ದ್ದರೆ
ಕಣ್ಮ-ರೆ-ಯಾ-ಗಿ-ಬಿಟ್ಟ
ಕಣ್ಮ-ರೆ-ಯಾ-ಗು-ವುದೇ
ಕಣ್ಮ-ರೆ-ಯಾ-ದ್ದ-ರಿಂದ
ಕಣ್ಮುಂದೆ
ಕಣ್ಮುಂ-ದೆಯೇ
ಕಣ್ಮುಚ್ಚಿ
ಕಣ್ಮು-ಚ್ಚಿ-ಕೊಂಡು
ಕಣ್ಮು-ಚ್ಚಿ-ದ-ನೆಂ-ದರೆ
ಕಣ್ಮು-ಚ್ಚಿ-ದಾಗ
ಕಣ್ಮು-ಮು-ಚ್ಚಿ-ಕೊಂ-ಡಿ-ದ್ದಾನೆ
ಕಣ್ರೆಪ್ಪೆ
ಕಣ್ರೆ-ಪ್ಪೆ-ಗಲು
ಕಣ್ರೊ
ಕತೆ
ಕತೆ-ಗಳನ್ನು
ಕತೆ-ಗಳೂ
ಕತೆ-ಯನ್ನು
ಕತ್ತನ್ನ
ಕತ್ತನ್ನು
ಕತ್ತ-ರಿಸಿ
ಕತ್ತ-ರಿ-ಸಿಕೊ
ಕತ್ತ-ರಿ-ಸಿ-ಕೊಂಡು
ಕತ್ತ-ರಿ-ಸಿ-ಹಾ-ಕಿ-ಬಿಟ್ಟ
ಕತ್ತ-ರಿ-ಸಿ-ಹಾ-ಕಿ-ಬಿ-ಡು-ತ್ತಾನೆ
ಕತ್ತ-ಲಲಿ
ಕತ್ತ-ಲ-ಳಿ-ಯಲಿ
ಕತ್ತ-ಲಾಗಿ
ಕತ್ತ-ಲಾ-ಗುವ
ಕತ್ತ-ಲಿ-ನಿಂದ
ಕತ್ತಲು
ಕತ್ತಿ-ಯ-ಲ-ಗಿನ
ಕತ್ತು
ಕಥ-ನ-ಕಾ-ರರ
ಕಥ-ನ-ಕಾ-ರರು
ಕಥ-ನ-ಕಾ-ರ-ರೊ-ಬ್ಬರು
ಕಥಾ-ಕಾ-ಲ-ಕ್ಷೇಪ
ಕಥಾ-ನಾ-ಯ-ಕ-ನಾದ
ಕಥಾ-ಪ್ರ-ಸಂ-ಗ-ಗಳನ್ನು
ಕಥೆ
ಕಥೆ-ಗಳ
ಕಥೆ-ಗಳನ್ನು
ಕಥೆ-ಗಳನ್ನೆಲ್ಲ
ಕಥೆ-ಗ-ಳಿ-ದ್ದುವು
ಕಥೆ-ಗಳು
ಕಥೆ-ಗಳೂ
ಕಥೆ-ಯಂ-ತಾ-ಗಿ-ಬಿ-ಡು-ತ್ತದೆ
ಕಥೆ-ಯನ್ನು
ಕಥೆ-ಯಾ-ಗಿ-ಬಿ-ಡು-ತ್ತ-ದೆ-ಎಂದು
ಕಥೆಯೂ
ಕಥೆ-ಯೇ-ನಾ-ದರೂ
ಕದ-ಡಿ-ಬಿ-ಟ್ಟ-ನಲ್ಲ
ಕದ-ಲ-ದಿ-ರು-ವಂ-ತಹ
ಕದ-ಲಿ-ಸದೆ
ಕದ-ಲಿ-ಸಲು
ಕದ-ಲಿ-ಸುವ
ಕದ-ಲು-ವು-ದಿಲ್ಲ
ಕದ್ದು
ಕನ-ಸನ್ನು
ಕನ-ಸನ್ನೇ
ಕನ-ಸಾ-ಗಿಯೇ
ಕನ-ಸಿನ
ಕನ-ಸಿ-ನಂತೆ
ಕನ-ಸಿ-ನಲ್ಲಿ
ಕನ-ಸಿ-ನಲ್ಲೂ
ಕನಸು
ಕನ-ಸು-ಗಳನ್ನೂ
ಕನಿ-ಕ-ರ-ದಿಂದ
ಕನಿಷ್ಠ
ಕನಿ-ಷ್ಠ-ಪಕ್ಷ
ಕನ್ನಡ
ಕನ್ನ-ಡ-ದಲ್ಲಿ
ಕನ್ನ-ಡ-ದಲ್ಲೇ
ಕನ್ನ-ಡ-ರೂ-ಪದ
ಕನ್ನ-ಡಿಯ
ಕನ್ನ-ಡಿ-ಯೊ-ಳ-ಗಿನ
ಕನ್ನೈ-ಲಾಲ್
ಕಪಿ
ಕಪ್ಪು
ಕಪ್ಪೆ-ಹಾ-ವು-ಗ-ಳಿಗೂ
ಕಪ್ಪೆ-ಗ-ಳಂತೆ
ಕಪ್ಪೆ-ಯನ್ನು
ಕಬ್ಬಿ-ಣದ
ಕಬ್ಬಿ-ಣ-ದ್ದಾ-ದ-ರೇನು
ಕಬ್ಬಿ-ಣವೊ
ಕಮಂ-ಡ-ಲ-ಭಿ-ಕ್ಷಾ-ಪಾ-ತ್ರೆ-ಗಳನ್ನು
ಕಮಂ-ಡಲು
ಕಮಲ
ಕಮ-ಲದ
ಕಮ-ಲ-ದಂತೆ
ಕಮ-ಲ-ದಿಂದ
ಕರ-ಕ-ರಗಿ
ಕರ-ಗತ
ಕರ-ಗ-ದಿ-ದ್ದರೆ
ಕರ-ಗದು
ಕರ-ಗಳಿಂದ
ಕರಗಿ
ಕರ-ಗಿಸಿ
ಕರ-ಗಿ-ಸಿ-ಬಿ-ಟ್ಟಿ-ದ್ದರು
ಕರ-ಗಿ-ಹೋ-ಗಲಿ
ಕರ-ಗಿ-ಹೋಗಿ
ಕರ-ಗಿ-ಹೋ-ಗು-ತ್ತಿದ್ದ
ಕರ-ಗಿ-ಹೋ-ಗು-ತ್ತೇನೆ
ಕರ-ಗುತ್ತ
ಕರ-ತಲ
ಕರ-ತ-ಲಾ-ಮ-ಲ-ಕ-ವಾ-ಗಿ-ಸಿ-ಕೊಂಡ
ಕರಾ-ರು-ವಾ-ಕ್ಕಾಗಿ
ಕರೀ-ಕಂ-ಬ-ಳಿ-ಯನ್ನು
ಕರು-ಣಾ-ಜ-ನಕ
ಕರು-ಣಾ-ಜ-ನ-ಕ-ವಾಗಿ
ಕರು-ಣಾ-ಪೂರ್ಣ
ಕರು-ಣಾ-ಪ್ರ-ವಾಹ
ಕರು-ಣಾ-ಮಯ
ಕರು-ಣಾ-ಮ-ಯನ
ಕರು-ಣಾ-ಮ-ಯ-ನಾದ
ಕರು-ಣಾ-ಸಾ-ಗ-ರನೇ
ಕರು-ಣಿ-ಸ-ಬಲ್ಲ
ಕರು-ಣಿ-ಸಲಿ
ಕರು-ಣಿ-ಸಿ-ದ-ವಳು
ಕರು-ಣಿಸು
ಕರು-ಣಿ-ಸುವ
ಕರು-ಣಿ-ಸು-ವಂತೆ
ಕರುಣೆ
ಕರು-ಣೆಯ
ಕರು-ಣೆ-ಯಲ್ಲಿ
ಕರು-ಣೆ-ಯಿಂದ
ಕರು-ಣೆ-ಯಿಂ-ದಿ-ರು-ತ್ತಿದ್ದ
ಕರು-ಣೆಯು
ಕರು-ಳಿಗೆ
ಕರೆ
ಕರೆಗೆ
ಕರೆ-ಗೋ-ಗೊ-ಡು-ವನು
ಕರೆ-ತಂದ
ಕರೆ-ತಂ-ದರು
ಕರೆ-ತಂ-ದಾಗ
ಕರೆ-ತಂ-ದಿದ್ದ
ಕರೆ-ತಂದು
ಕರೆ-ತನ್ನಿ
ಕರೆ-ತ-ರ-ಬೇ-ಕಾ-ಯಿತು
ಕರೆ-ತ-ರ-ಲಾ-ಯಿತು
ಕರೆ-ತೆ-ರ-ಲಾ-ಯಿತು
ಕರೆದ
ಕರೆ-ದ-ರಲ್ಲ
ಕರೆ-ದರು
ಕರೆ-ದ-ರುಓ
ಕರೆ-ದರೂ
ಕರೆ-ದರೆ
ಕರೆ-ದಳು
ಕರೆ-ದಾಗ
ಕರೆ-ದಿ-ದ್ದಾರೆ
ಕರೆ-ದಿ-ದ್ದಾರೋ
ಕರೆದು
ಕರೆ-ದು-ಕೊಂಡು
ಕರೆ-ದು-ಕೊಂ-ಡು-ಹೋ-ಗ-ಬೇ-ಕೆಂದು
ಕರೆ-ದು-ಕೊ-ಳ್ಳು-ತ್ತಿ-ದ್ದರು
ಕರೆ-ದು-ಕೊ-ಳ್ಳು-ವು-ದಕ್ಕೇ
ಕರೆ-ದು-ಕೊ-ಳ್ಳು-ವು-ದುಂಟು
ಕರೆ-ದುದು
ಕರೆದೆ
ಕರೆ-ದೆ-ಯಲ್ಲ
ಕರೆ-ದೊ-ಯ್ದರು
ಕರೆ-ದೊ-ಯ್ದರೆ
ಕರೆ-ದೊ-ಯ್ದಾರು
ಕರೆ-ದೊ-ಯ್ದಿತು
ಕರೆ-ದೊಯ್ದು
ಕರೆ-ದೊ-ಯ್ಯ-ತ್ತಿ-ರುವ
ಕರೆ-ದೊ-ಯ್ಯಲಿ
ಕರೆ-ದೊ-ಯ್ಯಲು
ಕರೆ-ದೊಯ್ಯು
ಕರೆ-ದೊ-ಯ್ಯು-ತ್ತಿತ್ತು
ಕರೆ-ದೊ-ಯ್ಯು-ತ್ತಿದ್ದ
ಕರೆ-ಯನ್ನು
ಕರೆ-ಯನ್ನೂ
ಕರೆ-ಯ-ಬ-ಹುದು
ಕರೆ-ಯ-ಲಾಗಿದೆ
ಕರೆ-ಯ-ಲಾ-ರಂ-ಭಿ-ಸಿ-ದರು
ಕರೆ-ಯ-ಲಾರೆ
ಕರೆ-ಯ-ಲಿಲ್ಲ
ಕರೆ-ಯಲು
ಕರೆ-ಯ-ಲ್ಪ-ಡು-ತ್ತಿ-ದ್ದುದು
ಕರೆ-ಯಿರಿ
ಕರೆ-ಯಿ-ರಿ-ಅ-ವರು
ಕರೆ-ಯಿ-ಸಿ-ಕೊಂಡು
ಕರೆ-ಯಿ-ಸಿ-ಕೊ-ಳ್ಳದೆ
ಕರೆ-ಯಿ-ಸಿ-ಕೊ-ಳ್ಳಲು
ಕರೆಯು
ಕರೆ-ಯು-ತಿ-ಹನು
ಕರೆ-ಯುತ್ತ
ಕರೆ-ಯು-ತ್ತಾರೋ
ಕರೆ-ಯು-ತ್ತಿದ್ದ
ಕರೆ-ಯು-ತ್ತಿ-ದ್ದರೂ
ಕರೆ-ಯು-ತ್ತಿ-ದ್ದಾ-ನಲ್ಲ
ಕರೆ-ಯು-ತ್ತಿ-ದ್ದಾರೆ
ಕರೆ-ಯು-ತ್ತಿ-ದ್ದಾರೋ
ಕರೆ-ಯು-ತ್ತಿ-ದ್ದಾಳೆ
ಕರೆ-ಯು-ತ್ತೇ-ವೆಯೋ
ಕರೆ-ಯು-ವು-ದಾ-ಗಲಿ
ಕರೆ-ಯು-ವು-ದೇ-ತಕ್ಕೆ
ಕರೆಸಿ
ಕರೆ-ಸಿ-ಕೊಂ-ಡಳು
ಕರೆ-ಸಿ-ಕೊಂ-ಡಿದ್ದ
ಕರೆ-ಸಿ-ಕೊಂಡು
ಕರೆ-ಸಿ-ಕೊ-ಳ್ಳು-ತ್ತಿ-ದ್ದರು
ಕರೆ-ಸಿದ
ಕರೆ-ಸು-ತ್ತೇನೆ
ಕರ್ಣ-ಪ್ರ-ಯಾ-ಗಕ್ಕೆ
ಕರ್ಣ-ಪ್ರ-ಯಾ-ಗ-ದಲ್ಲೇ
ಕರ್ತವ್ಯ
ಕರ್ತ-ವ್ಯದ
ಕರ್ತ-ವ್ಯ-ವನ್ನು
ಕರ್ತ-ವ್ಯ-ವೆಂದು
ಕರ್ನಲ್
ಕರ್ನಾ-ಟಕ
ಕರ್ಮ
ಕರ್ಮ-ಕ-ರ್ಮ-ಫ-ಲ-ಗಳ
ಕರ್ಮ-ಇ-ವು-ಗಳನ್ನು
ಕರ್ಮಕ್ಕೆ
ಕರ್ಮ-ಕ್ಷೇ-ತ್ರ-ಕ್ಕಿ-ಳಿ-ಯು-ವಂತೆ
ಕರ್ಮ-ಕ್ಷೇ-ತ್ರಕ್ಕೆ
ಕರ್ಮ-ಗಳನ್ನು
ಕರ್ಮ-ಗಳಿಂದ
ಕರ್ಮ-ಚಿ-ತಾನ್
ಕರ್ಮ-ನದಿ
ಕರ್ಮ-ನ-ದಿ-ಯಲಿ
ಕರ್ಮ-ನ-ದಿ-ಯಲ್ಲಿ
ಕರ್ಮ-ಯೋ-ಗಿ-ಗ-ಳಾ-ದರೂ
ಕರ್ಮ-ಯೋ-ಗಿ-ಯಂತೆ
ಕರ್ಮ-ವನ್ನೂ
ಕರ್ಮ-ಶ-ಕ್ತಿಯು
ಕಲ-ಕಾಡಿ
ಕಲ-ಕಿ-ಬಿ-ಟ್ಟಿತು
ಕಲ-ಕಿ-ಹೋಗಿ
ಕಲ-ರವ
ಕಲಾ-ಕಾ-ರರ
ಕಲಾ-ತ್ಮಕ
ಕಲಾ-ವಿದ
ಕಲಾ-ವಿ-ದರು
ಕಲಿತ
ಕಲಿ-ತ-ಕೆ-ಲವು
ಕಲಿ-ತ-ದ್ದ-ಲ್ಲದೆ
ಕಲಿ-ತದ್ದು
ಕಲಿ-ತ-ದ್ದೆಲ್ಲ
ಕಲಿ-ತರು
ಕಲಿ-ತರೆ
ಕಲಿ-ತಾ-ಯಿತು
ಕಲಿ-ತಿದ್ದ
ಕಲಿ-ತಿ-ದ್ದ-ನೆಂ-ದರೆ
ಕಲಿತು
ಕಲಿ-ತು-ಕೊಂಡ
ಕಲಿ-ತು-ಕೊಂ-ಡಿದ್ದ
ಕಲಿ-ತು-ಕೊಂಡು
ಕಲಿ-ತು-ಕೊ-ಳ್ಳು-ತ್ತಿ-ದ್ದೆವು
ಕಲಿಯ
ಕಲಿ-ಯ-ತೊ-ಡ-ಗಿದ
ಕಲಿ-ಯ-ದಿ-ದ್ದರೆ
ಕಲಿ-ಯ-ಬೇ-ಕಾ-ಗು-ತ್ತದೆ
ಕಲಿ-ಯ-ಬೇಕು
ಕಲಿ-ಯ-ಬೇ-ಡವೆ
ಕಲಿ-ಯ-ಲಾ-ರಂ-ಭಿ-ಸಿದ
ಕಲಿ-ಯ-ಲಾರೆ
ಕಲಿ-ಯ-ಲಿ-ಲ್ಲ-ವಲ್ಲ
ಕಲಿ-ಯಲು
ಕಲಿ-ಯ-ಲೇ-ಬೇ-ಕಾ-ಗಿತ್ತು
ಕಲಿ-ಯುಗ
ಕಲಿ-ಯು-ಗ-ದಲ್ಲಿ
ಕಲಿ-ಯುತ್ತ
ಕಲಿ-ಯು-ತ್ತಾರೆ
ಕಲಿ-ಯುವ
ಕಲಿ-ಯು-ವಂ-ಥದು
ಕಲಿ-ಯು-ವುದನ್ನು
ಕಲಿ-ಯು-ವು-ದಷ್ಟೇ
ಕಲಿ-ಯು-ವು-ದೆಂ-ದರೆ
ಕಲಿ-ಸ-ಬೇ-ಕಾ-ದರೆ
ಕಲಿ-ಸ-ಲೇ-ಬೇಕು
ಕಲಿ-ಸಿ-ಕೊ-ಟ್ಟಿ-ದ್ದಾನೆ
ಕಲಿ-ಸಿ-ಕೊ-ಡ-ಬೇಕು
ಕಲಿ-ಸಿ-ತು-ನಾವು
ಕಲಿ-ಸಿದ
ಕಲಿ-ಸಿ-ದರೋ
ಕಲಿ-ಸಿದ್ದ
ಕಲಿ-ಸು-ತ್ತಿದೆ
ಕಲಿ-ಸು-ತ್ತಿವೆ
ಕಲಿ-ಸುವ
ಕಲು-ಕಿತು
ಕಲು-ಷಿ-ತ-ರಾ-ಗದ
ಕಲೆ
ಕಲೆತು
ಕಲೆ-ದರೆ
ಕಲೆ-ಯಲ್ಲಿ
ಕಲೆ-ಯಲ್ಲೂ
ಕಲೈ-ಡೋ-ಸ್ಕೋ-ಪಿನ
ಕಲ್ಕತ್ತ
ಕಲ್ಕ-ತ್ತಕ್ಕೆ
ಕಲ್ಕ-ತ್ತದ
ಕಲ್ಕ-ತ್ತ-ದಂ-ತಹ
ಕಲ್ಕ-ತ್ತ-ದಲ್ಲಿ
ಕಲ್ಕ-ತ್ತ-ದ-ಲ್ಲಿ-ದ್ದಾಗ
ಕಲ್ಕ-ತ್ತ-ದ-ಲ್ಲಿನ
ಕಲ್ಕ-ತ್ತ-ದಿಂದ
ಕಲ್ಕ-ತ್ತ-ದೆ-ಡೆಗೆ
ಕಲ್ಕ-ತ್ತ-ವನ್ನು
ಕಲ್ಕತ್ತಾ
ಕಲ್ಪ-ತರು
ಕಲ್ಪ-ತ-ರು-ವಿನ
ಕಲ್ಪನೆ
ಕಲ್ಪ-ನೆ-ಗಳ
ಕಲ್ಪ-ನೆ-ಗ-ಳ-ನ್ನಿ-ಟ್ಟು-ಕೊಂ-ಡಿದ್ದ
ಕಲ್ಪ-ನೆ-ಗಳನ್ನು
ಕಲ್ಪ-ನೆ-ಗಳಿಂದ
ಕಲ್ಪ-ನೆ-ಗಳು
ಕಲ್ಪ-ನೆ-ಗ-ಳೆಲ್ಲ
ಕಲ್ಪ-ನೆಯ
ಕಲ್ಪ-ನೆ-ಯನ್ನು
ಕಲ್ಪ-ನೆ-ಯಾ-ದರೂ
ಕಲ್ಪ-ನೆ-ಯಾ-ಯಿತು
ಕಲ್ಪ-ನೆಯೇ
ಕಲ್ಪ-ನೆ-ಯೊಂದು
ಕಲ್ಪಿ-ಸಿ-ಕೊಂ-ಡಾಗ
ಕಲ್ಪಿ-ಸಿ-ಕೊಂ-ಡಿ-ರು-ವು-ದನ್ನೇ
ಕಲ್ಯಾಣ
ಕಲ್ಲಾ-ಗದೆ
ಕಲ್ಲಿನ
ಕಲ್ಲು-ಕಟ್ಟಿ
ಕಲ್ಲು-ಗಳನ್ನು
ಕಲ್ಲು-ಚ-ಪ್ಪಡಿ
ಕಲ್ಲು-ತೂ-ರು-ತ್ತಿದ್ದ
ಕಲ್ಲು-ಸ-ಕ್ಕರೆ
ಕಲ್ಲು-ಹಾ-ಕು-ವಂ-ತಹ
ಕಲ್ಲೆ-ದೆ-ಯ-ವ-ನಾಗಿ
ಕಲ್ಲೆ-ದೆ-ಯ-ವ-ನಿ-ರ-ಬೇಕು
ಕಲ್ಲೋ-ಲವೂ
ಕಳಚಿ
ಕಳ-ಚಿ-ಕೊಂಡು
ಕಳ-ಚಿತು
ಕಳ-ಚಿ-ದಾಗ
ಕಳ-ಚಿ-ಬೀ-ಳು-ತ್ತಿ-ರು-ವಂತೆ
ಕಳ-ಚಿ-ಹೋ-ಗಿ-ರು-ತ್ತವೆ
ಕಳ-ಚಿ-ಹೋ-ಗು-ತ್ತಿರು
ಕಳ-ವಳ
ಕಳ-ವ-ಳ-ಗೊಂ-ಡರು
ಕಳ-ವ-ಳ-ಗೊಂಡು
ಕಳ-ವ-ಳ-ದಿಂದ
ಕಳ-ವ-ಳ-ವಾ-ಯಿತು
ಕಳ-ವ-ಳಿ-ಸು-ತ್ತಿದ್ದೆ
ಕಳಿ-ಸಲೂ
ಕಳಿಸಿ
ಕಳಿ-ಸಿ-ಕೊಟ್ಟ
ಕಳಿ-ಸಿ-ಕೊ-ಟ್ಟರು
ಕಳಿ-ಸಿ-ಕೊ-ಡು-ತ್ತಿದ್ದ
ಕಳಿ-ಸಿ-ಕೊ-ಡು-ತ್ತಿ-ದ್ದರು
ಕಳಿ-ಸಿ-ಕೊ-ಡು-ವು-ದೇನು
ಕಳಿ-ಸಿದ
ಕಳಿ-ಸಿ-ದರು
ಕಳಿ-ಸಿ-ಬಿ-ಡು-ತ್ತಿ-ದ್ದರು
ಕಳಿ-ಸು-ತ್ತಿದ್ದ
ಕಳಿ-ಸು-ತ್ತೇನೆ
ಕಳಿ-ಸುವ
ಕಳಿ-ಸು-ವು-ದಕ್ಕೂ
ಕಳಿ-ಸು-ವುದು
ಕಳೆ
ಕಳೆ-ಗಿ-ಡ-ಗಳು
ಕಳೆ-ಗಿ-ಡ-ಗ-ಳೆಲ್ಲ
ಕಳೆದ
ಕಳೆ-ದಂತೆ
ಕಳೆ-ದಂ-ತೆಲ್ಲ
ಕಳೆ-ದದ್ದೂ
ಕಳೆ-ದ-ಮೇ-ಲಷ್ಟೇ
ಕಳೆ-ದ-ಮೇಲೂ
ಕಳೆ-ದ-ಮೇಲೆ
ಕಳೆ-ದರು
ಕಳೆ-ದ-ರು-ಹಾ-ಡಿ-ದರು
ಕಳೆ-ದರೂ
ಕಳೆ-ದವು
ಕಳೆ-ದಿದ್ದ
ಕಳೆ-ದಿ-ದ್ದರು
ಕಳೆ-ದಿರ
ಕಳೆ-ದಿ-ರ-ಬ-ಹುದು
ಕಳೆ-ದಿಲ್ಲ
ಕಳೆದು
ಕಳೆ-ದು-ಕೊಂ-ಡಂತೆ
ಕಳೆ-ದು-ಕೊಂ-ಡ-ವ-ನಿ-ಗಿಂತ
ಕಳೆ-ದು-ಕೊಂ-ಡಿ-ರ-ಲಿಲ್ಲ
ಕಳೆ-ದು-ಕೊಂ-ಡಿಲ್ಲ
ಕಳೆ-ದು-ಕೊಂಡು
ಕಳೆ-ದು-ಕೊಂ-ಡು-ಬಿಟ್ಟ
ಕಳೆ-ದು-ಕೊಂ-ಡು-ಬಿ-ಟ್ಟಿ-ದ್ದಾರೆ
ಕಳೆ-ದು-ಕೊಂ-ಡು-ಬಿ-ಡು-ತ್ತಿ-ದ್ದೆನೋ
ಕಳೆ-ದು-ಕೊಂ-ಡೇ-ವೆಂಬ
ಕಳೆ-ದು-ಕೊ-ಳ್ಳ-ಲಿಲ್ಲ
ಕಳೆ-ದು-ಕೊ-ಳ್ಳಲು
ಕಳೆ-ದು-ಕೊಳ್ಳು
ಕಳೆ-ದು-ಕೊ-ಳ್ಳು-ತ್ತಿ-ರ-ಲಿಲ್ಲ
ಕಳೆ-ದುವು
ಕಳೆ-ದು-ಹೋ-ಗ-ಲಿ-ಸ-ತ್ಯ-ಸಾ-ಕ್ಷಾ-ತ್ಕಾ-ರ-ವಾ-ಗು-ವ-ವ-ರಗೆ
ಕಳೆ-ದು-ಹೋ-ಯಿ-ತ-ಲ್ಲಮ್ಮಾ
ಕಳೆಯ
ಕಳೆ-ಯ-ಬೇಕು
ಕಳೆ-ಯ-ಬೇ-ಕೆಂದು
ಕಳೆ-ಯ-ಲಿ-ದ್ದರು
ಕಳೆ-ಯಲು
ಕಳೆ-ಯ-ವ-ರನು
ಕಳೆ-ಯಿತು
ಕಳೆ-ಯಿತೋ
ಕಳೆ-ಯುತ್ತ
ಕಳೆ-ಯು-ತ್ತಿತ್ತು
ಕಳೆ-ಯು-ತ್ತಿದ್ದ
ಕಳೆ-ಯು-ತ್ತಿ-ದ್ದರು
ಕಳೆ-ಯು-ತ್ತಿ-ದ್ದುವು
ಕಳೆ-ಯು-ತ್ತಿ-ದ್ದೆವು
ಕಳೆ-ಯುವ
ಕಳ್ಳ
ಕಳ್ಳ-ತ-ನ-ದ-ರೋಡೆ
ಕವನ
ಕವ-ನಕ್ಕೆ
ಕವ-ನದ
ಕವ-ನ-ದಲ್ಲಿ
ಕವ-ನ-ವನ್ನು
ಕವ-ನ-ವೊಂ-ದನ್ನು
ಕವಿ-ಕ-ಲ್ಪನೆ
ಕವಿ-ಗಳು
ಕವಿತ್ವ
ಕವಿ-ದಂ-ತಿತ್ತು
ಕವಿ-ದು-ಕೊಂ-ಡಿ-ದ್ದು-ವೆಂದರೆ
ಕವಿಯ
ಕವಿ-ಯಲ್ಲಿ
ಕವಿ-ಯು-ತ್ತಿದ್ದ
ಕವಿ-ಯು-ವಂ-ತಾ-ದಾಗ
ಕವಿಯೇ
ಕವಿ-ಸು-ವಂ-ತಹ
ಕಷ್ಟ
ಕಷ್ಟ-ಕಾ-ರ್ಪ-ಣ್ಯ-ಗಳ
ಕಷ್ಟ-ತಾ-ಪ-ತ್ರ-ಯ-ಗಳ
ಕಷ್ಟ-ಕ-ರ-ವಾದ
ಕಷ್ಟ-ಕಾ-ರ್ಪ-ಣ್ಯ-ಗಳನ್ನು
ಕಷ್ಟ-ಕಾ-ರ್ಪ-ಣ್ಯ-ಗಳೂ
ಕಷ್ಟ-ಕೋ-ಟ-ಲೆ-ಗಳನ್ನು
ಕಷ್ಟ-ಕೋ-ಟ-ಲೆ-ಗ-ಳೆಲ್ಲ
ಕಷ್ಟಕ್ಕೆ
ಕಷ್ಟ-ಗಳ
ಕಷ್ಟ-ಗ-ಳ-ನ್ನಲ್ಲ
ಕಷ್ಟ-ಗಳನ್ನು
ಕಷ್ಟ-ಗಳನ್ನೆಲ್ಲ
ಕಷ್ಟ-ಗಳಿಂದ
ಕಷ್ಟ-ಗ-ಳಿ-ಗಿಂ-ತಲೂ
ಕಷ್ಟ-ಗ-ಳಿಗೂ
ಕಷ್ಟ-ಗ-ಳಿಗೆ
ಕಷ್ಟ-ಗಳು
ಕಷ್ಟ-ಗಳೂ
ಕಷ್ಟ-ಗ-ಳೆಲ್ಲ
ಕಷ್ಟದ
ಕಷ್ಟ-ದ-ಲ್ಲಿ-ದ್ದಾನೆ
ಕಷ್ಟ-ದ-ಲ್ಲಿ-ದ್ದಾರೆ
ಕಷ್ಟ-ದ-ಲ್ಲಿ-ರು-ವುದನ್ನು
ಕಷ್ಟ-ದಲ್ಲೇ
ಕಷ್ಟ-ದಿಂ-ದಲೋ
ಕಷ್ಟ-ಪಟ್ಟು
ಕಷ್ಟ-ಪ-ಡುತ್ತ
ಕಷ್ಟ-ಪ-ರಂ-ಪ-ರೆ-ಗಳು
ಕಷ್ಟ-ಪ-ರಂ-ಪ-ರೆ-ಯನ್ನೇ
ಕಷ್ಟ-ವನ್ನು
ಕಷ್ಟ-ವಾ-ಗಲಿ
ಕಷ್ಟ-ವಾ-ಗಿ-ರ-ಬೇಕು
ಕಷ್ಟ-ವಾ-ಗು-ತ್ತಿತ್ತು
ಕಷ್ಟ-ವಾ-ಗು-ತ್ತಿದೆ
ಕಷ್ಟ-ವಾ-ಯಿತು
ಕಷ್ಟ-ವೆ-ನಿ-ಸಿತು
ಕಷ್ಟವೇ
ಕಷ್ಟವೋ
ಕಷ್ಟ-ಸಂ-ಕ-ಟ-ಗಳನ್ನು
ಕಷ್ಟ-ಸಂ-ಕ-ಟ-ಗಳು
ಕಷ್ಟ-ಸ-ಹಿ-ಷ್ಣು-ತೆ-ಯಿ-ಲ್ಲ-ದವ
ಕಷ್ಟ-ಸಾ-ಧ್ಯ-ವಾ-ದ್ದ-ರಿಂದ
ಕಸ-ಕ-ಡ್ಡಿ-ಗಳನ್ನು
ಕಸಬು
ಕಸ-ರತ್ತು
ಕಸ-ರ-ತ್ತು-ಗಳು
ಕಸಿದು
ಕಸಿ-ದು-ಕೊ-ಳ್ಳುವ
ಕಸಿ-ವಿ-ಸಿ-ಗೊಂಡ
ಕಸಿ-ವಿ-ಸಿ-ಗೊ-ಳ್ಳ-ಲಿಲ್ಲ
ಕಸೂತಿ
ಕಹಿ-ಯನ್ನು
ಕಾಂಕು-ರ್ಗಾ-ಚಿಗೆ
ಕಾಂಕು-ರ್ಗಾ-ಚಿಯ
ಕಾಂಕು-ರ್ಗಾ-ಚಿ-ಯ-ಲ್ಲಿದ್ದ
ಕಾಂಚ-ನ-ಗಳ
ಕಾಂಚ-ನ-ದಾಸೆ
ಕಾಂಚ-ನ-ವೆಂ-ಬು-ವಾ-ಸೆ-ಗಳಿಂದ
ಕಾಂಚ-ನವೊ
ಕಾಂಚ-ನಾ-ಸ-ಕ್ತಿ-ಯನ್ನು
ಕಾಂಪೌಂ-ಡಿನ
ಕಾಂಪೌಂ-ಡಿ-ನಲ್ಲೇ
ಕಾಂಪೌ-ಡಿ-ನಲ್ಲಿ
ಕಾಕ್ರಿ
ಕಾಗ-ದ-ದಲ್ಲಿ
ಕಾಗು-ಣಿ-ತ-ದೊಂ-ದಿಗೇ
ಕಾಗೆ
ಕಾಟ
ಕಾಡಿಗೆ
ಕಾಡಿದ
ಕಾಡಿನ
ಕಾಡಿ-ನಂ-ತಿತ್ತು
ಕಾಡಿ-ನಲ್ಲಿ
ಕಾಡಿ-ನಿಂದ
ಕಾಡು
ಕಾಡು-ತ್ತಲೇ
ಕಾಡು-ತ್ತಾನೆ
ಕಾಡು-ತ್ತಿದ್ದ
ಕಾಡು-ತ್ತಿ-ದ್ದರು
ಕಾಡು-ದಾರಿ
ಕಾಡು-ಹ-ರ-ಟೆ-ಯನ್ನು
ಕಾಣ
ಕಾಣ-ತೊಡ
ಕಾಣ-ತೊ-ಡ-ಗಿತು
ಕಾಣ-ತೊ-ಡ-ಗಿತ್ತು
ಕಾಣ-ತೊ-ಡ-ಗಿ-ದರು
ಕಾಣ-ತೊ-ಡ-ಗಿ-ದುವು
ಕಾಣ-ತೊ-ಡ-ಗಿದ್ದ
ಕಾಣ-ತೊ-ಡ-ಗಿ-ದ್ದುವು
ಕಾಣದ
ಕಾಣ-ದಂತೆ
ಕಾಣ-ದಿ-ದ್ದಾಳೆ
ಕಾಣ-ದಿ-ದ್ದು-ದ-ರಿಂದ
ಕಾಣ-ದಿ-ರಲು
ಕಾಣದೆ
ಕಾಣ-ಬಂದ
ಕಾಣ-ಬ-ಯ-ಸು-ವ-ವನು
ಕಾಣ-ಬ-ಲ್ಲ-ವ-ರಾ-ಗಿ-ದ್ದರು
ಕಾಣ-ಬಲ್ಲೆ
ಕಾಣ-ಬ-ಲ್ಲೆ-ನೆಂಬ
ಕಾಣ-ಬಹು
ಕಾಣ-ಬ-ಹು-ದಾ-ಗಿತ್ತು
ಕಾಣ-ಬ-ಹು-ದಾ-ಗಿದೆ
ಕಾಣ-ಬ-ಹುದು
ಕಾಣ-ಬೇ-ಕಾ-ಗಿದೆ
ಕಾಣ-ಬೇ-ಕಾ-ದರೆ
ಕಾಣ-ಬೇಕು
ಕಾಣ-ಬೇ-ಕೆಂಬ
ಕಾಣ-ಬೇ-ಕೆಂ-ಬ-ವನು
ಕಾಣ-ಲಾ-ರಂ-ಭಿ-ಸಿತು
ಕಾಣ-ಲಾ-ರಂ-ಭಿ-ಸಿದ್ದ
ಕಾಣ-ಲಾ-ರನು
ಕಾಣಲಿ
ಕಾಣ-ಲಿ-ದ್ದೇವೆ
ಕಾಣ-ಲಿ-ರು-ವುದು
ಕಾಣ-ಲಿಲ್ಲ
ಕಾಣಲು
ಕಾಣ-ಲೇ-ಬೇಕು
ಕಾಣ-ಲೇ-ಬೇ-ಕೆಂಬ
ಕಾಣ-ಸಿ-ಗದು
ಕಾಣ-ಸಿ-ಗ-ಲಾ-ರವು
ಕಾಣ-ಸಿ-ಗುವ
ಕಾಣಿ-ಕೆ-ಗಳನ್ನು
ಕಾಣಿ-ಕೆ-ಯನ್ನು
ಕಾಣಿ-ಕೆ-ಯಾಗಿ
ಕಾಣಿ-ಸ-ಬೇಕು
ಕಾಣಿ-ಸಲು
ಕಾಣಿಸಿ
ಕಾಣಿ-ಸಿ-ಕೊಂಡ
ಕಾಣಿ-ಸಿ-ಕೊಂ-ಡರು
ಕಾಣಿ-ಸಿ-ಕೊಂಡು
ಕಾಣಿ-ಸಿ-ಕೊ-ಡ-ಬ-ಲ್ಲ-ವ-ರಾರು
ಕಾಣಿ-ಸಿ-ಕೊಳ್ಳ
ಕಾಣಿ-ಸಿ-ಕೊ-ಳ್ಳ-ದಿ-ದ್ದರೆ
ಕಾಣಿ-ಸಿ-ಕೊ-ಳ್ಳ-ದಿ-ದ್ದಾಗ
ಕಾಣಿ-ಸಿ-ಕೊ-ಳ್ಳಲಿ
ಕಾಣಿ-ಸಿ-ಕೊ-ಳ್ಳಲೇ
ಕಾಣಿ-ಸಿ-ಕೊ-ಳ್ಳು-ತ್ತಿದ್ದ
ಕಾಣಿ-ಸಿ-ಕೊ-ಳ್ಳು-ತ್ತಿ-ದ್ದರು
ಕಾಣಿ-ಸಿತು
ಕಾಣಿ-ಸು-ತ್ತದೆ
ಕಾಣಿ-ಸು-ತ್ತ-ದೆ-ಯಲ್ಲಾ
ಕಾಣಿ-ಸು-ತ್ತ-ದೆಯೆ
ಕಾಣಿ-ಸು-ತ್ತಿ-ರ-ಬೇಕು
ಕಾಣಿ-ಸು-ತ್ತಿಲ್ಲ
ಕಾಣು
ಕಾಣು-ತ-ಲ-ವರು
ಕಾಣುತ್ತ
ಕಾಣು-ತ್ತದೆ
ಕಾಣು-ತ್ತ-ದೆ-ಶ-ರೀರ
ಕಾಣು-ತ್ತಲೇ
ಕಾಣು-ತ್ತಲ್ಲ
ಕಾಣು-ತ್ತಾನೆ
ಕಾಣು-ತ್ತಾರೆ
ಕಾಣುತ್ತಿ
ಕಾಣು-ತ್ತಿತ್ತು
ಕಾಣು-ತ್ತಿದೆ
ಕಾಣು-ತ್ತಿದ್ದ
ಕಾಣು-ತ್ತಿ-ದ್ದ-ಅ-ವನ
ಕಾಣು-ತ್ತಿ-ದ್ದ-ನಾ-ದರೂ
ಕಾಣು-ತ್ತಿ-ದ್ದರು
ಕಾಣು-ತ್ತಿ-ದ್ದರೂ
ಕಾಣು-ತ್ತಿ-ದ್ದಾನೆ
ಕಾಣು-ತ್ತಿ-ದ್ದಾರೆ
ಕಾಣು-ತ್ತಿ-ದ್ದೀರಿ
ಕಾಣು-ತ್ತಿ-ದ್ದುದು
ಕಾಣು-ತ್ತಿ-ದ್ದುವು
ಕಾಣು-ತ್ತಿ-ದ್ದೇನೆ
ಕಾಣು-ತ್ತಿ-ದ್ದೇ-ನೆ-ಅ-ವನು
ಕಾಣು-ತ್ತಿ-ರ-ಲಿಲ್ಲ
ಕಾಣು-ತ್ತಿ-ರಲು
ಕಾಣು-ತ್ತಿ-ರುವ
ಕಾಣು-ತ್ತಿ-ರು-ವಂ-ತೆಯೇ
ಕಾಣು-ತ್ತಿ-ರು-ವುದೇ
ಕಾಣು-ತ್ತಿ-ಲ್ಲ-ವಲ್ಲ
ಕಾಣು-ತ್ತಿವೆ
ಕಾಣು-ತ್ತೀ-ಯಲ್ಲ
ಕಾಣು-ತ್ತೀ-ಯೇನು
ಕಾಣು-ತ್ತೇ-ನೆ-ನಾನೇ
ಕಾಣು-ತ್ತೇ-ನೆ-ಭ-ಗ-ವಂತ
ಕಾಣು-ತ್ತೇನೋ
ಕಾಣುವ
ಕಾಣು-ವಂ-ತಹ
ಕಾಣು-ವಂ-ತಿಲ್ಲ
ಕಾಣು-ವ-ನಾ-ತನು
ಕಾಣು-ವಷ್ಟೇ
ಕಾಣು-ವುದ
ಕಾಣು-ವು-ದ-ಕ್ಕಿಂತ
ಕಾಣು-ವು-ದ-ಕ್ಕಿಂ-ತಲೂ
ಕಾಣು-ವು-ದಿ-ಲ್ಲವೆ
ಕಾಣು-ವುದು
ಕಾಣು-ವುದೇ
ಕಾಣೆ-ಯಾ-ಗಿ-ಬಿ-ಟ್ಟರು
ಕಾಣೋ
ಕಾತರ
ಕಾತ-ರ-ಗೊಂ-ಡರು
ಕಾತ-ರ-ಗೊಂ-ಡಿತ್ತು
ಕಾತ-ರತೆ
ಕಾತ-ರ-ತೆಗೆ
ಕಾತ-ರ-ತೆ-ಯನ್ನು
ಕಾತ-ರ-ತೆ-ಯಿಂದ
ಕಾತ-ರ-ತೆಯು
ಕಾತ-ರದ
ಕಾತ-ರ-ದಲ್ಲಿ
ಕಾತ-ರ-ದಿಂದ
ಕಾತ-ರ-ನಾ-ಗಿ-ದ್ದರೂ
ಕಾತ-ರ-ನಾ-ಗಿ-ದ್ದು-ದನ್ನು
ಕಾತ-ರ-ನಾ-ಗಿ-ದ್ದೇನೆ
ಕಾತ-ರ-ರಾಗಿ
ಕಾತ-ರ-ರಾ-ಗಿ-ದ್ದರು
ಕಾತ-ರ-ರಾ-ಗಿ-ದ್ದಾರೆ
ಕಾತ-ರ-ರಾ-ಗಿ-ದ್ದೀರಿ
ಕಾತ-ರ-ರಾ-ಗಿ-ರ-ಬೇ-ಕಿತ್ತು
ಕಾತ-ರ-ರಾ-ಗಿ-ರು-ತ್ತಾರೆ
ಕಾತ-ರ-ರಾ-ಗಿ-ರು-ತ್ತಿ-ದ್ದರು
ಕಾತ-ರಿ-ಸಿತು
ಕಾತ-ರಿ-ಸು-ತ್ತಿ-ರುವ
ಕಾಥೇ-ವಾ-ಡದ
ಕಾದದ್ದು
ಕಾದಿ-ತ್ತು-ಅ-ವರ
ಕಾದಿ-ರಿ-ಸ-ಲಾಗಿದೆ
ಕಾದಿ-ರು-ವುದು
ಕಾದು
ಕಾದು-ಕಾದು
ಕಾದು-ಕು-ಳಿ-ತಿ-ದ್ದರು
ಕಾದೂ
ಕಾದೆ
ಕಾನ-ನದ
ಕಾನೂ-ನಿನ
ಕಾನೂನು
ಕಾಪಾಡಿ
ಕಾಪಾ-ಡಿಕೊ
ಕಾಪಾ-ಡಿ-ಕೊಂಡು
ಕಾಪಾ-ಡಿ-ಕೊಳ್ಳ
ಕಾಪಾ-ಡಿ-ಕೊ-ಳ್ಳುವ
ಕಾಪಾ-ಡಿ-ದರು
ಕಾಪಾಡು
ಕಾಪಾ-ಡು-ತ್ತಾನೆ
ಕಾಫಿ
ಕಾಫಿ-ಯನ್ನು
ಕಾಮ
ಕಾಮ-ಕಾಂ-ಚ-ನ-ಗಳ
ಕಾಮ-ಕಾಂ-ಚ-ನ-ಗಳನ್ನು
ಕಾಮ-ಮೋ-ಹ-ಗಳು
ಕಾಮ-ಕಾಂ-ಚನ
ಕಾಮ-ಕಾಂ-ಚ-ನ-ಗಳ
ಕಾಮ-ಕಾಂ-ಚ-ನ-ಗಳನ್ನು
ಕಾಮ-ಕಾಂ-ಚ-ನ-ತ್ಯಾ-ಗ-ವ-ಲ್ಲವೆ
ಕಾಮ-ಕಾಂ-ಚ-ನ-ತ್ಯಾ-ಗವೇ
ಕಾಮ-ಕಾಂ-ಚ-ನದ
ಕಾಮ-ಕಾಂ-ಚ-ನ-ದಲ್ಲಿ
ಕಾಮ-ಕಾಂ-ಚ-ನ-ದಿಂದ
ಕಾಮ-ಕಾಂ-ಚ-ನ-ವೆಂಬ
ಕಾಮ-ಕಾಂ-ಚ-ನಾ-ಸ-ಕ್ತಿ-ಯಿಂದ
ಕಾಮ-ಕಾಂ-ಚ-ನಾ-ಸ-ಕ್ತಿಯು
ಕಾಮ-ಗಂ-ಧ-ದಿಂದ
ಕಾಮ-ನೆ-ಗಳನ್ನು
ಕಾಮ-ಬು-ದ್ಧಿ-ಯಿ-ರು-ವ-ವ-ನಿಗೆ
ಕಾಮ-ಭಾ-ವನೆ
ಕಾಮ-ರ-ಪು-ಕುರ
ಕಾಮ-ರ-ಪು-ರ-ಕು-ರ-ದ-ಲ್ಲಿದ್ದು
ಕಾಮವು
ಕಾಮಾ-ರ-ಪು-ಕು-ರಕ್ಕೆ
ಕಾಮಿ-ನಿ-ಯೆಲ್ಲಿ
ಕಾಮಿ-ನೀ-ಕಾಂ-ಚ-ನ-ವನ್ನು
ಕಾಯದೆ
ಕಾಯ-ಬೇ-ಕಾ-ಗಿಲ್ಲ
ಕಾಯ-ಬೇ-ಕಾ-ಯಿತು
ಕಾಯ-ಸ್ಥರ
ಕಾಯಿಲೆ
ಕಾಯಿ-ಲೆ-ಗಳನ್ನೆಲ್ಲ
ಕಾಯಿ-ಲೆ-ಗಳು
ಕಾಯಿ-ಲೆ-ಗ-ಳು-ಇ-ವು-ಗಳಿಂದ
ಕಾಯಿ-ಲೆ-ಗ-ಳೆಲ್ಲ
ಕಾಯಿ-ಲೆಗೆ
ಕಾಯಿ-ಲೆಯ
ಕಾಯಿ-ಲೆ-ಯನ್ನು
ಕಾಯಿ-ಲೆ-ಯಲ್ಲಿ
ಕಾಯಿ-ಲೆ-ಯಾ-ಗಿ-ಬಿ-ಡು-ತ್ತಿತ್ತು
ಕಾಯಿ-ಲೆ-ಯಾ-ಗಿ-ರ-ಬೇಕು
ಕಾಯಿ-ಲೆ-ಯಿಂದ
ಕಾಯಿ-ಲೆ-ಯಿಂ-ದಾ-ಗಿದ್ದ
ಕಾಯುತ್ತ
ಕಾಯು-ತ್ತಿ-ದ್ದರು
ಕಾಯು-ತ್ತಿ-ದ್ದ-ವ-ನಂತೆ
ಕಾಯು-ತ್ತಿ-ದ್ದಾರೆ
ಕಾಯು-ತ್ತಿರು
ಕಾರಣ
ಕಾರ-ಣ-ಕ್ಕ-ಲ್ಲದೆ
ಕಾರ-ಣ-ಕ್ಕಾಗಿ
ಕಾರ-ಣ-ಕ್ಕಾ-ಗಿಯೇ
ಕಾರ-ಣಕ್ಕೂ
ಕಾರ-ಣಕ್ಕೆ
ಕಾರ-ಣ-ಗಳನ್ನು
ಕಾರ-ಣ-ಗಳಿಂದ
ಕಾರ-ಣ-ಗ-ಳಿಂ-ದಾಗಿ
ಕಾರ-ಣ-ಗಳು
ಕಾರ-ಣ-ದಿಂದ
ಕಾರ-ಣ-ದಿಂ-ದಲೂ
ಕಾರ-ಣ-ದಿಂ-ದಲೇ
ಕಾರ-ಣ-ದಿಂ-ದಾಗಿ
ಕಾರ-ಣ-ನಾಗು
ಕಾರ-ಣ-ನೆಂಬ
ಕಾರ-ಣ-ವನ್ನು
ಕಾರ-ಣ-ವಾ-ಗಿತ್ತು
ಕಾರ-ಣ-ವಾ-ಗಿದ್ದ
ಕಾರ-ಣ-ವಾ-ಗಿ-ರ-ಬೇಕು
ಕಾರ-ಣ-ವಾ-ಗು-ತ್ತದೆ
ಕಾರ-ಣ-ವಾ-ದರೆ
ಕಾರ-ಣ-ವಾ-ಯಿತು
ಕಾರ-ಣ-ವಿತ್ತು
ಕಾರ-ಣ-ವಿದೆ
ಕಾರ-ಣ-ವಿ-ರ-ಬೇ-ಕೆಂದು
ಕಾರ-ಣ-ವಿ-ಲ್ಲ-ದಿ-ರ-ಲಿಲ್ಲ
ಕಾರ-ಣ-ವಿಷ್ಟೇ
ಕಾರ-ಣವೂ
ಕಾರ-ಣ-ವೆಂದರೆ
ಕಾರ-ಣ-ವೆಂದು
ಕಾರ-ಣವೇ
ಕಾರ-ಣ-ವೇನೆಂದರೆ
ಕಾರ-ಣ-ವೇ-ನೆಂದು
ಕಾರ-ಣಾಂ-ತರ
ಕಾರ-ಣಾಂ-ತ-ರ-ದಿಂದ
ಕಾರ-ಣಿಯೋ
ಕಾರುಣ್ಯ
ಕಾರು-ಣ್ಯವೂ
ಕಾರು-ಬಾ-ರನ್ನು
ಕಾರು-ಬಾರು
ಕಾರ್ಖಾ-ನೆಯ
ಕಾರ್ತಿಕ
ಕಾರ್ನಾಕ್
ಕಾರ್ನ್
ಕಾರ್ಮೋ-ಡ-ಗಳು
ಕಾರ್ಯ
ಕಾರ್ಯ-ಕ-ರ್ತ-ರೆ-ಲ್ಲರೂ
ಕಾರ್ಯ-ಕ-ಲಾ-ಪ-ಗಳನ್ನು
ಕಾರ್ಯ-ಕ-ಲಾ-ಪ-ಗಳಲ್ಲಿ
ಕಾರ್ಯ-ಕ-ಲಾ-ಪ-ಗಳು
ಕಾರ್ಯ-ಕ್ಕಾಗಿ
ಕಾರ್ಯಕ್ಕೆ
ಕಾರ್ಯ-ಕ್ರಮ
ಕಾರ್ಯ-ಕ್ರ-ಮ-ಗಳು
ಕಾರ್ಯ-ಕ್ರ-ಮದ
ಕಾರ್ಯ-ಕ್ರ-ಮ-ದಲ್ಲಿ
ಕಾರ್ಯ-ಕ್ರ-ಮ-ವನ್ನು
ಕಾರ್ಯ-ಕ್ರ-ಮ-ವಿ-ರು-ತ್ತಿತ್ತು
ಕಾರ್ಯ-ಕ್ರ-ಮ-ವೊಂ-ದ-ರಲ್ಲಿ
ಕಾರ್ಯ-ಗ-ತ-ಗೊ-ಳಿ-ಸಲು
ಕಾರ್ಯ-ಗ-ತ-ಗೊ-ಳಿ-ಸುವ
ಕಾರ್ಯ-ಗಳನ್ನು
ಕಾರ್ಯ-ಗಳನ್ನೂ
ಕಾರ್ಯ-ಗಳಲ್ಲಿ
ಕಾರ್ಯ-ಗ-ಳೆಲ್ಲ
ಕಾರ್ಯ-ಗಳೇ
ಕಾರ್ಯತಃ
ಕಾರ್ಯದ
ಕಾರ್ಯ-ದಲ್ಲಿ
ಕಾರ್ಯ-ದಿಂದ
ಕಾರ್ಯ-ನಿ-ಮಿ-ತ್ತ-ವಾಗಿ
ಕಾರ್ಯ-ಪ್ರ-ಣಾ-ಳಿ-ಕೆ-ಯನ್ನು
ಕಾರ್ಯ-ಪ್ರ-ಣಾ-ಳಿ-ಗಳಲ್ಲಿ
ಕಾರ್ಯ-ಮಗ್ನ
ಕಾರ್ಯ-ಯೋ-ಜನೆ
ಕಾರ್ಯ-ರೂ-ಪಕ್ಕೆ
ಕಾರ್ಯ-ವನ್ನು
ಕಾರ್ಯ-ವಿದೆ
ಕಾರ್ಯ-ವಿ-ಧಾ-ನ-ಗಳನ್ನು
ಕಾರ್ಯವು
ಕಾರ್ಯವೂ
ಕಾರ್ಯವೇ
ಕಾರ್ಯ-ಶೀ-ಲತೆ
ಕಾರ್ಯ-ಸಾ-ಧ-ನೆಗೆ
ಕಾರ್ಯ-ಸಿ-ದ್ಧಿಯೂ
ಕಾರ್ಯಾ-ರಂಭ
ಕಾರ್ಯಾ-ರ್ಥ-ವಾಗಿ
ಕಾರ್ಯೋ-ದ್ದೇಶ
ಕಾರ್ಯೋ-ದ್ದೇ-ಶ-ವನ್ನು
ಕಾಲ
ಕಾಲಂ-ದು-ಗೆಯ
ಕಾಲ-ಇ-ವು-ಗಳ
ಕಾಲ-ಕಾ-ಲಕ್ಕೆ
ಕಾಲಕ್ಕೆ
ಕಾಲ-ಕ್ರ-ಮ-ದಲ್ಲಿ
ಕಾಲ-ಕ್ರ-ಮ-ದಿಂದ
ಕಾಲದ
ಕಾಲ-ದಲ್ಲಿ
ಕಾಲ-ದಿಂ-ದಲೂ
ಕಾಲ-ಧ-ರ್ಮ-ಕ್ಕ-ನು-ಗು-ಣ-ವಾಗಿ
ಕಾಲ-ವನ್ನು
ಕಾಲ-ವ-ಶ-ನಾದ
ಕಾಲ-ವಾದ
ಕಾಲ-ವಿ-ಳಂಬ
ಕಾಲವೂ
ಕಾಲ-ಸ್ಥಿ-ತಿಗೆ
ಕಾಲಾಂ-ತ-ರ-ದಲ್ಲಿ
ಕಾಲಾ-ತೀತ
ಕಾಲಾ-ನಂ-ತರ
ಕಾಲಾ-ವ-ಕಾಶ
ಕಾಲಾ-ವ-ಧಿ-ಯಲ್ಲಿ
ಕಾಲಿಂ-ದೊ-ದೆದು
ಕಾಲಿಗೆ
ಕಾಲಿ-ಗೊಂದು
ಕಾಲಿ-ಟ್ಟರೂ
ಕಾಲಿ-ಟ್ಟವು
ಕಾಲಿ-ಟ್ಟಾಗ
ಕಾಲಿ-ಡು-ತ್ತಿದ್ದ
ಕಾಲಿ-ನಡಿ
ಕಾಲು
ಕಾಲು-ಗಳನ್ನು
ಕಾಲು-ದಾರಿ-ಯಾಗಿ
ಕಾಲುವೆ
ಕಾಲು-ವೆ-ಗ-ಳಾಗ
ಕಾಲು-ವೆ-ಯನ್ನು
ಕಾಲು-ವೆ-ಯೊಂ-ದರ
ಕಾಲು-ಹಾ-ಕಲೇ
ಕಾಲು-ಹಿ-ಡಿದು
ಕಾಲೇ-ಜಿಗೆ
ಕಾಲೇ-ಜಿಗೇ
ಕಾಲೇ-ಜಿನ
ಕಾಲೇ-ಜಿ-ನಲ್ಲಿ
ಕಾಲೇಜು
ಕಾಲೊ-ತ್ತುತ್ತ
ಕಾಲ್ನ-ಡಿಗೆ
ಕಾಲ್ನ-ಡಿ-ಗೆಯ
ಕಾಲ್ನ-ಡಿ-ಗೆ-ಯಲ್ಲೇ
ಕಾಳ-ಗ-ತ್ತಲು
ಕಾಳಜಿ
ಕಾಳಿ
ಕಾಳಿ-ಕಾ-ದೇವಿ
ಕಾಳಿ-ಕಾ-ದೇ-ವಿಯ
ಕಾಳಿಗೆ
ಕಾಳಿಯ
ಕಾಳಿ-ಯನ್ನು
ಕಾಳಿಯೇ
ಕಾಳೀ
ಕಾಳೀ-ದೇ-ವ-ಸ್ಥಾ-ನಕ್ಕೆ
ಕಾಳೀ-ದೇ-ವಾ-ಯದ
ಕಾಳೀ-ದೇ-ವಾ-ಲ-ಯದ
ಕಾಳೀ-ದೇ-ವಾ-ಲ-ಯ-ದಲ್ಲಿ
ಕಾಳೀ-ದೇ-ವಿಯ
ಕಾಳೀ-ಪ-ದ-ಘೋಷ್
ಕಾಳೀ-ಪೂ-ಜೆಯ
ಕಾಳೀ-ಪ್ರ-ಸಾದ
ಕಾಳೀ-ಪ್ರ-ಸಾ-ದ-ದತ್ತ
ಕಾಳೀ-ಪ್ರ-ಸಾ-ದನ
ಕಾಳೀ-ಪ್ರ-ಸಾ-ದ-ನಲ್ಲಿ
ಕಾಳೀ-ಪ್ರ-ಸಾ-ದ-ನಿಗೆ
ಕಾಳೀ-ಪ್ರ-ಸಾ-ದನೂ
ಕಾಳೀ-ಪ್ರ-ಸಾ-ದನೇ
ಕಾಳೀ-ಪ್ರ-ಸಾ-ದ-ರನ್ನು
ಕಾಳೀ-ಮಾ-ತೆ-ತಮ್ಮ
ಕಾಳ್ಗಿ-ಚ್ಚಿ-ನಂತೆ
ಕಾವನ್ನು
ಕಾವ-ಲಾಗಿ
ಕಾವ-ಲು-ಗಾ-ರ-ನಾ-ಗಿದ್ದ
ಕಾವಿ
ಕಾವಿ-ನಲ್ಲಿ
ಕಾವಿ-ನಿಂದ
ಕಾವಿ-ಬಟ್ಟೆ
ಕಾವಿ-ಬ-ಟ್ಟೆಯ
ಕಾವಿ-ಬ-ಟ್ಟೆ-ಯನ್ನೂ
ಕಾವಿ-ವ-ಸ್ತ್ರ-ವನ್ನು
ಕಾವಿ-ಶಾ-ಟೆಯ
ಕಾವ್ಯ
ಕಾವ್ಯ-ಕ-ವ-ನ-ಗಳಲ್ಲಿ
ಕಾವ್ಯ-ಗಳಲ್ಲಿ
ಕಾವ್ಯ-ಮಯ
ಕಾವ್ಯ-ವನ್ನು
ಕಾವ್ಯವು
ಕಾವ್ಯ-ವೆಲ್ಲ
ಕಾವ್ಯಾ-ಕಾ-ಶದ
ಕಾಶಿಗೆ
ಕಾಶಿಯ
ಕಾಶಿ-ಯನ್ನು
ಕಾಶಿ-ಯಲ್ಲಿ
ಕಾಶಿ-ಯ-ಲ್ಲಿ-ದ್ದಾಗ
ಕಾಶಿ-ಯ-ಲ್ಲಿನ
ಕಾಶಿ-ಯಿಂದ
ಕಾಶೀ
ಕಾಶೀ-ನಾಥ
ಕಾಶೀ-ಪುರ
ಕಾಶೀ-ಪು-ರಕ್ಕೆ
ಕಾಶೀ-ಪು-ರದ
ಕಾಶೀ-ಪು-ರ-ದಲ್ಲಿ
ಕಾಶೀ-ಯಾತ್ರೆ
ಕಾಶೀ-ವಿ-ಶ್ವ-ನಾ-ಥನ
ಕಾಶ್ಮೀ-ರ-ದಿಂದ
ಕಾಷಾಯ
ಕಾಷಾ-ಯ-ಧಾ-ರಿ-ಯೊ-ಬ್ಬರು
ಕಾಷಾ-ಯ-ವಸ್ತ್ರ
ಕಾಷಾ-ಯ-ವ-ಸ್ತ್ರ-ಗಳನ್ನು
ಕಾಷಾ-ಯ-ವ-ಸ್ತ್ರ-ಧಾ-ರಿ-ಗ-ಳಾದ
ಕಾಷಾ-ಯ-ವ-ಸ್ತ್ರ-ವ-ನ್ನಾ-ದರೂ
ಕಾಷಾ-ಯ-ವ-ಸ್ತ್ರ-ವನ್ನು
ಕಾಷಾ-ಯ-ವ-ಸ್ತ್ರ-ವೊಂದೇ
ಕಾಸನ್ನೂ
ಕಾಸಿನ
ಕಾಸಿ-ಲ್ಲ-ದ-ವನು
ಕಿಂಕ-ರ್ತ-ವ್ಯ-ವಿ-ಮೂ-ಢ-ನಾಗಿ
ಕಿಂಕ-ರ್ತ-ವ್ಯ-ವಿ-ಮೂ-ಢ-ರಾಗಿ
ಕಿಂಚಿತ್
ಕಿಂಚಿ-ತ್ತಾ-ದರೂ
ಕಿಂಚಿತ್ತೂ
ಕಿಂತು
ಕಿಛುಡಿ
ಕಿಛು-ಡಿ-ಯನ್ನು
ಕಿಟ-ಕಿಯ
ಕಿಟ-ಕಿ-ಯಿಂದ
ಕಿಡಿ
ಕಿಡಿ-ಗಳು
ಕಿಡಿಯ
ಕಿತ್ತ
ಕಿತ್ತಡಿ
ಕಿತ್ತಾ-ಟ-ಹ-ಠ-ಮಾ-ರಿ-ತನ
ಕಿತ್ತಾ-ಡಿ-ಕೊ-ಳ್ಳು-ತ್ತಿ-ರಲಿ
ಕಿತ್ತಾಡು
ಕಿತ್ತು
ಕಿತ್ತು-ಕೊಂ-ಡೆ-ಯಲ್ಲ
ಕಿತ್ತು-ಕೊ-ಳ್ಳ-ಬ-ಹುದು
ಕಿತ್ತು-ತಂದು
ಕಿತ್ತು-ತಿ-ನ್ನುವ
ಕಿತ್ತು-ಹಾ-ಕ-ಬೇಕು
ಕಿತ್ತು-ಹೋಗಿ
ಕಿತ್ತೆ-ಸೆ-ದು-ಬಿಟ್ಟ
ಕಿಮ್ಮತ್ತು
ಕಿರ-ಣ-ಕಾಂ-ತಿ-ಯಿಂದ
ಕಿರಿ-ಕಿ-ರಿಯೂ
ಕಿರಿ-ಚುತ್ತ
ಕಿರಿ-ದಾದ
ಕಿರಿ-ದಾ-ದರೆ
ಕಿರಿದು
ಕಿರಿದೆ
ಕಿರಿ-ಮ-ಗ-ನಾಗಿ
ಕಿರಿಯ
ಕಿರಿ-ಯ-ನಾ-ದರೂ
ಕಿರಿ-ಯರೇ
ಕಿರಿ-ಯುತ್ತ
ಕಿರು-ಗಾ-ತ್ರ-ದಲ್ಲಿ
ಕಿರು-ಚಿ-ಕೊಂಡ
ಕಿಲು-ಬಿನ್ನೂ
ಕಿಲುಬೇ
ಕಿವಿ
ಕಿವಿ-ಗಳನ್ನು
ಕಿವಿ-ಗ-ಳಿಗೆ
ಕಿವಿಗೂ
ಕಿವಿಗೆ
ಕಿವಿ-ಗೊಟ್ಟು
ಕಿವಿ-ಗೊ-ಡ-ಲಾರೆ
ಕಿವಿ-ಗೊ-ಡು-ತ್ತಿಲ್ಲ
ಕಿವಿ-ಗೊ-ಡು-ವುದನ್ನು
ಕಿವಿ-ತೆ-ರೆ-ದು-ಕೊಂಡು
ಕಿವಿಯ
ಕಿವಿ-ಯನ್ನು
ಕಿವಿ-ಯನ್ನೂ
ಕಿವಿ-ಯಲ್ಲಿ
ಕಿವಿ-ಯಿಂದ
ಕಿವಿ-ಯಿಟ್ಟು
ಕಿವಿ-ಯೆಲ್ಲ
ಕೀಟಲೆ
ಕೀರಲು
ಕೀರ್ತ-ನ-ಕಾ-ರರು
ಕೀರ್ತ-ನೆ-ಯನ್ನು
ಕೀರ್ತಿ
ಕೀರ್ತಿ-ಕಾಂ-ಚ-ನ-ವೆಂಬು
ಕೀರ್ತಿ-ಗಳ
ಕೀರ್ತಿ-ಯನ್ನು
ಕೀರ್ತಿಯೂ
ಕೀಲಿಕೈ
ಕೀಳ-ರಿ-ಮೆ-ಯನ್ನು
ಕೀಳು
ಕೀಳುವ
ಕೀಶೋ-ರನ
ಕುಂಚ-ಗಳಿಂದ
ಕುಂಠಿ-ತ-ಗೊ-ಳಿಸು
ಕುಂಡ-ಲಿನಿ
ಕುಂಡ-ಲಿ-ನೀ-ಶ-ಕ್ತಿ-ಯನ್ನು
ಕುಂದಿತು
ಕುಂದಿ-ಹೋ-ಗು-ತ್ತಿತ್ತು
ಕುಂಭಾ-ಭಿ-ಷೇಕ
ಕುಕ್ಕಿದ
ಕುಕ್ಕು-ತ್ತಿತ್ತು
ಕುಗ್ಗ-ಲಿಲ್ಲ
ಕುಗ್ಗಿ-ಹೋ-ಗು-ತ್ತಿತ್ತು
ಕುಛ್
ಕುಟೀ-ಚ-ಕ-ನಾಗಿ
ಕುಟೀ-ಚ-ಕ-ನಾ-ಗು-ತ್ತಾನೆ
ಕುಟೀ-ರ-ಗಳಲ್ಲಿ
ಕುಟೀ-ರ-ಗಳು
ಕುಟೀ-ರದ
ಕುಟೀ-ರ-ವನ್ನು
ಕುಟೀ-ರ-ವೊಂ-ದನ್ನು
ಕುಟುಂಬ
ಕುಟುಂ-ಬ-ಗಳಿಂದ
ಕುಟುಂ-ಬದ
ಕುಟುಂ-ಬ-ದ-ಲ್ಲಾ-ದ್ದ-ರಿಂದ
ಕುಟುಂ-ಬ-ದ-ವ-ರನ್ನು
ಕುಟುಂ-ಬ-ದ-ವರೆ-ಲ್ಲರೂ
ಕುಟುಂ-ಬ-ವ-ರ್ಗ-ದ-ವರ
ಕುಟುಂ-ಬ-ವಾ-ದು-ದ-ರಿಂದ
ಕುಟು-ಕಿ-ದ-ವ-ರಂತೆ
ಕುಟುಕು
ಕುಟ್ಟಿ
ಕುಡಿಕೆ
ಕುಡಿತ
ಕುಡಿ-ತ-ಕ್ಕಿ-ಳಿ-ದಿದ್ದೂ
ಕುಡಿ-ತಕ್ಕೂ
ಕುಡಿದ
ಕುಡಿ-ದಂತೆ
ಕುಡಿ-ದರೂ
ಕುಡಿ-ದರೆ
ಕುಡಿ-ದ-ರೇನು
ಕುಡಿದು
ಕುಡಿ-ದು-ಬಿಟ್ಟ
ಕುಡಿ-ಯಲು
ಕುಡಿಯು
ಕುಡಿ-ಯು-ತ್ತಾನೆ
ಕುಡಿ-ಯು-ತ್ತಿ-ರಲಿ
ಕುಡಿ-ಯು-ತ್ತಿ-ರು-ವಾಗ
ಕುಡಿ-ಯು-ವ-ವನು
ಕುಡಿ-ಯು-ವಷ್ಟೇ
ಕುಡಿ-ಸ-ಲಾ-ಯಿತು
ಕುಣಿಕೆ
ಕುಣಿ-ದರು
ಕುಣಿದು
ಕುಣಿ-ಯಿರೈ
ಕುತೂ-ಹಲ
ಕುತೂ-ಹ-ಲ-ಕರ
ಕುತೂ-ಹ-ಲ-ಕಾರಿ
ಕುತೂ-ಹ-ಲಕ್ಕೆ
ಕುತೂ-ಹ-ಲದ
ಕುತೂ-ಹ-ಲ-ದಲ್ಲಿ
ಕುತೂ-ಹ-ಲ-ದಿಂದ
ಕುತೂ-ಹ-ಲ-ವನ್ನು
ಕುತ್ತಿ-ಗೆಗೆ
ಕುದಿ-ಯಿತು
ಕುದುರೆ
ಕುದು-ರೆ-ಗಳನ್ನು
ಕುದು-ರೆಯ
ಕುದು-ರೆ-ಯನ್ನು
ಕುದು-ರೆ-ಯೊಂ-ದನ್ನು
ಕುದು-ರೆ-ಸ-ವಾ-ರಿ-ಯಲ್ಲೂ
ಕುಪ್ಪ-ಳಿಸು
ಕುಪ್ಪ-ಳಿ-ಸು-ವಂತೆ
ಕುಪ್ಪ-ಳಿ-ಸು-ವುದು
ಕುಮಾರ್
ಕುಮ್ಮ-ಕ್ಕು-ಪ್ರೋ-ತ್ಸಾಹ
ಕುರಿ-ತಂತೆ
ಕುರಿತಾ
ಕುರಿ-ತಾಗಿ
ಕುರಿ-ತಾ-ಗಿಯೇ
ಕುರಿ-ತಾ-ಗಿಯೋ
ಕುರಿ-ತಾದ
ಕುರಿತು
ಕುರಿತೇ
ಕುರುಡ
ಕುರು-ಡ-ನಿಗೆ
ಕುರು-ಡರ
ಕುರು-ಡ-ರೇ-ನನು
ಕುರುಡು
ಕುರು-ಡು-ಶ್ರದ್ಧೆ
ಕುರು-ಹಾಗಿ
ಕುರು-ಹು-ಗಳನ್ನು
ಕುರು-ಹು-ಗ-ಳೆಂದು
ಕುಲ-ಕು-ಕಿ-ದರೂ
ಕುಲಕ್ಕೂ
ಕುಲದ
ಕುಲ-ದೇ-ವ-ತೆಯ
ಕುಲ-ಪು-ರೋ-ಹಿ-ತರು
ಕುಲ-ಮದ
ಕುಲ-ವಂತ
ಕುಲಾ-ಭಿ-ಮಾನವೇ
ಕುಲುಕಿ
ಕುಳಿತ
ಕುಳಿ-ತ-ನೆಂ-ದರೆ
ಕುಳಿ-ತರು
ಕುಳಿ-ತರೆ
ಕುಳಿ-ತ-ಲ್ಲಿಂದ
ಕುಳಿ-ತ-ವ-ನಲ್ಲ
ಕುಳಿ-ತ-ವ-ರತ್ತ
ಕುಳಿ-ತ-ವ-ರಿ-ಗೆಲ್ಲ
ಕುಳಿ-ತ-ವರು
ಕುಳಿ-ತಾಗ
ಕುಳಿತಿ
ಕುಳಿ-ತಿದ್ದ
ಕುಳಿ-ತಿ-ದ್ದರು
ಕುಳಿ-ತಿ-ದ್ದರೆ
ಕುಳಿ-ತಿ-ದ್ದ-ವ-ರತ್ತ
ಕುಳಿ-ತಿ-ದ್ದ-ವರೆಲ್ಲ
ಕುಳಿ-ತಿ-ದ್ದಾಗ
ಕುಳಿ-ತಿ-ದ್ದಾನೆ
ಕುಳಿ-ತಿ-ದ್ದಾರೆ
ಕುಳಿ-ತಿ-ದ್ದಾಳೆ
ಕುಳಿ-ತಿ-ದ್ದು-ಬಿಟ್ಟ
ಕುಳಿ-ತಿದ್ದೆ
ಕುಳಿ-ತಿ-ರ-ಬೇಕು
ಕುಳಿ-ತಿ-ರ-ಲಾ-ಗದೆ
ಕುಳಿ-ತಿ-ರಲು
ಕುಳಿ-ತಿ-ರಲೂ
ಕುಳಿ-ತಿ-ರು-ತ್ತಿದ್ದ
ಕುಳಿ-ತಿ-ರು-ತ್ತಿ-ದ್ದರು
ಕುಳಿ-ತಿ-ರುವ
ಕುಳಿ-ತಿ-ರು-ವಂತೆ
ಕುಳಿ-ತಿ-ರು-ವುದನ್ನು
ಕುಳಿ-ತಿ-ರು-ವುದು
ಕುಳಿತು
ಕುಳಿ-ತು-ಕೊಂಡ
ಕುಳಿ-ತು-ಕೊಂ-ಡರೆ
ಕುಳಿ-ತು-ಕೊಂ-ಡ-ವನೇ
ಕುಳಿ-ತು-ಕೊಂ-ಡಿ-ದ್ದರು
ಕುಳಿ-ತು-ಕೊಂ-ಡಿ-ದ್ದಾನೆ
ಕುಳಿ-ತು-ಕೊಂ-ಡಿ-ರು-ವ-ವನು
ಕುಳಿ-ತು-ಕೊಂಡು
ಕುಳಿ-ತು-ಕೊಳ್ಳ
ಕುಳಿ-ತು-ಕೊ-ಳ್ಳ-ಬೇಕೆ
ಕುಳಿ-ತು-ಕೊ-ಳ್ಳಲು
ಕುಳಿ-ತು-ಕೊ-ಳ್ಳ-ವುದು
ಕುಳಿ-ತು-ಕೊ-ಳ್ಳು-ತ್ತಾನೆ
ಕುಳಿ-ತು-ಕೊ-ಳ್ಳು-ತ್ತಿದ್ದ
ಕುಳಿ-ತು-ಕೊ-ಳ್ಳುವ
ಕುಳಿ-ತು-ಕೊ-ಳ್ಳು-ವಂ-ತಿ-ರ-ಲಿಲ್ಲ
ಕುಳಿ-ತು-ಕೊ-ಳ್ಳು-ವಂ-ತಿಲ್ಲ
ಕುಳಿ-ತು-ಕೊ-ಳ್ಳು-ವ-ವ-ರಲ್ಲ
ಕುಳಿ-ತು-ಕೊ-ಳ್ಳು-ವಾಗ
ಕುಳಿ-ತು-ಕೊ-ಳ್ಳು-ವು-ದೇಕೆ
ಕುಳಿ-ತು-ಕೊ-ಳ್ಳೋಣ
ಕುಳಿ-ತು-ಬಿಟ್ಟ
ಕುಳಿ-ತು-ಬಿ-ಟ್ಟದ್ದ
ಕುಳಿ-ತು-ಬಿ-ಟ್ಟದ್ದು
ಕುಳಿ-ತು-ಬಿ-ಟ್ಟರು
ಕುಳಿ-ತು-ಬಿ-ಟ್ಟರೆ
ಕುಳಿ-ತು-ಬಿ-ಟ್ಟಿ-ದ್ದಾರೆ
ಕುಳಿ-ತು-ಬಿಟ್ಟೆ
ಕುಳಿ-ತು-ಬಿ-ಡ-ಬೇ-ಕೆಂಬ
ಕುಳಿ-ತು-ಬಿ-ಡು-ತ್ತಿದ್ದ
ಕುಳಿ-ತು-ಬಿ-ಡು-ತ್ತಿ-ದ್ದರು
ಕುಳಿ-ತು-ಬಿ-ಡು-ತ್ತಿ-ದ್ದು-ದನ್ನು
ಕುಳಿ-ತು-ಬಿ-ಡು-ತ್ತಿದ್ದೆ
ಕುಳಿ-ತು-ಬಿ-ಡು-ತ್ತೀರಿ
ಕುಳಿತೆ
ಕುಳಿ-ತೆ-ಯಲ್ಲ
ಕುಳಿತೇ
ಕುಳಿ-ತೊ-ಡ-ನೆಯೇ
ಕುಳ್ಳಿ-ರಿ-ಸ-ಲಾ-ಯಿತು
ಕುಳ್ಳಿ-ರಿಸಿ
ಕುಳ್ಳಿ-ರಿ-ಸಿ-ಕೊಂ-ಡರು
ಕುಳ್ಳಿ-ರಿ-ಸಿ-ಕೊಂಡು
ಕುಳ್ಳಿ-ರಿ-ಸಿದ
ಕುವೆಂಪು
ಕುಶ-ಲಾ-ನು-ಶಿಷ್ಟ
ಕುಶ-ಲಿ-ಗ-ಳಾದ
ಕುಶ-ಲಿ-ಯಾ-ಗಿ-ರ-ಬೇಕು
ಕುಶ-ಲೋsಸ್ಯ
ಕುಸಂ-ಸ್ಕಾ-ರ-ಗಳು
ಕುಸಿದು
ಕುಸಿ-ದು-ಬಿ-ದ್ದರೆ
ಕುಸಿ-ಯುವ
ಕುಸ್ತಿ
ಕುಸ್ತಿ-ಯಾ-ಡು-ತ್ತಿ-ದ್ದರೆ
ಕುಹ-ಕ-ದಿಂದ
ಕೂಗದೇ
ಕೂಗಾ-ಡು-ತ್ತಿ-ರ-ಲಿಲ್ಲ
ಕೂಗಿ
ಕೂಗಿ-ಕೊಂಡ
ಕೂಗಿ-ಕೊಂ-ಡ-ದ್ದನ್ನು
ಕೂಗಿ-ಕೊಂ-ಡರು
ಕೂಗಿ-ಕೊ-ಳ್ಳ-ಲಿ-ಲ್ಲವೆ
ಕೂಗಿ-ಕೊ-ಳ್ಳುವ
ಕೂಗಿದ
ಕೂಗಿ-ದರು
ಕೂಗಿ-ದರೂ
ಕೂಗಿ-ದಾಗ
ಕೂಗಿದ್ದು
ಕೂಗುತ್ತ
ಕೂಗು-ತ್ತಿ-ದ್ದಾನೆ
ಕೂಗು-ವು-ದ-ರೊ-ಳ-ಗಾಗಿ
ಕೂಡ
ಕೂಡದು
ಕೂಡಲು
ಕೂಡಲೆ
ಕೂಡಲೇ
ಕೂಡಿ
ಕೂಡಿ-ಕೊಂ-ಡರು
ಕೂಡಿಟ್ಟ
ಕೂಡಿ-ಟ್ಟಿದ್ದ
ಕೂಡಿಟ್ಟು
ಕೂಡಿ-ಡು-ವು-ದ-ರಲ್ಲಿ
ಕೂಡಿದ
ಕೂಡಿ-ದ-ವನು
ಕೂಡಿದೆ
ಕೂಡಿ-ದ್ದರೆ
ಕೂಡಿ-ದ್ದಾ-ಗಿ-ರು-ತ್ತಿತ್ತು
ಕೂಡಿದ್ದು
ಕೂಡಿ-ಬ-ರ-ದೆ-ಹೋ-ದರೂ
ಕೂಡಿ-ಬ-ರು-ವ-ವ-ರೆಗೆ
ಕೂಡಿ-ರು-ವಂ-ಥ-ವ-ನೊ-ಬ್ಬನು
ಕೂಡಿ-ಸಲು
ಕೂಡಿಸಿ
ಕೂಡಿ-ಸಿ-ಕೊಂಡು
ಕೂಡಿ-ಸಿ-ಬಿ-ಟ್ಟರೆ
ಕೂಡಿ-ಹಾ-ಕಿ-ದಳು
ಕೂತುಂಡೇ
ಕೂತುಂ-ಬಗೆ
ಕೂದಲು
ಕೂದ-ಲೆಳೆ
ಕೂಪ-ಮಂ-ಡೂ-ಕ-ದಂತೆ
ಕೂರಿ-ಸಿ-ಕೊಂಡು
ಕೂಲ-ಕ-ರ-ವಾ-ಗಿ-ರ-ಲಿಲ್ಲ
ಕೂಲಿ-ಗಳ
ಕೂಲಿ-ಯೊಂ-ದಿಗೆ
ಕೃತಕ
ಕೃತ-ಘ್ನ-ರಾಗಿ
ಕೃತ-ಜ್ಞತೆ
ಕೃತ-ಜ್ಞ-ತೆಯ
ಕೃತ-ಜ್ಞ-ತೆ-ಯ-ನ್ನ-ರ್ಪಿಸಿ
ಕೃತ-ಜ್ಞ-ತೆ-ಯಿಂದ
ಕೃತ-ಜ್ಞ-ನಾ-ಗಿ-ದ್ದೇನೆ
ಕೃತ-ಜ್ಞ-ರಾ-ಗಿ-ದ್ದರೂ
ಕೃತಾ-ರ್ಥ-ರಾ-ಗು-ತ್ತಾರೆ
ಕೃತಾ-ರ್ಥ-ವಾ-ಯಿತೋ
ಕೃತಿ-ಗಳ
ಕೃತಿ-ಗಳನ್ನು
ಕೃತಿ-ಗಳಲ್ಲಿ
ಕೃತಿಗೂ
ಕೃತಿ-ಯಲ್ಲೂ
ಕೃತಿ-ಯೊಂ-ದರ
ಕೃತ್ಯ
ಕೃಪಾ-ಕ-ಟಾಕ್ಷ
ಕೃಪಾ-ಕಾಂ-ಕ್ಷಿ-ಗ-ಳಾದ
ಕೃಪಾ-ನಂ-ದರು
ಕೃಪಾ-ನಂ-ದ-ರು-ಇ-ವ-ರು-ಗ-ಳೊ-ಡನೆ
ಕೃಪಾ-ನಂ-ದ-ರು-ಇಷ್ಟು
ಕೃಪಾ-ನಂ-ದರೂ
ಕೃಪಾ-ಪ್ರ-ವಾಹ
ಕೃಪಾ-ಶೀ-ರ್ವಾದ
ಕೃಪಾ-ಶೀ-ರ್ವಾ-ದ-ವನ್ನು
ಕೃಪೆ
ಕೃಪೆಗೆ
ಕೃಪೆ-ಗೊ-ಳ-ಗಾ-ಗ-ಬೇ-ಕಾ-ದರೆ
ಕೃಪೆ-ಮಾಡಿ
ಕೃಪೆ-ಯನ್ನು
ಕೃಪೆ-ಯಲ್ಲಿ
ಕೃಪೆ-ಯಾ-ಗು-ತ್ತದೆ
ಕೃಪೆ-ಯಾ-ಗು-ವ-ವ-ರೆಗೆ
ಕೃಪೆ-ಯಿಂದ
ಕೃಪೆ-ಯಿಂ-ದಲೇ
ಕೃಷಿ-ಕನ
ಕೃಷ್ಣ-ಜ-ನ್ಮಾ-ಷ್ಟ-ಮಿಗೆ
ಕೃಷ್ಣ-ನಾಗಿ
ಕೃಷ್ಣ-ನಾ-ಗಿ-ದ್ದನೋ
ಕೃಷ್ಣನೂ
ಕೃಷ್ಣರ
ಕೃಷ್ಣ-ರನ್ನು
ಕೃಷ್ಣ-ರಿ-ಗಾಗಿ
ಕೃಷ್ಣ-ರಿಗೆ
ಕೃಷ್ಣರು
ಕೃಷ್ಣರೂ
ಕೃಷ್ಣ-ರೆಂದೂ
ಕೃಷ್ಣರೇ
ಕೆಂಗಣ್ಣು
ಕೆಂಪು
ಕೆಚ್ಚೆ-ದೆಯ
ಕೆಟ್ಟ
ಕೆಟ್ಟಂ-ತಾ-ಗಿ-ಬಿ-ಡು-ತ್ತಿತ್ತು
ಕೆಡಕು
ಕೆಡಿ-ಸದೆ
ಕೆಡಿ-ಸ-ಲಿ-ಲ್ಲವೆ
ಕೆಡು-ಕನ್ನು
ಕೆಡು-ತ್ತಲೇ
ಕೆಡು-ವು-ದಿಲ್ಲ
ಕೆಣ-ಕಿ-ದರು
ಕೆತ್ತಿದ
ಕೆದಕಿ
ಕೆರಳಿ
ಕೆರ-ಳಿತು
ಕೆರ-ಳಿದ
ಕೆರ-ಳಿ-ಸಿತ್ತು
ಕೆರ-ಳು-ತ್ತದೆ
ಕೆರ-ಳು-ತ್ತಿತ್ತೋ
ಕೆರ-ವಾನ್
ಕೆರೆ-ಕುಂ-ಟೆ-ಗ-ಳಲ್ಲೇ
ಕೆರೆ-ಕುಂ-ಟೆ-ಗ-ಳಾ-ದರೆ
ಕೆರೆ-ಕುಂ-ಟೆ-ಗಳಿಂದ
ಕೆರೆ-ಕುಂ-ಟೆ-ಗ-ಳಿದ್ದ
ಕೆರೆಯ
ಕೆರೆ-ಯಲ್ಲಿ
ಕೆಲ
ಕೆಲ-ಕಾಲ
ಕೆಲ-ಕಾ-ಲದ
ಕೆಲ-ಕಾ-ಲ-ದಲ್ಲೇ
ಕೆಲ-ತಿಂ-ಗಳ
ಕೆಲ-ತಿಂ-ಗ-ಳ-ನಿಂದ
ಕೆಲ-ತಿಂ-ಗ-ಳಿ-ನಿಂದ
ಕೆಲ-ದಿನ
ಕೆಲ-ದಿ-ನ-ಗಳ
ಕೆಲ-ದಿ-ನ-ಗ-ಳಲ್ಲೇ
ಕೆಲ-ದಿ-ನ-ಗ-ಳ-ವ-ರೆಗೆ
ಕೆಲ-ದಿ-ನ-ಗ-ಳಾ-ಗಿ-ರ-ಬ-ಹುದು
ಕೆಲ-ದಿ-ನ-ಗ-ಳಿದ್ದು
ಕೆಲ-ದಿ-ನ-ಗಳು
ಕೆಲ-ಮ-ಟ್ಟಿ-ಗಾ-ದರೂ
ಕೆಲ-ಮೊಮ್ಮೆ
ಕೆಲರು
ಕೆಲ-ವಂ-ಶ-ವ-ನ್ನಾ-ದರೂ
ಕೆಲ-ವನ್ನು
ಕೆಲ-ವರ
ಕೆಲ-ವ-ರಂತೂ
ಕೆಲ-ವ-ರನ್ನು
ಕೆಲ-ವ-ರಾ-ದರೂ
ಕೆಲ-ವ-ರಿಗೆ
ಕೆಲ-ವರು
ಕೆಲ-ವ-ರ್ಷ-ಗಳ
ಕೆಲ-ವ-ರ್ಷ-ಗಳೇ
ಕೆಲವು
ಕೆಲವೇ
ಕೆಲ-ವೊಂದು
ಕೆಲ-ವೊಮ್ಮೆ
ಕೆಲಸ
ಕೆಲ-ಸಆ
ಕೆಲ-ಸ-ಕಾರ್ಯ
ಕೆಲ-ಸ-ಕಾ-ರ್ಯ-ಗಳನ್ನು
ಕೆಲ-ಸ-ಕಾ-ರ್ಯ-ಗಳನ್ನೆಲ್ಲ
ಕೆಲ-ಸ-ಕಾ-ರ್ಯ-ಗಳಲ್ಲಿ
ಕೆಲ-ಸ-ಕ್ಕಾಗಿ
ಕೆಲ-ಸಕ್ಕೆ
ಕೆಲ-ಸ-ಗಳನ್ನು
ಕೆಲ-ಸ-ಗಳು
ಕೆಲ-ಸದ
ಕೆಲ-ಸ-ಮಾ-ಡಿತು
ಕೆಲ-ಸ-ವಂತೂ
ಕೆಲ-ಸ-ವನ್ನು
ಕೆಲ-ಸ-ವನ್ನೂ
ಕೆಲ-ಸ-ವಾ-ಗ-ಬೇ-ಕಾ-ದ್ದಿದೆ
ಕೆಲ-ಸ-ವಾ-ಗಿತ್ತು
ಕೆಲ-ಸ-ವಾ-ಯಿತು
ಕೆಲ-ಸ-ವಾ-ಯಿತೆ
ಕೆಲ-ಸ-ವಿದೆ
ಕೆಲ-ಸವೂ
ಕೆಲ-ಸ-ವೆಂದರೆ
ಕೆಲ-ಸ-ವೇನೂ
ಕೆಲ-ಸ-ವೊಂ-ದನ್ನು
ಕೆಲ-ಹೊ-ತ್ತಿನ
ಕೆಲ-ಹೊತ್ತು
ಕೆಳ-ಕ್ಕೆ-ಸೆ-ದು-ಬಿಟ್ಟ
ಕೆಳ-ಗಡೆ
ಕೆಳ-ಗ-ಡೆ-ಯಿಂದ
ಕೆಳ-ಗ-ಳೆ-ಯಲು
ಕೆಳ-ಗಾಗಿ
ಕೆಳ-ಗಿನ
ಕೆಳ-ಗಿ-ನ-ವ-ರೆಗೂ
ಕೆಳ-ಗಿ-ರುವ
ಕೆಳ-ಗಿ-ಳಿದು
ಕೆಳಗೆ
ಕೆಳಗೋ
ಕೆಳ-ಭಾ-ಗ-ವನ್ನು
ಕೆಳ-ಮ-ಟ್ಟ-ಕ್ಕಿ-ಳಿ-ದಾನು
ಕೆಳ-ಸ್ತ-ರಕ್ಕೆ
ಕೇಂದ್ರ-ವಾಗಿ
ಕೇಂದ್ರೀ-ಕೃ-ತ-ವಾ-ಗಿವೆ
ಕೇಡು
ಕೇದಾ-ರ-ಗಳ
ಕೇರಿ-ಗ-ಳಿಗೆ
ಕೇಲಿ-ದ-ವರೆಲ್ಲ
ಕೇಳ-ದಿ-ದ್ದರೆ
ಕೇಳ-ದಿ-ರಲು
ಕೇಳದೆ
ಕೇಳ-ಬ-ಹುದು
ಕೇಳ-ಬ-ಹು-ದು-ನೀವು
ಕೇಳ-ಬ-ಹುದೇ
ಕೇಳ-ಬೇಕು
ಕೇಳಲಿ
ಕೇಳ-ಲಿ-ಕ್ಕಿದೆ
ಕೇಳ-ಲಿ-ದ್ದೇನೆ
ಕೇಳಲು
ಕೇಳಲೂ
ಕೇಳಲೇ
ಕೇಳಿ
ಕೇಳಿ-ಕೇಳಿ
ಕೇಳಿಕೊ
ಕೇಳಿ-ಕೊಂಡ
ಕೇಳಿ-ಕೊಂ-ಡರು
ಕೇಳಿ-ಕೊಂ-ಡರೂ
ಕೇಳಿ-ಕೊಂ-ಡರೆ
ಕೇಳಿ-ಕೊಂ-ಡಳು
ಕೇಳಿ-ಕೊಂ-ಡಾಗ
ಕೇಳಿ-ಕೊಂ-ಡಿರಾ
ಕೇಳಿ-ಕೊಂಡು
ಕೇಳಿ-ಕೊಂ-ಡೆಯಾ
ಕೇಳಿ-ಕೊ-ಳ್ಳ-ದಿ-ದ್ದರೆ
ಕೇಳಿ-ಕೊ-ಳ್ಳ-ಬಾ-ರದು
ಕೇಳಿ-ಕೊ-ಳ್ಳ-ಬೇಕು
ಕೇಳಿ-ಕೊ-ಳ್ಳ-ಲಾರೆ
ಕೇಳಿ-ಕೊ-ಳ್ಳಲಿ
ಕೇಳಿ-ಕೊ-ಳ್ಳಲೇ
ಕೇಳಿ-ಕೊ-ಳ್ಳ-ಲೇ-ಬೇಕು
ಕೇಳಿ-ಕೊ-ಳ್ಳು-ತ್ತ-ಲಿದ್ದ
ಕೇಳಿ-ಕೊ-ಳ್ಳು-ತ್ತಿ-ದ್ದರು
ಕೇಳಿ-ಕೊ-ಳ್ಳು-ತ್ತಿ-ರು-ವುದು
ಕೇಳಿ-ಕೊ-ಳ್ಳೋ-ಣ-ವೆಂ-ದಿ-ದ್ದೇನೆ
ಕೇಳಿದ
ಕೇಳಿ-ದ-ರ-ವರು
ಕೇಳಿ-ದರು
ಕೇಳಿ-ದರೂ
ಕೇಳಿ-ದರೆ
ಕೇಳಿ-ದಳು
ಕೇಳಿ-ದ-ವ-ರಿಗೆ
ಕೇಳಿ-ದ-ವ-ರಿ-ಗೆಲ್ಲ
ಕೇಳಿ-ದ-ವರೆಲ್ಲ
ಕೇಳಿ-ದ-ವರೇ
ಕೇಳಿ-ದ-ಷ್ಟನ್ನು
ಕೇಳಿ-ದಷ್ಟೂ
ಕೇಳಿ-ದಾಗ
ಕೇಳಿದೆ
ಕೇಳಿ-ದೊ-ಡನೆ
ಕೇಳಿ-ದೊ-ಡ-ನೆಯೇ
ಕೇಳಿದ್ದ
ಕೇಳಿ-ದ್ದರೆ
ಕೇಳಿದ್ದು
ಕೇಳಿದ್ದೆ
ಕೇಳಿ-ದ್ದೇನೆ
ಕೇಳಿ-ನೋ-ಡ-ಬಾ-ರದು
ಕೇಳಿ-ಬಂತು
ಕೇಳಿ-ಬಂ-ತು-ಶ್ರೀ-ರಾ-ಮ-ಕೃ-ಷ್ಣರ
ಕೇಳಿ-ಬಿ-ಟ್ಟಳು
ಕೇಳಿ-ಬಿ-ಟ್ಟಳೋ
ಕೇಳಿಯೇ
ಕೇಳಿ-ಯೇ-ಬಿಟ್ಟ
ಕೇಳಿ-ಸಿ-ಕೊಂಡ
ಕೇಳಿ-ಸಿ-ಕೊಂ-ಡ-ದ್ದೇನೋ
ಕೇಳಿ-ಸಿ-ಕೊಂಡು
ಕೇಳಿ-ಸಿ-ಕೊಂ-ಡೆಯಾ
ಕೇಳಿ-ಸಿ-ಕೊ-ಳ್ಳ-ಬಲ್ಲ
ಕೇಳಿ-ಸಿ-ಕೊ-ಳ್ಳು-ತ್ತಾನೆ
ಕೇಳಿ-ಸಿತು
ಕೇಳಿ-ಸಿಯೇ
ಕೇಳಿ-ಸು-ತ್ತಿದೆ
ಕೇಳು
ಕೇಳು-ಗ-ರ-ನ್ನದು
ಕೇಳು-ತ-ಕ್ತಾ-ರೆ-ಹಾ-ಜರಾ
ಕೇಳುತ್ತ
ಕೇಳು-ತ್ತ-ಕೇ-ಳುತ್ತ
ಕೇಳು-ತ್ತಲೇ
ಕೇಳು-ತ್ತಾ-ನಲ್ಲ
ಕೇಳು-ತ್ತಾನೆ
ಕೇಳು-ತ್ತಾರೆ
ಕೇಳು-ತ್ತಾ-ರೆ-ಅ-ವರು
ಕೇಳು-ತ್ತಿದ್ದ
ಕೇಳು-ತ್ತಿ-ದ್ದಂತೆ
ಕೇಳು-ತ್ತಿ-ದ್ದಂ-ತೆಯೇ
ಕೇಳು-ತ್ತಿ-ದ್ದರು
ಕೇಳು-ತ್ತಿ-ದ್ದ-ವರೆಲ್ಲ
ಕೇಳು-ತ್ತಿ-ದ್ದಾನೆ
ಕೇಳು-ತ್ತಿ-ದ್ದಾರೆ
ಕೇಳು-ತ್ತೀಯೋ
ಕೇಳು-ನಿ-ರ್ಭೀ-ತಿ-ಯಿಂದ
ಕೇಳುವ
ಕೇಳು-ವಂ-ತಿಲ್ಲ
ಕೇಳು-ವಂತೆ
ಕೇಳು-ವಂ-ತೆಯೂ
ಕೇಳು-ವ-ವ-ನಲ್ಲ
ಕೇಳು-ವ-ವ-ರಲ್ಲ
ಕೇಳು-ವಾಗ
ಕೇಳು-ವು-ದ-ಕ್ಕಾ-ಗ-ಲಿ-ಲ್ಲವೆ
ಕೇಳು-ವು-ದಕ್ಕೆ
ಕೇಳು-ವು-ದ-ಕ್ಕೆಂದೇ
ಕೇಳು-ವುದನ್ನು
ಕೇಳು-ವು-ದಿತ್ತು
ಕೇಳು-ವು-ದಿ-ತ್ತು-ಭ-ಗ-ವಂ-ತನ
ಕೇಳು-ವು-ದಿಲ್ಲ
ಕೇಳೋಣ
ಕೇಳ್ತೀಯೋ
ಕೇವಲ
ಕೇಶಕ್ಕೆ
ಕೇಶ-ವ-ವಿ-ಜ-ಯ-ಕೃ-ಷ್ಣ-ರಲ್ಲಿ
ಕೇಶ-ವ-ಚಂದ್ರ
ಕೇಶ-ವ-ಚಂ-ದ್ರ-ಸೇ-ನರ
ಕೇಶ-ವ-ನಲ್ಲಿ
ಕೇಶ-ವ-ಸೇ-ನ-ವಿ-ಜ-ಯ-ಕೃ-ಷ್ಣ-ರಲ್ಲಿ
ಕೇಶ-ವ-ಸೇ-ನ-ನಿಂದ
ಕೇಶಾ-ಕೇ-ಶಿ-ಗ-ಳಿಗೆ
ಕೇಸರಿ
ಕೇಸ-ರಿ-ಯಂತೆ
ಕೇಸ-ರಿ-ಯೊಲು
ಕೈ
ಕೈಂಕ-ರ್ಯವೇ
ಕೈಕಾ-ಲು-ಗಳನ್ನು
ಕೈಕಾ-ಲು-ಗ-ಳೆಲ್ಲ
ಕೈಕಾಲೇ
ಕೈಕೊ-ಟ್ಟು-ಕೊಂಡು
ಕೈಗ-ಡಿ-ಯಾ-ರ-ವನ್ನು
ಕೈಗಲ್ಲ
ಕೈಗ-ಲ್ಲವೆ
ಕೈಗಳನ್ನು
ಕೈಗಳಿಂದ
ಕೈಗ-ಳಿಂ-ದಲೇ
ಕೈಗೆ
ಕೈಗೆ-ಟು-ಕ-ದಿ-ದ್ದರೆ
ಕೈಗೆ-ತ್ತಿ-ಕೊಂಡ
ಕೈಗೆ-ತ್ತಿ-ಕೊಂ-ಡರು
ಕೈಗೆ-ತ್ತಿ-ಕೊಂಡು
ಕೈಗೆ-ತ್ತಿ-ಕೊ-ಳ್ಳ-ದಿ-ದ್ದರೆ
ಕೈಗೊಂಡ
ಕೈಗೊಂ-ಡರು
ಕೈಗೊಂ-ಡಿದ್ದ
ಕೈಗೊಂ-ಡಿ-ದ್ದರು
ಕೈಗೊಂ-ಡಿ-ದ್ದರೂ
ಕೈಗೊಂ-ಡಿ-ರಲಿ
ಕೈಗೊಂಡು
ಕೈಗೊಂ-ಡೇನೋ
ಕೈಗೊ-ಪ್ಪಿ-ಸಲು
ಕೈಗೊ-ಳ್ಳ-ಬೇ-ಕಾ-ದ-ದ್ದಿದೆ
ಕೈಗೊ-ಳ್ಳ-ಬೇಕು
ಕೈಗೊ-ಳ್ಳ-ಬೇ-ಕೆಂಬ
ಕೈಗೊ-ಳ್ಳು-ತ್ತಾರೆ
ಕೈಗೊ-ಳ್ಳು-ವಂತೆ
ಕೈಗೊ-ಳ್ಳು-ವ-ವರು
ಕೈಗೊ-ಳ್ಳು-ವ-ವ-ರೆಗೂ
ಕೈಚ-ಳಕ
ಕೈಚಾ-ಚಿ-ದರು
ಕೈಜೋ-ಡಿಸಿ
ಕೈತಪ್ಪಿ
ಕೈತುಂಬ
ಕೈನೋಡಿ
ಕೈಬಿಚ್ಚಿ
ಕೈಬಿ-ಟ್ಟಿದ್ದ
ಕೈಬಿ-ಟ್ಟು-ಬಿ-ಡು-ತ್ತೀರಾ
ಕೈಬಿ-ಡ-ಬೇ-ಕಾ-ಯಿತು
ಕೈಬಿಡು
ಕೈಬೀ-ಸಿ-ಕೊಂಡು
ಕೈಮೀ-ರುವ
ಕೈಮು-ಗಿ-ದರು
ಕೈಮು-ಗಿದು
ಕೈಮು-ರಿ-ದು-ಕೊಂ-ಡಿ-ದ್ದ-ವನು
ಕೈಮೂಳೆ
ಕೈಯ
ಕೈಯನ್ನು
ಕೈಯನ್ನೇ
ಕೈಯಲ್ಲಿ
ಕೈಯ-ಲ್ಲಿಟ್ಟ
ಕೈಯ-ಲ್ಲಿ-ಟ್ಟಿ-ದ್ದೇನೆ
ಕೈಯ-ಲ್ಲಿದೆ
ಕೈಯ-ಲ್ಲಿದ್ದ
ಕೈಯ-ಲ್ಲಿ-ರುವ
ಕೈಯಲ್ಲೂ
ಕೈಯಲ್ಲೇ
ಕೈಯಾರೆ
ಕೈಯಿಂದ
ಕೈಯಿಂ-ದ-ಲಾ-ದರೂ
ಕೈಯಿಂ-ದಲೇ
ಕೈಯೂ
ಕೈಯೆತ್ತಿ
ಕೈಲಾ-ಗ-ದ-ವನು
ಕೈಲಾ-ಸಕ್ಕೆ
ಕೈಲಾ-ಸ-ಬಾಬು
ಕೈಲಿದೆ
ಕೈವಾಡ
ಕೈಸು-ಟ್ಟು-ಕೊಂ-ಡಿ-ರ-ಬ-ಹುದು
ಕೈಸೇ-ರಿ-ದರೆ
ಕೈಸೇ-ರುವ
ಕೈಸೋತು
ಕೈಹ-ಚ್ಚಿ-ದರು
ಕೈಹ-ತ್ತ-ಲಿಲ್ಲ
ಕೈಹಾ-ಕಿದ
ಕೈಹಾ-ಕಿ-ದ-ವರು
ಕೈಹಾ-ಕು-ವುದನ್ನು
ಕೈಹಿ-ಡಿದು
ಕೈಹಿ-ಡಿ-ದು-ಕೊಂಡು
ಕೊಂಡ
ಕೊಂಡಂ-ತಹ
ಕೊಂಡ-ಕೊಂಡ
ಕೊಂಡದ್ದು
ಕೊಂಡನೋ
ಕೊಂಡರು
ಕೊಂಡಾಗ
ಕೊಂಡಾ-ಡಿ-ದರು
ಕೊಂಡಾ-ಡುತ್ತ
ಕೊಂಡಾ-ಡು-ತ್ತಾರೆ
ಕೊಂಡಾ-ಡು-ತ್ತಿ-ದ್ದರು
ಕೊಂಡಾ-ಡುವ
ಕೊಂಡಾ-ಡು-ವು-ದೇನು
ಕೊಂಡಾರು
ಕೊಂಡಾ-ರೆಯೆ
ಕೊಂಡಿತು
ಕೊಂಡಿತ್ತು
ಕೊಂಡಿದ್ದ
ಕೊಂಡಿ-ದ್ದನ್ನು
ಕೊಂಡಿ-ದ್ದರೆ
ಕೊಂಡಿ-ದ್ದ-ರೆಂ-ಬು-ದರ
ಕೊಂಡಿ-ದ್ದಾಗ
ಕೊಂಡಿದ್ದು
ಕೊಂಡಿದ್ದೆ
ಕೊಂಡಿ-ರ-ಬ-ಹುದು
ಕೊಂಡಿವೆ
ಕೊಂಡು
ಕೊಂಡು-ಕೊಂಡು
ಕೊಂಡು-ಕೊ-ಳ್ಳಲು
ಕೊಂಡು-ಕೊ-ಳ್ಳ-ಲೇ-ಬೇಕು
ಕೊಂಡು-ತಂದ
ಕೊಂಡು-ಬಿ-ಟ್ಟಿದ್ದ
ಕೊಂಡು-ಬಿ-ಟ್ಟಿ-ರು-ವುದನ್ನು
ಕೊಂಡು-ಬಿ-ಡು-ತ್ತಿತ್ತು
ಕೊಂಡುವು
ಕೊಂಡೆ
ಕೊಂಡೇ
ಕೊಂಡೇ-ಬಿಟ್ಟ
ಕೊಂಡೊ-ಯ್ಯಲು
ಕೊಂಡೊ-ಯ್ಯು-ತ್ತ-ದೆಯೋ
ಕೊಂಡೊ-ಯ್ಯು-ತ್ತವೆ
ಕೊಂಡೊ-ಯ್ಯು-ತ್ತಿ-ದ್ದರು
ಕೊಂಬೆಗೆ
ಕೊಂಬೆ-ಯಲ್ಲಿ
ಕೊಂಬೆ-ಯಿಂದ
ಕೊಕ್ಕನ್ನು
ಕೊಗ-ಡಲು
ಕೊಚ್ಚಿ-ಕೊಂಡು
ಕೊಚ್ಚು-ತ್ತಿದ್ದ
ಕೊಟ್ಚಿದ್ದೇ
ಕೊಟ್ಟ
ಕೊಟ್ಟ-ದ್ದ-ಲ್ಲದೆ
ಕೊಟ್ಟದ್ದು
ಕೊಟ್ಟದ್ದೂ
ಕೊಟ್ಟನೆ
ಕೊಟ್ಟ-ಬಿ-ಡು-ತ್ತಿ-ದ್ದರು
ಕೊಟ್ಟರು
ಕೊಟ್ಟರೆ
ಕೊಟ್ಟಳು
ಕೊಟ್ಟ-ವರ
ಕೊಟ್ಟಾರು
ಕೊಟ್ಟಾ-ಹಾ-ರ-ವ-ನ್ನವು
ಕೊಟ್ಟಿತ್ತು
ಕೊಟ್ಟಿದ್ದ
ಕೊಟ್ಟಿ-ದ್ದರು
ಕೊಟ್ಟಿ-ದ್ದಾನೆ
ಕೊಟ್ಟಿ-ದ್ದಾರೆ
ಕೊಟ್ಟಿ-ರ-ಲಿಲ್ಲ
ಕೊಟ್ಟು
ಕೊಟ್ಟು-ಬಿಟ್ಟ
ಕೊಟ್ಟು-ಬಿ-ಟ್ಟಾ-ಗಿದೆ
ಕೊಟ್ಟು-ಬಿ-ಟ್ಟಿ-ದ್ದಾರೆ
ಕೊಟ್ಟು-ಬಿಟ್ಟೆ
ಕೊಟ್ಟು-ಬಿ-ಡ-ದಂ-ತಾ-ಗಲಿ
ಕೊಟ್ಟು-ಬಿ-ಡಲು
ಕೊಟ್ಟು-ಬಿ-ಡು-ತ್ತಿದ್ದ
ಕೊಟ್ಟೇ
ಕೊಟ್ಟೇ-ಕೊ-ಡು-ತ್ತಾನೆ
ಕೊಠ-ಡಿ-ಯಲ್ಲಿ
ಕೊಡ
ಕೊಡ-ಗ-ಟ್ಟಲೆ
ಕೊಡದಿ
ಕೊಡ-ದಿ-ದ್ದರೆ
ಕೊಡ-ದಿ-ರಲಿ
ಕೊಡ-ದಿ-ರುವ
ಕೊಡದೆ
ಕೊಡದೇ
ಕೊಡ-ಬ-ಲ್ಲವು
ಕೊಡ-ಬಾ-ರದು
ಕೊಡ-ಬೇ-ಕಲ್ಲ
ಕೊಡ-ಬೇ-ಕಾ-ಗಿಲ್ಲ
ಕೊಡ-ಬೇ-ಕಾದ
ಕೊಡ-ಬೇ-ಕಾ-ದರೆ
ಕೊಡ-ಬೇಕು
ಕೊಡ-ಬೇ-ಕೆಂಬ
ಕೊಡ-ಬೇಕೇ
ಕೊಡ-ಬೇ-ಡವೆ
ಕೊಡ-ಲಾ-ಗ-ದಂ-ತಾ-ದಾಗ
ಕೊಡ-ಲಾಗಿದೆ
ಕೊಡ-ಲಾ-ರಂ-ಭಿ-ಸಿದ
ಕೊಡ-ಲಾ-ರದೆ
ಕೊಡ-ಲಾ-ರವು
ಕೊಡ-ಲಿ-ಲ್ಲ-ವೆಂದು
ಕೊಡಲು
ಕೊಡಲೂ
ಕೊಡ-ಲೇ-ಬೇಕು
ಕೊಡ-ವ-ವರು
ಕೊಡವಿ
ಕೊಡಿ
ಕೊಡಿ-ಸ-ಬ-ಲ್ಲರು
ಕೊಡಿ-ಸ-ಬ-ಹು-ದಾ-ಗಿತ್ತು
ಕೊಡಿ-ಸ-ಬೇಕು
ಕೊಡಿ-ಸ-ಲೇ-ಬೇಕು
ಕೊಡಿಸಿ
ಕೊಡಿ-ಸಿ-ದರು
ಕೊಡಿ-ಸು-ತ್ತೇನೆ
ಕೊಡಿ-ಸು-ವಂತೆ
ಕೊಡಿ-ಸೋಣ
ಕೊಡು
ಕೊಡು-ಗೆಯೂ
ಕೊಡು-ಗೈಯ
ಕೊಡುತ್ತ
ಕೊಡು-ತ್ತದೋ
ಕೊಡು-ತ್ತಲೇ
ಕೊಡು-ತ್ತಾನೆ
ಕೊಡು-ತ್ತಾರೆ
ಕೊಡು-ತ್ತಾ-ರೆ-ಯಾರ
ಕೊಡು-ತ್ತಿತ್ತು
ಕೊಡು-ತ್ತಿದ್ದ
ಕೊಡು-ತ್ತಿ-ದ್ದರು
ಕೊಡು-ತ್ತಿ-ದ್ದಳು
ಕೊಡು-ತ್ತಿ-ದ್ದ-ವರು
ಕೊಡು-ತ್ತಿಯೋ
ಕೊಡು-ತ್ತಿ-ರ-ಲಿಲ್ಲ
ಕೊಡು-ತ್ತಿ-ರು-ವುದು
ಕೊಡು-ತ್ತಿ-ಲ್ಲ-ವಲ್ಲ
ಕೊಡು-ತ್ತಿ-ಲ್ಲ-ವೆಂದು
ಕೊಡುತ್ತೀ
ಕೊಡು-ತ್ತೀ-ಯಲ್ಲ
ಕೊಡು-ತ್ತೀಯಾ
ಕೊಡು-ತ್ತೇನೆ
ಕೊಡು-ತ್ತೇವೆ
ಕೊಡುವ
ಕೊಡು-ವಂತೆ
ಕೊಡು-ವ-ವನೇ
ಕೊಡು-ವ-ವ-ರಲ್ಲ
ಕೊಡು-ವ-ವ-ರಾರು
ಕೊಡು-ವಷ್ಟು
ಕೊಡು-ವಾಗ
ಕೊಡುವು
ಕೊಡು-ವು-ದ-ಕ್ಕಾಗಿ
ಕೊಡು-ವು-ದಕ್ಕೆ
ಕೊಡು-ವು-ದ-ಕ್ಕೋ-ಸ್ಕರ
ಕೊಡು-ವು-ದಾ-ಗಲಿ
ಕೊಡು-ವು-ದಾ-ದರೆ
ಕೊಡು-ವುದು
ಕೊಡು-ವುದೂ
ಕೊಡೋಣ
ಕೊಡ್ತಾರೆ
ಕೊನೆ-ಗಾ-ದರೂ
ಕೊನೆ-ಗಾಲ
ಕೊನೆಗೂ
ಕೊನೆಗೆ
ಕೊನೆ-ಗೊಂ-ಡಂ-ತಾ-ಯಿತು
ಕೊನೆ-ಗೊಂ-ಡಿದೆ
ಕೊನೆ-ಗೊಂದು
ಕೊನೆ-ಗೊ-ಳಿ-ಸಿ-ಕೊಂ-ಡು-ಬಿ-ಡುವ
ಕೊನೆ-ಗೊ-ಳ್ಳು-ತ್ತ-ದ್ದಂತೆ
ಕೊನೆಯ
ಕೊನೆ-ಯಲ್ಲಿ
ಕೊನೆ-ಯ-ವ-ರೆಗೂ
ಕೊನೆಯೇ
ಕೊಯ್ದು
ಕೊಯ್ವನು
ಕೊರ-ಗ-ಬೇಡ
ಕೊರ-ಗಲೂ
ಕೊರ-ಗು-ತ್ತಿ-ರ-ಬೇಕು
ಕೊರ-ಡಿ-ನಂ-ತಾ-ಗಿ-ಬಿ-ಟ್ಟಿದೆ
ಕೊರತೆ
ಕೊರ-ತೆ-ಯಾ-ಗ-ಲೇ-ಬಾ-ರದು
ಕೊರ-ತೆ-ಯಿ-ರ-ಲಿಲ್ಲ
ಕೊರ-ತೆ-ಯಿ-ಲ್ಲದ
ಕೊರ-ತೆ-ಯಿ-ಲ್ಲ-ದಂತೆ
ಕೊರ-ತೆಯೂ
ಕೊರ-ತೆಯೆ
ಕೊರ-ಳನ್ನು
ಕೊರಾನು
ಕೊರೆ-ತ-ದಲ್ಲಿ
ಕೊರೆ-ಯು-ತ್ತಿ-ತ್ತು-ಒಂ-ದಾ-ದರೂ
ಕೊರೆ-ಯು-ತ್ತಿದೆ
ಕೊರೆ-ಯು-ತ್ತಿ-ದ್ದುವು
ಕೊರೆ-ಯುವ
ಕೊಲೆ-ಗ-ಡು-ಕ-ನನ್ನು
ಕೊಳ-ಅದು
ಕೊಳ-ಕಾದ
ಕೊಳದ
ಕೊಳ-ದಲ್ಲಿ
ಕೊಳವೆ
ಕೊಳ-ವೆ-ಗಳಿಂದ
ಕೊಳ-ವೆ-ಯನ್ನು
ಕೊಳೆ
ಕೊಳೆ-ಕೊ-ರ-ತೆಯೆ
ಕೊಳೆ-ಗೇ-ರಿ-ಗ-ಳಿಗೆ
ಕೊಳೆತ
ಕೊಳೆತು
ಕೊಳ್ಳ-ದಿ-ದ್ದರೆ
ಕೊಳ್ಳ-ದಿ-ರು-ವುದೇ
ಕೊಳ್ಳದೆ
ಕೊಳ್ಳ-ಬೇ-ಕಾ-ಗು-ತ್ತದೆ
ಕೊಳ್ಳ-ಬೇ-ಕಾದ
ಕೊಳ್ಳ-ಬೇ-ಕಾ-ದರೆ
ಕೊಳ್ಳ-ಬೇಕು
ಕೊಳ್ಳಲಿ
ಕೊಳ್ಳಲು
ಕೊಳ್ಳಲೇ
ಕೊಳ್ಳ-ಲೇ-ಬೇಕು
ಕೊಳ್ಳಿ
ಕೊಳ್ಳು-ತ್ತಾರೆ
ಕೊಳ್ಳುತ್ತಿ
ಕೊಳ್ಳು-ತ್ತಿದ್ದ
ಕೊಳ್ಳು-ತ್ತಿ-ದ್ದರು
ಕೊಳ್ಳು-ತ್ತಿ-ದ್ದಾ-ರೆಯೆ
ಕೊಳ್ಳುವ
ಕೊಳ್ಳು-ವಂ-ತಿದ್ದ
ಕೊಳ್ಳು-ವಂತೆ
ಕೊಳ್ಳು-ವ-ವ-ನಲ್ಲ
ಕೊಳ್ಳು-ವು-ದಕ್ಕೂ
ಕೊಳ್ಳು-ವುದು
ಕೊಳ್ಳೋಣ
ಕೋಂಡವು
ಕೋಟಿ
ಕೋಟಿ-ಕೋಟಿ
ಕೋಟಿ-ಗ-ಟ್ಟಲೆ
ಕೋಟಿ-ಭಾ-ನು-ಕ-ರ-ದೀಪ್ತ
ಕೋಟಿಯ
ಕೋಟಿ-ಸೂ-ರ್ಯ-ಪ್ರ-ಕಾ-ಶ-ದಿಂದ
ಕೋಟೆ-ಯನ್ನು
ಕೋಟ್ಯ-ಧೀ-ಶ-ರನ್ನೂ
ಕೋಟ್ಯ-ಧೀ-ಶ್ವರ
ಕೋಡಿ
ಕೋಣೆಗೆ
ಕೋಣೆಯ
ಕೋಣೆ-ಯನ್ನು
ಕೋಣೆ-ಯಲ್ಲಿ
ಕೋಣೆ-ಯ-ಲ್ಲಿನ
ಕೋಣೆ-ಯ-ಲ್ಲೆಲ್ಲ
ಕೋಣೆ-ಯಿಂದ
ಕೋಣೆ-ಯಿಂ-ದಾ-ಚೆಗೆ
ಕೋಣೆಯೂ
ಕೋಣೆ-ಯೊಂ-ದ-ರಲ್ಲಿ
ಕೋಣೆ-ಯೊ-ಳಕ್ಕೆ
ಕೋಣೆ-ಯೊ-ಳ-ಗಿನ
ಕೋಣೆ-ಯೊ-ಳಗೆ
ಕೋಣೆ-ಯೊ-ಳ-ಲೆವ
ಕೋತಿ
ಕೋತಿ-ಗಳ
ಕೋತಿ-ಗ-ಳಂ-ತೆಯೇ
ಕೋತಿ-ಗ-ಳಿಗೆ
ಕೋತಿ-ಗಳು
ಕೋತಿ-ಗ-ಳು-ಗ-ಟ್ಟಿ-ಯಾಗಿ
ಕೋತಿ-ಗ-ಳು-ಬ-ನಾ-ರಿ-ಸಿನ
ಕೋತಿಯ
ಕೋನ-ಗಳಿಂದ
ಕೋನ-ದಿಂದ
ಕೋಪ
ಕೋಪ-ಗೊಂ-ಡರು
ಕೋಪ-ಗೊಂಡು
ಕೋಪ-ತಾ-ಪ-ಗಳನ್ನು
ಕೋಪ-ತಾ-ಪ-ಗಳಲ್ಲಿ
ಕೋಪ-ದಿಂದ
ಕೋಪ-ವನ್ನು
ಕೋಪಿ-ಷ್ಟರು
ಕೋಪಿ-ಸಿ-ಕೊಂಡು
ಕೋಪಿ-ಸಿ-ಕೊಂ-ಡು-ಬಿ-ಟ್ಟೆಯಾ
ಕೋಪಿ-ಸಿ-ಕೊ-ಳ್ಳು-ತ್ತಾ-ರಲ್ಲ
ಕೋಮ-ಲ-ಸ್ಪರ್ಶ
ಕೋರಲಿ
ಕೋರಿ
ಕೋರಿ-ಕೆ-ಯಂತೆ
ಕೋರಿ-ಕೆ-ಯನ್ನಾ
ಕೋರಿ-ಕೆ-ಯನ್ನು
ಕೋರಿದ
ಕೋರಿ-ದಳು
ಕೋರು-ತ್ತೀಯಾ
ಕೋರ್ಟ
ಕೋರ್ಟಿಗೆ
ಕೋರ್ಟಿನ
ಕೋರ್ಟು
ಕೋರ್ಟು-ಕ-ಛೇ-ರಿ-ಗ-ಳಿಗೂ
ಕೋಲಾ-ಹ-ಲ-ವುಂ-ಟಾ-ಯಿತು
ಕೋಲಾ-ಹ-ಲವೇ
ಕೋಶ-ದೊ-ಳಗೆ
ಕೋಶಿ
ಕೌಟುಂ-ಬಿಕ
ಕೌತು-ಕ-ಮಯ
ಕೌಪೀನ
ಕೌಪೀ-ನಕ್ಕೆ
ಕೌಪೀ-ನ-ಧಾರೀ
ಕೌಪೀ-ನ-ವನ್ನು
ಕೌಮಾರ್ಯ
ಕೌಮಾ-ರ್ಯಕ್ಕೆ
ಕೌಶ-ಲದ
ಕೌಶ-ಲ-ದ-ಲ್ಲಿಲ್ಲ
ಕೌಶ-ಲ-ದಿಂ-ದಲೇ
ಕೌಶ-ಲ-ವನ್ನು
ಕ್ಕಾಗ-ಲಿಲ್ಲ
ಕ್ಕಾಗಿ
ಕ್ಕಾಗಿಯೇ
ಕ್ಕಾಗಿಯೋ
ಕ್ಕಾದರೂ
ಕ್ಕಿದ್ದಂತೆ
ಕ್ತಿತು-ಹಾಕಿ
ಕ್ಯಾಂಟ್
ಕ್ಯಾಂಪ್
ಕ್ಯಾನ್ಸರ್
ಕ್ಯಾನ್ಸರ್ಗೆ
ಕ್ರಮ
ಕ್ರಮ-ನಿ-ಯ-ಮ-ಗಳನ್ನು
ಕ್ರಮ-ಸಾ-ಧ-ನಾ-ಕ್ರ-ಮ-ಗಳನ್ನು
ಕ್ರಮ-ಕ್ರ-ಮ-ವಾಗಿ
ಕ್ರಮ-ದಲ್ಲಿ
ಕ್ರಮ-ದಿಂದ
ಕ್ರಮ-ಪ್ರ-ಕಾರ
ಕ್ರಮ-ಪ್ರ-ಕಾ-ರ-ವಾ-ಗಿಯೇ
ಕ್ರಮ-ವ-ನ್ನಿ-ಟ್ಟು-ಕೊಂ-ಡಿದ್ದ
ಕ್ರಮ-ವನ್ನು
ಕ್ರಮ-ವಾಗಿ
ಕ್ರಮ-ವೆಲ್ಲ
ಕ್ರಮವೇ
ಕ್ರಮಾ-ಗ-ತ-ವಾಗಿ
ಕ್ರಮಿ-ಸಿ-ದರು
ಕ್ರಮೇಣ
ಕ್ರಾಂತಿ
ಕ್ರಾಂತಿ-ಕಾರಿ
ಕ್ರಾಂತಿ-ಕಾರೀ
ಕ್ರಾಂತಿ-ಯನ್ನೇ
ಕ್ರಾಂತಿ-ಯಾ-ಗಿ-ಬಿ-ಟ್ಟಿತು
ಕ್ರಾಂತಿ-ಯುಂ-ಟಾ-ಗು-ತ್ತದೆ
ಕ್ರಿಕೆಟ್
ಕ್ರಿಕೆ-ಟ್ನಲ್ಲಿ
ಕ್ರಿಯಾ-ತ್ಮಕ
ಕ್ರಿಯಾ-ಶೀ-ಲ-ವಾ-ಗ-ಲೇ-ಬೇಕು
ಕ್ರಿಯಾ-ಶೀ-ಲ-ವಾಗಿ
ಕ್ರಿಯೆಯೂ
ಕ್ರಿಸ್ತನ
ಕ್ರಿಸ್ತ-ನಂ-ತೆಯೇ
ಕ್ರಿಸ್ತ-ನಿಗೆ
ಕ್ರಿಸ್ತ-ಭಾ-ವ-ದಿಂದ
ಕ್ರಿಸ್ಮಸ್
ಕ್ರೀಡೆ-ಗಳಲ್ಲಿ
ಕ್ರೂರ
ಕ್ರೂರ-ತನ
ಕ್ರೈಸ್ತ
ಕ್ರೈಸ್ತ-ಮು-ಸ-ಲ್ಮಾ-ನರ
ಕ್ರೈಸ್ತ-ಧ-ರ್ಮದ
ಕ್ರೈಸ್ತ-ಧ-ರ್ಮ-ವನ್ನು
ಕ್ರೈಸ್ತ-ಧ-ರ್ಮೀ-ಯರ
ಕ್ರೈಸ್ತ-ಮ-ತೀ-ಯ-ನಾದ
ಕ್ರೈಸ್ತರು
ಕ್ರೋಡೀ-ಕ-ರಿಸಿ
ಕ್ಲಿಷ್ಟ
ಕ್ಲಿಷ್ಟ-ವಾದ
ಕ್ಷಣ
ಕ್ಷಣ-ಕಾಲ
ಕ್ಷಣ-ಕಾ-ಲ-ದ-ವ-ರೆಗೆ
ಕ್ಷಣ-ಕಾ-ಲ-ವಾ-ದರೂ
ಕ್ಷಣ-ಕಾ-ಲವೂ
ಕ್ಷಣ-ಗಳಲ್ಲಿ
ಕ್ಷಣ-ದ-ಲ್ಲಾ-ದರೂ
ಕ್ಷಣ-ದಲ್ಲಿ
ಕ್ಷಣ-ದಲ್ಲೂ
ಕ್ಷಣ-ದಲ್ಲೇ
ಕ್ಷಣ-ದ-ವ-ರೆಗೂ
ಕ್ಷಣ-ದಿಂ-ದಲೇ
ಕ್ಷಣ-ಭಂ-ಗು-ರ-ತೆ-ಯನ್ನು
ಕ್ಷಣ-ಮಾ-ತ್ರ-ದಲ್ಲಿ
ಕ್ಷಣವೂ
ಕ್ಷಣವೇ
ಕ್ಷಣ-ವೊಂ-ದೊಂದೂ
ಕ್ಷಣಾ-ರ್ಧ-ದಲ್ಲಿ
ಕ್ಷಣಿಕ
ಕ್ಷತ್ರಿಯ
ಕ್ಷತ್ರಿ-ಯರು
ಕ್ಷಮಿ-ಸಿ-ದ್ದಾರೆ
ಕ್ಷಮೆ
ಕ್ಷಮೆ-ಯ-ನ್ನೇಕೆ
ಕ್ಷಾಮ
ಕ್ಷೀಣ-ಗೊಂ-ಡಿತು
ಕ್ಷೀಣ-ಗೊಂಡು
ಕ್ಷೀಣ-ವಾ-ಗಿತ್ತು
ಕ್ಷುದಿ-ರಾಮ
ಕ್ಷುದ್ರ
ಕ್ಷುದ್ರ-ವಾ-ದುದು
ಕ್ಷುಲ್ಲಕ
ಕ್ಷುಲ್ಲ-ಕ-ವಾ-ದ-ದ್ದನ್ನು
ಕ್ಷೆಮೆ
ಕ್ಷೇತ್ರ
ಕ್ಷೇತ್ರಕ್ಕೆ
ಕ್ಷೇತ್ರ-ಗಳ
ಕ್ಷೇತ್ರ-ಗಳಲ್ಲಿ
ಕ್ಷೇತ್ರ-ಗಳು
ಕ್ಷೇತ್ರದ
ಕ್ಷೇತ್ರ-ದಲ್ಲಿ
ಕ್ಷೇತ್ರ-ದಲ್ಲೇ
ಕ್ಷೇತ್ರ-ದೆ-ಡೆಗೆ
ಕ್ಷೇತ್ರ-ವಾ-ಗಿ-ರು-ವಾಗ
ಕ್ಷೇಮ-ಸ-ಮಾ-ಚಾ-ರ-ವನ್ನು
ಖ
ಖಂಡ-ಅ-ಖಂಡ
ಖಂಡಿತ
ಖಂಡಿ-ತ-ವಾಗಿ
ಖಂಡಿ-ತ-ವಾ-ಗಿಯೂ
ಖಂಡಿ-ತ-ವಾದ
ಖಂಡಿ-ತ-ವಾದಿ
ಖಂಡಿ-ಸಲು
ಖಂಡಿಸಿ
ಖಂಡಿ-ಸಿ-ದರು
ಖಂಡಿ-ಸಿ-ಬಿ-ಡು-ತ್ತಿದ್ದ
ಖಂಡಿ-ಸು-ತ್ತೇನೆ
ಖಂಡಿ-ಸು-ವಾಗ
ಖಚಿ-ತ-ಪ-ಡಿ-ಸಿ-ಕೊಂ-ಡದ್ದು
ಖಚಿ-ತ-ವಾ-ಯಿತು
ಖಡ-ಕ್ಕಾಗಿ
ಖಡಾ-ಖಂ-ಡಿ-ತ-ವಾಗಿ
ಖಡ್ಗ
ಖಡ್ಗ-ಮೃ-ಗ-ದೊಲು
ಖರ್ಚನ್ನು
ಖರ್ಚನ್ನೂ
ಖರ್ಚಾ-ಗು-ತ್ತಿದೆ
ಖರ್ಚಿಗೆ
ಖರ್ಚಿ-ನಲ್ಲೇ
ಖರ್ಚು
ಖರ್ಚು-ಮಾಡಿ
ಖರ್ಚು-ಮಾ-ಡು-ತ್ತಿದ್ದ
ಖರ್ಚು-ಮಾ-ಡು-ತ್ತಿದ್ದೆ
ಖರ್ಚೂ
ಖಲ್ವಿದಂ
ಖಾನ-ನಿಂದ
ಖಾನೆಯ
ಖಾನೆ-ಯಲ್ಲಿ
ಖಾನ್
ಖಾನ್-ಇ-ವ-ರಿಂ-ದಲೂ
ಖಾರ
ಖಾರ-ಮುಂದು
ಖಾರ-ವಾಗಿ
ಖಾರ-ವಾ-ಗಿಯೇ
ಖಾರವೇ
ಖಾರವೋ
ಖಿಚ-ಡಿ-ಒಂದು
ಖಿಚ-ಡಿಗೆ
ಖಿನ್ನ-ವಾ-ಗಿತ್ತು
ಖುಷಿ
ಖುಷಿ-ಪ-ಡಿ-ಸಲು
ಖುಷಿ-ಯಾ-ಗಿ-ಡು-ತ್ತಿದ್ದ
ಖುಷಿ-ಯಿಂ-ದಿ-ರ-ಬೇಕು
ಖುಷಿಯೊ
ಖೇತ್ರಿ
ಖೇದ
ಖೇದ-ಗೊಂಡ
ಖೇದ-ವಾ-ಯಿತು
ಖೋಟಾ
ಖ್ಯಾತ
ಗಂಗಾ
ಗಂಗಾ-ಜ-ಲ-ದಿಂದ
ಗಂಗಾ-ತೀರ
ಗಂಗಾ-ತೀ-ರದ
ಗಂಗಾ-ತೀ-ರ-ದಲ್ಲಿ
ಗಂಗಾ-ತೀ-ರ-ದ-ಲ್ಲೊಂದು
ಗಂಗಾ-ಧರ
ಗಂಗಾ-ನ-ದಿಗೆ
ಗಂಗಾ-ನ-ದಿಯ
ಗಂಗಾ-ನ-ದಿ-ಯನ್ನು
ಗಂಗಾ-ನ-ದಿ-ಯಲ್ಲಿ
ಗಂಗಾ-ಸ್ನಾನ
ಗಂಗಾ-ಸ್ನಾ-ನ-ಇ-ವು-ಗ-ಳ-ನ್ನೆಲ
ಗಂಗಾ-ಸ್ನಾ-ನವೂ
ಗಂಗೆ
ಗಂಗೆಗೆ
ಗಂಗೆ-ಗೇಂ
ಗಂಗೆ-ಗೇನ್
ಗಂಗೆಯ
ಗಂಗೆ-ಯಲ್ಲಿ
ಗಂಗೋತ್ರಿ
ಗಂಜಿ
ಗಂಜಿಯ
ಗಂಜಿ-ಯ-ನ್ನೆಲ್ಲ
ಗಂಟಲ
ಗಂಟ-ಲ-ನೋ-ವಿ-ನಿಂ-ದಾಗಿ
ಗಂಟ-ಲನ್ನು
ಗಂಟ-ಲಲ್ಲಿ
ಗಂಟ-ಲಿಗೆ
ಗಂಟ-ಲಿ-ನಿಂದ
ಗಂಟಲು
ಗಂಟ-ಲು-ಬೇನೆ
ಗಂಟಿ-ಕ್ಕು-ವು-ದರ
ಗಂಟಿ-ನಲ್ಲಿ
ಗಂಟು
ಗಂಟು-ಕ-ಟ್ಟಿ-ಕೊಳ್ಳಿ
ಗಂಟು-ಗಳನ್ನೂ
ಗಂಟು-ಮೋರೆ
ಗಂಟೆ
ಗಂಟೆ-ಗ-ಟ್ಟಲೆ
ಗಂಟೆ-ಗಳ
ಗಂಟೆ-ಗ-ಳ-ಅ-ಥವಾ
ಗಂಟೆ-ಗಳನ್ನು
ಗಂಟೆ-ಗಳಲ್ಲಿ
ಗಂಟೆ-ಗಳು
ಗಂಟೆ-ಗಿಂ-ತಲೂ
ಗಂಟೆಗೆ
ಗಂಟೆಗೇ
ಗಂಟೆಯ
ಗಂಟೆ-ಯ-ವ-ರೆಗೂ
ಗಂಟೆ-ಯಾ-ಗಿ-ರು-ತ್ತಿತ್ತು
ಗಂಟೆಯೂ
ಗಂಡಲ್ಲ
ಗಂಡಾಂ-ತರ
ಗಂಡಾಳು
ಗಂಡು
ಗಂಡು-ಗಲಿ
ಗಂಡು-ಗ-ಲಿ-ಗ-ಳಂ-ತಿ-ರ-ಬೇ-ಕೆಂ-ಬುದು
ಗಂಡು-ಗ-ಲಿ-ಯಾ-ಗ-ಲಿ-ರುವ
ಗಂಡು-ಪಾ-ರಿ-ವಾ-ಳದ
ಗಂಡು-ಮಗ
ಗಂಡು-ಮ-ಗುವೂ
ಗಂಡು-ಸಂ-ತಾ-ನ-ವನ್ನು
ಗಂಧ
ಗಂಧದ
ಗಂಧ-ರ್ವ-ಸ್ವ-ರ-ವನ್ನು
ಗಂಧ-ವಿ-ರ-ಲಿಲ್ಲ
ಗಂಧವೇ
ಗಂಭೀರ
ಗಂಭೀ-ರ-ಚಿಂ-ತ-ನೆ-ಯಲ್ಲಿ
ಗಂಭೀ-ರತೆ
ಗಂಭೀ-ರ-ಭಾವ
ಗಂಭೀ-ರ-ವಾಗಿ
ಗಂಭೀ-ರ-ವಾ-ಗಿತ್ತು
ಗಂಭೀ-ರ-ವಾ-ಗಿದೆ
ಗಂಭೀ-ರ-ವಾ-ಗುತ್ತ
ಗಂಭೀ-ರ-ವಾ-ಗು-ತ್ತಿದೆ
ಗಂಭೀ-ರ-ವಾ-ಹಿ-ನಿ-ಯಾದ
ಗಗ-ನ-ಚಂದ್ರ
ಗಗ-ನ-ಬ-ಲೆ-ಯನೆ
ಗಗ-ನವೇ
ಗಗನ್
ಗಗ-ನ್ಬಾ-ಬು-ವಿನ
ಗಜ
ಗಟ್ಟಿ
ಗಟ್ಟಿ-ಗನೋ
ಗಟ್ಟಿ-ಗರೇ
ಗಟ್ಟಿ-ಮ-ನ-ಸ್ಸಿನ
ಗಟ್ಟಿ-ಯಾಗಿ
ಗಟ್ಟಿ-ಯಾಗೇ
ಗಟ್ಟಿ-ಹೋ-ದಂ-ತಾಗಿ
ಗಡವ
ಗಡ-ಸಾಗಿ
ಗಡ-ಸಾ-ಯಿತು
ಗಡ-ಸು-ದ-ನಿ-ಯಲ್ಲಿ
ಗಡಿ-ಬಿಡಿ
ಗಡಿ-ಬಿ-ಡಿ-ಯಿಂದ
ಗಡಿ-ಯನ್ನು
ಗಣ-ನೆಗೆ
ಗಣಿತ
ಗಣಿ-ತ-ದಲ್ಲಿ
ಗಣಿ-ಸ-ದ-ವ-ರನು
ಗಣಿ-ಸ-ಬ-ಹುದು
ಗಣೇಶ
ಗಣೇ-ಶ-ಪ್ರ-ಯಾಗ
ಗಣೇ-ಶ-ಪ್ರ-ಯಾ-ಗಕ್ಕೆ
ಗಣ್ಣಿನ
ಗಣ್ಯ
ಗಣ್ಯ-ಸ್ಥಾ-ನ-ವನ್ನು
ಗತ-ವೈ-ಭ-ವವೂ
ಗತಿ
ಗತಿ-ಯಾ-ಗು-ತ್ತಿತ್ತು
ಗತಿ-ಯಿ-ಲ್ಲದೆ
ಗತಿ-ಯೇನು
ಗತ್ತು
ಗದ-ರಿ-ದರು
ಗದ-ರಿ-ದರೂ
ಗದ-ರಿಸಿ
ಗದ-ರಿ-ಸಿ-ದಾಗ
ಗದ-ರಿ-ಸುತ್ತ
ಗದ-ರಿ-ಸು-ತ್ತಿ-ದ್ದರು
ಗದಾ-ಧರ
ಗದಾ-ಧ-ರನ
ಗದಾ-ಧ-ರ-ನನ್ನು
ಗದಾ-ಧ-ರನೂ
ಗದೆ
ಗದ್ಗದ
ಗದ್ಯ-ವಾಗಿ
ಗಮನ
ಗಮ-ನ-ಕೊ-ಟ್ಟಿದ್ದ
ಗಮ-ನ-ಕೊ-ಟ್ಟಿ-ರ-ಬೇಕು
ಗಮ-ನ-ಕೊಟ್ಟು
ಗಮ-ನ-ಕೊ-ಡದೆ
ಗಮ-ನ-ಕೊ-ಡ-ಲಿಲ್ಲ
ಗಮ-ನಕ್ಕೆ
ಗಮ-ನ-ವನ್ನೂ
ಗಮ-ನ-ವಿ-ಟ್ಟಿ-ರು-ವುದು
ಗಮ-ನ-ವಿಟ್ಟು
ಗಮ-ನ-ವಿತ್ತು
ಗಮ-ನ-ವೆಲ್ಲ
ಗಮ-ನವೇ
ಗಮ-ನಾರ್ಹ
ಗಮ-ನಾ-ರ್ಹ-ವಾದ
ಗಮ-ನಿಸ
ಗಮ-ನಿ-ಸ-ತೊ-ಡ-ಗಿ-ದರು
ಗಮ-ನಿ-ಸದೆ
ಗಮ-ನಿ-ಸ-ಬ-ಹು-ದಾ-ಗಿದೆ
ಗಮ-ನಿ-ಸ-ಬ-ಹುದು
ಗಮ-ನಿ-ಸ-ಬೇ-ಕಾ-ಗಿ-ರ-ಲಿಲ್ಲ
ಗಮ-ನಿ-ಸ-ಬೇ-ಕಾ-ದ-ದ್ದಿದೆ
ಗಮ-ನಿ-ಸ-ಬೇಕು
ಗಮ-ನಿ-ಸಲು
ಗಮ-ನಿಸಿ
ಗಮ-ನಿ-ಸಿದ
ಗಮ-ನಿ-ಸಿ-ದರೂ
ಗಮ-ನಿ-ಸಿ-ದೆ-ಒಂದು
ಗಮ-ನಿ-ಸಿ-ದ್ದ-ವನು
ಗಮ-ನಿ-ಸುತ್ತ
ಗಮ-ನಿ-ಸು-ತ್ತಲೇ
ಗಮ-ನಿ-ಸು-ತ್ತಿದ್ದ
ಗಮ-ನಿ-ಸು-ತ್ತಿ-ದ್ದಂ-ತೆಯೇ
ಗಮ-ನಿ-ಸು-ತ್ತಿ-ದ್ದಾನೆ
ಗಮ-ನಿ-ಸು-ತ್ತಿದ್ದೆ
ಗಮ-ನೀಯ
ಗಮ-ನೆಲ್ಲ
ಗಯಾ-ಪ-ಟ್ಟ-ಣಕ್ಕೆ
ಗಯೆಗೆ
ಗಯೆ-ಯಲ್ಲಿ
ಗಯೆ-ಯಿಂದ
ಗರ-ಡಿ-ಮ-ನೆಗೆ
ಗರ-ಡಿ-ಮ-ನೆ-ಯಲ್ಲಿ
ಗರ-ಡಿ-ಯಲ್ಲಿ
ಗರಾಗಿ
ಗರಿಮಾ
ಗರಿ-ಮುರಿ
ಗರುಡ
ಗರೆ-ಯು-ತ್ತಿ-ದ್ದಾರೆ
ಗರ್ಭ-ಗು-ಡಿ-ಯಲ್ಲಿ
ಗರ್ಹಿತ
ಗಲಭೆ
ಗಲ-ಭೆ-ಯ-ನ್ನೆಲ್ಲ
ಗಲ-ಭೆ-ಯೆ-ಬ್ಬಿ-ಸಿ-ದರು
ಗಲಾಟೆ
ಗಲಾ-ಟೆ-ಗಳಿಂದ
ಗಲಾ-ಟೆ-ಗಿಟ್ಟು
ಗಲಾ-ಟೆ-ಗೆಲ್ಲ
ಗಲಾ-ಟೆ-ಯಾ-ದರೂ
ಗಲಾ-ಟೆಯೂ
ಗಲಾ-ಟೆ-ಯೆಲ್ಲ
ಗಲಿ
ಗಲಿ-ಬಿಲಿ
ಗಲಿ-ಬಿ-ಲಿ-ಗೊ-ಳ-ಗಾ-ಗಿ-ರ-ದಿ-ದ್ದರೆ
ಗಲಿ-ಬಿ-ಲಿ-ಯುಂ-ಟಾ-ಗಿ-ಬಿ-ಡು-ತ್ತಿತ್ತು
ಗಲೂ
ಗಲ್ಲಿ
ಗಲ್ಲಿಯ
ಗಲ್ಲು-ಗಂಬ
ಗಳ
ಗಳಂ-ತಿದ್ದ
ಗಳಂತೆ
ಗಳನ್ನು
ಗಳನ್ನೂ
ಗಳ-ನ್ನೆಲ್ಲ
ಗಳ-ನ್ನೆಲ್ಲಾ
ಗಳನ್ನೇ
ಗಳಲ್ಲ
ಗಳ-ಲ್ಲವೇ
ಗಳಲ್ಲಿ
ಗಳಲ್ಲೇ
ಗಳ-ವ-ರೆಗೆ
ಗಳ-ಹಿ-ದ್ದ-ನ್ನೆಲ್ಲ
ಗಳ-ಹು-ವಿ-ಕೆ-ಯನ್ನು
ಗಳಾ-ಗು-ತ್ತಿವೆ
ಗಳಿಂದ
ಗಳಿಂ-ದಲೂ
ಗಳಿಂ-ದಾಗಿ
ಗಳಿಗೆ
ಗಳಿಗೇ
ಗಳಿ-ಸ-ಬೇ-ಕಾ-ದರೆ
ಗಳಿ-ಸ-ಬೇಕು
ಗಳಿ-ಸಲು
ಗಳಿಸಿ
ಗಳಿ-ಸಿ-ಕೊ-ಳ್ಳಲಿ
ಗಳಿ-ಸಿದ
ಗಳಿ-ಸಿದೆ
ಗಳಿ-ಸಿದ್ದ
ಗಳಿ-ಸಿ-ದ್ದ-ವನು
ಗಳಿ-ಸುತ್ತ
ಗಳಿ-ಸು-ತ್ತಿದ್ದ
ಗಳಿ-ಸು-ತ್ತಿ-ದ್ದರು
ಗಳಿ-ಸು-ವು-ದ-ಕ್ಕಲ್ಲ
ಗಳಿ-ಸು-ವುದೇ
ಗಳು
ಗಳುಈ
ಗಳೂ
ಗಳೆ-ನ್ನು-ತ್ತಾ-ರೆಈ
ಗಳೆ-ರಡೂ
ಗಳೆಲ್ಲ
ಗಳೊಂ-ದಿಗೆ
ಗವಾ-ಯಿ-ಗಳ
ಗವಿ-ಯೊಂ-ದ-ರಲ್ಲಿ
ಗಹ-ಗ-ಹಿಸಿ
ಗಹನ
ಗಹ-ನ-ವಾದ
ಗಹ-ನ-ವಾ-ದದ್ದು
ಗಾಂಧೀ-ಜಿ-ಯ-ವರು
ಗಾಂಭೀರ್ಯ
ಗಾಂಭೀ-ರ್ಯ-ದಿಂದ
ಗಾಂಭೀ-ರ್ಯ-ವಿತ್ತು
ಗಾಗಿ
ಗಾಟಿ-ಕೆ-ತುಂ-ಟ-ತ-ನ-ಗಳ
ಗಾಡಿ
ಗಾಡಿಗೆ
ಗಾಡಿಯ
ಗಾಡಿ-ಯಲ್ಲಿ
ಗಾಡಿಯೂ
ಗಾಡು-ತ್ತಾನೆ
ಗಾಢ
ಗಾಢ-ಚಿಂ-ತನೆ
ಗಾಢ-ತಮ
ಗಾಢ-ಧ್ಯಾ-ನ-ದಲ್ಲಿ
ಗಾಢ-ವಾಗಿ
ಗಾಢ-ವಾ-ಗಿ-ದ್ದಂತೆ
ಗಾಢ-ವಾ-ಗಿ-ರು-ತ್ತಿ-ತ್ತೆಂ-ದರೆ
ಗಾಢ-ವಾದ
ಗಾಢ-ವಾ-ದಂ-ತೆಲ್ಲ
ಗಾಢ-ವಾ-ದದ್ದು
ಗಾಢ-ಸ-ಮಾ-ಧಿ-ಯಲ್ಲಿ
ಗಾಢ-ಸ-ಮಾ-ಧಿ-ಸ್ಥ-ರಾ-ಗಿ-ಬಿ-ಟ್ಟರು
ಗಾಣು-ವ-ವ-ರೆಗೆ
ಗಾತ್ರ-ಗ-ಳೆ-ಲ್ಲಿವೆ
ಗಾತ್ರ-ದಲ್ಲಿ
ಗಾತ್ರ-ದಲ್ಲೂ
ಗಾದರೂ
ಗಾನ
ಗಾನ-ವನು
ಗಾನ-ವನ್ನು
ಗಾನ-ಸು-ಧೆ-ಯಲ್ಲಿ
ಗಾಬರಿ
ಗಾಬ-ರಿ-ಗೊಂಡು
ಗಾಬ-ರಿ-ಯಾಗಿ
ಗಾಬ-ರಿ-ಯಾ-ಯಿತು
ಗಾಯ
ಗಾಯಕ
ಗಾಯ-ಕ-ನ-ನ್ನಾಗಿ
ಗಾಯ-ಕರ
ಗಾಯಕ್ಕೆ
ಗಾಯ-ಗಳನ್ನೆಲ್ಲ
ಗಾಯದ
ಗಾಯನ
ಗಾಯ-ನ-ಪ್ರ-ತಿ-ಭೆ-ಯೆ-ನ್ನು-ವುದು
ಗಾಯ-ವಾಗಿ
ಗಾಯವೇ
ಗಾಯೆ
ಗಾರ
ಗಾಳಿ
ಗಾಳಿಗೆ
ಗಿಂತ
ಗಿಂತಲೂ
ಗಿಡ-ಮ-ರ-ಬ-ಳ್ಳಿ-ಗಳ
ಗಿತು
ಗಿತ್ತು
ಗಿದು-ವಂತೆ
ಗಿದ್ದ
ಗಿದ್ದಾಗ
ಗಿದ್ದು
ಗಿನ
ಗಿಯೇ
ಗಿಯೋ
ಗಿರ-ಗಿ-ರನೆ
ಗಿರಿ-ಗು-ಹೆ-ಕಂ-ದ-ರದ
ಗಿರಿ-ನಾರ್
ಗಿರೀಶ
ಗಿರೀ-ಶ-ಚಂದ್ರ
ಗಿರೀ-ಶ-ಚಂ-ದ್ರನ
ಗಿರೀ-ಶ-ಚಂ-ದ್ರನೂ
ಗಿರೀ-ಶನ
ಗಿರೀ-ಶ-ನಂ-ತಹ
ಗಿರೀ-ಶ-ನನ್ನು
ಗಿರೀ-ಶ-ನಿ-ಗಾಗಿ
ಗಿರೀ-ಶ-ನಿಗೆ
ಗಿರೀಶ್
ಗಿರೀ-ಶ್ಚಂದ್ರ
ಗಿಲ್ಲ
ಗಿಳಿ-ಗಳು
ಗೀತಾ-ಮೃತ
ಗೀತೆ
ಗೀತೆ-ಗಳು
ಗೀತೆಯ
ಗೀತೆ-ಯನ್ನು
ಗೀತ್
ಗೀದನೆ
ಗೀಳಿ-ನಂ-ತೆಯೇ
ಗುಂಡು-ಗುಂ-ಡಾದ
ಗುಂಪಿ-ನಲ್ಲಿ
ಗುಂಪಿ-ನಲ್ಲೂ
ಗುಂಪಿ-ನ-ವರು
ಗುಂಪು
ಗುಂಪು-ಗಳನ್ನು
ಗುಂಪು-ಗ-ಳಾಗಿ
ಗುಂಪು-ಗುಂ-ಪಾಗಿ
ಗುಂಪೇ
ಗುಂಯಿ-ಗು-ಡ-ಲಾ-ರಂ-ಭಿ-ಸಿ-ದುವು
ಗುಜು
ಗುಟ್ಟನ್ನು
ಗುಟ್ಟಾಗಿ
ಗುಟ್ಟಿನ
ಗುಟ್ಟು
ಗುಟ್ಟೆಲ್ಲ
ಗುಟ್ಟೇನು
ಗುಡಾಣ
ಗುಡಿ-ಮ-ಸೀ-ದಿ-ಗ-ಳ-ಲ್ಲಾ-ಗಲಿ
ಗುಡಿ-ಯಲ್ಲೋ
ಗುಡಿ-ಯೊ-ಳದು
ಗುಡಿ-ಸ-ಲಿಗೆ
ಗುಡಿ-ಸ-ಲೊಂ-ದ-ರಲ್ಲಿ
ಗುಡು-ಗ-ಲಿ-ರುವ
ಗುಡು-ಗಿ-ದರು
ಗುಡು-ಗುಡಿ
ಗುಡು-ಗು-ಡಿ-ಯನ್ನು
ಗುಡು-ಗು-ತ್ತಿದ್ದ
ಗುಣ
ಗುಣ-ನ-ಡತೆ
ಗುಣ-ಮ-ಹಿ-ಮೆ-ಗಳನ್ನು
ಗುಣ-ವ್ಯ-ಕ್ತಿ-ತ್ವ-ತ-ತ್ತ್ವ-ಗಳನ್ನು
ಗುಣ-ಕ-ಥನ
ಗುಣ-ಕ್ಕಾ-ದರೂ
ಗುಣಕ್ಕೆ
ಗುಣ-ಗಳ
ಗುಣ-ಗಳನ್ನೂ
ಗುಣ-ಗಳನ್ನೆಲ್ಲ
ಗುಣ-ಗಳಿಂದ
ಗುಣ-ಗ-ಳಿವೆ
ಗುಣ-ಗಳು
ಗುಣ-ಗಾನ
ಗುಣ-ತ್ರಯ
ಗುಣ-ದ-ವನೋ
ಗುಣ-ನ-ಡ-ತೆಈ
ಗುಣ-ನ-ಡ-ತೆ-ಗಳನ್ನೆಲ್ಲ
ಗುಣ-ಪ-ಡಿ-ಸ-ಬಲ್ಲ
ಗುಣ-ಪ-ಡಿ-ಸು-ವಂತೆ
ಗುಣ-ಮಾ-ಧುರ್ಯ
ಗುಣ-ಮು-ಖ-ರಾ-ಗಲು
ಗುಣ-ಮು-ಖ-ರಾ-ಗಿ-ರ-ಲಿಲ್ಲ
ಗುಣ-ಲ-ಕ್ಷ-ಣ-ಗಳನ್ನು
ಗುಣ-ಲ-ಕ್ಷ-ಣ-ಗ-ಳಿ-ಲ್ಲ-ದಿ-ದ್ದಲ್ಲಿ
ಗುಣ-ವಂ-ತ-ರ-ನ್ನಾಗಿ
ಗುಣ-ವನ್ನು
ಗುಣ-ವ-ನ್ನು-ಎಂ-ದೆಂ-ದಿಗೂ
ಗುಣ-ವ-ನ್ನು-ಮಗ
ಗುಣ-ವಿ-ಶೇ-ಷ-ಗಳನ್ನು
ಗುಣ-ವಿ-ಶೇ-ಷ-ವೆಂದರೆ
ಗುಣ-ವೆಂದರೆ
ಗುಣ-ಹೊಂ-ದಿದ
ಗುತ್ತ
ಗುತ್ತದೆ
ಗುದ್ದನ್ನೂ
ಗುದ್ದಿ-ಸಿ-ಕೊ-ಳ್ಳಲಿ
ಗುದ್ದು
ಗುನು-ಗ-ಲಾ-ರಂ-ಭಿ-ಸಿದ
ಗುನು-ಗಿ-ಕೊ-ಳ್ಳು-ತ್ತಿ-ದ್ದರು
ಗುಪ್ತ
ಗುಪ್ತನ
ಗುಮಾಸ್ತೆ
ಗುಮಾ-ಸ್ತೆಯ
ಗುಮಾ-ಸ್ತೆ-ಯನ್ನು
ಗುರ-ಯಾ-ದ-ವರ
ಗುರಿ
ಗುರಿಗೆ
ಗುರಿ-ಪ-ಡಿ-ಸದೆ
ಗುರಿ-ಪ-ಡಿಸಿ
ಗುರಿ-ಮು-ಟ್ಟಲು
ಗುರಿಯ
ಗುರಿ-ಯ-ನ್ನಾ-ಗಿ-ಸಿ-ಕೊಂಡು
ಗುರಿ-ಯನ್ನು
ಗುರಿ-ಯಾ-ಗ-ಬೇ-ಕಾ-ದರೂ
ಗುರಿ-ಯಾ-ಗ-ಬೇ-ಕಾ-ಯಿತು
ಗುರಿ-ಯಾಗಿ
ಗುರಿ-ಯಾ-ಗಿತ್ತು
ಗುರಿ-ಯಾಗು
ಗುರಿ-ಯಾದ
ಗುರಿ-ಯಾ-ದರು
ಗುರಿ-ಯಾ-ದಾಗ
ಗುರಿ-ಯಾ-ಯಿತು
ಗುರಿಯು
ಗುರಿ-ಯೆಂ-ದರೆ
ಗುರು
ಗುರು-ಶಿಷ್ಯ
ಗುರು-ಆ-ಶ್ಚ-ರ್ಯ-ಕರ
ಗುರು-ಎ-ಲ್ಲವೂ
ಗುರು-ಕು-ಲ-ದಂ-ತಹ
ಗುರು-ಗಳ
ಗುರು-ಗ-ಳಂತೆ
ಗುರು-ಗ-ಳ-ನ್ನಾ-ಗಲಿ
ಗುರು-ಗ-ಳನ್ನೇ
ಗುರು-ಗ-ಳಾ-ಗ-ಬ-ಲ್ಲರು
ಗುರು-ಗ-ಳಿ-ಗಲ್ಲ
ಗುರು-ಗಳು
ಗುರು-ಗಳೂ
ಗುರು-ತನ್ನು
ಗುರು-ತರ
ಗುರು-ತಿ-ಸ-ದಿ-ರ-ಬೇಡ
ಗುರು-ತಿ-ಸ-ಬ-ಹು-ದಾ-ಗಿತ್ತು
ಗುರು-ತಿ-ಸ-ಬ-ಹುದು
ಗುರು-ತಿ-ಸ-ಬೇ-ಕೆಂ-ಬುದೇ
ಗುರು-ತಿಸಿ
ಗುರು-ತಿ-ಸಿದ
ಗುರು-ತಿ-ಸಿ-ದಳು
ಗುರು-ತಿ-ಸಿ-ದ-ವ-ರುಂಟು
ಗುರು-ತಿ-ಸಿದ್ದ
ಗುರು-ತಿ-ಸಿ-ದ್ದ-ರಲ್ಲಿ
ಗುರು-ತಿ-ಸಿ-ಬಿ-ಟ್ಟರು
ಗುರು-ತಿ-ಸು-ವು-ದಿಲ್ಲ
ಗುರುತು
ಗುರು-ತು-ಹಾ-ಕಿ-ಕೊಂಡು
ಗುರು-ತು-ಹಿ-ಡಿ-ದಿ-ದ್ದಾರೆ
ಗುರು-ತು-ಹಿ-ಡಿ-ದು-ಬಿಟ್ಟೆ
ಗುರು-ದೇ-ವನ
ಗುರು-ದೇ-ವ-ನನ್ನು
ಗುರು-ದೇ-ವ-ನಿಗೆ
ಗುರು-ದೇ-ವನೇ
ಗುರು-ದೇ-ವರು
ಗುರು-ಪು-ತ್ರರು
ಗುರು-ಭಕ್ತಿ
ಗುರು-ಭಾಯಿ
ಗುರು-ಭಾ-ಯಿ-ಗಳ
ಗುರು-ಭಾ-ಯಿ-ಗಳನ್ನು
ಗುರು-ಭಾ-ಯಿ-ಗಳನ್ನೆಲ್ಲ
ಗುರು-ಭಾ-ಯಿ-ಗ-ಳ-ಲ್ಲೊ-ಬ್ಬರು
ಗುರು-ಭಾ-ಯಿ-ಗಳಾ
ಗುರು-ಭಾ-ಯಿ-ಗಳಿಂದ
ಗುರು-ಭಾ-ಯಿ-ಗ-ಳಿಗೂ
ಗುರು-ಭಾ-ಯಿ-ಗ-ಳಿಗೆ
ಗುರು-ಭಾ-ಯಿ-ಗಳು
ಗುರು-ಭಾ-ಯಿ-ಗಳೂ
ಗುರು-ಭಾ-ಯಿ-ಗ-ಳೆ-ದು-ರಿಗೆ
ಗುರು-ಭಾ-ಯಿ-ಗ-ಳೆಲ್ಲ
ಗುರು-ಭಾ-ಯಿ-ಗ-ಳೊಂ-ದಿಗೆ
ಗುರು-ಭಾ-ಯಿ-ಗ-ಳೊ-ಡನೆ
ಗುರು-ಭಾ-ಯಿ-ಯನ್ನು
ಗುರು-ಭಾ-ಯಿ-ಯಾದ
ಗುರು-ಮಾ-ರ್ಗ-ದ-ರ್ಶನ
ಗುರು-ವನ್ನು
ಗುರು-ವಾಗಿ
ಗುರು-ವಾ-ಗು-ವಂತೆ
ಗುರು-ವಾ-ದ-ವನು
ಗುರು-ವಿಗೂ
ಗುರು-ವಿಗೆ
ಗುರು-ವಿ-ಗೊಂದು
ಗುರು-ವಿನ
ಗುರು-ವಿ-ನಲ್ಲಿ
ಗುರು-ವಿ-ನಿಂದ
ಗುರು-ವಿ-ನೊ-ಡನೆ
ಗುರುವು
ಗುರುವೂ
ಗುರು-ವೆಂದು
ಗುರು-ಶಿ-ಷ್ಯ-ರಿ-ಬ್ಬರೂ
ಗುರು-ಸ-ಹಾಯ
ಗುರು-ಸೇವೆ
ಗುರು-ಸೇ-ವೆ-ಗಾಗಿ
ಗುರು-ಸೇ-ವೆಯ
ಗುರು-ಸೇ-ವೆ-ಯನ್ನು
ಗುರು-ಸೇ-ವೆ-ಯನ್ನೂ
ಗುರು-ಸೇ-ವೆ-ಯಲ್ಲಿ
ಗುಲಾ-ಮ-ಗಿ-ರಿ-ಯಲ್ಲಿ
ಗುಲಾ-ಮ-ಗಿ-ರಿ-ಯೇನೂ
ಗುಲಾ-ಮ-ನಾ-ಗಿ-ರು-ವು-ದ-ಕ್ಕಿಂತ
ಗುಲಾ-ಮ-ನಾ-ದ-ವನು
ಗುಳು-ಗು-ಳು-ಗುಳು
ಗುಳು-ಗು-ಳು-ಗು-ಳು-ಗುಳು
ಗುವ
ಗುಹ
ಗುಹೆ
ಗುಹೆಯ
ಗುಹೆ-ಯನ್ನು
ಗುಹೆ-ಯಲ್ಲಿ
ಗುಹೆ-ಯಲ್ಲೇ
ಗುಹೆ-ಯೊ-ಳ-ಗಿಂದ
ಗೂಡಿ-ಕೊಂಡು
ಗೂಡಿದ
ಗೂಡಿನ
ಗೂಡಿ-ನ-ಲ್ಲಿ-ಟ್ಟು-ಕೊ-ಳ್ಳ-ಬೇಕು
ಗೂಢಾರ್ಥ
ಗೂಢಾ-ರ್ಥ-ವನ್ನು
ಗೃಹದ
ಗೃಹಸ್ಥ
ಗೃಹ-ಸ್ಥ-ಜೀ-ವ-ನಕ್ಕೆ
ಗೃಹ-ಸ್ಥ-ಧ-ರ್ಮ-ಕ್ಕನು
ಗೃಹ-ಸ್ಥ-ಧ-ರ್ಮ-ವನ್ನು
ಗೃಹ-ಸ್ಥ-ನಂತೆ
ಗೃಹ-ಸ್ಥ-ನಾ-ಗಿ-ದ್ದು-ಕೊಂಡೇ
ಗೃಹ-ಸ್ಥ-ನಾ-ದರೂ
ಗೃಹ-ಸ್ಥ-ನೊ-ಬ್ಬನ
ಗೃಹ-ಸ್ಥ-ಭ-ಕ್ತ-ರಾದ
ಗೃಹ-ಸ್ಥ-ಭ-ಕ್ತ-ರಿಂದ
ಗೃಹ-ಸ್ಥ-ಭ-ಕ್ತ-ರಿಗೆ
ಗೃಹ-ಸ್ಥ-ಭ-ಕ್ತ-ರಿ-ಗೆಲ್ಲ
ಗೃಹ-ಸ್ಥ-ಭ-ಕ್ತರು
ಗೃಹ-ಸ್ಥ-ಭ-ಕ್ತರೂ
ಗೃಹ-ಸ್ಥ-ರ-ಲ್ಲವೆ
ಗೃಹ-ಸ್ಥ-ರಾಗಿ
ಗೃಹ-ಸ್ಥ-ರಾ-ಗಿ-ದ್ದ-ವ-ರ-ಲ್ಲವೆ
ಗೃಹ-ಸ್ಥ-ರಾ-ಗಿ-ದ್ದು-ಕೊಂಡೇ
ಗೃಹ-ಸ್ಥ-ರಾ-ದರೂ
ಗೃಹ-ಸ್ಥ-ರಿಗೂ
ಗೃಹ-ಸ್ಥ-ರಿಗೆ
ಗೃಹ-ಸ್ಥರು
ಗೃಹ-ಸ್ಥರೂ
ಗೃಹ-ಸ್ಥ-ರೆಲ್ಲ
ಗೃಹೀ-ಭಕ್ತ
ಗೃಹೀ-ಭ-ಕ್ತ-ನಾದ
ಗೃಹೀ-ಭ-ಕ್ತನೂ
ಗೃಹೀ-ಭ-ಕ್ತರ
ಗೃಹೀ-ಭ-ಕ್ತ-ರಲ್ಲಿ
ಗೃಹೀ-ಭ-ಕ್ತರು
ಗೆದ್ದ
ಗೆದ್ದ-ವ-ನಿ-ಗಿಂತ
ಗೆದ್ದ-ವನು
ಗೆದ್ದ-ವನೇ
ಗೆದ್ದಿತು
ಗೆದ್ದು
ಗೆದ್ದು-ಕೊ-ಳ್ಳುವ
ಗೆರೆ
ಗೆಲು
ಗೆಲು-ವಿನ
ಗೆಲ್ಲ
ಗೆಲ್ಲದ
ಗೆಲ್ಲ-ಬ-ಲ್ಲನೋ
ಗೆಲ್ಲಲು
ಗೆಲ್ಲು-ವು-ದಂತೂ
ಗೆಲ್ಲು-ವುದನ್ನು
ಗೆಳೆಯ
ಗೆಳೆ-ಯ-ನನ್ನು
ಗೆಳೆ-ಯ-ನಾಗಿ
ಗೆಳೆ-ಯ-ನಿಗೆ
ಗೆಳೆ-ಯರ
ಗೆಳೆ-ಯ-ರನ್ನು
ಗೆಳೆ-ಯ-ರಿಂದ
ಗೆಳೆ-ಯ-ರಿ-ಬ್ಬರೂ
ಗೆಳೆ-ಯರು
ಗೆಳೆ-ಯ-ರೆಂ-ಬು-ವ-ರ-ರಿ-ಯರು
ಗೆಳೆ-ಯ-ರೊಂ-ದಿ-ಗಿನ
ಗೆಳೆ-ಯ-ರೊಂ-ದಿಗೆ
ಗೇಟಿನ
ಗೇಟು
ಗೇನೂ
ಗೇರುವ
ಗೇಲಿ
ಗೇಲಿ-ಮಾ-ಡು-ತ್ತಲೇ
ಗೈರು-ಹಾ-ಜ-ರಾ-ಗಿ-ದ್ದಾನೆ
ಗೊಂಡರು
ಗೊಂಡಿತು
ಗೊಂಡಿದೆ
ಗೊಂಡಿ-ದ್ದಾನೆ
ಗೊಂಡಿಲ್ಲ
ಗೊಂಡು
ಗೊಂದು
ಗೊಂಬೆ-ಗಳನ್ನು
ಗೊಣ
ಗೊತ್ತಲ್ಲ
ಗೊತ್ತಾ
ಗೊತ್ತಾ-ಗದೆ
ಗೊತ್ತಾ-ಗ-ಬೇಕು
ಗೊತ್ತಾ-ಗ-ಲಿಲ್ಲ
ಗೊತ್ತಾ-ಗಲೇ
ಗೊತ್ತಾ-ಗಿದೆ
ಗೊತ್ತಾ-ಗಿ-ಬಿ-ಟ್ಟಿತ್ತು
ಗೊತ್ತಾ-ಗಿ-ಬಿ-ಡು-ತ್ತದೆ
ಗೊತ್ತಾ-ಗಿ-ರ-ಲಿಲ್ಲ
ಗೊತ್ತಾ-ಗು-ತ್ತದೆ
ಗೊತ್ತಾ-ಗು-ತ್ತಲೇ
ಗೊತ್ತಾ-ಗು-ತ್ತಿದೆ
ಗೊತ್ತಾ-ದದ್ದು
ಗೊತ್ತಾ-ದಾಗ
ಗೊತ್ತಾ-ದುದು
ಗೊತ್ತಾ-ಯಿತು
ಗೊತ್ತಿ
ಗೊತ್ತಿತ್ತು
ಗೊತ್ತಿದೆ
ಗೊತ್ತಿ-ದೆಯೆ
ಗೊತ್ತಿ-ದೆ-ಯೆಂ-ಬುದು
ಗೊತ್ತಿ-ದ್ದರೆ
ಗೊತ್ತಿ-ರ-ಲಾ-ರದು
ಗೊತ್ತಿ-ರ-ಲಿಲ್ಲ
ಗೊತ್ತಿಲ್ಲ
ಗೊತ್ತಿ-ಲ್ಲವೆ
ಗೊತ್ತಿ-ಲ್ಲ-ವೆಂ-ದಲ್ಲ
ಗೊತ್ತು
ಗೊತ್ತು-ಗು-ರಿ-ಗಳ
ಗೊತ್ತು-ಗು-ರಿ-ಯಿ-ಲ್ಲದೆ
ಗೊತ್ತು-ಪ-ಡಿ-ಸಿದ
ಗೊತ್ತು-ಮಾಡಿ
ಗೊತ್ತು-ಮಾ-ಡಿ-ಟ್ಟು-ಕೊಂ-ಡಿ-ದ್ದರು
ಗೊತ್ತು-ಮಾ-ಡಿ-ದರು
ಗೊತ್ತೆ
ಗೊತ್ತೇ
ಗೊತ್ತೇನು
ಗೊಳಗೇ
ಗೊಳಿ-ಸಲು
ಗೊಳಿ-ಸಿ-ದರು
ಗೊಳಿ-ಸಿ-ದ್ದುವು
ಗೊಳಿಸು
ಗೋಗ-ರೆದ
ಗೋಚರ
ಗೋಚ-ರ-ವಾ-ಗಿ-ಬಿ-ಟ್ಟಿದೆ
ಗೋಚ-ರ-ವಾ-ದುವು
ಗೋಚ-ರ-ವಾ-ಯಿತು
ಗೋಚ-ರಿ-ಸ-ಲೇ-ಬೇಕು
ಗೋಚ-ರಿ-ಸಿ-ಬಿ-ಡು-ತ್ತಿತ್ತು
ಗೋಚ-ರಿ-ಸು-ತ್ತ-ದೆಂ-ದರೆ
ಗೋಚ-ರಿ-ಸು-ತ್ತಿದೆ
ಗೋಚ-ರಿ-ಸು-ವು-ದುಂಟು
ಗೋಜಿಗೆ
ಗೋಡೆ
ಗೋಡೆ-ಗಳ
ಗೋಡೆ-ಗಳನ್ನು
ಗೋಡೆ-ಗಳು
ಗೋಡೆ-ಗಳೂ
ಗೋಡೆಗೆ
ಗೋಡೆಯ
ಗೋಣಿ-ನಾ-ರಿನ
ಗೋಪಾಲ
ಗೋಪಾ-ಲ-ಇ-ವ-ರೊ-ಬ್ಬರೇ
ಗೋಪಾ-ಲ-ನನ್ನು
ಗೋಪಾ-ಲ-ನಿಗೆ
ಗೋಪಾ-ಲ-ನೊ-ಡನೆ
ಗೋಪಾ-ಲ-ನೊ-ಬ್ಬನೇ
ಗೋಪಾಲ್
ಗೋಪಿ-ಕಾ-ವ-ಸ್ತ್ರಾ-ಪ-ಹ-ರ-ಣದ
ಗೋಪುರ
ಗೋಪು-ರ-ಗ-ಳೆಲ್ಲ
ಗೋರಿ-ಗಳು
ಗೋರಿಯೂ
ಗೋಲಿ-ನಿಂದ
ಗೋಲಿ-ಯಾ-ಟ-ಎ-ಲ್ಲವೂ
ಗೋಳು
ಗೋಳು-ಹೊ-ಯ್ದು-ಕೊ-ಳ್ಳು-ವು-ದೆಂ-ದರೆ
ಗೋವ-ರ್ಧನ
ಗೋವ-ರ್ಧ-ನ-ಗಿ-ರಿ-ಯಿಂದ
ಗೋವಿಂ-ದನ
ಗೋಷ್ಠಿ-ಗ-ಳಿಗೆ
ಗೋಸ್ಕ-ರವೇ
ಗೋಸ್ವಾಮಿ
ಗೋಸ್ವಾ-ಮಿಯೇ
ಗೌತಮ
ಗೌರವ
ಗೌರ-ವ-ಪೂ-ಜ್ಯ-ಭಾ-ವ-ಗಳು
ಗೌರ-ವ-ಭ-ಕ್ತಿ-ಗಳ
ಗೌರ-ವಕ್ಕೂ
ಗೌರ-ವಕ್ಕೆ
ಗೌರ-ವ-ದಿಂದ
ಗೌರ-ವ-ದಿಂ-ದಲೇ
ಗೌರ-ವ-ಬು-ದ್ಧಿ-ಯನ್ನೂ
ಗೌರ-ವ-ಭಾ-ವ-ವಿತ್ತು
ಗೌರ-ವ-ವಿತ್ತು
ಗೌರ-ವಾ-ದರ
ಗೌರ-ವಾ-ದ-ರ-ಗ-ಳ-ನ್ನಿ-ಟ್ಟಿದ್ದ
ಗೌರ-ವಾ-ದ-ರ-ಗಳಿಂದ
ಗೌರ-ವಾ-ನ್ವಿತ
ಗೌರ-ವಿಸಿ
ಗೌರ-ವಿ-ಸುವ
ಗೌರ-ವಿ-ಸು-ವು-ದಿ-ಲ್ಲವೋ
ಗೌರಾಂ-ಗನು
ಗೌರಾಂ-ಗರು
ಗ್ರಂಥ-ಋಣ
ಗ್ರಂಥ-ಗಳ
ಗ್ರಂಥ-ಗಳನ್ನು
ಗ್ರಂಥ-ಗಳಲ್ಲಿ
ಗ್ರಂಥ-ಗಳು
ಗ್ರಂಥದ
ಗ್ರಂಥ-ದಂತೆ
ಗ್ರಂಥ-ದಲ್ಲಿ
ಗ್ರಂಥ-ವನ್ನು
ಗ್ರಂಥ-ವನ್ನೇ
ಗ್ರಂಥ-ವಾದ
ಗ್ರಂಥವು
ಗ್ರಂಥಾ-ಯ-ಲಕ್ಕೆ
ಗ್ರಂಥಾ-ಲ-ಯದ
ಗ್ರಂಥಾ-ಲ-ಯ-ದಲ್ಲಿ
ಗ್ರಂಥಾ-ಲ-ಯವೂ
ಗ್ರಂಥಾ-ಲ-ಯ-ವೊಂ-ದ-ರಿಂದ
ಗ್ರಹಕ್ಕೂ
ಗ್ರಹಣ
ಗ್ರಹ-ಣ-ಶಕ್ತಿ
ಗ್ರಹ-ಣ-ಸಾ-ಮರ್ಥ್ಯ
ಗ್ರಹ-ಣ-ಸಾ-ಮ-ರ್ಥ್ಯವೇ
ಗ್ರಹ-ಬಲ
ಗ್ರಹಿಕೆ
ಗ್ರಹಿ-ಸ-ಲಾ-ಗದ
ಗ್ರಹಿ-ಸ-ಲಾ-ರದೆ
ಗ್ರಹಿ-ಸಲು
ಗ್ರಹಿಸಿ
ಗ್ರಹಿ-ಸಿದ
ಗ್ರಹಿ-ಸಿ-ದ-ವನು
ಗ್ರಹಿ-ಸಿ-ದ್ದಳು
ಗ್ರಹಿ-ಸುವ
ಗ್ರಹಿ-ಸು-ವಲ್ಲಿ
ಗ್ರಾಮ
ಗ್ರಾಹ್ಯ
ಗ್ಲಾನಿ
ಗ್ಲಾನಿ-ರ್ಭ-ವತಿ
ಘಂಟಾ-ಘೋ-ಷ-ವಾಗಿ
ಘಂಟೆ-ಶಂ-ಖ-ಜಾ-ಗ-ಟೆ-ಗಳ
ಘಟನೆ
ಘಟ-ನೆ-ಗಳ
ಘಟ-ನೆ-ಗಳನ್ನು
ಘಟ-ನೆ-ಗಳನ್ನೂ
ಘಟ-ನೆ-ಗಳು
ಘಟ-ನೆ-ಗ-ಳೆಲ್ಲ
ಘಟ-ನೆಯ
ಘಟ-ನೆ-ಯನ್ನು
ಘಟ-ನೆ-ಯನ್ನೂ
ಘಟ-ನೆ-ಯ-ನ್ನೆಲ್ಲ
ಘಟ-ನೆ-ಯಲ್ಲ
ಘಟ-ನೆ-ಯಲ್ಲಿ
ಘಟ-ನೆ-ಯಾ-ಯಿತು
ಘಟ-ನೆ-ಯಿಂದ
ಘಟ-ನೆ-ಯಿಂ-ದಾಗಿ
ಘಟ-ನೆ-ಯಿದೆ
ಘಟ-ನೆ-ಯೊಂದು
ಘಟ-ನೆಯೋ
ಘಟ-ಸರ್ಪ
ಘಟಿ-ಸ-ಬ-ಹುದು
ಘಟ್ಟ
ಘನ
ಘನ-ತ-ರ-ವಾದ
ಘನ-ತ-ರ-ವಾ-ದದ್ದು
ಘನತೆ
ಘನ-ತೆ-ಯನ್ನು
ಘನ-ತೆ-ಯಿದೆ
ಘನ-ತೆ-ವೆ-ತ್ತ-ವರು
ಘನ-ವಾ-ಗಿ-ದ್ದಾ-ವೆಂದು
ಘನ-ವಾದ
ಘನ-ವ್ಯ-ಕ್ತಿ-ತ್ವದ
ಘನ-ಸ-ತ್ಯ-ವನ್ನು
ಘನೀ-ಭೂ-ತ-ವಾಗಿ
ಘರ-ವಾ-ಲಿನ
ಘರ-ವಾ-ಲಿ-ನ-ವರೆಲ್ಲ
ಘರ-ವಾಲ್
ಘಳಿಗೆ
ಘಳಿ-ಗೆ-ಯಲ್ಲಿ
ಘಳಿ-ಗೆ-ಯಲ್ಲೇ
ಘಳಿ-ಗೆ-ಯೊಂ-ದ-ರಲ್ಲಿ
ಘಾಜೀ-ಪು-ದಕ್ಕೆ
ಘಾಜೀ-ಪು-ರಕ್ಕೆ
ಘಾಜೀ-ಪು-ರ-ದ-ಲ್ಲಾದ
ಘಾಜೀ-ಪು-ರ-ದಲ್ಲಿ
ಘಾಜೀ-ಪು-ರ-ದ-ಲ್ಲಿದ್ದ
ಘಾಜೀ-ಪು-ರ-ದ-ಲ್ಲಿ-ದ್ದಾಗ
ಘಾಜೀ-ಪು-ರ-ದಿಂದ
ಘಾಟ್
ಘುಸೂರಿ
ಘೋರ
ಘೋಷ-ಗಳು
ಘೋಷ-ಣೆ-ಯಾ-ಗಿತ್ತು
ಘೋಷ-ಣೆ-ಯಿಂದ
ಘೋಷನ
ಘೋಷನೂ
ಘೋಷಿ-ಸಿ-ಕೊಂ-ಡಿತ್ತು
ಘೋಷಿ-ಸಿ-ದರು
ಘೋಷಿ-ಸಿ-ದ್ದರೋ
ಘೋಷಿಸು
ಘೋಷಿ-ಸುತ್ತ
ಘೋಷಿ-ಸು-ತ್ತದೆ
ಘೋಷಿ-ಸು-ತ್ತ-ದೆ-ನಾ-ಯ-ಮಾತ್ಮಾ
ಘೋಷಿ-ಸು-ತ್ತಿ-ದ್ದರೆ
ಘೋಷಿ-ಸು-ತ್ತಿ-ದ್ದಾರೆ
ಘೋಷಿ-ಸು-ವು-ದ-ಕ್ಕಾ-ಗಿಯೇ
ಘೋಷ್
ಚಂಚಲ
ಚಂಚ-ಲ-ಗೊ-ಳಿ-ಸು-ವು-ದಾ-ಗಲಿ
ಚಂಚ-ಲ-ತೆ-ಯಿಂದ
ಚಂಡ-ಮಾ-ರು-ತವು
ಚಂದ
ಚಂದ-ಮಾಮ
ಚಂದವೋ
ಚಂದ್ರ-ಮ-ಣಿ-ದೇ-ವಿಗೂ
ಚಂದ್ರ-ಮಣೀ
ಚಂದ್ರೇ-ಶ್ವರ
ಚಂಪ-ಕ-ಪುಷ್ಪ
ಚಕಿತ
ಚಕಿ-ತ-ಗೊಂಡ
ಚಕ್ಕರ್
ಚಕ್ರಕ್ಕೆ
ಚಕ್ರ-ವರ್ತಿ
ಚಕ್ರ-ವ-ರ್ತಿಗೆ
ಚಕ್ರ-ವ-ರ್ತಿಯ
ಚಕ್ರ-ವ-ರ್ತಿ-ಯಂತೆ
ಚಕ್ರ-ವ-ರ್ತಿ-ಯಾಗಿ
ಚಕ್ರ-ವ-ರ್ತಿ-ಯಾದ
ಚಕ್ರ-ವ-ರ್ತಿ-ಯೊಂ-ದಿಗೆ
ಚಕ್ರಾ-ಧಿ-ಪ-ತ್ಯ-ಗಳ
ಚಚ್ಚಿ-ಕೊಂಡು
ಚಟ
ಚಟ-ಪಟ
ಚಟರ್ಜಿ
ಚಟಾ-ಕಿ-ಗಳನ್ನು
ಚಟು-ವ-ಟಿಕೆ
ಚಟು-ವ-ಟಿ-ಕೆ-ಸಾ-ಧ-ನೆ-ಗಳ
ಚಟು-ವ-ಟಿ-ಕೆ-ಗಳಲ್ಲಿ
ಚಟು-ವ-ಟಿ-ಕೆ-ಯ-ನ್ನೆಲ್ಲ
ಚಟು-ವ-ಟಿ-ಕೆ-ಯಿಂದ
ಚಟು-ವ-ಟಿ-ಕೆ-ಯಿಂ-ದಿ-ರು-ತ್ತಿ-ದ್ದು-ದೇನೋ
ಚಟ್ಟೋ-ಪಾ-ಧ್ಯಾ-ಯ-ರೆಂಬ
ಚಡ-ಪ-ಡಿ-ಕೆ-ಯನ್ನು
ಚಡ-ಪ-ಡಿ-ಕೆ-ಯ-ನ್ನುಂ-ಟು-ಮಾ-ಡು-ತ್ತಿತ್ತು
ಚಡ-ಪ-ಡಿ-ಕೆ-ಯನ್ನೂ
ಚಡ-ಪ-ಡಿ-ಕೆ-ಯಲ್ಲಿ
ಚಡ-ಪ-ಡಿ-ಕೆ-ಯಿಂ-ದಾಗಿ
ಚಡ-ಪ-ಡಿ-ಸ-ಲಾ-ರಂ-ಭಿ-ಸಿ-ದ್ದಾರೆ
ಚಡ-ಪ-ಡಿ-ಸಿದ
ಚಡ-ಪ-ಡಿ-ಸು-ತ್ತಿ-ದ್ದಂತೆ
ಚಡ-ಪ-ಡಿ-ಸು-ತ್ತಿ-ದ್ದರು
ಚಡ-ಪ-ಡಿ-ಸು-ತ್ತಿ-ದ್ದಾನೆ
ಚಡ-ಪ-ಡಿ-ಸು-ತ್ತಿ-ದ್ದು-ದನ್ನು
ಚಡ-ಪ-ಡಿ-ಸು-ತ್ತಿದ್ದೆ
ಚಡ-ಪ-ಡಿ-ಸು-ತ್ತಿ-ರು-ವಂ-ತಿತ್ತು
ಚಡ-ಪ-ಡಿ-ಸು-ತ್ತಿ-ರು-ವುದು
ಚಡ್ಡಿ
ಚತುರ
ಚತು-ರೋ-ಪಾ-ಯ-ಗಳೂ
ಚದು-ರಂ-ಗ-ವನ್ನು
ಚದುರಿ
ಚನೆ-ಯನ್ನು
ಚಪಲ
ಚಪ-ಲ-ಚಿ-ತ್ತ-ರಾಗಿ
ಚಪ್ಪ-ಲಿ-ಯಿಲ್ಲ
ಚಪ್ಪಾಳೆ
ಚಮ-ತ್ಕಾ-ರಕ್ಕೆ
ಚರ-ಮಾ-ವಸ್ಥೆ
ಚರಿತ
ಚರಿತ್ರೆ
ಚರಿ-ತ್ರೆಯ
ಚರಿ-ತ್ರೆ-ಯಲ್ಲಿ
ಚರಿ-ಸ-ಬೇಕು
ಚರಿ-ಸು-ತ್ತಿ-ರುವ
ಚರ್ಚಿಸಿ
ಚರ್ಚಿ-ಸುತ್ತ
ಚರ್ಚಿ-ಸು-ತ್ತಿ-ದ್ದರು
ಚರ್ಚೆ
ಚರ್ಚೆ-ಇ-ವು-ಗ-ಳಿಗೆ
ಚರ್ಚೆ-ಗಳನ್ನು
ಚರ್ಚೆ-ಗಳಲ್ಲಿ
ಚರ್ಚೆ-ಮಾಡಿ
ಚರ್ಚೆಯ
ಚರ್ಚೆ-ಯನ್ನು
ಚರ್ಚೆ-ಯೆ-ದ್ದಿತು
ಚರ್ಚೆಯೇ
ಚರ್ಚ್
ಚರ್ಮದ
ಚಲ-ನ-ವ-ಲನ
ಚಲ-ನ-ವ-ಲ-ನ-ದಲ್ಲೂ
ಚಲ-ನ-ವ-ಲ-ನ-ವನ್ನೂ
ಚಲ-ನ-ವ-ಲ-ನವೂ
ಚಲ-ನೆ-ಯನ್ನೇ
ಚಲಾ-ಯಿ-ಸುತ್ತ
ಚಲಿ-ಸುತ್ತ
ಚಲ್ತಾ
ಚಳ-ವ-ಳಿ-ಗಾ-ರರೇ
ಚಳ-ವ-ಳಿ-ಯನ್ನು
ಚಳ-ವ-ಳಿ-ಯಾ-ಗಲಿ
ಚಳಿ
ಚಳಿ-ಗಾಲ
ಚಳಿ-ಗಾ-ಲದ
ಚಳಿ-ಯನ್ನು
ಚಳಿ-ಯಿಂದ
ಚಹರೆ
ಚಹಾ
ಚಾಕ-ಚ-ಕ್ಯ-ತೆ-ಯಿಂದ
ಚಾಗ-ವು-ದಿಟ
ಚಾಗಿಗೆ
ಚಾಗಿಯ
ಚಾಚಿ
ಚಾಚಿ-ಕೊಂಡು
ಚಾಚಿ-ಕೊ-ಳ್ಳು-ವುದು
ಚಾಟೂ-ಕ್ತಿ-ಗಳಿಂದ
ಚಾಡಿ
ಚಾತು-ರ್ಯ-ದಿಂದ
ಚಾಪೆ
ಚಾಪೆ-ಯನ್ನು
ಚಾರಿ-ತ್ರಿಕ
ಚಾರಿತ್ರ್ಯ
ಚಾರಿ-ತ್ರ್ಯ-ವನ್ನು
ಚಾರಿ-ತ್ರ್ಯ-ವನ್ನೂ
ಚಾರ್ವಾಕ
ಚಾವಟಿ
ಚಿಂತ-ಕ-ನೊಬ್ಬ
ಚಿಂತ-ಕರ
ಚಿಂತ-ನ-ಶೀಲ
ಚಿಂತ-ನ-ಶೀ-ಲ-ನಾದ
ಚಿಂತ-ನಾ-ಧಾ-ರೆಯ
ಚಿಂತನೆ
ಚಿಂತ-ನೆ-ಯಲ್ಲಿ
ಚಿಂತಾ-ಕ್ರಾಂ-ತ-ನಾ-ಗದೆ
ಚಿಂತಾ-ಕ್ರಾಂ-ತ-ನಾ-ಗಿ-ದ್ದಾನೆ
ಚಿಂತಾ-ಜ-ನ-ಕ-ವಾ-ಗಿತ್ತು
ಚಿಂತಾ-ವಿ-ಲಾ-ಪ-ರ-ಹಿತಂ
ಚಿಂತಿ-ಸದೆ
ಚಿಂತಿ-ಸ-ಬೇಡ
ಚಿಂತಿ-ಸ-ಲಾ-ರಂ-ಭಿ-ಸಿತು
ಚಿಂತಿ-ಸ-ಲಾ-ರಂ-ಭಿ-ಸು-ತ್ತೇನೆ
ಚಿಂತಿ-ಸಿ-ದ-ವರು
ಚಿಂತಿ-ಸುತ್ತ
ಚಿಂತಿ-ಸು-ತ್ತಿ-ದ್ದರು
ಚಿಂತಿ-ಸು-ತ್ತಿ-ದ್ದಾರೆ
ಚಿಂತಿ-ಸು-ತ್ತಿ-ರುವ
ಚಿಂತಿ-ಸು-ವಾಗ
ಚಿಂತೆ
ಚಿಂತೆ-ಗಳು
ಚಿಂತೆ-ಯಿಲ್ಲ
ಚಿಂತೆ-ಯೆಂ-ದರೆ
ಚಿಕಿ
ಚಿಕಿತ್ಸೆ
ಚಿಕಿ-ತ್ಸೆ-ಗಾಗಿ
ಚಿಕಿ-ತ್ಸೆಗೂ
ಚಿಕಿ-ತ್ಸೆ-ಯನ್ನು
ಚಿಕಿ-ತ್ಸೆ-ಯಿಂದ
ಚಿಕ್ಕ
ಚಿಕ್ಕ-ಚಿಕ್ಕ
ಚಿಕ್ಕಪ್ಪ
ಚಿಕ್ಕ-ಪ್ಪನ
ಚಿಕ್ಕ-ಪ್ಪ-ನನ್ನು
ಚಿಕ್ಕ-ಪ್ಪ-ನಾದ
ಚಿಕ್ಕಮ್ಮ
ಚಿಕ್ಕ-ಮ್ಮನ
ಚಿಕ್ಕ-ಮ್ಮ-ನಿ-ಗೊಂದು
ಚಿಕ್ಕ-ವನು
ಚಿಕ್ಕ-ವರು
ಚಿಟ್ಟೆ
ಚಿತೆಗೆ
ಚಿತೆಯ
ಚಿತ್ರ
ಚಿತ್ರ-ಗಳನ್ನು
ಚಿತ್ರಣ
ಚಿತ್ರ-ವನ್ನು
ಚಿತ್ರ-ವನ್ನೂ
ಚಿತ್ರ-ವಿ-ಚಿ-ತ್ರ-ವಾಗಿ
ಚಿತ್ರವೇ
ಚಿತ್ರಿ-ಸಿ-ದರೆ
ಚಿತ್ರಿ-ಸು-ತ್ತಿದ್ದ
ಚಿನ್ನದ
ಚಿನ್ನ-ದ್ದಾ-ದ-ರೇ-ನು-ಸ-ರ-ಪಳಿ
ಚಿನ್ಮ-ಯಿ-ಯಾಗಿ
ಚಿನ್ಮಯೀ
ಚಿಮ್ಮಲು
ಚಿಮ್ಮಿತು
ಚಿಮ್ಮಿ-ಸಿದ
ಚಿಮ್ಮು-ತ್ತಿತ್ತು
ಚಿಮ್ಮು-ತ್ತಿ-ದೆ-ಓ-ಜಸ್ಸು
ಚಿಮ್ಮು-ತ್ತಿ-ರು-ವುದೋ
ಚಿಮ್ಮು-ವಂತೆ
ಚಿರ
ಚಿರ-ನೂ-ತನ
ಚಿರ-ಪ-ರಿ-ಚಿತ
ಚಿರ-ಪ-ರಿ-ಚಿ-ತ-ನಾದ
ಚಿರ-ಪುತ್ರ
ಚಿರ-ಮುಕ್ತ
ಚಿರ-ಮು-ದ್ರಿ-ತ-ವಾಗಿ
ಚಿರ-ಮು-ದ್ರೆ-ಯ-ನ್ನೊ-ತ್ತಿ-ಬಿ-ಟ್ಟಿತು
ಚಿರ-ಶಾ-ಶ್ವತ
ಚಿರ-ಸ-ಮಾ-ಧಿ-ಯಲ್ಲಿ
ಚಿರ-ಸಾ-ಮ್ರಾ-ಜ್ಯದ
ಚಿರ-ಸ್ವಾ-ತಂತ್ರ್ಯ
ಚಿಲುಮೆ
ಚಿಹ್ನೆ-ಗಳು
ಚೀಟಿ-ಯಲ್ಲಿ
ಚೀನಾಕ್ಕೂ
ಚೀರಾ-ಟ-ವನ್ನು
ಚೀರು-ವಂ-ತಾ-ಗು-ತ್ತಿತ್ತು
ಚೀಲ
ಚೀಲ-ಗಳ
ಚೀಲ-ದಲ್ಲಿ
ಚುಚ್ಚು-ತ್ತಿತ್ತು
ಚುನಿ-ಬಾ-ಬು-ಇಷ್ಟು
ಚುರು-ಕಾಗಿ
ಚುರುಕು
ಚುರು-ಗು-ಟ್ಟಿದ್ದು
ಚುರು-ಗು-ಟ್ಟು-ತ್ತಿದೆ
ಚೂರು-ಗಳು
ಚೂರು-ಚೂ-ರಾ-ದುವು
ಚೆಂಡಿನ
ಚೆಂಡು
ಚೆದ-ರಿಸಿ
ಚೆನ್ನಾ
ಚೆನ್ನಾಗಿ
ಚೆನ್ನಾ-ಗಿತ್ತು
ಚೆನ್ನಾ-ಗಿದೆ
ಚೆನ್ನಾ-ಗಿ-ದೆಯೋ
ಚೆನ್ನಾ-ಗಿಯೇ
ಚೆನ್ನಾ-ಗಿರು
ಚೆನ್ನಾ-ಗಿ-ಲ್ಲ-ವೆಂಬ
ಚೆನ್ನಾ-ಗಿ-ಲ್ಲ-ವೆಂ-ಬುದು
ಚೆಲ್ಲು-ತ-ನ-ವಿ-ರ-ಲಿಲ್ಲ
ಚೇತ-ನ-ಗೊ-ಳಿ-ಸುವ
ಚೇತ-ನರು
ಚೇತ-ನ-ವನ್ನೇ
ಚೇತ-ರಿ-ಸಿ-ಕೊಂ-ಡರು
ಚೇತ-ರಿ-ಸಿ-ಕೊಂ-ಡ-ವ-ನಂತೆ
ಚೇತ-ರಿ-ಸಿ-ಕೊಂಡು
ಚೇತ-ರಿ-ಸಿ-ಕೊ-ಳ್ಳ-ಲಾ-ರಂ-ಭಿ-ಸಿ-ದರು
ಚೇತೋ-ಹಾ-ರಿ-ಯಾ-ಗಿತ್ತು
ಚೇಷ್ಟೆ
ಚೇಷ್ಟೆ-ಗಳನ್ನೆಲ್ಲ
ಚೇಷ್ಟೆಗೆ
ಚೇಷ್ಟೆ-ಯನ್ನು
ಚೈತ-ನ್ನ-ದೇ-ವ-ನ-ನ್ನು-ಅಷ್ಟೇ
ಚೈತನ್ಯ
ಚೈತ-ನ್ಯ-ಗಳ
ಚೈತ-ನ್ಯದ
ಚೈತ-ನ್ಯನ
ಚೈತ-ನ್ಯ-ಪೂ-ರ್ಣ-ವಾದ
ಚೈತ-ನ್ಯ-ಮ-ಯಿ-ಯಾಗಿ
ಚೈತ-ನ್ಯ-ವನ್ನು
ಚೈತ-ನ್ಯ-ವೀಗ
ಚೈತ-ನ್ಯವೇ
ಚೊಕ್ಕ
ಚೋಟುದ್ದ
ಚೌಕ-ಟ್ಟನ್ನು
ಚೌಕ-ಟ್ಟಿ-ನಲ್ಲಿ
ಚೌಕದ
ಚೌಧರಿ
ಛಂಗನೆ
ಛತ್ರ-ದಲ್ಲಿ
ಛತ್ರ-ದಲ್ಲೋ
ಛಲ
ಛಲ-ಇವು
ಛಾಯೆ
ಛಾಯೆಯೂ
ಛಾವಣಿ
ಛಾವ-ಣಿಯ
ಛಿ
ಛಿಲ್ಲೆಂದು
ಛೀ
ಛೀಛೀ-ಛೀಛೀ
ಛೀಮಾರಿ
ಛೀಮಾ-ರಿ-ಯಿಂದ
ಛೆ
ಛೇಡಿಸಿ
ಛೇಡಿ-ಸಿ-ದರು
ಛೇಡಿ-ಸು-ತ್ತಾರೆ
ಛೇಡಿ-ಸು-ತ್ತಿದ್ದ
ಛೇಡಿ-ಸು-ತ್ತಿ-ದ್ದರು
ಜಂಜ-ಡ-ಗಳ
ಜಂಬ
ಜಂಬ-ದಿಂದ
ಜಗ-ಜ್ಜ-ನ-ನಿಯ
ಜಗ-ತ್ತನ್ನು
ಜಗ-ತ್ತನ್ನೇ
ಜಗ-ತ್ತಾಗಿ
ಜಗತ್ತಿ
ಜಗ-ತ್ತಿಗೂ
ಜಗ-ತ್ತಿಗೆ
ಜಗ-ತ್ತಿ-ಗೆಲ್ಲ
ಜಗ-ತ್ತಿಗೇ
ಜಗ-ತ್ತಿನ
ಜಗ-ತ್ತಿ-ನ-ಲ್ಲಾ-ಗಲಿ
ಜಗ-ತ್ತಿ-ನಲ್ಲಿ
ಜಗ-ತ್ತಿ-ನ-ಲ್ಲಿ-ರು-ವು-ದೆಲ್ಲ
ಜಗ-ತ್ತಿ-ನ-ಲ್ಲೊಂದು
ಜಗ-ತ್ತಿ-ನಾ-ದ್ಯಂತ
ಜಗತ್ತು
ಜಗ-ತ್ತು-ಜ-ನ-ಗಳು
ಜಗ-ತ್ತು-ಪ-ರ-ಬ್ರ-ಹ್ಮ-ಗಳ
ಜಗ-ತ್ತು-ಗಳನ್ನು
ಜಗತ್ತೇ
ಜಗ-ತ್ಪ್ರ-ಸಿದ್ಧ
ಜಗ-ತ್ಪ್ರ-ಸಿ-ದ್ಧ-ರಾದ
ಜಗ-ತ್ಪ್ರ-ಸಿ-ದ್ಧ-ವಾದ
ಜಗ-ತ್ಪ್ರ-ಸಿ-ದ್ಧಿ-ಯನ್ನು
ಜಗದ
ಜಗ-ದಾ-ತ್ಮ-ಭಾ-ವ-ದಿಂ-ದಿರು
ಜಗ-ದೋ-ದ್ಧಾ-ರ-ಕನೂ
ಜಗ-ದ್ಭಾ-ವ-ವನ್ನು
ಜಗ-ದ್ವಿ-ಖ್ಯಾತ
ಜಗ-ನ್ನಾಥ
ಜಗ-ನ್ಮಾ-ತೃ-ತ್ವದ
ಜಗ-ನ್ಮಾತೆ
ಜಗ-ನ್ಮಾ-ತೆಗೆ
ಜಗ-ನ್ಮಾ-ತೆಯ
ಜಗ-ನ್ಮಾ-ತೆ-ಯನ್ನು
ಜಗ-ನ್ಮಾ-ತೆ-ಯನ್ನೇ
ಜಗ-ನ್ಮಾ-ತೆ-ಯಲ್ಲಿ
ಜಗ-ನ್ಮಾ-ತೆ-ಯಿಂದ
ಜಗ-ನ್ಮಾ-ತೆ-ಯೆಂದೇ
ಜಗ-ನ್ಮಾ-ತೆಯೇ
ಜಗ-ನ್ಮಾ-ತೆ-ಯೊಂ-ದಿಗೆ
ಜಗ-ಲಿಗೆ
ಜಗ-ಲಿಯ
ಜಗ-ಲಿ-ಯಿಂದ
ಜಗ-ಲಿ-ಯಿದ್ದು
ಜಗಳ
ಜಗ-ಳ-ವನ್ನು
ಜಗ-ಳ-ವಾ-ಗಿ-ದ್ದು-ದ-ರಿಂದ
ಜಗ-ಳ-ವಾ-ಡಿ-ಕೊ-ಳ್ಳ-ಬಾ-ರದು
ಜಗ-ಳ-ವಾ-ಡು-ತ್ತಿದ್ದ
ಜಗ-ಳ-ವೆದ್ದು
ಜಗ-ಳ-ವೇಕೆ
ಜಗ್ಗ-ಲಿಲ್ಲ
ಜಗ್ಗಾ-ಟ-ವಿ-ರ-ಲಿಲ್ಲ
ಜಗ್ಗಿ
ಜಗ್ಗು-ವ-ವ-ನಲ್ಲ
ಜಜ್ಜಿ
ಜಟ-ಕಾ-ದಲ್ಲಿ
ಜಟಾ-ಜೂ-ಟ-ಧಾ-ರಿ-ಗ-ಳಾಗಿ
ಜಟಿಲ
ಜಟಿ-ಲ-ತೆ
ಜಟೆ-ಗ-ಟ್ಟ-ಬೇ-ಕಾ-ದರೆ
ಜಟೆ-ಗ-ಟ್ಟಲೇ
ಜಟೆ-ಗಳು
ಜಟೆ-ಗ-ಳೆಲ್ಲ
ಜಟೆ-ಯಾ-ಗಲೂ
ಜಟೆ-ಯಾ-ಗಿ-ದೆಯೇ
ಜಟ್ಟಿಯ
ಜಡ
ಜಡ-ವಾ-ದಿ-ಗಳ
ಜನ
ಜನಕ
ಜನ-ಕ-ಆ-ಮ್ಲ-ಜ-ನಕ
ಜನ-ಕೋ-ಟಿಗೆ
ಜನ-ಕೋ-ಟಿಯ
ಜನ-ಕೋ-ಟಿ-ಯಲ್ಲಿ
ಜನ-ಗಳ
ಜನ-ಗಳು
ಜನ-ಗ-ಳೊಂ-ದಿಗೆ
ಜನ-ಜಂ-ಗುಳಿ
ಜನ-ಜಂ-ಗು-ಳಿಯ
ಜನ-ಜಂ-ಗು-ಳಿ-ಯಿದ್ದ
ಜನ-ಜ-ನಿ-ತ-ವಾ-ಗಿವೆ
ಜನ-ಜೀ-ವ-ನಕ್ಕೂ
ಜನ-ಜೀ-ವ-ನದ
ಜನ-ಜೀ-ವ-ನ-ವನ್ನು
ಜನ-ತೆಗೆ
ಜನ-ತೆಯ
ಜನ-ತೆ-ಯನ್ನು
ಜನದ
ಜನ-ದ-ಟ್ಟ-ಣೆ-ಯಿಂದ
ಜನನ
ಜನ-ನ-ವೃ-ದ್ಧಾ-ಪ್ಯ-ಮ-ರ-ಣ-ಗ-ಳುಂಟೆ
ಜನ-ನ-ದಿಂ-ದಾ-ರಂ-ಭಿಸಿ
ಜನ-ನ-ದೆ-ಡೆ-ಯಿಂ
ಜನ-ನ-ವಾ-ಗ-ಲಿದೆ
ಜನ-ಪದ
ಜನ-ಪ್ರಿ-ಯತೆ
ಜನ-ಪ್ರಿ-ಯ-ವಾ-ಯಿತು
ಜನ-ಮನ
ಜನ-ಮ-ನ-ದಲ್ಲಿ
ಜನರ
ಜನ-ರಂತೆ
ಜನ-ರನ್ನು
ಜನ-ರನ್ನೂ
ಜನ-ರಲ್
ಜನ-ರಲ್ಲಿ
ಜನ-ರ-ಲ್ಲೆಲ್ಲ
ಜನ-ರಿಂದ
ಜನ-ರಿ-ಗೀಗ
ಜನ-ರಿಗೆ
ಜನ-ರಿ-ಗೆ-ತೊಂ-ದ-ರೆ-ಯಾ-ಗು-ತ್ತಿ-ರು-ವಾಗ
ಜನ-ರಿ-ಗೆಲ್ಲ
ಜನ-ರಿ-ದ್ದಾರು
ಜನ-ರಿ-ರಲಿ
ಜನರು
ಜನ-ರು-ಇ-ವೆಲ್ಲ
ಜನರೂ
ಜನ-ರೆಂ-ದರೆ
ಜನ-ರೆಲ್ಲ
ಜನ-ರೆ-ಲ್ಲರೂ
ಜನರೇ
ಜನ-ರೊಂ-ದಿಗೆ
ಜನ-ವರಿ
ಜನ-ವ-ರಿಯ
ಜನ-ವ-ರ್ಗಕ್ಕೆ
ಜನ-ವ-ರ್ಗ-ಗಳು
ಜನ-ವ-ರ್ಗವೂ
ಜನ-ಸಂ-ದ-ಣಿ-ಯ-ಲ್ಲೆಲ್ಲೋ
ಜನ-ಸಂ-ಪ-ರ್ಕದ
ಜನ-ಸ-ಮೂ-ಹಕೆ
ಜನ-ಸಾ-ಮಾನ್ಯ
ಜನ-ಸಾ-ಮಾ-ನ್ಯರು
ಜನ-ಸೇವೆ
ಜನ-ಸೇ-ವೆಯ
ಜನ-ಸೇ-ವೆ-ಯನ್ನು
ಜನಾಂಗ
ಜನಾಂ-ಗದ
ಜನಾಂ-ಗವು
ಜನಾಂ-ಗವೇ
ಜನಾ-ರ್ದ-ನ-ನನ್ನು
ಜನಿ-ಸಿ-ದ-ವನು
ಜನಿ-ಸಿ-ದ್ದಾನೋ
ಜನಿ-ಸು-ತ್ತೇನೆ
ಜನಿ-ಸುವ
ಜನು-ಮ-ಜ-ನು-ಮಂ-ಗ-ಳಲು
ಜನೆಗೆ
ಜನ್ಮ
ಜನ್ಮ
ಜನ್ಮ-ಗಳ
ಜನ್ಮ-ಗಳೇ
ಜನ್ಮ-ಜ-ನ್ಮದಿ
ಜನ್ಮತಃ
ಜನ್ಮ-ತ-ಳೆದು
ಜನ್ಮ-ತಾ-ಳಿ-ದರೋ
ಜನ್ಮ-ತಾ-ಳು-ವು-ದಂತೂ
ಜನ್ಮ-ದಲ್ಲಿ
ಜನ್ಮ-ದಲ್ಲೇ
ಜನ್ಮ-ದಾ-ತೆ-ಅ-ವಳೇ
ಜನ್ಮ-ದಾ-ರ-ಭ್ಯ-ದಿಂ-ದಲೇ
ಜನ್ಮ-ದಿ-ನ-ವನ್ನು
ಜನ್ಮ-ಭೂ-ಮಿಯ
ಜನ್ಮ-ವೆತ್ತಿ
ಜನ್ಮ-ವೆ-ತ್ತಿದ
ಜನ್ಮ-ವೆ-ತ್ತಿ-ದವ
ಜನ್ಮ-ವೆ-ತ್ತಿ-ದ-ವ-ನಲ್ಲ
ಜನ್ಮ-ಸ್ಥಳ
ಜನ್ಮ-ಸ್ಥ-ಳದ
ಜನ್ಮಾಂ-ತ-ರ-ಗಳ
ಜನ್ಮೋ-ತ್ಸ-ವದ
ಜನ್ಮೋ-ದ್ದೇಶ
ಜಪ
ಜಪ-ತ-ಪ-ಧ್ಯಾ-ನ-ಗಳನ್ನು
ಜಪ-ತ-ಪ-ಗಳ
ಜಪ-ಧ್ಯಾನ
ಜಪ-ತಪ
ಜಪ-ತ-ಪಾ-ದಿ-ಗಳನ್ನು
ಜಪ-ಧ್ಯಾನ
ಜಪ-ಧ್ಯಾ-ನ-ಗಳನ್ನು
ಜಪ-ಮಾ-ಡು-ತ್ತಲೇ
ಜಪ-ಮಾಲೆ
ಜಪಿ-ಸು-ತ್ತಿ-ರ-ಬೇಕು
ಜಯ
ಜಯ-ಕಾರ
ಜಯ-ಭೇರಿ
ಜಯ-ರಾಂ-ಬಾಟಿ
ಜಯ-ರಾಂ-ಬಾ-ಟಿ-ಯಲ್ಲಿ
ಜಯ-ವಾ-ಗಲಿ
ಜಯ-ಶಾಲಿ
ಜಯ-ಶಾ-ಲಿ-ಗ-ಳಾ-ಗ-ಲಾ-ರರು
ಜಯ-ಶಾ-ಲಿ-ಗ-ಳಾ-ಗಿ-ದ್ದೇ-ವೇನು
ಜಯ-ಶಾ-ಲಿ-ಯಾ-ಗು-ತ್ತಿದ್ದ
ಜಯ-ಶಾ-ಲಿ-ಯಾ-ಯಿತು
ಜಯಿ-ಸಲು
ಜಯಿ-ಸ-ಲೇ-ಬೇ-ಕಾ-ಗು-ತ್ತದೆ
ಜಯಿ-ಸಿದ
ಜರಾ-ರೋ-ಗ-ಗಳು
ಜರು-ಗ-ಲಿಲ್ಲ
ಜರು-ಗಿ-ದಾಗ
ಜರ್ಝ-ರಿ-ತ-ಗೊ-ಳಿ-ಸಿ-ಬಿ-ಟ್ಟಿವೆ
ಜರ್ಝ-ರಿ-ತ-ವಾ-ಗಿತ್ತು
ಜರ್ಝ-ರಿ-ತ-ವಾ-ಗಿ-ಬಿ-ಟ್ಟಿದೆ
ಜಲ
ಜಲ-ಪಾ-ತ-ಗಳು
ಜಲ-ಮಾ-ರ್ಗ-ವಾ-ಗಿಯೇ
ಜವಾ
ಜವಾ-ನರು
ಜವಾ-ಬ್ದಾರಿ
ಜವಾ-ಬ್ದಾ-ರಿಯ
ಜವಾ-ಬ್ದಾ-ರಿ-ಯನ್ನು
ಜವಾ-ಬ್ದಾ-ರಿ-ಯನ್ನೂ
ಜಾಗ
ಜಾಗ-ಕ್ಕಾಗಿ
ಜಾಗಕ್ಕೆ
ಜಾಗ-ತಿಕ
ಜಾಗ-ದಲ್ಲಿ
ಜಾಗ-ದಿಂದ
ಜಾಗ-ರೂ-ಕ-ತೆ-ಯಿಂದ
ಜಾಗ-ವನ್ನು
ಜಾಗೃತ
ಜಾಗೃ-ತ-ಗೊಂಡು
ಜಾಗೃ-ತ-ಗೊ-ಳಿ-ಸಲು
ಜಾಗೃ-ತ-ಗೊ-ಳಿಸಿ
ಜಾಗೃ-ತ-ಗೊ-ಳಿ-ಸಿ-ಬಿ-ಟ್ಟಿತ್ತು
ಜಾಗೃ-ತ-ಗೊ-ಳಿ-ಸು-ವು-ದರ
ಜಾಗೃ-ತ-ನಾಗು
ಜಾಗೃ-ತ-ವಾಗಿ
ಜಾಗೃ-ತ-ವಾ-ಗಿ-ತ್ತೆ-ನ್ನು-ವುದು
ಜಾಗೃ-ತ-ವಾ-ಗಿ-ದೆಯೋ
ಜಾಗೃ-ತ-ವಾ-ಗಿ-ದ್ದಿ-ರ-ಬೇಕು
ಜಾಗೃ-ತ-ವಾ-ಗಿ-ಬಿ-ಟ್ಟರೆ
ಜಾಗೃ-ತ-ವಾ-ಗಿ-ಬಿ-ಟ್ಟಿತು
ಜಾಗೃ-ತ-ವಾ-ಗಿ-ಬಿ-ಟ್ಟಿತ್ತು
ಜಾಗೃ-ತ-ವಾ-ಗಿ-ಬಿ-ಟ್ಟಿದೆ
ಜಾಗೃ-ತ-ವಾ-ಗಿ-ಬಿ-ಟ್ಟುವು
ಜಾಗೃ-ತ-ವಾ-ಗಿ-ರು-ವಂತೆ
ಜಾಗೃ-ತ-ವಾ-ಗು-ತ್ತಿತ್ತು
ಜಾಗೃ-ತ-ವಾ-ಗು-ವಂತೆ
ಜಾಗೃ-ತ-ವಾದ
ಜಾಗೃ-ತ-ವಾ-ದಂ-ತೆಲ್ಲ
ಜಾಗೃ-ತ-ವಾ-ಯಿ-ತಲ್ಲ
ಜಾಗೃ-ತ-ವಾ-ಯಿತು
ಜಾಗೃ-ತ-ಸ್ಥಿ-ತಿ-ಯಲ್ಲೂ
ಜಾಗೃತಿ
ಜಾಗೃ-ತಿ-ಯ-ನ್ನುಂ-ಟು-ಮಾಡಿ
ಜಾಗೃ-ತಿ-ಯುಂ-ಟಾ-ಗಲಿ
ಜಾಗೃ-ದ-ವ-ಸ್ಥೆ-ಯಲ್ಲೇ
ಜಾಜ್ವ-ಲ್ಯ-ಮಾ-ನ-ವಾಗಿ
ಜಾಡ-ಮಾಲಿ
ಜಾಣ-ತನ
ಜಾಣರೇ
ಜಾಣ್ಮೆ-ಯಿಂದ
ಜಾತಕ
ಜಾತ-ಕ-ವನ್ನು
ಜಾತಾನಿ
ಜಾತಿ
ಜಾತಿ-ಕು-ಲ-ಗೋತ್ರ
ಜಾತಿ-ಕು-ಲ-ಗೋ-ತ್ರ-ಗಳ
ಜಾತಿ-ಕು-ಲ-ಗಳ
ಜಾತಿ-ಮ-ತ-ಧರ್ಮ
ಜಾತಿಗೆ
ಜಾತಿ-ನಿ-ಯ-ಮ-ವನ್ನು
ಜಾತಿ-ಪ-ದ್ಧತಿ
ಜಾತಿ-ಪ-ದ್ಧ-ತಿಯ
ಜಾತಿ-ಬುದ್ಧಿ
ಜಾತಿ-ಬ್ರಾ-ಹ್ಮ-ಣರ
ಜಾತಿ-ಭೇ-ದ-ದಿಂದ
ಜಾತಿ-ಭೇ-ದ-ವನ್ನು
ಜಾತಿಯ
ಜಾತಿ-ಯನ್ನು
ಜಾತಿ-ಯ-ವನು
ಜಾತಿ-ಯ-ವರ
ಜಾತಿ-ಯ-ವ-ರಲ್ಲಿ
ಜಾತಿ-ಯ-ವ-ರಾ-ದರೂ
ಜಾತಿ-ಯ-ವ-ರಿಗೆ
ಜಾತಿ-ಯ-ವರು
ಜಾತಿ-ಯ-ವ-ರೆಂದು
ಜಾತಿ-ಯಿಂದ
ಜಾತಿ-ಯೆ-ನ್ನು-ವು-ದೊಂದು
ಜಾತ್ರೆ
ಜಾತ್ರೆಗೆ
ಜಾತ್ರೆ-ಯಲ್ಲಿ
ಜಾದು-ಗಾರ
ಜಾದೂ
ಜಾನ-ಕೀ-ವರ
ಜಾನ್
ಜಾಯಂತೇ
ಜಾಯ-ಮಾ-ನ-ದಲ್ಲೇ
ಜಾರಿ
ಜಾರಿ-ದರೂ
ಜಾರಿ-ಬೀ-ಳು-ವು-ದ-ಲ್ಲದೆ
ಜಾರಿ-ಸಲು
ಜಾರು-ತ್ತಿತ್ತು
ಜಾರು-ಬಂ-ಡೆ-ಯಾಗಿ
ಜಾರೋ-ಣವೆ
ಜಾಲ-ದಲ್ಲಿ
ಜಾಲಾಡಿ
ಜಾವ
ಜಾವದ
ಜಿ
ಜಿಂಕೆಯ
ಜಿಂಕೆ-ಯಾ-ಗಿಯೇ
ಜಿಗಿದ
ಜಿಗಿ-ಯ-ಬ-ಹುದು
ಜಿಗಿ-ಯು-ತ್ತಾನೆ
ಜಿಜ್ಞಾಸು
ಜಿಜ್ಞಾ-ಸು-ಮು-ಮು-ಕ್ಷು-ಗ-ಳಿಗೆ
ಜಿಜ್ಞಾ-ಸು-ಗಳನ್ನು
ಜಿಜ್ಞಾಸೆ
ಜಿಜ್ಞಾ-ಸೆ-ಯೆ-ದ್ದಿ-ದೆ-ಶ್ರೀ-ರಾ-ಮ-ಕೃ-ಷ್ಣ-ರಿಗೆ
ಜಿನು-ಗುವ
ಜಿಲ್ಲಾ
ಜೀರ್ಣ-ಶ-ಕ್ತಿಯೂ
ಜೀವ
ಜೀವಂತ
ಜೀವಂ-ತ-ಳಾಗಿ
ಜೀವಂ-ತ-ವಾ-ಗಿ-ದ್ದು-ಕೊಂಡು
ಜೀವಂ-ತ-ವಾ-ಗಿ-ಸಿ-ರುವ
ಜೀವಂತಿ
ಜೀವ-ಕಳೆ
ಜೀವ-ಕ-ಳೆಯೇ
ಜೀವ-ಕೋ-ಟಿ-ಗ-ಳಾಗಿ
ಜೀವ-ಕೋ-ಟಿ-ಗಳು
ಜೀವ-ಕೋ-ಟಿ-ಗ-ಳೆ-ಲ್ಲರೂ
ಜೀವಕ್ಕೆ
ಜೀವ-ಜಂ-ತು-ಗ-ಳಲ್ಲೂ
ಜೀವ-ಜಂ-ತು-ಗ-ಳಾ-ಳಿಗೆ
ಜೀವದ
ಜೀವ-ದಾನ
ಜೀವನ
ಜೀವ-ನ
ಜೀವ-ನ-ಬೋ-ಧ-ನೆ-ಗಳನ್ನು
ಜೀವ-ನ-ಬೋ-ಧ-ನೆ-ಗಳು
ಜೀವ-ನ-ವ್ಯ-ಕ್ತಿ-ತ್ವ-ಬೋ-ಧ-ನೆ-ಗಳ
ಜೀವ-ನ-ಸಂ-ದೇ-ಶ-ಗಳ
ಜೀವ-ನ-ಸಂ-ದೇ-ಶ-ಗಳನ್ನು
ಜೀವ-ನ-ಸಂ-ದೇ-ಶ-ಗಳಲ್ಲಿ
ಜೀವ-ನ-ಸಂ-ದೇ-ಶ-ಗ-ಳೆ-ರಡೂ
ಜೀವ-ನ-ಕ್ಕಿ-ಳಿದು
ಜೀವ-ನಕ್ಕೆ
ಜೀವ-ನ-ಕ್ರಮ
ಜೀವ-ನ-ಕ್ರ-ಮ-ಗಳ
ಜೀವ-ನ-ಕ್ರ-ಮದ
ಜೀವ-ನ-ಕ್ರ-ಮ-ವನ್ನು
ಜೀವ-ನ-ಗ-ತಿ-ಯನ್ನು
ಜೀವ-ನ-ಗಳ
ಜೀವ-ನ-ಗಳನ್ನು
ಜೀವ-ನ-ಚ-ಕ್ರ-ದಲ್ಲಿ
ಜೀವ-ನ-ಚ-ರಿ-ತ್ರೆ-ಗಳನ್ನು
ಜೀವ-ನ-ಚ-ರಿ-ತ್ರೆಗೆ
ಜೀವ-ನ-ಚ-ರಿ-ತ್ರೆಯ
ಜೀವ-ನ-ಚ-ರಿ-ತ್ರೆ-ಯನ್ನು
ಜೀವ-ನ-ತ-ತ್ತ್ವಕ್ಕೆ
ಜೀವ-ನದ
ಜೀವ-ನ-ದಂ-ತಲ್ಲ
ಜೀವ-ನ-ದತ್ತ
ಜೀವ-ನ-ದಲ್ಲಿ
ಜೀವ-ನ-ದಲ್ಲೇ
ಜೀವ-ನ-ದಷ್ಟು
ಜೀವ-ನ-ದಿಂದ
ಜೀವ-ನ-ದು-ದ್ದಕ್ಕೂ
ಜೀವ-ನ-ರಂ-ಗಕ್ಕೆ
ಜೀವ-ನಲ್ಲ
ಜೀವ-ನಲ್ಲಿ
ಜೀವ-ನ-ವನ್ನು
ಜೀವ-ನ-ವ-ನ್ನು-ಕೈ-ಗೊ-ಳ್ಳು-ವು-ದರ
ಜೀವ-ನ-ವನ್ನೂ
ಜೀವ-ನ-ವ-ನ್ನೆಲ್ಲ
ಜೀವ-ನ-ವನ್ನೇ
ಜೀವ-ನ-ವಿಡೀ
ಜೀವ-ನವು
ಜೀವ-ನವೂ
ಜೀವ-ನ-ವೆಂದರೆ
ಜೀವ-ನ-ವೆಂ-ಬು-ವುದು
ಜೀವ-ನ-ವೆ-ನ್ನು-ವುದು
ಜೀವ-ನವೇ
ಜೀವ-ನ-ವೇನೂ
ಜೀವ-ನ-ವೊಂ-ದರ
ಜೀವ-ನ-ಸಾ-ರ್ಥ-ಕ್ಯ-ವನ್ನು
ಜೀವ-ನಾ-ದ-ರ್ಶ-ವನ್ನು
ಜೀವ-ನಾ-ದ-ರ್ಶ-ವನ್ನೂ
ಜೀವ-ನಾ-ಧಾರ
ಜೀವ-ನಾ-ಧಾ-ರಕ್ಕೆ
ಜೀವ-ನಾ-ನು-ಭ-ವ-ಗಳು
ಜೀವ-ನಿಗೂ
ಜೀವ-ನೋ-ದ್ದೇ-ಶ-ಗಳನ್ನು
ಜೀವ-ನೋ-ಪಾ-ಯ-ಕ್ಕಾಗಿ
ಜೀವ-ನ್ಮುಕ್ತ
ಜೀವ-ಮಾನ
ಜೀವ-ಮಾ-ನ-ದಲ್ಲೇ
ಜೀವ-ಮಾ-ನ-ಪ-ರ್ಯಂತ
ಜೀವರ
ಜೀವ-ರಂತೆ
ಜೀವ-ರಲ್ಲಿ
ಜೀವ-ರಲ್ಲೂ
ಜೀವ-ರಾ-ದರೆ
ಜೀವ-ರಿಗೆ
ಜೀವರೆ-ದೆ-ಗಳನ್ನು
ಜೀವರೆ-ದೆಗೆ
ಜೀವ-ವನ್ನೇ
ಜೀವವು
ಜೀವ-ಸ-ತ್ವ-ದಂತೆ
ಜೀವಾ-ತ್ಮನು
ಜೀವಾ-ತ್ಮನೂ
ಜೀವಿ
ಜೀವಿ-ಗಳನ್ನು
ಜೀವಿ-ಗ-ಳಲ್ಲೂ
ಜೀವಿ-ಗಳೂ
ಜೀವಿ-ಗ-ಳೆಲ್ಲ
ಜೀವಿಗೂ
ಜೀವಿ-ತಾ-ವ-ಧಿಯ
ಜೀವಿ-ತಾ-ವ-ಧಿ-ಯಲ್ಲಿ
ಜೀವಿ-ತಾ-ವ-ಧಿ-ಯಿಡೀ
ಜೀವಿ-ಯೊ-ಬ್ಬನು
ಜೀವಿ-ಸದೆ
ಜೀವಿ-ಸ-ಲಾ-ರಂ-ಭಿ-ಸಿ-ದರು
ಜೀವಿ-ಸಿ-ದ-ವ-ರಲ್ಲ
ಜೀವಿ-ಸಿ-ದ-ವರು
ಜೀವಿ-ಸಿ-ದೆವು
ಜೀವಿಸು
ಜೀವಿ-ಸುತ್ತ
ಜೀವಿ-ಸುವ
ಜೀವಿ-ಸು-ವುದು
ಜುಗುಪ್ಸೆ
ಜುಗು-ಪ್ಸೆ-ಗೊಂಡ
ಜುಗು-ಪ್ಸೆ-ಯಿಂ-ದಲೇ
ಜುಲೈ
ಜುಲ್ಫಿ-ಕರ್
ಜೂಜು
ಜೇಡಿ-ಮ-ಣ್ಣಿ-ನಿಂದ
ಜೇನು
ಜೇನು-ಕಂ-ಠ-ದಿಂದ
ಜೇನು-ಗೂ-ಡಿ-ನಂ-ತಾ-ಗಿತ್ತು
ಜೇನು-ಗೂಡು
ಜೇನು-ತುಪ್ಪ
ಜೇನು-ತು-ಪ್ಪ-ದಂತೆ
ಜೇನ್ನೊಣ
ಜೇನ್ನೊ-ಣ-ಗಳ
ಜೈ
ಜೈತನ್ಯ
ಜೈತ್ರ-ಯಾ-ತ್ರೆ-ಯನ್ನೂ
ಜೈನ
ಜೈನ-ಧ-ರ್ಮದ
ಜೊತೆ
ಜೊತೆ-ಗಾರ
ಜೊತೆ-ಗಾ-ರರೇ
ಜೊತೆ-ಗಿ-ದ್ದಾಗ
ಜೊತೆ-ಗಿ-ರು-ತ್ತೇನೆ
ಜೊತೆ-ಗೂಡಿ
ಜೊತೆ-ಗೂ-ಡಿ-ಕೊಂಡು
ಜೊತೆಗೆ
ಜೊತೆ-ಜೊ-ತೆಗೇ
ಜೊತೆ-ಯನ್ನು
ಜೊತೆ-ಯಲ್ಲಿ
ಜೊತೆ-ಯ-ಲ್ಲಿದ್ದ
ಜೊತೆ-ಯ-ಲ್ಲಿ-ದ್ದು-ಕೊಂಡು
ಜೊತೆ-ಯ-ಲ್ಲಿ-ದ್ದು-ಬಿ-ಟ್ಟರೆ
ಜೊತೆ-ಯ-ಲ್ಲಿ-ರು-ತ್ತಿ-ದ್ದರು
ಜೊತೆ-ಯ-ಲ್ಲಿ-ರು-ವಾಗ
ಜೊತೆ-ಯಲ್ಲೂ
ಜೊತೆ-ಯಲ್ಲೇ
ಜೊತೆ-ಯಾಗಿ
ಜೋ
ಜೋಡಿಸಿ
ಜೋಡಿ-ಸಿ-ಟ್ಟು-ಕೊ-ಳ್ಳು-ತ್ತಿದ್ದೆ
ಜೋತಾ-ಡ-ಬ-ಹುದು
ಜೋತಾ-ಡಿ-ಸಿ-ಕೊಂಡು
ಜೋತಾಡು
ಜೋತಾ-ಡು-ತ್ತಿದ್ದ
ಜೋತಾ-ಡು-ತ್ತಿ-ದ್ದಾನೆ
ಜೋತಾ-ಡು-ವು-ದು-ಮುಂ-ತಾದ
ಜೋಮು
ಜೋರಾಗಿ
ಜೋಲು
ಜೋಲು-ಮೋರೆ
ಜೋಳಿ-ಗೆ-ಯಿಂದ
ಜೋಶಿ
ಜೋಷಿ
ಜ್ಞಾತಾ
ಜ್ಞಾನ
ಜ್ಞಾನ
ಜ್ಞಾನ-ಭ-ಕ್ತಿ-ವೈ-ರಾ-ಗ್ಯ-ಗಳನ್ನು
ಜ್ಞಾನ-ಅ-ನು-ಭವ
ಜ್ಞಾನ-ಭ-ಕ್ತಿ-ವಿ-ವೇ-ಕ
ಜ್ಞಾನ-ಭ-ಕ್ತಿ-ಗಳನ್ನು
ಜ್ಞಾನ-ಭ-ಕ್ತಿ-ಗ-ಳಿ-ಗಾಗಿ
ಜ್ಞಾನ-ಭ-ಕ್ತಿ-ಗ-ಳೆ-ರಡೂ
ಜ್ಞಾನ-ಕೋ-ಶ-ವನ್ನು
ಜ್ಞಾನ-ಕ್ಕಿಂತ
ಜ್ಞಾನ-ಕ್ಷೇ-ತ್ರ-ಗಳನ್ನು
ಜ್ಞಾನ-ಖ-ಡ್ಗ-ದಿಂದ
ಜ್ಞಾನದ
ಜ್ಞಾನ-ದ-ಲ್ಲಾ-ಗಲಿ
ಜ್ಞಾನ-ದಾ-ಹವೂ
ಜ್ಞಾನ-ಪೀ-ಠದ
ಜ್ಞಾನ-ಭಂ-ಡಾ-ರ-ವೊಂ-ದರ
ಜ್ಞಾನ-ಮಾ-ರ್ಗವೇ
ಜ್ಞಾನ-ಮಾರ್ಗಿ
ಜ್ಞಾನ-ಯೋಗ
ಜ್ಞಾನ-ಯೋ-ಗಿ-ಗ-ಳಾ-ದರೂ
ಜ್ಞಾನ-ವನ್ನು
ಜ್ಞಾನ-ವ-ನ್ನು-ಭ-ಕ್ತಿ-ಯನ್ನು
ಜ್ಞಾನ-ವನ್ನೂ
ಜ್ಞಾನ-ವಾ-ಗಲಿ
ಜ್ಞಾನ-ವಾ-ದರೆ
ಜ್ಞಾನ-ವಿತ್ತು
ಜ್ಞಾನ-ವಿ-ದ್ದ-ವರು
ಜ್ಞಾನವು
ಜ್ಞಾನವೂ
ಜ್ಞಾನ-ವೆಂ-ಬುದು
ಜ್ಞಾನವೇ
ಜ್ಞಾನ-ಶಕ್ತಿ
ಜ್ಞಾನ-ಸಂ-ಪಾ-ದನೆ
ಜ್ಞಾನ-ಸಂ-ಪಾ-ದ-ನೆ-ಯಲ್ಲೂ
ಜ್ಞಾನ-ಸೂ-ರ್ಯನೇ
ಜ್ಞಾನ-ಸ್ವ-ರೂ-ಪಿಣಿ
ಜ್ಞಾನಾ-ಗ್ನಿ-ಯೆಂ-ಬುದು
ಜ್ಞಾನಾರ್
ಜ್ಞಾನಾ-ರ್ಜನೆ
ಜ್ಞಾನಿ
ಜ್ಞಾನಿ-ಗಳ
ಜ್ಞಾನಿ-ಯಾಗಿ
ಜ್ಞಾನೋ-ದ-ಯ-ವಾದ
ಜ್ಞಾಪಿ-ಸಿ-ಕೊಳ್ಳಿ
ಜ್ಯೇಷ್ಠ
ಜ್ಯೋತಿ
ಜ್ಯೋತಿಃ-ಪುಂ-ಜ-ವಾಗಿ
ಜ್ಯೋತಿ-ದ-ರ್ಶನ
ಜ್ಯೋತಿಯ
ಜ್ಯೋತಿ-ಯಂ-ತಿದೆ
ಜ್ಯೋತಿ-ಯಂತೆ
ಜ್ಯೋತಿ-ಯೊಂ-ದನ್ನು
ಜ್ಯೋತಿ-ಯೊಂ-ಬುದು
ಜ್ಯೋತಿ-ರ್ಮಯ
ಜ್ಯೋತಿ-ರ್ಮ-ಯ-ನಾದ
ಜ್ಯೋತಿಷ
ಜ್ಯೋತಿ-ಷ-ವನ್ನೂ
ಜ್ವರ
ಜ್ವರ-ಕ್ಕೀ-ಡಾಗಿ
ಜ್ವರ-ದಿಂದ
ಜ್ವಲಂತ
ಜ್ವಲಂ-ತ-ಗೊ-ಳಿ-ಸಿದ
ಜ್ವಲಂ-ತ-ವಾಗಿ
ಜ್ವಲಂ-ತ-ವಾ-ಗಿ-ರಿ-ಸಲು
ಜ್ವಲಿ-ಸ-ಬೇ-ಕಾದ
ಜ್ವಲಿ-ಸುವ
ಜ್ವಾಲಾ-ಮುಖಿ
ಜ್ವಾಲೆ
ಜ್ವಾಲೆಗೆ
ಜ್ವಾಲೆ-ಯನ್ನೂ
ಝಟಾ-ಪಟಿ
ಝಣ-ತ್ಕಾರ
ಝರಿ
ಝರಿ-ತೊರೆ
ಟನ್
ಟರು
ಟಾಗೋ-ರರು
ಟಾಗೋರ್
ಟಿಕೆ-ಟ್ಟನ್ನು
ಟಿಪ್ಪಣಿ
ಟಿಪ್ಪ-ಣಿ-ಗ-ಳ-ನ್ನಾ-ಗಲಿ
ಟಿಬೆ-ಟನ್ನು
ಟಿಬೆ-ಟಿಗೆ
ಟಿಬೆ-ಟಿನ
ಟಿಬೆ-ಟಿ-ನ-ಲ್ಲಿದ್ದ
ಟಿಬೆ-ಟ್ಟಿಗೂ
ಟಿಬೆ-ಟ್ಟಿಗೆ
ಟೀಕಿಸಿ
ಟೀಕಿ-ಸಿ-ದರು
ಟೀಕಿ-ಸಿ-ದರೆ
ಟೀಕಿ-ಸಿ-ದ್ದುಂಟು
ಟೀಕಿ-ಸುತ್ತ
ಟೀಕಿ-ಸು-ತ್ತಾನೆ
ಟೀಕಿ-ಸು-ತ್ತಿದ್ದ
ಟೀಕಿ-ಸು-ತ್ತಿ-ದ್ದರೆ
ಟೀಕೆ
ಟೀಕೆ-ಗಳನ್ನು
ಟೀಕೆ-ಮಾ-ಡಿ-ದ-ರು-ಬ್ರಾ-ಹ್ಮ-ಣ-ರ-ಲ್ಲ-ದ-ವರು
ಟೀಕೆಯ
ಟೆಹರಿ
ಟೆಹ-ರಿಗೆ
ಟೆಹ-ರಿಯ
ಟೆಹ-ರಿ-ಯಲ್ಲಿ
ಟೆಹ-ರಿ-ಯಲ್ಲೇ
ಟೆಹ-ರಿ-ಯಿಂದ
ಟೊಂಕ
ಟೊಳ್ಳು
ಟ್ಟವರು
ಟ್ಟುಕೊ
ಟ್ರಸ್ಟ್
ಟ್ರೈನು
ಠಣಲ್
ಠಣಾರ್
ಠಣ್
ಠಾಕೂ-ರ-ರಿಗೆ
ಠಾಕೂ-ರರು
ಠೀವಿ
ಠೀವಿ-ಯಿಂದ
ಡನೆ
ಡಾ
ಡಾಕ್ಟರು
ಡಾಕ್ಟರ್
ಡಾಮ-ಹೇಂ-ದ್ರ-ಲಾಲ
ಡಾರ್ವಿನ್
ಡಿಂಡಿ-ಮ-ವನ್ನು
ಡಿಸೆಂ-ಬ-ರಿ-ನ-ಲ್ಲೊಂದು
ಡಿಸೆಂ-ಬರ್
ಡುವ
ಡೆಕಾರ್ಟೆ
ಡೆಹ-ರಾಡೂ
ಡೆಹ-ರಾ-ಡೂ-ನಿಗೆ
ಡೆಹ-ರಾ-ಡೂ-ನಿ-ನಲ್ಲಿ
ಡೋಲಾ-ಯ-ಮಾ-ನ-ವಾ-ಗಿತ್ತು
ಡ್ತಿರುತ್ತೆ
ಣಾಮ-ವನ್ನು
ಣಾಮ-ವಾಗಿ
ಣಿಕೆಯ
ತಂಟೆ
ತಂಟೆ
ತಂಟೆ-ಮಾರಿ
ತಂಟೆಯ
ತಂಡ
ತಂಡದ
ತಂತಮ್ಮ
ತಂತಿ
ತಂತಿಯ
ತಂತಿ-ಯನ್ನು
ತಂತಿ-ವಾ-ದ್ಯ-ಗಳನ್ನೂ
ತಂತ್ರ
ತಂತ್ರ-ಗ-ಳಲ್ಲೂ
ತಂತ್ರ-ಶಾ-ಸ್ತ್ರ-ಗಳಲ್ಲಿ
ತಂತ್ರ-ಶಾ-ಸ್ತ್ರದ
ತಂತ್ರ-ಸಾ-ಧ-ಕ-ರಿಗೆ
ತಂತ್ರ-ಸಾ-ಧನಾ
ತಂತ್ರ-ಸಾ-ಧನೆ
ತಂತ್ರ-ಸಾ-ಧ-ನೆಯ
ತಂತ್ರ-ಸಾ-ಧ-ನೆ-ಯನ್ನು
ತಂತ್ರ-ಸಾ-ಧ-ನೆ-ಯೆಂ-ದರೆ
ತಂದ
ತಂದದ್ದೂ
ತಂದರು
ತಂದರೆ
ತಂದ-ವನೇ
ತಂದ-ವರು
ತಂದಿಟ್ಟು
ತಂದಿತು
ತಂದಿದ್ದ
ತಂದಿ-ದ್ದರೆ
ತಂದಿ-ದ್ದಾರೆ
ತಂದಿ-ದ್ದೇನೆ
ತಂದಿವೆ
ತಂದು
ತಂದು-ಕೊಂ-ಡಳು
ತಂದು-ಕೊಂಡಿ
ತಂದು-ಕೊಂ-ಡಿ-ದ್ದಾರೆ
ತಂದು-ಕೊಂಡು
ತಂದು-ಕೊಂಡೆ
ತಂದು-ಕೊಟ್ಟ
ತಂದು-ಕೊ-ಟ್ಟ-ಮೇಲೆ
ತಂದು-ಕೊ-ಟ್ಟರೆ
ತಂದು-ಕೊ-ಡ-ದಿ-ದ್ದರೆ
ತಂದು-ಕೊ-ಡುವ
ತಂದು-ಕೊ-ಳ್ಳಲು
ತಂದು-ಕೊ-ಳ್ಳು-ತ್ತೇವೆ
ತಂದು-ಬಿ-ಟ್ಟಿದ್ದ
ತಂದು-ಬಿ-ಡು-ತ್ತಾನೆ
ತಂದೆ
ತಂದೆ-ತಾಯಿ
ತಂದೆಗೂ
ತಂದೆ-ತಾ-ಯಂ-ದಿರ
ತಂದೆ-ತಾ-ಯಂ-ದಿರು
ತಂದೆ-ತಾ-ಯಿ-ಗಳು
ತಂದೆ-ತಾ-ಯಿ-ಯ-ರಿಂ-ದಲೂ
ತಂದೆಯ
ತಂದೆ-ಯಂತೆ
ತಂದೆ-ಯನ್ನು
ತಂದೆ-ಯಾದ
ತಂದೆ-ಯಾ-ದ-ವನು
ತಂದೆ-ಯಿಂದ
ತಂದೆ-ಯೆ-ನ್ನ-ಬ-ಹುದು
ತಂದೊ-ಡ್ಡು-ತ್ತಾರೆ
ತಂಪು-ಗೊ-ಳಿ-ಸ-ಬೇ-ಕಾ-ದ್ದಿದೆ
ತಂಬಿ-ಗೆ-ಬಿಂ-ದಿ-ಗೆ-ಗ-ಳಾ-ದರೆ
ತಂಬೂರಿ
ತಂಬೂ-ರಿ-ಇವೇ
ತಕ-ರಾರು
ತಕ್ಕ
ತಕ್ಕಂತೆ
ತಕ್ಕ-ಡಿ-ಯಲ್ಲಿ
ತಕ್ಕ-ಮ-ಟ್ಟಿಗೆ
ತಕ್ಕು-ದ-ಲ್ಲದ
ತಕ್ಷಣ
ತಕ್ಷ-ಣವೇ
ತಗ-ಲ-ಬೇ-ಕಾ-ಗಿದೆ
ತಗಲಿ
ತಗ-ಲಿ-ಕೊಂ-ಡಂ-ತೆಯೇ
ತಗ-ಲಿ-ಕೊಂ-ಡಿತು
ತಗ-ಲಿ-ದೆ-ಯೆಂಬ
ತಗ-ಲಿ-ಬಿ-ಟ್ಟರೆ
ತಗ-ಲಿಲ್ಲ
ತಗ-ಲುವ
ತಗು-ಲಿತು
ತಗು-ಲಿ-ಸಿ-ಕೊಂಡು
ತಗ್ಗಿ-ಸಿ-ಕೊಂಡು
ತಟ-ಸ್ಥ-ವಾ-ಗಿ-ದ್ದು-ಬಿ-ಟ್ಟರು
ತಟ್ಟದೊ
ತಟ್ಟ-ಲಾ-ರಂ-ಭಿ-ಸಿ-ದರು
ತಟ್ಟಿ
ತಟ್ಟಿ-ದರು
ತಟ್ಟಿ-ದಾ-ಗಲೇ
ತಟ್ಟುತ್ತ
ತಟ್ಟೆ
ತಟ್ಟೆ-ಯಲ್ಲಿ
ತಡ
ತಡ-ಕಾ-ಡುವೆ
ತಡ-ಮಾ-ಡದೆ
ತಡ-ವಾಗಿ
ತಡಿಕೆ
ತಡೆ
ತಡೆ-ಗ-ಟ್ಟಲು
ತಡೆ-ಗ-ಟ್ಟುವ
ತಡೆದ
ತಡೆ-ದರು
ತಡೆ-ದರೂ
ತಡೆದು
ತಡೆ-ದು-ಕೊಂ-ಡಿದ್ದ
ತಡೆ-ದು-ಕೊಂಡು
ತಡೆ-ದು-ಕೊಳ್ಳ
ತಡೆ-ದು-ಕೊ-ಳ್ಳ-ಲಾ-ಗ-ಲಿಲ್ಲ
ತಡೆ-ದು-ಕೊ-ಳ್ಳ-ಲಾ-ಗು-ತ್ತಿ-ರ-ಲಿಲ್ಲ
ತಡೆ-ದು-ಕೊ-ಳ್ಳ-ಲಾ-ರದೆ
ತಡೆ-ದು-ಕೊ-ಳ್ಳು-ವು-ದಕ್ಕೆ
ತಡೆ-ದು-ಕೊ-ಳ್ಳು-ವು-ದಾ-ಗದೆ
ತಡೆ-ಯ-ಲಾ-ರದ
ತಡೆ-ಯ-ಲಾ-ರದೆ
ತಡೆ-ಯಲು
ತಡೆ-ಯುತ್ತ
ತಡೆ-ಹಿ-ಡಿ-ದರೂ
ತಡೆ-ಹಿ-ಡಿ-ದಿ-ದ್ದುದು
ತಡೆ-ಹಿ-ಡಿ-ಯು-ತ್ತಿತ್ತು
ತಣಿ-ಸಲು
ತಣಿ-ಸು-ತಿ-ಹುದು
ತಣ್ಣ-ಗಾ-ಗ-ತೊ-ಡ-ಗಿತು
ತಣ್ಣ-ಗಾ-ಗಿ-ಬಿ-ಡು-ತ್ತಿದ್ದ
ತಣ್ಣ-ಗಿತ್ತು
ತಣ್ಣೀ-ರನ್ನು
ತಣ್ಣೀರು
ತಣ್ಣೀ-ರೆ-ರ-ಚು-ವಂತೆ
ತತ್
ತತ್-ಸತ್
ತತ್ಕಾ-ಲಕ್ಕೆ
ತತ್ಕ್ಷ-ಣದ
ತತ್ತ-ರಿ-ಸಿ-ಹೋದ
ತತ್ತ್ವ
ತತ್ತ್ವ-ಸಂ-ದೇ-ಶ-ಗಳನ್ನು
ತತ್ತ್ವಕ್ಕೆ
ತತ್ತ್ವ-ಗಳ
ತತ್ತ್ವ-ಗಳನ್ನು
ತತ್ತ್ವ-ಗಳನ್ನೂ
ತತ್ತ್ವ-ಗ-ಳನ್ನೇ
ತತ್ತ್ವ-ಗ-ಳಿಗೆ
ತತ್ತ್ವ-ಗಳು
ತತ್ತ್ವ-ಗಳೂ
ತತ್ತ್ವ-ಗ-ಳೆಲ್ಲ
ತತ್ತ್ವ-ಗ-ಳೆಷ್ಟು
ತತ್ತ್ವ-ಗ-ಳೇನು
ತತ್ತ್ವ-ಜ್ಞಾ-ನಿ-ಗಳು
ತತ್ತ್ವ-ಜ್ಞಾ-ನಿ-ಯಂತೆ
ತತ್ತ್ವ-ಜ್ಞಾ-ನಿ-ಯಾ-ಗಿದ್ದ
ತತ್ತ್ವ-ಜ್ಞಾ-ನಿಯೂ
ತತ್ತ್ವದ
ತತ್ತ್ವ-ದ-ರ್ಶಿ-ಗ-ಳೆಂ-ಬರು
ತತ್ತ್ವ-ದಲ್ಲಿ
ತತ್ತ್ವ-ದೊಂ-ದಿಗೆ
ತತ್ತ್ವ-ಮಸಿ
ತತ್ತ್ವ-ವನ್ನು
ತತ್ತ್ವ-ವಾದಿ
ತತ್ತ್ವ-ವಿ-ಚಾ-ರ-ವಾಗಿ
ತತ್ತ್ವಶಃ
ತತ್ತ್ವ-ಶಾಸ್ತ್ರ
ತತ್ತ್ವ-ಶಾ-ಸ್ತ್ರ-ಸಿ-ದ್ಧಾಂ-ತ-ಗಳನ್ನು
ತತ್ತ್ವ-ಶಾ-ಸ್ತ್ರ-ಗಳ
ತತ್ತ್ವ-ಶಾ-ಸ್ತ್ರ-ಜ್ಞರ
ತತ್ತ್ವ-ಶಾ-ಸ್ತ್ರ-ಜ್ಞ-ರಾದ
ತತ್ತ್ವ-ಶಾ-ಸ್ತ್ರ-ವನ್ನೂ
ತತ್ತ್ವ-ಸಾ-ಕ್ಷಾ-ತ್ಕಾರ
ತತ್ತ್ವಾ-ಭ್ಯಾ-ಸವೂ
ತತ್ಪ-ರಿ-ಣಾ-ಮ-ವಾಗಿ
ತತ್ಫ-ಲ-ವಾಗಿ
ತಥ್ಯ
ತಥ್ಯವೂ
ತಥ್ಯಾಂಶ
ತದ-ನಂ-ತ-ರವೇ
ತದ-ನು-ಸಾ-ರ-ವಾಗಿ
ತದಾ-ತ್ಮನಿ
ತದಾ-ತ್ಮಾನಂ
ತದ್
ತದ್ರೂ-ಪಿ-ಗ-ಳಾದ
ತದ್ವತ್
ತದ್ವ-ತ್ತಾಗಿ
ತದ್ವಿ-ಜಿ-ಜ್ಞಾ-ಸಸ್ವ
ತದ್ವಿ-ರುದ್ಧ
ತದ್ವಿ-ರು-ದ್ಧ-ವಾ-ಗಿಯೇ
ತನಕ
ತನ-ಗ-ರಿ-ವಿ-ಲ್ಲ-ದಂ-ತೆಯೇ
ತನ-ಗ-ರಿ-ವಿ-ಲ್ಲ-ದೆಯೇ
ತನ-ಗಾ-ಗಿದ್ದ
ತನ-ಗಾದ
ತನ-ಗಿಂತ
ತನ-ಗಿಂ-ತಲೂ
ತನ-ಗಿನ್ನೂ
ತನ-ಗುಂ-ಟಾದ
ತನಗೂ
ತನಗೆ
ತನಗೇ
ತನ-ಗೇನು
ತನ-ಗೇನೂ
ತನ-ಗೇನೋ
ತನ-ಗೊಂದು
ತನ-ಗೊ-ದ-ಗಿದ
ತನ-ಗೊಬ್ಬ
ತನು-ಮ-ನ-ಗಳನ್ನು
ತನು-ಮ-ನ-ಗ-ಳೊಂ-ದಿಗೆ
ತನು-ಮ-ನ-ಪೂ-ರ್ವ-ಕ-ವಾಗಿ
ತನ್ನ
ತನ್ನಂ-ತಹ
ತನ್ನಂ-ತೆಯೇ
ತನ್ನಂಥ
ತನ್ನಂ-ಥ-ವ-ನನ್ನು
ತನ್ನ-ತ-ನ-ವನ್ನು
ತನ್ನದೇ
ತನ್ನ-ದೊಂದು
ತನ್ನನ್ನು
ತನ್ನನ್ನೇ
ತನ್ನಲ್ಲಿ
ತನ್ನ-ಲ್ಲು-ದಿ-ಸಿದ
ತನ್ನ-ಲ್ಲೇನೋ
ತನ್ನವ
ತನ್ನ-ವ-ರ-ನ್ನಾ-ಗಿ-ಸಿ-ಕೊ-ಳ್ಳುವ
ತನ್ನ-ಷ್ಟಕ್ಕೆ
ತನ್ನ-ಷ್ಟಕ್ಕೇ
ತನ್ನಾತ್ಮ
ತನ್ನಿಂ-ತಾನೇ
ತನ್ನಿಂದ
ತನ್ನಿಂ-ದೇ-ನಾ-ದರೂ
ತನ್ನಿ-ಚ್ಛೆ-ಯಂತೆ
ತನ್ನೆ-ಡೆಗೆ
ತನ್ನೆ-ದು-ರಿ-ನಲ್ಲಿ
ತನ್ನೆ-ದುರು
ತನ್ನೆಲ್ಲ
ತನ್ನೊಂ-ದಿಗೆ
ತನ್ನೊಂ-ದಿಗೇ
ತನ್ನೊ-ಡನೆ
ತನ್ನೊ-ಳ-ಗಿನ
ತನ್ನೊ-ಳಗೆ
ತನ್ನೊ-ಳಗೇ
ತನ್ಮ-ಯ-ಚಿ-ತ್ತ-ನಾಗಿ
ತನ್ಮ-ಯ-ನಾಗಿ
ತನ್ಮ-ಯ-ನಾ-ಗಿ-ದ್ದು-ಕೊಂಡು
ತನ್ಮ-ಯ-ರಾಗಿ
ತನ್ಮೂ-ಲಕ
ತಪ
ತಪ-ಗಳ
ತಪ-ಗೈ-ದರು
ತಪ-ಶ್ಚ-ರ್ಯೆಯ
ತಪಸ್ವಿ
ತಪ-ಸ್ವಿ-ಗ-ಳೆಂದೂ
ತಪ-ಸ್ಸನ್ನಾ
ತಪ-ಸ್ಸಾ-ಧ-ನೆ-ಗಳಲ್ಲಿ
ತಪ-ಸ್ಸಾ-ಧ-ನೆ-ಯಲ್ಲಿ
ತಪ-ಸ್ಸಿ-ಗಾಗಿ
ತಪ-ಸ್ಸಿನ
ತಪ-ಸ್ಸಿ-ನಲ್ಲಿ
ತಪಸ್ಸು
ತಪಸ್ಸೇ
ತಪೋ-ಜೀ-ವ-ನ-ವನ್ನು
ತಪೋ-ನಿ-ರ-ತ-ರಾ-ಗಿದ್ದ
ತಪೋ-ಮಯ
ತಪ್ಪದೆ
ತಪ್ಪನ್ನು
ತಪ್ಪ-ಬ-ಹುದು
ತಪ್ಪ-ಲಲ್ಲೋ
ತಪ್ಪಾಗಿ
ತಪ್ಪಾ-ಗಿಯೇ
ತಪ್ಪಾ-ಯಿತೋ
ತಪ್ಪಿ
ತಪ್ಪಿ-ತೆಂ-ಬುದು
ತಪ್ಪಿ-ದಂ-ತಾ-ಯಿತು
ತಪ್ಪಿ-ದ್ದಲ್ಲ
ತಪ್ಪಿನ
ತಪ್ಪಿಯೇ
ತಪ್ಪಿ-ರು-ವುದು
ತಪ್ಪಿಲ್ಲ
ತಪ್ಪಿಸಿ
ತಪ್ಪಿ-ಸಿ-ಕೊಂಡು
ತಪ್ಪಿ-ಸಿ-ಕೊ-ಳ್ಳಲು
ತಪ್ಪಿ-ಸಿ-ಕೊ-ಳ್ಳಲೇ
ತಪ್ಪಿ-ಸುವ
ತಪ್ಪಿ-ಹೋ-ಯಿತು
ತಪ್ಪು
ತಪ್ಪು-ಕ-ಲ್ಪನೆ
ತಪ್ಪು-ಗಳನ್ನು
ತಪ್ಪು-ಗ-ಳಿ-ಗೆಲ್ಲ
ತಪ್ಪು-ತ್ತಾನೆ
ತಪ್ಪು-ದಾರಿ-ಗೆ-ಳೆ-ಯುವ
ತಪ್ಪು-ವು-ದನ್ನೂ
ತಪ್ಪು-ವು-ದಿಲ್ಲ
ತಪ್ಪು-ವುದೇ
ತಪ್ಪೇ
ತಬಲ
ತಬಲಾ
ತಬ್ಬಲಿ
ತಬ್ಬ-ಲಿ-ಯಾಗಿ
ತಬ್ಬ-ಲಿ-ಯಾದ
ತಬ್ಬ-ಲಿ-ಯಾ-ದರೂ
ತಬ್ಬಿ
ತಬ್ಬಿ-ಕೊಂಡು
ತಬ್ಬಿ-ಕೊಂ-ಡು-ಬಿಟ್ಟ
ತಬ್ಬಿ-ಬ್ಬಾ-ಗ-ಲಿಲ್ಲ
ತಬ್ಬಿ-ಬ್ಬಾಗಿ
ತಬ್ಬಿ-ಬ್ಬಾ-ದರು
ತಬ್ಬಿ-ಬ್ಬಾ-ಯಿತು
ತಬ್ಬಿಬ್ಬು
ತಮ-ಗ-ರಿ-ವಿ-ಲ್ಲ-ದಂ-ತೆಯೇ
ತಮ-ಗ-ರಿ-ವಿ-ಲ್ಲ-ದೆಯೇ
ತಮ-ಗಾಗಿ
ತಮ-ಗಾದ
ತಮ-ಗಿಂತ
ತಮ-ಗಿ-ಲ್ಲ-ವೆಂ-ಬುದು
ತಮ-ಗಿ-ಷ್ಟ-ವಾದ
ತಮಗೆ
ತಮಗೇ
ತಮ-ಗೇ-ನಾ-ದರೂ
ತಮ-ಗೊಂದು
ತಮ-ಗೊ-ಪ್ಪಿ-ಸಿದ
ತಮ-ತ-ಮಗೆ
ತಮಾಷೆ
ತಮಾ-ಷೆ-ಗಳಲ್ಲಿ
ತಮಾ-ಷೆ-ಗಳಿಂದ
ತಮಾ-ಷೆ-ಗಳು
ತಮಾ-ಷೆ-ಗ-ಳೆಲ್ಲ
ತಮಾ-ಷೆಯ
ತಮಾ-ಷೆ-ಯ-ನ್ನು-ಗು-ರು-ತಿ-ಸು-ವಲ್ಲಿ
ತಮಾ-ಷೆ-ಯಾಗಿ
ತಮಾ-ಷೆ-ಯಾ-ಗಿಯೇ
ತಮಾ-ಷೆಯೂ
ತಮಾ-ಷೆ-ಯೇ-ನೆಂ-ದರೆ
ತಮೋ-ಗುಣ
ತಮೋ-ಗು-ಣ-ದ-ವ-ರಲ್ಲಿ
ತಮ್ಮ
ತಮ್ಮಂ-ತೆಯೇ
ತಮ್ಮಂ-ಥ-ವ-ರಿ-ಗಂತೂ
ತಮ್ಮಂ-ದಿರ
ತಮ್ಮಂ-ದಿರು
ತಮ್ಮ-ಊ-ಹಾ-ಶ-ಕ್ತಿ-ಯನ್ನು
ತಮ್ಮ-ತಮ್ಮ
ತಮ್ಮ-ತ-ಮ್ಮದೇ
ತಮ್ಮ-ತ-ಮ್ಮಲ್ಲಿ
ತಮ್ಮ-ತ-ಮ್ಮೊ-ಳಗೆ
ತಮ್ಮ-ತ-ಮ್ಮೊ-ಳಗೇ
ತಮ್ಮ-ದ-ನ್ನಾಗಿ
ತಮ್ಮದೇ
ತಮ್ಮ-ನನ್ನು
ತಮ್ಮ-ನಿಗೆ
ತಮ್ಮನ್ನು
ತಮ್ಮನ್ನೂ
ತಮ್ಮ-ನ್ನೆಂ-ದಿಗೂ
ತಮ್ಮ-ನ್ನೆಲ್ಲ
ತಮ್ಮನ್ನೇ
ತಮ್ಮಲ್ಲಿ
ತಮ್ಮ-ಲ್ಲಿಗೆ
ತಮ್ಮ-ಲ್ಲಿಯೇ
ತಮ್ಮ-ಲ್ಲಿ-ರುವ
ತಮ್ಮ-ವನು
ತಮ್ಮ-ಷ್ಟಕ್ಕೆ
ತಮ್ಮ-ಷ್ಟಕ್ಕೇ
ತಮ್ಮಿಂದ
ತಮ್ಮಿಂ-ದಾ-ದಷ್ಟು
ತಮ್ಮಿಬ್ಬ
ತಮ್ಮೆ-ಡೆಗೆ
ತಮ್ಮೆ-ದುರು
ತಮ್ಮೆ-ದೆಯ
ತಮ್ಮೆ-ರಡೂ
ತಮ್ಮೆಲ್ಲ
ತಮ್ಮೆ-ಲ್ಲರ
ತಮ್ಮೆ-ಲ್ಲ-ರನ್ನೂ
ತಮ್ಮೆ-ಲ್ಲ-ರಿ-ಗಿಂತ
ತಮ್ಮೆ-ಲ್ಲ-ರಿಗೂ
ತಮ್ಮೊಂ-ದಿಗೆ
ತಮ್ಮೊ-ಡನೆ
ತಮ್ಮೊ-ಡ-ನೆಯೇ
ತಮ್ಮೊಬ್ಬ
ತಮ್ಮೊ-ಳಗೆ
ತಮ್ಮೊ-ಳಗೇ
ತಯಾ-ರಾದ
ತಯಾ-ರಾ-ದವೋ
ತಯಾ-ರಾ-ಯಿತು
ತಯಾರಿ
ತಯಾ-ರಿ-ಯನ್ನು
ತಯಾ-ರಿ-ಯೆಲ್ಲ
ತಯಾ-ರಿ-ಸಲು
ತಯಾ-ರಿಸಿ
ತಯಾ-ರಿ-ಸಿ-ಕೊಂಡು
ತಯಾ-ರಿ-ಸಿ-ಕೊಡು
ತಯಾ-ರಿ-ಸಿ-ಕೊ-ಡು-ತ್ತಿದ್ದ
ತಯಾ-ರಿ-ಸಿದ
ತಯಾ-ರಿ-ಸು-ತ್ತಿದ್ದ
ತಯಾ-ರಿ-ಸು-ವ-ವನು
ತಯಾ-ರು-ಗೊ-ಳಿ-ಸು-ತ್ತಿ-ದ್ದರು
ತರ-ಕಾರಿ
ತರ-ಕಾ-ರಿ-ಮೀನು
ತರ-ಕಾ-ರಿ-ಯನ್ನು
ತರ-ಕಾ-ರಿ-ಯನ್ನೇ
ತರ-ಕಾ-ರಿ-ಯ-ವ-ನ-ಲ್ಲ-ವಲ್ಲ
ತರ-ಕಾ-ರಿ-ಯೊಂ-ದಿಗೆ
ತರ-ಗತಿ
ತರ-ಗ-ತಿಗೂ
ತರ-ಗ-ತಿಗೇ
ತರ-ಗ-ತಿಯ
ತರ-ಗ-ತಿ-ಯನ್ನು
ತರ-ಗ-ತಿ-ಯಲ್ಲಿ
ತರ-ಗ-ತಿ-ಯ-ಲ್ಲಿ-ದ್ದರು
ತರ-ಗೆ-ಲೆ-ಯಂತೆ
ತರ-ತ-ರದ
ತರ-ದಂತೆ
ತರ-ಬಲ್ಲ
ತರ-ಬ-ಹುದು
ತರ-ಬೇ-ಕಾ-ಗಿತ್ತು
ತರ-ಬೇ-ಕಾ-ಗಿದೆ
ತರ-ಬೇ-ಕಾ-ದರೆ
ತರ-ಬೇತಿ
ತರ-ಲಾ-ರಂ-ಭಿ-ಸಿ-ದರು
ತರಲು
ತರಲೆ
ತರ-ಲೆ-ತ್ನಿ-ಸಿದ
ತರ-ಳರು
ತರ-ವಲ್ಲ
ತರಹ
ತರ-ಹದ
ತರಾ-ಟೆಗೆ
ತರಿ-ಸ-ಲಾ-ಗಿತ್ತು
ತರಿಸಿ
ತರಿ-ಸಿ-ಕೊಂಡು
ತರಿ-ಸಿ-ಕೊಳ್ಳು
ತರಿ-ಸಿದ್ದೆ
ತರುಣ
ತರು-ಣ-ಇಷ್ಟು
ತರು-ಣನ
ತರು-ಣ-ನಾ-ದರೂ
ತರು-ಣ-ನಿಗೆ
ತರು-ಣ-ನೊಬ್ಬ
ತರು-ಣ-ಭಕ್ತ
ತರು-ಣರ
ತರು-ಣ-ರಿ-ಗೆಲ್ಲ
ತರು-ಣರು
ತರು-ಣ-ರೆಲ್ಲ
ತರು-ಣ-ಶಿ-ಷ್ಯರ
ತರು-ಣ-ಶಿ-ಷ್ಯರು
ತರು-ತಲ
ತರು-ತ್ತಿತ್ತು
ತರು-ತ್ತಿದ್ದ
ತರು-ತ್ತಿ-ದ್ದರು
ತರು-ತ್ತಿ-ದ್ದಾರೆ
ತರು-ತ್ತಿ-ದ್ದೇನೆ
ತರುವ
ತರು-ವಲ್ಲಿ
ತರು-ವ-ವ-ರಲ್ಲ
ತರು-ವಾಯ
ತರು-ವು-ದ-ಕ್ಕಾಗಿ
ತರು-ವೆ-ನೆಂಬ
ತರೋಣ
ತರ್ಕ
ತರ್ಕ-ವಿ-ತರ್ಕ
ತರ್ಕ-ವಿ-ತ-ರ್ಕ-ಗಳ
ತರ್ಕ-ವಿ-ತ-ರ್ಕ-ಗ-ಳೆಲ್ಲ
ತರ್ಕ-ಚೂ-ಡಾ-ಮಣಿ
ತರ್ಕ-ಬ-ದ್ಧ-ವಾಗಿ
ತರ್ಕ-ಬ-ದ್ಧ-ವಾ-ಗಿ-ರು-ತ್ತಿತ್ತು
ತರ್ಕ-ಮಾ-ಡಿ-ದರು
ತರ್ಕ-ವ-ನ್ನೆಲ್ಲ
ತರ್ಕ-ಶ-ಕ್ತಿಈ
ತರ್ಕ-ಶಾಸ್ತ್ರ
ತರ್ಕ-ಶಾ-ಸ್ತ್ರ-ಜ್ಞರ
ತರ್ಕಿ-ಸು-ವವು
ತಲ-ದಲ್ಲಿ
ತಲು-ಪ-ಲಾ-ರ-ದಷ್ಟು
ತಲು-ಪಲು
ತಲು-ಪಿತು
ತಲು-ಪಿತ್ತು
ತಲು-ಪಿದ
ತಲು-ಪಿ-ದಂತೆ
ತಲು-ಪಿ-ದರು
ತಲು-ಪಿ-ದ-ವರು
ತಲು-ಪಿ-ದಾಗ
ತಲು-ಪಿ-ದಾ-ಗ-ಲೆಲ್ಲ
ತಲು-ಪಿ-ದೊ-ಡ-ನೆಯೇ
ತಲು-ಪಿ-ದ್ದ-ರಿಂದ
ತಲು-ಪಿ-ರು-ವುದನ್ನು
ತಲು-ಪಿ-ಸ-ಬೇ-ಕಾ-ಗಿತ್ತು
ತಲು-ಪು-ತ್ತಿ-ದ್ದಂತೆ
ತಲು-ಪು-ತ್ತಿ-ದ್ದಾನೆ
ತಲು-ಪು-ತ್ತಿ-ರು-ವುದು
ತಲು-ಪು-ವ-ಷ್ಟ-ರಲ್ಲಿ
ತಲೆ
ತಲೆ-ಕು-ತ್ತಿ-ಗೆ-ಗಳಿಂದ
ತಲೆ-ಕೂ-ದ-ಲನ್ನು
ತಲೆ-ಕೆಟ್ಟು
ತಲೆ-ಕೆ-ಡಿ-ಸಿ-ಕೊಂಡು
ತಲೆ-ಕೆ-ಡಿ-ಸಿ-ಕೊಳ್ಳ
ತಲೆ-ಕೆ-ಳ-ಗಾಗಿ
ತಲೆ-ಗಳ
ತಲೆ-ಗಳನ್ನೆಲ್ಲ
ತಲೆ-ಗಿಲೆ
ತಲೆಗೆ
ತಲೆ-ತ-ಗ್ಗಿಸಿ
ತಲೆ-ತ-ಗ್ಗಿ-ಸಿ-ಕೊಂಡು
ತಲೆ-ತ-ಗ್ಗಿ-ಸಿದ
ತಲೆ-ತ-ಲಾಂ-ತ-ರ-ದಿಂ-ದಲೂ
ತಲೆ-ತಿ-ರು-ಗಿಸಿ
ತಲೆ-ದೂ-ಗಿ-ದರು
ತಲೆ-ದೋರಿ
ತಲೆ-ದೋ-ರಿತು
ತಲೆ-ದೋ-ರಿ-ರು-ವು-ದ-ರಿಂದ
ತಲೆ-ನೋ-ವಾ-ಗಿ-ಬಿಟ್ಟೆ
ತಲೆ-ನೋವು
ತಲೆ-ಬಾ-ಗಲೇ
ತಲೆ-ಬುಡ
ತಲೆ-ಮುಟ್ಟಿ
ತಲೆ-ಮೇಲೆ
ತಲೆಯ
ತಲೆ-ಯನ್ನು
ತಲೆ-ಯಲ್ಲಿ
ತಲೆ-ಯೆತ್ತಿ
ತಲೆ-ಯೊ-ಳ-ಗೊಂದು
ತಲೆ-ಶೂಲೆ
ತಲೆ-ಸಿ-ಡಿ-ತ-ವಂತೆ
ತಲೆ-ಸು-ತ್ತಿ-ಬಂದು
ತಲೆ-ಹಾ-ಕಿ-ದರೂ
ತಲ್ಲ-ಣ-ಗೊ-ಳ್ಳು-ತ್ತಿ-ದ್ದಳು
ತಲ್ಲ-ಣಿಸಿ
ತಲ್ಲ-ಣಿ-ಸಿತು
ತಲ್ಲ-ಣಿ-ಸಿ-ದ್ದನ್ನು
ತಲ್ಲ-ಣಿ-ಸಿ-ಹೋ-ಗು-ತ್ತಾನೆ
ತಲ್ಲ-ಣಿ-ಸಿ-ಹೋ-ಯಿತು
ತಲ್ಲ-ಣಿಸು
ತಲ್ಲ-ಣಿ-ಸು-ತ್ತಿ-ದ್ದಾರೆ
ತಲ್ಲ-ಣಿ-ಸು-ವಂ-ತಹ
ತಲ್ಲೀ-ನ-ನಾ-ಗಿ-ಬಿಟ್ಟ
ತಲ್ಲೀ-ನ-ರಾ-ಗಲು
ತಲ್ಲೀ-ನ-ರಾಗಿ
ತಲ್ಲೀ-ನ-ಳಾ-ದಳು
ತಳ-ಮಳ
ತಳ-ಮ-ಳ-ಗು-ಟ್ಟು-ತ್ತಿದೆ
ತಳ-ಮ-ಳ-ಗೊಂ-ಡಿತು
ತಳ-ಮ-ಳ-ದಿಂದ
ತಳ-ಮ-ಳಿ-ಸಿ-ದರು
ತಳ-ವಾರ
ತಳ-ವೂ-ರಿದ್ದ
ತಳ-ಹದಿ
ತಳ-ಹ-ದಿ-ಯನ್ನು
ತಳ-ಹ-ದಿ-ಯನ್ನೇ
ತಳೆದ
ತಳೆ-ದ-ದ್ದ-ಕ್ಕೊಂದು
ತಳೆ-ದಿತ್ತು
ತಳೆ-ದಿ-ದ್ದ-ವರು
ತಳೆದು
ತಳೆ-ದುವು
ತಳೆ-ಯ-ಬಲ್ಲ
ತಳೆ-ಯಿತು
ತಳ್ಳಿ
ತಳ್ಳಿದ
ತಳ್ಳಿ-ಹಾ-ಕಿ-ದ್ದನ್ನು
ತಳ್ಳಿ-ಹಾ-ಕಿ-ಬಿ-ಟ್ಟರು
ತಳ್ಳಿ-ಹಾ-ಕಿ-ಬಿಟ್ಟೆ
ತಳ್ಳಿ-ಹಾ-ಕು-ವಂ-ತಿ-ರ-ಲಿಲ್ಲ
ತಳ್ಳು
ತವಕ
ತವ-ಕ-ವುಂ-ಟಾ-ಗು-ತ್ತಿತ್ತು
ತವ-ಕಿ-ಸು-ತ್ತಿದೆ
ತವ-ಕಿ-ಸು-ತ್ತಿ-ದ್ದೇನೆ
ತವ-ಕಿ-ಸು-ತ್ತಿ-ರುವ
ತವ-ರಿಗೆ
ತಹ-ತ-ಹಿ-ಸಿ-ದರೋ
ತಾಂಡ-ವ-ನೃತ್ಯ
ತಾಂಡ-ವ-ವ-ನ್ನಾ-ಡು-ತ್ತಿ-ರುವ
ತಾಂಡ-ವ-ವಾ-ಡು-ತ್ತಿದೆ
ತಾಂಡ-ವ-ವಾ-ಡು-ತ್ತಿ-ರುವ
ತಾಜಾ
ತಾಜ್
ತಾಜ್ಮ-ಹ-ಲನ್ನು
ತಾಜ್ಮ-ಹ-ಲಿನ
ತಾಣ
ತಾಣ-ಗಳು
ತಾತ
ತಾತನ
ತಾತ-ನಂತೆ
ತಾತ-ನಾದ
ತಾತನು
ತಾತ-ನೇ-ನಾ-ದರೂ
ತಾತಯ್ಯ
ತಾತ-ಯ್ಯನ
ತಾತ-ಯ್ಯ-ನಿಗೆ
ತಾತಾ
ತಾತಾಯ್ಯ
ತಾತ್ಕಾ-ಲಿಕ
ತಾತ್ಕಾ-ಲಿ-ಕ-ವಾಗಿ
ತಾತ್ಕಾ-ಲಿ-ಕ-ವಾ-ಗಿ-ಯಾ-ದರೂ
ತಾತ್ಕಾ-ಲಿ-ಕ-ವಾ-ದದ್ದು
ತಾತ್ಪರ್ಯ
ತಾತ್ಪ-ರ್ಯ-ವನ್ನು
ತಾಥೈಯ
ತಾದಾತ್ಮ್ಯ
ತಾದಾ-ತ್ಮ್ಯ-ದಿಂದ
ತಾದಾ-ತ್ಮ್ಯ-ಭಾ-ವವು
ತಾನ-ಲ್ಲಿಗೆ
ತಾನಾಗಿ
ತಾನಾ-ಗಿತ್ತು
ತಾನಾ-ಗಿಯೇ
ತಾನಾ-ದರೋ
ತಾನಿನ್ನು
ತಾನಿಲ್ಲಿ
ತಾನೀಗ
ತಾನು
ತಾನೂ
ತಾನೆ
ತಾನೆ-ಸ-ಗಿದ
ತಾನೇ
ತಾನೇಕೆ
ತಾನೇ-ತಾ-ನಾಗಿ
ತಾನೇ-ತಾ-ನಾ-ಗಿದೆ
ತಾನೇನೋ
ತಾನೊಬ್ಬ
ತಾನೊ-ಬ್ಬನೇ
ತಾಪ-ತ್ರಯ
ತಾಪ-ತ್ರ-ಯ-ಗಳಲ್ಲಿ
ತಾಪ-ತ್ರ-ಯವೂ
ತಾಪಸಿ
ತಾಮ-ಸಿಕ
ತಾಮುಂದು
ತಾಮ್ರದ
ತಾಯಂ-ದಿರ
ತಾಯಂ-ದಿರು
ತಾಯಿ
ತಾಯಿ-ಮ-ಕ್ಕಳು
ತಾಯಿ-ಮ-ಗನ
ತಾಯಿ-ಮಗು
ತಾಯಿ-ಗಾಗಿ
ತಾಯಿ-ಗಿಂತ
ತಾಯಿಗೆ
ತಾಯಿ-ತಂದೆ
ತಾಯಿ-ತಂ-ದೆ-ಯರ
ತಾಯಿ-ಮ-ಕ್ಕಳು
ತಾಯಿಯ
ತಾಯಿ-ಯನ್ನು
ತಾಯಿ-ಯನ್ನೂ
ತಾಯಿ-ಯ-ನ್ನೊಮ್ಮೆ
ತಾಯಿ-ಯಾದ
ತಾಯಿ-ಯಿಂದ
ತಾಯಿಯು
ತಾಯಿಯೂ
ತಾಯಿಯೇ
ತಾಯಿ-ಸಿಂ-ಹದ
ತಾಯೀ
ತಾಯ್ತಂದೆ
ತಾಯ್ತಂ-ದೆ-ಯರ
ತಾಯ್ತಂ-ದೆ-ಯ-ರಿಗೆ
ತಾಯ್ತಂ-ದೆ-ಯ-ರಿ-ದ್ದಾರೆ
ತಾಯ್ತಂ-ದೆ-ಯರು
ತಾಯ್ತಂ-ದೆ-ಯ-ರೊ-ಡನೆ
ತಾಯ್ನಾ-ಡನು
ತಾರಕ
ತಾರ-ಕ-ನನ್ನು
ತಾರ-ಕ-ನಾಥ
ತಾರ-ಕ-ನಾ-ಥ-ನಿಗೆ
ತಾರ-ಕ-ನಾ-ಥನೇ
ತಾರ-ಕ-ನಿಗೆ
ತಾರ-ಕೇ-ಶ್ವ-ರಕ್ಕೆ
ತಾರ-ತಮ್ಯ
ತಾರ-ತ-ಮ್ಯ-ವನ್ನು
ತಾರದೆ
ತಾರ-ಸಿಯ
ತಾರ-ಸ್ವ-ರ-ದಲ್ಲಿ
ತಾರಾ-ಬಲ
ತಾರೀಕು
ತಾರೀ-ಖಿನ
ತಾರೀ-ಖಿ-ನಂದು
ತಾರೀಖು
ತಾರು-ಣ್ಯಕ್ಕೆ
ತಾರು-ಣ್ಯ-ದಲ್ಲೇ
ತಾರೆ-ಗಳೇ
ತಾಳ
ತಾಳದೆ
ತಾಳ-ಬೇಕು
ತಾಳ-ಲಾ-ರಂ-ಭಿ-ಸಿ-ದರು
ತಾಳ-ಲಾ-ರದ
ತಾಳ-ಲಾ-ರದೆ
ತಾಳ-ಲಿದೆ
ತಾಳ-ಲಿಲ್ಲ
ತಾಳಲು
ತಾಳಲೂ
ತಾಳ-ಲೂ-ಬ-ಹುದು
ತಾಳಿ
ತಾಳಿ-ಕೊ-ಳ್ಳು-ತ್ತಿದ್ದೆ
ತಾಳಿತು
ತಾಳಿದ
ತಾಳಿ-ದರು
ತಾಳಿ-ದರೆ
ತಾಳಿ-ದರೋ
ತಾಳಿದೆ
ತಾಳಿ-ಬಿ-ಡು-ತ್ತಿ-ದ್ದರು
ತಾಳಿ-ರು-ವಾಗ
ತಾಳಿ-ರು-ವುದು
ತಾಳಿ-ರು-ವುದೇ
ತಾಳು
ತಾಳು-ತ್ತಿದೆ
ತಾಳು-ತ್ತಿದ್ದ
ತಾಳು-ತ್ತಿ-ದ್ದರು
ತಾಳು-ವಂತೆ
ತಾಳು-ವ-ನೆಂ-ದಾ-ದರೆ
ತಾಳು-ವಾಗ
ತಾಳು-ವು-ದ-ರಲ್ಲಿ
ತಾಳು-ವುದು
ತಾಳೆ-ಯಾಗ
ತಾಳೆ-ಯಾ-ಗದ
ತಾಳೆ-ಯಾ-ಗು-ತ್ತಿದೆ
ತಾಳೆ-ಯಾ-ಗುವ
ತಾಳೆ-ಯಾ-ಗು-ವುದು
ತಾಳ್ಮೆ
ತಾಳ್ಮೆ-ಗೆ-ಡದೆ
ತಾಳ್ಮೆ-ಯಿರು
ತಾಳ್ವೆ
ತಾವಾ-ಗಿಯೇ
ತಾವಿನ್ನು
ತಾವಿ-ಬ್ಬರೂ
ತಾವೀಗ
ತಾವು
ತಾವು-ತಾವೇ
ತಾವೂ
ತಾವೆಲ್ಲ
ತಾವೆ-ಲ್ಲರೂ
ತಾವೇ
ತಾವೇ-ತಾ-ವಾ-ಗಿ-ದ್ದಾರೆ
ತಾವೇನು
ತಾವೊಬ್ಬ
ತಾಸಿನ
ತಾಸು-ಗಳ
ತಾಸೂ
ತಿಂಗಳ
ತಿಂಗಳಲ್ಲಿ
ತಿಂಗ-ಳ-ಲ್ಲಿಅ
ತಿಂಗ-ಳ-ಲ್ಲೊಂದು
ತಿಂಗ-ಳ-ವ-ರೆಗೂ
ತಿಂಗ-ಳಾ-ದರೂ
ತಿಂಗಳಿಂದ
ತಿಂಗ-ಳಿಗೆ
ತಿಂಗ-ಳಿ-ನ-ಲ್ಲೊಂದು
ತಿಂಗಳು
ತಿಂಗ-ಳು-ಗ-ಟ್ಟಲೆ
ತಿಂಗ-ಳು-ಗಳಲ್ಲಿ
ತಿಂಗ-ಳು-ಗಳು
ತಿಂಡಿ
ತಿಂಡಿ-ಗಳನ್ನು
ತಿಂಡಿ-ತಿ-ನಿ-ಸು-ಗಳನ್ನು
ತಿಂಡಿ-ತೀರ್ಥ
ತಿಂದ-ದ್ದಕ್ಕೆ
ತಿಂದ-ದ್ದಲ್ಲ
ತಿಂದರೂ
ತಿಂದರೆ
ತಿಂದ-ರೇನು
ತಿಂದಿ-ದ್ದರೂ
ತಿಂದಿ-ದ್ದರೆ
ತಿಂದೀತೆ
ತಿಂದು
ತಿಂದು-ಕೊಂಡು
ತಿಂದು-ತೇ-ಗುವ
ತಿಂದು-ಬಿ-ಡು-ತ್ತಿದ್ದೆ
ತಿಂದು-ಹಾ-ಕಿ-ದರು
ತಿಂದು-ಹಾ-ಕಿ-ರ-ಬೇಕು
ತಿಂದೆ
ತಿಂದೇ
ತಿಂದೇ-ಬಿಟ್ಟ
ತಿಕ್ಕಿ-ದರು
ತಿತಿಕ್ಷಾ
ತಿತಿಕ್ಷೆ
ತಿತಿ-ಕ್ಷೆಯ
ತಿತಿ-ಕ್ಷೆ-ಯನ್ನು
ತಿತಿ-ಕ್ಷೆ-ಯೆಂ-ಬುದು
ತಿಥಿ
ತಿದ್ದದೆ
ತಿದ್ದಲು
ತಿದ್ದ-ಲೋ-ಸುಗ
ತಿದ್ದಿ
ತಿದ್ದಿ-ಕೊಂಡು
ತಿದ್ದಿ-ಸಿ-ದರು
ತಿದ್ದಿ-ಸು-ವಾಗ
ತಿದ್ದುತ್ತ
ತಿದ್ದು-ತ್ತಿ-ದ್ದರು
ತಿದ್ದು-ವು-ದ-ರ-ಲ್ಲೊಂದು
ತಿದ್ದು-ವುದು
ತಿನಿ-ಸು-ಗಳನ್ನೆಲ್ಲ
ತಿನ್ನ-ತೊ-ಡ-ಗಿದೆ
ತಿನ್ನ-ಬೇ-ಕಾ-ಯಿತು
ತಿನ್ನ-ಬೇ-ಕೆಂದು
ತಿನ್ನ-ಲಾ-ರಂ-ಭಿ-ಸಿತ್ತು
ತಿನ್ನಲು
ತಿನ್ನಲೂ
ತಿನ್ನಿ
ತಿನ್ನಿ-ಸ-ಬೇಕು
ತಿನ್ನಿ-ಸಲು
ತಿನ್ನಿಸಿ
ತಿನ್ನಿ-ಸಿದ
ತಿನ್ನಿ-ಸಿ-ದರು
ತಿನ್ನಿ-ಸಿ-ಯೇ-ಬಿ-ಟ್ಟರು
ತಿನ್ನಿ-ಸು-ವು-ದೇನು
ತಿನ್ನು-ತ್ತಿ-ರಲಿ
ತಿನ್ನು-ತ್ತಿ-ರ-ಲಿಲ್ಲ
ತಿನ್ನು-ತ್ತಿ-ರ-ಲಿ-ಲ್ಲವೋ
ತಿನ್ನು-ತ್ತಿ-ಲ್ಲವೇ
ತಿನ್ನು-ತ್ತೇನೆ
ತಿನ್ನು-ವ-ವರ
ತಿನ್ನು-ವು-ದಕ್ಕೆ
ತಿನ್ನು-ವು-ದ-ಲ್ಲದೆ
ತಿನ್ನು-ವು-ದಿಲ್ಲ
ತಿರ-ಗಾ-ಡ-ಲಾ-ರಂ-ಭಿ-ಸಿದ
ತಿರ-ಸ್ಕ-ರಿಸಿ
ತಿರ-ಸ್ಕ-ರಿ-ಸಿದ
ತಿರ-ಸ್ಕ-ರಿ-ಸಿ-ಬಿಟ್ಟ
ತಿರ-ಸ್ಕ-ರಿ-ಸಿ-ಬಿ-ಟ್ಟಿದ್ದ
ತಿರ-ಸ್ಕ-ರಿ-ಸು-ತ್ತಾನೆ
ತಿರ-ಸ್ಕ-ರಿ-ಸು-ತ್ತಿತ್ತು
ತಿರ-ಸ್ಕ-ರಿ-ಸು-ತ್ದಿದ್ದ
ತಿರ-ಸ್ಕ-ರಿ-ಸು-ವು-ದಿಲ್ಲ
ತಿರ-ಸ್ಕಾರ
ತಿರ-ಸ್ಕಾ-ರ-ದಿಂದ
ತಿರ-ಸ್ಕಾ-ರ-ವೇ-ನಿಲ್ಲ
ತಿರಿ-ಚಿ-ಮು-ರುಚಿ
ತಿರು-ಗ-ಲಾ-ರಂ-ಭಿ-ಸಿ-ದರು
ತಿರು-ಗಾ-ಟಕ್ಕೆ
ತಿರು-ಗಾ-ಡಿ-ದರು
ತಿರು-ಗಾ-ಡುತ್ತಿ
ತಿರು-ಗಾ-ಡು-ವ-ವನು
ತಿರು-ಗಾ-ಡು-ವಾಗ
ತಿರು-ಗಾ-ಡು-ವು-ದೇ-ಕೆಂ-ದರೆ
ತಿರುಗಿ
ತಿರು-ಗಿ-ಕೊಂ-ಡರೂ
ತಿರು-ಗಿ-ಕೊಂ-ಡಿತು
ತಿರು-ಗಿ-ಕೊಂ-ಡಿ-ರು-ವುದು
ತಿರು-ಗಿ-ಕೊಳ್ಳು
ತಿರು-ಗಿತು
ತಿರು-ಗಿ-ದುವು
ತಿರು-ಗಿ-ಬಿ-ಟ್ಟಿತು
ತಿರು-ಗಿ-ಬಿ-ಡು-ತ್ತಾರೆ
ತಿರು-ಗಿ-ಬಿ-ದ್ದರು
ತಿರು-ಗಿಯೂ
ತಿರು-ಗಿ-ಸ-ತೊ-ಡ-ಗಿ-ದ್ದರು
ತಿರು-ಗಿ-ಸು-ತ್ತಾ-ರೆ-ಹೇಗೆ
ತಿರು-ಗಿ-ಸು-ತ್ತಿ-ದ್ದಾರೆ
ತಿರು-ಗಿ-ಸು-ತ್ತಿ-ರು-ವ-ವನೂ
ತಿರು-ಗಿ-ಹೋ-ಗು-ತ್ತಿತ್ತು
ತಿರು-ಗು-ತ್ತಿತ್ತು
ತಿರು-ಗು-ತ್ತಿ-ದ್ದರು
ತಿರು-ಳಿಲ್ಲ
ತಿರು-ಳಿ-ಲ್ಲ-ದವು
ತಿರು-ಳಿ-ಲ್ಲದ್ದು
ತಿರುಳೇ
ತಿರುವಿ
ತಿರು-ವುತ್ತ
ತಿಳಿ
ತಿಳಿದ
ತಿಳಿ-ದ-ದ್ದ-ರಿಂ-ದಲೇ
ತಿಳಿ-ದರು
ತಿಳಿ-ದರೆ
ತಿಳಿ-ದ-ವರು
ತಿಳಿ-ದ-ವರೇ
ತಿಳಿ-ದಾಗ
ತಿಳಿ-ದಿತ್ತು
ತಿಳಿ-ದಿದೆ
ತಿಳಿ-ದಿದ್ದ
ತಿಳಿ-ದಿ-ದ್ದರು
ತಿಳಿ-ದಿ-ದ್ದರೆ
ತಿಳಿ-ದಿ-ರ-ಬ-ಹುದು
ತಿಳಿ-ದಿ-ರ-ಬೇ-ಕಾದ
ತಿಳಿ-ದಿ-ರ-ಲಿಲ್ಲ
ತಿಳಿ-ದಿ-ರುವ
ತಿಳಿ-ದಿ-ರು-ವಂತೆ
ತಿಳಿ-ದಿ-ರು-ವುದು
ತಿಳಿ-ದಿಲ್ಲ
ತಿಳಿ-ದಿ-ಲ್ಲವೆ
ತಿಳಿ-ದಿ-ಲ್ಲ-ವೆಂ-ದಲ್ಲ
ತಿಳಿದು
ತಿಳಿ-ದುಕೊ
ತಿಳಿ-ದು-ಕೊಂಡ
ತಿಳಿ-ದು-ಕೊಂ-ಡ-ಮೇಲೆ
ತಿಳಿ-ದು-ಕೊಂ-ಡರು
ತಿಳಿ-ದು-ಕೊಂ-ಡರೂ
ತಿಳಿ-ದು-ಕೊಂ-ಡ-ವರು
ತಿಳಿ-ದು-ಕೊಂ-ಡಿದ್ದೆ
ತಿಳಿ-ದು-ಕೊಂಡು
ತಿಳಿ-ದು-ಕೊಂ-ಡು-ಬಂ-ದಿದ್ದ
ತಿಳಿ-ದು-ಕೊಂ-ಡು-ಬಿ-ಟ್ಟಿ-ದ್ದಾ-ರಲ್ಲ
ತಿಳಿ-ದು-ಕೊಂಡೆ
ತಿಳಿ-ದು-ಕೊಂ-ಡೆಯಾ
ತಿಳಿ-ದು-ಕೊ-ಳ್ಳ-ಬ-ಲ್ಲ-ವ-ರಾ-ಗಿ-ದ್ದರು
ತಿಳಿ-ದು-ಕೊ-ಳ್ಳ-ಬ-ಲ್ಲೆ-ನೆಂಬ
ತಿಳಿ-ದು-ಕೊ-ಳ್ಳ-ಬ-ಹು-ದಾದ
ತಿಳಿ-ದು-ಕೊ-ಳ್ಳ-ಬ-ಹುದು
ತಿಳಿ-ದು-ಕೊ-ಳ್ಳ-ಬೇ-ಕಾ-ಗಿದೆ
ತಿಳಿ-ದು-ಕೊ-ಳ್ಳ-ಬೇಕು
ತಿಳಿ-ದು-ಕೊ-ಳ್ಳಲು
ತಿಳಿ-ದು-ಕೊಳ್ಳಿ
ತಿಳಿ-ದು-ಕೊ-ಳ್ಳು-ತ್ತಿ-ದ್ದರು
ತಿಳಿ-ದು-ಕೊ-ಳ್ಳುವ
ತಿಳಿ-ದು-ಕೊ-ಳ್ಳು-ವ-ವರೂ
ತಿಳಿ-ದು-ಕೊ-ಳ್ಳು-ವು-ದ-ಕ್ಕಿಂತ
ತಿಳಿ-ದು-ಕೊ-ಳ್ಳು-ವುದು
ತಿಳಿ-ದು-ಬಂತು
ತಿಳಿ-ದು-ಬಂ-ತು-ಇ-ಲ್ಲೊ-ಬ್ಬರು
ತಿಳಿ-ದು-ಬಂ-ತು-ಇ-ವರೇ
ತಿಳಿ-ದು-ಬಂ-ತು-ಇ-ವೆಲ್ಲ
ತಿಳಿ-ದು-ಬಂ-ದದ್ದು
ತಿಳಿ-ದು-ಬಂ-ದಾಗ
ತಿಳಿ-ದು-ಬಂ-ದಿತು
ತಿಳಿ-ದು-ಬಂ-ದಿದೆ
ತಿಳಿ-ದು-ಬಂ-ದಿ-ದೆ-ತಾವು
ತಿಳಿ-ದು-ಬಂ-ದುವು
ತಿಳಿ-ದು-ಬ-ರು-ತ್ತದೆ
ತಿಳಿ-ದು-ಬ-ರು-ತ್ತವೆ
ತಿಳಿ-ದು-ಬಿ-ಡು-ತ್ತಿತ್ತು
ತಿಳಿ-ದು-ಹೋ-ಯಿತು
ತಿಳಿ-ದೆದ್ದ
ತಿಳಿ-ದೆಯಾ
ತಿಳಿದೇ
ತಿಳಿ-ದೊ-ಡ-ನೆಯೇ
ತಿಳಿದೋ
ತಿಳಿ-ನೀ-ರಿನ
ತಿಳಿ-ಯ-ಗೊ-ಡಲೇ
ತಿಳಿ-ಯದ
ತಿಳಿ-ಯ-ದಂ-ತೆಯೇ
ತಿಳಿ-ಯ-ದಾ-ಗಿದೆ
ತಿಳಿ-ಯದು
ತಿಳಿ-ಯದೆ
ತಿಳಿ-ಯ-ದೆಯೋ
ತಿಳಿ-ಯದೊ
ತಿಳಿ-ಯದೋ
ತಿಳಿ-ಯ-ಪ-ಡಿ-ಸು-ತ್ತಿ-ದ್ದರು
ತಿಳಿ-ಯ-ಬ-ಲ್ಲರು
ತಿಳಿ-ಯ-ಬ-ಹು-ದಾ-ದದ್ದು
ತಿಳಿ-ಯ-ಬ-ಹು-ದಾ-ದ-ಷ್ಟನ್ನೂ
ತಿಳಿ-ಯ-ಬೇ-ಕಾ-ಗಿಲ್ಲ
ತಿಳಿ-ಯ-ಬೇಕು
ತಿಳಿ-ಯ-ಬೇಕೆ
ತಿಳಿ-ಯ-ಬೇಡ
ತಿಳಿ-ಯರು
ತಿಳಿ-ಯ-ಲಾ-ಗದ
ತಿಳಿ-ಯ-ಲಾ-ರದ
ತಿಳಿ-ಯ-ಲಿಲ್ಲ
ತಿಳಿ-ಯ-ಲಿ-ಲ್ಲ-ವೆಂದೇ
ತಿಳಿ-ಯಲು
ತಿಳಿ-ಯ-ಲೆಂದು
ತಿಳಿ-ಯ-ಹೇ-ಳಿ-ದರು
ತಿಳಿ-ಯ-ಹೇ-ಳಿ-ದರೂ
ತಿಳಿ-ಯ-ಹೇ-ಳು-ತ್ತಾರೆ
ತಿಳಿ-ಯ-ಹೇ-ಳು-ತ್ತಿ-ದ್ದರು
ತಿಳಿ-ಯಾಗಿ
ತಿಳಿ-ಯಿತು
ತಿಳಿ-ಯಿತೋ
ತಿಳಿ-ಯಿ-ರಾ-ತನ
ತಿಳಿ-ಯು-ತ್ತದೆ
ತಿಳಿ-ಯು-ತ್ತಿ-ತ್ತು-ಇದು
ತಿಳಿ-ಯು-ತ್ತಿ-ದ್ದರು
ತಿಳಿ-ಯು-ತ್ತಿ-ರ-ಲಿಲ್ಲ
ತಿಳಿ-ಯುವ
ತಿಳಿ-ಯು-ವಂ-ತಾ-ಗ-ಬಾ-ರದು
ತಿಳಿ-ಯು-ವಂ-ತಾ-ಗ-ಬೇಕು
ತಿಳಿ-ಯು-ವಂ-ತಿದೆ
ತಿಳಿ-ಯು-ವಂ-ತಿಲ್ಲ
ತಿಳಿ-ಯು-ವಂತೆ
ತಿಳಿ-ಯು-ವ-ವನು
ತಿಳಿ-ಯು-ವುದ
ತಿಳಿ-ಯು-ವು-ದ-ಕ್ಕಿಂತ
ತಿಳಿ-ಯು-ವುದೇ
ತಿಳಿ-ವ-ಳಿಕೆ
ತಿಳಿ-ವ-ಳಿ-ಕೆಗೆ
ತಿಳಿ-ವ-ಳಿ-ಕೆಯ
ತಿಳಿ-ವ-ಳಿ-ಕೆ-ಯನ್ನು
ತಿಳಿ-ವ-ಳಿ-ಕೆ-ಯಲ್ಲಿ
ತಿಳಿ-ವ-ಳಿ-ಕೆ-ಯಿದ್ದ
ತಿಳಿ-ವ-ಳಿ-ಕೆಯೂ
ತಿಳಿ-ವ-ಳಿ-ಕೆ-ಯೇ-ನಾ-ದರೂ
ತಿಳಿ-ವಿಗೆ
ತಿಳಿವು
ತಿಳಿ-ಸದೆ
ತಿಳಿ-ಸ-ಲಾದ
ತಿಳಿ-ಸಲು
ತಿಳಿ-ಸ-ಹೊ-ರ-ಟಾಗ
ತಿಳಿಸಿ
ತಿಳಿ-ಸಿ-ಕೊಟ್ಟ
ತಿಳಿ-ಸಿ-ಕೊ-ಟ್ಟರು
ತಿಳಿ-ಸಿ-ಕೊಟ್ಟು
ತಿಳಿ-ಸಿ-ಕೊ-ಟ್ಟು-ಬಿ-ಟ್ಟಿ-ದ್ದಾ-ನಲ್ಲ
ತಿಳಿ-ಸಿ-ಕೊ-ಟ್ಟು-ಬಿ-ಡು-ತ್ತಿ-ದ್ದರು
ತಿಳಿ-ಸಿ-ಕೊ-ಡ-ಬೇ-ಕಾದ
ತಿಳಿ-ಸಿ-ಕೊ-ಡ-ಬೇಕು
ತಿಳಿ-ಸಿ-ಕೊ-ಡ-ಲಾಗಿದೆ
ತಿಳಿ-ಸಿ-ಕೊ-ಡಲಿ
ತಿಳಿ-ಸಿ-ಕೊ-ಡಲು
ತಿಳಿ-ಸಿ-ಕೊಡು
ತಿಳಿ-ಸಿ-ಕೊ-ಡು-ತ್ತವೆ
ತಿಳಿ-ಸಿ-ಕೊ-ಡು-ತ್ತಾಳೆ
ತಿಳಿ-ಸಿ-ಕೊ-ಡು-ತ್ತಿ-ದ್ದರು
ತಿಳಿ-ಸಿ-ಕೊ-ಡು-ತ್ತಿ-ದ್ದಾರೆ
ತಿಳಿ-ಸಿ-ಕೊ-ಡು-ತ್ತಿ-ದ್ದುವು
ತಿಳಿ-ಸಿ-ಕೊ-ಡು-ವುದ
ತಿಳಿ-ಸಿದ
ತಿಳಿ-ಸಿ-ದರು
ತಿಳಿ-ಸಿ-ದರೂ
ತಿಳಿ-ಸಿ-ದಾಗ
ತಿಳಿ-ಸಿದೆ
ತಿಳಿ-ಸಿದ್ದ
ತಿಳಿ-ಸಿ-ದ್ದರು
ತಿಳಿ-ಸಿದ್ದು
ತಿಳಿ-ಸಿ-ಬಿ-ಟ್ಟನೋ
ತಿಳಿ-ಸಿ-ಬಿಟ್ಟು
ತಿಳಿ-ಸುತ್ತ
ತಿಳಿ-ಸು-ತ್ತಾರೆ
ತಿಳಿ-ಸು-ತ್ತಿ-ದ್ದರು
ತಿಳಿ-ಸು-ತ್ತೀಯಾ
ತಿಳಿ-ಸು-ವಂ-ತಿಲ್ಲ
ತಿಳಿ-ಸೆನು
ತೀಕ್ಷ
ತೀಕ್ಷ್ಣ
ತೀಕ್ಷ್ಣ-ತೆ-ಯನ್ನು
ತೀಕ್ಷ್ಣ-ವಾದ
ತೀತಾ-ನಂ-ದರು
ತೀರ
ತೀರದ
ತೀರ-ದಲ್ಲಿ
ತೀರ-ಬೇಕು
ತೀರಾ
ತೀರಿ
ತೀರಿ-ಕೊಂಡ
ತೀರಿ-ಕೊಂ-ಡ-ನೆಂಬ
ತೀರಿ-ಕೊಂ-ಡ-ಮೇ-ಲೆಯೇ
ತೀರಿ-ಕೊಂ-ಡರು
ತೀರಿ-ಕೊಂ-ಡಳು
ತೀರಿ-ಕೊಂ-ಡಾ-ರೆಂಬ
ತೀರಿ-ಸ-ಲಿಲ್ಲ
ತೀರಿ-ಹೋ-ಗಿ-ದ್ದಳು
ತೀರಿ-ಹೋ-ಗಿ-ಬಿ-ಟ್ಟರು
ತೀರಿ-ಹೋದ
ತೀರು-ತ್ತೇನೆ
ತೀರು-ವುದು
ತೀರ್ಥ
ತೀರ್ಥ-ಕ್ಷೇತ್ರ
ತೀರ್ಥ-ಕ್ಷೇ-ತ್ರ-ಗಳ
ತೀರ್ಥ-ಕ್ಷೇ-ತ್ರ-ಗಳನ್ನು
ತೀರ್ಥ-ಕ್ಷೇ-ತ್ರ-ವಾದ
ತೀರ್ಥ-ಯಾತ್ರೆ
ತೀರ್ಥ-ಯಾ-ತ್ರೆ-ಗಾಗಿ
ತೀರ್ಥ-ಯಾ-ತ್ರೆಗೆ
ತೀರ್ಥ-ಯಾ-ತ್ರೆಯ
ತೀರ್ಥ-ಯಾ-ತ್ರೆ-ಯನ್ನು
ತೀರ್ಥಾ-ಟನೆ
ತೀರ್ಥಾ-ಟ-ನೆ-ದೇ-ಶ-ಸಂ-ಚಾರ
ತೀರ್ಥಾ-ಟ-ನೆ-ಗಾಗಿ
ತೀರ್ಥಾ-ಟ-ನೆಗೆ
ತೀರ್ಥಾ-ಟ-ನೆ-ಗೆಂದು
ತೀರ್ಥಾ-ಟ-ನೆಯ
ತೀರ್ಥಾ-ಟ-ನೆ-ಯಲ್ಲಿ
ತೀರ್ಥಾ-ಟ-ನೆ-ಯೆಂ-ದರೆ
ತೀರ್ಮಾ
ತೀರ್ಮಾನ
ತೀರ್ಮಾ-ನಕ್ಕೆ
ತೀರ್ಮಾ-ನ-ವನ್ನು
ತೀರ್ಮಾನವೂ
ತೀರ್ಮಾ-ನಿಸಿ
ತೀರ್ಮಾ-ನಿ-ಸಿದ
ತೀರ್ಮಾ-ನಿ-ಸಿ-ದರು
ತೀರ್ಮಾ-ನಿ-ಸಿದ್ದ
ತೀರ್ಮಾ-ನಿ-ಸಿ-ದ್ದರು
ತೀರ್ಮಾ-ನಿ-ಸಿ-ದ್ದರೋ
ತೀರ್ಮಾ-ನಿ-ಸಿ-ಬಿ-ಟ್ಟಿ-ದ್ದರು
ತೀರ್ಮಾ-ನಿ-ಸಿ-ರು-ವು-ದ-ರಿಂದ
ತೀವ್ರ
ತೀವ್ರ-ಗ-ತಿ-ಯಿಂದ
ತೀವ್ರ-ಗಾ-ಮಿ-ಯಾ-ಗ-ಬೇಕು
ತೀವ್ರ-ಗೊ-ಳಿಸಿ
ತೀವ್ರ-ಗೊ-ಳಿ-ಸಿದ್ದ
ತೀವ್ರ-ಗೊ-ಳಿ-ಸು-ತ್ತಾನೆ
ತೀವ್ರ-ಗೊ-ಳಿ-ಸು-ವು-ದರ
ತೀವ್ರ-ಗೊ-ಳ್ಳು-ತ್ತದೆ
ತೀವ್ರ-ತರ
ತೀವ್ರ-ತ-ರ-ವಾದ
ತೀವ್ರತೆ
ತೀವ್ರ-ತೆ-ಯನ್ನು
ತೀವ್ರ-ತೆ-ಯನ್ನೂ
ತೀವ್ರ-ತೆಯೂ
ತೀವ್ರ-ವಾಗ
ತೀವ್ರ-ವಾಗಿ
ತೀವ್ರ-ವಾ-ಗಿ-ತ್ತೆಂ-ದರೆ
ತೀವ್ರ-ವಾ-ಗಿದೆ
ತೀವ್ರ-ವಾ-ಗಿ-ಬಿ-ಟ್ಟಿತು
ತೀವ್ರ-ವಾ-ಗಿ-ಬಿ-ಟ್ಟಿತ್ತು
ತೀವ್ರ-ವಾ-ಗಿಯೇ
ತೀವ್ರ-ವಾ-ಗು-ತ್ತಿದೆ
ತೀವ್ರ-ವಾದ
ತೀವ್ರ-ವಾ-ಯಿತು
ತೀವ್ರ-ಸಾ-ಧ-ನೆ-ಯಲ್ಲಿ
ತುಂಟ
ತುಂಟ-ತನ
ತುಂಟ-ತ-ನ-ಗ-ಳಿಗೆ
ತುಂಟ-ತ-ನದ
ತುಂಟ-ನಗೆ
ತುಂಟ-ನಾ-ದರೂ
ತುಂಟಾಟ
ತುಂಟಾ-ಟ-ಚ-ಟು-ವ-ಟಿ-ಕೆ-ಗಳು
ತುಂಡಾಗಿ
ತುಂಡು
ತುಂಡು-ಮಾ-ಡಿ-ದಂತೆ
ತುಂಬ
ತುಂಬ-ಬಲ್ಲ
ತುಂಬಿ
ತುಂಬಿ-ಕೊಂಡ
ತುಂಬಿ-ಕೊಂಡಿ
ತುಂಬಿ-ಕೊಂ-ಡಿತು
ತುಂಬಿ-ಕೊಂ-ಡಿತ್ತು
ತುಂಬಿ-ಕೊಂ-ಡಿದೆ
ತುಂಬಿ-ಕೊಂ-ಡಿದ್ದ
ತುಂಬಿ-ಕೊಂ-ಡಿ-ರು-ತ್ತದೆ
ತುಂಬಿ-ಕೊಂ-ಡಿವೆ
ತುಂಬಿ-ಕೊಂಡು
ತುಂಬಿ-ಕೊಟ್ಟ
ತುಂಬಿ-ಕೊ-ಳ್ಳಲು
ತುಂಬಿ-ಕೊ-ಳ್ಳು-ತ್ತಿ-ದ್ದು-ದನ್ನು
ತುಂಬಿತು
ತುಂಬಿತ್ತು
ತುಂಬಿದ
ತುಂಬಿ-ದರು
ತುಂಬಿದ್ದು
ತುಂಬಿ-ಬಂತು
ತುಂಬಿ-ಬಿ-ಟ್ಟಿತು
ತುಂಬಿ-ಬಿ-ಟ್ಟಿತ್ತು
ತುಂಬಿ-ಬಿ-ಟ್ಟಿ-ದ್ದಾನೆ
ತುಂಬಿ-ಬಿ-ಟ್ಟಿ-ದ್ದುವು
ತುಂಬಿ-ರು-ತ್ತಿದ್ದ
ತುಂಬಿ-ರುವ
ತುಂಬಿ-ರು-ವುದನ್ನು
ತುಂಬಿ-ರು-ವುದು
ತುಂಬಿ-ಹೋ-ಗಿ-ರು-ವಂತೆ
ತುಂಬು-ತ್ತಿದ್ದ
ತುಂಬು-ಹೃ-ದ-ಯದ
ತುಂಬೆಲ್ಲ
ತುಚ್ಛ-ಹೀನ
ತುಚ್ಛೀ-ಕ-ರಿಸಿ
ತುಚ್ಛೀ-ಕ-ರಿ-ಸಿದ
ತುಟಿ-ಗ-ಳಿಗೆ
ತುಟಿ-ಗಿ-ಟ್ಟರು
ತುಟಿ-ಪಿ-ಟ-ಕ್ಕೆ-ನ್ನದೆ
ತುಟಿಯ
ತುಟಿ-ಯಿಂದ
ತುಡಿತ
ತುಡಿ-ಯು-ತ್ತಿತ್ತು
ತುಣು-ಕನ್ನು
ತುಣು-ಕನ್ನೂ
ತುಣು-ಕು-ಗಳನ್ನು
ತುತ್ತ-ತು-ದಿ-ಯ-ವ-ರೆಗೂ
ತುತ್ತಿಗೇ
ತುತ್ತು
ತುದಿ
ತುದಿಗೆ
ತುದಿ-ಗೇ-ರಿ-ದ-ವರು
ತುಪ್ಪ
ತುಪ್ಪ-ವನ್ನು
ತುಮು-ಲ-ಗಳಲ್ಲಿ
ತುಮು-ಲ-ಗಳು
ತುಮು-ಲ-ದಿಂದ
ತುಮು-ಲ-ವನ್ನು
ತುಯ್ದಾ-ಟ-ಗಳ
ತುಯ್ದಾ-ಡು-ತ್ತಿತ್ತು
ತುಯ್ದಾ-ಡು-ತ್ತಿ-ರು-ತ್ತದೆ
ತುರೀಯ
ತುರೀ-ಯಾ-ನಂದ
ತುರೀ-ಯಾ-ನಂ-ದ-ರಂತೆ
ತುರೀ-ಯಾ-ನಂ-ದರು
ತುರೀ-ಯಾ-ನಂ-ದ-ರುಈ
ತುರ್ತು
ತುಲನೆ
ತುಳು-ಕಾ-ಡು-ತ್ತಿ-ರುವ
ತುಷಾ-ರ-ಧ-ವ-ಲ-ಕಾಂ-ತಿ-ಯಿಂದ
ತೂ
ತೂಗಾ-ಡು-ತ್ತಿದೆ
ತೂಗಿ
ತೂಗಿ-ನೋಡಿ
ತೂಗು-ಹಾ-ಕಿದ್ದ
ತೂಗು-ಹಾ-ಕಿ-ದ್ದರು
ತೂಗು-ಹಾ-ಕಿ-ರು-ತ್ತಿ-ದ್ದರು
ತೂರಾ-ಡು-ತ್ತಲೇ
ತೂರಿ-ದ್ದರು
ತೂರುತ್ತ
ತೃಪ್ತ-ನಾ-ದರೆ
ತೃಪ್ತ-ರಾ-ಗಿ-ಬಿ-ಡು-ತ್ತಾರೆ
ತೃಪ್ತಿ
ತೃಪ್ತಿ-ಕರ
ತೃಪ್ತಿ-ಪ-ಟ್ಟು-ಕೊ-ಳ್ಳುವ
ತೃಪ್ತಿ-ಪ-ಡಿ-ಸಿ-ಕೊ-ಳ್ಳು-ತ್ತಾರೆ
ತೃಪ್ತಿಯ
ತೃಪ್ತಿ-ಯಾ-ಗ-ಲಿಲ್ಲ
ತೃಪ್ತಿ-ಯಾ-ದರೆ
ತೃಪ್ತಿ-ಯಿಂ-ದಿ-ರ-ಬೇಕು
ತೃಪ್ತಿ-ಯಿಲ್ಲ
ತೃಪ್ತಿ-ಯೆಂಬ
ತೃಫ್ತಿ-ಯಿಲ್ಲ
ತೃಷೆ
ತೃಷೆಗೂ
ತೃಷ್ಣೆ-ಯನ್ನು
ತೃಷ್ಣೆಯು
ತೆಂದು
ತೆಕ್ಕೆಗೆ
ತೆಗ-ಳಿ-ಕೆ-ಗಳ
ತೆಗೆ-ದರೂ
ತೆಗೆ-ದಿ-ಟ್ಟಿದ್ದು
ತೆಗೆ-ದಿ-ಟ್ಟು-ಕೊಂಡ
ತೆಗೆ-ದಿ-ಡುವ
ತೆಗೆದು
ತೆಗೆ-ದು-ಕೊಂ-ಡ-ದ್ದ-ರಿಂದ
ತೆಗೆ-ದು-ಕೊಂ-ಡ-ನೆಂ-ದರೆ
ತೆಗೆ-ದು-ಕೊಂ-ಡಿ-ರಲ್ಲ
ತೆಗೆ-ದು-ಕೊಂಡು
ತೆಗೆ-ದು-ಕೊಂ-ಡು-ಬಿ-ಟ್ಟ-ರ-ಲ್ಲ-ಅಂ-ತಹ
ತೆಗೆ-ದು-ಕೊ-ಳ್ಳದೆ
ತೆಗೆ-ದು-ಕೊ-ಳ್ಳ-ಬೇ-ಕಾ-ಗಿದ್ದ
ತೆಗೆ-ದು-ಕೊ-ಳ್ಳ-ಬೇ-ಕಾದ
ತೆಗೆ-ದು-ಕೊ-ಳ್ಳಲು
ತೆಗೆ-ದು-ಕೊಳ್ಳು
ತೆಗೆ-ದು-ಕೊ-ಳ್ಳು-ತ್ತಿ-ದ್ದರು
ತೆಗೆ-ದು-ಕೊ-ಳ್ಳುವ
ತೆಗೆ-ದು-ಕೊ-ಳ್ಳು-ವು-ದ-ಕ್ಕಾ-ದರೂ
ತೆಗೆ-ದು-ಕೊ-ಳ್ಳು-ವು-ದ-ಕ್ಕೋ-ಸ್ಕರ
ತೆಗೆ-ದು-ಬಿ-ಟ್ಟರೆ
ತೆಗೆ-ಯ-ದೆಯೇ
ತೆಗೆ-ಯ-ಲಿಲ್ಲ
ತೆಗೆ-ಯಲು
ತೆಗೆ-ಯ-ಲೇ-ಬೇಕು
ತೆಗೆ-ಯ-ಲೊ-ಪ್ಪು-ತ್ತಿಲ್ಲ
ತೆಗೆ-ಯುವ
ತೆಗೆ-ಸಿ-ಕೊ-ಟ್ಟಿದ್ದ
ತೆಪ್ಪಗೆ
ತೆರ-ದಲಿ
ತೆರದಿ
ತೆರ-ನಿದೆ
ತೆರ-ಮಾ-ತೃ-ಭಾ-ವದಿ
ತೆರವು
ತೆರೆ-ಗ-ಣ್ಣಿಗೂ
ತೆರೆ-ಗಳು
ತೆರೆ-ದಿ-ಟ್ಟಳು
ತೆರೆ-ದಿ-ಟ್ಟು-ಕೊಂಡ
ತೆರೆ-ದಿ-ಡು-ತ್ತಾರೆ
ತೆರೆದು
ತೆರೆ-ದು-ಕೊಂಡ
ತೆರೆ-ದು-ಕೊಂ-ಡಂತೆ
ತೆರೆ-ದು-ಕೊಂ-ಡಿತು
ತೆರೆ-ದು-ಕೊಂಡು
ತೆರೆದೇ
ತೆರೆ-ಯದು
ತೆರೆ-ಯನ್ನು
ತೆರೆ-ಯ-ಬೇಡ
ತೆರೆ-ಯ-ಲಾ-ಗು-ತ್ತದೆ
ತೆರೆ-ಯಲು
ತೆರೆ-ಯುವ
ತೆರೆ-ಯು-ವ-ಷ್ಟ-ರಲ್ಲಿ
ತೆರೆ-ಸು-ವಲ್ಲಿ
ತೆಳು-ವಾದ
ತೆಳು-ವೆಂದರೆ
ತೇ
ತೇಗುವ
ತೇಜಃ-ಪುಂ-ಜ-ವಾಗಿ
ತೇಜಃ-ಪುಂ-ಜ-ವಾದ
ತೇಜಸ್ವಿ
ತೇಜ-ಸ್ವಿ-ಯಾಗಿ
ತೇಜ-ಸ್ವಿ-ಯಾದ
ತೇಜಸ್ವೀ
ತೇಜ-ಸ್ಸನ್ನು
ತೇಜ-ಸ್ಸಿನ
ತೇಜ-ಸ್ಸಿ-ನಿಂದ
ತೇಜಸ್ಸು
ತೇಜೋ-ಮಯ
ತೇಜೋ-ಮ-ಯ-ವಾಗಿ
ತೇಜೋ-ಮ-ಯ-ವಾದ
ತೇಜೋ-ವಂ-ತ-ವಾ-ಗಿ-ದೆ-ಯೆಂ-ದರೆ
ತೇರ್ಗ-ಡೆ-ಯಾ-ಗಲು
ತೇರ್ಗ-ಡೆ-ಯಾ-ದ-ವನು
ತೇಲಲಿ
ತೇಲಾ-ಡು-ತ್ತಿ-ದ್ದರು
ತೇಲಿ-ಕೊಂಡು
ತೇಲಿ-ಸಿ-ಬಿ-ಟ್ಟರು
ತೇಲುತ್ತ
ತೇಲು-ತ್ತಿ-ರುವ
ತೇವ-ವಾ-ಗು-ತ್ತಿ-ದ್ದುವು
ತೈತ್ತಿ-ರೀಯ
ತೈನಾ-ತಿ-ಗಳು
ತೊಂದರೆ
ತೊಂದ-ರೆ-ಗ-ಳಿಂ-ದಾಗಿ
ತೊಂದ-ರೆ-ಗೀ-ಡಾ-ಗ-ದಿರು
ತೊಂದ-ರೆ-ಪ-ಡು-ತ್ತೀರಿ
ತೊಂದ-ರೆ-ಯಾ-ಗ-ದಂತೆ
ತೊಂದ-ರೆ-ಯಾ-ಗ-ದಿ-ರಲಿ
ತೊಂದ-ರೆ-ಯಾ-ಗಿ-ರ-ಬೇಕು
ತೊಂದ-ರೆ-ಯಾ-ಗು-ವು-ದಿಲ್ಲ
ತೊಂದ-ರೆ-ಯಿತ್ತು
ತೊಂದ-ರೆ-ಯಿದೆ
ತೊಂದ-ರೆ-ಯಿ-ರ-ಬ-ಹುದೆ
ತೊಂದ-ರೆ-ಯೇನೂ
ತೊಂದ-ರೆ-ಯೊಂ-ದರ
ತೊಟ್ಟಿ-ದ್ದರೂ
ತೊಟ್ಟು-ಕೊಂಡು
ತೊಡ-ಕನ್ನು
ತೊಡ-ಕಾ-ಗಿ-ರುವ
ತೊಡ-ಗ-ದಂತೆ
ತೊಡ-ಗ-ಬೇಕು
ತೊಡ-ಗಲು
ತೊಡಗಿ
ತೊಡ-ಗಿತ್ತು
ತೊಡ-ಗಿದ
ತೊಡ-ಗಿ-ದರು
ತೊಡ-ಗಿ-ದಳು
ತೊಡ-ಗಿ-ದಾಗ
ತೊಡ-ಗಿ-ದಾ-ಗ-ಲೆಲ್ಲ
ತೊಡ-ಗಿದೆ
ತೊಡ-ಗಿದ್ದ
ತೊಡ-ಗಿ-ದ್ದರು
ತೊಡ-ಗಿ-ದ್ದಾಗ
ತೊಡ-ಗಿ-ದ್ದಾ-ಗಲೂ
ತೊಡ-ಗಿ-ದ್ದಾರೆ
ತೊಡ-ಗಿ-ರಲಿ
ತೊಡ-ಗಿ-ರು-ತ್ತಿ-ದ್ದರು
ತೊಡ-ಗಿ-ರು-ವಂ-ತಹ
ತೊಡ-ಗಿ-ರು-ವಾಗ
ತೊಡ-ಗು-ತ್ತಾರೆ
ತೊಡ-ಗು-ತ್ತಾ-ರೆಯೋ
ತೊಡ-ಗು-ತ್ತಿ-ದ್ದರು
ತೊಡ-ಗು-ತ್ತಿ-ದ್ದೆವು
ತೊಡ-ಗುವ
ತೊಡ-ಗು-ವಂತೆ
ತೊಡ-ಗುವೆ
ತೊಡಲು
ತೊಡಿ-ಸ-ಲಾ-ಯಿತು
ತೊಡಿ-ಸ-ಲಿಲ್ಲ
ತೊಡು-ಗು-ತ್ತಾನೆ
ತೊಡು-ಗುವ
ತೊಡು-ತ್ತಿ-ದ್ದರು
ತೊಡೆ-ದು-ಹಾ-ಕುವ
ತೊದ-ಲಿದ
ತೊರೆ
ತೊರೆ-ಗಳು
ತೊರೆ-ದ-ವರು
ತೊರೆ-ದಾ-ದರೂ
ತೊರೆದು
ತೊರೆ-ದು-ಬ-ರು-ವಂತೆ
ತೊರೆ-ದೊಂ-ದೆ-ಯೆಂ-ಬು-ದ-ನ-ರಿಯೆ
ತೊಲ-ಗ-ಬೇ-ಕಾ-ದರೆ
ತೊಲ-ಗಲಿ
ತೊಲೆ
ತೊಲೆ-ಯನ್ನು
ತೊಲೆ-ಯೊಂ-ದನ್ನು
ತೊಳಲಿ
ತೊಳ-ಲುತ
ತೊಳ-ಲು-ತ್ತಿದ್ದ
ತೊಳ-ಲು-ತ್ತಿ-ದ್ದರು
ತೊಳ-ಲು-ತ್ತಿ-ರು-ವ-ವ-ನೊಬ್ಬ
ತೊಳ-ಸಿ-ಬಂದು
ತೊಳೆ
ತೊಳೆದು
ತೊಳೆ-ದು-ಕೊಳ್ಳಿ
ತೊವ್ವೆ-ಅನ್ನ
ತೋ
ತೋಚ-ದಂ-ತಾ-ಯಿತು
ತೋಚದೆ
ತೋಚ-ಲಿಲ್ಲ
ತೋಚಲೇ
ತೋಚಿ-ತು-ಇದು
ತೋಚು-ತ್ತಿಲ್ಲ
ತೋಟಕ್ಕೆ
ತೋಟದ
ತೋಟ-ದಲ್ಲಿ
ತೋಟ-ವಿತ್ತು
ತೋಡಿ
ತೋಡಿ-ಕೊಂಡ
ತೋಡಿ-ಕೊಂ-ಡಿದ್ದ
ತೋಡಿ-ಕೊಂಡು
ತೋಡಿ-ಕೊ-ಳ್ಳು-ತ್ತಾನೆ
ತೋತಾ-ಪುರಿ
ತೋತಾ-ಪು-ರಿಗೆ
ತೋಯುತ್ತ
ತೋರ-ಗೊ-ಡದೆ
ತೋರದ
ತೋರ-ದಿ-ದ್ದರೆ
ತೋರ-ದಿ-ದ್ದು-ದಕ್ಕೆ
ತೋರದು
ತೋರದೆ
ತೋರ-ಬೇಕು
ತೋರ-ಲಾ-ರಂ-ಭಿ-ಸಿತು
ತೋರ-ಲಿಲ್ಲ
ತೋರಲು
ತೋರಾ
ತೋರಾ-ಣಿಕೆ
ತೋರಾ-ಣಿ-ಕೆ-ಯ-ಲ್ಲವೆ
ತೋರಾ-ಣಿ-ಕೆ-ಯೆಂದು
ತೋರಿ-ಕೆಗೆ
ತೋರಿ-ಕೆಯ
ತೋರಿ-ಕೆಯು
ತೋರಿತು
ತೋರಿದ
ತೋರಿ-ದಾಗ
ತೋರಿ-ದು-ವಾ-ದರೂ
ತೋರಿದ್ದ
ತೋರಿದ್ದು
ತೋರಿ-ಬ-ರುವ
ತೋರಿ-ಬ-ರು-ವ-ವನು
ತೋರಿ-ಸ-ಕೊ-ಡ-ಲೆಂದು
ತೋರಿ-ಸ-ಬೇಕು
ತೋರಿ-ಸ-ಬೇಡ
ತೋರಿ-ಸಲು
ತೋರಿ-ಸ-ಹೋ-ದಾಗ
ತೋರಿಸಿ
ತೋರಿ-ಸಿ-ಕೊಟ್ಟ
ತೋರಿ-ಸಿ-ಕೊ-ಟ್ಟದ್ದೋ
ತೋರಿ-ಸಿ-ಕೊ-ಟ್ಟ-ನಂತೆ
ತೋರಿ-ಸಿ-ಕೊ-ಟ್ಟಳು
ತೋರಿ-ಸಿ-ಕೊ-ಟ್ಟಿತ್ತು
ತೋರಿ-ಸಿ-ಕೊ-ಟ್ಟಿ-ದ್ದಾಳೆ
ತೋರಿ-ಸಿ-ಕೊಟ್ಟು
ತೋರಿ-ಸಿ-ಕೊ-ಡ-ಬ-ಲ್ಲ-ವರು
ತೋರಿ-ಸಿ-ಕೊ-ಡ-ಬ-ಲ್ಲ-ವ-ರೊ-ಬ್ಬರು
ತೋರಿ-ಸಿ-ಕೊ-ಡ-ಬ-ಹು-ದ-ಲ್ಲವೆ
ತೋರಿ-ಸಿ-ಕೊ-ಡು-ತ್ತಾನೋ
ತೋರಿ-ಸಿ-ಕೊ-ಡು-ತ್ತಿ-ದ್ದರು
ತೋರಿ-ಸಿ-ಕೊ-ಡು-ತ್ತೇನೆ
ತೋರಿ-ಸಿ-ಕೊ-ಡು-ವ-ವ-ರಿ-ದ್ದಿ-ದ್ದರೆ
ತೋರಿ-ಸಿ-ಕೊ-ಳ್ಳಲು
ತೋರಿ-ಸಿ-ಕೊ-ಳ್ಳುತ್ತ
ತೋರಿ-ಸಿ-ತು-ಆ-ಗಲಿ
ತೋರಿ-ಸಿದ
ತೋರಿ-ಸಿ-ದರು
ತೋರಿ-ಸಿ-ದ-ಳಷ್ಟೆ
ತೋರಿ-ಸಿದ್ದು
ತೋರಿ-ಸಿ-ದ್ದುಂ-ಟು-ಆ-ದರೆ
ತೋರಿ-ಸಿಲ್ಲ
ತೋರಿಸು
ತೋರಿ-ಸುತ್ತ
ತೋರಿ-ಸು-ತ್ತಲೂ
ತೋರಿ-ಸು-ತ್ತವೆ
ತೋರಿ-ಸು-ತ್ತಿದ್ದ
ತೋರಿ-ಸು-ತ್ತಿ-ರು-ವಾಗ
ತೋರಿ-ಸುವ
ತೋರಿ-ಸು-ವಂ-ತಹ
ತೋರಿ-ಸು-ವಂತೆ
ತೋರಿ-ಸು-ವು-ದ-ಕ್ಕಾಗಿ
ತೋರಿ-ಸು-ವು-ದಾಗಿ
ತೋರಿ-ಸು-ವು-ದಾ-ದರೂ
ತೋರು
ತೋರುತ್ತ
ತೋರು-ತ್ತದೆ
ತೋರು-ತ್ತ-ವೆಯೇ
ತೋರು-ತ್ತಾ-ರೆ-ಅ-ದ-ರಲ್ಲೂ
ತೋರು-ತ್ತಿತ್ತು
ತೋರು-ತ್ತಿದೆ
ತೋರು-ತ್ತಿದ್ದ
ತೋರು-ತ್ತಿ-ದ್ದರು
ತೋರು-ತ್ತಿ-ದ್ದಾ-ರಲ್ಲ
ತೋರು-ತ್ತಿ-ದ್ದುವು
ತೋರು-ತ್ತಿ-ರುವ
ತೋರುವ
ತೋರು-ವ-ನಾ-ತನು
ತೋರು-ವ-ವ-ರಾರು
ತೋರು-ವ-ವರು
ತೋರುವು
ತೋರು-ವು-ದಲ್ಲ
ತೋಳ-ಲ್ಲೆ-ತ್ತಿ-ಕೊಂ-ಡಾಗ
ತೋಳಿ-ನ-ಲ್ಲೆ-ತ್ತಿ-ಕೊ-ಳ್ಳು-ವಂತೆ
ತ್ಕಾರ
ತ್ಕಾರ-ಕ್ಕಾಗಿ
ತ್ಕಾರದ
ತ್ಕಾರ್ಯ-ಗಳನ್ನು
ತ್ತದೆ
ತ್ತಲೇ
ತ್ತಲ್ಲ
ತ್ತವೆ
ತ್ತವೆಯೋ
ತ್ತಾನೆ
ತ್ತಾನೆ-ಶ್ರೀ-ರಾ-ಮ-ಕೃ-ಷ್ಣರ
ತ್ತಾರೆ
ತ್ತಾರೋ
ತ್ತಿತ್ತು
ತ್ತಿತ್ತು-ಎಲ್ಲ
ತ್ತಿತ್ತೆಂ-ದರೆ
ತ್ತಿದೆ
ತ್ತಿದೆಯೋ
ತ್ತಿದ್ದ
ತ್ತಿದ್ದಂತೆ
ತ್ತಿದ್ದ-ಇಂದು
ತ್ತಿದ್ದ-ರಾ-ದರೂ
ತ್ತಿದ್ದರು
ತ್ತಿದ್ದರೆ
ತ್ತಿದ್ದರೊ
ತ್ತಿದ್ದರೋ
ತ್ತಿದ್ದಳು
ತ್ತಿದ್ದ-ವಳು
ತ್ತಿದ್ದವು
ತ್ತಿದ್ದಾಗ
ತ್ತಿದ್ದಾನೆ
ತ್ತಿದ್ದಾ-ರಲ್ಲ
ತ್ತಿದ್ದಾರೆ
ತ್ತಿದ್ದುದು
ತ್ತಿದ್ದುದೂ
ತ್ತಿದ್ದುವು
ತ್ತಿದ್ದೆಓ
ತ್ತಿದ್ದೆನೋ
ತ್ತಿರ-ಲಿಲ್ಲ
ತ್ತಿರುವ
ತ್ತಿರು-ವ-ವರು
ತ್ತಿರು-ವಾಗ
ತ್ತಿರು-ವುದು
ತ್ತೀಯಾ
ತ್ತೀಯೋ
ತ್ತೆಂದರೆ
ತ್ತೆಂಬುದು
ತ್ತೇನೆ
ತ್ತೊಂದು
ತ್ತ್
ತ್ಮಕ
ತ್ಮಿಕ
ತ್ಮಿಕ-ತೆ-ಯೆ-ನ್ನು-ವುದು
ತ್ಯಜಿ-ಸ-ಬೇಕು
ತ್ಯಜಿಸಿ
ತ್ಯಜಿ-ಸಿ-ಬಿ-ಡು-ತ್ತಾನೆ
ತ್ಯಜಿ-ಸಿ-ಬಿ-ಡುವ
ತ್ಯಜಿ-ಸಿ-ರುವ
ತ್ಯಾಗ
ತ್ಯಾಗ-ತ-ಪ-ಸ್ಸು-ಗ-ಳಲ್ಲೂ
ತ್ಯಾಗ-ವೈ-ರಾಗ್ಯ
ತ್ಯಾಗ-ವೈ-ರಾ-ಗ್ಯ-ಗಳ
ತ್ಯಾಗ-ವೈ-ರಾ-ಗ್ಯ-ಭಾವ
ತ್ಯಾಗಕ್ಕೂ
ತ್ಯಾಗಕ್ಕೆ
ತ್ಯಾಗ-ಜೀ-ವನ
ತ್ಯಾಗ-ಜೀ-ವ-ನದ
ತ್ಯಾಗ-ಜೀ-ವ-ನ-ವನ್ನು
ತ್ಯಾಗ-ಜೀ-ವ-ನ-ವನ್ನೇ
ತ್ಯಾಗ-ಜೀ-ವ-ನವೇ
ತ್ಯಾಗದ
ತ್ಯಾಗ-ದ-ಲ್ಲಾ-ಗಲಿ
ತ್ಯಾಗ-ದಲ್ಲಿ
ತ್ಯಾಗ-ಬು-ದ್ದಿ-ಇ-ವೆಲ್ಲ
ತ್ಯಾಗ-ಬುದ್ಧಿ
ತ್ಯಾಗ-ಬು-ದ್ಧಿ-ಯನ್ನು
ತ್ಯಾಗ-ಭ-ರಿ-ತ-ರಾದ
ತ್ಯಾಗ-ಮಯ
ತ್ಯಾಗ-ಮಾ-ಡ-ಬೇ-ಕಾ-ದದ್ದು
ತ್ಯಾಗ-ಮಾಡಿ
ತ್ಯಾಗ-ವನ್ನು
ತ್ಯಾಗ-ವನ್ನೂ
ತ್ಯಾಗ-ವನ್ನೇ
ತ್ಯಾಗ-ವಿತ್ತು
ತ್ಯಾಗ-ವಿ-ಲ್ಲದೆ
ತ್ಯಾಗ-ವೆಂ-ಬು-ದರ
ತ್ಯಾಗ-ವೆಂ-ಬುದು
ತ್ಯಾಗ-ವೈ-ರಾಗ್ಯ
ತ್ಯಾಗ-ವೈ-ರಾ-ಗ್ಯ-ಗ-ಳನ್ನೇ
ತ್ಯಾಗ-ವೈ-ರಾ-ಗ್ಯ-ಗಳು
ತ್ಯಾಗಿ-ಗಳ
ತ್ಯಾಗಿ-ಗ-ಳ-ನ್ನಾಗಿ
ತ್ಯಾಗಿ-ಗ-ಳ-ನ್ನಾ-ಗಿಯೇ
ತ್ಯಾಗಿ-ಗ-ಳಾ-ಗಿ-ರ-ಬೇ-ಕಾ-ಗು-ತ್ತದೆ
ತ್ಯಾಗಿ-ಗ-ಳಾ-ಗಿ-ರ-ಬೇಕು
ತ್ಯಾಗಿ-ಗಳು
ತ್ಯಾಗಿ-ಚಾ-ಗಿ-ಗಳ
ತ್ಯಾಗಿಯ
ತ್ಯಾಗಿ-ಯಂತೆ
ತ್ಯಾಜ್ಯ
ತ್ರಯೋ-ದ-ಶಿಯ
ತ್ರಿಕ-ರ-ಣ-ಪೂ-ರ್ವ-ಕ-ವಾಗಿ
ತ್ರಿಗುಣಾ
ತ್ರಿಗು-ಣಾ-ತೀ-ತಾ-ನಂದ
ತ್ರಿಗು-ಣಾ-ತೀ-ತಾ-ನಂ-ದರು
ತ್ರಿಭು-ವನ
ತ್ರಿವಿ-ಕ್ರ-ಮ-ಗಾ-ತ್ರಕ್ಕೆ
ತ್ರೈಲಿಂಗ
ತ್ರೈಲಿಂ-ಗ-ಸ್ವಾ-ಮಿ-ಗಳನ್ನು
ತ್ರೈಲಿಂ-ಗ-ಸ್ವಾ-ಮಿ-ಗಳು
ತ್ರೈಲೋ-ಕ್ಯ-ನಾಥ
ತ್ರೈಲೋ-ಕ್ಯ-ನಾ-ಥ-ಘೋಷ್
ತ್ವರಿ-ತದ
ತ್ಸಂಬಂ-ಧ-ವಾದ
ತ್ಸರ್ಯ
ತ್ಸೆಗೂ
ಥಂಡಿ
ಥಂಡಿಯ
ಥಕ
ಥಳ-ಕಿನ
ಥಳಿ-ಸ-ತೊ-ಡ-ಗಿ-ದರು
ಥೇಯ-ರಾದ
ದಂ
ದಂಗಾ-ಗಿ-ಬಿ-ಡ-ಬೇ-ಕಾ-ಗಿತ್ತು
ದಂಗಾದ
ದಂಗು
ದಂಡ
ದಂಡ-ಕ-ಮಂ-ಡಲು
ದಂಡ-ಕ-ಮಂ-ಡ-ಲು-ಗಳು
ದಂಡ-ಗಳನ್ನು
ದಂಡೋ-ಪಾ-ಯವೇ
ದಂತಹ
ದಂತೂ
ದಂತೆ
ದಂಪ-ತಿ-ಗಳ
ದಂಪ-ತಿ-ಗಳು
ದಕ್ಕಾಗಿ
ದಕ್ಕೂ
ದಕ್ಕೆ
ದಕ್ಕೇ
ದಕ್ಕೋ-ಸ್ಕರ
ದಕ್ಷ-ತೆ-ಯಿಂದ
ದಕ್ಷಮ್
ದಕ್ಷಿ-ಣ-ಭಾ-ರ-ತ-ದಲ್ಲಿ
ದಕ್ಷಿ-ಣ-ಭಾ-ರ-ತೀ-ಯ-ರಾದ
ದಕ್ಷಿ-ಣೇ-ಶ್ವರ
ದಕ್ಷಿ-ಣೇ-ಶ್ವ-ರ-ಕ-ಲ್ಕ-ತ್ತ-ಗಳ
ದಕ್ಷಿ-ಣೇ-ಶ್ವ-ರಕ್ಕೆ
ದಕ್ಷಿ-ಣೇ-ಶ್ವ-ರ-ಕ್ಕೊಮ್ಮೆ
ದಕ್ಷಿ-ಣೇ-ಶ್ವ-ರದ
ದಕ್ಷಿ-ಣೇ-ಶ್ವ-ರ-ದಲ್ಲಿ
ದಕ್ಷಿ-ಣೇ-ಶ್ವ-ರ-ದ-ಲ್ಲಿ-ದ್ದಾ-ಗಲೇ
ದಕ್ಷಿ-ಣೇ-ಶ್ವ-ರ-ದ-ಲ್ಲಿ-ರು-ವ-ವ-ರೆಗೂ
ದಕ್ಷಿ-ಣೇ-ಶ್ವ-ರ-ದಲ್ಲೇ
ದಕ್ಷಿ-ಣೇ-ಶ್ವ-ರ-ದ-ಲ್ಲೊಬ್ಬ
ದಕ್ಷಿ-ಣೇ-ಶ್ವ-ರ-ದ-ವ-ರೆಗೂ
ದಕ್ಷಿ-ಣೇ-ಶ್ವ-ರ-ದಿಂದ
ದಕ್ಷೀ-ಣೇ-ಶ್ವ-ರಕ್ಕೆ
ದಟ್ಟ-ವಾಗಿ
ದಟ್ಟ-ವಾದ
ದಡ
ದಡಕೆ
ದಡಕ್ಕೆ
ದಡ-ಗಳು
ದಡದ
ದಡ-ದಲ್ಲಿ
ದಡ-ದಿಂದ
ದಡ-ಬಡ
ದಡ-ಬ-ಡಿಸಿ
ದಡ-ಸೇ-ರಿತು
ದಡಿ-ಯಲ್ಲಿ
ದಡ್ಡ
ದಣಿ-ದರೂ
ದಣಿ-ದಿ-ದ್ದಾರೆ
ದಣಿದು
ದಣಿ-ಯಿರೈ
ದಣಿ-ವ-ರಿ-ಯದ
ದಣಿ-ವಾ-ದಾಗ
ದಣಿ-ವಾ-ರಿ-ಸಿ-ಕೊ-ಳ್ಳುವ
ದಣಿ-ವಿನ
ದಣಿವು
ದಣಿ-ವೆ-ನ್ನದೆ
ದಣಿವೋ
ದತ್ತ
ದತ್ತನ
ದತ್ತ-ನಿಗೆ
ದತ್ತನೂ
ದದ್ದೂ
ದನ-ಕ-ರು-ಗ-ಳು-ಎ-ಲ್ಲವೂ
ದನಿ
ದನಿ-ಗೂ-ಡಿ-ಸಿದ
ದನಿ-ಯಲ್ಲಿ
ದನ್ನು
ದಬಾ-ಯಿ-ಸಿದ
ದಬ್ಬಾ-ಳಿಕೆ
ದಮ
ದಯ
ದಯ-ಪಾ-ಲಿಸು
ದಯ-ಪಾ-ಲಿ-ಸು-ತ್ತಾಳೆ
ದಯ-ಮಾಡಿ
ದಯ-ವಿಟ್ಟು
ದಯಾ-ನಂದ
ದಯಾ-ನಿಧೆ
ದಯಾ-ಪೂರ್ಣ
ದಯಾ-ಮ-ಯ-ನಾ-ಗಿ-ದ್ದಲ್ಲಿ
ದಯಾ-ಮ-ಯ-ನಾದ
ದಯಾ-ಸಾ-ಗರ
ದಯೆ
ದಯೆ-ಯನ್ನು
ದರಲ್ಲಿ
ದರು
ದರೂ
ದರೆ
ದರೋ
ದರ್ಜೆ-ಗ-ಳಿ-ಸಿದ
ದರ್ಜೆಯ
ದರ್ಜೆ-ಯಲ್ಲಿ
ದರ್ಬಾ-ರಿ-ನಲ್ಲಿ
ದರ್ಬಾರು
ದರ್ಮ-ಪ್ರ-ಸಾ-ರ-ಕಾ-ರ್ಯವು
ದರ್ವಾ-ನ-ಹು-ಡು-ಗ-ರಿ-ಗೆಲ್ಲ
ದರ್ಶ-ಕ-ರಾಗಿ
ದರ್ಶನ
ದರ್ಶ-ನ-ಗಿ-ರ್ಶ-ನ-ಗ-ಳಾ-ಗಿ-ದ್ದಿ-ದ್ದರೆ
ದರ್ಶ-ನ-ಕ್ಕಾಗಿ
ದರ್ಶ-ನ-ಗಳ
ದರ್ಶ-ನ-ಗ-ಳ-ನ್ನಾ-ಗಲಿ
ದರ್ಶ-ನ-ಗಳನ್ನು
ದರ್ಶ-ನ-ಗಳಲ್ಲಿ
ದರ್ಶ-ನ-ಗ-ಳಾ-ಗು-ತ್ತಲೇ
ದರ್ಶ-ನ-ಗ-ಳಾ-ದವು
ದರ್ಶ-ನ-ಗಳು
ದರ್ಶ-ನ-ಗ-ಳೆಷ್ಟು
ದರ್ಶ-ನ-ಗಾ-ಳಾ-ದುವು
ದರ್ಶ-ನದ
ದರ್ಶ-ನ-ದಲ್ಲಿ
ದರ್ಶ-ನ-ದಿಂದ
ದರ್ಶ-ನ-ದ್ಲಲಿ
ದರ್ಶ-ನ-ಭಾ-ಗ್ಯ-ವಾ-ಯಿತು
ದರ್ಶ-ನ-ಲಾ-ಭ-ವಾ-ಗ-ದಿದ್ದ
ದರ್ಶ-ನ-ವನ್ನು
ದರ್ಶ-ನ-ವನ್ನೇ
ದರ್ಶ-ನ-ವಾ-ಗ-ಲಿ-ಲ್ಲ-ವಲ್ಲ
ದರ್ಶ-ನ-ವಾ-ಗಿತ್ತು
ದರ್ಶ-ನ-ವಾ-ಗು-ತ್ತಿದೆ
ದರ್ಶ-ನ-ವಾ-ಗುವ
ದರ್ಶ-ನ-ವಾ-ಗು-ವಂತೆ
ದರ್ಶ-ನವೇ
ದರ್ಶ-ನ-ವೊಂ-ದ-ರಲ್ಲಿ
ದರ್ಶ-ನಾ-ನು-ಭ-ವ-ವನ್ನು
ದರ್ಶ-ನಾ-ನು-ಭ-ವ-ವಾ-ಗು-ತ್ತಿತ್ತು
ದರ್ಶ-ನಾ-ರ್ಥಿ-ಗಳ
ದರ್ಶಿ-ಸಿ-ದ್ದೇನೆ
ದಲ್ಲದೆ
ದಲ್ಲಿ
ದಲ್ಲಿದ್ದ
ದಲ್ಲಿ-ರು-ವಂ-ತಾ-ಯಿತು
ದಲ್ಲಿ-ರು-ವ-ವ-ರೆಗೂ
ದಲ್ಲೂ
ದಲ್ಲೆ-ದ್ದಿದ್ದ
ದಲ್ಲೇ
ದಳ-ಗ-ಳಂತೆ
ದಳದ
ದಳೆಂ-ದರೆ
ದವ-ಡೆ-ಗಳು
ದವ-ಡೆಗೆ
ದವ-ತಾರ
ದವನು
ದವ-ರಲ್ಲಿ
ದವರು
ದವು
ದಶ-ನಾಮೀ
ದಷ್ಟ-ಪುಷ್ಟ
ದಷ್ಟ-ಪು-ಷ್ಟ-ವಾ-ಗಿತ್ತು
ದಷ್ಟ-ಪು-ಷ್ಟ-ವಾದ
ದಸ್ತ-ಗಿರಿ
ದಹನ
ದಹ-ನ-ಕಾರ್ಯ
ದಹ-ನ-ಮಾ-ಡ-ಬಾ-ರ-ದೆಂಬ
ದಹ-ನ-ಮಾ-ಡಿದ್ದು
ದಹ-ನ-ಮಾ-ಡಿ-ದ್ದೇಕೆ
ದಾ
ದಾಂಪ-ತ್ಯ-ಸು-ಖವೂ
ದಾಖ-ಲಿ-ಸಿ-ದರು
ದಾಖಲೆ
ದಾಖ-ಲೆ-ಗಳ
ದಾಗ
ದಾಗಿ
ದಾಗಿತ್ತು
ದಾಗಿಯೇ
ದಾಟ-ಬೇ-ಕಿತ್ತು
ದಾಟಲು
ದಾಟಿ
ದಾಟಿ-ಕೊಂಡು
ದಾಟಿದೆ
ದಾಟಿ-ಬಿ-ಟ್ಟರೆ
ದಾಟಿಸು
ದಾಟುವ
ದಾತಾ
ದಾದಿ-ಯ-ರನ್ನು
ದಾನ-ಧ-ರ್ಮಾದಿ
ದಾನಂದ
ದಾನ-ಧ-ರ್ಮಾದಿ
ದಾನ-ವನು
ದಾನಿ-ಗಳು
ದಾನಿ-ಗಳೆ
ದಾಪು-ಗಾಲು
ದಾಯ-ಗಳ
ದಾಯವೇ
ದಾರದ
ದಾರಿ
ದಾರಿ-ಖ-ರ್ಚಿಗೆ
ದಾರಿ-ಗ-ಳಿಂ-ದಲೂ
ದಾರಿ-ಗಳು
ದಾರಿ-ಗಾ-ಣದೆ
ದಾರಿ-ತಪ್ಪಿ
ದಾರಿ-ತ-ಪ್ಪಿ-ಯಾನು
ದಾರಿ-ತ-ಪ್ಪಿಸು
ದಾರಿ-ತೋ-ರ-ಬ-ಲ್ಲನೆ
ದಾರಿ-ತೋ-ರ-ಬ-ಲ್ಲ-ವನು
ದಾರಿದ್ರ್ಯ
ದಾರಿ-ದ್ರ್ಯದ
ದಾರಿ-ದ್ರ್ಯ-ವನ್ನು
ದಾರಿ-ಪಕ್ಕದ
ದಾರಿಯ
ದಾರಿ-ಯಂತೆ
ದಾರಿ-ಯನ್ನು
ದಾರಿ-ಯ-ಲ್ಲದ
ದಾರಿ-ಯಲ್ಲಿ
ದಾರಿ-ಯ-ಲ್ಲಿ-ರುವ
ದಾರಿ-ಯಲ್ಲೂ
ದಾರಿ-ಯ-ಲ್ಲೆಲ್ಲೂ
ದಾರಿ-ಯ-ಲ್ಲೆಲ್ಲೋ
ದಾರಿ-ಯಾಗಿ
ದಾರಿ-ಯು-ದ್ದಕ್ಕೂ
ದಾರಿಯೂ
ದಾರಿ-ಯೇನು
ದಾರಿ-ಹಿ-ಡಿದು
ದಾರುಣ
ದಾರ್
ದಾರ್ಶ-ನಿ-ಕನ
ದಾಳಿ-ಮಾ-ಡಿ-ಬಿ-ಟ್ಟುವು
ದಾಸ
ದಾಸ-ನ-ನ್ನಾಗಿ
ದಾಸ-ನಾಗಿ
ದಾಸ-ನಿಗೆ
ದಾಸ-ನಿ-ಗೇನು
ದಾಸನು
ದಾಸ-ನೆಂ-ಬುದೆ
ದಾಸ-ಯ್ಯ-ಗಳನ್ನು
ದಾಸ-ಯ್ಯ-ಗ-ಳಾ-ಗಿ-ಬಿ-ಡು-ತ್ತಾರೆ
ದಾಸಾ-ನು-ದಾಸ
ದಾಸ್
ದಾಸ್ಯ-ದ-ಲ್ಲಿ-ರು-ವ-ವನು
ದಿಂಡರು
ದಿಂದ
ದಿಂದಲೂ
ದಿಂದಲೇ
ದಿಂದಲೋ
ದಿಂದಾ-ಗಲಿ
ದಿಂದಾಗಿ
ದಿಂಬಿನ
ದಿಂಬು-ಗಳ
ದಿಂಬು-ಗಳನ್ನು
ದಿಕ್ಕು
ದಿಕ್ಕು-ಗ-ಳಿಂ-ದಲೂ
ದಿಕ್ಕು-ತೋ-ಚದೆ
ದಿಕ್ಕೆಟ್ಟ
ದಿಕ್ಕೆ-ಡದೆ
ದಿಗಂ-ತ-ದಲ್ಲಿ
ದಿಗಂ-ಬ-ರ-ರಾಗಿ
ದಿಗಿ-ಲು-ಬಿ-ದ್ದರು
ದಿಗೂ
ದಿಗ್ಭ್ರ-ಮೆಯ
ದಿಗ್ಭ್ರಾಂತ
ದಿಗ್ಭ್ರಾಂ-ತ-ನಾಗಿ
ದಿಗ್ಭ್ರಾಂ-ತಿ-ಯಾ-ಯಿತು
ದಿಙ್ಮುಖ
ದಿಙ್ಮೂ-ಢ-ರಾಗಿ
ದಿಟ್ಟ-ತ-ನದ
ದಿಟ್ಟ-ತ-ನ-ದಿಂದ
ದಿಟ್ಟ-ತ-ನ-ವನ್ನು
ದಿಟ್ಟಿ-ಸ-ಲಾ-ರಂ-ಭಿ-ಸಿ-ದರು
ದಿಟ್ಟಿಸಿ
ದಿಟ್ಟಿ-ಸಿದ
ದಿಟ್ಟಿ-ಸಿ-ದರು
ದಿಟ್ಟಿ-ಸುತ್ತ
ದಿಟ್ಟಿ-ಸು-ತ್ತಿ-ದ್ದಾರೆ
ದಿಟ್ಟು-ಕೊಂ-ಡಿ-ದ್ದರು
ದಿದ್ದರೂ
ದಿದ್ದರೆ
ದಿನ
ದಿನಂ-ಪ್ರ-ತಿಯ
ದಿನಕ್ಕೆ
ದಿನ-ಗಳ
ದಿನ-ಗ-ಳ-ದಂ-ತೆಲ್ಲ
ದಿನ-ಗಳನ್ನು
ದಿನ-ಗಳಲ್ಲಿ
ದಿನ-ಗ-ಳ-ಲ್ಲೆಲ್ಲ
ದಿನ-ಗ-ಳಲ್ಲೇ
ದಿನ-ಗ-ಳ-ಲ್ಲೊಮ್ಮೆ
ದಿನ-ಗ-ಳ-ವ-ರೆಗೂ
ದಿನ-ಗ-ಳ-ವ-ರೆಗೆ
ದಿನ-ಗ-ಳಾ-ಗ-ಬೇ-ಕಿತ್ತು
ದಿನ-ಗ-ಳಾಗಿ
ದಿನ-ಗ-ಳಾ-ಗಿವೆ
ದಿನ-ಗ-ಳಾ-ದರೂ
ದಿನ-ಗ-ಳಾ-ದುವು
ದಿನ-ಗಳಿಂದ
ದಿನ-ಗ-ಳಿಂ-ದಲೂ
ದಿನ-ಗ-ಳಿ-ಗಿಂತ
ದಿನ-ಗ-ಳಿ-ರು-ವಾಗ
ದಿನ-ಗ-ಳಿವೆ
ದಿನ-ಗಳು
ದಿನ-ಗಳೇ
ದಿನ-ಚ-ರಿ-ಯ-ನ್ನಾ-ಗಲಿ
ದಿನದ
ದಿನ-ದಂದು
ದಿನ-ದಂದೇ
ದಿನ-ದಲ್ಲಿ
ದಿನ-ದಿಂದ
ದಿನ-ದಿನ
ದಿನ-ದಿ-ನಕ್ಕೆ
ದಿನ-ದೊ-ಳ-ಗೆಲ್ಲ
ದಿನ-ನಿ-ತ್ಯ-ವೆಂ-ಬಂತೆ
ದಿನ-ಬೆ-ಳ-ಗಾ-ದರೆ
ದಿನ-ವಂತೂ
ದಿನ-ವನ್ನು
ದಿನ-ವನ್ನೂ
ದಿನ-ವಾದ
ದಿನ-ವಾ-ದರೂ
ದಿನ-ವಿಡೀ
ದಿನವೂ
ದಿನ-ವೆಲ್ಲ
ದಿನವೇ
ದಿನ-ವೊಂದು
ದಿನಾ-ಚ-ರ-ಣೆ-ಯ-ನನು
ದಿನಾಲೂ
ದಿನೇ-ದಿನೇ
ದಿರಿ
ದಿರು-ವಷ್ಟು
ದಿಲ್ಲ
ದಿವಂ-ಗತ
ದಿವಸ
ದಿವ-ಸ-ಗಳು
ದಿವಾನ
ದಿವಾ-ನರ
ದಿವಾ-ನ-ರಾದ
ದಿವಾ-ನರು
ದಿವಾ-ನ್-ಇ-ಹ-ಫೀಜ್
ದಿವಾಳಿ
ದಿವ್ಯ
ದಿವ್ಯ-ಗುರು
ದಿವ್ಯ-ಜ-ನ-ನ-ದಿಂ-ದಾಗಿ
ದಿವ್ಯ-ಜ್ಯೋ-ತಿ-ಯನ್ನು
ದಿವ್ಯ-ತೆಯ
ದಿವ್ಯ-ತೆ-ಯನ್ನು
ದಿವ್ಯ-ತೆ-ಯೊಂದೇ
ದಿವ್ಯ-ದ-ರ್ಶ-ನ-ವಾದ
ದಿವ್ಯ-ದೃ-ಷ್ಟಿಗೆ
ದಿವ್ಯ-ಧಾ-ಮದ
ದಿವ್ಯ-ಪೂ-ಜೆ-ಯನ್ನು
ದಿವ್ಯ-ಪ್ರೇ-ಮಕೆ
ದಿವ್ಯ-ಬಾ-ಲ-ಕನು
ದಿವ್ಯ-ಭಾವ
ದಿವ್ಯ-ಭಾ-ವ-ಗ-ಳು-ಇ-ವು-ಗಳನ್ನು
ದಿವ್ಯ-ಭಾ-ವ-ದಲ್ಲಿ
ದಿವ್ಯ-ಭಾ-ವ-ದಲ್ಲೇ
ದಿವ್ಯ-ಲೀ-ಲೆ-ಯಾಗಿ
ದಿವ್ಯ-ವ-ಚ-ನ-ಗಳ
ದಿವ್ಯ-ವಾ-ಣಿಯ
ದಿವ್ಯ-ವಾ-ಣಿ-ಯನ್ನು
ದಿವ್ಯ-ವಾದ
ದಿವ್ಯ-ಶ-ಕ್ತಿ-ಯಿಂದ
ದಿವ್ಯ-ಶಿಶು
ದಿವ್ಯ-ಶಿ-ಶು-ವನ್ನೇ
ದಿವ್ಯ-ಸಂ-ದೇ-ಶ-ಗಳನ್ನು
ದಿವ್ಯ-ಸಂ-ದೇ-ಶ-ವನ್ನು
ದಿವ್ಯ-ಸ್ಪ-ರ್ಶ-ದಿಂ-ದಾಗಿ
ದಿವ್ಯಾ-ನಂದ
ದಿವ್ಯಾ-ನಂ-ದದ
ದಿವ್ಯಾ-ನಂ-ದ-ವನ್ನು
ದಿವ್ಯಾ-ನು-ಭವ
ದಿವ್ಯಾ-ನು-ಭ-ವ-ಗಳನ್ನು
ದಿವ್ಯಾ-ನು-ಭ-ವ-ಗಳನ್ನೆಲ್ಲ
ದಿವ್ಯಾ-ನು-ಭ-ವ-ಗ-ಳಾ-ದಾವು
ದಿವ್ಯಾ-ನು-ಭ-ವ-ಗ-ಳೆಲ್ಲ
ದಿವ್ಯಾ-ನು-ಭ-ವದ
ದಿವ್ಯಾ-ನು-ಭ-ವ-ದಲ್ಲಿ
ದಿವ್ಯಾ-ನು-ಭ-ವ-ವೊಂ-ದನ್ನು
ದಿವ್ಯಾ-ಮೃತ
ದಿವ್ಯಾ-ವ-ಸ್ಥೆ-ಯನ್ನು
ದಿವ್ಯೋ-ದ್ದೇ-ಶದ
ದಿಸೆ-ಯಲ್ಲಿ
ದೀಕ್ಷೆ
ದೀಕ್ಷೆಯ
ದೀಕ್ಷೆ-ಯನ್ನು
ದೀತೆಂಬ
ದೀನ
ದೀನ-ದ-ರಿ-ದ್ರರ
ದೀನ-ದ-ಲಿ-ತರ
ದೀನ-ನಾ-ಗು-ವ-ವ-ನಲ್ಲ
ದೀನ-ಬಂಧು
ದೀನ-ರೊಂ-ದಿಗೆ
ದೀನಾ-ರ್ತ-ರಿಗೆ
ದೀಪ
ದೀಪ-ಗಳನ್ನು
ದೀಪದ
ದೀಪಾ-ವ-ಳಿ-ಗ-ಳಂ-ತಹ
ದೀಪ್ತಿ
ದೀಪ್ತಿ-ಯಿಂದ
ದೀರ್ಘ
ದೀರ್ಘ-ಕಾಲ
ದೀರ್ಘ-ಕಾ-ಲದ
ದೀರ್ಘ-ಕಾ-ಲ-ದ-ವ-ರೆಗೆ
ದೀರ್ಘ-ದಂ-ಡ-ಪ್ರ-ಣಾಮ
ದೀರ್ಘ-ವಾಗಿ
ದೀರ್ಘಾ-ಲೋ-ಚನೆ
ದೀರ್ಘಾ-ವ-ಧಿಯ
ದುಂಡು-ಗಿನ
ದುಂಬಿ-ಗ-ಳಂತೆ
ದುಂಬಿ-ಸು-ತ್ತಿ-ದ್ದರು
ದುಃಖ
ದುಃಖ-ದಾರಿ-ದ್ರ್ಯ-ಗಳನ್ನು
ದುಃಖ-ಕ್ಕೀ-ಡಾ-ಗು-ತ್ತಾನೆ
ದುಃಖಕ್ಕೆ
ದುಃಖ-ಗಳನ್ನೂ
ದುಃಖ-ಗೊಂಡು
ದುಃಖದ
ದುಃಖ-ದಾರಿ-ದ್ರ್ಯ-ಗಳು
ದುಃಖ-ದಿಂದ
ದುಃಖ-ನಿ-ವಾ-ರಣೆ
ದುಃಖ-ಪ-ರಂ-ಪ-ರೆ-ಯಿಂದ
ದುಃಖ-ಪೂರ್ಣ
ದುಃಖ-ವನ್ನು
ದುಃಖ-ವಾಗ
ದುಃಖ-ವಾ-ಗಿದೆ
ದುಃಖ-ವಾ-ಗು-ತ್ತಿ-ದೆ-ಶ್ರೀ-ರಾ-ಮ-ಕೃ-ಷ್ಣರ
ದುಃಖ-ವಾ-ಯಿತು
ದುಃಖ-ವಾ-ಯಿ-ತೆಂ-ಬು-ದನ್ನು
ದುಃಖ-ವು-ಕ್ಕಿ-ಬಂತು
ದುಃಖವೂ
ದುಃಖ-ವೊ-ದ-ಗುವ
ದುಃಖ-ಸಂ-ಕ-ಟ-ಗ-ಳೆಲ್ಲ
ದುಃಖಿ-ತ-ರಾಗಿ
ದುಃಖಿ-ತ-ರಾ-ದ-ವ-ರನ್ನು
ದುಃಖಿಸು
ದುಃಖಿ-ಸು-ವುದನ್ನು
ದುಃಖೋ-ದ್ವೇಗ
ದುಃಸ್ಥಿ-ತಿ-ಯನ್ನು
ದುಗ-ಡ-ದು-ಮ್ಮಾ-ನ-ವ-ನ್ನಾ-ಗಲಿ
ದುಗುಡ
ದುಡಿದು
ದುಡಿ-ಯ-ಬಲ್ಲ
ದುಡಿ-ಯಲು
ದುಡು-ಕು-ಮಾ-ತಿ-ಗಾಗಿ
ದುಡು-ಕುವೆ
ದುನ್ನಿ-ಖಾನ್
ದುಪ್ಪ-ಟದ
ದುರ-ದೃಷ್ಟ
ದುರ-ದೃ-ಷ್ಟ-ಕರ
ದುರ-ದೃ-ಷ್ಟಕ್ಕೆ
ದುರ-ದೃ-ಷ್ಟದ
ದುರ-ಸ್ತಿ-ಗಾಗಿ
ದುರ-ಹಂ-ಕಾ-ರ-ವೆಂದು
ದುರಾ-ಚಾ-ರ-ಗಳ
ದುರು-ಗುಟ್ಟಿ
ದುರು-ಗು-ಟ್ಟಿ-ಕೊಂಡು
ದುರು-ದ್ದೇಶ
ದುರು-ಪ-ಯೋ-ಗಕ್ಕೆ
ದುರು-ಪ-ಯೋ-ಗ-ಪ-ಡಿ-ಸಿ-ಕೊ-ಳ್ಳು-ವಂ-ತಾ-ಗ-ದಿ-ರಲಿ
ದುರ್ಗತಿ
ದುರ್ಗಾ-ಪೂಜೆ
ದುರ್ಗಾ-ಪ್ರ-ಸಾದ
ದುರ್ಗಾ-ಪ್ರ-ಸಾ-ದನ
ದುರ್ಗಾ-ಪ್ರ-ಸಾ-ದ-ನನ್ನೇ
ದುರ್ಗುಣ
ದುರ್ಗೆ
ದುರ್ಗೆ-ಯರ
ದುರ್ಘ-ಟ-ನೆ-ಯನ್ನು
ದುರ್ಬಲ
ದುರ್ಬ-ಲ-ಗೊಂಡು
ದುರ್ಬ-ಲ-ತೆ-ಯನ್ನು
ದುರ್ಬ-ಲ-ನಾ-ಗ-ಲಿಲ್ಲ
ದುರ್ಬ-ಲ-ನಾ-ಗು-ತ್ತಿದ್ದೆ
ದುರ್ಬ-ಲ-ರಾಗಿ
ದುರ್ಬ-ಲ-ರಿಗೆ
ದುರ್ಬ-ಲ-ವಾ-ಗದು
ದುರ್ಬ-ಲ-ವೆಂದು
ದುರ್ಬಾ-ರನ್ನು
ದುರ್ಭಿ-ಕ್ಷದ
ದುರ್ಭೇ-ದ್ಯ-ವಾ-ಯಿತು
ದುರ್ವಿ-ನಿ-ಯೋಗ
ದುರ್ವ್ಯಯ
ದುಷ್ಟ
ದುಷ್ಟರ
ದುಷ್ಟ-ಶಿ-ಕ್ಷ-ಣ-ಶಿ-ಷ್ಟ-ರ-ಕ್ಷಣ
ದುಷ್ಟ-ಶಿ-ಕ್ಷ-ಣ-ಶಿ-ಷ್ಟ-ರ-ಕ್ಷ-ಣೆ-ಗಳನ್ನು
ದುಷ್ಟ-ಸಂ-ಪ್ರ-ದಾಯ
ದುಷ್ಪರಿ
ದುಷ್ಪ-ರಿ-ಣಾ-ಮ-ಗಳನ್ನೂ
ದುಷ್ಪ-ರಿ-ಣಾ-ಮ-ವನ್ನು
ದುಷ್ಪ-ರಿ-ಣಾ-ಮ-ವಾ-ಗ-ಬ-ಹು-ದೆಂಬ
ದುಸ್ಥಿತಿ
ದುಸ್ಥಿ-ತಿಯೇ
ದೂಡ-ಲ್ಪಟ್ಟ
ದೂಡು-ತ್ತಿತ್ತು
ದೂಡು-ತ್ತಿ-ದ್ದುದು
ದೂಡು-ತ್ತೇನೆ
ದೂಡುವೆ
ದೂಡೆಲೈ
ದೂತ-ನ-ನ್ನಾಗಿ
ದೂರ
ದೂರಕೆ
ದೂರಕ್ಕೆ
ದೂರದ
ದೂರ-ದ-ಡ-ವಿ-ಯೊ-ಳೆಲ್ಲಿ
ದೂರ-ದಲ್ಲಿ
ದೂರ-ದ-ಲ್ಲಿದ್ದು
ದೂರ-ದ-ಲ್ಲಿ-ದ್ದೇನೆ
ದೂರ-ದ-ಲ್ಲಿ-ರುವ
ದೂರ-ದಲ್ಲೇ
ದೂರ-ದಿಂ-ದಲೇ
ದೂರ-ದೂ-ರದ
ದೂರ-ವಾ-ಗ-ದಂತೆ
ದೂರ-ವಾ-ಗಿ-ಬಿ-ಡು-ತ್ತಿ-ದ್ದರು
ದೂರ-ವಾ-ಗಿ-ರ-ಲಿಲ್ಲ
ದೂರ-ವಾ-ಗು-ತ್ತದೆ
ದೂರ-ವಾ-ಗು-ತ್ತವೆ
ದೂರ-ವಾ-ಗು-ತ್ತಿ-ದ್ದಾ-ರೆ-ಶಕ್ತ
ದೂರ-ವಾ-ಗುವ
ದೂರ-ವಾದ
ದೂರ-ವಾ-ಯಿತು
ದೂರ-ವಿ-ದ್ದಾನು
ದೂರ-ವಿ-ದ್ದು-ಕೊಂಡು
ದೂರ-ವಿ-ರ-ಬೇಕು
ದೂರ-ವಿ-ರ-ಬೇ-ಕೆಂದು
ದೂರ-ವಿ-ರುವ
ದೂರವೇ
ದೂರ-ಸ್ಥ-ಳ-ಗ-ಳಿಗೆ
ದೂರಾಗಿ
ದೂರು-ಗಳು
ದೂಷ-ಣೀಯ
ದೃಗ್ಗೋ-ಚ-ರ-ವಾ-ಗು-ತ್ತಿದ್ದ
ದೃಢ
ದೃಢ-ಚಿ-ತ್ತ-ತೆ-ಯನ್ನೂ
ದೃಢ-ಚಿ-ತ್ತ-ರಾ-ದ-ವರ
ದೃಢತೆ
ದೃಢ-ನಂ-ಬಿ-ಕೆ-ಯಿದೆ
ದೃಢ-ನಿ-ರ್ಧಾರ
ದೃಢ-ನಿ-ರ್ಧಾ-ರದ
ದೃಢ-ನಿ-ರ್ಧಾ-ರ-ದಿಂದ
ದೃಢ-ನಿ-ರ್ಧಾ-ರ-ವನ್ನು
ದೃಢ-ನಿ-ಶ್ಚಯ
ದೃಢ-ನಿ-ಶ್ಚ-ಯದ
ದೃಢ-ನೆ-ಲೆ-ಯನ್ನು
ದೃಢ-ಪ-ಟ್ಟಿತು
ದೃಢ-ಪ-ಡಿ-ಸಿ-ಕೊಂಡ
ದೃಢ-ಪ-ಡಿ-ಸಿ-ಕೊಂಡೆ
ದೃಢ-ಪ-ಡಿ-ಸು-ತ್ತಿ-ದ್ದಾರೆ
ದೃಢ-ಬು-ದ್ಧಿ-ಯನ್ನೂ
ದೃಢ-ಬು-ದ್ಧಿ-ಯಿಂದ
ದೃಢ-ಮ-ನ-ಸ್ಕ-ನಾಗಿ
ದೃಢ-ವಾ-ಗಲು
ದೃಢ-ವಾಗಿ
ದೃಢ-ವಾ-ಗಿತ್ತು
ದೃಢ-ವಾ-ಗುತ್ತ
ದೃಢ-ವಾ-ಗು-ತ್ತದೆ
ದೃಢ-ವಾದ
ದೃಢ-ವಾ-ಯಿತು
ದೃಢ-ವಿ-ಶ್ವಾ-ಸ-ವಿತ್ತು
ದೃಢವೂ
ದೃಢ-ಸಂ-ಕಲ್ಪ
ದೃಢ-ಸಂ-ಕ-ಲ್ಪ-ವನ್ನು
ದೃಶ್ಯ
ದೃಶ್ಯ-ಇ-ವೆ-ಲ್ಲವೂ
ದೃಶ್ಯ-ಗಳ
ದೃಶ್ಯ-ಗಳನ್ನು
ದೃಶ್ಯ-ಗಳು
ದೃಶ್ಯ-ಗಳೂ
ದೃಶ್ಯದ
ದೃಶ್ಯ-ದಲ್ಲಿ
ದೃಶ್ಯ-ವನ್ನು
ದೃಶ್ಯವೂ
ದೃಷ್ಟಾಂ-ತ-ವಾಗಿ
ದೃಷ್ಟಿ
ದೃಷ್ಟಿ-ಕೋನ
ದೃಷ್ಟಿ-ಕೋ-ನ-ಕ್ಕಿಂತ
ದೃಷ್ಟಿ-ಕೋ-ನ-ಗಳಿಂದ
ದೃಷ್ಟಿ-ಕೋ-ನದ
ದೃಷ್ಟಿ-ಕೋ-ನ-ವನ್ನು
ದೃಷ್ಟಿಗೆ
ದೃಷ್ಟಿಯ
ದೃಷ್ಟಿ-ಯದು
ದೃಷ್ಟಿ-ಯನ್ನು
ದೃಷ್ಟಿ-ಯಲ್ಲಿ
ದೃಷ್ಟಿ-ಯಿಂದ
ದೃಷ್ಟಿ-ಯಿಂ-ದಲ್ಲ
ದೃಷ್ಟಿ-ಯಿಂ-ದ-ಸಾ-ಮಾ-ಜಿ-ಕ-ವಾಗಿ
ದೃಷ್ಟಿ-ಯಿ-ಟ್ಟು-ಕೊ-ಳ್ಳ-ಬೇಡ
ದೃಷ್ಟಿ-ಸಿದ
ದೆಂದ-ರೇ-ನರ್ಥ
ದೆಂಬಂತೆ
ದೆಯೆ
ದೆವ್ವದ
ದೆಸೆ-ಯಿಂ
ದೆಸೆ-ಯಿಂದ
ದೆಹಲಿ
ದೆಹ-ಲಿಗೆ
ದೆಹ-ಲಿಯ
ದೆಹ-ಲಿ-ಯನ್ನು
ದೆಹ-ಲಿ-ಯಲ್ಲಿ
ದೆಹ-ಲಿ-ಯ-ಲ್ಲೆಲ್ಲ
ದೆಹ-ಲಿ-ಯಲ್ಲೇ
ದೆಹ-ಲಿ-ಯಿಂದ
ದೇಗು-ಲ-ವನ್ನು
ದೇಙ-ಮ-ನೋ-ಬು-ದ್ಧಿ-ಗ-ಳಿಗೆ
ದೇಜಗೌ
ದೇದೀ-ಪ್ಯ-ಮಾ-ನ-ವಾದ
ದೇವ
ದೇವ-ದೇ-ವ-ತೆ-ಗಳ
ದೇವ-ದೇ-ವಿ-ಯರ
ದೇವ-ತಾ-ಮ-ನುಷ್ಯ
ದೇವತೆ
ದೇವ-ತೆ-ಗಳ
ದೇವ-ತೆ-ಗಳನ್ನೂ
ದೇವ-ತೆ-ಗ-ಳಿಂ-ದಲೇ
ದೇವ-ತೆ-ಗ-ಳೆಂದು
ದೇವ-ತೆಯ
ದೇವ-ದೇ-ವನ
ದೇವ-ದೇ-ವಿ-ಯರ
ದೇವ-ದೇ-ವಿ-ಯ-ರನ್ನು
ದೇವ-ದೇ-ವಿ-ಯ-ರಿಗೂ
ದೇವ-ನ-ಸ್ಥಾ-ನದ
ದೇವ-ನಾದ
ದೇವ-ನಿ-ದ್ದಾನೆ
ದೇವ-ಮಾ-ನ-ವ-ನೆಂದು
ದೇವ-ಮಾ-ನ-ವರ
ದೇವ-ಮಾ-ನ-ವರೇ
ದೇವ-ಮೂರ್ತಿ
ದೇವರ
ದೇವ-ರಂ-ತಹ
ದೇವ-ರಂತೆ
ದೇವ-ರ-ಡಿಗೆ
ದೇವ-ರ-ನಾ-ಮ-ಗಳ
ದೇವ-ರ-ನಾ-ಮ-ಗಳನ್ನು
ದೇವ-ರನ್ನು
ದೇವ-ರನ್ನೇ
ದೇವ-ರ-ಪೂಜೆ
ದೇವ-ರಾದ
ದೇವ-ರಿ-ಗಾಗಿ
ದೇವ-ರಿಗೆ
ದೇವ-ರಿ-ದ್ದಾನೆ
ದೇವರು
ದೇವ-ರು-ಧ-ರ್ಮ-ಸಾ-ಧನೆ
ದೇವರೆ
ದೇವ-ರೆಂ-ದರೆ
ದೇವರೇ
ದೇವ-ರೊ-ಬ್ಬ-ನಿ-ರ-ಬ-ಹುದು
ದೇವ-ರೊ-ಬ್ಬನೇ
ದೇವ-ಲೋ-ಕ-ದಲ್ಲಿ
ದೇವ-ಶರ್ಮಾ
ದೇವ-ಸ್ಥಾನ
ದೇವ-ಸ್ಥಾ-ನಕ್ಕೆ
ದೇವ-ಸ್ಥಾ-ನ-ಗಳ
ದೇವ-ಸ್ಥಾ-ನದ
ದೇವ-ಸ್ಥಾ-ನ-ದಲ್ಲಿ
ದೇವ-ಸ್ಥಾ-ನ-ವನ್ನು
ದೇವಾ
ದೇವಾ-ಲಯ
ದೇವಾ-ಲ-ಯಕ್ಕೆ
ದೇವಾ-ಲ-ಯ-ಗ-ಳಿಗೂ
ದೇವಾ-ಲ-ಯದ
ದೇವಾ-ಲ-ಯ-ವ-ನ್ನಾ-ದರೂ
ದೇವಾ-ಲ-ಯ-ವಿದೆ
ದೇವಾ-ಲ-ಯ-ವಿ-ರು-ವುದು
ದೇವಾ-ಲ-ಯವೂ
ದೇವಿ
ದೇವಿಗೆ
ದೇವಿಯ
ದೇವಿ-ಯನ್ನು
ದೇವಿ-ಯ-ಲ್ಲವೆ
ದೇವಿಯೂ
ದೇವಿ-ಯೆ-ದುರು
ದೇವೇಂ-ದ್ರ-ನಾಥ
ದೇಶ-ಕಾ-ಲ-ಗ-ಳೆಂಬ
ದೇಶ-ಕಾ-ಲ-ಗಳೇ
ದೇಶ-ಗಳ
ದೇಶ-ಗಳನ್ನು
ದೇಶ-ಪ್ರೇಮ
ದೇಶ-ವಿ-ದೇ-ಶ-ಗಳ
ದೇಶ-ಸಂ-ಚಾರ
ದೇಶ-ಸಂ-ಚಾ-ರಕ್ಕೆ
ದೇಶ-ಸಂ-ಚಾ-ರ-ವನ್ನು
ದೇಶಾಂ-ತರ
ದೇಹ
ದೇಹ-ಮ-ನ-ಸ್ಸು-ಗಳ
ದೇಹ-ಮ-ನ-ಸ್ಸು-ಗಳನ್ನು
ದೇಹಕ್ಕೆ
ದೇಹ-ತ್ಯಾಗ
ದೇಹದ
ದೇಹ-ದಲ್ಲಿ
ದೇಹ-ದಿಂದ
ದೇಹ-ಧ-ರಿಸಿ
ದೇಹ-ಪ್ರಜ್ಞೆ
ದೇಹ-ಬಂ-ಧ-ನ-ವನ್ನು
ದೇಹ-ಬುದ್ಧಿ
ದೇಹ-ಬು-ದ್ಧಿ-ಯಾ-ಗಲಿ
ದೇಹ-ಭಾ-ವ-ನೆಯೇ
ದೇಹ-ರ-ಕ್ಷ-ಣೆ-ಗಾಗಿ
ದೇಹ-ವನ್ನು
ದೇಹ-ವನ್ನೇ
ದೇಹ-ವಲ್ಲ
ದೇಹವು
ದೇಹ-ವೆಂಬ
ದೇಹ-ಸು-ಖ-ವನ್ನು
ದೇಹ-ಸ್ಥಿತಿ
ದೇಹ-ಸ್ಥಿ-ತಿ-ಯನ್ನು
ದೇಹಾ-ರೋಗ್ಯ
ದೇಹಾವ
ದೇಹಾ-ವ-ಸಾ-ನ-ದೊಂ-ದಿಗೇ
ದೈನಂ-ದಿನ
ದೈನ್ಯದ
ದೈವತ್ವ
ದೈವ-ತ್ವ-ದಲ್ಲಿ
ದೈವ-ತ್ವ-ವನ್ನು
ದೈವ-ದತ್ತ
ದೈವ-ಪ್ರ-ಪಂ-ಚಕ್ಕೂ
ದೈವ-ಭಕ್ತ
ದೈವ-ಭಕ್ತಿ
ದೈವ-ಭಕ್ತೆ
ದೈವ-ಶ್ರ-ದ್ಧೆ-ಯಿಂ-ದೊ-ಡ-ಗೂ-ಡಿದ
ದೈವ-ಶ್ರ-ದ್ಧೆ-ಯೆಂ-ಬುದು
ದೈವಿಕ
ದೈವಿ-ಕತೆ
ದೈವೀ
ದೈವೀ-ತೇ-ಜ-ಸ್ಸಿ-ನಿಂದ
ದೈವೀ-ಪು-ರು-ಷ-ನಾಗಿ
ದೈವೀ-ಭಾ-ವ-ದ-ಲ್ಲಿ-ರು-ವಾ-ಗ-ಲ-ಲ್ಲದೆ
ದೈವೀ-ಶ-ಕ್ತಿಗೆ
ದೈವೀ-ಶ-ಕ್ತಿ-ಸಂ-ಪ-ನ್ನ-ರಾದ
ದೈವೇಚ್ಛೆ
ದೈವೇ-ಚ್ಛೆ-ಯೆಂ-ದ-ರಿತು
ದೈಹಿಕ
ದೊಂದಿಗೆ
ದೊಂದು
ದೊಡನೆ
ದೊಡ್ಡ
ದೊಡ್ಡದು
ದೊಡ್ಡ-ದೊಂದು
ದೊಡ್ಡ-ದೊಡ್ಡ
ದೊಡ್ಡದೋ
ದೊಡ್ಡ-ಮ-ನುಷ್ಯ
ದೊಡ್ಡ-ವ-ನಾಗಿ
ದೊಡ್ಡ-ವ-ನಾ-ಗುತ್ತ
ದೊಡ್ಡ-ವ-ನಾದ
ದೊಡ್ಡ-ವ-ನಾ-ದಂ-ತೆಲ್ಲ
ದೊಡ್ಡ-ವ-ರು-ಚಿ-ಕ್ಕ-ವರು
ದೊಡ್ಡ-ವ-ಳಾ-ದಂತೆ
ದೊಡ್ಡ-ಸ್ತಿಕೆ
ದೊಡ್ಡ-ಸ್ತಿ-ಕೆ-ಯ-ವ-ರನ್ನು
ದೊಣ್ಣೆ
ದೊಣ್ಣೆ-ಯಿಂದ
ದೊಪ್ಪನೆ
ದೊಬ್ಬ-ರಿಗೆ
ದೊರ-ಕ-ಬ-ಹು-ದಾ-ದಂ-ತಹ
ದೊರ-ಕ-ಲಾ-ರದು
ದೊರ-ಕಲಿ
ದೊರ-ಕ-ಲಿಲ್ಲ
ದೊರ-ಕಿತ್ತು
ದೊರ-ಕಿದ
ದೊರ-ಕಿ-ದಂ-ತಾ-ಯಿತು
ದೊರ-ಕಿ-ದಾಗ
ದೊರ-ಕಿದೆ
ದೊರ-ಕಿವೆ
ದೊರ-ಕಿ-ಸಿ-ಕೊ-ಡ-ಲಾ-ರದೋ
ದೊರ-ಕಿ-ಸಿ-ಕೊ-ಡು-ತ್ತಾನೋ
ದೊರ-ಕಿ-ಸಿ-ಕೊ-ಡು-ವಂತೆ
ದೊರ-ಕು-ತ್ತದೆ
ದೊರ-ಕು-ತ್ತದೊ
ದೊರ-ಕುವ
ದೊರ-ಕು-ವಂ-ತಾ-ಗ-ಬೇಕು
ದೊರ-ಕು-ವಂ-ತಾ-ಯಿತು
ದೊರ-ಕು-ವಂ-ತಿಲ್ಲ
ದೊರ-ಕು-ವ-ವ-ರೆಗೆ
ದೊರೆ-ತಂ-ತಾ-ಯಿತು
ದೊರೆ-ತರೂ
ದೊರೆ-ತಾಗ
ದೊರೆ-ತಿದೆ
ದೊರೆ-ತಿ-ವೆ-ಮ-ಹಾ-ತ-ಪ-ಸ್ಸಿನ
ದೊರೆ-ಯ-ಲಿಲ್ಲ
ದೊರೆ-ಯಿತು
ದೊರೆ-ಯು-ತ್ತದೆ
ದೊರೆ-ಯು-ತ್ತಿತ್ತು
ದೊರೆ-ವ-ನ-ವನು
ದೋಣಿ
ದೋಣಿಗೆ
ದೋಣಿ-ಮ-ನೆಯ
ದೋಣಿಯ
ದೋಣಿ-ಯನ್ನು
ದೋಣಿ-ಯಲ್ಲಿ
ದೋಣಿ-ಯಲ್ಲೇ
ದೋಣಿ-ಯ-ವ-ರಿಗೆ
ದೋಣಿ-ಯ-ವರು
ದೋಣಿ-ಯಿಂದ
ದೋಪಾ-ದಿ-ಯಲ್ಲಿ
ದೋಷ
ದೋಷ-ಪೂ-ರ್ಣ-ವಾ-ಗಿ-ದ್ದರೆ
ದೋಷ-ವಿಲ್ಲ
ದೋಷ-ವೆಂದರೆ
ದೋಷಾ-ರೋ-ಪಣೆ
ದೌರ್ಜ-ನ್ಯ-ದಿಂದ
ದೌರ್ಬ-ಲ್ಯ-ಗ-ಳಿ-ರು-ತ್ತವೆ
ದೌರ್ಭಾ-ಗ್ಯ-ವಾ-ಗು-ತ್ತದೆ
ದ್
ದ್ಗಾತ್ರ-ದಲ್ಲಿ
ದ್ದಂತೆ
ದ್ದಕ್ಕೆ
ದ್ದನೋ
ದ್ದನ್ನು
ದ್ದರಿಂದ
ದ್ದರು
ದ್ದರೂ
ದ್ದರೆ
ದ್ದವನು
ದ್ದವನೇ
ದ್ದವರೆ-ಲ್ಲರೂ
ದ್ದಾನಂತೆ
ದ್ದಾನಲ್ಲ
ದ್ದಾನೆ
ದ್ದಾನೆ-ಜ-ನ-ಸೇ-ವೆಯೇ
ದ್ದಾರಲ್ಲ
ದ್ದಾರೆ
ದ್ದೀಯಾ
ದ್ದೀರೋ
ದ್ದುದ-ರಿಂದ
ದ್ದುದು
ದ್ದುವು
ದ್ದೇವೆ
ದ್ದೇಶ
ದ್ದೇಶ-ಕ್ಕಾಗಿ
ದ್ದೇಶಿಸಿ
ದ್ಯೋತ-ಕ-ವಾದ
ದ್ರಷ್ಟಾ-ರರು
ದ್ರಾಕ್ಷಿ
ದ್ವಂದ್ವ-ಕ್ಕೆ-ಡೆ-ಯಿ-ಲ್ಲ-ದಂ-ತಹ
ದ್ವಂದ್ವ-ಕ್ಕೆ-ಡೆ-ಯಿ-ಲ್ಲ-ದಂತೆ
ದ್ವಂದ್ವ-ಗಳು
ದ್ವಂದ್ವ-ರಾ-ಜ್ಯದ
ದ್ವಾರ
ದ್ವಾರ-ಕಾ-ದಾ-ಸರ
ದ್ವಾರ-ಕಾ-ದಾ-ಸ-ರೆಂ-ಬ-ವರ
ದ್ವಾರ-ದಿಂದ
ದ್ವಾರ-ಪಾ-ಲಕ
ದ್ವಾರ-ಪಾ-ಲ-ಕನ
ದ್ವಾರ-ಪಾ-ಲ-ಕ-ನೊಬ್ಬ
ದ್ವಿಗು-ಣ-ವಾ-ಗು-ತ್ತದೆ
ದ್ವಿತೀಯ
ದ್ವೇಷ
ದ್ವೇಷಾ-ಸೂಯೆ
ದ್ವೇಷಾ-ಸೂ-ಯೆ-ಗಳನ್ನು
ದ್ವೈತ
ದ್ವೈತ-ಅ-ದ್ವೈ-ತ-ವಿ-ಶಿ-ಷ್ಟಾ-ದ್ವೈತ
ದ್ವೈತ-ತ-ತ್ತ್ವ-ವನ್ನು
ಧಕ್ಕೆ
ಧಕ್ಕೆ-ಯಾ-ಗು-ವಂ-ತಹ
ಧಕ್ಕೆ-ಯಾ-ಯಿತು
ಧಗ-ಧ-ಗಿ-ಸು-ತ್ತಿದ್ದ
ಧನ
ಧನ-ಮದ
ಧನ-ವಂ-ತ-ರ-ನ್ನಾ-ಗಲ್ಲ
ಧನ-ವನ್ನೂ
ಧನವೂ
ಧನ-ವೆ-ಲ್ಲಿಂದ
ಧನ-ಸಂ-ಗ್ರಹ
ಧನ-ಸ-ಹಾ-ಯದ
ಧನ-ಸ-ಹಾ-ಯ-ವನ್ನೇ
ಧನ-ಸ-ಹಾ-ಯ-ವನ್ನೋ
ಧನಿ-ಕ-ನೆಂಬ
ಧನಿ-ಕರೇ
ಧನ್ಯ
ಧನ್ಯ-ತೆಯ
ಧನ್ಯ-ನಾ-ದ-ವ-ನಿಗೆ
ಧನ್ಯ-ರಾ-ಗು-ವು-ದ-ರಲ್ಲಿ
ಧನ್ಯ-ವಾದ
ಧನ್ಯ-ವಾ-ದ-ಗ-ಳ-ನ್ನ-ರ್ಪಿಸಿ
ಧಮ್ಮ-ಪ-ದದ
ಧರಿ-ಸ-ಬ-ಹುದು
ಧರಿ-ಸ-ಬ-ಹು-ದು-ಅ-ದ-ರಲ್ಲಿ
ಧರಿ-ಸ-ಬೇ-ಕಾದ್ದು
ಧರಿ-ಸ-ಲಿ-ಲ್ಲ-ವಲ್ಲ
ಧರಿಸಿ
ಧರಿ-ಸಿ-ಕೊಂಡು
ಧರಿ-ಸಿದ್ದ
ಧರಿ-ಸಿ-ರ-ಲಿಲ್ಲ
ಧರಿ-ಸಿ-ರುವ
ಧರಿ-ಸು-ತ್ತಿ-ದ್ದರು
ಧರಿ-ಸು-ತ್ತಿ-ರ-ಲಿಲ್ಲ
ಧರಿ-ಸುವ
ಧರಿ-ಸು-ವುದು
ಧರೆ-ಗಿ-ಳಿದ
ಧರೆ-ಗಿ-ಳಿದು
ಧರೆಗೆ
ಧರೆ-ಯಲ್ಲಿ
ಧರೆ-ಯೊ-ಳಿಲ್ಲ
ಧರ್ಮ
ಧರ್ಮ-ಇ-ವು-ಗಳ
ಧರ್ಮಕ್ಕೆ
ಧರ್ಮ-ಗಳ
ಧರ್ಮ-ಗಳನ್ನು
ಧರ್ಮ-ಗ-ಳಿಂ-ದಲೂ
ಧರ್ಮ-ಗಳೂ
ಧರ್ಮ-ಗ-ಳೆ-ಲ್ಲವೂ
ಧರ್ಮ-ಗು-ರು-ಗಳ
ಧರ್ಮ-ಗು-ರು-ಗ-ಳಂತೆ
ಧರ್ಮ-ಗು-ರು-ಗಳಲ್ಲಿ
ಧರ್ಮ-ಗು-ರು-ವಿನ
ಧರ್ಮ-ಗ್ಲಾ-ನಿ-ಯಾ-ಗು-ತ್ತಿ-ದ್ದರೆ
ಧರ್ಮದ
ಧರ್ಮ-ದಲ್ಲಿ
ಧರ್ಮ-ದ-ವರೂ
ಧರ್ಮ-ದೊಂ-ದಿಗೆ
ಧರ್ಮ-ಪ-ರಿ-ಪಾ-ಲನೆ
ಧರ್ಮ-ಪ್ರ-ಚಾ-ರ-ಕರು
ಧರ್ಮ-ಪ್ರ-ಚಾ-ರ-ಕ-ರೊಂ-ದಿಗೆ
ಧರ್ಮ-ಪ್ರ-ಚಾ-ರ-ಕ್ಕಾಗಿ
ಧರ್ಮ-ಪ್ರ-ಸಾರ
ಧರ್ಮ-ಪ್ರ-ಸಾ-ರ
ಧರ್ಮ-ಪ್ರ-ಸಾ-ರ-ಕ-ನ-ನ್ನಾಗಿ
ಧರ್ಮ-ಪ್ರ-ಸಾ-ರ-ಕಾ-ರ್ಯವು
ಧರ್ಮ-ಪ್ರ-ಸಾ-ರ-ಕ್ಕಂತೂ
ಧರ್ಮ-ಪ್ರ-ಸಾ-ರಕ್ಕೆ
ಧರ್ಮ-ಪ್ರ-ಸಾ-ರದ
ಧರ್ಮ-ಬೋ-ಧನೆ
ಧರ್ಮ-ಬೋ-ಧ-ನೆ-ಯನ್ನು
ಧರ್ಮ-ಬೋ-ಧ-ನೆ-ಯೆಂ-ದರೆ
ಧರ್ಮ-ರ-ಕ್ಷ-ಣ-ಕೋ-ವಿದ
ಧರ್ಮ-ವನ್ನು
ಧರ್ಮ-ವಲ್ಲ
ಧರ್ಮ-ವಾದ
ಧರ್ಮವು
ಧರ್ಮವೂ
ಧರ್ಮ-ವೆಂ-ಬುದು
ಧರ್ಮ-ವೆ-ನ್ನು-ವುದು
ಧರ್ಮ-ಸಂ-ಸ್ಥಾ-ಪ-ನಾ-ರ್ಥಾಯ
ಧರ್ಮ-ಸಂ-ಸ್ಥಾ-ಪನೆ
ಧರ್ಮ-ಸಂ-ಸ್ಥಾ-ಪ-ನೆಯ
ಧರ್ಮ-ಸಂ-ಸ್ಥೆ-ಯನ್ನು
ಧರ್ಮ-ಸೂಕ್ಷ್ಮ
ಧರ್ಮಸ್ಯ
ಧರ್ಮಾಸ್ತೇ
ಧಾಂಡಿಗ
ಧಾಂಡಿ-ಗನೇ
ಧಾಟಿ-ಯಿಂ-ದಲೇ
ಧಾಮಕ್ಕೆ
ಧಾರ-ಣ-ಶ-ಕ್ತಿಯೂ
ಧಾರಣೆ
ಧಾರ-ಳ-ವಾ-ಗಿಯೇ
ಧಾರಾ-ಕಾ-ರ-ವಾಗಿ
ಧಾರಾ-ಳದ
ಧಾರಾ-ಳ-ವಾಗಿ
ಧಾರಾಳಿ
ಧಾರೆ-ಯಾಗಿ
ಧಾರೆ-ಯಿಂದ
ಧಾರೆ-ಯೆ-ರೆದು
ಧಾರೆ-ಯೆ-ರೆ-ಯಲು
ಧಾರೆ-ಯೆ-ರೆ-ಯಲೂ
ಧಾರ್ಮಿಕ
ಧಾವಿಸಿ
ಧಾವಿ-ಸಿದ
ಧಾವಿ-ಸಿ-ದರು
ಧಾವಿ-ಸಿ-ಬಂ-ದರು
ಧಾವಿ-ಸಿ-ಬಿ-ಡು-ತ್ತಿತ್ತು
ಧಾವಿ-ಸುವ
ಧಾವಿ-ಸು-ವುದು
ಧಿಕ್ಕ-ರಿಸಿ
ಧಿಮಾಕು
ಧಿಸಿ
ಧೀರ
ಧೀರ-ಕ-ಶ್ಚಿ-ದ್ಧೀರಃ
ಧೀರ-ಗಂ-ಭೀರ
ಧೀರ-ಗಂ-ಭೀ-ರ-ವಾಗಿ
ಧೀರ-ತ-ನವೂ
ಧೀರ-ತ-ನವೇ
ಧೀರ-ತೆಯ
ಧೀರ-ತೆ-ಯಿಂದ
ಧೀರ-ನ-ಲ್ಲವೆ
ಧೀರ-ನಾ-ಗಿರು
ಧೀರ-ನಾ-ಗು-ವುದು
ಧೀರನೆ
ಧೀರ-ನೊ-ಬ್ಬ-ನಿ-ದ್ದಾ-ನೆ-ನ-ರೇಂ-ದ್ರ-ನಾಥ
ಧೀರ-ಪ್ರ-ಯತ್ನ
ಧೀರ-ರ-ನ್ನಾಗಿ
ಧೀರರೇ
ಧುನಿ
ಧುನಿಯ
ಧುನಿ-ಯನ್ನು
ಧುನಿಯು
ಧುಮು-ಕಿ-ದ್ದ-ರಿಂದ
ಧುಮು-ಧು-ಮಿಸಿ
ಧುರೀ-ಣರು
ಧುರೀ-ಣರೂ
ಧುರೀ-ಣ-ರೆಲ್ಲ
ಧೂಪ-ಗಳನ್ನು
ಧೂಪ-ದೀ-ಪಾದಿ
ಧೂಮ-ಪಾನ
ಧೂರ್ತ-ಚೋ-ರರು
ಧೃತ-ರಾ-ಷ್ಟ್ರ-ರಾ-ಗಿ-ಬಿ-ಡು-ತ್ತೇವೆ
ಧೃತಿ-ಗೆ-ಡದೆ
ಧೃತಿ-ಗೆ-ಡ-ಲಿಲ್ಲ
ಧೆರ್ಯ
ಧೇಂಡಿ
ಧೈರ್ಯ
ಧೈರ್ಯ-ಔ-ದಾ-ರ್ಯ-ಗಳು
ಧೈರ್ಯ-ದಿಂದ
ಧೈರ್ಯ-ದಿಂ-ದಿ-ದ-ನೆ-ಲ್ಲ-ರಾ-ಲಿಸೆ
ಧೈರ್ಯ-ದೊಂ-ದಿಗೆ
ಧೈರ್ಯ-ಮಾಡಿ
ಧೈರ್ಯ-ವನ್ನೂ
ಧೈರ್ಯ-ವಾಗಿ
ಧೈರ್ಯ-ವಾ-ಗಿರಿ
ಧೈರ್ಯ-ವಿ-ದ್ದುದು
ಧೈರ್ಯ-ಶಾ-ಲಿನಿ
ಧೋತಿ
ಧೋತಿ-ಯ-ನ್ನಲ್ಲ
ಧೋತಿ-ಯನ್ನು
ಧೋತಿ-ಯನ್ನೇ
ಧೋರಣೆ
ಧೋರ-ಣೆ-ಪ್ರ-ಯ-ತ್ನ-ಗಳು
ಧೋರ-ಣೆ-ಗಳಿಂದ
ಧೋರ-ಣೆ-ಯೇ-ನೆಂ-ದ-ರೆ-ಅ-ವರು
ಧೋರೆ
ಧ್ಯಾನ
ಧ್ಯಾನ-ಅ-ಧ್ಯ-ಯ-ನ-ಭ-ಜನೆ
ಧ್ಯಾನ-ಜಪ
ಧ್ಯಾನ-ಜ-ಪ-ಪ್ರಾ-ರ್ಥ-ನೆ-ಭ-ಜ-ನೆ-ಗಳಲ್ಲಿ
ಧ್ಯಾನ-ಜ-ಪ-ಪ್ರಾ-ರ್ಥ-ನೆ-ಗಳು
ಧ್ಯಾನ-ಪ್ರಾ-ರ್ಥ-ನೆ-ಶಾ-ಸ್ತ್ರಾ-ಧ್ಯ-ಯ-ನ-ಗಳಲ್ಲಿ
ಧ್ಯಾನ-ಭ-ಜ-ನೆ-ಗಳಲ್ಲಿ
ಧ್ಯಾನ-ಸ-ಮಾ-ಧಿಯ
ಧ್ಯಾನಕ್ಕೆ
ಧ್ಯಾನ-ಕ್ರ-ಮ-ವನ್ನು
ಧ್ಯಾನ-ಗಳಲ್ಲಿ
ಧ್ಯಾನ-ಗಳೇ
ಧ್ಯಾನ-ಜಪ
ಧ್ಯಾನದ
ಧ್ಯಾನ-ದಲ್ಲಿ
ಧ್ಯಾನ-ದಲ್ಲೇ
ಧ್ಯಾನ-ದಿಂದ
ಧ್ಯಾನ-ದಿಂ-ದೆದ್ದು
ಧ್ಯಾನ-ನಿ-ರ-ತ-ನಾ-ಗ-ಬೇಕು
ಧ್ಯಾನ-ನಿ-ರ-ತ-ನಾಗಿ
ಧ್ಯಾನ-ನಿ-ರ-ತ-ನಾ-ಗಿ-ಬಿ-ಡು-ತ್ತಿದ್ದ
ಧ್ಯಾನ-ನಿ-ರ-ತ-ನಾದ
ಧ್ಯಾನ-ನಿ-ರ-ತ-ರಾ-ಗಿ-ದ್ದರು
ಧ್ಯಾನ-ನಿ-ರ-ತ-ರಾ-ಗಿ-ರ-ಲಾ-ರಂ-ಭಿ-ಸಿ-ದರು
ಧ್ಯಾನ-ಭಾವ
ಧ್ಯಾನ-ಮಗ್ನ
ಧ್ಯಾನ-ಮ-ಗ್ನ-ನಾಗಿ
ಧ್ಯಾನ-ಮ-ಗ್ನ-ನಾ-ಗಿದ್ದ
ಧ್ಯಾನ-ಮ-ಗ್ನ-ನಾದ
ಧ್ಯಾನ-ಮ-ಗ್ನ-ರಾಗಿ
ಧ್ಯಾನ-ಮ-ಗ್ನ-ರಾ-ಗಿ-ದ್ದಾರೆ
ಧ್ಯಾನ-ಮ-ಗ್ನ-ರಾ-ಗಿ-ರ-ತೊ-ಡ-ಗಿ-ದರು
ಧ್ಯಾನ-ಮಾ-ಡಲು
ಧ್ಯಾನ-ಮಾಡು
ಧ್ಯಾನ-ಮಾ-ಡು-ತ್ತಿದ್ದ
ಧ್ಯಾನ-ಲೀನ
ಧ್ಯಾನ-ಲೀ-ನ-ನಾ-ಗಲು
ಧ್ಯಾನ-ಲೀ-ನ-ನಾಗಿ
ಧ್ಯಾನ-ಲೀ-ನ-ವಾ-ಗಿ-ಬಿ-ಟ್ಟಿತು
ಧ್ಯಾನ-ವನ್ನು
ಧ್ಯಾನ-ವೆಂ-ಬುದು
ಧ್ಯಾನ-ಸಿದ್ಧ
ಧ್ಯಾನಾ-ದಿ-ಗಳಲ್ಲಿ
ಧ್ಯಾನಾ-ನಂ-ದ-ಕ್ಕಿಂ-ತಲೂ
ಧ್ಯಾನಾ-ನಂ-ದದ
ಧ್ಯಾನಾ-ನಂ-ದ-ದಲ್ಲಿ
ಧ್ಯಾನಾ-ಭ್ಯಾಸ
ಧ್ಯಾನಾ-ವಸ್ಥೆ
ಧ್ಯಾನಿ-ಸಲಿ
ಧ್ಯಾನಿ-ಸಿದ
ಧ್ಯಾನಿ-ಸಿದ್ಧ
ಧ್ಯಾನಿ-ಸುತ್ತ
ಧ್ಯಾಯ
ಧ್ಯೇಯ
ಧ್ಯೇಯ-ದಿಂದ
ಧ್ಯೇಯ-ಮಂತ್ರ
ಧ್ಯೇಯ-ವನ್ನು
ಧ್ಯೇಯ-ವಾ-ಗಲಿ
ಧ್ಯೇಯಾ-ದ-ರ್ಶ-ಗಳನ್ನು
ಧ್ಯೇಯೋ-ದ್ದೇ-ಶ-ಗ-ಳಿಗೆ
ಧ್ರುವ-ತಾ-ರೆ-ಯೆಂ-ದರೆ
ಧ್ವನಿ
ಧ್ವನಿ-ಗಳು
ಧ್ವನಿ-ಯಲ್ಲಿ
ಧ್ವನಿ-ಸ-ಬೇ-ಕಾ-ಗಿ-ರು-ವುದು
ನ
ನಂಟ-ನಾದ
ನಂಟು
ನಂತರ
ನಂತ-ರವೂ
ನಂತೆ
ನಂತೆಯೇ
ನಂದ
ನಂದರ
ನಂದ-ರಾಗಿ
ನಂದ-ರಿಗೆ
ನಂದರು
ನಂಬ-ಬೇಕಾ
ನಂಬ-ಬೇಡಿ
ನಂಬ-ಲ-ನ-ರ್ಹ-ವೆಂದು
ನಂಬ-ಲಾ-ಗದ
ನಂಬ-ಲಾ-ಗ-ದಂ-ತಹ
ನಂಬ-ಲಿಲ್ಲ
ನಂಬಲೇ
ನಂಬಿ
ನಂಬಿಕೆ
ನಂಬಿ-ಕೆ
ನಂಬಿ-ಕೆ-ಗಳ
ನಂಬಿ-ಕೆ-ಗಳನ್ನು
ನಂಬಿ-ಕೆ-ಗಳು
ನಂಬಿ-ಕೆಯ
ನಂಬಿ-ಕೆ-ಯನ್ನು
ನಂಬಿ-ಕೆ-ಯ-ನ್ನೆಲ್ಲ
ನಂಬಿ-ಕೆ-ಯಲ್ಲ
ನಂಬಿ-ಕೆ-ಯಲ್ಲಿ
ನಂಬಿ-ಕೆ-ಯಾ-ಗ-ಲಿ-ಲ್ಲವೆ
ನಂಬಿ-ಕೆ-ಯಾ-ಯಿತು
ನಂಬಿ-ಕೆ-ಯಿ-ಟ್ಟಿ-ದ್ದಾರೆ
ನಂಬಿ-ಕೆ-ಯಿತ್ತು
ನಂಬಿ-ಕೆ-ಯಿ-ರ-ಲಿಲ್ಲ
ನಂಬಿ-ಕೆ-ಯಿಲ್ಲ
ನಂಬಿ-ಕೆ-ಯಿ-ಲ್ಲ-ದವ
ನಂಬಿ-ಕೆ-ಯಿ-ಲ್ಲ-ವಲ್ಲ
ನಂಬಿ-ಕೆಯೇ
ನಂಬಿ-ಕೊಂ-ಡಿದ್ದ
ನಂಬಿ-ದರು
ನಂಬಿ-ದ-ವ-ರೆಂ-ದರೆ
ನಂಬಿ-ದ್ದ-ರಿಂ-ದಲೇ
ನಂಬಿ-ದ್ದರೂ
ನಂಬಿದ್ದಾ
ನಂಬಿ-ದ್ದಾರೆ
ನಂಬಿ-ಬಿ-ಟ್ಟರೆ
ನಂಬಿ-ಬಿ-ಡು-ತ್ತಾರೆ
ನಂಬಿ-ಬಿ-ಡೋ-ಣವೇ
ನಂಬಿ-ಯೇನು
ನಂಬು
ನಂಬು-ತ್ತಾರೆ
ನಂಬು-ತ್ತಿ-ರ-ಲಿಲ್ಲ
ನಂಬು-ತ್ತೀಯಾ
ನಂಬು-ವ-ವ-ನಲ್ಲ
ನಂಬು-ವ-ವನೇ
ನಂಬು-ವುದು
ನಕ್ಕ
ನಕ್ಕ-ದ್ದಲ್ಲ
ನಕ್ಕದ್ದು
ನಕ್ಕರು
ನಕ್ಕು
ನಕ್ಕು-ಬಿಟ್ಟ
ನಕ್ಕು-ಬಿ-ಟ್ಟರು
ನಕ್ಷತ್ರ
ನಕ್ಷ-ತ್ರ-ಗಳು
ನಕ್ಷ-ತ್ರ-ಗ-ಳೆ-ಲ್ಲವೂ
ನಗ-ಬ-ಹುದು
ನಗರ
ನಗ-ರಕ್ಕೆ
ನಗ-ರಕ್ಕೋ
ನಗ-ರ-ಗಳು
ನಗ-ರದ
ನಗ-ರ-ದಲ್ಲಿ
ನಗ-ರ-ದಿಂದ
ನಗ-ರ-ವಾ-ಗಲಿ
ನಗ-ರವು
ನಗ-ಲಾ-ರಂ-ಭಿ-ಸಿ-ದರು
ನಗಿಸಿ
ನಗಿ-ಸು-ತ್ತಾನೆ
ನಗಿ-ಸು-ತ್ತಿದ್ದ
ನಗಿ-ಸು-ತ್ತಿ-ರುವ
ನಗು
ನಗುತ್ತ
ನಗು-ತ್ತಲೇ
ನಗು-ತ್ತಾನೆ
ನಗು-ತ್ತಿದ್ದ
ನಗು-ನ-ಗುತ್ತ
ನಗು-ನ-ಗು-ತ್ತಲೇ
ನಗು-ವನ್ನು
ನಗು-ವ-ರು-ಳಿ-ದ-ವರೆ-ಲ್ಲರೂ
ನಗು-ವಿ-ನಲ್ಲಿ
ನಗುವು
ನಗು-ವುದನ್ನು
ನಗು-ವು-ದಿಲ್ಲ
ನಗುವೋ
ನಗೆ
ನಗೆ-ಗ-ಡ-ಲಿ-ನಲ್ಲಿ
ನಗೆಯ
ನಗೆ-ಯನ್ನು
ನಗೆ-ಯಿಂದ
ನಗೋರ್
ನಗ್ನ-ಸಂನ್ಯಾಸಿ
ನಟ-ಕೃಷ್ಣ
ನಟನೆ
ನಟ-ರಾ-ಜ-ನಾ-ಟ್ಯ-ವಾ-ಡು-ವ-ವರೆ-ಲ್ಲರ
ನಡತೆ
ನಡ-ತೆ-ಯ-ಲ್ಲೇ-ನಾ-ದರೂ
ನಡ-ವ-ಳಿಕೆ
ನಡ-ವ-ಳಿ-ಕೆಯ
ನಡ-ವ-ಳಿ-ಕೆ-ಯ-ಗಳನ್ನು
ನಡ-ವ-ಳಿ-ಕೆಯೇ
ನಡವೆ
ನಡ-ಸಿ-ಕೊಂಡು
ನಡಿ
ನಡಿಗೆ
ನಡಿ-ಗೆಯ
ನಡು
ನಡು-ಗ-ದಿಹ
ನಡು-ಗಿತು
ನಡು-ಗಿ-ಸುವ
ನಡುಗು
ನಡು-ಗು-ತ್ತಿತ್ತು
ನಡು-ಗು-ತ್ತಿದೆ
ನಡು-ಗು-ತ್ತಿ-ರು-ವಾ-ಗಲೂ
ನಡು-ಗುವ
ನಡು-ದಾರಿ-ಯಲ್ಲಿ
ನಡು-ರ-ಸ್ತೆ-ಯಲ್ಲಿ
ನಡು-ವಣ
ನಡು-ವಿನ
ನಡುವೆ
ನಡು-ವೆಯೂ
ನಡೆ
ನಡೆದ
ನಡೆ-ದಂ-ತೆಲ್ಲ
ನಡೆ-ದ-ದ್ದಲ್ಲ
ನಡೆ-ದದ್ದು
ನಡೆ-ದರು
ನಡೆ-ದರೆ
ನಡೆ-ದಾಗ
ನಡೆ-ದಾ-ಡು-ತ್ತಿದ್ದ
ನಡೆ-ದಾ-ಡು-ತ್ತಿ-ದ್ದಂತೆ
ನಡೆ-ದಿದೆ
ನಡೆ-ದಿ-ರ-ಬ-ಹುದು
ನಡೆ-ದಿ-ರ-ಲಿಲ್ಲ
ನಡೆದು
ನಡೆ-ದು-ಕೊಂಡ
ನಡೆ-ದು-ಕೊಂ-ಡ-ರಲ್ಲ
ನಡೆ-ದು-ಕೊಂ-ಡರು
ನಡೆ-ದು-ಕೊಂ-ಡರೆ
ನಡೆ-ದು-ಕೊಂ-ಡಾ-ಗ-ಲೆಲ್ಲ
ನಡೆ-ದು-ಕೊಂಡು
ನಡೆ-ದು-ಕೊಂಡೇ
ನಡೆ-ದು-ಕೊಂ-ಡೇನು
ನಡೆ-ದು-ಕೊಳ್ಳ
ನಡೆ-ದು-ಕೊ-ಳ್ಳತ್ತ
ನಡೆ-ದು-ಕೊ-ಳ್ಳಲು
ನಡೆ-ದು-ಕೊ-ಳ್ಳುವ
ನಡೆ-ದು-ಕೊ-ಳ್ಳು-ವುದನ್ನು
ನಡೆ-ದು-ದ-ನ್ನೆಲ್ಲ
ನಡೆ-ದು-ಬಿ-ಡ-ಬೇಕು
ನಡೆ-ದು-ಬಿ-ಡು-ತ್ತದೆ
ನಡೆ-ದುವು
ನಡೆ-ದು-ಹೋ-ಗಿ-ಬಿ-ಟ್ಟಿತು
ನಡೆ-ದು-ಹೋ-ಗು-ತ್ತಿತ್ತು
ನಡೆ-ದು-ಹೋ-ಗು-ತ್ತಿ-ರು-ವಾಗ
ನಡೆ-ದು-ಹೋದ
ನಡೆ-ದು-ಹೋ-ದದ್ದು
ನಡೆ-ದು-ಹೋ-ಯಿತು
ನಡೆ-ದು-ಹೋ-ಯಿತೋ
ನಡೆದೇ
ನಡೆ-ದೇ-ಬಿ-ಟ್ಟರು
ನಡೆ-ದೇ-ಹೋ-ಯಿತು
ನಡೆಯ
ನಡೆ-ಯ-ದಂತೆ
ನಡೆ-ಯ-ಬ-ಹುದು
ನಡೆ-ಯ-ಬ-ಹುದೆ
ನಡೆ-ಯ-ಬಾ-ರದು
ನಡೆ-ಯ-ಬೇ-ಕಾದ
ನಡೆ-ಯ-ಬೇಕು
ನಡೆ-ಯ-ಲಾರ
ನಡೆ-ಯಲಿ
ನಡೆ-ಯ-ಲಿಲ್ಲ
ನಡೆ-ಯಲೇ
ನಡೆ-ಯಿತು
ನಡೆ-ಯಿರಿ
ನಡೆಯು
ನಡೆ-ಯುತ್ತ
ನಡೆ-ಯು-ತ್ತದೆ
ನಡೆ-ಯು-ತ್ತ-ದೆಯೆ
ನಡೆ-ಯು-ತ್ತ-ಲಿದ್ದು
ನಡೆ-ಯು-ತ್ತಲೇ
ನಡೆ-ಯು-ತ್ತವೆ
ನಡೆ-ಯು-ತ್ತಿತ್ತು
ನಡೆ-ಯು-ತ್ತಿದೆ
ನಡೆ-ಯು-ತ್ತಿದ್ದ
ನಡೆ-ಯು-ತ್ತಿ-ದ್ದರೂ
ನಡೆ-ಯು-ತ್ತಿ-ದ್ದಾನೆ
ನಡೆ-ಯು-ತ್ತಿ-ದ್ದುವು
ನಡೆ-ಯು-ತ್ತಿ-ರು-ವಾಗ
ನಡೆ-ಯು-ತ್ತಿ-ವೆಯೋ
ನಡೆ-ಯು-ವವ
ನಡೆ-ಯು-ವು-ದಕ್ಕೇ
ನಡೆ-ಯು-ವುದೇ
ನಡೆಸ
ನಡೆ-ಸ-ಬೇಕು
ನಡೆ-ಸಲು
ನಡೆಸಿ
ನಡೆ-ಸಿ-ಕೊಂಡು
ನಡೆ-ಸಿ-ಕೊ-ಡ-ದಿ-ದ್ದಾ-ರೆಯೆ
ನಡೆ-ಸಿ-ಕೊ-ಳ್ಳ-ಬೇ-ಕೆಂ-ಬುದು
ನಡೆ-ಸಿದ
ನಡೆ-ಸಿ-ದರು
ನಡೆ-ಸಿ-ದ-ವನು
ನಡೆ-ಸಿ-ದ-ವ-ರಿಗೆ
ನಡೆ-ಸಿ-ದಾಗ
ನಡೆ-ಸಿ-ದ್ದರು
ನಡೆ-ಸಿ-ದ್ದಾನೆ
ನಡೆ-ಸಿ-ದ್ದಾರೆ
ನಡೆ-ಸಿದ್ದು
ನಡೆಸು
ನಡೆ-ಸುತ್ತ
ನಡೆ-ಸು-ತ್ತಲೇ
ನಡೆ-ಸು-ತ್ತಾರೆ
ನಡೆ-ಸು-ತ್ತಿತ್ತು
ನಡೆ-ಸು-ತ್ತಿದ್ದ
ನಡೆ-ಸು-ತ್ತಿ-ದ್ದಾ-ನಲ್ಲ
ನಡೆ-ಸು-ತ್ತಿ-ದ್ದಾನೆ
ನಡೆ-ಸು-ತ್ತಿ-ದ್ದೆವು
ನಡೆ-ಸು-ತ್ತಿ-ರು-ವಂ-ತಿತ್ತು
ನಡೆ-ಸುವ
ನಡೆ-ಸು-ವಂ-ತಹ
ನಡೆ-ಸು-ವಂ-ತಾ-ದರು
ನಡೆ-ಸು-ವಂತೆ
ನಡೆ-ಸು-ವಾಗ
ನಡೆ-ಸು-ವು-ದ-ರಲ್ಲಿ
ನಡೆ-ಸು-ವು-ದಾಗಿ
ನಡೆ-ಸು-ವು-ದಾ-ದರೂ
ನದಿ-ಗಳ
ನದಿ-ಗಳಲ್ಲಿ
ನದಿ-ಗಳು
ನದಿಯ
ನದಿ-ಯಂತೆ
ನದಿ-ಯನ್ನು
ನದಿ-ಯಲ್ಲಿ
ನದಿ-ಯೊ-ಳಗೆ
ನದೀ-ತೀ-ರದ
ನದೀ-ತೀ-ರ-ದಲ್ಲಿ
ನನ-ಗಂತೂ
ನನ-ಗ-ನಿ-ಸಿತು
ನನ-ಗ-ನಿ-ಸಿ-ದರೆ
ನನ-ಗ-ನಿ-ಸು-ತ್ತಿದೆ
ನನ-ಗ-ನ್ನಿ-ಸು-ತ್ತಿದೆ
ನನ-ಗ-ವೆಲ್ಲ
ನನ-ಗಾಗಿ
ನನ-ಗಾದ
ನನ-ಗಿ-ದೇನು
ನನ-ಗಿ-ನ್ನೇನೂ
ನನ-ಗಿ-ರುವ
ನನ-ಗಿಲ್ಲ
ನನ-ಗಿಷ್ಟ
ನನ-ಗೀಗ
ನನಗೂ
ನನಗೆ
ನನ-ಗೇಕೆ
ನನ-ಗೇ-ನಾ-ಗ-ಬೇ-ಕಾ-ಗಿದೆ
ನನ-ಗೇ-ನಾ-ಗು-ತ್ತಿದೆ
ನನ-ಗೇ-ನಾ-ದರೂ
ನನ-ಗೇನು
ನನ-ಗೇನೂ
ನನ-ಗೇನೋ
ನನ-ಗೊಂದು
ನನ-ಗೊಂದೂ
ನನ-ಗೊಬ್ಬ
ನನಗೋ
ನನ-ಸಾ-ಗು-ವು-ದ-ರಲ್ಲಿ
ನನ್ನ
ನನ್ನಂ-ತಹ
ನನ್ನಂ-ಥ-ವ-ನಿಗೆ
ನನ್ನಂ-ಥ-ವನು
ನನ್ನ-ದಾ-ಗಿದೆ
ನನ್ನದು
ನನ್ನದೇ
ನನ್ನನ್ನು
ನನ್ನನ್ನೂ
ನನ್ನನ್ನೇ
ನನ್ನ-ನ್ನೇ-ನಾ-ದರೂ
ನನ್ನ-ನ್ನೊಮ್ಮೆ
ನನ್ನಲ್ಲಿ
ನನ್ನ-ಲ್ಲೊಂದು
ನನ್ನ-ವ-ನಾದ
ನನ್ನ-ಷ್ಟಕ್ಕೆ
ನನ್ನಿಂದ
ನನ್ನಿಂ-ದ-ಲಾ-ದರೂ
ನನ್ನಿಂ-ದಲೇ
ನನ್ನಿಂ-ದಾಗಿ
ನನ್ನಿಂ-ದೇ-ನಾ-ಗ-ಬೇಕೋ
ನನ್ನಿಂ-ದೇನೂ
ನನ್ನಿಚ್ಛೆ
ನನ್ನಿ-ಚ್ಛೆ-ಯಂ-ತೆಯೇ
ನನ್ನಿಯ
ನನ್ನಿ-ಯ-ರಿ-ವಾ-ನಂ-ದ-ವಾ-ಹಿ-ನಿ-ಯೆಲ್ಲಿ
ನನ್ನಿ-ಷ್ಟ-ದಂ-ತೆಯೇ
ನನ್ನು
ನನ್ನೆ-ಡೆಗೆ
ನನ್ನೆ-ದರು
ನನ್ನೊಂ-ದಿಗೆ
ನನ್ನೊ-ಡನೆ
ನನ್ನೊ-ಡ-ನೆಯೇ
ನನ್ನೊಬ್ಬ
ನನ್ನೊ-ಳ-ಗಿ-ರು-ವುದು
ನನ್ನೊ-ಳಗೆ
ನಮ-ಗಾ-ಗು-ತ್ತದೆ
ನಮ-ಗಾ-ದೀತು
ನಮ-ಗಿನ್ನೂ
ನಮ-ಗಿ-ರುವ
ನಮಗೂ
ನಮಗೆ
ನಮ-ಗೆಲ್ಲ
ನಮ-ಗೆಲ್ಲಿ
ನಮಗೇ
ನಮ-ಗೇಕೆ
ನಮ-ಗೋ-ಸ್ಕ-ರ-ವಾ-ದರೂ
ನಮ-ಗ್ಯಾ-ರಿಗೂ
ನಮ-ಸ್ಕ-ರಿಸಿ
ನಮ-ಸ್ಕ-ರಿ-ಸಿದ
ನಮ-ಸ್ಕ-ರಿ-ಸಿ-ದಳು
ನಮ-ಸ್ಕ-ರಿ-ಸುತ್ತ
ನಮ-ಸ್ಕ-ರಿ-ಸು-ತ್ತಿದ್ದ
ನಮ-ಸ್ಕ-ರಿ-ಸು-ತ್ತೇನೆ
ನಮ-ಸ್ಕ-ರಿ-ಸು-ವುದನ್ನು
ನಮ-ಸ್ಕಾರ
ನಮ-ಸ್ಕಾ-ರ-ಗಳನ್ನು
ನಮಿಸಿ
ನಮಿ-ಸಿ-ದರು
ನಮೂ-ನೆಯ
ನಮ್ಮ
ನಮ್ಮದು
ನಮ್ಮ-ನಮ್ಮ
ನಮ್ಮ-ನ-ಮ್ಮೊ-ಳ-ಗಿನ
ನಮ್ಮ-ನ್ನಾ-ಗಲಿ
ನಮ್ಮನ್ನು
ನಮ್ಮ-ನ್ನೆಲ್ಲ
ನಮ್ಮನ್ನೇ
ನಮ್ಮಲ್ಲಿ
ನಮ್ಮ-ಲ್ಲಿ-ರುವ
ನಮ್ಮ-ಲ್ಲೊ-ಬ್ಬರು
ನಮ್ಮಾ-ತ್ಮ-ನಾ-ಶಕೆ
ನಮ್ಮಿಂದ
ನಮ್ಮಿಂ-ದಂತೂ
ನಮ್ಮಿ-ಬ್ಬರ
ನಮ್ಮಿ-ಬ್ಬ-ರಲ್ಲಿ
ನಮ್ಮೀ
ನಮ್ಮೆ-ಲ್ಲರ
ನಮ್ಮೊ-ಡನೆ
ನಮ್ರ-ತೆ-ಯಿಂದ
ನಯ-ನ-ಗಳು
ನಯ-ನ-ದ್ವಯ
ನಯ-ನ-ಮ-ನೋ-ಹರ
ನಯ-ನ-ಮ-ನೋ-ಹ-ರ-ವಾ-ಗಿತ್ತು
ನಯ-ವಾ-ಗಿಯೇ
ನಯ-ವಾದ
ನರ-ಕ-ಗ-ಳಾ-ಸೆ-ಭ-ಯ-ಗ-ಳ-ನೆಲ್ಲ
ನರ-ಕ-ಭೀ-ತಿ-ಯಿಂದ
ನರ-ಕವೇ
ನರ-ದೌ-ರ್ಬ-ಲ್ಯದ
ನರ-ನಾ-ಡಿ-ಗ-ಳ-ಲ್ಲೆಲ್ಲ
ನರ-ಪುಷಿ
ನರ-ಭ-ಕ್ಷ-ಕ-ನಿ-ರ-ಬೇಕು
ನರ-ಮಂ-ಡಲ
ನರ-ರೂ-ಪಿ-ಯಾದ
ನರಳಿ
ನರ-ಳಿದ್ದೂ
ನರ-ಳುತ್ತ
ನರ-ಳು-ತ್ತಿದ್ದ
ನರ-ಳು-ತ್ತಿ-ದ್ದರು
ನರ-ಳು-ತ್ತಿ-ದ್ದಾರೆ
ನರ-ಳು-ತ್ತಿ-ರುವ
ನರ-ಳು-ವಂತೆ
ನರ-ವಂ-ದ್ಯಮ್
ನರ-ವೇ-ಷ-ಧಾ-ರಿ-ಯಾಗಿ
ನರಿ-ಬು-ದ್ಧಿ-ಯಾ-ದರೆ
ನರಿ-ಮಂತ್ರಿ
ನರಿ-ಮಂ-ತ್ರಿಯ
ನರೆಂದ್ರ
ನರೇಂದ್ರ
ನರೇಂ-ದ್ರ
ನರೇಂ-ದ್ರ-ಅದೇ
ನರೇಂ-ದ್ರ-ಎ-ಷ್ಟಾ-ದರೂ
ನರೇಂ-ದ್ರ-ಚ-ಕ್ರ-ವರ್ತಿ
ನರೇಂ-ದ್ರನ
ನರೇಂ-ದ್ರ-ನಂ-ತಹ
ನರೇಂ-ದ್ರ-ನಂತೂ
ನರೇಂ-ದ್ರ-ನಂ-ಥ-ವ-ನನ್ನು
ನರೇಂ-ದ್ರ-ನಂ-ಥ-ವರು
ನರೇಂ-ದ್ರ-ನಗೆ
ನರೇಂ-ದ್ರ-ನತ್ತ
ನರೇಂ-ದ್ರ-ನ-ದಾ-ಗಿತ್ತು
ನರೇಂ-ದ್ರ-ನದು
ನರೇಂ-ದ್ರ-ನದೇ
ನರೇಂ-ದ್ರ-ನ-ನ್ನಂತೂ
ನರೇಂ-ದ್ರ-ನನ್ನು
ನರೇಂ-ದ್ರ-ನನ್ನೇ
ನರೇಂ-ದ್ರ-ನ-ನ್ನೊಮ್ಮೆ
ನರೇಂ-ದ್ರ-ನ-ಲ್ಲವೆ
ನರೇಂ-ದ್ರ-ನಲ್ಲಿ
ನರೇಂ-ದ್ರ-ನ-ಲ್ಲೀಗ
ನರೇಂ-ದ್ರ-ನಲ್ಲೂ
ನರೇಂ-ದ್ರ-ನಲ್ಲೇ
ನರೇಂ-ದ್ರ-ನ-ಲ್ಲೊಂದು
ನರೇಂ-ದ್ರ-ನಾಗಿ
ನರೇಂ-ದ್ರ-ನಾ-ಗಿ-ದ್ದಾಗ
ನರೇಂ-ದ್ರ-ನಾಥ
ನರೇಂ-ದ್ರ-ನಾ-ಥ-ಮುಂದೆ
ನರೇಂ-ದ್ರ-ನಾದ
ನರೇಂ-ದ್ರನಿ
ನರೇಂ-ದ್ರ-ನಿಂದ
ನರೇಂ-ದ್ರ-ನಿಂ-ದಲೂ
ನರೇಂ-ದ್ರ-ನಿಗ
ನರೇಂ-ದ್ರ-ನಿ-ಗಂತೂ
ನರೇಂ-ದ್ರ-ನಿ-ಗಾ-ಗಲಿ
ನರೇಂ-ದ್ರ-ನಿ-ಗಾಗಿ
ನರೇಂ-ದ್ರ-ನಿ-ಗಾದ
ನರೇಂ-ದ್ರ-ನಿ-ಗಿದ್ದ
ನರೇಂ-ದ್ರ-ನಿಗೂ
ನರೇಂ-ದ್ರ-ನಿಗೆ
ನರೇಂ-ದ್ರ-ನಿ-ಗೆ-ಆ-ಗ-ಲಪ್ಪ
ನರೇಂ-ದ್ರ-ನಿ-ಗೇನೋ
ನರೇಂ-ದ್ರ-ನಿ-ಗೊಂದು
ನರೇಂ-ದ್ರ-ನಿ-ಗೊ-ದ-ಗಿದ
ನರೇಂ-ದ್ರ-ನಿಗೋ
ನರೇಂ-ದ್ರ-ನಿದ್ದ
ನರೇಂ-ದ್ರ-ನಿ-ಲ್ಲದ
ನರೇಂ-ದ್ರ-ನೀಗ
ನರೇಂ-ದ್ರನು
ನರೇಂ-ದ್ರನೂ
ನರೇಂ-ದ್ರ-ನೆಂದ
ನರೇಂ-ದ್ರ-ನೆಂಬ
ನರೇಂ-ದ್ರ-ನೆ-ಡೆಗೆ
ನರೇಂ-ದ್ರ-ನೆನ್ನು
ನರೇಂ-ದ್ರನೇ
ನರೇಂ-ದ್ರ-ನೇನೂ
ನರೇಂ-ದ್ರ-ನೇನೋ
ನರೇಂ-ದ್ರ-ನೊಂ-ದಿ-ಗಿನ
ನರೇಂ-ದ್ರ-ನೊಂ-ದಿಗೆ
ನರೇಂ-ದ್ರ-ನೊ-ಡನೆ
ನರೇಂ-ದ್ರ-ನೊ-ಬ್ಬ-ನನ್ನೇ
ನರೇಂ-ದ್ರ-ನೊ-ಬ್ಬನೇ
ನರೇಂ-ದ್ರ-ಪ್ರೇ-ಮಕ್ಕೆ
ನರೇಂ-ದ್ರರೂ
ನರೇಂ-ದ್ರಾದಿ
ನರೇಂ-ದ್ರಾ-ದಿ-ಗಳ
ನರೇಂ-ದ್ರಾ-ದಿ-ಗಳು
ನರೇಂ-ದ್ರಾ-ದಿ-ಗ-ಳೆಲ್ಲ
ನರೇ-ನನ
ನರೇ-ನ-ನನ್ನು
ನರೇನ್
ನರ್ತನ
ನರ್ತ-ನ-ಕಲೆ
ನರ್ತ-ನ-ಗಳು
ನರ್ತಿ-ಸ-ಲಾ-ರಂ-ಭಿ-ಸಿ-ದರು
ನರ್ತಿಸಿ
ನರ್ತಿ-ಸಿ-ದರು
ನರ್ತಿ-ಸು-ವುದನ್ನು
ನರ್ಮ-ದಾ-ತೀ-ರಕ್ಕೆ
ನಲ-ವತ್ತು
ನಲಿ-ದಾ-ಟ-ಗಳಲ್ಲಿ
ನಲಿ-ವುದೊ
ನಲು-ಗಿ-ಸದೆ
ನಲ್ಲ
ನಲ್ಲಿ
ನಲ್ಲೇ
ನವ-ಗೋ-ಪಾಲ
ನವ-ಚೇ-ತ-ನ-ವನ್ನು
ನವ-ಚೇ-ತ-ನ-ವ-ನ್ನುಂ-ಟು-ಮಾ-ಡಿತು
ನವ-ಜಾತ
ನವ-ನೂ-ತನ
ನವ-ಭಾ-ರ-ತದ
ನವ-ಯು-ವ-ಕರ
ನವ-ಯು-ವ-ಕರು
ನವ-ಶಕ್ತಿ
ನವ-ಸಂ-ಕ್ರ-ಮಣ
ನವ-ಸಂ-ನ್ಯಾ-ಸಿ-ಗಳು
ನವ-ಸ್ಫೂ-ರ್ತಿ-ಯನ್ನು
ನವಾ-ಬ-ನಂತೆ
ನವಾ-ಬರ
ನವಿಲು
ನವೆಂ-ಬ-ರಿ-ನಲ್ಲಿ
ನವೋ-ತ್ಸಾಹ
ನವೋ-ಲ್ಲಾ-ಸ-ವನ್ನು
ನಶಿಸಿ
ನಶ್ವರ
ನಷ್ಟ
ನಷ್ಟ-ವಾ-ಗು-ತ್ತಿತ್ತು
ನಷ್ಟ-ವಾ-ಗು-ವುದು
ನಸು-ಕಿ-ನಲ್ಲೇ
ನಸು-ಗ-ತ್ತ-ಲಾ-ಗು-ತ್ತಲೇ
ನಸು-ಗೆಂ-ಪಿನ
ನಸು-ನಕ್ಕ
ನಸು-ನ-ಕ್ಕರು
ನಹ-ಬತ್
ನಾ
ನಾಂದಿ-ಯಾ-ಗಿ-ಬಿ-ಟ್ಟಿ-ರು-ತ್ತದೆ
ನಾಕ
ನಾಗ-ಬೇಕು
ನಾಗ-ರ-ಹಾ-ವಿನ
ನಾಗ-ರ-ಹಾವು
ನಾಗ-ರ-ಹಾ-ವೇ-ನಾ-ದರೂ
ನಾಗ-ರಿ-ಕತೆ
ನಾಗ-ರಿ-ಕ-ತೆಯ
ನಾಗ-ರಿ-ಕ-ತೆಯು
ನಾಗ-ರಿ-ಕ-ನಾದ
ನಾಗ-ರಿ-ಕ-ರಿಗೆ
ನಾಗ-ರಿ-ಕರು
ನಾಗಿ
ನಾಗಿದ್ದ
ನಾಗಿ-ಬಿ-ಟ್ಟ-ನೆಂ-ದರೆ
ನಾಗಿ-ಬಿ-ಟ್ಟಿ-ದ್ದಾನೆ
ನಾಗಿ-ಬಿ-ಡ-ಲಿಲ್ಲ
ನಾಗು-ವಂತೆ
ನಾಚಿ
ನಾಚಿಕೆ
ನಾಚಿ-ಕೆ-ಕ-ಸಿ-ವಿಸಿ
ನಾಚಿ-ಕೆ-ಯಾ-ಗು-ವು-ದಿ-ಲ್ಲವೇ
ನಾಚಿ-ಕೆ-ಯಾ-ಯಿ-ತಂತೆ
ನಾಚಿ-ಕೆ-ಯಾ-ಯಿತು
ನಾಚಿ-ಕೆ-ಯಾ-ಯಿ-ತೆಂ-ದರೆ
ನಾಚಿ-ಕೆ-ಯಿಂದ
ನಾಚಿ-ಕೆ-ಯಿಲ್ಲ
ನಾಚಿ-ಗೆ-ಗೇಡು
ನಾಚಿ-ಸುವ
ನಾಚೇ
ನಾಟಕ
ನಾಟ-ಕಕ್ಕೆ
ನಾಟ-ಕ-ಗಳನ್ನು
ನಾಟ-ಕದ
ನಾಟ-ಕ-ವಾ-ಡು-ತ್ತಿ-ದ್ದಾರೆ
ನಾಟ-ಕಾ-ಭ್ಯಾಸ
ನಾಟ-ಕೀ-ಯ-ವಾಗಿ
ನಾಟಿತು
ನಾಟು-ವಂತೆ
ನಾಟ್ಯ-ನಿ-ಪುಣ
ನಾಡ-ಬೇಡಿ
ನಾಡ-ಲಾ-ರಂ-ಭಿ-ಸಿ-ದರೆ
ನಾಡಲು
ನಾಡಲೂ
ನಾಡಿ
ನಾಡಿ-ಗಳ
ನಾಡಿಗೆ
ನಾಡಿ-ದರೂ
ನಾಡಿ-ದ್ದರು
ನಾಡಿದ್ದು
ನಾಡಿನ
ನಾಡಿ-ನಲ್ಲಿ
ನಾಡಿ-ಬ-ಡಿತ
ನಾಡು
ನಾಡು-ತ್ತಿ-ದ್ದರು
ನಾಣ್ಯ
ನಾಣ್ಯದ
ನಾಣ್ಯ-ವನ್ನು
ನಾಣ್ಯವೋ
ನಾಥ
ನಾಥ-ನ-ಗರ
ನಾಥ-ನ-ಗ-ರ-ದಲ್ಲಿ
ನಾದ
ನಾದ-ದೊಂ-ದಿಗೆ
ನಾದರೂ
ನಾದರೆ
ನಾದರೋ
ನಾದ-ವನ್ನು
ನಾದವೇ
ನಾದಿತ
ನಾದ್ಯಂತ
ನಾನ-ದನ್ನು
ನಾನ-ರಿತೆ
ನಾನಲ್ಲ
ನಾನ-ಲ್ಲಿಗೆ
ನಾನ-ವನ
ನಾನ-ವ-ನನ್ನು
ನಾನ-ವ-ರಿಗೆ
ನಾನಹೆ
ನಾನಾ
ನಾನಾಗ
ನಾನಾ-ಗಲೇ
ನಾನಾ-ಗಿಯೇ
ನಾನಾ-ದರೂ
ನಾನಾ-ಬ-ಗೆಯ
ನಾನಾ-ವಿ-ಧ-ವಾಗಿ
ನಾನಿಂದು
ನಾನಿ-ದ-ನ್ನೆಲ್ಲ
ನಾನಿ-ದ್ದೇನೆ
ನಾನಿ-ನಿ-ತನ್ನೆ
ನಾನಿನ್ನು
ನಾನಿ-ನ್ನೇನು
ನಾನಿ-ಲ್ಲದೆ
ನಾನಿಲ್ಲಿ
ನಾನಿ-ಲ್ಲಿಗೆ
ನಾನಿಲ್ಲೇ
ನಾನಿ-ವತ್ತು
ನಾನೀಗ
ನಾನು
ನಾನು-ನೀನು
ನಾನು-ಹೀಗೆ
ನಾನೂ
ನಾನೆಂಥ
ನಾನೆಂದೆ
ನಾನೆಂ-ದೆ-ನ-ನಗೆ
ನಾನೆ-ಣಿ-ಸಿದೆ
ನಾನೆ-ಣಿ-ಸಿ-ದ್ದೆ-ನೀ-ನೊಂದು
ನಾನೆ-ಲ್ಲಿಗೆ
ನಾನೆಲ್ಲೂ
ನಾನೆಷ್ಟು
ನಾನೆಷ್ಟೇ
ನಾನೆ-ಷ್ಟೊಂದು
ನಾನೇ
ನಾನೇಕೆ
ನಾನೇ-ನನ್ನ
ನಾನೇನು
ನಾನೊಂದು
ನಾನೊಬ್ಬ
ನಾನೊ-ಬ್ಬನೇ
ನಾನೋ
ನಾಪತ್ತೆ
ನಾಬಲ್ಲೆ
ನಾಮ
ನಾಮ-ರೂ-ಪ-ಗಳ
ನಾಮ-ಕ-ರ-ಣದ
ನಾಮ-ಕ-ರ-ಣ-ವಾ-ಯಿತು
ನಾಮ-ದಿಂದ
ನಾಮ-ರೂ-ಪ-ಗ-ಳಿಗೆ
ನಾಮ-ರೂ-ಪ-ಗಳೂ
ನಾಮ-ರೂ-ಪಾ-ತೀ-ತ-ನಾ-ತನು
ನಾಮ-ರೂ-ಪಾ-ತೀ-ತವು
ನಾಮ-ವನ್ನು
ನಾಮ-ಸಂ-ಕೀ-ರ್ತನೆ
ನಾಮಾ-ವ-ಳಿ-ಗಳನ್ನು
ನಾಮಾ-ವ-ಳಿ-ಯನ್ನು
ನಾಮುಂದು
ನಾಯಕ
ನಾಯ-ಕ
ನಾಯ-ಕತ್ವ
ನಾಯ-ಕ-ತ್ವ-ದಲ್ಲಿ
ನಾಯ-ಕ-ತ್ವ-ವನ್ನು
ನಾಯ-ಕ-ತ್ವವೇ
ನಾಯ-ಕ-ನ-ನ್ನಾಗಿ
ನಾಯ-ಕ-ನಲ್ಲಿ
ನಾಯ-ಕ-ನಾಗಿ
ನಾಯ-ಕ-ನಾ-ಗಿದ್ದ
ನಾಯ-ಕ-ನಾ-ಗು-ವು-ದೆಂ-ದರೆ
ನಾಯ-ಕ-ನಾದ
ನಾಯ-ಕ-ನೆಂದು
ನಾಯ-ಕ-ರ-ಲ್ಲವೆ
ನಾಯ-ಕೀ-ಭಾ-ವದ
ನಾಯಿ
ನಾಯಿಯ
ನಾಯಿ-ಯಂತೆ
ನಾರ-ದೀಯ
ನಾರಾ-ಯಣ
ನಾರಾ-ಯ-ಣನ
ನಾರಾ-ಯ-ಣ-ನ-ನ್ನಾಗಿ
ನಾರಾ-ಯ-ಣ-ನನ್ನು
ನಾರಾ-ಯ-ಣ-ನೆಂ-ಬಂತೆ
ನಾರೀ-ಕು-ಲ-ದಿಂದ
ನಾರು-ತ್ತದೆ
ನಾರುವ
ನಾರುವು
ನಾಲಿಗೆ
ನಾಲಿ-ಗೆ-ಯಲ್ಲಿ
ನಾಲ್ಕ-ನೆಯ
ನಾಲ್ಕನೇ
ನಾಲ್ಕ-ರಂದು
ನಾಲ್ಕಾರು
ನಾಲ್ಕು
ನಾಲ್ಕೈದು
ನಾಲ್ವರು
ನಾಳಿನ
ನಾಳೆ
ನಾಳೆಯ
ನಾವಾ-ಗಲೇ
ನಾವಾ-ದರೂ
ನಾವಿಕ
ನಾವಿ-ಕನ
ನಾವಿ-ಕ-ನನ್ನು
ನಾವಿ-ಕ-ನಿಗೆ
ನಾವಿ-ಕನೂ
ನಾವಿ-ಬ್ಬರೂ
ನಾವಿ-ಬ್ಬರೇ
ನಾವಿಲ್ಲಿ
ನಾವೀಗ
ನಾವೀ-ಗಾ-ಗಲೇ
ನಾವು
ನಾವು-ಗಳು
ನಾವೂ
ನಾವೆಲ್ಲ
ನಾವೆ-ಲ್ಲರೂ
ನಾವೇ
ನಾವೇಕೆ
ನಾಶಕ್ಕೆ
ನಾಶ-ಗೊ-ಳಿ-ಸಿ-ದರೆ
ನಾಶ-ಮಾ-ಡದ
ನಾಶ-ಮಾ-ಡ-ಬಲ್ಲ
ನಾಶ-ವಾಗಿ
ನಾಶ-ವಾ-ಗಿ-ಹೋ-ಯಿತು
ನಾಸಿ-ಕಾ-ಗ್ರ-ದಲ್ಲಿ
ನಾಸ್ತಿಕ
ನಾಸ್ತಿ-ಕತೆ
ನಾಸ್ತಿ-ಕ-ತೆಗೂ
ನಾಸ್ತಿ-ಕ-ತೆಯ
ನಾಸ್ತಿ-ಕ-ತೆ-ಯಿಂದ
ನಾಸ್ತಿ-ಕ-ತೆಯೂ
ನಾಸ್ತಿ-ಕ-ತೆ-ಯೆ-ನ್ನು-ವುದು
ನಾಸ್ತಿ-ಕ-ತೆ-ಯೇ-ನಲ್ಲ
ನಾಸ್ತಿ-ಕ-ನಂತೆ
ನಾಸ್ತಿ-ಕ-ನಲ್ಲ
ನಾಸ್ತಿ-ಕ-ನಾ-ಗ-ದಂತೆ
ನಾಸ್ತಿ-ಕ-ನಾ-ಗಲು
ನಾಸ್ತಿ-ಕ-ನಾ-ಗಿ-ಬಿಟ್ಟ
ನಾಸ್ತಿ-ಕ-ನಾ-ಗಿ-ಬಿ-ಟ್ಟಿ-ದ್ದಾ-ನಂತೆ
ನಾಸ್ತಿ-ಕ-ನಾ-ಗಿಯೇ
ನಾಸ್ತಿ-ಕ-ನಾ-ದದ್ದೂ
ನಾಸ್ತಿ-ಕ-ನಾ-ದಾನು
ನಾಸ್ತಿ-ಕ-ವಾ-ದಿ-ಗಳ
ನಾಸ್ತಿ-ಕ್ಯದ
ನಾಸ್ತಿ-ಕ್ಯ-ವನ್ನು
ನಿಂತ
ನಿಂತಂತೆ
ನಿಂತದ್ದೇ
ನಿಂತರು
ನಿಂತರೆ
ನಿಂತ-ಲ್ಲಿಂದ
ನಿಂತಾಗ
ನಿಂತಾ-ರೆಯೆ
ನಿಂತಿತು
ನಿಂತಿತ್ತು
ನಿಂತಿದೆ
ನಿಂತಿದ್ದ
ನಿಂತಿ-ದ್ದರು
ನಿಂತಿ-ದ್ದಾನೆ
ನಿಂತಿ-ದ್ದಾರೆ
ನಿಂತಿ-ದ್ದಾಳೆ
ನಿಂತಿ-ದ್ದುವು
ನಿಂತಿರ
ನಿಂತಿ-ರ-ಬೇ-ಕಾ-ಗಿತ್ತು
ನಿಂತಿ-ರ-ಲಿಲ್ಲ
ನಿಂತಿ-ರು-ತ್ತದೆ
ನಿಂತಿ-ರು-ತ್ತಿದ್ದ
ನಿಂತಿ-ರುವ
ನಿಂತಿ-ರು-ವಂತೆ
ನಿಂತಿ-ರು-ವನು
ನಿಂತಿ-ರು-ವ-ಳೆಂ-ದರೆ
ನಿಂತಿವೆ
ನಿಂತು
ನಿಂತು-ಕೊಂಡು
ನಿಂತು-ಕೊ-ಳ್ಳು-ವಂತೆ
ನಿಂತು-ಬಿಟ್ಟ
ನಿಂತು-ಬಿ-ಟ್ಟರು
ನಿಂತು-ಬಿ-ಟ್ಟರೆ
ನಿಂತು-ಬಿ-ಟ್ಟಿ-ದ್ದರು
ನಿಂತು-ಬಿ-ಟ್ಟಿ-ದ್ದಾಳೆ
ನಿಂತು-ಬಿ-ಡು-ತ್ತಿದ್ದ
ನಿಂತುವು
ನಿಂತು-ಹೋ-ಗು-ತ್ತವೆ
ನಿಂತು-ಹೋ-ಗು-ತ್ತಿತ್ತು
ನಿಂತೆ
ನಿಂತೇ-ಹೋ-ದಂ-ತಾ-ಯಿತು
ನಿಂದ
ನಿಂದಕ
ನಿಂದಲೂ
ನಿಂದಲೇ
ನಿಂದಾಚೆ
ನಿಂದಿ-ಪ-ರೆಲ್ಲ
ನಿಂದಿಸಿ
ನಿಂದಿ-ಸು-ವ-ವ-ರು-ನಿಂ-ದಿ-ಸ-ಲ್ಪ-ಡು-ವ-ವರು
ನಿಂದೆ
ನಿಂದೆ-ಯ-ನುಂ-ಬರು
ನಿಂದೆ-ಯ-ನ್ನಾ-ಗಲಿ
ನಿಃಸ್ವಾರ್ಥ
ನಿಕಟ
ನಿಕ-ಟ-ಗೊ-ಳಿ-ಸಿತು
ನಿಕ-ಟ-ವಾಗಿ
ನಿಕ-ಟ-ವಾದ
ನಿಕ-ಟ-ಸಂ-ಪ-ರ್ಕ-ದ-ಲ್ಲಿದ್ದ
ನಿಕ್ಕು-ವ-ವ-ನಲ್ಲ
ನಿಖರ
ನಿಖ-ರತೆ
ನಿಖ-ರ-ವಾಗಿ
ನಿಖ-ರ-ವಾದ
ನಿಖಿಲ
ನಿಗ-ದ್ಯತೇ
ನಿಗೂಢ
ನಿಗೆ
ನಿಚ್ಚ-ಳ-ವಾ-ಗ-ತೊ-ಡ-ಗಿತು
ನಿಚ್ಚ-ಳ-ವಾಗಿ
ನಿಚ್ಚ-ಳ-ವಾ-ಗಿತ್ತು
ನಿಚ್ಚ-ವಾ-ಗಿಹ
ನಿಜ
ನಿಜ-ಇ-ವರ
ನಿಜಕ್ಕೂ
ನಿಜ-ಜೀ-ವ-ನ-ದಲ್ಲಿ
ನಿಜ-ತ್ವ-ವನ್ನು
ನಿಜ-ವ-ನ-ರಿ-ತ-ವರೆಲ್ಲೊ
ನಿಜ-ವ-ನ-ರಿ-ಯಲು
ನಿಜ-ವನು
ನಿಜ-ವಾಗಿ
ನಿಜ-ವಾ-ಗಿ-ದ್ದರೆ
ನಿಜ-ವಾ-ಗಿಯೂ
ನಿಜ-ವಾ-ಗುತ್ತ
ನಿಜ-ವಾ-ಗು-ವುದನ್ನು
ನಿಜ-ವಾದ
ನಿಜ-ವಾ-ದರೂ
ನಿಜ-ವಾ-ದಲ್ಲಿ
ನಿಜವೆ
ನಿಜ-ವೆಂ-ದಾ-ಗು-ತ್ತ-ದೆ-ಯ-ಲ್ಲವೆ
ನಿಜ-ವೆಂದು
ನಿಜ-ವೆಂದೇ
ನಿಜ-ವೆನ್ನು
ನಿಜವೇ
ನಿಜ-ವ್ಯ-ಕ್ತಿ-ತ್ವದ
ನಿಜ-ವ್ಯ-ಕ್ತಿ-ತ್ವ-ವೊಂದು
ನಿಜ-ಸ್ವ-ಭಾ-ವ-ವನ್ನು
ನಿಜ-ಸ್ವ-ರೂಪ
ನಿಜ-ಸ್ವ-ರೂ-ಪ-ಇಂ-ಥ-ವು-ಗಳ
ನಿಜ-ಸ್ವ-ರೂ-ಪದ
ನಿಜ-ಸ್ವ-ರೂ-ಪ-ದಿಂದ
ನಿಜ-ಸ್ವ-ರೂ-ಪ-ವನ್ನು
ನಿಜಾಂ-ಶ-ವನ್ನು
ನಿಜಾಂ-ಶ-ವಿ-ದ್ದರೂ
ನಿಜಾರ್ಥ
ನಿಟ್ಟಿ-ಸಿ-ದಳು
ನಿಟ್ಟಿ-ಸುತ್ತ
ನಿಟ್ಟು-ಸಿರು
ನಿಟ್ಟು-ಸಿ-ರೆ-ಳೆ-ಯು-ತ್ತಿ-ದ್ದರು
ನಿಟ್ಟು-ಸಿ-ರೊಂದು
ನಿತ್ಯ
ನಿತ್ಯ-ಗೋ-ಪಾ-ಲನೇ
ನಿತ್ಯ-ಜೀ-ವ-ನದ
ನಿತ್ಯ-ಜೀ-ವ-ನ-ದಲ್ಲಿ
ನಿತ್ಯ-ಜೀ-ವ-ನ-ದ-ಲ್ಲಿಯೇ
ನಿತ್ಯ-ಜ್ಞಾ-ನ-ಕ್ಕಾಗಿ
ನಿತ್ಯ-ತೃಪ್ತಿ
ನಿತ್ಯದ
ನಿತ್ಯ-ನಿ-ಚ್ಚ-ವಾದ
ನಿತ್ಯ-ಮು-ಕ್ತನು
ನಿತ್ಯ-ಮು-ಕ್ತನೂ
ನಿತ್ಯ-ಮು-ಕ್ತರು
ನಿತ್ಯವು
ನಿತ್ಯ-ಶಾಂತಿ
ನಿತ್ಯ-ಸ-ತ್ಯ-ದೆ-ಡೆಗೆ
ನಿತ್ಯ-ಸಿದ್ಧ
ನಿತ್ಯ-ಸಿ-ದ್ಧ-ನಾದ
ನಿತ್ಯ-ಸಿ-ದ್ಧರು
ನಿತ್ಯಾ-ನಂದ
ನಿತ್ಯಾ-ನಂ-ದನು
ನಿತ್ಯಾ-ನಂ-ದ-ಪ್ರಾ-ಪ್ತಿಯ
ನಿತ್ಯಾ-ನಂ-ದ-ಮೂರ್ತಿ
ನಿತ್ಯಾ-ನಂ-ದ-ವನ್ನು
ನಿತ್ರಾ-ಣ-ನಾಗಿ
ನಿತ್ರಾ-ಣ-ನಾ-ಗಿದ್ದ
ನಿತ್ರಾ-ಣ-ರಾ-ಗಿ-ದ್ದ-ರೆಂ-ದರೆ
ನಿದ-ರ್ಶನ
ನಿದ-ರ್ಶ-ನ-ವಾದ
ನಿದ್ದಾನೆ
ನಿದ್ದೆ
ನಿದ್ದೆ-ಗ-ಣ್ಣಿಗೆ
ನಿದ್ರಾ-ಮ-ಗ್ನ-ನಾ-ಗಿ-ರು-ವಾಗ
ನಿದ್ರಾ-ವ-ಸ್ಥೆ-ಯ-ಲ್ಲಿ-ರುವ
ನಿದ್ರಾ-ಹಾ-ರ-ಗಳು
ನಿದ್ರಿ-ಸಿದ್ದೂ
ನಿದ್ರಿಸು
ನಿದ್ರಿ-ಸು-ತ್ತಿ-ದ್ದಾನೆ
ನಿದ್ರಿ-ಸು-ತ್ತಿ-ದ್ದಾರೆ
ನಿದ್ರೆ
ನಿದ್ರೆ-ಸ್ವ-ಪ್ನ-ಗ-ಳಂ-ತಲ್ಲ
ನಿದ್ರೆಗೆ
ನಿದ್ರೆಯ
ನಿದ್ರೆ-ಯನ್ನೂ
ನಿದ್ರೆ-ಯಲ್ಲಿ
ನಿದ್ರೆಯೇ
ನಿದ್ರೆ-ಹೋ-ಗು-ತ್ತಿದ್ದ
ನಿಧ-ನ-ಕ್ಕಾಗಿ
ನಿಧ-ನ-ದಿಂದ
ನಿಧ-ನ-ನಾ-ದ-ನೆಂಬ
ನಿಧ-ನಾ-ನಂ-ತರ
ನಿಧಾನ
ನಿಧಾ-ನ-ವಾಗಿ
ನಿಧಾ-ನ-ವಾ-ಗಿ-ಯಾ-ದರೂ
ನಿಧಾ-ನ-ವಾ-ಗುತ್ತ
ನಿಧಾ-ನ-ವಾ-ಯಿತು
ನಿಧಿ
ನಿನ
ನಿನ-ಗಾಗಿ
ನಿನ-ಗಿಂತ
ನಿನ-ಗಿನ್ನೂ
ನಿನ-ಗಿಲ್ಲ
ನಿನ-ಗೀಗ
ನಿನಗೂ
ನಿನಗೆ
ನಿನಗೇ
ನಿನ-ಗೇ-ನಾ-ದರೂ
ನಿನ-ಗೇನು
ನಿನ-ಗೇನೂ
ನಿನ-ಗೇ-ನೇನು
ನಿನ-ಗೋ-ಸ್ಕರ
ನಿನ-ಗ್ಯಾರೋ
ನಿನಾದ
ನಿನ್ನ
ನಿನ್ನಂ-ತಹ
ನಿನ್ನಂಥ
ನಿನ್ನ-ದಾ-ಗು-ತ್ತದೆ
ನಿನ್ನದು
ನಿನ್ನನು
ನಿನ್ನನ್ನು
ನಿನ್ನಲ್ಲಿ
ನಿನ್ನ-ಲ್ಲಿಯೆ
ನಿನ್ನಿಂದ
ನಿನ್ನಿಷ್ಟ
ನಿನ್ನಿ-ಷ್ಟ-ದಂ-ತೆಯೇ
ನಿನ್ನೀ
ನಿನ್ನೆ
ನಿನ್ನೆ-ದ-ಯೊ-ಳ-ಗಿ-ರು-ತಿ-ರಲು
ನಿನ್ನೆ-ದು-ರಿಗೇ
ನಿನ್ನೆ-ದುರು
ನಿನ್ನೊ-ಡನೆ
ನಿನ್ನೊ-ಳ-ಗಿನ
ನಿನ್ನೊ-ಳಗೆ
ನಿನ್ನೊ-ಳಗೇ
ನಿಪುಣ
ನಿಪು-ಣನೋ
ನಿಬ-ದ್ಧ-ರಾ-ಗಿ-ರ-ಬೇ-ಕಾ-ಗು-ತ್ತದೆ
ನಿಬ-ದ್ಧ-ವಾ-ಗಿತ್ತೋ
ನಿಬ್ಬೆ-ರ-ಗಾದ
ನಿಭಾ-ಯಿ-ಸ-ಬಲ್ಲ
ನಿಭಾ-ಯಿ-ಸಲು
ನಿಭಾ-ಯಿಸಿ
ನಿಭಾ-ಯಿ-ಸು-ವ-ವ-ರಿಗೆ
ನಿಭಾ-ವಣೆ
ನಿಮ-ಗ-ದೆಲ್ಲ
ನಿಮ-ಗ-ವೆಲ್ಲ
ನಿಮಗೂ
ನಿಮಗೆ
ನಿಮ-ಗೆಲ್ಲ
ನಿಮ-ಗೆ-ಲ್ಲ-ರಿಗೂ
ನಿಮಗೇ
ನಿಮ-ಗೇ-ನಾ-ದರೂ
ನಿಮ-ಗೋ-ಸ್ಕರ
ನಿಮಾ-ಯಿ-ಚಂದ್ರ
ನಿಮಿರಿ
ನಿಮಿಷ
ನಿಮಿ-ಷ-ಗ-ಳ-ಕಾ-ಲ-ವನ್ನು
ನಿಮಿ-ಷ-ಗಳಲ್ಲಿ
ನಿಮಿ-ಷ-ಗ-ಳಲ್ಲೇ
ನಿಮಿ-ಷ-ಗ-ಳ-ವ-ರೆಗೆ
ನಿಮಿ-ಷ-ಗ-ಳಾದ
ನಿಮಿ-ಷ-ದಲ್ಲಿ
ನಿಮಿ-ಷ-ಮಾ-ತ್ರ-ದಲ್ಲಿ
ನಿಮಿ-ಷವೂ
ನಿಮಿ-ಷಾ-ರ್ಧ-ದಲ್ಲಿ
ನಿಮ್ಮ
ನಿಮ್ಮಂ-ತಹ
ನಿಮ್ಮದೇ
ನಿಮ್ಮ-ನಿಮ್ಮ
ನಿಮ್ಮನ್ನು
ನಿಮ್ಮ-ಲ್ಲಿಗೆ
ನಿಮ್ಮೆ-ಲ್ಲರ
ನಿಮ್ಮೊ-ಡ-ನಿದ್ದು
ನಿಮ್ಮೊ-ಳ-ಗಿಂ-ದಲೇ
ನಿಯಂ-ತ್ರ-ಣ-ವಿಲ್ಲ
ನಿಯಂ-ತ್ರಿ-ಸಿ-ಕೊಂಡು
ನಿಯಂ-ತ್ರಿ-ಸಿ-ಕೊ-ಳ್ಳ-ಲಾ-ರ-ದಾದ
ನಿಯಂ-ತ್ರಿ-ಸು-ವಂ-ತಹ
ನಿಯತ
ನಿಯ-ತ-ಕಾ-ಲಿ-ಕ-ಗಳು
ನಿಯಮ
ನಿಯ-ಮ
ನಿಯ-ಮ-ದೊ-ಣ್ಣೆ-ಯಿಂ
ನಿಯ-ಮ-ನಿ-ರ್ದೇ-ಶನ
ನಿಯ-ಮ-ನಿ-ರ್ಬಂ-ಧ-ವನ್ನೂ
ನಿಯ-ಮ-ಗಳನ್ನು
ನಿಯ-ಮ-ಗಳನ್ನೂ
ನಿಯ-ಮ-ಗ-ಳ-ಲ್ಲೊಂದು
ನಿಯ-ಮ-ಗ-ಳಿ-ಗ-ನು-ಸಾ-ರ-ವಾಗಿ
ನಿಯ-ಮ-ಗ-ಳಿಗೆ
ನಿಯ-ಮ-ಗಳು
ನಿಯ-ಮದ
ನಿಯ-ಮ-ಬ-ದ್ಧ-ವಾಗಿ
ನಿಯ-ಮ-ವನ್ನು
ನಿಯ-ಮಿ-ತ-ವಾಗಿ
ನಿಯ-ಮಿ-ಸಿ-ದ್ದಾರೆ
ನಿಯು-ಕ್ತ-ವಾ-ಗ-ಲಿ-ರುವ
ನಿಯೋ-ಜಿ-ತ-ರಾ-ಗಿ-ರುವ
ನಿಯೋ-ಜಿ-ಸ-ಬೇ-ಕಾ-ದರೆ
ನಿರಂ-ಕು-ಶ-ಸ್ವ-ಚ್ಛಂ-ದ-ವೆಂ-ಬಂತೆ
ನಿರಂ-ಜನ
ನಿರಂ-ಜ-ನಾ-ನಂದ
ನಿರಂ-ತರ
ನಿರಂ-ತ-ರ-ವಾಗಿ
ನಿರಂ-ತ-ರವೂ
ನಿರ-ಗ್ನಿ-ಗ-ಳಾ-ದ್ದ-ರಿಂ-ದ-ಎಂ-ದರೆ
ನಿರತ
ನಿರ-ತ-ನಾ-ಗಿದ್ದ
ನಿರ-ತ-ನಾ-ಗಿ-ದ್ದು-ದ-ಲ್ಲದೆ
ನಿರ-ತ-ನಾದ
ನಿರ-ತರಾ
ನಿರ-ತ-ರಾಗಿ
ನಿರ-ತ-ರಾ-ಗಿ-ದ್ದಾರೆ
ನಿರ-ತ-ರಾ-ಗಿ-ದ್ದು-ಬಿ-ಡೋಣ
ನಿರ-ತ-ರಾ-ಗಿ-ಬಿ-ಟ್ಟರು
ನಿರ-ತ-ರಾ-ಗಿ-ಬಿ-ಡು-ತ್ತಾರೆ
ನಿರ-ತ-ರಾ-ಗಿ-ರು-ತ್ತಿ-ದ್ದರು
ನಿರ-ತ-ರಾ-ಗಿ-ರು-ವುದೇ
ನಿರ-ತ-ರಾ-ಗು-ತ್ತಿ-ದ್ದರು
ನಿರ-ತ-ರಾ-ದರು
ನಿರ-ತ-ಳಾ-ಗು-ತ್ತಿ-ದ್ದಳು
ನಿರತೇ
ನಿರರ್
ನಿರ-ರ್ಗ-ಳ-ವಾಗಿ
ನಿರ-ರ್ಥಕ
ನಿರ-ರ್ಥ-ಕ-ವ-ಲ್ಲವೆ
ನಿರಾ-ಕ-ರಣೆ
ನಿರಾ-ಕ-ರ-ಣೆಯೇ
ನಿರಾ-ಕರಿ
ನಿರಾ-ಕ-ರಿ-ಸ-ದಂ-ತಾ-ಗಲಿ
ನಿರಾ-ಕ-ರಿ-ಸ-ದಿ-ದ್ದುದು
ನಿರಾ-ಕ-ರಿ-ಸ-ದಿ-ರಲಿ
ನಿರಾ-ಕ-ರಿ-ಸ-ಲಿಲ್ಲ
ನಿರಾ-ಕ-ರಿಸಿ
ನಿರಾ-ಕ-ರಿ-ಸಿ-ತ್ತೆಂ-ದಲ್ಲ
ನಿರಾ-ಕ-ರಿ-ಸಿದ
ನಿರಾ-ಕ-ರಿ-ಸಿ-ದರೂ
ನಿರಾ-ಕ-ರಿ-ಸಿ-ದರೆ
ನಿರಾ-ಕ-ರಿ-ಸಿ-ದಾಗ
ನಿರಾ-ಕ-ರಿ-ಸಿ-ಬಿ-ಟ್ಟರು
ನಿರಾ-ಕ-ರಿ-ಸಿ-ಬಿ-ಟ್ಟರೆ
ನಿರಾ-ಕ-ರೋತ್
ನಿರಾ-ಕಾರ
ನಿರಾ-ಕಾ-ರ-ಇದು
ನಿರಾ-ಕಾ-ರನೂ
ನಿರಾ-ಕಾ-ರನೆ
ನಿರಾ-ಕಾ-ರ-ಬ್ರಹ್ಮ
ನಿರಾ-ಕಾ-ರ-ವಾ-ದಿ-ಯಾ-ಗಿದ್ದ
ನಿರಾ-ಕು-ರ್ಯಾಮ್
ನಿರಾ-ತಂ-ಕದ
ನಿರಾ-ತಂ-ಕ-ವಾಗಿ
ನಿರಾ-ಶ-ನಾ-ಗಿ-ರ-ಲಿಲ್ಲ
ನಿರಾ-ಶ-ನಾ-ಗು-ತ್ತೀಯೋ
ನಿರಾ-ಶ-ನಾದ
ನಿರಾ-ಶ-ರಾಗಿ
ನಿರಾ-ಶಾ-ದಾ-ಯಕ
ನಿರಾ-ಶೆಯ
ನಿರಾ-ಶೆ-ಯಾ-ಗಿದೆ
ನಿರಾ-ಶೆ-ಯಾ-ಯಿತು
ನಿರಾ-ಸಕ್ತ
ನಿರಾ-ಸ-ಕ್ತ-ಸಪ್ಪೆ
ನಿರಾ-ಸೆ-ಯಾಗು
ನಿರೀ-ಕ್ಷಿ-ಸ-ಬ-ಹುದು
ನಿರೀ-ಕ್ಷಿ-ಸ-ಬೇಕು
ನಿರೀ-ಕ್ಷಿಸಿ
ನಿರೀ-ಕ್ಷಿ-ಸಿ-ಕೊಂಡು
ನಿರೀ-ಕ್ಷಿ-ಸಿದ
ನಿರೀ-ಕ್ಷಿ-ಸಿ-ರ-ಲಿಲ್ಲ
ನಿರೀ-ಕ್ಷಿ-ಸುತ್ತ
ನಿರೀ-ಕ್ಷಿ-ಸುತ್ತಾ
ನಿರೀ-ಕ್ಷಿ-ಸು-ತ್ತಾರೆ
ನಿರೀ-ಕ್ಷಿ-ಸು-ವಂ-ತೆಯೇ
ನಿರೀ-ಕ್ಷಿ-ಸು-ವ-ವ-ರಲ್ಲ
ನಿರೀಕ್ಷೆ
ನಿರೀ-ಕ್ಷೆ-ಭ-ರ-ವ-ಸೆ-ಗ-ಳೆಲ್ಲ
ನಿರೀ-ಕ್ಷೆಗೆ
ನಿರೀ-ಕ್ಷೆ-ಯ-ಲ್ಲಿದೆ
ನಿರೀ-ಕ್ಷೆ-ಯಲ್ಲೇ
ನಿರೀ-ಕ್ಷೆಯೂ
ನಿರೂ-ಪಿ-ಸು-ವು-ದರ
ನಿರೋ
ನಿರೋ-ಧಿ-ಸು-ತ್ತಿದ್ದ
ನಿರ್ಗ-ತಿಕ
ನಿರ್ಗ-ತಿ-ಕ-ವಾ-ಗಿದ್ದ
ನಿರ್ಗ-ಮಿಸಿ
ನಿರ್ಗ-ಮಿ-ಸಿದ
ನಿರ್ಗ-ಮಿ-ಸು-ತ್ತಿ-ದ್ದಂ-ತೆಯೇ
ನಿರ್ಗು-ಣ-ನಿ-ರಾ-ಕಾ-ರ-ನೆಂದು
ನಿರ್ಗು-ಣ-ನಿ-ರಾ-ಕಾ-ರ-ವಾದ
ನಿರ್ಗು-ಣನೂ
ನಿರ್ಜನ
ನಿರ್ಜ-ನ-ವಾದ
ನಿರ್ಜೀವ
ನಿರ್ದ-ಯಿ-ಗ-ಳ-ಲ್ಲ-ವಲ್ಲ
ನಿರ್ದಿಷ್ಟ
ನಿರ್ದಿ-ಷ್ಟ-ವಾದ
ನಿರ್ದೇ-ಶ-ನದ
ನಿರ್ದೇ-ಶಿ-ಸಿ-ರ-ಬ-ಹುದು
ನಿರ್ಧ-ರಿಸಿ
ನಿರ್ಧ-ರಿ-ಸಿ-ದರು
ನಿರ್ಧ-ರಿ-ಸಿ-ಬಿಟ್ಚ
ನಿರ್ಧ-ರಿ-ಸಿ-ರ-ಬೇಕು
ನಿರ್ಧ-ರಿ-ಸು-ತ್ತದೆ
ನಿರ್ಧ-ರಿ-ಸು-ವುದು
ನಿರ್ಧಾರ
ನಿರ್ಧಾ-ರಕ್ಕೆ
ನಿರ್ಧಾ-ರ-ದಿಂದ
ನಿರ್ಧಾ-ರ-ವನ್ನು
ನಿರ್ಧಾ-ರ-ವಾ-ಗುವ
ನಿರ್ಧಾ-ರ-ವೇನೆಂದರೆ
ನಿರ್ನಾಮ
ನಿರ್ಬಂಧ
ನಿರ್ಬಂ-ಧ-ಗ-ಳಾ-ವುವೂ
ನಿರ್ಭ-ಯತೆ
ನಿರ್ಭ-ರ-ರಾಗಿ
ನಿರ್ಮ-ಲ-ವಾಗಿ
ನಿರ್ಮ-ಲ-ವಾ-ಗಿದೆ
ನಿರ್ಮಾಣ
ನಿರ್ಮಾ-ಣ-ಗೊಂಡ
ನಿರ್ಮಾ-ಣ-ಗೊ-ಳ್ಳ-ಲಿ-ದ್ದಾನೆ
ನಿರ್ಮಾ-ಣ-ಗೊ-ಳ್ಳು-ವುದು
ನಿರ್ಮಾ-ಣ-ದಲ್ಲಿ
ನಿರ್ಮಾ-ಣ-ವಾ-ಗ-ಬೇಕು
ನಿರ್ಮಾ-ಣ-ವಾ-ಗುವ
ನಿರ್ಮಾ-ಣ-ವಾ-ಯಿತು
ನಿರ್ಮಾ-ಣವೇ
ನಿರ್ಮಾ-ಲಾ-ಕಾ-ಶದ
ನಿರ್ಮಿ-ಸ-ಬೇಕು
ನಿರ್ಮಿ-ಸ-ಬೇ-ಕೆಂದು
ನಿರ್ಮಿ-ಸಿ-ದರು
ನಿರ್ಮಿ-ಸಿ-ದ-ವರು
ನಿರ್ಮಿ-ಸಿದ್ದ
ನಿರ್ಮಿ-ಸಿ-ದ್ದಾನೆ
ನಿರ್ಮಿ-ಸಿ-ದ್ದುವು
ನಿರ್ಮಿ-ಸುವ
ನಿರ್ಮೂಲ
ನಿರ್ಮೂ-ಲ-ಗೊ-ಳಿ-ಸಲು
ನಿರ್ಮೂ-ಲ-ಗೊ-ಳಿಸಿ
ನಿರ್ಮೂ-ಲನ
ನಿರ್ಮೂ-ಲ-ನ-ಕ್ಕಾಗಿ
ನಿರ್ಯಾ-ಣದ
ನಿರ್ಯಾ-ಣಾ-ನಂ-ತರ
ನಿರ್ಯೋ-ಚ-ನೆ-ಯಿಂದ
ನಿರ್ಲ-ಕ್ಷಿಸಿ
ನಿರ್ಲ-ಕ್ಷಿ-ಸಿ-ಬಿಟ್ಟ
ನಿರ್ಲಕ್ಷ್ಯ
ನಿರ್ಲ-ಕ್ಷ್ಯ-ವೇನು
ನಿರ್ಲಿ-ಪ್ತತೆ
ನಿರ್ವ-ಹ-ಣೆ-ಗಾಗಿ
ನಿರ್ವ-ಹಿಲು
ನಿರ್ವಹಿ-ಸಲು
ನಿರ್ವಹಿಸಿ-ಕೊಂಡು
ನಿರ್ವಹಿಸಿ-ಕೊಳ್ಳು-ತ್ತಾರೆ
ನಿರ್ವಹಿಸು-ತ್ತಿದ್ದುದ-ರಿಂದ
ನಿರ್ವಿ-ಕಲ್ಪ
ನಿರ್ವೇ-ದ-ಮಾ-ಯಾತ್
ನಿಲು-ವನ್ನು
ನಿಲು-ವಿಗೆ
ನಿಲುವು
ನಿಲ್ದಾ-ಣಕ್ಕೆ
ನಿಲ್ದಾ-ಣ-ದಲ್ಲಿ
ನಿಲ್ಲದೆ
ನಿಲ್ಲ-ಬೇ-ಕಾ-ಗು-ತ್ತದೆ
ನಿಲ್ಲ-ಬೇಕು
ನಿಲ್ಲ-ಲಾ-ರ-ದ-ವ-ಳಾ-ಗಿ-ದ್ದಾಳೆ
ನಿಲ್ಲ-ಲಾ-ರರು
ನಿಲ್ಲ-ಲಾರೆ
ನಿಲ್ಲ-ಲಿದೆ
ನಿಲ್ಲ-ಲಿಲ್ಲ
ನಿಲ್ಲಲೆ
ನಿಲ್ಲಿ-ಸದೆ
ನಿಲ್ಲಿ-ಸ-ಲಾ-ಗು-ತ್ತಿತ್ತು
ನಿಲ್ಲಿ-ಸಲು
ನಿಲ್ಲಿ-ಸಲೇ
ನಿಲ್ಲಿಸಿ
ನಿಲ್ಲಿ-ಸಿ-ದರು
ನಿಲ್ಲಿ-ಸಿ-ದುವು
ನಿಲ್ಲಿ-ಸಿ-ದ್ದ-ರಿಂದ
ನಿಲ್ಲಿ-ಸಿದ್ದು
ನಿಲ್ಲಿ-ಸಿ-ಬಿ-ಟ್ಟರು
ನಿಲ್ಲಿ-ಸಿ-ಬಿಡು
ನಿಲ್ಲಿ-ಸಿಲ್ಲ
ನಿಲ್ಲಿಸು
ನಿಲ್ಲಿ-ಸು-ತ್ತಾನೋ
ನಿಲ್ಲಿ-ಸು-ತ್ತಿದ್ದ
ನಿಲ್ಲಿ-ಸು-ವಂತೆ
ನಿಲ್ಲು-ತ್ತ-ದೆಯೋ
ನಿಲ್ಲು-ತ್ತವೆ
ನಿಲ್ಲು-ತ್ತಿತ್ತು
ನಿಲ್ಲು-ತ್ತಿ-ದ್ದರು
ನಿಲ್ಲು-ತ್ತಿರ
ನಿಲ್ಲುವ
ನಿಲ್ಲು-ವಂಥ
ನಿಲ್ಲು-ವುದನ್ನು
ನಿಲ್ಲು-ವು-ದರ
ನಿವ-ರ್ತಂತೇ
ನಿವಾ-ರಣೆ
ನಿವಾ-ರಿ-ಸಿಕೊ
ನಿವೃತ್ತ
ನಿವೃತ್ತಿ
ನಿವೃ-ತ್ತಿ-ಪ-ರ-ವಾದ
ನಿಶ್ಚಯ
ನಿಶ್ಚ-ಯಕ್ಕೆ
ನಿಶ್ಚ-ಯ-ವಾಗಿ
ನಿಶ್ಚ-ಯ-ವಾ-ಗಿದ್ದ
ನಿಶ್ಚ-ಯ-ವಾದ
ನಿಶ್ಚ-ಯ-ವಾ-ಯಿ-ತಲ್ಲ
ನಿಶ್ಚ-ಯ-ವಿ-ರ-ಲಿಲ್ಲ
ನಿಶ್ಚ-ಯಿಸಿ
ನಿಶ್ಚ-ಯಿ-ಸಿದ
ನಿಶ್ಚ-ಯಿ-ಸಿ-ದರು
ನಿಶ್ಚ-ಯಿ-ಸಿ-ದರೆ
ನಿಶ್ಚ-ಯಿ-ಸಿದ್ದ
ನಿಶ್ಚ-ಯಿ-ಸು-ತ್ತಿ-ದ್ದರು
ನಿಶ್ಚಲ
ನಿಶ್ಚಿಂ-ತ-ರಾ-ಗಿ-ದ್ದು-ಬಿ-ಟ್ಟಿ-ದ್ದರು
ನಿಶ್ಚಿಂ-ತೆಯ
ನಿಶ್ಚಿತ
ನಿಶ್ಚಿ-ತ-ವಾದ
ನಿಶ್ಚಿ-ತಾ-ಭಿ-ಪ್ರಾಯ
ನಿಶ್ಚೇ-ಷ್ಟಿತ
ನಿಶ್ಚೇ-ಷ್ಟಿ-ತ-ನಾಗಿ
ನಿಶ್ಚೇ-ಷ್ಟಿ-ತ-ರಾಗಿ
ನಿಶ್ಶಂ-ಕೆ-ಯಿಂದ
ನಿಶ್ಶ-ಕ್ತ-ರಾಗಿ
ನಿಶ್ಶ-ಕ್ತ-ರಾ-ಗಿ-ಬಿ-ಟ್ಟಿ-ದ್ದರು
ನಿಶ್ಶಕ್ತಿ-ಯಿಂದ
ನಿಶ್ಶಕ್ತಿ-ಯೆಲ್ಲ
ನಿಷತ್ತು
ನಿಷ-ದ್ಧ-ವೆ-ನಿ-ಸಿ-ಕೊಂಡು
ನಿಷಿದ್ಧ
ನಿಷ್ಕ-ರು-ಣಿ-ಯಾ-ದೆಯಾ
ನಿಷ್ಕ-ಲ್ಮಶ
ನಿಷ್ಕ-ಲ್ಮ-ಶ-ವಾದ
ನಿಷ್ಕ-ಲ್ಮಷ
ನಿಷ್ಕ-ಳಂಕ
ನಿಷ್ಕಾಮ
ನಿಷ್ಕಾ-ಮ-ಕರ್ಮ
ನಿಷ್ಠು-ರ-ತೆ-ಗಳು
ನಿಷ್ಠೆ
ನಿಷ್ಠೆ-ಯನ್ನು
ನಿಷ್ಠೆ-ಯನ್ನೇ
ನಿಷ್ಠೆ-ಯಿಂದ
ನಿಷ್ಣಾತ
ನಿಷ್ಪ್ರ-ಯೋ-ಜಕ
ನಿಷ್ಪ್ರ-ಯೋ-ಜ-ಕ-ನಂತೆ
ನಿಷ್ಪ್ರ-ಯೋ-ಜ-ಕ-ರಾ-ಗು-ತ್ತಾರೆ
ನಿಸ-ಬ-ಹುದು
ನಿಸ್ವಾರ್ಥ
ನಿಸ್ಸಂ-ಕೋ-ಚ-ವಾಗಿ
ನಿಸ್ಸಂ-ದೇ-ಹ-ವಾಗಿ
ನಿಸ್ಸಂ-ಶಯ
ನಿಸ್ಸ-ಹಾ-ಯ-ಕ-ರಾ-ದರು
ನಿಸ್ಸೀಮ
ನಿಸ್ಸೀ-ಮ-ನಾದ
ನೀ
ನೀಗಲಿ
ನೀಚ
ನೀಜಿ
ನೀಡ
ನೀಡ-ತೊ-ಡ-ಗಿ-ದ್ದರು
ನೀಡ-ಲಾರ
ನೀಡ-ಲಾ-ರಂ-ಭಿ-ಸಿ-ದರು
ನೀಡಲು
ನೀಡಿ
ನೀಡಿತು
ನೀಡಿದ
ನೀಡಿ-ದರು
ನೀಡಿ-ದರೋ
ನೀಡಿ-ದಳು
ನೀಡಿ-ದ-ವರೂ
ನೀಡಿ-ದಾಗ
ನೀಡಿ-ದುವು
ನೀಡಿದ್ದ
ನೀಡಿ-ದ್ದ-ರಿಂದ
ನೀಡಿ-ದ್ದಳು
ನೀಡಿ-ದ್ದಾರೆ
ನೀಡಿ-ದ್ದಾ-ರೆಂ-ಬುದು
ನೀಡು
ನೀಡು-ತ್ತಿತ್ತು
ನೀಡು-ತ್ತಿದ್ದ
ನೀಡು-ತ್ತಿ-ದ್ದರು
ನೀಡು-ತ್ತಿ-ರುವ
ನೀಡುವ
ನೀಡು-ವಂ-ತಾ-ಗ-ಬೇಕು
ನೀಡು-ವಂತೆ
ನೀಡು-ವಂಥ
ನೀಡು-ವ-ವರು
ನೀಡು-ವು-ದ-ಕ್ಕಾಗಿ
ನೀಡು-ವು-ದಾಗಿ
ನೀಡು-ವು-ದೆಂ-ದ-ರೇನು
ನೀಡೆಲೈ
ನೀತಿ
ನೀತಿ
ನೀತಿ-ಕ-ತೆ-ಗಳನ್ನು
ನೀತಿ-ತ-ತ್ತ್ವ-ವನ್ನು
ನೀತಿ-ಪರ
ನೀತಿಯ
ನೀತಿ-ವಂ-ತ-ನಾ-ಗಿ-ದ್ದಲ್ಲಿ
ನೀತಿ-ವಂ-ತ-ನಾ-ಗಿರ
ನೀನ-ದನ್ನು
ನೀನ-ರ-ಸ-ದಿರು
ನೀನಿನ್ನೂ
ನೀನಿರು
ನೀನೀಗ
ನೀನು
ನೀನು-ಗ-ಳ-ಳಿದು
ನೀನೂ
ನೀನೆಂ-ದೆಂದು
ನೀನೆನ್ನ
ನೀನೆ-ನ್ನುವ
ನೀನೇ
ನೀನೇಕೆ
ನೀನೇನು
ನೀನೇನೂ
ನೀನೇನೋ
ನೀನೊಂದು
ನೀನೊಬ್ಬ
ನೀನೊ-ಬ್ಬನೇ
ನೀರ
ನೀರ-ನ್ನಾ-ದರೂ
ನೀರನ್ನು
ನೀರವ
ನೀರ-ವತೆ
ನೀರ-ವ-ತೆ-ಯನ್ನು
ನೀರ-ವ-ವಾ-ಗಿದೆ
ನೀರಸ
ನೀರ-ಸ-ನಿ-ರ-ರ್ಥಕ
ನೀರ-ಸ-ವಾಗಿ
ನೀರಾ-ಗು-ತ್ತದೆ
ನೀರಿ
ನೀರಿಗೆ
ನೀರಿನ
ನೀರಿ-ನೊ-ಳಗೇ
ನೀರು
ನೀರು-ಹಾ-ವಲ್ಲ
ನೀರು-ಹಾವು
ನೀರೂ
ನೀರೆಲ್ಲ
ನೀರೊ-ಳಗೆ
ನೀವ-ಲ್ಲದೆ
ನೀವಾರೂ
ನೀವಿನ್ನು
ನೀವಿನ್ನೂ
ನೀವಿ-ರದೆ
ನೀವಿರು
ನೀವು
ನೀವು-ಗಳು
ನೀವೂ
ನೀವೆ-ನ್ನು-ವುದು
ನೀವೆಲ್ಲ
ನೀವೆ-ಲ್ಲರೂ
ನೀವೇ
ನೀವೇಕೆ
ನೀವೇಕೋ
ನೀವೇನೂ
ನೀವೊ-ಬ್ಬರೇ
ನುಂಗಲೂ
ನುಂಗಿ
ನುಂಗಿ-ಕೊಂಡು
ನುಗ್ಗಿ
ನುಗ್ಗಿ-ದ-ನ-ಲ್ಲದೆ
ನುಗ್ಗಿ-ದ್ದ-ರಿಂದ
ನುಗ್ಗಿ-ಬಿಟ್ಟ
ನುಗ್ಗು-ತ್ತಾನೆ
ನುಗ್ಗು-ತ್ತಿದ್ದ
ನುಚ್ಚು-ನೂರು
ನುಡಿ
ನುಡಿ-ಗಳ
ನುಡಿ-ಗಳನ್ನು
ನುಡಿ-ಗ-ಳಿಗೆ
ನುಡಿ-ಗಳು
ನುಡಿದ
ನುಡಿ-ದ-ದ-ಯ-ಮಾಡಿ
ನುಡಿ-ದರು
ನುಡಿಯ
ನುಡಿ-ಯ-ದಿ-ರಲೂ
ನುಡಿ-ಯ-ಲಾ-ರೆವು
ನುಡಿ-ಯು-ತ್ತಾರೆ
ನುಡಿಯೂ
ನುಡಿ-ವನೀ
ನುಡಿಸ
ನುಡಿ-ಸು-ತ್ತಿದ್ದ
ನುಡಿ-ಸು-ವ-ವನೂ
ನುಡಿ-ಸು-ವುದನ್ನು
ನುಡಿ-ಸು-ವು-ದ-ರಲ್ಲಿ
ನುಭ-ವ-ಗಳು
ನುಭ-ವದ
ನುರಿತ
ನುರಿ-ತ-ವನು
ನುರಿ-ತ-ವರು
ನುರಿ-ತ-ವಳು
ನುಸು-ಳಿದ
ನೂಕಲಿ
ನೂತನ
ನೂರಕ್ಕೆ
ನೂರಾಗಿ
ನೂರಾರು
ನೂರು
ನೂರು-ಪಟ್ಟು
ನೂರೆಂಟು
ನೂರ್ಮ-ಡಿ-ಗೊಂ-ಡಿತು
ನೃತ್ಯ
ನೃತ್ಯ-ಕ-ಲೆ-ಯನ್ನೂ
ನೃತ್ಯ-ವನ್ನು
ನೃಸಿಂ-ಹ-ದ-ತ್ತ-ನಿಗೆ
ನೆಂಟ-ನಾದ
ನೆಂಟ-ರೆಲ್ಲ
ನೆಂದರೆ
ನೆಂದು
ನೆಂದೂ
ನೆಂಬ
ನೆಗೆ-ಯು-ವುದು
ನೆಚ್ಚಿನ
ನೆಟ್ಚಿವೆ
ನೆಟ್ಟ
ನೆಟ್ಟ-ದೃ-ಷ್ಟಿ-ಯಿಂದ
ನೆಟ್ಟ-ನೋ-ಟ-ದಿಂದ
ನೆಟ್ಟ-ವನು
ನೆಟ್ಟಿತ್ತು
ನೆಟ್ಟಿದೆ
ನೆಟ್ಟು
ನೆಟ್ಟುವು
ನೆನಪಾ
ನೆನ-ಪಾಗಿ
ನೆನ-ಪಾಗು
ನೆನ-ಪಾ-ಗು-ತ್ತದೆ
ನೆನ-ಪಾ-ಗು-ತ್ತಿವೆ
ನೆನ-ಪಾ-ಯಿ-ತಂತೆ
ನೆನ-ಪಾ-ಯಿತು
ನೆನ-ಪಿಗೆ
ನೆನ-ಪಿಟ್ಟು
ನೆನ-ಪಿ-ಟ್ಟು-ಕೊಂ-ಡಿದ್ದ
ನೆನ-ಪಿ-ಟ್ಟು-ಕೊ-ಳ್ಳು-ವುದು
ನೆನ-ಪಿ-ಟ್ಟುಕೋ
ನೆನ-ಪಿದೆ
ನೆನ-ಪಿ-ದೆಯಾ
ನೆನ-ಪಿನ
ನೆನ-ಪಿ-ನ-ಲ್ಲಿ-ಟ್ಟು-ಕೊಂಡು
ನೆನ-ಪಿ-ರ-ಬೇಕು
ನೆನ-ಪಿ-ಲ್ಲವೆ
ನೆನ-ಪಿ-ಸಿ-ಕ-ಳ್ಳ-ಬೇಕು
ನೆನ-ಪಿ-ಸಿ-ಕೊಂಡು
ನೆನ-ಪಿ-ಸಿ-ಕೊ-ಡು-ತ್ತಾ-ಳೆ-ಎಲ್ಲ
ನೆನ-ಪಿ-ಸಿ-ಕೊ-ಡು-ತ್ತಿದ್ದ
ನೆನ-ಪಿ-ಸಿ-ಕೊ-ಳ್ಳ-ಬ-ಹುದು
ನೆನ-ಪಿ-ಸಿ-ದರು
ನೆನಪು
ನೆನ-ಪು-ಗಳನ್ನು
ನೆನ-ಪು-ಗಳಿಂದ
ನೆನ-ಪು-ಗಳು
ನೆನಪೇ
ನೆನೆದು
ನೆನೆ-ವೆನು
ನೆನೆ-ಸಿ-ಕೊಂ-ಡರೇ
ನೆನೆ-ಸಿ-ಕೊಂಡು
ನೆಪ
ನೆಪ-ಮಾತ್ರ
ನೆಪ-ವನ್ನು
ನೆಪೋ-ಲಿ-ಯ-ನ್ನನ
ನೆಮ್ಮ-ದಿ-ಯಿಂ-ದಿ-ದ್ದಾರೆ
ನೆಮ್ಮ-ದಿ-ಯಿಂ-ದಿ-ರು-ತ್ತಿದ್ದೆ
ನೆಮ್ಮ-ದಿಯೇ
ನೆಯ
ನೆಯದು
ನೆರ-ಳಾಟ
ನೆರ-ಳಿ-ನಲ್ಲಿ
ನೆರ-ಳಿ-ರು-ವು-ದ-ರಿಂದ
ನೆರಳು
ನೆರ-ವನ್ನು
ನೆರ-ವನ್ನೂ
ನೆರವಾ
ನೆರ-ವಾ-ಗ-ತೊ-ಡ-ಗಿದ
ನೆರ-ವಾ-ಗ-ದಿ-ದ್ದರೆ
ನೆರ-ವಾ-ಗ-ಬ-ಲ್ಲುದು
ನೆರ-ವಾ-ಗ-ಬೇ-ಕಾ-ಗಿದೆ
ನೆರ-ವಾ-ಗ-ಬೇಕು
ನೆರ-ವಾ-ಗಲು
ನೆರ-ವಾ-ಗ-ವು-ದ-ಕ್ಕಾ-ಗಿಯೇ
ನೆರ-ವಾಗು
ನೆರ-ವಾ-ಗು-ತ್ತದೆ
ನೆರ-ವಾ-ಗು-ತ್ತವೆ
ನೆರ-ವಾ-ಗು-ತ್ತಿ-ದ್ದುವು
ನೆರ-ವಾ-ಗು-ವಂ-ತಹ
ನೆರ-ವಾ-ಗು-ವು-ದ-ಕ್ಕೋ-ಸ್ಕರ
ನೆರ-ವಿಗೆ
ನೆರ-ವಿ-ಗೊ-ದಗಿ
ನೆರ-ವಿ-ನಿಂದ
ನೆರವು
ನೆರ-ವೇರಿ
ನೆರ-ವೇ-ರಿತು
ನೆರ-ವೇ-ರಿ-ಸಿದ
ನೆರ-ವೇ-ರಿ-ಸಿ-ದರು
ನೆರ-ವೇ-ರಿ-ಸು-ತ್ತಾರೆ
ನೆರ-ವೇ-ರಿ-ಸು-ವಲ್ಲಿ
ನೆರ-ವೇ-ರು-ತ್ತಿತ್ತು
ನೆರೆ-ಕೆ-ರೆ-ಯಲ್ಲಿ
ನೆರೆ-ದಿದ್ದ
ನೆರೆ-ದಿ-ದ್ದರು
ನೆರೆ-ದಿ-ದ್ದವ
ನೆರೆ-ಹೊರೆ
ನೆರೆ-ಹೊ-ರೆಯ
ನೆರೆ-ಹೊ-ರೆ-ಯ-ವರೂ
ನೆಲ
ನೆಲದ
ನೆಲ-ದೊ-ಳಗೆ
ನೆಲ-ದೊ-ಳೋ-ಡುವ
ನೆಲವೇ
ನೆಲ-ಸದೊ
ನೆಲ-ಸ-ಮಾಧಿ
ನೆಲ-ಸಿ-ದರು
ನೆಲ-ಸಿ-ಬಿ-ಟ್ಟಿ-ದ್ದರು
ನೆಲ-ಸಿ-ಲ್ಲವೋ
ನೆಲೆ
ನೆಲೆ-ಗೊಂ-ಡಿತ್ತು
ನೆಲೆ-ಗೊಂ-ಡಿ-ರು-ತ್ತವೆ
ನೆಲೆ-ಗೊ-ಳಿ-ಸಿದ
ನೆಲೆ-ಗೊ-ಳಿ-ಸಿ-ದ್ದಾರೋ
ನೆಲೆ-ಗೊ-ಳಿ-ಸುವ
ನೆಲೆ-ನಿಂತ
ನೆಲೆ-ನಿಂತು
ನೆಲೆ-ನಿಂ-ತು-ಬಿ-ಟ್ಟಿ-ದ್ದರು
ನೆಲೆ-ನಿ-ಲ್ಲ-ಬಾ-ರದು
ನೆಲೆ-ನಿ-ಲ್ಲಲು
ನೆಲೆ-ನಿ-ಲ್ಲಿ-ಸಲು
ನೆಲೆ-ನಿ-ಲ್ಲು-ವಂ-ತಾ-ಗು-ತ್ತದೆ
ನೆಲೆ-ನಿ-ಲ್ಲು-ವಂ-ತಾ-ಯಿತು
ನೆಲೆ-ಯಾ-ಗಿ-ರು-ವುದೊ
ನೆಲೆ-ವೀಡು
ನೆಲೆಸಿ
ನೆಲೆ-ಸಿದೆ
ನೆಲೆ-ಸಿ-ದ್ದಾರೆ
ನೆಲೆ-ಸಿ-ದ್ದಾಳೆ
ನೆಲೆ-ಸಿ-ರು-ವ-ವರು
ನೆಲ್ಲ
ನೆಲ್ಲಿ-ಕಾ-ಯಿ-ಯ-ಲ್ಲವೆ
ನೆವ-ದಿಂದ
ನೇ
ನೇತೃ-ತ್ವ-ದಲ್ಲಿ
ನೇತ್ರ-ದ್ವಯ
ನೇನ
ನೇಪಾ-ಳಕ್ಕೆ
ನೇಮಕ
ನೇಮ-ಕ-ವಾ-ಯಿತು
ನೇಮಿ-ಸು-ತ್ತಿ-ದ್ದಾರೆ
ನೇಯ್ದು-ಕೊ-ಳ್ಳು-ತ್ತಿ-ರು-ವುದನ್ನು
ನೇರ
ನೇರ-ವಾಗಿ
ನೇರ-ವಾದ
ನೇವ-ರಿಸಿ
ನೇವ-ರಿ-ಸುತ್ತ
ನೇವ-ರಿ-ಸು-ತ್ತಿ-ರು-ವುದನ್ನು
ನೈಜ
ನೈಜ-ವಾದ
ನೈಜ-ಸ್ವ-ರೂ-ಪ-ವನ್ನು
ನೈತಿಕ
ನೈತಿ-ಕ-ಆ-ಧ್ಯಾ-ತ್ಮಿಕ
ನೈತಿ-ಕ-ತೆಯ
ನೈತಿ-ಕ-ತೆ-ಯೆಂ-ಬುದು
ನೈತಿ-ಕ-ತೆಯೇ
ನೈನಿ-ತಾ-ಲಿಗೆ
ನೈವೇದ್ಯ
ನೈವೇ-ದ್ಯಕ್ಕೂ
ನೈವೇ-ದ್ಯಾ-ದಿ-ಗಳನ್ನು
ನೊಂದ
ನೊಂದಿಗೆ
ನೊಂದಿ-ದ್ದಾರೆ
ನೊಂದು
ನೊಂದು-ಕೊಂ-ಡಳು
ನೊಣ
ನೊಣೆ-ಯು-ತ್ತದೆ
ನೊಬ್ಬನ
ನೊಬ್ಬ-ನ-ನ್ನು-ಳಿದು
ನೊಬ್ಬ-ನಿ-ರು-ವು-ದಾ-ದರೆ
ನೊಬ್ಬನೇ
ನೋಟಕ್ಕೆ
ನೋಟದ
ನೋಟ-ದಲ್ಲಿ
ನೋಟ-ದಲ್ಲೇ
ನೋಟ-ಮಾ-ತ್ರ-ದಿಂದ
ನೋಟ-ಮಾ-ತ್ರ-ದಿಂ-ದಲೇ
ನೋಟವೇ
ನೋಡ
ನೋಡ-ದಿ-ದ್ದರೆ
ನೋಡ-ದಿರು
ನೋಡದೆ
ನೋಡ-ನೋ-ಡು-ತ್ತಿ-ದ್ದಂತೆ
ನೋಡ-ನೋ-ಡು-ತ್ತಿ-ದ್ದಂ-ತೆಯೇ
ನೋಡಪ್ಪ
ನೋಡ-ಬ-ಹುದು
ನೋಡ-ಬ-ಹುದೆ
ನೋಡ-ಬೇ-ಕಂತೆ
ನೋಡ-ಬೇ-ಕಾದ
ನೋಡ-ಬೇಕು
ನೋಡ-ಬೇ-ಕೆಂಬ
ನೋಡಯ್ಯ
ನೋಡಯ್ಯಾ
ನೋಡ-ಲಾ-ರಂ-ಭಿ-ಸಿದ
ನೋಡ-ಲಾ-ರಂ-ಭಿ-ಸಿ-ದರು
ನೋಡ-ಲಾರೆ
ನೋಡಲಿ
ನೋಡ-ಲಿ-ದ್ದೇವೆ
ನೋಡ-ಲಿ-ರು-ವಂತೆ
ನೋಡ-ಲಿಲ್ಲ
ನೋಡಲು
ನೋಡ-ಲೆಂದು
ನೋಡಲೇ
ನೋಡ-ಲೇ-ಬೇಕು
ನೋಡಿ
ನೋಡಿ-ಒಂದು
ನೋಡಿಕೊ
ನೋಡಿ-ಕೊಂಡ
ನೋಡಿ-ಕೊಂ-ಡರು
ನೋಡಿ-ಕೊಂ-ಡರೆ
ನೋಡಿ-ಕೊಂ-ಡಾಗ
ನೋಡಿ-ಕೊಂಡು
ನೋಡಿ-ಕೊಂ-ಡು-ಬ-ರಲು
ನೋಡಿ-ಕೊಳ್ಳ
ನೋಡಿ-ಕೊ-ಳ್ಳ-ತೊ-ಡ-ಗಿ-ದರು
ನೋಡಿ-ಕೊ-ಳ್ಳ-ಬೇಕಾ
ನೋಡಿ-ಕೊ-ಳ್ಳ-ಬೇ-ಕಾ-ಗಿತ್ತು
ನೋಡಿ-ಕೊ-ಳ್ಳ-ಬೇ-ಕಾ-ಗಿದ್ದ
ನೋಡಿ-ಕೊ-ಳ್ಳ-ಬೇ-ಕಾ-ದ-ವನು
ನೋಡಿ-ಕೊ-ಳ್ಳ-ಬೇಕು
ನೋಡಿ-ಕೊ-ಳ್ಳ-ಬೇ-ಕು-ತಾನು
ನೋಡಿ-ಕೊ-ಳ್ಳಲು
ನೋಡಿ-ಕೊಳ್ಳಿ
ನೋಡಿ-ಕೊ-ಳ್ಳು-ತ್ತಿದ್ದ
ನೋಡಿ-ಕೊ-ಳ್ಳು-ತ್ತಿ-ರ-ಬೇ-ಕಾ-ಗು-ತ್ತ-ದೆ-ಜಾ-ತ್ಯ-ಭಿ-ಮಾನ
ನೋಡಿ-ಕೊ-ಳ್ಳುವ
ನೋಡಿ-ಕೊ-ಳ್ಳು-ವು-ದ-ಕ್ಕೋ-ಸ್ಕ-ರವೇ
ನೋಡಿಕೋ
ನೋಡಿತು
ನೋಡಿದ
ನೋಡಿ-ದಂತೆ
ನೋಡಿ-ದ-ಅ-ತ್ಯಂತ
ನೋಡಿ-ದರು
ನೋಡಿ-ದರೂ
ನೋಡಿ-ದರೆ
ನೋಡಿ-ದರೇ
ನೋಡಿ-ದಳು
ನೋಡಿ-ದ-ವನೇ
ನೋಡಿ-ದ-ವ-ರಿ-ಗೆಲ್ಲ
ನೋಡಿ-ದ-ವರೆಲ್ಲ
ನೋಡಿ-ದಾಗ
ನೋಡಿ-ದಾ-ಗಲೂ
ನೋಡಿ-ದಾ-ಗಲೇ
ನೋಡಿ-ದಾ-ಗಿ-ನಿಂ-ದಲೂ
ನೋಡಿದೆ
ನೋಡಿ-ದೆ-ನಾನು
ನೋಡಿ-ದೆಯಾ
ನೋಡಿ-ದೆವು
ನೋಡಿ-ದ್ದ-ಅ-ವರು
ನೋಡಿ-ದ್ದಳು
ನೋಡಿದ್ದೇ
ನೋಡಿ-ದ್ದೇನೆ
ನೋಡಿ-ದ್ದೇ-ನೆ-ಆ-ದರೆ
ನೋಡಿ-ದ್ದೇವೆ
ನೋಡಿ-ಬಿಟ್ಟ
ನೋಡಿ-ಬಿ-ಡೋಣ
ನೋಡಿಯೇ
ನೋಡಿ-ಯೇ-ಬಿ-ಡ-ಬೇಕು
ನೋಡಿ-ಯೇ-ಬಿ-ಡೋಣ
ನೋಡಿ-ರ-ಬೇ-ಕು-ಒಂದು
ನೋಡಿಲ್ಲ
ನೋಡಿ-ಲ್ಲವೇ
ನೋಡಿಲ್ಲಿ
ನೋಡಿ-ಹೋ-ಗುವ
ನೋಡು
ನೋಡುತ್ತ
ನೋಡು-ತ್ತ-ನೋ-ಡುತ್ತ
ನೋಡು-ತ್ತಲೇ
ನೋಡು-ತ್ತಾನೆ
ನೋಡು-ತ್ತಾ-ನೆ-ಅದು
ನೋಡು-ತ್ತಾ-ನೆ-ಅ-ಪ-ರಿ-ಚಿತ
ನೋಡು-ತ್ತಾ-ನೆ-ನ-ರೇಂದ್ರ
ನೋಡು-ತ್ತಾ-ನೆ-ಪಕ್ಕ-ದಲ್ಲಿ
ನೋಡು-ತ್ತಾ-ನೆ-ಶ್ರೀ-ರಾ-ಮ-ಕೃ-ಷ್ಣರು
ನೋಡು-ತ್ತಾ-ನೆ-ಸೊ-ಳ್ಳೆ-ಗಳು
ನೋಡು-ತ್ತಾರೆ
ನೋಡು-ತ್ತಾ-ರೆ-ಅ-ವನು
ನೋಡು-ತ್ತಾ-ರೆ-ಅ-ವರ
ನೋಡು-ತ್ತಾ-ರೆ-ಎದೆ
ನೋಡು-ತ್ತಾ-ರೆ-ಕೌ-ಪೀನ
ನೋಡು-ತ್ತಾ-ರೆ-ತ್ರಿ-ಗು-ಣಾ-ತೀತಾ
ನೋಡು-ತ್ತಾ-ರೆ-ನ-ರೇಂದ್ರ
ನೋಡು-ತ್ತಾ-ರೆ-ಪ-ವಾ-ಹಾರಿ
ನೋಡು-ತ್ತಾ-ಳೆಆ
ನೋಡು-ತ್ತಾ-ಳೆ-ಎ-ದು-ರಿಗೆ
ನೋಡು-ತ್ತಾ-ಳೆ-ಮ-ಗನ
ನೋಡು-ತ್ತಿತ್ತು
ನೋಡು-ತ್ತಿದೆ
ನೋಡು-ತ್ತಿದ್ದ
ನೋಡು-ತ್ತಿ-ದ್ದಂ-ತೆಯೇ
ನೋಡು-ತ್ತಿ-ದ್ದರು
ನೋಡು-ತ್ತಿ-ದ್ದರೆ
ನೋಡು-ತ್ತಿ-ದ್ದಾಗ
ನೋಡು-ತ್ತಿ-ದ್ದಾನೆ
ನೋಡು-ತ್ತಿ-ದ್ದಾರೆ
ನೋಡು-ತ್ತಿ-ದ್ದೀ-ಯಲ್ಲ
ನೋಡು-ತ್ತಿ-ದ್ದೀಯೆ
ನೋಡು-ತ್ತಿದ್ದೆ
ನೋಡು-ತ್ತಿ-ದ್ದೇನೆ
ನೋಡು-ತ್ತಿ-ರ-ಲಿಲ್ಲ
ನೋಡು-ತ್ತಿರಿ
ನೋಡು-ತ್ತಿರು
ನೋಡು-ತ್ತಿ-ರು-ತ್ತೇನ
ನೋಡು-ತ್ತಿ-ರು-ವ-ವ-ನು-ಇದು
ನೋಡು-ತ್ತಿ-ರು-ವುದ
ನೋಡು-ತ್ತಿ-ರು-ವುದು
ನೋಡು-ತ್ತೇ-ನ-ಅ-ವ-ನಲ್ಲಿ
ನೋಡು-ತ್ತೇನೆ
ನೋಡು-ತ್ತೇ-ನೆ-ನ-ನ-ಗೀಗ
ನೋಡು-ತ್ತೇವೆ
ನೋಡುವ
ನೋಡು-ವಂತೆ
ನೋಡು-ವ-ವರ
ನೋಡು-ವ-ವ-ರಲ್ಲ
ನೋಡು-ವ-ವ-ರೆಗೂ
ನೋಡು-ವಷ್ಟು
ನೋಡು-ವಿರಿ
ನೋಡು-ವು-ದಕ್ಕೆ
ನೋಡು-ವುದನ್ನು
ನೋಡು-ವು-ದಿಲ್ಲ
ನೋಡು-ವುದು
ನೋಡು-ವು-ದೆಂ-ದರೇ
ನೋಡು-ವುದೇ
ನೋಡುವೆ
ನೋಡೇ-ಬಿ-ಡು-ವುದು
ನೋಡೋಣ
ನೋಡ್ರಪ್ಪ
ನೋಯಿ-ಸದೆ
ನೋಯು-ವು-ದಾ-ಗಲಿ
ನೋವನ್ನು
ನೋವಾ-ಗು-ತ್ತಿತ್ತು
ನೋವಾ-ಗು-ತ್ತಿದೆ
ನೋವಾ-ಗು-ವಂತೆ
ನೋವಾ-ಯಿತು
ನೋವಿಗೆ
ನೋವಿನ
ನೋವಿ-ನಿಂದ
ನೋವು
ನೋವು-ನ-ಲಿ-ವು-ಗಳ
ನೋವೂ
ನೌಮಿ
ನ್ನಾಗಿ-ರಿ-ಸು-ತ್ತಿದ್ದ
ನ್ನಾಡು-ತ್ತಾ-ರಲ್ಲ
ನ್ನಾಡು-ತ್ತಿದ್ದ
ನ್ನಾದರೂ
ನ್ನುಂಟು-ಮಾ-ಡಿತು
ನ್ನುತ್ತಿ-ದ್ದರು
ನ್ನೆಲ್ಲ
ನ್ಮುಖ-ರಾ-ದರು
ನ್ಯಾಯ
ನ್ಯಾಯ-ವಿ-ತ-ರಣೆ
ನ್ಯಾಯಾ-ಧೀಶ
ನ್ಯಾಯಾ-ಧೀ-ಶ-ನಾದ
ನ್ಯಾಯಾ-ಧೀ-ಶನು
ನ್ಯಾಯಾ-ಲ-ಯಕ್ಕೆ
ನ್ಯಾಯಾ-ಲ-ಯದ
ನ್ಯಾಯಾ-ಲ-ಯ-ದಲ್ಲಿ
ನ್ಯೂನ-ತೆ-ಯೆಂ-ಬು-ದಾ-ಗಲಿ
ನ್ಯೂಯಾ-ರ್ಕ್ನಲ್ಲಿ
ನ್ಯೂಹ್ಯಾಂ-ಪ್ಷೈ-ರಿನ
ಪಂಕ್ತಿ-ಪಂಕ್ತಿ
ಪಂಗ-ಡ-ಗಳ
ಪಂಗ-ಡದ
ಪಂಚ-ವಟಿ
ಪಂಚ-ವ-ಟಿಯ
ಪಂಚ-ವ-ಟಿ-ಯಲ್ಲಿ
ಪಂಚಾಂ-ಗ-ವನ್ನು
ಪಂಚೆ
ಪಂಚೆ-ಶ-ಲ್ಯ-ಗಳನ್ನು
ಪಂಜಾ-ಬಿ-ನತ್ತ
ಪಂಜಾ-ಬಿ-ನಿಂದ
ಪಂಡಿತ
ಪಂಡಿ-ತ-ನಿಗೆ
ಪಂಡಿ-ತ-ರನ್ನು
ಪಂಡಿ-ತ-ರ-ನ್ನೆಲ್ಲ
ಪಂಡಿ-ತರು
ಪಂಡಿ-ತರೂ
ಪಂಡಿ-ತ-ರೆಂ-ಬ-ರು-ಮೇಣ್
ಪಂಡಿ-ತ-ರೆ-ಲ್ಲರೂ
ಪಂಡಿ-ತರೋ
ಪಂಥ
ಪಂಥ-ವನ್ನು
ಪಂಥ-ವನ್ನೋ
ಪಂಥವು
ಪಂಥ-ವೇ-ನಾ-ಗಿ-ರ-ಲಿಲ್ಲ
ಪಂದ್ಯ
ಪಂದ್ಯ-ಗ-ಳಲ್ಲೂ
ಪಂದ್ಯ-ಗಳು
ಪಂದ್ಯ-ಗ-ಳೇನೋ
ಪಂದ್ಯಾ-ವ-ಳಿ-ಯೊಂ-ದ-ರಲ್ಲಿ
ಪಕ್ಕಕ್ಕೆ
ಪಕ್ಕದ
ಪಕ್ಕ-ದಲ್ಲಿ
ಪಕ್ಕ-ದ-ಲ್ಲಿಯೇ
ಪಕ್ಕ-ದ-ಲ್ಲಿ-ರುವ
ಪಕ್ಕ-ದಲ್ಲೇ
ಪಕ್ಕ-ವಾದ್ಯ
ಪಕ್ವ-ಗೊಂಡು
ಪಕ್ವ-ವಾ-ಗಿದೆ
ಪಕ್ಷ
ಪಕ್ಷ-ಕ್ಕೊ-ಮ್ಮೆಯೋ
ಪಕ್ಷ-ಪಾ-ತ-ಗಳನ್ನು
ಪಕ್ಷಿ-ಗ-ಳಾ-ಗಲಿ
ಪಕ್ಷಿ-ಗಳು
ಪಕ್ಷಿ-ಗಳೂ
ಪಖ-ವಾ-ಜ್-ತ-ಬ-ಲಾ-ಗಳನ್ನೂ
ಪಚ್ಚೆ
ಪಟ-ಗಳನ್ನು
ಪಟ್ಟ
ಪಟ್ಟ-ಣ-ದಲ್ಲಿ
ಪಟ್ಟರು
ಪಟ್ಟರೂ
ಪಟ್ಟಾಗಿ
ಪಟ್ಟಿ
ಪಟ್ಟಿ-ಯನ್ನು
ಪಟ್ಟು
ಪಟ್ಟು-ಕೊ-ಳ್ಳದೆ
ಪಟ್ಟು-ಬಿ-ಡದೆ
ಪಠ-ನ-ದೊಂ-ದಿಗೆ
ಪಠ್ಯೇ-ತರ
ಪಡ-ಲಿಲ್ಲ
ಪಡಿ-ಸ-ಲಾ-ಯಿತು
ಪಡಿ-ಸಿ-ಕೊಂಡ
ಪಡಿ-ಸು-ತ್ತಿದ್ದ
ಪಡಿ-ಸು-ತ್ತೇನೆ
ಪಡಿ-ಸುವ
ಪಡುತ್ತ
ಪಡು-ತ್ತಿದ್ದ
ಪಡು-ತ್ತಿ-ದ್ದರು
ಪಡು-ತ್ತಿ-ದ್ದ-ವರೂ
ಪಡು-ತ್ತಿ-ರ-ಲಿಲ್ಲ
ಪಡು-ವಂ-ತಿತ್ತು
ಪಡೆದ
ಪಡೆ-ದಂ-ತಹ
ಪಡೆ-ದದ್ದು
ಪಡೆ-ದದ್ದೇ
ಪಡೆ-ದ-ಮೇಲೆ
ಪಡೆ-ದರು
ಪಡೆ-ದರೆ
ಪಡೆ-ದ-ವನೋ
ಪಡೆ-ದ-ವರು
ಪಡೆ-ದಿದ್ದ
ಪಡೆ-ದಿದ್ದು
ಪಡೆದು
ಪಡೆ-ದು-ಕೊಂಡ
ಪಡೆ-ದು-ಕೊಂ-ಡರು
ಪಡೆ-ದು-ಕೊಂ-ಡ-ವರು
ಪಡೆ-ದು-ಕೊಂ-ಡಾ-ದ-ಮೇಲೆ
ಪಡೆ-ದು-ಕೊಂಡು
ಪಡೆ-ದು-ಕೊಂ-ಡು-ಬಿ-ಡ-ಬ-ಹು-ದೇನು
ಪಡೆ-ದು-ಕೊಂಡೇ
ಪಡೆ-ದು-ಕೊ-ಳ್ಳ-ಬೇ-ಕಾ-ದರೆ
ಪಡೆ-ದು-ಕೊ-ಳ್ಳ-ಬೇಕು
ಪಡೆ-ದು-ಕೊ-ಳ್ಳಲು
ಪಡೆ-ದು-ಕೊ-ಳ್ಳು-ತ್ತಾ-ರೆಂದು
ಪಡೆ-ದು-ಕೊ-ಳ್ಳು-ತ್ತಿ-ದ್ದರು
ಪಡೆ-ದು-ಕೊ-ಳ್ಳುವ
ಪಡೆ-ದು-ಕೊ-ಳ್ಳು-ವು-ದಾಗಿ
ಪಡೆ-ದು-ಕೊ-ಳ್ಳು-ವುದು
ಪಡೆ-ದು-ಕೊ-ಳ್ಳು-ವುದೋ
ಪಡೆ-ದು-ತ-ರು-ವಂ-ತೆಯೂ
ಪಡೆದೇ
ಪಡೆ-ಯ-ಬ-ಹುದು
ಪಡೆ-ಯ-ಬ-ಹುದೋ
ಪಡೆ-ಯ-ಬಾ-ರದು
ಪಡೆ-ಯ-ಬೇ-ಕಾ-ಗಿತ್ತು
ಪಡೆ-ಯ-ಬೇ-ಕಾ-ದರೂ
ಪಡೆ-ಯ-ಬೇ-ಕಾ-ದರೆ
ಪಡೆ-ಯ-ಬೇಕು
ಪಡೆ-ಯ-ಲಿಲ್ಲ
ಪಡೆ-ಯಲು
ಪಡೆಯು
ಪಡೆ-ಯುತ್ತ
ಪಡೆ-ಯು-ತ್ತಾನೆ
ಪಡೆ-ಯು-ತ್ತಿ-ದ್ದರು
ಪಡೆ-ಯು-ತ್ತಿ-ದ್ದಾರೋ
ಪಡೆ-ಯು-ತ್ತೇನೆ
ಪಡೆ-ಯುವ
ಪಡೆ-ಯು-ವಂ-ತಾ-ಯಿತು
ಪಡೆ-ಯು-ವಂತೆ
ಪಡೆ-ಯು-ವಿಕೆ
ಪಡೆ-ಯು-ವು-ದ-ಕ್ಕಾ-ಗು-ವು-ದಿಲ್ಲ
ಪಡೆ-ಯು-ವು-ದರ
ಪಡೆ-ವ-ರಿಲ್ಲ
ಪಣ-ತೊಟ್ಟು
ಪಣ-ವಾ-ಗಿಸಿ
ಪತಂ-ಜಲಿ
ಪತನ
ಪತಿ
ಪತಿ-ಗೃಹ
ಪತಿ-ಗೃ-ಹಕ್ಕೆ
ಪತಿ-ದೇ-ವ-ರನ್ನು
ಪತಿಯ
ಪತಿ-ಯಂತೆ
ಪತಿ-ಯನ್ನು
ಪತಿ-ವಿ-ಯೋ-ಗದ
ಪತ್ತೆ
ಪತ್ತೆ-ಹ-ಚ್ಚದೆ
ಪತ್ನಿ
ಪತ್ನಿಗೆ
ಪತ್ನಿಯ
ಪತ್ನಿ-ಯನ್ನು
ಪತ್ನಿ-ಯನ್ನೇ
ಪತ್ನಿ-ಯೊ-ಡ-ಗೂ-ಡಿದ
ಪತ್ರ
ಪತ್ರಕ್ಕೆ
ಪತ್ರ-ಗಳನ್ನು
ಪತ್ರ-ಗಳಲ್ಲಿ
ಪತ್ರ-ಗಳು
ಪತ್ರದ
ಪತ್ರ-ದಲ್ಲಿ
ಪತ್ರ-ದಿಂದ
ಪತ್ರ-ವನ್ನು
ಪತ್ರ-ವೊಂ-ದನ್ನು
ಪಥಕ್ಕೆ
ಪಥ-ಗ-ಳಲ್ಲೂ
ಪಥ-ದಲಿ
ಪಥ-ದಲ್ಲಿ
ಪಥ್ಯ-ಪಾನ
ಪಥ್ಯ-ಪಾ-ನ-ಗಳು
ಪಥ್ಯ-ಪಾ-ನದ
ಪಥ್ಯೋ-ಪ-ಚಾ-ರ-ಗ-ಳಿಗೆ
ಪದ
ಪದಕ್ಕೂ
ಪದಕ್ಕೆ
ಪದ-ಗಳನ್ನು
ಪದ-ಗಳಲ್ಲಿ
ಪದ-ಗ-ಳಷ್ಟೇ
ಪದ-ತಲ
ಪದ-ತ-ಲ-ದಲ್ಲಿ
ಪದದ
ಪದ-ದಡಿ
ಪದ-ಪ್ರ-ಯೋಗ
ಪದ-ವ-ನ್ನಾ-ದರೂ
ಪದ-ವನ್ನು
ಪದ-ವನ್ನೂ
ಪದ-ವಿ-ಪ್ರ-ಶಸ್ತಿ
ಪದ-ವಿಯ
ಪದ-ವಿ-ಯಂ-ತಹ
ಪದ-ವೀ-ಧರ
ಪದಾ-ರ್ಥ-ಗಳನ್ನು
ಪದಾ-ರ್ಥ-ಗಳನ್ನೆಲ್ಲ
ಪದಾ-ರ್ಥ-ಗ-ಳನ್ನೇ
ಪದಾ-ರ್ಥ-ಗಳು
ಪದೇ-ಪದೇ
ಪದ್ಧತಿ
ಪದ್ಧ-ತಿ-ಸಂ-ಪ್ರ-ದಾ-ಯ-ಗಳನ್ನೂ
ಪದ್ಧ-ತಿ-ಗಳನ್ನು
ಪದ್ಧ-ತಿ-ಗ-ಳು-ಎಲ್ಲ
ಪದ್ಧ-ತಿಯ
ಪದ್ಧ-ತಿ-ಯನ್ನು
ಪದ್ಮ-ಇ-ತರ
ಪದ್ಮ-ಗಳಲ್ಲಿ
ಪದ್ಮದ
ಪದ್ಮ-ದಂತೆ
ಪದ್ಮ-ಪ-ತ್ರದ
ಪದ್ಮ-ಲೋ-ಚನ
ಪದ್ಮ-ಶೇ-ಖರ್
ಪದ್ಯ-ಗಳನ್ನು
ಪಯ-ಣಿ-ಸುತ್ತ
ಪಯ-ಣಿ-ಸು-ತ್ತಿ-ದ್ದಾರೆ
ಪರ
ಪರಂ-ಪರೆ
ಪರಂ-ಪ-ರೆಯ
ಪರಂ-ಪ-ರೆ-ಯನ್ನು
ಪರಂ-ಪ-ರೆ-ಯಲ್ಲಿ
ಪರ-ಕೀ-ಯರ
ಪರ-ಕೀ-ಯರು
ಪರ-ದಾ-ಟ-ವೆಲ್ಲ
ಪರ-ದಾ-ಡ-ಬೇ-ಕಾ-ಗು-ವು-ದಿಲ್ಲ
ಪರ-ದಾಡು
ಪರ-ದಾ-ಡು-ತ್ತಾರೋ
ಪರ-ದಾ-ಡುವ
ಪರದೆ
ಪರ-ದೆ-ಗಿ-ರ-ದೆ-ಗಳನ್ನೆಲ್ಲ
ಪರ-ದೆಯ
ಪರ-ದೆ-ಯನ್ನು
ಪರ-ದೇಸಿ
ಪರ-ಬ್ರಹ್ಮ
ಪರ-ಬ್ರ-ಹ್ಮದ
ಪರ-ಬ್ರ-ಹ್ಮನ
ಪರ-ಬ್ರ-ಹ್ಮ-ನೊಂ-ದಿಗೆ
ಪರ-ಬ್ರ-ಹ್ಮ-ವ-ನ್ನು-ಳಿದು
ಪರ-ಬ್ರ-ಹ್ಮ-ವ-ಸ್ತು-ವನ್ನು
ಪರ-ಬ್ರ-ಹ್ಮ-ವ-ಸ್ತು-ವಿ-ನಲ್ಲಿ
ಪರ-ಬ್ರ-ಹ್ಮ-ವ-ಸ್ತುವು
ಪರ-ಬ್ರ-ಹ್ಮ-ವ-ಸ್ತು-ವೆಂದು
ಪರ-ಬ್ರ-ಹ್ಮ-ವ-ಸ್ತುವೇ
ಪರ-ಬ್ರ-ಹ್ಮವೇ
ಪರ-ಬ್ರ-ಹ್ಮ-ಸ್ವ-ರೂಪಿ
ಪರಮ
ಪರ-ಮ-ಗು-ರು-ವನ್ನು
ಪರ-ಮ-ತ-ಗಳ
ಪರ-ಮ-ಪ-ರಿ-ಶುದ್ಧ
ಪರ-ಮ-ಪ-ವಿತ್ರ
ಪರ-ಮ-ಪ-ವಿ-ತ್ರರೇ
ಪರ-ಮ-ಪಾ-ವ-ನ-ಕರ
ಪರ-ಮ-ಪಾ-ವ-ನ-ಕ-ರ-ವಾದ
ಪರ-ಮ-ಪಾ-ವ-ನ-ಕ-ರ-ವಾ-ದ-ದ್ದೆಂಬ
ಪರ-ಮ-ಪು-ರು-ಷಾ-ರ್ಥ-ವೆಂದು
ಪರ-ಮ-ಪು-ಷಿ-ಗಳು
ಪರ-ಮ-ಪು-ಷಿ-ಯೊ-ಬ್ಬನು
ಪರ-ಮ-ಪ್ರ-ಶಾಂತ
ಪರ-ಮ-ಪ್ರ-ಶಾಂ-ತ-ವಾ-ಗಿತ್ತು
ಪರ-ಮ-ಪ್ರೀ-ತಿಯ
ಪರ-ಮ-ಭ-ಕ್ತ-ನಾದ
ಪರ-ಮ-ಭಕ್ತೆ
ಪರ-ಮ-ಭೋ-ಗದ
ಪರ-ಮ-ಯೋ-ಗಿ-ಗ-ಳಾದ
ಪರ-ಮ-ವಾ-ತ್ಸ-ಲ್ಯ-ದಿಂದ
ಪರ-ಮ-ಶಾಂತಿ
ಪರ-ಮ-ಶ್ರೇ-ಷ್ಠ-ವಾ-ದದ್ದು
ಪರ-ಮ-ಸ-ತ್ಯದ
ಪರ-ಮ-ಸ-ತ್ಯ-ದಲ್ಲಿ
ಪರ-ಮ-ಸ-ತ್ಯ-ವನ್ನು
ಪರ-ಮ-ಸಾ-ಧನೆ
ಪರ-ಮ-ಸೌ-ಭಾ-ಗ್ಯವೇ
ಪರ-ಮ-ಹಂ-ಸರ
ಪರ-ಮ-ಹಂ-ಸ-ರಾ-ಗಿ-ಬಿ-ಡು-ತ್ತೇವೆ
ಪರ-ಮ-ಹಂ-ಸರು
ಪರ-ಮ-ಹಂ-ಸಾ-ವಸ್ಥೆ
ಪರ-ಮಾ-ಣು-ವಿ-ನಲ್ಲಿ
ಪರ-ಮಾತ್ಮ
ಪರ-ಮಾ-ತ್ಮನ
ಪರ-ಮಾ-ತ್ಮ-ನನ್ನು
ಪರ-ಮಾ-ತ್ಮ-ನಲ್ಲಿ
ಪರ-ಮಾ-ತ್ಮನೂ
ಪರ-ಮಾ-ತ್ಮ-ನೆಂಬ
ಪರ-ಮಾ-ತ್ಮ-ನೊ-ಡನೆ
ಪರ-ಮಾ-ದ-ರದ
ಪರ-ಮಾ-ದ-ರ್ಶವೇ
ಪರ-ಮಾ-ದೃ-ಷ್ಟದ
ಪರ-ಮಾ-ದ್ಭುತ
ಪರ-ಮಾ-ರ್ಥ-ಸ-ತ್ಯವೇ
ಪರ-ಮಾ-ಶ್ಚರ್ಯ
ಪರ-ಮಾ-ಶ್ಚ-ರ್ಯ-ಗೊಂಡ
ಪರ-ಮಾ-ಶ್ಚ-ರ್ಯದ
ಪರ-ಮೋಚ್ಚ
ಪರ-ಮೋ-ದಾ-ರ-ವಾದ
ಪರ-ಮೋ-ದ್ದೇಶ
ಪರ-ಮೋ-ಲ್ಲಾ-ಸ-ದಿಂದ
ಪರ-ರಾ-ಗ-ಬೇಕು
ಪರ-ರಾ-ಜ್ಯ-ವನ್ನು
ಪರ-ರಾ-ಷ್ಟ್ರ-ಗಳ
ಪರ-ರಾ-ಷ್ಟ್ರೀ-ಯರ
ಪರ-ವಶ
ಪರ-ವಾಗಿ
ಪರ-ವಾ-ಗಿಲ್ಲ
ಪರ-ವಾದ
ಪರ-ವಾ-ದರು
ಪರವು
ಪರ-ವೂ-ರು-ಗ-ಳಿಗೆ
ಪರ-ಸ್ಪರ
ಪರ-ಸ್ಪ-ರರ
ಪರ-ಸ್ವರ
ಪರ-ಹಿತ
ಪರ-ಹಿ-ತ-ಕ್ಕಾಗಿ
ಪರ-ಹಿ-ತ-ದೃ-ಷ್ಟಿ-ಯನ್ನು
ಪರಾ-ಕಾ-ಷ್ಠೆಗೆ
ಪರಾ-ಕಾ-ಷ್ಠೆ-ಯನ್ನು
ಪರಾ-ಕಾ-ಷ್ಠೆ-ಯೆಂ-ಬಂತೆ
ಪರಾ-ಕಾ-ಷ್ಠೆಯೇ
ಪರಾ-ಕ್ರಮ
ಪರಾ-ಕ್ರ-ಮ-ಗಳಿಂದ
ಪರಾ-ಕ್ರ-ಮ-ದಿಂದ
ಪರಾ-ಕ್ರ-ಮಿ-ಯ-ಲ್ಲಿವೆ
ಪರಾ-ಜ-ಯ-ಗೊ-ಳಿ-ಸಿದ
ಪರಾ-ಧೀ-ನತೆ
ಪರಾರಿ
ಪರಾ-ರಿ-ಯಾ-ದರು
ಪರಿ
ಪರಿ-ಗ-ಣಿಸಿ
ಪರಿ-ಗ-ಣಿ-ಸಿ-ದರು
ಪರಿ-ಗ-ಣಿ-ಸಿ-ದ-ವನು
ಪರಿ-ಗ-ಣಿ-ಸಿ-ದ-ವ-ನೇ-ನಲ್ಲ
ಪರಿ-ಗ-ಣಿ-ಸಿ-ದ-ವರೂ
ಪರಿ-ಗ-ಣಿ-ಸಿ-ರುವ
ಪರಿ-ಗ-ಳೆಷ್ಟು
ಪರಿ-ಗ್ರ-ಹದ
ಪರಿ-ಗ್ರ-ಹಿ-ಸಲು
ಪರಿ-ಗ್ರ-ಹಿಸಿ
ಪರಿ-ಚಯ
ಪರಿ-ಚ-ಯ-ಪ-ತ್ರ-ವನ್ನೂ
ಪರಿ-ಚ-ಯ-ವಾ-ಗ-ಬೇ-ಕಾ-ಗಿದೆ
ಪರಿ-ಚ-ಯ-ವಾ-ಗಿತ್ತು
ಪರಿ-ಚ-ಯ-ವಾ-ಗಿದ್ದು
ಪರಿ-ಚ-ಯ-ವಾ-ಗು-ತ್ತದೆ
ಪರಿ-ಚ-ಯ-ವಾದ
ಪರಿ-ಚ-ಯ-ವಾ-ದಂತೆ
ಪರಿ-ಚ-ಯ-ವಾ-ಯಿತು
ಪರಿ-ಚ-ಯ-ವಿ-ರುವ
ಪರಿ-ಚ-ಯವೂ
ಪರಿ-ಚ-ಯವೇ
ಪರಿ-ಚ-ಯ-ಸ್ಥ-ನೊ-ಬ್ಬನು
ಪರಿ-ಚ-ಯ-ಸ್ಥರ
ಪರಿ-ಚ-ಯಿಸಿ
ಪರಿ-ಚ-ಯಿ-ಸಿ-ಕೊಟ್ಟ
ಪರಿ-ಚಾ-ರ-ಕ-ರಿಗೆ
ಪರಿ-ಚಾ-ರ-ಕರು
ಪರಿ-ಚಿ-ತ-ನಾದ
ಪರಿ-ಚಿ-ತರು
ಪರಿ-ಚಿ-ತ-ರು-ಅ-ಪ-ರಿ-ಚಿ-ತರು
ಪರಿ-ಜ್ಞಾನ
ಪರಿ-ಜ್ಞಾ-ನ-ವನ್ನೂ
ಪರಿ-ಣ-ತ-ನ-ನ್ನಾಗಿ
ಪರಿ-ಣ-ತ-ರಾದ
ಪರಿ-ಣತಿ
ಪರಿ-ಣಮಿ
ಪರಿ-ಣ-ಮಿ-ಸ-ಬ-ಲ್ಲುದು
ಪರಿ-ಣ-ಮಿಸಿ
ಪರಿ-ಣ-ಮಿ-ಸಿತು
ಪರಿ-ಣ-ಮಿ-ಸಿತ್ತು
ಪರಿ-ಣ-ಮಿ-ಸಿ-ದು-ದನ್ನು
ಪರಿ-ಣ-ಮಿ-ಸಿದೆ
ಪರಿ-ಣ-ಮಿ-ಸು-ತ್ತದೆ
ಪರಿ-ಣಾಮ
ಪರಿ-ಣಾ-ಮ-ಕಾರಿ
ಪರಿ-ಣಾ-ಮ-ಕಾ-ರಿ-ಯಾ-ಗದೆ
ಪರಿ-ಣಾ-ಮ-ದಿಂ-ದಾಗಿ
ಪರಿ-ಣಾ-ಮ-ವನ್ನು
ಪರಿ-ಣಾ-ಮ-ವ-ನ್ನುಂ-ಟು-ಮಾ-ಡಿತು
ಪರಿ-ಣಾ-ಮ-ವ-ನ್ನುಂ-ಟು-ಮಾ-ಡಿ-ದುವು
ಪರಿ-ಣಾ-ಮ-ವಾಗಿ
ಪರಿ-ಣಾ-ಮ-ವುಂಟಾ
ಪರಿ-ಣಾ-ಮ-ವುಂ-ಟಾ-ದು-ದ-ರಲ್ಲಿ
ಪರಿ-ಣಾ-ಮ-ವೇ-ನಾ-ದೀ-ತೆಂದು
ಪರಿ-ತ-ಪಿ-ಸಿ-ದರೊ
ಪರಿ-ತಾಪ
ಪರಿ-ತ್ಯಾಗ
ಪರಿ-ತ್ಯಾಗಿ
ಪರಿ-ತ್ಯಾ-ಗಿ-ಗ-ಳಾದ
ಪರಿ-ತ್ಯಾ-ಗಿ-ಯಾದ
ಪರಿ-ಧಿ-ಯಾ-ಚೆ-ಗಿನ
ಪರಿ-ನಿ-ರ್ಯಾಣ
ಪರಿ-ಪ-ರಿ-ಯಾಗಿ
ಪರಿ-ಪಾಕ
ಪರಿ-ಪಾ-ಕ-ಗೊಂಡು
ಪರಿ-ಪಾಠ
ಪರಿ-ಪಾ-ಠ-ವಿದೆ
ಪರಿ-ಪಾ-ಲಿ-ಸಿ-ಕೊಂಡು
ಪರಿ-ಪಾ-ಲಿ-ಸಿ-ದ-ರು-ಎ-ರಡು
ಪರಿ-ಪಾ-ಲಿಸು
ಪರಿ-ಪಾ-ಲಿ-ಸುತ್ತ
ಪರಿ-ಪೂರ್ಣ
ಪರಿ-ಪೂ-ರ್ಣ-ಗೊ-ಳಿ-ಸುವು
ಪರಿ-ಪೂ-ರ್ಣ-ಜ್ಞಾ-ನ-ವನ್ನು
ಪರಿ-ಪೂ-ರ್ಣತೆ
ಪರಿ-ಪೂ-ರ್ಣ-ತೆಯ
ಪರಿ-ಪೂ-ರ್ಣ-ನಲ್ಲ
ಪರಿ-ಪೂ-ರ್ಣ-ರಾ-ಗಿ-ಲ್ಲ-ವೆಂದು
ಪರಿ-ಪೂ-ರ್ಣ-ವಾಗಿ
ಪರಿ-ಪೂ-ರ್ಣ-ವಾ-ಗಿ-ರುವು
ಪರಿ-ಪೂ-ರ್ಣ-ವಾದ
ಪರಿ-ಪೂ-ರ್ಣ-ವಾ-ದ-ದ್ದಾ-ಗಿದೆ
ಪರಿ-ಪೂ-ರ್ಣ-ವಾ-ದ-ವು-ಗಳು
ಪರಿ-ಪ್ರಾ-ಜ-ಕ-ನಾಗಿ
ಪರಿ-ಭಾ-ವಿ-ಸಿ-ಸ-ದ್ದ-ರಲ್ಲೂ
ಪರಿ-ಭಾ-ವಿಸು
ಪರಿ-ಮ-ಳ-ವನ್ನು
ಪರಿ-ಮಾ-ಣ-ದ-ಲ್ಲಿ-ದ್ದರೆ
ಪರಿ-ಮಿ-ತಿ-ಯನ್ನು
ಪರಿ-ಮಿ-ತಿ-ಯೊ-ಳಗೇ
ಪರಿ-ಯನ್ನು
ಪರಿ-ಯಾಗಿ
ಪರಿ-ಯಿಂದ
ಪರಿ-ಯೆಂತು
ಪರಿ-ಯೇನು
ಪರಿ-ವ-ರ್ತನೆ
ಪರಿ-ವ-ರ್ತ-ನೆ-ಇ-ವೆಲ್ಲ
ಪರಿ-ವ-ರ್ತ-ನೆ-ಗಳ
ಪರಿ-ವ-ರ್ತ-ನೆ-ಗೊಂಡ
ಪರಿ-ವ-ರ್ತ-ನೆ-ಗೊಂ-ಡುವು
ಪರಿ-ವ-ರ್ತ-ನೆ-ಯನ್ನು
ಪರಿ-ವ-ರ್ತ-ನೆ-ಯಾ-ಗು-ತ್ತದೆ
ಪರಿ-ವ-ರ್ತ-ನೆ-ಯುಂಟಾ
ಪರಿ-ವ-ರ್ತ-ನೆ-ಯುಂ-ಟಾ-ಗಿದೆ
ಪರಿ-ವ-ರ್ತಿ-ತ-ವಾ-ಗಿತ್ತು
ಪರಿ-ವ-ರ್ತಿ-ತ-ವಾ-ಯಿತು
ಪರಿ-ವ-ರ್ತಿ-ಸ-ಬಲ್ಲ
ಪರಿ-ವ-ರ್ತಿ-ಸ-ಬೇ-ಕಾ-ದರೆ
ಪರಿ-ವ-ರ್ತಿಸಿ
ಪರಿ-ವ-ರ್ತಿ-ಸು-ತ್ತಿ-ದ್ದರು
ಪರಿವೆ
ಪರಿ-ವೆ-ಯಾ-ದರೂ
ಪರಿ-ವೆಯೂ
ಪರಿ-ವೆಯೇ
ಪರಿ-ವ್ರ-ಜನ
ಪರಿ-ವ್ರ-ಜ-ನದ
ಪರಿ-ವ್ರಾ-ಜಕ
ಪರಿ-ವ್ರಾ-ಜ-ಕ-ನಾಗಿ
ಪರಿ-ವ್ರಾ-ಜ-ಕ-ಮ-ನೋ-ಭಾ-ವವು
ಪರಿ-ವ್ರಾ-ಜ-ಕ-ರಾಗಿ
ಪರಿ-ವ್ರಾ-ಜ-ಕ-ರಾ-ಗಿದ್ದ
ಪರಿ-ವ್ರಾ-ಜ-ಕ-ರಾ-ಗಿಯೇ
ಪರಿ-ವ್ರಾ-ಜ-ಕರು
ಪರಿ-ಶೀ-ಲನಾ
ಪರಿ-ಶೀ-ಲಿಸಿ
ಪರಿ-ಶುದ್ಧ
ಪರಿ-ಶು-ದ್ಧ
ಪರಿ-ಶು-ದ್ಧ-ಗೊ-ಳಿ-ಸಿ-ಕೊ-ಳ್ಳು-ತ್ತಾನೆ
ಪರಿ-ಶು-ದ್ಧ-ನಾ-ಗ-ಬೇಕು
ಪರಿ-ಶು-ದ್ಧ-ನಾ-ಗಿ-ರ-ಬೇಕು
ಪರಿ-ಶು-ದ್ಧ-ನಾ-ಗಿ-ರು-ತ್ತಾ-ನೆಯೋ
ಪರಿ-ಶು-ದ್ಧ-ವಾ-ಗಿ-ದೆಯೋ
ಪರಿ-ಶು-ದ್ಧ-ವಾ-ಗಿ-ಬಿ-ಟ್ಟಿದೆ
ಪರಿ-ಶು-ದ್ಧ-ವಾದ
ಪರಿ-ಶು-ದ್ಧಾತ್ಮ
ಪರಿ-ಶು-ದ್ಧಾ-ತ್ಮ-ರಲ್ಲಿ
ಪರಿ-ಶು-ದ್ಧಾ-ತ್ಮರು
ಪರಿ-ಶೋ-ಧಿಸಿ
ಪರಿ-ಶ್ರಮ
ಪರಿ-ಶ್ರ-ಮದ
ಪರಿ-ಶ್ರ-ಮ-ದಿಂದ
ಪರಿ-ಸ-ಮಾ-ಪ್ತಿಯ
ಪರಿ-ಸ-ರಕ್ಕೆ
ಪರಿ-ಸ-ರ-ಗ-ಳು-ಅ-ವನ
ಪರಿ-ಸ-ರದ
ಪರಿ-ಸ-ರ-ದಲ್ಲಿ
ಪರಿ-ಸ-ರ-ದಲ್ಲೂ
ಪರಿ-ಸ್ಥಿತಿ
ಪರಿ-ಸ್ಥಿ-ತಿ-ಗಳನ್ನು
ಪರಿ-ಸ್ಥಿ-ತಿ-ಗಳು
ಪರಿ-ಸ್ಥಿ-ತಿ-ಗಾಗಿ
ಪರಿ-ಸ್ಥಿ-ತಿಗೂ
ಪರಿ-ಸ್ಥಿ-ತಿಗೆ
ಪರಿ-ಸ್ಥಿ-ತಿಯ
ಪರಿ-ಸ್ಥಿ-ತಿ-ಯನ್ನು
ಪರಿ-ಸ್ಥಿ-ತಿ-ಯಲ್ಲಿ
ಪರಿ-ಸ್ಥಿ-ತಿ-ಯ-ಲ್ಲಿ-ದ್ದೇನೆ
ಪರಿ-ಸ್ಥಿ-ತಿ-ಯಲ್ಲೂ
ಪರಿ-ಸ್ಥಿ-ತಿ-ಯಲ್ಲೇ
ಪರಿ-ಸ್ಥಿ-ತಿ-ಯಿಂದ
ಪರಿ-ಸ್ಥಿ-ತಿ-ಯಿತ್ತು
ಪರಿ-ಸ್ಥಿ-ತಿಯೇ
ಪರಿ-ಹ-ರಿ-ಸಿ-ಕೊಂ-ಡರು
ಪರಿ-ಹ-ರಿ-ಸಿ-ಕೊ-ಳ್ಳ-ಬ-ಹುದು
ಪರಿ-ಹ-ರಿ-ಸಿ-ಕೊ-ಳ್ಳುತ್ತ
ಪರಿ-ಹ-ರಿ-ಸಿ-ಕೊ-ಳ್ಳು-ತ್ತಿದ್ದ
ಪರಿ-ಹ-ರಿ-ಸಿ-ದ್ದಾರೆ
ಪರಿ-ಹ-ರಿ-ಸು-ವು-ದಾಗಿ
ಪರಿ-ಹಾ-ರ-ವನ್ನು
ಪರಿ-ಹಾ-ರ-ವಾ-ಗ-ಬೇ-ಕಾ-ಗಿವೆ
ಪರಿ-ಹಾ-ರ-ವಾ-ಗು-ವ-ವ-ರೆಗೂ
ಪರಿ-ಹಾ-ರ-ವಾದ
ಪರಿ-ಹಾ-ರ-ವಾ-ದುವು
ಪರಿ-ಹಾಸ
ಪರೀಕ್ಷಾ
ಪರೀ-ಕ್ಷಾ-ಕಾ-ಲ-ವಾಗಿ
ಪರೀ-ಕ್ಷಾ-ದೃ-ಷ್ಟಿ-ಯಿಂದ
ಪರೀಕ್ಷಿ
ಪರೀ-ಕ್ಷಿ-ಸಲು
ಪರೀ-ಕ್ಷಿಸಿ
ಪರೀ-ಕ್ಷಿ-ಸಿದ
ಪರೀ-ಕ್ಷಿ-ಸಿ-ದರು
ಪರೀ-ಕ್ಷಿ-ಸಿಯೇ
ಪರೀ-ಕ್ಷಿ-ಸಿ-ಯೇ-ಬಿ-ಡ-ಬೇಕು
ಪರೀ-ಕ್ಷಿ-ಸು-ತ್ತಿದ್ದ
ಪರೀ-ಕ್ಷಿ-ಸು-ತ್ತಿದ್ದೆ
ಪರೀಕ್ಷೆ
ಪರೀ-ಕ್ಷೆ-ನಿ-ರೀ-ಕ್ಷೆ-ಗ-ಳಾದ
ಪರೀ-ಕ್ಷೆ-ಗಳ
ಪರೀ-ಕ್ಷೆ-ಗಳನ್ನು
ಪರೀ-ಕ್ಷೆ-ಗಳಲ್ಲಿ
ಪರೀ-ಕ್ಷೆ-ಗ-ಳಿಗೆ
ಪರೀ-ಕ್ಷೆ-ಗ-ಳಿ-ಗೆಲ್ಲ
ಪರೀ-ಕ್ಷೆ-ಗಳು
ಪರೀ-ಕ್ಷೆ-ಗ-ಳೆಲ್ಲ
ಪರೀ-ಕ್ಷೆ-ಗಾಗಿ
ಪರೀ-ಕ್ಷೆಗೆ
ಪರೀ-ಕ್ಷೆ-ಮಾ-ಡಲು
ಪರೀ-ಕ್ಷೆ-ಮಾಡಿ
ಪರೀ-ಕ್ಷೆಯ
ಪರೀ-ಕ್ಷೆ-ಯನ್ನೇ
ಪರೀ-ಕ್ಷೆ-ಯಲ್ಲಿ
ಪರೀ-ಕ್ಷೆ-ಯಿಂ-ದಲೇ
ಪರೀ-ಕ್ಷೆ-ಯೆಂ-ದರೆ
ಪರೀಕ್ಷ್ಯ
ಪರೋ-ಕ್ಷ-ವಾಗಿ
ಪರ್ಯಾಪ್ತ
ಪರ್ಯಾ-ಯ-ವಾಗಿ
ಪರ್ವ-ಕಾಲ
ಪರ್ವತ
ಪರ್ವ-ತಕ್ಕೆ
ಪರ್ವ-ತ-ಪ್ರ-ದೇ-ಶದ
ಪರ್ವ-ತ-ಪ್ರ-ಯಾ-ಣವು
ಪರ್ವ-ತ-ವ-ನ್ನೇ-ರುವ
ಪರ್ವ-ತ-ಶಿ-ಖ-ರ-ಗಳ
ಪರ್ವ-ತ-ಶಿ-ಖ-ರ-ಗಳು
ಪರ್ವ-ತ-ಶಿ-ಖ-ರ-ಗಳೇ
ಪರ್ವ-ತ-ಶಿ-ಖ-ರ-ವೊಂ-ದ-ರಲ್ಲಿ
ಪರ್ವ-ತೋ-ಪಮ
ಪರ್ವ-ತೋ-ಪ-ಮ-ವಾದ
ಪರ್ವ-ದಿನ
ಪರ್ಷಿ-ಯನ್
ಪರ್ಸಿ
ಪಲಾ-ಯನ
ಪಲಾ-ವಿನ
ಪಲ್ಯ
ಪಲ್ಯ-ಗಳನ್ನು
ಪಲ್ಯ-ಗ-ಳಿಗೆ
ಪಲ್ಯ-ದ-ಲ್ಲಿ-ರುವ
ಪಲ್ಯ-ದಿಂದ
ಪಲ್ಲ-ಟ-ಗೊ-ಳಿ-ಸಿ-ಬಿ-ಟ್ಟಿ-ದ್ದಾನೆ
ಪಲ್ಲವಿ
ಪಳ-ಗಿ-ದ-ವನೇ
ಪಳ-ಗಿ-ಸುವ
ಪವಾಡ
ಪವಾ-ಡ-ಗಳು
ಪವಾ-ಡ-ದಂತೆ
ಪವಾ-ಡ-ದೋ-ಪಾ-ದಿ-ಯಲ್ಲಿ
ಪವಾ-ಡ-ವ-ನ್ನೇ-ನಾ-ದರೂ
ಪವಾ-ಡ-ಶ-ಕ್ತಿ-ಯನ್ನು
ಪವಾ-ಹಾರಿ
ಪವಿತ್ರ
ಪವಿ-ತ್ರ-ತೆ-ಪ-ರಿ-ಶು-ದ್ಧತೆ
ಪವಿ-ತ್ರ-ತೆಯ
ಪವಿ-ತ್ರ-ತೆ-ಯನ್ನು
ಪವಿ-ತ್ರ-ತೆ-ಯ-ಲ್ಲಾ-ಗಲಿ
ಪವಿ-ತ್ರ-ನಾ-ಗಿ-ರ-ಬೇಕು
ಪವಿ-ತ್ರ-ಳಾ-ಗಿ-ಲ್ಲದೆ
ಪಶ್ಚಾ-ತ್ತಾ-ಪ-ಪ-ರಿ-ತಾಪ
ಪಶ್ಚಾ-ತ್ತಾ-ಪ-ಗೊಂಡು
ಪಶ್ಚಾ-ತ್ತಾ-ಪದ
ಪಶ್ಚಿಮ
ಪಶ್ಚಿ-ಮದ
ಪಸ-ರಿ-ಸು-ತ್ತಾನೆ
ಪಸಾ-ಡೆ-ನ-ದಲ್ಲಿ
ಪಾಂಡ-ವರ
ಪಾಂಡಿತ್ಯ
ಪಾಂಡಿ-ತ್ಯದ
ಪಾಂಡಿ-ತ್ಯ-ದಿಂ-ದಲೂ
ಪಾಂಡಿ-ತ್ಯ-ವನ್ನೂ
ಪಾಕ
ಪಾಕ-ಶಾ-ಸ್ತ್ರದ
ಪಾಕ-ಶಾ-ಸ್ತ್ರ-ನಿ-ಪು-ಣ-ರೆಲ್ಲ
ಪಾಚಿ
ಪಾಚಿ-ಯಿಂದ
ಪಾಠ
ಪಾಠ-ಗಳನ್ನು
ಪಾಠ-ಗಳನ್ನೆಲ್ಲ
ಪಾಠ-ಗಳು
ಪಾಠ-ಗ-ಳೆ-ಲ್ಲವೂ
ಪಾಠದ
ಪಾಠ-ದಲ್ಲೂ
ಪಾಠ-ಪ-ಟ್ಟಿ-ಗಷ್ಟೇ
ಪಾಠ-ಪ-ಟ್ಟಿ-ಯ-ಲ್ಲಿ-ರುವ
ಪಾಠ-ವನ್ನು
ಪಾಠ-ವನ್ನೂ
ಪಾಠ-ವಾ-ಗಿತ್ತು
ಪಾಠ-ಶಾ-ಲೆಗೆ
ಪಾಠ-ಶಾ-ಲೆ-ಯಲ್ಲಿ
ಪಾಠ-ಶಾ-ಲೆ-ಯೊಂ-ದನ್ನು
ಪಾಡನ್ನು
ಪಾಡಿಗೆ
ಪಾಣಿ-ನಿಯ
ಪಾತಾ-ಳ-ಲೋ-ಕಕ್ಕೆ
ಪಾತ್ರ
ಪಾತ್ರ-ಧಾ-ರಿಗೆ
ಪಾತ್ರ-ಧಾ-ರಿ-ಯನ್ನು
ಪಾತ್ರ-ನಾ-ಗಿದ್ದ
ಪಾತ್ರ-ನಾದ
ಪಾತ್ರ-ನೆಂ-ಬು-ದನ್ನು
ಪಾತ್ರ-ರಾ-ಗ-ಬೇ-ಕಾ-ಗು-ತ್ತದೆ
ಪಾತ್ರ-ರಾ-ಗಿ-ದ್ದ-ವರು
ಪಾತ್ರ-ರಾ-ದಂ-ತಹ
ಪಾತ್ರ-ರಾ-ದರು
ಪಾತ್ರ-ವನ್ನು
ಪಾತ್ರ-ವಿದೆ
ಪಾತ್ರ-ವಿ-ದೆ-ಯೆಂದು
ಪಾತ್ರ-ವೇನು
ಪಾತ್ರೆ-ಪ-ಡ-ಗ-ಗಳು
ಪಾತ್ರೆಗೆ
ಪಾತ್ರೆಯ
ಪಾತ್ರೆ-ಯನ್ನು
ಪಾತ್ರೆಯೂ
ಪಾತ್ರೆ-ಯೊಂ-ದ-ರಲ್ಲಿ
ಪಾದ
ಪಾದಕ್ಕೆ
ಪಾದ-ಗಳ
ಪಾದ-ಗಳನ್ನು
ಪಾದ-ಗಳಲ್ಲಿ
ಪಾದ-ಗ-ಳಿಗೆ
ಪಾದ-ಧೂ-ಳಿ-ಯನ್ನು
ಪಾದ-ರ-ಕ್ಷೆ-ಗಳು
ಪಾದ-ರ-ಕ್ಷೆಯೂ
ಪಾದ-ರಿ-ಗಳು
ಪಾದ-ವನ್ನು
ಪಾದಿ-ಯಲ್ಲಿ
ಪಾಧೂ-ಳಿ-ಯನ್ನು
ಪಾನ-ಕದ
ಪಾನ-ಕ-ವನ್ನು
ಪಾನ-ಕ-ವಿದೆ
ಪಾನ-ಮಾ-ಡ-ಬೇ-ಕೆಂ-ಬ-ವರೆಲ್ಲ
ಪಾನಿ
ಪಾನಿ-ಹಾ-ಟಿ-ಯಲ್ಲಿ
ಪಾಪ
ಪಾಪ-ಪುಣ್ಯ
ಪಾಪ-ಪು-ಣ್ಯ-ಗಳ
ಪಾಪಕೆ
ಪಾಪ-ಗಳು
ಪಾಪದ
ಪಾಪ-ಪು-ಣ್ಯ-ಗಳನ್ನು
ಪಾಪ-ಪು-ಣ್ಯಾ-ತೀ-ತನು
ಪಾಪ-ಫಲ
ಪಾಪ-ರಾ-ಶಿ-ಯನ್ನು
ಪಾಪ-ವನ್ನು
ಪಾಪ-ವಿ-ಮೋ-ಚನೆ
ಪಾಪಿ
ಪಾಮ-ರ-ನೇನು
ಪಾರಂ-ಗ-ತ-ರಾ-ದರು
ಪಾರ-ಮಾ-ರ್ಥಿಕ
ಪಾರ-ಮಾ-ರ್ಥಿ-ಕ-ರನ್ನು
ಪಾರ-ವಿ-ಲ್ಲದ
ಪಾರವೇ
ಪಾರಸೀ
ಪಾರ-ಸೀ-ಸಂ-ಸ್ಕೃತ
ಪಾರಾ-ಗ-ಬ-ಲ್ಲೆನೆ
ಪಾರಾ-ಗ-ಬೇಕು
ಪಾರಾ-ಗಲೇ
ಪಾರಾಗಿ
ಪಾರಾ-ಗಿ-ದ್ದೇ-ನೆಯೇ
ಪಾರಾ-ಗಿ-ದ್ದೇ-ನೆಯೋ
ಪಾರಾಗು
ಪಾರಾದ
ಪಾರಿ-ವಾಳ
ಪಾರಿ-ವಾ-ಳ-ಗಳು
ಪಾರಿ-ವಾ-ಳ-ದಂತೆ
ಪಾರು
ಪಾರು-ಪತ್ಯ
ಪಾರು-ಪ-ತ್ಯ-ವನ್ನು
ಪಾರು-ಮಾ-ಡ-ಬೇಕು
ಪಾರು-ಮಾ-ಡ-ಲಾ-ರದೋ
ಪಾರು-ಮಾಡಿ
ಪಾರು-ಮಾ-ಡಿದ
ಪಾರು-ಮಾ-ಡಿ-ಬಿ-ಟ್ಟ-ನಲ್ಲ
ಪಾರ್ಥಿವ
ಪಾರ್ವ-ತೀ-ಚ-ರಣ
ಪಾರ್ಶ್ವ-ವಾಯು
ಪಾಲನೆ
ಪಾಲ-ನೆ-ಪೋ-ಷ-ಣೆ-ಯಲ್ಲೇ
ಪಾಲ-ನೆಯ
ಪಾಲಿ-ಗಂತೂ
ಪಾಲಿ-ಗಿ-ದ್ದುವು
ಪಾಲಿಗೆ
ಪಾಲಿಗೇ
ಪಾಲಿ-ಗೊಂದು
ಪಾಲಿ-ಟನ್
ಪಾಲಿನ
ಪಾಲಿ-ಸಲು
ಪಾಲಿಸಿ
ಪಾಲಿ-ಸಿ-ಕೊಂಡು
ಪಾಲಿ-ಸಿದ
ಪಾಲಿ-ಸಿ-ದರೆ
ಪಾಲಿ-ಸಿದ್ದೇ
ಪಾಲಿಸು
ಪಾಲಿ-ಸು-ತ್ತಿದ್ದ
ಪಾಲಿ-ಸು-ವುದು
ಪಾಲು
ಪಾಲು-ದಾ-ರ-ರಾಗಿ
ಪಾಲು-ದಾ-ರರು
ಪಾಲ್ಗೊಂಡು
ಪಾಲ್ನನ್ನು
ಪಾಳು-ಬಿದ್ದ
ಪಾಳು-ಬಿ-ದ್ದಿದ್ದ
ಪಾವತಿ
ಪಾವ-ನ-ಕರ
ಪಾವ-ನ-ವಾ-ಯಿತೋ
ಪಾವಿತ್ರ್ಯ
ಪಾವಿ-ತ್ರ್ಯ-ವೆಂ-ಬುದು
ಪಾಶ-ಗಳ
ಪಾಶದ
ಪಾಶ-ಮು-ಕ್ತ-ನಾಗು
ಪಾಶವ
ಪಾಶ-ವನ್ನು
ಪಾಶ್ಚಾತ್ಯ
ಪಾಶ್ಚಾ-ತ್ಯ-ಪೌ-ರ್ವಾತ್ಯ
ಪಾಶ್ಚಾ-ತ್ಯರ
ಪಾಶ್ಚಾ-ತ್ಯ-ರಿ-ಗೆಲ್ಲ
ಪಾಶ್ಚಾ-ತ್ಯರು
ಪಾಸಾ-ಗು-ತ್ತಾನೆ
ಪಾಸು
ಪಾಸು-ಮಾ-ಡಿ-ದರೆ
ಪಾಸು-ಮಾಡು
ಪಿಂಡಾಂಡ
ಪಿಂಡಾಂ-ಡ-ಗಳ
ಪಿಂಡಾಂ-ಡ-ವಾದ
ಪಿತಾ-ಮ-ಹ-ನೆಂಬ
ಪಿತ್ರಾ-ರ್ಜಿತ
ಪಿಪಾಸೆ
ಪಿಪಾ-ಸೆಯೂ
ಪಿಶಾ-ಚಿ-ಗಳು
ಪಿಸ-ಲಿಲ್ಲ
ಪಿಸು-ಗು-ಟ್ಟಿದ
ಪಿಸು-ದ-ನಿ-ಯಲ್ಲಿ
ಪೀಟರ್
ಪೀಠ
ಪೀಠದ
ಪೀಠೋ-ಪ-ಕ-ರ-ಣ-ಗಳು
ಪೀಡಿ-ತ-ವಾದ
ಪೀಡಿ-ಸಲಿ
ಪೀಡಿ-ಸಿ-ದಾ-ಗ-ಲೆಲ್ಲ
ಪೀಡಿ-ಸು-ತ್ತಿ-ದ್ದುವು
ಪೀಡಿ-ಸು-ವು-ದಿತ್ತು
ಪೀಳಿ-ಗೆ-ಯ-ವರೆಲ್ಲ
ಪುಂಗಿಯ
ಪುಕ್ಕ-ಲ್ರಯ್ಯ
ಪುಜು-ತ್ವ-ವನ್ನು
ಪುಟ
ಪುಟ-ಕ್ಕಿಂತ
ಪುಟ-ಗಳ
ಪುಟ-ಗಳನ್ನು
ಪುಟ-ಗ-ಳನ್ನೇ
ಪುಟದ
ಪುಟ-ಪು-ಟ-ಗ-ಳನ್ನೇ
ಪುಟ-ವನ್ನು
ಪುಟ-ವಿಟ್ಟ
ಪುಟ-ವೊಂ-ದನ್ನು
ಪುಟಿ-ದೆದ್ದು
ಪುಟಿ-ಯ-ಬಲ್ಲ
ಪುಟಿ-ಯು-ತ್ತಿತ್ತು
ಪುಟ್ಟ
ಪುಟ್ಟ-ಪುಟ್ಟ
ಪುಡಿ-ಪು-ಡಿ-ಮಾಡು
ಪುಡಿ-ಪು-ಡಿ-ಯಾ-ಯಿತು
ಪುಡಿ-ಯಂತೆ
ಪುಣ-ವನ್ನು
ಪುಣಿ
ಪುಣ್ಯ
ಪುಣ್ಯ-ಕಾರ್ಯ
ಪುಣ್ಯಕೆ
ಪುಣ್ಯ-ಕ್ಷೇ-ತ್ರ-ಗ-ಳಿಗೆ
ಪುಣ್ಯ-ಕ್ಷೇ-ತ್ರ-ವಾದ
ಪುಣ್ಯ-ಗ-ಳೆಂ-ಬವು
ಪುಣ್ಯ-ಫಲ
ಪುತ್ತಳಿ
ಪುತ್ಥಳಿ
ಪುತ್ರ
ಪುತ್ರ-ಪ್ರ-ಸ-ವಕ್ಕೆ
ಪುತ್ರ-ಸಂ-ತಾನ
ಪುನಃ
ಪುನ-ರಾ-ರಂ-ಭಿ-ಸಿ-ದ್ದರು
ಪುನ-ರು-ಚ್ಚ-ರಿ-ಸಿದ
ಪುನ-ರು-ತ್ಥಾನ
ಪುನ-ರು-ದ್ಧಾ-ರ-ಕರು
ಪುನ-ರ್ವಿ-ವಾ-ಹ-ವನ್ನು
ಪುನೀ-ತ-ರಾ-ದಂ-ತಹ
ಪುನೀ-ತ-ವಾ-ಗಿ-ವೆಯೋ
ಪುರ
ಪುರ-ದಲ್ಲಿ
ಪುರ-ಸ್ಕ-ರಿಸು
ಪುರ-ಸ್ಕ-ರಿ-ಸು-ವುದ
ಪುರ-ಸ್ಕಾ-ರ-ವನ್ನು
ಪುರಾಣ
ಪುರಾ-ಣದ
ಪುರಾ-ಣ-ದಲ್ಲಿ
ಪುರಾ-ಣ-ಪ್ರ-ಸಿದ್ಧ
ಪುರಾ-ತನ
ಪುರಾವೆ
ಪುರೀ-ಕ್ಷೇ-ತ್ರಕ್ಕೆ
ಪುರುಷ
ಪುರು-ಷ-ನಾ-ಗಲಿ
ಪುರು-ಷ-ನೊಬ್ಬ
ಪುರು-ಷರ
ಪುರು-ಷ-ರಾ-ಗಿ-ದ್ದರೆ
ಪುರು-ಷರು
ಪುರು-ಷ-ಸಿಂ-ಹ-ನಾದ
ಪುರು-ಷ-ಸಿಂ-ಹ-ರಾ-ಗು-ತ್ತಾರೆ
ಪುರು-ಷ-ಸ್ವ-ಭಾವ
ಪುರು-ಷೋ-ತ್ತ-ಮಾ-ನಂದ
ಪುರು-ಷೋ-ತ್ತ-ಮಾ-ನಂ-ದರು
ಪುರೋ-ಗಾಮಿ
ಪುರೋ-ಹಿ-ತರು
ಪುಳ-ಕಿ-ತ-ಗೊಂಡ
ಪುಷಿ
ಪುಷಿ-ಗಳು
ಪುಷಿ-ಪ-ರಂ-ಪ-ರೆಯ
ಪುಷಿ-ಮು-ನಿ-ಗಳ
ಪುಷಿಯ
ಪುಷಿ-ಯನ್ನು
ಪುಷ್ಕ-ಳ-ವಾಗಿ
ಪುಷ್ಟ-ವಾದ
ಪುಷ್ಟಿ-ಯಾಗಿ
ಪುಷ್ಪಾ-ರ್ಚನೆ
ಪುಷ್ಪಾ-ಲಂ-ಕಾರ
ಪುಸ್ತಕ
ಪುಸ್ತ-ಕ-ಗಳ
ಪುಸ್ತ-ಕ-ಗಳನ್ನು
ಪುಸ್ತ-ಕ-ಗಳನ್ನೆಲ್ಲ
ಪುಸ್ತ-ಕ-ಗಳಿಂದ
ಪುಸ್ತ-ಕ-ಗಳು
ಪುಸ್ತ-ಕದ
ಪುಸ್ತ-ಕ-ದಲ್ಲಿ
ಪುಸ್ತ-ಕ-ಪಾಂ-ಡಿ-ತ್ಯ-ದಂತೆ
ಪುಸ್ತ-ಕ-ವನ್ನು
ಪೂಜಾ-ಕಾ-ರ್ಯ-ಗಳನ್ನು
ಪೂಜಾ-ಕಾ-ರ್ಯ-ದಲ್ಲಿ
ಪೂಜಾ-ಗೃ-ಹಕ್ಕೆ
ಪೂಜಾ-ದಿ-ಗಳ
ಪೂಜಾ-ದಿ-ಗಳನ್ನು
ಪೂಜಾ-ದಿ-ಗಳು
ಪೂಜಾ-ದಿ-ಗ-ಳೆಲ್ಲ
ಪೂಜಿ-ಸ-ದಿ-ರಲು
ಪೂಜಿ-ಸ-ಲಾ-ಯಿತು
ಪೂಜಿ-ಸ-ಲ್ಪ-ಡು-ತ್ತಿ-ರುವ
ಪೂಜಿಸಿ
ಪೂಜಿ-ಸಿ-ದರು
ಪೂಜಿ-ಸುತ್ತ
ಪೂಜಿ-ಸು-ತ್ತಾ-ನ-ಲ್ಲವೆ
ಪೂಜಿ-ಸು-ತ್ತಾರೆ
ಪೂಜಿ-ಸು-ತ್ತಿದ್ದ
ಪೂಜಿ-ಸು-ತ್ತಿ-ದ್ದರು
ಪೂಜಿ-ಸು-ತ್ತಿ-ದ್ದೇವೆ
ಪೂಜಿ-ಸು-ವುದು
ಪೂಜೆ
ಪೂಜೆ-ಕೈಂ-ಕರ್ಯ
ಪೂಜೆ-ಧ್ಯಾ-ನ-ಗಳಲ್ಲಿ
ಪೂಜೆ-ಪ್ರಾ-ರ್ಥ-ನೆ-ಕೀ-ರ್ತ-ನೆ-ಗಳಲ್ಲಿ
ಪೂಜೆ-ಪ್ರಾ-ರ್ಥ-ನೆ-ಗಳ
ಪೂಜೆ-ಪ್ರಾ-ರ್ಥ-ನೆ-ಗಳನ್ನು
ಪೂಜೆ-ವ್ರ-ತ-ತ-ಪಸ್ಸು
ಪೂಜೆ-ಗಾಗಿ
ಪೂಜೆಗೆ
ಪೂಜೆಯ
ಪೂಜೆ-ಯನ್ನು
ಪೂಜೆ-ಯನ್ನೇ
ಪೂಜೆ-ಯಲ್ಲಿ
ಪೂಜೆಯೋ
ಪೂಜ್ಯ
ಪೂಜ್ಯ-ಪೂ-ಜ-ಕ-ರಿ-ಬ್ಬರೂ
ಪೂಜ್ಯ-ತೆಯ
ಪೂಜ್ಯ-ದೃಷ್ಟಿ
ಪೂಜ್ಯ-ಬು-ದ್ಧಿ-ಯಿಂದ
ಪೂಜ್ಯ-ಭಾ-ವ-ವಾಗಿ
ಪೂಜ್ಯರು
ಪೂರೈ-ಸ-ಬಂದ
ಪೂರೈ-ಸುವ
ಪೂರೈ-ಸು-ವುದು
ಪೂರ್ಣ
ಪೂರ್ಣ-ಗೊ-ಳಿ-ಸಿದ್ದೂ
ಪೂರ್ಣ-ಗೊ-ಳಿ-ಸು-ವುದೇ
ಪೂರ್ಣ-ಚಂದ್ರ
ಪೂರ್ಣ-ತೆ-ಯನ್ನು
ಪೂರ್ಣ-ರೂಪ
ಪೂರ್ಣ-ವಾ-ಗ-ಬೇ-ಕಾ-ದರೆ
ಪೂರ್ಣ-ವಾಗಿ
ಪೂರ್ಣ-ವಾ-ಗು-ವುದ
ಪೂರ್ಣವೋ
ಪೂರ್ಣ-ಶ್ರ-ದ್ಧೆ-ಯಿಂದ
ಪೂರ್ಣಾ-ಹು-ತಿಯ
ಪೂರ್ತಿ
ಪೂರ್ತಿ-ಯಾ-ಗಂತೂ
ಪೂರ್ತಿ-ಯಾಗಿ
ಪೂರ್ವ
ಪೂರ್ವ-ಗ್ರಹ
ಪೂರ್ವ-ಗ್ರ-ಹ-ಗ-ಳೆಂಬ
ಪೂರ್ವ-ಜ-ನ್ಮದ
ಪೂರ್ವ-ಜರು
ಪೂರ್ವ-ತ-ಯಾ-ರಿ-ಯಾಗಿ
ಪೂರ್ವ-ದಿ-ಗಂ-ತವು
ಪೂರ್ವ-ಭಾಗ
ಪೂರ್ವ-ವೃ-ತ್ತಾಂ-ತ-ವೇನು
ಪೂರ್ವ-ಸೂ-ಚಿ-ಗಳು
ಪೂರ್ವಾ
ಪೂರ್ವಾ-ಪ-ರ-ಗಳನ್ನೆಲ್ಲ
ಪೆಚ್ಚು-ಮೋರೆ
ಪೆಟ್ಟಿ-ಗೆಯ
ಪೆಟ್ಟಿ-ಗೆ-ಯ-ಲ್ಲಿಟ್ಟು
ಪೆನ್ನಿಂಗ್
ಪೆನ್ನಿಂ-ಗ್ಟನ್
ಪೇಚಿಗೆ
ಪೇಟ
ಪೇಟೆಗೋ
ಪೇರಿ-ಕೊಂಡು
ಪೇರಿ-ಸಿ-ಕೊಂಡು
ಪೈಕಿ
ಪೈಜಾ-ಮ-ಜು-ಬ್ಬ-ಗಳಲ್ಲಿ
ಪೈಜಾ-ಮ-ಜು-ಬ್ಬ-ವನ್ನೇ
ಪೈಲ-ವಾ-ನ-ನಂ-ತೆ-ಆತ
ಪೈಲ್ವಾ-ನ-ರಿಂದ
ಪೊದೆ-ಯೊಂ-ದರ
ಪೊಳ್ಳು-ತ-ನ-ವನ್ನು
ಪೋತ್ಸಾ-ಹಿ-ಸು-ತ್ತಿದ್ದ
ಪೋಷಕ
ಪೋಷಿಸಿ
ಪೌರುಷ
ಪೌರು-ಷ-ಪೂರ್ಣ
ಪೌರು-ಷ-ವಂತ
ಪೌರು-ಷ-ವನ್ನು
ಪೌರ್ವಾತ್ಯ
ಪೌಷ್ಟಿಕ
ಪೌಷ್ಟಿ-ಕಾ-ಹಾ-ರ-ವನ್ನು
ಪ್ಯಾರಾ
ಪ್ಯಾರಾ-ಗ-ಳನ್ನೇ
ಪ್ಯಾರಾದ
ಪ್ರಕಟ
ಪ್ರಕ-ಟ-ಗೊಂ-ಡಿ-ದ್ದಾನೆ
ಪ್ರಕ-ಟ-ಗೊ-ಳ್ಳ-ಲಿ-ರು-ವುದನ್ನು
ಪ್ರಕ-ಟ-ಗೊ-ಳ್ಳಲು
ಪ್ರಕ-ಟ-ಗೊ-ಳ್ಳು-ತ್ತವೆ
ಪ್ರಕ-ಟ-ಪ-ಡಿ-ಸಿ-ದ್ದಾನೆ
ಪ್ರಕ-ಟ-ಪ-ಡಿ-ಸು-ತ್ತಾರೆ
ಪ್ರಕ-ಟ-ರಾಗಿ
ಪ್ರಕ-ಟ-ವಾಗಿ
ಪ್ರಕ-ಟ-ವಾ-ಗಿ-ದ್ದೇನೆ
ಪ್ರಕ-ಟ-ವಾ-ಗು-ತ್ತಿತ್ತು
ಪ್ರಕ-ಟ-ವಾ-ಗು-ತ್ತಿದ್ದ
ಪ್ರಕ-ಟ-ವಾ-ಗು-ವುದನ್ನು
ಪ್ರಕ-ಟಿ-ಸಲು
ಪ್ರಕ-ಟಿ-ಸಿದ
ಪ್ರಕ-ಟಿ-ಸುವು
ಪ್ರಕಾರ
ಪ್ರಕಾ-ರವೇ
ಪ್ರಕಾಶ
ಪ್ರಕಾ-ಶ-ಕರ
ಪ್ರಕಾ-ಶ-ಕರು
ಪ್ರಕಾ-ಶ-ದಿಂದ
ಪ್ರಕಾ-ಶ-ಮಾ-ನ-ವಾಗಿ
ಪ್ರಕಾ-ಶ-ವನ್ನು
ಪ್ರಕಾ-ಶ-ವಾ-ಗು-ತ್ತದೆ
ಪ್ರಕಾ-ಶವು
ಪ್ರಕಾ-ಶಿ-ತ-ವಾಗು
ಪ್ರಕಾ-ಶಿ-ಸ-ಲಿ-ರು-ವುದನ್ನು
ಪ್ರಕಾ-ಶಿ-ಸು-ತ್ತಿ-ದ್ದಾಳೆ
ಪ್ರಕೃತ
ಪ್ರಕೃತಿ
ಪ್ರಕೃ-ತಿ-ಪು-ರು-ಷರ
ಪ್ರಕೃ-ತಿಗೆ
ಪ್ರಕೃ-ತಿ-ಜೀ-ವರ
ಪ್ರಕೃ-ತಿ-ಪೂಜೆ
ಪ್ರಕೃ-ತಿಯ
ಪ್ರಕೃ-ತಿ-ಯನ್ನು
ಪ್ರಕೃ-ತಿ-ಸ್ಥ-ನಾ-ಗ-ಬೇಕು
ಪ್ರಕೃ-ತಿ-ಸ್ಥ-ರಾಗಿ
ಪ್ರಕ್ರಿಯೆ
ಪ್ರಕ್ರಿ-ಯೆ-ಯನ್ನು
ಪ್ರಕ್ಷು-ಬ್ಧ-ತೆ-ಯನ್ನೂ
ಪ್ರಕ್ಷು-ಬ್ಧ-ವಾ-ಗಿ-ಸಿತ್ತು
ಪ್ರಖರ
ಪ್ರಖ-ರ-ಗೊ-ಳಿ-ಸು-ತ್ತಿದ್ದ
ಪ್ರಖ-ರ-ವಾ-ಗಿತ್ತು
ಪ್ರಖ-ರ-ವಾದ
ಪ್ರಖ್ಯಾತ
ಪ್ರಖ್ಯಾ-ತ-ಶ್ರೀ-ಮಂತ
ಪ್ರಗತಿ
ಪ್ರಗ-ತಿ-ಅ-ಪ-ಗ-ತಿ-ಗಳ
ಪ್ರಗ-ತಿಗೆ
ಪ್ರಗ-ತಿ-ಪರ
ಪ್ರಗ-ತಿಯ
ಪ್ರಗ-ತಿ-ಯನ್ನು
ಪ್ರಚಂಡ
ಪ್ರಚಂ-ಡ-ವಾದ
ಪ್ರಚ-ಲಿ-ತ-ವಾ-ಗ-ತೊ-ಡ-ಗಿತು
ಪ್ರಚ-ಲಿ-ತ-ವಾ-ಗಿದೆ
ಪ್ರಚ-ಲಿ-ತ-ವಿತ್ತು
ಪ್ರಚ-ಲಿ-ತ-ವಿದ್ದ
ಪ್ರಚ-ಲಿ-ತ-ವಿ-ರುವ
ಪ್ರಚಾರ
ಪ್ರಚಾ-ರ-ಕರ
ಪ್ರಚಾ-ರ-ಕರು
ಪ್ರಚಾ-ರ-ವಾ-ಗಿ-ಬಿ-ಟ್ಟಿತ್ತು
ಪ್ರಚಾ-ರ-ವಾ-ಯಿತು
ಪ್ರಚೋ-ದ-ಕ-ವಾಗಿ
ಪ್ರಚೋ-ದಿಸಿ
ಪ್ರಚೋ-ದಿ-ಸಿದ
ಪ್ರಚೋ-ದಿ-ಸಿ-ರ-ಬೇಕು
ಪ್ರಚೋ-ದಿ-ಸಿ-ರ-ಲಿ-ಲ್ಲವೆ
ಪ್ರಚೋ-ದಿ-ಸು-ತ್ತದೆ
ಪ್ರಚೋ-ದಿ-ಸುವ
ಪ್ರಜೆ-ಗಳ
ಪ್ರಜೆ-ಗಳು
ಪ್ರಜ್ಞೆ
ಪ್ರಜ್ಞೆ-ತಪ್ಪಿ
ಪ್ರಜ್ಞೆಯೇ
ಪ್ರಜ್ಞೆ-ಯೊಂದು
ಪ್ರಜ್ವ-ಲ-ಗೊ-ಳಿ-ಸಿ-ದರು
ಪ್ರಜ್ವ-ಲಿ-ಸ-ಲಾ-ರಂ-ಭಿ-ಸಿತು
ಪ್ರಜ್ವ-ಲಿ-ಸು-ತ್ತಿತ್ತು
ಪ್ರಜ್ವ-ಲಿ-ಸು-ತ್ತಿದೆ
ಪ್ರಜ್ವ-ಲಿ-ಸು-ತ್ತಿ-ರುವ
ಪ್ರಜ್ವ-ಲಿ-ಸು-ವಂತೆ
ಪ್ರಣಾಮ
ಪ್ರಣಾ-ಮ-ಮಾಡಿ
ಪ್ರಣಾ-ಳಿ-ಗಳನ್ನು
ಪ್ರಣಾ-ಳಿ-ಯ-ನ್ನೇನೋ
ಪ್ರಣಾ-ಳಿ-ಯಿದೆ
ಪ್ರತಾ-ಪ-ಚಂದ್ರ
ಪ್ರತಿ
ಪ್ರತಿ-ಕ್ರಿ-ಯಿ-ಸದೆ
ಪ್ರತಿ-ಕ್ರಿ-ಯಿ-ಸು-ತ್ತೀಯೋ
ಪ್ರತಿ-ಕ್ರಿಯೆ
ಪ್ರತಿ-ಕ್ರಿ-ಯೆ-ಯನ್ನು
ಪ್ರತಿ-ಕ್ರಿ-ಯೆ-ಯನ್ನೂ
ಪ್ರತಿ-ಕ್ಷಣ
ಪ್ರತಿ-ಗಳು
ಪ್ರತಿಜ್ಞೆ
ಪ್ರತಿ-ದಿನ
ಪ್ರತಿ-ದಿ-ನದ
ಪ್ರತಿ-ದಿ-ನವೂ
ಪ್ರತಿ-ದಿ-ನ-ವೆಂ-ಬಂತೆ
ಪ್ರತಿ-ಧ್ವ-ನಿ-ಸು-ತ್ತಿತ್ತು
ಪ್ರತಿ-ನಿ-ಧಿ-ಯಂ-ತಿದ್ದ
ಪ್ರತಿ-ನಿ-ಧಿಸಿ
ಪ್ರತಿ-ನಿ-ಧಿ-ಸು-ತ್ತವೆ
ಪ್ರತಿ-ನಿ-ಧಿ-ಸು-ವಂ-ತಾ-ಗಿ-ದ್ದರು
ಪ್ರತಿ-ನು-ಡಿ-ಯ-ಲಿಲ್ಲ
ಪ್ರತಿ-ಪಾ-ದಿ-ಸ-ಲಾ-ಗಿ-ರುವ
ಪ್ರತಿ-ಪಾ-ದಿ-ಸಿ-ದರು
ಪ್ರತಿ-ಪಾ-ದಿ-ಸು-ತ್ತಿತ್ತು
ಪ್ರತಿ-ಪಾ-ದಿ-ಸುವ
ಪ್ರತಿ-ಫ-ಲ-ಕೆಟ್ಟ
ಪ್ರತಿ-ಫ-ಲಾ-ಪೇ-ಕ್ಷೆಯ
ಪ್ರತಿ-ಫ-ಲಾ-ಪೇ-ಕ್ಷೆ-ಯನ್ನೂ
ಪ್ರತಿ-ಬಂ-ಧಕ
ಪ್ರತಿ-ಬಂ-ಧ-ಗ-ಳಿ-ದ್ದು-ದ-ರಿಂದ
ಪ್ರತಿ-ಬಿಂಬ
ಪ್ರತಿ-ಬಿಂ-ಬಿ-ಸಿ-ದ್ದಾರೆ
ಪ್ರತಿ-ಬಿಂ-ಬಿ-ಸುವ
ಪ್ರತಿ-ಬಿಂ-ಬಿ-ಸು-ವು-ದೆಂದು
ಪ್ರತಿ-ಭ-ಟನೆ
ಪ್ರತಿ-ಭ-ಟ-ನೆಯ
ಪ್ರತಿ-ಭ-ಟಿಸಿ
ಪ್ರತಿ-ಭ-ಟಿ-ಸಿದ
ಪ್ರತಿ-ಭ-ಟಿ-ಸಿ-ದರು
ಪ್ರತಿ-ಭ-ಟಿ-ಸಿ-ದರೋ
ಪ್ರತಿ-ಭ-ಟಿ-ಸಿ-ದಾಗ
ಪ್ರತಿ-ಭ-ಟಿ-ಸು-ತ್ತಿದ್ದ
ಪ್ರತಿಭಾ
ಪ್ರತಿ-ಭಾ-ನ್ವಿತ
ಪ್ರತಿ-ಭಾ-ಪ್ರ-ದ-ರ್ಶ-ನ-ಕ್ಕಾಗಿ
ಪ್ರತಿ-ಭಾ-ವಂತ
ಪ್ರತಿ-ಭಾ-ವಂ-ತರು
ಪ್ರತಿಭೆ
ಪ್ರತಿ-ಭೆ-ಗಳನ್ನೂ
ಪ್ರತಿ-ಭೆಗೆ
ಪ್ರತಿ-ಭೆ-ಯನ್ನು
ಪ್ರತಿ-ಭೆ-ಯಿದೆ
ಪ್ರತಿ-ಮಾ-ತಾ-ಡದೆ
ಪ್ರತಿಮೆ
ಪ್ರತಿ-ಮೆ-ಗ-ಳ-ನ್ನಿಟ್ಟು
ಪ್ರತಿ-ಮೆ-ಯನ್ನು
ಪ್ರತಿ-ಯನ್ನು
ಪ್ರತಿ-ಯಾಗಿ
ಪ್ರತಿ-ಯೊಂ-ದಕ್ಕೂ
ಪ್ರತಿ-ಯೊಂದು
ಪ್ರತಿ-ಯೊಬ್ಬ
ಪ್ರತಿ-ಯೊ-ಬ್ಬನ
ಪ್ರತಿ-ಯೊ-ಬ್ಬ-ನಲ್ಲೂ
ಪ್ರತಿ-ಯೊ-ಬ್ಬನೂ
ಪ್ರತಿ-ಯೊ-ಬ್ಬ-ರಲ್ಲೂ
ಪ್ರತಿ-ಯೊ-ಬ್ಬ-ರಿಗೂ
ಪ್ರತಿ-ಯೊ-ಬ್ಬರೂ
ಪ್ರತಿ-ರಾತ್ರಿ
ಪ್ರತಿ-ರಾ-ತ್ರಿಯೂ
ಪ್ರತಿ-ರೂ-ಪವೂ
ಪ್ರತಿ-ವ-ರ್ಷವೂ
ಪ್ರತಿ-ವಾದ
ಪ್ರತಿ-ವಾ-ದ-ದಿಂದ
ಪ್ರತಿ-ವಾ-ದಿ-ಗಳ
ಪ್ರತಿ-ವಾ-ದಿಯ
ಪ್ರತಿ-ವಾ-ದಿಸಿ
ಪ್ರತಿ-ಷ್ಠಾ-ಪ-ಕರು
ಪ್ರತಿ-ಷ್ಠಾ-ಪ-ನೆಗೆ
ಪ್ರತಿ-ಷ್ಠಾ-ಪಿ-ಸ-ಲಾ-ಯಿತು
ಪ್ರತಿ-ಷ್ಠಾ-ಪಿ-ಸಲು
ಪ್ರತಿ-ಷ್ಠಾ-ಪಿ-ಸ-ಲೇ-ಬೇಕು
ಪ್ರತಿ-ಷ್ಠಾ-ಪಿಸಿ
ಪ್ರತಿ-ಷ್ಠಾ-ಪಿ-ಸಿ-ಕೊಂಡು
ಪ್ರತಿ-ಷ್ಠಾ-ಪಿ-ಸಿದ
ಪ್ರತಿ-ಷ್ಠಾ-ಪಿ-ಸಿದ್ದ
ಪ್ರತಿ-ಷ್ಠಾ-ಪಿ-ಸು-ವುದು
ಪ್ರತಿ-ಷ್ಠಾ-ಪಿ-ಸು-ವು-ದೆಂದು
ಪ್ರತಿ-ಷ್ಠಿ-ತ-ನಾದ
ಪ್ರತಿ-ಸ-ಲದ
ಪ್ರತಿ-ಸ-ಲವೂ
ಪ್ರತೀಕ
ಪ್ರತೀ-ಕ-ವಾದ
ಪ್ರತ್ಯಕ್ಷ
ಪ್ರತ್ಯ-ಕ್ಷ-ಳಾ-ದಳು
ಪ್ರತ್ಯ-ಕ್ಷ-ವಾಗಿ
ಪ್ರತ್ಯ-ಕ್ಷ-ವಾ-ದು-ದನ್ನು
ಪ್ರತ್ಯ-ಗಾ-ತ್ಮಾ-ನ-ಮೈ-ಕ್ಷತ್
ಪ್ರತ್ಯು-ತ್ತರ
ಪ್ರತ್ಯು-ತ್ಪ-ನ್ನ-ಮತಿ
ಪ್ರತ್ಯೇಕ
ಪ್ರತ್ಯೇ-ಕಿ-ಸು-ತ್ತದೆ
ಪ್ರಥಮ
ಪ್ರದ-ಕ್ಷಿಣೆ
ಪ್ರದ-ರ್ಶನ
ಪ್ರದ-ರ್ಶ-ನ-ಕ್ಕಾಗಿ
ಪ್ರದ-ರ್ಶ-ನ-ಗಳನ್ನು
ಪ್ರದ-ರ್ಶ-ನ-ವನ್ನು
ಪ್ರದ-ರ್ಶ-ನಾ-ಲ-ಗ-ಳಿಗೋ
ಪ್ರದೇಶ
ಪ್ರದೇ-ಶಕ್ಕೆ
ಪ್ರದೇ-ಶ-ಗಳ
ಪ್ರದೇ-ಶ-ಗಳಲ್ಲಿ
ಪ್ರದೇ-ಶದ
ಪ್ರದೇ-ಶ-ದತ್ತ
ಪ್ರದೇ-ಶ-ದಲ್ಲಿ
ಪ್ರದೇ-ಶ-ದಿಂದ
ಪ್ರಧಾನ
ಪ್ರಧಾ-ನ-ವಾಗಿ
ಪ್ರಧಾ-ನ-ವಾದ
ಪ್ರಪಂಚ
ಪ್ರಪಂ-ಚಕ್ಕೆ
ಪ್ರಪಂ-ಚದ
ಪ್ರಪಂ-ಚ-ದಲ್ಲಿ
ಪ್ರಪಂ-ಚ-ದಲ್ಲೆಲ್ಲ
ಪ್ರಪಂ-ಚ-ವನ್ನು
ಪ್ರಪಂ-ಚ-ವನ್ನೇ
ಪ್ರಪಂ-ಚವು
ಪ್ರಪಂ-ಚಾನು
ಪ್ರಪ-ತ್ತಿ-ಯಾ-ಗಲಿ
ಪ್ರಫು-ಲ್ಲ-ವಾದ
ಪ್ರಫು-ಲ್ಲ-ವಾ-ಯಿತು
ಪ್ರಬಲ
ಪ್ರಬ-ಲ-ವಾ-ಗಿ-ದ್ದಿ-ರ-ಬೇಕು
ಪ್ರಬ-ಲ-ವಾಗು
ಪ್ರಬ-ಲ-ವಾದ
ಪ್ರಬ-ಲ-ವಾ-ಯಿತು
ಪ್ರಬುದ್ಧ
ಪ್ರಬು-ದ್ಧತೆ
ಪ್ರಬು-ದ್ಧನೆ
ಪ್ರಭಾವ
ಪ್ರಭಾ-ವಕ್ಕೆ
ಪ್ರಭಾ-ವ-ಕ್ಕೊ-ಳ-ಗಾ-ಗು-ವುದನ್ನು
ಪ್ರಭಾ-ವ-ಗ-ಳಿಗೆ
ಪ್ರಭಾ-ವದ
ಪ್ರಭಾ-ವ-ದಿಂ-ದಲೋ
ಪ್ರಭಾ-ವ-ವನ್ನು
ಪ್ರಭಾ-ವ-ಶಾಲಿ
ಪ್ರಭಾ-ವ-ಶಾ-ಲಿ-ಯಾದ
ಪ್ರಭಾ-ವಿತ
ಪ್ರಭಾ-ವಿ-ತ-ನಾ-ಗ-ಬಾ-ರದು
ಪ್ರಭಾ-ವಿ-ತ-ನಾಗಿ
ಪ್ರಭಾ-ವಿ-ತ-ನಾದ
ಪ್ರಭಾ-ವಿ-ತ-ರಾ-ಗು-ತ್ತಾರೆ
ಪ್ರಭಾ-ವಿ-ತ-ರಾದ
ಪ್ರಭಾ-ವಿ-ತ-ರಾ-ದ-ವರು
ಪ್ರಭಾ-ವಿ-ತ-ವಾಗಿ
ಪ್ರಭು
ಪ್ರಭು-ತ್ವ-ದಿಂದ
ಪ್ರಭು-ತ್ವ-ವನ್ನು
ಪ್ರಭುವೆ
ಪ್ರಭೆ
ಪ್ರಭೆ-ಯನ್ನು
ಪ್ರಭೆ-ಯಲ್ಲಿ
ಪ್ರಮದ
ಪ್ರಮ-ದ-ದಾಸ
ಪ್ರಮ-ದ-ಬಾ-ಬು-ಗಳ
ಪ್ರಮ-ದ-ಬಾ-ಬು-ಗಳಿಂದ
ಪ್ರಮ-ದ-ಬಾ-ಬು-ಗಳು
ಪ್ರಮ-ದ-ಬಾ-ಬು-ವಿಗೆ
ಪ್ರಮಾ-ಣ-ದಲ್ಲಿ
ಪ್ರಮಾ-ಣ-ವನ್ನು
ಪ್ರಮಾ-ಣ-ವಿ-ಲ್ಲದೆ
ಪ್ರಮಾ-ಣೀ-ಕ-ರಿ-ಸು-ತ್ತಾನೆ
ಪ್ರಮುಖ
ಪ್ರಮು-ಖ-ನಾ-ದ-ವನು
ಪ್ರಮು-ಖ-ವಾದ
ಪ್ರಮು-ಖ-ವಾ-ದದ್ದು
ಪ್ರಯತ್ನ
ಪ್ರಯ-ತ್ನ-ಗಳ
ಪ್ರಯ-ತ್ನ-ಗಳಲ್ಲಿ
ಪ್ರಯ-ತ್ನ-ದಲ್ಲಿ
ಪ್ರಯ-ತ್ನ-ಪಟ್ಟ
ಪ್ರಯ-ತ್ನ-ಪ-ಟ್ಟರೂ
ಪ್ರಯ-ತ್ನ-ಪಟ್ಟೆ
ಪ್ರಯ-ತ್ನ-ಪ-ಡು-ತ್ತಿ-ದ್ದರು
ಪ್ರಯ-ತ್ನ-ಪ-ಡು-ತ್ತಿ-ದ್ದರೆ
ಪ್ರಯ-ತ್ನ-ಮಾ-ಡಿ-ದರು
ಪ್ರಯ-ತ್ನ-ಮಾ-ಡಿ-ದರೂ
ಪ್ರಯ-ತ್ನ-ವನ್ನು
ಪ್ರಯ-ತ್ನ-ವನ್ನೇ
ಪ್ರಯ-ತ್ನವೂ
ಪ್ರಯ-ತ್ನ-ವೆ-ನ್ನು-ವುದು
ಪ್ರಯ-ತ್ನವೇ
ಪ್ರಯ-ತ್ನಿ-ಸ-ಬ-ಹುದು
ಪ್ರಯ-ತ್ನಿ-ಸಿದ
ಪ್ರಯ-ತ್ನಿ-ಸಿ-ದರೂ
ಪ್ರಯ-ತ್ನಿ-ಸಿ-ದುವು
ಪ್ರಯ-ತ್ನಿಸು
ಪ್ರಯ-ತ್ನಿ-ಸುತ್ತ
ಪ್ರಯ-ತ್ನಿ-ಸು-ತ್ತಾರೆ
ಪ್ರಯ-ತ್ನಿ-ಸು-ತ್ತಿತ್ತು
ಪ್ರಯ-ತ್ನಿ-ಸು-ತ್ತಿದ್ದ
ಪ್ರಯ-ತ್ನಿ-ಸು-ತ್ತಿ-ದ್ದರು
ಪ್ರಯ-ತ್ನಿ-ಸು-ತ್ತಿ-ದ್ದಾಳೆ
ಪ್ರಯಾ-ಗ-ವನ್ನು
ಪ್ರಯಾಣ
ಪ್ರಯಾ-ಣಕ್ಕೆ
ಪ್ರಯಾ-ಣ-ಗಳನ್ನು
ಪ್ರಯಾ-ಣದ
ಪ್ರಯಾ-ಣ-ದಿಂದ
ಪ್ರಯಾ-ಣ-ದಿಂ-ದಾಗಿ
ಪ್ರಯಾ-ಣ-ವೆಲ್ಲ
ಪ್ರಯಾ-ಣಿ-ಕ-ರಿಗೆ
ಪ್ರಯಾ-ಸ-ಕರ
ಪ್ರಯೋ-ಗದ
ಪ್ರಯೋ-ಗವೆ
ಪ್ರಯೋ-ಗಿಸ
ಪ್ರಯೋ-ಗಿ-ಸ-ಬೇ-ಕಾದ
ಪ್ರಯೋ-ಗಿಸಿ
ಪ್ರಯೋ-ಗಿ-ಸಿ-ದರು
ಪ್ರಯೋ-ಗಿ-ಸಿ-ಬಿಟ್ಟ
ಪ್ರಯೋ-ಗಿ-ಸಿ-ಯೂ-ಬಿಟ್ಟ
ಪ್ರಯೋ-ಜನ
ಪ್ರಯೋ-ಜ-ನ-ವಾ-ಗ-ಲಿಲ್ಲ
ಪ್ರಯೋ-ಜ-ನ-ವಾ-ಗ-ಲಿ-ಲ್ಲ-ವಲ್ಲ
ಪ್ರಯೋ-ಜ-ನ-ವಾಗು
ಪ್ರಯೋ-ಜ-ನ-ವಾ-ದರೂ
ಪ್ರಯೋ-ಜ-ನ-ವಾ-ದೀತು
ಪ್ರಯೋ-ಜ-ನ-ವಿಲ್ಲ
ಪ್ರಯೋ-ಜ-ನ-ವಿ-ಲ್ಲಈ
ಪ್ರಯೋ-ಜ-ನವೂ
ಪ್ರಲೋ
ಪ್ರಲೋ-ಭನೆ
ಪ್ರಲೋ-ಭ-ನೆ-ಗಳ
ಪ್ರಲೋ-ಭ-ನೆ-ಗಳನ್ನೂ
ಪ್ರಲೋ-ಭ-ನೆ-ಗ-ಳಿಗೆ
ಪ್ರಲೋ-ಭ-ನೆ-ಗಳು
ಪ್ರಳ-ಯ-ಗ-ರ್ಜನೆ
ಪ್ರವಚ
ಪ್ರವ-ಚನ
ಪ್ರವ-ಚ-ನ-ಗಳು
ಪ್ರವ-ಚ-ನ-ದಿಂದ
ಪ್ರವ-ಹಿ-ಸ-ಲಿದೆ
ಪ್ರವಾಸ
ಪ್ರವಾ-ಸ-ಗ-ಳಿಗೋ
ಪ್ರವಾ-ಸದ
ಪ್ರವಾಹ
ಪ್ರವಾ-ಹ-ದಂತೆ
ಪ್ರವಾ-ಹ-ದಲ್ಲಿ
ಪ್ರವಾ-ಹ-ದೋ-ಪಾ-ದಿ-ಯಲ್ಲಿ
ಪ್ರವಾ-ಹ-ವನ್ನು
ಪ್ರವೀಣ
ಪ್ರವೀ-ಣ-ನಾ-ಗಿದ್ದ
ಪ್ರವೀ-ಣ-ನಾ-ಗಿ-ದ್ದರೂ
ಪ್ರವೀ-ಣ-ನಾದ
ಪ್ರವೀ-ಣ-ರಾ-ಗಲು
ಪ್ರವೃತ್ತಿ
ಪ್ರವೃ-ತ್ತಿಯ
ಪ್ರವೃ-ತ್ತಿ-ಯನ್ನು
ಪ್ರವೃ-ತ್ತಿ-ಯ-ನ್ನುಂ-ಟು-ಮಾ-ಡು-ತ್ತದೋ
ಪ್ರವೃ-ತ್ತಿ-ಯು-ಳ್ಳ-ವರು
ಪ್ರವೇಶ
ಪ್ರವೇ-ಶ-ದ್ವಾ-ರದ
ಪ್ರವೇ-ಶ-ಮಾಡಿ
ಪ್ರವೇ-ಶ-ಮಾ-ಡಿತು
ಪ್ರವೇ-ಶ-ಮಾ-ಡುವ
ಪ್ರವೇ-ಶ-ವಿಲ್ಲ
ಪ್ರವೇ-ಶಿ-ಸಲು
ಪ್ರವೇ-ಶಿಸಿ
ಪ್ರವೇ-ಶಿ-ಸಿತು
ಪ್ರವೇ-ಶಿ-ಸಿದ
ಪ್ರವೇ-ಶಿ-ಸು-ತ್ತಿ-ರ-ಲಿಲ್ಲ
ಪ್ರವೇ-ಶಿ-ಸು-ವಲ್ಲಿ
ಪ್ರಶಂ-ಸಿ-ಸಲು
ಪ್ರಶಂ-ಸೆಯ
ಪ್ರಶಂ-ಸೆ-ಯಿಂದ
ಪ್ರಶ-ಸ್ತ-ವಾ-ಗಿತ್ತು
ಪ್ರಶ-ಸ್ತ-ವಾ-ಗಿದೆ
ಪ್ರಶ-ಸ್ತಿಯ
ಪ್ರಶಾಂತ
ಪ್ರಶಾಂ-ತ
ಪ್ರಶಾಂ-ತತೆ
ಪ್ರಶಾಂ-ತ-ತೆ-ಯಿಂದ
ಪ್ರಶಾಂ-ತ-ವಾ-ಗಿತ್ತು
ಪ್ರಶಾಂ-ತ-ವಾ-ಗಿದ್ದ
ಪ್ರಶಾಂ-ತ-ವಾದ
ಪ್ರಶ್ನಿ-ಸ-ಲಾ-ರಂ-ಭಿ-ಸಿದ
ಪ್ರಶ್ನಿಸಿ
ಪ್ರಶ್ನಿ-ಸಿದ
ಪ್ರಶ್ನಿ-ಸಿ-ದರು
ಪ್ರಶ್ನಿ-ಸಿ-ದರೂ
ಪ್ರಶ್ನಿ-ಸಿ-ದಾಗ
ಪ್ರಶ್ನಿಸು
ಪ್ರಶ್ನಿ-ಸು-ತ್ತಿದ್ದ
ಪ್ರಶ್ನಿ-ಸು-ತ್ತಿ-ದ್ದಾ-ರಲ್ಲ
ಪ್ರಶ್ನಿ-ಸುವ
ಪ್ರಶ್ನೆ
ಪ್ರಶ್ನೆ-ಪ-ರಿ-ಪ್ರ-ಶ್ನೆ-ಗ-ಳಿಗೆ
ಪ್ರಶ್ನೆ-ಗಳನ್ನು
ಪ್ರಶ್ನೆ-ಗ-ಳಿಂ-ದಾಗಿ
ಪ್ರಶ್ನೆ-ಗ-ಳಿಗೂ
ಪ್ರಶ್ನೆ-ಗ-ಳಿಗೆ
ಪ್ರಶ್ನೆಗೂ
ಪ್ರಶ್ನೆಗೆ
ಪ್ರಶ್ನೆ-ಯ-ನ್ನಾ-ದರೂ
ಪ್ರಶ್ನೆ-ಯನ್ನು
ಪ್ರಶ್ನೆಯೂ
ಪ್ರಶ್ನೆ-ಯೇನು
ಪ್ರಶ್ನೆ-ಯೇ-ಳ-ಬ-ಹು-ದು-ನಿ-ಜಕ್ಕೂ
ಪ್ರಶ್ನೆ-ಯೇ-ಳ-ಬ-ಹು-ದು-ಶ್ರೀ-ರಾ-ಮ-ಕೃ-ಷ್ಣರು
ಪ್ರಶ್ನೆ-ಯೊಂದು
ಪ್ರಸಂಗ
ಪ್ರಸಂ-ಗ-ಗಳೂ
ಪ್ರಸಂ-ಗದ
ಪ್ರಸಂ-ಗ-ವನ್ನು
ಪ್ರಸಂ-ಗ-ವೇ-ರ್ಪ-ಟ್ಟಿತು
ಪ್ರಸಂ-ಗ-ವೊಂದು
ಪ್ರಸನ್ನ
ಪ್ರಸ-ನ್ನ-ಗಂ-ಭೀರ
ಪ್ರಸಾದ
ಪ್ರಸಾ-ದ-ವನ್ನು
ಪ್ರಸಾ-ದ-ವನ್ನೇ
ಪ್ರಸಾರ
ಪ್ರಸಾ-ರದ
ಪ್ರಸಿದ್ಧ
ಪ್ರಸಿ-ದ್ಧ-ನಾ-ಗಿದ್ದ
ಪ್ರಸಿ-ದ್ಧ-ನಾ-ದ-ಪ-ರ-ಮ-ಹಂ-ಸ-ನಾದ
ಪ್ರಸಿ-ದ್ಧ-ರಾದ
ಪ್ರಸಿ-ದ್ಧ-ವಾ-ಗಿದೆ
ಪ್ರಸಿದ್ಧಿ
ಪ್ರಸಿ-ದ್ಧಿ-ಗೊಂಡು
ಪ್ರಸಿ-ದ್ಧಿ-ಯನ್ನು
ಪ್ರಸೂತ
ಪ್ರಸ್ತಾಪ
ಪ್ರಸ್ತಾ-ಪ-ವೆ-ತ್ತಿ-ದಾಗ
ಪ್ರಸ್ತಾ-ಪ-ವೊಂದು
ಪ್ರಸ್ತಾ-ಪಿಸಿ
ಪ್ರಸ್ತಾ-ಪಿ-ಸು-ತ್ತಿ-ರ-ಲಿಲ್ಲ
ಪ್ರಸ್ತಾ-ಪಿ-ಸು-ವು-ದಿತ್ತು
ಪ್ರಸ್ತುತ
ಪ್ರಸ್ತು-ತ-ವಾ-ಗಿಯೇ
ಪ್ರಸ್ಫು-ಟ-ವಾಗಿ
ಪ್ರಾಂತ-ಗಳಲ್ಲಿ
ಪ್ರಾಂತ-ದಲ್ಲಿ
ಪ್ರಾಕಾಮ್ಯ
ಪ್ರಾಚೀ-ನ-ಅ-ರ್ವಾ-ಚೀನ
ಪ್ರಾಚ್ಯ-ಪಾ-ಶ್ಚಾತ್ಯ
ಪ್ರಾಣ
ಪ್ರಾಣಕ್ಕೆ
ಪ್ರಾಣ-ಗಳನ್ನು
ಪ್ರಾಣ-ತ್ಯಾಗ
ಪ್ರಾಣದ
ಪ್ರಾಣ-ಪ್ರಿಯ
ಪ್ರಾಣ-ವನ್ನೇ
ಪ್ರಾಣ-ಸ-ಮಾ-ನ-ರಾದ
ಪ್ರಾಣಾ-ಪಾಯ
ಪ್ರಾಣಿ
ಪ್ರಾಣಿ-ಗಳನ್ನು
ಪ್ರಾಣಿ-ಗ-ಳಾ-ಗಲಿ
ಪ್ರಾಣಿ-ಗಳಿಂದ
ಪ್ರಾಣಿ-ಪ-ಕ್ಷಿ-ಗ-ಳೆ-ಲ್ಲವೂ
ಪ್ರಾಣಿ-ವ-ನ-ವೊಂ-ದಕ್ಕೆ
ಪ್ರಾಣಿ-ವ-ರ್ಗಕ್ಕೆ
ಪ್ರಾಣಿ-ವ-ಲ-ಯಕ್ಕೂ
ಪ್ರಾಣೋ-ತ-ಮಣ
ಪ್ರಾಣೋ-ತ-ಮ-ಣದ
ಪ್ರಾತಃ-ಕಾ-ಲ-ಸಾ-ಯಂ-ಕಾ-ಲ-ಗಳು
ಪ್ರಾತಃ-ಕಾ-ಲದ
ಪ್ರಾಥ-ಮಿಕ
ಪ್ರಾಧಾನ
ಪ್ರಾಧಾನ್ಯ
ಪ್ರಾಧ್ಯಾ-ಪ-ಕ-ರನ್ನೂ
ಪ್ರಾಧ್ಯಾ-ಪ-ಕರು
ಪ್ರಾಪಂ-ಚಿಕ
ಪ್ರಾಪಂ-ಚಿ-ಕ-ತೆಯ
ಪ್ರಾಪಂ-ಚಿ-ಕ-ತೆ-ಯನ್ನು
ಪ್ರಾಪಂ-ಚಿ-ಕರ
ಪ್ರಾಪಂ-ಚಿ-ಕ-ರಿಗೂ
ಪ್ರಾಪಂ-ಚಿ-ಕ-ರಿಗೆ
ಪ್ರಾಪಂ-ಚಿ-ಕ-ರಿ-ಗೆಲ್ಲ
ಪ್ರಾಪಂ-ಚಿ-ಕರು
ಪ್ರಾಪಂ-ಚಿ-ಕ-ರೆಲ್ಲ
ಪ್ರಾಪಂ-ಚಿ-ಕ-ವಾದ
ಪ್ರಾಪ್ತ-ವ-ಯ-ಸ್ಕ-ನಾದ
ಪ್ರಾಪ್ತವಾ
ಪ್ರಾಪ್ತ-ವಾ-ಗಿ-ತ್ತೆಂಬ
ಪ್ರಾಪ್ತ-ವಾ-ಗಿದೆ
ಪ್ರಾಪ್ತ-ವಾಗು
ಪ್ರಾಪ್ತ-ವಾ-ಗು-ತ್ತದೆ
ಪ್ರಾಪ್ತ-ವಾ-ಗು-ವ-ವ-ರೆಗೆ
ಪ್ರಾಪ್ತ-ವಾ-ದದ್ದು
ಪ್ರಾಪ್ತ-ವಾ-ದಾಗ
ಪ್ರಾಪ್ತ-ವಾ-ಯಿತು
ಪ್ರಾಪ್ತಿ
ಪ್ರಾಪ್ತಿಯ
ಪ್ರಾಮಾ-ಣಿಕ
ಪ್ರಾಮಾ-ಣಿ-ಕತೆ
ಪ್ರಾಮಾ-ಣಿ-ಕ-ತೆ-ಯನ್ನು
ಪ್ರಾಮಾ-ಣಿ-ಕ-ತೆ-ಯಿಂದ
ಪ್ರಾಮಾ-ಣಿ-ಕ-ತೆ-ಯಿತ್ತು
ಪ್ರಾಮಾ-ಣಿ-ಕ-ನಾದ
ಪ್ರಾಮಾ-ಣಿ-ಕ-ರಾದ
ಪ್ರಾಮಾ-ಣಿ-ಕ-ವಾಗಿ
ಪ್ರಾಮಾ-ಣಿ-ಕ-ವಾ-ಗಿ-ರು-ತ್ತಿತ್ತು
ಪ್ರಾಮಾ-ಣಿ-ಕ-ವಾ-ದ-ವು-ಗಳು
ಪ್ರಾಮಾ-ಣ್ಯದ
ಪ್ರಾಮು-ಖ್ಯತೆ
ಪ್ರಾಮು-ಖ್ಯ-ವಾ-ಗಿತ್ತು
ಪ್ರಾಮು-ಖ್ಯ-ವಾದ
ಪ್ರಾಯ-ಗಳನ್ನು
ಪ್ರಾಯ-ಪ್ರ-ಬು-ದ್ಧ-ನಾ-ಗು-ವ-ವರೆ-ಗಾ-ದರೂ
ಪ್ರಾಯ-ಪ್ರ-ಬು-ದ್ಧ-ರಾದ
ಪ್ರಾಯಶಃ
ಪ್ರಾಯೋ-ಪ-ವೇಶ
ಪ್ರಾರಂಭ
ಪ್ರಾರಂ-ಭ-ದಲ್ಲಿ
ಪ್ರಾರಂ-ಭ-ವಾಗಿ
ಪ್ರಾರಂ-ಭ-ವಾ-ಗಿದ್ದ
ಪ್ರಾರಂ-ಭ-ವಾ-ಗಿದ್ದು
ಪ್ರಾರಂ-ಭ-ವಾ-ಗು-ತ್ತದೆ
ಪ್ರಾರಂ-ಭ-ವಾ-ದರೆ
ಪ್ರಾರಂ-ಭ-ವಾ-ದಾಗ
ಪ್ರಾರಂ-ಭ-ವಾ-ದ್ದ-ರಿಂದ
ಪ್ರಾರಂ-ಭ-ವಾ-ಯಿತು
ಪ್ರಾರಂ-ಭ-ವಿ-ರು-ತ್ತದೆ
ಪ್ರಾರಂ-ಭಿ-ಸಲು
ಪ್ರಾರಂ-ಭಿಸಿ
ಪ್ರಾರಂ-ಭಿ-ಸಿದ
ಪ್ರಾರಂ-ಭಿ-ಸಿ-ದ-ಈಗ
ಪ್ರಾರಂ-ಭಿ-ಸಿ-ದರು
ಪ್ರಾರಂ-ಭಿ-ಸಿ-ದಾಗ
ಪ್ರಾರಂ-ಭಿ-ಸಿ-ಯಾ-ಗಿ-ದೆ-ಯಲ್ಲ
ಪ್ರಾರಂ-ಭಿಸು
ಪ್ರಾರಂ-ಭಿ-ಸು-ತ್ತಿದ್ದ
ಪ್ರಾರ್ಥನಾ
ಪ್ರಾರ್ಥ-ನಾ-ಗೋ-ಷ್ಠಿಗೆ
ಪ್ರಾರ್ಥ-ನಾ-ದಿ-ಗಳ
ಪ್ರಾರ್ಥ-ನಾ-ಮಂ-ದಿ-ರ-ದಲ್ಲಿ
ಪ್ರಾರ್ಥ-ನಾ-ಶ್ಲೋ-ಕ-ಗಳನ್ನು
ಪ್ರಾರ್ಥನೆ
ಪ್ರಾರ್ಥ-ನೆ-ಧ್ಯಾ-ನ-ಗಳನ್ನು
ಪ್ರಾರ್ಥ-ನೆ-ಭ-ಜ-ನೆಯ
ಪ್ರಾರ್ಥ-ನೆ-ಗಳ
ಪ್ರಾರ್ಥ-ನೆಗೆ
ಪ್ರಾರ್ಥ-ನೆ-ಯನ್ನು
ಪ್ರಾರ್ಥ-ನೆ-ಯನ್ನೂ
ಪ್ರಾರ್ಥ-ನೆ-ಯ-ನ್ನೆಲ್ಲ
ಪ್ರಾರ್ಥಿಸಿ
ಪ್ರಾರ್ಥಿ-ಸಿಕೊ
ಪ್ರಾರ್ಥಿ-ಸಿ-ಕೊಂ-ಡದ್ದು
ಪ್ರಾರ್ಥಿ-ಸಿ-ಕೊಂ-ಡಿ-ದ್ದೇನೆ
ಪ್ರಾರ್ಥಿ-ಸಿ-ಕೊಂಡು
ಪ್ರಾರ್ಥಿ-ಸಿ-ಕೊಂಡೆ
ಪ್ರಾರ್ಥಿ-ಸಿ-ಕೊ-ಳ್ಳ-ಬೇಕು
ಪ್ರಾರ್ಥಿ-ಸಿ-ಕೊ-ಳ್ಳ-ಲಾ-ರಂ-ಭಿ-ಸಿ-ದಳು
ಪ್ರಾರ್ಥಿ-ಸಿ-ಕೊಳ್ಳಿ
ಪ್ರಾರ್ಥಿ-ಸಿ-ಕೊ-ಳ್ಳು-ತ್ತಾನೆ
ಪ್ರಾರ್ಥಿ-ಸಿ-ಕೊ-ಳ್ಳು-ತ್ತಿದ್ದ
ಪ್ರಾರ್ಥಿ-ಸಿ-ಕೊ-ಳ್ಳು-ತ್ತಿ-ದ್ದರು
ಪ್ರಾರ್ಥಿ-ಸಿ-ಕೊ-ಳ್ಳು-ತ್ತಿ-ದ್ದಳು
ಪ್ರಾರ್ಥಿ-ಸಿ-ಕೊ-ಳ್ಳು-ತ್ತಿ-ದ್ದುದು
ಪ್ರಾರ್ಥಿ-ಸಿ-ಕೊ-ಳ್ಳು-ತ್ತಿ-ದ್ದೇ-ನಲ್ಲ
ಪ್ರಾರ್ಥಿ-ಸಿ-ಕೊ-ಳ್ಳು-ವುದು
ಪ್ರಾರ್ಥಿ-ಸಿಕೋ
ಪ್ರಾರ್ಥಿ-ಸಿದ
ಪ್ರಾರ್ಥಿ-ಸಿ-ದೆಓ
ಪ್ರಾರ್ಥಿ-ಸು-ತ್ತಿ-ದ್ದರು
ಪ್ರಾರ್ಥಿ-ಸು-ವು-ದಾ-ಗಲಿ
ಪ್ರಾಶಸ್ತ್ಯ
ಪ್ರಾಶ-ಸ್ತ್ಯ-ವನ್ನು
ಪ್ರಿಂಟ-ರ್ಸ್
ಪ್ರಿನ್ಸಿ-ಪಾ-ಲರ
ಪ್ರಿನ್ಸಿ-ಪಾ-ಲರೇ
ಪ್ರಿನ್ಸಿ-ಪಾಲ್
ಪ್ರಿಯ
ಪ್ರಿಯ-ತ-ಮನ
ಪ್ರಿಯ-ತಮೆ
ಪ್ರಿಯ-ದ-ರ್ಶ-ನ-ರಾ-ಗಿ-ರು-ತ್ತಾರೆ
ಪ್ರಿಯ-ದೇ-ವ-ನಾದ
ಪ್ರಿಯ-ನಾ-ಗಿದ್ದ
ಪ್ರಿಯ-ನಾ-ಯ-ಕನೂ
ಪ್ರಿಯನೂ
ಪ್ರಿಯ-ವಲ್ಲ
ಪ್ರಿಯ-ವ-ಲ್ಲವೆ
ಪ್ರಿಯ-ವಾದ
ಪ್ರಿಯ-ಶಿ-ಷ್ಯನ
ಪ್ರಿಯ-ಸ-ಖನೆ
ಪ್ರಿಯೆಯು
ಪ್ರಿಯೆ-ಯೆಂ-ಬಂತೆ
ಪ್ರೀತಿ
ಪ್ರೀತಿ
ಪ್ರೀತಿ-ಆ-ದರ
ಪ್ರೀತಿ-ವಿ-ಶ್ವಾಸ
ಪ್ರೀತಿ-ವಿ-ಶ್ವಾ-ಸ-ಗೌ-ರ-ವ-ಗಳು
ಪ್ರೀತಿ-ವಿ-ಶ್ವಾ-ಸ-ಗಳನ್ನು
ಪ್ರೀತಿ-ವಿ-ಶ್ವಾ-ಸದ
ಪ್ರೀತಿ-ವಿ-ಶ್ವಾ-ಸವೇ
ಪ್ರೀತಿ-ವಿ-ಶ್ವಾ-ಸ-ವೊಂದೇ
ಪ್ರೀತಿಗೆ
ಪ್ರೀತಿ-ಪಾ-ತ್ರ-ಳಾ-ಗಿ-ದ್ದ-ವಳು
ಪ್ರೀತಿ-ಪಾ-ತ್ರ-ವಾದ
ಪ್ರೀತಿ-ಪೂ-ರ್ವಕ
ಪ್ರೀತಿಯ
ಪ್ರೀತಿ-ಯನ್ನು
ಪ್ರೀತಿ-ಯನ್ನೂ
ಪ್ರೀತಿ-ಯನ್ನೆ
ಪ್ರೀತಿ-ಯಿಂದ
ಪ್ರೀತಿ-ಯಿಂ-ದಲೋ
ಪ್ರೀತಿ-ಯಿಂ-ದಾಗಿ
ಪ್ರೀತಿ-ಯುತ
ಪ್ರೀತಿ-ಯೆಂ-ಬುದು
ಪ್ರೀತಿ-ಯೆಲ್ಲ
ಪ್ರೀತಿಯೇ
ಪ್ರೀತಿ-ಯೇನು
ಪ್ರೀತಿ-ವಿ-ಶ್ವಾ-ಸ-ಗ-ಳಿಗೆ
ಪ್ರೀತಿ-ಸ-ಬ-ಹುದು
ಪ್ರೀತಿಸಿ
ಪ್ರೀತಿ-ಸಿ-ದರೆ
ಪ್ರೀತಿಸು
ಪ್ರೀತಿ-ಸು-ತ್ತಾ-ನೆಯೋ
ಪ್ರೀತಿ-ಸು-ತ್ತಿ-ದ್ದ-ರಾ-ದರೂ
ಪ್ರೀತಿ-ಸು-ತ್ತಿ-ದ್ದಾರೆ
ಪ್ರೀತಿ-ಸು-ತ್ತಿ-ದ್ದು-ದ-ರಲ್ಲಿ
ಪ್ರೀತಿ-ಸು-ತ್ತಿ-ರು-ವುದು
ಪ್ರೀತಿ-ಸು-ತ್ತಿಲ್ಲ
ಪ್ರೀತಿ-ಸು-ತ್ತೀರಿ
ಪ್ರೀತಿ-ಸು-ತ್ತೇನೆ
ಪ್ರೀತಿ-ಸು-ತ್ತೇ-ವೆಯೋ
ಪ್ರೀತಿ-ಸುವ
ಪ್ರೀತಿ-ಸು-ವಲ್ಲಿ
ಪ್ರೀತಿ-ಸು-ವುದು
ಪ್ರೆಸಿ-ಡೆನ್ಸಿ
ಪ್ರೇಕ್ಷಕ
ಪ್ರೇಕ್ಷ-ಕರ
ಪ್ರೇಕ್ಷ-ಕ-ರಲ್ಲಿ
ಪ್ರೇಕ್ಷ-ಕ-ರ-ಲ್ಲೊಬ್ಬ
ಪ್ರೇಕ್ಷ-ಕರು
ಪ್ರೇಕ್ಷ-ಕ-ರೆಲ್ಲ
ಪ್ರೇಕ್ಷ-ಣ
ಪ್ರೇತ
ಪ್ರೇತ-ಕಳೆ
ಪ್ರೇತಕ್ಕೆ
ಪ್ರೇತದ
ಪ್ರೇತ-ದಂತೆ
ಪ್ರೇತ-ನಾ-ದ-ನಂತೆ
ಪ್ರೇತ-ವಾ-ದರೂ
ಪ್ರೇತ-ವಾ-ದ್ದ-ರಿಂದ
ಪ್ರೇಮ
ಪ್ರೇಮ-ಪಾ-ವಿ-ತ್ರ್ಯ-ಗ-ಳಲ್ಲೂ
ಪ್ರೇಮ-ವಿ-ಶ್ವಾ-ಸ-ಗಳನ್ನು
ಪ್ರೇಮ-ಸೌಂ-ದ-ರ್ಯ-ಕ-ಲೆ-ಗಳ
ಪ್ರೇಮ-ಕ್ಕಿಂ-ತಲೂ
ಪ್ರೇಮಕ್ಕೆ
ಪ್ರೇಮ-ಗಳ
ಪ್ರೇಮ-ಗಳನ್ನು
ಪ್ರೇಮ-ಗ-ಳೆಂ-ಬವು
ಪ್ರೇಮದ
ಪ್ರೇಮ-ದ-ಲ್ಲಾ-ಗಲಿ
ಪ್ರೇಮ-ದಿಂದ
ಪ್ರೇಮ-ಧ-ನ-ವನು
ಪ್ರೇಮ-ಪುರ
ಪ್ರೇಮ-ಪೂ-ರಿತ
ಪ್ರೇಮ-ಪ್ರ-ವಾಹ
ಪ್ರೇಮ-ಬಂ-ಧನ
ಪ್ರೇಮ-ಭ-ಕ್ತಿಯ
ಪ್ರೇಮ-ಭ-ರಿ-ತ-ನಾ-ಗಿಯೇ
ಪ್ರೇಮ-ಮಯ
ಪ್ರೇಮ-ವನ್ನು
ಪ್ರೇಮ-ವಾ-ಗಲಿ
ಪ್ರೇಮ-ವಿ-ರು-ವಲ್ಲಿ
ಪ್ರೇಮ-ವೆಂ-ಬುದು
ಪ್ರೇಮ-ವೆಂ-ಬು-ವುದು
ಪ್ರೇಮ-ವೆ-ನಿ-ಸು-ತ್ತದೆ
ಪ್ರೇಮ-ವೊಂದೇ
ಪ್ರೇಮ-ಸಂ-ಬಂ-ಧ-ವನ್ನು
ಪ್ರೇಮ-ಸ್ವ-ರೂ-ಪ-ನಾದ
ಪ್ರೇಮಾ-ನಂದ
ಪ್ರೇಮಾ-ನಂ-ದ-ರನ್ನೂ
ಪ್ರೇಮಾ-ನಂ-ದರು
ಪ್ರೇಮಿಯೂ
ಪ್ರೇಮೀ
ಪ್ರೇಮೋ-ತ್ಸಾ-ಹದ
ಪ್ರೇರ-ಣೆ-ಯಾ-ಗು-ತ್ತಿದೆ
ಪ್ರೇರೇ
ಪ್ರೇರೇ-ಪ-ಣೆ-ಯುಂ-ಟಾ-ಯಿತು
ಪ್ರೇರೇಪಿ
ಪ್ರೇರೇ-ಪಿಸು
ಪ್ರೇರೇ-ಪಿ-ಸು-ತ್ತಿತ್ತು
ಪ್ರೇರೇ-ಪಿ-ಸು-ತ್ತಿದ್ದ
ಪ್ರೈಷ-ಮಂ-ತ್ರ-ವನ್ನು
ಪ್ರೊಜೆ-ಕ್ಟರ್
ಪ್ರೊಫೆ-ಸರ್
ಪ್ರೋತ್ಸಾ-ಹದ
ಪ್ರೋತ್ಸಾ-ಹಿ-ಸ-ದಿ-ರು-ವುದನ್ನು
ಪ್ರೋತ್ಸಾ-ಹಿಸಿ
ಪ್ರೋತ್ಸಾ-ಹಿ-ಸಿದ
ಪ್ರೋತ್ಸಾ-ಹಿ-ಸಿ-ದರು
ಪ್ರೋತ್ಸಾ-ಹಿ-ಸಿದ್ದು
ಪ್ರೋತ್ಸಾ-ಹಿ-ಸು-ತ್ತಾನೆ
ಪ್ರೋತ್ಸಾ-ಹಿ-ಸು-ತ್ತಿ-ದ್ದುದು
ಪ್ರೋತ್ಸಾ-ಹಿ-ಸು-ವು-ದಿಲ್ಲ
ಪ್ರೌಢ-ವಾ-ಗಿದೆ
ಪ್ರೌಢವೂ
ಪ್ಲಸ್
ಫಕೀರ
ಫಕೀ-ರನ
ಫಕೀ-ರ-ನನ್ನು
ಫಕೀ-ರ-ನಾ-ಗಿ-ಬಿಟ್ಟೆ
ಫಕೀ-ರರೂ
ಫಟಿಂಗ
ಫಲ
ಫಲ-ದಿಂದ
ಫಲ-ಪು-ಷ್ಪ-ಗಳ
ಫಲ-ವನ್ನೂ
ಫಲ-ವಾಗಿ
ಫಲ-ವಾದ
ಫಲವೋ
ಫಲ-ಹಾ-ರಿಣೀ
ಫಲಿ-ತಾಂ-ಶ-ವನ್ನು
ಫಲಿ-ಸ-ಲಿಲ್ಲ
ಫಲಿ-ಸಿತು
ಫಲಿ-ಸಿ-ದ್ದನ್ನು
ಫಲಿ-ಸು-ತ್ತದೆ
ಫಿಕ್ಟೆ
ಫೀಸ-ನ್ನೇನೋ
ಫೀಸು
ಫೀಸ್
ಫೆಬ್ರ-ವರಿ
ಫೆರೆಂಡು
ಫ್ಲೂ
ಬಂಗಾರ
ಬಂಗಾಳ
ಬಂಗಾ-ಳಕ್ಕೆ
ಬಂಗಾ-ಳದ
ಬಂಗಾ-ಳ-ದ-ಲ್ಲಂತೂ
ಬಂಗಾ-ಳ-ದಲ್ಲಿ
ಬಂಗಾ-ಳ-ದಿಂದ
ಬಂಗಾ-ಳವು
ಬಂಗಾ-ಳವೇ
ಬಂಗಾಳಿ
ಬಂಗಾ-ಳಿ-ಗಳ
ಬಂಗಾ-ಳಿ-ಗ-ಳ-ಲ್ಲಿದ್ದ
ಬಂಗಾ-ಳಿ-ಗಳು
ಬಂಗಾ-ಳಿ-ಗ-ಳೆ-ಲ್ಲರೂ
ಬಂಗಾ-ಳಿ-ಯಲ್ಲಿ
ಬಂಗಾ-ಳಿ-ಯ-ಲ್ಲಿಲ್ಲ
ಬಂಗಾಳೀ
ಬಂಟ
ಬಂಡಾ-ಯ-ಗಾ-ರ-ನಂತೆ
ಬಂಡಾ-ಯದ
ಬಂತು
ಬಂತು-ಆ-ತನ
ಬಂತೋ
ಬಂದ
ಬಂದಂ-ತಾ-ಯಿತು
ಬಂದಂ-ತಿ-ರುವ
ಬಂದಂತೆ
ಬಂದ-ದ್ದನ್ನು
ಬಂದ-ದ್ದರ
ಬಂದ-ದ್ದ-ರಿಂದ
ಬಂದ-ದ್ದಾ-ದರೂ
ಬಂದದ್ದು
ಬಂದದ್ದೂ
ಬಂದದ್ದೇ
ಬಂದ-ದ್ದೇಕೆ
ಬಂದನೆ
ಬಂದ-ನೆಂ-ದರೆ
ಬಂದನೊ
ಬಂದನೋ
ಬಂದ-ಬಂ-ದ-ವರೆ-ಲ್ಲ-ರನ್ನೂ
ಬಂದ-ಬರಿ
ಬಂದ-ಮಾ-ತ್ರಕ್ಕೆ
ಬಂದ-ಮೇಲೂ
ಬಂದ-ಮೇಲೆ
ಬಂದ-ರಿಗೆ
ಬಂದ-ರಿ-ನಲ್ಲಿ
ಬಂದರು
ಬಂದ-ರು-ಏ-ನೆಂ-ದರೆ
ಬಂದ-ರು-ತೀವ್ರ
ಬಂದ-ರು-ನ-ರೇಂ-ದ್ರ-ನನ್ನು
ಬಂದರೂ
ಬಂದರೆ
ಬಂದರೋ
ಬಂದಳು
ಬಂದ-ವನೇ
ಬಂದ-ವ-ರನ್ನು
ಬಂದ-ವ-ರ-ಲ್ಲವೆ
ಬಂದ-ವ-ರಿಗೆ
ಬಂದ-ವರು
ಬಂದ-ವರೇ
ಬಂದ-ವರೋ
ಬಂದಾಗ
ಬಂದಾ-ಗ-ಲೆಲ್ಲ
ಬಂದಾನು
ಬಂದಾರು
ಬಂದಿ-ತಲ್ಲ
ಬಂದಿತು
ಬಂದಿ-ತು-ಕೋ-ತಿಯ
ಬಂದಿ-ತೆಂ-ದರೆ
ಬಂದಿತ್ತು
ಬಂದಿತ್ತೆ
ಬಂದಿದೆ
ಬಂದಿದ್ದ
ಬಂದಿ-ದ್ದನ್ನು
ಬಂದಿ-ದ್ದರು
ಬಂದಿ-ದ್ದ-ರು-ಏ-ನೆಂ-ದರೆ
ಬಂದಿ-ದ್ದರೆ
ಬಂದಿ-ದ್ದ-ವ-ನ-ಲ್ಲವೆ
ಬಂದಿ-ದ್ದ-ವ-ರ-ಲ್ಲದೆ
ಬಂದಿ-ದ್ದಾಗ
ಬಂದಿ-ದ್ದಾ-ಗಲೇ
ಬಂದಿ-ದ್ದಾರೆ
ಬಂದಿ-ದ್ದಾ-ರೆಂಬ
ಬಂದಿ-ದ್ದಾ-ರೆಂ-ಬುದು
ಬಂದಿ-ದ್ದಿ-ರಲೂ
ಬಂದಿ-ದ್ದೀ-ಯಲ್ಲ
ಬಂದಿ-ದ್ದೇನೆ
ಬಂದಿ-ದ್ದೇವೆ
ಬಂದಿ-ರ-ಬ-ಹುದು
ಬಂದಿ-ರ-ಬ-ಹುದೇ
ಬಂದಿ-ರ-ಲಿಲ್ಲ
ಬಂದಿ-ರ-ಲಿ-ಲ್ಲ-ವೆಂದೆ
ಬಂದಿರಿ
ಬಂದಿ-ರುವ
ಬಂದಿ-ರು-ವಂತೆ
ಬಂದಿ-ರು-ವುದು
ಬಂದಿ-ರುವೆ
ಬಂದಿ-ರು-ವೆ-ಯೆಂದು
ಬಂದಿ-ಳಿ-ದರು
ಬಂದೀತು
ಬಂದು
ಬಂದು-ಬಿಟ್ಟ
ಬಂದು-ಬಿ-ಟ್ಟರು
ಬಂದು-ಬಿಟ್ಟಿ
ಬಂದು-ಬಿ-ಟ್ಟಿತ್ತು
ಬಂದು-ಬಿ-ಟ್ಟಿದೆ
ಬಂದು-ಬಿ-ಟ್ಟಿ-ದ್ದರು
ಬಂದು-ಬಿ-ಟ್ಟಿ-ದ್ದಾರೆ
ಬಂದು-ಬಿಟ್ಟೆ
ಬಂದು-ಬಿಡು
ಬಂದು-ಬಿ-ಡು-ತ್ತದೆ
ಬಂದು-ಬಿ-ಡು-ತ್ತಾನೆ
ಬಂದು-ಬಿ-ದ್ದಿದೆ
ಬಂದುವು
ಬಂದು-ಹೋ-ಗಲು
ಬಂದು-ಹೋಗು
ಬಂದು-ಹೋ-ಗು-ತ್ತಿ-ದ್ದರು
ಬಂದು-ಹೋ-ಗು-ತ್ತಿದ್ದು
ಬಂದೂ-ಬಿಟ್ಟ
ಬಂದೆ
ಬಂದೆ-ನಪ್ಪ
ಬಂದೆವು
ಬಂದೇ
ಬಂದೊ-ಡನೆ
ಬಂದೊ-ದ-ಗಲಿ
ಬಂದೊ-ದ-ಗಿತು
ಬಂದೊ-ದ-ಗುವ
ಬಂದ್
ಬಂಧನ
ಬಂಧ-ನ-ಬಿ-ಡು-ಗ-ಡೆ-ಗ-ಳೆಂ-ಬುದೇ
ಬಂಧ-ನ-ಕಾ-ರಿ-ಯಲ್ಲ
ಬಂಧ-ನ-ಕಾ-ರಿಯೇ
ಬಂಧ-ನಕ್ಕೆ
ಬಂಧ-ನ-ಕ್ಕೊ-ಳ-ಗಾ-ದ-ವನೇ
ಬಂಧ-ನ-ಗಳ
ಬಂಧ-ನ-ಗಳನ್ನು
ಬಂಧ-ನ-ಗಳನ್ನೂ
ಬಂಧ-ನ-ಗಳನ್ನೆಲ್ಲ
ಬಂಧ-ನ-ಗಳಿಂದ
ಬಂಧ-ನದ
ಬಂಧ-ನ-ದಿಂದ
ಬಂಧ-ನ-ವಾದ
ಬಂಧ-ನವು
ಬಂಧ-ನ-ವೆಂದರೆ
ಬಂಧ-ನ-ವೆಂ-ಬು-ದೊಂದು
ಬಂಧ-ಮು-ಕ್ತ-ರಾ-ಗ-ಬೇಕು
ಬಂಧಮ್
ಬಂಧಿ
ಬಂಧಿ-ಸಲು
ಬಂಧಿ-ಸಿ-ಟ್ಟಿದ್ದ
ಬಂಧಿ-ಸಿದ್ದ
ಬಂಧಿ-ಸಿ-ಬಿ-ಟ್ಟಿದೆ
ಬಂಧಿ-ಸಿ-ರುವ
ಬಂಧಿ-ಸುವ
ಬಂಧು
ಬಂಧು-ಬ-ಳ-ಗ-ದ-ವರ
ಬಂಧು-ಗಳ
ಬಂಧು-ಗ-ಳಿಗೆ
ಬಂಧು-ಗಳು
ಬಂಧು-ಗ-ಳೊಂ-ದಿಗೆ
ಬಂಧು-ವ-ರ್ಗ-ದ-ವರೂ
ಬಗ-ಲಲ್ಲಿ
ಬಗ-ಲ-ಲ್ಲಿ-ಟ್ಟಿ-ದ್ದರು
ಬಗ-ಲಿ-ನಲ್ಲಿ
ಬಗ-ವಂ-ತ-ನೊ-ಬ್ಬನೇ
ಬಗೆ
ಬಗೆ-ಗ-ಣ್ಣಿನ
ಬಗೆ-ಗ-ಳಿಂ-ದಲೂ
ಬಗೆ-ಗಾ-ಣದೆ
ಬಗೆ-ಗಿನ
ಬಗೆಗೂ
ಬಗೆಗೆ
ಬಗೆಗೇ
ಬಗೆ-ದು-ನೋ-ಡು-ವಂ-ತಹ
ಬಗೆ-ಬ-ಗೆಯ
ಬಗೆ-ಬ-ಗೆ-ಯಾಗಿ
ಬಗೆಯ
ಬಗೆ-ಯದು
ಬಗೆ-ಯದೇ
ಬಗೆ-ಯ-ನ-ರಿಯೆ
ಬಗೆ-ಯಲ್ಲಿ
ಬಗೆ-ಯಾಗಿ
ಬಗೆ-ಯಿಂದ
ಬಗೆಯೇ
ಬಗೆ-ಹ-ರಿ-ಯಿತು
ಬಗೆ-ಹ-ರಿ-ಯುವ
ಬಗೆ-ಹ-ರಿ-ಸಲು
ಬಗೆ-ಹ-ರಿ-ಸಿ-ಕೊಂಡು
ಬಗೆ-ಹ-ರಿ-ಸಿ-ಕೊ-ಳ್ಳುವ
ಬಗ್ಗಿ
ಬಗ್ಗಿ-ಸ-ಬೇಕು
ಬಗ್ಗು-ವ-ವ-ನಲ್ಲ
ಬಗ್ಗೆ
ಬಗ್ಗೆ-ಯಾ-ದರೂ
ಬಗ್ಗೆಯೂ
ಬಗ್ಗೆಯೇ
ಬಟ್ಟ-ಲ-ನ್ನೆತ್ತಿ
ಬಟ್ಟಲು
ಬಟ್ಟೆ
ಬಟ್ಟೆ-ಗಳ
ಬಟ್ಟೆ-ಗಳನ್ನು
ಬಟ್ಟೆ-ಗ-ಳಿಗೆ
ಬಟ್ಟೆ-ಗಳು
ಬಟ್ಟೆ-ಬ-ರೆಯ
ಬಟ್ಟೆಯ
ಬಟ್ಟೆ-ಯ-ನ್ನಾ-ದರೂ
ಬಟ್ಟೆ-ಯನ್ನು
ಬಟ್ಟೆ-ಯನ್ನೂ
ಬಟ್ಟೆ-ಯ-ಲ್ಲಿ-ದ್ದರು
ಬಟ್ಟೆ-ಯ-ಲ್ಲಿ-ದ್ದರೂ
ಬಟ್ಟೆ-ಯೆಂ-ದರೆ
ಬಟ್ವಿ
ಬಡ
ಬಡ-ತನ
ಬಡ-ತ-ನ-ಕಷ್ಟ
ಬಡ-ತ-ನಕ್ಕೆ
ಬಡ-ತ-ನದ
ಬಡ-ತ-ನವೇ
ಬಡ-ಪಾಯಿ
ಬಡ-ಪಾ-ಯಿ-ಯ-ನ್ನೇಕೆ
ಬಡ-ಪೆ-ಟ್ಟಿಗೆ
ಬಡ-ಬ-ಗ್ಗರ
ಬಡವ
ಬಡ-ವ-ನೆಂ-ದರೆ
ಬಡ-ವ-ನೊ-ಬ್ಬ-ನಿಗೆ
ಬಡ-ವರ
ಬಡ-ವ-ರಾ-ಗಿ-ರ-ಬ-ಹುದು
ಬಡ-ವ-ರಾ-ದರು
ಬಡ-ವ-ರಾ-ದರೂ
ಬಡ-ವ-ರಿ-ಗಾಗಿ
ಬಡ-ವ-ರಿಗೆ
ಬಡ-ವಾ-ಗು-ತ್ತವೆ
ಬಡ-ಸಂ-ನ್ಯಾ-ಸಿ-ಗ-ಳ-ಲ್ಲವೆ
ಬಡಾ
ಬಡಾ-ಸಾ-ಹೇ-ಬರ
ಬಡಾ-ಸಾ-ಹೇ-ಬ-ರನ್ನು
ಬಡಿ-ದಂತಾ
ಬಡಿ-ದ-ದ್ದ-ಲ್ಲದೆ
ಬಡಿ-ದರು
ಬಡಿ-ದರೆ
ಬಡಿ-ದ-ವ-ನಂತೆ
ಬಡಿ-ದಾ-ಡಿ-ಕೊಂ-ಡರು
ಬಡಿದು
ಬಡಿ-ದುವು
ಬಡಿ-ದು-ಹಾ-ಕು-ತ್ತೇವೆ
ಬಡಿ-ದೆ-ಬ್ಬಿಸಿ
ಬಡಿ-ದೆ-ಬ್ಬಿ-ಸಿ-ದಂ-ತಾ-ಯಿತು
ಬಡಿ-ದೆ-ಬ್ಬಿ-ಸು-ವಂ-ತಹ
ಬಡಿ-ಬ-ಡಿದು
ಬಡಿ-ಯು-ತ್ತ-ದೇನೋ
ಬಡಿ-ಸಿ-ಕೊಂ-ಡರು
ಬಡಿ-ಸಿದ
ಬಡಿ-ಸಿ-ದಳು
ಬಡಿಸು
ಬಡಿ-ಸು-ತ್ತಿದ್ದ
ಬಡಿ-ಸು-ತ್ತಿ-ರುವ
ಬಡಿ-ಸು-ತ್ತೇನೆ
ಬಡಿ-ಸು-ವಂಥ
ಬಣ್ಣದ
ಬಣ್ಣನೆ
ಬಣ್ಣ-ನೆ-ಯಲ್ಲ
ಬಣ್ಣ-ನೆ-ಯೆಲ್ಲ
ಬಣ್ಣಿ-ಸ-ಬ-ಲ್ಲ-ವ-ರಿಲ್ಲ
ಬಣ್ಣಿ-ಸ-ಲಾ-ಗದು
ಬಣ್ಣಿ-ಸ-ಲಾ-ರಂ-ಭಿ-ಸಿದ
ಬಣ್ಣಿ-ಸಲಿ
ಬಣ್ಣಿ-ಸ-ಲ್ಪ-ಟ್ಟಿ-ರುವ
ಬಣ್ಣಿಸಿ
ಬಣ್ಣಿ-ಸಿದ
ಬಣ್ಣಿ-ಸಿ-ದರು
ಬಣ್ಣಿ-ಸಿ-ದಾಗ
ಬಣ್ಣಿ-ಸುತ್ತ
ಬಣ್ಣಿ-ಸು-ತ್ತದೆ
ಬಣ್ಣಿ-ಸು-ತ್ತಾನೆ
ಬಣ್ಣಿ-ಸು-ತ್ತಾರೆ
ಬಣ್ಣಿ-ಸು-ತ್ತಿ-ದ್ದಳು
ಬಣ್ಣಿ-ಸುವ
ಬಣ್ಣಿ-ಸು-ವ-ವರು
ಬಣ್ಣಿ-ಸು-ವು-ದಿತ್ತು
ಬತ್ತಲಿ
ಬತ್ತ-ಲೆ-ಯಾಗಿ
ಬತ್ತಿ-ಹೋ-ಗಲಿ
ಬತ್ತಿ-ಹೋ-ಗಿ-ರ-ಲಿಲ್ಲ
ಬದ-ನೇ-ಕಾಯಿ
ಬದರಿ
ಬದ-ರಿ-ಕೇ-ದಾರ
ಬದ-ರಿ-ಕೇ-ದಾ-ರ-ಗ-ಳಿಗೆ
ಬದ-ರಿ-ಕಾ-ಶ್ರ-ಮಕ್ಕೆ
ಬದ-ರಿ-ಕಾ-ಶ್ರ-ಮದ
ಬದರೀ
ಬದ-ರೀ-ಕೇ-ದಾ-ರದ
ಬದ-ರೀ-ದತ್ತ
ಬದಲಾ
ಬದ-ಲಾ-ಗದ
ಬದ-ಲಾ-ಗಲೇ
ಬದ-ಲಾಗಿ
ಬದ-ಲಾ-ಗಿ-ಹೋ-ಯಿತು
ಬದ-ಲಾ-ಗು-ತ್ತಿತ್ತು
ಬದ-ಲಾ-ದಂತೆ
ಬದ-ಲಾ-ಯಿ-ಸಲು
ಬದ-ಲಾ-ಯಿಸಿ
ಬದ-ಲಾ-ಯಿ-ಸಿ-ಕೊ-ಳ್ಳ-ಬೇ-ಕೆಂ-ದರೆ
ಬದ-ಲಾ-ಯಿ-ಸಿ-ದಾ-ಗ-ಲೆಲ್ಲ
ಬದ-ಲಾ-ಯಿ-ಸು-ತ್ತಲೇ
ಬದ-ಲಾ-ಯಿ-ಸುವ
ಬದ-ಲಾ-ಯಿ-ಸು-ವು-ದಿಲ್ಲ
ಬದ-ಲಾ-ವಣೆ
ಬದ-ಲಾ-ವ-ಣೆ-ಯುಂ-ಟಾ-ಗಿ-ಬಿ-ಟ್ಟಿತು
ಬದ-ಲಿ-ಗ-ನೊ-ಬ್ಬನ
ಬದ-ಲಿಗೆ
ಬದ-ಲಿ-ಗೆ-ಇನ್ನೂ
ಬದ-ಲಿ-ಸ-ಬೇ-ಕಾ-ಯಿತು
ಬದ-ಲಿ-ಸಿ-ಕೊಂಡು
ಬದ-ಲಿ-ಸಿ-ಕೊಳ್ಳು
ಬದ-ಲಿ-ಸಿ-ದ್ದಾ-ಗಲಿ
ಬದ-ಲಿ-ಸಿ-ಬಿ-ಡು-ತ್ತಾನೆ
ಬದಲು
ಬದಿ-ಗಿ-ಟ್ಟ-ಬಿ-ಡು-ತ್ತಿ-ದ್ದರೋ
ಬದಿ-ಗೊತ್ತಿ
ಬದಿ-ಗೊ-ತ್ತಿ-ದರೆ
ಬದಿ-ಗೊ-ತ್ತಿ-ದಳು
ಬದಿ-ಗೊ-ತ್ತಿ-ದು-ದನ್ನು
ಬದಿ-ಗೊ-ತ್ತಿ-ಬಿ-ಡು-ತ್ತಿದ್ದೆ
ಬದಿ-ಯಲ್ಲಿ
ಬದಿ-ಯ-ಲ್ಲಿ-ರುವ
ಬದಿ-ಯಲ್ಲೇ
ಬದಿ-ಯ-ಲ್ಲೊಂದು
ಬದು-ಕನ್ನು
ಬದು-ಕ-ಬಂ-ದ-ವ-ನಲ್ಲ
ಬದು-ಕ-ಬೇ-ಕೆ-ನ್ನು-ವ-ವನು
ಬದು-ಕ-ಲಾರೆ
ಬದು-ಕಿ-ದರು
ಬದು-ಕಿ-ದರೆ
ಬದು-ಕಿ-ಯಾ-ದರೂ
ಬದು-ಕಿ-ಯಾ-ರೆಂಬ
ಬದು-ಕಿ-ರ-ಲಾರ
ಬದು-ಕಿ-ರ-ಲಾರೆ
ಬದು-ಕಿ-ರಲಿ
ಬದು-ಕಿ-ರು-ವ-ವ-ರೆಗೆ
ಬದು-ಕುತ್ತ
ಬದು-ಕುವ
ಬದು-ಕು-ವು-ದೆಂ-ದರೆ
ಬದ್ಧ
ಬದ್ಧ-ನಾ-ಗಿದ್ದ
ಬದ್ಧ-ನಾ-ದ-ವನು
ಬದ್ಧ-ಮಾನ
ಬದ್ಧಿ-ಶ-ಕ್ತಿ-ಇ-ಚ್ಛಾ-ಶ-ಕ್ತಿ-ಗ-ಳೆಲ್ಲ
ಬನ್ನಿ
ಬನ್ನಿರಿ
ಬಯಕೆ
ಬಯ-ಕೆ-ಗಳನ್ನು
ಬಯ-ಕೆ-ಗಳನ್ನೆಲ್ಲ
ಬಯ-ಕೆ-ಯ-ನ್ನಿ-ಡುವ
ಬಯ-ಕೆ-ಯನ್ನು
ಬಯ-ಕೆ-ಯನ್ನೂ
ಬಯ-ಕೆ-ಯಿಂದ
ಬಯ-ಲಾ-ಗು-ತ್ತದೆ
ಬಯ-ಲಿನ
ಬಯಲು
ಬಯ-ಲು-ಪ-ಡಿ-ಸು-ತ್ತಾರೆ
ಬಯ-ಲು-ಮಾ-ಡು-ತ್ತಾನೆ
ಬಯ-ಲು-ಸೀ-ಮೆ-ಯಲ್ಲೇ
ಬಯ-ಸದೆ
ಬಯಸಿ
ಬಯ-ಸಿ-ದಂ-ತಹ
ಬಯ-ಸಿ-ದರು
ಬಯ-ಸಿ-ದಾಗ
ಬಯ-ಸಿದ್ದ
ಬಯ-ಸಿ-ದ್ದರು
ಬಯ-ಸು-ತ್ತಾರೆ
ಬಯ-ಸು-ತ್ತಿದ್ದೀ
ಬಯ-ಸುವ
ಬಯ-ಸು-ವ-ವನು
ಬಯ-ಸು-ವ-ವ-ರಾ-ದರೆ
ಬರ-ಕೂ-ಡದು
ಬರ-ಗಾ-ಲದ
ಬರ-ತೊ-ಡ-ಗಿದ
ಬರ-ದಿ-ದ್ದರೆ
ಬರ-ದಿ-ದ್ದುದೇ
ಬರದೆ
ಬರದೇ
ಬರ-ಬ-ರುತ್ತ
ಬರ-ಬಲ್ಲ
ಬರ-ಬ-ಲ್ಲೆಯಾ
ಬರ-ಬ-ಹುದು
ಬರ-ಬ-ಹು-ದೆಂದು
ಬರ-ಬಾ-ರದು
ಬರ-ಬೇ-ಕಾ-ಗಿತ್ತು
ಬರ-ಬೇ-ಕಾ-ಗಿದೆ
ಬರ-ಬೇ-ಕಾ-ಗು-ತ್ತದೆ
ಬರ-ಬೇ-ಕಾದ
ಬರ-ಬೇ-ಕಾ-ದರೆ
ಬರ-ಬೇ-ಕಾ-ಯಿತು
ಬರ-ಬೇಕು
ಬರ-ಬೇ-ಕೆಂದು
ಬರ-ಬೇಡಿ
ಬರ-ಮಾಡಿ
ಬರ-ಮಾ-ಡಿ-ಕೊಂ-ಡರು
ಬರ-ಮಾ-ಡಿ-ಕೊಂ-ಡಳು
ಬರ-ಮಾ-ಡಿ-ಕೊಂ-ಡಿದ್ದ
ಬರ-ಮಾ-ಡಿ-ಕೊಂಡು
ಬರ-ಲಪ್ಪಾ
ಬರ-ಲಾ-ರಂ-ಭ-ವಾ-ಯಿತು
ಬರ-ಲಾ-ರಂ-ಭಿ-ಸಿತು
ಬರ-ಲಾ-ರಂ-ಭಿ-ಸಿದ
ಬರ-ಲಾ-ರಂ-ಭಿ-ಸಿ-ದರು
ಬರ-ಲಾ-ರಂ-ಭಿ-ಸಿ-ದ-ರು-ಗೌ-ರೀ-ಪಂ-ಡಿತ
ಬರ-ಲಾ-ರಂ-ಭಿ-ಸಿ-ದವು
ಬರ-ಲಾ-ರಂ-ಭಿ-ಸಿ-ದ್ದರು
ಬರ-ಲಾ-ರಂ-ಭಿ-ಸಿ-ದ್ದಾರೆ
ಬರ-ಲಾ-ರರು
ಬರಲಿ
ಬರ-ಲಿದೆ
ಬರ-ಲಿ-ದ್ದಾರೆ
ಬರ-ಲಿ-ರುವ
ಬರ-ಲಿಲ್ಲ
ಬರ-ಲಿ-ಲ್ಲ-ಎಷ್ಟು
ಬರ-ಲಿ-ಲ್ಲ-ವೆಂ-ದೇ-ನಲ್ಲ
ಬರ-ಲಿ-ಲ್ಲವೋ
ಬರಲು
ಬರಲೇ
ಬರ-ಲೇ-ಬೇ-ಕಾ-ಗಿತ್ತು
ಬರ-ವ-ಣಿ-ಗೆಯ
ಬರ-ವನ್ನೇ
ಬರ-ವಿ-ಗಾಗಿ
ಬರ-ಹ-ವಿ-ಲ್ಲದ
ಬರಾರಿ
ಬರಾ-ರಿ-ಯಲ್ಲಿ
ಬರಿ-ಗ-ಣ್ಣಿಗೆ
ಬರಿಗೈ
ಬರಿ-ಗೈ-ಯಲ್ಲಿ
ಬರಿ-ದಾ-ಗಿ-ರು-ತ್ತಿತ್ತು
ಬರಿ-ದಾ-ಯಿತು
ಬರಿದೆ
ಬರಿಯ
ಬರಿ-ಸಿ-ದ್ದಾಳೆ
ಬರಿ-ಸು-ತ್ತಿದ್ದ
ಬರಿ-ಹೊ-ಟ್ಟೆ-ಯಲ್ಲಿ
ಬರೀ
ಬರು
ಬರು-ತ್ತದೆ
ಬರು-ತ್ತ-ದೆಂ-ದರೆ
ಬರು-ತ್ತ-ದೆಯೆ
ಬರು-ತ್ತ-ದೆಯೋ
ಬರು-ತ್ತದೋ
ಬರು-ತ್ತಲೇ
ಬರು-ತ್ತವೆ
ಬರು-ತ್ತ-ವೆ-ಆ-ಧ್ಯಾ-ತ್ಮಿಕ
ಬರುತ್ತಾ
ಬರು-ತ್ತಾನೆ
ಬರು-ತ್ತಾ-ರ-ಲ್ಲವೆ
ಬರು-ತ್ತಾರೆ
ಬರುತ್ತಿ
ಬರು-ತ್ತಿತ್ತು
ಬರು-ತ್ತಿದೆ
ಬರು-ತ್ತಿ-ದೆ-ಯೇನೋ
ಬರು-ತ್ತಿದ್ದ
ಬರು-ತ್ತಿ-ದ್ದಂ-ತೆಯೇ
ಬರು-ತ್ತಿ-ದ್ದರು
ಬರು-ತ್ತಿ-ದ್ದರೆ
ಬರು-ತ್ತಿ-ದ್ದರೋ
ಬರು-ತ್ತಿ-ದ್ದ-ವ-ರಲ್ಲಿ
ಬರು-ತ್ತಿ-ದ್ದವು
ಬರು-ತ್ತಿ-ದ್ದಾ-ನಲ್ಲ
ಬರು-ತ್ತಿ-ದ್ದಾನೆ
ಬರು-ತ್ತಿ-ದ್ದಾರೆ
ಬರು-ತ್ತಿ-ದ್ದಾ-ರೆಂಬ
ಬರು-ತ್ತಿ-ದ್ದು-ದಕ್ಕೆ
ಬರು-ತ್ತಿ-ರ-ಲಿಲ್ಲ
ಬರು-ತ್ತಿ-ರ-ವುದನ್ನು
ಬರು-ತ್ತಿರು
ಬರು-ತ್ತಿ-ರುವ
ಬರು-ತ್ತಿ-ರು-ವಂ-ತಿತ್ತು
ಬರು-ತ್ತಿ-ರು-ವಾಗ
ಬರು-ತ್ತಿ-ರು-ವುದನ್ನು
ಬರು-ತ್ತಿ-ರು-ವುದು
ಬರು-ತ್ತಿ-ರುವೆ
ಬರು-ತ್ತಿಲ್ಲ
ಬರು-ತ್ತೀ-ರಲ್ಲ
ಬರು-ತ್ತೀರೋ
ಬರುತ್ತೇ
ಬರು-ತ್ತೇನೆ
ಬರು-ತ್ತೇ-ನೆ-ಎ-ನ್ನ-ಬೇ-ಕಾ-ಯಿತು
ಬರುವ
ಬರು-ವಂ-ತಾ-ದರೆ
ಬರು-ವಂ-ತಿತ್ತು
ಬರು-ವಂ-ತಿಲ್ಲ
ಬರು-ವಂತೆ
ಬರು-ವ-ರೆಂದು
ಬರು-ವ-ವನೇ
ಬರು-ವ-ವ-ರಿ-ದ್ದಾರೆ
ಬರು-ವ-ವರು
ಬರು-ವ-ವ-ರೆಗೆ
ಬರು-ವ-ವಳ
ಬರು-ವಾಗ
ಬರು-ವಾ-ಗಲೇ
ಬರು-ವುದ
ಬರು-ವು-ದಕ್ಕೆ
ಬರು-ವು-ದ-ಕ್ಕೊಂದು
ಬರು-ವುದನ್ನು
ಬರು-ವು-ದಾಗಿ
ಬರು-ವು-ದಿಲ್ಲ
ಬರು-ವು-ದಿ-ಲ್ಲ-ವಲ್ಲ
ಬರು-ವು-ದಿ-ಲ್ಲವೆ
ಬರು-ವುದು
ಬರು-ವು-ದು-ಹೋ-ಗು-ವು-ದನ್ನೇ
ಬರು-ವುದೂ
ಬರು-ವುದೇ
ಬರು-ವು-ದೇನು
ಬರುವೆ
ಬರೆ
ಬರೆದ
ಬರೆ-ದ-ದ್ದು-ನನ್ನ
ಬರೆ-ದದ್ದೂ
ಬರೆ-ದರು
ಬರೆ-ದರೋ
ಬರೆ-ದಳು
ಬರೆ-ದಿಟ್ಟು
ಬರೆ-ದಿ-ಟ್ಟು-ಕೊಂ-ಡಿ-ದ್ದರು
ಬರೆ-ದಿ-ಟ್ಟು-ಕೊಂ-ಡಿದ್ದು
ಬರೆ-ದಿ-ಡ-ಲಿಲ್ಲ
ಬರೆ-ದಿದೆ
ಬರೆ-ದಿದ್ದ
ಬರೆ-ದಿ-ದ್ದನ್ನು
ಬರೆ-ದಿ-ದ್ದರು
ಬರೆ-ದಿ-ದ್ದಾ-ದಲ್ಲ
ಬರೆ-ದಿ-ದ್ದಾರೆ
ಬರೆ-ದಿಲ್ಲ
ಬರೆದು
ಬರೆ-ದು-ಕೊ-ಟ್ಟರು
ಬರೆ-ದು-ಬಿಟ್ಟ
ಬರೆ-ಯ-ಬ-ಲ್ಲ-ವ-ನಾ-ಗಿದ್ದ
ಬರೆ-ಯ-ಬೇ-ಕಾಗಿ
ಬರೆ-ಯ-ಬೇ-ಕಾ-ಗಿತ್ತು
ಬರೆ-ಯಲೂ
ಬರೆಯು
ಬರೆ-ಯು-ತ್ತಾರೆ
ಬರೆ-ಯು-ತ್ತಿ-ದ್ದರು
ಬರೆ-ಯು-ತ್ತಿ-ದ್ದಾರೆ
ಬರೆ-ಯುವ
ಬಲ
ಬಲ-ಗೈ-ಯಲ್ಲಿ
ಬಲ-ಗೈ-ಯಿಂದ
ಬಲ-ಗೊಂ-ಡಿತು
ಬಲ-ಗೊಂಡು
ಬಲ-ಗೊ-ಳ್ಳ-ತೊ-ಡ-ಗಿತು
ಬಲದ
ಬಲ-ದಿಂದ
ಬಲ-ದಿಂ-ದಲೇ
ಬಲ-ಪ-ಡಿ-ಸಿ-ಕೊಂಡ
ಬಲ-ಪ-ಡಿ-ಸಿ-ಕೊಳ್ಳ
ಬಲ-ಪ-ಡಿ-ಸಿ-ಕೊ-ಳ್ಳ-ಬೇಕು
ಬಲ-ಪ-ಡಿ-ಸಿ-ಕೊ-ಳ್ಳುವ
ಬಲ-ಪ-ಡಿ-ಸಿ-ಕೊ-ಳ್ಳೋಣ
ಬಲ-ಪ-ಡಿ-ಸು-ತ್ತಿ-ದ್ದರು
ಬಲ-ಪಾ-ದ-ವ-ನ್ನೆತ್ತಿ
ಬಲ-ಭಾ-ಗಕ್ಕೆ
ಬಲ-ಭಾ-ಗ-ದಲ್ಲಿ
ಬಲ-ಯು-ತ-ವಾದ
ಬಲ-ರಾಮ
ಬಲ-ರಾ-ಮನ
ಬಲ-ರಾ-ಮ-ಬಾಬು
ಬಲ-ರಾ-ಮ-ಬಾ-ಬು-ವಿನ
ಬಲ-ರಾ-ಮ-ಬಾ-ಬುವೂ
ಬಲ-ರಾಮ್
ಬಲ-ರಾ-ಮ್ಬಾಬು
ಬಲ-ವಂತ
ಬಲ-ವಂ-ತಕ್ಕೆ
ಬಲ-ವಂ-ತ-ದಿಂದ
ಬಲ-ವಂ-ತ-ದಿಂ-ದಾಗಿ
ಬಲ-ವಂ-ತ-ಪ-ಡಿ-ಸಿ-ದಾಗ
ಬಲ-ವಂ-ತ-ವಾಗಿ
ಬಲ-ವ-ತ್ತ-ರ-ವಾಗಿ
ಬಲ-ವಾಗಿ
ಬಲ-ವಾ-ಗಿದ್ದ
ಬಲ-ವಾ-ಗಿಯೇ
ಬಲ-ವಾದ
ಬಲ-ವಾ-ಯಿತು
ಬಲ-ವಿ-ದೆಯೆ
ಬಲವೂ
ಬಲ-ಶಾ-ಲಿ-ಯಾ-ಗಿ-ರ-ಬೇಕು
ಬಲ-ಹು-ಬ್ಬಿನ
ಬಲಾ-ತ್ಕಾ-ರ-ದಿಂದ
ಬಲಾ-ತ್ಕಾ-ರ-ದಿಂ-ದಲ್ಲ
ಬಲಿ
ಬಲಿ-ದಾನ
ಬಲಿ-ಯಾ-ಗ-ಲಾ-ರಂ-ಭಿ-ಸಿ-ದ್ದರು
ಬಲಿ-ಯಾ-ಗಲು
ಬಲಿ-ಯಾಗಿ
ಬಲಿ-ಯಾ-ಗಿ-ರು-ವುದನ್ನು
ಬಲಿ-ಯಾ-ಗು-ತ್ತಿ-ರು-ವು-ದರ
ಬಲಿ-ಯಾ-ದ-ದ್ದ-ರಲ್ಲಿ
ಬಲಿ-ಯು-ತ್ತಿತ್ತು
ಬಲಿ-ಷ್ಠ-ಗೊಂಡು
ಬಲಿ-ಷ್ಠ-ನಾ-ದರೆ
ಬಲಿ-ಷ್ಠ-ರೇ-ನಿ-ರ್ಧಾ-ರ-ದಲ್ಲಿ
ಬಲು
ಬಲೆಯ
ಬಲೆ-ಯನ್ನು
ಬಲೆ-ಯಲ್ಲಿ
ಬಲೆ-ಯಿಂದ
ಬಲ್ಲ
ಬಲ್ಲ-ದ-ವ-ನೇ-ನಲ್ಲ
ಬಲ್ಲ-ನೆಂದೂ
ಬಲ್ಲರು
ಬಲ್ಲ-ವ-ನಾ-ಗಿದ್ದ
ಬಲ್ಲ-ವರ
ಬಲ್ಲ-ವ-ರಾರು
ಬಲ್ಲ-ವ-ರಿ-ಗೆಲ್ಲ
ಬಲ್ಲ-ವರು
ಬಲ್ಲೆ
ಬಲ್ಲೆವು
ಬಳ-ಪದ
ಬಳ-ಲಿಕೆ
ಬಳ-ಲಿ-ಕೆ-ಯನ್ನು
ಬಳ-ಲಿ-ಕೆ-ಯಾ-ಗಿ-ದ್ದರೂ
ಬಳ-ಲಿತ್ತು
ಬಳ-ಲಿದ
ಬಳ-ಲಿದೆ
ಬಳ-ಲಿ-ದ್ದಾರೆ
ಬಳ-ಸ-ಬ-ಹುದು
ಬಳಸಿ
ಬಳ-ಸಿ-ದ್ದರು
ಬಳ-ಸುತ್ತ
ಬಳ-ಸು-ತ್ತಾನೆ
ಬಳ-ಸು-ತ್ತಾರೆ
ಬಳ-ಸು-ಮಾ-ತು-ಗ-ಳಾ-ಗಲಿ
ಬಳ-ಸು-ವು-ದಕ್ಕೆ
ಬಳ-ಸು-ವು-ದನ್ನಾ
ಬಳಿ
ಬಳಿಕ
ಬಳಿಗೂ
ಬಳಿಗೆ
ಬಳಿಗೇ
ಬಳಿ-ಗೋಡಿ
ಬಳಿ-ಗೋ-ಡಿದ
ಬಳಿ-ಗೋ-ಡು-ತ್ತಿದ್ದ
ಬಳಿ-ದು-ಕೊಂಡು
ಬಳಿ-ದು-ಕೊಂಡೇ
ಬಳಿ-ಯಲ್ಲಿ
ಬಳಿ-ಯ-ಲ್ಲಿದ್ದ
ಬಳಿ-ಯಲ್ಲೇ
ಬಳಿ-ಯಿದ್ದ
ಬಳಿ-ಯಿ-ರು-ತ್ತದೆ
ಬಳಿ-ಯಿ-ರುವ
ಬಳಿ-ಯಿ-ರು-ವ-ವರೆ-ಲ್ಲರೂ
ಬಳಿ-ಸಾರಿ
ಬಳಿ-ಸಾ-ರಿ-ದರು
ಬಳೆ-ಗಳನ್ನು
ಬಳೆ-ಗಳನ್ನೂ
ಬಳ್ಳಿ-ಗಳು
ಬವ-ಣೆ-ಗಳ
ಬವ-ಣೆ-ಗಳನ್ನು
ಬವ-ಣೆಗೆ
ಬವ-ಣೆ-ಯನ್ನು
ಬವ-ಣೆ-ಯೊ-ಳಗೆ
ಬವಳಿ
ಬಸ-ವ-ಳಿದು
ಬಸು
ಬಹ
ಬಹಳ
ಬಹ-ಳ-ವಾಗಿ
ಬಹ-ಳ-ವಾ-ಗಿಯೇ
ಬಹ-ಳವೇ
ಬಹ-ಳ-ಷ್ಟಿವೆ
ಬಹ-ಳಷ್ಟು
ಬಹಿ-ರಂಗ
ಬಹಿ-ರಂ-ಗ-ಪ-ಡಿ-ಸು-ವಂ-ತಾ-ಯಿ-ತ-ಲ್ಲವೆ
ಬಹಿ-ರಂ-ಗ-ವ-ಲ-ಯಕ್ಕೆ
ಬಹಿ-ರ್ಮು-ಖ-ರಾ-ದರು
ಬಹು
ಬಹು-ಎ-ತ್ತ-ರ-ದಲ್ಲಿ
ಬಹು-ಕಾಲ
ಬಹು-ಕಾ-ಲದ
ಬಹು-ಜನ
ಬಹು-ತೇಕ
ಬಹು-ದಾ-ಗಿತ್ತು
ಬಹು-ದಿ-ನ-ಗ-ಳ-ವ-ರೆಗೆ
ಬಹುದು
ಬಹು-ದು-ಇ-ದೇನು
ಬಹು-ದೂರ
ಬಹು-ದೂ-ರದ
ಬಹುದೆ
ಬಹು-ದೆಂಬ
ಬಹು-ದೇ-ವತಾ
ಬಹುನಾ
ಬಹು-ಬೇಗ
ಬಹು-ಭಾ-ಗ-ವನ್ನು
ಬಹು-ಮ-ಖ್ಯ-ವಾದ
ಬಹು-ಮ-ಟ್ಟಿಗೆ
ಬಹು-ಮಾ-ನ-ವಾಗಿ
ಬಹು-ಮುಖ
ಬಹು-ಮು-ಖ-ವಾ-ದದ್ದು
ಬಹು-ಮುಖ್ಯ
ಬಹು-ಮು-ಖ್ಯ-ವಾದ
ಬಹು-ವ-ರ್ಷ-ಗಳ
ಬಹು-ವಾಗಿ
ಬಹುಶಃ
ಬಹೂ-ದಕ
ಬಹೂ-ದ-ಕ-ನಾ-ಗು-ತ್ತಾನೆ
ಬಹೂ-ದ-ಕ-ನೆಂ-ದರೆ
ಬಹೆ
ಬಾ
ಬಾಂಕು-ಬಿ-ಹಾರಿ
ಬಾಂಧವ್ಯ
ಬಾಂಧ-ವ್ಯಕ್ಕೆ
ಬಾಂಧ-ವ್ಯದ
ಬಾಂಧ-ವ್ಯ-ವ-ನ್ನಿ-ಟ್ಟು-ಕೊಂ-ಡಿ-ದ್ದರು
ಬಾಂಧ-ವ್ಯ-ವನ್ನು
ಬಾಂಧ-ವ್ಯ-ವಿತ್ತು
ಬಾಂಬಿ-ನಂತೆ
ಬಾಕಿ
ಬಾಗಲಿ
ಬಾಗಿ
ಬಾಗಿಲ
ಬಾಗಿ-ಲ-ನ-ಲ್ಲಿ-ಹುದು
ಬಾಗಿ-ಲನ್ನು
ಬಾಗಿ-ಲಲ್ಲಿ
ಬಾಗಿ-ಲಲ್ಲೇ
ಬಾಗಿ-ಲ-ವ-ರೆಗೆ
ಬಾಗಿ-ಲಿಗೆ
ಬಾಗಿಲು
ಬಾಗಿ-ಲು-ಗಳನ್ನು
ಬಾಗ್ಬ-ಜಾ-ರಿ-ನಲ್ಲಿ
ಬಾಡಿಗೆ
ಬಾಡಿ-ಗೆಗೆ
ಬಾಡಿ-ಗೆಯ
ಬಾಡಿ-ಗೆ-ಯನ್ನು
ಬಾಡಿ-ಗೆ-ಯನ್ನೂ
ಬಾಣ-ದಂತೆ
ಬಾಣ-ಸಿಗ
ಬಾಧಕ
ಬಾಧ-ಕ-ವಾ-ದದ್ದು
ಬಾಧ-ಕ-ವಾ-ದ-ವು-ಗಳು
ಬಾಧಿ-ತ-ವಾ-ಗದು
ಬಾನೊ-ಳಾ-ಡುವ
ಬಾಬಾ
ಬಾಬಾಜಿ
ಬಾಬಾ-ಜಿ-ಯ-ವರ
ಬಾಬಾ-ಜಿ-ಯ-ವ-ರನ್ನು
ಬಾಬಾ-ಜಿ-ಯ-ವ-ರಿಗೆ
ಬಾಬಾರ
ಬಾಬಾ-ರನ್ನು
ಬಾಬಾ-ರಿಂದ
ಬಾಬಾ-ರಿಗೆ
ಬಾಬು
ಬಾಬು-ರಾಮ
ಬಾಬು-ರಾ-ಮ-ಇ-ವರೆಲ್ಲ
ಬಾಬು-ರಾ-ಮನ
ಬಾಬು-ರಾಮ್
ಬಾಬು-ವಿಗೆ
ಬಾಬು-ವಿನ
ಬಾಯಲ್ಲಿ
ಬಾಯ-ಲ್ಲಿ-ರು-ತ್ತಿತ್ತು
ಬಾಯಿ
ಬಾಯಿಂದ
ಬಾಯಿ-ಗಳ
ಬಾಯಿ-ಗಿ-ಟ್ಟು-ಕೊಂ-ಡರೆ
ಬಾಯಿ-ಗೀ-ಡಾ-ಗು-ತ್ತಿ-ರು-ವಂತೆ
ಬಾಯಿಗೆ
ಬಾಯಿ-ಪಾಠ
ಬಾಯಿ-ಬಿ-ಟ್ಟಿ-ರ-ಲಿಲ್ಲ
ಬಾಯಿ-ಬಿಟ್ಟು
ಬಾಯಿ-ಮಾ-ತಾ-ಗಿ-ರಲು
ಬಾಯಿ-ಮು-ಚ್ಚಿಸಿ
ಬಾಯಿ-ಮು-ಚ್ಚಿ-ಸಿ-ಬಿ-ಡು-ತ್ತಿ-ದ್ದರು
ಬಾಯೊ-ಳಗೆ
ಬಾಯ್ತುಂಬ
ಬಾಯ್ದೆ-ರೆದ
ಬಾರ
ಬಾರದ
ಬಾರ-ದಂತೆ
ಬಾರ-ದಪ್ಪ
ಬಾರ-ದಿ-ದ್ದರೆ
ಬಾರ-ದಿ-ದ್ದಾಗ
ಬಾರದು
ಬಾರದೆ
ಬಾರ-ದೆಯೇ
ಬಾರದೇ
ಬಾರದ್ದು
ಬಾರಯ್ಯ
ಬಾರಾ
ಬಾರಾ-ನ-ಗೋ-ರನ್ನು
ಬಾರಾ-ನ-ಗೋ-ರಿಗೆ
ಬಾರಾ-ನ-ಗೋ-ರಿನ
ಬಾರಾ-ನ-ಗೋ-ರಿ-ನಿಂದ
ಬಾರಾ-ನ-ಗೋರ್
ಬಾರಾ-ನ-ಗೋರ್ನ
ಬಾರಾ-ನಾ-ಗೋರ್
ಬಾರಿ
ಬಾರಿಗೆ
ಬಾರಿಯೂ
ಬಾರಿ-ಸ-ತೊ-ಡ-ಗಿ-ದರು
ಬಾರಿಸಿ
ಬಾರಿ-ಸು-ತ್ತಲೇ
ಬಾಲ
ಬಾಲಕ
ಬಾಲ-ಕ-ನಂತೆ
ಬಾಲ-ಕರು
ಬಾಲ-ಕ-ಲ್ಪನೆ
ಬಾಲ-ಕ-ಸ್ವ-ಭಾವ
ಬಾಲ-ಚಂ-ದ್ರ-ಧ-ರ-ಮಿಂದು
ಬಾಲ-ಚಂ-ದ್ರ-ನನ್ನು
ಬಾಲ-ಜ-ಗ-ತ್ತನ್ನು
ಬಾಲ-ನಾ-ಗಿ-ರು-ವಾ-ಗಲೇ
ಬಾಲ-ಭಾ-ಷೆ-ಯಲ್ಲಿ
ಬಾಲ-ರಾ-ಮನ
ಬಾಲ-ರಾ-ಮ-ನನ್ನು
ಬಾಲ-ವಿ-ಧವೆ
ಬಾಲ-ವಿ-ಧ-ವೆ-ಯಾ-ಗುವ
ಬಾಲ-ವಿ-ಧ-ವೆ-ಯಿ-ದ್ದಳು
ಬಾಲ-ಸ-ಹಜ
ಬಾಲ-ಸ-ಹ-ಜ-ವಾದ
ಬಾಲ-ಹೃ-ದ-ಯಕ್ಕೆ
ಬಾಲಿ
ಬಾಲಿಶ
ಬಾಲ್ಯ
ಬಾಲ್ಯ-ಕೌ-ಮಾ-ರ್ಯ-ಗ-ಳಿಂ-ದಲೇ
ಬಾಲ್ಯದ
ಬಾಲ್ಯ-ದಲ್ಲಿ
ಬಾಲ್ಯ-ದಲ್ಲೇ
ಬಾಲ್ಯ-ದಿಂ-ದಲೂ
ಬಾಲ್ಯ-ದಿಂ-ದಲೇ
ಬಾಲ್ಯ-ವನ್ನು
ಬಾಲ್ಯ-ವಿ-ವಾಹ
ಬಾಲ್ಯ-ವಿ-ವಾ-ಹದ
ಬಾಲ್ಯ-ವಿ-ವಾ-ಹವೇ
ಬಾಲ್ಯ-ವೆಂಬ
ಬಾಲ್ಯವೇ
ಬಾಳ-ಮೋ-ಹವು
ಬಾಳಲಿ
ಬಾಳಿನ
ಬಾಳುಗೆ
ಬಾಳು-ತ್ತಿ-ದ್ದ-ವನು
ಬಾಳೆಂಬ
ಬಾಳೆಯ
ಬಾಳ್ವೆ
ಬಾವ
ಬಾವಲಿ
ಬಾವಿ-ಗ-ಳ-ಕೆ-ರೆ-ಗಳ
ಬಾವಿ-ಸಿ-ದ್ದುಂಟು
ಬಾವುಲ್
ಬಾಹು-ಗಳನ್ನು
ಬಾಹ್ಯ
ಬಾಹ್ಯ-ಜ-ಗ-ತ್ತಿನ
ಬಾಹ್ಯ-ಪ್ರಜ್ಞೆ
ಬಾಹ್ಯ-ಪ್ರ-ಜ್ಞೆ-ಯನ್ನೇ
ಬಾಹ್ಯ-ಪ್ರ-ಜ್ಞೆಯೇ
ಬಾಹ್ಯ-ಪ್ರ-ಪಂಚ
ಬಾಹ್ಯ-ಪ್ರ-ಪಂ-ಚಕ್ಕೆ
ಬಾಹ್ಯ-ಪ್ರ-ಪಂ-ಚ-ದಲ್ಲಿ
ಬಾಹ್ಯಾ-ಚ-ರ-ಣೆ-ಗ-ಳಿಗೆ
ಬಾಹ್ಯಾ-ಡಂ-ಬ-ರದ
ಬಿ
ಬಿಂಕ
ಬಿಂಬಿ-ಸು-ತ್ತಿತ್ತು
ಬಿಎ
ಬಿಎಲ್
ಬಿಎ-ವ-ರೆಗೆ
ಬಿಕ್ಕಿ-ದರು
ಬಿಕ್ಕಿ-ಬಿಕ್ಕಿ
ಬಿಗಿದ
ಬಿಗಿ-ದಿಟ್ಟ
ಬಿಗಿ-ದಿ-ಟ್ಟಿ-ರುವ
ಬಿಗಿ-ದಿಹ
ಬಿಗಿ-ಯಲು
ಬಿಗಿ-ಯಾಗಿ
ಬಿಗಿ-ಯು-ತ್ತವೆ
ಬಿಗಿ-ವ-ರಿ-ರಿ-ವರು
ಬಿಗಿ-ಹಿ-ಡಿ-ದು-ಕೊಂಡು
ಬಿಗು-ಮಾ-ನ-ಗ-ಳಿ-ರ-ಲಿಲ್ಲ
ಬಿಚ್ಚಿ
ಬಿಚ್ಚಿ-ಕೊಂಡು
ಬಿಚ್ಚು
ಬಿಟ್ಟ
ಬಿಟ್ಟನೋ
ಬಿಟ್ಟ-ಮೇಲೆ
ಬಿಟ್ಟರು
ಬಿಟ್ಟರೂ
ಬಿಟ್ಟರೆ
ಬಿಟ್ಟ-ರೇನೋ
ಬಿಟ್ಟಳು
ಬಿಟ್ಟಾ-ಗಿ-ನಿಂ-ದಲೂ
ಬಿಟ್ಟಾ-ರೆಯೆ
ಬಿಟ್ಟಿತು
ಬಿಟ್ಟಿತ್ತು
ಬಿಟ್ಟಿದೆ
ಬಿಟ್ಟಿದ್ದ
ಬಿಟ್ಟಿ-ದ್ದರೆ
ಬಿಟ್ಟಿ-ದ್ದಾರೆ
ಬಿಟ್ಟಿ-ದ್ದಾ-ರೆಯೇ
ಬಿಟ್ಟಿ-ರ-ಲಿಲ್ಲ
ಬಿಟ್ಟಿ-ರು-ವ-ವ-ರ-ಲ್ಲವೆ
ಬಿಟ್ಟಿ-ರು-ವು-ದಾ-ಗಿಯೂ
ಬಿಟ್ಟಿವೆ
ಬಿಟ್ಟು
ಬಿಟ್ಟು-ಕೊ-ಡ-ಬಾ-ರದು
ಬಿಟ್ಟು-ಕೊ-ಡ-ಲಾರೆ
ಬಿಟ್ಟು-ಕೊ-ಡಲು
ಬಿಟ್ಟು-ಕೊ-ಡು-ವವ
ಬಿಟ್ಟು-ಕೊ-ಡು-ವ-ವ-ನಲ್ಲ
ಬಿಟ್ಟು-ಬಂ-ದರೆ
ಬಿಟ್ಟು-ಬಂ-ದಿ-ದ್ದ-ರಾ-ದರೂ
ಬಿಟ್ಟು-ಬಿಟ್ಟ
ಬಿಟ್ಟು-ಬಿ-ಟ್ಟರೆ
ಬಿಟ್ಟು-ಬಿ-ಟ್ಟಿ-ದ್ದರೆ
ಬಿಟ್ಟು-ಬಿ-ಟ್ಟಿ-ದ್ದಾರೆ
ಬಿಟ್ಟು-ಬಿಟ್ಟೆ
ಬಿಟ್ಟು-ಬಿ-ಡ-ಬೇಕು
ಬಿಟ್ಟು-ಬಿ-ಡ-ಬೇ-ಕೆಂ-ದಿ-ದ್ದೇನೆ
ಬಿಟ್ಟು-ಬಿ-ಡಲು
ಬಿಟ್ಟು-ಬಿಡಿ
ಬಿಟ್ಟು-ಬಿಡು
ಬಿಟ್ಟು-ಬಿ-ಡು-ತ್ತಾನೆ
ಬಿಟ್ಟು-ಬಿ-ಡುವ
ಬಿಟ್ಟು-ಬಿ-ಡು-ವು-ದೊಂದೇ
ಬಿಟ್ಟುವು
ಬಿಟ್ಟು-ಹೋ-ಗಲು
ಬಿಟ್ಟು-ಹೋ-ಗಿ-ದ್ದಾನೆ
ಬಿಟ್ಟೆ
ಬಿಟ್ಟೆ-ಯಲ್ಲ
ಬಿಡ-ಗಡೆ
ಬಿಡದೆ
ಬಿಡ-ದೆ-ಮಾ-ಡಿ-ಕೊಂಡು
ಬಿಡ-ಬಲ್ಲ
ಬಿಡ-ಬೇ-ಕಾಗಿ
ಬಿಡ-ಬೇ-ಕಾ-ಯಿತು
ಬಿಡ-ಬೇಕು
ಬಿಡ-ಬೇ-ಕೆಂದೇ
ಬಿಡ-ಲಾರ
ಬಿಡ-ಲಾ-ರವು
ಬಿಡ-ಲಾ-ರೆವು
ಬಿಡ-ಲಿಲ್ಲ
ಬಿಡ-ಲಿ-ಲ್ಲ-ವಲ್ಲ
ಬಿಡಲು
ಬಿಡಲೂ
ಬಿಡಲೇ
ಬಿಡ-ಲೇ-ಬಾ-ರದು
ಬಿಡಾರ
ಬಿಡಿ
ಬಿಡಿ-ಸ-ಲಾ-ಗದ
ಬಿಡಿ-ಸ-ಲಾ-ರದ
ಬಿಡಿ-ಸಲು
ಬಿಡಿಸಿ
ಬಿಡಿ-ಸಿ-ಕೊಂಡು
ಬಿಡಿ-ಸಿ-ಕೊಂ-ಡು-ಹೋಗಿ
ಬಿಡಿ-ಸಿ-ಕೊ-ಳ್ಳ-ಬೇ-ಕಾದ್ದೇ
ಬಿಡಿ-ಸಿ-ಕೊ-ಳ್ಳು-ವುದು
ಬಿಡಿ-ಸಿ-ದರು
ಬಿಡಿ-ಸು-ತ್ತಿದ್ದ
ಬಿಡಿ-ಸುವ
ಬಿಡು
ಬಿಡು-ಗಡೆ
ಬಿಡು-ಗ-ಡೆ-ಗೆ-ಎಂ-ದರೆ
ಬಿಡುತ್ತ
ಬಿಡು-ತ್ತದೆ
ಬಿಡು-ತ್ತಾನೆ
ಬಿಡು-ತ್ತಾ-ನೆಯೆ
ಬಿಡು-ತ್ತಿತ್ತು
ಬಿಡು-ತ್ತಿದ್ದ
ಬಿಡು-ತ್ತಿ-ದ್ದರು
ಬಿಡು-ತ್ತಿ-ರ-ಲಿಲ್ಲ
ಬಿಡು-ತ್ತೇನೆ
ಬಿಡು-ವಂತೆ
ಬಿಡು-ವ-ವ-ನಲ್ಲ
ಬಿಡು-ವ-ವ-ನಿ-ದ್ದೇನೆ
ಬಿಡು-ವ-ವ-ರಲ್ಲ
ಬಿಡು-ವಾ-ಗಿ-ದ್ದಾ-ಗ-ಲೆಲ್ಲ
ಬಿಡು-ವಿದೆ
ಬಿಡು-ವಿನ
ಬಿಡು-ವು-ದಕ್ಕೂ
ಬಿಡು-ವುದೂ
ಬಿಡು-ವುದೆ
ಬಿಡು-ವು-ದೆಂದು
ಬಿಡು-ವುದೋ
ಬಿತ್ತ-ರಿ-ಸ-ಬೇಕು
ಬಿತ್ತ-ರಿ-ಸ-ಲಾರ
ಬಿತ್ತ-ರಿ-ಸಲೂ
ಬಿತ್ತ-ರಿ-ಸ-ಲೆಂದು
ಬಿತ್ತ-ರಿಸಿ
ಬಿತ್ತಿ-ದಂ-ತಾ-ಯಿತು
ಬಿತ್ತಿ-ದಂತೆ
ಬಿತ್ತಿ-ದಾ-ತನು
ಬಿತ್ತು
ಬಿದಿಯು
ಬಿದಿ-ರನ್ನು
ಬಿದ್ದ
ಬಿದ್ದ-ದ್ದನ್ನು
ಬಿದ್ದ-ದ್ದ-ರಿಂದ
ಬಿದ್ದದ್ದೂ
ಬಿದ್ದರೆ
ಬಿದ್ದಾಗ
ಬಿದ್ದಿದೆ
ಬಿದ್ದಿದ್ದ
ಬಿದ್ದಿ-ದ್ದಾನೆ
ಬಿದ್ದಿ-ರಲು
ಬಿದ್ದಿ-ರು-ತ್ತಿ-ದ್ದು-ದನ್ನು
ಬಿದ್ದಿ-ರುವ
ಬಿದ್ದಿ-ರು-ವುದನ್ನು
ಬಿದ್ದಿ-ಹುದು
ಬಿದ್ದು
ಬಿದ್ದು-ಬಿಟ್ಟ
ಬಿದ್ದು-ಬಿ-ಟ್ಟರು
ಬಿದ್ದು-ಬಿ-ಟ್ಟಳು
ಬಿದ್ದು-ಬಿ-ಟ್ಟಿ-ದ್ದಾನೆ
ಬಿದ್ದು-ಬಿ-ಡು-ತ್ತಿದ್ದ
ಬಿದ್ದು-ಹೋಗಿ
ಬಿದ್ದು-ಹೋ-ಗು-ತ್ತದೆ
ಬಿದ್ದು-ಹೋ-ಗು-ತ್ತಿತ್ತು
ಬಿದ್ದು-ಹೋದ
ಬಿದ್ದು-ಹೋ-ದುವು
ಬಿದ್ದು-ಹೋ-ಯಿತು
ಬಿದ್ದೆ
ಬಿರಿ-ಯು-ವಂತೆ
ಬಿರು-ಕನ್ನೇ
ಬಿರುಕು
ಬಿರು-ಕು-ಬಿ-ಟ್ಟಿತು
ಬಿರು-ಗಾ-ಳಿಗೆ
ಬಿರು-ಗಾ-ಳಿ-ಯನ್ನು
ಬಿರು-ಗಾ-ಳಿ-ಯನ್ನೇ
ಬಿರುದು
ಬಿರು-ಸಾದ
ಬಿರು-ಸಿ-ನಿಂದ
ಬಿರುಸು
ಬಿರುಸೇ
ಬಿಲಾ-ಸ-ಪುರ
ಬಿಲೇ
ಬಿಲ್ಲಂ-ಬೆ-ರ-ಗಾ-ದರು
ಬಿಲ್ವ-ಸ-ಮಿ-ತ್ತು-ಗಳನ್ನು
ಬಿಲ್ವ-ಪ-ತ್ರೆ-ಗಳನ್ನು
ಬಿಲ್ವ-ಪ-ತ್ರೆ-ಯನ್ನು
ಬಿಲ್ವ-ವೃ-ಕ್ಷದ
ಬಿಳ-ಲು-ಗಳನ್ನು
ಬಿಳಿ
ಬಿಳಿಯ
ಬಿಳಿ-ಯ-ಬ-ಟ್ಟೆ-ಯ-ಲ್ಲಿ-ದ್ದರೂ
ಬಿಳಿ-ಯ-ರನ್ನು
ಬಿಳೀ
ಬಿಸಿ
ಬಿಸಿ-ಬಿಸಿ
ಬಿಸಿ-ಯಾ-ಗದೆ
ಬಿಸಿ-ಯೇ-ರಿತು
ಬಿಸಿ-ಯೇ-ರು-ವಂತೆ
ಬಿಸಿಯೊ
ಬಿಸಿಯೋ
ಬಿಸಿ-ರ-ಕ್ತದ
ಬಿಸಿ-ಲಿ-ನಲ್ಲಿ
ಬಿಸಿ-ಲಿ-ನಲ್ಲೂ
ಬಿಸುಡು
ಬಿಸುಡೈ
ಬೀಗ
ಬೀಗ-ಮುದ್ರೆ
ಬೀಗ-ಮು-ದ್ರೆ-ಯ-ನಿಟ್ಟ
ಬೀಗ-ಮು-ದ್ರೆ-ಯನ್ನು
ಬೀಗ-ವನ್ನು
ಬೀಗಿತು
ಬೀಗು-ತ್ತಿತ್ತು
ಬೀಜ
ಬೀಜ-ದಂತೆ
ಬೀಡು
ಬೀದಿ-ಗಳಲ್ಲಿ
ಬೀದಿ-ಯಲ್ಲಿ
ಬೀಭತ್ಸ
ಬೀರ-ಬ-ಹು-ದಾದ
ಬೀರ-ಲಾ-ರಂ-ಭಿ-ಸಿತ್ತು
ಬೀರ-ಲಿ-ಕ್ಕಿದೆ
ಬೀರ-ಲಿಲ್ಲ
ಬೀರಿ
ಬೀರಿತು
ಬೀರಿದ
ಬೀರಿ-ದರು
ಬೀರಿ-ದರೋ
ಬೀರಿ-ದುವು
ಬೀರಿದೆ
ಬೀರಿದ್ದ
ಬೀರಿ-ದ್ದುವು
ಬೀರಿ-ಬಿ-ಟ್ಟಿತು
ಬೀರಿ-ಬಿ-ಟ್ಟಿತ್ತು
ಬೀರು
ಬೀರುತ್ತ
ಬೀರು-ತ್ತವೆ
ಬೀರು-ತ್ತಿತ್ತು
ಬೀರು-ವಂ-ತಹ
ಬೀರು-ವಂ-ತಾ-ಗ-ಬೇಕು
ಬೀರು-ವಷ್ಟು
ಬೀರು-ವಿಕೆ
ಬೀರು-ವುದು
ಬೀಳ-ದಂತೆ
ಬೀಳದೆ
ಬೀಳ-ಬ-ಹು-ದೆಂ-ಬುದು
ಬೀಳಲಿ
ಬೀಳ-ಲಿಲ್ಲ
ಬೀಳಲು
ಬೀಳಿ-ಸಲೇ
ಬೀಳಿ-ಸು-ವಂತೆ
ಬೀಳು-ತ್ತ-ದೆಯೇ
ಬೀಳು-ತ್ತವೆ
ಬೀಳು-ತ್ತಿದ್ದ
ಬೀಳುವ
ಬೀಳು-ವಂ-ತಾ-ಗಿತ್ತು
ಬೀಳು-ವಂತೆ
ಬೀಳು-ವ-ದೆ-ಲ್ಲಿಗೆ
ಬೀಳು-ವುದನ್ನು
ಬೀಳು-ವು-ದು-ಇ-ದ-ನ್ನೆಲ್ಲ
ಬೀಳ್ಕೊ-ಟ್ಟರು
ಬೀಳ್ಗೊಂಡ
ಬೀಳ್ಗೊಂ-ಡರು
ಬೀಳ್ಗೊಂ-ಡಿದ್ದ
ಬೀಳ್ಗೊಂಡು
ಬೀಳ್ಗೊಡ
ಬೀಳ್ಗೊ-ಡಲು
ಬೀಳ್ಗೊ-ಳ್ಳ-ಲಿ-ದ್ದೇನೆ
ಬೀಳ್ಗೊ-ಳ್ಳಲು
ಬೀಸ-ಣಿಗೆ
ಬೀಸ-ಬಲ್ಲೆ
ಬೀಸ-ಬೇಕೋ
ಬೀಸಿ
ಬೀಸಿದ
ಬೀಸಿ-ದ-ನೆಂ-ದರೆ
ಬೀಸು-ತಿ-ಹುದು
ಬೀಸುತ್ತ
ಬೀಸು-ತ್ತಿದೆ
ಬೀಸು-ತ್ತಿ-ದ್ದಾನೆ
ಬೀಸು-ತ್ತಿ-ರು-ವುದೇ
ಬೀಸುವ
ಬೀಸು-ವು-ದಕ್ಕೆ
ಬೀಸು-ವು-ದ-ರಲ್ಲಿ
ಬುಗ್ಗೆ-ಯನ್ನೇ
ಬುಡ-ದಲ್ಲಿ
ಬುಡ-ಮೇಲು
ಬುಡ-ಸ-ಮೇತ
ಬುಡ-ಸ-ಹಿತ
ಬುದ್ಧ
ಬುದ್ಧ-ಗ-ಯೆಗೆ
ಬುದ್ಧ-ಗ-ಯೆಯ
ಬುದ್ಧ-ಗ-ಯೆ-ಯಲ್ಲಿ
ಬುದ್ಧ-ದೇ-ವ-ನಂ-ತೆಯೇ
ಬುದ್ಧನ
ಬುದ್ಧ-ನನ್ನು
ಬುದ್ಧ-ನಿಂದ
ಬುದ್ಧ-ನಿಗೆ
ಬುದ್ಧನು
ಬುದ್ಧನೂ
ಬುದ್ಧ-ಭ-ಗ-ವಂ-ತನ
ಬುದ್ಧಿ
ಬುದ್ಧಿ-ವೈ-ಚಾ-ರಿ-ಕ-ತೆ-ಗ-ಳೆಲ್ಲ
ಬುದ್ಧಿ-ಹೃ-ದ-ಯ-ಗಳ
ಬುದ್ಧಿ-ಹೃ-ದ-ಯ-ಗಳನ್ನು
ಬುದ್ಧಿ-ಗ-ಳಾ-ಚೆಗೆ
ಬುದ್ಧಿ-ಗ-ಳೆ-ಲ್ಲ-ವನ್ನೂ
ಬುದ್ಧಿ-ಗಿದ್ಧಿ
ಬುದ್ಧಿಗೂ
ಬುದ್ಧಿಗೆ
ಬುದ್ಧಿ-ಗ್ರಾ-ಹ್ಯ-ವಲ್ಲ
ಬುದ್ಧಿ-ಪೂ-ರ್ವ-ಕ-ವಾಗಿ
ಬುದ್ಧಿ-ಮತ್ತೆ
ಬುದ್ಧಿ-ಮ-ತ್ತೆಗೆ
ಬುದ್ಧಿ-ಮ-ತ್ತೆ-ಯಿಂ-ದಲೂ
ಬುದ್ಧಿಯ
ಬುದ್ಧಿ-ಯಂತೆ
ಬುದ್ಧಿ-ಯನ್ನು
ಬುದ್ಧಿ-ಯ-ನ್ನು-ಪ-ಯೋ-ಗಿಸಿ
ಬುದ್ಧಿ-ಯನ್ನೂ
ಬುದ್ಧಿ-ಯಲ್ಲಿ
ಬುದ್ಧಿ-ಯಿಂದ
ಬುದ್ಧಿ-ಯಿಂ-ದಲೂ
ಬುದ್ಧಿ-ಯಿತ್ತೋ
ಬುದ್ಧಿಯು
ಬುದ್ಧಿ-ವಂತ
ಬುದ್ಧಿ-ವಂ-ತ-ನಾದ
ಬುದ್ಧಿ-ವಂ-ತ-ರಾಗಿ
ಬುದ್ಧಿ-ವಂ-ತ-ರಾ-ದ-ವರು
ಬುದ್ಧಿ-ವಂ-ತ-ರು-ಮೂ-ರ್ಖರು
ಬುದ್ಧಿ-ವಂ-ತಿಕೆ
ಬುದ್ಧಿ-ವಂ-ತಿ-ಕೆಯ
ಬುದ್ಧಿ-ವಂ-ತಿ-ಕೆ-ಯಿಂದ
ಬುದ್ಧಿ-ವಂ-ತಿ-ಕೆಯೇ
ಬುದ್ಧಿ-ವಾದ
ಬುದ್ಧಿ-ವಾ-ದಕ್ಕೆ
ಬುದ್ಧಿ-ವಾ-ದ-ದಂತೆ
ಬುದ್ಧಿ-ಶಕ್ತಿ
ಬುದ್ಧಿ-ಶ-ಕ್ತಿ-ಇ-ಚ್ಛಾ-ಶ-ಕ್ತಿ-ಗ-ಳಿಗೆ
ಬುದ್ಧಿ-ಶ-ಕ್ತಿ-ಸಾ-ಮ-ರ್ಥ್ಯ-ಗಳನ್ನು
ಬುದ್ಧಿ-ಶ-ಕ್ತಿಗೆ
ಬುದ್ಧಿ-ಶ-ಕ್ತಿ-ಗೊಂದು
ಬುದ್ಧಿ-ಶ-ಕ್ತಿಯ
ಬುದ್ಧಿ-ಶ-ಕ್ತಿ-ಯನ್ನು
ಬುದ್ಧಿ-ಶ-ಕ್ತಿ-ಯನ್ನೂ
ಬುದ್ಧಿ-ಶ-ಕ್ತಿ-ಯಲ್ಲಿ
ಬುದ್ಧಿ-ಶ-ಕ್ತಿ-ಯಿಂದ
ಬುದ್ಧಿ-ಶ-ಕ್ತಿ-ಯಿದೆ
ಬುದ್ಧಿ-ಶ-ಕ್ತಿ-ಯೆಂ-ಬುದು
ಬುದ್ಧಿ-ಶ-ಕ್ತಿ-ಯೊ-ಡನೆ
ಬುದ್ಧಿ-ಶ-ಕ್ತಿ-ಸಂ-ಪ-ನ್ನ-ನಾ-ದರೂ
ಬುದ್ಧಿ-ಶಾಲಿ
ಬುದ್ಧಿ-ಶಾ-ಲಿಗೆ
ಬುದ್ಧಿ-ಸಾ-ಮ-ರ್ಥ್ಯದ
ಬುದ್ಧಿ-ಸಾ-ಮ-ರ್ಥ್ಯ-ವನ್ನು
ಬುದ್ಧಿ-ಸಾ-ಮ-ರ್ಥ್ಯ-ವನ್ನೂ
ಬುದ್ಧಿ-ಸ್ವಾ-ಸ್ಥ್ಯ-ವನ್ನು
ಬೂಟಾ-ಟಿಕೆ
ಬೂಟು-ಗಳನ್ನು
ಬೂಟು-ಗಳೂ
ಬೂದಿ
ಬೂದಿ-ಯನ್ನು
ಬೃಂದಾ-ವನ
ಬೃಂದಾ-ವ-ನಕ್ಕೆ
ಬೃಂದಾ-ವ-ನ-ಗಳ
ಬೃಂದಾ-ವ-ನದ
ಬೃಂದಾ-ವ-ನ-ದಲ್ಲೇ
ಬೃಂದಾ-ವ-ನ-ದಿಂದ
ಬೃಂದಾ-ವ-ನ-ಯಾತ್ರೆ
ಬೃಂದಾ-ವ-ನ-ಲೀ-ಲೆಯು
ಬೃಂದಾ-ವ-ನ-ವನ್ನು
ಬೃಹ
ಬೃಹತ್
ಬೃಹ-ತ್ಕಾರ್ಯ
ಬೃಹ-ತ್ಕಾ-ರ್ಯ-ಗಳನ್ನು
ಬೃಹ-ತ್ತಾಗಿ
ಬೃಹ-ತ್ತಾದ
ಬೃಹ-ದಾ-ಕಾ-ರದ
ಬೆಂಕಿ
ಬೆಂಕಿಗೆ
ಬೆಂಕಿಯ
ಬೆಂಕಿ-ಯನ್ನು
ಬೆಂಕಿ-ಯಲ್ಲಿ
ಬೆಂಗ-ಳೂ-ರಿನ
ಬೆಂಗಾ-ವ-ಲಾಗಿ
ಬೆಂಚಿನ
ಬೆಂಚಿ-ನಲ್ಲೇ
ಬೆಂಚಿ-ನಿಂದ
ಬೆಂಚು-ಗಳ
ಬೆಂಡಾದ
ಬೆಂದ
ಬೆಂದಿ-ಲ್ಲದ್ದೋ
ಬೆಂದು
ಬೆಂಬ-ಲಕ್ಕೆ
ಬೆಂಬ-ಲ-ವಾ-ಗಿ-ದ್ದರು
ಬೆಂಬ-ಲಿ-ಸು-ತ್ತಿದ್ದ
ಬೆಕ್ಕಸ
ಬೆಕ್ಕ-ಸ-ಬೆ-ರ-ಗಾಗಿ
ಬೆಕ್ಕ-ಸ-ಬೆ-ರ-ಗಾ-ಗು-ತ್ತಿ-ದ್ದರು
ಬೆಚ್ಚ-ಗಾ-ಯಿತು
ಬೆಚ್ಚಿ
ಬೆಚ್ಚಿತು
ಬೆಚ್ಚಿ-ಬಿದ್ದ
ಬೆಚ್ಚಿ-ಬಿದ್ದು
ಬೆಟ್ಟ
ಬೆಟ್ಟ-ಗಳು
ಬೆಟ್ಟ-ಗು-ಡ್ಡ-ಗಳ
ಬೆಟ್ಟದ
ಬೆಟ್ಟ-ವನ್ನೇ
ಬೆಡ-ಗಿಗೆ
ಬೆಣ್ಣೆ
ಬೆತ್ತ
ಬೆತ್ತ-ದಿಂದ
ಬೆತ್ತ-ದೇಟು
ಬೆದ-ರಿ-ಸದೆ
ಬೆನ್ತ್ಯಾಮ್
ಬೆನ್ನ-ಮೇಲೆ
ಬೆನ್ನ-ಲುಬು
ಬೆನ್ನಿಗೆ
ಬೆನ್ನು
ಬೆನ್ನು-ಮಾಡಿ
ಬೆನ್ನು-ಹ-ತ್ತದೆ
ಬೆನ್ನೇರಿ
ಬೆಪ್ಪಾಗಿ
ಬೆಪ್ಪು-ಗಟ್ಟಿ
ಬೆರ-ಗಾ-ಗದೆ
ಬೆರ-ಗಾಗಿ
ಬೆರ-ಗಾ-ಗಿದ್ದ
ಬೆರ-ಗಾಗು
ಬೆರ-ಗಾ-ಗು-ತ್ತಿ-ದ್ದರು
ಬೆರ-ಗಾ-ಗು-ವ-ವರೇ
ಬೆರ-ಗಾ-ದರು
ಬೆರ-ಗು-ಗ-ಣ್ಣಿ-ನಿಂದ
ಬೆರ-ಗು-ಗೊ-ಳಿ-ಸ-ಬೇ-ಕಾ-ದ-ವನು
ಬೆರಳ
ಬೆರಳು
ಬೆರ-ಳು-ಗಳನ್ನು
ಬೆರ-ಳು-ಗಳಲ್ಲಿ
ಬೆರ-ಳು-ಮಾಡಿ
ಬೆರೆ-ತಿದ್ದ
ಬೆರೆ-ತು-ಕೊಂ-ಡಿವೆ
ಬೆರೆ-ಯ-ದಿ-ರು-ವಂತೆ
ಬೆರೆ-ಯ-ಬಲ್ಲ
ಬೆರೆ-ಯು-ತ್ತಾರೆ
ಬೆರೆ-ಯು-ತ್ತಿದ್ದ
ಬೆರೆ-ಯು-ವು-ದುನ್ನು
ಬೆರೆಸಿ
ಬೆರೆ-ಸಿದ
ಬೆಲೆ
ಬೆಲೆ-ಕ-ಟ್ಟು-ವ-ವನು
ಬೆಲೆ-ಬಾ-ಳುವ
ಬೆಲೆಯೂ
ಬೆಳ-ಕನು
ಬೆಳ-ಕನ್ನು
ಬೆಳ-ಕನ್ನೆ
ಬೆಳ-ಕಿಗೆ
ಬೆಳ-ಕಿನ
ಬೆಳ-ಕಿ-ನಲ್ಲಿ
ಬೆಳ-ಕಿ-ನೆ-ಡೆಗೆ
ಬೆಳಕು
ಬೆಳ-ಗ-ಬಲ್ಲ
ಬೆಳ-ಗ-ಲಾ-ರಂ-ಭಿ-ಸು-ತ್ತಿತ್ತು
ಬೆಳ-ಗ-ಲಿ-ರುವ
ಬೆಳ-ಗಾ-ಗು-ತ್ತಲೇ
ಬೆಳ-ಗಾ-ಗೆದ್ದು
ಬೆಳ-ಗಾದ
ಬೆಳ-ಗಾ-ವಿಗೆ
ಬೆಳಗಿ
ಬೆಳ-ಗಿತು
ಬೆಳ-ಗಿ-ದಂ-ತಹ
ಬೆಳ-ಗಿನ
ಬೆಳಗು
ಬೆಳ-ಗು-ತ್ತ-ದೆಯೋ
ಬೆಳ-ಗು-ತ್ತಾನೆ
ಬೆಳ-ಗು-ತ್ತಿತ್ತು
ಬೆಳ-ಗು-ತ್ತಿದೆ
ಬೆಳ-ಗು-ತ್ತಿದ್ದ
ಬೆಳ-ಗು-ತ್ತಿ-ದ್ದರೂ
ಬೆಳ-ಗು-ತ್ತಿ-ದ್ದಾನೆ
ಬೆಳ-ಗು-ತ್ತಿ-ದ್ದಾರೆ
ಬೆಳ-ಗು-ತ್ತಿ-ದ್ದಾಳೆ
ಬೆಳ-ಗು-ತ್ತಿ-ರುವ
ಬೆಳ-ಗು-ತ್ತಿವೆ
ಬೆಳ-ಗು-ವಂತೆ
ಬೆಳ-ಗು-ವು-ದ-ರಲ್ಲಿ
ಬೆಳ-ಗ್ಗಿ-ನಿಂದ
ಬೆಳಗ್ಗೆ
ಬೆಳ-ದಿಂ-ಗ-ಳಿ-ನ-ಲ್ಲಿ-ಹೀಗೆ
ಬೆಳ-ವ-ಣಿ-ಗೆಗೂ
ಬೆಳ-ವ-ಣಿ-ಗೆಗೆ
ಬೆಳ-ವ-ಣಿ-ಗೆಯ
ಬೆಳ-ವ-ಣಿ-ಗೆ-ಯನ್ನು
ಬೆಳೆ
ಬೆಳೆದ
ಬೆಳೆ-ದಂ-ತೆಲ್ಲ
ಬೆಳೆ-ದರೂ
ಬೆಳೆ-ದ-ವರು
ಬೆಳೆ-ದಿತ್ತು
ಬೆಳೆ-ದಿ-ದ್ದೀರಿ
ಬೆಳೆ-ದಿ-ರು-ವ-ವರು
ಬೆಳೆದು
ಬೆಳೆ-ದು-ಕೊಂಡು
ಬೆಳೆ-ದು-ನಿಂತ
ಬೆಳೆ-ದು-ಬಂದ
ಬೆಳೆ-ದು-ಬಂ-ದದ್ದು
ಬೆಳೆ-ದು-ಬಂ-ದವು
ಬೆಳೆ-ದು-ಬಂ-ದಿತ್ತು
ಬೆಳೆ-ದು-ಬಂ-ದಿದ್ದ
ಬೆಳೆ-ದು-ಬಂ-ದಿ-ದ್ದರೆ
ಬೆಳೆ-ದು-ಬಂ-ದಿ-ರುವ
ಬೆಳೆ-ದು-ಬಂ-ದು-ಬಿ-ಟ್ಟಿದೆ
ಬೆಳೆ-ದು-ಬ-ರು-ತ್ತಿತ್ತು
ಬೆಳೆ-ದು-ಬಿ-ಟ್ಟಿತ್ತು
ಬೆಳೆಯ
ಬೆಳೆ-ಯ-ಬ-ಹುದು
ಬೆಳೆ-ಯಲು
ಬೆಳೆ-ಯಲೂ
ಬೆಳೆ-ಯಿತು
ಬೆಳೆಯು
ಬೆಳೆ-ಯುತ್ತ
ಬೆಳೆ-ಯು-ತ್ತದೆ
ಬೆಳೆ-ಯು-ತ್ತವೆ
ಬೆಳೆ-ಯು-ತ್ತಿತ್ತು
ಬೆಳೆ-ಯು-ತ್ತಿದೆ
ಬೆಳೆ-ಯುವ
ಬೆಳೆ-ಯು-ವು-ದಿ-ಲ್ಲಪ್ಪ
ಬೆಳೆ-ಯು-ವುದು
ಬೆಳೆ-ಸ-ಬೇ-ಕಾ-ಯಿತು
ಬೆಳೆ-ಸ-ಬೇಕು
ಬೆಳೆ-ಸಿ-ಕೊಂ-ಡರೆ
ಬೆಳೆ-ಸಿ-ಕೊಂಡು
ಬೆಳೆ-ಸಿ-ಕೊ-ಳ್ಳ-ಬ-ಹುದೆ
ಬೆಳೆ-ಸಿ-ಕೊ-ಳ್ಳಲು
ಬೆಳೆ-ಸಿ-ಕೊ-ಳ್ಳುವ
ಬೆಳೆ-ಸಿ-ಕೊ-ಳ್ಳು-ವಂತೆ
ಬೆಳೆ-ಸಿದ್ದ
ಬೆಳೆ-ಸು-ವುದು
ಬೆಳ್ಳಿಯ
ಬೆಳ್ಳು-ಳ್ಳಿ-ಯಿ-ಟ್ಟಿದ್ದ
ಬೆಳ್ವ-ಕ್ಕಿ-ಗಳು
ಬೆವತು
ಬೆಸುಗೆ
ಬೆಸೆದು
ಬೆಸೆ-ಯಿತು
ಬೇಕಲ್ಲ
ಬೇಕ-ಲ್ಲವೆ
ಬೇಕಾ-ಗ-ಬ-ಹು-ದಾದ
ಬೇಕಾ-ಗ-ಬ-ಹುದು
ಬೇಕಾಗಿ
ಬೇಕಾ-ಗಿತ್ತು
ಬೇಕಾ-ಗಿದೆ
ಬೇಕಾ-ಗಿ-ದ್ದಾರೆ
ಬೇಕಾ-ಗಿ-ದ್ದು-ದೊಂ-ದೇ-ನ-ರೇಂದ್ರ
ಬೇಕಾ-ಗಿ-ರು-ವು-ದ-ರಿಂದ
ಬೇಕಾ-ಗಿ-ರು-ವುದು
ಬೇಕಾ-ಗಿಲ್ಲ
ಬೇಕಾಗು
ಬೇಕಾ-ಗು-ತ್ತದೆ
ಬೇಕಾ-ಗು-ತ್ತವೆ
ಬೇಕಾ-ಗು-ತ್ತಾನೆ
ಬೇಕಾ-ಗು-ತ್ತಿತ್ತು
ಬೇಕಾ-ಗು-ವಷ್ಟು
ಬೇಕಾ-ಗು-ವು-ದಿಲ್ಲ
ಬೇಕಾದ
ಬೇಕಾ-ದಂತೆ
ಬೇಕಾ-ದದ್ದು
ಬೇಕಾ-ದರೂ
ಬೇಕಾ-ದರೆ
ಬೇಕಾ-ದ-ವರೇ
ಬೇಕಾ-ದ-ಷ್ಟನ್ನು
ಬೇಕಾ-ದ-ಷ್ಟಾ-ಯಿತು
ಬೇಕಾ-ದು-ದ-ಕ್ಕಿಂತ
ಬೇಕಾ-ದು-ದಾ-ದರೂ
ಬೇಕಾ-ದುದು
ಬೇಕಾ-ದುವು
ಬೇಕಾ-ದ್ದನ್ನು
ಬೇಕಾ-ದ್ದ-ರಿಂದ
ಬೇಕಾ-ಯಿತು
ಬೇಕಿ-ದ್ದರೆ
ಬೇಕಿಲ್ಲ
ಬೇಕು
ಬೇಕು-ಎಂದು
ಬೇಕು-ಬೇ-ಕೆಂ-ದು-ದನ್ನು
ಬೇಕು-ಬೇ-ಕೆಂದೇ
ಬೇಕೆ
ಬೇಕೆಂ-ದರೆ
ಬೇಕೆಂದು
ಬೇಕೆಂದೇ
ಬೇಕೆಂಬ
ಬೇಕೆಂ-ಬ-ವರೋ
ಬೇಕೆಂ-ಬಷ್ಟು
ಬೇಕೆಂ-ಬು-ದನ್ನು
ಬೇಕೆಂ-ಬುದು
ಬೇಕೆ-ನ್ನುವ
ಬೇಕೆ-ನ್ನು-ವು-ದಾ-ದರೆ
ಬೇಕೇ-ಬೇಕು
ಬೇಕೋ
ಬೇಗ
ಬೇಗನೆ
ಬೇಗೆ
ಬೇಗೆ-ಗ-ಳೆ-ಲ್ಲ-ವನ್ನೂ
ಬೇಟೆಗೆ
ಬೇಡ
ಬೇಡ-ಬಾ-ರದು
ಬೇಡ-ಬೇಡ
ಬೇಡ-ಬೇ-ಡ-ವೆಂ-ದರೂ
ಬೇಡ-ಲಾರೆ
ಬೇಡ-ವಾ-ಗಿದೆ
ಬೇಡವೆ
ಬೇಡಿ
ಬೇಡಿಕೊ
ಬೇಡಿ-ಕೊಂಡ
ಬೇಡಿ-ಕೊಂಡು
ಬೇಡಿ-ಕೊ-ಳ್ಳು-ತ್ತೇನೆ
ಬೇಡಿ-ತ-ರು-ತ್ತಿ-ದ್ದ-ವರು
ಬೇಡಿ-ದರು
ಬೇಡಿ-ದರೂ
ಬೇಡಿ-ದರೆ
ಬೇಡಿ-ದಾಗ
ಬೇಡಿ-ಯೇನು
ಬೇಡು-ತ್ತಿ-ದ್ದಾರೆ
ಬೇಡುವ
ಬೇಡು-ವು-ದಕ್ಕೂ
ಬೇಡು-ವುದು
ಬೇಡು-ವೆನು
ಬೇನೆ-ಯನ್ನು
ಬೇನೆ-ಯಿಂದ
ಬೇಯಿಸಿ
ಬೇಯಿ-ಸಿದ
ಬೇಯಿ-ಸಿದ್ದೋ
ಬೇರ-ಬೇರೆ
ಬೇರಾ-ರಾ-ದರೂ
ಬೇರಾವ
ಬೇರಿನ
ಬೇರಿ-ನಂತೆ
ಬೇರು
ಬೇರು-ಸ-ಹಿತ
ಬೇರೂರಿ
ಬೇರೂ-ರಿತು
ಬೇರೂ-ರಿದ್ದು
ಬೇರೂ-ರಿ-ರುವ
ಬೇರೆ
ಬೇರೆ-ಬೇರೆ
ಬೇರೆ-ಬೇ-ರೆ-ಯಲ್ಲ
ಬೇರೆ-ಬೇ-ರೆ-ಯಾಗಿ
ಬೇರೆ-ಯಲ್ಲ
ಬೇರೆ-ಯ-ಲ್ಲ-ವೆಂ-ಬು-ದನ್ನು
ಬೇರೆ-ಯ-ವರ
ಬೇರೆ-ಯಾಗಿ
ಬೇರೆ-ಯಾ-ಗಿತ್ತು
ಬೇರೆಯೆ
ಬೇರೆಯೇ
ಬೇರೆ-ಲ್ಲ-ವನ್ನೂ
ಬೇರೆ-ಲ್ಲಿಗೂ
ಬೇರೆ-ಲ್ಲಿಗೆ
ಬೇರೇ-ನನ್ನೂ
ಬೇರೇ-ನಾ-ದರೂ
ಬೇರೇನು
ಬೇರೇನೂ
ಬೇರೊಂ-ದಿ-ಲ್ಲದ
ಬೇರೊಂದು
ಬೇರ್ಪ-ಡೆ-ಯಾ-ಗು-ತ್ತಾ-ರೆ-ಒಂದು
ಬೇಲೂರು
ಬೇಳೆ
ಬೇಳೆ-ಕಾಳು-ಗ-ಳನ್ನೋ
ಬೇಳೆ-ಯಿಂದ
ಬೇವಿನ
ಬೇಸತ್ತು
ಬೇಸರ
ಬೇಸ-ರ-ದಿಂ-ದಿ-ರು-ವಂತೆ
ಬೇಸ-ರ-ವಾ-ಗಿದೆ
ಬೇಸ-ರ-ವಾ-ದಾ-ಗ-ಲೆಲ್ಲ
ಬೇಸ-ರ-ವಾ-ಯಿತು
ಬೇಸ-ರವೂ
ಬೇಸ-ರಿ-ಸದೆ
ಬೇಸ-ರಿ-ಸ-ಲಿಲ್ಲ
ಬೇಸ-ರಿ-ಸಿ-ಕೊ-ಳ್ಳ-ಬಾ-ರದು
ಬೇಸ-ರಿ-ಸಿ-ಕೊ-ಳ್ಳ-ಲಿಲ್ಲ
ಬೇಸ-ರಿ-ಸು-ತ್ತಿ-ರ-ಲಿಲ್ಲ
ಬೇಸಿ-ಗೆ-ಯಲ್ಲಿ
ಬೈಠಕ್
ಬೈಠ-ಕ್ಖಾ-ನೆಗೆ
ಬೈದರು
ಬೈದಿ-ದ್ದಾನೆ
ಬೈದು
ಬೈದು-ಕೊ-ಳ್ಳುತ್ತ
ಬೈದು-ಕೊ-ಳ್ಳು-ತ್ತಿ-ದ್ದ-ರುಛೆ
ಬೈಬ-ಲಿನ
ಬೈಬಲು
ಬೈಬಲ್
ಬೈಬ-ಲ್ಲಿ-ನ-ಲ್ಲಿ-ರುವ
ಬೈಯ-ಲಿಲ್ಲ
ಬೈಯಲು
ಬೈಯಲೂ
ಬೈಯ-ಲೂ-ಬಾ-ರದು
ಬೈಯು-ವುದು
ಬೈರಾಗಿ
ಬೈರಾ-ಗಿ-ಗಳ
ಬೈರಾ-ಗಿ-ಗ-ಳಂ-ತಲ್ಲ
ಬೈರಾ-ಗಿಗೆ
ಬೊಂಬೆ
ಬೊಂಬೆ-ಗಳನ್ನು
ಬೊಚ್ಚು-ಬಾಯಿ
ಬೋಧನೆ
ಬೋಧ-ನೆ
ಬೋಧ-ನೆ-ಗಳ
ಬೋಧ-ನೆ-ಗಳನ್ನು
ಬೋಧ-ನೆ-ಗಳನ್ನೂ
ಬೋಧ-ನೆ-ಗಳಲ್ಲಿ
ಬೋಧ-ನೆ-ಗ-ಳಲ್ಲೇ
ಬೋಧ-ನೆ-ಗಳು
ಬೋಧ-ನೆ-ಗಿಂತ
ಬೋಧ-ನೆಯ
ಬೋಧ-ನೆ-ಯ-ನ್ನಾ-ಗಲಿ
ಬೋಧ-ನೆ-ಯನ್ನು
ಬೋಧ-ನೆ-ಯನ್ನೇ
ಬೋಧಿ-ವೃಕ್ಷ
ಬೋಧಿ-ವೃ-ಕ್ಷದ
ಬೋಧಿ-ಸ-ಲಾಗಿದೆ
ಬೋಧಿ-ಸ-ಲಿ-ಲ್ಲ-ವೆಂದು
ಬೋಧಿ-ಸಲು
ಬೋಧಿ-ಸಲೇ
ಬೋಧಿ-ಸಿದ
ಬೋಧಿ-ಸಿ-ದರು
ಬೋಧಿ-ಸಿ-ದ-ರೆಂ-ಬು-ದರ
ಬೋಧಿ-ಸಿ-ದ-ವಳೇ
ಬೋಧಿ-ಸಿ-ದ್ದರೋ
ಬೋಧಿ-ಸಿದ್ದು
ಬೋಧಿ-ಸಿ-ದ್ದೇನು
ಬೋಧಿಸು
ಬೋಧಿ-ಸು-ತ್ತಿ-ದ್ದರು
ಬೋಧಿ-ಸು-ತ್ತಿ-ದ್ದಾರೆ
ಬೋಧಿ-ಸು-ತ್ತಿ-ರ-ಲಿಲ್ಲ
ಬೋಧಿ-ಸು-ತ್ತೇನೆ
ಬೋಧಿ-ಸುವ
ಬೋಧಿ-ಸು-ವ-ವರು
ಬೋಧಿ-ಸು-ವು-ದಾ-ದರೂ
ಬೋಧೆ-ಯಾ-ಗಿ-ದೆಯೋ
ಬೋಧೆ-ಯುಂ-ಟಾ-ಗ-ಬೇ-ಕಾ-ದರೆ
ಬೋರ-ಲಾಗಿ
ಬೋರ್ಡ್
ಬೋಸನ
ಬೋಸ-ನೆಂಬ
ಬೋಸ್
ಬೌದ್ಧ
ಬೌದ್ಧ-ತ-ತ್ತ್ವ-ಗಳನ್ನೂ
ಬೌದ್ಧ-ಧ-ರ್ಮದ
ಬೌದ್ಧಿಕ
ಬೌದ್ಧಿ-ಕ-ಆ-ಧ್ಯಾ-ತ್ಮಿಕ
ಬೌದ್ಧಿ-ಕತೆ
ಬೌದ್ಧಿ-ಕ-ತೆಗೂ
ಬೌದ್ಧಿ-ಕ-ತೆಗೆ
ಬೌದ್ಧಿ-ಕ-ವಾಗಿ
ಬ್ದಾರಿ
ಬ್ದಾರಿ-ಯೆಲ್ಲ
ಬ್ಬನ
ಬ್ಯಾನರ್ಜಿ
ಬ್ಯಾನ-ರ್ಜಿ-ಯ-ವರ
ಬ್ಯಾನ-ರ್ಜಿ-ಯ-ವ-ರನ್ನು
ಬ್ಯಾನ-ರ್ಜಿ-ಯ-ವರು
ಬ್ರಜೇ-ನನ
ಬ್ರಜೇ-ನ್ಬಾಬು
ಬ್ರಹ್ಮ
ಬ್ರಹ್ಮ-ಎಂ-ದರೆ
ಬ್ರಹ್ಮ-ಚರ್ಯ
ಬ್ರಹ್ಮ-ಚ-ರ್ಯ-ಗೃ-ಹ-ಸ್ಥ-ಸಂ-ನ್ಯಾಸಾ
ಬ್ರಹ್ಮ-ಚ-ರ್ಯದ
ಬ್ರಹ್ಮ-ಚ-ರ್ಯ-ದಿಂದ
ಬ್ರಹ್ಮ-ಚ-ರ್ಯ-ವನ್ನು
ಬ್ರಹ್ಮ-ಚಾರಿ
ಬ್ರಹ್ಮ-ಚಾ-ರಿ-ಯಾ-ಗಿದ್ದ
ಬ್ರಹ್ಮ-ಜ್ಞಾನ
ಬ್ರಹ್ಮ-ಜ್ಞಾ-ನ-ದ-ವರೆ-ಗಿನ
ಬ್ರಹ್ಮ-ಜ್ಞಾ-ನ-ದ-ವ-ರೆಗೆ
ಬ್ರಹ್ಮ-ಜ್ಞಾ-ನ-ವನ್ನು
ಬ್ರಹ್ಮ-ಜ್ಞಾನಿ
ಬ್ರಹ್ಮ-ಜ್ಞಾ-ನಿ-ಗಳು
ಬ್ರಹ್ಮದ
ಬ್ರಹ್ಮ-ದೈತ್ಯ
ಬ್ರಹ್ಮ-ನು-ಸಿರು
ಬ್ರಹ್ಮ-ಭಾವ
ಬ್ರಹ್ಮ-ಭಾ-ವ-ವನ್ನು
ಬ್ರಹ್ಮ-ಮ-ಯ-ವಾಗಿ
ಬ್ರಹ್ಮ-ರಾ-ಕ್ಷಸ
ಬ್ರಹ್ಮ-ವಂತೆ
ಬ್ರಹ್ಮ-ವನ್ನು
ಬ್ರಹ್ಮ-ವನ್ನೇ
ಬ್ರಹ್ಮ-ವಿಷ್ಣು
ಬ್ರಹ್ಮವೂ
ಬ್ರಹ್ಮವೇ
ಬ್ರಹ್ಮಾಂಡ
ಬ್ರಹ್ಮಾಂ-ಡ-ದಲ್ಲಿ
ಬ್ರಹ್ಮಾಂ-ಡ-ವನ್ನೇ
ಬ್ರಹ್ಮಾಂ-ಡವೇ
ಬ್ರಹ್ಮಾದಿ
ಬ್ರಹ್ಮಾ-ನಂದ
ಬ್ರಹ್ಮಾ-ನಂ-ದ-ರಿಗೆ
ಬ್ರಹ್ಮಾ-ನಂ-ದರು
ಬ್ರಹ್ಮಾ-ನಂ-ದವೂ
ಬ್ರಹ್ಮಾಸ್ಮಿ
ಬ್ರಹ್ಮೇತಿ
ಬ್ರಾಂಕೈ-ಟಿ-ಸಿಗೆ
ಬ್ರಾಂಕೈ-ಟಿಸ್
ಬ್ರಾಹ್ಮ
ಬ್ರಾಹ್ಮಣ
ಬ್ರಾಹ್ಮ-ಣ-ನಾ-ದ-ವನು
ಬ್ರಾಹ್ಮ-ಣ-ನೆಂದು
ಬ್ರಾಹ್ಮ-ಣ-ರಲ್ಲ
ಬ್ರಾಹ್ಮ-ಣ-ರ-ಲ್ಲ-ದಿ-ರ-ಬ-ಹುದು
ಬ್ರಾಹ್ಮ-ಣರು
ಬ್ರಾಹ್ಮ-ಣ-ರು-ಅ-ಸ್ಪೃ-ಶ್ಯ-ರು-ಎ-ಲ್ಲರೂ
ಬ್ರಾಹ್ಮ-ಣ-ವಂ-ಶ-ದಲ್ಲಿ
ಬ್ರಾಹ್ಮ-ಣ-ವ-ರ್ಗದ
ಬ್ರಾಹ್ಮಣಿ
ಬ್ರಾಹ್ಮ-ಣಿ-ಯನ್ನು
ಬ್ರಾಹ್ಮಣೋ
ಬ್ರಾಹ್ಮ-ಣ್ಯ-ಕ್ಕಿಂತ
ಬ್ರಾಹ್ಮ-ಣ್ಯವು
ಬ್ರಾಹ್ಮ-ಧು-ರೀ-ಣ-ನಾದ
ಬ್ರಾಹ್ಮ-ಸ-ದ-ಸ್ಯನ
ಬ್ರಾಹ್ಮ-ಸ-ದ-ಸ್ಯ-ರೆಲ್ಲ
ಬ್ರಾಹ್ಮ-ಸ-ಮಾಜ
ಬ್ರಾಹ್ಮ-ಸ-ಮಾ-ಜಕ್ಕೆ
ಬ್ರಾಹ್ಮ-ಸ-ಮಾ-ಜದ
ಬ್ರಾಹ್ಮ-ಸ-ಮಾ-ಜ-ದಲ್ಲಿ
ಬ್ರಾಹ್ಮ-ಸ-ಮಾ-ಜ-ದಲ್ಲೂ
ಬ್ರಾಹ್ಮ-ಸ-ಮಾ-ಜ-ದ-ವರು
ಬ್ರಾಹ್ಮ-ಸ-ಮಾ-ಜ-ದಿಂದ
ಬ್ರಾಹ್ಮ-ಸ-ಮಾ-ಜ-ವನ್ನು
ಬ್ರಾಹ್ಮ-ಸ-ಮಾ-ಜ-ವನ್ನೇ
ಬ್ರಾಹ್ಮ-ಸ-ಮಾ-ಜವು
ಬ್ರಾಹ್ಮ-ಸ-ಮಾ-ಜ-ವೆಂ-ಬುದು
ಬ್ರಾಹ್ಮ-ಸ-ಮಾ-ಜೀ-ಯ-ರಿಂದ
ಬ್ರಾಹ್ಮ-ಸ-ಮಾ-ಜೀ-ಯರು
ಬ್ರಿಟಾ-ನಿ-ಕದ
ಬ್ರಿಟಿ-ಷರ
ಬ್ರಿಟಿಷ್
ಭಂಗ
ಭಂಗ-ವಾಗಿ
ಭಂಗಿಯ
ಭಕ್ತ
ಭಕ್ತ-ಭ-ಗ-ವಂ-ತ-ರಲ್ಲಿ
ಭಕ್ತ-ಶಿ-ಷ್ಯ-ರಿಗೆ
ಭಕ್ತ-ಶಿ-ಷ್ಯ-ರೆ-ಲ್ಲರೂ
ಭಕ್ತ-ಗ-ಣ-ವನ್ನೂ
ಭಕ್ತ-ಜ-ನರ
ಭಕ್ತ-ಜ-ನ-ರಲ್ಲಿ
ಭಕ್ತನ
ಭಕ್ತ-ನ-ನ್ನಾ-ಗಿಯೂ
ಭಕ್ತ-ನಾ-ಗಲು
ಭಕ್ತ-ನಾದ
ಭಕ್ತ-ನಾ-ದ-ವನು
ಭಕ್ತ-ನಿಗೆ
ಭಕ್ತನೆ
ಭಕ್ತ-ನೆಂದು
ಭಕ್ತನೇ
ಭಕ್ತ-ನೊ-ಡನೆ
ಭಕ್ತರ
ಭಕ್ತ-ರದು
ಭಕ್ತ-ರ-ನೇ-ಕ-ರಲ್ಲಿ
ಭಕ್ತ-ರನ್ನು
ಭಕ್ತ-ರ-ನ್ನು-ಮು-ಖ್ಯ-ವಾಗಿ
ಭಕ್ತ-ರ-ನ್ನೆಲ್ಲ
ಭಕ್ತ-ರಲ್ಲಿ
ಭಕ್ತ-ರಾಗಿ
ಭಕ್ತ-ರಿಂ-ದಾ-ವೃ-ತ-ರಾಗಿ
ಭಕ್ತ-ರಿ-ಗಂತೂ
ಭಕ್ತ-ರಿಗೂ
ಭಕ್ತ-ರಿಗೆ
ಭಕ್ತ-ರಿ-ಗೆಲ್ಲ
ಭಕ್ತರು
ಭಕ್ತರೂ
ಭಕ್ತ-ರೆ-ಡೆಗೆ
ಭಕ್ತ-ರೆ-ದುರು
ಭಕ್ತ-ರೆಲ್ಲ
ಭಕ್ತ-ರೆ-ಲ್ಲರೂ
ಭಕ್ತರೇ
ಭಕ್ತ-ರೇನೋ
ಭಕ್ತ-ರೊಂ-ದಿ-ಗಿನ
ಭಕ್ತ-ರೊಂ-ದಿಗೆ
ಭಕ್ತ-ರೊ-ಡನೆ
ಭಕ್ತ-ರೊ-ಬ್ಬರು
ಭಕ್ತ-ವರ
ಭಕ್ತ-ವ-ರೇ-ಣ್ಯ-ನಾದ
ಭಕ್ತ-ವ-ರ್ಗಕ್ಕೆ
ಭಕ್ತ-ವೃಂದ
ಭಕ್ತ-ವೃಂ-ದದ
ಭಕ್ತ-ವೃಂ-ದ-ದಲ್ಲಿ
ಭಕ್ತ-ವೃಂ-ದ-ದ-ಲ್ಲೊಂದು
ಭಕ್ತ-ವೃಂ-ದ-ವೆಂ-ಬುದು
ಭಕ್ತ-ಶ್ರೇ-ಷ್ಠರು
ಭಕ್ತಾ-ದಿ-ಗಳಲ್ಲಿ
ಭಕ್ತಾ-ದಿ-ಗ-ಳಿ-ಗಾ-ಗಲಿ
ಭಕ್ತಾ-ದಿ-ಗಳು
ಭಕ್ತಾ-ದಿ-ಗ-ಳೆಲ್ಲ
ಭಕ್ತಿ
ಭಕ್ತಿ-ಕ-ರ್ಮ-ರಾ-ಜ-ಯೋ-ಗಿ-ಗ-ಳಿಗೂ
ಭಕ್ತಿ-ಗೌ-ರವ
ಭಕ್ತಿ-ಜ್ಞಾ-ನ-ಗಳ
ಭಕ್ತಿ-ಪ್ರೀತಿ
ಭಕ್ತಿ-ಪ್ರೀ-ತಿ-ಗೌ-ರ-ವ-ವ-ನ್ನಿಟ್ಟು
ಭಕ್ತಿ-ಪ್ರೇಮ
ಭಕ್ತಿ-ಪ್ರೇ-ಮ-ಗಳನ್ನು
ಭಕ್ತಿ-ಯೋ-ಗ-ಗಳಿಂದ
ಭಕ್ತಿ-ವಿ-ಶ್ವಾಸ
ಭಕ್ತಿ-ವಿ-ಶ್ವಾ-ಸ-ಗಳು
ಭಕ್ತಿ-ಶ್ರದ್ಧೆ
ಭಕ್ತಿ-ಶ್ರ-ದ್ಧೆ-ಗಳು
ಭಕ್ತಿ-ಗೀ-ತೆ-ಗಳು
ಭಕ್ತಿಗೆ
ಭಕ್ತಿ-ಗೊ-ಲಿ-ಯುವ
ಭಕ್ತಿ-ಗೌ-ರ-ವ-ಗಳಿಂದ
ಭಕ್ತಿ-ಜ್ಞಾ-ನ-ಗಳನ್ನು
ಭಕ್ತಿ-ಪರ
ಭಕ್ತಿ-ಪ್ರ-ಧಾನ
ಭಕ್ತಿ-ಪ್ರ-ಧಾ-ನ-ವಾದ
ಭಕ್ತಿ-ಭಾ-ವದ
ಭಕ್ತಿ-ಭಾ-ವ-ದಲ್ಲಿ
ಭಕ್ತಿ-ಭಾ-ವ-ಪೂ-ರ್ಣ-ವಾದ
ಭಕ್ತಿ-ಭಾ-ವ-ರಂ-ಜಿ-ತ-ವಾ-ಗಿ-ಬಿ-ಟ್ಟಿತು
ಭಕ್ತಿ-ಭಾ-ವ-ವನ್ನು
ಭಕ್ತಿ-ಭಾ-ವಾ-ವೇ-ಶ-ದಿಂದ
ಭಕ್ತಿ-ಮತಿ
ಭಕ್ತಿ-ಮಾ-ರ್ಗದ
ಭಕ್ತಿ-ಮಾ-ರ್ಗ-ದಲ್ಲಿ
ಭಕ್ತಿ-ಮಾ-ರ್ಗವು
ಭಕ್ತಿ-ಮಾ-ರ್ಗವೇ
ಭಕ್ತಿಯ
ಭಕ್ತಿ-ಯನ್ನು
ಭಕ್ತಿ-ಯಾ-ಗಲಿ
ಭಕ್ತಿ-ಯಿಂದ
ಭಕ್ತಿ-ಯಿ-ರು-ವಲ್ಲಿ
ಭಕ್ತಿ-ಯುತ
ಭಕ್ತಿಯೂ
ಭಕ್ತಿ-ಯೆಂ-ಬುದು
ಭಕ್ತಿ-ಯೊಂದು
ಭಕ್ತಿ-ಯೊ-ಚ್ಚ-ರ-ಗೊ-ಳಲು
ಭಕ್ತಿ-ರಿಗೆ
ಭಕ್ತಿ-ಶಾಸ್ತ್ರ
ಭಕ್ತಿ-ಸಂ-ಗೀತ
ಭಕ್ತಿ-ಸಾ-ಧನೆ
ಭಕ್ತಿ-ಸಾ-ಧ-ನೆಯ
ಭಕ್ತೆ-ಯರ
ಭಕ್ಷ್ಯ-ಗಳನ್ನು
ಭಕ್ಷ್ಯ-ವನ್ನು
ಭಗ
ಭಗವ
ಭಗ-ವಂತ
ಭಗ-ವಂ-ತನ
ಭಗ-ವಂ-ತ-ನಂ-ತೆಯೇ
ಭಗ-ವಂ-ತ-ನನ್ನು
ಭಗ-ವಂ-ತ-ನನ್ನೇ
ಭಗ-ವಂ-ತ-ನಲ್ಲಿ
ಭಗ-ವಂ-ತ-ನಿಂದ
ಭಗ-ವಂ-ತ-ನಿ-ಗಲ್ಲ
ಭಗ-ವಂ-ತ-ನಿ-ಗಾಗಿ
ಭಗ-ವಂ-ತ-ನಿಗೆ
ಭಗ-ವಂ-ತ-ನಿ-ರು-ವುದು
ಭಗ-ವಂ-ತನು
ಭಗ-ವಂ-ತನೂ
ಭಗ-ವಂ-ತ-ನೆಂ-ದರೆ
ಭಗ-ವಂ-ತ-ನೆ-ಡೆಗೆ
ಭಗ-ವಂ-ತನೇ
ಭಗ-ವಂ-ತ-ನೊಂ-ದಿ-ಗಿನ
ಭಗ-ವಂ-ತ-ನೊ-ಬ್ಬನೇ
ಭಗ-ವ-ತಿಯ
ಭಗ-ವ-ತ್ಕೃಪೆ
ಭಗ-ವ-ತ್ಪ್ರೇಮ
ಭಗ-ವ-ತ್ಪ್ರೇ-ಮದ
ಭಗ-ವ-ತ್ಶ-ಕ್ತಿ-ಯಿಂದ
ಭಗ-ವ-ತ್ಸಂ-ಬಂ-ಧ-ವಾದ
ಭಗ-ವ-ತ್ಸ-ದೃಶ
ಭಗ-ವ-ತ್ಸಾಕ್ಷಾ
ಭಗ-ವ-ತ್ಸಾ-ಕ್ಷಾ-ತ್ಕಾರ
ಭಗ-ವ-ತ್ಸಾ-ಕ್ಷಾ-ತ್ಕಾ-ರ-ಕ್ಕಾಗಿ
ಭಗ-ವ-ತ್ಸಾ-ಕ್ಷಾ-ತ್ಕಾ-ರಕ್ಕೂ
ಭಗ-ವ-ತ್ಸಾ-ಕ್ಷಾ-ತ್ಕಾ-ರಕ್ಕೆ
ಭಗ-ವ-ತ್ಸಾ-ಕ್ಷಾ-ತ್ಕಾ-ರದ
ಭಗ-ವ-ತ್ಸಾ-ಕ್ಷಾ-ತ್ಕಾ-ರ-ದಲ್ಲಿ
ಭಗ-ವ-ತ್ಸಾ-ಕ್ಷಾ-ತ್ಕಾ-ರ-ವನ್ನೇ
ಭಗ-ವ-ತ್ಸಾ-ಕ್ಷಾ-ತ್ಕಾ-ರ-ವೆಂಬ
ಭಗ-ವ-ತ್ಸಾ-ಕ್ಷಾ-ತ್ಕಾ-ರವೇ
ಭಗ-ವ-ತ್ಸಾ-ಕ್ಷಾ-ರಕ್ಕೆ
ಭಗ-ವ-ತ್ಸ್ವ-ರೂ-ಪಿ-ಗ-ಳಂತೆ
ಭಗ-ವ-ತ್ಸ್ವ-ರೂ-ಪಿ-ಗಳೇ
ಭಗ-ವ-ತ್ಸ್ವ-ರೂ-ಪಿ-ಯಾಗಿ
ಭಗ-ವ-ದಾ-ನಂ-ದ-ದಲ್ಲಿ
ಭಗ-ವ-ದಾ-ನಂ-ದ-ವನ್ನು
ಭಗ-ವ-ದಿ-ಚ್ಛೆಗೆ
ಭಗ-ವ-ದಿ-ಚ್ಛೆ-ಯಂತೆ
ಭಗ-ವ-ದ್ಗೀತೆ
ಭಗ-ವ-ದ್ಗೀ-ತೆ-ಯಲ್ಲಿ
ಭಗ-ವ-ದ್ದ-ರ್ಶನ
ಭಗ-ವ-ದ್ದ-ರ್ಶ-ನ-ಕ್ಕಾಗಿ
ಭಗ-ವ-ದ್ದ-ರ್ಶ-ನಕ್ಕೆ
ಭಗ-ವ-ದ್ದ-ರ್ಶ-ನದ
ಭಗ-ವ-ದ್ಧ್ಯಾನ
ಭಗ-ವ-ದ್ಭ-ಕ್ತಿಯ
ಭಗ-ವ-ದ್ಭಾವ
ಭಗ-ವ-ದ್ಭಾ-ವ-ದಲ್ಲಿ
ಭಗ-ವ-ದ್ವಿ-ಚಾ-ರ-ದಲ್ಲಿ
ಭಗ-ವ-ದ್ವ್ಯಾ-ಕು-ಲತೆ
ಭಗ-ವನ್
ಭಗ-ವ-ನ್ನಾಮ
ಭಗ-ವ-ನ್ನಾ-ಮ-ವನ್ನು
ಭಗ-ವ-ನ್ನಾ-ಮ-ಸ್ಮ-ರಣೆ
ಭಗ-ವ-ನ್ಮಯ
ಭಗ-ವಾನೇ
ಭಗ-ವಾನ್
ಭಗೀ-ರಥ
ಭಜ
ಭಜನ
ಭಜನೆ
ಭಜ-ನೆ-ಪ್ರಾ-ರ್ಥ-ನೆ-ಧ್ಯಾನ
ಭಜ-ನೆ-ಪ್ರಾ-ರ್ಥ-ನೆ-ಧ್ಯಾ-ನ-ಗಳನ್ನು
ಭಜ-ನೆ-ಪ್ರಾ-ರ್ಥ-ನೆ-ಧ್ಯಾ-ನಾ-ದಿ-ಗಳನ್ನು
ಭಜ-ನೆ-ವೇದ
ಭಜ-ನೆ-ಗ-ಳ-ಷ್ಟ-ರಿಂ-ದಲೇ
ಭಜ-ನೆ-ಮಾ-ಡು-ತ್ತಿ-ದ್ದರು
ಭಜ-ನೆಯ
ಭಜ-ನೆ-ಯೇನೂ
ಭಟರು
ಭಟ್ಟಾ-ಚಾರ್ಯ
ಭಟ್ಟಾ-ಚಾ-ರ್ಯರ
ಭಟ್ಟಾ-ಚಾ-ರ್ಯರು
ಭಟ್ಟಾ-ಚಾ-ರ್ಯರೂ
ಭದ್ರ-ಪ-ಡಿ-ಸಿ-ದಂತೆ
ಭನೆಗೂ
ಭಯ
ಭಯ
ಭಯಂ-ಕರ
ಭಯಂ-ಕ-ರ-ವಾ-ಗಿತ್ತು
ಭಯಂ-ಕ-ರ-ವಾದ
ಭಯಂ-ಕ-ರ-ವಾದ್ದು
ಭಯ-ಗೊಂ-ಡಿ-ದ್ದರು
ಭಯ-ಗೌ-ರ-ವ-ಗಳಿಂದ
ಭಯ-ದಿಂದ
ಭಯ-ದಿಂ-ದಲೋ
ಭಯ-ದೊ-ಳ-ಗಿ-ರು-ವೆವು
ಭಯ-ಪ-ಟ್ಟು-ಕೊಂಡು
ಭಯ-ಭೀ-ತ-ನಾಗಿ
ಭಯ-ವಾ-ಗ-ದಿ-ರು-ತ್ತ-ದೆಯೆ
ಭಯ-ವಾಗಿ
ಭಯ-ವಾ-ಗು-ತ್ತ-ದೆಯೇ
ಭಯ-ವಾ-ಗು-ತ್ತಿತ್ತು
ಭಯ-ವಾ-ಗು-ತ್ತಿದೆ
ಭಯ-ವಿಲ್ಲ
ಭಯ-ವಿ-ಹ್ವ-ಲ-ರಾ-ದರು
ಭಯ-ವುಂ-ಟಾ-ಯಿತು
ಭಯ-ವೆಲ್ಲಿ
ಭಯ-ವೆ-ಲ್ಲಿ-ಯದು
ಭಯವೋ
ಭಯಾ-ಕ್ರಾಂ-ತ-ರಾಗಿ
ಭರ-ತ-ಇ-ಳಿ-ತ-ಗ-ಳಿ-ಲ್ಲದ
ಭರ-ತ-ಖಂ-ಡಕ್ಕೆ
ಭರ-ತನ
ಭರ-ದಲ್ಲಿ
ಭರ-ದಿಂದ
ಭರ-ವಸೆ
ಭರ-ವ-ಸೆ-ಯನ್ನೂ
ಭರ-ವ-ಸೆ-ಯಿಂ-ದಾಗಿ
ಭರ-ವ-ಸೆ-ಯುಂ-ಟಾ-ಗಿದೆ
ಭರಾ-ಟೆ-ಯನ್ನು
ಭರಿ-ತ-ನಾಗಿ
ಭರಿ-ತ-ನಾದ
ಭರಿ-ತ-ರಾ-ಗಿ-ದ್ದಾರೆ
ಭರಿ-ತ-ರಾ-ಗು-ತ್ತಿ-ದ್ದರು
ಭರಿ-ಸ-ಲಾ-ರದ
ಭರಿಸಿ
ಭರ್ಜರಿ
ಭಲೇ
ಭವ
ಭವ-ತಾ-ರಿಣಿ
ಭವ-ತಾ-ರಿಣೀ
ಭವದ
ಭವ-ನದ
ಭವ-ನಾಥ
ಭವ-ನಾ-ಥ-ನದ್ದು
ಭವ-ರೋ-ಗವು
ಭವ-ವನ್ನು
ಭವವೇ
ಭವ-ಸಾ-ಗ-ರ-ವನ್ನು
ಭವಿಷ್ಯ
ಭವಿ-ಷ್ಯ-ಜೀ-ವ-ನದ
ಭವಿ-ಷ್ಯ-ತ್ತಿನ
ಭವಿ-ಷ್ಯದ
ಭವಿ-ಷ್ಯ-ವ-ನ್ನೆಲ್ಲ
ಭವಿ-ಷ್ಯ-ವಾಣಿ
ಭವಿ-ಷ್ಯ-ವಾ-ಣಿ-ಯನ್ನು
ಭವಿ-ಷ್ಯ-ವಿದೆ
ಭವಿ-ಷ್ಯ-ವೆಲ್ಲ
ಭವಿ-ಸಿ-ದುವು
ಭವಿ-ಸಿ-ರು-ವಂತೆ
ಭವ್ಯ
ಭವ್ಯ-ಚಿ-ತ್ರ-ವನ್ನು
ಭವ್ಯ-ತೆ-ಯನ್ನು
ಭವ್ಯ-ನಾ-ದ-ದಿಂದ
ಭವ್ಯ-ವಾ-ಗಿದೆ
ಭವ್ಯ-ವಾದ
ಭಸ್ಮ
ಭಸ್ಮ-ಮಾ-ಡಿ-ಬಿ-ಡ-ಬ-ಲ್ಲದು
ಭಸ್ಮ-ವಾ-ಗ-ಲಿದೆ
ಭಾಗ
ಭಾಗ-ಲ್ಪುರ
ಭಾಗ-ಲ್ಪು-ರಕ್ಕೆ
ಭಾಗ-ಲ್ಪು-ರ-ದಲ್ಲಿ
ಭಾಗ-ಲ್ಪು-ರ-ದ-ಲ್ಲಿದ್ದ
ಭಾಗ-ಲ್ಪು-ರ-ದಿಂದ
ಭಾಗ-ವ-ತದ
ಭಾಗ-ವ-ನ್ನಾ-ದರೂ
ಭಾಗ-ವನ್ನು
ಭಾಗ-ವ-ಹಿ-ಸ-ಲಾ-ರಂ-ಭಿ-ಸಿದ
ಭಾಗ-ವ-ಹಿ-ಸಲು
ಭಾಗ-ವ-ಹಿ-ಸಿ-ದರು
ಭಾಗ-ವ-ಹಿ-ಸುತ್ತ
ಭಾಗ-ವ-ಹಿ-ಸು-ತ್ತಿದ್ದ
ಭಾಗವೇ
ಭಾಗಶಃ
ಭಾಗಿ-ಗ-ಳಾ-ಗು-ತ್ತೇವೆ
ಭಾಗಿ-ಯಾ-ಗು-ತ್ತಿತ್ತು
ಭಾಗ್ಯ-ವಂ-ತರು
ಭಾಗ್ಯ-ವಾಗಿ
ಭಾಗ್ಯ-ವಿ-ಶೇ-ಷ-ದಿಂದ
ಭಾಗ್ಯವೂ
ಭಾಗ್ಯ-ಶಾ-ಲಿ-ಗಳಲ್ಲಿ
ಭಾನು-ವಾರ
ಭಾಯಿ-ಗಳೂ
ಭಾಯ್
ಭಾರತ
ಭಾರ-ತಕ್ಕೆ
ಭಾರ-ತದ
ಭಾರ-ತ-ದ-ಲ್ಲಷ್ಟೇ
ಭಾರ-ತ-ದಲ್ಲಿ
ಭಾರ-ತ-ದಾ-ದ್ಯಂತ
ಭಾರ-ತ-ರಾ-ಷ್ಟ್ರ-ವನ್ನು
ಭಾರ-ತ-ವನ್ನು
ಭಾರ-ತ-ವಿಡೀ
ಭಾರ-ತ-ವೇ-ನಾ-ದರೂ
ಭಾರ-ತೀಯ
ಭಾರ-ತೀ-ಯ-ತೆಯ
ಭಾರ-ತೀ-ಯ-ರನ್ನು
ಭಾರ-ತೀ-ಯ-ರಾದ
ಭಾರ-ತೀ-ಯರು
ಭಾರ-ತೀ-ಯ-ರೆಲ್ಲ
ಭಾರದ
ಭಾರ-ದಿಂದ
ಭಾರ-ವನ್ನೂ
ಭಾರ-ವ-ನ್ನೆಲ್ಲ
ಭಾರ-ವಾಗಿ
ಭಾರ-ವಾ-ಗು-ವಿಕೆ
ಭಾರ-ವಾದ
ಭಾರೀ
ಭಾವ
ಭಾವಕ್ಕೆ
ಭಾವ-ಗಳ
ಭಾವ-ಗಳನ್ನೂ
ಭಾವ-ಗಳಲ್ಲಿ
ಭಾವ-ಗ-ಳಿಗೆ
ಭಾವ-ಗಳು
ಭಾವ-ಗ್ರ-ಹಣ
ಭಾವ-ಚಿತ್ರ
ಭಾವ-ಚಿ-ತ್ರ-ವ-ನ್ನಿ-ರಿಸಿ
ಭಾವ-ಜ-ಗ-ತ್ತನ್ನು
ಭಾವ-ಜೀ-ವನ
ಭಾವ-ಜೀವಿ
ಭಾವ-ಜೀ-ವಿ-ಗಳ
ಭಾವ-ತ-ರಂ-ಗ-ವನ್ನೇ
ಭಾವದ
ಭಾವ-ದಂ-ತ-ರಂ-ಗ-ವನ್ನು
ಭಾವ-ದ-ಲೆ-ಗಳು
ಭಾವ-ದಲ್ಲಿ
ಭಾವ-ದ-ಲ್ಲಿ-ದ್ದಾಗ
ಭಾವ-ದ-ಲ್ಲಿ-ರು-ವಾಗ
ಭಾವ-ದಲ್ಲೇ
ಭಾವ-ದಿಂದ
ಭಾವ-ದುಂಬಿ
ಭಾವ-ದುಂ-ಬಿ-ಬಂ-ದಾಗ
ಭಾವ-ದು-ಬ್ಬ-ರ-ವನ್ನು
ಭಾವ-ದೆ-ತ್ತ-ರಕ್ಕೆ
ಭಾವ-ದೊಂ-ದಿಗೆ
ಭಾವ-ನಾ-ತ್ಮಕ
ಭಾವ-ನಾ-ತ್ಮ-ಕ-ತೆಗೆ
ಭಾವನೆ
ಭಾವ-ನೆ-ಮಾ-ತು
ಭಾವ-ನೆ-ಗಳ
ಭಾವ-ನೆ-ಗಳನ್ನು
ಭಾವ-ನೆ-ಗಳನ್ನೆಲ್ಲ
ಭಾವ-ನೆ-ಗಳಲ್ಲಿ
ಭಾವ-ನೆ-ಗ-ಳ-ಲ್ಲೆಲ್ಲ
ಭಾವ-ನೆ-ಗಳಿಂದ
ಭಾವ-ನೆ-ಗ-ಳಿಗೆ
ಭಾವ-ನೆ-ಗಳು
ಭಾವ-ನೆ-ಗಳೂ
ಭಾವ-ನೆ-ಗ-ಳೆಲ್ಲ
ಭಾವ-ನೆಯ
ಭಾವ-ನೆ-ಯನ್ನು
ಭಾವ-ನೆ-ಯಿಂದ
ಭಾವ-ನೆಯೂ
ಭಾವ-ನೆಯೇ
ಭಾವ-ನೆ-ಯೇ-ನೆಂ-ದರೆ
ಭಾವ-ಪ-ರ-ವ-ಶ-ತೆ-ಇ-ವು-ಗಳನ್ನೆಲ್ಲ
ಭಾವ-ಪ-ರ-ವ-ಶ-ನಾ-ಗು-ತಿದ್ದ
ಭಾವ-ಪ-ರ-ವ-ಶ-ನಾ-ದಂತೆ
ಭಾವ-ಪ-ರ-ವ-ಶ-ರಾ-ಗಿ-ಬಿ-ಡುತ್ತಿ
ಭಾವ-ಪ-ರ-ವ-ಶ-ರಾ-ಗು-ತ್ತಿ-ದ್ದರು
ಭಾವ-ಪ-ರ-ವ-ಶ-ರಾ-ದಂತೆ
ಭಾವ-ಪ-ರ-ವ-ಶ-ವಾ-ಗಿ-ಬಿ-ಡು-ತ್ತಿತ್ತು
ಭಾವ-ಪೂ-ರ್ಣ-ವಾಗಿ
ಭಾವ-ಪೂ-ರ್ಣ-ವಾದ
ಭಾವ-ಪ್ರ-ಕಾ-ಶ-ವನ್ನು
ಭಾವ-ಪ್ರ-ಚೋ-ದ-ನೆ-ಯಾ-ದರೂ
ಭಾವ-ಭ-ರಿತ
ಭಾವ-ಭ-ರಿ-ತ-ನಾಗಿ
ಭಾವ-ಭ-ರಿ-ತ-ವಾದ
ಭಾವ-ಮುಖ
ಭಾವ-ಮು-ಖ-ವೆಂದರೆ
ಭಾವ-ರಂ-ಜಿ-ತ-ವಾ-ಗಿದೆ
ಭಾವ-ರಾ-ಜ್ಯ-ದಲ್ಲಿ
ಭಾವ-ಲ-ಹ-ರಿ-ಯನ್ನು
ಭಾವ-ಲ-ಹ-ರಿಯು
ಭಾವ-ವನ್ನು
ಭಾವ-ವ-ನ್ನೆ-ಬ್ಬಿ-ಸುವ
ಭಾವ-ವಿ-ದೆಯೇ
ಭಾವ-ವುಕ್ಕಿ
ಭಾವ-ವೆಂದರೆ
ಭಾವವೇ
ಭಾವ-ಸ-ಮಾ-ಧಿ-ಗೇ-ರ-ದಿ-ರಲು
ಭಾವ-ಸ-ಮಾ-ಧಿ-ಗೇ-ರಿ-ದರು
ಭಾವ-ಸ-ಮಾ-ಧಿ-ಗೇ-ರುವ
ಭಾವ-ಸ-ಮಾ-ಧಿಯ
ಭಾವ-ಸ-ಮಾ-ಧಿ-ಯಂ-ತಹ
ಭಾವ-ಸ-ಮಾ-ಧಿ-ಸ್ಥ-ರಾಗಿ
ಭಾವ-ಸೂಕ್ಷ್ಮ
ಭಾವ-ಸ್ಥ-ರಾ-ದರು
ಭಾವ-ಸ್ಥಿ-ತಿ-ಗೇ-ರ-ದಂತೆ
ಭಾವ-ಸ್ಥಿ-ತಿ-ಯಲ್ಲಿ
ಭಾವ-ಸ್ಥಿ-ತಿ-ಯ-ಲ್ಲಿ-ದ್ದರೆ
ಭಾವ-ಸ್ಥಿ-ತಿ-ಯ-ಲ್ಲಿ-ರ-ಬೇ-ಕೆಂದು
ಭಾವ-ಸ್ವಾ-ತಂ-ತ್ರ್ಯ-ದಲ್ಲಿ
ಭಾವ-ಹೀ-ನ-ರಾದ
ಭಾವಾರ್ಥ
ಭಾವಾ-ರ್ಥ-ವೇನೆಂದರೆ
ಭಾವಾ-ವ-ಸ್ಥೆ-ಯಲ್ಲಿ
ಭಾವಾ-ವ-ಸ್ಥೆ-ಯಿಂದ
ಭಾವಾ-ವ-ಸ್ಥೆ-ಯುಂ-ಟಾ-ಗು-ವು-ದಿ-ಲ್ಲ-ವಲ್ಲ
ಭಾವಾ-ವಿ-ಷ್ಟ-ರಾ-ದರು
ಭಾವಾ-ವೇಗ
ಭಾವಾ-ವೇ-ಗಕ್ಕೆ
ಭಾವಾ-ವೇ-ಗದ
ಭಾವಾ-ವೇ-ಗ-ದಿಂದ
ಭಾವಾ-ವೇ-ಗ-ವನ್ನು
ಭಾವಾ-ವೇ-ಗ-ವನ್ನೇ
ಭಾವಾ-ವೇಶ
ಭಾವಾ-ವೇ-ಶದ
ಭಾವಾ-ವೇ-ಶ-ದಂ-ತಲ್ಲ
ಭಾವಾ-ವೇ-ಶ-ದಲ್ಲೇ
ಭಾವಾ-ವೇ-ಶ-ದಿಂದ
ಭಾವಾ-ವೇ-ಶ-ಭ-ರಿ-ತ-ನಾಗಿ
ಭಾವಾ-ವೇ-ಶ-ಭ-ರಿ-ತ-ರಾಗಿ
ಭಾವಾ-ವೇ-ಶ-ವುಂ-ಟಾ-ದು-ದನ್ನು
ಭಾವಾ-ವೇ-ಶ-ವೆಂ-ಬುದು
ಭಾವಾ-ವೇ-ಶ-ವೆಲ್ಲ
ಭಾವಿ
ಭಾವಿ-ಸ-ಬೇಕಾ
ಭಾವಿ-ಸ-ಬೇಕು
ಭಾವಿ-ಸಲಿ
ಭಾವಿಸಿ
ಭಾವಿ-ಸಿ-ಕೊಂ-ಡರು
ಭಾವಿ-ಸಿ-ಕೊಂಡು
ಭಾವಿ-ಸಿ-ಕೊ-ಳ್ಳು-ತ್ತಿದ್ದ
ಭಾವಿ-ಸಿದ
ಭಾವಿ-ಸಿ-ದ-ಇ-ವರ
ಭಾವಿ-ಸಿ-ದರು
ಭಾವಿ-ಸಿ-ದ-ರು-ಇ-ವನು
ಭಾವಿ-ಸಿ-ದರೆ
ಭಾವಿ-ಸಿ-ದರೇ
ಭಾವಿ-ಸಿ-ದಾಗ
ಭಾವಿ-ಸಿ-ದಿ-ರೇನು
ಭಾವಿ-ಸಿದ್ದ
ಭಾವಿ-ಸಿ-ದ್ದ-ರಾತ್ರಿ
ಭಾವಿ-ಸಿ-ದ್ದರು
ಭಾವಿ-ಸಿ-ಯಾರು
ಭಾವಿ-ಸಿ-ಯಾಳು
ಭಾವಿ-ಸಿಯೇ
ಭಾವಿ-ಸಿ-ರ-ಬೇಕು
ಭಾವಿ-ಸಿ-ರ-ಲಿಲ್ಲ
ಭಾವಿ-ಸಿ-ರು-ವಂ-ತಿದೆ
ಭಾವಿಸು
ಭಾವಿ-ಸುತ್ತ
ಭಾವಿ-ಸು-ತ್ತಾರೆ
ಭಾವಿ-ಸು-ತ್ತಿದ್ದ
ಭಾವಿ-ಸು-ತ್ತಿ-ದ್ದರು
ಭಾವಿ-ಸು-ತ್ತಿ-ದ್ದಾರೆ
ಭಾವಿ-ಸು-ತ್ತೇನೆ
ಭಾವಿ-ಸು-ತ್ತೇವೆ
ಭಾವಿ-ಸುವ
ಭಾವಿ-ಸು-ವು-ದಾ-ದರೆ
ಭಾವಿ-ಸೋಣ
ಭಾವೀ
ಭಾವೀ-ಶಿ-ಷ್ಯರ
ಭಾವುಕ
ಭಾವು-ಕ-ತೆ-ಯನ್ನು
ಭಾವು-ಕರ
ಭಾವೈ-ಕ್ಯಕ್ಕೆ
ಭಾವೋ-ತ್ಕ-ಟ-ತೆಯ
ಭಾವೋ-ತ್ಕ-ರ್ಷಕ್ಕೂ
ಭಾವೋ-ತ್ಸಾಹ
ಭಾವೋ-ದ್ರೇ-ಕದ
ಭಾವೋ-ದ್ರೇ-ಕ-ವನ್ನು
ಭಾವೋ-ದ್ರೇ-ಕ-ವೆಲ್ಲ
ಭಾವೋ-ದ್ವೇಗ
ಭಾವೋ-ನ್ಮ-ತ್ತತೆ
ಭಾವೋ-ನ್ಮ-ತ್ತ-ತೆ-ಯನ್ನು
ಭಾವೋ-ನ್ಮ-ತ್ತ-ನಾ-ಗಿ-ರ-ಬೇಕು
ಭಾವೋ-ನ್ಮ-ತ್ತ-ರಾಗಿ
ಭಾಷಣ
ಭಾಷ-ಣ-ಕಾ-ರ-ನಾದ
ಭಾಷ-ಣ-ವನ್ನು
ಭಾಷ-ಣ-ವನ್ನೇ
ಭಾಷಾ-ಶೈಲಿ
ಭಾಷೆ
ಭಾಷೆ-ಗಳ
ಭಾಷೆ-ಗಳನ್ನು
ಭಾಷೆ-ಗಳಲ್ಲಿ
ಭಾಷೆ-ಗ-ಳು-ಇ-ವೆ-ಲ್ಲ-ದರ
ಭಾಷೆಯ
ಭಾಷೆ-ಯನ್ನು
ಭಾಷೆ-ಯಲ್ಲಿ
ಭಾಷೆ-ಯಲ್ಲೇ
ಭಾಷೆ-ಯಿಂದ
ಭಾಷೆ-ಯೆಂ-ದರೆ
ಭಾಷೆಯೇ
ಭಾಸ
ಭಾಸ-ವಾ-ಗು-ತ್ತದೆ
ಭಾಸ-ವಾ-ಗು-ತ್ತಿ-ತ್ತಲ್ಲ
ಭಾಸ-ವಾ-ಗು-ತ್ತಿತ್ತು
ಭಾಸ-ವಾ-ಗು-ತ್ತಿದೆ
ಭಾಸ-ವಾ-ಯಿತು
ಭಾಸ್ಕ-ರ
ಭಾಸ್ಕರಾ
ಭಾಸ್ಕ-ರಾ-ನಂ-ದ-ರನ್ನು
ಭಾಸ್ಕ-ರಾ-ನಂ-ದರು
ಭಾಸ್ಕ-ರಾ-ನಂ-ದರೂ
ಭಿಕಾರಿ
ಭಿಕಾ-ರಿ-ಯಾ-ಗಿ-ಬಿಟ್ಟೆ
ಭಿಕ್ಷಾ
ಭಿಕ್ಷಾ-ನ್ನ-ವನ್ನು
ಭಿಕ್ಷಾ-ನ್ನವೂ
ಭಿಕ್ಷಾ-ಪಾತ್ರೆ
ಭಿಕ್ಷಾ-ಪಾ-ತ್ರೆ-ಗಳನ್ನು
ಭಿಕ್ಷಾ-ಪಾ-ತ್ರೆ-ಯನ್ನು
ಭಿಕ್ಷುಕ
ಭಿಕ್ಷು-ಕ-ನಂತೆ
ಭಿಕ್ಷು-ಕ-ರನ್ನು
ಭಿಕ್ಷು-ಕ-ರು-ಇ-ವ-ರನ್ನು
ಭಿಕ್ಷೆ
ಭಿಕ್ಷೆ-ಗಾಗಿ
ಭಿಕ್ಷೆಗೆ
ಭಿಕ್ಷೆ-ಬೇಡಿ
ಭಿಕ್ಷೆಯ
ಭಿಕ್ಷೆ-ಯನ್ನು
ಭಿಕ್ಷೆ-ಯನ್ನೇ
ಭಿಕ್ಷೆ-ಯಿಂದ
ಭಿಕ್ಷೆಯೂ
ಭಿನ್ನ
ಭಿನ್ನ-ವಾ-ಗಿ-ರು-ತ್ತಿತ್ತು
ಭಿನ್ನಾ-ಭಿ-ಪ್ರಾಯ
ಭಿನ್ನಾ-ಭಿ-ಪ್ರಾ-ಯ-ಗಳ
ಭಿನ್ನಾ-ಭಿ-ಪ್ರಾ-ಯದ
ಭಿನ್ನಾ-ಭಿ-ಪ್ರಾ-ಯ-ವಾ-ದರೆ
ಭಿನ್ನಾ-ಭಿ-ಪ್ರಾ-ಯ-ವುಂ-ಟಾಗಿ
ಭಿನ್ನಾ-ಭಿ-ಪ್ರಾ-ಯವೇ
ಭೀತಿ-ಯಿಂದ
ಭುಕ್ತಿ
ಭುಕ್ತಿ-ಮು-ಕ್ತಿ-ಗಳನ್ನು
ಭುಗಿ-ಲೆ-ದ್ದಿತು
ಭುಗಿ-ಲೆ-ದ್ದಿದ್ದ
ಭುಗಿ-ಲೆ-ದ್ದು-ಬಿ-ಟ್ಟಿತು
ಭುಗಿ-ಲೆ-ದ್ದು-ಬಿ-ಟ್ಟಿದೆ
ಭುಗಿ-ಲ್ಲೆ-ದ್ದಿತು
ಭುಜ-ಗಳು
ಭುಜದ
ಭುಜಿ-ಸಿ-ದರು
ಭುವ-ನಂ-ಗ-ಳನೆ
ಭುವ-ನ-ಮೋ-ಹಿನೀ
ಭುವ-ನ-ವನ್ನೇ
ಭುವ-ನೇ-ಶ್ವರಿ
ಭುವ-ನೇ-ಶ್ವ-ರಿಗೆ
ಭುವ-ನೇ-ಶ್ವ-ರಿಯ
ಭುವ-ನೇ-ಶ್ವ-ರಿಯೂ
ಭುವ-ನೇ-ಶ್ವರೀ
ಭುವಿ-ಯನ್ನು
ಭುವಿ-ಯಲ್ಲಿ
ಭೂಗತ
ಭೂಗೋ-ಳದ
ಭೂಗೋ-ಳ-ದಲ್ಲಿ
ಭೂತ
ಭೂತ
ಭೂತ-ಭ-ವಿ-ಷ್ಯ-ಗ-ಳೆಲ್ಲ
ಭೂತ-ವ-ರ್ತ-ಮಾ-ನ-ಭ-ವಿ-ಷ್ಯ-ತ್ಕಾ-ಲ-ಗಳ
ಭೂತ-ವ-ರ್ತ-ಮಾ-ನ-ಭ-ವಿ-ಷ್ಯ-ತ್ಕಾ-ಲ-ಗಳನ್ನೆಲ್ಲ
ಭೂತ-ದಂತೆ
ಭೂತಾನಿ
ಭೂದೇವ
ಭೂದೇ-ವಿ-ಯನ್ನು
ಭೂಮಿ
ಭೂಮಿಗೆ
ಭೂಮಿಯ
ಭೂಮಿ-ಯತ್ತ
ಭೂಮಿ-ಯನ್ನು
ಭೂಮಿ-ಯಲ್ಲೇ
ಭೂಮಿಯೇ
ಭೂಷಣ
ಭೃಗು
ಭೇಟಿ
ಭೇಟಿ-ಕೊ-ಡಲು
ಭೇಟಿ-ನೀಡಿ
ಭೇಟಿ-ಮಾಡಿ
ಭೇಟಿ-ಮಾ-ಡು-ತ್ತಿರ
ಭೇಟಿ-ಯಲ್ಲಿ
ಭೇಟಿಯಾ
ಭೇಟಿ-ಯಾ-ಗುವ
ಭೇಟಿ-ಯಾದ
ಭೇಟಿ-ಯಾ-ದಾಗ
ಭೇಟಿ-ಯಾದೆ
ಭೇಟಿ-ಯಾ-ಯಿತು
ಭೇಟಿ-ಯಿಂ-ದಲೂ
ಭೇಟಿಯೂ
ಭೇದ
ಭೇದ-ಗ-ಳೆ-ಲ್ಲ-ವನ್ನೂ
ಭೇದ-ಗಳೇ
ಭೇದ-ಬು-ದ್ಧಿ-ಯನ್ನು
ಭೇದ-ವನು
ಭೇದ-ವನ್ನು
ಭೇದ-ವಿಲ್ಲ
ಭೇದ-ವಿ-ಲ್ಲದೆ
ಭೇದವೂ
ಭೇದ-ವೆಂ-ಬುವ
ಭೇದವೇ
ಭೇದಿ-ಸಿ-ಕೊಂಡು
ಭೇದಿ-ಸಿಯೇ
ಭೇದಿ-ಸು-ವು-ದರ
ಭೇಷ್
ಭೈರವಿ
ಭೋಗ
ಭೋಗ-ಯೋ-ಗ-ಗಳ
ಭೋಗಕ್ಕೆ
ಭೋಗ-ಗಳ
ಭೋಗ-ಗಳನ್ನು
ಭೋಗ-ಗಳನ್ನೂ
ಭೋಗ-ಗಳನ್ನೆಲ್ಲ
ಭೋಗ-ಗಳಲ್ಲಿ
ಭೋಗ-ಭ-ರಿತ
ಭೋಗ-ಭಾ-ಗ್ಯ-ಗಳ
ಭೋಗ-ವನ್ನು
ಭೋಗ-ವಿ-ಮು-ಕ್ತ-ರ-ನ್ನಾ-ಗಿಸಿ
ಭೋಗವು
ಭೋಗ-ವೆಂ-ಬುದು
ಭೋಗಾ-ಸಕ್ತಿ
ಭೋಗಾ-ಸ್ಕ-ತಿ-ಯಿಂದ
ಭೋಗಿ-ಗ-ಳಾಗಿ
ಭೋಗಿ-ಗಳೇ
ಭೋಜನ
ಭೋಜ-ನ-ವನ್ನು
ಭೋರ್ಗ-ರೆ-ಯು-ತ್ತಿತ್ತು
ಭೋಲಾ
ಭೋಲಾ-ನಾಥ
ಭೋಲಾ-ನಾ-ಥ-ನನ್ನು
ಭೌತ-ವಾ-ದಿ-ಗಳ
ಭೌತ-ವಿ-ಜ್ಞಾ-ನ-ಗ-ಳಿಗೆ
ಭೌತಿಕ
ಭ್ರಮಿಸಿ
ಭ್ರಮಿ-ಸಿ-ದ್ದಾರೆ
ಭ್ರಮಿ-ಸುತ್ತ
ಭ್ರಮಿ-ಸುವ
ಭ್ರಮೆ
ಭ್ರಮೆ-ಗೀ-ಡಾ-ಗುವು
ಭ್ರಮೆ-ಯನ್ನು
ಭ್ರಮೆ-ಯಾ-ಗಿ-ರ-ಲೂ-ಬ-ಹು-ದೇನೋ
ಭ್ರಮೆ-ಯಾ-ಗು-ತ್ತಿ-ರ-ಬೇಕು
ಭ್ರಮೆ-ಯಿಂದ
ಭ್ರಮೆ-ಯಿ-ರ-ಬ-ಹುದು
ಭ್ರಷ್ಟ-ರಾ-ಗುವ
ಭ್ರಾಂತಿ
ಭ್ರಾಂತಿಯ
ಭ್ರಾಂತಿ-ಯಿಂದ
ಭ್ರಾಂತಿಯೇ
ಭ್ರಾತೃ-ವಾ-ತ್ಸ-ಲ್ಯ-ದಿಂದ
ಭ್ರೂಮ-ಧ್ಯ-ದಲ್ಲಿ
ಭ್ರೂಮ-ಧ್ಯ-ವನ್ನು
ಮಂಕಾ-ಗಿ-ಬಿ-ಟ್ಟಿದೆ
ಮಂಕು-ಬ-ಡಿ-ದ-ವ-ನಂತೆ
ಮಂಗ-ಮಾಯ
ಮಂಗ-ಲ-ಕರ
ಮಂಗಳ
ಮಂಗ-ಳ-ಕರ
ಮಂಗ-ಳ-ವಾ-ಗಲಿ
ಮಂಗ-ಳ-ವಾರ
ಮಂಗ-ಳಾ-ರತಿ
ಮಂಚದ
ಮಂಜಿ-ನಂ-ತಾ-ಗಿ-ಬಿ-ಟ್ಟಿತು
ಮಂಜು-ಗಡ್ಡೆ
ಮಂಜು-ಗೆಡ್ಡೆ
ಮಂಜುಳ
ಮಂಜೂರು
ಮಂಡಿ-ಸು-ತ್ತಿ-ದ್ದಂತೆ
ಮಂಡಿ-ಸುವ
ಮಂತ್ರ
ಮಂತ್ರ-ಗಳನ್ನು
ಮಂತ್ರ-ದಂತೆ
ಮಂತ್ರ-ದೀಕ್ಷೆ
ಮಂತ್ರ-ಮು-ಗ್ಧ-ನಾ-ಗಿ-ಬಿಡು
ಮಂತ್ರ-ಮು-ಗ್ಧ-ವಾಗು
ಮಂತ್ರ-ವನ್ನು
ಮಂತ್ರ-ವ-ನ್ನು-ಚ್ಚ-ರಿ-ಸುತ್ತ
ಮಂತ್ರವೋ
ಮಂತ್ರ-ಸಿ-ದ್ಧರು
ಮಂತ್ರಿ-ಗಳು
ಮಂತ್ರಿ-ಯಂತೆ
ಮಂತ್ರಿ-ಯಾಗು
ಮಂತ್ರೋ-ಚ್ಚಾ-ರಣೆ
ಮಂಥನ
ಮಂದ
ಮಂದ-ಮಾ-ರು-ತವೂ
ಮಂದ-ಸ್ಮಿ-ತ-ಳಾಗಿ
ಮಂದ-ಸ್ಮಿ-ತ-ವ-ದ-ನ-ರಾಗಿ
ಮಂದ-ಹಾಸ
ಮಂದ-ಹಾ-ಸ-ದಿಂದ
ಮಂದ-ಹಾ-ಸ-ವನು
ಮಂದಾ-ಕಿನೀ
ಮಂದಾ-ನಿ-ಲ-ಇ-ದುವೆ
ಮಂದಿ
ಮಂದಿಗೆ
ಮಂದಿ-ಯೆಲ್ಲ
ಮಂದಿ-ಯೊಂ-ದಿಗೆ
ಮಂದಿ-ರಕ್ಕೆ
ಮಂದಿ-ರದ
ಮಂದಿ-ರ-ದಲ್ಲಿ
ಮಂದಿ-ರ-ವನ್ನು
ಮಂದಿ-ರ-ವಾ-ಗಿ-ರಲಿ
ಮಂದಿ-ರೋ-ದ್ಯಾ-ನಕ್ಕೆ
ಮಂದೆ
ಮಂಪರು
ಮಕ-ರ-ಸಂ-ಕ್ರಾಂ-ತಿಯ
ಮಕ್ಕಳ
ಮಕ್ಕ-ಳಂತೆ
ಮಕ್ಕ-ಳನ್ನು
ಮಕ್ಕ-ಳನ್ನೂ
ಮಕ್ಕ-ಳ-ನ್ನೆಲ್ಲ
ಮಕ್ಕ-ಳಾ-ಗಿ-ದ್ದುವು
ಮಕ್ಕಳಿ
ಮಕ್ಕ-ಳಿಂದ
ಮಕ್ಕ-ಳಿ-ಗಾಗಿ
ಮಕ್ಕ-ಳಿಗೂ
ಮಕ್ಕ-ಳಿಗೆ
ಮಕ್ಕ-ಳಿರಾ
ಮಕ್ಕಳು
ಮಕ್ಕ-ಳು-ದು-ರ್ಗಾ-ಪ್ರ-ಸಾದ
ಮಕ್ಕಳೂ
ಮಕ್ಕ-ಳೆಂದು
ಮಕ್ಕ-ಳೆಲ್ಲ
ಮಕ್ಕ-ಳೆಷ್ಟೋ
ಮಗ
ಮಗನ
ಮಗ-ನನ್ನು
ಮಗ-ನಾಗಿ
ಮಗ-ನಾ-ದ್ದ-ರಿಂದ
ಮಗನಿ
ಮಗ-ನಿ-ಗಂತೂ
ಮಗ-ನಿಗೆ
ಮಗ-ನೊಂ-ದಿಗೆ
ಮಗ-ನೊಂ-ದಿಗೇ
ಮಗಳು
ಮಗ-ಳು-ಶಾ-ರದೆ
ಮಗು
ಮಗು-ವನ್ನು
ಮಗು-ವ-ನ್ನೇನೋ
ಮಗು-ವಲ್ಲ
ಮಗು-ವ-ಲ್ಲವೇ
ಮಗು-ವಿ-ಗಾಗಿ
ಮಗು-ವಿಗೆ
ಮಗು-ವಿನ
ಮಗು-ವಿ-ನೊಂ-ದಿಗೆ
ಮಗು-ವಿ-ಲ್ಲ-ವೆಂಬ
ಮಗುವು
ಮಗೂ
ಮಗ್ಗು-ಲಲ್ಲಿ
ಮಗ್ನ
ಮಗ್ನ-ನಾಗಿ
ಮಗ್ನ-ನಾ-ಗಿ-ಬಿಟ್ಟ
ಮಗ್ನ-ನಾ-ಗಿ-ಬಿ-ಟ್ಟಿ-ದ್ದಾ-ನೆಯೋ
ಮಗ್ನ-ನಾ-ಗಿ-ಬಿ-ಡು-ತ್ತಿದ್ದ
ಮಗ್ನ-ರಾಗಿ
ಮಗ್ನ-ವಾ-ಗಿ-ದ್ದರೆ
ಮಗ್ನ-ವಾ-ಗು-ತ್ತಿತ್ತು
ಮಘ-ದಲ
ಮಜುಮ
ಮಜು-ಮ-ದಾರ
ಮಟ-ಗು-ಟ್ಟುತ್ತ
ಮಟ್ಟ
ಮಟ್ಟಕ್ಕೆ
ಮಟ್ಟದ
ಮಟ್ಟ-ದ-ವ-ರಿ-ದ್ದಾ-ರೆಂ-ಬುದು
ಮಟ್ಟದ್ದು
ಮಟ್ಟವೇ
ಮಟ್ಟಿ-ಗಾ-ದರೂ
ಮಟ್ಟಿಗೆ
ಮಟ್ಟಿ-ಗೆಂ-ದರೆ
ಮಟ್ಟಿ-ಗೆ-ಶಾ-ಸ್ತ್ರಕ್ಕೆ
ಮಠ
ಮಠ-ಗು-ರು-ಭಾ-ಯಿ-ಗಳು
ಮಠ-ಕ್ಕಾಗಿ
ಮಠಕ್ಕೆ
ಮಠ-ಗಳಲ್ಲಿ
ಮಠದ
ಮಠ-ದಲ್ಲಿ
ಮಠ-ದ-ಲ್ಲಿನ
ಮಠ-ದ-ಲ್ಲಿ-ರ-ಲಿಲ್ಲ
ಮಠ-ದಲ್ಲೇ
ಮಠ-ದಿಂದ
ಮಠ-ದೊಳ
ಮಠ-ನಿ-ವಾ-ಸಿ-ಗ-ಳಾದ
ಮಠ-ವನ್ನು
ಮಠ-ವಾ-ಸಿ-ಗ-ಳಾದ
ಮಠವೇ
ಮಠ-ಸ್ಥಾ-ಪನೆ
ಮಡದಿ
ಮಡ-ದಿಯ
ಮಡ-ದಿ-ಯನ್ನು
ಮಡ-ದಿ-ಯಾಗಿ
ಮಡಿ
ಮಡಿ-ಯಲ್ಲಿ
ಮಡಿ-ಯಾಗಿ
ಮಡಿ-ಲಲ್ಲಿ
ಮಡಿ-ಲಿಗೆ
ಮಡಿ-ವಂತ
ಮಡಿ-ವಂ-ತ-ನೆಂದು
ಮಡಿ-ವಂ-ತಿ-ಕೆ-ಗಳನ್ನು
ಮಡಿ-ವಂ-ತಿ-ಕೆಯೋ
ಮಡಿ-ಸಿ-ಬಿ-ಡಿಸಿ
ಮಡು-ವಿಗೆ
ಮಣ-ಗ-ಟ್ಟಲೆ
ಮಣಿದು
ಮಣಿ-ಸು-ವುದು
ಮಣ್ಣಿಗೇ
ಮಣ್ಣಿ-ನಿಂದ
ಮಣ್ಣು
ಮತ
ಮತ-ಧ-ರ್ಮ-ಗಳ
ಮತ-ಪಂ-ಥ-ವನ್ನೂ
ಮತಕ್ಕೆ
ಮತ-ಗಳ
ಮತ-ಗ-ಳಿವೆ
ಮತ-ಗಳು
ಮತ-ಧ-ರ್ಮ-ಗಳ
ಮತ-ಪಂ-ಥ-ಗಳ
ಮತ-ವನ್ನು
ಮತ-ವನ್ನೋ
ಮತವು
ಮತಾಂ-ತರ
ಮತಾಂ-ತ-ರಿ-ಸು-ವುದು
ಮತಾಂ-ಧರ
ಮತೀ-ಯರ
ಮತ್ತ-ರಾ-ಗಿ-ರು-ವ-ವ-ರೆಗೆ
ಮತ್ತ-ವರ
ಮತ್ತಷ್ಟು
ಮತ್ತಾ-ರಿಗೆ
ಮತ್ತಾರು
ಮತ್ತಿ-ತರ
ಮತ್ತಿ-ತ-ರರು
ಮತ್ತಿ-ತರು
ಮತ್ತಿ-ನಲ್ಲಿ
ಮತ್ತಿ-ನ್ನೆಂದೂ
ಮತ್ತಿ-ನ್ನೇನು
ಮತ್ತು
ಮತ್ತೂ
ಮತ್ತೆ
ಮತ್ತೆಂದೂ
ಮತ್ತೆ-ಮತ್ತೆ
ಮತ್ತೆ-ಯನ್ನು
ಮತ್ತೆ-ರಡು
ಮತ್ತೊಂದು
ಮತ್ತೊಬ್ಬ
ಮತ್ತೊ-ಬ್ಬರ
ಮತ್ತೊ-ಬ್ಬ-ರೆಂ-ದರೆ
ಮತ್ತೊಮ್ಮೆ
ಮಥು-ರ-ನಾ-ಥ-ರಿ-ಬ್ಬರೂ
ಮಥುರಾ
ಮಥು-ರಾ-ನಾಥ
ಮಥು-ರಾ-ನಾ-ಥನ
ಮದುವೆ
ಮದು-ವೆ-ಮಾಡಿ
ಮದು-ವೆ-ಮಾ-ಡಿ-ಕೊಂಡು
ಮದು-ವೆಯ
ಮದು-ವೆ-ಯನ್ನು
ಮದು-ವೆ-ಯಾಗಿ
ಮದು-ವೆ-ಯಾ-ಗು-ವಂತೆ
ಮದು-ವೆ-ಯಾ-ಗು-ವು-ದ-ರಿಂದ
ಮದು-ವೆ-ಯಾ-ಗು-ವು-ದಿಲ್ಲ
ಮದು-ವೆ-ಯಾ-ಗು-ವು-ದಿ-ಲ್ಲ-ವೆಂಬ
ಮದು-ವೆ-ಯಾ-ಗು-ವುದೇ
ಮದು-ವೆ-ಯಾ-ದ-ವ-ರ-ಲ್ಲವೆ
ಮದು-ವೆ-ಯಾ-ದಾಗ
ಮದು-ವೆ-ಯೆಂ-ಬುದು
ಮದು-ವೆ-ಯೊಂದು
ಮದ್ಯ-ಪಾನ
ಮದ್ಯ-ಪಾ-ನದ
ಮಧು-ಕ-ರ-ವೃತ್ತಿ
ಮಧು-ಕರಿ
ಮಧು-ಕರೀ
ಮಧು-ಮಿ-ಶ್ರಿತ
ಮಧುರ
ಮಧು-ರ-ಗಂ-ಭೀರ
ಮಧು-ರ-ಗಾ-ಯ-ನ-ವನ್ನು
ಮಧು-ರ-ತರ
ಮಧು-ರ-ನು-ಡಿ-ಗಳ
ಮಧು-ರ-ಭಾವ
ಮಧು-ರ-ವಾಗಿ
ಮಧು-ರ-ವಾ-ಗಿತ್ತು
ಮಧು-ರ-ವಾದ
ಮಧು-ರ-ಸ್ಮೃತಿ
ಮಧು-ವನ್ನು
ಮಧು-ವನ್ನೇ
ಮಧ್ಯ
ಮಧ್ಯಂ-ತರ
ಮಧ್ಯದ
ಮಧ್ಯ-ದಲ್ಲಿ
ಮಧ್ಯ-ದಲ್ಲೂ
ಮಧ್ಯ-ದಿಂದ
ಮಧ್ಯ-ಭಾ-ಗ-ದಲ್ಲಿ
ಮಧ್ಯ-ಭಾ-ರತ
ಮಧ್ಯ-ಮ-ಧ್ಯ-ದಲ್ಲಿ
ಮಧ್ಯ-ರಾತ್ರಿ
ಮಧ್ಯ-ರಾ-ತ್ರಿಯ
ಮಧ್ಯ-ರಾ-ತ್ರಿ-ಯಲ್ಲಿ
ಮಧ್ಯಾಹ್ನ
ಮಧ್ಯಾ-ಹ್ನದ
ಮಧ್ಯಾ-ಹ್ನ-ವಾ-ದರೂ
ಮಧ್ಯಾ-ಹ್ನ-ವಾ-ಯಿತು
ಮಧ್ಯೆ
ಮಧ್ಯೆ-ಮಧ್ಯೆ
ಮನ
ಮನಃ-ಪ-ಟ-ಲದ
ಮನಃ-ಸ್ಥಿತಿ
ಮನಃ-ಸ್ಥಿ-ತಿ-ಯನ್ನೂ
ಮನ-ಕ-ರಗ
ಮನ-ಗಂಡ
ಮನ-ಗಂ-ಡಿದ್ದ
ಮನ-ಗಂ-ಡಿ-ದ್ದರು
ಮನ-ಗಂಡು
ಮನ-ಗಾ-ಣಲು
ಮನ-ಗಾ-ಣಿ-ಸಲು
ಮನ-ಗಾ-ಣುತ್ತ
ಮನ-ಗಾ-ಣು-ವಂ-ತಾ-ಯಿತು
ಮನ-ಗೊ-ಡ-ಬೇಕು
ಮನದ
ಮನ-ದ-ಟ್ಟಾ-ಗಿ-ತ್ತೆಂದು
ಮನ-ದ-ಟ್ಟಾ-ಯಿತು
ಮನ-ದಟ್ಟು
ಮನ-ದ-ಣಿಯೆ
ಮನ-ದಲ್ಲಿ
ಮನ-ದ-ಲ್ಲಿ-ಟ್ಟು-ಕೊಂಡು
ಮನ-ದ-ಲ್ಲೆದ್ದ
ಮನ-ದಲ್ಲೇ
ಮನ-ದಾಳ
ಮನ-ದಿಂ
ಮನ-ದಿಂ-ದಾ-ಚೆಗೆ
ಮನ-ದೊ-ಳಗೇ
ಮನ-ದೊ-ಳೆದ್ದ
ಮನನ
ಮನ-ನಾ-ಟು-ವಂತೆ
ಮನ-ನೀಯ
ಮನ-ಬಿಚ್ಚಿ
ಮನ-ಮು-ಟ್ಟಿ-ಸು-ವು-ದ-ಕ್ಕಾಗಿ
ಮನ-ಮು-ಟ್ಟು-ವಂತೆ
ಮನ-ಮೋ-ಹಕ
ಮನ-ಮೋ-ಹನ
ಮನ-ರಂ-ಜನೆ
ಮನ-ರಂ-ಜ-ನೆ-ಯಾ-ಗಲಿ
ಮನ-ವ-ರಿಕೆ
ಮನ-ವ-ರಿ-ಕೆ-ಯಾ-ಗಿತ್ತು
ಮನ-ವ-ರಿ-ಕೆ-ಯಾ-ಗಿ-ರ-ಬೇಕು
ಮನ-ವ-ರಿ-ಕೆ-ಯಾ-ಗುತ್ತ
ಮನ-ವ-ರಿ-ಕೆ-ಯಾ-ಗು-ತ್ತಿದೆ
ಮನ-ವ-ರಿ-ಕೆ-ಯಾದ
ಮನ-ವ-ರಿ-ಕೆ-ಯಾ-ಯಿತು
ಮನ-ವ-ರಿತು
ಮನ-ವಿಟ್ಟು
ಮನವು
ಮನ-ವು-ಬೀ-ಗ-ಮು-ದ್ರೆ-ಯ-ನಿಟ್ಟ
ಮನವೆ
ಮನ-ವೊ-ಪ್ಪುವ
ಮನ-ವೊ-ಲಿ-ಸಿ-ಕೊ-ಳ್ಳುವ
ಮನ-ಶ್ಶಾಸ್ತ್ರ
ಮನ-ಶ್ಶಾ-ಸ್ತ್ರ-ಗ-ಳಿಗೆ
ಮನ-ಶ್ಶಾ-ಸ್ತ್ರದ
ಮನ-ಶ್ಶುದ್ಧಿ-ಯಿ-ಲ್ಲದ
ಮನಸಾ
ಮನ-ಸಾರೆ
ಮನ-ಸೂ-ರೆ-ಗೊಂ-ಡಿ-ದ್ದುವು
ಮನ-ಸೋ-ತಿದ್ದ
ಮನ-ಸ್ಥಿತ
ಮನ-ಸ್ಥಿ-ತಿ-ಯನ್ನು
ಮನ-ಸ್ಥಿ-ತಿ-ಯಲ್ಲಿ
ಮನ-ಸ್ಥಿ-ತಿ-ಯ-ಲ್ಲಿದ್ದ
ಮನ-ಸ್ಥಿ-ತಿ-ಯಲ್ಲೇ
ಮನ-ಸ್ಥಿ-ತಿ-ಯ-ಲ್ಲೇನೋ
ಮನ-ಸ್ಸ-ನಿಂದ
ಮನ-ಸ್ಸ-ನ್ನಾ-ದರೂ
ಮನ-ಸ್ಸನ್ನು
ಮನ-ಸ್ಸನ್ನೂ
ಮನ-ಸ್ಸ-ನ್ನೆಲ್ಲ
ಮನ-ಸ್ಸನ್ನೇ
ಮನ-ಸ್ಸಾ-ಗ-ಲಿಲ್ಲ
ಮನ-ಸ್ಸಾ-ಗಿದೆ
ಮನ-ಸ್ಸಾ-ಗಿ-ರ-ಬ-ಹುದೆ
ಮನ-ಸ್ಸಾ-ಗು-ತ್ತಲೇ
ಮನ-ಸ್ಸಾ-ಯಿತು
ಮನಸ್ಸಿ
ಮನ-ಸ್ಸಿ-ಗಾಗ
ಮನ-ಸ್ಸಿ-ಗಿನ್ನೂ
ಮನ-ಸ್ಸಿಗೂ
ಮನ-ಸ್ಸಿಗೆ
ಮನ-ಸ್ಸಿ-ಟ್ಟಿ-ದ್ದಾರೋ
ಮನ-ಸ್ಸಿನ
ಮನ-ಸ್ಸಿ-ನಲ್ಲಿ
ಮನ-ಸ್ಸಿ-ನ-ಲ್ಲಿದ್ದ
ಮನ-ಸ್ಸಿ-ನಲ್ಲೂ
ಮನ-ಸ್ಸಿ-ನ-ಲ್ಲೆದ್ದ
ಮನ-ಸ್ಸಿ-ನಲ್ಲೇ
ಮನ-ಸ್ಸಿ-ನ-ಲ್ಲೊಂದು
ಮನ-ಸ್ಸಿ-ನ-ವರ
ಮನ-ಸ್ಸಿ-ನಿಂದ
ಮನ-ಸ್ಸಿ-ನಿಂ-ದಲೇ
ಮನ-ಸ್ಸಿಲ್ಲ
ಮನ-ಸ್ಸೀಗ
ಮನಸ್ಸು
ಮನ-ಸ್ಸು-ಇಂ-ದ್ರಿ-ಯ-ಗ-ಳಿಂ-ದಲೇ
ಮನ-ಸ್ಸು-ದೇ-ಹ-ಗಳು
ಮನ-ಸ್ಸು-ಬು-ದ್ಧಿ-ಗಳು
ಮನ-ಸ್ಸು-ಗಳನ್ನೆಲ್ಲ
ಮನಸ್ಸೂ
ಮನ-ಸ್ಸೆಲ್ಲ
ಮನಸ್ಸೇ
ಮನು-ಕುಲ
ಮನು-ಜನ
ಮನುಷ್ಯ
ಮನು-ಷ್ಯ-ಜನ್ಮ
ಮನು-ಷ್ಯ-ದೇಹ
ಮನು-ಷ್ಯನ
ಮನು-ಷ್ಯ-ನನ್ನು
ಮನು-ಷ್ಯ-ನಲ್ಲಿ
ಮನು-ಷ್ಯ-ನಾ-ದ-ವನು
ಮನು-ಷ್ಯ-ನಿಗೆ
ಮನು-ಷ್ಯನು
ಮನು-ಷ್ಯನೂ
ಮನು-ಷ್ಯನೇ
ಮನು-ಷ್ಯರ
ಮನು-ಷ್ಯ-ರನ್ನು
ಮನು-ಷ್ಯ-ರಿ-ರಲಿ
ಮನು-ಷ್ಯರು
ಮನು-ಷ್ಯ-ರೂ-ಪ-ದಿಂದ
ಮನು-ಷ್ಯ-ರೆಂಬ
ಮನು-ಷ್ಯ-ರೆ-ಲ್ಲರೂ
ಮನೆ
ಮನೆ-ಕಡೆ
ಮನೆ-ಗಳ
ಮನೆ-ಗಳನ್ನು
ಮನೆ-ಗ-ಳನ್ನೇ
ಮನೆ-ಗಳಲ್ಲಿ
ಮನೆ-ಗ-ಳಲ್ಲೇ
ಮನೆ-ಗ-ಳ-ವರೂ
ಮನೆ-ಗ-ಳ-ವ-ರೆಗೆ
ಮನೆ-ಗಳಿಂದ
ಮನೆ-ಗ-ಳಿಗೆ
ಮನೆ-ಗಳು
ಮನೆಗೂ
ಮನೆಗೆ
ಮನೆ-ಗೆ-ಲ-ಸಕ್ಕೆ
ಮನೆಗೇ
ಮನೆಗೋ
ಮನೆ-ತನ
ಮನೆ-ತ-ನದ
ಮನೆ-ತ-ನ-ದ-ವರು
ಮನೆ-ತ-ನ-ದ-ವ-ರೆಂದೋ
ಮನೆ-ತ-ನ-ದ-ವರೆಲ್ಲ
ಮನೆ-ಬಾ-ಗಿ-ಲಿಗೆ
ಮನೆ-ಬಿಟ್ಟು
ಮನೆ-ಮಂದಿ
ಮನೆ-ಮಂ-ದಿ-ಗಂತೂ
ಮನೆ-ಮಂ-ದಿಗೂ
ಮನೆ-ಮಂ-ದಿಗೆ
ಮನೆ-ಮಂ-ದಿಯ
ಮನೆ-ಮಂ-ದಿ-ಯನ್ನೂ
ಮನೆ-ಮಂ-ದಿ-ಯ-ನ್ನೆಲ್ಲ
ಮನೆ-ಮಂ-ದಿಯೂ
ಮನೆ-ಮಂ-ದಿ-ಯೆಲ್ಲ
ಮನೆ-ಮಂ-ದಿ-ಯೊ-ಳಗೆ
ಮನೆ-ಮ-ನೆ-ಗ-ಳಲ್ಲೂ
ಮನೆ-ಮ-ನೆಗೆ
ಮನೆ-ಮಾ-ತಾ-ಗಿತ್ತು
ಮನೆಯ
ಮನೆ-ಯನ್ನು
ಮನೆ-ಯ-ಲ್ಲಯೇ
ಮನೆ-ಯಲ್ಲಿ
ಮನೆ-ಯ-ಲ್ಲಿ-ಟ್ಟು-ಕೊ-ಳ್ಳಲು
ಮನೆ-ಯ-ಲ್ಲಿ-ದ್ದ-ವ-ರಿಂದ
ಮನೆ-ಯ-ಲ್ಲಿ-ದ್ದ-ವರೆ-ಲ್ಲರ
ಮನೆ-ಯ-ಲ್ಲಿ-ದ್ದಾಗ
ಮನೆ-ಯ-ಲ್ಲಿದ್ದು
ಮನೆ-ಯ-ಲ್ಲಿನ
ಮನೆ-ಯ-ಲ್ಲಿಯೇ
ಮನೆ-ಯಲ್ಲೂ
ಮನೆ-ಯಲ್ಲೇ
ಮನೆ-ಯ-ಲ್ಲೊಂದು
ಮನೆ-ಯ-ಲ್ಲೊಬ್ಬ
ಮನೆ-ಯ-ವರ
ಮನೆ-ಯ-ವ-ರ-ನ್ನೆಲ್ಲ
ಮನೆ-ಯ-ವ-ರಾ-ರಿಗೂ
ಮನೆ-ಯ-ವ-ರಿಂದ
ಮನೆ-ಯ-ವ-ರಿಗೆ
ಮನೆ-ಯ-ವ-ರಿ-ಗೆಲ್ಲ
ಮನೆ-ಯ-ವರು
ಮನೆ-ಯ-ವ-ರು-ಏ-ನ್ರಯ್ಯ
ಮನೆ-ಯ-ವರೂ
ಮನೆ-ಯ-ವರೆಲ್ಲ
ಮನೆ-ಯ-ವರೆ-ಲ್ಲ-ರಿಗೂ
ಮನೆ-ಯ-ವರೆ-ಲ್ಲರೂ
ಮನೆ-ಯ-ವ-ರೊಂ-ದಿಗೆ
ಮನೆ-ಯಾ-ಗಿದ್ದು
ಮನೆ-ಯಿಂದ
ಮನೆ-ಯಿ-ದ್ದುದು
ಮನೆ-ಯಿ-ಲ್ಲ-ದ-ವರು
ಮನೆಯು
ಮನೆಯೂ
ಮನೆ-ಯೊಂ-ದರ
ಮನೆ-ಯೊಂ-ದ-ರಲ್ಲಿ
ಮನೆ-ಯೊಂದು
ಮನೆ-ಯೊ-ಳ-ಗಿ-ದ್ದ-ವರೆಲ್ಲ
ಮನೆ-ವಾ-ರ್ತೆಯ
ಮನೊ-ಭಾ-ವವೇ
ಮನೊಯ
ಮನೋ
ಮನೋ-ದಾ-ರ್ಢ್ಯ-ವನ್ನು
ಮನೋ-ಧರ್ಮ
ಮನೋ-ಬು-ದ್ಧಿ-ಗಳನ್ನು
ಮನೋ-ಭಾವ
ಮನೋ-ಭಾ-ವ-ಇ-ವು-ಗಳಿಂದ
ಮನೋ-ಭಾ-ವಕ್ಕೆ
ಮನೋ-ಭಾ-ವ-ಗಳೂ
ಮನೋ-ಭಾ-ವದ
ಮನೋ-ಭಾ-ವ-ದಲ್ಲಿ
ಮನೋ-ಭಾ-ವ-ದ-ವ-ನಾ-ದ್ದ-ರಿಂದ
ಮನೋ-ಭಾ-ವ-ದ-ವ-ರಿಗೆ
ಮನೋ-ಭಾ-ವ-ದ-ವರು
ಮನೋ-ಭಾ-ವ-ದಿಂದ
ಮನೋ-ಭಾ-ವ-ವನ್ನು
ಮನೋ-ಭಾ-ವವು
ಮನೋ-ಭಾ-ವವೂ
ಮನೋ-ಭಿ-ಪ್ರಾಯ
ಮನೋ-ರಥ
ಮನೋ-ರ-ಥ-ಗಳನ್ನು
ಮನೋ-ವೃತ್ತಿ
ಮನೋ-ವೃ-ತ್ತಿಗೆ
ಮನೋ-ವೃ-ತ್ತಿಯೂ
ಮನೋ-ವೃ-ತ್ತಿ-ಯೆಂ-ದರೆ
ಮನೋ-ಸಾ-ಮ್ರಾ-ಜ್ಯ-ದಲ್ಲಿ
ಮನೋ-ಹರ
ಮನೋ-ಹ-ರ-ವಾ-ಗಿದೆ
ಮನ್ನಣೆ
ಮನ್ನ-ಣೆಗೆ
ಮನ್ನ-ಣೆ-ಯಿತ್ತು
ಮನ್ನ-ಣೆ-ಯಿ-ರ-ಲಿಲ್ಲ
ಮನ್ನಿಸಿ
ಮನ್ನಿ-ಸು-ತ್ತಿ-ರ-ಲಿಲ್ಲ
ಮನ್ಮಥ
ಮನ್ಮ-ಥ-ನಾಥ
ಮನ್ಮ-ಥ-ನಾ-ಥನ
ಮನ್ಮ-ಥ-ಬಾಬು
ಮಬ್ಬು
ಮಮ-ಕಾರ
ಮಮತೆ
ಮಯ
ಮಯಿ
ಮರ
ಮರ-ಕೋ-ತಿ-ಯಾಟ
ಮರಕ್ಕೆ
ಮರ-ಗಳನ್ನು
ಮರ-ಗಿ-ಡ-ಬ-ಳ್ಳಿ-ಗಳು
ಮರಣ
ಮರ-ಣ-ಕಾ-ಲ-ದಲ್ಲಿ
ಮರ-ಣ-ಕ್ಕೀ-ಡಾ-ಗುವ
ಮರ-ಣದ
ಮರ-ಣ-ದಿಂದ
ಮರ-ಣ-ದೆ-ಡೆ-ಗಾ-ಗೆ-ಳೆ-ವು-ದೆ-ಮ್ಮನು
ಮರ-ಣವು
ಮರ-ಣ-ವೆಂ-ಬುದು
ಮರ-ಣವೇ
ಮರ-ಣ-ವೊಂದೇ
ಮರ-ಣ-ಶ-ಯ್ಯೆ-ಯಲ್ಲಿ
ಮರ-ಣಾ-ನಂ-ತರ
ಮರದ
ಮರ-ದಿಂದ
ಮರ-ಳ-ತೊ-ಡ-ಗಿತು
ಮರಳಿ
ಮರ-ಳಿತು
ಮರ-ಳಿದ
ಮರ-ಳಿ-ದರು
ಮರ-ಳಿ-ದರೆ
ಮರ-ಳಿ-ದಳು
ಮರ-ಳಿ-ದಾಗ
ಮರ-ಳಿನ
ಮರ-ಳಿಲ್ಲ
ಮರ-ಳು-ಗಾ-ಡಾ-ಗು-ತ್ತದೆ
ಮರ-ಳು-ತ್ತಿ-ರು-ವಂತೆ
ಮರ-ಳುವ
ಮರ-ಳು-ವು-ದೆಮ್ಮ
ಮರ-ವನ್ನು
ಮರ-ವಿತ್ತು
ಮರ-ವೆಂದರೆ
ಮರ-ವೆ-ಯಾ-ಗು-ತ್ತಿತ್ತು
ಮರಿಗೆ
ಮರಿ-ತು-ಟಿ-ಗಿ-ಟ್ಟು-ಕೊಂಡು
ಮರಿ-ಯ-ಲ್ಲವೆ
ಮರಿ-ಸಿಂಹ
ಮರಿ-ಸಿಂ-ಹ-ಮ-ಹಾ-ಸಿಂಹ
ಮರೀ-ಚಿಕೆ
ಮರು
ಮರು-ಕ-ಳಿಸ
ಮರು-ಕ-ಳಿ-ಸ-ತೊ-ಡ-ಗಿತ್ತು
ಮರು-ಕ-ಳಿ-ಸ-ದಂತೆ
ಮರು-ಕ-ಳಿ-ಸಿ-ಬಿ-ಟ್ಟಿತು
ಮರು-ಕ್ಷಣ
ಮರು-ಕ್ಷ-ಣಕ್ಕೆ
ಮರು-ಕ್ಷ-ಣಕ್ಕೇ
ಮರು-ಕ್ಷ-ಣ-ದಲ್ಲೇ
ಮರು-ಕ್ಷ-ಣವೇ
ಮರು-ಗುತ್ತೀ
ಮರು-ಗು-ವಂ-ತಾದ
ಮರು-ಘ-ಳಿ-ಗೆಯೇ
ಮರು-ತ್ತು-ಗಳೇ
ಮರು-ದನಿ
ಮರು-ದಿನ
ಮರು-ದಿ-ನದ
ಮರು-ದಿ-ನ-ದಂದು
ಮರು-ದಿ-ನ-ದಿಂ-ದಲೇ
ಮರು-ದಿ-ನವೇ
ಮರು-ಪ್ರಶ್ನೆ
ಮರು-ಪ್ರ-ಶ್ನೆಯೇ
ಮರು-ಭೂ-ಮಿ-ಯಾ-ಗಲಿ
ಮರು-ಮ-ರೀ-ಚಿಕೆ
ಮರು-ಮಾ-ತಿ-ಲ್ಲದೆ
ಮರುಳ
ಮರು-ಳ-ನಂತೆ
ಮರು-ಳಾಗಿ
ಮರೆತ
ಮರೆ-ತಾ-ಗಿತ್ತು
ಮರೆ-ತಿಲ್ಲ
ಮರೆ-ತೀಯೆ
ಮರೆತು
ಮರೆ-ತು-ಬಿ-ಟ್ಟಿ-ದ್ದಾನೆ
ಮರೆ-ತು-ಬಿ-ಡು-ವಂತೆ
ಮರೆತೆ
ಮರೆ-ತೆಯೋ
ಮರೆ-ತೇ-ಬಿಟ್ಟ
ಮರೆ-ತೇ-ಬಿಟ್ಟೆ
ಮರೆ-ಮಾಚು
ಮರೆ-ಯ-ಬ-ಹು-ದಾ-ಗಿತ್ತು
ಮರೆ-ಯ-ಲಿಲ್ಲ
ಮರೆ-ಯಲ್ಲಿ
ಮರೆ-ಯಾಗಿ
ಮರೆ-ಯಾ-ಗಿದೆ
ಮರೆ-ಯಾ-ಗಿವೆ
ಮರೆ-ಯಾ-ಗು-ತ್ತಿವೆ
ಮರೆ-ಯಾದ
ಮರೆ-ಯು-ತ್ತಿದ್ದ
ಮರೆ-ಯುವ
ಮರೆ-ಯು-ವಂತೆ
ಮರೆ-ಯು-ವು-ದ-ಕ್ಕಾಗಿ
ಮರೆ-ಹೊ-ಗ-ಬೇ-ಕಾದ
ಮರ್ತ್ಯ
ಮರ್ತ್ಯನ
ಮರ್ತ್ಯ-ನಂತೆ
ಮರ್ತ್ಯ-ಲೋ-ಕದ
ಮರ್ದಿ-ಸಲು
ಮರ್ಮ
ಮರ್ಮ-ಮೌ-ಲ್ಯ-ಗಳನ್ನೂ
ಮರ್ಮ-ಗಳನ್ನು
ಮರ್ಮ-ವ-ನ್ನ-ರಿ-ಯ-ಲಾ-ಗದೆ
ಮರ್ಮ-ವನ್ನು
ಮರ್ಮ-ವನ್ನೂ
ಮರ್ಮ-ವೇನು
ಮರ್ಮ-ಸ್ಪರ್ಶಿ
ಮರ್ಯಾ-ದೆ-ಯ-ನ್ನಾ-ದರೂ
ಮರ್ಯಾ-ದೆ-ಯನ್ನೇ
ಮಲ-ಗಲು
ಮಲಗಿ
ಮಲ-ಗಿ-ಕೊಂಡ
ಮಲ-ಗಿ-ಕೊಂ-ಡರೆ
ಮಲ-ಗಿ-ಕೊಂ-ಡಳು
ಮಲ-ಗಿ-ಕೊಂ-ಡಾಗ
ಮಲ-ಗಿ-ಕೊಂ-ಡಿ-ದ್ದಾ-ನಲ್ಲ
ಮಲ-ಗಿ-ಕೊಂ-ಡಿ-ದ್ದಾನೆ
ಮಲ-ಗಿ-ಕೊ-ಳ್ಳಲು
ಮಲ-ಗಿ-ಕೊಳ್ಳು
ಮಲ-ಗಿ-ಕೊ-ಳ್ಳು-ತ್ತಿದ್ದ
ಮಲ-ಗಿ-ಕೊ-ಳ್ಳು-ವಾಗ
ಮಲ-ಗಿ-ದರು
ಮಲ-ಗಿದ್ದ
ಮಲ-ಗಿ-ದ್ದನೋ
ಮಲ-ಗಿ-ದ್ದರೂ
ಮಲ-ಗಿ-ದ್ದಾನೆ
ಮಲ-ಗಿ-ದ್ದಾರೆ
ಮಲ-ಗಿ-ದ್ದಾ-ರೆಂದು
ಮಲ-ಗಿದ್ದೆ
ಮಲ-ಗಿ-ಬಿ-ಟ್ಟರು
ಮಲ-ಗಿ-ಬಿ-ಟ್ಟಿ-ದ್ದಾನೆ
ಮಲ-ಗಿ-ಬಿ-ಟ್ಟಿ-ದ್ದಾರೆ
ಮಲ-ಗಿ-ಬಿ-ಡು-ತ್ತಿದ್ದೆ
ಮಲ-ಗಿ-ರುವ
ಮಲ-ಗಿ-ಸಿದ
ಮಲ-ಗಿಹ
ಮಲ-ಗು-ತ್ತಿ-ದ್ದರು
ಮಲಿನ
ಮಲಿ-ನ-ಗೊ-ಳ್ಳು-ವು-ದಿಲ್ಲ
ಮಲಿ-ನ-ತೆ-ಯೆಂ-ಬು-ದಾ-ಗಲಿ
ಮಲೇ-ರಿಯಾ
ಮಲೇ-ರಿ-ಯಾಗೆ
ಮಲೇ-ರಿ-ಯಾ-ದಿಂದ
ಮಲ್ಲ-ಯು-ದ್ಧ-ಗಳಲ್ಲಿ
ಮಲ್ಲಿಗೆ
ಮಳೆ
ಮಳೆ-ಬಿ-ಸಿ-ಲು-ಚ-ಳಿಗೆ
ಮಳೆ-ಗ-ರೆ-ಯು-ವು-ದ-ಕ್ಕಾಗಿ
ಮಳೆ-ಯಲ್ಲಿ
ಮಸ್ತ-ಕ-ಕ್ಕಿ-ಳಿ-ಸಿ-ಬಿ-ಡು-ತ್ತಿದ್ದ
ಮಹ
ಮಹಂ-ತರ
ಮಹಂ-ತರು
ಮಹಡಿ
ಮಹ-ಡಿಯ
ಮಹ-ಡಿ-ಯ-ನ್ನೇರಿ
ಮಹ-ತ್ಕಾರ್ಯ
ಮಹ-ತ್ತರ
ಮಹ-ತ್ತ-ರ-ವಾದ
ಮಹ-ತ್ತ-ರ-ವಾ-ದದ್ದು
ಮಹತ್ವ
ಮಹ-ತ್ವದ
ಮಹ-ತ್ವ-ಪೂರ್ಣ
ಮಹ-ತ್ವ-ಪೂ-ರ್ಣ-ವಾದ
ಮಹ-ತ್ವ-ವನ್ನು
ಮಹ-ತ್ವ-ವನ್ನೂ
ಮಹ-ತ್ವ-ವಿತ್ತು
ಮಹ-ತ್ವ-ವಿಲ್ಲ
ಮಹ-ದ-ಭಿ-ಲಾಷೆ
ಮಹರ್ಷಿ
ಮಹ-ರ್ಷಿ-ಗಳನ್ನೆಲ್ಲ
ಮಹ-ರ್ಷಿ-ಗ-ಳಿಗೆ
ಮಹ-ರ್ಷಿ-ಗಳು
ಮಹ-ಲಿನ
ಮಹಾ
ಮಹಾ-ಕ-ಥೆ-ಯನ್ನು
ಮಹಾ-ಕವಿ
ಮಹಾ-ಕಾ-ರು-ಣಿಕ
ಮಹಾ-ಕಾರ್ಯ
ಮಹಾ-ಕಾ-ರ್ಯಕ್ಕೂ
ಮಹಾ-ಕಾ-ರ್ಯಕ್ಕೆ
ಮಹಾ-ಕಾ-ರ್ಯ-ಗಳನ್ನು
ಮಹಾ-ಕಾ-ರ್ಯ-ವನ್ನು
ಮಹಾ-ಕಾ-ರ್ಯ-ವನ್ನೂ
ಮಹಾ-ಕಾ-ರ್ಯ-ವೆಂ-ಬುದೇ
ಮಹಾ-ಕಾ-ರ್ಯ-ವೆ-ಸ-ಗ-ಬಲ್ಲ
ಮಹಾ-ಕಾ-ವ್ಯ-ಗಳನ್ನು
ಮಹಾ-ಕೇಂದ್ರ
ಮಹಾ-ಗು-ರು-ವಿ-ನಿಂದ
ಮಹಾ-ಜ್ಞಾನಿ
ಮಹಾ-ಜ್ಞಾ-ನಿಯೂ
ಮಹಾ-ತಾ-ಯಿಯ
ಮಹಾತ್ಮ
ಮಹಾ-ತ್ಮ-ನ-ನ್ನಾ-ಗಿ-ಸಿದ
ಮಹಾ-ತ್ಮ-ನನ್ನು
ಮಹಾ-ತ್ಮ-ನನ್ನೇ
ಮಹಾ-ತ್ಮನೆ
ಮಹಾ-ತ್ಮನೇ
ಮಹಾ-ತ್ಮರ
ಮಹಾ-ತ್ಮರು
ಮಹಾ-ತ್ಮರೇ
ಮಹಾ-ತ್ಮ-ವಾ-ಸ-ವಾ-ಗಿದ್ದ
ಮಹಾತ್ಮಾ
ಮಹಾ-ತ್ಯಾಗ
ಮಹಾ-ತ್ಯಾ-ಗಕ್ಕೆ
ಮಹಾ-ತ್ಯಾ-ಗದ
ಮಹಾ-ತ್ಯಾ-ಗ-ದಿಂದ
ಮಹಾ-ತ್ಯಾಗಿ
ಮಹಾ-ತ್ವಾ-ಕಾಂ-ಕ್ಷೆ-ಯೆಂ-ದರೆ
ಮಹಾ-ದ-ರ್ಶ-ಗಳನ್ನು
ಮಹಾ-ದೇವ
ಮಹಾ-ದೇ-ವನ
ಮಹಾ-ದ್ಭು-ತ-ವಾ-ಗಿತ್ತು
ಮಹಾ-ಧ್ಯ-ಕ್ಷ-ರಿಗೆ
ಮಹಾ-ನಂದ
ಮಹಾ-ನ-ಗರ
ಮಹಾ-ನಿ-ರ್ಣ-ಯ-ವೇನೆಂದರೆ
ಮಹಾ-ನು-ಭಾ-ವರೇ
ಮಹಾನ್
ಮಹಾ-ಪಂ-ಡಿತ
ಮಹಾ-ಪೀಠ
ಮಹಾ-ಪು-ರುಷ
ಮಹಾ-ಪು-ರು-ಷನ
ಮಹಾ-ಪು-ರು-ಷರ
ಮಹಾ-ಪು-ರು-ಷ-ರನ್ನು
ಮಹಾ-ಪು-ರು-ಷ-ರಲ್ಲಿ
ಮಹಾ-ಪು-ರು-ಷ-ರೆಂದು
ಮಹಾ-ಪು-ರು-ಷರೇ
ಮಹಾ-ಪೂರ
ಮಹಾ-ಪೂ-ರ-ವನ್ನೇ
ಮಹಾ-ಪೂ-ರ-ವಾಗಿ
ಮಹಾ-ಪ್ರ-ಚಂ-ಡ-ಆ-ದರೆ
ಮಹಾ-ಪ್ರ-ಭವೂ
ಮಹಾ-ಪ್ರ-ಭು-ವಿನ
ಮಹಾ-ಪ್ರ-ಸಾ-ದ-ವೆಂದು
ಮಹಾ-ಭ-ಕ್ತ-ನಾದ
ಮಹಾ-ಭ-ಕ್ತನೇ
ಮಹಾ-ಭ-ಕ್ತ-ರೆಂದು
ಮಹಾ-ಭಾ-ರ-ತ-ದಲ್ಲಿ
ಮಹಾ-ಭಾ-ರ-ತ-ವನ್ನು
ಮಹಾ-ಭಾವ
ಮಹಾ-ಮ-ಹಿ-ಮನ
ಮಹಾ-ಮಾ-ತೆಯ
ಮಹಾ-ಮಾಯೆ
ಮಹಾ-ಮಾ-ಲೆಯ
ಮಹಾ-ಮುನಿ
ಮಹಾ-ಮೇಳ
ಮಹಾ-ಮೌನ
ಮಹಾ-ರಾ-ಜನ
ಮಹಾ-ರಾ-ಜನೂ
ಮಹಾ-ರಾ-ಜನೇ
ಮಹಾ-ರಾ-ಜರ
ಮಹಾ-ರಾಯ
ಮಹಾ-ರ್ಕಾ-ಯ-ವಿದೆ
ಮಹಾ-ವೀ-ರ-ನಂತೆ
ಮಹಾ-ವ್ಯ-ಥೆ-ಯಾಗಿ
ಮಹಾ-ಶಕ್ತಿ
ಮಹಾ-ಶ-ಕ್ತಿಯ
ಮಹಾ-ಶ-ಕ್ತಿ-ಯೊಂ-ದಿ-ಗಿನ
ಮಹಾ-ಶ-ತ್ರು-ಗಳನ್ನು
ಮಹಾ-ಶಯ
ಮಹಾ-ಶ-ಯರೆ
ಮಹಾ-ಶ-ಯರೇ
ಮಹಾ-ಶಿ-ಷ್ಯ-ನಿ-ರ್ಮಾಣ
ಮಹಾ-ಸಂ-ಘದ
ಮಹಾ-ಸಂ-ಘವೂ
ಮಹಾ-ಸಂ-ತ-ರೆಂದು
ಮಹಾ-ಸಂ-ಪ್ರ-ದಾ-ಯ-ಸ್ಥರು
ಮಹಾ-ಸಂಸ್ಥೆ
ಮಹಾ-ಸ-ತ್ಯ-ವನ್ನು
ಮಹಾ-ಸ-ಮಾ-ಧಿಗೆ
ಮಹಾ-ಸ-ಮಾ-ಧಿ-ಯಾ-ಗಿತ್ತು
ಮಹಾ-ಸ-ರೋ-ವರ
ಮಹಾ-ಸ-ರೋ-ವ-ರದ
ಮಹಾ-ಸಾ-ಮ್ರಾ-ಜ್ಯದ
ಮಹಾ-ಸಿಂಹ
ಮಹಾ-ಸಿಂ-ಹದ
ಮಹಾ-ಸಿಂ-ಹ-ದಂ-ತಿದ್ದ
ಮಹಾ-ಸಿಂ-ಹ-ದೆ-ಡೆಗೆ
ಮಹಿಮ
ಮಹಿಮಾ
ಮಹಿ-ಮಾ-ವಂ-ತ-ರ-ನ್ನಾಗಿ
ಮಹಿಮೆ
ಮಹಿ-ಮೆ-ಮ-ಹ-ತ್ತು-ಗಳನ್ನು
ಮಹಿ-ಮೆ-ಗಳನ್ನು
ಮಹಿ-ಮೆಯ
ಮಹಿ-ಮೆ-ಯನ್ನು
ಮಹಿ-ಮೆ-ಯಲ್ಲಿ
ಮಹಿಳೆ
ಮಹಿ-ಳೆ-ಯರ
ಮಹಿ-ಳೆ-ಯ-ರಿಗೆ
ಮಹಿ-ಳೆ-ಯರು
ಮಹಿ-ಳೆ-ಯ-ರೆಲ್ಲ
ಮಹೀಂದ್ರ
ಮಹೇಂದ್ರ
ಮಹೇಂ-ದ್ರ-ನ-ನ್ನು-ದ್ದೇ-ಶಿಸಿ
ಮಹೇಂ-ದ್ರ-ನಾತ
ಮಹೇಂ-ದ್ರ-ನಾಥ
ಮಹೇಂ-ದ್ರ-ನಾ-ಥ-ಇಷ್ಟು
ಮಹೇಂ-ದ್ರ-ನಾ-ಥನ
ಮಹೇಂ-ದ್ರ-ನಾ-ಥ-ನಿಗೆ
ಮಹೇಂ-ದ್ರ-ನಿಗೆ
ಮಹೇಂ-ದ್ರ-ಲಾಲ
ಮಹೇಂ-ದ್ರ-ಲಾಲ್
ಮಹೇ-ಶ್ವ-ರನೇ
ಮಹೋ
ಮಹೋ-ದ್ದೇಶ
ಮಾ
ಮಾ
ಮಾsಹಂ
ಮಾಂಸ
ಮಾಂಸಕ್ಕೆ
ಮಾಂಸ-ಗಳು
ಮಾಂಸ-ದಿಂದ
ಮಾಂಸ-ಲ-ವಾದ
ಮಾಂಸಾ-ಹಾರ
ಮಾಂಸಾ-ಹಾ-ರ-ವನ್ನು
ಮಾಡ
ಮಾಡ-ತೊ-ಡಗಿ
ಮಾಡ-ತೊ-ಡ-ಗಿದ
ಮಾಡ-ತೊ-ಡ-ಗಿ-ದರು
ಮಾಡತ್ತೆ
ಮಾಡ-ದಿ-ರಲಿ
ಮಾಡದೆ
ಮಾಡ-ದೆಯೇ
ಮಾಡ-ಬಲ್ಲ
ಮಾಡ-ಬ-ಲ್ಲದೋ
ಮಾಡ-ಬ-ಲ್ಲರು
ಮಾಡ-ಬ-ಹು-ದಾಗಿ
ಮಾಡ-ಬ-ಹು-ದಾ-ಗಿದ್ದ
ಮಾಡ-ಬ-ಹುದು
ಮಾಡ-ಬಾ-ರ-ದೆಂದು
ಮಾಡ-ಬೇ-ಕಲ್ಲ
ಮಾಡ-ಬೇ-ಕ-ಲ್ಲವೆ
ಮಾಡ-ಬೇಕಾ
ಮಾಡ-ಬೇ-ಕಾಗಿ
ಮಾಡ-ಬೇ-ಕಾ-ಗಿತ್ತು
ಮಾಡ-ಬೇ-ಕಾ-ಗಿದೆ
ಮಾಡ-ಬೇ-ಕಾ-ಗಿ-ಬಂ-ದರೆ
ಮಾಡ-ಬೇ-ಕಾ-ಗಿಯೇ
ಮಾಡ-ಬೇ-ಕಾ-ಗಿ-ರುವ
ಮಾಡ-ಬೇ-ಕಾ-ಗಿಲ್ಲ
ಮಾಡ-ಬೇ-ಕಾ-ಗು-ತ್ತದೆ
ಮಾಡ-ಬೇ-ಕಾದ
ಮಾಡ-ಬೇ-ಕಾ-ದ-ದ್ದಿದೆ
ಮಾಡ-ಬೇ-ಕಾ-ದದ್ದು
ಮಾಡ-ಬೇ-ಕಾ-ದರೂ
ಮಾಡ-ಬೇ-ಕಾ-ದರೆ
ಮಾಡ-ಬೇ-ಕಾ-ದ-ವ-ರಲ್ಲಿ
ಮಾಡ-ಬೇ-ಕಾ-ಯಿತು
ಮಾಡ-ಬೇ-ಕಿಲ್ಲ
ಮಾಡ-ಬೇಕು
ಮಾಡ-ಬೇ-ಕೆಂದು
ಮಾಡ-ಬೇ-ಕೆಂಬ
ಮಾಡ-ಬೇ-ಕೆಂ-ಬ-ವನು
ಮಾಡ-ಬೇ-ಕೆಂ-ಬು-ದರ
ಮಾಡ-ಬೇಕೇ
ಮಾಡ-ಬೇಡ
ಮಾಡ-ಬೇಡಿ
ಮಾಡ-ಲಪ್ಪ
ಮಾಡ-ಲಾಗಿದೆ
ಮಾಡ-ಲಾ-ದೀತು
ಮಾಡ-ಲಾರ
ಮಾಡ-ಲಾ-ರಂ-ಭಿಸಿ
ಮಾಡ-ಲಾ-ರಂ-ಭಿ-ಸಿದ
ಮಾಡ-ಲಾ-ರಂ-ಭಿ-ಸಿದೆ
ಮಾಡ-ಲಾ-ರಂ-ಭಿ-ಸು-ತ್ತಾನೆ
ಮಾಡ-ಲಾ-ರದು
ಮಾಡ-ಲಾರೆ
ಮಾಡ-ಲಾ-ರೆವು
ಮಾಡಲಿ
ಮಾಡ-ಲಿದೆ
ಮಾಡ-ಲಿ-ರು-ವ-ವನು
ಮಾಡ-ಲಿಲ್ಲ
ಮಾಡಲು
ಮಾಡಲೂ
ಮಾಡಲೆ
ಮಾಡ-ಲೇ-ಬೇ-ಕಾ-ಗಿದೆ
ಮಾಡ-ಲೇ-ಬೇ-ಕಾ-ಗು-ತ್ತದೆ
ಮಾಡ-ಲೇ-ಬೇಕು
ಮಾಡ-ಲೊ-ಲ್ಲಿರಿ
ಮಾಡ-ಲ್ಪ-ಟ್ಟಂತೆ
ಮಾಡಿ
ಮಾಡಿ-ಕೊಂಡ
ಮಾಡಿ-ಕೊಂ-ಡ-ದ್ದ-ರಿಂದ
ಮಾಡಿ-ಕೊಂ-ಡದ್ದು
ಮಾಡಿ-ಕೊಂ-ಡ-ಮೇಲೆ
ಮಾಡಿ-ಕೊಂ-ಡರು
ಮಾಡಿ-ಕೊಂ-ಡಳು
ಮಾಡಿ-ಕೊಂ-ಡ-ವರು
ಮಾಡಿ-ಕೊಂ-ಡ-ವರೆಲ್ಲ
ಮಾಡಿ-ಕೊಂಡಿ
ಮಾಡಿ-ಕೊಂ-ಡಿದ್ದ
ಮಾಡಿ-ಕೊಂ-ಡಿ-ದ್ದರು
ಮಾಡಿ-ಕೊಂ-ಡಿ-ದ್ದಳು
ಮಾಡಿ-ಕೊಂ-ಡಿ-ರುವ
ಮಾಡಿ-ಕೊಂ-ಡಿ-ರು-ವಂ-ತಾ-ಗ-ಬೇಕು
ಮಾಡಿ-ಕೊಂಡು
ಮಾಡಿ-ಕೊಂ-ಡು-ಬಿಟ್ಟ
ಮಾಡಿ-ಕೊಂ-ಡು-ಬಿ-ಟ್ಟರು
ಮಾಡಿ-ಕೊಂ-ಡು-ಬಿ-ಟ್ಟರೆ
ಮಾಡಿ-ಕೊಂ-ಡು-ಬಿ-ಡ-ಬೇಕು
ಮಾಡಿ-ಕೊಂ-ಡು-ಬಿ-ಡು-ತ್ತಾನೋ
ಮಾಡಿ-ಕೊಂಡೆ
ಮಾಡಿ-ಕೊಟ್ಟ
ಮಾಡಿ-ಕೊ-ಟ್ಟರೆ
ಮಾಡಿ-ಕೊ-ಡ-ಬ-ಲ್ಲರು
ಮಾಡಿ-ಕೊ-ಡ-ಬೇಕು
ಮಾಡಿ-ಕೊಡು
ಮಾಡಿ-ಕೊ-ಡುವ
ಮಾಡಿ-ಕೊಳ್ಳ
ಮಾಡಿ-ಕೊ-ಳ್ಳ-ಬ-ಲ್ಲರು
ಮಾಡಿ-ಕೊ-ಳ್ಳ-ಬ-ಲ್ಲುದೇ
ಮಾಡಿ-ಕೊ-ಳ್ಳ-ಬ-ಹು-ದಾ-ಗಿತ್ತು
ಮಾಡಿ-ಕೊ-ಳ್ಳ-ಬ-ಹುದು
ಮಾಡಿ-ಕೊ-ಳ್ಳ-ಬಾ-ರದು
ಮಾಡಿ-ಕೊ-ಳ್ಳ-ಬೇ-ಕಾದ
ಮಾಡಿ-ಕೊ-ಳ್ಳ-ಬೇ-ಕಾ-ದರೂ
ಮಾಡಿ-ಕೊ-ಳ್ಳ-ಬೇ-ಕಾ-ದರೆ
ಮಾಡಿ-ಕೊ-ಳ್ಳ-ಬೇ-ಕಾ-ಯಿ-ತಂತೆ
ಮಾಡಿ-ಕೊ-ಳ್ಳ-ಬೇ-ಕೆಂಬ
ಮಾಡಿ-ಕೊ-ಳ್ಳ-ಬೇ-ಕೆಂ-ಬ-ವ-ರಲ್ಲೇ
ಮಾಡಿ-ಕೊ-ಳ್ಳಲಿ
ಮಾಡಿ-ಕೊ-ಳ್ಳಲು
ಮಾಡಿ-ಕೊ-ಳ್ಳಲೇ
ಮಾಡಿ-ಕೊ-ಳ್ಳ-ಲೇ-ಬೇಕು
ಮಾಡಿ-ಕೊಳ್ಳಿ
ಮಾಡಿ-ಕೊ-ಳ್ಳು-ತ್ತಾನೆ
ಮಾಡಿ-ಕೊ-ಳ್ಳು-ತ್ತಾರೆ
ಮಾಡಿ-ಕೊ-ಳ್ಳು-ತ್ತಿದ್ದ
ಮಾಡಿ-ಕೊ-ಳ್ಳು-ತ್ತಿ-ದ್ದರು
ಮಾಡಿ-ಕೊ-ಳ್ಳು-ತ್ತಿ-ದ್ದಾರೆ
ಮಾಡಿ-ಕೊ-ಳ್ಳು-ತ್ತೇನೆ
ಮಾಡಿ-ಕೊ-ಳ್ಳುವ
ಮಾಡಿ-ಕೊ-ಳ್ಳು-ವಂ-ತಾಗ
ಮಾಡಿ-ಕೊ-ಳ್ಳು-ವಂತೆ
ಮಾಡಿ-ಕೊ-ಳ್ಳು-ವು-ದ-ಕ್ಕಿಂತ
ಮಾಡಿ-ಕೊ-ಳ್ಳು-ವು-ದರ
ಮಾಡಿ-ಕೊ-ಳ್ಳು-ವುದು
ಮಾಡಿ-ಕೊ-ಳ್ಳು-ವುದೇ
ಮಾಡಿ-ಕೊ-ಳ್ಳೋಣ
ಮಾಡಿ-ಟ್ಟಿ-ದ್ದರು
ಮಾಡಿಟ್ಟು
ಮಾಡಿ-ಟ್ಟು-ಹೋ-ಗು-ವು-ದಲ್ಲ
ಮಾಡಿತು
ಮಾಡಿದ
ಮಾಡಿ-ದಂ-ತಹ
ಮಾಡಿ-ದಂ-ತಾ-ಗು-ತ್ತದೆ
ಮಾಡಿ-ದಂ-ತಿತ್ತು
ಮಾಡಿ-ದಂ-ತಿದೆ
ಮಾಡಿ-ದಂತೆ
ಮಾಡಿ-ದ-ಮೇ-ಲೆಯೇ
ಮಾಡಿ-ದ-ರಲ್ಲ
ಮಾಡಿ-ದ-ರ-ಲ್ಲದೆ
ಮಾಡಿ-ದರು
ಮಾಡಿ-ದರೂ
ಮಾಡಿ-ದರೆ
ಮಾಡಿ-ದಳು
ಮಾಡಿ-ದ-ವ-ನಂತೆ
ಮಾಡಿ-ದ-ವನು
ಮಾಡಿ-ದ-ವ-ರನ್ನು
ಮಾಡಿ-ದ-ವ-ರಲ್ಲ
ಮಾಡಿ-ದ-ವ-ರಾ-ದ್ದ-ರಿಂ-ದ-ಅ-ವರ
ಮಾಡಿ-ದ-ವರು
ಮಾಡಿ-ದಷ್ಟೂ
ಮಾಡಿ-ದಾಗ
ಮಾಡಿ-ದಾ-ಗಲೂ
ಮಾಡಿ-ದಾ-ಗ-ಲೆಲ್ಲ
ಮಾಡಿ-ದಾ-ಗಲೇ
ಮಾಡಿ-ದಿರಿ
ಮಾಡಿದೆ
ಮಾಡಿ-ದೊ-ಡ-ನೆಯೇ
ಮಾಡಿದ್ದ
ಮಾಡಿ-ದ್ದ-ಕ್ಕಾಗಿ
ಮಾಡಿ-ದ್ದ-ನ್ನೆಲ್ಲ
ಮಾಡಿ-ದ್ದ-ರಾ-ದರೂ
ಮಾಡಿ-ದ್ದ-ರಿಂದ
ಮಾಡಿ-ದ್ದರು
ಮಾಡಿ-ದ್ದರೂ
ಮಾಡಿ-ದ್ದರೆ
ಮಾಡಿ-ದ್ದ-ಲ್ಲದೆ
ಮಾಡಿ-ದ್ದ-ವನು
ಮಾಡಿ-ದ್ದಾ-ಗಿದೆ
ಮಾಡಿ-ದ್ದಾ-ದರೂ
ಮಾಡಿ-ದ್ದಾ-ರಲ್ಲ
ಮಾಡಿ-ದ್ದಾರೆ
ಮಾಡಿ-ದ್ದಾರೋ
ಮಾಡಿ-ದ್ದಿರ
ಮಾಡಿದ್ದೀ
ಮಾಡಿದ್ದು
ಮಾಡಿ-ದ್ದುವು
ಮಾಡಿದ್ದೂ
ಮಾಡಿದ್ದೇ
ಮಾಡಿ-ದ್ದೇ-ನೆಂ-ದರೆ
ಮಾಡಿ-ನೋಡಿ
ಮಾಡಿ-ನೋಡು
ಮಾಡಿ-ಬಿಟ್ಟ
ಮಾಡಿ-ಬಿ-ಟ್ಟರು
ಮಾಡಿ-ಬಿ-ಟ್ಟರೂ
ಮಾಡಿ-ಬಿಟ್ಟಿ
ಮಾಡಿ-ಬಿ-ಟ್ಟಿ-ತು-ನ-ರೇಂ-ದ್ರನ
ಮಾಡಿ-ಬಿ-ಟ್ಟಿತ್ತು
ಮಾಡಿ-ಬಿ-ಟ್ಟಿದ್ದ
ಮಾಡಿ-ಬಿ-ಟ್ಟಿ-ದ್ದಳು
ಮಾಡಿ-ಬಿ-ಟ್ಟಿ-ದ್ದಾರೆ
ಮಾಡಿ-ಬಿ-ಟ್ಟಿ-ದ್ದೇ-ನೆ-ಇ-ವ-ರಿಗೆ
ಮಾಡಿ-ಬಿಟ್ಟೆ
ಮಾಡಿ-ಬಿ-ಟ್ಟೆವು
ಮಾಡಿ-ಬಿ-ಡ-ಬೇಕು
ಮಾಡಿ-ಬಿ-ಡಲು
ಮಾಡಿ-ಬಿಡಿ
ಮಾಡಿ-ಬಿಡು
ಮಾಡಿ-ಬಿ-ಡು-ತ್ತಾನೆ
ಮಾಡಿ-ಬಿ-ಡು-ತ್ತಿದ್ದ
ಮಾಡಿ-ಬಿ-ಡು-ತ್ತೇನೆ
ಮಾಡಿ-ಬಿ-ಡು-ತ್ತೇವೆ
ಮಾಡಿ-ಬಿ-ಡುವ
ಮಾಡಿ-ಮಾಡಿ
ಮಾಡಿ-ಯಾ-ಗಿತ್ತು
ಮಾಡಿ-ಯಾರು
ಮಾಡಿಯೂ
ಮಾಡಿಯೇ
ಮಾಡಿ-ರ-ಬೇಕು
ಮಾಡಿ-ರ-ಲಿಲ್ಲ
ಮಾಡಿ-ರುವ
ಮಾಡಿ-ರು-ವುದನ್ನು
ಮಾಡಿಲ್ಲ
ಮಾಡಿ-ಲ್ಲವೆ
ಮಾಡಿ-ಲ್ಲ-ವೆಂದು
ಮಾಡಿ-ಲ್ಲವೋ
ಮಾಡಿವೆ
ಮಾಡಿ-ಸ-ಬೇಕು
ಮಾಡಿ-ಸ-ಬೇ-ಕೆಂಬ
ಮಾಡಿ-ಸಲು
ಮಾಡಿಸಿ
ಮಾಡಿ-ಸಿ-ಕೊಟ್ಟ
ಮಾಡಿ-ಸಿ-ಕೊ-ಟ್ಟಿ-ದ್ದರೂ
ಮಾಡಿ-ಸಿ-ಕೊ-ಡ-ಬ-ಲ್ಲ-ವ-ರನ್ನು
ಮಾಡಿ-ಸಿ-ಕೊ-ಡ-ಬೇ-ಕಾದ
ಮಾಡಿ-ಸಿ-ಕೊಡಿ
ಮಾಡಿ-ಸಿ-ಕೊ-ಡು-ವು-ದಾಗಿ
ಮಾಡಿ-ಸಿ-ಕೊ-ಳ್ಳಲು
ಮಾಡಿ-ಸಿ-ಕೊಳ್ಳಿ
ಮಾಡಿ-ಸಿದ
ಮಾಡಿ-ಸಿ-ದರು
ಮಾಡಿ-ಸಿ-ದಳು
ಮಾಡಿ-ಸಿ-ದ್ದಳು
ಮಾಡಿ-ಸು-ತ್ತಾನೆ
ಮಾಡಿ-ಸು-ತ್ತಿದ್ದ
ಮಾಡಿ-ಸು-ತ್ತಿ-ದ್ದರು
ಮಾಡಿ-ಸು-ತ್ತಿ-ದ್ದಾರೆ
ಮಾಡಿ-ಸುವ
ಮಾಡಿ-ಸು-ವುದು
ಮಾಡಿ-ಹೋ-ಗಿ-ದ್ದಾರೆ
ಮಾಡೀತು
ಮಾಡು
ಮಾಡು-ಅಂತ
ಮಾಡುತ್ತ
ಮಾಡು-ತ್ತ-ಕು-ಳಿ-ತಿ-ರು-ವುದನ್ನು
ಮಾಡು-ತ್ತದೆ
ಮಾಡು-ತ್ತ-ಮಾ-ಡುತ್ತ
ಮಾಡು-ತ್ತಲೇ
ಮಾಡು-ತ್ತವೆ
ಮಾಡು-ತ್ತ-ಹೋಗು
ಮಾಡು-ತ್ತಾನೆ
ಮಾಡು-ತ್ತಾ-ನೆ-ಎಲ್ಲ
ಮಾಡು-ತ್ತಾನೋ
ಮಾಡು-ತ್ತಾರೆ
ಮಾಡು-ತ್ತಾ-ರೆಂದು
ಮಾಡು-ತ್ತಾರೋ
ಮಾಡುತ್ತಿ
ಮಾಡು-ತ್ತಿತ್ತು
ಮಾಡು-ತ್ತಿ-ದೆಯೋ
ಮಾಡು-ತ್ತಿದ್ದ
ಮಾಡು-ತ್ತಿ-ದ್ದಂ-ತೆಯೇ
ಮಾಡು-ತ್ತಿ-ದ್ದನೇ
ಮಾಡು-ತ್ತಿ-ದ್ದರು
ಮಾಡು-ತ್ತಿ-ದ್ದರೂ
ಮಾಡು-ತ್ತಿ-ದ್ದರೆ
ಮಾಡು-ತ್ತಿ-ದ್ದಳು
ಮಾಡು-ತ್ತಿ-ದ್ದಾಗ
ಮಾಡು-ತ್ತಿ-ದ್ದಾ-ನಲ್ಲ
ಮಾಡು-ತ್ತಿ-ದ್ದಾನೆ
ಮಾಡು-ತ್ತಿ-ದ್ದಾ-ರಲ್ಲ
ಮಾಡು-ತ್ತಿ-ದ್ದಾರೆ
ಮಾಡು-ತ್ತಿ-ದ್ದಾರೋ
ಮಾಡು-ತ್ತಿದ್ದೀ
ಮಾಡು-ತ್ತಿ-ದ್ದೀಯ
ಮಾಡು-ತ್ತಿ-ದ್ದೀಯಾ
ಮಾಡು-ತ್ತಿ-ದ್ದೀರಿ
ಮಾಡು-ತ್ತಿ-ದ್ದು-ದ-ಕ್ಕಾಗಿ
ಮಾಡು-ತ್ತಿ-ದ್ದುದು
ಮಾಡು-ತ್ತಿ-ದ್ದುವು
ಮಾಡು-ತ್ತಿದ್ದೆ
ಮಾಡು-ತ್ತಿ-ದ್ದೆವು
ಮಾಡು-ತ್ತಿ-ದ್ದೇವೆ
ಮಾಡು-ತ್ತಿ-ರ-ಬೇಕು
ಮಾಡು-ತ್ತಿ-ರ-ಲಿಲ್ಲ
ಮಾಡು-ತ್ತಿ-ರಲೇ
ಮಾಡು-ತ್ತಿರು
ಮಾಡು-ತ್ತಿ-ರುವ
ಮಾಡು-ತ್ತಿ-ರು-ವಂತೆ
ಮಾಡು-ತ್ತಿ-ರು-ವಾಗ
ಮಾಡು-ತ್ತಿ-ರು-ವಾ-ಗಲೂ
ಮಾಡು-ತ್ತಿ-ರು-ವಾ-ಗಲೇ
ಮಾಡು-ತ್ತಿ-ರು-ವುದನ್ನು
ಮಾಡು-ತ್ತಿ-ರು-ವು-ದ-ರಿಂದ
ಮಾಡು-ತ್ತಿ-ರು-ವುದು
ಮಾಡು-ತ್ತಿ-ವೆ-ಯಲ್ಲ
ಮಾಡು-ತ್ತೀಯ
ಮಾಡು-ತ್ತೇನೆ
ಮಾಡು-ತ್ತೇ-ವೆಂದು
ಮಾಡು-ದು-ವು-ದೆಂ-ದರೆ
ಮಾಡುವ
ಮಾಡು-ವಂ-ತಾ-ಗ-ಬೇಕು
ಮಾಡು-ವಂ-ತಿ-ರ-ಲಿಲ್ಲ
ಮಾಡು-ವಂತೆ
ಮಾಡು-ವ-ವ-ರಂತೆ
ಮಾಡು-ವ-ವ-ರನ್ನು
ಮಾಡು-ವ-ವ-ರಲ್ಲಿ
ಮಾಡು-ವ-ವ-ರಿಗೆ
ಮಾಡು-ವ-ವರು
ಮಾಡು-ವಾಗ
ಮಾಡು-ವಾ-ಗಲೂ
ಮಾಡುವು
ಮಾಡು-ವುದ
ಮಾಡು-ವು-ದ-ಕ್ಕಾಗಿ
ಮಾಡು-ವು-ದ-ಕ್ಕಾ-ಗಿಯೇ
ಮಾಡು-ವು-ದ-ಕ್ಕಿಂ-ತಲೂ
ಮಾಡು-ವು-ದಕ್ಕೂ
ಮಾಡು-ವು-ದಕ್ಕೆ
ಮಾಡು-ವುದನ್ನು
ಮಾಡು-ವು-ದರ
ಮಾಡು-ವು-ದ-ರಿಂದ
ಮಾಡು-ವು-ದ-ಲ್ಲದೆ
ಮಾಡು-ವು-ದಾ-ದರೂ
ಮಾಡು-ವು-ದಿತ್ತು
ಮಾಡು-ವು-ದಿ-ರಲಿ
ಮಾಡು-ವು-ದಿಲ್ಲ
ಮಾಡು-ವು-ದಿ-ಲ್ಲವೆ
ಮಾಡು-ವು-ದೀಗ
ಮಾಡು-ವುದು
ಮಾಡು-ವು-ದು-ಇದೇ
ಮಾಡು-ವುದೂ
ಮಾಡು-ವು-ದೆಂ-ದರೆ
ಮಾಡು-ವು-ದೆಂದು
ಮಾಡು-ವು-ದೆಲ್ಲ
ಮಾಡು-ವುದೇ
ಮಾಡುವೆ
ಮಾಡೋಣ
ಮಾಡೋ-ದಿ-ಲ್ಲವೆ
ಮಾತ
ಮಾತಂ-ಗಿನೀ
ಮಾತಂ-ಗಿ-ನೀ-ದೇವಿ
ಮಾತಂತೂ
ಮಾತನಾ
ಮಾತ-ನಾ-ಡ-ದಂತೆ
ಮಾತ-ನಾ-ಡ-ದಿ-ದ್ದರೆ
ಮಾತ-ನಾ-ಡ-ಬ-ಯಸಿ
ಮಾತ-ನಾ-ಡ-ಬಲ್ಲ
ಮಾತ-ನಾ-ಡ-ಬ-ಲ್ಲರು
ಮಾತ-ನಾ-ಡ-ಬಲ್ಲೆ
ಮಾತ-ನಾ-ಡ-ಬ-ಹುದು
ಮಾತ-ನಾ-ಡ-ಬೇಕು
ಮಾತ-ನಾ-ಡ-ಬೇಡ
ಮಾತ-ನಾ-ಡ-ಲಾ-ರಂ-ಭಿಸಿ
ಮಾತ-ನಾ-ಡ-ಲಾ-ರಂ-ಭಿ-ಸಿದ
ಮಾತ-ನಾ-ಡ-ಲಾ-ರಂ-ಭಿ-ಸಿ-ದರು
ಮಾತ-ನಾ-ಡ-ಲಿಲ್ಲ
ಮಾತ-ನಾ-ಡಲು
ಮಾತ-ನಾ-ಡಲೂ
ಮಾತ-ನಾ-ಡಲೇ
ಮಾತ-ನಾಡಿ
ಮಾತ-ನಾ-ಡಿ-ಕೊ-ಳ್ಳಲು
ಮಾತ-ನಾ-ಡಿ-ಕೊ-ಳ್ಳು-ತ್ತಿ-ದ್ದಾ-ರಲ್ಲ
ಮಾತ-ನಾ-ಡಿ-ಕೊ-ಳ್ಳು-ತ್ತಿ-ರು-ವಾಗ
ಮಾತ-ನಾ-ಡಿ-ಕೊ-ಳ್ಳು-ವುದೂ
ಮಾತ-ನಾ-ಡಿದ
ಮಾತ-ನಾ-ಡಿ-ದರೂ
ಮಾತ-ನಾ-ಡಿ-ದರೆ
ಮಾತ-ನಾ-ಡಿ-ದಾಗ
ಮಾತ-ನಾ-ಡಿದ್ದ
ಮಾತ-ನಾ-ಡಿದ್ದು
ಮಾತ-ನಾ-ಡಿ-ಯಾನು
ಮಾತ-ನಾ-ಡಿ-ಸಲಿ
ಮಾತ-ನಾ-ಡಿ-ಸಲೇ
ಮಾತ-ನಾ-ಡಿಸಿ
ಮಾತ-ನಾ-ಡಿ-ಸಿದ
ಮಾತ-ನಾ-ಡಿ-ಸಿ-ದರು
ಮಾತ-ನಾ-ಡಿ-ಸುತ್ತ
ಮಾತ-ನಾ-ಡಿ-ಸು-ತ್ತಿದ್ದ
ಮಾತ-ನಾ-ಡಿ-ಸು-ತ್ತಿ-ದ್ದರು
ಮಾತ-ನಾಡು
ಮಾತ-ನಾ-ಡುತ್ತ
ಮಾತ-ನಾ-ಡು-ತ್ತಲೇ
ಮಾತ-ನಾ-ಡು-ತ್ತಿ-ದೆಯೋ
ಮಾತ-ನಾ-ಡು-ತ್ತಿದ್ದ
ಮಾತ-ನಾ-ಡು-ತ್ತಿ-ದ್ದಂತೆ
ಮಾತ-ನಾ-ಡು-ತ್ತಿ-ದ್ದರು
ಮಾತ-ನಾ-ಡು-ತ್ತಿ-ದ್ದ-ರು-ಬ-ಹುಶಃ
ಮಾತ-ನಾ-ಡು-ತ್ತಿ-ದ್ದರೆ
ಮಾತ-ನಾ-ಡು-ತ್ತಿ-ದ್ದಾರೆ
ಮಾತ-ನಾ-ಡು-ತ್ತಿ-ದ್ದೀರೋ
ಮಾತ-ನಾ-ಡು-ತ್ತಿದ್ದು
ಮಾತ-ನಾ-ಡು-ತ್ತಿ-ರ-ಲಿಲ್ಲ
ಮಾತ-ನಾ-ಡು-ತ್ತಿ-ರು-ತ್ತಾನೆ
ಮಾತ-ನಾ-ಡು-ತ್ತಿ-ರು-ವಾಗ
ಮಾತ-ನಾ-ಡು-ತ್ತೀಯೆ
ಮಾತ-ನಾ-ಡುವ
ಮಾತ-ನಾ-ಡು-ವಂ-ತಾ-ಗ-ಬೇಕು
ಮಾತ-ನಾ-ಡು-ವ-ವರು
ಮಾತ-ನಾ-ಡು-ವಷ್ಟೇ
ಮಾತ-ನಾ-ಡು-ವಾಗ
ಮಾತ-ನಾ-ಡು-ವು-ದ-ಕ್ಕಿಂತ
ಮಾತ-ನಾ-ಡು-ವು-ದಕ್ಕೆ
ಮಾತ-ನಾ-ಡು-ವು-ದ-ನ್ನಾ-ಗಲಿ
ಮಾತ-ನಾ-ಡು-ವು-ದ-ರಲ್ಲಿ
ಮಾತ-ನಾ-ಡು-ವುದು
ಮಾತ-ನಾ-ಡು-ವುದೂ
ಮಾತ-ನಾ-ಡು-ವುದೇ
ಮಾತನ್ನು
ಮಾತನ್ನೂ
ಮಾತ-ನ್ನೆತ್ತಿ
ಮಾತ-ನ್ನೆಲ್ಲ
ಮಾತನ್ನೇ
ಮಾತ-ನ್ನೇನೂ
ಮಾತಲ್ಲ
ಮಾತಾ-ಗು-ತ್ತದೆ
ಮಾತಾ-ಡ-ಲಿಲ್ಲ
ಮಾತಾಡಿ
ಮಾತಾ-ಡಿ-ಕೊಂ-ಡರು
ಮಾತಾ-ಡಿಸಿ
ಮಾತಾ-ಡು-ತ್ತಿ-ದ್ದರೂ
ಮಾತಾ-ಡು-ತ್ತಿ-ದ್ದೀಯೆ
ಮಾತಿಗೂ
ಮಾತಿಗೆ
ಮಾತಿಗೇ
ಮಾತಿದು
ಮಾತಿದೆ
ಮಾತಿ-ದೆ-ಬಹ್ತಾ
ಮಾತಿನ
ಮಾತಿ-ನಂತೆ
ಮಾತಿ-ನಲ್ಲಿ
ಮಾತಿ-ನಲ್ಲೂ
ಮಾತಿ-ನ-ವ-ಳಲ್ಲ
ಮಾತಿ-ನಿಂದ
ಮಾತಿ-ನಿಂ-ದಲೇ
ಮಾತಿ-ನಿಂ-ದಾ-ಗಲಿ
ಮಾತಿ-ನಿಂ-ದೆಲ್ಲ
ಮಾತಿ-ಲ್ಲದೆ
ಮಾತು
ಮಾತು-ಮಾ-ರ್ಗ-ದ-ರ್ಶ-ನ-ಗ-ಳಿ-ಗಿಂತ
ಮಾತುಆ
ಮಾತು-ಕತೆ
ಮಾತು-ಕ-ತೆ-ಗಳನ್ನು
ಮಾತು-ಕ-ತೆ-ಗಳಲ್ಲಿ
ಮಾತು-ಕ-ತೆ-ಗಳು
ಮಾತು-ಕ-ತೆ-ಯನ್ನು
ಮಾತು-ಕ-ತೆ-ಯಾ-ಡುತ್ತ
ಮಾತು-ಕ-ತೆ-ಯೇನೂ
ಮಾತು-ಕೊಡು
ಮಾತು-ಗ-ಲ-ನ್ನೆಲ್ಲ
ಮಾತು-ಗಳ
ಮಾತು-ಗ-ಳ-ನ್ನಾ-ಡಿ-ದರೂ
ಮಾತು-ಗ-ಳ-ನ್ನಾ-ಡಿ-ದರೆ
ಮಾತು-ಗಳನ್ನು
ಮಾತು-ಗಳನ್ನೂ
ಮಾತು-ಗಳನ್ನೆಲ್ಲ
ಮಾತು-ಗ-ಳನ್ನೇ
ಮಾತು-ಗಳಲ್ಲಿ
ಮಾತು-ಗ-ಳಾ-ವುವೂ
ಮಾತು-ಗಳಿಂದ
ಮಾತು-ಗ-ಳಿಂ-ದಾಗಿ
ಮಾತು-ಗ-ಳಿಗೂ
ಮಾತು-ಗಳು
ಮಾತು-ಗಳೂ
ಮಾತು-ಗ-ಳೆಲ್ಲ
ಮಾತು-ಗ-ಳೆ-ಲ್ಲವೂ
ಮಾತು-ಗ-ಳೇನು
ಮಾತು-ಮಾ-ತಿಗೂ
ಮಾತೂ
ಮಾತೃ
ಮಾತೃತ್ವ
ಮಾತೃ-ತ್ವ-ವನ್ನು
ಮಾತೃ-ಬಾಷೆ
ಮಾತೃ-ಭೂ-ಮಿಯ
ಮಾತೃ-ಭೂ-ಮಿ-ಯನ್ನು
ಮಾತೆ-ತ್ತಿ-ದರೆ
ಮಾತೆಯ
ಮಾತೆ-ಯರ
ಮಾತೆಯೇ
ಮಾತೆಲ್ಲ
ಮಾತೇ
ಮಾತೋ
ಮಾತ್ಮನ
ಮಾತ್ಮ-ನನ್ನು
ಮಾತ್ರ
ಮಾತ್ರಕ್ಕೆ
ಮಾತ್ರ-ದಿಂದ
ಮಾತ್ರ-ವಲ್ಲ
ಮಾತ್ರ-ವ-ಲ್ಲದೆ
ಮಾತ್ರವೆ
ಮಾತ್ರವೇ
ಮಾಧುರ್ಯ
ಮಾಧು-ರ್ಯ-ದಲ್ಲಿ
ಮಾಧು-ರ್ಯ-ವಿತ್ತು
ಮಾಧ್ಯ-ಮದ
ಮಾನ-ಕೀ-ರ್ತಿ-ಗ-ಳ-ಲ್ಲಿ-ತ-ನ-ಗಿಂತ
ಮಾನಕ್ಕೆ
ಮಾನ-ದಂಡ
ಮಾನವ
ಮಾನ-ವ-ಕೋ-ಟಿಯ
ಮಾನ-ವ-ಜ-ನ್ಮದ
ಮಾನ-ವ-ತೆ-ಯನ್ನು
ಮಾನ-ವ-ತ್ವ-ದೈ-ವ-ತ್ವ-ಗಳನ್ನು
ಮಾನ-ವನ
ಮಾನ-ವ-ನಿಗೆ
ಮಾನ-ವನು
ಮಾನ-ವ-ಪ್ರ-ಪಂ-ಚಕ್ಕೂ
ಮಾನ-ವ-ಮಾತ್ರ
ಮಾನ-ವ-ರೂ-ಪ-ಗಳಲ್ಲಿ
ಮಾನ-ವ-ರೂ-ಪ-ದಿಂದ
ಮಾನ-ವ-ಶ-ರೀ-ರ-ವನ್ನು
ಮಾನ-ವ-ಸ-ಹಜ
ಮಾನ-ವ-ಸ-ಹ-ಜ-ವಾ-ದ-ದ್ದೆಂದು
ಮಾನ-ವ-ಸೇ-ವೆಯ
ಮಾನ-ವೀಯ
ಮಾನ-ವೀ-ಯ-ವಾ-ಗಿ-ರು-ವುದನ್ನು
ಮಾನ-ವೀ-ಯ-ವಾದ
ಮಾನ-ಸಿಕ
ಮಾನ-ಸಿ-ಕ-ವಾಗಿ
ಮಾನುಷ
ಮಾನು-ಷ-ವಾದ
ಮಾನು-ಷ-ವೈ-ಭ-ವ-ಗಳ
ಮಾನ್ಯು
ಮಾಫಿ
ಮಾಮ
ಮಾಮನ
ಮಾಮಾ
ಮಾಯ
ಮಾಯ-ವಾಗಿ
ಮಾಯ-ವಾ-ಗಿತ್ತು
ಮಾಯ-ವಾ-ಗಿ-ಬಿ-ಟ್ಟಿ-ದ್ದಾನೆ
ಮಾಯ-ವಾ-ಗಿ-ಬಿ-ಟ್ಟಿ-ರು-ತ್ತಿ-ದ್ದುವು
ಮಾಯಾ
ಮಾಯಾ-ಜಿಂ-ಕೆ-ಯಂತೆ
ಮಾಯಾ-ಪಾ-ಶ-ಗಳನ್ನು
ಮಾಯಾ-ಪಾ-ಶ-ವ-ನನು
ಮಾಯಾ-ಪಾ-ಶ-ವನ್ನು
ಮಾಯಾ-ಬಂ-ಧ-ನ-ಗಳನ್ನೆಲ್ಲ
ಮಾಯಾ-ಲಾಂ-ದ್ರ-ದ-ಸ್ಲೈಡ್
ಮಾಯಾ-ಸಾ-ಗ-ರ-ವನ್ನು
ಮಾಯೆ
ಮಾಯೆಯ
ಮಾಯೆ-ಯನ್ನು
ಮಾಯೆಯು
ಮಾಯ್ಹಾ-ಡೊಂದು
ಮಾರಿ
ಮಾರುತ
ಮಾರು-ತ-ನೊಲು
ಮಾರು-ಹೋ-ಗಿತ್ತು
ಮಾರು-ಹೋ-ಗಿ-ಯಾ-ಗಿತ್ತು
ಮಾರು-ಹೋದ
ಮಾರ್ಗ
ಮಾರ್ಗಕ್ಕೆ
ಮಾರ್ಗ-ಗಳ
ಮಾರ್ಗ-ಗ-ಳನ್ನೇ
ಮಾರ್ಗ-ಗಳಲ್ಲಿ
ಮಾರ್ಗ-ಗಳಿಂದ
ಮಾರ್ಗ-ಗ-ಳಿವೆ
ಮಾರ್ಗ-ಗಳು
ಮಾರ್ಗದ
ಮಾರ್ಗ-ದ-ರ್ಶಕ
ಮಾರ್ಗ-ದ-ರ್ಶ-ಕ-ನಾಗಿ
ಮಾರ್ಗ-ದ-ರ್ಶ-ಕ-ರಾ-ಗಿ-ದ್ದರು
ಮಾರ್ಗ-ದ-ರ್ಶ-ಕ-ವಾಗಿ
ಮಾರ್ಗ-ದ-ರ್ಶನ
ಮಾರ್ಗ-ದ-ರ್ಶ-ನ-ನಿ-ರ್ದೇ-ಶನ
ಮಾರ್ಗ-ದ-ರ್ಶ-ನ-ದಲ್ಲಿ
ಮಾರ್ಗ-ದ-ರ್ಶ-ನ-ವಿದೆ
ಮಾರ್ಗ-ದಲ್ಲಿ
ಮಾರ್ಗ-ದ-ಲ್ಲಿ-ದ್ದಾನೆ
ಮಾರ್ಗ-ವನ್ನು
ಮಾರ್ಗ-ವನ್ನೇ
ಮಾರ್ಗ-ವಾಗಿ
ಮಾರ್ಗ-ವಿ-ರ-ಲೇ-ಬೇಕು
ಮಾರ್ಗವೂ
ಮಾರ್ಗ-ವೆಂದರೆ
ಮಾರ್ಗವೇ
ಮಾರ್ಗ-ವೇನು
ಮಾರ್ಗ-ವೊಂದು
ಮಾರ್ಗ-ಸೂ-ಚಿ-ಯಾ-ಗಿದೆ
ಮಾರ್ಗಾ-ವ-ಲಂ-ಬಿ-ಗ-ಳಿಗೂ
ಮಾರ್ಚ್
ಮಾರ್ನು-ಡಿ-ಯು-ತ್ತಿತ್ತು
ಮಾರ್ಪಾ-ಡಿನ
ಮಾರ್ಮಿ-ಕ-ವಾ-ಗಿದೆ
ಮಾರ್ವಾಡಿ
ಮಾಲಿ-ಕ-ನಾದ
ಮಾಲೆ-ಗಳನ್ನು
ಮಾಳಿಗೆ
ಮಾವಿನ
ಮಾವಿ-ನ-ಮ-ರದ
ಮಾಸ-ಗಳು
ಮಾಸದ
ಮಾಸ-ಲಾ-ಯಿತು
ಮಾಸ-ಲಿಲ್ಲ
ಮಾಸ್ಟರ್
ಮಾಸ್ತ-ರರ
ಮಾಸ್ತ-ರ-ರಾ-ದ-ರೆಂ-ದಲ್ಲ
ಮಾಸ್ತ-ರಿ-ಕೆಯ
ಮಾಸ್ತ-ರಿಗೆ
ಮಾಸ್ತರು
ಮಾಹಿತಿ
ಮಿಂಚಿನ
ಮಿಂಚಿ-ನಂತೆ
ಮಿಂಚು
ಮಿಂಚು-ಸಿ-ಡಿ-ಲು-ಗ-ಳಂತೆ
ಮಿಂಚು-ನೋ-ಟ-ಗ-ಳಾದ
ಮಿಂಚೈ
ಮಿಂದು
ಮಿಕ್ಕ
ಮಿಕ್ಕ-ವರೆಲ್ಲ
ಮಿಕ್ಕೆಲ್ಲ
ಮಿಗಿ-ಲಾಗಿ
ಮಿಗಿ-ಲಾದ
ಮಿಗಿಲು
ಮಿಗು-ತ್ತಿತ್ತು
ಮಿಚಿ-ಗನ್
ಮಿಠಾಯಿ
ಮಿಡಿ-ತ-ವನ್ನು
ಮಿಡಿ-ತ-ವೆ-ನ್ನು-ವುದು
ಮಿತ
ಮಿತ-ವ್ಯ-ಯದ
ಮಿತಿ-ಗಳನ್ನೂ
ಮಿತಿ-ಮೀ-ರಿತು
ಮಿತಿ-ಮೀ-ರು-ತ್ತಿತ್ತು
ಮಿತಿ-ಯಿ-ದೆಯೆ
ಮಿತ್ರ
ಮಿತ್ರ-ಇ-ವರು
ಮಿತ್ರನ
ಮಿತ್ರ-ನಿಗೂ
ಮಿತ್ರನು
ಮಿತ್ರರ
ಮಿತ್ರ-ರಾದ
ಮಿತ್ರ-ರಿಗೆ
ಮಿತ್ರ-ರಿ-ಗೊಂದು
ಮಿತ್ರರು
ಮಿತ್ರ-ರೆಂಬ
ಮಿತ್ರ-ರೊಂ-ದಿಗೆ
ಮಿಥ್ಯೆ
ಮಿನು-ಗು-ತ್ತಿತ್ತು
ಮಿನು-ಗು-ತ್ತಿ-ದವೆ
ಮಿಲ-ನ-ಗೊ-ಳ್ಳಲು
ಮಿಲ-ನದ
ಮಿಲ್
ಮಿಳಿ-ತ-ವಾ-ಗಿತ್ತು
ಮಿಳಿ-ತ-ವಾ-ಗು-ತ್ತದೆ
ಮಿಶ್ರ
ಮಿಶ್ರ-ಣ-ದಿಂ-ದಾಗಿ
ಮಿಶ್ರಿತ
ಮಿಷ-ನರಿ
ಮಿಷ-ನ್ನಿನ
ಮಿಷ್ಟ-ಮ-ಮರ
ಮಿಸು-ಕ-ಲಿಲ್ಲ
ಮಿಸು-ಕಲೇ
ಮೀಟುತ್ತ
ಮೀತಿ-ಮೀರಿ
ಮೀನಿನ
ಮೀನು
ಮೀನು-ಮಾಂಸ
ಮೀಯರ
ಮೀರ-ತ್ತಿಗೆ
ಮೀರ-ತ್ತಿ-ನಲ್ಲಿ
ಮೀರ-ತ್ತಿ-ನ-ಲ್ಲಿ-ದ್ದಾಗ
ಮೀರ-ತ್ತಿ-ನ-ಲ್ಲಿ-ರುವ
ಮೀರ-ತ್ತಿ-ನಿಂದ
ಮೀರ-ದಿ-ರು-ವುದನ್ನು
ಮೀರಿ
ಮೀರಿತು
ಮೀರಿದ
ಮೀರಿ-ದ-ವ-ನನ್ನು
ಮೀರಿ-ದ-ವನೂ
ಮೀರಿ-ಸ-ಬಲ್ಲ
ಮೀರಿ-ಸಿ-ದ-ವನು
ಮೀರಿ-ಸಿ-ದ್ದಾರೆ
ಮೀರಿ-ಸಿ-ಬಿಟ್ಟ
ಮೀರಿ-ಸುವ
ಮೀರಿ-ಸು-ವಂ-ಥವು
ಮೀರಿ-ಸು-ವ-ವರು
ಮೀರಿ-ಹ-ನಾ-ವ-ನಿ-ರು-ವನು
ಮೀಸ-ಲಾ-ಗಿ-ಡ-ಬೇಕು
ಮೀಸ-ಲಾದ
ಮುಂಗಂ-ಡರು
ಮುಂಗಾ-ಣ-ಬ-ಲ್ಲ-ವ-ರಾ-ಗಿ-ದ್ದರು
ಮುಂಚೆಯೇ
ಮುಂಜಾನೆ
ಮುಂಜಾ-ನೆಯ
ಮುಂಜಾ-ನೆ-ಯಲ್ಲಿ
ಮುಂಜಾ-ವಿ-ನಲ್ಲೇ
ಮುಂಡ-ಕೋ-ಪ-ನಿ-ಷ-ತ್ತಿನ
ಮುಂತಾದ
ಮುಂತಾ-ದ-ವರ
ಮುಂದಕೆ
ಮುಂದಕ್ಕೆ
ಮುಂದಲೆ
ಮುಂದಾ-ಗ-ದಿ-ದ್ದು-ದನ್ನು
ಮುಂದಾ-ಗ-ಲಿಲ್ಲ
ಮುಂದಾಗಿ
ಮುಂದಾ-ಗಿ-ದ್ದರು
ಮುಂದಾಗು
ಮುಂದಾ-ಗು-ತ್ತಿದ್ದ
ಮುಂದಾದ
ಮುಂದಾ-ದರು
ಮುಂದಾ-ದರೂ
ಮುಂದಾ-ದಾರು
ಮುಂದಾ-ದು-ದ-ರಿಂ-ದ-ಲಾ-ದರೂ
ಮುಂದಾ-ಳಾ-ಗಿಯೇ
ಮುಂದಾಳು
ಮುಂದಾ-ಳು-ಗಳು
ಮುಂದಾ-ಳ್ತ-ನದ
ಮುಂದಾ-ಳ್ತ-ನ-ದಲ್ಲಿ
ಮುಂದಿಟ್ಟ
ಮುಂದಿ-ಟ್ಟರು
ಮುಂದಿಟ್ಟು
ಮುಂದಿ-ಟ್ಟು-ಕೊಂಡು
ಮುಂದಿ-ಡ-ಬೇ-ಕೆಂದು
ಮುಂದಿ-ಡಲೂ
ಮುಂದಿದ್ದ
ಮುಂದಿನ
ಮುಂದಿ-ನಿಂದ
ಮುಂದು
ಮುಂದು-ಮುಂ-ದಕ್ಕೆ
ಮುಂದು-ವರಿ
ಮುಂದು-ವ-ರಿ-ಕೆಯೇ
ಮುಂದು-ವ-ರಿದ
ಮುಂದು-ವ-ರಿ-ದಳು
ಮುಂದು-ವ-ರಿ-ದಾಗ
ಮುಂದು-ವ-ರಿದು
ಮುಂದು-ವ-ರಿ-ಯ-ಬ-ಹು-ದಾದ
ಮುಂದು-ವ-ರಿ-ಯ-ಬೇ-ಕಾ-ಗು-ತ್ತದೆ
ಮುಂದು-ವ-ರಿ-ಯ-ಬೇ-ಕೆಂದು
ಮುಂದು-ವ-ರಿ-ಯ-ಲಿಲ್ಲ
ಮುಂದು-ವ-ರಿ-ಯಲು
ಮುಂದು-ವ-ರಿ-ಯಿತು
ಮುಂದು-ವ-ರಿ-ಯಿರಿ
ಮುಂದು-ವ-ರಿ-ಯು-ತ್ತಾನೆ
ಮುಂದು-ವ-ರಿ-ಯು-ತ್ತಿತ್ತು
ಮುಂದು-ವ-ರಿ-ಯು-ತ್ತಿತ್ತೋ
ಮುಂದು-ವ-ರಿ-ಯು-ತ್ತಿದೆ
ಮುಂದು-ವ-ರಿ-ಯು-ತ್ತಿದ್ದ
ಮುಂದು-ವ-ರಿ-ಯು-ತ್ತಿ-ದ್ದಾ-ರೆಂ-ಬು-ದಕ್ಕೆ
ಮುಂದು-ವ-ರಿ-ಯು-ತ್ತಿ-ರು-ವುದನ್ನು
ಮುಂದು-ವ-ರಿ-ಯು-ತ್ತೇನೆ
ಮುಂದು-ವ-ರಿ-ಯು-ವಂತೆ
ಮುಂದು-ವ-ರಿ-ಸ-ಬೇಕು
ಮುಂದು-ವ-ರಿಸಿ
ಮುಂದು-ವ-ರಿ-ಸಿ-ಕೊಂಡು
ಮುಂದು-ವ-ರಿ-ಸಿದ
ಮುಂದು-ವ-ರಿ-ಸುತ್ತ
ಮುಂದು-ವ-ರಿ-ಸು-ತ್ತಾನೆ
ಮುಂದು-ವ-ರಿ-ಸು-ತ್ತಿದ್ದ
ಮುಂದು-ವ-ರಿ-ಸು-ವು-ದಿ-ಲ್ಲವೆ
ಮುಂದೂ-ಡಿದೆ
ಮುಂದೂ-ಡು-ತ್ತಲೇ
ಮುಂದೆ
ಮುಂದೆಂದೂ
ಮುಂದೆಯೂ
ಮುಂದೆಯೇ
ಮುಂದೇನು
ಮುಂದೊಂದು
ಮುಂದೊಮ್ಮೆ
ಮುಂಬೆ-ಳ-ಕನ್ನು
ಮುಂಬೆ-ಳ-ಕಿ-ನಲ್ಲಿ
ಮುಂಬೆ-ಳಕು
ಮುಂಭಾ-ಗ-ದ-ಲ್ಲಿ-ರುವ
ಮುಕ-ಮಂ-ಡಲ
ಮುಕ್ತ-ಕಂ-ಠ-ದಿಂದ
ಮುಕ್ತ-ನಾ-ಗ-ಬ-ಲ್ಲೆನೆ
ಮುಕ್ತ-ನಾಗಿ
ಮುಕ್ತ-ನಾ-ಗಿ-ಬಿ-ಡ-ಬೇಕು
ಮುಕ್ತ-ನಾ-ಗಿ-ರ-ಬೇ-ಕೆಂಬ
ಮುಕ್ತ-ನಾ-ಗಿ-ಹೋ-ಗಿ-ಬಿ-ಡು-ತ್ತಿದ್ದ
ಮುಕ್ತ-ನಾ-ಗುವ
ಮುಕ್ತನು
ಮುಕ್ತನೆ
ಮುಕ್ತ-ರಾ-ಗೋಣ
ಮುಕ್ತಾ-ಯ-ಗೊಂ-ಡಿತು
ಮುಕ್ತಾ-ಯ-ಗೊ-ಳ್ಳು-ವಂ-ಥ-ವಲ್ಲ
ಮುಕ್ತಿ
ಮುಕ್ತಿ-ಗಾಗಿ
ಮುಕ್ತಿಗೂ
ಮುಕ್ತಿಗೆ
ಮುಕ್ತಿ-ಗೆ-ಕಾರಣ-ವಾ-ಗು-ತ್ತದೆ
ಮುಕ್ತಿ-ದಾ-ಯ-ಕ-ವಾ-ದದ್ದು
ಮುಕ್ತಿ-ಬಂ-ಧ-ಗ-ಳಿ-ಲ್ಲ-ದಾ-ತ್ಮವು
ಮುಕ್ತಿಯ
ಮುಕ್ತಿ-ಯನ್ನು
ಮುಕ್ತಿ-ಯಾ-ಗಲಿ
ಮುಕ್ತಿಯು
ಮುಕ್ತಿ-ಯೆಂ-ಬುದು
ಮುಖ
ಮುಖಂಡ
ಮುಖಂ-ಡ-ತ್ವ-ದಲ್ಲಿ
ಮುಖಂ-ಡ-ನಂತೆ
ಮುಖಂ-ಡ-ನಾ-ಗಿದ್ದ
ಮುಖಂ-ಡರ
ಮುಖಂ-ಡ-ರ-ಲ್ಲೊ-ಬ್ಬ-ನಾದ
ಮುಖಂ-ಡ-ರಿಗೂ
ಮುಖಂ-ಡರು
ಮುಖಂ-ಡ-ರು-ಇ-ವರೆಲ್ಲ
ಮುಖಂ-ಡ-ರೊ-ಬ್ಬರು
ಮುಖಕ್ಕೆ
ಮುಖ-ಗಳನ್ನು
ಮುಖ-ಗಳು
ಮುಖ-ಗ-ಳೆಂ-ಬಂತೆ
ಮುಖದ
ಮುಖ-ದಲ್ಲಿ
ಮುಖ-ದಿಂ-ದಲೇ
ಮುಖ-ಭಾವ
ಮುಖ-ಭಾ-ವ-ವನ್ನು
ಮುಖ-ಮಂ-ಡಲ
ಮುಖ-ಮಂ-ಡ-ಲವೂ
ಮುಖ-ಮುದ್ರೆ
ಮುಖ-ಮು-ದ್ರೆ-ಯನ್ನು
ಮುಖರ್ಜಿ
ಮುಖ-ರ್ಜಿಯ
ಮುಖ-ರ್ಜಿ-ಯ-ವರ
ಮುಖರ್ಜೀ
ಮುಖ-ಲ-ಕ್ಷಣ
ಮುಖ-ವನ್ನು
ಮುಖ-ವನ್ನೇ
ಮುಖ-ವಾಡ
ಮುಖ-ಸ್ತು-ತಿಯ
ಮುಖಾ-ಮುಖಿ
ಮುಖಾ-ಮು-ಖಿ-ಯಾಗಿ
ಮುಖ್ಯ
ಮುಖ್ಯ-ಗು-ಮಾಸ್ತೆ
ಮುಖ್ಯ-ರಾದ
ಮುಖ್ಯ-ಲೋ-ಕದ
ಮುಖ್ಯ-ವಾಗಿ
ಮುಖ್ಯ-ವಾದ
ಮುಖ್ಯವೋ
ಮುಖ್ಯಾಂಶ
ಮುಖ್ಯಾಂ-ಶ-ಗಳನ್ನು
ಮುಖ್ಯಾಂ-ಶ-ವನ್ನು
ಮುಖ್ಯೋಪಾ
ಮುಖ್ಯೋ-ಪಾ-ಧ್ಯಾಯ
ಮುಗಿ
ಮುಗಿದ
ಮುಗಿ-ದ-ಮೇಲೆ
ಮುಗಿ-ದ-ಹಾ-ಗೆಯೇ
ಮುಗಿ-ದಿಲ್ಲ
ಮುಗಿದು
ಮುಗಿ-ದುವು
ಮುಗಿ-ದು-ಹೋಗ
ಮುಗಿ-ದು-ಹೋ-ಗಿ-ತ್ತು-ವಿ-ಶ್ವ-ನಾ-ಥನ
ಮುಗಿ-ದು-ಹೋ-ಗಿ-ದ್ದುವು
ಮುಗಿ-ದು-ಹೋ-ಯಿತು
ಮುಗಿ-ಯ-ಲಿಲ್ಲ
ಮುಗಿ-ಯಿತು
ಮುಗಿ-ಯುವ
ಮುಗಿ-ಯು-ವು-ದ-ರೊ-ಳಗೆ
ಮುಗಿ-ವ-ವ-ರೆಗೂ
ಮುಗಿ-ಸದೆ
ಮುಗಿ-ಸ-ಬೇಕು
ಮುಗಿ-ಸಲು
ಮುಗಿ-ಸ-ಲುಂಟೆ
ಮುಗಿಸಿ
ಮುಗಿ-ಸಿ-ಕೊಂಡು
ಮುಗಿ-ಸಿದ
ಮುಗಿ-ಸಿದೆ
ಮುಗಿ-ಸಿ-ಬಿ-ಡು-ತ್ತದೆ
ಮುಗಿ-ಸಿ-ಬಿ-ಡು-ತ್ತೇನೆ
ಮುಗಿ-ಸಿ-ಯೂ-ಬಿಟ್ಟ
ಮುಗಿ-ಸು-ವ-ವ-ರೆಗೆ
ಮುಗು-ಳು-ನಕ್ಕು
ಮುಗು-ಳ್ನ-ಕ್ಕರು
ಮುಗು-ಳ್ನಕ್ಕು
ಮುಗು-ಳ್ನ-ಗುತ್ತ
ಮುಗು-ಳ್ನ-ಗು-ತ್ತಲೇ
ಮುಗು-ಳ್ನ-ಗು-ವೊಂ-ದನ್ನು
ಮುಗು-ಳ್ನಗೆ
ಮುಗ್ಗ-ಟ್ಟನ್ನು
ಮುಗ್ಧ
ಮುಗ್ಧ-ತೆ-ಯನ್ನು
ಮುಗ್ಧ-ನಲ್ಲ
ಮುಗ್ಧ-ಭಾ-ವ-ದಿಂದ
ಮುಗ್ಧ-ಸ್ವ-ಭಾ-ವದ
ಮುಚ್ಚ-ಳ-ವನ್ನು
ಮುಚ್ಚಿ
ಮುಚ್ಚಿ-ಕೊಂಡು
ಮುಚ್ಚಿ-ಕೊ-ಳ್ಳು-ತ್ತಿತ್ತು
ಮುಚ್ಚಿ-ಡ-ದಿ-ರು-ವುದು
ಮುಚ್ಚಿ-ಡು-ವಂ-ಥದು
ಮುಚ್ಚಿದೆ
ಮುಚ್ಚಿದ್ದ
ಮುಚ್ಚಿ-ಬಿ-ಟ್ಟರು
ಮುಚ್ಚಿಸಿ
ಮುಚ್ಚಿ-ಸಿ-ಬಿ-ಡು-ತ್ತಿದ್ದ
ಮುಚ್ಚು-ಮರೆ
ಮುಚ್ಚು-ಮ-ರೆ-ಯಿ-ಲ್ಲದ
ಮುಚ್ಚು-ವಂ-ತಹ
ಮುಟ್ಟದ
ಮುಟ್ಟದೊ
ಮುಟ್ಟ-ಬ-ಲ್ಲರೆ
ಮುಟ್ಟಲು
ಮುಟ್ಟಾದ
ಮುಟ್ಟಿ
ಮುಟ್ಟಿತು
ಮುಟ್ಟಿದ
ಮುಟ್ಟಿ-ದರೂ
ಮುಟ್ಟಿ-ದ್ದೇ-ನೆಂ-ದ-ರಿತು
ಮುಟ್ಟಿ-ನೋ-ಡಿದ
ಮುಟ್ಟಿ-ನೋಡು
ಮುಟ್ಟು
ಮುಟ್ಟು-ತ್ತಿತ್ತು
ಮುಟ್ಟು-ತ್ತಿರ
ಮುಟ್ಟುವ
ಮುಟ್ಟು-ವಂತೆ
ಮುಟ್ಟು-ವು-ದಿಲ್ಲ
ಮುಟ್ಟು-ವುದು
ಮುಡಿ-ಪಾಗಿ
ಮುಡಿ-ಪಾ-ಗಿ-ಟ್ಟಿ-ರು-ವು-ದ-ರಿಂದ
ಮುತ್ತ-ಜ್ಜಿಯೂ
ಮುತ್ತಾ-ತ-ರಾ-ದಿ-ಯಾಗಿ
ಮುತ್ತಿ
ಮುತ್ತಿ-ಕೊಂ-ಡಿತು
ಮುತ್ತಿ-ಕೊಂ-ಡಿ-ದ್ದು-ವೆಂದರೆ
ಮುತ್ತಿ-ಕೊಂ-ಡಿವೆ
ಮುತ್ತಿ-ಕೊಂಡು
ಮುತ್ತಿ-ದುವು
ಮುತ್ತು-ತ್ತಿ-ರುವ
ಮುತ್ತು-ರ-ತು-ನ-ಗಳೊ
ಮುತ್ತೆಲ್ಲ
ಮುದ-ಗೊಂಡ
ಮುದಿ
ಮುದಿ-ಕಣ್ಣು
ಮುದುಕ
ಮುದು-ಕನ
ಮುದು-ಕ-ನನ್ನು
ಮುದು-ಕ-ನಿಗೆ
ಮುದು-ಕನೇ
ಮುದು-ಕನೋ
ಮುದು-ಕಪ್ಪ
ಮುದು-ಡಿ-ಹೋ-ಗು-ತ್ತಿತ್ತು
ಮುದುರಿ
ಮುದ್ದಾದ
ಮುದ್ದಿಗೆ
ಮುದ್ದಿನ
ಮುದ್ದಿ-ಸಲಿ
ಮುದ್ದಿಸಿ
ಮುದ್ದಿ-ಸು-ವಂತೆ
ಮುದ್ದು
ಮುದ್ದೆ-ಮಾಡಿ
ಮುದ್ರ-ಕರು
ಮುದ್ರಣ
ಮುದ್ರೆ-ಇವು
ಮುದ್ರೆ-ಯನ್ನು
ಮುದ್ರೆ-ಯ-ನ್ನೊತ್ತಿ
ಮುದ್ರೆ-ಯ-ನ್ನೊ-ತ್ತಿ-ದುವು
ಮುದ್ರೆ-ಯೊ-ತ್ತಿ-ದುವು
ಮುನಿ-ಸಿ-ನಿಂದ
ಮುನಿಸು
ಮುನ್ನಡೆ
ಮುನ್ನ-ಡೆದ
ಮುನ್ನ-ಡೆ-ದರು
ಮುನ್ನ-ಡೆದು
ಮುನ್ನ-ಡೆ-ಯ-ಬೇ-ಕೆಂದು
ಮುನ್ನ-ಡೆ-ಯಲು
ಮುನ್ನ-ಡೆ-ಯು-ತ್ತಲೇ
ಮುನ್ನ-ಡೆ-ಯು-ತ್ತಿ-ದ್ದರು
ಮುನ್ನ-ಡೆ-ಯು-ತ್ತಿ-ರು-ವುದನ್ನು
ಮುನ್ನ-ಡೆ-ಯು-ವಂ-ತಾ-ಗಿದೆ
ಮುನ್ನ-ಡೆ-ಯು-ವು-ದೊಂದೇ
ಮುನ್ನ-ಡೆ-ಸ-ಬಲ್ಲ
ಮುನ್ನ-ಡೆಸಿ
ಮುನ್ನ-ಡೆ-ಸಿ-ಕೊಂಡು
ಮುನ್ನ-ಡೆಸು
ಮುನ್ನ-ಡೆ-ಸುತ್ತಿ
ಮುನ್ನ-ಡೆ-ಸು-ತ್ತಿ-ದ್ದಾರೆ
ಮುನ್ನ-ಡೆ-ಸು-ತ್ತಿ-ರು-ವ-ವರು
ಮುನ್ನ-ಡೆ-ಸುವ
ಮುನ್ನು-ಗ್ಗಲು
ಮುನ್ನುಗ್ಗಿ
ಮುನ್ನು-ಗ್ಗಿದ
ಮುನ್ನು-ಗ್ಗು-ವುದು
ಮುನ್ನೂ-ರು-ಮು-ನ್ನೂರು
ಮುನ್ನೋಟ
ಮುನ್ಸೂ-ಚನೆ
ಮುನ್ಸೂ-ಚ-ನೆ-ಗಳು
ಮುಬ್ಬು-ಕ-ವಿ-ಸುವ
ಮುಮು-ಕ್ಷು-ಗ-ಳಾ-ದವ
ಮುಮು-ಕ್ಷು-ಗ-ಳಿಗೆ
ಮುಮು-ಕ್ಷು-ಗಳು
ಮುಮು-ಕ್ಷುತ್ವ
ಮುರಿದ
ಮುರಿ-ದಂತೆ
ಮುರಿ-ದರೆ
ಮುರಿದು
ಮುರಿ-ದು-ಕೊಂಡ
ಮುರಿ-ದು-ಕೊಂಡು
ಮುರಿ-ದು-ಬಿ-ದ್ದರೂ
ಮುರು-ಕಲು
ಮುರು-ಸಾ-ಗು-ತ್ತಿತ್ತು
ಮುಲಾ-ಜನ್ನೂ
ಮುಳು
ಮುಳು-ಗ-ಬೇ-ಕೆಂಬ
ಮುಳು-ಗಲು
ಮುಳುಗಿ
ಮುಳು-ಗಿ-ದರೆ
ಮುಳು-ಗಿ-ದಳು
ಮುಳು-ಗಿ-ದ-ವರ
ಮುಳು-ಗಿ-ದ್ದರೂ
ಮುಳು-ಗಿ-ದ್ದಾಗ
ಮುಳು-ಗಿ-ದ್ದು-ಬಿ-ಟ್ಟರೆ
ಮುಳು-ಗಿ-ಬಿ-ಟ್ಟರು
ಮುಳು-ಗಿ-ಬಿ-ಟ್ಟಿದೆ
ಮುಳು-ಗಿ-ಬಿ-ಡಲು
ಮುಳು-ಗಿ-ಬಿಡು
ಮುಳು-ಗಿ-ಬಿ-ಡು-ವು-ದಿ-ಲ್ಲವೆ
ಮುಳು-ಗಿ-ರಲು
ಮುಳು-ಗಿರು
ಮುಳು-ಗಿ-ರುವ
ಮುಳು-ಗಿ-ರು-ವ-ವರೂ
ಮುಳು-ಗಿ-ಸಿತು
ಮುಳು-ಗಿ-ಸಿ-ಬಿ-ಡ-ಬೇಕು
ಮುಳು-ಗಿ-ಸಿ-ಬಿ-ಡು-ತ್ತಿದ್ದ
ಮುಳು-ಗಿ-ಹೋಗ
ಮುಳು-ಗಿ-ಹೋ-ದರು
ಮುಳು-ಗು-ವುದು
ಮುಳು-ಗೇ-ಳುವ
ಮುಷ್ಟಾ-ಮುಷ್ಟಿ
ಮುಷ್ಟಿ-ಯನ್ನು
ಮುಷ್ಟಿ-ಯು-ದ್ಧ-ದಲ್ಲಿ
ಮುಸ-ಲ್ಮಾನ
ಮುಸ-ಲ್ಮಾ-ನ-ನನ್ನು
ಮುಸ-ಲ್ಮಾ-ನನೂ
ಮುಸ-ಲ್ಮಾ-ನರ
ಮುಸ-ಲ್ಮಾ-ನರು
ಮುಸ-ಲ್ಮಾ-ನರೂ
ಮುಸು-ಕನ್ನು
ಮುಸು-ಕ-ಬಲ್ಲ
ಮುಸು-ಕಿ-ನಲ್ಲಿ
ಮುಸುಕು
ಮುಸ್ಲಿ-ಮರ
ಮುಹೂ-ರ್ತ-ಗ-ಳ-ಲ್ಲೊಂದು
ಮುಹೂ-ರ್ತ-ದಿಂದ
ಮುಹೂ-ರ್ತ-ದಿಂ-ದಲೇ
ಮುಹೂ-ರ್ತವು
ಮೂಕ-ನಾದ
ಮೂಕ-ರಾ-ಗಿ-ಬಿ-ಟ್ಟರು
ಮೂಕ-ವಿ-ಸ್ಮಿತ
ಮೂಡಿ
ಮೂಡಿತು
ಮೂಡಿದ
ಮೂಡಿ-ಬ-ರ-ತೊ-ಡ-ಗಿತ್ತು
ಮೂಡಿ-ಬ-ರ-ಬೇಕೇ
ಮೂಡಿ-ಬ-ರ-ಲಾ-ರಂ-ಭಿ-ಸಿ-ದುವು
ಮೂಡಿ-ಬರು
ಮೂಡಿ-ಬ-ರು-ತ್ತಿ-ರುವ
ಮೂಡಿಸಿ
ಮೂಡಿ-ಸಿದ್ದ
ಮೂಡಿ-ಸಿ-ದ್ದಾರೆ
ಮೂಡು-ತ್ತವೆ
ಮೂಡು-ತ್ತಿತ್ತು
ಮೂಡುವ
ಮೂಡು-ವಂ-ತಾ-ಗು-ತ್ತದೆ
ಮೂಢ
ಮೂಢ-ನಂ-ಬಿ-ಕೆ-ಗಳ
ಮೂಢ-ನಂ-ಬಿ-ಕೆ-ಗಳನ್ನೂ
ಮೂರ-ನೆಯ
ಮೂರ-ನೆ-ಯ-ದಾಗಿ
ಮೂರ-ನೆ-ಯ-ವಳು
ಮೂರನೇ
ಮೂರು
ಮೂರು-ನಾಲ್ಕು
ಮೂರು-ಜೀ-ವ-ರಲ್ಲಿ
ಮೂರು-ದಿ-ನ-ಗಳ
ಮೂರು-ವರೆ-ನಾಲ್ಕು
ಮೂರೂ
ಮೂರೇ
ಮೂರ್ಖ
ಮೂರ್ಖ-ತನ
ಮೂರ್ತ-ಮ-ಹೇ-ಶ್ವ-ರ-ಮು-ಜ್ಜ್ವಲ
ಮೂರ್ತ-ರೂಪ
ಮೂರ್ತಿ
ಮೂರ್ತಿ-ಪೂ-ಜೆ-ಯನ್ನು
ಮೂರ್ತಿ-ಪೂ-ಜೆ-ಯನ್ನೇ
ಮೂರ್ತಿ-ಪೂ-ಜೆ-ಯಲ್ಲಿ
ಮೂರ್ತಿ-ಯಾದ
ಮೂರ್ತಿಯೇ
ಮೂರ್ನಾಲ್ಕು
ಮೂಲ
ಮೂಲಕ
ಮೂಲ-ಕವೇ
ಮೂಲ-ಕಾ-ರಣ
ಮೂಲ-ಕಾರಣ-ವನ್ನು
ಮೂಲತಃ
ಮೂಲ-ಧ-ನ-ವಾ-ಗಿ-ಟ್ಟರೆ
ಮೂಲ-ಭೂತ
ಮೂಲವೋ
ಮೂಲೆ-ಗ-ಳಿಗೂ
ಮೂಲೆ-ಗಳೇ
ಮೂಲೆ-ಮೂ-ಲೆ-ಗಳನ್ನು
ಮೂಲೆ-ಮೂ-ಲೆ-ಯ-ನ್ನೆಲ್ಲ
ಮೂಲೆ-ಯಲ್ಲಿ
ಮೂಲೋ-ದ್ದೇಶ
ಮೂಳೆ-ಗಳನ್ನು
ಮೂಳೆ-ಗಳು
ಮೂಳೆ-ಮಾಂ-ಸ-ಗಳು
ಮೂವತ್ತು
ಮೂವರ
ಮೂವ-ರನ್ನು
ಮೂವ-ರನ್ನೂ
ಮೂವರು
ಮೂವರೂ
ಮೂಸೆ-ಯಲ್ಲಿ
ಮೃಗ-ರಾ-ಜ-ನಾದ
ಮೃತ-ಪ್ರಾ-ಯ-ರಾಗಿ
ಮೃತ-ಪ್ರಾ-ಯ-ವಾ-ಗಿ-ದ್ದರೆ
ಮೃತ್ಯು
ಮೃತ್ಯು-ಮು-ಖ-ದಿಂದ
ಮೃತ್ಯು-ವನ್ನೇ
ಮೃತ್ಯು-ವ-ಶ-ವಾ-ಗು-ವಂ-ಥ-ದನ್ನು
ಮೃತ್ಯು-ವಿನ
ಮೃತ್ಯು-ವಿ-ನಿಂದ
ಮೃತ್ಯು-ಶ-ಯ್ಯೆ-ಯಲ್ಲಿ
ಮೃತ್ಯು-ಸ್ವ-ರೂ-ಪ-ವಾದ
ಮೃದು
ಮೃದು-ಗೊಂಡು
ಮೃದು-ದ-ನಿ-ಯಲ್ಲಿ
ಮೃದು-ಮ-ಧುರ
ಮೃದು-ಲ-ತರ
ಮೃದು-ವಾಗಿ
ಮೆಂಟ್
ಮೆಚ್ಚಿ
ಮೆಚ್ಚಿ-ಕೊಂಡ
ಮೆಚ್ಚಿ-ಕೊಂ-ಡಾ-ರೇನು
ಮೆಚ್ಚಿ-ಕೊಂ-ಡಿದ್ದ
ಮೆಚ್ಚಿ-ಕೊಂ-ಡಿ-ದ್ದರು
ಮೆಚ್ಚಿ-ಕೊಂ-ಡಿ-ದ್ದಾ-ಗಲಿ
ಮೆಚ್ಚಿ-ಕೊ-ಳ್ಳ-ದಿ-ರ-ಲಿಲ್ಲ
ಮೆಚ್ಚಿ-ಕೊ-ಳ್ಳುತ್ತ
ಮೆಚ್ಚಿ-ಕೊ-ಳ್ಳು-ತ್ತಾನೆ
ಮೆಚ್ಚಿ-ಕೊ-ಳ್ಳುವ
ಮೆಚ್ಚಿ-ಗೆಯ
ಮೆಚ್ಚಿ-ಗೆ-ಯಾ-ಗಿತ್ತು
ಮೆಚ್ಚಿದ್ದ
ಮೆಚ್ಚುಗೆ
ಮೆಚ್ಚು-ಗೆಯ
ಮೆಚ್ಚು-ಗೆ-ಯಾ-ದುವು
ಮೆಚ್ಚು-ಗೆ-ಯಿಂದ
ಮೆಚ್ಚು-ತ್ತಿ-ರ-ಲಿಲ್ಲ
ಮೆಟ್ಟಿ
ಮೆಟ್ಟಿ-ನಿಂ-ತಿತು
ಮೆಟ್ಟಿ-ನಿಂತು
ಮೆಟ್ಟಿಲ
ಮೆಟ್ಟಿ-ಲಾಗಿ
ಮೆಟ್ಟಿಲು
ಮೆಟ್ಟಿ-ಲು-ಗಳ
ಮೆಟ್ಟಿ-ಲು-ಗಳನ್ನು
ಮೆಟ್ಟಿ-ಲೇರಿ
ಮೆಟ್ಟಿ-ಲೇ-ರಿ-ದ್ದಾನೆ
ಮೆಟ್ರೊ
ಮೆಟ್ರೊ-ಪಾ-ಲಿ-ಟನ್
ಮೆಟ್ರೋ
ಮೆಟ್ರೋ-ಪಾ-ಲಿ-ಟನ್
ಮೆಣ-ಸಿ-ನ-ಕಾ-ಯಿ-ಯನ್ನು
ಮೆದು
ಮೆದು-ಳಿಗೆ
ಮೆದು-ಳಿನ
ಮೆದು-ಳಿ-ನಲ್ಲಿ
ಮೆದು-ಳು-ನ-ರ-ಮಂ-ಡ-ಲ-ಶ-ರೀ-ರ-ಗಳ
ಮೆರ-ವ-ಣಿಗೆ
ಮೆರ-ವ-ಣಿ-ಗೆ-ಯಲ್ಲಿ
ಮೆರು-ಗಿಗೆ
ಮೆರೆದ
ಮೆರೆ-ದಾ-ಡುತ್ತ
ಮೆರೆ-ಯು-ತ್ತಿ-ದ್ದಾರೆ
ಮೆರೆ-ವರು
ಮೆರೆ-ಸಿ-ಯಾರು
ಮೆಲಕು
ಮೆಲು-ಕು-ಹಾ-ಕು-ತ್ತಿ-ದ್ದರು
ಮೆಲು-ದ-ನಿ-ಯಲ್ಲಿ
ಮೆಲೆ
ಮೆಲ್ಲಗೆ
ಮೆಲ್ಲನೆ
ಮೆಲ್ಲ-ನೆದ್ದು
ಮೆಲ್ಲ-ಮೆ-ಲ್ಲನೆ
ಮೇ
ಮೇ-ಜೂನ್
ಮೇಘ-ಗ-ಳಿಗೆ
ಮೇಣ
ಮೇಣ್
ಮೇಧಯಾ
ಮೇಧಾವಿ
ಮೇಧಾವೀ
ಮೇನೆ-ಗಳ
ಮೇರೆಗೆ
ಮೇಲಂತೂ
ಮೇಲಕ್ಕೆ
ಮೇಲ-ಕ್ಕೆತ್ತಿ
ಮೇಲ-ಕ್ಕೆ-ತ್ತಿ-ಬಿ-ಟ್ಟರು
ಮೇಲ-ಕ್ಕೋ-ಡಿ-ದರು
ಮೇಲಲ್ಲ
ಮೇಲಷ್ಟೇ
ಮೇಲ-ಷ್ಟೇ-ತಿ-ಳಿ-ಯ-ಬೇ-ಕಾ-ಗಿದೆ
ಮೇಲಾ-ಗಿದ್ದ
ಮೇಲಾ-ದರೂ
ಮೇಲಾ-ಯಿತು
ಮೇಲಿ-ಟ್ಟಿದ್ದ
ಮೇಲಿ-ಡಲೇ
ಮೇಲಿ-ಡುವ
ಮೇಲಿದೆ
ಮೇಲಿದ್ದ
ಮೇಲಿನ
ಮೇಲಿ-ನಂತೆ
ಮೇಲಿ-ನಿಂದ
ಮೇಲಿ-ರಲಿ
ಮೇಲಿ-ರಿ-ಸ-ಲಾ-ಯಿತು
ಮೇಲಿ-ರಿ-ಸಿ-ಕೊಂಡು
ಮೇಲಿ-ರಿ-ಸಿದ್ದ
ಮೇಲಿ-ರುವ
ಮೇಲಿ-ರು-ವಂ-ತಹ
ಮೇಲುಂ-ಟಾದ
ಮೇಲು-ಮೇಲೆ
ಮೇಲೂ
ಮೇಲೆ
ಮೇಲೆ-ಇ-ಲ್ಲ-ಎ-ನ್ನು-ವು-ದ-ಕ್ಕಾ-ಗು-ತ್ತ-ದೆಯೆ
ಮೇಲೆ-ತಾನು
ಮೇಲೆ-ತ್ತ-ಬೇ-ಕಾದ
ಮೇಲೆ-ತ್ತಲು
ಮೇಲೆತ್ತಿ
ಮೇಲೆ-ತ್ತಿ-ಕೊ-ಳ್ಳು-ವು-ದರ
ಮೇಲೆ-ತ್ತುವ
ಮೇಲೆದ್ದ
ಮೇಲೆ-ದ್ದಿ-ದ್ದುವು
ಮೇಲೆ-ಬ-ಹೂ-ದ-ಕ-ನಾದ
ಮೇಲೆಯೇ
ಮೇಲೆ-ರಗಿ
ಮೇಲೆಲ್ಲ
ಮೇಲೇ
ಮೇಲೇ-ನಾ-ದರೂ
ಮೇಲೇರಿ
ಮೇಲೇ-ರಿದ
ಮೇಲೇ-ಳೇಳು
ಮೇಲೊಂದು
ಮೇಲೊ-ರ-ಗಿ-ಕೊಂಡು
ಮೇಲ್ಗಡೆ
ಮೇಲ್ಗ-ಡೆಯ
ಮೇಲ್ನೋ-ಟಕ್ಕೆ
ಮೇಲ್ಪಂ-ಕ್ತಿ-ಯನ್ನು
ಮೇಲ್ಭಾ-ಗ-ದಲ್ಲಿ
ಮೇಲ್ಮೈ-ನೋ-ಟದ
ಮೇಲ್ಮೈ-ಯಲ್ಲಿ
ಮೇಳ-ದ-ವರು
ಮೈ
ಮೈಕ-ಟ್ಟನ್ನು
ಮೈಕ-ಟ್ಟಾ-ದರೂ
ಮೈಕಟ್ಟು
ಮೈಕಾಂತಿ
ಮೈಕೈ
ಮೈಕೊ-ಡಹಿ
ಮೈಗೂ
ಮೈಗೂ-ಡಿ-ಸಿ-ಕೊಂಡ
ಮೈಗೂ-ಡಿ-ಸಿ-ಕೊ-ಳ್ಳ-ಬೇಕು
ಮೈಗೂ-ಡಿ-ಸಿ-ಕೊ-ಳ್ಳು-ವಂ-ತಾ-ಗಲಿ
ಮೈಗೆ
ಮೈಗೆಲ್ಲ
ಮೈತಿ-ಳಿ-ದಾಗ
ಮೈತಿ-ಳಿದು
ಮೈತಿ-ಳಿ-ದೆ-ದ್ದಾಗ
ಮೈತುಂಬ
ಮೈದ-ಡು-ವುತ್ತ
ಮೈದ-ಳೆದು
ಮೈದ-ಳೆ-ಯಿತು
ಮೈದ-ಳೆ-ಯು-ವಂತೆ
ಮೈದಾನ
ಮೈದಾ-ನ-ದಲ್ಲಿ
ಮೈದಾಳಿ
ಮೈದೋ-ರ-ದಿ-ದ್ದಾಳೆ
ಮೈದೋ-ರದು
ಮೈದೋ-ರಿತ್ತು
ಮೈಬ-ಣ್ಣ-ದೊಂ-ದಿಗೆ
ಮೈಮ-ನ-ವನು
ಮೈಮ-ರೆ-ತದ್ದು
ಮೈಮ-ರೆ-ತರು
ಮೈಮ-ರೆತು
ಮೈಮ-ರೆ-ಯು-ತ್ತಿದ್ದ
ಮೈಮುಟ್ಟಿ
ಮೈಮು-ಟ್ಟಿ-ದಿರೋ
ಮೈಮೇ-ಲಿ-ಟ್ಟು-ಬಿ-ಟ್ಟರು
ಮೈಮೇ-ಲಿದ್ದ
ಮೈಮೇ-ಲಿನ
ಮೈಮೇಲೆ
ಮೈಮೇ-ಲೇನೂ
ಮೈಮೇ-ಲೊಂದು
ಮೈಯ
ಮೈಯನ್ನು
ಮೈಯಲ್ಲ
ಮೈಯು-ರಿ-ಯು-ತ್ತಿದೆ
ಮೈಲಿ
ಮೈಲಿ-ಗಳ
ಮೈಲಿ-ಗಳನ್ನು
ಮೈಲಿ-ಗೆಯ
ಮೈಲಿ-ಗೆ-ಯಾ-ಗಿ-ಬಿ-ಡು-ತ್ತದೆ
ಮೈಲಿ-ಯಷ್ಟು
ಮೈವ-ಡೆ-ದಾತ್ಮ
ಮೈವೆ-ತ್ತಿ-ರು-ವಂತೆ
ಮೈಸೂ-ರಿನ
ಮೈಸೂರು
ಮೊಂಡಾಟ
ಮೊಂಡು-ಬು-ದ್ಧಿಯ
ಮೊಂಡು-ವಾ-ದ-ವಲ್ಲ
ಮೊಕ-ದ್ದಮೆ
ಮೊಗ್ಗೇ
ಮೊಘಲ್
ಮೊಟ-ಕು-ಗೊ-ಳಿಸಿ
ಮೊಟ್ಟ
ಮೊಟ್ಟ-ಮೊ-ದಲ
ಮೊಟ್ಟ-ಮೊ-ದಲು
ಮೊತ್ತ-ಮೊ-ದ-ಲ-ನೆ-ಯ-ದಾಗಿ
ಮೊತ್ತ-ಮೊ-ದ-ಲಿ-ನಿಂ-ದಲೂ
ಮೊತ್ತ-ಮೊ-ದಲು
ಮೊದ-ಮೊ-ದ-ಲಿಗೆ
ಮೊದ-ಮೊ-ದಲು
ಮೊದಲ
ಮೊದ-ಲ-ನೆಯ
ಮೊದ-ಲ-ನೆ-ಯ-ದಾಗಿ
ಮೊದ-ಲ-ನೆ-ಯದು
ಮೊದ-ಲನೇ
ಮೊದ-ಲಾದ
ಮೊದ-ಲಾ-ದ-ವರ
ಮೊದ-ಲಾ-ದ-ವು-ಗಳನ್ನೆಲ್ಲ
ಮೊದ-ಲಾ-ದು-ವನ್ನು
ಮೊದ-ಲಾ-ದು-ವು-ಗಳ
ಮೊದಲಿ
ಮೊದ-ಲಿಗ
ಮೊದ-ಲಿ-ಗ-ನಾಗಿ
ಮೊದ-ಲಿ-ಗ-ನಾದ
ಮೊದ-ಲಿ-ಗ-ನೆಂ-ಬಂತೆ
ಮೊದ-ಲಿಗೆ
ಮೊದ-ಲಿನ
ಮೊದ-ಲಿ-ನಂ-ತಾದ
ಮೊದ-ಲಿ-ನಂ-ತೆಯೇ
ಮೊದ-ಲಿ-ನಿಂ-ದಲೂ
ಮೊದ-ಲಿ-ನಿಂಲೂ
ಮೊದಲು
ಮೊದಲೂ
ಮೊದಲೇ
ಮೊನೆ-ಯುಳ್ಳ
ಮೊರೆ-ಯಿಟ್ಟ
ಮೊರೆ-ಯಿ-ಟ್ಟರು
ಮೊರೆ-ಯಿ-ಡು-ತ್ತಾರೆ
ಮೊರೆ-ಯಿ-ಡು-ತ್ತಿ-ದ್ದರು
ಮೊರೆ-ಹೋದೆ
ಮೊಳ-ಕಾ-ಲನ್ನು
ಮೊಳಕೆ
ಮೊಳ-ಕೆ-ಯಲ್ಲಿ
ಮೊಳ-ಗ-ಬೇ-ಕಾ-ಗಿ-ದೆ-ಯ-ಲ್ಲವೆ
ಮೊಳ-ಗಿತು
ಮೊಳ-ಗಿ-ದ್ದಾನೆ
ಮೊಳ-ಗಿಸಿ
ಮೊಳ-ಗಿ-ಸುತ್ತ
ಮೊಳಗು
ಮೊಳ-ಗು-ತ್ತಿ-ದ್ದುವು
ಮೊಳೆತು
ಮೊಹ-ಲ್ಲ-ದಲ್ಲಿ
ಮೋಂಘೀರ್
ಮೋಕ್ಷವೇ
ಮೋಜೂ
ಮೋಡ-ಗಳೇ
ಮೋಡಿ
ಮೋಡಿ-ಗೊ-ಳ-ಗಾ-ದ-ವ-ರಂತೆ
ಮೋರಿ-ಯೊ-ಳಗೆ
ಮೋರೆ
ಮೋಸ
ಮೋಸವು
ಮೋಸ-ಹೋ-ಗು-ತ್ತೇವೆ
ಮೋಹ
ಮೋಹ-ಕ-ವಾ-ಗಿತ್ತು
ಮೋಹ-ಗಳನ್ನು
ಮೋಹ-ಗೊ-ಳಿ-ಪರು
ಮೋಹ-ವನ್ನು
ಮೋಹ-ವಲ್ಲ
ಮೋಹ-ವಾಗಿ
ಮೋಹವು
ಮೋಹ-ವೆ-ನ್ನಿ-ಸು-ತ್ತದೆ
ಮೋಹ-ವೆ-ನ್ನು-ವುದು
ಮೋಹವೇ
ಮೌಖಿ-ಕ-ವಲ್ಲ
ಮೌಢ್ಯ
ಮೌಢ್ಯ-ವೆಂದು
ಮೌನ
ಮೌನದ
ಮೌನ-ದಲ್ಲಿ
ಮೌನ-ವಾಗಿ
ಮೌನ-ವಾ-ಗಿದ್ದು
ಮೌನ-ವಾ-ಗಿಯೇ
ಮೌನ-ವಾ-ಗಿ-ರು-ತ್ತಾ-ರೆಂಬ
ಮೌನವೇ
ಮೌನ-ವೇನು
ಮೌನಾ-ಶ್ರು-ಧಾ-ರೆ-ಯಲಿ
ಮೌಲ್ಯ-ಗಳ
ಮ್ಲಾನ-ವ-ದ-ನ-ರಾಗಿ
ಯ
ಯಂತಿದ್ದ
ಯಂತೆ
ಯಂತ್ರೋ-ಪ-ಕ-ರ-ಣ-ಗಳನ್ನೆಲ್ಲ
ಯಕ್ಷ-ಪ್ರ-ಶ್ನೆ-ಯನ್ನು
ಯಜ-ಮಾನ
ಯಜ-ಮಾ-ನನ
ಯಜ-ಮಾ-ನ-ನಾದ
ಯಜ-ಮಾ-ನ-ನಾ-ದ-ವನು
ಯಜ-ಮಾ-ನ-ನಿಗೆ
ಯಜ್ಞೇ-ಶ್ವರ
ಯತೋ
ಯತ್ನ-ವನ್ನೂ
ಯತ್ನಿ-ಸಿ-ದಳು
ಯತ್ನಿಸು
ಯತ್ಪ್ರ-ಯಂ-ತ್ಯ-ಭಿ-ಸಂ-ವಿ-ಶಂತಿ
ಯಥಾ
ಯಥಾ-ರ್ಥ-ವಾಗಿ
ಯಥಾ-ವ-ತ್ತಾಗಿ
ಯಥಾ-ಶಕ್ತಿ
ಯಥೇ-ಚ್ಛ-ವಾಗಿ
ಯದ
ಯದಾ
ಯದು-ಮ-ಲ್ಲಿಕ
ಯದು-ಮ-ಲ್ಲಿ-ಕನ
ಯನ್ನು
ಯನ್ನೂ
ಯನ್ನೇ
ಯಮ-ಪಾ-ಶವು
ಯಮುನಾ
ಯಮು-ನೋತ್ರಿ
ಯಲ್ಲ
ಯಲ್ಲಿ
ಯಲ್ಲಿದ್ದ
ಯಲ್ಲಿ-ದ್ದರೂ
ಯಲ್ಲಿ-ದ್ದಾ-ಗಲೇ
ಯಲ್ಲಿ-ದ್ದು-ಕೊಂಡು
ಯವನು
ಯವರ
ಯವ-ರನ್ನು
ಯವ-ರಿಗೆ
ಯವರು
ಯವ-ರೆಗೆ
ಯಶ-ಸ್ವಿ-ಯಾಗಿ
ಯಶ-ಸ್ವಿ-ಯಾ-ಗಿ-ದ್ದರು
ಯಶ-ಸ್ವಿ-ಯಾ-ಗಿಯೇ
ಯಶ-ಸ್ವಿ-ಯಾ-ಗಿ-ಸಲು
ಯಶ-ಸ್ವಿ-ಯಾ-ಗು-ವಂತೆ
ಯಶ-ಸ್ವಿ-ಯಾ-ದ-ವರು
ಯಶ-ಸ್ವಿ-ಯಾ-ಯಿತು
ಯಶ-ಸ್ವಿಯೂ
ಯಶ-ಸ್ಸನ್ನೂ
ಯಶ-ಸ್ಸಿಗೆ
ಯಶಸ್ಸು
ಯಶ್ವಸೀ
ಯಾ
ಯಾಕಪ್ಪ
ಯಾಕಪ್ಪಾ
ಯಾಕಾ-ಗ-ಬಾ-ರದು
ಯಾಕಿ-ಲ್ಲಿಗೆ
ಯಾಕೆ
ಯಾಕೋ
ಯಾಗ-ದಂತೆ
ಯಾಗ-ಲಾರ
ಯಾಗಲಿ
ಯಾಗಿ
ಯಾಗಿತ್ತು
ಯಾಗಿ-ದ್ದರೆ
ಯಾಗಿಯೇ
ಯಾಗಿ-ರು-ತ್ತಿತ್ತು
ಯಾಗು-ತ್ತಿದೆ
ಯಾಗು-ತ್ತಿ-ರು-ವುದನ್ನು
ಯಾಚಿ-ಸಿದ
ಯಾಚಿ-ಸಿ-ದರು
ಯಾತನೆ
ಯಾತ-ನೆ-ಗಳನ್ನು
ಯಾತ-ನೆ-ಗಳಿಂದ
ಯಾತ-ನೆ-ಗ-ಳಿ-ಗೀ-ಡಾಗಿ
ಯಾತ-ನೆ-ಪ-ಡು-ವು-ದ-ಕ್ಕಿಂತ
ಯಾತ-ನೆಯ
ಯಾತ-ನೆ-ಯನ್ನು
ಯಾತ-ನೆ-ಯಿಂದ
ಯಾತ-ನೆಯೇ
ಯಾತ್ರಿಕ
ಯಾತ್ರಿ-ಕ-ನಾಗಿ
ಯಾತ್ರೆ
ಯಾತ್ರೆ-ಯನ್ನು
ಯಾದ
ಯಾದರೂ
ಯಾದರೆ
ಯಾದಳು
ಯಾದ-ವ-ಗಿರಿ
ಯಾದ-ವನು
ಯಾಮ
ಯಾಮ-ಕ್ಕೊಂ-ದ-ರಂತೆ
ಯಾಮ-ಗಳಲ್ಲಿ
ಯಾಯಿತು
ಯಾರ
ಯಾರ-ಕೈ-ಲಾ-ದರೂ
ಯಾರ-ದಾ-ದರೂ
ಯಾರದು
ಯಾರ-ದ್ದಪ್ಪಾ
ಯಾರ-ನ್ನಾ-ದರೂ
ಯಾರನ್ನು
ಯಾರನ್ನೂ
ಯಾರನ್ನೋ
ಯಾರಲ್ಲಿ
ಯಾರಾ
ಯಾರಾ-ದರೂ
ಯಾರಾ-ದ-ರೊಬ್ಬ
ಯಾರಾ-ದ-ರೊ-ಬ್ಬ-ರನ್ನು
ಯಾರಾ-ದ-ರೊ-ಬ್ಬ-ರಿಂದ
ಯಾರಾ-ದ-ರೊ-ಬ್ಬರು
ಯಾರಿ
ಯಾರಿಂ-ದ-ತಾನೆ
ಯಾರಿಂ-ದಲೂ
ಯಾರಿ-ಗಾಗಿ
ಯಾರಿ-ಗಾ-ದರೂ
ಯಾರಿ-ಗಿ-ರು-ತ್ತದೆ
ಯಾರಿಗೂ
ಯಾರಿಗೆ
ಯಾರಿ-ಗೆ-ತಾನೆ
ಯಾರಿ-ದಂಲೂ
ಯಾರಿ-ದ್ದಾರೆ
ಯಾರಿ-ರ-ಬ-ಹುದು
ಯಾರು
ಯಾರು-ತಾನೆ
ಯಾರು-ಯಾರು
ಯಾರೂ
ಯಾರೆಂ-ದ-ರ-ವರು
ಯಾರೆಂ-ದರೆ
ಯಾರೆಂ-ಬುದು
ಯಾರೇ
ಯಾರೇನು
ಯಾರೊಂ-ದಿ-ಗಾ-ದರೂ
ಯಾರೊಂ-ದಿಗೂ
ಯಾರೊಂ-ದಿಗೋ
ಯಾರೊ-ಡ-ನೆ-ಯಾ-ದರೂ
ಯಾರೊ-ಬ್ಬರೂ
ಯಾರೋ
ಯಾರ್ಯಾ-ರಿಂದ
ಯಾರ್ಯಾರು
ಯಾವ
ಯಾವ-ಗಲೂ
ಯಾವತ್ತೋ
ಯಾವನ
ಯಾವನು
ಯಾವನೋ
ಯಾವ-ಯಾವ
ಯಾವ-ಯಾ-ವಾಗ
ಯಾವ-ಯಾ-ವುದು
ಯಾವಾ
ಯಾವಾಗ
ಯಾವಾ-ಗ-ಲಾ-ದರೂ
ಯಾವಾ-ಗ-ಲಾ-ದ-ರೊಮ್ಮೆ
ಯಾವಾ-ಗಲೂ
ಯಾವಾ-ಗಲೋ
ಯಾವಾ-ಗೆಂ-ದರೆ
ಯಾವುದ
ಯಾವು-ದಕ್ಕೂ
ಯಾವು-ದದು
ಯಾವು-ದ-ನ್ನಾ-ದರೂ
ಯಾವುದನ್ನು
ಯಾವು-ದನ್ನೂ
ಯಾವು-ದನ್ನೇ
ಯಾವು-ದರ
ಯಾವು-ದ-ರಿಂದ
ಯಾವು-ದಾ-ದರೂ
ಯಾವು-ದಾ-ದ-ರೊಂದು
ಯಾವು-ದಿದೆ
ಯಾವು-ದಿ-ರ-ಬ-ಹುದು
ಯಾವುದು
ಯಾವು-ದು-ಸಂ-ನ್ಯಾ-ಸ-ಜೀ-ವ-ನವೆ
ಯಾವುದೂ
ಯಾವು-ದೆಂ-ದರೆ
ಯಾವುದೇ
ಯಾವುದೋ
ಯಾವುವು
ಯಾವುವೂ
ಯಾವೊಂದು
ಯಿಂದ
ಯಿಂದಲೇ
ಯಿಂದಾಗಿ
ಯಿಂದಿ-ರುವ
ಯಿತು
ಯಿತೋ
ಯಿತ್ತು
ಯಿಲ್ಲದೆ
ಯುಂಟಾ-ಗಲು
ಯುಂಟಾ-ಗಿದೆ
ಯುಕ್ತ
ಯುಕ್ತತೆ
ಯುಕ್ತ-ತೆ-ಗಳನ್ನೂ
ಯುಕ್ತಾ-ಯು-ಕ್ತ-ತೆಯ
ಯುಕ್ತಿ
ಯುಕ್ತಿ-ಯು-ಕ್ತ-ವಾಗಿ
ಯುಕ್ತಿ-ಯು-ಕ್ತ-ವಾ-ಗಿ-ವೆ-ಯೆಂ-ಬು-ದ-ರಲ್ಲಿ
ಯುಕ್ತಿ-ಯು-ಕ್ತ-ವಾದ
ಯುಗ
ಯುಗಕ್ಕೆ
ಯುಗ-ಕ್ಕೊಬ್ಬ
ಯುಗ-ಗಳ
ಯುಗ-ಗಳಲ್ಲಿ
ಯುಗ-ಗ-ಳಿಗೆ
ಯುಗದ
ಯುಗ-ದ-ಲ್ಲಂತೂ
ಯುಗ-ದಲ್ಲಿ
ಯುಗ-ಪು-ರು-ಷ-ರನ್ನು
ಯುಗ-ಯು-ಗ-ಗಳ
ಯುಗ-ಯು-ಗ-ಗ-ಳಿಂ-ದಲೂ
ಯುಗ-ಯು-ಗವು
ಯುಗ-ವಾಗಿ
ಯುಗ-ವೆಂದರೆ
ಯುಗವೇ
ಯುಗಾಂ-ತ್ಯ-ದಲ್ಲಿ
ಯುಗಾ-ವ-ತಾರ
ಯುಗೇ
ಯುದ್ದಕ್ಕೂ
ಯುದ್ಧದ
ಯುದ್ಧ-ನೌ-ಕೆ-ಯನ್ನು
ಯುದ್ಧ-ಭೂ-ಮಿ-ಯಲ್ಲಿ
ಯುವ
ಯುವಕ
ಯುವ-ಕ-ಯು-ವ-ತಿ-ಯರು
ಯುವ-ಕನೂ
ಯುವ-ಕರ
ಯುವ-ಕ-ರಂ-ತಲ್ಲ
ಯುವ-ಕ-ರನ್ನು
ಯುವ-ಕ-ರ-ನ್ನು-ತ-ನ್ನೆ-ಡೆಗೆ
ಯುವ-ಕ-ರ-ನ್ನೆಲ್ಲ
ಯುವ-ಕ-ರಲ್ಲಿ
ಯುವ-ಕ-ರ-ಲ್ಲಿಯೇ
ಯುವ-ಕ-ರಾ-ರಿಗೂ
ಯುವ-ಕ-ರಿಗೆ
ಯುವ-ಕ-ರಿ-ಗೆಲ್ಲ
ಯುವ-ಕರು
ಯುವ-ಕ-ರೆಲ್ಲ
ಯುವ-ಕ-ರೆ-ಲ್ಲರ
ಯುವ-ಕರೇ
ಯುವ-ಕ-ರೇಆ
ಯುವ-ಕ-ರೇ-ನಾ-ದರೂ
ಯುವ-ಕ-ಶಿ-ಷ್ಯ-ರಿಗೆ
ಯುವ-ಕ-ಶಿ-ಷ್ಯರು
ಯುವ-ಕ-ಶಿ-ಷ್ಯರೂ
ಯುವ-ಕ-ಸಂನ್ಯಾಸಿ
ಯುವ-ಜ-ನ-ತೆಗೆ
ಯುವ-ಜ-ನರ
ಯುವ-ಜ-ನರು
ಯುವತಿ
ಯುವ-ಮಿ-ತ್ರ-ರಿಗೆ
ಯುವ-ಶಿಷ್ಯ
ಯುವ-ಶಿ-ಷ್ಯರ
ಯುವ-ಶಿ-ಷ್ಯ-ರನ್ನು
ಯುವ-ಶಿ-ಷ್ಯ-ರ-ನ್ನೆಲ್ಲ
ಯುವ-ಶಿ-ಷ್ಯ-ರಲ್ಲೇ
ಯುವ-ಶಿ-ಷ್ಯ-ರ-ಲ್ಲೊ-ಬ್ಬ-ನಾದ
ಯುವ-ಶಿ-ಷ್ಯ-ರಿ-ಗಂತೂ
ಯುವ-ಶಿ-ಷ್ಯ-ರಿಗೂ
ಯುವ-ಶಿ-ಷ್ಯ-ರಿಗೆ
ಯುವ-ಶಿ-ಷ್ಯರು
ಯುವ-ಶಿ-ಷ್ಯರೂ
ಯುವ-ಶಿ-ಷ್ಯ-ರೆಲ್ಲ
ಯುವ-ಶಿ-ಷ್ಯ-ರೆ-ಲ್ಲರೂ
ಯುವ-ಶಿ-ಷ್ಯ-ರೊಂ-ದಿಗೂ
ಯುವ-ಸಂನ್ಯಾಸಿ
ಯುವ-ಸಂ-ನ್ಯಾ-ಸಿ-ಗಳ
ಯುವ-ಸಂ-ನ್ಯಾ-ಸಿ-ಗಳನ್ನು
ಯುವ-ಸಂ-ನ್ಯಾ-ಸಿ-ಗ-ಳ-ಲ್ಲದೆ
ಯುವ-ಸಂ-ನ್ಯಾ-ಸಿ-ಗ-ಳಿಗೆ
ಯುವ-ಸಂ-ನ್ಯಾ-ಸಿ-ಗಳು
ಯುವ-ಸಂ-ನ್ಯಾ-ಸಿ-ಗ-ಳೆ-ಲ್ಲರ
ಯುವ-ಸಾ-ಧ-ಕರ
ಯುವ-ಸಾ-ಧ-ಕ-ರಲ್ಲೂ
ಯುವ-ಸಾ-ಧ-ಕರು
ಯುವ-ಸಾ-ಧ-ಕ-ರೆಲ್ಲ
ಯುವ-ಸಾ-ಧ-ಕ-ರೊ-ಳ-ಗಿನ
ಯುವ-ಸ್ನೇ-ಹಿ-ತರ
ಯೆಂದರೆ
ಯೆಂಬುದು
ಯೆನ್ನು-ವುದು
ಯೇನ
ಯೇನು
ಯೇಳು-ತ್ತದೆ
ಯೊಂದಕ್ಕೆ
ಯೊಂದಿಗೆ
ಯೊಂದು
ಯೊಂದೂ
ಯೊಬ್ಬನ
ಯೊಬ್ಬನು
ಯೊಬ್ಬ-ರಿ-ಲ್ಲ-ದಿ-ದ್ದರೆ
ಯೋಗ
ಯೋಗ-ಆ-ಧ್ಯಾ-ತ್ಮಿಕ
ಯೋಗ-ಶಾಸ್ತ್ರ
ಯೋಗ-ಕ್ಷೆ-ಮ-ವನ್ನು
ಯೋಗ-ಕ್ಷೇಮ
ಯೋಗ-ಕ್ಷೇ-ಮದ
ಯೋಗ-ಗಳ
ಯೋಗದ
ಯೋಗ-ದಲ್ಲಿ
ಯೋಗ-ದೀಕ್ಷೆ
ಯೋಗ-ದೀ-ಕ್ಷೆ-ಯನ್ನು
ಯೋಗ-ಭ್ಯಾಸ
ಯೋಗ-ಮಾ-ರ್ಗ-ಗಳನ್ನು
ಯೋಗ-ವನ್ನು
ಯೋಗವು
ಯೋಗ-ಶಾ-ಸ್ತ್ರ-ಗಳು
ಯೋಗ-ಸೂ-ತ್ರ-ದಲ್ಲಿ
ಯೋಗಾ-ನಂದ
ಯೋಗಾ-ನಂ-ದರ
ಯೋಗಾ-ನಂ-ದ-ರಿಗೆ
ಯೋಗಾ-ನಂ-ದರು
ಯೋಗಾ-ಭ್ಯಾ-ಸ-ಗಳ
ಯೋಗಿ-ಯ-ತಿ-ಗಳನ್ನು
ಯೋಗಿ-ಗಳ
ಯೋಗಿಗೆ
ಯೋಗಿಯ
ಯೋಗಿ-ಯಿಂದ
ಯೋಗಿ-ವ-ರ್ಯರು
ಯೋಗಿ-ಸ-ಬೇ-ಕಾ-ದರೂ
ಯೋಗಿ-ಸಲು
ಯೋಗೀಂದ್ರ
ಯೋಗೋ-ದ್ಯಾನ
ಯೋಗ್ಯ
ಯೋಗ್ಯ-ತಾ-ಪತ್ರ
ಯೋಗ್ಯತೆ
ಯೋಗ್ಯ-ತೆಗೆ
ಯೋಗ್ಯ-ತೆ-ಯ-ನ್ನ-ರಿ-ತಿದ್ದ
ಯೋಗ್ಯ-ತೆ-ಯಿಂ-ದಲೂ
ಯೋಗ್ಯ-ತೆ-ಯಿ-ರ-ಬೇ-ಕಾ-ಗು-ತ್ತದೆ
ಯೋಗ್ಯ-ತೆಯೇ
ಯೋಗ್ಯ-ವಲ್ಲ
ಯೋಗ್ಯ-ವಾ-ಗಿದೆ
ಯೋಗ್ಯ-ವಾ-ಗಿ-ರ-ಲಿಲ್ಲ
ಯೋಗ್ಯವೂ
ಯೋಚನೆ
ಯೋಚ-ನೆ-ಯಾ-ಯಿತು
ಯೋಚ-ನೆಯೇ
ಯೋಚಿ-ಸ-ಬೇಕು
ಯೋಚಿಸಿ
ಯೋಚಿ-ಸಿದ
ಯೋಚಿ-ಸಿ-ದರು
ಯೋಚಿ-ಸಿ-ದ-ರು-ಶ್ರೀ-ರಾ-ಮ-ಕೃ-ಷ್ಣರ
ಯೋಚಿ-ಸಿದ್ದ
ಯೋಚಿ-ಸಿ-ನೋಡಿ
ಯೋಚಿ-ಸು-ತ್ತಿದ್ದ
ಯೋಚಿ-ಸು-ತ್ತಿ-ದ್ದಂತೆ
ಯೋಚಿ-ಸು-ತ್ತಿರ
ಯೋಜನೆ
ಯೋಜ-ನೆ-ಗಳನ್ನು
ಯೋಜ-ನೆ-ಗಳಲ್ಲಿ
ಯೋಜ-ನೆ-ಗ-ಳಿವೆ
ಯೋಜ-ನೆ-ಯನ್ನು
ಯೋಧ-ರನ್ನು
ಯೋಪಥಿ
ಯೋಸಿ-ಸು-ತ್ತಿ-ದೇವೆ
ಯೌವನ
ಯೌವ-ನದ
ಯೌವ-ನ-ದಲ್ಲಿ
ಯೌವ-ನ-ಭ-ರಿತ
ಯೌವ-ನ-ಭ-ರಿ-ತ-ರಾದ
ರ
ರಂಗಕ್ಕೆ
ರಂಗ-ಮಂ-ಟ-ಪ-ದ-ಲ್ಲಿ-ರು-ವುದನ್ನು
ರಂಗ-ಮಂ-ಟ-ಪ-ವಾ-ಯಿತು
ರಂಗು-ರಂ-ಗಾಗಿ
ರಂಜ-ನೀಯ
ರಂಜಿ-ತ-ನಾ-ಗಿದ್ದ
ರಂಜಿ-ತ-ವಾ-ಯಿತು
ರಂಜಿ-ಸು-ತ್ತಿದ್ದ
ರಂದು
ರಂಪ
ರಂಪಾಟ
ರಂಭಿ-ಸಿ-ದರು
ರಕ್ತ
ರಕ್ತ-ಗತ
ರಕ್ತ-ಗ-ತ-ವಾಗಿ
ರಕ್ತ-ಮಾಂ-ಸ-ಗಳ
ರಕ್ತ-ವನ್ನು
ರಕ್ತ-ವ-ರ್ಣ-ದಿಂದ
ರಕ್ತ-ಸ್ರಾ-ವ-ವಾ-ಗುವ
ರಕ್ತ-ಸ್ರಾ-ವ-ವಾ-ಯಿ-ತೆಂದು
ರಕ್ಷಣೆ
ರಕ್ಷ-ಣೆ-ಯಲ್ಲಿ
ರಕ್ಷಿ-ಸ-ಬೇಕು
ರಕ್ಷಿಸಿ
ರಕ್ಷಿ-ಸಿ-ಕೊಂಡು
ರಕ್ಷಿ-ಸಿ-ಕೊ-ಳ್ಳ-ಬೇಕು
ರಕ್ಷಿ-ಸಿ-ಡ-ಲಾಗಿದೆ
ರಕ್ಷಿ-ಸಿತು
ರಕ್ಷಿ-ಸಿ-ದ್ದಾರೆ
ರಕ್ಷಿ-ಸು-ತ್ತಾರೆ
ರಕ್ಷಿ-ಸುವ
ರಕ್ಷೆ
ರಘು-ನಾಥ
ರಚನಾ
ರಚಿ-ಸ-ಬ-ಹು-ದಾ-ಗಿದೆ
ರಚಿಸಿ
ರಚಿ-ಸಿದ
ರಚಿ-ಸಿ-ದರು
ರಚಿ-ಸಿದ್ದ
ರಚಿಸು
ರಚಿ-ಸು-ತ್ತಾರೆ
ರಚಿ-ಸು-ತ್ತಿ-ದ್ದಂ-ತಿ-ತ್ತು-ಒಬ್ಬ
ರಜ-ಪು-ತಾನ
ರಜ-ಪು-ತಾ-ನಕ್ಕೆ
ರಜಾ
ರಜಾ-ದಿಸ
ರಜೆ-ಯ-ಲ್ಲಂತೂ
ರಜೋ-ಗುಣ
ರಜೋ-ಗು-ಣ-ತ-ಮೋ-ಗುಣ
ರಜೋ-ಗು-ಣ-ದಿಂದ
ರಣ-ಹ-ದ್ದಿಗೆ
ರಣ-ಹದ್ದು
ರಣ-ಹ-ದ್ದು-ಗ-ಳಂತೆ
ರತನ್
ರತ್ನ
ರತ್ನಕ್ಕೆ
ರತ್ನ-ಗಳ
ರತ್ನ-ಗಳನ್ನು
ರತ್ನ-ದಂ-ಥವು
ರತ್ನ-ವನ್ನು
ರತ್ನ-ವಾದ
ರಥ
ರನ್ನ-ಗಣಿ
ರನ್ನಾ-ದರೂ
ರನ್ನು
ರನ್ನೂ
ರಪ-ರ-ಪನೆ
ರಭಸ
ರಭ-ಸಕ್ಕೆ
ರಭ-ಸ-ದಲ್ಲಿ
ರಭ-ಸ-ದಿಂದ
ರಭ-ಸ-ಪೂರ್ಣ
ರಭ-ಸ-ವನ್ನು
ರಮ-ಣೀಯ
ರಮಿ-ಸು-ತ್ತಿದ್ದ
ರಮ್ಯ
ರಮ್ಯ-ವಾದ
ರಲ್ಲ
ರಲ್ಲದೆ
ರಲ್ಲಿ
ರಲ್ಲೂ
ರಲ್ಲೊಂದು
ರವೆ-ಗಂ-ಜಿ-ಯನ್ನು
ರಸ
ರಸಜ್ಞ
ರಸ-ದೌ-ತ-ಣ-ವಾ-ಗು-ತ್ತಿತ್ತು
ರಸ-ಭ-ರಿ-ತ-ವಾದ
ರಸ-ವ-ತ್ತಾಗಿ
ರಸ-ವನ್ನು
ರಸಾ-ಯ-ನ-ಶಾ-ಸ್ತ್ರ-ವನ್ನು
ರಸ್ತೆ
ರಸ್ತೆ-ಗಳಲ್ಲಿ
ರಸ್ತೆ-ಗಳು
ರಸ್ತೆಗೆ
ರಸ್ತೆಯ
ರಸ್ತೆ-ಯಲ್ಲಿ
ರಸ್ತೆ-ಯ-ವ-ರೆಗೆ
ರಸ್ತೆ-ಯಾ-ದರೂ
ರಸ್ತೆ-ಯು-ದ್ದಕ್ಕೂ
ರಹಸ್ಯ
ರಹ-ಸ್ಯ-ಗಳನ್ನು
ರಹ-ಸ್ಯ-ಗ-ಳಲ್ಲೂ
ರಹ-ಸ್ಯವ
ರಹ-ಸ್ಯ-ವನ್ನು
ರಹ-ಸ್ಯ-ವಾಗಿ
ರಹ-ಸ್ಯ-ವಾದ
ರಹ-ಸ್ಯವು
ರಾಕ್ಷಸ
ರಾಕ್ಷ-ಸರೂ
ರಾಕ್ಷ-ಸ-ವಿ-ದೆ-ಯಪ್ಪ
ರಾಖಾಲ
ರಾಖಾ-ಲನ
ರಾಖಾ-ಲ-ನನ್ನು
ರಾಖಾ-ಲ-ನಿ-ಗೆಂ-ದರು
ರಾಖಾ-ಲನೂ
ರಾಖಾ-ಲ-ನೆಂದ
ರಾಖಾ-ಲರ
ರಾಖಾ-ಲಾದಿ
ರಾಖಾಲ್
ರಾಗ
ರಾಗದ
ರಾಗ-ವಾಗಿ
ರಾಗಿ
ರಾಗಿದ್ದ
ರಾಗಿ-ದ್ದರು
ರಾಗಿ-ದ್ದಾರೆ
ರಾಗಿ-ಬಿ-ಟ್ಟಿ-ದ್ದರು
ರಾಗಿಯೇ
ರಾಗಿ-ರು-ವಾಗ
ರಾಜ
ರಾಜ-ಮ-ಹಾ-ರಾ-ಜರು
ರಾಜ-ಕೀಯ
ರಾಜ-ಕು-ಮಾ-ರನ
ರಾಜ-ಕು-ಮಾ-ರ-ನಂತೆ
ರಾಜ-ಕು-ಮಾ-ರ-ನಾಗಿ
ರಾಜ-ಗಾಂ-ಭೀರ್ಯ
ರಾಜ-ಠೀ-ವಿ-ಯೇ-ಆ-ದರೆ
ರಾಜ-ತೇ-ಜಸ್ಸು
ರಾಜ-ತ್ವ-ದ-ಚ-ಕ್ರಾ-ಧಿ-ಪ-ತ್ಯದ
ರಾಜ-ಧಾನಿ
ರಾಜ-ಧಾ-ನಿ-ಯಾಗಿ
ರಾಜ-ಧಾ-ನಿ-ಯಾ-ಗಿದ್ದ
ರಾಜನ
ರಾಜ-ನಾಗಿ
ರಾಜ-ನಾ-ಗು-ತ್ತೇನೆ
ರಾಜ-ನಾ-ರಾ-ಯಣ
ರಾಜ-ನಾ-ರಾ-ಯ-ಣನ
ರಾಜ-ನಾ-ರಾ-ಯ-ಣ-ನಿಗೆ
ರಾಜ-ನೆಂದು
ರಾಜ-ಪು-ರದ
ರಾಜ-ಮ-ನೆ-ತ-ನದ
ರಾಜ-ಮ-ಹಾ-ರಾ-ಜ-ರು-ಗಳನ್ನೂ
ರಾಜ-ಮಾ-ರ್ಗ-ಗಳು
ರಾಜ-ಯ-ತಿ-ರಾಜ
ರಾಜ-ಯೋ-ಗ-ವನ್ನು
ರಾಜ-ಯೋ-ಗಿಯ
ರಾಜರ
ರಾಜರು
ರಾಜ-ವೈ-ಭ-ವದ
ರಾಜ-ಸಿಕ
ರಾಜ-ಸ್ಥಾನ
ರಾಜಾ
ರಾಜಾಧಿ
ರಾಜಾ-ಧಿ-ರಾ-ಜ-ನಿಗೆ
ರಾಜಾ-ರೋ-ಷ-ವಾಗಿ
ರಾಜೀ-ನಾಮೆ
ರಾಜೇಂ-ದ್ರ-ಲಾಲ್
ರಾಜ್ಕು-ಮಾರ
ರಾಜ್ಕು-ಮಾ-ರನ
ರಾಜ್ಕು-ಮಾ-ರ-ನಿಗೆ
ರಾಜ್ಕು-ಮಾರ್
ರಾಜ್ಯಕ್ಕೆ
ರಾಜ್ಯದ
ರಾಜ್ಯ-ಭಾರ
ರಾಜ್ಯ-ಭೋ-ಗ-ವನ್ನು
ರಾಜ್ಯ-ವನ್ನೂ
ರಾಣಿ
ರಾತ್ರಿ
ರಾತ್ರಿ-ಗಳು
ರಾತ್ರಿಯ
ರಾತ್ರಿ-ಯಂತೂ
ರಾತ್ರಿ-ಯನ್ನು
ರಾತ್ರಿ-ಯ-ನ್ನು-ಇಲ್ಲೇ
ರಾತ್ರಿ-ಯಲ್ಲಿ
ರಾತ್ರಿ-ಯಾ-ಗು-ವು-ದನ್ನೇ
ರಾತ್ರಿ-ಯಾ-ಯಿತು
ರಾತ್ರಿ-ಯಿಡೀ
ರಾತ್ರಿ-ಯು-ದ್ದಕ್ಕೂ
ರಾತ್ರಿಯೂ
ರಾತ್ರಿ-ಯೆಲ್ಲ
ರಾದ
ರಾದರು
ರಾದರೂ
ರಾದೀರಿ
ರಾಧಾ
ರಾಧಾ-ಕೃಷ್ಣ
ರಾಧಾ-ಕೃ-ಷ್ಣರ
ರಾಧಾ-ಕುಂ-ಡಕ್ಕೆ
ರಾಧಾ-ಕೃಷ್ಣ
ರಾಧೆ
ರಾಧೆಯ
ರಾಧೆ-ಯನ್ನು
ರಾಧೆ-ಯಾಗಿ
ರಾಧೆಯೇ
ರಾಮ
ರಾಮ-ಕೃ-ಷ್ಣರೇ
ರಾಮ-ಸೀ-ತೆ-ಯರ
ರಾಮ-ಕು-ಮಾರ
ರಾಮ-ಕು-ಮಾ-ರ-ನನ್ನು
ರಾಮ-ಕೃಷ್ಣ
ರಾಮ-ಕೃ-ಷ್ಣ-ನಾಗಿ
ರಾಮ-ಕೃ-ಷ್ಣರ
ರಾಮ-ಕೃ-ಷ್ಣರು
ರಾಮ-ಕೃ-ಷ್ಣ-ಸಂ-ಘದ
ರಾಮ-ಕೃ-ಷ್ಣ-ಸಂ-ಘ-ವೆಂ-ಬುದು
ರಾಮ-ಕೃ-ಷ್ಣಾ-ನಂದ
ರಾಮ-ಕೃ-ಷ್ಣಾ-ನಂ-ದ-ರನ್ನೂ
ರಾಮ-ಕೃ-ಷ್ಣಾ-ನಂ-ದ-ರಿಗೆ
ರಾಮ-ಕೃ-ಷ್ಣಾ-ನಂ-ದರು
ರಾಮ-ಕೃ-ಷ್ಣಾ-ನಂ-ದ-ರೊ-ಬ್ಬ-ರನ್ನು
ರಾಮ-ಚಂದ್ರ
ರಾಮ-ಚಂ-ದ್ರ-ದ-ತ್ತನ
ರಾಮ-ಚಂ-ದ್ರ-ನಿಗೆ
ರಾಮ-ದ-ಯಾಲ
ರಾಮ-ದ-ಯಾ-ಲ-ನನ್ನು
ರಾಮ-ದ-ಯಾ-ಲ-ನಿಗೆ
ರಾಮ-ದ-ಯಾ-ಲನೂ
ರಾಮ-ದ-ಯಾಲ್
ರಾಮನ
ರಾಮ-ನನ್ನು
ರಾಮ-ನಾ-ಗಿ-ದ್ದನೋ
ರಾಮ-ನಾ-ಮ-ವನ್ನು
ರಾಮನೂ
ರಾಮ-ಪ್ರ-ಸನ್ನ
ರಾಮ-ಪ್ರ-ಸಾ-ದನ
ರಾಮ-ಭಕ್ತ
ರಾಮ-ಮಂ-ತ್ರ-ದೀ-ಕ್ಷೆ-ಯನ್ನು
ರಾಮ-ಮಂ-ತ್ರ-ವನ್ನು
ರಾಮ-ಮಂ-ತ್ರ-ವ-ನ್ನು-ಚ್ಚ-ರಿ-ಸುತ್ತ
ರಾಮ-ಮೋ-ಹನ
ರಾಮ-ಲಾಲ
ರಾಮ-ಲಾ-ಲನ
ರಾಮ-ಲಾ-ಲ-ನನ್ನು
ರಾಮ-ಲಾ-ಲ-ನಿಗೂ
ರಾಮ-ಲೀಲಾ
ರಾಮಾ-ಯಣ
ರಾಮಾ-ಯ-ಣ-ಮ-ಹಾ-ಭಾ-ರ-ತ-ಗಳ
ರಾಮಾ-ಯ-ಣ-ಮ-ಹಾ-ಭಾ-ರ-ತ-ಗ-ಳೆಂಬ
ರಾಮಾ-ಯ-ಣದ
ರಾಮಾ-ಯ-ಣ-ವನ್ನು
ರಾಮಾ-ಯಾ-ಣವೇ
ರಾಮೇ-ಶ್ವ-ರಕ್ಕೆ
ರಾಮೇ-ಶ್ವ-ರದ
ರಾಮ್ಖಾ-ಡಿ-ನ-ರೇಂ-ದ್ರನ
ರಾಮ್ಬಾಬು
ರಾಮ್ಬಾ-ಬು-ವಿನ
ರಾಯ-ಪುರ
ರಾಯ-ಪು-ರಕ್ಕೆ
ರಾಯ-ಪು-ರ-ಗ-ಳಿ-ಗೆಲ್ಲ
ರಾಯ-ಪು-ರ-ದಲ್ಲಿ
ರಾಯ-ಪು-ರ-ದ-ಲ್ಲಿದ್ದ
ರಾಯ-ಭಾ-ರಿ-ಯಾಗಿ
ರಾಯಿತು
ರಾಯ್
ರಾರಿಗೂ
ರಾವಣ
ರಾಶಿ-ಯನ್ನೇ
ರಾಷ್ಟ್ರ
ರಾಷ್ಟ್ರ-ಕವಿ
ರಾಷ್ಟ್ರಕ್ಕೇ
ರಾಷ್ಟ್ರ-ಗಳ
ರಾಷ್ಟ್ರ-ಗಳಲ್ಲಿ
ರಾಷ್ಟ್ರ-ಗ-ಳಲ್ಲೂ
ರಾಷ್ಟ್ರ-ಗ-ಳಿಗೆ
ರಾಷ್ಟ್ರದ
ರಾಷ್ಟ್ರ-ದಲ್ಲಿ
ರಾಷ್ಟ್ರ-ನಾ-ಡಿ-ಯಲ್ಲಿ
ರಾಷ್ಟ್ರ-ಪ್ರಜ್ಞೆ
ರಾಷ್ಟ್ರ-ಪ್ರೇ-ಮದ
ರಾಷ್ಟ್ರ-ಪ್ರೇಮಿ
ರಾಷ್ಟ್ರ-ಪ್ರೇ-ಮಿ-ಯಾ-ಗ-ಬೇ-ಕಾ-ದ-ವನು
ರಾಷ್ಟ್ರ-ಭ-ಕ್ತಿಯ
ರಾಷ್ಟ್ರ-ವನ್ನು
ರಾಷ್ಟ್ರವೇ
ರಾಷ್ಟ್ರ-ಹಿತ
ರಾಷ್ಟ್ರೀಯ
ರಾಷ್ಟ್ರೀ-ಯ-ತಾ-ಭಾ-ವದ
ರಾಷ್ಟ್ರೋ-ದ್ಧಾರ
ರಾಸ-ಮಣಿ
ರಾಸ-ಮ-ಣಿಯ
ರಾಸ್
ರಿಂದ
ರಿಂದಲೇ
ರಿಗಿಂತ
ರಿಗೂ
ರಿಗೆ
ರಿಯಾ-ಯಿತಿ
ರಿವೆಟ್
ರೀತಿ
ರೀತಿ-ರ-ಹ-ಸ್ಯ-ಗಳನ್ನೂ
ರೀತಿ-ಗಳಲ್ಲಿ
ರೀತಿ-ನೀ-ತಿ-ಗಳು
ರೀತಿಯ
ರೀತಿ-ಯನ್ನು
ರೀತಿ-ಯಲ್ಲಿ
ರೀತಿ-ಯ-ಲ್ಲಿಯೇ
ರೀತಿ-ಯಲ್ಲೂ
ರೀತಿ-ಯಲ್ಲೇ
ರೀತಿ-ಯಾಗಿ
ರೀತಿ-ಯಾ-ದರೆ
ರೀತಿ-ಯಿಂದ
ರೀತಿ-ಯಿಂ-ದಲೂ
ರೀತಿಯೇ
ರುಗ್ಣ-ಶ-ಯ್ಯೆ-ಯಲ್ಲಿ
ರುಚಿ
ರುಚಿ-ಕ-ರ-ವಾದ
ರುಚಿ-ಯನ್ನು
ರುಚಿ-ಯಾ-ಗಿಯೇ
ರುಚಿ-ಯಾದ
ರುಚಿ-ಸ-ಲಾ-ರವು
ರುಚಿ-ಸು-ತ್ತಿಲ್ಲ
ರುಚಿ-ಸು-ವು-ದಿಲ್ಲ
ರುಜು
ರುಜು-ವಾ-ತು-ಪ-ಡಿ-ಸಿ-ಕೊ-ಳ್ಳದೆ
ರುದ್ರ
ರುದ್ರ-ನಾ-ಥನ
ರುದ್ರಾಕ್ಷಿ
ರುದ್ರಾ-ಕ್ಷಿ-ಮಾ-ಲೆ-ಯನ್ನು
ರುವ
ರುವು-ದಕ್ಕೂ
ರುವುದನ್ನು
ರೂಢ-ಮೂ-ಲ-ವಾ-ಗಿತ್ತು
ರೂಢಿ
ರೂಢಿಗೆ
ರೂಢಿ-ಗೊ-ಳಿ-ಸು-ವು-ದ-ಕ್ಕಾಗಿ
ರೂಢಿ-ಯಲ್ಲಿ
ರೂಢಿ-ಯ-ಲ್ಲಿ-ರ-ಲಿಲ್ಲ
ರೂಢಿ-ಯ-ಲ್ಲಿ-ರು-ವಂತೆ
ರೂಢಿ-ಯಾ-ಗಿತ್ತು
ರೂಢಿ-ಯಾ-ಗಿ-ರ-ಲಿಲ್ಲ
ರೂಢಿಸಿ
ರೂಢಿ-ಸಿ-ಕೊಂಡ
ರೂಪ
ರೂಪಕ್ಕೆ
ರೂಪ-ಗಳನ್ನು
ರೂಪ-ಗಳಿಂದ
ರೂಪ-ತ-ಳೆ-ಯು-ತ್ತಿದ್ದ
ರೂಪ-ದಲ್ಲಿ
ರೂಪ-ದ-ಲ್ಲಿದ್ದ
ರೂಪ-ದ-ಲ್ಲಿ-ರ-ಬ-ಹು-ದಾದ
ರೂಪ-ದಲ್ಲೂ
ರೂಪ-ದಿಂದ
ರೂಪ-ರೇಖೆ
ರೂಪ-ಲಾ-ವ-ಣ್ಯ-ವನ್ನು
ರೂಪ-ವನ್ನು
ರೂಪ-ವಾದ
ರೂಪಾಯಿ
ರೂಪಾ-ಯಿ-ಗಳನ್ನು
ರೂಪಾ-ಯಿ-ಗಳಲ್ಲಿ
ರೂಪಾ-ಯಿ-ಗ-ಳಾ-ದರೂ
ರೂಪಾ-ಯಿ-ಗೆಲ್ಲ
ರೂಪಾ-ಯಿಯ
ರೂಪಾ-ಯಿ-ಯಷ್ಟು
ರೂಪಿ-ತ-ಗೊಂ-ಡಿತ್ತು
ರೂಪಿ-ಸ-ದಿ-ದ್ದರೆ
ರೂಪಿ-ಸಲು
ರೂಪಿ-ಸಿ-ಕೊಂ-ಡಿ-ದ್ದರು
ರೂಪಿ-ಸಿ-ಕೊಂಡು
ರೂಪಿ-ಸಿ-ಕೊ-ಳ್ಳ-ದಿ-ದ್ದರೆ
ರೂಪಿ-ಸಿ-ಕೊ-ಳ್ಳಲು
ರೂಪಿ-ಸಿ-ಕೊ-ಳ್ಳುವು
ರೂಪಿ-ಸಿ-ಕೊ-ಳ್ಳೋಣ
ರೂಪಿ-ಸಿದ
ರೂಪಿ-ಸಿ-ದರು
ರೂಪಿ-ಸಿ-ದರೆ
ರೂಪಿ-ಸಿ-ದಳು
ರೂಪಿ-ಸು-ತ್ತಿದ್ದ
ರೂಪಿ-ಸು-ತ್ತಿ-ದ್ದುವು
ರೂಪು-ಗೊ-ಳ್ಳ-ಲಿದೆ
ರೂಪು-ಗೊಳ್ಳು
ರೂಪು-ಗೊ-ಳ್ಳು-ತ್ತದೆ
ರೂಪು-ರೇ-ಖೆ-ಯನ್ನು
ರೆಂದರೆ
ರೆಂದೋ
ರೆನ್ನಿ-ಸಿ-ಕೊಂ-ಡ-ವರು
ರೆನ್ನುವ
ರೆನ್ನು-ವುದು
ರೆಲ್ಲ
ರೆಲ್ಲರೂ
ರೇಖಾ-ಗ-ಣಿ-ತದ
ರೇಖಾ-ಗ-ಣಿ-ತ-ವನ್ನು
ರೇಖೆ
ರೇಖೆ-ಯೊಂದು
ರೇಗಾಡಿ
ರೇಗಾ-ಡು-ವುದನ್ನು
ರೇಗಿ
ರೇಗಿತು
ರೇಗಿ-ಬಿ-ಟ್ಟರು
ರೇಗಿ-ಬಿ-ಡು-ತ್ತಿದ್ದ
ರೇಗಿ-ಸು-ತ್ತಿ-ದ್ದುದೂ
ರೇಗಿ-ಸು-ವು-ದರ
ರೇನಲ್ಲ
ರೇನಾ-ದರೂ
ರೇನು
ರೈಲಿ-ನಲ್ಲಿ
ರೈಲು
ರೈಲು-ದಾರಿ
ರೈಲು-ನಿ-ಲ್ದಾಣ
ರೈಲು-ನಿ-ಲ್ದಾ-ಣದ
ರೈಲು-ಪ್ರ-ಯಾಣ
ರೈಲು-ಬಂಡಿ
ರೈಲು-ಮಾ-ರ್ಗ-ವಾಗಿ
ರೈಲ್ವೆ
ರೊಂದು
ರೊಡನೆ
ರೊಬ್ಬರು
ರೊಮಾಂ-ಚಿತ
ರೊಯ್ಯನೆ
ರೋಗ
ರೋಗ-ಗ್ರಸ್ತ
ರೋಗದ
ರೋಗ-ದಿಂ-ದಾಗಿ
ರೋಗ-ಪೀ-ಡಿ-ತ-ರಾ-ದರು
ರೋಗ-ವನ್ನು
ರೋಚ-ಕ-ವಾ-ಗಿ-ದ್ದರೂ
ರೋದಿ-ಸ-ಲಾ-ರಂ-ಭಿ-ಸಿ-ದರು
ರೋದಿ-ಸುತ್ತ
ರೋದಿ-ಸು-ತ್ತಿ-ದ್ದರು
ರೋಮ-ಕೂ-ಪ-ದಿಂ-ದಲೂ
ರೋಮ-ಗ-ಳೆಲ್ಲ
ರೋಮಾಂಚ
ರೋಮಾಂ-ಚ-ಕಾರಿ
ರೋಮಾಂ-ಚ-ಕಾ-ರಿ-ಯಾದ
ರೋಮಾಂ-ಚ-ಕಾ-ರಿ-ಯಾ-ಯಿತು
ರೋಮಾಂ-ಚ-ಕಾರೀ
ರೋಮಾಂ-ಚ-ನದ
ರೋಮಾಂ-ಚ-ನ-ವಾ-ಗಿ-ಬಿ-ಡು-ತ್ತದೆ
ರೋಮ್
ರೋಷ-ಗೊಂಡ
ರೋಷದ
ರೋಸಿ-ಹೋಗಿ
ರೋಸಿ-ಹೋ-ಗಿ-ರ-ಬೇಕು
ಲಂಗರು
ಲಂಗು-ಲ-ಗಾ-ಮಿ-ಲ್ಲದ
ಲಕ್ನೋ
ಲಕ್ನೋಗೆ
ಲಕ್ನೋ-ದಿಂದ
ಲಕ್ಷ-ಗ-ಟ್ಟಲೆ
ಲಕ್ಷ-ಣ-ಗಳನ್ನೂ
ಲಕ್ಷ-ಣ-ಗಳನ್ನೆಲ್ಲ
ಲಕ್ಷ-ಣ-ಗ-ಳಿವೆ
ಲಕ್ಷ-ಣ-ಗಳು
ಲಕ್ಷ-ಣ-ಗಳೇ
ಲಕ್ಷ-ಣ-ವಂ-ತ-ನಾದ
ಲಕ್ಷ-ಣ-ವನ್ನು
ಲಕ್ಷ-ಣ-ವಲ್ಲ
ಲಕ್ಷ-ಣ-ವಾ-ಗಿರ
ಲಕ್ಷ-ಣವೇ
ಲಕ್ಷ-ಣ-ವೇನು
ಲಕ್ಷ-ಣ-ವೇನೆಂದರೆ
ಲಕ್ಷಾಂ-ತರ
ಲಕ್ಷ್ಮಿ-ಸ-ರ-ಸ್ವತಿ
ಲಕ್ಷ್ಮೀ-ನಾ-ರಾ-ಯಣ
ಲಕ್ಷ್ಯವೇ
ಲಘಿಮಾ
ಲಘು
ಲಬ್ಧಾ
ಲಭಿ-ಸ-ಬೇ-ಕಾ-ದರೆ
ಲಭಿ-ಸ-ಲಿಲ್ಲ
ಲಭಿ-ಸಿತು
ಲಭಿ-ಸಿ-ದರೆ
ಲಭಿ-ಸಿ-ಬಿ-ಟ್ಟರೆ
ಲಭಿ-ಸಿ-ಯಾನು
ಲಭಿ-ಸು-ತ್ತದೆ
ಲಭ್ಯಃ
ಲಭ್ಯ-ವಾ-ಗಿವೆ
ಲಭ್ಯ-ವಾಗು
ಲಭ್ಯ-ವಾ-ದ-ಹೊ-ರತು
ಲಭ್ಯ-ವಾ-ಯಿತು
ಲಭ್ಯ-ವಿ-ದ್ದು-ವು-ಯಾ-ರೆಂ-ದ-ರ-ವ-ರಿ-ಗಲ್ಲ
ಲಭ್ಯ-ವಿ-ರುವ
ಲಭ್ಯ-ವಿಲ್ಲ
ಲಯದ
ಲಯ-ಬ-ದ್ಧ-ವಾಗಿ
ಲಯ-ವಾ-ಗಿ-ಬಿ-ಟ್ಟಿತ್ತು
ಲಯ-ವಾ-ಗಿ-ಹೋ-ಯಿತು
ಲವ-ಲ-ವಿ-ಕೆಯೂ
ಲವ-ಲೇ-ಶವೂ
ಲಾ
ಲಾಗು-ತ್ತದೆ
ಲಾಟೀ-ನು-ಗಳನ್ನು
ಲಾಟು
ಲಾಟು-ವಿನ
ಲಾಠಿ
ಲಾಠಿ-ಗಳೂ
ಲಾಠಿ-ಯನ್ನು
ಲಾಡಿ-ಸಿ-ದರೋ
ಲಾಭ
ಲಾಭ-ವೇನೂ
ಲಾರಂ-ಭಿ-ಸಿತು
ಲಾರಂ-ಭಿ-ಸಿದ
ಲಾರಂ-ಭಿ-ಸಿ-ದರು
ಲಾರದ
ಲಾರದು
ಲಾರದೆ
ಲಾರರು
ಲಾರೆ
ಲಾರೆವು
ಲಾಲಾ
ಲಾಲಿಸಿ
ಲಾಹೋರ್
ಲಿಂಗ-ಭೇ-ದ-ಗ-ಳುಂಟೆ
ಲಿಂಗ-ವ-ರಿ-ಯದ
ಲಿಕ್ಕಿ-ದೆ-ಯೆಂ-ದರೆ
ಲಿಲ್ಲ
ಲಿಲ್ಲ-ಅ-ಷ್ಟ-ರ-ಲ್ಲಿಯೇ
ಲಿಲ್ಲವೋ
ಲೀನ-ಗೊ-ಳಿ-ಸಿ-ದಾಗ
ಲೀನ-ಗೊ-ಳಿ-ಸಿ-ಬಿ-ಡ-ಬೇ-ಕೆಂಬ
ಲೀನ-ನಾಗಿ
ಲೀನ-ನಾಗು
ಲೀನ-ರಾ-ಗಿರ
ಲೀನ-ರಾ-ದರು
ಲೀನ-ವಾಗಿ
ಲೀನ-ವಾ-ಗಿ-ಬಿ-ಟ್ಟರು
ಲೀನ-ವಾ-ಗಿ-ಬಿ-ಟ್ಟಿತ್ತು
ಲೀನ-ವಾ-ಗಿ-ಬಿ-ಡು-ತ್ತಿತ್ತು
ಲೀನ-ವಾ-ಗಿ-ಬಿ-ಡು-ತ್ತಿದ್ದ
ಲೀನ-ವಾ-ಗು-ತ್ತಿತ್ತು
ಲೀಲಾ-ಕಾ-ರ್ಯ-ವನ್ನು
ಲೀಲಾ-ಜಾಲ
ಲೀಲಾ-ನಾ-ಟಕ
ಲೀಲಾ-ಸ-ಮಾ-ಪ್ತಿಯ
ಲೀಲಾ-ಸ-ಹ-ಚ-ರ-ನಾ-ಗಿ-ದ್ದ-ಕೊಂಡು
ಲೀಲಾ-ಸ-ಹ-ಚ-ರ-ರಾಗಿ
ಲೀಲೆ
ಲೀಲೆ-ಗಳು
ಲೀಲೆ-ಯನ್ನು
ಲೀಲೆ-ಯಾಡಿ
ಲೀಲೆ-ಯಾ-ಡಿದ
ಲೀಲೆ-ಯೆ-ನ್ನೆದೆ
ಲುಪ್ತ-ವಾ-ಗುವ
ಲುಬ್ಬಾಕ್
ಲೆಕ್ಕ
ಲೆಕ್ಕ-ವನ್ನು
ಲೆಕ್ಕ-ವಿ-ಡು-ತ್ತಿದ್ದ
ಲೆಕ್ಕ-ವೆಂ-ದ-ರಾ-ಗದು
ಲೆಕ್ಕಾ-ಚಾರ
ಲೆಕ್ಕಿ-ಸದೆ
ಲೆಕ್ಕಿ-ಸ-ಲಿಲ್ಲ
ಲೆಕ್ಕಿ-ಸು-ವವ
ಲೆಕ್ಕಿ-ಸು-ವ-ವ-ನಲ್ಲ
ಲೇಖಕ
ಲೇಖ-ಕನ
ಲೇಖ-ಕನು
ಲೇಖ-ಕರು
ಲೇಖ-ನ-ಭಾ-ಷ-ಣ-ಗಳ
ಲೇಖ-ನ-ವನ್ನು
ಲೇಪಿ-ಸಿ-ಕೊಂಡು
ಲೇರಿದ
ಲೇವಡಿ
ಲೇವ-ಡಿ-ಮಾ-ಡಿದ
ಲೇಶವೂ
ಲೈಂಗಿಕ
ಲೈಂಗಿ-ಕ-ತೆ-ಯಲ್ಲೇ
ಲೋಕ-ಕ-ಲ್ಯಾಣ
ಲೋಕ-ಕ-ಲ್ಯಾ-ಣ-ಕಾ-ರ್ಯ-ವೆಂ-ಬುದು
ಲೋಕ-ಕ-ಲ್ಯಾ-ಣ-ಕ್ಕಾಗಿ
ಲೋಕ-ಕ-ಲ್ಯಾ-ಣದ
ಲೋಕಕ್ಕೆ
ಲೋಕ-ಗಳ
ಲೋಕ-ಗು-ರು-ಗಳು
ಲೋಕದ
ಲೋಕ-ದಲ್ಲಿ
ಲೋಕ-ದ-ಲ್ಲಿ-ರು-ವಂತೆ
ಲೋಕ-ದಿಂದ
ಲೋಕ-ದೃ-ಷ್ಟಿ-ಗೆ-ಸಂ-ಸಾ-ರ-ಸ್ಥರೇ
ಲೋಕ-ದೃ-ಷ್ಟಿ-ಯಿಂದ
ಲೋಕ-ನ-ದಿಂದ
ಲೋಕ-ಮಾ-ನ್ಯ-ತೆ-ಗಾಗಿ
ಲೋಕ-ಲೋ-ಕಾಂ-ತ-ರ-ಗಳನ್ನೆಲ್ಲ
ಲೋಕ-ವನ್ನೇ
ಲೋಕ-ವಿ-ಖ್ಯಾತ
ಲೋಕ-ವಿ-ಖ್ಯಾ-ತ-ರಾ-ದರೋ
ಲೋಕ-ಶಿ-ಕ್ಷಣ
ಲೋಕ-ಶಿ-ಕ್ಷ-ಣ-ಕ್ಕಾಗಿ
ಲೋಕ-ಸೇವೆ
ಲೋಕ-ಹಿತ
ಲೋಕ-ಹಿ-ತ-ಚಿಂ-ತನೆ
ಲೋಕಾ-ನು-ಭವ
ಲೋಕಾ-ನು-ಭ-ವಿ-ಯಾದ
ಲೋಕಾನ್
ಲೋಟವೂ
ಲೋಪ-ದೋಷ
ಲೋಪ-ದೋ-ಷ-ಗಳನ್ನು
ಲೋಪ-ದೋ-ಷ-ಗಳನ್ನೂ
ಲೋಪ-ದೋ-ಷ-ಗ-ಳಿಂ-ದಲೂ
ಲೋಪ-ದೋ-ಷ-ಗ-ಳೇನೇ
ಲೋಭ-ವಾ-ಗು-ತ್ತದೆ
ಲೌಕಿಕ
ಲೌಕಿ-ಕ-ವಿ-ಷ-ಯ-ವಾ-ಸನೆ
ಲ್ಲಿದ್ದು
ಲ್ಲಿರಿ-ಸ-ಲಾ-ಯಿತು
ಲ್ಲುಂಟಾ-ಯಿತು
ಲ್ಲೆಲ್ಲ
ಲ್ಲೆಲ್ಲೂ
ಲ್ಲೊಂದು
ಲ್ಲೊಬ್ಬ
ಲ್ಲೊಬ್ಬರು
ಳಾಗಿ-ರ-ಲಿ-ಕ್ಕಿಲ್ಲ
ವಂಚ-ಕ-ನಾ-ಗ-ಬೇಡ
ವಂಚಿ-ತ-ನಾ-ದೆ-ನಲ್ಲ
ವಂಚಿ-ತ-ರಾ-ಗ-ಲಿಲ್ಲ
ವಂಚಿ-ಸಿ-ಕೊಳ್ಳ
ವಂತ
ವಂತನ
ವಂತ-ನನ್ನು
ವಂತನು
ವಂತ-ರಲ್ಲಿ
ವಂತರು
ವಂತಹ
ವಂತ-ಹದು
ವಂತಾ-ಗಲು
ವಂತಿಗೆ
ವಂತಿಲ್ಲ
ವಂತೂ
ವಂತೆ
ವಂಥ-ದಲ್ಲ
ವಂಥದೇ
ವಂಥ-ದೇ-ನಿದೆ
ವಂದೇ
ವಂದ್ಯ-ನಾದ
ವಂದ್ಯ-ಮಿಹ
ವಂಶದ
ವಂಶ-ವ-ಲ್ಲವೆ
ವಂಶ-ಸ್ಥರು
ವಂಶಾ-ಭಿ-ವೃ-ದ್ಧಿಗೇ
ವಕಾ-ಲ-ತ್ತನ್ನು
ವಕಾ-ಲತ್ತು
ವಕಾ-ಲ-ತ್ತೇನೂ
ವಕೀಲ
ವಕೀ-ಲನ
ವಕೀ-ಲ-ನ-ನ್ನಾಗಿ
ವಕೀ-ಲ-ನಾಗಿ
ವಕೀ-ಲ-ನಾದ
ವಕೀ-ಲ-ನಾ-ದ್ದ-ರಿಂದ
ವಕೀ-ಲನೋ
ವಕೀ-ಲರ
ವಕೀ-ಲ-ರಾ-ದೀರಿ
ವಕೀ-ಲ-ರಿಂದ
ವಕೀ-ಲರು
ವಕೀಲಿ
ವಕೀ-ಲಿ-ಕೆಯ
ವಕೀ-ಲಿ-ಯಲ್ಲಿ
ವಕೀ-ಲಿ-ವೃ-ತ್ತಿ-ಯಲ್ಲಿ
ವಕ್ತಾ
ವಕ್ರ-ವಾದ
ವಕ್ರೋ-ಕ್ತಿಯ
ವಚ-ನ-ವೇದ
ವಚ-ನ-ವೇ-ದದ
ವಚ-ನ-ವೇ-ದ-ದಲ್ಲಿ
ವಚ-ನೀಯ
ವಜ್ರದ
ವಜ್ರ-ಸ-ಮ-ವಾಗಿ
ವಟು-ಗಳು
ವಠಾ-ರಕ್ಕೆ
ವಣೆ-ಯಾ-ಯಿತು
ವತ್ತಾಗಿ
ವದಂ-ತಿ-ಯೊಂದು
ವನ
ವನ-ಭೋ-ಜ-ನಕ್ಕೆ
ವನ-ಭೋ-ಜ-ವನ್ನೂ
ವನ-ರಾ-ಜನೇ
ವನ-ರಾ-ಜಿ-ಗಳ
ವನ-ರಾಶಿ
ವನ-ಸಿರಿ
ವನು
ವನ್ನಾ-ಗಲಿ
ವನ್ನಾಗಿ
ವನ್ನಿ-ತ್ತಿ-ದ್ದರು
ವನ್ನು
ವನ್ನುಂ-ಟು-ಮಾ-ಡಿತು
ವನ್ನುಂ-ಟು-ಮಾ-ಡಿ-ದುವು
ವನ್ನುಂ-ಟು-ಮಾ-ಡಿ-ದ್ದ-ವೆಂದರೆ
ವನ್ನೂ
ವನ್ನೆಲ್ಲ
ವಪ್ಪ
ವಯ-ಸ್ಕರು
ವಯ-ಸ್ಸಿಗೆ
ವಯ-ಸ್ಸಿನ
ವಯ-ಸ್ಸಿ-ನಲ್ಲಿ
ವಯ-ಸ್ಸಿ-ನಲ್ಲೇ
ವಯ-ಸ್ಸಿ-ನ-ವ-ನಾ-ಗಿ-ರು-ವಾ-ಗಲೇ
ವಯ-ಸ್ಸಿ-ನ-ವ-ರೆಗೆ
ವಯ-ಸ್ಸಿ-ನಿಂದ
ವಯಸ್ಸು
ವಯೋ-ಗು-ಣಕ್ಕೆ
ವಯೋ-ಮಿ-ತಿ-ಯನ್ನು
ವಯೋ-ವೃದ್ಧ
ವಯೋ-ವೃ-ದ್ಧರು
ವರಣ
ವರ-ದ-ಕ್ಷಿಣೆ
ವರದಿ
ವರ-ದಿ-ಯಿಂದ
ವರ-ನನ್ನು
ವರ-ಪ್ರ-ಸಾ-ದ-ವ-ಲ್ಲ-ವೇನು
ವರ-ಮಾನವೇ
ವರ-ವನ್ನು
ವರಾಂ-ಡಕ್ಕೆ
ವರಾಂ-ಡದ
ವರಾಂ-ಡ-ದಲ್ಲಿ
ವರಾ-ಹ-ನ-ಗರ
ವರಿ-ಸ-ಲಾ-ರರು
ವರು
ವರು-ಣ-ನನ್ನು
ವರು-ಣನು
ವರೆ-ಗಿನ
ವರೆಗೂ
ವರೆಗೆ
ವರ್ಗಕ್ಕೆ
ವರ್ಗ-ಗ-ಳಾಗಿ
ವರ್ಗದ
ವರ್ಗ-ದ-ವರೆಲ್ಲ
ವರ್ಗೀ-ಕ-ರಿ-ಸ-ಬ-ಹುದು
ವರ್ಚ-ಸ್ಸನ್ನು
ವರ್ಜಿತ
ವರ್ಜಿ-ಸ-ಬೇ-ಕಾ-ಯಿತು
ವರ್ಜಿ-ಸ-ಬೇ-ಕೆಂ-ಬುದು
ವರ್ಡ್ಸ್
ವರ್ಡ್ಸ-್ವ-ರ್ತ್
ವರ್ಣ
ವರ್ಣದ
ವರ್ಣ-ನಾ-ತೀತ
ವರ್ಣನೆ
ವರ್ಣ-ನೆ-ಯನ್ನು
ವರ್ಣ-ಮ-ಯ-ವಾಗಿ
ವರ್ಣ-ಮಾ-ಲೆ-ಯನ್ನು
ವರ್ಣವು
ವರ್ಣಿ-ಸ-ಬಲ್ಲ
ವರ್ಣಿ-ಸ-ಲ-ಸಾಧ್ಯ
ವರ್ಣಿ-ಸಲಿ
ವರ್ಣಿ-ಸಲು
ವರ್ಣಿಸಿ
ವರ್ಣಿ-ಸಿ-ದರು
ವರ್ಣಿ-ಸಿ-ದರೂ
ವರ್ಣಿ-ಸಿ-ರ-ಬ-ಹುದು
ವರ್ಣಿ-ಸುತ್ತ
ವರ್ಣಿ-ಸು-ತ್ತಾನೆ
ವರ್ಣಿ-ಸು-ತ್ತಾರೆ
ವರ್ಣಿ-ಸು-ತ್ತಿದ್ದ
ವರ್ಣಿ-ಸು-ತ್ತಿ-ದ್ದಂತೆ
ವರ್ಣಿ-ಸು-ತ್ತಿ-ದ್ದರು
ವರ್ಣಿ-ಸುವ
ವರ್ಣಿ-ಸು-ವಲ್ಲಿ
ವರ್ಣಿ-ಸು-ವು-ದ-ರ-ಲ್ಲಿಲ್ಲ
ವರ್ಣಿ-ಸು-ವುದು
ವರ್ತನೆ
ವರ್ತ-ನೆ-ಇ-ವು-ಗ-ಳಿಂ-ದೆಲ್ಲ
ವರ್ತ-ನೆ-ಗ-ಳೆಲ್ಲ
ವರ್ತ-ನೆ-ಯನ್ನು
ವರ್ತ-ನೆ-ಯ-ನ್ನೆಲ್ಲ
ವರ್ತ-ನೆ-ಯ-ಲ್ಲಾ-ಗಲಿ
ವರ್ತ-ನೆ-ಯಿಂದ
ವರ್ತ-ಮಾನ
ವರ್ತಿಸಿ
ವರ್ತಿ-ಸಿದ
ವರ್ತಿ-ಸಿ-ದ್ದನ್ನು
ವರ್ತಿ-ಸು-ತ್ತಾರೆ
ವರ್ತಿ-ಸು-ತ್ತಿದ್ದ
ವರ್ತಿ-ಸು-ವುದನ್ನು
ವರ್ತಿ-ಸು-ವು-ದಿಲ್ಲ
ವರ್ತುಲ
ವರ್ತ್
ವರ್ಧಿ-ಸಿ-ಕೊ-ಳ್ಳಲು
ವರ್ಷ
ವರ್ಷ-ಕಾಲ
ವರ್ಷ-ಗ-ಟ್ಟಲೆ
ವರ್ಷ-ಗಳ
ವರ್ಷ-ಗಳಲ್ಲಿ
ವರ್ಷ-ಗ-ಳ-ವರೆ-ಗಾ-ದರೂ
ವರ್ಷ-ಗ-ಳ-ವ-ರೆಗೂ
ವರ್ಷ-ಗ-ಳ-ವ-ರೆಗೆ
ವರ್ಷ-ಗ-ಳಿಂ-ದಲೂ
ವರ್ಷ-ಗಳು
ವರ್ಷ-ಗಳೇ
ವರ್ಷದ
ವರ್ಷ-ದಲ್ಲಿ
ವರ್ಷ-ದ-ಲ್ಲಿ-ದ್ದಾನೆ
ವರ್ಷ-ದ-ವ-ನಿ-ರು-ವಾಗ
ವರ್ಷ-ದಿಂ-ದಲೂ
ವರ್ಷ-ವನ್ನು
ವರ್ಷ-ವಷ್ಟೆ
ವರ್ಷ-ವಾ-ಗಿತ್ತು
ವರ್ಷ-ವಾ-ದರೆ
ವರ್ಷ-ವಿಡೀ
ವರ್ಷ-ವೇನೋ
ವರ್ಷಾಂ-ತ-ರ-ಗಳ
ವರ್ಷಾಂ-ತ್ಯ-ದಲ್ಲಿ
ವರ್ಷಿ-ಸಲು
ವರ್ಷಿ-ಸು-ತ್ತಿ-ದ್ದಾರೆ
ವಲ-ಯಕ್ಕೆ
ವಲ-ಯ-ದಿಂದ
ವಲ್ಲವೆ
ವಲ್ಲಿ
ವವನೂ
ವವರು
ವಶ
ವಶಕ್ಕೆ
ವಶ-ದಲ್ಲೇ
ವಶ-ನಾ-ಗು-ವ-ವ-ನಲ್ಲ
ವಶ-ಪ-ಡಿ-ಸಿ-ಕೊಂ-ಡ-ವರು
ವಶಿತ್ವ
ವಶೀ-ಕ-ರ-ಣವೇ
ವಷ್ಟರ
ವಷ್ಟ-ರಲ್ಲಿ
ವಸತಿ
ವಸ-ತಿಗೂ
ವಸ-ತಿ-ಯನ್ನು
ವಸೂಲು
ವಸ್ತಾದ
ವಸ್ತು
ವಸ್ತು-ಗಳ
ವಸ್ತು-ಗಳನ್ನು
ವಸ್ತು-ಗಳನ್ನೆಲ್ಲ
ವಸ್ತು-ಗಳು
ವಸ್ತು-ಗಳೂ
ವಸ್ತು-ವನ್ನು
ವಸ್ತು-ವಲ್ಲ
ವಸ್ತು-ವಾ-ಗಿದ್ದ
ವಸ್ತು-ವಾದ
ವಸ್ತು-ವಿ-ಗಾಗಿ
ವಸ್ತು-ವಿನ
ವಸ್ತು-ವಿ-ನಿಂದ
ವಸ್ತುವೂ
ವಸ್ತು-ವೂಈ
ವಸ್ತು-ವೆಂದು
ವಸ್ತು-ಸ್ಥಿ-ತಿ-ಯೇ-ನೆಂ-ಬು-ದನ್ನು
ವಸ್ತ್ರ
ವಸ್ತ್ರ-ಗಳನ್ನು
ವಸ್ತ್ರದ
ವಸ್ತ್ರ-ಧಾ-ರಿ-ಗ-ಳಾಗಿ
ವಸ್ತ್ರ-ಧಾ-ರಿ-ಯಾದ
ವಸ್ತ್ರ-ವ-ನ್ನಾ-ದರೂ
ವಸ್ತ್ರಾ-ಭ-ರ-ಣ-ಗಳಿಂದ
ವಸ್ಥೆ-ಯ-ಲ್ಲಿ-ದ್ದು-ದ-ರಿಂದ
ವಹಿ-ಸ-ಬೇ-ಕಾದ
ವಹಿಸಿ
ವಹಿ-ಸಿ-ಕೊಂಡ
ವಹಿ-ಸಿ-ಕೊಂ-ಡಿದ್ದ
ವಹಿ-ಸಿ-ಕೊಂ-ಡಿ-ದ್ದರು
ವಹಿ-ಸಿ-ಕೊಂ-ಡಿ-ದ್ದಾನೆ
ವಹಿ-ಸಿ-ಕೊಂಡು
ವಹಿ-ಸಿ-ಕೊಡು
ವಹಿ-ಸಿ-ಕೊ-ಳ್ಳು-ತ್ತಿ-ದ್ದರು
ವಹಿ-ಸಿ-ಕೊ-ಳ್ಳು-ತ್ತೇನೆ
ವಹಿ-ಸಿ-ಕೊ-ಳ್ಳು-ವ-ವರು
ವಹಿ-ಸಿಕೋ
ವಹಿ-ಸಿಯೂ
ವಾ
ವಾಂತಿ
ವಾಂತಿ-ಯನ್ನು
ವಾಕ್ಪ-ಟು-ತ್ವಕ್ಕೆ
ವಾಕ್ಯ-ಗಳ
ವಾಕ್ಯ-ಗಳನ್ನು
ವಾಕ್ಯ-ಗಳಲ್ಲಿ
ವಾಕ್ಯ-ಗ-ಳಾ-ಗಲಿ
ವಾಕ್ಯ-ಗಳಿಂದ
ವಾಕ್ಯ-ಗಳು
ವಾಕ್ಯದ
ವಾಕ್ಯ-ದಿಂದ
ವಾಕ್ಯ-ವನ್ನು
ವಾಕ್ಯ-ವಾ-ಕ್ಯ-ಗ-ಳನ್ನೇ
ವಾಗ
ವಾಗ-ದಿ-ದ್ದರೆ
ವಾಗಲಿ
ವಾಗ-ಲಿಲ್ಲ
ವಾಗಿ
ವಾಗಿ-ಡಲು
ವಾಗಿತ್ತು
ವಾಗಿದೆ
ವಾಗಿ-ದೆ-ಯೆಂ-ಬು-ದನ್ನು
ವಾಗಿದ್ದ
ವಾಗಿ-ದ್ದರೂ
ವಾಗಿ-ಬಿ-ಟ್ಟಿತು
ವಾಗಿ-ಬಿ-ಟ್ಟಿತ್ತು
ವಾಗಿಯೂ
ವಾಗಿಯೇ
ವಾಗಿ-ರ-ಲಿಲ್ಲ
ವಾಗಿ-ರು-ತ್ತಿತ್ತು
ವಾಗಿ-ರು-ತ್ತಿದ್ದ
ವಾಗಿ-ರು-ವ-ವ-ರ-ಲ್ಲವೆ
ವಾಗಿಲ್ಲ
ವಾಗಿ-ಹೋ-ಗು-ತ್ತ-ದೆಯೋ
ವಾಗು-ತ್ತದೆ
ವಾಗುವ
ವಾಗು-ವಂತೆ
ವಾಗ್ದಾನ
ವಾಗ್ದೇ-ವಿಯೇ
ವಾಗ್ಧಾ-ರೆ-ಯನ್ನು
ವಾಗ್ಧಾ-ರೆ-ಯಿಂದ
ವಾಗ್ಮಿ-ಗಳು
ವಾಗ್ಮಿ-ಗಳೂ
ವಾಗ್ಮಿಯ
ವಾಗ್ಮಿಯೋ
ವಾಗ್ಮಿ-ವೀರ
ವಾಗ್ವಾದ
ವಾಗ್ವಾ-ದ-ಕ್ಕಿ-ಳಿ-ದರು
ವಾಗ್ವಾ-ಹಿ-ನಿಯ
ವಾಗ್ವೈ-ಖರಿ
ವಾಗ್ವೈ-ಖ-ರಿ-ಯನ್ನು
ವಾಗ್ವೈ-ಖ-ರಿ-ಯಿಂದ
ವಾಗ್ಸಾ-ಮ-ರ್ಥ್ಯ-ದಿಂದ
ವಾಚೋ
ವಾಣಿ-ಯನ್ನು
ವಾಣಿ-ಯಿಂದ
ವಾತ
ವಾತ-ರೋಗ
ವಾತ-ರೋ-ಗ-ದಿಂದ
ವಾತಾ
ವಾತಾ-ವ-ರಣ
ವಾತಾ-ವ-ರ-ಣ-ದಲ್ಲಿ
ವಾತಾ-ವ-ರ-ಣ-ದಿಂದ
ವಾತಾ-ವ-ರ-ಣ-ವನ್ನು
ವಾತಾ-ವ-ರ-ಣ-ವ-ನ್ನುಂ-ಟು-ಮಾ-ಡು-ತ್ತಿ-ದ್ದುವು
ವಾತಾ-ವ-ರ-ಣ-ವೆಲ್ಲ
ವಾತಾ-ವ-ರ-ಣವೇ
ವಾತಾ-ವ-ರ-ಣ-ವೇನು
ವಾತಾ-ವ-ರ-ಣ-ವೇ-ರ್ಪ-ಟ್ಟಿತು
ವಾತ್ಸಲ್ಯ
ವಾತ್ಸ-ಲ್ಯದ
ವಾತ್ಸ-ಲ್ಯ-ದಿಂದ
ವಾತ್ಸ-ಲ್ಯ-ಪೂರ್ಣ
ವಾತ್ಸ-ಲ್ಯ-ವನ್ನು
ವಾದ
ವಾದ-ವಿ-ವಾದ
ವಾದ-ವಿ-ವಾ-ದ-ಗಳ
ವಾದ-ವಿ-ವಾ-ದ-ಗಳನ್ನೆಲ್ಲ
ವಾದ-ವಿ-ವಾ-ದ-ಗ-ಳಿಗೆ
ವಾದ-ವಿ-ವಾ-ದ-ದಲ್ಲಿ
ವಾದ-ವಿ-ವಾ-ದ-ವೇ-ರ್ಪ-ಡು-ತ್ತಿತ್ತು
ವಾದ-ಕ್ಕಿ-ಳಿ-ದರೆ
ವಾದಕ್ಕೂ
ವಾದಕ್ಕೆ
ವಾದ-ಗಳ
ವಾದ-ಗಳನ್ನು
ವಾದ-ಗಳಲ್ಲಿ
ವಾದ-ಗ-ಳಿಗೂ
ವಾದ-ಗಳು
ವಾದ-ಗಳೇ
ವಾದದ
ವಾದ-ದಲ್ಲಿ
ವಾದ-ದ್ದಾ-ದರೆ
ವಾದದ್ದು
ವಾದ-ದ್ದೆಂದು
ವಾದ-ನ-ರೇಂ-ದ್ರನ
ವಾದ-ಮಾ-ಡುವ
ವಾದರು
ವಾದರೂ
ವಾದ-ವನ್ನು
ವಾದ-ವನ್ನೂ
ವಾದ-ವ-ನ್ನೆಲ್ಲ
ವಾದ-ವಿ-ವಾ-ದ-ಗಳನ್ನು
ವಾದ-ವಿ-ವಾ-ದ-ಗಳು
ವಾದ-ವಿ-ವಾ-ದ-ದಲ್ಲಿ
ವಾದವೂ
ವಾದಾಗ
ವಾದಿಸಿ
ವಾದಿ-ಸಿ-ದರು
ವಾದಿ-ಸಿ-ದರೂ
ವಾದಿ-ಸಿ-ದರೆ
ವಾದಿ-ಸು-ತ್ತಾನೆ
ವಾದಿ-ಸು-ತ್ತಿದ್ದ
ವಾದಿ-ಸು-ತ್ತಿ-ದ್ದರು
ವಾದಿ-ಸು-ವ-ವನು
ವಾದುವು
ವಾದ್ಯ
ವಾದ್ಯ-ಗಳನ್ನು
ವಾಪಸು
ವಾಪಸ್ಸು
ವಾಯವ್ಯ
ವಾಯಿತು
ವಾಯಿತೆ
ವಾಯಿ-ತೆಂ-ದರೆ
ವಾಯಿ-ತೆಂ-ದಲ್ಲ
ವಾಯಿ-ತೆಂಬ
ವಾಯು-ವಿ-ಹಾ-ರಕ್ಕೋ
ವಾರ
ವಾರಂಟು
ವಾರಕ್ಕೆ
ವಾರ-ಕ್ಕೊಮ್ಮೆ
ವಾರ-ಕ್ಕೊ-ಮ್ಮೆಯೋ
ವಾರ-ಗ-ಟ್ಟಲೆ
ವಾರ-ಗಳನ್ನು
ವಾರ-ಗಳಲ್ಲಿ
ವಾರ-ಣಾಸೀ
ವಾರದ
ವಾರ-ದಲ್ಲಿ
ವಾರ-ದಲ್ಲೇ
ವಾರ-ದ-ಲ್ಲೊಂದು
ವಾರ-ದೊ-ಳ-ಗಾಗಿ
ವಾರ-ವನ್ನು
ವಾರ-ವಿತ್ತು
ವಾರಾ-ಣಸಿ
ವಾರಾ-ಣ-ಸಿಗೆ
ವಾರಾ-ಣ-ಸಿಯ
ವಾರಾ-ಣ-ಸಿ-ಯನ್ನು
ವಾರಾ-ಣ-ಸಿ-ಯಲ್ಲಿ
ವಾರಾ-ಣ-ಸಿ-ಯ-ಲ್ಲಿದ್ದ
ವಾರಾ-ಣ-ಸಿ-ಯ-ಲ್ಲಿ-ದ್ದು-ಕೊಂಡು
ವಾರಾ-ಣ-ಸಿ-ಯಲ್ಲೇ
ವಾರಾ-ಣ-ಸಿ-ಯಿಂದ
ವಾರಾ-ಣಸೀ
ವಾರ್ತೆ
ವಾರ್ಷಿಕ
ವಾರ್ಷಿ-ಕೋ-ತ್ಸವ
ವಾಲಿ-ಕೊಂ-ಡಿತ್ತು
ವಾಲಿ-ಕೊಂ-ಡಿ-ರು-ವಂತೆ
ವಾಲಿ-ಕೊಂ-ಡಿಲ್ಲ
ವಾಲೀಸ್
ವಾಲ್ಮೀ-ಕಿ-ವ್ಯಾಸ
ವಾಸ
ವಾಸ-ಕ್ಕಂತೂ
ವಾಸ-ಕ್ಕಾ-ಗಿಯೇ
ವಾಸ-ಕ್ಕೆಂದು
ವಾಸದ
ವಾಸನೆ
ವಾಸ-ಮಾ-ಡುವ
ವಾಸ-ವಾಗಿ
ವಾಸ-ವಾ-ಗಿದ್ದ
ವಾಸ-ವಾ-ಗಿ-ದ್ದರು
ವಾಸ-ವಾ-ಗಿ-ದ್ದ-ವನು
ವಾಸ-ವಾ-ಗಿ-ದ್ದ-ವ-ರೆಂ-ದರೆ
ವಾಸ-ವಾ-ಗಿ-ದ್ದಾನೆ
ವಾಸ-ವಾ-ಗಿ-ದ್ದಾರೆ
ವಾಸ-ವಾ-ಗಿದ್ದು
ವಾಸ-ವಾ-ಗಿ-ದ್ದು-ಕೊಂಡು
ವಾಸ-ವಾ-ಗಿರಿ
ವಾಸ-ವಾ-ಗಿರು
ವಾಸ-ವಾ-ಗಿ-ರು-ತ್ತಾನೆ
ವಾಸ-ವಾ-ಗಿ-ರು-ವಂ-ತಹ
ವಾಸ-ವಾ-ಗಿ-ರು-ವಂ-ತಾ-ಗ-ಬೇಕು
ವಾಸ-ಸ್ಥ-ಳ-ವನ್ನು
ವಾಸಿ
ವಾಸಿ-ಮಾ-ಡಿ-ಕೊ-ಳ್ಳ-ಬೇಕು
ವಾಸಿ-ಯಾ-ಗಲಿ
ವಾಸಿ-ಯಾ-ಗಲು
ವಾಸಿ-ಯಾಗಿ
ವಾಸಿ-ಯಾ-ಗಿ-ಬಿ-ಡಲಿ
ವಾಸಿ-ಯಾ-ಗು-ತ್ತಿಲ್ಲ
ವಾಸಿ-ಯಾ-ಗು-ವು-ದಿ-ರಲಿ
ವಾಸಿ-ಯೇನೋ
ವಾಸಿ-ಸ-ಲಾ-ರಂ-ಭಿ-ಸಿದ
ವಾಸಿ-ಸ-ಲಾ-ರಂ-ಭಿ-ಸಿ-ದರು
ವಾಸಿ-ಸುತ್ತ
ವಾಸಿ-ಸು-ತ್ತಿದ್ದ
ವಾಸಿ-ಸು-ತ್ತಿ-ರುವ
ವಾಸಿ-ಸು-ವಂ-ತಹ
ವಾಸೆ-ಗಳಿಂದ
ವಾಸ್ತ-ವ-ವಾಗಿ
ವಾಸ್ತ-ವಿಕ
ವಾಸ್ತ-ವಿ-ಕತೆ
ವಾಸ್ತ-ವಿ-ಕ-ತೆಯ
ವಾಸ್ತ-ವ್ಯಕ್ಕೂ
ವಾಸ್ತ-ವ್ಯ-ವನ್ನು
ವಿಂಗ-ಡಿ-ಸ-ಬ-ಹು-ದು-ಸ-ಮ-ರ್ಥರು
ವಿಂಗ-ಡಿ-ಸಲೇ
ವಿಂಧ್ಯ-ಪ-ರ್ವ-ತದ
ವಿಕ-ಸನ
ವಿಕ-ಸ-ನಕ್ಕೆ
ವಿಕ-ಸ-ನ-ಗೊ-ಳ್ಳ-ಲಿ-ಕ್ಕಿದೆ
ವಿಕ-ಸಿ-ತ-ವಾ-ಗ-ತೊ-ಡ-ಗಿತ್ತು
ವಿಕೃ-ತ-ಗೊಂ-ಡಿ-ರು-ವುದನ್ನು
ವಿಕೋ-ಪಕ್ಕೆ
ವಿಖ್ಯಾತ
ವಿಗ್ರ-ಹ-ಗಳ
ವಿಗ್ರ-ಹ-ಗ-ಳ-ನ್ನಿ-ಟ್ಟಿದ್ದ
ವಿಗ್ರ-ಹ-ಗಳನ್ನು
ವಿಗ್ರ-ಹ-ಗ-ಳ-ಲ್ಲವೆ
ವಿಗ್ರ-ಹ-ಗಳು
ವಿಗ್ರ-ಹ-ಗ-ಳೆ-ದು-ರಿ-ನಲ್ಲಿ
ವಿಗ್ರ-ಹದ
ವಿಗ್ರ-ಹ-ಪೂಜೆ
ವಿಗ್ರ-ಹ-ರೂ-ಪ-ದಲ್ಲಿ
ವಿಗ್ರ-ಹ-ವನ್ನು
ವಿಗ್ರ-ಹ-ವ-ನ್ನು-ಅಲ್ಲೇ
ವಿಗ್ರ-ಹ-ವಾಗಿ
ವಿಚ-ಲಿ-ತ-ಗೊಳ್ಳು
ವಿಚ-ಲಿ-ತ-ನಾ-ಗು-ತ್ತಿ-ರ-ಲಿಲ್ಲ
ವಿಚ-ಲಿ-ತ-ರಾ-ದರು
ವಿಚ-ಲಿ-ತ-ರಾ-ದರೂ
ವಿಚಾರ
ವಿಚಾ-ರ-ಗಳ
ವಿಚಾ-ರ-ಗಳನ್ನು
ವಿಚಾ-ರ-ಗಳನ್ನೂ
ವಿಚಾ-ರ-ಗಳನ್ನೆಲ್ಲ
ವಿಚಾ-ರ-ಗ-ಳ-ನ್ನೊ-ಳ-ಗೊಂಡ
ವಿಚಾ-ರ-ಗಳಲ್ಲಿ
ವಿಚಾ-ರ-ಗಳಿಂದ
ವಿಚಾ-ರ-ಗಳು
ವಿಚಾ-ರ-ಗಳೂ
ವಿಚಾ-ರಣೆ
ವಿಚಾ-ರದ
ವಿಚಾ-ರ-ದಲ್ಲಿ
ವಿಚಾ-ರ-ದ-ಲ್ಲಿದ್ದ
ವಿಚಾ-ರ-ದೃಷ್ಟಿ
ವಿಚಾ-ರ-ಧಾರೆ
ವಿಚಾ-ರ-ಧಾ-ರೆ-ಇ-ವು-ಗಳನ್ನು
ವಿಚಾ-ರ-ಧಾ-ರೆಗೆ
ವಿಚಾ-ರ-ಧಾ-ರೆಯ
ವಿಚಾ-ರ-ಧಾ-ರೆ-ಯನ್ನು
ವಿಚಾ-ರ-ಧಾ-ರೆ-ಯನ್ನೇ
ವಿಚಾ-ರ-ಧಾ-ರೆ-ಯಲ್ಲಿ
ವಿಚಾ-ರ-ಧಾ-ರೆ-ಯಿಂದ
ವಿಚಾ-ರ-ಪರ
ವಿಚಾ-ರ-ಪೂರ್ಣ
ವಿಚಾ-ರ-ಮಾ-ರ್ಗದ
ವಿಚಾ-ರ-ಯೋ-ಗ್ಯ-ವಾ-ಗಿವೆ
ವಿಚಾ-ರ-ಲ-ಹರಿ
ವಿಚಾ-ರ-ಲ-ಹ-ರಿಯು
ವಿಚಾ-ರ-ವಂ-ತರ
ವಿಚಾ-ರ-ವಂ-ತಿಕೆ
ವಿಚಾ-ರ-ವಂ-ತಿ-ಕೆಗೆ
ವಿಚಾ-ರ-ವನ್ನು
ವಿಚಾ-ರ-ವನ್ನೂ
ವಿಚಾ-ರ-ವಾಗಿ
ವಿಚಾ-ರ-ವಾ-ದರೆ
ವಿಚಾ-ರ-ವಾದಿ
ವಿಚಾ-ರ-ವಾ-ದಿಯೂ
ವಿಚಾ-ರ-ವಿ-ನಿ-ಮಯ
ವಿಚಾ-ರ-ವೆಲ್ಲ
ವಿಚಾ-ರವೇ
ವಿಚಾ-ರ-ಶಕ್ತಿ
ವಿಚಾ-ರ-ಶ-ಕ್ತಿಯ
ವಿಚಾ-ರ-ಶ-ಕ್ತಿ-ಯಿಂದ
ವಿಚಾ-ರ-ಶ-ಕ್ತಿ-ಯೆನ್ನು
ವಿಚಾ-ರ-ಸ್ವಾ-ತಂ-ತ್ರ್ಯದ
ವಿಚಾ-ರ-ಸ್ವಾ-ತಂ-ತ್ರ್ಯ-ವ-ನ್ನಿ-ತ್ತಿದ್ದ
ವಿಚಾ-ರ-ಸ್ವಾ-ತಂ-ತ್ರ್ಯ-ವನ್ನು
ವಿಚಾ-ರಿಸಿ
ವಿಚಾ-ರಿ-ಸಿ-ಕೊ-ಳ್ಳು-ತಿದ್ದ
ವಿಚಾ-ರಿ-ಸಿ-ದರು
ವಿಚಿತ್ರ
ವಿಚಿ-ತ್ರ-ವನ್ನು
ವಿಚಿ-ತ್ರ-ವಪ್ಪಾ
ವಿಚಿ-ತ್ರ-ವಾಗಿ
ವಿಚಿ-ತ್ರ-ವಾದ
ವಿಚಿ-ತ್ರ-ವಾ-ದದ್ದೇ
ವಿಜ-ಯ-ಕೃಷ್ಣ
ವಿಜ-ಯ-ಕೃ-ಷ್ಣರ
ವಿಜ-ಯೋ-ತ್ಸಾ-ಹ-ದಿಂದ
ವಿಜೃಂ-ಭಿ-ಸ-ಲಿದೆ
ವಿಜೃಂ-ಭಿ-ಸು-ತ್ತಿ-ರು-ವಂತೆ
ವಿಜ್ಞಾನ
ವಿಜ್ಞಾ-ನ-ಕಲೆ
ವಿಜ್ಞಾ-ನ-ಗಳನ್ನು
ವಿಜ್ಞಾ-ನವೂ
ವಿಜ್ಞಾ-ನ-ಸ್ಥಿತಿ
ವಿಜ್ಞಾ-ನಾ-ನಂದ
ವಿಟ್ಟು
ವಿಡಂ-ಬ-ನೆ-ಗಳ
ವಿತ-ರಿ-ಸು-ತಿ-ಹನು
ವಿತರ್ಕ
ವಿತ-ರ್ಕ-ಗಳಲ್ಲಿ
ವಿತ-ರ್ಕ-ಗ-ಳಿಗೆ
ವಿತ-ರ್ಕ-ಗಳು
ವಿತ್ತು
ವಿದ-ಲನ
ವಿದಾಯ
ವಿದೆ-ಯಲ್ಲ
ವಿದೇ-ಶ-ಗ-ಳಲ್ಲೂ
ವಿದ್ದರೆ
ವಿದ್ದು-ದ-ರಿಂದ
ವಿದ್ಯ-ಮಾನ
ವಿದ್ಯಾ
ವಿದ್ಯಾ-ಸುಂ-ದರ
ವಿದ್ಯಾ-ಧಿ-ದೇ-ವ-ತೆ-ಯಾದ
ವಿದ್ಯಾ-ಬು-ದ್ಧಿ-ಯನ್ನೂ
ವಿದ್ಯಾ-ಭ್ಯಾಸ
ವಿದ್ಯಾ-ಭ್ಯಾ-ಸಕ್ಕೆ
ವಿದ್ಯಾ-ಭ್ಯಾ-ಸದ
ವಿದ್ಯಾ-ಭ್ಯಾ-ಸ-ದಲ್ಲಿ
ವಿದ್ಯಾ-ಭ್ಯಾ-ಸ-ದಿಂದ
ವಿದ್ಯಾ-ಭ್ಯಾ-ಸ-ವನ್ನು
ವಿದ್ಯಾರ್ಥಿ
ವಿದ್ಯಾ-ರ್ಥಿ-ಗಳ
ವಿದ್ಯಾ-ರ್ಥಿ-ಗ-ಳಂತೆ
ವಿದ್ಯಾ-ರ್ಥಿ-ಗ-ಳಾದ
ವಿದ್ಯಾ-ರ್ಥಿ-ಗ-ಳಿಗೆ
ವಿದ್ಯಾ-ರ್ಥಿ-ಗಳು
ವಿದ್ಯಾ-ರ್ಥಿ-ಗ-ಳೆಲ್ಲ
ವಿದ್ಯಾ-ರ್ಥಿ-ಯನ್ನು
ವಿದ್ಯಾ-ರ್ಥಿ-ಯಾಗಿ
ವಿದ್ಯಾ-ರ್ಥಿ-ಯಾಗಿದ್ದಾನೆ
ವಿದ್ಯಾ-ವಂತ
ವಿದ್ಯಾ-ವಂ-ತರು
ವಿದ್ಯಾ-ಶಾಲೆ
ವಿದ್ಯಾ-ಸಂಸ್ಥೆ
ವಿದ್ಯಾ-ಸಾ-ಗ-ರರ
ವಿದ್ಯಾ-ಸಾ-ಗ-ರ-ರಿಂದ
ವಿದ್ಯಾ-ಸಾ-ಗ-ರರು
ವಿದ್ಯುತ್
ವಿದ್ಯು-ತ್ತಿ-ನಂ-ತಹ
ವಿದ್ಯೆ
ವಿದ್ಯೆ-ಗಳನ್ನು
ವಿದ್ಯೆ-ಗಳಲ್ಲಿ
ವಿದ್ಯೆಯ
ವಿದ್ಯೆ-ಯನ್ನು
ವಿದ್ಯೆ-ಯನ್ನೂ
ವಿದ್ಯೆ-ಯಿಂದ
ವಿದ್ರಾ-ವಕ
ವಿದ್ವತ್ತೂ
ವಿದ್ವ-ತ್ಪೂರ್ಣ
ವಿದ್ವಾಂ-ಸ-ನಿಗೆ
ವಿದ್ವಾಂ-ಸರ
ವಿದ್ವಾಂ-ಸ-ರಿಂ-ದಲೂ
ವಿದ್ವಾಂ-ಸರು
ವಿದ್ವಾಂ-ಸರೂ
ವಿದ್ವಾಂ-ಸ-ರೆಂದೂ
ವಿದ್ವಾಂ-ಸ-ರೊ-ಬ್ಬ-ರೊಂ-ದಿಗೆ
ವಿಧ
ವಿಧದ
ವಿಧ-ದಲ್ಲಿ
ವಿಧರ್
ವಿಧವಾ
ವಿಧ-ವಾದ
ವಿಧ-ವೆ-ಯರ
ವಿಧಾನ
ವಿಧಾ-ನ-ಗಳ
ವಿಧಾ-ನ-ಗಳನ್ನು
ವಿಧಾ-ನ-ಗಳು
ವಿಧಾ-ನ-ದಲ್ಲಿ
ವಿಧಾ-ನ-ವನ್ನು
ವಿಧಾ-ನ-ವನ್ನೂ
ವಿಧಿ
ವಿಧಿ-ಗಳ
ವಿಧಿ-ಗ-ಳೆಲ್ಲ
ವಿಧಿ-ನಿ-ಯಮ
ವಿಧಿ-ನಿ-ಷೇ-ಧ-ಗ-ಳೆಲ್ಲ
ವಿಧಿ-ಯಿಚ್ಛೆ
ವಿಧಿ-ಯು-ಕ್ತ-ವಾಗಿ
ವಿಧಿ-ವ-ತ್ತಾಗಿ
ವಿಧಿ-ವೈ-ಚಿ-ತ್ರ್ಯವೇ
ವಿಧಿ-ಸಿ-ದರು
ವಿಧಿ-ಸುವ
ವಿಧೇ-ಯ-ತೆ-ಯಿಂ-ದಿ-ರ-ಬೇ-ಕಾ-ಗು-ತ್ತದೆ
ವಿಧ್ಯಾ-ಭ್ಯಾಸ
ವಿಧ್ಯುಕ್ತ
ವಿನಂ-ತಿ-ಪೂ-ರ್ವ-ಕ-ವಾದ
ವಿನಂ-ತಿಸಿ
ವಿನಂ-ತಿ-ಸಿ-ಕೊಂಡ
ವಿನಂ-ತಿ-ಸಿ-ಕೊಂ-ಡರು
ವಿನ-ತೆ-ಯನ್ನು
ವಿನ-ಯ-ದಿಂದ
ವಿನ-ಯ-ಮೂರ್ತಿ
ವಿನ-ಯ-ವನ್ನು
ವಿನಾ-ಯತಿ
ವಿನಾ-ಯಿತಿ
ವಿನಾ-ಯಿ-ತಿ-ಯನ್ನೇ
ವಿನಾ-ಯಿ-ತಿ-ಯನ್ನೋ
ವಿನಿ
ವಿನಿ-ಯೋ-ಗಿ-ಸ-ಬೇ-ಕಾ-ಗು-ತ್ತದೆ
ವಿನಿ-ಯೋ-ಗಿ-ಸ-ಬೇಕು
ವಿನಿ-ಯೋ-ಗಿ-ಸು-ತ್ತಿದ್ದ
ವಿನಿ-ಯೋ-ಗಿ-ಸು-ವು-ದಕ್ಕೂ
ವಿನೂ-ತನ
ವಿನ್ಯಾ-ಸವೂ
ವಿನ್ಯಾ-ಸವೇ
ವಿಪ-ರೀತ
ವಿಪ-ರೀ-ತ-ವಾಗಿ
ವಿಪು-ಲ-ವಾ-ಗಿ-ಸು-ವಿಕೆ
ವಿಪ್ರ-ಶ್ರೇ-ಷ್ಠನೂ
ವಿಫ-ಲ-ವಾದ
ವಿಭ-ಜಿ-ಸು-ವಂತೆ
ವಿಭಾಗ
ವಿಭಾ-ಗಿ-ಸು-ವೆ-ಯಲ್ಲ
ವಿಭಿನ್ನ
ವಿಭಿ-ನ್ನ-ವಾಗಿ
ವಿಭಿ-ನ್ನ-ವಾ-ಗಿತ್ತು
ವಿಭಿ-ನ್ನ-ವಾದ
ವಿಭಿ-ನ್ನ-ವಾ-ದದ್ದು
ವಿಮ-ರ್ಶಾ-ತ್ಮಕ
ವಿಮ-ರ್ಶಾ-ದೃ-ಷ್ಟಿ-ಯಿಂದ
ವಿಮ-ರ್ಶಿಸಿ
ವಿಮ-ರ್ಶಿ-ಸಿ-ನೋಡಿ
ವಿಮ-ರ್ಶೆಗೆ
ವಿಮ-ರ್ಶೆ-ಮಾಡಿ
ವಿಮ-ರ್ಶೆಯ
ವಿಮ-ರ್ಶೆ-ಯನ್ನು
ವಿಮಾ-ರ್ಶಾ-ತ್ಮಕ
ವಿಮು-ಖತೆ
ವಿಮು-ಖ-ತೆಯ
ವಿಮು-ಖ-ನಾ-ಗಿದ್ದ
ವಿರ-ಕ್ತನೆ
ವಿರ-ಕ್ತಿ-ಯುಂ-ಟಾಗು
ವಿರಜಾ
ವಿರ-ಜಾ-ಹೋ-ಮದ
ವಿರ-ಮಿ-ಸಲು
ವಿರ-ಲಿಲ್ಲ
ವಿರಳ
ವಿರಸ
ವಿರಹ
ವಿರ-ಹ-ವನ್ನು
ವಿರಾಗಿ
ವಿರಾ-ಗಿಯ
ವಿರಾ-ಗಿ-ಯಂತೆ
ವಿರಾ-ಜಿ-ಸು-ತ್ತಾನೆ
ವಿರಾ-ಜಿ-ಸು-ತ್ತಿದ್ದ
ವಿರಾ-ಜಿ-ಸು-ತ್ತಿ-ದ್ದರೆ
ವಿರಾ-ಜಿ-ಸು-ತ್ತಿ-ದ್ದಾರೆ
ವಿರಾ-ಜಿ-ಸು-ತ್ತಿ-ರು-ವುದನ್ನು
ವಿರಾ-ಮದ
ವಿರುದ್ಧ
ವಿರು-ದ್ಧ-ವಾ-ಗ-ಬ-ಹುದು
ವಿರು-ದ್ಧ-ವಾಗಿ
ವಿರು-ದ್ಧ-ವಾ-ಗಿದ್ದ
ವಿರು-ದ್ಧ-ವಾ-ಗು-ತ್ತದೋ
ವಿರು-ದ್ಧ-ವಾದ
ವಿರು-ದ್ಧ-ವಾ-ಯಿತು
ವಿರೋ-ಧ-ಗಳನ್ನು
ವಿರೋ-ಧವೇ
ವಿರೋ-ಧಾ-ಭಾ-ಸ-ಗಳನ್ನು
ವಿರೋಧಿ
ವಿರೋ-ಧಿ-ಸಿ-ದರೂ
ವಿರೋ-ಧಿಸು
ವಿರೋ-ಧಿ-ಸು-ತ್ತಿತ್ತು
ವಿರೋ-ಧಿ-ಸು-ತ್ತಿದ್ದ
ವಿರೋ-ಧಿ-ಸುವ
ವಿಲಂ-ಗನಾ
ವಿಲ-ಕ್ಷಣ
ವಿಲಾ-ಪಿ-ಸದೆ
ವಿಲಾಸ
ವಿಲಾ-ಸ-ಕ್ಕಾ-ಗಿಯೇ
ವಿಲಾ-ಸ-ಗಳ
ವಿಲಾ-ಸ-ಗಳಿಂದ
ವಿಲಾ-ಸ-ಗಳು
ವಿಲಾ-ಸ-ದಲ್ಲೇ
ವಿಲಾ-ಸ-ಪೂ-ರ್ಣ-ವಾದ
ವಿಲಾಸೀ
ವಿಲಿಯಂ
ವಿಲಿ-ವಿಲಿ
ವಿಲ್ಲ
ವಿವರ
ವಿವ-ರ-ಗಳನ್ನು
ವಿವ-ರ-ಗ-ಳಾ-ಗಲಿ
ವಿವ-ರ-ಗ-ಳಿಗೆ
ವಿವ-ರ-ಗಳೂ
ವಿವ-ರಣೆ
ವಿವ-ರ-ಣೆ-ಗ-ಳೊಂ-ದಿಗೆ
ವಿವ-ರ-ಣೆ-ಯನ್ನೂ
ವಿವ-ರ-ಣೆ-ಯಿಂ-ದಾ-ಗಲಿ
ವಿವ-ರ-ಪೂ-ರ್ಣ-ವಾದ
ವಿವ-ರ-ವಾಗಿ
ವಿವ-ರ-ವಿ-ವ-ರ-ವಾಗಿ
ವಿವ-ರಿಸ
ವಿವ-ರಿ-ಸ-ತೊ-ಡ-ಗಿದ
ವಿವ-ರಿ-ಸ-ಬ-ಹುದು
ವಿವ-ರಿ-ಸ-ಲಾ-ಗದ
ವಿವ-ರಿ-ಸ-ಲಾಗಿದೆ
ವಿವ-ರಿ-ಸ-ಹೊ-ರ-ಟರೆ
ವಿವ-ರಿಸಿ
ವಿವ-ರಿ-ಸಿದ
ವಿವ-ರಿ-ಸಿ-ದರು
ವಿವ-ರಿ-ಸಿ-ದರೂ
ವಿವ-ರಿ-ಸಿ-ದಾಗ
ವಿವ-ರಿ-ಸಿ-ದ್ದಾನೆ
ವಿವ-ರಿ-ಸಿದ್ದು
ವಿವ-ರಿ-ಸುತ್ತ
ವಿವ-ರಿ-ಸು-ತ್ತಾನೆ
ವಿವ-ರಿ-ಸು-ತ್ತಾರೆ
ವಿವ-ರಿ-ಸು-ತ್ತಿದ್ದ
ವಿವ-ರಿ-ಸು-ತ್ತಿ-ದ್ದರು
ವಿವ-ರಿ-ಸು-ವಾಗ
ವಿವ-ರಿ-ಸು-ವುದು
ವಿವಾದ
ವಿವಾ-ದಕ್ಕೆ
ವಿವಾ-ದದ
ವಿವಾ-ದಾ-ಸ್ಪದ
ವಿವಾಹ
ವಿವಾ-ಹ-ಗಳು
ವಿವಾ-ಹದ
ವಿವಾ-ಹ-ವನ್ನು
ವಿವಾ-ಹ-ವಾದ
ವಿವಿ-ದಿ-ಶಾ-ನಂದ
ವಿವಿಧ
ವಿವೇಕ
ವಿವೇ-ಕ-ಚೂ-ಡಾ-ಮ-ಣಿ-ಯನ್ನು
ವಿವೇ-ಕ-ಚೂ-ಡಾ-ಮ-ಣಿ-ಯಲ್ಲಿ
ವಿವೇ-ಕ-ದಿಂದ
ವಿವೇ-ಕ-ಪೂ-ರ್ಣ-ವಾಗಿ
ವಿವೇ-ಕ-ಪೂ-ರ್ಣ-ವಾ-ಗಿದೆ
ವಿವೇ-ಕ-ಪ್ರ-ಜ್ಞೆಯ
ವಿವೇ-ಕ-ಪ್ರ-ಜ್ಞೆ-ಯಿ-ರ-ಬೇಕು
ವಿವೇ-ಕ-ಬು-ದ್ಧಿ-ಯನ್ನು
ವಿವೇಕಾ
ವಿವೇಕಾನಂದ
ವಿವೇಕಾನಂದ-ನ-ನ್ನಾಗಿ
ವಿವೇಕಾನಂದ-ನಾಗಿ
ವಿವೇಕಾನಂದಮ್
ವಿವೇಕಾನಂದರ
ವಿವೇಕಾನಂದ-ರದು
ವಿವೇಕಾನಂದ-ರ-ನ್ನಾಗಿ
ವಿವೇಕಾನಂದ-ರನ್ನು
ವಿವೇಕಾನಂದ-ರ-ಲ್ಲಂತೂ
ವಿವೇಕಾನಂದ-ರಾ-ಗ-ಬೇಕಾ
ವಿವೇಕಾನಂದ-ರಾಗಿ
ವಿವೇಕಾನಂದ-ರಾ-ಗು-ವ-ವರೆ-ಗಿನ
ವಿವೇಕಾನಂದ-ರಾದ
ವಿವೇಕಾನಂದ-ರಾ-ದಾಗ
ವಿವೇಕಾನಂದ-ರಿಗೆ
ವಿವೇಕಾನಂದ-ರಿಗೇ
ವಿವೇಕಾನಂದರು
ವಿವೇಕಾನಂದ-ರೆಂಬ
ವಿವೇಕಾನಂದರೇ
ವಿವೇ-ಕಿ-ಯಾ-ದ-ವನು
ವಿವೇಕೀ
ವಿವೇ-ಕೋ-ದ-ಯ-ವಾದ
ವಿವೇ-ಚನಾ
ವಿವೇ-ಚ-ನಾ-ಶಕ್ತಿ
ವಿವೇ-ಚ-ನಾ-ಶ-ಕ್ತಿ-ಯನ್ನು
ವಿವೇ-ಚಿ-ಸುತ್ತ
ವಿಶ-ದ-ವಾಗಿ
ವಿಶಾಲ
ವಿಶಾ-ಲ-ದೃ-ಷ್ಟಿಯ
ವಿಶಾ-ಲ-ದೃ-ಷ್ಟಿ-ಯಿಂದ
ವಿಶಾ-ಲ-ದೃ-ಷ್ಟಿಯೇ
ವಿಶಾ-ಲ-ಬು-ದ್ಧಿ-ಯ-ವನೂ
ವಿಶಾ-ಲ-ವಾಗಿ
ವಿಶಾ-ಲ-ವಾದ
ವಿಶಾ-ಲ-ವಾ-ದದ್ದು
ವಿಶಾ-ಲ-ವಾ-ದುದು
ವಿಶಿಷ್ಟ
ವಿಶಿ-ಷ್ಟ-ವಾದ
ವಿಶಿ-ಷ್ಟ-ವಾ-ದದ್ದು
ವಿಶಿ-ಷ್ಟಾ-ದ್ವೈತ
ವಿಶೇಷ
ವಿಶೇ-ಷಣ
ವಿಶೇ-ಷ-ಣ-ಗಳಿಂದ
ವಿಶೇ-ಷ-ತೆ-ಯಿತ್ತು
ವಿಶೇ-ಷ-ವಾಗಿ
ವಿಶೇ-ಷ-ವಾದ
ವಿಶ್ರ-ಮಿ-ಸ-ತೊ-ಡ-ಗಿದ
ವಿಶ್ರ-ಮಿಸಿ
ವಿಶ್ರ-ಮಿ-ಸು-ತ್ತಿ-ದ್ದರು
ವಿಶ್ರ-ಮಿ-ಸು-ತ್ತಿ-ರಲಿ
ವಿಶ್ರ-ಮಿ-ಸು-ವಂತೆ
ವಿಶ್ರಾಂ-ತಿ-ಗಳ
ವಿಶ್ರಾಂ-ತಿ-ಯ-ಲ್ಲಿದ್ದ
ವಿಶ್ರಾಂ-ತಿ-ಯಿಲ್ಲ
ವಿಶ್ರಾಂ-ತಿಯೇ
ವಿಶ್ಲೇ-ಷಣಾ
ವಿಶ್ಲೇ-ಷ-ಣೆಗೆ
ವಿಶ್ಲೇ-ಷಿಸಿ
ವಿಶ್ಲೇ-ಷಿ-ಸು-ತ್ತಿದ್ದ
ವಿಶ್ಲೇ-ಷಿ-ಸುವ
ವಿಶ್ವ-ಕೋಶ
ವಿಶ್ವ-ಜ್ಯೋ-ತಿ-ಯಾ-ದರು
ವಿಶ್ವದ
ವಿಶ್ವ-ದಲ್ಲಿ
ವಿಶ್ವ-ದಲ್ಲೇ
ವಿಶ್ವ-ದ-ಲ್ಲೊಂದು
ವಿಶ್ವ-ದಾ-ದ್ಯಂತ
ವಿಶ್ವ-ಧ-ರ್ಮಕ್ಕೆ
ವಿಶ್ವ-ನಾಥ
ವಿಶ್ವ-ನಾ-ಥ-ದತ್ತ
ವಿಶ್ವ-ನಾ-ಥನ
ವಿಶ್ವ-ನಾ-ಥ-ನಂ-ತಹ
ವಿಶ್ವ-ನಾ-ಥ-ನಂ-ತೆಯೇ
ವಿಶ್ವ-ನಾ-ಥ-ನನ್ನು
ವಿಶ್ವ-ನಾ-ಥ-ನಲ್ಲಿ
ವಿಶ್ವ-ನಾ-ಥ-ನಿಂದ
ವಿಶ್ವ-ನಾ-ಥ-ನಿ-ಗಿನ್ನೂ
ವಿಶ್ವ-ನಾ-ಥ-ನಿಗೂ
ವಿಶ್ವ-ನಾ-ಥ-ನಿಗೆ
ವಿಶ್ವ-ನಾ-ಥನು
ವಿಶ್ವ-ನಾ-ಥನೂ
ವಿಶ್ವ-ನಾ-ಥನೇ
ವಿಶ್ವ-ನಾ-ಥ-ನೇನೂ
ವಿಶ್ವ-ಬ್ರ-ಹ್ಮಾಂಡ
ವಿಶ್ವ-ಮಾನವ
ವಿಶ್ವ-ಮಾ-ಯಾ-ಧೀ-ಶ-ನಾ-ತನು
ವಿಶ್ವ-ವನ್ನೇ
ವಿಶ್ವ-ವ-ರಿ-ಯಲಿ
ವಿಶ್ವ-ವಿ-ಖ್ಯಾ-ತ-ರಾದ
ವಿಶ್ವ-ವಿ-ಖ್ಯಾ-ತರು
ವಿಶ್ವ-ವಿ-ಜೇತ
ವಿಶ್ವ-ವಿ-ಜೇ-ತ-ರಾದ
ವಿಶ್ವ-ವಿ-ದ್ಯಾ-ನಿ-ಲ-ಯವೂ
ವಿಶ್ವ-ವಿ-ದ್ಯಾ-ಲ-ಯ-ಗಳು
ವಿಶ್ವ-ವಿ-ದ್ಯಾ-ಲ-ಯದ
ವಿಶ್ವವೇ
ವಿಶ್ವ-ವ್ಯಾ-ಪ-ಕ-ವಾಗಿ
ವಿಶ್ವ-ಶಾಂ-ತಿಗೆ
ವಿಶ್ವಾ-ತ್ಮ-ನಾದ
ವಿಶ್ವಾ-ತ್ಮ-ಭಾ-ವದ
ವಿಶ್ವಾ-ತ್ಮ-ಭಾ-ವ-ವನ್ನು
ವಿಶ್ವಾ-ನಾಥ
ವಿಶ್ವಾಸ
ವಿಶ್ವಾ-ಸ-ಶ್ರ-ದ್ಧೆ-ಪ್ರೀ-ತಿ-ಪ್ರಾ-ಮಾ-ಣಿ-ಕ-ತೆ-ಗ-ಳಿಗೆ
ವಿಶ್ವಾ-ಸಕ್ಕೆ
ವಿಶ್ವಾ-ಸ-ಗಳು
ವಿಶ್ವಾ-ಸ-ದಿಂದ
ವಿಶ್ವಾ-ಸ-ದಿಂ-ದಿದ್ದು
ವಿಶ್ವಾ-ಸ-ದಿಂ-ದಿ-ರು-ತ್ತಿದ್ದ
ವಿಶ್ವಾ-ಸ-ಪಾತ್ರ
ವಿಶ್ವಾ-ಸ-ಪೂರ್ಣ
ವಿಶ್ವಾ-ಸ-ವಾಗಿ
ವಿಶ್ವಾ-ಸ-ವಿ-ಟ್ಟ-ವ-ರೆಂ-ದರೆ
ವಿಶ್ವಾ-ಸ-ವಿ-ಟ್ಟಿ-ದ್ದರು
ವಿಶ್ವಾ-ಸ-ವಿ-ಟ್ಟಿ-ದ್ದಾರೆ
ವಿಶ್ವಾ-ಸ-ವಿತ್ತು
ವಿಶ್ವಾ-ಸ-ವಿ-ರು-ವು-ದಾ-ದರೆ
ವಿಶ್ವಾ-ಸ-ವುಂ-ಟಾಗಿ
ವಿಶ್ವಾ-ಸವೇ
ವಿಶ್ವಾ-ಸ-ವೇನು
ವಿಶ್ವಾ-ಸ-ವೊಂ-ದಿ-ದ್ದರೆ
ವಿಶ್ವಾ-ಸಿ-ಗ-ನಾಗಿ
ವಿಶ್ವಾ-ಸಿ-ಗಳೂ
ವಿಶ್ವೇ-ಶ್ವ-ರನೇ
ವಿಷ
ವಿಷ-ಜಾ-ಲ-ದಿಂದ
ವಿಷಮ
ವಿಷಯ
ವಿಷ-ಯ-ಕ್ಕಾಗಿ
ವಿಷ-ಯಕ್ಕೂ
ವಿಷ-ಯಕ್ಕೆ
ವಿಷ-ಯ-ಗಳ
ವಿಷ-ಯ-ಗ-ಳಂತೂ
ವಿಷ-ಯ-ಗಳನ್ನು
ವಿಷ-ಯ-ಗಳನ್ನೂ
ವಿಷ-ಯ-ಗಳನ್ನೆಲ್ಲ
ವಿಷ-ಯ-ಗ-ಳನ್ನೇ
ವಿಷ-ಯ-ಗಳಲ್ಲಿ
ವಿಷ-ಯ-ಗ-ಳಲ್ಲೂ
ವಿಷ-ಯ-ಗ-ಳ-ಲ್ಲೊಂದು
ವಿಷ-ಯ-ಗ-ಳಿಗೆ
ವಿಷ-ಯ-ಗಳು
ವಿಷ-ಯ-ಗ-ಳೆಂ-ದರೆ
ವಿಷ-ಯ-ಗ-ಳೆಲ್ಲ
ವಿಷ-ಯದ
ವಿಷ-ಯ-ದ-ಲ್ಲಂತೂ
ವಿಷ-ಯ-ದ-ಲ್ಲಾ-ದರೂ
ವಿಷ-ಯ-ದ-ಲ್ಲಾ-ದರೆ
ವಿಷ-ಯ-ದಲ್ಲಿ
ವಿಷ-ಯ-ದ-ಲ್ಲಿನ್ನೂ
ವಿಷ-ಯ-ದಲ್ಲೂ
ವಿಷ-ಯ-ದಲ್ಲೇ
ವಿಷ-ಯ-ವಂತೂ
ವಿಷ-ಯ-ವನ್ನು
ವಿಷ-ಯ-ವನ್ನೂ
ವಿಷ-ಯ-ವ-ನ್ನೆಲ್ಲ
ವಿಷ-ಯ-ವಾ-ಗಲಿ
ವಿಷ-ಯ-ವಾಗಿ
ವಿಷ-ಯ-ವಾ-ಗಿಯೇ
ವಿಷ-ಯ-ವಾಗೇ
ವಿಷ-ಯ-ವಾದ
ವಿಷ-ಯ-ವಾ-ದರೂ
ವಿಷ-ಯ-ವಾ-ಯಿ-ತಲ್ಲ
ವಿಷ-ಯ-ವಿದೆ
ವಿಷ-ಯವೂ
ವಿಷ-ಯ-ವೆಂದರೆ
ವಿಷ-ಯ-ವೆಲ್ಲ
ವಿಷ-ಯವೇ
ವಿಷ-ಯ-ವೇನೆಂದರೆ
ವಿಷ-ಯ-ವೊಂ-ದನ್ನು
ವಿಷ-ಯ-ಸಂ-ಗ್ರ-ಹಣೆ
ವಿಷ-ಯ-ಸು-ಖದ
ವಿಷ-ಯ-ಸ್ಪ-ಷ್ಟ-ತೆ-ಗಾಗಿ
ವಿಷಾ-ದ-ನೀ-ಯ-ವಾ-ದ-ದ್ದೇ-ನಿದೆ
ವಿಷ್ಣು
ವಿಷ್ಣು-ವಿನ
ವಿಸ-ರ್ಜನೆ
ವಿಸ್ತ-ರಿ-ಸ-ಬೇ-ಕಾ-ದ-ವನು
ವಿಸ್ತ-ರಿ-ಸಿ-ಕೊ-ಳ್ಳು-ವು-ದರ
ವಿಸ್ತ-ರಿ-ಸು-ವು-ದ-ಕ್ಕಾಗಿ
ವಿಸ್ತಾ-ರ-ವಾದ
ವಿಸ್ಮಯ
ವಿಸ್ಮ-ಯ-ಸಂ-ತೋ-ಷ-ಗಳಿಂದ
ವಿಸ್ಮ-ಯ-ಗ-ಳ-ಲ್ಲೊಂ-ದಾದ
ವಿಸ್ಮ-ಯ-ಗೊಂಡ
ವಿಸ್ಮ-ಯ-ಗೊಂಡು
ವಿಸ್ಮ-ಯ-ದಿಂದ
ವಿಸ್ಮ-ಯ-ಮೂ-ಕ-ನಾಗಿ
ವಿಸ್ಮ-ಯ-ಮೂ-ಕ-ರಾಗಿ
ವಿಸ್ಮ-ಯ-ಮೂ-ಕ-ರಾ-ಗಿ-ಬಿ-ಟ್ಟರು
ವಿಸ್ಮ-ಯ-ಮೂ-ಕ-ರಾ-ಗು-ತ್ತಾರೆ
ವಿಸ್ಮ-ಯಾ-ನಂ-ದ-ಗೊಂಡು
ವಿಸ್ಮ-ಯಾ-ನಂ-ದ-ಭ-ರಿ-ತ-ರಾ-ದರು
ವಿಸ್ಮಿ-ತ-ನಾಗಿ
ವಿಸ್ಮಿ-ತ-ನಾದೆ
ವಿಹ-ರಿ-ಸು-ತ್ತಿದ್ದ
ವಿಹ-ರಿ-ಸು-ತ್ತಿ-ದ್ದಾನೆ
ವಿಹ-ರಿ-ಸು-ತ್ತಿ-ರುವ
ವಿಹ-ರಿ-ಸು-ತ್ತಿ-ರು-ವ-ವನು
ವಿಹಾರ
ವಿಹಾ-ರಕ್ಕೆ
ವಿಹಾ-ರದ
ವೀಕ್ಷ-ಣ-ಮಾ-ತ್ರ-ದಿಂ-ದಲೇ
ವೀಕ್ಷಿಸಿ
ವೀಕ್ಷಿ-ಸಿ-ದರು
ವೀಕ್ಷಿ-ಸು-ವು-ದರ
ವೀಣಾ-ನಾ-ದ-ವನ್ನು
ವೀರ
ವೀರ-ಕೇ-ಸರಿ
ವೀರ-ತ್ವಕ್ಕೆ
ವೀರ-ರ-ಲ್ಲವೆ
ವೀರರೇ
ವೀರ-ಸಂನ್ಯಾಸಿ
ವೀರ-ಸಂ-ನ್ಯಾ-ಸಿ-ಗ-ಳಾದ
ವೀರ-ಸಂ-ನ್ಯಾ-ಸಿಯ
ವೀರಾ-ಗ್ರ-ಣಿ-ಗಳು
ವೀರಾ-ಧಿ-ವೀ-ರ-ನಾದ
ವೀರಾ-ವೇ-ಶ-ದಿಂದ
ವೀರೇ-ಶ್ವರ
ವುದ-ಕ್ಕಾ-ಗಿಯೇ
ವುದ-ಕ್ಕೋ-ಸ್ಕರ
ವುದನ್ನು
ವುದ-ರ-ಲ್ಲಿ-ದ್ದರು
ವುದ-ರಿಂ-ದಲ್ಲ
ವುದಲ್ಲ
ವುದಾ-ದಲ್ಲಿ
ವುದಿಲ್ಲ
ವುದಿ-ಲ್ಲವೋ
ವುದು
ವುದೇ
ವುದೋ
ವೃಕ್ಷ-ಕಾ-ರ್ಯಕೆ
ವೃಕ್ಷ-ಗಳ
ವೃಕ್ಷದ
ವೃತ್ತ-ಪ-ತ್ರಿ-ಕೆ-ಗಳನ್ನು
ವೃತ್ತಾಂ-ತ-ವ-ನ್ನೆಲ್ಲ
ವೃತ್ತಾ-ಕಾ-ರ-ವಾಗಿ
ವೃತ್ತಿ
ವೃತ್ತಿಯ
ವೃತ್ತಿ-ಯಲ್ಲಿ
ವೃತ್ತಿ-ಯಿಂದ
ವೃತ್ತಿಯೋ
ವೃದ್ಧ
ವೃದ್ಧಿ
ವೃದ್ಧಿ-ಗೊ-ಳಿ-ಸಿ-ಕೊ-ಳ್ಳು-ವಂ-ತಾ-ಯಿತು
ವೃದ್ಧಿ-ಗೊ-ಳಿ-ಸು-ವುದೇ
ವೃದ್ಧಿ-ಗೊ-ಳ್ಳ-ಲಾ-ರಂ-ಭಿ-ಸಿ-ದ್ದುವು
ವೃದ್ಧಿ-ಯಾಗ
ವೃದ್ಧಿ-ಯಾ-ಗು-ತ್ತಿ-ದ್ದಂ-ತೆಯೇ
ವೃದ್ಧಿ-ಯಾ-ಗು-ವು-ದಿಲ್ಲ
ವೆಂಥದು
ವೆಂದರೆ
ವೆಂದು
ವೆಂಬಂತೆ
ವೆಂಬುದು
ವೆಚ್ಚಕ್ಕೆ
ವೆಚ್ಚ-ಮಾ-ಡು-ತ್ತಿ-ದ್ದರು
ವೆಚ್ಚ-ವನ್ನೂ
ವೆನಿ-ಸಿತು
ವೆನ್ನು-ವುದು
ವೆಯೂ
ವೆಲ್ಲ
ವೇಗ
ವೇಗ-ಗೊ-ಳ್ಳು-ತ್ತದೆ
ವೇಗ-ದಲ್ಲಿ
ವೇಗ-ವಾಗಿ
ವೇಣೀ
ವೇದ
ವೇದ-ಶಾ-ಸ್ತ್ರ-ಪು-ರಾ-ಣ-ದ-ರ್ಶ-ನ-ದಾ-ಚೆಗೇ
ವೇದ-ಶಾ-ಸ್ತ್ರ-ಗಳ
ವೇದ-ಶಾ-ಸ್ತ್ರ-ಗಳಲ್ಲಿ
ವೇದ-ಗಳ
ವೇದ-ಗ-ಳಲ್ಲೂ
ವೇದ-ಗಳು
ವೇದ-ಗಳೇ
ವೇದ-ತ-ನು-ಮು-ಜ್ಝಿತ
ವೇದನೆ
ವೇದ-ನೆಯ
ವೇದ-ನೆ-ಯಾ-ಗು-ತ್ತಿತ್ತು
ವೇದ-ವಾ-ಕ್ಯ-ಗಳ
ವೇದ-ಶಾ-ಸ್ತ್ರ-ಗಳ
ವೇದಾಂತ
ವೇದಾಂ-ತ-ಗಳನ್ನು
ವೇದಾಂ-ತ-ತ-ತ್ತ್ವ-ಬೋ-ಧನೆ
ವೇದಾಂ-ತದ
ವೇದಾಂ-ತ-ದಲ್ಲಿ
ವೇದಾಂ-ತ-ವನ್ನು
ವೇದಾಂ-ತವು
ವೇದಾಂ-ತವೇ
ವೇದಾಂತಿ
ವೇದಾಂ-ತಿ-ಯಾದ
ವೇದಾ-ಧ್ಯ-ಯ-ನದ
ವೇದಿ-ಕೆ-ಗಳ
ವೇದಿ-ಕೆಯ
ವೇದೋ-ಪ-ನಿ-ಷ-ತ್ತು-ಗಳ
ವೇನಲ್ಲ
ವೇನಿ-ರ-ಬ-ಹುದು
ವೇನು
ವೇನೂ
ವೇಳೆ
ವೇಳೆ-ಗಾ-ಗಲೇ
ವೇಳೆಗೆ
ವೇಳೆ-ಯಲ್ಲಿ
ವೇಳೆ-ಯಲ್ಲೂ
ವೇಳೆ-ಯ-ಲ್ಲೆಲ್ಲ
ವೇಳೆ-ಯೆಲ್ಲ
ವೇಶ್ಯಾ-ಸಂ-ಪ-ರ್ಕಕ್ಕೆ
ವೇಶ್ಯೆ-ಯನ್ನು
ವೇಶ್ಯೆ-ಯರ
ವೇಷ-ಪ-ಲ್ಲಟ
ವೇಷ-ಭೂ-ಷ-ಣ-ದ-ಲ್ಲಾ-ಗಲಿ
ವೈಕುಂಠ
ವೈಕುಂ-ಠ-ನಾಥ
ವೈಕುಂ-ಠ-ನಾ-ಥ-ನಿಗೆ
ವೈಕುಂ-ಠ-ನಿಗೆ
ವೈಕುಂ-ಠನೂ
ವೈಖರಿ
ವೈಚಾ-ರಿಕ
ವೈಚಾ-ರಿ-ಕ-ಬು-ದ್ಧಿಗೆ
ವೈಚಿ-ತ್ರ್ಯ-ವಿತ್ತು
ವೈಚಿ-ತ್ರ್ಯ-ವೇನು
ವೈಜ್ಞಾ-ನಿಕ
ವೈದ್ಯ
ವೈದ್ಯ-ಕೀಯ
ವೈದ್ಯನ
ವೈದ್ಯ-ನಾಥ
ವೈದ್ಯ-ನಾ-ಥದ
ವೈದ್ಯ-ನಾ-ಥ-ದಲ್ಲಿ
ವೈದ್ಯ-ನಾ-ಥ-ದಿಂದ
ವೈದ್ಯ-ನಾದ
ವೈದ್ಯರ
ವೈದ್ಯ-ರನ್ನು
ವೈದ್ಯ-ರಾದ
ವೈದ್ಯ-ರಾ-ದೀರಿ
ವೈದ್ಯ-ರಿಗೂ
ವೈದ್ಯ-ರಿಗೆ
ವೈದ್ಯರು
ವೈಭ-ವ-ವೈ-ಶಿಷ್ಟ್ಯ
ವೈಭ-ವದ
ವೈಭ-ವ-ಪೂರ್ಣ
ವೈಭ-ವ-ವನ್ನು
ವೈಭೋ-ಗದ
ವೈಮ-ನಸ್ಯ
ವೈಯ-ಕ್ತಿಕ
ವೈಯ-ಕ್ತಿ-ಕ-ವಾಗಿ
ವೈಯ-ಕ್ತಿ-ಕ-ವಾದ
ವೈರಾಗ್ಯ
ವೈರಾ-ಗ್ಯ-ಗಳ
ವೈರಾ-ಗ್ಯ-ಗ-ಳನ್ನೇ
ವೈರಾ-ಗ್ಯದ
ವೈರಾ-ಗ್ಯ-ದಂ-ತಲ್ಲ
ವೈರಾ-ಗ್ಯ-ಪ-ರ-ವಾದ
ವೈರಾ-ಗ್ಯ-ಪ್ರ-ಚೋ-ದ-ಕ-ವಾ-ಗಿ-ದ್ದುವು
ವೈರಾ-ಗ್ಯ-ಬುದ್ಧಿ
ವೈರಾ-ಗ್ಯ-ಬು-ದ್ಧಿ-ಇವು
ವೈರಾ-ಗ್ಯ-ಭಾ-ವ-ವನ್ನು
ವೈರಾ-ಗ್ಯ-ವನ್ನು
ವೈರಾ-ಗ್ಯ-ವಲ್ಲ
ವೈರಾ-ಗ್ಯ-ವು-ದಿಸು
ವೈರಾ-ಗ್ಯ-ವೊಂದು
ವೈರಾ-ಗ್ಯ-ಶಾಲಿ
ವೈರಿ-ಗಳನ್ನೂ
ವೈರಿ-ಗಳು
ವೈವಾ-ಹಿಕ
ವೈವಿ-ಧ್ಯ-ಪೂ-ರ್ಣ-ವಾದ
ವೈವಿ-ಧ್ಯ-ಮಯ
ವೈಶಾ-ಲ್ಯದ
ವೈಶಾ-ಲ್ಯ-ದೆ-ಡೆಗೆ
ವೈಶಿಷ್ಟ್ಯ
ವೈಶಿ-ಷ್ಟ್ಯ-ವೇನು
ವೈಶ್ಯ-ನಾ-ದರೂ
ವೈಶ್ಯನೋ
ವೈಷ್ಣವ
ವೈಷ್ಣ-ವ-ಇ-ವ-ರಲ್ಲಿ
ವೈಷ್ಣ-ವ-ಚ-ರಣ
ವೈಷ್ಣ-ವ-ನನ್ನೂ
ವೈಷ್ಣ-ವ-ಶಾ-ಸ್ತ್ರ-ಗ-ಳಿಗೆ
ವೊಂದನ್ನು
ವೊಂದಿತ್ತು
ವೊಮ್ಮೆ
ವ್ಯಂಗ್ಯದ
ವ್ಯಂಗ್ಯ-ವಾಗಿ
ವ್ಯಕ್ತ
ವ್ಯಕ್ತ-ಗೊಂ-ಡಿ-ದೆ-ಯೆಂದ
ವ್ಯಕ್ತ-ಗೊ-ಳಿ-ಸು-ತ್ತಿದ್ದ
ವ್ಯಕ್ತ-ಗೊ-ಳಿ-ಸು-ತ್ತಿ-ದ್ದುದು
ವ್ಯಕ್ತ-ಗೊ-ಳ್ಳು-ವು-ದಿಲ್ಲ
ವ್ಯಕ್ತ-ನಾ-ಗ-ತೊ-ಡ-ಗಿದ್ದ
ವ್ಯಕ್ತ-ನಾ-ಗು-ತ್ತಾನೆ
ವ್ಯಕ್ತ-ಪ-ಡಿ-ಸ-ಬ-ಹುದು
ವ್ಯಕ್ತ-ಪ-ಡಿಸಿ
ವ್ಯಕ್ತ-ಪ-ಡಿ-ಸಿದ
ವ್ಯಕ್ತ-ಪ-ಡಿ-ಸಿ-ದರು
ವ್ಯಕ್ತ-ಪ-ಡಿ-ಸಿ-ದಳು
ವ್ಯಕ್ತ-ಪ-ಡಿ-ಸಿ-ದಾಗ
ವ್ಯಕ್ತ-ಪ-ಡಿ-ಸಿಯೂ
ವ್ಯಕ್ತ-ಪ-ಡಿ-ಸಿ-ಯೂ-ಬಿಟ್ಟ
ವ್ಯಕ್ತ-ಪ-ಡಿ-ಸು-ತ್ತಿದ್ದ
ವ್ಯಕ್ತ-ಪ-ಡಿ-ಸುವ
ವ್ಯಕ್ತ-ಪ-ಡಿ-ಸು-ವಾಗ
ವ್ಯಕ್ತ-ಪ-ಡಿ-ಸು-ವುದ
ವ್ಯಕ್ತ-ಪ-ಡಿ-ಸು-ವುದೇ
ವ್ಯಕ್ತ-ರಾ-ಗ-ಬ-ಲ್ಲರು
ವ್ಯಕ್ತ-ವಾ-ಗ-ತೊ-ಡ-ಗಿತು
ವ್ಯಕ್ತ-ವಾ-ಗ-ತೊ-ಡ-ಗಿ-ದ್ದುವು
ವ್ಯಕ್ತ-ವಾ-ಗಲು
ವ್ಯಕ್ತ-ವಾ-ಗಿತ್ತು
ವ್ಯಕ್ತ-ವಾ-ಗಿದೆ
ವ್ಯಕ್ತ-ವಾ-ಗಿರು
ವ್ಯಕ್ತ-ವಾ-ಗಿ-ರು-ವ-ವಳು
ವ್ಯಕ್ತ-ವಾ-ಗುತ್ತ
ವ್ಯಕ್ತ-ವಾ-ಗು-ತ್ತದೆ
ವ್ಯಕ್ತ-ವಾ-ಗು-ತ್ತಿದ್ದ
ವ್ಯಕ್ತ-ವಾ-ಗು-ತ್ತಿ-ದ್ದವು
ವ್ಯಕ್ತ-ವಾ-ಗು-ತ್ತಿ-ರು-ವಂತೆ
ವ್ಯಕ್ತ-ವಾದ
ವ್ಯಕ್ತ-ವಾ-ದಂ-ತಹ
ವ್ಯಕ್ತಿ
ವ್ಯಕ್ತಿ-ಗಳ
ವ್ಯಕ್ತಿ-ಗ-ಳಂ-ತಲ್ಲ
ವ್ಯಕ್ತಿ-ಗ-ಳಂತೆ
ವ್ಯಕ್ತಿ-ಗಳನ್ನು
ವ್ಯಕ್ತಿ-ಗಳನ್ನೂ
ವ್ಯಕ್ತಿ-ಗಳಲ್ಲಿ
ವ್ಯಕ್ತಿ-ಗ-ಳಿಗೂ
ವ್ಯಕ್ತಿ-ಗ-ಳಿಗೆ
ವ್ಯಕ್ತಿ-ಗ-ಳಿ-ದ್ದಾರೆ
ವ್ಯಕ್ತಿ-ಗಳು
ವ್ಯಕ್ತಿ-ಗ-ಳೆಂ-ದರೆ
ವ್ಯಕ್ತಿ-ಗ-ಳೆ-ನ್ನು-ವುದು
ವ್ಯಕ್ತಿಗೆ
ವ್ಯಕ್ತಿ-ಚಿ-ತ್ರ-ಣದ
ವ್ಯಕ್ತಿತ್ವ
ವ್ಯಕ್ತಿ-ತ್ವ-ಜೀ-ವ-ನ
ವ್ಯಕ್ತಿ-ತ್ವ-ಇ-ವು-ಗಳು
ವ್ಯಕ್ತಿ-ತ್ವಕ್ಕೆ
ವ್ಯಕ್ತಿ-ತ್ವ-ಗಳನ್ನು
ವ್ಯಕ್ತಿ-ತ್ವದ
ವ್ಯಕ್ತಿ-ತ್ವ-ದಲ್ಲಿ
ವ್ಯಕ್ತಿ-ತ್ವ-ದಿಂದ
ವ್ಯಕ್ತಿ-ತ್ವ-ನಿ-ರ್ಮಾ-ಣದ
ವ್ಯಕ್ತಿ-ತ್ವ-ವ-ನ್ನಾ-ಗಲಿ
ವ್ಯಕ್ತಿ-ತ್ವ-ವನ್ನು
ವ್ಯಕ್ತಿ-ತ್ವ-ವನ್ನೂ
ವ್ಯಕ್ತಿ-ತ್ವ-ವನ್ನೇ
ವ್ಯಕ್ತಿ-ತ್ವ-ವಿನ್ನೂ
ವ್ಯಕ್ತಿ-ತ್ವವೂ
ವ್ಯಕ್ತಿ-ತ್ವ-ವೆಲ್ಲ
ವ್ಯಕ್ತಿ-ತ್ವವೇ
ವ್ಯಕ್ತಿ-ನೀವು
ವ್ಯಕ್ತಿ-ಪೂ-ಜೆ-ಗ-ಳಿಗೂ
ವ್ಯಕ್ತಿಯ
ವ್ಯಕ್ತಿ-ಯನ್ನು
ವ್ಯಕ್ತಿ-ಯಲ್ಲಿ
ವ್ಯಕ್ತಿ-ಯ-ವಿ-ಶೇ-ಷತಃ
ವ್ಯಕ್ತಿ-ಯಾಗಿ
ವ್ಯಕ್ತಿ-ಯಾ-ಗಿ-ಬಿ-ಟ್ಟಿದ್ದ
ವ್ಯಕ್ತಿ-ಯಾಗು
ವ್ಯಕ್ತಿಯು
ವ್ಯಕ್ತಿ-ಯೆಂ-ಬಂತೆ
ವ್ಯಕ್ತಿ-ಯೆ-ನ್ನ-ಬೇಕು
ವ್ಯಕ್ತಿಯೇ
ವ್ಯಕ್ತಿ-ಯೊ-ಡನೆ
ವ್ಯಕ್ತಿ-ಯೊಬ್ಬ
ವ್ಯತಿ-ರಿ-ಕ್ತ-ವಾಗಿ
ವ್ಯತಿ-ರಿ-ಕ್ತ-ವಾದ
ವ್ಯತ್ಯಾಸ
ವ್ಯತ್ಯಾ-ಸ-ವಾ-ಗ-ದಂತೆ
ವ್ಯತ್ಯಾ-ಸ-ವಿತ್ತು
ವ್ಯತ್ಯಾ-ಸ-ವಿದೆ
ವ್ಯತ್ಯಾ-ಸ-ವಿಲ್ಲ
ವ್ಯಥೆ
ವ್ಯಭಿ-ಚಾರ
ವ್ಯಭಿ-ಚಾ-ರಿಯಾ
ವ್ಯಯ
ವ್ಯಯಿ-ಸು-ತ್ತಲೇ
ವ್ಯರ್ಥ
ವ್ಯರ್ಥ-ಇದು
ವ್ಯರ್ಥ-ವಾಗಿ
ವ್ಯರ್ಥ-ವಾ-ಯಿತು
ವ್ಯವ-ಧಾ-ನವೇ
ವ್ಯವ-ಸ್ಥಿ-ತ-ಗೊ-ಳಿ-ಸಿ-ಕೊಂಡು
ವ್ಯವಸ್ಥೆ
ವ್ಯವ-ಸ್ಥೆ-ಮಾ-ಡಿದ
ವ್ಯವ-ಸ್ಥೆ-ಮಾಡು
ವ್ಯವ-ಸ್ಥೆ-ಯನ್ನು
ವ್ಯವ-ಸ್ಥೆ-ಯನ್ನೂ
ವ್ಯವ-ಸ್ಥೆ-ಯಾ-ಗ-ಬೇ-ಕಲ್ಲ
ವ್ಯವ-ಹಾರ
ವ್ಯವ-ಹಾ-ರ-ಗಳನ್ನು
ವ್ಯವ-ಹಾ-ರ-ಗಳನ್ನೆಲ್ಲ
ವ್ಯವ-ಹಾ-ರ-ದಲ್ಲಿ
ವ್ಯವ-ಹಾ-ರ-ವ-ನ್ನಿ-ಟ್ಟು-ಕೊಂ-ಡಿ-ದ್ದರು
ವ್ಯವ-ಹಾ-ರ-ವನ್ನು
ವ್ಯವ-ಹಾ-ರ-ವ-ನ್ನೆಲ್ಲ
ವ್ಯವ-ಹಾ-ರ-ವೆಲ್ಲ
ವ್ಯವ-ಹಾ-ರ-ವೆ-ಲ್ಲ-ಸಾ-ರ್ವ-ಜ-ನಿಕ
ವ್ಯಾಕ-ರಣ
ವ್ಯಾಕ-ರ-ಣವೂ
ವ್ಯಾಕುಲ
ವ್ಯಾಕು-ಲ-ಗೊಂ-ಡಾಗ
ವ್ಯಾಕು-ಲ-ಗೊಂಡು
ವ್ಯಾಕು-ಲ-ಗೊ-ಳ್ಳು-ತ್ತಿ-ದ್ದುದು
ವ್ಯಾಕು-ಲತೆ
ವ್ಯಾಕು-ಲ-ತೆಯ
ವ್ಯಾಕು-ಲ-ತೆ-ಯನ್ನು
ವ್ಯಾಕು-ಲ-ತೆ-ಯಿಂದ
ವ್ಯಾಕು-ಲ-ತೆ-ಯಿ-ದ್ದರೆ
ವ್ಯಾಕು-ಲ-ತೆ-ಯಿ-ರುವ
ವ್ಯಾಕು-ಲ-ತೆ-ಯಿ-ರು-ವಲ್ಲಿ
ವ್ಯಾಕು-ಲ-ತೆ-ಯುಂ-ಟಾ-ಗಿದೆ
ವ್ಯಾಕು-ಲ-ನಾ-ಗಿ-ದ್ದಾನೆ
ವ್ಯಾಕು-ಲ-ನಾ-ಗಿ-ರ-ಲಿ-ಕ್ಕಿಲ್ಲ
ವ್ಯಾಕು-ಲ-ರಾಗಿ
ವ್ಯಾಕು-ಲಿ-ತ-ನಾ-ಗಿದ್ದ
ವ್ಯಾಕು-ಲಿ-ತ-ರಾ-ಗ-ಬೇಕು
ವ್ಯಾಕು-ಲಿ-ತ-ರಾ-ಗಿ-ದ್ದರು
ವ್ಯಾಖ್ಯಾನ
ವ್ಯಾಜ್ಯದ
ವ್ಯಾಧಿ
ವ್ಯಾಧಿಯ
ವ್ಯಾಪ-ಕ-ವಾಗಿ
ವ್ಯಾಪಾ-ರಿ-ಗಳು
ವ್ಯಾಪಾ-ರಿ-ಯ-ಲ್ಲದೆ
ವ್ಯಾಪಾ-ರೋ-ದ್ಯಮ
ವ್ಯಾಪಿ-ಸಿ-ಕೊಂ-ಡಿತು
ವ್ಯಾಪಿ-ಸಿತು
ವ್ಯಾಪಿ-ಸಿ-ರುವ
ವ್ಯಾಯ-ಮ-ಶಾ-ಲೆಯ
ವ್ಯಾಯಾಮ
ವ್ಯಾಯಾ-ಮ-ಮ-ಲ್ಲ-ಯು-ದ್ಧಾ-ದಿ-ಗಳ
ವ್ಯಾಯಾ-ಮ-ಗಳ
ವ್ಯಾಯಾ-ಮ-ಶಾಲೆ
ವ್ಯಾಯಾ-ಮ-ಶಾ-ಲೆಗೆ
ವ್ಯಾಯಾ-ಮ-ಶಾ-ಲೆಯ
ವ್ಯಾವ-ಹಾ-ರಿಕ
ವ್ಯಾವೋಹ
ವ್ಯಾಸ-ವಾ-ಲ್ಮೀ-ಕಿ-ಗಳೂ
ವ್ಯಾಸನ
ವ್ಯೋಂ
ವ್ರಜೇಂ-ದ್ರ-ನಾಥ
ವ್ರತ
ವ್ರತ-ಕ್ಕೀಗ
ವ್ರತ-ಗಳನ್ನು
ವ್ರತ-ದಂತೆ
ವ್ರತ-ಧಾ-ರಣೆ
ವ್ರತ-ಧಾ-ರಿ-ಯಾಗಿ
ವ್ರತ-ನಿ-ಯ-ಮ-ಗ-ಳ-ನ್ನಾ-ಚ-ರಿಸಿ
ವ್ರತ-ವನ್ನು
ಶಂಕ-ರಾ-ಚಾ-ರ್ಯ-ರನ್ನು
ಶಂಕ-ರಾ-ಚಾ-ರ್ಯರು
ಶಂಕಿ-ಸ-ಲಾ-ಗ-ಲಿಲ್ಲ
ಶಂಕಿಸಿ
ಶಂಕಿ-ಸಿ-ದ್ದರೂ
ಶಂಕಿ-ಸಿ-ರ-ಬ-ಹುದು
ಶಂಕಿ-ಸು-ವಂ-ತಹ
ಶಂಕೆ
ಶಂಖ-ಜಾ-ಗ-ಟೆ-ಗಳನ್ನು
ಶಕ್ತ-ನು-ಅವ
ಶಕ್ತಿ
ಶಕ್ತಿ-ಆ-ಧ್ಯಾ-ತ್ಮಿಕ
ಶಕ್ತಿ-ಉ-ತ್ಸಾಹ
ಶಕ್ತಿ-ಉ-ತ್ಸಾ-ಹ-ಗಳು
ಶಕ್ತಿ-ಸಾ-ಧ್ಯ-ತೆ-ಗ-ಳ-ನ್ನು-ಳ್ಳ-ವನು
ಶಕ್ತಿ-ಸಾ-ಮ-ರ್ಥ್ಯ-ಗಳ
ಶಕ್ತಿ-ಸಾ-ಹ-ಸ-ಗ-ಳೆಲ್ಲ
ಶಕ್ತಿ-ಕೇಂ-ದ್ರ-ವಾಗಿ
ಶಕ್ತಿ-ಕೇಂ-ದ್ರವೇ
ಶಕ್ತಿ-ಗಳ
ಶಕ್ತಿ-ಗಳಿಂದ
ಶಕ್ತಿ-ಗಳು
ಶಕ್ತಿ-ತ-ರಂ-ಗವು
ಶಕ್ತಿ-ಪೂರ್ಣ
ಶಕ್ತಿ-ಪೂ-ರ್ಣ-ವಾ-ಗಿದೆ
ಶಕ್ತಿ-ಪೂ-ರ್ಣ-ವಾದ
ಶಕ್ತಿ-ಪ್ರ-ದ-ರ್ಶನ
ಶಕ್ತಿ-ಮೀರಿ
ಶಕ್ತಿಯ
ಶಕ್ತಿ-ಯ-ಡ-ಗಿ-ರುವ
ಶಕ್ತಿ-ಯನ್ನು
ಶಕ್ತಿ-ಯನ್ನೂ
ಶಕ್ತಿ-ಯ-ನ್ನೆಲ್ಲ
ಶಕ್ತಿ-ಯಾಗಿ
ಶಕ್ತಿ-ಯಾ-ಗಿತ್ತು
ಶಕ್ತಿ-ಯಾದ
ಶಕ್ತಿ-ಯಿಂದ
ಶಕ್ತಿ-ಯಿಂ-ದಲೂ
ಶಕ್ತಿ-ಯಿತ್ತು
ಶಕ್ತಿ-ಯಿ-ದೆಯೆ
ಶಕ್ತಿ-ಯಿ-ದ್ದಾಳೆ
ಶಕ್ತಿ-ಯಿ-ರು-ವ-ವ-ರನ್ನು
ಶಕ್ತಿ-ಯಿಲ್ಲ
ಶಕ್ತಿ-ಯಿ-ಲ್ಲ-ದಂ-ತಾ-ಗಿದೆ
ಶಕ್ತಿ-ಯುತ
ಶಕ್ತಿ-ಯು-ತ-ವಾಗಿ
ಶಕ್ತಿಯೂ
ಶಕ್ತಿ-ಯೆಲ್ಲ
ಶಕ್ತಿ-ಯೊಂ-ದರ
ಶಕ್ತಿ-ಯೊಂದು
ಶಕ್ತಿ-ಶಾಲಿ
ಶಕ್ತಿ-ಶಾ-ಲಿ-ಗ-ಳಾದ
ಶಕ್ತಿ-ಶಾ-ಲಿ-ಯಾಗಿ
ಶಕ್ತಿ-ಶಾ-ಲಿ-ಯಾದ
ಶತ-ದಳ
ಶತ-ಪಾಲು
ಶತ-ಮಾ-ನಕ್ಕೆ
ಶತ-ಮಾ-ನ-ಗಳ
ಶತ-ಮಾ-ನ-ಗಳಲ್ಲಿ
ಶತ-ಮಾ-ನ-ಗ-ಳಿಂ-ದಲೂ
ಶತ-ಮಾ-ನ-ಗ-ಳಿಗೂ
ಶತ-ಮಾ-ನದ
ಶತಾ-ಯ-ಗ-ತಾಯ
ಶತ್ರುವು
ಶನಿ-ವಾರ
ಶಬ್ದ
ಶಬ್ದ-ಕೋ-ಶ-ದಲ್ಲಿ
ಶಬ್ದ-ಕೋ-ಶ-ವೆಲ್ಲ
ಶಬ್ದಕ್ಕೆ
ಶಬ್ದ-ಗಳ
ಶಬ್ದ-ಗಳನ್ನು
ಶಬ್ದ-ಗಳಿಂದ
ಶಬ್ದ-ಗಳು
ಶಬ್ದದ
ಶಬ್ದ-ವನ್ನು
ಶಬ್ದ-ವನ್ನೂ
ಶಬ್ದ-ವನ್ನೇ
ಶಬ್ದವೂ
ಶಮ
ಶಮ-ನ-ಗೊ-ಳಿ-ಸಲು
ಶಯ್ಯೆ-ಯಲ್ಲೂ
ಶರ-ಚ್ಚಂದ್ರ
ಶರ-ಚ್ಚಂ-ದ್ರ-ಗುಪ್ತ
ಶರ-ಚ್ಚಂ-ದ್ರನ
ಶರ-ಚ್ಚಂ-ದ್ರ-ನಿಗೆ
ಶರ-ಚ್ಚಂ-ದ್ರನೂ
ಶರ-ಚ್ಚಂ-ದ್ರ-ನೆಂದ
ಶರ-ಣರ
ಶರ-ಣರು
ಶರ-ಣ-ರೆಂಬ
ಶರಣಾ
ಶರ-ಣಾ-ಗ-ತ-ರಾಗಿ
ಶರ-ಣಾ-ಗ-ತ-ರಾದ
ಶರ-ಣಾ-ಗ-ತ-ರಾ-ದ-ವರೆ-ಲ್ಲ-ರಿಗೂ
ಶರ-ಣಾ-ಗ-ತಿ-ಇವೇ
ಶರ-ಣಾ-ಗ-ತಿ-ಭಾ-ವ-ದಿಂದ
ಶರ-ಣಾಗಿ
ಶರ-ಣಾ-ಗಿ-ದ್ದು-ಕೊಂಡು
ಶರ-ಣಾ-ಗಿ-ದ್ದೇನೆ
ಶರ-ಣಾಗು
ಶರ-ಣಾ-ಗು-ವು-ದ-ರಿಂ-ದಲ್ಲ
ಶರಣು
ಶರತ್
ಶರೀರ
ಶರೀ-ರ-ಇಂ-ದ್ರಿ-ಯ-ಮ-ನಸ್ಸು
ಶರೀ-ರ-ಮಾ-ತು-ಮ-ನ-ಸ್ಸು-ಗಳಿಂದ
ಶರೀ-ರಕ್ಕೆ
ಶರೀ-ರ-ಗಳನ್ನು
ಶರೀ-ರ-ತ್ಯಾಗ
ಶರೀ-ರದ
ಶರೀ-ರ-ದಂತೆ
ಶರೀ-ರ-ದಲ್ಲಿ
ಶರೀ-ರ-ದ-ಲ್ಲಿದೆ
ಶರೀ-ರ-ದ-ಲ್ಲಿ-ರಲು
ಶರೀ-ರ-ದಲ್ಲೂ
ಶರೀ-ರ-ದಿಂದ
ಶರೀ-ರ-ದೊ-ಳಗೆ
ಶರೀ-ರ-ಧಾ-ರಿ-ಗ-ಳಾದ
ಶರೀ-ರ-ಪ್ರಜ್ಞೆ
ಶರೀ-ರ-ಪ್ರ-ಜ್ಞೆ-ಯನ್ನು
ಶರೀ-ರ-ಪ್ರ-ಜ್ಞೆಯೂ
ಶರೀ-ರ-ರ-ಕ್ಷ-ಣೆ-ಗಾಗಿ
ಶರೀ-ರ-ರ-ಚ-ನೆಯೂ
ಶರೀ-ರ-ಲ-ಕ್ಷಣ
ಶರೀ-ರ-ವನ್ನು
ಶರೀ-ರ-ವ-ನ್ನೆಲ್ಲ
ಶರೀ-ರ-ವನ್ನೇ
ಶರೀ-ರ-ವಿದೆ
ಶರೀ-ರವು
ಶರೀ-ರವೂ
ಶರೀ-ರ-ವೆಲ್ಲ
ಶರೀ-ರ-ಶಕ್ತಿ
ಶರೀ-ರ-ಶ್ರಮ
ಶರೀ-ರ-ಸಂ-ಪ-ತ್ತನ್ನು
ಶರೀ-ರ-ಸಂ-ಪ-ತ್ತನ್ನೂ
ಶರೀ-ರ-ಸಾ-ಮು-ದ್ರಿಕ
ಶಲ್ಯ
ಶವಕ್ಕೆ
ಶವದ
ಶಶ-ಧರ
ಶಶಿ
ಶಶಿಗೆ
ಶಶಿ-ಭೂ-ಷಣ
ಶಶಿ-ಭೂ-ಷ-ಣನೂ
ಶಶಿಯ
ಶಹ-ರಾ-ನ್ಪು-ರಕ್ಕೆ
ಶಹ-ರಾ-ನ್ಪು-ರ-ದ-ವ-ರೆಗೆ
ಶಾಂತ
ಶಾಂತ-ಗಂ-ಭೀರ
ಶಾಂತ-ಶ-ರ-ಣಾ-ಗತಿ
ಶಾಂತ-ಗೊಂಡ
ಶಾಂತ-ಗೊ-ಳಿ-ಸಲು
ಶಾಂತ-ಗೊ-ಳಿ-ಸಿ-ದಾಗ
ಶಾಂತ-ಗೊ-ಳಿ-ಸು-ವು-ದರ
ಶಾಂತ-ಗೊ-ಳ್ಳು-ತ್ತದೆ
ಶಾಂತ-ಚಿತ್ತ
ಶಾಂತತೆ
ಶಾಂತ-ನಾ-ಗು-ವುದು
ಶಾಂತ-ನಾದ
ಶಾಂತ-ವಾಗಿ
ಶಾಂತ-ವಾ-ಗಿತ್ತು
ಶಾಂತ-ವಾ-ಗಿಯೇ
ಶಾಂತ-ವಾದ
ಶಾಂತಿ
ಶಾಂತಿ-ಆ-ನಂ-ದ-ಗ-ಳಾಗಿ
ಶಾಂತಿ-ಸಂ-ತೋಷ
ಶಾಂತಿ-ಸ-ಮಾ-ಧಾನ
ಶಾಂತಿ-ಧಾ-ಮ-ದೆ-ಡೆಗೆ
ಶಾಂತಿಯ
ಶಾಂತಿ-ಯನ್ನು
ಶಾಂತಿ-ಯಲ್ಲಿ
ಶಾಂತಿ-ಯಿಂ-ದಿ-ರಲಿ
ಶಾಂತಿ-ಯೆಲ್ಲಿ
ಶಾಕ್ತ
ಶಾರದಾ
ಶಾರ-ದಾ-ದೇವಿ
ಶಾರ-ದಾ-ದೇ-ವಿ-ಯ-ವರ
ಶಾರ-ದಾ-ದೇ-ವಿ-ಯ-ವರು
ಶಾರ-ದಾ-ನಂದ
ಶಾರ-ದಾ-ನಂ-ದ-ರಿಗೆ
ಶಾರ-ದಾ-ನಂ-ದರು
ಶಾರ-ದಾ-ನಂ-ದರೂ
ಶಾರ-ದಾ-ಪ್ರ-ಸನ್ನ
ಶಾರ-ದಾ-ಪ್ರ-ಸ-ನ್ನ-ಇಷ್ಟು
ಶಾರದೆ
ಶಾರೀ-ರಿಕ
ಶಾರೀ-ರಿ-ಕ-ಮಾ-ನ-ಸಿಕ
ಶಾಲಾ
ಶಾಲಾ-ಕಾ-ಲೇ-ಜು-ಗಳನ್ನೂ
ಶಾಲಾ-ವಿ-ದ್ಯಾ-ಭ್ಯಾ-ಸಕ್ಕೆ
ಶಾಲಿ-ಯಾದ
ಶಾಲೆ
ಶಾಲೆಗೆ
ಶಾಲೆಗೇ
ಶಾಲೆಯ
ಶಾಲೆ-ಯನ್ನು
ಶಾಲೆ-ಯಲ್ಲಿ
ಶಾಲೆ-ಯಲ್ಲೇ
ಶಾಲೆ-ಯಿಂದ
ಶಾಲೆಯೂ
ಶಾಶ್ವತ
ಶಾಶ್ವ-ತ-ಗೊ-ಳಿ-ಸು-ವಂ-ತಹ
ಶಾಶ್ವ-ತ-ವಾದ
ಶಾಶ್ವ-ತ-ವೆಂದು
ಶಾಸ್ತ್ರ
ಶಾಸ್ತ್ರ-ಸಿ-ದ್ಧಾಂ-ತ-ಗಳನ್ನು
ಶಾಸ್ತ್ರಕ್ಕೆ
ಶಾಸ್ತ್ರ-ಗಳ
ಶಾಸ್ತ್ರ-ಗಳನ್ನು
ಶಾಸ್ತ್ರ-ಗ-ಳ-ನ್ನೋ-ದಿದ
ಶಾಸ್ತ್ರ-ಗಳಲ್ಲಿ
ಶಾಸ್ತ್ರ-ಗ-ಳಲ್ಲೂ
ಶಾಸ್ತ್ರ-ಗ-ಳಿಗೂ
ಶಾಸ್ತ್ರ-ಗಳು
ಶಾಸ್ತ್ರ-ಗಳೂ
ಶಾಸ್ತ್ರ-ಗ್ರಂ-ಥ-ಗಳ
ಶಾಸ್ತ್ರ-ಗ್ರಂ-ಥ-ಗಳನ್ನು
ಶಾಸ್ತ್ರ-ಗ್ರಂ-ಥ-ಗ-ಳ-ಲ್ಲಾ-ಗಲಿ
ಶಾಸ್ತ್ರ-ಗ್ರಂ-ಥ-ಗ-ಳೆಷ್ಟು
ಶಾಸ್ತ್ರ-ಜ್ಞಾ-ನವೂ
ಶಾಸ್ತ್ರ-ತ-ತ್ತ್ವ-ಗಳ
ಶಾಸ್ತ್ರ-ದಲ್ಲಿ
ಶಾಸ್ತ್ರ-ಪಂ-ಡಿ-ತ-ರಲ್ಲ
ಶಾಸ್ತ್ರ-ಪ್ರ-ಕಾರ
ಶಾಸ್ತ್ರ-ವಾಕ್ಯ
ಶಾಸ್ತ್ರ-ವಾ-ಕ್ಯ-ಗಳ
ಶಾಸ್ತ್ರ-ವಾ-ಕ್ಯ-ಗಳನ್ನು
ಶಾಸ್ತ್ರ-ವಿ-ಚಾರ
ಶಾಸ್ತ್ರ-ವಿ-ಚಾ-ರ-ಗಳೂ
ಶಾಸ್ತ್ರ-ವಿಧಿ
ಶಾಸ್ತ್ರ-ವ್ಯಾ-ಖ್ಯಾನ
ಶಾಸ್ತ್ರ-ಸ-ಮ್ಮ-ತ-ವಾದ
ಶಾಸ್ತ್ರಾ-ಧ್ಯ-ಯನ
ಶಾಸ್ತ್ರಾ-ಧ್ಯ-ಯ-ನ-ವನ್ನು
ಶಾಸ್ತ್ರಿ
ಶಾಸ್ತ್ರೀಯ
ಶಾಸ್ತ್ರೀ-ಯಂ-ತಹ
ಶಾಸ್ತ್ರೋ-ಕ್ತ-ವಾಗಿ
ಶಾಸ್ತ್ರೋ-ಕ್ತಿ-ಗಳನ್ನು
ಶಿಕಾ-ಗೋದ
ಶಿಕ್ಷಣ
ಶಿಕ್ಷ-ಣ-ಕೇಂದ್ರ
ಶಿಕ್ಷ-ಣ-ಕೌ-ಶ-ಲ-ವನ್ನು
ಶಿಕ್ಷ-ಣ-ಕ್ಕಾಗಿ
ಶಿಕ್ಷ-ಣ-ತ-ಜ್ಞರೂ
ಶಿಕ್ಷ-ಣದ
ಶಿಕ್ಷ-ಣ-ದಿಂದ
ಶಿಕ್ಷ-ಣ-ವನ್ನು
ಶಿಕ್ಷ-ಣ-ವಿ-ಧಾನ
ಶಿಕ್ಷ-ಣ-ವಿ-ನ್ಯಾ-ಸದ
ಶಿಕ್ಷೆ-ಗಳನ್ನು
ಶಿಖರ
ಶಿಖ-ರ-ಗ-ಳತ್ತ
ಶಿಖ-ರ-ಗ-ಳ-ನ್ನೇರಿ
ಶಿಖ-ರ-ಗ-ಳಿಗೂ
ಶಿಖ-ರ-ಗಳು
ಶಿಖ-ರದ
ಶಿಖ-ರ-ದಲ್ಲಿ
ಶಿಖ-ರ-ದಿಂದ
ಶಿಖ-ರ-ದೆ-ಡೆಗೆ
ಶಿಖ-ರ-ವನ್ನೇ
ಶಿಖ-ರವೇ
ಶಿಥಿಲ
ಶಿಥಿ-ಲ-ವಾ-ಗಿತ್ತು
ಶಿರ
ಶಿರ-ಚ್ಛೇ-ದನ
ಶಿರ-ಬಾಗಿ
ಶಿರ-ಬಾ-ಗು-ವು-ದಿಲ್ಲ
ಶಿರ-ಸ್ತೇ-ದಾ-ರ-ನಾದ
ಶಿರೋ-ನಾಮೆ
ಶಿರೋ-ನಾ-ಮೆ-ಯನ್ನು
ಶಿರೋ-ನಾ-ಮೆ-ಯಲ್ಲಿ
ಶಿವ
ಶಿವ-ಶ-ಕ್ತಿ-ಯರು
ಶಿವ-ಕೃಪೆ
ಶಿವ-ಗುರು
ಶಿವ-ದೇ-ವಾ-ಲ-ಯ-ದಲ್ಲಿ
ಶಿವ-ಧಾ-ಮ-ವನ್ನು
ಶಿವನ
ಶಿವ-ನನ್ನು
ಶಿವ-ನ-ಲ್ಲವೇ
ಶಿವ-ನಾಥ
ಶಿವ-ನಿಂದೆ
ಶಿವ-ನಿಗೆ
ಶಿವನೇ
ಶಿವ-ಪೂಜೆ
ಶಿವ-ಪೂ-ಜೆ-ಶಿ-ವ-ಧ್ಯಾ-ನ-ಗ-ಳಲ್ಲೇ
ಶಿವ-ಪೂ-ಜೆಗೆ
ಶಿವ-ಪೂ-ಜೆ-ಯಲ್ಲಿ
ಶಿವ-ಮ-ಯವೇ
ಶಿವ-ರಾತ್ರಿ
ಶಿವ-ರಾ-ತ್ರಿಯ
ಶಿವ-ರಾ-ತ್ರಿಯೇ
ಶಿವ-ಸ್ಮ-ರಣೆ
ಶಿವ-ಸ್ವ-ರೂ-ಪಿ-ಯೆಂದು
ಶಿವಾನಂ
ಶಿವಾ-ನಂದ
ಶಿವಾ-ನಂ-ದ-ರಿಗೆ
ಶಿವಾ-ನಂ-ದರು
ಶಿಶು
ಶಿಶು-ಭಾ-ವ-ವನ್ನು
ಶಿಶು-ರೂಪ
ಶಿಶು-ವನ್ನು
ಶಿಶು-ವಿನ
ಶಿಶು-ವಿ-ನೊಲು
ಶಿಶು-ವೆಂದು
ಶಿಶುವೇ
ಶಿಷ್ಟಾ-ಚಾರ
ಶಿಷ್ಯ
ಶಿಷ್ಯ-ಸ-ಖ-ನಾದ
ಶಿಷ್ಯನ
ಶಿಷ್ಯ-ನನ್ನು
ಶಿಷ್ಯ-ನಲ್ಲಿ
ಶಿಷ್ಯ-ನಾಗಿ
ಶಿಷ್ಯ-ನಾ-ಗಿದ್ದ
ಶಿಷ್ಯ-ನಾ-ಗಿ-ಬಿಟ್ಟ
ಶಿಷ್ಯ-ನಾ-ಗು-ತ್ತಾನೆ
ಶಿಷ್ಯ-ನಾದ
ಶಿಷ್ಯ-ನಿಂ-ದಲೇ
ಶಿಷ್ಯ-ನಿಗೂ
ಶಿಷ್ಯ-ನಿಗೆ
ಶಿಷ್ಯನೂ
ಶಿಷ್ಯ-ನೆ-ದುರು
ಶಿಷ್ಯನೇ
ಶಿಷ್ಯ-ನೊಂ-ದಿಗೆ
ಶಿಷ್ಯ-ನೊ-ಬ್ಬನ
ಶಿಷ್ಯ-ನೊ-ಬ್ಬ-ನನ್ನು
ಶಿಷ್ಯ-ನೊ-ಬ್ಬನು
ಶಿಷ್ಯರ
ಶಿಷ್ಯ-ರಂತೂ
ಶಿಷ್ಯ-ರನ್ನು
ಶಿಷ್ಯ-ರ-ನ್ನೆಲ್ಲ
ಶಿಷ್ಯ-ರಲ್ಲಿ
ಶಿಷ್ಯ-ರಲ್ಲೂ
ಶಿಷ್ಯ-ರ-ಲ್ಲೆಲ್ಲ
ಶಿಷ್ಯ-ರ-ಲ್ಲೊ-ಬ್ಬ-ನಾದ
ಶಿಷ್ಯ-ರಾ-ಗು-ವ-ವರು
ಶಿಷ್ಯ-ರಿಂದ
ಶಿಷ್ಯ-ರಿ-ಗಂತೂ
ಶಿಷ್ಯ-ರಿ-ಗಿಂತ
ಶಿಷ್ಯ-ರಿಗೂ
ಶಿಷ್ಯ-ರಿಗೆ
ಶಿಷ್ಯ-ರಿ-ಗೇನೋ
ಶಿಷ್ಯರು
ಶಿಷ್ಯ-ರು-ಭ-ಕ್ತರು
ಶಿಷ್ಯರೂ
ಶಿಷ್ಯ-ರೆಂ-ಬು-ದನ್ನು
ಶಿಷ್ಯ-ರೆ-ದುರು
ಶಿಷ್ಯ-ರೆಲ್ಲ
ಶಿಷ್ಯ-ರೇ-ಅ-ದ-ರಲ್ಲೂ
ಶಿಷ್ಯ-ರೇನೋ
ಶಿಷ್ಯ-ರೊಂ-ದಿ-ಗಿ-ನ-ದ್ದ-ಕ್ಕಿಂತ
ಶಿಷ್ಯ-ರೊಂ-ದಿಗೆ
ಶಿಷ್ಯ-ವರ್ಗ
ಶಿಷ್ಯ-ವ-ರ್ಗ-ದಲ್ಲಿ
ಶಿಷ್ಯ-ವೃ-ತ್ತಿ-ಯಲ್ಲಿ
ಶಿಷ್ಯ-ಸ-ಮು-ದಾ-ಯದ
ಶಿಸ್ತಿಗೆ
ಶಿಸ್ತು-ಬದ್ಧ
ಶೀಘ್ರ-ದಲ್ಲೇ
ಶೀಲ
ಶೀಲ-ನ-ಡ-ವ-ಳಿ-ಕೆ-ಗಳು
ಶೀಲ-ಪೂರ್ಣ
ಶೀಲ-ವನ್ನು
ಶೀಲ-ವೆಂಬ
ಶುಕ-ಮ-ಹ-ರ್ಷಿ-ಗಳು
ಶುಕ-ಮ-ಹ-ರ್ಷಿಯೂ
ಶುಚಿ
ಶುಚಿ-ಮಾಡಿ
ಶುದ್ಧ
ಶುದ್ಧ-ಚೈ-ತ-ನ್ಯದ
ಶುದ್ಧ-ಚೈ-ತ-ನ್ಯವೇ
ಶುದ್ಧ-ವಾ-ಗಿ-ರು-ತ್ತ-ದೆಯೋ
ಶುದ್ಧ-ವಾ-ಗಿ-ರು-ವು-ದಿ-ಲ್ಲವೋ
ಶುದ್ಧ-ವಾದ
ಶುದ್ಧ-ಸಂ-ನ್ಯಾ-ಸದ
ಶುದ್ಧ-ಸ-ತ್ತ್ವ-ನಾ-ಗು-ತ್ತಾನೆ
ಶುದ್ಧ-ಹೃ-ದ-ಯದ
ಶುದ್ಧ-ಹೃ-ದ-ಯ-ದಿಂದ
ಶುನಾತೆ
ಶುಭ-ದಿನ
ಶುಭ-ದಿ-ನ-ದಲ್ಲಿ
ಶುಭ-ಮು-ಹೂರ್ತ
ಶುಭ-ರಾತ್ರಿ
ಶುಭ-ಸೂ-ಚನೆ
ಶುಭ-ಸೂ-ಚ-ನೆ-ಗ-ಳಾಗಿ
ಶುಭ್ರ
ಶುರು
ಶುರು-ಮಾ-ಡಿ-ದನೋ
ಶುರು-ಮಾ-ಡಿ-ದರು
ಶುರು-ಮಾ-ಡಿ-ಬಿ-ಟ್ಟೆಯಾ
ಶುರು-ವಾದ್ದ
ಶುರು-ವಾ-ಯಿತು
ಶುಲ್ಕ
ಶುಲ್ಕ-ವನ್ನು
ಶುಲ್ಕ-ವಿ-ನಾ-ಯಿತಿ
ಶುಶ್ರೂಷೆ
ಶುಶ್ರೂ-ಷೆಗೆ
ಶುಶ್ರೂ-ಷೆಯ
ಶುಷ್ಕ
ಶುಷ್ಕ-ವಾಗಿ
ಶೂಟ್
ಶೂದ್ರ
ಶೂದ್ರನೋ
ಶೂನ್ಯತೆ
ಶೂನ್ಯ-ತೆ-ಯನ್ನು
ಶೂನ್ಯ-ತೆಯೇ
ಶೂನ್ಯ-ದೊ-ಳಗೆ
ಶೂನ್ಯ-ವನ್ನೂ
ಶೃಂಗಾ-ಗ್ರ-ಗಳ
ಶೃಂಗಾ-ರ-ಭಾವ
ಶೇಖ-ರಿಸಿ
ಶೇಖ-ರಿ-ಸಿಕೊ
ಶೈಲಿ-ಯಲ್ಲಿ
ಶೋಕ-ದುಃ-ಖ-ಗಳು
ಶೋಕ-ಗ್ರ-ಸ್ತ-ರಾ-ದರು
ಶೋಕಿ
ಶೋಕಿ-ಪು-ದೇಕೆ
ಶೋಕಿಯ
ಶೋಕಿ-ಲಾ-ಲ-ರನ್ನು
ಶೋಕಿ-ಸುತ್ತ
ಶೋಚ-ನೀಯ
ಶೋಚ-ನೀ-ಯ-ವಾ-ದದ್ದು
ಶೋಫೆ-ನೋ-ವರ್
ಶೋಭಾ-ಯ-ಮಾ-ನ-ಳಾ-ಗಿ-ದ್ದಾಳೆ
ಶೋಭಿ-ಸು-ತ್ತಿತ್ತು
ಶೋಭೆ-ಗೊ-ಳಿ-ಸಿ-ದ್ದಾನೆ
ಶೌರ್ಯ
ಶ್
ಶ್ಚರ್ಯ
ಶ್ಚರ್ಯಕ್ಕೆ
ಶ್ಚರ್ಯೆ-ಯಲ್ಲಿ
ಶ್ಮಶಾ-ನ-ಘ-ಟ್ಟ-ದಲ್ಲಿ
ಶ್ಯಾಮ-ಪು-ಕುರ
ಶ್ಯಾಮ-ಪು-ಕು-ರಕ್ಕೆ
ಶ್ಯಾಮ-ಪು-ಕು-ರ-ದಲ್ಲೇ
ಶ್ಯಾಮ-ಪು-ಕು-ರ-ವಾ-ಸದ
ಶ್ಯಾಮಲ
ಶ್ಯಾಮಲ್
ಶ್ಯಾಮ-ಲ್ದಾ-ಸನ
ಶ್ಯಾಮ-ಳನ್ನು
ಶ್ಯಾಮ-ಸುಂ-ದರಿ
ಶ್ಯಾಮ-ಸುಂ-ದರೀ
ಶ್ರದ್ದೆ-ಯನ್ನು
ಶ್ರದ್ಧಾ
ಶ್ರದ್ಧಾ-ಭ-ಕ್ತಿ-ಗ-ಳ-ಲ್ಲಾ-ಗಲಿ
ಶ್ರದ್ಧಾ-ಭ-ಕ್ತಿ-ಗಳಲ್ಲಿ
ಶ್ರದ್ಧಾ-ಭ-ಕ್ತಿ-ಗಳಿಂದ
ಶ್ರದ್ಧೆ
ಶ್ರದ್ಧೆ-ಗ-ಲ್ಲದೆ
ಶ್ರದ್ಧೆ-ಗಳನ್ನು
ಶ್ರದ್ಧೆ-ಗಳಿಂದ
ಶ್ರದ್ಧೆ-ಗೇನು
ಶ್ರದ್ಧೆಯ
ಶ್ರದ್ಧೆ-ಯನ್ನು
ಶ್ರದ್ಧೆ-ಯಲ್ಲ
ಶ್ರದ್ಧೆ-ಯಾ-ಗಲಿ
ಶ್ರದ್ಧೆ-ಯಿಂದ
ಶ್ರದ್ಧೆ-ಯಿಂ-ದೇ-ನಾ-ದೀತು
ಶ್ರದ್ಧೆ-ಯಿ-ಟ್ಟಿದ್ದ
ಶ್ರದ್ಧೆ-ಯಿ-ರ-ಬೇ-ಕಾ-ಗು-ತ್ತದೆ
ಶ್ರದ್ಧೆ-ಯಿ-ರ-ಲಿಲ್ಲ
ಶ್ರದ್ಧೆಯೇ
ಶ್ರಮ-ಗ-ಳ-ವ-ರಿಗೂ
ಶ್ರಮದ
ಶ್ರಮ-ವ-ನ್ನಷ್ಟೇ
ಶ್ರಮಿ-ಸ-ಬೇ-ಕೆ-ನ್ನುವ
ಶ್ರಮಿ-ಸು-ತ್ತಿ-ದ್ದರು
ಶ್ರಮಿ-ಸುವ
ಶ್ರಾವ-ಣ-ಮಾ-ಸದ
ಶ್ರಿರಾ-ಮ-ಕೃ-ಷ್ಣ-ನ-ರೇಂ-ದ್ರರ
ಶ್ರಿರಾ-ಮ-ಕೃ-ಷ್ಣರ
ಶ್ರಿರಾ-ಮ-ಕೃ-ಷ್ಣರು
ಶ್ರೀ
ಶ್ರೀಆ-ದಿ-ಶಂ-ಕ-ರರೂ
ಶ್ರೀಕೃಷ್ಣ
ಶ್ರೀಕೃ-ಷ್ಣ-ಚೈ-ತ-ನ್ಯನ
ಶ್ರೀಕೃ-ಷ್ಣ-ಚೈ-ತ-ನ್ಯನು
ಶ್ರೀಕೃ-ಷ್ಣ-ಜ-ನ್ಮಾ-ಷ್ಟ-ಮಿಯ
ಶ್ರೀಕೃ-ಷ್ಣನ
ಶ್ರೀಕೃ-ಷ್ಣನು
ಶ್ರೀಕೃ-ಷ್ಣನೂ
ಶ್ರೀಕೃ-ಷ್ಣ-ಪ-ರ-ಮಾ-ತ್ಮನು
ಶ್ರೀಕೃಷ್ಣಾ
ಶ್ರೀಕೃ-ಷ್ಣಾ-ಷ್ಟ-ಮಿಯ
ಶ್ರೀಗು-ರು-ಸೇ-ವೆ-ಯನ್ನೇ
ಶ್ರೀಗೌ-ರಾಂ-ಗನು
ಶ್ರೀದು-ರ್ಗೆಯ
ಶ್ರೀನ-ಗ-ರಕ್ಕೆ
ಶ್ರೀನ-ಗ-ರದ
ಶ್ರೀನ-ಗ-ರ-ದಲ್ಲಿ
ಶ್ರೀನ-ಗ-ರ-ದಿಂದ
ಶ್ರೀಪ್ರ-ಸಾ-ದನು
ಶ್ರೀಮಂತ
ಶ್ರೀಮಂ-ತ-ನಲ್ಲ
ಶ್ರೀಮಂ-ತ-ನಾ-ಗಿದ್ದ
ಶ್ರೀಮಂ-ತ-ನಿಗೆ
ಶ್ರೀಮಂ-ತ-ನೇನೂ
ಶ್ರೀಮಂ-ತರ
ಶ್ರೀಮಂ-ತ-ರನ್ನು
ಶ್ರೀಮಂ-ತ-ರಾ-ಗಿ-ರ-ಬ-ಹುದು
ಶ್ರೀಮಂ-ತರು
ಶ್ರೀಮಂ-ತ-ರು-ದ-ರಿ-ದ್ರರು
ಶ್ರೀಮಂ-ತ-ರು-ಹ-ಣ್ಣು-ಹಂ-ಪಲು
ಶ್ರೀಮಂ-ತ-ರೇ-ನಲ್ಲ
ಶ್ರೀಮಂ-ತಿ-ಕೆಯ
ಶ್ರೀಮಂ-ತಿ-ಕೆ-ಯಲ್ಲಿ
ಶ್ರೀಮ-ದ್ಭಾ-ಗ-ವ-ತ-ದಲ್ಲಿ
ಶ್ರೀಮ-ದ್ಭಾ-ಗ-ವ-ತ-ವನ್ನು
ಶ್ರೀಮಾತೆ
ಶ್ರೀಮಾ-ತೆ-ಯ-ವರ
ಶ್ರೀಮಾ-ತೆ-ಯ-ವ-ರಿಗೂ
ಶ್ರೀಮಾ-ತೆ-ಯ-ವ-ರಿಗೆ
ಶ್ರೀಮಾ-ತೆ-ಯ-ವರು
ಶ್ರೀರಾಮ
ಶ್ರೀರಾ-ಮ-ಕ-ಷ್ಣರು
ಶ್ರೀರಾ-ಮ-ಕೃಷ್ಣ
ಶ್ರೀರಾ-ಮ-ಕೃ-ಷ್ಣ-ನ-ರೇಂ-ದ್ರರ
ಶ್ರೀರಾ-ಮ-ಕೃ-ಷ್ಣ-ದ-ರ್ಶ-ನದ
ಶ್ರೀರಾ-ಮ-ಕೃ-ಷ್ಣ-ನಾಮ
ಶ್ರೀರಾ-ಮ-ಕೃ-ಷ್ಣ-ಭಾ-ವವು
ಶ್ರೀರಾ-ಮ-ಕೃ-ಷ್ಣರ
ಶ್ರೀರಾ-ಮ-ಕೃ-ಷ್ಣ-ರಂ-ತಹ
ಶ್ರೀರಾ-ಮ-ಕೃ-ಷ್ಣ-ರಂತೂ
ಶ್ರೀರಾ-ಮ-ಕೃ-ಷ್ಣ-ರದು
ಶ್ರೀರಾ-ಮ-ಕೃ-ಷ್ಣ-ರದೇ
ಶ್ರೀರಾ-ಮ-ಕೃ-ಷ್ಣ-ರ-ದ್ದಾ-ಗಿ-ರ-ಬೇಕು
ಶ್ರೀರಾ-ಮ-ಕೃ-ಷ್ಣ-ರನ್ನು
ಶ್ರೀರಾ-ಮ-ಕೃ-ಷ್ಣ-ರನ್ನೂ
ಶ್ರೀರಾ-ಮ-ಕೃ-ಷ್ಣ-ರನ್ನೇ
ಶ್ರೀರಾ-ಮ-ಕೃ-ಷ್ಣ-ರಲ್ಲಿ
ಶ್ರೀರಾ-ಮ-ಕೃ-ಷ್ಣ-ರ-ಲ್ಲಿದೆ
ಶ್ರೀರಾ-ಮ-ಕೃ-ಷ್ಣ-ರ-ಲ್ಲಿದ್ದ
ಶ್ರೀರಾ-ಮ-ಕೃ-ಷ್ಣ-ರ-ಲ್ಲಿನ
ಶ್ರೀರಾ-ಮ-ಕೃ-ಷ್ಣ-ರಾಗಿ
ಶ್ರೀರಾ-ಮ-ಕೃ-ಷ್ಣ-ರಾ-ಡಿದ
ಶ್ರೀರಾ-ಮ-ಕೃ-ಷ್ಣ-ರಾ-ಡುವ
ಶ್ರೀರಾ-ಮ-ಕೃ-ಷ್ಣ-ರಾ-ದರೋ
ಶ್ರೀರಾ-ಮ-ಕೃ-ಷ್ಣ-ರಿಂದ
ಶ್ರೀರಾ-ಮ-ಕೃ-ಷ್ಣ-ರಿ-ಗಂತೂ
ಶ್ರೀರಾ-ಮ-ಕೃ-ಷ್ಣ-ರಿ-ಗಾ-ಗಲಿ
ಶ್ರೀರಾ-ಮ-ಕೃ-ಷ್ಣ-ರಿ-ಗಾಗಿ
ಶ್ರೀರಾ-ಮ-ಕೃ-ಷ್ಣ-ರಿ-ಗಾ-ಗು-ತ್ತಿದ್ದ
ಶ್ರೀರಾ-ಮ-ಕೃ-ಷ್ಣ-ರಿ-ಗಾದ
ಶ್ರೀರಾ-ಮ-ಕೃ-ಷ್ಣ-ರಿಗೂ
ಶ್ರೀರಾ-ಮ-ಕೃ-ಷ್ಣ-ರಿಗೆ
ಶ್ರೀರಾ-ಮ-ಕೃ-ಷ್ಣ-ರಿಗೇ
ಶ್ರೀರಾ-ಮ-ಕೃ-ಷ್ಣ-ರಿನ್ನೂ
ಶ್ರೀರಾ-ಮ-ಕೃ-ಷ್ಣ-ರೀಗ
ಶ್ರೀರಾ-ಮ-ಕೃ-ಷ್ಣರು
ಶ್ರೀರಾ-ಮ-ಕೃ-ಷ್ಣರೂ
ಶ್ರೀರಾ-ಮ-ಕೃ-ಷ್ಣ-ರೂಪ
ಶ್ರೀರಾ-ಮ-ಕೃ-ಷ್ಣ-ರೆಂ-ದರು
ಶ್ರೀರಾ-ಮ-ಕೃ-ಷ್ಣ-ರೆಂಬ
ಶ್ರೀರಾ-ಮ-ಕೃ-ಷ್ಣ-ರೆ-ದುರು
ಶ್ರೀರಾ-ಮ-ಕೃ-ಷ್ಣ-ರೆ-ನ್ನು-ತ್ತಾರೆ
ಶ್ರೀರಾ-ಮ-ಕೃ-ಷ್ಣರೇ
ಶ್ರೀರಾ-ಮ-ಕೃ-ಷ್ಣ-ರೇನೋ
ಶ್ರೀರಾ-ಮ-ಕೃ-ಷ್ಣ-ರೊಂ-ದಿಗೆ
ಶ್ರೀರಾ-ಮ-ಕೃ-ಷ್ಣ-ರೊ-ಡ್ಡಿದ
ಶ್ರೀರಾ-ಮ-ಕೃ-ಷ್ಣ-ರೊ-ಬ್ಬರೇ
ಶ್ರೀರಾ-ಮ-ಕೃ-ಷ್ಣಾ-ವ-ತಾರ
ಶ್ರೀರಾ-ಮ-ಕೃ-ಷ್ಣಾ-ವ-ತಾ-ರದ
ಶ್ರೀರಾ-ಮ-ಕೃ-ಷ್ಣಾ-ಶ್ರಮ
ಶ್ರೀರಾ-ಮ-ಚಂ-ದ್ರನ
ಶ್ರೀರಾ-ಮನ
ಶ್ರೀರಾ-ಮನು
ಶ್ರೀರಾ-ಮ-ಮಂ-ದಿ-ರದ
ಶ್ರೀವಿ-ವೇ-ಕಾ-ನಂದ
ಶ್ರೀಶಾ-ರ-ದಾ-ದೇವಿ
ಶ್ರೀಶಾ-ರ-ದಾ-ದೇ-ವಿ-ಯ-ವರ
ಶ್ರೀಶಾ-ರ-ದಾ-ದೇ-ವಿ-ಯ-ವ-ರನ್ನು
ಶ್ರೀಶಾ-ರ-ದಾ-ದೇ-ವಿ-ಯ-ವರು
ಶ್ರುತಿ
ಶ್ರುತಿ-ಲಯ
ಶ್ರುತಿ-ಬ-ದ್ಧ-ವಾಗಿ
ಶ್ರುತಿಯ
ಶ್ರುತೇನ
ಶ್ರೇಣಿ-ಯನ್ನು
ಶ್ರೇಣಿ-ಯಲ್ಲಿ
ಶ್ರೇಯ-ಸ್ಸಿ-ಗಾಗಿ
ಶ್ರೇಯಸ್ಸು
ಶ್ರೇಯ-ಸ್ಸುಂ-ಟಾ-ಗು-ತ್ತದೆ
ಶ್ರೇಷ್ಠ
ಶ್ರೇಷ್ಠ-ತೆ-ಯನ್ನು
ಶ್ರೇಷ್ಠ-ನಾದ
ಶ್ರೇಷ್ಠ-ನಾ-ದ-ವನು
ಶ್ರೇಷ್ಠ-ಭ-ಕ್ತರ
ಶ್ರೇಷ್ಠರು
ಶ್ರೇಷ್ಠ-ವಾ-ದದ್ದು
ಶ್ರೇಷ್ಠ-ವಾ-ದ-ವಿ-ಚಾ-ರ-ಸ-ಮ್ಮ-ತ-ವಾದ
ಶ್ರೋತೃ-ಗ-ಳೆಂ-ದರೆ
ಶ್ಲೋಕ-ಗಳ
ಶ್ಲೋಕ-ಗಳನ್ನು
ಶ್ವಾಸ-ನಾ-ಳದ
ಶ್ವಾಸೋ-ಚ್ಛ್ವಾಸ
ಶ್ವೇತ-ವ-ರ್ಣ-ದಿಂದ
ಷಡ್
ಷರ-ತ್ತಿಗೆ
ಷರ-ತ್ತಿನ
ಷೋಡಶೀ
ಷ್
ಷ್ಟಮಿ
ಷ್ಟಿದ್ದರೆ
ಷ್ಠಾನ
ಸಂ
ಸಂ
ಸಂಕಟ
ಸಂಕ-ಟ-ಗಳ
ಸಂಕ-ಟ-ಗಳು
ಸಂಕ-ಟದ
ಸಂಕ-ಟ-ಪಟ್ಟ
ಸಂಕ-ಟ-ಮ-ಯ-ವಾದ
ಸಂಕ-ಟ-ಮ-ಯ-ವಾ-ದದ್ದು
ಸಂಕ-ಟ-ಮ-ಯ-ವೆಂಬ
ಸಂಕ-ಟ-ವ-ನ್ನ-ನು-ಭ-ವಿ-ಸಿ-ದರು
ಸಂಕ-ಟ-ವಾ-ಗು-ತ್ತಿತ್ತು
ಸಂಕ-ಟ-ವಾ-ಯಿತು
ಸಂಕ-ಲನ
ಸಂಕಲ್ಪ
ಸಂಕ-ಲ್ಪ-ಗಳನ್ನು
ಸಂಕ-ಲ್ಪ-ದಿಂದ
ಸಂಕ-ಲ್ಪ-ಮಾ-ಡಿ-ದ್ದ-ರಿಂದ
ಸಂಕ-ಲ್ಪ-ಮಾ-ತ್ರ-ದಿಂದ
ಸಂಕ-ಲ್ಪಿ-ಸಿ-ದ-ರಾ-ಯಿ-ತು-ಕಾ-ಯಿಲೆ
ಸಂಕ-ಲ್ಪಿ-ಸಿ-ದ್ದರು
ಸಂಕೀ-ರ್ತನೆ
ಸಂಕೀ-ರ್ತ-ನೆಯ
ಸಂಕೀ-ರ್ತ-ನೆ-ಯನ್ನು
ಸಂಕೀ-ರ್ತ-ನೆಯೇ
ಸಂಕು-ಚಿತ
ಸಂಕು-ಚಿ-ತತೆ
ಸಂಕು-ಚಿ-ತ-ತೆ-ಯಿಂದ
ಸಂಕು-ಚಿ-ತ-ವಾ-ದ-ವು-ಗಳು
ಸಂಕೇತ
ಸಂಕೇ-ತ-ವಾಗಿ
ಸಂಕೇ-ತವೇ
ಸಂಕೋ-ಚ-ದಿಂದ
ಸಂಕೋ-ಚ-ಪ-ಟ್ಟು-ಕೊ-ಳ್ಳು-ವಂಥ
ಸಂಕೋ-ಚ-ವಾ-ಯಿತು
ಸಂಕೋ-ಚ-ವಿಲ್ಲ
ಸಂಕೋ-ಲೆಯ
ಸಂಕೋ-ಲೆ-ಯಾ-ಗಿ-ಬಿಟ್ಟೆ
ಸಂಕ್ರ-ಮಣ
ಸಂಕ್ಷಿ-ಪ್ತ-ವಾಗಿ
ಸಂಖ್ಯೆ
ಸಂಖ್ಯೆಯ
ಸಂಖ್ಯೆಯೂ
ಸಂಗ
ಸಂಗ-ಡವೇ
ಸಂಗಡಿ
ಸಂಗ-ಡಿಗ
ಸಂಗ-ಡಿ-ಗನೂ
ಸಂಗ-ಡಿ-ಗ-ನೊ-ಬ್ಬ-ನನ್ನು
ಸಂಗ-ಡಿ-ಗ-ರ-ನ್ನೆಲ್ಲ
ಸಂಗ-ಡಿ-ಗರೂ
ಸಂಗ-ಡಿ-ಗ-ರೆಲ್ಲ
ಸಂಗತಿ
ಸಂಗ-ತಿ-ಗ-ಳಷ್ಟೆ
ಸಂಗ-ತಿ-ಯನ್ನು
ಸಂಗ-ತಿ-ಯೆಂ-ದರೆ
ಸಂಗ-ತಿ-ಯೆಂ-ಬು-ದ-ರಲ್ಲಿ
ಸಂಗ-ತ್ಯಾಗ
ಸಂಗ-ದಲ್ಲಿ
ಸಂಗ-ಮ-ದಲ್ಲಿ
ಸಂಗ-ಮ-ಸ್ಥಳ
ಸಂಗ-ಮ-ಸ್ಥ-ಳ-ವಾದ
ಸಂಗಾ-ತಿ-ಗಳ
ಸಂಗಾ-ತಿ-ಗಳು
ಸಂಗಾ-ತಿ-ಗ-ಳೆ-ಲ್ಲರ
ಸಂಗೀತ
ಸಂಗೀ-ತಕ್ಕೆ
ಸಂಗೀ-ತ-ಗಾ-ರ-ರನ್ನು
ಸಂಗೀ-ತ-ಗಾ-ರ-ರಾದ
ಸಂಗೀ-ತ-ಗಾ-ರ-ರಿಂದ
ಸಂಗೀ-ತ-ಗಾ-ರರು
ಸಂಗೀ-ತದ
ಸಂಗೀ-ತ-ದಲ್ಲಿ
ಸಂಗೀ-ತ-ದಲ್ಲೂ
ಸಂಗೀ-ತ-ಪ್ರೇ-ಮ-ವನ್ನೂ
ಸಂಗೀ-ತ-ವನ್ನು
ಸಂಗೀ-ತ-ವನ್ನೂ
ಸಂಗೀ-ತ-ವೆ-ನಿ-ಸ-ಲಾ-ರದು
ಸಂಗೀ-ತ-ವೆ-ನ್ನು-ವುದು
ಸಂಗೀ-ತವೇ
ಸಂಗೀ-ತಾ-ಭ್ಯಾಸ
ಸಂಗ್ರ-ಹ-ವಾ-ಯಿತು
ಸಂಗ್ರಹಿ
ಸಂಗ್ರ-ಹಿಸಿ
ಸಂಗ್ರ-ಹಿ-ಸಿ-ದಂ-ತಹ
ಸಂಘ
ಸಂಘಕ್ಕೆ
ಸಂಘ-ಜೀ-ವನ
ಸಂಘ-ಟನೆ
ಸಂಘ-ಟಿ-ತ-ರಾಗಿ
ಸಂಘ-ಟಿ-ಸ-ಲ್ಪಟ್ಟ
ಸಂಘ-ಟಿ-ಸಿತು
ಸಂಘದ
ಸಂಘ-ದಲ್ಲಿ
ಸಂಘರ್ಷ
ಸಂಘ-ರ್ಷ-ವನ್ನು
ಸಂಘ-ವನ್ನು
ಸಂಘ-ವೆ-ನ್ನು-ವುದು
ಸಂಘ-ಸಂ-ಸ್ಥೆ-ಗಳನ್ನು
ಸಂಚ-ರಿ-ಸ-ತೊ-ಡ-ಗಿತು
ಸಂಚ-ರಿಸಿ
ಸಂಚ-ರಿ-ಸಿ-ದ-ರಲ್ಲಾ
ಸಂಚ-ರಿ-ಸುತ್ತ
ಸಂಚ-ರಿ-ಸುತ್ತಿ
ಸಂಚ-ರಿ-ಸು-ತ್ತಿದ್ದ
ಸಂಚ-ರಿ-ಸು-ತ್ತಿ-ದ್ದರು
ಸಂಚ-ರಿ-ಸು-ತ್ತಿ-ದ್ದಾಗ
ಸಂಚ-ರಿ-ಸು-ತ್ತಿ-ರ-ಬೇಕು
ಸಂಚ-ರಿ-ಸುವ
ಸಂಚ-ರಿ-ಸು-ವಾಗ
ಸಂಚ-ಲ-ನಕ್ಕೂ
ಸಂಚಾರ
ಸಂಚಾ-ರಕ್ಕೆ
ಸಂಚಾ-ರ-ವನ್ನೂ
ಸಂಚಾ-ರ-ವಾ-ಯಿತು
ಸಂಚಾ-ರಿಗೆ
ಸಂಜೆ
ಸಂಜೆ-ಗ-ತ್ತ-ಲಾ-ಗು-ತ್ತಿತ್ತು
ಸಂಜೆ-ಗ-ತ್ತಲು
ಸಂಜೆಯ
ಸಂಜೆ-ಯ-ವ-ರೆಗೂ
ಸಂಜೆ-ಯಾ-ದ್ದ-ರಿಂದ
ಸಂಜೆ-ಯಿಂ-ದಲೇ
ಸಂಜೆಯೂ
ಸಂಜೆ-ಯೆಲ್ಲ
ಸಂಜ್ಞೆ-ಗಳ
ಸಂಜ್ಞೆ-ಯಿಂ-ದಲೇ
ಸಂತ
ಸಂತತ
ಸಂತ-ನಾ-ಗ-ಲಿ-ರುವ
ಸಂತ-ನಾ-ಗಲು
ಸಂತ-ನಿಂದ
ಸಂತ-ನೊ-ಬ್ಬನ
ಸಂತನೋ
ಸಂತರ
ಸಂತ-ರಾದ
ಸಂತರು
ಸಂತಸ
ಸಂತ-ಸ-ಗೊಂ-ಡರು
ಸಂತ-ಸದ
ಸಂತ-ಸ-ಪಟ್ಟ
ಸಂತಾ-ನ-ಭಾವ
ಸಂತಾ-ನೋ-ತ್ಪತ್ತಿ
ಸಂತು
ಸಂತುಷ್ಟ
ಸಂತು-ಷ್ಟ-ರಾದ
ಸಂತು-ಷ್ಟಿ-ಪ-ಡಿ-ಸು-ವು-ದರ
ಸಂತೃಪ್ತ
ಸಂತೃ-ಪ್ತಿ-ಯಿಂ-ದಲೋ
ಸಂತೆಯ
ಸಂತೆಯೇ
ಸಂತೈ-ಸಿದ
ಸಂತೋಷ
ಸಂತೋ-ಷ-ಕ-ರ-ವಾದ
ಸಂತೋ-ಷ-ಕ್ಕಾಗಿ
ಸಂತೋ-ಷ-ಗಳಿಂದ
ಸಂತೋ-ಷ-ಚಿ-ತ್ತ-ರಾಗಿ
ಸಂತೋ-ಷದ
ಸಂತೋ-ಷ-ದಿಂದ
ಸಂತೋ-ಷ-ದಿಂ-ದಲೇ
ಸಂತೋ-ಷ-ಪಟ್ಟ
ಸಂತೋ-ಷ-ಪ-ಟ್ಟರು
ಸಂತೋ-ಷ-ಪ-ಡಿ-ಸಿದ
ಸಂತೋ-ಷ-ಪ-ಡು-ತ್ತಾರೆ
ಸಂತೋ-ಷ-ವನ್ನು
ಸಂತೋ-ಷ-ವನ್ನೇ
ಸಂತೋ-ಷ-ವಾಗಿ
ಸಂತೋ-ಷ-ವಾ-ಗಿ-ರ-ಬ-ಲ್ಲರು
ಸಂತೋ-ಷ-ವಾ-ಗಿ-ರ-ಬ-ಹುದು
ಸಂತೋ-ಷ-ವಾ-ಗು-ತ್ತಿದೆ
ಸಂತೋ-ಷ-ವಾ-ಯಿತು
ಸಂತೋ-ಷ-ವಿಲ್ಲ
ಸಂತೋ-ಷವೂ
ಸಂತೋ-ಷವೇ
ಸಂತೋ-ಷಿ-ಸು-ತ್ತಿ-ದ್ದರು
ಸಂತೋ-ಷಿ-ಸು-ತ್ತೇವೆ
ಸಂದರ್ಭ
ಸಂದ-ರ್ಭ-ಗ-ಳಂತೂ
ಸಂದ-ರ್ಭ-ಗಳಲ್ಲಿ
ಸಂದ-ರ್ಭ-ಗ-ಳಲ್ಲೂ
ಸಂದ-ರ್ಭ-ಗ-ಳ-ಲ್ಲೆಲ್ಲ
ಸಂದ-ರ್ಭ-ಗ-ಳಲ್ಲೇ
ಸಂದ-ರ್ಭ-ಗ-ಳಿವೆ
ಸಂದ-ರ್ಭ-ಗಳು
ಸಂದ-ರ್ಭ-ದಲ್ಲಿ
ಸಂದ-ರ್ಭ-ದ-ಲ್ಲಿನ
ಸಂದ-ರ್ಭ-ದಲ್ಲೇ
ಸಂದ-ರ್ಭ-ವನ್ನು
ಸಂದ-ರ್ಭ-ವೆಂದು
ಸಂದ-ರ್ಭ-ವೊ-ದ-ಗಿತು
ಸಂದ-ರ್ಭ-ವೊ-ದ-ಗಿ-ದಾ-ಗ-ಲೆಲ್ಲ
ಸಂದ-ರ್ಭ-ವೊ-ದ-ಗಿದೆ
ಸಂದ-ರ್ಶ-ಕ-ರನ್ನು
ಸಂದ-ರ್ಶ-ಕ-ರೇ-ನಾ-ದರೂ
ಸಂದ-ರ್ಶನ
ಸಂದ-ರ್ಶ-ನ-ಕಾ-ಲದ
ಸಂದ-ರ್ಶ-ನ-ಗಳಲ್ಲಿ
ಸಂದ-ರ್ಶ-ನವೂ
ಸಂದ-ರ್ಶಿ-ಸ-ಬ-ಹುದು
ಸಂದ-ರ್ಶಿ-ಸಲು
ಸಂದ-ರ್ಶಿಸಿ
ಸಂದ-ರ್ಶಿ-ಸಿದ
ಸಂದ-ರ್ಶಿ-ಸಿ-ದರು
ಸಂದ-ರ್ಶಿ-ಸುವ
ಸಂದ-ರ್ಶಿ-ಸು-ವು-ದಾ-ಗಿತ್ತು
ಸಂದಿ-ಗೊಂ-ದಿ-ಯ-ಲ್ಲೆಲ್ಲ
ಸಂದಿಗ್ಧ
ಸಂದಿತ್ತು
ಸಂದೂ-ಕ-ದ-ಲ್ಲಿಟ್ಟು
ಸಂದೇಶ
ಸಂದೇ-ಶ-ಗಳನ್ನು
ಸಂದೇ-ಶ-ಗಳಿಂದ
ಸಂದೇ-ಶದ
ಸಂದೇ-ಶ-ದೊಂ-ದಿಗೆ
ಸಂದೇ-ಶ-ವ-ನ್ನೀ-ಯಲು
ಸಂದೇ-ಶ-ವನ್ನು
ಸಂದೇಹ
ಸಂದೇ-ಹಕ್ಕೆ
ಸಂದೇ-ಹ-ಗಳನ್ನು
ಸಂದೇ-ಹ-ಗಳು
ಸಂದೇ-ಹ-ದಿಂದ
ಸಂದೇ-ಹ-ವಿ-ದ್ದರೆ
ಸಂದೇ-ಹ-ವಿಲ್ಲ
ಸಂದೇ-ಹ-ವುಂ-ಟಾ-ಗಿ-ಬಿ-ಟ್ಟಿ-ತ್ತು-ತಾವು
ಸಂದೇ-ಹವೇ
ಸಂದೇ-ಹ-ವೇ-ನಿದೆ
ಸಂದೇ-ಹಿಯೂ
ಸಂದೇ-ಹಿ-ಸಿ-ದೆ-ನ-ಲ್ಲ-ಎಂಬ
ಸಂಧ-ರ್ಭ-ದಲ್ಲಿ
ಸಂಧಿ-ಕಾ-ಲ-ದಲ್ಲಿ
ಸಂಧಿ-ಸಲು
ಸಂಧಿಸಿ
ಸಂಧಿ-ಸಿದ
ಸಂಧಿ-ಸಿ-ದರು
ಸಂಧಿ-ಸಿ-ದ-ರು-ಆ-ನಂದ
ಸಂಧಿ-ಸಿ-ದಾ-ಗಲೂ
ಸಂಧಿ-ಸು-ತ್ತವೆ
ಸಂಧಿ-ಸುವ
ಸಂಧ್ಯಾ-ರಾ-ಗ-ದಲ್ಲಿ
ಸಂನ್ಯಾಸ
ಸಂನ್ಯಾ-ಸ-ಜೀ-ವನ
ಸಂನ್ಯಾ-ಸ-ಜೀ-ವ-ನಕ್ಕೆ
ಸಂನ್ಯಾ-ಸ-ಜೀ-ವ-ನದ
ಸಂನ್ಯಾ-ಸ-ಜೀ-ವ-ನ-ವನ್ನು
ಸಂನ್ಯಾ-ಸ-ಜೀ-ವ-ನವೇ
ಸಂನ್ಯಾ-ಸದ
ಸಂನ್ಯಾ-ಸ-ದೀಕ್ಷೆ
ಸಂನ್ಯಾ-ಸ-ದೀ-ಕ್ಷೆ-ಯನ್ನು
ಸಂನ್ಯಾ-ಸ-ಧ-ರ್ಮ-ಕ್ಕ-ನು-ಸಾ-ರ-ವಾಗಿ
ಸಂನ್ಯಾ-ಸ-ಧ-ರ್ಮಕ್ಕೆ
ಸಂನ್ಯಾ-ಸ-ನಾ-ಮ-ಗಳನ್ನು
ಸಂನ್ಯಾ-ಸ-ನಿ-ಷ್ಠೆ-ಯನ್ನು
ಸಂನ್ಯಾ-ಸ-ರೇಖೆ
ಸಂನ್ಯಾ-ಸ-ರೇ-ಖೆ-ಯಂತೆ
ಸಂನ್ಯಾ-ಸ-ವನ್ನು
ಸಂನ್ಯಾ-ಸವೋ
ಸಂನ್ಯಾ-ಸ-ವ್ರತ
ಸಂನ್ಯಾ-ಸ-ಸ್ವೀ-ಕಾ-ರಕ್ಕೆ
ಸಂನ್ಯಾಸಿ
ಸಂನ್ಯಾ-ಸಿ-ಗಳ
ಸಂನ್ಯಾ-ಸಿ-ಗ-ಳ-ನ್ನಾಗಿ
ಸಂನ್ಯಾ-ಸಿ-ಗ-ಳ-ನ್ನಾ-ಗಿ-ಸಿ-ದ್ದರು
ಸಂನ್ಯಾ-ಸಿ-ಗಳನ್ನು
ಸಂನ್ಯಾ-ಸಿ-ಗಳನ್ನೂ
ಸಂನ್ಯಾ-ಸಿ-ಗ-ಳಲ್ಲ
ಸಂನ್ಯಾ-ಸಿ-ಗ-ಳ-ಲ್ಲವೆ
ಸಂನ್ಯಾ-ಸಿ-ಗಳಲ್ಲಿ
ಸಂನ್ಯಾ-ಸಿ-ಗ-ಳಲ್ಲೂ
ಸಂನ್ಯಾ-ಸಿ-ಗ-ಳಾ-ಗ-ಬೇಕು
ಸಂನ್ಯಾ-ಸಿ-ಗ-ಳಾ-ಗ-ಬೇ-ಕೆಂದು
ಸಂನ್ಯಾ-ಸಿ-ಗ-ಳಾಗಿ
ಸಂನ್ಯಾ-ಸಿ-ಗ-ಳಾ-ದ-ರಲ್ಲ
ಸಂನ್ಯಾ-ಸಿ-ಗ-ಳಾ-ದರೂ
ಸಂನ್ಯಾ-ಸಿ-ಗಳಿ
ಸಂನ್ಯಾ-ಸಿ-ಗ-ಳಿಗೂ
ಸಂನ್ಯಾ-ಸಿ-ಗ-ಳಿಗೆ
ಸಂನ್ಯಾ-ಸಿ-ಗ-ಳಿ-ಬ್ಬರೂ
ಸಂನ್ಯಾ-ಸಿ-ಗಳು
ಸಂನ್ಯಾ-ಸಿ-ಗಳೂ
ಸಂನ್ಯಾ-ಸಿ-ಗ-ಳೆಂ-ದರೆ
ಸಂನ್ಯಾ-ಸಿ-ಗ-ಳೆಂ-ಬುದೇ
ಸಂನ್ಯಾ-ಸಿ-ಗ-ಳೆ-ನಿ-ಸ-ಲಾ-ರರು
ಸಂನ್ಯಾ-ಸಿ-ಗ-ಳೆ-ನಿ-ಸಲು
ಸಂನ್ಯಾ-ಸಿ-ಗ-ಳೆಲ್ಲ
ಸಂನ್ಯಾ-ಸಿ-ಗ-ಳೆ-ಲ್ಲರೂ
ಸಂನ್ಯಾ-ಸಿ-ಗಳೇ
ಸಂನ್ಯಾ-ಸಿ-ಗ-ಳೊಂ-ದಿಗೆ
ಸಂನ್ಯಾ-ಸಿ-ಗೀತೆ
ಸಂನ್ಯಾ-ಸಿ-ಗೀ-ತೆಯ
ಸಂನ್ಯಾ-ಸಿ-ಗೀ-ತೆ-ಯಲ್ಲಿ
ಸಂನ್ಯಾ-ಸಿಗೆ
ಸಂನ್ಯಾ-ಸಿ-ಜೀ-ವ-ನಕ್ಕೆ
ಸಂನ್ಯಾ-ಸಿಯ
ಸಂನ್ಯಾ-ಸಿ-ಯಂತೆ
ಸಂನ್ಯಾ-ಸಿ-ಯನ್ನು
ಸಂನ್ಯಾ-ಸಿ-ಯ-ಲ್ಲ-ವಲ್ಲ
ಸಂನ್ಯಾ-ಸಿ-ಯ-ಲ್ಲ-ಸಂ-ನ್ಯಾ-ಸ-ವನ್ನೂ
ಸಂನ್ಯಾ-ಸಿ-ಯಾ-ಗ-ಬೇಕು
ಸಂನ್ಯಾ-ಸಿ-ಯಾ-ಗ-ಬೇ-ಕೆಂ-ದಿ-ರು-ವ-ವನು
ಸಂನ್ಯಾ-ಸಿ-ಯಾ-ಗ-ಬೇ-ಕೆಂಬ
ಸಂನ್ಯಾ-ಸಿ-ಯಾ-ಗಲು
ಸಂನ್ಯಾ-ಸಿ-ಯಾಗಿ
ಸಂನ್ಯಾ-ಸಿ-ಯಾಗು
ಸಂನ್ಯಾ-ಸಿ-ಯಾ-ಗು-ತ್ತಾ-ನಂ-ತಲ್ಲ
ಸಂನ್ಯಾ-ಸಿ-ಯಾ-ಗು-ತ್ತೇನೆ
ಸಂನ್ಯಾ-ಸಿ-ಯಾ-ಗು-ವು-ದಕ್ಕೂ
ಸಂನ್ಯಾ-ಸಿ-ಯಾ-ಗು-ವು-ದ-ರಲ್ಲಿ
ಸಂನ್ಯಾ-ಸಿ-ಯಾದ
ಸಂನ್ಯಾ-ಸಿ-ಯಾ-ದರೂ
ಸಂನ್ಯಾ-ಸಿ-ಯಾ-ದ-ವನು
ಸಂನ್ಯಾ-ಸಿ-ಯಿ-ರ-ಬ-ಹುದು
ಸಂನ್ಯಾ-ಸಿಯು
ಸಂನ್ಯಾ-ಸಿಯೂ
ಸಂನ್ಯಾ-ಸಿಯೆ
ಸಂನ್ಯಾ-ಸಿಯೇ
ಸಂನ್ಯಾ-ಸಿ-ಯೊ-ಬ್ಬನ
ಸಂನ್ಯಾಸೀ
ಸಂನ್ಯಾ-ಸೀ-ಪು-ತ್ರರು
ಸಂನ್ಯಾ-ಸೀ-ಬಂ-ಧು-ಗಳ
ಸಂನ್ಯಾ-ಸೀ-ಬಂ-ಧು-ಗ-ಳಾದ
ಸಂನ್ಯಾ-ಸೀ-ಬಂ-ಧು-ಗ-ಳಿ-ಗೆಲ್ಲ
ಸಂನ್ಯಾ-ಸೀ-ಶ-ರೀ-ರಕ್ಕೆ
ಸಂನ್ಯಾ-ಸೀ-ಶಿ-ಷ್ಯ-ರಾದ
ಸಂನ್ಯಾ-ಸೀ-ಶಿ-ಷ್ಯ-ರೆಲ್ಲ
ಸಂಪ-ತ್ತನ್ನು
ಸಂಪ-ತ್ತ-ನ್ನೆಲ್ಲ
ಸಂಪ-ತ್ತಿ-ಗಿಂತ
ಸಂಪ-ತ್ತಿಗೆ
ಸಂಪತ್ತು
ಸಂಪ-ತ್ತು-ಅ-ಧಿ-ಕಾ-ರ-ಗಳ
ಸಂಪ-ತ್ತು-ಗಳಲ್ಲಿ
ಸಂಪ-ನ್ನ-ನಾದ
ಸಂಪ-ನ್ಮೂ-ಲ-ಗ-ಳಾ-ಗಲಿ
ಸಂಪ-ರ್ಕ-ಬಾಂ-ಧ-ವ್ಯ-ಗಳು
ಸಂಪ-ರ್ಕ-ಸಂ-ಬಂ-ಧ-ವ-ನ್ನಿ-ಟ್ಟು-ಕೊಂಡು
ಸಂಪ-ರ್ಕಕ್ಕೆ
ಸಂಪ-ರ್ಕದ
ಸಂಪ-ರ್ಕ-ದಲ್ಲಿ
ಸಂಪ-ರ್ಕ-ದಿಂದ
ಸಂಪ-ರ್ಕ-ದಿಂ-ದಾಗಿ
ಸಂಪ-ರ್ಕ-ವನ್ನು
ಸಂಪ-ರ್ಕ-ವಿ-ಟ್ಟ-ಕೊಂ-ಡಿದ್ದ
ಸಂಪಾ-ದನೆ
ಸಂಪಾ-ದ-ನೆಯ
ಸಂಪಾ-ದ-ನೆ-ಯಲ್ಲೂ
ಸಂಪಾ-ದ-ನೆ-ಯೆಲ್ಲ
ಸಂಪಾ-ದಿ-ಸ-ಬೇ-ಕಾ-ದರೆ
ಸಂಪಾ-ದಿ-ಸಲು
ಸಂಪಾ-ದಿಸಿ
ಸಂಪಾ-ದಿ-ಸಿ-ಕೊಂಡ
ಸಂಪಾ-ದಿ-ಸಿ-ಕೊಂ-ಡು-ಬಿ-ಟ್ಟರೆ
ಸಂಪಾ-ದಿ-ಸಿ-ಕೊ-ಳ್ಳಲು
ಸಂಪಾ-ದಿ-ಸಿ-ದರೂ
ಸಂಪಾ-ದಿ-ಸಿ-ದ್ದನ್ನು
ಸಂಪಾ-ದಿಸು
ಸಂಪಾ-ದಿ-ಸುವ
ಸಂಪಿಗೆ
ಸಂಪುಟ
ಸಂಪು-ಟ-ಗಳನ್ನು
ಸಂಪು-ಟ-ಗ-ಳನ್ನೇ
ಸಂಪು-ಟ-ಗಳಲ್ಲಿ
ಸಂಪು-ಟ-ಗ-ಳ-ಲ್ಲಿನ
ಸಂಪು-ಟ-ಗಳು
ಸಂಪು-ಟ-ವಾದ
ಸಂಪೂರ್ಣ
ಸಂಪೂ-ರ್ಣ-ವಾಗಿ
ಸಂಪೂ-ರ್ಣ-ಸ್ವಾ-ತಂತ್ರ್ಯ
ಸಂಪ್ರ
ಸಂಪ್ರ-ದಾಯ
ಸಂಪ್ರ-ದಾ-ಯ-ಗಳ
ಸಂಪ್ರ-ದಾ-ಯ-ಗಳನ್ನು
ಸಂಪ್ರ-ದಾ-ಯ-ಗಳನ್ನೆಲ್ಲ
ಸಂಪ್ರ-ದಾ-ಯ-ಗಳಲ್ಲಿ
ಸಂಪ್ರ-ದಾ-ಯ-ಗ-ಳಲ್ಲೇ
ಸಂಪ್ರ-ದಾ-ಯ-ಗಳಿಂದ
ಸಂಪ್ರ-ದಾ-ಯ-ಗಳು
ಸಂಪ್ರ-ದಾ-ಯದ
ಸಂಪ್ರ-ದಾ-ಯ-ದಂತೆ
ಸಂಪ್ರ-ದಾ-ಯ-ದ-ವರು
ಸಂಪ್ರ-ದಾ-ಯ-ಪ್ರಿಯ
ಸಂಪ್ರ-ದಾ-ಯ-ವನ್ನು
ಸಂಪ್ರ-ದಾ-ಯವೇ
ಸಂಪ್ರ-ದಾ-ಯ-ಶ-ರ-ಣ-ರನ್ನು
ಸಂಪ್ರ-ದಾ-ಯಸ್ಥ
ಸಂಬಂಧ
ಸಂಬಂ-ಧ-ಗಳನ್ನು
ಸಂಬಂ-ಧ-ಗಳನ್ನೂ
ಸಂಬಂ-ಧ-ಗ-ಳನ್ನೇ
ಸಂಬಂ-ಧ-ಗಳು
ಸಂಬಂ-ಧ-ಗ-ಳೆಂಬ
ಸಂಬಂ-ಧ-ದಲ್ಲಿ
ಸಂಬಂ-ಧ-ಪಟ್ಟ
ಸಂಬಂ-ಧ-ಪ-ಟ್ಟಂತೆ
ಸಂಬಂ-ಧ-ವಂತೂ
ಸಂಬಂ-ಧ-ವನ್ನು
ಸಂಬಂ-ಧ-ವಾಗಿ
ಸಂಬಂ-ಧ-ವಾದ
ಸಂಬಂ-ಧವು
ಸಂಬಂ-ಧವೇ
ಸಂಬಂಧಿ
ಸಂಬಂ-ಧಿ-ಕರ
ಸಂಬಂ-ಧಿ-ಕ-ರಿ-ರಲಿ
ಸಂಬಂ-ಧಿ-ಕರು
ಸಂಬಂ-ಧಿ-ಗ-ಳಿಗೆ
ಸಂಬಂ-ಧಿ-ಗಳೇ
ಸಂಬಂ-ಧಿ-ಯಾದ
ಸಂಬಂ-ಧಿ-ಸಿದ
ಸಂಬಂ-ಧಿ-ಸಿ-ದಂತೆ
ಸಂಬಂ-ಧಿ-ಸಿ-ದು-ದೇ-ನನ್ನೂ
ಸಂಬಂ-ಧಿ-ಸಿದ್ದೇ
ಸಂಬ-ಳದ
ಸಂಬೋ-ಧಿ-ಸು-ವುದನ್ನು
ಸಂಬೋ-ಧಿ-ಸೋಣ
ಸಂಭವ
ಸಂಭ-ವ-ವಿತ್ತು
ಸಂಭ-ವ-ವಿದೆ
ಸಂಭ-ವ-ವಿ-ದ್ದರೆ
ಸಂಭ-ವ-ವಿ-ರು-ತ್ತದೆ
ಸಂಭ-ವ-ವಿ-ಲ್ಲ-ದಿ-ದ್ದರೆ
ಸಂಭ-ವವೇ
ಸಂಭ-ವಾಮಿ
ಸಂಭ-ವಿಸಿ
ಸಂಭ-ವಿ-ಸಿತು
ಸಂಭ-ವಿ-ಸಿದೆ
ಸಂಭ-ವಿ-ಸಿ-ದ್ದ-ರಿಂದ
ಸಂಭ-ವಿ-ಸು-ತ್ತಿವೆ
ಸಂಭ-ವಿ-ಸು-ವಂ-ಥ-ವು-ಗಳು
ಸಂಭ-ವಿ-ಸು-ವು-ದಿದೆ
ಸಂಭಾ-ವಿತ
ಸಂಭಾ-ಷಣ
ಸಂಭಾ-ಷಣಾ
ಸಂಭಾ-ಷಣೆ
ಸಂಭಾ-ಷ-ಣೆ-ಗಳನ್ನು
ಸಂಭಾ-ಷ-ಣೆಯ
ಸಂಭಾ-ಷ-ಣೆ-ಯನ್ನು
ಸಂಭಾ-ಷ-ಣೆ-ಯ-ನ್ನೆಲ್ಲ
ಸಂಭಾ-ಷ-ಣೆ-ಯಲ್ಲಿ
ಸಂಭಾ-ಷಿ-ಸಿದ
ಸಂಭಾ-ಷಿ-ಸಿ-ದರು
ಸಂಭಾ-ಷಿ-ಸು-ವು-ದೆಂ-ದರೆ
ಸಂಭ್ರ-ಮ-ದಿಂದ
ಸಂಮಿ-ಶ್ರಣ
ಸಂಯಮ
ಸಂಯ-ಮಕ್ಕೆ
ಸಂಯ-ಮ-ಗೊ-ಳಿ-ಸಲು
ಸಂಯ-ಮ-ದಿಂ-ದಿದ್ದು
ಸಂಯ-ಮ-ಪೂ-ರ್ಣ-ವಾದ
ಸಂಯ-ಮ-ವನ್ನು
ಸಂಯೋ-ಗವೇ
ಸಂರ-ಕ್ಷಿ-ಸ-ಬೇಕು
ಸಂರ-ಕ್ಷಿ-ಸು-ವಂ-ತೆಯೂ
ಸಂವ-ತ್ಸ-ರ-ಗಳ
ಸಂವಾದ
ಸಂವೇ-ದ-ನೆಯ
ಸಂಶಯ
ಸಂಶ-ಯ-ಅ-ಪ-ನಂ-ಬಿ-ಕೆ-ಗಳಿಂದ
ಸಂಶ-ಯ-ನಾ-ಸ್ತಿ-ಕ-ತೆ-ಯೆಲ್ಲ
ಸಂಶ-ಯಕ್ಕೆ
ಸಂಶ-ಯ-ಕ್ಕೆ-ಡೆ-ಯಿ-ಲ್ಲ-ದಂತೆ
ಸಂಶ-ಯ-ಗಳ
ಸಂಶ-ಯ-ಗಳನ್ನು
ಸಂಶ-ಯ-ಗಳನ್ನೆಲ್ಲ
ಸಂಶ-ಯ-ಗ-ಳೆಲ್ಲ
ಸಂಶ-ಯ-ಗ್ರಸ್ತ
ಸಂಶ-ಯದ
ಸಂಶ-ಯ-ದಿಂದ
ಸಂಶ-ಯ-ವಿಲ್ಲ
ಸಂಶ-ಯ-ವು-ಳಿ-ಯ-ಲಿಲ್ಲ
ಸಂಶ-ಯ-ವೆ-ದ್ದುದು
ಸಂಶ-ಯವೇ
ಸಂಶೋ-ಧನಾ
ಸಂಶ್ರ-ಯವೂ
ಸಂಸ-ರ್ಗ-ಇವು
ಸಂಸಾರ
ಸಂಸಾ-ರ-ಜೀ-ವನ
ಸಂಸಾ-ರ-ಜೀ-ವ-ನವೆ
ಸಂಸಾ-ರ-ಜೀ-ವ-ನವೇ
ಸಂಸಾ-ರ-ತ್ಯಾಗ
ಸಂಸಾ-ರದ
ಸಂಸಾ-ರ-ದಲ್ಲಿ
ಸಂಸಾ-ರ-ದ-ಲ್ಲಿದ್ದು
ಸಂಸಾ-ರ-ವ-ನ್ನಂತೂ
ಸಂಸಾ-ರ-ವನ್ನು
ಸಂಸಾ-ರವೋ
ಸಂಸಾ-ರಿ-ಗಳ
ಸಂಸಾ-ರಿ-ಯಾದ
ಸಂಸ್ಕ-ರ-ಣಕ್ಕೆ
ಸಂಸ್ಕಾ-ರ-ಗಳ
ಸಂಸ್ಕಾ-ರ-ಗಳಿಂದ
ಸಂಸ್ಕಾ-ರ-ಗಳು
ಸಂಸ್ಕಾ-ರ-ಬಂ-ಧ-ನ-ಗಳನ್ನು
ಸಂಸ್ಕಾ-ರ-ರಾ-ಶಿ-ಯನ್ನು
ಸಂಸ್ಕಾ-ರವೇ
ಸಂಸ್ಕೃತ
ಸಂಸ್ಕೃ-ತ-ಇಂ-ಗ್ಲಿಷ್
ಸಂಸ್ಕೃ-ತ-ಜ್ಞಾ-ನ-ವನ್ನು
ಸಂಸ್ಕೃ-ತ-ಜ್ಞಾ-ನವೂ
ಸಂಸ್ಕೃ-ತ-ದಲ್ಲಿ
ಸಂಸ್ಕೃತಿ
ಸಂಸ್ಕೃ-ತಿ-ನಾ-ಗ-ರಿ-ಕ-ತೆ-ಗಳು
ಸಂಸ್ಕೃ-ತಿ-ಪ-ರಂ-ಪ-ರೆ-ಗಳು
ಸಂಸ್ಕೃ-ತಿ-ಗಳ
ಸಂಸ್ಕೃ-ತಿ-ಗ-ಳತ್ತ
ಸಂಸ್ಕೃ-ತಿ-ಗಳನ್ನು
ಸಂಸ್ಕೃ-ತಿಗೆ
ಸಂಸ್ಕೃ-ತಿಯ
ಸಂಸ್ಕೃ-ತಿ-ಯನ್ನು
ಸಂಸ್ಕೃ-ತಿ-ಯಿಂದ
ಸಂಸ್ಕೃ-ತಿಯೂ
ಸಂಸ್ಕೃ-ತಿ-ಯೊಂದು
ಸಂಸ್ಥಾ-ಪನಾ
ಸಂಸ್ಥಾ-ಪನೆ
ಸಂಸ್ಥಾ-ಪ-ನೆಯೇ
ಸಂಸ್ಥಾ-ಪಿ-ಸುವ
ಸಂಸ್ಥೆ
ಸಂಸ್ಥೆಗೆ
ಸಂಸ್ಥೆಯ
ಸಂಸ್ಥೆ-ಯನ್ನು
ಸಂಸ್ಥೆ-ಯಲ್ಲಿ
ಸಂಸ್ಥೆ-ಯಲ್ಲೇ
ಸಂಸ್ಥೆ-ಯಾ-ಗಲಿ
ಸಂಸ್ಥೆ-ಯಿದು
ಸಂಸ್ವ-ರ್ಶ-ವೆ-ನ್ನು-ವುದು
ಸಕಲ
ಸಕ-ಲ-ರನ್ನೂ
ಸಕ-ಲ-ರಲ್ಲು
ಸಕ-ಲ-ರಿಗೂ
ಸಕ-ಲರೂ
ಸಕ-ಲ-ವನ್ನೂ
ಸಕ-ಲವೂ
ಸಕ-ಲೈ-ಶ್ವ-ರ್ಯದ
ಸಕ-ಲೈ-ಶ್ವ-ರ್ಯ-ವನ್ನೂ
ಸಕಾ-ರಾ-ತ್ಮ-ಕ-ವಾ-ದದ್ದು
ಸಕಾ-ಲಕ್ಕೆ
ಸಕಾ-ಲ-ದಲ್ಲಿ
ಸಕಾ-ಲಿಕ
ಸಕ್ಕರೆ
ಸಖ
ಸಗುಣ
ಸಗು-ಣ-ನಿ-ರಾ-ಕಾರ
ಸಗು-ಣ-ನಿ-ರಾ-ಕಾ-ರ-ಬ್ರ-ಹ್ಮದ
ಸಗು-ಣ-ನಿ-ರ್ಗುಣ
ಸಗು-ಣ-ನಿ-ರ್ಗು-ಣ-ಗಳ
ಸಗು-ಣ-ಸಾ-ಕಾರ
ಸಗು-ಣ-ಸಾ-ಕಾ-ರ-ನೆಂದು
ಸಗು-ಣನೂ
ಸಚೇ-ತ-ನ-ವಾಗಿ
ಸಚ್ಚಿದಾ
ಸಚ್ಚಿ-ದಾನಂ
ಸಚ್ಚಿ-ದಾ-ನಂದ
ಸಚ್ಚಿ-ದಾ-ನಂ-ದ-ಗು-ರು-ವಾದ
ಸಚ್ಚಿ-ದಾ-ನಂ-ದನ
ಸಚ್ಚಿ-ದಾ-ನಂ-ದವೇ
ಸಚ್ಚಿ-ದಾ-ನಂ-ದ-ಸಾ-ಗರ
ಸಚ್ಚಿ-ದಾ-ನಂ-ದ-ಸಾ-ಗ-ರ-ದಿಂದ
ಸಜ್ಜನ
ಸಜ್ಜ-ನ-ಸು-ಚ-ರಿ-ತ್ರ-ನೆ-ನಿ-ಸಿ-ಕೊ-ಳ್ಳ-ಬೇಕಾ
ಸಜ್ಜ-ನ-ನಾ-ಗಿ-ರ-ಬೇಕು
ಸಜ್ಜ-ನ-ರ-ಲ್ಲದೆ
ಸಜ್ಜ-ನ-ರಾ-ಗಿ-ರ-ಬ-ಹುದು
ಸಡಿ-ಲಿ-ಸಿ-ದ್ದಿ-ರ-ಬ-ಹುದು
ಸಣ್ಣ
ಸಣ್ಣ-ಗಾಗಿ
ಸಣ್ಣ-ಗಾ-ಗಿ-ಬಿ-ಟ್ಟರು
ಸಣ್ಣ-ಪುಟ್ಟ
ಸಣ್ಣ-ಮಂ-ಚದ
ಸತತ
ಸತಾ-ಯಿ-ಸು-ತ್ತಿ-ದ್ದಾ-ನಲ್ಲ
ಸತಾ-ಯಿ-ಸು-ವಷ್ಟು
ಸತಿಯು
ಸತೀ-ಶ-ಚಂದ್ರ
ಸತ್
ಸತ್-ಚಿ-ತ್-ಆ-ನಂದ
ಸತ್-ಚಿ-ತ್-ಆ-ನಂ-ದ-ಸ್ವ-ರೂ-ಪ-ನಾದ
ಸತ್ಕ-ರಿ-ಸ-ಬೇಕು
ಸತ್ಕ-ರಿ-ಸ-ಲೇ-ಬೇ-ಕಂತೆ
ಸತ್ಕ-ರಿ-ಸಿದ
ಸತ್ಕ-ರಿ-ಸಿ-ದರು
ಸತ್ಕ-ರಿ-ಸು-ವು-ದ-ಕ್ಕಾಗಿ
ಸತ್ಕ-ರ್ಮ-ಗಲ
ಸತ್ಕಾ-ರದ
ಸತ್ಕಾ-ರ್ಯ-ಗ-ಳಿ-ಗಾಗಿ
ಸತ್ಕಾ-ರ್ಯ-ಗಳು
ಸತ್ಕು-ಲ-ಪ್ರ-ಸೂ-ತ-ನಾದ
ಸತ್ತ
ಸತ್ತ-ದ್ದ-ರಿಂದ
ಸತ್ತರು
ಸತ್ತು
ಸತ್ತು-ಹೋ-ಗಿರ
ಸತ್ತೇ
ಸತ್ತೇ-ಹೋ-ಗಿ-ಬಿ-ಡ-ಬೇ-ಕಾ-ಗಿತ್ತು
ಸತ್ತೇ-ಹೋ-ದ-ನೆಂದು
ಸತ್ತ್ವ
ಸತ್ತ್ವ-ಗುಣ
ಸತ್ತ್ವ-ಪ-ರೀಕ್ಷೆ
ಸತ್ತ್ವ-ರ-ಜ-ಸ್ತ-ಮೋ-ಗು-ಣ-ಗ-ಳಿ-ಗ-ನು-ಸಾ-ರ-ವಾಗಿ
ಸತ್ತ್ವ-ವನ್ನೇ
ಸತ್ತ್ವ-ವಿದೆ
ಸತ್ತ್ವ-ವಿ-ದೆಯೆ
ಸತ್ತ್ವ-ವಿಲ್ಲ
ಸತ್ಪ-ರಿ-ಣಾ-ಮ-ವ-ನ್ನುಂ-ಟು-ಮಾ-ಡಿ-ದವು
ಸತ್ಪ-ರಿ-ವ-ರ್ತ-ನೆ-ಯನ್ನು
ಸತ್ಪು-ರು-ಷ-ನನ್ನು
ಸತ್ಪು-ರು-ಷರ
ಸತ್ಫ-ಲ-ವನ್ನು
ಸತ್ಯ
ಸತ್ಯ-ಅ-ದನ್ನು
ಸತ್ಯ-ಎಂದು
ಸತ್ಯ-ಗ-ಳನ್ನೇ
ಸತ್ಯ-ಜ್ಞಾ-ನ-ಕ್ಕಾಗಿ
ಸತ್ಯ-ಜ್ಞಾ-ನಾ-ನಂ-ದ-ಸ್ವ-ರೂ-ಪ-ನಾದ
ಸತ್ಯತೆ
ಸತ್ಯ-ತೆಯ
ಸತ್ಯ-ತೆ-ಯನ್ನು
ಸತ್ಯದ
ಸತ್ಯ-ದಂ-ತೆಯೇ
ಸತ್ಯ-ದ-ರ್ಶನ
ಸತ್ಯ-ದೂ-ರ-ವಾದ
ಸತ್ಯ-ದೆ-ಡೆಗೆ
ಸತ್ಯ-ಧಾ-ಮದ
ಸತ್ಯ-ಪ-ರ-ತೆಯೇ
ಸತ್ಯ-ವನ್ನು
ಸತ್ಯ-ವ-ಲ್ಲ-ವೆಂದು
ಸತ್ಯ-ವಾಗಿ
ಸತ್ಯ-ವಾ-ಗಿಯೂ
ಸತ್ಯ-ವಾ-ಯಿತು
ಸತ್ಯ-ವಿತ್ತು
ಸತ್ಯವು
ಸತ್ಯವೆ
ಸತ್ಯ-ವೆಂದರೆ
ಸತ್ಯ-ವೆಂದು
ಸತ್ಯ-ವೆಂ-ಬುದು
ಸತ್ಯ-ವೆಂ-ಬು-ವು-ದಲ್ಲಿ
ಸತ್ಯ-ವೇ-ನೆಂ-ಬು-ದನ್ನು
ಸತ್ಯ-ವೊಂದ
ಸತ್ಯ-ವ್ರ-ತ-ನಾದ
ಸತ್ಯ-ಶೋ-ಧಕ
ಸತ್ಯ-ಶೋ-ಧ-ಕರು
ಸತ್ಯ-ಶೋ-ಧನೆ
ಸತ್ಯ-ಶೋ-ಧ-ನೆಯ
ಸತ್ಯ-ಸಂ-ಗ-ತಿ-ಯನ್ನು
ಸತ್ಯ-ಸಂ-ಧ-ತೆ-ಯನ್ನು
ಸತ್ಯ-ಸಾ-ಕ್ಷಾ-ತ್ಕಾರ
ಸತ್ಯ-ಸಾ-ಕ್ಷಾ-ತ್ಕಾ-ರ-ಕ್ಕಾಗಿ
ಸತ್ಯ-ಸಾ-ಕ್ಷಾ-ತ್ಕಾ-ರಕ್ಕೆ
ಸತ್ಯ-ಸಾ-ಕ್ಷಾ-ತ್ಕಾ-ರದ
ಸತ್ಯ-ಸಾ-ಕ್ಷಾ-ತ್ಕಾ-ರ-ದೆ-ಡೆಗೆ
ಸತ್ಯ-ಸಾ-ಕ್ಷಾ-ತ್ಕಾ-ರ-ವೆಂದರೆ
ಸತ್ಯ-ಸ್ವ-ರೂ-ಪ-ನಾದ
ಸತ್ಯ-ಸ್ವ-ರೂಪಿ
ಸತ್ಯಾಂ-ಶ-ಗಳು
ಸತ್ರ-ದಲ್ಲಿ
ಸತ್ರ-ವೊಂ-ದನ್ನು
ಸತ್ವ-ಗುಣ
ಸತ್ವದ
ಸತ್ವ-ರ-ಹಿತ
ಸತ್ಸಂ-ಗ-ದಲ್ಲಿ
ಸತ್ಸಂಪ್ರ
ಸತ್ಸ-ಹ-ವಾ-ಸದ
ಸತ್ಸ-ಹ-ವಾ-ಸ-ದಲ್ಲಿ
ಸದ-ವ-ಕಾಶ
ಸದ-ವ-ಕಾ-ಶ-ವಾ-ಯಿತು
ಸದ-ಸ್ಯ-ನಲ್ಲೂ
ಸದ-ಸ್ಯ-ನಾ-ಗಿದ್ದ
ಸದ-ಸ್ಯ-ರಂ-ತಲ್ಲ
ಸದ-ಸ್ಯ-ರನ್ನು
ಸದ-ಸ್ಯ-ರಾ-ಗಿ-ದ್ದರು
ಸದ-ಸ್ಯ-ರಿಗೂ
ಸದ-ಸ್ಯರು
ಸದ-ಸ್ಯ-ರೆಲ್ಲ
ಸದಾ
ಸದಾ-ಚಾ-ರ-ಸಂ-ಪ-ನ್ನ-ನಾ-ಗಿ-ರ-ಬೇಕು
ಸದಾ-ನಂದ
ಸದಾ-ನಂ-ದರು
ಸದಾ-ಶಿವ
ಸದು-ದ್ದೇಶ
ಸದ್ಗುಣ
ಸದ್ಗು-ಣ-ಗಳನ್ನು
ಸದ್ಗು-ಣ-ಗಳು
ಸದ್ಗು-ಣ-ವಂ-ತ-ನಂತೆ
ಸದ್ಗೃ-ಹ-ಸ್ಥರ
ಸದ್ದನ್ನು
ಸದ್ದಿ-ಲ್ಲದೆ
ಸದ್ಯ-ದಲ್ಲೇ
ಸನಾ-ತನ
ಸನಿ-ಹ-ದಲ್ಲಿ
ಸನಿ-ಹ-ದಲ್ಲೇ
ಸನ್ನಾ-ಹ-ದ-ಲ್ಲಿತ್ತು
ಸನ್ನಿ-ಧಾ-ನ-ದಲ್ಲೇ
ಸನ್ನಿಧಿ
ಸನ್ನಿ-ಧಿ-ನ-ಹ-ವಾ-ಸ-ದಿಂ-ದಾಗಿ
ಸನ್ನಿ-ಧಿ-ಯಲ್ಲಿ
ಸನ್ನಿ-ಧಿ-ಯ-ಲ್ಲಿ-ದ್ದಾಗ
ಸನ್ನಿ-ಧಿ-ಯ-ಲ್ಲಿ-ರು-ವ-ವ-ರೆಗೂ
ಸನ್ನಿ-ಧಿ-ಯೆಂ-ಬುದು
ಸನ್ನಿ-ವೇ-ಶಕ್ಕೆ
ಸನ್ನಿ-ವೇ-ಶ-ಗಳಲ್ಲಿ
ಸನ್ನಿ-ವೇ-ಶ-ಗ-ಳ-ಲ್ಲಿಯೇ
ಸನ್ನಿ-ವೇ-ಶ-ಗಳಿಂದ
ಸನ್ನಿ-ವೇ-ಶ-ದಲ್ಲಿ
ಸನ್ನಿ-ವೇ-ಶ-ವನ್ನು
ಸನ್ನಿ-ಹಿ-ತ-ವಾ-ಗಿತ್ತು
ಸನ್ನಿ-ಹಿ-ತ-ವಾ-ಗಿದೆ
ಸನ್ನಿ-ಹಿ-ತ-ವಾ-ಗು-ತ್ತಿತ್ತು
ಸನ್ಮಾನ
ಸನ್ಮಾ-ನ-ವಾ-ದಷ್ಟು
ಸನ್ಯಾಲ
ಸಪ-ತ್ನೀ-ಕ-ರಾಗಿ
ಸಪ್ತ
ಸಪ್ತ-ಪ-ರ್ಷಿ-ಗಳ
ಸಪ್ಪ-ಳ-ವನ್ನೂ
ಸಪ್ಪೆ
ಸಪ್ಪೆ-ಮೋರೆ
ಸಭಿ-ಕ-ರಿಗೆ
ಸಭಿ-ಕ-ರೆಲ್ಲ
ಸಭೆ
ಸಭೆ-ಗಳಲ್ಲಿ
ಸಭೆ-ಗ-ಳಿಗೆ
ಸಭೆಗೆ
ಸಭೆಯ
ಸಭೆ-ಯಲ್ಲಿ
ಸಭ್ಯ-ತೆ-ಯಿಂದ
ಸಭ್ಯ-ರಂತೆ
ಸಭ್ಯ-ಸ್ಥನ
ಸಮ
ಸಮಂ-ಜ-ಸ-ವಾದ
ಸಮ-ಕಾ-ಲೀನ
ಸಮಗ್ರ
ಸಮ-ಗ್ರ-ದೃ-ಷ್ಟಿ-ಯಿಂದ
ಸಮ-ಗ್ರ-ವಾಗಿ
ಸಮ-ಗ್ರ-ವಾದ
ಸಮ-ತೋ-ಲ-ಸಾ-ಮ-ರ-ಸ್ಯ-ಗಳನ್ನು
ಸಮ-ನಾಗಿ
ಸಮನೆ
ಸಮ-ನ್ವ-ಯ-ಗೊ-ಳಿ-ಸಿ-ದರು
ಸಮ-ನ್ವ-ಯ-ವನ್ನು
ಸಮಯ
ಸಮ-ಯಕ್ಕೆ
ಸಮ-ಯ-ಗಳಲ್ಲಿ
ಸಮ-ಯ-ದಲ್ಲಿ
ಸಮ-ಯ-ದಲ್ಲೂ
ಸಮ-ಯ-ದಲ್ಲೇ
ಸಮ-ಯ-ನ-ಷ್ಟ-ಶ-ಕ್ತಿ-ನಷ್ಟ
ಸಮ-ಯ-ಪ್ರ-ಜ್ಞೆಯೂ
ಸಮ-ಯ-ವನ್ನು
ಸಮ-ಯ-ವ-ನ್ನೆಲ್ಲ
ಸಮ-ಯ-ವೆಲ್ಲ
ಸಮ-ಯವೇ
ಸಮ-ಯ-ಸ್ಫೂರ್ತಿ
ಸಮರ
ಸಮ-ರ-ಸ-ತೆ-ಯನ್ನು
ಸಮ-ರ-ಸ-ದಿಂದ
ಸಮರ್ಥ
ಸಮ-ರ್ಥ-ನಾ-ಗ-ಬೇ-ಕಾ-ದರೆ
ಸಮ-ರ್ಥ-ನಾ-ಗಿದ್ದ
ಸಮ-ರ್ಥ-ನಾ-ಗು-ತ್ತಾನೆ
ಸಮ-ರ್ಥ-ನಾ-ಗುವೆ
ಸಮ-ರ್ಥ-ನಾದ
ಸಮ-ರ್ಥ-ನಾ-ದದ್ದು
ಸಮ-ರ್ಥ-ರಾಗಿ
ಸಮ-ರ್ಥ-ರಾ-ಗಿ-ದ್ದರು
ಸಮ-ರ್ಥ-ರೆ-ನ್ನ-ಬ-ಹುದು
ಸಮ-ರ್ಥ-ಳ-ಲ್ಲವೆ
ಸಮ-ರ್ಥ-ವಾ-ಗ-ಲಿಲ್ಲ
ಸಮ-ರ್ಥ-ವಾಗಿ
ಸಮ-ರ್ಥಿ-ಸ-ಲಿ-ಲ್ಲ-ವಾ-ದರೂ
ಸಮ-ರ್ಥಿಸಿ
ಸಮ-ರ್ಥಿ-ಸಿದ
ಸಮ-ರ್ಥಿ-ಸಿ-ದರು
ಸಮ-ರ್ಥಿ-ಸುವ
ಸಮ-ರ್ಪ-ಕ-ವಾದ
ಸಮ-ರ್ಪಣೆ
ಸಮ-ರ್ಪಿಸಿ
ಸಮ-ರ್ಪಿ-ಸಿ-ಕೊಂ-ಡ-ವನು
ಸಮ-ರ್ಪಿ-ಸಿ-ಕೊಂ-ಡಿ-ರುವ
ಸಮ-ರ್ಪಿ-ಸಿ-ಕೊಂಡು
ಸಮ-ರ್ಪಿ-ಸಿ-ಕೊಂ-ಡು-ಬಿಡು
ಸಮ-ರ್ಪಿ-ಸಿ-ದರು
ಸಮ-ವಸ್ತ್ರ
ಸಮ-ಸ್ಕ-ರಿಸಿ
ಸಮಸ್ತ
ಸಮಸ್ಯೆ
ಸಮ-ಸ್ಯೆ-ಗಳನ್ನು
ಸಮ-ಸ್ಯೆ-ಗಳನ್ನೂ
ಸಮ-ಸ್ಯೆ-ಗಳು
ಸಮ-ಸ್ಯೆ-ಗಳೂ
ಸಮ-ಸ್ಯೆಗೂ
ಸಮ-ಸ್ಯೆಯ
ಸಮ-ಸ್ಯೆ-ಯೇ-ನೆಂ-ದ-ರೆ-ಅ-ವ-ತಾ-ರ-ಗಳು
ಸಮ-ಸ್ಯೆ-ಯೊಂದು
ಸಮಾ-ಗ-ಮ-ದಿಂದ
ಸಮಾ-ಗ-ಮ-ವಾ-ದದ್ದು
ಸಮಾ-ಚಾರ
ಸಮಾಜ
ಸಮಾ-ಜಕ್ಕೂ
ಸಮಾ-ಜಕ್ಕೆ
ಸಮಾ-ಜದ
ಸಮಾ-ಜ-ದಲ್ಲಿ
ಸಮಾ-ಜ-ದ-ವರು
ಸಮಾ-ಜ-ದಿಂದ
ಸಮಾ-ಜ-ವನ್ನು
ಸಮಾ-ಜವು
ಸಮಾ-ಜ-ಶಾಸ್ತ್ರ
ಸಮಾ-ಜ-ಹಿತ
ಸಮಾ-ಧಾನ
ಸಮಾ-ಧಾ-ನ-ಕರ
ಸಮಾ-ಧಾ-ನ-ಕ-ರ-ವಾಗಿ
ಸಮಾ-ಧಾ-ನ-ಗಳನ್ನು
ಸಮಾ-ಧಾ-ನ-ಗೊಂ-ಡಿತ್ತು
ಸಮಾ-ಧಾ-ನ-ಗೊಂಡು
ಸಮಾ-ಧಾ-ನ-ಗೊ-ಳಿ-ಸಿ-ದರು
ಸಮಾ-ಧಾ-ನ-ಚಿ-ತ್ತದ
ಸಮಾ-ಧಾ-ನದ
ಸಮಾ-ಧಾ-ನ-ದಿಂ-ದಲೇ
ಸಮಾ-ಧಾ-ನ-ದಿಂ-ದಿ-ರು-ವುದನ್ನು
ಸಮಾ-ಧಾ-ನ-ಪ-ಟ್ಟು-ಕೊ-ಳ್ಳಲು
ಸಮಾ-ಧಾ-ನ-ಪ-ಡಿಸಿ
ಸಮಾ-ಧಾ-ನ-ಪ-ಡಿ-ಸಿ-ದರು
ಸಮಾ-ಧಾ-ನ-ಪ-ಡಿ-ಸುತ್ತ
ಸಮಾ-ಧಾ-ನ-ಪ-ಡಿ-ಸುವ
ಸಮಾ-ಧಾ-ನ-ವನ್ನು
ಸಮಾ-ಧಾ-ನ-ವಾ-ಗು-ತ್ತಿಲ್ಲ
ಸಮಾ-ಧಾ-ನ-ವಾ-ಗು-ವು-ದಿ-ರಲಿ
ಸಮಾ-ಧಾ-ನ-ವಾ-ಯಿತು
ಸಮಾ-ಧಾ-ನ-ವಿಲ್ಲ
ಸಮಾ-ಧಾ-ನ-ವುಂ-ಟು-ಮಾ-ಡಿ-ದಾಗ
ಸಮಾ-ಧಾ-ನವೂ
ಸಮಾ-ಧಾ-ನವೇ
ಸಮಾ-ಧಾ-ನ-ಸ್ಥಿ-ತಿಗೆ
ಸಮಾಧಿ
ಸಮಾ-ಧಿ-ಗಿಂ-ತಲೂ
ಸಮಾ-ಧಿ-ಗೇ-ರ-ದಂತೆ
ಸಮಾ-ಧಿ-ಗೇರಿ
ಸಮಾ-ಧಿ-ಗೇ-ರಿ-ದರು
ಸಮಾ-ಧಿ-ಗೇ-ರು-ತ್ತಾನೆ
ಸಮಾ-ಧಿ-ಗೇ-ರುವ
ಸಮಾ-ಧಿ-ಗೇ-ರು-ವುದನ್ನು
ಸಮಾ-ಧಿ-ಗೇ-ರು-ವುದು
ಸಮಾ-ಧಿ-ಮ-ಗ್ನ-ನಾ-ಗಿ-ದ್ದು-ಬಿ-ಡ-ಬೇ-ಕೆಂದು
ಸಮಾ-ಧಿ-ಮ-ಗ್ನ-ರಾಗಿ
ಸಮಾ-ಧಿ-ಮ-ಗ್ನ-ರಾ-ಗಿ-ಬಿ-ಟ್ಟರು
ಸಮಾ-ಧಿ-ಮ-ಗ್ನ-ರಾ-ದು-ದಕ್ಕೆ
ಸಮಾ-ಧಿ-ಮ-ಗ್ನ-ವಾದ
ಸಮಾ-ಧಿಯ
ಸಮಾ-ಧಿ-ಯನ್ನು
ಸಮಾ-ಧಿ-ಯಲ್ಲಿ
ಸಮಾ-ಧಿ-ಯಿಂದ
ಸಮಾ-ಧಿ-ಯಿಂ-ದಿ-ಳಿದು
ಸಮಾ-ಧಿ-ಯಿಂ-ದೆದ್ದು
ಸಮಾ-ಧಿ-ಸುಖ
ಸಮಾ-ಧಿ-ಸು-ಖ-ಕ್ಕಾಗಿ
ಸಮಾ-ಧಿ-ಸು-ಖದ
ಸಮಾ-ಧಿ-ಸ್ಥ-ನಾ-ಗಿ-ಬಿಟ್ಟ
ಸಮಾ-ಧಿ-ಸ್ಥ-ನಾ-ಗಿ-ಬಿಟ್ಟೆ
ಸಮಾ-ಧಿ-ಸ್ಥ-ರಾಗಿ
ಸಮಾ-ಧಿ-ಸ್ಥ-ರಾ-ಗಿ-ಬಿ-ಟ್ಟರು
ಸಮಾ-ಧಿ-ಸ್ಥ-ರಾ-ಗಿ-ಬಿ-ಟ್ಟಿ-ದ್ದರು
ಸಮಾ-ಧಿ-ಸ್ಥ-ರಾ-ಗಿ-ಬಿ-ಟ್ಟಿ-ರು-ತ್ತಿ-ದ್ದರು
ಸಮಾ-ಧಿ-ಸ್ಥ-ರಾ-ದರು
ಸಮಾ-ಧಿ-ಸ್ಥ-ರಾ-ದು-ದನ್ನು
ಸಮಾ-ಧಿ-ಸ್ಥಿ-ತಿ-ಗಾಗಿ
ಸಮಾ-ಧಿ-ಸ್ಥಿ-ತಿ-ಗಿಂ-ತಲೂ
ಸಮಾ-ಧಿ-ಸ್ಥಿ-ತಿಗೆ
ಸಮಾ-ಧಿ-ಸ್ಥಿ-ತಿ-ಗೇರ
ಸಮಾ-ಧಿ-ಸ್ಥಿ-ತಿ-ಗೇ-ರಿ-ದರು
ಸಮಾ-ಧಿ-ಸ್ಥಿ-ತಿ-ಗೇ-ರಿಸಿ
ಸಮಾ-ಧಿ-ಸ್ಥಿ-ತಿ-ಗೇ-ರಿ-ಸಿ-ದ್ದರು
ಸಮಾ-ಧಿ-ಸ್ಥಿ-ತಿ-ಗೇ-ರುವ
ಸಮಾ-ಧಿ-ಸ್ಥಿ-ತಿ-ಯನ್ನು
ಸಮಾ-ಧಿ-ಸ್ಥಿ-ತಿ-ಯನ್ನೂ
ಸಮಾ-ಧಿ-ಸ್ಥಿ-ತಿ-ಯಲ್ಲಿ
ಸಮಾ-ಧಿ-ಸ್ಥಿ-ತಿ-ಯಲ್ಲೂ
ಸಮಾ-ಧಿ-ಸ್ಥಿ-ತಿ-ಯಲ್ಲೇ
ಸಮಾ-ಧಿ-ಸ್ಥಿ-ತಿ-ಯಿಂದ
ಸಮಾ-ಧಿ-ಸ್ಥಿ-ತಿ-ಯುಂ-ಟಾ-ಗು-ತ್ತಿ-ದ್ದು-ದ-ರಿಂದ
ಸಮಾನ
ಸಮಾ-ನ-ದೃ-ಷ್ಟಿ-ಯಿಂದ
ಸಮಾ-ನ-ವಾಗಿ
ಸಮಾ-ನ-ಸ್ಕಂ-ಧ-ನೆಂ-ಬಂತೆ
ಸಮಾ-ನಾ-ರ್ಥದ
ಸಮಾ-ರಂಭ
ಸಮಾ-ರಂ-ಭ-ವೊಂ-ದನ್ನು
ಸಮಾ-ಲೋ-ಚಿ-ಸು-ತ್ತಿ-ದ್ದರು
ಸಮಾ-ವೇಶ
ಸಮೀಪ
ಸಮೀ-ಪದ
ಸಮೀ-ಪ-ದಲ್ಲಿ
ಸಮೀ-ಪ-ದಲ್ಲೇ
ಸಮೀ-ಪಿ-ಸಲೂ
ಸಮೀ-ಪಿಸಿ
ಸಮೀ-ಪಿ-ಸಿತು
ಸಮೀ-ಪಿ-ಸಿದೆ
ಸಮೀ-ಪಿ-ಸು-ತ್ತಿತ್ತು
ಸಮೀ-ಪಿ-ಸು-ತ್ತಿದೆ
ಸಮೀ-ಪಿ-ಸು-ತ್ತಿ-ದೆ-ಯೆಂಬು
ಸಮೀ-ಪಿ-ಸು-ತ್ತಿ-ರು-ವುದ
ಸಮೀ-ಪಿ-ಸು-ತ್ತಿವೆ
ಸಮೃದ್ಧ
ಸಮೃ-ದ್ಧ-ವಾ-ಗಿದೆ
ಸಮೃ-ದ್ಧಿಯ
ಸಮೇ-ತ-ವಾಗಿ
ಸಮ್ಮತ
ಸಮ್ಮ-ತ-ವಾ-ಗಲೇ
ಸಮ್ಮ-ತ-ವಾ-ದುವು
ಸಮ್ಮ-ತ-ವಾ-ಯಿತು
ಸಮ್ಮ-ತಿ
ಸಮ್ಮ-ತಿ-ಯನ್ನು
ಸಮ್ಮ-ತಿ-ಯಿದೆ
ಸಮ್ಮ-ತಿಯೂ
ಸಮ್ಮ-ತಿ-ಸಿ-ದರು
ಸಮ್ಮ-ನಿ-ರಲು
ಸಮ್ಮನೆ
ಸಮ್ಮಾ-ನಿತ
ಸಮ್ಮಿ-ಲನ
ಸಮ್ಮು-ಖ-ದಲ್ಲಿ
ಸಮ್ಮು-ಖ-ದಲ್ಲೇ
ಸಮ್ಮೇ-ಳ-ನ-ದಿಂದ
ಸಮ್ಮೋ-ಹಿ-ನಿ-ವ-ಶೀ-ಕ-ರಣ
ಸಮ್ಮೋ-ಹಿ-ನಿಗೆ
ಸಮ್ಮೋ-ಹಿನೀ
ಸರ-ಕಾರ
ಸರ-ಕಾರೀ
ಸರಕು
ಸರದಿ
ಸರ-ಪಣಿ
ಸರ-ಪ-ಣಿ-ಯಿಂದ
ಸರ-ಪಳಿ
ಸರ-ಪ-ಳಿ-ಗಳ
ಸರ-ಪ-ಳಿ-ಗಳಲ್ಲಿ
ಸರ-ಪ-ಳಿ-ಯೆಂ-ದರೆ
ಸರ-ಪ-ಳಿಯೇ
ಸರಳ
ಸರ-ಳ-ಶುದ್ಧ
ಸರ-ಳ-ತೆ-ಯನ್ನು
ಸರ-ಳ-ತೆಯೂ
ಸರ-ಳ-ಬಾ-ಡಿಗೆ
ಸರ-ಳ-ವಾದ
ಸರ-ಳ-ಶ್ರ-ದ್ಧೆ-ಯನ್ನು
ಸರ-ಸ-ರನೆ
ಸರ-ಸ-ಲೇ-ಬೇ-ಕಾ-ಗು-ತ್ತದೆ
ಸರ-ಸ-ವಾಗಿ
ಸರ-ಸ್ವತಿ
ಸರ-ಸ್ವ-ತಿಗೆ
ಸರ-ಸ್ವ-ತಿ-ಯ-ವ-ರಿಂದ
ಸರಾಫು
ಸರಾ-ಸ-ರಿ-ಗಿಂತ
ಸರಿ
ಸರಿ-ದದ್ದೇ
ಸರಿ-ದರು
ಸರಿ-ದಾರಿಗೆ
ಸರಿದು
ಸರಿ-ದೂ-ಗುವ
ಸರಿ-ದೂ-ಗು-ವಂ-ಥ-ವ-ರನ್ನು
ಸರಿ-ಪ-ಡಿ-ಸುವ
ಸರಿ-ಬ-ರ-ಲಿಲ್ಲ
ಸರಿ-ಯಲ್ಲ
ಸರಿ-ಯ-ಲ್ಲವೆ
ಸರಿ-ಯಾಗಿ
ಸರಿ-ಯಾ-ಗಿದೆ
ಸರಿ-ಯಾ-ಗಿ-ದ್ದು-ಬಿಟ್ಟ
ಸರಿ-ಯಾ-ಗಿಯೇ
ಸರಿ-ಯಾ-ಗಿ-ರು-ವ-ವರು
ಸರಿ-ಯಾ-ಗಿ-ರು-ವಾಗ
ಸರಿ-ಯಾಗೇ
ಸರಿ-ಯಾದ
ಸರಿ-ಯಾ-ದ-ದ್ದನ್ನೇ
ಸರಿ-ಯಾ-ದ್ದದು
ಸರಿ-ಯಿ-ರ-ಲಿಲ್ಲ
ಸರಿ-ಯಿ-ಲ್ಲ-ದ್ದ-ರಿಂದ
ಸರಿ-ಯು-ತ್ತಾನೆ
ಸರಿ-ಯು-ತ್ತಿದೆ
ಸರಿಯೆ
ಸರಿ-ಯೆಂದ
ಸರಿ-ಯೆಂದು
ಸರಿ-ಯೆ-ಶ್ರೀ-ರಾಮ
ಸರಿಯೇ
ಸರಿ-ವಿ-ರು-ದ್ಧ-ವಾಗಿ
ಸರಿ-ಸ-ಮ-ನಾಗಿ
ಸರಿ-ಸ-ಮ-ರಿಲ್ಲ
ಸರಿ-ಸರಿ
ಸರಿ-ಸಾ-ಟಿ-ಯಿಲ್ಲ
ಸರಿ-ಸಾ-ಟಿ-ಯಿ-ಲ್ಲದ
ಸರಿಸಿ
ಸರಿಸು
ಸರಿ-ಸು-ವು-ದಕ್ಕೆ
ಸರಿ-ಹೊಂ-ದು-ತ್ತ-ದೆಯೋ
ಸರಿ-ಹೋ-ಗಿ-ಬಿ-ಡು-ತ್ತಿತ್ತು
ಸರಿ-ಹೋ-ದಾನು
ಸರಿ-ಹೋ-ಯಿತು
ಸರೋ-ವರ
ಸರೋ-ವ-ರದ
ಸರೋ-ವ-ರ-ದಲ್ಲಿ
ಸರ್
ಸರ್ಕಸ್
ಸರ್ಕಾರ
ಸರ್ಕಾ-ರನು
ಸರ್ಕಾರೀ
ಸರ್ಕಾರ್
ಸರ್ಪ
ಸರ್ಪವೋ
ಸರ್ಪಾ-ಸ್ತ್ರ-ದಂತೆ
ಸರ್ರನೆ
ಸರ್ವ
ಸರ್ವಂ
ಸರ್ವ-ಜೀ-ವ-ರಲ್ಲೂ
ಸರ್ವಜ್ಞ
ಸರ್ವ-ಜ್ಞ-ತ್ವ-ದೆ-ಡೆಗೆ
ಸರ್ವ-ಜ್ಞ-ನಾದ
ಸರ್ವ-ಜ್ಞಾ-ನಿಯು
ಸರ್ವಜ್ಞೆ
ಸರ್ವ-ತಂತ್ರ
ಸರ್ವ-ತಿ-ರ-ಸ್ಕಾ-ರದ
ಸರ್ವತೋ
ಸರ್ವ-ತೋ-ಮುಖ
ಸರ್ವದಾ
ಸರ್ವ-ದುಃ-ಖಾ-ನಾಂ
ಸರ್ವ-ಧರ್ಮ
ಸರ್ವ-ಧ-ರ್ಮ-ಗಳ
ಸರ್ವ-ಧ-ರ್ಮ-ಗಳನ್ನೂ
ಸರ್ವ-ನಿ-ಯ-ಮಾ-ತೀ-ತವು
ಸರ್ವ-ಪಾ-ಪವ
ಸರ್ವ-ಭಾ-ವ-ಗಳನ್ನು
ಸರ್ವರ
ಸರ್ವ-ರಾ-ತ್ಮನು
ಸರ್ವ-ರಿ-ಗಿ-ರಲಿ
ಸರ್ವ-ರಿಗೂ
ಸರ್ವ-ವನ್ನೂ
ಸರ್ವ-ವೆ-ನ್ನುತ
ಸರ್ವ-ವ್ಯಾಪಿ
ಸರ್ವ-ಶ-ಕ್ತಿ-ಯನ್ನೂ
ಸರ್ವ-ಸಂಗ
ಸರ್ವ-ಸಂ-ಗ-ಪ-ರಿ-ತ್ಯಾಗ
ಸರ್ವ-ಸಂ-ಗ-ಪ-ರಿ-ತ್ಯಾ-ಗಿ-ಗ-ಳೆಂ-ಬು-ದೇನೋ
ಸರ್ವ-ಸಂ-ಗ-ಪ-ರಿ-ತ್ಯಾ-ಗಿ-ಯಾದ
ಸರ್ವ-ಸ-ಮ-ರ್ಪಣೆ
ಸರ್ವ-ಸಾ-ಕ್ಷಿ-ಯಾದ
ಸರ್ವ-ಸಾ-ಧ-ನೆ-ಗಳ
ಸರ್ವ-ಸಿ-ದ್ಧಿ-ಗಳ
ಸರ್ವಸ್ವ
ಸರ್ವ-ಸ್ವ-ವನ್ನೂ
ಸರ್ವ-ಸ್ವವೂ
ಸರ್ವ-ಸ್ವ-ವೆಂದರೆ
ಸರ್ವಾಂಗ
ಸರ್ವಾಂ-ಗ-ಸುಂ-ದರ
ಸರ್ವಾಂ-ಗ-ಸುಂ-ದ-ರ-ವಾಗಿ
ಸರ್ವಾಂ-ಗ-ಸುಂ-ದ-ರ-ವಾದ
ಸರ್ವಾ-ನು-ಮತ
ಸರ್ವೇ
ಸರ್ವೇ-ಸಾ-ಮಾನ್ಯ
ಸರ್ವೇ-ಸಾ-ಮಾ-ನ್ಯ-ನಾದ
ಸರ್ವೇ-ಸಾ-ಮಾ-ನ್ಯ-ವಾ-ಯಿತು
ಸಲ
ಸಲಕ್ಕೆ
ಸಲ-ವಂತೂ
ಸಲ-ವಾ-ದರೂ
ಸಲವೂ
ಸಲ-ವೆಂತೂ
ಸಲಹಿ
ಸಲ-ಹುವ
ಸಲಹೆ
ಸಲ-ಹೆ-ಗಳನ್ನು
ಸಲ-ಹೆ-ಗಳನ್ನೆಲ್ಲ
ಸಲ-ಹೆ-ಗಳು
ಸಲ-ಹೆಯ
ಸಲ-ಹೆ-ಯಂತೆ
ಸಲ-ಹೆ-ಯನ್ನು
ಸಲ-ಹೆ-ಯನ್ನೂ
ಸಲಿ-ಗೆಯ
ಸಲಿ-ಗೆ-ಯಿಂದ
ಸಲಿ-ಗೆ-ಯಿಂ-ದಿರ
ಸಲಿ-ಲದ
ಸಲೀ-ಸಾಗಿ
ಸಲು
ಸಲು-ವಾಗಿ
ಸಲ್ಲ-ಬೇ-ಕಾ-ದರೆ
ಸಲ್ಲಾ-ಪ-ದಲ್ಲಿ
ಸಲ್ಲಿ-ಸ-ಬ-ಹುದು
ಸಲ್ಲಿ-ಸ-ಬೇ-ಕಾದ
ಸಲ್ಲಿ-ಸ-ಬೇಕು
ಸಲ್ಲಿ-ಸ-ಲಾ-ರೆ-ನಪ್ಪ
ಸಲ್ಲಿ-ಸಲು
ಸಲ್ಲಿಸಿ
ಸಲ್ಲಿ-ಸಿ-ದರೆ
ಸಲ್ಲಿ-ಸಿ-ದಳು
ಸಲ್ಲಿ-ಸಿ-ದಾಗ
ಸಲ್ಲಿ-ಸಿ-ದ್ದೇ-ನಪ್ಪ
ಸಲ್ಲಿ-ಸು-ತ್ತಿದ್ದ
ಸಲ್ಲಿ-ಸು-ತ್ತಿ-ದ್ದರು
ಸಲ್ಲಿ-ಸು-ತ್ತಿ-ದ್ದರೆ
ಸಲ್ಲಿ-ಸು-ತ್ತಿ-ರು-ತ್ತಾರೆ
ಸಲ್ಲಿ-ಸು-ತ್ತಿ-ರು-ವಂತೆ
ಸಲ್ಲಿ-ಸು-ತ್ತಿ-ರು-ವಾಗ
ಸಲ್ಲಿ-ಸು-ತ್ತೇನೆ
ಸಲ್ಲಿ-ಸುವ
ಸಲ್ಲಿ-ಸು-ವು-ದನ್ನೂ
ಸವರಿ
ಸವ-ಲತ್ತೂ
ಸವಾರ
ಸವಾರಿ
ಸವಾ-ಲಿ-ನಂತೆ
ಸವಾ-ಲೊ-ಡ್ಡಿ-ದರು
ಸವಿದ
ಸವಿ-ದಿದ್ದ
ಸವಿದು
ಸವಿ-ಯನ್ನು
ಸವಿ-ಯಲು
ಸವಿ-ಯುತ್ತ
ಸವಿ-ಸ್ತಾ-ರ-ವಾಗಿ
ಸವೆ-ಸು-ವು-ದ-ಕ್ಕಾಗಿ
ಸಶ-ಕ್ತ-ವಾ-ಗ-ಲೇ-ಬೇಕು
ಸಶ-ರೀ-ರಿ-ಯಾಗಿ
ಸಸ್ನೇಹ
ಸಸ್ಯ-ವ-ರ್ಗಕ್ಕೆ
ಸಸ್ಯ-ವ-ಲ-ಯಕ್ಕೂ
ಸಸ್ಯಾ-ಹಾ-ರವೇ
ಸಸ್ಯಾ-ಹಾ-ರಿ-ಯಾ-ಗಿ-ಬಿ-ಟ್ಟಿ-ದ್ದಾನೆ
ಸಸ್ಯಾ-ಹಾ-ರಿಯೇ
ಸಹ
ಸಹ-ಎಂ-ದರೆ
ಸಹ-ಕ-ರಿ-ಸಿ-ದರು
ಸಹ-ಕಾರ
ಸಹ-ಕಾ-ರ-ದಿಂದ
ಸಹ-ಕಾ-ರ-ವಾ-ಗಲಿ
ಸಹ-ಚ-ರ-ನೊಂ-ದಿಗೆ
ಸಹಜ
ಸಹ-ಜ-ಸ-ಮಂ-ಜ-ಸ-ವಾ-ಗಿದೆ
ಸಹ-ಜ-ಅ-ವ-ರಲ್ಲಿ
ಸಹ-ಜ-ವಾಗಿ
ಸಹ-ಜ-ವಾ-ಗಿದ್ದ
ಸಹ-ಜ-ವಾ-ಗಿ-ಬಿ-ಡು-ತ್ತದೆ
ಸಹ-ಜ-ವಾ-ಗಿಯೇ
ಸಹ-ಜ-ವಾ-ಗುತ್ತ
ಸಹ-ಜ-ವಾದ
ಸಹ-ಜವೇ
ಸಹ-ಜ-ಸ್ಥಿ-ತಿಗೆ
ಸಹನಂ
ಸಹ-ನೆಗೆ
ಸಹ-ನೆ-ತಪ್ಪಿ
ಸಹ-ನೆಯ
ಸಹ-ನೆ-ಯಿಂದ
ಸಹ-ನೆ-ಯಿಂ-ದಿ-ರು-ತ್ತಿದ್ದ
ಸಹ-ಪಾ-ಠಿ-ಗಳ
ಸಹ-ಪಾ-ಠಿ-ಗ-ಳೆಲ್ಲ
ಸಹ-ಪಾ-ಠಿ-ಗ-ಳೊಂ-ದಿಗೆ
ಸಹ-ಪಾ-ಠಿಯ
ಸಹ-ಪಾ-ಠಿ-ಯಾದ
ಸಹ-ಪಾ-ಠಿಯೇ
ಸಹ-ಭಾ-ಗಿ-ಗ-ಳಾ-ಗು-ವಂತೆ
ಸಹ-ವಾ-ಸಕ್ಕೂ
ಸಹ-ವಾ-ಸಕ್ಕೆ
ಸಹ-ವಾ-ಸದ
ಸಹ-ವಾ-ಸ-ದಲ್ಲಿ
ಸಹ-ವಾ-ಸವೂ
ಸಹ-ವಿ-ದ್ಯಾರ್ಥಿ
ಸಹ-ಶಿ-ಷ್ಯ-ನಾದ
ಸಹ-ಶಿ-ಷ್ಯ-ರನ್ನು
ಸಹ-ಶಿ-ಷ್ಯ-ರ-ನ್ನೆಲ್ಲ
ಸಹ-ಶಿ-ಷ್ಯ-ರಿ-ಗೆಲ್ಲ
ಸಹ-ಶಿ-ಷ್ಯ-ರೆಲ್ಲ
ಸಹ-ಶಿ-ಷ್ಯ-ರೊಂ-ದಿಗೆ
ಸಹ-ಸ್ರ-ದಳ
ಸಹ-ಸ್ರ-ದ-ಳದ
ಸಹ-ಸ್ರ-ದ್ವೀ-ಪೋ-ದ್ಯಾನ
ಸಹ-ಸ್ರ-ನಾಮ
ಸಹಾ-ನು-ಭೂತಿ
ಸಹಾ-ನು-ಭೂ-ತಿ-ಯಲ್ಲ
ಸಹಾ-ನು-ಭೂ-ತಿ-ಯಿಂದ
ಸಹಾ-ನು-ಭೂ-ತಿ-ಯಿದೆ
ಸಹಾ-ನು-ಭೂ-ತಿ-ಯಿ-ರ-ಲಿಲ್ಲ
ಸಹಾಯ
ಸಹಾ-ಯ-ಸ-ಹ-ಕಾ-ರ-ಗ-ಳಿ-ಗಾಗಿ
ಸಹಾ-ಯ-ಕ-ನಾಗಿ
ಸಹಾ-ಯ-ಕ-ವಾ-ಯಿತು
ಸಹಾ-ಯ-ಕಾರಿ
ಸಹಾ-ಯಕ್ಕೆ
ಸಹಾ-ಯ-ಕ್ಕೊ-ದ-ಗಿದ
ಸಹಾ-ಯ-ದಿಂದ
ಸಹಾ-ಯ-ದಿಂ-ದಲೇ
ಸಹಾ-ಯ-ಮಾ-ಡು-ತ್ತದೆ
ಸಹಾ-ಯ-ವನ್ನು
ಸಹಾ-ಯ-ವನ್ನೂ
ಸಹಾ-ಯ-ವ-ನ್ನೆಲ್ಲ
ಸಹಾ-ಯ-ವ-ನ್ನೇಕೆ
ಸಹಾ-ಯ-ವಾ-ಗು-ತ್ತಿ-ದ್ದುವು
ಸಹಾ-ಯ-ವಾ-ಗು-ವಂ-ತಹ
ಸಹಾ-ಯ-ವಾ-ದೀತು
ಸಹಾ-ಯ-ವಾ-ಯಿತು
ಸಹಾ-ಯವೂ
ಸಹಾ-ಯ-ಹ-ಸ್ತ-ದಿಂದ
ಸಹಾ-ಯ-ಹ-ಸ್ತ-ವನ್ನು
ಸಹಿತ
ಸಹಿ-ತಿ-ನಿಸು
ಸಹಿ-ಸ-ಬೇ-ಕಾ-ಯಿತು
ಸಹಿ-ಸ-ಲಾ-ರದ
ಸಹಿ-ಸ-ಲಾರೆ
ಸಹಿ-ಸಿ-ಕೊಂ-ಡರೆ
ಸಹಿ-ಸಿ-ಕೊಂ-ಡಿ-ರ-ಬ-ಹುದು
ಸಹಿ-ಸಿ-ಕೊಂಡು
ಸಹಿ-ಸಿ-ಕೊ-ಳ್ಳ-ಬೇ-ಕಾ-ಗು-ತ್ತದೆ
ಸಹಿ-ಸಿ-ಕೊ-ಳ್ಳ-ಲಾರೆ
ಸಹಿ-ಸಿ-ಕೊ-ಳ್ಳಲು
ಸಹಿ-ಸಿ-ಕೊ-ಳ್ಳು-ತ್ತಾರೆ
ಸಹಿ-ಸಿ-ಕೊ-ಳ್ಳು-ತ್ತಿ-ದ್ದರು
ಸಹಿ-ಸಿ-ಕೊ-ಳ್ಳು-ತ್ತಿ-ರು-ವುದು
ಸಹಿ-ಸಿ-ಕೊ-ಳ್ಳುವ
ಸಹಿ-ಸಿ-ಕೊ-ಳ್ಳು-ವುದು
ಸಹಿ-ಸಿ-ಯಾನೆ
ಸಹಿ-ಸು-ತಿ-ರುವೆ
ಸಹಿ-ಸು-ತ್ತಿ-ರ-ಲಿಲ್ಲ
ಸಹಿ-ಸು-ವು-ದ-ಕ್ಕಾ-ಗು-ವು-ದಿಲ್ಲ
ಸಹಿ-ಸು-ವು-ದಾ-ಗಲಿ
ಸಹೃ-ದಯಿ
ಸಹ್ಯ-ವಾ-ಗು-ತ್ತಿಲ್ಲ
ಸಾ
ಸಾಂಕೇ-ತಿ-ಕ-ವಾಗಿ
ಸಾಂಕ್ರಾ-ಮಿ-ಕ-ವಾ-ದದ್ದು
ಸಾಂಗ-ವಾಗಿ
ಸಾಂತ-ವಾದ
ಸಾಂತ್ವನ
ಸಾಂತ್ವ-ನ-ಗೊ-ಳಿ-ಸ-ಲೆ-ತ್ನಿ-ಸಿದ
ಸಾಂತ್ವ-ನ-ಗೊ-ಳಿ-ಸುವ
ಸಾಂತ್ವ-ನ-ವನ್ನು
ಸಾಂದ-ರ್ಭಿ-ಕ-ವಾಗಿ
ಸಾಂಪ್ರ-ದಾ-ಯಿಕ
ಸಾಂಸ್ಕೃ-ತಿಕ
ಸಾಕಪ್ಪ
ಸಾಕ-ವ-ನಿಗೆ
ಸಾಕಷ್ಟು
ಸಾಕಾ-ಗದೆ
ಸಾಕಾ-ಗಿತ್ತು
ಸಾಕಾ-ಗಿ-ತ್ತು-ಇಡೀ
ಸಾಕಾ-ಗು-ತ್ತಿತ್ತು
ಸಾಕಾ-ಗು-ವಷ್ಟು
ಸಾಕಾ-ಗು-ವುದೆ
ಸಾಕಾ-ಯಿತು
ಸಾಕಾರ
ಸಾಕಾ-ರ-ನಿ-ರಾ-ಕಾರ
ಸಾಕಾ-ರ-ತ್ವ-ವನ್ನು
ಸಾಕಾ-ರನೂ
ಸಾಕಾ-ರನೆ
ಸಾಕಾ-ರ-ಮೂರ್ತಿ
ಸಾಕಾ-ರ-ಮೂ-ರ್ತಿ-ಯಾದ
ಸಾಕಾ-ರ-ಸ್ವ-ರೂ-ಪ-ವನ್ನು
ಸಾಕಿ
ಸಾಕಿದ್ದ
ಸಾಕಿ-ಸ-ಲಹಿ
ಸಾಕು
ಸಾಕು-ತ-ನ್ನ-ಷ್ಟಕ್ಕೇ
ಸಾಕು-ತ್ತಿ-ದ್ದೀಯ
ಸಾಕ್ಷಾ
ಸಾಕ್ಷಾತ್
ಸಾಕ್ಷಾ-ತ್ಕ-ರಿಸಿ
ಸಾಕ್ಷಾ-ತ್ಕ-ರಿ-ಸಿ-ಕೊಂ-ಡಿ-ದ್ದ-ವರು
ಸಾಕ್ಷಾ-ತ್ಕ-ರಿ-ಸಿ-ಕೊ-ಳ್ಳದೆ
ಸಾಕ್ಷಾ-ತ್ಕ-ರಿ-ಸಿ-ಕೊ-ಳ್ಳ-ಲಾ-ಗು-ವುದೆ
ಸಾಕ್ಷಾ-ತ್ಕ-ರಿ-ಸಿ-ಕೊ-ಳ್ಳ-ಲೂ-ಬ-ಹುದು
ಸಾಕ್ಷಾ-ತ್ಕ-ರಿ-ಸಿ-ಕೊ-ಳ್ಳುವ
ಸಾಕ್ಷಾ-ತ್ಕ-ರಿ-ಸಿ-ಕೊ-ಳ್ಳು-ವು-ದರ
ಸಾಕ್ಷಾ-ತ್ಕಾರ
ಸಾಕ್ಷಾ-ತ್ಕಾ-ರ-ಕ್ಕಾಗಿ
ಸಾಕ್ಷಾ-ತ್ಕಾ-ರ-ಕ್ಕಿ-ರು-ವುದ
ಸಾಕ್ಷಾ-ತ್ಕಾ-ರಕ್ಕೆ
ಸಾಕ್ಷಾ-ತ್ಕಾ-ರ-ಗಳ
ಸಾಕ್ಷಾ-ತ್ಕಾ-ರ-ಗಳನ್ನು
ಸಾಕ್ಷಾ-ತ್ಕಾ-ರ-ಗಳಿಂದ
ಸಾಕ್ಷಾ-ತ್ಕಾ-ರ-ಗ-ಳೆಲ್ಲ
ಸಾಕ್ಷಾ-ತ್ಕಾ-ರದ
ಸಾಕ್ಷಾ-ತ್ಕಾ-ರ-ದಿಂ-ದಾಗಿ
ಸಾಕ್ಷಾ-ತ್ಕಾ-ರ-ದೆ-ಡೆಗೆ
ಸಾಕ್ಷಾ-ತ್ಕಾ-ರ-ವನ್ನು
ಸಾಕ್ಷಾ-ತ್ಕಾ-ರ-ವನ್ನೇ
ಸಾಕ್ಷಾ-ತ್ಕಾ-ರ-ವಾಗ
ಸಾಕ್ಷಾ-ತ್ಕಾ-ರ-ವಾ-ಗ-ದಿದ್ದ
ಸಾಕ್ಷಾ-ತ್ಕಾ-ರ-ವಾ-ಗ-ದಿ-ರು-ವುದು
ಸಾಕ್ಷಾ-ತ್ಕಾ-ರ-ವಾ-ಗ-ಬೇ-ಕಾ-ದರೆ
ಸಾಕ್ಷಾ-ತ್ಕಾ-ರ-ವಾ-ಗ-ಲಿ-ಲ್ಲ-ವಲ್ಲ
ಸಾಕ್ಷಾ-ತ್ಕಾ-ರ-ವಾ-ಗ-ಲಿ-ಲ್ಲ-ವಲ್ಲಾ
ಸಾಕ್ಷಾ-ತ್ಕಾ-ರ-ವಾಗಿ
ಸಾಕ್ಷಾ-ತ್ಕಾ-ರ-ವಾ-ಗು-ವ-ವ-ರೆಗೂ
ಸಾಕ್ಷಾ-ತ್ಕಾ-ರ-ವಾದ
ಸಾಕ್ಷಾ-ತ್ಕಾ-ರ-ವಾ-ಯಿ-ತೆಂ-ದರೆ
ಸಾಕ್ಷಾ-ತ್ಕಾ-ರ-ವಿನ್ನೂ
ಸಾಕ್ಷಾ-ತ್ಕಾ-ರ-ವೆಂದರೆ
ಸಾಕ್ಷಾ-ತ್ಕಾ-ರವೇ
ಸಾಕ್ಷಾ-ತ್ಕಾ-ರಿ-ಸಿ-ಕೊ-ಳ್ಳ-ಬ-ಹುದು
ಸಾಕ್ಷಾ-ತ್ಕಾ-ರಿ-ಸಿ-ಕೊ-ಳ್ಳು-ವು-ದರ
ಸಾಕ್ಷಾ-ತ್ತಾಗಿ
ಸಾಕ್ಷಿ
ಸಾಕ್ಷಿ-ಅಕ್ಷಿ
ಸಾಕ್ಷಿ-ಗ-ಳಾ-ಗಿ-ರು-ವಂತೆ
ಸಾಕ್ಷಿ-ಯಾ-ತನು
ಸಾಕ್ಷಿ-ರೂ-ಪ-ದಿಂದ
ಸಾಕ್ಷ್ಯ-ವೊಂ-ದನ್ನು
ಸಾಗ-ತೊ-ಡ-ಗಿತು
ಸಾಗ-ತೊ-ಡ-ಗಿತ್ತು
ಸಾಗ-ಬೇ-ಕಾ-ಗಿ-ತ್ತು-ಹ-ದಿ-ನೈದು
ಸಾಗರ
ಸಾಗ-ರದ
ಸಾಗ-ರ-ದಂ-ತಿದೆ
ಸಾಗ-ರ-ದಲ್ಲಿ
ಸಾಗ-ರ-ನಿ-ಹನು
ಸಾಗ-ರ-ವನ್ನು
ಸಾಗ-ರ-ವಿ-ದ್ದಂತೆ
ಸಾಗಲು
ಸಾಗಿ
ಸಾಗಿತ್ತು
ಸಾಗಿ-ದರು
ಸಾಗಿದೆ
ಸಾಗಿದ್ದ
ಸಾಗಿ-ದ್ದಾರೆ
ಸಾಗಿ-ಸಿ-ಬಿ-ಟ್ಟಿ-ದ್ದರು
ಸಾಗಿ-ಸು-ವು-ದ-ಕ್ಕಾಗಿ
ಸಾಗು
ಸಾಗು-ತ್ತಾನೆ
ಸಾಗು-ತ್ತಿದೆ
ಸಾಗು-ತ್ತಿದ್ದ
ಸಾಗು-ತ್ತಿ-ದ್ದರು
ಸಾಗು-ತ್ತಿ-ದ್ದಾರೆ
ಸಾಗು-ವಂತೆ
ಸಾಗೆಲೈ
ಸಾತ್ತ್ವಿಕ
ಸಾತ್ವಿ-ಕ-ವಾ-ದ್ದ-ರಿಂದ
ಸಾದ್ಯ-ವಿ-ಲ್ಲ-ದಷ್ಟು
ಸಾದ್ಯವೆ
ಸಾಧಕ
ಸಾಧ-ಕ-ಬಾ-ಧ-ಕ-ಗ-ಳೇನು
ಸಾಧ-ಕನ
ಸಾಧ-ಕ-ನ-ಅ-ವ-ನ-ತಿಗೆ
ಸಾಧ-ಕ-ನಲ್ಲಿ
ಸಾಧ-ಕ-ನಾದ
ಸಾಧ-ಕ-ನಾ-ದ-ವನೂ
ಸಾಧ-ಕನು
ಸಾಧ-ಕರ
ಸಾಧ-ಕ-ರನ್ನು
ಸಾಧ-ಕ-ರ-ಲ್ಲ-ದ-ವರೂ
ಸಾಧ-ಕ-ರಲ್ಲಿ
ಸಾಧ-ಕ-ರಾ-ದರೆ
ಸಾಧ-ಕ-ರಿಗೂ
ಸಾಧ-ಕ-ರಿಗೆ
ಸಾಧ-ಕರು
ಸಾಧ-ಕರೂ
ಸಾಧ-ಕ-ರೆ-ನ್ನಿ-ಸಿ-ಕೊಂ-ಡ-ವರು
ಸಾಧ-ಕ-ರೊಂ-ದಿಗೆ
ಸಾಧ-ಕ-ವಾ-ಗು-ತ್ತದೆ
ಸಾಧ-ಕ-ಶ್ರೇ-ಷ್ಠ-ನೊಬ್ಬ
ಸಾಧಕಿ
ಸಾಧನ
ಸಾಧ-ನ-ವಾ-ಯಿತು
ಸಾಧ-ನವೇ
ಸಾಧನಾ
ಸಾಧ-ನಾ-ಕಾ-ಲ-ದಲ್ಲಿ
ಸಾಧ-ನಾ-ಕ್ರ-ಮ-ವ-ನ್ನ-ನು-ಸ-ರಿಸಿ
ಸಾಧ-ನಾ-ದಿ-ಗಳ
ಸಾಧ-ನಾ-ನಿ-ರ-ತ-ನಾ-ಗಿದ್ದು
ಸಾಧ-ನಾ-ಪ-ಥ-ದಿಂದ
ಸಾಧ-ನಾ-ವಿ-ಧಾನ
ಸಾಧನೆ
ಸಾಧ-ನೆ-ಅ-ಧ್ಯ-ಯನ
ಸಾಧ-ನೆ-ಜಿ-ಜ್ಞಾ-ಸೆ-ಗಳಲ್ಲಿ
ಸಾಧ-ನೆ-ಭ-ಜ-ನೆ-ಧ್ಯಾ-ನ-ಅ-ಧ್ಯ-ಯ-ನ-ಗ-ಳ-ಲ್ಲದೆ
ಸಾಧ-ನೆ-ಗಳ
ಸಾಧ-ನೆ-ಗಳನ್ನು
ಸಾಧ-ನೆ-ಗಳಲ್ಲಿ
ಸಾಧ-ನೆ-ಗ-ಳಲ್ಲೂ
ಸಾಧ-ನೆ-ಗಳು
ಸಾಧ-ನೆ-ಗಾಗಿ
ಸಾಧ-ನೆ-ಗಿಳಿ-ದಾ-ಗಲೂ
ಸಾಧ-ನೆಗೆ
ಸಾಧ-ನೆಯ
ಸಾಧ-ನೆ-ಯ-ನ್ನಾಗಿ
ಸಾಧ-ನೆ-ಯನ್ನು
ಸಾಧ-ನೆ-ಯನ್ನೂ
ಸಾಧ-ನೆ-ಯನ್ನೇ
ಸಾಧ-ನೆ-ಯಲ್ಲಿ
ಸಾಧ-ನೆ-ಯಲ್ಲೇ
ಸಾಧ-ನೆ-ಯಿಂದ
ಸಾಧ-ನೆ-ಯಿಂ-ದಾಗಿ
ಸಾಧ-ನೆಯೂ
ಸಾಧ-ನೆ-ಯೆಂ-ಬುದು
ಸಾಧ-ನೆ-ಯೆಲ್ಲ
ಸಾಧ-ನೆಯೇ
ಸಾಧ-ನೆ-ಯೊಂ-ದಿಗೆ
ಸಾಧನೋ
ಸಾಧಾ-ನ-ಮಯ
ಸಾಧಾ-ರಣ
ಸಾಧಿ-ಸದೆ
ಸಾಧಿ-ಸ-ಬಲ್ಲ
ಸಾಧಿ-ಸ-ಬಲ್ಲೆ
ಸಾಧಿ-ಸ-ಬ-ಲ್ಲೆ-ನೆಂಬ
ಸಾಧಿ-ಸ-ಬ-ಹುದು
ಸಾಧಿ-ಸ-ಬೇ-ಕಾ-ಗಿತ್ತು
ಸಾಧಿ-ಸ-ಬೇ-ಕಾದ
ಸಾಧಿ-ಸ-ಬೇ-ಕಾ-ದರೆ
ಸಾಧಿ-ಸ-ಬೇ-ಕಾ-ದ್ದೇ-ನಿದೆ
ಸಾಧಿ-ಸ-ಬೇ-ಕೆಂದು
ಸಾಧಿ-ಸ-ಬೇ-ಕೆಂಬ
ಸಾಧಿ-ಸ-ಬೇ-ಕೆಂ-ಬುದು
ಸಾಧಿ-ಸ-ಲಾ-ಗದು
ಸಾಧಿ-ಸ-ಲಾ-ಗು-ವು-ದಿಲ್ಲ
ಸಾಧಿ-ಸ-ಲಿ-ದ್ದಾನೆ
ಸಾಧಿ-ಸಲು
ಸಾಧಿ-ಸ-ಲ್ಪ-ಡ-ಬೇ-ಕಾ-ದರೆ
ಸಾಧಿ-ಸ-ಹೊ-ರಟ
ಸಾಧಿ-ಸ-ಹೊ-ರ-ಟ-ವ-ರಲ್ಲೇ
ಸಾಧಿಸಿ
ಸಾಧಿ-ಸಿ-ಕೊಂಡ
ಸಾಧಿ-ಸಿ-ಕೊ-ಳ್ಳ-ಬೇ-ಕೆಂದು
ಸಾಧಿ-ಸಿ-ಕೊ-ಳ್ಳಲು
ಸಾಧಿ-ಸಿ-ಕೊ-ಳ್ಳಲೇ
ಸಾಧಿ-ಸಿ-ದರು
ಸಾಧಿ-ಸಿ-ದ-ವನ
ಸಾಧಿ-ಸಿ-ದಿರಿ
ಸಾಧಿ-ಸಿದ್ದ
ಸಾಧಿ-ಸಿ-ದ್ದುದು
ಸಾಧಿ-ಸಿ-ದ್ದೇನು
ಸಾಧಿ-ಸಿಯೇ
ಸಾಧಿ-ಸಿ-ಯೇ-ನು-ಆ-ದ್ದ-ರಿಂದ
ಸಾಧಿ-ಸು-ತ್ತಿ-ದ್ದೇವೆ
ಸಾಧಿ-ಸು-ತ್ತೀಯೆ
ಸಾಧಿ-ಸು-ತ್ತೀರಿ
ಸಾಧಿ-ಸು-ವಂತೆ
ಸಾಧಿ-ಸುವು
ಸಾಧಿ-ಸು-ವು-ದ-ಕ್ಕಿದೆ
ಸಾಧಿ-ಸು-ವುದು
ಸಾಧು
ಸಾಧು-ಸಂ-ತರ
ಸಾಧು-ಸಂ-ತರೂ
ಸಾಧು-ಸಂನ್ಯಾಸಿ
ಸಾಧು-ಸಂ-ನ್ಯಾ-ಸಿ-ಗಳ
ಸಾಧು-ಸಂ-ನ್ಯಾ-ಸಿ-ಗಳು
ಸಾಧು-ಎಂ-ದರೆ
ಸಾಧು-ಗಳ
ಸಾಧು-ಗಳನ್ನು
ಸಾಧು-ಗ-ಳಾ-ಗ-ಬ-ಹುದು
ಸಾಧು-ಗ-ಳಾ-ಗಿ-ರ-ಬ-ಹುದು
ಸಾಧು-ಗ-ಳಿಗೆ
ಸಾಧು-ಗ-ಳಿನ್ನೂ
ಸಾಧು-ಗಳು
ಸಾಧು-ಗ-ಳು-ಭಿ-ಕ್ಷು-ಕರು
ಸಾಧು-ಗಳೂ
ಸಾಧು-ಗ-ಳೆಂದು
ಸಾಧುತ್ವ
ಸಾಧು-ತ್ವ-ವನ್ನು
ಸಾಧು-ವನ್ನು
ಸಾಧು-ವಾ-ಗಿದ್ದೆ
ಸಾಧುವಿ
ಸಾಧು-ವಿಗೆ
ಸಾಧು-ವಿ-ನಂತೆ
ಸಾಧುವೂ
ಸಾಧುವೆ
ಸಾಧು-ವೊಬ್ಬ
ಸಾಧು-ವೊ-ಬ್ಬನ
ಸಾಧು-ವೊ-ಬ್ಬನು
ಸಾಧು-ವೊ-ಬ್ಬರು
ಸಾಧು-ಸಂ-ತರ
ಸಾಧು-ಸಂ-ನ್ಯಾ-ಸಿ-ಗಳನ್ನು
ಸಾಧು-ಸಂ-ನ್ಯಾ-ಸಿ-ಗಳು
ಸಾಧ್ಯ
ಸಾಧ್ಯತೆ
ಸಾಧ್ಯ-ತೆ-ಗಳನ್ನು
ಸಾಧ್ಯ-ತೆಯೇ
ಸಾಧ್ಯ-ವಾ-ಗದ
ಸಾಧ್ಯ-ವಾ-ಗ-ದಿ-ದ್ದಲ್ಲಿ
ಸಾಧ್ಯ-ವಾ-ಗ-ದಿ-ರ-ಬ-ಹುದು
ಸಾಧ್ಯ-ವಾ-ಗದೆ
ಸಾಧ್ಯ-ವಾ-ಗದೋ
ಸಾಧ್ಯ-ವಾ-ಗ-ಬೇ-ಕಾ-ದರೆ
ಸಾಧ್ಯ-ವಾ-ಗ-ಲಿಲ್ಲ
ಸಾಧ್ಯ-ವಾ-ಗ-ಲಿ-ಲ್ಲ-ವೆಂದರೆ
ಸಾಧ್ಯ-ವಾ-ಗಲೇ
ಸಾಧ್ಯ-ವಾ-ಗಿದೆ
ಸಾಧ್ಯ-ವಾ-ಗಿಲ್ಲ
ಸಾಧ್ಯ-ವಾ-ಗಿ-ಲ್ಲ-ವಲ್ಲ
ಸಾಧ್ಯ-ವಾಗು
ಸಾಧ್ಯ-ವಾ-ಗು-ತ್ತವೆ
ಸಾಧ್ಯ-ವಾ-ಗು-ತ್ತಿತ್ತು
ಸಾಧ್ಯ-ವಾ-ಗು-ತ್ತಿದೆ
ಸಾಧ್ಯ-ವಾ-ಗು-ತ್ತಿ-ರ-ಲಿಲ್ಲ
ಸಾಧ್ಯ-ವಾ-ಗು-ತ್ತಿಲ್ಲ
ಸಾಧ್ಯ-ವಾ-ಗು-ತ್ತಿ-ಲ್ಲ-ವಲ್ಲ
ಸಾಧ್ಯ-ವಾ-ಗು-ವಂ-ತಿ-ರ-ಲಿಲ್ಲ
ಸಾಧ್ಯ-ವಾ-ಗು-ವಂ-ಥ-ದಲ್ಲ
ಸಾಧ್ಯ-ವಾ-ಗು-ವು-ದಿಲ್ಲ
ಸಾಧ್ಯ-ವಾದ
ಸಾಧ್ಯ-ವಾ-ದದ್ದು
ಸಾಧ್ಯ-ವಾ-ದರೆ
ಸಾಧ್ಯ-ವಾ-ದಷ್ಟು
ಸಾಧ್ಯ-ವಾ-ದೀತು
ಸಾಧ್ಯ-ವಾದ್ದ
ಸಾಧ್ಯ-ವಾ-ಯಿತು
ಸಾಧ್ಯ-ವಾ-ಯಿತೇ
ಸಾಧ್ಯ-ವಾ-ಯಿತೋ
ಸಾಧ್ಯ-ವಿದೆ
ಸಾಧ್ಯ-ವಿ-ರದ
ಸಾಧ್ಯ-ವಿ-ರ-ಲಿಲ್ಲ
ಸಾಧ್ಯ-ವಿ-ರು-ವಾಗ
ಸಾಧ್ಯ-ವಿಲ್ಲ
ಸಾಧ್ಯ-ವಿ-ಲ್ಲ-ಎಂದು
ಸಾಧ್ಯ-ವಿ-ಲ್ಲ-ದಂ-ತಹ
ಸಾಧ್ಯ-ವಿ-ಲ್ಲ-ದಾ-ದಾಗ
ಸಾಧ್ಯ-ವಿ-ಲ್ಲ-ದ್ದನ್ನು
ಸಾಧ್ಯ-ವಿ-ಲ್ಲ-ವಲ್ಲ
ಸಾಧ್ಯ-ವಿ-ಲ್ಲ-ವಾ-ಯಿತು
ಸಾಧ್ಯ-ವಿ-ಲ್ಲವೆ
ಸಾಧ್ಯ-ವಿ-ಲ್ಲ-ವೆಂಬ
ಸಾಧ್ಯವೂ
ಸಾಧ್ಯವೆ
ಸಾಧ್ಯ-ವೆಂದರೆ
ಸಾಧ್ಯ-ವೆಂದು
ಸಾಧ್ಯವೇ
ಸಾಧ್ಯ-ವೇನು
ಸಾಧ್ಯವೋ
ಸಾಧ್ಯಾ-ವಾ-ಗು-ತ್ತ-ದೆಯೋ
ಸಾನವು
ಸಾನ್ನಿ-ಧ್ಯದ
ಸಾನ್ನಿ-ಧ್ಯ-ದಲ್ಲಿ
ಸಾನ್ನಿ-ಧ್ಯ-ದ-ಲ್ಲಿ-ರುತ್ತ
ಸಾನ್ನಿ-ಧ್ಯ-ದಿಂ-ದಲೇ
ಸಾನ್ನಿ-ಧ್ಯ-ದಿಂ-ದಾ-ಗಿಯೇ
ಸಾನ್ನಿ-ಧ್ಯ-ವನ್ನು
ಸಾನ್ನಿ-ಧ್ಯವೇ
ಸಾಪೇ-ಕ್ಷ-ವಾ-ದದ್ದೇ
ಸಾಬೀತು
ಸಾಬೀ-ತು-ಪ-ಡಿ-ಸಿ-ದಳು
ಸಾಬೀ-ತು-ಪ-ಡಿ-ಸು-ತ್ತಿ-ದ್ದರು
ಸಾಬೀ-ತು-ಪ-ಡಿ-ಸು-ವಲ್ಲಿ
ಸಾಮಂತ
ಸಾಮಧಿ
ಸಾಮ-ರಸ್ಯ
ಸಾಮ-ರ-ಸ್ಯದ
ಸಾಮರ್ಥ್ಯ
ಸಾಮ-ರ್ಥ್ಯ-ಯೋ-ಗ್ಯತೆ
ಸಾಮ-ರ್ಥ್ಯ-ಗಳನ್ನು
ಸಾಮ-ರ್ಥ್ಯ-ಗಳಲ್ಲಿ
ಸಾಮ-ರ್ಥ್ಯ-ಗಳು
ಸಾಮ-ರ್ಥ್ಯದ
ಸಾಮ-ರ್ಥ್ಯ-ದಿಂದ
ಸಾಮ-ರ್ಥ್ಯ-ವ-ಡ-ಗಿ-ರು-ವುದನ್ನು
ಸಾಮ-ರ್ಥ್ಯ-ವನ್ನು
ಸಾಮ-ರ್ಥ್ಯ-ವನ್ನೂ
ಸಾಮ-ರ್ಥ್ಯ-ವೆಂ-ಥದು
ಸಾಮ-ರ್ಥ್ಯ-ಶಾಲೀ
ಸಾಮಾ-ಜಿಕ
ಸಾಮಾ-ಜಿ-ಕರ
ಸಾಮಾ-ಜಿ-ಕ-ವ್ಯ-ವ-ಸ್ಥೆಯ
ಸಾಮಾ-ಧಾ-ನ-ಪ-ಡಿ-ಸುತ್ತ
ಸಾಮಾನು
ಸಾಮಾ-ನು-ಗಳನ್ನು
ಸಾಮಾ-ನು-ಗಳನ್ನೆಲ್ಲ
ಸಾಮಾ-ನು-ಗಳಲ್ಲಿ
ಸಾಮಾ-ನು-ಗಳೂ
ಸಾಮಾನ್ಯ
ಸಾಮಾ-ನ್ಯತಃ
ಸಾಮಾ-ನ್ಯ-ನಂತೆ
ಸಾಮಾ-ನ್ಯ-ನಲ್ಲ
ಸಾಮಾ-ನ್ಯನೂ
ಸಾಮಾ-ನ್ಯ-ರಂತೆ
ಸಾಮಾ-ನ್ಯ-ರಾದ
ಸಾಮಾ-ನ್ಯ-ರಿಗೂ
ಸಾಮಾ-ನ್ಯ-ರಿಗೆ
ಸಾಮಾ-ನ್ಯರು
ಸಾಮಾ-ನ್ಯರೂ
ಸಾಮಾ-ನ್ಯ-ರೆಂದು
ಸಾಮಾ-ನ್ಯ-ವಾಗಿ
ಸಾಮಾ-ನ್ಯ-ವಾದ
ಸಾಮಾ-ನ್ಯ-ವಾ-ದು-ವು-ಗ-ಳಲ್ಲ
ಸಾಮೂ-ಹಿಕ
ಸಾಮ್ರಾಜ್ಯ
ಸಾಮ್ರಾ-ಜ್ಯಕ್ಕೆ
ಸಾಮ್ರಾ-ಜ್ಯದ
ಸಾಮ್ರಾ-ಜ್ಯ-ವನ್ನು
ಸಾಯ
ಸಾಯಂ-ಕಾಲ
ಸಾಯಲಿ
ಸಾಯ-ಲಿಲ್ಲ
ಸಾಯು-ತ್ತಾ-ರಲ್ಲ
ಸಾಯು-ತ್ತಾ-ರೆಯೇ
ಸಾರ
ಸಾರಂ-ಗ-ವ-ನ-ವನ್ನು
ಸಾರ-ಲಾ-ರಂ-ಭಿ-ಸಿದ
ಸಾರ-ಲಿ-ರು-ವಂ-ತಹ
ಸಾರ-ಲಿಲ್ಲ
ಸಾರಲು
ಸಾರ-ವನ್ನು
ಸಾರ-ವನ್ನೇ
ಸಾರ-ವಾ-ದ-ದ್ದಲ್ಲ
ಸಾರ-ಸ್ವತ
ಸಾರಾಂಶ
ಸಾರಾಂ-ಶ-ವನ್ನು
ಸಾರಾ-ನಾ-ಥಕ್ಕೂ
ಸಾರಿ
ಸಾರಿದ
ಸಾರಿ-ದಳು
ಸಾರು
ಸಾರು-ತ್ತಾ-ರೆ-ಮ-ಹಾ-ಕಾ-ರ್ಯ-ಗಳು
ಸಾರು-ತ್ತಿ-ದ್ದರು
ಸಾರು-ತ್ತಿ-ದ್ದೀ-ಯಲ್ಲ
ಸಾರುವ
ಸಾರು-ವುದು
ಸಾರು-ವುವು
ಸಾರೈ
ಸಾರೋ-ಟನ್ನು
ಸಾರೋ-ಟಿಗೆ
ಸಾರೋ-ಟಿನ
ಸಾರೋ-ಟಿ-ನಲ್ಲಿ
ಸಾರೋ-ಟಿ-ನಲ್ಲೇ
ಸಾರೋಟು
ಸಾರೋ-ಟು-ಗಳಲ್ಲಿ
ಸಾರೋ-ಟು-ಗ-ಳಿಲ್ಲಿ
ಸಾರೋ-ಟು-ವಾಲ
ಸಾರೋ-ಟು-ವಾ-ಲನ
ಸಾರೋ-ಟು-ವಾ-ಲನೇ
ಸಾರೋ-ಟು-ವಾಲಾ
ಸಾರೋ-ಟು-ಸ-ವಾರ
ಸಾರ್
ಸಾರ್ಥಕ
ಸಾರ್ಥ-ಕ-ತೆ-ಯನ್ನೂ
ಸಾರ್ಥ-ಕ-ವಾ-ಯಿತು
ಸಾರ್ವ-ಕಾ-ಲಿ-ಕ-ವಾ-ದ-ವು-ಗಳು
ಸಾರ್ವ-ಜ-ನಿಕ
ಸಾರ್ವ-ಭೌಮ
ಸಾಲ
ಸಾಲ-ಗಳ
ಸಾಲ-ಗಾ-ರ-ರೆಲ್ಲ
ಸಾಲದು
ಸಾಲ-ದು-ಅ-ವನು
ಸಾಲದೆ
ಸಾಲ-ದ್ದಕ್ಕೆ
ಸಾಲನ್ನು
ಸಾಲ-ಲಿಲ್ಲ
ಸಾಲಾಗಿ
ಸಾಲಿ-ನಲ್ಲಿ
ಸಾಲಿ-ರು-ವುದು
ಸಾಲು-ಗಳನ್ನು
ಸಾಲು-ಗಳಲ್ಲಿ
ಸಾಲು-ಗಳಿಂದ
ಸಾಲು-ಗಳು
ಸಾಲ್ವುದೆ
ಸಾವ-ಕಾ-ಶವೇ
ಸಾವ-ರಿ-ಸಿ-ಕೊಂಡು
ಸಾವಿನ
ಸಾವಿ-ನಿಂ-ದಾಗಿ
ಸಾವಿರ
ಸಾವಿ-ರ-ಗ-ಟ್ಟಲೆ
ಸಾವಿ-ರ-ದ-ಷ್ಟಿತ್ತು
ಸಾವಿ-ರ-ಪಟ್ಟು
ಸಾವು
ಸಾಷ್ಟಾಂಗ
ಸಾಸಿವೆ
ಸಾಹ
ಸಾಹರ
ಸಾಹ-ರನ್ನೂ
ಸಾಹರು
ಸಾಹಸ
ಸಾಹ-ಸ-ಕೃ-ತ್ಯ-ವನ್ನು
ಸಾಹ-ಸ-ಗಳಲ್ಲಿ
ಸಾಹ-ಸ-ಗಳು
ಸಾಹ-ಸ-ಪ್ರ-ವೃತ್ತಿ
ಸಾಹ-ಸ-ಪ್ರ-ವೃ-ತ್ತಿ-ಯೆ-ನ್ನು-ವುದು
ಸಾಹ-ಸ-ವನ್ನು
ಸಾಹ-ಸ-ವಾಗಿ
ಸಾಹ-ಸವೂ
ಸಾಹಿ-ತಿ-ಗಳು
ಸಾಹಿತ್ಯ
ಸಾಹಿ-ತ್ಯ-ಕೃ-ತಿ-ಗಳ
ಸಾಹಿ-ತ್ಯದ
ಸಾಹಿ-ತ್ಯ-ದಲ್ಲಿ
ಸಾಹೇ-ಬರ
ಸಾಹೇ-ಬರು
ಸಿ
ಸಿಂಗ-ರಿ-ಸ-ದನು
ಸಿಂಗ್
ಸಿಂಡ-ರಿ-ಸಿ-ಕೊಂಡು
ಸಿಂಪ-ಡಿಸಿ
ಸಿಂಹ
ಸಿಂಹ-ಗ-ರ್ಜನೆ
ಸಿಂಹದ
ಸಿಂಹ-ದಂತೆ
ಸಿಂಹ-ಬುದ್ಧಿ
ಸಿಂಹ-ಮಹೋ
ಸಿಂಹ-ರಾ-ಜ-ನೊಂ-ದಿಗೆ
ಸಿಂಹ-ಸ-ದೃಶ
ಸಿಂಹಾ-ವ-ಲೋ-ಕನ
ಸಿಕ್ಕ
ಸಿಕ್ಕಂ-ತಾ-ಯಿತು
ಸಿಕ್ಕ-ಲಿಲ್ಲ
ಸಿಕ್ಕಾಗ
ಸಿಕ್ಕಾ-ಗ-ಲೆಲ್ಲ
ಸಿಕ್ಕಾನು
ಸಿಕ್ಕಿ
ಸಿಕ್ಕಿ-ಕೊಂ-ಡ-ರೇನು
ಸಿಕ್ಕಿ-ಕೊಂ-ಡ-ವ-ರಂತೆ
ಸಿಕ್ಕಿ-ಕೊಂಡು
ಸಿಕ್ಕಿ-ಕೊಂ-ಡು-ಬಿ-ಟ್ಟವು
ಸಿಕ್ಕಿ-ಕೊ-ಳ್ಳು-ತ್ತಿದ್ದ
ಸಿಕ್ಕಿತು
ಸಿಕ್ಕಿತೋ
ಸಿಕ್ಕಿದ
ಸಿಕ್ಕಿ-ದಂ-ತಾ-ಯಿತು
ಸಿಕ್ಕಿ-ದ-ನ-ಲ್ಲ-ದೇ-ವ-ರನ್ನು
ಸಿಕ್ಕಿ-ದ-ವನು
ಸಿಕ್ಕಿ-ದಾಗ
ಸಿಕ್ಕಿದೆ
ಸಿಕ್ಕಿ-ದ್ದ-ಅ-ವನೇ
ಸಿಕ್ಕಿ-ದ್ದನ್ನು
ಸಿಕ್ಕಿ-ದ್ದ-ನ್ನೆಲ್ಲ
ಸಿಕ್ಕಿ-ಬಿ-ಟ್ಟರೆ
ಸಿಕ್ಕಿ-ಬಿ-ಟ್ಟಿತ್ತು
ಸಿಕ್ಕಿ-ಬಿ-ಟ್ಟಿ-ದ್ದಾನೆ
ಸಿಕ್ಕಿ-ಬೀಳು
ಸಿಕ್ಕಿಲ್ಲ
ಸಿಕ್ಕಿವೆ
ಸಿಕ್ಕಿ-ಹಾ-ಕಿ-ಕೊಂ-ಡರೆ
ಸಿಕ್ಕಿ-ಹಾ-ಕಿ-ಕೊಂಡು
ಸಿಕ್ಕೀತು
ಸಿಕ್ಖರು
ಸಿಗ-ದಂ-ತಾ-ದಾಗ
ಸಿಗ-ದಿ-ದ್ದು-ದ-ರಿಂದ
ಸಿಗದೆ
ಸಿಗ-ದೆ-ಹೋ-ಗ-ಬೇ-ಕಾ-ದರೆ
ಸಿಗ-ದೆ-ಹೋ-ದರೆ
ಸಿಗ-ಬೇ-ಕಲ್ಲ
ಸಿಗ-ಬೇ-ಕಾ-ದರೂ
ಸಿಗ-ಲಿಲ್ಲ
ಸಿಗಲೇ
ಸಿಗು-ತ್ತದೆ
ಸಿಗು-ತ್ತ-ದೆಯೋ
ಸಿಗು-ತ್ತಾ-ನೆಯೆ
ಸಿಗು-ತ್ತಿತ್ತು
ಸಿಗು-ತ್ತಿದೆ
ಸಿಗುವ
ಸಿಗು-ವಂ-ತಾ-ಗ-ಬೇಕಾ
ಸಿಗು-ವಂ-ತಿ-ರ-ಬೇಕು
ಸಿಗು-ವಂ-ಥ-ದಲ್ಲ
ಸಿಗು-ವನೋ
ಸಿಗು-ವು-ದಿ-ರಲಿ
ಸಿಗು-ವು-ದಿಲ್ಲ
ಸಿಟ್ಟಿ-ಗೆ-ದ್ದರು
ಸಿಟ್ಟು
ಸಿಟ್ಟು-ಗೊಂಡು
ಸಿಟ್ಟೂ
ಸಿಟ್ಟೇ-ರಿತು
ಸಿಟ್ಟೇ-ರಿರ
ಸಿಡಿ-ದೆ-ರ-ಗು-ವ-ವ-ರೆಗೂ
ಸಿಡಿ-ಮಿಡಿ
ಸಿಡಿ-ಮಿ-ಡಿ-ಯನ್ನು
ಸಿಡಿಲ
ಸಿಡಿ-ಲಿ-ನಂ-ತಹ
ಸಿಡಿಲು
ಸಿಡಿಲೆ
ಸಿಡಿಲೈ
ಸಿಡುಬು
ಸಿತಾರ್
ಸಿದ
ಸಿದ-ನೇನೋ
ಸಿದರು
ಸಿದ್ದು
ಸಿದ್ಧ
ಸಿದ್ಧ-ಪ್ರ-ಬ-ದ್ಧ-ನಾದ
ಸಿದ್ಧ-ಗೊ-ಳಿ-ಸು-ತ್ತಿ-ದ್ದರು
ಸಿದ್ಧ-ಗೊ-ಳಿ-ಸುವ
ಸಿದ್ಧತೆ
ಸಿದ್ಧ-ತೆ-ಗಳು
ಸಿದ್ಧ-ನಾ-ಗಿದ್ದ
ಸಿದ್ಧ-ನಾ-ಗಿಯೇ
ಸಿದ್ಧ-ನಾ-ಗಿರ
ಸಿದ್ಧ-ನಾ-ಗಿರು
ಸಿದ್ಧ-ನಾ-ಗಿ-ರು-ವ-ವನು
ಸಿದ್ಧ-ನಾ-ಗಿಲ್ಲ
ಸಿದ್ಧ-ನಾದ
ಸಿದ್ಧ-ನಾ-ದ-ದ್ದನ್ನು
ಸಿದ್ಧ-ನಿ-ದ್ದೇನೆ
ಸಿದ್ಧ-ನಿ-ರು-ವಾಗ
ಸಿದ್ಧ-ನಿ-ರು-ವೆಯಾ
ಸಿದ್ಧನೆ
ಸಿದ್ಧ-ಪ-ಡಿ-ಸಿ-ಕೊಂ-ಡರು
ಸಿದ್ಧ-ಪ-ಡಿ-ಸಿ-ದ್ದರು
ಸಿದ್ಧ-ಪ-ಡಿ-ಸು-ತ್ತಾನೆ
ಸಿದ್ಧ-ಪು-ರು-ಷ-ನಾ-ಗಿ-ಬಿ-ಟ್ಟರೆ
ಸಿದ್ಧ-ಪು-ರು-ಷನೋ
ಸಿದ್ಧ-ಪು-ರು-ಷರು
ಸಿದ್ಧ-ಮಂತ್ರ
ಸಿದ್ಧ-ರಾಗಿ
ಸಿದ್ಧ-ರಾ-ಗಿ-ದ್ದಾರೆ
ಸಿದ್ಧ-ರಾ-ಗಿ-ರ-ದಿ-ದ್ದರೂ
ಸಿದ್ಧ-ರಾ-ಗಿ-ರು-ವುದು
ಸಿದ್ಧ-ರಾ-ದರು
ಸಿದ್ಧ-ರಿ-ದ್ದರು
ಸಿದ್ಧ-ಳಾಗಿ
ಸಿದ್ಧ-ಳಾ-ದಳು
ಸಿದ್ಧ-ವಾ-ಗಿ-ರು-ವಾಗ
ಸಿದ್ಧ-ವಾ-ದಂ-ತಿತ್ತು
ಸಿದ್ಧ-ಸಿ-ಕೊ-ಡು-ವು-ದಾಗಿ
ಸಿದ್ಧಾಂತ
ಸಿದ್ಧಾಂ-ತ-ಗಳ
ಸಿದ್ಧಾಂ-ತ-ಗಳನ್ನು
ಸಿದ್ಧಾಂ-ತ-ಗಳನ್ನೆಲ್ಲ
ಸಿದ್ಧಾಂ-ತ-ಗಳಲ್ಲಿ
ಸಿದ್ಧಾಂ-ತ-ಗಳು
ಸಿದ್ಧಾಂ-ತ-ಗಳೂ
ಸಿದ್ಧಾಂ-ತ-ಗ-ಳೆಷ್ಟು
ಸಿದ್ಧಾಂ-ತದ
ಸಿದ್ಧಾಂ-ತ-ವನ್ನು
ಸಿದ್ಧಾಂ-ತವೇ
ಸಿದ್ಧಾಂ-ತಾ-ದಿ-ಗ-ಳೆಲ್ಲ
ಸಿದ್ಧಾಂ-ತಿ-ಗಳು
ಸಿದ್ಧಿ-ಗಳನ್ನು
ಸಿದ್ಧಿ-ಗಳನ್ನೆಲ್ಲ
ಸಿದ್ಧಿ-ಗಳು
ಸಿದ್ಧಿ-ಗಳೂ
ಸಿದ್ಧಿ-ಗ-ಳೆಲ್ಲ
ಸಿದ್ಧಿ-ಗ-ಳೊಂ-ದಿಗೆ
ಸಿದ್ಧಿ-ಗಾಗಿ
ಸಿದ್ಧಿ-ಯನ್ನು
ಸಿದ್ಧಿ-ಯಾ-ಯಿತು
ಸಿದ್ಧಿಯೂ
ಸಿದ್ಧಿ-ಯೊಂ-ದನ್ನೇ
ಸಿದ್ಧಿ-ಸ-ರ್ವ-ಸ್ವ-ವನ್ನೇ
ಸಿದ್ಧಿ-ಸ-ಲೇ-ಬೇಕು
ಸಿದ್ಧಿ-ಸಿ-ಕೊಂಡ
ಸಿದ್ಧಿ-ಸಿ-ಕೊಂ-ಡಿ-ದ್ದಾ-ರೆಯೋ
ಸಿದ್ಧಿ-ಸಿ-ಕೊಂಡು
ಸಿದ್ಧಿ-ಸಿ-ಕೊಂ-ಡು-ಬಿ-ಡ-ಬ-ಹುದು
ಸಿದ್ಧಿ-ಸಿ-ಕೊ-ಳ್ಳ-ಬಲ್ಲೆ
ಸಿದ್ಧಿ-ಸಿ-ಕೊ-ಳ್ಳ-ಬ-ಹುದು
ಸಿದ್ಧಿ-ಸಿ-ಕೊ-ಳ್ಳಲು
ಸಿದ್ಧಿ-ಸಿ-ಕೊ-ಳ್ಳುವ
ಸಿದ್ಧಿ-ಸಿ-ಕೊ-ಳ್ಳು-ವ-ವ-ನಿ-ದ್ದಾನೆ
ಸಿದ್ಧಿ-ಸಿತು
ಸಿದ್ಧಿ-ಸಿ-ದ-ಮೇಲೆ
ಸಿದ್ಧಿ-ಸು-ವಂ-ಥ-ದಲ್ಲ
ಸಿನ್ಹ
ಸಿಪಾ-ಯಿ-ಗಳ
ಸಿಪಾ-ಯಿ-ಗಳಿಂದ
ಸಿಪಾ-ಯಿ-ಗಳು
ಸಿಮು-ಲಿಯಾ
ಸಿಮು-ಲ್ತಾಲಾ
ಸಿಮ್ಲಾ
ಸಿಯೂ
ಸಿರಾ-ಪಿಸ್
ಸಿರಿ
ಸಿಲು-ಕ-ದಿಹ
ಸಿಲುಕಿ
ಸಿಲು-ಕಿ-ಕೊಂ-ಡಿತು
ಸಿಲು-ಕಿ-ಕೊಂಡು
ಸಿಲು-ಕಿ-ಕೊಂ-ಡು-ಬಿ-ಟ್ಟಾನೋ
ಸಿಲು-ಕಿ-ಕೊಂ-ಡು-ಬಿ-ಟ್ಟಿ-ದ್ದಾ-ರೆ-ಸಂ-ನ್ಯಾ-ಸಿಯ
ಸಿಲು-ಕಿದ
ಸಿಲು-ಕಿ-ದ್ದನ್ನು
ಸಿಳ್ಳಿನ
ಸಿಹಿ-ತಿಂಡಿ
ಸಿಹಿ-ತಿಂ-ಡಿ-ಗಳನ್ನು
ಸಿಹಿ-ತಿ-ನಿ-ಸನ್ನೂ
ಸೀತಾ-ರಾ-ಮರ
ಸೀತಾ-ರಾ-ಮ-ರನ್ನು
ಸೀತಾ-ರಾ-ಮರು
ಸೀತಾ-ರಾ-ಮರ
ಸೀತಾ-ರಾಮ್
ಸೀತಾಳೆ
ಸೀತೆ
ಸೀದಾ
ಸೀದಾ-ಸೀದಾ
ಸೀದು
ಸೀಮಾ-ತೀ-ತ-ನಾ-ಗ-ಬಲ್ಲ
ಸೀಮಾ-ಬ-ದ್ಧ-ನಾದ
ಸೀಮಿ-ತ-ಗೊ-ಳಿ-ಸಲು
ಸೀಮಿ-ತ-ಗೊ-ಳಿ-ಸಿ-ಕೊಂ-ಡಾಗ
ಸೀಮಿ-ತ-ಗೊ-ಳಿ-ಸಿ-ದ-ವ-ನಲ್ಲ
ಸೀಮಿ-ತ-ವಾ-ಗಿ-ರ-ಲಿಲ್ಲ
ಸೀರೆ
ಸೀರೆಯ
ಸೀರೆ-ಯನ್ನೇ
ಸೀರೆ-ಯೊಂದು
ಸೀಲರು
ಸೀಳು-ವಂ-ತಹ
ಸುಂದರ
ಸುಂದ-ರ-ವಾ-ಗಿತ್ತು
ಸುಂದ-ರ-ವಾ-ಗಿಯೇ
ಸುಂದ-ರ-ವಾದ
ಸುಂದ-ರಾಂ-ಗ-ನೆಂದು
ಸುಕೋ-ಮಲ
ಸುಖ
ಸುಖ-ದುಃಖ
ಸುಖ-ಕ್ಕಾಗಿ
ಸುಖ-ಗಳೇ
ಸುಖದ
ಸುಖ-ದಲ್ಲಿ
ಸುಖ-ಭೋ-ಗ-ಗ-ಳಾ-ಗಲಿ
ಸುಖ-ವನ್ನು
ಸುಖ-ವಾ-ಗಿ-ರ-ಬ-ಹುದು
ಸುಖ-ವಾ-ಗಿ-ರು-ವುದೇ
ಸುಖಾ-ನು-ಭ-ವಕ್ಕೆ
ಸುಖಾ-ಭಿ-ಲಾ-ಷೆ-ಯ-ನ್ನೊ-ಳ-ಗೊಂ-ಡಂತೆ
ಸುಖಿ-ಸು-ತ್ತಿ-ರ-ಬೇ-ಕೆಂಬ
ಸುಗಂ-ಧ-ವನ್ನು
ಸುಗ-ಮ-ಸು-ಖ-ಕ-ರ-ವಾ-ಗಿಯೇ
ಸುಗ-ಮ-ವಾ-ಗು-ವು-ದೆಂಬ
ಸುಗ-ಮ-ವಾ-ದ-ದ್ದೇ-ನಲ್ಲ
ಸುಟ್ಟ
ಸುಟ್ಟು
ಸುಟ್ಟು-ಬಿ-ಟ್ಟೆ-ನೆಂದು
ಸುಟ್ಟು-ಬಿ-ಡೋಣ
ಸುಟ್ಟು-ಹಾ-ಕು-ತ್ತಿದೆ
ಸುಟ್ಟು-ಹೋಗಿ
ಸುಡು-ತ್ತಿದೆ
ಸುಡು-ತ್ತಿ-ರುವ
ಸುತನ
ಸುತರ
ಸುತ-ರಾಂ
ಸುತ್ತ
ಸುತ್ತ-ಮುತ್ತ
ಸುತ್ತ-ಮು-ತ್ತಲ
ಸುತ್ತ-ಮು-ತ್ತೆಲ್ಲ
ಸುತ್ತ-ಲಿದ್ದ
ಸುತ್ತ-ಲಿ-ದ್ದ-ವರೆ-ಲ್ಲ-ರನ್ನೂ
ಸುತ್ತ-ಲಿನ
ಸುತ್ತಲೂ
ಸುತ್ತಾ-ಡ-ಬೇ-ಕೆಂದು
ಸುತ್ತಾಡಿ
ಸುತ್ತಾ-ಡಿದ
ಸುತ್ತಾಡು
ಸುತ್ತಾ-ಡುತ್ತ
ಸುತ್ತಾ-ಡು-ತ್ತಿದ್ದ
ಸುತ್ತಾ-ಡು-ತ್ತಿ-ದ್ದರು
ಸುತ್ತಾ-ಡು-ತ್ತಿ-ದ್ದೀಯ
ಸುತ್ತಾ-ಡೋಣ
ಸುತ್ತಿ
ಸುತ್ತಿ-ಕೊಂ-ಡಿದ್ದ
ಸುತ್ತಿ-ಕೊಂಡು
ಸುತ್ತಿತ್ತು
ಸುತ್ತಿದ
ಸುತ್ತಿ-ದರೂ
ಸುತ್ತಿದ್ದ
ಸುತ್ತಿ-ದ್ದರು
ಸುತ್ತು-ಗಟ್ಟಿ
ಸುತ್ತು-ಬ-ರ-ತೊ-ಡ-ಗಿದ
ಸುತ್ತು-ವ-ರಿ-ದಿ-ರು-ವುದನ್ನು
ಸುತ್ತು-ಹಾಕಿ
ಸುದೀರ್ಘ
ಸುದೀ-ರ್ಘ-ವಾಗಿ
ಸುದೃ-ಢ-ಗೊಂಡು
ಸುದೃ-ಢ-ವಾದ
ಸುದ್ದಿ
ಸುದ್ದಿ-ಕೇಳಿ
ಸುದ್ದಿ-ಗಳನ್ನೆಲ್ಲ
ಸುದ್ದಿ-ಯನ್ನು
ಸುದ್ದಿ-ಯಲ್ಲಿ
ಸುದ್ದಿಯೇ
ಸುದ್ಧಿ
ಸುಧಾ-ರಕ
ಸುಧಾ-ರ-ಕ-ನಾ-ಗ-ಬ-ಹು-ದಾ-ಗಿತ್ತು
ಸುಧಾ-ರ-ಕ-ರಿಗೆ
ಸುಧಾ-ರ-ಕರೂ
ಸುಧಾ-ರಣೆ
ಸುಧಾ-ರ-ಣೆ-ಗ-ಳೆಲ್ಲ
ಸುಧಾ-ರ-ಣೆಯ
ಸುಧಾ-ರಿ-ಸ-ದಿ-ದ್ದು-ದ-ರಿಂದ
ಸುಧಾ-ರಿ-ಸ-ಲಿ-ಲ್ಲ-ವೆಂಬ
ಸುಧಾ-ರಿಸಿ
ಸುಧಾ-ರಿ-ಸಿ-ಕೊಂಡ
ಸುಧಾ-ರಿ-ಸಿ-ಕೊಂ-ಡರು
ಸುಧಾ-ರಿ-ಸಿ-ಕೊಂಡು
ಸುಧಾ-ರಿ-ಸಿ-ಕೊಂ-ಡೆದ್ದು
ಸುಧಾ-ರಿ-ಸಿ-ಕೊ-ಳ್ಳ-ಲಾ-ಗು-ವಂತೆ
ಸುಧಾ-ರಿ-ಸಿತು
ಸುಧಾ-ರಿ-ಸಿ-ದಂತೆ
ಸುಧಾ-ರಿ-ಸಿ-ದ್ದ-ರಿಂದ
ಸುಧೀರ್ಘ
ಸುನಿ-ಶ್ಚಿ-ತ-ವಾದ
ಸುಪ-ರಿಚಿ
ಸುಪ-ರಿ-ಚಿತ
ಸುಪು-ಷ್ಟ-ವಾದ
ಸುಪ್ತ-ವಾ-ಗಿದ್ದ
ಸುಪ್ತ-ವಾ-ಗಿ-ರುವ
ಸುಪ್ಪ-ತ್ತಿಗೆ
ಸುಪ್ಪ-ತ್ತಿ-ಗೆ-ಯಿಂದ
ಸುಪ್ರ-ಸಿದ್ಧ
ಸುಪ್ರೀ-ತ-ರಾದ
ಸುಬೋಧ
ಸುಬೋ-ಧಾ-ನಂದ
ಸುಮ-ಧುರ
ಸುಮ-ಧು-ರ-ವಾಗಿ
ಸುಮ-ಧು-ರ-ವಾ-ದದ್ದು
ಸುಮಾ-ರಿಗೆ
ಸುಮಾರು
ಸುಮ್ಮ
ಸುಮ್ಮ-ನಾ-ಗ-ಬೇ-ಕಾ-ಯಿತು
ಸುಮ್ಮ-ನಾಗಿ
ಸುಮ್ಮ-ನಾ-ಗಿ-ಬಿಟ್ಟ
ಸುಮ್ಮ-ನಾ-ಗಿ-ಬಿ-ಟ್ಟರು
ಸುಮ್ಮ-ನಾ-ಗಿ-ಬಿ-ಡು-ತ್ತಾನೆ
ಸುಮ್ಮ-ನಾ-ಗಿ-ಸಿ-ದ್ದರು
ಸುಮ್ಮ-ನಾದ
ಸುಮ್ಮ-ನಾ-ದರು
ಸುಮ್ಮ-ನಿದ್ದ
ಸುಮ್ಮ-ನಿದ್ದು
ಸುಮ್ಮ-ನಿ-ದ್ದು-ಬಿಟ್ಟ
ಸುಮ್ಮ-ನಿ-ದ್ದು-ಬಿ-ಟ್ಟರು
ಸುಮ್ಮ-ನಿ-ದ್ದು-ಬಿ-ಟ್ಟೆ-ಹೇ-ಳಿ-ಕೊಂಡು
ಸುಮ್ಮ-ನಿ-ರಲು
ಸುಮ್ಮ-ನಿ-ರು-ತ್ತಿ-ದ್ದರು
ಸುಮ್ಮ-ನಿ-ರು-ವಂ-ತಿಲ್ಲ
ಸುಮ್ಮ-ನಿ-ರು-ವಂತೆ
ಸುಮ್ಮ-ನಿರೋ
ಸುಮ್ಮನೆ
ಸುಮ್ಮ-ನೆಯೇ
ಸುಮ್ಮ-ಸು-ಮ್ಮನೆ
ಸುಯಲ್
ಸುರ-ನ-ರ-ರಿಂದ
ಸುರ-ಕ್ಷಿ-ತ-ವಾಗಿ
ಸುರಿದ
ಸುರಿ-ದರು
ಸುರಿ-ದ-ರೆಂ-ದರೂ
ಸುರಿ-ಯ-ತೊ-ಡ-ಗಿತು
ಸುರಿ-ಯ-ಬ-ಹುದು
ಸುರಿ-ಯು-ತ್ತಿದೆ
ಸುರಿ-ಯು-ತ್ತಿ-ದ್ದಳು
ಸುರಿ-ಯು-ವು-ದ-ಕ್ಕಾಗಿ
ಸುರಿ-ವುದೊ
ಸುರಿ-ಸಿ-ದರೂ
ಸುರಿ-ಸಿ-ದ್ದಕ್ಕೆ
ಸುರಿ-ಸುತ್ತ
ಸುರಿ-ಸು-ತ್ತಾರೆ
ಸುರಿ-ಸು-ತ್ತಿ-ದ್ದಾನೆ
ಸುರಿ-ಸು-ತ್ತಿ-ದ್ದಾರೆ
ಸುರಿ-ಸು-ವಂ-ತಹ
ಸುರಿ-ಸು-ವುದನ್ನು
ಸುರುಳಿ
ಸುರು-ಸು-ತ್ತೀಯೆ
ಸುರೇಂದ್ರ
ಸುರೇಂ-ದ್ರ-ನಾಥ
ಸುರೇಂ-ದ್ರ-ನಾ-ಥ-ಮಿ-ತ್ರನೂ
ಸುರೇಂ-ದ್ರ-ನಿಗೆ
ಸುರೇಂ-ದ್ರ-ಬಾಬು
ಸುರೇಶ್
ಸುಲಭ
ಸುಲ-ಭ
ಸುಲ-ಭದ
ಸುಲ-ಭ-ದಲ್ಲಿ
ಸುಲ-ಭ-ವಲ್ಲ
ಸುಲ-ಭ-ವಾಗಿ
ಸುಲ-ಭ-ವಾ-ಗಿ-ರ-ಲಿಲ್ಲ
ಸುಲ-ಭ-ವಾ-ಯಿತು
ಸುಲ-ಭವೆ
ಸುಲ-ಭೋ-ಪಾಯ
ಸುಲ-ಭೋ-ಪಾ-ಯ-ವೆಂದರೆ
ಸುಳಿ-ದದ್ದೇ
ಸುಳಿ-ದರೂ
ಸುಳಿ-ದಾ-ಡಿ-ತು-ನಿ-ಜಕ್ಕೂ
ಸುಳಿ-ದಾ-ಡಿದ
ಸುಳಿ-ದಾ-ಡು-ತ್ತಿದೆ
ಸುಳಿಯ
ಸುಳಿ-ಯದು
ಸುಳಿ-ಯ-ದೊ-ಮೇಣ್
ಸುಳಿ-ಯ-ಲಾ-ರಳು
ಸುಳಿ-ಯಲ್ಲಿ
ಸುಳಿ-ಯವೋ
ಸುಳಿ-ಯಿಂದ
ಸುಳಿ-ಯು-ತ್ತಿ-ದ್ದುವು
ಸುಳಿ-ಯು-ವುದೂ
ಸುಳಿ-ಯೊ-ಳಗೆ
ಸುಳಿವು
ಸುಳಿ-ವುದೊ
ಸುಳಿವೇ
ಸುಳ್ಳಲ್ಲ
ಸುಳ್ಳ-ಲ್ಲ-ವೆಂ-ದಾ-ಯಿತು
ಸುಳ್ಳಾ-ಗಲು
ಸುಳ್ಳಾ-ಡಿ-ದಂ-ತಾ-ಗು-ತ್ತದೆ
ಸುಳ್ಳಿನ
ಸುಳ್ಳು
ಸುಳ್ಳು-ಪ-ಳ್ಳನ್ನು
ಸುಳ್ಳೆಂದು
ಸುವ
ಸುವಂ-ತಿ-ರ-ಬೇಕು
ಸುವ-ರ್ಣಾ-ಕ್ಷ-ರ-ಗಳಲ್ಲಿ
ಸುವ-ಷ್ಟ-ರಲ್ಲೇ
ಸುವಾ-ಸನೆ
ಸುಶಿ-ಕ್ಷಿತ
ಸುಶಿ-ಕ್ಷಿ-ತ-ರಾದ
ಸುಸಂ-ಘ-ಟಿತ
ಸುಸಂ-ಘ-ಟಿ-ತ-ರಾ-ದದ್ದೇ
ಸುಸಂ-ಬದ್ಧ
ಸುಸಂ-ಬ-ದ್ಧ-ವಾ-ಗಿದೆ
ಸುಸ್ತಾ-ಗಿ-ಬಿ-ಡು-ತ್ತಿ-ದ್ದರು
ಸುಸ್ತಾದ
ಸುಸ್ಥಿ-ತಿಗೆ
ಸುಸ್ಪಷ್ಟ
ಸೂಕ್ತ
ಸೂಕ್ತ-ಎಂದು
ಸೂಕ್ತ-ಮಾ-ರ್ಗ-ದ-ರ್ಶನ
ಸೂಕ್ತ-ವಾಗಿ
ಸೂಕ್ತ-ವಾದ
ಸೂಕ್ತ-ವಾ-ದು-ದನ್ನು
ಸೂಕ್ಷ-ಬು-ದ್ಧಿಗೆ
ಸೂಕ್ಷವೂ
ಸೂಕ್ಷ್ಮ
ಸೂಕ್ಷ್ಮ-ಗಳನ್ನೆಲ್ಲ
ಸೂಕ್ಷ್ಮ-ತ-ರ-ವಾದ
ಸೂಕ್ಷ್ಮ-ತೆ-ಗಳನ್ನು
ಸೂಕ್ಷ್ಮ-ತೆ-ಯಿಂದ
ಸೂಕ್ಷ್ಮ-ಬು-ದ್ಧಿಯ
ಸೂಕ್ಷ್ಮ-ಮ-ತಿ-ಯನ್ನೂ
ಸೂಕ್ಷ್ಮ-ರೂ-ಪ-ವನ್ನು
ಸೂಕ್ಷ್ಮ-ವನ್ನು
ಸೂಕ್ಷ್ಮ-ವಾಗಿ
ಸೂಕ್ಷ್ಮ-ವಾದ
ಸೂಕ್ಷ್ಮ-ವಿ-ಚಾ-ರ-ಗಳನ್ನೆಲ್ಲ
ಸೂಕ್ಷ್ಮ-ಶಕ್ತಿ
ಸೂಕ್ಷ್ಮಾ-ತಿ-ಸೂ-ಕ್ಷ್ಮ-ವಾದ
ಸೂಚನೆ
ಸೂಚ-ನೆ-ಸಂ-ದೇ-ಶ-ಗಳನ್ನು
ಸೂಚ-ನೆ-ಗಳನ್ನು
ಸೂಚ-ನೆ-ಗಳು
ಸೂಚ-ನೆ-ಯ-ನ್ನಿ-ತ್ತರು
ಸೂಚ-ನೆ-ಯ-ರಿತು
ಸೂಚ-ನೆ-ಯಾ-ಗಿ-ರು-ವಂ-ತಿದೆ
ಸೂಚ-ನೆ-ಯಿ-ತ್ತರು
ಸೂಚ-ನೆಯೇ
ಸೂಚಿ-ತ-ವಾ-ಗಿದ್ದ
ಸೂಚಿಸಿ
ಸೂಚಿ-ಸಿ-ದರು
ಸೂಚಿ-ಸಿ-ದಳು
ಸೂಚಿ-ಸಿ-ದ-ವ-ರಲ್ಲಿ
ಸೂಚಿ-ಸಿ-ದಾಗ
ಸೂಚಿಸು
ಸೂಚಿ-ಸು-ತ್ತಿತ್ತು
ಸೂಚಿ-ಸು-ತ್ತಿದ್ದ
ಸೂಚಿ-ಸು-ತ್ತಿ-ದ್ದರು
ಸೂಚಿ-ಸು-ತ್ತಿ-ದ್ದವು
ಸೂಚಿ-ಸು-ತ್ತಿ-ದ್ದಾರೆ
ಸೂಚಿ-ಸುವ
ಸೂಚಿ-ಸು-ವುದೇ
ಸೂತ-ಕದ
ಸೂರೆ-ಗೊಂ-ಡಿ-ದ್ದ-ನೆಂ-ದರೆ
ಸೂರ್ಯ
ಸೂರ್ಯ-ಚಂ-ದ್ರ-ನ-ಕ್ಷ-ತ್ರ-ಗಳಿಂದ
ಸೂರ್ಯ-ಚಂ-ದ್ರರೇ
ಸೂರ್ಯ-ನಂತೆ
ಸೂರ್ಯ-ನಿಗೆ
ಸೂರ್ಯ-ಮಂ-ಡ-ಲ-ಗಳು
ಸೂರ್ಯ-ಮಂ-ಡ-ಲ-ದಲ್ಲಿ
ಸೂರ್ಯ-ವಂಶ
ಸೂರ್ಯಾ-ಸ್ತದ
ಸೂರ್ಯಾ-ಸ್ತ-ಮಾನ
ಸೂರ್ಯೋ-ದ-ಯಕ್ಕೆ
ಸೂಸು-ತ್ತಿದೆ
ಸೃಜಾ-ಮ್ಯ-ಹಮ್
ಸೃಜಿ-ಸಿ-ದ್ದಾನೆ
ಸೃಷ್ಟಿ
ಸೃಷ್ಟಿ-ಸ್ಥಿ-ತಿ-ಲ-ಯ-ಗಳ
ಸೃಷ್ಟಿ-ಕರ್ತ
ಸೃಷ್ಟಿಯ
ಸೃಷ್ಟಿ-ಯಲ್ಲ
ಸೃಷ್ಟಿ-ಯಲ್ಲಿ
ಸೃಷ್ಟಿ-ಯಾ-ಗಲು
ಸೃಷ್ಟಿ-ಯಾದ
ಸೃಷ್ಟಿಯೂ
ಸೃಷ್ಟಿ-ರ-ಚನಾ
ಸೃಷ್ಟಿ-ಸ-ಲ್ಪ-ಟ್ಟಿ-ರುವ
ಸೃಷ್ಟಿ-ಸಿ-ದ-ವಳು
ಸೆಪ್ಟೆಂ-ಬರ್
ಸೆರೆ-ಯಾ-ಳಾ-ಗಿ-ರು-ವಂತೆ
ಸೆರೆ-ಹಿ-ಡಿ-ದು-ಬಿ-ಟ್ಟಿತು
ಸೆರೆ-ಹಿ-ಡಿ-ದು-ಬಿ-ಟ್ಟಿ-ದ್ದುವು
ಸೆಲೆ-ಯಾಗಿ
ಸೆಲೆ-ಯಾ-ಗಿದ್ದ
ಸೆಳೆತ
ಸೆಳೆ-ತಕ್ಕೆ
ಸೆಳೆದು
ಸೆಳೆ-ದು-ಕೊಂಡು
ಸೆಳೆ-ದು-ಕೊ-ಳ್ಳು-ವುದು
ಸೆಳೆ-ಯ-ದಿ-ರಲಿ
ಸೆಳೆ-ಯ-ಲಾ-ರಂ-ಭಿ-ಸಿತ್ತು
ಸೆಳೆ-ಯ-ಲ್ಪಟ್ಟ
ಸೆಳೆ-ಯು-ತ್ತಾರೆ
ಸೆಳೆ-ಯು-ತ್ತಿದೆ
ಸೆಳೆ-ಯು-ತ್ತಿದ್ದ
ಸೆಳೆ-ಯು-ತ್ತಿ-ದ್ದುವು
ಸೆಳೆ-ಯುವ
ಸೆಳೆ-ಯು-ವಂ-ತಹ
ಸೆಳೆ-ಯು-ವಂ-ತಿತ್ತು
ಸೆಳೆ-ವನು
ಸೇಠ್
ಸೇಠ್ಜಿ
ಸೇಠ್ಜಿಯ
ಸೇಠ್ಜಿ-ಯೊ-ಬ್ಬನ
ಸೇತುವು
ಸೇತು-ವೆ-ಯಾ-ಗು-ತ್ತದೆ
ಸೇದ-ಬೇ-ಕಾ-ಯಿತು
ಸೇದಲು
ಸೇದ-ಹೊ-ರ-ಟರು
ಸೇದಿ
ಸೇದಿದ
ಸೇದಿ-ದರು
ಸೇದಿ-ದರೆ
ಸೇದಿ-ದ-ವರು
ಸೇದಿ-ನೋ-ಡಿದ
ಸೇದು
ಸೇದು-ತ್ತಿ-ದ್ದುದು
ಸೇದುವ
ಸೇದು-ವಷ್ಟು
ಸೇದು-ವುದು
ಸೇನ
ಸೇನ-ಇ-ವರು
ಸೇನ-ನಂ-ಥ-ವನ
ಸೇನನೂ
ಸೇನಾ-ಧಿ-ಪ-ತಿ-ಯಂತೆ
ಸೇನ್
ಸೇರದು
ಸೇರ-ಬೇ-ಕಾ-ದರೆ
ಸೇರಲು
ಸೇರಿ
ಸೇರಿ-ಕೊಂಡ
ಸೇರಿ-ಕೊಂ-ಡರು
ಸೇರಿ-ಕೊಂ-ಡರೆ
ಸೇರಿ-ಕೊಂ-ಡಿತ್ತು
ಸೇರಿ-ಕೊಂ-ಡಿದೆ
ಸೇರಿ-ಕೊಂ-ಡಿದ್ದ
ಸೇರಿ-ಕೊಂ-ಡಿ-ದ್ದ-ನಾ-ದರೂ
ಸೇರಿ-ಕೊಂ-ಡಿ-ದ್ದುವು
ಸೇರಿ-ಕೊಂ-ಡಿ-ರು-ವು-ದಿ-ಲ್ಲವೆ
ಸೇರಿ-ಕೊಂಡು
ಸೇರಿ-ಕೊ-ಳ್ಳು-ತ್ತ-ವೆಯೋ
ಸೇರಿ-ಕೊ-ಳ್ಳು-ವಂತೆ
ಸೇರಿತು
ಸೇರಿತ್ತು
ಸೇರಿದ
ಸೇರಿ-ದಂತೆ
ಸೇರಿ-ದರು
ಸೇರಿ-ದ-ವ-ನಾ-ಗಿ-ಬಿಟ್ಟ
ಸೇರಿ-ದ-ವ-ನಾದ್ದ
ಸೇರಿ-ದ-ವನು
ಸೇರಿ-ದ-ವ-ರದು
ಸೇರಿ-ದ-ವ-ರಾ-ಗಿ-ದ್ದರು
ಸೇರಿ-ದ-ವ-ರಾ-ಗಿ-ಬಿ-ಟ್ಟರು
ಸೇರಿ-ದ-ವ-ರಾ-ದರು
ಸೇರಿ-ದ-ವರು
ಸೇರಿ-ದ-ವ-ರೆಂಬ
ಸೇರಿದೆ
ಸೇರಿದ್ದ
ಸೇರಿ-ದ್ದರು
ಸೇರಿ-ದ್ದ-ವರೇ
ಸೇರಿ-ದ್ದಾರೆ
ಸೇರಿ-ಬಿ-ಟ್ಟರು
ಸೇರಿ-ಬಿ-ಡು-ತ್ತಾನೋ
ಸೇರಿಯೂ
ಸೇರಿ-ರು-ವು-ದ-ರಿಂದ
ಸೇರಿ-ಸ-ಬೇಕೆ
ಸೇರಿ-ಸ-ಲಾ-ಯಿತು
ಸೇರಿ-ಸಲು
ಸೇರಿಸಿ
ಸೇರಿ-ಸಿ-ಕೊಂ-ಡರು
ಸೇರಿ-ಸಿ-ಕೊಂ-ಡಿದ್ದ
ಸೇರಿ-ಸಿ-ಕೊಂಡು
ಸೇರಿ-ಸಿ-ಕೊ-ಳ್ಳ-ಬ-ಹು-ದಿತ್ತು
ಸೇರಿ-ಸಿದ
ಸೇರಿ-ಸು-ತ್ತಾ-ನೇ-ನಮ್ಮಾ
ಸೇರಿ-ಸು-ತ್ತಾರೆ
ಸೇರಿ-ಸುವ
ಸೇರಿ-ಸು-ವು-ದಿಲ್ಲ
ಸೇರಿ-ಸು-ವುದು
ಸೇರಿ-ಹೋ-ಗಿ-ಬಿ-ಡು-ತ್ತದೆ
ಸೇರು-ತ್ತಿ-ದ್ದರು
ಸೇರುವ
ಸೇರು-ವು-ದ-ರಲ್ಲಿ
ಸೇರು-ವುದು
ಸೇರು-ವುದೋ
ಸೇರ್ಪಡೆ
ಸೇವಕ
ಸೇವ-ಕನು
ಸೇವಾ-ಕಾ-ರ್ಯ-ಗ-ಳಲ್ಲೂ
ಸೇವಾ-ಪ-ರತೆ
ಸೇವಾ-ಪಾ-ತ್ರರು
ಸೇವಾ-ಸಂ-ಸ್ಥೆ-ಯೊಂ-ದರ
ಸೇವಿ-ಸಲು
ಸೇವಿಸಿ
ಸೇವಿ-ಸು-ವಂತೆ
ಸೇವೆ
ಸೇವೆ-ಶು-ಶ್ರೂ-ಷೆ-ಗಳನ್ನು
ಸೇವೆ-ಗಾಗಿ
ಸೇವೆ-ಗಿಂ-ತಲೂ
ಸೇವೆ-ಗಿಳಿ-ಯುವ
ಸೇವೆಗೆ
ಸೇವೆಯ
ಸೇವೆ-ಯ-ನ್ನಾ-ದರೂ
ಸೇವೆ-ಯನ್ನು
ಸೇವೆ-ಯಲ್ಲಿ
ಸೇವೆ-ಯಾ-ಗು-ವಂ-ತಿ-ದ್ದರೆ
ಸೈನಿ-ಕರ
ಸೈನಿ-ಕ-ರನ್ನು
ಸೈನಿ-ಕ-ರನ್ನೇ
ಸೈನಿ-ಕ-ರಿಂ-ದಲೂ
ಸೈನಿ-ಕ-ರಿಗೆ
ಸೈನಿ-ಕರು
ಸೈನಿ-ಕ-ಸ-ಹ-ಜ-ವಾದ
ಸೊಂಟ
ಸೊಂಟಕ್ಕೆ
ಸೊಂಟ-ಕ್ಕೊಂದು
ಸೊಂಟದ
ಸೊಗ-ದಲಿ
ಸೊಗವ
ಸೊಗ-ಸಾದ
ಸೊಗ-ಸೀತು
ಸೊಗಸು
ಸೊಗ-ಸೇ-ನಿದೆ
ಸೊತ್ತೂ
ಸೊನ್ನೆ
ಸೊನ್ನೆಯ
ಸೊನ್ನೆಯೇ
ಸೊಪ್ಪು
ಸೊಬಗು
ಸೊರಗಿ
ಸೊರ-ಗಿ-ದರೂ
ಸೊರ-ಗಿ-ಹೋ-ಗಿ-ರು-ವುದು
ಸೊಳ್ಳೆ-ಕ್ರಿ-ಮಿ-ಕೀ-ಟ-ಗಳೂ
ಸೊಳ್ಳೆ-ಗಳ
ಸೊಳ್ಳೆ-ಗ-ಳಿಗೆ
ಸೊಳ್ಳೆ-ಗಳು
ಸೊಳ್ಳೆ-ಪ-ರ-ದೆ-ಯೊ-ಳಗೆ
ಸೋಂಕಿತು
ಸೋಂಬೇ-ರಿ-ಗಳನ್ನೆಲ್ಲ
ಸೋಗಿನ
ಸೋಗು
ಸೋಗು-ಗಾ-ರಿ-ಕೆಯ
ಸೋತ
ಸೋತಿ-ದೆ-ಆ-ದ್ದ-ರಿಂದ
ಸೋತಿ-ದ್ದಾನೆ
ಸೋತು
ಸೋದರ
ಸೋದ-ರ-ಸೋ-ದ-ರಿ-ಯ-ರನ್ನೂ
ಸೋದ-ರ-ಸೋ-ದ-ರಿ-ಯ-ರಲ್ಲಿ
ಸೋದ-ರ-ಸೋ-ದ-ರಿ-ಯ-ರೆಲ್ಲ
ಸೋದ-ರ-ನಂತೆ
ಸೋದ-ರ-ನಾದ
ಸೋದ-ರ-ನಿ-ಗಾಗಿ
ಸೋದ-ರ-ಮಾವ
ಸೋದ-ರರು
ಸೋದ-ರರೇ
ಸೋದ-ರ-ಶಿಷ್ಯ
ಸೋದ-ರ-ಶಿ-ಷ್ಯನೂ
ಸೋದ-ರ-ಶಿ-ಷ್ಯರ
ಸೋದ-ರ-ಶಿ-ಷ್ಯ-ರನ್ನೂ
ಸೋದ-ರ-ಶಿ-ಷ್ಯ-ರ-ನ್ನೆಲ್ಲ
ಸೋದ-ರ-ಶಿ-ಷ್ಯ-ರಲ್ಲಿ
ಸೋದ-ರ-ಶಿ-ಷ್ಯ-ರಿಗೆ
ಸೋದ-ರ-ಸಂನ್ಯಾಸಿ
ಸೋದ-ರ-ಸಂ-ನ್ಯಾ-ಸಿ-ಗಳ
ಸೋದ-ರ-ಸಂ-ನ್ಯಾ-ಸಿ-ಗ-ಳಂತೆ
ಸೋದ-ರ-ಸಂ-ನ್ಯಾ-ಸಿ-ಗಳನ್ನು
ಸೋದ-ರ-ಸಂ-ನ್ಯಾ-ಸಿ-ಗಳನ್ನೆಲ್ಲ
ಸೋದ-ರ-ಸಂ-ನ್ಯಾ-ಸಿ-ಗ-ಳಲ್ಲೇ
ಸೋದ-ರ-ಸಂ-ನ್ಯಾ-ಸಿ-ಗ-ಳಾದ
ಸೋದ-ರ-ಸಂ-ನ್ಯಾ-ಸಿ-ಗ-ಳಿಗೆ
ಸೋದ-ರ-ಸಂ-ನ್ಯಾ-ಸಿ-ಗಳು
ಸೋದ-ರ-ಸಂ-ನ್ಯಾ-ಸಿ-ಗ-ಳೆಲ್ಲ
ಸೋದ-ರ-ಸಂ-ನ್ಯಾ-ಸಿ-ಗ-ಳೊಂದಿ
ಸೋದ-ರ-ಸಂ-ನ್ಯಾ-ಸಿ-ಗ-ಳೊಂ-ದಿ-ಗಿನ
ಸೋದ-ರ-ಸಂ-ನ್ಯಾ-ಸಿಗೆ
ಸೋದ-ರ-ಸಂ-ನ್ಯಾ-ಸಿ-ಯಾದ
ಸೋದ-ರ-ಸಂ-ನ್ಯಾ-ಸಿಯೇ
ಸೋದ-ರ-ಸಂ-ನ್ಯಾ-ಸಿ-ಯೊ-ಬ್ಬನು
ಸೋದ-ರ-ಸೋ-ದ-ರಿ-ಯ-ರೆ-ಲ್ಲರೂ
ಸೋದರಿ
ಸೋದ-ರಿಯ
ಸೋದ-ರಿ-ಯಂತೆ
ಸೋದ-ರಿ-ಯರೂ
ಸೋದ-ರಿ-ಯೊ-ಬ್ಬಳು
ಸೋಮ-ವಾರ
ಸೋಮ-ವಾ-ರ-ಗ-ಳಂದು
ಸೋಮ-ವಾ-ರ-ದಂದು
ಸೋಮಾರಿ
ಸೋಮಾ-ರಿ-ಗ-ಳಾದ
ಸೋರಿತು
ಸೋರು-ತ್ತಿತ್ತು
ಸೋಲಿ-ಸ-ಲಾ-ಗ-ಲಿಲ್ಲ
ಸೋಲಿ-ಸು-ತ್ತಾನೆ
ಸೋಲು-ಗಳನ್ನು
ಸೋಲು-ತ್ತಾನೆ
ಸೋಲೊಪ್ಪ
ಸೋಲೊ-ಪ್ಪಿ-ಕೊ-ಳ್ಳ-ದಿ-ರುವ
ಸೋಲೊ-ಪ್ಪಿ-ಕೊ-ಳ್ಳು-ವಂತೆ
ಸೌಂದರ್ಯ
ಸೌಂದ-ರ್ಯಕ್ಕೆ
ಸೌಂದ-ರ್ಯ-ವನ್ನು
ಸೌಜ-ನ್ಯಾದಿ
ಸೌತೆ-ಕಾ-ಯನ್ನು
ಸೌತೆ-ಕಾ-ಯಿ-ಯನ್ನು
ಸೌಧ-ಗ-ಳಲ್ಲೇ
ಸೌಭಾ-ಗ್ಯ-ಗಳೂ
ಸೌಲಭ್ಯ
ಸ್ಕರ
ಸ್ಕಾಟಿಷ್
ಸ್ಟೇಜಿನ
ಸ್ಟೇಷನ್
ಸ್ಟ್ರಾಂಗ್
ಸ್ತಂಭೀ-ಭೂ-ತ-ರಾಗಿ
ಸ್ತಬ್ಧ-ನಾಗಿ
ಸ್ತಬ್ಧ-ನಾದ
ಸ್ತಬ್ಧ-ರಾ-ಗಿ-ಬಿ-ಟ್ಟರು
ಸ್ತಬ್ಧ-ರಾ-ಗಿ-ಬಿ-ಡು-ತ್ತಿ-ದ್ದರು
ಸ್ತಬ್ಧ-ರಾ-ಗು-ತ್ತಿ-ದ್ದರು
ಸ್ತಬ್ಧ-ವಾಗಿ
ಸ್ತರ-ಕ್ಕೇ-ರಿ-ಸುವ
ಸ್ತರ-ಕ್ಕೇ-ರು-ತ್ತಿಲ್ಲ
ಸ್ತರ-ಗಳ
ಸ್ತರ-ದಲ್ಲಿ
ಸ್ತಿಮಿ-ತ-ವನ್ನು
ಸ್ತುತಿ-ನಿಂ-ದೆ-ಗ-ಳಾ-ವುವೂ
ಸ್ತುತಿ-ಸಿ-ದರೆ
ಸ್ತುತಿ-ಸು-ತ್ತಿ-ದೆಯೋ
ಸ್ತೋತ್ರ
ಸ್ತೋತ್ರ-ಮಂ-ತ್ರ-ಗಳನ್ನು
ಸ್ತೋತ್ರ-ಗಳ
ಸ್ತ್ರಿಯನ್ನು
ಸ್ತ್ರೀ-ಪು-ರು-ಷರು
ಸ್ತ್ರೀಕು-ಲದ
ಸ್ತ್ರೀಭಾ-ವ-ದಿಂ-ದಲೂ
ಸ್ತ್ರೀಯನ್ನು
ಸ್ತ್ರೀಯರ
ಸ್ತ್ರೀಯ-ರಂ-ತೆಯೇ
ಸ್ತ್ರೀಯ-ರನ್ನೂ
ಸ್ತ್ರೀಯಾ-ಗಲಿ
ಸ್ತ್ರೀಯೆಂದೇ
ಸ್ತ್ರೀವಿ-ದ್ಯಾ-ಭ್ಯಾ-ಸ-ವನ್ನು
ಸ್ತ್ರೀಶ-ರೀ-ರವೇ
ಸ್ತ್ರೀಸ್ವ-ಭವ
ಸ್ಥಳ
ಸ್ಥಳಕ್ಕೂ
ಸ್ಥಳಕ್ಕೆ
ಸ್ಥಳ-ಗಳಲ್ಲಿ
ಸ್ಥಳ-ಗ-ಳ-ಲ್ಲಿನ
ಸ್ಥಳ-ಗ-ಳಾದ
ಸ್ಥಳ-ಗ-ಳಿಗೆ
ಸ್ಥಳ-ಗಳು
ಸ್ಥಳ-ಗ-ಳೆ-ಲ್ಲ-ವನ್ನೂ
ಸ್ಥಳದ
ಸ್ಥಳ-ದಲ್ಲಿ
ಸ್ಥಳ-ದ-ಲ್ಲಿ-ಇಂ-ತಹ
ಸ್ಥಳ-ದ-ಲ್ಲಿದ್ದು
ಸ್ಥಳ-ದಲ್ಲೂ
ಸ್ಥಳ-ದಲ್ಲೇ
ಸ್ಥಳ-ವನ್ನು
ಸ್ಥಳ-ವನ್ನೂ
ಸ್ಥಳ-ವ-ಲ್ಲವೆ
ಸ್ಥಳ-ವಾ-ದರೂ
ಸ್ಥಳ-ವಿದೆ
ಸ್ಥಳವೂ
ಸ್ಥಳಾಂ-ತ-ರ-ಗೊ-ಳಿ-ಸು-ವ-ವ-ರೆಗೂ
ಸ್ಥಾನ
ಸ್ಥಾನ-ಮಾನ
ಸ್ಥಾನದ
ಸ್ಥಾನ-ದಲ್ಲಿ
ಸ್ಥಾನ-ಮಾ-ನ-ಗಳೂ
ಸ್ಥಾನ-ವ-ನ್ನಿ-ತ್ತಿ-ದ್ದಾರೆ
ಸ್ಥಾನ-ವನ್ನು
ಸ್ಥಾನ-ವಿದೆ
ಸ್ಥಾನ-ವಿಲ್ಲ
ಸ್ಥಾನ-ವಿ-ಲ್ಲ-ದಿ-ರು-ವುದನ್ನು
ಸ್ಥಾಪ-ಕ-ರಾದ
ಸ್ಥಾಪ-ನೆ-ಯಾದ
ಸ್ಥಾಪಿ-ತ-ವಾದ
ಸ್ಥಾಪಿ-ಸ-ಬೇ-ಕಾದ
ಸ್ಥಾಪಿ-ಸಿದ
ಸ್ಥಾಪಿ-ಸಿ-ದ್ದ-ವನು
ಸ್ಥಾಪಿ-ಸಿ-ದ್ದಾರೆ
ಸ್ಥಾಪಿ-ಸಿದ್ದು
ಸ್ಥಾಪಿ-ಸು-ತ್ತಾನೆ
ಸ್ಥಾಪಿ-ಸು-ವ-ವ-ಳಿ-ದ್ದಾಳೆ
ಸ್ಥಾಪಿ-ಸು-ವು-ದ-ಕ್ಕಾಗಿ
ಸ್ಥಾವ-ರ-ಗ-ಳಿ-ದ್ದವು
ಸ್ಥಾವ-ರ-ವೊಂ-ದನ್ನು
ಸ್ಥಿತಿ
ಸ್ಥಿತಿ-ಇಂ-ತಹ
ಸ್ಥಿತಿ-ಗ-ತಿ-ಗಳ
ಸ್ಥಿತಿ-ಗ-ತಿ-ಗಳು
ಸ್ಥಿತಿಗೆ
ಸ್ಥಿತಿ-ಗೇ-ರಲು
ಸ್ಥಿತಿ-ಗೇ-ರಿ-ದ-ವರು
ಸ್ಥಿತಿ-ಗೇ-ರಿಸಿ
ಸ್ಥಿತಿ-ಗೇ-ರುವ
ಸ್ಥಿತಿ-ಗೇ-ರು-ವಂತೆ
ಸ್ಥಿತಿಯ
ಸ್ಥಿತಿ-ಯನ್ನು
ಸ್ಥಿತಿ-ಯನ್ನೂ
ಸ್ಥಿತಿ-ಯನ್ನೇ
ಸ್ಥಿತಿ-ಯಲ್ಲಿ
ಸ್ಥಿತಿ-ಯ-ಲ್ಲಿ-ದ್ದಾಗ
ಸ್ಥಿತಿ-ಯ-ಲ್ಲಿ-ದ್ದೆನೋ
ಸ್ಥಿತಿ-ಯ-ಲ್ಲಿ-ರ-ಲಿಲ್ಲ
ಸ್ಥಿತಿ-ಯ-ಲ್ಲಿ-ರಿ-ಸುವ
ಸ್ಥಿತಿ-ಯ-ಲ್ಲಿ-ರುವ
ಸ್ಥಿತಿ-ಯ-ಲ್ಲಿ-ರು-ವಂತೆ
ಸ್ಥಿತಿ-ಯ-ಲ್ಲಿ-ರು-ವಾಗ
ಸ್ಥಿತಿ-ಯಲ್ಲೂ
ಸ್ಥಿತಿ-ಯಲ್ಲೇ
ಸ್ಥಿತಿ-ಯಾದ
ಸ್ಥಿತಿ-ಯಿಂದ
ಸ್ಥಿತಿ-ಯಿಂ-ದಲೂ
ಸ್ಥಿತಿ-ಯಿದೆ
ಸ್ಥಿತಿ-ಯೆಂದು
ಸ್ಥಿತಿ-ಯೊಂ-ದನ್ನು
ಸ್ಥಿತಿಯೋ
ಸ್ಥಿಮಿ-ತ-ಬುದ್ಧಿ
ಸ್ಥಿಮಿ-ತ-ವನ್ನು
ಸ್ಥಿರ-ಗ-ತಿ-ಯಿಂದ
ಸ್ಥಿರ-ಗೊ-ಳಿಸ
ಸ್ಥಿರತೆ
ಸ್ಥಿರ-ತೆ-ಯನ್ನು
ಸ್ಥಿರ-ವಾಗಿ
ಸ್ಥಿರ-ವಾ-ಗಿ-ಟ್ಟು-ಕೊ-ಳ್ಳಲು
ಸ್ಥಿರ-ವಾ-ಗಿ-ಬಿ-ಟ್ಟಿತ್ತು
ಸ್ಥಿರ-ವಾ-ಗಿ-ರಿ-ಸಿ-ಕೊಂ-ಡರು
ಸ್ಥೂಲ
ಸ್ಥೂಲ-ಜ-ಗ-ತ್ತನ್ನು
ಸ್ಥೂಲ-ವಾಗಿ
ಸ್ಥೈರ್ಯ
ಸ್ಥೈರ್ಯ-ಗೆ-ಡದೆ
ಸ್ನಾನ
ಸ್ನಾನಕ್ಕೆ
ಸ್ನಾನಾ-ದಿ-ಗ-ಳಿಗೆ
ಸ್ನೇಹ
ಸ್ನೇಹ-ಗೌ-ರ-ವ-ಪೂರ್ಣ
ಸ್ನೇಹ-ಪ್ರೇ-ಮ-ಭ-ಕ್ತಿ-ಗಳನ್ನು
ಸ್ನೇಹ-ಪಾ-ಶ-ವನ್ನು
ಸ್ನೇಹ-ಪೂರ್ಣ
ಸ್ನೇಹ-ಬಾಂ-ಧವ್ಯ
ಸ್ನೇಹಿತ
ಸ್ನೇಹಿ-ತನ
ಸ್ನೇಹಿ-ತ-ನಂತೆ
ಸ್ನೇಹಿ-ತ-ನಾ-ಗಿದ್ದ
ಸ್ನೇಹಿ-ತ-ನಾದ
ಸ್ನೇಹಿ-ತ-ನಿ-ಗಾಗಿ
ಸ್ನೇಹಿ-ತ-ನಿಗೆ
ಸ್ನೇಹಿ-ತನೂ
ಸ್ನೇಹಿ-ತನೇ
ಸ್ನೇಹಿ-ತನೊ
ಸ್ನೇಹಿ-ತ-ನೊಂ-ದಿಗೆ
ಸ್ನೇಹಿ-ತ-ನೊಬ್ಬ
ಸ್ನೇಹಿ-ತ-ನೊ-ಬ್ಬನ
ಸ್ನೇಹಿ-ತ-ನೊ-ಬ್ಬ-ನೊಂ-ದಿಗೆ
ಸ್ನೇಹಿ-ತನೋ
ಸ್ನೇಹಿ-ತರ
ಸ್ನೇಹಿ-ತ-ರ-ನ್ನಾಗಿ
ಸ್ನೇಹಿ-ತ-ರನ್ನು
ಸ್ನೇಹಿ-ತ-ರ-ನ್ನು-ಬಂ-ಧು-ಗಳನ್ನು
ಸ್ನೇಹಿ-ತ-ರ-ನ್ನೆಲ್ಲ
ಸ್ನೇಹಿ-ತ-ರನ್ನೇ
ಸ್ನೇಹಿ-ತ-ರಲ್ಲಿ
ಸ್ನೇಹಿ-ತ-ರಲ್ಲೂ
ಸ್ನೇಹಿ-ತ-ರಲ್ಲೇ
ಸ್ನೇಹಿ-ತ-ರಾಗಿ
ಸ್ನೇಹಿ-ತ-ರಾದ
ಸ್ನೇಹಿ-ತ-ರಿಂ-ದಲೂ
ಸ್ನೇಹಿ-ತ-ರಿಗೆ
ಸ್ನೇಹಿ-ತರು
ಸ್ನೇಹಿ-ತ-ರು-ಹಿ-ತೈ-ಷಿ-ಗಳು
ಸ್ನೇಹಿ-ತರೂ
ಸ್ನೇಹಿ-ತ-ರೆಲ್ಲ
ಸ್ನೇಹಿ-ತ-ರೊಂ-ದಿಗೆ
ಸ್ನೇಹಿ-ತ-ರೊ-ಡನೆ
ಸ್ಪಂದಿಸಿ
ಸ್ಪಂದಿ-ಸುತ್ತ
ಸ್ಪಂದಿ-ಸು-ತ್ತಿತ್ತು
ಸ್ಪರು-ಷ-ಮಣಿ
ಸ್ಪರ್ಧೆ
ಸ್ಪರ್ಶ
ಸ್ಪರ್ಶದ
ಸ್ಪರ್ಶ-ದಿಂದ
ಸ್ಪರ್ಶ-ಮಾ-ತ್ರ-ದಿಂಂದ
ಸ್ಪರ್ಶ-ಮಾ-ತ್ರ-ದಿಂದ
ಸ್ಪರ್ಶ-ಮಾ-ತ್ರ-ದಿಂ-ದಲೇ
ಸ್ಪರ್ಶ-ವನ್ನು
ಸ್ಪರ್ಶಿಸಿ
ಸ್ಪರ್ಶಿ-ಸಿ-ದರು
ಸ್ಪರ್ಶಿ-ಸಿ-ದಾಗ
ಸ್ಪರ್ಶಿ-ಸಿದ್ದೇ
ಸ್ಪರ್ಶಿ-ಸುತ್ತ
ಸ್ಪಲ್ಪ
ಸ್ಪಷ್ಟ
ಸ್ಪಷ್ಟ-ಕ-ಲ್ಪನೆ
ಸ್ಪಷ್ಟ-ದ-ರ್ಶ-ನಾ-ನು-ಭ-ವ-ದಿಂದ
ಸ್ಪಷ್ಟ-ಪ-ಡಿ-ಸು-ತ್ತಿ-ದ್ದಾರೆ
ಸ್ಪಷ್ಟ-ವಾಗಿ
ಸ್ಪಷ್ಟ-ವಾ-ಗಿತ್ತು
ಸ್ಪಷ್ಟ-ವಾ-ಗಿ-ಬಿ-ಟು-ತ್ತದೆ
ಸ್ಪಷ್ಟ-ವಾ-ಗು-ತ್ತದೆ
ಸ್ಪಷ್ಟ-ವಾದ
ಸ್ಪಷ್ಟ-ವಾ-ಯಿತು
ಸ್ಪಿನೋಸಾ
ಸ್ಪೆನ್ಸ-ರನ
ಸ್ಪೆನ್ಸರ್
ಸ್ಫಟಿ-ಕ-ದಷ್ಟು
ಸ್ಫಟಿ-ಕ-ಸ-ದೃಶ
ಸ್ಫುಟ-ವಾಗಿ
ಸ್ಫುರಣೆ
ಸ್ಫುರ-ಣೆಗೆ
ಸ್ಫುರ-ಣೆ-ಯನ್ನು
ಸ್ಫುರ-ಣೆ-ಯಾ-ಗ-ದಿ-ದ್ದಲ್ಲಿ
ಸ್ಫುರ-ಣೆ-ಯಾ-ಗಿ-ಬಿ-ಟ್ಟಿತು
ಸ್ಫುರಿ-ಸುವ
ಸ್ಫೂರ್ತಿ
ಸ್ಫೂರ್ತಿ-ಸಂ-ತೋ-ಷ-ಗಳನ್ನು
ಸ್ಫೂರ್ತಿ-ಗೊಂ-ಡ-ರೆಂ-ದರೆ
ಸ್ಫೂರ್ತಿ-ಗೊಂ-ಡಿದ್ದ
ಸ್ಫೂರ್ತಿ-ಗೊ-ಳಿ-ಸ-ತೊ-ಡ-ಗಿ-ದರು
ಸ್ಫೂರ್ತಿ-ಗೊ-ಳಿ-ಸ-ಲಾ-ರಂ-ಭಿ-ಸಿದ
ಸ್ಫೂರ್ತಿ-ಗೊ-ಳಿ-ಸು-ತ್ತಿದ್ದ
ಸ್ಫೂರ್ತಿಯ
ಸ್ಫೂರ್ತಿ-ಯುಂ-ಟು-ಮಾ-ಡಿದ್ದ
ಸ್ಫೂರ್ತಿ-ಯುತ
ಸ್ಫೂರ್ತಿ-ಯು-ತ-ನಾಗಿ
ಸ್ಫೂರ್ತಿ-ಯು-ತ-ವಾಗಿ
ಸ್ಫೋಟ-ಗೊಂ-ಡಂ-ತಾಗಿ
ಸ್ಫೋಟ-ಗೊ-ಳ್ಳು-ತ್ತಿತ್ತು
ಸ್ಫೋಟಿ-ಸಿತು
ಸ್ಮರ-ಣ-ಮ-ನ-ನ-ದಲ್ಲಿ
ಸ್ಮರ-ಣ-ಶ-ಕ್ತಿಯ
ಸ್ಮರಣೆ
ಸ್ಮರ-ಣೆ-ಯನ್ನು
ಸ್ಮರ-ಣೆ-ಯಲ್ಲಿ
ಸ್ಮರಿ-ಸ-ಬ-ಹು-ದಾ-ಗಿದೆ
ಸ್ಮರಿ-ಸ-ಬ-ಹುದು
ಸ್ಮರಿಸಿ
ಸ್ಮರಿ-ಸಿ-ಕೊಂಡ
ಸ್ಮರಿ-ಸಿ-ಕೊಂಡು
ಸ್ಮಶಾ-ನ-ಕ್ಕೊ-ಯ್ಯಲು
ಸ್ಮಶಾ-ನ-ಗಳನ್ನು
ಸ್ಮಶಾ-ನ-ದಲ್ಲಿ
ಸ್ಮಶಾ-ನ-ವಿತ್ತು
ಸ್ಮಾರಕ
ಸ್ಮಾರ-ಕ-ರೂ-ಪ-ವಾದ
ಸ್ಮಾರ-ಕ-ವನ್ನು
ಸ್ಮಾರ-ಕ-ವೊಂ-ದನ್ನು
ಸ್ಮೃತಿ
ಸ್ಮೃತಿ-ಗಳಲ್ಲಿ
ಸ್ಮೃತಿ-ಚಿ-ತ್ರ-ವನ್ನು
ಸ್ಮೃತಿಯೂ
ಸ್ರೋತ-ವಾ-ಗ-ಲಿವೆ
ಸ್ಲೇಟನ್ನು
ಸ್ವಂತ
ಸ್ವಂತಕ್ಕೆ
ಸ್ವಂತ-ಕ್ಕೋ-ಸ್ಕರ
ಸ್ವಂತದ
ಸ್ವಂತ-ದ-ವರ
ಸ್ವಂತ-ದ-ವ-ರ-ನ್ನಾಗಿ
ಸ್ವಂತ-ದ-ವರು
ಸ್ವಂತದ್ದೇ
ಸ್ವಂತಿಕೆ
ಸ್ವಂತಿ-ಕೆ-ಯಿಂದ
ಸ್ವಗ-ತ-ವೆಂ-ಬಂತೆ
ಸ್ವಜ-ನ-ಪ್ರೇ-ಮದ
ಸ್ವತಂತ್ರ
ಸ್ವತಂ-ತ್ರ-ನಾಗಿ
ಸ್ವತಂ-ತ್ರ-ನಾ-ಗಿಯೇ
ಸ್ವತಂ-ತ್ರ-ನಾದ
ಸ್ವತಂ-ತ್ರ-ಮ-ನೋ-ಭಾ-ವ-ದಿಂದ
ಸ್ವತಂ-ತ್ರ-ರಾ-ಗಿದ್ದು
ಸ್ವತಂ-ತ್ರ-ವಾಗಿ
ಸ್ವತಃ
ಸ್ವತ್ತಾ-ಗು-ವಂತೆ
ಸ್ವತ್ತು
ಸ್ವದೇ-ಶ-ದಲ್ಲೂ
ಸ್ವಪ್ನ-ಗಳನ್ನು
ಸ್ವಪ್ನ-ಗಳನ್ನೆಲ್ಲ
ಸ್ವಪ್ನದ
ಸ್ವಪ್ನ-ದ-ರ್ಶ-ನದ
ಸ್ವಪ್ರ-ಯ-ತ್ನ-ದಿಂದ
ಸ್ವಭಾವ
ಸ್ವಭಾ-ವ-ಸ್ವ-ರೂ-ಪ-ಗಳ
ಸ್ವಭಾ-ವಕ್ಕೆ
ಸ್ವಭಾ-ವ-ಗ-ತ-ವಾ-ಯಿತು
ಸ್ವಭಾ-ವ-ಗಳ
ಸ್ವಭಾ-ವ-ಗ-ಳನ್ನ
ಸ್ವಭಾ-ವ-ಗಳು
ಸ್ವಭಾ-ವತಃ
ಸ್ವಭಾ-ವದ
ಸ್ವಭಾ-ವ-ದಲ್ಲಿ
ಸ್ವಭಾ-ವ-ದ-ಲ್ಲೊಂದು
ಸ್ವಭಾ-ವ-ದ-ವನು
ಸ್ವಭಾ-ವ-ದ-ವ-ರಾ-ಗಿ-ರು-ತ್ತಾರೆ
ಸ್ವಭಾ-ವ-ದ-ವರು
ಸ್ವಭಾ-ವ-ವನ್ನು
ಸ್ವಭಾ-ವ-ವೆಂ-ಥದು
ಸ್ವಭಾ-ವ-ವೆ-ನ್ನು-ವುದು
ಸ್ವಭಾ-ವವೇ
ಸ್ವಭಾ-ವ-ಸ-ಹಜ
ಸ್ವಭಾ-ವ-ಸ-ಹ-ಜ-ವಾ-ಗಿದ್ದ
ಸ್ವಯಂ
ಸ್ವರ
ಸ್ವರ-ದಿಂದ
ಸ್ವರ-ವಿನ್ನೂ
ಸ್ವರ-ವೊಂದು
ಸ್ವರೂಪ
ಸ್ವರೂ-ಪದ
ಸ್ವರೂ-ಪ-ದಲ್ಲಿ
ಸ್ವರೂ-ಪ-ದಿಂದ
ಸ್ವರೂ-ಪ-ನಾದ
ಸ್ವರೂ-ಪರು
ಸ್ವರೂ-ಪ-ರೆಂದು
ಸ್ವರೂ-ಪ-ವ-ನ್ನಾ-ಗಲಿ
ಸ್ವರೂ-ಪ-ವನ್ನು
ಸ್ವರೂ-ಪಿ-ಯೆಂದು
ಸ್ವರ್ಗ-ನ-ರಕ
ಸ್ವರ್ಗ-ದ-ಲ್ಲಾ-ಗಲಿ
ಸ್ವರ್ಗವೇ
ಸ್ವರ್ಣ-ಪ-ದಕ
ಸ್ವರ್ಣ-ಮಯಿ
ಸ್ವರ್ಣ-ಮುಖಿ
ಸ್ವರ್ಶ-ಜ್ಞಾ-ನವೇ
ಸ್ವರ್ಶದ
ಸ್ವರ್ಶ-ದಿಂ-ದೆಲ್ಲ
ಸ್ವರ್ಶ-ವಾ-ಯಿತೋ
ಸ್ವಲ್ಪ
ಸ್ವಲ್ಪ-ದೂರ
ಸ್ವಲ್ಪ-ಮ-ಟ್ಟಿಗೆ
ಸ್ವಲ್ಪ-ಮ-ಟ್ಟಿನ
ಸ್ವಲ್ಪ-ಮಾತ್ರ
ಸ್ವಲ್ಪ-ವನ್ನು
ಸ್ವಲ್ಪ-ವಾ-ದರೂ
ಸ್ವಲ್ಪವೂ
ಸ್ವಲ್ಪವೇ
ಸ್ವಲ್ಪ-ಸ್ವಲ್ಪ
ಸ್ವಲ್ಪ-ಹೊ-ತ್ತಿನ
ಸ್ವಲ್ಪ-ಹೊತ್ತು
ಸ್ವಷ್ಟ
ಸ್ವಷ್ಟ-ವಾಗಿ
ಸ್ವಷ್ಟ-ವಾ-ಗಿದೆ
ಸ್ವಷ್ಟ-ವಾ-ಗು-ತ್ತದೆ
ಸ್ವಸಂ-ತೋ-ಷ-ಕ್ಕಾಗಿ
ಸ್ವಸು-ಖ-ವನ್ನು
ಸ್ವಸ್ಥ
ಸ್ವಸ್ಥ-ಳಕ್ಕೆ
ಸ್ವಸ್ವ-ರೂ-ಪದ
ಸ್ವಸ್ವ-ರೂ-ಪ-ವನ್ನು
ಸ್ವಾಗತ
ಸ್ವಾಗ-ತಿಸಿ
ಸ್ವಾಗ-ತಿ-ಸಿ-ದ-ನ-ಲ್ಲದೆ
ಸ್ವಾಗ-ತಿ-ಸಿ-ದ-ರೆ-ನ್ನ-ಬ-ಹುದು
ಸ್ವಾಗ-ತಿ-ಸು-ತ್ತಿ-ದ್ದಳು
ಸ್ವಾತಂತ್ರ್ಯ
ಸ್ವಾತಂ-ತ್ರ್ಯಕ್ಕೆ
ಸ್ವಾತಂ-ತ್ರ್ಯದ
ಸ್ವಾತಂ-ತ್ರ್ಯ-ದೆ-ಡೆಗೆ
ಸ್ವಾತಂ-ತ್ರ್ಯ-ಪ್ರಿ-ಯರು
ಸ್ವಾತಂ-ತ್ರ್ಯ-ವನ್ನು
ಸ್ವಾತಂ-ತ್ರ್ಯ-ವಿ-ತ್ತಿ-ದ್ದರು
ಸ್ವಾಧೀನ
ಸ್ವಾಧೀ-ನ-ಪ-ಡಿ-ಸಿ-ಕೊಂ-ಡ-ವನೇ
ಸ್ವಾಧೀ-ನ-ಪ-ಡಿ-ಸಿ-ಕೊಂ-ಡಿದ್ದ
ಸ್ವಾಧೀ-ನ-ಪ-ಡಿ-ಸಿ-ಕೊ-ಳ್ಳಲು
ಸ್ವಾಧೀ-ನ-ಪ-ಡಿ-ಸಿ-ಕೊ-ಳ್ಳು-ವ-ವ-ರೆಗೆ
ಸ್ವಾಧೀ-ನ-ಪ-ಡಿ-ಸಿ-ಕೊ-ಳ್ಳು-ವು-ದ-ಕ್ಕಾಗಿ
ಸ್ವಾನು-ಭ-ವದ
ಸ್ವಾಭಾ-ವಿ-ಕ-ವಾ-ಗಿಯೇ
ಸ್ವಾಭಿ
ಸ್ವಾಭಿ-ಮಾನ
ಸ್ವಾಭಿ-ಮಾ-ನದ
ಸ್ವಾಭಿ-ಮಾ-ನ-ಪ್ರಜ್ಞೆ
ಸ್ವಾಭಿ-ಮಾ-ನಿ-ಗ-ಳಾದ
ಸ್ವಾಮಿ
ಸ್ವಾಮಿ-ಗಳ
ಸ್ವಾಮಿ-ಗಳನ್ನು
ಸ್ವಾಮಿ-ಗ-ಳಿಗೆ
ಸ್ವಾಮಿ-ಗ-ಳಿ-ದ್ದಾರೆ
ಸ್ವಾಮಿ-ಗ-ಳಿ-ಬ್ಬರ
ಸ್ವಾಮಿ-ಗ-ಳಿ-ಬ್ಬ-ರಿಗೂ
ಸ್ವಾಮಿ-ಗ-ಳಿ-ಬ್ಬರೂ
ಸ್ವಾಮಿ-ಗಳು
ಸ್ವಾಮಿ-ಗಳೂ
ಸ್ವಾಮಿ-ಗ-ಳೆ-ಲ್ಲ-ರನ್ನೂ
ಸ್ವಾಮಿಜೀ
ಸ್ವಾಮೀ
ಸ್ವಾಮೀಜಿ
ಸ್ವಾಮೀ-ಜಿ-ಗಂತೂ
ಸ್ವಾಮೀ-ಜಿ-ಗಿಂತ
ಸ್ವಾಮೀ-ಜಿಗೂ
ಸ್ವಾಮೀ-ಜಿಗೆ
ಸ್ವಾಮೀ-ಜಿಯ
ಸ್ವಾಮೀ-ಜಿ-ಯ-ದಾ-ಗಿತ್ತು
ಸ್ವಾಮೀ-ಜಿ-ಯ-ದಾ-ಯಿತು
ಸ್ವಾಮೀ-ಜಿ-ಯದೂ
ಸ್ವಾಮೀ-ಜಿ-ಯನ್ನು
ಸ್ವಾಮೀ-ಜಿ-ಯಲ್ಲಿ
ಸ್ವಾಮೀ-ಜಿ-ಯ-ವರ
ಸ್ವಾಮೀ-ಜಿ-ಯ-ವ-ರನ್ನು
ಸ್ವಾಮೀ-ಜಿ-ಯ-ವ-ರಿ-ಗಿತ್ತು
ಸ್ವಾಮೀ-ಜಿ-ಯ-ವರು
ಸ್ವಾಮೀ-ಜಿಯೂ
ಸ್ವಾಮೀ-ಜಿಯೇ
ಸ್ವಾಮೀ-ಜಿ-ಯೊಂ-ದಿ-ಗಿ-ದ್ದರು
ಸ್ವಾಮೀ-ಜಿ-ಯೊಂ-ದಿಗೆ
ಸ್ವಾಮೀಜೀ
ಸ್ವಾಮೀ-ಜೀಯ
ಸ್ವಾಮೀ-ಜೀ-ಯ-ವ-ರಿಗೆ
ಸ್ವಾಮೀ-ಜೀ-ಯ-ವರೇ
ಸ್ವಾರ-ಸ್ಯ-ಕರ
ಸ್ವಾರ-ಸ್ಯ-ಕ-ರ-ವಾಗಿ
ಸ್ವಾರ-ಸ್ಯ-ವನ್ನು
ಸ್ವಾರ-ಸ್ಯ-ವಾಗಿ
ಸ್ವಾರ-ಸ್ಯ-ವೇ-ನಿದೆ
ಸ್ವಾರ್ಥ
ಸ್ವಾರ್ಥತೆ
ಸ್ವಾರ್ಥದ
ಸ್ವಾರ್ಥ-ಬು-ದ್ಧಿ-ಯನ್ನು
ಸ್ವಾರ್ಥ-ರ-ಹಿ-ತ-ವಾದ
ಸ್ವಾರ್ಥ-ವನ್ನು
ಸ್ವಾರ್ಥ-ವೆಲ್ಲಿ
ಸ್ವಾರ್ಥೋ
ಸ್ವಾರ್ಥೋ-ದ್ದೇ-ಶ-ದಿಂದ
ಸ್ವಾರ್ಥೋ-ದ್ದೇ-ಶ-ವೇನೂ
ಸ್ವಾವ-ಲಂಬಿ
ಸ್ವೀಕರಾ
ಸ್ವೀಕ-ರಿ-ಸ-ಬ-ಲ್ಲರು
ಸ್ವೀಕ-ರಿ-ಸ-ಬೇಕು
ಸ್ವೀಕ-ರಿ-ಸ-ಬೇ-ಕೆಂದು
ಸ್ವೀಕ-ರಿ-ಸ-ಲಾರ
ಸ್ವೀಕ-ರಿ-ಸ-ಲಾ-ರಂ-ಭಿ-ಸಿದ
ಸ್ವೀಕ-ರಿ-ಸಲಿ
ಸ್ವೀಕ-ರಿ-ಸಲು
ಸ್ವೀಕ-ರಿಸಿ
ಸ್ವೀಕ-ರಿ-ಸಿದ
ಸ್ವೀಕ-ರಿ-ಸಿ-ದರು
ಸ್ವೀಕ-ರಿ-ಸಿ-ದ-ವರು
ಸ್ವೀಕ-ರಿ-ಸಿ-ದಾಗ
ಸ್ವೀಕ-ರಿ-ಸಿದ್ದ
ಸ್ವೀಕ-ರಿ-ಸಿ-ದ್ದರೂ
ಸ್ವೀಕ-ರಿ-ಸಿ-ದ್ದ-ವರೇ
ಸ್ವೀಕ-ರಿ-ಸಿದ್ದು
ಸ್ವೀಕ-ರಿ-ಸಿ-ಯಾ-ಗಿ-ಬಿ-ಟ್ಟಿದೆ
ಸ್ವೀಕ-ರಿಸು
ಸ್ವೀಕ-ರಿ-ಸುತ್ತ
ಸ್ವೀಕ-ರಿ-ಸು-ತ್ತಾರೆ
ಸ್ವೀಕ-ರಿ-ಸು-ತ್ತಿದ್ದ
ಸ್ವೀಕ-ರಿ-ಸು-ತ್ತಿ-ರ-ಲಿಲ್ಲ
ಸ್ವೀಕ-ರಿ-ಸುವ
ಸ್ವೀಕ-ರಿ-ಸು-ವಂ-ತಾ-ಗಲು
ಸ್ವೀಕ-ರಿ-ಸು-ವಂತೆ
ಸ್ವೀಕ-ರಿ-ಸು-ವ-ವರು
ಸ್ವೀಕ-ರಿ-ಸು-ವುದನ್ನು
ಸ್ವೀಕ-ರಿ-ಸು-ವು-ದರ
ಸ್ವೀಕ-ರಿ-ಸು-ವು-ದ-ರಲ್ಲಿ
ಸ್ವೀಕ-ರಿ-ಸು-ವು-ದ-ರ-ಲ್ಲಿ-ದ್ದೇನೆ
ಸ್ವೀಕ-ರಿ-ಸು-ವುದು
ಸ್ವೀಕ-ರಿ-ಸು-ವು-ದೆಂದು
ಸ್ವೀಕಾರ
ಸ್ವೂಅ-ರ್ಟ್
ಸ್ವೇಚ್ಛಾ-ಚಾರಿ
ಸ್ವೇಚ್ಛೆ
ಸ್ವೇಚ್ಛೆ-ಯಾಗಿ
ಹಂಗನ್ನೂ
ಹಂಗನ್ನೇ
ಹಂಗು
ಹಂಚಿ-ಕೊಂ-ಡರು
ಹಂಚಿ-ಕೊಂಡು
ಹಂಚಿ-ಕೊಂ-ಡೇನು
ಹಂಚಿ-ಕೊ-ಡದೆ
ಹಂಚಿ-ದರು
ಹಂಚಿ-ಬಿ-ಡು-ತ್ತಿ-ದ್ದರು
ಹಂಚು-ತ್ತಿ-ದ್ದಾನೆ
ಹಂತ
ಹಂತ-ದಲ್ಲಿ
ಹಂತ-ವನ್ನು
ಹಂತ-ವೊಂ-ದನ್ನು
ಹಂದ-ರದ
ಹಂದಿ-ಮಾಂಸ
ಹಂಬಲ
ಹಂಬ-ಲದ
ಹಂಬ-ಲ-ವನ್ನು
ಹಂಬ-ಲ-ವ-ಲ್ಲವೆ
ಹಂಬ-ಲ-ವಿ-ರುವ
ಹಂಬ-ಲವು
ಹಂಬ-ಲ-ವುಂಟಾ
ಹಂಬ-ಲ-ವೆಂದರೆ
ಹಂಬ-ಲಿ-ಕೆ-ಯಲ್ಲಿ
ಹಂಬ-ಲಿ-ಕೆ-ಯೆಲ್ಲ
ಹಂಬ-ಲಿ-ಸಿದ
ಹಂಬ-ಲಿ-ಸಿದ್ದ
ಹಂಬ-ಲಿ-ಸು-ತ್ತಿ-ದ್ದರು
ಹಂಬ-ಲಿ-ಸು-ತ್ತಿ-ದ್ದಾನೆ
ಹಂಬ-ಲಿ-ಸು-ತ್ತಿ-ರು-ವಾಗ
ಹಂಬ-ಲಿ-ಸು-ವಂ-ತಹ
ಹಂಬ-ಲಿ-ಸು-ವುದನ್ನು
ಹಕೀ-ಮ-ರನ್ನು
ಹಕೀ-ಮ-ರಿಗೆ
ಹಕ್ಕಿ-ಯಂತೆ
ಹಕ್ಕಿಲ್ಲ
ಹಕ್ಕು-ಗಳನ್ನೂ
ಹಕ್ಕು-ಗ-ಳಿ-ಗಾಗಿ
ಹಗ-ಲಿ-ನಲ್ಲಿ
ಹಗ-ಲಿ-ನಷ್ಟು
ಹಗ-ಲಿ-ರು-ಳು-ಗಳು
ಹಗ-ಲಿ-ರುಳೂ
ಹಗ-ಲಿ-ರು-ಳೆ-ನ್ನದೆ
ಹಗಲು
ಹಗ-ಲು-ಹೊತ್ತು
ಹಗಲೂ
ಹಗ-ಲೂ-ರಾತ್ರಿ
ಹಗ-ಲೆಲ್ಲ
ಹಗುರ
ಹಗು-ರ-ಗೊಂಡ
ಹಗು-ರ-ಮಾ-ಡಿ-ಕೊ-ಳ್ಳುವ
ಹಗು-ರ-ವಾಗಿ
ಹಗು-ರ-ವಾ-ಗು-ವಿಕೆ
ಹಗು-ರ-ವಾದ
ಹಗ್ಗ
ಹಗ್ಗ-ರಾ-ಟೆ-ಗಳ
ಹಗ್ಗದ
ಹಗ್ಗವ
ಹಚ್ಚ
ಹಚ್ಚ-ಲಾ-ಯಿತು
ಹಚ್ಚಲು
ಹಚ್ಚ-ಹ-ಸಿ-ರಾದ
ಹಚ್ಚಿ
ಹಚ್ಚಿ-ಕೊಂಡು
ಹಚ್ಚಿ-ಕೊಂ-ಡು-ಬಿ-ಟ್ಟಿ-ದ್ದೀ-ರಲ್ಲ
ಹಚ್ಚಿ-ಕೊ-ಳ್ಳು-ತ್ತಾರೆ
ಹಚ್ಚಿ-ಕೊ-ಳ್ಳು-ತ್ತಿ-ರ-ಲಿಲ್ಲ
ಹಚ್ಚಿ-ಕೊ-ಳ್ಳುವ
ಹಚ್ಚಿ-ಕೊ-ಳ್ಳು-ವು-ದ-ರಲ್ಲಿ
ಹಚ್ಚಿ-ನೋ-ಡದೆ
ಹಚ್ಚಿ-ಬಿ-ಟ್ಟಿ-ದ್ದಾನೆ
ಹಜಾ-ರ-ಗಳಿಂದ
ಹಜಾ-ರ-ದಲ್ಲಿ
ಹಟ
ಹಟ-ತೊಟ್ಟು
ಹಟ-ಮಾ-ರಿ-ತನ
ಹಟ-ವಾ-ದಿ-ಯಲ್ಲ
ಹಠ
ಹಠ-ಮಾ-ರಿ-ಗಳ
ಹಠ-ಯೋಗ
ಹಠ-ಯೋ-ಗದ
ಹಠ-ಯೋ-ಗಿಯ
ಹಠ-ಯೋ-ಗಿ-ಯೆಂದು
ಹಠ-ಹಿ-ಡಿದ
ಹಠ-ಹಿ-ಡಿದು
ಹಠಾತ್
ಹಠಾ-ತ್ತನೆ
ಹಠಾ-ತ್ತಾಗಿ
ಹಡು-ಗರು
ಹಡುಗು
ಹಡೆದ
ಹಣ
ಹಣ-ಅ-ಧಿ-ಕಾ-ರ-ಗಳ
ಹಣ-ಆ-ಸ್ತಿ-ಗಾಗಿ
ಹಣ-ಹೆ-ಸ-ರು-ಕೀ-ರ್ತಿ-ಗ-ಳೆಂಬ
ಹಣ-ಕ್ಕಾಗಿ
ಹಣ-ತೆಯ
ಹಣದ
ಹಣ-ದೊಂ-ದಿಗೆ
ಹಣ-ಬೇಕು
ಹಣ-ವನ್ನು
ಹಣ-ವನ್ನೂ
ಹಣ-ವ-ನ್ನೆಲ್ಲ
ಹಣ-ವನ್ನೇ
ಹಣ-ವನ್ನೋ
ಹಣ-ವಿ-ದೆಯೆ
ಹಣ-ವಿ-ರ-ಲಿಲ್ಲ
ಹಣ-ವಿಲ್ಲ
ಹಣವೂ
ಹಣೆ
ಹಣೆ-ಗಿ-ಟ್ಟು-ಕೊಂಡ
ಹಣೆಗೆ
ಹಣೆಯ
ಹಣೆ-ಯಲ್ಲಿ
ಹಣೆ-ಯಲ್ಲೂ
ಹಣ್ಣನ್ನು
ಹಣ್ಣು-ಗಳನ್ನು
ಹಣ್ಣು-ಗಾಯಿ
ಹತಾ-ಶ-ನಾ-ಗಿ-ರ-ಲಿಲ್ಲ
ಹತಾ-ಶ-ಳಾ-ದಂತೆ
ಹತೋಟಿ
ಹತೋ-ಟಿಗೆ
ಹತೋ-ಟಿ-ಯ-ಲ್ಲಿ-ಡಲು
ಹತ್ತನ್ನು
ಹತ್ತ-ಬಾ-ರದು
ಹತ್ತ-ಬೇ-ಡವೋ
ಹತ್ತ-ಬೇ-ಡ್ರಪ್ಪ
ಹತ್ತ-ರೊ-ಳಗೆ
ಹತ್ತಲ್ಲ
ಹತ್ತ-ಲ್ಲವೆ
ಹತ್ತಾಗಿ
ಹತ್ತಾ-ಯಿತು
ಹತ್ತಾರು
ಹತ್ತಿ
ಹತ್ತಿ-ಕೊಂ-ಡರೆ
ಹತ್ತಿ-ಕೊಂ-ಡಿತು
ಹತ್ತಿ-ಕ್ಕ-ಲಾ-ದೀತೆ
ಹತ್ತಿ-ಕ್ಕಿ-ಕೊಂ-ಡರು
ಹತ್ತಿ-ತೇನು
ಹತ್ತಿದ
ಹತ್ತಿ-ದ-ವರ
ಹತ್ತಿ-ನಿಂ-ತರು
ಹತ್ತಿರ
ಹತ್ತಿ-ರಕ್ಕೆ
ಹತ್ತಿ-ರದ
ಹತ್ತಿ-ರ-ದಲ್ಲಿ
ಹತ್ತಿ-ರ-ದ-ಲ್ಲಿದ್ದ
ಹತ್ತಿ-ರ-ದ-ಲ್ಲಿ-ದ್ದರೂ
ಹತ್ತಿ-ರ-ದಲ್ಲೇ
ಹತ್ತಿ-ರವೂ
ಹತ್ತಿ-ರವೇ
ಹತ್ತು
ಹತ್ತು-ಹ-ನ್ನೆ-ರಡು
ಹತ್ತು-ಹ-ನ್ನೊಂದು
ಹತ್ತು-ಗಂ-ಟೆಗೆ
ಹತ್ತು-ಪಾಲು
ಹತ್ತು-ವುದು
ಹತ್ತೂ
ಹತ್ತೊಂ-ಬ-ತ್ತ-ನೆಯ
ಹತ್ತೊಂ-ಬ-ತ್ತನೇ
ಹತ್ತೊಂ-ಬತ್ತು
ಹತ್ರಾ-ಸಿಗೆ
ಹತ್ರಾ-ಸಿಗೇ
ಹತ್ರಾ-ಸಿ-ನಲ್ಲಿ
ಹತ್ರಾ-ಸಿ-ನಿಂದ
ಹತ್ರಾಸ್
ಹದಕ್ಕೆ
ಹದ-ಗೆ-ಡುತ್ತ
ಹದ-ಗೊ-ಳ್ಳು-ವಂತೆ
ಹದ-ವ-ರಿತು
ಹದಿ-ನಾ-ರನೇ
ಹದಿ-ನಾ-ರರ
ಹದಿ-ನಾರು
ಹದಿ-ನಾ-ಲ್ಕನೇ
ಹದಿ-ನಾಲ್ಕು
ಹದಿ-ನೆಂಟು
ಹದಿ-ನೆಂ-ಟು
ಹದಿ-ನೆಂ-ಟು-ಹ-ತ್ತೊಂ-ಬ-ತ್ತ-ನೆಯ
ಹದಿ-ನೇ-ಳರ
ಹದಿ-ನೈ-ದ-ನೆಯ
ಹದಿ-ನೈದು
ಹದಿ-ನೈ-ದು-ಇ-ಪ್ಪತ್ತು
ಹದಿ-ಮೂ-ರು-ಹ-ದಿ-ನಾಲ್ಕು
ಹದೀ-ನೇ-ಳ-ನೆಯ
ಹನಿ
ಹನಿ-ಗೂ-ಡಿದ
ಹನು-ಮಂತ
ಹನು-ಮಂ-ತನ
ಹನು-ಮಂ-ತ-ನನ್ನು
ಹನು-ಮಂ-ತನು
ಹನು-ಮಂ-ತ-ನೇಕೋ
ಹನ್ನೆ-ರಡು
ಹನ್ನೆ-ರ-ಡು-ಹ-ದಿ-ನಾಲ್ಕು
ಹನ್ನೊಂ-ದಾ-ಯಿತು
ಹನ್ನೊಂದು
ಹಬ್ಬ
ಹಬ್ಬ-ಉ-ತ್ಸ-ವ-ಗಳ
ಹಬ್ಬದ
ಹಬ್ಬ-ಹ-ರಿ-ದಿ-ನ-ಗಳನ್ನು
ಹಬ್ಬಿ-ಕೊಂಡು
ಹರ-ಕಲು
ಹರಕೆ
ಹರಟಿ
ಹರ-ಟುತ್ತ
ಹರಟೆ
ಹರ-ಟೆ-ಕೊಚ್ಚಿ
ಹರ-ಡ-ಬೇ-ಕಾ-ದ-ವನು
ಹರಡಿ
ಹರ-ಡಿ-ಕೊಂ-ಡಿ-ರುವ
ಹರ-ಡಿತು
ಹರ-ಡಿತ್ತು
ಹರ-ಡಿದ
ಹರ-ಡಿ-ದಾಗ
ಹರ-ಡಿ-ಯಾ-ಗಿದೆ
ಹರ-ಡಿ-ರು-ತ್ತಿದ್ದ
ಹರಡು
ಹರ-ಣೆಗೆ
ಹರ-ಮೋ-ಹಿನಿ
ಹರ-ಳು-ಗಳೊ
ಹರಸಿ
ಹರ-ಸಿದ
ಹರ-ಸಿ-ದರು
ಹರ-ಸು-ತ್ತಿ-ದ್ದರು
ಹರ-ಸು-ತ್ತಿ-ರು-ವಂತೆ
ಹರ-ಹರ
ಹರಿ
ಹರಿ-ಜ-ನರ
ಹರಿತ
ಹರಿ-ತ-ವಾ-ಗಲು
ಹರಿ-ತ-ವಾದ
ಹರಿ-ದಾ-ಡಿ-ಕೊಂಡು
ಹರಿ-ದಾ-ಡುವ
ಹರಿ-ದಾಸ
ಹರಿ-ದಾ-ಸನ
ಹರಿ-ದಾ-ಸ-ನಿಗೆ
ಹರಿ-ದಾಸ್
ಹರಿ-ದಿನ
ಹರಿ-ದಿ-ರು-ತ್ತಿತ್ತು
ಹರಿದು
ಹರಿ-ದು-ಬ-ರ-ಲಿತ್ತು
ಹರಿ-ದು-ಹೋ-ಗ-ಬ-ಹು-ದಾ-ಗಿತ್ತು
ಹರಿ-ದು-ಹೋ-ಗ-ಬ-ಹುದು
ಹರಿ-ದು-ಹೋ-ಗು-ತ್ತಿತ್ತು
ಹರಿ-ದು-ಹೋ-ದಂ-ತಾಗಿ
ಹರಿ-ದ್ವಾ-ರಕ್ಕೆ
ಹರಿ-ದ್ವಾ-ರ-ಕ್ಕೆಂದು
ಹರಿ-ದ್ವಾ-ರದ
ಹರಿ-ನಾಥ
ಹರಿ-ನಾ-ಥನೂ
ಹರಿ-ಪಾಲ
ಹರಿ-ಪ್ರ-ಸನ್ನ
ಹರಿ-ಯದೆ
ಹರಿ-ಯನ್ನು
ಹರಿ-ಯ-ಬೇ-ಕಾ-ದರೆ
ಹರಿ-ಯ-ಲಾ-ರಂ-ಭಿ-ಸಿತು
ಹರಿ-ಯಿತು
ಹರಿ-ಯಿ-ಸಿದ
ಹರಿ-ಯಿ-ಸು-ತ್ತಿದ್ದ
ಹರಿ-ಯಿ-ಸು-ತ್ತಿ-ದ್ದಾಳೆ
ಹರಿ-ಯು-ತ್ತದೆ
ಹರಿ-ಯು-ತ್ತಿತ್ತು
ಹರಿ-ಯು-ತ್ತಿದ್ದ
ಹರಿ-ಯು-ತ್ತಿ-ದ್ದರೆ
ಹರಿ-ಯು-ತ್ತಿ-ದ್ದಾಳೆ
ಹರಿ-ಯು-ತ್ತಿ-ದ್ದು-ದರ
ಹರಿ-ಯು-ತ್ತಿರ
ಹರಿ-ಯು-ತ್ತಿ-ರು-ವುದೋ
ಹರಿ-ಯುವ
ಹರಿ-ಯು-ವಂ-ತಿ-ರು-ತ್ತಿತ್ತು
ಹರಿ-ಯು-ವಂತೆ
ಹರಿ-ಯು-ವು-ದ-ರ-ಲ್ಲಿದೆ
ಹರಿ-ಯೇನೋ
ಹರಿ-ವುದೊ
ಹರಿ-ಸ-ತೊ-ಡ-ಗಿ-ದರು
ಹರಿ-ಸ-ಬಲ್ಲ
ಹರಿ-ಸ-ಬೇ-ಕಾ-ಗಿತ್ತು
ಹರಿಸಿ
ಹರಿ-ಸಿದ
ಹರಿ-ಸಿ-ದರು
ಹರಿ-ಸಿ-ದರೆ
ಹರಿ-ಸಿ-ದ್ದರು
ಹರಿ-ಸುತ್ತ
ಹರಿ-ಸು-ತ್ತಿದ್ದ
ಹರಿ-ಸು-ತ್ತಿ-ರು-ವುದನ್ನು
ಹರಿ-ಸುವ
ಹರಿ-ಹಂ-ಚಿ-ಹೋ-ದರೆ
ಹರೀಶ
ಹರೆ-ಯದ
ಹರ್ಬ-ರ್ಟ್
ಹರ್ಷ-ಚಿ-ತ್ತ-ದಿಂ-ದಿ-ರಲು
ಹಲ-ವರೆಲ್ಲ
ಹಲ-ವಾರ
ಹಲ-ವಾರು
ಹಲ-ವಿ-ಧ-ವಾಗಿ
ಹಲವು
ಹಲ-ವು-ಹತ್ತು
ಹಲ್ಲಿ-ಗಳ
ಹಲ್ಲು
ಹಲ್ಲು-ಕಿ-ರಿ-ಯುತ್ತ
ಹಳ-ತಾ-ಗದ
ಹಳ-ಬ-ರು-ಹೊ-ಸ-ಬರು
ಹಳಿ-ದನೋ
ಹಳೆಯ
ಹಳ್ಳಿ
ಹಳ್ಳಿ-ಗಳ
ಹಳ್ಳಿಗೆ
ಹಳ್ಳಿಯ
ಹಳ್ಳಿ-ಯಲ್ಲಿ
ಹಳ್ಳಿ-ಯ-ಲ್ಲಿ-ದ್ದರು
ಹಳ್ಳಿ-ಯೊ-ಳಗೆ
ಹವ-ಣಿ-ಸಿ-ದರೆ
ಹವ-ನ-ಹೋ-ಮ-ಗಳನ್ನೂ
ಹವಿ-ಷ್ಯಾ-ನ್ನಕ್ಕೆ
ಹವೆ
ಹವ್ಯಾ-ಸ-ವನ್ನು
ಹಸಿ-ದರೂ
ಹಸಿ-ದಿ-ದೆಯೋ
ಹಸಿ-ದಿ-ದ್ದಾನೆ
ಹಸಿ-ದಿ-ದ್ದೀರಿ
ಹಸಿದು
ಹಸಿಯೊ
ಹಸಿ-ರಾದ
ಹಸಿರು
ಹಸಿ-ವಾ-ಗಿದೆ
ಹಸಿ-ವಿನ
ಹಸಿ-ವಿ-ನಿಂದ
ಹಸಿವು
ಹಸಿ-ವು-ಬಾ-ಯಾ-ರಿ-ಕೆ-ಗಳಿಂದ
ಹಸಿ-ವು-ಬಾ-ಯಾ-ರಿ-ಕೆಯ
ಹಸಿ-ವೆ-ಯನ್ನೇ
ಹಸಿ-ವೆ-ಯಿಂದ
ಹಸುರೆ
ಹಸುಳೆ
ಹಸು-ಳೆಯ
ಹಸು-ವಿತ್ತು
ಹಸು-ವಿನ
ಹಸ್ತ-ಸಾ-ಮು-ದ್ರಿಕ
ಹಾಂ
ಹಾಕ-ದಿ-ದ್ದರೂ
ಹಾಕ-ದಿ-ದ್ದರೆ
ಹಾಕ-ದಿ-ದ್ದು-ದಕ್ಕೆ
ಹಾಕ-ದಿ-ರು-ವ-ವರ
ಹಾಕದೆ
ಹಾಕ-ಬ-ಲ್ಲ-ವ-ರನ್ನು
ಹಾಕ-ಬೇ-ಕಾ-ಗುತ್ತೆ
ಹಾಕ-ಲಾ-ಗು-ತ್ತದೆ
ಹಾಕ-ಲಾ-ರಂ-ಭಿ-ಸಿದ
ಹಾಕ-ಲಿಲ್ಲ
ಹಾಕಲು
ಹಾಕ-ಲೇನೂ
ಹಾಕಿ
ಹಾಕಿ-ಕೊಂಡ
ಹಾಕಿ-ಕೊಂ-ಡಿತ್ತು
ಹಾಕಿ-ಕೊಂ-ಡಿದ್ದ
ಹಾಕಿ-ಕೊಂ-ಡಿ-ರು-ತ್ತಿ-ರ-ಲಿಲ್ಲ
ಹಾಕಿ-ಕೊಂಡು
ಹಾಕಿ-ಕೊ-ಟ್ಟರು
ಹಾಕಿ-ಕೊ-ಳ್ಳದೆ
ಹಾಕಿ-ಕೊ-ಳ್ಳಲೂ
ಹಾಕಿ-ಕೊ-ಳ್ಳು-ತ್ತಿದ್ದ
ಹಾಕಿ-ಟ್ಟರೂ
ಹಾಕಿ-ಟ್ಟು-ಕೊಂ-ಡರು
ಹಾಕಿಡ
ಹಾಕಿತ್ತು
ಹಾಕಿದ
ಹಾಕಿ-ದಂತೆ
ಹಾಕಿ-ದಂ-ತೆಯೇ
ಹಾಕಿ-ದರು
ಹಾಕಿ-ದ-ವರು
ಹಾಕಿದೆ
ಹಾಕಿದ್ದ
ಹಾಕಿ-ದ್ದರು
ಹಾಕಿ-ದ್ದೀರೋ
ಹಾಕಿ-ಬಿಟ್ಟ
ಹಾಕಿ-ಬಿ-ಟ್ಟರು
ಹಾಕಿ-ಬಿ-ಟ್ಟಿದ್ದ
ಹಾಕು
ಹಾಕುತ್ತ
ಹಾಕು-ತ್ತಿದ್ದ
ಹಾಕು-ತ್ತಿ-ದ್ದರು
ಹಾಕು-ತ್ತೀರೋ
ಹಾಕು-ವ-ವ-ರಲ್ಲ
ಹಾಕುವು
ಹಾಕು-ವುದು
ಹಾಗಯೇ
ಹಾಗಲ್ಲ
ಹಾಗ-ಲ್ಲ-ದಿ-ದ್ದರೆ
ಹಾಗ-ಲ್ಲದೆ
ಹಾಗಾ-ಗ-ದಂತೆ
ಹಾಗಾ-ಗ-ಲಿಲ್ಲ
ಹಾಗಾಗಿ
ಹಾಗಾ-ಗುತ್ತೀ
ಹಾಗಾ-ಗು-ವು-ದಕ್ಕೆ
ಹಾಗಾ-ಗು-ವು-ದಿಲ್ಲ
ಹಾಗಾ-ದರೆ
ಹಾಗಾ-ದಾಗ
ಹಾಗಾ-ಯಿತು
ಹಾಗಿತ್ತು
ಹಾಗಿದ
ಹಾಗಿ-ದ್ದರೆ
ಹಾಗಿ-ರಲಿ
ಹಾಗಿ-ರು-ತ್ತದೆ
ಹಾಗಿ-ರು-ತ್ತೇನೆ
ಹಾಗಿ-ರು-ವಾಗ
ಹಾಗೂ
ಹಾಗೆ
ಹಾಗೆಂದ
ಹಾಗೆಂ-ದರೆ
ಹಾಗೆಂ-ದಿ-ದ್ದರು
ಹಾಗೆಂದು
ಹಾಗೆಂದೂ
ಹಾಗೆಂದೆ
ಹಾಗೆಯೇ
ಹಾಗೆಲ್ಲ
ಹಾಗೆ-ಹಾ-ಗೆಯೇ
ಹಾಗೇ
ಹಾಗೇಕೆ
ಹಾಗೇ-ನಾ-ದರೂ
ಹಾಗೇ-ನಿಲ್ಲ
ಹಾಜರಾ
ಹಾಜ-ರಾನ
ಹಾಜ-ರಾ-ನಂತೂ
ಹಾಜ-ರಾ-ನದು
ಹಾಜ-ರಾ-ನಿಗೆ
ಹಾಜ-ರಾ-ನೊಂ-ದಿಗೆ
ಹಾಜರಿ
ಹಾಡ
ಹಾಡ-ತೊ-ಡ-ಗಿದ
ಹಾಡ-ತೊ-ಡ-ಗಿ-ದ-ನೆಂ-ದರೆ
ಹಾಡನು
ಹಾಡ-ನು-ಲಿಯೈ
ಹಾಡ-ನ್ನಂತೂ
ಹಾಡನ್ನು
ಹಾಡ-ಬ-ಗುದು
ಹಾಡ-ಬ-ಲ್ಲರು
ಹಾಡ-ಬ-ಲ್ಲ-ವ-ನಾ-ಗಿದ್ದ
ಹಾಡ-ಬೇಕು
ಹಾಡಲಾ
ಹಾಡ-ಲಾ-ರಂ-ಭಿ-ಸಿದ
ಹಾಡ-ಲಿಲ್ಲ
ಹಾಡಲು
ಹಾಡಾದ
ಹಾಡಿ
ಹಾಡಿ-ಕೊಂಡು
ಹಾಡಿ-ಕೊ-ಳ್ಳು-ತ್ತಿದ್ದ
ಹಾಡಿ-ಕೊ-ಳ್ಳು-ತ್ತಿ-ದ್ದಳು
ಹಾಡಿದ
ಹಾಡಿ-ದರು
ಹಾಡಿ-ದಾಗ
ಹಾಡಿದೆ
ಹಾಡಿನ
ಹಾಡಿ-ನಲ್ಲಿ
ಹಾಡಿ-ನಿಂದ
ಹಾಡಿ-ನಿಂ-ದೆ-ಚ್ಚ-ರಿಸು
ಹಾಡಿರಿ
ಹಾಡು
ಹಾಡು-ಗಳನ್ನು
ಹಾಡು-ಗ-ಳೆಲ್ಲ
ಹಾಡು-ಗಾರ
ಹಾಡು-ಗಾ-ರನ
ಹಾಡು-ಗಾ-ರ-ನಾಗಿ
ಹಾಡು-ಗಾ-ರರು
ಹಾಡು-ಗಾ-ರಿ-ಕೆ-ಯನ್ನು
ಹಾಡು-ಗಾ-ರಿ-ಕೆ-ಯ-ಲ್ಲ-ಲ್ಲದೆ
ಹಾಡು-ಗಾ-ರಿ-ಕೆ-ಯಲ್ಲಿ
ಹಾಡು-ಗಾ-ರಿ-ಕೆ-ಯಲ್ಲೇ
ಹಾಡು-ಗಾ-ರಿ-ಕೆಯೇ
ಹಾಡುತ್ತ
ಹಾಡು-ತ್ತಲೇ
ಹಾಡುತ್ತಾ
ಹಾಡು-ತ್ತಾನೆ
ಹಾಡು-ತ್ತಾರೆ
ಹಾಡು-ತ್ತಿದ್ದ
ಹಾಡು-ತ್ತಿ-ದ್ದ-ಒ-ಮ್ಮೊಮ್ಮೆ
ಹಾಡು-ತ್ತಿ-ದ್ದರು
ಹಾಡು-ತ್ತಿ-ದ್ದರೆ
ಹಾಡು-ತ್ತಿ-ದ್ದ-ವನು
ಹಾಡು-ವ-ವ-ರಲ್ಲ
ಹಾಡು-ವ-ವರೇ
ಹಾಡು-ವು-ದ-ಕ್ಕಾಗಿ
ಹಾಡು-ವುದನ್ನು
ಹಾಡುವೆ
ಹಾಡೆಂ-ಬು-ದು-ಓಂ-ಕಾ-ರ-ವೆಂ-ಬು-ದು-ಭೋ-ಗದ
ಹಾಡೈ
ಹಾಡೊಂ-ದನ್ನು
ಹಾತೊ-ರೆದ
ಹಾತೊ-ರೆ-ಯ-ಲಾ-ರಂ-ಭಿ-ಸಿದ
ಹಾತೊ-ರೆ-ಯು-ತ್ತಿ-ದ್ದೀ-ಯಲ್ಲ
ಹಾದಿ-ಯಲ್ಲಿ
ಹಾದು
ಹಾದು-ಹೋ-ಗು-ತ್ತಿತ್ತು
ಹಾದು-ಹೋದ
ಹಾದು-ಹೋ-ದ-ವ-ರ-ಲ್ಲವೆ
ಹಾದು-ಹೋ-ದವು
ಹಾದು-ಹೋ-ದುವು
ಹಾನಿ
ಹಾನಿ-ಕಾ-ರಕ
ಹಾನಿ-ಯಾ-ಗ-ಲಿಲ್ಲ
ಹಾಯಾಗಿ
ಹಾಯಾ-ಗಿ-ರ-ಬ-ಹು-ದಿ-ತ್ತಲ್ಲ
ಹಾಯಾ-ಗಿ-ರು-ವುದೇ
ಹಾಯಿ-ಸಿ-ದರು
ಹಾರ
ಹಾರ-ಗಳಿಂದ
ಹಾರಾ-ಟ-ಗ-ಳೆಲ್ಲ
ಹಾರಾ-ಡುತ್ತ
ಹಾರಾ-ಡು-ತ್ತಿ-ರು-ತ್ತದೆ
ಹಾರಾ-ಡುವ
ಹಾರಿ
ಹಾರಿದ
ಹಾರಿ-ದರು
ಹಾರಿಯೇ
ಹಾರಿಸಿ
ಹಾರಿ-ಸುತ್ತ
ಹಾರಿ-ಹೋ-ಗಲು
ಹಾರಿ-ಹೋಗಿ
ಹಾರಿ-ಹೋ-ಗಿ-ಬಿ-ಡ-ಬೇಕು
ಹಾರಿ-ಹೋ-ಗಿರ
ಹಾರಿ-ಹೋ-ಗು-ತ್ತದೆ
ಹಾರು-ತ್ತದೆ
ಹಾರು-ತ್ತಿ-ರುವ
ಹಾರುವ
ಹಾರು-ವುದು
ಹಾರೈಕೆ
ಹಾರೈಸಿ
ಹಾರೈ-ಸು-ತ್ತೇನೆ
ಹಾಲನ್ನು
ಹಾಳಾಗಿ
ಹಾಳಾ-ಗಿ-ರು-ವುದೇ
ಹಾಳಾ-ಗಿ-ಹೋ-ಗು-ತ್ತಿತ್ತು
ಹಾಳಾ-ಗಿ-ಹೋ-ದೇನು
ಹಾಳಾ-ಗುವ
ಹಾಳು-ಬಿದ್ದ
ಹಾಳು-ಮಾ-ಡು-ತ್ತಿ-ರು-ವ-ವನು
ಹಾವು
ಹಾಸಿ-ಕೊಂ-ಡರೆ
ಹಾಸಿಗೆ
ಹಾಸಿ-ಗೆಯ
ಹಾಸಿ-ಗೆ-ಯನ್ನು
ಹಾಸಿ-ಗೆ-ಯಾ-ಗಿ-ರ-ಲಿಲ್ಲ
ಹಾಸಿದ
ಹಾಸುಗೆ
ಹಾಸ್ಯ-ಕ್ಕೊಂದು
ಹಾಸ್ಯ-ಚ-ಟಾ-ಕಿ-ಗಳನ್ನು
ಹಾಸ್ಯದ
ಹಾಸ್ಯ-ಪ್ರ-ವೃತ್ತಿ
ಹಾಸ್ಯ-ಪ್ರ-ವೃ-ತ್ತಿ-ಯೇನೂ
ಹಾಸ್ಯ-ಪ್ರಿ-ಯ-ನೆಂ-ದರೆ
ಹಾಸ್ಯ-ಮಯ
ಹಾಸ್ಯಾ-ಸ್ಪ-ದ-ವಾ-ಗು-ವ-ಷ್ಟ-ರ-ವ-ರೆಗೆ
ಹಾಸ್ಯಾ-ಸ್ಪ-ದ-ವಾ-ದ-ದ್ದೇ-ನಾ-ದರೂ
ಹಿ
ಹಿಂಜ-ರಿದು
ಹಿಂಜ-ರಿ-ಯ-ದಿರು
ಹಿಂಜ-ರಿ-ಯದೆ
ಹಿಂಜ-ರಿ-ಯ-ಲಾ-ರಂ-ಭಿಸಿ
ಹಿಂಜ-ರಿ-ಯುತ್ತ
ಹಿಂಜ-ರಿ-ಯು-ತ್ತಿ-ದ್ದರೂ
ಹಿಂಡಿ
ಹಿಂಡಿದ
ಹಿಂಡಿ-ದಂ-ತಾ-ಗು-ತ್ತದೆ
ಹಿಂಡಿ-ದರೆ
ಹಿಂಡು-ತ್ತಿ-ದ್ದಾರೋ
ಹಿಂಡುವ
ಹಿಂಡು-ವುದು
ಹಿಂಡೊಂದು
ಹಿಂದಕೆ
ಹಿಂದಕ್ಕೆ
ಹಿಂದಿ
ಹಿಂದಿಈ
ಹಿಂದಿ-ಗಿಂತ
ಹಿಂದಿ-ಗಿಂ-ತಲೂ
ಹಿಂದಿನ
ಹಿಂದಿ-ನಂ-ತೆಯೇ
ಹಿಂದಿ-ಯಲ್ಲಿ
ಹಿಂದಿರು
ಹಿಂದಿ-ರುಗ
ಹಿಂದಿ-ರು-ಗ-ದಂತೆ
ಹಿಂದಿ-ರು-ಗದೆ
ಹಿಂದಿ-ರು-ಗ-ಬೇ-ಕಾ-ಗು-ತ್ತಿತ್ತು
ಹಿಂದಿ-ರು-ಗ-ಬೇ-ಕಾ-ಯಿತು
ಹಿಂದಿ-ರು-ಗ-ಲಾ-ರೆ-ನೆಂದು
ಹಿಂದಿ-ರು-ಗ-ಲಿಲ್ಲ
ಹಿಂದಿ-ರು-ಗಲು
ಹಿಂದಿ-ರು-ಗ-ಲೇ-ಬೇಕು
ಹಿಂದಿ-ರುಗಿ
ಹಿಂದಿ-ರು-ಗಿದ
ಹಿಂದಿ-ರು-ಗಿ-ದ-ಮೇ-ಲಷ್ಟೇ
ಹಿಂದಿ-ರು-ಗಿ-ದರು
ಹಿಂದಿ-ರು-ಗಿ-ದಾಗ
ಹಿಂದಿ-ರು-ಗಿ-ದೆ-ನೋ-ಡೋಣ
ಹಿಂದಿ-ರು-ಗಿ-ದ್ದ-ರಾ-ದರೂ
ಹಿಂದಿ-ರು-ಗಿ-ದ್ದರು
ಹಿಂದಿ-ರು-ಗಿದ್ದು
ಹಿಂದಿ-ರು-ಗಿ-ಬಂ-ದರು
ಹಿಂದಿ-ರು-ಗಿ-ಬಂ-ದಾಗ
ಹಿಂದಿ-ರು-ಗಿ-ಬಾ-ರ-ದಿ-ದ್ದು-ದನ್ನು
ಹಿಂದಿ-ರು-ಗಿ-ರು-ವ-ವರೂ
ಹಿಂದಿ-ರು-ಗಿಸು
ಹಿಂದಿ-ರು-ಗಿ-ಸು-ವಂತೆ
ಹಿಂದಿ-ರು-ಗಿ-ಸು-ವು-ದಿಲ್ಲ
ಹಿಂದಿ-ರು-ಗು-ತ್ತ-ದೆಯೋ
ಹಿಂದಿ-ರು-ಗುವ
ಹಿಂದಿ-ರು-ಗು-ವಂತೆ
ಹಿಂದಿ-ರು-ಗು-ವ-ವ-ರೆಗೆ
ಹಿಂದಿ-ರು-ಗು-ವಾಗ
ಹಿಂದಿ-ರು-ಗು-ವು-ದ-ರೊ-ಳಗೆ
ಹಿಂದಿ-ರು-ಗು-ವು-ದಾಗಿ
ಹಿಂದಿ-ರು-ಗು-ವು-ದಿಲ್ಲ
ಹಿಂದಿ-ರು-ಗು-ವುದು
ಹಿಂದಿ-ರುವ
ಹಿಂದೀ
ಹಿಂದು-ಗಳ
ಹಿಂದು-ಮುಂದು
ಹಿಂದು-ರಿಗಿ
ಹಿಂದು-ರು-ಗ-ಗೊ-ಡದೆ
ಹಿಂದು-ಳಿದ
ಹಿಂದು-ವಾದ
ಹಿಂದೂ
ಹಿಂದೂ-ಇ-ಸ್ಲಾಂ
ಹಿಂದೂ-ಮು-ಸ-ಲ್ಮಾ-ನ-ಕ್ರೈ-ಸ್ತ-ರಿಗೂ
ಹಿಂದೂ-ಗಳ
ಹಿಂದೂ-ಗಳನ್ನು
ಹಿಂದೂ-ಗಳಲ್ಲಿ
ಹಿಂದೂ-ಗ-ಳು-ಇ-ವರೆಲ್ಲ
ಹಿಂದೂ-ಧರ್ಮ
ಹಿಂದೂ-ಧ-ರ್ಮಕ್ಕೆ
ಹಿಂದೂ-ಧ-ರ್ಮದ
ಹಿಂದೂ-ಧ-ರ್ಮ-ದಲ್ಲಿ
ಹಿಂದೂ-ಧ-ರ್ಮ-ದಿಂದ
ಹಿಂದೂ-ಧ-ರ್ಮ-ವನ್ನು
ಹಿಂದೂ-ಧ-ರ್ಮ-ವೆಂದರೆ
ಹಿಂದೂ-ಧ-ರ್ಮ-ಸಾ-ರವೇ
ಹಿಂದೂ-ರಾ-ಷ್ಟ್ರದ
ಹಿಂದೂ-ಸಂ-ಪ್ರ-ದಾ-ಯ-ಗಳಲ್ಲಿ
ಹಿಂದೂ-ಸ-ಮಾ-ಜವು
ಹಿಂದೂ-ಸ್ಥಾ-ನದ
ಹಿಂದೂ-ಸ್ಥಾ-ನ-ದಲ್ಲೆಲ್ಲ
ಹಿಂದೆ
ಹಿಂದೆಂ-ದಿ-ಗಿಂತ
ಹಿಂದೆಂದೂ
ಹಿಂದೆ-ಗೆ-ದು-ಕೊಂಡು
ಹಿಂದೆ-ಗೆ-ದು-ಕೊಂ-ಡು-ಬಿ-ಡು-ತ್ತಾಳೆ
ಹಿಂದೆ-ಮುಂದೆ
ಹಿಂದೆ-ಯಷ್ಟೆ
ಹಿಂದೆಯೂ
ಹಿಂದೆಯೇ
ಹಿಂದೆ-ಳೆ-ದು-ಕೊಂಡು
ಹಿಂದೇಟು
ಹಿಂದೊಂದು
ಹಿಂದೊಮ್ಮೆ
ಹಿಂಬಾ-ಗಿಲ
ಹಿಂಬಾ-ಲಿ-ಸ-ಕೂ-ಡದು
ಹಿಂಬಾ-ಲಿಸಿ
ಹಿಂಬಾ-ಲಿ-ಸು-ತ್ತದೆ
ಹಿಂಬಾ-ಲಿ-ಸುತ್ತಿ
ಹಿಂಬಾ-ಲಿ-ಸುವ
ಹಿಂಭಾ-ಗ-ದಲ್ಲಿ
ಹಿಂಭಾ-ಗ-ದ-ಲ್ಲೊಂದು
ಹಿಂಸೆ-ಯಾ-ಗದೆ
ಹಿಗ್ಗಿಸಿ
ಹಿಗ್ಗಿ-ಹೋ-ದನೆ
ಹಿಗ್ಗು
ಹಿಗ್ಗು-ತ್ತಿತ್ತು
ಹಿಚ್ಚು-ಗಾ-ರಿ-ಕೆ-ಯಿಲ್ಲ
ಹಿಡ-ಕೊಂಡು
ಹಿಡ-ತಕ್ಕೂ
ಹಿಡಿತ
ಹಿಡಿ-ತಕ್ಕೆ
ಹಿಡಿ-ತ-ದಿಂದ
ಹಿಡಿದ
ಹಿಡಿ-ದಂ-ತಾ-ಗಿ-ದ್ದ-ರಿಂದ
ಹಿಡಿ-ದರು
ಹಿಡಿ-ದರೆ
ಹಿಡಿ-ದಿ-ಟ್ಟು-ಕೊಂ-ಡು-ಬಿ-ಟ್ಟಿ-ದ್ದರು
ಹಿಡಿ-ದಿ-ಟ್ಟು-ಕೊ-ಳ್ಳ-ಬೇ-ಕಾ-ದರೆ
ಹಿಡಿ-ದಿ-ಡ-ಲಾ-ರವು
ಹಿಡಿ-ದಿ-ದೆ-ಯೆಂದೇ
ಹಿಡಿ-ದಿದ್ದ
ಹಿಡಿದು
ಹಿಡಿ-ದು-ಕೊಂ-ಡರೆ
ಹಿಡಿ-ದು-ಕೊಂ-ಡಳು
ಹಿಡಿ-ದು-ಕೊಂ-ಡಿತ್ತು
ಹಿಡಿ-ದು-ಕೊಂ-ಡಿ-ದ್ದ-ಳೆಂ-ದರೆ
ಹಿಡಿ-ದು-ಕೊಂಡು
ಹಿಡಿ-ದು-ಕೊ-ಳ್ಳುತ್ತೆ
ಹಿಡಿ-ಮೆ-ಣ-ಸನ್ನು
ಹಿಡಿ-ಯ-ಬೇ-ಕಾಗಿ
ಹಿಡಿ-ಯ-ಬೇ-ಕಾ-ಯಿತು
ಹಿಡಿ-ಯ-ಲಿ-ಲ್ಲವೆ
ಹಿಡಿ-ಯಲು
ಹಿಡಿ-ಯಿತು
ಹಿಡಿಯು
ಹಿಡಿ-ಯುತ್ತ
ಹಿಡಿ-ಯು-ವಂ-ಥವು
ಹಿಡಿ-ಯು-ವ-ವರು
ಹಿಡಿ-ಯೋಣ
ಹಿಡಿಸ
ಹಿಡಿಸಿ
ಹಿಡಿ-ಸಿ-ರ-ಲಿಲ್ಲ
ಹಿಡಿ-ಸು-ತ್ತಿಲ್ಲ
ಹಿಡೀರಿ
ಹಿಡು-ದು-ಕೊಂಡು
ಹಿತ
ಹಿತ-ಕ್ಕಾಗಿ
ಹಿತ-ಕ್ಕಾ-ಗಿಯೇ
ಹಿತರು
ಹಿತ-ವಾದ
ಹಿತ-ವೆ-ನಿ-ಸಿತು
ಹಿತ-ಸಾ-ಧನೆ
ಹಿತೈ-ಷಿ-ಗಳು
ಹಿತ್ತಲ
ಹಿನ್ನಲೆ
ಹಿನ್ನ-ಲೆ-ಯಲ್ಲಿ
ಹಿನ್ನೆಲೆ
ಹಿನ್ನೆ-ಲೆ-ಗಳು
ಹಿನ್ನೆ-ಲೆ-ಯನ್ನು
ಹಿನ್ನೆ-ಲೆ-ಯಲ್ಲಿ
ಹಿನ್ನೆ-ಲೆ-ಯ-ಲ್ಲಿ-ರುವ
ಹಿನ್ನೆ-ಲೆ-ಯಲ್ಲೂ
ಹಿಮ-ರಾ-ಶಿ-ಇ-ವು-ಗ-ಳೆಲ್ಲ
ಹಿಮ-ವಂ-ತನ
ಹಿಮಾ
ಹಿಮಾ-ಚ್ಛಾ-ದಿತ
ಹಿಮಾಲ
ಹಿಮಾ-ಲಯ
ಹಿಮಾ-ಲ-ಯಕ್ಕೆ
ಹಿಮಾ-ಲ-ಯದ
ಹಿಮಾ-ಲ-ಯ-ದತ್ತ
ಹಿಮಾ-ಲ-ಯ-ದಲ್ಲಿ
ಹಿಮಾ-ಲ-ಯ-ದ-ಲ್ಲಿ-ರುವ
ಹಿಮಾ-ಲ-ಯ-ದಲ್ಲೇ
ಹಿಮಾ-ಲ-ಯ-ವನ್ನು
ಹಿಮ್ಮೆ-ಟ್ಟಿ-ದು-ವೆಂದರೆ
ಹಿರಿ-ದಾ-ದವು
ಹಿರಿ-ಮೆ-ಮ-ಹಿ-ಮೆ-ಗಳನ್ನು
ಹಿರಿ-ಮೆಯ
ಹಿರಿ-ಮೆ-ಯನ್ನು
ಹಿರಿಯ
ಹಿರಿ-ಯ-ಗೋ-ಪಾಲ
ಹಿರಿ-ಯಣ್ಣ
ಹಿರಿ-ಯ-ನಾದ
ಹಿರಿ-ಯನೂ
ಹಿರಿ-ಯರ
ಹಿರಿ-ಯ-ರನ್ನು
ಹಿರಿ-ಯ-ರಾ-ದ-ವ-ರೊ-ಡನೆ
ಹಿರಿ-ಯ-ರಿಂದ
ಹಿರಿ-ಯರು
ಹಿಸು-ಕಿ-ಹಾ-ಕಿ-ಬಿ-ಡು-ತ್ತದೆ
ಹಿಸು-ಕಿ-ಹಾ-ಕಿ-ಬಿ-ಡುತ್ತೆ
ಹೀ
ಹೀಗ-ಳೆಯು
ಹೀಗ-ಳೆ-ಯು-ತ್ತಿದ್ದ
ಹೀಗ-ಳೆ-ಯು-ತ್ತಿ-ದ್ದ-ವನು
ಹೀಗಾ-ಗಿ-ಬಿ-ಟ್ಟಿದೆ
ಹೀಗಾ-ದರೂ
ಹೀಗಿದೆ
ಹೀಗಿದ್ದ
ಹೀಗಿ-ದ್ದರೂ
ಹೀಗಿ-ದ್ದರೆ
ಹೀಗಿ-ರು-ವಾಗ
ಹೀಗಿವೆ
ಹೀಗೂ
ಹೀಗೆ
ಹೀಗೆಂದ
ಹೀಗೆಂ-ದ-ದ್ದನ್ನು
ಹೀಗೆಂ-ದದ್ದೇ
ಹೀಗೆಂ-ದರು
ಹೀಗೆಂ-ದಾಗ
ಹೀಗೆಂದು
ಹೀಗೆ-ನ್ನುತ್ತ
ಹೀಗೆ-ನ್ನು-ತ್ತಾರೆ
ಹೀಗೆ-ನ್ನು-ತ್ತಿದ್ದೀ
ಹೀಗೆಯೇ
ಹೀಗೆಲ್ಲ
ಹೀಗೇ
ಹೀಗೋ
ಹೀಗ್
ಹೀನ
ಹೀಯಾ-ಳಿ-ಸ-ಬ-ಹುದೆ
ಹೀಯಾ-ಳಿಸು
ಹೀರಲು
ಹೀರಿ-ಬಿ-ಟ್ಟಿದೆ
ಹೀರು-ತ್ತ-ದೆಯೋ
ಹೀರು-ತ್ತೀಯೆ
ಹೀರು-ತ್ತೇನೆ
ಹೀಹೆ
ಹುಂಕಾರ
ಹುಕ್ಕ
ಹುಕ್ಕದ
ಹುಕ್ಕಾ
ಹುಕ್ಕಾ-ಕೊ-ಳ-ವೆ-ಗಳನ್ನೂ
ಹುಕ್ಕಾ-ವನ್ನು
ಹುಚ್ಚ
ಹುಚ್ಚ-ನಂತೆ
ಹುಚ್ಚ-ನ-ನ್ನಲ್ಲ
ಹುಚ್ಚ-ನಿ-ರ-ಬೇಕು
ಹುಚ್ಚ-ನೆಂದು
ಹುಚ್ಚ-ರಂ-ತಾ-ಗಿ-ಬಿ-ಟ್ಟಿ-ದ್ದ-ರು-ಭ-ಗ-ವಂ-ತನ
ಹುಚ್ಚ-ರಂತೂ
ಹುಚ್ಚ-ರಾ-ಗು-ತ್ತಾರೆ
ಹುಚ್ಚ-ರೆಂದು
ಹುಚ್ಚರೇ
ಹುಚ್ಚು
ಹುಚ್ಚು-ಕ-ಲ್ಪ-ನೆಯೋ
ಹುಚ್ಚುಚ್ಚಾ
ಹುಚ್ಚು-ತನ
ಹುಚ್ಚು-ಸಾ-ಹಸ
ಹುಚ್ಚೆ
ಹುಚ್ಚೆ-ದ್ದು-ಬಿ-ಟ್ಟಿ-ದ್ದಾರೆ
ಹುಟ್ಟ
ಹುಟ್ಟ-ಬ-ಹುದು
ಹುಟ್ಟ-ಬೇ-ಕಾ-ಯಿ-ತಂತೆ
ಹುಟ್ಟಾ-ಸಂ-ನ್ಯಾ-ಸಿ-ಯ-ಲ್ಲವೆ
ಹುಟ್ಟಿ
ಹುಟ್ಟಿ-ಕೊಂಡ
ಹುಟ್ಟಿ-ಕೊಂ-ಡಿತು
ಹುಟ್ಟಿ-ಕೊಂ-ಡಿತೆ
ಹುಟ್ಟಿ-ಕೊಂ-ಡಿತ್ತು
ಹುಟ್ಟಿ-ಕೊಂ-ಡಿದೆ
ಹುಟ್ಟಿ-ಕೊ-ಳ್ಳ-ಲಿಲ್ಲ
ಹುಟ್ಟಿ-ಕೊ-ಳ್ಳಲು
ಹುಟ್ಟಿ-ಕೊ-ಳ್ಳಲೂ
ಹುಟ್ಟಿ-ಕೊ-ಳ್ಳು-ತ್ತವೆ
ಹುಟ್ಟಿ-ಕೊ-ಳ್ಳು-ತ್ತಿತ್ತು
ಹುಟ್ಟಿ-ಕೊ-ಳ್ಳು-ವಂ-ತಾ-ಗಿದೆ
ಹುಟ್ಟಿತು
ಹುಟ್ಟಿತ್ತು
ಹುಟ್ಟಿದ
ಹುಟ್ಟಿ-ದ-ವ-ನಲ್ಲ
ಹುಟ್ಟಿ-ದುದು
ಹುಟ್ಟಿದೆ
ಹುಟ್ಟಿ-ದ್ದ-ರಲ್ಲಿ
ಹುಟ್ಟಿ-ನಿಂದ
ಹುಟ್ಟಿ-ನಿಂ-ದಲೇ
ಹುಟ್ಟಿ-ಬಂದ
ಹುಟ್ಟಿ-ಬ-ರು-ವುದು
ಹುಟ್ಟಿ-ಬಿ-ಟ್ಟಿತು
ಹುಟ್ಟಿ-ಬಿ-ಟ್ಟಿತ್ತು
ಹುಟ್ಟಿ-ಬೆ-ಳೆದ
ಹುಟ್ಟಿಯೂ
ಹುಟ್ಟಿ-ಸು-ತ್ತಾನೆ
ಹುಟ್ಟು
ಹುಟ್ಟು-ಗುಣ
ಹುಟ್ಟು-ಗು-ಣವೇ
ಹುಟ್ಟು-ತ್ತದೆ
ಹುಟ್ಟು-ವವೋ
ಹುಟ್ಟು-ವು-ದಿಲ್ಲ
ಹುಟ್ಟು-ವುದು
ಹುಟ್ಟು-ಸಾ-ವು-ಗಳ
ಹುಟ್ಟು-ಹಾ-ಕಿ-ದರೆ
ಹುಡ-ಗ-ರೇ-ನಾ-ದರೂ
ಹುಡ-ಗು-ರೆಲ್ಲ
ಹುಡಿ
ಹುಡಿಯು
ಹುಡಿ-ಯೊಳೆ
ಹುಡು
ಹುಡು-ಕ-ತೊ-ಡ-ಗಿ-ದರು
ಹುಡು-ಕ-ಬೇ-ಕಾ-ಯಿತು
ಹುಡು-ಕಲು
ಹುಡು-ಕಾಟ
ಹುಡು-ಕಾಡಿ
ಹುಡುಕಿ
ಹುಡು-ಕಿ-ಕೊಂಡು
ಹುಡು-ಕಿ-ನೋ-ಡಿ-ದರೂ
ಹುಡು-ಕಿಸಿ
ಹುಡು-ಕು-ತ-ಲಿ-ಹೆಯೋ
ಹುಡು-ಕು-ತಿಹೆ
ಹುಡು-ಕು-ತ್ತಿದ್ದ
ಹುಡು-ಕು-ವು-ದಾ-ಗಲಿ
ಹುಡು-ಕುವೆ
ಹುಡುಗ
ಹುಡು-ಗನ
ಹುಡು-ಗ-ನಂತೆ
ಹುಡು-ಗ-ನನ್ನು
ಹುಡು-ಗ-ನಲ್ಲಿ
ಹುಡು-ಗ-ನಾ-ಗಿ-ದ್ದಾ-ಗಲೇ
ಹುಡು-ಗ-ನಾ-ಗಿ-ರ-ಬ-ಹುದು
ಹುಡು-ಗ-ನಾ-ದರೂ
ಹುಡು-ಗ-ನಿ-ಗಾಗಿ
ಹುಡು-ಗ-ನಿಗೆ
ಹುಡು-ಗ-ನಿದ್ದ
ಹುಡು-ಗನೋ
ಹುಡು-ಗರ
ಹುಡು-ಗ-ರಂತೆ
ಹುಡು-ಗ-ರನ್ನು
ಹುಡು-ಗ-ರನ್ನೂ
ಹುಡು-ಗ-ರ-ನ್ನೆಲ್ಲ
ಹುಡು-ಗ-ರ-ಲ್ಲವೆ
ಹುಡು-ಗ-ರಲ್ಲಿ
ಹುಡು-ಗ-ರಲ್ಲೇ
ಹುಡು-ಗ-ರಷ್ಟೆ
ಹುಡು-ಗರಿ
ಹುಡು-ಗ-ರಿಂದ
ಹುಡು-ಗ-ರಿಗೂ
ಹುಡು-ಗ-ರಿಗೆ
ಹುಡು-ಗ-ರಿ-ಬ್ಬರೂ
ಹುಡು-ಗ-ರಿಲ್ಲ
ಹುಡು-ಗರು
ಹುಡು-ಗ-ರು-ನಾ-ವೆಲ್ಲ
ಹುಡು-ಗರೂ
ಹುಡು-ಗ-ರೆಲ್ಲ
ಹುಡು-ಗ-ರೆ-ಲ್ಲ-ರಿಗೂ
ಹುಡು-ಗ-ರೆ-ಲ್ಲರೂ
ಹುಡು-ಗ-ರೊಂ-ದಿಗೆ
ಹುಡುಗಿ
ಹುಡು-ಗಿ-ಗಿನ್ನೂ
ಹುಡು-ಗಿಗೆ
ಹುಡು-ಗು-ತ-ನ-ವಿಲ್ಲ
ಹುಡುಗ್ರಾ
ಹುಣ್ಣಾ-ಗು-ವಷ್ಟು
ಹುದು-ಗಿದೆ
ಹುದು-ಗಿದ್ದು
ಹುದ್ದೆ
ಹುದ್ದೆ-ಯನ್ನು
ಹುಬ್ಬು-ಗಳ
ಹುಬ್ಬು-ಗಳನ್ನು
ಹುಮ್ಮ-ಸ್ಸಿ-ನೊಂ-ದಿಗೆ
ಹುರಿ
ಹುರಿ-ದುಂ-ಬಿ-ಸಿ-ದರು
ಹುರಿ-ದುಂ-ಬಿ-ಸು-ತ್ತಿ-ದ್ದರು
ಹುರು-ಪಿ-ನಿಂದ
ಹುರು-ಪುಂ-ಟಾ-ಗಿತ್ತು
ಹುಲಿ
ಹುಲು-ಮಾನವ
ಹುಲ್ಲು
ಹುಲ್ಲು-ಗ-ರಿ-ಗಳು
ಹುಲ್ಲು-ಬ-ಣ-ವೆ-ಗಳ
ಹುಲ್ಲು-ಹಾಸೇ
ಹುಳಿ
ಹುಳು-ವಿ-ಗಿಂ-ತಲೂ
ಹುಸಿ
ಹೂಂ
ಹೂಗಳನ್ನು
ಹೂಡ-ಬೇಕು
ಹೂಡಿ-ದರು
ಹೂಡಿದ್ದ
ಹೂಡಿಯೂ
ಹೂಳ-ಪು-ಗ-ಣ್ಣು-ಗಳ
ಹೂವಿನ
ಹೂವು-ಗಳನ್ನು
ಹೂವು-ಗಳನ್ನೆಲ್ಲ
ಹೂಹಾ-ರ-ಗಳಿಂದ
ಹೃತ್ಪೂ-ರ್ವಕ
ಹೃತ್ಪೂ-ರ್ವ-ಕ-ವಾಗಿ
ಹೃತ್ಪೂ-ರ್ವ-ಕ-ವಾ-ಗಿಯೇ
ಹೃದಯ
ಹೃದ-ಯ-ಬು-ದ್ಧಿ-ಗಳ
ಹೃದ-ಯಂ-ಗ-ಮ-ವಾಗಿ
ಹೃದ-ಯಕ್ಕೆ
ಹೃದ-ಯ-ಗ-ಳಲ್ಲೂ
ಹೃದ-ಯ-ಗ-ಳಿಗೆ
ಹೃದ-ಯದ
ಹೃದ-ಯ-ದ-ಮೃ-ತಾ-ನಂ-ದ-ರೂ-ಪನು
ಹೃದ-ಯ-ದ-ಲ್ಲಾ-ಗಲೇ
ಹೃದ-ಯ-ದಲ್ಲಿ
ಹೃದ-ಯ-ದಲ್ಲೂ
ಹೃದ-ಯ-ದ-ಲ್ಲೇ-ಕಾ-ಣ-ಬ-ಹುದು
ಹೃದ-ಯ-ದಾ-ಳ-ದಿಂದ
ಹೃದ-ಯ-ದಿಂದ
ಹೃದ-ಯ-ದುಂಬಿ
ಹೃದ-ಯ-ದೊ-ಳಗೆ
ಹೃದ-ಯ-ನಿ-ಗ-ಲ್ಲದೆ
ಹೃದ-ಯ-ಬಾಬು
ಹೃದ-ಯ-ಬಾ-ಬು-ವಿನ
ಹೃದ-ಯ-ಮಂ-ದಿ-ರದ
ಹೃದ-ಯ-ರಾಮ
ಹೃದ-ಯ-ವಂ-ತಿಕೆ
ಹೃದ-ಯ-ವನ್ನು
ಹೃದ-ಯ-ವನ್ನೇ
ಹೃದ-ಯವು
ಹೃದ-ಯವೂ
ಹೃದ-ಯವೇ
ಹೃದ-ಯ-ವೈ-ಶಾಲ್ಯ
ಹೃದ-ಯ-ವೊಂ-ದ-ರಿಂದ
ಹೃದ-ಯ-ಸೂಕ್ಷ್ಮ
ಹೃದ-ಯ-ಸ್ಪರ್ಶಿ
ಹೃದ-ಯಾಂ-ತ-ರಾ-ಳ-ದಿಂದ
ಹೃದ-ಯಾಂ-ತ-ರಾ-ಳ-ವೆಲ್ಲ
ಹೃದ-ಯಾ-ಘಾ-ತ-ದಿಂದ
ಹೃದ-ಯಾ-ಘಾ-ತ-ವಾ-ಗಿದ್ದು
ಹೃದ್ಗತ
ಹೃನ್ಮ-ನ-ಗಳನ್ನು
ಹೃನ್ಮ-ನ-ಗಳು
ಹೃಷೀ
ಹೃಷೀ-ಕೇ-ಶಕ್ಕೆ
ಹೃಷೀ-ಕೇ-ಶದ
ಹೃಷೀ-ಕೇ-ಶ-ದತ್ತ
ಹೃಷೀ-ಕೇ-ಶ-ದಲ್ಲಿ
ಹೃಷೀ-ಕೇ-ಶ-ದಿಂದ
ಹೃಷೀ-ಕೇ-ಶ-ವನ್ನು
ಹೃಷೀ-ಕೇ-ಶವು
ಹೆಂಗಸು
ಹೆಂಡತಿ
ಹೆಂಡ-ತಿ
ಹೆಂಡ-ತಿ-ಮ-ಕ್ಕ-ಳನ್ನೂ
ಹೆಂಡ-ತಿ-ಮ-ಕ್ಕಳು
ಹೆಂಡ-ತಿ-ಮ-ಕ್ಕ-ಳೊಂ-ದಿಗೆ
ಹೆಂಡ-ತಿಯ
ಹೆಂಡ-ತಿ-ಯ-ಲ್ಲವೆ
ಹೆಂಡ-ತಿ-ಯಾಗಿ
ಹೆಗಲ
ಹೆಗ-ಲ-ಮೇಲೆ
ಹೆಗ-ಲೇ-ರಿತು
ಹೆಗಲ್
ಹೆಗ್ಗ-ಳಿ-ಕೆ-ಯನ್ನು
ಹೆಗ್ಗು-ರುತು
ಹೆಚ್ಚನ್ನು
ಹೆಚ್ಚ-ಲಾ-ರಂ-ಭಿ-ಸಿ-ದುವು
ಹೆಚ್ಚಾ-ಗ-ಬೇ-ಕೆಂ-ಬು-ದನ್ನು
ಹೆಚ್ಚಾಗಿ
ಹೆಚ್ಚಾ-ಗಿ-ಬಿ-ಟ್ಟಿತು
ಹೆಚ್ಚಾ-ಗಿ-ಬಿ-ಟ್ಟಿತ್ತು
ಹೆಚ್ಚಾ-ಗಿಯೇ
ಹೆಚ್ಚಾ-ಗಿ-ರ-ಲಿಲ್ಲ
ಹೆಚ್ಚಾ-ಗಿ-ರು-ವ-ವ-ರಲ್ಲಿ
ಹೆಚ್ಚಾ-ಗಿ-ರು-ವುದು
ಹೆಚ್ಚಾದ
ಹೆಚ್ಚಾ-ದಂತೆ
ಹೆಚ್ಚಾ-ದರೆ
ಹೆಚ್ಚಾ-ದು-ದ-ರಿಂದ
ಹೆಚ್ಚಾ-ಯಿತು
ಹೆಚ್ಚಿಗೆ
ಹೆಚ್ಚಿತು
ಹೆಚ್ಚಿತ್ತು
ಹೆಚ್ಚಿನ
ಹೆಚ್ಚಿ-ನ-ವ-ರಿಗೆ
ಹೆಚ್ಚಿ-ನ-ವರೆಲ್ಲ
ಹೆಚ್ಚಿ-ನವು
ಹೆಚ್ಚಿ-ನ-ವೆಲ್ಲ
ಹೆಚ್ಚಿ-ಸಿ-ಕೊ-ಳ್ಳಲೂ
ಹೆಚ್ಚಿ-ಸು-ತ್ತಿ-ದ್ದರು
ಹೆಚ್ಚು
ಹೆಚ್ಚು-ಕ-ಡಿಮೆ
ಹೆಚ್ಚು-ಕ-ಡಿ-ಮೆ-ಯಾ-ಗಿ-ಬಿ-ಟ್ಟಿ-ದ್ದರೂ
ಹೆಚ್ಚುತ್ತ
ಹೆಚ್ಚು-ತ್ತಲೇ
ಹೆಚ್ಚು-ತ್ತಿತ್ತು
ಹೆಚ್ಚು-ತ್ತಿತ್ತೇ
ಹೆಚ್ಚು-ಹೆಚ್ಚು
ಹೆಚ್ಚೆ
ಹೆಚ್ಚೆಂ-ದರೆ
ಹೆಚ್ಚೆಚ್ಚು
ಹೆಚ್ಚೇ
ಹೆಚ್ಚೇನೂ
ಹೆಜ್ಜೆ
ಹೆಜ್ಜೆಯ
ಹೆಜ್ಜೆ-ಯನ್ನೂ
ಹೆಜ್ಜೆ-ಯೊಂ-ದ-ನ್ನಿ-ಟ್ಟಿ-ದ್ದಾನೆ
ಹೆಜ್ಜೆ-ಹಾ-ಕಲೂ
ಹೆಜ್ಜೆ-ಹಾ-ಕುತ್ತ
ಹೆಜ್ಜೆ-ಹೆ-ಜ್ಜೆಗೂ
ಹೆಜ್ಜೆ-ಹೆ-ಜ್ಜೆ-ಯಾಗಿ
ಹೆಡೆ
ಹೆಡೆ-ಬಿ-ಚ್ಚಿ-ಕೊಂಡು
ಹೆಡೆ-ಯೆತ್ತಿ
ಹೆಣೆ-ದು-ಕೊಂ-ಡಿ-ರು-ವುದನ್ನು
ಹೆಣ್ಣಿ-ಗ-ತನ
ಹೆಣ್ಣಿನ
ಹೆಣ್ಣು
ಹೆಣ್ಣು-ಹೊ-ನ್ನು-ಮ-ಣ್ಣು-ಗಳ
ಹೆಣ್ಣು-ಒ-ಬ್ಬಳು
ಹೆಣ್ಣು-ಮ-ಕ್ಕ-ಳೆಂದು
ಹೆಣ್ಣು-ಮ-ಣ್ಣಿನ
ಹೆತ್ತ
ಹೆತ್ತ-ತಾ-ಯಿಗೆ
ಹೆತ್ತ-ವ-ರಿಗೂ
ಹೆತ್ತ-ವ-ರಿಗೆ
ಹೆತ್ತ-ವ-ಳಾದ
ಹೆದ-ರ-ದಿರಿ
ಹೆದ-ರ-ಬೇಡಿ
ಹೆದರಿ
ಹೆದ-ರಿಕೆ
ಹೆದ-ರಿ-ಕೆ-ಯಾ-ಯಿತು
ಹೆದ-ರಿ-ಕೆ-ಯಿಂ-ದಲೋ
ಹೆದ-ರಿ-ಕೊಂಡು
ಹೆದ-ರಿ-ಕೊಂಡೇ
ಹೆದ-ರಿ-ಕೊ-ಳ್ಳದೆ
ಹೆದ-ರಿ-ಕೊ-ಳ್ಳ-ಬೇಡ
ಹೆದ-ರಿ-ಕೊ-ಳ್ಳು-ವುದೂ
ಹೆದ-ರಿ-ನ-ನಗೆ
ಹೆದ-ರಿ-ಬಿ-ಟ್ಟಿ-ದ್ದರು
ಹೆದ-ರಿ-ಸ-ಬೇ-ಕಂತೆ
ಹೆದ-ರಿ-ಸಲು
ಹೆದ-ರಿಸಿ
ಹೆದ-ರಿ-ಸಿ-ದರೂ
ಹೆದ-ರಿ-ಸಿಯೂ
ಹೆದ-ರಿ-ಸು-ತ್ತಿ-ದ್ದಳು
ಹೆದ-ರಿ-ಸು-ತ್ತೀಯೋ
ಹೆದ-ರು-ವ-ವರು
ಹೆಬ್ಬಾ-ಗಿಲ
ಹೆಮ್ಮೆ
ಹೆಮ್ಮೆ-ಪಟ್ಟು
ಹೆಮ್ಮೆಯ
ಹೆಮ್ಮೆ-ಯಿಂದ
ಹೆರು-ವುದು
ಹೆಸ-ರ-ನ್ನಂತೂ
ಹೆಸ-ರ-ನ್ನಿ-ಟ್ಟು-ಕೊಂಡ
ಹೆಸ-ರ-ನ್ನಿ-ತ್ತರು
ಹೆಸ-ರನ್ನು
ಹೆಸ-ರನ್ನೇ
ಹೆಸ-ರಾಂತ
ಹೆಸ-ರಾ-ಗಿ-ದ್ದ-ವರು
ಹೆಸ-ರಾದ
ಹೆಸ-ರಾ-ದ-ವರು
ಹೆಸ-ರಿಗೆ
ಹೆಸ-ರಿದೆ
ಹೆಸ-ರಿನ
ಹೆಸ-ರಿ-ನಲ್ಲಿ
ಹೆಸ-ರಿ-ನಲ್ಲೂ
ಹೆಸ-ರಿ-ನಲ್ಲೇ
ಹೆಸ-ರಿ-ನಿಂದ
ಹೆಸ-ರಿ-ಸ-ಲಾಗಿದೆ
ಹೆಸ-ರಿ-ಸಿದ
ಹೆಸರು
ಹೆಸ-ರು-ಕೀರ್ತಿ
ಹೆಸ-ರು-ಕೀ-ರ್ತಿ-ಗಳನ್ನು
ಹೆಸ-ರು-ಗ-ಳ-ನ್ನಿ-ಟ್ಟು-ಕೊಂ-ಡರೂ
ಹೆಸ-ರು-ಗ-ಳಿಂ-ದಲೇ
ಹೆಸ-ರು-ಗಳು
ಹೆಸ-ರು-ಗ-ಳೆಲ್ಲ
ಹೆಸ-ರು-ವಾ-ಸಿ-ಯಾ-ದ-ವರು
ಹೆಸರೇ
ಹೇ
ಹೇಗದು
ಹೇಗಾ-ದರು
ಹೇಗಾ-ದರೂ
ಹೇಗಾ-ದಾನು
ಹೇಗಾ-ದೀತು
ಹೇಗಿತ್ತೋ
ಹೇಗಿದೆ
ಹೇಗಿ-ದ್ದರೂ
ಹೇಗಿ-ದ್ದರೆ
ಹೇಗಿ-ದ್ದಾನೆ
ಹೇಗಿರ
ಹೇಗಿ-ರಿ-ಸಿ-ದರೆ
ಹೇಗಿ-ರು-ತ್ತ-ದೆಂದು
ಹೇಗಿ-ರು-ತ್ತ-ದೆಯೋ
ಹೇಗೆ
ಹೇಗೆಂ-ದರೆ
ಹೇಗೆಂ-ಬು-ದರ
ಹೇಗೆ-ತಾನೆ
ಹೇಗೆ-ನ್ನಿ-ಸಿತು
ಹೇಗೋ
ಹೇಡಿ-ಗಳು
ಹೇಡಿ-ತನ
ಹೇತು
ಹೇತುವು
ಹೇಮ-ಚಂದ್ರ
ಹೇಯ-ವಾದ
ಹೇಯ್
ಹೇರ-ಲಿಲ್ಲ
ಹೇರಳ
ಹೇರ-ಳ-ವಾಗಿ
ಹೇರ-ಳ-ವಾ-ಗಿದೆ
ಹೇರಿ-ಕೊಂ-ಡಿದೆ
ಹೇರಿ-ಕೊಂಡು
ಹೇರಿ-ರ-ಲಿಲ್ಲ
ಹೇರು-ವು-ದರ
ಹೇರು-ವುದು
ಹೇಳ
ಹೇಳ-ತೊ-ಡ-ಗಿದ
ಹೇಳ-ತೊ-ಡ-ಗಿ-ದರು
ಹೇಳ-ದಿ-ರಲು
ಹೇಳ-ದೆಯೇ
ಹೇಳ-ಬ-ರು-ವಂ-ತಿಲ್ಲ
ಹೇಳ-ಬಲ್ಲ
ಹೇಳ-ಬ-ಲ್ಲ-ವ-ರಾರು
ಹೇಳ-ಬಲ್ಲೆ
ಹೇಳ-ಬ-ಹು-ದಾ-ಗಿ-ತ್ತಾ-ದರೂ
ಹೇಳ-ಬ-ಹುದು
ಹೇಳ-ಬೇ-ಕಾ-ಗಿ-ತ್ತ-ಲ್ಲವೆ
ಹೇಳ-ಬೇ-ಕಾ-ಗಿತ್ತು
ಹೇಳ-ಬೇ-ಕಾ-ಗಿಯೇ
ಹೇಳ-ಬೇ-ಕಾಗು
ಹೇಳ-ಬೇಕು
ಹೇಳ-ಬೇ-ಕೆಂ-ದರೆ
ಹೇಳ-ಲಾ-ಗ-ದಿ-ರ-ಬ-ಹುದು
ಹೇಳ-ಲಾಗು
ಹೇಳ-ಲಾ-ರಂ-ಭಿ-ಸಿದ
ಹೇಳ-ಲಾ-ರಂ-ಭಿ-ಸಿ-ದ-ರು-ಸ-ಕಾ-ಲ-ದಲ್ಲಿ
ಹೇಳ-ಲಾ-ರದ
ಹೇಳ-ಲಾರೆ
ಹೇಳಲಿ
ಹೇಳ-ಲಿಲ್ಲ
ಹೇಳ-ಲಿ-ಲ್ಲವೆ
ಹೇಳಲು
ಹೇಳಲೂ
ಹೇಳ-ಲೆ-ತ್ನಿ-ಸಿ-ದರು
ಹೇಳ-ಲೇ-ಬೇ-ಕಾ-ಗಿಲ್ಲ
ಹೇಳ-ಲೇ-ಬೇಕು
ಹೇಳ-ಲ್ಪ-ಟ್ಟಿ-ರುವ
ಹೇಳ-ಹೆ-ಸ-ರಿ-ಲ್ಲ-ದಂ-ತಾ-ಗಿ-ಬಿ-ಡುವ
ಹೇಳ-ಹೊ-ರ-ಟಂ-ತಿತ್ತು
ಹೇಳ-ಹೋ-ಗಿ-ರ-ಲಿಲ್ಲ
ಹೇಳಾ-ರಂ-ಭಿ-ಸಿದ
ಹೇಳಿ
ಹೇಳಿ-ಕ-ಳಿಸಿ
ಹೇಳಿ-ಕ-ಳಿ-ಸಿದ
ಹೇಳಿ-ಕ-ಳಿ-ಸಿ-ದರೂ
ಹೇಳಿಕೊ
ಹೇಳಿ-ಕೊಂಡ
ಹೇಳಿ-ಕೊಂ-ಡ-ಪ್ರ-ಬಲ
ಹೇಳಿ-ಕೊಂ-ಡಾಗ
ಹೇಳಿ-ಕೊಂ-ಡಿ-ದ್ದ-ರೆಂ-ಬುದು
ಹೇಳಿ-ಕೊಂಡು
ಹೇಳಿ-ಕೊಂಡೆ
ಹೇಳಿ-ಕೊಟ್ಟ
ಹೇಳಿ-ಕೊ-ಟ್ಟದ್ದು
ಹೇಳಿ-ಕೊ-ಟ್ಟರು
ಹೇಳಿ-ಕೊಟ್ಟೆ
ಹೇಳಿ-ಕೊ-ಡ-ಬೇ-ಕಾದ
ಹೇಳಿ-ಕೊ-ಡು-ತ್ತಿ-ದ್ದರು
ಹೇಳಿ-ಕೊ-ಡು-ವಂತೆ
ಹೇಳಿ-ಕೊ-ಳ್ಳ-ಬಹು
ಹೇಳಿ-ಕೊ-ಳ್ಳಲು
ಹೇಳಿ-ಕೊಳ್ಳಿ
ಹೇಳಿ-ಕೊಳ್ಳು
ಹೇಳಿ-ಕೊ-ಳ್ಳುತ್ತ
ಹೇಳಿ-ಕೊ-ಳ್ಳು-ತ್ತಲೂ
ಹೇಳಿ-ಕೊ-ಳ್ಳು-ತ್ತಾನೆ
ಹೇಳಿ-ಕೊ-ಳ್ಳು-ತ್ತಾರೆ
ಹೇಳಿ-ಕೊ-ಳ್ಳು-ತ್ತಿ-ದ್ದರು
ಹೇಳಿ-ತೀ-ರ-ದಷ್ಟು
ಹೇಳಿದ
ಹೇಳಿ-ದಂತೆ
ಹೇಳಿ-ದ-ಈಗ
ಹೇಳಿ-ದ-ನಾನೂ
ಹೇಳಿ-ದ-ರಾ-ದರೂ
ಹೇಳಿ-ದರು
ಹೇಳಿ-ದರೂ
ಹೇಳಿ-ದರೆ
ಹೇಳಿ-ದಳು
ಹೇಳಿ-ದ-ವನೂ
ಹೇಳಿ-ದ-ವನೇ
ಹೇಳಿ-ದ-ಸ್ವಾಮಿ
ಹೇಳಿ-ದಾಗ
ಹೇಳಿ-ದು-ದನ್ನು
ಹೇಳಿದೆ
ಹೇಳಿ-ದೆ-ಅಂದು
ಹೇಳಿ-ದೆ-ಅಮ್ಮಾ
ಹೇಳಿ-ದೆ-ನನ್ನ
ಹೇಳಿ-ದೆ-ನೆಂಬ
ಹೇಳಿ-ದೆ-ಯಾರ
ಹೇಳಿ-ದೆಯೊ
ಹೇಳಿದ್ದ
ಹೇಳಿ-ದ್ದಕ್ಕೆ
ಹೇಳಿ-ದ್ದನ್ನು
ಹೇಳಿ-ದ್ದ-ನ್ನೆಲ್ಲ
ಹೇಳಿ-ದ್ದರು
ಹೇಳಿ-ದ್ದರೂ
ಹೇಳಿ-ದ್ದರೋ
ಹೇಳಿ-ದ್ದಾರೆ
ಹೇಳಿ-ದ್ದಾಳೆ
ಹೇಳಿದ್ದು
ಹೇಳಿ-ದ್ದುಂಟು
ಹೇಳಿ-ದ್ದುಂ-ಟು-ತ್ಯಾಗ
ಹೇಳಿ-ದ್ದೇ-ನೆ-ನಾನು
ಹೇಳಿ-ಬಿಟ್ಟ
ಹೇಳಿ-ಬಿ-ಟ್ಟ-ತಾನು
ಹೇಳಿ-ಬಿ-ಟ್ಟರು
ಹೇಳಿ-ಬಿ-ಟ್ಟರೆ
ಹೇಳಿ-ಬಿ-ಟ್ಟಾಗ
ಹೇಳಿ-ಬಿ-ಡ-ಬ-ಹುದು
ಹೇಳಿ-ಬಿ-ಡು-ತ್ತಾರೆ
ಹೇಳಿ-ಬಿ-ಡು-ತ್ತಿದ್ದ
ಹೇಳಿ-ಯಾರು
ಹೇಳಿ-ಯೇ-ಬಿಟ್ಟ
ಹೇಳಿ-ರ-ಬ-ಹುದು
ಹೇಳಿ-ರ-ಲಿಲ್ಲ
ಹೇಳಿ-ರು-ವಂತೆ
ಹೇಳಿ-ರು-ವುದು
ಹೇಳಿಲ್ಲ
ಹೇಳಿ-ಸಿ-ಕೊ-ಳ್ಳುವ
ಹೇಳಿ-ಸಿ-ದ್ದಲ್ಲ
ಹೇಳಿ-ಸು-ತ್ತಿದ್ದ
ಹೇಳಿ-ಸುವ
ಹೇಳು
ಹೇಳುತ್ತ
ಹೇಳು-ತ್ತದೆ
ಹೇಳು-ತ್ತಲೇ
ಹೇಳು-ತ್ತಾ-ನಲ್ಲ
ಹೇಳು-ತ್ತಾನೆ
ಹೇಳು-ತ್ತಾ-ನೆ-ಅ-ವನ
ಹೇಳು-ತ್ತಾ-ನೆ-ಬೇ-ಡಪ್ಪ
ಹೇಳು-ತ್ತಾನೋ
ಹೇಳು-ತ್ತಾರೆ
ಹೇಳು-ತ್ತಾ-ರೆ-ಕಾ-ಮ-ಕಾಂ-ಚ-ನ-ವನ್ನು
ಹೇಳು-ತ್ತಾ-ರೆ-ಮೊ-ದಲು
ಹೇಳುತ್ತಿ
ಹೇಳು-ತ್ತಿತ್ತು
ಹೇಳು-ತ್ತಿದೆ
ಹೇಳು-ತ್ತಿದ್ದ
ಹೇಳು-ತ್ತಿ-ದ್ದಂತೆ
ಹೇಳು-ತ್ತಿ-ದ್ದ-ರಷ್ಟೆ
ಹೇಳು-ತ್ತಿ-ದ್ದರು
ಹೇಳು-ತ್ತಿ-ದ್ದ-ರು-ಇ-ದೆಲ್ಲ
ಹೇಳು-ತ್ತಿ-ದ್ದ-ರು-ತಾವು
ಹೇಳು-ತ್ತಿ-ದ್ದ-ರು-ಶ್ರದ್ಧೆ
ಹೇಳು-ತ್ತಿ-ದ್ದ-ರು-ಸಾ-ಧ-ಕ-ನಾ-ದ-ವನು
ಹೇಳು-ತ್ತಿ-ದ್ದಳು
ಹೇಳು-ತ್ತಿ-ದ್ದಾ-ನಲ್ಲ
ಹೇಳು-ತ್ತಿ-ದ್ದಾನೆ
ಹೇಳು-ತ್ತಿ-ದ್ದಾ-ನೆ-ದೇ-ವರು
ಹೇಳು-ತ್ತಿ-ದ್ದಾರೆ
ಹೇಳು-ತ್ತಿ-ದ್ದಾ-ರೆ-ತಾವು
ಹೇಳು-ತ್ತಿ-ದ್ದಾ-ರೆ-ಮೊ-ದಲು
ಹೇಳು-ತ್ತಿ-ದ್ದಾ-ರೇನು
ಹೇಳು-ತ್ತಿ-ದ್ದೀರಿ
ಹೇಳು-ತ್ತಿ-ದ್ದು-ದ-ರಲ್ಲಿ
ಹೇಳು-ತ್ತಿ-ದ್ದುದು
ಹೇಳು-ತ್ತಿ-ದ್ದುವು
ಹೇಳು-ತ್ತಿರು
ಹೇಳು-ತ್ತಿ-ರು-ತ್ತೀಯ
ಹೇಳು-ತ್ತಿ-ರು-ತ್ತೇ-ನೆ-ಏ-ನಯ್ಯ
ಹೇಳು-ತ್ತಿ-ರುವ
ಹೇಳು-ತ್ತಿ-ರು-ವಂ-ತಿತ್ತು
ಹೇಳು-ತ್ತಿ-ರು-ವಾಗ
ಹೇಳು-ತ್ತಿ-ರು-ವುದು
ಹೇಳುತ್ತೀ
ಹೇಳು-ತ್ತೀಯಾ
ಹೇಳು-ತ್ತೀಯೆ
ಹೇಳು-ತ್ತೀರಿ
ಹೇಳು-ತ್ತೇನೆ
ಹೇಳು-ತ್ತೇ-ನೆ-ನಾನು
ಹೇಳುವ
ಹೇಳು-ವಂ-ತಿತ್ತು
ಹೇಳು-ವಂತೆ
ಹೇಳು-ವ-ವ-ರಲ್ಲ
ಹೇಳು-ವ-ವ-ರಿ-ಲ್ಲ-ವಲ್ಲ
ಹೇಳು-ವ-ಷ್ಟ-ರಲ್ಲಿ
ಹೇಳು-ವಾಗ
ಹೇಳು-ವುದನ್ನು
ಹೇಳು-ವು-ದಲ್ಲೂ
ಹೇಳು-ವುದಾ
ಹೇಳು-ವು-ದಾ-ದರೆ
ಹೇಳು-ವು-ದಿತ್ತು
ಹೇಳು-ವು-ದಿಲ್ಲ
ಹೇಳು-ವು-ದಿ-ಲ್ಲವೆ
ಹೇಳು-ವುದು
ಹೇಳು-ವುದೂ
ಹೇಳೋಣ
ಹೇವ-ರಿಕೆ
ಹೇಸ್ಟೀ
ಹೈ
ಹೈಏ-ನೇ-ನಿ-ವೆಯೋ
ಹೈಕೋ-ರ್ಟಿನ
ಹೊಂಗ-ನ-ಸಿಗೆ
ಹೊಂಗ-ನಸು
ಹೊಂದದ
ಹೊಂದ-ಬಲ್ಲ
ಹೊಂದ-ಬ-ಲ್ಲೆನೆ
ಹೊಂದ-ಬೇ-ಕೆಂಬ
ಹೊಂದ-ಲಿ-ದ್ದರು
ಹೊಂದ-ಲಿವೆ
ಹೊಂದಲು
ಹೊಂದಾ-ಣಿಕೆ
ಹೊಂದಾ-ಣಿ-ಕೆಗೆ
ಹೊಂದಾ-ಣಿ-ಕೆ-ಯಿ-ರು-ವುದು
ಹೊಂದಿ
ಹೊಂದಿ-ಕೆ-ಯಾ-ಗ-ದಿ-ದ್ದರೆ
ಹೊಂದಿ-ಕೊಂ-ಡು-ಬಿ-ಟ್ಟ-ನೆಂ-ದರೆ
ಹೊಂದಿ-ಕೊಳ್ಳ
ಹೊಂದಿ-ಕೊ-ಳ್ಳ-ಬಲ್ಲ
ಹೊಂದಿ-ಕೊ-ಳ್ಳು-ವಂ-ತಾ-ಗಿ-ಬಿ-ಟ್ಟಿತ್ತು
ಹೊಂದಿ-ಕೊ-ಳ್ಳು-ವಂತೆ
ಹೊಂದಿದ
ಹೊಂದಿ-ದರು
ಹೊಂದಿ-ದುವು
ಹೊಂದಿದ್ದ
ಹೊಂದಿ-ದ್ದರು
ಹೊಂದಿದ್ದು
ಹೊಂದಿ-ರ-ಬೇಕು
ಹೊಂದುತ್ತ
ಹೊಂದು-ತ್ತ-ದೆಯೋ
ಹೊಂದು-ತ್ತಿ-ರುವ
ಹೊಂದು-ತ್ತಿ-ರು-ವುದನ್ನು
ಹೊಂದುವ
ಹೊಂದು-ವನೋ
ಹೊಕ್ಕ
ಹೊಕ್ಕು
ಹೊಕ್ಕು-ಬಿ-ಟ್ಟಿದೆ
ಹೊಗ-ಳ-ಬೇ-ಕೆಂ-ದರೆ
ಹೊಗ-ಳಿ-ಕೆ
ಹೊಗ-ಳಿ-ಕೆ-ತೆ-ಗ-ಳಿ-ಕೆ-ಗ-ಳಿ-ಗೆಲ್ಲ
ಹೊಗ-ಳಿ-ಕೆ-ತೆ-ಗ-ಳಿ-ಕೆಗೂ
ಹೊಗ-ಳಿ-ಕೆಯ
ಹೊಗ-ಳಿ-ಕೆ-ಯ-ನ್ನಾ-ಗಲಿ
ಹೊಗ-ಳಿ-ಕೆಯೂ
ಹೊಗ-ಳಿ-ಕೆಯೇ
ಹೊಗ-ಳಿ-ದರು
ಹೊಗ-ಳಿ-ಸಿ-ಕೊಂ-ಬರು
ಹೊಗ-ಳು-ತ್ತಿ-ದ್ದರು
ಹೊಗ-ಳು-ವ-ರಾರು
ಹೊಗ-ಳು-ವ-ವ-ರು-ಹೊ-ಗ-ಳಿ-ಸಿ-ಕೊಳ್ಳು
ಹೊಗೆ
ಹೊಚ್ಚ
ಹೊಟ್ಟೆ
ಹೊಟ್ಟೆಗೆ
ಹೊಟ್ಟೆ-ತುಂಬ
ಹೊಟ್ಟೆ-ಪಾ-ಡಿನ
ಹೊಟ್ಟೆ-ಯಂತೂ
ಹೊಟ್ಟೆ-ಯಲ್ಲಿ
ಹೊಡೆತ
ಹೊಡೆ-ದರು
ಹೊಡೆ-ದಾಡಿ
ಹೊಡೆ-ಯಲು
ಹೊಡೆ-ಯು-ತ್ತಿ-ದ್ದರೆ
ಹೊಡೆ-ಯು-ವ-ಷ್ಟ-ರಲ್ಲಿ
ಹೊಣೆ
ಹೊಣೆ-ಗಾ-ರಿಕೆ
ಹೊಣೆ-ಗಾ-ರಿ-ಕೆಯ
ಹೊಣೆ-ಗಾ-ರಿ-ಕೆ-ಯನ್ನು
ಹೊಣೆ-ಗಾ-ರಿ-ಕೆ-ಯಿತ್ತು
ಹೊಣೆ-ಗಾ-ರಿ-ಕೆ-ಯೊಂದು
ಹೊತ್ತ
ಹೊತ್ತಾ-ಗಿ-ಹೋ-ಯಿತು
ಹೊತ್ತಾದ
ಹೊತ್ತಾ-ದರೂ
ಹೊತ್ತಾ-ಯಿ-ತೆನ್ನಿ
ಹೊತ್ತಿ-ಗಾ-ಗಲೇ
ಹೊತ್ತಿಗೆ
ಹೊತ್ತಿ-ದ್ದರು
ಹೊತ್ತಿ-ದ್ದೇನೆ
ಹೊತ್ತಿನ
ಹೊತ್ತಿ-ನಲ್ಲಿ
ಹೊತ್ತಿ-ನಲ್ಲೇ
ಹೊತ್ತಿ-ನಿಂದ
ಹೊತ್ತಿ-ರು-ತ್ತದೋ
ಹೊತ್ತಿಸಿ
ಹೊತ್ತಿ-ಸಿ-ಕೊಂಡು
ಹೊತ್ತಿ-ಸಿ-ದುವು
ಹೊತ್ತು
ಹೊತ್ತು-ಕೊಂಡು
ಹೊತ್ತೇನೂ
ಹೊದಿಕೆ
ಹೊದಿ-ಕೆಯ
ಹೊದ್ದಿದ್ದ
ಹೊದ್ದು
ಹೊನಲು
ಹೊನಲೇ
ಹೊನ್ನಿನ
ಹೊಮ್ಮಿ
ಹೊಮ್ಮಿ-ಬಂ-ದಂತೆ
ಹೊಮ್ಮಿ-ಸು-ತ್ತಿವೆ
ಹೊಮ್ಮು
ಹೊಮ್ಮು-ತ್ತಿತ್ತು
ಹೊಮ್ಮು-ತ್ತಿದೆ
ಹೊಯ್ದಾ-ಡು-ತ್ತಿತ್ತು
ಹೊರ
ಹೊರ-ಕ-ವ-ಚ-ವ-ನ್ನಷ್ಟೇ
ಹೊರಕ್ಕೆ
ಹೊರ-ಕ್ಕೆ-ಳೆ-ಯು-ತ್ತಿದೆ
ಹೊರ-ಗಡೆ
ಹೊರ-ಗ-ಡೆಯ
ಹೊರ-ಗ-ಡೆಯೇ
ಹೊರ-ಗಣ
ಹೊರ-ಗಿಂದ
ಹೊರ-ಗಿನ
ಹೊರ-ಗಿ-ನಿಂದ
ಹೊರ-ಗಿ-ನ್ನೊಂ-ದೆ-ನ-ಬೇಡ
ಹೊರಗೂ
ಹೊರಗೆ
ಹೊರ-ಗೆ-ಡಹಿ
ಹೊರ-ಗೆ-ಡ-ಹು-ತ್ತಾನೆ
ಹೊರ-ಗೆ-ಡ-ಹು-ತ್ತಾರೆ
ಹೊರ-ಗೆಯೇ
ಹೊರ-ಗೆ-ಲ್ಲಿಯು
ಹೊರ-ಗೆ-ಸೆದು
ಹೊರಗೇ
ಹೊರ-ಚಿ-ಮ್ಮು-ತ್ತಿ-ರುವು
ಹೊರಟ
ಹೊರ-ಟ-ಕ್ಷ-ಣ-ದಿಂ-ದಲೂ
ಹೊರ-ಟದ್ದು
ಹೊರ-ಟರು
ಹೊರ-ಟ-ರು-ಸ್ವಾಮಿ
ಹೊರ-ಟರೆ
ಹೊರ-ಟಳು
ಹೊರ-ಟ-ವರು
ಹೊರ-ಟಾಗ
ಹೊರ-ಟಾ-ಗ-ಲಂತೂ
ಹೊರ-ಟಾ-ಗಲೂ
ಹೊರ-ಟಾ-ಗ-ಲೆಲ್ಲ
ಹೊರಟಿ
ಹೊರ-ಟಿತು
ಹೊರ-ಟಿದ್ದ
ಹೊರ-ಟಿ-ದ್ದರು
ಹೊರ-ಟಿ-ದ್ದರೂ
ಹೊರ-ಟಿ-ದ್ದಾಗ
ಹೊರ-ಟಿ-ದ್ದಾನೆ
ಹೊರ-ಟಿ-ದ್ದಾರೆ
ಹೊರ-ಟಿದ್ದೀ
ಹೊರ-ಟಿ-ದ್ದೇನೆ
ಹೊರಟು
ಹೊರ-ಟು-ನಿಂ-ತರು
ಹೊರ-ಟು-ನಿಂ-ತಿ-ದ್ದಾರೆ
ಹೊರ-ಟು-ಬಂದ
ಹೊರ-ಟು-ಬಂದು
ಹೊರ-ಟು-ಬ-ರು-ತ್ತೇನೆ
ಹೊರ-ಟು-ಬಿಟ್ಟ
ಹೊರ-ಟು-ಬಿ-ಟ್ಟರು
ಹೊರ-ಟು-ಬಿ-ಟ್ಟರೆ
ಹೊರ-ಟು-ಬಿ-ಟ್ಟಿದೆ
ಹೊರ-ಟು-ಬಿ-ಟ್ಟಿದ್ದ
ಹೊರ-ಟು-ಬಿ-ಟ್ಟಿ-ದ್ದರು
ಹೊರ-ಟು-ಬಿ-ಟ್ಟಿ-ದ್ದರೆ
ಹೊರ-ಟು-ಬಿ-ಟ್ಟಿ-ದ್ದಾನೆ
ಹೊರ-ಟು-ಬಿ-ಟ್ಟಿರು
ಹೊರ-ಟು-ಬಿಟ್ಟೆ
ಹೊರ-ಟು-ಬಿ-ಡ-ಬ-ಲ್ಲರು
ಹೊರ-ಟು-ಬಿ-ಡ-ಬೇ-ಕೆಂದು
ಹೊರ-ಟು-ಬಿ-ಡ-ಬೇ-ಕೆಂಬ
ಹೊರ-ಟು-ಬಿ-ಡ-ಲಿ-ದ್ದೇವೆ
ಹೊರ-ಟು-ಬಿ-ಡಲು
ಹೊರ-ಟು-ಬಿ-ಡಲೆ
ಹೊರ-ಟು-ಬಿ-ಡು-ತ್ತಾನೆ
ಹೊರ-ಟು-ಬಿ-ಡು-ತ್ತಾರೆ
ಹೊರ-ಟು-ಬಿ-ಡು-ತ್ತಿದ್ದ
ಹೊರ-ಟು-ಬಿ-ಡು-ತ್ತಿ-ದ್ದರು
ಹೊರ-ಟು-ಬಿ-ಡು-ತ್ತಿ-ದ್ದ-ರೇನೋ
ಹೊರ-ಟು-ಬಿ-ಡು-ತ್ತಿದ್ದೆ
ಹೊರ-ಟು-ಬಿ-ಡುವ
ಹೊರ-ಟು-ಹೋ-ಗ-ಬಾ-ರದು
ಹೊರ-ಟು-ಹೋ-ಗಲು
ಹೊರ-ಟು-ಹೋಗಿ
ಹೊರ-ಟು-ಹೋ-ಗಿದೆ
ಹೊರ-ಟು-ಹೋ-ಗಿ-ದ್ದರು
ಹೊರ-ಟು-ಹೋ-ಗಿ-ದ್ದಾರೆ
ಹೊರ-ಟು-ಹೋ-ಗಿ-ದ್ದೆವು
ಹೊರ-ಟು-ಹೋ-ಗಿ-ಬಿ-ಟ್ಟಿದ್ದ
ಹೊರ-ಟು-ಹೋ-ಗಿ-ಬಿ-ಟ್ಟಿ-ದ್ದರೆ
ಹೊರ-ಟು-ಹೋ-ಗಿ-ಬಿ-ಡು-ತ್ತದೆ
ಹೊರ-ಟು-ಹೋ-ಗು-ತ್ತಾರೆ
ಹೊರ-ಟು-ಹೋ-ಗು-ತ್ತಿತ್ತು
ಹೊರ-ಟು-ಹೋ-ಗು-ತ್ತಿ-ದ್ದೀ-ಯಲ್ಲ
ಹೊರ-ಟು-ಹೋ-ಗು-ವಂ-ತಾ-ಗ-ಲೆಂದು
ಹೊರ-ಟು-ಹೋ-ಗು-ವಾಗ
ಹೊರ-ಟು-ಹೋ-ಗು-ವುದು
ಹೊರ-ಟು-ಹೋದ
ಹೊರ-ಟು-ಹೋ-ದ-ಮೇಲೆ
ಹೊರ-ಟು-ಹೋ-ದರು
ಹೊರ-ಟು-ಹೋ-ದರೇ
ಹೊರ-ಟು-ಹೋ-ಯಿ-ತಲ್ಲ
ಹೊರ-ಟು-ಹೋ-ಯಿತು
ಹೊರಟೆ
ಹೊರ-ಟೇ-ಬಿಟ್ಟ
ಹೊರ-ಟೇ-ಬಿ-ಟ್ಟರು
ಹೊರ-ಟೇ-ಹೋ-ಗಿ-ಬಿ-ಡ-ಬ-ಹುದು
ಹೊರ-ಡ-ಬೇ-ಕೆಂ-ದಿದ್ದ
ಹೊರ-ಡ-ಲಿ-ದ್ದೇನೆ
ಹೊರ-ಡ-ಲಿಲ್ಲ
ಹೊರ-ಡಲು
ಹೊರ-ಡಲೇ
ಹೊರ-ಡಿ-ಸಿದ್ದ
ಹೊರ-ಡು-ತ್ತಾನೆ
ಹೊರ-ಡು-ತ್ತಿದ್ದ
ಹೊರ-ಡು-ತ್ತಿ-ದ್ದೇನೆ
ಹೊರ-ಡು-ತ್ತಿಲ್ಲ
ಹೊರ-ಡುವ
ಹೊರ-ಡು-ವಾಗ
ಹೊರ-ಡು-ವಾ-ಗಲೂ
ಹೊರ-ಡು-ವುದನ್ನು
ಹೊರತು
ಹೊರ-ತೆ-ಗೆದು
ಹೊರ-ದೂ-ಡಿ-ದಂ-ತಾಗಿ
ಹೊರ-ದೂ-ಡು-ವು-ದಾ-ದರೂ
ಹೊರ-ನೋ-ಟಕ್ಕೆ
ಹೊರ-ಪ್ರ-ಪಂ-ಚದ
ಹೊರ-ಬಂದ
ಹೊರ-ಬಂ-ದಿದೆ
ಹೊರ-ಬಂ-ದಿದ್ದ
ಹೊರ-ಬರ
ಹೊರ-ಬ-ರಲು
ಹೊರ-ಬಿದ್ದ
ಹೊರಲು
ಹೊರ-ಳಿತು
ಹೊರ-ಸೂ-ಸ-ಲಾ-ರಂ-ಭಿ-ಸಿ-ದಾಗ
ಹೊರ-ಸೂ-ಸಲು
ಹೊರ-ಸೂ-ಸುತ್ತ
ಹೊರ-ಸೂ-ಸು-ತ್ತಿತ್ತು
ಹೊರ-ಸೂ-ಸು-ತ್ತಿದೆ
ಹೊರ-ಸೂ-ಸು-ತ್ತಿ-ರು-ವುದನ್ನು
ಹೊರ-ಹಾ-ಕ-ಲಿಲ್ಲ
ಹೊರ-ಹೊಮ್ಮಿ
ಹೊರ-ಹೊ-ಮ್ಮಿತು
ಹೊರ-ಹೊ-ಮ್ಮಿದ
ಹೊರಿ-ಸು-ವ-ವರ
ಹೊರುವ
ಹೊರು-ವುದು
ಹೊರೆ-ಯನ್ನು
ಹೊರೆ-ಯಾ-ಗದ
ಹೊರೆ-ಯಾಗಿ
ಹೊಲಿ-ಗೆಯ
ಹೊಳೆ-ಯ-ಲಿಲ್ಲ
ಹೊಳೆ-ಯಿತು
ಹೊಳೆ-ಯಿ-ತು-ಮ-ಕ್ಕ-ಳನ್ನು
ಹೊಳೆ-ಯುವ
ಹೊಳೆವ
ಹೊಳೆ-ಹೊ-ಳೆ-ಯುವ
ಹೊಸ
ಹೊಸ-ಕಿ-ಬಿ-ಡ-ಬೇಕೆ
ಹೊಸ-ತ-ರಲ್ಲಿ
ಹೊಸ-ದಾಗಿ
ಹೊಸ-ದಾ-ಗಿತ್ತು
ಹೊಸದು
ಹೊಸ-ದೇ-ನಾ-ಗಿ-ರ-ಲಿಲ್ಲ
ಹೊಸ-ದೇನೂ
ಹೊಸ-ದೊಂದು
ಹೊಸ-ಬ-ನಾದ
ಹೊಸ-ಬೆ-ಳ-ಕನ್ನು
ಹೊಸ-ಹೊಸ
ಹೊಸೆ-ಯುವ
ಹೋ
ಹೋಗ
ಹೋಗ-ದಂತೆ
ಹೋಗದೆ
ಹೋಗ-ಬಲ್ಲ
ಹೋಗ-ಬ-ಹುದು
ಹೋಗ-ಬ-ಹು-ದೆಂದು
ಹೋಗ-ಬಾ-ರದು
ಹೋಗ-ಬೇ-ಕಾ-ಗಿದೆ
ಹೋಗ-ಬೇ-ಕಾ-ಗು-ತ್ತಿತ್ತು
ಹೋಗ-ಬೇ-ಕಾ-ದ-ದ್ದಿಲ್ಲ
ಹೋಗ-ಬೇ-ಕಾ-ದರೆ
ಹೋಗ-ಬೇ-ಕಾ-ದ-ವರು
ಹೋಗ-ಬೇ-ಕಾ-ಯಿತು
ಹೋಗ-ಬೇಕು
ಹೋಗ-ಬೇಡಿ
ಹೋಗ-ಲಾ-ಗು-ತ್ತಿದೆ
ಹೋಗ-ಲಾ-ಡಿಸಿ
ಹೋಗ-ಲಾ-ಡಿ-ಸಿ-ಕೊಂ-ಡ-ವರು
ಹೋಗ-ಲಾ-ಡಿ-ಸಿ-ದರು
ಹೋಗ-ಲಾ-ಡಿ-ಸು-ವಂತೆ
ಹೋಗ-ಲಾ-ಯಿತು
ಹೋಗ-ಲಾರ
ಹೋಗ-ಲಾ-ರಂ-ಭಿ-ಸಿದ
ಹೋಗ-ಲಾ-ರಂ-ಭಿ-ಸಿ-ದ್ದರು
ಹೋಗ-ಲಾ-ರದು
ಹೋಗ-ಲಾ-ರದೆ
ಹೋಗ-ಲಾ-ರರು
ಹೋಗಲಿ
ಹೋಗ-ಲಿಲ್ಲ
ಹೋಗಲು
ಹೋಗಲೂ
ಹೋಗಲೇ
ಹೋಗ-ಲೇ-ಬೇ-ಕಾ-ಗಿದೆ
ಹೋಗ-ಲೇ-ಬೇ-ಕಾ-ಯಿತು
ಹೋಗ-ಲೇ-ಬೇಕು
ಹೋಗ-ಲೇ-ಬೇಡ
ಹೋಗ-ಲೊಲ್ಲೆ
ಹೋಗಾ-ಕಡೆ
ಹೋಗಿ
ಹೋಗಿದೆ
ಹೋಗಿದ್ದ
ಹೋಗಿ-ದ್ದಂತೆ
ಹೋಗಿ-ದ್ದ-ರಿಂದ
ಹೋಗಿ-ದ್ದರು
ಹೋಗಿ-ದ್ದರೆ
ಹೋಗಿ-ದ್ದಾಗ
ಹೋಗಿ-ದ್ದಾರೆ
ಹೋಗಿ-ದ್ದಿ-ರ-ಬೇಕು
ಹೋಗಿ-ದ್ದು-ಬಿ-ಡೋಣ
ಹೋಗಿದ್ದೆ
ಹೋಗಿ-ಬಂ-ದ-ವನೇ
ಹೋಗಿ-ಬ-ರ-ಬೇ-ಕಾ-ಗು-ತ್ತಿತ್ತು
ಹೋಗಿ-ಬ-ರಲು
ಹೋಗಿ-ಬ-ರು-ತ್ತಿ-ದ್ದ-ನಾ-ದರೂ
ಹೋಗಿ-ಬ-ರು-ತ್ತಿ-ದ್ದರು
ಹೋಗಿ-ಬ-ರು-ವು-ದಾಗಿ
ಹೋಗಿ-ಬಿ-ಟ್ಟುವು
ಹೋಗಿ-ಬಿ-ಡು-ವು-ದೆಂದು
ಹೋಗಿ-ರ-ಬ-ಹುದು
ಹೋಗಿ-ರ-ಬೇಕಾ
ಹೋಗಿ-ರ-ಬೇ-ಕಾ-ಯಿತು
ಹೋಗಿ-ರ-ಬೇ-ಕೆಂಬ
ಹೋಗಿ-ರ-ಲಿಲ್ಲ
ಹೋಗಿ-ರು-ತ್ತಿದ್ದ
ಹೋಗಿ-ರುವ
ಹೋಗಿ-ರು-ವು-ದೊಂದೇ
ಹೋಗು
ಹೋಗುತ್ತ
ಹೋಗು-ತ್ತದೆ
ಹೋಗು-ತ್ತ-ದೆಯೇ
ಹೋಗು-ತ್ತಾನೆ
ಹೋಗು-ತ್ತಾನೋ
ಹೋಗು-ತ್ತಾರೆ
ಹೋಗುತ್ತಿ
ಹೋಗು-ತ್ತಿತ್ತು
ಹೋಗು-ತ್ತಿದ್ದ
ಹೋಗು-ತ್ತಿ-ದ್ದರು
ಹೋಗು-ತ್ತಿ-ದ್ದ-ರೆ-ಹು-ಡುಗ
ಹೋಗು-ತ್ತಿ-ದ್ದಾಗ
ಹೋಗು-ತ್ತಿ-ದ್ದಾನೆ
ಹೋಗು-ತ್ತಿ-ದ್ದಾರೆ
ಹೋಗು-ತ್ತಿ-ದ್ದುದು
ಹೋಗು-ತ್ತಿ-ದ್ದೆವು
ಹೋಗು-ತ್ತಿ-ದ್ದೇನೆ
ಹೋಗು-ತ್ತಿ-ದ್ದೇ-ವೆಂ-ಬು-ದನ್ನು
ಹೋಗು-ತ್ತಿ-ರು-ತ್ತೇನೆ
ಹೋಗು-ತ್ತಿ-ರುವ
ಹೋಗು-ತ್ತಿ-ರು-ವು-ದಾಗಿ
ಹೋಗು-ತ್ತಿ-ವೆಯೋ
ಹೋಗುತ್ತೀ
ಹೋಗು-ತ್ತೀಯೇ
ಹೋಗು-ತ್ತೀಯೋ
ಹೋಗು-ತ್ತೀರಿ
ಹೋಗು-ತ್ತೇನೆ
ಹೋಗು-ತ್ತೇವೆ
ಹೋಗುವ
ಹೋಗು-ವಂ-ತಿಲ್ಲ
ಹೋಗು-ವಂತೆ
ಹೋಗು-ವಂ-ತೆಯೂ
ಹೋಗು-ವ-ವ-ನಲ್ಲ
ಹೋಗು-ವ-ಷ್ಟ-ರಲ್ಲಿ
ಹೋಗು-ವ-ಷ್ಟ-ರಲ್ಲೇ
ಹೋಗು-ವಾಗ
ಹೋಗು-ವಾ-ಗ-ಲೆಲ್ಲ
ಹೋಗು-ವಾ-ಗಲೇ
ಹೋಗು-ವು-ದ-ಕ್ಕಾಗಿ
ಹೋಗು-ವು-ದಕ್ಕೆ
ಹೋಗು-ವುದನ್ನು
ಹೋಗು-ವು-ದಾಗಿ
ಹೋಗು-ವು-ದಿತ್ತು
ಹೋಗು-ವು-ದಿಲ್ಲ
ಹೋಗು-ವುದು
ಹೋಗು-ವು-ದೆಂ-ದರೆ
ಹೋಗು-ವುದೇ
ಹೋಗೆಲೈ
ಹೋಗೋ
ಹೋಗೋಣ
ಹೋಗೋ-ಣವೆ
ಹೋಟೆ-ಲಿ-ನಲ್ಲಿ
ಹೋದ
ಹೋದಂ-ತಹ
ಹೋದಂ-ತಾ-ಗಿದೆ
ಹೋದಂ-ತಾ-ಯಿತು
ಹೋದಂತೆ
ಹೋದಂ-ತೆಲ್ಲ
ಹೋದ-ದ್ದ-ರಿಂದ
ಹೋದ-ದ್ದುಂಟು
ಹೋದನೋ
ಹೋದ-ರಲ್ಲ
ಹೋದರು
ಹೋದರೂ
ಹೋದರೆ
ಹೋದಲ್ಲೆಲ್ಲ
ಹೋದಳು
ಹೋದ-ವ-ರಿ-ದ್ದಾ-ರೆಯೇ
ಹೋದಾಗ
ಹೋದಾ-ಗಿ-ನಿಂದ
ಹೋದಾನು
ಹೋದು-ದನ್ನು
ಹೋದುವು
ಹೋದೆ
ಹೋದೇನು
ಹೋಮ
ಹೋಮಕ್ಕೆ
ಹೋಮದ
ಹೋಮ-ದಲ್ಲಿ
ಹೋಮಾ-ಗ್ನಿಗೆ
ಹೋಮಿ
ಹೋಮಿಯೋ
ಹೋಮಿ-ಯೋ-ಪತಿ
ಹೋಮಿ-ಯೋ-ಪಥಿ
ಹೋಯಿ-ತಲ್ಲ
ಹೋಯಿತು
ಹೋಯ್
ಹೋರಾಟ
ಹೋರಾ-ಟ-ಗಳನ್ನು
ಹೋರಾ-ಟ-ಗ-ಳಿ-ರ-ಬಾ-ರದು
ಹೋರಾ-ಟದ
ಹೋರಾ-ಟ-ವನ್ನು
ಹೋರಾ-ಟವೇ
ಹೋರಾ-ಡ-ಬೇಕು
ಹೋರಾ-ಡ-ಲೇ-ಬೇಕು
ಹೋರಾಡಿ
ಹೋರಾ-ಡಿದ
ಹೋರಾ-ಡು-ತ್ತಿತ್ತು
ಹೋರಾ-ಡು-ತ್ತಿ-ದ್ದಾ-ನೆ-ಇ-ದ್ದ-ಕ್ಕಿ-ದ್ದಂತೆ
ಹೋರಾ-ಡು-ವುದನ್ನು
ಹೋಲಿ-ಸ-ಬ-ಹುದು
ಹೋಲಿ-ಸ-ಲಾಗಿದೆ
ಹೋಲಿಸಿ
ಹೋಲಿ-ಸಿ-ದರೆ
ಹೋಲಿ-ಸುತ್ತ
ಹೋಲಿ-ಸು-ತ್ತಿ-ದ್ದರು
ಹೋಲಿ-ಸು-ತ್ತಿ-ದ್ದೀ-ರಲ್ಲ
ಹೋಲು-ತ್ತಿ-ರು-ವುದನ್ನು
ಹೋಳಿ
ಹೋಳು-ಗಳನ್ನು
ಹೌದಪ್ಪ
ಹೌದಲ್ಲ
ಹೌದಾ
ಹೌದಾ-ದಲ್ಲಿ
ಹೌದು
ಹೌದೆ
ಹೌದೆಂದ
ಹೌದೆ-ನ್ನಿ-ಸಿತು
ಹೌದೇ-ನಮ್ಮ
ಹೌದೋ
ಹೌಹಾ-ರಿ-ಬಿ-ಟ್ಟ-ರಲ್ಲ
ಹ್ಞೂ
ಹ್ಯಾಮಿ-ಲ್ಟನ್
ಹ್ಯೂಮ್
ಹ್ರಸ್ವ-ರೂಪ
್ಣ
}
