
\chapter{ಮನದೊಳೆದ್ದ ಮಹೋದ್ದೇಶ}

\noindent

ಸ್ವಾಮೀಜಿ ಬಾರಾನಗೋರ್ ಮಠಕ್ಕೆ ಹಿಂದಿರುಗಿ ಬಂದಿದ್ದಾರೆ. ಹಿಂದಿನಂತೆಯೇ ತಮ್ಮ ಸೋದರಸಂನ್ಯಾಸಿಗಳ ನೆಚ್ಚಿನ ನಾಯಕನಾಗಿ, ಗೆಳೆಯನಾಗಿ, ಮಾರ್ಗದರ್ಶಕನಾಗಿ ವಾಸವಾಗಿ ದ್ದಾರೆ. ಅವರ ಆಗಮನದಿಂದಾಗಿ ಮಠದಲ್ಲಿ ಹೊಸ ಉತ್ಸಾಹವುಂಟಾಗಿದೆ. ಶ್ರೀರಾಮಕೃಷ್ಣರ ದಿವ್ಯವಚನಗಳ ನೆನಪುಗಳನ್ನು ಜಾಗೃತಗೊಳಿಸುವುದರ ಮೂಲಕ, ಹಾಗೂ ತಮ್ಮ ಪರಿವ್ರಾಜಕ ಜೀವನದ ವೈವಿಧ್ಯಮಯ ಅನುಭವಗಳನ್ನು ಸ್ವಾರಸ್ಯಕರವಾಗಿ ನಿರೂಪಿಸುವುದರ ಮೂಲಕ ಅವರು ತಮ್ಮ ಸೋದರಸಂನ್ಯಾಸಿಗಳ ಬುದ್ಧಿ-ಹೃದಯಗಳನ್ನು ಪ್ರಜ್ವಲಗೊಳಿಸಿದರು. ಅವರ ಅಪಾರ ಜ್ಞಾನ ಹಾಗೂ ಅನುಭವಗಳು ಆ ಯುವಸಂನ್ಯಾಸಿಗಳ ಬೌದ್ಧಿಕ ಬೆಳವಣಿಗೆಗೆ ಪ್ರಚೋದಕವಾಗಿ ದ್ದುವು. ಒಂದು ದೃಷ್ಟಿಯಿಂದ, ಆ ಯುವಕರು ತಮ್ಮ ಕಾಲೇಜು ವಿದ್ಯಾಭ್ಯಾಸವನ್ನು ಅರ್ಧಕ್ಕೆ ಬಿಟ್ಟುಬಂದಿದ್ದರಾದರೂ ಅದರಿಂದ ಅವರಿಗಾದ ನಷ್ಟ ಅಗಣ್ಯ ಎಂದೇ ಹೇಳಬೇಕು. ಏಕೆಂದರೆ, ವಿಶ್ವವಿದ್ಯಾಲಯದ ಜ್ಞಾನಕ್ಕಿಂತ ಮಿಗಿಲಾದ ಜ್ಞಾನಭಂಡಾರವೊಂದರ ಸಾನ್ನಿಧ್ಯವೇ ಅವರಿಗೆ ದೊರಕಿತ್ತು. ಯಾವುದೇ ವಿಷಯದ ಬಗ್ಗೆ ಸಂಪೂರ್ಣ ಪ್ರಭುತ್ವದಿಂದ ಮಾತನಾಡಬಲ್ಲ ಸ್ವಾಮೀಜಿಯ ಮಾರ್ಗದರ್ಶನ ಅವರಿಗೆ ದೊರಕಿತ್ತು. ಎಂದರೆ ಸ್ವಾಮೀಜಿ ತಮ್ಮ ಸೋದರ ಸಂನ್ಯಾಸಿಗಳ ಮಾಸ್ತರರಾದರೆಂದಲ್ಲ. ಅವರು ಯಾವುದಾದರೊಂದು ವಿಷಯದ ಬಗ್ಗೆ ಮಾತ ನಾಡಲಾರಂಭಿಸಿದರೆ ಇತರರೆಲ್ಲ ಸುತ್ತಲೂ ಕುಳಿತು ತನ್ಮಯರಾಗಿ ಕೇಳುತ್ತಿದ್ದರು. ಕೆಲವೊಂದು ಸಲವಂತೂ ಒಂದೇ ವಿಷಯವೇ ಎಷ್ಟೋ ದಿನಗಳವರೆಗೆ ಮುಂದುವರಿಯುತ್ತಿತ್ತು. ತಮ್ಮ ಹಲವಾರು ಅನುಭವಗಳನ್ನು ಹಾಗೂ ತಮ್ಮ ಮನದಲ್ಲಿ ರೂಪತಳೆಯುತ್ತಿದ್ದ ವಿನೂತನ ಭಾವನೆಗಳನ್ನು ಸ್ವಾಮೀಜಿ ವರ್ಣಿಸುತ್ತಿದ್ದರು.

ಮಠದಲ್ಲಿ ಸ್ವಾಮೀಜಿಯ ಸಾನ್ನಿಧ್ಯದ ಆವಶ್ಯಕತೆ ಬಹಳವಾಗಿಯೇ ಇತ್ತು. ಏಕೆಂದರೆ, ಶ್ರೀರಾಮಕೃಷ್ಣರ ಅವತಾರೋದ್ದೇಶವನ್ನು ಕಾರ್ಯಗತಗೊಳಿಸುವ ಹೊಣೆಗಾರಿಕೆ ಮುಖ್ಯವಾಗಿ ಅವರದ್ದೇ ಅಲ್ಲವೆ? ಆದರೆ ಪರಿಸ್ಥಿತಿ ಮಾತ್ರ ಈ ಸಂನ್ಯಾಸಿಗಳಿಗೆ ಯಾವ ರೀತಿಯಲ್ಲೂ ಅನು ಕೂಲಕರವಾಗಿರಲಿಲ್ಲ. ಮೊದಲನೆಯದಾಗಿ, ಈ ಸಂನ್ಯಾಸ ಸಂಪ್ರದಾಯವೇ ಬಂಗಾಳಕ್ಕೆ ತೀರ ಹೊಸದು. ಆದ್ದರಿಂದ ಅವರಿಗೆ ಸಮಾಜದಿಂದ ಯಾವುದೇ ಬಗೆಯ ಸಹಕಾರ ಸಿಗುವುದಿರಲಿ, ಎಲ್ಲ ರೀತಿಯಿಂದಲೂ ವಿರೋಧವೇ ಎದುರಾಗುತ್ತಿತ್ತು. ಇದರ ಜೊತೆಗೆ ಮಠದ ಆರ್ಥಿಕ ಪರಿಸ್ಥಿತಿ ತೀರಾ ಚಿಂತಾಜನಕವಾಗಿತ್ತು. ನಾವು ನೋಡಿದಂತೆ, ಮೊದಲಿನಿಂದಲೂ ಮಠದ ಸ್ಥಿತಿಗತಿಗಳು ಮಠದ ಕಟ್ಟಡದ ಅವಸ್ಥೆಯಂತೆಯೇ ಇದ್ದುವು. ಅಲ್ಲದೆ, ಮಠದ ಆಧಾರಸ್ತಂಭ ಗಳಂತಿದ್ದ ಬಲರಾಮಬಾಬುವೂ ಸುರೇಂದ್ರನಾಥಮಿತ್ರನೂ ಕ್ರಮವಾಗಿ ೧೮೯೦ರ ಏಪ್ರಿಲ್ ಮತ್ತು ಮೇ ತಿಂಗಳುಗಳಲ್ಲಿ ತೀರಿಕೊಂಡರು. ಇದರಿಂದಾಗಿ ಹಿಂದೆಂದಿಗಿಂತ ಹೆಚ್ಚು ದುಸ್ಥಿತಿ ಏರ್ಪಟ್ಟಿತು. ಜೊತೆಗೆ ಶ್ರೀರಾಮಕೃಷ್ಣರ ಪವಿತ್ರ ಅವಶೇಷಗಳನ್ನು ಕಾಪಾಡಿಕೊಳ್ಳುವ ಹಾಗೂ ಸಂಸ್ಥಾಪಿಸುವ ಹೊಣೆ ಈ ಸಂನ್ಯಾಸಿಗಳ ಮೇಲಿದೆ. ಈಗ ಮಠದ ಖರ್ಚನ್ನು ವಹಿಸಿಕೊಳ್ಳುವವರು ಯಾರಿದ್ದಾರೆ? ಈ ಸಂನ್ಯಾಸಿಗಳೆಲ್ಲ ಬೇರೆಲ್ಲಿಗೆ ಹೋಗುವುದು? ಶ್ರೀರಾಮಕೃಷ್ಣರನ್ನು ಎಲ್ಲಿ ಪ್ರತಿಷ್ಠಾಪಿಸುವುದು? ಅವರಿಗೀಗ ದಿಕ್ಕು ತೋಚದಂತಾಯಿತು. ಗಿರೀಶ, ಮಹೇಂದ್ರನಾಥ ಮತ್ತಿತರ ಭಕ್ತರೇನೋ ತಮ್ಮಿಂದಾದಷ್ಟು ನೆರವು ನೀಡಿದರು. ಆದರೆ ಅವರಾದರೂ ಹೆಚ್ಚು ಶ್ರೀಮಂತರೇನಲ್ಲ; ಅವರಿಂದ ಹೆಚ್ಚು ಹಣವನ್ನು ನಿರೀಕ್ಷಿಸುವಂತೆಯೇ ಇರಲಿಲ್ಲ.

ಈ ಎಲ್ಲ ಕಷ್ಟಗಳ ನಡುವೆ ಸ್ವಾಮೀಜಿಯ ಮನಸ್ಸಿನಲ್ಲಿ ಬೇರೊಂದು ಚಿಂತೆ ಕೊರೆಯುತ್ತಿದೆ. ಬಂಗಾಳದಲ್ಲಿ ಶ್ರೀರಾಮಕೃಷ್ಣರ ಹೆಸರನ್ನು ಶಾಶ್ವತಗೊಳಿಸುವಂತಹ ಒಂದು ಸ್ಮಾರಕವನ್ನು, ಕಡೆಗೆ ಗಂಗಾತೀರದಲ್ಲೊಂದು ದೇವಾಲಯವನ್ನಾದರೂ ನಿರ್ಮಿಸಬೇಕು ಎಂಬ ಆಕಾಂಕ್ಷೆ ಅವರಲ್ಲಿ ತೀವ್ರವಾಗಿದೆ. ಆದರೆ ಅದಕ್ಕೆ ಬೇಕಾಗುವಷ್ಟು ಧನಸಂಗ್ರಹ ಮಾಡುವ ಪರಿಯೆಂತು? ಆ ಪಾಳುಬಿದ್ದ ಕಟ್ಟಡದಲ್ಲಿ, ಒಂದು ಹೊತ್ತಿನ ತುತ್ತಿಗೇ ಗತಿಯಿಲ್ಲದೆ ವಾಸಿಸುತ್ತಿರುವ ಈ ಭಿಕಾರಿ ಸಂನ್ಯಾಸಿಗಳನ್ನು ಕಂಡು ಯಾರುತಾನೆ ಸಾವಿರಗಟ್ಟಲೆ ರೂಪಾಯಿ ಕೊಡಲು ಮುಂದಾದಾರು? ಅನಂತ ಶಕ್ತಿಯಡಗಿರುವ ಪವಿತ್ರ ಚೇತನರು ಇವರು ಎಂದಾಗಲಿ, ಇವರಿಂದ ಮಹಾಸಂಸ್ಥೆ ಯೊಂದು ರೂಪುಗೊಳ್ಳಲಿದೆ ಎಂದಾಗಲಿ ಜನರು ಹೇಗೆ ಅರಿಯಬಲ್ಲರು? ಅಂತೂ ಸ್ವಾಮೀಜಿ ಬಂಗಾಳದಲ್ಲಿ ಯಾವ ಭರವಸೆಯನ್ನೂ ಕಾಣದೆ, ತಮ್ಮ ಮಿತ್ರರಾದ ಪ್ರಮದಬಾಬುಗಳಿಂದ ನೆರವನ್ನೂ ಸಲಹೆಯನ್ನೂ ಕೋರಿ, (ಸುರೇಂದ್ರ ತೀರಿಕೊಂಡ ಮರುದಿನದಂದು) ದೀರ್ಘ ಪತ್ರವೊಂದನ್ನು ಬರೆಯುತ್ತಾರೆ:

“... ಮಹಾಶಯರೆ, ಈ ಮೊದಲೇ ನಿಮಗೆ ಹೇಳಿರುವಂತೆ ನಾನು ಶ್ರೀರಾಮಕೃಷ್ಣರ ದಾಸಾನುದಾಸ. ತನುಮನಗಳನ್ನು ಅವರ ಪಾದಗಳಲ್ಲಿ ಸಮರ್ಪಿಸಿಕೊಂಡಿರುವ ನಾನು, ಅವರ ಆಜ್ಞೆಯನ್ನು ಕಡೆಗಣಿಸಲಾರೆ. ಅವರಿಂದಲೇ ಸಂಘಟಿಸಲ್ಪಟ್ಟ ಈ ಸಂನ್ಯಾಸಿಗಳ ಸಂಘಕ್ಕೆ, ಏನಾದರಾಗಲಿ, ನನ್ನನ್ನು ನಾನು ಅರ್ಪಿಸಿಕೊಳ್ಳಬೇಕು. ನನ್ನ ಪಾಲಿಗೆ ಸ್ವರ್ಗವೇ ಬರಲಿ, ನರಕವೇ ಎದುರಾಗಲಿ, ಮೋಕ್ಷವೇ ದೊರಕಲಿ ಅಥವಾ ಇನ್ನೇನೇ ಬಂದೊದಗಲಿ, ನಾನು ಈ ಸಂಘವನ್ನು ಸಂರಕ್ಷಿಸಬೇಕು ಎಂಬುದೇ ಶ್ರೀರಾಮಕೃಷ್ಣರ ಆದೇಶವಾಗಿತ್ತು.

“ಸರ್ವಸಂಗ ಪರಿತ್ಯಾಗ ಮಾಡಿದ ಅವರ ಈ ಎಲ್ಲ ಶಿಷ್ಯರೂ ಒಟ್ಟಾಗಿರಬೇಕು ಎಂಬುದು ಶ್ರೀರಾಮಕೃಷ್ಣರ ಆಜ್ಞೆಯಾಗಿತ್ತು. ಮತ್ತು, ಇವರನ್ನೆಲ್ಲ ಒಟ್ಟಾಗಿರಿಸಬೇಕು ಎಂದು ಅವರೇ ನನ್ನನ್ನು ನಿಯಮಿಸಿದ್ದಾರೆ. ನಾವುಗಳು ತೀರ್ಥಾಟನೆಗೆಂದು ಅಲ್ಲಿ-ಇಲ್ಲಿ ಹೋಗಬಹುದು. ಆದರೆ ಅದು ತಾತ್ಕಾಲಿಕ ಮಾತ್ರ; ಮತ್ತೆ ಮಠಕ್ಕೆ ಹಿಂದಿರುಗಲೇಬೇಕು. ಸ್ವತಃ ಶ್ರೀರಾಮಕೃಷ್ಣರ ಅಭಿಪ್ರಾಯವೇನಾಗಿತ್ತೆಂದರೆ, ಯಾರು ಆಧ್ಯಾತ್ಮಿಕ ಜೀವನದಲ್ಲಿ ಅತ್ಯುನ್ನತ ಮಟ್ಟದ ಪರಿ ಪೂರ್ಣತೆಯನ್ನು ಸಿದ್ಧಿಸಿಕೊಂಡಿದ್ದಾರೆಯೋ ಅವರು ಮಾತ್ರ ಸದಾ ಪರಿವ್ರಾಜಕರಾಗಿ ಓಡಾಡಿ ಕೊಂಡಿರಬಹುದು; ಅಲ್ಲಿಯವರೆಗೆ ಒಂದು ಜಾಗದಲ್ಲಿ ನೆಲೆನಿಂತು ಆಧ್ಯಾತ್ಮಿಕ ಸಾಧನೆಯಲ್ಲಿ ನಿರತರಾಗಿರುವುದೇ ಸೂಕ್ತ–ಎಂದು. ಆದ್ದರಿಂದ ಅವರ ಈ ಆದೇಶವನ್ನು ಪರಿಪಾಲಿಸು ವುದಕ್ಕಾಗಿಯೇ ಅವರ ಸಂನ್ಯಾಸೀಶಿಷ್ಯರೆಲ್ಲ ಬಾರಾನಗೋರಿನ ಪಾಳುಬಿದ್ದ ಮನೆಯಲ್ಲಿ ಕಲೆತು ನೆಲೆಸಿದ್ದಾರೆ. ಮತ್ತು, ಶ್ರೀರಾಮಕೃಷ್ಣರ ಇಬ್ಬರು ಗೃಹಸ್ಥಭಕ್ತರಾದ ಬಲರಾಮ ಬೋಸ್ ಹಾಗೂ ಸುರೇಂದ್ರನಾಥ ಮಿತ್ರ ಇವರು ಇಲ್ಲಿಯವರೆಗೆ ಈ ಸಂನ್ಯಾಸಿಗಳ ಆಹಾರದ ಹಾಗೂ ಮನೆ ಬಾಡಿಗೆಯ ಖರ್ಚನ್ನು ವಹಿಸಿಕೊಂಡು ಬಂದರು.

“ಹಲವಾರು ಕಾರಣಗಳಿಂದಾಗಿ, ಶ್ರೀರಾಮಕೃಷ್ಣರ ಶರೀರವನ್ನು ದಹನ ಮಾಡಬೇಕಾಯಿತು. ಇದೊಂದು ಆಕ್ಷೇಪಣೀಯ ಕೃತ್ಯ ಎಂಬುದರಲ್ಲಿ ಸಂದೇಹವಿಲ್ಲ.\footnote{*ನೋಡಿ: ಅನುಬಂಧ ೭.} ಅಂತೂ, ಅವರ ಅವಶೇಷ ಗಳನ್ನು ರಕ್ಷಿಸಿಡಲಾಗಿದೆ. ಈಗ ಇವುಗಳನ್ನು ಗಂಗೆಯ ದಡದ ಮೇಲೆ ಒಂದು ಸರಿಯಾದ ಜಾಗದಲ್ಲಿ ವಿಧಿಯುಕ್ತವಾಗಿ ಪ್ರತಿಷ್ಠಾಪಿಸಿ ಒಂದು ದೇವಸ್ಥಾನವನ್ನು ಕಟ್ಟಿದರೆ, ನಾವು ನಮ್ಮ ತಲೆಗಳ ಮೇಲಿರುವ ಪಾಪದ ಹೊರೆಯನ್ನು ಕೆಲಮಟ್ಟಿಗಾದರೂ ಪರಿಹರಿಸಿಕೊಳ್ಳಬಹುದು ಎಂಬುದು ನನ್ನ ಭಾವನೆ. ಅವರ ಅವಶೇಷಗಳು, ಅವರ ಪೀಠ ಹಾಗೂ ಅವರ ಭಾವಚಿತ್ರ –ಇವುಗಳಿಗೆ ದಿನವೂ ವಿಧಿಯುಕ್ತವಾಗಿ ಪೂಜೆ ನಡೆಯುತ್ತಿದೆ. ಬಹುಶಃ ನಿಮಗೆ ತಿಳಿದಿರುವಂತೆ, ನನ್ನೊಬ್ಬ ಸೋದರಸಂನ್ಯಾಸಿಯೇ ಈ ಪೂಜಾಕಾರ್ಯದಲ್ಲಿ ಹಗಲಿರುಳೂ ನಿರತರಾಗಿದ್ದಾರೆ. ಈ ಪೂಜಾದಿಗಳ ಖರ್ಚನ್ನೂ ಮೇಲೆ ಹೇಳಿದ ಇಬ್ಬರು ಮಹಾನುಭಾವರೇ ವಹಿಸಿಕೊಳ್ಳುತ್ತಿದ್ದರು. ಈಗ ಯಾರ ದಿವ್ಯಜನನದಿಂದಾಗಿ ಬಂಗಾಳಿಗಳ ಜನಾಂಗವೇ ಕೃತಾರ್ಥವಾಯಿತೋ, ಸಮಸ್ತ ಬಂಗಾಳವೇ ಪಾವನವಾಯಿತೋ, ಅಂತಹ ಶ್ರೀರಾಮಕೃಷ್ಣರ ಜನ್ಮಸ್ಥಳದ ಸಮೀಪದಲ್ಲಿ ಎಲ್ಲಾದ ರೊಂದು ಕಡೆ, ಒಂದು ಸ್ಮಾರಕವನ್ನು ನಿರ್ಮಾಣ ಮಾಡಲು ನಮ್ಮಿಂದ ಇನ್ನೂ ಸಾಧ್ಯವಾಗಲಿಲ್ಲ ವೆಂದರೆ ಇದಕ್ಕಿಂತ ಶೋಚನೀಯವಾದದ್ದು ಇನ್ನೇನಿದೆ? ಅಲ್ಲದೆ, ಭಾರತೀಯರನ್ನು ಪಾಶ್ಚಾತ್ಯ ನಾಗರಿಕತೆಯ ಥಳಕಿನ ವಿಲಾಸಗಳಿಂದ ಪಾರು ಮಾಡುವುದಕ್ಕಾಗಿಯೇ ಜನ್ಮವೆತ್ತಿದ, ಮತ್ತು ಆ ಕಾರಣಕ್ಕಾಗಿಯೇ ತಮ್ಮ ಹೆಚ್ಚಿನ ಸಂನ್ಯಾಸೀ ಶಿಷ್ಯರನ್ನು ಕಾಲೇಜು ಯುವಕರಲ್ಲಿಯೇ ಆರಿಸಿ ಕೊಂಡಂತಹ ಶ್ರೀರಾಮಕೃಷ್ಣರಿಗಾಗಿ ಒಂದು ಸ್ಮಾರಕವನ್ನು ನಿರ್ಮಾಣ ಮಾಡಲು ನಮಗಿನ್ನೂ ಸಾಧ್ಯವಾಗಲಿಲ್ಲವೆಂದರೆ, ಇದಕ್ಕಿಂತ ವಿಷಾದನೀಯವಾದದ್ದೇನಿದೆ?

“ಬಲರಾಮ್ ಬಾಬು ಹಾಗೂ ಸುರೇಂದ್ರಬಾಬು ಇವರಿಬ್ಬರಿಗೂ, ಗಂಗೆಯ ದಡದ ಮೇಲೊಂದು ಜಾಗವನ್ನು ಕೊಂಡುಕೊಳ್ಳಲೇಬೇಕು, ಅಲ್ಲಿ ಶ್ರೀರಾಮಕೃಷ್ಣರ ಅವಶೇಷಗಳನ್ನು ಪ್ರತಿಷ್ಠಾಪಿಸಿ ಒಂದು ಸ್ಮಾರಕವನ್ನು ಕಟ್ಟಿಸಬೇಕು, ಹಾಗೂ ಈ ಸಂನ್ಯಾಸೀಶಿಷ್ಯರೆಲ್ಲ ಅಲ್ಲೇ ವಾಸವಾಗಿರುವಂತಾಗಬೇಕು ಎಂಬ ತೀವ್ರವಾದ ಇಚ್ಛೆಯಿತ್ತು. ಈ ಕಾರ್ಯಕ್ಕಾಗಿ ಒಂದು ಸಾವಿರ ರೂಪಾಯಿಗಳನ್ನು ಕೊಡಲು ಸುರೇಂದ್ರಬಾಬು ಮುಂದೆ ಬಂದಿದ್ದರು; ಅಲ್ಲದೆ ಇನ್ನೂ ಹೆಚ್ಚಿನ ಹಣವನ್ನು ನೀಡುವುದಾಗಿ ವಾಗ್ದಾನ ಮಾಡಿದ್ದರು. ಆದರೆ ಅಚಿಂತ್ಯನಾದ ಭಗವಂತನ ಇಚ್ಛೆ ಯಂತೆ ಅವರು ನಿನ್ನೆ ರಾತ್ರಿ (ಮೇ ೨೫ರಂದು) ಈ ಲೋಕವನ್ನೇ ತ್ಯಜಿಸಿ ಹೊರಟುಹೋದರು.

“ಈಗ ಈ ಸಂನ್ಯಾಸಿಗಳೆಲ್ಲ ಶ್ರೀರಾಮಕೃಷ್ಣರ ಅಸ್ಥಿಯನ್ನು ತೆಗೆದುಕೊಂಡು ಎಲ್ಲಿಗೆ ಹೋಗು ವುದೋ ತಿಳಿಯದಾಗಿದೆ. ನಮ್ಮ ಬಂಗಾಳೀ ಜನ ಮಣಗಟ್ಟಲೆ ಮಾತನಾಡುವುದರಲ್ಲಿ ಗಟ್ಟಿಗರೇ ಹೊರತು ಕಿಂಚಿತ್ತೂ ಕಾರ್ಯರೂಪಕ್ಕೆ ತರುವವರಲ್ಲ ಎಂಬುದು ನಿಮಗೆ ಗೊತ್ತಲ್ಲ! ಈ ಯುವಕರೆಲ್ಲ ಸಂನ್ಯಾಸಿಗಳು; ಈ ಕ್ಷಣದಲ್ಲಿ ಎಲ್ಲದರಿಂದಲೂ ಬಿಡಿಸಿಕೊಂಡು ತಮ್ಮತಮ್ಮ ದಾರಿ ಹಿಡಿದು ಹೊರಟುಬಿಡಬಲ್ಲರು. ಆದರೆ ಇವರನ್ನೆಲ್ಲ ಒಟ್ಟಾಗಿರುವಂತೆ ನೋಡಿಕೊಳ್ಳಬೇಕಾ ದವನು ನಾನು; ಇವರ ಸೇವಕ ನಾನು. ನನಗೆ ತುಂಬ ದುಃಖವಾಗುತ್ತಿದೆ–ಶ್ರೀರಾಮಕೃಷ್ಣರ ಅವಶೇಷಗಳನ್ನು ಪ್ರತಿಷ್ಠಾಪಿಸಲು ಒಂದು ತುಂಡು ಜಾಗ ಸಿಗದೆ ಹೋಯಿತಲ್ಲ! ಇದನ್ನು ಭಾವಿಸಿದರೇ ನನಗೆ ಎದೆ ಹಿಂಡಿದಂತಾಗುತ್ತದೆ.

“ಕಲ್ಕತ್ತದಂತಹ ಸ್ಥಳದಲ್ಲಿ ಸಾವಿರ ರೂಪಾಯಿಗೆಲ್ಲ ಜಾಗ ಕೊಂಡುಕೊಂಡು ದೇವಸ್ಥಾನ ವನ್ನು ಕಟ್ಟಿಸಲು ಸಾಧ್ಯವೇ ಇಲ್ಲ. ನಮಗೆ ಬೇಕಾಗಬಹುದಾದ ಜಾಗಕ್ಕೆ ಕನಿಷ್ಠಪಕ್ಷ ಐದರಿಂದ ಏಳುಸಾವಿರ ರೂಪಾಯಿಗಳಾದರೂ ಬೇಕಾಗುತ್ತದೆ.”

“ಶ್ರೀರಾಮಕೃಷ್ಣರ ಶಿಷ್ಯರ ಪಾಲಿಗೆ ಈಗ ಉಳಿದಿರುವ ಏಕಮಾತ್ರ ಸ್ನೇಹಿತ ಹಾಗೂ ಪೋಷಕ ರೆಂದರೆ ನೀವೊಬ್ಬರೇ. ವಾಯವ್ಯ ಪ್ರಾಂತದಲ್ಲಿ ನಿಮ್ಮ ಕೀರ್ತಿಯೂ ಸ್ಥಾನಮಾನಗಳೂ ನಿಜಕ್ಕೂ ಜನಜನಿತವಾಗಿವೆ. ನಿಮ್ಮ ಪರಿಚಯಸ್ಥರ ಸಂಖ್ಯೆಯೂ ಅಲ್ಲಿ ಹೇರಳವಾಗಿದೆ. ಆದ್ದರಿಂದ, ನಿಮಗೆ ಪರಿಚಯವಿರುವ ಅನುಕೂಲಸ್ಥ ದೈವಭಕ್ತ ಜನರಿಂದ ಈ ಕಾರ್ಯಕ್ಕಾಗಿ ವಂತಿಗೆ ಕೇಳುವ ಬಗ್ಗೆ ಆಲೋಚಿಸಬೇಕೆಂದು ನಿಮ್ಮನ್ನು ನಾನು ಬೇಡಿಕೊಳ್ಳುತ್ತೇನೆ. ನಮ್ಮ ಈ ಆಲೋಚನೆಯು ಯುಕ್ತ ವಾದದ್ದೆಂದು ನೀವು ಭಾವಿಸುವುದಾದರೆ, ನಿಮ್ಮ ಅನುಮತಿಯ ಮೇರೆಗೆ ನಾನು ತಕ್ಷಣ ನಿಮ್ಮಲ್ಲಿಗೆ ಹೊರಟುಬರುತ್ತೇನೆ. ಈ ಒಂದು ಉದಾತ್ತ ಕಾರ್ಯಕ್ಕಾಗಿ, ನನ್ನ ಪಾಲಿನ ಭಗವಂತನ ಹಾಗೂ ಅವನ ಮಕ್ಕಳ ಸಲುವಾಗಿ ಮನೆಯಿಂದ ಮನೆಗೆ ಹೋಗಿ ಭಿಕ್ಷೆ ಬೇಡುವುದಕ್ಕೂ ನನಗೆ ಸಂಕೋಚವಿಲ್ಲ.

“ನೀವು ನನ್ನನ್ನು ಕೇಳಬಹುದು–‘ನೀವು ಸಂನ್ಯಾಸಿಗಳಲ್ಲವೆ, ಇಂಥ ವಿಚಾರಗಳನ್ನೆಲ್ಲ ಹಚ್ಚಿ ಕೊಂಡು ಏಕೆ ತೊಂದರೆಪಡುತ್ತೀರಿ?’ ಅಂತ. ಅದಕ್ಕೆ ನಾನು ಹೇಳುತ್ತೇನೆ–‘ನಾನು ಶ್ರೀರಾಮ ಕೃಷ್ಣರ ದಾಸ; ಆದ್ದರಿಂದ ಅವರು ಜನ್ಮ ತಳೆದು ಅಭೂತಪೂರ್ವ ಆಧಾತ್ಮಿಕ ಸಾಧನೆ ಮಾಡಿದಂತಹ ಈ ನಾಡಿನಲ್ಲಿ ಅವರ ಹೆಸರನ್ನು ಅಮರಗೊಳಿಸುವುದಕ್ಕೋಸ್ಕರ ಹಾಗೂ ಅವರ ಶಿಷ್ಯರಿಗೆ ಅವರ ಮಹಾದರ್ಶಗಳನ್ನು ಆಚರಣೆಗೆ ತರುವಲ್ಲಿ ಕಿಂಚಿತ್ತಾದರೂ ಸಹಾಯ ಮಾಡು ವುದಕ್ಕೋಸ್ಕರ, ಕಡೆಗೆ ಕಳ್ಳತನ-ದರೋಡೆ ಮಾಡಬೇಕಾಗಿಬಂದರೆ ಅದಕ್ಕೂ ನಾನು ಸಿದ್ಧ. ಅವರ ಈ ನಾಡಿನಲ್ಲಿ ಸ್ಮಾರಕರೂಪವಾದ ದೇಗುಲವನ್ನು ಕಟ್ಟಲು ನಮ್ಮಿಂದ ಸಾಧ್ಯವಾಗದಿದ್ದಲ್ಲಿ ಅದು ಅತ್ಯಂತ ಶೋಚನೀಯ ವಿಷಯ. ಬಂಗಾಳದ ಪರಿಸ್ಥಿತಿ ನಿಜಕ್ಕೂ ಅವಮಾನಕರವಾಗಿದೆ. ಇಲ್ಲಿನ ಜನ ತ್ಯಾಗವೆಂಬುದರ ಅರ್ಥವನ್ನು ಕನಸಿನಲ್ಲೂ ಕಂಡರಿಯರು. ವಿಲಾಸ ಹಾಗೂ ಇಂದ್ರಿಯ ಭೋಗವೆಂಬುದು ಈ ಜನತೆಯ ಸತ್ತ್ವವನ್ನೇ ಹಿಂಡಿ ಹೀರಿಬಿಟ್ಟಿದೆ. ಭಗವಂತ ಈ ಜನತೆಗೆ ಒಂದಿಷ್ಟು ತ್ಯಾಗಬುದ್ಧಿಯನ್ನು, ಒಂದಿಷ್ಟು ಪಾರಮಾರ್ಥಿಕ ಬುದ್ಧಿಯನ್ನು ಕರುಣಿಸಲಿ!... ”

ಸ್ವಾಮೀಜಿಯ ಈ ದೀರ್ಘ, ವಿನಂತಿಪೂರ್ವಕವಾದ ಪತ್ರಕ್ಕೆ ಪ್ರಮದಬಾಬುಗಳಿಂದ ಬಂದ ಪ್ರತಿಕ್ರಿಯೆ ಅದೆಷ್ಟು ತಣ್ಣಗಿತ್ತು ಎಂಬುದನ್ನು ನಾವು ಸ್ವಾಮೀಜಿ ಅವರಿಗೆ ಬರೆಯುವ ಮರು ಪತ್ರದ ಧಾಟಿಯಿಂದಲೇ ತಿಳಿದುಕೊಳ್ಳಬಹುದು. ಅದರಲ್ಲಿ ಸ್ವಾಮೀಜಿ ಬರೆಯುತ್ತಾರೆ: “ನಿಮ್ಮ ಸಲಹೆಗಳು ನಿಜಕ್ಕೂ ಯುಕ್ತಿಯುಕ್ತವಾಗಿವೆಯೆಂಬುದರಲ್ಲಿ ಸಂದೇಹವಿಲ್ಲ. ಎಲ್ಲವೂ ಭಗ ವಂತನ ಇಚ್ಛೆಯಂತೆಯೇ ಆಗುವುದೆಂಬ ಮಾತು ತುಂಬ ಸತ್ಯ. ನಾವೂ ಕೂಡ ಈಗ ಚಿಕ್ಕಚಿಕ್ಕ ಗುಂಪುಗಳಾಗಿ ಬೇರೆಬೇರೆ ಕಡೆಗೆ ಹೊರಟುಬಿಡಲಿದ್ದೇವೆ... ” ಪ್ರಮದಬಾಬುಗಳು ಸಹಾಯಕ್ಕೆ ಬದಲಾಗಿ ಸಲಹೆ ನೀಡಿದ್ದಾರೆಂಬುದು ಈ ಪತ್ರದಿಂದ ಸ್ಪಷ್ಟವಾಗುತ್ತದೆ. ಬಹುಶಃ ಅವರ ಸಲಹೆ ಈ ರೀತಿ ಇದ್ದಿರಬೇಕು: “ನಿಮ್ಮ ಉದ್ದೇಶವೇನೋ ಒಳ್ಳೆಯದೇ. ಆದರೆ ಭಗವಂತನ ಇಚ್ಛೆಯೊಂದಿದೆಯೆಲ್ಲ? ಅವನ ಕೃಪೆಯಾಗುವವರೆಗೆ ಕಾದಿರುವುದು ಒಳ್ಳೆಯದು...” ಎಂದು. ಪ್ರಮದಬಾಬುಗಳಿಂದ ಇಂತಹ ನಿರಾಶಾದಾಯಕ ಉತ್ತರ ಬಂದರೂ ಸ್ವಾಮೀಜಿ ಧೃತಿಗೆಡಲಿಲ್ಲ. ಇಂದಲ್ಲ ನಾಳೆ ತಮ್ಮ ಕನಸು ನನಸಾಗುವುದರಲ್ಲಿ ಅವರಿಗೆ ಸಂದೇಹವೇ ಇರಲಿಲ್ಲ. ಆದರೆ ತತ್ಕ್ಷಣದ ಪರಿಸ್ಥಿತಿ ಮಾತ್ರ ಪರೀಕ್ಷಾಕಾಲವಾಗಿ ಪರಿಣಮಿಸಿದೆ. ಆದ್ದರಿಂದ, ಸರಿಯಾದ ಕಾಲ ಕೂಡಿಬರುವವರೆಗೆ ತಾವೆಲ್ಲ ಬೇರೆಬೇರೆಯಾಗಿ ಹೊರಟು ಪರಿವ್ರಾಜಕ ಜೀವನ ನಡೆಸುವ ಆಲೋಚನೆಯನ್ನು ಸೋದರ ಸಂನ್ಯಾಸಿಗಳ ಮುಂದಿಟ್ಟರು. ಮಠದ ಆವಶ್ಯಕತೆಗಳನ್ನು ಭರಿಸಿ ಕೊಡುತ್ತಿದ್ದ ಇಬ್ಬರು ಪ್ರಮುಖ ಭಕ್ತರು ಕಣ್ಮುಚ್ಚಿದಾಗ, ಅವರಿಗೆ ಇದಲ್ಲದೆ ಬೇರೆ ಯಾವ ದಾರಿಯೂ ಕಾಣಲಿಲ್ಲ.

ಇಷ್ಟಾದರೂ ಸ್ವಾಮೀಜಿಯ ಉತ್ಸಾಹ ಮಾತ್ರ ಅದಮ್ಯವೆನ್ನಬೇಕು. ಇದೇ ಸಮಯದಲ್ಲಿ ಆಲ್ಮೋರಾದಲ್ಲಿದ್ದ ಸ್ವಾಮಿ ಶಾರದಾನಂದರಿಗೆ ಅವರು ಬರೆಯುವ ಎರಡು ಪತ್ರಗಳಲ್ಲಿ ಅದು ವ್ಯಕ್ತವಾಗುತ್ತದೆ: “ಗಿರೀಶ್ಚಂದ್ರ ಘೋಷ್ ಮಠದ ಖರ್ಚನ್ನು ವಹಿಸಿಕೊಂಡಿದ್ದಾನೆ. ತತ್ಕಾಲಕ್ಕೆ ಮಠದ ದಿನ ನಿತ್ಯದ ವ್ಯವಹಾರವನ್ನು ಚೆನ್ನಾಗಿ ನಿರ್ವಹಿಸಿಕೊಂಡು ಹೋಗಲಾಗುತ್ತಿದೆ...” ನೀನು ಇಲ್ಲಿಗೆ ಬಂದಿರುವುದು ಒಳ್ಳೆಯದು. ದಿವಂಗತ ಮಹೀಂದ್ರ ಮುಖರ್ಜಿಯವರ ಪತ್ನಿ ಒಂದು ಮಠವನ್ನು ಕಟ್ಟಿಸಿಕೊಡಲು ಶಕ್ತಿಮೀರಿ ಪ್ರಯತ್ನಿಸುತ್ತಿದ್ದಾಳೆ. ಅಲ್ಲದೆ ಸುರೇಂದ್ರನಾಥ ಮಿತ್ರ ಕೂಡ ಮಠಕ್ಕಾಗಿ ಒಂದು ಸಾವಿರ ರೂಪಾಯಿ ಬಿಟ್ಟುಹೋಗಿದ್ದಾನೆ. ಆದ್ದರಿಂದ ಸದ್ಯದಲ್ಲೇ ಗಂಗಾತೀರದಲ್ಲಿ ಒಂದು ಕಟ್ಟಡ ನಿರ್ಮಾಣವಾಗುವ ಸಂಭವವಿದೆ.” ಆದರೆ ಈ ಮಾತುಗಳೆಲ್ಲ ಅವರ ಆಶಾಪೂರ್ಣ ಮನೋಭಾವವನ್ನು ತೋರುತ್ತವೆಯೇ ಹೊರತು ಇನ್ನೇನೂ ಅಲ್ಲ. ಏಕೆಂದರೆ, ಅವರ ಈ ಮಾತುಗಳಾವುವೂ ತತ್ಕಾಲಕ್ಕೆ ಕಾರ್ಯರೂಪಕ್ಕೆ ಬರಲಿಲ್ಲ.

ಶ್ರೀರಾಮಕೃಷ್ಣರಿಗಾಗಿ ಸ್ಮಾರಕವನ್ನು ನಿರ್ಮಿಸುವ ಹಾಗೂ ಸೋದರಸಂನ್ಯಾಸಿಗಳಿಗೆ ತರಬೇತಿ ನೀಡುವ ಪ್ರಯತ್ನಗಳ ನಡುವೆಯೂ, ತಾವೂ ಪರಿವ್ರಾಜಕರಾಗಿ ಹೊರಟುಬಿಡಬೇಕೆಂಬ ಆಸೆ ಸ್ವಾಮೀಜಿಯ ಮನಸ್ಸಿನಲ್ಲಿ ಮತ್ತೆ ಭುಗಿಲೆದ್ದಿತು. ಹಿಮಾಲಯದತ್ತ ಹಾರಿಹೋಗಿಬಿಡಬೇಕು ಎಂದು ಅವರ ಹೃದಯ ಕಾತರಿಸಿತು. ಆದರೆ ದಿನೇದಿನೇ ಮಠದಲ್ಲಿ ಸೋದರಸಂನ್ಯಾಸಿಗಳೊಂದಿ ಗಿನ ಹಾಗೂ ಭಕ್ತರೊಂದಿಗಿನ ಸಂಬಂಧ ಹೆಚ್ಚು ಆತ್ಮೀಯವಾಗುತ್ತಿತ್ತು; ಮಠದ ಹೊಣೆಗಾರಿಕೆ ಇನ್ನಷ್ಟು ಹೆಚ್ಚಾಗಿ ಬೆಳೆಯುತ್ತಿತ್ತು. ಇವೆಲ್ಲ ಅವರು ಪರಿವ್ರಾಜಕ ಜೀವನವನ್ನು ಪ್ರಾರಂಭಿಸಲು ಅಡಚಣೆಗಳಾಗಿದ್ದುವು. ಹೀಗಿರುವಾಗ ತಾವು ತಮ್ಮ ಪರಿವ್ರಾಜಕ ಜೀವನವನ್ನುಕೈಗೊಳ್ಳುವುದರ ಮೂಲಕ ಮಾತ್ರವೇ ತಮ್ಮ ಆತ್ಮವಿಶ್ವಾಸವನ್ನು ಹೆಚ್ಚಿಸಿಕೊಳ್ಳಲೂ ಶ್ರೀರಾಮಕೃಷ್ಣರ ಸಂದೇಶ ಗಳನ್ನು ಯಶಸ್ವಿಯಾಗಿ ಬಿತ್ತರಿಸಲೂ ಸಾಧ್ಯ ಎಂಬುದು ಅವರಿಗೆ ದೃಢವಾಗುತ್ತ ಬಂದಿತು. ಆದ್ದರಿಂದ ಅವರು ಶಾರದಾನಂದರಿಗೆ ಬರೆಯುತ್ತಾರೆ: “ಹಿಮಾಲಯಕ್ಕೆ ಹಾರಿಹೋಗಲು ನಾನು ಕಾತರನಾಗಿದ್ದೇನೆ... ಈ ಪತ್ರ ನಿನ್ನ ಕೈಸೇರುವ ಮೊದಲೇ ನಾನು ಇಲ್ಲಿಂದ ಹೊರಟುಬಿಟ್ಟಿರು ತ್ತೇನೆ. ಮುಂದೇನು ಮಾಡಬೇಕೆಂಬುದರ ಬಗ್ಗೆ ನನ್ನದೇ ಆದ ಯೋಜನೆಗಳಿವೆ. ಆದರೆ ಅವೆಲ್ಲ ಗುಟ್ಟಿನ ವಿಚಾರ; ನಿನಗೀಗ ತಿಳಿಸುವಂತಿಲ್ಲ.”

ಅಂತೂ ಸ್ವಾಮೀಜಿ ಮತ್ತೆ ಪರಿವ್ರಾಜಕರಾಗಿ ಹೊರಟುಬಿಡಲು ಸಿದ್ಧರಾಗಿದ್ದಾರೆ. ಈ ಸಲ ತಮ್ಮ ನೆಚ್ಚಿನ ಸಂನ್ಯಾಸೀ ಬಂಧು ಸ್ವಾಮಿ ಅಖಂಡಾನಂದರನ್ನೂ ತಮ್ಮೊಡನೆ ಕರೆದೊಯ್ಯಲಿ ದ್ದರು. ಅವರ ಕೋರಿಕೆಯಂತೆ ಅಖಂಡಾನಂದಅರು ಕಾಶ್ಮೀರದಿಂದ ಹಿಂದಿರುಗಿಬಂದರು. ಇವರಿಗೆ ಹಿಮಾಲಯದ ದಾರಿಗಳು ಸ್ಥಳಗಳು, ಪದ್ಧತಿಗಳು–ಎಲ್ಲ ಸುಪರಿಚಿತ. ಆದ್ದರಿಂದ ತಮ್ಮ ಪ್ರಯಾಣ ಹೆಚ್ಚು ಸುಗಮವಾಗುವುದೆಂಬ ಭರವಸೆ ಸ್ವಾಮೀಜಿಯದಾಗಿತ್ತು.

ಈ ಸಲ ಸ್ವಾಮೀಜಿ ಯಾವ ಬಗೆಯ ದೃಢನಿರ್ಧಾರದಿಂದ ಹೊರಟಿದ್ದಾರೆ ಎಂಬುದಕ್ಕೆ, ಅವರು ತಮ್ಮ ಸೋದರಸಂನ್ಯಾಸಿಗಳ ಮುಂದೆ ಆಡುವ ಮಾತೇ ನಿದರ್ಶನ. ಅವರೆನ್ನುತ್ತಾರೆ: “ನಾನು ನನ್ನ ಸ್ಪರ್ಶಮಾತ್ರದಿಂದಲೇ ಮನುಷ್ಯನನ್ನು ಪರಿವರ್ತನೆ ಮಾಡುವಂತಾಗಬೇಕು. ಅಂತಹ ಉನ್ನತ ಸಾಕ್ಷಾತ್ಕಾರವಾಗುವವರೆಗೂ ನಾನು ಹಿಂದಿರುಗುವುದಿಲ್ಲ.” ಮಾತಿನಿಂದಾಗಲಿ ತರ್ಕ ದಿಂದಾಗಲಿ ಸುದೀರ್ಘ ವಿವರಣೆಯಿಂದಾಗಲಿ, ಯಾವ ಪರಿವರ್ತನೆಯನ್ನು ಉಂಟು ಮಾಡಲು, ಯಾವ ಭಾವನೆಯನ್ನು ಅರ್ಥಪಡಿಸಲು ಕೆಲವೊಮ್ಮೆ ಸಾಧ್ಯವಾಗದೋ, ಅಂಥದನ್ನು ಸಮರ್ಥ ವ್ಯಕ್ತಿಯೊಬ್ಬ ತನ್ನ ಸ್ಪರ್ಶಮಾತ್ರದಿಂದ ಸಾಧಿಸಬಲ್ಲ; ತನ್ನ ವಿಚಾರಲಹರಿಯು ಇನ್ನೊಬ್ಬರಲ್ಲಿ ಹರಿಯುವಂತೆ ಮಾಡಬಲ್ಲ. ಹಿಂದೆ ಸ್ವತಃ ಶ್ರೀರಾಮಕೃಷ್ಣರು ತಮ್ಮ ಅಲೌಕಿಕ ಸ್ಪರ್ಶದಿಂದ ನರೇಂದ್ರನ ಅಂತರಾಳವನ್ನೇ ಕಲಕಾಡಿ, ಅವನೊಳಗೆ ಅಪೂರ್ವ ಆಧ್ಯಾತ್ಮಿಕ ಜ್ಞಾನವನ್ನು ಪ್ರಚೋದಿಸಿರಲಿಲ್ಲವೆ? ಅಂತಹ ಅತ್ಯುನ್ನತ ಆಧ್ಯಾತ್ಮಿಕ ಶಕ್ತಿಯನ್ನು ಪಡೆದು ಕೊಳ್ಳಲು ಸ್ವಾಮೀಜಿ ಈಗ ಹೊರಟುನಿಂತಿದ್ದಾರೆ.

ಈ ಸಲ ಹೊರಡುವ ಮೊದಲು ಅವರು, ಶ್ರೀಮಾತೆ ಶಾರದಾದೇವಿಯವರ ಆಶೀರ್ವಾದವನ್ನು ಪಡೆದುಕೊಂಡು ಬರಲು ಇಚ್ಛಿಸಿದರು. ಈ ಸಮಯದಲ್ಲಿ ಶ್ರೀಮಾತೆಯವರು ಗಂಗೆಯ ಇನ್ನೊಂದು ಬದಿಯಲ್ಲಿರುವ ಘುಸೂರಿ ಎಂಬ ಹಳ್ಳಿಯಲ್ಲಿದ್ದರು. ಅಖಂಡಾನಂದರೊಂದಿಗೆ ಸ್ವಾಮೀಜಿ ಅಲ್ಲಿಗೆ ಹೋದರು; ಶ್ರೀಮಾತೆಯವರಿಗೆ ಸಮಸ್ಕರಿಸಿ, “ಅಮ್ಮಾ ನಾನು ಹಿಮಾಲ ಯದ ಕಡೆಗೆ ಹೊರಟಿದ್ದೇನೆ. ಅತ್ಯುನ್ನತ ಜ್ಞಾನ ಪ್ರಾಪ್ತವಾಗುವವರೆಗೆ ನಾನು ಹಿಂದಿರುಗುವುದಿಲ್ಲ. ದಯವಿಟ್ಟು ನಾನು ಯಶಸ್ವಿಯಾಗುವಂತೆ ಹರಸಿ” ಎಂದು ಕೇಳಿಕೊಂಡರು. ಶಾರದಾದೇವಿ ಯವರು ಶ್ರೀರಾಮಕೃಷ್ಣರ ಹೆಸರಿನಲ್ಲಿ ಹೃತ್ಪೂರ್ವಕವಾಗಿ ಹರಸಿದರು. ಬಳಿಕ ಕೇಳಿದರು: “ಮಗು, ನಿನ್ನ ತಾಯಿಯನ್ನೊಮ್ಮೆ ನೋಡಿಕೊಂಡು ಬರುವುದಿಲ್ಲವೆ?” ಅದಕ್ಕೆ ಸ್ವಾಮೀಜಿ, “ಅಮ್ಮಾ, ನೀವೇ ನನ್ನ ತಾಯಿ!” ಎಂದುತ್ತರಿಸಿದರು. ಅವರ ಈ ಭಾವವನ್ನು ಕಂಡ ಶ್ರೀಮಾತೆಯವರು ಹೃದಯದುಂಬಿ ಮತ್ತೊಮ್ಮೆ ಹರಸಿದರು. ಈ ಹೃದಯಸ್ಪರ್ಶಿ ದೃಶ್ಯವನ್ನು ನಾವು ನಮ್ಮ ಕಣ್ಮುಂದೆ ತಂದುಕೊಂಡು ಭಾವಿಸಿ ನೋಡಬೇಕು. ಈ ಇಬ್ಬರು ಸಿಂಹಸದೃಶ ಸಂನ್ಯಾಸಿಗಳು ಮಹಾಮಾತೆಯ ಸಮ್ಮುಖದಲ್ಲಿ ವಿನಯದಿಂದ ನಿಂತು ಆಶೀರ್ವಾದವನ್ನು ಬೇಡುತ್ತಿದ್ದಾರೆ; ಅವರನ್ನು ತಮ್ಮ ಹೆತ್ತ ತಾಯಿಗಿಂತ ಮಿಗಿಲಾಗಿ ಭಾವಿಸುತ್ತಿದ್ದಾರೆ; ಸುಪ್ರೀತರಾದ ಶ್ರೀಮಾತೆ ಯವರು ಅವರ ಮೇಲೆ ತಮ್ಮ ಕೃಪಾಶೀರ್ವಾದವನ್ನು ವರ್ಷಿಸುತ್ತಿದ್ದಾರೆ...

ಬಳಿಕ ಶಾರದಾದೇವಿಯವರು ಅಖಂಡಾನಂದರಿಗೆ, “ಮಗು, ನನ್ನ ಸರ್ವಸ್ವವನ್ನೂ ನಿನ್ನ ಕೈಯಲ್ಲಿಟ್ಟಿದ್ದೇನೆ. ನೀನು ಪರ್ವತಪ್ರದೇಶದ ಜೀವನಕ್ಕೆ ಒಗ್ಗಿಕೊಂಡವನು. ಅವನಿಗೆ ಯಾವ ರೀತಿಯಲ್ಲೂ ತೊಂದರೆಯಾಗದಂತೆ ನೋಡಿಕೋ” ಎಂದು ಪ್ರೀತಿಯಿಂದ ಎಚ್ಚರಿಸಿದರು. ‘ನನ್ನ ಸರ್ವಸ್ವ’ ಎಂದರೆ ಯಾವುದದು? ಸ್ವಾಮೀಜಿ! ಶ್ರೀಮಾತೆಯವರ ನೆಚ್ಚಿನ ನರೇನ್! ಇಲ್ಲಿ ಶ್ರೀಮಾತೆಯವರು ಅವರನ್ನು ತಮ್ಮ ಸರ್ವಸ್ವ ಎಂದು ಕರೆದುದು ತುಂಬ ಮಾರ್ಮಿಕವಾಗಿದೆ.

ಮಠದಲ್ಲಿ ಸುಮಾರು ಮೂರು ತಿಂಗಳ ಕಾಲ ಉಳಿದುಕೊಂಡಿದ್ದ ಸ್ವಾಮೀಜಿ, ೧೮೯೦ರ ಜುಲೈ ತಿಂಗಳ ಮಧ್ಯದ ವೇಳೆಗೆ ಬಾರಾನಗೋರನ್ನು ಬಿಟ್ಟು ಹೊರಟರು. ಹಿಂದೆ ಎರಡು-ಮೂರು ಸಲ ಪರಿವ್ರಾಜಕರಾಗಿ ಹೊರಟಾಗಲೂ, ತಮ್ಮ ಗುರಿಯನ್ನು ಸಾಧಿಸದೆ ಹಿಂದಿರುಗಲಾರೆನೆಂದು ತೀರ್ಮಾನ ಮಾಡಿಯೇ ಹೊರಟಿದ್ದರು. ಆದರೆ ಪ್ರತಿಸಲವೂ ಯಾವುದಾದರೊಂದು ಕಾರಣದಿಂದ ಹಿಂದಿರುಗಬೇಕಾಯಿತು. ಈ ಸಲ ಮಾತ್ರ ಸ್ವಾಮೀಜಿ ಮತ್ತೆ ಮಠಕ್ಕೆ ಹಿಂದಿರುಗಿದ್ದು ಸುದೀರ್ಘ ಸಪ್ತ ಸಂವತ್ಸರಗಳ ಅನಂತರವೇ; ಪಾಶ್ಚಾತ್ಯ ರಾಷ್ಟ್ರಗಳಲ್ಲಿ ವೇದಾಂತದ ಡಿಂಡಿಮವನ್ನು ಬಾರಿಸಿ, ವಿಶ್ವವಿಜೇತರಾದ ಮೇಲೆಯೇ!

