
\chapter{ಮತ್ತೆ ಮಠದಲ್ಲಿ}

\noindent

ಬಾರಾನಗೋರ್ ಮಠದಲ್ಲಿ ಸ್ವಾಮೀಜಿ ಪೂಜೆ, ಪ್ರಾರ್ಥನೆ, ಧ್ಯಾನ-ಜಪ ಹಾಗೂ ಶಾಸ್ತ್ರಾಧ್ಯಯನವನ್ನು ತೀವ್ರಗೊಳಿಸಿ ಒಂದು ವಿಶೇಷ ಆಧ್ಯಾತ್ಮಿಕ ವಾತಾವರಣವನ್ನು ನಿರ್ಮಾಣ ಮಾಡಿದರು. ಮುಂದೆ ತಾವೆಲ್ಲ ಶ್ರೀರಾಮಕೃಷ್ಣರ ಸಂದೇಶಗಳನ್ನು ವಿಶ್ವದಾದ್ಯಂತ ಪ್ರಸಾರ ಮಾಡಬೇಕು ಎಂಬ ತಮ್ಮ ಕಾರ್ಯಪ್ರಣಾಳಿಕೆಯನ್ನು ವರ್ಣಿಸುತ್ತ ಸೋದರಸಂನ್ಯಾಸಿಗಳನ್ನು ಸ್ಫೂರ್ತಿಗೊಳಿಸತೊಡಗಿದರು. ಸ್ವಾಮೀಜಿ ಮುಂದೆ ಪಾಶ್ಚಾತ್ಯ ರಾಷ್ಟ್ರಗಳಲ್ಲಿ ಹಾಗೂ ಸಮಸ್ತ ಭಾರತದಲ್ಲಿ ಯಾವ ದಿವ್ಯ ಸಂದೇಶಗಳನ್ನು ಬಿತ್ತರಿಸಿ ಲೋಕವಿಖ್ಯಾತರಾದರೋ ಅವುಗಳೆಲ್ಲ ಅವರ ಸೋದರಸಂನ್ಯಾಸಿಗಳಿಗೆ ಹೊಸದೇನಾಗಿರಲಿಲ್ಲ. ಏಕೆಂದರೆ, ಅವುಗಳನ್ನೆಲ್ಲ ಸ್ವಾಮೀಜಿ ಬಾರಾನಗೋರ್ ಮಠದ ದಿನಗಳಲ್ಲೇ ವಿವರಿಸಿ ಹೇಳಿದ್ದರು. ಭಾರತವನ್ನು ಅಖಂಡ ರಾಷ್ಟ್ರ ವನ್ನಾಗಿ ಸಮಗ್ರ ದೃಷ್ಟಿಯಿಂದ ನೋಡುವಂತೆ ಅವರ ದೃಷ್ಟಿಕೋನವನ್ನು ಸ್ವಾಮೀಜಿ ವಿಶಾಲ ಗೊಳಿಸಿದರು. ಆಧುನಿಕ ವಿಚಾರಗಳಿಂದ ಪ್ರಭಾವಿತರಾದ ಜನರ ಮನಸ್ಸಿಗೆ ಒಪ್ಪಿಗೆಯಾಗುವಂತೆ ಹಿಂದೂಧರ್ಮವನ್ನು ಅವರ ಮುಂದಿಡಬೇಕೆಂದು ಸ್ವಾಮೀಜಿ ತಮ್ಮ ಸೋದರ ಸಂನ್ಯಾಸಿಗಳಿಗೆ ಒತ್ತಿ ಹೇಳುತ್ತಿದ್ದರು. ಕಾಲಧರ್ಮಕ್ಕನುಗುಣವಾಗಿ ಸನಾತನ ಧರ್ಮವನ್ನು ವಿವರಿಸಿ, ಚಿಂತನಶೀಲ ಮೇಧಾವೀ ಜನವರ್ಗವೂ ಅದನ್ನು ಸ್ವೀಕರಿಸುವಂತೆ ಮಾಡಲು ತಮ್ಮ ಗುರುಭಾಯಿಗಳಿಗೆ ತರಬೇತಿ ಕೊಡುತ್ತಿದ್ದರು. ಅಲ್ಲದೆ ನಾಸ್ತಿಕವಾದಿಗಳ, ಜಡವಾದಿಗಳ ಹಾಗೂ ಮತಾಂಧರ ಟೀಕೆಗಳನ್ನು ಸಮರ್ಥವಾಗಿ ಎದುರಿಸಿ, ನಿಜವಾದ ಹಿಂದೂ ಶ್ರದ್ಧೆಯನ್ನು ಸಂರಕ್ಷಿಸುವಂತೆಯೂ ಅವರಿಗೆ ಶಿಕ್ಷಣ ನೀಡಿದರು.

ಮಠದಲ್ಲಿ ತೀವ್ರಗತಿಯಿಂದ ಹಾಗೂ ನಿಯಮಬದ್ಧವಾಗಿ ಶಾಸ್ತ್ರಾಧ್ಯಯನ ಕಾರ್ಯ ನಡೆಯ ಬೇಕೆಂಬುದು ಸ್ವಾಮೀಜಿಯ ಹಂಬಲ. ಆದರೆ ಮಠದಲ್ಲಿ ಶಾಸ್ತ್ರಗ್ರಂಥಗಳ ಕೊರತೆ, ಕೊಂಡು ತರಲು ಹಣವಿಲ್ಲ. ಅಲ್ಲದೆ ಆಶ್ರಮವಾಸಿಗಳ ಸಂಸ್ಕೃತಜ್ಞಾನವೂ ಹೆಚ್ಚಾಗಬೇಕೆಂಬುದನ್ನು ಸ್ವಾಮೀಜಿ ಕಂಡುಕೊಂಡಿದ್ದರು. ಆದ್ದರಿಂದ ಅವರು, ವಾರಾಣಸಿಯಲ್ಲಿದ್ದ ತಮ್ಮ ಸ್ನೇಹಿತ ಪ್ರಮದದಾಸ ಮಿತ್ರರಿಗೆ ಪತ್ರ ಬರೆದು ಕೆಲವು ಅಮೂಲ್ಯ ಗ್ರಂಥಗಳನ್ನು ಎರವಲು ತರಿಸಿ ಕೊಂಡರು. ಇವುಗಳಲ್ಲಿ ಪಾಣಿನಿಯ ಸಂಸ್ಕೃತ ವ್ಯಾಕರಣವೂ ಒಂದು.

ಈ ಅವಧಿಯಲ್ಲಿ ಸ್ವಾಮೀಜಿ ಸಾಮಾಜಿಕ ಪದ್ಧತಿ-ಸಂಪ್ರದಾಯಗಳನ್ನೂ ಹಲವಾರು ಶಾಸ್ತ್ರ ವಾಕ್ಯಗಳ ಅಸಂಬದ್ಧತೆಗಳನ್ನೂ ತೀವ್ರವಾಗಿ ವಿಮರ್ಶಿಸಿ ನೋಡುತ್ತಿದ್ದರು. ಈಗಾಗಲೇ ತಮ್ಮ ಪರಿವ್ರಾಜಕ ದಿನಗಳಲ್ಲಿ ಅವರು, ಹಲವಾರು ಅನಿಷ್ಟ ಸಂಪ್ರದಾಯಗಳು ಸಮಾಜದ ಮೇಲೆ ಎಂತಹ ಹೊರೆಯಾಗಿ ನಿಂತಿವೆ ಎಂಬುದನ್ನು ಕಂಡಿದ್ದರು. ಉದಾಹರಣೆಗೆ, ಶೂದ್ರ ಜಾತಿ ಯವರಿಗೆ ವೇದಾಧ್ಯಯನದ ಹಕ್ಕಿಲ್ಲ ಎನ್ನುತ್ತವೆ ಶಾಸ್ತ್ರಗಳು. ಆದರೆ ರೂಢಿಯಲ್ಲಿ ‘ಜಾತಿ’ಯ ಅರ್ಥವೇ ವಿಕೃತಗೊಂಡಿರುವುದನ್ನು ಸ್ವಾಮೀಜಿ ಕಂಡುಕೊಂಡರು. ನಿಜಕ್ಕೂ ಈ ‘ಜಾತಿ’ ಎಂಬುದು ಒಬ್ಬನ ವೈಯಕ್ತಿಕ ಗುಣ, ಯೋಗ್ಯತೆ ಹಾಗೂ ಕರ್ಮ–ಇವುಗಳನ್ನು ಅವಲಂಬಿಸಿ ನಿರ್ಧಾರವಾಗುವ ವಿಚಾರ. ಯಾವನು ಯಾವ ಗುಣದವನೋ, ಯಾರದು ಯಾವ ವೃತ್ತಿಯೋ, ಯಾವನ ಮನೋಭಾವ ಎಂಥದೋ ಅದದು ಅವನವನ ಜಾತಿಯನ್ನು ನಿರ್ಧರಿಸುತ್ತದೆ. ಎಂದರೆ, ಬ್ರಾಹ್ಮಣವಂಶದಲ್ಲಿ ಜನಿಸಿದವನು ಬ್ರಾಹ್ಮಣನೆಂದು ಹೇಳಲಾಗದಿರಬಹುದು; ವೃತ್ತಿಯಿಂದ ಆತ ವೈಶ್ಯನೋ ಶೂದ್ರನೋ ಆಗಬಹುದು... ಆದರೆ ಕಾಲಕ್ರಮದಲ್ಲಿ, ಯಾವನು ಯಾವ ಜಾತಿಯವರಲ್ಲಿ ಜನಿಸಿದ್ದಾನೋ ಅದೇ ಅವನ ಜಾತಿ ಎಂಬ ರೂಢಿ ಬೆಳೆದುಬಂದುಬಿಟ್ಟಿದೆ. ಈ ಎಲ್ಲ ತಪ್ಪು ಸಂಪ್ರದಾಯಗಳಿಂದ ಪರಸ್ಪರ ದ್ವೇಷ, ತಿರಸ್ಕಾರ ಹುಟ್ಟಿಕೊಳ್ಳುವಂತಾಗಿದೆ. ಸ್ವಾಮೀಜಿ ಇವೆಲ್ಲವುಗಳ ಕುರಿತಾಗಿ ಆಳವಾಗಿ ಆಲೋಚನೆ ಮಾಡಿ ಒಂದು ತೀರ್ಮಾನಕ್ಕೆ ಬಂದರು–ಏನೆಂದರೆ, ಭಾರತವೇನಾದರೂ ಉದ್ಧಾರವಾಗಬೇಕಾದರೆ ಇಲ್ಲಿನ ಜನಸಾಮಾನ್ಯ ರೆಲ್ಲರೂ ವೇದೋಪನಿಷತ್ತುಗಳ ಸನಾತನ ಸತ್ಯವನ್ನು ತಿಳಿಯುವಂತಾಗಬೇಕು; ಅದಕ್ಕಾಗಿ ಉಪನಿಷತ್ತುಗಳ ಭವ್ಯ ಸಂದೇಶಗಳನ್ನು ವಿಶಾಲ ಜನತೆಯ ಮುಂದೆ ಬಿತ್ತರಿಸಬೇಕು, ಎಂದು.

ಸ್ವಾಮೀಜಿ ಈ ವಿಚಾರಗಳ ಕುರಿತಾಗಿ ತಾವು ಆಲೋಚಿಸುತ್ತಿದ್ದುದಲ್ಲದೆ, ತಮ್ಮ ಸ್ನೇಹಿತರಾದ ಪ್ರಮದದಾಸ ಮಿತ್ರರೊಂದಿಗೆ ವಿಚಾರವಿನಿಮಯ ಮಾಡಿಕೊಳ್ಳುತ್ತಿದ್ದರು. ಇವರು ಬಹಳ ದೊಡ್ಡ ಪಂಡಿತರು; ವೇದ-ಶಾಸ್ತ್ರಗಳಲ್ಲಿ ನಿಖರ ಜ್ಞಾನವಿದ್ದವರು. ವಯಸ್ಸಿನಲ್ಲಿ ಸ್ವಾಮೀಜಿಗಿಂತ ಎಷ್ಟೋ ಹಿರಿಯರು. ವೇದ-ಶಾಸ್ತ್ರಗಳ ಅಧ್ಯಯನಕ್ಕಾಗಿಯೇ ತಮ್ಮ ಜೀವನವನ್ನು ಮುಡಿಪಾಗಿ ಟ್ಟವರು ಇವರು. ವೇದಗಳ ಪ್ರಾಮಾಣ್ಯದ ಕುರಿತಾಗಿ ಕರ್ಮ ನಿಯಮದ ವಿಷಯವಾಗಿ, ವಿವಿಧ ಶಾಸ್ತ್ರವಾಕ್ಯಗಳ ಭಿನ್ನಾಭಿಪ್ರಾಯಗಳ ಕುರಿತಾಗಿ, ಸ್ಮೃತಿಗಳಲ್ಲಿ ಮೇಲ್ನೋಟಕ್ಕೆ ಅರ್ಥರಹಿತ ವೆಂಬಂತೆ ತೋರಿಬರುವ ಅನೇಕ ವಿಧಿಗಳ ವಿಚಾರವಾಗಿ ಸ್ವಾಮೀಜಿ ಇವರಿಗೆ ಹಲವಾರು ಪತ್ರ ಗಳನ್ನು ಬರೆದು ತಮ್ಮ ಸಂದೇಹಗಳನ್ನು ಪರಿಹರಿಸಿಕೊಂಡರು. ಆಧ್ಯಾತ್ಮಿಕ ಅನುಭವ ಎಷ್ಟು ಮುಖ್ಯವೋ ಶಾಸ್ತ್ರಜ್ಞಾನವೂ ಅಷ್ಟೇ ಮುಖ್ಯ–ಲೋಕದ ಬಹುತೇಕ ಜನರ ವಿಷಯದಲ್ಲಿ ಈ ಮಾತು ಅನ್ವಯವಾಗುತ್ತದೆ. ಶಾಸ್ತ್ರವಾಕ್ಯಗಳ ಮರ್ಮಗಳನ್ನು ಅರಿಯದಿದ್ದರೆ, ನಮ್ಮ ಜೀವನ ಕ್ರಮ-ಸಾಧನಾಕ್ರಮಗಳನ್ನು ವ್ಯವಸ್ಥಿತಗೊಳಿಸಿಕೊಂಡು ಸರಿಯಾದ ರೀತಿಯಲ್ಲಿ ಮುನ್ನಡೆಯಲು ಸಾಧ್ಯವಾಗುವುದಿಲ್ಲ. ಅಲ್ಲದೆ ಶಾಸ್ತ್ರವಾಕ್ಯಗಳನ್ನು ಮನಸ್ಸಿಗೆ ಬಂದಂತೆ ಒಬ್ಬೊಬ್ಬರು ಒಂದೊಂದು ರೀತಿಯಲ್ಲಿ ಅರ್ಥೈಸಿ ಪ್ರಸಾರ ಮಾಡಿದ್ದೇ ಹಲವಾರು ಅನರ್ಥಗಳಿಗೆ ಕಾರಣ. ಆದ್ದರಿಂದ ಸ್ವಾಮೀಜಿ ಶಾಸ್ತ್ರಗಳ ನಿಜವಾದ, ನಿಖರವಾದ ಅರ್ಥವನ್ನು ಗ್ರಹಿಸಲು, ಅವುಗಳ ಮಹತ್ವವನ್ನು ಅರಿಯಲು ಪ್ರಯತ್ನಪಡುತ್ತಿದ್ದರು. ಶಾಸ್ತ್ರಗಳನ್ನು ಅರಿಯದವರು ಹೇಳುತ್ತಾರೆ: “ಶಾಸ್ತ್ರಗಳಲ್ಲಿ ಏನಿದೆ? ಅವುಗಳನ್ನು ಓದುವುದರಿಂದ ಏನು ಪ್ರಯೋಜನ?” ಎಂದು. ಶಾಸ್ತ್ರಗಳನ್ನೋದಿದ ಪಂಡಿತರೂ ಅವಿವೇಕದಿಂದ ವರ್ತಿಸುವುದನ್ನು ಕಂಡಾಗ ಅವರ ಅನುಮಾನ ಮತ್ತಷ್ಟು ದೃಢ ವಾಗುತ್ತದೆ. ಆದರೆ ಶಾಸ್ತ್ರವಾಕ್ಯಗಳನ್ನು ಓದಿ, ಅರಿತು, ಆಲೋಚಿಸಿ, ಅನುಷ್ಠಾನಕ್ಕೆ ತಂದರೆ ಮಾತ್ರ ಅವುಗಳ ಪ್ರಯೋಜನ ಅರಿವಿಗೆ ಬರುತ್ತದೆ. ಉಪನಿಷದ್ವಾಕ್ಯಗಳಿಗೆ ಅನುಸಾರ ವಾಗಿ ಜೀವನವನ್ನು ರೂಪಿಸಿಕೊಳ್ಳಲು ಎಲ್ಲರಿಂದಲೂ ಸಾಧ್ಯವಾಗದಿರಬಹುದು. ಆದರೆ ಹಾಗೆಂದ ಮಾತ್ರಕ್ಕೆ ಶಾಸ್ತ್ರ ಗಳಲ್ಲಿ ಏನೂ ತಿರುಳಿಲ್ಲ ಎಂದು ಬಿಡುವುದೆ? ದ್ರಾಕ್ಷಿ ಕೈಗೆಟುಕದಿದ್ದರೆ ‘ಹುಳಿ’ ಎಂದುಬಿಡುವುದೆ? ಆದ್ದರಿಂದ, ಯಾವ ಶಾಸ್ತ್ರಗಳು ವಿವೇಕೀ ಜನವರ್ಗಕ್ಕೆ ಸಕಾಲಿಕ ನೀತಿ- ನಿಯಮ-ನಿರ್ದೇಶನ ಗಳನ್ನು ಕೊಟ್ಟು ನಡೆಸಿಕೊಂಡು ಹೋಗುತ್ತಿವೆಯೋ, ಆ ಶಾಸ್ತ್ರಗಳ ಮರ್ಮವನ್ನು, ಮಹತ್ವವನ್ನು ಅರಿತುಕೊಳ್ಳುವ ಪ್ರಯತ್ನ ಮಾಡುತ್ತಿದ್ದಾರೆ ಸ್ವಾಮೀಜಿ. ಶಾಸ್ತ್ರ ಗಳಲ್ಲಿ ಕಂಡುಬರುವ ಎಲ್ಲ ವಿರೋಧಾಭಾಸಗಳನ್ನು, ವಿವಾದಾಸ್ಪದ ವಿಷಯಗಳನ್ನು ಸಮರ್ಥ ವಾಗಿ ತರ್ಕಬದ್ಧವಾಗಿ ವಿವರಿಸ ಬಲ್ಲ ದೃಷ್ಟಿಕೋನವನ್ನು ಹೊಂದುವ ಪ್ರಯತ್ನ ಮಾಡುತ್ತಿದ್ದಾರೆ. ಅವರು ತಮ್ಮ ಒಂದು ಪತ್ರ ದಲ್ಲಿ ಪ್ರಮದದಾಸ ಮಿತ್ರರಿಗೆ ಬರೆಯುತ್ತಾರೆ: “ನಾನು ಆ ಮಂಗಳ ಸ್ವರೂಪನಾದ ಭಗವಂತ ನಲ್ಲಿ ಶ್ರದ್ಧೆಯನ್ನು ಕಳೆದುಕೊಂಡಿಲ್ಲ; ಮುಂದೆಂದೂ ಕಳೆದುಕೊಳ್ಳು ವವನೂ ಅಲ್ಲ. ಶಾಸ್ತ್ರಗಳಲ್ಲಿ ನನಗಿರುವ ಶ್ರದ್ಧೆ ಅಚಲವಾದದ್ದು. ಆದರೆ ಈ ಆರೇಳು ವರ್ಷಗಳಲ್ಲಿ ಮಾತ್ರ ನನ್ನ ಮನಸ್ಸು ನಾನಾ ಬಗೆಯ ಹೋರಾಟಗಳನ್ನು ಎದುರಿಸಬೇಕಾಯಿತು; ನಾನಾ ಬಗೆಯ ವಿರೋಧಗಳನ್ನು ಎದುರಿಸಬೇಕಾಯಿತು. ಈಗ ನನಗೆ ನಿಜವಾದ ಆದರ್ಶಶಾಸ್ತ್ರದ ಅನುಗ್ರಹವಾಗಿದೆ; ನಾನು ನಿಜವಾದ ಆದರ್ಶಮಾನವನನ್ನು ಕಂಡಿದ್ದೇನೆ. ಆದರೂ ಒಂದು ಮಾರ್ಗದಲ್ಲಿ ನಿರಂತರವಾಗಿ ಸಾಗಿ ಗುರಿಮುಟ್ಟಲು ನನ್ನಿಂದ ಸಾಧ್ಯವಾಗಿಲ್ಲ. ಇದೇ ನನ್ನ ಪಾಲಿನ ಒಂದು ಮಹಾವ್ಯಥೆಯಾಗಿ ಪರಿಣಮಿಸಿದೆ.” ನಿಜ; ಶಾಸ್ತ್ರವಿಚಾರಗಳೂ ಕ್ಲಿಷ್ಟ, ಜನ ಜೀವನದ ವಿನ್ಯಾಸವೂ ಕ್ಲಿಷ್ಟ. ಆ ಶಾಸ್ತ್ರಗಳಿಗೂ ಈ ಜನಜೀವನಕ್ಕೂ ಹೊಂದಾಣಿಕೆ ಬರುವ ಬಗೆ ಹೇಗೆ? ಸ್ವಾಮೀಜಿ ದೀರ್ಘವಾಗಿ ಆಲೋಚಿಸುತ್ತಿದ್ದಾರೆ.

ಬಾರಾನಗೋರ್ ಮಠದಲ್ಲಿ ವಾಸಿಸುತ್ತಿದ್ದ ಸ್ವಾಮೀಜೀಯವರಿಗೆ ಬೇಡಬೇಡವೆಂದರೂ ತಮ್ಮ ಮನೆಮಂದಿಯ ದಾರಿದ್ರ್ಯದ ಪರಿಸ್ಥಿತಿ ಕಣ್ಣಿಗೆ ಬೀಳುವಂತಾಗಿತ್ತು. ಶ್ರೀರಾಮಕೃಷ್ಣರು ನೀಡಿದ್ದ ಅಭಯದಂತೆ, ಅವರ ಮನೆಯವರಿಗೆ ಸಾಧಾರಣ ಅನ್ನ ಬಟ್ಟೆಗಳಿಗೆ ಕೊರತೆಯಿರಲಿಲ್ಲ. ಆದರೆ ಕೋರ್ಟು ಮೊಕದ್ದಮೆ ಅವರನ್ನು ತೀರ ಹಣ್ಣುಗಾಯಿ ಮಾಡಿಬಿಟ್ಟಿತ್ತು. ಇದನ್ನು ಕಂಡು ಸ್ವಾಮೀಜಿಗೆ ಸಹಿಸಲಾರದ ಸಂಕಟವಾಗುತ್ತಿತ್ತು. ಅವರು ಸರ್ವಸಂಗಪರಿತ್ಯಾಗಿಗಳೆಂಬುದೇನೋ ನಿಜ. ಆದರೆ ದಾರಿದ್ರ್ಯದ ದವಡೆಗೆ ಸಿಲುಕಿಕೊಂಡು ದುಃಖಿತರಾದವರನ್ನು ಕಂಡು ಮನಕರಗ ದಿರುವಷ್ಟು ನಿರ್ದಯಿಗಳಲ್ಲವಲ್ಲ! ಒಬ್ಬ ಮಗನಾಗಿ ತಾವು ತಮ್ಮ ತಾಯಿಗೆ ಸಲ್ಲಿಸಬೇಕಾದ ಕರ್ತವ್ಯದ ಪುಣವನ್ನು ತೀರಿಸಲಿಲ್ಲ ಎಂಬ ವ್ಯಥೆ ಅವರ ಅಂತರಾಳವನ್ನು ಆಗಾಗ ಚುಚ್ಚುತ್ತಿತ್ತು. ಒಮ್ಮೊಮ್ಮೆ ಅವರ ಮನಸ್ಸು ಭಾವಗಳ ಬಿರುಗಾಳಿಗೆ ಸಿಕ್ಕಿಕೊಂಡು ತುಯ್ದಾಡುತ್ತಿತ್ತು. ಇಂತಹ ಸ್ಥಿತಿಯಲ್ಲಿ ಅವರೊಮ್ಮೆ ಪ್ರಮದದಾಸ ಮಿತ್ರರಿಗೆ ಬರೆಯುತ್ತಾರೆ: “ನಾನು ಕಲ್ಕತ್ತದ ಸಮೀಪ ದಲ್ಲೇ ಇರುವುದರಿಂದ ಅವರೆಲ್ಲರ ಕಷ್ಟಕಾರ್ಪಣ್ಯಗಳನ್ನು ಕಾಣಬೇಕಾಗಿದೆ. ಇದನ್ನೆಲ್ಲ ಕಂಡು ಕೆಲವು ಸಲ ನನ್ನ ರಜೋಗುಣ ಕೆರಳುತ್ತದೆ. ನನ್ನ ಅಹಂಬುದ್ಧಿ ಬೃಹತ್ತಾಗಿ ಬೆಳೆದು, ಕಾರ್ಯಮಗ್ನ ನಾಗುವಂತೆ ನನ್ನನ್ನು ಪ್ರಚೋದಿಸುತ್ತದೆ. ಇಂತಹ ಸಂದರ್ಭಗಳಲ್ಲಿ ನನ್ನ ಮನಸ್ಸು ಭಯಂಕರ ವಾಗಿ ತುಯ್ದಾಡುತ್ತಿರುತ್ತದೆ. ಆದ್ದರಿಂದಲೇ ನಿಮಗೆ ಬರೆದದ್ದು–ನನ್ನ ಮನಸ್ಸು ಉಗ್ರವಾಗಿದೆ, ಎಂದು. ಅಂತೂ ಈಗ ಆ ಕೋರ್ಟಿನ ಮೊಕದ್ದಮೆ ಕೊನೆಗೊಂಡಿದೆ. ಆದ್ದರಿಂದ ಇನ್ನು ಕೆಲವು ದಿನಗಳಲ್ಲಿ ನಾನು ಇಲ್ಲಿನ ಎಲ್ಲ ವ್ಯವಹಾರಗಳನ್ನು ಮುಗಿಸಿಕೊಂಡು, ಎಂದೆಂದಿಗೂ ಈ ಸ್ಥಳಕ್ಕೆ ವಿದಾಯ ಹೇಳಿ ಹೊರಟುಹೋಗುವಂತಾಗಲೆಂದು ಹಾರೈಸಿ. ಆ ಭಗವತ್ಶಕ್ತಿಯಿಂದ ನನ್ನ ಹೃದಯ ಬಲಿಷ್ಠಗೊಂಡು ಈ ಎಲ್ಲ ಮಾಯಾಪಾಶಗಳನ್ನು ಕತ್ತರಿಸಿ ಮುನ್ನಡೆಯಲು ಸಾಧ್ಯವಾಗು ವಂತೆ ಆಶೀರ್ವದಿಸಿ.”

ಈ ಸಮಯದಲ್ಲಿ ಆಗಾಗ ಸ್ವಾಮೀಜಿಯ ಮನಸ್ಸಿನಲ್ಲಿ ತೀರ್ಥಾಟನೆಯ ಇಚ್ಛೆ ಪ್ರಬಲವಾಗು ತ್ತಿತ್ತು. ಪರಮಪಾವನಕರವಾದ ವಾರಾಣಸಿಗೆ ಹೋಗಿ ಅಲ್ಲಿ ವಿಶ್ವನಾಥನ ಸನ್ನಿಧಿಯಲ್ಲಿ ಜಪತಪ ಗಳಲ್ಲಿ ಮುಳುಗಬೇಕೆಂಬ ತವಕವುಂಟಾಗುತ್ತಿತ್ತು. ಅಲ್ಲದೆ, ಅವರ ವಿಶ್ವಾಸಿಗಳೂ ವಿದ್ವಾಂಸರೂ ಆದ ಪ್ರಮದದಾಸ ಮಿತ್ರರು ಅಲ್ಲಿರುವುದು ಇನ್ನೊಂದು ಆಕರ್ಷಣೆ. ಇವರೊಂದಿಗೆ ಶಾಸ್ತ್ರವಿಚಾರ ಗಳ ಸೂಕ್ಷ್ಮಗಳನ್ನೆಲ್ಲ ವಿವರವಿವರವಾಗಿ ಚರ್ಚಿಸಿ ತಿಳಿದುಕೊಳ್ಳುವ ಕಾತರ ಅವರಿಗೆ. ಈ ಎಲ್ಲ ತುಯ್ದಾಟಗಳ ನಡುವೆ ಅವರಿಗೆ ಕಲ್ಕತ್ತದ ಜೀವನ ಅಸಹನೀಯವಾಗತೊಡಗಿತ್ತು.

ಈ ವೇಳೆಗೆ ಅವರ ಸೋದರಸಂನ್ಯಾಸಿಯಾದ ಸ್ವಾಮಿ ಅಖಂಡಾನಂದರು ಹಿಮಾಲಯ ಪ್ರದೇಶದಲ್ಲಿ ಸಂಚರಿಸುತ್ತಿದ್ದರು. ಎಷ್ಟೋ ಸಲ ಟಿಬೆಟ್ಟಿಗೂ ಹೋದ ಬಗ್ಗೆ ಅವರು ಸ್ವಾಮೀಜಿಗೆ ಪತ್ರ ಬರೆಯುತ್ತಿದ್ದರು; ಅಲ್ಲಿನ ಜನರ ಜೀವನಕ್ರಮಗಳ ಕುರಿತಾಗಿ ತಿಳಿಸುತ್ತಿದ್ದರು. ಇವರಲ್ಲದೆ ಇತರ ನಾಲ್ವರು ಸೋದರಸಂನ್ಯಾಸಿಗಳು ಹಿಮಾಲಯದಲ್ಲೇ ಪರಿವ್ರಾಜಕರಾಗಿ ತಿರುಗಾಡುತ್ತಿ ದ್ದರು. ಆದ್ದರಿಂದ ಈಗ ಸ್ವಾಮೀಜಿಗೆ ತಾವೂ ಹಿಮಾಲಯಕ್ಕೆ ಹೋಗಿರಬೇಕೆಂಬ ತವಕ ಹುಟ್ಟಿಕೊಂಡಿತು.

