
\chapter{ವ್ಯಕ್ತಿತ್ವ ವಿಕಸನ}

\noindent

ನರೇಂದ್ರನು ಶ್ರೀರಾಮಕೃಷ್ಣರ ಗರಡಿಯಲ್ಲಿ ಕಠಿಣ ತರಬೇತಿ ಪಡೆಯುತ್ತ ಹೊಸ ಹೊಸ ಪಾಠಗಳನ್ನು ಕಲಿಯುತ್ತ ಶಕ್ತಿಶಾಲಿಯಾಗಿ ಮುನ್ನಡೆಯುತ್ತಿರುವುದನ್ನು ಈವರೆಗೆ ನೋಡಿದೆವು. ಆದರೆ ಅದು ಅವನ ವ್ಯಕ್ತಿತ್ವನಿರ್ಮಾಣದ ಕಾರ್ಯಪ್ರಣಾಳಿಗಳಲ್ಲಿ ಒಂದು ಮಾತ್ರ. ಇನ್ನೂ ಅನೇಕ ಅಂಶಗಳು–ಹಲವಾರು ಗ್ರಂಥಗಳು, ತತ್ತ್ವಗಳು, ವಾದವಿವಾದಗಳು, ವ್ಯಕ್ತಿಗಳು, ದೃಶ್ಯಗಳು, ಪರಿಸರಗಳು–ಅವನ ಮೇಲೆ ಪ್ರಭಾವ ಬೀರುತ್ತ ಆತನನ್ನು ರೂಪಿಸುತ್ತಿದ್ದುವು. ಅವನ ಇಲ್ಲಿಯ ವರೆಗಿನ ಜೀವನವನ್ನು ಸಿಂಹಾವಲೋಕನ ಮಾಡಿದಾಗ ಅಂತಹ ಮೂರು ಮುಖ್ಯಾಂಶಗಳನ್ನು ಗುರುತಿಸಬಹುದು. ಮೊದಲ ಅಂಶವೆಂದರೆ ಅವನ ಆಧ್ಯಾತ್ಮಿಕ ಮನೋವೃತ್ತಿ; ಅರ್ಥಾತ್ ತನ್ನ ನಿಜವ್ಯಕ್ತಿತ್ವವೊಂದು ಬೇರೆಯೇ ಇದೆ ಎಂಬುದರ ಬಗ್ಗೆ ಅವನಿಗಿದ್ದ ಅರಿವು. ಎರಡನೆಯದೆಂದರೆ ಅವನ ತಂದೆ-ತಾಯಿ ಹಾಗೂ ಮನೆತನದ ಪ್ರಭಾವ ಮತ್ತು ಅವನ ಆಳವಾದ ಅಧ್ಯಯನದ ಪರಿಣಾಮ. ಮೂರನೆಯದಾಗಿ, ಅವನನ್ನು ಅಶಾಂತಿಯ ತುಮುಲದಿಂದ ಶಾಂತಿಧಾಮದೆಡೆಗೆ, ಸಂಶಯದ ಸುಳಿಯಿಂದ ಶ್ರದ್ಧೆಯ ಶಿಖರದೆಡೆಗೆ ಕರೆದೊಯ್ಯತ್ತಿರುವ ಗುರು ಶ್ರೀರಾಮಕೃಷ್ಣರ ಮಾರ್ಗದರ್ಶನ.

ನರೇಂದ್ರನ ಮೇಲೆ ಅವನ ಮನೆತನ, ಅದರಲ್ಲೂ ಮುಖ್ಯವಾಗಿ ಆತನ ತಾಯ್ತಂದೆಯರು ಬೀರಿದ ಪ್ರಭಾವ ಎಷ್ಟು ಆಳವಾದದ್ದು ಎಂಬುದನ್ನು ಹಿಂದೆಯೇ ಕಂಡಿದ್ದೇವೆ. ಉದಾತ್ತವಾಗಿ ಆಲೋಚಿಸುವ, ಉನ್ನತವಾಗಿ ಭಾವಿಸುವ ಹಾಗೂ ಸಭ್ಯತೆಯಿಂದ ನಡೆದುಕೊಳ್ಳುವ ಪಾಠಗಳನ್ನು ಅವನು ತಾಯಿಯಿಂದ ಕಲಿತರೆ, ತಂದೆಯಿಂದ ಔದಾರ್ಯ-ಪೌರುಷಗಳನ್ನೂ ರಾಷ್ಟ್ರದ ಸತ್ಸಂಪ್ರ ದಾಯಗಳ ಬಗೆಗೆ ಗೌರವಬುದ್ಧಿಯನ್ನೂ ಮೈಗೂಡಿಸಿಕೊಂಡ. ತಂದೆ ವಿಶ್ವನಾಥನ ದೆಸೆಯಿಂ ದಾಗಿಯೇ ಅವನು ಕ್ರೈಸ್ತ ಹಾಗೂ ಇಸ್ಲಾಂ ಸಂಸ್ಕೃತಿಗಳತ್ತ ಗಮನ ಹರಿಸಿ, ಕ್ರಮೇಣ ತನ್ನ ಜಾಗತಿಕ ಜ್ಞಾನವನ್ನು ವೃದ್ಧಿಗೊಳಿಸಿಕೊಳ್ಳುವಂತಾಯಿತು. ಅವನಿಗೆ ಈ ಬಗೆಯ ಜ್ಞಾನಸಂಪಾದನೆ ಅತ್ಯಾವಶ್ಯಕವಾಗಿತ್ತು. ಏಕೆಂದರೆ ಮುಂದೆ ಅವನು ವೀರಸಂನ್ಯಾಸಿ ವಿವೇಕಾನಂದರಾಗಬೇಕಾ ದವನು. ಆದ್ದರಿಂದ ಪರರಾಷ್ಟ್ರಗಳ ಸಂಸ್ಕೃತಿಗಳನ್ನು, ಧರ್ಮಗಳನ್ನು ವಿಮರ್ಶಾತ್ಮಕ ದೃಷ್ಟಿ ಯಿಂದ ಅರಿತುಕೊಳ್ಳಬೇಕಾದ ಆವಶ್ಯಕತೆಯಿತ್ತು. ಆದ್ದರಿಂದಲೇ ಅವನು ಪ್ರಾಚ್ಯ-ಪಾಶ್ಚಾತ್ಯ ತತ್ತ್ವಶಾಸ್ತ್ರ, ಇತಿಹಾಸ, ಕಲೆ, ವಿಜ್ಞಾನ ಮತ್ತು ವಿಶೇಷವಾಗಿ ಧರ್ಮ–ಇವುಗಳ ಸಮಗ್ರ ಪರಿಚಯ ಮಾಡಿಕೊಳ್ಳಲು ಅಷ್ಟೊಂದು ಕಾತರನಾಗಿದ್ದುದನ್ನು ನೋಡುತ್ತೇವೆ. ಅದರಲ್ಲೂ ಮುಖ್ಯವಾಗಿ, ಪಾಶ್ಚಾತ್ಯ ತತ್ತ್ವಶಾಸ್ತ್ರಗಳ ಮರ್ಮ-ಮೌಲ್ಯಗಳನ್ನೂ ಸಾರ್ಥಕತೆಯನ್ನೂ ಅರಿತುಕೊಂಡು ಸ್ವಾಧೀನಪಡಿಸಿಕೊಳ್ಳಲು ಅವನು ದೃಢನಿರ್ಧಾರದ ಪ್ರಯತ್ನದಲ್ಲಿ ನಿರತನಾದ.

ಈ ತತ್ತ್ವಶಾಸ್ತ್ರ-ಸಿದ್ಧಾಂತಗಳನ್ನು ಅಧ್ಯಯಿಸಿದಂತೆ ಇವುಗಳಲ್ಲಿ ಹೆಚ್ಚಿನವೆಲ್ಲ ಕೇವಲ ಬುದ್ಧಿ ಚಮತ್ಕಾರಕ್ಕೆ ಸಂಬಂಧಿಸಿದ ಅಂಶಗಳು ಎನ್ನುವುದು ನರೇಂದ್ರನ ಅರಿವಿಗೆ ಬಂದಿತು. ಇವುಗಳಲ್ಲಿ ಮನುಷ್ಯನ ಭಾವನಾತ್ಮಕತೆಗೆ ಒಂದು ಸ್ಥಾನವಿಲ್ಲದಿರುವುದನ್ನು ಕಂಡ. ಇವುಗಳೆಲ್ಲ ವಿಚಾರ ಸ್ವಾತಂತ್ರ್ಯಕ್ಕೆ ಅವಕಾಶ ಕೊಡದಿರುವ ‘ಕಣ್ಣುಮುಚ್ಚಿಕೊಂಡು ನಂಬು’ ಎನ್ನುವ ಜಾತಿಯ ಸಿದ್ಧಾಂತಗಳು ಎಂದೆನಿಸಿತು ಅವನಿಗೆ. ಈ ಸಿದ್ಧಾಂತಗಳಲ್ಲಿ ಎಷ್ಟೇ ಬುದ್ಧಿವಂತಿಕೆಯಿಂದ, ಚಾಕಚಕ್ಯತೆಯಿಂದ ಸತ್ಯದ ರೂಪುರೇಖೆಯನ್ನು ವರ್ಣಿಸಿರಬಹುದು. ಆದರೆ ನರೇಂದ್ರನಿಗೆ ಬೇಕಾದದ್ದು ಸತ್ಯದ ಮೇಲ್ಮೈನೋಟದ ಬಣ್ಣನೆಯಲ್ಲ; ಅವನಿಗೆ ಬೇಕಾಗಿರುವುದು ಸಾಕ್ಷಾತ್ ಸತ್ಯ–ಅದನ್ನು ‘ದೇವರು’ ಎಂದರೂ ಸರಿ, ಇನ್ನಾವ ಹೆಸರಿನಿಂದ ಕರೆದರೂ ಸರಿ. ಯಾವುದು ಮನುಷ್ಯನಲ್ಲಿ ಆಧ್ಯಾತ್ಮಿಕ ಪ್ರವೃತ್ತಿಯನ್ನುಂಟುಮಾಡುತ್ತದೋ, ಯಾವುದು ಮನುಷ್ಯನ ರಚನಾ ತ್ಮಕ ಶಕ್ತಿಯ ಮೂಲವೋ, ಯಾವುದು ಮನುಷ್ಯನ ಬುದ್ಧಿಶಕ್ತಿ-ಇಚ್ಛಾಶಕ್ತಿಗಳಿಗೆ ಅತ್ಯುನ್ನತವಾದ ಮತ್ತು ಅತ್ಯಂತ ಉದಾತ್ತವಾದ ಸ್ಫುರಣೆಯನ್ನು ಕೊಡುತ್ತದೋ ಅದು ಮಾತ್ರವೇ ನಿಜವಾದ ತತ್ತ್ವಶಾಸ್ತ್ರ; ಈ ಗುಣಲಕ್ಷಣಗಳಿಲ್ಲದಿದ್ದಲ್ಲಿ ಅದೆಲ್ಲ ಕೆಲಸಕ್ಕೆ ಬಾರದ ಕಂತೆ ಎಂಬ ತೀರ್ಮಾನಕ್ಕೆ ಬಂದ ನರೇಂದ್ರ.

ತನ್ನ ಕಾಲೇಜು ಅಧ್ಯಯನದ ಐದು ವರ್ಷಗಳ ಅವಧಿಯಲ್ಲಿ ಅವನು, ಆಗಿನ ಜಗತ್ತಿನ ವಿಚಾರಧಾರೆಯ ಮೇಲೆ ಗಾಢ ಪ್ರಭಾವ ಬೀರಿದ್ದ ಅನೇಕ ಮಹಾ ವಿದ್ವಾಂಸರ ಸಿದ್ಧಾಂತಗಳನ್ನೆಲ್ಲ ಓದಿ ಅರಗಿಸಿಕೊಂಡ. ಜಾನ್ ಸ್ವೂಅರ್ಟ್ ಮಿಲ್, ಡೆಕಾರ್ಟೆ, ಹ್ಯೂಮ್, ಬೆನ್ತ್ಯಾಮ್, ಸ್ಪಿನೋಸಾ, ಡಾರ್ವಿನ್, ಸ್ಪೆನ್ಸರ್, ಕ್ಯಾಂಟ್, ಫಿಕ್ಟೆ, ಶೋಫೆನೋವರ್ ಮೊದಲಾದವರ ಅಭಿ ಪ್ರಾಯಗಳನ್ನು ಪರಿಪೂರ್ಣವಾಗಿ ಅರಿತುಕೊಂಡ. ಇವುಗಳಲ್ಲಿ ಕೆಲವು ನಾಸ್ತಿಕ್ಯವನ್ನು ಎತ್ತಿ ಹಿಡಿಯುವಂಥವು. ‘ದೇವರೊಬ್ಬನಿರಬಹುದು, ಆದರೆ ಆತನ ಬಗ್ಗೆ ಅರಿತುಕೊಳ್ಳಲು ಸಾಧ್ಯವಿಲ್ಲ’ ಎಂದು ಸಾರುವುವು ಕೆಲವು. ಹಾಗೆಯೇ, ವಿಶ್ವದ ಉಗಮದ ಮತ್ತು ಬೆಳವಣಿಗೆಯ ಬಗ್ಗೆ, ಮಾನವನ ನಿಜ ಸ್ವಭಾವದ ಬಗ್ಗೆ, ಗೊತ್ತುಗುರಿಗಳ ಬಗ್ಗೆ ಬಗೆಬಗೆಯಾಗಿ ತರ್ಕಿಸುವವು ಕೆಲವು. ಇವುಗಳನ್ನೆಲ್ಲ ಮನನ ಮಾಡಿದ್ದಲ್ಲದೆ. ಮನುಷ್ಯನ ಮೆದುಳು-ನರಮಂಡಲ-ಶರೀರಗಳ ಕಾರ್ಯ ವಿಧಾನವನ್ನೂ ಪರಸ್ಪರ ಸಾಮರಸ್ಯದ ಮರ್ಮವನ್ನೂ ಅರ್ಥಮಾಡಿಕೊಳ್ಳಲು ನರೇಂದ್ರ ಆಗಾಗ ಗೆಳೆಯರ ಜೊತೆಯಲ್ಲಿ ಕಲ್ಕತ್ತದ ವೈದ್ಯಕೀಯ ಕಾಲೇಜಿಗೆ ಹೋಗಿ, ಉಪನ್ಯಾಸಗಳನ್ನು ಕೇಳುತ್ತಿದ್ದ.

ಈ ವಿವಿಧ ಸಿದ್ಧಾಂತಗಳ ಪರಿಚಯ ಮಾಡಿಕೊಂಡದ್ದರಿಂದ ಅವನಿಗಾದ ಲಾಭ ಅಪಾರ. ಇವುಗಳಿಂದಾಗಿ ಅವನ ಆಲೋಚನಾಶಕ್ತಿ ಬಲಗೊಂಡಿತು; ವಿವೇಚನಾಶಕ್ತಿ ಇನ್ನಷ್ಟು ಹರಿತ ಗೊಂಡಿತು. ಯಾವುದೇ ವಿಷಯದಲ್ಲೇ ಆಗಲಿ ನಿರ್ಧಾರ ತೆಗೆದುಕೊಳ್ಳುವ ಅವನ ಸಾಮರ್ಥ್ಯ ದುರ್ಭೇದ್ಯವಾಯಿತು. ಇವುಗಳಲ್ಲಿ ಮೊದಲಿಗೆ ಹರ್ಬರ್ಟ್ ಸ್ಪೆನ್ಸರನ ಅಜ್ಞೇಯತಾವಾದ– ದೇವರನ್ನು ಅರಿತುಕೊಳ್ಳಲು ಸಾಧ್ಯವಿಲ್ಲವೆಂಬ ವಾದ–ನರೇಂದ್ರನ ಕುತೂಹಲವನ್ನು ಬಹಳ ವಾಗಿ ಕೆರಳಿಸಿತ್ತು. ಸಾಂಪ್ರದಾಯಿಕ ಭಾವನೆಗಳನ್ನು ಬುಡಸಹಿತ ಅಲುಗಾಡಿಸಿಬಿಡುವಂಥದು ಸ್ಪೆನ್ಸರನ ಸಿದ್ಧಾಂತ. ಇಂತಹ ಕ್ರಾಂತಿಕಾರಿ ವಿಚಾರಧಾರೆಗೆ ತನ್ನನ್ನು ಒಡ್ಡಿಕೊಂಡರೂ ನರೇಂದ್ರ ತನ್ನ ಮೂಲ ವ್ಯಕ್ತಿತ್ವವನ್ನು ಹಾಗೆಯೇ ಉಳಿಸಿಕೊಂಡಿದ್ದ. ಇದೊಂದು ಅಸಾಮಾನ್ಯ ವಿಷಯ. ಅವನ ಭಾವನಾತ್ಮಕ ಸ್ವಭಾವ, ಆದರ್ಶಕ್ಕಾಗಿ ಜೀವಿಸುವ ಛಲ–ಇವು ಅವನಲ್ಲಿ ಆಳವಾಗಿ ಬೇರೂರಿದ್ದು ಅವನ ಹಿಂದಿನ ದೃಷ್ಟಿಕೋನವನ್ನು ಸ್ಥಿರವಾಗಿ ನಿಲ್ಲಿಸಿದುವು. ಅವನಲ್ಲಿ ರಕ್ತಗತ ವಾಗಿದ್ದ ವಿಶಾಲದೃಷ್ಟಿಯೇ ಅವನು ಒಬ್ಬ ನಾಸ್ತಿಕನಾಗದಂತೆ, ಅಥವಾ ‘ಅದೃಷ್ಟವಾದಿ’ಯಾಗ ದಂತೆ ರಕ್ಷಿಸಿತು. ಅಜ್ಞೇಯತಾವಾದದಲ್ಲೇ ವಿರಮಿಸಲು ಅವನ ದೈವಶ್ರದ್ಧೆಯಿಂದೊಡಗೂಡಿದ ಅಸಾಮಾನ್ಯ ಬುದ್ಧಿ ಒಪ್ಪಲಿಲ್ಲ. ಸೃಷ್ಟಿ-ಸ್ಥಿತಿ-ಲಯಗಳ ಹಿಂದೆ, ಬರಿಗಣ್ಣಿಗೆ ಕಾಣದ ಶಕ್ತಿಯೊಂದು ಇದ್ದೇ ಇದೆ ಎಂದು ಅವನ ಅಂತರಂಗ ಮಾರ್ನುಡಿಯುತ್ತಿತ್ತು. ಅಜ್ಞೇಯತಾವಾದಕ್ಕಿಂತಲೂ ವಿಶಾಲದೃಷ್ಟಿಯ ಇತರ ಕೆಲವು ವಾದಗಳು ಹೆಚ್ಚು ಸಮಾಧಾನಕರವಾಗಿ ತೋರಿದುವಾದರೂ ಅವನು ಅಲ್ಲಿಗೇ ನಿಲ್ಲಲಿಲ್ಲ. ದೃಢನೆಲೆಯನ್ನು ಕಾಣುವ ಕಾತರದಲ್ಲಿ ಅವಸರದ ನಿರ್ಧಾರಕ್ಕೆ ಬರದೆ ಆ ಸಿದ್ಧಾಂತಗಳನ್ನು ಮತ್ತೆ ಅಷ್ಟೇ ಕಠಿಣ ವಿಶ್ಲೇಷಣೆಗೆ ಒಳಪಡಿಸಿದ. ಕೊನೆಗೆ ಅವುಗಳಲ್ಲೂ ಅಂತಿಮ ಸತ್ಯವನ್ನು ಕಾಣದೆ ಅವನ್ನೂ ತಿರಸ್ಕರಿಸಿದ.

ಈ ಮಧ್ಯೆ ಅವನು ಹಿಂದೂ ಸಾಮಾಜಿಕವ್ಯವಸ್ಥೆಯ ಕಡುವಿರೋಧಿಯಾಗಿದ್ದ. ಇಡೀ ರಾಷ್ಟ್ರವೇ ಬ್ರಾಹ್ಮಣವರ್ಗದ ಸೆರೆಯಾಳಾಗಿರುವಂತೆ ಅವನಿಗೆ ತೋರಿತು. ಜಾತಿ-ಮತ-ಧರ್ಮ ಗಳು ವಿಚಿತ್ರವಾಗಿ ಹೆಣೆದುಕೊಂಡಿರುವುದನ್ನು ಗಮನಿಸಿ ರೋಷಗೊಂಡ. ಭಾವಜೀವಿಗಳ ಜೀವನದಲ್ಲಿ ಇದೊಂದು ಸಂದಿಗ್ಧ ಹಂತ. ಈ ಹಂತದಲ್ಲಿ ಅವರಿಗೆ ನೈತಿಕತೆಯ ಮೇಲಿನ ವಿಶ್ವಾಸ ಕುಸಿಯುವ ಸಂಭವವಿರುತ್ತದೆ. ದೇವರು-ಧರ್ಮ-ಸಾಧನೆ ಎಲ್ಲವೂ ಸುಳ್ಳು, ಅರ್ಥವಿಹೀನ ಎಂದು ತೋರಿದಾಗ ವ್ಯಕ್ತಿಯನ್ನು ಸಂಯಮಗೊಳಿಸಲು ಅಲ್ಲೇನಿರುತ್ತದೆ? ನಿಜಕ್ಕೂ ನರೇಂದ್ರನಿಗೆ ಇದೊಂದು ಸಂಕ್ರಮಣ ಕಾಲ. ಆಗಲೂ ಅವನು ದಿಕ್ಕೆಡದೆ, ಮನಸ್ಸಿನ ಸ್ಥಿಮಿತವನ್ನು ಕಾಪಾಡಿ ಕೊಳ್ಳಲು ಸಮರ್ಥನಾದದ್ದು ಅವನ ಅದ್ಭುತ ಅಂತಸ್ಸತ್ವಕ್ಕೆ, ಸಂಯಮಕ್ಕೆ ಪುರಾವೆ.

ತನ್ನ ನೈಜಸ್ವರೂಪವನ್ನು ಅರಿತುಕೊಳ್ಳಲೇಬೇಕು, ಅಜ್ಞಾನದ ಬಲೆಯಿಂದ ಪಾರಾಗಬೇಕು ಎಂಬ ತೀವ್ರ ಚಡಪಡಿಕೆಯಲ್ಲಿ ನರೇಂದ್ರನ ಮನಸ್ಸು ಇಂದ್ರಿಯಸುಖವೆಂಬ ಸುಳಿಯ ಸೆಳೆತ ವನ್ನು ಮೀರಿ, ಉನ್ನತಮಟ್ಟದ ಬೌದ್ಧಿಕ ಪ್ರಪಂಚವನ್ನು ಹೊಕ್ಕು ಸಂಚರಿಸತೊಡಗಿತು. ದೇವ ನೊಬ್ಬನಿರುವುದಾದರೆ ಅವನನ್ನು ಸಾಕ್ಷಾತ್ಕರಿಸಿಕೊಳ್ಳುವ ಮಾರ್ಗವನ್ನು ಕಂಡುಕೊಳ್ಳುವುದು ಅವನ ಪಾಲಿಗೆ ಅತ್ಯಂತ ಅವಸರದ ಅಗತ್ಯವಾಗಿ ತೋರುತ್ತಿತ್ತು. ಜೀವನದ ರಹಸ್ಯವು ಬಗೆಹರಿಯುವ ವರೆಗೂ ಅವನ ಜೀವಕ್ಕೆ ಸಮಾಧಾನವಿಲ್ಲ. ನಾಸ್ತಿಕತೆ ಆಗಾಗ ತಲೆಹಾಕಿದರೂ ಅದು ತಾತ್ಕಾಲಿಕ ಮಾತ್ರ. ‘ಸಾಕ್ಷಾತ್ಕಾರ ಮಾಡಿಕೊಳ್ಳಲೇಬೇಕು! ಅದಕ್ಕೊಂದು ಮಾರ್ಗವಿರಲೇಬೇಕು! ಆದರೆ ಅದನ್ನು ಬಲ್ಲವರು ಯಾರು? ಆ ಪರಮಸತ್ಯವನ್ನು ಬಣ್ಣಿಸುವ, ವಿಶ್ಲೇಷಿಸುವ ತತ್ತ್ವಗಳೆಷ್ಟು, ಸಿದ್ಧಾಂತಗಳೆಷ್ಟು! ವೇದ ಪುರಾಣ ಬೈಬಲು ಕೊರಾನು ಮೊದಲಾದ ಶಾಸ್ತ್ರಗ್ರಂಥಗಳೆಷ್ಟು! ಆದರೆ ಇಷ್ಟೆಲ್ಲ ಇದ್ದರೂ, ತನ್ನ ಸಾಕ್ಷಾತ್ಕಾರಕ್ಕೆ ಅವು ನೆರವಾಗದಿದ್ದರೆ, ಸತ್ಯಧಾಮದ ಬಾಗಿಲನ್ನು ತೆರೆಸುವಲ್ಲಿ ಅಸಮರ್ಥವಾದರೆ, ಅವಕ್ಕೆ ಕಾಸಿನ ಬೆಲೆಯೂ ಇಲ್ಲ! ಅವು ಎಷ್ಟು ಸ್ವಾರಸ್ಯಕರ ವಾಗಿದ್ದರೂ, ರೋಚಕವಾಗಿದ್ದರೂ, ಅವುಗಳ ಪ್ರಯೋಜನ ಸೊನ್ನೆಯೇ ಸರಿ!’–ಇದು ಅವನ ವಿಚಾರಲಹರಿ.

ವಿಜ್ಞಾನ ಎಂದು ನಾವು ಯಾವುದನ್ನು ಕರೆಯುತ್ತೇವೆಯೋ ಅದು ನಮ್ಮ ಅನುಭವಕ್ಕೆ ಬರುವ ವಿಷಯಗಳನ್ನು ಮಾತ್ರ ಒಪ್ಪಿಕೊಳ್ಳುತ್ತದೆ. ಇಂದ್ರಿಯಗಳ ಮೂಲಕ ನಮ್ಮ ತಿಳಿವಿಗೆ ಬರುವ ಜ್ಞಾನವೆಂಬುದು ದೇಶಕಾಲಗಳೇ ಮೊದಲಾದ ಉಪಾಧಿಗಳಿಗೆ ಒಳಪಟ್ಟದ್ದು. ಆದ್ದರಿಂದ ಇಂದ್ರಿಯಗೋಚರವಾದ ಈ ಬಾಹ್ಯಪ್ರಪಂಚ ‘ನಿಜಕ್ಕೂ’ ಹೇಗಿದೆ ಎಂದು ನಾವು ಅರಿತುಕೊಳ್ಳ ಲಾರೆವು. ಹಾಗೆಯೇ, ಮಾನವನ ನಿಜಸ್ವಭಾವವನ್ನು ಮಾನವನು ಕಂಡುಕೊಳ್ಳಲಾರ–ಈ ವಾದ ವನ್ನು ಪಾಶ್ಚಾತ್ಯ ವಿಜ್ಞಾನವೂ ಒಪ್ಪಿಕೊಳ್ಳುತ್ತದೆ. ನರೇಂದ್ರನಿಗೆ ಇದು ಸಮ್ಮತ. ಆದರೆ ಸನಾತನ ಸತ್ಯವೆಂಬುದು ಬುದ್ಧಿಗೂ ಇಂದ್ರಿಯಗಳಿಗೂ ಅತೀತವಾದ ವಸ್ತು. ಹೀಗಿರುವಾಗ, ಈ ಬುದ್ಧಿ ಮತ್ತು ಇಂದ್ರಿಯಗೋಚರವಾದ ವಿಷಯಗಳನ್ನೇ ಸಂಪೂರ್ಣವಾಗಿ ತಿಳಿಯಲಾರದ ವಿಜ್ಞಾನ ವೆಂಬುದು, ಅತೀಂದ್ರಿಯ ವಸ್ತುವಾದ ದೇವರನ್ನು ಮುಟ್ಟಲು ಹೇಗೆ ಸಾಧ್ಯ? ಆದ್ದರಿಂದ ನಮ್ಮ ‘ನಿಜವ್ಯಕ್ತಿತ್ವ’ದ ಬಗ್ಗೆ ನಮಗಿರುವ ಕಲ್ಪನೆಯಾದರೂ ಇಂದ್ರಿಯಾನುಭವಗಳಿಗೆ ಒಳಪಟ್ಟಿರು ವಂಥದೇ; ಆದ್ದರಿಂದ ಸಾಪೇಕ್ಷವಾದದ್ದೇ. ಇದಕ್ಕೆ ಅತೀತವಾದ ‘ಆತ್ಮ’ವೊಂದಿದೆಯೆಂಬುದನ್ನು ಸಾಬೀತುಪಡಿಸುವಲ್ಲಿ ವಿಜ್ಞಾನ ಸೋತಿದೆ–ಆದ್ದರಿಂದ ‘ಅಂತಿಮ ಸತ್ಯ’ದ ಬಗ್ಗೆ ಯಾವುದೇ ನಿರ್ಧಾರಕ್ಕೆ ಬರಲು ಪಾಶ್ಚಾತ್ಯ ಪಂಡಿತರು ಅಸಮರ್ಥರಾಗಿದ್ದಾರೆ ಎಂಬುದನ್ನು ನರೇಂದ್ರ ಕಂಡುಕೊಂಡ. 

ಆದರೆ ಪಾಶ್ಚಾತ್ಯರ ಭೌತಿಕ ಜ್ಞಾನದ ಮೇಲೂ ಅವರ ವಿಶ್ಲೇಷಣಾ ವಿಧಾನಗಳ ಮೇಲೂ ಅವನಿಗೆ ವಿಶೇಷ ಗೌರವವಿತ್ತು. ಈ ಬಗೆಯ ವಿಶ್ಲೇಷಣಾ ಮಾರ್ಗಗಳಿಂದ ಅವನು ಶ್ರೀರಾಮ ಕೃಷ್ಣರ ಅತೀಂದ್ರಿಯ ಅನುಭವಗಳನ್ನು ವಿಮರ್ಶಿಸಿ ನೋಡಿದ. ಯಾವಯಾವುದು ಈ ಪರೀಕ್ಷೆ ಯಲ್ಲಿ ಉತ್ತೀರ್ಣವಾಯಿತೋ ಅಂಥದನ್ನು ಮಾತ್ರ ಒಪ್ಪಿಕೊಂಡ. ಅವನು ಸತ್ಯ ಸಾಕ್ಷಾತ್ಕಾರಕ್ಕಾಗಿ ಅಷ್ಟೊಂದು ಕಾತರನಾಗಿದ್ದರೂ ಯಾವುದೋ ಭಯದಿಂದಲೋ ಇಲ್ಲವೆ ಹೊರಗಡೆಯ ಒತ್ತಡ ದಿಂದಲೋ ಯಾವುದನ್ನೂ ಒಪ್ಪಿಕೊಳ್ಳುವವನಲ್ಲ; ಇವು ಯಾವುವೂ ತನ್ನ ವೈಚಾರಿಕ ದೃಷ್ಟಿಗೆ ಹೊಂದಿಕೆಯಾಗದಿದ್ದರೆ ಪ್ರಾಮಾಣಿಕನಾದ ನಾಸ್ತಿಕನಾಗಿಯೇ ಉಳಿಯಲೂ ಆತ ಸಿದ್ಧ. ಸತ್ಯ ಸಾಕ್ಷಾ ತ್ಕಾರಕ್ಕಾಗಿ ಪ್ರಪಂಚದ ಸಕಲ ಭೋಗಗಳನ್ನೂ, ಅಷ್ಟೇಕೆ, ತನ್ನ ಜೀವನವನ್ನೂ ಬಲಿದಾನ ಮಾಡಲು ಅವನು ಸಿದ್ಧನಾಗಿದ್ದ. ಪಾಶ್ಚಾತ್ಯ-ಪೌರ್ವಾತ್ಯ ತತ್ತ್ವಶಾಸ್ತ್ರ ಗಳು ಹಾಗೂ ವಿಜ್ಞಾನ-ಕಲೆ ಗಳನ್ನೆಲ್ಲ ಅವನು ಅಷ್ಟೊಂದು ಅಧ್ಯಯನ ಮಾಡಿದ್ದು ಕೇವಲ ಲೋಕಮಾನ್ಯತೆಗಾಗಿ ಅಲ್ಲ, ಭಗವತ್ಸಾಕ್ಷಾತ್ಕಾರಕ್ಕಾಗಿ. ತಾನು ಅರಸುತ್ತಿರುವ ಸತ್ಯವನ್ನು ಕಂಡುಕೊಳ್ಳಲು ವಿಚಾರಮಾರ್ಗದ ತುತ್ತತುದಿಯವರೆಗೂ ಹೋಗಿ, ಇನ್ನು ಮುಂದುವರಿಯಲು ಸಾಧ್ಯವಿಲ್ಲದಾದಾಗ, ಪುನಃ ತನ್ನಲ್ಲಿ ಸಹಜವಾಗಿದ್ದ ಶ್ರದ್ಧೆಯ ಪಥಕ್ಕೆ ಮರಳಿ ಬರಬೇಕಾಯಿತು. ಇದನ್ನು ನಾವು ಮುಂದೆ ನೋಡಲಿದ್ದೇವೆ.

ನರೇಂದ್ರನ ಅಧ್ಯಯನವು ಕೇವಲ ತತ್ತ್ವಶಾಸ್ತ್ರ, ಭೌತವಿಜ್ಞಾನಗಳಿಗೆ ಸೀಮಿತವಾಗಿರಲಿಲ್ಲ. ಇತಿಹಾಸವೂ ಅವನ ನೆಚ್ಚಿನ ವಿಷಯಗಳಲ್ಲೊಂದು. ಬೇರೆಬೇರೆ ಸನ್ನಿವೇಶಗಳಲ್ಲಿ, ಒತ್ತಡಗಳಲ್ಲಿ ಮನುಷ್ಯನ ಶೀಲ-ನಡವಳಿಕೆಗಳು ಹೇಗೆ ಉನ್ನತಿಗೇರಿದುವು, ಹೇಗೆ ಪತನ ಹೊಂದಿದುವು– ಇದನ್ನು ಅರಿತುಕೊಳ್ಳುವಲ್ಲಿ ಅವನಿಗೆ ವಿಶೇಷ ಉತ್ಸಾಹ, ಕುತೂಹಲ. ಅವನ ಪಾಲಿಗೆ ಚರಿತ್ರೆ ಯೆಂದರೆ ರಾಷ್ಟ್ರಗಳ ಆಶೋತ್ತರಗಳ ಮತ್ತು ಪ್ರಗತಿ-ಅಪಗತಿಗಳ ಶತಮಾನಗಳ ದಾಖಲೆ.

ಕಾವ್ಯ-ಕವನಗಳಲ್ಲಿ ನರೇಂದ್ರನಿಗಿದ್ದ ಆಸಕ್ತಿ ಅಪಾರ. ಏಕೆಂದರೆ ಕಾವ್ಯವು ಭಾವುಕರ ಭಾಷೆ. ಅವನ ಕಾವ್ಯಾಕಾಶದ ಧ್ರುವತಾರೆಯೆಂದರೆ ವರ್ಡ್ಸ್ ವರ್ತ್ ಮಹಾಕವಿ. ಇತಿಹಾಸ, ತತ್ತ್ವಶಾಸ್ತ್ರ, ಕಾವ್ಯ, ವಿಜ್ಞಾನ ಇವೆಲ್ಲ ಒಂದೇ ಪರಮಸತ್ಯದ ಬೇರೆ ಬೇರೆ ಮುಖಗಳೆಂಬಂತೆ ವರ್ಡ್ಸ್ವರ್ತ್ ಕವಿಯಲ್ಲಿ ದಾರ್ಶನಿಕನ ಶಕ್ತಿಯಿತ್ತು. ಆದ್ದರಿಂದಲೇ ನರೇಂದ್ರನಿಗೆ ಅವನ ಕಾವ್ಯಗಳಲ್ಲಿ ವಿಶೇಷ ಒಲವು.

ನರೇಂದ್ರ ಯಾವುದೇ ಕಾರ್ಯದಲ್ಲಿ ತೊಡಗಿರಲಿ, ಅದನ್ನವನು ಅದರ ಪ್ರಾಶಸ್ತ್ಯ. ಯುಕ್ತತೆ ಗಳನ್ನು ಮನಗಂಡು ಮಾಡುತ್ತಿದ್ದನೇ ಹೊರತು ಎಂದಿಗೂ ಮತ್ತೊಬ್ಬರ ಬಲಾತ್ಕಾರದಿಂದಲ್ಲ. ಅವನದು ಸ್ವತಂತ್ರ ಮನೋಭಾವ. ಆತ್ಮವಿಕಾಸದ ಹಾದಿಯಲ್ಲಿ ಸ್ವತಂತ್ರ ಮನೋಭಾವವು ಆಧಾರಭೂತವಾದ ಗುಣ ಎಂಬುದು ಅವನ ಅಭಿಮತ. ಆದರೆ ಈ ಸ್ವಾತಂತ್ರ್ಯದ ಹಿನ್ನೆಲೆಯಲ್ಲಿ ವಿವೇಕಪ್ರಜ್ಞೆಯಿರಬೇಕು; ಎಂದರೆ, ಸ್ವಾತಂತ್ರ್ಯವನ್ನು ವಿವೇಕಪೂರ್ಣವಾಗಿ ಉಪಯೋಗಿಸಿಕೊಳ್ಳ ಬೇಕು ಎಂಬುದನ್ನು ಆತ ಬಲ್ಲ. ದಾಸ್ಯದಲ್ಲಿರುವವನು ಯಜಮಾನನ ಅಂಕೆಗೊಳಗಾಗಿ ನೀತಿ ನಿಯಮಗಳನ್ನು ಪಾಲಿಸಿದರೆ, ಯಜಮಾನನಾದವನು ಅವುಗಳನ್ನು ಸ್ವತಂತ್ರನಾಗಿಯೇ ಪಾಲಿಸು ತ್ತಾನೆ. ವಿವೇಕಿಯಾದವನು ಯಜಮಾನನಿಗೆ ಸಮ, ಅವಿವೇಕಿಯಾದವನು ದಾಸನಿಗೆ ಸಮ ಎಂಬುದು ನರೇಂದ್ರನ ಅಭಿಮತ. ತಾನು ಸದಾಚಾರಸಂಪನ್ನನಾಗಿರಬೇಕು, ನೀತಿವಂತನಾಗಿರ ಬೇಕು ಎಂಬುದರ ಕಡೆಗೆ ಅವನು ವಿಶೇಷ ಗಮನ ಹರಿಸುತ್ತಿದ್ದ. ಆಗಾಗ ಅವನಲ್ಲಿ ನಾಸ್ತಿಕತೆಯ ಚಿಹ್ನೆಗಳು ಕಂಡುಬಂದರೂ, ವಿಲಾಸಪೂರ್ಣವಾದ ಪ್ರಾಪಂಚಿಕ ಜೀವನವನ್ನು ಮಾತ್ರ ಅವನು ತಿರಸ್ಕರಿಸಿಬಿಟ್ಟಿದ್ದ.

ನಾವು ಹಿಂದೆಯೇ ಕಂಡಂತೆ ಅವನಲ್ಲಿ ಸಂನ್ಯಾಸದ ಮನೋವೃತ್ತಿ ಸಹಜವಾಗಿಯೇ ಬೆಳೆದು ಬಂದಿತ್ತು. ಆದರೂ ಅವನು ಇಹಜೀವನವನ್ನೇ ಕಡೆಗಣಿಸುವವನಲ್ಲ. ಬದುಕಿದರೆ ಚೆನ್ನಾಗಿಯೇ ಬದುಕಬೇಕೆನ್ನುವವನು ಅವನು. ಅವನಲ್ಲಿ ಹಸುಳೆಯ ಹುಮ್ಮಸ್ಸಿನೊಂದಿಗೆ ತೀವ್ರ ಆಧ್ಯಾತ್ಮಿಕ ಮನೋವೃತ್ತಿಯೂ ಸೇರಿಕೊಂಡಿತ್ತು. ಇದನ್ನು ಆಗಾಗ ಅವನ ಮಾತುಗಳಲ್ಲಿ ಕಾಣಬಹು ದಾಗಿತ್ತು. ಅವನೊಮ್ಮೆ ಬಿ.ಎಲ್. ಪರೀಕ್ಷೆಗಾಗಿ ಸಿದ್ಧತೆ ನಡೆಸುತ್ತಿದ್ದ ಸಮಯದಲ್ಲಿ ಇದ್ದ ಕ್ಕಿದ್ದಂತೆ ತನ್ನ ಸ್ನೇಹಿತನ ಮುಂದೆ ಉದ್ಗರಿಸುತ್ತಾನೆ: “ನಾನು ಈ ಪರೀಕ್ಷೆಗೆ ಕುಳಿತುಕೊಳ್ಳುವ ಆಲೋಚನೆಯನ್ನೇ ಬಿಟ್ಟುಬಿಡಬೇಕೆಂದಿದ್ದೇನೆ. ಈ ಪರೀಕ್ಷೆಗಳಿಗೆಲ್ಲ ಏನರ್ಥ! ನಾನು ಸ್ವತಂತ್ರ ನಾಗಬೇಕು!” ಮದುವೆಯೆಂಬುದು ಆಧ್ಯಾತ್ಮಿಕ ಜೀವನಕ್ಕೆ ಒಂದು ಅಡ್ಡಿ ಎಂಬುದನ್ನು ಅವನು ತಾರುಣ್ಯದಲ್ಲೇ ಗುರುತಿಸಿದ್ದ. ಇನ್ನೊಮ್ಮೆ ಅದೇ ಸ್ನೇಹಿತನಿಗೆ ಹೇಳುತ್ತಾನೆ: “ನೀನು ಮದುವೆ ಯಾದವನು, ಗೃಹಸ್ಥಜೀವನಕ್ಕೆ ಬದ್ಧನಾದವನು. ಆದರೆ ನಾನು ಸ್ವತಂತ್ರ. ನನ್ನದು ಸಂನ್ಯಾಸ ಜೀವನ. ಇದು ಖಂಡಿತ!” ಅವನ ನಾಸ್ತಿಕ ಸಿದ್ಧಾಂತವೇ ಅವನಿಗೆ ಜೀವನದ ಪೊಳ್ಳುತನವನ್ನು ತೋರಿಸಿಕೊಟ್ಟಿತ್ತು. ಅದನ್ನು ಧಿಕ್ಕರಿಸಿ ನಿಲ್ಲುವ ಏಕಮಾತ್ರ ಮಾರ್ಗವೆಂದರೆ ಸಂನ್ಯಾಸ ಎಂಬ ನಿರ್ಧಾರ ಅವನಲ್ಲಿ ಬೆಳೆಯುತ್ತಿತ್ತು.

ಈ ಹಂತದಲ್ಲಿ ನರೇಂದ್ರನಿಗೆ ಅವನ ಬುದ್ಧಿವಂತಿಕೆಯೇ ಸತ್ಯಸಾಕ್ಷಾತ್ಕಾರಕ್ಕೆ ದೊಡ್ಡ ತಡೆ ಯಾಗಿ ಪರಿಣಮಿಸಿತ್ತು. ಏಕೆಂದರೆ ಉಪನಿಷತ್ತು ಘೋಷಿಸುತ್ತದೆ–‘ನಾಯಮಾತ್ಮಾ ಪ್ರವಚ ನೇನ ಲಭ್ಯಃ ನ ಮೇಧಯಾ ನ ಬಹುನಾ ಶ್ರುತೇನ’ ಎಂದರೆ, ಆತ್ಮಸಾಕ್ಷಾತ್ಕಾರವು ಪ್ರವಚನದಿಂದ ಪ್ರಾಪ್ತವಾಗು ವಂಥದಲ್ಲ; ಬುದ್ಧಿಯಿಂದಲೂ ಶಾಸ್ತ್ರ ಪಾಂಡಿತ್ಯದಿಂದಲೂ ಸಿದ್ಧಿಸುವಂಥದಲ್ಲ. ಹಾಗಾದರೆ ಈ ಬುದ್ಧಿಯ ಸದ್ದನ್ನು ಅಡಗಿಸುವುದು ಹೇಗೆ? ಶ್ರದ್ಧೆಯ ಹೆಸರಿನಲ್ಲಿ ಬುದ್ಧಿಯ ಬೆಳವಣಿಗೆಯನ್ನು ಹೊಸಕಿಬಿಡಬೇಕೆ? ಕೂಡದು! ಬುದ್ಧಿಯ ಬೆಳವಣಿಗೆಯನ್ನು ಕುಂಠಿತಗೊಳಿಸು ವುದಲ್ಲ ಉಪಾಯ, ವೃದ್ಧಿಗೊಳಿಸುವುದೇ ಉಪಾಯ! ಬುದ್ಧಿಯು ಎಷ್ಟು ವಿಶಾಲವಾಗಿ, ಆಳ ವಾಗಿ, ಉನ್ನತವಾಗಿ, ಸಮರ್ಥವಾಗಿ ಬೆಳೆಯಲು ಸಾಧ್ಯವೋ, ಅಷ್ಟೂ ಬೆಳೆಯಲು ಅವಕಾಶ ಕೊಟ್ಟು, ಅದು ತನ್ನ ಪರಿಮಿತಿಯನ್ನು ಅರಿತುಕೊಳ್ಳುವಂತೆ ಮಾಡುವುದೇ ಉಪಾಯ. ಅನಂತ ನಾದ ಭಗವಂತನನ್ನು ಸಾಂತವಾದ ತಾನು ಅರಿತುಕೊಳ್ಳಲಾಗದು ಎಂಬುದರ ಅರಿವು ಮನಸ್ಸಿಗಾಗ ಬೇಕು. ಆಗ ಹೃದಯದ ಭಾವದೊಂದಿಗೆ ಅದು ಮಿಳಿತವಾಗುತ್ತದೆ. ಭಾವದಿಂದ ಭಕ್ತಿ ಉದಿಸು ತ್ತದೆ; ಭಕ್ತಿ ಉದಿಸಿದಾಗ ಭಗವಂತನ ಕೃಪೆಯಾಗುತ್ತದೆ. ಕೃಪೆಯಿಂದಲೇ ಸಾಕ್ಷಾತ್ಕಾರ. ಶ್ರೀರಾಮ ಕೃಷ್ಣರ ದಿವ್ಯ ಸಂಪರ್ಕದಿಂದ ನರೇಂದ್ರನ ಬುದ್ಧಿ-ವೈಚಾರಿಕತೆಗಳೆಲ್ಲ ಪರಿಪಾಕ ಹೊಂದಲಿವೆ. ಅವನು ಇನ್ನೂ ಎಷ್ಟೋ ಕಷ್ಟಗಳನ್ನು ಎದುರಿಸಬೇಕಾಗಿದೆ. ಅವನ ಎಷ್ಟೋ ಸಂಶಯಗಳೆಲ್ಲ ಇನ್ನೂ ಪರಿಹಾರವಾಗಬೇಕಾಗಿವೆ. ಹೀಗೆ ನಾನಾ ಬಗೆಯ ಬವಣೆಗಳ ಮೂಸೆಯಲ್ಲಿ ಪರಿಪಾಕ ಹೊಂದಿ, ಕಟ್ಟಕಡೆಗೆ ಗುರುವಿನ ಪಾದಗಳಿಗೆ ಶರಣಾಗಿ, ಗುರು ಹೇಳಿದ್ದನ್ನೆಲ್ಲ ಮರುಮಾತಿಲ್ಲದೆ ಒಪ್ಪಿಕೊಳ್ಳುವ ಹದಕ್ಕೆ ಬರಬೇಕಾಗಿದೆ! ಶ್ರೀರಾಮಕೃಷ್ಣರಿಂದ ಪ್ರತಿಯೊಂದು ವಿಷಯವನ್ನೂ ಸಂದೇಹಕ್ಕೆ ಆಸ್ಪದವಿಲ್ಲದ ರೀತಿಯಲ್ಲಿ ಅರಿತು ಒಪ್ಪಿಕೊಳ್ಳುವಂತಾಗಲು ಅವನು ಹೆಜ್ಜೆಹೆಜ್ಜೆಗೂ ಹೋರಾಡಿದ. ಆದರೆ ಹಾಗೆ ಹೋರಾಡಿ ತಿಳಿದುಕೊಂಡಮೇಲೆ ಮಾತ್ರ ಆ ವಿಷಯದಲ್ಲಿ ಇನ್ನು ಸಂಶಯವುಳಿಯಲಿಲ್ಲ; ಆ ವಿಷಯದ ಪರಿಪೂರ್ಣ ಜ್ಞಾನ ಅವನಿಗೆ ಉಂಟಾಯಿತು ಎಂದೇ ಅರ್ಥ. ಒಂದು ಗಮನಾರ್ಹ ಅಂಶವೆಂದರೆ, ಕೊನೆಗೊಂದು ದಿನ ತಾನು ಯಶಸ್ವಿಯಾಗಿಯೇ ತೀರುತ್ತೇನೆ ಎನ್ನುವ ವಿಶ್ವಾಸ ಅವನಲ್ಲಿ ಯಾವಾಗಲೂ ಇದ್ದೇ ಇತ್ತು. ನಿಜ, ಅವನಂತಹ ಪರಿಶುದ್ಧ ಹೃದಯನಿಗಲ್ಲದೆ ಮತ್ತಾರಿಗೆ ದೇವರು ಲಭಿಸಿಯಾನು?

ಅಂತೂ ಈ ಬಗೆಯ ಸಂಶಯ ಹಾಗೂ ಅನಿಶ್ಚಿತತೆಗಳ ಸುಳಿಯಿಂದ ಬೇಗನೆ ಪಾರಾಗಬೇಕು ಎಂದು ಹಂಬಲಿಸುತ್ತಿದ್ದಾನೆ ನರೇಂದ್ರ; ಎಂದೆಂದೂ ಬದಲಾಗದ ನಿತ್ಯಜ್ಞಾನಕ್ಕಾಗಿ, ಶಾಂತಿ ಸಮಾಧಾನಗಳನ್ನು ಸುರಿಸುವಂತಹ ಸತ್ಯಜ್ಞಾನಕ್ಕಾಗಿ ಹಂಬಲಿಸುತ್ತಿದ್ದಾನೆ. ಅವನಲ್ಲಿ ಆಗಾಗ ಕಂಡುಬರುವ ‘ನಾಸ್ತಿಕತೆ’ಯೂ ನಿಜವಾದ ನಾಸ್ತಿಕತೆಯೇನಲ್ಲ; ಅದು ಅವನ ಹೃದಯದ ವ್ಯಾಕುಲತೆಯ ಇನ್ನೊಂದು ರೂಪ ಅಷ್ಟೆ. ‘ಸಾಕ್ಷಾತ್ಕಾರವಿನ್ನೂ ಆಗಲಿಲ್ಲವಲ್ಲ’ ಎಂಬ ತೀವ್ರ ಚಡಪಡಿಕೆಯಿಂದಾಗಿ ಅವನನ್ನು ಒಂದು ಬಗೆಯ ಶೂನ್ಯತೆ, ಜುಗುಪ್ಸೆ ಆವರಿಸಿಬಿಟ್ಟಿದೆ. ಆದರೆ ಅದೇಕೆಂಬುದನ್ನು ಹೇಳಲು ಮಾತ್ರ ಅವನಿಂದ ಸಾಧ್ಯವಾಗುತ್ತಿಲ್ಲ. ಒಂದು ಬಗೆಯ ದಿಕ್ಕೆಟ್ಟ ಮನಸ್ಥಿತಿಯಲ್ಲಿ ಹೇಳುತ್ತಾನೆ: “ನನಗೊಂದೂ ತೋಚುತ್ತಿಲ್ಲ” ಎಂದು. ಸಾಮಾನ್ಯ ತತ್ತ್ವವಾದಿ ಯಾದರೆ ‘ನನಗೆ ಗೊತ್ತಿಲ್ಲ’ ಎಂದು ಉದಾಸೀನನಾಗಿ ಹೇಳಿ ಸುಮ್ಮನಾಗಿಬಿಡುತ್ತಾನೆ. ಆದರೆ ನಿಜವಾದ ವ್ಯಾಕುಲತೆಯಿರುವ, ಮುಂದೆ ಸಂತನಾಗಲಿರುವ ಸಾಧಕ ನರೇಂದ್ರ ಯಾತನೆ ತುಂಬಿದ ಹೃದಯದಿಂದ ಹೇಳುತ್ತಾನೆ: “ನನಗೊಂದೂ ತೋಚುತ್ತಿಲ್ಲ, ಈಗೇನು ಮಾಡಲಿ?” ಎಂದು. ಸಾಮಾನ್ಯರಾದ ಪ್ರಾಪಂಚಿಕರಿಗೆಲ್ಲ ಈ ಆದರ್ಶಗಳು, ಸಂಸ್ಕೃತಿ-ಪರಂಪರೆಗಳು ಉಳಿದರೂ ಅಷ್ಟೆ, ಅಳಿದರೂ ಅಷ್ಟೆ–ಅದರಿಂದ ಅವರಿಗೆ ಸಂತೋಷವೂ ಇಲ್ಲ, ದುಃಖವೂ ಇಲ್ಲ. ಏಕೆಂದರೆ, ಈ ಪ್ರಪಂಚ ಎಂಥ ಸಂಕಟಮಯವಾದದ್ದು ಎಂಬ ಅರಿವಾದಾಗಲೇ ನಿಜವಾದ ವೈರಾಗ್ಯವುದಿಸು ವುದು. ಆದರೆ ಪ್ರಾಪಂಚಿಕರಿಗೆ ಈ ಪ್ರಪಂಚ ಸಂಕಟಮಯವೆಂಬ ಅರಿವೇ ಇಲ್ಲದಿರುವಾಗ ಅವರಿಗೆಲ್ಲಿ ಬರಬೇಕು ವೈರಾಗ್ಯ!

ಒಂದು ಬಹುಮುಖ್ಯ ವಿಷಯವೆಂದರೆ, ತನ್ನ ಅಂತರಂಗದಲ್ಲಿ ನಾನಾಬಗೆಯ ತುಮುಲಗಳು ನಡೆಯುತ್ತಿದ್ದರೂ ನರೇಂದ್ರ ತನ್ನ ಆಧ್ಯಾತ್ಮಿಕ ಸಾಧನೆಗಳನ್ನು ಬಿಡದೆಮಾಡಿಕೊಂಡು ಬರುತ್ತಿದ್ದ. ತಪ್ಪದೆ ಧ್ಯಾನಾಭ್ಯಾಸ ಮಾಡುತ್ತಿದ್ದ. ಧ್ಯಾನದಿಂದ ಮನಸ್ಸನ್ನು ಶಾಂತಗೊಳಿಸಲು ಸಾಧ್ಯವಾದ್ದ ರಿಂದ ಅಪೂರ್ವ ಸಮಾಧಾನವನ್ನು ಅನುಭವಿಸುತ್ತಿದ್ದ. ಹೀಗೆ ಮನಸ್ಸನ್ನು ಶಾಂತಗೊಳಿಸಿದಾಗ, ತನ್ನ ಮನಸ್ಸಿನ ಅತ್ಯಂತ ಒಳವಲಯದೊಳಗೆ ಪ್ರವೇಶಿಸಲು ಸಾಧ್ಯವಾಗುತ್ತಿತ್ತು. ಆಗ ಅವನ ಸಂಶಯದ ಮನಸ್ಸು ಹೊರಗೆಯೇ ನಿಂತಿರಬೇಕಾಗಿತ್ತು. ಏಕೆಂದರೆ ಈ ಸಂಶಯದ ಮನಸ್ಸಿಗೆ ಅಂತರಂಗದ ತೀರಾ ಒಳಕ್ಕೆ ಪ್ರವೇಶವಿಲ್ಲ. ಶ್ರೀರಾಮಕೃಷ್ಣರು ಅವನಿಗೆ ಮೊದಲ ಎರಡು- ಮೂರು ಸಂದರ್ಶನಗಳಲ್ಲಿ ಮಾಡಿಸಿಕೊಟ್ಟ ಅನುಭವಗಳೆಲ್ಲ ಈಗ ಧ್ಯಾನಕ್ಕೆ ವಿಶೇಷವಾಗಿ ನೆರವಾಗುತ್ತಿದ್ದುವು. ಪರಮಸತ್ಯದಲ್ಲಿ ಮನಸ್ಸನ್ನು ನೆಲೆನಿಲ್ಲಿಸಲು ಸಹಾಯವಾಗುತ್ತಿದ್ದುವು. ಅಲ್ಲದೆ ಶ್ರೀರಾಮಕೃಷ್ಣರ ಈ ಮಾತುಗಳು ಅವನ ಮೇಲೆ ವಿಶೇಷವಾಗಿ ಪ್ರಭಾವ ಬೀರಿದ್ದುವು: “ನೋಡು, ಮನುಷ್ಯನ ಪ್ರಾಮಾಣಿಕ ಪ್ರಾರ್ಥನೆಯನ್ನು ಭಗವಂತ ಖಂಡಿತ ಕೇಳುತ್ತಾನೆ. ನಾನು ಪ್ರತಿಜ್ಞೆ ಮಾಡಿ ಹೇಳುತ್ತೇನೆ: ಭಗವಂತನನ್ನು ನೀನು ನನ್ನನ್ನು ಕಾಣುವುದಕ್ಕಿಂತ ಹೆಚ್ಚು ಸ್ಪಷ್ಟವಾಗಿ ಕಾಣಬಲ್ಲೆ; ನನ್ನೊಂದಿಗೆ ಮಾತನಾಡುವುದಕ್ಕಿಂತ ಹೆಚ್ಚಿನ ಆತ್ಮೀಯತೆಯಿಂದ ಅವನೊಂದಿಗೆ ಮಾತನಾಡಬಲ್ಲೆ. ಅವನ ವಾಣಿಯನ್ನು ಕೇಳಬಹುದು, ಅವನ ಸ್ಪರ್ಶವನ್ನು ಅನುಭವಿಸಬಹುದು.” ಮತ್ತೆ ಹೇಳುತ್ತಿದ್ದರು: “ನೀನು ನಾನಾ ದೇವತೆಗಳ ಅಸ್ತಿತ್ವವನ್ನು ಒಪ್ಪದೆ ಬದಿಗೊತ್ತಿದರೆ ಏನೂ ಚಿಂತೆಯಿಲ್ಲ. ಆದರೆ ನಿನಗೆ ಆ ಒಂದು ಪರಮಸತ್ಯದ ಮೇಲೆ ವಿಶ್ವಾಸವಿರುವುದಾದರೆ, ಸಮಸ್ತ ವಿಶ್ವದ ಚಲನೆಯನ್ನೇ ನಿಯಂತ್ರಿಸುವಂತಹ ಆ ಪರಮ ಚೈತನ್ಯದ ಮೇಲೆ ವಿಶ್ವಾಸವಿರುವುದಾದರೆ ಸಾಕು. ‘ಹೇ ಭಗವಂತ, ನೀನು ಯಾರೆಂಬುದು ನನಗೆ ತಿಳಿಯದು. ಆದ್ದರಿಂದ ನೀನೇ ನಿನ್ನ ಸ್ವರೂಪವನ್ನು ತೋರುವ ಕೃಪೆ ಮಾಡು’ ಅಂತ ಪ್ರಾರ್ಥಿಸಿಕೊ. ನಿನ್ನ ಪ್ರಾರ್ಥನೆ ಪ್ರಾಮಾಣಿಕ ವಾದದ್ದಾದರೆ ಅವನದನ್ನು ಕೇಳಿಯೇ ಕೇಳುತ್ತಾನೆ.” ಶ್ರೀರಾಮಕೃಷ್ಣರ ಈ ಮಾತುಗಳಿಂದ ನರೇಂದ್ರನಿಗೆ ತನ್ನ ಸಾಧನೆಯ ಕಡೆಗೆ ಇನ್ನಷ್ಟು ಮನಸ್ಸು ಕೊಡಲು ಸಹಾಯವಾಯಿತು; ವಿಶ್ವಾಸ ಹೆಚ್ಚಿತು. ಇನ್ನೂ ಹೆಚ್ಚಿನ ರಭಸದಿಂದ ಆಧ್ಯಾತ್ಮಿಕ ಸಾಧನೆಯಲ್ಲಿ ನಿರತನಾದ. ಹ್ಯಾಮಿಲ್ಟನ್ ಎಂಬ ಪಾಶ್ಚಾತ್ಯ ಚಿಂತಕನೊಬ್ಬ ಹೇಳುತ್ತಾನೆ: ‘ಮಾನವನ ಬುದ್ಧಿಶಕ್ತಿಯೆಂಬುದು ಭಗವಂತನ ಕುರಿತಾಗಿ ಎಲ್ಲೋ ಕಿಂಚಿತ್ ಪರಿಚಯ ಮಾಡಿಕೊಳ್ಳಬಲ್ಲುದೇ ಹೊರತು ಅದಕ್ಕಿಂತ ಮುಂದೆ ಹೋಗಲಾರದು. ಅವನನ್ನು ಯಥಾರ್ಥವಾಗಿ ತಿಳಿದುಕೊಳ್ಳಲು ಈ ಬುದ್ಧಿಯಿಂದ ಸಾಧ್ಯವಿಲ್ಲ. ಬುದ್ಧಿಯಿಂದ ತಿಳಿದುಕೊಳ್ಳಬಹುದಾದ ಸಿದ್ಧಾಂತಾದಿಗಳೆಲ್ಲ ಮುಗಿದಮೇಲೆ ನಿಜವಾದ ಧರ್ಮ, ನಿಜವಾದ ಆಧ್ಯಾತ್ಮಿಕತೆ ಪ್ರಾರಂಭವಾಗುತ್ತದೆ.’ ಈ ಮಾತುಗಳು ಮತ್ತೆಮತ್ತೆ ನರೇಂದ್ರನ ಮನಸ್ಸಿಗೆ ಬರಲಾರಂಭಿಸಿದವು. ಏಕೆಂದರೆ ಅವನಿಗೆ ಮೊದಮೊದಲು ತನ್ನ ಬುದ್ಧಿಶಕ್ತಿಯ ಮೇಲೆ ವಿಶೇಷ ನಂಬಿಕೆಯಿತ್ತು; ಇಂತಹ ತೀಕ್ಷ ್ಣ ಬುದ್ಧಿಯಿಂದ ಏನನ್ನಾದರೂ ತಿಳಿದುಕೊಳ್ಳಬಲ್ಲೆನೆಂಬ ವಿಶ್ವಾಸವಿತ್ತು. ಆದರೆ ಈ ಬುದ್ಧಿಯಿಂದ ಏನನ್ನು ಬೇಕಾದರೂ ತಿಳಿದುಕೊಳ್ಳಬಹುದು, ಆದರೆ ಭಗವಂತನನ್ನು ತಿಳಿದುಕೊಳ್ಳುವುದು ಮಾತ್ರ ಅಸಾಧ್ಯ ಎಂಬುದು ಈಗೀಗ ಅವನಿಗೆ ಅರ್ಥವಾಗು ತ್ತಿದೆ! ಸಾಧನೆ ಮಾಡುತ್ತ ಹೋದಂತೆ, ಉಪನಿಷದ್ವಾಕ್ಯಗಳ ಈ ಸತ್ಯ ಅವನಿಗೆ ಮನವರಿಕೆಯಾಗುತ್ತಿದೆ–

\textit{‘ಯತೋ ವಾಚೋ ನಿವರ್ತಂತೇ ಅಪ್ರಾಪ್ಯ ಮನಸಾ ಸಹ}’–ಎಂದರೆ, ಪರಮಾತ್ಮನನ್ನು ತಿಳಿಯಲೆಂದು ಹೊರಟ ಮಾತುಗಳೂ ಮನಸ್ಸೂ ಅವನನ್ನು ಅರ್ಥಮಾಡಿಕೊಳ್ಳಲಾಗದೆ ಹಿಂದಿರು ಗಿದುವಂತೆ! ಮಾತಿನ ಮೂಲಕ ಏನನ್ನು ಬೇಕಾದರೂ ವರ್ಣಿಸಿ ವಿವರಿಸಬಹುದು, ಆದರೆ ಭಗವಂತನನ್ನು ಮಾತ್ರ ಯಥಾವತ್ತಾಗಿ ವರ್ಣಿಸಿ ವಿವರಿಸುವುದು ಸಾಧ್ಯವಿಲ್ಲ. ನಮ್ಮ ಮನಸ್ಸಿ ನಿಂದ ಏನನ್ನು ಬೇಕಾದರೂ ಗ್ರಹಿಸಿ ಅರ್ಥಮಾಡಿಕೊಳ್ಳಬಹುದು, ಆದರೆ ಅನಂತನಾದ ಭಗವಂತ ನನ್ನು ಮಾತ್ರ ಯಥಾವತ್ತಾಗಿ ಗ್ರಹಿಸಿ ಅರಿತುಕೊಳ್ಳುವುದು ಸಾಧ್ಯವಿಲ್ಲ. ಈ ಮಾತುಗಳ ಸತ್ಯತೆ ಅರ್ಥವಾದಂತೆಲ್ಲ ಅವನು ಹೆಚ್ಚೆಚ್ಚು ಧ್ಯಾನನಿರತನಾದ. ಶಾಸ್ತ್ರಗ್ರಂಥಗಳನ್ನು ಮನವಿಟ್ಟು ಓದ ಲಾರಂಭಿಸಿದ. ಧ್ಯಾನ-ಅಧ್ಯಯನ-ಭಜನೆ ಇವುಗಳಲ್ಲೇ ಮಗ್ನನಾಗಿಬಿಟ್ಟ.

ಶ್ರೀರಾಮಕೃಷ್ಣರ ಸಲಹೆಯ ಪ್ರಕಾರ ಅವನೀಗ ಹೊಸ ಬಗೆಯ ಧ್ಯಾನಕ್ರಮವನ್ನು ರೂಢಿಸಿ ಕೊಂಡ. ಈ ಹಿಂದೆ ಅವನು ಧ್ಯಾನ ಮಾಡುತ್ತಿದ್ದುದು ಬ್ರಾಹ್ಮಸಮಾಜದ ನಂಬಿಕೆಯ ಪ್ರಕಾರ ಸಗುಣ-ನಿರಾಕಾರ ಬ್ರಹ್ಮದ ಮೇಲೆ. ಈಗ ಅವನು, ‘ಹೇ ಭಗವನ್, ಕೃಪೆಮಾಡಿ ನನಗೆ ನಿನ್ನ ನಿಜಸ್ವರೂಪವನ್ನು ತೋರಿಸು’ ಎಂದು ತನ್ನ ಹೃದಯಾಂತರಾಳದಿಂದ ಪ್ರಾರ್ಥಿಸಿಕೊಳ್ಳುತ್ತಿದ್ದ. ಹೀಗೆ ಪ್ರಾರ್ಥಿಸಿಕೊಂಡು ಗಾಢ ಧ್ಯಾನದಲ್ಲಿ ಲೀನವಾಗಿಬಿಡುತ್ತಿದ್ದ. ಈಗ ಅವನಿಗೆ ತನ್ನ ಶರೀರಪ್ರಜ್ಞೆಯೂ ಇಲ್ಲ, ಸಮಯಪ್ರಜ್ಞೆಯೂ ಇಲ್ಲ! ಪ್ರತಿ ರಾತ್ರಿಯೂ ಉಳಿದವರೆಲ್ಲ ನಿದ್ರಿಸು ತ್ತಿರುವಾಗ ಅವನು ಈ ರೀತಿ ಧ್ಯಾನಮಗ್ನನಾಗಿ ಕುಳಿತಿರುತ್ತಿದ್ದ. ಆಗ ಅವನ ಅಂತರಂಗ ಒಂದು ಅಪೂರ್ವ ಶಾಂತಿಯನ್ನು ಅನುಭವಿಸುತ್ತಿತ್ತು. ಬಳಿಕ ಅವನನ್ನು ಧ್ಯಾನದ ಅಮಲು ಆವರಿಸಿ ಕೊಂಡುಬಿಡುತ್ತಿತ್ತು. ಆ ಅನುಭವದಿಂದ ಮನಸ್ಸನ್ನು ಹಿಂದೆಳೆದುಕೊಂಡು ಆಸನವನ್ನು ಬಿಟ್ಟು ಎದ್ದೇಳುವ ಮನಸ್ಸಾಗುತ್ತಲೇ ಇರಲಿಲ್ಲ. ಒಂದು ದಿನ ಹೀಗೆ ಧ್ಯಾನವನ್ನು ಮುಗಿಸಿ ಕುಳಿತಿದ್ದಾಗ, ಅವನಿಗೆ ಭಗವಾನ್ ಬುದ್ಧನ ದರ್ಶನಭಾಗ್ಯವಾಯಿತು. ಆ ಅನುಭವವನ್ನು ತನ್ನ ಗುರುಭಾಯಿ ಶರಚ್ಚಂದ್ರನ ಹತ್ತಿರ ಹೇಳಿಕೊಳ್ಳುತ್ತಾನೆ:

“... ಒಂದು ದಿನ ಹೀಗೆಯೇ ಧ್ಯಾನ ಮುಗಿಸಿ ಅದೇ ಆನಂದದ ಸ್ಥಿತಿಯಲ್ಲಿ ಕುಳಿತಿದ್ದಾಗ, ಇದ್ದಕ್ಕಿದ್ದಂತೆ ಒಂದು ಅದ್ಭುತವಾದ ಸಂನ್ಯಾಸಿಯ ಆಕೃತಿ ಪ್ರತ್ಯಕ್ಷವಾದುದನ್ನು ಕಂಡೆ. ಆತ ಎಲ್ಲಿಂದ ಬಂದನೋ ತಿಳಿಯಲಿಲ್ಲ. ಅವನು ನನ್ನ ಮುಂದೆ ಸ್ವಲ್ಪ ದೂರದಲ್ಲಿ ನಿಂತಿದ್ದ. ಕೋಣೆಯಲ್ಲೆಲ್ಲ ಒಂದು ದಿವ್ಯ ಜ್ಯೋತಿಯ ಪ್ರಕಾಶ ತುಂಬಿಕೊಂಡಿತು. ಅವನು ಕಾಷಾಯವಸ್ತ್ರ ಧರಿಸಿದ್ದ; ಕೈಯಲ್ಲಿ ಕಮಂಡಲು ಇತ್ತು. ಮುಖ ಪರಮಪ್ರಶಾಂತವಾಗಿತ್ತು. ಅವನ ಮನಸ್ಸು ಸಂಪೂರ್ಣ ಅಂತರ್ಮುಖವಾಗಿರುವುದನ್ನು ಆ ಮುಖಮುದ್ರೆ ಸೂಚಿಸುತ್ತಿತ್ತು. ಆ ಪ್ರಶಾಂತತೆ ಯನ್ನು ಕಂಡು ನಾನು ಬೆರಗಾಗಿ ಅವನೆಡೆಗೆ ಆಕರ್ಷಿತನಾದೆ. ಅವನು ಮೆಲ್ಲನೆ ಹೆಜ್ಜೆಹಾಕುತ್ತ ನನ್ನೆಡೆಗೆ ನಡೆದು ಬಂದ. ಅವನ ದೃಷ್ಟಿ ನನ್ನ ಮೇಲೆಯೇ ನೆಟ್ಟಿತ್ತು. ನನಗೆ ಅವನು ಏನನ್ನೋ ಹೇಳಹೊರಟಂತಿತ್ತು, ಆ ಮುಖಭಾವ. ಆದರೆ ಅದೇಕೋ ನನಗೆ ತುಂಬ ಹೆದರಿಕೆಯಾಯಿತು. ಅಲ್ಲಿನ್ನು ಕುಳಿತಿರಲಾಗದೆ ಆಸನದಿಂದೆದ್ದು ಸರಸರನೆ ಕೋಣೆಯ ಬಾಗಿಲನ್ನು ತೆರೆದುಕೊಂಡು ಹೊರಗೆ ಬಂದುಬಿಟ್ಟೆ. ಆದರೆ ಮರುಕ್ಷಣವೇ, ‘ಛೆ, ಇದೆಂತಹ ಹುಚ್ಚು ಹೆದರಿಕೆ!’ ಎನ್ನಿಸಿತು. ಧೈರ್ಯ ತಂದುಕೊಂಡು ಮತ್ತೆ ಕೋಣೆಗೆ ಹಿಂದಿರುಗಿದೆ–ನೋಡೋಣ, ಆ ಸಂನ್ಯಾಸಿ ಏನು ಹೇಳುತ್ತಾನೋ ಅಂತ. ಆದರೆ, ಅಯ್ಯೋ, ಅವನು ಅಲ್ಲಿರಲಿಲ್ಲ! ಅವನು ಮತ್ತೆ ಬರಬಹುದೆಂದು ಬಹಳ ಹೊತ್ತು ಕಾದೆ. ಕಾದದ್ದು ವ್ಯರ್ಥವಾಯಿತು, ಅವನು ಕಾಣಿಸಿಕೊಳ್ಳಲೇ ಇಲ್ಲ. ‘ಆ ಸಂನ್ಯಾಸಿ ಹೇಳುವುದನ್ನು ಕೇಳಲೂ ನಿಲ್ಲದೆ ಓಡಿಹೋಗಿಬಿಟ್ಟೆನಲ್ಲ, ಎಂಥ ಮೂರ್ಖ ನಾನು!’ ಎಂದು ತುಂಬ ಖೇದವಾಯಿತು. ನನ್ನ ಮೇಲೆಯೇ ನನಗೆ ಬೇಸರವಾಯಿತು. ನಾನು ಎಷ್ಟೋ ಜನ ಸಂನ್ಯಾಸಿಗಳನ್ನು ನೋಡಿದ್ದೇನೆ, ಆದರೆ ಆತನ ಮುಖದಲ್ಲಿ ವ್ಯಕ್ತವಾದಂತಹ ಭಾವಪ್ರಕಾಶವನ್ನು ಮಾತ್ರ ಇನ್ನಾರಲ್ಲೂ ಕಂಡಿಲ್ಲ. ಆ ಮುಖ ನನ್ನ ಹೃದಯದಲ್ಲಿ ಅಚ್ಚಳಿಯದ ಮುದ್ರೆಯನ್ನೊತ್ತಿ ಬಿಟ್ಟಿತ್ತು. ಇದೆಲ್ಲ ನನ್ನ ಭ್ರಮೆಯಾಗಿರಲೂಬಹುದೇನೋ. ಆದರೆ ಎಷ್ಟೋ ಸಲ ನನಗೆ ಅನಿಸುತ್ತದೆ–ನಾನು ಆ ದಿನ ಪಡೆದದ್ದು ಸಾಕ್ಷಾತ್ ಬುದ್ಧಭಗವಂತನ ದರ್ಶನವನ್ನೇ ಎಂದು.”

ಶ್ರೀರಾಮಕೃಷ್ಣರ ದಿವ್ಯ ಸಂಪರ್ಕದಲ್ಲಿ ನರೇಂದ್ರನ ಆಧ್ಯಾತ್ಮಿಕ ಸಾಧನೆ ರಭಸದಿಂದ ಸಾಗತೊಡಗಿತ್ತು. ಅವನಿಗೆ ಹೊಸಹೊಸ ಆಧ್ಯಾತ್ಮಿಕ ಅನುಭವಗಳಾಗುತ್ತಿದ್ದುವು. ಆದರೆ ಈ ಆಧ್ಯಾತ್ಮಿಕ ಅನುಭವಗಳಿಗೂ ಭಗವತ್ಸಾಕ್ಷಾತ್ಕಾರಕ್ಕೂ ಬಹಳ ವ್ಯತ್ಯಾಸವಿದೆ. ಆಧ್ಯಾತ್ಮಿಕ ಅನುಭವ ವೆಂಬುದು ಸಾಧಕನು ಸಾಧನೆಯ ಸರಿಯಾದ ಮಾರ್ಗದಲ್ಲಿದ್ದಾನೆ ಎನ್ನುವುದನ್ನು ಸೂಚಿಸುವುದೇ ಹೊರತು ಭಗವತ್ಸಾಕ್ಷಾತ್ಕಾರವನ್ನೇ ಅಲ್ಲ. ನಿಜವಾದ ಸಾಕ್ಷಾತ್ಕಾರವಾದ ಮೇಲೆ ಇನ್ನು ಅಲ್ಲಿ ಸಂಶಯ ಹೋರಾಟಗಳಿರಬಾರದು. ಅಲ್ಲಿ ದುಃಖದ ಕಿಲುಬೇ ಇರಬಾರದು. ಅಲ್ಲದೆ ಶ್ರೀರಾಮ ಕೃಷ್ಣರು ಹೇಳುವಂತೆ, ಭಗವಂತ ನಮ್ಮೊಡನೆ ಮಾತನಾಡುವಂತಾಗಬೇಕು. ನಮ್ಮ ಮಾತಿಗೆ ಅವನ ಮರುದನಿ ಸಿಗುವಂತಿರಬೇಕು. ನರೇಂದ್ರನಿಗೆ ಈ ಬಗೆಯ ಸ್ಪಷ್ಟ ಸಾಕ್ಷಾತ್ಕಾರವನ್ನು ಮಾಡಿಕೊಳ್ಳಬೇಕೆಂಬ ಹಂಬಲ ಅಧಿಕವಾಗುತ್ತಿತ್ತು. ಅದನ್ನು ಪಡೆದೇ ತೀರಬೇಕು ಎಂದು ಅವನು ದೃಢಸಂಕಲ್ಪ ಮಾಡಿಬಿಟ್ಟ. ಸಾಮಾನ್ಯ ಜನಕೋಟಿಯ ಹಂಬಲಿಕೆಯೆಲ್ಲ ಇಂದ್ರಿಯ ಭೋಗಗಳ ಕಡೆಗೆ ಹರಿಯುತ್ತಿದ್ದರೆ, ನರೇಂದ್ರನಲ್ಲಿ ಅದು ಜೀವನದ ರಹಸ್ಯವನ್ನು ಭೇದಿಸುವುದರ ಕಡೆಗೆ ಹರಿಯುತ್ತಿತ್ತು. ಅವನ ಹೃದಯದ ಏಕೈಕ ಹಂಬಲವೆಂದರೆ ಸಾಕ್ಷಾತ್ಕಾರ. ಇಂತಹ ಪ್ರಾಮಾಣಿಕ ವ್ಯಾಕುಲತೆಯಿರುವಲ್ಲಿ ಸಾಕ್ಷಾತ್ಕಾರ ಸಿದ್ಧಿಸಲೇಬೇಕು! ಪ್ರಾಪಂಚಿಕವಾದ ಎಲ್ಲ ಜ್ಞಾನ-ಅನುಭವ ಗಳನ್ನೂ ವಿಶ್ಲೇಷಿಸಿ ನೋಡುತ್ತ, ಕೊನೆಗೆ ಅವನೊಂದು ತೀರ್ಮಾನಕ್ಕೆ ಬಂದ: ಈ ಪ್ರಪಂಚಾನು ಭವವೇ ಆಗಲಿ, ಈ ಪ್ರಾಪಂಚಿಕ ಜ್ಞಾನವೇ ಆಗಲಿ–ಇವೆಲ್ಲ ತಿರುಳಿಲ್ಲದವು, ಬರೀ ಟೊಳ್ಳು ಮತ್ತು ಆತ್ಮಸಾಕ್ಷಾತ್ಕಾರಕ್ಕೆ ಬಾಧಕವಾದವುಗಳು’ ಎಂದು. ಮುಂಡಕೋಪನಿಷತ್ತಿನ ಒಂದು ಮಾತು ಹೀಗಿದೆ–

\textit{‘ಪರೀಕ್ಷ್ಯ ಲೋಕಾನ್ ಕರ್ಮಚಿತಾನ್ ಬ್ರಾಹ್ಮಣೋ ನಿರ್ವೇದಮಾಯಾತ್’.} ಎಂದರೆ, ‘ಬ್ರಾಹ್ಮಣನಾದವನು, ಅರ್ಥಾತ್ ವಿವೇಕಿಯಾದವನು, ಲೋಕಲೋಕಾಂತರಗಳನ್ನೆಲ್ಲ ವಿವೇಕ ದಿಂದ ಪರೀಕ್ಷೆಮಾಡಿ ನೋಡಿ, ಕೊನೆಗೆ ಅವುಗಳ ವಿಷಯದಲ್ಲಿ ವೈರಾಗ್ಯ ತಾಳಬೇಕು, ನಿರಾಸಕ್ತ ನಾಗಬೇಕು’ ಎಂದು. ಬುದ್ಧಿಯನ್ನುಪಯೋಗಿಸಿ ಪರೀಕ್ಷೆ ಮಾಡಿ ನೋಡಿದಾಗ, ಈ ಲೋಕದ ಸುಖಗಳೇ ಆಗಲಿ ಇತರ ಲೋಕಗಳ ಸುಖಗಳೇ ಆಗಲಿ ಕೇವಲ ಕ್ಷಣಿಕ, ಅಷ್ಟೇ ಅಲ್ಲ, ಅವುಗಳನ್ನು ಅನುಭವಿಸಿ ಮುಗಿಸಿದ ಕೂಡಲೇ ಅಪಾರ ದುಃಖ, ಆದ್ದರಿಂದ ಲೌಕಿಕ ಅನುಭವಗಳಾವುವೂ ನಿತ್ಯಶಾಂತಿ ಹಾಗೂ ನಿತ್ಯಾನಂದವನ್ನು ಕೊಡಲಾರವು ಎಂಬುದು ತಿಳಿದುಬರುತ್ತದೆ. ಆದ್ದರಿಂದ ವಿವೇಕಿಯಾದವನು ಈ ಪ್ರಾಪಂಚಿಕ ಸುಖದ ಬೆನ್ನುಹತ್ತದೆ ನಿತ್ಯಾನಂದಪ್ರಾಪ್ತಿಯ ಕಡೆಗೆ ಮನಗೊಡಬೇಕು. ಇಲ್ಲಿ ನರೇಂದ್ರನೂ ಚೆನ್ನಾಗಿ ವಿಮರ್ಶಿಸಿನೋಡಿ, ‘ಇವೆಲ್ಲ ತಿರುಳಿಲ್ಲದ್ದು, ಆತ್ಮಸಾಕ್ಷಾತ್ಕಾರಕ್ಕೆ ಬಾಧಕವಾದದ್ದು’ ಎಂಬ ಸತ್ಯವನ್ನು ಮನಗಂಡ. ಹಾಗಯೇ ಯೋಚನೆ ಮಾಡುತ್ತಮಾಡುತ್ತ, ಸಮಸ್ತ ಸೃಷ್ಟಿಯೂ ಯಾವುದೋ ಒಂದು ಅನಿರ್ವಚನೀಯ ಸತ್ಯವೊಂದ ರಿಂದ ಉದ್ಭವಿಸಿದೆ ಎಂಬುದರಲ್ಲಿ ನಂಬಿಕೆ ಬೆಳೆಯಿತು. 

ತೈತ್ತಿರೀಯ ಉಪನಿಷತ್ತು ಹೇಳುತ್ತದೆ–

\begin{myquote}
ಯತೋ ವಾ ಇಮಾನಿ ಭೂತಾನಿ ಜಾಯಂತೇ । ಯೇನ ಜಾತಾನಿ ಜೀವಂತಿ ।\\ಯತ್ಪ್ರಯಂತ್ಯಭಿಸಂವಿಶಂತಿ । ತದ್ವಿಜಿಜ್ಞಾಸಸ್ವ । ತದ್ ಬ್ರಹ್ಮೇತಿ ।
\end{myquote}

ಭೃಗು ತನ್ನ ತಂದೆಯಾದ ವರುಣನನ್ನು ಕೇಳುತ್ತಾನೆ: ‘ಹೇ ಭಗವನ್, ನನಗೆ ಪರಬ್ರಹ್ಮದ ಕುರಿತಾಗಿ ತಿಳಿಸಿಕೊಡು’ ಎಂದು. ಆಗ ವರುಣನು ಮೇಲಿನಂತೆ ಉತ್ತರಿಸುತ್ತಾನೆ: ‘ಯಾವುದ ರಿಂದ ಈ ಜೀವಿಗಳೆಲ್ಲ ಹುಟ್ಟುವವೋ, ಹುಟ್ಟಿದ ಜೀವಿಗಳೆಲ್ಲ ಯಾವುದರಿಂದ ಜೀವಿಸು ತ್ತವೆಯೋ ಮತ್ತು ಕೊನೆಯಲ್ಲಿ ಯಾವುದನ್ನು ಕುರಿತು ಹೋಗಿ ಸೇರಿಕೊಳ್ಳುತ್ತವೆಯೋ ಅದನ್ನು ಅರಿತುಕೊ; ಅದೇ ಬ್ರಹ್ಮ.’ ಇಂತಹ ಪರಬ್ರಹ್ಮದ ಅಸ್ತಿತ್ವದಲ್ಲಿ ನರೇಂದ್ರನಿಗೆ ವಿಶ್ವಾಸ ಹೆಚ್ಚಿತು.

ಹೀಗೆ ಅವನು ತನ್ನ ಬುದ್ಧಿಯಿಂದ, ವಿವೇಕದಿಂದ ತಿಳಿಯಬಹುದಾದಷ್ಟನ್ನೂ ತಿಳಿದ. ಆದರೆ ಪರಮಾತ್ಮ ಅವಾಙ್ಮಾನಸಗೋಚರ. ಆದ್ದರಿಂದ ಅವನನ್ನು ತಿಳಿಯಲು ಬೇಕಾದದ್ದು ಬುದ್ಧಿಶಕ್ತಿ ಯಲ್ಲ, ಆತ್ಮಶಕ್ತಿ. ಮನೋಬುದ್ಧಿಗಳನ್ನು ಆತ್ಮದಲ್ಲಿ ಲೀನಗೊಳಿಸಿದಾಗ ನಿಜವಾದ ಆಧ್ಯಾತ್ಮಿಕ ಜೀವನ ಆರಂಭವಾಗುತ್ತದೆ. ನರೇಂದ್ರ ತನ್ನ ದಿವ್ಯ ಗುರು ಶ್ರೀರಾಮಕೃಷ್ಣರ ಮಾರ್ಗದರ್ಶನದಲ್ಲಿ ಮೆಲ್ಲಮೆಲ್ಲನೆ ಈ ಆಧ್ಯಾತ್ಮಿಕ ಜೀವನವನ್ನು ಬಲಪಡಿಸಿಕೊಂಡ. ನಾಸ್ತಿಕ ಮನೋಭಾವದಿಂದ ಆಸ್ತಿಕ ಮನೋಭಾವಕ್ಕೆ ಏರುವ ಕಾರ್ಯ ಸ್ವಲ್ಪ ಸಾವಕಾಶವೇ. ಪುನಃ ಈ ಆಸ್ತಿಕ ಮನೋಭಾವ ದಿಂದ, ಪ್ರಾಮಾಣಿಕ ಪ್ರಾರ್ಥನೆ ಹಾಗೂ ಧ್ಯಾನದ ಸ್ಥಿತಿಗೆ ಏರುವುದಕ್ಕೂ ಕಾಲಾವಕಾಶ ಬೇಕಾಗು ತ್ತದೆ. ತನ್ನ ಇಂದ್ರಿಯಗಳನ್ನು ಶಾಂತಗೊಳಿಸುವುದರ ಮೂಲಕ, ಇಂದ್ರಿಯ ಭೋಗಗಳನ್ನು ತ್ಯಾಗ ಮಾಡುವುದರ ಮೂಲಕ ಹೃತ್ಪೂರ್ವಕ ಪ್ರಾರ್ಥನೆ ಮಾಡಲು, ಗಾಢತಮ ಧ್ಯಾನಲೀನನಾಗಲು ಸಮರ್ಥನಾದ. ಧ್ಯಾನ ಗಾಢವಾದಂತೆಲ್ಲ ಅವನ ಭಗವದ್ದರ್ಶನದ ಹಂಬಲ ಇನ್ನಷ್ಟು ತೀವ್ರ ವಾಯಿತು. ಏಕೆಂದರೆ, ಭಗವಂತನ ನೇರ ದರ್ಶನದಿಂದ ಆಗುವ ಆನಂದ, ಧ್ಯಾನಾನಂದಕ್ಕಿಂತಲೂ ಎಷ್ಟೋ ಪಾಲು ಮಿಗಿಲು.

ಅವನು ಭಗವಂತನ ಸ್ವರೂಪದ ಕುರಿತಂತೆ ಗಾಢವಾಗಿ ಆಲೋಚನೆ ನಡೆಸಿದ. ಹಗಲಿರುಳೂ ಕೇವಲ ಆ ಸಚ್ಚಿದಾನಂದನ ಕುರಿತಾಗಿಯೇ ಚಿಂತನೆ, ಆಲೋಚನೆ. ಧ್ಯಾನವೆಂಬುದು ಅವನಿಗೆ ಸ್ವಭಾವಗತವಾಯಿತು. ಸತ್ಯದರ್ಶನ ಮಾಡಬೇಕೆಂಬ ತೀವ್ರ ಕಾತರತೆಯು ಅವನ ಬುದ್ಧಿ ನಿರ್ಮಿಸಿದ್ದ ಆಲೋಚನೆಗಳ, ಕಲ್ಪನೆಗಳ ಗೋಡೆಗಳನ್ನು ನುಚ್ಚುನೂರು ಮಾಡಿತು. ಇದರಿಂದಾಗಿ ಅವನ ಅಂತರ್ದೃಷ್ಟಿ ಬೆಳೆಯಲು ಅವಕಾಶವಾಯಿತು. ನಿದ್ರಾಮಗ್ನನಾಗಿರುವಾಗ ಇಂದ್ರಿಯ ಲೋಕದ ಪರಿಧಿಯಾಚೆಗಿನ ಅನೇಕ ದೃಶ್ಯಗಳನ್ನು ಅಸ್ಪಷ್ಟವಾಗಿ ಕಾಣುತ್ತಿದ್ದ. ಕೆಲವೊಮ್ಮೆ ಬೆಳಗ್ಗೆ ಏಳುವಾಗ ಅನಿರ್ವಚನೀಯವಾದ ಆನಂದದಿಂದ ತುಂಬಿರುತ್ತಿದ್ದ. ಅವನ ಆ ಅನುಭವಗಳು ಸಾಧಾರಣ ನಿದ್ರೆ-ಸ್ವಪ್ನಗಳಂತಲ್ಲ. ಪರಮಸತ್ಯದ ಮಿಂಚುನೋಟಗಳಾದ ಈ ಬಗೆಯ ಆನಂದಾ ನುಭವಗಳು ಅವನಿಗೆ ಪ್ರತಿದಿನವೂ ಆಗುತ್ತಿದ್ದುವು. ಅಲ್ಲದೆ, ತಾನು ಈ ದೇಹವಲ್ಲ, ತಾನು ಈ ದೇಹದಿಂದ ಬೇರೆ ಎಂಬ ಭಾವನೆ ಅವನಲ್ಲಿ ಮತ್ತೆಮತ್ತೆ ಮೂಡುತ್ತಿತ್ತು.

ಆದರೆ, ಶ್ರೀರಾಮಕೃಷ್ಣರು ಅವನಲ್ಲಿ ಸುಪ್ತವಾಗಿದ್ದ ಆಧ್ಯಾತ್ಮಿಕತೆಯನ್ನು ಜಾಗೃತಗೊಳಿಸಲು ಪ್ರಯತ್ನಪಡುತ್ತಿದ್ದರೆ, ಅವನು ಮಾತ್ರ ತನ್ನ ಬುದ್ಧಿಶಕ್ತಿಯಿಂದ ಅದನ್ನು ನಿರೋಧಿಸುತ್ತಿದ್ದ. ಅವರ ಅನುಭವದ ಮಾತುಗಳನ್ನೆಲ್ಲ ಅಲ್ಲಗಳೆದು ಟೀಕಿಸುತ್ತಿದ್ದ. ಆದರೆ ಶ್ರೀರಾಮಕೃಷ್ಣರೇ ಹೇಳುತ್ತಿದ್ದಂತೆ ಅವರೇನೂ ಸಾಮಾನ್ಯವಾದ ನೀರುಹಾವಲ್ಲ, ಮಹಾ ಘಟಸರ್ಪ! ಅವರು ಹೇಳುವಂತೆ, ನೀರುಹಾವು ಕಪ್ಪೆಯನ್ನು ಹಿಡಿದರೆ ಅದನ್ನು ಪೂರ್ಣ ನುಂಗಲೂ ಆರದೆ ಬಿಡಲೂ ಆರದೆ ಒದ್ದಾಡುತ್ತಿರುತ್ತದೆ. ಆದರೆ ನಾಗರಹಾವೇನಾದರೂ ಕಪ್ಪೆಯನ್ನು ಹಿಡಿದರೆ, ಅದು ಮೂರು ಸಲ ಕೂಗುವುದರೊಳಗಾಗಿ ಅದನ್ನು ಮುಗಿಸಿಬಿಡುತ್ತದೆ. ಹಾಗೆಯೇ ಸಾಮಾನ್ಯ ಗುರುಗಳು ತಮ್ಮ ಶಿಷ್ಯರನ್ನು ಉದ್ಧಾರ ಮಾಡಲೂ ಆರದೆ, ಬಿಡಲೂ ಆರದೆ ಒದ್ದಾಡುತ್ತಿರುತ್ತಾರೆ–ನೀರುಹಾವು- ಕಪ್ಪೆಗಳಂತೆ ಗುರುವಿಗೂ ಸಂಕಟ, ಶಿಷ್ಯನಿಗೂ ಸಂಕಟ. ಆದರೆ ಸಮರ್ಥ ಗುರುವಾದವನು ನಾಗರಹಾವಿನ ಹಾಗೆ. ಶಿಷ್ಯ ‘ಹಾಂ ಹೂಂ’ ಎನ್ನುವುದರೊಳಗಾಗಿ ಅವನನ್ನು ಸಮಾಧಾನಸ್ಥಿತಿಗೆ ತಂದುಬಿಡುತ್ತಾನೆ. ಇಂತಹ ಸಮರ್ಥ ಗುರುವಿನಲ್ಲಿ ನರೇಂದ್ರ ಕ್ರಮೇಣ ಸಂಪೂರ್ಣ ಶರಣಾ ಗುತ್ತ ಬಂದ; ಅವನ ಪ್ರತಿಭಟನೆ ಕುಂದಿತು.

ಆದರೆ ನರೇಂದ್ರ ಹೀಗೆ ಪರಿವರ್ತನೆಗೊಂಡ ಮತ್ತು ಬೆಳಕನ್ನು ಕಂಡ ಕಥೆ ಅವರ್ಣನೀಯ, ಅತಿ ಸೂಕ್ಷ್ಮ. ಶ್ರೀರಾಮಕೃಷ್ಣರು ಇವುಗಳನ್ನೆಲ್ಲ ಮಾತುಗಳಿಂದ ವಿವರಿಸಲಾಗದ ಒಂದು ಅಲೌಕಿಕ ವಿಧಾನದಲ್ಲಿ ಸಾಧಿಸಿದರು. ನರೇಂದ್ರನ ಹೊರ ನೋಟದ ಚಡಪಡಿಕೆಯನ್ನು ಮತ್ತು ಅವನ ಬೌದ್ಧಿಕ ಹೋರಾಟವನ್ನು ಮಾತ್ರ ಅವನ ಕೆಲವು ಸ್ನೇಹಿತರು ಸ್ವಲ್ಪಮಟ್ಟಿಗೆ ಗಮನಿಸಲು ಸಮರ್ಥರಾಗಿದ್ದರು. ಆದರೆ ಈ ಅದ್ಭುತ ಮಾರ್ಪಾಡಿನ ಹಿಂದಿರುವ ವಿವರಗಳನ್ನು ಶ್ರೀರಾಮ ಕೃಷ್ಣರು ಮತ್ತು ಪ್ರಾಯಶಃ ನರೇಂದ್ರ ಮಾತ್ರ ಬಲ್ಲರು.

