
\chapter{ಬಹೂದಕ}

\noindent

ಹೀಗೆ ಈ ಯುವಸಂನ್ಯಾಸಿಗಳು ಪರಿಪೂರ್ಣ ಆಧ್ಯಾತ್ಮಿಕ ಜೀವನವನ್ನು ನಡೆಸುತ್ತ ಬಂದರು. ತನ್ಮೂಲಕ ಶ್ರೀರಾಮಕೃಷ್ಣ ಮಹಾಸಂಘವೂ ಈಗ ಒಂದು ಆಕಾರ ತಳೆಯಿತು. ಈ ಸಂಘದ ಧ್ಯೇಯೋದ್ದೇಶಗಳಿಗೆ ಸಂಬಂಧಿಸಿದಂತೆ ಎರಡು ಅಂಶಗಳನ್ನು ಗಮನಿಸಬಹುದು. ಮೊದಲ ನೆಯದು: ಸಂನ್ಯಾಸಿಗಳೆಲ್ಲ ಮಠದಲ್ಲಿ ಒಂದೇ ಕಡೆ ಒಮ್ಮತದಿಂದಿದ್ದುಕೊಂಡು, ಶ್ರೀರಾಮ ಕೃಷ್ಣರ ಸಂದೇಶಗಳನ್ನು ಪ್ರಸಾರ ಮಾಡುವ ಆದರ್ಶ; ಎರಡನೆಯದು: ಸಂನ್ಯಾಸಿಗಳಿಗೆ ಸಹಜ ವಾದ ಸ್ವತಂತ್ರಮನೋಭಾವದಿಂದ ಪರಿವ್ರಾಜಕರಾಗಿ ತೀರ್ಥಾಟನೆ ಮಾಡುವ ಅಂಶ. ಹೀಗೆ, ಪರಸ್ಪರ ಕಾರ್ಯತಃ ವ್ಯತಿರಿಕ್ತವಾದ ಈ ಎರಡೂ ಅಂಶಗಳು ಪ್ರತಿಯೊಬ್ಬ ಸದಸ್ಯನಲ್ಲೂ ಇದ್ದೇ ಇದ್ದುವೆನ್ನಬಹುದು. ಅದರಲ್ಲೂ ಸ್ವಾಮಿ ವಿವೇಕಾನಂದರಲ್ಲಂತೂ (ಇನ್ನುಮುಂದೆ ಅವರನ್ನು ರಾಮಕೃಷ್ಣರ ಸಂಘದಲ್ಲಿ ಹಾಗೂ ಭಕ್ತವೃಂದದಲ್ಲಿ ರೂಢಿಯಲ್ಲಿರುವಂತೆ ‘ಸ್ವಾಮೀಜಿ’ ಎಂದು ಸಂಬೋಧಿಸೋಣ) ಈ ಎರಡೂ ಮನೋಭಾವಗಳೂ ಬಲವತ್ತರವಾಗಿ ಕೆಲಸ ಮಾಡುತ್ತಿದ್ದುವು. ಕೆಲವು ಸಲವಂತೂ ಅವರೊಳಗೆ ಒಂದು ಹೋರಾಟವೇ ಆರಂಭವಾಗಿಬಿಡುತ್ತಿತ್ತು–ಮಠದಲ್ಲೇ ನೆಲೆ ನಿಲ್ಲಲೆ, ಅಥವಾ ಪರಿಪ್ರಾಜಕನಾಗಿ ಹೊರಟುಬಿಡಲೆ?–ಎಂದು. ಪರಿವ್ರಾಜಕರಾಗಿ ತೀರ್ಥಾಟನೆ-ದೇಶಸಂಚಾರ ಮಾಡುವುದು ಸಂನ್ಯಾಸಿಗಳಲ್ಲಿ ಬೆಳೆದುಬಂದಿರುವ ಸಂಪ್ರದಾಯ. ಆದ್ದರಿಂದ ಅವರಿಗೆ ಸಹಜವಾದ ಪರಿವ್ರಾಜಕಮನೋಭಾವವು ಈ ಯುವಸಂನ್ಯಾಸಿಗಳನ್ನು ಮಠ ತ್ಯಜಿಸಿ ದೂರ ಹೋಗುವಂತೆ, ಏಕಾಂತದಲ್ಲಿದ್ದುಕೊಂಡು ಸಾಧನೆ ಮಾಡುವಂತೆ ಪ್ರೇರೇಪಿಸು ತ್ತಿತ್ತು. ಇವರಲ್ಲಿ ಒಬ್ಬಿಬ್ಬರನ್ನು ಬಿಟ್ಟು ಮಿಕ್ಕವರೆಲ್ಲ ಒಂದಲ್ಲ ಒಂದು ವೇಳೆಯಲ್ಲಿ ಪರಿವ್ರಾಜಕ ರಾಗಿ ಹೊರಟು ಹಲವಾರು ವರ್ಷಗಳ ಬಳಿಕ ಮಠಕ್ಕೆ ಹಿಂದಿರುಗಿದರು ಎನ್ನುವುದನ್ನು ನೋಡಲಿ ದ್ದೇವೆ. ಆದರೆ ಒಂದು ಅತಿ ಮುಖ್ಯವಾದ ಅಂಶವೆಂದರೆ, ಮಠದ ಅಸ್ತಿತ್ವಕ್ಕೇ ಧಕ್ಕೆಯಾಗುವಂತಹ ಪರಿಸ್ಥಿತಿ ಎಂದೂ ಬರಲಿಲ್ಲ ಎಂಬುದು. ಯಾವಾಗಲೂ ಸಂನ್ಯಾಸಿಗಳ ಒಂದು ಗುಂಪು ಮಠದಲ್ಲಿ ಇದ್ದೇ ಇರುತ್ತಿತ್ತು. ಇವರು ಶ್ರೀರಾಮಕೃಷ್ಣರ ಪೂಜಾದಿಗಳನ್ನು, ಆಧ್ಯಾತ್ಮಿಕ ಸಾಧನೆಯನ್ನು ನಡೆಸುತ್ತಲೇ ಇದ್ದು ಮಠದಲ್ಲಿ ಆಧ್ಯಾತ್ಮಿಕ ವಾತಾವರಣವನ್ನು ನಿರ್ಮಾಣ ಮಾಡಿದ್ದರು. ಮಠ ಸ್ಥಾಪನೆಯಾದ ಮೊದಲ ನಾಲ್ಕು ವರ್ಷಗಳ ಕಾಲ ಸ್ವಾಮೀಜಿಯವರು ಸಣ್ಣಪುಟ್ಟ ತೀರ್ಥಯಾತ್ರೆ ಗಳಿಗೆ ಹೋಗಿಬರುತ್ತಿದ್ದರು. ಆದರೆ ಮುಂದಿನ ಏಳು ವರ್ಷಗಳ ದೀರ್ಘಕಾಲ ಅವರು ಮಠಕ್ಕೆ ಹಿಂದಿರುಗದೆ ನಿರಂತರ ಪರಿವ್ರಾಜಕರಾಗಿಯೇ ಉಳಿದರು.

ಹಿಂದೂ ಸಂನ್ಯಾಸಿಗಳಿಗೆ ತೀರ್ಥಾಟನೆಯಲ್ಲಿ ತುಂಬ ಪ್ರೇಮ. ಹಿಂದಿಯಲ್ಲಿ ಒಂದು ಮಾತಿದೆ–‘ಬಹ್ತಾ ಪಾನಿ, ಚಲ್ತಾ ಸಾಧು.’ ಹಾಗೆಂದರೆ, ನೀರು ಯಾವಾಗಲೂ ಹರಿಯುತ್ತಿರ ಬೇಕು, ಸಂನ್ಯಾಸಿ ಯಾವಾಗಲೂ ಸಂಚರಿಸುತ್ತಿರಬೇಕು ಎಂದು. ಹೀಗಿದ್ದರೆ ನೀರೂ ನಾರುವು ದಿಲ್ಲ, ಸಂನ್ಯಾಸಿಯೂ ಕೆಡುವುದಿಲ್ಲ. ನೀರು ಒಂದೇ ಕಡೆ ನಿಂತರೆ ಪಾಚಿ ಕಟ್ಟುತ್ತದೆ, ಕೊಳೆತು ನಾರುತ್ತದೆ. ಬಳಿಕ ಸೊಳ್ಳೆ-ಕ್ರಿಮಿಕೀಟಗಳೂ ಹುಟ್ಟಿಕೊಳ್ಳುತ್ತವೆ. ಹಾಗೆಯೇ ಸಂನ್ಯಾಸಿಗಳು ಒಂದೇ ಕಡೆ ಹಲವಾರು ವರ್ಷ ಇರುವುದರಿಂದ ಅವರಿಗೆ ಆ ಸ್ಥಳದ ಮೇಲೆ ಆಸಕ್ತಿ-ಮಮಕಾರ ಹುಟ್ಟ ಬಹುದು; ಆ ಊರಿನ ವ್ಯಕ್ತಿಗಳ ಮೇಲೆ ಆಸಕ್ತಿ ಬೆಳೆಯಬಹುದು; ಕೆಲವು ವ್ಯಕ್ತಿಗಳ ಮೇಲೆ ವಿಶೇಷ ವ್ಯಾವೋಹ ಹುಟ್ಟಿಕೊಳ್ಳಲೂ ಬಹುದು. ಸಂನ್ಯಾಸಿಗಳೆಂದರೆ ಸರ್ವಸಂಗಪರಿತ್ಯಾಗ ಮಾಡಿ ದವರು. ಸಂಗತ್ಯಾಗ ಮಾಡಿದರೆ ಮಾತ್ರ ಆಧ್ಯಾತ್ಮಿಕತೆಯಲ್ಲಿ ಪ್ರಗತಿ. ಈ ಕಾರಣದಿಂದಲೇ ಅವರು ಒಂದು ಸ್ಥಳದಲ್ಲಿ ಬಹಳ ಕಾಲ ನಿಲ್ಲದೆ ಸದಾ ಸಂಚಾರ ಮಾಡುತ್ತಿರಬೇಕು. ಆದ್ದರಿಂದಲೇ ಅವರಿಗೆ ಪರಿವ್ರಾಜಕರು ಎಂಬ ಹೆಸರು. ಅಲ್ಲದೆ ತೀರ್ಥಾಟನೆಯೆಂದರೆ ಯಾರಿಗೆ ಇಷ್ಟವಿಲ್ಲ! ಯಾವ ಕ್ಷೇತ್ರಗಳಲ್ಲಿ ಭಗವಂತ ವಿಶೇಷವಾಗಿ ಆರಾಧಿಸಲ್ಪಡುತ್ತಿದ್ದಾನೆಯೋ, ಯಾವ ಕ್ಷೇತ್ರಗಳು ಅಸಂಖ್ಯಾತ ಸಾಧು-ಸಂತರ ಸಮಾಗಮದಿಂದ ಮತ್ತಷ್ಟು ಪುನೀತವಾಗಿವೆಯೋ ಅಂಥಲ್ಲಿಗೆ ಹೋಗಲು ಯಾರಿಗೆತಾನೆ ಇಷ್ಟವಿಲ್ಲ! ಅದರಲ್ಲೂ ಭಗವಂತನ ಸಾಕ್ಷಾತ್ಕಾರವನ್ನೇ ಜೀವನದ ಏಕೈಕ ಗುರಿಯನ್ನಾಗಿಸಿಕೊಂಡು ಸರ್ವಸಂಗಪರಿತ್ಯಾಗ ಮಾಡಿರುವ ಬಾರಾನಗೋರ್ ಮಠದ ನೂತನ ಸಂನ್ಯಾಸಿಗಳಿಗೂ ಆ ಇಚ್ಛೆ ಉಂಟಾಗಿದೆ.

೧೮೮೭ರ ಫೆಬ್ರವರಿ ಅಂತ್ಯದಲ್ಲಿ ಶಾರದಾನಂದ, ಅಭೇದಾನಂದ ಹಾಗೂ ಪ್ರೇಮಾನಂದ– ಇವರು ಪುರೀಕ್ಷೇತ್ರಕ್ಕೆ ಹೊರಟರು. ಅಖಂಡಾನಂದರು ಟಿಬೆಟಿನ ಕಡೆಗೆ ಹೊರಟರು.

ಮೇ ತಿಂಗಳಲ್ಲೊಂದು ದಿನ ತ್ರಿಗುಣಾತೀತಾನಂದರು ತಾವು ಎಲ್ಲಿಗೋ ಹೋಗುವುದಾಗಿ ಹೇಳಿ ಇದ್ದಕ್ಕಿದ್ದಂತೆ ಕಾಣೆಯಾಗಿಬಿಟ್ಟರು. ತಾವು ಯಾವ ಕಡೆಗೆ ಹೋಗುತ್ತೇವೆ ಎಂಬುದನ್ನು ಯಾರ ಹತ್ತಿರವೂ ಬಾಯಿಬಿಟ್ಟಿರಲಿಲ್ಲ. ಸ್ವಾಮೀಜಿ ಆ ದಿನ ಮಠದಲ್ಲಿರಲಿಲ್ಲ. ಕಾರಣಾಂತರ ದಿಂದ ಕಲ್ಕತ್ತಕ್ಕೆ ಹೋಗಿದ್ದರು. ಅಲ್ಲಿಂದ ಹಿಂದಿರುಗಿ ಬಂದಾಗ ನೋಡುತ್ತಾರೆ–ತ್ರಿಗುಣಾತೀತಾ ನಂದರು ನಾಪತ್ತೆ! ಅವರಿಗೆ ತುಂಬ ಕಳವಳವಾಯಿತು, ಸಿಟ್ಟೂ ಬಂದಿತು. ಏಕೆಂದರೆ ತ್ರಿಗುಣಾ ತೀತಾನಂದರು ವಯಸ್ಸಿನಲ್ಲಿ ಬಹಳ ಚಿಕ್ಕವರು (ವಯಸ್ಸು ಸುಮಾರು ೧೮-೧೯ ವರ್ಷ ಅಷ್ಟೆ); ಲೋಕಾನುಭವ ಸಾಲದು. ಯಾರಾದರೂ ದುಷ್ಟರ ಕೈಗೆ ಸಿಕ್ಕಿಕೊಂಡರೇನು ಗತಿ? ಅಥವಾ ಸಾಕ್ಷಾತ್ಕಾರ ಮಾಡಿಕೊಳ್ಳಬೇಕೆಂಬ ಉತ್ಸಾಹದಲ್ಲಿ, ಅತಿ ಸಾಧನೆ ಮಾಡಿ ಅಪಾಯಕ್ಕೆ ಗುರಿಯಾದ ರೇನು ಗತಿ?–ಎಂದು ಸ್ವಾಮೀಜಿ ತಳಮಳಿಸಿದರು. ಆದರೆ ತ್ರಿಗುಣಾತೀತಾನಂದರು ಹೋಗು ವಾಗ ಒಂದು ಪತ್ರವನ್ನು ಬರೆದಿಟ್ಟು ಹೋಗಿದ್ದರು. ದಕ್ಷಿಣೇಶ್ವರಕ್ಕೆ ಹೋಗಿದ್ದ ಬ್ರಹ್ಮಾನಂದರು ಅಲ್ಲಿಂದ ಹಿಂದಿರುಗಿದಾಗ ಆ ಪತ್ರವನ್ನು ಸ್ವಾಮೀಜಿಯ ಕೈಗೆ ಕೊಟ್ಟರು. ಆ ಪತ್ರದ ಒಕ್ಕಣೆ ಈ ರೀತಿ ಇತ್ತು: “ನಾನು ಮಠದಲ್ಲಿ ಉಳಿದುಕೊಳ್ಳುವುದು ಅಪಾಯಕರ ಎಂದು ತೋರುತ್ತಿದೆ. ಮನೆಯವರಿಂದ ನನಗೆ ತೊಂದರೆಯಿದೆ. ಆದ್ದರಿಂದ ನಾನು ಕಾಲ್ನಡಿಗೆಯಲ್ಲೇ ಬೃಂದಾವನಕ್ಕೆ ಹೋಗುತ್ತಿದ್ದೇನೆ.”

ಈಗ ಬ್ರಹ್ಮಾನಂದರು ಸ್ವಾಮೀಜಿಗೆ ಹೇಳುತ್ತಾರೆ: “ನೋಡಿದೆಯಾ, ಅವನು ಹೊರಟುಹೋದ ದ್ದಕ್ಕೆ ಕಾರಣಗಳು ಇವು. ಅಲ್ಲದೆ ನಾವಾದರೂ ಇಲ್ಲಿ ಉಳಿದುಕೊಂಡು ಸಾಧಿಸಿದ್ದೇನು? ಶ್ರೀ ರಾಮಕೃಷ್ಣರು ಯಾವಾಗಲೂ ನಮ್ಮನ್ನು ಭಗವಂತನ ಸಾಕ್ಷಾತ್ಕಾರ ಮಾಡಿಕೊಳ್ಳುವಂತೆ ಹುರಿ ದುಂಬಿಸುತ್ತಿದ್ದರು. ನಾವು ಅದರಲ್ಲಿ ಜಯಶಾಲಿಗಳಾಗಿದ್ದೇವೇನು? ಆದ್ದರಿಂದ ನಾವೆಲ್ಲರೂ ನರ್ಮದಾತೀರಕ್ಕೆ ಹೋಗಿ ಸಾಧನೆ ಮಾಡೋಣ.”

ಸ್ವಾಮೀಜಿ: “ಆದರೆ ಅಲ್ಲಿ ಇಲ್ಲಿ ಅಲೆದಾಡುವುದರಿಂದ ನೀನು ಹೇಳುತ್ತಿರುವ ಆ ಬ್ರಹ್ಮಜ್ಞಾನ ವನ್ನು ಪಡೆದುಕೊಂಡುಬಿಡಬಹುದೇನು?”

ತಕ್ಷಣ ಅಲ್ಲಿದ್ದ ಒಬ್ಬ ಭಕ್ತ ಕೇಳುತ್ತಾನೆ: “ಹಾಗಾದರೆ ನೀವೆಲ್ಲ ಪ್ರಪಂಚವನ್ನು ತ್ಯಾಗ ಮಾಡಿ ಬಂದದ್ದೇಕೆ?”

ಬ್ರಹ್ಮಜ್ಞಾನವನ್ನು ಪಡೆದುಕೊಳ್ಳಲು ಸುಲಭವಲ್ಲ ಎಂದಮೇಲೆ, ಸಂಸಾರವನ್ನು ತ್ಯಾಗ ಮಾಡುವುದಾದರೂ ಏಕೆ? ಮದುವೆ ಮಾಡಿಕೊಂಡು ಹಾಯಾಗಿರಬಹುದಿತ್ತಲ್ಲ! ಈಗ ನೋಡಿ; ಆ ಕಡೆ ಬ್ರಹ್ಮಾನಂದವೂ ಇಲ್ಲ, ಈ ಕಡೆ ದಾಂಪತ್ಯಸುಖವೂ ಇಲ್ಲ! ಆದ್ದರಿಂದ ಮದುವೆಮಾಡಿ ಕೊಂಡು ಹಾಯಾಗಿರುವುದೇ ಒಳ್ಳೆಯದಲ್ಲವೆ?–ಇದು ಆ ಭಕ್ತನ ಇಂಗಿತ. ಅದಕ್ಕೆ ಸ್ವಾಮೀಜಿ ಹೇಳುತ್ತಾರೆ:

“ರಾಮ ಸಿಕ್ಕಲಿಲ್ಲ ಅಂತ ಶ್ಯಾಮಳನ್ನು ಕಟ್ಟಿಕೊಳ್ಳೋಣವೆ? ಭಗವಂತನ ಸಾಕ್ಷಾತ್ಕಾರವಾಗ ಲಿಲ್ಲ ಅಂದಮಾತ್ರಕ್ಕೆ ಮದುವೆ ಮಾಡಿಕೊಂಡು ಸಂತಾನೋತ್ಪತ್ತಿ ಮಾಡುತ್ತ ಕುಳಿತುಕೊಳ್ಳಬೇಕೆ? ಏನು ಹೇಳುತ್ತೀ ನೀನು?”

ಸ್ವಾಮೀಜಿಯ ಈ ಮಾತಿನಲ್ಲಿ, ಕೈಗೊಂಡ ಆದರ್ಶವನ್ನು ದೃಢಬುದ್ಧಿಯಿಂದ ಪಾಲಿಸಿಕೊಂಡು ಬರುವ ಛಲ ಎದ್ದುಕಾಣುತ್ತದೆ.

ಆದರೆ ಅವರೆಲ್ಲರ ಆಶ್ಚರ್ಯಕ್ಕೆ, ತ್ರಿಗುಣಾತೀತಾನಂದರು ಒಂದೆರಡು ದಿನಗಳಲ್ಲೇ ಮಠಕ್ಕೆ ಹಿಂದಿರುಗಿ ಬಂದರು. ಬಹಳ ದೂರ ಹೋಗಲು ಅವರಿಂದ ಸಾಧ್ಯವಾಗಲಿಲ್ಲ.

ಇಲ್ಲಿ ಸ್ವಾಮೀಜಿಯವರು ಪರಿವ್ರಾಜಕ ಜೀವನವನ್ನು ಅನುಮೋದಿಸುವುದಿಲ್ಲ ಎಂಬಂತೆ ತೋರುತ್ತದೆ. ಆದರೆ ಇನ್ನು ಕೆಲವೇ ದಿನಗಳಲ್ಲಿ ಅವರೇ ಪರಿವ್ರಾಜಕರಾಗಿ ದೇಶಸಂಚಾರಕ್ಕೆ ಹೊರಡುವುದನ್ನು ನಾವು ನೋಡುತ್ತೇವೆ. ಹಾಗಾದರೆ ಈಗೇಕೆ ಆ ರೀತಿ ಹೇಳಿದರು? ಅದಕ್ಕೆ ಉತ್ತರವಿಷ್ಟೆ: ಭಗವಂತನ ಸಾಕ್ಷಾತ್ಕಾರಕ್ಕೆ ನಿಜವಾಗಿಯೂ ಈ ಅಲೆದಾಟವೇನೂ ಸಹಾಯ ಮಾಡುವುದಿಲ್ಲ. ಒಂದು ಕಡೆ ನೆಲೆನಿಂತು ಸಾಧನೆ ಮಾಡಿದರೆ ಮಾತ್ರ ಸಾಕ್ಷಾತ್ಕಾರ ಮಾಡಿಕೊಳ್ಳ ಬಹುದು. ಸ್ವಯಂ ಶ್ರೀರಾಮಕೃಷ್ಣರೇ ಎಲ್ಲೂ ಅಲೆದಾಡಿದವರಲ್ಲ. ಅವರು ದಕ್ಷಿಣೇಶ್ವರದಲ್ಲೇ ನೆಲೆನಿಂತು ಸ್ಥಿರಗತಿಯಿಂದ ಸಾಧನೆ ಮಾಡಿದವರು. ಅವರೇ ಹೇಳುತ್ತಿದ್ದರು–“ಸಾಧಕನಾದವನು ಮೊದಲು ‘ಬಹೂದಕ’ನಾಗುತ್ತಾನೆ, ಬಳಿಕ ‘ಕುಟೀಚಕ’ನಾಗುತ್ತಾನೆ” ಎಂದು. ಬಹೂದಕನೆಂದರೆ, ಹಲವಾರು ಸ್ಥಳಗಳಲ್ಲಿನ ಹಲವಾರು ಬಾವಿಗಳ-ಕೆರೆಗಳ ನೀರು ಕುಡಿಯುವವನು. ಅರ್ಥಾತ್ ಹಲವು ಹದಿನೆಂಟು ಕಡೆ ತಿರುಗಾಡುವವನು. ಇವನು ಹಲವಾರು ಕಡೆ ತಿರುಗಾಡುವುದೇಕೆಂದರೆ, ಇವನ ಚಂಚಲ ಮನಸ್ಸಿಗೆ ಅನ್ನಿಸುತ್ತದೆ–‘ಆಹ್! ಹೃಷೀಕೇಶಕ್ಕೆ ಹೋಗಿ ತಪಸ್ಸು ಮಾಡಿದರೆ ಬೇಗ ಭಗವಂತನನ್ನು ಸಾಕ್ಷಾತ್ಕಾರಿಸಿಕೊಳ್ಳಬಹುದು’ ಎಂದು. ಸರಿ; ಹೃಷೀಕೇಶಕ್ಕೆ ಹೋಗುತ್ತಾನೆ, ಸಾಧನೆ ಮಾಡುತ್ತಾನೆ. ಆದರೆ ಭಗವಂತ ದರ್ಶನ ಕೊಡುವ ಸೂಚನೆ ಕಂಡುಬರುವುದಿಲ್ಲ. ಇವನ ಚಂಚಲ ಮನಸ್ಸಿಗೆ ಆಗ ಹೃಷೀಕೇಶವು ರುಚಿಸುವುದಿಲ್ಲ. ಆಗ ಅವನಿಗೆ ರಾಮೇಶ್ವರದ ನೆನಪಾಗು ತ್ತದೆ. ಸರಿ, ಹೊರಡುತ್ತಾನೆ. ದಾರಿಯಲ್ಲಿ ಇನ್ನೂ ಹಲವಾರು ತೀರ್ಥಕ್ಷೇತ್ರಗಳ ಹೆಸರುಗಳೆಲ್ಲ ಕಿವಿಗೆ ಬೀಳುತ್ತವೆ. ಅಲ್ಲಿಗೆಲ್ಲಾ ಹೋಗಿ ನೋಡುತ್ತಾನೆ. ತೀರ್ಥ ಕುಡಿಯುತ್ತಾನೆ. ಹಾಗೇ ರಾಮೇಶ್ವರಕ್ಕೆ ಬರುತ್ತಾನೆ. ಅಲ್ಲಿಯೂ ಸಾಧನೆ ಮಾಡಲಾರಂಭಿಸುತ್ತಾನೆ. ಆದರೆ ಭಗವಂತನ ಸಾಕ್ಷಾತ್ಕಾರ ಮಾತ್ರ ಆಗುವುದಿಲ್ಲ. ಭಗವಂತನ ಸಾಕ್ಷಾತ್ಕಾರವಾಗದಿರುವುದು ಆ ಕ್ಷೇತ್ರದ ಶಕ್ತಿಯ ಅಭಾವದಿಂದಲ್ಲ, ಇವನ ಮನಸ್ಸಿನ ಚಂಚಲತೆಯಿಂದ. ಈ ಸತ್ಯ ಅವನ ಮನಸ್ಸಿಗೆ ಯಾವಾಗ ಗೋಚರಿಸುತ್ತದೆಂದರೆ, ಹಲವಾರು ಕ್ಷೇತ್ರಗಳ ತೀರ್ಥ ಕುಡಿದ ಮೇಲೆ–‘ಬಹೂದಕ’ನಾದ ಮೇಲೆ. ಆಗ ಅವನು ಒಂದು ಸ್ಥಳವನ್ನು ಆರಿಸಿಕೊಂಡು, ಒಂದು ಕುಟೀರವನ್ನು ಕಟ್ಟಿಕೊಂಡು, ಅಲ್ಲಿ ಸ್ಥಿರವಾಗಿ ನೆಲೆನಿಂತು ನಿಶ್ಚಲ ಮನಸ್ಸಿನಿಂದ ಸಾಧನೆಗೆ ತೊಡುಗುತ್ತಾನೆ. ಹೀಗೆ ಅವನು ‘ಕುಟೀಚಕ’ನಾಗುತ್ತಾನೆ. ನಿಜವಾಗಿ ಸಾಧನೆ ಮಾಡಬೇಕೆಂಬವನು ಕೊನೆಗೆ ಕುಟೀಚಕನಾಗಿ ದೃಢ ವಾಗಿ ನಿಲ್ಲಬೇಕಾಗುತ್ತದೆ. ಕ್ರಮವಾಗಿ, ನಿಯಮಬದ್ಧವಾಗಿ ಸಾಧನೆ ಮಾಡಬೇಕಾದರೆ ಅಲ್ಲಿ ಇಲ್ಲಿ ಅಲೆದಾಡುವುದರಿಂದ ಸಾಧ್ಯವೆ? ‘ಹಾಗಾದರೆ, ಸ್ವಾಮಿ ವಿವೇಕಾನಂದರೇ ಏಳು ವರ್ಷಗಳ ಕಾಲ ಭಾರತವಿಡೀ ಸಂಚರಿಸಿದರಲ್ಲಾ!’ ಎಂದರೆ, ಅದು ಬೇರೆಯೇ ವಿಚಾರ. ಇಡೀ ಭಾರತದ ಸಾಮಾಜಿಕ, ಸಾಂಸ್ಕೃತಿಕ, ಆಧ್ಯಾತ್ಮಿಕ ಹಿನ್ನೆಲೆಯನ್ನು ಹಾಗೂ ಪ್ರಕೃತ ಪರಿಸ್ಥಿತಿಗಳನ್ನು ಅವರು ಕಣ್ಣಾರೆ ಕಂಡು ಅರಿತುಕೊಳ್ಳಬೇಕಾದ ಆವಶ್ಯಕತೆಯಿತ್ತು. ಭಾರತದಾದ್ಯಂತ ನೂರಾರು ಜನರ ಸ್ನೇಹ-ಪ್ರೇಮ-ಭಕ್ತಿಗಳನ್ನು ಗೆದ್ದು, ಅವರಿಂದ ತಮ್ಮ ಧರ್ಮಪ್ರಸಾರ ಕಾರ್ಯಕ್ಕೆ ನೆರವನ್ನು ಪಡೆಯಬೇಕಾಗಿತ್ತು. ಅಲ್ಲದೆ, ಸ್ವತಃ ವಿವೇಕಾನಂದರ ವ್ಯಕ್ತಿತ್ವ ಕೂಡ ಮತ್ತಷ್ಟು ದೃಢವೂ ಪ್ರೌಢವೂ ಆಗಬೇಕಾಗಿತ್ತು. ಶ್ರೀರಾಮಕೃಷ್ಣರೇ ಹೇಳಿದಂತೆ, ಅವರು ಮುಂದೆ ಲೋಕಶಿಕ್ಷಣ ಕಾರ್ಯ ಕೈಗೊಳ್ಳಬೇಕಾದದ್ದಿದೆ. ಜನತೆಗೆ ಬೋಧನೆ ಮಾಡಬೇಕಾದದ್ದಿದೆ. ಗುಲಾಮಗಿರಿಯಲ್ಲಿ ಮುಳುಗಿ ಹೋಗಿದ್ದ ಭಾರತವನ್ನು ಮೇಲೆತ್ತಬೇಕಾದ ಮಹಾರ್ಕಾಯವಿದೆ. ಆದ್ದರಿಂದ ಅವರು ದೇಶಸಂಚಾರ ಮಾಡಲೇಬೇಕಾಗಿದೆ. ಆದರೆ ಅವರಂತೆಯೇ ಎಲ್ಲರೂ ಬಾರಾನಗೋರ್ ಮಠ ದಿಂದ ಹೊರಟುಬಿಟ್ಟರೆ ಇವರ ಸಂಘವೆನ್ನುವುದು ಹೇಳಹೆಸರಿಲ್ಲದಂತಾಗಿಬಿಡುವ ಸಾಧ್ಯತೆ ಯಿತ್ತು. ಆದ್ದರಿಂದ ಕೆಲವರಾದರೂ ಅಲ್ಲಿ ಉಳಿದುಕೊಳ್ಳುವಂತೆ ಮಾಡುವ ಉದ್ದೇಶದಿಂದಲೇ ಸ್ವಾಮೀಜಿಯವರು ಸೋದರಸಂನ್ಯಾಸಿಗಳ ವಿಷಯದಲ್ಲಿ ಪರಿವ್ರಾಜಕ ಜೀವನವನ್ನು ಪ್ರೋತ್ಸಾಹಿಸುವುದಿಲ್ಲ.

ದಿನ ಕಳೆದಂತೆ, ಪರಿವ್ರಾಜಕರಾಗಿ ಹೊರಟುಬಿಡಬೇಕೆಂಬ ಹಂಬಲ ಸ್ವಾಮೀಜಿಯಲ್ಲಿ ತೀವ್ರ ವಾಯಿತು. ಇತರ ಸೋದರ ಸಂನ್ಯಾಸಿಗಳಲ್ಲಿ ಈ ತೀರ್ಥಾಟನೆಯ ಹಂಬಲ ನೂರಕ್ಕೆ ಐವತ್ತರ ಷ್ಟಿದ್ದರೆ, ಸ್ವಾಮೀಜಿಯಲ್ಲಿ ಅದು ನೂರಕ್ಕೆ ಸಾವಿರದಷ್ಟಿತ್ತು. ಹಿಂದೆ ಅವರು ಸೋದರಸಂನ್ಯಾಸಿ ಗಳಿಗೆ ‘ಸುಮ್ಮನೆ ಅಲೆದಾಡಿ ಏನು ಸಾಧಿಸುತ್ತೀರಿ?’ ಎಂದಿದ್ದರೂ ಕೂಡ, ಈಗ ಅದೇ ಪರಿವ್ರಜನದ ಗುಣಗಾನ ಮಾಡುತ್ತಿದ್ದರು. ಈಗ ತಪಸ್ಸಿಗಾಗಿ, ತೀರ್ಥಯಾತ್ರೆಗಾಗಿ ಹೊರಟು ನಿಂತ ಸೋದರಸಂನ್ಯಾಸಿಗಳ ಬಗ್ಗೆ ಅವರ ಧೋರಣೆಯೇನೆಂದರೆ–‘ಅವರು ತಂತಮ್ಮ ಅನು ಭವವನ್ನು ತಾವೇ ಗಳಿಸಿಕೊಳ್ಳಲಿ. ಈ ಮಠದಿಂದ ಬಿಡಿಸಿಕೊಂಡುಹೋಗಿ, ಸ್ವತಂತ್ರರಾಗಿದ್ದು, ತಮ್ಮ ಸತ್ತ್ವಪರೀಕ್ಷೆ ಮಾಡಿಕೊಳ್ಳಲಿ, ಈ ಹೊಸ ಜೀವನಾನುಭವಗಳು ಅವರನ್ನು ಧೀರರನ್ನಾಗಿ, ಅಜೇಯರನ್ನಾಗಿ ಮಾಡುತ್ತವೆ. ಹೀಗೆ ಅವರು ಪುರುಷಸಿಂಹರಾಗುತ್ತಾರೆ’ ಎಂಬುದಾಗಿತ್ತು. ಅದನ್ನೇ ತಮ್ಮ ಗುರುಭಾಯಿಗಳ ಮುಂದೆ ವ್ಯಕ್ತಪಡಿಸಿ, ಪರಿವ್ರಾಜಕ ಜೀವನ ಕೈಗೊಳ್ಳುವಂತೆ ಅವರನ್ನು ಪ್ರೋತ್ಸಾಹಿಸಿದರು. ಅದಾಗಲೇ ಕೆಲವರು ತಪಸ್ಸಿಗಾಗಿ ದೂರಸ್ಥಳಗಳಿಗೆ ಹೋಗಿ ಯಾಗಿತ್ತು. ಉಳಿದವರಲ್ಲೂ ಆ ಇಚ್ಛೆ ಇದ್ದೇ ಇತ್ತು, ಈಗ ಸ್ವಾಮೀಜಿಯ ಅನುಮತಿ-ಪ್ರೋತ್ಸಾಹ ಸಿಕ್ಕಿದಾಗ ಉಳಿದವರೂ ಒಬ್ಬೊಬ್ಬರಾಗಿ ಹೊರಟರು–ಸ್ವಾಮಿ ರಾಮಕೃಷ್ಣಾನಂದರೊಬ್ಬರನ್ನು ಬಿಟ್ಟು. ರಾಮಕೃಷ್ಣಾನಂದರು ಮಠದಲ್ಲಿ ಶ್ರೀರಾಮಕೃಷ್ಣರ ಪೂಜೆಯನ್ನೇ ತಮ್ಮ ಪ್ರಧಾನ ಸಾಧನೆಯನ್ನಾಗಿ ಮಾಡಿಕೊಂಡವರು. ಅವರೂ ತೀರ್ಥಾಟನೆಗಾಗಿ ಅಥವಾ ಸಾಧನೆಗಾಗಿ ದೇಶಾಂತರ ಹೊರಟರೆ ಇಲ್ಲಿ ತಮ್ಮ ಗುರುದೇವನ ಪೂಜಾದಿಗಳನ್ನು ಬಿಡದೆ ಮಾಡುವವರು ಯಾರು? ಆದ್ದರಿಂದ ಎಷ್ಟೇ ಒತ್ತಾಯ ಮಾಡಿದರೂ ಅವರು ಹೊರಡಲೇ ಇಲ್ಲ. ಸ್ವಾಮೀಜಿ ಈಗ ಅವರನ್ನು ತೀರ್ಥಾಟನೆ ಮಾಡುವಂತೆ ಒತ್ತಾಯಿಸಿದ್ದರೂ, ಮುಂದೆ ತಾವೇ “ನಿಜಕ್ಕೂ ಅವನು ಮಠದ ಬೆನ್ನಲುಬು!”ಎಂದು ಕೊಂಡಾಡುತ್ತಾರೆ.

ಸ್ವಾಮೀಜಿಯ ಮನಸ್ಸು ಈಗ, ಮಠದಿಂದ ಹೊರಟು ಪರಿವ್ರಾಜಕರಾಗಿ ದೀರ್ಘಕಾಲ ಸುತ್ತಾಡಬೇಕೆಂದು ತುಂಬ ಕಾತರಗೊಂಡಿತ್ತು. ಆದರೆ, ಹಾಗೆ ಮಾಡಲು ಅವರಿಗೆ ಬಹಳ ಕಾಲ ಸಾಧ್ಯವಾಗಲಿಲ್ಲ. ಎಷ್ಟೋ ಸಲ ಹಾಗೆ ಹೊರಟಿದ್ದರೂ, ಕೆಲವೇ ತಿಂಗಳಲ್ಲಿ ಹಿಂದಿರುಗಿ ಬಂದು ಬಿಡಬೇಕಾಯಿತು. ಪ್ರತಿ ಸಲ ಹೊರಡುವಾಗಲೂ, “ಇನ್ನು ನಾನು ಹಿಂದಿರುಗುವುದು ಬಹಳ ವರ್ಷಗಳ ಮೇಲೆಯೇ” ಎಂದು ತಮ್ಮ ಗುರುಭಾಯಿಗಳಿಗೆ ಹೇಳುತ್ತಿದ್ದರು. ಆದರೆ ಒಂದಲ್ಲ ಒಂದು ಕಾರಣದಿಂದಾಗಿ ಹಿಂದಿರುಗಬೇಕಾಗುತ್ತಿತ್ತು. ಒಂದು ಸಲ ತೀವ್ರವಾಗಿ ಕಾಯಿಲೆ ಬಿದ್ದದ್ದರಿಂದ ಹಿಂದಿರುಗಿ ಬರಬೇಕಾಯಿತು. ಇನ್ನೊಮ್ಮೆ, ತಮ್ಮ ಸೋದರಸಂನ್ಯಾಸಿಗಳಲ್ಲೇ ಒಬ್ಬರಿಗೆ ಕಾಯಿಲೆ ಎಂಬ ಕಾರಣದಿಂದ ಬರಬೇಕಾಯಿತು. ಮತ್ತೊಮ್ಮೆ, ಶ್ರೀರಾಮಕೃಷ್ಣರ ಗೃಹೀಭಕ್ತನೂ ತಮ್ಮೆಲ್ಲರ ಆಪ್ತನೂ ಆದ ಬಲರಾಮ್ ಬಾಬು ತೀರಿಕೊಂಡನೆಂಬ ಸುದ್ದಿ ತಿಳಿಯಿತು. ಅದನ್ನು ಕೇಳಿ ದುಃಖಿತರಾಗಿ, ಆ ಸಂದರ್ಭದಲ್ಲಿ ಬಲರಾಮನ ಕುಟುಂಬದವರನ್ನು ಸಂಧಿಸಿ ಅವರಿಗೆ ಸಾಂತ್ವನ ನೀಡಲು ಹಿಂದಿರುಗಿ ಬರುವುದನ್ನು ನೋಡಲಿದ್ದೇವೆ. ಅಂತೂ, ಮುಂದೆ ೧೮೯೦ರಲ್ಲಿ ಬಾರಾನಗೋರ್ ಮಠವನ್ನು ಬಿಟ್ಟು ಹೊರಟು, ಸಮಸ್ತ ಭಾರತದ ಸಂಚಾರವನ್ನೂ, ಪಶ್ಚಿಮ ದೇಶಗಳ ಜೈತ್ರಯಾತ್ರೆಯನ್ನೂ ಮುಗಿಸಿಕೊಂಡು ಹಿಂದಿರುಗುವವರೆಗೆ, ಏಳು ವರ್ಷಗಳ ದೀರ್ಘಕಾಲ ಮಠದಿಂದ ದೂರ ಉಳಿದಿದ್ದರು.

ಆದರೆ ಒಂದು ದೃಷ್ಟಿಯಿಂದ, ಅವರು ಹಾಗೆ ವಾಪಸು ಬರುತ್ತಿದ್ದುದಕ್ಕೆ, ಇವೆಲ್ಲ ನೆಪಮಾತ್ರ ಕಾರಣಗಳು ಎನ್ನಬೇಕು. ನಿಜಕ್ಕೂ, ಸ್ವಾಮೀಜಿಯವರು ತಮ್ಮ ಇತರ ಸೋದರಸಂನ್ಯಾಸಿಗಳಂತೆ ನಿರ್ಯೋಚನೆಯಿಂದ ಹೊರಟುಬಿಡಲು ಸಾಧ್ಯವಿರಲಿಲ್ಲ. ಅವರು ಸಂಘದ ನಾಯಕರಲ್ಲವೆ? ಆದ್ದರಿಂದ, ಮಠ ಒಂದು ಸುಸಂಘಟಿತ ಸ್ಥಿತಿಗೆ ಬರುವುದಕ್ಕೆ ಮುಂಚೆಯೇ ಅವರು ಒಂದೇ ಸಲಕ್ಕೆ ಹೊರಟುಬಿಟ್ಟಿದ್ದರೆ, ಸಂಘ ನಿರ್ಮಾಣ ಕಾರ್ಯದ ಮಹೋದ್ದೇಶ ಕನಸಾಗಿಯೇ ಉಳಿಯುತ್ತಿತ್ತು. ಆದರೆ ಈ ಮಠ ಹಾಗೂ ಸೋದರಸಂನ್ಯಾಸಿಗಳೊಂದಿಗಿನ ತಮ್ಮ ಸಂಬಂಧ ವನ್ನು ಅವರು ‘ಚಿನ್ನದ ಸರಪಳಿ’ ಎಂದು ಕೆಲವೊಮ್ಮೆ ಭಾವಿಸುತ್ತಿದ್ದರು. ಅಂತೂ, ಕಾಲ ಪಕ್ವವಾಗಿದೆ ಎನ್ನುವವರೆಗೆ ಅವರು ಮಠಕ್ಕೆ ಆಗಾಗ ಹಿಂದಿರುಗಿ ಬರುತ್ತಲೇ ಇರಬೇಕಾಯಿತು.

ಪರಿವ್ರಾಜಕ ವಿವೇಕಾನಂದರದು ಎಲ್ಲರ ಕಣ್ಮನ ಸೆಳೆಯುವಂತಹ ವ್ಯಕ್ತಿತ್ವ. ಒಬ್ಬ ಸಾಮಾನ್ಯ ಸಂನ್ಯಾಸಿಯಂತೆ ಕಾವಿಬಟ್ಟೆ ತೊಟ್ಟಿದ್ದರೂ ಅವರಲ್ಲಿ ರಾಜತೇಜಸ್ಸು, ರಾಜಗಾಂಭೀರ್ಯ ಎದ್ದು ಕಾಣುತ್ತಿತ್ತು. ಅವರ ಸುದೃಢವಾದ ಎತ್ತರದ ಮೈಕಟ್ಟು, ದಷ್ಟಪುಷ್ಟವಾದ ಅಂಗಾಂಗಗಳು, ತೇಜಃಪುಂಜವಾದ ಕಣ್ಣುಗಳು, ಮನೋದಾರ್ಢ್ಯವನ್ನು ಪ್ರತಿಬಿಂಬಿಸುವ ಮುಖಲಕ್ಷಣ, ಪ್ರಭಾವ ಶಾಲಿಯಾದ, ಮಹಿಮೆಯನ್ನು ಸಾರಿ ಹೇಳುವ ಒಟ್ಟು ವ್ಯಕ್ತಿತ್ವ–ಇವುಗಳು ಅವರು ಹೋದ ಹೋದಲ್ಲೆಲ್ಲ ಜನ ಅವರನ್ನೇ ದಿಟ್ಟಿಸಿ ನೋಡುವಂತೆ ಮಾಡುತ್ತಿದ್ದುವು. ಅವರ ಶರೀರದ ಪ್ರತಿಯೊಂದು ಚಲನವಲನವೂ ನಯನಮನೋಹರವಾಗಿತ್ತು. ನೋಡಿದ ಕೂಡಲೇ ಕುತೂಹಲ ಮಿಶ್ರಿತ ಗೌರವ-ಭಕ್ತಿಗಳ ಭಾವವನ್ನೆಬ್ಬಿಸುವ ವ್ಯಕ್ತಿತ್ವ. ಕೈಯಲ್ಲಿ ದಂಡ, ಕಮಂಡಲು ಹಾಗೂ ಒಂದು ಪುಟ್ಟ ಗಂಟು; ಗಂಟಿನಲ್ಲಿ ಎರಡು ಪುಸ್ತಕಗಳು. ಒಂದು ಭಗವದ್ಗೀತೆ; ಇನ್ನೊಂದು, \eng{The Imitation of Christ} (‘ಕ್ರಿಸ್ತನ ಅನುಸರಣೆ’). ಇವಿಷ್ಟನ್ನು ಹಿಡಿದುಕೊಂಡು, ಕಾಷಾಯ ವಸ್ತ್ರಧಾರಿಯಾದ ಸ್ವಾಮೀಜಿ, ಶಾಂತವಾಗಿ, ಗಂಭೀರ ಮುಖಮುದ್ರೆಯನ್ನು ಧರಿಸಿ, ಆದರೆ ಹೃದಯದಲ್ಲಿ ಪರಮೋಲ್ಲಾಸದಿಂದ ಪರಿವ್ರಾಜಕರಾಗಿ ಸಾಗುತ್ತಿದ್ದರು.

ತಮ್ಮ ಪರಿವ್ರಾಜಕ ದಿನಗಳ ಬಗ್ಗೆ ಸ್ವಾಮೀಜಿ ದಿನಚರಿಯನ್ನಾಗಲಿ, ಇತರ ಟಿಪ್ಪಣಿಗಳನ್ನಾಗಲಿ ಬರೆದಿಡಲಿಲ್ಲ. ಅಲ್ಲದೆ ಅವುಗಳ ಬಗ್ಗೆ ಮುಂದೆ ಇತರರಿಗೆ ತಿಳಿಸಿದ್ದು ಕೂಡ ಅಪರೂಪ. ಯಾವಾಗಲಾದರೂ ಹಗುರವಾಗಿ ಯಾವುದಾದರೊಂದು ವಿಷಯವನ್ನು ಪ್ರಸ್ತಾಪಿಸುವುದಿತ್ತು, ಅಷ್ಟೆ. ಈ ದಿನಗಳಲ್ಲಿ ತಮಗಾದ ಆಧ್ಯಾತ್ಮಿಕ ಅನುಭವಗಳನ್ನಂತೂ ಅವರು ತಮ್ಮ ಸೋದರ ಸಂನ್ಯಾಸಿಗಳ ಹತ್ತಿರವೂ ಪ್ರಸ್ತಾಪಿಸುತ್ತಿರಲಿಲ್ಲ. ಆದ್ದರಿಂದ ಅವರ ಪ್ರಯಾಣದ ಅನುಭವಗಳ ವಿಷಯವಾಗಲಿ, ಜನಸಂಪರ್ಕದ ವಿವರಗಳಾಗಲಿ ನಮಗೆ ತಿಳಿದಿರುವುದು ಸ್ವಲ್ಪವೇ. ಹೀಗಿದ್ದರೂ ಅವರ ವಿವಿಧ ಅನುಭವಗಳಲ್ಲಿ ಕೆಲವನ್ನು ತಿಳಿಯಲು ಸಾಧ್ಯವಾಗಿದೆ. ಹೇಗೆಂದರೆ, ಎಷ್ಟೋ ಸಲ ಅವರ ಕೆಲವು ಸೋದರಸಂನ್ಯಾಸಿಗಳು ಅವರ ಜೊತೆಯಲ್ಲಿರುತ್ತಿದ್ದರು. ರಾಮಕೃಷ್ಣಾನಂದರನ್ನೂ ಅದ್ಭುತಾನಂದರನ್ನೂ ಬಿಟ್ಟು ಉಳಿದವರೆಲ್ಲರೂ ಮೊದಲ ನಾಲ್ಕು ವರ್ಷಗಳ ಅವಧಿಯಲ್ಲಿ ಬೇರಬೇರೆ ಸಮಯಗಳಲ್ಲಿ ಸ್ವಾಮೀಜಿಯೊಂದಿಗಿದ್ದರು. ಅದರಲ್ಲೂ ಅಖಂಡಾನಂದರು ಹೆಚ್ಚು ಕಾಲ ಇದ್ದರು. ಇವರೆಲ್ಲರ ಮೂಲಕ ಸ್ವಾಮೀಜಿಯ ಜೀವನದ ಈ ದಿನಗಳ ಬಗ್ಗೆ ಕೆಲವು ವಿವರ ಗಳು ದೊರಕಿವೆ. ಅಲ್ಲದೆ ಮುಂದೆ ಅವರು ಭಾರತದ ಬೇರೆಬೇರೆ ಪ್ರಾಂತಗಳಲ್ಲಿ ತಿರುಗಾಡುವಾಗ, ಅವರ ಸಂಪರ್ಕಕ್ಕೆ ಬಂದು, ಅವರಿಂದ ಗಾಢವಾಗಿ ಪ್ರಭಾವಿತರಾದವರು ನೂರಾರು ಜನ. ಇವರಲ್ಲನೇಕರು ಅವರ ಸಂಬಂಧವಾಗಿ ಹಲವಾರು ವಿಷಯಗಳನ್ನು, ಸಂಭಾಷಣೆಗಳನ್ನು ಬರೆ ದಿಟ್ಟುಕೊಂಡಿದ್ದರು. ಇನ್ನು ಅನೇಕರು ಮುಂದೆ ತಮ್ಮ ನೆನಪುಗಳಿಂದ ಸ್ಮೃತಿಚಿತ್ರವನ್ನು ರಚಿಸಿ ದಾಖಲಿಸಿದರು. ಈ ದಾಖಲೆಗಳ ಮೂಲಕ ನಮಗೆ ಹಲವಾರು ವಿಷಯಗಳು ತಿಳಿದುಬರುತ್ತವೆ. ಇವಲ್ಲದೆ, ಸ್ವಾಮೀಜಿ ತಮ್ಮ ಸೋದರಸಂನ್ಯಾಸಿಗಳಿಗೆ ಹಾಗೂ ಇತರ ಭಕ್ತರಿಗೆ ಆಗಾಗ ಬರೆದ ಹಲವಾರು ಪತ್ರಗಳು ಸಿಕ್ಕಿವೆ. ಈ ಎಲ್ಲ ಆಧಾರಗಳ ಸಹಾಯದಿಂದ ಅವರು ಜಗದ್ವಿಖ್ಯಾತ ಸ್ವಾಮಿ ವಿವೇಕಾನಂದರಾಗುವವರೆಗಿನ ಅವಧಿಯ ಒಂದು ಸ್ಥೂಲ ಚಿತ್ರವನ್ನು ರಚಿಸಬಹುದಾಗಿದೆ.

ಪರಿವ್ರಾಜಕರಾಗಿದ್ದ ಆ ದಿನಗಳಲ್ಲಿ ಸ್ವಾಮೀಜಿ ತಮ್ಮ ಅಪಾರ ವಿದ್ಯೆಯನ್ನೂ ಸಾಮರ್ಥ್ಯ ಗಳನ್ನೂ ಅಡಗಿಸಿಕೊಂಡು ಒಬ್ಬ ತೀರ ಸಾಮಾನ್ಯ ಸಂನ್ಯಾಸಿಯಂತೆ ಸುತ್ತಾಡುತ್ತಿದ್ದರು. ಅವ ರಾಗಿಯೇ ಬಾಯಿಬಿಟ್ಟು ಮಾತನಾಡದಿದ್ದರೆ ಅವರಿಗೆ ಇಂಗ್ಲಿಷ್ ಭಾಷೆಯಲ್ಲಿ ಪ್ರಚಂಡ ಜ್ಞಾನವಿತ್ತು ಎಂಬುದನ್ನು ಊಹಿಸಲು ಯಾರಿಗೂ ಸಾಧ್ಯವಿರಲಿಲ್ಲ. ಕೆಲವೊಮ್ಮೆ ಅವರು ಭಿಕ್ಷೆಗಾಗಿ ಮನೆಮನೆಗೆ ಹೋಗದೆ, ತಾನಾಗಿಯೇ ಏನು ಬರುತ್ತದೆಯೋ ಅಷ್ಟರಲ್ಲೇ ಜೀವಿಸುವುದು ಎಂದು ನಿಶ್ಚಯಿಸುತ್ತಿದ್ದರು. ಆದರೆ ಬಡಪಾಯಿ ಸಂನ್ಯಾಸಿಯನ್ನು ತಾವಾಗಿಯೇ ಕರೆದು ಊಟ ಬಡಿಸು ವವರು ಎಷ್ಟು ಜನರಿದ್ದಾರು? ಎಷ್ಟೋ ದಿನ ಉಪವಾಸವೇ ಗತಿಯಾಗುತ್ತಿತ್ತು. ಇವರಾಗಿ ಭಿಕ್ಷೆ ಕೇಳುವವರಲ್ಲ; ಜನ ತಾವಾಗಿಯೇ ಭಿಕ್ಷೆ ಹಾಕುವವರಲ್ಲ. ಈ ವ್ರತದಂತೆ ಅವರು ಅತಿ ಹೆಚ್ಚೆಂದರೆ ಐದು ದಿನಗಳವರೆಗೂ ನಿರಂತರವಾಗಿ ಉಪವಾಸವಿದ್ದದ್ದೂ ಉಂಟು! ಎಷ್ಟೋ ಸಲ ಹಾಳುಬಿದ್ದ ಗುಡಿಯಲ್ಲೋ ಛತ್ರದಲ್ಲೋ ರಾತ್ರಿಯನ್ನು ಕಳೆಯುತ್ತಿದ್ದರು. ಇನ್ನು ಕೆಲವು ಸಲ ಕಾಡಿನ ನಡುವೆ, ಮರದ ಬುಡದಲ್ಲಿ ನಿದ್ರಿಸಿದ್ದೂ ಉಂಟು. ಒಮ್ಮೊಮ್ಮೆ ವಿಶಾಲವಾದ ಮೈದಾನದಲ್ಲಿ ಪ್ರಯಾಣ ಮಾಡುತ್ತಿರುವಾಗಲೇ ಕತ್ತಲಾಗಿ ಬಿಡುತ್ತಿತ್ತು. ಮುಂದೆ ಹೋಗಲು ಸಾಧ್ಯವಿಲ್ಲ, ಹಿಂದೆ ಹೋಗುವಂತೆಯೂ ಇಲ್ಲ. ಆಗ ನೆಲವೇ ಹಾಸುಗೆ, ಗಗನವೇ ಹೊದಿಕೆ! ಮಳೆ-ಬಿಸಿಲು-ಚಳಿಗೆ ಸಿಕ್ಕಿಕೊಳ್ಳುತ್ತಿದ್ದ ಸಂದರ್ಭಗಳಂತೂ ಹಲವಾರು. ಈ ದಿನಗಳಲ್ಲಿ ಸ್ವಾಮೀಜಿ ಹಣ ಮುಟ್ಟುವುದಿಲ್ಲ ಎಂದು ವ್ರತ ಮಾಡಿದ್ದರು. ಕೆಲವು ಸಲ ಯಾರಾದರೂ ತುಂಬ ಬಲವಂತ ಮಾಡಿದಾಗ, ಅವರಿಂದ ಮುಂದಿನ ಊರಿಗೊಂದು ರೈಲು ಟಿಕೆಟ್ಟನ್ನು ಮಾತ್ರ ಪಡೆದುಕೊಳ್ಳುತ್ತಿದ್ದರು. ಉಳಿದಂತೆ ಸಾಮಾನ್ಯವಾಗಿ ಕಾಲ್ನಡಿಗೆಯಲ್ಲೇ ಪ್ರಯಾಣ. ಇಂತಹ ಎಲ್ಲ ಕಷ್ಟಗಳನ್ನು ತುಟಿಪಿಟಕ್ಕೆನ್ನದೆ ಸಹಿಸಿಕೊಂಡು, ಆಂತರ್ಯದಲ್ಲಿ ಆನಂದವನ್ನು ತುಂಬಿಕೊಂಡು ಮುನ್ನಡೆಯುತ್ತಿದ್ದರು, ವೀರಸಂನ್ಯಾಸಿ ವಿವೇಕಾನಂದರು. ಈ ಸ್ಥಿತಿಯನ್ನೇ ಅವರು ತಮ್ಮ ‘ಸಂನ್ಯಾಸಿಗೀತೆ’ಯಲ್ಲಿ \eng{( Song of the Sannyasin)} ಹೃದಯಂಗಮವಾಗಿ ಬಣ್ಣಿಸುತ್ತಾರೆ–

\begin{myquote}
ಗಗನವೇ ಮನೆ! ಹಸುರೆ ಹಾಸಿಗೆ! ಮನೆಯು ಸಾಲ್ವುದೆ ಚಾಗಿಗೆ?\\ಹಸಿಯೊ ಬಿಸಿಯೊ ಬಿದಿಯು ಕೊಟ್ಟಾಹಾರವನ್ನವು ಯೋಗಿಗೆ!\\ಏನು ತಿಂದರೆ ಏನು ಕುಡಿದರೆ ಏನು ಆತ್ಮಗೆ ಕೊರತೆಯೆ?\\ಸರ್ವಪಾಪವ ತಿಂದು ತೇಗುವ ಗಂಗೆಗೇನ್ ಕೊಳೆ ಕೊರತೆಯೆ?\\ನೀನು ಮಿಂಚೈ! ನೀನು ಸಿಡಿಲೈ! ಮೊಳಗು ಸಂನ್ಯಾಸಿ!
\end{myquote}

\begin{flushright}
ಓಂ ತ್ತ್ ಸತ್ ಓಂ
\end{flushright}

\noindent

ಸುಮಾರು ೧೮೮೭ರ ಮಧ್ಯಭಾಗದಲ್ಲಿ ಸ್ವಾಮೀಜಿ ಮೊಟ್ಟಮೊದಲು ಬಾರಿ ಬಾರಾನಗೋರ್ ಮಠದಿಂದ ಹೊರಟು, ಸ್ವಾಮಿ ಪ್ರೇಮಾನಂದರು ಹಾಗೂ ಫಕೀರ ಬಾಬು ಎಂಬ ಒಬ್ಬ ಗೃಹಸ್ಥ ಭಕ್ತನೊಡನೆ ವಾರಾಣಸಿಗೆ ಬಂದರು. ಈ ವಾರಾಣಸೀ (ವಾರಣಾಸೀ, ಕಾಶೀ) ಕ್ಷೇತ್ರ ಪುರಾತನ ಕಾಲದಿಂದಲೂ ಪರಮಪಾವನಕರವಾದದ್ದೆಂಬ ಪ್ರಸಿದ್ಧಿಯನ್ನು ಗಳಿಸಿದೆ. ಇದು ಸಾಧು-ಸಂನ್ಯಾಸಿ ಗಳ ನೆಲೆವೀಡು;ಆಧ್ಯಾತ್ಮಿಕ ವಿದ್ಯೆಯ ಮಹಾಕೇಂದ್ರ. ಇದು ಕಾಶೀವಿಶ್ವನಾಥನ ಮಹಾಪೀಠ. ಪಕ್ಕದಲ್ಲೇ ಪವಿತ್ರ ಗಂಗೆ ಹರಿಯುತ್ತಿದ್ದಾಳೆ. ಅಸಂಖ್ಯಾತ ಭಕ್ತರು ಅರ್ಘ್ಯ ಕೊಡುತ್ತ ಪ್ರಾರ್ಥನೆ ಸಲ್ಲಿಸುತ್ತಿರುತ್ತಾರೆ. ಅನೇಕಾನೇಕ ದೇವಸ್ಥಾನಗಳ, ಅದರಲ್ಲೂ ಜಗತ್ಪ್ರಸಿದ್ಧವಾದ ವಿಶ್ವನಾಥ, ಅನ್ನಪೂರ್ಣೆ, ದುರ್ಗೆಯರ ದೇವಸ್ಥಾನಗಳ ನಗರ ವಾರಾಣಸಿ. ಭಗವಾನ್ ಗೌತಮ ಬುದ್ಧನೂ ಶ್ರೀಆದಿಶಂಕರರೂ ಧರ್ಮಬೋಧನೆಯನ್ನು ಮಾಡಿದ ತಾಣ ಇದು. ಈ ಎಲ್ಲ ಪವಿತ್ರ ದೃಶ್ಯಗಳು, ಭಾವನೆಗಳು ಸ್ವಾಮೀಜಿಯ ಮನಸ್ಸಿನ ಮೇಲೆ ಆಳವಾದ ಪರಿಣಾಮವನ್ನುಂಟುಮಾಡಿದುವು.

ಒಂದು ದಿನ ಬೆಳಗ್ಗೆ ಸ್ವಾಮೀಜಿ, ಶ್ರೀದುರ್ಗೆಯ ದರ್ಶನ ಮಾಡಿಕೊಂಡು ಬರುತ್ತಿದ್ದಾರೆ. ದಾರಿಯ ಒಂದು ಪಕ್ಕದಲ್ಲಿ ಎತ್ತರದ ಗೋಡೆ, ಇನ್ನೊಂದು ಪಕ್ಕಕ್ಕೆ ವಿಶಾಲವಾದ ಸರೋವರ. ಆಗ ಕೋತಿಗಳ ಹಿಂಡೊಂದು ಎದುರಾಗಿ ಅವರನ್ನು ಮುತ್ತಿಕೊಂಡಿತು. ಅವು ಅವರನ್ನು ಆ ದಾರಿ ಯಾಗಿ ಮುಂದುವರಿಯಲು ಬಿಡುವಂತೆ ಕಾಣಲಿಲ್ಲ. ಸ್ವಾಮೀಜಿ ಬೇಗ ದಾಪುಗಾಲು ಹಾಕುತ್ತ ಅಲ್ಲಿಂದ ಹಿಂದಿರುಗಲು ನೋಡಿದರು. ಆದರೆ ಆ ಕೋತಿಗಳು–ಬನಾರಿಸಿನ ಮೂರುವರೆ-ನಾಲ್ಕು ಅಡಿ ಎತ್ತರದ ರಾಕ್ಷಸ ಕೋತಿಗಳು–ಗಟ್ಟಿಯಾಗಿ ಕಿರಿಚುತ್ತ, ವಿಚಿತ್ರವಾಗಿ ಹಲ್ಲುಕಿರಿಯುತ್ತ ಅಟ್ಟಿಸಿಕೊಂಡು ಬಂದುವು, ಅವರ ಕಾಲುಹಿಡಿದು ಎಳೆಯಲು ಪ್ರಯತ್ನಿಸಿದುವು! ಆಗ ಸ್ವಾಮೀಜಿ ಓಡಲಾರಂಭಿಸಿದರು. ಅವರು ಓಡಿದಂತೆಲ್ಲ ಕೋತಿಗಳು ಇನ್ನೂ ವೇಗವಾಗಿ ಅಟ್ಟಿಸಿಕೊಂಡು ಬಂದು ಕಚ್ಚಲು ಪ್ರಯತ್ನಿಸಿದುವು! ಸ್ವಾಮೀಜಿ ಇನ್ನು ಓಡದಾದರು; ಅವುಗಳ ಕೈಗೆ ಸಿಕ್ಕಿಬೀಳು ವುದೇ ಖಂಡಿತ ಎನ್ನಿಸಿತು. ನಿಸ್ಸಹಾಯಕರಾದರು. ಈ ಸಮಯಕ್ಕೆ ಸರಿಯಾಗಿ ಎದುರುಗಡೆಯಿಂದ ಬಂದ ವೃದ್ಧ ಸಾಧುವೊಬ್ಬರು ಕೂಗಿದ್ದು ಅವರ ಕಿವಿಗೆ ಬಿತ್ತು: “ಆ ಪ್ರಾಣಿಗಳನ್ನು ಎದುರಿಸು!”\eng{ “Face the brute!”} ಈ ಮಾತು ಮಿಂಚಿನಂತೆ ಕೆಲಸಮಾಡಿತು. ಸ್ವಾಮೀಜಿಗೆ ಏನು ಮಾಡಬೇಕು ಎನ್ನುವುದು ಹೊಳೆಯಿತು. ಕೋತಿಗಳಿಗೆ ಬೆನ್ನುಮಾಡಿ ಓಡುತ್ತಿದ್ದವರು ಇದ್ದಕ್ಕಿದ್ದಂತೆ ಧೈರ್ಯ ವಾಗಿ ಎದುರಿಸಿ ನಿಂತರು. ಆತ್ಮರಕ್ಷಣೆಗೆ ಸಿದ್ಧರಾಗಿ ನಿಂತು ಅವುಗಳನ್ನು ದುರುಗುಟ್ಟಿ ನೋಡಿ ದರು. ಏನಾಶ್ಚರ್ಯ! ತಕ್ಷಣ ಆ ಗಡವ ಕೋತಿಗಳು ಓಟಕಿತ್ತುವು! ಅಬ್ಬ, ಎಂತಹ ಬಿಡಗಡೆ! ವಿಸ್ಮಯ-ಸಂತೋಷಗಳಿಂದ ಸ್ವಾಮೀಜಿ, ಸಕಾಲದಲ್ಲಿ ಸಹಾಯಕ್ಕೊದಗಿದ ಆ ವೃದ್ಧ ಸಾಧುವಿಗೆ ಕೃತಜ್ಞತೆಯಿಂದ ನಮಿಸಿದರು. ಆ ಸಾಧುವೂ ಮುಗುಳ್ನಗುತ್ತ ಪ್ರತಿ ನಮಸ್ಕಾರ ಮಾಡಿ ಹೊರಟು ಹೋದರು.

ಈ ಘಟನೆಯಿಂದ ಸ್ವಾಮೀಜಿ ಒಂದು ಅದ್ಭುತವಾದ ಪಾಠವನ್ನು ಕಲಿತರು. ಮುಂದೆ ಅವರು ನ್ಯೂಯಾರ್ಕ್ನಲ್ಲಿ ಭಾಷಣ ಮಾಡುವಾಗ ಈ ಘಟನೆಯನ್ನು ಉದಾಹರಿಸುತ್ತ ಹೇಳುತ್ತಾರೆ: “ನಮ್ಮ ಇಡೀ ಜೀವನಕ್ಕೆ ಅನ್ವಯಿಸುವ ದೊಡ್ಡ ಪಾಠ ಇದು–ಕಷ್ಟಗಳನ್ನು ಎದುರಿಸಿ; ಧೈರ್ಯ ವಾಗಿ ಎದುರಿಸಿ. ಭಯಂಕರ ಕಷ್ಟಕಾರ್ಪಣ್ಯಗಳನ್ನು ಕಂಡು ಹೆದರಿ ಓಡಿಹೋಗುವ ಬದಲು ಎದುರಿಸಿ ನಿಂತದ್ದೇ ಆದರೆ, ಅವು ಆ ಕೋತಿಗಳಂತೆಯೇ ಪಲಾಯನ ಮಾಡುತ್ತವೆ. ನಾವು ಬಂಧಮುಕ್ತರಾಗಬೇಕು, ಚಿರಸ್ವಾತಂತ್ರ್ಯ ಗಳಿಸಬೇಕು ಎಂದಿದ್ದರೆ–ಅದು ಪ್ರಕೃತಿಯನ್ನು ವಶ ಮಾಡಿಕೊಳ್ಳುವುದರ ಮೂಲಕವೇ ಹೊರತು ಪ್ರಕೃತಿಗೆ ಶರಣಾಗುವುದರಿಂದಲ್ಲ, ಓಡಿಹೋಗು ವುದರಿಂದಲ್ಲ. ಹೇಡಿಗಳು ಎಂದೆಂದಿಗೂ ಜಯಶಾಲಿಗಳಾಗಲಾರರು. ನಮ್ಮಲ್ಲಿರುವ ಭಯ- ಅಜ್ಞಾನಗಳೂ, ನಮ್ಮನ್ನು ಕಿತ್ತುತಿನ್ನುವ ಕಷ್ಟಕಾರ್ಪಣ್ಯಗಳೂ ತೊಲಗಬೇಕಾದರೆ, ನಾವು ಅವು ಗಳನ್ನು ಎದುರಿಸಿ ಹೋರಾಡಬೇಕು.”

ವಾರಾಣಸಿಯಲ್ಲಿ ಸ್ವಾಮೀಜಿ ದ್ವಾರಕಾದಾಸರೆಂಬವರ ಆಶ್ರಮದಲ್ಲಿ ಉಳಿದುಕೊಂಡಿದ್ದರು. ಈ ಆಶ್ರಮಕ್ಕೆ ಅನೇಕ ಸಾಧುಸಂನ್ಯಾಸಿಗಳು, ಪಂಡಿತರು ಬಂದುಹೋಗುತ್ತಿದ್ದು, ಸ್ವಾಮೀಜಿಗೆ ಇವರೆಲ್ಲರ ಪರಿಚಯವಾಯಿತು. ಭೂದೇವ ಮುಖ್ಯೋಪಾಧ್ಯಾಯ ಎಂಬವರು ಅವರ ಲ್ಲೊಬ್ಬರು. ಇವರು ಮಹಾ ಪಂಡಿತರು ಹಾಗೂ ಬಂಗಾಳೀ ಲೇಖಕರು. ಸ್ವಾಮೀಜಿ ಈ ಪಂಡಿತ ರೊಡನೆ ಹಿಂದೂಧರ್ಮದ ಆದರ್ಶಗಳ ಕುರಿತಾಗಿ ಸುಧೀರ್ಘ ಸಂಭಾಷಣೆಗಳನ್ನು ನಡೆಸಿದರು. ಬಳಿಕ ಆ ಪಂಡಿತರು ಸ್ವಾಮೀಜಿಯ ಸಂಬಂಧವಾಗಿ ದ್ವಾರಕಾದಾಸರ ಹತ್ತಿರ ಹೇಳುತ್ತಾರೆ: “ಅದ್ಭುತ! ಇಷ್ಟೊಂದು ಸಣ್ಣ ವಯಸ್ಸಿನಲ್ಲಿ ಎಂಥ ವಿಶಾಲ ಅನುಭವ, ಎಂಥ ತೀಕ್ಷ್ಣ ಅಂತರ್ದೃಷ್ಟಿ! ಮುಂದೆ ಇವರೊಬ್ಬ ಮಹಾನ್ ವ್ಯಕ್ತಿಯಾಗಿ ಬೆಳಗುವುದರಲ್ಲಿ ನನಗೆ ಸಂದೇಹವೇ ಇಲ್ಲ!”

ಸ್ವಾಮೀಜಿ ಕಾಶಿಯಲ್ಲಿದ್ದಾಗ ಮಹಾ ಸಂತರಾದ ತ್ರೈಲಿಂಗಸ್ವಾಮಿಗಳನ್ನು ಸಂದರ್ಶಿಸಿದರು. ತ್ರೈಲಿಂಗಸ್ವಾಮಿಗಳು ಅಲ್ಲಿನ ಒಂದು ಶಿವದೇವಾಲಯದಲ್ಲಿ ವಾಸವಾಗಿದ್ದರು. ಸದಾ ಸಮಾಧಿ ಸ್ಥಿತಿಯಲ್ಲೇ ಇರುತ್ತಿದ್ದವರು ಅವರು. ಸಮಾಧಿಯಿಂದಿಳಿದು ಬಂದಾಗ ಯಾರಾದರೂ ಒಂದಿಷ್ಟು ಆಹಾರವನ್ನು ತಂದುಕೊಟ್ಟರೆ ತೆಗೆದುಕೊಳ್ಳುತ್ತಿದ್ದರು; ಇಲ್ಲದಿದ್ದರೆ ಉಪವಾಸ. ತಮ್ಮ ಇಳಿ ವಯಸ್ಸಿನಲ್ಲಿ ಮಾತ್ರ ಕೆಲವೊಮ್ಮೆ ಸಮಾಧಿಸ್ಥಿತಿಯಿಂದ ಇಳಿದುಬಂದು, ದರ್ಶನಾರ್ಥಿಗಳ ಪ್ರಶ್ನೆ ಗಳಿಗೆ ಬರವಣಿಗೆಯ ಮೂಲಕ ಉತ್ತರ ಕೊಡುತ್ತಿದ್ದರು. ಶ್ರೀರಾಮಕೃಷ್ಣರೂ ಹಿಂದೆ ಕಾಶಿಗೆ ಬಂದಾಗ ಅವರನ್ನು ಭೇಟಿ ಮಾಡಿದ್ದರು. 

ಇನ್ನೊಂದು ದಿನ ಸ್ವಾಮೀಜಿ, ತಪಸ್ವಿಗಳೆಂದೂ ವಿದ್ವಾಂಸರೆಂದೂ ಪ್ರಸಿದ್ಧರಾದ ಸ್ವಾಮಿ ಭಾಸ್ಕರಾನಂದರನ್ನು ಭೇಟಿ ಮಾಡಿದರು. ಇವರು ಮೈಮೇಲೆ ಬಟ್ಟೆಯನ್ನೂ ಧರಿಸುತ್ತಿರಲಿಲ್ಲ. ಸ್ವಾಮೀಜಿ ಇವರೊಂದಿಗೆ ಸಂಭಾಷಣೆ ನಡೆಸು ತ್ತಿದ್ದಾಗ ಕಾಮಕಾಂಚನ ತ್ಯಾಗದ ವಿಷಯ ಬಂದಿತು. ಶ್ರೀರಾಮಕೃಷ್ಣರು ಹೇಳುತ್ತಿದ್ದರು– ‘ಭಗವಂತನ ಸಾಕ್ಷಾತ್ಕಾರವಾಗಬೇಕಾದರೆ ಕಾಮ ಕಾಂಚನಗಳ ಮೇಲಿನ ಆಸಕ್ತಿಯನ್ನು ಸಂಪೂರ್ಣ ಜಯಿಸಲೇಬೇಕಾಗುತ್ತದೆ’ ಎಂದು. ಮತ್ತು ಅದೇ ಅವರ ಬೋಧನೆಯ ಪಲ್ಲವಿ. ಆದರೆ ಭಾಸ್ಕರಾ ನಂದರು ಒಂದು ಬಗೆಯ ಅಧಿಕಾರವಾಣಿ ಯಿಂದ, “ಕಾಮಕಾಂಚನಗಳನ್ನು ಸಂಪೂರ್ಣವಾಗಿ ಗೆಲ್ಲಲು ಯಾರಿಂದಲೂ ಸಾಧ್ಯವಿಲ್ಲ” ಎಂದು ಬಿಟ್ಟರು. ಸ್ವಾಮೀಜಿ ಆಶ್ಚರ್ಯದಿಂದ ನುಡಿದರು, “ಸ್ವಾಮಿ! ಏನು ಹೇಳುತ್ತಿದ್ದೀರಿ! ಕಾಮ ಕಾಂಚನಾಸಕ್ತಿಯನ್ನು ಸಂಪೂರ್ಣ ಹೋಗಲಾಡಿಸಿಕೊಂಡವರು ಎಷ್ಟೋ ಜನ ಇದ್ದೇ ಇದ್ದಾರೆ. ನಿಜವಾದ ಸಂನ್ಯಾಸಜೀವನದ ಆಧಾರವೇ ಕಾಮಕಾಂಚನತ್ಯಾಗವಲ್ಲವೆ! ಅಷ್ಟೇ ಅಲ್ಲ, ಕಡೆಯ ಪಕ್ಷ ಅಂತಹ ಒಬ್ಬರನ್ನಾದರೂ ನಾನು ಕಣ್ಣಾರೆ ನೋಡಿದ್ದೇನೆ.” ಆಗ ಭಾಸ್ಕರಾನಂದರು ಮುಗುಳ್ನಗುತ್ತ, “ನೀವಿನ್ನೂ ಮಗು. ನಿಮಗವೆಲ್ಲ ಏನು ಗೊತ್ತಾಗುತ್ತದೆ!” ಎಂದರು. ಇಂತಹ ಅಸಡ್ಡೆಯ ಮಾತನ್ನು ಕೇಳಿ ಸ್ವಾಮೀಜಿಯ ರಕ್ತ ಕುದಿಯಿತು. ‘ನಾನು ಶ್ರೀರಾಮಕೃಷ್ಣರ ಜೀವನವನ್ನು ಕಣ್ಣಾರೆ ಕಂಡವನು. ಅವರ ಪರಮಪರಿಶುದ್ಧ ಪವಿತ್ರ ವ್ಯಕ್ತಿತ್ವವನ್ನು ವರ್ಷಗಟ್ಟಲೆ ಕಂಡು ಅಧ್ಯಯಿಸಿದವನು. ಅಂಥವರ ಬಗ್ಗೆಯೂ ಇವರು ಇಂತಹ ಹಗುರವಾದ ಮಾತುಗಳ ನ್ನಾಡುತ್ತಾರಲ್ಲ! ಸಂನ್ಯಾಸ ಜೀವನದ ತಳಹದಿಯನ್ನೇ ಪ್ರಶ್ನಿಸುತ್ತಿದ್ದಾರಲ್ಲ! ಇವರು ಶ್ರೀರಾಮ ಕೃಷ್ಣರ ಹಾಗೂ ಇನ್ನೆಷ್ಟೋ ಶ್ರೇಷ್ಠ ಸಂನ್ಯಾಸಿಗಳ ತ್ಯಾಗ-ವೈರಾಗ್ಯ ನಿಷ್ಠೆಯನ್ನೇ ಅಲ್ಲಗಳೆಯುತ್ತಿ ದ್ದಾರಲ್ಲ’ ಎಂದು ಅವರಿಗೆ ರೇಗಿತು. ತಕ್ಷಣ ಅವರು ತಮ್ಮ ಪ್ರಖರ ವಾಗ್ವೈಖರಿಯಿಂದ, ತೇಜಸ್ವಿಯಾದ ವಿಚಾರಧಾರೆಯಿಂದ, ತಮ್ಮ ಸ್ವಂತ ಅನುಭವಗಳ ಆಧಾರದ ಮೇಲೆ ಭಾಸ್ಕರಾ ನಂದರ ವಾದವನ್ನು ಖಂಡಿಸಿದರು. ಭಾಸ್ಕರಾನಂದರು ಮಹಾಸಂತರೆಂದು ಹೆಸರಾದವರು, ವಯೋವೃದ್ಧರು. ಅವರಿಗೆ ಹೋಲಿಸಿದರೆ ಸ್ವಾಮೀಜಿ ಬಹಳ ಕಿರಿಯರೇ ಸರಿ. ಆದರೆ ಬೆಂಕಿಯ ಕಿಡಿ, ಗಾತ್ರದಲ್ಲಿ ಕಿರಿದಾದರೆ ಕಾವಿನಲ್ಲಿ ಕಿರಿದೆ? ಸ್ವಾಮೀಜಿ ತಮ್ಮ ವಿಚಾರಧಾರೆಯನ್ನು ಮಂಡಿಸುತ್ತಿದ್ದಂತೆ ಸ್ವಾಮಿ ಭಾಸ್ಕರಾನಂದರೂ ಅವರ ಶಿಷ್ಯರೂ ಸ್ತಂಭೀಭೂತರಾಗಿ ಕುಳಿತರು. ಭಾಸ್ಕರಾನಂದರು ತಮ್ಮ ಸುತ್ತ ಕುಳಿತವರತ್ತ ತಿರುಗಿ, “ಈತನ ನಾಲಿಗೆಯಲ್ಲಿ ಸ್ವಯಂ ವಾಗ್ದೇವಿಯೇ ನೆಲೆಸಿದ್ದಾಳೆ! ಈತನ ಬುದ್ಧಿಶಕ್ತಿ ಪ್ರಖರ ಜ್ಯೋತಿಯಂತಿದೆ!” ಎಂದು ಉದ್ಗಾರ ಮಾಡಿದರು. ಆದರೆ ಸ್ವಾಮೀಜಿ ಆ ಮೆಚ್ಚಿಗೆಯ ಮಾತನ್ನು ಲೆಕ್ಕಿಸದೆ, ಜುಗುಪ್ಸೆಯಿಂದಲೇ ಅಲ್ಲಿಂದೆದ್ದು ಹೊರಟುಹೋದರು. 

ಒಂದು ದಿನ ಅವರು ಕಾಶಿಗೆ ಸಮೀಪದ ಸಾರಾನಾಥಕ್ಕೂ ಭೇಟಿ ನೀಡಿದರು. ಭಗವಾನ್ ಬುದ್ಧನು ತನ್ನ ಸಂದೇಶಗಳನ್ನು ಮೊಟ್ಟಮೊದಲ ಸಲ ಸಾರಿದ ಇಲ್ಲಿನ ಸಾರಂಗವನವನ್ನು ವೀಕ್ಷಿಸಿದರು.

ಪವಿತ್ರ ವಾರಾಣಸಿಯಲ್ಲಿ ಒಂದು ವಾರವನ್ನು ಕಳೆದ ಸ್ವಾಮೀಜಿ ನೇರವಾಗಿ ಬಾರಾನಗೋರಿಗೆ ಹಿಂದಿರುಗಿದರು. ಇತ್ತ ವಾರಾಣಸಿಯಲ್ಲಿ ಅವರ ಸಂಪರ್ಕಕ್ಕೆ ಬಂದಿದ್ದ ಅನೇಕ ಸಾಧುಗಳೂ ಭಕ್ತರೂ ಸ್ವಾಮೀಜಿ ಇದ್ದಕ್ಕಿದ್ದಂತೆ ಎಲ್ಲಿಗೆ ಹೋಗಿರಬಹುದು ಎಂದು ಆಶ್ಚರ್ಯಪಟ್ಟರು. ಮತ್ತೆ ತಮ್ಮ ಗುರುಭಾಯಿಗಳೊಡನೆ ಸಾಧನೆ-ಅಧ್ಯಯನ ಹಾಗೂ ವಿಚಾರವಿನಿಮಯ ನಡೆಸುವ ಉದ್ದೇಶದಿಂದ ಸ್ವಾಮೀಜಿ ಮಠಕ್ಕೆ ಹಿಂದಿರುಗಿದ್ದರು. ಕೆಲವೊಮ್ಮೆ, ಸೇವಾಸಂಸ್ಥೆಯೊಂದರ ಮೂಲಕ ದೀನದರಿದ್ರರ ಉದ್ಧಾರಕ್ಕಾಗಿ ಶ್ರಮಿಸುವ ದೃಶ್ಯ ಅವರ ಮನಃಪಟಲದ ಮೇಲೆ ಅಸ್ಪಷ್ಟವಾಗಿ ಹಾದುಹೋಗುತ್ತಿತ್ತು. ಮನುಷ್ಯನನ್ನು ಭಗವಂತನ ಅಂಶ ಎಂದು ಭಾವಿಸಿ ಸೇವೆ ಮಾಡುವ ಭಾವನೆ ಅವರನ್ನು ಸಂಪೂರ್ಣವಾಗಿ ಆವರಿಸಿಬಿಡುತ್ತಿತ್ತು. ವೇದಾಂತದ ತತ್ತ್ವಗಳನ್ನು ನಿಜಜೀವನದಲ್ಲಿ ಕಾರ್ಯಗತಗೊಳಿಸಲು ಇದಕ್ಕಿಂತ ಉತ್ತಮವಾದ ಮಾರ್ಗ ಯಾವುದು! ಧರ್ಮದ ಬಗೆಗಿನ ಈ ಹೊಸ ಕಲ್ಪನೆಯನ್ನು ಮುಂದಿಟ್ಟು ತಮ್ಮ ಸೋದರಸಂನ್ಯಾಸಿಗಳನ್ನು ಉತ್ತೇಜಿಸುವ ಪ್ರಯತ್ನ ಮಾಡುತ್ತಿದ್ದರು. ಅವರೆಲ್ಲ ಆಗಷ್ಟೇ ಸಂನ್ಯಾಸ ಸ್ವೀಕರಿಸಿದ್ದ ಯುವಕರು. ಅಲ್ಲದೆ, ಸಮಾಜದ ಮೇಲೆ ಜಾತಿಪದ್ಧತಿಯ ಹಾಗೂ ಸಂಪ್ರದಾಯಗಳ ಹಿಡಿತ ಇನ್ನೂ ಬಲವಾಗಿದ್ದ ದಿನಗಳು ಅವು. ಅಂತಹ ಸಂದರ್ಭದಲ್ಲಿ ಸ್ವಾಮೀಜಿ, ಹರಿಜನರ ಕೇರಿಗಳಿಗೆ ಹೋಗಿ ಅವರಿಗೆ ಬೋಧನೆ ಮಾಡುವಂತೆ ಅವರ ಕಣ್ದೆರೆಸುವಂತೆ ತಮ್ಮ ಗುರುಭಾಯಿಗಳನ್ನು ಒತ್ತಾಯಿ ಸುತ್ತಿದ್ದರು. ಆದರೆ ಈ ಗುರುಭಾಯಿಗಳು ಧರ್ಮಬೋಧನೆಯೆಂದರೆ ಹಿಂದೇಟು ಹಾಕುತ್ತಿದ್ದರು. ಅವರ ಮುಂದಿದ್ದ ಮೊದಲ ಗುರಿಯೆಂದರೆ ಭಗವತ್ಸಾಕ್ಷಾತ್ಕಾರ. ಅದು ಸಿದ್ಧಿಸಿದಮೇಲೆ ಧರ್ಮ ಪ್ರಸಾರದ ಪ್ರಶ್ನೆ ಎಂಬುದು ಅವರ ನಿಲುವು. ಭಗವತ್ಸಾಕ್ಷಾತ್ಕಾರದ ಮೂಲಕ ಧರ್ಮಪ್ರಸಾರಕ್ಕೆ ಬೇಕಾದ ಅರ್ಹತೆಯನ್ನು ಪಡೆದುಕೊಳ್ಳಬೇಕು ಎಂಬ ವಾದವನ್ನು ಸ್ವಾಮೀಜಿಯೂ ಈ ಹಿಂದೆ ಅನುಮೋದಿಸುತ್ತಿದ್ದರು. ಆದರೂ ಕೆಲವೊಮ್ಮೆ ಧರ್ಮಪ್ರಸಾರದ ಉತ್ಸಾಹ ಅವರ ಮನಸ್ಸಿನಲ್ಲಿ ಚಿಮ್ಮುತ್ತಿತ್ತು. ಇಂತಹ ಭಾವ ಜಾಗೃತವಾದ ಒಂದು ಸಂದರ್ಭದಲ್ಲಿ, ಅವರು, ಧರ್ಮಪ್ರಸಾರ- ಉಪದೇಶಗಳನ್ನೆಲ್ಲ ವಿರೋಧಿಸುತ್ತಿದ್ದ ಒಬ್ಬ ಸೋದರಸಂನ್ಯಾಸಿಗೆ ಹೇಳುತ್ತಾರೆ: “ನೋಡು, ಪ್ರತಿಯೊಬ್ಬನೂ ತನಗೆ ತಿಳಿದೋ ತಿಳಿಯದೆಯೋ ಬೋಧನೆಯನ್ನು ಮಾಡುತ್ತಲೇ ಇರುತ್ತಾನೆ. ಇತರರು ಯಾವುದನ್ನು ತಮಗರಿವಿಲ್ಲದಂತೆಯೇ, ಅನುದ್ದೇಶಪೂರ್ವಕವಾಗಿಯೇ ಮಾಡು ತ್ತಾರೋ, ಅದನ್ನೇ ನಾನು ಉದ್ದೇಶಪೂರ್ವಕವಾಗಿ, ಉತ್ಸಾಹದಿಂದ ಮಾಡುತ್ತೇನೆ. ಯಾವ ಅಡೆತಡೆಯನ್ನೂ ಲೆಕ್ಕಿಸುವವನಲ್ಲ ನಾನು. ಕಡೆಗೆ, ನನ್ನ ಸೋದರಸಂನ್ಯಾಸಿಗಳಾದ ನೀವು ಕೂಡ ಅದನ್ನು ಒಪ್ಪದಿದ್ದರೂ ಸರಿಯೆ. ನಾನು ಅಂತ್ಯಜರ ಕೊಳೆಗೇರಿಗಳಿಗೆ ಹೋಗಿ ಬೋಧಿಸುತ್ತೇನೆ. ಬೋಧನೆ ಎಂದರೇನರ್ಥ? ನಮ್ಮ ಭಾವನೆಗಳನ್ನು ವ್ಯಕ್ತಪಡಿಸುವುದೇ ಬೋಧನೆ. ತ್ರೈಲಿಂಗ ಸ್ವಾಮಿಗಳು ಮಾತನ್ನೇ ಆಡದೆ ಯಾವಾಗಲೂ ಮೌನವಾಗಿರುತ್ತಾರೆಂಬ ಮಾತ್ರಕ್ಕೆ ಅವರು ಬೋಧನೆಯನ್ನೇ ಮಾಡುವುದಿಲ್ಲ ಎಂದು ಭಾವಿಸಿದಿರೇನು? ಅವರ ಮೌನವೇ ಒಂದು ಬೋಧನೆ!”

ಮುಂದೆ ಸ್ವಾಮೀಜಿ ಮಾಡಿದ್ದಾದರೂ ಅದನ್ನೇ–ಅಲ್ಲಿಯವರೆಗೆ ಯಾವುದನ್ನು ಇತರ ಸಾಧು ಸಂತರು ಮೆಲುದನಿಯಲ್ಲಿ ವೈಯಕ್ತಿಕವಾಗಿ ಬೋಧಿಸಿದ್ದರೋ ಅದನ್ನೇ ಅವರು ಜನಮನ ಮುಟ್ಟುವಂತೆ ತಮ್ಮ ಧೀರಗಂಭೀರ ವಾಣಿಯಿಂದ ಘೋಷಿಸಿದರು. ಅವರ ಈ ಕಾರ್ಯವು ಬಾರಾನಗೋರ್ ಮಠದಲ್ಲಿ ಪ್ರಾರಂಭವಾಯಿತು. ಅವರ ಮೊಟ್ಟಮೊದಲ ಶ್ರೋತೃಗಳೆಂದರೆ ಅವರ ಸೋದರಸಂನ್ಯಾಸಿಗಳು ಹಾಗೂ ಕೆಲವು ಮಂದಿ ಗೃಹಸ್ಥಭಕ್ತರು.

ಆದರೆ ಸ್ವಾಮೀಜಿ ಬಾರಾನಗೋರ್ ಮಠದಲ್ಲಿ ಹೆಚ್ಚು ದಿನ ಉಳಿದುಕೊಳ್ಳಲಿಲ್ಲ. ಪರಿವ್ರಾಜಕ ಜೀವನದ ಆಕರ್ಷಣೆ ಅವರನ್ನು ಮತ್ತೆ ಸೆಳೆಯಲಾರಂಭಿಸಿತ್ತು. ಆದ್ದರಿಂದ ಮತ್ತೆ ಉತ್ತರಭಾರತದ ಪುಣ್ಯಕ್ಷೇತ್ರಗಳಿಗೆ ಯಾತ್ರೆ ಹೊರಟು ವಾರಾಣಸಿಗೆ ಬಂದರು. ಈ ಬಾರಿ ಅವರಿಗೆ ಪ್ರಮದದಾಸ ಮಿತ್ರರೆಂಬ ಪ್ರಸಿದ್ಧ ಸಂಸ್ಕೃತ ವಿದ್ವಾಂಸರ ಭೇಟಿಯಾಯಿತು. ಇವರಿಗೆ ಈಗಾಗಲೇ ಸ್ವಾಮಿ ಅಖಂಡಾನಂದರ ಪರಿಚಯವಾಗಿದ್ದು, ಅವರಿಂದ ಸ್ವಾಮೀಜಿಯ ಬಗ್ಗೆ ಕೇಳಿ ತಿಳಿದಿದ್ದರು. ಈಗ ಸ್ವಾಮೀಜಿಯವರನ್ನು ಪ್ರತ್ಯಕ್ಷ ಕಂಡಾಗ ಇಬ್ಬರ ನಡುವೆ ಗಾಢ ಸ್ನೇಹ ಬೆಳೆಯಿತು.

ವಾರಾಣಸಿಯಲ್ಲಿ ಕೆಲದಿನಗಳಿದ್ದು, ಅಲ್ಲಿಂದ ಸ್ವಾಮೀಜಿ ಅಯೋಧ್ಯೆಯ ಕಡೆಗೆ ಹೊರಟರು. ಅಯೋಧ್ಯೆ ಶ್ರೀರಾಮಚಂದ್ರನ ರಾಜಧಾನಿಯಾಗಿ ಮೆರೆದ ನಗರ. ಸ್ವಾಮೀಜಿ ತಮ್ಮ ಬಾಲ್ಯ ದಿಂದಲೇ ರಾಮಾಯಣ ಮಹಾಕಥೆಯನ್ನು ಕೇಳಿ ಪ್ರಭಾವಿತರಾದವರು. ಶ್ರೀರಾಮಕೃಷ್ಣರಿಂದ ರಾಮಮಂತ್ರದೀಕ್ಷೆಯನ್ನು ಪಡೆದವರು. ಈಗ ಅಯೋಧ್ಯೆಯನ್ನು ಕಂಡಾಗ ರಾಮಾಯಣದ ದೃಶ್ಯಗಳೂ ಅಯೋಧ್ಯೆಯ ಗತವೈಭವವೂ ಅವರ ಕಣ್ಮುಂದೆ ಹಾದುಹೋದವು. ಅಲ್ಲಿಂದ ಅವರು ಲಕ್ನೋಗೆ ಬಂದರು. ಇಲ್ಲಿ ನವಾಬರ ಕಾಲದ ಸಂಸ್ಕೃತಿಯ ಅವಶೇಷಗಳನ್ನು ಕಂಡು ಅವುಗಳ ಸೌಂದರ್ಯಕ್ಕೆ ಬೆರಗಾದರು. ಲಕ್ನೋದಿಂದ ಅವರ ಪ್ರಯಾಣ ಆಗ್ರಾಕ್ಕೆ. ಮೊಘಲ್ ಸಾಮ್ರಾಜ್ಯದ ನೆನಪುಗಳನ್ನು ಜೀವಂತವಾಗಿಸಿರುವ ನಗರ ಆಗ್ರಾ. ಇತಿಹಾಸದಲ್ಲಿ ಪರಿಣತರಾದ ಸ್ವಾಮೀಜಿಗೆ ಆಗ್ರಾದ ರಸ್ತೆಗಳಲ್ಲಿ ನಡೆದಾಡುತ್ತಿದ್ದಂತೆ ಮೊಘಲ್ ಸಾಮ್ರಾಜ್ಯದ ದಿನಗಳ ಚಿತ್ರ ಕಣ್ಮುಂದೆ ಕಟ್ಟಿದಂತಾಯಿತು. ಇಲ್ಲಿ ಭಾರತೀಯ ಕಲಾಕಾರರ ಕೌಶಲವನ್ನು ನೋಡುತ್ತ ಸ್ವಾಮೀಜಿ ಆಶ್ಚರ್ಯಚಿಕಿತರಾದರು. ಅದರಲ್ಲೂ ಜಗತ್ತಿನ ವಿಸ್ಮಯಗಳಲ್ಲೊಂದಾದ ತಾಜ್ ಮಹಲಿನ ಸೌಂದರ್ಯ ಅವರ ಮನಸ್ಸನ್ನು ಸೆರೆಹಿಡಿದುಬಿಟ್ಟಿತು. ತಾಜ್ಮಹಲನ್ನು ಅವರು ಹಲವು ಸಲ ಬಂದು ನೋಡಿದರು. ಬೇರೆಬೇರೆ ದೃಷ್ಟಿಕೋನಗಳಿಂದ ನೋಡಿದರು. ಮುಂಜಾನೆಯ ಮುಂಬೆಳಕಿನಲ್ಲಿ, ಮಧ್ಯಾಹ್ನದ ಬಿಸಿಲಿನಲ್ಲಿ, ಸೂರ್ಯಾಸ್ತದ ಸಂಧ್ಯಾರಾಗದಲ್ಲಿ, ಸುಂದರವಾದ ಬೆಳದಿಂಗಳಿನಲ್ಲಿ–ಹೀಗೆ ಬೇರೆ ಬೇರೆ ಸಮಯಗಳಲ್ಲಿ, ಬೇರೆಬೇರೆ ಭಾವಗಳಲ್ಲಿ ನೋಡಿದರು. ಅಲ್ಲದೆ, ಅದು ತಮ್ಮ ಪ್ರಿಯ ಜನ್ಮಭೂಮಿಯ ಹೆಮ್ಮೆಯ ವಸ್ತು ಎಂಬ ದೃಷ್ಟಿಯಿಂದ ನೋಡಿ ಆನಂದಿಸಿದರು. ತಾಜ್ಮಹಲಿನ ಕುರಿತಾಗಿ ಅವರೊಮ್ಮೆ ಹೇಳುತ್ತಾರೆ: “ಆ ಭವ್ಯ ಭವನದ ಪ್ರತಿಯೊಂದು ಅಂಗುಲವೂ ಒಂದೊಂದು ದಿನವಿಡೀ ಗಮನಿಸಿ ನೋಡುವಷ್ಟು ಯೋಗ್ಯವಾಗಿದೆ. ನಿಜಕ್ಕೂ ಆ ಭವನದ ಸಮಗ್ರ ಪರಿಚಯ ಮಾಡಿಕೊಳ್ಳಬೇಕಾದರೆ ಕಡಿಮೆಯೆಂದರೆ ಆರು ತಿಂಗಳಾದರೂ ಬೇಕು.”

೧೮೮೮ರ ಆಗಸ್ಟ್ ವೇಳೆಗೆ ಸ್ವಾಮೀಜಿ ಆಗ್ರಾದಿಂದ ಹೊರಟು ಬೃಂದಾವನ ಕ್ಷೇತ್ರದೆಡೆಗೆ ಸಾಗಿದರು. ಹೆಚ್ಚಿನ ಪ್ರಯಾಣವೆಲ್ಲ ರೈಲಿನಲ್ಲಿ. ಆದರೆ ಕಡೆಯ ಮೂವತ್ತು ಮೈಲಿಗಳನ್ನು ಕಾಲ್ನಡಿಗೆಯಲ್ಲೇ ಕ್ರಮಿಸಿದರು. ಬೃಂದಾವನಕ್ಕೆ ಸುಮಾರು ಎರಡು ಮೈಲಿ ಇದೆಯೆನ್ನುವಾಗ, ದಾರಿಯ ಪಕ್ಕದಲ್ಲಿ ಒಬ್ಬ ಮನುಷ್ಯ ಆರಾಮವಾಗಿ ಕುಳಿತುಕೊಂಡು ಹುಕ್ಕಾ ಸೇದುತ್ತಿದ್ದುದು ಅವರ ಕಣ್ಣಿಗೆ ಬಿತ್ತು. ನಡೆದು ನಡೆದು ಆಯಾಸಗೊಂಡಿದ್ದ ಅವರಿಗೆ, ಸ್ವಲ್ಪ ಹುಕ್ಕಾ ಸೇದಿ ದಣಿವಾರಿಸಿಕೊಳ್ಳುವ ಮನಸ್ಸಾಯಿತು. ಹತ್ತಾರು ವರ್ಷಗಳ ಅಭ್ಯಾಸವಲ್ಲವೆ! ಸರಿ; ಆ ಮನುಷ್ಯ ನನ್ನು ಮಾತಾಡಿಸಿ, ಒಂದರೆಡು ಸಲ ‘ದಂ’ ಎಳೆಯಲು ಕೊಡುವಂತೆ ಕೇಳಿಕೊಂಡರು. ಆದರೆ ಅವನು ತುಂಬ ಅಧೈರ್ಯಗೊಂಡು, “ಇಲ್ಲ ಸ್ವಾಮೀಜಿ! ನಾನೊಬ್ಬ ಜಾಡಮಾಲಿ. ನನ್ನ ಹುಕ್ಕ ವನ್ನು ಉಪಯೋಗಿಸಿದರೆ ನಿಮಗೆ ಮೈಲಿಗೆಯಾಗಿಬಿಡುತ್ತದೆ” ಎಂದ. ಇದನ್ನು ಕೇಳಿ ಸ್ವಾಮೀಜಿಗೆ ಹೌದೆನ್ನಿಸಿತು. ಅವನ ಕೈಯ ಹುಕ್ಕಾ ಸೇದಲು ಹಿಂಜರಿದು, ಹಾಗೇ ಮುಂದೆ ಸಾಗಿದರು. ಆದರೆ ಸ್ವಲ್ಪದೂರ ಹೋಗುವಷ್ಟರಲ್ಲಿ ಅವರ ಮನಸ್ಸಿನಲ್ಲೊಂದು ಭಾವನೆ ಉದ್ಭವಿಸಿತು: ‘ಏನಾ ಶ್ಚರ್ಯ! ನಾನು ಸಂನ್ಯಾಸ ವ್ರತಧಾರಣೆ ಮಾಡಿದವನು; ಜಾತಿ-ಕುಲಗಳ ಅಭಿಮಾನವನ್ನು ತೊರೆ ದವನು. ಅಂಥದರಲ್ಲಿ, ಆ ಮನುಷ್ಯ ತಾನೊಬ್ಬ ಜಾಡಮಾಲಿ ಎಂದಾಗ ನನ್ನಲ್ಲಿ ಈ ಜಾತಿಬುದ್ಧಿ ಜಾಗೃತವಾಯಿತಲ್ಲ! ಛೆ! ಇದು ಯುಗಯುಗಗಳಿಂದಲೂ ಅಂಟಿಕೊಂಡು ಬಂದ ಅಭ್ಯಾಸದ ಪರಿಣಾಮ.’ ತಕ್ಷಣ ಅವರು ಹಿಂದಿರುಗಿ, ಅವನ ಬಳಿಗೆ ಬಂದರು. ಆತನ ಪಕ್ಕದಲ್ಲೇ ಕುಳಿತು, “ಅಣ್ಣ, ಪರವಾಗಿಲ್ಲ! ನನಗೂ ಸ್ವಲ್ಪ ಹುಕ್ಕ ತಯಾರಿಸಿಕೊಡು” ಎಂದು ಪ್ರೀತಿಯಿಂದ ಕೇಳಿ ಕೊಂಡರು. ಆಗ ಪುನಃ ಆತ ಸಂಕೋಚದಿಂದ ಹಿಂಜರಿಯುತ್ತ, “ಸ್ವಾಮಿ, ನೀವು ಪೂಜ್ಯರು. ನಾನಾದರೂ ಅಂತ್ಯಜ” ಎಂದು ಎಷ್ಟೋ ಹೇಳಿದ. ಆದರೆ ಸ್ವಾಮೀಜಿ ಬಿಡಲೇ ಇಲ್ಲ. ಅವನಿಂದ ಹುಕ್ಕಾ ತಯಾರಿಸಿಕೊಂಡು ಸೇದಿ, ಅವನಿಗೆ ಕೃತಜ್ಞತೆ ಹೇಳಿ ಮುಂದೆ ಸಾಗಿದರು.

ಮುಂದೊಮ್ಮೆ ಗಿರೀಶ್ಚಂದ್ರ ಘೋಷ್ ಈ ವಿಷಯವನ್ನು ಕೇಳಿದಾಗ ಅರ್ಧಹಾಸ್ಯವಾಗಿ ಹೇಳುತ್ತಾನೆ, “ಸ್ವಾಮೀಜಿ, ನಿಮಗೆ ಹುಕ್ಕಾ ಸೇದುವ ಚಟ ಇದೆಯಲ್ಲವೆ? ಆದ್ದರಿಂದ ನಿಮಗೆ ಆ ಭಂಗಿಯ ಕೈಯಿಂದಲಾದರೂ ಅದನ್ನು ತೆಗೆದುಕೊಂಡು ಸೇದುವಷ್ಟು ಆಸೆಯಾಗಿಬಿಟ್ಟಿತು, ಅಷ್ಟೆ!” ತಮಗಿಂತ ಹಿರಿಯನೂ ಆತ್ಮೀಯನೂ ಆದ ಗಿರೀಶನ ಈ ತಮಾಷೆಯ ಮಾತಿಗೆ ಸ್ವಾಮೀಜಿ ಉತ್ತರಿಸುತ್ತಾರೆ, “ಇಲ್ಲ ಜಿ. ಸಿ., (ಜಿ. ಸಿ. ಎನ್ನುವುದು ಗಿರೀಶ್ಚಂದ್ರ ಎಂಬುದರ ಹ್ರಸ್ವರೂಪ) ನಿಜಕ್ಕೂ ನಾನು ನನ್ನನ್ನು ಪರೀಕ್ಷೆಮಾಡಿ ನೋಡಿಕೊಳ್ಳುವುದಕ್ಕೋಸ್ಕರವೇ ಹಾಗೆ ಮಾಡಿದೆ. ಸಂನ್ಯಾಸ ಸ್ವೀಕರಿಸಿದ ಮೇಲೆ ಪ್ರತಿಯೊಬ್ಬನೂ ತನ್ನನ್ನು ತಾನು ಪರೀಕ್ಷೆ ಮಾಡಿ ನೋಡಿಕೊಳ್ಳಬೇಕು–ತಾನು ಜಾತಿ-ಕುಲ-ಗೋತ್ರಗಳ ಬಂಧನದಿಂದ ಪಾರಾಗಿದ್ದೇನೆಯೋ ಇಲ್ಲವೋ ಎಂದು. ಸಂನ್ಯಾಸದ ಆದರ್ಶವನ್ನು ಕಟ್ಟುನಿಟ್ಟಾಗಿ ಪಾಲಿಸುವುದು ಬಹಳ ಕಷ್ಟ. ಮಾತಿಗೂ ಕೃತಿಗೂ ವ್ಯತ್ಯಾಸವಾಗದಂತೆ ನೋಡಿಕೊಳ್ಳಬೇಕು.” ಇನ್ನೊಮ್ಮೆ ಇದೇ ಘಟನೆಯನ್ನು ತಿಳಿಸುತ್ತ ಒಬ್ಬ ಶಿಷ್ಯನಿಗೆ ಹೇಳುತ್ತಾರೆ: “ಏನಪ್ಪ, ಸಂನ್ಯಾಸದ ಆದರ್ಶಗಳನ್ನು ನಿತ್ಯ ಜೀವನದಲ್ಲಿ ಪರಿಪಾಲಿಸಿಕೊಂಡು ಬರುವುದ ಸುಲಭದ ಕೆಲಸ ಅಂತಂದುಕೊಂಡೆಯಾ? ಸಂನ್ಯಾಸ ಜೀವನದಷ್ಟು ಕಷ್ಟಕರವಾದ ಮಾರ್ಗ ಇನ್ನಾವುದೂ ಇಲ್ಲ. ಅತಿ ಎತ್ತರವಾದ, ಕಡಿದಾದ ಪರ್ವತ ಶಿಖರದ ಮೇಲೆ ನಡೆದುಹೋಗುತ್ತಿರುವಾಗ ಸ್ವಲ್ಪ ಜಾರಿದರೂ ನೀನು ಬೀಳುವದೆಲ್ಲಿಗೆ ಹೇಳು? ಕೆಳಗೆ ಆಳವಾದ ಕಣಿವೆಯಲ್ಲಿ! ಯಾವಾಗ ವ್ಯಕ್ತಿಯೊಬ್ಬ ಸಂನ್ಯಾಸದ ಆದರ್ಶವ್ನು ಸ್ವೀಕರಾ ಮಾಡುತ್ತಾನೋ ಅವನು ಪ್ರತಿದಿನ, ಪ್ರತಿಕ್ಷಣ ತನ್ನನ್ನು ತಾನು ಪರೀಕ್ಷಿಸಿ ನೋಡಿಕೊಳ್ಳುತ್ತಿರಬೇಕಾಗುತ್ತದೆ–ಜಾತ್ಯಭಿಮಾನ, ಕುಲಾಭಿಮಾನವೇ ಮೊದಲಾದ ಬಂಧನಗಳಿಂದ ತಾನು ಪಾರಾಗಿದ್ದೇನೆಯೇ ಎಂದು. ಆ ಘಟನೆ ನನಗೊಂದು ಒಳ್ಳೆಯ ಪಾಠ ಕಲಿಸಿತು–ನಾವು ಯಾರನ್ನೂ ಕಡೆಗಣಿಸಬಾರದು; ಬದಲಾಗಿ ಎಲ್ಲರನ್ನೂ ಸಮಾನದೃಷ್ಟಿಯಿಂದ ಭಗವಂತನ ಮಕ್ಕಳೆಂದು ಭಾವಿಸಬೇಕು ಅಂತ.”

ಬೃಂದಾವನಕ್ಕೆ ಬಂದ ಸ್ವಾಮೀಜಿ ಅಲ್ಲಿ ಬಲರಾಮ ಬಾಬುವಿನ ಪೂರ್ವಜರು ಕಟ್ಟಿಸಿದ್ದ ಸತ್ರದಲ್ಲಿ ಇಳಿದುಕೊಂಡರು. ಬೃಂದಾವನಕ್ಕೆ ಬಂದೊಡನೆ ಅವರಲ್ಲೊಂದು ದಿವ್ಯಭಾವ ಉತ್ಪನ್ನ ವಾಗಿಬಿಟ್ಟಿತು. ತಮ್ಮ ಹೃದಯದ ದ್ವಾರ ಸಂಪೂರ್ಣ ತೆರೆದುಕೊಂಡಂತೆ ಭಾಸವಾಯಿತು. ಶ್ರೀಕೃಷ್ಣನ ದಿವ್ಯ ಆಧ್ಯಾತ್ಮಿಕ ಲೀಲೆ ನಡೆದ ಸ್ಥಳವಲ್ಲವೆ ಅದು? ಆ ಅದ್ಭುತ ಘಟನೆಗಳೆಲ್ಲ ನೆನಪಿಗೆ ಬಂದು ಅವರ ಹೃದಯ ಭಕ್ತಿಭಾವರಂಜಿತವಾಗಿಬಿಟ್ಟಿತು. ಶ್ರೀಕೃಷ್ಣನು ಲೀಲೆಯಾಡಿದ ಪ್ರತಿಯೊಂದು ಸ್ಥಳವನ್ನೂ ಸಂದರ್ಶಿಸುವ ನಿರ್ಧಾರ ಮಾಡಿದರು. ಏಕೆಂದರೆ, ಆ ಪ್ರತಿಯೊಂದು ಸ್ಥಳದ ಪರಿಸರದಲ್ಲೂ ಅವರಿಗೆ ಆ ದಿವ್ಯ ಲೀಲೆಗಳು ಈಗಲೂ ನಡೆಯುತ್ತಿವೆಯೋ ಎಂಬಷ್ಟು ಸಚೇತನವಾಗಿ ಕಾಣತೊಡಗಿತ್ತು.

ಸ್ವಾಮೀಜಿ ಗೋವರ್ಧನ ಪರ್ವತಕ್ಕೆ ಪ್ರದಕ್ಷಿಣೆ ಹಾಕಲು ಹೊರಟರು. ಇದು ಬಹು ದೊಡ್ಡ ಪ್ರದಕ್ಷಿಣೆ. ಈ ಪ್ರದಕ್ಷಿಣೆ ಮಾಡುವ ಕಾಲದಲ್ಲಿ ತಾವು ಭಿಕ್ಷೆ ಬೇಡಬಾರದು; ತಾನಾಗಿಯೇ ಏನು ಬರುತ್ತದೋ ಅದನ್ನೇ ಸ್ವೀಕರಿಸಬೇಕು ಎಂದು ತೀರ್ಮಾನಿಸಿದರು. ಮೊದಲ ದಿವಸ ಪ್ರದಕ್ಷಿಣೆ ಹೊರಟಿದ್ದಾರೆ; ಮಧ್ಯಾಹ್ನದ ಹೊತ್ತಿಗೆ ಹಸಿವು ಪ್ರಾರಂಭವಾಯಿತು. ನಡೆದಂತೆಲ್ಲ ಹಸಿವು ಇನ್ನಷ್ಟು ತೀವ್ರವಾಯಿತು. ಜೊತೆಗೆ ಮಳೆ ಬೇರೆ ಸುರಿಯುತ್ತಿದೆ! ಹಸಿವು, ಬಳಲಿಕೆ ಸೇರಿ ಬವಳಿ ಬಂದು ಬೀಳುವ ಹಾಗಾಯಿತು. ಆದರೂ ಅವಡುಗಚ್ಚಿ ಮುನ್ನಡೆದರು. ಭಿಕ್ಷೆ ಬೇಡಲಾರೆ ಎಂದು ಸಂಕಲ್ಪಮಾಡಿದ್ದರಿಂದ ಭಗವಂತನಲ್ಲಿ ಶರಣಾಗತರಾಗಿ ಸಾಗುತ್ತಿದ್ದಾರೆ. ಈ ಹೊತ್ತಿಗೆ ಹಿಂದಿ ನಿಂದ ಯಾವನೋ ಒಬ್ಬ “ಬಾಬಾಜಿ!” ಎಂದು ಕೂಗಿ ಕರೆದ ಶಬ್ದ ಕೇಳಿಸಿತು. ಆದರೆ ಸ್ವಾಮೀಜಿ ಅತ್ತ ಗಮನ ಕೊಡದೆ ಮುನ್ನಡೆಯುತ್ತಲೇ ಇದ್ದರು. ಕ್ರಮೇಣ ಆ ಧ್ವನಿ ಇನ್ನಷ್ಟು ಹತ್ತಿರ ಬಂದಿತು. ಆ ಮನುಷ್ಯ ಕೂಗುತ್ತಿದ್ದಾನೆ, “ಬಾಬಾಜಿ, ನಿಮಗೋಸ್ಕರ ಊಟ ತಂದಿದ್ದೇನೆ; ದಯವಿಟ್ಟು ಸ್ವೀಕರಿಸಬೇಕು” ಎಂದು. ನಿಜಕ್ಕೂ ಇದು ಭಗವಂತನೇ ಕಳಿಸಿದ ಭಿಕ್ಷೆ ಎಂಬುದರಲ್ಲಿ ಸಂದೇಹವೇನಿದೆ? ಆದರೂ ಸ್ವಾಮೀಜಿ ಇದನ್ನು ಪರೀಕ್ಷೆ ಮಾಡಿ ನೋಡಿಬಿಡೋಣ ಎಂದು ಓಡಲಾರಂಭಿಸಿದರು. ಆದರೆ ಆ ಮನುಷ್ಯ ಬಿಡದೆ, “ಸ್ವಲ್ಪ ತಾಳಿ, ಸ್ವಲ್ಪ ತಾಳಿ, ಬಾಬಾಜಿ” ಎಂದು ಕೂಗುತ್ತ ತಾನೂ ಓಡಿಬಂದ! ಸ್ವಾಮೀಜಿ ವಿಪರೀತ ಹಸಿದು ಬಳಲಿದ್ದಾರೆ; ಆದರೂ ಶಕ್ತಿಮೀರಿ ಓಡುತ್ತಿದ್ದಾರೆ! ಈ ಓಟದ ಪಂದ್ಯ ಒಂದು ಮೈಲಿಯಷ್ಟು ದೂರ ನಡೆಯಿತು. ಕಡೆಗೂ ಆ ಮನುಷ್ಯ ಪಟ್ಟುಬಿಡದೆ, ಸ್ವಾಮೀಜಿಯನ್ನು ಸಮೀಪಿಸಿ, ಹಿಡಿದು ನಿಲ್ಲಿಸಿ, “ದಯವಿಟ್ಟು ಊಟ ವನ್ನು ಸ್ವೀಕರಿಸಬೇಕು” ಎಂದು ಅಂಗಲಾಚಿ ಬೇಡಿಕೊಂಡ. ಆಗ ಸ್ವಾಮೀಜಿ ಮರುಮಾತಿಲ್ಲದೆ ಊಟ ಮಾಡಿದರು. ಆ ಮನುಷ್ಯ ಹೊರಟ ಕೆಲವೇ ಕ್ಷಣಗಳಲ್ಲಿ ಸ್ವಾಮೀಜಿ ಅವನ ಕಡೆಗೆ ನೋಡುತ್ತಾರೆ–ಅವನು ಮಾಯವಾಗಿಬಿಟ್ಟಿದ್ದಾನೆ! ಯಾರಿರಬಹುದು ಅವನು?! ಸ್ವಾಮೀಜಿಗೆ ಭಾವದುಂಬಿ ಬಂದಿತು. ಅನಾಥರಕ್ಷಕನಾದ ಭಗವಂತ ತನ್ನ ಭಕ್ತರನ್ನು ನೋಡಿಕೊಳ್ಳುವ ಪರಿಯನ್ನು ಕಂಡು ಅವರ ಕಂಗಳಿಂದ ನೀರು ಧಾರೆಯಾಗಿ ಹರಿಯಿತು. ಭಾವಾವೇಶದಿಂದ “ಜೈ ರಾಧೆ! ಜೈ ಶ್ರೀಕೃಷ್ಣ!” ಎಂದು ಗಟ್ಟಿಯಾಗಿ ಜಯಕಾರ ಮಾಡಿದರು.

ಗೋವರ್ಧನಗಿರಿಯಿಂದ ಸ್ವಾಮೀಜಿ ‘ರಾಧಾಕುಂಡ’ಕ್ಕೆ ಬಂದರು. ಅಲ್ಲಿನ ಕೆರೆಯಲ್ಲಿ ಸ್ನಾನ ಮಾಡಬೇಕೆಂದು ಅವರಿಗೆ ಮನಸ್ಸಾಯಿತು. ಆಗ ಅವರ ಮೈಮೇಲಿದ್ದ ಬಟ್ಟೆಯೆಂದರೆ ಒಂದು ತುಂಡು ಕೌಪೀನ ಮಾತ್ರ. ಆದ್ದರಿಂದ ಸ್ನಾನಕ್ಕೆ ಇಳಿಯುವ ಮೊದಲು ಕೌಪೀನವನ್ನು ಒಗೆದು ಕೆರೆಯ ದಡದ ಮೇಲೆ ಒಣಗಲು ಹಾಕಿದರು. ಆದರೆ ಸ್ನಾನ ಮುಗಿಸಿಕೊಂಡು ಬಂದು ನೋಡುತ್ತಾರೆ–ಕೌಪೀನ ಮಾಯ! ಎಲ್ಲಿಗೆ ಹೋಗಿರಬಹುದು ಅದು? ಗಾಳಿಗೆ ಹಾರಿಹೋಗಿರ ಬಹುದೆ? ಆ ಕೌಪೀನಕ್ಕೆ ಇನ್ನು ಯಾವ ಕಳ್ಳ ಬಂದಾನು! ಸ್ವಾಮೀಜಿ ಸುತ್ತಮುತ್ತ ನೋಡುತ್ತಾರೆ; ಅಲ್ಲೇ ಒಂದು ಮರದ ಮೇಲೆ ಕುಳಿತ ಒಂದು ಕೋತಿಯ ಕೈಯಲ್ಲಿದೆ ಕೌಪೀನ! ಕೋತಿಯ ಚೇಷ್ಟೆಗೆ ಮಿತಿಯಿದೆಯೆ! ಅಥವಾ... ಸ್ವಾಮೀಜಿಯ ಬ್ರಹ್ಮಚರ್ಯದ ತೇಜಸ್ಸನ್ನು ನೋಡಿ ಅದಕ್ಕೂ ಬ್ರಹ್ಮಚರ್ಯ ಪಾಲನೆ ಮಾಡುವ ಮನಸ್ಸಾಗಿರಬಹುದೆ? ಸ್ವಾಮೀಜಿಗಂತೂ ಬಹಳ ಸಂಕಟವಾಯಿತು. ಆ ಕೋತಿ ಕೌಪೀನವನ್ನು ಬೀಳಿಸುವಂತೆ ಮಾಡಲು ನಾನಾ ಉಪಾಯಗಳನ್ನು ಪ್ರಯೋಗಿಸಿ ನೋಡಿದರು. ಆದರೆ ಕೋತಿ ಯಾಕೋ ದೃಢನಿಶ್ಚಯ ಮಾಡಿದಂತಿತ್ತು. ಕೌಪೀನ ವನ್ನು ಬೀಳಿಸಲೇ ಇಲ್ಲ. ಆಗ ಸ್ವಾಮೀಜಿಗೆ ಕೋಪ ಬಂದಿತು–ಕೋತಿಯ ಮೇಲಲ್ಲ, ರಾಧೆಯ ಮೇಲೆ! ಅವಳೇ ಅಲ್ಲವೆ ಅಲ್ಲಿನ ಅಧಿಷ್ಠಾತೃ ದೇವತೆ? ಅವಳ ಪ್ರಭಾವ ಇರುವಂತಹ ಈ ಸ್ಥಳದಲ್ಲಿಇಂತಹ ಅನ್ಯಾಯ ನಡೆಯಬಹುದೆ ಎಂದು ರಾಧೆಯ ಮೇಲೆ ಕೋಪ. ತಾನಿನ್ನು ನಗರಕ್ಕೆ ಹೋಗುವುದಿಲ್ಲ; ಬದಲಾಗಿ ಕಾಡಿಗೆ ಹೋಗಿ ಅಲ್ಲೇ ಉಪವಾಸವಿದ್ದು ಪ್ರಾಣತ್ಯಾಗ ಮಾಡಿಬಿಡು ತ್ತೇನೆ ಎಂದು ನಿರ್ಧಾರ ಮಾಡಿಬಿಟ್ಟರು. ತಕ್ಷಣ ಕಾಡಿನ ಕಡೆಗೆ ನಡೆದರು. ಅಷ್ಟರಲ್ಲಿ ಎದುರುಗಡೆ ಯಿಂದ ಯಾರೋ ಒಬ್ಬ ಅಪರಿಚಿತ–ಇವನು ಆ ಘಟನೆಯನ್ನೆಲ್ಲ ನೋಡಿರಬೇಕು–ಒಂದು ತುಂಡು ಹೊಸ ಕಾವಿಬಟ್ಟೆಯನ್ನೂ ಸ್ವಲ್ಪ ಆಹಾರವನ್ನೂ ತಂದು ಅವರಿಗೆ ಅರ್ಪಿಸಿದ. ಇದನ್ನೆಲ್ಲ ಕಂಡು ಸ್ವಾಮೀಜಿಗೆ ಬಹಳ ಆಶ್ಚರ್ಯವಾಯಿತು. ಇದೊಂದು ಪವಾಡದಂತೆ ತೋರಿತು. ಊಟ ಮುಗಿಸಿಕೊಂಡು ಪುನಃ ಆ ಸರೋವರದ ಬಳಿಗೆ ಬಂದು ನೋಡುತ್ತಾರೆ–ಅವರ ಕೌಪೀನ ಯಥಾ ಸ್ಥಳದಲ್ಲಿ ಬಂದುಬಿದ್ದಿದೆ! ಈಗ ಅವರಿಗೆ ಒಂದು ವಿಷಯ ದೃಢವಾಯಿತು: ಏನೆಂದರೆ, ತಮ್ಮ ಪ್ರೀತಿಯ ಭಗವಂತ ತಮ್ಮನ್ನು ಎಲ್ಲಿಗೆ ಹೋದರೂ ಬಿಡದೆ ಕಾಪಾಡುತ್ತಾನೆ ಎಂದು. ಅಂತೂ ಬೃಂದಾವನಕ್ಕೆ ಬಂದದ್ದರಿಂದ ಸ್ವಾಮೀಜಿಗೆ ಗೋಪಿಕಾವಸ್ತ್ರಾಪಹರಣದ ಒಂದು ಅನುಭವವಾದಂತಾಯಿತು.

ಈಗ ಸ್ವಾಮೀಜಿಯ ಮನಸ್ಸು ಹರಿದ್ವಾರದ\footnote{*ಹಿಮಾಲಯದಲ್ಲಿರುವ ಬದರೀನಾಥ (ಹರಿ) ಕ್ಷೇತ್ರಕ್ಕೂ ಕೇದಾರನಾಥ (ಹರ) ಕ್ಷೇತ್ರಕ್ಕೂ ಇದರ ಮೂಲಕವೇ ಹೋಗಬೇಕು. ಆದ್ದರಿಂದ ಈ ಸ್ಥಳವನ್ನು ಹರಿದ್ವಾರ, ಹರದ್ವಾರ ಎಂಬ ಎರಡು ಹೆಸರುಗಳಿಂದ ಕರೆಯುತ್ತಾರೆ.} ಕಡೆಗೆ ಹೊರಳಿತು. ಅಲ್ಲಿಂದ ಮುಂದೆ ಹಿಮಾಲಯದಲ್ಲಿರುವ ಬದರಿ ಕೇದಾರಗಳ ಯಾತ್ರೆ ಮಾಡಬೇಕೆಂದು ಅವರು ತೀರ್ಮಾನಿಸಿದರು. ಆದರೆ ತಮ್ಮ ಗುರುಭಾಯಿಯಾದ ಸ್ವಾಮಿ ಅದ್ವೈತಾನಂದರು ಬೃಂದಾವನಕ್ಕೆ ಬರುತ್ತಿದ್ದಾರೆಂಬ ಸುದ್ದಿ ತಲುಪಿದ್ದರಿಂದ ಇನ್ನೂ ಕೆಲವು ದಿನ ಬೃಂದಾವನದಲ್ಲೇ ಉಳಿದುಕೊಂಡು, ಅದ್ವೈತಾನಂದ ರನ್ನು ಭೇಟಿಮಾಡಿ ಬಳಿಕ ಹಿಮಾಲಯದ ಕಡೆಗೆ ಹೊರಟರು. 

ಹರಿದ್ವಾರಕ್ಕೆ ಹೋಗುವ ದಾರಿಯಲ್ಲಿ ಹತ್ರಾಸ್ ರೈಲುನಿಲ್ದಾಣ ಸಿಗುತ್ತದೆ. ಈ ನಿಲ್ದಾಣಕ್ಕೆ ಬಂದು ಒಂದೆಡೆ ಕುಳಿತುಬಿಟ್ಟಿದ್ದಾರೆ ಸ್ವಾಮೀಜಿ. ನಡೆದು ನಡೆದು ಹಸಿವು, ದಣಿವು ಸೇರಿದೆ; ಇದು ಅವರನ್ನು ನೋಡಿದ ಕೂಡಲೇ ತಿಳಿಯುವಂತಿದೆ. ಶರಚ್ಚಂದ್ರಗುಪ್ತ ಎಂಬ ಉತ್ಸಾಹೀ ಯುವಕ ಆ ರೈಲುನಿಲ್ದಾಣದ ಅಸಿಸ್ಟೆಂಟ್ ಸ್ಟೇಷನ್ ಮಾಸ್ಟರ್. ಸ್ವಾಮೀಜಿ ದಣಿದು ಕುಳಿತಿರುವ ದೃಶ್ಯ ಅವನ ಕಣ್ಣಿಗೆ ಬಿತ್ತು. ಅವನೊಬ್ಬ ಅದ್ಭುತ ವ್ಯಕ್ತಿಯೆನ್ನಬೇಕು. ಅವನ ಇಡೀ ವ್ಯಕ್ತಿತ್ವವನ್ನು ಮೂರು ಪದಗಳಲ್ಲಿ ಕ್ರೋಡೀಕರಿಸಿ ಹೇಳಿಬಿಡಬಹುದು: ಮಾಧುರ್ಯ, ಪ್ರಾಮಾಣಿಕತೆ ಮತ್ತು ಪೌರುಷ. ಈತ ಸ್ವಾಮೀಜಿಯನ್ನು ಕಂಡ ಪ್ರಥಮ ನೋಟದಲ್ಲೇ ಅವರ ಸುತ್ತ ಒಂದು ಅಲೌಕಿಕ ಆಧ್ಯಾತ್ಮಿಕ ಪ್ರಕಾಶ ಸುತ್ತುವರಿದಿರುವುದನ್ನು ಕಂಡು ಅವರೆಡೆಗೆ ತೀವ್ರವಾಗಿ ಆಕರ್ಷಿತನಾದ. ಸ್ವಾಮೀಜಿ ಸಿದ್ಧಪುರುಷರು; ಸ್ವತಃ ಶರಚ್ಚಂದ್ರನೂ ಶುದ್ಧಹೃದಯದ ಯುವಕ. ಆದ್ದರಿಂದ ಆತ ಅವರ ಆಧ್ಯಾತ್ಮಿಕ ಪ್ರಭೆಯನ್ನು ಗುರುತಿಸಿದ್ದರಲ್ಲಿ ಅಚ್ಚರಿಯೇನಿಲ್ಲ. ಅವರಿಗೆ ತನ್ನಿಂದೇನಾದರೂ ಸೇವೆಯಾಗುವಂತಿದ್ದರೆ ಮಾಡೋಣ ಎಂದು ಅವರ ಹತ್ತಿರಕ್ಕೆ ಬಂದು ನಮಸ್ಕರಿಸಿ, ತನ್ನ ಪರಿಚಯ ಹೇಳಿಕೊಂಡ. ಬಳಿಕ ಕೇಳಿದ, “ಸ್ವಾಮೀಜಿ, ನೀವು ಹಸಿದಿದ್ದೀರಿ, ಅಲ್ಲವೆ?” “ಹೌದು ಹೌದು; ಚೆನ್ನಾಗಿಯೇ ಹಸಿವಾಗಿದೆ!” ಎಂದರು ಸ್ವಾಮೀಜಿ. ಆಗ ಅವನು, “ಹಾಗಾದರೆ, ದಯ ವಿಟ್ಟು ನನ್ನ ಮನೆಗೆ ಬನ್ನಿ” ಎಂದು ಆಹ್ವಾನಿಸಿದ. ಆದರೆ ಅವನೆಷ್ಟಾದರೂ ಬ್ರಹ್ಮಚಾರಿ; ಮನೆ ಯಲ್ಲಿ ಅವನೊಬ್ಬನೇ. ಅವನ ಮನೆಯಲ್ಲಿ ಇವರಿಗೇನು ಸಿಕ್ಕೀತು? ಆದ್ದರಿಂದ ಸ್ವಾಮೀಜಿ ಮುಗುಳ್ನಗುತ್ತ ಕೇಳಿದರು, “ನಿಮ್ಮ ಮನೆಗೆ ಬಂದರೆ ತಿನ್ನಲು ನನಗೇನು ಕೊಡುತ್ತೀ?” ತಕ್ಷಣ ಶರಚ್ಚಂದ್ರ ಪಾರಸೀ ಭಾಷೆಯ ಒಂದು ಕವನವನ್ನು ಉದ್ಧರಿಸಿದ. ಅದರ ಭಾವಾರ್ಥ ಇದು: “ಓ ನನ್ನ ಪ್ರಿಯಸಖನೆ, ಅತ್ಯಂತ ವಿಶ್ವಾಸದಿಂದ ನೀನು ನನ್ನ ಮನೆಗೆ ಬಂದಿರುವೆ; ನಿನಗೆ ನಾನು ನನ್ನ ಹೃದಯದ ಮಾಂಸದಿಂದ ಅತ್ಯಂತ ರುಚಿಯಾದ ಭಕ್ಷ್ಯವನ್ನು ಮಾಡಿ ಬಡಿಸುತ್ತೇನೆ.” ಈ ಉತ್ತರದಿಂದ ಸಂತುಷ್ಟರಾದ ಸ್ವಾಮೀಜಿ, ಅವನ ಆಹ್ವಾನವನ್ನು ಮನ್ನಿಸಿ ಅವನೊಂದಿಗೆ ಹೋದರು. ಅವನು ತಯಾರಿಸಿ ಬಡಿಸಿದ ಅಡಿಗೆಯನ್ನು ಸಂತೋಷದಿಂದ ಭುಜಿಸಿದರು.

ಶರಚ್ಚಂದ್ರ ತನ್ನ ಅಂದಿನ ಕೆಲಸಕಾರ್ಯಗಳನ್ನು ಮುಗಿಸಿಕೊಂಡು ಉಳಿದ ಸಮಯವೆಲ್ಲ ಸ್ವಾಮೀಜಿಯ ಬಳಿಯಲ್ಲೇ ಇದ್ದುಬಿಟ್ಟ. ಅವರ ಕಣ್ಣುಗಳೇ ಅವನನ್ನು ಸಂಪೂರ್ಣವಾಗಿ ಸೆರೆಹಿಡಿದುಬಿಟ್ಟಿದ್ದುವು! ಆ ಆಕರ್ಷಣೆಯಿಂದ ಬಿಡಿಸಿಕೊಳ್ಳುವುದು ಅಸಾಧ್ಯವಾಗಿತ್ತು. ಆದ್ದ ರಿಂದ, ಕೆಲದಿನಗಳ ಮಟ್ಟಿಗಾದರೂ ತನ್ನ ಮನೆಯಲ್ಲಿದ್ದು ಆತಿಥ್ಯ ಸ್ವೀಕರಿಸಬೇಕು ಎಂದು ಬೇಡಿಕೊಂಡ. ಸ್ವಾಮೀಜಿ ಅದಕ್ಕೊಪ್ಪಿದರು. ಈಗ ಶರಚ್ಚಂದ್ರ, “ಸ್ವಾಮೀಜಿ, ನನಗೆ ಜ್ಞಾನವನ್ನು ಅನುಗ್ರಹಿಸಿ” ಎಂದು ಕೇಳಿಕೊಂಡ. ಅದಕ್ಕುತ್ತರವಾಗಿ ಸ್ವಾಮೀಜಿ, ‘ವಿದ್ಯಾ-ಸುಂದರ’ ಎಂಬ ಬಂಗಾಳೀ ಕೃತಿಯೊಂದರ ಮಾತನ್ನು ಉದ್ಧರಿಸಿದರು:

“ನಿನಗೆ ‘ವಿದ್ಯಾ’ ಬೇಕೆನ್ನುವುದಾದರೆ,\footnote{*ಇದು ಆ ಕಥೆಯಲ್ಲಿ, ಮಾಲಿನಿ ಎಂಬಾಕೆ ಕಥಾನಾಯಕನಾದ ಸುಂದರನಿಗೆ ಹೇಳುವ ಮಾತು. `ವಿದ್ಯಾ' ಎಂಬುದು ಕಥಾನಾಯಕಿಯ ಹೆಸರು. `ವಿದ್ಯೆ' ಎನ್ನುವುದಕ್ಕೂ ಬಂಗಾಳಿಯಲ್ಲಿ `ವಿದ್ಯಾ' ಎಂದೇ ಹೇಳುವುದು.} ನಿನ್ನ ಸುಂದರ ಮುಖಕ್ಕೆ ಬೂದಿ ಬಳಿದುಕೊಂಡು ಬಾ. ಇಲ್ಲದಿದ್ದರೆ, ನಿನ್ನ ದಾರಿ ನೋಡಿಕೊ.”

ಎರಡರ್ಥ ಬರುವಂತೆ ಸ್ವಾಮೀಜಿ ಈ ವಾಕ್ಯವನ್ನು ಬಳಸಿದ್ದರು. ಆದರೆ ಹೆಚ್ಚು ತಿಳಿಯದ ಶರಚ್ಚಂದ್ರ ಅದನ್ನು ಅಕ್ಷರಶಃ ಸ್ವೀಕರಿಸಿ, ಮುಖಕ್ಕೆ ಬೂದಿ ಬಳಿದುಕೊಂಡೇ ಬಂದುಬಿಟ್ಟ! ಅದನ್ನು ಕಂಡು ಸ್ವಾಮೀಜಿ ಗಟ್ಟಿಯಾಗಿ ನಕ್ಕುಬಿಟ್ಟರು. ಆದರೆ ಆತನ ಸರಳಶ್ರದ್ಧೆಯನ್ನು ಅವರು ಮೆಚ್ಚಿಕೊಳ್ಳದಿರಲಿಲ್ಲ. ಆದರೆ, ನಿಜಕ್ಕೂ ಶರಚ್ಚಂದ್ರನಿಗೆ ಸಂನ್ಯಾಸದ ಕಲ್ಪನೆ ಹೊಸದೇನೂ ಅಲ್ಲ. ಹಿಂದೆಯೇ ಅವನ ಅಣ್ಣ ಸಂನ್ಯಾಸಿಯಾಗಿ ಹೊರಟುಹೋಗಿಬಿಟ್ಟಿದ್ದ.

ಹತ್ರಾಸಿನಲ್ಲಿ ಬ್ರಜೇನ್ಬಾಬು ಎಂಬ ತಮ್ಮ ಹಳೆಯ ಪರಿಚಯಸ್ಥನೊಬ್ಬನು ವಾಸವಾಗಿರು ವುದನ್ನು ತಿಳಿದು ಸ್ವಾಮೀಜಿ ತಾವಾಗಿಯೇ ಅವನ ಮನೆಯನ್ನು ಹುಡುಕಿಕೊಂಡು ಹೋದರು. ಆತ ಅವರನ್ನು ತಕ್ಷಣ ಗುರುತಿಸಿ ವಿಶ್ವಾಸದಿಂದ ಸ್ವಾಗತಿಸಿದನಲ್ಲದೆ, ತನ್ನ ಮನೆಯಲ್ಲೂ ಕೆಲವು ದಿನ ಇರಬೇಕು ಎಂದು ಕೇಳಿಕೊಂಡ. ಸ್ವಾಮೀಜಿ ಅದಕ್ಕೊಪ್ಪಿ, ಕೆಲವು ದಿನ ಈತನ ಮನೆಯ ಲ್ಲಿದ್ದು ಹಿಂದಿರುಗುವುದಾಗಿ ಶರಚ್ಚಂದ್ರನಿಗೆ ಹೇಳಿದರು. ಅವರು ಬ್ರಜೇನನ ಮನೆಯಲ್ಲಿ ಉಳಿದು ಕೊಂಡಿದ್ದಾಗ ಹತ್ರಾಸ್ ಪಟ್ಟಣದಲ್ಲಿ ವಾಸವಾಗಿದ್ದ ಬಂಗಾಳಿಗಳೆಲ್ಲರೂ ಅವರನ್ನು ನೋಡಲು, ಅವರ ದಿವ್ಯವಾಣಿಯನ್ನು ಕೇಳಲು ಧಾವಿಸಿ ಬರಲಾರಂಭಿಸಿದರು. ಆ ಜನಗಳೊಂದಿಗೆ ಸ್ವಾಮೀಜಿ ಹಿಂದೂಧರ್ಮದ ಬಗ್ಗೆ, ಭಾರತದ ಹಿರಿಮೆಯ ಬಗ್ಗೆ ಸ್ಫೂರ್ತಿಯುತವಾಗಿ ಮಾತನಾಡುತ್ತಿದ್ದರು. ಅದರಿಂದ ಈ ಬಂಗಾಳಿಗಳು ಎಷ್ಟು ಸ್ಫೂರ್ತಿಗೊಂಡರೆಂದರೆ, ಅವರ ಮಾತುಗಳನ್ನು ಕೇಳುತ್ತ ಗಂಟೆಗಟ್ಟಲೆ ಕುಳಿತುಬಿಡುತ್ತಿದ್ದರು. ಈ ದಿನಗಳ ಬಗ್ಗೆ ಶರಚ್ಚಂದ್ರನ ಸ್ನೇಹಿತನಾದ ನಟಕೃಷ್ಣ ಎಂಬವನು ಮುಂದೊಮ್ಮೆ ಹೇಳುತ್ತಾನೆ:

“ಸ್ವಾಮೀಜಿಯೊಂದಿಗೆ ನಿರಂತರ ಆಧ್ಯಾತ್ಮಿಕ ಸಂಭಾಷಣೆಯಲ್ಲಿ ಭಾಗವಹಿಸುತ್ತ ಅವರ ಸತ್ಸಹವಾಸದಲ್ಲಿ ಕಳೆದ ಆ ದಿನಗಳು, ನಿಜಕ್ಕೂ ನಮ್ಮ ಜೀವಮಾನದಲ್ಲೇ ಅತ್ಯಂತ ಸಾರ್ಥಕ ದಿನಗಳು. ಅವರ ಪವಿತ್ರ ಸಹವಾಸದ ಪರಿಣಾಮವಾಗಿ ಆ ಊರಿನ ಬಂಗಾಳಿಗಳಲ್ಲಿದ್ದ ಒಳಜಗಳಗಳೆಲ್ಲ ತನ್ನಿಂತಾನೇ ಪರಿಹಾರವಾದುವು. ನಮ್ಮ ಸಮಾಜದಲ್ಲಿ, ತಾವು ವಯೋವೃದ್ಧ ರೆಂದೋ ಉನ್ನತ ಮನೆತನದವರೆಂದೋ ಹೆಮ್ಮೆ ಪಡುತ್ತಿದ್ದವರೂ ಕೂಡ, ಈ ತರುಣ ಸಂನ್ಯಾಸಿಯ ಮುಂದೆ ಮಕ್ಕಳಂತೆ ಕುಳಿತುಕೊಂಡು, ತಮ್ಮ ಅಧಿಕಾರಮದ, ಕುಲಮದ, ಧನಮದ ಗಳನ್ನೆಲ್ಲ ಮರೆತು, ಆಧ್ಯಾತ್ಮಿಕ ವಿಚಾರಗಳನ್ನು ಕೇಳಿ ತಿಳಿದುಕೊಳ್ಳುತ್ತಿದ್ದರು. ಸಂಜೆಯ ಹೊತ್ತಿಗೆ ಭಜನೆಯ ಕಾರ್ಯಕ್ರಮವಿರುತ್ತಿತ್ತು. ಸ್ವಾಮೀಜಿಯ ಮಧುರ ಕಂಠವನ್ನು ಕೇಳಿ, ಅಲ್ಲಿ ನೆರೆದಿದ್ದವ ರೆಲ್ಲ ಭಾವಪರವಶರಾಗುತ್ತಿದ್ದರು. ಅವರ ಕಂಠಸ್ವರವನ್ನು ಕೇಳಿದವರಿಗೆ ಇನ್ನಷ್ಟು ಕೇಳುವ ಹಂಬಲ; ಕೇಳಿದಷ್ಟೂ ತೃಪ್ತಿಯಿಲ್ಲ.”

ಆಗಾಗ ಸ್ವಾಮೀಜಿ ಯಾವುದೋ ಗಂಭೀರಚಿಂತನೆಯಲ್ಲಿ ಮುಳುಗಿ, ಮ್ಲಾನವದನರಾಗಿ ಕುಳಿತುಬಿಡುತ್ತಿದ್ದುದನ್ನು ಶರಚ್ಚಂದ್ರ ಗಮನಿಸುತ್ತಿದ್ದ. ಒಂದು ದಿನ ಅಂತಹ ಸಂದರ್ಭದಲ್ಲಿ ಆತ ಕೇಳಿಯೇಬಿಟ್ಟ, “ಸ್ವಾಮೀಜಿ, ನೀವೇಕೋ ತುಂಬ ಬೇಸರದಿಂದಿರುವಂತೆ ಕಾಣುತ್ತಿದ್ದೀರಿ; ಏಕೆ?” ಸ್ವಾಮೀಜಿ ಸ್ವಲ್ಪ ಹೊತ್ತು ಹಾಗೇ ಇದ್ದು, ಬಳಿಕ ನುಡಿದರು: “ಮಗು, ಒಂದು ಮಹಾ ಕಾರ್ಯವನ್ನು ಸಾಧಿಸಬೇಕಾದ ಹೊಣೆ ನನ್ನದಾಗಿದೆ. ಅದನ್ನು ಸಾಧಿಸಬೇಕೆಂಬುದು ನನ್ನ ಗುರುವಿನ ಆಜ್ಞೆ. ಆದರೆ ನನ್ನ ಅಸಾಮರ್ಥ್ಯವನ್ನೇ ಕಾಣುತ್ತ ನನಗೆ ನಿರಾಶೆಯಾಗಿದೆ. ಅದು ಯಾವ ಕಾರ್ಯ ಗೊತ್ತೇನು? ನಮ್ಮ ಮಾತೃಭೂಮಿಯನ್ನು ಮೇಲೆತ್ತುವ ಬೃಹತ್ಕಾರ್ಯ! ಧರ್ಮ, ಆಧ್ಯಾತ್ಮಿಕತೆ ಗಳು ತೀರಾ ಅಧೋಗತಿಗಿಳಿದಿವೆ. ಎಲ್ಲೆಲ್ಲೂ ಹಸಿವಿನ ಮಾರಿ ತಾಂಡವವಾಡುತ್ತಿದೆ. ನನ್ನ ಭಾರತ ಸಶಕ್ತವಾಗಲೇಬೇಕು; ಕ್ರಿಯಾಶೀಲವಾಗಲೇಬೇಕು. ಅದು ತನ್ನ ಆಧ್ಯಾತ್ಮಿಕ ಶಕ್ತಿಯಿಂದ ಇಡೀ ಪ್ರಪಂಚದಲ್ಲೆಲ್ಲ ಪ್ರಭಾವ ಬೀರುವಂತಾಗಬೇಕು.”

ಈ ಮಾತುಗಳನ್ನು ಕೇಳಿ ವಿಸ್ಮಯಮೂಕನಾಗಿ ಕುಳಿತ ಶರಚ್ಚಂದ್ರ. ಬಳಿಕ ಅಷ್ಟೇ ರಭಸದಿಂದ, ಅಷ್ಟೇ ಉತ್ಸಾಹದಿಂದ ಹೇಳಿದ, “ಸ್ವಾಮೀಜಿ, ನಾನಿದ್ದೇನೆ! ನನ್ನಿಂದೇನಾಗಬೇಕೋ ಹೇಳಿ.”

ಸ್ವಾಮೀಜಿ (ಗಂಭೀರವಾಗಿ): “ಕೈಯಲ್ಲಿ ಕಮಂಡಲು ಭಿಕ್ಷಾಪಾತ್ರೆಗಳನ್ನು ಹಿಡಿದು ಮಹೋ ದ್ದೇಶಕ್ಕಾಗಿ ದುಡಿಯಲು ಸಿದ್ಧನಿರುವೆಯಾ? ಮನೆಮನೆಗೆ ಹೋಗಿ ಭಿಕ್ಷೆ ಎತ್ತಬಲ್ಲೆಯಾ?”

ಶರಚ್ಚಂದ್ರ: “ಓ, ಖಂಡಿತವಾಗಿ ಸ್ವಾಮೀಜಿ!”

ಶರಚ್ಚಂದ್ರನ ಧೈರ್ಯವನ್ನೂ ದೃಢಬುದ್ಧಿಯನ್ನೂ ಕಂಡು ಸ್ವಾಮೀಜಿ ಬಹಳ ಸಂತೋಷ ಪಟ್ಟರು.

ಈ ಸಂಭಾಷಣೆಯನ್ನು ಕೇಳಿದಾಗ ಒಂದು ಪ್ರಶ್ನೆ ಏಳಬಹುದು. ರಾಷ್ಟ್ರೋದ್ಧಾರ ಮಾಡುವು ದಕ್ಕೂ, ಕಮಂಡಲ-ಭಿಕ್ಷಾಪಾತ್ರೆಗಳನ್ನು ಹಿಡಿದುಕೊಂಡು ಸಂನ್ಯಾಸಿಯಾಗುವುದಕ್ಕೂ ಏನು ಸಂಬಂಧ? ರಾಷ್ಟ್ರವನ್ನು ಉದ್ಧಾರ ಮಾಡಬೇಕಾದರೆ ಹಣಬೇಕು; ಆದ್ದರಿಂದ ‘ಚೆನ್ನಾಗಿ ಸಂಪಾದಿಸು’ ಎಂದು ಹೇಳಬೇಕಾಗಿತ್ತು ಸ್ವಾಮೀಜಿ. ರಾಷ್ಟ್ರವನ್ನು ಪ್ರಗತಿಯ ಹಾದಿಯಲ್ಲಿ ಕರೆದೊಯ್ಯಲು ಉನ್ನತ ಅಧಿಕಾರ ಬೇಕು; ಆದ್ದರಿಂದ ಐ.ಎ.ಎಸ್. ಪರೀಕ್ಷೆ ಪಾಸುಮಾಡು ಎಂದೋ, ಮಂತ್ರಿಯಾಗು ಎಂದೋ ಹೇಳಬೇಕಾಗಿತ್ತಲ್ಲವೆ? ಅದನ್ನು ಬಿಟ್ಟು, ‘ಸಂನ್ಯಾಸಿಯಾಗು ತ್ತೀಯಾ?’ಎಂದೇಕೆ ಕೇಳಿದರು? ಏಕೆಂದರೆ, ಯಾವನು ತ್ಯಾಗ ಮಾಡುತ್ತಾನೋ ಅವನು ಮಾತ್ರ ಮಹಾಕಾರ್ಯಗಳನ್ನು ಸಾಧಿಸಬಲ್ಲ. ಬುದ್ಧನು ರಾಜ್ಯಭೋಗವನ್ನು ತ್ಯಾಗಮಾಡಿ ಜ್ಞಾನ ಸಂಪಾದಿಸಿ ದ್ದರಿಂದ, ರಾಜನಾಗಿ ಮಾಡಬಹುದಾಗಿದ್ದ ಸೇವೆಗಿಂತಲೂ ಹೆಚ್ಚಿನ ಸೇವೆಯನ್ನು ಮಾಡಲು ಸಾಧ್ಯವಾಯಿತು. ಮಹಾತ್ಮಾ ಗಾಂಧೀಜಿಯವರು ಮಹಾತ್ಯಾಗ ಮಾಡಿದಮೇಲೆಯೇ ಸಮಗ್ರ ಭಾರತದ ಜನ ಅವರನ್ನು ಅನುಸರಿಸಿ ನಡೆದರು; ಅವರೊಂದಿಗೆ ಸಹಕರಿಸಿದರು; ಅವರೊಂದಿಗೆ ಪ್ರಾಣ ಕೊಟ್ಟರು; ಸ್ವಾತಂತ್ರ್ಯವನ್ನು ಪಡೆದರು. ಆದ್ದರಿಂದ ನಿಜವಾದ ಮಹಾಕಾರ್ಯವೆಂಬುದೇ ನಾದರೂ ಸಾಧಿಸಲ್ಪಡಬೇಕಾದರೆ, ನಿಜವಾದ ಸೇವೆ ಸಲ್ಲಬೇಕಾದರೆ, ಅಲ್ಲಿ ಪರಿಪೂರ್ಣವಾದ ತ್ಯಾಗ ಇರಲೇ ಬೇಕಾಗುತ್ತದೆ. ಆತ್ಮೋದ್ಧಾರ ಮಾಡಿಕೊಳ್ಳಬೇಕಾದರೂ, ರಾಷ್ಟ್ರೋದ್ಧಾರ ಮಾಡಬೇಕಾದರೂ ಸ್ವಸುಖವನ್ನು ತ್ಯಾಗ ಮಾಡಬೇಕಾಗುತ್ತದೆ. ಇದನ್ನೇ ಸ್ವಾಮಿ ವಿವೇಕಾನಂದರು ಮುಂದೆ ಘಂಟಾಘೋಷವಾಗಿ ಸಾರುತ್ತಾರೆ–“ಮಹಾಕಾರ್ಯಗಳು ಮಹಾತ್ಯಾಗದಿಂದ ಮಾತ್ರ ಸಾಧ್ಯ!” ಎಂದು.

ಹರಿದ್ವಾರಕ್ಕೆಂದು ಹೊರಟಿದ್ದ ಸ್ವಾಮೀಜಿ ಹತ್ರಾಸಿಗೆ ಬಂದು ಬಹಳ ದಿನಗಳಾಗಿ ಬಿಟ್ಟಿವೆ. ಆದ್ದರಿಂದ, ತಾವಿನ್ನು ಹೊರಟುಬಿಡಬೇಕೆಂದು ನಿಶ್ಚಯಿಸಿ, ಒಂದು ದಿನ ಶರಚ್ಚಂದ್ರನಿಗೆ ಹೇಳುತ್ತಾರೆ: “ನೋಡು, ನಾನಿನ್ನು ಇಲ್ಲಿ ನಿಲ್ಲಲಾರೆ. ಸಂನ್ಯಾಸಿಯಾದವನು ಒಂದೇ ಸ್ಥಳದಲ್ಲಿ ಬಹಳ ದಿನ ಇರಬಾರದು. ಅಲ್ಲದೆ ನನಗೆ ಈಗಾಗಲೇ ನಿಮ್ಮ ಮೇಲೆಲ್ಲ ಒಂದು ಬಗೆಯ ಆಸಕ್ತಿ ಬೆಳೆಯುತ್ತಿದೆ. ಆಧ್ಯಾತ್ಮಿಕ ಜೀವನದಲ್ಲಿ ಇದೂ ಒಂದು ಬಂಧನ.” ಶರಚ್ಚಂದ್ರನಿಗೆ ಸ್ವಾಮೀಜಿ ಯನ್ನು ಬಿಟ್ಟುಕೊಡಲು ಸ್ವಲ್ಪವೂ ಇಷ್ಟವಿಲ್ಲ. ಆದ್ದರಿಂದ ಅವರನ್ನು ಇನ್ನೂ ಕೆಲವು ದಿನ ಉಳಿದುಕೊಳ್ಳುವಂತೆ ಬೇಡಿಕೊಂಡ. ಅದಕ್ಕೆ ಅವರದೊಂದೇ ಉತ್ತರ: “ನನ್ನನ್ನು ಒತ್ತಾಯಿಸ ಬೇಡ.” ಶರಚ್ಚಂದ್ರನಿಗೆ ಅವರಲ್ಲಿ ಅಪಾರ ಪ್ರೀತಿ-ವಿಶ್ವಾಸ ಬೆಳೆದುಬಿಟ್ಟಿತ್ತು. ಅವರನ್ನು ಬಿಟ್ಟು ಕೊಡಲಾರದೆ, “ಸ್ವಾಮೀಜಿ, ನನಗೆ ಮಂತ್ರದೀಕ್ಷೆ ನೀಡಿ ನಿಮ್ಮ ಶಿಷ್ಯನಾಗಿ ಮಾಡಿಕೊಳ್ಳಿ” ಎಂದು ಗೋಗರೆದ. ಆಗ ಅವರೆಂದರು, “ಏನು, ನನ್ನ ಶಿಷ್ಯನಾದ ಮಾತ್ರಕ್ಕೆ ಆಧ್ಯಾತ್ಮಿಕ ಜೀವನದಲ್ಲಿ ಎಲ್ಲವನ್ನೂ ಸಿದ್ಧಿಸಿಕೊಂಡುಬಿಡಬಹುದು ಎಂದು ತಿಳಿದುಕೊಂಡೆಯಾ? ಭಗವಂತನೇ ಎಲ್ಲವೂ ಆಗಿದ್ದಾನೆ ಎಂಬುದನ್ನು ನೆನಪಿಟ್ಟುಕೋ. ಆಗ ನೀನು ಮಾಡಿದ ಪ್ರತಿಯೊಂದು ಕಾರ್ಯವೂ ನಿನ್ನ ಪ್ರಗತಿಗೆ ನೆರವಾಗುತ್ತದೆ. ನಾನು ಆಗಾಗ ಬಂದು ನಿನ್ನ ಜೊತೆಗಿರುತ್ತೇನೆ. ಈಗಂತೂ ನಾನು ಹಿಮಾಲಯದ ಕಡೆಗೆ ಹೋಗಲೇಬೇಕಾಗಿದೆ.” ಕಡೆಗೆ ಅವನು, “ಸ್ವಾಮೀಜಿ, ನೀವು ಏನು ಬೇಕಾದರೂ ಹೇಳಿಕೊಳ್ಳಿ; ನೀವು ಎಲ್ಲಿಗೇ ಹೋದರೂ ನಾನೂ ನಿಮ್ಮನ್ನು ಹಿಂಬಾಲಿಸಿ ಬರುವವನೇ” ಎಂದುಬಿಟ್ಟ. ಸ್ವಾಮೀಜಿ ಅವನನ್ನೇ ಒಂದುಕ್ಷಣ ದಿಟ್ಟಿಸುತ್ತ ನುಡಿದರು, “ನಿಜಕ್ಕೂ ನನ್ನ ಜೊತೆಗೆ ಬರಬಲ್ಲೆಯಾ? ಹಾಗಾದರೆ ಈಗಲೇ ನನ್ನ ಭಿಕ್ಷಾಪಾತ್ರೆಯನ್ನು ತೆಗೆದುಕೊಂಡು ಹೋಗಿ, ನಿನ್ನ ರೈಲ್ವೆ ಕೂಲಿಗಳ ಹತ್ತಿರ ಭಿಕ್ಷೆಬೇಡಿ ತೆಗೆದುಕೊಂಡು ಬಾ ನೋಡೋಣ!” ಸ್ವಾಮೀಜಿ ಹೀಗೆಂದದ್ದೇ ತಡ, ಶರಚ್ಚಂದ್ರ ಭಿಕ್ಷಾಪಾತ್ರೆಯನ್ನು ಹಿಡಿದು ರೈಲ್ವೆ ಕೂಲಿಗಳ ವಠಾರಕ್ಕೆ ನಡೆದೇ ಬಿಟ್ಟ! ತನ್ನ ಗುರುವಿನ ಆಣತಿಯಂತೆ ಭಿಕ್ಷೆ ಸಂಗ್ರಹಿಸಿ ತಂದ. ಸ್ವಾಮೀಜಿಗೆ ಒಪ್ಪಿಗೆ ಯಾಯಿತು. ಅವನಿಗೆ ಮಂತ್ರದೀಕ್ಷೆ ನೀಡಿ ಅನುಗ್ರಹಿಸಿ, ತಮ್ಮ ಮೊಟ್ಟಮೊದಲ ಶಿಷ್ಯನಾಗಿ ಸ್ವೀಕರಿಸಿದರು.

ಅವನೀಗ ತನ್ನ ಗುರುವನ್ನು ಅನುಸರಿಸಿ ಹೃಷೀಕೇಶಕ್ಕೆ ಹೊರಡಲು ಸಿದ್ಧನಾದ. ಆದರೆ ಅವನದು ಜವಾಬ್ದಾರಿಯ ಹುದ್ದೆ; ಇದ್ದಕ್ಕಿದ್ದಂತೆ ರಜಾ ಹಾಕಿ ಹೋಗುವಂತಿಲ್ಲ. ಅವನ ಅದೃಷ್ಟಕ್ಕೆ, ಬದಲಿಗನೊಬ್ಬನ ನೇಮಕವಾಯಿತು. ಒಡನೆಯೇ ಅವನು ಸ್ವಾಮೀಜಿಯ ಜೊತೆಯಲ್ಲಿ ಹೃಷೀ ಕೇಶಕ್ಕೆ ಹೊರಟುಬಿಟ್ಟ. ಶಹರಾನ್ಪುರದವರೆಗೆ (೨೩೦ ಮೈಲಿ) ರೈಲುಪ್ರಯಾಣ; ಅನಂತರ ಕಾಲ್ನಡಿಗೆ. ಅವನಿಗೆ ಇಂತಹ ಪ್ರಯಾಸಕರ ಪ್ರಯಾಣದ ಅಭ್ಯಾಸವಿರಲಿಲ್ಲ. ಸಾಲದ್ದಕ್ಕೆ ಹೆಗಲ ಮೇಲೆ ಭಾರದ ಗಂಟು. ಆದ್ದರಿಂದ ದಣಿವೋ ದಣಿವು. ಪರಿವ್ರಾಜಕ ಜೀವನ ಎಷ್ಟು ಕಠಿಣ ಎಂಬುದರ ಕಲ್ಪನೆಯಾಯಿತು ಅವನಿಗೆ. ಆದರೆ ಸ್ವಾಮೀಜಿಯ ಬಲದಿಂದ, ಅವರ ಪ್ರೇಮದ ಆರೈಕೆಯಿಂದ ಆ ಕಷ್ಟಗಳನ್ನು ಸಹಿಸಿಕೊಳ್ಳಲು ಸಾಧ್ಯವಾಯಿತು. ಸ್ವಾಮೀಜಿಯ ಮಾತುಕತೆಗಳನ್ನು ಆಲಿಸುತ್ತ, ಅವನಿಗೆ ಎಷ್ಟೋ ಸಲ ಪ್ರಯಾಣದ ಆಯಾಸವೆಲ್ಲ ಮರವೆಯಾಗುತ್ತಿತ್ತು. ಮುಂದೆ ಶರಚ್ಚಂದ್ರ ತನ್ನ ಸಂನ್ಯಾಸೀಬಂಧುಗಳ ಬಳಿ ಈ ದಿನಗಳ ಅನುಭವಗಳನ್ನು ವರ್ಣಿಸುತ್ತ ಹೇಳುತ್ತಾನೆ: “ಹಿಮಾಲಯ ಪ್ರದೇಶದಲ್ಲಿ ಸಂಚರಿಸುತ್ತಿದ್ದ ಆ ಸಮಯದಲ್ಲಿ ಒಂದು ದಿನ ನಾನು ಹಸಿವು-ಬಾಯಾರಿಕೆಗಳಿಂದ ಕಂಗಾಲಾಗಿ ಪ್ರಜ್ಞೆತಪ್ಪಿ ಬಿದ್ದೆ. ಆ ಕೊರೆತದಲ್ಲಿ ನಾನು ಸತ್ತೇಹೋಗಿಬಿಡಬೇಕಾಗಿತ್ತು. ಆದರೆ ಸ್ವಾಮೀಜಿ, ನನ್ನನ್ನು ತಮ್ಮ ಹೆಗಲಮೇಲೆ ಹೊತ್ತು ಸಾಗಿ ರಕ್ಷಣೆ ಮಾಡಿದರು. ಇನ್ನೊಂದು ಸಲ ನಮ್ಮ ಪ್ರಯಾಣಕ್ಕೆ ಯಾರೋ ಒಂದು ಕುದುರೆಯನ್ನು ಕೊಟ್ಟಿದ್ದರು. ನಾವಿಬ್ಬರೂ ಆ ಕುದುರೆಯ ಮೇಲೆ ಕುಳಿತು ಒಂದು ನದಿಯನ್ನು ದಾಟಬೇಕಿತ್ತು. ಅದು ಅಷ್ಟೇನೂ ಆಳವಿರಲಿಲ್ಲ; ಆದರೆ ಬಹಳ ರಭಸದಿಂದ ಹರಿಯುತ್ತಿತ್ತು. ಅಲ್ಲದೆ, ನೆಲ ವಂತೂ ವಿಪರೀತ ಜಾರುತ್ತಿತ್ತು. ಆದ್ದರಿಂದ ನನ್ನನ್ನು ಕುದುರೆಯ ಮೇಲೆ ಕುಳ್ಳಿರಿಸಿ, ಸ್ವಾಮೀಜಿ ತಾವು ಮುಂದಿನಿಂದ ಕುದುರೆಯನ್ನು ಎಚ್ಚರಿಕೆಯಿಂದ ನಡೆಸಿಕೊಂಡು ಸಾಗಿದರು. ಹೀಗೆ ಹಲವಾರು ಸಲ ತಮ್ಮ ಪ್ರಾಣವನ್ನೇ ಪಣವಾಗಿಸಿ ನನ್ನನ್ನು ಕಾಪಾಡಿದರು. ಸ್ವಾಮೀಜಿಯನ್ನು ನಾನು ಏನೆಂದು ಬಣ್ಣಿಸಲಿ! ಅವರ ವ್ಯಕ್ತಿತ್ವವನ್ನು ವರ್ಣಿಸಲು ನನಗೆ ತಿಳಿದಿರುವ ಶಬ್ದ ಒಂದೇ–ಪ್ರೇಮ, ಪ್ರೇಮ, ಪ್ರೇಮ. ಅವರ ವ್ಯಕ್ತಿತ್ವವೆಲ್ಲ ಪ್ರೇಮಮಯ. ಕೆಲವು ಸಲ ಕಾಯಿಲೆ ಯಿಂದಾಗಿ ನಾನು ದುರ್ಬಲನಾಗುತ್ತಿದ್ದೆ. ನಡೆಯುವುದಕ್ಕೇ ಕಷ್ಟವಾಗುತ್ತಿತ್ತು. ಆಗ ಸ್ವಾಮೀಜಿಯೇ ನನ್ನ ಸಾಮಾನುಗಳನ್ನೆಲ್ಲ ಹೊತ್ತು ತರುತ್ತಿದ್ದರು. ಆ ಸಾಮಾನುಗಳಲ್ಲಿ ನನ್ನ ಬೂಟುಗಳೂ ಇರುತ್ತಿದ್ದುವು.”

ಮುಂದೆ ಒಂದಾನೊಂದು ಸಂದರ್ಭದಲ್ಲಿ ಶರಚ್ಚಂದ್ರ ಯಾವುದೋ ಕಾರಣದಿಂದಾಗಿ ದುಃಖಗೊಂಡು, “ಸ್ವಾಮೀಜಿ, ನೀವು ನನ್ನ ಕೈಬಿಟ್ಟುಬಿಡುತ್ತೀರಾ?” ಎಂದು ಕೇಳುತ್ತಾನೆ. ಆಗ ಸ್ವಾಮೀಜಿ ಛೇಡಿಸುತ್ತಾರೆ: “ಛಿ ಮೂರ್ಖ, ನಿನ್ನ ಬೂಟುಗಳನ್ನು ಕೂಡ ಹೊತ್ತಿದ್ದೇನೆ ನಾನು, ನೆನಪಿದೆಯಾ?”

ಇನ್ನೊಂದು ಸಲ ಅವರು ಅರಣ್ಯವೊಂದರಲ್ಲಿ ಸಂಚರಿಸುತ್ತಿದ್ದಾಗ, ಅಲ್ಲೊಂದು ಕಡೆ ಮನುಷ್ಯನ ಮೂಳೆಗಳು ಬಿದ್ದಿರುವುದನ್ನು ಕಂಡರು. ಜೊತೆಯಲ್ಲೇ, ಕೊಳೆತು ವಾಸನೆ ಬೀರು ತ್ತಿರುವ ಕಾವಿಬಟ್ಟೆಯ ಚೂರುಗಳು! ಎದೆ ತಲ್ಲಣಿಸುವಂತಹ ದೃಶ್ಯ! ಸ್ವಾಮೀಜಿ ಅವುಗಳನ್ನು ತೋರಿಸುತ್ತ ನುಡಿದರು, “ನೋಡಿಲ್ಲಿ, ಯಾರೋ ಒಬ್ಬ ಸಂನ್ಯಾಸಿಯನ್ನು ಒಂದು ಹುಲಿ ತಿಂದುಹಾಕಿರಬೇಕು! ಏನು, ಭಯವಾಗುತ್ತದೆಯೇ?” ತಕ್ಷಣ, ಶರಚ್ಚಂದ್ರನೆಂದ, “ನೀವಿರು ವಾಗ ಭಯವೆಲ್ಲಿ ಸ್ವಾಮೀಜಿ?” ಅವರಲ್ಲಿ ಅವನಿಗೆ ಅದೆಷ್ಟು ವಿಶ್ವಾಸ! ನಿಜ, ವನರಾಜನೇ ಜೊತೆಯಲ್ಲಿರುವಾಗ ಇತರ ಪ್ರಾಣಿಗಳಿಂದ ಭಯವೆಲ್ಲಿಯದು?

ಹೃಷೀಕೇಶವನ್ನು ತಲುಪಿದ ಸ್ವಾಮೀಜಿ, ಅಲ್ಲಿ ಗುಡಿಸಲೊಂದರಲ್ಲಿ ಇಳಿದುಕೊಂಡು, ಇತರ ಸಾಧು-ಸಂನ್ಯಾಸಿಗಳ ಸಹವಾಸದಲ್ಲಿ ತಮ್ಮ ಶಿಷ್ಯನೊಂದಿಗೆ ಧ್ಯಾನಾದಿಗಳಲ್ಲಿ ನಿರತರಾದರು. ಸುತ್ತಲೂ ಶುಭ್ರ ಹಿಮಾಚ್ಛಾದಿತ ಬೆಟ್ಟಗಳು; ತಿಳಿಯಾಗಿ ಹರಿಯುವ ಗಂಗೆಯ ಕಲರವ; ತಪ ಶ್ಚರ್ಯಕ್ಕೆ ಅತ್ಯಂತ ಸೂಕ್ತವಾದ ಸ್ಥಳ. ಕೆಲವು ದಿನಗಳ ಬಳಿಕ ಅವರು ಬದರಿ-ಕೇದಾರಗಳಿಗೆ ಹೋಗಲು ನಿರ್ಧರಿಸಿದರು. ಹಿಮಾಲಯದ ಉನ್ನತ ಶಿಖರಗಳನ್ನೇರಿ ಹೋಗುವ ಉತ್ಸಾಹ ಅವರಲ್ಲಿ ತುಡಿಯುತ್ತಿತ್ತು. ಆದರೆ ಶರಚ್ಚಂದ್ರನ ಆರೋಗ್ಯ ಸರಿಯಿರಲಿಲ್ಲ. ಪರ್ವತವನ್ನೇರುವ ಕೆಲಸವಂತೂ ಅವನಿಂದ ಸಾಧ್ಯವೇ ಇರಲಿಲ್ಲ. ಆದ್ದರಿಂದ ಅವರು ತಮ್ಮ ಬಯಕೆಯನ್ನು ಒತ್ತಿ ಹಿಡಿಯಬೇಕಾಯಿತು. ಆದರೆ ಅವರು ಹುಟ್ಟಾಸಂನ್ಯಾಸಿಯಲ್ಲವೆ? ಸಂನ್ಯಾಸಿಗಳಿಗೂ ಹಿಮಾ ಲಯದ ಶಿಖರಗಳಿಗೂ ಯುಗಯುಗಗಳ ನಂಟು. ಕೆಲದಿನಗಳಲ್ಲೇ ಆ ಅಭಿಲಾಷೆ ಅವರಲ್ಲಿ ಮತ್ತೊಮ್ಮೆ ಜಾಗೃತವಾಯಿತು. ಇನ್ನು ತಡೆಯಲಾರದೆ ಒಂದು ದಿನ ಇದ್ದಕ್ಕಿದ್ದಂತೆ, “ಛೆ!ನೀನು ನನ್ನ ಕಾಲಿಗೊಂದು ಸಂಕೋಲೆಯ ತರಹ ಆಗಿಬಿಟ್ಟೆ! ನಾನೊಬ್ಬನೇ ಹಾಯಾಗಿ ಅಲೆದಾಡಿ ಕೊಂಡಿದ್ದೆ. ಈಗ ನೀನೊಬ್ಬ ಬಂದು ನನಗೆ ಒಳ್ಳೇ ತಲೆನೋವಾಗಿಬಿಟ್ಟೆ! ಸಾಕಾಯಿತು! ನಾನು ನನ್ನ ದಾರಿಹಿಡಿದು ಹೊರಟು ಹೋಗುತ್ತೇನೆ. ಇನ್ನು ನನ್ನಿಂದ ಇಲ್ಲಿರಲು ಸಾಧ್ಯವಿಲ್ಲ” ಎಂದು ಚೆನ್ನಾಗಿ ರೇಗಾಡಿ, ದಂಡ-ಕಮಂಡಲು ಹಿಡಿದುಕೊಂಡು ಹೊರಟೇಬಿಟ್ಟರು. ಆಗ ಕಿಛುಡಿ (ತೊವ್ವೆ-ಅನ್ನ) ತಯಾರಿಸುತ್ತಿದ್ದ ಶರಚ್ಚಂದ್ರ ಒಂದು ಮಾತನ್ನೂ ಆಡದೆ ಮಂಕುಬಡಿದವನಂತೆ ಕುಳಿತುಬಿಟ್ಟ.

ಸುಮಾರು ಮೂರು-ನಾಲ್ಕು ಗಂಟೆ ಕಳೆದಿರಬಹುದು; ಸ್ವಾಮೀಜಿ ವಾಪಸು ಬಂದುಬಿಟ್ಟರು. ಶಿಷ್ಯನ ಹಿಂದೆ ನಿಂತುಕೊಂಡು ಮೃದುದನಿಯಲ್ಲಿ ನುಡಿದರು, “ಓ ಶರತ್, ನನಗೇನಾದರೂ ಸ್ವಲ್ಪ ತಿನ್ನಲು ಕೊಡುತ್ತೀಯಾ? ನನಗೆ ಭಯಂಕರ ಹಸಿವು!”

ಶರಚ್ಚಂದ್ರನಿಗೆ ಹೋದ ಜೀವ ಬಂದಂತಾಯಿತು. “ಖಂಡಿತ, ಸ್ವಾಮೀಜಿ, ಈ ಕ್ಷಣ ಕೊಟ್ಟು ಬಿಟ್ಟೆ!” ಎಂದುದ್ಗರಿಸಿದ.

ಸ್ವಾಮೀಜಿ: “ನೀನಿನ್ನೂ ಊಟ ಮಾಡಿಲ್ಲವೆ?”

ಶರಚ್ಚಂದ್ರ: “ನೀವು ಇಲ್ಲಿಲ್ಲದ ಮೇಲೆ ಹೇಗೆ ಮಾಡಲಿ?”

ಸ್ವಾಮೀಜಿ: “ನಿಜವಾಗಿಯೂ ನೀನು ನನ್ನ ಕಾಲಿಗೊಂದು ಸಂಕೋಲೆಯಾಗಿಬಿಟ್ಟೆ ಕಣಯ್ಯಾ. ನಾನು ಸಾಕಷ್ಟು ದೂರ ಹೋಗಿದ್ದೆ. ಆದರೆ ಇಲ್ಲೊಬ್ಬ ಮೂರ್ಖ ಒಬ್ಬಂಟಿಗನಾಗಿದ್ದಾನಲ್ಲ ಎನ್ನುವುದು ನೆನಪಾಯಿತು. ಏನು ಮಾಡಿಕೊಂಡುಬಿಡುತ್ತಾನೋ ಎಂದು ಯೋಚನೆಯಾಯಿತು. ನೋಡು, ನಾನು ನಿನಗೋಸ್ಕರ ಹಿಂದಿರುಗಿ ಬಂದೆ.”

ಬಳಿಕ ಶಿಷ್ಯ ಕಿಛುಡಿಯನ್ನು ಬಡಿಸಿದ. ಗುರುಶಿಷ್ಯರಿಬ್ಬರೂ ಕುಳಿತುಕೊಂಡು ಆನಂದದಿಂದ ಊಟ ಮಾಡಿದರು.

ಸ್ವಾಮೀಜಿ ಹಿಂದುರಿಗಿ ಬಂದದ್ದು ಒಳ್ಳೆಯದೇ ಆಯಿತೆನ್ನಬೇಕು. ಏಕೆಂದರೆ ಮೊದಲೇ ಆರೋಗ್ಯ ಸರಿಯಿಲ್ಲದ್ದರಿಂದ ಶರಚ್ಚಂದ್ರ ತೀವ್ರ ಕಾಯಿಲೆಗೆ ಗುರಿಯಾದ. ಅಲ್ಲಿ ಸರಿಯಾದ ಪಥ್ಯೋಪಚಾರಗಳಿಗೆ ಅವಕಾಶವಿಲ್ಲದ್ದರಿಂದ, ಬೇರೆ ಉಪಾಯಗಾಣದೆ ಸ್ವಾಮೀಜಿ ಪುನಃ ಅವನನ್ನು ಹತ್ರಾಸಿಗೇ ಕರೆತರಬೇಕಾಯಿತು. ಅಂತೂ ಅವರ ಬದರೀ-ಕೇದಾರದ ಆಲೋಚನೆ ಯನ್ನು ಕೈಬಿಡಬೇಕಾಯಿತು. ಆದರೆ ಹತ್ರಾಸಿಗೆ ತಲುಪಿದಾಗ ಸ್ವತಃ ಸ್ವಾಮೀಜಿಯೇ ಮಲೇರಿಯಾ ರೋಗಪೀಡಿತರಾದರು. ಗುರುಶಿಷ್ಯರಿಬ್ಬರೂ ಕಾಯಿಲೆಯಿಂದ ಸೊರಗಿ ಸಣ್ಣಗಾಗಿಬಿಟ್ಟರು. ಬಹಳವೇ ಕಷ್ಟವಾಯಿತು. ಈ ವರ್ತಮಾನ, ಹೇಗೋ ಏನೋ, ಬಾರಾನಗೋರ್ ಮಠದ ಸೋದರಸಂನ್ಯಾಸಿಗಳಿಗೆ ತಲುಪಿತು. ಅವರು ಸ್ವಾಮೀಜಿಗೆ ಪತ್ರಗಳನ್ನು ಬರೆದು ಕೂಡಲೇ ಮಠಕ್ಕೆ ಹಿಂದಿರುಗಿ ಬರುವಂತೆ ವಿನಂತಿಸಿಕೊಂಡರು. ಆ ಸಮಯಕ್ಕೆ, ತೀರ್ಥಯಾತ್ರೆ ಮಾಡುತ್ತ ಹತ್ರಾಸಿಗೆ ಬಂದ ಸ್ವಾಮಿ ಶಿವಾನಂದರಿಗೆ ಸ್ವಾಮೀಜಿ ಕಾಯಿಲೆ ಬಿದ್ದಿರುವ ವರ್ತಮಾನ ಸಿಕ್ಕಿತು. ಅವರು ತಮ್ಮ ತೀರ್ಥಯಾತ್ರೆಯನ್ನು ಅಲ್ಲಿಗೆ ಮೊಟಕುಗೊಳಿಸಿ, ಸ್ವಾಮೀಜಿಯನ್ನು ಕರೆದು ಕೊಂಡು ಬಾರಾನಗೋರ್ ಮಠಕ್ಕೆ ಹಿಂದಿರುಗಿದರು. ಹತ್ರಾಸಿನಿಂದ ಹೊರಡುವಾಗ ಸ್ವಾಮೀಜಿ, “ಸ್ವಲ್ಪ ಸುಧಾರಿಸಿಕೊಂಡು ತಕ್ಷಣ ಬಾರಾನಗೋರ್ ಮಠಕ್ಕೆ ಬಂದು ಬಿಡು” ಎಂದು ಶರಚ್ಚಂದ್ರ ನಿಗೆ ಆದೇಶಿಸಿದರು. ಅದರಂತೆ ಅವನು ತನ್ನ ಕೆಲಸಕ್ಕೆ ರಾಜೀನಾಮೆ ನೀಡಿ, ಆರೋಗ್ಯ ಸುಧಾರಿಸಿ ದೊಡನೆ ಬಾರಾನಗೋರ್ ಮಠಕ್ಕೆ ಬಂದು ಬಿಟ್ಟ. ಇವನೇ ಸ್ವಾಮಿ ವಿವೇಕಾನಂದರ ಮೊದಲ ಶಿಷ್ಯ. ಮಠದಲ್ಲಿ ಎಲ್ಲರೂ ಇವನನ್ನು ಭ್ರಾತೃವಾತ್ಸಲ್ಯದಿಂದ ಸ್ವಾಗತಿಸಿ ಬರಮಾಡಿ ಕೊಂಡರು. ಸ್ವಾಮೀಜಿ ಅವನಿಗೆ ಸಂನ್ಯಾಸದೀಕ್ಷೆ ನೀಡಿ, ‘ಸ್ವಾಮಿ ಸದಾನಂದ’ ಎಂಬ ಹೆಸರನ್ನಿತ್ತರು.

ಸ್ವಾಮೀಜಿ ಈ ಬಾರಿ ಬಾರಾನಗೋರ್ ಮಠಕ್ಕೆ ಹಿಂದಿರುಗಿದಾಗ (೧೮೮೮ರ ಅಂತ್ಯ ಭಾಗ) ಅನೇಕ ಸೋದರಸಂನ್ಯಾಸಿಗಳು ತೀರ್ಥಾಟನೆಗೆ ಹೊರಟುಹೋಗಿದ್ದರು. ಆದರೆ ಉಳಿದ ಗುರು ಭಾಯಿಗಳೂ ಗೃಹಸ್ಥಭಕ್ತರೂ ತಮ್ಮ ನೆಚ್ಚಿನ ‘ನರೇನ’ನನ್ನು ಕಂಡು ಬಹಳ ಸಂತಸಗೊಂಡರು. ಈ ಸಲ ಸ್ವಾಮೀಜಿ ಸುಮಾರು ಒಂದು ಇಡೀ ವರ್ಷವನ್ನು ಬಾರಾನಗೋರ್ ಮಠದಲ್ಲೇ ಕಳೆಯಲಿದ್ದರು.

ಈ ಅವಧಿಯಲ್ಲಿ ಸ್ವಾಮೀಜಿ, ಎರಡು ಚಿಕ್ಕ ಪ್ರಯಾಣಗಳನ್ನು ಕೈಗೊಂಡರು. ಒಂದು ಸಲ ಅವರು ಶ್ರೀಮಾತೆ ಶಾರದಾದೇವಿಯವರು ಹಾಗೂ ಕೆಲವು ಸೋದರ ಸಂನ್ಯಾಸಿಗಳೊಂದಿಗೆ ಕಾಮಾರಪುಕುರಕ್ಕೆ ಹೊರಟರು. ಆದರೆ ದಾರಿಯಲ್ಲಿ ತಮಗೆ ಜ್ವರ ಹಾಗೂ ವಾಂತಿ ಶುರುವಾದ್ದ ರಿಂದ ಸ್ವಾಮೀಜಿ ಕಲ್ಕತ್ತಕ್ಕೆ ಹಿಂದಿರುಗಬೇಕಾಯಿತು. ಹೋಮಿಯೋಪಥಿ ಚಿಕಿತ್ಸೆಯಿಂದ ಸ್ವಲ್ಪ ಮಟ್ಟಿಗೆ ಸುಧಾರಿಸಿಕೊಂಡರು.

ಆದರೂ ಅವರು ತೀರಾ ನಿಶ್ಶಕ್ತರಾಗಿಬಿಟ್ಟಿದ್ದರು. ಆದ್ದರಿಂದ ೧೮೮೯ರ ಬೇಸಿಗೆಯಲ್ಲಿ ವೈದ್ಯನಾಥದ ಬಳಿಯಿರುವ ಸಿಮುಲ್ತಾಲಾ ಎಂಬಲ್ಲಿಗೆ ಹೋದರು. ಇಲ್ಲಿನ ಹವೆ ತುಂಬ ಆರೋಗ್ಯಕರವಾದದ್ದೆಂದು ಪ್ರಸಿದ್ಧ. ಆದರೆ ಕೆಲವೇ ದಿನಗಳಲ್ಲಿ, ಅಲ್ಲಿ ಅವರಿಗೆ ಆಮಶಂಕೆ ಪ್ರಾರಂಭವಾದ್ದರಿಂದ ಮತ್ತೆ ಮಠಕ್ಕೆ ಮರಳಿದರು.

