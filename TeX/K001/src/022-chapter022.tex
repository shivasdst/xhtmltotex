
\chapter{ವಿಶ್ವದ ವೈಶಾಲ್ಯದೆಡೆಗೆ}

\noindent

ಸೋದರಸಂನ್ಯಾಸಿಗಳ ಸ್ನೇಹಪಾಶವನ್ನು ಕಡಿದುಕೊಂಡು ಹೊರಟ ಸ್ವಾಮೀಜಿ ದೆಹಲಿಗೆ ಬಂದು ತಲುಪಿದರು. ಶತಮಾನಗಳಿಂದಲೂ ಹಿಂದೂ ಹಾಗೂ ಮೊಘಲ್ ರಾಜರ ರಾಜಧಾನಿಯಾಗಿದ್ದ ನಗರ ದೆಹಲಿ. ಅನೇಕಾನೇಕ ಚಾರಿತ್ರಿಕ ಘಟನೆಗಳ ತಾಣ. ರಾಜ ಮಹಾರಾಜರ ಗೋರಿಗಳು, ಅರಮನೆಗಳು, ರಾಜವೈಭವದ ಅವಶೇಷಗಳು–ಇವು ದೆಹಲಿಯನ್ನು ‘ಭಾರತದ ಪುರಾತನ ರೋಮ್’ ಎಂಬಂತೆ ಮಾಡಿವೆ. ಇಲ್ಲಿ ಬೀಸುತ್ತಿರುವುದೇ ರಾಜತ್ವದ-ಚಕ್ರಾಧಿಪತ್ಯದ ಗಾಳಿ ಎನ್ನಬಹುದು. ಈಗ ಇಲ್ಲಿಗೆ ಬಂದಿರುವ ಸ್ವಾಮೀಜಿಯದೂ ರಾಜಠೀವಿಯೇ–ಆದರೆ ಅವರ ಬಳಿಯಿರುವ ಸರ್ವಸ್ವವೆಂದರೆ ದಂಡ-ಕಮಂಡಲುಗಳು ಹಾಗೂ ಒಂದು ಪುಟ್ಟ ಗಂಟು ಮಾತ್ರ!

ಇಲ್ಲಿ ಅವರು ಸೇಠ್ ಶ್ಯಾಮಲ್ ದಾಸ್ ಎಂಬಾತನ ಅತಿಥಿಯಾಗಿ ಉಳಿದುಕೊಂಡರು. ಅವರು ದೆಹಲಿಯಲ್ಲೆಲ್ಲ ಸುತ್ತಾಡಿ, ನೋಡಬೇಕಾದ ಸ್ಥಳಗಳೆಲ್ಲವನ್ನೂ ನೋಡಿದರು. ದೆಹಲಿಯ ಮಹಾ ಚಕ್ರಾಧಿಪತ್ಯಗಳ ಅವಶೇಷಗಳು ಈ ಯುವ ಸಂನ್ಯಾಸಿಯ ಮನಃಪಟಲದ ಮೇಲೆ ಸಮಸ್ತ ಮಾನುಷವೈಭವಗಳ ಕ್ಷಣಭಂಗುರತೆಯನ್ನು ಮುದ್ರೆಯೊತ್ತಿದುವು! ಅಲ್ಲದೆ ಅನಾದಿಯಾದ, ಅನಂತವಾದ ಆತ್ಮನ ಚಿರಸಾಮ್ರಾಜ್ಯದ ಬಗೆಗಿನ ಅವರ ಶ್ರದ್ಧೆ ಮತ್ತಷ್ಟು ಬಲಗೊಂಡಿತು.

ಸ್ವಾಮೀಜಿ ಸರ್ವಸಂಗಪರಿತ್ಯಾಗಿಯಾದ ಸಂನ್ಯಾಸಿಯೂ ಹೌದು; ಜೊತೆಗೆ ಉಜ್ವಲ ರಾಷ್ಟ್ರ ಪ್ರೇಮಿಯೂ ಇತಿಹಾಸಕಾರರೂ ಹೌದು. ಅವರ ದೃಷ್ಟಿಗೆ ಈ ನಗರವು ಭವ್ಯ ಭಾರತೀಯ ಸಂಸ್ಕೃತಿಯ ಸಂಕೇತವಾಗಿ ಕಾಣುತ್ತಿದೆ! ಒಂದೊಂದು ಗೋರಿಯೂ ಒಂದೊಂದು ಅರ ಮನೆಯೂ ಅವರ ಮನದಲ್ಲಿ ಹಲವಾರು ನೆನಪುಗಳನ್ನು, ಭಾವನೆಗಳನ್ನು ಹೊಮ್ಮಿಸುತ್ತಿವೆ! ಒಂದೊಂದು ದೃಶ್ಯವೂ ಅವರಿಗೆ ಒಂದೊಂದು ಪಾಠವನ್ನು ಕಲಿಸುತ್ತಿದೆ!

ಹೀಗೆ ಸ್ವಾಮೀಜಿ ದೆಹಲಿಯಲ್ಲಿ ಏಕಾಂಗಿಯಾಗಿ ಸುತ್ತಾಡುತ್ತ ತಮ್ಮ ಭಾವನೆಗಳಲ್ಲಿ, ಕಲ್ಪನೆ ಗಳಲ್ಲಿ, ಯೋಜನೆಗಳಲ್ಲಿ ಮುಳುಗಿದ್ದಾಗ ಅನಿರೀಕ್ಷಿತವಾಗಿ ಅವರಿಗೆ ತಾವು ಮೀರತ್ತಿನಲ್ಲಿ ಬೀಳ್ಗೊಂಡಿದ್ದ ಗುರುಭಾಯಿಗಳ ಭೇಟಿಯಾಯಿತು! ಆ ಗುರುಭಾಯಿಗಳಿಗೂ ಇದು ಅಷ್ಟೇ ಅನಿರೀಕ್ಷಿತ. ಅವರಿಗೇನೋ ಸ್ವಾಮೀಜಿಯನ್ನು ಕಂಡು ತುಂಬ ಸಂತೋಷವಾಯಿತು. ಆದರೆ, ಅವರು ತಮ್ಮನ್ನು ಬೇಕೆಂದೇ ಹಿಂಬಾಲಿಸಿ ಬರುತ್ತಿದ್ದಾರೆ ಎಂದು ಭಾವಿಸಿದ ಸ್ವಾಮೀಜಿ ನುಡಿ ದರು: “ನೋಡಿ, ನಾನಾಗಲೇ ನಿಮಗೆ ಸ್ಪಷ್ಟವಾಗಿ ಹೇಳಿದ್ದೇನೆ–ನಾನು ಮಾಡಬೇಕಾದ ಕೆಲಸವಿದೆ; ಆದ್ದರಿಂದ ಯಾರೂ ನನ್ನನ್ನು ಹಿಂಬಾಲಿಸಿ ಬರಬೇಡಿ; ನನ್ನಷ್ಟಕ್ಕೆ ನನ್ನನ್ನು ಬಿಟ್ಟುಬಿಡಿ ಎಂದು. ಈಗ ಮತ್ತೆ ಒತ್ತಿಹೇಳುತ್ತಿದ್ದೇನೆ–ನನ್ನ ಮಾತಿನಂತೆ ನೀವು ನಡೆಯಬೇಕು. ಈಗ ನಾನು ದೆಹಲಿಯಿಂದ ಹೊರಡುತ್ತಿದ್ದೇನೆ. ನನ್ನನ್ನು ನೀವು ಯಾರೂ ಹಿಂಬಾಲಿಸಿ ಬರಕೂಡದು. ಇನ್ನು ಇಷ್ಟರಮೇಲೂ ಯಾರಾದರೂ ಹಿಂಬಾಲಿಸಿ ಬಂದದ್ದೇ ಆದರೆ ಅವರು ಅಪಾಯಕ್ಕೆ ಸಿದ್ಧರಾಗಿ ಬರಬೇಕಾಗುತ್ತದೆ, ಅಷ್ಟೆ. ಏಕೆಂದರೆ ನಾನು ನನ್ನ ಎಲ್ಲ ಹಳೆಯ ಸಂಬಂಧಗಳನ್ನು ಕಡಿದುಕೊಂಡು ಬಿಡುವವನಿದ್ದೇನೆ. ನನ್ನ ಆತ್ಮ ನನ್ನನ್ನು ಎಲ್ಲಿಗೆ ಕರೆದೊಯ್ದರೆ ಅಲ್ಲಿಗೆ ಹೋಗುತ್ತೇನೆ. ಅದು ಅರಣ್ಯವಾಗಲಿ, ಮರುಭೂಮಿಯಾಗಲಿ, ಅಥವಾ ಜನದಟ್ಟಣೆಯಿಂದ ಕೂಡಿದ ನಗರವಾಗಲಿ– ಎಲ್ಲೆಂದರಲ್ಲಿಗೆ ಹೋಗುತ್ತಿರುತ್ತೇನೆ... ನಾನು ಹೊರಟೆ.”

ಅವರ ಮಾತಿನಲ್ಲಿ ವ್ಯಕ್ತವಾದ ದೃಢನಿರ್ಧಾರವನ್ನು ಕಂಡು ಸೋದರಸಂನ್ಯಾಸಿಗಳೆಲ್ಲ ಚಕಿತ ರಾಗಿ ನುಡಿದರು: “ನರೇನ್, ನಿಜಕ್ಕೂ ನೀನು ಇಲ್ಲಿರುವುದೇ ನಮಗೆ ಗೊತ್ತಿರಲಿಲ್ಲ. ನಾವು ದೆಹಲಿಯನ್ನು ನೋಡಲು ಇಲ್ಲಿಗೆ ಬಂದೆವು ಅಷ್ಟೇ. ಆದರೆ ಇಲ್ಲಿಗೆ ಬಂದಾಗ ನಮಗೆ ತಿಳಿದುಬಂತು–ಇಲ್ಲೊಬ್ಬರು ‘ಸ್ವಾಮಿ ವಿವಿದಿಶಾನಂದ’ ಎಂಬ ಹೆಸರಿನ ಇಂಗ್ಲಿಷ್ ಮಾತನಾ ಡುವ ಸ್ವಾಮಿಗಳಿದ್ದಾರೆ ಎಂದು. ಅವರು ಯಾರೋ ನೋಡಬೇಕು ಎಂಬ ಕುತೂಹಲದಿಂದ ನಾವು ಬಂದರೆ, ಇಲ್ಲಿ ನೀನು ಭೇಟಿಯಾದೆ!”

ಈಗ ಸ್ವಾಮೀಜಿ ಸಮಾಧಾನಗೊಂಡು ಸುಮ್ಮನಾದರು.

ಆ ಗುರುಭಾಯಿಗಳ ಮಾತಿನಿಂದ, ಸ್ವಾಮೀಜಿ ತಮ್ಮ ಹೆಸರನ್ನು ‘ಸ್ವಾಮಿ ವಿವಿದಿಶಾನಂದ’ ಎಂದು ಹೇಳಿಕೊಂಡಿದ್ದರೆಂಬುದು ತಿಳಿದುಬರುತ್ತದೆ. ಸಂನ್ಯಾಸ ಸ್ವೀಕಾರ ಮಾಡಿದಾಗ ಅವರು ತಮ್ಮ ಹೆಸರನ್ನು ‘ಸ್ವಾಮಿ ವಿವೇಕಾನಂದ’ ಎಂದಿಟ್ಟುಕೊಂಡಿದ್ದರೂ ಈ ಹೆಸರು ಅವರ ಆಪ್ತ ರಾರಿಗೂ ರೂಢಿಯಾಗಿರಲಿಲ್ಲ. ಏಕೆಂದರೆ ಈ ಸಂನ್ಯಾಸಿಗಳೆಲ್ಲ ಇನ್ನೂ ತಮ್ಮನ್ನು ಪೂರ್ವಾ ಶ್ರಮದ ಹೆಸರುಗಳಿಂದಲೇ ಕರೆದುಕೊಳ್ಳುತ್ತಿದ್ದರು. ಅವರ ಪಾಲಿಗೆ ಸ್ವಾಮೀಜಿ ‘ನರೇನ್ ಭಾಯ್’ ಅಷ್ಟೆ. ಇನ್ನು ‘ವಿವಿದಿಶಾನಂದ’ ಎಂಬ ಹೆಸರನ್ನಂತೂ ಅವರು ಕೇಳಿಯೇ ಇರಲಾರರು. ಆದ್ದರಿಂದ, ಈ ‘ಇಂಗ್ಲಿಷ್ ಮಾತನಾಡುವ ಸಂನ್ಯಾಸಿ’ ಯಾರಿರಬಹುದು ಎಂದು ನೋಡ ಬಂದರೆ, ಅದು ‘ನರೇಂದ್ರ’ನೇ ಆಗಿರಬೇಕೆ!

ಸ್ವಾಮೀಜಿ ‘ನಾನಿನ್ನು ಹೊರಟೆ’ ಎಂದು ಹೇಳಿದರಾದರೂ ದೆಹಲಿಯಲ್ಲೇ ಇನ್ನೂ ಕೆಲವು ದಿನ ಉಳಿದುಕೊಂಡರು. ಊಟದ ವೇಳೆಯಲ್ಲಿ ಮಾತ್ರ ಸ್ವಾಮೀಜಿ ಹಾಗೂ ಅವರ ಸೋದರಸಂನ್ಯಾಸಿ ಗಳು ಸೇಠ್ ಶ್ಯಾಮಲ್ದಾಸನ ಮನೆಯಲ್ಲಿ ಒಟ್ಟಾಗಿ ಸೇರುತ್ತಿದ್ದರು.

ದೆಹಲಿಯಲ್ಲಿ ಸ್ವಾಮೀಜಿ, ತಮ್ಮ ಗಂಟಲ ತೊಂದರೆಯೊಂದರ ಸಂಬಂಧವಾಗಿ ಅಲ್ಲಿನ ಪ್ರಸಿದ್ಧ ಬಂಗಾಳೀ ವೈದ್ಯನಾದ ಡಾ. ಹೇಮಚಂದ್ರ ಸೇನ್ ಎಂಬವನನ್ನು ಕಂಡಿದ್ದರು. ಅದೇಕೋ ಈತ ಸ್ವಾಮೀಜಿಯ ವಿಷಯದಲ್ಲಿ ಸಂಪೂರ್ಣ ವಿರುದ್ಧವಾಗಿದ್ದ. ಅಲ್ಲದೆ, ಅಖಂಡಾನಂದರ ಬಳಿ, ಸ್ವಾಮೀಜಿಯ ಬಗ್ಗೆ ಹಗುರವಾಗಿ ಮಾತನಾಡಿದ್ದ. ಇದಾದ ಕೆಲವು ದಿನಗಳ ಬಳಿಕ ಆತ ಅಖಂಡಾನಂದರನ್ನು ಕಂಡು, ಮತ್ತೊಮ್ಮೆ ಸ್ವಾಮೀಜಿಯನ್ನು ಭೇಟಿ ಮಾಡುವ ಇಚ್ಛೆಯನ್ನು ವ್ಯಕ್ತಪಡಿಸಿ, ಒಂದು ಸಂಜೆ ಸ್ವಾಮೀಜಿ ಹಾಗೂ ಅವರ ಸೋದರಸಂನ್ಯಾಸಿಗಳನ್ನು ತನ್ನ ಮನೆಗೆ ಆಹ್ವಾನಿಸಿದ. ಜೊತೆಗೆ ಹಲವಾರು ಕಾಲೇಜು ಪ್ರಾಧ್ಯಾಪಕರನ್ನೂ ಆಹ್ವಾನಿಸಿದ್ದ. ಎಲ್ಲರೂ ಸೇರಿದ್ದ ಈ ಸಂದರ್ಭದಲ್ಲಿ ದೊಡ್ಡದೊಂದು ಚರ್ಚೆ ಪ್ರಾರಂಭವಾಯಿತು. ಸುಶಿಕ್ಷಿತರಾದ ಇವರು ಸ್ವಾಮೀಜಿಯನ್ನು ಹಲವಾರು ವಿಷಯಗಳ ಬಗ್ಗೆ ಆಳವಾಗಿ ಪ್ರಶ್ನಿಸಿದರು. ಸ್ವಾಮೀಜಿ ತಮ್ಮ ಸಹಜ ವಾಗ್ವೈಖರಿಯಿಂದ, ವಿದ್ವತ್ಪೂರ್ಣ ಉತ್ತರಗಳಿಂದ ಎಲ್ಲರ ಮೇಲೂ ಪ್ರಭಾವ ಬೀರಿ ದರು. ಸ್ವಾಮೀಜಿ ಹಾಗೂ ಅವರ ಗುರುಭಾಯಿಗಳಿಂದ ವಿಶೇಷವಾಗಿ ಆಕರ್ಷಿತನಾದ ಡಾ. ಸೇನ್ ಅವರನ್ನು ಮರುದಿನ ತನ್ನ ಮನೆಗೆ ಔತಣಕ್ಕೆ ಆಹ್ವಾನಿಸಿ ಸತ್ಕರಿಸಿದ.

ಬಳಿಕ ಕೆಲವೇ ದಿನಗಳಲ್ಲಿ ಸೋದರಸಂನ್ಯಾಸಿಗಳೆಲ್ಲ ಒಬ್ಬೊಬ್ಬರಾಗಿ ದೆಹಲಿಯಿಂದ ಹೊರ ಟರು. ಬ್ರಹ್ಮಾನಂದರು ಹಾಗೂ ತುರೀಯಾನಂದರು ಪಂಜಾಬಿನತ್ತ ಹೊರಟರು. ಶಾರದಾ ನಂದರು ಎಟಾವಾ ಎಂಬಲ್ಲಿಗೆ ಹೋದರು. ಶಾರದಾನಂದರು ಅನಾರೋಗ್ಯಪೀಡಿತರಾಗಿದ್ದರಿಂದ ಕೃಪಾನಂದರೂ ಅವರೊಂದಿಗೆ ಹೋದರು. ಅಖಂಡಾನಂದರು ಬೃಂದಾವನದ ಕಡೆಗೆ ನಡೆದರು. ಆದರೆ ತಾವು ಮಾತ್ರ ಸ್ವಾಮೀಜಿಯ ಬೆನ್ನು ಬಿಡಲಾರೆವು ಎಂದು ಮನದಲ್ಲೇ ನಿರ್ಧರಿಸಿರಬೇಕು. ಸ್ವಾಮೀಜಿ ರಜಪುತಾನಕ್ಕೆ (ಈಗಿನ ರಾಜಸ್ಥಾನ) ಹೊರಟರು. ಅವರು ಯಾವ ಕಡೆಗೆ ಹೊರಟಿ ದ್ದಾರೆ ಎಂಬುದು ಅವರ ಗುರುಭಾಯಿಗಳಿಗೆ ತಿಳಿದುಬಂದಿತು. ಅವರನ್ನು ಹಿಂಬಾಲಿಸುವ ಇಚ್ಛೆಯೂ ಇದ್ದೇ ಇದ್ದಿತಾದರೂ, ಹಾಗೆ ಮಾಡಲು ಧೈರ್ಯ ಸಾಲಲಿಲ್ಲ. ತಮ್ಮ ನಾಯಕನಾದ ನರೇಂದ್ರನ ಹೃದಯವು, ತಾನು ಪೂರೈಸಬಂದ ದಿವ್ಯೋದ್ದೇಶದ ಸಾಧನೆಗಾಗಿ ತೀವ್ರ ವ್ಯಾಕುಲ ಗೊಂಡಿದೆ; ಮತ್ತು, ಅವನನ್ನು ಶ್ರೀರಾಮಕೃಷ್ಣರೇ ಮಾರ್ಗದರ್ಶನ ಮಾಡುತ್ತ ಮುನ್ನಡೆಸುತ್ತಿ ದ್ದಾರೆ; ಆದ್ದರಿಂದ ಜಗನ್ಮಾತೆಯ ಇಚ್ಛೆಯಂತೆ ಅವನನ್ನೀಗ ಅವನಷ್ಟಕ್ಕೆ ಬಿಟ್ಟುಬಿಡಬೇಕು ಎಂಬುದು ಅವರಿಗೆ ಮನವರಿಕೆಯಾಗಿತ್ತು. ಹನಿಗೂಡಿದ ಕಂಗಳಿಂದ, ಭಾರವಾದ ಹೃದಯದಿಂದ ಅವರು ತಮ್ಮ ನರೇನನನ್ನು ಬೀಳ್ಕೊಟ್ಟರು. ಆದರೆ ಸ್ವಾಮೀಜಿ ಮಾತ್ರ ಸಂತೋಷಚಿತ್ತರಾಗಿ ತಮ್ಮ ಹೃದಯದ ಕೊನೆಯ ಬಂಧನವಾದ ಸೋದರಸಂನ್ಯಾಸಿಗಳ ಪ್ರೀತಿಯಿಂದ ಬಿಡಿಸಿಕೊಂಡು ಪರಿವ್ರಾಜಕರಾಗಿ ಮುನ್ನಡೆದರು. ಅವರಿಗೀಗ ‘ಧಮ್ಮಪದ’ದ ಸಾಲುಗಳು ನೆನಪಾಗುತ್ತಿವೆ:

\begin{myquote}
ದಾರಿಯಲ್ಲದ ದಾರಿಯಲ್ಲೂ ಸಾಗು ಧೀರನೆ ಮುಂದಕೆ!\\ಅಂಜದಳುಕದೆ ನಿಖಿಲ ಭುವನಂಗಳನೆ ಗಣನೆಗೆ ತಾರದೆ,\\ಖಡ್ಗಮೃಗದೊಲು ನುಗ್ಗಿ ನಡೆ ನೀ ಒಂಟಿಯಾಗಿಯೆ ಪಥದಲಿ!\\ಪ್ರಳಯಗರ್ಜನೆ ಸಿಡಿಲೆ ಬಡಿದರು ನಡುಗದಿಹ ಕೇಸರಿಯೊಲು,\\ಗಗನಬಲೆಯನೆ ಬೀಸಿ ಎಸೆದರು ಸಿಲುಕದಿಹ ಮಾರುತನೊಲು,\\ನೀರಿನೊಳಗೇ ನೀರ ಮೀರಿದ ಪದ್ಮಪತ್ರದ ತೆರದಲಿ,\\ವೀರಕೇಸರಿ, ನುಗ್ಗಿ ನಡೆ ಏಕಾಂಗಿಯಾಗಿಯೆ ಪಥದಲಿ!
\end{myquote}

\noindent

ಬುದ್ಧಭಗವಂತನ ಈ ನುಡಿಗಳು ಅವರಿಗೊಂದು ನವಸ್ಫೂರ್ತಿಯನ್ನು ನೀಡಿದುವು. ಎಲ್ಲ ಬಂಧನಗಳನ್ನೂ ಕಳಚಿಕೊಂಡು, ಎಲ್ಲ ‘ಗಂಟು’ಗಳನ್ನೂ ಬಿಡಿಸಿಕೊಂಡು, ಎಲ್ಲ ಮಿತಿಗಳನ್ನೂ ಅತಿಕ್ರಮಿಸಿ ಮುನ್ನಡೆದರು ಸ್ವಾಮೀಜಿ, ಧೀರಗಂಭೀರ ಕೇಸರಿಯಂತೆ! ಶ್ರೀರಾಮನ ಕಾರ್ಯವನ್ನು ಸಾಧಿಸಹೊರಟ ಮಹಾವೀರನಂತೆ!!

\begin{myquote}
ನಿಜವನರಿತವರೆಲ್ಲೊ ಕೆಲವರು, ನಗುವರುಳಿದವರೆಲ್ಲರೂ\\ನಿನ್ನ ಕಂಡರೆ, ಹೇ ಮಹಾತ್ಮನೆ! ಕುರುಡರೇನನು ಬಲ್ಲರು?\\ಗಣಿಸದವರನು ಹೋಗು, ಮುಕ್ತನೆ, ನೀನು ಊರಿಂದೂರಿಗೆ\\ಸೊಗವ ಬಯಸದೆ, ಅಳಲಿಗಳುಕದೆ! ಕತ್ತಲಲಿ ಸಂಚಾರಿಗೆ\\ನಿನ್ನ ಬೆಳಕನು ನೀಡೆಲೈ, ಸಂ-\\ಸಾರ ಮಾಯೆಯ ದೂಡೆಲೈ
\end{myquote}

\begin{myquote}
ಇಂತು ದಿನದಿನ ಕರ್ಮಶಕ್ತಿಯು ಮುಗಿವವರೆಗೂ ಸಾಗೆಲೈ\\ನಾನು ನೀನುಗಳಳಿದು, ಆತ್ಮದೊಳಿಳಿದು, ಕಡೆಯೊಳು ಹೋಗೆಲೈ\\ಏಳು, ಮೇಲೇಳೇಳು ಸಾಧುವೆ ಹಾಡು ಚಾಗಿಯ ಹಾಡನು\\ಹಾಡಿನಿಂದೆಚ್ಚರಿಸು ಮಲಗಿಹ ನಮ್ಮ ಈ ತಾಯ್ನಾಡನು\\ತತ್ತ್ವಮಸಿ ಎಂದರಿತು ಹಾಡೈ, ವೀರಸಂನ್ಯಾಸಿ–\\ಓಂ ತತ್ ಸತ್ ಓಂ!
\end{myquote}

