
\chapter*{ಮೊದಲ ಮಾತು}

\noindent

ಸ್ವಾಮಿ ವಿವೇಕಾನಂದರ ಸಮಗ್ರ ಜೀವನಚರಿತ್ರೆಯ ಮೊದಲ ಸಂಪುಟವಾದ ಈ ಗ್ರಂಥವನ್ನು “ವೀರಸಂನ್ಯಾಸಿ ವಿವೇಕಾನಂದ” ಎಂದು ಹೆಸರಿಸಲಾಗಿದೆ. ಆದರೆ ಈ ಹೆಸರನ್ನು ಸೂಚಿಸಿದಾಗ ನಮ್ಮಲ್ಲೊಬ್ಬರು ನನಗೆ ಹೇಳಿದರು: “ಆ ‘ಸಂನ್ಯಾಸಿ’ ಎನ್ನುವ ಪದಪ್ರಯೋಗ ಬೇಡ. ಏಕೆಂದರೆ ಹೊರಗಡೆ ಜನರಲ್ಲಿ ಸಂನ್ಯಾಸಿ ಎನ್ನುವ ಪದಕ್ಕೆ ಒಳ್ಳೆಯ ಅಭಿಪ್ರಾಯವಿಲ್ಲ. ಜನಗಳು ಈ ಸಂನ್ಯಾಸಿ ಎಂಬ ಪದವನ್ನು ಬೈಯಲು, ತಿರಸ್ಕರಿಸಿ ಮಾತನಾಡಲು ಬಳಸುತ್ತಾರೆ. ಆದ್ದರಿಂದ ಬೇರೇನಾದರೂ ಶಿರೋನಾಮೆ ಇಡಿ.” ಆಗ ನಾನೆಂದೆ: “ಜನ ಕೆಡಿಸದೆ ಇರುವ ಶಬ್ದ ಯಾವುದಿದೆ? ‘ಮಂಗಳಾರತಿ’ ಎನ್ನುವ ಶಬ್ದವನ್ನು ಕೆಡಿಸಲಿಲ್ಲವೆ? ‘ಸಹಸ್ರನಾಮ’ ಎನ್ನುವ ಶಬ್ದವನ್ನು ಕೆಡಿಸಲಿಲ್ಲವೆ? ‘ಪುರಾಣ’ ‘ರಾಮಾಯಾಣ’ವೇ ಮೊದಲಾದ ಹಲವಾರು ಶಬ್ದಗಳನ್ನು ಅಪಪ್ರಯೋಗ ಮಾಡಿ ಕೆಡಿಸಲಿಲ್ಲವೆ? ಹೀಗಿರುವಾಗ ಈ ‘ಸಂನ್ಯಾಸಿ’ ಎಂಬ ಶಬ್ದವೂ ಜನರ ಬಾಯಿಗೆ ಬಲಿಯಾದದ್ದರಲ್ಲಿ ಆಶ್ಚರ್ಯವೇನೂ ಇಲ್ಲ.”

ಅಲ್ಲದೆ ಈ ಸಂನ್ಯಾಸಿ ಎಂಬ ಪದದ ದುರುಪಯೋಗಕ್ಕೆ ಇನ್ನೊಂದು ಕಾರಣವೂ ಇದೆ. ಏನೆಂದರೆ, ಈ ಸಂನ್ಯಾಸಿಗಳಲ್ಲೂ ಕೂಡ ಕೆಲವರು ಅಲ್ಲಲ್ಲಿ ಅರ್ಜುನಸಂನ್ಯಾಸಿಯಂತೆ, ರಾವಣ ಸಂನ್ಯಾಸಿಯಂತೆ ವರ್ತಿಸಿದ್ದನ್ನು ಜನ ಕಂಡಿರುತ್ತಾರೆ. ಆದ್ದರಿಂದ ಸಹಜವಾಗಿಯೇ ಈ‘ಸಂನ್ಯಾಸಿ’ ಎಂಬ ಶಬ್ದವನ್ನೇ ತಿರಸ್ಕರಿಸಿ, ಅಪಪ್ರಯೋಗ ಮಾಡಿ ಬಳಸುವುದಕ್ಕೆ ಸಾಧ್ಯವಿದೆ. ಆದರೆ ಸಂನ್ಯಾಸದ ಘನ ಆದರ್ಶವನ್ನು ತಮ್ಮ ಜೀವನದಲ್ಲಿ ಜಾಜ್ವಲ್ಯಮಾನವಾಗಿ ಬೆಳಗಿದಂತಹ ಶ್ರೀ ಸ್ವಾಮಿ ವಿವೇಕಾನಂದರ ಜೀವನ\eng{-}ಸಂದೇಶಗಳನ್ನು ನಾವು ಅಧ್ಯಯನ ಮಾಡಿದಾಗ ‘ಸಂನ್ಯಾಸಿ’ ಎನ್ನುವ ಈ ಪದದ ಕುರಿತಾದ ಕ್ಷುಲ್ಲಕ ಭಾವನೆ ತಾನೇತಾನಾಗಿ ದೂರವಾಗುತ್ತದೆ.

ನಿಜಕ್ಕೂ ಇದೊಂದು ವಿಚಿತ್ರ ಪರಿಸ್ಥಿತಿ! ಸಂನ್ಯಾಸಿಗಳು ತಮ್ಮನ್ನು ಧೈರ್ಯವಾಗಿ ‘ಸಂನ್ಯಾಸಿ ಗಳು’ ಎಂದು ಕರೆದುಕೊಳ್ಳುವುದಕ್ಕೇ ಸಂಕೋಚಪಟ್ಟುಕೊಳ್ಳುವಂಥ ಪರಿಸ್ಥಿತಿ! ಆದ್ದರಿಂದಲೇ ಸಂನ್ಯಾಸಿಗಳು ತಮ್ಮನ್ನು ‘ಸಾಧುಗಳು’ ಎಂದು ಕರೆದುಕೊಳ್ಳುವುದುಂಟು. ಆದರೆ ‘ಸಾಧು’ ಎನ್ನುವ ಪದಕ್ಕೂ ‘ಸಂನ್ಯಾಸಿ’ ಎನ್ನುವ ಪದಕ್ಕೂ ಬಹಳ ಅರ್ಥವ್ಯತ್ಯಾಸವಿದೆ.‘ಸಾಧು’ ಎಂದರೆ ಒಳ್ಳೆಯವನು, ಸಜ್ಜನ ಎಂದರ್ಥ. ‘ಸಂನ್ಯಾಸಿ’ ಎಂದರೆ ಸರ್ವಸಂಗ ಪರಿತ್ಯಾಗ ಮಾಡಿದವನು ಎಂದರ್ಥ. ಆದ್ದರಿಂದ ಯಾರು ಬೇಕಾದರೂ ಸಾಧುಗಳಾಗಬಹುದು. ಆದರೆ ಯಾರೆಂದರವರು ಸಂನ್ಯಾಸಿಗಳೆನಿಸಲಾರರು. ಗೃಹಸ್ಥರೂ ಸಾಧುಗಳಾಗಿರಬಹುದು, ಎಂದರೆ ಸಜ್ಜನರಾಗಿರಬಹುದು. ಆದರೆ ಅವರು ಸರ್ವಸಂಗಪರಿತ್ಯಾಗ ಮಾಡಿದ ಸಂನ್ಯಾಸಿಗಳೆನಿಸಲು ಸಾಧ್ಯವಿಲ್ಲ. ಆದರೆ ಸಂನ್ಯಾಸಿಗಳನ್ನು ಸಾಧುಗಳೆಂದು ಕರೆಯಬಹುದು. ಏಕೆಂದರೆ ಇವರು ಸಜ್ಜನರಲ್ಲದೆ ಹೋದರೆ ಸಂನ್ಯಾಸಿಗಳು ಹೇಗಾದರು? ಸ್ವಾಮಿ ವಿವೇಕಾನಂದರೇ ಹೇಳಿದಂತೆ, ಸಂನ್ಯಾಸಿಯಾಗಬೇಕೆಂದಿರುವವನು ಮೊದಲು ಸಜ್ಜನನಾಗಿರಬೇಕು. ಆದರೆ ತಮಾಷೆಯೇನೆಂದರೆ ಈ ‘ಸಾಧು’ ಎನ್ನುವ ಪದವನ್ನಾದರೂ ಜನ ಸುಮ್ಮನೆ ಬಿಟ್ಟಿದ್ದಾರೆಯೇ? ‘ಸಾಧು’ಎಂದರೆ ಕೈಲಾಗದವನು ಎಂಬ ಅರ್ಥವನ್ನು ಕೊಟ್ಟುಬಿಟ್ಟಿದ್ದಾರೆ. ಆದರೆ ಜನ ಎಷ್ಟೇ ಅಪಾರ್ಥ ಮಾಡಿದರೂ ನಿಜಾರ್ಥ ಎಂದಿಗೂ ನಶಿಸಿ ಹೋಗುವುದಿಲ್ಲ.

ಇನ್ನು ‘ವೀರಸಂನ್ಯಾಸಿ ವಿವೇಕಾನಂದ’ ಎಂಬ ಈ ಶಿರೋನಾಮೆಯಲ್ಲಿ ವಿವೇಕಾನಂದರೆಂಬ ಸಂನ್ಯಾಸಿಗಳನ್ನು ‘ವೀರಸಂನ್ಯಾಸಿ’ ಎಂದು ಕರೆಯಲಾಗಿದೆ. ಹಾಗಾದರೆ ಮಿಕ್ಕ ಸಂನ್ಯಾಸಿಗಳು ವೀರರಲ್ಲವೆ ಎಂದು ಕೇಳಿದರೆ, ಸಂನ್ಯಾಸಿಗಳೆಲ್ಲರೂ ವೀರರೇ ಸರಿ. ಆದರೆ ವಿವೇಕಾನಂದರು ಮಾತ್ರ ವೀರಾಗ್ರಣಿಗಳು. ಈ ಸತ್ಯವನ್ನು ಅವರ ಜೀವನದಲ್ಲಿ ನಾವೇ ಕಂಡುಕೊಳ್ಳಬಹುದು. ಆದ್ದರಿಂದ ವೀರಸಂನ್ಯಾಸಿ ಎಂಬ ವಿಶೇಷಣ ನೇರವಾಗಿ ವಿವೇಕಾನಂದರಿಗೇ ಅನ್ವಯವಾಗುತ್ತದೆ.

ಈಗ ಇನ್ನೊಂದು ಅಂಶ. ಸಂನ್ಯಾಸಿಯು ಸರ್ವವನ್ನೂ ತ್ಯಜಿಸಿ ಅಹಿಂಸಾ ವ್ರತಧಾರಣೆ ಮಾಡಿದವನು. ಅಂದಮೇಲೆ ಅಲ್ಲಿ ವೀರತ್ವಕ್ಕೆ ಅವಕಾಶವೆಲ್ಲಿದೆ? ಹಾಗಿರುವಾಗ ಅವನು ವೀರಸಂನ್ಯಾಸಿ ಹೇಗಾದಾನು? ಅಥವಾ ಒಬ್ಬ ಸಂನ್ಯಾಸಿ ‘ವೀರ’ ಹೇಗಾದಾನು? ಯಾವ ವೈರಿಗಳನ್ನೂ ನಾಶಮಾಡದ, ಯಾವ ರಾಜ್ಯವನ್ನೂ ಗೆಲ್ಲದ ಸಂನ್ಯಾಸಿಯು ‘ವೀರ’ ಹೇಗಾದಾನು? ಇದಕ್ಕೆ ಉತ್ತರವಿಷ್ಟೆ–ಸಾವಿರಾರು ಯೋಧರನ್ನು ಗೆದ್ದವನಿಗಿಂತ ತನ್ನ ಇಂದ್ರಿಯಗಳನ್ನು ಗೆದ್ದವನೇ ನಿಜವಾದ ವೀರ! ಪರರಾಜ್ಯವನ್ನು ಗೆದ್ದವನಿಗಿಂತ ಆತ್ಮರಾಜ್ಯವನ್ನು ಸ್ವಾಧೀನಪಡಿಸಿಕೊಂಡವನೇ ನಿಜವಾದ ಧೀರ–ಕಶ್ಚಿದ್ಧೀರಃ ಪ್ರತ್ಯಗಾತ್ಮಾನಮೈಕ್ಷತ್! ಈ ಬಗೆಯ ವೀರಸಂನ್ಯಾಸಿಯ ವರ್ಣನೆ ಮಾಡುತ್ತಾರೆ, ಸ್ವಯಂ ವೀರಸಂನ್ಯಾಸಿಗಳಾದ ವಿವೇಕಾನಂದರು. ಅವರು ಇಂಗ್ಲಿಷಿನಲ್ಲಿ ರಚಿಸಿದ \eng{‘ Song of the Sannyasin’} ಎಂಬ ತಮ್ಮ ಕವನದಲ್ಲಿ ವೀರಸಂನ್ಯಾಸಿಯ ಸ್ಪಷ್ಟ ಚಿತ್ರಣ ಮೂಡಿಸಿದ್ದಾರೆ. ಈ ಕವನವನ್ನು ಸುಂದರ ಕನ್ನಡದಲ್ಲಿ ಪ್ರತಿಬಿಂಬಿಸಿದ್ದಾರೆ ರಾಷ್ಟ್ರಕವಿ ಕುವೆಂಪು ಅವರು. ಆ ಕವನದ ಕನ್ನಡ ರೂಪ ಎಷ್ಟು ನಿಖರ, ರಭಸಪೂರ್ಣ ಹಾಗೂ ತೇಜೋವಂತವಾಗಿದೆಯೆಂದರೆ ಸ್ವಾಮಿ ವಿವೇಕಾನಂದರು ಅದನ್ನು ಮೂಲತಃ ಕನ್ನಡದಲ್ಲೇ ಬರೆದರೋ ಎನ್ನುವಂತಿದೆ! ಕನ್ನಡರೂಪದ ಈ ಕವನಕ್ಕೆ ಸಂನ್ಯಾಸಿ ಗೀತೆ ಎಂದು ಹೆಸರು. ಓದಿನೋಡಿ ಇದನ್ನು\footnote{ಇದರ ತಾತ್ಪರ್ಯವನ್ನು ಅನುಬಂಧದಲ್ಲಿ (ಪುಟ ೩೦೦) ನೋಡಬಹುದು}–

\begin{verse}
ಏಳು ಮೇಲೇಳೇಳು ಸಾಧುವೆ, ಹಾಡು ಚಾಗಿಯ ಹಾಡನು;\\ಹಾಡಿನಿಂದೆಚ್ಚರಿಸು ಮಲಗಿಹ ನಮ್ಮ ಈ ತಾಯ್ನಾಡನು!\\ದೂರದಡವಿಯೊಳೆಲ್ಲಿ ಲೌಕಿಕವಿಷಯವಾಸನೆ ಮುಟ್ಟದೊ,\\ಎಲ್ಲಿ ಗಿರಿಗುಹೆಕಂದರದ ಬಳಿ ಜಗದ ಗಲಿಬಿಲಿ ತಟ್ಟದೊ,\\ ಎಲ್ಲಿ ಕಾಮವು ಸುಳಿಯದೊ,–ಮೇಣ್\\ಎಲ್ಲಿ ಜೀವವು ತಿಳಿಯದೊ\\ಕೀರ್ತಿ ಕಾಂಚನವೆಂಬುವಾಸೆಗಳಿಂದ ಜನಿಸುವ ಭ್ರಾಂತಿಯ,\\ಎಲ್ಲಿ ಆತ್ಮವು ಪಡೆದು ನಲಿವುದೊ ನಿಚ್ಚವಾಗಿಹ ಶಾಂತಿಯ,\\ನನ್ನಿಯರಿವಾನಂದವಾಹಿನಿಯೆಲ್ಲಿ ಸಂತತ ಹರಿವುದೊ,\\ಎಲ್ಲಿ ಎಡೆಬಿಡದಿರದ ತೃಪ್ತಿಯ ಝರಿ ನಿರಂತರ ಸುರಿವುದೊ,\\ಅಲ್ಲಿ ಮೂಡಿದ ಹಾಡನುಲಿಯೈ, ವೀರ ಸಂನ್ಯಾಸಿ–
\end{verse}

\begin{flushright}
ಓಂ! ತತ್! ಸತ್! ಓಂ!
\end{flushright}

\begin{verse}
ಕುಟ್ಟಿ ಪುಡಿಪುಡಿಮಾಡು ಮಾಯೆಯು ಕಟ್ಟಿಬಿಗಿದಿಹ ಹಗ್ಗವ;\\ಕಿತ್ತು ಬಿಸುಡೈ ಹೊಳೆವ ಹೊನ್ನಿನ ಹೆಣ್ಣುಮಣ್ಣಿನ ಕಗ್ಗವ!\\ಮುದ್ದಿಸಲಿ ಪೀಡಿಸಲಿ ದಾಸನು ದಾಸನೆಂಬುದೆ ಸತ್ಯವು;\\ಕಬ್ಬಿಣವೊ? ಕಾಂಚನವೊ? ಕಟ್ಟಿದ ಕಣ್ಣಿ ಕಣ್ಣಿಯೆ ನಿತ್ಯವು.\\ಪಾಪ ಪುಣ್ಯಗಳೆಂಬವು, –ಮಾ-\\ತ್ಸರ್ಯ ಪ್ರೇಮಗಳೆಂಬವು\\ದ್ವಂದ್ವರಾಜ್ಯದ ಧೂರ್ತಚೋರರು; ಬಿಟ್ಟು ಕಳೆ, ಕಳೆಯವರನು!\\ಮೋಹಗೊಳಿಪರು, ಬಿಗಿವರಿರಿವರು; ಎಚ್ಚರಿಕೆಯಿಂದವರನು \\ತಳ್ಳು ದೂರಕೆ, ಓ ವಿರಕ್ತನೆ! ಹಾಡು ಚಾಗಿಯ ಹಾಡನು!\\ಹಾಡಿನಿಂದೆಚ್ಚರಿಸು ಮಲಗಿಹ ನಮ್ಮ ಈ ತಾಯ್ನಾಡನು!\\ಹಾಡು ಮುಕ್ತಿಯ ಗಾನವನು, ಓ ವೀರ ಸಂನ್ಯಾಸಿ–
\end{verse}

\begin{flushright}
ಓಂ! ತತ್! ಸತ್! ಓಂ!
\end{flushright}

\begin{verse}
ಕತ್ತಲಳಿಯಲಿ; ಮುಬ್ಬುಕವಿಸುವ ಭವದ ತೃಷ್ಣೆಯು ಬತ್ತಲಿ;\\ಬಾಳಮೋಹವು ಮರುಮರೀಚಿಕೆ; ಮಾಯೆ ಕೆತ್ತಿದ ಪುತ್ತಳಿ;\\ಜನನದೆಡೆಯಿಂ ಮರಣದೆಡೆಗಾಗೆಳೆವುದೆಮ್ಮನು ದೇಹವು!\\ಜನ್ಮಜನ್ಮದಿ ಮರಳಿ ಮರಳುವುದೆಮ್ಮ ಬಿಗಿಯಲು ಮೋಹವು!\\ತನ್ನ ಜಯಿಸಿದ ಶಕ್ತನು–ಅವ\\ ನೆಲ್ಲ ಜಯಿಸಿದ ಮುಕ್ತನು\\ಎಂಬುದನು ತಿಳಿ; ಹಿಂಜರಿಯದಿರು. ಸಂನ್ಯಾಸಿಯೇ, ನಡೆ ಮುಂದಕೆ.\\ಗುರಿಯು ದೊರಕುವವರೆಗೆ ನಡೆ ನಡೆ; ನೋಡದಿರು ನೀ ಹಿಂದಕೆ.\\ಏಳು ಮೇಲೇಳೇಳು ಸಾಧುವೆ, ಹಾಡು ಚಾಗಿಯ ಹಾಡನು;\\ಹಾಡಿನಿಂದೆಚ್ಚರಿಸು ಮಲಗಿಹ ನಮ್ಮ ಈ ತಾಯ್ನಾಡನು!\\ಹಾಡು ಸಿದ್ಧನೆ, ಓ ಪ್ರಬುದ್ಧನೆ, ಹಾಡು ಸಂನ್ಯಾಸಿ–
\end{verse}

\begin{flushright}
ಓಂ! ತತ್! ಸತ್! ಓಂ!
\end{flushright}

\begin{verse}
“ಬೆಳೆಯ ಕೊಯ್ವನು ಬಿತ್ತಿದಾತನು; ಪಾಪ ಪಾಪಕೆ ಕಾರಣ;\\ವೃಕ್ಷಕಾರ್ಯಕೆ ಬೀಜ ಕಾರಣ; ಪುಣ್ಯ ಪುಣ್ಯಕೆ ಕಾರಣ;\\ಹುಟ್ಟಿ ಮೈವಡೆದಾತ್ಮ ಬಾಳಿನ ಬಲೆಯ ತಪ್ಪದೆ ಹೊರುವುದು;\\ಕಟ್ಟು ಮೀರಿಹನಾವನಿರುವನು? ಕಟ್ಟು ಕಟ್ಟನೆ ಹೆರುವುದು!”\\ಎಂದು ಪಂಡಿತರೆಂಬರು–ಮೇಣ್\\ತತ್ತ್ವದರ್ಶಿಗಳೆಂಬರು!\\ಆದೊಡೇನಂತಾತ್ಮವೆಂಬುದು ನಾಮರೂಪಾತೀತವು;\\ಮುಕ್ತಿಬಂಧಗಳಿಲ್ಲದಾತ್ಮವು ಸರ್ವನಿಯಮಾತೀತವು!\\ತತ್ತ್ವಮಸಿ ಎಂದರಿತು, ಸಾಧುವೆ, ಹಾಡು ಚಾಗಿಯ ಹಾಡನು!\\ಹಾಡಿನಿಂದೆಚ್ಚರಿಸು ಮಲಗಿಹ ನಮ್ಮ ಈ ತಾಯ್ನಾಡನು!\\ಸಾರು ಸಿದ್ಧನೆ, ವಿಶ್ವವರಿಯಲಿ! ಹಾಡು ಸಂನ್ಯಾಸಿ–
\end{verse}

\begin{flushright}
ಓಂ! ತತ್! ಸತ್! ಓಂ!
\end{flushright}

\begin{verse}
ತಂದೆ ತಾಯಿಯು ಸತಿಯು ಮಕ್ಕಳು ಗೆಳೆಯರೆಂಬುವರರಿಯರು;\\ಕನಸು ಕಾಣುತಲವರು ಸೊನ್ನೆಯ ಸರ್ವವೆನ್ನುತ ಮೆರೆವರು.\\ಲಿಂಗವರಿಯದ ಆತ್ಮವಾರಿಗೆ ಮಗುವು? ಆರಿಗೆ ತಾತನು?\\ಆರ ಮಿತ್ರನು? ಆರ ಶತ್ರುವು? ಒಂದೆಯಾಗಿರುವಾತನು?\\ಆತ್ಮವೆಲ್ಲಿಯು ಇರುವುದು;–ಮೇಣ್\\ ಆತ್ಮವೊಂದಾಗಿರುವುದು.\\ಭೇದವೆಂಬುವ ತೋರಿಕೆಯು ನಮ್ಮಾತ್ಮನಾಶಕೆ ಹೇತುವು.\\ಭೇದವನು ತೊರೆದೊಂದೆಯೆಂಬುದನರಿಯೆ ಮುಕ್ತಿಗೆ ಸೇತುವು.\\ಧೈರ್ಯದಿಂದಿದನೆಲ್ಲರಾಲಿಸೆ ಹಾಡು ಚಾಗಿಯ ಹಾಡನು!\\ಹಾಡಿನಿಂದೆಚ್ಚರಿಸು ಮಲಗಿಹ ನಮ್ಮ ಈ ತಾಯ್ನಾಡನು!\\ಸಾರು, ಜೀವನ್ಮುಕ್ತ! ಸಾರೈ ಧೀರ ಸಂನ್ಯಾಸಿ–
\end{verse}

\begin{flushright}
ಓಂ! ತತ್! ಸತ್! ಓಂ!
\end{flushright}

\begin{verse}
ಇರುವುದೊಂದೇ! ನಿತ್ಯಮುಕ್ತನು, ಸರ್ವಜ್ಞಾನಿಯು ಆತ್ಮನು!\\ನಾಮರೂಪಾತೀತನಾತನು; ಪಾಪಪುಣ್ಯಾತೀತನು!\\ವಿಶ್ವಮಾಯಾಧೀಶನಾತನು; ಕನಸು ಕಾಣುವನಾತನು!\\ಸಾಕ್ಷಿಯಾತನು; ಪ್ರಕೃತಿಜೀವರ ತೆರದಿ ತೋರುವನಾತನು!\\ಎಲ್ಲಿ ಮುಕ್ತಿಯ ಹುಡುಕುವೆ?–ಏ-\\ಕಿಂತು ಸುಮ್ಮನೆ ದುಡುಕುವೆ?\\ಇಹವು ತೋರದು, ಪರವು ತೋರದು; ಗುಡಿಯೊಳದು ಮೈದೋರದು.\\ವೇದ ತೋರದು, ಶಾಸ್ತ್ರ ತೋರದು; ಮತವು ಮುಕ್ತಿಯ ತೋರದು!\\ನಿನ್ನ ಕೈಲಿದೆ ನಿನ್ನ ಬಿಗಿದಿಹ ಕಬ್ಬಿಣದ ಯಮಪಾಶವು;\\ಬರಿದೆ ಶೋಕಿಪುದೇಕೆ? ಬಿಡು, ಬಿಡು! ನಿನಗೆ ನೀನೇ ಮೋಸವು!\\ಬೇಡ, ಪಾಶವ ಕಡಿದು ಕೈಬಿಡು! ಹಾಡು ಸಂನ್ಯಾಸಿ–
\end{verse}

\begin{flushright}
ಓಂ! ತತ್! ಸತ್! ಓಂ!
\end{flushright}

\begin{verse}
“ಶಾಂತಿ ಸರ್ವರಿಗಿರಲಿ” ಉಲಿಯೈ, “ಜೀವಜಂತುಗಳಾಳಿಗೆ\\ಹಿಂಸೆಯಾಗದೆ ಇರಲಿ ಎನ್ನಿಂದೆಲ್ಲ ಸೊಗದಲಿ ಬಾಳುಗೆ!\\ಬಾನೊಳಾಡುವ, ನೆಲದೊಳೋಡುವ ಸರ್ವರಾತ್ಮನು ನಾನಹೆ;\\ನಾಕ ನರಕಗಳಾಸೆಭಯಗಳನೆಲ್ಲ ಮನದಿಂ ದೂಡುವೆ!”\\ದೇಹ ಬಾಳಲಿ, ಬೀಳಲಿ;–ಅದು\\ಕರ್ಮನದಿಯಲಿ ತೇಲಲಿ!\\ಕೆಲರು ಹಾರಗಳಿಂದ ಸಿಂಗರಿಸದನು ಪೂಜಿಸಿ ಬಾಗಲಿ!\\ಕೆಲರು ಕಾಲಿಂದೊದೆದು ನೂಕಲಿ! ಹುಡಿಯು ಹುಡಿಯೊಳೆ ಹೋಗಲಿ!\\ಎಲ್ಲ ಒಂದಿರಲಾರು ಹೊಗಳುವರಾರು ಹೊಗಳಿಸಿಕೊಂಬರು?\\ನಿಂದೆ ನಿಂದಿಪರೆಲ್ಲ ಕೂಡಲು ಯಾರು ನಿಂದೆಯನುಂಬರು?\\ಪಾಶಗಳ ಕಡಿ! ಬಿಸುಡು, ಕಿತ್ತಡಿ! ಹಾಡು ಸಂನ್ಯಾಸಿ–
\end{verse}

\begin{flushright}
ಓಂ! ತತ್! ಸತ್! ಓಂ!
\end{flushright}

\begin{verse}
ಎಲ್ಲಿ ಕಾಮಿನಿಯೆಲ್ಲಿ ಕಾಂಚನದಾಸೆ ನೆಲೆಯಾಗಿರುವುದೊ,\\ಸತ್ಯವೆಂಬುವುದಲ್ಲಿ ಸುಳಿಯದು! ಎಲ್ಲಿ ಕಾಮವು ಇರುವುದೊ\\ಅಲ್ಲಿ ಮುಕ್ತಿಯು ನಾಚಿ ತೋರದು! ಎಲ್ಲಿ ಸುಳಿವುದೊ ಭೋಗವು\\ಅಲ್ಲಿ ತೆರೆಯದು ಮಾಯೆ ಬಾಗಿಲನಲ್ಲಿಹುದು ಭವರೋಗವು;\\ಎಲ್ಲಿ ನೆಲಸದೊ ಚಾಗವು,–ದಿಟ\\ವಲ್ಲಿ ಸೇರದು ಯೋಗವು!\\ಗಗನವೇ ಮನೆ! ಹಸುರೆ ಹಾಸಿಗೆ! ಮನೆಯು ಸಾಲ್ವುದೆ ಚಾಗಿಗೆ?\\ಹಸಿಯೊ, ಬಿಸಿಯೋ? ಬಿದಿಯು ಕೊಟ್ಟಾಹಾರವನ್ನವು ಯೋಗಿಗೆ!\\ಏನು ತಿಂದರೆ, ಏನು ಕುಡಿದರೆ, ಏನು? ಆತ್ಮಗೆ ಕೊರತೆಯೆ?\\ಸರ್ವಪಾಪವ ತಿಂದುತೇಗುವ ಗಂಗೆಗೇಂ ಕೊಳೆ-ಕೊರತೆಯೆ?\\ನೀನು ಮಿಂಚೈ! ನೀನು ಸಿಡಿಲೈ! ಮೊಳಗು ಸಂನ್ಯಾಸಿ–
\end{verse}

\begin{flushright}
ಓಂ! ತತ್! ಸತ್! ಓಂ!
\end{flushright}

\begin{verse}
ನಿಜವನರಿತವರೆಲ್ಲೊ ಕೆಲವರು; ನಗುವರುಳಿದವರೆಲ್ಲರೂ\\ನಿನ್ನ ಕಂಡರೆ, ಹೇ ಮಹಾತ್ಮನೆ! ಕುರುಡರೇನನು ಬಲ್ಲರು?\\ಗಣಿಸದವರನು ಹೋಗು, ಮುಕ್ತನೆ, ನೀನು ಊರಿಂದೂರಿಗೆ\\ಸೊಗವ ಬಯಸದೆ, ಅಳಲಿಗಳುಕದೆ! ಕತ್ತಲಲಿ ಸಂಚಾರಿಗೆ\\ನಿನ್ನ ಬೆಳಕನು ನೀಡೆಲೈ;–ಸಂ\\ಸಾರ ಮಾಯೆಯ ದೂಡೆಲೈ!\\ಇಂತು ದಿನದಿನ ಕರ್ಮಶಕ್ತಿಯು ಮುಗಿವವರೆಗೂ ಸಾಗೆಲೈ!\\ನಾನು ನೀನುಗಳಳಿದು ಆತ್ಮದೊಳಿಳಿದು ಕಡೆಯೊಳು ಹೋಗೆಲೈ!\\ಏಳು, ಮೇಲೇಳೇಳು, ಸಾಧುವೆ, ಹಾಡು ಚಾಗಿಯ ಹಾಡನು!\\ತತ್ತ್ವಮಸಿ ಎಂದರಿತು ಹಾಡೈ, ಧೀರ ಸಂನ್ಯಾಸಿ–
\end{verse}

\begin{flushright}
ಓಂ! ತತ್! ಸತ್! ಓಂ!
\end{flushright}

ಇಂಥ ವೀರಸಂನ್ಯಾಸಿಗಳಾದ ವಿವೇಕಾನಂದರ ಧೀರ ಗಂಭೀರ ಜೀವನವನ್ನು ಅಧ್ಯಯನ ಮಾಡಿದರೆ ಎಂತಹ ಸಪ್ಪೆ ಮನುಷ್ಯನೂ ವೀರಾವೇಶದಿಂದ ಒಮ್ಮೆ ಮೈಕೊಡಹಿ ಏಳದಿರಲಾರ.

\begin{flushright}
\textbf{ಸ್ವಾಮಿ ಪುರುಷೋತ್ತಮಾನಂದ}
\end{flushright}

\chapter*{ಶ್ರೀವಿವೇಕಾನಂದ ಸ್ತೋತ್ರ}

\begin{verse}
ಮೂರ್ತಮಹೇಶ್ವರಮುಜ್ಜ್ವಲ ಭಾಸ್ಕರ-\\ಮಿಷ್ಟಮಮರ ನರವಂದ್ಯಮ್\\ವಂದೇ ವೇದತನುಮುಜ್ಝಿತ ಗರ್ಹಿತ\\ಕಾಮಕಾಂಚನ ಬಂಧಮ್
\end{verse}

\begin{verse}
ಕೋಟಿಭಾನುಕರದೀಪ್ತ ಸಿಂಹಮಹೋ\\ಕಟಿತಟಕೌಪೀನವಂತಮ್\\ಅಭೀರಭೀಃ ಹುಂಕಾರ ನಾದಿತ ದಿಙ್ಮುಖ\\ಪ್ರಚಂಡ ತಾಂಡವನೃತ್ಯ
\end{verse}

\begin{verse}
ಭುಕ್ತಿ ಮುಕ್ತಿ 
 ಕೃಪಾಕಟಾಕ್ಷ ಪ್ರೇಕ್ಷಣ-\\ಮಘದಲ ವಿದಲನ ದಕ್ಷಮ್\\ಬಾಲಚಂದ್ರಧರಮಿಂದು ವಂದ್ಯಮಿಹ\\ನೌಮಿ ಗುರು ವಿವೇಕಾನಂದಮ್
\end{verse}

\begin{flushright}
\textbf{ಶರಚ್ಚಂದ್ರ ದೇವಶರ್ಮಾ (‘ಇಂದು’)}
\end{flushright}

ಮಹೇಶ್ವರನೇ ಅವತರಿಸಿದಂತಿರುವ, ಸೂರ್ಯನಂತೆ ಉಜ್ಜ್ವಲಿಸುತ್ತಿರುವ, ಸುರ-ನರರಿಂದ ವಂದ್ಯನಾದ, ವೇದಗಳೇ ದೇಹವನ್ನು ಧರಿಸಿ ಬಂದಂತಿರುವ, ಹೇಯವಾದ ಕಾಮ-ಕಾಂಚನಗಳನ್ನು ಪೂರ್ಣವಾಗಿ ಜಯಿಸಿದ, ಕೋಟಿ ಸೂರ್ಯ ಕಿರಣಕಾಂತಿಯಿಂದ ಜ್ವಲಿಸುವ ಕೌಪೀನಧಾರೀ ಪುರುಷಸಿಂಹನಾದ, ‘ಹೆದರದಿರಿ! ಹೆದರದಿರಿ!’ ಎಂಬ ಭವ್ಯನಾದದಿಂದ ಪ್ರಚಂಡ ತಾಂಡವವನ್ನಾಡುತ್ತಿರುವ, ಕಡೆ ಗಣ್ಣಿನ ನೋಟಮಾತ್ರದಿಂದಲೇ ಭುಕ್ತಿ-ಮುಕ್ತಿಗಳನ್ನು ಕರುಣಿಸುವ ಮತ್ತು ಪಾಪರಾಶಿಯನ್ನು ನಾಶಮಾಡಬಲ್ಲ, ಬಾಲಚಂದ್ರನನ್ನು ಧರಿಸಿರುವ ಸಾಕ್ಷಾತ್ ಶಿವನೇ ಆದ ಮತ್ತು ‘ಇಂದು’ವಂದ್ಯನಾದ ಗುರು ಶ್ರೀ ವಿವೇಕಾನಂದರಿಗೆ ನಮಸ್ಕರಿಸುತ್ತೇನೆ.

