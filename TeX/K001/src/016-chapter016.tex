
\chapter{ಸಂಘ ಸಂಸ್ಥಾಪನೆ}

\noindent

ನಿಜಕ್ಕೂ, ಭಗವತ್ಸಾಕ್ಷಾತ್ಕಾರ ಮಾಡಿಕೊಂಡ ಮಹಾತ್ಮರ ಮರಣವೆಂಬುದು ಮರಣವೇ ಅಲ್ಲ. ಅಂಥವರು ದೇಹತ್ಯಾಗ ಮಾಡಿದಾಗ ಅವರ ಭಕ್ತರಿಗೆ, ಅನುಚರರಿಗೆ ತಾತ್ಕಾಲಿಕವಾಗಿ ದುಃಖವಾಗ ಬಹುದು, ಇಲ್ಲವೆಂದಲ್ಲ. ಆದರೆ ಆ ಮಹಾತ್ಮರ ಜೀವನ ದರ್ಶನದಿಂದ ಮುಮುಕ್ಷುಗಳಾದವ ರಲ್ಲಿ ಭಗವತ್ಸಾಕ್ಷಾತ್ಕಾರದ ಹಂಬಲ ಇನ್ನಷ್ಟು ತೀವ್ರಗೊಳ್ಳುತ್ತದೆ. ಶ್ರೀರಾಮಕೃಷ್ಣರ ಮಹಾ ಸಮಾಧಿಯ ಅನಂತರವೂ ಹಾಗೆಯೇ ಆಯಿತು. ಅವರ ಶಿಷ್ಯರಲ್ಲಿ ಅತ್ಯುನ್ನತ ಆಧ್ಯಾತ್ಮಿಕ ಸಾಕ್ಷಾತ್ಕಾರ ಮಾಡಿಕೊಳ್ಳಬೇಕೆಂಬ ತೀವ್ರ ವ್ಯಾಕುಲತೆ ಉದ್ದೀಪನಗೊಂಡಿತು. ಈಗ ತಮ್ಮೊಳಗೆ ಏನೋ ಒಂದು ಹೊಸ ಶಕ್ತಿ-ಉತ್ಸಾಹ ಜಾಗೃತವಾಗಿರುವಂತೆ ಅವರಿಗೆ ಭಾಸವಾಯಿತು. ಅದು ತಮ್ಮ ಪ್ರಿಯ ಗುರು ಶ್ರೀರಾಮಕೃಷ್ಣರ ಸಾನ್ನಿಧ್ಯದಿಂದಾಗಿಯೇ ಉಂಟಾದ ಸ್ಫೂರ್ತಿ ಎಂಬ ಅರಿವೂ ಅವರಿಗುಂಟಾಗುತ್ತಿತ್ತು. ಶ್ರೀರಾಮಕೃಷ್ಣರು ಯಾವ ಉದ್ದೇಶವನ್ನಿಟ್ಟುಕೊಂಡು ಈ ಜಗತ್ತಿನಲ್ಲಿ ಜನ್ಮತಾಳಿದರೋ, ಯಾವ ಕಾರ್ಯಗಳನ್ನು ಸಾಧಿಸಬೇಕೆಂದು ಇಲ್ಲಿಗೆ ಆಗಮಿಸಿ ದರೋ ಆ ಕಾರ್ಯಗಳೆಲ್ಲ ಅವರ ದೇಹಾವಸಾನದೊಂದಿಗೇ ಮುಕ್ತಾಯಗೊಳ್ಳುವಂಥವಲ್ಲ ಎಂಬುದು ಈ ಶಿಷ್ಯರಿಗೆ ತಿಳಿದಿತ್ತು. ಶ್ರೀರಾಮಕೃಷ್ಣರ ಅವತಾರೋದ್ದೇಶವೆಂಬುದು ಆಧ್ಯಾತ್ಮಿಕ ಜೀವನಗಳ ಹಾಗೂ ಆಧ್ಯಾತ್ಮಿಕ ಜ್ಞಾನದ ರೂಪದಲ್ಲಿ ಮುಂದೆ ನಿರಂತರವಾಗಿ ಪ್ರವಹಿಸಲಿದೆ. ಅವರ ಭಕ್ತರು ಹಾಗೂ ಶಿಷ್ಯರೇ–ಅದರಲ್ಲೂ ಶ್ರೀರಾಮಕೃಷ್ಣರ ಹೆಸರಿನಲ್ಲಿ ತ್ಯಾಗಮಾಡಿ ಬರಲಿರುವ ಈ ಯುವಕರೇ–ಆ ಪ್ರವಾಹವನ್ನು ಒಯ್ಯಲಿರುವ ಪ್ರಮುಖ ಕಾಲುವೆಗಳಾಗ ಬೇಕಾಗಿದೆ.

ಆದರೆ ಮೊದಮೊದಲಿಗೆ ಶ್ರೀರಾಮಕೃಷ್ಣರು ಕಣ್ಮರೆಯಾದ್ದರಿಂದ ಈ ಶಿಷ್ಯರೂ ದಿಗ್ಭ್ರಾಂತ ರಾದರು. ಮುಂದೇನು ಮಾಡಬೇಕೆಂದು ಬಗೆಗಾಣದೆ ಕಂಗಾಲಾದರು. ಶ್ರೀರಾಮಕೃಷ್ಣರ ದೇಹಾವ ಸಾನವು ಅನಿರೀಕ್ಷಿತವಾದದ್ದೇನಲ್ಲ. ಆದರೂ ಅದು ಆಕಸ್ಮಿಕವಾಗಿ ನಡೆದುಹೋಯಿತೋ ಎಂಬಂತೆ ಅಸಹಾಯಕರಾದರು. ತಾವೀಗ ಸಂಸಾರವನ್ನು ತ್ಯಾಗಮಾಡಿ ಬಂದರೂ ಇರುವುದೆಲ್ಲಿ, ತಮಗೆ ದಿಕ್ಕು ತೋರುವವರಾರು? ಅಲ್ಲದೆ ಕಾಶೀಪುರದ ಉದ್ಯಾನಗೃಹವಾದರೋ ಬಾಡಿಗೆ ಮನೆ. ಆಗಸ್ಟ್ ತಿಂಗಳ ಅಂತ್ಯದವರೆಗಿನ ಬಾಡಿಗೆಯನ್ನು ಮಾತ್ರ ಮುಂದಾಗಿ ಪಾವತಿ ಮಾಡಿದ್ದಾಗಿದೆ. ಶಿಷ್ಯರಿಗೇನೋ ಆ ಉದ್ಯಾನಗೃಹದಲ್ಲೇ ಉಳಿದುಕೊಂಡು ಶ್ರೀರಾಮಕೃಷ್ಣರನ್ನು ಪೂಜಿಸುತ್ತ ಆಧ್ಯಾತ್ಮಿಕ ಸಾಧನೆಯಲ್ಲಿ ತೊಡಗುವ ಆಸೆ. ಆದರೆ ಮನೆಯ ಬಾಡಿಗೆ ಕೊಡುವವರಾರು? ತಾವೆಲ್ಲ ಸಂಘಟಿತರಾಗಿ, ಅಲ್ಲೇ ವಾಸಿಸುತ್ತ ತ್ಯಾಗಜೀವನ ನಡೆಸುವುದಾಗಿ ನರೇಂದ್ರ ಹೇಳಿದ. ಆದರೆ ಗೃಹೀಭಕ್ತರಲ್ಲಿ ಹೆಚ್ಚಿನವರಿಗೆ ಅದು ಒಪ್ಪಿಗೆಯಾಗಲಿಲ್ಲ. “ಸುಮ್ಮನೆ ನಿಮ್ಮನಿಮ್ಮ ಮನೆಗಳಿಗೆ ಹೋಗಿ; ಓದಿ ಬುದ್ಧಿವಂತರಾಗಿ. ಇದರಿಂದ ನಿಮಗೂ ಹಿತ, ಹೆತ್ತವರಿಗೂ ಸಂತೋಷ” ಎಂದು ಬೋಧಿಸಿದರು. ತಾವು ಇಷ್ಟು ದಿನ ನೀಡುತ್ತಿದ್ದ ನೆರವನ್ನು ನಿಲ್ಲಿಸಿಬಿಟ್ಟರು. ಮತ್ತು, “ಸಂಸಾರಜೀವನವೇ ಶ್ರೇಷ್ಠ ಅಂತಲೇ ಶ್ರೀರಾಮಕೃಷ್ಣರು ಹೇಳುತ್ತಿದ್ದುದು. ಅವರೇ ಸ್ವತಃ ಗೃಹಸ್ಥರಾಗಿದ್ದವರಲ್ಲವೆ!” ಎಂದೂ ಕೆಲವು ಭಕ್ತರು ತರ್ಕಮಾಡಿದರು. ಆದರೆ ನರೇಂದ್ರ ಈ ವಾದವನ್ನೆಲ್ಲ ಧಿಕ್ಕರಿಸಿ, ತಾವೆಲ್ಲ ತ್ಯಾಗಜೀವನವನ್ನೇ ಕೈಗೊಳ್ಳುವವರು, ಮತ್ತು ಶ್ರೀರಾಮ ಕೃಷ್ಣರು ತಮಗೆ ತೋರಿಸಿಕೊಟ್ಟ ಮಾರ್ಗವೂ ಅದೇ ಎಂದು ಖಡಾಖಂಡಿತವಾಗಿ ಹೇಳಿಬಿಟ್ಟ. ಬಲರಾಮ, ಮಾಸ್ಟರ್, ಸುರೇಂದ್ರ ಮೊದಲಾದ ಕೆಲವೇ ಮಂದಿ ಭಕ್ತರಿಗೆ ಮಾತ್ರ ಅವನ ಮಾತಿನ ತಥ್ಯ ಮನವರಿಕೆಯಾಯಿತು. ಅವರುಗಳು ಶಿಷ್ಯರ ಬೆಂಬಲಕ್ಕೆ ನಿಂತರು.

ಶ್ರೀರಾಮಕೃಷ್ಣರು ಇಚ್ಛಿಸಿದ್ದಂತೆ, ಅವರ ಅವಶೇಷಗಳನ್ನು ಪ್ರತಿಷ್ಠಾಪಿಸಲು ಗಂಗೆಯ ದಡದ ಮೇಲೆ ಒಂದು ಕಟ್ಟಡವನ್ನು ನಿರ್ಮಿಸಬೇಕೆಂದು ಈ ಶಿಷ್ಯರು ಮನಸ್ಸು ಮಾಡಿದ್ದರು. ಈಗ ಈ ಬಗ್ಗೆ ಶಿಷ್ಯರ ಹಾಗೂ ಗೃಹೀಭಕ್ತರ ನಡುವೆ ಭಿನ್ನಾಭಿಪ್ರಾಯ ತಲೆದೋರಿತು. ಕೊನೆಗೆ, ಗಂಗಾತೀರ ದಲ್ಲಿ ಒಂದು ಜಾಗವನ್ನು ಕೊಂಡು, ಅಲ್ಲಿ ಕಟ್ಟಡ ಕಟ್ಟಬೇಕು; ಅಲ್ಲಿಯವರೆಗೂ ಅವಶೇಷಗಳನ್ನು ಕಾಶೀಪುರದ ಉದ್ಯಾನಗೃಹದಲ್ಲೇ ಇಟ್ಟಿರುವುದು ಎಂಬ ತೀರ್ಮಾನಕ್ಕೆ ಬಂದರು. ಆದರೆ ಈ ತೀರ್ಮಾನವೂ ಶೀಘ್ರದಲ್ಲೇ ಬಿದ್ದುಹೋಯಿತು. ಏಕೆಂದರೆ ಜಾಗವನ್ನು ಕೊಂಡುಕೊಳ್ಳಲು, ಕಟ್ಟಡ ಕಟ್ಟಿಸಲು ಹಣ ಕೊಡವವರು ಯಾರು? ಪ್ರಾಯಶಃ ನರೇಂದ್ರ ವಿವೇಕಾನಂದರಾಗಿ, ಅಮೆರಿಕೆಗೆ ಹೋಗಿ ಜಯಭೇರಿ ಬಾರಿಸಿ ಹಿಂದಿರುಗಿದ ಮೇಲೆಯೇ ಗಂಗೆಯ ದಡದ ಮೇಲೆ ಕಟ್ಟಡ ನಿರ್ಮಾಣವಾಗಬೇಕು ಎಂಬುದು ಶ್ರೀರಾಮಕೃಷ್ಣರ ಇಚ್ಛೆಯಾಗಿತ್ತೇನೋ, ಯಾರು ಬಲ್ಲರು! ಈಗಂತೂ ರಾಮಚಂದ್ರ ದತ್ತನ ಮುಖಂಡತ್ವದಲ್ಲಿ ದೇವೇಂದ್ರನಾಥ ಮಜುಮ ದಾರ್, ನಿತ್ಯಗೋಪಾಲನೇ ಮೊದಲಾದ ಗೃಹಸ್ಥರು ಆ ಅವಶೇಷಗಳನ್ನು ಕಾಂಕುರ್ಗಾಚಿಯಲ್ಲಿದ್ದ ರಾಮಚಂದ್ರ ದತ್ತನ ಉದ್ಯಾನದಲ್ಲೇ ಪ್ರತಿಷ್ಠಾಪಿಸುವುದೆಂದು ತೀರ್ಮಾನಿಸಿದರು.

ಆದರೆ ಈ ಯುವಶಿಷ್ಯರು ಅವಶೇಷವನ್ನು ಅವರ ಕೈಗೊಪ್ಪಿಸಲು ನಿರಾಕರಿಸಿಬಿಟ್ಟರು. ಆಗ ಆ ಗೃಹಸ್ಥಭಕ್ತರಿಗೆಲ್ಲ ಸಿಟ್ಟೇರಿತು. “ಎಲಾ, ಇವರೆಲ್ಲ ನಮ್ಮ ಕಣ್ಮುಂದೆಯೇ ಬೆಳೆದ ಹುಡುಗರು. ಈಗ ನಮ್ಮ ಮಾತಿಗೇ ಎದುರಾಡುತ್ತಾರಲ್ಲ!” ಎಂದು ಕೋಪಗೊಂಡರು. ಆದರೆ ಯುವಶಿಷ್ಯ ರಿಗೆ, “ಶ್ರೀರಾಮಕೃಷ್ಣರು ನಮ್ಮ ಪ್ರೀತಿಯ ಗುರು. ನಮ್ಮ ಮೂಲಕ ಅವರು ಧರ್ಮಪ್ರಸಾರ ಕಾರ್ಯ ಮಾಡಿಸುವ ಉದ್ದೇಶವನ್ನಿಟ್ಟುಕೊಂಡಿದ್ದರು. ಆದ್ದರಿಂದ ಈ ಅವಶೇಷ ನಮ್ಮ ವಶದಲ್ಲೇ ಇರಬೇಕಾದದ್ದು” ಎಂಬ ಹಟ. ಹೀಗೆ ಅವರ ಕೋಪ, ಇವರ ಹಟ ಸೇರಿ ಝಟಾಪಟಿ ಶುರುವಾಯಿತು. ಈ ಕಡೆ ಶಶಿ ಮತ್ತು ನಿರಂಜನ ಆ ಅವಶೇಷಕ್ಕೆ ಬಲವಾದ ಕಾವಲಾಗಿ ನಿಂತಿದ್ದರು. ಇಬ್ಬರೂ ಬಲಿಷ್ಠರೇ–ನಿರ್ಧಾರದಲ್ಲಿ ಶಶಿ ಬಲಿಷ್ಠನಾದರೆ, ಆಕಾರ-ಗಾತ್ರದಲ್ಲಿ ನಿರಂಜನ. ಏನೇ ಆಗಲಿ ಅವಶೇಷವನ್ನು ಮಾತ್ರ ಬಿಟ್ಟುಕೊಡಬಾರದು ಎಂದು ಸ್ಥಿರವಾಗಿ ನಿಂತುಬಿಟ್ಟಿದ್ದರು. ಉಳಿದ ಯುವಕಶಿಷ್ಯರೂ ಇವರಿಗೆ ಬೆಂಬಲವಾಗಿದ್ದರು. ಆಗ ಗೃಹಸ್ಥಭಕ್ತರು ನರೇಂದ್ರನ ಬಳಿಗೆ ಬಂದು, ಅದನ್ನು ತಮಗೆ ಕೊಡಿಸುವಂತೆ ಕೇಳಿಕೊಂಡರು. ಆಗ ನರೇಂದ್ರ ತನ್ನ ಸೋದರಶಿಷ್ಯರನ್ನೆಲ್ಲ ಕರೆದು, “ಸೋದರರೇ, ನಾವೀಗ ಸ್ವಲ್ಪ ವಿಚಾರ ಮಾಡಿ ನೋಡ ಬೇಕಾಗುತ್ತದೆ. ನಾವು ಈ ರೀತಿ ಜಗಳವಾಡಿಕೊಳ್ಳಬಾರದು. ಇಲ್ಲದೆ ಹೋದರೆ ಮುಂದೆ ಜನಗಳು ‘ಶ್ರೀರಾಮಕೃಷ್ಣರ ಶಿಷ್ಯರು ಅವರ ಅವಶೇಷಕ್ಕಾಗಿ ಬಡಿದಾಡಿಕೊಂಡರು’ ಅಂತ ಮಾತನಾಡಿ ಕೊಳ್ಳುತ್ತಾರೆ. ಅಲ್ಲದೆ, ನಾವೆಲ್ಲ ಮುಂದೆ ಎಲ್ಲಿ ಉಳಿದುಕೊಳ್ಳುವುದು ಎನ್ನುವುದೇ ನಿರ್ಧಾರ ವಾಗಿಲ್ಲ. ಆದ್ದರಿಂದ ಅಸ್ಥಿಯನ್ನು ಅವರೇ ಇಟ್ಟುಕೊಳ್ಳಲಿ, ಇಷ್ಟರ ಮೇಲೆ, ರಾಮ್ಬಾಬು ವಾದರೂ ಆ ಅಸ್ಥಿಯನ್ನು ಶ್ರೀರಾಮಕೃಷ್ಣರ ಹೆಸರಿನಲ್ಲೇ ಪ್ರತಿಷ್ಠಾಪಿಸಿ ಪೂಜಿಸುತ್ತಾನಲ್ಲವೆ? ಅದೂ ಒಳ್ಳೆಯದೇ! ನಾವೂ ಅಲ್ಲಿಗೆ ಹೋಗಿ ಪೂಜೆ ಸಲ್ಲಿಸಬಹುದು. ಈ ಅಸ್ಥಿಗೆ ಬದಲಾಗಿ ನಾವು ಶ್ರೀರಾಮಕೃಷ್ಣರ ಬೋಧನೆಗಳನ್ನು ಕೈಗೆತ್ತಿಕೊಂಡು ಅವುಗಳಿಗನುಸಾರವಾಗಿ ನಮ್ಮ ಜೀವನವನ್ನು ರೂಪಿಸಿಕೊಳ್ಳೋಣ. ನಾವು ಅವರ ಆದರ್ಶಗಳನ್ನು ಪ್ರಾಮಾಣಿಕ ಭಾವದಿಂದ ಸ್ವೀಕರಿಸಿ, ಅವುಗಳಿಗೆ ತಕ್ಕಂತೆ ಬದುಕಿದರೆ, ಈ ಅವಶೇಷಗಳನ್ನು ಪೂಜೆ ಮಾಡುವುದಕ್ಕಿಂತಲೂ ಎಷ್ಟೋ ಪಾಲು ಉತ್ತಮವಾದ ಕೆಲಸವನ್ನು ಮಾಡಿದಂತಾಗುತ್ತದೆ” ಎಂದು ಬುದ್ಧಿ ಹೇಳಿದ.

ಆದರೆ ಯುವಶಿಷ್ಯರು ಯೋಚಿಸಿದರು–ಶ್ರೀರಾಮಕೃಷ್ಣರ ಪವಿತ್ರ ಅವಶೇಷಗಳನ್ನು ಗೃಹಸ್ಥ ಭಕ್ತರಿಗೆ ಕೊಡಲೇಬೇಕು ಎನ್ನುವುದಾದರೆ, ಖಂಡಿತವಾಗಿ ಪೂರ್ತಿಯಾಗಂತೂ ಕೊಡಬಾರದು; ತಾವು ಒಂದಿಷ್ಟನ್ನು ಇಟ್ಟುಕೊಂಡು ಉಳಿದದ್ದನ್ನು ಕೊಡೋಣ, ಎಂದು. ಈ ವಿಷಯವನ್ನು ನರೇಂದ್ರನೊಂದಿಗೆ ಚರ್ಚಿಸಿ ಒಂದು ತೀರ್ಮಾನಕ್ಕೆ ಬಂದರು. ಅದರಂತೆ, ಅರ್ಧಕ್ಕಿಂತಲೂ ಹೆಚ್ಚಿನ ಅಸ್ಥಿ-ಭಸ್ಮಗಳನ್ನು ಇನ್ನೊಂದು ಪಾತ್ರೆಗೆ ಹಾಕಿಟ್ಟುಕೊಂಡರು. ಅದರ ಪರಿಣಾಮವಾಗಿ ಆ ರಾತ್ರಿ ಎಲ್ಲರೂ ಗಾಢಧ್ಯಾನದಲ್ಲಿ ಮುಳುಗಿಹೋದರು. ತಾವು ತೆಗೆದಿಟ್ಟುಕೊಂಡ ಅಸ್ಥಿ-ಭಸ್ಮದ ಪಾತ್ರೆಯ ಮುಚ್ಚಳವನ್ನು ಬೆಸುಗೆ ಹಾಕಿ ಬಲರಾಮ ಬಾಬುವಿನ ಮನೆಗೆ ಗುಟ್ಟಾಗಿ ಕಳಿಸಿಕೊಟ್ಟರು. ಬಲರಾಮ ಗೃಹಸ್ಥನಾದರೂ ಅವನು ಈ ಯುವಕರ ಪರ. ಅವನ ಮನೆಯಲ್ಲೇ ಈ ಪವಿತ್ರ ಅವಶೇಷಗಳಿಗೆ ಪ್ರತಿದಿನ ಪೂಜೆ ನಡೆಯುತ್ತ ಬಂತು. ಒಂದಲ್ಲ ಒಂದು ದಿನ ಗಂಗೆಯ ದಡದಲ್ಲಿ ಒಂದು ಜಾಗವನ್ನು ಕೊಂಡು, ಅಲ್ಲಿ ಈ ಪವಿತ್ರ ಅವಶೇಷವನ್ನು ಪ್ರತಿಷ್ಠಾಪಿಸಲೇಬೇಕು ಎಂದು ಆ ಯುವಕಶಿಷ್ಯರು ದೃಢನಿರ್ಧಾರ ಮಾಡಿಬಿಟ್ಟರು. ಮತ್ತು, ಏನೂ ಆಗಲಿಲ್ಲವೆಂಬಂತೆ ಇನ್ನೊಂದು ಅವಶೇಷದ ಪಾತ್ರೆಯನ್ನು ರಾಮಚಂದ್ರ ದತ್ತನಿಗೆ ಒಪ್ಪಿಸಿಕೊಟ್ಟರು. ಬಳಿಕ ಎಲ್ಲರೂ ಸೇರಿ, ಆ ಅವಶೇಷವನ್ನು ಶ್ರೀಕೃಷ್ಣಜನ್ಮಾಷ್ಟಮಿಯ ದಿನದಂದು, ಎಂದರೆ ಆಗಸ್ಟ್ ೨೩ರಂದು ಕಾಂಕುರ್ಗಾಚಿಯ ಉದ್ಯಾನದಲ್ಲಿ ಪ್ರತಿಷ್ಠಾಪಿಸುವುದು ಎಂದು ತೀರ್ಮಾನಿಸಿದರು.

ಕೃಷ್ಣಜನ್ಮಾಷ್ಟಮಿಗೆ ಇನ್ನೂ ಒಂದು ವಾರವಿತ್ತು. ಈಗ ಕಾಶೀಪುರದ ಉದ್ಯಾನಗೃಹದಲ್ಲಿ ಲಾಟು, ತಾರಕನಾಥ ಮತ್ತು ಹಿರಿಯ ಗೋಪಾಲ ಮಾತ್ರ ಉಳಿದುಕೊಂಡಿದ್ದು, ಉಳಿದವರೆಲ್ಲ ತತ್ಕಾಲಕ್ಕೆ ತಮ್ಮತಮ್ಮ ಮನೆಗಳಿಗೆ ಹಿಂದಿರುಗಿದ್ದರು. ಆದರೆ ಪ್ರತಿದಿನ ಭಕ್ತ-ಶಿಷ್ಯರೆಲ್ಲರೂ ಉದ್ಯಾನಗೃಹದಲ್ಲಿ ಒಟ್ಟಾಗಿ ಸೇರಿ ಧ್ಯಾನ-ಭಜನೆಗಳಲ್ಲಿ ತೊಡಗುತ್ತಿದ್ದರು. ತಮ್ಮ ಪ್ರಾಣಪ್ರಿಯ ರಾದ ಶ್ರೀರಾಮಕೃಷ್ಣರ ಸ್ಮರಣೆಯಲ್ಲಿ ತೊಡಗುತ್ತಿದ್ದರು. ತಮ್ಮ ಗುರುದೇವನ ಕೊನೆಯ ದಿನಗಳ ಹಲವು ಘಟನೆಗಳನ್ನು, ಮಾತುಕತೆಗಳನ್ನು ಮತ್ತು ದಕ್ಷಿಣೇಶ್ವರದ ದಿವ್ಯಾನಂದದ ದಿನಗಳನ್ನು ಮೆಲುಕುಹಾಕುತ್ತಿದ್ದರು. ನರೇಂದ್ರ ಶ್ರೀರಾಮಕೃಷ್ಣರ ಅಲೌಕಿಕ ವ್ಯಕ್ತಿತ್ವವನ್ನು ಮತ್ತು ಅವರ ಜೀವನದ ಹಲವಾರು ಅದ್ಭುತ ಘಟನೆಗಳನ್ನು ವಿವರಿಸುತ್ತ ಅವರನ್ನೆಲ್ಲ ಸ್ಫೂರ್ತಿಗೊಳಿಸುತ್ತಿದ್ದ.

ಈ ಸಂದರ್ಭದಲ್ಲೇ ಒಮ್ಮೆ, ಶ್ರೀರಾಮಕೃಷ್ಣರಿನ್ನೂ ತಮ್ಮೊಡನೆಯೇ ಇದ್ದಾರೆ ಎಂಬುದನ್ನು ಅವರಿಗೆ ಮನವರಿಕೆ ಮಾಡಿಸುವ ಅಪೂರ್ವ ಘಟನೆಯೊಂದು ನಡೆಯಿತು. ಒಂದು ದಿನ ನರೇಂದ್ರನೂ ಹರೀಶ ಎಂಬ ಸೋದರಶಿಷ್ಯನೂ ಉದ್ಯಾನದ ಕೊಳದ ಬದಿಯಲ್ಲಿ ನಿಂತು ಮಾತನಾಡುತ್ತಿದ್ದರು–ಬಹುಶಃ ಶ್ರೀರಾಮಕೃಷ್ಣರ ಕುರಿತಾಗಿಯೇ ಇರಬೇಕು. ರಾತ್ರಿ ಸುಮಾರು ಎಂಟು ಗಂಟೆಯ ಸಮಯ. ಆಗ ಇದ್ದಕ್ಕಿದ್ದಂತೆ ನರೇಂದ್ರನಿಗೆ ಅಲ್ಲೊಬ್ಬ ಜ್ಯೋತಿರ್ಮಯನಾದ ವ್ಯಕ್ತಿ ನಿಂತಿರುವಂತೆ ಕಂಡಿತು. ಆ ವ್ಯಕ್ತಿ ಮೈತುಂಬ ವಸ್ತ್ರ ಹೊದ್ದು, ಗೇಟಿನ ಕಡೆಯಿಂದ ಅವರ ಬಳಿಗೇ ಮೆಲ್ಲನೆ ನಡೆದುಕೊಂಡು ಬರುತ್ತಿರುವಂತಿತ್ತು. ನರೇಂದ್ರ ಕಣ್ಣಗಲಿಸಿಕೊಂಡು ಅತ್ತಲೇ ನೋಡುತ್ತಿದ್ದಾನೆ. ಅದು ಅಗಲಿದ ತಮ್ಮ ಗುರುದೇವನೇ ಆಗಿರಬಹುದೆ...! ಆದರೆ ತನಗೇನೋ ಭ್ರಮೆಯಾಗುತ್ತಿರಬೇಕು ಎಂದುಕೊಂಡು ಸುಮ್ಮನೆಯೇ ಇದ್ದ. ಆದರೆ ಜೊತೆಯಲ್ಲಿದ್ದ ಹರೀಶ ಉದ್ವೇಗದ ದನಿಯಲ್ಲಿ, “ಅದೇನದು! ನರೇನ್, ಅಲ್ಲಿ ನೋಡು, ಅಲ್ಲಿ ನೋಡು!” ಎಂದು ಪಿಸುಗುಟ್ಟಿದ. ಎಂದರೆ, ಅವನಿಗೂ ಈ ದೃಶ್ಯ ಗೋಚರವಾಗಿಬಿಟ್ಟಿದೆ! ಈಗ ನರೇಂದ್ರ ಗಟ್ಟಿಯಾಗಿ “ಯಾರದು?” ಎಂದು ಕೂಗಿದ. ಅದನ್ನು ಕೇಳಿ ಮನೆಯೊಳಗಿದ್ದವರೆಲ್ಲ ಹೊರಗೆ ಓಡಿಬಂದರು. ಅಷ್ಟರಲ್ಲಿ ಆ ವ್ಯಕ್ತಿ ಇವರಿಬ್ಬರು ನಿಂತಲ್ಲಿಂದ ಸುಮಾರು ಹತ್ತು ಗಜ ದೂರ ದಲ್ಲಿದ್ದ ಮಲ್ಲಿಗೆ ಪೊದೆಯೊಂದರ ಬಳಿಗೆ ಬಂದು ಅದೃಶ್ಯವಾಯಿತು. ಎಲ್ಲರೂ ಲಾಟೀನುಗಳನ್ನು ತಂದು ಮೂಲೆಮೂಲೆಗಳನ್ನು ಹುಡುಕಿ ನೋಡಿದರು. ಎಲ್ಲೂ ಏನೂ ಕಾಣಲಿಲ್ಲ. ನರೇಂದ್ರನ ಮನಸ್ಸಿನಲ್ಲಿ ಮಾತ್ರ ಈ ಸ್ಪಷ್ಟ ದರ್ಶನ ತನ್ನ ಚಿರಮುದ್ರೆಯನ್ನೊತ್ತಿಬಿಟ್ಟಿತು.

ಶ್ರೀಕೃಷ್ಣಾಷ್ಟಮಿಯ ದಿನ ಬಂದಿತು. ಭಕ್ತರೆಲ್ಲರೂ ಸೇರಿ ಶ್ರೀರಾಮಕೃಷ್ಣರ ಅವಶೇಷವನ್ನು ಕಾಂಕುರ್ಗಾಚಿಗೆ ಕೊಂಡೊಯ್ಯಲು ಸಿದ್ಧರಾದರು. ನರೇಂದ್ರನ ಮುಂದಾಳ್ತನದಲ್ಲಿ ಮೆರವಣಿಗೆ ಹೊರಟಿತು. ಅವಶೇಷದ ಪಾತ್ರೆಯನ್ನು ಶಶಿ ತಲೆಯ ಮೇಲೆ ಹೊತ್ತು ನಡೆದ. ಅವನ ಹಿಂದೆ ರಾಮಚಂದ್ರ ದತ್ತ ಇನ್ನಿತರರು ಸಾಗಿದರು. ದಾರಿಯುದ್ದಕ್ಕೂ ಭಗವನ್ನಾಮ ಸಂಕೀರ್ತನೆ ಮಾಡುತ್ತ ಮುನ್ನಡೆದು ಕಾಂಕುರ್ಗಾಚಿಯ ಉದ್ಯಾನವನ್ನು ತಲುಪಿದರು. ಅಲ್ಲಿ ಶ್ರೀರಾಮಕೃಷ್ಣರ ಅವಶೇಷವನ್ನು ವಿಧಿಯುಕ್ತವಾಗಿ ಪ್ರತಿಷ್ಠಾಪಿಸಲಾಯಿತು. ಕಾಲಾಂತರದಲ್ಲಿ ಆ ಸ್ಥಳದಲ್ಲಿ ಒಂದು ಮಂದಿರವನ್ನು ಕಟ್ಟಲಾಯಿತು. ಈಗ ಈ ಸ್ಥಳ ‘ಯೋಗೋದ್ಯಾನ’ ಎಂಬ ಹೆಸರಿನಿಂದ ಪ್ರಸಿದ್ಧವಾಗಿದೆ. ಇಲ್ಲಿ ಪ್ರತಿವರ್ಷವೂ ಆ ದಿನದಂದು ಶ್ರೀರಾಮಕೃಷ್ಣರ ಉತ್ಸವ ನಡೆದುಕೊಂಡು ಬರುತ್ತಿದೆ.

ಅವಶೇಷದ ಪ್ರತಿಷ್ಠಾಪನೆಗೆ ಎರಡು ದಿನ ಮೊದಲೇ ಬಲರಾಮಬಾಬು ಶಾರದಾದೇವಿ ಯವರನ್ನು ತನ್ನ ಮನೆಗೆ ಕರೆತಂದಿದ್ದ. ಪತಿವಿಯೋಗದ ದುಃಖವನ್ನು ಶಮನಗೊಳಿಸಲು ಅವರಿಗಾಗಿ ಆತ ತೀರ್ಥಯಾತ್ರೆಯ ಏರ್ಪಾಡು ಮಾಡಿದ. ಆಗಸ್ಟ್ ೩೦ರಂದು ಶಾರದಾದೇವಿ ಯವರು ಲಾಟು, ಕಾಳೀಪ್ರಸಾದ, ಯೋಗೀಂದ್ರ ಹಾಗೂ ಕೆಲವು ಭಕ್ತೆಯರ ಜೊತೆಯಲ್ಲಿ ಬೃಂದಾವನಯಾತ್ರೆ ಹೊರಟರು.

ಕಾಶೀಪುರದ ಉದ್ಯಾನಗೃಹವನ್ನು ಬಿಟ್ಟುಬಿಡುವ ಮೊದಲೇ ಯುವಕಶಿಷ್ಯರು ಶ್ರೀರಾಮಕೃಷ್ಣರ ಹಾಸಿಗೆ, ಬಟ್ಟೆಗಳು, ಪೀಠೋಪಕರಣಗಳು, ಪಾತ್ರೆ-ಪಡಗಗಳು ಮೊದಲಾದವುಗಳನ್ನೆಲ್ಲ ಬಲ ರಾಮ ಬಾಬುವಿನ ಮನೆಗೆ ಸಾಗಿಸಿಬಿಟ್ಟಿದ್ದರು. ಮುಂದೆ ಅವರು ತಮ್ಮ ವಾಸಸ್ಥಳವನ್ನು ಬದಲಾಯಿಸಿದಾಗಲೆಲ್ಲ ಆ ಅಮೂಲ್ಯ ವಸ್ತುಗಳನ್ನು ತಮ್ಮೊಂದಿಗೆ ಕೊಂಡೊಯ್ಯುತ್ತಿದ್ದರು. ಈ ವಸ್ತುಗಳನ್ನೆಲ್ಲ ಇಂದಿಗೂ ಬೇಲೂರು ಮಠದಲ್ಲಿ ಶ್ರೀರಾಮಕೃಷ್ಣರ ಜನ್ಮೋತ್ಸವದ ದಿನಗಳಲ್ಲಿ ನೋಡಲು ಅವಕಾಶವಿದೆ.

ಆಗಸ್ಟ್ ತಿಂಗಳ ಅಂತ್ಯಕ್ಕೆ ಉದ್ಯಾನಗೃಹವನ್ನು ತೆರವು ಮಾಡಬೇಕಾಗಿತ್ತು. ಯುವ ಶಿಷ್ಯರಲ್ಲಿ ಮೂವರು ಈಗಾಗಲೇ ಶ್ರೀಮಾತೆಯವರ ಜೊತೆಯಲ್ಲಿ ತೀರ್ಥಯಾತ್ರೆಗೆ ಹೋಗಿದ್ದರು. ಉಳಿದ ವರು ತಮ್ಮತಮ್ಮ ಮನೆಗಳಿಗೆ ಹಿಂದಿರುಗಿ ಕಾಲೇಜು ವಿದ್ಯಾಭ್ಯಾಸದಲ್ಲಿ ಮತ್ತೆ ತೊಡಗಿದ್ದರು. ಆದರೆ, ಒಂದು ದೃಷ್ಟಿಯಲ್ಲಿ, ಶ್ರೀರಾಮಕೃಷ್ಣರು ಅವರನ್ನೆಲ್ಲ ಅದಾಗಲೇ ಸಂನ್ಯಾಸಿಗಳನ್ನಾಗಿ ಮಾಡಿಯಾಗಿತ್ತು. ಅಲ್ಲದೆ ಅವರು ನರೇಂದ್ರನಿಗೆ ಸ್ಪಷ್ಟವಾಗಿ ಹೇಳಿದ್ದರು: ‘ಈ ಹುಡುಗರು ಮನೆಗಳಿಗೆ ಹಿಂದಿರುಗದಂತೆ ಮತ್ತು ಅವರು ಸಾಧನೆಯಲ್ಲಿ ತೊಡಗುವಂತೆ ಮಾಡುವ ಜವಾ ಬ್ದಾರಿ ನಿನ್ನದು’ ಎಂದು. ಈಗ ಅವರ ಮುಂದಿನ ದಾರಿಯೇನು? ನೆಲೆನಿಲ್ಲಲು ಅವರಿಗೆ ಸ್ಥಳವಾದರೂ ಎಲ್ಲಿದೆ? ಏನೇ ಆದರೂ ಅವರನ್ನೆಲ್ಲ ಕರೆತಂದು ಸಂನ್ಯಾಸಜೀವನ ನಡೆಸುವಂತೆ ಮಾಡಲೇಬೇಕು ಎಂದು ನರೇಂದ್ರನಂತೂ ದೃಢನಿರ್ಧಾರ ಮಾಡಿಬಿಟ್ಟಿದ್ದ. ಆದರೆ ಎಷ್ಟೋ ಮಂದಿ ಗೃಹಸ್ಥಭಕ್ತರಿಗೆ ಇದು ಒಪ್ಪಿಗೆಯಾಗಲಿಲ್ಲ. “ಈ ಹುಡುಗರು ಮನೆಗಳನ್ನು ಬಿಟ್ಟುಬಂದರೆ ಜೀವನ ನಡೆಸುವುದಾದರೂ ಹೇಗೆ? ಇವರು ಸಾಧಾರಣ ಬೈರಾಗಿಗಳ ಹಾಗೆ ಅಲೆದಾಡಿಕೊಂಡಿರು ವಂತೆ ಬಿಟ್ಟುಬಿಡಲು ನಮ್ಮಿಂದಂತೂ ಸಾಧ್ಯವಿಲ್ಲ! ಇವರಿಗೆಲ್ಲ ಉಜ್ವಲವಾದ ಭವಿಷ್ಯವಿದೆ. ಆದ್ದರಿಂದ ಸುಮ್ಮನೆ ಮನೆಗಳಲ್ಲೇ ಉಳಿದುಕೊಳ್ಳಲಿ, ಅದೇ ಬುದ್ಧಿವಂತಿಕೆಯ ಕೆಲಸ, ಹಾಗೆ ಮಾಡಿದರೆ, ಇವರೂ ಸಂತೋಷವಾಗಿರಬಹುದು, ಮನೆಯವರೂ ಸಂತೋಷಪಡುತ್ತಾರೆ” ಎಂದು ಭಾವಿಸಿ ಅವರಿಗೆ ಬುದ್ಧಿವಾದ ಹೇಳಿದರು. ಶ್ರೀರಾಮಕೃಷ್ಣರ ಅವಶೇಷದ ವಿಷಯವಾಗಿ ಈಗಾಗಲೇ ಜಗಳವಾಗಿದ್ದುದರಿಂದ ಆ ಭಕ್ತರಲ್ಲಿ ಹೆಚ್ಚಿನವರಿಗೆ ಈ ಯುವಕರ ಮೇಲೆ ಸಹಾನುಭೂತಿಯಿರಲಿಲ್ಲ. ಆದರೆ ಅವರ ಪೈಕಿ ಬಲರಾಮ ಬಾಬು, ಸುರೇಂದ್ರನಾಥ, ಮಹೇಂದ್ರ ನಾಥ ಗುಪ್ತ ಮೊದಲಾದ ಕೆಲವೇ ಮಂದಿ ಗೃಹಸ್ಥ ಭಕ್ತರು ಪರಿಸ್ಥಿತಿಯನ್ನು ಸರಿಯಾಗಿ ಅರ್ಥಮಾಡಿಕೊಂಡರು. ಮಹಾತ್ಯಾಗದ ಆದರ್ಶವನ್ನು ತಮ್ಮೊಳಗೆ ತುಂಬಿಕೊಂಡು ಮನೆಗಳನ್ನು ಅದಾಗಲೇ ತ್ಯಜಿಸಿರುವ ಈ ಹುಡುಗರು ಪುನಃ ಮನೆಗಳಿಗೆ ಹಿಂದಿರುಗುವುದು ಸಾಧ್ಯವೇ ಇಲ್ಲ; ಮತ್ತು ಈಗ ಮನೆಗಳಿಗೆ ಹಿಂದಿರುಗಿರುವವರೂ ಕೂಡ ಎಲ್ಲರಂತೆ ವಿದ್ಯಾಭ್ಯಾಸವನ್ನು ಮುಂದು ವರಿಸಲಾರರು ಎಂಬುದನ್ನು ಅವರು ಸ್ಪಷ್ಟವಾಗಿ ಕಂಡರು.

ಈ ಸಂದಿಗ್ಧ ಕಾಲವನ್ನು ನೆನಪಿಸಿಕೊಂಡು ಮುಂದೆ ಸ್ವಾಮಿ ವಿವೇಕಾನಂದರು ತಮ್ಮ ಗುರುಭಾಯಿಯಾದ ಸ್ವಾಮಿ ಬ್ರಹ್ಮಾನಂದರಿಗೆ ಒಂದು ಪತ್ರದಲ್ಲಿ ಬರೆಯುತ್ತಾರೆ: “... ರಾಖಾಲ್, ಅಂದು ಶ್ರೀರಾಮಕೃಷ್ಣರು ಶರೀರತ್ಯಾಗ ಮಾಡಿದ ಮೇಲೆ, ಆ ಗೃಹಸ್ಥರೆಲ್ಲ ನಮ್ಮನ್ನು ನಿಷ್ಪ್ರಯೋಜಕ ಉಡಾಳರು ಅಂತ ಕಡೆಗಣಿಸಿದ್ದು ನಿನಗೆ ನೆನಪಿರಬೇಕು. ಕೇವಲ ಬಲರಾಮ್, ಸುರೇಶ್ ಮಿತ್ರ, ಮಾಸ್ಟರ್ ಮಹಾಶಯ, ಚುನಿಬಾಬು–ಇಷ್ಟು ಜನ ಮಾತ್ರ ಆ ಕಷ್ಟದ ದಿನಗಳಲ್ಲಿ ನಮ್ಮ ಸ್ನೇಹಿತರಾಗಿ ಉಳಿದುಕೊಂಡರು. ಇವರಿಗೆ ನಾವು ಎಷ್ಟು ಕೃತಜ್ಞರಾಗಿದ್ದರೂ ಸಾಲದು.” ಅಮೆರಿಕದ ಪಸಾಡೆನದಲ್ಲಿ ಭಾಷಣ ಮಾಡುತ್ತ ವಿವೇಕಾನಂದರು ಹೇಳುತ್ತಾರೆ: “....ಶ್ರೀರಾಮಕೃಷ್ಣರ ನಿರ್ಯಾಣದ ಬಳಿಕ ನಾವು ಕಷ್ಟದ ದಿನಗಳನ್ನು ಎದುರಿಸಬೇಕಾಯಿತು. ನಮಗೆ ಸ್ನೇಹಿತರು-ಹಿತೈಷಿಗಳು ಯಾರೂ ಇರಲಿಲ್ಲ. ಅರ್ಧಂಬರ್ಧ ತಿಳಿವಳಿಕೆಯ, ಕೇವಲ ಭಾವರಾಜ್ಯದಲ್ಲಿ ಚರಿಸುತ್ತಿರುವ ಕೆಲವು ಹುಡುಗರ ಮಾತನ್ನು ಯಾರು ಕೇಳುತ್ತಾರೆ? ಯೋಚಿಸಿ ನೋಡಿ–ಒಂದು ಹತ್ತು-ಹನ್ನೆರಡು ಹುಡುಗರು, ದೊಡ್ಡದೊಡ್ಡ ವಿಚಾರಗಳನ್ನು ಜನರ ಮುಂದೆ ಬಣ್ಣಿಸುತ್ತ, ತಾವು ತಮ್ಮ ಜೀವನದಲ್ಲಿ ಅವುಗಳನ್ನೆಲ್ಲ ಸಾಧಿಸಿ ತೋರಿಸುವುದಾಗಿ ಹೇಳುತ್ತಿದ್ದಾರೆ! ಇದನ್ನು ಕೇಳಿದವರೆಲ್ಲ ನಕ್ಕರು; ಬಳಿಕ ನಗೆ ನಿಲ್ಲಿಸಿ ಸಿಟ್ಟಿಗೆದ್ದರು, ಛೀಮಾರಿ ಹಾಕಿದರು. ಅಲ್ಲದೆ ನಮಗೆ ತೊಂದರೆ ಕೊಡಲೂ ಶುರುಮಾಡಿದರು. ಕೇವಲ ಒಬ್ಬ ಹುಡುಗನ ಕಲ್ಪನೆಗಳ ಬಗ್ಗೆ ಯಾರು ಸಹಾನುಭೂತಿ ತೋರುತ್ತಾರೆ–ಅದರಲ್ಲೂ ಆ ಕಲ್ಪನೆಗಳಿಂದ ಇತರ ಎಷ್ಟೋ ಜನರಿಗೆ‘ತೊಂದರೆ’ಯಾಗುತ್ತಿರುವಾಗ! ನನ್ನ ಬಗ್ಗೆ ಸಹಾನುಭೂತಿ ತೋರುವವರು ಯಾರಿ ದ್ದರು? ಯಾರೂ ಇಲ್ಲ, ಒಬ್ಬರನ್ನು ಬಿಟ್ಟು... ”

ನಾವು ನೋಡಲಿರುವಂತೆ, ಆ ‘ಒಬ್ಬರು’ ಬೇರೆ ಯಾರೂ ಅಲ್ಲ, ಶ್ರೀಮಾತೆ ಶಾರದಾದೇವಿಯವರು.

ಗೃಹಸ್ಥಭಕ್ತರು ನರೇಂದ್ರಾದಿಗಳ ಬಗ್ಗೆ ಸಹಾನುಭೂತಿ ತೋರದಿದ್ದುದಕ್ಕೆ ಇನ್ನೂ ಒಂದು ಪ್ರಮುಖ ಕಾರಣವಿತ್ತು. ಅದೇನೆಂದರೆ, ಅವರು ಶ್ರೀರಾಮಕೃಷ್ಣರ ಬೋಧನೆಗಳನ್ನು ಅರ್ಥ ಮಾಡಿಕೊಂಡ ಬಗೆಯೇ ಇವರಿಗಿಂತ ಸಂಪೂರ್ಣ ವಿಭಿನ್ನವಾಗಿತ್ತು. “ಶ್ರೀರಾಮಕೃಷ್ಣರು ಸಂನ್ಯಾಸದ ಆದರ್ಶವನ್ನು ಬೋಧಿಸಲೇ ಇಲ್ಲ; ಸಂಸಾರಜೀವನವೇ ಶ್ರೇಷ್ಠ ಅಂತಲೇ ಅವರು ಹೇಳುತ್ತಿದ್ದುದು. ಅವರೇ ಸ್ವತಃ ಗೃಹಸ್ಥರಾಗಿ ಗೃಹಸ್ಥಧರ್ಮವನ್ನು ಎತ್ತಿ ಹಿಡಿಯಲಿಲ್ಲವೆ! ಹಾಗಿರುವಾಗ ನೀವುಗಳು ಸಂನ್ಯಾಸಿಗಳಾಗಿ ಮಠವನ್ನು ಸ್ಥಾಪಿಸಬೇಕಾದ ಅಗತ್ಯವೇ ಇಲ್ಲ” ಎಂದು ವಾದಿಸಿದರು.\footnote{*ನೋಡಿ: ಅನುಬಂಧ ೬.} ಬಲರಾಮ, ಮಾಸ್ಟರ್ ಮಹಾಶಯ, ಸುರೇಂದ್ರ ಮೊದಲಾದ ಕೆಲವೇ ಭಕ್ತರು ಯುವಶಿಷ್ಯರ ಬೆಂಬಲಕ್ಕೆ ನಿಂತರು.

ಇವರಲ್ಲದೆ ಶಿಷ್ಯರ ನಿಲುವನ್ನು ಸಂಪೂರ್ಣವಾಗಿ ಸಮರ್ಥಿಸಿದ ಮತ್ತೊಬ್ಬರೆಂದರೆ ಶ್ರೀ ಶಾರದಾದೇವಿಯವರು. ತಮ್ಮ ಆಧ್ಯಾತ್ಮಿಕ ಸುತರ ಶ್ರೇಯಸ್ಸಿಗಾಗಿ ಅವರು ಶ್ರೀರಾಮಕೃಷ್ಣರಲ್ಲಿ ಕಂಬನಿಗರೆದು ಮೊರೆಯಿಟ್ಟರು: “ಪ್ರಭು, ನೀವು ನರವೇಷಧಾರಿಯಾಗಿ ಬಂದಿರಿ; ಲೀಲೆಯಾಡಿ ದಿರಿ; ಈ ಕೆಲವು ಮಂದಿಯೊಂದಿಗೆ ಒಡನಾಡಿದಿರಿ. ಆದರೆ ಇಷ್ಟಕ್ಕೆ ಎಲ್ಲವೂ ಮುಗಿದುಹೋಗ ಬೇಕೆ? ಇಷ್ಟೇ ಆದರೆ ನೀವು ಈ ದುಃಖಪೂರ್ಣ ಜಗತ್ತಿಗೆ ಅವತರಿಸಿ ಬಂದದ್ದಾದರೂ ಏಕೆ? ಆಹಾರಕ್ಕಾಗಿ ಒಂದೆಡೆಯಿಂದ ಇನ್ನೊಂದೆಡೆಗೆ ಅಲೆದಾಡುವ ಸಾಧುಗಳು ಬಹಳ ಮಂದಿ ಇದ್ದಾರೆ. ಆದರೆ ನಿಮ್ಮ ಹೆಸರಿನಲ್ಲಿ ಮನೆಬಿಟ್ಟು ಬಂದಿರುವ ನನ್ನ ಮಕ್ಕಳು ತುತ್ತು ಅನ್ನಕ್ಕಾಗಿ ಪರದಾಡು ವುದನ್ನು ನಾನು ಸಹಿಸಲಾರೆ. ನಿಮ್ಮನ್ನು ಕೈಮುಗಿದು ಬೇಡಿಕೊಳ್ಳುತ್ತೇನೆ. ನಿಮ್ಮ ಹೆಸರಿನಲ್ಲಿ ತ್ಯಾಗ ಮಾಡಿ ಬಂದವರಿಗೆ ಜೀವನಾಧಾರಕ್ಕೆ ಏನೇನೂ ಕೊರತೆಯಾಗಲೇಬಾರದು. ಅವರೆಲ್ಲ ಒಟ್ಟಿಗೆ ಇದ್ದು ನಿಮ್ಮ ಧ್ಯೇಯಾದರ್ಶಗಳನ್ನು ಅನುಷ್ಠಾನ ಮಾಡಿಕೊಂಡಿರುವಂತಾಗಬೇಕು. ಮುಂದೆ, ಪ್ರಪಂಚದ ತಾಪತ್ರಯಗಳಲ್ಲಿ ತೊಳಲಿ ಬೆಂಡಾದ ಜನ ತಮ್ಮ ಬಳಿಗೆ ಬಂದಾಗ, ಇವರು ನಿಮ್ಮ ಜೀವನ-ಸಂದೇಶಗಳನ್ನು ತಿಳಿಸಿ, ಅವರಿಗೆ ಸಾಂತ್ವನ ನೀಡುವಂತಾಗಬೇಕು. ಈ ಉದ್ದೇಶಕ್ಕಾಗಿ ಬಂದವರಲ್ಲವೆ ನೀವು? ಆದರೆ ನನ್ನ ಮಕ್ಕಳು ಹೀಗೆ ಅಲೆದಾಡುವುದನ್ನು ಕಂಡಾಗ ನನ್ನ ಹೃದಯ ಹಿಂಡಿದಂತಾಗುತ್ತದೆ.”

ಶ್ರೀಶಾರದಾದೇವಿಯವರ ವಾತ್ಸಲ್ಯಪೂರ್ಣ ಹೃದಯದಿಂದ ಹೊರಹೊಮ್ಮಿದ ಈ ಪ್ರಾರ್ಥನೆ ಯನ್ನು ಶ್ರೀರಾಮಕೃಷ್ಣರು ನಡೆಸಿಕೊಡದಿದ್ದಾರೆಯೆ? ಒಂದು ದಿನ ಸುರೇಂದ್ರನಾಥ ಆಫೀಸಿನಿಂದ ಮನೆಗೆ ಹಿಂದಿರುಗಿ ಎಂದಿನಂತೆ ಧ್ಯಾನಕ್ಕೆ ಕುಳಿತಿದ್ದಾನೆ; ಆಗ ಶ್ರೀರಾಮಕೃಷ್ಣರು ಅವನ ಮುಂದೆ ಕಾಣಿಸಿಕೊಂಡು ನುಡಿದರು: “ಓ! ಏನು ಮಾಡುತ್ತಿದ್ದೀಯ ನೀನು? ಅಲ್ಲಿ ನನ್ನ ಮಕ್ಕಳೆಲ್ಲ ಬೀದಿಯಲ್ಲಿ ಅಲೆದಾಡುತ್ತಿದ್ದಾರೆ. ಅವರ ಅವಸ್ಥೆಯನ್ನು ನೋಡು! ಹೋಗು, ತಡಮಾಡದೆ ಅವರಿಗೊಂದು ವ್ಯವಸ್ಥೆಮಾಡು.” ತಕ್ಷಣ ಸುರೇಂದ್ರ ದಡಬಡಿಸಿ ಎದ್ದು ನರೇಂದ್ರನ ಮನೆಗೆ ಓಡಿದ. ತನಗಾದ ದರ್ಶನವನ್ನು ಬಣ್ಣಿಸಿ ಅಶ್ರುಭರಿತನಾಗಿ ಹೇಳಿದ: “ನರೇನ್, ನೀವೆಲ್ಲ ಎಲ್ಲಿಗೆ ಹೋಗುತ್ತೀರಿ? ನಾವೀಗ ಒಂದು ಬಾಡಿಗೆ ಮನೆ ಹಿಡಿಯೋಣ. ನೀವು ಹುಡುಗರೆಲ್ಲ ಅದರಲ್ಲಿ ವಾಸವಾಗಿರಿ. ಆ ಮನೆ ಶ್ರೀರಾಮಕೃಷ್ಣರ ಮಂದಿರವಾಗಿರಲಿ, ನಾವು ಗೃಹಸ್ಥರು ಆಗಾಗ ಬಂದು ನಿಮ್ಮೊಡನಿದ್ದು, ನಿಮ್ಮ ಆಧ್ಯಾತ್ಮಿಕ ಆನಂದದಲ್ಲಿ ಭಾಗಿಗಳಾಗುತ್ತೇವೆ, ಮತ್ತು ತಾಪತ್ರಯ ಗಳಿಂದ ಬೆಂದ ನಮ್ಮ ಹೃದಯಕ್ಕೆ ಶಾಂತಿ ತಂದುಕೊಳ್ಳುತ್ತೇವೆ. ಇದುವರೆಗೂ ನಾನು ಶ್ರೀರಾಮ ಕೃಷ್ಣರಿಗಾಗಿ ಒಂದಿಷ್ಟು ಹಣವನ್ನು ಖರ್ಚುಮಾಡುತ್ತಿದ್ದೆ. ಇನ್ನುಮೇಲೆ ಅದನ್ನೇ ಸಂತೋಷದಿಂದ ನಿಮಗೆ ಕೊಡುತ್ತೇನೆ.” ಇದನ್ನು ಕೇಳುತ್ತಿದ್ದಂತೆಯೇ ನರೇಂದ್ರನಿಗೆ ಶ್ರೀರಾಮಕೃಷ್ಣರೇ ಹಿನ್ನೆಲೆ ಯಲ್ಲಿದ್ದುಕೊಂಡು ತಮ್ಮನ್ನು ಮುನ್ನಡೆಸುತ್ತಿದ್ದಾರೆ ಎನ್ನುವುದು ಗೋಚರವಾಯಿತು. ಹನಿ ಗೂಡಿದ ಕಂಗಳಿಂದ ಅವನು ಸುರೇಂದ್ರನಿಗೆ ಧನ್ಯವಾದ ತಿಳಿಸಿ ಮುಂದಿನ ಕಾರ್ಯದಲ್ಲಿ ತೊಡಗಿದ.

ದಕ್ಷಿಣೇಶ್ವರ-ಕಲ್ಕತ್ತಗಳ ನಡುವಿನ ಬಾರಾನಗೋರ್ (ವರಾಹನಗರ) ಎಂಬಲ್ಲಿ ಗಂಗಾತೀರ ದಲ್ಲೇ ಮನೆಯೊಂದು ಸಿಕ್ಕಿತು. ನಿಜಕ್ಕೂ, ಆ ಮನೆಯನ್ನು ಬಾಡಿಗೆಗೆ ಹಿಡಿಯುವವರು ಒಳ್ಳೇ ಧೀರರೇ ಆಗಿರಬೇಕಾಗಿತ್ತು. ಅದೊಂದು ಪಾಳುಬಿದ್ದಿದ್ದ, ಎಲ್ಲರೂ ಕೈಬಿಟ್ಟಿದ್ದ ಮಹಡಿ ಮನೆ. ಈಗಲೋ ಆಗಲೋ ಕುಸಿದು ಬೀಳಲು ಸಿದ್ಧವಾದಂತಿತ್ತು. ಜೊತೆಗೆ, ‘ದೆವ್ವದ ಮನೆ’ ಎಂಬ ಕೀರ್ತಿ ಬೇರೆ ಅದಕ್ಕೆ! ನೆಲ ಅಂತಸ್ತು ಹಾವು ಹಲ್ಲಿಗಳ ಆವಾಸಸ್ಥಾನ. ಕಾಂಪೌಂಡಿನ ಗೇಟು ಬಿದ್ದುಹೋಗಿ ಎಷ್ಟೋ ವರ್ಷವಾಗಿತ್ತು. ಮೇಲಿನ ಮಾಳಿಗೆ ಇದ್ದುದರಲ್ಲಿ ವಾಸಿಯೇನೋ. ಆದರೆ ಅಲ್ಲೊಂದು ಉದ್ದನೆಯ ಜಗಲಿಯಿದ್ದು ಅದೂ ಶಿಥಿಲವಾಗಿತ್ತು. ಈ ಯುವಕರ ವಾಸಕ್ಕೆಂದು ಗೊತ್ತುಪಡಿಸಿದ ಕೋಣೆಯೂ ಅಷ್ಟೆ, ಮನುಷ್ಯರ ವಾಸಕ್ಕಂತೂ ಯೋಗ್ಯವಾಗಿರಲಿಲ್ಲ! ಬಹುಶಃ ಆ ಯುವಸಂನ್ಯಾಸಿಗಳಲ್ಲದೆ ಬೇರೆ ಯಾರೂ ಅಲ್ಲಿ ವಾಸಿಸುವಂತಹ ಧೈರ್ಯ ಮಾಡಿದ್ದಿರ ಲಾರರು. ಮನೆಯ ಒಂದು ಬದಿಯಲ್ಲಿ ಕಳೆಗಿಡಗಳು ಬಳ್ಳಿಗಳು ಹೇರಳವಾಗಿ ಬೆಳೆದುಕೊಂಡು ಕಾಡಿನಂತಿತ್ತು. ಹಿಂಭಾಗದಲ್ಲೊಂದು ಕೊಳ–ಅದು ಹಚ್ಚಹಸಿರಾದ ಪಾಚಿಯಿಂದ ತುಂಬಿದ್ದು ದಲ್ಲದೆ ಲಕ್ಷಗಟ್ಟಲೆ ಸೊಳ್ಳೆಗಳಿಗೆ ಉಗಮಸ್ಥಾನವಾಗಿತ್ತು. ಒಟ್ಟಿನಲ್ಲಿ ಆ ಮನೆಯ ಒಳಹೊರಗೆಲ್ಲ ಎಂಥದೋ ಪ್ರೇತಕಳೆ! ಇದೇ ಶ್ರೀರಾಮಕೃಷ್ಣ ಮಹಾಸಂಘದ ಮೊಟ್ಟಮೊದಲ ‘ಮಠ.’ ಆದರೆ ಎಲ್ಲ ಬಿಟ್ಟು ಈ ಕಟ್ಟಡವನ್ನೇಕೆ ಹಿಡಿದರು? ಕಾರಣ ತುಂಬ ಸರಳ–ಬಾಡಿಗೆ ಬಹಳ ಕಡಿಮೆ; ತಿಂಗಳಿಗೆ ಹನ್ನೊಂದು ರೂಪಾಯಿ. ಜೊತೆಗೆ ಇನ್ನೂ ಕೆಲವು ಅನುಕೂಲಗಳಿದ್ದುವು–ಇದು ಗಂಗಾನದಿಗೆ ಹತ್ತಿರದಲ್ಲೇ ಇದ್ದುದರಿಂದ ಸ್ನಾನಾದಿಗಳಿಗೆ ಬಹಳ ಅನುಕೂಲವಾಗಿತ್ತು. ಅಲ್ಲದೆ ಶ್ರೀರಾಮಕೃಷ್ಣರ ಅಂತ್ಯಕ್ರಿಯೆ ನಡೆದ ಪವಿತ್ರ ಸ್ಥಳವೂ ಇಲ್ಲಿಗೆ ಹತ್ತಿರ. ಜೊತೆಗೆ ಇದು ನಗರದ ಗಡಿಬಿಡಿ ಗಲಾಟೆಗಳಿಂದ ದೂರದಲ್ಲಿದ್ದು ತುಂಬ ಪ್ರಶಾಂತವಾಗಿತ್ತು. ಆದ್ದರಿಂದ ಯುವಶಿಷ್ಯರು ಸಂತೋಷದಿಂದಲೇ ಈ ಮನೆಯನ್ನು ತಮ್ಮದನ್ನಾಗಿ ಮಾಡಿಕೊಂಡರು.

ಯುವಶಿಷ್ಯರು ಒಬ್ಬೊಬ್ಬರಾಗಿ ಬಂದು ಈ ಮಠದಲ್ಲಿ ವಾಸಿಸಲಾರಂಭಿಸಿದರು. ವಾರಾಣಸಿ ಯಲ್ಲಿದ್ದ ತಾರಕನಾಥನಿಗೆ ನರೇಂದ್ರ ತಂತಿ ಕೊಟ್ಟು ಕರೆಸಿದ. ಅವನು ಬಂದು ಹಿರಿಯ ಗೋಪಾಲನೊಡನೆ ಮಠದಲ್ಲಿ ವಾಸಿಸಲಾರಂಭಿಸಿದ. ಸುಮಾರು ಒಂದು ತಿಂಗಳ ಬಳಿಕ ಮಠದ ವಿಚಾರ ತಿಳಿದು ಕಾಳೀಪ್ರಸಾದನೂ ಬೃಂದಾವನದಿಂದ ಹಿಂದಿರುಗಿ ಬಂದ. ಹೀಗೆ ಶ್ರೀರಾಮ ಕೃಷ್ಣರು ದೇಹತ್ಯಾಗ ಮಾಡಿದ ಆರು ವಾರಗಳಲ್ಲಿ ಮಠ ಅಸ್ತಿತ್ವಕ್ಕೆ ಬಂದಿತು.

ಪ್ರಾರಂಭದಲ್ಲಿ ಮಠದಲ್ಲೇ ವಾಸವಾಗಿದ್ದವರೆಂದರೆ ತಾರಕನಾಥ, ಹಿರಿಯಗೋಪಾಲ ಮತ್ತು ಕಾಳೀಪ್ರಸಾದ ಈ ಮೂವರು. ಉಳಿದವರೆಲ್ಲ ಆಗಾಗ ಬಂದು ಹೋಗುತ್ತಿದ್ದರು. ಕೆಲವರು ತಮ್ಮ ಕಾಲೇಜು ವಿದ್ಯಾಭ್ಯಾಸವನ್ನು ಪುನರಾರಂಭಿಸಿದ್ದರು. ಮಠವನ್ನು ಸೇರುವುದೋ ಬಿಡುವುದೋ ಎಂದು ಅನೇಕರ ಮನಸ್ಸು ಇನ್ನೂ ಹೊಯ್ದಾಡುತ್ತಿತ್ತು. ನರೇಂದ್ರ ತನ್ನ ಈ ಸ್ನೇಹಿತರ ಮನೆ ಗಳಿಗೇ ಹೋಗಿ, ತಕ್ಷಣ ಹೊರಟುಬಂದು ಮಠವನ್ನು ಸೇರಿಕೊಳ್ಳುವಂತೆ ತನ್ನ ಪ್ರಖರ ವಿಚಾರ ಧಾರೆಯಿಂದ ಅವರನ್ನು ಸ್ಫೂರ್ತಿಗೊಳಿಸಲಾರಂಭಿಸಿದ. ಅವರೊಂದಿಗೆ ಹಲವಿಧವಾಗಿ ಚರ್ಚಿಸಿ, ತ್ಯಾಗಜೀವನವೇ ಪರಮಶ್ರೇಷ್ಠವಾದದ್ದು ಎಂದು ಮನಮುಟ್ಟುವಂತೆ ವಿವರಿಸಿದ. ಶ್ರೀರಾಮ ಕೃಷ್ಣರ ಜೀವನ ಮತ್ತು ಸಂದೇಶಗಳನ್ನು ಉಜ್ವಲವಾಗಿ ವರ್ಣಿಸಿ ಅವರ ಮನಸ್ಸನ್ನೂ ಹೃದಯ ವನ್ನೂ ಗೆದ್ದುಕೊಳ್ಳುವ ಪ್ರಯತ್ನ ಮಾಡಿದ.

ಇದೇ ಸಂದರ್ಭದಲ್ಲಿ ಈ ಯುವಕರೆಲ್ಲರ ವೈರಾಗ್ಯ ಮನೋಭಾವವೂ ಸುದೃಢಗೊಂಡು ಒಂದು ನಿಶ್ಚಿತ ನಿಲುವಿಗೆ ಬಂದು ನಿಲ್ಲುವಂಥ ಪ್ರಸಂಗವೊಂದು ತಾನಾಗಿಯೇ ಒದಗಿಬಂತು. ಶಿಷ್ಯರಲ್ಲೊಬ್ಬನಾದ ಬಾಬುರಾಮನ ತಾಯಿ ಮಾತಂಗಿನೀದೇವಿ ಶ್ರೀರಾಮಕೃಷ್ಣರ ಪರಮಭಕ್ತೆ. ಈ ತರುಣ ಶಿಷ್ಯರ ಬಗ್ಗೆ ಅವಳಿಗೆ ವಿಶೇಷ ಆದರ, ಅಭಿಮಾನ. ಅವಳು ತ್ಯಾಗಜೀವನವನ್ನು ಸ್ವೀಕರಿಸಿದ್ದ ನರೇಂದ್ರ ಹಾಗೂ ಇನ್ನಿಬ್ಬರು ಮೂವರನ್ನು ತನ್ನ ಹಳ್ಳಿಯ ಮನೆಗೆ ಬಂದು ಆತಿಥ್ಯ ಸ್ವೀಕರಿಸುವಂತೆ ಕೋರಿದಳು. ಈ ಸುದ್ದಿ ತಿಳಿದಾಗ ಇನ್ನೂ ಅನೇಕರು ನರೇಂದ್ರನೊಂದಿಗೆ ತಾವೂ ಹೋಗಲು ಉತ್ಸಾಹ ತಾಳಿದರು. ಕಡೆಗೆ ಅದು ಸಾಕಷ್ಟು ದೊಡ್ಡ ಗುಂಪೇ ಆಯಿತು. ೧೮೮೬ರ ಡಿಸೆಂಬರಿನಲ್ಲೊಂದು ದಿನ ನರೇಂದ್ರ, ಬಾಬುರಾಮ, ಶರಚ್ಚಂದ್ರ, ಶಶಿಭೂಷಣ, ತಾರಕನಾಥ, ಕಾಳೀಪ್ರಸಾದ, ನಿರಂಜನ, ಗಂಗಾಧರ, ಶಾರದಾಪ್ರಸನ್ನ–ಇಷ್ಟು ಜನ ರೈಲುಮಾರ್ಗವಾಗಿ ಹೊರಟರು. ತಮ್ಮೊಡನೆ ತಾಳ ತಂಬೂರಿ ತಬಲ ಮೊದಲಾದ ವಾದ್ಯಗಳನ್ನು ಒಯ್ದಿದ್ದರು. ಉದ್ದಕ್ಕೂ ಭಜನೆ ಮಾಡುತ್ತ, ಭಗವದಾನಂದವನ್ನು ಸವಿಯುತ್ತ ಪ್ರಯಾಣ ಮಾಡಿದರು. ಹರಿಪಾಲ ನಿಲ್ದಾಣದಲ್ಲಿ ಇಳಿದು ಅಲ್ಲಿಂದ ಬಾಬುರಾಮನ ಊರಾದ ಆಂಟ್ಪುರಕ್ಕೆ ಗಾಡಿಯಲ್ಲಿ ಸಾಗಿದರು. ಎಲ್ಲರನ್ನೂ ಮಾತಂಗಿನೀ ದೇವಿ ವಾತ್ಸಲ್ಯದಿಂದ ಬರಮಾಡಿಕೊಂಡಳು.

ಆಂಟ್ ಪುರ ಒಂದು ಸುಂದರ ಗ್ರಾಮ. ಆ ನೀರವ ಪ್ರಶಾಂತ ವಾತಾವರಣದಲ್ಲಿ, ಈ ಯುವಸಾಧಕರೊಳಗಿನ ಆಧ್ಯಾತ್ಮಿಕ ಜ್ವಾಲೆ ಪ್ರಜ್ವಲಿಸಲಾರಂಭಿಸಿತು. ಆ ಜ್ವಾಲೆಗೆ ನರೇಂದ್ರನ ಉತ್ಸಾಹದ ಬೀಸಣಿಗೆ ಗಾಳಿ ಬೇರೆ! ಸ್ವಯಂ ಶ್ರೀರಾಮಕೃಷ್ಣರ ಚೈತನ್ಯವೇ ಅವನ ಮೂಲಕ ಮಾತನಾಡುತ್ತಿದೆಯೋ, ಅವನ ಮೂಲಕ ಕೆಲಸ ಮಾಡುತ್ತಿದೆಯೋ ಎನ್ನುವಂತಿತ್ತು. ಸಂನ್ಯಾಸ ಜೀವನದ ದರ್ಶನ ಅವನ ಕನಸು ಮನಸ್ಸುಗಳನ್ನೆಲ್ಲ ತುಂಬಿಬಿಟ್ಟಿತ್ತು. ಆದ್ದರಿಂದ ಅವನು ಆಗಾಗ ಗುಡುಗುತ್ತಿದ್ದ: “ಸಿಂಹಸದೃಶ ವ್ಯಕ್ತಿತ್ವ ನಿರ್ಮಾಣವೇ ನಮ್ಮ ಜೀವನದ ಧ್ಯೇಯವಾಗಲಿ. ಅದೇ ನಮ್ಮ ಏಕಮಾತ್ರ ಆಧ್ಯಾತ್ಮಿಕ ಸಾಧನೆಯೂ ಆಗಲಿ. ನೀರಸ-ನಿರರ್ಥಕ ಅಧ್ಯಯನವನ್ನೆಲ್ಲ ಬಿಟ್ಟುಬಿಡಿ. ಜಗತ್ತಿನ ಥಳಕಿನ ವಿಲಾಸಗಳು ನಮ್ಮ ಮನಸ್ಸನ್ನು ಕ್ಷಣಕಾಲವೂ ಸೆಳೆಯದಿರಲಿ. ಭಗವಂತನ ಸಾಕ್ಷಾತ್ಕಾರವೇ ನಮ್ಮ ಜೀವನದ ಏಕಮಾತ್ರ ಗುರಿ. ನಮ್ಮ ಗುರುದೇವರು ತಮ್ಮ ಜೀವನದ ಮೂಲಕ ತೋರಿಸಿಕೊಟ್ಟ ಆದರ್ಶವೇ ಇದು–ನಾವು ಭಗವಂತನನ್ನು ಸಾಕ್ಷಾತ್ಕರಿಸಿ ಕೊಳ್ಳಲೇಬೇಕು.”

ಅವನ ಈ ಸ್ಫೂರ್ತಿಯ ಕರೆಯನ್ನು ಕೇಳಿ ಗುರುಭಾಯಿಗಳೆಲ್ಲ ಒಮ್ಮತದಿಂದ ಸಾಧನೋ ನ್ಮುಖರಾದರು. ತಮ್ಮನ್ನೆಲ್ಲ ಯಾವುದೋ ಅದ್ಭುತ ಶಕ್ತಿ ಒಂದುಗೂಡಿಸುತ್ತಿರುವಂತೆ ಅವರಿಗೆ ಭಾಸವಾಗುತ್ತಿತ್ತು. ತಾವೆಲ್ಲ ಒಂದೇ ಶರೀರ, ಒಂದೇ ಮನಸ್ಸು, ಒಂದೇ ಆತ್ಮವಾಗಿ ಬೆಳೆಯು ತ್ತಿರುವ ಅನುಭವವಾಯಿತು. ಹಗಲಿರುಳೂ ಸಾಧನೆಯಲ್ಲಿ ಮುಳುಗಿಹೋದರು. ಅವರ ಮನಸ್ಸಿ ನಲ್ಲಿ ಶ್ರೀರಾಮಕೃಷ್ಣರೂಪ; ನಾಲಿಗೆಯಲ್ಲಿ ಶ್ರೀರಾಮಕೃಷ್ಣನಾಮ! ಅವರೆಲ್ಲರ ಮನಸ್ಸಿನಲ್ಲೂ ತ್ಯಾಗಜೀವನದ ಆದರ್ಶ ಜಾಗೃತವಾಗಿಬಿಟ್ಟಿತು; ಸಂನ್ಯಾಸಜೀವನವನ್ನು ಸ್ವೀಕರಿಸುವ ಇಚ್ಛೆ ಪ್ರಜ್ವಲಿಸಲಾರಂಭಿಸಿತು. ಈಗ ಅವರಲ್ಲಿ, ತಮ್ಮ ಉದ್ಧಾರಕ್ಕಾಗಿ ಮಾತ್ರವಲ್ಲದೆ ಪರಹಿತಕ್ಕಾಗಿ ಸಂನ್ಯಾಸಜೀವನವನ್ನು ಕೈಗೊಳ್ಳಬೇಕೆಂಬ ಆಕಾಂಕ್ಷೆ ಕ್ರಮೇಣ ಬಲವಾಯಿತು. ಪ್ರತಿಯೊಬ್ಬನ ನಿರ್ಧಾರಕ್ಕೆ ಇತರರು ಸಾಕ್ಷಿಗಳಾಗಿರುವಂತೆ, ಎಲ್ಲರೂ ಒಟ್ಟಾಗಿ ಸಂನ್ಯಾಸ ಸ್ವೀಕರಿಸುವ ಮನಸ್ಸು ಮಾಡಿದರು. ಪ್ರತಿಯೊಬ್ಬನೂ ತನ್ನ ಸೋದರಶಿಷ್ಯರಲ್ಲಿ ಅಪಾರ ಆಧ್ಯಾತ್ಮಿಕ ಶಕ್ತಿ ವ್ಯಕ್ತವಾಗಿರು ವುದನ್ನು ಕಾಣುತ್ತಿದ್ದ. ಇದರಿಂದ ಅವರಲ್ಲಿ ಪರಸ್ವರ ಪ್ರೀತಿ, ಗೌರವ ಪೂಜ್ಯತೆಯ ಭಾವ ನೂರ್ಮಡಿಗೊಂಡಿತು. ಅವರ ಪರಸ್ಪರ ಬಂಧನ ಮತ್ತಷ್ಟು ಬಲಗೊಳ್ಳತೊಡಗಿತು. ಇದು ತೀರ ಸಹಜವೇ ಎನ್ನಬೇಕು. ಏಕೆಂದರೆ, ಅವರೆಲ್ಲರ ಹೃದಯಗಳಲ್ಲೂ ಈಗ ಶ್ರೀರಾಮಕೃಷ್ಣರೇ ನೆಲಸಿಬಿಟ್ಟಿದ್ದರು.

ಅದೊಂದು ನಿರ್ಮಾಲಾಕಾಶದ ರಾತ್ರಿ; ಎಲ್ಲೆಲ್ಲೂ ಮಹಾಮೌನ ತಾನೇತಾನಾಗಿದೆ. ಈ ಯುವಸಾಧಕರೆಲ್ಲ ಸೇರಿ ಒಂದು ಧುನಿಯನ್ನು ಹೊತ್ತಿಸಿಕೊಂಡು ಸುತ್ತಲೂ ಕುಳಿತು ಧ್ಯಾನಮಗ್ನ ರಾಗಿದ್ದಾರೆ. ಧ್ಯಾನ ನಿರಾತಂಕವಾಗಿ ಸಾಗಿದೆ. ಬಹಳ ಹೊತ್ತಿನ ಮೇಲೆ ಈಗ ಎಲ್ಲರೂ ಬಹಿರ್ಮುಖರಾದರು. ನರೇಂದ್ರ ತನ್ನ ಗುರುಭಾಯಿಗಳ ಮುಂದೆ ತ್ಯಾಗಜೀವನದ ಘನತೆಯನ್ನು ಬಣ್ಣಿಸಲಾರಂಭಿಸಿದ. ಆಗ ಅವನಲ್ಲಿ ಮಹಾಕಾರುಣಿಕ ಏಸುಕ್ರಿಸ್ತನ ಸ್ಫುರಣೆಯಾಗಿಬಿಟ್ಟಿತು. ಅವನು ಏಸುವಿನ ಅದ್ಭುತ ಜೀವನ ಕಥೆಯನ್ನು ಸ್ಫೂರ್ತಿಯುತನಾಗಿ ವಿವರಿಸತೊಡಗಿದ. ಕ್ರಿಸ್ತನ ಜನನದಿಂದಾರಂಭಿಸಿ, ಅವನ ಸಂಪೂರ್ಣ ಜೀವನ, ಪರಿನಿರ್ಯಾಣ ಹಾಗೂ ಪುನರುತ್ಥಾನ– ಇವುಗಳನ್ನೆಲ್ಲ ಮನಮುಟ್ಟುವಂತೆ ತಿಳಿಸಿದ. ಎಲ್ಲ ಅಡಚಣೆಗಳನ್ನೂ ಎದುರಿಸಿ, ಏಸುಕ್ರಿಸ್ತನ ಜೀವನ-ಬೋಧನೆಗಳನ್ನು ಜಗತ್ತಿನಾದ್ಯಂತ ಪ್ರಸಾರ ಮಾಡುವಂತೆ ಸಂತ ಪಾಲ್ನನ್ನು ಪ್ರೇರೇಪಿ ಸಿದ ಶಕ್ತಿ-ಉತ್ಸಾಹಗಳು, ಈಗ ಈ ಯುವಸಾಧಕರಲ್ಲೂ ಸ್ವಲ್ಪಮಟ್ಟಿಗೆ ಜಾಗೃತವಾಗುವಂತೆ ಮಾಡಿಬಿಟ್ಟಿತು–ನರೇಂದ್ರನ ಆ ಅದ್ಭುತ ವಾಗ್ವೈಖರಿ! ಅವರಲ್ಲೆಲ್ಲ ನವೋತ್ಸಾಹ ಉತ್ಪನ್ನ ವಾಗುವಂತೆ ಅವನು ನುಡಿದ: “ನೀವು ಪ್ರತಿಯೊಬ್ಬರೂ ಏಸುಕ್ರಿಸ್ತರೇ ಆಗಿಬಿಡಬೇಕು. ಮತ್ತು ಅವನಂತೆಯೇ ಜಗತ್ತಿನ ನೊಂದ ಜನತೆಯನ್ನು ಸಾಂತ್ವನಗೊಳಿಸುವ ಪ್ರಯತ್ನ ಮಾಡಬೇಕು. ಕ್ರಿಸ್ತನಂತೆಯೇ ನೀವೂ ನಿಮ್ಮ ಸ್ವಾರ್ಥವನ್ನು ತ್ಯಜಿಸಬೇಕು. ಮತ್ತು ಭಗವಂತನನ್ನು ಸಾಕ್ಷಾತ್ಕರಿಸಿ ಕೊಳ್ಳಬೇಕು.”

ಉರಿಯುತ್ತಿರುವ ಧುನಿಯ ಮುಂದೆ ಕುಳಿತ ಅವರೆಲ್ಲರ ಮುಖಗಳು ಅಗ್ನಿತೇಜದಿಂದ ಬೆಳಗುತ್ತಿವೆ. ಉರಿಯುವ ಕಟ್ಟಿಗೆಯ ಚಟಪಟ ಸದ್ದನ್ನು ಬಿಟ್ಟರೆ ನೀರವ ಮೌನದ ಆ ರಾತ್ರಿಯಲ್ಲಿ ಎಲ್ಲರೂ ಪರಸ್ಪರರ ಸಮ್ಮುಖದಲ್ಲಿ ಹಾಗೂ ಸರ್ವಸಾಕ್ಷಿಯಾದ ಭಗವಂತನ ಸಮ್ಮುಖದಲ್ಲಿ ಸಂನ್ಯಾಸದ ವ್ರತವನ್ನು ಕೈಗೊಂಡರು. ಹೀಗೆ ಮಹಾತ್ಯಾಗ ಮೈದಳೆಯಿತು ಆ ಹಳ್ಳಿಯಲ್ಲಿ; ಮಹಾತ್ಯಾಗದ ಫಲವಾದ ಮಹಾನಂದ ಹಾಗೂ ಅಮೃತತ್ವದ ಅಲೆ ಅಲ್ಲೆಲ್ಲ ವ್ಯಾಪಿಸಿತು.

ಬಳಿಕ ಅವರಿಗೆ ತಿಳಿದುಬಂತು–ಇವೆಲ್ಲ ನಡೆದದ್ದು ಕ್ರಿಸ್ಮಸ್ ಈವ್ ದಿನದಂದು (ಎಂದರೆ, ಕ್ರಿಸ್ಮಸ್ ಹಬ್ಬದ ಹಿಂದಿನ ದಿನವಾದ ಡಿಸೆಂಬರ್ ೨೪ರಂದು) ಎಂದು. ತನಗರಿವಿಲ್ಲದಂತೆಯೇ ಅಂದು ನರೇಂದ್ರ ಕ್ರಿಸ್ತಭಾವದಿಂದ ರಂಜಿತನಾಗಿದ್ದ; ಕ್ರಿಸ್ತನ ಜೀವನದಿಂದ ಸ್ಫೂರ್ತಿಗೊಂಡಿದ್ದ; ಅಲ್ಲದೆ, ಅದನ್ನೇ ಇತರರಿಗೂ ತಿಳಿಸಿ ಅವರಲ್ಲೂ ಸ್ಫೂರ್ತಿಯುಂಟುಮಾಡಿದ್ದ. ಇದೊಂದು ಅಪೂರ್ವ ಸಂಯೋಗವೇ ಸರಿ.

ಮುಂದೊಮ್ಮೆ ಸ್ವಾಮಿ ಶಿವಾನಂದರು (ತಾರಕನಾಥ) ಈ ದಿನಗಳನ್ನು ನೆನಪಿಸಿಕೊಂಡು ಹೇಳು ತ್ತಾರೆ: “ನಿಜಕ್ಕೂ ನಾವೆಲ್ಲರೂ ಸೇರಿ ಸುಸಂಘಟಿತರಾದದ್ದೇ ಆಂಟ್ಪುರದಲ್ಲಿ. ಶ್ರೀರಾಮ ಕೃಷ್ಣರು ಹಿಂದೆಯೇ ನಮ್ಮನ್ನೆಲ್ಲ ಸಂನ್ಯಾಸಿಗಳನ್ನಾಗಿ ಮಾಡಿದ್ದರಾದರೂ ಆ ತ್ಯಾಗ-ವೈರಾಗ್ಯಭಾವ ಆಂಟ್ಪುರದಲ್ಲಿ ದೃಢಪಟ್ಟಿತು.”

ಈ ಯುವಕರು ಆಂಟ್ ಪುರದಲ್ಲಿ ಸುಮಾರು ಒಂದು ವಾರ ಇದ್ದು, ಹಿಂದಿರುಗಿ ಬರುವಾಗ ತಾರಕೇಶ್ವರಕ್ಕೆ ಹೋಗಿ, ಅಲ್ಲಿ ಸಂನ್ಯಾಸಿಗಳ ದೇವನಾದ ಶಿವನನ್ನು ಪೂಜಿಸಿ ಕಲ್ಕತ್ತಕ್ಕೆ ಮರಳಿದರು.

ಇದಾದ ಕೆಲವೇ ದಿನಗಳಲ್ಲಿ ಶಶಿಭೂಷಣ, ಶರಚ್ಚಂದ್ರ, ಶಾರದಾಪ್ರಸನ್ನ, ನಿರಂಜನ, ಸುಬೋಧ ಹಾಗೂ ಬಾಬುರಾಮ–ಇವರೆಲ್ಲ ಒಬ್ಬೊಬ್ಬರಾಗಿ ಬಂದು ಬಾರಾನಾಗೋರ್ ಮಠದಲ್ಲಿ ನೆಲಸಿದರು. ಇಷ್ಟು ದಿನ ಮೋಂಘೀರ್ ಎಂಬಲ್ಲಿದ್ದ ರಾಖಾಲನೂ ಜನವರಿಯ ವೇಳೆಗೆ ಹಿಂದಿರುಗಿದ. ಗಂಗಾಧರ ಈ ವೇಳೆಗಾಗಲೇ ಸಂಘಕ್ಕೆ ಸೇರಿಕೊಂಡಿದ್ದನಾದರೂ ಮರು ವರ್ಷ ಟಿಬೆಟ್ಟಿಗೆ ಯಾತ್ರಿಕನಾಗಿ ಹೋಗಿ, ಮೂರು ವರ್ಷಗಳ ಬಳಿಕ ಹಿಂದಿರುಗಿದ ಮೇಲಷ್ಟೇ ಮಠದಲ್ಲಿ ನೆಲೆನಿಂತ. ಬೃಂದಾವನಕ್ಕೆ ಹೋಗಿದ್ದ ಲಾಟು ಆರು ತಿಂಗಳ ಬಳಿಕ, ಮತ್ತು ಶಾರದಾದೇವಿಯವರ ಜೊತೆಯಲ್ಲಿ ಹೋಗಿದ್ದ ಯೋಗೀಂದ್ರ ಒಂದು ವರ್ಷದ ಮೇಲೆ ಮಠಕ್ಕೆ ಸೇರಿದರು. ಹರಿನಾಥನೂ ೧೮೮೭ರ ಸುಮಾರಿಗೆ ಸೇರಿಕೊಂಡ. ಹರಿಪ್ರಸನ್ನ ಕಾರಣಾಂತರದಿಂದ ತುಂಬ ತಡವಾಗಿ, ಎಂದರೆ, ೧೮೯೫ರಲ್ಲಿ ಸೇರಿದ.

ಹೀಗೆ ನರೇಂದ್ರನ ನಾಯಕತ್ವದಲ್ಲಿ ರಾಮಕೃಷ್ಣಸಂಘವೆಂಬುದು ಒಂದು ರೂಪಕ್ಕೆ ಬಂದಿತು. ಈ ಮಧ್ಯೆ ಅವನು ತನ್ನ ಮನೆಯ ಕಡೆಗೂ ಗಮನ ಹರಿಸಬೇಕಾಗಿತ್ತು. ಅವನ ಮನೆಗೆ ಸಂಬಂಧಿಸಿದ ಮೊಕದ್ದಮೆ ನ್ಯಾಯಾಲಯದಲ್ಲಿ ನಡೆಯುತ್ತಲಿದ್ದು ಅವನೇ ಓಡಾಡಬೇಕಾಗಿತ್ತು. ಆದರೆ ತಿಂಗಳಲ್ಲಿ ಹೆಚ್ಚಿನ ದಿನಗಳನ್ನು ಮಠದಲ್ಲೇ ಕಳೆಯುತ್ತಿದ್ದ. ಹೀಗೆ ಅವನು ಆಗಾಗ ಕೆಲವು ದಿನಗಳ ಮಟ್ಟಿಗೆ ಅಲ್ಲಿ ಇಲ್ಲಿ ಹೋಗಿಬರುತ್ತಿದ್ದನಾದರೂ ಮುಂದೆ ಪರಿವ್ರಾಜಕ ಸಂನ್ಯಾಸಿಯಾಗಿ ಸುದೀರ್ಘ ಪ್ರಯಾಣ ಕೈಗೊಳ್ಳುವವರೆಗೂ, ಸುಮಾರು ಮೂರು ವರ್ಷಗಳ ಕಾಲ ನಿರಂತರವಾಗಿ ತನ್ನ ಸೋದರಶಿಷ್ಯರಿಗೆ ಮಾರ್ಗದರ್ಶನ ನೀಡುತ್ತಿದ್ದ; ಅವರಲ್ಲಿ ಮತ್ತೆಮತ್ತೆ ಸ್ಫೂರ್ತಿ ತುಂಬುತ್ತಿದ್ದ.

ಬಾರಾನಗೋರ್ ಮಠದಲ್ಲಿ, ಶ್ರೀರಾಮಕೃಷ್ಣರು ಜೀವಂತವಾಗಿದ್ದುಕೊಂಡು ಸದಾ ತಮ್ಮನ್ನು ಹರಸುತ್ತಿರುವಂತೆ ಆ ಯುವಸಂನ್ಯಾಸಿಗಳಿಗೆ ಭಾಸವಾಗುತ್ತಿತ್ತು. ಪ್ರತಿದಿನವೂ ಶ್ರೀರಾಮಕೃಷ್ಣರಿಗೆ ಭಕ್ತಿಯುತ ಪೂಜೆ ನೆರವೇರುತ್ತಿತ್ತು. ಅಂತಸ್ಸತ್ವವನ್ನು ಬಡಿದೆಬ್ಬಿಸುವಂತಹ ಭಜನೆ-ವೇದ ಘೋಷಗಳು ನಡೆಯುತ್ತಿದ್ದುವು. ಶ್ರೀರಾಮಕೃಷ್ಣರಿಗೆ ಪುಷ್ಪಾಲಂಕಾರ ಮಾಡಿ, ಧೂಪದೀಪಾದಿ ಗಳನ್ನು ಹಚ್ಚಿ, ಆರತಿ, ಬೆಳಗಿ, ಶಂಖ-ಜಾಗಟೆಗಳನ್ನು ಮೊಳಗಿಸುತ್ತ ದಿವ್ಯಪೂಜೆಯನ್ನು ಮಾಡು ತ್ತಿದ್ದರು. ಅಲ್ಲದೆ ಪ್ರತಿದಿನವೂ ಮಡಿಯಲ್ಲಿ ತಯಾರಿಸಿದ ಆಹಾರವನ್ನು ನೈವೇದ್ಯ ಮಾಡು ತ್ತಿದ್ದರು. ಸಂಜೆ ಆರತಿಯ ವೇಳೆಗೆ ಎಲ್ಲರೂ ಸೇರಿ,

\begin{myquote}
ಜಯ ಶಿವ ಓಂಕಾರ ಭಜ ಶಿವ ಓಂಕಾರ\\ಬ್ರಹ್ಮವಿಷ್ಣು ಸದಾಶಿವ ಹರಹರ ಮಹಾದೇವ
\end{myquote}

\noindent

ಎಂಬ ನಾಮಾವಳಿಯನ್ನು ಉಚ್ಚ ಕಂಠದಿಂದ ಆನಂದೋತ್ಸಾಹಭರಿತರಾಗಿ ಹಾಡುತ್ತಿದ್ದರು.

 ಈ ಯುವಸಾಧಕರೆಲ್ಲ ಅಂದು ಆಂಟ್ಪುರದಲ್ಲಿ ಧುನಿಯ ಮುಂದೆ ಕುಳಿತು ಸಂನ್ಯಾಸವ್ರತ ವನ್ನು ಕೈಗೊಂಡಿದ್ದರೂ ಅದು ಶಾಸ್ತ್ರೋಕ್ತವಾಗಿ, ನಿಯಮಗಳಿಗನುಸಾರವಾಗಿ ನಡೆದಿರಲಿಲ್ಲ. ಆದ್ದರಿಂದ ಈಗ ತಾವು ವಿಧಿವತ್ತಾಗಿ ಸಂನ್ಯಾಸ ಸ್ವೀಕರಿಸುವ ಬಗ್ಗೆ ನರೇಂದ್ರ ತನ್ನ ಸೋದರಶಿಷ್ಯರ ಅಭಿಪ್ರಾಯ ಕೇಳಿದ. ಅವರೆಲ್ಲರೂ ಅದಕ್ಕೆ ಹೃತ್ಪೂರ್ವಕವಾಗಿ ಸಮ್ಮತಿಸಿದರು. ವಿರಜಾ ಹೋಮದ ಹಾಗೂ ಸಂನ್ಯಾಸಸ್ವೀಕಾರಕ್ಕೆ ಸಂಬಂಧಪಟ್ಟ ಇತರ ಕ್ರಮ-ನಿಯಮಗಳನ್ನು ಕಾಳೀ ಪ್ರಸಾದ ತಿಳಿದುಕೊಂಡುಬಂದಿದ್ದ. ಜನವರಿ ಮೂರನೇ ವಾರದಲ್ಲಿ ಒಂದು ದಿನವನ್ನು ಗೊತ್ತು ಪಡಿಸಲಾಯಿತು. ಅಂದು ಮುಂಜಾನೆ ಶಿಷ್ಯರೆಲ್ಲ ಗಂಗೆಯಲ್ಲಿ ಮಿಂದು, ಶ್ರೀರಾಮಕೃಷ್ಣರ ಅವಶೇಷ ಹಾಗೂ ಭಾವಚಿತ್ರವನ್ನಿರಿಸಿ ಪೂಜಿಸುತ್ತಿದ್ದ ಕೋಣೆಯಲ್ಲಿ ಸೇರಿದರು. ಶಶಿ ಎಂದಿನಂತೆ ಶ್ರೀರಾಮಕೃಷ್ಣರ ಪೂಜೆ ಮಾಡಿದ. ಬಳಿಕ ವಿರಜಾ ಹೋಮ ಪ್ರಾರಂಭ. ಹೋಮಕ್ಕೆ ಸಂಬಂಧಿಸಿದ ಮಂತ್ರಗಳನ್ನು ಕಾಳೀಪ್ರಸಾದ ಉಚ್ಚರಿಸುತ್ತಿದ್ದಂತೆ ಒಬ್ಬೊಬ್ಬರಾಗಿ ಎಲ್ಲರೂ ಎಳ್ಳುತುಪ್ಪ ಸೇರಿದ ಬಿಲ್ವಪತ್ರೆಯನ್ನು ಅಗ್ನಿಕುಂಡದಲ್ಲಿ ಆಹುತಿ ಕೊಟ್ಟರು. ಮೊದಲು ನರೇಂದ್ರ, ಬಳಿಕ ರಾಖಾಲ, ನಿರಂಜನ, ಶಶಿ ಹಾಗೂ ಇತರರು ಮಂತ್ರವನ್ನುಚ್ಚರಿಸುತ್ತ ಆಹುತಿ ನೀಡಿದರು. ಹಿಂದೆ ಶ್ರೀರಾಮಕೃಷ್ಣರೇ ಇವರಿಗೆಲ್ಲ ಕಾಷಾಯವಸ್ತ್ರಗಳನ್ನು ನೀಡಿ ಸಂನ್ಯಾಸಿಗಳನ್ನಾಗಿಸಿದ್ದರು. ಈಗ ಈ ವಿಧ್ಯುಕ್ತ ಕರ್ಮಗಳಿಂದ ಅದು ಮತ್ತೊಮ್ಮೆ ದೃಢಪಟ್ಟಿತು, ಅಷ್ಟೆ. ವಿಧಿಗಳೆಲ್ಲ ಮುಗಿದ ಮೇಲೆ ನರೇಂದ್ರ ತನ್ನ ಸೋದರಶಿಷ್ಯರಿಗೆ ಹೊಸ ಸಂನ್ಯಾಸನಾಮಗಳನ್ನು ಕೊಟ್ಟ. ಅವು ಹೀಗಿವೆ:

\begin{tabular}{@{}ccc@{}}
ನರೇಂದ್ರ & – & ಸ್ವಾಮಿ ವಿವೇಕಾನಂದ \\
ರಾಖಾಲ & – & ಸ್ವಾಮಿ ಬ್ರಹ್ಮಾನಂದ \\
ಬಾಬುರಾಮ & – & ಸ್ವಾಮಿ ಪ್ರೇಮಾನಂದ \\
ಶಶಿಭೂಷಣ & – & ಸ್ವಾಮಿ ರಾಮಕೃಷ್ಣಾನಂದ \\
ಶರಚ್ಚಂದ್ರ & – & ಸ್ವಾಮಿ ಶಾರದಾನಂದ \\
ನಿರಂಜನ & – & ಸ್ವಾಮಿ ನಿರಂಜನಾನಂದ \\
ಕಾಳೀಪ್ರಸಾದ & – & ಸ್ವಾಮಿ ಅಭೇದಾನಂದ \\
ಶಾರದಾಪ್ರಸನ್ನ & – & ಸ್ವಾಮಿ ತ್ರಿಗುಣಾತೀತಾನಂದ \\
\end{tabular}

ಇನ್ನುಳಿದ ಗುರುಭಾಯಿಗಳು ಅಮೇಲೆ ಬೇರೆಬೇರೆ ಸಮಯಗಳಲ್ಲಿ ಸಂನ್ಯಾಸ ದೀಕ್ಷೆಯನ್ನು ವಿಧಿವತ್ತಾಗಿ ಪಡೆದುಕೊಂಡರು. ಅವರ ಹೊಸ ಹೆಸರುಗಳು ಇವು:

\begin{tabular}{@{}ccc@{}}
ತಾರಕನಾಥ & – & ಸ್ವಾಮಿ ಶಿವಾನಂದ \\
ಹಿರಿಯ ಗೋಪಾಲ & – & ಸ್ವಾಮಿ ಅದ್ವೈತಾನಂದ \\
ಲಾಟು & – & ಸ್ವಾಮಿ ಅದ್ಭುತಾನಂದ \\
ಯೋಗೀಂದ್ರ & – & ಸ್ವಾಮಿ ಯೋಗಾನಂದ \\
ಹರಿನಾಥ & – & ಸ್ವಾಮಿ ತುರೀಯಾನಂದ \\
ಗಂಗಾಧರ & – & ಸ್ವಾಮಿ ಅಖಂಡಾನಂದ \\
ಸುಬೋಧ & – & ಸ್ವಾಮಿ ಸುಬೋಧಾನಂದ \\
ಹರಿಪ್ರಸನ್ನ & – & ಸ್ವಾಮಿ ವಿಜ್ಞಾನಾನಂದ \\
\end{tabular}

‘ರಾಮಕೃಷ್ಣಾನಂದ’ ಎಂಬ ಹೆಸರನ್ನು ತಾನೇ ಇಟ್ಟುಕೊಳ್ಳಲು ನರೇಂದ್ರ ಬಯಸಿದ್ದ. ಆದರೆ ಶಶಿಗೆ ಶ್ರೀರಾಮಕೃಷ್ಣರಲ್ಲಿದ್ದ ಅದ್ಭುತ ಭಕ್ತಿ-ಪ್ರೇಮಗಳನ್ನು ಕಂಡು ಅವನಿಗೇ ಆ ಹೆಸರನ್ನು ಕೊಟ್ಟು ತಾನು ವಿವೇಕಾನಂದ ಎಂಬ ಹೆಸರನ್ನಿಟ್ಟುಕೊಂಡ. ಆದರೆ ಮೊದಮೊದಲು ನರೇಂದ್ರ ಆ ಹೆಸರಿನಿಂದ ಕರೆಯಲ್ಪಡುತ್ತಿದ್ದುದು ಅಪರೂಪ. ಮುಂದೆ ಪರಿವ್ರಾಜಕನಾಗಿ ಹೊರಟ ಮೇಲಂತೂ ತನ್ನ ಹೆಸರನ್ನು ಸಚ್ಚಿದಾನಂದ, ವಿವಿದಿಶಾನಂದ ಎಂಬಿತ್ಯಾದಿಯಾಗಿ ಬದಲಿಸಿಕೊಳ್ಳು ತ್ತಲೇ ಇದ್ದ. ಸೋದರಸಂನ್ಯಾಸಿಗಳು ತನ್ನ ಸುಳಿವು ಹಿಡಿದು ಅನುಸರಿಸಿ ಬರಬಾರದು ಎನ್ನುವುದು ಅದರ ಉದ್ದೇಶ. (ಆದರೆ ಮುಂದೆ ೧೮೯೩ರಲ್ಲಿ ಸ್ವಾಮೀಜಿ ಅಮೆರಿಕೆಗೆ ಹೊರಟಾಗ ತಮ್ಮ ಶಿಷ್ಯನಾದ ಖೇತ್ರಿ ಮಹಾರಾಜನ ಅಪೇಕ್ಷೆಯಂತೆ ಸ್ವಾಮಿ ವಿವೇಕಾನಂದ ಎಂಬ ಹೆಸರನ್ನೇ ಸ್ಥಿರವಾಗಿರಿಸಿಕೊಂಡರು.)

ಈಗ ಈ ಯುವಸಾಧಕರು ವಿಧಿವತ್ತಾಗಿ ಸಂನ್ಯಾಸ ಸ್ವೀಕರಿಸಿದ್ದರೂ ಅವರು ಮಠದೊಳ ಗಿದ್ದಾಗ ಮಾತ್ರ ಕಾವಿವಸ್ತ್ರವನ್ನು ಧರಿಸುತ್ತಿದ್ದರು. ಹೊರಗೆ ಹೋಗುವಾಗಲೆಲ್ಲ ತಮ್ಮ ಎಂದಿನ ಉಡುಗೆಯನ್ನೆ ತೊಡುತ್ತಿದ್ದರು. ಅಲ್ಲದೆ, ಹೊಸ ಹೆಸರುಗಳನ್ನಿಟ್ಟುಕೊಂಡರೂ ಬಹಳ ವರ್ಷ ಗಳವರೆಗೆ ತಮ್ಮನ್ನು ಹಳೆಯ ಹೆಸರುಗಳಿಂದಲೇ ಕರೆದುಕೊಳ್ಳುತ್ತಿದ್ದರು. ಇದಕ್ಕೆ ಕಾರಣವೂ ಇಲ್ಲದಿರಲಿಲ್ಲ. ಆ ದಿನಗಳಲ್ಲಿ ಬಂಗಾಳದಲ್ಲಿ ಇವರು ಅನುಸರಿಸುತ್ತಿದ್ದ ದಶನಾಮೀ ಪದ್ಧತಿಯ ಸಂನ್ಯಾಸ ಸಂಪ್ರದಾಯ ರೂಢಿಯಲ್ಲಿರಲಿಲ್ಲ. ಅಲ್ಲದೆ ಸಂನ್ಯಾಸಿಗಳ ಬಗ್ಗೆ ಸಮಾಜದಲ್ಲಿ ಗೌರವ ವಿರಲಿಲ್ಲ. ಆದ್ದರಿಂದ ಜನ ಇವರ ಬಗ್ಗೆ ವಿಚಿತ್ರವಾಗಿ ಭಾವಿಸಿ ತೊಂದರೆ ಕೊಡುವ ಸಂಭವ ವಿತ್ತು. ಕೆಲವೊಮ್ಮೆ ತೊಂದರೆ ಆಗಿಯೂ ಇತ್ತು. ಜೊತೆಗೆ, ಸ್ವಾತಂತ್ರ್ಯ ಹೋರಾಟದ ಆ ಕಾಲ ದಲ್ಲಿ ಸಂನ್ಯಾಸಿಗಳನ್ನು ಕಂಡರೆ, ಅವರು ವೇಷಪಲ್ಲಟ ಮಾಡಿಕೊಂಡ ಚಳವಳಿಗಾರರೇ ಇರ ಬೇಕೆಂದು ಶಂಕಿಸಿ, ಅವರನ್ನು ಹಿಡಿದು ಬಂಧಿಸುವ ಪ್ರಸಂಗಗಳೂ ನಡೆಯುತ್ತಿದ್ದುವು. ಸಾಲದ್ದಕ್ಕೆ ಇವರೆಲ್ಲ ಇನ್ನೂ ೨೦-೨೫ರ ನವಯುವಕರು. ಜನ ಇವರ ಸಂನ್ಯಾಸನಿಷ್ಠೆಯನ್ನು ಶಂಕಿಸಿದ್ದರೂ ಆಶ್ಚರ್ಯವಿರಲಿಲ್ಲ. ನರೇಂದ್ರನ ವಿಷಯದಲ್ಲಿ ಇನ್ನೂ ಒಂದು ತೊಂದರೆಯಿತ್ತು. ಏನೆಂದರೆ, ಅವನು ಆಗಾಗ ಮನೆಗೂ ಕೋರ್ಟುಕಛೇರಿಗಳಿಗೂ ಹೋಗಬೇಕಾಗುತ್ತಿತ್ತು. ಅಂತೂ, ಕಾಲ ಕ್ರಮೇಣ ಈ ನವಸಂನ್ಯಾಸಿಗಳು ಸದಾ ಕಾಷಾಯ ವಸ್ತ್ರಧಾರಿಗಳಾಗಿ ಓಡಾಡುವ ಸ್ಥಿತಿಗೆ ಬಂದರು.

ಶಿಷ್ಯನಾಗಿದ್ದ ತರುಣ ನರೇಂದ್ರ, ಜಗದ್ವಿಖ್ಯಾತ ವಿವೇಕಾನಂದ ಗುರುವಾಗಿ ಹೊಂದಿದ ಪರಿವರ್ತನೆ ಒಂದು ರಾತ್ರಿಯಲ್ಲಿ ನಡೆದದ್ದಲ್ಲ. ಅವನು ಉಪವಾಸ-ವನವಾಸಗಳನ್ನು ಅನುಭವಿಸ ಬೇಕಾಯಿತು; ನಾನಾ ವಿಧವಾದ ಶಾರೀರಿಕ-ಮಾನಸಿಕ ಯಾತನೆಗಳನ್ನು ಸಹಿಸಬೇಕಾಯಿತು; ಕಷ್ಟಪರಂಪರೆಯನ್ನೇ ಎದುರಿಸಬೇಕಾಯಿತು. ಆದ್ದರಿಂದ ಅದೊಂದು ಪವಾಡದೋಪಾದಿಯಲ್ಲಿ ನಡೆದುಹೋದ ಘಟನೆಯಲ್ಲ; ಕ್ರಮಾಗತವಾಗಿ ಬೆಳೆದುಬಂದ ಪ್ರಕ್ರಿಯೆ. ಸ್ವಾಮಿ ವಿವೇಕಾ ನಂದರ ಬೌದ್ಧಿಕ-ಆಧ್ಯಾತ್ಮಿಕ ಶಕ್ತಿಗಳು ಆಗಸದೆತ್ತರಕ್ಕೆ ಬೆಳೆದು ನಿಲ್ಲುವುದನ್ನು ಕಂಡಾಗ ಸಾಧಕರು, ಸತ್ಯಶೋಧಕರು ವಿಸ್ಮಯಮೂಕರಾಗುತ್ತಾರೆ. ಆದರೆ ಅವರ ಜೀವನವು ಅಷ್ಟೇ ಮಾನವೀಯವಾಗಿರುವುದನ್ನು ಕಂಡಾಗ, ಪ್ರತಿಯೊಬ್ಬ ಸಾಮಾನ್ಯನೂ ಕೂಡ ಅವರ ಬಗ್ಗೆ ಪ್ರೀತಿ-ಆದರ ತಾಳುವುದರಲ್ಲಿ ಸಂಶಯವಿಲ್ಲ. ಇನ್ನು ಮುಂದೆ ನಾವು ಕಾಣಲಿರುವುದು, ವ್ಯಕ್ತಿ ಯೊಬ್ಬನ ಅಂತಶ್ಶಕ್ತಿಯು ಮಹಾಪೂರವಾಗಿ ಹೊರಹೊಮ್ಮಿ ಅವನನ್ನು ತ್ರಿವಿಕ್ರಮಗಾತ್ರಕ್ಕೆ ಹಿಗ್ಗಿಸಿ ಮಹಾತ್ಮನನ್ನಾಗಿಸಿದ ರೋಮಾಂಚಕಾರೀ ಕಥೆಯನ್ನು.

ಬಾರಾನಗೋರ್ ಮಠನಿವಾಸಿಗಳಾದ ಯುವಸಂನ್ಯಾಸಿಗಳು ತೀವ್ರತರ ಸಾಧನಾ ಜೀವನವನ್ನು ಕೈಗೊಂಡಿದ್ದರು. ಮುಂಜಾನೆ ಮೂರು ಗಂಟೆಗೇ ಎದ್ದು ಮುಖ ತೊಳೆದು ಅಥವಾ ಸ್ನಾನ ಮಾಡಿ ಧ್ಯಾನಕ್ಕೆ ಕುಳಿತುಬಿಡುತ್ತಿದ್ದರು. ಅವರಲ್ಲಿ ಭುಗಿಲೆದ್ದಿದ್ದ ವೈರಾಗ್ಯ ಅದೆಷ್ಟು ತೀವ್ರ! ಈ ಪ್ರಪಂಚ ಇದೆಯೋ ಇಲ್ಲವೋ ಎಂಬುದರ ಪರಿವೆಯಾದರೂ ಅವರಿಗಿತ್ತೇನು! ಅವರ ಪೈಕಿ, ಸ್ವಾಮಿ ರಾಮಕೃಷ್ಣಾನಂದರಿಗೆ ಮಾತ್ರ ಗುರುದೇವನ ಕೈಂಕರ್ಯವೇ ಪರಮಸಾಧನೆ. ಅವರ ಪಾಲಿಗೆ ಶ್ರೀರಾಮಕೃಷ್ಣರ ಭಾವಚಿತ್ರ ಒಂದು ನಿರ್ಜೀವ ವಸ್ತುವಲ್ಲ. ಅಲ್ಲಿ ಸಾಕ್ಷಾತ್ ಶ್ರೀರಾಮಕೃಷ್ಣರು ಸಶರೀರಿಯಾಗಿ ಇರುವರೆಂದೇ ಅವರ ಅವಿಚಲ ನಂಬಿಕೆ. ಗುರುದೇವನ ಪೂಜೆ-ಕೈಂಕರ್ಯ ಗಳೊಂದಿಗೆ ತಮ್ಮ ಸೋದರಸಂನ್ಯಾಸಿಗಳ ಯೋಗಕ್ಷೇಮದ ಭಾರವನ್ನೂ ಅವರೇ ಹೊತ್ತಿದ್ದರು. ಮನೆಗೆ ತಾಯಿ ಹೇಗೋ ಹಾಗೆ ಮಠಕ್ಕೆ ಸ್ವಾಮಿ ರಾಮಕೃಷ್ಣಾನಂದರು. ಶ್ರೀರಾಮಕೃಷ್ಣರ ನೈವೇದ್ಯಕ್ಕೂ, ಸೋದರ ಸಂನ್ಯಾಸಿಗಳ ಊಟಕ್ಕೂ, ಆಹಾರ ಪದಾರ್ಥಗಳನ್ನು– ಸಾಮಾನ್ಯವಾಗಿ ಭಿಕ್ಷೆ ಬೇಡಿ–ತರುತ್ತಿದ್ದವರು ಇವರೇ. ಇತರರೆಲ್ಲ ಬೆಳಗ್ಗೆ ಧ್ಯಾನಕ್ಕೆ ಕುಳಿತರೆ ಕೆಲವೊಮ್ಮೆ ಏಳುವಾಗ ಸಾಯಂಕಾಲ ನಾಲ್ಕು ಅಥವಾ ಐದು ಗಂಟೆಯಾಗಿರುತ್ತಿತ್ತು. ರಾಮಕೃಷ್ಣಾನಂದರು ಅವರಿಗಾಗಿ ಅಡಿಗೆ ಮಾಡಿ ಕಾದು ಕುಳಿತಿರುತ್ತಿದ್ದರು. ಕಾದೂ ಕಾದೂ ಕೊನೆಗೂ ಎದ್ದು ಬಾರ ದಿದ್ದರೆ, ತಾವೇ ಅವರನ್ನು ಬಲಾತ್ಕಾರದಿಂದ ಎಬ್ಬಿಸಿ ಒತ್ತಾಯ ಮಾಡಿ ಎರಡು ತುತ್ತು ಊಟ ಮಾಡಿಸುತ್ತಿದ್ದರು. ಶ್ರೀರಾಮಕೃಷ್ಣರ ಮೇಲೂ, ತಮ್ಮ ಸೋದರಸಂನ್ಯಾಸಿಗಳ ಮೇಲೂ ಅವರಿ ಗಿದ್ದ ಭಕ್ತಿ-ವಿಶ್ವಾಸಗಳು ಅದ್ವಿತೀಯ.

ಬಾರಾನಗೋರ್ ಮಠದಲ್ಲಿನ ಜೀವನ ನಿಜಕ್ಕೂ ತೀರಾ ಬಡತನದ ಜೀವನ. ಆದರೆ ಬಡತನಕ್ಕೆ ಹೆದರುವವರು ಯಾರು! ಶ್ರೀರಾಮಕೃಷ್ಣರು ತೋರಿಸಿಕೊಟ್ಟ ಆಧ್ಯಾತ್ಮಿಕ ಸಾಧನಾ ಮಾರ್ಗದಲ್ಲಿ ಮುನ್ನಡೆಯುವುದೊಂದೇ ಆಲೋಚನೆ ಅವರಿಗೆ. ಆದ್ದರಿಂದ ಎಷ್ಟೋ ದಿನ ನಿದ್ರೆಯನ್ನೂ ಮರೆತು ರಾತ್ರಿಯೆಲ್ಲ ಧ್ಯಾನಮಗ್ನರಾಗಿ ಕುಳಿತು ಬಿಡುತ್ತಿದ್ದರು. ಧ್ಯಾನದಲ್ಲಿ ಎಂತಹ ಆನಂದವೆಂದರೆ ಎದ್ದು ಭಿಕ್ಷೆಗೆ ಹೋಗಲೂ ಮನಸ್ಸಿಲ್ಲ! ಭಗವಂತನ ಇಚ್ಛೆಯಿಂದ ತಾನಾಗಿಯೇ ಏನು ಬರುತ್ತದೆಯೋ ಅದರಲ್ಲೇ ಜೀವನ. ಮಠಕ್ಕೆ ಸಂಬಂಧಪಟ್ಟ ಪ್ರತಿಯೊಂದು ಸಣ್ಣಪುಟ್ಟ ಕೆಲಸವನ್ನು ಮಾಡಲೂ ‘ನಾಮುಂದು ತಾಮುಂದು’ ಎಂಬ ಸ್ಪರ್ಧೆ. ಎಷ್ಟೋ ದಿನ ಅವರಿಗೆ ಊಟಕ್ಕೆ ಏನೂ ಇರುತ್ತಿರಲಿಲ್ಲ. ಆದರೆ ತಮಗೊಂದು ಶರೀರವಿದೆ ಎಂಬುದನ್ನೂ ಮರೆತು ಧ್ಯಾನ ಮಾಡುತ್ತಿದ್ದರು, ಭಜನೆ ಮಾಡುತ್ತಿದ್ದರು, ಶಾಸ್ತ್ರವಿಚಾರ ಮಾಡುತ್ತಿದ್ದರು. ಅವರಲ್ಲಿದ್ದ ಉಡುಗೆ ಯೆಂದರೆ ಒಂದು ಕೌಪೀನ; ಜೊತೆಗೆ ಒಂದೆರಡು ತುಂಡು ಕಾವಿ ಬಟ್ಟೆಗಳು, ಮತ್ತು ಅವರಷ್ಟೂ ಮಂದಿಗೆ ಸೇರಿದಂತೆ ಒಂದು ಪಂಚೆ, ಒಂದು ಶಲ್ಯ. ಈ ಪಂಚೆ-ಶಲ್ಯಗಳನ್ನು ತಂತಿಯ ಮೇಲೆ ತೂಗುಹಾಕಿರುತ್ತಿದ್ದರು. ಪೇಟೆಗೋ ನಗರಕ್ಕೋ ಹೋಗಬೇಕಾದವರು ಸಭ್ಯರಂತೆ ಕಾಣಿಸಿಕೊಳ್ಳ ಬೇಕಾದ್ದರಿಂದ ಈ ವಸ್ತ್ರಗಳನ್ನು ಧರಿಸಿಕೊಂಡು ಹೋಗುತ್ತಿದ್ದರು. ನೆಲದ ಮೇಲೊಂದು ಚಾಪೆ ಹಾಸಿಕೊಂಡರೆ ಅದೇ ಅವರ ಹಾಸಿಗೆ. ಜಪಮಾಲೆ ಹಾಗೂ ಒಂದು ತಂಬೂರಿ–ಇವೇ ಅವರ ಆಸ್ತಿ. ಆ ಶಿಥಿಲ ಗೋಡೆಗಳ ಮೇಲೆಯೇ ದೇವದೇವಿಯರ ಹಾಗೂ ಸಾಧುಸಂತರ ಕೆಲವು ಪಟಗಳನ್ನು ತೂಗುಹಾಕಿದ್ದರು. ಎಲ್ಲರಿಗೂ ಸೇರಿದಂತೆ ಸುಮಾರು ಒಂದು ನೂರು ಪುಸ್ತಕಗಳ ‘ಗ್ರಂಥಾಲಯ’ವೂ ಇತ್ತು. ಇಷ್ಟೇ ಆ ಮಠದ ಸರ್ವಸ್ವ. ಗೃಹೀಭಕ್ತ ಸುರೇಂದ್ರನಾಥ ಇವರಿಗೆ ತಿಂಗಳಿಗೆ ನೂರು ರೂಪಾಯಿ ಕೊಡಲಾರಂಭಿಸಿದ. ಅದರಿಂದಲೂ ಅವನಿಗೆ ತೃಪ್ತಿಯಾಗಲಿಲ್ಲ. ಅಲ್ಲಿನ ಪರಿಸ್ಥಿತಿಯನ್ನು ಇತರ ಭಕ್ತರಿಗೂ ಗುಟ್ಟಾಗಿ ತಿಳಿಸಿ ಒಂದಿಷ್ಟು ವಂತಿಗೆ ಎತ್ತಿ, ಆಗಾಗ ಸ್ವಲ್ಪ ಹೆಚ್ಚಿಗೆ ಹಣವನ್ನೋ ಬೇಳೆಕಾಳುಗಳನ್ನೋ ಮಠಕ್ಕೆ ಕಳಿಸಿಕೊಡುತ್ತಿದ್ದ. ಇದರಿಂದಾಗಿ ಮಠದ ಕಡುಬಡತನ ಎಷ್ಟೋ ಕಡಿಮೆಯಾಯಿತು.

ಈ ಮಠಕ್ಕೆ ಗೃಹೀಭಕ್ತರು ಕೆಲವರು ಆಗಾಗ ಬಂದುಹೋಗುತ್ತಿದ್ದರು. ಇವರಲ್ಲದೆ ಕೆಲವರು ಅನಪೇಕ್ಷಿತ ಅತಿಥಿಗಳೂ ಒಮ್ಮೊಮ್ಮೆ ಬರುತ್ತಿದ್ದರು. ಅವರು ಯಾರೆಂದರೆ ಆ ಯುವಸಂನ್ಯಾಸಿ ಗಳ ತಾಯ್ತಂದೆಯರು ಮತ್ತು ‘ಹಿತೈಷಿ’ಗಳು. ಈ ಯುವಕರನ್ನು ಹೇಗಾದರೂ ಮಾಡಿ ವಾಪಸು ಮನೆಗೆ ಕರೆದುಕೊಂಡು ಹೋಗುವ ಆಸೆಯಿಂದ ಅವರು ಬರುತ್ತಿದ್ದರು. ತಮ್ಮ ಮಕ್ಕಳನ್ನು ಮನೆಗೆ ಹಿಂದಿರುಗುವಂತೆ ಕಣ್ಣೀರ್ಗರೆದು ಕೇಳಿಕೊಳ್ಳುತ್ತಿದ್ದರು. ಕಣ್ಣೀರಿಗೂ ಕರಗದಿದ್ದರೆ ಗದರಿಸಿ ಹೆದರಿಸಿ ನೋಡುತ್ತಿದ್ದರು. ಆದರೆ ಯಾರೇನು ಮಾಡಿದರೂ ಈ ಯುವಸಂನ್ಯಾಸಿಗಳನ್ನು ಕದಲಿಸಲು ಸಾಧ್ಯವಾಗುತ್ತಿರಲಿಲ್ಲ. ಇವರದ್ದು ಅಭಾವ ವೈರಾಗ್ಯವಲ್ಲ; ಪರಿಪೂರ್ಣ ವೈರಾಗ್ಯ. ಆದ್ದರಿಂದ ತಾಯ್ತಂದೆಯರ ಆಜ್ಞಾರೂಪದ ಮಾತುಗಳನ್ನೂ ಲೆಕ್ಕಿಸಲಿಲ್ಲ. ಆದರೆ ಇವರು ತಾಯ್ತಂದೆಯರೊಡನೆ ವಾದಮಾಡುವ ಬದಲಾಗಿ ಮೌನವಾಗಿಯೇ ಇದ್ದುಬಿಡುತ್ತಿದ್ದರು. ಇಲ್ಲವೆ ಅವರಿಂದ ತಪ್ಪಿಸಿಕೊಂಡು ತಿರುಗುತ್ತಿದ್ದರು. ತಮ್ಮ ಚತುರೋಪಾಯಗಳೂ ವಿಫಲವಾದ ದ್ದನ್ನು ಕಂಡ ತಾಯ್ತಂದೆಯರ ಸಿಟ್ಟು ಈಗ ನರೇಂದ್ರನ ಮೇಲೆ ತಿರುಗಿತು. “ಅವನೇ ಈ ಎಲ್ಲ ಅನಿಷ್ಟಗಳಿಗೂ ಮೂಲಕಾರಣ. ರಾಮಕೃಷ್ಣರು ಹೊರಟುಹೋದ ಮೇಲೆ ನಮ್ಮ ಹುಡುಗರು ಮನೆಗೆ ಬಂದು ಓದುಬರಹದ ಕಡೆಗೆ ಗಮನ ಹರಿಸಿದ್ದರು. ಆದರೆ ಆ ನರೇಂದ್ರ ಬಂದು ಎಲ್ಲ ಬುಡಮೇಲು ಮಾಡಿಬಿಟ್ಟ” ಎಂದು ಬೈದುಕೊಳ್ಳುತ್ತ ಹಿಂದಿರುಗಿದರು.

ಇನ್ನು ಕೆಲವರಿಗೆ ಇಷ್ಟು ಸಣ್ಣ ವಯಸ್ಸಿನ ಯುವಕರು ಸಂನ್ಯಾಸಿಗಳಾದರಲ್ಲ ಎಂದು ಆಶ್ಚರ್ಯ. ಕೆಲವು ಗೃಹಸ್ಥರು ಒಂದು ಬಗೆಯ ಕುಹಕದಿಂದ ಅವರನ್ನು ಕೇಳುವುದಿತ್ತು, “ಏನು, ನೀವೆಲ್ಲ ಸಂನ್ಯಾಸ ತೆಗೆದುಕೊಂಡಿರಲ್ಲ, ಏನು ಸಾಧಿಸಿದಿರಿ?” ಆಗ ನರೇಂದ್ರ ಖಾರವಾಗಿಯೇ ಹೇಳುತ್ತಿದ್ದ, “ಒಂದು ವೇಳೆ ಸಾಕ್ಷಾತ್ಕಾರ ಆಗದೆಯಿದ್ದರೂ ಏನೀಗ? ಇನ್ನೂ ಸಾಕ್ಷಾತ್ಕಾರವಾಗ ಲಿಲ್ಲ ಎಂದಮಾತ್ರಕ್ಕೆ ನಾವು ಒಮ್ಮೆ ಉಗುಳಿಬಿಟ್ಟ ಇಂದ್ರಿಯಜೀವನಕ್ಕೇ ಹಿಂದಿರುಗಿ ಹೋಗೋಣವೆ? ನಮ್ಮ ಉನ್ನತ ಆದರ್ಶದಿಂದ ಕೆಳಗೆ ಜಾರೋಣವೆ?”

ಬಾರಾನಗೋರ್ ಮಠವಾಸಿಗಳಾದ ಯುವಸಾಧಕರು ಭಗವಂತನ ಸಾಕ್ಷಾತ್ಕಾರಕ್ಕಾಗಿ ತೀವ್ರ ವಾಗಿ ಹಂಬಲಿಸುತ್ತಿದ್ದರು. ನರೇಂದ್ರನಂತೂ ಅತ್ಯುನ್ನತ ಸತ್ಯಸಾಕ್ಷಾತ್ಕಾರಕ್ಕಾಗಿ, ನಿರ್ವಿಕಲ್ಪ ಸಮಾಧಿಸುಖಕ್ಕಾಗಿ ವ್ಯಾಕುಲಿತನಾಗಿದ್ದ. ಕೆಲವೊಮ್ಮೆ ಅವನು ಉದ್ಗರಿಸುತ್ತಿದ್ದ, “ನನಗಾದ ಆ ಅನುಭವಗಳಿಂದೆಲ್ಲ ಏನು ಪ್ರಯೋಜನ? ಸುವರ್ಣಾಕ್ಷರಗಳಲ್ಲಿ ಪ್ರಕಾಶಮಾನವಾಗಿ ಬೆಳಗು ತ್ತಿರುವ ಮಂತ್ರವನ್ನು ಕಂಡಿದ್ದೇನೆ. ಕಾಳಿ ಹಾಗೂ ಇತರ ದೇವದೇವಿಯರನ್ನು ದರ್ಶಿಸಿದ್ದೇನೆ. ಆದರೆ ಆಹಾ, ಈಗ ಆ ಸಮಾಧಿಯ ಶಾಂತಿಯೆಲ್ಲಿ! ಸಮಾಧಿಯ ಆನಂದವೆಲ್ಲಿ! ನನಗೆ ಯಾವುದೂ ಬೇಡವಾಗಿದೆ. ಭಕ್ತರ ಜೊತೆಯಲ್ಲಿ ಮಾತನಾಡುವುದೂ ಬೇಸರವಾಗಿದೆ. ದೇವರೇ ಇಲ್ಲವೋ ಏನೋ ಎನ್ನಿಸುತ್ತದೆ! ನನ್ನಿಂದ ಸತ್ಯಸಾಕ್ಷಾತ್ಕಾರ ಮಾಡಿಕೊಳ್ಳಲು ಸಾಧ್ಯವಾಗದೆ ಹೋದರೆ, ಉಪವಾಸ ಮಾಡಿ ಪ್ರಾಣತ್ಯಾಗ ಮಾಡಿಬಿಡುತ್ತೇನೆ.” ಸಾಮಾನ್ಯ ಸಾಧಕರಾದರೆ ಒಂದು ಸಣ್ಣ ಅನುಭವವಾದರೂ, ಸಾಕ್ಷಾತ್ಕಾರವೇ ಆಗಿಬಿಟ್ಟಿತು–ಇನ್ನೇನೂ ಆಗಬೇಕಾದದ್ದಿಲ್ಲ ಎನ್ನು ವಷ್ಟರ ಮಟ್ಟಿಗೆ ತೃಪ್ತರಾಗಿಬಿಡುತ್ತಾರೆ. ಆದರೆ ನರೇಂದ್ರನಿಗೆ ಅಂತಹ ಅಪೂರ್ವ ಅನುಭವ ಗಳೆಲ್ಲ ಆಗಿದ್ದರೂ ಇನ್ನೂ ಸಂತೋಷವಿಲ್ಲ, ತೃಪ್ತಿಯಿಲ್ಲ. ಏಕಿರಬಹುದು? ಅವನು ಅದಾಗಲೇ ಶ್ರೀರಾಮಕೃಷ್ಣರ ಕೃಪೆಯಿಂದ ಒಮ್ಮೆ ನಿರ್ವಿಕಲ್ಪ ಸಮಾಧಿಯ ಅಮೃತವನ್ನುಂಡವನ್ನಲ್ಲವೆ! ಆ ಅನುಭವದ ಮುಂದೆ ಉಳಿದೆಲ್ಲ ಅನುಭವಗಳೂ ನೀರಸವಾಗಿ ಕಾಣುತ್ತಿವೆ.

ಬಾರಾನಗೋರ್ ಮಠದಲ್ಲಿನ ಜೀವನವೆಂದರೆ ಅದೊಂದು ಅಪೂರ್ವ ಸಾಧಾನಮಯ ಜೀವನ. ಎಷ್ಟೋ ದಿನ ಮುಂಜಾನೆ ನಾಮಸಂಕೀರ್ತನೆ ಪ್ರಾರಂಭವಾದರೆ, ಸಂಜೆಯವರೆಗೂ ಮುಂದುವರಿಯುತ್ತಿತ್ತು. ಊಟ ವಿಶ್ರಾಂತಿಗಳ ನೆನಪೇ ಆಗುತ್ತಿರಲಿಲ್ಲ. ಭಗವದ್ದರ್ಶನದ ಹಂಬಲ ಅವರಲ್ಲಿ ಎಷ್ಟೊಂದು ತೀವ್ರವಾಗಿತ್ತೆಂದರೆ, ಧ್ಯಾನ ಮಾಡುತ್ತ ಕುಳಿತವರು ಹಾಗೇ ಪ್ರಾಯೋಪವೇಶ\footnote{*ಪ್ರಾಣತ್ಯಾಗ ಮಾಡಲು ನಿರ್ಧರಿಸಿ ಅನ್ನ-ನೀರು ಬಿಟ್ಟು ಒಂದೆಡೆ ಮಲಗಿಬಿಡುವುದು.} ಮಾಡಿದ್ದರೂ ಆಶ್ಚರ್ಯವಿಲ್ಲ! ಅನೇಕ ವರ್ಷಗಳ ಬಳಿಕ ಸ್ವಾಮಿ ವಿವೇಕಾ ನಂದರು ತಮ್ಮ ಶಿಷ್ಯನೊಬ್ಬನ ಮುಂದೆ ಹೇಳುತ್ತಾರೆ: “ ಆ ದಿನಗಳಲ್ಲಿ ಎಷ್ಟೋ ಬಾರಿ ನಮಗೆ ಊಟಕ್ಕೆ ಏನೇನೂ ಇರುತ್ತಿರಲಿಲ್ಲ. ಕೆಲವು ಸಲ ಅಕ್ಕಿ ಇರುತ್ತಿತ್ತು. ಆದರೆ ಉಪ್ಪೇ ಇರುತ್ತಿರಲಿಲ್ಲ! ಅನ್ನ, ಉಪ್ಪು ಮತ್ತು ಬೇಯಿಸಿದ ಸೊಪ್ಪು ತಿಂದುಕೊಂಡು ತಿಂಗಳುಗಟ್ಟಲೆ ಜೀವಿಸಿದೆವು. ಆದರೆ ಏನಿತ್ತೋ ಏನಿಲ್ಲವೋ, ನಮಗ್ಯಾರಿಗೂ ಅದರತ್ತ ಲಕ್ಷ್ಯವೇ ಇರಲಿಲ್ಲ. ನಾವು ಸಂನ್ಯಾಸಿ ಗಳು. ನಾಳಿನ ಚಿಂತೆ ನಮಗೆಲ್ಲಿ? ನಮ್ಮನಮ್ಮ ಆಧ್ಯಾತ್ಮಿಕ ಸಾಧನೆಯ ಪ್ರವಾಹದಲ್ಲಿ ನಾವು ಕೊಚ್ಚಿಕೊಂಡು ಹೋಗುತ್ತಿದ್ದೆವು. ಆಹ್! ಅದೆಂತಹ ದಿನಗಳು! ನಮ್ಮ ತಪಶ್ಚರ್ಯೆಯ ತೀವ್ರತೆಯನ್ನು ಕಂಡು ಮನುಷ್ಯರಿರಲಿ ರಾಕ್ಷಸರೂ ಹೆದರಿ ಓಡಿಹೋಗಬೇಕು! ಒಂದು ವಿಷಯ ವಂತೂ ನಿಜ. ಪರಿಸ್ಥಿತಿ ನಮಗೆ ಎಷ್ಟೆಷ್ಟು ವಿರುದ್ಧವಾಗುತ್ತದೋ ಅಷ್ಟಷ್ಟೂ ನಮ್ಮ ಅಂತಸ್ಸತ್ವ ವ್ಯಕ್ತವಾಗುತ್ತ ಬರುತ್ತದೆ.” ವಿವೇಕಾನಂದರು ಸಾಮಾನ್ಯವಾಗಿ ಇಂತಹ ವಿಷಯಗಳ ಬಗ್ಗೆ ಯಾರೊಂದಿಗೂ ಮಾತನಾಡುತ್ತಿರಲಿಲ್ಲ. ಆದರೆ ತಮ್ಮ ಶಿಷ್ಯರಲ್ಲಿ ವೈರಾಗ್ಯ, ತಿತಿಕ್ಷೆಯ ಗುಣ ಗಳನ್ನು ಉದ್ದೀಪನಗೊಳಿಸುವ ಉದ್ದೇಶದಿಂದ ಅವರ ಮುಂದೆ ಬಿಚ್ಚು ಮನಸ್ಸಿನಿಂದ ಇವು ಗಳನ್ನೆಲ್ಲ ಹೇಳುವುದಿತ್ತು.

ಈ ಯುವಸಂನ್ಯಾಸಿಗಳು ಪ್ರತಿದಿನದ ಸಾಧನೆ-ಭಜನೆ-ಧ್ಯಾನ-ಅಧ್ಯಯನಗಳಲ್ಲದೆ ಶ್ರೀಕೃಷ್ಣಾ ಷ್ಟಮಿ, ಶಿವರಾತ್ರಿಯೇ ಮೊದಲಾದ ಹಬ್ಬಹರಿದಿನಗಳನ್ನು ಸಂಭ್ರಮದಿಂದ ಆಚರಿಸುತ್ತಿದ್ದರು. ಅವರು ಮೊದಲ ಸಲ ಆಚರಿಸಿದ ಶಿವರಾತ್ರಿ ಉತ್ಸವದ ವಿವರಗಳನ್ನು ಮಹೇಂದ್ರನಾಥ ಬರೆದಿಟ್ಟುಕೊಂಡಿದ್ದು, ಅದನ್ನು ‘ವಚನವೇದ’ದಲ್ಲಿ ಕಾಣಬಹುದು.

ಅಂದು ೧೮೮೭ ಫೆಬ್ರವರಿ ೨೧ ನೇ ತಾರೀಖು, ಸೋಮವಾರ. ನರೇಂದ್ರ, ರಾಖಾಲ, ನಿರಂಜನ, ಶರಚ್ಚಂದ್ರ, ಶಶಿ ಕಾಳೀಪ್ರಸಾದ, ಬಾಬುರಾಮ, ತಾರಕನಾಥ, ಹಿರಿಯ ಗೋಪಾಲ, ಶಾರದಾ ಪ್ರಸನ್ನ, ಹರೀಶ ಮತ್ತು ಮಹೇಂದ್ರನಾಥ–ಇಷ್ಟು ಜನರೂ ಅಂದು ಸೇರಿದ್ದರು. ಮುಂಜಾನೆ ಶಿವನ ಸ್ತೋತ್ರಗಳ ಪಠನದೊಂದಿಗೆ ಉತ್ಸವ ಆರಂಭವಾಯಿತು. ನರೇಂದ್ರನೇ ಈಚೆಗೆ ರಚಿಸಿದ್ದ, ‘ತಾಥೈಯ ತಾಥೈಯ ನಾಚೇ ಭೋಲಾ’ (ಭೋಲಾ ಎಂದರೆ ಶಿವ) ಎಂಬ ಹಾಡನ್ನು ಹಾಡುತ್ತ ರಾಖಾಲ ಮತ್ತು ತಾರಕನಾಥ ನರ್ತಿಸಿದರು. ಶಶಿ ಎಂದಿನಂತೆ ಶ್ರೀರಾಮ ಕೃಷ್ಣರ ಪೂಜೆ ಮಾಡಿದ. ಎಲ್ಲರೂ ದಿನವಿಡೀ ಉಪವಾಸವಿದ್ದು ಶಿವನ ಪೂಜೆ-ಧ್ಯಾನಗಳಲ್ಲಿ ನಿರತರಾದರು.

ರಾತ್ರಿಯ ಪೂಜೆಗಾಗಿ ಸಂಜೆಯಿಂದಲೇ ಸಿದ್ಧತೆಗಳು ನಡೆದುವು. ಬಿಲ್ವಪತ್ರೆಗಳನ್ನು ಸಂಗ್ರಹಿಸಿ ಹೋಮಕ್ಕೆ ಬಿಲ್ವ-ಸಮಿತ್ತುಗಳನ್ನು ಸಿದ್ಧಪಡಿಸಿಕೊಂಡರು. ಎಲ್ಲ ದೇವದೇವಿಯರ ಮುಂದೆ ಗಂಧದ ಕಡ್ಡಿಗಳನ್ನು ಹಚ್ಚಲಾಯಿತು. ಮಠದ ಕಾಂಪೌಂಡಿನಲ್ಲೇ ಇರುವ ಒಂದು ಬಿಲ್ವವೃಕ್ಷದ ಕೆಳಗೆ ಶಿವಪೂಜೆಗೆ ಸಿದ್ಧತೆ ನಡೆದಿದೆ. ಯಾಮಕ್ಕೊಂದರಂತೆ ನಾಲ್ಕು ಯಾಮಗಳಲ್ಲಿ (ಯಾಮ = ಜಾವ; ಮೂರು ತಾಸುಗಳ ಅವಧಿ) ನಾಲ್ಕು ಪೂಜೆ ಆಗಬೇಕಾಗಿದೆ. ಸುಮಾರು ಒಂಬತ್ತು ಗಂಟೆಯ ವೇಳೆಗೆ ಎಲ್ಲರೂ ಮರದ ಬುಡದಲ್ಲಿ ಸೇರಿದರು. ಒಬ್ಬರು ಪೂಜೆ ಪ್ರಾರಂಭಿಸಿದರು. ನರೇಂದ್ರ ಹಾಗೂ ಇತರರು ಮೈಗೆಲ್ಲ ಭಸ್ಮ ಲೇಪಿಸಿಕೊಂಡು ಆಗಾಗ ಬಿಲ್ವವೃಕ್ಷದ ಸುತ್ತ ಹಾಡು ಹೇಳಿಕೊಂಡು ನರ್ತಿಸಿದರು. ಕೆಲವೊಮ್ಮೆ “ಶಿವಗುರು ಶಿವಗುರು ಹರಹರ ವ್ಯೋಂ ವ್ಯೋಂ” ಎಂದು ತಾರಸ್ವರದಲ್ಲಿ ಚಪ್ಪಾಳೆ ತಟ್ಟುತ್ತ ಹಾಡಿದರು. ಅದು ಅಮಾವಾಸ್ಯೆಯ ಮಧ್ಯರಾತ್ರಿ; ಎಲ್ಲೆಲ್ಲೂ ಕಾಳಗತ್ತಲು, ಎಲ್ಲೆಲ್ಲೂ ನೀರವತೆ, ಕಾಷಾಯವಸ್ತ್ರಧಾರಿಗಳಾದ ಈ ಸಂನ್ಯಾಸಿಗಳು ತಮ್ಮ ಯೌವನಭರಿತ ಕಂಠಗಳಿಂದ “ಶಿವಗುರು ಶಿವಗುರು ಹರಹರ ವ್ಯೋಂ ವ್ಯೋಂ” ಎಂದು ಘೋಷಿಸುತ್ತಿದ್ದರೆ, ಸಿಡಿಲಿನಂತಹ ಆ ನಿನಾದ ಸುತ್ತಲೂ ಹರಡಿ ಅನಂತ ಸಚ್ಚಿದಾನಂದಸಾಗರ ದಲ್ಲಿ ಲೀನವಾಗುತ್ತಿತ್ತು! ಪೂಜಾದಿಗಳು ಮುಗಿದು ಹೋಮದ ಕೊನೆಯಲ್ಲಿ ಪೂರ್ಣಾಹುತಿಯ ವೇಳೆಗೆ, ಸಮಸ್ತ ದೇವದೇವಿಯರ ಹೆಸರಿನಲ್ಲಿ ಹಾಗೂ ವಿಶ್ವದಾದ್ಯಂತ ಜನ್ಮವೆತ್ತಿ ಬಂದ ಸಮಸ್ತ ಅವತಾರಪುರುಷರ ಹೆಸರಿನಲ್ಲಿ ಹೋಮಾಗ್ನಿಗೆ ಆಹುತಿಯನ್ನು ಅರ್ಪಿಸಲಾಯಿತು. ಈಗ ಅರುಣೋದಯಕಾಲ ಸಮೀಪಿಸಿತು. ಪೂರ್ವದಿಗಂತವು ರಕ್ತವರ್ಣದಿಂದ ರಂಜಿತವಾಯಿತು. ಆ ಪವಿತ್ರ ಪ್ರಾತಃಕಾಲದ ಮುಂಬೆಳಕಿನಲ್ಲಿ ಯುವಸಂನ್ಯಾಸಿಗಳು ಗಂಗಾಸ್ನಾನ ಮಾಡಿ ಪೂಜಾಗೃಹಕ್ಕೆ ಬಂದು ಶ್ರೀರಾಮಕೃಷ್ಣರಿಗೆ ಸಾಷ್ಟಾಂಗ ಪ್ರಣಾಮ ಮಾಡಿದರು. ಬಳಿಕ ಎಲ್ಲರೂ ಬಂದು ಹಜಾರದಲ್ಲಿ ಸೇರಿದರು. ಅಲ್ಲಿ ತಾನು ಕಂಡ ದೃಶ್ಯವನ್ನು ಮಹೇಂದ್ರನಾಥ ವರ್ಣಿಸುತ್ತಾನೆ:

“ನರೇಂದ್ರನು ನೂತನ ಕಾಷಾಯವಸ್ತ್ರವನ್ನು ಧರಿಸಿದ್ದ. ಆ ವಸ್ತ್ರದ ಉಜ್ವಲ ಕೇಸರಿ ವರ್ಣವು ಅವನ ತೇಜೋಮಯವಾದ ಮುಖ ಹಾಗೂ ಮೈಬಣ್ಣದೊಂದಿಗೆ ಬಹಳ ಚೆನ್ನಾಗಿ ಮಿಳಿತವಾಗಿತ್ತು. ಅವನ ಶರೀರದ ಪ್ರತಿಯೊಂದು ರೋಮಕೂಪದಿಂದಲೂ ದಿವ್ಯ ಪ್ರಭೆ ಹೊರಸೂಸುತ್ತಿತ್ತು. ಅವನ ಮುಕಮಂಡಲ ಒಂದು ಪ್ರಖರ ತೇಜಸ್ಸಿನಿಂದ ಶೋಭಿಸುತ್ತಿತ್ತು. ಜೊತೆಗೆ ಒಂದು ಅಪೂರ್ವ ಕಾರುಣ್ಯವೂ ಅಲ್ಲಿ ಕಾಣುತ್ತಿತ್ತು. ಅವನನ್ನು ನೋಡಿದವರಿಗೆಲ್ಲ ಅನ್ನಿಸುತ್ತಿತ್ತು–ಶ್ರೀರಾಮಕೃಷ್ಣರ ಕಾರ್ಯವನ್ನು ಈಡೇರಿಸಲು ಸಚ್ಚಿದಾನಂದಸಾಗರದಿಂದ ಎದ್ದುಬಂದ ಒಂದು ಅಲೆಯೇ ಇವನು ಎಂದು. ಅವನಿಗೆ ಈಗಷ್ಟೇ ಇಪ್ಪತ್ನಾಲ್ಕು ವರ್ಷ; ಜೈತನ್ಯ ಮಹಾಪ್ರಭವೂ ಸಂಸಾರ ತ್ಯಾಗ ಮಾಡಿದ್ದು ಈ ವಯಸ್ಸಿನಲ್ಲೇ... ”

ಶಿವರಾತ್ರಿಯ ಪೂಜೆ, ಉಪವಾಸ ವ್ರತ, ಗಂಗಾಸ್ನಾನ–ಇವುಗಳನ್ನೆಲ ಮುಗಿಸಿ ಆ ಯುವಸಂನ್ಯಾಸಿಗಳು ಹಜಾರದಲ್ಲಿ ಕುಳಿತಿದ್ದಾರೆ. ಆ ಸಮಯಕ್ಕೆ ಸರಿಯಾಗಿ ಬಲರಾಮ ಬೋಸ್ ಅವರಿಗಾಗಿ ಹಣ್ಣುಗಳನ್ನು ಸಿಹಿತಿನಿಸನ್ನೂ ಕಲಿಸಿಕೊಟ್ಟಿದ್ದಾನೆ. ನರೇಂದ್ರಾದಿಗಳು ಅದನ್ನು ಆದರದಿಂದ ಸ್ವೀಕರಿಸುತ್ತ ಉದ್ಗರಿಸಿದರು, “ಧನ್ಯ ಬಲರಾಮ, ನಿಜಕ್ಕೂ ಧನ್ಯ!”

\delimiter

ಈ ಯುವಕರೆಲ್ಲ ಅನುಕೂಲಸ್ಥ ಕುಟುಂಬಗಳಿಂದ ಬಂದವರು; ತಮ್ಮ ತಾಯ್ತಂದೆಯರೊಡನೆ ಸುಖದಲ್ಲಿ ಬೆಳೆದವರು. ಈಗ ಬಡತನ ತಾಂಡವವಾಡುತ್ತಿರುವ ಈ ಪಾಳುಬಿದ್ದ ಕಟ್ಟಡದಲ್ಲಿ ಎಲ್ಲ ಬಗೆಯ ಅನನುಕೂಲತೆಗಳ ನಡುವೆಯೂ ನೆಮ್ಮದಿಯಿಂದಿದ್ದಾರೆ! ಅವರ ಒಂದೇ ಒಂದು ಚಿಂತೆಯೆಂದರೆ, ಇನ್ನೂ ಭಗವಂತನ ದರ್ಶನವಾಗಲಿಲ್ಲವಲ್ಲ ಎಂದು. ಅದಕ್ಕಾಗಿ ತೀವ್ರತರ ಸಾಧನೆಗಳನ್ನು ಮಾಡುತ್ತಿದ್ದಾರೆ. ಆದರೆ ತಮ್ಮ ಸಾಧನೆಯಲ್ಲಿ ಅವರಿಗೆ ತೃಪ್ತಿಯಿಲ್ಲ. ಆಗಾಗ ನಿಟ್ಟುಸಿರು ಬಿಡುತ್ತ ಉದ್ಗರಿಸುತ್ತಾರೆ, “ಓಹ್, ನಿಜಕ್ಕೂ ಶ್ರೀರಾಮಕೃಷ್ಣರ ತ್ಯಾಗ, ಭಗವದ್ವ್ಯಾಕುಲತೆ ಅದೆಷ್ಟು ಅದ್ಭುತ! ಅವರು ಸಾಧಿಸಿದ್ದ ಹದಿನಾರನೇ ಒಂದು ಭಾಗವನ್ನಾದರೂ ಸಾಧಿಸಿಕೊಳ್ಳಲು ನಮಗೆ ಸಾಧ್ಯವಾಗುತ್ತಿಲ್ಲವಲ್ಲ!”

ಮಠದಲ್ಲಿ ತೀವ್ರ ಆಧ್ಯಾತ್ಮಿಕ ಸಾಧನೆಯೊಂದಿಗೆ ಆಳವಾದ ತತ್ತ್ವಾಭ್ಯಾಸವೂ ನಡೆಯುತ್ತಿತ್ತು. ಪಾಶ್ಚಾತ್ಯ ತತ್ತ್ವಶಾಸ್ತ್ರಜ್ಞರಾದ ಕ್ಯಾಂಟ್, ಹೆಗಲ್, ಸ್ಪೆನ್ಸರ್ ಮುಂತಾದವರ ವಾದಗಳನ್ನು, ಅಲ್ಲದೆ ನಾಸ್ತಿಕವಾದಿಗಳ ಹಾಗೂ ಭೌತವಾದಿಗಳ ಸಿದ್ಧಾಂತಗಳನ್ನು ಅಧ್ಯಯನ ಮಾಡುತ್ತಿದ್ದರು. ಧರ್ಮದೊಂದಿಗೆ ಇತಿಹಾಸ ಸಮಾಜಶಾಸ್ತ್ರ ಸಾಹಿತ್ಯ ಕಲೆ ವಿಜ್ಞಾನಗಳನ್ನು ಅಭ್ಯಾಸಮಾಡಿ ಅವುಗಳ ಕುರಿತಾಗಿ ಚರ್ಚಿಸುತ್ತಿದ್ದರು. ಇಂದು ನರೇಂದ್ರ ‘ದೇವರು ಇಲ್ಲವೇ ಇಲ್ಲ; ದೇವರ ಅಸ್ತಿತ್ವವೆಂಬುದೇ ಒಂದು ಮಿಥ್ಯೆ’ ಎಂದು ವಾದಿಸಿ ಸಿದ್ಧಪಡಿಸುತ್ತಾನೆ. ಮರುದಿನ, ‘ಬಗವಂತನೊಬ್ಬನೇ ಸತ್ಯ’ಎಂದು ವಾದಿಸಿ ಪ್ರಮಾಣೀಕರಿಸುತ್ತಾನೆ! ಈ ಯುವಸಾಧಕರು ಷಡ್ ದರ್ಶನಗಳನ್ನು ಚೆನ್ನಾಗಿ ವಿಮರ್ಶಿಸಿ, ಅವುಗಳಲ್ಲಿ ಪರಸ್ಪರ ತಾಳೆಯಾಗುವ ವಿಷಯಗಳು ಯಾವುವು, ಪರಸ್ಪರ ವಿರುದ್ಧವಾದ ವಿಷಯಗಳು ಯಾವುವು ಎಂಬುದನ್ನು ಅರಿಯುತ್ತಿದ್ದರು; ವೇದಾಂತ ತತ್ತ್ವಗಳನ್ನೂ ಬೌದ್ಧತತ್ತ್ವಗಳನ್ನೂ ಹೋಲಿಸಿ ನೋಡುತ್ತಿದ್ದರು; ಇವುಗಳಲ್ಲಿ ಸಾಮಾನ್ಯವಾದ ಅಂಶಗಳು ಯಾವುವು, ಬುದ್ಧನ ಉಪದೇಶಗಳ ವೈಶಿಷ್ಟ್ಯವೇನು ಎಂಬುದನ್ನೆಲ್ಲ ವಿಚಾರ ಮಾಡುತ್ತಿದ್ದರು. ಕೆಲವೊಮ್ಮೆ ಕ್ರೈಸ್ತ ಪಾದರಿಗಳು ಬಂದು ಇವರೊಂದಿಗೆ ಚರ್ಚೆಮಾಡಿ, ಕ್ರೈಸ್ತಧರ್ಮದ ಹೆಗ್ಗಳಿಕೆಯನ್ನು ಬಣ್ಣಿಸುವುದಿತ್ತು. ಆಗ ನರೇಂದ್ರ ಅವರ ಸಿದ್ಧಾಂತಗಳ ಲೋಪದೋಷಗಳನ್ನು ತೋರಿಸಿಕೊಟ್ಟು, ಅವರು ಸೋಲೊಪ್ಪಿಕೊಳ್ಳುವಂತೆ ಮಾಡುತ್ತಿದ್ದ; ಬಳಿಕ ಏಸುಕ್ರಿಸ್ತನ ದಿವ್ಯತೆಯನ್ನು ಮನಮುಟ್ಟುವಂತೆ ಅವರಿಗೇ ವರ್ಣಿಸುತ್ತಿದ್ದ.

ಅವನು ಗುರುಭಾಯಿಗಳೆದುರಿಗೆ ಗುರುದೇವನ ಸಂಬಂಧವಾಗಿ ಒಂದು ಸ್ವತಂತ್ರ ವಿಚಾರಧಾರೆಯನ್ನು ಹರಿಯಿಸುತ್ತಿದ್ದ. ಶ್ರೀರಾಮಕೃಷ್ಣರ ಜೀವನ-ಸಂದೇಶಗಳ ಐತಿಹಾಸಿಕ ಮಹತ್ವವನ್ನು ಎತ್ತಿತೋರಿಸುತ್ತಿದ್ದ. ಭವಿಷ್ಯದ ಹಿಂದೂ ಜನಾಂಗದ ಮೇಲೆ ಅವರ ಜೀವನ-ಬೋಧನೆಗಳು ಬೀರಬಹುದಾದ ಪ್ರಭಾವವನ್ನು ವಿಶ್ಲೇಷಿಸುತ್ತಿದ್ದ.

ಈ ಯುವಸಾಧಕರ ನಡುವೆ ಎಂತಹ ಸ್ನೇಹ-ಗೌರವಪೂರ್ಣ ಬಾಂಧವ್ಯವಿತ್ತು. ಅವರು ಯಾವ ಬಗೆಯ ಪ್ರಾಮಾಣಿಕತೆಯಿಂದ ಸಾಧನೆ-ಜಿಜ್ಞಾಸೆಗಳಲ್ಲಿ ತೊಡಗಿರುತ್ತಿದ್ದರು ಎಂಬುದರ ಪರಿಚಯ ಮಾಡಿಕೊಳ್ಳಲು ‘ವಚನವೇದ’ದ ಪುಟವೊಂದನ್ನು ನೋಡಬೇಕು.

ಅಂದು ೭ನೇ ಮೇ, ೧೮೮೭, ಸೋದರಸಂನ್ಯಾಸಿಯೊಬ್ಬನು ತನಗಿನ್ನೂ ಭಗವಂತನ ಸಾಕ್ಷಾತ್ಕಾರವಾಗಲಿಲ್ಲವಲ್ಲ ಎಂದು ಚಿಂತಾಕ್ರಾಂತನಾಗಿದ್ದಾನೆ. ಅವನಿಗೆ ನರೇಂದ್ರ ಹೇಳುತ್ತಾನೆ:

“ಏನಯ್ಯ, ನೀನು ಗೀತೆಯನ್ನು ಓದಿಲ್ಲವೆ? ಭಗವಂತ ಪ್ರತಿಯೊಬ್ಬನ ಹೃದಯದಲ್ಲೂ ನೆಲೆಸಿ ದ್ದಾನೆ, ನಮ್ಮನ್ನೆಲ್ಲ ಈ ಜೀವನಚಕ್ರದಲ್ಲಿ ತಿರುಗಿಸುತ್ತಿರುವವನೂ ಅವನೇ ಎಂಬುದನ್ನು ನೀನು ಓದಿಲ್ಲವೆ? ಒಂದು ದೃಷ್ಟಿಯಿಂದ ನೋಡಿದರೆ, ನಾವು ಹರಿದಾಡುವ ಹುಳುವಿಗಿಂತಲೂ ಕಡೆ. ಇಂತಹ ನಾವು, ಅನಂತವಾದ ಭಗವಂತನ ನಿಜಸ್ವರೂಪವನ್ನು ಅರಿಯಲು ಸಾಧ್ಯವೇನು? ಈ ವಿಶ್ವದಲ್ಲಿ ಮಾನವನ ಸ್ಥಾನವನ್ನು ಸ್ವಲ್ಪ ಭಾವಿಸಿ ನೋಡು, ಆಕಾಶದಲ್ಲಿರುವ ಅಸಂಖ್ಯಾತ ನಕ್ಷತ್ರಗಳೆಲ್ಲವೂ ಒಂದೊಂದು ಸೂರ್ಯಮಂಡಲಗಳು. ನಾವೆಲ್ಲ ಇರುವುದು ಇಂತಹ ಒಂದು ಸೂರ್ಯಮಂಡಲದಲ್ಲಿ. ಸೂರ್ಯನಿಗೆ ಹೋಲಿಸಿದರೆ ನಮ್ಮ ಭೂಮಿ ಎಂಬುದೊಂದು ಪುಟ್ಟ ಚೆಂಡು. ಈ ಚೆಂಡಿನ ಮೇಲ್ಮೈಯಲ್ಲಿ ಮನುಷ್ಯ ಎನ್ನುವವನು ಒಂದು ಅಣು! ಇಂತಹ ನಾವು, ಆ ಭಗವಂತನನ್ನು ಅರಿತುಕೊಳ್ಳಬಲ್ಲೆವೇನು?”

ಬಳಿಕ ನರೇಂದ್ರ ಒಂದು ಹಾಡು ಹೇಳಿದ. ಆ ಹಾಡಿನ ಭಾವಾರ್ಥವೇನೆಂದರೆ, ‘ಹೇ ದೇವ, ನಾನು ನಿನಗೆ ಸದಾ ಶರಣಾಗಿದ್ದೇನೆ. ನನ್ನನ್ನು ಈ ಜಗತ್ತಿನ ಎಲ್ಲ ಅಪಾಯಗಳಿಂದ, ಪ್ರಲೋಭನೆ ಗಳಿಂದ ಪಾರುಮಾಡಿ ನನಗೆ ನಿರಂತರ ಮಾರ್ಗದರ್ಶನ ನೀಡು’ ಎಂದು. ನರೇಂದ್ರ ಮತ್ತೆ ಸೋದರಸಂನ್ಯಾಸಿಗೆ ಹೇಳುತ್ತಾನೆ: “ಭಗವಂತನಲ್ಲಿ ಸಂಪೂರ್ಣ ಶರಣಾಗು. ಭಗವಂತನ ಪಾದ ಪದ್ಮಗಳಲ್ಲಿ ನಿನ್ನನ್ನು ನೀನು ಸಂಪೂರ್ಣವಾಗಿ ಸಮರ್ಪಿಸಿಕೊಂಡುಬಿಡು. ಶ್ರೀರಾಮಕೃಷ್ಣರು ಹೇಳಿದ ಮಾತು ನಿನಗೆ ನೆನಪಿಲ್ಲವೆ? ಭಗವಂತನೆಂದರೆ ಅವನೊಂದು ದೊಡ್ಡ ಸಕ್ಕರೆ ಬೆಟ್ಟ. ನೀನೊಂದು ಇರುವೆ. ನಿನಗೆ ಆ ಸಕ್ಕರೆ ಬೆಟ್ಟದ ಒಂದು ಕಣವೇ ಬೇಕಾದಷ್ಟಾಯಿತು. ಆದರೆ ನೀನು ಆ ಇಡೀ ಬೆಟ್ಟವನ್ನೇ ಹೊತ್ತುಕೊಂಡು ಹೋಗಿ ನಿನ್ನ ಗೂಡಿನಲ್ಲಿಟ್ಟುಕೊಳ್ಳಬೇಕು ಎಂದು ನೋಡುತ್ತಿದ್ದೀಯೆ! ಅಂತಹ ಆ ಶುಕಮಹರ್ಷಿಯೂ ಹೆಚ್ಚೆಂದರೆ ಒಂದು ದೊಡ್ಡ ಇರುವೆ, ಅಷ್ಟೆ. ಆದ್ದರಿಂದ ನಾನು ಕಾಳೀಪ್ರಸಾದನಿಗೆ ಹೇಳುತ್ತಿರುತ್ತೇನೆ–‘ಏನಯ್ಯ, ನೀನು ನಿನ್ನ ಅಡಿ ಗೋಲಿನಿಂದ ಭಗವಂತನ ಉದ್ದಗಲವನ್ನು ಅಳೆಯುತ್ತೀಯಾ?’ ಅಂತ. ಭಗವಂತ ಅಪಾರವಾದ ದಯಾಸಾಗರ. ಅವನು ನಿನ್ನ ಮೇಲೆ ಕೃಪೆ ಮಾಡಿಯೇ ಮಾಡುತ್ತಾನೆ. ‘ಹೇ ಭಗವನ್, ನಮ್ಮನ್ನು ಸದಾ ಕಾಪಾಡು. ಯಾವಾಗಲೂ ನಿನ್ನ ದಯೆ ನಮ್ಮ ಮೇಲಿರಲಿ. ಅಸತ್ಯದಿಂದ ಸತ್ಯದೆಡೆಗೆ, ಕತ್ತಲಿನಿಂದ ಬೆಳಕಿನೆಡೆಗೆ, ಮೃತ್ಯುವಿನಿಂದ ಅಮೃತತ್ವದೆಡೆಗೆ ಸದಾ ನಮ್ಮನ್ನು ಮುನ್ನಡೆಸು’ ಎಂದು ಅವನನ್ನು ಪ್ರಾರ್ಥಿಸಿಕೋ.”

ಇನ್ನೊಬ್ಬ ಗುರುಭಾಯಿ ನರೇಂದ್ರನನ್ನು ಕೇಳುತ್ತಾನೆ: “ಭಗವಂತನನ್ನು ಯಾವ ರೀತಿಯಲ್ಲಿ ಪ್ರಾರ್ಥಿಸಿಕೊಳ್ಳಬೇಕು?”

ನರೇಂದ್ರ: “ಭಗವಂತನ ನಾಮವನ್ನು ಉಚ್ಚರಿಸಿದರಾಯಿತು. ಶ್ರೀರಾಮಕೃಷ್ಣರು ನಮಗೆ ಹೇಳಿಕೊಟ್ಟದ್ದು ಹಾಗೆಯೇ ಅಲ್ಲವೆ?”

ಸೋದರಶಿಷ್ಯ: “ನೀನು ಈಗೇನೋ ಹೇಳುತ್ತೀ, ‘ಭಗವಂತ ಇದ್ದಾನೆ, ಪ್ರಾರ್ಥಿಸಿಕೋ’ ಅಂತ. ಆದರೆ ನೀನು ಇನ್ನೊಂದು ಭಾವದಲ್ಲಿರುವಾಗ ಹೇಳುತ್ತೀ, ‘ಚಾರ್ವಾಕ ಹಾಗೂ ಇನ್ನು ಕೆಲವು ಸಿದ್ಧಾಂತಗಳ ಪ್ರಕಾರ, ಭಗವಂತ ಇಲ್ಲವೇ ಇಲ್ಲ; ಈ ಪ್ರಪಂಚ ಸೃಷ್ಟಿಯಾಗಲು ಹೊರಗಿನ ಶಕ್ತಿಯೊಂದರ ಆವಶ್ಯಕತೆಯೇನೂ ಇಲ್ಲ; ಅದು ತಾನಾಗಿಯೇ ಹುಟ್ಟುತ್ತದೆ, ಬೆಳೆಯುತ್ತದೆ, ಅಳಿಯುತ್ತದೆ’ ಅಂತ!”

ನರೇಂದ್ರ (ಮುಗುಳ್ನಗುತ್ತ): “ನೀನು ರಸಾಯನಶಾಸ್ತ್ರವನ್ನು ಓದಿಲ್ಲವೆ? ಅದರಲ್ಲಿ, ಜಲ ಜನಕ-ಆಮ್ಲಜನಕ ಸೇರಿ ನೀರಾಗುತ್ತದೆ, ಆದರೆ ಅವು ಹಾಗೆ ಸೇರಬೇಕಾದರೆ ಮಾತ್ರ ಮನುಷ್ಯನ ನೆರವು ಅಥವಾ ಒಂದು ಶಕ್ತಿಯ ನೆರವು ಬೇಕಾಗುತ್ತದೆ ಅಂತ ಹೇಳುವುದಿಲ್ಲವೆ? ಈ ಜಗತ್ತಿನಲ್ಲಿ ಯಾವುದನ್ನೇ ಆಗಲಿ ಪರಸ್ಪರ ಸೇರಿಸಲು ಹಿನ್ನೆಲೆಯಲ್ಲಿ ಒಂದು ಚೈತನ್ಯ ಕೆಲಸ ಮಾಡುತ್ತಿರಲೇ ಬೇಕು ಎಂಬುದನ್ನು ಪ್ರತಿಯೊಬ್ಬರೂ ಒಪ್ಪಿಕೊಳ್ಳುತ್ತಾರೆ. ಈ ಜಗತ್ತನ್ನು ನಡೆಸುವಂತಹ ಆ ಸರ್ವಜ್ಞ ಚೈತನ್ಯವೇ ಭಗವಂತ.”

ಸೋದರಶಿಷ್ಯ: “ಆದರೆ ಭಗವಂತ ಕರುಣಾಮಯ ಎಂಬುದನ್ನು ನಾವು ಕಂಡುಕೊಳ್ಳುವುದು ಹೇಗೆ?”

ನರೇಂದ್ರ: “ಇದಕ್ಕೆ ನಿನ್ನ ದಯಾಪೂರ್ಣ ಹೃದಯವೇ ಸಾಕ್ಷಿ! ವೇದಗಳು ಇದನ್ನೇ ಹೇಳು ತ್ತವೆ. ಮನುಷ್ಯನ ಹೃದಯದಲ್ಲಿ ಒಂದಿಷ್ಟಾದರೂ ದಯೆಯನ್ನು ಇಟ್ಟಂತಹ ಆ ಭಗವಂತನು ನಿಜಕ್ಕೂ ಕರುಣಾಸಾಗರನೇ ಸರಿ. ಶ್ರೀರಾಮಕೃಷ್ಣರು ಹೇಳುತ್ತಿದ್ದರು–‘ಶ್ರದ್ಧೆ ಬಹಳ ಮುಖ್ಯ ವಾದ ವಿಷಯ’ ಅಂತ. ಭಗವಂತ ನಮ್ಮ ಹತ್ತಿರದಲ್ಲೇ ಇದ್ದಾನೆ. ಅವನನ್ನು ಕಾಣಬೇಕಾದರೆ ನಿನಗೆ ಶ್ರದ್ಧೆಯಿರಬೇಕಾಗುತ್ತದೆ.”

ಸೋದರಶಿಷ್ಯ (ತಮಾಷೆಯ ದನಿಯಲ್ಲಿ): “ನರೇನ್, ನೀನು ಒಂದು ಸಲ ‘ಭಗವಂತ ಇಲ್ಲವೇ ಇಲ್ಲ’ ಅಂತ ಹೇಳುತ್ತೀಯೆ. ಇನ್ನೊಂದು ಸಲ, ‘ಭಗವಂತ ಇದ್ದಾನೆ’ ಎನ್ನುತ್ತೀಯೆ. ನೀನು ಹೀಗೆ ನಿನ್ನ ಅಭಿಪ್ರಾಯಗಳನ್ನು ಬದಲಾಯಿಸುತ್ತಲೇ ಹೋದರೆ ನಿನ್ನ ಮಾತನ್ನು ನಂಬುವುದು ಹೇಗೆ?”

ನರೇಂದ್ರ: “ಆದರೆ ನೋಡು, ನಾನು ಈ ಅಭಿಪ್ರಾಯವನ್ನು ಮಾತ್ರ ಬದಲಾಯಿಸುವುದಿಲ್ಲ; ಏನೆಂದರೆ, ನಾವು ಅಹಂಕಾರದಿಂದ ಮತ್ತರಾಗಿರುವವರೆಗೆ, ನಾವು ಆಸೆಗಳಿಗೆ ಬಲಿಯಾಗಿ ಅಲೆದಾಡುತ್ತಿರುವವರೆಗೆ, ನಮಗೆ ಭಗವಂತನ ಮೇಲೆ ಭಕ್ತಿ-ಶ್ರದ್ಧೆ ಹುಟ್ಟುವುದಿಲ್ಲ.”

ಬಳಿಕ ಭಾವಾವೇಶಭರಿತನಾಗಿ ಈ ಅರ್ಥದ ಹಾಡೊಂದನ್ನು ಹಾಡಲಾರಂಭಿಸಿದ: ‘ಭಗ ವಂತನು ಪರಮ ಕರುಣಾಮಯನಾದ ತಂದೆ-ತಾಯಿ. ಅವನಲ್ಲಿ ಶರಣಾಗತರಾದವರೆಲ್ಲರಿಗೂ ಅವನು ಆಶ್ರಯ ಕೊಟ್ಟೇಕೊಡುತ್ತಾನೆ.’ ಭಕ್ತಿಭಾವಪೂರ್ಣವಾದ ಆ ಹಾಡು ನರೇಂದ್ರನ ಮಧುಮಿಶ್ರಿತ ಕಂಠಶ್ರೀಯಿಂದ ಹೊರಹೊಮ್ಮಿ ಸುತ್ತಲಿದ್ದವರೆಲ್ಲರನ್ನೂ ಆನಂದಾಮೃತದಲ್ಲಿ ಮುಳುಗಿಸಿತು. “ನಾಗರಹಾವು ಪುಂಗಿಯ ನಾದವನ್ನು ಹೆಡೆಯೆತ್ತಿ ಆಲಿಸುತ್ತ ಮಂತ್ರಮುಗ್ಧವಾಗು ವಂತೆ, ನರೇಂದ್ರ ಹಾಡಿದಾಗ ಅಂತರ್ಯಾಮಿಯಾದ ಭಗವಂತನೂ ಮಂತ್ರಮುಗ್ಧನಾಗಿಬಿಡು ತ್ತಾನೆ” ಎಂದು ಶ್ರೀರಾಮಕೃಷ್ಣರು ಹೇಳುತ್ತಿದ್ದುದು ಮಹೇಂದ್ರನಾಥನಿಗೆ ನೆನಪಾಯಿತು.

ಶ್ರೀರಾಮಕೃಷ್ಣಭಾವವು ಯುವಸಂನ್ಯಾಸಿಗಳೆಲ್ಲರ ಹೃನ್ಮನಗಳನ್ನು ಆವರಿಸಿಬಿಟ್ಟಿತ್ತು. ಶ್ರೀ ರಾಮಕೃಷ್ಣರ ಸಾಕ್ಷಾತ್ ಸಾನ್ನಿಧ್ಯವನ್ನು ಅಲ್ಲಿ ಕಾಣಬಹುದಾಗಿತ್ತು. ಭಗವತ್ಸಾಕ್ಷಾತ್ಕಾರದ ವ್ಯಾಕುಲತೆ ಯಿಂದ ಕೂಡಿದ ಈ ಯುವಕರಿಗೆ ಹಗಲೂ ಇರುಳೂ ಒಂದೇ ಆಗಿಬಿಟ್ಟಿತ್ತು. ಸಮಯ ಸರಿದದ್ದೇ ತಿಳಿಯುತ್ತಿರಲಿಲ್ಲ. ಅವರ ಮನಸ್ಸೆಲ್ಲ ಆಧ್ಯಾತ್ಮಿಕ ಸಾಧನೆಯಲ್ಲಿ, ಆಧ್ಯಾತ್ಮಿಕ ಭಾವದಲ್ಲಿ ಮುಳುಗಿಬಿಟ್ಟಿದೆ. ನಿಜಕ್ಕೂ ಅವರೆಲ್ಲ ಹುಚ್ಚರಂತಾಗಿಬಿಟ್ಟಿದ್ದರು–ಭಗವಂತನ ದರ್ಶನಕ್ಕಾಗಿ ಹುಚ್ಚು! ಅವರು ತಮ್ಮ ಪಥದಲ್ಲಿ ಸರಿಯಾದ ರೀತಿಯಲ್ಲಿ ಮುಂದುವರಿಯುತ್ತಿದ್ದಾರೆಂಬುದಕ್ಕೆ ಶುಭಸೂಚನೆಗಳಾಗಿ ಅವರಿಗೆ ನಾನಾ ಬಗೆಯ ಆಧ್ಯಾತ್ಮಿಕ ಅನುಭವಗಳಾಗುತ್ತಿದ್ದುವು. ಕೆಲವರು ತೀವ್ರ ಧ್ಯಾನಮಗ್ನರಾಗಿ, ಅಲುಗಾಡದೆ ಗಂಟೆಗಟ್ಟಲೆ ಕುಳಿತುಬಿಡುತ್ತಿದ್ದರು. ಇನ್ನು ಕೆಲವರು ಭಕ್ತಿಭಾವದಲ್ಲಿ ಮೈಮರೆತು ಹಾಡುಗಳನ್ನು ಹೇಳಿಕೊಳ್ಳುತ್ತಿದ್ದರು. ಮತ್ತೆ ಕೆಲವರು ರಾತ್ರಿಯ ವೇಳೆಯಲ್ಲಿ ಧುನಿಯ ಮುಂದೆ ಕುಳಿತು, ಇಲ್ಲವೆ ಸಮೀಪದ ಸ್ಮಶಾನದಲ್ಲಿ ಧ್ಯಾನ ಮಾಡು ತ್ತಿದ್ದರು. ಕೆಲವರಂತೂ ಹಗಲಿರುಳೂ ಎಡೆಬಿಡದೆ ಜಪಮಾಡುತ್ತಲೇ ಇರುತ್ತಿದ್ದರು. ಅಂತೂ, ಪ್ರತಿಯೊಬ್ಬನೂ ಭಗವಂತನ ಸಾಕ್ಷಾತ್ಕಾರ ಮಾಡಿಕೊಳ್ಳಲೇಬೇಕು ಎಂಬ ದೃಢನಿಶ್ಚಯ ಮಾಡಿಬಿಟ್ಟಿದ್ದ.

ನರೇಂದ್ರನೂ ಉಳಿದವರಂತೆಯೇ ತೀವ್ರಸಾಧನೆಯಲ್ಲಿ ನಿರತನಾಗಿದ್ದ. ಆದರೆ ಅವನಿಗೆ ಸಾಧನೆಯ ಜೊತೆಗೆ ಇನ್ನೂ ಒಂದು ಹೊಣೆಗಾರಿಕೆಯಿತ್ತು. ಅದು ಸೋದರಸಂನ್ಯಾಸಿಗಳ ಸಾಧನಾದಿಗಳ ಕಡೆಗೂ ಗಮನವಿಟ್ಟಿರುವುದು. ಅವನನ್ನು ಶ್ರೀರಾಮಕೃಷ್ಣರೇ ಈ ಯುವಕರೆಲ್ಲರ ನಾಯಕನೆಂದು ನೇಮಕ ಮಾಡಿದ್ದರೂ, ಅವನು ತನ್ನ ಸ್ವಂತ ಯೋಗ್ಯತೆಯಿಂದಲೂ ಅವರ ನಾಯಕನಾಗಿದ್ದ. ಮಾತ್ರವಲ್ಲ, ಅವನು ಅವರೆಲ್ಲರ ಸೋದರನಂತೆ, ಆತ್ಮೀಯ ಸ್ನೇಹಿತನಂತೆ ಇದ್ದ. ಅವನ ಮೂಲಕ ಮಾತನಾಡುವುದು ಸಾಕ್ಷಾತ್ ಶ್ರೀರಾಮಕೃಷ್ಣರೇ ಎಂಬ ಭಾವನೆ ಅವರಲ್ಲಿ ಮತ್ತೆಮತ್ತೆ ಏಳುತ್ತಿತ್ತು. ಆಧ್ಯಾತ್ಮಿಕವಾಗಿ ಅವನೊಂದು ಮಹಾಸಿಂಹದಂತಿದ್ದ. ಆದ್ದರಿಂದ ಆತನ ಬಗೆಗಿನ ಅವರ ಪ್ರೀತಿಯೆಂಬುದು ಪೂಜ್ಯಭಾವವಾಗಿ ಪರಿವರ್ತಿತವಾಗಿತ್ತು. ಶ್ರೀರಾಮ ಕೃಷ್ಣರು ಅವನ ಕುರಿತಾಗಿ ಹೇಳಿದ ಮಾತುಗಳೆಲ್ಲ ಅವರ ಕಿವಿಯಲ್ಲಿ ಇನ್ನೂ ಮೊಳಗುತ್ತಿದ್ದುವು. ಅವನ ಜ್ವಲಿಸುವ ಕಣ್ಣುಗಳು, ಪ್ರಫುಲ್ಲವಾದ ಮುಖ, ವಿದ್ಯುತ್ತಿನಂತಹ ಮಾತುಗಳು, ಅವರಲ್ಲಿ ಅವನಿಟ್ಟಿದ್ದ ಪ್ರೀತಿ-ವಿಶ್ವಾಸ, ಅವನ ಸ್ಫೂರ್ತಿಯ ನುಡಿಗಳು, ಅವರನ್ನೆಲ್ಲ ಸದಾ ಉತ್ಸಾಹಭರಿತರ ನ್ನಾಗಿರಿಸುತ್ತಿದ್ದ ಅವನ ವರ್ತನೆ–ಇವುಗಳಿಂದೆಲ್ಲ ಅವರು ಆತನಲ್ಲಿ ಶ್ರೀರಾಮಕೃಷ್ಣರೇ ಆವಿರ್ ಭವಿಸಿರುವಂತೆ ಕಾಣುತ್ತಿದ್ದರು. ಎಷ್ಟೋ ಸಲ ಅವನು ಅವರನ್ನು ಗದರಿಸಿ ಬುದ್ಧಿ ಹೇಳ ಬೇಕಾಗುತ್ತಿತ್ತು. ಅವರೆಲ್ಲ ತಪ್ಪದೆ ಸರಿಯಾಗಿ ಸಾಧನೆ ಮಾಡುತ್ತಾರೋ ಇಲ್ಲವೋ ಎಂದೇನೂ ಅವನು ಗಮನಿಸಬೇಕಾಗಿರಲಿಲ್ಲ; ಆದರೆ ಅವರು ಅತಿಯಾದ ಸಾಧನೆ ಮಾಡಿ ಅಪಾಯಕ್ಕೆ ಗುರಿ ಯಾಗದಂತೆ ನೋಡಿಕೊಳ್ಳಬೇಕಾಗಿತ್ತು. ಯಾರಾದರೂ ಅತಿಯಾದ ಸಾಧನೆ ಮಾಡುವುದು ಕಂಡುಬಂದರೆ, “ಏನು, ‘ನಾವೂ ಸಾಧನೆ ಮಾಡಿ ಪರಮಹಂಸರಾಗಿಬಿಡುತ್ತೇವೆ’ ಎಂದುಕೊಂಡಿ ದ್ದೀರೋ? ಅದೆಲ್ಲ ಎಂದಿಗೂ ಸಾಧ್ಯವಿಲ್ಲ! ಯುಗಕ್ಕೊಬ್ಬ ರಾಮಕೃಷ್ಣರು ಹುಟ್ಟಬಹುದು, ಅಷ್ಟೆ” ಎಂದು ಛೇಡಿಸುತ್ತಿದ್ದ.

ಸ್ವಾಮಿ ವಿವೇಕಾನಂದರ ಪ್ರಥಮ ಸಂನ್ಯಾಸೀಶಿಷ್ಯರಾದ ಸ್ವಾಮಿ ಸದಾನಂದರು\footnote{*ಇವರು ವಿವೇಕಾನಂದರ ಶಿಷ್ಯರಾದ ಮತ್ತು ಮಠಕ್ಕೆ ಸೇರಿದ ಕಥೆಯನ್ನು ಮುಂದಿನ ಅಧ್ಯಾಯದಲ್ಲಿ ನೋಡಲಿದ್ದೇವೆ.} ಮುಂದೆ ಬಾರಾನಗೋರ್ ಮಠದ ದಿನಗಳನ್ನು ಸ್ಮರಿಸಿಕೊಂಡು ಹೇಳುತ್ತಾರೆ: “ಆಗ ಸ್ವಾಮೀಜಿಯವರು ದಿನಕ್ಕೆ ಇಪ್ಪತ್ತನಾಲ್ಕು ಗಂಟೆಗಳ ಪ್ರಕಾರ ಕೆಲಸ ಮಾಡುತ್ತಿದ್ದರು. ‘ಏಳಿ, ಎದ್ದೇಳಿ! ದಿವ್ಯಾಮೃತ ವನ್ನು ಪಾನಮಾಡಬೇಕೆಂಬವರೆಲ್ಲ ಎದ್ದೇಳಿ’ ಎಂದು ರಾಗವಾಗಿ ಹಾಡುತ್ತ ಸೋದರಸಂನ್ಯಾಸಿಗಳ ನ್ನೆಲ್ಲ ನಸುಕಿನಲ್ಲೇ ಎಬ್ಬಿಸುತ್ತಿದ್ದರು, ಮತ್ತು ಅರ್ಧರಾತ್ರಿ ಕಳೆದಮೇಲೂ ಮಠದ ತಾರಸಿಯ ಮೇಲೆ ಗುರುಭಾಯಿಗಳೊಂದಿಗೆ ಕುಳಿತು, ಭಾವಾವೇಶದಿಂದ ಭಜನೆಮಾಡುತ್ತಿದ್ದರು. ನೆರೆಹೊರೆ ಯವರು ಬಂದು ಹಾಡುವುದನ್ನು ನಿಲ್ಲಿಸುವಂತೆ ಕೇಳಿಕೊಂಡರೂ ಭಜನೆಯೇನೂ ನಿಲ್ಲುತ್ತಿರ ಲಿಲ್ಲ! ಸ್ವಾಮೀಜಿ ತಮ್ಮ ಸುಮಧುರ ಕಂಠದಿಂದ ‘ಸೀತಾರಾಮ್ ಸೀತಾರಾಮ್’ ಇಲ್ಲವೆ, ‘ರಾಧಾಕೃಷ್ಣ’ ಎಂಬ ನಾಮಾವಳಿಗಳನ್ನು ಹೇಳಿಕೊಡುತ್ತಿದ್ದರು. ಆ ದಿನಗಳಲ್ಲಿ ಯಾರಿಗೂ ವಿಶ್ರಾಂತಿಯೇ ಇರಲಿಲ್ಲ. ಎಷ್ಟೋ ಜನ ಭಕ್ತಾದಿಗಳು ಬರುತ್ತಿದ್ದರು, ಹೋಗುತ್ತಿದ್ದರು, ಪಂಡಿತರು ಬಂದು ವಾದ ಮಾಡುತ್ತಿದ್ದರು, ಚರ್ಚೆ ಮಾಡುತ್ತಿದ್ದರು. ಆದರೆ ಸ್ವಾಮಿಜೀ ಮಾತ್ರ ಯಾವ ಕಾರಣಕ್ಕೂ ಒಂದು ಕ್ಷಣವೂ ಬೇಸರಿಸುತ್ತಿರಲಿಲ್ಲ.”

ಶ್ರೀರಾಮಕೃಷ್ಣ ಮಹಾಸಂಘದ ಇತಿಹಾಸದಲ್ಲಿ ತೀವ್ರ ಆಧ್ಯಾತ್ಮಿಕ ಸಾಧನೆ ಎಂಬುದರ ಇನ್ನೊಂದು ಹೆಸರೇ ‘ಬಾರಾನಗೋರ್ ಮಠ’ ಎಂಬಂತಾಗಿದೆ. ಬಾರಾನಗೋರ್ನ ಮಠದ ಹೆಸರನ್ನು ಕೇಳಿದರೆ ಸಾಕು, ಆಧ್ಯಾತ್ಮಿಕ ಸಾಧನೆಯ ನೆನಪಾಗುತ್ತದೆ. ಯಾರ್ಯಾರು ಆ ಯುವ ಸಂನ್ಯಾಸಿಗಳ ಸಂಪರ್ಕಕ್ಕೆ ಬರುತ್ತಿದ್ದರೋ, ಅವರಿಗೆಲ್ಲ ಈ ‘ಭಗವಂತನ ಹುಚ್ಚು’ ಹಿಡಿಯು ತ್ತಿತ್ತು. ನಿರಂತರ ಆಧ್ಯಾತ್ಮಿಕ ಸಾಧನೆಯಿಂದಾಗಿ ಈ ಸಂನ್ಯಾಸಿಗಳ ವ್ಯಕ್ತಿತ್ವವೆಲ್ಲ ಭಗವನ್ಮಯ ವಾಗಿಬಿಟ್ಟಿತ್ತು. ಶ್ರೀರಾಮಕೃಷ್ಣರು ತಮ್ಮ ಸ್ವಂತದವರನ್ನಾಗಿ ಮಾಡಿಕೊಂಡಿದ್ದ ಆ ಯುವಕ ರೆಲ್ಲರೂ ಈಗ ತಮ್ಮ ಗುರುದೇವನ ವ್ಯಕ್ತಿತ್ವದ ಒಂದಲ್ಲ ಒಂದು ಅಂಶವನ್ನು ಪ್ರತಿನಿಧಿಸುವಂತಾಗಿದ್ದರು.

