
\chapter{ಹಿಮವಂತನ ಮಡಿಲಲ್ಲಿ}

\noindent

ಸ್ವಾಮೀಜಿ ತಮ್ಮ ಮನಸ್ಸಿನಿಂದ ಮಠದ ಹಾಗೂ ಸಂನ್ಯಾಸೀಬಂಧುಗಳ ಮೇಲಿನ ಬಂಧನಗಳ ನ್ನೆಲ್ಲ ಬಿಡಿಸಿಕೊಂಡು, ಹಗುರಗೊಂಡ ಹೃದಯದಿಂದ, ತಮ್ಮ ಉದ್ದೇಶ ಸಿದ್ಧಿಯೊಂದನ್ನೇ ಮನದಲ್ಲಿಟ್ಟುಕೊಂಡು, ನೇರವಾಗಿ ಹಿಮಾಲಯದತ್ತ ಆನಂದದಿಂದ ಸಾಗಿದ್ದಾರೆ. ಅವರ ಸಂಗಡಿ ಗರಾಗಿ ಸ್ವಾಮಿ ಅಖಂಡಾನಂದರಿದ್ದಾರೆ. ಹಿಮಾಲಯದಲ್ಲೇ ಜನ್ಮತಳೆದು ಹರಿದು ಬರುವ ಪವಿತ್ರ ಗಂಗೆಯ ದಡದ ಮೇಲೆ ಕಾಲ್ನಡಿಗೆಯಲ್ಲೇ ಪ್ರಯಾಣ ಮಾಡುವ ಯೋಜನೆ ಹೂಡಿದರು ಸ್ವಾಮೀಜಿ. ಇದರಿಂದ ಒಂದು ಆಧ್ಯಾತ್ಮಿಕ ಸಾಧನೆಯೂ ಆಗುತ್ತದೆ, ಒಂದು ಸಾಹಸವೂ ಆಗುತ್ತದೆ. ಕಲ್ಕತ್ತವನ್ನು ಬಿಟ್ಟು ಹೊರಟಕ್ಷಣದಿಂದಲೂ ಸ್ವಾಮೀಜಿ ಸ್ವತಂತ್ರ ಹಕ್ಕಿಯಂತೆ ಆನಂದ ಭರಿತರಾಗಿದ್ದಾರೆ. ಅವರು ಬಹಳವಾಗಿ ಹಂಬಲಿಸಿದ್ದ ಏಕಾಂತತೆ, ಹಳ್ಳಿಗಳ ಪರಿಶುದ್ಧ- ಪ್ರಶಾಂತ ವಾತಾವರಣ, ದಾರಿಯುದ್ದಕ್ಕೂ ಸಿಗುವ ಹೊಸಹೊಸ ಸ್ಥಳಗಳು, ಅಲ್ಲಲ್ಲಿ ಭೇಟಿಯಾ ಗುವ ಹೊಸ ಹೊಸ ಜನರು–ಇವೆಲ್ಲ ಅವರ ಮನಸ್ಸಿಗೆ ನವೋಲ್ಲಾಸವನ್ನು ತಂದಿವೆ. ಜೊತೆಗೆ, ಕಲ್ಕತ್ತದಲ್ಲಿನ ಹಳೆಯ ನೆನಪುಗಳು, ಚಿಂತೆಗಳು ಮರೆಯಾಗಿವೆ. ಆದ್ದರಿಂದ ಅವರೀಗ ಗಂಗಾ ನದಿಯ ದಾರಿಯಾಗಿ ಒಬ್ಬ ಸಾಧಾರಣ ಸಾಧುವಿನಂತೆ ಪಯಣಿಸುತ್ತಿದ್ದಾರೆ.

ಕಲ್ಕತ್ತವನ್ನು ಬಿಟ್ಟಮೇಲೆ ಸುಮಾರು ಹದಿನೈದು-ಇಪ್ಪತ್ತು ದಿನಗಳ ದೀರ್ಘ ಪ್ರಯಾಣದ ಬಳಿಕ ಅವರು ಉಳಿದುಕೊಂಡ ಮೊದಲ ಸ್ಥಳ ಭಾಗಲ್ಪುರ. ಸ್ವಾಮಿಗಳಿಬ್ಬರೂ ಗಂಗಾತೀರದಲ್ಲಿ ನಡೆದು ಹೋಗುತ್ತಿದ್ದಾರೆ; ದೀರ್ಘ ಪ್ರಯಾಣದಿಂದಾಗಿ ಆಯಾಸಗೊಂಡಿದ್ದಾರೆ. ಆದರೆ ಮುಖದಲ್ಲಿ ತ್ಯಾಗ ವೈರಾಗ್ಯಗಳ ದೀಪ್ತಿ ಬೆಳಗುತ್ತಿದೆ. ಆ ಊರಿನ ಒಬ್ಬ ಪ್ರಮುಖ ನಾಗರಿಕನಾದ ಕುಮಾರ್ ನಿತ್ಯಾನಂದ ಸಿಂಗ್ ಎಂಬಾತ ಅವರನ್ನು ಕಂಡು ಒಡನೆಯೇ ಆಕರ್ಷಿತನಾದ. ಅವರು ಸಾಧಾರಣ ಬೈರಾಗಿಗಳಂತಲ್ಲ ಎನ್ನುವುದನ್ನು ಆತ ಸ್ಪಷ್ಟವಾಗಿ ಕಂಡ. ಸ್ವಾಮಿಗಳ ಬಳಿಗೆ ಬಂದು ನಮಸ್ಕರಿಸಿ, ಸ್ವಲ್ಪ ಸಂಭಾಷಣೆ ನಡೆಸಿದ. ಸ್ವಾಮೀಜಿಯ ವಾಗ್ವೈಖರಿಯನ್ನು ಕಂಡಮೇಲಂತೂ ಅವರು ಅಸಾಮಾನ್ಯ ವ್ಯಕ್ತಿಗಳೆನ್ನುವುದು ಇನ್ನಷ್ಟು ಖಚಿತವಾಯಿತು. ಬಳಿಕ ಇಬ್ಬರನ್ನೂ ತನ್ನ ಪರಿಚಿತನಾದ ಮನ್ಮಥನಾಥ ಚೌಧರಿ ಎಂಬವನ ಮನೆಗೆ ಕರೆದುಕೊಂಡು ಬಂದ. ಇದೇ ಸಮಯದಲ್ಲಿ ಅವನ ಮನೆಯಲ್ಲಿ ಮಥುರಾನಾಥ ಸಿನ್ಹ ಎಂಬವನು ಅತಿಥಿಯಾಗಿ ಉಳಿದು ಕೊಂಡಿದ್ದ. ವೃತ್ತಿಯಿಂದ ಈತ ವಕೀಲ; ಧಾರ್ಮಿಕ ವಿಚಾರಗಳಲ್ಲಿ ಸಾಕಷ್ಟು ತಿಳಿವಳಿಕೆಯಿದ್ದ ವನು. ಈ ಇಬ್ಬರು ಸ್ವಾಮಿಗಳನ್ನು ಕಂಡಕೂಡಲೇ ಅವನು ಅವರತ್ತ ಸೆಳೆಯಲ್ಪಟ್ಟ. ಮಥುರಾ ನಾಥ ಈ ಹಿಂದೆಯೇ ಬ್ರಾಹ್ಮಸಮಾಜದಲ್ಲಿ ಸ್ವಾಮೀಜಿಯನ್ನು ನೋಡಿದ್ದ–ಅವರು ಯುವಕ ನರೇಂದ್ರನಾಗಿದ್ದಾಗ, ಈತ ಅವರೊಡನೆ ಸಾಹಿತ್ಯ, ತತ್ತ್ವಶಾಸ್ತ್ರ ಹಾಗೂ ಧರ್ಮದ ಕುರಿತಾಗಿ ಕೆಲಕಾಲ ಸಂಭಾಷಿಸಿದ. ಜ್ಞಾನ ಹಾಗೂ ಆಧ್ಯಾತ್ಮಿಕತೆ ಅವರ ಉಸಿರಿನಲ್ಲೇ ತುಂಬಿಹೋಗಿರುವಂತೆ ಆತನಿಗೆ ಭಾಸವಾಯಿತು. ಅವರ ಪ್ರತಿಯೊಂದು ಭಾವನೆಯ ಹಿಂದೆಯೂ ತೀವ್ರ, ನಿಃಸ್ವಾರ್ಥ ದೇಶಪ್ರೇಮ ತುಂಬಿರುವುದನ್ನು ಆತ ಕಂಡುಕೊಂಡ.

ಈ ಇಬ್ಬರು ಸ್ವಾಮಿಗಳನ್ನು ಕುಮಾರ್ ನಿತ್ಯಾನಂದ ಸಿಂಗ್ ತನ್ನ ಮನೆಗೆ ಕರೆತಂದಾಗ, ಅವರನ್ನು ಕಂಡ ಮನ್ಮಥನಾಥ ಚೌಧರಿ ಅವರನ್ನು ಯಾರೋ ಭಿಕ್ಷುಕ ಸಂನ್ಯಾಸಿಗಳು ಎಂದು ಭಾವಿಸಿ ಅವರ ಕಡೆಗೆ ಹೆಚ್ಚು ಗಮನ ಕೊಟ್ಟಿರಲಿಲ್ಲ. ಆ ದಿನ ಮಧ್ಯಾಹ್ನದ ಊಟವಾದ ಮೇಲೆ ಎಲ್ಲರೂ ಕುಳಿತುಕೊಂಡಿದ್ದರು. ಆಗ ಈ ಮನ್ಮಥಬಾಬು ಸ್ವಾಮಿಗಳನ್ನು ಅವರಷ್ಟಕ್ಕೆ ಬಿಟ್ಟು ತಾನು ಬೌದ್ಧಧರ್ಮದ ಕುರಿತಾದ ಒಂದು ಇಂಗ್ಲಿಷ್ ಪುಸ್ತಕವನ್ನು ಓದುತ್ತ ಕುಳಿತುಬಿಟ್ಟ. ಓದು ಬರಹವಿಲ್ಲದ ಈ ಸಂನ್ಯಾಸಿಗಳೊಂದಿಗೆ ಏನು ಮಾತು ಎಂಬುದು ಅವನ ಭಾವನೆ. ಸ್ವಲ್ಪ ಹೊತ್ತಾದ ಮೇಲೆ ಸ್ವಾಮೀಜಿ ಅವನನ್ನು, “ನೀವು ಓದುತ್ತಿರುವ ಪುಸ್ತಕ ಯಾವುದದು?” ಎಂದು ಕೇಳಿದರು. ಆಗ ಅವನು ಆ ಪುಸ್ತಕದ ಶಿರೋನಾಮೆಯನ್ನು ಹೇಳಿ, ಬಳಿಕ ಕೇಳಿದ: “ನಿಮಗೆ ಇಂಗ್ಲಿಷ್ ಬರುತ್ತದೆಯೆ?” ಎಂದು. “ಏನೋ ಸ್ವಲ್ಪ ಮಟ್ಟಿಗೆ” ಎಂದರು ಸ್ವಾಮೀಜಿ. ಈಗ ಆತ ಅವರೊಂದಿಗೆ ಬೌದ್ಧ ತತ್ತ್ವಗಳ ಮೇಲೆ ಸಂಭಾಷಣೆ ಪ್ರಾರಂಭಿಸಿದ. ಸ್ವಲ್ಪ ಹೊತ್ತಿನಲ್ಲೇ ಅವನಿಗೆ ಅರಿವಾಯಿತು–ಸ್ವಾಮೀಜಿ ತನಗಿಂತ ಸಾವಿರ ಪಾಲು ಹೆಚ್ಚು ತಿಳಿದವರು ಎಂದು. ಅವರು ಹಲವಾರು ಇಂಗ್ಲಿಷ್ ಪುಸ್ತಕಗಳಿಂದ ಉದಾಹರಿಸಿ ನಿರರ್ಗಳವಾಗಿ ಮಾತನಾಡುತ್ತಿದ್ದಂತೆ ಮನ್ಮಥಬಾಬು ಹಾಗೂ ಮಥುರನಾಥರಿಬ್ಬರೂ ಬೆಕ್ಕಸಬೆರಗಾಗಿ ಕಿವಿಗೊಟ್ಟು ಆಲಿಸಿದರು. ಇಂತಹವರು ಅತಿಥಿಗಳಾಗಿ ಬಂದಿರುವುದು ತಮ್ಮ ಅದೃಷ್ಟವಿಶೇಷ ಎಂದರಿತ ಮನ್ಮಥನಾಥ, ಅವರನ್ನು ಕೆಲದಿನ ತಮ್ಮ ಮನೆಯಲ್ಲಿ ವಿಶ್ರಮಿಸಿ ಮುಂದುವರಿಯಬೇಕೆಂದು ಒತ್ತಾಯಿಸಿ ಒಪ್ಪಿಸಿದ.

ಮನ್ಮಥನಾಥ ಬ್ರಾಹ್ಮಸಮಾಜದ ಕಟ್ಟಾ ಅನುಯಾಯಿಯಾಗಿದ್ದ. ಹಿಂದೂ ಧರ್ಮದ ತತ್ತ್ವಗಳನ್ನು ಅರಿತುಕೊಳ್ಳಲಾಗದೆ, ಅವುಗಳಿಂದ ವಿಮುಖನಾಗಿದ್ದ. ಸ್ವಾಮೀಜಿ ಅವನಿಗೆ ಹಿಂದೂ ಧರ್ಮದ ಹಲವಾರು ಅಂಶಗಳನ್ನು ತಿಳಿಸಿಕೊಟ್ಟು ಅವನ ಕಣ್ಣುತೆರೆಸಿದರು. ಆತ ಈಗ ನಿಜವಾದ ಅರ್ಥದಲ್ಲಿ ಒಬ್ಬ ಹಿಂದುವಾದ.

ಮತ್ತೊಂದು ದಿನ, ಯೋಗ-ಆಧ್ಯಾತ್ಮಿಕ ಸಾಧನೆ ಇವುಗಳ ಪ್ರಸ್ತಾಪ ಬಂದಿತು. ಮನ್ಮಥನಾಥ, ಪ್ರಸಿದ್ಧ ತತ್ತ್ವಜ್ಞಾನಿಯೂ ಸಮಾಜ ಸುಧಾರಕರೂ ಆದ ಸ್ವಾಮಿ ದಯಾನಂದ ಸರಸ್ವತಿಯವರಿಂದ ಅದಾಗಲೇ ಈ ವಿಷಯವಾಗಿ ಶಿಕ್ಷಣ ಪಡೆದಿದ್ದ. ಅಂದು ಸ್ವಾಮೀಜಿ ಹೇಳಿದ ವಿಷಯಗಳು ಅವನು ಈಗಾಗಲೇ ಕೇಳಿ ತಿಳಿದಿದ್ದ ವಿಷಯಗಳಿಗೆ ಅನುಗುಣವಾಗಿದ್ದುವು. ಅಲ್ಲದೆ ಅದಕ್ಕಿಂತಲೂ ಹೆಚ್ಚಿನ ಹಲವಾರು ವಿಚಾರಗಳನ್ನು ಅವರಿಂದ ಕೇಳಿದ. ಇವರೊಬ್ಬರು ಅಸಾಧಾರಣ ವ್ಯಕ್ತಿಯೇ ನಿಜ ಎಂಬುದು ಅವನಿಗೆ ದೃಢವಾಯಿತು.

ಬಳಿಕ ಸ್ವಾಮೀಜಿಯ ಸಂಸ್ಕೃತಜ್ಞಾನವನ್ನು ಪರೀಕ್ಷೆ ಮಾಡಲು ಅವನು ತನ್ನ ಬಳಿಯಿದ್ದ ಉಪನಿಷತ್ತುಗಳನ್ನೆಲ್ಲ ಹೊರತೆಗೆದು, ಅವುಗಳಲ್ಲಿನ ಕೆಲವು ಅತ್ಯಂತ ಕ್ಲಿಷ್ಟವಾದ ಶ್ಲೋಕಗಳ ಬಗ್ಗೆ ಕೆದಕಿ ಪ್ರಶ್ನಿಸಿದ. ಸ್ವಾಮೀಜಿ ಆ ಅಂಶಗಳ ಮೇಲೆ ಹೊಸ ಬೆಳಕನ್ನು ಬೀರುವಂತಹ ಉತ್ತರ ಗಳನ್ನು ನೀಡಿದಾಗ ಅವರ ಅಸಾಮಾನ್ಯ ಪಾಂಡಿತ್ಯವನ್ನೂ ಸಂಸ್ಕೃತ ಪರಿಜ್ಞಾನವನ್ನೂ ಕಂಡು ನಿಬ್ಬೆರಗಾದ. ಅಲ್ಲದೆ, ಅವರು ಉಪನಿಷತ್ತುಗಳ ಶ್ಲೋಕಗಳನ್ನು ಸುಮಧುರವಾಗಿ, ನಿರರ್ಗಳವಾಗಿ ಉಚ್ಚರಿಸುವ ಕ್ರಮವನ್ನು ಕಂಡು ಅವನಿಗೆ ಆನಂದ, ಆಶ್ಚರ್ಯ. ಹೀಗೆ ಸ್ವಾಮೀಜಿಗೆ ಇಂಗ್ಲಿಷ್, ಸಂಸ್ಕೃತ ಹಾಗೂ ಯೋಗದ ಮೇಲಿರುವ ಪ್ರಭುತ್ವವನ್ನು ಕಂಡು ಅವರೆಡೆಗೆ ತೀವ್ರವಾಗಿ ಆಕರ್ಷಿತನಾಗಿಬಿಟ್ಟ. ಮತ್ತು, ತನ್ನ ಮನೆಯನ್ನು ಬಿಟ್ಟು ಬೇರೆಲ್ಲಿಗೂ ಹೋಗಲು ಅವರನ್ನು ಬಿಡಲೇಬಾರದು ಎಂದು ನಿರ್ಧರಿಸಿ, ಬಹಳವಾಗಿ ಒತ್ತಾಯಿಸಲಾರಂಭಿಸಿದ!

ಒಂದು ದಿನ ಸ್ವಾಮೀಜಿ ತಮ್ಮಷ್ಟಕ್ಕೆ ತಾವೇ ಹಾಡೊಂದನ್ನು ಗುನುಗಿಕೊಳ್ಳುತ್ತಿದ್ದರು. ಇದನ್ನು ಗಮನಿಸಿದ ಮನ್ಮಥನಾಥ, “ನಿಮಗೆ ಹಾಡಲು ಗೊತ್ತಿದೆಯೆ?” ಎಂದು ಕೇಳಿದ. “ಏನೋ ಅಲ್ಪ ಸ್ವಲ್ಪ” ಎಂದರು ಸ್ವಾಮೀಜಿ. ಮನೆಯಲ್ಲಿದ್ದವರೆಲ್ಲರ ಬಲವಂತಕ್ಕೆ ಸ್ವಾಮೀಜಿ ಹಾಡಿದರು. ಅದನ್ನು ಕೇಲಿದವರೆಲ್ಲ ಮೋಡಿಗೊಳಗಾದವರಂತೆ ಕುಳಿತರು. ಇವರು ಇಂಗ್ಲಿಷ್-ಸಂಸ್ಕೃತ ಭಾಷೆಗಳಲ್ಲಿ, ಯೋಗ-ಶಾಸ್ತ್ರ ವಿಚಾರಗಳಲ್ಲಿ ಎಂತಹ ಪಂಡಿತರೋ ಸಂಗೀತದಲ್ಲೂ ಅಷ್ಟೇ ಪ್ರತಿಭಾವಂತರು ಎಂಬುದನ್ನು ಕಂಡುಕೊಂಡ ಮನ್ಮಥನಾಥ ಪರಮಾಶ್ಚರ್ಯಗೊಂಡ. ಈ ಸವಿಯನ್ನು ಇತರರಿಗೂ ಉಣಬಡಿಸಬೇಕೆಂಬ ಆಶಯದಿಂದ,“ಸ್ವಾಮೀಜಿ, ನೀವು ಒಪ್ಪಿಗೆ ಕೊಡುವುದಾದರೆ, ಈ ಊರಿನ ಕೆಲವು ಸಂಗೀತಗಾರರನ್ನು ಇಲ್ಲಿಗೆ ಕರೆಸುತ್ತೇನೆ. ನೀವು ದಯವಿಟ್ಟು ಹಾಡಬೇಕು” ಎಂದು ವಿನಂತಿಸಿಕೊಂಡ. ಸ್ವಾಮೀಜಿ ಅದಕ್ಕೆ ಸಮ್ಮತಿಸಿದರು. ಮನ್ಮಥನಾಥ ಬಹಳ ಸಂತೋಷಪಟ್ಟ. ಹಲವಾರು ಸಂಗೀತಗಾರರನ್ನು ಆಹ್ವಾನಿಸಿದ. ಅವರಲ್ಲಿ ಕೆಲವರಂತೂ ಉಸ್ತಾದರು, ಅಲ್ಲಿನ ಅತ್ಯಂತ ನಿಪುಣ ಸಂಗೀತಗಾರರು. ಸಂಜೆಯ ವೇಳೆಗೆ ಕಾರ್ಯಕ್ರಮ ಪ್ರಾರಂಭವಾಯಿತು. ಸ್ವಾಮೀಜಿ ತಮ್ಮ ದಿವ್ಯ ಮಧುರ ಕಂಠದಿಂದ ಹಾಡಲಾ ರಂಭಿಸಿದರು. ಮನ್ಮಥನಾಥ ಭಾವಿಸಿದ್ದ–ರಾತ್ರಿ ಸುಮಾರು ಒಂಬತ್ತು-ಹತ್ತು ಗಂಟೆಯವರೆಗೂ ಸ್ವಾಮೀಜಿ ಹಾಡಬಗುದು ಎಂಎದು. ಆದ್ದರಿಂದ ಆಹ್ವಾನಿತರಿಗೆ ಊಟದ ವ್ಯವಸ್ಥೆಯನ್ನೂ ಮಾಡಿರಲಿಲ್ಲ. ಆದರೆ ಗಂಟೆ ಹತ್ತಾಯಿತು, ಹನ್ನೊಂದಾಯಿತು; ಸ್ವಾಮೀಜಿ ಹಾಡುವುದನ್ನು ನಿಲ್ಲಿಸಲೇ ಇಲ್ಲ. ಬದಲಾಗಿ ಇನ್ನಷ್ಟು ಹೆಚ್ಚಿನ ಭಕ್ತಿಭಾವಾವೇಶದಿಂದ ಹಾಡುತ್ತಲೇ ಇದ್ದಾರೆ! ಸಂಗೀತವನ್ನು ಕೇಳುತ್ತಿದ್ದವರೆಲ್ಲ ಹಸಿವು-ಬಾಯಾರಿಕೆಯ ನೆನಪೇ ಇಲ್ಲದೆ ಕುಳಿತು ಕೇಳುತ್ತಿದ್ದಾರೆ ಒಬ್ಬರಾದರೂ ಕುಳಿತಲ್ಲಿಂದ ಜರುಗಲಿಲ್ಲ. ಕೊನೆಗೆ ಬೆಳಗಿನ ಜಾವ ಮೂರು ಗಂಟೆಯ ಹೊತ್ತಿಗೆ ಸ್ವಾಮೀಜಿ ಹಾಡುವುದನ್ನು ನಿಲ್ಲಿಸಿದರು. ಆದರೆ ಅವರು ಹಾಡನ್ನು ನಿಲ್ಲಿಸಿದ್ದು ತಮಗೆ ಆಯಾಸ ವಾಯಿತೆಂದಲ್ಲ, ತಬಲಾ ಪಕ್ಕವಾದ್ಯ ನುಡಿಸುತ್ತಿದ್ದ ಕೈಲಾಸಬಾಬು ಎಂಬವನು, ಕೈಸೋತು ಮರ ಗಟ್ಟಿಹೋದಂತಾಗಿ ತಬಲಾ ನುಡಿಸುವುದನ್ನು ನಿಲ್ಲಿಸಿದ್ದರಿಂದ! ವಿಸ್ಮಯಗೊಂಡ ಮನ್ಮಥ ಬಾಬು ಉದ್ಗರಿಸುತ್ತಾನೆ: “ಇಂತಹ ಅತಿಮಾನುಷ ಶಕ್ತಿಯನ್ನು ನಾನು ಈ ಹಿಂದೆ ಕಂಡಿರಲಿಲ್ಲ; ಮಂದೆ ಕಾಣಬಲ್ಲೆನೆಂಬ ನಿರೀಕ್ಷೆಯೂ ಇಲ್ಲ!” ಸ್ವಾಮೀಜಿಯ ಕೀರ್ತಿ ಊರಲ್ಲೆಲ್ಲ ಹರಡಿತು. ಮರುದಿನ ಸಂಜೆ, ಹಿಂದಿನ ದಿನ ಬಂದಿದ್ದವರಲ್ಲದೆ ಇನ್ನೂ ಹಲವಾರು ಜನ ಯಾವ ಆಮಂತ್ರಣವೂ ಇಲ್ಲದೆ ತಾವಾಗಿಯೇ ಬಂದುಬಿಟ್ಟರು. ತಬಲಾ ನುಡಿಸುವವನೂ ಬಂದಿದ್ದ. ಆದರೆ ಅಂದು ಸ್ವಾಮೀಜಿ ಹಾಡಲಿಲ್ಲ. ಎಲ್ಲರಿಗೂ ಬಹಳ ನಿರಾಶೆಯಾಯಿತು. ಸ್ವಾಮೀಜಿ ಭಾವದುಂಬಿಬಂದಾಗ, ಸ್ವಸಂತೋಷಕ್ಕಾಗಿ ಹಾಡುವವರೇ ಹೊರತು ಪ್ರತಿಭಾಪ್ರದರ್ಶನಕ್ಕಾಗಿ ಹಾಡುವವರಲ್ಲ. 

ಒಮ್ಮೆ ಮನ್ಮಥನಾಥ, “ಸ್ವಾಮೀಜಿ, ಈ ಊರಿನ ಶ್ರೀಮಂತ ವ್ಯಕ್ತಿಗಳನ್ನು ನಿಮಗೆ ಪರಿಚಯಿಸಿ ಕೊಡುತ್ತೇನೆ. ನೀವು ಒಪ್ಪುವುದಾದರೆ ನನ್ನ ಸಾರೋಟಿನಲ್ಲೇ ಹೋಗಿ ಅವರನ್ನು ಭೇಟಿ ಮಾಡೋಣ” ಎಂದು ಸಲಹೆ ಮಾಡಿದ. ಆಗ ಸ್ವಾಮೀಜಿ ಆ ಸಲಹೆಯನ್ನು ತಿರಸ್ಕರಿಸಿ, ಕಡ್ಡಿ ತುಂಡುಮಾಡಿದಂತೆ ಹೇಳುತ್ತಾರೆ: “ಶ್ರೀಮಂತರನ್ನು ಹುಡುಕಿಕೊಂಡು ಹೋಗಿ ಭೇಟಿ ಮಾಡು ವುದು ಸಂನ್ಯಾಸಿಯ ಧರ್ಮವಲ್ಲ!” ಇಂತಹ ಉತ್ತರವನ್ನು ಆತ ನಿರೀಕ್ಷಿಸಿರಲಿಲ್ಲ. ಅವರ ಪ್ರಖರ ವೈರಾಗ್ಯಬುದ್ಧಿ ಅವನ ಮೇಲೆ ಆಳವಾದ ಪರಿಣಾಮವನ್ನುಂಟುಮಾಡಿತು. ಕೆಲವೇ ದಿನಗಳ ಸ್ವಾಮೀಜಿಯ ಸತ್ಸಂಗದಲ್ಲಿ ಮನ್ಮಥನಾಥ ಕಲಿತ ಪಾಠಗಳು ಹಲವಾರು; ಪಡೆದ ಆಧ್ಯಾತ್ಮಿಕ ಮಾರ್ಗದರ್ಶನ ಅಮೂಲ್ಯ.

ಆತ ಸ್ವಭಾವತಃ ಆಧ್ಯಾತ್ಮಿಕ ಜೀವನದತ್ತ ಒಲವಿದ್ದವನು. ಬಾಲ್ಯದಿಂದಲೂ ಅವನಿಗೆ ಏಕಾಂತವಾಸದಲ್ಲಿದ್ದುಕೊಂಡು ಆಧ್ಯಾತ್ಮಿಕ ಸಾಧನೆಯಲ್ಲಿ ತೊಡುಗುವ ಇಚ್ಛೆಯಿತ್ತು. ಈಗ ಸ್ವಾಮೀಜಿಯ ಸಂಪರ್ಕಕ್ಕೆ ಬಂದಮೇಲೆ ಅವನ ಈ ಆಕಾಂಕ್ಷೆ ಪ್ರಬಲವಾಯಿತು. ಆದ್ದರಿಂದ ತನ್ನದೊಂದು ಸಲಹೆಯನ್ನು ಅವರ ಮುಂದಿಟ್ಟ: “ಸ್ವಾಮೀಜಿ, ನಾವಿಬ್ಬರೂ ಬೃಂದಾವನಕ್ಕೆ ಹೋಗಿದ್ದುಬಿಡೋಣ. ನಮ್ಮಿಬ್ಬರ ಹೆಸರಿನಲ್ಲೂ ಮುನ್ನೂರು-ಮುನ್ನೂರು ರೂಪಾಯಿಗಳನ್ನು ಅಲ್ಲಿ ಗೋವಿಂದನ ದೇವಸ್ಥಾನದಲ್ಲಿ ಮೂಲಧನವಾಗಿಟ್ಟರೆ, ಜೀವಮಾನಪರ್ಯಂತ ಗೋವಿಂದನ ಪ್ರಸಾದವನ್ನು ಪಡೆಯಬಹುದು. ಹೀಗೆ ನಾವು ಯಾರಿಗೂ ಹೊರೆಯಾಗದ ರೀತಿ ಯಲ್ಲಿ, ಪವಿತ್ರ ಯಮುನಾ ನದಿಯ ತೀರದ ನಿರ್ಜನ ಪ್ರದೇಶದಲ್ಲಿ ಹಗಲಿರುಳೂ ಜಪ-ಧ್ಯಾನ ಗಳಲ್ಲಿ ನಿರತರಾಗಿದ್ದುಬಿಡೋಣ.” ಅದಕ್ಕೆ ಸ್ವಾಮೀಜಿ ಹೇಳುತ್ತಾರೆ: “ನಿಜ ನಿಜ; ಆ ತರಹದ ಮನೋಭಾವದವರಿಗೆ ಈ ಯೋಜನೆ ಬಹಳ ಒಳ್ಳೆಯದೆ. ಆದರೆ ಎಲ್ಲರಿಗೂ ಅಲ್ಲ” ಎಂದು. ‘ಎಲ್ಲರಿಗೂ ಅಲ್ಲ’ ಎಂದರೆ, ‘ತಮ್ಮಂಥವರಿಗಂತೂ ಅಲ್ಲ’ ಎಂಬುದು ಸ್ವಾಮೀಜಿಯ ಇಂಗಿತ. ಪ್ರತಿಯೊಬ್ಬ ಸಾಧಕನ ಮನೋಭಾವ, ಹಾಗೂ ಆ ಮನೋಭಾವಕ್ಕೆ ಅನುಗುಣವಾಗಿ ಆತ ಆರಿಸಿ ಕೊಳ್ಳುವ ಸಾಧನಾ ವಿಧಾನಗಳು ಬೇರೆಬೇರೆ. ಅದರಲ್ಲೂ ಮುಂದೆ ವಿಶ್ವದಲ್ಲೊಂದು ಕ್ರಾಂತಿ ಯನ್ನೇ ಉಂಟುಮಾಡಬೇಕಾಗಿರುವ ಸ್ವಾಮೀಜಿ, ಸಾಧನೆಯ ಹೆಸರಿನಲ್ಲಿ ಯಾವುದೋ ನಿರ್ಜನ ಪ್ರದೇಶದಲ್ಲಿ ಜೀವಮಾನಪರ್ಯಂತ ಕುಳಿತಿರಲು ಸಾಧ್ಯವೆ?

ಭಾಗಲ್ಪುರದಲ್ಲಿದ್ದ ಅವಧಿಯಲ್ಲಿ ಸ್ವಾಮೀಜಿ ಸಮೀಪದ ಬರಾರಿ ಮತ್ತು ನಾಥನಗರ ಎಂಬ ಎರಡು ಸ್ಥಳಗಳಿಗೆ ಭೇಟಿ ನೀಡಿದರು. ಬರಾರಿಯಲ್ಲಿ ಅವರು ಪಾರ್ವತೀಚರಣ ಮುಖ್ಯೋಪಾ ಧ್ಯಾಯ ಎಂಬ ಸತ್ಪುರುಷನನ್ನು ಸಂದರ್ಶಿಸಿದರು. ನಾಥನಗರದಲ್ಲಿ ಒಂದು ಪ್ರಸಿದ್ಧ ಜೈನ ದೇವಾಲಯವಿದೆ. ಸ್ವಾಮೀಜಿ ಅಲ್ಲಿಗೆ ಭೇಟಿನೀಡಿ, ಅಲ್ಲಿನ ಆಚಾರ್ಯರೊಂದಿಗೆ ಜೈನಧರ್ಮದ ಕುರಿತಾಗಿ ಸಂಭಾಷಣೆ ನಡೆಸಿದರು.

ಭಾಗಲ್ಪುರದಲ್ಲಿ ಸ್ವಾಮೀಜಿಯ ಹೆಸರು ಹರಡಿದಾಗ, ಅಲ್ಲಿನ ಬ್ರಾಹ್ಮಣವರ್ಗದ ಒಂದು ಗುಂಪಿನವರು ತಿಳಿದುಕೊಳ್ಳಲು ಇಚ್ಛಿಸಿದ ಒಂದು ‘ಮುಖ್ಯ’ ವಿಷಯವೆಂದರೆ, ‘ಅವರು ಯಾವ ಜಾತಿಗೆ ಸೇರಿದವರು?’ ಎಂಬುದು. ಸ್ವಾಮೀಜಿ ಹುಟ್ಟಿನಿಂದ ಬ್ರಾಹ್ಮಣರಲ್ಲ, ಕ್ಷತ್ರಿಯರು ಎಂದು ಕಂಡುಕೊಂಡ ಕೆಲವರು ತಕ್ಷಣ ಟೀಕೆಮಾಡಿದರು–ಬ್ರಾಹ್ಮಣರಲ್ಲದವರು ಸಂನ್ಯಾಸ ತೆಗೆದು ಕೊಂಡದ್ದು ಸರಿಯಲ್ಲ, ಎಂದು. ಈ ಟೀಕೆ ಮನ್ಮಥನಾಥನ ಕಿವಿಗೆ ಬಿತ್ತು. ಜುಗುಪ್ಸೆಗೊಂಡ ಆತ ಹೇಳುತ್ತಾನೆ: “ಒಬ್ಬ ಮನುಷ್ಯ ಸಂತನಾಗಲು, ಭಕ್ತನಾಗಲು ಅವನ ಜಾತಿ ಅಡ್ಡಿಯಾಗುವು ದಿಲ್ಲ ಎಂಬುದು ಅವರಿಗೇನು ಗೊತ್ತು? ಅಥವಾ, ಸ್ವಾಮೀಜಿಯ ಯೋಗ್ಯತೆ ಏನೆಂಬುದಾದರೂ ಅವರಿಗೆ ಗೊತ್ತೆ? ಆದರೆ, ಒಬ್ಬ ಸಂನ್ಯಾಸಿಗೆ ಅವನು ಧರಿಸಿರುವ ಕಾಷಾಯವಸ್ತ್ರವೊಂದೇ ಯೋಗ್ಯತೆ ಎನ್ನುವುದಾದರೆ, ಅವನ ಜಾತಿ ಯಾವುದು ಎಂದು ನೋಡುವುದು ಸರಿಯೇ! ಇಲ್ಲದೆಹೋದರೆ ಈ ಬ್ರಾಹ್ಮಣರು ಒಬ್ಬ ಸಂನ್ಯಾಸಿಯನ್ನು ಆತ ಜಾತಿಯಿಂದ ಬ್ರಾಹ್ಮಣ ಎಂಬ ಕಾರಣಕ್ಕಲ್ಲದೆ ಇನ್ಯಾವ ದೃಷ್ಟಿಯಿಂದ ಅವನನ್ನು ಪೂಜಿಸಿ ಸಮಾಧಾನಪಟ್ಟುಕೊಳ್ಳಲು ಸಾಧ್ಯ! ಸ್ವಾಮೀಜಿ ಬ್ರಾಹ್ಮಣರಲ್ಲದಿರಬಹುದು; ಆದರೆ ಅವರು ಬ್ರಹ್ಮಜ್ಞಾನಿಗಳು. ಆದ್ದರಿಂದ ಅವರ ಬ್ರಾಹ್ಮಣ್ಯವು ಜಾತಿಬ್ರಾಹ್ಮಣರ ಬ್ರಾಹ್ಮಣ್ಯಕ್ಕಿಂತ ಹತ್ತುಪಾಲು ಹೆಚ್ಚು ಶ್ರೇಷ್ಠವಾದದ್ದು! ಇಷ್ಟರ ಮೇಲೆ, ಸ್ವಾಮೀಜಿಯಲ್ಲಿ ನಿಜವಾದ ಸಾಧುತ್ವ ಹಾಗೂ ದೈವತ್ವ ಕಂಡುಬರುತ್ತಿರುವಾಗ ಯಾರು ತಾನೆ ಅವರನ್ನು ಪೂಜಿಸದಿರಲು ಸಾಧ್ಯ?”

ಸ್ವಾಮೀಜಿ ಭಾಗಲ್ಪುರಕ್ಕೆ ಬಂದು ಅದಾಗಲೇ ಐದಾರು ದಿನಗಳಾಗಿವೆ. ಇನ್ನು ಅಲ್ಲಿ ನಿಲ್ಲಲಾರರು. ಏಕೆಂದರೆ, ಅವರು ನೇರವಾಗಿ ಹಿಮಾಲಯಕ್ಕೆ ಹೊರಟವರು. ಆದರೆ ಮನ್ಮಥನಾಥ ಅವರನ್ನು ಸುಲಭವಾಗಿ ಬಿಟ್ಟುಕೊಡುವವನಲ್ಲ. ಅವರಿಂದ ದೂರವಾಗುವ ಆಲೋಚನೆಯೇ ಅವನಿಗೆ ಸಹ್ಯವಾಗುತ್ತಿಲ್ಲ! ಆದ್ದರಿಂದ ಒಂದು ದಿನ ಆತ ಯಾವುದೋ ಕಾರ್ಯನಿಮಿತ್ತವಾಗಿ ಹೊರಗೆ ಹೋಗಿದ್ದ ಸಮಯ ನೋಡಿಕೊಂಡು ಸ್ವಾಮೀಜಿ ಮನೆಯಲ್ಲಿದ್ದವರಿಂದ ಬೀಳ್ಗೊಂಡು ಹೊರಟೇಬಿಟ್ಟರು. ಮನೆಗೆ ಹಿಂದಿರುಗಿದಮೇಲಷ್ಟೇ ಅವನಿಗೆ ಇಬ್ಬರು ಸ್ವಾಮಿಗಳೂ ಹೊರಟು ಹೋದರು ಎಂಬುದು ತಿಳಿಯಿತು; ತುಂಬ ದುಃಖವಾಯಿತು. ತಕ್ಷಣ ಊರಲ್ಲೆಲ್ಲ ಹುಡುಕಿಸಿ ನೋಡಿದ; ಎಲ್ಲೂ ಅವರ ಸುಳಿವೇ ಸಿಗಲಿಲ್ಲ. ಆದರೂ ಸ್ವಾಮೀಜಿಯನ್ನು ಕಾಣಲೇಬೇಕೆಂಬ ಅವನ ಇಚ್ಛೆ ಕಡಿಮೆಯಾಗಲಿಲ್ಲ. ಒಮ್ಮೆ ಮಾತಿನ ಸಂದರ್ಭದಲ್ಲಿ, ತಾವು ಬದರಿಕಾಶ್ರಮದ ಕಡೆಗೆ ಹೋಗುತ್ತಿರುವುದಾಗಿ ಸ್ವಾಮೀಜಿ ಅವನಿಗೆ ಹೇಳಿದ್ದರು. ಆದ್ದರಿಂದ ಆತ ಅವರನ್ನು ಅರಸುತ್ತ ಆಲ್ಮೋರದವರೆಗೂ ಹೋದ. ಆದರೆ ಅಷ್ಟೊತ್ತಿಗಾಗಲೇ ಅವರು ಅಲ್ಲಿಂದಲೂ ಹೊರಟುಬಿಟ್ಟಿದ್ದರು. ಅವರು ಈಗಾಗಲೇ ಹಿಮಾಲಯದಲ್ಲಿ ಬಹಳ ದೂರ ಹೋಗಿದ್ದಿರಬೇಕು ಎಂದು ಊಹಿಸಿ ಮನ್ಮಥನಾಥ ನಿರಾಶನಾದ. ‘ನಿಜ, ಎಲ್ಲ ನನ್ನಿಚ್ಛೆಯಂತೆಯೇ ನಡೆಯಬೇಕು ಎಂದು ನಾನಾದರೂ ಏಕೆ ಯೋಚಿಸಬೇಕು? ಅಲ್ಲದೆ ಇಡೀ ವಿಶ್ವವೇ ಸ್ವಾಮೀಜಿಯ ಕಾರ್ಯ ಕ್ಷೇತ್ರವಾಗಿರುವಾಗ ಅವರು ನನ್ನ ಮನೆಯಲ್ಲೇ ಕೂಪಮಂಡೂಕದಂತೆ ಏಕೆ ಕುಳಿತಿರಬೇಕು?’ ಎಂದು ಆಲೋಚಿಸಿ ಸಮಾಧಾನ ತಂದುಕೊಂಡು ಹಿಂದಿರುಗಿದ.

ಭಾಗಲ್ಪುರದಿಂದ ಹೊರಟ ಸ್ವಾಮಿಗಳಿಬ್ಬರೂ ವೈದ್ಯನಾಥ ಕ್ಷೇತ್ರಕ್ಕೆ ಬಂದು ತಲುಪಿದರು. ಇಲ್ಲಿನ ಪ್ರಸಿದ್ಧ ಶಿವ ದೇವಸ್ಥಾನವನ್ನು ಸಂದರ್ಶಿಸಲು ಸ್ವಾಮಿ ಅಖಂಡಾನಂದರು ಇಚ್ಛಿಸಿದ್ದರು. ಇಲ್ಲಿ ಸ್ವಾಮೀಜಿಗೆ ರಾಜನಾರಾಯಣ ಬೋಸನೆಂಬ ಹೆಸರಾಂತ ಬ್ರಾಹ್ಮಸದಸ್ಯನ ಭೇಟಿಯಾ ಯಿತು. ಈತ ಸಮಾಜ ಸುಧಾರಕ, ರಾಷ್ಟ್ರಪ್ರೇಮಿ. ಇವನಿಗೆ ‘ಭಾರತದ ರಾಷ್ಟ್ರೀಯತಾಭಾವದ ಪಿತಾಮಹ’ನೆಂಬ ಗೌರವ ಸಂದಿತ್ತು. ಆದರೆ ಇವನಲ್ಲೊಂದು ವೈಚಿತ್ರ್ಯವಿತ್ತು. ಇವನ ಭಾರತ ಪ್ರೇಮವೆಂಬುದು ಪಾಶ್ಚಾತ್ಯ ಸಂಸ್ಕೃತಿಯ ಹಾಗೂ ಇಂಗ್ಲಿಷ್ ಭಾಷೆಯ ವಿರುದ್ಧವಾಗಿ ತಿರುಗಿ ಕೊಂಡಿತ್ತು. ಇಂಗ್ಲಿಷಿನಲ್ಲಿ ಮಾತನಾಡುವುದನ್ನಾಗಲಿ, ಇಂಗ್ಲಿಷ್ ಪದಗಳನ್ನು ಬಳಸುವುದನ್ನಾ ಗಲಿ ಆತ ತೀವ್ರವಾಗಿ ಆಕ್ಷೇಪಿಸುತ್ತಿದ್ದ. ಇದು ಹಾಸ್ಯಾಸ್ಪದವಾಗುವಷ್ಟರವರೆಗೆ ಬೆಳೆದಿತ್ತು.

ಸ್ವಾಮೀಜಿಗೆ ರಾಜನಾರಾಯಣನ ಈ ವಿಚಿತ್ರ ನಡವಳಿಕೆಯ ಬಗ್ಗೆ ತಿಳಿದಿತ್ತು. ಆದ್ದರಿಂದ ತಮಗೆ ಇಂಗ್ಲಿಷ್ ಗೊತ್ತಿದೆಯೆಂಬುದು ಅವನಿಗೆ ಸ್ವಲ್ಪವೂ ಅರಿವಾಗದಂತೆ ನೋಡಿಕೊಳ್ಳ ಬೇಕೆಂದು ಅಖಂಡಾನಂದರಿಗೆ ಮೊದಲೇ ತಿಳಿಸಿದ್ದರು. ಸಂಭಾಷಣೆ ಆರಂಭವಾಯಿತು. ನಿಜಕ್ಕೂ ಆತನೊಂದಿಗಿನ ಸಂಭಾಷಣೆ ಸ್ವಾಮೀಜಿಯ ಪಾಲಿಗೆ ಉಪಯುಕ್ತವಾಗಿತ್ತು. ಭಾರತೀಯತೆಯ ಕುರಿತಾದ ಆತನ ಅನೇಕ ಅಭಿಪ್ರಾಯಗಳು ಅವರಿಗೆ ಸಂಪೂರ್ಣ ಸಮ್ಮತವಾದುವು. ಆದರೆ ಈ ನಡುವೆ ಅಲ್ಲೊಂದು ತಮಾಷೆಯ ಪ್ರಸಂಗವೇರ್ಪಟ್ಟಿತು. ಸ್ವಾಮೀಜಿ ಅಪ್ಪಟ ಬಂಗಾಳಿಯಲ್ಲಿ ನಿರರ್ಗಳವಾಗಿ ಮಾತನಾಡುತ್ತಿದ್ದು, ಮಧ್ಯೆ ಒಂದೇ ಒಂದು ಇಂಗ್ಲಿಷ್ ಪದವನ್ನೂ ಪ್ರಯೋಗಿಸ ದಂತೆ ಎಚ್ಚರವಹಿಸಿದರು. ಅದನ್ನು ಕಂಡು ರಾಜನಾರಾಯಣನಿಗೆ, ಈ ಸಂನ್ಯಾಸಿಗಳಿಗೆ ಇಂಗ್ಲಿಷಿನ ಗಂಧವೇ ಇಲ್ಲ ಎಂಬ ಅಭಿಪ್ರಾಯವುಂಟಾಯಿತು. ಆದರೆ ಅವರ ವಾಗ್ವೈಖರಿ ಹಾಗೂ ಬುದ್ಧಿಯ ತೀಕ್ಷ್ಣತೆಯನ್ನು ನೋಡಿ ಆತ ಬಹಳ ಸಂತೋಷಪಟ್ಟ. ಮಾತಿನ ಮಧ್ಯದಲ್ಲಿ ಅವನೇ ಆಕಸ್ಮಿಕವಾಗಿ ‘ಪ್ಲಸ್\eng{’ (plus)} ಎಂಬ ಇಂಗ್ಲಿಷ್ ಪದವನ್ನು ಪ್ರಯೋಗಿಸಿಬಿಟ್ಟ. ತಕ್ಷಣ ಸ್ವಾಮಿಗಳಿಗೆ ಅದು ಅರ್ಥವಾಗಿರಲಾರದು ಎಂದು ಭಾವಿಸಿ, ತನ್ನ ಬೆರಳುಗಳಲ್ಲಿ ‘ಪ್ಲಸ್’ ಗುರುತನ್ನು ತೋರಿಸಿ ಹೇಳಿದ. ಏಕೆಂದರೆ ‘ಪ್ಲಸ್’ ಎಂಬುದರ ಸಮಾನಾರ್ಥದ ಪದ ಬಂಗಾಳಿಯಲ್ಲಿಲ್ಲ. ಆತನ ಒದ್ದಾಟವನ್ನು ಕಂಡು ಸ್ವಾಮಿಗಳಿಬ್ಬರಿಗೂ ತಡೆಯಲಾರದ ನಗು. ಆದರೂ ಈಚೆಗೆ ಬರುವವರೆಗೆ ನಗುವನ್ನು ಹೇಗೋ ಹತ್ತಿಕ್ಕಿಕೊಂಡರು. ಅಖಂಡಾನಂದರಿಗಂತೂ ಗುಟ್ಟನ್ನು ಕಾಪಾಡಿಕೊಳ್ಳ ಬೇಕಾದರೆ ವಿಪರೀತ ಕಷ್ಟವಾಯಿತು. ಏಕೆಂದರೆ ತಮ್ಮ ಅಚ್ಚುಮೆಚ್ಚಿನ ವಿವೇಕಾನಂದರ ಗುಣ ಗಾನ ಮಾಡುವುದೆಂದರೆ ಅವರಿಗೆ ಎಲ್ಲಿಲ್ಲದ ಸಂತೋಷ. ಸ್ವಾಮೀಜಿಯವರು ಸ್ವತಃ ಆಂಗ್ಲ ರಿಗಿಂತ ಸೊಗಸಾದ ಇಂಗ್ಲಿಷ್ ಮಾತನಾಡಬಲ್ಲರು ಎಂಬುದು, ಪಾಪ, ಆ ಮುದುಕನಿಗೆ ಗೊತ್ತಾಗಲೇ ಇಲ್ಲ!

ಆದರೆ ಹಲವಾರು ವರ್ಷಗಳ ಬಳಿಕ ಸ್ವಾಮೀಜಿ ವಿಶ್ವವಿಖ್ಯಾತರಾದ ಮೇಲೆ ರಾಜನಾರಾಯಣ ಬಾಬುವಿಗೆ ತಿಳಿದುಬಂತು–ಇವರೇ ಅಂದು ತಾನು ಭೇಟಿಯಾದ ಸಂನ್ಯಾಸಿ ಎಂದು. ಆಗ ಆತ ಆಶ್ಚರ್ಯಾಘಾತಗೊಂಡು, “ಎಲ ಎಲಾ! ಅಂದು ಅವರು ನನ್ನೊಡನೆ ಅಷ್ಟು ಹೊತ್ತು ಮಾತ ನಾಡಿದರೂ ಅವರಿಗೆ ಇಂಗ್ಲಿಷ್ ಭಾಷೆ ತಿಳಿದಿರಬಹುದು ಎಂದು ನನಗೆ ಸ್ವಲ್ಪವೂ ಗೊತ್ತಾಗಲೇ ಇಲ್ಲವಲ್ಲ! ಅವರು ನಿಜಕ್ಕೂ ಒಬ್ಬ ಅದ್ಭುತ ವ್ಯಕ್ತಿ” ಎಂದುದ್ಗರಿಸುತ್ತಾನೆ.

ವೈದ್ಯನಾಥದಿಂದ ವಾರಾಣಸಿಗೆ ಬಂದು ತಲುಪಿದ ಸ್ವಾಮಿಗಳು ಪ್ರಮದದಾಸ ಮಿತ್ರರ ಮನೆಯಲ್ಲಿ ಕೆಲವು ದಿನಗಳ ಮಟ್ಟಿಗೆ ಉಳಿದುಕೊಂಡರು. ಈ ಸಮಯದಲ್ಲಿ ಸ್ವಾಮೀಜಿ, ಅವರ ಜೊತೆಯಲ್ಲಿ ಹಲವಾರು ಆಧ್ಯಾತ್ಮಿಕ ವಿಚಾರಗಳ ಕುರಿತಾಗಿ ಗಂಟೆಗಟ್ಟಲೆ ಸಂಭಾಷಿಸಿದರು. ಆದರೆ ಅವರಿಗೆ ತುಷಾರಧವಲಕಾಂತಿಯಿಂದ ಕೂಡಿದ ಹಿಮಾಲಯವನ್ನು ನೋಡುವ ತವಕ. ಆದ್ದರಿಂದ ಕೆಲದಿನಗಳಲ್ಲೇ ವಾರಾಣಸಿಯಿಂದ ಹೊರಟುನಿಂತರು. ಹೊರಡುವ ಸಮಯದಲ್ಲಿ ಪ್ರಮದ ಬಾಬು ಹಾಗೂ ಇನ್ನಿತರರ ಸಮ್ಮುಖದಲ್ಲಿ ಸ್ವಾಮೀಜಿ ಹೇಳುತ್ತಾರೆ:“ಈಗ ನಾನು ಕಾಶಿಯನ್ನು ಬಿಟ್ಟು ಹೊರಡುತ್ತಿದ್ದೇನೆ. ನಾನು ಈ ಸಮಾಜದ ಮೇಲೆ ಬಾಂಬಿನಂತೆ ಸಿಡಿದೆರಗುವವರೆಗೂ ಇಲ್ಲಿಗೆ ಮತ್ತೆ ಬರುವುದಿಲ್ಲ. ಆಗ ಈ ಸಮಾಜವು ನನ್ನನ್ನು ನಾಯಿಯಂತೆ ಹಿಂಬಾಲಿಸುತ್ತದೆ.” ಇದು ಸುಮಾರು ಇಪ್ಪತ್ತೆಂಟು ವರ್ಷದ ಯುವಕಸಂನ್ಯಾಸಿ ವಿವೇಕಾನಂದರು ಅಂದು ನುಡಿದ ನುಡಿ. ಅವರ ನುಡಿ ಅಕ್ಷರಶಃ ಸತ್ಯವಾಯಿತು. ಮುಂದೆ ಅವರು ಪಾಶ್ಚಾತ್ಯ ದೇಶಗಳನ್ನು ತಮ್ಮ ನವನೂತನ ಭಾವನೆಗಳಿಂದ ಅಲುಗಾಡಿಸಿ, ಜಗತ್ಪ್ರಸಿದ್ಧಿಯನ್ನು ಹೊಂದಿ ಹಿಂದಿರುಗಿಬಂದಾಗ, ಇಡೀ ಭಾರತದ ಮೇಲೆ ಬಾಂಬಿನಂತೆ ಎರಗುತ್ತಾರೆ. ಸಾಮಾನ್ಯ ಜನರಿರಲಿ, ರಾಜ-ಮಹಾರಾಜರು ಕೂಡ ವಿವೇಕಾನಂದರನ್ನು ಹೊತ್ತ ಸಾರೋಟನ್ನು ತಮ್ಮ ಕೈಗಳಿಂದಲೇ ಎಳೆಯುತ್ತಾರೆ. ಆದರೆ ಇಲ್ಲಿ ಸ್ವಾಮೀಜಿ ನಾಯಿಯ ಉಪಮಾನವನ್ನು ಕೊಟ್ಟದ್ದು ತಿರಸ್ಕಾರದಿಂದ ಎಂದು ಭಾವಿಸಬೇಕಾ ಗಿಲ್ಲ. ವಿಶ್ವಾಸ-ಶ್ರದ್ಧೆ-ಪ್ರೀತಿ-ಪ್ರಾಮಾಣಿಕತೆಗಳಿಗೆ ಹೆಸರಾದ ಪ್ರಾಣಿ ನಾಯಿ. ಆದ್ದರಿಂದ, ಮುಂದೆ ತಾವು ಎಂತಹ ಚೈತನ್ಯಪೂರ್ಣವಾದ ಸಂದೇಶಗಳನ್ನು, ಆದರ್ಶಗಳನ್ನು ಈ ಸಮಾಜಕ್ಕೆ ನೀಡ ಲಿಕ್ಕಿದೆಯೆಂದರೆ, ಅದರಿಂದ ಈ ಸಮಾಜ ಸಂಪೂರ್ಣ ಪ್ರಭಾವಿತವಾಗಿ ತಮ್ಮನ್ನು ಪ್ರಾಮಾಣಿಕ ವಾಗಿ ಅನುಸರಿಸುವಂತಾಗುತ್ತದೆ ಎಂಬುದು ಸ್ವಾಮೀಜೀಯ ಮಾತಿನ ತಾತ್ಪರ್ಯ. ಈ ಮಾತನ್ನು ಹೇಳುವಾಗ ಅವರು ಕೇವಲ ಒಬ್ಬ ಸಾಧಾರಣ, ನಿರ್ಗತಿಕ ಸಂನ್ಯಾಸಿ. ಹೀಗಿರುವಾಗ, ಅವರಿಗೆ ತಮ್ಮ ಆತ್ಮಶಕ್ತಿಯ ಮೇಲೆ ಎಂತಹ ದೃಢವಿಶ್ವಾಸವಿತ್ತು ಎಂಬುದರ ಅರಿವು ನಮಗಾಗುತ್ತದೆ.

ಆದರೆ ಸ್ವಾಮೀಜಿಯ ಈ ಮಾತಿನಲ್ಲಿ ಆಕ್ರೋಶವೂ ತುಂಬಿರುವುದನ್ನು ಗಮನಿಸಬಹುದು. ಇದಕ್ಕೆ ಅವರ ಹಾಗೂ ಪ್ರಮದದಾಸ ಮಿತ್ರರ ನಡುವೆ ಅನೇಕ ವಿಚಾರಗಳಲ್ಲಿ ಉಂಟಾಗಿದ್ದ ಅಭಿಪ್ರಾಯಭೇದವೇ ಮುಖ್ಯ ಕಾರಣ. ಶ್ರೀರಾಮಕೃಷ್ಣರಿಗಾಗಿ ಸ್ಮಾರಕವೊಂದನ್ನು ನಿರ್ಮಿಸುವ ವಿಚಾರದಲ್ಲಿ ಪ್ರಮದಬಾಬುಗಳು ಹೇಗೆ ಸ್ವಾಮೀಜಿಯ ಉತ್ಸಾಹಕ್ಕೆ ತಣ್ಣೀರೆರಚುವಂತೆ ಉತ್ತರಿಸಿದರು ಎಂಬುದನ್ನು ನೋಡಿದ್ದೇವೆ. ಇದಲ್ಲದೆ, ಅವರು ಶ್ರೀರಾಮಕೃಷ್ಣರ ಜೀವನ- ಬೋಧನೆಗಳನ್ನು ಅರಿತುಕೊಳ್ಳುವ ವಿಚಾರದಲ್ಲಿ ಆಸಕ್ತಿ ತೋರದೆ, ವೇದಾಂತಿಯಾದ ವಿವೇಕಾ ನಂದರು ಶ್ರೀರಾಮಕೃಷ್ಣರನ್ನು ಅಷ್ಟೊಂದಾಗಿ ಕೊಂಡಾಡುವ ಅಗತ್ಯವೇ ಇಲ್ಲ ಎಂಬಂತೆ ಮಾತ ನಾಡಿದ್ದರು. ಮತ್ತು ಸ್ವಾಮೀಜಿ ತಮಗೆ ಸಹಜವಾದ ಸಂನ್ಯಾಸಧರ್ಮಕ್ಕನುಸಾರವಾಗಿ ‘ವಿರಾಗಿ’ ಯಂತೆ ಜೀವಿಸದೆ, “ಮಠ--ಗುರುಭಾಯಿಗಳು ಇವರನ್ನೆಲ್ಲ ಕಟ್ಟಿಕೊಂಡು ಒದ್ದಾಡುತ್ತಿರುವುದ ಕ್ಕಾಗಿ” ಆಕ್ಷೇಪಿಸಿದ್ದರು. ಅಲ್ಲದೆ, ಪ್ರಮದದಾಸ ಮಿತ್ರರು ಮಹಾಸಂಪ್ರದಾಯಸ್ಥರು, ಮಡಿ ವಂತರು, ಆಹಾರಾದಿಗಳ ವಿಷಯದಲ್ಲಿ ಸ್ವಾಮೀಜಿಯ ಹಗುರ ಧೋರಣೆ ಅವರಿಗೆ ಸ್ವಲ್ಪವೂ ಹಿಡಿಸಿರಲಿಲ್ಲ. ಇದೆಲ್ಲದರಿಂದಾಗಿ ಅವರ ನಡುವೆ ಭಿನ್ನಾಭಿಪ್ರಾಯದ ಕಂದಕವುಂಟಾಗತೊಡ ಗಿತ್ತು. ಆದ್ದರಿಂದಲೇ ಕಡೆಗೆ ಸ್ವಾಮೀಜಿ ರೋಸಿಹೋಗಿ ಹಾಗೆ ಉದ್ಗರಿಸಿದುದು.

ಕಾಶಿಯಿಂದ ನೇರವಾಗಿ ಹಿಮಾಲಯದ ಕಡೆಗೆ ನಡೆದುಬಿಡಬೇಕು ಎಂದು ಸ್ವಾಮೀಜಿ ಕಾತರರಾಗಿದ್ದರು. ಆದರೆ ಅಖಂಡಾನಂದರು ಅವರಿಗೆ ಜಾನಕೀವರ ಶರಣರೆಂಬ ಸತ್ಪುರುಷರ ಸಂದರ್ಶನ ಮಾಡಿಸಲು ಅವರನ್ನು ಒತ್ತಾಯದಿಂದ ಅಯೋಧ್ಯೆಗೆ ಕರೆದೊಯ್ದರು. ಜಾನಕೀವರ ಶರಣರು ಅಯೋಧ್ಯೆಯ ಶ್ರೀರಾಮಮಂದಿರದ ಮಹಂತರು. ಅವರು ಒಬ್ಬ ಮಹಾತ್ಯಾಗಿ ಹಾಗೂ ಮಹಾಭಕ್ತರೆಂದು ಹೆಸರಾಗಿದ್ದವರು; ಅಲ್ಲದೆ ಸಂಸ್ಕೃತ ಹಾಗೂ ಪಾರಸೀ ಭಾಷೆಗಳಲ್ಲಿ ಪಂಡಿತರು. ಇಬ್ಬರು ಸಂನ್ಯಾಸಿಗಳನ್ನೂ ಅವರು ತಮ್ಮ ಆಶ್ರಮಕ್ಕೆ ಆದರದಿಂದ ಬರಮಾಡಿ ಕೊಂಡರು. ಭಕ್ತಿಯ ವಿಚಾರವಾಗಿ ದೀರ್ಘಕಾಲ ಸಂಭಾಷಣೆ ನಡೆಯಿತು. ಸ್ವಾಮೀಜಿಗೆ ಇವರ ಅಪೂರ್ವ ಜ್ಞಾನ-~~-ಭಕ್ತಿ--ವೈರಾಗ್ಯಗಳನ್ನು ಕಂಡು ಬಹಳ ಸಂತೋಷವಾಯಿತು. ತಾವು ಅಯೋಧ್ಯೆಗೆ ಬಂದದ್ದು ಸಾರ್ಥಕವಾಯಿತು ಎಂದು ಅವರು ಅಖಂಡಾನಂದರ ಮುಂದೆ ಹೇಳಿದರು.

ಕೆಲವು ದಿನ ಜಾನಕೀವರ ಶರಣರ ಆತಿಥ್ಯವನ್ನು ಸವಿದ ಸಂನ್ಯಾಸಿಗಳಿಬ್ಬರೂ ಅಯೋಧ್ಯೆ ಯಿಂದ ಹೊರಟು, ಬೆಟ್ಟಗುಡ್ಡಗಳ ದಾರಿಯಾಗಿ ನಡೆಯುತ್ತ ನೈನಿತಾಲಿಗೆ ಬಂದು ಸೇರಿದರು. ಅಲ್ಲಿ ರಾಮಪ್ರಸನ್ನ ಭಟ್ಟಾಚಾರ್ಯ ಎಂಬವರ ಅತಿಥಿಗಳಾಗಿ ಆರು ದಿನಗಳನ್ನು ಕಳೆದು, ಬದರಿಕಾಶ್ರಮದ ದಾರಿಯಲ್ಲಿರುವ ಆಲ್ಮೋರಕ್ಕೆ ಹೊರಟರು. ಇಬ್ಬರೂ ಕೈಯಲ್ಲಿ ಒಂದು ಕಾಸನ್ನೂ ಇಟ್ಟುಕೊಂಡಿಲ್ಲ; ಕಾಲ್ನಡಿಗೆಯಲ್ಲೇ ಪ್ರಯಾಣ. ದಾರಿಯಲ್ಲಿ ಭಗವಂತ ಏನನ್ನು ದೊರಕಿಸಿಕೊಡುತ್ತಾನೋ ಅದನ್ನೇ ಮಹಾಪ್ರಸಾದವೆಂದು ಸ್ವೀಕರಿಸುವುದು ಎಂದು ತೀರ್ಮಾನಿಸಿ ದ್ದರು. ಹೀಗೆ ದಟ್ಟವಾದ ಕಾಡಿನಲ್ಲಿ ಕಾಲುದಾರಿಯಾಗಿ ಸಾಗಿ, ಮೂರು ದಿನಗಳ ಮೇಲೆ ಕಾಕ್ರಿ ಘಾಟ್ ಎಂಬ ಸ್ಥಳವನ್ನು ತಲುಪಿದರು. ಇಲ್ಲಿಂದ ಆಲ್ಮೋರಕ್ಕೆ ಸುಮಾರು ಹದಿನೈದು ಮೈಲಿ, ಇಲ್ಲಿ ಕೋಶಿ ಹಾಗೂ ಸುಯಲ್ ಎಂಬ ಎರಡು ನದಿಗಳು ಸಂಧಿಸುತ್ತವೆ. ಅತ್ಯಂತ ರಮಣೀಯ ವಾದ ಸ್ಥಳ. ರಾತ್ರಿಯನ್ನುಇಲ್ಲೇ ಕಳೆಯಲು ನಿರ್ಧರಿಸಿದರು. ಇಲ್ಲಿನ ಪ್ರಶಾಂತ ವಾತಾವರಣವನ್ನು ಕಂಡು ಮುದಗೊಂಡ ಸ್ವಾಮೀಜೀ ಅಖಂಡಾನಂದರಿಗೆ ಹೇಳುತ್ತಾರೆ: “ಈ ಸ್ಥಳ ನಿಜಕ್ಕೂ ತುಂಬ ಮನೋಹರವಾಗಿದೆ. ಧ್ಯಾನಕ್ಕೆ ಇದು ಎಷ್ಟು ಪ್ರಶಸ್ತವಾಗಿದೆ, ನೋಡು!” ಹೀಗೆಂದ ಅವರು ಸಂಗಮದಲ್ಲಿ ಮಿಂದು ಬಂದು, ಒಂದು ಅಶ್ವತ್ಥ ವೃಕ್ಷದ ಕೆಳಗೆ ಧ್ಯಾನಕ್ಕೆ ಕುಳಿತರು. ಶರೀರ ಪ್ರಜ್ಞೆಯೇ ಇಲ್ಲದಂತಹ ಗಾಢಧ್ಯಾನದಲ್ಲಿ ಮುಳುಗಿಬಿಟ್ಟರು. ಬಹಳ ಹೊತ್ತಿನ ನಂತರ ಮೈ ತಿಳಿದೆದ್ದ ಸ್ವಾಮೀಜಿ ಅಖಂಡಾನಂದರೆದುರು ತಮಗಾದ ದಿವ್ಯಾನುಭವವೊಂದನ್ನು ಹೊರಗೆಡಹಿ ದರು: “ಗಂಗಾಧರ! ನಾನಿಂದು ಇಲ್ಲಿ ಕಳೆದ ಈ ಮುಹೂರ್ತವು ನನ್ನ ಜೀವನದಲ್ಲೇ ಅತ್ಯಂತ ಮಹತ್ವಪೂರ್ಣವಾದ ಮುಹೂರ್ತಗಳಲ್ಲೊಂದು. ಇವತ್ತು ಈ ಅಶ್ವತ್ಥವೃಕ್ಷದ ಕೆಳಗೆ ನನ್ನ ಜೀವನದ ದೊಡ್ಡ ಸಮಸ್ಯೆಯೊಂದು ಬಗೆಹರಿಯಿತು. ಬ್ರಹ್ಮಾಂಡ ಹಾಗೂ ಪಿಂಡಾಂಡಗಳ ಏಕತೆಯನ್ನು ನಾನಿಂದು ಕಂಡುಕೊಂಡೆ. ಸಮಸ್ತ ಬ್ರಹ್ಮಾಂಡದಲ್ಲಿ ಏನೇನಿದೆಯೋ. ಅದೆಲ್ಲವೂ ಪಿಂಡಾಂಡವಾದ ಈ ಶರೀರದಲ್ಲಿದೆ. ನಾನು ಒಂದು ಪರಮಾಣುವಿನಲ್ಲಿ ಸಕಲ ಬ್ರಹ್ಮಾಂಡವೇ ಅಡಗಿರುವುದನ್ನು ಕಂಡೆ.” ಅದ್ಭುತ! ಭಾರತದ ಪರಮಪುಷಿಗಳು ಹಿಂದೆ ಉಪನಿಷತ್ತುಗಳಲ್ಲಿ ಯಾವ ಘನಸತ್ಯವನ್ನು ಘೋಷಿಸಿದ್ದರೋ, ಅದನ್ನು ಇಂದು ಸ್ವಾಮೀಜಿ ತಮ್ಮ ಧ್ಯಾನಾವಸ್ಥೆ ಯಲ್ಲಿ ಸಾಕ್ಷಾತ್ತಾಗಿ ಕಾಣುತ್ತಿದ್ದಾರೆ! ವೇದವಾಕ್ಯಗಳ ಸತ್ಯತೆಯನ್ನು ಸ್ಪಷ್ಟವಾಗಿ ಕಂಡುಕೊಳ್ಳುತ್ತಿ ದ್ದಾರೆ! ಆ ದಿನವಿಡೀ ಸ್ವಾಮೀಜಿ ಅದೇ ಉನ್ನತ ಮನಸ್ಥಿತಿಯಲ್ಲೇ ಇದ್ದರು. ತಮಗಾದ ಆ ದಿವ್ಯ ಸಾಕ್ಷಾತ್ಕಾರದ ವಿಷಯವಾಗಿಯೇ ತಮ್ಮ ಸಹಚರನೊಂದಿಗೆ ಮಾತನಾಡುತ್ತಿದ್ದರು. ಈ ಅನು ಭವದ ಕುರಿತಾದ ಕೆಲವು ತುಣುಕುಗಳನ್ನು ಅವರ ಟಿಪ್ಪಣಿ ಪುಸ್ತಕದಲ್ಲಿ ಕಾಣಬಹುದಾಗಿತ್ತು.

ಅವರಿಗಾದ ಆ ಅನುಭವದ ಒಂದು ಕಲ್ಪನೆಯನ್ನು, ಮುಂದೆ ಅವರು ಅಮೆರಿಕದಲ್ಲಿ ನೀಡಿದ ಅ“ಸಮಗ್ರ ವಿಶ್ವ–ಬ್ರಹ್ಮಾಂಡ ಹಾಗೂ ಪಿಂಡಾಂಡ” ಎಂಬ ಉಪನ್ಯಾಸದಲ್ಲಿ ಕಾಣಬಹುದಾಗಿದೆ.

ಮರುದಿನ ಇಬ್ಬರೂ ಕಾಕ್ರಿ ಘಾಟ್ ನಿಂದ ಮುಂದಕ್ಕೆ ಹೊರಟರು. ಮತ್ತೆ ಕಾಲ್ನಡಿಗೆಯಲ್ಲೇ ಪ್ರಯಾಣ. ಸುಮಾರು ೧೨ ಮೈಲಿ ನಡೆದಿರಬಹುದು. ಅಲ್ಲಿಂದ ಆಲ್ಮೋರಕ್ಕೆ ಎರಡೇ ಮೈಲಿ, ಆದರೆ ಇಬ್ಬರೂ ವಿಪರೀತ ದಣಿದಿದ್ದಾರೆ. ಹೊಟ್ಟೆಯಂತೂ ಹಸಿವಿನಿಂದ ಚುರುಗುಟ್ಟುತ್ತಿದೆ. ಇನ್ನು ಒಂದು ಹೆಜ್ಜೆಯನ್ನೂ ಇಡಲು ಸಾಧ್ಯವಾಗುವಂತಿರಲಿಲ್ಲ. ಆದ್ದರಿಂದ ಇಬ್ಬರೂ ಅಲ್ಲೇ ರಸ್ತೆ ಬದಿಯಲ್ಲಿ ಒಂದು ಕಡೆ ಕುಳಿತುಬಿಟ್ಟರು. ಸ್ವಾಮೀಜಿ ಬಳಲಿಕೆಯನ್ನು ತಾಳಲಾರದೆ ಕುಸಿದು ನೆಲದ ಮೇಲೆ ಒರಗಿಕೊಂಡರು. ಅಖಂಡಾನಂದರು ಕುಡಿಯಲು ನೀರನ್ನಾದರೂ ತರೋಣ ವೆಂದು ಹೊರಟರು.

ಅಲ್ಲೇ ರಸ್ತೆಯ ಬದಿಯಲ್ಲೊಂದು ಮುಸಲ್ಮಾನರ ಸ್ಮಶಾನವಿತ್ತು. ಜುಲ್ಫಿಕರ್ ಆಲಿ ಎಂಬ ಒಬ್ಬ ಮುಸಲ್ಮಾನ ಫಕೀರ ಇದರ ಕಾವಲುಗಾರನಾಗಿದ್ದ. ಈತ ಸ್ವಾಮೀಜಿಯ ಅವಸ್ಥೆಯನ್ನು ಕಂಡು ತಕ್ಷಣ ತನ್ನ ಗುಡಿಸಲಿಗೆ ಹೋಗಿ ಬಂದು ಸೌತೆಕಾಯನ್ನು ತಂದುಕೊಟ್ಟ. ಪಾಪ, ಆ ಬಡ ಫಕೀರನ ಬಳಿ ಕೊಡಲು ಇದ್ದದ್ದು ಅದೊಂದೇ. ಆದರೆ ಸ್ವಾಮೀಜಿ ಎಷ್ಟು ನಿತ್ರಾಣರಾಗಿದ್ದರೆಂದರೆ ಅದನ್ನು ಕೊಯ್ದು ಬಾಯಿಗೆ ಹಾಕಿಕೊಳ್ಳಲೂ ಅವರಿಗೆ ಸಾಧ್ಯವಿರಲಿಲ್ಲ. ಆದ್ದರಿಂದ, “ನನಗೆ ಎದ್ದು ತಿನ್ನಲೂ ಶಕ್ತಿಯಿಲ್ಲ. ನೀನೇ ಅದನ್ನು ಸ್ವಲ್ಪ ನನ್ನ ಬಾಯಿಗೆ ಹಾಕು” ಎಂದರು. ಆಗ ಆ ಫಕೀರ “ಸ್ವಾಮಿ, ಅದು ಹೇಗಾದೀತು! ನಾನೊಬ್ಬ ಮುಸಲ್ಮಾನ!” ಎಂದು ಪ್ರತಿಭಟಿಸಿದ. ಸ್ವಾಮೀಜಿ ಮುಗುಳ್ನಗುವೊಂದನ್ನು ತಂದುಕೊಂಡು ನುಡಿದರು, “ಅದೇನು ಪರವಾಗಿಲ್ಲ. ನಾವೆಲ್ಲ ಅಣ್ಣತಮ್ಮಂದಿರೇ ಅಲ್ಲವೆ?” ಅವರ ಒತ್ತಾಯದ ಮೇರೆಗೆ ಫಕೀರ ಸೌತೆಕಾಯಿಯನ್ನು ತಾನೇ ತಿನ್ನಿಸಿದ. ಬಳಿಕ ಸ್ವಾಮೀಜಿ ಸ್ವಲ್ಪ ಚೇತರಿಸಿಕೊಂಡರು. ಈ ಘಟನೆಯನ್ನು ನೆನಪಿಸಿಕೊಂಡು ಅವರು ಮುಂದೆ ಆಗಾಗ ಹೇಳುತ್ತಿದ್ದರು, “ನನಗೆ ಅಷ್ಟೊಂದು ಆಯಾಸ ಹಿಂದೆಂದೂ ಆಗಿರಲಿಲ್ಲ. ಆತ ನಿಜಕ್ಕೂ ನನಗೆ ಜೀವದಾನ ಮಾಡಿದ ” ಎಂದು.

ಇದಾದ ಏಳು ವರ್ಷಗಳ ಬಳಿಕ, ಜಗದ್ವಿಖ್ಯಾತ ವಿವೇಕಾನಂದರಾಗಿ ಭಾರತಕ್ಕೆ ಮರಳಿದ ಮೇಲೆ ಸ್ವಾಮೀಜಿ, ಒಮ್ಮೆ ಆಲ್ಮೋರಕ್ಕೆ ಭೇಟಿ ನೀಡಿದರು. ಆಗ ಅವರಿಗೆ ಅಲ್ಲಿನ ನಾಗರಿಕರು ಭವ್ಯ ಸ್ವಾಗತ ನೀಡಿ, ಅವರನ್ನು ಊರ ಸುತ್ತ ಮೆರವಣಿಗೆಯಲ್ಲಿ ಕರೆದೊಯ್ದರು. ಸಮಾರಂಭ ಭವನದ ಮೂಲೆಯಲ್ಲಿ ಜನರೊಂದಿಗೆ ನಿಂತು ಆ ಫಕೀರ ಕುತೂಹಲದಿಂದ ನೋಡುತ್ತಿರುವುದು ಸ್ವಾಮೀಜಿಯ ಕಣ್ಣಿಗೆ ಬಿತ್ತು. ಆ ಫಕೀರ ಅವರನ್ನು ಮರೆತುಬಿಟ್ಟಿದ್ದಾನೆ. ಆದರೆ ಸ್ವಾಮೀಜಿ ಮರೆತಿಲ್ಲ; ಅಷ್ಟೇ ಅಲ್ಲ, ಅವನನ್ನು ಆ ಗುಂಪಿನಲ್ಲೂ ಗುರುತುಹಿಡಿದಿದ್ದಾರೆ! ಒಡನೆಯೇ ಅವರು ತಮ್ಮ ಸುತ್ತ ಇದ್ದವರಿಗೆಲ್ಲ ಆ ಫಕೀರನನ್ನು ತೋರಿಸಿ, ತಮ್ಮ ಜೀವವನ್ನೇ ಉಳಿಸಿದವನು ಅವನು ಎಂದು ಅಂದಿನ ಆ ಘಟನೆಯನ್ನು ವಿವರಿಸಿದರು. ಅಲ್ಲದೆ ಅವನ ಉಪಕಾರವನ್ನು ಸ್ಮರಿಸಿ ಕೃತಜ್ಞತೆಯ ಕುರುಹಾಗಿ ಅವನಿಗೆ ಒಂದಿಷ್ಟು ಹಣವನ್ನು ಕೊಡಿಸಿದರು. ಆ ಫಕೀರ ಸ್ವಾಮೀಜಿಗೆ ಸೌತೆಕಾಯಿಯನ್ನು ತಿನ್ನಿಸಿದ ಸ್ಥಳದಲ್ಲೇ ಈಚೆಗೆ (೧೯೭೧ರಲ್ಲಿ) ಭಕ್ತರೊಬ್ಬರು ಸ್ವಾಮಿ ವಿವೇಕಾ ನಂದರ ಹೆಸರಿನಲ್ಲಿ ಸತ್ರವೊಂದನ್ನು ಕಟ್ಟಿಸಿದ್ದಾರೆ.

ಬಯಲು ಪ್ರದೇಶದಿಂದ ಆಲ್ಮೋರದ ಪರ್ವತ ಪ್ರದೇಶದತ್ತ ಕೈಗೊಂಡ ದೀರ್ಘ ಪ್ರಯಾಣ ಸ್ವಾಮಿಗಳ ಪಾಲಿಗೆ ಅತ್ಯಂತ ಉಲ್ಲಾಸಕರವಾಗಿತ್ತು. ಆ ಪರ್ವತ ಪ್ರದೇಶಗಳ ಪ್ರಶಾಂತ- ಮನೋಹರ ವಾತಾವರಣದಲ್ಲಿ ಸ್ವಾಮೀಜಿಯ ಹೃನ್ಮನಗಳು ಅಪೂರ್ವ ಶಾಂತಿಯನ್ನು ಅನು ಭವಿಸಿದುವು. ಅಲ್ಲಿನ ಗಾಳಿ ಅವರಲ್ಲಿ ನವಚೇತನವನ್ನುಂಟುಮಾಡಿತು. ಈ ಸುದೀರ್ಘ ಕಾಲ್ನಡಿಗೆಯ ಪ್ರಯಾಣದಿಂದ ಹಾಗೂ ಕಾಲಕಾಲಕ್ಕೆ ಅನ್ನಾಹಾರಗಳಿಲ್ಲದ್ದರಿಂದ ದೇಹಕ್ಕೆ ಸಾಕಷ್ಟು ಬಳಲಿಕೆಯಾಗಿದ್ದರೂ, ಅವರ ಪಾಲಿಗೆ ಈ ಪರ್ವತಪ್ರಯಾಣವು ಆನಂದದ ಪರಾಕಾಷ್ಠೆಯೇ ಆಗಿದ್ದಿತು.

ಹೀಗೆ ಸುದೀರ್ಘ ಪ್ರಯಾಣ ಮಾಡಿ, ೧೮೯೦ರ ಆಗಸ್ಟ್ ತಿಂಗಳ ಕೊನೆಯಲ್ಲಿ ಇಬ್ಬರೂ ಆಲ್ಮೋರಕ್ಕೆ ಬಂದು ತಲುಪಿದರು. ಅಖಂಡಾನಂದರಿಗೆ ಇದೆಲ್ಲ ಸುಪರಿಚಿತ ಸ್ಥಳ. ಅವರು ಸ್ವಾಮೀಜಿಯನ್ನು ಅಂಬಾ ದತ್ತ ಎಂಬವನ ಮನೆಗೆ ಕರೆದುಕೊಂಡು ಹೋದರು. ಆ ಸಮಯದಲ್ಲಿ ಸ್ವಾಮಿ ಶಾರದಾನಂದರೂ ಸ್ವಾಮಿ ಕೃಪಾನಂದರೂ (ಶ್ರೀರಾಮಕೃಷ್ಣರ ಯುವಶಿಷ್ಯರಲ್ಲೊಬ್ಬನಾದ ವೈಕುಂಠನಾಥ ಸನ್ಯಾಲ) ಆಲ್ಮೋರದಲ್ಲೇ ಲಾಲಾ ಬದರೀ ಸಾಹ ಎಂಬವರ ಅತಿಥಿಗಳಾಗಿದ್ದು, ಸ್ವಾಮೀಜಿಯ ಬರವನ್ನೇ ಕಾಯುತ್ತಿದ್ದರು. ಸ್ವಾಮೀಜಿ ಬಂದಿದ್ದಾರೆಂಬ ಸಂತಸದ ಸುದ್ದಿ ಕೇಳಿದೊಡನೆಯೇ ಇಬ್ಬರೂ ತಮ್ಮ ಆತಿಥೇಯರಾದ ಬದರಿ ಸಾಹರನ್ನೂ ಕರೆದುಕೊಂಡು ಅವರನ್ನು ನೋಡಲು ಧಾವಿಸಿದರು. ಆದರೆ ಇಷ್ಟು ಹೊತ್ತಿಗಾಗಲೇ ಸ್ವಾಮೀಜಿ ಇವರನ್ನು ನೋಡಲು ಅರ್ಧದಾರಿ ಬಂದುಬಿಟ್ಟಿದ್ದರು. ನಡುದಾರಿಯಲ್ಲಿ ಪರಸ್ಪರ ಸಂಧಿಸಿದರು–ಆನಂದ ಸಮ್ಮಿಲನ! ಬದರೀ ಸಾಹರು ಸ್ವಾಮೀಜಿಯನ್ನು ಸ್ವಾಗತಿಸಿ ತಮ್ಮ ಮನೆಗೆ ಕರೆತಂದರು.

ಇಲ್ಲಿ ಸ್ವಾಮೀಜಿ ಆ ಊರಿನ ಶಿರಸ್ತೇದಾರನಾದ ಶ್ರೀಕೃಷ್ಣ ಜೋಶಿ ಎಂಬವನನ್ನು ಭೇಟಿಮಾಡಿ ದರು. ಮಾತಿನ ಮಧ್ಯೆ ಈತ, “ಸ್ವಾಮೀಜಿ, ಗೃಹಸ್ಥರಾಗಿದ್ದುಕೊಂಡೇ ಧರ್ಮಪರಿಪಾಲನೆ ಮಾಡಲು ಸಾಧ್ಯವಿರುವಾಗ ಸಂನ್ಯಾಸವನ್ನು ತೆಗೆದುಕೊಳ್ಳಬೇಕಾದ ಆವಶ್ಯಕತೆಯಾದರೂ ಏನು?” ಎಂದು ಕೇಳಿದ. ಆಗ ಅವರು ಸಂನ್ಯಾಸದ ಮಹಿಮೆಯನ್ನು, ತ್ಯಾಗಜೀವನದ ಘನತೆಯನ್ನು ಎತ್ತಿಹಿಡಿಯುತ್ತ, ತುಂಬ ಪ್ರಭಾವಶಾಲಿಯಾದ ವಾಕ್ಯಗಳಿಂದ ವರ್ಣಿಸಿದರು. ಸರ್ವಧರ್ಮಗಳ ಪರಮಾದರ್ಶವೇ ತ್ಯಾಗ ಎಂಬುದನ್ನು ಮನಮುಟ್ಟುವಂತೆ ತಿಳಿಸಿದರು. ಸ್ವಾನುಭವದ ಆಧಾರದ ಮೇಲೆ ಸ್ವಾಮೀಜಿ ಈ ವಿಷಯವನ್ನು ವಿವರಿಸಿದಾಗ ಅವರ ಅಸಾಮಾನ್ಯ ಜ್ಞಾನವನ್ನು ಮನಗಂಡ ಶ್ರೀಕೃಷ್ಣ ಜೋಶಿ, ಅವರಿಗೆ ಸಾಷ್ಟಾಂಗ ಪ್ರಣಾಮಮಾಡಿ ಬೀಳ್ಗೊಂಡ.

ಸ್ವಾಮೀಜಿ ಹಿಮಾಲಯಕ್ಕೆ ಬಂದದ್ದರ ಉದ್ದೇಶ ತೀವ್ರ ಆಧ್ಯಾತ್ಮಿಕ ಸಾಧನೆಯಲ್ಲಿ ಮುಳುಗಿ ಬಿಡಬೇಕು ಎನ್ನುವುದು. ಇದು ಅವರ ಬಹುವರ್ಷಗಳ ಹಂಬಲ. ಆದ್ದರಿಂದ ಬದರೀ ಸಾಹರ ಮನೆಯಲ್ಲಿ ಗುರುಭಾಯಿಗಳೊಂದಿಗೆ ಕೆಲವು ದಿನಗಳನ್ನು ಕಳೆದ ಬಳಿಕ, ಅವರು ಏಕಾಂಗಿಯಾಗಿ ಹೊರಟುಬಿಟ್ಟರು. ಸಮೀಪದ ಪರ್ವತಶಿಖರವೊಂದರಲ್ಲಿ ಒಂದು ಗುಹೆಯನ್ನು ಕಂಡುಕೊಂಡು, ಅಲ್ಲಿ ಧ್ಯಾನಮಗ್ನರಾಗಿರತೊಡಗಿದರು. ಹಗಲಿರುಳುಗಳು ಗಾಢ ಧ್ಯಾನದಲ್ಲಿ ಕಳೆದುವು. ಗುಹೆಯ ಒಳಗೂ ಪರಮಶಾಂತಿ, ಹೊರಗೂ ನೀರವ ಮೌನ. ಅತ್ಯುನ್ನತ ಸಾಕ್ಷಾತ್ಕಾರವನ್ನು ಸಾಧಿಸಿಕೊಳ್ಳಲೇ ಬೇಕು ಎಂದು ದೃಢನಿರ್ಧಾರ ಮಾಡಿಬಿಟ್ಟಿದ್ದಾರೆ ಸ್ವಾಮೀಜಿ. ಅವರ ಧ್ಯಾನ ಬರಬರುತ್ತ ಹೆಚ್ಚು ತೀವ್ರವಾಗುತ್ತಿದೆ, ಹೆಚ್ಚು ಉಗ್ರವಾಗುತ್ತಿದೆ. ಜೊತೆಜೊತೆಗೇ ಅನೇಕ ಉನ್ನತ ಅಧ್ಯಾತ್ಮಾನುಭ ಗಳಾಗುತ್ತಿವೆ. ಹೃದಯಾಂತರಾಳವೆಲ್ಲ ಆಧ್ಯಾತ್ಮಜ್ಯೋತಿಯಿಂದ ಬೆಳಗುತ್ತಿದೆ. ತತ್ಫಲವಾಗಿ ಅವರ ಮುಖಮಂಡಲವೂ ದೈವೀತೇಜಸ್ಸಿನಿಂದ ಕಂಗೊಳಿಸಲಾರಂಭಿಸಿದೆ. ಇಂತಹ ಸ್ಥಿತಿಯಲ್ಲಿ, ತಾವು ಹಿಂದೊಮ್ಮೆ ಶ್ರೀರಾಮಕೃಷ್ಣರ ಕೃಪೆಯಿಂದ ಸವಿದಿದ್ದ ನಿರ್ವಿಕಲ್ಪ ಸಮಾಧಿಯನ್ನು ಮತ್ತೆ ಹೊಂದುವ ಪ್ರಯತ್ನದಲ್ಲಿ ತೊಡಗಿದರು ಸ್ವಾಮೀಜಿ. ಆದರೆ ಏನು ಮಾಡಿದರೂ ಅವರ ಮನಸ್ಸು ಆ ಸ್ತರಕ್ಕೇರುತ್ತಿಲ್ಲ; ಅದರ ಬದಲಾಗಿ ಕರ್ಮಕ್ಷೇತ್ರಕ್ಕಿಳಿಯುವಂತೆ ಪ್ರೇರಣೆಯಾಗುತ್ತಿದೆ! ಯಾವುದೋ ಪ್ರಚಂಡ ಶಕ್ತಿ ಅವರನ್ನು ಆಧ್ಯಾತ್ಮ ಸಾಧನಾಪಥದಿಂದ ಹೊರಕ್ಕೆಳೆಯುತ್ತಿದೆ! ಸ್ವಾಮೀಜಿ ಮತ್ತೆ ಮತ್ತೆ ಪ್ರಯತ್ನಮಾಡಿದರು; ಆದರೆ ಅದೇ ಅವ್ಯಕ್ತ ಶಕ್ತಿ ಅವರನ್ನು ಕರ್ಮ ಕ್ಷೇತ್ರದೆಡೆಗೆ ದೂಡುತ್ತಿತ್ತು. ಇದೆಂಥ ವಿಚಿತ್ರ ಸ್ಥಿತಿ! ಹಾಗಾದರೆ, ಅವರಿಗೆ ಆ ಅತ್ಯುನ್ನತ ಸ್ಥಿತಿ ಗೇರುವ ಸಾಮರ್ಥ್ಯ ಇನ್ನೂ ಬಂದಿರಲಿಲ್ಲವೆಂದೆ? ಅಲ್ಲ! ಅವರ ಈ ಸ್ಥಿತಿಯ ಕುರಿತಾಗಿ ಸ್ವಾಮಿ ಅಖಂಡಾನಂದರು ಮುಂದೊಮ್ಮೆ ಹೇಳುತ್ತಾರೆ: “ಶುದ್ಧಸಂನ್ಯಾಸದ ಆದರ್ಶಕ್ಕನುಗುಣವಾಗಿ ತಾವು ಪರಿಪೂರ್ಣ ಶಾಂತಿಯಲ್ಲಿ ಮುಳುಗಿ ತಲ್ಲೀನರಾಗಲು ಅವರು ಎಷ್ಟೇ ಪ್ರಯತ್ನಪಟ್ಟರೂ, ಪ್ರತಿ ಬಾರಿಯೂ ಯಾವುದೋ ಒಂದು ಪರಿಸ್ಥಿತಿಯ ಒತ್ತಡದಿಂದಾಗಿ ಅವರು ಆ ಪ್ರಯತ್ನವನ್ನೇ ಬಿಡಬೇಕಾಗಿ ಬರುತ್ತಿತ್ತು. ಏಕೆಂದರೆ, ಅವರು ಈ ಜಗತ್ತಿನಲ್ಲಿ ಸಾಧಿಸಬೇಕಾದ ಮಹಾಕಾರ್ಯ ವೊಂದಿತ್ತು. ಆ ಕಾರ್ಯವನ್ನು ಈಡೇರಿಸುವಂತೆ ಅವರ ಸ್ವಭಾವವೇ ಅವರನ್ನು ಬಲವಾಗಿ ಪ್ರೇರೇಪಿಸುತ್ತಿತ್ತು.”

ಈ ಸಂಬಂಧವಾಗಿ ಹಿಂದಿನ ಘಟನೆಗಳನ್ನೂ ಇಲ್ಲಿ ನೆನಪಿಸಿಕೊಳ್ಳಬಹುದು. ಹಿಂದೆ ಶ್ರೀರಾಮ ಕೃಷ್ಣರು ನರೇಂದ್ರನಿಗೆ ನಿರ್ವಿಕಲ್ಪ ಸಮಾಧಿಯ ಅನುಭವ ಮಾಡಿಸಿಕೊಟ್ಟ ಮೇಲೆ ಹೇಳಿದ್ದರು, “... ನಿನ್ನ ಈ ಅನುಭವವನ್ನು ಪೆಟ್ಟಿಗೆಯಲ್ಲಿಟ್ಟು ಬೀಗ ಹಾಕಲಾಗುತ್ತದೆ; ಮತ್ತು ಅದರ ಕೀಲಿಕೈ ನನ್ನ ಬಳಿಯಿರುತ್ತದೆ. ನೀನು ನಿನ್ನ ಪಾಲಿನ ಕಾರ್ಯವನ್ನು ಮಾಡಿ ಮುಗಿಸಿದ ಮೇಲಷ್ಟೇ ಆ ಬೀಗಮುದ್ರೆಯನ್ನು ತೆರೆಯಲಾಗುತ್ತದೆ” ಎಂದು. ಅಲ್ಲದೆ ತಮ್ಮ ಕಡೇ ದಿನಗಳಲ್ಲೊಮ್ಮೆ, ‘ನೀನು ಧರ್ಮಪ್ರಸಾರ ಮಾಡಬೇಕಾಗುತ್ತದೆ’ ಎಂದು ಒಂದು ಚೀಟಿಯಲ್ಲಿ ಬರೆದು ತೋರಿಸಿ ದ್ದರು. ‘ನಾನದನ್ನು ಮಾಡಲಾರೆ’ ಎಂದು ಅವನು ಪ್ರತಿಭಟಿಸಿದಾಗ ಶ್ರೀರಾಮಕೃಷ್ಣರು, ‘ನೀನು \textbf{ಮಾಡಲೇ}ಬೇಕಾಗುತ್ತದೆ’ ಎಂದಿದ್ದರು. ಆದ್ದರಿಂದ, ಈಗ ಸ್ವಾಮೀಜಿ ತಾವಾಗಿಯೇ ಪ್ರಯತ್ನ ಪಟ್ಟು ನಿರ್ವಿಕಲ್ಪ ಸಮಾಧಿಗೇರಿ ದಿವ್ಯಾನಂದವನ್ನು ಅನುಭವಿಸುತ್ತ ಕುಳಿತುಕೊಳ್ಳುವಂತಿರಲಿಲ್ಲ. ಶ್ರೀರಾಮಕೃಷ್ಣರು ತಮಗೊಪ್ಪಿಸಿದ ಕಾರ್ಯವನ್ನು ಅವರು ಸಾಧಿಸಬೇಕಾಗಿತ್ತು. ಆದ್ದರಿಂದಲೇ ಅವರು ಧ್ಯಾನದಲ್ಲಿ ಸಮಾಧಿಗೇರುವ ಪ್ರಯತ್ನ ಮಾಡಿದಾಗಲೆಲ್ಲ ಅಗೋಚರ ಶಕ್ತಿಯೊಂದು ಅವರನ್ನು ಕರ್ಮಕ್ಷೇತ್ರಕ್ಕೆ ದೂಡುತ್ತಿದ್ದುದು. ಶ್ರೀರಾಮಕೃಷ್ಣರು ಇನ್ನೂ ಆ ಪೆಟ್ಟಿಗೆಯ ಬೀಗ ಮುದ್ರೆಯನ್ನು ತೆಗೆಯಲೊಪ್ಪುತ್ತಿಲ್ಲ!

ಹೀಗೆ ಸ್ವಾಮೀಜಿ ಏಕಾಂತದಲ್ಲಿ ತೀವ್ರ ಸಾಧನೆಗಳನ್ನು ನಡೆಸಿ, ಕೆಲವು ದಿನಗಳ ಬಳಿಕ ಲಾಲಾ ಬದರೀ ಸಾಹರ ಮನೆಗೆ ಹಿಂದಿರುಗಿದರು. ಆಗ ಅವರಿಗೊಂದು ದಾರುಣ ಸುದ್ದಿ ಕಾದಿತ್ತು–ಅವರ ಒಬ್ಬಳು ಸೋದರಿ ಆತ್ಮಹತ್ಯೆ ಮಾಡಿಕೊಂಡಳು ಎಂಬ ತಂತಿ ವರ್ತಮಾನ ತಲುಪಿತ್ತು. ಈ ತಂತಿಯನ್ನು ಹಿಂಬಾಲಿಸಿ ಒಂದು ವಿವರಪೂರ್ಣವಾದ ಪತ್ರ ಬಂದಿತು. ಬಾಲ್ಯದಲ್ಲೇ ವಿವಾಹ ವಾಗಿ ಪತಿಗೃಹಕ್ಕೆ ಹೋಗಿದ್ದ ಆ ಸೋದರಿ, ಅಲ್ಲಿನ ಸಂಪ್ರದಾಯಸ್ಥ ಪರಿಸರಕ್ಕೆ ಹೊಂದಿಕೊಳ್ಳ ಲಾರದೆ ತೀವ್ರ ಕಷ್ಟಗಳನ್ನು ಎದುರಿಸಬೇಕಾಗಿತ್ತು. ಕಡೆಗೆ ಕಷ್ಟಗಳು ಅಸಹನೀಯವಾಗಿ ಆತ್ಮಹತ್ಯೆ ಮಾಡಿಕೊಂಡಿದ್ದಳು. ಸೋದರಿಯ ದುರದೃಷ್ಟಕರ ಸಾವು ಸ್ವಾಮೀಜಿಗೆ ಒಂದು ದೊಡ್ಡ ಆಘಾತ ವನ್ನುಂಟುಮಾಡಿತು. ಅಲ್ಲದೆ ಆಕೆ ಅವರಿಗೆ ವಿಶೇಷ ಪ್ರೀತಿಪಾತ್ರಳಾಗಿದ್ದವಳು. ಆದ್ದರಿಂದ ಈ ಘಟನೆ ಅವರ ಅಂತರಾಳವನ್ನೇ ಕಲಕಿಬಿಟ್ಟಿತು. ಅವರು ಸರ್ವಸಂಗ ಪರಿತ್ಯಾಗಿಯಾದ ಮಹಾನ್ ಸಂನ್ಯಾಸಿಯಿರಬಹುದು. ಆದರೆ ಒಡಹುಟ್ಟಿದ ಸೋದರಿಯೊಬ್ಬಳು ಸಮಾಜದ ದೌರ್ಜನ್ಯದಿಂದ ನೊಂದು, ಹೀಗೆ ಘೋರ ಮರಣಕ್ಕೀಡಾಗುವ ಪ್ರಸಂಗ ಒದಗಿದ್ದನ್ನು ಕಂಡಾಗ ಅವರ ಹೃದಯ ದಲ್ಲಿ ಭಾರತದ ಸಮಸ್ತ ಸ್ತ್ರೀಕುಲದ ಸಮಸ್ಯೆಯ ಕುರಿತಾದ ಪ್ರಜ್ಞೆಯೊಂದು ಜಾಗೃತವಾಯಿತು. ಮಹಿಳೆಯರ ವಿದ್ಯಾಭ್ಯಾಸ ಹಾಗೂ ಸರ್ವತೋಮುಖ ಉನ್ನತಿಯ ಬಗ್ಗೆ ಅವರಲ್ಲಿ ತೀವ್ರ ಕಾಳಜಿ ಯುಂಟಾಗಲು ಈ ಸಾವು ಕೆಲಮಟ್ಟಿಗಾದರೂ ಕಾರಣವಾಗಿರಬೇಕು. ಆದರೆ ತತ್ಕಾಲಕ್ಕೆ ಅವರು ಈ ಬಗ್ಗೆ ಏನೂ ಮಾಡುವಂತಿರಲಿಲ್ಲ.

ಬದರೀ ಸಾಹರು ಸ್ವಾಮೀಜಿಯನ್ನು ತುಂಬ ಆದರದಿಂದ ನೋಡಿಕೊಂಡರು. ಅವರ ದೈವಭಕ್ತಿ ಯನ್ನೂ ಆದರಪೂರ್ವಕ ಆತಿಥ್ಯವನ್ನೂ ಕಂಡು ಅತೀವ ಆನಂದಿತರಾದ ಸ್ವಾಮೀಜಿ, “ಇಂತಹ ಒಬ್ಬ ಭಕ್ತರನ್ನು ಕಾಣುವುದು ತೀರ ಅಪರೂಪ ” ಎಂದು ಉದ್ಗರಿಸಿದರು. ಇವರು ಸ್ವಾಮೀಜಿಯ ಒಬ್ಬ ಆಪ್ತ ಭಕ್ತರಾಗಿ ಉಳಿದುಕೊಂಡರು. ಹೀಗೆ ಆಲ್ಮೋರದಲ್ಲಿ ಕೆಲವು ದಿನಗಳನ್ನು ಕಳೆದ ಸ್ವಾಮೀಜಿ, ಇನ್ನಷ್ಟು ಉತ್ತರಕ್ಕೆ ಸಾಗಲು ಬಯಸಿದರು. ಧ್ಯಾನಮಗ್ನರಾಗಿ ಸಮಾಧಿಸ್ಥಿತಿಗೇರ ಬೇಕೆಂಬ ಉತ್ಕಟೇಚ್ಛೆ ಅವರನ್ನಿನ್ನೂ ಬಿಟ್ಟಿರಲಿಲ್ಲ. ಸೋದರಿಯ ಸಾವು ಅವರ ಮನಸ್ಸನ್ನು ಪ್ರಕ್ಷುಬ್ಧವಾಗಿಸಿತ್ತು; ಆದರೆ ಆಧ್ಯಾತ್ಮಿಕ ಸಾಧನೆಯ ಹಂಬಲವು ಈ ಪ್ರಕ್ಷುಬ್ಧತೆಯನ್ನೂ ಮೀರಿ ನಿಂತು, ಅವರನ್ನು ಹಿಮಾಲಯದ ಶಿಖರಗಳತ್ತ ಸಾಗುವಂತೆ ಪ್ರೇರೇಪಿಸುತ್ತಿತ್ತು. ಕಡೆಗೆ, ಸೆಪ್ಟೆಂಬರ್ ೫ ರಂದು ಸ್ವಾಮೀಜಿ ತಮ್ಮ ಮೂವರು ಗುರುಭಾಯಿಗಳು ಹಾಗೂ ಒಬ್ಬ ಸಾಮಾನು ಹೊರುವ ಕೂಲಿಯೊಂದಿಗೆ ಘರವಾಲಿನ ಮಾರ್ಗವಾಗಿ ಕರ್ಣಪ್ರಯಾಗಕ್ಕೆ ಬಂದರು. ಇಲ್ಲೊಬ್ಬ ಸರಕಾರೀ ವೈದ್ಯರು ಅವರನ್ನೆಲ್ಲ ಆದರದಿಂದ ಸತ್ಕರಿಸಿದರು.

ಅಲ್ಲಿಂದ ಮುಂದೆ ಅವರು ಬದರಿಕಾಶ್ರಮದ ದಾರಿಯಾಗಿ ಹೊರಟರು. ಆದರೆ ಬದರಿ-ಕೇದಾರ ಗಳಲ್ಲಿ ಕ್ಷಾಮ ತಲೆದೋರಿರುವುದರಿಂದ ಸರಕಾರ ಆ ಕಡೆಗೆ ಹೋಗುವ ದಾರಿಯನ್ನು ಮುಚ್ಚಿದೆ ಎಂದು ತಿಳಿದುಬಂತು. ಇದರಿಂದಾಗಿ ಬದರಿಕಾಶ್ರಮಕ್ಕೆ ಹೋಗುವ ತಮ್ಮ ಯೋಜನೆಯನ್ನು ಸ್ವಾಮೀಜಿ ಕೈಬಿಡಬೇಕಾಯಿತು. ಈಗ ಅವರು ಶ್ರೀನಗರದ ಕಡೆಗೆ ಪ್ರಯಾಣ ಹೊರಟರು. ದಾರಿಯುದ್ದಕ್ಕೂ ಕಂಡುಬರುತ್ತಿದ್ದ ತಿಳಿನೀರಿನ ತೊರೆಗಳು, ಮನೋಹರ ಜಲಪಾತಗಳು, ಹಚ್ಚ ಹಸಿರಾದ ವನಸಿರಿ, ಪರಮ ಪ್ರಶಾಂತತೆಯಿಂದ ಕೂಡಿದ ನಿರ್ಜನ ತಾಣಗಳು, ಅಲ್ಲಲ್ಲಿ ದೃಗ್ಗೋಚರವಾಗುತ್ತಿದ್ದ ಶುಭ್ರ ಹಿಮರಾಶಿ–ಇವುಗಳೆಲ್ಲ ಅವರ ಮನಸ್ಸಿಗೆ ಅತ್ಯಂತ ಆನಂದ ವನ್ನುಂಟುಮಾಡಿದುವು.

ಸ್ವಾಮೀಜಿ ಮತ್ತು ಅವರ ಗುರುಭಾಯಿಗಳು ಶ್ರೀನಗರಕ್ಕೆ ಹೋಗುವ ದಾರಿಯಲ್ಲಿ ರುದ್ರ ಪ್ರಯಾಗವನ್ನು ತಲುಪಿದರು. ಇದು ಮಂದಾಕಿನೀ ಹಾಗೂ ಅಲಕನಂದಾ ನದಿಗಳ ಸಂಗಮಸ್ಥಳ. ಇಲ್ಲಿ ರುದ್ರನಾಥನ ದೇವಾಲಯವಿದೆ. ಆದರೆ ಇಲ್ಲಿಗೆ ಬಂದು ತಲುಪುತ್ತಿದ್ದಂತೆ ಸ್ವಾಮೀಜಿಗೂ ಅಖಂಡಾನಂದರಿಗೂ ಅನಾರೋಗ್ಯವುಂಟಾಯಿತು. ಅಖಂಡಾನಂದರಿಗೆ ಕರ್ಣಪ್ರಯಾಗದಲ್ಲೇ ಶ್ವಾಸನಾಳದ ಉರಿಯೂತ (ಬ್ರಾಂಕೈಟಿಸ್) ಪ್ರಾರಂಭವಾಗಿದ್ದು, ಈಗ ಅದು ಉಲ್ಬಣಿಸಿತು. ಆದರೆ ಅದೃಷ್ಟವಶಾತ್ ಆ ಪ್ರದೇಶಕ್ಕೆ ಕಾರ್ಯಾರ್ಥವಾಗಿ ಬಂದು ಬಿಡಾರ ಹೂಡಿದ್ದ ಬದರೀದತ್ತ ಜೋಷಿ ಎಂಬ ಸರ್ಕಾರೀ ಅಧಿಕಾರಿಯೊಂದಿಗೆ ಭೇಟಿಯಾಯಿತು. ಈತ ಸ್ವಾಮಿಗಳಿಬ್ಬರಿಗೂ ಆಯುರ್ವೇದ ಔಷಧಿಯನ್ನು ಕೊಟ್ಟ. ಇದರಿಂದ ಅವರ ಕಾಯಿಲೆ ಹತೋಟಿಗೆ ಬಂದಿತು. ಹೀಗೆ ಇಬ್ಬರೂ ಸುಧಾರಿಸಿಕೊಂಡ ಮೇಲೆ ಬದರೀದತ್ತ ಜೋಷಿ ಸ್ವಾಮಿಗಳೆಲ್ಲರನ್ನೂ ಮೇನೆಗಳ ಮೂಲಕ ಅವರ ಮುಂದಿನ ಗುರಿಯಾದ ಶ್ರೀನಗರಕ್ಕೆ ಕಳಿಸಿಕೊಟ್ಟ.

ಶ್ರೀನಗರದಲ್ಲಿ ಅವರೆಲ್ಲರೂ ಅಲಕನಂದಾ ನದೀತೀರದ ಒಂದು ಏಕಾಂತ ಸ್ಥಳದಲ್ಲಿ ಕುಟೀರವೊಂದನ್ನು ಆಶ್ರಯಿಸಿದರು. ಇಲ್ಲಿ ಅವರು ಮಧುಕರಿ ಭಿಕ್ಷೆಯನ್ನು ಅವಲಂಬಿಸಿ ಸುಮಾರು ಒಂದೂವರೆ ತಿಂಗಳ ಕಾಲ ವಾಸವಾಗಿದ್ದರು. ಸ್ವಾಮಿಗಳಿಬ್ಬರ ಆರೋಗ್ಯವೂ ಸುಧಾರಿಸಿತು. ದಿನಗಳನ್ನು ಧ್ಯಾನ-ಪ್ರಾರ್ಥನೆ-ಶಾಸ್ತ್ರಾಧ್ಯಯನಗಳಲ್ಲಿ ಆನಂದದಿಂದ ಕಳೆದರು. ಈ ದಿನಗಳಲ್ಲಿ ಸ್ವಾಮೀಜಿ ತಮ್ಮ ಗುರುಭಾಯಿಗಳಿಗೆ ಉಪನಿಷತ್ತುಗಳನ್ನು ಬೋಧಿಸಿದರು. ಇಲ್ಲಿ ಒಬ್ಬ ಶಾಲಾ ಉಪಾಧ್ಯಾಯ ಇವರ ಸಂಪರ್ಕಕ್ಕೆ ಬಂದ. ಈತ ಹುಟ್ಟಿನಿಂದ ವೈಶ್ಯನಾದರೂ ಈಚೆಗೆ ಕ್ರೈಸ್ತ ಮತಕ್ಕೆ ಮತಾಂತರ ಹೊಂದಿದ್ದ. ಸ್ವಾಮೀಜಿ ಇವನಿಗೆ ಹಿಂದೂ ಧರ್ಮದ ವೈಭವ-ವೈಶಿಷ್ಟ್ಯ ಗಳನ್ನು ಮನಮುಟ್ಟುವಂತೆ ವಿವರಿಸಿ ಹೇಳಿದರು. ಇದರಿಂದ ತುಂಬ ಪ್ರಭಾವಿತನಾದ ಆ ಉಪಾಧ್ಯಾಯ ಸಂಪೂರ್ಣ ಪರಿವರ್ತನೆ ಹೊಂದಿ, ಹಿಂದೂಧರ್ಮದ ತೆಕ್ಕೆಗೆ ಹಿಂದಿರುಗಿದ. ಸ್ವಾಮೀಜಿ ಹಾಗೂ ಅವರ ಗುರುಭಾಯಿಗಳ ಮೇಲೆ ಇವನಿಗೆ ವಿಶೇಷ ವಿಶ್ವಾಸವುಂಟಾಗಿ ಹಲವಾರು ಸಲ ಅವರನ್ನು ತನ್ನ ಮನೆಗೆ ಆಹ್ವಾನಿಸಿ ಸತ್ಕರಿಸಿದ.

ಶ್ರೀನಗರದಿಂದ ಸ್ವಾಮೀಜಿ ಮತ್ತು ಅವರ ಗುರುಭಾಯಿಗಳು ಟೆಹರಿಯ ದಾರಿಯಲ್ಲಿ ಸಾಗಿ ದರು. ಗಂಗಾ ಮತ್ತು ಯಮುನಾ ನದಿಗಳ ಉಗಮಸ್ಥಾನಗಳಾದ ಗಂಗೋತ್ರಿ ಯಮುನೋತ್ರಿ ಗಳೆರಡೂ ಇರುವುದು ಟೆಹರಿಯಲ್ಲೇ. ಪ್ರಯಾಣ ಮಾಡುತ್ತ, ಕತ್ತಲಾಗುವ ವೇಳೆಗೆ ಘರವಾಲ್ ಎಂಬ ಒಂದು ಹಳ್ಳಿಗೆ ಬಂದರು. ದಾರಿಯಲ್ಲೆಲ್ಲೂ ಭಿಕ್ಷೆ ಸಿಗದಿದ್ದುದರಿಂದ ಹಸಿವು ಕಿತ್ತು ತಿನ್ನಲಾರಂಭಿಸಿತ್ತು. ದಾರಿಪಕ್ಕದ ಒಂದು ಮುರುಕಲು ಛತ್ರದಲ್ಲಿ ಇಳಿದುಕೊಂಡು, ಬಳಿಕ ಹಳ್ಳಿಯೊಳಗೆ ಪ್ರವೇಶಿಸಿ ಭಿಕ್ಷೆ ಕೇಳಿದರು. ಆದರೆ ಎಷ್ಟು ಬೇಡಿದರೂ ಒಬ್ಬ ಮನುಷ್ಯನೂ ಓಗೊಡಲಿಲ್ಲ, ಒಂದು ತುತ್ತು ಭಿಕ್ಷೆಯೂ ಸಿಗಲಿಲ್ಲ. ‘ಇದೆಂಥ ದುರ್ಭಿಕ್ಷದ ಜಾಗಕ್ಕೆ ಬಂದೆ ವಪ್ಪ!’ ಎನ್ನಿಸತೊಡಗಿತು ಅವರಿಗೆ. ಆದರೆ ಈ ಜನರು ಭಿಕ್ಷೆ ಹಾಕದಿದ್ದುದಕ್ಕೆ ಕಾರಣವೆಂದರೆ ಅಲ್ಲಿನ ಒಂದು ವಿಚಿತ್ರ ಸಂಪ್ರದಾಯ. ಈ ಘರವಾಲಿನ ಜನಗಳ ಬಗ್ಗೆ ಒಂದು ಮಾತೇ ಪ್ರಚಲಿತವಿತ್ತು:

\begin{myquote}
ಘರವಾಲಿನವರೆಲ್ಲ ಕೊಡುಗೈಯ ದಾನಿಗಳು–\\ಅವರಂಥ ದಾನಿಗಳೆ ಧರೆಯೊಳಿಲ್ಲ;\\ಆದರೊಂದೇ ನಿಯಮ-ದೊಣ್ಣೆಯಿಂ ಬೆದರಿಸದೆ\\ಅವರಿಂದ ದಾನವನು ಪಡೆವರಿಲ್ಲ!
\end{myquote}

\noindent

ಭಿಕ್ಷೆ ಹಾಕಲೇನೂ ಅಭ್ಯಂತರವಿಲ್ಲವಂತೆ; ಆದರೆ ಅವರು ಕೈಯೆತ್ತಿ ಕೊಡಬೇಕಾದರೆ, ‘ಕೊಡದಿ ದ್ದರೆ ದೊಣ್ಣೆಯಿಂದ ಬಡಿದುಹಾಕುತ್ತೇವೆ’ ಎಂದು ಹೆದರಿಸಬೇಕಂತೆ! ಅದೃಷ್ಟಕ್ಕೆ, ಅಖಂಡಾ ನಂದರಿಗೆ ಈ ವಿಷಯ ತಿಳಿದಿತ್ತು; ಈಗ ಇದ್ದಕ್ಕಿದ್ದಂತೆ ನೆನಪಾಯಿತು. ಕೂಡಲೇ ಈ ಸಂನ್ಯಾಸಿಗಳು ಮನೆಗಳ ಮಂದೆ ನಿಂತು, ತಮ್ಮ ದಂಡಗಳನ್ನು ಮೇಲೆತ್ತಿ, “ಭಿಕ್ಷೆ ಹಾಕುತ್ತೀರೋ, ದೊಣ್ಣೆ ಬೀಸಬೇಕೋ!” ಎಂದು ಗಂಟಲು ಬಿರಿಯುವಂತೆ ಅಬ್ಬರಿಸಿದರು. ಆಗ ಹಳ್ಳಿಯ ಜನರೆಲ್ಲ ಬಹಳ ವಿನಯದಿಂದ ಆಹಾರವನ್ನು ತಂದಿಟ್ಟು ಕೈಮುಗಿದರು! ಬಿಸಿಯಾಗದೆ ಬೆಣ್ಣೆ ಕರಗದು ಎಂಬ ಮಾತು ಸುಮ್ಮನೆ ಹುಟ್ಟಿಕೊಂಡಿತೆ? ಕೊನೆಗೂ ‘ದಂಡೋಪಾಯ’ವೇ ಸಿದ್ಧಿಸಿತು.

ಮರುದಿನ ಅವರು ಅಲ್ಲಿಂದ ಹೊರಟು ಟೆಹರಿಗೆ ಬಂದು ತಲುಪಿದರು. ಗಂಗೆಯ ದಡದಲ್ಲಿ ಸಾಧುಗಳ ವಾಸಕ್ಕಾಗಿಯೇ ಕಟ್ಟಿದ್ದ ಎರಡು ಕುಟೀರಗಳಲ್ಲಿ ಉಳಿದುಕೊಂಡು, ಭಿಕ್ಷೆಯ ಮೇಲೆ ಜೀವಿಸುತ್ತ, ಆಧ್ಯಾತ್ಮಿಕ ಸಾಧನೆಗಳಲ್ಲಿ ತೊಡಗಿದರು. ಇಲ್ಲಿ ಅವರಿಗೆ ಟೆಹರಿ ರಾಜ್ಯದ ದಿವಾನ ರಾದ ರಘುನಾಥ ಭಟ್ಟಾಚಾರ್ಯರ ಪರಿಚಯವಾಯಿತು. ಅವರ ಅಪೇಕ್ಷೆಯ ಮೇರೆಗೆ ಸ್ವಾಮೀಜಿ ಅವರ ಮನೆಯಲ್ಲಿ ಕೆಲದಿನ ಇಳಿದುಕೊಂಡರು. ಆದರೆ ಸ್ವಾಮೀಜಿ ಗಂಗಾತೀರದ ಒಂದು ಸೂಕ್ತ ಸ್ಥಳದಲ್ಲಿದ್ದು ಧ್ಯಾನದಲ್ಲಿ ಮುಳುಗಿಬಿಡಲು ಕಾತರರಾಗಿದ್ದರು. ಇದನ್ನು ತಿಳಿದು ಅವರ ಆತಿ ಥೇಯರಾದ ರಘುನಾಥ ಭಟ್ಟಾಚಾರ್ಯರು ಅವರ ನೆರವಿಗೆ ಒದಗಿದರು. ಗಂಗಾ ಹಾಗೂ ವಿಲಂಗನಾ ನದಿಗಳ ಸಂಗಮಸ್ಥಳವಾದ ಗಣೇಶಪ್ರಯಾಗ ಎಂಬಲ್ಲಿ ಒಂದು ಸೊಗಸಾದ ಕುಟೀರವನ್ನು ಕಟ್ಟಿಸಿಕೊಟ್ಟರು. ಆದರೆ ಸ್ವಾಮೀಜಿಯ ದುರದೃಷ್ಟ! ಈ ವೇಳೆಗೆ ಸ್ವಾಮಿ ಅಖಂಡಾನಂದರು ಮತ್ತೆ ಬ್ರಾಂಕೈಟಿಸಿಗೆ ಗುರಿಯಾದರು. ಅವರನ್ನು ಪರೀಕ್ಷಿಸಿದ ವೈದ್ಯರು, ಅವರಿಗೆ ಪರ್ವತ ಪ್ರದೇಶ ಒಗ್ಗುವುದಿಲ್ಲವೆಂದು ತಿಳಿಸಿ, ಚಳಿಗಾಲ ಬೇರೆ ಸಮೀಪಿಸುತ್ತಿರುವುದ ರಿಂದ ಕೂಡಲೇ ಬಯಲು ಪ್ರದೇಶಕ್ಕೆ ಹೋಗಿರುವುದೊಂದೇ ಉಪಾಯ ಎಂದು ಹೇಳಿದರು. ಗಣೇಶಪ್ರಯಾಗಕ್ಕೆ ಹೊರಡಲು ಎಲ್ಲ ಸಿದ್ಧವಾಗಿರುವಾಗ ಹೀಗಾಗಿಬಿಟ್ಟಿದೆ. ಆದರೆ ಸ್ವಾಮೀಜಿ ಸ್ವಲ್ಪವೂ ಬೇಸರಿಸಲಿಲ್ಲ. ಶ್ರೀರಾಮಕೃಷ್ಣರು ಈ ಸೋದರಸಂನ್ಯಾಸಿಗಳನ್ನೆಲ್ಲ ಅವರ ಕೈಗಲ್ಲವೆ ಒಪ್ಪಿಸಿರುವುದು? ಕೂಡಲೇ ಅವರು ತಮ್ಮ ಯೋಜನೆಯನ್ನು ಬದಲಾಯಿಸಿ, ಡೆಹರಾಡೂನಿಗೆ ಹೋಗಲು ನಿರ್ಧರಿಸಿದರು. ಮತ್ತು ದಿವಾನರ ಬಳಿಗೆ ಹೋಗಿ ವಿಷಯವನ್ನು ವಿವರಿಸಿ, ಮುಂದೆ ಯಾವಾಗಲಾದರೂ ಅವರ ಸೇವೆಯನ್ನು ಪಡೆದುಕೊಳ್ಳುವುದಾಗಿ ಹೇಳಿದರು. ದಿವಾನರು ಸ್ವಾಮೀಜಿ ಹಾಗೂ ಅವರ ಗುರುಭಾಯಿಗಳ ಪ್ರಯಾಣಕ್ಕೆ ವ್ಯವಸ್ಥೆ ಮಾಡಿದರಲ್ಲದೆ, ಡೆಹರಾಡೂ ನಿನ ಸರಕಾರೀ ವೈದ್ಯರಿಗೆ ಅವರ ಬಗ್ಗೆ ಒಂದು ಪರಿಚಯ ಪತ್ರವನ್ನು ಕೊಟ್ಟು ಬೀಳ್ಗೊಂಡರು.

ಟೆಹರಿಯಿಂದ ಹೊರಟ ಈ ಸಂನ್ಯಾಸಿಗಳು ಪಯಣಿಸುತ್ತ ರಾಜಪುರದ ಹತ್ತಿರ ಬರುತ್ತಿರು ವಾಗ, ಆ ಕಡೆಯಿಂದ ಕಾಷಾಯಧಾರಿಯೊಬ್ಬರು ಅವರ ಕಡೆಗೇ ಬರುತ್ತಿರುವುದು ಕಂಡಿತು. ನೋಡಿದರೆ... ತುರೀಯಾನಂದರಂತೆ ಕಾಣುತ್ತಿದ್ದಾರೆ. ಹತ್ತಿರ ಬಂದಂತೆ ನೋಡುತ್ತಾರೆ– ಹೌದು, ಅವರೇ! “ಓ ನಮ್ಮ ಹರಿ!” ಎಂದು ಎಲ್ಲರೂ ಉತ್ಸಾಹದಿಂದ ಗಟ್ಟಿಯಾಗಿ ಕೂಗಿ ಕೊಂಡರು. ತಮ್ಮ ಪ್ರೀತಿಯ ಗುರುಭಾಯಿಯನ್ನು ಹೀಗೆ ಅನಿರೀಕ್ಷಿತವಾಗಿ ಕಂಡು ಎಲ್ಲರಿಗೂ ಅತ್ಯಾನಂದವಾಯಿತು. ಈಗ ಸ್ವಾಮೀಜಿ ತಮ್ಮ ಸಂನ್ಯಾಸೀಬಂಧುಗಳಾದ ಅಖಂಡಾನಂದರು, ಶಾರದಾನಂದರು, ತುರೀಯಾನಂದರು ಹಾಗೂ ಕೃಪಾನಂದರು–ಇವರುಗಳೊಡನೆ ಡೆಹರಾಡೂ ನಿಗೆ ನಡೆದರು. ಅಲ್ಲಿನ ಸರಕಾರೀ ವೈದ್ಯರನ್ನು ಭೇಟಿಯಾದಾಗ ಅವರು ಅಖಂಡಾನಂದರನ್ನು ಚೆನ್ನಾಗಿ ಪರೀಕ್ಷಿಸಿ, ಸೂಕ್ತ ಔಷಧವನ್ನು ಕೊಟ್ಟರು. ಮತ್ತು ಬಯಲುಸೀಮೆಯಲ್ಲೇ ವಾಸವಾಗಿದ್ದು ಕೊಂಡು ಔಷಧಿ-ಪಥ್ಯಗಳನ್ನು ಮುಂದುವರಿಸಬೇಕು ಎಂದು ಸೂಚಿಸಿದರು.

ಈಗ ಡೆಹರಾಡೂನಿನಲ್ಲಿ ಇವರೆಲ್ಲ ಅರ್ಧಂಬರ್ಧ ಕಟ್ಟಿದ್ದ ಥಂಡಿ ಮನೆಯೊಂದರಲ್ಲಿ ಬೀಡು ಬಿಟ್ಟರು. ಆದರೆ ಅಖಂಡಾನಂದರು ಮಾತ್ರ ಅಲ್ಲಿರಲು ಸಾಧ್ಯವೇ ಇರಲಿಲ್ಲ. ಆದ್ದರಿಂದ ಅವರಿ ಗಾಗಿ ಬೇರೊಂದು ವಸತಿಯನ್ನು ಹುಡುಕಬೇಕಾಯಿತು. ಸ್ವಾಮೀಜಿ ತಮ್ಮ ಸೋದರನಿಗಾಗಿ ಆಶ್ರಯವನ್ನರಸುತ್ತ ಮನೆಯಿಂದ ಮನೆಗೆ ಅಲೆದರು. ಆದರೆ ಅವರಿಗೆ ವಸತಿ ನೀಡಲು ಯಾರೂ ಮುಂದಾಗಲಿಲ್ಲ. ಎಷ್ಟಾದರೂ ಹರಕಲು ಕಾವಿಶಾಟೆಯ ಬಡಸಂನ್ಯಾಸಿಗಳಲ್ಲವೆ? ಸಾಕಷ್ಟು ಅಲೆದಾಡಿದ ಮೇಲೆ, ಹಿಂದೆ ಕಾಲೇಜಿನಲ್ಲಿ ಸ್ವಾಮೀಜಿಯ ಸಹಪಾಠಿಯೇ ಆಗಿದ್ದ ಹೃದಯಬಾಬು ಎಂಬೊಬ್ಬ ಶಾಲಾ ಅಧ್ಯಾಪಕನ ಭೇಟಿಯಾಯಿತು. ಈಗ ಆತ ಮತಾಂತರ ಹೊಂದಿ ಕ್ರೈಸ್ತ ನಾಗಿದ್ದ. ಅವನು ಅಖಂಡಾನಂದರನ್ನು ತನ್ನ ಮನೆಯಲ್ಲಿಟ್ಟುಕೊಳ್ಳಲು ಸಂತೋಷದಿಂದ ಮುಂದಾದ. ಅಖಂಡಾನಂದರು ಅವನ ಮನೆಗೆ ಹೋಗಿ ಉಳಿದುಕೊಂಡರು. ಆದರೆ ಮಡಿವಂತ ಹಿಂದೂ ಸಂಪ್ರದಾಯಗಳಲ್ಲೇ ಹುಟ್ಟಿಬೆಳೆದ ಅಖಂಡಾನಂದರಿಗೆ ಆತನ ಮನೆಯ ರೀತಿನೀತಿಗಳು ಸ್ವಲ್ಪವೂ ಸರಿಬರಲಿಲ್ಲ. ಆದ್ದರಿಂದ ತಮ್ಮ ಗುರುಭಾಯಿಗಳು ವಾಸವಾಗಿದ್ದ ಮನೆಗೇ ವಾಪಸ್ಸು ಬಂದುಬಿಟ್ಟರು. ಈಗ ಸ್ವಾಮೀಜಿ ಅವರಿಗಾಗಿ ಸೂಕ್ತ ವಸತಿ ಹುಡುಕಿಕೊಂಡು ಮತ್ತೆ ಹೊರಟರು. ಎಷ್ಟೆಷ್ಟೋ ಸುತ್ತಾಡಿದ ಮೇಲೆ ಕಡೆಗೆ ಪಂಡಿತ ಆನಂದನಾರಾಯಣ ಎಂಬೊಬ್ಬ ವಕೀಲರು ಅವರೆಲ್ಲರನ್ನೂ ನೋಡಿಕೊಳ್ಳಲು ಮುಂದಾದರು. ಅವರಿಗಾಗಿಯೇ ಒಂದು ಪುಟ್ಟ ಮನೆಯನ್ನು ಬಾಡಿಗೆಗೆ ಗೊತ್ತು ಮಾಡಿ, ಅದರಲ್ಲಿ ಅಖಂಡಾನಂದರಿಗೆ ಪಥ್ಯಪಾನದ ವ್ಯವಸ್ಥೆ ಮಾಡಿದರು.

ಹೃದಯಬಾಬುವಿನ ಮನೆಯಲ್ಲಿ ಸ್ವಾಮೀಜಿಗೆ ಕೆಲವು ಕ್ರೈಸ್ತ ಧರ್ಮಪ್ರಚಾರಕರೊಂದಿಗೆ ಭೇಟಿಯಾಯಿತು. ಆ ಕಾಲದಲ್ಲಿ ಈ ಪ್ರಚಾರಕರ ಮುಖ್ಯ ಕೆಲಸವೆಂದರೆ ಹಿಂದೂಧರ್ಮವನ್ನು ತುಚ್ಛೀಕರಿಸಿ, ಕ್ರೈಸ್ತಧರ್ಮದ ಮಹಿಮೆಯನ್ನು ಸಾರುವುದು; ಹಾಗೂ ಹಿಂದೂಗಳನ್ನು ಕ್ರೈಸ್ತ ಧರ್ಮಕ್ಕೆ ಮತಾಂತರಿಸುವುದು. ಈಗ ಕ್ರೈಸ್ತಮತೀಯನಾದ ಹೃದಯಬಾಬುವಿನ ಮನೆಯಲ್ಲಿ ಹಿಂದೂ ಸಂನ್ಯಾಸಿಗಳನ್ನು ಕಂಡು ಕಸಿವಿಸಿಗೊಂಡ ಆ ಪ್ರಚಾರಕರು ಸ್ವಾಮೀಜಿಯೊಂದಿಗೆ ವಾಗ್ವಾದಕ್ಕಿಳಿದರು. ಹಿಂದೂಧರ್ಮವನ್ನು ಮೂಢನಂಬಿಕೆಗಳ ಕಂತೆ ಎಂದು ಅಪಹಾಸ್ಯ ಮಾಡಲು ನೋಡಿದರು. ಸ್ವತಃ ಕ್ರಿಸ್ತನ ಆರಾಧಕರೂ ಕ್ರೈಸ್ತಧರ್ಮವನ್ನು ಆಳವಾಗಿ ಅಧ್ಯಯಿಸಿದವರೂ ಆದ ಸ್ವಾಮೀಜಿ, ಅವರ ವಾದಗಳನ್ನು ನುಚ್ಚುನೂರು ಮಾಡಿದ್ದಲ್ಲದೆ, ಅವರ ಪವಿತ್ರ ಗ್ರಂಥವಾದ ಬೈಬಲ್ಲಿನಲ್ಲಿರುವ ಹಲವಾರು ಲೋಪದೋಷಗಳನ್ನು, ಅಸಂಬದ್ಧತೆಗಳನ್ನು ಎತ್ತಿ ತೋರಿಸಿದರು. ಆ ಧರ್ಮಪ್ರಚಾರಕರು ಇದನ್ನೆಲ್ಲ ಕೇಳಿದವರೇ ಅಲ್ಲ. ಕೊನೆಗೆ ಅವರು ಸ್ವಾಮೀಜಿಯ ಮಾತಿನ ರಭಸವನ್ನು ತಾಳಲಾರದೆ ಕಾಲಿಗೆ ಬುದ್ಧಿ ಹೇಳಿದರು. ಆದರೆ ತಮ್ಮ ಆತಿಥೇಯನಾದ ಹೃದಯ ಬಾಬುವಿನ ಮನೆಯಲ್ಲೇ ಕ್ರೈಸ್ತಧರ್ಮೀಯರ ವಿರುದ್ಧವಾಗಿ ವಾದ ಮಾಡಿದ್ದಕ್ಕಾಗಿ ಸ್ವಾಮೀಜಿ ಆಮೇಲೆ ಅವನ ಕ್ಷಮೆ ಕೇಳಿದರು.

ಸ್ವಾಮೀಜಿ ತಮ್ಮ ಗುರುಭಾಯಿಗಳೊಂದಿಗೆ ಸುಮಾರು ಮೂರು ವಾರ ಇದ್ದರು. ಬಳಿಕ ಅಲಹಾಬಾದಿಗೆ ಹೋಗಿ ತಮ್ಮ ಒಬ್ಬ ಸ್ನೇಹಿತನ ಮನೆಯಲ್ಲಿದ್ದು ವಿಶ್ರಮಿಸುವಂತೆ ಅಖಂಡಾ ನಂದರಿಗೆ ಹೇಳಿ, ತಾವು ಮತ್ತೆ ಸಮಾಧಿಸುಖದ ಬಯಕೆ ಹೊತ್ತು ಹೃಷೀಕೇಶದತ್ತ ಹೊರಟರು. ತುರೀಯಾನಂದರು, ಶಾರದಾನಂದರು ಮತ್ತು ಕೃಪಾನಂದರು ಕೂಡ ಅವರೊಂದಿಗೆ ನಡೆದರು. ಇತ್ತ ಅಲಹಾಬಾದಿಗೆ ಹೊರಟ ಅಖಂಡಾನಂದರು ದಾರಿಯಲ್ಲಿ ಶಹರಾನ್ಪುರಕ್ಕೆ ಬಂದು, ಬಾಂಕುಬಿಹಾರಿ ಎಂಬ ವಕೀಲನ ಮನೆಯಲ್ಲಿ ಉಳಿದುಕೊಂಡರು. ಬಳಿಕ ಆತನ ಸಲಹೆಯಂತೆ ಅಲಹಾಬಾದಿಗೆ ಬದಲಾಗಿ ಮೀರತ್ತಿಗೆ ಹೋಗಿ, ತ್ರೈಲೋಕ್ಯನಾಥಘೋಷ್ ಎಂಬ ಸರ್ಕಾರೀ ವೈದ್ಯನ ಆಶ್ರಯದಲ್ಲಿ ಉಳಿದುಕೊಂಡರು.

ಇತ್ತ ಸ್ವಾಮೀಜಿ ಮತ್ತೊಮ್ಮೆ ಪುರಾಣಪ್ರಸಿದ್ಧ ಪುಣ್ಯಕ್ಷೇತ್ರವಾದ ಹೃಷೀಕೇಶಕ್ಕೆ ಬಂದಿದ್ದಾರೆ. ಇದು ನಯನಮನೋಹರ, ಪರಮಪ್ರಶಾಂತ ಸ್ಥಳ, ಸುತ್ತಲೂ ಬೆಟ್ಟಗಳು, ಕಣಿವೆಗಳು ಹರಡಿ ಕೊಂಡಿವೆ; ಗಂಗೆ ಅರ್ಧವೃತ್ತಾಕಾರದಲ್ಲಿ ಹರಿಯುತ್ತಿದ್ದಾಳೆ. ಆ ದಿನಗಳಲ್ಲಿ ಆ ಪ್ರದೇಶ ಸಂಪೂರ್ಣ ಅರಣ್ಯಮಯವಾಗಿತ್ತು. ಸುತ್ತಮುತ್ತೆಲ್ಲ ಸಂನ್ಯಾಸಿಗಳ ಆಶ್ರಮಗಳು, ಕುಟೀರಗಳು, ಅಲ್ಲಿನ ವಾತಾವರಣವೆಲ್ಲ ಪರಮಪರಿಶುದ್ಧ, ಪಾವನಕರ. ಈ ದಿನಗಳನ್ನು ನೆನಪಿಸಿಕೊಂಡು ಮುಂದೊಮ್ಮೆ ಸ್ವಾಮೀಜಿ ಹೇಳುತ್ತಾರೆ: “... ನಾನು ಹೃಷೀಕೇಶದಲ್ಲಿ ಹಲವಾರು ಮಹಾತ್ಮ ರನ್ನು ಕಂಡೆ. ಅವರಲ್ಲೊಬ್ಬ ಹುಚ್ಚನಂತೆ ಕಾಣಿಸಿಕೊಳ್ಳುತ್ತಿದ್ದ. ರಸ್ತೆಯಲ್ಲಿ ಆತ ಬತ್ತಲೆಯಾಗಿ ನಡೆದಾಡುತ್ತಿದ್ದ. ಹುಡುಗರೆಲ್ಲ ಅವನ ಮೇಲೆ ಕಲ್ಲುಗಳನ್ನು ತೂರುತ್ತ, ಅವನನ್ನು ಹಿಂಬಾಲಿಸುತ್ತಿ ದ್ದರು. ಅವನ ತಲೆ-ಕುತ್ತಿಗೆಗಳಿಂದ ರಕ್ತ ಸೋರುತ್ತಿತ್ತು. ಆದರೆ ಆ ಮಹಾತ್ಮ ಆನಂದದಿಂದ ನಗುನಗುತ್ತಲೇ ಇದ್ದ! ನಾನವನನ್ನು ಕರೆದೊಯ್ದು, ಅವನ ಗಾಯಗಳನ್ನೆಲ್ಲ ತೊಳೆದು, ಬಟ್ಟೆ ಸುಟ್ಟ ಬೂದಿಯನ್ನು ಹಚ್ಚಿ, ಪಟ್ಟಿ ಕಟ್ಟಿದೆ. ಅವನಿಗೆ ಔಷಧೋಪಚಾರ ಮಾಡುವಾಗಲೂ ಆತ ನಿರಂತರವಾಗಿ ನಗುತ್ತಲೇ ಹೇಳಿದ–‘ನಾನೂ ಆ ಹುಡುಗರೂ ಕಲ್ಲುಗಳನ್ನು ಎಸೆದುಕೊಂಡು ತುಂಬ ತಮಾಷೆಯಾಗಿ ಆಡಿಕೊಂಡಿದ್ದೆವು!... ಭಗವಂತ ಹೀಗೆಲ್ಲ ಆಟವಾಡುತ್ತಾನೆ’ ಎಂದು. ಇಂತಹ ಅನೇಕ ಮಹಾತ್ಮರು, ಸಾಮಾನ್ಯ ಜನ ತಮಗೆ ಸುಮ್ಮನೆ ತೊಂದರೆ ಕೊಡದಿರಲಿ ಎಂದು ವಿಚಿತ್ರ ರೀತಿಯಲ್ಲಿ ಅಡಗಿಕೊಂಡಿರುತ್ತಾರೆ. ಜನಜಂಗುಳಿ ಅವರಿಗೊಂದು ಬಾಧಕ. ಒಬ್ಬ ಮಹಾಪುರುಷ ತನ್ನ ಗುಹೆಯ ಎದುರಿಗೆ ಮನುಷ್ಯರ ಮೂಳೆಗಳನ್ನು ಹರಡಿರುತ್ತಿದ್ದ. ಇವನಾರೋ ನರಭಕ್ಷಕನಿರಬೇಕು ಎಂದು ಜನ ಭ್ರಮಿಸಿ ದೂರವೇ ಉಳಿಯುವಂತೆ ಮಾಡಲು ಈ ಉಪಾಯ! ಮತ್ತೊಬ್ಬ ಸಾಧು ಜನರ ಮೇಲೆ ಅಕಾರಣವಾಗಿ ಕಲ್ಲುತೂರುತ್ತಿದ್ದ... ”

ಸ್ವಾಮೀಜಿ ಹಾಗೂ ಅವರ ಸೋದರಸಂನ್ಯಾಸಿಗಳು ಅಲ್ಲಿನ ಚಂದ್ರೇಶ್ವರ ಮಹಾದೇವನ ದೇವ ಸ್ಥಾನದ ಬಳಿಯಲ್ಲಿ, ತಾವೇ ಕುಟೀರವೊಂದನ್ನು ಕಟ್ಟಿಕೊಂಡು ವಾಸಿಸಲಾರಂಭಿಸಿದರು. ಊಟಕ್ಕೆ ಭಿಕ್ಷೆ ಇದ್ದೇ ಇದೆ. ಇಲ್ಲಿ ಮತ್ತೊಮ್ಮೆ ಸ್ವಾಮೀಜಿ ಅತ್ಯಂತ ತೀವ್ರವಾದ ತಪಸ್ಸಾಧನೆಯಲ್ಲಿ ಮುಳುಗಿಬಿಡಲು ತೀರ್ಮಾನಿಸಿದರು. ಆದರೆ ವಿಧಿಯಿಚ್ಛೆ ಬೇರೆಯಾಗಿತ್ತು. ಸಾಧನೆಯಲ್ಲಿ ನಿರತರಾ ಗುವ ಮೊದಲೇ ಅವರು ತೀವ್ರ ಜ್ವರ ಹಾಗೂ ಗಂಟಲ ನೋವಿಗೆ ಗುರಿಯಾಗಿ ಹಾಸಿಗೆ ಹಿಡಿದರು. ಒಂದು ದಿನ ಸೋದರಸಂನ್ಯಾಸಿಗಳೆಲ್ಲ ತಮ್ಮ ಕುಟೀರವನ್ನು ವಿಸ್ತರಿಸುವುದಕ್ಕಾಗಿ ಬಿದಿರನ್ನು ಕತ್ತರಿಸಿ ತರಲು ಕಾಡಿಗೆ ಹೋಗಿದ್ದರು. ಹಿಂದಿರುಗಿ ಬಂದಾಗ ನೋಡುತ್ತಾರೆ, ಸ್ವಾಮೀಜಿ ಸಂಪೂರ್ಣ ನಿಶ್ಶಕ್ತರಾಗಿ ಮಲಗಿಬಿಟ್ಟಿದ್ದಾರೆ. ನೋಡನೋಡುತ್ತಿದ್ದಂತೆ ಪರಿಸ್ಥಿತಿ ಹದಗೆಡುತ್ತ ಬಂತು. ಮೂವರು ಗುರುಭಾಯಿಗಳೂ ಅಸಹಾಯಕರಾಗಿ ಸುಮ್ಮನೆ ನೋಡುತ್ತಿದ್ದಾರೆ. ಸುತ್ತಲೂ ೩೦ ಮೈಲಿಗಳ ದೂರದಲ್ಲಿ ಯಾವುದೇ ವೈದ್ಯಕೀಯ ಸೌಲಭ್ಯ ದೊರಕುವಂತಿಲ್ಲ. ಸ್ವಾಮೀಜಿ ದೇಹಪ್ರಜ್ಞೆ ಕಳೆದುಕೊಂಡು, ಒರಟು ಕಂಬಳಿಗಳ ಮೇಲೆ ಮಲಗಿದ್ದಾರೆ. ಕ್ರಮೇಣ ಅವರ ನಾಡಿಬಡಿತ ನಿಧಾನವಾಯಿತು; ದೇಹ ತಣ್ಣಗಾಗತೊಡಗಿತು. ಗುರುಭಾಯಿಗಳು ದುಃಖೋದ್ವೇಗ ದಿಂದ ಚಡಪಡಿಸುತ್ತಿದ್ದಂತೆ, ತಲೆಯ ಭಾಗವನ್ನು ಬಿಟ್ಟು ಮೈಯಲ್ಲ ಮಂಜಿನಂತಾಗಿಬಿಟ್ಟಿತು; ನಾಡಿ ಮತ್ತಷ್ಟು ಕ್ಷೀಣಗೊಂಡು ಕಡೆಗೆ ನಿಂತೇಹೋದಂತಾಯಿತು! ಇನ್ನವರು ಬದುಕಿಯಾರೆಂಬ ಆಸೆ ಉಳಿಯಲಿಲ್ಲ. ಗುರುಭಾಯಿಗಳು ಕಿಂಕರ್ತವ್ಯವಿಮೂಢರಾಗಿ ರೋದಿಸಲಾರಂಭಿಸಿದರು. ತಮ್ಮ ಪ್ರೀತಿಯ ನರೇನನ ಪ್ರಾಣಕ್ಕೆ ಬದಲಾಗಿ ತಮ್ಮ ಪ್ರಾಣಗಳನ್ನು ಒಯ್ಯುವಂತೆ ಭಗವಂತ ನಲ್ಲಿ ಮೊರೆಯಿಟ್ಟರು. ಪರಿಸ್ಥಿತಿ ಹೀಗಿರುವಾಗ ಯಾರೋ ಅಪರಿಚಿತ ಸಾಧುವೊಬ್ಬ ಅಲ್ಲಿಗೆ ಬಂದ. ಅವನು ಮೈಮೇಲೊಂದು ಕಂಬಳಿ ಹೊದ್ದಿದ್ದ. ಈ ಸಂನ್ಯಾಸಿಗಳು ಹೀಗೆ ಅಳುತ್ತ ಕುಳಿತಿರುವುದನ್ನು ಕಂಡು ಕಾರಣವೇನೆಂದು ಕೇಳಿದ. ಏಕೆಂದು ಹೇಳಬೇಕಾಗಿಯೇ ಇಲ್ಲ; ಇದಿರಲ್ಲೇ ಸ್ವಾಮೀಜಿ ಮೃತಪ್ರಾಯರಾಗಿ ಮಲಗಿದ್ದಾರೆ. ಆ ಸಾಧು ತನ್ನ ಜೋಳಿಗೆಯಿಂದ ಅಶ್ವತ್ಥದ ಬೇರಿನ ಪುಡಿಯಂತೆ ಹಾಗೂ ಜೇನುತುಪ್ಪದಂತೆ ಕಾಣುತ್ತಿದ್ದ ಎರಡು ವಸ್ತುಗಳನ್ನು ತೆಗೆದು, ಎರಡನ್ನೂ ಬೆರೆಸಿ, ಸ್ವಾಮೀಜಿಯ ಬಾಯೊಳಗೆ ಬಲವಂತವಾಗಿ ತಳ್ಳಿದ. ಔಷಧ ಪವಾಡದಂತೆ ಕೆಲಸ ಮಾಡಿತು. ಕೆಲ ನಿಮಿಷಗಳಲ್ಲೇ ಅವರ ದೇಹ ಬೆಚ್ಚಗಾಯಿತು, ನಾಡಿಬಡಿತ ಹೆಚ್ಚಾಯಿತು. ಪ್ರಾಣಾಪಾಯ ತಪ್ಪಿತೆಂಬುದು ಖಚಿತವಾಯಿತು. ಆ ಸಾಧುವನ್ನು ಆ ಸಮಯಕ್ಕೆ ಸರಿಯಾಗಿ ಅಲ್ಲಿಗೆ ಬರುವಂತೆ ಮಾಡಿದ್ದು ದೇವರೇ ಅಲ್ಲದೆ ಮತ್ತಾರು? ಶ್ರೀರಾಮಕೃಷ್ಣರ ಇಚ್ಛೆಗೆ ಸಂಪೂರ್ಣ ಶರಣಾಗಿ, ಜಗನ್ಮಾತೆಯ ಕಾರ್ಯವನ್ನು ನಡೆಸಲು ನಿಯೋಜಿತರಾಗಿರುವ ವಿವೇಕಾನಂದರಿಗೆ ಇಂಥ ಸಾವು ಉಂಟಾದೀತೆ?

ಸ್ವಲ್ಪ ಹೊತ್ತಿನಲ್ಲಿ ಸ್ವಾಮೀಜಿ ಕಣ್ದೆರೆದು ಮಾತನಾಡುವ ಪ್ರಯತ್ನ ಮಾಡಿದರು. ಆದರೆ ಸ್ವರವಿನ್ನೂ ಗಟ್ಟಿಯಾಗಿ ಹೊರಡುತ್ತಿಲ್ಲ. ಆಗ ಗುರುಭಾಯಿಗಳಲ್ಲೊಬ್ಬರು ಅವರ ತುಟಿಯ ಹತ್ತಿರ ತಮ್ಮ ಕಿವಿಯಿಟ್ಟು ಕೇಳಿದರು. “ಏನೂ ಹೆದರಬೇಡಿ, ಧೈರ್ಯವಾಗಿರಿ. ನಾನು ಸಾಯ ಲಾರೆ!” ಎಂದರು ಸ್ವಾಮೀಜಿ. ಮತ್ತೆ ಕ್ರಮೇಣ ಚೇತರಿಸಿಕೊಳ್ಳಲಾರಂಭಿಸಿದರು. ಆಮೇಲೆ ಒಮ್ಮೆ ಅವರು ಗುರುಭಾಯಿಗಳಿಗೆ ಹೇಳುತ್ತಾರೆ: “ಆ ಅರೆಪ್ರಜ್ಞಾವಸ್ಥೆಯಲ್ಲಿದ್ದಾಗ, ನಾನು ಈ ಜಗತ್ತಿನಲ್ಲಿ ಈಡೇರಿಸಬೇಕಾದ ಒಂದು ಮಹತ್ವದ ಕಾರ್ಯವಿದೆ; ಮತ್ತು ಅದನ್ನು ಮಾಡಿ ಮುಗಿಸುವವರೆಗೆ ನನಗೆ ವಿಶ್ರಾಂತಿಯಿಲ್ಲ ಎಂದು ಕಂಡುಕೊಂಡೆ.” ನಿಜಕ್ಕೂ ಅವರಲ್ಲಿ ಅಪರಿ ಮಿತ ಆಧ್ಯಾತ್ಮಿಕ ಶಕ್ತಿ ತುಂಬಿಕೊಳ್ಳುತ್ತಿದ್ದುದನ್ನು ಅವರ ಗುರುಭಾಯಿಗಳು ಸ್ವಷ್ಟವಾಗಿ ಕಾಣುತ್ತಿ ದ್ದರು. ಅದು ಅವರೊಳಗೆ ಪ್ರಚಂಡ ಒತ್ತಡದಿಂದ ಹುದುಗಿದ್ದು, ಸೂಕ್ತ ರೀತಿಯಲ್ಲಿ ಹೊರ ಚಿಮ್ಮಲು ಚಡಪಡಿಸುತ್ತಿರುವಂತಿತ್ತು.

ಈ ಒಂದು ಅನುಭವದಿಂದಾಗಿ ಸೋದರಸಂನ್ಯಾಸಿಗಳಿಗೆ, ತಮ್ಮ ಜೀವನದಲ್ಲಿ ಸ್ವಾಮೀಜಿಯ ಪಾತ್ರ ಎಷ್ಟು ಮಹತ್ತರವಾದದ್ದು ಎಂಬ ಅರಿವುಂಟಾಯಿತು. ಶ್ರೀರಾಮಕೃಷ್ಣರಂತೂ ದೇಹತ್ಯಾಗ ಮಾಡಿ ಹೊರಟುಹೋಗಿದ್ದಾರೆ. ಈಗ ಸ್ವಾಮೀಜಿಯೂ ಹೊರಟುಬಿಟ್ಟರೆ ತಮ್ಮೆಲ್ಲರ ಗತಿ ಯೇನು! ಸಂಘದ ಗತಿಯೇನು! ಸ್ವಾಮೀಜಿ ಇಲ್ಲದೆ ಹೋದರೆ ತಾವು ಅನಾಥರೇ ಸರಿ. ಸ್ವಾಮೀಜಿ ಯೊಬ್ಬರಿಲ್ಲದಿದ್ದರೆ ಈ ಜಗತ್ತು ತಮ್ಮ ಪಾಲಿಗೆ ಕೇವಲ ಒಂದು ಮರಳುಗಾಡಾಗುತ್ತದೆ ಎಂಬುದು ಈ ಸೋದರ ಸಂನ್ಯಾಸಿಗಳ ಅರಿವಿಗೆ ಬಂತು.

ಈ ಮಧ್ಯೆ, ಟೆಹರಿಯ ಮಹಾರಾಜನೂ ಆತನ ದಿವಾನರಾದ ರಘುನಾಥ ಭಟ್ಟಾಚಾರ್ಯರೂ ಹೃಷೀಕೇಶದ ಮೂಲಕ ಅಜ್ಮೀರಕ್ಕೆ ಪ್ರಯಾಣ ಮಾಡುತ್ತಿದ್ದರು. ಸ್ವಾಮೀಜಿ ತೀವ್ರ ಅನಾರೋಗ್ಯ ದಿಂದ ಮಲಗಿದ್ದಾರೆಂದು ಕೇಳಿದ ದಿವಾನರು ಒಡನೆಯೇ ಬಂದು ನೋಡಿದರು. ಅವರು ಸ್ವಾಮೀಜಿಗೆ ದೆಹಲಿಯ ಒಬ್ಬ ಹಕೀಮರನ್ನು ಕಂಡು ಚಿಕಿತ್ಸೆಯನ್ನು ಪಡೆಯುವಂತೆ ಸೂಚಿಸಿದರು. ಮತ್ತು ಆ ಹಕೀಮರಿಗೆ ಒಂದು ಪರಿಚಯಪತ್ರವನ್ನೂ ಇವರ ಕುಟೀರದ ದುರಸ್ತಿಗಾಗಿ ಹಣವನ್ನೂ ಕೊಟ್ಟು ಬೀಳ್ಗೊಂಡರು.

ಆದರೆ ಸ್ವಾಮೀಜಿ ಹೃಷೀಕೇಶದಿಂದ ಏಕಾಂಗಿಯಾಗಿ ಹರಿದ್ವಾರಕ್ಕೆ ಬಂದರು–ತೀವ್ರ ಸಾಧನೆ ಯಲ್ಲಿ ತೊಡಗಿ, ಮನಸ್ಸನ್ನು ಸಮಾಧಿಯಲ್ಲಿ ಲೀನಗೊಳಿಸಿಬಿಡಬೇಕೆಂಬ ಅದೇ ಬಯಕೆಯಿಂದ. ಸ್ವಲ್ಪ ಆರೋಗ್ಯ ಸುಧಾರಿಸಿ ಶಕ್ತಿ ಬಂದರೆ ಸಾಕು, ಮತ್ತೆ ಸಾಧನೆಗೆ ಮನಸ್ಸು ಮಾಡುತ್ತಾರೆ! ಅವರ ಉತ್ಸಾಹ ಅಷ್ಟು ಅದಮ್ಯ. ಆ ಸಮಯದಲ್ಲಿ, ಹರಿದ್ವಾರಕ್ಕೆ ಮೂರು ಮೈಲಿ ದೂರದಲ್ಲಿ ರುವ ಕಂಖಲ್ ಎಂಬಲ್ಲಿ ತಪೋನಿರತರಾಗಿದ್ದ ಸ್ವಾಮಿ ಬ್ರಹ್ಮಾನಂದರು, ಸ್ವಾಮೀಜಿ ಬಂದಿರುವ ಸುದ್ದಿ ತಿಳಿದು ಹರಿದ್ವಾರಕ್ಕೆ ಬಂದರು. ಬಳಿಕ ಶಾರದಾನಂದರು, ಕೃಪಾನಂದರು ಹಾಗೂ ತುರೀಯಾನಂದರು–ಈ ಮೂವರು ಕೂಡ ಹರಿದ್ವಾರಕ್ಕೆ ಬಂದರು. ಇಲ್ಲಿಂದ ಎಲ್ಲರೂ ಒಟ್ಟಾಗಿ ದೆಹಲಿಗೆ ಹೊರಟರು. ಇಲ್ಲಿಗೆ, ಹಿಮಾಲಯದಲ್ಲಿ ವಾಸವಾಗಿದ್ದು ಧ್ಯಾನಾನಂದದಲ್ಲಿ ಲೀನರಾಗಿರ ಬೇಕೆಂಬ ಸ್ವಾಮೀಜಿಯ ಬಹುಕಾಲದ ಹಂಬಲ ಕೊನೆಗೊಂಡಂತಾಯಿತು.

ದಾರಿಯಲ್ಲಿ ಶಹರಾನ್ಪುರಕ್ಕೆ ಹೋದಾಗ, ಅವರಿಗೆ ಅಖಂಡಾನಂದರು ಮೀರತ್ತಿನಲ್ಲಿರುವ ವಿಷಯ ತಿಳಿಯಿತು. ಸರಿ; ಎಲ್ಲರೂ ಅವರನ್ನು ಅನುಸರಿಸಿಕೊಂಡು ಮೀರತ್ತಿಗೆ ಹೋದರು (೧೮೯೦ರ ಡಿಸೆಂಬರ್).

ತಮ್ಮ ಗುರುಭಾಯಿಗಳನ್ನೆಲ್ಲ ಕಂಡು ಅಖಂಡಾನಂದರಿಗೆ ಬಹಳ ಸಂತೋಷವಾಯಿತು. ಆದರೆ ತಮ್ಮ ನಾಯಕನಾದ ಸ್ವಾಮೀಜಿಯ ದೇಹದ ಮೇಲೆ ಕಾಯಿಲೆಯಿಂದಾಗಿದ್ದ ದುಷ್ಪರಿ ಣಾಮವನ್ನು ಕಂಡು ಅವರಿಗೆ ಆಘಾತವಾಯಿತು. ಸ್ವಾಮೀಜಿ ಅಷ್ಟು ಇಳಿದುಹೋದದ್ದನ್ನು ಅವರೆಂದೂ ಕಂಡಿರಲಿಲ್ಲ. ಸ್ವಾಮೀಜಿಯನ್ನು ನೋಡಿಕೊಳ್ಳಬೇಕಾಗಿದ್ದ ತಾವೇ ಹೀಗೆ ಕಾಯಿಲೆ ಬಿದ್ದದ್ದನ್ನು ನೆನೆಸಿಕೊಂಡು ಅಖಂಡಾನಂದರು ಅದೆಷ್ಟು ಪರಿತಪಿಸಿದರೊ!

ಈಗ ಅವರು ತಮ್ಮನ್ನು ಇರಿಸಿಕೊಂಡು ಶುಶ್ರೂಷೆ ಮಾಡುತ್ತಿದ್ದ ಡಾ. ತ್ರೈಲೋಕ್ಯನಾಥ ಘೋಷನ ಮನೆಯಲ್ಲೇ ಸ್ವಾಮೀಜಿಯ ವಾಸ್ತವ್ಯಕ್ಕೂ ಚಿಕಿತ್ಸೆಗೂ ಏರ್ಪಾಡು ಮಾಡಿದರು. ಉಳಿದವರು ಯಜ್ಞೇಶ್ವರ ಬಾಬು ಎಂಬುವರ ಮನೆಯಲ್ಲಿ ಇಳಿದುಕೊಂಡರು. ಕೆಲದಿನಗಳ ಬಳಿಕ ಸ್ವಾಮೀಜಿಯೂ ಸೇರಿದಂತೆ ಎಲ್ಲ ಸೋದರ ಸಂನ್ಯಾಸಿಗಳೂ ಸೇಠ್ಜಿಯೊಬ್ಬನ ಆಹ್ವಾನಕ್ಕೊಪ್ಪಿ ಆತನ ಉದ್ಯಾನಗೃಹದಲ್ಲಿ ಇಳಿದುಕೊಂಡರು. ಅವರ ಆತಿಥೇಯನಾದ ಸೇಠ್ಜಿ ಅವರಿಗೆ ಬೇಕಾದ ಅನುಕೂಲತೆಗಳನ್ನೆಲ್ಲ ಮಾಡಿಕೊಟ್ಟ. ಸ್ವಾಮೀಜಿ ತಮ್ಮ ಅನಾರೋಗ್ಯದಿಂದ ಇನ್ನೂ ಪೂರ್ಣ ಗುಣಮುಖರಾಗಿರಲಿಲ್ಲ. ಪರಿವ್ರಾಜಕ ದಿನಗಳಲ್ಲಿ ಕೈಗೊಂಡಿದ್ದ ತೀವ್ರ ಸಾಧನೆಗಳು, ನಿರಂತರ ಕಾಲ್ನಡಿಗೆ, ಸಕಾಲಕ್ಕೆ ಆಹಾರ-ವಿಶ್ರಾಂತಿಗಳಿಲ್ಲದಿದ್ದುದು; ಇವೆಲ್ಲದರ ಜೊತೆಗೆ ಆಗಾಗ ತಗಲಿ ಕೊಳ್ಳುತ್ತಿದ್ದ ಕಾಯಿಲೆಗಳು–ಇವುಗಳಿಂದ ಅವರ ಶರೀರ ಜರ್ಝರಿತವಾಗಿತ್ತು. ಮತ್ತೆ ಕಾಯಿಲೆ ಮರುಕಳಿಸದಂತೆ ಎಚ್ಚರವಹಿಸಿ ನಿಯಮಿತವಾಗಿ ಔಷಧ-ಆಹಾರಗಳನ್ನು ತೆಗೆದುಕೊಂಡದ್ದರಿಂದ ಅವರ ಆರೋಗ್ಯ ಸುಧಾರಿಸಿ, ಶರೀರ ದೃಢವಾಗುತ್ತ ಬಂದಿತು.

ಇದೇ ಸಮಯಕ್ಕೆ, ಪರಿವ್ರಾಜಕರಾಗಿ ಸುತ್ತಾಡುತ್ತಿದ್ದ ಸ್ವಾಮಿ ಅದ್ವೈತಾನಂದರು ಮೀರತ್ತಿಗೆ ಬಂದು ಗುರುಭಾಯಿಗಳನ್ನು ಕೂಡಿಕೊಂಡರು. ಈಗ ಸೇಠ್ಜಿಯ ಮನೆ ಮತ್ತೊಂದು ಬಾರಾ ನಗೋರ್ ಮಠವೇ ಆದಂತಾಯಿತು. ಏಕೆಂದರೆ, ಇಲ್ಲೀಗ ವಿವೇಕಾನಂದರು, ಬ್ರಹ್ಮಾನಂದರು, ಶಾರದಾನಂದರು, ತುರೀಯಾನಂದರು, ಅಖಂಡಾನಂದರು ಮತ್ತು ಅದ್ವೈತಾನಂದರು–ಹೀಗೆ ರಾಮಕೃಷ್ಣಸಂಘದ ಆರು ಜನ ಸೋದರ ಸಂನ್ಯಾಸಿಗಳು, ಮತ್ತವರ ಗುರುಭಾಯಿಯಾದ ಸ್ವಾಮಿ ಕೃಪಾನಂದರು–ಇಷ್ಟು ಜನ ಒಟ್ಟಾಗಿ ಸೇರಿದ್ದಾರೆ. ಸಹಜವಾಗಿಯೇ ಇಲ್ಲೊಂದು ಅದ್ಭುತ ಆಧ್ಯಾತ್ಮಿಕ ವಾತಾವರಣವೇರ್ಪಟ್ಟಿತು. ಧ್ಯಾನ-ಜಪ-ಪ್ರಾರ್ಥನೆ-ಭಜನೆಗಳಲ್ಲಿ ದಿನಗಳನ್ನು ಕಳೆ ದರು. ಅಲ್ಲದೆ ಸ್ವಾಮೀಜಿಯ ನೇತೃತ್ವದಲ್ಲಿ ಶಾಸ್ತ್ರಗ್ರಂಥಗಳನ್ನು, ಸಂಸ್ಕೃತ-ಇಂಗ್ಲಿಷ್ ಸಾಹಿತ್ಯದ ಹಲವಾರು ಶ್ರೇಷ್ಠ ಗ್ರಂಥಗಳನ್ನು ಅಧ್ಯಯಿಸಿದರು.

ತಮ್ಮ ಜ್ಞಾನಕೋಶವನ್ನು ವರ್ಧಿಸಿಕೊಳ್ಳಲು ಸ್ವಾಮೀಜಿ ಸದಾ ಕಾತರರಾಗಿರುತ್ತಿದ್ದರು. ಇದು ಅವರ ಹುಟ್ಟುಗುಣ. ಮೀರತ್ತಿನಲ್ಲಿದ್ದಾಗ ಅವರು, ಅಖಂಡಾನಂದರ ಮೂಲಕ ಸಮೀಪದ ಗ್ರಂಥಾಲಯವೊಂದರಿಂದ ಸರ್ ಜಾನ್ ಲುಬ್ಬಾಕ್ ಎಂಬವನ ಕೃತಿಗಳನ್ನು ತರಿಸಿಕೊಳ್ಳು ತ್ತಿದ್ದರು. ಈ ಗ್ರಂಥಗಳು ಗಾತ್ರದಲ್ಲೂ ವಿಷಯದಲ್ಲೂ ಬಹಳ ಹಿರಿದಾದವು. ಸ್ವಾಮೀಜಿ ಈ ಕೃತಿಗಳ ಸಂಪುಟಗಳನ್ನು ಪ್ರತಿದಿನ ಒಂದೊಂದಾಗಿ ತರಿಸಿಕೊಂಡು ಮರುದಿನ ಹಿಂದಿರುಗಿಸು ತ್ತಿದ್ದರು. ಇದನ್ನು ಗಮನಿಸುತ್ತಿದ್ದ ಗ್ರಂಥಾಲಯದ ಅಧಿಕಾರಿ ಭಾವಿಸಿದ–ಇವರ ಸ್ನೇಹಿತರು ಸುಮ್ಮನೆ ಓದುವ ನಟನೆ ಮಾಡುತ್ತಿರಬೇಕು ಎಂದು. ಒಂದು ದಿನ ಆತ ಅಸಹನೆಗೊಂಡು, ಅಖಂಡಾನಂದರನ್ನು ವ್ಯಂಗ್ಯವಾಗಿ ಕೇಳಿದ: “ಏನು, ನಿಮ್ಮ ಆ ಸ್ನೇಹಿತರು ಪುಸ್ತಕದ ಹೊಳೆಯುವ ಹೊರಕವಚವನ್ನಷ್ಟೇ ನೋಡಿ ಹಿಂದಿರುಗಿಸುವಂತೆ ಕಾಣುತ್ತದೆ?” ಆತ ಹೀಗೆಂದದ್ದನ್ನು ಕೇಳಿ ತಿಳಿದ ಸ್ವಾಮೀಜಿ ತಾವೇ ಬಂದು ಆ ಅಧಿಕಾರಿಯನ್ನು ಕಂಡು, “ಸ್ವಾಮಿ, ನೀವು ಕಳಿಸಿದ ಪುಸ್ತಕ ಗಳನ್ನೆಲ್ಲ ನಾನು ಓದಿದ್ದೇನೆ. ಇದರ ಬಗ್ಗೆ ನಿಮಗೇನಾದರೂ ಸಂದೇಹವಿದ್ದರೆ ನೀವು ಯಾವ ಪ್ರಶ್ನೆಯನ್ನಾದರೂ ಕೇಳಬಹುದು” ಎಂದರು. ನೋಡಿಯೇಬಿಡೋಣ ಎಂದು ಆತ ಹಲವಾರು ಪ್ರಶ್ನೆಗಳನ್ನು ಕೇಳಿದ; ಪ್ರತಿಯೊಂದಕ್ಕೂ ಅತ್ಯಂತ ಸಮರ್ಪಕವಾದ ಉತ್ತರ ಬಂದಿತು! ಆಗ ಆಶ್ಚರ್ಯದಿಂದ ಅವನ ಬಾಯಿ ಕಟ್ಟಿಹೋಯಿತು. ತನ್ನ ದುಡುಕುಮಾತಿಗಾಗಿ ಕ್ಷಮೆ ಯಾಚಿಸಿದ. ಅಖಂಡಾನಂದರಿಗೂ ಅವನಷ್ಟೇ ಆಶ್ಚರ್ಯವಾಗಿತ್ತು. ಆಮೇಲೆ ಅವರು ಇದು ಹೇಗೆ ಸಾಧ್ಯ ವಾಯಿತು ಎಂದು ಕೇಳಿದಾಗ ಸ್ವಾಮೀಜಿ ಹೇಳುತ್ತಾರೆ: “ನಾನು ಒಂದು ಪುಸ್ತಕವನ್ನು ಓದುವಾಗ ಒಂದೊಂದೇ ಪದವನ್ನು ಓದುವುದಿಲ್ಲ; ವಾಕ್ಯವಾಕ್ಯಗಳನ್ನೇ ಓದುತ್ತೇನೆ. ಕೆಲವೊಮ್ಮೆ ಪ್ಯಾರಾ ಪ್ಯಾರಾಗಳನ್ನೇ ಓದಿಕೊಂಡು ಹೋಗುತ್ತೇನೆ. ಎಂದರೆ, ಕಲೈಡೋಸ್ಕೋಪಿನ\footnote{*ಕಲೈಡೋಸ್ಕೋಪು ಎನ್ನುವುದೊಂದು ಮಕ್ಕಳ ಮನರಂಜನೆಯ ವಸ್ತು. ಅದರಲ್ಲಿ ಮೂರು ಕನ್ನಡಿಗಳನ್ನು ತ್ರಿಕೋನಾಕಾರವಾಗಿ ಇಟ್ಟು ಕೊಳವೆಯಲ್ಲಿ ಕೂಡಿಸಿರುತ್ತಾರೆ. ಕೊಳವೆಯ ಒಂದು ತುದಿಯಲ್ಲಿ ಬಳೆಯ ಚೂರಿ ನಂತಹ ಹೊಳೆಯುವ ಪದಾರ್ಥಗಳನ್ನು ಬಿದ್ದು ಹೋಗುದಂತೆ ಇಟ್ಟು. ಕೊಳವೆಯ ಇನ್ನೊಂದು ತುದಿಯಿಂದ ನೋಡಲು ಅವಕಾಶ ಮಾಡಿರುತ್ತಾರೆ. ಅಲ್ಲಿಂದ ನೋಡುತ್ತ ಕೊಳವೆಯನ್ನು ಮೆಲ್ಲನೆ ತಿರುಗಿಸಿದರೆ ಕನ್ನಡಿಗಳ ಪ್ರತಿಫಲನದಿಂದ ನೂರಾರು ವರ್ಣಪ್ರಭೇದಗಳುಂಟಾಗುವುದನ್ನು ಕಾಣಬಹುದು.} ರೂಪದಲ್ಲಿ ಓದಿ ಕೊಂಡು ಹೋಗುತ್ತೇನೆ.”

ಇದೊಂದು ಅತ್ಯಂತ ಅಚ್ಚರಿಯ ಸಂಗತಿಯೆಂಬುದರಲ್ಲಿ ಸಂದೇಹವೇ ಇಲ್ಲ. ಸಾಮಾನ್ಯವಾದ ಒಂದು ‘ಚಂದಮಾಮ’ ಓದಿಯೇ ಎಲ್ಲವನ್ನೂ ನೆನಪಿಟ್ಟುಕೊಳ್ಳುವುದು ಎಷ್ಟು ಕಷ್ಟ! ಹಾಗಿರುವಾಗ ಅಂತಹ ಕ್ಲಿಷ್ಟ ವಿಚಾರಗಳನ್ನೊಳಗೊಂಡ ಬೃಹತ್ ಗ್ರಂಥಗಳನ್ನು ಓದಿ ನೆನಪಿಟ್ಟು ಕೊಳ್ಳುವುದು ಸಾಧ್ಯವೆಂದರೆ! ಅದೂ ಒಂದೇ ಸಲ ಓದಿದ್ದು. ಒಂದೇ ದಿನದಲ್ಲಿ ಓದಿ ಮುಗಿ ಸಿದ್ದು! ಆದರೆ ಇತರರ ದೃಷ್ಟಿಗೆ ಇದೊಂದು ಪರಮಾಶ್ಚರ್ಯದ ವಿಷಯವಾದರೂ ವಿವೇಕಾ ನಂದರಿಗೆ ಈ ಅದ್ಭುತ ಗ್ರಹಣ ಶಕ್ತಿಯೂ ಧಾರಣಶಕ್ತಿಯೂ ಸ್ವಭಾವಸಹಜವಾಗಿದ್ದ ಗುಣಗಳು. ಇವುಗಳನ್ನು ಅವರು ಬಾಲ್ಯದಿಂದಲೂ ಬೆಳೆಸಿಕೊಂಡು ಬಂದಿದ್ದನ್ನು ನೋಡಿದ್ದೇವೆ. ಅವರ ಜೀವನದ ಕೊನೆಯವರೆಗೂ ಈ ಶಕ್ತಿ ಕಿಂಚಿತ್ತೂ ಮಾಸಲಿಲ್ಲ. ಇದನ್ನು ತೋರಿಸುವ ಕೆಲವು ಘಟನೆಗಳನ್ನು ಇಲ್ಲಿ ಉಲ್ಲೇಖಿಸಬಹುದು.

ಶ್ರೀರಾಮಕೃಷ್ಣರು ತೀವ್ರ ಕಾಯಿಲೆಯಿಂದ ಹಾಸಿಗೆ ಹಿಡಿದಿದ್ದ ಸಂದರ್ಭ. ಅವರನ್ನು ನೋಡಿ ಕೊಳ್ಳುತ್ತಿದ್ದ ಡಾ.ಮಹೇಂದ್ರಲಾಲ ಸರ್ಕಾರ್ ಆ ಕಾಲದ ಪ್ರಸಿದ್ಧ ಹೋಮಿಯೋ ವೈದ್ಯರ ಲ್ಲೊಬ್ಬ. ಅಲ್ಲದೆ ಅಸಾಧಾರಣ ಬುದ್ಧಿಶಾಲಿ ಹಾಗೂ ಜ್ಞಾನಿ. ಒಮ್ಮೆ ಶ್ರೀರಾಮಕೃಷ್ಣರು ಆತನಿಗೆ ನರೇಂದ್ರನೊಡನೆ ಮಾತುಕತೆ ನಡೆಸುವಂತೆ ಹೇಳಿದರು. ಯಾವುದೋ ವಿಷಯವಾಗಿ ಇಬ್ಬರೂ ವಾದ-ವಿವಾದ ನಡೆಸಿದರು. ಮಾತಿನ ಸಂದರ್ಭದಲ್ಲಿ ಸರ್ಕಾರ ತನ್ನ ವಾದವನ್ನು ಸಮರ್ಥಿಸುವ ಒಂದು ಪುಸ್ತಕವನ್ನು ಹೆಸರಿಸಿದ. ತಕ್ಷಣ ನರೇಂದ್ರ ಕೇಳಿದ, “ಮಹಾಶಯರೆ, ನೀವು ಆ ಪುಸ್ತಕವನ್ನು ಆಮೂಲಾಗ್ರವಾಗಿ ಓದಿದ್ದೀರೋ ಅಥವಾ ಸುಮ್ಮನೆ ಮೇಲುಮೇಲೆ ನೋಡಿ ತಿರುವಿ ಹಾಕಿದ್ದೀರೋ ಎಂದು ಕೇಳಬಹುದೇ?”ಸರ್ಕಾರ ಅವಾಕ್ಕಾಗಿಬಿಟ್ಟ. ಆದರೆ ಮರುಕ್ಷಣವೇ, “ಇಲ್ಲ, ಆ ಪುಸ್ತಕವನ್ನು ನಾನು ಸಂಪೂರ್ಣವಾಗಿ ಓದಿಲ್ಲ” ಎಂದು ಒಪ್ಪಿಕೊಂಡ. ಆಗ ನರೇಂದ್ರ, ಆ ಪುಸ್ತಕದಲ್ಲಿ ಪ್ರತಿಪಾದಿಸಲಾಗಿರುವ ಅಭಿಪ್ರಾಯ ಅದಲ್ಲ ಎಂದು ಹೇಳಿ. ನಿಜಾಂಶವನ್ನು ಎತ್ತಿತೋರಿಸಿದ. ಮಹೇಂದ್ರಲಾಲ ವಿಸ್ಮಯಾನಂದಗೊಂಡು, ನರೇಂದ್ರನನ್ನು ಹೃತ್ಪೂರ್ವಕವಾಗಿ ಹರಸಿದ.

ಮುಂದೆ ಬೇಲೂರು ಮಠ ಸ್ಥಾಪನೆಯಾದ ಮೇಲಿನ ಮಾತಿದು. ಮಠದ ಗ್ರಂಥಾಯಲಕ್ಕೆ ಸುಪ್ರಸಿದ್ಧ ವಿಶ್ವಕೋಶ ‘ಎನ್ಸೈಕ್ಲೋಪೀಡಿಯ ಬ್ರಿಟಾನಿಕ’ದ ಹೊಸ ಆವೃತ್ತಿಯ ಸಂಪುಟಗಳನ್ನು ತರಿಸಲಾಗಿತ್ತು. ಒಂದು ದಿನ ಸ್ವಾಮೀಜಿ ತಮ್ಮ ಶಿಷ್ಯ ಶರಚ್ಚಂದ್ರ ಚಕ್ರವರ್ತಿಯೊಂದಿಗೆ ಗ್ರಂಥಾಲಯದಲ್ಲಿ ಅಡ್ಡಾಡುವಾಗ, ಈ ಸಂಪುಟಗಳನ್ನು ನೋಡಿದ ಶಿಷ್ಯ ಅಚ್ಚರಿಯಿಂದ, “ಇವು ಗಳನ್ನು ಓದಿ ಮುಗಿಸಲು ಒಂದು ಜೀವಮಾನ ಸಾಲದು!” ಎಂದುದ್ಗರಿಸಿದ. ಸ್ವಾಮೀಜಿ ಅವನತ್ತ ತಿರುಗಿ, “ಏಕೆ ಸಾಧ್ಯವಿಲ್ಲ? ಮೊದಲ ಹತ್ತು ಸಂಪುಟಗಳಲ್ಲಿ ಯಾವ ವಿಷಯದ ಬಗ್ಗೆಯಾದರೂ ನನ್ನನ್ನು ಪ್ರಶ್ನಿಸು?” ಎಂದರು. ದಂಗಾದ ಶಿಷ್ಯ ಕೇಳಿದ: “ಏನು! ಆಗಲೇ ಹತ್ತು ಸಂಪುಟಗಳನ್ನು ಓದಿಬಿಟ್ಟಿರಾ!” ಹೌದೆಂದ ಸ್ವಾಮೀಜಿ ಅವನಿಗೆ ಮತ್ತೆ ಸವಾಲೊಡ್ಡಿದರು. ಆಗ ಶರಚ್ಚಂದ್ರ ಆ ಹತ್ತು ಸಂಪುಟಗಳಲ್ಲಿನ ಹಲವಾರು ಕ್ಲಿಷ್ಟ ವಿಚಾರಗಳನ್ನು ಮಧ್ಯಮಧ್ಯದಲ್ಲಿ ಆಯ್ದು ಪ್ರಶ್ನೆಗಳನ್ನು ಕೇಳಿದ. ಸ್ವಾಮೀಜಿ ಅವುಗಳಿಗೆಲ್ಲ ಕರಾರುವಾಕ್ಕಾಗಿ ಉತ್ತರಿಸಿದರಲ್ಲದೆ, ಕೆಲವೊಮ್ಮೆ ಪಂಕ್ತಿಪಂಕ್ತಿ ಗಳನ್ನೇ ಪುಟಪುಟಗಳನ್ನೇ ಬಾಯಿಪಾಠ ಮಾಡಿದಂತೆ ಒಪ್ಪಿಸಿಬಿಟ್ಟರು!ಶರಚ್ಚಂದ್ರನಿಗೆ ಕಣ್ಣಾರೆ ಕಂಡರೂ ನಂಬಲಾಗದಂತಹ ಅವಸ್ಥೆ! ಅವರ ಈ ಅತಿಮಾನುಷ ಸ್ಮರಣಶಕ್ತಿಯ ಗುಟ್ಟೇನು ಎಂದು ಪ್ರಶ್ನಿಸಿದಾಗ, ಸ್ವಾಮೀಜಿ “ಕಟ್ಟುನಿಟ್ಟಿನ ಬ್ರಹ್ಮಚರ್ಯ ಹಾಗೂ ಅಭ್ಯಾಸದಿಂದ ಅದನ್ನು ಸಾಧಿಸಬಹುದು” ಎಂದುತ್ತರಿಸಿದರು. ಆದರೆ ಶರಚ್ಚಂದ್ರನೆಂದ, “ಕೇವಲ ಬ್ರಹ್ಮಚರ್ಯದಿಂದ ಇದು ಸಾಧ್ಯವೆಂದು ಒಪ್ಪಿಕೊಳ್ಳಲಾಗುವುದಿಲ್ಲ. ಮತ್ತೂ ಇನ್ನೇನೋ ವಿಶೇಷ ಇರಲೇಬೇಕು...” ಸ್ವಾಮೀಜಿ ಪ್ರತಿನುಡಿಯಲಿಲ್ಲ.

ಸ್ವಾಮೀಜಿಯವರ ಜೀವನದಲ್ಲಿ ಇಂತಹ ಘಟನೆಗಳು ಹಲವಾರು. ಇವುಗಳಲ್ಲಿ ಒಂದನ್ನು ಮುಂದೆ ಅವರು ಬೆಳಗಾವಿಗೆ ಬಂದಾಗ ಹಾಗೂ ಮತ್ತೆ ಕೆಲವನ್ನು ಅವರ ಜೀವನದ ಇತರ ಸಂದರ್ಭಗಳಲ್ಲಿ ನೋಡಲಿದ್ದೇವೆ.

ಅವರು ತಮ್ಮ ಗುರುಭಾಯಿಗಳೊಂದಿಗೆ ಈ ವೇಳೆಗಾಗಲೇ ಸಾಕಷ್ಟು ಸಮಯವನ್ನು ಕಳೆದಿದ್ದರು. ಅಖಂಡಾನಂದರಂತೂ ಬಾರಾನಗೋರನ್ನು ಬಿಟ್ಟಾಗಿನಿಂದಲೂ ಅವರ ಜೊತೆಗಾರ ರಾಗಿಯೇ ಇದ್ದರು. ಈಗ ಹಲವಾರು ತೀರ್ಥಕ್ಷೇತ್ರಗಳನ್ನು ಸಂದರ್ಶಿಸಿ, ತೀವ್ರ ಆಧ್ಯಾತ್ಮಿಕ ಸಾಧನೆ ಗಳನ್ನು ನಡೆಸಿದ ಮೇಲೆ ಸ್ವಾಮೀಜಿಯ ಮನಸ್ಸು ಬೇರೊಂದು ರೀತಿಯಲ್ಲಿ ಚಿಂತಿಸಲಾರಂಭಿಸಿತು. ಮಠದ ಹಾಗೂ ಸೋದರಸಂನ್ಯಾಸಿಗಳೊಂದಿಗಿನ ಬಂಧನಗಳನ್ನು ತಾತ್ಕಾಲಿಕವಾಗಿ ಕಡಿದುಕೊಳ್ಳ ಬೇಕೆಂದೇ ಅವರು ಬಾರಾನಗೋರಿನಿಂದ ಹೊರಟದ್ದು. ಆದರೆ ಈ ಗುರುಭಾಯಿಗಳು ಅವರೊ ಡನೆ ಮತ್ತೆ ಸೇರಿದ್ದಾರೆ. ಆದ್ದರಿಂದ ಇವರೆಲ್ಲರಿಂದಲೂ ಬಿಡಿಸಿಕೊಂಡು ತಾವು ಏಕಾಂಗಿಯಾಗಿ ಮುನ್ನಡೆಯಬೇಕೆಂದು ಸ್ವಾಮೀಜಿ ದೃಢನಿಶ್ಚಯ ಮಾಡಿಬಿಟ್ಟರು. ಹಿಮಾಲಯದ ಗುಹೆಯಲ್ಲಿ ಕುಳಿತು ಸಮಾಧಿಸುಖ ಅನುಭವಿಸುವ ಯೋಗವು ತಮಗಿಲ್ಲವೆಂಬುದು ಅವರಿಗೆ ಈ ವೇಳೆಗೆ ಮನದಟ್ಟಾಗಿತ್ತೆಂದು ತೋರುತ್ತದೆ. ಕಡೆಗೊಂದು ದಿನ ಅವರು ತಮ್ಮ ಗುರುಭಾಯಿಗಳಿಗೆ, “ನಾನೀಗ ಇಲ್ಲಿಂದ ಏಕಾಂಗಿಯಾಗಿ ಹೊರಡಲಿದ್ದೇನೆ. ನನಗೆ ನನ್ನ ಮುಂದಿನ ಕಾರ್ಯಯೋಜನೆ ಏನೆಂಬುದು ತಿಳಿದುಬಂದಿದೆ... ನನ್ನನ್ನು ನೀವಾರೂ ಹಿಂಬಾಲಿಸಕೂಡದು” ಎಂದು ಹೇಳಿಬಿಟ್ಟರು.

ಆದರೆ ಅವರು ಹೀಗೆ ತಮ್ಮನ್ನೆಲ್ಲ ಬಿಟ್ಟು ಹೊರಟುಹೋಗುವುದು ಅವರ ಗುರುಭಾಯಿಗಳಾ ರಿಗೂ ಸ್ವಲ್ಪವೂ ಇಷ್ಟವಿಲ್ಲ. ಅವರ ಸ್ನೇಹಿತ, ಮಾರ್ಗದರ್ಶಕ, ಗುರು–ಎಲ್ಲವೂ ಸ್ವಾಮೀಜಿಯೇ. ಅಖಂಡಾನಂದರಂತೂ, ಸ್ವಾಮೀಜಿ ಹೀಗೆ ಹೇಳಬಹುದು ಎಂದು ಹೆದರಿಕೊಂಡೇ ಇದ್ದರು. ತಮ್ಮ ಪ್ರಾಣಸಮಾನರಾದ ಸ್ವಾಮೀಜಿಯನ್ನು ಬಿಟ್ಟುಕೊಡಲು ಅವರು ಹೇಗೆತಾನೆ ಒಪ್ಪಿಯಾರು? ಅಲ್ಲದೆ ‘ನರೇಂದ್ರನ ಯೋಗಕ್ಷೇಮದ ಹೊಣೆ ನಿನ್ನದು’ ಎಂದು ಶ್ರೀಶಾರದಾದೇವಿಯವರು ಬೇರೆ ವಿಶೇಷವಾಗಿ ಹೇಳಿದ್ದಾರೆ. ಆದ್ದರಿಂದ ಅಖಂಡಾನಂದರು ಈಗ ಅವರನ್ನು ಏನಾದರೂ ಮಾಡಿ ತಡೆಯಲು ಒಂದು ನೆಪ ತೆಗೆದು, “ನರೇನ್, ನೀನು ಕರೆದೆಯಲ್ಲ ಅಂತ ನಾನು ಮಧ್ಯ ಏಷ್ಯಾಗೆ ಹೋಗುವ ಕಾರ್ಯಕ್ರಮವನ್ನು ಬಿಟ್ಟು ಬಾರಾನಗೋರಿಗೆ ಬಂದೆ. ಈಗ ನೋಡಿದರೆ ನೀನು ನನ್ನನ್ನು ಇಲ್ಲಿ ಬಿಟ್ಟು ಹೊರಟುಹೋಗುತ್ತಿದ್ದೀಯಲ್ಲ!” ಎಂದರು. ಆದರೆ ಸ್ವಾಮೀಜಿ ಆ ವಾದವನ್ನು ಒಮ್ಮೆಗೇ ತಳ್ಳಿಹಾಕಿಬಿಟ್ಟರು: “ನೋಡು, ಸೋದರಸಂನ್ಯಾಸಿಗಳ ಸಂಗ ಎನ್ನು ವುದು ಆಧ್ಯಾತ್ಮಿಕ ಸಾಧನೆಗೆ ಒಂದು ಪ್ರತಿಬಂಧಕ. ಈಗ ನೀನೇ ನೋಡು, ನೀನು ಟೆಹರಿಯಲ್ಲಿ ಕಾಯಿಲೆ ಬಿದ್ದದ್ದರಿಂದ ತಾನೆ ನನಗೆ ಸಾಧನೆಯಲ್ಲಿ ತೊಡಗಲು ಸಾಧ್ಯವಾಗಲಿಲ್ಲ! ಈ ಸೋದರಸಂನ್ಯಾಸಿಗಳು ಎನ್ನುವ ಮಾಯಾಪಾಶವನ್ನು ಕತ್ತರಿಸಿ ಹಾಕದಿದ್ದರೆ ಆಧ್ಯಾತ್ಮಿಕ ಸಾಧನೆ ಅಸಾಧ್ಯ. ನಾನು ಸಾಧನೆ ಮಾಡಬೇಕು ಎಂದು ಹೊರಟಾಗಲೆಲ್ಲ ಶ್ರೀರಾಮಕೃಷ್ಣರು ಏನಾದ ರೊಂದು ಅಡ್ಡಿಯನ್ನು ತಂದೊಡ್ಡುತ್ತಾರೆ! ಇನ್ನು ನಾನು ಹೊರಟುಬಿಟ್ಟೆ. ನಾನು ಒಬ್ಬನೇ ಹೋಗುತ್ತೇನೆ. ಎಲ್ಲಿಗೆ ಹೋಗುತ್ತೇನೆ, ಎಲ್ಲಿರುತ್ತೇನೆ ಎನ್ನುವುದನ್ನು ಯಾರಿಗೂ ಹೇಳುವುದಿಲ್ಲ.” ಅಖಂಡಾನಂದರು ಅದನ್ನೊಂದು ಸವಾಲಿನಂತೆ ಪರಿಗಣಿಸಿ, ಅರ್ಧಹಾಸ್ಯವಾಗಿ ನುಡಿದರು: “ಇರಲಿ, ನೀನು ಪಾತಾಳಲೋಕಕ್ಕೆ ಹೋದರೂ ಸರಿಯೆ, ನಾನು ನಿನ್ನನ್ನು ಪತ್ತೆಹಚ್ಚದೆ ಬಿಡು ವುದಿಲ್ಲ, ನೋಡುತ್ತಿರು.” ಕಡೆಗೆ ೧೮೯೧ರ ಜನವರಿಯ ಕೊನೆಯ ವಾರದಲ್ಲೊಂದು ದಿನ, ಸ್ವಾಮೀಜಿ ತಮ್ಮ ಗುರುಭಾಯಿಗಳಿಗೆ ವಿದಾಯ ಹೇಳಿ, ಎಲ್ಲಿಗೆ ಎಂದು ತಿಳಿಸದೆ ಮೀರತ್ತಿನಿಂದ ಏಕಾಂಗಿಯಾಗಿ ಹೊರಟರು.

