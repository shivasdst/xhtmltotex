
\chapter{ಸಂಕಟಗಳ ಸುಳಿಯಲ್ಲಿ}

\noindent

ಈಗ ನರೇಂದ್ರ ತನ್ನ ಬಿ.ಎ. ಪದವಿಯ ಕೊನೆಯ ವರ್ಷದಲ್ಲಿದ್ದಾನೆ. ಈವರೆಗೂ ಅವನ ಜೀವನ ಯಾತ್ರೆ ಸುಗಮ-ಸುಖಕರವಾಗಿಯೇ ಸಾಗಿದೆ. ನಾವೀಗಾಗಲೇ ನೋಡಿದಂತೆ, ಯಾವುದಕ್ಕೂ ಕೊರತೆಯಿಲ್ಲದ ಸಂತೃಪ್ತ ಕುಟುಂಬ ಅವನದು. ಅವನ ತಂದೆ ವಿಶ್ವನಾಥದತ್ತ ಸಮಾಜದಲ್ಲಿ ತುಂಬ ಪ್ರಭಾವಶಾಲಿ ವ್ಯಕ್ತಿ. ವಕೀಲಿವೃತ್ತಿಯಲ್ಲಿ ಸಾಕಷ್ಟು ಯಶಸ್ಸನ್ನೂ ಧನವನ್ನೂ ಗಳಿಸಿ ಶ್ರೀಮಂತ ಬದುಕನ್ನು ಬಾಳುತ್ತಿದ್ದವನು. ಆದರೆ ಅವನು ಅಷ್ಟೊಂದು ಸಂಪಾದಿಸಿದರೂ ಹತ್ತಾರು ಜನರಿಂದ ಕೂಡಿದ ತನ್ನ ಅವಿಭಕ್ತ ಕುಟುಂಬದ ನಿರ್ವಹಣೆಗಾಗಿ, ಮನೆಯಲ್ಲಿ ಸಂಗೀತ ಕಛೇರಿ ಗಳನ್ನು ಏರ್ಪಡಿಸುವುದಕ್ಕಾಗಿ, ಸ್ನೇಹಿತರನ್ನು-ಬಂಧುಗಳನ್ನು ಸತ್ಕರಿಸುವುದಕ್ಕಾಗಿ, ಕೇಳಿದವರಿಗೆ ಸಹಾಯ ಕೊಡುವುದಕ್ಕಾಗಿ ಕೈಬಿಚ್ಚಿ ಖರ್ಚುಮಾಡುತ್ತಿದ್ದ. ಆದ್ದರಿಂದ ಅವನ ಸಂಪಾದನೆಯೆಲ್ಲ ಅಲ್ಲಲ್ಲಿಗೆ ಸರಿಹೋಗಿಬಿಡುತ್ತಿತ್ತು. ಸಾಲದ್ದಕ್ಕೆ, ಅವನ ಚಿಕ್ಕಪ್ಪನಾದ ಕಾಳೀಪ್ರಸಾದದತ್ತ ಕೂಡ ಯಾವ ಉದ್ಯೋಗವನ್ನೂ ಮಾಡದೆ ಅವನ ಮನೆಯಲ್ಲೇ ತಳವೂರಿದ್ದ. ಈತ ಪಿತ್ರಾರ್ಜಿತ ಸಂಪತ್ತನ್ನು ಕರಗಿಸಿ ತಿನ್ನುವುದಲ್ಲದೆ ವಿಶ್ವನಾಥನ ಸಂಪಾದನೆಯಲ್ಲೂ ಒಂದಿಷ್ಟನ್ನು ಕಸಿದು ಕೊಳ್ಳುತ್ತಿದ್ದ. ಚಿಕ್ಕಪ್ಪನ ಜೊತೆಯಲ್ಲಿ ಜಗಳವೇಕೆ ಎಂದು ವಿಶ್ವನಾಥ ಅವನು ಕೇಳಿದಷ್ಟನ್ನು ಕೊಟ್ಟುಬಿಡುತ್ತಿದ್ದ.

ತನ್ನ ವ್ಯವಹಾರವನ್ನು ನಿಭಾಯಿಸಲು ವಿಶ್ವನಾಥ ಒಂದು ವಕೀಲಿ ಸಂಸ್ಥೆಯನ್ನು ಕಟ್ಟಿದ್ದ. ಇದರಲ್ಲಿ ಇನ್ನೂ ಕೆಲವು ವಕೀಲರು ಅವನೊಂದಿಗೆ ಪಾಲುದಾರರಾಗಿ ಇದ್ದರು. ಮೊದಮೊದಲು ಇವರೊಳಗಿನ ವ್ಯವಹಾರ ಚೆನ್ನಾಗಿಯೇ ನಡೆಯಿತು. ಆದರೆ ವಿಶ್ವನಾಥನು ಕಾರ್ಯನಿಮಿತ್ತವಾಗಿ ಪರವೂರುಗಳಿಗೆ ಹೋಗಿದ್ದಾಗ ಈ ಪಾಲುದಾರರು ಸಂಸ್ಥೆಯ ಹೆಸರಿನಲ್ಲಿ ಧಾರಾಳವಾಗಿ ಸಾಲ ಮಾಡಿ ಆ ಹಣವನ್ನೆಲ್ಲ ತಿಂದುಹಾಕಿದರು. ಅವನ ಸಂಸ್ಥೆ ಯಾವ ಕ್ಷಣದಲ್ಲಾದರೂ ದಿವಾಳಿ ಏಳುವ ಸ್ಥಿತಿಗೆ ತಲುಪುತ್ತಿರುವುದು ಬೇರೆ ಯಾರಿಗೂ ಗೊತ್ತಾಗಿರಲಿಲ್ಲ. ಸ್ವತಃ ವಿಶ್ವನಾಥನಿಗೂ ಇದು ಗೊತ್ತಾದದ್ದು ತೀರ ತಡವಾಗಿ. ಅವನ ಮನೆಮಂದಿಗಂತೂ ಈ ಸಾಲಗಳ ವಿಷಯವೆಲ್ಲ ತಿಳಿದುಬಂದದ್ದು ಅವನು ತೀರಿಕೊಂಡಮೇಲೆಯೇ. ಅವನೊಬ್ಬನ ಸಂಪಾದನೆಯ ಬಲದಿಂದ ಸಂಸಾರ ರಥ ಹೇಗೋ ನಡೆಯುತ್ತಿತ್ತು. ಆದರೆ ಮೇಲ್ನೋಟಕ್ಕೆ ಮಾತ್ರ ಅವರು ಹಿಂದಿನ ಶ್ರೀಮಂತ ಜೀವನವನ್ನೇ ನಡೆಸುತ್ತಿರುವಂತಿತ್ತು. ಆದರೆ ಈ ಮಧ್ಯೆ ದುರ್ಗಾಪ್ರಸಾದನ ಅಲ್ಪಸ್ವಲ್ಪ ಪಿತ್ರಾರ್ಜಿತ ಆಸ್ತಿಯ ಸಲುವಾಗಿ ಜಗಳವೆದ್ದು ಕೊನೆಗೆ ಆ ಮನೆಯನ್ನು ಬಿಟ್ಟು ಹೊರಬರ ಬೇಕಾಯಿತು. ಈಗ ವಿಶ್ವನಾಥ ತನ್ನ ಹೆಂಡತಿ-ಮಕ್ಕಳೊಂದಿಗೆ ಒಂದು ಬಾಡಿಗೆ ಮನೆಯಲ್ಲಿ ವಾಸಿಸಲಾರಂಭಿಸಿದ.

ಈ ನಡುವೆ ನರೇಂದ್ರ ಎರಡನೆಯ ವರ್ಷದ ಬಿ.ಎ. ತರಗತಿಯಲ್ಲಿ ಓದುತ್ತಿದ್ದಾಗಲೇ ಬಿ.ಎಲ್. ತರಗತಿಗೂ ಸೇರಿಕೊಂಡಿದ್ದ. ಮಗನನ್ನು ಒಬ್ಬ ಸಮರ್ಥ ವಕೀಲನನ್ನಾಗಿ ಮಾಡುವ ಉದ್ದೇಶದಿಂದ ವಿಶ್ವನಾಥ ಅವನನ್ನು ನಿಮಾಯಿಚಂದ್ರ ಬಸು ಎಂಬುವನ ವಕೀಲಿ ಸಂಸ್ಥೆಗೆ ಸೇರಿಸಿದ. ಅವನನ್ನು ತನ್ನ ಸಂಸ್ಥೆಯಲ್ಲೇ ಸೇರಿಸಿಕೊಳ್ಳಬಹುದಿತ್ತು; ಆದರೆ ಸ್ವಂತದವರ ಮಾತು-ಮಾರ್ಗದರ್ಶನಗಳಿಗಿಂತ ಇತರರದು ಹೆಚ್ಚು ಪರಿಣಾಮಕಾರಿ ಎಂದು ವಿಶ್ವನಾಥ ಭಾವಿ ಸಿದನೇನೋ. ನರೇಂದ್ರನನ್ನು ವಕೀಲಿಯಲ್ಲಿ ವಿಶೇಷ ಶಿಕ್ಷಣಕ್ಕಾಗಿ ಇಂಗ್ಲೆಂಡಿಗೆ ಕಳಿಸುವ ಅಭಿಪ್ರಾಯವೂ ಇತ್ತು. ಇದಕ್ಕೆ ನರೇಂದ್ರನ ಸಮ್ಮತಿಯೂ ಇದ್ದಂತಿತ್ತು. ಅದರೆ ಅಷ್ಟರಲ್ಲೇ ಆಕಸ್ಮಿಕವೊಂದು ಸಂಭವಿಸಿ ಅದರ ಸಾಧ್ಯತೆಯೇ ಇಲ್ಲದಂತಾಯಿತು.

ನರೇಂದ್ರ ಅತ್ತ ಶ್ರೀರಾಮಕೃಷ್ಣರ ಪ್ರಭಾವಕ್ಕೆ ಸಿಲುಕಿ ತೀವ್ರ ಆಧ್ಯಾತ್ಮಿಕ ಸಾಧನೆಯಲ್ಲಿ ನಿರತನಾಗಿದ್ದುದಲ್ಲದೆ, ಇತ್ತ ತಂದೆಯ ಆಶಯಕ್ಕೆ ಅನುಸಾರವಾಗಿ ವಕೀಲಿ ವೃತ್ತಿಯ ಸಾಧನೆಯ ಕಡೆಗೂ ಗಮನ ಕೊಡಲಾರಂಭಿಸಿದ. ಇಷ್ಟರಲ್ಲೇ ವಿಶ್ವನಾಥ ಪ್ರಾಪ್ತವಯಸ್ಕನಾದ ಮಗನಿಗೆ ಮದುವೆ ಮಾಡುವ ಪ್ರಯತ್ನದಲ್ಲಿ ಮತ್ತೆ ತೊಡಗಿದ. ಈ ಹಿಂದೆಯೇ ಮದುವೆಯ ಪ್ರಸ್ತಾಪ ಬಂದಿದ್ದಾಗ ನರೇಂದ್ರ ಅದನ್ನು ಖಡಾಖಂಡಿತವಾಗಿ ತಳ್ಳಿಹಾಕಿದ್ದನ್ನು ನೋಡಿದ್ದೇವೆ. ಆದರೆ ವಿಶ್ವನಾಥ ಆ ವಿಷಯದಲ್ಲಿನ್ನೂ ನಿರಾಶನಾಗಿರಲಿಲ್ಲ. ಮದುವೆಯೊಂದು ಆಗಿಬಿಟ್ಟರೆ ದೋಣಿಗೆ ಲಂಗರು ಹಾಕಿದಂತೆ ಮಗ ಸಂಸಾರದ ಕಡೆ ಗಮನ ಕೊಡುತ್ತ ಮನೆಯಲ್ಲೇ ಇರುವಂತಾಗುತ್ತದೆ, ಹೊಣೆಗಾರಿಕೆಯ ಅರಿವು ಮೂಡುವಂತಾಗುತ್ತದೆ, ಆಗ ಅವನ ಹಾರಾಟಗಳೆಲ್ಲ ತಾನಾಗಿಯೇ ನಿಂತುಹೋಗುತ್ತವೆ ಎಂಬುದು ಅವನ ಆಲೋಚನೆ. ಆದರೆ ನರೇಂದ್ರನ ಪ್ರತಿಭಟನೆಯ ವಿಷಯ ಹಾಗಿರಲಿ, ಅವನ ಮದುವೆಯ ಪ್ರಸ್ತಾಪ ಬಂದಾಗಲೆಲ್ಲ ಯಾವುದೋ ಒಂದು ಅನಿರೀಕ್ಷಿತ ಕಷ್ಟವೋ ಅನನುಕೂಲತೆಯೋ ತಲೆದೋರಿ ಅದು ಅಲ್ಲಿಗೇ ನಿಂತುಹೋಗುತ್ತಿತ್ತು. ಆದರೆ ಇದರಲ್ಲಿ ಅಂತಹ ಆಶ್ಚರ್ಯವೇನಿಲ್ಲ ಎನ್ನಬೇಕು. ಏಕೆಂದರೆ, ಮದುವೆಯ ಪ್ರಸ್ತಾಪವೊಂದು ಬಂದಿದೆ ಎಂದು ಗೊತ್ತಾಗುತ್ತಲೇ ಇತ್ತ ಶ್ರೀರಾಮಕೃಷ್ಣರು ವ್ಯಾಕುಲರಾಗಿ, ‘ಅಮ್ಮಾ, ಈ ಮದುವೆ ನಡೆಯದಂತೆ ಮಾಡಿಬಿಡು’ ಎಂದು ಪ್ರಾರ್ಥಿಸುತ್ತಿದ್ದರು! ಅವರ ಪ್ರಾರ್ಥನೆಗೆ ಜಗನ್ಮಾತೆ ಇಲ್ಲವೆಂದಾಳೆಯೆ? ಶ್ರೀರಾಮಕೃಷ್ಣರ ಈ ವರ್ತನೆಯನ್ನು ಕಂಡು ಸಾಮಾನ್ಯ ಪ್ರಾಪಂಚಿಕರು, ‘ಅಲ್ಲ, ಭುವನೇಶ್ವರಿ ದೇವಿ ಪೂಜೆ-ವ್ರತ-ತಪಸ್ಸು ಮಾಡಿ ನರೇಂದ್ರನಂತಹ ಗಂಡುಸಂತಾನವನ್ನು ಪಡೆದರೆ, ಅವಳ ವಂಶಾಭಿವೃದ್ಧಿಗೇ ಕಲ್ಲುಹಾಕುವಂತಹ ಪ್ರಾರ್ಥನೆ ಮಾಡು ತ್ತಿದ್ದಾರಲ್ಲ ಇವರು!’ ಎಂದುಕೊಳ್ಳಬಹುದು. ಆದರೆ ಶ್ರೀರಾಮಕೃಷ್ಣರಿಗೆ ಚೆನ್ನಾಗಿ ಗೊತ್ತಿತ್ತು– ಯಾರೋ ಒಬ್ಬಳನ್ನು ಮದುವೆ ಮಾಡಿಕೊಂಡು ತನ್ನೆಲ್ಲ ಪ್ರೀತಿಯನ್ನೂ ಅವಳೊಬ್ಬಳಿಗೇ ಸುರಿಯುವುದಕ್ಕಾಗಿ ಹುಟ್ಟಿದವನಲ್ಲ ನರೇಂದ್ರ. ಅಥವಾ ಸಂಸಾರ ಕಟ್ಟಿಕೊಂಡು ಕೇವಲ ಅದರ ಪಾಲನೆ-ಪೋಷಣೆಯಲ್ಲೇ ಜೀವನವನ್ನು ಸವೆಸುವುದಕ್ಕಾಗಿ ಜನ್ಮವೆತ್ತಿದವನಲ್ಲ ಎಂದು. ಅವನು ಅವತರಿಸಿರುವುದು ಜೀವರ ಉದ್ಧಾರಕ್ಕಾಗಿ, ಸಮಸ್ತ ಮಾನವಕೋಟಿಯ ಮೇಲೆ ತನ್ನ ಪ್ರೀತಿಯ ಮಳೆಗರೆಯುವುದಕ್ಕಾಗಿ ಎನ್ನುವುದು ಅವರ ದಿವ್ಯದೃಷ್ಟಿಗೆ ಚೆನ್ನಾಗಿ ತಿಳಿದಿತ್ತು. ಆದ್ದರಿಂದ ಮದುವೆಯ ಮಾತುಕತೆಗಳು ಬಿದ್ದುಹೋದ ವರ್ತಮಾನ ತಲುಪಿದಾಗಲೆಲ್ಲ ಸಮಾಧಾನದ ನಿಟ್ಟುಸಿರೆಳೆಯುತ್ತಿದ್ದರು. ಆದರೆ ಹೀಗೆ ಎಷ್ಟು ಸಂಬಂಧಗಳು ಕೂಡಿಬರದೆಹೋದರೂ ವಿಶ್ವನಾಥ ತಾಳ್ಮೆಗೆಡದೆ ಹೊಸ ಸಂಬಂಧವನ್ನು ಹುಡುಕುತ್ತಿದ್ದ. ಹೀಗೆ ಹೊಸ ಹುರುಪಿನಿಂದ ಹುಡುಕಿ ಕೊನೆಗೊಂದು ಪ್ರಖ್ಯಾತ-ಶ್ರೀಮಂತ ಮನೆತನದ ಸಂಬಂಧವನ್ನು ಏರ್ಪಡಿಸಿದ. ಆ ಹೆಣ್ಣಿನ ಕಡೆಯವರು ಹೇರಳವಾಗಿ ವರದಕ್ಷಿಣೆ ಕೊಡಲು ಮುಂದಾಗಿದ್ದರು. ಅಲ್ಲದೆ ವರನನ್ನು ಉನ್ನತ ಶಿಕ್ಷಣಕ್ಕಾಗಿ ಇಂಗ್ಲೆಂಡಿಗೆ ಕಳಿಸುವುದಕ್ಕೂ ಸಿದ್ಧರಿದ್ದರು. ಹಿರಿಯರ ನಡುವೆ ಮಾತುಕತೆ ನಡೆ ಯಿತು. ಯಾವ ವಿಷಯಕ್ಕೂ ತಕರಾರು ಹುಟ್ಟಿಕೊಳ್ಳಲಿಲ್ಲ. ವಿಶ್ವನಾಥನಿಗೆ ತುಂಬ ಸಮಾಧಾನ ವಾಯಿತು, ಕೊನೆಗಾದರೂ ಒಂದು ಸರಿಯಾದ ಸಂಬಂಧ ನಿಶ್ಚಯವಾಯಿತಲ್ಲ ಎಂದು. ಆದರೆ ನರೇಂದ್ರ ತಾನು ಮದುವೆಯಾಗುವುದೇ ಇಲ್ಲ ಎಂದು ಹಠಹಿಡಿದ. ಏನಾದರೂ ವಿಶ್ವನಾಥ ಅವನ ಮಾತಿಗೆ ಗಮನಕೊಡಲಿಲ್ಲ. ಹುಡುಗರು ತಮ್ಮ ಪೌರುಷವನ್ನು ತೋರಿಸಿಕೊಳ್ಳಲು ಹಾಗೆಲ್ಲ ಆಡುತ್ತಾರೆ ಎಂದೇ ಭಾವಿಸಿದ್ದ ಆತ. ಈಗ ಪರಿಸ್ಥಿತಿ ಹೇಗೆ ಮುಂದುವರಿಯುತ್ತಿತ್ತೋ ಏನೋ, ಆದರೆ ಕೆಲದಿನಗಳಲ್ಲೇ ವಿಶ್ವನಾಥ ಹಠಾತ್ತಾಗಿ ಹೃದಯಾಘಾತದಿಂದ ಕಾಲವಾದ. ಮದುವೆಯ ಸಂಭವವೇ ಇಲ್ಲವಾಯಿತು. ಆಗಂತೂ ನರೇಂದ್ರ ಸರ್ವತಂತ್ರ ಸ್ವತಂತ್ರನಾದ. ಅವಿವಾಹಿತನಾಗಿ ಉಳಿಯುವ ಅವನ ನಿರ್ಧಾರವನ್ನು ಬದಲಾಯಿಸಲು ಯಾರಿಂದಲೂ ಸಾಧ್ಯವಿಲ್ಲವಾಯಿತು. ಆಮೇಲೂ ಮನೆಮಂದಿ ಅವನನ್ನು ಮದುವೆಯಾಗುವಂತೆ ಬಲವಂತಪಡಿಸಿದಾಗ ನರೇಂದ್ರ, “ನೀವೆಲ್ಲ ಸೇರಿ ನನ್ನನ್ನು ಮುಳುಗಿಸಿಬಿಡಬೇಕು ಅಂತ ಆಲೋಚಿಸಿದ್ದೀರೇನು? ಮದುವೆಯಾಗಿ ಬಿಟ್ಟರೆ ನನ್ನ ಕಥೆ ಮುಗಿದಹಾಗೆಯೇ!” ಎಂದು ರೇಗಿಬಿಡುತ್ತಿದ್ದ.

ಮುಂದಿನ ವಿಷಯ ಹಾಗಿರಲಿ; ಇನ್ನೂ ಬಿ.ಎ. ಓದುತ್ತಿರುವಾಗಲೇ ಅವನು ಎಷ್ಟೋ ಸಲ ತನ್ನ ಆಪ್ತ ಸ್ನೇಹಿತರ ಮುಂದೆ ಸಂನ್ಯಾಸಜೀವನದ ಘನತೆಯನ್ನು ವರ್ಣಿಸುತ್ತಿದ್ದ; ತಾನೂ ಸಂನ್ಯಾಸಿಯಾಗಲು ಮನಸ್ಸು ಮಾಡಿರುವುದನ್ನು ಸೂಚಿಸುತ್ತಿದ್ದ. ಆದರೆ ಸ್ನೇಹಿತರಿಗೆ ಇದೆಲ್ಲ ಸರಿ ಕಾಣಲಿಲ್ಲ. ಅವನ ಮೇಲಿನ ವಿಶ್ವಾಸದಿಂದ, “ಏನಪ್ಪ ನರೇನ್, ನೀನೇಕೆ ಒಂದು ವ್ಯವಸ್ಥೆ ಮಾಡಿಕೊಂಡು ನೆಲೆನಿಲ್ಲಬಾರದು? ನಿನಗೆ ಒಳ್ಳೇ ಭವಿಷ್ಯವಿದೆ. ನೀನು ಮನಸ್ಸು ಮಾಡಿದರೆ ಒಂದು ಒಳ್ಳೆಯ ಹುದ್ದೆಯನ್ನು ಪಡೆದುಕೊಂಡು ಸುಖವಾಗಿರಬಹುದು” ಎಂದು ಬುದ್ಧಿ ಹೇಳಲೆತ್ನಿಸಿದರು. ಆದರೆ ನರೇಂದ್ರ ಈ ಸಲಹೆಗಳನ್ನೆಲ್ಲ ಒಂದೇ ಸಲಕ್ಕೆ ತಿರಸ್ಕರಿಸಿಬಿಟ್ಟ: “ನೋಡಿ, ನಾನು ಹಿಂದೆ ಹೆಸರು-ಕೀರ್ತಿ, ಸ್ಥಾನ-ಮಾನ, ಸಂಪತ್ತು-ಅಧಿಕಾರಗಳ ಕನಸು ಕಂಡಿದ್ದೆ. ಆದರೆ ವಿಚಾರ ಮಾಡಿ ನೋಡಿದಂತೆ, ಇಂದಲ್ಲ ನಾಳೆ ಮೃತ್ಯು ಬಂದು ಇವೆಲ್ಲವನ್ನೂ ನುಂಗಿ ನೊಣೆಯುತ್ತದೆ ಅಂತ ಮನವರಿಕೆಯಾಯಿತು. ಎಂದಿದ್ದರೂ ಮೃತ್ಯುವಶವಾಗುವಂಥದನ್ನು ಬೆಳೆಸಿಕೊಂಡು ಏನಾಗಬೇಕು? ಆದ್ದರಿಂದ ಸಂನ್ಯಾಸಜೀವನವೇ ಶ್ರೇಷ್ಠವಾದದ್ದು. ಏಕೆಂದರೆ, ಮೃತ್ಯುವನ್ನೇ ಬದಿಗೊತ್ತಿ ಅಮರತೆಗೇರುವ ಪ್ರಯತ್ನ ಮಾಡುತ್ತಾನೆ ಸಂನ್ಯಾಸಿ. ಪ್ರಾಪಂಚಿಕರೆಲ್ಲ ಈ ಅನಿತ್ಯವಾದ ಭೋಗಭಾಗ್ಯಗಳ ಹಿಂದೆ ಅಲೆದಾಡಿದರೆ, ಸಂನ್ಯಾಸಿ ನಿತ್ಯಸತ್ಯದೆಡೆಗೆ ಸಾಗುತ್ತಾನೆ.”

ಈ ವಾದ ಅವನ ಸ್ನೇಹಿತರಿಗೆ ಅತಿ ವಿಚಿತ್ರವಾಗಿ ತೋರಿತು. ಅವರಲ್ಲೊಬ್ಬ ಹೇಳಿದ–“ಈಗ ಕಷ್ಟ ಏನು ಅಂದರೆ ನರೇಂದ್ರ ಆ ದಕ್ಷಿಣೇಶ್ವರದ ಮುದುಕನ ಸಹವಾಸಕ್ಕೆ ಬಿದ್ದುಬಿಟ್ಟಿದ್ದಾನೆ. ಆ ಮುದುಕನೋ, ಯಾವಾಗಲೂ ದೇವರ ವಿಷಯವಾಗಿಯೇ ಮಾತನಾಡುತ್ತಿರುತ್ತಾನೆ. ಆಗಾಗ ಸಮಾಧಿಗೇರುತ್ತಾನೆ. ಸಂನ್ಯಾಸಜೀವನ ನಡೆಸುತ್ತಿದ್ದಾನೆ. ಅವನಿಗೆ ಈ ಪ್ರಪಂಚದ ಗಂಧವೇ ಇಲ್ಲ. ಆ ಮುದುಕನೇ ನಮ್ಮ ನರೇಂದ್ರನ ಬುದ್ಧಿಯನ್ನು ಪಲ್ಲಟಗೊಳಿಸಿಬಿಟ್ಟಿದ್ದಾನೆ. ಇವನ ತಲೆತಿರುಗಿಸಿ ಭವಿಷ್ಯವನ್ನೆಲ್ಲ ಹಾಳುಮಾಡುತ್ತಿರುವವನು ಅವನೇ. ನರೇನ್, ನಿನಗೇನಾದರೂ ಬುದ್ಧಿಗಿದ್ಧಿ ಇದ್ದರೆ ಅವನ ಬಳಿಗೆ ಹೋಗುವುದನ್ನು ನಿಲ್ಲಿಸಿಬಿಡು. ಇಲ್ಲದೆಹೋದರೆ, ನಿನ್ನ ಭವಿಷ್ಯವೆಲ್ಲ ಹಾಳಾಗಿ ಹೋಗುವುದು ಖಂಡಿತ. ನಿನ್ನಲ್ಲಿ ಒಳ್ಳೇ ಪ್ರತಿಭೆಯಿದೆ, ನೀನು ಮನಸ್ಸು ಮಾಡಿದರೆ ಬೇಕಾದ್ದನ್ನು ಸಾಧಿಸಬಲ್ಲೆ ಕಣೊ! ಸುಮ್ಮನೆ ಅಲ್ಲಿಗೆ ಹೋಗುವುದನ್ನು ಬಿಟ್ಟುಬಿಡು.” ಆಗ ನರೇಂದ್ರ ಗಂಭೀರವಾಗಿ ಹೇಳಿದ: “ನೋಡಿ ನಿಮಗದೆಲ್ಲ ಅರ್ಥವಾಗುವುದಿಲ್ಲ. ನಿಮಗೇ ಏನು, ನನಗೂ ಅರ್ಥವಾಗುತ್ತಿಲ್ಲ!... ಇಲ್ಲ ನನಗೂ ಅರ್ಥವಾಗುತ್ತಿಲ್ಲ. ಆದರೆ... ಆ ಮುದುಕನನ್ನು, ಆ \textbf{ಮಹಾತ್ಮ}ನನ್ನು ನಾನು ಪ್ರೀತಿಸುತ್ತೇನೆ.” ಇದಕ್ಕೆ ಅವನ ಸ್ನೇಹಿತರು ಏನು ತಾನೆ ಹೇಳಿಯಾರು?

ಮದುವೆಯ ವಿಷಯದಲ್ಲಿ ನರೇಂದ್ರನ ಮೇಲೆ ಮನೆಮಂದಿಯ ಒತ್ತಡ ಹೆಚ್ಚಾಗಿರುವುದು ತಿಳಿದಾಗ ಶ್ರೀರಾಮಕೃಷ್ಣರು ಅವನ ಬಗ್ಗೆ ಕಾತರಗೊಂಡರು. ಕೆಲವು ದಿನಗಳವರೆಗೆ ಅವನು ದಕ್ಷಿಣೇಶ್ವರಕ್ಕೆ ಬರದಿದ್ದರೆ ತಾವೇ ಅವನ ಮನೆಗೆ ಹೋಗಿ ಮಾತನಾಡಿಸುತ್ತಿದ್ದರು. ಅವನಿಗೆ ಧ್ಯಾನಕ್ಕೆ ಹಾಗೂ ಆಧ್ಯಾತ್ಮಿಕ ಜೀವನಕ್ಕೆ ಸಂಬಂಧಿಸಿದಂತೆ ವಿಶೇಷ ಸಲಹೆಗಳನ್ನು ನೀಡುತ್ತಿದ್ದರು. ಎಲ್ಲಿ ಅವನು ತಾಯ್ತಂದೆಯರ, ಬಂಧು-ಬಳಗದವರ ಬಲವಂತಕ್ಕೆ ಮಣಿದು ಮದುವೆಯ ಬಂಧನಕ್ಕೆ ಒಳಗಾಗಿಬಿಡುತ್ತಾನೋ ಎಂದು ಅವರು ಹೆದರಿಬಿಟ್ಟಿದ್ದರು. ಆದ್ದರಿಂದ ಕಠಿಣ ಬ್ರಹ್ಮಚರ್ಯವನ್ನು ಬಿಡದೆ ಪಾಲಿಸಿಕೊಂಡು ಬರುವಂತೆ ಪ್ರೋತ್ಸಾಹಿಸಿದರು. ಒಂದು ದಿನ ಅವರೆನ್ನುತ್ತಾರೆ, “ನೋಡು, ಮನುಷ್ಯ ಹನ್ನೆರಡು ವರ್ಷಗಳವರೆಗೆ ಕಟ್ಟುನಿಟ್ಟಾದ ಬ್ರಹ್ಮಚರ್ಯ ವನ್ನು ಪಾಲಿಸಿದ್ದೇ ಆದರೆ ಅವನಲ್ಲೊಂದು ಸೂಕ್ಷ್ಮ ಶಕ್ತಿ ಬೆಳೆಯುತ್ತದೆ. ಸಾಧಾರಣ ಬುದ್ಧಿವಂತಿಕೆ ಯಿಂದ ಗ್ರಹಿಸಲಾಗದ ಗಹನ ವಿಚಾರಗಳನ್ನೂ ಅವನು ಈ ಶಕ್ತಿಯಿಂದ ಯಥಾವತ್ತಾಗಿ ಗ್ರಹಿಸುವ ಸಾಮರ್ಥ್ಯವನ್ನು ಪಡೆಯುತ್ತಾನೆ. ಈ ಶಕ್ತಿಯ ಸಹಾಯದಿಂದ ಅವನು ಭಗವಂತನ ಪ್ರತ್ಯಕ್ಷ ದರ್ಶನ ಮಾಡಲೂ ಸಮರ್ಥನಾಗುತ್ತಾನೆ. ಅವನ ಪರಿಶುದ್ಧ ಗ್ರಹಣಸಾಮರ್ಥ್ಯವೇ ಅವನನ್ನು ಭಗವತ್ಸಾಕ್ಷಾತ್ಕಾರಕ್ಕೆ ಅರ್ಹನನ್ನಾಗಿ ಮಾಡುತ್ತದೆ.” ನಿಜಕ್ಕೂ ಅವರು ನರೇಂದ್ರನಿಗೆ ಬ್ರಹ್ಮಚರ್ಯದ ಪಾಠವನ್ನು ಹೊಸದಾಗಿ ಹೇಳಿಕೊಡಬೇಕಾದ ಅಗತ್ಯವೇನೂ ಇರಲಿಲ್ಲ. ಏಕೆಂ ದರೆ, ನಾವು ನೋಡಿದಂತೆ ಅವನು ಮೊದಲಿನಿಂದಲೂ ಬ್ರಹ್ಮಚರ್ಯವನ್ನು ಪಾಲಿಸಿಕೊಂಡು ಬಂದವನೇ. ಆದರೆ ಅವನೀಗ ಪರಿಸ್ಥಿತಿಯ ಒತ್ತಡಕ್ಕೆ ಸಿಲುಕಿ ಬೇರೆಯವರ ಬಲವಂತದಿಂದಾಗಿ ಎಲ್ಲಿ ಮದುವೆಯ ಬಂಧನಕ್ಕೆ ಸಿಲುಕಿಕೊಂಡುಬಿಟ್ಟಾನೋ ಎಂಬ ಬಾಲಸಹಜ ಭೀತಿಯಿಂದ ಶ್ರೀರಾಮಕೃಷ್ಣರು ಇದನ್ನೆಲ್ಲ ಹೇಳುತ್ತಿದ್ದರಷ್ಟೆ.

ಯಾವಾಗ ನರೇಂದ್ರ ‘ಮದುವೆ ಬೇಡ, ಸಂನ್ಯಾಸಿಯಾಗುತ್ತೇನೆ’ ಎಂದು ಹಠಹಿಡಿದು ಕುಳಿತು ಬಿಟ್ಟನೋ ಆಗ ಮನೆಯವರ ಮನಸ್ಸಿನಲ್ಲಿ ಶ್ರೀರಾಮಕೃಷ್ಣರ ಬಗ್ಗೆ ಸಂಶಯದ ಮೊಳಕೆ ಕಾಣಿಸಿ ಕೊಂಡಿತು. ಅವರ ಪ್ರಭಾವಕ್ಕೆ ಒಳಗಾಗಿರುವುದರಿಂದಲೇ ಈಗ ಇವನು ಮದುವೆಯಾಗುವುದಿಲ್ಲ ಎನ್ನುತ್ತಿರುವುದು ಎಂದು ಅರ್ಥೈಸಿದರು. ಒಂದು ದಿನ ಶ್ರೀರಾಮಕೃಷ್ಣರು ನರೇಂದ್ರನ ಮನೆಗೆ ಬಂದು, ಅವನ ಕೋಣೆಯಲ್ಲಿ ವೈಯಕ್ತಿಕವಾಗಿ ಮಾತನಾಡುತ್ತಿದ್ದಾರೆ; ಆಜೀವ ಬ್ರಹ್ಮಚರ್ಯ ಪಾಲನೆಯ ಬಗ್ಗೆ ಬೋಧಿಸುತ್ತಿದ್ದಾರೆ. ಆಗ ಅವನ ಅಜ್ಜಿ ಮರೆಯಲ್ಲಿ ನಿಂತು ಅವರ ಸಂಭಾಷಣೆಯನ್ನೆಲ್ಲ ಕೇಳಿಸಿಕೊಂಡು ನರೇಂದ್ರನ ತಾಯ್ತಂದೆಯರ ಹತ್ತಿರ ವರದಿ ಮಾಡಿ ಬಿಟ್ಟಳು. ಅಂದಿನಿಂದ, ಅವನಿಗೆ ಸಾಧ್ಯವಾದಷ್ಟು ಬೇಗನೆ ಮದುವೆ ಮಾಡಿಬಿಡುವ ಪ್ರಯತ್ನ ಆರಂಭವಾಯಿತು. ಆದರೆ ಹಿಂದೆ ಹೇಳಿದಂತೆ, ಮತ್ತೆಮತ್ತೆ ಮದುವೆ ಪ್ರಯತ್ನ ಮಾಡಿದಾಗಲೂ ಎಲ್ಲ ನಿಶ್ಚಯವಾದ ಮೇಲೆ ಯಾವುದೋ ಒಂದು ಕ್ಷುಲ್ಲಕ ಕಾರಣದಿಂದ ಎಲ್ಲ ಬಿದ್ದುಹೋಗುತ್ತಿತ್ತು.

ಈ ವೇಳೆಗೆ, ಎಂದರೆ ೧೮೮೪ರ ಪ್ರಾರಂಭದಲ್ಲಿ, ನರೇಂದ್ರನಿಗೆ ಜಗತ್ತಿನ ಘೋರ ವಾಸ್ತವಿಕತೆ ಯನ್ನು ಮುಖಾಮುಖಿಯಾಗಿ ಎದುರಿಸಬೇಕಾದ ಸಂದರ್ಭ ಒದಗಿಬಂತು. ಅವನ ಉಲ್ಲಾಸ ಮಯ ಸ್ವಭಾವಕ್ಕೆ, ನಿಶ್ಚಿಂತೆಯ ಲಘು ಪ್ರಕೃತಿಗೆ ದೊಡ್ಡ ಹೊಡೆತ ಬಿತ್ತು. ಅಂದು ಫೆಬ್ರವರಿ ೨೫; ಬಾರಾನಗೋರಿನ ತನ್ನ ಸ್ನೇಹಿತನೊಬ್ಬನ ಕೋರಿಕೆಯಂತೆ ಆಹ್ವಾನಿತರೆದುರು ಭಕ್ತಿಸಂಗೀತ ವನ್ನು ಹಾಡಲು ನರೇಂದ್ರ ಹೋಗಿದ್ದ. ಕಾರ್ಯಕ್ರಮ ಮುಗಿದ ಮೇಲೆ ರಾತ್ರಿಯ ಭೋಜನ ಮುಗಿಸಿಕೊಂಡು ಎಲ್ಲರೂ ಅಲ್ಲೇ ವಿಶ್ರಮಿಸುತ್ತಿದ್ದರು. ರಾತ್ರಿ ಎರಡು ಗಂಟೆಗೆ ನರೇಂದ್ರನಿ ಗೊಂದು ದಾರುಣ ವಾರ್ತೆ ಬಂತು–ಆತನ ತಂದೆ ಹೃದಯಾಘಾತದಿಂದ ತೀರಿಕೊಂಡರು ಎಂದು. ಅದನ್ನು ಕೇಳಿ ಅವನ ಕೈಕಾಲೇ ಆಡಲಿಲ್ಲ. ಹೇಗೋ ಮಾಡಿ ಮನೆಗೆ ಧಾವಿಸಿದ. ಅಷ್ಟು ಹೊತ್ತಿಗೆ ಎಲ್ಲ ಮುಗಿದುಹೋಗಿತ್ತು–ವಿಶ್ವನಾಥನ ಶರೀರವನ್ನು ಸ್ಮಶಾನಕ್ಕೊಯ್ಯಲು ಸಿದ್ಧತೆ ನಡೆಯುತ್ತಿತ್ತು. ತಾಯಿ ಸೋದರಸೋದರಿಯರೆಲ್ಲರೂ ರೋದಿಸುತ್ತಿದ್ದರು. ಒಂದು ತಿಂಗಳ ಹಿಂದೆಯೇ ವಿಶ್ವನಾಥನಿಗೆ ಒಮ್ಮೆ ಹೃದಯಾಘಾತವಾಗಿದ್ದು, ಆಗಿನಿಂದ ವಿಶ್ರಾಂತಿಯಲ್ಲಿದ್ದ. ಆದರೆ ತನ್ನ ತಂದೆ ಇದ್ದಕ್ಕಿದ್ದಂತೆ ಹೀಗೆ ತೀರಿಕೊಂಡಾರೆಂಬ ಕಲ್ಪನೆಯೇ ನರೇಂದ್ರನ ಮನಸ್ಸಿಗೆ ಬಂದಿರಲಿಲ್ಲ ಎಂದು ತೋರುತ್ತದೆ. ಅವನೀಗ ದಿಕ್ಕುತೋಚದೆ ಅಳಲೂ ಆರದೆ, ಮಾತನಾಡಲೂ ಆರದೆ ನಿಂತುಬಿಟ್ಟ. ಬಳಿಕ ಕಣ್ಣೀರ ಕೋಡಿ ಹರಿಯಿತು. ಅವನೇ ಹಿರಿಯ ಮಗನಾದ್ದರಿಂದ ಉತ್ತರಕ್ರಿಯೆಗಳನ್ನೆಲ್ಲ ನೆರವೇರಿಸಿದ.

ವಿಶ್ವನಾಥನ ಹಠಾತ್ ನಿಧನದಿಂದ ಅವನ ಸಂಸಾರ ಅತ್ಯಂತ ಅಸಹಾಯಕ ಪರಿಸ್ಥಿತಿಗೆ ಸಿಕ್ಕಿ ಕೊಂಡಿತು. ಅವನ ಸಾವಿನಿಂದಾಗಿ ಶೋಕಿಸುತ್ತ ಕುಳಿತಿರಲೂ ಮನೆಮಂದಿಗೆ ಅವಕಾಶವಾಗ ಲಿಲ್ಲ–ಅಷ್ಟರಲ್ಲಿಯೇ ಕಷ್ಟಕೋಟಲೆಗಳೆಲ್ಲ ರಣಹದ್ದುಗಳಂತೆ ಮುತ್ತಿದುವು. ವಿಶ್ವನಾಥ ತೀರಿ ಕೊಂಡ ನೆಂಬ ಸುದ್ದಿ ಕೇಳಿದ ಕೂಡಲೆ ಸಾಲಗಾರರೆಲ್ಲ ಮನೆ ಬಾಗಿಲು ತಟ್ಟಲಾರಂಭಿಸಿದರು. ಏಳೆಂಟು ಜನರ ಸಂಸಾರ. ಅವರಲ್ಲಿ ನರೇಂದ್ರನೊಬ್ಬನೇ ಕೆಲಸ ಮಾಡಬಲ್ಲ ಗಂಡುಮಗ. ಅವನಿಗೋ ಇನ್ನೂ ಯಾವ ಉದ್ಯೋಗವೂ ಇಲ್ಲ. ಆದ್ದರಿಂದ ಮನೆಗೆ ಬೇರೆ ವರಮಾನವೇ ಇರ ಲಿಲ್ಲ. ಈಗ ನರೇಂದ್ರನೇ ಅವರಿಗೆಲ್ಲ ಅನ್ನ ಕಾಣಿಸಬೇಕು. ಕಷ್ಟದ ದಿನಗಳು ಕಾಲಿಟ್ಟವು. ನರೇಂದ್ರ ಶ್ರೀಮಂತಿಕೆಯ ಸುಪ್ಪತ್ತಿಗೆಯಿಂದ ಇದ್ದಕ್ಕಿದ್ದಂತೆ ಬಡತನದ ಬವಣೆಯೊಳಗೆ ಬಿದ್ದ. ಕಷ್ಟಗಳ ಕಾರ್ಮೋಡಗಳು ಎಷ್ಟು ದಟ್ಟವಾಗಿ ಕವಿದುಕೊಂಡಿದ್ದುವೆಂದರೆ, ಇನ್ನು ಬೆಳಕು ಕಾಣಲು ಸಾಧ್ಯವೇ ಇಲ್ಲವೇನೋ ಎಂಬಂತೆ ಕಂಡುಬರುತ್ತಿತ್ತು. ಆದರೆ ನರೇಂದ್ರ ಗಂಡುಗಲಿ. ಈ ದುರದೃಷ್ಟ ಪರಂಪರೆಯನ್ನು ಧೈರ್ಯದಿಂದ, ಅಂತಸ್ಸತ್ವದಿಂದ, ಸಾಮರ್ಥ್ಯದಿಂದ ಎದುರಿಸಿದ. ಅವನು ತನಗೆ ತಾನೇ ಪ್ರಭು. ಸಂಕಟಗಳ ಒತ್ತಡಕ್ಕೆ ಸಿಲುಕಿ ದಾಸನಾಗಿ ದೀನನಾಗುವವನಲ್ಲ. ಆದರೆ ನಿಜಕ್ಕೂ ಅವನನಿಗೀಗ ಬಂದ ಪರಿಸ್ಥಿತಿ ಅದೆಷ್ಟು ಘೋರ! ಅವನೀಗ ಬಿ.ಎಲ್. ಪರೀಕ್ಷೆಗಾಗಿ ಓದುತ್ತಿದ್ದಾನೆ. ಅವನ ಕಾಲೇಜಿನಲ್ಲಿ ಈಗ ಅತಿ ಬಡವನೆಂದರೆ ಅವನೇ! ಕಾಲಿಗೆ ಒಂದು ಜೊತೆ ಸರಿಯಾದ ಪಾದರಕ್ಷೆಯೂ ಇಲ್ಲ. ತೊಡಲು ಅತ್ಯಂತ ಸಾಮಾನ್ಯವಾದ ಒರಟು ಬಟ್ಟೆಗಳು. ಎಷ್ಟೋ ಸಲ ಅವನು ಕಾಲೇಜಿಗೆ ಊಟವಿಲ್ಲದೆ ಬರಿಹೊಟ್ಟೆಯಲ್ಲಿ ಹೋಗಬೇಕಾಯಿತು. ಆಗಾಗ ಹಸಿವು ತಾಳಲಾರದೆ ಬಸವಳಿದು ತಲೆಸುತ್ತಿಬಂದು ಬಿದ್ದದ್ದೂ ಉಂಟು! ಕೆಲವೊಮ್ಮೆ ಅವನ ಸ್ನೇಹಿತರ ಆಹ್ವಾನದ ಮೇರೆಗೆ ಅವರ ಮನೆಗಳಿಗೆ ಹೋಗುತ್ತಿದ್ದ. ಅವರೊಂದಿಗೆ ಸಂತೋಷವಾಗಿ ಗಂಟೆಗಟ್ಟಲೆ ಮಾತನಾಡುತ್ತಿದ್ದ. ಅವರು ಒಮ್ಮೊಮ್ಮೆ ಅವನನ್ನು ಊಟಕ್ಕೇಳುವಂತೆ ಹೇಳು ತ್ತಿದ್ದರು. ಆದರೆ ಅವನಿಗೆ ತಕ್ಷಣ ತನ್ನ ಮನೆಯಲ್ಲಿನ ನಿರ್ಗತಿಕ ದಾರುಣ ಚಿತ್ರ ಕಣ್ಮುಂದೆ ಕಟ್ಟುತ್ತಿತ್ತು. ಮನೆಯವರೆಲ್ಲ ಅಲ್ಲಿ ಗಂಜಿ ಕುಡಿಯುತ್ತಿರುವಾಗ ತಾನಿಲ್ಲಿ ಪುಷ್ಕಳವಾಗಿ ಊಟ ಮಾಡಲೆ... ‘ಇಲ್ಲ ಇಲ್ಲ. ನನಗೆ ಯಾವುದೋ ಒಂದು ತ್ವರಿತದ ಕೆಲಸವಿದೆ’ ಎಂದು ಹೇಳಿ ಅಲ್ಲಿಂದ ಹೊರಟುಬಿಡುತ್ತಿದ್ದ. ಮನೆಯಲ್ಲೂ ಸಾಧ್ಯವಾದಷ್ಟು ಕಡಿಮೆ ಊಟ ಮಾಡುತ್ತಿದ್ದ– ಇತರರಾದರೂ ಸ್ವಲ್ಪ ಹೆಚ್ಚಾಗಿ ಉಣ್ಣಲಿ ಎಂಬ ಉದ್ದೇಶದಿಂದ. ಇನ್ನು ಕೆಲವೊಮ್ಮೆ ತಾಯಿ ಅವನನ್ನು ಊಟಕ್ಕೆ ಕರೆದರೆ, “ನನ್ನ ಊಟ ಆಗಿದೆಯಮ್ಮ; ಇವೊತ್ತು ನನಗೆ ಸ್ನೇಹಿತನ ಮನೆಯಲ್ಲಿ ಊಟ ಆಯಿತು. ನೀವೆಲ್ಲ ಊಟ ಮಾಡಿ” ಎಂದು ಸುಳ್ಳು ಹೇಳಿ ಉಪವಾಸವಿದ್ದು ಬಿಡುತ್ತಿದ್ದ! ಅದೂ, ತಾನು ನಿಜಕ್ಕೂ ಊಟ ಮಾಡಿದವನಂತೆ ಗೆಲುವಿನ ಮುಖದಿಂದಲೇ ಹೇಳುತ್ತಿದ್ದ. ಅವನು ಹೇಳುತ್ತಿರುವುದು ಸುಳ್ಳೆಂದು ಯಾರಿಗೂ ತಿಳಿಯುತ್ತಿರಲಿಲ್ಲ. ತನಗೊದಗಿದ ದುಃಖ-ದಾರಿದ್ರ್ಯಗಳನ್ನು ಅವನು ಹಗುರವಾಗಿ ಭಾವಿಸಿ ಕಡೆಗಣಿಸಲು ಪ್ರಯತ್ನಿಸುತ್ತಿದ್ದ. ಅವನು ಮಾತ್ರವಲ್ಲ, ಮನೆಯವರೆಲ್ಲರೂ ತಮ್ಮ ದಾರಿದ್ರ್ಯವನ್ನು ಜನರ ಕಣ್ಣಿನಿಂದ ಮರೆಮಾಚು ತ್ತಿದ್ದರು. ಎಷ್ಟಾದರೂ ಅವರದು ಶ್ರೀಮಂತ, ಗೌರವಾನ್ವಿತ ದತ್ತ ವಂಶವಲ್ಲವೆ?

ಎಷ್ಟೋ ಸಲ ಮನೆಯಲ್ಲಿ ಎಲ್ಲರಿಗೂ ಸಾಕಾಗುವಷ್ಟು ಅಕ್ಕಿ ಬೇಳೆ ಮುಂತಾದ ಅಡಿಗೆ ಪದಾರ್ಥಗಳು ಇರುತ್ತಿರಲಿಲ್ಲ; ಕೈಯೂ ಬರಿದಾಗಿರುತ್ತಿತ್ತು. ಆಗ ನರೇಂದ್ರ ತಾಯಿಗೆ ಹೇಳು ತ್ತಿದ್ದ–‘ಇಂದು ನನ್ನ ಸ್ನೇಹಿತರು ಊಟಕ್ಕೆ ಕರೆದಿದ್ದಾರೆ. ನಾನು ಅಲ್ಲಿಗೆ ಹೋಗುತ್ತಿದ್ದೇನೆ’ ಎಂದು. ಹೀಗೆ ಸುಳ್ಳು ಹೇಳಿ ದಿನವೆಲ್ಲ ಊಟವಿಲ್ಲದೆ ಇದ್ದುಬಿಡುತ್ತಿದ್ದ! ಅಭಿಮಾನಿಯಾದ ನರೇಂದ್ರ ಯಾರ ಮುಂದೆಯೂ ಅದನ್ನೆಲ್ಲ ಹೇಳಿಕೊಳ್ಳುತ್ತಲೂ ಇರಲಿಲ್ಲ. ಅವನ ಶ್ರೀಮಂತ ಸ್ನೇಹಿತರು ಅವನನ್ನು ಹಾಡುವುದಕ್ಕಾಗಿ ತಮ್ಮ ಮನೆಗೋ ಉದ್ಯಾನವನಕ್ಕೋ ಆಮಂತ್ರಿಸು ತ್ತಿದ್ದರು. ತಪ್ಪಿಸಿಕೊಳ್ಳಲು ಬೇರೆ ದಾರಿ ಸಿಗದಂತಾದಾಗ ಒಪ್ಪಿಕೊಳ್ಳುತ್ತಿದ್ದ. ಆದರೆ ತಾನು ಎಂತಹ ಪರಿಸ್ಥಿತಿಯಲ್ಲಿದ್ದೇನೆ ಎಂಬುದನ್ನು ನರೇಂದ್ರ ತಾನಾಗಿ ಹೇಳಿದವನೂ ಅಲ್ಲ, ಆ ಸ್ನೇಹಿತರು ಕೂಡ ಕೇಳಿ ತಿಳಿದುಕೊಳ್ಳುವವರೂ ಅಲ್ಲ. ಆದರೂ ಅವರಲ್ಲಿ ಒಬ್ಬೊಬ್ಬರು ಕೆಲ ವೊಮ್ಮೆ ಅವನನ್ನು ಕೇಳುವುದಿತ್ತು, ‘ಏನು, ಇವೊತ್ತು ಸ್ವಲ್ಪ ಸುಸ್ತಾದ ಹಾಗೆ ಕಾಣುತ್ತೀಯಲ್ಲ ಏಕೆ?’ ಎಂದು. ಈ ಸ್ನೇಹಿತರಲ್ಲಿ ಒಬ್ಬ ಮಾತ್ರ ನರೇಂದ್ರನ ಪರಿಸ್ಥಿತಿಯನ್ನು ಹೇಗೋ ತಿಳಿದು ಕೊಂಡುಬಿಟ್ಟಿದ್ದ. ಅವನು ಅಜ್ಞಾತವಾಗಿ ನರೇಂದ್ರನ ತಾಯಿಗೆ ಆಗಾಗ ಹಣ ಕಳಿಸುತ್ತಿದ್ದ. ಇದನ್ನು ತಿಳಿದ ನರೇಂದ್ರ ಮುಂದೆ ಹೇಳುತ್ತಾನೆ–‘ಅವನ ಈ ಉಪಕಾರಕ್ಕಾಗಿ ನಾನು ಅವನಿಗೆ ಎಂದೆಂ ದಿಗೂ ಕೃತಜ್ಞನಾಗಿದ್ದೇನೆ’ ಎಂದು.

ನರೇಂದ್ರನ ಸ್ನೇಹಿತರಲ್ಲಿ ಎಷ್ಟೋ ಜನ ಶ್ರೀಮಂತರು. ಅವರಿಗೆ ಆತನ ಬಡತನ ಹೊರ ನೋಟಕ್ಕೆ ಕಾಣುತ್ತಿರಲಿಲ್ಲ. ಕೆಲವೊಮ್ಮೆ ಅವರು ತಮ್ಮ ಭರ್ಜರಿ ಕುದುರೆ ಸಾರೋಟುಗಳಲ್ಲಿ ಅವನ ಮನೆಗೆ ಬಂದು ಅವನನ್ನು ವಾಯುವಿಹಾರಕ್ಕೋ, ಸಣ್ಣಪುಟ್ಟ ಪ್ರವಾಸಗಳಿಗೋ ಕರೆಯು ತ್ತಿದ್ದರು. ನರೇಂದ್ರ ಸ್ವಲ್ಪವೂ ಸಂಶಯಕ್ಕೆ ಆಸ್ಪದ ಕೊಡದೆ ನಗುನಗುತ್ತ ಅವರ ಜೊತೆ ಹೊರಟುಬಿಡುತ್ತಿದ್ದ. ಆದರೆ ಅವನ ಶರೀರ ಸೊರಗಿಹೋಗಿರುವುದು ಸ್ಪಷ್ಟವಾಗಿ ಕಾಣುತ್ತಿದೆ ಯಲ್ಲ! ಬಡತನವೇ ಇದಕ್ಕೆ ಕಾರಣವೆಂದು ಅವನ ಸ್ನೇಹಿತರಿಗೆ ತೋಚಲೇ ಇಲ್ಲ. ಬಹುಶಃ ತಂದೆಯ ಸಾವಿನ ದುಃಖದಿಂದ ಕೊರಗುತ್ತಿರಬೇಕು ಎಂದು ಭಾವಿಸಿ ಅವರು ಸುಮ್ಮನಾಗಿ ಬಿಟ್ಟರು. ನಿಜವಾದ ಆಧ್ಯಾತ್ಮಿಕ ಸಾಧನೆಯೆಂಬುದು, ಅಂತಸ್ಸತ್ತ್ವವೆಂಬುದು ಜೀವನದಲ್ಲಿ ವಿಶೇಷವಾಗಿ ನೆರವಿಗೆ ಬರುವುದು ಇಲ್ಲಿಯೇ. ಅವುಗಳ ಪ್ರಾಮುಖ್ಯತೆ ಅರ್ಥವಾಗುವುದು ಇಂತಹ ಸನ್ನಿವೇಶಗಳಲ್ಲಿಯೇ. ಕಷ್ಟಸಂಕಟಗಳು ಪ್ರಾಪ್ತವಾದಾಗ ಧೃತಿಗೆಡದೆ ಸಹಿಸಿಕೊಳ್ಳಲು ಹಾಗೂ ಜೋಲುಮೋರೆ ಹಾಕಿಕೊಳ್ಳದೆ ಹರ್ಷಚಿತ್ತದಿಂದಿರಲು ಈ ಆಧ್ಯಾತ್ಮಿಕ ದೃಷ್ಟಿ ನೆರವಾಗುತ್ತದೆ. ಜೀವನದಲ್ಲಿ ಕಷ್ಟಸಂಕಟಗಳು ಎಲ್ಲರಿಗೂ ಬರುತ್ತವೆ–ಆಧ್ಯಾತ್ಮಿಕ ವ್ಯಕ್ತಿಗಳಿಗೂ ಬರುತ್ತವೆ, ಪ್ರಾಪಂಚಿಕರಿಗೂ ಬರುತ್ತವೆ. ಆದರೆ ನಿಜವಾದ ಆಧ್ಯಾತ್ಮಿಕರು ಸಹಿಸಿಕೊಳ್ಳುತ್ತಾರೆ, ಪ್ರಾಪಂಚಿಕರು ವಿಲಿವಿಲಿ ಒದ್ದಾಡುತ್ತಾರೆ.

ನಿಜಕ್ಕೂ ನರೇಂದ್ರನಿಗೆ ಒದಗಿಬಂದ ಕಷ್ಟಸಂಕಟಗಳು ಸಾಮಾನ್ಯವಾದುವುಗಳಲ್ಲ. ತಂದೆಯ ಮರಣವೊಂದೇ ಆಗಿದ್ದರೆ ಕೆಲಕಾಲದಲ್ಲೇ ಮರೆಯಬಹುದಾಗಿತ್ತು. ಆದರೆ ಕಿತ್ತುತಿನ್ನುವ ದಾರಿದ್ರ್ಯ ಪ್ರಾಪ್ತವಾಗಿದೆ. ಅಂಥ ನಿರಾತಂಕದ, ವೈಭೋಗದ ಜೀವನ ನಡೆಸಿದವರಿಗೆ ಇದ್ದ ಕ್ಕಿದ್ದಂತೆ ಈ ದುರ್ಗತಿ! ಇಂಥದನ್ನು ಸಹಿಸುವುದಾಗಲಿ ಕಡೆಗಣಿಸುವುದಾಗಲಿ ಸಾಧ್ಯವೆ? ದಿನಬೆಳಗಾದರೆ ದಾರಿದ್ರ್ಯ ತನ್ನ ಕ್ರೂರ ಮುಖವನ್ನು ತೋರಿಸಿ ಅಣಕಿಸುತ್ತಿದೆ. ಇವೆಲ್ಲ ಸಾಲ ದೆಂಬಂತೆ ಈಗ ಇನ್ನೊಂದು ಕಷ್ಟ ಅಂಟಿಕೊಂಡು ಬಂತು. ವಿಶ್ವನಾಥನಿಂದ ನಾನಾ ಬಗೆಯ ಸಹಾಯ ಪಡೆದ ಅನ ಸ್ವಂತ ಸಂಬಂಧಿಗಳೇ ಈಗ ಆಸ್ತಿಗಾಗಿ ತಿರುಗಿಬಿದ್ದರು. ಅವನ ಮನೆಯನ್ನು ಕಸಿದುಕೊಳ್ಳುವ ಉದ್ದೇಶದಿಂದ ಯಾವುದೋ ಆಧಾರದ ಮೇಲೆ ನ್ಯಾಯಾಲಯಕ್ಕೆ ಹೋದರು. ನಾವೀಗಾಗಲೇ ನೋಡಿದಂತೆ, ಈಚೆಗೆ ಕೆಲತಿಂಗಳನಿಂದ ವಿಶ್ವನಾಥನೂ ಅವನ ಮನೆಮಂದಿಯೂ ಈ ವ್ಯಾಜ್ಯದ ಕಾರಣದಿಂದ ಬೇರೊಂದು ಬಾಡಿಗೆ ಮನೆಯಲ್ಲಿ ವಾಸವಾಗಿದ್ದರು. ತನ್ನ ತಂದೆ ತೀರಿಹೋದ ಮೇಲೂ ನರೇಂದ್ರ ಕೆಲದಿನಗಳವರೆಗೆ ಈ ಮನೆಯಲ್ಲಯೇ ಇದ್ದ. ಆದರೆ ಬಾಡಿಗೆಯನ್ನೂ ಕೊಡಲಾಗದಂತಾದಾಗ ತನ್ನ ಅಜ್ಜಿಯ ಮನೆಗೆ ಹೋಗಿರಬೇಕಾಯಿತು. ಇತ್ತ ಅವರ ಸ್ವಂತ ಮನೆಯ ಕೋರ್ಟು ವ್ಯವಹಾರ ವರ್ಷಗಟ್ಟಲೆ ನಡೆಯಿತು. ತಮ್ಮ ಸಂಸಾರದ ವ್ಯವಹಾರವೆಲ್ಲಸಾರ್ವಜನಿಕ ವಿಷಯವಾಯಿತಲ್ಲ ಎಂದು ಭುವನೇಶ್ವರಿ ತುಂಬ ನೊಂದುಕೊಂಡಳು. ಈ ಜಗಳವನ್ನು ತಮ್ಮತಮ್ಮೊಳಗೇ ಶಾಂತವಾಗಿ ಬಗೆಹರಿಸಿಕೊಳ್ಳುವ ಪ್ರಯತ್ನ ಮಾಡಿದಳು. ಆದರೆ ತಮ್ಮತಮ್ಮೊಳಗೇ ಶಾಂತವಾಗಿ ಬಗೆಹರಿಸಿಕೊಳ್ಳುವ ಪ್ರಯತ್ನ ಮಾಡಿದಳು. ಆದರೆ ಕಾಳೀಪ್ರಸಾದನ ಹೆಂಡತಿ ಹೊಂದಾಣಿಕೆಗೆ ಒಪ್ಪಲಿಲ್ಲ. ಅಂತೂ ನರೇಂದ್ರ ಕೋರ್ಟಿಗೆ ಹೋಗಬೇಕಾಯಿತು. ವರ್ಷಗಟ್ಟಲೆ ಅಲೆದಾಡಬೇಕಾಯಿತು. ಈ ಸಂದರ್ಭದಲ್ಲಿ ಅವನು ವಕೀಲಿ ಅಭ್ಯಾಸ ಮಾಡುತ್ತಿದ್ದ ಸಂಸ್ಥೆಯ ನಿಮಾಯಿಚಂದ್ರ ಮತ್ತು ಬ್ಯಾನರ್ಜಿ ಎಂಬುವರು ತುಂಬ ನೆರವಿಗೆ ಬಂದರು. ಅಲ್ಲದೆ ನರೇಂದ್ರ ದಿನಬೆಳಗಾದರೆ ದಾರಿದ್ರ್ಯ ತನ್ನ ಕ್ರೂರ ಮುಖವನ್ನು ತೋರಿಸಿ ಅಣಕಿಸುತ್ತಿದೆ. ಇವೆಲ್ಲ ಸಾಲ ದೆಂಬಂತೆ ಈಗ ಇನ್ನೊಂದು ಕಷ್ಟ ಅಂಟಿಕೊಂಡು ಬಂತು. ವಿಶ್ವನಾಥನಿಂದ ನಾನಾ ಬಗೆಯ ಸಹಾಯ ಪಡೆದ ಅವನ ಸ್ವಂತ ಸಂಬಂಧಿಗಳೇ ಈಗ ಆಸ್ತಿಗಾಗಿ ತಿರುಗಿಬಿದ್ದರು. ಅವನ ಮನೆಯನ್ನು ಕಸಿದುಕೊಳ್ಳುವ ಉದ್ದೇಶದಿಂದ ಯಾವುದೋ ಆಧಾರದ ಮೇಲೆ ನ್ಯಾಯಾಲಯಕ್ಕೆ ಹೋದರು. ನಾವೀಗಾಗಲೇ ನೋಡಿದಂತೆ, ಈಚೆಗೆ ಕೆಲತಿಂಗಳಿನಿಂದ ವಿಶ್ವನಾಥನೂ ಅವನ ಮನೆಮಂದಿಯೂ ಈ ವ್ಯಾಜ್ಯದ ಕಾರಣದಿಂದ ಬೇರೊಂದು ಬಾಡಿಗೆ ಮನೆಯಲ್ಲಿ ವಾಸವಾಗಿ ದ್ದರು. ತನ್ನ ತಂದೆ ತೀರಿಹೋದ ಮೇಲೂ ನರೇಂದ್ರ ಕೆಲದಿನಗಳವರೆಗೆ ಈ ಮನೆಯಲ್ಲಿಯೇ ಇದ್ದ. ಆದರೆ ಬಾಡಿಗೆಯನ್ನೂ ಕೊಡಲಾಗದಂತಾದಾಗ ತನ್ನ ಅಜ್ಜಿಯ ಮನೆಗೆ ಹೋಗಿರಬೇಕಾ ಯಿತು. ಇತ್ತ ಅವರ ಸ್ವಂತ ಮನೆಯ ಕೋರ್ಟು ವ್ಯವಹಾರ ವರ್ಷಗಟ್ಟಲೆ ನಡೆಯಿತು. ತಮ್ಮ ಸಂಸಾರದ ವ್ಯವಹಾರವೆಲ್ಲ ಸಾರ್ವಜನಿಕ ವಿಷಯವಾಯಿತಲ್ಲ ಎಂದು ಭುವನೇಶ್ವರಿ ತುಂಬ ನೊಂದುಕೊಂಡಳು. ಈ ಜಗಳವನ್ನು ತಮ್ಮತಮ್ಮೊಳಗೇ ಶಾಂತವಾಗಿ ಬಗೆಹರಿಸಿಕೊಳ್ಳುವ ಪ್ರಯತ್ನ ಮಾಡಿದಳು. ಆದರೆ ಕಾಳೀಪ್ರಸಾದನ ಹೆಂಡತಿ ಹೊಂದಾಣಿಕೆಗೆ ಒಪ್ಪಲಿಲ್ಲ. ಅಂತೂ ನರೇಂದ್ರ ಕೋರ್ಟಿಗೆ ಹೋಗಬೇಕಾಯಿತು. ವರ್ಷಗಟ್ಟಲೆ ಅಲೆದಾಡಬೇಕಾಯಿತು. ಈ ಸಂದರ್ಭ ದಲ್ಲಿ ಅವನು ವಕೀಲಿ ಅಭ್ಯಾಸ ಮಾಡುತ್ತಿದ್ದ ಸಂಸ್ಥೆಯ ನಿಮಾಯಿಚಂದ್ರ ಮತ್ತು ಬ್ಯಾನರ್ಜಿ ಎಂಬುವರು ತುಂಬ ನೆರವಿಗೆ ಬಂದರು. ಅಲ್ಲದೆ ನರೇಂದ್ರ ಕೂಡ ಈ ವಾದದಲ್ಲಿ ತನ್ನ ವಕೀಲಿಕೆಯ ಪ್ರತಿಭೆಯನ್ನು ತೋರಿಸಿದ. ಇದನ್ನು ಕಂಡು ಪ್ರತಿವಾದಿಗಳ ಕಡೆಯವನಾದ ಬ್ರಿಟಿಷ್ ವಕೀಲ ಕೂಡ ಹೃತ್ಪೂರ್ವಕವಾಗಿ ಮೆಚ್ಚಿಕೊಂಡ. ಅಂತೂ ಕೆಲವರ್ಷಗಳೇ ಕಳೆದಮೇಲೆ ನರೇಂದ್ರನ ಮನೆಮಂದಿಗೆ ನ್ಯಾಯ ಸಿಕ್ಕಿತು; ಅವನ ಪಾಲಿಗೆ ಬರಬೇಕಾದ ಆಸ್ತಿ-ಮನೆ ಸಿಕ್ಕಿತು, ಅವನ ತಾಯಿಯೂ ಸೋದರ ಸೋದರಿಯರೂ ಮತ್ತೆ ಆ ಮನೆಯಲ್ಲಿ ವಾಸಿಸಲಾರಂಭಿಸಿದರು. ಅಷ್ಟು ಹೊತ್ತಿಗೆ ಬಂಧುಗಳ ಕಿರಿಕಿರಿಯೂ ಕಡಿಮೆಯಾಗಿತ್ತು. (ಆ ವೇಳೆಗೆ ನರೇಂದ್ರ ಸಂನ್ಯಾಸಿ ಯಾಗಿ ಹೊರಟುಬಿಟ್ಟಿದ್ದ.) ಆದರೆ ಅಲ್ಲಿಯವರೆಗೂ ಅವರು ಅನುಭವಿಸಿದ್ದು ಸಾಧಾರಣ ಕಷ್ಟಗಳನ್ನಲ್ಲ. ಹಲವಾರು ವರ್ಷಗಳವರೆಗೆ ಅನ್ನಬಟ್ಟೆಗೂ ಪರದಾಡುವ ಪರಿಸ್ಥಿತಿಯಿತ್ತು. ನರೇಂದ್ರ ಮನೆಯವರೆಲ್ಲರಿಗೂ ಒಂದು ತುತ್ತು ಅನ್ನದ ವ್ಯವಸ್ಥೆ ಮಾಡಲು ಹೋರಾಡಿದ. ಆದರೆ ಅಷ್ಟು ಬುದ್ಧಿವಂತ, ವಿದ್ಯಾವಂತ, ಲಕ್ಷಣವಂತನಾದ ನರೇಂದ್ರನಿಗೂ ಒಂದು ಸರಿಯಾದ ಕೆಲಸ ಸಿಗಲೇ ಇಲ್ಲ. ಹೇಗೋ ಮಾಡಿ ಮೆಟ್ರೊಪಾಲಿಟನ್ ಶಾಲೆಯಲ್ಲಿ ಮಾಸ್ತರಿಕೆಯ ಕೆಲಸವೊಂದನ್ನು ಸಂಪಾದಿಸಿಕೊಂಡ. ಆದರೆ ಇದು ಕೇವಲ ಕೆಲತಿಂಗಳ ಕಾಲ ಮಾತ್ರ.

ಭುವನೇಶ್ವರೀ ದೇವಿಯ ಸಂಸಾರವನ್ನು ಅಲ್ಲೋಲಕಲ್ಲೋಲ ಮಾಡಿದ ಈ ಚಂಡಮಾರುತವು ತಾಯಿ-ಮಗನ ಸಂಬಂಧ ಹಿಂದಿಗಿಂತ ನೂರುಪಟ್ಟು ದೃಢವಾಗಲು, ಆತ್ಮೀಯವಾಗಲು ನೆರವಾ ಯಿತು. ಈ ಸಂಧರ್ಭದಲ್ಲಿ ಭುವನೇಶ್ವರಿ ತಾನು ಪತಿ ವಿಶ್ವನಾಥನಲ್ಲಿ ಕಂಡು ಮೆಚ್ಚಿಕೊಂಡಿದ್ದ ಒಂದು ಗುಣವನ್ನು–ಎಂದೆಂದಿಗೂ ಸೋಲೊಪ್ಪಿಕೊಳ್ಳದಿರುವ ಮಹಾ ಗುಣವನ್ನು–ಮಗ ನರೇಂದ್ರನಲ್ಲಿ ಕಂಡುಕೊಂಡಳು. ಅಷ್ಟೇ ಅಲ್ಲ, ಸ್ವತಃ ಭುವನೇಶ್ವರಿಯ ಅಂತಸ್ಸತ್ತ್ವವು ಕೂಡ ಅಗ್ನಿಪರೀಕ್ಷೆಯ ಈ ಸಂಧಿಕಾಲದಲ್ಲಿ ಬೆಳಕಿಗೆ ಬಂತು. ಆ ಮಹಾತಾಯಿಯ ಬಗ್ಗೆ ಶಾರದಾ ನಂದರು ಬರೆಯುತ್ತಾರೆ:

“ಪತಿಯ ನಿಧನಾನಂತರ ಇದ್ದಕ್ಕಿದ್ದಂತೆ ಕಷ್ಟದ ಮಡುವಿಗೆ ದೂಡಲ್ಪಟ್ಟ ಭುವನೇಶ್ವರೀ ದೇವಿ ತನ್ನ ಅದ್ಭುತವಾದ ಅಂತಸ್ಸತ್ತ್ವ, ಆಶ್ಚರ್ಯಕರವಾದ ತಾಳ್ಮೆ, ಶಾಂತತೆ, ಮಿತವ್ಯಯದ ಗುಣ ಮತ್ತು ಎಂಥ ಪರಿಸ್ಥಿತಿಗೂ ಹೊಂದಿಕೊಳ್ಳಬಲ್ಲ ಶಕ್ತಿ ಇವುಗಳನ್ನು ವ್ಯಕ್ತಪಡಿಸಿದಳು. ಸಂಸಾರ ವನ್ನು ನಿರ್ವಹಿಸಲು ಅದುವರೆಗೂ ತಿಂಗಳಿಗೆ ಒಂದು ಸಾವಿರ ರೂಪಾಯಿ ಖರ್ಚು ಮಾಡು ತ್ತಿದ್ದವಳು ಈಗ ಕೇವಲ ಮೂವತ್ತು ರೂಪಾಯಿಗಳಲ್ಲಿ ಇಡೀ ಸಂಸಾರದ ಖರ್ಚನ್ನು ನಡೆಸ ಬೇಕಾಗಿ ಬಂತು. ಆದರೆ ಒಮ್ಮೆಯಾದರೂ ಆಕೆ ಹತಾಶಳಾದಂತೆ ಕಾಣಲಿಲ್ಲ. ಅವಳು ತನ್ನ ಸಂಸಾರವನ್ನು ಅಷ್ಟು ಕಡಿಮೆ ಆದಾಯದಲ್ಲಿ ಅದೆಷ್ಟು ಜಾಣ್ಮೆಯಿಂದ ಅಚ್ಚುಕಟ್ಟಾಗಿ ನಿಭಾಯಿಸಿ ದಳೆಂದರೆ, ಆ ಸಂಸಾರವನ್ನು ನೋಡಿದವರೆಲ್ಲ ಅವರ ಆದಾಯ ಬಹಳ ಹೆಚ್ಚಾಗಿಯೇ ಇರ ಬೇಕೆಂದು ಭಾವಿಸುತ್ತಿದ್ದರು. ನಿಜಕ್ಕೂ ಆಕೆಯ ಪರಿಸ್ಥಿತಿಯನ್ನು ನೆನೆಸಿಕೊಂಡರೇ ಮೈ ನಡುಗು ತ್ತದೆ. ಅವರ ಆದಾಯ ತಿಂಗಳಿಗೆ ಇಂತಿಷ್ಟು ಎನ್ನುವುದು ಕೂಡ ನಿಶ್ಚಯವಿರಲಿಲ್ಲ. ಆದರೆ ಅದಷ್ಟರಲ್ಲೇ ಭುವನೇಶ್ವರಿ ತನ್ನ ವೃದ್ಧ ತಾಯಿಯನ್ನೂ ಮಕ್ಕಳನ್ನೂ ಸಾಕಿ ಸಲಹಿ, ಮಕ್ಕಳ ವಿದ್ಯಾಭ್ಯಾಸದ ಖರ್ಚನ್ನೂ ನೋಡಿಕೊಳ್ಳಬೇಕಾಗಿತ್ತು. ಆಕೆಯ ಪತಿಯ ಹಣ-ಅಧಿಕಾರಗಳ ಸಹಾಯದಿಂದಲೇ ಮೇಲೆ ಬಂದಿದ್ದ ಬಂಧುವರ್ಗದವರೂ ಕೂಡ ಈಗ ಅತ್ಯಂತ ಕೃತಘ್ನರಾಗಿ ವರ್ತಿಸಿ, ಅವರ ಆಸ್ತಿಯನ್ನೂ ಕಸಿದುಕೊಳ್ಳುವ ಪ್ರಯತ್ನ ನಡೆಸಿದ್ದರು. ಹಿರಿಯ ಮಗ ನರೇಂದ್ರ ನಿಗೆ ಒಂದು ಸರಿಯಾದ ಕೆಲಸವೂ ಸಿಗದೆ ಆತ ಕಡೆಗೆ ವೈರಾಗ್ಯಭಾವವನ್ನು ತಾಳಿ ಪ್ರಪಂಚವನ್ನೇ ತ್ಯಜಿಸಿಬಿಡುವ ಆಲೋಚನೆಯಲ್ಲಿದ್ದ. ಇಂತಹ ದಾರುಣ ಪರಿಸ್ಥಿತಿಯಲ್ಲೂ ಆಕೆ ಸ್ಥಿಮಿತಬುದ್ಧಿ ಯಿಂದ ವರ್ತಿಸಿದ ಪರಿಯನ್ನು ಕಂಡವರಲ್ಲಿ ಆಕೆಯ ಬಗ್ಗೆ ಗೌರವ-ಪೂಜ್ಯಭಾವಗಳು ತಾನೇ ತಾನಾಗಿ ಉದಿಸುತ್ತವೆ.”

ಭುವನೇಶ್ವರಿ ಎದುರಿಸಿದ ಕಷ್ಟಗಳು ಈ ರೀತಿಯಾದರೆ, ನರೇಂದ್ರನ ಕಷ್ಟಸಂಕಟಗಳು ಇನ್ನೂ ಅಸಹನೀಯವಾದವು. ತಂದೆ ಸತ್ತ ಸೂತಕದ ದಿನಗಳು ಕೂಡ ಇನ್ನೂ ಮುಗಿದಿಲ್ಲ. ದುಃಖದ ಕಣ್ಣೀರೇ ಇನ್ನೂ ಆರಿಲ್ಲ, ಅಷ್ಟರಲ್ಲೇ ಅವನು ಕೆಲಸಕ್ಕಾಗಿ ಅಲೆದಾಡುವಂತಾಯಿತು. ಮಧ್ಯಾಹ್ನದ ಉರಿಬಿಸಿಲಿನಲ್ಲಿ ಕೈಯಲ್ಲಿ ಅರ್ಜಿ ಹಿಡಿದುಕೊಂಡು ಕಛೇರಿಯಿಂದ ಕಛೇರಿಗೆ ಎಡತಾಕಿದ. ಹೊಟ್ಟೆಯಲ್ಲಿ ಹಸಿವಿನ ಬೆಂಕಿ, ತಲೆಯ ಮೇಲೆ ಸುಡುತ್ತಿರುವ ಸೂರ್ಯ, ಕಾಲಿಗೆ ಚಪ್ಪಲಿಯಿಲ್ಲ. ಆದರೆ ಹಸಿವು ಎಂದು ಕುಳಿತುಕೊಳ್ಳುವಂತಿಲ್ಲ; ಕಾಲು ಸುಡುತ್ತಿದೆ ಎಂದು ಸುಮ್ಮನಿರುವಂತಿಲ್ಲ. ಏಕೆಂದರೆ, ಉದ್ಯೋಗ ಸಿಗದೆಹೋದರೆ ತನಗೆ ಮಾತ್ರವಲ್ಲ, ಮನೆಮಂದಿಯ ಹೊಟ್ಟೆಗೆ ಸೊನ್ನೆ. ನರೇಂದ್ರ ಯಾರನ್ನೂ ಸಾಲ ಕೇಳುವವನಲ್ಲ; ಸ್ನೇಹಿತರು, ಸಂಬಂಧಿಕರು ತಾವಾಗಿಯೇ ಕೊಡುವವರಲ್ಲ! ಕೆಲವೊಮ್ಮೆ ಅವನ ಒಬ್ಬಿಬ್ಬರು ಆತ್ಮೀಯ ಸ್ನೇಹಿತರು ಅವನ ಆ ದುರದೃಷ್ಟದ ದಿನಗಳಲ್ಲಿ ಸಹಾನುಭೂತಿಯಿಂದ ಅವನ ಉದ್ಯೋಗಾನ್ವೇಷಣೆಯಲ್ಲಿ ನೆರವಾಗಲು ಬಂದರು. ಆದರೆ ಅವನ ಪಾಲಿಗೆ ಪ್ರತಿಯೊಂದು ಕಡೆಯಲ್ಲೂ ಬಾಗಿಲು ಮುಚ್ಚಿಕೊಳ್ಳುತ್ತಿತ್ತು. ಅವನಿಗೀಗ ಮೊತ್ತಮೊದಲನೆಯದಾಗಿ ಜೀವನದ ಕಠೋರ ವಾಸ್ತವಿಕತೆಯ ದರ್ಶನವಾಗುತ್ತಿದೆ. ಅದನ್ನು ಕಂಡಾಗ ಅವನಿಗನ್ನಿಸತೊಡಗಿತು–ಈ ಜಗತ್ತಿನಲ್ಲಿ ಸ್ವಾರ್ಥರಹಿತವಾದ ಸಹಾನುಭೂತಿ, ಸ್ನೇಹ ಎಂಬವುಗಳೆಲ್ಲ ಕೇವಲ ಮಾತನಾಡಲು ಉಪಯೋಗಿಸುವ ಔಪಚಾರಿಕ ಪದಗಳಷ್ಟೇ ಎಂದು. ಇಲ್ಲಿ ದುರ್ಬಲರಿಗೆ, ಬಡವರಿಗೆ, ಅಸಹಾಯಕರಿಗೆ ಸ್ಥಾನವಿಲ್ಲ ಎಂದು ಅವನಿಗೆ ತೋರಿತು. ಕೆಲವೇ ದಿನಗಳ ಹಿಂದೆ ಯಾರ್ಯಾರು ಅವನಿಗೆ ಬೇಕಾದ ಸಹಾಯವನ್ನೆಲ್ಲ ಮಾಡಲು ಮುಂದಾಗು ತ್ತಿದ್ದರೋ ಈಗ ಅವರೇ ಅವನನ್ನು ಕಂಡ ತಕ್ಷಣ ಮುಖ ತಿರುಗಿಸತೊಡಗಿದ್ದರು. ಹೋಗಲಿ, ಅವರಿಗೂ ಏನೋ ತೊಂದರೆಯಿರಬಹುದೆ? ಹಾಗೇನಿಲ್ಲ! ಅವರೆಲ್ಲ ಸಾಕಷ್ಟು ಧನಿಕರೇ. ಆದರೆ ಅಸಹಾಯಕ ಸ್ಥಿತಿಯಲ್ಲಿರುವ ನರೇಂದ್ರನನ್ನು ಕಂಡರೆ ಮಾತ್ರ ದೂರವಾಗುತ್ತಿದ್ದಾರೆ–‘ಶಕ್ತ ನಾದರೆ ನೆಂಟರೆಲ್ಲ ಹಿತರು, ಅಶಕ್ತನಾದರೆ ಅವರೇ ವೈರಿಗಳು ಲೋಕದಲ್ಲಿ’ ಎಂಬಂತೆ. ಇದನ್ನೆಲ್ಲ ನೋಡುತ್ತನೋಡುತ್ತ ನರೇಂದ್ರನಿಗೆ ಕೆಲವೊಮ್ಮೆ ಅನ್ನಿಸುತ್ತಿತ್ತು–ಈ ಪ್ರಪಂಚವು ದೇವರ ಸೃಷ್ಟಿಯಲ್ಲ, ದೆವ್ವದ ಸೃಷ್ಟಿ ಎಂದು.

ಒಂದು ದಿನ ಅವನು ಕೆಲಸಕ್ಕಾಗಿ ಅಲೆದಾಡುತ್ತ ಆಯಾಸಗೊಂಡು ‘ಅಕ್ಬರ್ಲೋನಿ ಮಾನ್ಯು ಮೆಂಟ್’ ಎಂಬ ಎತ್ತರದ ಕಟ್ಟಡದ ನೆರಳಿನಲ್ಲಿ ಕುಳಿತುಕೊಂಡಿದ್ದಾನೆ. ಆಗ ಅವನ ಇಬ್ಬರು ಸ್ನೇಹಿತರು ಆಕಸ್ಮಿಕವಾಗಿ ಅವನನ್ನು ನೋಡಿ ಅಲ್ಲಿಗೆ ಬಂದರು. ಅವನ ದುಃಸ್ಥಿತಿಯನ್ನು ಕಂಡ ಒಬ್ಬ ಸ್ನೇಹಿತ ಅವನನ್ನು ಸಮಾಧಾನಪಡಿಸುವ ಉದ್ದೇಶದಿಂದ ಮೆಲ್ಲನೆ ಒಂದು ಹಾಡನ್ನು ಗುನುಗಲಾರಂಭಿಸಿದ:

\begin{myquote}
ಬೀಸುತಿಹುದು ಮಂದಾನಿಲ–ಇದುವೆ ಬ್ರಹ್ಮನುಸಿರು;\\ಮೈಮನವನು ತಣಿಸುತಿಹುದು, ಇದುವೆ ಅವನ ಕರುಣೆಯು!
\end{myquote}

\noindent

ಅವನು ಈ ಹಾಡನ್ನು ಅರ್ಧ ಕೂಡ ಹೇಳಿಲ್ಲ, ಅಷ್ಟರಲ್ಲೇ ನರೇಂದ್ರ ಅವನನ್ನು ತಡೆದ. ಅವನಿಗೆ ಈ ಹಾಡನ್ನು ಕೇಳಿ ಸಮಾಧಾನವಾಗುವುದಿರಲಿ, ಅಸಹಾಯಕರಾಗಿದ್ದ ತನ್ನ ತಾಯಿ ಹಾಗೂ ತಮ್ಮಂದಿರ ನೆನಪು ಒತ್ತರಿಸಿಬಂದು, ಯಾರೋ ತಲೆಯ ಮೇಲೆ ದೊಣ್ಣೆಯಿಂದ ಬಡಿದಂತಾ ಯಿತು. ಅವನು ಆಕ್ರೋಶದ ದನಿಯಲ್ಲಿ ನುಡಿದ–“ದಯಮಾಡಿ ನಿನ್ನ ಆ ಹಾಡನ್ನು ನಿಲ್ಲಿಸು ತ್ತೀಯಾ! ಚಿನ್ನದ ತಟ್ಟೆಯಲ್ಲಿ ಊಟ ಮಾಡುವವರಿಗೆ ಸೊಗಸೀತು ಇಂಥಾ ಹಾಡುಗಳೆಲ್ಲ. ಹಸಿವೆಯನ್ನೇ ಅನುಭವಿಸದವರಿಗೆ ಸೊಗಸೀತು. ನಿಜ, ಈ ತರಹದ ಹಾಡುಗಳನ್ನು ಒಂದು ಕಾಲದಲ್ಲಿ ನಾನೂ ಇಷ್ಟಪಟ್ಟಿದ್ದೆ. ಆದರೆ ಇಂದು ಜೀವನದ ಬೀಭತ್ಸ ದೃಶ್ಯದ ಮುಂದೆ ಅವೆಲ್ಲ ವಿಡಂಬನೆಗಳ ಹಾಗೆ ಕಂಡುಬರುತ್ತಿವೆ!” ನರೇಂದ್ರನ ಸಿಡಿಮಿಡಿಯನ್ನು ಕೇಳಿ ಅವನ ಸ್ನೇಹಿತ ಮತ್ತೆ ಮಾತನಾಡುವ ಸಾಹಸ ಮಾಡಲಿಲ್ಲ. ಆದರೆ ಅವನ ಬಾಯಿಂದ ಆ ಮಾತನ್ನು ಹೊರಡಿಸಿದ್ದ ಅವನೊಡಲೊಳಗಿನ ದುಃಖದ ತೀವ್ರತೆ ಆ ಸ್ನೇಹಿತನಿಗೆ ಹೇಗೆ ಅರ್ಥವಾಗಬೇಕು!

ನರೇಂದ್ರನ ಹಳೆಯ ಸ್ನೇಹಿತರು ಕೆಲವರು ಅನ್ಯಾಯಮಾರ್ಗಗಳಿಂದ ಹಣ ಸಂಪಾದನೆ ಮಾಡುತ್ತಿದ್ದರು. ಕಡಮೆ ಕಾಲಾವಧಿಯಲ್ಲಿ ಹೆಚ್ಚು ಹಣವನ್ನು ಸುಲಭವಾಗಿ ಸಂಪಾದಿಸಬೇಕಾದರೆ ಇದೇ ಏಕಮಾತ್ರ ಉಪಾಯ ಎಂಬುದು ಅವರ ನಿಶ್ಚಿತ ನಿಲುವು. ಅವರಿಗೂ ನರೇಂದ್ರನಿಗೆ ಒದಗಿಬಂದಂತಹ ದುರದೃಷ್ಟದ ಪರಿಸ್ಥಿತಿಗಳು ಎದುರಾಗಿದ್ದುದರಿಂದಲೇ ಅವರು ಆ ಮಾರ್ಗ ಗಳನ್ನು ಹಿಡಿಯಬೇಕಾಗಿ ಬಂದಿತ್ತು. ಈಗ ನರೇಂದ್ರನೂ ಇದ್ದಕ್ಕಿದ್ದಂತೆ ಬಡತನದ ಬವಣೆಗೆ ಸಿಲುಕಿದ್ದನ್ನು ಕಂಡ ಆ ಸ್ನೇಹಿತರು ಅವನ ಮೇಲಿನ ಸಹಾನುಭೂತಿಯಿಂದ ಅವನನ್ನು ತಮ್ಮೊಂದಿಗೆ ಸೇರಿಕೊಳ್ಳುವಂತೆ ಆಹ್ವಾನಿಸಿದರು. ಹೀಗಾದರೂ ಅವನ ಬಡತನ ನೀಗಲಿ ಎಂಬುದು ಅವರ ‘ಸದುದ್ದೇಶ’. ಆದರೆ ನರೇಂದ್ರನ ನೈತಿಕತೆಯೆಂಬುದು ಮರಳಿನ ಮೇಲೆ ಕಟ್ಟಿದ ಕಟ್ಟಡವಲ್ಲ. ಅವನು ಅಡ್ಡದಾರಿಗಿಳಿಯಲಿಲ್ಲ ಎಂಬುದನ್ನು ಹೇಳಬೇಕಾಗಿಯೇ ಇಲ್ಲವಲ್ಲ!

ಅವನ ಮುಂದೆ ಮಾಯಾಜಿಂಕೆಯಂತೆ ಸುಳಿದಾಡಿದ ಪ್ರಲೋಭನೆಗಳು ಅನೇಕ. ಹಿಂದಿ ನಿಂದಲೂ ಅವನ ಮೇಲೆ ಕಣ್ಣಿಟ್ಟಿದ್ದ ಒಬ್ಬಳು ಶ್ರೀಮಂತ ಮಹಿಳೆ ಅವನಿಗೊಂದು ಅಶ್ಲೀಲ ಸಲಹೆಯನ್ನು ನೀಡಿ, ಅವನ ಕಷ್ಟಗಳನ್ನೆಲ್ಲ ತಾನು ಪರಿಹರಿಸುವುದಾಗಿ ಸೂಚಿಸಿದಳು. ಆದರೆ ಅವನು ತುಂಬ ಕಟುವಾಗಿ ಅವಳ ಸಲಹೆಯನ್ನು ತುಚ್ಛೀಕರಿಸಿದ. ಇನ್ನೊಮ್ಮೆ ಮತ್ತೊಬ್ಬ ಹೆಂಗಸು ಇದೇ ರೀತಿಯ ಇಂಗಿತವನ್ನು ವ್ಯಕ್ತಪಡಿಸಿದಾಗ ನರೇಂದ್ರ ಅವಳಿಗೆ ಕನಿಕರದಿಂದ ಹೇಳಿದ– “ನೀನು ದೇಹಸುಖವನ್ನು ಅರಸುತ್ತ ಜೀವನವನ್ನೆಲ್ಲ ವ್ಯರ್ಥ ಮಾಡಿಕೊಂಡೆ. ಮೃತ್ಯುವಿನ ಕಪ್ಪು ನೆರಳು ನಿನ್ನ ಮುಂದೆಯೇ ಸುಳಿದಾಡುತ್ತಿದೆ. ಅದನ್ನೆದುರಿಸಲು ಏನಾದರೂ ಸಿದ್ಧತೆ ಮಾಡಿಕೊಂಡಿ ದ್ದೀಯಾ? ಈಗಲಾದರೂ ಇಂಥ ಹೀನ ಬಯಕೆಗಳನ್ನೆಲ್ಲ ತ್ಯಜಿಸಿ ಭಗವಂತನ ಸ್ಮರಣೆ ಮಾಡು.” ನರೇಂದ್ರನಿಗೊದಗಿದ ಕಷ್ಟಪರಂಪರೆಗಳು ಅವನನ್ನು ನೈತಿಕತೆಯ ಶಿಖರದಿಂದ ಜಾರಿಸಲು ಸಮರ್ಥವಾಗಲಿಲ್ಲ ಎನ್ನುವುದೊಂದು ಮಹತ್ವದ ಅಂಶ.

ಆದರೆ ನರೇಂದ್ರನಿಗೆ ಈಗ ಎದುರಾಗಿರುವ ಕಷ್ಟಗಳಿಗೆ ಒಂದು ವಿಶೇಷ ಅರ್ಥವಿದೆ ಎನ್ನಬಹುದು. ಈ ಕಷ್ಟಗಳೆಲ್ಲ ಅವನಿಗೆ ಸಹನೆಯ ಪಾಠವನ್ನು ಕಲಿಸುತ್ತಿವೆ. ಬೃಹತ್ಕಾರ್ಯಗಳನ್ನು ಸಾಧಿಸಬೇಕಾದರೆ ಬೃಹತ್ತಾದ ಕಷ್ಟಗಳನ್ನು ಸಹಿಸಿಕೊಳ್ಳಬೇಕಾಗುತ್ತದೆ. ಕಷ್ಟಸಹಿಷ್ಣುತೆಯಿಲ್ಲದವ ನಿಂದ ಏನನ್ನೂ ಸಾಧಿಸಲಾಗದು. ನರೇಂದ್ರ ಮುಂದೆ ಸ್ವಾಮಿ ವಿವೇಕಾನಂದರಾದಾಗ ಅವರ ಜೀವನವೇನೂ ಹೂವಿನ ಹಾಸಿಗೆಯಾಗಿರಲಿಲ್ಲ. ಈಗ ನರೇಂದ್ರನಾಗಿ ಅನುಭವಿಸುತ್ತಿರುವ ಕಷ್ಟಗಳಿಗಿಂತಲೂ ಎಷ್ಟೋ ಪಾಲು ಹೆಚ್ಚಿನ ಪರ್ವತೋಪಮ ಕಷ್ಟಗಳನ್ನು ಎದುರಿಸಬೇಕಾಯಿತು. ಅದಕ್ಕೆ ಪೂರ್ವತಯಾರಿಯಾಗಿ ಅವನು ಇಂದು ಈ ಬಗೆಯ ಕಷ್ಟಗಳನ್ನು ಅನುಭವಿಸುತ್ತಿದ್ದಾನೆ ಎನ್ನಬಹುದು. ಕಷ್ಟಗಳನ್ನು ಧೀರತೆಯಿಂದ ಸಹಿಸಿಕೊಳ್ಳುವುದು ಒಂದು ದೊಡ್ಡ ಗುಣ. ಗೊಣ ಗದೆ, ಮರುಮಾತಿಲ್ಲದೆ ಸಹಿಸಿಕೊಂಡರೆ ನಮ್ಮಲ್ಲಿ ಅಪಾರ ಶಕ್ತಿ ಉತ್ಪನ್ನವಾಗುತ್ತದೆ. ಈ ಬಗೆಯ ಸಹನೆಗೆ ‘ತಿತಿಕ್ಷೆ’ ಎಂದು ಹೆಸರು. ‘ವಿವೇಕಚೂಡಾಮಣಿ’ಯಲ್ಲಿ ಈ ತಿತಿಕ್ಷೆಯ ಲಕ್ಷಣವನ್ನು ವಿವರಿಸಲಾಗಿದೆ:

\begin{verse}
ಸಹನಂ ಸರ್ವದುಃಖಾನಾಂ ಅಪ್ರತೀಕಾರಪೂರ್ವಕಂ\\ಚಿಂತಾವಿಲಾಪರಹಿತಂ ಸಾ ತಿತಿಕ್ಷಾ ನಿಗದ್ಯತೇ ॥
\end{verse}

\noindent

ಎಂದರೆ, ಎಲ್ಲ ಬಗೆಯ ದುಃಖಗಳನ್ನೂ, ಯಾವುದೇ ಬಗೆಯ ಪ್ರತಿಕ್ರಿಯೆಯನ್ನೂ ತೋರದೆ, ಯಾವುದೇ ರೀತಿಯಿಂದ ಚಿಂತಾಕ್ರಾಂತನಾಗದೆ ಮತ್ತು ವಿಲಾಪಿಸದೆ ಸಹಿಸಿಕೊಳ್ಳುವ ಗುಣವನ್ನು ತಿತಿಕ್ಷೆ ಎನ್ನುತ್ತಾರೆ. ಭಗವತ್ಸಾಕ್ಷಾತ್ಕಾರ ಮಾಡಿಕೊಳ್ಳಬೇಕೆಂಬವರಲ್ಲೇ ಆಗಲಿ, ಮಹಾಕಾರ್ಯ ವೊಂದನ್ನು ಸಾಧಿಸಹೊರಟವರಲ್ಲೇ ಆಗಲಿ ಇರಲೇಬೇಕಾದ ಒಂದು ಗುಣವೆಂದರೆ ಈ ತಿತಿಕ್ಷೆ. ಶಮ ದಮ ಉಪರತಿ ತಿತಿಕ್ಷೆ ಸಮಾಧಾನ ಶ್ರದ್ಧಾ ಎನ್ನುವ ಈ ಆರು ‘ಸಂಪತ್ತು’ಗಳಲ್ಲಿ ತಿತಿಕ್ಷೆಯೆಂಬುದು ಒಂದು ದೊಡ್ಡ ಸಂಪತ್ತು. ಈ ಬಗೆಯ ತಿತಿಕ್ಷೆಯನ್ನು ಅಭ್ಯಾಸ ಮಾಡುವುದ ಕ್ಕಾಗಿಯೋ ಎಂಬಂತೆ ನರೇಂದ್ರನೀಗ ಬಗೆಬಗೆಯ ಕಷ್ಟ ಪರಂಪರೆಯನ್ನು ಎದುರಿಸಬೇಕಾಗಿದೆ.

ಹೀಗೆ ನಾನಾ ಬಗೆಯ ಕಷ್ಟಗಳಿಗೆ, ಪ್ರಲೋಭನೆಗಳಿಗೆ ಒಳಗಾದರೂ ಅವನು ಭಗವಂತನ ಅಸ್ತಿತ್ವದಲ್ಲಿ, ಆತನ ಕರುಣೆಯಲ್ಲಿ ನಂಬಿಕೆಯನ್ನು ಕಳೆದುಕೊಂಡಿರಲಿಲ್ಲ. ಪ್ರತಿದಿನ ಬೆಳಗ್ಗೆ ಎದ್ದೊಡನೆಯೇ ಭಗವಂತನ ಸ್ಮರಣೆ ಮಾಡಿ, ಬಳಿಕ ಕೆಲಸ ಹುಡುಕಲು ಹೊರಡುತ್ತಿದ್ದ. ಒಂದು ದಿನ ಎಂದಿನಂತೆ ಪ್ರಾರ್ಥನೆ ಮಾಡುತ್ತಿದ್ದಾನೆ, ಆಗ ಅದನ್ನು ಕೇಳಿಸಿಕೊಂಡ ಅವನ ತಾಯಿ ಇದ್ದಕ್ಕಿದ್ದಂತೆ, “ಛೀ ದಡ್ಡ! ಸಾಕು ಸುಮ್ಮನಿರೋ! ನೀನು ನಿನ್ನ ಬಾಲ್ಯದಿಂದಲೂ ಆ ದೇವರಿಗಾಗಿ ಗಂಟಲು ಕಿತ್ತುಕೊಂಡೆಯಲ್ಲ, ಕೊನೆಗೂ ಅವನು ನಿನಗೆ ಏನು ಮಾಡಿದ?” ಎಂದು ಬಿಟ್ಟಳು. ಸ್ವತಃ ತಾಯಿಯೇ, ತನ್ನನ್ನು ಮಡಿಲಲ್ಲಿ ಕುಳ್ಳಿರಿಸಿಕೊಂಡು ಬಾಲ್ಯದಿಂದಲೂ ಭಗವಂತನ ಕುರಿತಾಗಿ ಬೋಧಿಸಿದವಳೇ ಹೀಗೆ ಹೇಳಿಬಿಟ್ಟಾಗ ನರೇಂದ್ರ ತತ್ತರಿಸಿಹೋದ. ಅವನ ಶ್ರದ್ಧೆಯ ಗೋಪುರ ಬಿರುಕುಬಿಟ್ಟಿತು. ಮನಸ್ಸಿನಲ್ಲಿ ಸಂಶಯ ಸುಳಿದಾಡಿತು–‘ನಿಜಕ್ಕೂ ದೇವರು ಇರು ವುದು ಹೌದೆ? ಮನುಷ್ಯ ವ್ಯಾಕುಲತೆಯಿಂದ ಕರೆದಾಗ ಅವನು ಅದನ್ನು ಕೇಳಿಸಿಕೊಳ್ಳುತ್ತಾನೆ ಎನ್ನುತ್ತಾರಲ್ಲ, ಅದು ಸತ್ಯವೆ? ಹಾಗಾದರೆ ನಾನು ಇಷ್ಟೊಂದು ಪ್ರಾರ್ಥಿಸಿಕೊಳ್ಳುತ್ತಿದ್ದೇನಲ್ಲ, ಅವನೇಕೆ ಕಿವಿಗೊಡುತ್ತಿಲ್ಲ? ಅವನ ಮಂಗಳಕರ ಸೃಷ್ಟಿಯಲ್ಲಿ ಇಷ್ಟೊಂದು ಅಮಂಗಳ, ದುಃಖಸಂಕಟಗಳೆಲ್ಲ ಇವೆಯಲ್ಲ, ಏಕೆ? ದಯಾಮಯನಾದ ಭಗವಂತನ ಸೃಷ್ಟಿಯಲ್ಲಿ ಭೂತ- ಪಿಶಾಚಿಗಳು ರಾಜ್ಯಭಾರ ಮಾಡುತ್ತಿವೆಯಲ್ಲ, ಏಕೆ?’ “ಭಗವಂತನು ಕರುಣಾಮಯ ಹಾಗೂ ಮಂಗಳಕರ ಎನ್ನುವುದಾದಲ್ಲಿ, ಬರಗಾಲದ ಸಮಯದಲ್ಲಿ ಕೋಟಿಗಟ್ಟಲೆ ಜನರು ಒಂದು ತುತ್ತು ಅನ್ನವಿಲ್ಲದೆ ಹಸಿವೆಯಿಂದ ನರಳಿ ಸಾಯುತ್ತಾರಲ್ಲ, ಏಕೆ?” ಎಂಬ ಪಂಡಿತ ಈಶ್ವರಚಂದ್ರ ವಿದ್ಯಾಸಾಗರರ ಮಾತುಗಳು ಅವನ ಕಿವಿಯಲ್ಲಿ ಗುಂಯಿಗುಡಲಾರಂಭಿಸಿದುವು. ಅವನಿಗೀಗ ಆ ದೇವರ ಮೇಲೆ ಭಯಂಕರ ಕೋಪ ಬಂದುಬಿಟ್ಟಿದೆ. ಯಾವಾಗ ಸ್ವತಃ ತನ್ನ ಹೆತ್ತ ತಾಯಿಯೇ ‘ಭಗವಂತ ನಿನಗೆ ಏನು ಮಾಡಿದ?’ ಎಂದು ಕೇಳಿಬಿಟ್ಟಳೋ ಆಗ ಅವನಿಗೂ ಅನ್ನಿಸಿತು, ‘ಹೌದಲ್ಲ! ಭಗವಂತ ನನಗೆ ಏನು ಮಾಡಿದ?’ ಎಂದು. ತಾಳಲಾರದ ಕಷ್ಟಗಳು ಮುತ್ತಿ ಕೊಂಡಾಗ ಎಂಥವರೂ ಕಂಗಾಲಾಗುವುದು ಸಹಜ. ಹೃದಯದಲ್ಲಿ ನಾಸ್ತಿಕತೆಯ ಭಾವನೆ ಹುಟ್ಟುವುದು ಇಂತಹ ಸಂದರ್ಭಗಳಲ್ಲೇ. ಆದರೆ ಇದೇ ನರೇಂದ್ರ ಮುಂದೆ ಸ್ವಾಮಿ ವಿವೇಕಾ ನಂದರಾಗಿ ಸಮಸ್ತ ಜಗತ್ತಿಗೆ ಆಸ್ತಿಕತೆಯ ಸಂದೇಶವನ್ನು ಧೀರತೆಯಿಂದ ಸಾರುವ ದೃಶ್ಯವನ್ನು ಕಂಡಾಗ, ಈಗ ಅವನಲ್ಲಿ ಉಂಟಾಗಿರುವ ನಾಸ್ತಿಕತೆಯೆನ್ನುವುದು ಕೇವಲ ತಾತ್ಕಾಲಿಕವಾದದ್ದು ಎಂಬುದನ್ನು ನಾವು ತಿಳಿಯಬೇಕು.

ನರೇಂದ್ರನದು ಮುಚ್ಚುಮರೆಯಿಲ್ಲದ ಸ್ವಭಾವ. ತನ್ನ ಮನಸ್ಸಿನಲ್ಲೆದ್ದ ಯಾವುದೇ ಆಲೋ ಚನೆಯನ್ನು ಇತರರಿಂದ ಮುಚ್ಚಿಡದಿರುವುದು ಅವನಲ್ಲಿ ಬಾಲ್ಯದಿಂದಲೂ ಬಂದ ಒಂದು ಗುಣ. ಯಾರದಾದರೂ ಹೆದರಿಕೆಯಿಂದಲೋ ಅಥವಾ ಒತ್ತಡದಿಂದಲೋ ತನ್ನ ಭಾವನೆಗಳನ್ನು ಅದುಮಿ ಅಡಗಿಸಿಡುವ ಸ್ವಭಾವವೇ ಅವನದಲ್ಲ. ಆದ್ದರಿಂದ ಈಗಲೂ ಕೂಡ ತನ್ನ ಮನಸ್ಸಿನಲ್ಲಿ ಧುಮುಧುಮಿಸಿ ಬರುವ ಭಾವನೆಗಳನ್ನು ಘಂಟಾಘೋಷವಾಗಿ ಪ್ರಕಟಿಸಲು ಪ್ರಾರಂಭಿಸಿದ. ಅವನೀಗ ಜಗತ್ತಿಗೆಲ್ಲ ಒಂದು ವಿಷಯವನ್ನು ಸಾರಿ ಹೇಳುತ್ತಿದ್ದಾನೆ–“ದೇವರು ಎನ್ನುವುದೇನೂ ಇಲ್ಲ. ಈ ದೇವರು ದಿಂಡರು ಎಲ್ಲ ಬರೀ ಸುಳ್ಳು; ಕೇವಲ ಕಂತೆಪುರಾಣ. ಅಥವಾ ಒಂದು ವೇಳೆ ಅವನಿರುವುದೇ ನಿಜವಾದರೂ ಅವನನ್ನು ಕರೆಯುವುದಾಗಲಿ ಪ್ರಾರ್ಥಿಸುವುದಾಗಲಿ ನಿರರ್ ಥಕ.” ನಾವು ಹಿಂದೆಯೇ ನೋಡಿದಂತೆ ಯಾವ ವಿಷಯವಾಗಿ ವಾದಿಸಿದರೂ ತೀವ್ರವಾಗಿಯೇ ವಾದಿಸುವವನು ನರೇಂದ್ರ. ಈಗ ಅವನ ನಾಸ್ತಿಕತೆಯ ವಾದವೂ ಅಷ್ಟೇ ಪ್ರಖರವಾಗಿತ್ತು. ಅದನ್ನು ಕೇಳಿದ ಜನರು ಭಾವಿಸಿದರು–ಇವನು ಸಂಪೂರ್ಣ ನಾಸ್ತಿಕನಾಗಿಬಿಟ್ಟ ಎಂದು. ಸುದ್ದಿ ಒಬ್ಬರಿಂ ದೊಬ್ಬರಿಗೆ ಬಹುಬೇಗ ಹರಡಿತು. ಒಂದನ್ನು ಹತ್ತಾಗಿ ಮಾಡುವುದು, ಹತ್ತನ್ನು ನೂರಾಗಿ ಮಾಡುವುದು ಜನಜಂಗುಳಿಯ ಸಹಜ ಸ್ವಭಾವ. ‘ನರೇಂದ್ರ ಪೂರ್ತಿ ನಾಸ್ತಿಕನಾಗಿಬಿಟ್ಟಿದ್ದಾನಂತೆ’ ಎಂಬ ಈ ಮಾತಿನಲ್ಲಿ ಜನರಿಗೆ ಅಷ್ಟೇನೂ ರುಚಿ ಕಾಣಲಿಲ್ಲ. ಅದರಲ್ಲಿ ಅಂಥ ಸ್ವಾರಸ್ಯವೇನಿದೆ? ಆದ್ದರಿಂದ ಇನ್ನೊಂದೆರಡನ್ನು ಸೇರ್ಪಡೆ ಮಾಡಿದರು. “ನರೇಂದ್ರ ಸಂಪೂರ್ಣ ನಾಸ್ತಿಕನಾಗಿಬಿಟ್ಟ ನಂತೆ!” “ಅಷ್ಟೇ ಆಗಿದ್ದರೆ ಪರವಾಗಿಲ್ಲ ಅನ್ನಬಹುದಾಗಿತ್ತು; ಕುಡಿತ ಬೇರೆ ಶುರು ಮಾಡಿಬಿಟ್ಟಿ ದ್ದಾನಂತೆ!” “ಅಯ್ಯೋ ಅದನ್ನೇನು ಹೇಳುತ್ತೀರಿ! ವೇಶ್ಯೆಯರ ಸಹವಾಸಕ್ಕೂ ಬಿದ್ದಿದ್ದಾನೆ ಎನ್ನು ತ್ತಾರೆ!”–ಹೀಗೆ ಸುದ್ದಿ ಭರದಿಂದ ಪ್ರಚಾರವಾಯಿತು. ಕಡೆಗೊಂದು ದಿನ ಅದು ನರೇಂದ್ರನ ಕಿವಿಯನ್ನೂ ತಲುಪಿತು. ಇದನ್ನು ಕೇಳಿದಾಗ ಅವನಿಗೆಷ್ಟು ರೋಸಿಹೋಗಿರಬೇಕು! ಇಂಥ ಮಾತು ಗಳು ತನ್ನ ತಾಯಿಯ ಕಿವಿಗೆ ಬಿದ್ದರೆ ಆಕೆ ಏನು ಭಾವಿಸಿಯಾಳು! ತಾನೇನೋ ನಾಸ್ತಿಕತೆಯ ಒಂದು ಮಾತನಾಡಿದರೆ ಈ ಜನ ಹೀಗೆ ಬಾಯಿಗೆ ಬಂದಂತೆ ಅಪಪ್ರಚಾರ ಮಾಡಲಾರಂಭಿಸಿ ದ್ದಾರಲ್ಲ, ಇದೆಂಥ ಅನ್ಯಾಯ! ಆದರೆ ಜನರಲ್ಲೆಲ್ಲ ಇದು ಹರಡಿಯಾಗಿದೆ. ಇನ್ನು ಇದು ಸತ್ಯವಲ್ಲವೆಂದು ಸಾಬೀತು ಮಾಡುವುದಾದರೂ ಸಾಧ್ಯವೆ? ನರೇಂದ್ರನಿಗೆ ಬಹಳ ದುಃಖ ವಾಯಿತು. ಆದರೆ ದುಃಖದಿಂದ ಅವನು ಕುಗ್ಗಲಿಲ್ಲ, ದುರ್ಬಲನಾಗಲಿಲ್ಲ. ಬದಲಾಗಿ ಅವನ ಮನಸ್ಸು ಇನ್ನಷ್ಟು ಗಡಸಾಯಿತು. ಆಗ ಅವನು ಇನ್ನಷ್ಟು ರಾಜಾರೋಷವಾಗಿ ಹೇಳಲಾರಂಭಿಸಿದ –“ಈ ಸಂಕಟಮಯವಾದ ಜಗತ್ತಿನಲ್ಲಿ ದುಃಖದಿಂದ ನರಳುತ್ತಿರುವ ಮನುಷ್ಯ ಅದನ್ನು ಮರೆಯುವುದಕ್ಕಾಗಿ ಒಂದಿಷ್ಟು ಕುಡಿದರೂ ಅದರಲ್ಲೇನೂ ದೋಷವಿಲ್ಲ. ಅಷ್ಟು ಮಾತ್ರವಲ್ಲ, ಅಂಥ ಅಕೃತ್ಯಗಳು ನಿಜವಾಗಿಯೂ ಶಾಂತಿ-ಸಮಾಧಾನ ಕೊಡಬಲ್ಲವು ಎಂದು ನನಗನಿಸಿದರೆ, ನಾನೂ ಕೂಡ ಯಾರಿಗೂ ಹೆದರಿಕೊಳ್ಳದೆ ನನ್ನಿಷ್ಟದಂತೆಯೇ ನಡೆದುಕೊಂಡೇನು!” ಅವನು ಹೀಗೆಲ್ಲ ಮಾತನಾಡಿದ್ದು ಜನ ತನ್ನ ಮೇಲೆ ಅನ್ಯಾಯವಾಗಿ ದೋಷಾರೋಪಣೆ ಮಾಡುತ್ತಿ ದ್ದಾರಲ್ಲ ಎಂಬ ರೋಷದ ಭರದಲ್ಲಿ. ಆದರೆ ಜನರಿಗೀಗ ಒಂದು ಪ್ರತ್ಯಕ್ಷ ಆಧಾರವೇ ಸಿಕ್ಕಂತಾಯಿತು. ಈಗ ಅವರೆಲ್ಲ ಅನ್ನಲಾರಂಭಿಸಿದರು–“ನೋಡಿದಿರಾ! ನಾವು ಹೇಳಿದ್ದು ಸುಳ್ಳಲ್ಲ. ನರೇಂದ್ರ ನಾಸ್ತಿಕನಾದದ್ದೂ ಹೌದು, ಕುಡಿತಕ್ಕಿಳಿದಿದ್ದೂ ಹೌದು, ವ್ಯಭಿಚಾರಿಯಾ ದದ್ದೂ ಹೌದು. ಅವನೇ ಒಪ್ಪಿಕೊಳ್ಳುತ್ತಿದ್ದಾನಲ್ಲ! ಮನುಷ್ಯ ಸುಖಕ್ಕಾಗಿ ಏನು ಮಾಡಿದರೂ ತಪ್ಪಿಲ್ಲ ಅಂತ ಅವನೇ ಹೇಳುತ್ತಿದ್ದಾನಲ್ಲ!” ಎಂದು. ಅಂತೂ ಈ ಜನರಿಗೆ ನರೇಂದ್ರ ‘ದಾರಿ ತಪ್ಪಿರುವುದು’ ಮತ್ತಷ್ಟು ದೃಢವಾಯಿತು.

ಇದೆಲ್ಲ ಇಲ್ಲಿಗೇ ನಿಲ್ಲಲಿಲ್ಲ. ದಕ್ಷಿಣೇಶ್ವರದವರೆಗೂ ಹೋಯಿತು ಈ ಅಪಲಾಪ. ಶ್ರೀರಾಮ ಕೃಷ್ಣರು ಮಾತ್ರ ಒಂದು ಮಾತನ್ನೂ ಆಡದೆ ತಟಸ್ಥವಾಗಿದ್ದುಬಿಟ್ಟರು. ಅವರ ಭಕ್ತರಿಗಂತೂ ಇದೊಂದು ನಂಬಲಾಗದ ಸುದ್ದಿ. ಆದರೆ ಹಲವಾರು ಜನ ಮಾತನಾಡಿಕೊಳ್ಳುತ್ತಿರುವಾಗ ನಿಜಾಂಶವಿದ್ದರೂ ಇರಬಹುದು ಎಂಬ ಅನುಮಾನವುಂಟಾಯಿತು. ಕೊನೆಗೆ ನರೇಂದ್ರನ ಬಳಿಗೇ ಹೋಗಿ ಮುಖಾಮುಖಿ ಮಾತನಾಡಿ ಸತ್ಯವೇನೆಂಬುದನ್ನು ತಿಳಿದುಕೊಂಡು ಬರುವುದೇ ಸರಿ ಎಂದು ನಿರ್ಧರಿಸಿ, ಅವರಲ್ಲಿ ಕೆಲವರು ಬಂದು ಕೇಳಿದರು, “ಏನಪ್ಪ ನರೇನ್, ಜನ ಏನೇನೋ ಮಾತನಾಡಿಕೊಳ್ಳುತ್ತಿದ್ದಾರಲ್ಲ! ಕೇಳಿ ನಮಗೆಲ್ಲ ತುಂಬ ಬೇಸರವಾಯಿತು...” ಅವರು ಮಾತನಾಡಿದ ರೀತಿ ನೋಡಿದರೆ ಅವರು ಆ ಸುದ್ದಿಯಲ್ಲಿ ಕೆಲವಂಶವನ್ನಾದರೂ ನಂಬಿದ್ದಾ ರೆನ್ನುವುದು ಸ್ಪಷ್ಟವಾಗಿತ್ತು. ಅದನ್ನು ಕಂಡು ನರೇಂದ್ರನಿಗೆ ಮತ್ತಷ್ಟು ನೋವಾಯಿತು. ಅವನ ಅಭಿಮಾನಕ್ಕೆ ಧಕ್ಕೆಯಾಯಿತು. ‘ಛೆ, ಇದೇನಿದು? ಇವರೂ ನನ್ನನ್ನು ಅರ್ಥಮಾಡಿಕೊಳ್ಳಲಾರದೆ ಹೋದರಲ್ಲ! ನನ್ನನ್ನು ಅಂಥ ನೀಚ ಎಂದು ತಿಳಿದುಕೊಂಡುಬಿಟ್ಟಿದ್ದಾರಲ್ಲ!’ ಎನ್ನಿಸಿತು. ಅವರು ಪಾಪ-ಪುಣ್ಯಗಳ ಮಾತನ್ನೆತ್ತಿ, ಬುದ್ಧಿ ಹೇಳಲು ಹೊರಟಾಗಲಂತೂ ಇನ್ನಷ್ಟು ಕೆರಳಿ, “ನೋಡಿ, ನರಕಭೀತಿಯಿಂದ ದೇವರನ್ನು ನಂಬುವುದು ಹೇಡಿತನ” ಎಂದು ಕಟುವಾಗಿ ಹೇಳಿಬಿಟ್ಟ. ಪಾಶ್ಚಾತ್ಯ ಚಿಂತಕರ ಅಭಿಪ್ರಾಯಗಳನ್ನೆಲ್ಲ ಉದಾಹರಿಸಿ ದೇವರ ಅಸ್ತಿತ್ವದ ಕುರಿತಾದ ಅವರ ನಂಬಿಕೆಯನ್ನೆಲ್ಲ ಈಡಾಡಿದ. ಇದರ ಪರಿಣಾಮವಾಗಿ ಆ ಭಕ್ತರು ಅವನನ್ನು ಇನ್ನಷ್ಟು ಅಪಾರ್ಥ ಮಾಡಿಕೊಂಡರು. ‘ಇವನು ಸಂಪೂರ್ಣ ಹಾಳಾಗಿರುವುದೇ ನಿಜ’ ಎಂದು ನಿರ್ಧರಿಸಿ, ಅಲ್ಲಿಂದ ಹೊರಟರು. ಇದನ್ನು ಕಂಡು ಅವನಿಗೆ ಒಂದು ಬಗೆಯಲ್ಲಿ ಸಂತೋಷವೇ ಆಯಿತು. ‘ಹೇಗಿದ್ದರೂ ತಪ್ಪಾಗಿಯೇ ಭಾವಿಸುವ ಜನ ಇನ್ನಷ್ಟು ಚೆನ್ನಾಗಿಯೇ ತಪ್ಪಾಗಿ ತಿಳಿದುಕೊಂಡು ಹೋಗಲಿ, ಏನೀಗ!’ ಎಂದುಕೊಂಡ. ಆದರೆ ಇನ್ನೊಂದು ಬಗೆಯಲ್ಲಿ ಆಲೋಚಿಸಿದಾಗ ದುಃಖವುಕ್ಕಿಬಂತು. ಆ ಭಕ್ತರು ಶ್ರೀರಾಮಕೃಷ್ಣರ ಬಳಿಗೆ ಹೋಗಿ ತಾವು ಕಲ್ಪಿಸಿಕೊಂಡಿರುವುದನ್ನೇ ಹೇಳಿ, ಅವರೂ ಅದನ್ನು ನಂಬಿಬಿಟ್ಟರೆ!... ಆಮೇಲೆ ತಾನಲ್ಲಿಗೆ ಹೋಗಿ ಮುಖ ತೋರಿಸುವುದಾದರೂ ಹೇಗೆ? ಆದರೆ ಮರುಕ್ಷಣಕ್ಕೇ ಅವನ ಮನಸ್ಸು ಗಡಸಾಗಿ ಪ್ರತಿಕ್ರಿಯೆ ತೋರಿಸಿತು–‘ಆಗಲಿ, ಪರವಾಗಿಲ್ಲ; ಜನದ ಮಾತಿನಿಂದೆಲ್ಲ ಅವರು ನನ್ನನ್ನು ಅಳೆಯುವಂತಿದ್ದರೆ ನಾನು ಯಾರನ್ನೂ ಲೆಕ್ಕಿಸುವವ ನಲ್ಲ.’ ಆದರೆ ಶ್ರೀರಾಮಕೃಷ್ಣರು ತಮ್ಮ ನೆಚ್ಚಿನ ಶಿಷ್ಯನನ್ನು ಎಂದಿಗಾದರೂ ಅಪಾರ್ಥ ಮಾಡಿ ಕೊಂಡಾರೆಯೆ? ತಮ್ಮ ಸ್ವಂತ ಭಕ್ತರೇ ಬಂದು ನರೇಂದ್ರನ ಬಗ್ಗೆ ತಾವು ಕಂಡುಕೇಳಿದ ಸುದ್ದಿಗಳನ್ನೆಲ್ಲ ತಿಳಿಸಿದರೂ ಪ್ರತಿಕ್ರಿಯಿಸದೆ ಸುಮ್ಮನೆಯೇ ಇದ್ದರು. ಅವರು ಹಾಗೆ ಸುಮ್ಮನಿದ್ದು ದಕ್ಕೆ ಕಾರಣ ನರೇಂದ್ರನ ಮೇಲಿನ ಮೋಹವಲ್ಲ. ಬದಲಾಗಿ ಆತ ಪುಟವಿಟ್ಟ ಅಪರಂಜಿಯಂತೆ ಪರಿಶುದ್ಧ ಎಂಬುದು ಅವರಿಗೆ ಗೊತ್ತು. ಶ್ರೀರಾಮಕೃಷ್ಣರ ಪ್ರಿಯ ಶಿಷ್ಯರಲ್ಲೊಬ್ಬನಾದ ಭವನಾಥ ಅಶ್ರುಭರಿತ ಕಂಗಳಿಂದ ಹೇಳಿದ–“ಸ್ವಾಮಿ, ನರೇಂದ್ರ ಇಷ್ಟೊಂದು ಕೆಳಮಟ್ಟಕ್ಕಿಳಿದಾನು ಅಂತ ನಾನು ಕನಸಿನಲ್ಲೂ ಭಾವಿಸಿರಲಿಲ್ಲ” ಎಂದು. ಆಗ ಮಾತ್ರ ಶ್ರೀರಾಮಕೃಷ್ಣರು ರೇಗಿಬಿಟ್ಟರು: “ಮೂರ್ಖ, ಹಾಗೆಲ್ಲ ಮಾತನಾಡಬೇಡ! ನರೇಂದ್ರ ಎಂದಿಗೂ ದಾರಿತಪ್ಪಿ ನಡೆಯಲಾರ ಅಂತ ಜಗನ್ಮಾತೆ ನನಗೆ ಸ್ಪಷ್ಟವಾಗಿ ಹೇಳಿದ್ದಾಳೆ. ನೋಡಿಕೊ, ನೀನು ಇನ್ನೊಂದು ಸಲ ಏನಾದರೂ ಅವನ ಬಗ್ಗೆ ಹಾಗೆಲ್ಲ ಮಾತನಾಡಿದರೆ ನಾನು ನಿನ್ನ ಮುಖ ಕೂಡ ನೋಡುವುದಿಲ್ಲ!” ಮುಂದೆ ಈ ವಿಷಯಗಳೆಲ್ಲ ತಿಳಿದುಬಂದಾಗ ನರೇಂದ್ರನಿಗಾದ ಆಶ್ಚರ್ಯ-ಆನಂದ ಅಷ್ಟಿಷ್ಟಲ್ಲ. ತನ್ನ ಮೇಲಿನ ಅವರ ವಿಶ್ವಾಸ ಪ್ರೇಮಗಳನ್ನು ಮನಗಂಡು ಅವನ ಕಣ್ಣಲ್ಲಿ ನೀರು ತುಂಬಿಬಂತು.

ನರೇಂದ್ರ ಪರಿಸ್ಥಿತಿಯ ಒತ್ತಡಕ್ಕೆ ಸಿಲುಕಿ, ತೋರಿಕೆಗೆ ನಾಸ್ತಿಕನಂತೆ ಮಾತಾಡುತ್ತಿದ್ದರೂ ಅವನಿಗೇ ಅದರಲ್ಲಿ ನಂಬಿಕೆಯಿರಲಿಲ್ಲ. ಬಾಲ್ಯದಿಂದಲೂ ತನಗಾಗಿದ್ದ ವಿವಿಧ ಆಧ್ಯಾತ್ಮಿಕ ದರ್ಶನಗಳ, ಅನುಭವಗಳ ನೆನಪು ಮರುಕಳಿಸತೊಡಗಿತ್ತು. ಅದರಲ್ಲೂ ಶ್ರೀರಾಮಕೃಷ್ಣರ ದಿವ್ಯ ಸಂಪರ್ಕದಿಂದ ತಾನು ಪಡೆದುಕೊಂಡ ಅತೀಂದ್ರಿಯ ಅನುಭವಗಳು ಅವನ ಮನಸ್ಸಿನಲ್ಲಿ ನಿಚ್ಚಳವಾಗಿ ಅಚ್ಚೊತ್ತಿದ್ದವು. ಈ ಅನುಭವಗಳು ಅವನನ್ನು ನಾಸ್ತಿಕನಾಗಲು ಬಿಡಲಾರವು. ದೈವಶ್ರದ್ಧೆಯೆಂಬುದು ಅವನಲ್ಲಿ ರೂಢಮೂಲವಾಗಿತ್ತು. ಎಲ್ಲ ಕಷ್ಟ-ತಾಪತ್ರಯಗಳ ನಡು ವೆಯೂ ಅವನ ಮನಸ್ಸೀಗ ತೀವ್ರವಾಗಿ ಆಲೋಚಿಸುತ್ತಿದೆ–‘ಭಗವಂತನೆಂಬವನು ಇರಲೇ ಬೇಕು. ಭಗವಂತನನ್ನು ಸಾಕ್ಷಾತ್ಕರಿಸಿಕೊಳ್ಳದೆ ಹೋದರೆ ಈ ಜೀವನಕ್ಕೆ ಅರ್ಥವೇನಿದೆ? ಸೊಗಸೇನಿದೆ? ನಾನು ಆ ಸಾಕ್ಷಾತ್ಕಾರದ ಮಾರ್ಗವನ್ನು ಕಂಡುಕೊಳ್ಳಲೇಬೇಕು.’ ಹೀಗೆ ಅವನ ಮನಸ್ಸು ಸಂಶಯ ನಂಬಿಕೆಗಳ ನಡುವೆ ಡೋಲಾಯಮಾನವಾಗಿತ್ತು.

ದಿನಗಳ ಮೇಲೆ ದಿನಗಳು ಉರುಳುತ್ತಿವೆ. ಈ ನಡುವೆ ಅವನ ಆರ್ಥಿಕ ಪರಿಸ್ಥಿತಿ ಇನ್ನೂ ಹಾಗೇ ಇದೆ–ಮೊದಲಿನ ದುಸ್ಥಿತಿಯೇ ಮುಂದುವರಿಯುತ್ತಿದೆ. ಕೆಲಸಕ್ಕಾಗಿ ಅಲೆಯುತ್ತಲೇ ಇದ್ದಾನೆ. ಒಂದು ದಿನ ಹೀಗೆಯೇ ಕೆಲಸ ಹುಡುಕಿಕೊಂಡು ಹೊರಟಿದ್ದಾನೆ; ಬೆಳಗ್ಗಿನಿಂದ ಸುತ್ತಾಡು ತ್ತಿದ್ದಾನೆ. ಆ ದಿನ ಮಳೆ ಬೇರೆ. ಮಧ್ಯಾಹ್ನವಾಯಿತು. ಸಂಜೆಯೂ ಆಯಿತು. ಕೆಲಸ ಸಿಗಲಿಲ್ಲ, ಊಟವೂ ಇಲ್ಲ. ಒಂದು ಕಡೆ ದಿನವಿಡೀ ಊಟವಿಲ್ಲದೆ ಹಸಿದಿದ್ದಾನೆ, ಇನ್ನೊಂದು ಕಡೆ ಮಳೆಯಲ್ಲಿ ತೋಯುತ್ತ ಅಲೆದಲೆದು ಸೋತಿದ್ದಾನೆ. ಆದರಿನ್ನೇನು ತಾನೆ ಮಾಡಲು ಸಾಧ್ಯ? ಸೋತ ಕಾಲುಗಳನ್ನು ಮನೆಯ ಕಡೆಗೆ ಬಲವಂತವಾಗಿ ಎಳೆದುಕೊಂಡು ಹೋಗುತ್ತಿದ್ದಾನೆ. ಮನಸ್ಸು ಮಂಕಾಗಿಬಿಟ್ಟಿದೆ. ನಿಶ್ಶಕ್ತಿಯಿಂದ ಇನ್ನು ಒಂದು ಹೆಜ್ಜೆ ಮುಂದಿಡಲೂ ಸಾಧ್ಯವಿಲ್ಲ ವೆನಿಸಿತು. ಕಣ್ಣುಕತ್ತಲೆ ಬಂದು ಅಲ್ಲೇ ರಸ್ತೆಯ ಪಕ್ಕದ ಮನೆಯೊಂದರ ಗೋಡೆಗೆ ಒರಗಿ ಕುಸಿದು ಕುಳಿತುಕೊಂಡ. ಒಂದು ಕ್ಷಣ ಪ್ರಜ್ಞೆ ತಪ್ಪಿದಂತಾಯಿತು. ಮರುಕ್ಷಣವೇ ಎಚ್ಚರವಾಯಿತು. ಆಗ ಅವನ ಮನಸ್ಸನ್ನು ನಾನಾ ಬಗೆಯ ಆಲೋಚನೆಗಳು ಮುತ್ತಿಕೊಂಡು ದಾಳಿಮಾಡಿಬಿಟ್ಟುವು. ಅವನು ಆ ಆಲೋಚನೆಗಳನ್ನು ಓಡಿಸಲೂ ಆರ, ಅಷ್ಟೊಂದು ನಿತ್ರಾಣನಾಗಿದ್ದ. ಆಗ ಒಂದು ಅದ್ಭುತ ಸಂಭವಿಸಿತು. ಇದ್ದಕ್ಕಿದ್ದಂತೆ ಯಾವುದೋ ಅಲೌಕಿಕ ಶಕ್ತಿಯ ಪ್ರಭಾವದಿಂದಲೋ ಎಂಬಂತೆ ತನ್ನ ಅಂತರಾತ್ಮವನ್ನು ಸುತ್ತಿದ್ದ ಆವರಣಗಳೆಲ್ಲ ಒಂದೊಂದಾಗಿ ಕಳಚಿಹೋಗುತ್ತಿರು ವಂತೆ ಅವನಿಗೆ ಭಾಸವಾಯಿತು. ಇಷ್ಟು ದಿನವೂ ಹಲವಾರು ಸಮಸ್ಯೆಗಳು ಅವನ ಮನಸ್ಸಿನಲ್ಲಿ ಸೇರಿಕೊಂಡು ಅವನನ್ನು ಪೀಡಿಸುತ್ತಿದ್ದುವು: ‘ಭಗವತ್ಕೃಪೆ ಮತ್ತು ವಿಧಿನಿಯಮ ಎಂಬವು ಈ ಜಗತ್ತಿನಲ್ಲಿ ಒಟ್ಟೊಟ್ಟಿಗೆ ಇರಲು ಸಾಧ್ಯವೆ? ಭಗವಂತನು ನಿಜಕ್ಕೂ ದಯಾಮಯನಾಗಿದ್ದಲ್ಲಿ, ಅವನದೇ ಆದ ಈ ಸೃಷ್ಟಿಯಲ್ಲಿ ಇಷ್ಟೊಂದು ದುಃಖದಾರಿದ್ರ್ಯಗಳು, ಕಷ್ಟಸಂಕಟಗಳು ಇರಲು ಹೇಗೆ ಸಾಧ್ಯ?’–ಇಂಥವು. ಆದರೆ ಅವನ ಆತ್ಮವನ್ನು ಮುಚ್ಚಿದ್ದ ತೆರೆಗಳು ಒಂದೊಂದಾಗಿ ಕಳಚಿದಾಗ ಈ ಎಲ್ಲ ಸಮಸ್ಯೆಗಳೂ ಆಶ್ಚರ್ಯಕರವಾಗಿ ಪರಿಹಾರವಾದುವು. ಆಳವಾದ ಆತ್ಮಾವ ಲೋಕನದಿಂದ ಸೃಷ್ಟಿಯ ಈ ನಿಯಮಗಳು ಅವನಿಗೆ ಗೋಚರವಾದುವು. ಈಗ ಅವನ ಮನಸ್ಸು ಸಂಪೂರ್ಣ ಶಾಂತವಾಗಿತ್ತು, ಸಮಾಧಾನಗೊಂಡಿತ್ತು, ಮೆಲ್ಲನೆ ಅಲ್ಲಿಂದೆದ್ದು ಮನೆಯ ಕಡೆಗೆ ಹೆಜ್ಜೆ ಹಾಕಿದ. ಆಶ್ಚರ್ಯ! ಅವನ ನಿಶ್ಶಕ್ತಿಯೆಲ್ಲ ಮಾಯವಾಗಿತ್ತು. ಅವನಲ್ಲಿ ಒಂದು ನವಶಕ್ತಿ ಉದಿಸಿತ್ತು. ಮನಸ್ಸಿನಲ್ಲಿ ಒಂದು ವಿಶೇಷ ಹುರುಪುಂಟಾಗಿತ್ತು. ಅವನು ಅಪಾರವಾದ ಶಾಂತಿ ಯನ್ನು ಅನುಭವಿಸಲಾರಂಭಿಸಿದ.

ಈ ಅಲೌಕಿಕ ಅನುಭವವಾದ ಮೇಲೆ ನರೇಂದ್ರನ ಸ್ವಭಾವದಲ್ಲೊಂದು ಮಹತ್ತರ ಬದಲಾ ವಣೆಯಾಯಿತು. ಅವನು ಜನರ ಹೊಗಳಿಕೆ-ತೆಗಳಿಕೆಗಳಿಗೆಲ್ಲ ಕಿವಿಗೊಡುವುದನ್ನು ಪೂರ್ತಿಯಾಗಿ ಬಿಟ್ಟುಬಿಟ್ಟ. ಅವನು ಹಿಂದೆಯೇ ತಿಳಿದಿದ್ದ ಒಂದು ವಿಷಯ ಈಗ ಇನ್ನಷ್ಟು ಸ್ಪಷ್ಟವಾಯಿತು. ತಾನು ಸಾಮಾನ್ಯ ಜನರಂತೆ ಹಣ ಸಂಪಾದನೆ ಮಾಡಿ ಸಂಸಾರ ಸಾಗಿಸುವುದಕ್ಕಾಗಿ ಜನ್ಮವೆತ್ತಿದವ ನಲ್ಲ; ಇನ್ನು ಇಂದ್ರಿಯಸುಖಗಳನ್ನು ಅನುಭವಿಸುವ ಮಾತಂತೂ ದೂರವೇ ಉಳಿಯಿತು– ಎಂಬ ಈ ನಂಬಿಕೆ ಅವನಲ್ಲಿ ಆಳವಾಗಿ ಬೇರೂರಿತು. ಎಂದಮೇಲೆ, ತನ್ನ ಈ ಪರದಾಟವೆಲ್ಲ ನಿರರ್ಥಕವಲ್ಲವೆ? ಆದ್ದರಿಂದ, ಈ ಮಾಯಾಬಂಧನಗಳನ್ನೆಲ್ಲ ಕತ್ತರಿಸಿಕೊಂಡು ತನ್ನ ತಾತ ನಂತೆಯೇ ಸಂನ್ಯಾಸಿಯಾಗಿ ಮನೆ ಬಿಟ್ಟು ಹೊರಡಲು ಗುಟ್ಟಾಗಿ ಸಿದ್ಧತೆ ನಡೆಸಿ ಅದಕ್ಕೊಂದು ದಿನವನ್ನೂ ನಿಶ್ಚಯಿಸಿದ. ಅವನ ಅದೃಷ್ಟಕ್ಕೆ, ತಾನು ಹೊರಡಬೇಕೆಂದಿದ್ದ ದಿನದಂದೇ ಶ್ರೀರಾಮ ಕೃಷ್ಣರು ಕಲ್ಕತ್ತಕ್ಕೆ ಬರುವವರಿದ್ದಾರೆ ಎಂಬ ವರ್ತಮಾನ ಸಿಕ್ಕಿತು. ಇದನ್ನು ಕೇಳಿ ಅವನಿಗೆ ಬಹಳ ಸಂತೋಷವಾಯಿತು. ‘ಆಹ್! ನಿಜಕ್ಕೂ ಇದು ಶುಭಸೂಚನೆ. ಅಂದು ಗುರುಗಳ ಆಶೀರ್ವಾದ ಪಡೆದುಕೊಂಡೇ ಸರ್ವಸಂಗ ಪರಿತ್ಯಾಗ ಮಾಡಿಬಿಡುತ್ತೇನೆ’ ಎಂದು ಮನಸ್ಸಿನಲ್ಲೇ ಹೇಳಿಕೊಂಡ. ಅಂತೆಯೇ ಆ ದಿನ ಹೋಗಿ ಶ್ರೀರಾಮಕೃಷ್ಣರನ್ನು ಭೇಟಿ ಮಾಡಿ ನಮಸ್ಕರಿಸಿದ. ಅವನನ್ನು ಕಂಡ ತಕ್ಷಣವೇ ಶ್ರೀರಾಮಕೃಷ್ಣರಿಗೆ ಅವನ ಮನದ ಉದ್ದೇಶ ತಿಳಿದುಹೋಯಿತು. ಆದ್ದರಿಂದ “ನೋಡು, ನೀನು ಇಂದು ದಕ್ಷಿಣೇಶ್ವರಕ್ಕೆ ಬಂದು ನನ್ನೊಡನೆಯೇ ಈ ರಾತ್ರಿಯನ್ನು ಕಳೆಯ ಬೇಕು” ಎಂದು ಒತ್ತಾಯಪಡಿಸಿದರು. ಆದರೆ ನರೇಂದ್ರ ಅಂದೇ ಸರ್ವಸಂಗ ಪರಿತ್ಯಾಗ ಮಾಡಿ ಹೊರಟುಹೋಗಲು ಯೋಜನೆ ಹಾಕಿಕೊಂಡು ಸಿದ್ಧನಾಗಿರುವವನು. ಆದ್ದರಿಂದ ದಕ್ಷಿಣೇಶ್ವರಕ್ಕೆ ಹೋಗುವುದನ್ನು ತಪ್ಪಿಸಿಕೊಳ್ಳಲು ನಾನಾ ಕಾರಣಗಳನ್ನು ಒಡ್ಡಿದ. ಆದರೆ ಏನೂ ಪ್ರಯೋಜನ ವಾಗಲಿಲ್ಲ. ಅವನು ಅವರ ಜೊತೆಯಲ್ಲಿ ಹೋಗಲೇಬೇಕಾಯಿತು. ಇಬ್ಬರೂ ಒಂದೇ ಗಾಡಿಯಲ್ಲಿ ಸಾಗಿದರು. ದಾರಿಯಲ್ಲಿ ವಿಶೇಷವಾದ ಮಾತುಕತೆಯೇನೂ ನಡೆಯಲಿಲ್ಲ. ಆದರೆ ದಕ್ಷಿಣೇಶ್ವರ ದಲ್ಲಿ ತಮ್ಮ ಕೋಣೆಯನ್ನು ತಲುಪಿದೊಡನೆಯೇ ಶ್ರೀರಾಮಕೃಷ್ಣರು ಭಾವಾವಿಷ್ಟರಾದರು. ಆ ಸ್ಥಿತಿಯಲ್ಲೇ ಮೆಲ್ಲನೆ ನರೇಂದ್ರನ ಬಳಿಸಾರಿದರು. ಅತ್ಯಂತ ವಿಶ್ವಾಸದಿಂದ ಅವನನ್ನು ಸ್ಪರ್ಶಿಸುತ್ತ, ಕಂಬನಿದುಂಬಿ ಒಂದು ಹಾಡನ್ನು ಹಾಡಿದರು:

\begin{myquote}
ನುಡಿಯಲಾರೆವು ನಾವು ರಾಧಾ, ನುಡಿಯದಿರಲೂ ಅರೆವು!\\ನಾವು ನಿನ್ನನು ಕಳೆದುಕೊಂಡೇವೆಂಬ ಭಯದೊಳಗಿರುವೆವು
\end{myquote}

ಈ ಹಾಡನ್ನು ಅವರು ನರೇಂದ್ರನ ಕುರಿತೇ ಹಾಡಿದರು. ಅಲ್ಲಿಯವರೆಗೂ ನರೇಂದ್ರ ತನ್ನೊಳಗಿನ ಭಾವಾವೇಗವನ್ನು ತಡೆದುಕೊಂಡಿದ್ದ. ಆದರೆ ಶ್ರೀರಾಮಕೃಷ್ಣರು ಆ ಮರ್ಮಸ್ಪರ್ಶಿ ಯಾದ ಹಾಡನ್ನು ಹಾಡಿದಾಗ ಮಾತ್ರ ಇನ್ನು ತಡೆದುಕೊಳ್ಳುವುದಾಗದೆ ಧಾರಾಕಾರವಾಗಿ ಕಣ್ಣೀರು ಸುರಿಸುತ್ತ ಅತ್ತುಬಿಟ್ಟ.

ಹೀಗೆ ಶಿಷ್ಯನ ಮನೋಭಿಪ್ರಾಯ ಗುರುವಿಗೆ ತಿಳಿದಿದೆ, ಗುರುವಿನ ಹಾಡಿನ ಅರ್ಥ ಶಿಷ್ಯನಿಗೂ ಗೊತ್ತಾಗಿದೆ. ಆದ್ದರಿಂದ ಇಬ್ಬರೂ ಕಣ್ಣೀರು ಸುರಿಸುತ್ತಿದ್ದಾರೆ. ಆದರೆ ಇವರಿಬ್ಬರೂ ಹೀಗೆ ಮಾತಿಲ್ಲದೆ ಕಣ್ಣೀರು ಹರಿಸುತ್ತಿರುವುದನ್ನು ಕಂಡ ಭಕ್ತರಿಗೆಲ್ಲ ಆಶ್ಚರ್ಯವೋ ಆಶ್ಚರ್ಯ. ಪಾಪ, ಅವರಿಗೆ ಇವರಿಬ್ಬರ ನಡುವಿನ ಗುಟ್ಟು ಹೇಗೆ ಅರ್ಥವಾಗಬೇಕು! ಸ್ವಲ್ಪ ಹೊತ್ತಿನ ಮೇಲೆ ಶ್ರೀರಾಮಕೃಷ್ಣರು ಸಹಜ ಸ್ಥಿತಿಗೆ ಬಂದಾಗ ಒಬ್ಬ ಭಕ್ತ ಅವರಿಬ್ಬರೂ ಹಾಗೆ ಕಣ್ಣೀರು ಸುರಿಸಿದ್ದಕ್ಕೆ ಕಾರಣ ಕೇಳಿದ. ಅದಕ್ಕೆ ಶ್ರೀರಾಮಕೃಷ್ಣರು ಮುಗುಳುನಕ್ಕು “ಅದೆಲ್ಲ ನಮ್ಮನಮ್ಮೊಳಗಿನ ವಿಷಯ” ಎಂದುಬಿಟ್ಟರು. ಹೀಗೆ ಆ ದಿನ ಕಳೆಯಿತು, ರಾತ್ರಿಯಾಯಿತು. ಶ್ರೀರಾಮಕೃಷ್ಣರು ಇತರ ಭಕ್ತರನ್ನೆಲ್ಲ ಕಳಿಸಿ ನರೇಂದ್ರನೊಬ್ಬನನ್ನೇ ಹತ್ತಿರಕ್ಕೆ ಕರೆದು ಕುಳ್ಳಿರಿಸಿಕೊಂಡು ನುಡಿದರು, “ನರೇನ್, ನನಗೆ ಗೊತ್ತು, ನೀನು ಜನ್ಮ ತಾಳಿರುವುದೇ ಜಗನ್ಮಾತೆಯ ಕಾರ್ಯವನ್ನು ಸಾಧಿಸುವು ದಕ್ಕಾಗಿ ಎಂದು. ಆದರೆ ನಾನು ಬದುಕಿರುವವರೆಗೆ ಮಾತ್ರ ನೀನು ಮನೆಯಲ್ಲೇ ಇರಬೇಕು.” ಹೀಗೆ ಹೇಳುತ್ತ ಮತ್ತೆ ಗಟ್ಟಿಯಾಗಿ ಅತ್ತುಬಿಟ್ಟರು. ಅಂತೂ ತಾನು ಅಂದೇ ಸರ್ವಸಂಗ ಪರಿತ್ಯಾಗ ಮಾಡಿಬಿಡಬೇಕು ಎಂಬ ನಿರ್ಧಾರವನ್ನು ನರೇಂದ್ರ ಬದಲಿಸಬೇಕಾಯಿತು. ಮರುದಿನ ಬೆಳಗ್ಗೆ ಎದ್ದು, ಶ್ರೀರಾಮಕೃಷ್ಣರಿಗೆ ನಮಿಸಿ ಮನೆಗೆ ಹೊರಟ.

ಈಗ ತನ್ನ ತಾಯಿಯನ್ನೂ ಸೋದರ-ಸೋದರಿಯರನ್ನೂ ಸಲಹುವ ಚಿಂತೆ ಮತ್ತೆ ಅವನ ಹೆಗಲೇರಿತು. ಪುನಃ ಉದ್ಯೋಗದ ಬೇಟೆಗೆ ಹೊರಟ. ಯಾವುದೋ ವಕೀಲರ ಆಫೀಸಿನಲ್ಲಿ ಸಣ್ಣ ಕೆಲಸ ಸಿಕ್ಕಿತು. ಕೆಲವು ಪುಸ್ತಕಗಳನ್ನು ಅನುವಾದಿಸುವ ಕೆಲಸವನ್ನೂ ಮಾಡಿದ. ಇದರಿಂದೆಲ್ಲ ಎರಡು ಹೊತ್ತಿನ ಊಟಕ್ಕಾಗುವಷ್ಟು ಸಿಗುತ್ತಿತ್ತು ಅಷ್ಟೆ. ಇದಾವುದೂ ಶಾಶ್ವತವಾದ ಉದ್ಯೋಗ ವೇನೂ ಆಗಿರಲಿಲ್ಲ. ತಾಯಿಯ ಹಾಗೂ ಒಡಹುಟ್ಟಿದವರ ಅಶನಾರ್ಥಕ್ಕೆ ಒಂದು ಶಾಶ್ವತ ವ್ಯವಸ್ಥೆ ಮಾಡಬೇಕೆಂದು ಅವನೆಷ್ಟೋ ಪ್ರಯತ್ನಪಟ್ಟ. ಆದರೆ ಕೊನೆಗೂ ಅದು ಅವನಿಂದ ಸಾಧ್ಯವಾಗದೆ ಹೋಯಿತು. ಭಗವಂತನ ಲೀಲೆ ನಿಜಕ್ಕೂ ವಿಚಿತ್ರವಾದದ್ದೇ ಸರಿ. ವಿದ್ಯಾ ಬುದ್ಧಿ ತೇಜಸ್ಸು ಪರಾಕ್ರಮಗಳಿಂದ ತುಂಬಿ ತುಳುಕಾಡುತ್ತಿರುವ ನರೇಂದ್ರನಂತಹ ಅಸಾಧಾರಣ ತರುಣನಿಗೆ ಒಂದು ಸರಿಯಾದ ಕೆಲಸ ಸಿಗದೆಹೋಗಬೇಕಾದರೆ ಅದನ್ನು ಭಗವಂತನ ಲೀಲೆಯೆನ್ನೆದೆ ಇನ್ನೇನು ತಾನೆ ಹೇಳೋಣ!

ಹೀಗಿರುವಾಗ, ಒಂದು ದಿನ ನರೇಂದ್ರನಿಗೆ ಇದ್ದಕ್ಕಿದ್ದಂತೆ ಒಂದು ಆಲೋಚನೆ ಹೊಳೆಯಿತು. ‘ಹೇಗಿದ್ದರೂ ಭಗವಂತ ಶ್ರೀರಾಮಕೃಷ್ಣರ ಪ್ರಾರ್ಥನೆಯನ್ನು ಕೇಳುತ್ತಾನಲ್ಲ, ತಾನೇಕೆ ಅವರನ್ನೇ ಕೇಳಿಕೊಳ್ಳಬಾರದು...? ತನಗೆ ಒದಗಿರುವ ಆರ್ಥಿಕ ಬವಣೆಯನ್ನು ಹೋಗಲಾಡಿಸುವಂತೆ ಒಂದು ಮಾತು ಏಕೆ ಕೇಳಿನೋಡಬಾರದು?’ ಶ್ರೀರಾಮಕೃಷ್ಣರು ತನ್ನ ಯಾವ ಕೋರಿಕೆಯನ್ನಾ ದರೂ ಈಡೇರಿಸಲು ಸಿದ್ಧ ಎಂಬ ಭರವಸೆ ಅವನಿಗಿತ್ತು. ಹೀಗಿರುವಾಗ ಈ ಸಣ್ಣ ಕೋರಿಕೆಯನ್ನು ಈಡೇರಿಸಲಾರರೆ? ಸರಿ, ತಡಮಾಡದೆ ದಕ್ಷಿಣೇಶ್ವರಕ್ಕೆ ಓಡಿದ. ಅವರನ್ನು ಕೇಳಿಕೊಂಡ:

“ಮಹಾಶಯರೆ, ನನ್ನ ತಾಯಿ, ಸೋದರ-ಸೋದರಿಯರೆಲ್ಲ ಸರಿಯಾದ ಊಟ-ಬಟ್ಟೆಗೂ ಗತಿ ಯಿಲ್ಲದೆ ನರಳುತ್ತಿದ್ದಾರೆ. ನಾನು ಶಕ್ತಿಮೀರಿ ಪ್ರಯತ್ನಿಸಿದರೂ ನನಗೊಂದು ಸರಿಯಾದ ಉದ್ಯೋಗ ಸಿಕ್ಕಿಲ್ಲ. ಹೇಗಿದ್ದರೂ ನೀವು ಜಗನ್ಮಾತೆಯನ್ನು ಚೆನ್ನಾಗಿ ಬಲ್ಲವರು, ಅವಳ ಜೊತೆ ಯಲ್ಲಿ ಮಾತನಾಡುವವರು. ಈಗ ನನ್ನ ಸಂಸಾರದ ಕಷ್ಟಗಳನ್ನೆಲ್ಲ ಹೋಗಲಾಡಿಸುವಂತೆ ನಿಮ್ಮ ಕಾಳಿಯನ್ನು ನನ್ನ ಪರವಾಗಿ ದಯವಿಟ್ಟು ಕೇಳಿಕೊಳ್ಳಬೇಕು.”

ಶ್ರೀರಾಮಕೃಷ್ಣರು: “ಅಯ್ಯೋ! ನಾನು ಆ ತರಹದ ಪ್ರಾರ್ಥನೆಯನ್ನೆಲ್ಲ ಸಲ್ಲಿಸಲಾರೆನಪ್ಪ! ಆದರೆ ನೀನೇ ಯಾಕೆ ಕಾಳಿಯ ಹತ್ತಿರ ಹೋಗಿ ಕೇಳಿಕೊಳ್ಳಬಾರದು? ನೀನು ಅವಳನ್ನು ಒಪ್ಪಿ ಕೊಳ್ಳದಿರುವುದೇ ನಿನ್ನ ಈ ಎಲ್ಲ ಕಷ್ಟಗಳಿಗೂ ಕಾರಣ.”

ನರೇಂದ್ರ: “ಆದರೆ ನನಗೆ ಆಕೆಯ ಪರಿಚಯವೇ ಇಲ್ಲ. ಆದ್ದರಿಂದ ದಯವಿಟ್ಟು ನನ್ನ ಪರವಾಗಿ ನೀವೇ ಅವಳಿಗೆ ಹೇಳಿ. ನೀವು ಹೇಳಲೇಬೇಕು.”

ಶ್ರೀರಾಮಕೃಷ್ಣರು (ವಾತ್ಸಲ್ಯ ತುಂಬಿದ ದನಿಯಲ್ಲಿ): “ನರೇನ್! ನಿಜ ಹೇಳಬೇಕೆಂದರೆ, ನಾನು ನಿನಗಾಗಿ ಬಹಳ ಸಲ ಪ್ರಾರ್ಥನೆ ಸಲ್ಲಿಸಿದ್ದೇನಪ್ಪ. ಆದರೆ ನೀನೇ ಜಗನ್ಮಾತೆಯನ್ನು ಒಪ್ಪಿಕೊಳ್ಳುವುದಿಲ್ಲವಲ್ಲ, ಆದ್ದರಿಂದ ಅವಳು ನನ್ನ ಆ ಪ್ರಾರ್ಥನೆಯನ್ನು ಈಡೇರಿಸಿಕೊಡಲು ಒಪ್ಪುತ್ತಿಲ್ಲ... ಸರಿ, ಆಗಲಿ; ಇವೊತ್ತು ಮಂಗಳವಾರ, ಒಳ್ಳೆಯ ದಿನ. ಇವೊತ್ತು ರಾತ್ರಿ ನೀನು ಕಾಳೀ ದೇವಾಲಯಕ್ಕೆ ಹೋಗಿ ಅವಳಿಗೆ ಸಾಷ್ಟಾಂಗ ಪ್ರಣಾಮ ಮಾಡು. ಬಳಿಕ ನಿನಗೆ ಬೇಕಾದ ವರವನ್ನು ಕೇಳಿಕೊ. ಅವಳು ಅನುಗ್ರಹಿಸುತ್ತಾಳೆ. ಅವಳು ಸರ್ವಜ್ಞೆ, ಜ್ಞಾನಸ್ವರೂಪಿಣಿ, ಪರಬ್ರಹ್ಮನ ಆದ್ಯಾಶಕ್ತಿ. ಅವಳು ಇಚ್ಛಾಮಾತ್ರದಿಂದ ಸಮಸ್ತ ಬ್ರಹ್ಮಾಂಡವನ್ನೇ ಸೃಷ್ಟಿಸಿದವಳು! ಆಕೆ ಮನಸ್ಸು ಮಾಡಿದರೆ ಆಗದಿರುವುದೇನಿದೆ?”

ಅತ್ಯಾಶ್ಚರ್ಯದ ಸಂಗತಿಯೆಂದರೆ ನರೇಂದ್ರನಿಗೆ ಅವರಾಡಿದ ಪ್ರತಿಯೊಂದು ಮಾತೂ ಅಕ್ಷರಶಃ ನಂಬಿಕೆಯಾಯಿತು. ಏಕೆಂದರೆ ಶ್ರೀರಾಮಕೃಷ್ಣರು ಸದಾ ಜಗನ್ಮಾತೆಯೊಂದಿಗೆ ವಾಸ ವಾಗಿರುವವರಲ್ಲವೆ? ಅಲ್ಲದೆ, ಅವನನ್ನೀಗ ತಾಳಲಾರದ ಕಷ್ಟಗಳು ಬಂದು ಮುತ್ತಿಕೊಂಡಿವೆ ಯಲ್ಲ! ಸರಿ, ರಾತ್ರಿಯಾಗುವುದನ್ನೇ ಕಾಯುತ್ತ ಕುಳಿತ. ಕೊನೆಗೆ, ಸುಮಾರು ಒಂಬತ್ತು ಗಂಟೆಯ ಹೊತ್ತಿಗೆ ಶ್ರೀರಾಮಕೃಷ್ಣರ ಅನುಜ್ಞೆ ಪಡೆದು ಕಾಳೀ ದೇವಾಲಯಕ್ಕೆ ಹೊರಟ. ಹೋಗುವಾಗಲೇ ಅವನನ್ನೊಂದು ದೈವೀ ಭಾವೋನ್ಮತ್ತತೆ ಆವರಿಸಿಕೊಂಡುಬಿಟ್ಟಿತ್ತು. ಸರಿಯಾಗಿ ಹೆಜ್ಜೆಹಾಕಲೂ ಸಾಧ್ಯವಾಗುತ್ತಿಲ್ಲ ಅವನಿಗೆ! ‘ನಾನೀಗ ಜೀವಂತ ದೇವಿಯ ಬಳಿಗೆ ಹೋಗುತ್ತಿದ್ದೇನೆ, ಅವಳ ಮಾತುಗಳನ್ನು ಕೇಳಲಿದ್ದೇನೆ’ ಎಂಬ ಭಾವವೇ ಅವನ ನರನಾಡಿಗಳಲ್ಲೆಲ್ಲ ತುಡಿಯುತ್ತಿತ್ತು. ತೂರಾಡುತ್ತಲೇ ದೇವಸ್ಥಾನವನ್ನು ಪ್ರವೇಶಿಸಿದ. ನೋಡುತ್ತಾನೆ, ಗರ್ಭಗುಡಿಯಲ್ಲಿ ಭವತಾರಿಣಿ ಸಾಕ್ಷಾತ್ ಚಿನ್ಮಯಿಯಾಗಿ ನಿಂತಿದ್ದಾಳೆ! ಚೈತನ್ಯಮಯಿಯಾಗಿ ನಿಂತಿದ್ದಾಳೆ! ದಿವ್ಯ ಸೌಂದರ್ಯ ದಿಂದ ಶೋಭಾಯಮಾನಳಾಗಿದ್ದಾಳೆ! ಭಕ್ತರ ಅಭೀಷ್ಟಗಳನ್ನು ಈಡೇರಿಸಲು ಸಿದ್ಧಳಾಗಿ, ದಿವ್ಯ ವಾತ್ಸಲ್ಯವನ್ನು ಹರಿಯಿಸುತ್ತಿದ್ದಾಳೆ. ಇಂಥ ಜಗನ್ಮಾತೆಯನ್ನು ಕಂಡು ನರೇಂದ್ರ ಭಾವಾವೇಶ ಭರಿತನಾಗಿ, ಆನಂದೋನ್ಮತ್ತನಾಗಿ ದೀರ್ಘದಂಡಪ್ರಣಾಮ ಮಾಡಿದ. ಬಳಿಕ ಕೈಜೋಡಿಸಿ ಕೊಂಡು, “ಅಮ್ಮಾ, ಜಗನ್ಮಾತೆ! ವಿವೇಕ ಕೊಡು, ವೈರಾಗ್ಯ ಕೊಡು, ಜ್ಞಾನ ಕೊಡು, ಭಕ್ತಿ ಕೊಡು! ಅಮ್ಮಾ, ನನಗೆ ನಿರಂತರವೂ ನಿನ್ನ ದರ್ಶನವಾಗುವಂತೆ ಅನುಗ್ರಹಿಸು!” ಎಂದು ಪ್ರಾರ್ಥಿಸಿದ.

ಈಗ ಅವನ ಇಡೀ ವ್ಯಕ್ತಿತ್ವದಲ್ಲಿ ಒಂದು ಅಪೂರ್ವವಾದ ದಿವ್ಯ ಶಾಂತಿ ನೆಲೆಸಿದೆ. ಅವನ ಹೃದಯದಲ್ಲಿ ಜಗನ್ಮಾತೆ ಪ್ರಕಾಶಿಸುತ್ತಿದ್ದಾಳೆ.

ಆನಂದಭರಿತನಾಗಿ ತಮ್ಮ ಬಳಿಗೆ ಬಂದ ನರೇಂದ್ರನನ್ನು ಶ್ರೀರಾಮಕೃಷ್ಣರು ಕೇಳಿದರು: “ಏನು, ಹೋದ ಕೆಲಸವಾಯಿತೆ? ನಿನ್ನ ಆರ್ಥಿಕ ಮುಗ್ಗಟ್ಟನ್ನು ಹೋಗಲಾಡಿಸುವಂತೆ ಜಗನ್ಮಾತೆ ಯನ್ನು ಕೇಳಿಕೊಂಡೆಯಾ?”

ನರೇಂದ್ರ ತಬ್ಬಿಬ್ಬಾಗಿ ನುಡಿದ: “ಇಲ್ಲ ಮಹಾಶಯರೆ, ಎಲ್ಲ ಮರೆತೇಬಿಟ್ಟೆ. ಈಗ ನಾನೇನು ಮಾಡಲಿ?”

ಶ್ರೀರಾಮಕೃಷ್ಣರು: “ಇನ್ನೇನು ಮಾಡುವುದು? ಮತ್ತೆ ಹೋಗು, ನಿನ್ನ ಆವಶ್ಯಕತೆಗಳನ್ನೆಲ್ಲ ಕೇಳಿಕೊ.”

ನರೇಂದ್ರ ಮತ್ತೊಮ್ಮೆ ಕಾಳಿಯನ್ನು ನೋಡಲು ಹೊರಟ. ಆದರೆ ಪುನಃ ಅದೇ ಭಾವಾವೇಶ! ಬೇರೆಲ್ಲವನ್ನೂ ಮರೆತೇಬಿಟ್ಟ. ಸಾಷ್ಟಾಂಗ ಪ್ರಣಾಮ ಮಾಡಿ ಜ್ಞಾನ-ಭಕ್ತಿಗಳನ್ನು ಕರುಣಿಸುವಂತೆ ಹೃತ್ಪೂರ್ವಕವಾಗಿ ಬೇಡಿಕೊಂಡು ಹಿಂದಿರುಗಿದ. ಶ್ರೀರಾಮಕೃಷ್ಣರು ಕೇಳಿದರು:

“ಏನು, ಈ ಸಲವಾದರೂ ಸರಿಯಾಗಿ ಕೇಳಿಕೊಂಡೆಯಾ? ಅಥವಾ ಮತ್ತೆ ಮರೆತೆಯೋ?”

ನರೇಂದ್ರ (ನಾಚಿಕೆಯಿಂದ): “ಹೌದು, ಈ ಸಲವೂ ಮರೆತೇಬಿಟ್ಟೆ. ಅಲ್ಲಿ ಹೋಗುವಷ್ಟರಲ್ಲೇ ಯಾವುದೋ ಅಲೌಕಿಕ ಶಕ್ತಿಯ ಪ್ರಭಾವಕ್ಕೆ ಸಿಕ್ಕಿಕೊಂಡು ಮನೆಯ ವಿಷಯವನ್ನೆಲ್ಲ ಮರೆತೆ. ಕೇವಲ ಜ್ಞಾನ-ಭಕ್ತಿಗಳಿಗಾಗಿ ಪ್ರಾರ್ಥಿಸಿಕೊಂಡೆ. ಈಗ ಉಪಾಯವೇನು? ನೀವೇ ಹೇಳಿ.”

ಶ್ರೀರಾಮಕೃಷ್ಣರು: “ಛೆ, ಎಂಥಾ ದಡ್ಡ ನೀನು! ಇದೊಂದೆರಡು ಮಾತುಗಳನ್ನು ನೆನಪಿಟ್ಟು ಕೊಂಡು ಹೋಗಿ ಕೇಳುವುದಕ್ಕಾಗಲಿಲ್ಲವೆ ನಿನ್ನಿಂದ? ಸರಿ, ಪುನಃ ಹೋಗು. ಈ ಸಲವಾದರೂ ನಿನಗೆ ಬೇಕಾದ್ದನ್ನು ಬೇಡಿಕೊ. ಮರೆತೀಯೆ ಮತ್ತೆ!”

ಈ ಸಲ ಅವನು ಮನಸ್ಸು ಗಟ್ಟಿ ಮಾಡಿಕೊಂಡು, ದೇವಿಯೆದುರು ತನ್ನ ಬಯಕೆಯನ್ನಿಡುವ ದೃಢ ನಿರ್ಧಾರದಿಂದ ಹೊರಟ. ತಾನು ಬಂದ ಉದ್ದೇಶವನ್ನು ಮರೆಯಲಿಲ್ಲ. ಆದರೆ ದೇವಿಯ ಮುಂದೆ ನಿಂತಾಗ ಮಾತ್ರ ಅವನಿಗೆ ಇನ್ನಿಲ್ಲದಷ್ಟು ನಾಚಿಕೆಯಾಯಿತು. ‘ಆಹಾ, ಸಾಕ್ಷಾತ್ ಜಗನ್ಮಾತೆಯ ಬಳಿಗೆ ಬಂದು ಎಂಥಾ ಕ್ಷುಲ್ಲಕವಾದದ್ದನ್ನು ಕೇಳಲು ಹೊರಟಿದ್ದೇನೆ, ಚಕ್ರವರ್ತಿಯ ಬಳಿಗೆ ಹೋಗಿ ನಾಲ್ಕು ಬದನೇಕಾಯಿ ಕೇಳುವಂತೆ! ಛಿ, ನಾನೆಂಥ ಮೂರ್ಖ!’ ಎಂದು ಆಲೋಚಿಸಿ, ತುಂಬ ದುಃಖದಿಂದ ಜಗಜ್ಜನನಿಯ ಮುಂದೆ ಶಿರಬಾಗಿ ನಮಸ್ಕರಿಸುತ್ತ, “ಅಮ್ಮಾ, ಜಗನ್ಮಾತೆ! ನನಗೆ ಜ್ಞಾನ ಭಕ್ತಿ ವೈರಾಗ್ಯಗಳನ್ನೇ ಕರುಣಿಸು. ನನಗಿನ್ನೇನೂ ಬೇಡ” ಎಂದು ಮೊರೆಯಿಟ್ಟ.

ಹೀಗೆ ಪ್ರಾರ್ಥಿಸಿಕೊಂಡು ಹೊರಗೆ ಬರುತ್ತಿದ್ದಂತೆಯೇ ಅವನು ವಿಸ್ಮಯಗೊಂಡು ಆಲೋಚಿ ಸಿದ: ‘ನನಗೇನಾಗುತ್ತಿದೆ ಇವತ್ತು! ನನ್ನ ಮನಸ್ಸು ನನ್ನ ಅಧೀನದಲ್ಲೇ ಇದ್ದಂತಿಲ್ಲವಲ್ಲಾ?....ಹೌದು, ಇದೆಲ್ಲ ಶ್ರೀರಾಮಕೃಷ್ಣರದೇ ಕೈವಾಡ. ಇಲ್ಲದಿದ್ದರೆ ಹೀಗೆಲ್ಲ ಆಗಲು ಸಾಧ್ಯವಿಲ್ಲ. ಮೂರು ಸಲ ಬಂದು ಮೂರು ಸಲವೂ ನನ್ನ ಮನಸ್ಸಿನ ಕೋರಿಕೆಯನ್ನು ಸಲ್ಲಿಸಲು ಆಗಲಿಲ್ಲವೆಂದರೆ!’

ನೇರವಾಗಿ ಶ್ರೀರಾಮಕೃಷ್ಣರ ಬಳಿಗೆ ಬಂದು ಹೇಳಿದ: “ ಮಹಾಶಯರೆ, ನನ್ನ ಮೇಲೆ ಮೋಡಿ ಹಾಕಿ ನಾನು ಎಲ್ಲವನ್ನೂ ಮರೆಯುವಂತೆ ಮಾಡುತ್ತಿರುವುದು ನೀವೇ ಎಂದು ನನಗೆ ಅರ್ಥ ವಾಗಿದೆ. ಈಗ ನೀವೇ ನನ್ನ ಮೇಲೆ ಅನುಗ್ರಹ ಮಾಡಬೇಕು. ನನ್ನ ಮನೆಮಂದಿಯ ಅನ್ನ-ಬಟ್ಟೆಗೆ ಕೊರತೆಯಿಲ್ಲದಂತೆ ಮಾಡುವುದು ಈಗ ನಿಮ್ಮದೇ ಜವಾಬ್ದಾರಿ.”

ಶ್ರೀರಾಮಕೃಷ್ಣರು: “ಮಗೂ ನರೇನ್, ನಾನಾಗಲೇ ಹೇಳಲಿಲ್ಲವೆ? ನನ್ನ ಬಾಯಿಂದ ಅಂಥ ಪ್ರಾರ್ಥನೆ ಬಾರದಪ್ಪ. ನೀನು ಏನು ಕೇಳಿಕೊಂಡರೆ ಅದನ್ನು ಆಕೆ ದಯಪಾಲಿಸುತ್ತಾಳೆ ಅಂತ ಕೂಡ ನಾನು ಹೇಳಿದೆ. ಆದರೆ ನೀನೇ ನಿನಗೆ ಬೇಕಾದ್ದನ್ನು ಕೇಳಿಕೊಳ್ಳದಿದ್ದರೆ ನಾನು ತಾನೆ ಏನು ಮಾಡಲಿ? ಐಹಿಕ ಸುಖವನ್ನು ಅನುಭವಿಸುವುದು ನಿನ್ನ ಹಣೆಯಲ್ಲಿ ಬರೆದಿಲ್ಲ, ಅಷ್ಟೇ!”

ಆದರೆ ನರೇಂದ್ರ ಅಷ್ಟಕ್ಕೆ ಬಿಡದೆ ಮತ್ತೆಮತ್ತೆ ಗೋಗರೆದ: “ಅದೆಲ್ಲ ಆಗುವುದಿಲ್ಲ. ನನಗೋ ಸ್ಕರ ನೀವು ಕೇಳಿಕೊಳ್ಳಲೇಬೇಕು. ನೀವು ಒಂದು ಮಾತು ಹೇಳಿದರೆ ಸಾಕು, ನನ್ನ ಮನೆಯವರ ಕಷ್ಟಗಳೆಲ್ಲ ದೂರವಾಗುತ್ತವೆ, ನನಗೆ ಚೆನ್ನಾಗಿ ಗೊತ್ತು.”

ಕಟ್ಟಕಡೆಗೆ ಶ್ರೀರಾಮಕೃಷ್ಣರಿಂದ ಅಭಯ ದೊರೆಯಿತು: “ಸರಿ, ನಡೆ; ಅವರಿನ್ನು ಕನಿಷ್ಠ ಅನ್ನ-ಬಟ್ಟೆಗಳಿಗಾಗಿ ಪರದಾಡಬೇಕಾಗುವುದಿಲ್ಲ.”

ಈ ಮಾತು ಸತ್ಯವಾಗಿ ಪರಿಣಮಿಸಿತು. ಭಗವಂತನ ಸ್ವರೂಪವನ್ನು ಪರಿಪೂರ್ಣವಾಗಿ ಅರಿಯುವಲ್ಲಿ ನರೇಂದ್ರನಿಗೆ ಈ ಅನುಭವ ಬಹಳವಾಗಿ ಸಹಾಯ ಮಾಡಿತು. ‘ಭಗವಂತ ಸಾಕಾರನೂ ಹೌದು, ನಿರಾಕಾರನೂ ಹೌದು; ಅವನು ಸಗುಣನೂ ಹೌದು, ನಿರ್ಗುಣನೂ ಹೌದು; ಅವನನ್ನು ತಂದೆಯೆನ್ನಬಹುದು, ತಾಯಿ ಎನ್ನಲೂ ಬಹುದು’ ಎಂಬ ಸತ್ಯ ಅವನಿಗೆ ಇದೀಗ ಅರ್ಥವಾಗತೊಡಗಿದೆ. ಇಲ್ಲಿಯವರೆಗೆ ಅವನು ಭಗವಂತನ ಸಾಕಾರ ಸ್ವರೂಪವನ್ನಾಗಲಿ ಮಾತೃತ್ವ ವನ್ನಾಗಲಿ ಒಪ್ಪಿಕೊಳ್ಳುತ್ತಿರಲಿಲ್ಲ. ಭಗವಂತನನ್ನು ವಿಗ್ರಹರೂಪದಲ್ಲಿ ಅರ್ಚಿಸಿ ಆರಾಧಿಸುವುದ ರಿಂದ ಆಧ್ಯಾತ್ಮಿಕ ಸಾಧನೆಯಲ್ಲಿ ವಿಶೇಷ ಅನುಕೂಲತೆಯಿದೆ ಎಂಬ ಸತ್ಯವನ್ನು ಒಪ್ಪಿಕೊಳ್ಳು ತ್ತಿರಲಿಲ್ಲ. ಈಗ ಈ ಹೊಸ ಅನುಭವವಾದನಂತರ ಭಗವಂತನ ಸಾಕಾರಸ್ವರೂಪವನ್ನು ಒಪ್ಪಿ ಕೊಂಡು ಸಾಕಾರಮೂರ್ತಿಯಾದ ಕಾಳಿಯ ಮುಂದೆ ಉದ್ದಂಡ ಪ್ರಣಾಮ ಮಾಡುತ್ತಾನೆ. ಭಗವಂತನ ಮಾತೃತ್ವವನ್ನು ಒಪ್ಪಿಕೊಂಡು ಅದೇ ಜಗನ್ಮಾತೆಯನ್ನು ಪ್ರಾರ್ಥಿಸಿಕೊಳ್ಳುತ್ತಾನೆ. ನಿಜಕ್ಕೂ ಅವನಲ್ಲಿ ಇದೊಂದು ಅದ್ಭುತ ಪರಿವರ್ತನೆ; ಆತನ ಶಿಷ್ಯವೃತ್ತಿಯಲ್ಲಿ ಮಹತ್ವದ ಘಟ್ಟ. ಏಕೆಂದರೆ ಅವನು ಯಾವುದೇ ವಿಚಾರವನ್ನೂ ಸುಮ್ಮನೆ ಒಪ್ಪುವವನೇ ಅಲ್ಲ. ಆದ್ದ ರಿಂದಲೇ ಇಷ್ಟು ದಿನವೂ ಸಾಕಾರ ಭಗವಂತನ ಪೂಜೆಯನ್ನು ‘ಮೌಢ್ಯ’ ಎಂದು ಹೀಯಾಳಿಸು ತ್ತಿದ್ದ. ಇನ್ನು ಮುಂದೆ ಅವನು ಹಾಗೆಂದೂ ಮಾಡಲಾರ. ಏಕೆಂದರೆ ಅರ್ಧಂಬರ್ಧ ನಂಬಿಕೆ ಯೆಂಬುದು ಅವನ ಜಾಯಮಾನದಲ್ಲೇ ಇಲ್ಲ. ಯಾವುದನ್ನೇ ಆಗಲಿ ಒಮ್ಮೆ ಒಪ್ಪಿ ಸ್ವೀಕರಿಸಿದ ನೆಂದರೆ ಅದನ್ನು ಮತ್ತೆ ಬಿಡಲಾರ. ಮಾತ್ರವಲ್ಲ. ಅತಿ ಶೀಘ್ರದಲ್ಲೇ ಅವನೊಳಗಿನ ಭಕ್ತಿ ಮೊಳೆತು ಬೃಹತ್ತಾಗಿ ನಿಲ್ಲಲಿದೆ. ಹೀಗೆ ಅವನ ಆಧ್ಯಾತ್ಮಿಕ ಜೀವನವು ಪರಿಪೂರ್ಣತೆಯ ದಾರಿಯಲ್ಲಿ ಮುನ್ನಡೆಯುವಂತಾಗಿದೆ. ಶ್ರೀರಾಮಕೃಷ್ಣರಿಗಂತೂ ಇದು ಬಹಳ ಸಂತೋಷದ ವಿಷಯ. ಮುಂದೆ ಸ್ವಾಮಿ ವಿವೇಕಾನಂದರಾಗಿ ಜಗತ್ತಿಗೆ ಆಧ್ಯಾತ್ಮಿಕ ಜೀವನದ ಸಮಗ್ರ ಪರಿಚಯ ಮಾಡಿಸಿಕೊಡಬೇಕಾದ ನರೇಂದ್ರ, ಇಂದು ಮೂರ್ತಿಪೂಜೆಯನ್ನೇ ಒಪ್ಪಿಕೊಳ್ಳದಿದ್ದರೆ ಹೇಗೆ? ಭಗವಂತನ ಮಾತೃತ್ವವನ್ನು ಒಪ್ಪಿಕೊಳ್ಳದಿದ್ದರೆ ಹೇಗೆ? ಆಧ್ಯಾತ್ಮಿಕ ಜೀವನವೆಂಬುವುದು ಪರಿ ಪೂರ್ಣವಾಗಬೇಕಾದರೆ ಅದರ ಸಮಸ್ತ ಅಂಶಗಳನ್ನೂ, ಅವುಗಳ ಪ್ರಾಶಸ್ತ್ಯ ಹಾಗೂ ಉಪ ಯುಕ್ತತೆಗಳನ್ನೂ ಅರಿತಿರಬೇಕಾಗುತ್ತದೆ. ಇಲ್ಲದಿದ್ದರೆ ಅಂಥವನ ಜ್ಞಾನವೆಂಬುದು ಆನೆಯನ್ನು ಮುಟ್ಟಿನೋಡಿದ ಕುರುಡರ ಕಥೆಯಂತಾಗಿಬಿಡುತ್ತದೆ. ಆದರೆ ಶ್ರೀರಾಮಕೃಷ್ಣರು ಅವತರಿಸಿರು ವುದು ಸರ್ವಧರ್ಮ ಸಮನ್ವಯವನ್ನು ಸ್ಥಾಪಿಸುವುದಕ್ಕಾಗಿ; ಸರ್ವಭಾವಗಳನ್ನು ಪುರಸ್ಕರಿಸುವುದ ಕ್ಕಾಗಿ; ಸರ್ವಸಾಧನೆಗಳ ಮರ್ಮವನ್ನು ಮನಮುಟ್ಟಿಸುವುದಕ್ಕಾಗಿ. ಹೀಗಿರುವಾಗ, ಅವರ ಪಟ್ಟ ಶಿಷ್ಯನಾದ ನರೇಂದ್ರ, ಶಾಸ್ತ್ರಸಮ್ಮತವಾದ ಮೂರ್ತಿಪೂಜೆಯನ್ನು ನಿರಾಕರಿಸಿಬಿಟ್ಟರೆ ಹೇಗೆ? ಶ್ರೇಷ್ಠಭಕ್ತರ ಅನುಭವಸಿದ್ಧ ವಿಷಯವಾದ ಭಗವಂತನ ಜಗನ್ಮಾತೃತ್ವದ ಅಂಶವನ್ನೇ ಒಪ್ಪಿ ಕೊಳ್ಳದಿದ್ದರೆ ಹೇಗೆ? ಆದರೀಗ ಅವನಲ್ಲಿ ಪರಿವರ್ತನೆಯುಂಟಾಗಿದೆ. ಆಧ್ಯಾತ್ಮಿಕ ಸಾಧನೆಯ ಮಾರ್ಗದಲ್ಲಿ ಅವನು ದೊಡ್ಡ ಹೆಜ್ಜೆಯೊಂದನ್ನಿಟ್ಟಿದ್ದಾನೆ.

ಮರುದಿನ ಮಧ್ಯಾಹ್ನ; ಶ್ರೀರಾಮಕೃಷ್ಣರು ತಮ್ಮ ಕೋಣೆಯಲ್ಲಿ ಒಬ್ಬರೇ ಕುಳಿತಿದ್ದಾರೆ. ನರೇಂದ್ರ ಹೊರಗೆ ಜಗಲಿಯ ಮೇಲೆ ನಿದ್ರಿಸುತ್ತಿದ್ದಾನೆ. ಆಗ ವೈಕುಂಠನಾಥ ಸನ್ಯಾಲ ಎಂಬೊಬ್ಬ ತರುಣಭಕ್ತ ಅಲ್ಲಿಗೆ ಬಂದ. ‘ಇದೇನು ಇವನು ಈ ಹೊತ್ತಿನಲ್ಲಿ ಮಲಗಿಕೊಂಡಿದ್ದಾನಲ್ಲ!’ ಎಂದು ಅವನಿಗೆ ಆಶ್ಚರ್ಯವಾಗಿರಬೇಕು. ಆತ ತಮಗೆ ನಮಸ್ಕಾರ ಮಾಡಿದೊಡನೆಯೇ ಶ್ರೀರಾಮ ಕೃಷ್ಣರು ಉತ್ಸಾಹದಿಂದ ಹೇಳತೊಡಗಿದರು: “ನೋಡು, ಈ ಹುಡುಗ ಇದ್ದಾನಲ್ಲ, ಇವನು ಬಹಳ ಒಳ್ಳೆಯ ಹುಡುಗ. ನರೇಂದ್ರ ಅಂತ ಇವನ ಹೆಸರು. ಇವನು ಇಲ್ಲಿಯವರೆಗೂ ಜಗನ್ಮಾತೆ ಯನ್ನು ಒಪ್ಪಿಕೊಳ್ಳುತ್ತಿರಲಿಲ್ಲ, ಆದರೆ ನಿನ್ನೆ ಒಪ್ಪಿಕೊಂಡುಬಿಟ್ಟ! ಈಚೆಗೆ ಅವನು ವಿಪರೀತ ಕಷ್ಟದಲ್ಲಿದ್ದಾನೆ. ಆದ್ದರಿಂದ ಅವನಿಗೆ ನಾನು, ಹಣಕ್ಕಾಗಿ ಜಗನ್ಮಾತೆಯನ್ನು ಸ್ವಲ್ಪ ಪ್ರಾರ್ಥಿಸಿಕೊ ಎಂದು ಹೇಳಿದೆ. ಆದರೆ ಅವನಿಂದ ಅದು ಸಾಧ್ಯವಾಗಲೇ ಇಲ್ಲ. ದೇವಿಯನ್ನು ಹಾಗೆ ಕೇಳುವುದಕ್ಕೆ ಅವನಿಗೆ ನಾಚಿಕೆಯಾಯಿತಂತೆ. ದೇವಸ್ಥಾನಕ್ಕೆ ಹೋಗಿಬಂದವನೇ ದೇವಿಯ ಮೇಲೆ ತನಗೊಂದು ಹಾಡು ಹೇಳಿಕೊಡುವಂತೆ ನನ್ನನ್ನು ಕೇಳಿಕೊಂಡ. ನಾನು ಹೇಳಿಕೊಟ್ಟೆ. ನಿನ್ನೆ ರಾತ್ರಿಯಿಡೀ ಅದನ್ನೇ ಹಾಡುತ್ತಾ ಇದ್ದುಬಿಟ್ಟ. ಆದ್ದರಿಂದ ಅವನೀಗ ಮಲಗಿಕೊಂಡಿದ್ದಾನೆ.”

ಹೀಗೆಂದ ಶ್ರೀರಾಮಕೃಷ್ಣರು ತುಂಬ ಮುಗ್ಧ ಸಂತೋಷದಿಂದ ಮತ್ತೆ ಹೀಗೆಂದರು: “ಇವನು ಜಗನ್ಮಾತೆಯನ್ನು ಒಪ್ಪಿಕೊಂಡಿದ್ದು ದೊಡ್ಡ ಅದ್ಭುತವಲ್ಲವೆ?”

ವೈಕುಂಠ: “ಹೌದು, ನಿಜಕ್ಕೂ ಅದ್ಭುತವೇ.”

ಸ್ವಲ್ಪ ಹೊತ್ತು ಸುಮ್ಮನಿದ್ದ ಶ್ರೀರಾಮಕೃಷ್ಣರು ಪುನಃ ನುಡಿದರು: “ಆಶ್ಚರ್ಯ! ಅಲ್ಲವೆ? ಏನೆನ್ನುತ್ತೀಯಾ?”

ಹೀಗೆ ಕೆಲಹೊತ್ತು ಶ್ರೀರಾಮಕೃಷ್ಣರ ಬಾಯಲ್ಲಿ ಮತ್ತೆಮತ್ತೆ ಅದೇ ಮಾತು ಅದೇ ಉದ್ಗಾರ ಬರುತ್ತಿತ್ತು. ನರೇಂದ್ರ ಜಗನ್ಮಾತೆಯನ್ನು ಒಪ್ಪಿಕೊಳ್ಳುವುದು ಅವರ ದೃಷ್ಟಿಯಲ್ಲಿ ಎಷ್ಟು ಪ್ರಾಮುಖ್ಯವಾಗಿತ್ತು ಎಂದು ಇದರಿಂದ ಊಹಿಸಬಹುದು.

ಅಪರಾಹ್ನ ನಾಲ್ಕು ಗಂಟೆಯ ಹೊತ್ತಿಗೆ ನರೇಂದ್ರ ಮನೆಗೆ ಹೊರಟು ಶ್ರೀರಾಮಕೃಷ್ಣರಿಂದ ಬೀಳ್ಗೊಳ್ಳಲು ಬಂದ. ಅವನನ್ನು ಕಂಡೊಡನೆ ಅವರು ಉತ್ಸುಕರಾಗಿ ಅವನ ಬಳಿಸಾರಿ, ಹೆಚ್ಚುಕಡಿಮೆ ಅವನಿಗೆ ತಗಲಿಕೊಂಡಂತೆಯೇ ಕುಳಿತರು. ಬಳಿಯಿದ್ದ ವೈಕುಂಠನಾಥನಿಗೆ ತಮ್ಮಿಬ್ಬ ರನ್ನೂ ತೋರಿಸಿಕೊಳ್ಳುತ್ತ ಹೇಳಿದರು: “ನೋಡು! ನಾನೇ ಇವನು ಮತ್ತು ಇವನೇ ನಾನು ಎನ್ನುವುದು ನನಗೆ ಸ್ಪಷ್ಟವಾಗಿ ಕಾಣುತ್ತಿದೆ. ನಮ್ಮಿಬ್ಬರಲ್ಲಿ ಯಾವ ಭೇದವೂ ಇಲ್ಲ. ಗಂಗೆಯ ನೀರಿನ ಮೇಲೆ ತೇಲುತ್ತಿರುವ ಒಂದು ಕಡ್ಡಿ ನೀರನ್ನು ಎರಡಾಗಿ ವಿಭಜಿಸುವಂತೆ ಕಂಡರೂ ನಿಜಕ್ಕೂ ಅದು ಒಂದೆ ತಾನೆ! ಹಾಗೇ ನಾವಿಬ್ಬರೂ ಕೂಡ. ನಾನು ಹೇಳಿದ್ದು ನಿನಗೆ ಅರ್ಥ ವಾಯಿತೆ? ಏನು ಹೇಳುತ್ತಿ?”

ಆದರೆ ವೈಕುಂಠನಿಗೆ ಈ ಮಾತುಗಳೆಲ್ಲ ಹೇಗೆ ಅರ್ಥವಾಗಬೇಕು ಪಾಪ! ಅವರಿಬ್ಬರೂ ಬೇರೆಬೇರೆಯಾಗಿ ಕುಳಿತಿರುವುದು ಕಣ್ಣೆದುರಿಗೇ ಸ್ಪಷ್ಟವಾಗಿ ಕಾಣುತ್ತಿದೆ. ಅಂಥದರಲ್ಲಿ ಶ್ರೀರಾಮ ಕೃಷ್ಣರು ತಾವಿಬ್ಬರೂ ಒಂದೇ, ಯಾವ ಭೇದವೂ ಇಲ್ಲ ಎಂದು ಹೇಳಿಬಿಟ್ಟರೆ ಅವನು ಏನೆಂದು ಅರ್ಥಮಾಡಿಕೊಳ್ಳಬೇಕು? ಅವರಿಬ್ಬರಲ್ಲಿ ಪರಸ್ಪರ ಆತ್ಮೀಯತೆ-ಸ್ನೇಹ ಅಷ್ಟೊಂದು ನಿಕಟ ವಾಗಿದೆ ಎಂದೆ? ಅಥವಾ ಅವರಿಬ್ಬರೂ ಎರಡು ಶರೀರ, ಒಂದೇ ಆತ್ಮ ಎಂದೆ? ಅವನು ಸುಮ್ಮನೆ ಆಶ್ಚರ್ಯದಿಂದ ನೋಡುತ್ತಿದ್ದಾನೆ.

ಶ್ರೀರಾಮಕೃಷ್ಣರು, ಹೀಗೆ ಮಾತನಾಡುತ್ತ ವೈಕುಂಠನಾಥನಿಗೆ ಹುಕ್ಕ ತಯಾರಿಸಲು ಹೇಳಿ ದರು.\footnote{*ನೋಡಿ: ಅನುಬಂಧ ೨.} ಬಳಿಕ ಒಂದೆರಡು ಬಾರಿ ಅದನ್ನು ಸೇದಿ ತಮ್ಮ ಕೈಯನ್ನು ನರೇಂದ್ರನತ್ತ ಚಾಚಿ, “ಇಗೊ, ನೀನೂ ಒಂದು ಸಲ ಸೇದು” ಎನ್ನುತ್ತ ಕೊಳವೆಯನ್ನು ಹಿಡಿದಿದ್ದ ತಮ್ಮ ಮುಷ್ಟಿಯನ್ನು ಅವನ ತುಟಿಗಿಟ್ಟರು. ಸಹಜವಾಗಿಯೇ ಅವನು ತುಂಬ ಕಸಿವಿಸಿಗೊಂಡ. ಗುರುವಿನ ಕೈಯನ್ನು ತನ್ನ ತುಟಿಯಿಂದ ಮುಟ್ಟಿ ಎಂಜಲು ಮಾಡಲು ಹೇಗೆ ತಾನೆ ಸಾಧ್ಯ? ಆಗ ಶ್ರೀರಾಮಕೃಷ್ಣರು, “ಇದೇನು ಬುದ್ಧಿ ನಿನ್ನದು! ನಾನು ನಿನಗಿಂತ ಬೇರೆಯೆ? ಇದು (ತಾವು) ನಾನೇ ಮತ್ತು ಅದು (ನರೇಂದ್ರ) ಕೂಡ ನಾನೇ” ಎನ್ನುತ್ತ ಪುನಃ ತಮ್ಮ ಕೈಯನ್ನು ಅವನ ಬಾಯಿಗೆ ಹಿಡಿದರು. ಬೇರೆ ದಾರಿಗಾಣದೆ ಅವನು ಅವರ ಕೈಯಿಂದಲೇ ಸೇದಬೇಕಾಯಿತು. ಒಂದೆರಡು ಬಾರಿ ಸೇದಿದ ಮೇಲೆ ಅವರು ಅದೇ ಕೈಯಿಂದ ಮತ್ತೆ ತಾವು ಸೇದಹೊರಟರು. ತಕ್ಷಣ ನರೇಂದ್ರ ಅವರನ್ನು ತಡೆಯುತ್ತ, “ಮಹಾಶಯರೆ, ದಯವಿಟ್ಟು ನಿಮ್ಮ ಕೈಯನ್ನು ಮೊದಲು ತೊಳೆದುಕೊಳ್ಳಿ” ಎಂದ. ಆದರೆ ಅವನ ಮಾತನ್ನು ನಿರ್ಲಕ್ಷಿಸಿ, “ನೀನಿನ್ನೂ ನಿನ್ನೀ ಅಲ್ಪ ಭೇದಬುದ್ಧಿಯನ್ನು ಬಿಡಲಿಲ್ಲವಲ್ಲ!” ಎನ್ನುತ್ತ ಅದೇ ಕೈಯಿಂದಲೇ ಹುಕ್ಕ ಸೇದಿದರು. ಅಲ್ಲದೆ, ಒಂದು ವಿಶೇಷ ಭಾವಾವಸ್ಥೆಯಲ್ಲಿ ಆನಂದದಿಂದ ಮಾತುಕತೆಯಾಡುತ್ತ ಕುಳಿತರು. 

ನರೇಂದ್ರ ತಮ್ಮಿಂದ ಬೇರೆಯಲ್ಲವೆಂಬುದನ್ನು ಅವನಿಗೂ ಇತರರಿಗೂ ತಿಳಿಸಿಕೊಡುವುದ ಕ್ಕಾಗಿಯೇ ಈ ಹುಕ್ಕ ಸೇದುವ ನಾಟಕವಾಡುತ್ತಿದ್ದಾರೆ ಶ್ರೀರಾಮಕೃಷ್ಣರು. ಇದನ್ನೆಲ್ಲ ನೋಡುತ್ತಿದ್ದ ವೈಕುಂಠನಾಥನಿಗೆ ಪರಮಾಶ್ಚರ್ಯ. ಯಾವ ಶ್ರೀರಾಮಕೃಷ್ಣರು, ತಮಗಾಗಿ ತಂದ ಆಹಾರದಲ್ಲಿ ಮೊದಲೇ ಒಂದು ಸ್ವಲ್ಪವನ್ನು ಬೇರೆ ಯಾರಾದರೂ ತಿಂದಿದ್ದರೆ ಅದನ್ನು ತಿನ್ನುತ್ತಿರಲಿಲ್ಲವೋ, ಅಷ್ಟೇಕೆ, ತಮಗೆ ತಂದ ಆಹಾರವನ್ನು ಆಘ್ರಾಣಿಸಿದ್ದನ್ನು ಗಮನಿಸಿದರೂ ಅದನ್ನು ಮುಟ್ಟುತ್ತಿರ ಲಿಲ್ಲವೋ, ಅಂಥವರು ಇಂದು ನರೇಂದ್ರನ ವಿಷಯದಲ್ಲಿ ಮಾತ್ರ ಇಷ್ಟು ವಿಭಿನ್ನವಾಗಿ ನಡೆದು ಕೊಂಡಿದ್ದನ್ನು ನೋಡಿ ಅವನಿಗೆ ನಂಬಲೇ ಆಗುತ್ತಿಲ್ಲ. ಅಂತೂ ಈ ಘಟನೆಯಿಂದ ಶ್ರೀರಾಮ ಕೃಷ್ಣರು ನರೇಂದ್ರನ ಮೇಲಿಟ್ಟಿದ್ದ ಆತ್ಮೀಯತೆ, ಪ್ರೀತಿ-ವಿಶ್ವಾಸಗಳನ್ನು ಪ್ರತ್ಯಕ್ಷವಾಗಿ ನೋಡಲು ಅವನಿಗೊಂದು ಅವಕಾಶವಾಯಿತು.

ಹೀಗೆಯೇ ಬಹಳ ಹೊತ್ತು ಕಳೆದು ಹೋಯಿತು. ರಾತ್ರಿ ಎಂಟು ಗಂಟೆಯ ವೇಳೆಗೆ ನರೇಂದ್ರನೂ ವೈಕುಂಠನೂ ಶ್ರೀರಾಮಕೃಷ್ಣರಿಂದ ಬೀಳ್ಗೊಂಡು ಕಲ್ಕತ್ತಕ್ಕೆ ಹೊರಟರು.

ಶ್ರೀರಾಮಕೃಷ್ಣರ ಈ ಬಗೆಯ ಅಲೌಕಿಕವಾದ ಸ್ನೇಹ, ಪ್ರೀತಿ-ವಿಶ್ವಾಸಗಳನ್ನು ಅನುಭವಿಸಿದ ನರೇಂದ್ರ ಮುಂದೆ ಆಗಾಗ ಹೇಳುತ್ತಿದ್ದ: “ಮೊಟ್ಟ ಮೊದಲ ಬಾರಿ ನೋಡಿದಾಗಿನಿಂದಲೂ ನನ್ನನ್ನು ಒಂದೇ ಸಮನಾಗಿ ನಂಬಿದವರೆಂದರೆ ಶ್ರೀರಾಮಕೃಷ್ಣರೊಬ್ಬರೇ. ನನ್ನಲ್ಲಿ ಏಕಪ್ರಕಾರ ವಾಗಿ ವಿಶ್ವಾಸವಿಟ್ಟವರೆಂದರೆ ಅವರು ಮಾತ್ರವೇ. ನನ್ನ ಸ್ವಂತ ತಾಯಿ ಮತ್ತು ಸೋದರರು ಕೂಡ ನನ್ನನ್ನು ಅಷ್ಟರಮಟ್ಟಿಗೆ ನಂಬಲಿಲ್ಲ. ಶ್ರೀರಾಮಕೃಷ್ಣರ ಈ ಅವಿಚ್ಛಿನ್ನ ಪ್ರೀತಿ ಮತ್ತು ವಿಶ್ವಾಸವೇ ನನ್ನನ್ನು ಅವರೊಂದಿಗೆ ಎಂದೆಂದಿಗೂ ಬಂಧಿಸಿಬಿಟ್ಟಿದೆ. ಹೃತ್ಪೂರ್ವಕವಾಗಿ ಪ್ರೀತಿ ಸುವ ಬಗೆ ಏನೆಂಬುದು ಅವರಿಗೆ ಮಾತ್ರ ತಿಳಿದಿತ್ತು. ಪ್ರಾಪಂಚಿಕರ ಪ್ರೀತಿಯೆಲ್ಲ ಕೇವಲ ಸೋಗು, ತೋರಾಣಿಕೆ. ಅದರ ಹಿಂದೆ ಸ್ವಾರ್ಥ ತುಂಬಿಕೊಂಡಿರುತ್ತದೆ.”

ತನ್ನನ್ನು ಶ್ರೀರಾಮಕೃಷ್ಣರೊಂದಿಗೆ ಬಿಗಿದಿಟ್ಟಿರುವ ಅಭೇದ್ಯವಾದ ಬಂಧನವೆಂದರೆ ಅವರ ಅಲೌಕಿಕ, ನಿಸ್ವಾರ್ಥ ಪ್ರೇಮ ಎನ್ನುತ್ತಿದ್ದಾನೆ ನರೇಂದ್ರ. ಅವನು ಅದಾಗಲೇ ಅವರ ಅಪೂರ್ವ ಸಮಾಧಿಸ್ಥಿತಿಯನ್ನೂ ಪ್ರಾಮಾಣಿಕ ತ್ಯಾಗವನ್ನೂ ಸತ್ಯಪರತೆಯೇ ಮೊದಲಾದ ಇತರ ಅನೇಕ ಆಶ್ಚರ್ಯಕರ ಗುಣಗಳನ್ನೂ ಕಂಡು ಬೆರಗಾಗಿದ್ದ, ಮನಸೋತಿದ್ದ. ಆದರೆ ಅವನನ್ನು ಸಂಪೂರ್ಣ ವಾಗಿ ಪರಾಜಯಗೊಳಿಸಿದ ಅಂಶವೆಂದರೆ ಅವರ ಪ್ರೀತಿ-ವಿಶ್ವಾಸವೊಂದೇ. ಯಾವುದೇ ಕಾರಣಕ್ಕೂ ಅವರಿಂದ ದೂರವಾಗದಂತೆ ಅವನನ್ನು ತಡೆಹಿಡಿದಿದ್ದುದು ಆ ಪ್ರೇಮವೊಂದೇ. ಅವನೇನೂ ಜಗತ್ತನ್ನು ಕಂಡರಿಯದ ಮುಗ್ಧನಲ್ಲ. ಅವನಾಗಲೇ ಅಸಂಖ್ಯಾತ ಜನರನ್ನು ಕಂಡಿ ದ್ದಾನೆ, ಅವರ ಪರಸ್ಪರ ವಿಶ್ವಾಸಪೂರ್ಣ ಸಂಬಂಧಗಳನ್ನೂ ಸಾಕಷ್ಟು ಕಂಡಿದ್ದಾನೆ. ಅದೆಲ್ಲವನ್ನೂ ನೋಡಿ ನೋಡಿ, ಆ ಬಗೆಯ ಪ್ರೀತಿಯನ್ನೂ ಸವಿದು, ಈಗ ಶ್ರೀರಾಮಕೃಷ್ಣರ ಪ್ರೀತಿಯ ರೀತಿ ಯನ್ನು ಕಂಡಮೇಲೆ ಅವನಿಗೆ ಅರ್ಥವಾಗುತ್ತಿದೆ–ಜಗತ್ತಿನ ಜನರ ‘ಪ್ರೇಮ’ ಎನ್ನುವುದೆಲ್ಲ ಕಾಮಗಂಧದಿಂದ ಕೂಡಿದ್ದು ಎಂದು. ನಿಸ್ವಾರ್ಥ ಪ್ರೇಮದ ಲಕ್ಷಣವೇನೆಂದರೆ ಅಲ್ಲಿ ಯಾವುದೇ ಬಗೆಯ ಪ್ರತಿಫಲಾಪೇಕ್ಷೆಯ ನಿರೀಕ್ಷೆಯೂ ಇರಲಾರದು. ಅವನು ಶ್ರೀರಾಮಕೃಷ್ಣರಲ್ಲಿ ಕಂಡು ಅನುಭವಿಸಿದುದು ಆ ಬಗೆಯ ನಿಸ್ವಾರ್ಥ ಪ್ರೇಮವನ್ನು. 

ಇಲ್ಲೊಂದು ಪ್ರಶ್ನೆಯೇಳಬಹುದು–ನಿಜಕ್ಕೂ ನಿಸ್ವಾರ್ಥ ಪ್ರೇಮ ಎನ್ನುವಂಥದೇನಾದರೂ ಇರಲು ಸಾಧ್ಯವೆ? ಶ್ರೀರಾಮಕೃಷ್ಣರು ನರೇಂದ್ರನಿಗೆ ಸುರಿದ ಪ್ರೀತಿಯ ಹಿನ್ನೆಲೆಯಲ್ಲೂ ಸ್ವಾರ್ಥ ವಿದೆಯಲ್ಲ?–ಮುಂದೆ ಅವನ ಮೂಲಕ ಧರ್ಮಪ್ರಸಾರ ಮಾಡಿಸಬೇಕು ಎಂಬ ಸ್ವಾರ್ಥೋ ದ್ದೇಶ! ಇದಕ್ಕೆ ಉತ್ತರವಿಷ್ಟೆ–ಈ ಧರ್ಮಪ್ರಸಾರ ಕಾರ್ಯದಿಂದ, ಲೋಕಕಲ್ಯಾಣ ಕಾರ್ಯ ದಿಂದ ಶ್ರೀರಾಮಕೃಷ್ಣರಿಗಾಗಲಿ ನರೇಂದ್ರನಿಗಾಗಲಿ ಲಾಭವೇನೂ ಇಲ್ಲ. ಅವರ ಸ್ವಂತಕ್ಕೆ ಇವುಗಳಿಂದೆಲ್ಲ ಏನೇನೂ ‘ಪ್ರಯೋಜನ’ವಿಲ್ಲ. ಅವರಲ್ಲಿ, ಇವುಗಳಿಂದ ಜಗತ್ತಿನ ಜನರಿಗೆ ಹಿತ ವಾಗಲಿ, ಮಂಗಳವಾಗಲಿ ಎಂಬ ಭಾವವಿದೆಯೇ ಹೊರತು ಸ್ವಾರ್ಥೋದ್ದೇಶವೇನೂ ಇಲ್ಲ. ಸ್ವಾರ್ಥ ಎನ್ನುವುದು ಎಲ್ಲಿ ಬರುತ್ತದೆಂದರೆ ಕಾರ್ಯೋದ್ದೇಶವನ್ನು ತಾನು, ತನ್ನ ಹೆಂಡತಿ- ಮಕ್ಕಳು, ತನ್ನ ಇಷ್ಟಮಿತ್ರರು–ಇಷ್ಟಕ್ಕೆ ಸೀಮಿತಗೊಳಿಸಿಕೊಂಡಾಗ ಮಾತ್ರ. ವಿಶ್ವವನ್ನೇ ತಬ್ಬಿ ಕೊಂಡಾಗ ಸ್ವಾರ್ಥವೆಲ್ಲಿ ಬಂತು! ಆದ್ದರಿಂದ ಶ್ರೀರಾಮಕೃಷ್ಣರು ಮುಂದೆ ನರೇಂದ್ರನಿಂದ ಲೋಕಕಲ್ಯಾಣ ಕಾರ್ಯವನ್ನು ಮಾಡಿಸಬೇಕೆಂಬ ಉದ್ದೇಶವಿಟ್ಟುಕೊಂಡಿದ್ದರೂ ಅದರಲ್ಲಿ ಸ್ವಾರ್ಥದ ಗಂಧವಿರಲಿಲ್ಲ. ಇಂತಹ ಪ್ರೇಮದ ಇನ್ನೊಂದು ಲಕ್ಷಣವೇನೆಂದರೆ ಅಲ್ಲಿ ನಂಬಿಕೆ- ವಿಶ್ವಾಸಗಳು ಸಂಪೂರ್ಣವಾಗಿ ನೆಲೆಗೊಂಡಿರುತ್ತವೆ. ಶ್ರೀರಾಮಕೃಷ್ಣರು ಅದಾಗಲೇ ತಮ್ಮ ಪ್ರೀತಿಗೆ ಅವನು ಪಾತ್ರನೆಂಬುದನ್ನು ಕಂಡುಕೊಂಡಿದ್ದಾರೆ. ಅವನೆಂದೂ ದಾರಿತಪ್ಪಿ ನಡೆಯುವವ ನಲ್ಲ ಎಂಬುದನ್ನು ಜಗನ್ಮಾತೆಯೇ ಅವರಿಗೆ ತೋರಿಸಿಕೊಟ್ಟಿದ್ದಾಳೆ. ಆದ್ದರಿಂದಲೇ ಅವರು ಅವನಲ್ಲಿ ಅಪಾರ ನಂಬಿಕೆಯಿಟ್ಟಿದ್ದಾರೆ, ಮತ್ತು ಅವನ ಮೇಲಿನ ಅವರ ಪ್ರೇಮ ಅಷ್ಟೊಂದು ಪರಿಪೂರ್ಣವಾದದ್ದಾಗಿದೆ. ಅವರ ಈ ದಿವ್ಯ ಪ್ರೇಮ-ವಿಶ್ವಾಸಗಳನ್ನು ಕಂಡೇ ನರೇಂದ್ರನೆನ್ನು ತ್ತಾನೆ–“ಶ್ರೀರಾಮಕೃಷ್ಣರ ಈ ಪ್ರೀತಿ-ವಿಶ್ವಾಸವೇ ನನ್ನನ್ನು ಅವರೊಂದಿಗೆ ಎಂದೆಂದಿಗೂ ಬಂಧಿಸಿಬಿಟ್ಟಿದೆ. ಈ ಲೋಕದಲ್ಲಿ ಇತರರನ್ನು ಪ್ರೀತಿಸುವ ಬಗೆ ಯಾರಿಗಾದರೂ ಗೊತ್ತಿದ್ದರೆ, ಅದು ಶ್ರೀರಾಮಕೃಷ್ಣರಿಗೆ ಮಾತ್ರವೇ... ”

