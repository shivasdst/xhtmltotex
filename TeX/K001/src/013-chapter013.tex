
\chapter{ಮಹಾಗುರುವಿನಿಂದ ಮಹಾಶಿಷ್ಯನಿರ್ಮಾಣ}

\noindent

ಶ್ರೀರಾಮಕೃಷ್ಣರು ನರೇಂದ್ರನನ್ನು ಎಷ್ಟೇ ಪ್ರೀತಿಸಬಹುದು, ಅವನಲ್ಲಿ ಎಷ್ಟೇ ಸಲಿಗೆಯಿಂದಿರ ಬಹುದು; ಆದರೆ ಅವರ ಜೊತೆಯಲ್ಲಿ ಬದುಕುವುದೆಂದರೆ ಅದೊಂದು ತಪಸ್ಸೇ ಸರಿ. ಅದೊಂದು ಕಟ್ಟುನಿಟ್ಟಿನ ಆಧ್ಯಾತ್ಮಿಕ ಸಾಧನೆಯೇ ಸರಿ! ಶ್ರೀರಾಮಕೃಷ್ಣರ ಸನ್ನಿಧಿಯಲ್ಲಿದ್ದಾಗ ಪ್ರತಿಯೊಬ್ಬನ ಮನಸ್ಸೂ ಶಾರೀರಿಕ ವಲಯದಿಂದ ಮೇಲೇರಿ ಆತ್ಮರಾಜ್ಯವನ್ನು ಸಮೀಪಿಸುತ್ತಿತ್ತು. ಅವರ ಜೊತೆಯಲ್ಲಿದ್ದುಕೊಂಡು ಅವರ ಭಾವದೆತ್ತರಕ್ಕೆ ಏರಬೇಕಾದರೆ ಭಕ್ತರಲ್ಲಿ ಅತ್ಯಂತ ಉನ್ನತ ಮಟ್ಟದ ಪ್ರಜ್ಞೆ ಇರಬೇಕಾಗಿತ್ತು, ಸಾಕಷ್ಟು ಉಚ್ಚತರ ನೈತಿಕತೆಯೇ ಇರಬೇಕಾಗಿತ್ತು. ಅವರ ದಿವ್ಯ ಸನ್ನಿಧಿಯೆಂಬುದು ಭಕ್ತರ ಭಾವನೆಗಳನ್ನು ಬಡಿದೆಬ್ಬಿಸಿ, ಉನ್ನತ ಆಧ್ಯಾತ್ಮಿಕ ಆನಂದದೆಡೆಗೆ ಕರೆದೊಯ್ಯುತ್ತಿತ್ತು, ಭಾವೋನ್ಮತ್ತತೆಯನ್ನು ನೀಡುತ್ತಿತ್ತು. ಶ್ರೀರಾಮಕೃಷ್ಣರು ಎಷ್ಟೋ ಸಲ ತಮ್ಮ ದಿವ್ಯಾನುಭವಗಳನ್ನು ಶಬ್ದಗಳಿಂದ ವರ್ಣಿಸಲು ಪ್ರಯತ್ನಿಸುತ್ತಿದ್ದರು. ಆದರೆ, ವರ್ಣಿಸುತ್ತ ಹೋದಂತೆ ಅವರ ಮನಸ್ಸು ಶಬ್ದಗಳನ್ನು ಮೀರಿ ಭಗವಂತನೆಡೆಗೆ ಧಾವಿಸಿಬಿಡುತ್ತಿತ್ತು. ಆಧ್ಯಾತ್ಮಿಕ ಜ್ಯೋತಿಯೊಂಬುದು ಅವರಲ್ಲಿ ಪ್ರಚಂಡ ಪ್ರಕಾಶದಿಂದ ಪ್ರಜ್ವಲಿಸುತ್ತಿತ್ತು. ಅದರ ಪ್ರಭೆಯಲ್ಲಿ ಪ್ರತಿಯೊಬ್ಬನ ಅಂತಸ್ಸತ್ತ್ವವೂ ಬೆಳಗಲಾರಂಭಿಸುತ್ತಿತ್ತು.

ಶ್ರೀರಾಮಕೃಷ್ಣರ ಆತ್ಮೀಯತೆ ಮತ್ತು ಸಲಿಗೆಯ ಮಾಧುರ್ಯದಲ್ಲಿ ಅವರ ಶಿಷ್ಯರೆಲ್ಲ ಹೇಗೆ ಆನಂದದಿಂದ ತೇಲಾಡುತ್ತಿದ್ದರು ಎಂಬುದನ್ನು ಅವರ ಪಾದಗಳ ಬಳಿಯಲ್ಲಿ ಕುಳಿತು ಕಣ್ಣಾರೆ ಕಂಡವರು ಮಾತ್ರ ಬಲ್ಲರು. ಅವರಲ್ಲಿ ಅಹಂಕಾರದ ಲವಲೇಶವೂ ಇರಲಿಲ್ಲ. ಬಿಂಕ- ಬಿಗುಮಾನಗಳಿರಲಿಲ್ಲ. ಅಲ್ಲಿದ್ದುದೆಲ್ಲ ತುಂಬ ಸರಳವಾದ ಸಲಿಗೆಯ ಮಾನವೀಯ ಸಂಬಂಧ; ಸಹಜವಾದ ಮೃದುಮಧುರ ಸಂಬಂಧ. ಇವುಗಳಿಂದಾಗಿ ಅವರ ಸುತ್ತಮುತ್ತ ಒಂದು ಆತ್ಮೀಯ ವಾತಾವರಣ ನೆಲೆಗೊಂಡಿತ್ತು. ಅದರಲ್ಲೊಂದು ಅಲೌಕಿಕತೆ ಎದ್ದುಕಾಣುತ್ತಿತ್ತು. ಅಲ್ಲಿ ಭಗವಂತನ ಸಾನ್ನಿಧ್ಯವೇ ಏರ್ಪಟ್ಟಿತ್ತು ಎಂದರೆ ಅದು ಸತ್ಯದೂರವಾದ ಮಾತಲ್ಲ. ಹೀಗಿದ್ದರೂ, ದಕ್ಷಿಣೇಶ್ವರದ ಆ ಭವ್ಯ ವೃಕ್ಷಗಳ ತಲದಲ್ಲಿ ಮತ್ತು ಶ್ರೀರಾಮಕೃಷ್ಣರ ಕೋಣೆಯಲ್ಲಿ ಎಷ್ಟೋ ಸಲ ನಗೆಯ ಹೊನಲೇ ಹರಿಯುತ್ತಿತ್ತು. ಶ್ರೀರಾಮಕೃಷ್ಣರೂ ಅವರ ಶಿಷ್ಯರೂ ಸೇರಿ ಅಲ್ಲಿ ಆತ್ಮೀಯವಾಗಿ ಮಾತನಾಡುತ್ತ ಹಾಸ್ಯಚಟಾಕಿಗಳನ್ನು ಹಾರಿಸುತ್ತ ಹೊಟ್ಟೆ ತುಂಬ ನಗುತ್ತಿದ್ದ ಸಂದರ್ಭಗಳು ಅದೆಷ್ಟೊ! ಆದರೆ ಶ್ರೀರಾಮಕೃಷ್ಣರು ಸಾಂದರ್ಭಿಕವಾಗಿ ಆಡುವ ಒಂದೆರಡು ಮಾತುಗಳು ಆ ಹಾಸ್ಯದ ವಾತಾವರಣವನ್ನು ಬದಲಾಯಿಸಿ, ಒಂದು ಮಂಗಳಕರ ಆಧ್ಯಾತ್ಮಿಕ ವಾತಾವರಣವನ್ನುಂಟುಮಾಡುತ್ತಿದ್ದುವು; ಆಗ ಮಾನುಷ ಆನಂದ ಮರೆಯಾಗಿ, ದಿವ್ಯಾನಂದ ಉದ್ಭವವಾಗುತ್ತಿತ್ತು. ಆ ದೃಶ್ಯವನ್ನು ನರೇಂದ್ರ ಮುಂದೆ ಈ ರೀತಿಯಾಗಿ ವರ್ಣಿಸುತ್ತಾನೆ:

“ನಾವೆಲ್ಲ ಶ್ರೀರಾಮಕೃಷ್ಣರ ಸನ್ನಿಧಿಯಲ್ಲಿ ಅನುಭವಿಸಿದ ಆನಂದವನ್ನು ಶಬ್ದಗಳಿಂದ ವರ್ಣಿಸಿ ತಿಳಿಸಲು ಸಾಧ್ಯವೇ ಇಲ್ಲ. ಅವರು ತಮಾಷೆ ಮಾಡುತ್ತಲೇ, ನಗುನಗುತ್ತಲೇ ನಮಗೇ ಅರಿವಾಗ ದಂತೆ ನಮ್ಮನ್ನೆಲ್ಲ ತಿದ್ದಿ. ತರಬೇತಿ ಕೊಟ್ಟು, ನಮ್ಮ ಆಧ್ಯಾತ್ಮಿಕ ಜೀವನವನ್ನು ರೂಪಿಸಿದ ಪರಿ ಪರಮಾದ್ಭುತ. ಶ್ರೀರಾಮಕೃಷ್ಣರು ಒಬ್ಬ ನುರಿತ ಪೈಲವಾನನಂತೆ–ಆತ ತನ್ನ ಹೊಸ ಶಿಷ್ಯನಿಗೆ ತರಬೇತಿ ಕೊಡುವಾಗ ಬಹಳ ಎಚ್ಚರಿಕೆಯಿಂದ ನಿಧಾನವಾಗಿ ಮುಂದುವರಿಯುತ್ತಾನೆ. ಕೆಲವು ಸಲ ಬಹಳ ಕಷ್ಟದಿಂದಲೋ ಎಂಬಂತೆ ಶಿಷ್ಯನನ್ನು ಸೋಲಿಸುತ್ತಾನೆ; ಇನ್ನು ಕೆಲವು ಸಲ ಬೇಕುಬೇಕೆಂದೇ ಅವನ ಕೈಯಲ್ಲಿ ತಾನೇ ಸೋಲುತ್ತಾನೆ. ಹೀಗೆ ಶಿಷ್ಯನಲ್ಲಿ ಆತ್ಮವಿಶ್ವಾಸ ಹುಟ್ಟಿಸುತ್ತಾನೆ. ಶ್ರೀರಾಮಕೃಷ್ಣರು ನಮಗೆಲ್ಲ ಶಿಕ್ಷಣ ಕೊಟ್ಟ ಬಗೆ ತದ್ವತ್ ಅದೇ ರೀತಿ. ಅವರು ನಮ್ಮ ಪ್ರತಿಯೊಬ್ಬನಲ್ಲೂ ಸುಪ್ತವಾಗಿರುವ ದೈವತ್ವವನ್ನು ಕಾಣಬಲ್ಲವರಾಗಿದ್ದರು. ನಮ್ಮಲ್ಲಿ ಆತ್ಮಜ್ಯೋತಿ ಅತ್ಯಂತ ಕಿರುಗಾತ್ರದಲ್ಲಿ ಬೆಳಗುತ್ತಿದ್ದರೂ, ಮುಂದೆ ಸಕಾಲದಲ್ಲಿ ಅದು ಬೃಹ ದ್ಗಾತ್ರದಲ್ಲಿ ಪ್ರಕಾಶಿಸಲಿರುವುದನ್ನು ಅವರು ಮುಂಗಾಣಬಲ್ಲವರಾಗಿದ್ದರು. ಆ ಭವ್ಯ ಚಿತ್ರವನ್ನು ಈಗಲೇ ನಮ್ಮ ಕಣ್ಮುಂದೆ ಹಿಡಿಯುತ್ತ ನಮ್ಮನ್ನು ಹುರಿದುಂಬಿಸುತ್ತಿದ್ದರು. ನಮ್ಮ ಕುರಿತು ಪ್ರಶಂಸೆಯ ಮಾತುಗಳನ್ನೇ ಆಡುತ್ತಿದ್ದರು. ಆದರೆ ಅದರೊಂದಿಗೆ, ಆತ್ಮಸಾಕ್ಷಾತ್ಕಾರದ ದಾರಿ ಯಲ್ಲಿ ಅಡ್ಡವಾಗಿರುವ ಪ್ರಾಪಂಚಿಕತೆಯ ಸುಳಿಯೊಳಗೆ ಸಿಲುಕಿಕೊಂಡು ನಾವು ಮುಳುಗಿಹೋಗ ದಂತೆ ಎಚ್ಚರಿಕೆಯನ್ನು ಕೊಡುತ್ತಿದ್ದರು. ಅಲ್ಲದೆ, ನಮ್ಮ ನಿತ್ಯಜೀವನದ ಪ್ರತಿಯೊಂದು ಅಂಶ ವನ್ನೂ ಸೂಕ್ಷ್ಮವಾಗಿ ಪರಿಶೀಲಿಸಿ ನೋಡಿ, ನಮ್ಮನ್ನು ಒಂದು ಪರಿಮಿತಿಯೊಳಗೇ ಇಟ್ಟಿರುತ್ತಿ ದ್ದರು. ಆದರೆ ಇವೆಲ್ಲವನ್ನೂ ಅವರು ಶಿಷ್ಯರ ಅರಿವಿಗೇ ಬಾರದಂತೆ ಸದ್ದಿಲ್ಲದೆ ಮಾಡುತ್ತಿದ್ದರು. ಇದೇ ಅವರ ಶಿಕ್ಷಣದ ರಹಸ್ಯ; ಇದೇ ಅವರು ಶಿಷ್ಯರ ಜೀವನಗಳನ್ನು ರೂಪಿಸುತ್ತಿದ್ದ ವಿಧಾನ.

“ಒಮ್ಮೆ ನಾನು ಧ್ಯಾನ ಮಾಡಲು ಕುಳಿತಾಗ, ನನ್ನ ಮನಸ್ಸು ಏನು ಮಾಡಿದರೂ ಏಕಾಗ್ರವಾಗದೆ ಹೋಯಿತು. ನಾನು ಅದನ್ನು ಶ್ರೀರಾಮಕೃಷ್ಣರಿಗೆ ತಿಳಿಸಿ, ಅವರ ಸಲಹೆ ಕೇಳಿದೆ. ಆಗ ಅವರು ಈ ವಿಷಯದಲ್ಲಿ ತಮ್ಮ ಕೆಲವು ಅನುಭವಗಳನ್ನು ನನಗೆ ತಿಳಿಸಿ, ಕೆಲವು ಉಪಯುಕ್ತ ಸಲಹೆಗಳನ್ನು ಕೊಟ್ಟರು. ಇನ್ನೊಂದು ಸಲ, ನಾನು ಮುಂಜಾನೆಯ ಶಾಂತವಾದ ಸಮಯದಲ್ಲಿ ಧ್ಯಾನಕ್ಕೆ ಕುಳಿತುಕೊಂಡರೆ, ಹತ್ತಿರದಲ್ಲಿದ್ದ ಗೋಣಿನಾರಿನ ಕಾರ್ಖಾನೆಯ ಸಿಳ್ಳಿನ ಶಬ್ದ ಧ್ಯಾನಕ್ಕೆ ಭಂಗ ತರುತ್ತಿತ್ತು. ಶ್ರೀರಾಮಕೃಷ್ಣರಲ್ಲಿ ಇದನ್ನು ಹೇಳಿಕೊಂಡಾಗ, ಅವರು ಆ ಸಿಳ್ಳಿನ ಶಬ್ದದ ಮೇಲೆಯೇ ಮನಸ್ಸನ್ನು ಏಕಾಗ್ರಗೊಳಿಸುವಂತೆ ಹೇಳಿದರು. ನಾನು ಅದರಂತೆಯೇ ಮಾಡಿದೆ. ಆ ಉಪಾಯ ತುಂಬ ಯಶಸ್ವಿಯಾಯಿತು. ಮತ್ತೊಮ್ಮೆ, ಎಷ್ಟೇ ಪ್ರಯತ್ನಮಾಡಿದರೂ ಶರೀರಪ್ರಜ್ಞೆಯನ್ನು ಮೀರಿ ಸಂಪೂರ್ಣ ಏಕಾಗ್ರತೆಯಿಂದ ಧ್ಯಾನಮಾಡಲು ಸಾಧ್ಯವೇ ಆಗುತ್ತಿರಲಿಲ್ಲ. ನಾನಾಗ ಅವರ ಮೊರೆಹೋದೆ. ಆಗ ಅವರು ತಮ್ಮ ಗುರು ತೋತಾಪುರಿ ಮಾಡಿದ್ದ ಉಪಾಯವನ್ನೇ ನನ್ನ ಮೇಲೂ ಪ್ರಯೋಗಿಸಿದರು. ನನ್ನ ಭ್ರೂಮಧ್ಯವನ್ನು ತಮ್ಮ ಉಗುರಿನಿಂದ ಬಲವಾಗಿ ಒತ್ತಿ ಹಿಡಿದು, ‘ಈ ನೋವಿನ ಜಾಗದಲ್ಲಿ ನಿನ್ನ ಮನಸ್ಸನ್ನು ಏಕಾಗ್ರಗೊಳಿಸು’ ಎಂದರು. ನಾನು ಹಾಗೆಯೇ ಮಾಡಲು ಪ್ರಯತ್ನಪಟ್ಟೆ. ನೋಡುತ್ತೇನೆ–ನನಗೀಗ ಭ್ರೂಮಧ್ಯದಲ್ಲಿ ಮನಸ್ಸನ್ನು ಎಷ್ಟು ಹೊತ್ತು ಬೇಕಾದರೂ ಏಕಾಗ್ರಗೊಳಿಸಲು ಸಾಧ್ಯವಾಗುತ್ತಿದೆ! ಅಲ್ಲದೆ, ಶರೀರಪ್ರಜ್ಞೆಯನ್ನು ಸಂಪೂರ್ಣ ಮರೆತು ಕುಳಿತುಕೊಳ್ಳಲು ಸಾಧ್ಯವಾಗುತ್ತಿದೆ! ಶರೀರವನ್ನೇ ಮರೆತ ಮೇಲೆ ಮೈಕೈ ನೋಯುವುದಾಗಲಿ, ಮನಸ್ಸನ್ನು ಅವು ಚಂಚಲಗೊಳಿಸುವುದಾಗಲಿ ದೂರವೇ ಉಳಿಯಿತು. ಆದ್ದರಿಂದ ಧ್ಯಾನ ನಿರಾತಂಕವಾಗಿ ನಡೆಯಿತು.

“ಪಂಚವಟಿ ನಿರ್ಜನವಾದ, ಬಹಳ ಪ್ರಶಾಂತವಾದ ಸ್ಥಳ. ಅಲ್ಲದೆ, ಸ್ವತಃ ಶ್ರೀರಾಮಕೃಷ್ಣರು ಆಧ್ಯಾತ್ಮಿಕ ಸಾಧನೆಗಳನ್ನು ಮಾಡಿ ಸಾಕ್ಷಾತ್ಕಾರಗಳನ್ನು ಪಡೆದುಕೊಂಡ ಪವಿತ್ರ ಸ್ಥಳ. ಆದ್ದರಿಂದ ನಮ್ಮ ಪಾಲಿಗೆ ಆ ಸ್ಥಳ ಧ್ಯಾನಕ್ಕೆ ಬಹಳ ಪ್ರಶಸ್ತವಾಗಿತ್ತು. ಅಲ್ಲಿ ನಾವು ಧ್ಯಾನ ಜಪತಪಾದಿಗಳನ್ನು ಮಾಡುವುದಲ್ಲದೆ, ಬಹಳಷ್ಟು ಸಮಯವನ್ನು ಸಂತೋಷಕರವಾದ ಮಾತುಕತೆ, ನಲಿದಾಟಗಳಲ್ಲಿ ಕಳೆಯುತ್ತಿದ್ದೆವು. ಈ ನಮ್ಮ ಚಟುವಟಿಕೆಗಳಲ್ಲಿ ಕೆಲವೊಮ್ಮೆ ಶ್ರೀರಾಮಕೃಷ್ಣರೂ ಪಾಲ್ಗೊಂಡು, ಅಲ್ಲಿನ ನಿಷ್ಕಲ್ಮಶ ಆನಂದವನ್ನು ಇನ್ನಷ್ಟು ಹೆಚ್ಚಿಸುತ್ತಿದ್ದರು. ಆ ಪಂಚವಟಿಯಲ್ಲಿ ಸಂತೋಷ ದಿಂದ ಓಡಾಡುವುದು, ಕುಣಿದು ಕುಪ್ಪಳಿಸುವುದು, ಮರಗಳನ್ನು ಹತ್ತುವುದು, ಬಿಳಲುಗಳನ್ನು ಹಿಡಿದುಕೊಂಡು ಜೋತಾಡುವುದು–ಮುಂತಾದ ನಾನಾ ರೀತಿಯ ಆಮೋದ-ಪ್ರಮೋದಗಳಲ್ಲಿ ತೊಡಗುತ್ತಿದ್ದೆವು. ಕೆಲವೊಮ್ಮೆ ವನಭೋಜವನ್ನೂ ನಡೆಸುತ್ತಿದ್ದೆವು. ಅಲ್ಲಿ ಮೊದಲ ಸಲ ವನ ಭೋಜನವನ್ನು ಏರ್ಪಡಿಸಿದ್ದಾಗ ಅಡುಗೆ ಮಾಡಿದ್ದವನು ನಾನೇ. ಇದನ್ನು ತಿಳಿದ ಶ್ರೀರಾಮ ಕೃಷ್ಣರೂ ಅಂದು ಅದರಲ್ಲಿ ಭಾಗವಹಿಸಿದರು. ಆದರೆ ಅವರು ಬ್ರಾಹ್ಮಣರು ಮಾಡಿದ ಅಡುಗೆ ಯನ್ನು ಮಾತ್ರ ಸ್ವೀಕರಿಸಬಲ್ಲರು ಎಂಬುದು ನನಗೆ ತಿಳಿದಿತ್ತು. ಆದ್ದರಿಂದ ಅವರಿಗಾಗಿ ಕಾಳೀ ದೇವಾಲಯದ ಪ್ರಸಾದವನ್ನೇ ತರಿಸಿದ್ದೆ. ಆದರೆ ಅವರು ‘ನಿನ್ನಂಥ ಪರಿಶುದ್ಧಾತ್ಮರು ಮಾಡಿದ ಅಡುಗೆಯನ್ನು ಸ್ವೀಕರಿಸುವುದರಲ್ಲಿ ಏನೂ ದೋಷವಿಲ್ಲ’ ಎನ್ನುತ್ತ, ನಾನೆಷ್ಟು ತಡೆದರೂ ಕೇಳದೆ, ನಾನು ಮಾಡಿದ ಅಡುಗೆಯನ್ನೇ ಮೆಚ್ಚಿಕೊಳ್ಳುತ್ತ, ಆನಂದದಿಂದ ಊಟ ಮಾಡಿದರು.”

ಈ ವರದಿಯಿಂದ ನರೇಂದ್ರಾದಿಗಳು ಶ್ರೀರಾಮಕೃಷ್ಣರ ದಿವ್ಯ ಸಂಗದಲ್ಲಿ ಕಳೆದ ಆನಂದದ ದಿನಗಳ, ಅವರ ಪಡೆದ ಶಿಕ್ಷಣವಿನ್ಯಾಸದ ಒಂದು ಸ್ಥೂಲ ಪರಿಚಯವಾಗುತ್ತದೆ ನಮಗೆ.

ನರೇಂದ್ರ ಈಗೀಗ ನಿಜಸ್ವರೂಪದಿಂದ ವ್ಯಕ್ತನಾಗತೊಡಗಿದ್ದ. ತಾಯಿಸಿಂಹದ ಮುಂದೆ ಮರಿಸಿಂಹ ಕುಣಿದು ಕುಪ್ಪಳಿಸುವಂತೆ, ಶ್ರೀರಾಮಕೃಷ್ಣರ ಸನ್ನಿಧಿಯಲ್ಲಿ ಅವನು ಸ್ವೇಚ್ಛೆಯಾಗಿ ವಿಹರಿಸುತ್ತಿದ್ದ. ತಂದೆಯ ಮರಣಾನಂತರ ಕೆಲ ಸಮಯ ಅವನ ಶಕ್ತಿ-ಸಾಹಸಗಳೆಲ್ಲ ಒಳಗೆ ಸೇರಿಕೊಂಡಿದ್ದುವು. ಈಗ ಆ ದುಃಖದ ದಿನಗಳು ಮರೆಯಾಗುತ್ತಿವೆ. ಜೊತೆಗೆ ಶ್ರೀರಾಮಕೃಷ್ಣರ ವಾತ್ಸಲ್ಯದ ರಕ್ಷೆ ಸಿಕ್ಕಿದೆ. ಆದ್ದರಿಂದ ಅವನ ಶಕ್ತಿ-ಸಾಹಸಗಳೆಲ್ಲ ಈಗ ಮತ್ತೊಮ್ಮೆ ಪೂರ್ಣರೂಪ ದಿಂದ ವ್ಯಕ್ತವಾಗತೊಡಗಿದ್ದುವು. ಅವನ ಅನಂತ ಚೈತನ್ಯವೀಗ ಆಧ್ಯಾತ್ಮಿಕ ಶಕ್ತಿಯಾಗಿ ಪ್ರವಾಹ ದೋಪಾದಿಯಲ್ಲಿ ಹರಿಯಲಾರಂಭಿಸಿತು. ಅದನ್ನು ಕಂಡು ಶ್ರೀರಾಮಕೃಷ್ಣರಿಗೆ ಬಹಳ ಆನಂದ ವಾಗಿಬಿಟ್ಟಿತು. ಆತನ ಆಧ್ಯಾತ್ಮಿಕ ಪ್ರಗತಿಯನ್ನು ಅವರು ಸ್ಪಷ್ಟವಾಗಿ ಕಾಣುತ್ತಿದ್ದರು. ಶ್ರೀರಾಮ ಕೃಷ್ಣರು ತೀವ್ರ ಸಾಧನೆಯ ಫಲವಾಗಿ ಅತ್ಯುನ್ನತ ಸಾಕ್ಷಾತ್ಕಾರವನ್ನು ಪಡೆದು ಆಧ್ಯಾತ್ಮಿಕ ಸಾಮ್ರಾಜ್ಯದ ಚಕ್ರವರ್ತಿಯಂತೆ ವಿರಾಜಿಸುತ್ತಿದ್ದರೆ, ಆ ಮಹಾಸಾಮ್ರಾಜ್ಯದ ಉತ್ತರಾಧಿಕಾರಿ ಯಾದ ರಾಜಕುಮಾರನಂತೆ ಬೆಳಗುತ್ತಿದ್ದ ನರೇಂದ್ರ–ಅದೇ ರಾಜಗಾಂಭೀರ್ಯ, ಆದೇ ತೇಜಸ್ಸು, ಅದೇ ಪರಾಕ್ರಮ ಅವನಲ್ಲಿ. ಶ್ರೀರಾಮಕೃಷ್ಣರು ಅವನಿಗೆ ಸಂಪೂರ್ಣಸ್ವಾತಂತ್ರ್ಯ ವನ್ನಿತ್ತಿದ್ದರು. ಅವನ ಮನಸ್ಸು ಎಷ್ಟು ಸ್ವತಂತ್ರವಾಗಿ, ಹೊಸ ರೀತಿಯಲ್ಲಿ ಆಲೋಚನೆ ಮಾಡಬಲ್ಲದೋ ಅಷ್ಟೂ ಅವಕಾಶ ಕೊಟ್ಟಿದ್ದರು. ಅವನ ಅಂತಸ್ಸತ್ವ ಪೂರ್ಣ ತೇಜಸ್ಸಿನಿಂದ ಪ್ರಕಟಗೊಳ್ಳಲು ಎಲ್ಲ ರೀತಿಯಲ್ಲೂ ಸಹಾಯ ಮಾಡುತ್ತಿದ್ದರು. ಅವನ ಜಾಗೃತ ಮನಸ್ಸೇ ಅವನಿಗೆ ಗುರುವಾಗುವಂತೆ ಅನುಕೂಲಿಸಿಕೊಟ್ಟರು. ಅವರವರ ಪ್ರಾಮಾಣಿಕ ಪ್ರಯತ್ನವೇ ಕಾಲಕ್ರಮದಲ್ಲಿ ಅವರವರ ಹೃದಯದಲ್ಲಿ ಜ್ಯೋತಿ ಬೆಳಗುವಂತೆ ಮಾಡುತ್ತದೆ ಎಂಬುದು ಶ್ರೀರಾಮಕೃಷ್ಣರ ನಿಶ್ಚಿತ ಅಭಿಪ್ರಾಯ. ಆದ್ದರಿಂದ ಅವರು ನರೇಂದ್ರನಿಗೆ ಮನಬಿಚ್ಚಿ ಮಾತ ನಾಡಲು ನಡೆದುಕೊಳ್ಳಲು, ಸಂಪೂರ್ಣ ಸ್ವಾತಂತ್ರ್ಯವಿತ್ತಿದ್ದರು. ಅಲ್ಲದೆ, “ನೋಡು, ನರೇನ್, ನಾನು ಹೇಳಿದೆನೆಂಬ ಕಾರಣಕ್ಕಾಗಿ ನೀನು ಯಾವುದನ್ನೂ ಒಪ್ಪಿಕೊಳ್ಳಬೇಕಾಗಿಲ್ಲ. ನೀನೇ ಅದನ್ನೆಲ್ಲ ಪರೀಕ್ಷೆ ಮಾಡಿ ನೋಡು. ನೀನು ಗುರಿಯನ್ನು ಸೇರುವುದು ಸತ್ಯವನ್ನು \textit{ಸಾಕ್ಷಾತ್ಕಾರಿಸಿಕೊಳ್ಳುವುದರ} ಮೂಲಕ ಮಾತ್ರವೇ” ಎನ್ನುತ್ತಿದ್ದರು.

ಈ ಅನುಮತಿಯ ಮೇರೆಗೆ ನರೇಂದ್ರ ಅವರ ಪ್ರತಿಯೊಂದು ಮಾತನ್ನೂ ಪರೀಕ್ಷೆ ಮಾಡಿದ್ದ ರಿಂದಲೇ ಅವರನ್ನು ಆಮೂಲಾಗ್ರವಾಗಿ ಅರಿತುಕೊಳ್ಳಲು ಸಾಧ್ಯವಾಯಿತು. ಶಿಷ್ಯರಲ್ಲೆಲ್ಲ ಅವರ ಮಹತ್ವವನ್ನು ಸಂಪೂರ್ಣವಾಗಿ ಅರ್ಥಮಾಡಿಕೊಂಡವನೆಂದರೆ ಅವನೇ ಎನ್ನಬಹುದು. ಏಕೆಂದರೆ, ಶ್ರೀರಾಮಕೃಷ್ಣರ ಮಾತನ್ನೂ ಶಂಕಿಸಿ ಪ್ರಶ್ನಿಸುವ ಧೈರ್ಯವಿದ್ದುದು ಅವನಿಗೆ ಮಾತ್ರ. ಅವರ ಪ್ರತಿಯೊಂದು ಮಾತನ್ನೂ ಅವನು ತನ್ನ ವಿಮರ್ಶೆಯ ತಕ್ಕಡಿಯಲ್ಲಿ ತೂಗಿನೋಡಿ ಸಂಶಯಗಳನ್ನು ಪರಿಹರಿಸಿಕೊಳ್ಳುತ್ತಿದ್ದ. ಅವೆಲ್ಲ ಪರಿಹಾರವಾದ ಮೇಲಷ್ಟೇ ಅವರ ವಾಕ್ಯಗಳಲ್ಲಿ ಶ್ರದ್ಧೆ ತಾಳುತ್ತಿದ್ದ. ಆದರೆ ಇತರ ಶಿಷ್ಯರ ವಿಚಾರವೇ ಬೇರೆ. ಅವರು ಶ್ರೀರಾಮಕೃಷ್ಣರ ಮುಖ ಕಮಲದಿಂದ ಹೊರಬಿದ್ದ ಮಾತುಗಳನ್ನೆಲ್ಲ ಮರುಮಾತಿಲ್ಲದೆ ಸ್ವೀಕರಿಸುವವರು. ಏಕೆಂದರೆ ಅವರೆಲ್ಲ ‘ಭಕ್ತರು’. ಶ್ರೀರಾಮಕೃಷ್ಣರ ಮೇಲಿನ ಅವರ ಭಕ್ತಿ ಪ್ರೀತಿ ಅಪಾರ. ತತ್ಪರಿಣಾಮವಾಗಿ ಅವರಿಗೆ ಶ್ರೀರಾಮಕೃಷ್ಣರ ಮಾತುಗಳಲ್ಲಿ ಸಹಜವಾಗಿಯೇ ಅವಿಚಲ ಶ್ರದ್ಧೆ. ಈ ಪ್ರೀತಿಯ ಬಲದಿಂದಲೇ ಅವರು ಶ್ರೀರಾಮಕೃಷ್ಣರನ್ನು ಅರ್ಥಮಾಡಿಕೊಳ್ಳುತ್ತಿದ್ದರು. ಆದರೆ ನರೇಂದ್ರ ಹಾಗಲ್ಲ, ಅವನು ಅವರಾಡುವ ಮಾತುಗಳನ್ನೆಲ್ಲ ಪ್ರಶ್ನಿಸುತ್ತಿದ್ದ. ಕೆಲವೊಮ್ಮೆ ಅವರ ಮಾತು ಗಳನ್ನು ಕೇಳಿ ಪರಿಹಾಸ ಮಾಡುತ್ತಿದ್ದ, ಟೀಕಿಸುತ್ತಿದ್ದ. ಹೀಗೆಲ್ಲ ಮಾಡಿದ ಮಾತ್ರಕ್ಕೆ ಅವನಿಗೆ ಅವರ ಮೇಲೆ ಪ್ರೀತಿ ಇರಲಿಲ್ಲವೆಂದೇನೂ ಅಲ್ಲ. ಹಾಗೆ ನೋಡಿದರೆ, ಇತರ ಶಿಷ್ಯರ ಗುರುಭಕ್ತಿ ಗಿಂತ ಅವನ ಗುರುಭಕ್ತಿ ಅತ್ಯಂತ ವಿಶಿಷ್ಟವಾದದ್ದು, ಅತಿಶಯವಾದದ್ದು. ಈ ಭಕ್ತಿಯ ಪರಿಣಾಮವಾಗಿ ಅವನು ಅವರಿಗಾಗಿ ಯಾವ ಸೇವೆಯನ್ನಾದರೂ ಮಾಡಲು ಸಿದ್ಧನಾದದ್ದನ್ನು ಮುಂದೆ ನಾವು ನೋಡಲಿದ್ದೇವೆ. ಆದರೆ ಈಗಂತೂ ಅವನ ಯಾವುದೇ ಉಪದೇಶವನ್ನಾಗಲಿ, ಚೆನ್ನಾಗಿ ಪರೀಕ್ಷಿಸಿಯೇ ಸ್ವೀಕರಿಸುವ ಕ್ರಮವನ್ನಿಟ್ಟುಕೊಂಡಿದ್ದ. ಆ ಪ್ರತಿಯೊಂದು ಉಪದೇಶದ ತಥ್ಯವೂ ತನ್ನ ಮನಸ್ಸಿಗೆ ಗೋಚರಿಸಲೇಬೇಕು ಎಂಬುದು ಅವನ ನಿಲುವು. ಆದರೆ ಅವತಾರ ಪುರುಷರ, ಸಂತರ ಎಷ್ಟೋ ಉಪದೇಶಗಳು ಬುದ್ಧಿಗ್ರಾಹ್ಯವಲ್ಲ; ಆದ್ದರಿಂದ ಅಲ್ಲಿ ವೈಚಾರಿಕ ಬುದ್ಧಿಗೆ ಪ್ರವೇಶವಿಲ್ಲ. ಇದೂ ಅವನಿಗೆ ತಿಳಿದಿಲ್ಲವೆಂದಲ್ಲ. ಆದರೆ, ವೈಚಾರಿಕಬುದ್ಧಿಗೆ ಗ್ರಾಹ್ಯ ವಾಗುವ ಅಂಶಗಳನ್ನಾದರೂ ಅವರು ಯುಕ್ತಿಯುಕ್ತವಾಗಿ ತಿಳಿಸಿಕೊಡಲಿ ಎಂಬುದು ಅವನ ಆಶಯ.

ತನ್ನಲ್ಲಿ ಸಹಜವಾಗಿದ್ದ ಆತ್ಮಶಕ್ತಿ ಹಾಗೂ ಹರಿತವಾದ ತರ್ಕಶಕ್ತಿ–ಈ ಎರಡು ಶಕ್ತಿಗಳ ಸಹಾಯದಿಂದ ನರೇಂದ್ರ ಸತ್ಯಶೋಧನೆ ಮಾಡುತ್ತ ಸಾಕ್ಷಾತ್ಕಾರದೆಡೆಗೆ ವೇಗವಾಗಿ ಸಾಗಿದ್ದ. ಅದನ್ನು ಕಂಡು ಆತನ ಬಗ್ಗೆ ಶ್ರೀರಾಮಕೃಷ್ಣರ ಹೃದಯ ತುಂಬಿ ಬರುತ್ತಿತ್ತು. ಒಂದು ಸಂದರ್ಭದಲ್ಲಿ ತಮ್ಮ ಸುತ್ತ ಕುಳಿತಿದ್ದ ಶಿಷ್ಯರಿಗೆ ನರೇಂದ್ರನನ್ನು ತೋರಿಸುತ್ತ ಅವರು ಉದ್ಗರಿಸುತ್ತಾರೆ: “ನೋಡಿ ಅವನನ್ನು! ಅವನ ಗ್ರಹಣಶಕ್ತಿ ಅದೆಷ್ಟು ಅದ್ಭುತವಾಗಿದೆ! ಅವನಲ್ಲಿ ಪ್ರಖರ ಜ್ಞಾನವೆಂಬುದು ಪಾರವಿಲ್ಲದ ಸಾಗರದಂತಿದೆ. ಸಾಕ್ಷಾತ್ ಮಹಾಮಾಯೆ ಕೂಡ ಅವನ ಹತ್ತಿರ ಸುಳಿಯಲಾರಳು. ಆಕೆ ಅವನಿಗಿಂತ ಹತ್ತು ಅಡಿ ದೂರದಲ್ಲೇ ನಿಲ್ಲಬೇಕು! ಅವನಿಗೆ ಈ ಆಧ್ಯಾತ್ಮಿಕ ಹಾಗೂ ಬೌದ್ಧಿಕ ವೈಭವವನ್ನು ಕರುಣಿಸಿದವಳು ಅವಳೇ. ಆದರೆ ಆ ವೈಭವದ ಮುಂದೆ, ಪ್ರಖರ ತೇಜಸ್ಸಿನ ಮುಂದೆ ಅವಳೇ ನಿಲ್ಲಲಾರದವಳಾಗಿದ್ದಾಳೆ!”

ಆದರೆ ಶ್ರೀರಾಮಕೃಷ್ಣರೇ ಕೆಲವೊಮ್ಮೆ ಜಗನ್ಮಾತೆಯನ್ನು ‘ಅಮ್ಮಾ, ದಯಮಾಡಿ ನಿನ್ನ ಮಾಯೆಯನ್ನು ನರೇಂದ್ರನ ಮೇಲೆ ಸ್ವಲ್ಪ ಹರಡು’ ಎಂದು ಪ್ರಾರ್ಥಿಸಿಕೊಳ್ಳುತ್ತಿದ್ದರು. ಸಾಮಾನ್ಯ ಜನರ ವಿಷಯದಲ್ಲಾದರೆ, ‘ನಿನ್ನ ಮಾಯೆಯ ಆವರಣವನ್ನು ಸ್ವಲ್ಪ ಸರಿಸು’ ಎಂದು ಪ್ರಾರ್ಥಿಸಿ ಕೊಳ್ಳಬೇಕಾಗುತ್ತದೆ. ಆದರೆ ನರೇಂದ್ರನ ವಿಷಯದಲ್ಲಿ ಅದು ತದ್ವಿರುದ್ಧ. ಏಕೆಂದರೆ ಮುಂದೆ ಅವನಿಂದ ಲೋಕಕಲ್ಯಾಣದ ಕಾರ್ಯ ಆಗಬೇಕಾಗಿದೆ. ಸ್ವಲ್ಪ ಮಟ್ಟಿಗಾದರೂ ಮಾಯೆಯ ಅವರಣವಿಲ್ಲದೆ ಹೋದರೆ ಲೋಕಹಿತಚಿಂತನೆ ಮಾಡಲು ಸಾಧ್ಯವಾಗುವುದಿಲ್ಲ. ಏಕೆಂದರೆ, ಮಾಯೆಯ ಒಂದಂಶವಾದರೂ ಇಲ್ಲದಿದ್ದರೆ ಮನುಷ್ಯ ಶುದ್ಧಸತ್ತ್ವನಾಗುತ್ತಾನೆ. ಅಂಥವನ ಮನಸ್ಸನ್ನು ಅಲ್ಲಿಂದ ಕೆಳಗಳೆಯಲು ಸಾಧ್ಯವೇ ಆಗುವುದಿಲ್ಲ. ಹೀಗೆ ಅವನು ಲೋಕದ ಪಾಲಿಗೆ, ಇದ್ದೂ ಇಲ್ಲದವನಾಗಿಬಿಡುತ್ತಾನೆ. ಹಾಗೆಯೇ ನರೇಂದ್ರನೂ ತನ್ನ ಆನಂದದಲ್ಲಿ ತಾನೇ ಮುಳುಗಿದ್ದುಬಿಟ್ಟರೆ ಲೋಕಕ್ಕೆ ಅವನಿಂದ ಏನು ಪ್ರಯೋಜನವಾದೀತು? ಆದ್ದರಿಂದಲೇ ಶ್ರೀರಾಮಕೃಷ್ಣರು ಜಗನ್ಮಾತೆಯನ್ನು ಹಾಗೆ ಪ್ರಾರ್ಥಿಸಿಕೊಳ್ಳುತ್ತಿದ್ದುದು.

ಶ್ರೀರಾಮಕೃಷ್ಣರ ಶಿಕ್ಷಣವಿಧಾನ ನಿಜಕ್ಕೂ ಬಹಳ ವಿಶಿಷ್ಟವಾದದ್ದು, ಅದ್ಭುತವಾದದ್ದು. ತರುಣಶಿಷ್ಯರು ಸೇರಿ ತಮ್ಮತಮ್ಮೊಳಗೆ ಆತ್ಮತತ್ತ್ವದ ಕುರಿತಾಗಿಯೋ ವೇದಾಂತದ ಕುರಿತಾ ಗಿಯೋ ಚರ್ಚೆ ಮಾಡುವುದಿತ್ತು. ಆಗ ಶ್ರೀರಾಮಕೃಷ್ಣರು ತಾವು ಮಧ್ಯೆ ಪ್ರವೇಶ ಮಾಡದೆ ಅವರ ಮಾತುಗಳನ್ನು ಮೌನವಾಗಿ ಆಲಿಸುತ್ತಿದ್ದರು. ಆದರೆ ಆ ಶಿಷ್ಯರಿಗೆ ಗಂಟೆಗಟ್ಟಲೆ ತರ್ಕ ಮಾಡಿಯೂ ತಿಳಿಯಲಾಗದ ವಿಚಾರವನ್ನು ಅವರು ಕೇವಲ ಒಂದೇ ಮಾತಿನಲ್ಲಿ ಇಲ್ಲವೆ ಒಂದು ಹಾಡಿನ ಮೂಲಕ, ಅಥವಾ ತಮ್ಮ ನೋಟಮಾತ್ರದಿಂದ ತಿಳಿಸಿಕೊಟ್ಟುಬಿಡುತ್ತಿದ್ದರು. ಅವರ ಬೋಧನೆಯ ಪಲ್ಲವಿ ಮಾತ್ರ ಇದೊಂದೇ: ‘ಭಗವತ್ಸಾಕ್ಷಾತ್ಕಾರವೇ ಜೀವನದ ಗುರಿ; ಸಾಕ್ಷಾತ್ಕಾರವಾಯಿತೆಂದರೆ ಈ ತರ್ಕ-ವಿತರ್ಕಗಳೆಲ್ಲ ನಿಂತುಹೋಗುತ್ತವೆ; ಹೃದಯದಲ್ಲಿ ಜ್ಞಾನ ಜ್ಯೋತಿ ಪ್ರಕಾಶವಾಗುತ್ತದೆ.’

ಒಂದು ದಿನ ನರೇಂದ್ರನೂ ಇತರ ಶಿಷ್ಯರೂ ಸೇರಿ ಬಿಸಿಬಿಸಿ ವಾದ-ವಿವಾದದಲ್ಲಿ ತೊಡಗಿ ದ್ದಾರೆ. ‘ಭಗವಂತ ಸಾಕಾರನೆ, ನಿರಾಕಾರನೆ? ಭಗವಂತ ಅವತಾರ ತಾಳುವುದು ನಿಜವೆ ಅಥವಾ ಈ ಅವತಾರಗಳೆಲ್ಲ ಕಟ್ಟುಕಥೆಗಳೆ?’–ಇದು ವಾದದ ವಿಷಯ. ಆಗಿನ್ನೂ ನರೇಂದ್ರ ಕಟ್ಟಾ ನಿರಾಕಾರವಾದಿಯಾಗಿದ್ದ. ಅವತಾರತತ್ತ್ವವನ್ನು ಹೀಗಳೆಯುತ್ತಿದ್ದ. ಆದರೆ ಉಳಿದ ತರುಣರೆಲ್ಲ ಸಹಜವಾಗಿಯೇ ಭಕ್ತಿಪ್ರಧಾನ ಸ್ವಭಾವದವರು. ಅಲ್ಲದೆ, ಶ್ರೀರಾಮಕೃಷ್ಣರ ಬೋಧನೆಗಳಲ್ಲಿ ವಿಶ್ವಾಸ ತಳೆದಿದ್ದವರು. ಎಲ್ಲರೂ ಬಹಳ ಬುದ್ಧಿವಂತ ಯುವಕರೇ. ಆದ್ದರಿಂದ ವಾದ ಒಳ್ಳೇ ರಭಸದಿಂದ ಸಾಗಿತ್ತು. ಶಾಸ್ತ್ರಗಳ, ಸಿದ್ಧಾಂತಗಳ ಆಧಾರದ ಮೇಲೆ ವಾದ ನಡೆಯಿತು. ಶಾಸ್ತ್ರವಾಕ್ಯ ಗಳೂ ಮುಗಿದುವು; ಸಿದ್ಧಾಂತದ ಉದಾಹರಣೆಗಳೂ ಮುಗಿದುವು. ಕೊನೆಗೆ ವಾದದಲ್ಲಿ ಗೆದ್ದವನು ನರೇಂದ್ರನೇ. ನಾಲ್ಕಾರು ಜನ ಸೇರಿಯೂ ಅವನನ್ನು ಸೋಲಿಸಲಾಗಲಿಲ್ಲ. ಅವನು ಉಳಿದೆಲ್ಲರ ವಾದಗಳನ್ನು ಕತ್ತರಿಸಿಹಾಕಿಬಿಟ್ಟ. ಆ ಹೊತ್ತಿಗೆ ಸರಿಯಾಗಿ ಶ್ರೀರಾಮಕೃಷ್ಣರು ಭಾವಸ್ಥಿತಿಯಲ್ಲಿ ಅಲ್ಲಿಗೆ ಬಂದರು; ಬರುತ್ತಿದ್ದ ಹಾಗೆಯೇ ಅವರೆಲ್ಲರ ವಾದ-ವಿವಾದಗಳಿಗೆ ಉತ್ತರವೋ ಎಂಬಂತೆ, ಸಂತ ರಾಮಪ್ರಸಾದನ ಹಾಡೊಂದನ್ನು ಭಾವಪೂರ್ಣವಾಗಿ ಹಾಡ ತೊಡಗಿದರು:

\begin{verse}
 ದೇವದೇವನ ನಿಜವನರಿಯಲು ಮನವು ತೊಳಲುತ ಬಳಲಿದೆ;\\
 ಬೀಗಮುದ್ರೆಯನಿಟ್ಟ ಕೋಣೆಯೊಳಲೆವ ಮನುಜನ ತೆರನಿದೆ!\\
 ದಿವ್ಯಪ್ರೇಮಕೆ ದೊರೆವನವನು, ಶ್ರದ್ಧೆಗಲ್ಲದೆ ಒಲಿಯನು\\
 ವೇದ-ಶಾಸ್ತ್ರ-ಪುರಾಣ-ದರ್ಶನದಾಚೆಗೇ ನಿಂತಿರುವನು!\\
 ಭಕ್ತಿಗೊಲಿಯುವ ಹೃದಯದಮೃತಾನಂದರೂಪನು ಎಂಬರು;\\
 ಇದನರಿತೆ ಆ ಯೋಗಿವರ್ಯರು ಯುಗಯುಗವು ತಪಗೈದರು!\\
 ಭಕ್ತಿ ಎಚ್ಚರಗೊಳಲು ಎದೆಯಲಿ ಅವನೆ ನಿನ್ನನು ಸೆಳೆವನು\\
 ಈ ರಹಸ್ಯವ ಜಗದ ಸಂತೆಯ ಜನಸಮೂಹಕೆ ತಿಳಿಸೆನು!\\
 ಶ್ರೀಪ್ರಸಾದನು ನುಡಿವನೀ ತೆರ–‘ಮಾತೃಭಾವದಿ ನೆನೆವೆನು;\\
 ನನ್ನ ಸೂಚನೆಯರಿತು ನೀವೇ ತಿಳಿಯಿರಾತನ ನಿಜವನು’
\end{verse}

ಈ ಹಾಡನ್ನು ಕೇಳುತ್ತಲೇ ಶಿಷ್ಯರೆಲ್ಲ ಆನಂದೋದ್ವೇಗದಿಂದ ಸ್ತಬ್ಧರಾಗಿಬಿಟ್ಟರು. ಆ ಹಾಡು ಅವರೆಲ್ಲರ ವಾದ ವಿವಾದದ ಕಾವನ್ನು ಆರಿಸಿಬಿಟ್ಟಿತು. ವಾದ ಮಾಡುವುದು ಮನಸ್ಸು-ಬುದ್ಧಿಗಳು. ಆದರೆ ಭಗವಂತನಿರುವುದು ಮನಸ್ಸು ಬುದ್ಧಿಗಳಾಚೆಗೆ. ಉಪನಿಷತ್ತು ಘೋಷಿಸುತ್ತದೆ– ‘ಯತೋ ವಾಚೋ ನಿವರ್ತಂತೇ ಅಪ್ರಾಪ್ಯ ಮನಸಾ ಸಹ... ’ ಮಾತಿಗೆ ಅತೀತನಾದವನನ್ನು ಮಾತಿನಿಂದ ವರ್ಣಿಸುವುದು ಹೇಗೆ? ಮನಸ್ಸಿಗೂ ಮೀರಿದವನನ್ನು ಮನಸ್ಸಿನಿಂದ ಅರಿತುಕೊಳ್ಳು ವುದು ಹೇಗೆ? ಆದ್ದರಿಂದಲೇ ಶ್ರೀರಾಮಕೃಷ್ಣರು ಹಾಡುತ್ತಾರೆ:

‘(ಮನವು)ಬೀಗಮುದ್ರೆಯನಿಟ್ಟ ಕೋಣೆಯೊಳಲೆವ ಮನುಜನ ತೆರನಿದೆ!’ 

ಭಗವಂತ ಶಾಸ್ತ್ರಗಳಲ್ಲೂ ಇಲ್ಲ, ತಂತ್ರಗಳಲ್ಲೂ ಇಲ್ಲ, ವೇದಗಳಲ್ಲೂ ಇಲ್ಲ. ಅವನಿರುವುದು ದಿವ್ಯ ಪ್ರೇಮವಿರುವಲ್ಲಿ, ಶುದ್ಧ ಭಕ್ತಿಯಿರುವಲ್ಲಿ. ಅಲ್ಲದೆ,

‘ಭಕ್ತಿಯೊಚ್ಚರಗೊಳಲು ಎದೆಯಲಿ ಅವನೆ ನಿನ್ನನು ಸೆಳೆವನು...!’

ಹಾಡಿನ ಭಾವವನ್ನು ಗ್ರಹಿಸಿದ ನರೇಂದ್ರಾದಿಗಳು ಒಮ್ಮೆಗೇ ಮೂಕರಾಗಿಬಿಟ್ಟರು. ಅವರೆಲ್ಲರ ಪ್ರಶ್ನೆ-ಪರಿಪ್ರಶ್ನೆಗಳಿಗೆ ಸರಿಯಾದ ಉತ್ತರ ಸಿಕ್ಕಿಬಿಟ್ಟಿತ್ತು.

ನಿಜಕ್ಕೂ ಭಗವಂತನ ವಿಷಯವನ್ನು ತಿಳಿಸಹೊರಟಾಗ ಶ್ರೀರಾಮಕೃಷ್ಣರು ಉಪಯೋಗಿಸು ತ್ತಿದ್ದ ಭಾಷೆಯೆಂದರೆ ‘ಸಾಕ್ಷಾತ್ಕಾರದ ಭಾಷೆ.’ ಉಳಿದವರೆಲ್ಲ ಭಗವಂತನ ಅಸ್ತಿತ್ವದ ಕುರಿತಾಗಿ ಶಾಸ್ತ್ರಗಳ ಭಾಷೆಯಿಂದ ವಿಶ್ಲೇಷಿಸಿ ವಿವರಿಸಹೊರಟರೆ, ಶ್ರೀರಾಮಕೃಷ್ಣರು ನೇರವಾದ ತಮ್ಮ ಈ ಸಾಕ್ಷಾತ್ಕಾರದ ಭಾಷೆಯಲ್ಲಿ ಹೇಳಿಬಿಡುತ್ತಾರೆ. ಉಳಿದವರು ಗಂಟೆಗಟ್ಟಲೆ ಶಾಸ್ತ್ರವಾಕ್ಯಗಳನ್ನು ಉದಾಹರಿಸಿ ವಿವರಿಸಿದರೂ ಅರ್ಥವಾಗದಿದ್ದುದು, ಶ್ರೀರಾಮಕೃಷ್ಣರ ಒಂದೇ ಒಂದು ವಾಕ್ಯದಿಂದ ಅಥವಾ ಹಾಡಿನಿಂದ ಸ್ಪಷ್ಟವಾಗಿಬಿಟುತ್ತದೆ! ಏಕೆಂದರೆ ಅವರು ಕೇವಲ ಶಾಸ್ತ್ರಪಂಡಿತರಲ್ಲ, ದ್ರಷ್ಟಾರರು; ಭಗವಂತನನ್ನು ಮುಖಾಮುಖಿಯಾಗಿ ಕಂಡವರು. ಹಾಗಿರುವಾಗ ಅವರು ತರ್ಕ-ವಿತರ್ಕಗಳ ಮೂಲಕ ಆತನ ಬಗ್ಗೆ ತಿಳಿಸಿಕೊಡಬೇಕಾದ ಆವಶ್ಯಕತೆಯೇನಿದೆ? ಕೂದಲೆಳೆ ಯನ್ನು ಸೀಳುವಂತಹ ವ್ಯರ್ಥ ವಾದಗಳ ಮರೆಹೊಗಬೇಕಾದ ಅಗತ್ಯವೇನಿದೆ? ಪಂಚಾಂಗವನ್ನು ಹಿಂಡಿದರೆ ಮಳೆ ಬರುತ್ತದೆಯೆ? ಹಾಗೆಯೇ, ಶಾಸ್ತ್ರವಾಕ್ಯಗಳನ್ನು ಬಿಡಿಸಿ ನೋಡಿದರೆ ದೇವರು ಸಿಗುತ್ತಾನೆಯೆ? ಶ್ರೀರಾಮಕೃಷ್ಣರು ಕೂಡ ಬುದ್ಧದೇವನಂತೆಯೇ ಶಾಸ್ತ್ರ-ಸಿದ್ಧಾಂತಗಳನ್ನು ಬಗೆದುನೋಡುವಂತಹ ತರ್ಕ ವಿತರ್ಕಗಳಿಗೆ ಗಮನ ಕೊಡುವವರಲ್ಲ. ಅಲ್ಲದೆ, ನಿಜವಾದ ಆಧ್ಯಾತ್ಮಿಕತೆಯೆಂಬುದು ದೊಡ್ಡ ದೊಡ್ಡ ಸಿದ್ಧಾಂತಗಳನ್ನು ರಮ್ಯವಾದ ಶಬ್ದಗಳ ಮೂಲಕ ವರ್ಣಿಸುವುದರಲ್ಲಿಲ್ಲ; ಅಥವಾ ಶಾಸ್ತ್ರವ್ಯಾಖ್ಯಾನ ಕೌಶಲದಲ್ಲಿಲ್ಲ. ನಿಜವಾದ ಆಧ್ಯಾತ್ಮಿಕತೆಯಿರು ವುದು ಭಗವತ್ಸಾಕ್ಷಾತ್ಕಾರದಲ್ಲಿ. ಆದ್ದರಿಂದ ಶಿಷ್ಯರು ತರ್ಕ ವಿತರ್ಕಗಳಲ್ಲಿ ತೊಡಗಿದಾಗಲೆಲ್ಲ ಶ್ರೀರಾಮಕೃಷ್ಣರು ಅವರ ಆಲೋಚನಾಪಥವನ್ನು ಅಲ್ಲಿಂದ ತಪ್ಪಿಸಿ ಸತ್ಯಸಾಕ್ಷಾತ್ಕಾರದೆಡೆಗೆ ತಿರುಗಿಸುತ್ತಾರೆ–ಹೇಗೆ ಹೋರಾಡಿ ಸಾಧನೆ ಮಾಡಬೇಕು, ಹೇಗೆ ಸಂಯಮದಿಂದಿದ್ದು ತಪಸ್ಸನ್ನಾ ಚರಿಸಬೇಕು, ಹೇಗೆ ಭಗವದ್ದರ್ಶನಕ್ಕಾಗಿ ವ್ಯಾಕುಲಿತರಾಗಬೇಕು ಎಂಬುದರ ಕಡೆಗೆ ಅವರ ಗಮನ ಸೆಳೆಯುತ್ತಾರೆ. ತರ್ಕ ವಿತರ್ಕಗಳು ತುಂಬ ಮುಂದುವರಿದು ರಭಸ ಹೆಚ್ಚಾದಂತೆ ಕಂಡುಬಂದರೆ ಶ್ರೀರಾಮಕೃಷ್ಣರು ಅಸಹನೆಗೊಂಡು, “ಇನ್ನು ಈ ಮಾತೆಲ್ಲ ಇಷ್ಟಕ್ಕೇ ಸಾಕು” ಎನ್ನುತ್ತಿದ್ದರು. ಮಣಗಟ್ಟಲೆ ಮಾತನಾಡುವ ಪಂಡಿತರನ್ನು ಅವರು ರಣಹದ್ದಿಗೆ ಹೋಲಿಸುತ್ತಿದ್ದರು. “ರಣಹದ್ದು ಬಹಳ ಎತ್ತರದಲ್ಲೇನೋ ಹಾರಾಡುತ್ತಿರುತ್ತದೆ. ಆದರೆ ಅದರ ದೃಷ್ಟಿ ಮಾತ್ರ ಕೆಳಗೆ ನೆಲದ ಮೇಲಿರುವ ಕೊಳೆತ ಶವದ ಮೇಲೆಯೇ. ಹಾಗೆಯೇ ಈ ಪಂಡಿತರು ಶಾಸ್ತ್ರತತ್ತ್ವಗಳ ದೊಡ್ಡ ದೊಡ್ಡ ಮಾತುಗಳನ್ನಾಡಿದರೂ ಅವರ ಗಮನ ಇರುವುದೆಲ್ಲ ಹೆಸರು-ಕೀರ್ತಿ, ಕಾಮ-ಕಾಂಚನಗಳ ಕಡೆಗೆ” ಎನ್ನುತ್ತಿದ್ದರು.

ಇನ್ನೆಷ್ಟೋ ಸಲ ಶ್ರೀರಾಮಕೃಷ್ಣರ ಮೌನವೇ ಅತ್ಯಂತ ಸ್ಪಷ್ಟವಾದ ಉಪದೇಶವಾಗಿ ಪರಿಣಮಿ ಸುತ್ತಿತ್ತು. ಕೆಲವೊಮ್ಮೆ ಶಿಷ್ಯರು ತಮ್ಮ ವಾದ-ವಿವಾದಗಳನ್ನೆಲ್ಲ ಮುಗಿಸಿ ನೋಡಿದರೆ, ಅವರು ಸಮಾಧಿಸ್ಥರಾಗಿಬಿಟ್ಟಿರುತ್ತಿದ್ದರು! ಇದೇನಿದು, ಇವರು ಹೀಗೆ ಸಮಾಧಿಮಗ್ನರಾದುದಕ್ಕೆ ಕಾರಣ ವೇನಿರಬಹುದು ಎಂದು ಆಲೋಚಿಸಿದಾಗ ತಿಳಿಯುತ್ತಿತ್ತು–ಇದು ತಮ್ಮ ವ್ಯರ್ಥ ತರ್ಕ-ವಿತರ್ಕ ಗಳ ವಿರುದ್ಧ ಶ್ರೀರಾಮಕೃಷ್ಣರ ಪ್ರತಿಭಟನೆ ಎಂದು! ‘ನೋಡಿ, ನೀವು ಯಾವ ಭಗವಂತನ ವಿಷಯವಾಗಿ ಇಷ್ಟೆಲ್ಲ ಮಾತನಾಡುತ್ತಿದ್ದೀರೋ, ಅಂಥ ಭಗವಂತನನ್ನು ಅರಿಯಬೇಕಾದದ್ದು ಮೌನದ ಮೂಲಕ; ಅವನನ್ನು ಅರಿತು ಆನಂದಿಸಬೇಕಾದದ್ದು ಹೀಗೆ ಸಮಾಧಿಸ್ಥಿತಿಯಲ್ಲಿ– ನೋಡಿ!’ ಎಂದು ಅವರು ಹೇಳುತ್ತಿರುವಂತಿತ್ತು.

ಆದರೆ ವಾದ-ವಿವಾದಗಳ ಉಪಯುಕ್ತತೆಯನ್ನೂ ಶ್ರೀರಾಮಕೃಷ್ಣರು ಚೆನ್ನಾಗಿಯೇ ಅರಿತಿ ದ್ದರು. ವಿಚಾರಶಕ್ತಿ ಹರಿತವಾಗಲು ಮತ್ತು ವಿಷಯಸ್ಪಷ್ಟತೆಗಾಗಿ ಕೆಲವೊಮ್ಮೆ ವಾದ ಮಾಡಬೇಕಾ ಗುತ್ತದೆ. ಆದ್ದರಿಂದಲೇ ಅವರು ಸಾಮಾನ್ಯತಃ ಶಿಷ್ಯರ ಚರ್ಚೆಗಳಲ್ಲಿ ನಡುವೆ ಪ್ರವೇಶಿಸುತ್ತಿರಲಿಲ್ಲ, “ಹುಡುಗರು ಚರ್ಚೆ ಮಾಡಲಿ; ಹಾಗೆ ಮಾಡುವುದರಿಂದ ಚೆನ್ನಾಗಿ ಕಲಿಯುತ್ತಾರೆ” ಎನ್ನುತ್ತಿ ದ್ದರು. ಅಲ್ಲದೆ ಕೆಲವು ಸಲ ತಾವೂ ಅವರ ಚರ್ಚೆಯನ್ನು ಕೇಳಿ ಆನಂದಿಸುತ್ತಿದ್ದರು. ಶಿಷ್ಯರು ಚರ್ಚೆ ಮಾಡುವ ರೀತಿಯನ್ನು ನೋಡಿಯೇ ಅವರ ಆಧ್ಯಾತ್ಮಿಕತೆಯ ಆಳವನ್ನು ಕಂಡುಕೊಳ್ಳು ತ್ತಿದ್ದರು. ನಿಜಕ್ಕೂ ಶ್ರೀರಾಮಕೃಷ್ಣರ ಸಾನ್ನಿಧ್ಯವೇ ಒಂದು ಶಿಕ್ಷಣಕೇಂದ್ರ ಎನ್ನಬಹುದು; ಒಂದು ಉನ್ನತ ಆಧ್ಯಾತ್ಮಿಕ ವಿದ್ಯಾಶಾಲೆ ಎನ್ನಬಹುದು. ಶ್ರೀರಾಮಕೃಷ್ಣರ ಸಮರ್ಥ ಮಾರ್ಗದರ್ಶನದಲ್ಲಿ ಎಲ್ಲರೂ ತಮ್ಮತಮ್ಮ ಜೀವನಾದರ್ಶವನ್ನು ಕಂಡುಕೊಳ್ಳಲು ಅನುಕೂಲಿಸುವಂತಿತ್ತು ಆ ವಾತಾ ವರಣ. ಇನ್ನು ಕೆಲವೊಮ್ಮೆ ಶ್ರೀರಾಮಕೃಷ್ಣರು ಶಿಷ್ಯರ ಚರ್ಚೆಯನ್ನು ಗಮನಿಸಿ, ಅವರ ತಪ್ಪು ಗ್ರಹಿಕೆ, ತಪ್ಪು ವಾದಗಳನ್ನು ತೋರಿಸಿಕೊಡುತ್ತಿದ್ದರು. ಎಷ್ಟೋ ಸಲ ನರೇಂದ್ರ ತನ್ನ ಭಾರತೀಯ ಹಾಗೂ ಪಾಶ್ಚಾತ್ಯ ತತ್ತ್ವಶಾಸ್ತ್ರಗಳ ಜ್ಞಾನದ ಬಲದಿಂದ, ತನ್ನ ವಾಗ್ವೈಖರಿಯಿಂದ, ತನ್ನ ಬಲಯುತವಾದ ಆಲೋಚನಾ ಶಕ್ತಿಯಿಂದ ಎಂಥವರನ್ನೂ ಸುಲಭವಾಗಿ ಬಾಯಿಮುಚ್ಚಿಸಿ ಬಿಡುತ್ತಿದ್ದ. ಅಂತಹ ಸಂದರ್ಭಗಳಲ್ಲಿ ಶ್ರೀರಾಮಕೃಷ್ಣರೂ ಮಧ್ಯೆ ಪ್ರವೇಶ ಮಾಡುತ್ತಿದ್ದರು. ಒಂದು ದಿನ ಶ್ರದ್ಧೆಯ ಕುರಿತಾಗಿ ಚರ್ಚೆಯೆದ್ದಿತು. ಮುಕ್ತಿ ಗಳಿಸಬೇಕಾದರೆ ಶ್ರದ್ಧೆಯೇ ಸಾಧನ ಎಂದು ಇತರರೆಲ್ಲ ವಾದಿಸಿದರೆ, ನರೇಂದ್ರ ಹೇಳುತ್ತಾನೆ, “ಕೇವಲ ಶ್ರದ್ಧೆಯಿಂದೇನಾದೀತು? ಮುಕ್ತಿ ಪಡೆಯಲು ಬೇಕಾದದ್ದು ಶ್ರದ್ಧೆಯಲ್ಲ, ಜ್ಞಾನ. ಶ್ರದ್ಧೆ ಎನ್ನುವುದು ಬರೀ ಕುರುಡು–ಶ್ರದ್ಧೆ ಯೆಂದರೆ ಅಂಧವಿಶ್ವಾಸ” ಎಂದು. ಆಗ ಶ್ರೀರಾಮಕೃಷ್ಣರು ಮಧ್ಯೆ ಪ್ರವೇಶಿಸಿ ನುಡಿದರು: “ಅಂಧಶ್ರದ್ಧೆ ಎಂದೆಯಲ್ಲ, ಅದೇನದು? ಶ್ರದ್ಧೆ ಯಾವಾಗಲೂ ಅಂಧವೇ! ಶ್ರದ್ಧೆಗೇನು ಕಣ್ಣಿದೆಯೆ? ಅದನ್ನು ನೀನು ‘ಅಂಧಶ್ರದ್ಧೆ’ ಎಂದು ಕರೆಯುವುದೇತಕ್ಕೆ? ಸುಮ್ಮನೆ ‘ಶ್ರದ್ಧೆ’ ಎನ್ನು; ಇಲ್ಲವೆ ‘ಜ್ಞಾನ’ ಎನ್ನು. ಅದರ ಬದಲು ಕಣ್ಣಿರುವ ಶ್ರದ್ಧೆ, ಕಣ್ಣಿಲ್ಲದ ಶ್ರದ್ಧೆ ಅಂತ ಶ್ರದ್ಧೆ ಯನ್ನೇ ವಿಭಾಗಿಸುವೆಯಲ್ಲ!” ಇದನ್ನು ಕೇಳಿ ನರೇಂದ್ರನ ಬಾಯಿ ಕಟ್ಟಿಯೇ ಹೋಯಿತು. ಒಂದು ಮಾತನ್ನೂ ಆಡಲಾರದೆ ತೆಪ್ಪಗೆ ಕುಳಿತುಬಿಟ್ಟ. ಆಲೋಚಿಸಿದಂತೆಲ್ಲ ಅವನಿಗೆ ‘ಶ್ರದ್ಧೆ ಯಾವಾಗಲೂ ಅಂಧವೇ; ಶ್ರದ್ಧೆ ಎನ್ನು, ಇಲ್ಲವೆ ಜ್ಞಾನ ಎನ್ನು’–ಎನ್ನುವ ಮಾತಿನ ಸತ್ಯತೆ ನಿಚ್ಚಳವಾಗತೊಡಗಿತು. ಹೀಗೆ ಶ್ರೀರಾಮಕೃಷ್ಣರು ಶ್ರದ್ಧೆಯ ನಿಜಸ್ವರೂಪದ ಬಗ್ಗೆ ಅವನಲ್ಲಿ ಹೊಸ ಅರಿವನ್ನೇ ಉಂಟುಮಾಡಿದರು.

ಶ್ರೀರಾಮಕೃಷ್ಣರ ದಿವ್ಯ ಶಿಕ್ಷಣದಿಂದ ನರೇಂದ್ರನಿಗೆ, ನಿಧಾನವಾಗಿಯಾದರೂ ನಿಶ್ಚಯವಾಗಿ, ಅರ್ಥವಾಗತೊಡಗಿತು–ನಿಜವಾದ ಧರ್ಮ ಎಂದರೆ ಅದು ಸಾಕ್ಷಾತ್ಕಾರ; ಮನುಷ್ಯನಾದವನು ಭಗವಂತನನ್ನು ಕಾಣಲೇಬೇಕು ಎಂದು. ಮನುಷ್ಯನು ಭಗವಂತನ ಕುರಿತಾಗಿ ಆಲೋಚನೆ, ತರ್ಕ ವಿತರ್ಕ ಮಾಡುವುದು ಒಳ್ಳೆಯದೇ; ಆದರೆ ಆತನನ್ನು ಪ್ರತ್ಯಕ್ಷವಾಗಿ ಕಾಣುವುದೇ ಅತ್ಯುನ್ನತ ವಾದದ್ದು, ಪರಮಶ್ರೇಷ್ಠವಾದದ್ದು ಎಂಬ ವಿಷಯ ಅವನಿಗೆ ಮನದಟ್ಟಾಯಿತು. ಭಗವಂತನ ಸಾಕ್ಷಾತ್ಕಾರಕ್ಕೆ ಪ್ರೀತಿಪೂರ್ವಕ ಸಹನೆಯಿಂದ ಕೂಡಿದ ದೀರ್ಘಕಾಲದ ಸಾಧನೆ ಅತ್ಯಾವಶ್ಯಕ ಎಂಬುದೂ ಅರಿವಾಯಿತು.

ಶಿಷ್ಯರು ಶ್ರೀರಾಮಕೃಷ್ಣರನ್ನು ಕೇಳುವುದಿತ್ತು–“ಭಗವಂತನ ಸಾಕ್ಷಾತ್ಕಾರಕ್ಕೆ ಉಪಾಯವೇನು, ಯಾವ ಮಾರ್ಗ ಶ್ರೇಷ್ಠವಾದದ್ದು?” ಎಂದು. ಅದಕ್ಕೆ ಶ್ರೀರಾಮಕೃಷ್ಣರು ಒಂದು ಸುಲಭ ವಿಧಾನವನ್ನು ಸೂಚಿಸುತ್ತಿದ್ದರು: “ಅವನನ್ನೇ ಪ್ರಾರ್ಥನೆ ಮಾಡಿಕೊಳ್ಳಿ!” ಎಂದು. “ಪ್ರಾರ್ಥನೆ ಮಾಡುವುದು ಹೇಗೆ?” ಎಂಬ ಶಿಷ್ಯರ ಪ್ರಶ್ನೆಗೆ ಅವರು ಉತ್ತರಿಸುತ್ತಿದ್ದರು: “ನೀವು ಹೇಗೆ ಬೇಕಾದರೂ ಪ್ರಾರ್ಥನೆ ಮಾಡಿಕೊಳ್ಳಬಹುದು. ನಿಮಗೆ ಯಾವ ರೀತಿ ಇಷ್ಟವಾಗುತ್ತದೆಯೋ, ಸರಿಹೊಂದುತ್ತದೆಯೋ ಆ ರೀತಿಯಲ್ಲೇ ಪ್ರಾರ್ಥಿಸಿಕೊಳ್ಳಿ. ಭಗವಂತ ಇರುವೆಯ ಹೆಜ್ಜೆಯ ಸಪ್ಪಳವನ್ನೂ ಕೇಳಿಸಿಕೊಳ್ಳಬಲ್ಲ.” ಹೀಗೆ ಸಾಧನೆಯ ಕುರಿತಾದ ಎಲ್ಲ ಪ್ರಶ್ನೆಗಳಿಗೂ ಸುಲಭ- ಸಮಾಧಾನಕರವಾಗಿ ಉತ್ತರಿಸುತ್ತಿದ್ದರು. ಪ್ರಾರ್ಥನೆ, ಧ್ಯಾನ, ಜಪ-ತಪಗಳ ಬಗ್ಗೆ ವಿಶದವಾಗಿ ತಿಳಿಸಿಕೊಡುತ್ತಿದ್ದರು. ಭಗವಂತನ ಸಾಕಾರ-ನಿರಾಕಾರ ಸಗುಣ-ನಿರ್ಗುಣ ತತ್ತ್ವಗಳ ಬಗ್ಗೆ ಹೀಗೆ ನ್ನುತ್ತಿದ್ದರು: “ಭಗವಂತ ಸಾಕಾರನೂ ಹೌದು, ನಿರಾಕಾರನೂ ಹೌದು; ಅವನು ಸಗುಣನೂ ಹೌದು, ನಿರ್ಗುಣನೂ ಹೌದು; ಮತ್ತು ಅವನು ಇವೆಲ್ಲಕ್ಕೂ ಅತೀತನೂ ಹೌದು. ಸಾಕ್ಷಾತ್ಕಾರ ವಾದಾಗ ಅವನು ಮನುಷ್ಯನ ಅಂತರಂಗದಲ್ಲಿ ವ್ಯಕ್ತನಾಗುತ್ತಾನೆ. ಅವನು ಭಕ್ತರ ಸಂತೋಷಕ್ಕಾಗಿ ಯಾವ ಆಕಾರವನ್ನಾದರೂ ತಳೆಯಬಲ್ಲ.” ವಿಗ್ರಹಪೂಜೆ ಮಾಡುವುದು ಸರಿಯೇ ತಪ್ಪೇ ಎಂಬ ಪ್ರಶ್ನೆಗೆ ಅವರೆನ್ನುತ್ತಾರೆ: “ಭಗವದ್ದರ್ಶನಕ್ಕೆ ಸಹಾಯವಾಗುವಂತಹ ಯಾವುದೇ ಬಗೆಯ ಪೂಜೆ ಯಾದರೂ ಅದು ಸರಿಯೇ. ಆದರೆ ಮುಖ್ಯವಾಗಿ ಬೇಕಾದುದು ತೀವ್ರ ವ್ಯಾಕುಲತೆ.”

ಶ್ರೀರಾಮಕೃಷ್ಣರು ಎಲ್ಲ ಬಗೆಯ ಆಧ್ಯಾತ್ಮಿಕ ಸಾಧನೆಗಳಲ್ಲೂ ನುರಿತವರು; ಎಲ್ಲ ಸಾಧನಾ ಪಥಗಳಲ್ಲೂ ಶಾಸ್ತ್ರೋಕ್ತವಾಗಿ ಸಾಧನೆ ಮಾಡಿ ಭಗವದ್ದರ್ಶನ ಮಾಡಿದವರು. ಹಿಂದೂ ಶಾಸ್ತ್ರಗಳು ನಾನಾಬಗೆಯ ಆಧ್ಯಾತ್ಮಿಕ ಸಾಧನೆಗಳನ್ನು, ಯೋಗಮಾರ್ಗಗಳನ್ನು ತಿಳಿಸಿಕೊಡುತ್ತವೆ. ಏಕೆಂದರೆ ಒಬ್ಬೊಬ್ಬನ ಅಭಿರುಚಿ, ಮನೋಭಾವ ಒಂದೊಂದು ಬಗೆ. ಅವರವರು ತಮಗಿಷ್ಟವಾದ ಮಾರ್ಗವನ್ನು ಆರಿಸಿಕೊಂಡು ಮುಂದುವರಿಯಲು ಶಾಸ್ತ್ರಗಳ ಸಮ್ಮತಿಯಿದೆ. ಆದರೆ ಈ ಶಾಸ್ತ್ರಗಳಲ್ಲಿ ತಿಳಿಸಲಾದ ಮಾರ್ಗಗಳಲ್ಲಿ ಕೆಲವು ರಾಜಮಾರ್ಗಗಳು, ಇನ್ನು ಕೆಲವು ಕಾಡು ದಾರಿಗಳು. ತಂತ್ರಸಾಧನೆ ಎಂಬ ಒಂದು ಬಗೆಯ ಆಧ್ಯಾತ್ಮಿಕ ಸಾಧನೆ ಬಂಗಾಳದಲ್ಲಿ ಹೆಚ್ಚಾಗಿ ಪ್ರಚಲಿತವಾಗಿದೆ. ಇದನ್ನು ವಕ್ರವಾದ, ಕೊಳಕಾದ ಗಲ್ಲಿ ರಸ್ತೆಗೆ ಹೋಲಿಸಬಹುದು. ಈ ಸಾಧನೆಯಲ್ಲಿ ಸ್ತ್ರೀ-ಪುರುಷರು ಜೊತೆಯಾಗಿ ಸಾಧನೆ ಮಾಡುವ ಪ್ರಣಾಳಿಯಿದೆ. ಲೈಂಗಿಕ ತೃಷ್ಣೆಯನ್ನು ಕ್ರಮೇಣ ನಿರ್ಮೂಲಗೊಳಿಸಿ, ಮನಸ್ಸನ್ನು ಭಗವಂತನಲ್ಲಿ ನೆಲೆಗೊಳಿಸುವ ಒಂದು ವಿಶಿಷ್ಟ ಪ್ರಕ್ರಿಯೆಯನ್ನು ಇದರಲ್ಲಿ ತಿಳಿಸಿಕೊಡಲಾಗಿದೆ. ಆದರೆ ಬಂಗಾಳದಲ್ಲಿ ಈ ಸಾಧನೆಯನ್ನು ಅವಲಂಬಿಸಿದವರಲ್ಲಿ ಅತಿ ಹೆಚ್ಚಿನ ಸಂಖ್ಯೆಯ ಸಾಧಕರು ಕೇವಲ ಲೈಂಗಿಕತೆಯಲ್ಲೇ ಮುಳುಗಿರು ತ್ತಾರೆ. ಒಟ್ಟಿನಲ್ಲಿ, ಈ ಸಾಧನೆಯನ್ನು ಗಂಭೀರವಾಗಿ ಪರಿಗಣಿಸಿದವರೂ ಕೂಡ ಬೇಗನೆ ದಾರಿತಪ್ಪಿ ಭ್ರಷ್ಟರಾಗುವ ಸಂಭವವೇ ಹೆಚ್ಚು. ಆದ್ದರಿಂದ ತಂತ್ರಸಾಧನೆಯೆಂದರೆ ನಾಗರಿಕರಿಗೆ ಹೇವರಿಕೆ; ಅಂತಹ ‘ಸಾಧಕ’ರನ್ನು ಕಂಡರೆ ಸಿಡಿಮಿಡಿ. ತರುಣ ನರೇಂದ್ರನಿಗಂತೂ ಇಂತಹ ಭಾವನೆ ಇನ್ನಷ್ಟು ತೀವ್ರ. ಇದು ಶ್ರೀರಾಮಕೃಷ್ಣರಿಗೂ ಗೊತ್ತು. ಆದರೆ ಅವರು ಸ್ವಯಂ ಈ ಬಗೆಯ ಸಾಧನೆಯನ್ನು ಪರಿಶುದ್ಧ ರೂಪದಲ್ಲಿ ನಡೆಸಿ ಯಶಸ್ವಿಯಾದವರು. (ವಿವರಗಳಿಗೆ ನೋಡಿ: ಯುಗಾವತಾರ ಶ್ರೀರಾಮಕೃಷ್ಣ, ಸಂ. ೧.) ಅಲ್ಲದೆ ಅವರ ದೃಷ್ಟಿಕೋನ ಅತ್ಯಂತ ವಿಶಾಲ; ಇತರೆಲ್ಲರಿಗಿಂತ ವಿಭಿನ್ನ.

ಒಮ್ಮೆ ಸಂಭಾಷಣೆಯ ಸಂದರ್ಭದಲ್ಲಿ ತಂತ್ರಸಾಧನೆಯ ಪ್ರಸ್ತಾಪ ಬಂದಿತು. ಆಗ ಶ್ರೀರಾಮ ಕೃಷ್ಣರು ನರೇಂದ್ರನಿಗೆ ಅದರ ಮರ್ಮವನ್ನು ವಿವರಿಸುತ್ತ ಹೇಳುತ್ತಾರೆ: “ಈ ತಂತ್ರಸಾಧಕರಿಗೆ ತಮ್ಮ ಸಾಧನೆಯಲ್ಲಿ ಸರಿಯಾದ ರೀತಿಯಲ್ಲಿ ಮುನ್ನಡೆಯಲು ಸಾಧ್ಯವಾಗುವುದಿಲ್ಲ. ಅವರಲ್ಲಿ ಹೆಚ್ಚಿನವರೆಲ್ಲ ಸಾಧನೆಯ ಹೆಸರಿನಲ್ಲಿ ತಮ್ಮ ಕೀಳು ಕಾಮನೆಗಳನ್ನು ತೃಪ್ತಿಪಡಿಸಿಕೊಳ್ಳುತ್ತಾರೆ ಅಷ್ಟೆ. ಆದರೆ ನೋಡು ನರೇನ್, ನೀನು ಈ ವಿಷಯಗಳನ್ನೆಲ್ಲ ತಿಳಿದುಕೊಂಡು ತಲೆಕೆಡಿಸಿಕೊಳ್ಳ ಬೇಕಾಗಿಲ್ಲ. ನನ್ನ ಮಾರ್ಗವೇನು ಗೊತ್ತೆ? ನನ್ನದು ಸಂತಾನಭಾವ. ಸಮಸ್ತ ಸ್ತ್ರೀಯರನ್ನೂ ಜಗನ್ಮಾತೆಯ ಸ್ವರೂಪರೆಂದು ತಿಳಿಯುವವನು ನಾನು. ಇದು ಅತ್ಯಂತ ಪರಿಶುದ್ಧವಾದ ಮನೋ ಭಾವ. ಇದರಲ್ಲಿ ಅಪಾಯಕ್ಕೆ ಅವಕಾಶವೇ ಇಲ್ಲ. ಸ್ತ್ರೀಯನ್ನು ಸೋದರಿ ಎಂದು ತಿಳಿಯುವುದ ರಲ್ಲೂ ತಪ್ಪಿಲ್ಲ. ಆದರೆ ಇನ್ನುಳಿದ ಭಾವನೆಗಳು ಬಹಳ ಅಪಾಯಕರ. ತಂತ್ರಶಾಸ್ತ್ರದ ನಾಯಕ- ನಾಯಕೀಭಾವದ ಸಾಧನೆಯ ಮೂಲಕ ಧ್ಯೇಯವನ್ನು ಮುಟ್ಟುವುದು ತೀರಾ ಕಷ್ಟದ ಕೆಲಸ. ಭಗವಂತನನ್ನು ಸೇರಲು ನಾನಾ ಮಾರ್ಗಗಳಿವೆ. ಈ ತಂತ್ರ ಸಾಧನೆಯ ಮಾರ್ಗಗಳು ಮಾತ್ರ ಜಾಡಮಾಲಿ ಪ್ರವೇಶಮಾಡುವ ಹಿತ್ತಲ ದಾರಿಯಂತೆ. ಮನೆಯನ್ನು ಮುಂಭಾಗದಲ್ಲಿರುವ ಹೆಬ್ಬಾಗಿಲ ದಾರಿಯಾಗಿ ಪ್ರವೇಶ ಮಾಡುವುದೇ ಒಳ್ಳೆಯದಲ್ಲವೆ?”

ಹೀಗೆನ್ನುತ್ತ ಉನ್ನತ ಭಾವದಲ್ಲಿ ಹೇಳತೊಡಗಿದರು:

“ನೋಡು, ನಾನಾ ಮತಗಳಿವೆ; ನಾನಾ ಮಾರ್ಗಗಳಿವೆ. ಆದರೆ ಅವು ಯಾವುವೂ ನನಗೀಗ ರುಚಿಸಲಾರವು. ಈ ಭಿನ್ನ ಭಿನ್ನ ಮಾರ್ಗಗಳ ಅನುಯಾಯಿಗಳು ತಮ್ಮತಮ್ಮಲ್ಲಿ ಹೊಡೆದಾಡಿ ಕೊಳ್ಳುತ್ತಾರೆ. ನೀವು ನನ್ನ ಸ್ವಂತದವರು; ಇಲ್ಲಿ ಪರಕೀಯರು ಯಾರೂ ಇಲ್ಲ. ನಾನೊಂದು ವಿಷಯವನ್ನು ಹೇಳುತ್ತೇನೆ ಕೇಳಿ: ನಾನು ಸ್ಪಷ್ಟವಾಗಿ ಕಾಣುತ್ತಿದ್ದೇನೆ–ಅವನು ಪ್ರಭು, ನಾನವನ ದಾಸ ಎಂದು. ಇನ್ನು ಕೆಲವು ಸಲ ಕಾಣುತ್ತೇನೆ–ಭಗವಂತ ಪೂರ್ಣ, ನಾನು ಅವನ ಅಂಶ ಎಂದು. ಮತ್ತೆ ಕೆಲವೊಮ್ಮೆ ಕಾಣುತ್ತೇನೆ–ನಾನೇ ಅವನು, ಅವನೇ ನಾನು ಎಂದು.”

ಹೀಗೆ ದ್ವೈತ, ವಿಶಿಷ್ಟಾದ್ವೈತ, ಅದ್ವೈತ–ಈ ಮೂರು ತತ್ತ್ವಗಳೂ ಸಮಾನವಾಗಿ ಸತ್ಯ ಎಂದು ಸೂಚಿಸುತ್ತಿದ್ದಾರೆ, ಶ್ರೀರಾಮಕೃಷ್ಣರು. ನರೇಂದ್ರ ಈ ಎಲ್ಲ ಮಾತುಗಳನ್ನೂ ಗಮನವಿಟ್ಟು ಕೇಳುತ್ತಿದ್ದ. ಅವರ ಈ ದಿವ್ಯ, ಅನುಭವಪೂರ್ಣ ವಾಕ್ಯಗಳು ಅವನ ಮೇಲೆ ಅತೀವ ಪ್ರಭಾವ ಬೀರಿದುವು. ತಾನು ಬ್ರಾಹ್ಮಸಮಾಜದಲ್ಲಿ ಕಲಿತದ್ದೆಲ್ಲ ಎಷ್ಟು ಸಂಕುಚಿತ ದೃಷ್ಟಿಯದು ಎನ್ನು ವುದು ಈಗ ಅವನಿಗೆ ಅರ್ಥವಾಗತೊಡಗಿತ್ತು. ಕ್ರೈಸ್ತ, ಬ್ರಾಹ್ಮ ಮೊದಲಾದ ಸಿದ್ಧಾಂತಗಳಲ್ಲಿ ‘ಪಾಪ’ ‘ಪಾಪವಿಮೋಚನೆ’ ಇಂತಹ ತತ್ತ್ವಗಳಿಗೆ ಬಹುಮುಖ್ಯ ಸ್ಥಾನ. ‘ಮಾನವನು ಹುಟ್ಟಿ ನಿಂದಲೇ ಪಾಪಿ, ಆದರೆ ಅವನು ಪಾಪವನ್ನು ಕಳೆದುಕೊಂಡು ಪರಿಶುದ್ಧನಾಗಬೇಕು’ ಎಂದು ಅವುಗಳ ಬೋಧನೆ. ಆದರೆ ಶ್ರೀರಾಮಕೃಷ್ಣರು ‘ಪಾಪ’ ಎಂಬ ಪದವನ್ನು ಕೇಳಲೂ ಇಷ್ಟಪಡು ತ್ತಿರಲಿಲ್ಲ ಅವರ ಶಬ್ದಕೋಶದಲ್ಲಿ ‘ಜನ್ಮತಃ ಪಾಪಿ’ ಎಂಬ ಮಾತೇ ಇರಲಿಲ್ಲ. ಮಾನವನಿಗೆ ಹುಟ್ಟಿನಿಂದಲೇ ಕೆಲವು ಇತಿಮಿತಿಗಳಿರುತ್ತವೆ, ದೌರ್ಬಲ್ಯಗಳಿರುತ್ತವೆ ಎಂಬುದನ್ನು ಅವರು ಒಪ್ಪುತ್ತಿದ್ದರು. ಆದರೆ ಮಾನವನ ಈ ಇತಿಮಿತಿಗಳನ್ನೇ ಒತ್ತಿ ಹೇಳುವವರಲ್ಲ. ಈ ಇತಿಮಿತಿ ಗಳನ್ನು, ಬಂಧನಗಳನ್ನು ಮೀರಿ ನಡೆಯುವುದೇ ಮಾನವಜನ್ಮದ ಗುರಿ ಎಂಬುದನ್ನು ಅವರು ತೋರಿಸಿಕೊಡುತ್ತಿದ್ದರು. ಹೀಗೆ ಅವರು ತಮ್ಮ ಶಿಷ್ಯರ ಮನೋಭಾವವನ್ನು ಸಂಪೂರ್ಣವಾಗಿ ಪರಿವರ್ತಿಸುತ್ತಿದ್ದರು.

ಒಂದು ದಿನ ನರೇಂದ್ರ, ಮಹೇಂದ್ರನಾಥನ ಮುಂದೆ ಕಾಲೇಜು ಯುವಕರ ದುರಾಚಾರಗಳ ವಿಚಾರವಾಗಿ ಟೀಕೆ ಮಾಡುತ್ತಿದ್ದ. ಅದನ್ನು ಕೇಳಿಸಿಕೊಂಡ ಶ್ರೀರಾಮಕೃಷ್ಣರು ಒಡನೆಯೇ ಹೇಳುತ್ತಾರೆ: “ಇಂತಹ ಅಲ್ಪ ವಿಚಾರಗಳ ಬಗ್ಗೆ ಏಕೆ ಮಾತನಾಡುತ್ತೀಯೆ? ಭಗವಂತನ ವಿಚಾರವಾಗಿ ಮಾತ್ರ ಮಾತನಾಡು. ಬೇರೆ ಯಾವ ವಿಷಯವೂ ಬೇಡ.” ಆಗಾಗಲೇ ಕಂಡ ತಪ್ಪುಗಳನ್ನು ಅಲ್ಲಲ್ಲೇ ತಿದ್ದಿ ಸರಿಪಡಿಸುವ ಅವರ ಶಿಕ್ಷಣಕೌಶಲವನ್ನು ಇಲ್ಲಿ ನಾವು ಕಾಣಬಹುದಾಗಿದೆ.

ಶ್ರೀರಾಮಕೃಷ್ಣರ ಬೋಧನೆಯ ಮರ್ಮವನ್ನು, ಅವರ ಮಾತುಗಳ ಅಂತರಾರ್ಥವನ್ನು ಗ್ರಹಿಸುವಲ್ಲಿ ನರೇಂದ್ರ ಮಿಕ್ಕೆಲ್ಲ ಶಿಷ್ಯರಿಗಿಂತ ಹೆಚ್ಚು ಪ್ರವೀಣ. ಅವರ ದಿವ್ಯವಾಣಿಯ ಆಧ್ಯಾ ತ್ಮಿಕ ಭಾವಗಳಿಗೆ ಅವನ ಹೃದಯ ಸರಿಸಮನಾಗಿ ಸ್ಪಂದಿಸುತ್ತಿತ್ತು. ಶ್ರೀರಾಮಕೃಷ್ಣರ ಜೀವನ ವೆಂಬುದು ಅವರ ಸಾಕ್ಷಾತ್ಕಾರಗಳಿಂದ ಹಾಗೂ ಸಂದೇಶಗಳಿಂದ ಕೂಡಿದ ಒಂದು ಭವ್ಯ ಗ್ರಂಥದಂತೆ. ಅವರ ಇತರ ಶಿಷ್ಯರು ಅದರ ಎಲ್ಲೋ ಒಂದೊಂದು ಪುಟವನ್ನು ಓದಿ ಅರ್ಥಮಾಡಿಕೊಳ್ಳುತ್ತಿದ್ದರೆ, ನರೇಂದ್ರ ಆ ಗ್ರಂಥದ ಸಂಪುಟ ಸಂಪುಟಗಳನ್ನೇ ಅರಗಿಸಿಕೊಳ್ಳು ತ್ತಿದ್ದ. ಅವನ ಭಾವಗ್ರಹಣ ಸಾಮರ್ಥ್ಯ ಅಷ್ಟು ತೀಕ್ಷ್ಣ. ಒಮ್ಮೆ ಶ್ರೀರಾಮಕೃಷ್ಣರು ತಮ್ಮ ಕೋಣೆಯಲ್ಲಿ ನರೇಂದ್ರಾದಿ ಶಿಷ್ಯರೊಂದಿಗೆ ಕುಳಿತಿದ್ದಾರೆ; ವೈಷ್ಣವ ಸಿದ್ಧಾಂತದ ವಿಷಯವಾಗಿ ಸಂಭಾಷಣೆ ನಡೆಯುತ್ತಿದೆ. ಶ್ರೀರಾಮಕೃಷ್ಣರು ಚೈತನ್ಯ ಮಹಾಪ್ರಭುವಿನ ಸಿದ್ಧಾಂತದ ರೂಪರೇಖೆ ಗಳ ಕುರಿತಾಗಿ ಹೇಳುತ್ತಿದ್ದಾರೆ:

“ಚೈತನ್ಯನ ಬೋಧನೆಯ ಪ್ರಕಾರ ಅವನ ಅನುಯಾಯಿಗಳು ಮೂರು ಮುಖ್ಯವಾದ ಅಂಶ ಗಳನ್ನು ಅಭ್ಯಾಸ ಮಾಡಬೇಕು: ಒಂದು–ಭಗವನ್ನಾಮದಲ್ಲಿ ಅಭಿರುಚಿ; ಎರಡು–ವೈಷ್ಣವ ಭಕ್ತರ ಸೇವೆ; ಮೂರು–ಜೀವರಲ್ಲಿ ದಯೆ. ಭಗವಂತ ಮತ್ತು ಅವನ ಹೆಸರು ಬೇರೆಬೇರೆಯಲ್ಲ; ಆದ್ದರಿಂದ ಸದಾ ಅವನ ನಾಮವನ್ನು ಜಪಿಸುತ್ತಿರಬೇಕು. ಅಂತೆಯೇ ಭಗವಂತ ಮತ್ತು ಭಕ್ತ, ಎಂದರೆ ವಿಷ್ಣು ಮತ್ತು ವೈಷ್ಣವ–ಇವರಲ್ಲಿ ಪರಸ್ವರ ಭೇದವಿಲ್ಲ; ಆದ್ದರಿಂದ ಪ್ರತಿಯೊಬ್ಬ ವೈಷ್ಣವನನ್ನೂ ಪೂಜ್ಯಬುದ್ಧಿಯಿಂದ ಗೌರವಿಸಿ ಸತ್ಕರಿಸಬೇಕು. ಇನ್ನು ಈ ಇಡೀ ಜಗತ್ತೇ ಭಗವಂತನಿಗೆ ಸೇರಿರುವುದರಿಂದ ಸಕಲ ಜೀವರಲ್ಲೂ ದಯೆ ತೋರಬೇಕು.”

ಹೀಗೆ ನಿಧಾನವಾಗಿ ಎಲ್ಲರಿಗೂ ಅರ್ಥವಾಗುವಂತೆ ವಿವರಿಸುತ್ತಿದ್ದ ಶ್ರೀರಾಮಕೃಷ್ಣರು, ‘ಸಕಲ ಜೀವರಲ್ಲಿ ದಯೆ’ ಎಂಬ ಈ ಮಾತನ್ನು ಉಚ್ಚರಿಸುತ್ತಿದ್ದಂತೆಯೇ ಸಮಾಧಿಸ್ಥರಾಗಿಬಿಟ್ಟರು. ಶಿಷ್ಯರೆಲ್ಲ, ಅದಕ್ಕೆ ಕಾರಣವೇನೆಂದು ತಿಳಿಯದೆ ಮೌನವಾಗಿ ನೋಡುತ್ತಿದ್ದಾರೆ. ಸ್ವಲ್ಪ ಹೊತ್ತಾದ ಬಳಿಕ ಶ್ರೀರಾಮಕೃಷ್ಣರು ಅರ್ಧಭಾವಾವಸ್ಥೆಗೆ ಇಳಿದು, ತಮ್ಮಷ್ಟಕ್ಕೇ ಎಂಬಂತೆ ಉದ್ಗರಿಸಿದರು: “ಏನು, ಜೀವರಲ್ಲಿ ದಯೆ! ಜೀವರಲ್ಲಿ ದಯೆ! ಏ ಹುಲುಮಾನವ! ಜೀವರಲ್ಲಿ ದಯೆ ತೋರಲು ನೀನು ಯಾರು! ಇಲ್ಲ, ಅದು ಸಾಧ್ಯವಿಲ್ಲ. ಜೀವರಿಗೆ ದಯೆ ತೋರುವುದಲ್ಲ. ಜೀವರಿಗೆ ನೀನು ಮಾಡಬೇಕಾದದ್ದು ಸೇವೆ. ಜೀವನಲ್ಲಿ ಶಿವನನ್ನು ಕಂಡು ಸೇವೆ ಮಾಡಬೇಕು. ಜೀವನಲ್ಲ ಶಿವ; ಜೀವನಲ್ಲ ಶಿವ!”

ಶ್ರೀರಾಮಕೃಷ್ಣರು ಅರ್ಧಭಾವಾವಸ್ಥೆಯಲ್ಲಿ ಆಡಿದ ಈ ಕೆಲವು ಮಾತುಗಳನ್ನು ಅಲ್ಲಿ ಕುಳಿತಿ ದ್ದವರೆಲ್ಲರೂ ಕೇಳಿಸಿಕೊಂಡದ್ದೇನೋ ನಿಜವೇ. ಆದರೆ ಆ ಸ್ಫೂರ್ತಿಯುತ ನುಡಿಗಳ ಗೂಢಾರ್ಥ ವನ್ನು ಪೂರ್ಣವಾಗಿ ಗ್ರಹಿಸಿದವನು ನರೇಂದ್ರನೊಬ್ಬನೇ. ಬಳಿಕ ಕೋಣೆಯಿಂದ ಹೊರಬಂದ ಮೇಲೆ ಅವನು ತಾನು ಕಂಡುಕೊಂಡ ವಿನೂತನ ವಿಚಾರಗಳನ್ನು ಇತರರ ಮುಂದೆ ಉತ್ಸಾಹದಿಂದ ಬಣ್ಣಿಸುತ್ತಾನೆ.

“ಆಹ್! ಇಂದು ನಾನು ಶ್ರೀರಾಮಕೃಷ್ಣರ ಮಾತಿನಲ್ಲಿ ಅದ್ಭುತವಾದ ಒಂದು ಹೊಸ ಬೆಳಕನ್ನು ಕಂಡುಕೊಂಡೆ. ವೇದಾಂತವು ಬೋಧಿಸುವ ಜ್ಞಾನದ ತತ್ತ್ವದೊಂದಿಗೆ ಭಕ್ತಿಯ ಆದರ್ಶವನ್ನು ಅವರು ಎಷ್ಟು ಸುಮಧುರವಾಗಿ ಸಮನ್ವಯಗೊಳಿಸಿದರು! ವೇದಾಂತದ ತತ್ತ್ವಗಳೆಲ್ಲ ಶುಷ್ಕ; ಪ್ರೀತಿ ಅನುಕಂಪೆಯೇ ಮೊದಲಾದ ಮಾನವಸಹಜ ಭಾವನೆಗಳು ಆ ತತ್ತ್ವಗಳಿಗೆ ಒಗ್ಗುವುದಿಲ್ಲ ಎಂಬುದು ಸಾಮಾನ್ಯವಾದ ನಂಬಿಕೆ. ಇತರರು ಅವುಗಳನ್ನು ವ್ಯಾಖ್ಯಾನ ಮಾಡುವಾಗಲೂ ಅದು ನಿಜವೆಂದೇ ಭಾಸವಾಗುತ್ತದೆ. ಆದರೆ ಶ್ರೀರಾಮಕೃಷ್ಣರು ನೀಡಿದ ಅರ್ಥ ಎಷ್ಟು ಭವ್ಯವಾಗಿದೆ, ಎಷ್ಟು ಸಹಜ-ಸಮಂಜಸವಾಗಿದೆ, ನೋಡಿ! ಪ್ರಚಲಿತವಿರುವ ಭಾವನೆಯೇನೆಂದರೆ, ಜ್ಞಾನಯೋಗ ವನ್ನು ಅಭ್ಯಾಸ ಮಾಡಬೇಕಾದರೆ ಸಮಾಜದಿಂದ ದೂರವಿರಬೇಕು, ಪ್ರೇಮ ಭಕ್ತಿ ಅನುಕಂಪೆ ಮೊದಲಾದ ಭಾವನೆಗಳನ್ನೆಲ್ಲ ಬೇರುಸಹಿತ ಕಿತ್ತುಹಾಕಬೇಕು ಎಂದು. ವೇದಾಂತವನ್ನು ಜನ ಅರ್ಥ ಮಾಡಿಕೊಂಡಿರುವ ರೀತಿ ಹೀಗೆ. ಸಾಧಕನಾದವನೂ ಹೀಗೆಯೇ ಅಪಾರ್ಥ ಮಾಡಿಕೊಂಡು, ಈ ಜಗತ್ತು-ಜನಗಳು ತನ್ನ ಸಾಧನೆಗೆ ತೊಡಕಾಗಿರುವ ಕಂಟಕಗಳು ಎಂಬ ಭಾವನೆಯಿಂದ ಅವುಗಳನ್ನೆಲ್ಲ ತಿರಸ್ಕರಿಸುತ್ತಾನೆ. ಆದರೆ ಶ್ರೀರಾಮಕೃಷ್ಣರು ದಿವ್ಯಭಾವದಲ್ಲಿ ನುಡಿದ ಆ ವಿವೇಕ ವಾಣಿಯನ್ನು ಕೇಳಿದಾಗ ನನಗೆ ಅರ್ಥವಾಯಿತು–ಸಾಧಕರೆನ್ನಿಸಿಕೊಂಡವರು ಸಮಾಜದಿಂದ ದೂರವಿದ್ದುಕೊಂಡು ಸರ್ವಸಂಗ ಪರಿತ್ಯಾಗ ಮಾಡಿ ಯಾವ ಆದರ್ಶಗಳನ್ನು ಅಭ್ಯಾಸ ಮಾಡು ತ್ತಾರೋ ಅದೇ ಆದರ್ಶಗಳನ್ನು ಅವರು ತಮ್ಮ ಮನೆಯಲ್ಲಿಯೇ ಇದ್ದುಕೊಂಡು ಅಭ್ಯಾಸ ಮಾಡಬಹುದು, ತಮ್ಮ ನಿತ್ಯಜೀವನದಲ್ಲಿಯೇ ಆಚರಣೆಗೆ ತರಬಹುದು ಎಂದು. ಒಬ್ಬ ಮನುಷ್ಯ ಯಾವ ಕ್ಷೇತ್ರದಲ್ಲೇ ಇರಲಿ, ಯಾವ ಉದ್ಯೋಗವನ್ನೇ ಕೈಗೊಂಡಿರಲಿ, ಅವನು ತಿಳಿದಿರಬೇಕಾದ ಒಂದು ಸತ್ಯವೆಂದರೆ ಸ್ವಯಂ ಭಗವಂತನೇ ಈ ಜಗತ್ತಾಗಿ ಮತ್ತು ಜೀವಕೋಟಿಗಳಾಗಿ ವ್ಯಕ್ತ ಗೊಂಡಿದ್ದಾನೆ ಎಂದು. ಭಗವಂತ ಈ ಜಗತ್ತಿನಲ್ಲಿ ಅಂತರ್ಗತನಾಗಿಯೂ ಇದ್ದಾನೆ, ಜಗತ್ತಿಗೆ ಅತೀತನಾಗಿಯೂ ಇದ್ದಾನೆ. ನಾವು ಪ್ರೀತಿ ತೋರಿಸುವ, ಗೌರವಿಸುವ ಅಥವಾ ಸೇವೆ ಸಲ್ಲಿಸುವ ಈ ಜೀವಕೋಟಿಗಳು ಸ್ವಯಂ ಭಗವಂತನ ವ್ಯಕ್ತ ರೂಪ, ಮತ್ತು ಹೀಗಿದ್ದರೂ ಅವನು ಇವೆಲ್ಲಕ್ಕೂ ಅತೀತನೂ ಹೌದು. ಈ ಒಂದು ತತ್ತ್ವವನ್ನು ಪ್ರತಿಯೊಬ್ಬನೂ ಮನದಟ್ಟು ಮಾಡಿ ಕೊಳ್ಳಬೇಕು. ಜನರಲ್ಲಿ ಜನಾರ್ದನನನ್ನು ಕಂಡಾಗ, ಜೀವರಲ್ಲಿ ಶಿವನನ್ನು ಕಂಡಾಗ ನಮ್ಮಲ್ಲಿ ಅಹಂಕಾರ ಉಳಿದುಕೊಳ್ಳಲು ಅವಕಾಶವಾಗುವುದಿಲ್ಲ. ಏಕೆಂದರೆ ಭಗವಂತನ ಮುಂದೆಯೇ ನಮ್ಮ ಅಹಂಕಾರವನ್ನು ತೋರಿಸಲು ಸಾಧ್ಯವೆ? ಮತ್ತು ನಾವು ಯಾರ ಬಗ್ಗೆಯೂ ದ್ವೇಷಾಸೂಯೆ ತಾಳಲೂ ಸಾಧ್ಯವಿಲ್ಲ. ಪ್ರೇಮಸ್ವರೂಪನಾದ ಭಗವಂತನ ಕುರಿತಾಗಿ ದ್ವೇಷಾಸೂಯೆಗಳನ್ನು ತಾಳಲು ಸಾಧ್ಯವೆ? ಅಂತೆಯೇ ಈ ಪರಮಸತ್ಯವನ್ನು ಅರಿತಾಗ, ಮನುಷ್ಯ ಇತರರ ಮೇಲೆ ‘ದಯೆ’ ತೋರುವ ಧಿಮಾಕು ಮಾಡುವುದಿಲ್ಲ. ಭಗವಂತನ ಮೇಲೆಯೇ ದಯೆ ತೋರುವು ದೆಂದರೇನರ್ಥ! ಆದ್ದರಿಂದ ಜೀವರಲ್ಲಿ ದೇವನಿದ್ದಾನೆ ಎಂದು ಅರಿವಾದಾಗ ಸಾಧಕನು ಜೀವರ ‘ಸೇವೆ’ ಮಾಡುತ್ತಾನೆ, ತನ್ಮೂಲಕ ತನ್ನ ಹೃದಯವನ್ನು ಪರಿಶುದ್ಧಗೊಳಿಸಿಕೊಳ್ಳುತ್ತಾನೆ. ಹೀಗೆ ಭಾವದಿಂದ ಸೇವೆ ಮಾಡುವ ಸಾಧಕ ಬಹುಬೇಗ ಸಾಕ್ಷಾತ್ಕಾರ ಮಾಡಿಕೊಳ್ಳುತ್ತಾನೆ.”

ಶ್ರೀರಾಮಕೃಷ್ಣರ ದಿವ್ಯವಾಣಿಯನ್ನು ಕೇಳಿ ತನ್ನಲ್ಲುದಿಸಿದ ಭಾವನೆಗಳ ಮಹಾಪೂರವನ್ನೇ ಹರಿಸುತ್ತ ನರೇಂದ್ರ ಮುಂದುವರಿಸುತ್ತಾನೆ:

“ನಿಜಕ್ಕೂ ಶ್ರೀರಾಮಕೃಷ್ಣರ ಈ ಮಾತುಗಳು ಭಕ್ತಿಮಾರ್ಗದ ಮೇಲೆ ಒಂದು ಹೊಸ ಬೆಳಕನ್ನು ಬೀರುತ್ತವೆ. ಭಗವಂತ ಪ್ರತಿಯೊಬ್ಬರಲ್ಲೂ ಅಂತರ್ಯಾಮಿಯಾಗಿದ್ದಾನೆ ಎಂಬ ಸತ್ಯವನ್ನು ಮನ ಗಾಣುವವರೆಗೆ ನಿಜವಾದ ಭಕ್ತಿಯೆಂಬುದು ಬಹುದೂರದ ಮಾತು. ಸರ್ವಜೀವರಲ್ಲೂ ದೇವ ನಿದ್ದಾನೆ ಎಂಬ ಭಾವದಿಂದ ಸೇವೆ ಸಲ್ಲಿಸಿದಾಗ ಸಾಧಕನಲ್ಲಿ ನಿಜವಾದ ಭಕ್ತಿ, ನಿಜವಾದ ನಿಷ್ಠೆ ಮೂಡುತ್ತವೆ. ಕರ್ಮಯೋಗಿಗಳಾದರೂ ಸರಿಯೆ. ಜ್ಞಾನಯೋಗಿಗಳಾದರೂ ಸರಿಯೆ–ಶ್ರೀರಾಮ ಕೃಷ್ಣರ ಈ ಮಾತುಗಳನ್ನು ಅರ್ಥಮಾಡಿಕೊಂಡಲ್ಲಿ ಇಬ್ಬರಿಗೂ ಆಧ್ಯಾತ್ಮಿಕ ಉನ್ನತಿಯಾಗುವು ದರಲ್ಲಿ ಸಂದೇಹವಿಲ್ಲ. ಮಾನವಶರೀರವನ್ನು ಧರಿಸಿ ಬಂದ ಯಾರಿಂದಲೂ ಕ್ಷಣಕಾಲವಾದರೂ ಕರ್ಮ ಮಾಡದೆ ಸಮ್ಮನಿರಲು ಸಾಧ್ಯವಿಲ್ಲ. ಆದ್ದರಿಂದ ನಮ್ಮ ಚಟುವಟಿಕೆಯನ್ನೆಲ್ಲ ಮಾನವ ರೂಪದಿಂದ ಓಡಾಡುತ್ತಿರುವ ಭಗವಂತನ ಸೇವೆಗೆ ಮೀಸಲಾಗಿಡಬೇಕು. ಇದರಿಂದ ಆಧ್ಯಾತ್ಮಿಕ ಪ್ರಗತಿ ವೇಗಗೊಳ್ಳುತ್ತದೆ. ಭಗವಂತನ ಇಚ್ಛೆಯಿದ್ದರೆ, ನಾನು ಈ ಘನಸತ್ಯವನ್ನು ಸಮಸ್ತ ಜಗತ್ತಿಗೆ ಸಾರುವ ದಿನವೊಂದು ಶೀಘ್ರದಲ್ಲೇ ಒದಗಿಬಂದೀತು! ನಾನು ಈ ಮಹಾಸತ್ಯವನ್ನು ಸರ್ವರ ಸ್ವತ್ತಾಗುವಂತೆ ಮಾಡುತ್ತೇನೆ. ಬುದ್ಧಿವಂತರು-ಮೂರ್ಖರು, ಶ್ರೀಮಂತರು-ದರಿದ್ರರು, ಬ್ರಾಹ್ಮಣರು-ಅಸ್ಪೃಶ್ಯರು–ಎಲ್ಲರೂ ಈ ಸತ್ಯವನ್ನು ಅರಿಯುವಂತೆ ಮಾಡುತ್ತೇನೆ.”

ಶ್ರೀರಾಮಕೃಷ್ಣರ ಒಂದೆರಡು ಮಾತುಗಳಲ್ಲಿ ಇಷ್ಟೆಲ್ಲ ಗೂಢಾರ್ಥ ಅಡಗಿದೆ ಎಂಬುದನ್ನು ಮನಗಾಣುತ್ತ ಮತ್ತು ನರೇಂದ್ರನ ಆವೇಶಭರಿತ ವಾಗ್ಧಾರೆಯನ್ನು ಆಲಿಸುತ್ತ ಸಹಶಿಷ್ಯರೆಲ್ಲ ವಿಸ್ಮಯಮೂಕರಾಗಿ ನಿಂತರು. ನರೇಂದ್ರನು ಶ್ರೀರಾಮಕೃಷ್ಣರ ಉದ್ಗಾರದಲ್ಲಿ ಅಷ್ಟೊಂದು ಅರ್ಥವನ್ನು ಕಂಡುಕೊಂಡದ್ದು ಒಂದು ವಿಚಾರವಾದರೆ, ಈ ಘನಸತ್ಯವನ್ನು ತಾನು ಜಗತ್ತಿ ನಾದ್ಯಂತ ಪ್ರಸಾರ ಮಾಡುವ ಮುನ್ಸೂಚನೆ ಕೊಡುತ್ತಿರುವುದು ಇನ್ನೊಂದು ಮುಖ್ಯಾಂಶ. ಭವಿಷ್ಯದ ವಿವೇಕಾನಂದರಾಗಿ, ಶ್ರೀರಾಮಕೃಷ್ಣರ ರಾಯಭಾರಿಯಾಗಿ ಅವನು ಹೇಗೆ ರೂಪುಗೊಳ್ಳು ತ್ತಿದ್ದಾನೆ ಎಂಬುದನ್ನು ನಾವಿಲ್ಲಿ ಗಮನಿಸಬಹುದಾಗಿದೆ.

ಶ್ರೀರಾಮಕೃಷ್ಣರ ಜೀವನ-ಸಂದೇಶಗಳೆರಡೂ ಸನಾತನ ಹಿಂದೂ ಧರ್ಮದ ತಿರುಳೇ ಆಗಿವೆ. ಈ ಸನಾತನ ಧರ್ಮವೆಂಬುದು ಸರ್ವ ಭಾವಗಳನ್ನೂ ಒಳಗೊಂಡ ಸಮಗ್ರ ದೃಷ್ಟಿಯ ಧರ್ಮ. ಅದರ ಆದರ್ಶಗಳು ಸಾರ್ವಕಾಲಿಕವಾದವುಗಳು. ಹಿಂದೂ ಧರ್ಮ ಯಾವ ಮತ-ಪಂಥವನ್ನೂ ತಿರಸ್ಕರಿಸುವುದಿಲ್ಲ; ಬದಲಾಗಿ ಪ್ರತಿಯೊಂದಕ್ಕೂ ಅದರದರ ಸ್ಥಾನವನ್ನು ಕೊಟ್ಟು ಪುರಸ್ಕರಿಸು ತ್ತದೆ. ಶ್ರೀರಾಮಕೃಷ್ಣರ ಜೀವನ ಮತ್ತು ಬೋಧನೆಗಳು ಇಂತಹ ವಿಶಾಲ ಭಾವದ ಸನಾತನ ಧರ್ಮವನ್ನು ಪ್ರತಿನಿಧಿಸುತ್ತವೆ. ಆದ್ದರಿಂದ ನರೇಂದ್ರ ಅವರ ಸಂಪರ್ಕಕ್ಕೆ ಬಂದಮೇಲೆ ಹಿಂದೂ ಆಚಾರ-ವಿಚಾರಗಳೆಲ್ಲದರಲ್ಲೂ ಒಂದು ಹೊಸ ಬೆಳಕನ್ನೆ ಕಾಣಲಾರಂಭಿಸಿದ್ದ. ಬಾಲ್ಯದಿಂದಲೇ ಅವನಿಗೆ ಹಿರಿಯರಿಂದ ಹಿಂದೂಧರ್ಮದ ಸಿದ್ಧಾಂತಗಳ, ಆಚಾರಗಳ ಪರಿಚಯವಾಗಿತ್ತು ಎಂಬುದನ್ನು ನಾವು ಬಲ್ಲೆವು. ಆಗ ಅವುಗಳನ್ನೆಲ್ಲ ಅವನು ಮರುಪ್ರಶ್ನೆಯೇ ಇಲ್ಲದೆ ಸ್ವೀಕರಿಸಿದ್ದ. ಆದರೆ ದೊಡ್ಡವನಾಗುತ್ತ ಬುದ್ಧಿ ಬೆಳೆದಂತೆಲ್ಲ ಅವುಗಳನ್ನು ಪ್ರಶ್ನಿಸಲಾರಂಭಿಸಿದ. ಬುದ್ಧಿ ಜಾಗೃತವಾದಂತೆಲ್ಲ ಹೊಸಹೊಸ ಸಂದೇಹಗಳು ಮೇಲೆದ್ದಿದ್ದುವು. ಆದರೆ ಶ್ರೀರಾಮಕೃಷ್ಣರ ಸಂಪರ್ಕಕ್ಕೆ ಬಂದಾಗ, ಅವರಲ್ಲಿ ಹಿಂದೂಧರ್ಮದ ಪ್ರತಿಯೊಂದು ಆದರ್ಶವೂ ಅರ್ಥಪೂರ್ಣ ವಾಗಿ ಒಡಮೂಡಿರುವುದನ್ನು ಕ್ರಮೇಣ ಕಂಡುಕೊಂಡ. ತದನಂತರವೇ ಅವನು ಹಿಂದೂಧರ್ಮ ದಲ್ಲಿ ಮಾನವತೆಯನ್ನು ಮುನ್ನಡೆಸಬಲ್ಲ ಸಾಮರ್ಥ್ಯವಡಗಿರುವುದನ್ನು ಅರಿತದ್ದು. ಹಿಂದೂ ಧರ್ಮದಲ್ಲಿ ಇಂತಹ ಶಕ್ತಿ ಇರಬಹುದೆಂದು ಹಿಂದೆ ಆತ ಊಹಿಸಿಯೂ ಇರಲಿಲ್ಲ. ಶ್ರೀರಾಮ ಕೃಷ್ಣರ ಹಿಂದೂಧರ್ಮವೆಂದರೆ ಅದು ಸಕಾರಾತ್ಮಕವಾದದ್ದು, ಅನುಷ್ಠಾನಾತ್ಮಕವಾದದ್ದು. ನರೇಂದ್ರ ತನ್ನ ವಿಚಾರಶಕ್ತಿಯ ರಭಸದಲ್ಲಿ ಹಿಂದೂಧರ್ಮದ ಆಧ್ಯಾತ್ಮಿಕ ಆದರ್ಶಗಳನ್ನು, ಅಸಂಖ್ಯಾತ ದೇವ-ದೇವಿಯರ ಅಸ್ತಿತ್ವವನ್ನು ಎಷ್ಟೇ ಶಂಕಿಸಿರಬಹುದು, ಆದರೆ ಹಿಂದೂಧರ್ಮದ ಚೈತನ್ಯವೇ ಜ್ವಲಂತವಾಗಿ ಬೆಳಗುತ್ತಿರುವ ಶ್ರೀರಾಮಕೃಷ್ಣರ ಪ್ರಾಮಾಣಿಕತೆಯನ್ನು ಮಾತ್ರ ಅವನಿಂದ ಶಂಕಿಸಲಾಗಲಿಲ್ಲ. ಮೊದಮೊದಲು, ತಾನು ನಂಬಿಕೊಂಡಿದ್ದ ಬ್ರಾಹ್ಮಸಮಾಜವನ್ನೇ ಧರ್ಮಸಂಸ್ಥಾಪನೆ ಮಾಡಬಲ್ಲ ಒಂದು ಮಹಾಸಂಸ್ಥೆ ಎಂದು ಅವನು ಭಾವಿಸಿದ್ದ. ಆ ಬ್ರಾಹ್ಮಸಮಾಜದಲ್ಲೂ ಆಧ್ಯಾತ್ಮಿಕವಾಗಿ ಮುಂದುವರಿದ ಸಾಧಕರು ಇರಲಿಲ್ಲವೆಂದಲ್ಲ. ಆದರೆ ಯಾವಾಗ ಅವನು ಶ್ರೀರಾಮಕೃಷ್ಣರ ದಿವ್ಯ ಸಂಪರ್ಕಕ್ಕೆ ಬಂದನೊ ಆಗ ಅವನಿಗೆ ಅರ್ಥವಾಯಿತು –ಬ್ರಾಹ್ಮಸಮಾಜವೆಂಬುದು ಆಧ್ಯಾತ್ಮಿಕ ಜೀವಕಳೆಯೇ ಇಲ್ಲದ ಒಂದು ಸಾಮಾನ್ಯ ಸಂಸ್ಥೆ ಎಂದು. ಬ್ರಾಹ್ಮಸಮಾಜದ ಪ್ರಮುಖ ಶಕ್ತಿಯಾದ ಕೇಶವಚಂದ್ರ ಸೇನನೂ ಶ್ರೀರಾಮಕೃಷ್ಣರ ಪದತಲದಲ್ಲಿ ಕುಳಿತು ಆಧ್ಯಾತ್ಮಿಕತೆಯ ಪಾಠ ಕಲಿಯುವುದನ್ನು ಅವನು ಕಾಣುತ್ತಿದ್ದ.

ಅವನು ಹಿಂದೂಧರ್ಮದ ಸಾರವನ್ನು ಅರಿತದ್ದೇ ಶ್ರೀರಾಮಕೃಷ್ಣರ ಸನ್ನಿಧಿಯಲ್ಲಿ. ಅವರ ಭಾವಪೂರ್ಣವಾದ ಪೂಜೆ, ಮಾತುಕತೆ, ಬೋಧನೆ, ಅವರ ದಿವ್ಯಭಾವಗಳು–ಇವುಗಳನ್ನು ವೀಕ್ಷಿಸುವುದರ ಮೂಲಕವೇ ಅವನಿಗೆ ಹಿಂದೂ ಧರ್ಮದ ನಿಜವಾದ ಸ್ವರೂಪ ಅರ್ಥವಾದದ್ದು. ಶ್ರೀರಾಮಕೃಷ್ಣರ ಬೋಧನೆಗಳನ್ನು ಅವನೆಷ್ಟೇ ವಿರೋಧಿಸಿದರೂ ಅವು ಅವನ ಒಳಹೊಕ್ಕು ಬಿಟ್ಟುವು. ಅವರು ತಮ್ಮ ಸಾಕ್ಷಾತ್ಕಾರದ ದಿವ್ಯಾನುಭವಗಳನ್ನೆಲ್ಲ ಅವನೊಳಗೆ ಹರಿಸಿದರು. ತಮ್ಮ ಚೇತನವನ್ನೇ ಅವನೊಳಕ್ಕೆ ಸುರಿದರೆಂದರೂ ಅತಿಶಯೋಕ್ತಿಯಲ್ಲ. ಆದರೆ ಅವರಿದನ್ನು ಹೇಗೆ ಸಾಧಿಸಿದರು ಎಂಬುದು ಮಾತ್ರ ಯಾರಿಗೂ ಅರ್ಥವಾಗುವಂತಿಲ್ಲ. ಅಲ್ಲದೆ, ಇದು ವಿವರಿಸಿ ಹೇಳಲೂ ಸಾದ್ಯವಿಲ್ಲದಷ್ಟು ಸೂಕ್ಷ್ಮವಾದ ವಿಚಾರ. ಅಂತೂ ಸಂಶಯಗ್ರಸ್ತ ನರೇಂದ್ರ ನಾಶವಾಗಿ ಭಕ್ತ ನರೇಂದ್ರ ಉದಿಸಿದ; ಆಧ್ಯಾತ್ಮಿಕ ನರೇಂದ್ರ ಉದಿಸಿದ. ‘ನಾಸ್ತಿಕ’ ನರೇಂದ್ರ ಮಾಯವಾಗಿ ಆಸ್ತಿಕ ನರೇಂದ್ರನಾದ. ಅಷ್ಟೇಕೆ, ವಿಶ್ವಮಾನವ ವಿವೇಕಾನಂದನಾಗಿ ನಿರ್ಮಾಣಗೊಂಡ.

ಮೊದಮೊದಲು ನರೇಂದ್ರನಿಗೆ ತೀವ್ರವಾದ ಸಾಧನೆಗಳ ಬಗ್ಗೆ, ಮತ್ತು ತತ್ಫಲವಾಗಿ ಭಕ್ತಿ ಭಾವೋತ್ಕಟತೆಯ ಬಗ್ಗೆ ಒಂದು ಬಗೆಯ ಅನಾದರ, ಶಂಕೆ ಇದ್ದಿತು. ಈ ಭಾವಾವೇಶವೆಲ್ಲ ಆರೋಗ್ಯವಂತ ದೃಢ ಮನಸ್ಸಿನ ಲಕ್ಷಣವಲ್ಲ ಎಂಬ ನಂಬಿಕೆ ಅವನದು–ಬ್ರಾಹ್ಮಸಮಾಜದ ಸಂಪರ್ಕದ ಪ್ರಭಾವ! ಈ ತಪ್ಪು ಕಲ್ಪನೆಯನ್ನು ತಿದ್ದಲೋಸುಗ ಒಂದು ದಿನ ಶ್ರೀರಾಮಕೃಷ್ಣರು ಅವನನ್ನು ಕೇಳಿದರು: “ನೋಡು, ಭಗವಂತ ಪಾನಕದ ಸಾಗರವಿದ್ದಂತೆ. ಅದರೊಳಗೆ ಮುಳುಗಲು ನೀನು ಇಷ್ಟಪಡುವುದಿಲ್ಲವೆ? ಈಗ, ಒಂದು ದೊಡ್ಡ ಪಾತ್ರೆಯ ತುಂಬ ಪಾನಕವಿದೆ ಎಂದಿ ಟ್ಟುಕೊ. ನೀನು ಆ ಪಾನಕವನ್ನು ಹೀರಲು ತವಕಿಸುತ್ತಿರುವ ಒಂದು ನೊಣ ಎಂದು ಭಾವಿಸು. ಈಗ ಹೇಳು, ಆ ರಸವನ್ನು ಹೇಗೆ ಹೀರುತ್ತೀಯೆ?”

ನರೇಂದ್ರ: “ಪಾತ್ರೆಯ ಅಂಚಿನಲ್ಲಿ ಕುಳಿತುಕೊಂಡು ಎಚ್ಚರಿಕೆಯಿಂದ ನಿಧಾನವಾಗಿ ಹೀರುತ್ತೇನೆ.”

ಶ್ರೀರಾಮಕೃಷ್ಣರು: “ಏಕೆ, ಅದರೊಳಗೇ ಮುಳುಗಿಬಿಡುವುದಿಲ್ಲವೆ?”

ನರೇಂದ್ರ: “ಅದು ಹೇಗಾದೀತು? ಅದರೊಳಗೆ ಇಳಿದರೆ ಮುಳುಗಿ ಸತ್ತೇ ಹೋದೇನು!”

ಆಗ ಶ್ರೀರಾಮಕೃಷ್ಣರು ಅವನಿಗೆ ವಿವರಿಸುತ್ತಾರೆ: “ಓ, ಅದು ಹಾಗಲ್ಲ, ಭಗವಂತನೆಂದರೆ ಸಚ್ಚಿದಾನಂದ ಸಾಗರ, ಅಮೃತಸಾಗರ! ಅದರಲ್ಲಿ ಮುಳುಗಿದರೆ ಮರಣದ ಭಯವಿಲ್ಲ. ಭಗವಂತನನ್ನು ಪ್ರೀತಿಸುವಲ್ಲಿ ಅತಿರೇಕಕ್ಕೆ ಹೋಗಬಾರದು ಎನ್ನುವವರು ಅಜ್ಞಾನಿಗಳು, ಭಗ ವಂತನನ್ನು ಪ್ರೀತಿಸಿ, ಅದು ಅತಿರೇಕವಾಗುವುದು ಎಂಬುದೇನಾದರೂ ಇದೆಯೇನು? ಆದ್ದರಿಂದ, ನಾನು ಹೇಳುತ್ತೇನೆ ಕೇಳು–ನಿರ್ಭೀತಿಯಿಂದ ಸಚ್ಚಿದಾನಂದ ಸಾಗರದಲ್ಲಿ ಮುಳುಗಿಬಿಡು.”

ಈ ಮಾತು ನರೇಂದ್ರನಿಗೆ ತುಂಬ ಸಹಾಯಕವಾಯಿತು. ಈ ಜಗತ್ತಿನಲ್ಲಿ ಯಾವುದೇ ವಿಷಯವನ್ನು ಅನುಭವಿಸುವಾಗ ಅತಿರೇಕವಾದರೂ ಅದು ದುಃಖಕ್ಕೆ ನಾಶಕ್ಕೆ ಕಾರಣವಾಗುತ್ತದೆ. ಆಹಾರ-ಪಾನೀಯಗಳೇ ಆಗಲಿ, ಮನರಂಜನೆಯಾಗಲಿ, ಸುಖಭೋಗಗಳಾಗಲಿ ಯಾವುದು ಅತಿ ಯಾದರೂ ಕೇಡು ತಪ್ಪಿದ್ದಲ್ಲ. ಆದರೆ ಭಗವಂತನನ್ನು ಪ್ರೀತಿಸುವಲ್ಲಿ ಅತಿರೇಕವಾದರೆ ಆನಂದವೂ ಅಧಿಕವಾಗುತ್ತದೆಯೇ ಹೊರತು, ಅದರಿಂದ ದುಃಖವೊದಗುವ ಭಯವಿಲ್ಲ. ಅಂತೂ, ಈ ವಿಷಯ ಮನವರಿಕೆಯಾದ ಮೇಲೆ ಸಾಧನೆಯಲ್ಲಿ ಆಳವಾಗಿ ಮುಳುಗಲು ನರೇಂದ್ರನಿಗೆ ಹೊಸ ಶಕ್ತಿ ದೊರಕಿದಂತಾಯಿತು.

ಈ ಸಮಯದಲ್ಲಿ ಅವನಿಗೆ ಆಗಾಗ ಅನೇಕ ವಿಚಿತ್ರ ಅನುಭವಗಳಾಗುತ್ತಿದ್ದುವು. ಎಷ್ಟೋ ಸಲ ಮನೆಯಲ್ಲಿ ಕುಳಿತು ಗಾಢ ಧ್ಯಾನದಲ್ಲಿ ತೊಡಗಿದ್ದಾಗ, ತನ್ನೆದುರಿನಲ್ಲಿ ಶ್ರೀರಾಮಕೃಷ್ಣರನ್ನು ಸ್ಪಷ್ಟವಾಗಿ ಕಾಣುತ್ತಿದ್ದ. ಒಂದು ದಿನ ರಾತ್ರಿ ಕನಸಿನಲ್ಲಿ ಶ್ರೀರಾಮಕೃಷ್ಣರು ತನ್ನನ್ನು ಕರೆದು, ‘ನಾನು ನಿನಗೆ ರಾಧೆಯನ್ನು ತೋರಿಸಿಕೊಡುತ್ತೇನೆ’ ಎಂದಂತೆ ಕಂಡ. ನರೇಂದ್ರ ಅವರನ್ನು ಅನುಸರಿಸಿ ಹೋದ. ಸ್ವಲ್ಪ ದೂರ ನಡೆದ ಮೇಲೆ ಶ್ರೀರಾಮಕೃಷ್ಣರು ಇದ್ದಕ್ಕಿದ್ದಂತೆ ಹಿಂದೆ ತಿರುಗಿ, ‘ಇನ್ನೆಲ್ಲಿಗೆ ಹೋಗುತ್ತೀ ನೀನು?’ ಎಂದರು. ಹೀಗೆ ಹೇಳಿ, ತಕ್ಷಣವೇ ರಾಧೆಯಾಗಿ ಪರಿವರ್ತನೆ ಗೊಂಡು ನಿಂತರು! ತನ್ನೆದುರು ಅತಿ ಸುಂದರವಾದ ರಾಧೆಯ ರೂಪ ಕಂಗೊಳಿಸುತ್ತಿರುವುದನ್ನು ಅವನು ಕಂಡ. ಅವನ ಜಾಗೃತ ಮನಸ್ಸಿನ ಮೇಲೆ ಈ ಘಟನೆ ತೀವ್ರ ಪ್ರಭಾವ ಬೀರಿತು. ಎಷ್ಟರಮಟ್ಟಿಗೆಂದರೆ, ಹಿಂದೆ ಬ್ರಾಹ್ಮಸಮಾಜದ ನಿರಾಕಾರಬ್ರಹ್ಮ ಪರವಾದ ಹಾಡುಗಳನ್ನು ಮಾತ್ರ ಹಾಡುತ್ತಿದ್ದವನು ಈಗ ರಾಧೆಯ ಉತ್ಕಟ ಪ್ರೇಮಕ್ಕೆ ಸಂಬಂಧಿಸಿದ ಹಾಡುಗಳನ್ನು ಹಾಡಲಾರಂಭಿಸಿದ! ಈ ಸ್ವಪ್ನದ ವಿಷಯವನ್ನು ತಿಳಿದಾಗ ಸಹಶಿಷ್ಯರಿಗೆಲ್ಲ ಪರಮಾಶ್ಚರ್ಯ. “ನೀನು ಆ ಕನಸನ್ನು ನಿಜಕ್ಕೂ ನಂಬುತ್ತೀಯಾ?” ಎಂದು ಅವರಲ್ಲೊಬ್ಬರು ಕೇಳಿದಾಗ, “ಖಂಡಿತ ವಾಗಿ” ಎಂದುತ್ತರಿಸುತ್ತಾನೆ. ಅವನಂತಹ ವಿಚಾರವಾದಿ ಈ ಕನಸನ್ನು ನಿಜವೆಂದು ನಂಬಬೇಕಾ ದರೆ ಅದೆಷ್ಟು ನೈಜವಾದ ಅನುಭವವಾಗಿರಬೇಕು! ಹಿಂದೆ ರಾಧಾ-ಕೃಷ್ಣರ ಸಂಬಂಧವನ್ನು ಅಶ್ಲೀಲ ವೆಂದು ಟೀಕಿಸುತ್ತಿದ್ದ ಅವನೀಗ ಆ ಭಾವವನ್ನು ಪೂಜ್ಯ ದೃಷ್ಟಿಯಿಂದ ನೋಡಲು ಪ್ರಾರಂಭಿಸಿ ದ್ದಾನೆ. ನಿಜ, ಸ್ಪಷ್ಟವಾದ ಅನುಭವುಂಟಾದಾಗ ತಿಳಿವಳಿಕೆಯಲ್ಲಿ ಕ್ರಾಂತಿಯುಂಟಾಗುತ್ತದೆ.

ಶ್ರೀರಾಮಕೃಷ್ಣರ ಭಕ್ತರಲ್ಲಿ ನವಗೋಪಾಲ, ಮನಮೋಹನ ಮೊದಲಾದ ಕೆಲವರು ಭಗವನ್ನಾಮವನ್ನು ಉಚ್ಚರಿಸುತ್ತ ಭಾವೋನ್ಮತ್ತರಾಗಿ ನಿಶ್ಚೇಷ್ಟಿತರಾಗಿ ಬಿದ್ದಿರುತ್ತಿದ್ದುದನ್ನು ನರೇಂದ್ರ ನೋಡುತ್ತಿದ್ದ. ಕ್ರಮೇಣ ಅವನಿಗೆ, ‘ಇವರೆಲ್ಲ ಭಾಗ್ಯವಂತರು, ಇವರಿಗೆ ಅದೆಷ್ಟು ಸುಲಭವಾಗಿ ಇಂಥ ಆನಂದದ ಸ್ಥಿತಿ ಪ್ರಾಪ್ತವಾಗುತ್ತದೆ! ಆದರೆ ನನಗೆ ಮಾತ್ರ ಈ ಬಗೆಯ ಭಾವಾವಸ್ಥೆಯುಂಟಾಗುವುದಿಲ್ಲವಲ್ಲ’ ಎನ್ನಿಸಿ ದುಃಖವಾಯಿತು. ತಾನೂ ಕೂಡ ಶರೀರಪ್ರಜ್ಞೆ ಯನ್ನು ಸಂಪೂರ್ಣವಾಗಿ ಮರೆತು ಭಾವೋನ್ಮತ್ತನಾಗಿರಬೇಕು ಎಂಬ ತೀವ್ರ ಹಂಬಲವುಂಟಾ ಯಿತು. ಶ್ರೀರಾಮಕೃಷ್ಣರಲ್ಲಿ ತನ್ನ ಅಳಲನ್ನೂ ಬಯಕೆಯನ್ನೂ ತೋಡಿಕೊಂಡ. ಆಗ ಶ್ರೀರಾಮ ಕೃಷ್ಣರು, “ನರೇನ್, ಇಷ್ಟಕ್ಕೆಲ್ಲ ಚಿಂತಿಸಬೇಡ. ಅದೆಲ್ಲ ಏನು ಮಹಾ ಅಂತ ತಿಳಿದುಕೊಂಡೆ? ನೋಡು, ಒಂದು ಆನೆ ಸಣ್ಣ ಕೊಳದಲ್ಲಿ ಇಳಿದರೆ ಆ ಕೊಳದ ನೀರೆಲ್ಲ ಛಿಲ್ಲೆಂದು ಹೊರಗೆ ಹಾರುತ್ತದೆ. ಆ ಕೊಳದಲ್ಲಿ ಒಂದು ಅಲ್ಲೋಲಕಲ್ಲೋಲವೇ ಆಗಿಬಿಡುತ್ತದೆ. ಆದರೆ ಅದೇ ಆನೆ ಒಂದು ದೊಡ್ಡ ಸರೋವರದಲ್ಲಿ ಇಳಿದರೆ ಅಲ್ಲಿ ಯಾವ ದೊಡ್ಡ ಅಲ್ಲೋಲ ಕಲ್ಲೋಲವೂ ಇರುವುದಿಲ್ಲ. ಈ ಭಕ್ತರೆಲ್ಲ ಸಣ್ಣಪುಟ್ಟ ಕೆರೆಕುಂಟೆಗಳಿದ್ದ ಹಾಗೆ. ಭಗವತ್ಪ್ರೇಮದ ಮಹಾಶಕ್ತಿ ಒಂದು ಸ್ವಲ್ಪ ಅವರಲ್ಲಿ ಪ್ರವೇಶ ಮಾಡಿಬಿಟ್ಟರೂ ಸಾಕು, ಕೆರೆಕುಂಟೆಗಳಿಂದ ನೀರು ಚಿಮ್ಮುವಂತೆ ಅವರು ವರ್ತಿಸುತ್ತಾರೆ. ಆದರೆ ನೀನು ಒಂದು ಮಹಾಸರೋವರದ ಹಾಗೆ. ನಿನ್ನೊಳಗೆ ಭಾವ ಉಂಟಾದರೂ ಅದು ಹೀಗೆ ವ್ಯಕ್ತಗೊಳ್ಳುವುದಿಲ್ಲ. ಆದ್ದರಿಂದ ನೀನು ಇದಕ್ಕಾಗಿ ಕೊರಗಬೇಡ” ಎಂದು ಸಮಾಧಾನ ಹೇಳಿದರು.

ಈ ದಿನಗಳಲ್ಲಿ ನರೇಂದ್ರ ಹೇಗೆ ತ್ಯಾಗ-ವೈರಾಗ್ಯಗಳ ಮತ್ತು ಆಧ್ಯಾತ್ಮಿಕ ಶಕ್ತಿಯ ಜ್ವಲಂತ ಮೂರ್ತಿಯೇ ಆಗಿದ್ದ ಎಂಬುದಕ್ಕೆ ನಿದರ್ಶನವಾದ ಘಟನೆಯೊಂದು ನಡೆಯಿತು. ಒಮ್ಮೆ ಅವನ ಕೆಲವು ಶ್ರೀಮಂತ ಸ್ನೇಹಿತರು ವಿಹಾರದ ಕಾರ್ಯಕ್ರಮವೊಂದರಲ್ಲಿ ಭಾಗವಹಿಸಲು ತಮ್ಮ ಉದ್ಯಾನವನಕ್ಕೆ ಆಹ್ವಾನಿಸಿದರು. ನರೇಂದ್ರ ಎಂದಿನಂತೆ ಸಂತೋಷದಿಂದ ಒಪ್ಪಿ ಅವರೊಂದಿಗೆ ಹೋದ. ಆ ಶ್ರೀಮಂತ ಯುವಕರು ವಿಲಾಸಕ್ಕಾಗಿಯೇ ಅಲ್ಲಿ ನೆರೆದಿದ್ದರು. ಕೆಲವರು ಹಾಡು ಗಳನ್ನು ಹಾಡಿದರು. ನರೇಂದ್ರ ತಾನೂ ಹಾಡಿ ಗೆಳೆಯರನ್ನು ಸಂತೋಷಪಡಿಸಿದ. ಸ್ವಲ್ಪಹೊತ್ತಿನ ಮೇಲೆ ಅವನಿಗೆ ಆಯಾಸವಾಯಿತು. ಗೆಳೆಯರಿಂದ ಬೀಳ್ಗೊಂಡು ಅಲ್ಲೇ ಒಂದು ಕೋಣೆಯ ಏಕಾಂತದಲ್ಲಿ ವಿಶ್ರಮಿಸತೊಡಗಿದ. ಆಗ ಅವನ ಸ್ನೇಹಿತರು ಅವನನ್ನು ಖುಷಿಪಡಿಸಲು ಅಲ್ಲಿಗೆ ಒಬ್ಬಳು ವೇಶ್ಯೆಯನ್ನು ಕಳಿಸಿದರು. ಮುಗ್ಧ ಸ್ವಭಾವದ ನರೇಂದ್ರ ಆಕೆಯ ಉದ್ದೇಶವನ್ನರಿಯದೆ ಅವಳನ್ನು ತನ್ನ ಸ್ವಂತ ಸೋದರಿಯಂತೆ ಭಾವಿಸಿ ಮಾತನಾಡಿಸಿದ. ಆಕೆ ತನ್ನ ಜೀವನದ ನೋವು-ನಲಿವುಗಳ ಕಥೆಯನ್ನು ಅವನ ಮುಂದೆ ತೆರೆದಿಟ್ಟಳು. ಕ್ರಮೇಣ ಅವನು ತನ್ನ ಕಡೆಗೆ ಸಹಾನುಭೂತಿ ತಾಳಿ ಗಮನಕೊಟ್ಟಿದ್ದ ಸಮಯ ನೋಡಿಕೊಂಡು ಅವನನ್ನು ಆಕರ್ಷಿಸಿ ಮೋಹ ಗೊಳಿಸಲು ಯತ್ನಿಸಿದಳು. ಆಗ ನರೇಂದ್ರನಿಗೆ ಅವಳ ದುರುದ್ದೇಶ ತಿಳಿದುಹೋಯಿತು.

ತಕ್ಷಣವೇ ಅವನು ಎದ್ದುನಿಂತ; ಶ್ರೀರಾಮಕೃಷ್ಣರನ್ನು ಸ್ಮರಿಸಿಕೊಂಡ. ಬಳಿಕ ಅವಳಿಗೆ ಗಂಭೀರವಾಗಿ ಹೇಳಿದ: “ನೋಡು, ನಾನೀಗ ಹೋಗಬೇಕಾಗಿದೆ. ನಿನ್ನ ಬಗ್ಗೆ ನನಗೆ ನಿಜವಾಗಿಯೂ ಸಹಾನುಭೂತಿಯಿದೆ. ನಿನಗೆ ಒಳ್ಳೆಯದಾಗಲಿ ಎಂದು ಹಾರೈಸುತ್ತೇನೆ.” ಹೀಗೆಂದು ಕೂಡಲೇ ಅಲ್ಲಿಂದ ಹೊರಟುಬಿಟ್ಟ; ಅಪಾಯದಿಂದ ಪಾರಾದ.

ನರೇಂದ್ರನಲ್ಲಿ ಸುಪ್ತವಾಗಿದ್ದ ಅನಂತ ಶಕ್ತಿ-ಸಾಮರ್ಥ್ಯಗಳ ಮೇಲೆ ಶ್ರೀರಾಮಕೃಷ್ಣರಿಗೆ ಸಂಪೂರ್ಣ ವಿಶ್ವಾಸವಿತ್ತು. ಅವನು ಮುಂದೆ ಎಂತಹ ಬೃಹತ್ ಕಾರ್ಯಗಳನ್ನು ಸಾಧಿಸಲಿದ್ದಾನೆ ಎಂಬುದು ಅವರಿಗೆ ಚೆನ್ನಾಗಿ ತಿಳಿದಿತ್ತು. ಆ ಕುರಿತಾಗಿ ಅವರು ಆಗಾಗ ಅವನ ಮುಂದೆಯೂ ಹೇಳುತ್ತಿದ್ದರು. ಅವರ ಈ ಪ್ರೋತ್ಸಾಹದ ಮಾತುಗಳಿಂದಾಗಿ ನರೇಂದ್ರನಲ್ಲೂ ವಿಶೇಷ ಆತ್ಮ ವಿಶ್ವಾಸ ಹುಟ್ಟಿತ್ತು. ತಾನು ಈ ಭುವಿಯಲ್ಲಿ ಎಲ್ಲರಂತೆ ಬದುಕಬಂದವನಲ್ಲ ಎಂಬ ನಂಬಿಕೆ ದೃಢವಾಯಿತು. ಒಂದು ಸಲ ಅವನು ತನ್ನ ಸ್ನೇಹಿತರೊಂದಿಗೆ ಮಾತನಾಡುತ್ತ ಹೇಳುತ್ತಾನೆ: “ಏನು, ನೀವೆಲ್ಲ ಮುಂದೆ ಅತಿ ಹೆಚ್ಚೆಂದರೆ ವಕೀಲರಾದೀರಿ, ವೈದ್ಯರಾದೀರಿ ಅಥವಾ ನ್ಯಾಯಾಧೀಶ ರಾದೀರಿ, ಅಲ್ಲವೆ? ತಾಳಿ, ನನ್ನ ದಾರಿಯನ್ನು ನಾನು ಬೇರೆಯೇ ರೀತಿಯಲ್ಲಿ ರೂಪಿಸಿಕೊಳ್ಳುವು ದನ್ನು ನೀವೇ ನೋಡುವಿರಿ.”

ಶ್ರೀರಾಮಕೃಷ್ಣರ ಬಳಿಗೆ ಬರುತ್ತಿದ್ದವರಲ್ಲಿ ಎಷ್ಟೋ ಜನ ಪ್ರಖ್ಯಾತ ವೈದ್ಯರು, ದೊಡ್ಡ ದೊಡ್ಡ ವಕೀಲರು, ಮಹಾ ಪಂಡಿತರು, ಧಾರ್ಮಿಕ ಮುಖಂಡರು–ಇವರೆಲ್ಲ ಇದ್ದರು. ಆದರೆ ಶ್ರೀರಾಮ ಕೃಷ್ಣರು ಈ ಎಲ್ಲ ಭಕ್ತರಲ್ಲಿ ನರೇಂದ್ರನೇ ಅತಿ ಶ್ರೇಷ್ಠನಾದವನು ಎಂದು ಘಂಟಾಘೋಷವಾಗಿ ಸಾರುತ್ತಿದ್ದರು. ಈ ಭಕ್ತರಲ್ಲಿ ಲೌಕಿಕ ದೃಷ್ಟಿಯಿಂದ–ಸಾಮಾಜಿಕವಾಗಿ, ಅರ್ಥಿಕವಾಗಿ, ಸ್ಥಾನ ಮಾನ-ಕೀರ್ತಿಗಳಲ್ಲಿ–ತನಗಿಂತ ಎಷ್ಟೋ ಉನ್ನತ ಮಟ್ಟದವರಿದ್ದಾರೆಂಬುದು ನರೇಂದ್ರನಿಗೆ ಚೆನ್ನಾಗಿಯೇ ಗೊತ್ತಿತ್ತು. ಹೀಗಿದ್ದರೂ ಶ್ರೀರಾಮಕೃಷ್ಣರು ತನಗೆ ಅತ್ಯುನ್ನತ ಸ್ಥಾನವನ್ನಿತ್ತಿದ್ದಾರೆ ಎಂಬ ಅಂಶ ಅವನ ಹೆಚ್ಚಿನ ಆತ್ಮವಿಶ್ವಾಸಕ್ಕೆ ಕಾರಣವಾಗಿತ್ತು.

ಈ ಮಾತನ್ನು ಹೇಳಿದಾಗ ‘ಆದರೆ ಅವನು ಯಾರ ಹೊಗಳಿಕೆ-ತೆಗಳಿಕೆಗೂ ಬೆಲೆ ಕೊಡುವವನೇ ಅಲ್ಲವಲ್ಲ! ಇಂಥವನು ಶ್ರೀರಾಮಕೃಷ್ಣರ ಪ್ರಶಂಸೆಯಿಂದ ಅಷ್ಟೊಂದು ಹಿಗ್ಗಿಹೋದನೆ?’ ಎಂಬ ಪ್ರಶ್ನೆಯೊಂದು ಏಳುವ ಸಂಭವವಿದೆ. ನಿಜಕ್ಕೂ ಅವನು ಶ್ರೀರಾಮಕೃಷ್ಣರ ಮಾತಿಗೆ ಅಷ್ಟೊಂದು ಮಹತ್ವವಿತ್ತು, ಅದರಲ್ಲಿ ನಂಬಿಕೆ ತಳೆದದ್ದಕ್ಕೊಂದು ಬಲವಾದ ಕಾರಣವೇ ಇತ್ತು. ಏನದು? ಅವರ ಸ್ವಭಾವವನ್ನು ಅವನು ಬಹಳ ಚೆನ್ನಾಗಿ ಅರ್ಥಮಾಡಿಕೊಂಡಿದ್ದ. ಅವರೆಂದೂ ಲೌಕಿಕ ದೃಷ್ಟಿಯಿಂದ ವ್ಯಕ್ತಿಗಳನ್ನು ಅಳೆಯುತ್ತಿದ್ದವರಲ್ಲ. ಅವರಿಗೆ ತಮ್ಮದೇ ಆದ ಮಾನದಂಡ ವಿತ್ತು. ಒಬ್ಬ ವ್ಯಕ್ತಿಯಲ್ಲಿ ನಿಜವಾದ ಸತ್ತ್ವ ಇದೆಯೆ? ನೈತಿಕ ಬಲವಿದೆಯೆ? ಆಧ್ಯಾತ್ಮಿಕ ಶಕ್ತಿಯಿದೆಯೆ? ಹಾಗಿದ್ದರೆ ಮಾತ್ರ ಅವನೊಬ್ಬ ವ್ಯಕ್ತಿ ಎಂದು ಅವರ ಅಭಿಮತ. ಒಮ್ಮೆ ಅವರ ಬಳಿಗೆ ಒಬ್ಬ ಶ್ರೀಮಂತ ಬರುತ್ತಾನೆ. ಅವನು ಸಾಮಾನ್ಯ ಶ್ರೀಮಂತನಲ್ಲ, ಕೋಟ್ಯಧೀಶ್ವರ. ಇದರಿಂದಾಗಿ ಅವನಿಗೆ ‘ರಾಜಾ’ ಎಂಬ ಬಿರುದು ಬಂದಿತ್ತು. ಅವನಿಗೆ ಶ್ರೀರಾಮಕೃಷ್ಣರು ಹೇಳುತ್ತಾರೆ: “ನೋಡು, ಜನ ನಿನ್ನನ್ನು ‘ರಾಜಾ’ ಎಂದು ಕರೆಯಬಹುದು. ಆದರೆ ನಾನು ನಿನ್ನನ್ನು ಹಾಗೆ ಕರೆಯಲಾರೆ! ಹಾಗೆ ಕರೆದರೆ ನಾನು ಸುಳ್ಳಾಡಿದಂತಾಗುತ್ತದೆ.” ‘ರಾಜ, ಎಂಬ ಮಾತಿಗೆ ಒಂದು ಘನತೆಯಿದೆ. ಒಬ್ಬನನ್ನು ರಾಜನೆಂದು ಕರೆದಾಗ ಅವನಲ್ಲಿ ಗಾಂಭೀರ್ಯ, ತೇಜಸ್ಸು, ಬುದ್ಧಿಶಕ್ತಿ, ಹೃದಯವೈಶಾಲ್ಯ ಮೊದಲಾದ ಸದ್ಗುಣಗಳನ್ನು ನಿರೀಕ್ಷಿಸಬಹುದು; ಆದರೆ ಅವನು ಧನಿಕನೆಂಬ ಒಂದೇ ಕಾರಣದಿಂದ ಅವನನ್ನು ರಾಜನೆಂದು ಕರೆದರೆ ಅದು ಸುಳ್ಳು ಮಾತಾಗುತ್ತದೆ. ಆತ್ಮವಂಚನೆಯಾಗುತ್ತದೆ,–ಇದು ಶ್ರೀರಾಮಕೃಷ್ಣರ ಅಭಿಪ್ರಾಯ. ಹೀಗೆ ಅವರು ವ್ಯಕ್ತಿಗಳ ಗುಣ, ನಡತೆ, ಯೋಗ್ಯತೆ, ಅಂತಸ್ಸತ್ವ–ಇವುಗಳನ್ನೆಲ್ಲ ಗಮನಿಸಿ ಯಾವ ಅಭಿಪ್ರಾಯಕೊಟ್ಟರೂ ಅದು ಅಷ್ಟು ಪ್ರಾಮಾಣಿಕವಾಗಿರುತ್ತಿತ್ತು. ಆದ್ದರಿಂದ ಅವರು ಲೋಕದೃಷ್ಟಿಯಿಂದ ನೋಡಿದರೆ ಸಾಮಾನ್ಯನಂತೆ ಕಂಡು ಬರುತ್ತಿದ್ದ ಯುವಕ ನರೇಂದ್ರನನ್ನು ತಮ್ಮ ಭಕ್ತರು ಹಾಗೂ ಶಿಷ್ಯರಲ್ಲೆಲ್ಲ ಅಗ್ರಗಣ್ಯನೆಂದು ಹೇಳುತ್ತಿದ್ದುದರಲ್ಲಿ ಸಂಪೂರ್ಣ ಪ್ರಾಮಾಣಿಕತೆಯಿತ್ತು, ಸತ್ಯವಿತ್ತು.

ಶ್ರೀರಾಮಕೃಷ್ಣರ ಮಾರ್ಗದರ್ಶನದಲ್ಲಿ ಪ್ರೀತಿಪೂರ್ವಕ ಆರೈಕೆಯಲ್ಲಿ ನರೇಂದ್ರನ ವ್ಯಕ್ತಿತ್ವ ಸರ್ವಾಂಗಸುಂದರವಾಗಿ ವಿಕಸಿತವಾಗತೊಡಗಿತ್ತು. ಯುವಕ ನರೇಂದ್ರನ ಶಾರೀರಿಕ ವ್ಯಕ್ತಿತ್ವ ಪ್ರಸನ್ನ-ಗಂಭೀರ. ಮೃಗರಾಜನಾದ ಸಿಂಹದಂತೆ ಅವನ ಚಲನವಲನ. ಅವನ ಮನಸ್ಸು ಸದಾ ಒಂದಲ್ಲ ಒಂದು ಆಲೋಚನೆಯಲ್ಲಿ ಮಗ್ನ; ಆದ್ದರಿಂದ ಅವನ ನಡಿಗೆ ಒಮ್ಮೆ ವೇಗ, ಇನ್ನು ಕೆಲವೊಮ್ಮೆ ನಿಧಾನ. ಆದರೆ ಒಂದು ಬಗೆಯ ಬಾಲಸಹಜವಾದ ಲವಲವಿಕೆಯೂ ಸರಳತೆಯೂ ಅವನಲ್ಲಿತ್ತು. ಇದರಿಂದಾಗಿ ಅವನನ್ನು ಬಲ್ಲವರಿಗೆಲ್ಲ ಅವನೆಂದರೆ ಏನೋ ಸಂತೋಷ. ಸರಾಸರಿಗಿಂತ ಹೆಚ್ಚಾದ ಎತ್ತರ. ಅದಕ್ಕೆ ತಕ್ಕಂತೆ ಭುಜಗಳು ವಿಶಾಲ ಹಾಗೂ ದಷ್ಟಪುಷ್ಟ; ಸುದೃಢವಾದ ಎದೆ. ಸ್ವಲ್ಪ ಉಬ್ಬಿಕೊಂಡಂತಿದ್ದ ಮುಂದಲೆ ಅವನ ಬುದ್ಧಿಶಕ್ತಿಯನ್ನೂ ಪ್ರಬುದ್ಧತೆ ಯನ್ನೂ ಸೂಚಿಸುತ್ತಿತ್ತು. ಒಟ್ಟಿನಲ್ಲಿ ಅವನದು ಸರ್ವಾಂಗಸುಂದರವಾದ ಶರೀರ. ಅವನ ಪ್ರತಿಯೊಂದು ಚಲನವಲನದಲ್ಲೂ ಶಕ್ತಿ ಸಾಮರ್ಥ್ಯ ತೇಜಸ್ಸು ಎದ್ದು ಕಾಣುತ್ತಿತ್ತು. ಅವನನ್ನು ಸುಂದರಾಂಗನೆಂದು ಸುಲಭವಾಗಿ ಹೇಳಬಹುದಾಗಿತ್ತಾದರೂ ಅವನಲ್ಲಿ ಹೆಣ್ಣಿಗತನ ಕಿಂಚಿತ್ತೂ ಇರಲಿಲ್ಲ. ಅವನ ಇಡೀ ವ್ಯಕ್ತಿತ್ವದಲ್ಲಿ ಎದ್ದುಕಾಣುತ್ತಿದ್ದ ಅಂಶವೇನೆಂದರೆ ಆತನ ನಯನದ್ವಯ. ಕಮಲದ ದಳಗಳಂತೆ ಅವುಗಳ ಆಕಾರ. ಅಷ್ಟು ಎದ್ದುತೋರುತ್ತಿದ್ದರೂ ಅವು ಉಬ್ಬುಗಣ್ಣು ಗಳಲ್ಲ. ಅವನ ಭಾವನೆಗಳು ಬದಲಾದಂತೆ ಆ ಹೂಳಪುಗಣ್ಣುಗಳ ಬಣ್ಣದ ಛಾಯೆಯೂ ಬದಲಾಗುತ್ತಿತ್ತು! ಅವನ ಚುರುಕು ಬುದ್ಧಿಯನ್ನೂ ಸೂಕ್ಷ್ಮಮತಿಯನ್ನೂ ಆ ನೇತ್ರದ್ವಯ ಪ್ರತಿ ಬಿಂಬಿಸುತ್ತಿತ್ತು. ಅವನು ನೆಟ್ಟನೋಟದಿಂದ ನೋಡುತ್ತಿದ್ದಾಗ ಆ ಕಂಗಳು ಜ್ಯೋತಿಃಪುಂಜವಾಗಿ ಕಾಣುತ್ತಿದ್ದುವು. ಇತರ ಸಮಯಗಳಲ್ಲಿ, ಅವುಗಳಿಂದ ಆನಂದೋತ್ಸಾಹದ ಕಿಡಿಗಳು ಹೊಮ್ಮು ತ್ತಿದ್ದವು! ಅವನು ಯಾರೊಡನೆಯಾದರೂ ಮಾತನಾಡುವಾಗ, ಆಗ ಅಲ್ಲಿರುವುದೆಲ್ಲ ಆ ವ್ಯಕ್ತಿ ಮಾತ್ರವೇ ಏನೋ ಎಂಬಷ್ಟು ತಾದಾತ್ಮ್ಯದಿಂದ ಮಾತನಾಡುತ್ತಿದ್ದ. ಇದರಿಂದಾಗಿ, ಅವನೊಡನೆ ಸಂಭಾಷಿಸುವುದೆಂದರೆ ಎಲ್ಲರಿಗೂ ಒಂದು ವಿಶೇಷ ಸಂತೋಷದ ಅನುಭವವಾಗಿತ್ತು.

ನರೇಂದ್ರನದು ಮಾಂಸಲವಾದ ಜಟ್ಟಿಯ ಮೈಕಟ್ಟಾದರೂ ಅವನ ಮುಖದ ಕಡೆಗೊಮ್ಮೆ ದೃಷ್ಟಿ ಹರಿಸಿದರೆ ಆ ಶರೀರದ ಕಡೆಗಿನ ಗಮನ ಹೊರಟುಹೋಗುತ್ತಿತ್ತು. ತುಂಬಿಕೊಂಡು ಪುಷ್ಟವಾದ ಅವನ ದವಡೆಗಳು ಅವನ ದೃಢನಿಶ್ಚಯದ ಮನೋಭಾವವನ್ನು ಸೂಚಿಸುತ್ತಿದ್ದವು. ಅವನ ಮುಖವನ್ನು ನೋಡಿದಾಗ ಕೆಲವರಿಗೆ ಅವನೊಬ್ಬ ಭಾವಜೀವಿ ಎನ್ನಿಸುತ್ತಿತ್ತು. ಇನ್ನು ಕೆಲವರು, ಅವನೊಬ್ಬ ತೀವ್ರವಾಗಿ ಆಲೋಚಿಸುವ ಚಿಂತನಶೀಲ ವ್ಯಕ್ತಿ ಎಂದು ಭಾವಿಸುತ್ತಿದ್ದರು. ಮತ್ತೆ ಕೆಲವರಿಗೆ ಆದರ್ಶ ಪ್ರೇಮ-ಸೌಂದರ್ಯ-ಕಲೆಗಳ ಲೋಕದಲ್ಲಿ ವಿಹರಿಸುತ್ತಿರುವ ಯುವಕ ನಂತೆ ತೋರುತ್ತಿದ್ದ. ಆದರೆ ಅವನೊಬ್ಬ ಸತ್ಕುಲಪ್ರಸೂತನಾದ ಸಂಭಾವಿತ ಎಂಬುದನ್ನು ಎಲ್ಲರೂ ಕಾಣಬಹುದಾಗಿತ್ತು. ಅವನ ಮುಗುಳ್ನಗೆ ತುಂಬ ಮೋಹಕವಾಗಿತ್ತು. ಅದನ್ನು ನೋಡು ತ್ತಿದ್ದರೆ ನೋಡುವವರ ಮನಸ್ಸು ಅದರಲ್ಲೇ ಲೀನವಾಗಿಬಿಡುತ್ತಿತ್ತು. ಆದರೆ ಅವನು ಗಂಭೀರ ನಾಗಿಬಿಟ್ಟನೆಂದರೆ ಅವನ ಮುಖಭಾವವನ್ನು ಕಂಡು ಸ್ನೇಹಿತರೆಲ್ಲ ಸ್ತಬ್ಧರಾಗಿಬಿಡುತ್ತಿದ್ದರು –ಅಂಥ ಗಂಭೀರತೆ! ಇದೇ ನರೇಂದ್ರ ಇನ್ನು ಕೆಲವು ಸಲ ತನ್ನ ಸೋದರ ಶಿಷ್ಯರಿಗೆ ಒಬ್ಬ ಪುಟ್ಟ ಬಾಲಕನಂತೆ ಕಂಡುಬರುತ್ತಿದ್ದ. ಕೆಲವು ಸಲ ಅವರೊಂದಿಗೆ ಜಗಳವಾಡುತ್ತಿದ್ದ, ಕಿತ್ತಾಡು ತ್ತಿದ್ದ, ಹಠ ಮಾಡುತ್ತಿದ್ದ. ಆದರೆ ಅವನ ಈ ಕಿತ್ತಾಟ-ಹಠಮಾರಿತನ ಕೂಡ ಎಷ್ಟು ಚೆನ್ನಾಗಿರು ತ್ತಿತ್ತೆಂದರೆ ಅವನ ಗುರುಭಾಯಿಗಳು ಅದರಿಂದ ಸಂತೋಷವನ್ನೇ ಪಡುತ್ತಿದ್ದರು. ಅವನ ಮೇಲೆ ಇನ್ನಷ್ಟು ಪ್ರೀತಿಯನ್ನೆ ತಾಳುತ್ತಿದ್ದರು. ನಿಜಕ್ಕೂ ಇದೊಂದು ವಿಶೇಷವಾದ ಸ್ಥಿತಿ. ಒಬ್ಬನಲ್ಲಿ ಪರಿಶುದ್ಧವಾದ ಹೃದಯ, ನಿಷ್ಕಲ್ಮಶವಾದ ಪ್ರೀತಿ ಇದ್ದಾಗ ಮಾತ್ರ ಇತರರನ್ನು ಹೀಗೆ ಸಂತೋಷ ವಾಗಿಡಲು ಸಾಧ್ಯ. ನರೇಂದ್ರನ ಕೋಪತಾಪಗಳಲ್ಲಿ ಆ ಬಗೆಯ ಆಕರ್ಷಣೆಯಿತ್ತು. ಆದ್ದರಿಂದಲೇ ಅವನ ಕೋಪತಾಪಗಳನ್ನು ಕಂಡಾಗ ಗುರುಭಾಯಿಗಳಿಗೆ ಅವನ ಮೇಲಿನ ಪ್ರೀತಿ ಇನ್ನಷ್ಟು ಹೆಚ್ಚುತ್ತಿತ್ತು. ಅವನು ವಾದವಿವಾದದಲ್ಲಿ ತೊಡಗಿದಾಗ ರಭಸ ಹೆಚ್ಚಾದರೆ, ಅವನ ಕಣ್ಣುಗಳು ಒಂದು ವಿಶೇಷ ಪ್ರಕಾಶವನ್ನು ಹೊರಸೂಸುತ್ತ ಅವನೊಳಗಿನ ಅಪಾರ ಶಕ್ತಿಯನ್ನು ಸೂಚಿಸು ತ್ತಿದ್ದುವು. ಆದರೆ ಅವನು ತನ್ನ ಆಲೋಚನೆಗಳಲ್ಲೇ ಮುಳುಗಿ ಕುಳಿತುಬಿಟ್ಟರೆ ಅವನ ಸುತ್ತ ಒಂದು ಬಗೆಯ ವಿಮುಖತೆಯ ಭಾವ ಆವರಿಸಿಬಿಡುತ್ತಿತ್ತು. ಆಗ ಅವನನ್ನು ಆ ಭಾವದಿಂದ ಕದಲಿಸುವ ಎದೆಗಾರಿಕೆ ಯಾರಿಗೂ ಇರುತ್ತಿರಲಿಲ್ಲ. ನಿಜಕ್ಕೂ ಈ ಬಗೆಯ ತೀವ್ರ ವಿಮುಖತೆ ಯೆಂಬುದು ಅವನ ವ್ಯಕ್ತಿತ್ವದ ಒಂದು ಪ್ರಧಾನ ಲಕ್ಷಣವೇ ಆಗಿತ್ತು.

ನರೇಂದ್ರ ಸ್ವಭಾವತಃ ಅಸಾಧಾರಣ ಮೇಧಾವಿ. ಅವನ ಮನಸ್ಸಿನಲ್ಲಿ ನಾನಾ ಭಾವನೆಗಳು ಏಳುತ್ತಿದ್ದುವು; ನಾನಾ ಆಲೋಚನೆಗಳು ಸುಳಿಯುತ್ತಿದ್ದುವು; ನಾನಾ ಭಾವಗಳು ಪ್ರಕಾಶಿತವಾಗು ತ್ತಿದ್ದುವು. ಕೆಲವೊಮ್ಮೆ ಇದ್ದಕ್ಕಿದ್ದಂತೆ ತನ್ನ ಪ್ರಕೃತ ಪರಿಸರದ ಮೇಲೆ ಒಂದು ಬಗೆಯ ಅಸಹನೆ ತಾಳುತ್ತಿದ್ದ. ಇನ್ನು ಕೆಲವು ಸಲ, ಫಲಿತಾಂಶವನ್ನು ಲೆಕ್ಕಿಸದೆ ಕರ್ಮಯೋಗಿಯಂತೆ ಪ್ರೀತಿಯುತ ಸಹನೆಯಿಂದಿರುತ್ತಿದ್ದ. ಅವನು ಪಡುತ್ತಿದ್ದ ಕಷ್ಟಗಳನ್ನೆಲ್ಲ ಎಣಿಸಿದರೆ ಅವನ ಹೃದಯ ಕಲ್ಲಾಗದೆ ಇದ್ದದ್ದೇ ಒಂದು ಆಶ್ಚರ್ಯ. ಈ ಬಗೆಯ ಶಾಂತತೆ, ನಿರ್ಲಿಪ್ತತೆ, ಸಂತೋಷ ಅವನಿಗೆ ಸ್ವಭಾವ ಸಹಜವಾಗಿ ಪರಿಣಮಿಸಿತು, ಮತ್ತು ಅವನ ಜೀವನವಿಡೀ ಇದನ್ನು ಕಾಣಬಹುದು. ಯಾರಿಗಾಗಿ ಅವನು ಅಷ್ಟೆಲ್ಲ ಕಷ್ಟಗಳನ್ನು ಅನುಭವಿಸಿದನೋ, ಅವರಿಂದಲೂ ಅವನಿಗೆ ಪ್ರತಿಯಾಗಿ ಪ್ರೀತಿ- ವಿಶ್ವಾಸ ದೊರಕಲಿಲ್ಲ. ಆದರೂ ಅವನೆಂದೂ ಮನಸ್ಸಿನ ಸ್ತಿಮಿತವನ್ನು ಕಳೆದುಕೊಳ್ಳಲಿಲ್ಲ, ಅಶಾಂತನಾಗಲಿಲ್ಲ; ಯಾವಾಗಲೂ ಪ್ರೇಮಭರಿತನಾಗಿಯೇ ಉಳಿದುಕೊಂಡ. ತನ್ನನ್ನು ಇತರರು ಅರ್ಥಮಾಡಿಕೊಳ್ಳಲಿಲ್ಲವೆಂದು ಜುಗುಪ್ಸೆ ತಾಳಲಿಲ್ಲ. ಮುಂದೆ ಸ್ವಾಮಿ ವಿವೇಕಾನಂದರು ಈ ಕುರಿತಾಗಿ ಹೇಳುತ್ತಾರೆ: “ಇತರರು ನನ್ನನ್ನು ಅರ್ಥಮಾಡಿಕೊಳ್ಳಬೇಕು ಅಂತ ನಾನೇಕೆ ತಾನೆ ನಿರೀಕ್ಷಿಸಬೇಕು? ಅವರಿಗೆ ನನ್ನ ಮೇಲೆ ವಿಶ್ವಾಸವೊಂದಿದ್ದರೆ ಬೇಕಾದಷ್ಟಾಯಿತು. ಇಷ್ಟರ ಮೇಲೆ ನಾನಾದರೂ ಏನು ಮಹಾ? ಜಗನ್ಮಾತೆಗೆ ಎಲ್ಲವೂ ಗೊತ್ತು. ತನ್ನ ಕಾರ್ಯವನ್ನು ನೋಡಿ ಕೊಳ್ಳಲು ಆಕೆ ಸಮರ್ಥಳಲ್ಲವೆ? ನಾನಿಲ್ಲದೆ ಏನೂ ಆಗುವಂತಿಲ್ಲ ಎಂದು ನಾನೇಕೆ ಭಾವಿಸಲಿ?”

ಶ್ರೀರಾಮಕೃಷ್ಣರು ತೋರಿದ ದಿವ್ಯ ಪ್ರೇಮ ನರೇಂದ್ರನಲ್ಲಿ ಹಲವಾರು ಪರಿವರ್ತನೆಗಳ ನ್ನುಂಟುಮಾಡಿತು. ಅವನಲ್ಲಿ ಉಂಟಾದ ಬುದ್ಧಿ-ಹೃದಯಗಳ ಮಧುರ ಸಾಮರಸ್ಯ ಇವುಗಳ ಲ್ಲೊಂದು. ಸಾಮಾನ್ಯವಾಗಿ, ಬುದ್ಧಿ ಹೆಚ್ಚಾಗಿರುವವರಲ್ಲಿ ಹೃದಯವಂತಿಕೆ ಕಡಿಮೆ; ಹೃದಯ ವಂತರಲ್ಲಿ ಬುದ್ಧಿವಂತಿಕೆ ಸ್ವಲ್ಪ ಕಡಿಮೆ. ಇವೆರಡೂ ಸಮ ಪ್ರಮಾಣದಲ್ಲಿ ಬೆಳೆದಿರುವವರು ತೀರ ವಿರಳ. ನರೇಂದ್ರನಲ್ಲೀಗ ಪ್ರಚಂಡ ಬುದ್ಧಿಶಕ್ತಿಯೊಡನೆ ಪ್ರೇಮ, ಅನುಕಂಪ ಮೊದಲಾದ ಮಧುರ ಭಾವನೆಗಳು ವೃದ್ಧಿಗೊಳ್ಳಲಾರಂಭಿಸಿದ್ದುವು. ಇದನ್ನು ಸಾಧಿಸಿದ್ದುದು ಶ್ರೀರಾಮಕೃಷ್ಣರ ದಿವ್ಯ ಪ್ರೀತಿಯೇ. ನರೇಂದ್ರನಿಗೆ ಅವರು ಏನೇನನ್ನು ಕಲಿಸಿದರೋ, ಅವನ ಮೇಲೆ ಎಷ್ಟೆಷ್ಟು ಪ್ರಭಾವ ಬೀರಿದರೋ, ಅವೆಲ್ಲ ತಮ್ಮ ಪ್ರೀತಿಯ ಮಾಧ್ಯಮದ ಮೂಲಕವೇ. ನರೇಂದ್ರನಂತಹ ಪ್ರಚಂಡ ಬುದ್ಧಿಶಾಲಿಗೆ ಕಲಿಸಬೇಕಾದರೆ ಬುದ್ಧಿಯ ಭಾಷೆ ಕೆಲಸಕ್ಕೆ ಬರುವಂತಿಲ್ಲ; ಪ್ರೀತಿಯ ಭಾಷೆಯೇ ಆಗಬೇಕು. ಸ್ವಭಾವತಃ ನರೇಂದ್ರ ಜಿಜ್ಞಾಸು, ತತ್ತ್ವಜ್ಞಾನಿಯಾಗಿದ್ದ. ಶ್ರೀರಾಮಕೃಷ್ಣರು ಅವನನ್ನು ಒಬ್ಬ ಭಕ್ತನನ್ನಾಗಿಯೂ ಮಾಡಿದರು. ಆದರೆ ಅವನಲ್ಲಿ ಭಕ್ತಿಭಾವವನ್ನು ಮಾತ್ರ ತುಂಬಿದರು ಎಂದಲ್ಲ; ಅವನಿಗೆ ಅತ್ಯುನ್ನತ ಆತ್ಮಜ್ಞಾನದ ಅನುಭವವನ್ನು ನೀಡಿದವರೂ ಅವರೇ. ಅವನ ಬೌದ್ಧಿಕ ಜ್ಞಾನವು ಭಗವದ್ಭಕ್ತಿಯ ಸಂಪರ್ಕದಿಂದ ಮೃದುಗೊಂಡು ಹದಗೊಳ್ಳುವಂತೆ ಮಾಡಿದರು. ಮೇಲ್ನೋಟಕ್ಕೆ ಅವನೊಬ್ಬ ತತ್ತ್ವಜ್ಞಾನಿಯಂತೆ ಕಾಣುತ್ತಿದ್ದನಾದರೂ ವಾಸ್ತವಿಕ ವಾಗಿ ಅವನು ಭಕ್ತನೆ. ಆತನ ಅಂಗಲಕ್ಷಣಗಳ ಆಧಾರದ ಮೇಲೆ ಅವನನ್ನು ಭಕ್ತನೆಂದು ತೀರ್ಮಾ ನಿಸಬಹುದು ಎಂದು ಶ್ರೀರಾಮಕೃಷ್ಣರು ಹೇಳುತ್ತಿದ್ದರು. “ಸಾಮಾನ್ಯವಾಗಿ ಜ್ಞಾನಿಗಳ ಶರೀರಲಕ್ಷಣ ಶುಷ್ಕವಾಗಿ ಕಂಡುಬರುತ್ತದೆ. ಆದರೆ ಭಕ್ತರು ಸಾಮಾನ್ಯವಾಗಿ ಪ್ರಿಯದರ್ಶನರಾಗಿರುತ್ತಾರೆ” ಎಂದು ಅವರೆನ್ನುತ್ತಿದ್ದರು. ಮುಂದೆ ಸ್ವಾಮಿ ವಿವೇಕಾನಂದರೇ ಹೇಳಿಕೊಳ್ಳುತ್ತಾರೆ: “ಶ್ರೀರಾಮ ಕೃಷ್ಣರು ಮೇಲ್ನೋಟಕ್ಕೆ ಸಂಪೂರ್ಣ ಭಕ್ತ; ಆದರೆ ಅವರ ಅಂತರಂಗದಲ್ಲಿ ಜ್ಞಾನ ತುಂಬಿತ್ತು. ನಾನು ಹೊರನೋಟಕ್ಕೆ ಕೇವಲ ಜ್ಞಾನಿಯಾಗಿ ಕಂಡರೂ ನನ್ನೊಳಗಿರುವುದು ಕೇವಲ ಭಕ್ತಿ.”

ಆದರೆ ಈ ಜ್ಞಾನ-ಭಕ್ತಿಗಳೆರಡೂ ಬೇರೆಬೇರೆಯಲ್ಲ, ಒಂದೇ ನಾಣ್ಯದ ಎರಡು ಮುಖಗಳು ಎಂಬುದು, ಅನುಭವಕ್ಕೆ ಬಂದಮೇಲೆ ತಿಳಿಯುವ ಸತ್ಯ. ಆದರೆ ಬೌದ್ಧಿಕತೆಗೂ ಆಧ್ಯಾತ್ಮಿಕತೆಗೂ ಬಹಳ ವ್ಯತ್ಯಾಸವಿದೆ. ವಿಚಾರದ ಮೂಲಕ ತಿಳಿಯಬಹುದಾದದ್ದು ಬೌದ್ಧಿಕತೆ; ಆತ್ಮಾನುಭವದ ಮೂಲಕ ತಿಳಿಯಬಹುದಾದದ್ದು ಆಧ್ಯಾತ್ಮಿಕತೆ. ಶ್ರೀರಾಮಕೃಷ್ಣರ ಶಿಕ್ಷಣದಿಂದ ನರೇಂದ್ರ ಈ ಅಂಶವನ್ನು ಮನಗಂಡ.

ನರೇಂದ್ರನ ವ್ಯಕ್ತಿತ್ವದ ಒಂದು ಮುಖವನ್ನು ಬಡತನ-ಕಷ್ಟ ಸಂಕಟಗಳು ರೂಪಿಸಿದರೆ, ಅವನ ಒಡನಾಡಿಗಳಿಂದ ಮತ್ತೊಂದು ಭಾಗ ರೂಪಿತಗೊಂಡಿತ್ತು. ಶ್ರೀರಾಮಕೃಷ್ಣರು ಅವನ ವ್ಯಕ್ತಿತ್ವದ ಈ ಎರಡೂ ಮುಖಗಳನ್ನು ತಿದ್ದಿ, ಅವನನ್ನು ಭವಿಷ್ಯದ ವಿವೇಕಾನಂದನನ್ನಾಗಿ ನಿರ್ಮಿಸಿದರು. ಅವನಿಗೆ ಬಂದ ಬಡತನ ಮತ್ತು ಇತರ ಕಷ್ಟಗಳ ಕುರಿತಾಗಿ ಅವರೆನ್ನುತ್ತಾರೆ, “ನರೇಂದ್ರನಿಗೆ ಯಾವ ಕಷ್ಟಗಳೂ ಬಾರದೆ, ಅವನು ಕೇವಲ ಶ್ರೀಮಂತಿಕೆಯಲ್ಲಿ ಬೆಳೆದುಬಂದಿದ್ದರೆ, ಅವನ ಜೀವನದ ದಿಕ್ಕು ಬೇರೆಯೇ ಆಗಿ ಬಿಡುತ್ತಿತ್ತು” ಎಂದು. ಬಹುಶಃ ಅವನೊಬ್ಬ ದೊಡ್ಡ ರಾಜ ಕಾರಣಿಯೋ, ವಕೀಲನೋ, ವಾಗ್ಮಿಯೋ ಆಗಬಹುದಾಗಿತ್ತು. ಅಥವಾ ಪ್ರಮುಖ ಸಮಾಜ ಸುಧಾರಕನಾಗಬಹುದಾಗಿತ್ತು. ಆದರೆ ಬಡತನ ಬವಣೆಯನ್ನು ಅವನು ಸ್ವತಃ ಅನುಭವಿಸಿದ್ದ ರಿಂದಲೇ ಅವನು ಬಡವರಿಗಾಗಿ ಮರುಗುವಂತಾದ.

ನರೇಂದ್ರ ತನ್ನ ತಂದೆಯನ್ನು ಕಳೆದುಕೊಂಡು, ಅವನ ಸಂಸಾರ ನಿರ್ಗತಿಕವಾಗಿದ್ದ ಸಂದರ್ಭ. ಆಗೊಂದು ದಿನ ಶ್ರೀರಾಮಕೃಷ್ಣರು ಭಕ್ತರೊಂದಿಗೆ ಮಾತನಾಡುತ್ತಿದ್ದ ಸಂದರ್ಭದಲ್ಲಿ ಒಬ್ಬ ಶ್ರೀಮಂತನಿಗೆ, “ನೋಡು, ನರೇಂದ್ರನ ತಂದೆ ತೀರಿಹೋಗಿಬಿಟ್ಟರು. ಅವನ ಮನೆಯವರು ಬಹಳ ಕಷ್ಟದಲ್ಲಿದ್ದಾರೆ. ಅವನ ಸ್ನೇಹಿತರು ಈಗ ಅವನಿಗೆ ಏನಾದರೂ ಸಹಾಯ ಮಾಡಲು ಸಾಧ್ಯವಾದರೆ ಒಳ್ಳೆಯದು” ಎಂದು ಹೇಳಿದರು. ನರೇಂದ್ರ ಆಗ ಅಲ್ಲೇ ಇದ್ದನಾದರೂ ಏನೂ ಮಾತನಾಡಲಿಲ್ಲ. ಆದರೆ ಸ್ವಲ್ಪ ಹೊತ್ತಿಗೆ ಆ ಶ್ರೀಮಂತ ಅಲ್ಲಿಂದ ಹೊರಟುಹೋದಮೇಲೆ ಅಸಮಾಧಾನದಿಂದ, “ಮಹಾಶಯರೆ, ನೀವು ಅದನ್ನೆಲ್ಲ ಅವನ ಹತ್ತಿರ ಏಕೆ ಹೇಳಬೇಕಾಗಿತ್ತು?” ಎಂದ. ತನ್ನ ಸ್ವಾಭಿ ಮಾನಕ್ಕೆ ಮನೆತನದ ಗೌರವಕ್ಕೆ ಧಕ್ಕೆಯಾಯಿತು ಎಂಬ ಬೇಸರ ಅದರಲ್ಲಿ ವ್ಯಕ್ತವಾಗಿತ್ತು. ಆಗ ಶ್ರೀರಾಮಕೃಷ್ಣರು ಕಂಬನಿದುಂಬಿ ನುಡಿದರು: “ಓ ನರೇನ್, ನಿನಗೋಸ್ಕರ ನಾನು ಏನು ಬೇಕಾದರೂ ಮಾಡಲು ಸಿದ್ಧ ಎಂಬುದು ನಿನಗೆ ಗೊತ್ತಿಲ್ಲವೆ? ಬೇಕೆಂದರೆ ನಿನಗಾಗಿ ನಾನು ಮನೆಮನೆಗೆ ಹೋಗಿ ಭಿಕ್ಷೆ ಬೇಡಿಯೇನು!” ಇದನ್ನು ಕೇಳಿದಾಗ ನರೇಂದ್ರನಿಗೆ ಕಣ್ಣೀರು ಉಕ್ಕಿಬಂತು. ಶ್ರೀರಾಮಕೃಷ್ಣರ ನಿಷ್ಕಾಮ ಪ್ರೇಮಕ್ಕಿಂತಲೂ ಅವನ ಸ್ವಾಭಿಮಾನ ಹೆಚ್ಚೆ! ಆದ್ದರಿಂದಲೇ ಮುಂದೆ ಸ್ವಾಮಿ ವಿವೇಕಾನಂದರು, “ಶ್ರೀರಾಮಕೃಷ್ಣರು ತಮ್ಮ ಅಪಾರ ಪ್ರೀತಿ ಯಿಂದ ನನ್ನನ್ನು ತಮ್ಮ ದಾಸನನ್ನಾಗಿ ಮಾಡಿಕೊಂಡುಬಿಟ್ಟರು” ಎನ್ನುವುದು. ಇಲ್ಲಿ, ಶ್ರೀರಾಮ ಕೃಷ್ಣರ ಪ್ರೀತಿ ದೊಡ್ಡದೋ ಅಥವಾ ನರೇಂದ್ರ ಅವರ ಪ್ರೀತಿಯನ್ನು ಅರ್ಥಮಾಡಿಕೊಂಡದ್ದು ದೊಡ್ಡದೋ ಎನ್ನುವುದು ಇನ್ನೊಂದು ಅಂಶ. ಏಕೆಂದರೆ ಪ್ರೀತಿಯ ಸೂಕ್ಷ್ಮವನ್ನು ಅರ್ಥಮಾಡಿ ಕೊಳ್ಳಬೇಕಾದರೆ ಹೃದಯಸೂಕ್ಷ್ಮ ಬೇಕಾಗುತ್ತದೆ. ಕೊಟ್ಟ ಪ್ರೀತಿಯನ್ನು ಸ್ವೀಕರಿಸಿ ಆಸ್ವಾದಿಸಲೂ ಯೋಗ್ಯತೆಯಿರಬೇಕಾಗುತ್ತದೆ.

ಈ ಪ್ರೀತಿಯ ರೀತಿಯಲ್ಲಿ ಎಷ್ಟೋ ಸಲ ಒಂದು ದೋಷ ಸಂಭವಿಸುವುದಿದೆ. ಏನೆಂದರೆ, ನಾವು ಯಾರನ್ನು ವಿಶೇಷವಾಗಿ ಪ್ರೀತಿಸುತ್ತೇವೆಯೋ ಅವರ ಮೇಲೆ ವಿಶೇಷ ಮಮಕಾರ ಬೆಳೆದು, ‘ಪ್ರೀತಿ’ ಎಂಬುದು ‘ಮೋಹ’ವಾಗಿ ಪರಿಣಮಿಸುತ್ತದೆ. ಆಗ ಅವರು ಮಾಡುವ ತಪ್ಪುಗಳಿಗೆಲ್ಲ ಕಣ್ಣುಮುಚ್ಚಿಕೊಂಡುಬಿಡುತ್ತೇವೆ, ಧೃತರಾಷ್ಟ್ರರಾಗಿಬಿಡುತ್ತೇವೆ. ಆದರೆ ಶ್ರೀರಾಮಕೃಷ್ಣರು ನರೇಂದ್ರನನ್ನು ಅಷ್ಟೊಂದು ಪ್ರೀತಿಸುತ್ತಿದ್ದರಾದರೂ, ಅವನ ನಡತೆಯಲ್ಲೇನಾದರೂ ದೋಷ ಕಂಡುಬಂದರೆ ತಕ್ಷಣ ಅದನ್ನು ತಿದ್ದದೆ ಬಿಡುತ್ತಿರಲಿಲ್ಲ. ಅವನ ಮನಸ್ಸಿನಲ್ಲಿ ಅವನ ವ್ಯಕ್ತಿತ್ವಕ್ಕೆ ತಕ್ಕುದಲ್ಲದ ಒಂದು ಆಲೋಚನೆ ಒಮ್ಮೆ ಸುಳಿದರೂ ಸಾಕು, ಶ್ರೀರಾಮಕೃಷ್ಣರಿಗೆ ಅದು ಸ್ಪಷ್ಟ ವಾಗಿ ಗೋಚರಿಸಿಬಿಡುತ್ತಿತ್ತು. ಒಮ್ಮೆ ಅವನು ಒಬ್ಬ ವ್ಯಕ್ತಿಯೊಡನೆ ತುಂಬ ಸಲಿಗೆಯಿಂದ ಓಡಾಡುತ್ತಿರುವುದನ್ನು ಅವರು ಕಂಡರು. ಆ ವ್ಯಕ್ತಿ ಅವರ ಭಕ್ತನೇ. ಆದರೆ ಆತ ಹಿಂದೆ ತುಂಬ ವಿಲಾಸೀ ಜೀವನ ನಡೆಸಿದವನು. ಆದ್ದರಿಂದ ಅವರು ನರೇಂದ್ರನನ್ನು ಕರೆದು ಅವನ ಜೊತೆಯಲ್ಲಿ ಬೆರೆಯದಿರುವಂತೆ ಎಚ್ಚರಿಕೆ ನೀಡಿದರು. ಆಗ ನರೇಂದ್ರ ಅದನ್ನು ಪ್ರತಿಭಟಿಸಿ, “ಆದರೆ ಈಗ ಅವನು ತನ್ನ ಮಾರ್ಗವನ್ನು ಬದಲಿಸಿಕೊಂಡು ಒಳ್ಳೆಯ ಜೀವನ ನಡೆಸುತ್ತಿದ್ದಾನಲ್ಲ?” ಎಂದ. ಅದಕ್ಕೆ ಶ್ರೀರಾಮಕೃಷ್ಣರೆನ್ನುತ್ತಾರೆ: “ನೋಡು, ಬೆಳ್ಳುಳ್ಳಿಯಿಟ್ಟಿದ್ದ ಪಾತ್ರೆಯನ್ನು ಎಷ್ಟೇ ತೊಳೆ ದರೂ ಒಂದು ಸ್ವಲ್ಪವಾದರೂ ವಾಸನೆ ಉಳಿದೇ ಉಳಿಯುತ್ತದೆ. ಹಾಗೆಯೇ ಕಾಮಕಾಂಚನದಲ್ಲಿ ಒಮ್ಮೆ ಮುಳುಗಿದವರ ಹೃದಯವೂ ಕೂಡ. ಆದರೆ ನೀವೆಲ್ಲ ಪರಿಶುದ್ಧಾತ್ಮರು. ನಿಮ್ಮ ಹೃದಯಗಳಿಗೆ ಕಾಮಕಾಂಚನದ ಕಿಲುಬಿನ್ನೂ ತಗಲಿಲ್ಲ. ಕಾಗೆ ಕುಕ್ಕಿದ ಮಾವಿನ ಹಣ್ಣನ್ನು ನೋಡಿಲ್ಲವೇ? ಅದನ್ನು ದೇವರಿಗೆ ನೈವೇದ್ಯ ಮಾಡುವುದಿರಲಿ, ಮನುಷ್ಯರು ತಿನ್ನಲೂ ಅದು ಯೋಗ್ಯವಲ್ಲ. ಪ್ರಾಪಂಚಿಕ ಭೋಗಗಳನ್ನು ಅನುಭವಿಸಿದ ಭಕ್ತರೆಲ್ಲ ಬೇರೆಯೇ ವರ್ಗಕ್ಕೆ ಸೇರಿದವರು.”

ತಾನು ಶ್ರೀರಾಮಕೃಷ್ಣರ ನಿಕಟಸಂಪರ್ಕದಲ್ಲಿದ್ದ ಆ ನಾಲ್ಕೈದು ವರ್ಷಗಳಲ್ಲಿ ನರೇಂದ್ರ ಪಡೆದುಕೊಂಡ ಶಿಕ್ಷಣ ಈ ಬಗೆಯದು. ಆಧ್ಯಾತ್ಮಿಕತೆಯ ಬಗ್ಗೆ ಮತ್ತು ತನ್ನ ಆಧ್ಯಾತ್ಮಿಕ ಭವಿಷ್ಯದ ಬಗ್ಗೆ ನರೇಂದ್ರ ಈ ಹಿಂದೆಯೇ ಕೆಲವು ಕಲ್ಪನೆಗಳನ್ನಿಟ್ಟುಕೊಂಡಿದ್ದ. ಆದರೆ ಈಗ ಶ್ರೀರಾಮ ಕೃಷ್ಣರ ದಿವ್ಯ ಪರಿಸರದಲ್ಲಿ ಅವೆಲ್ಲವನ್ನೂ ಮೀರಿ ಬೆಳೆದ. ಅವನು ನಿಖರವಾಗಿ ಯಾವ ಮುಹೂರ್ತದಿಂದ ಅವರನ್ನು ಗುರುವಾಗಿ ಸ್ವೀಕರಿಸಿದ ಎಂದು ಹೇಳುವುದು ಕಷ್ಟ. ಆಧ್ಯಾತ್ಮಿಕಾ ನುಭವದ ದೃಷ್ಟಿಯಿಂದ ನೋಡಿದರೆ, ಶ್ರೀರಾಮಕೃಷ್ಣರ ಪ್ರಥಮ ಸ್ಪರ್ಶದ ಕ್ಷಣದಿಂದಲೇ ಅವನು ಅವರ ಶಿಷ್ಯನಾಗಿಬಿಟ್ಟ ಎನ್ನಬಹುದು. ಆದರೆ ಅವನು ಬುದ್ಧಿಪೂರ್ವಕವಾಗಿ ಅವರನ್ನು ಗುರುವಾಗಿ ಸ್ವೀಕರಿಸಿದ್ದು ಅವರು ತನ್ನ ಎಲ್ಲ ಪ್ರಶ್ನೆಗಳಿಗೆ ತೃಪ್ತಿಕರ ಉತ್ತರ ನೀಡಿ ತನ್ನ ಬೌದ್ಧಿಕತೆಗೆ ಸಮಾಧಾನವುಂಟುಮಾಡಿದಾಗ ಎನ್ನಬಹುದು. ಮತ್ತು, ತನ್ನ ಪ್ರತಿಯೊಂದು ಪ್ರಶ್ನೆಗೂ ಪ್ರತಿ ಯೊಂದು ಸಮಸ್ಯೆಗೂ ಸಮಂಜಸವಾದ, ಯುಕ್ತಿಯುಕ್ತವಾದ ಉತ್ತರವನ್ನು ಕೊಟ್ಟು ಶ್ರೀರಾಮ ಕೃಷ್ಣರು ಅವನನ್ನು ಒಪ್ಪಿಸಿದ್ದರಿಂದಲೇ ಮುಂದೆ ಅವನು ಜಗದ ಜನರನ್ನು ಒಪ್ಪಿಸುವ ಸಾಮರ್ಥ್ಯವನ್ನು ಪಡೆಯುವಂತಾಯಿತು.

ಅಂತೂ ನರೇಂದ್ರನಿಗೆ ಶ್ರೀರಾಮಕೃಷ್ಣರನ್ನು ಗುರುವಾಗಿ ಸ್ವೀಕರಿಸಲು ಸಾಕಷ್ಟು ಸಮಯವೇ ಹಿಡಿಯಿತು ಎನ್ನಬೇಕು. ಆದರೆ ಒಮ್ಮೆ ಸ್ವೀಕರಿಸಿದ ಮೇಲೆ ಮಾತ್ರ ಅವನು ಎಂದೆಂದಿಗೂ ಅವರ ಶಿಷ್ಯನೇ. ಇನ್ನು ಅವರಿಬ್ಬರ ನಡುವಿನ ಗುರು-ಶಿಷ್ಯ ಸಂಬಂಧ ಎಂಬುದು ಮತ್ತೆ ಸಂದೇಹ ಹಾಗೂ ಅಭಿಪ್ರಾಯಭೇದಗಳಿಂದ ಬಾಧಿತವಾಗದು, ದುರ್ಬಲವಾಗದು. ಅವರ ನಡುವಿನ ಈ ನಿಕಟ ಸಂಪರ್ಕ-ಬಾಂಧವ್ಯಗಳು ಐದು ವರ್ಷಗಳ ದೀರ್ಘ ಕಾಲ ಬೆಳೆದುಬಂದವು. ಅವರಿಬ್ಬರ ಪ್ರತಿಯೊಂದು ಭೇಟಿಯೂ ಇಬ್ಬರಿಗೂ ಅಪೂರ್ವ ಸ್ಫೂರ್ತಿ-ಸಂತೋಷಗಳನ್ನು ತಂದಿತು. ಪ್ರತಿಸಲದ ಸಂದರ್ಶನವೂ ಆ ಮಧುರ ಬಾಂಧವ್ಯವನ್ನು ಇನ್ನಷ್ಟು ಮತ್ತಷ್ಟು ನಿಕಟಗೊಳಿಸಿತು, ಬಲವಾಗಿ ಬೆಸೆಯಿತು. ಪ್ರತಿಸಲವೂ ನರೇಂದ್ರ ಹೊಸಹೊಸ ವಿಚಾರಗಳನ್ನು, ಹೊಸಹೊಸ ಭಾವನೆಗಳನ್ನು, ಹೊಸಹೊಸ ಆದರ್ಶಗಳನ್ನು ಕಲಿತು ಮೈಗೂಡಿಸಿಕೊಂಡ. ಹೀಗೆ ಬರಬರುತ್ತ ಅವನೀಗ ಆಧ್ಯಾತ್ಮಿಕತೆಯ ವಿಷಯದಲ್ಲಿ ಒಂದು ಪರ್ಯಾಪ್ತ ಸ್ಥಿತಿಯನ್ನು ತಲುಪುತ್ತಿದ್ದಾನೆ. ಶ್ರೀರಾಮಕೃಷ್ಣರೂ ತಮ್ಮಲ್ಲಿರುವ ಆಧ್ಯಾತ್ಮಿಕ ಸಂಪತ್ತನ್ನೆಲ್ಲ ಶಿಷ್ಯನಿಗೆ ಧಾರೆಯೆರೆದು ಕೊಡು ತ್ತಿದ್ದಾರೆ. ಕಾಮಕಾಂಚನವೆಂಬ ಮಹಾಶತ್ರುಗಳನ್ನು ನಿರ್ನಾಮ ಮಾಡಿ ಆಧ್ಯಾತ್ಮಿಕ ಸಾಮ್ರಾಜ್ಯ ವನ್ನು ವಶಪಡಿಸಿಕೊಂಡವರು ಶ್ರೀರಾಮಕೃಷ್ಣರು. ಈ ಆಧ್ಯಾತ್ಮಿಕ ಸಾಮ್ರಾಜ್ಯವನ್ನು ಜಗತ್ತಿನಲ್ಲಿ ವಿಸ್ತರಿಸಬೇಕಾದವನು ನರೇಂದ್ರ. ಪರಮಾತ್ಮನೆಂಬ ಸಾಗರದ ಆಳಕ್ಕೆ ಮುಳುಗಿ ಆಧ್ಯಾತ್ಮಿಕ ಅನುಭವಗಳ ಅಮೂಲ್ಯ ರತ್ನಗಳನ್ನು ಆರಿಸಿ ತಂದವರು ಶ್ರೀರಾಮಕೃಷ್ಣರು. ಆ ದಿವ್ಯ ರತ್ನಗಳ ಪ್ರಭೆಯನ್ನು ತೋರಿಸಿ ಜಗದ ಜನರನ್ನು ಬೆರಗುಗೊಳಿಸಬೇಕಾದವನು ನರೇಂದ್ರ. ಸಕಲ ಸಾಕ್ಷಾತ್ಕಾರಗಳ ಸಾಕಾರಮೂರ್ತಿ ಶ್ರೀರಾಮಕೃಷ್ಣರು; ಆ ಸಾಕ್ಷಾತ್ಕಾರಗಳ ವೈಭವವನ್ನು ಬಣ್ಣಿಸಿ ಪ್ರಸಾರ ಮಾಡಲಿರುವವನು ನರೇಂದ್ರ.

