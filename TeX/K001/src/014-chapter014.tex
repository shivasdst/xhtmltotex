
\chapter{ಮುಂದಾಳ್ತನದ ಮುಂಬೆಳಕು}

\noindent

ಶ್ರೀರಾಮಕೃಷ್ಣರ ಭಕ್ತವೃಂದವೆಂಬುದು ೧೮೮೫-೮೬ರ ವೇಳೆಗೆ ಒಂದು ಸ್ಪಷ್ಟ ಆಕಾರ ತಳೆದಿತ್ತು, ಮತ್ತು ಅವರಲ್ಲಿ ಅಂತರಂಗವಲಯಕ್ಕೆ ಮತ್ತು ಬಹಿರಂಗವಲಯಕ್ಕೆ ಸೇರಿದವರೆಂಬ ಎರಡು ವಿಭಿನ್ನ ಗುಂಪುಗಳನ್ನು ಗುರುತಿಸಬಹುದಾಗಿತ್ತು. ಈ ವೇಳೆಗೆ ಭಕ್ತಾದಿಗಳಲ್ಲಿ ಮತ್ತು ಶಿಷ್ಯರಲ್ಲಿ ಒಂದು ಜಿಜ್ಞಾಸೆ ಎದ್ದಿತು–ಶ್ರೀರಾಮಕೃಷ್ಣರು ಅವತಾರಪುರುಷರೋ ಅಲ್ಲವೋ ಎಂದು. ಆದರೆ ನರೇಂದ್ರನ ಗಮನವೇ ಬೇರೆ ದಿಸೆಯಲ್ಲಿ. ಅವನಿಗೆ ಶ್ರೀರಾಮಕೃಷ್ಣರು ಅವತಾರ ಹೌದೋ ಅಲ್ಲವೋ ಎಂಬುದರಲ್ಲಿ ಆಸಕ್ತಿಯಿಲ್ಲ. ಅವನ ಗಮನೆಲ್ಲ ಶ್ರೀರಾಮಕೃಷ್ಣರ ಶೀಲಪೂರ್ಣ ವ್ಯಕ್ತಿತ್ವದ ಕಡೆಗೆ. ಸಾಕ್ಷಾತ್ಕಾರದ ಹಾದಿಯಲ್ಲಿ ಅತ್ಯಾವಶ್ಯಕವಾದ ಅಂಶವೆಂದರೆ ಈ ಶೀಲ ಅಥವಾ ಚಾರಿತ್ರ್ಯ; ಶಾಸ್ತ್ರಗಳೂ ಅಲ್ಲ, ಸಿದ್ಧಾಂತಗಳೂ ಅಲ್ಲ. ಇತರರೆಲ್ಲರ ದೃಷ್ಟಿಕೋನಕ್ಕಿಂತ ನರೇಂದ್ರನ ದೃಷ್ಟಿಕೋನ ಎಷ್ಟೋ ಪಾಲು ಹೆಚ್ಚು ವಿಶಾಲವಾದದ್ದು. ಉಳಿದವರು ಅವನ ದೃಷ್ಟಿ ಯಿಂದ ನೋಡಲು ಸಾಧ್ಯವೇ ಇರಲಿಲ್ಲ. ಅವನು ಶ್ರೀರಾಮಕೃಷ್ಣರನ್ನು ಇಂತಹ ವಿಶಾಲ ದೃಷ್ಟಿ ಕೋನದಿಂದ ನಿರೀಕ್ಷಿಸಿ ಪರೀಕ್ಷಿಸುತ್ತಿದ್ದ. ಅವರ ಸಮಗ್ರ ವ್ಯಕ್ತಿತ್ವವನ್ನು ಅರಿತುಕೊಳ್ಳಬೇಕು ಎಂಬುದು ಆತನ ಅಭಿಲಾಷೆ. ಇತರ ಶಿಷ್ಯರು-ಭಕ್ತರು ಅವರ ಬೋಧನೆಯ ಅಥವಾ ವ್ಯಕ್ತಿತ್ವದ ಯಾವುದೋ ಒಂದಂಶವನ್ನು ಮಾತ್ರ ಒಪ್ಪಿ ಸ್ವೀಕರಿಸುವುದನ್ನು ಕಂಡು ಅವನು ಅಸಹನೆ ತಾಳು ತ್ತಿದ್ದ. ಅವನು ಶ್ರೀರಾಮಕೃಷ್ಣರನ್ನು ಕೇವಲ ಅವತಾರತತ್ತ್ವದ ಚೌಕಟ್ಟಿನಲ್ಲಿ ಬಂಧಿಸಲು ಇಷ್ಟ ಪಡುತ್ತಿರಲಿಲ್ಲ, ಮತ್ತು ಅವರು ಅವತಾರಪುರುಷರೆಂಬ ಒಂದೇ ಮಾತಿನ ಆಧಾರದ ಮೇಲೆ ಅವರಿಗೆ ಗೌರವ ಸಲ್ಲಿಸುವುದನ್ನೂ ಮೆಚ್ಚುತ್ತಿರಲಿಲ್ಲ. ಜಗತ್ತಿನ ಸಮಸ್ತ ಮತ-ಧರ್ಮಗಳ ಆಧ್ಯಾತ್ಮಿಕ ಆದರ್ಶಗಳಿಗೂ ಶ್ರೀರಾಮಕೃಷ್ಣರ ಜೀವನವು ಮಾರ್ಗಸೂಚಿಯಾಗಿದೆ ಎಂದು ಅವನು ಕಂಡುಕೊಂಡಿದ್ದ. ಅಧ್ಯಾತ್ಮ ಜಗತ್ತಿನಲ್ಲಿ ಹೊಸ ಮಾರ್ಗಗಳನ್ನೇ ನಿರ್ಮಿಸಿದವರು ಅವರು ಎಂದೂ ಅರಿತುಕೊಂಡ. ಅವರ ಒಂದೊಂದು ಉದ್ಗಾರವೂ ಆಧ್ಯಾತ್ಮಿಕ ಜೀವನದ ಅತ್ಯುನ್ನತ ಸಾಧ್ಯತೆಗಳನ್ನು ಪ್ರತಿಬಿಂಬಿಸುವುದೆಂದು ತೀರ್ಮಾನಿಸಿದ. ತನ್ನ ಶರೀರ-ಇಂದ್ರಿಯ-ಮನಸ್ಸು ಗಳಿಂದ ಸೀಮಾಬದ್ಧನಾದ ಮಾನವನು ತನ್ನ ಜೀವನೋದ್ದೇಶಗಳನ್ನು ವಿಸ್ತರಿಸಿಕೊಳ್ಳುವುದರ ಮೂಲಕ ಸೀಮಾತೀತನಾಗಬಲ್ಲ ಎಂಬ ಸತ್ಯವನ್ನು ಶ್ರೀರಾಮಕೃಷ್ಣರ ಸನ್ನಿಧಿಯಲ್ಲಿ ಕಂಡು ಕೊಂಡ. ನಿಜಕ್ಕೂ ಜಗತ್ತಿನ ಎಲ್ಲ ಸಂತರ ಪ್ರಯತ್ನವೂ ಇದೇ–ಸಾಂತದಿಂದ ಅನಂತದೆಡೆಗೆ ಧಾವಿಸುವ ಪ್ರಯತ್ನ; ಬಂಧನದಿಂದ ಸ್ವಾತಂತ್ರ್ಯದೆಡೆಗೆ ಧಾವಿಸುವ ಪ್ರಯತ್ನ. ಆದರೆ ಶ್ರೀರಾಮಕೃಷ್ಣರಲ್ಲಿ ಈ ಪ್ರಯತ್ನ ಪರಾಕಾಷ್ಠೆಗೆ ತಲುಪಿರುವುದನ್ನು ಅವನು ನೋಡಿದ. ಅಲ್ಲದೆ ಅವರು ಸನಾತನ ಧರ್ಮದ ಪುನರುದ್ಧಾರಕರು, ಸನಾತನ ಧರ್ಮದ ಪ್ರತಿಷ್ಠಾಪಕರು ಎಂದು ಗುರುತಿಸಿದ. ಅವರಲ್ಲಿ ಒಬ್ಬ ನೂತನ ಬುದ್ಧನನ್ನು, ನೂತನ ಶಂಕರಾಚಾರ್ಯರನ್ನು, ನೂತನ ಚೈತನ್ನದೇವನನ್ನು–ಅಷ್ಟೇ ಅಲ್ಲ, ಇನ್ನೂ ಹೆಚ್ಚಿನ ಶಕ್ತಿಯನ್ನೂ ಮಹತ್ವವನ್ನೂ ನರೇಂದ್ರ ಕಂಡುಕೊಂಡ. ನಿಜಕ್ಕೂ ಹದಿನೆಂಟು-ಹತ್ತೊಂಬತ್ತನೆಯ ಶತಮಾನಗಳಲ್ಲಿ ಸನಾತನ ಧರ್ಮಕ್ಕೆ ಉಂಟಾಗಿದ್ದ ಗ್ಲಾನಿ ಬಹು ದೊಡ್ಡದು. ಏಕೆಂದರೆ ಹಿಂದಿನ ಯುಗಗಳಲ್ಲಿ ಅನೈತಿಕತೆ, ಅನಾಚಾರ ಗಳಿಂದಾಗಿ ಧರ್ಮಗ್ಲಾನಿಯಾಗುತ್ತಿದ್ದರೆ, ಪ್ರಸ್ತುತ ಸಂದರ್ಭದಲ್ಲಿ ಪರರಾಷ್ಟ್ರೀಯರ, ವಿಧರ್ ಮೀಯರ ಆಕ್ರಮಣದಿಂದಾಗಿ ಸನಾತನ ಧರ್ಮವು ಬುಡಸಮೇತ ಲುಪ್ತವಾಗುವ ಅಪಾಯಕ್ಕೆ ಗುರಿಯಾಗಿತ್ತು. ಈ ಹಿನ್ನಲೆಯಲ್ಲಿ ಇಂದಿನ ಅವತಾರದ ವಿಶೇಷ ಹೊಣೆಗಾರಿಕೆಯನ್ನು ಅವನು ಸ್ಪಷ್ಟವಾಗಿ ಗುರುತಿಸಿದ. ಹಿಂದಿನ ಯುಗಗಳಲ್ಲಿ ಅವತಾರಪುರುಷರೂ ಆಚಾರ್ಯಪುರುಷರೂ ಭಾರತದಲ್ಲಷ್ಟೇ ಧರ್ಮಸಂಸ್ಥಾಪನೆಯ ಕಾರ್ಯವನ್ನು ಮಾಡಿದ್ದರೆ ಸಾಕಾಗಿತ್ತು. ಭಗವಾನ್ ಬುದ್ಧನ ಕಾಲದಲ್ಲಿ ದರ್ಮಪ್ರಸಾರಕಾರ್ಯವು ಇತರ ಪೌರ್ವಾತ್ಯ ರಾಷ್ಟ್ರಗಳಲ್ಲೂ ಮುಂದುವರಿ ಯಿತು. ಆದರೆ ಇಂದು ಶ್ರೀರಾಮಕೃಷ್ಣಾವತಾರದ ಕಾಲದಲ್ಲಿ ಈ ಧರ್ಮಪ್ರಸಾರಕಾರ್ಯವು ವಿಶ್ವದಾದ್ಯಂತ ನಡೆಯಬೇಕಾದ ಆವಶ್ಯಕತೆಯಿದೆ. ಅಲ್ಲದೆ, ಇಂತಹ ಒಂದು ಬೃಹತ್ ಕಾರ್ಯ ವನ್ನು ಯಶಸ್ವಿಯಾಗಿಸಲು ಬೇಕಾದ ಅನಂತ ಆಧ್ಯಾತ್ಮಿಕ ಶಕ್ತಿ ಶ್ರೀರಾಮಕೃಷ್ಣರಲ್ಲಿದೆ ಎಂಬುದು ಕ್ರಮೇಣ ನರೇಂದ್ರನ ಅರಿವಿಗೆ ಬಂದಿತು. ಈ ಎಲ್ಲ ವಿಚಾರಗಳೂ ನರೇಂದ್ರನಲ್ಲಿ ಹುಟ್ಟಿದುದು ದಕ್ಷಿಣೇಶ್ವರದ ದಿನಗಳಲ್ಲಿ. ಮುಂದೆ ಕಾಶೀಪುರದ ಉದ್ಯಾನವನದ ದಿನಗಳಲ್ಲಿ, ಎಂದರೆ ಶ್ರೀರಾಮಕೃಷ್ಣರ ಜೀವಿತಾವಧಿಯ ಕಡೆಯ ಸುಮಾರು ಒಂದು ವರ್ಷದಲ್ಲಿ ಅವನೊಳಗೆ ಈ ಭಾವನೆಗಳೆಲ್ಲ ಪಕ್ವಗೊಂಡು ನಿಶ್ಚಿತ ರೂಪ ತಳೆದುವು.

೧೮೮೫ರ ಬೇಸಿಗೆಯಲ್ಲಿ ತೀವ್ರ ಉಷ್ಣದಿಂದಾಗಿ ಶ್ರೀರಾಮಕೃಷ್ಣರಿಗೆ ಅತೀವ ಕಷ್ಟವೆನಿಸಿತು. ಇದನ್ನು ಕಂಡು ಭಕ್ತರು ಅವರಿಗೆ ಮಂಜುಗೆಡ್ಡೆ ಸೇವಿಸಲು ಸಲಹೆ ಮಾಡಿದರು. ಶ್ರೀರಾಮಕೃಷ್ಣರು ಹಾಗೆಯೇ ಮಾಡಿದರು. ಇದರಿಂದಾಗಿ ಅವರಿಗೆ ಎಷ್ಟೋ ಅನುಕೂಲವಾದಂತೆ ಅನ್ನಿಸಿತು. ಇದನ್ನು ಕಂಡು ಭಕ್ತಾದಿಗಳೆಲ್ಲ ದಕ್ಷಿಣೇಶ್ವರಕ್ಕೆ ಬರುವಾಗ ಮಂಜುಗೆಡ್ಡೆ ತರಲಾರಂಭಿಸಿದರು. ಶ್ರೀರಾಮ ಕೃಷ್ಣರು ಆ ಮಂಜುಗಡ್ಡೆ ಬೆರೆಸಿದ ಪಾನಕವನ್ನು ಮುಗ್ಧ ಬಾಲಕನಂತೆ ಸಂತೋಷದಿಂದ ಕುಡಿಯು ತ್ತಿದ್ದರು. ಆದರೆ ಸುಮಾರು ಒಂದೆರಡು ತಿಂಗಳ ಬಳಿಕ ಅವರ ಗಂಟಲಲ್ಲಿ ನೋವು ಕಾಣಿಸಿ ಕೊಂಡಿತು. ಮತ್ತೆರಡು ತಿಂಗಳಾದರೂ ನೋವು ಕಡಿಮೆಯಾಗುವ ಸೂಚನೆಯೇ ಕಂಡು ಬರಲಿಲ್ಲ. ಅಲ್ಲದೆ ಮೇ ತಿಂಗಳ ಹೊತ್ತಿಗೆ ಆ ಕಾಯಿಲೆಯಲ್ಲಿ ಕೆಲವು ಹೊಸ ಲಕ್ಷಣಗಳು ಕಾಣಿಸಿ ಕೊಂಡುವು. ಜೊತೆಗೆ, ಶ್ರೀರಾಮಕೃಷ್ಣರು ಹೆಚ್ಚಾಗಿ ಮಾತನಾಡಿದಾಗ ನೋವು ಉಲ್ಬಣಿಸುತ್ತಿತ್ತು. ಇವೆಲ್ಲ ಥಂಡಿಯ ದೆಸೆಯಿಂದ ಉಂಟಾದ ಉರಿಯೂತದ ಪರಿಣಾಮ ಎಂದು ಭಾವಿಸಿ ಅದಕ್ಕೆ ಪಟ್ಟಿಯನ್ನು ಕಟ್ಟಿದರು. ಆದರೆ ಹಲವು ದಿನಗಳು ಕಳೆದರೂ ಅದರಿಂದ ಏನೂ ಪ್ರಯೋಜನ ಕಂಡುಬರಲಿಲ್ಲ. ಆಗ ಒಬ್ಬ ಭಕ್ತ ಕಲ್ಕತ್ತದಿಂದ ಡಾಕ್ಟರ್ ರಾಖಾಲ್ ಎಂಬವರನ್ನು ಕರೆತಂದ. ಅವರು ಚೆನ್ನಾಗಿ ಪರೀಕ್ಷೆ ಮಾಡಿನೋಡಿ ಗಂಟಲ ಒಳಗೂ ಹೊರಗೂ ಹಚ್ಚಲು ಔಷಧವನ್ನು ಕೊಟ್ಟರು. ಬಳಿಕ, ಶ್ರೀರಾಮಕೃಷ್ಣರು ಹೆಚ್ಚಾಗಿ ಮಾತನಾಡದಂತೆ ಮತ್ತು ಸಮಾಧಿಸ್ಥಿತಿಗೆ ಏರದಂತೆ ನೋಡಿಕೊಳ್ಳಬೇಕು ಎಂದು ಅವರ ಆರೈಕೆ ಮಾಡುತ್ತಿದ್ದ ಪರಿಚಾರಕರಿಗೆ ಎಚ್ಚರಿಕೆ ಹೇಳಿದರು. ಪರಿಚಾರಕರು ‘ಆಗಲಿ’ ಎಂದೇನೋ ಹೇಳಿದರು. ಆದರೆ ಶ್ರೀರಾಮಕೃಷ್ಣರನ್ನು ಸಮಾಧಿಗೇರದಂತೆ ತಡೆಯಲು ಯಾರಿಂದತಾನೆ ಸಾಧ್ಯ!

ಜ್ಯೇಷ್ಠ ಮಾಸದ ತ್ರಯೋದಶಿಯ ದಿನ. (ಮೇ-ಜೂನ್ ತಿಂಗಳು) ಅಂದು ದಕ್ಷಿಣೇಶ್ವರಕ್ಕೆ ಕೆಲವು ಮೈಲಿ ದೂರವಿರುವ ಪಾನಿಹಾಟಿಯಲ್ಲಿ ಜಾತ್ರೆ. ಭಕ್ತರಿಗೆಲ್ಲ ಶ್ರೀರಾಮಕೃಷ್ಣರನ್ನು ಜೊತೆ ಗೂಡಿಕೊಂಡು ಈ ಜಾತ್ರೆಗೆ ಹೋಗಬೇಕು ಎಂಬ ಇಚ್ಛೆ. ಶ್ರೀರಾಮಕೃಷ್ಣರು ಒಪ್ಪಿಕೊಂಡು ಹೊರಟರು; ಅಲ್ಲಿ ಇಡೀ ದಿನವನ್ನು ಭಾವಾವೇಶದಲ್ಲೇ ಕಳೆದರು–ಹಾಡಿದರು, ನರ್ತಿಸಿದರು, ಮತ್ತೆಮತ್ತೆ ಸಮಾಧಿಗೇರಿದರು.\footnote{*ಅಂದಿನ ಉತ್ಸವದ ವಿವರಗಳಿಗೆ ನೋಡಿ: `ಯುಗಾವತಾರ ಶ್ರೀರಾಮಕೃಷ್ಣ' (ಸಂಪುಟ ೪).} ಅಲ್ಲಿನ ಆ ಪ್ರೇಮೋತ್ಸಾಹದ ವಾತಾವರಣದಲ್ಲಿ ಶ್ರೀರಾಮ ಕೃಷ್ಣರನ್ನು ಭಾವಸ್ಥಿತಿಗೇರದಂತೆ ತಡೆಯಲು ಯಾರಿಂದಲೂ ಸಾಧ್ಯವಾಗಲಿಲ್ಲ. ಆದರೆ ಜಾತ್ರೆ ಮುಗಿಸಿಕೊಂಡು ಬರುತ್ತಿದ್ದಂತೆಯೇ ಗಂಟಲ ಕಾಯಿಲೆ ಹೆಚ್ಚಾಗಿಬಿಟ್ಟಿತು. ವೈದ್ಯರು ಬಂದು ಮತ್ತೆ ಪರೀಕ್ಷೆ ಮಾಡಿನೋಡಿ ಔಷಧ ಬದಲಾಯಿಸಿ ಕೊಟ್ಟರು, ಪಥ್ಯ-ಪಾನ ವಿಧಿಸಿದರು. ಅವರು ಹೇಳಿದ ಎಲ್ಲ ನಿಯಮಗಳನ್ನೂ ಶ್ರೀರಾಮಕೃಷ್ಣರು ಪರಿಪಾಲಿಸಿದರು–ಎರಡು ವಿಷಯಗಳನ್ನು ಬಿಟ್ಟು, ಒಂದು ಸಮಾಧಿಗೇರುವುದು; ಎರಡು, ಭಕ್ತರೊಡನೆ ಮಾತನಾಡುವುದು. ಭಗವ ತ್ಸಂಬಂಧವಾದ ಮಾತುಕತೆ ನಡೆದಾಗ, ಭಾವಭರಿತವಾದ ಕೀರ್ತನೆಯನ್ನು ಕೇಳಿದಾಗ ತಕ್ಷಣ ಭಾವಸಮಾಧಿಗೇರದಿರಲು ಅವರಿಂದ ಸಾಧ್ಯವೇ ಇಲ್ಲ. ಅಂತೆಯೇ ಆಧ್ಯಾತ್ಮಿಕ ಮಾರ್ಗದರ್ಶನ ವನ್ನು ಅರಿಸಿ ಬಂದ, ಜೀವನದಲ್ಲಿ ನೊಂದು ಬೆಂದು ಸಾಂತ್ವನವನ್ನು ಬಯಸಿ ಬಂದ ಜನರನ್ನು ಕಂಡಾಗ ಅವರಿಗೆ ನಾಲ್ಕು ಸಮಾಧಾನದ ಮಾತುಗಳನ್ನು ಹೇಳದಿರಲು ಶ್ರೀರಾಮಕೃಷ್ಣರಿಂದ ಸಾಧ್ಯವೇ ಇಲ್ಲ. ಇದರಿಂದ ತಮ್ಮ ಗಂಟಲ ಮೇಲೆ ಎಂತಹ ದುಷ್ಪರಿಣಾಮವಾಗಬಹುದೆಂಬ ಅರಿವಿದ್ದರೂ ಮಾತನಾಡುತ್ತಲೇ ಇದ್ದರು–ಭಕ್ತಾನುಕಂಪೆ! ಅಕಾಲದಲ್ಲೆಲ್ಲ ಭಕ್ತರು ಬರುತ್ತಿ ದ್ದುದರಿಂದ ಮತ್ತು ಯಾವಾಗೆಂದರೆ ಆಗ ಸಮಾಧಿಸ್ಥಿತಿಯುಂಟಾಗುತ್ತಿದ್ದುದರಿಂದ ಅವರ ಪಥ್ಯ-ಪಾನಗಳು ಸಮಯಕ್ಕೆ ಸರಿಯಾಗಿ ನಡೆಸಲು ಸಾಧ್ಯವಾಗುತ್ತಿರಲಿಲ್ಲ. ಆದರೂ ಶ್ರೀರಾಮ ಕೃಷ್ಣರೆಂದೂ ಬೇಸರಿಸದೆ, ಭಕ್ತರ ಹಿತಕ್ಕಾಗಿ ತಮ್ಮ ಶಕ್ತಿಯ ಪ್ರತಿಯೊಂದು ತುಣುಕನ್ನೂ ವ್ಯಯಿಸುತ್ತಲೇ ಹೋದರು.

ದಿನದಿಂದ ದಿನಕ್ಕೆ ಗಂಟಲುಬೇನೆ ಹೊಸ ರೂಪ ತಾಳುತ್ತಿದೆ; ಪರಿಸ್ಥಿತಿ ಗಂಭೀರವಾಗುತ್ತಿದೆ. ಭಕ್ತರೆಲ್ಲ ಆತಂಕಗೊಂಡಿದ್ದಾರೆ. ನರೇಂದ್ರ ಶ್ರೀರಾಮಕೃಷ್ಣರ ದೇಹಸ್ಥಿತಿಯನ್ನು ಗಮನಿಸುತ್ತಲೇ ಇದ್ದಾನೆ. ಒಂದು ದಿನ, ಅವರ ಗಂಟಲಿನಿಂದ ರಕ್ತಸ್ರಾವವಾಯಿತೆಂದು ತಿಳಿದಾಗ ತನ್ನ ಸ್ನೇಹಿತ ನೊಬ್ಬನ ಬಳಿ ಹೇಳುತ್ತಾನೆ: “ಬಹುಶಃ ನಮ್ಮೆಲ್ಲರ ಪ್ರೀತಿಪಾತ್ರವಾದ ವಸ್ತು ನಮ್ಮ ಕೈತಪ್ಪಿ ಹೋಗುತ್ತದೆ ಅಂತ ನನಗನ್ನಿಸುತ್ತಿದೆ. ಈ ಕಾಯಿಲೆಯ ಬಗ್ಗೆ ನಾನು ವೈದ್ಯಕೀಯ ಗ್ರಂಥಗಳನ್ನು ಓದಿದ್ದೇನೆ ಮತ್ತು ಕೆಲವು ವೈದ್ಯ ಸ್ನೇಹಿತರನ್ನು ಕೇಳಿದ್ದೇನೆ. ಅವರೆಲ್ಲ ಹೇಳುವ ಪ್ರಕಾರ ಈ ಗಂಟಲುಬೇನೆ ಕ್ಯಾನ್ಸರ್ಗೆ ತಿರುಗಿಕೊಂಡರೂ ಆಶ್ಚರ್ಯವಿಲ್ಲ... ಕ್ಯಾನ್ಸರ್ಗೆ ಔಷಧವನ್ನು ಇನ್ನೂ ಯಾರೂ ಕಂಡುಹಿಡಿದಿಲ್ಲ... ”

ಶ್ರೀರಾಮಕೃಷ್ಣರ ಕಾಯಿಲೆ ದಕ್ಷಿಣೇಶ್ವರದಲ್ಲಿ ಸುಧಾರಿಸದಿದ್ದುದರಿಂದ ಅವರನ್ನು ಇನ್ನೂ ಉತ್ತಮ ಚಿಕಿತ್ಸೆಗಾಗಿ ಕಲ್ಕತ್ತಕ್ಕೆ ಕರೆದುಕೊಂಡುಹೋಗಬೇಕೆಂದು ಭಕ್ತರು ಯೋಚಿಸಿದರು. ಇದಕ್ಕೆ ಶ್ರೀರಾಮಕೃಷ್ಣರು ಕೂಡಲೇ ಒಪ್ಪಿದರು. ಸರಿ, ಕಲ್ಕತ್ತದ ಬಾಗ್ಬಜಾರಿನಲ್ಲಿ ಒಂದು ಪುಟ್ಟ ಮನೆಯನ್ನು ಗೊತ್ತುಮಾಡಿ, ಅಲ್ಲಿಗೆ ಕರೆದುಕೊಂಡು ಹೋಗಲಾಯಿತು. ಆದರೆ ಶ್ರೀರಾಮ ಕೃಷ್ಣರಿಗೆ ಆ ಮನೆ ಸ್ವಲ್ಪವೂ ಇಷ್ಟವಾಗದೆ, ಕೂಡಲೇ ಅಲ್ಲಿಂದ ಹೊರಟು ತಮ್ಮ ಆಪ್ತ ಭಕ್ತನಾದ ಬಲರಾಮ ಬೋಸನ ಮನೆಗೆ ಬಂದುಬಿಟ್ಟರು. ಬಳಿಕ ಒಂದು ವಾರದಲ್ಲೇ ಶ್ಯಾಮಪುಕುರ ಮೊಹಲ್ಲದಲ್ಲಿ ಒಂದು ಮನೆಯನ್ನು ಬಾಡಿಗೆಗೆ ಗೊತ್ತುಮಾಡಿ, ಶ್ರೀರಾಮಕೃಷ್ಣರನ್ನು ಅಲ್ಲಿಗೆ ಕರೆತರಲಾಯಿತು. ಈ ಮನೆ ತಕ್ಕಮಟ್ಟಿಗೆ ಉತ್ತಮವಾಗಿತ್ತು. ಈಗ ಕಲ್ಕತ್ತದ ಪ್ರಸಿದ್ಧ ಹೋಮಿ ಯೋಪಥಿ ವೈದ್ಯನಾದ ಡಾ ॥ ಮಹೇಂದ್ರಲಾಲ್ ಸರ್ಕಾರ್ ಎಂಬವನು ಶ್ರೀರಾಮಕೃಷ್ಣರಿಗೆ ಚಿಕಿತ್ಸೆ ನೀಡಲು ಒಪ್ಪಿಕೊಂಡ. ನರೇಂದ್ರ ತನ್ನ ಸೋದರಶಿಷ್ಯರ ಸಹಕಾರದಿಂದ ಅವರ ಶುಶ್ರೂಷೆಗೆ ತಕ್ಕ ವ್ಯವಸ್ಥೆ ಮಾಡಿದ. ಊಟೋಪಚಾರಗಳನ್ನು ಮಾಡಲು ದಕ್ಷಿಣೇಶ್ವರದಿಂದ ಶ್ರೀಶಾರದಾದೇವಿ ಯವರನ್ನು ಕರೆತರಲಾಯಿತು. ನರೇಂದ್ರನು ತಮ್ಮ ಗುರುದೇವನಿಗೆ ಅತೀವ ನಿಷ್ಠೆಯಿಂದ, ಉತ್ಸಾಹದಿಂದ ಸೇವೆ ಮಾಡುವ ಕ್ರಮವನ್ನು ನೋಡಿ ಇತರ ಶಿಷ್ಯರಲ್ಲೂ ಹೊಸ ಶಕ್ತಿ-ಉತ್ಸಾಹ ಗಳು ಉದಿಸಿದುವು. ತಮ್ಮ ಮನೆಗಳನ್ನು ಮರೆತು, ತನುಮನಪೂರ್ವಕವಾಗಿ ಗುರುಸೇವೆಯಲ್ಲಿ ತೊಡಗುವ ನಿರ್ಧಾರ ಮಾಡಿದರು, ಮತ್ತು ನರೇಂದ್ರನ ಬುದ್ಧಿವಾದದಂತೆ ಶ್ರದ್ಧಾಭಕ್ತಿಗಳಿಂದ ಗುರುಸೇವೆ ಮಾಡಿ, ತನ್ಮೂಲಕ ಭಗವತ್ಸಾಕ್ಷಾತ್ಕಾರ ಮಾಡಿಕೊಳ್ಳುವ ಪ್ರಯತ್ನದಲ್ಲಿ ನಿರತ ರಾದರು. ಆದರೆ ಅವರು ಹೀಗೆ ಮನೆ, ಕಾಲೇಜು, ಅಧ್ಯಯನವನ್ನೆಲ್ಲ ಬದಿಗೊತ್ತಿದುದನ್ನು ಕಂಡು ಅಸಮಾಧಾನಗೊಂಡ ತಂದೆತಾಯಂದಿರು ಬಂದು ಅವರನ್ನೆಲ್ಲ ಕರೆದುಕೊಂಡು ಹೋಗುವ ಪ್ರಯತ್ನ ಮಾಡಿದರು. ಬಹುಶಃ ನರೇಂದ್ರನ ಕುಮ್ಮಕ್ಕು-ಪ್ರೋತ್ಸಾಹ ಇಲ್ಲದೆ ಹೋಗಿದ್ದರೆ ಆ ಶಿಷ್ಯರೆಲ್ಲ ತಂದೆತಾಯಂದಿರ ಬಲವಂತಕ್ಕೆ ಮಣಿದು, ಕೈಗೊಂಡ ಸೇವೆಯನ್ನು ಅಲ್ಲಿಗೆ ಬಿಟ್ಟು ಹೊರಟುಬಿಡುತ್ತಿದ್ದರೇನೋ. ಆದರೆ ನರೇಂದ್ರನ ಬಲದ ಮುಂದೆ ಆ ತಂದೆ ತಾಯಂದಿರ ಪ್ರಯತ್ನ ನಡೆಯಲಿಲ್ಲ.

ಈ ಶಿಷ್ಯರೇನೋ ತಾಯ್ತಂದೆಯರ ಕರೆಯನ್ನೂ ನಿರಾಕರಿಸಿ ಶ್ಯಾಮಪುಕುರದಲ್ಲೇ ನಿಂತಿರ ಬಹುದು. ಆದರೆ ಅವರಿಗೆಲ್ಲ ಊಟ-ತಿಂಡಿಗಳ ವ್ಯವಸ್ಥೆಯಾಗಬೇಕಲ್ಲ. ಶ್ರೀರಾಮಕೃಷ್ಣರ ಚಿಕಿ ತ್ಸೆಗೂ ಹಣ ಬೇಕು, ಇವರೆಲ್ಲರ ಅಶನಾದಿಗಳಿಗೂ ಹಣ ಬೇಕು, ಮನೆಯ ಬಾಡಿಗೆಯನ್ನೂ ಕೊಡಬೇಕು. ಶ್ರೀರಾಮಕೃಷ್ಣರ ಬಳಿ ಕೂಡಿಟ್ಟ ಹಣವಿದೆಯೆ? ಅವರು ತಿಂಗಳಿಗೆ ಏಳು ರೂಪಾಯಿ ಸಂಬಳದ ಮೇಲೆ ಜೀವಿಸಿದವರು! ಕೆಲವು ಮಂದಿ ಅನಕೂಲಸ್ಥ ಭಕ್ತರೇನೋ ಇದ್ದರು. ಆದರೆ ಇಷ್ಟೊಂದು ಜನರಿಗೆ ತಿಂಗಳುಗಟ್ಟಲೆ ತಿಂಡಿತೀರ್ಥ ಕೊಟ್ಟು ಪೂರೈಸುವುದು ಸುಲಭವೆ? ಕೆಲವೊಮ್ಮೆ ಭಕ್ತರೂ ಶಿಷ್ಯರೂ ಅಧೈರ್ಯಕ್ಕೊಳಗಾಗುತ್ತಿದ್ದರು; ಮನಸ್ಸು ವಿಚಲಿತಗೊಳ್ಳು ತ್ತಿತ್ತು. ಮುಂದೆ ಏನು ಗತಿ ಎಂದು ಚಿಂತಿಸುತ್ತಿದ್ದರು. ಅವರೆಲ್ಲ ಹೀಗೆ ಹೆದರಿ ಅಸಹಾಯಕ ರಾಗಿರುವಾಗ ಶ್ರೀರಾಮಕೃಷ್ಣರ ದೈವೀಶಕ್ತಿಗೆ ಹೊಸ ಸಾಕ್ಷ್ಯವೊಂದನ್ನು ಕಾಣುತ್ತಿದ್ದರು. ಆಗ ಅವರು ತಮಗೆ ತಾವೇ ಬೈದುಕೊಳ್ಳುತ್ತಿದ್ದರು–“ಛೆ, ಇದೇನಿದು! ನಮ್ಮದು ಇದೆಂತಹ ಅಪ ನಂಬಿಕೆ! ಹಣದ ಬಗ್ಗೆ ನಮಗೇಕೆ ಚಿಂತೆ? ಶ್ರೀರಾಮಕೃಷ್ಣರೇ ಖಂಡಿತವಾಗಿ ಎಲ್ಲವನ್ನೂ ಒದಗಿಸಿಕೊಡುತ್ತಾರೆ.” ಶ್ರೀರಾಮಕೃಷ್ಣರಿಗೆ ಯಾವುದೇ ರೂಪದಿಂದ ಸೇವೆ ಮಾಡಿದರೂ ಅದ ರಿಂದ ತಮ್ಮ ಆಧ್ಯಾತ್ಮಿಕ ಉನ್ನತಿಯಾಗುವುದರಲ್ಲಿ ಸಂದೇಹವಿಲ್ಲ ಎಂಬುದು ಅವರಿಗೆಲ್ಲ ಮನದಟ್ಟಾಯಿತು. ಅಲ್ಲದೆ ಶ್ರೀರಾಮಕೃಷ್ಣರಿಗೆ ಈಗ ಬಂದಿರುವ ಈ ಕಾಯಿಲೆ ತಮ್ಮೆಲ್ಲರಿಗೂ ಯಥಾಶಕ್ತಿ ಗುರುಸೇವೆ ಮಾಡಲು ಒದಗಿರುವ ಒಂದು ಸದವಕಾಶ ಎಂದು ಆ ಭಕ್ತರೂ ಶಿಷ್ಯರೂ ಭಾವಿಸಿದರು. ಆದ್ದರಿಂದ ಈಗ ಗುರುಸೇವೆಗಾಗಿ ಸಾಧ್ಯವಾದಷ್ಟು ತಮ್ಮ ಹಣವನ್ನು ವಿನಿ ಯೋಗಿಸಲು ಭಕ್ತರು ನಿಶ್ಚಯಿಸಿದರು. ಯುವಕಶಿಷ್ಯರೂ ತಮ್ಮ ಶಕ್ತಿಮೀರಿ ಶ್ರೀರಾಮಕೃಷ್ಣರ ವೈಯಕ್ತಿಕ ಸೇವೆಯಲ್ಲಿ ತೊಡಗಲು ಪಣತೊಟ್ಟು ನಿಂತರು. ಅಲ್ಲದೆ, ಶ್ರೀರಾಮಕೃಷ್ಣರ ಅಪೂರ್ವ ಆಧ್ಯಾತ್ಮಿಕ ಶಕ್ತಿ ವಿಶೇಷವಾಗಿ ಪ್ರಕಟವಾಗುವುದನ್ನು ಕಾಣುತ್ತ ಅವರ ಆನಂದೋತ್ಸಾಹಗಳು ಇಮ್ಮಡಿಯಾದುವು. ಶ್ರೀರಾಮಕೃಷ್ಣರು ದಕ್ಷಿಣೇಶ್ವರದಲ್ಲಿರುವವರೆಗೂ ಎಷ್ಟೋ ಭಕ್ತರಿಗೆ ಅಲ್ಲಿ ಯವರೆಗೆ ಹೋಗಲು ಸಾಧ್ಯವಾಗುತ್ತಿರಲಿಲ್ಲ. ಈಗ ಶ್ಯಾಮಪುಕುರಕ್ಕೆ ಬಂದು ದರ್ಶನ ಪಡೆದು ಕೊಂಡು ಹೋಗಲು ಎಲ್ಲರಿಗೂ ಬಹಳ ಅನುಕೂಲವಾಯಿತು. ಕಾಯಿಲೆಯ ನೆವದಿಂದ ಅವರು ‘ಕಾಡಿನಿಂದ ನಾಡಿಗೆ’ ಬಂದದ್ದು ಭಕ್ತಜನರ ಭಾಗ್ಯವಾಗಿ ಪರಿಣಮಿಸಿತು.

ಈ ಮಧ್ಯೆ ಭಕ್ತವೃಂದದಲ್ಲೊಂದು ದೊಡ್ಡ ಜಿಜ್ಞಾಸೆಯೆದ್ದಿದೆ–‘ಶ್ರೀರಾಮಕೃಷ್ಣರಿಗೆ ಇಂತಹ ಕಾಯಿಲೆ ಏಕಾದರೂ ಬಂದಿರಬಹುದು?’ ಎಲ್ಲರೂ ತಮ್ಮತಮ್ಮದೇ ಆದ ರೀತಿಯಲ್ಲಿ ಊಹಿಸು ತ್ತಿದ್ದಾರೆ. ಇದು ತೀರ ಸಹಜವೇ. ಏಕೆಂದರೆ, ಪರಮಯೋಗಿಗಳಾದ ಶ್ರೀರಾಮಕೃಷ್ಣರ ಶರೀರಕ್ಕೆ, ಆ ಪರಮ ಪರಿಶುದ್ಧ ಶರೀರಕ್ಕೆ ಕಾಯಿಲೆ ಬರಲು ಹೇಗೆತಾನೆ ಸಾಧ್ಯ! ಆದರೆ ಇದಕ್ಕೊಂದು ಕಾರಣ ಇರಬೇಕಲ್ಲವೆ?–ಎಂಬುದು ಭಕ್ತರ ಕುತೂಹಲ, ಜಿಜ್ಞಾಸೆ. ಅವರಲ್ಲಿ ಒಂದು ಗುಂಪಿನವರು ಹೇಳುತ್ತಿದ್ದರು–“ಇದೆಲ್ಲ ಜಗನ್ಮಾತೆಯ ಇಚ್ಛೆಯಲ್ಲದೆ ಬೇರೇನು! ಶ್ರೀರಾಮ ಕೃಷ್ಣರು ಸಂಪೂರ್ಣವಾಗಿ ಆಕೆಯ ಮೇಲೆಯೇ ನಿರ್ಭರರಾಗಿ, ಅವಳ ಉಪಕರಣವೇ ಆಗಿ ಬಿಟ್ಟಿರುವವರಲ್ಲವೆ? ಆದ್ದರಿಂದ, ಯಾವುದೋ ಒಂದು ಉದ್ದೇಶವನ್ನು ಪರಿಪೂರ್ಣಗೊಳಿಸುವು ದಕ್ಕೋಸ್ಕರ ಅವಳೇ ಈ ಕಾಯಿಲೆಯನ್ನು ಬರಿಸಿದ್ದಾಳೆ” ಎಂದು. ಇನ್ನೊಂದು ಗುಂಪಿನವರು ಹೇಳುತ್ತಿದ್ದರು: “ಇಲ್ಲ, ಇಲ್ಲ; ಈ ಕಾಯಿಲೆಯನ್ನು ಶ್ರೀರಾಮಕೃಷ್ಣರು ತಾವೇ ತಂದುಕೊಂಡಿ ದ್ದಾರೆ. ಅವರ ಪ್ರತಿಯೊಂದು ಕ್ರಿಯೆಯೂ ಜನತೆಯ ಹಿತಕ್ಕಾಗಿಯೇ. ಆದ್ದರಿಂದ ಇದನ್ನೂ ಅವರು ತಮ್ಮ ಮೇಲೆ ತಾವೇ ತಂದುಕೊಂಡಿದ್ದಾರೆ, ಅಷ್ಟೆ.” ಆದರೆ ಮೂರನೆಯ ಗುಂಪಿನವರು ಬೇರೆಯೇ ರೀತಿಯಲ್ಲಿ ಯೋಚಿಸಿ ಒಂದು ನಿರ್ಧಾರಕ್ಕೆ ಬಂದಿದ್ದರು–ಏನೆಂದರೆ, ಜನ್ಮ- ಜರಾ-ರೋಗಗಳು ಮನುಷ್ಯ ಶರೀರಕ್ಕೆ ಸಹಜವಾಗಿಯೇ ಸಂಭವಿಸುವಂಥವುಗಳು; ಇವಕ್ಕೆಲ್ಲ ಅತೀಂದ್ರಿಯವಾದ ಕಾರಣಗಳನ್ನು ಹುಡುಕುವುದಾಗಲಿ, ದೈವಿಕ ಕಾರಣವನ್ನು ಕೊಡುವುದಾಗಲಿ ಅನಗತ್ಯ ಎಂದು. ಈ ರೀತಿಯಾಗಿ ಚಿಂತಿಸಿದವರು ಸುಶಿಕ್ಷಿತರಾದ ಯುವಕಶಿಷ್ಯರು. ಆದರೆ ಅವರು ಈ ಕಾಯಿಲೆ ಮಾನವಸಹಜವಾದದ್ದೆಂದು ನಂಬಿದ್ದರೂ, ಗುರು ಮಹಿಮೆಯಲ್ಲಿ ಕಿಂಚಿತ್ತೂ ಅವಿಶ್ವಾಸ ತಾಳದೆ ತಮ್ಮ ಸರ್ವಸ್ವವನ್ನೂ ತ್ಯಾಗ ಮಾಡಿ, ಶ್ರೀರಾಮಕೃಷ್ಣರ ಸೇವೆ ಮಾಡಲು ಟೊಂಕ ಕಟ್ಟಿ ನಿಂತಿದ್ದರು. ಹೀಗೆ ಗುರುಸೇವೆಯ ಮೂಲಕವೇ ಭಗವತ್ಸಾಕ್ಷಾತ್ಕಾರವೆಂಬ ಆದರ್ಶ ವನ್ನು ಸಿದ್ಧಿಸಿಕೊಳ್ಳುವ ಸಂಕಲ್ಪ ಮಾಡಿದ್ದರು. ಇವರಿಗೆಲ್ಲ ನರೇಂದ್ರನೇ ನಾಯಕನೆಂದು ಹೇಳಬೇಕಾಗಿಯೇ ಇಲ್ಲ. ಹೀಗೆ ಈ ಮೂರು ವರ್ಗದ ಭಕ್ತರು ಶ್ರೀರಾಮಕೃಷ್ಣರನ್ನು ‘ಭಗವ ದವತಾರ,’ ‘ಅತಿಮಾನವ’ ಮತ್ತು ‘ದೇವತಾಮನುಷ್ಯ’ ಎಂದು ಬೇರೆಬೇರೆಯಾಗಿ ಪರಿಭಾವಿಸು ತ್ತಿದ್ದರಾದರೂ, ಇವರೆಲ್ಲರಿಗೂ ಒಂದು ವಿಷಯವಂತೂ ಸ್ಪಷ್ಟವಾಗಿತ್ತು. ಅದೇನೆಂದರೆ, ಶ್ರೀ ರಾಮಕೃಷ್ಣರ ಬೋಧನೆಯನ್ನು ಪರಿಪಾಲಿಸುತ್ತ, ಅವರು ತೋರಿಸಿಕೊಟ್ಟ ಆದರ್ಶಗಳನ್ನು ಅನು ಷ್ಠಾನ ಮಾಡುತ್ತ, ಅವರಿಗೆ ಸೇವೆ ಸಲ್ಲಿಸುತ್ತಿದ್ದರೆ ತಾವು ತಮ್ಮ ಗುರಿ ಸೇರುವುದರಲ್ಲಿ ಸಂದೇಹವೇ ಇಲ್ಲ ಎಂದು. 

ಹೆಚ್ಚಿನ ಸಂಖ್ಯೆಯ ಭಕ್ತರು ಶ್ರೀರಾಮಕೃಷ್ಣರನ್ನು ಅವತಾರವೆಂದು ಕರೆಯುತ್ತಿದ್ದರೂ, ಅವರು ಅವತಾರಪುರುಷರೋ ಅಲ್ಲವೋ ಎಂಬುದು ತಮ್ಮ ತಿಳಿವಳಿಕೆಗೆ ಎಟುಕದ ವಿಷಯ ಎಂಬುದು ನರೇಂದ್ರನ ಅಭಿಮತ. ಆದರೆ ಅವನು ಆ ವಿಚಾರದಲ್ಲಿ ಹಟವಾದಿಯಲ್ಲ. ಸಾಮಾನ್ಯ ಮರ್ತ್ಯ ನಂತೆ ಕಣ್ಣೆದುರಿನಲ್ಲೇ ಕಾಯಿಲೆಯಿಂದ ನರಳುತ್ತ ಮಲಗಿರುವ ಶ್ರೀರಾಮಕೃಷ್ಣರು ಮರುಕ್ಷಣ ದಲ್ಲೇ ದೈವೀಪುರುಷನಾಗಿ ವ್ಯಕ್ತರಾಗಬಲ್ಲರು ಎಂಬುದನ್ನು ಅವನು ಒಪ್ಪಲೇಬೇಕಾಗಿತ್ತು. ಅವರ ವ್ಯಕ್ತಿತ್ವದಲ್ಲಿ ಮಾನವತ್ವ-ದೈವತ್ವಗಳನ್ನು ವಿಂಗಡಿಸಲೇ ಸಾಧ್ಯವಿಲ್ಲದಂತಹ ಸ್ಥಿತಿಯೊಂದನ್ನು ಅವನು ಕಾಣಲಾರಂಭಿಸಿದ್ದ. ನಿಯತ ಸಾಧನೆಯ ಮೂಲಕ ನಮ್ಮನಮ್ಮ ವ್ಯಕ್ತಿತ್ವದ ಸಂಕುಚಿತತೆ ಯಿಂದ ಪಾರಾಗಿ ಅನಂತತೆಯ ಅನುಭವ ಮಾಡಿಕೊಳ್ಳುವುದೇ ‘ಧರ್ಮ’ ಎಂಬುದರ ನಿಜವಾದ ಅರ್ಥ: ಇದನ್ನು ಅವನು ಶ್ರೀರಾಮಕೃಷ್ಣರ ಜೀವನದಲ್ಲಿ ಪ್ರತಿದಿನವೆಂಬಂತೆ ಕಾಣತೊಡಗಿದ್ದ. ಹೀಗೆ ಅವರ ಸಾನ್ನಿಧ್ಯದಲ್ಲಿರುತ್ತ ಸಕಲ ವೇದೋಪನಿಷತ್ತುಗಳ ಸಾರವನ್ನೇ ಅವರಲ್ಲಿ ಕಂಡು ಕೊಂಡ.

ಆದರೆ ಅವನು ಶ್ರೀರಾಮಕೃಷ್ಣರನ್ನು ‘ಅವತಾರ’ವೆಂದೂ ಸಾರಲಿಲ್ಲ, ‘ಸಾಮಾನ್ಯ ಮಾನವ’ ನೆಂದೂ ಕರೆಯಲಿಲ್ಲ. ಒಮ್ಮೆ ಡಾ ॥ ಸರ್ಕಾರ್ ಈ ವಿಷಯದ ಪ್ರಸ್ತಾಪವೆತ್ತಿದಾಗ ಅವನೆನ್ನು ತ್ತಾನೆ: “ನಾವು ಶ್ರೀರಾಮಕೃಷ್ಣರನ್ನು ‘ಭಗವತ್ಸದೃಶ ವ್ಯಕ್ತಿ’ ಎಂದು ಭಾವಿಸುತ್ತೇವೆ. ಇದು ಹೇಗೆಂದರೆ ಸಸ್ಯವಲಯಕ್ಕೂ ಪ್ರಾಣಿವಲಯಕ್ಕೂ ಮಧ್ಯೆ ಒಂದು ಸ್ಥಿತಿ ಇದೆ. ಈ ಮಧ್ಯಂತರ ಸ್ಥಿತಿಯಲ್ಲಿರುವ ಜೀವಿಗಳನ್ನು ಸಸ್ಯವರ್ಗಕ್ಕೆ ಸೇರಿಸಬೇಕೆ ಅಥವಾ ಪ್ರಾಣಿವರ್ಗಕ್ಕೆ ಸೇರಿಸಬೇಕೆ ಎಂಬುದನ್ನು ನಿರ್ಧರಿಸುವುದು ಕಷ್ಟ. ಅದೇ ತರಹ ಮಾನವಪ್ರಪಂಚಕ್ಕೂ ದೈವಪ್ರಪಂಚಕ್ಕೂ ಮಧ್ಯೆ ಒಂದು ಸ್ಥಿತಿಯಿದೆ. ಈ ಮಧ್ಯಂತರ ಸ್ಥಿತಿಯ ವ್ಯಕ್ತಿಯನ್ನು ಮನುಷ್ಯ ಎಂದು ತಿಳಿಯಬೇಕೆ ಅಥವಾ ದೇವರು ಎಂದು ತಿಳಿಯಬೇಕೆ ಅಂತ ನಿರ್ಧರಿಸುವುದು ಬಹಳ ಕಷ್ಟ. ಹಾಗಾಗಿ ನಾನು ಇವರನ್ನು ಭಗವಂತ ಅಂತ ಹೇಳಲಾರೆ. ಆದರೆ ಇವರು ದೇವರಂತಹ ಮನುಷ್ಯ. ಆದ್ದರಿಂದ ನಾವು ಇವರನ್ನು ಪೂಜಿಸುತ್ತಿದ್ದೇವೆ.”

ಆದರೆ ಶ್ರೀರಾಮಕೃಷ್ಣರು ನಿಜಕ್ಕೂ ಯಾರು ಎಂಬುದನ್ನು ಅವನು ತಿಳಿಯುವ ಕಾಲ ಶೀಘ್ರದಲ್ಲೇ ಬರಲಿದೆ!

ತಮ್ಮ ಇಹಲೋಕಲೀಲೆಯ ಪರಿಸಮಾಪ್ತಿಯ ದಿನ ಬಹುಬೇಗ ಸಮೀಪಿಸುತ್ತಿದೆಯೆಂಬು ದನ್ನು ಕಂಡು ಶ್ರೀರಾಮಕೃಷ್ಣರು ತಮ್ಮ ಶಿಷ್ಯರಲ್ಲಿ ಭಗವತ್ಸಾಕ್ಷಾತ್ಕಾರ ವ್ಯಾಕುಲತೆಯನ್ನು ಪ್ರಜ್ವಲಿಸುವಂತೆ ಮಾಡಲು ಇನ್ನಷ್ಟು ಕಾತರರಾಗಿದ್ದರು. ‘ಈ ಜನ್ಮದಲ್ಲೇ ಭಗವಂತನ ಸಾಕ್ಷಾ ತ್ಕಾರ ಮಾಡಿಕೊಳ್ಳುತ್ತೇನೆ’ ಎನ್ನುವ ತೀವ್ರ ವ್ಯಾಕುಲತೆಯಿದ್ದರೆ ಮಾತ್ರ ಮನುಷ್ಯ ತನ್ನ ಸಾಧನೆ ಯನ್ನು ಅಷ್ಟೇ ತೀವ್ರಗೊಳಿಸುತ್ತಾನೆ. ಆದ್ದರಿಂದಲೇ ಶ್ರೀರಾಮಕೃಷ್ಣರ ಆ ಕಾತರ. ಕಾಮಕಾಂಚನ ಗಳ ಮೇಲಿನ ಮೋಹವನ್ನು ಸಂಪೂರ್ಣ ನಾಶಗೊಳಿಸಿದರೆ ಮಾತ್ರ ಸಾಕ್ಷಾತ್ಕಾರ ಸಾಧ್ಯ ಎಂಬ ಸತ್ಯವನ್ನು ಅವರು ಮನಸ್ಸಿಗೆ ನಾಟುವಂತೆ ತಿಳಿಸುತ್ತಿದ್ದರು. ಶ್ಯಾಮಪುಕುರವಾಸದ ಈ ದಿನಗಳಲ್ಲಿ ಅವರ ಮಾತುಗಳೆಲ್ಲವೂ ವಿಶೇಷವಾಗಿ ವೈರಾಗ್ಯಪ್ರಚೋದಕವಾಗಿದ್ದುವು. ಅವರ ಅಕಳಂಕ ಜೀವನ, ಅವರ ಶಕ್ತಿಪೂರ್ಣವಾದ ಬೋಧನೆ, ಅವರು ಭಗವದ್ಭಾವದಲ್ಲಿ ಮುಳುಗೇಳುವ ಮತ್ತು ಸಹಜವಾಗಿ ಸಮಾಧಿಸ್ಥಿತಿಗೇರುವ ದೃಶ್ಯ–ಇವೆಲ್ಲವೂ ಉಜ್ಜ್ವಲ ಜ್ಯೋತಿಯಂತೆ ಶಿಷ್ಯರ ಗಮನ ವನ್ನು ಭವ್ಯವಾದ ಆಧ್ಯಾತ್ಮಿಕ ಆದರ್ಶದೆಡೆಗೆ ಸೆಳೆಯುತ್ತಿದ್ದುವು. ನಾನಾ ಬಗೆಯ ತುಮುಲಗಳಲ್ಲಿ ಸಿಲುಕಿ ತೊಳಲುತ್ತಿದ್ದ ನರೇಂದ್ರನ ಮನಸ್ಸು ಕೂಡ ಈಗ ಶ್ರೀರಾಮಕೃಷ್ಣರ ವ್ಯಕ್ತಿತ್ವ-ಜೀವನ- ಬೋಧನೆಗಳಲ್ಲೇ ನೆಲೆನಿಲ್ಲುವಂತಾಯಿತು. ಅವರ ಬೋಧನೆಗಳು ಆತನ ಹೃದಯದಲ್ಲಿ ಆಳವಾಗಿ ಬೇರೂರಿ ಸ್ಥಿರವಾಗಿ ನಿಂತುವು.

ಶ್ರೀರಾಮಕೃಷ್ಣರ ಚಿಕಿತ್ಸೆಯನ್ನು ಕೈಗೊಂಡಿದ್ದ ಡಾ ॥ ಸರ್ಕಾರ ಮಹಾ ಮೇಧಾವಿ, ವಿಚಾರ ವಂತ, ಸಹೃದಯಿ. ಶ್ರೀರಾಮಕೃಷ್ಣರನ್ನೂ ಅವರ ಭಕ್ತಗಣವನ್ನೂ ಇವನು ತನ್ನ ಸೂಕ್ಷ್ಮ ವಿಮರ್ಶಾದೃಷ್ಟಿಯಿಂದ ನೋಡುತ್ತ, ಅವರೆಲ್ಲರ ಬಗ್ಗೆ ತನ್ನ ಅನಿಸಿಕೆಗಳನ್ನು ನಿಶ್ಶಂಕೆಯಿಂದ ವ್ಯಕ್ತಪಡಿಸುತ್ತಿದ್ದ. ನರೇಂದ್ರನ ಪರಿಚಯವಾದ ಮೇಲೆ ಇಬ್ಬರ ನಡುವೆ ಆಗಾಗ ಅನೇಕ ವಿಷಯಗಳ ಬಗ್ಗೆ ಮಾತುಕತೆ ನಡೆಯಿತು. ನರೇಂದ್ರನ ಹರಿತ ಬುದ್ಧಿ, ಪ್ರಾಮಾಣಿಕತೆ ಮತ್ತು ಸಂಭಾಷಣಾ ಸಾಮರ್ಥ್ಯಗಳನ್ನು ಕಂಡ ಸರ್ಕಾರ ಬಹಳ ಸಂತೋಷಪಟ್ಟ. ಒಮ್ಮೆಯಂತೂ ಅವನ ಬಾಯಿಂದ ಹಾಡುಗಳನ್ನು ಕೇಳಿ ಆನಂದಾತಿಶಯದಿಂದ ಅವನನ್ನು ಆಲಿಂಗಿಸಿಕೊಂಡುಬಿಟ್ಟ. ಬಳಿಕ ಶ್ರೀರಾಮಕೃಷ್ಣರಿಗೆ ಹೇಳುತ್ತಾನೆ: “ನಿಮ್ಮ ಬಳಿಗೆ ಆಧ್ಯಾತ್ಮಿಕ ಶಿಕ್ಷಣ ಪಡೆಯಲು ಬರು ತ್ತಿರುವವರು ಇವನಂತಹ ಯುವಕರು ಎಂಬುದನ್ನು ಕಂಡು ನನಗೆ ಬಹಳ ಸಂತೋಷವಾಗುತ್ತಿದೆ. ನರೇಂದ್ರ ಒಂದು ನಿಜವಾದ ರತ್ನ. ಅವನು ಜೀವನದ ಯಾವುದೇ ಕ್ಷೇತ್ರದಲ್ಲಿ ಕಾಲಿಟ್ಟರೂ ಅಲ್ಲಿ ಬೆಳಗಬಲ್ಲ.” ಆಗ ಶ್ರೀರಾಮಕೃಷ್ಣರು ಮೆಚ್ಚುಗೆಯಿಂದ ನರೇಂದ್ರನನ್ನೇ ಈಕ್ಷಿಸುತ್ತ ಹೇಳುತ್ತಾರೆ: “ಅದ್ವೈತಗೋಸ್ವಾಮಿಯ ವ್ಯಾಕುಲ ಭಕ್ತಿಯ ಕರೆಗೆ ಓಗೊಟ್ಟು ಶ್ರೀಕೃಷ್ಣಚೈತನ್ಯನು ಅವತಾರ ತಾಳಿದ ಎನ್ನುವ ಮಾತಿದೆ. ಹಾಗೆಯೇ ನರೇಂದ್ರನಿಗಾಗಿ ಇಲ್ಲಿನ ಆವಿರ್ಭಾವ.”

ಶ್ರೀರಾಮಕೃಷ್ಣರಾಡಿದ ಈ ಒಂದು ಪುಟ್ಟ ವಾಕ್ಯದ ಭಾವ ಅಗಾಧವಾಗಿದೆ. ‘ನರೇಂದ್ರನಿಗಾಗಿ ಇಲ್ಲಿನ ಆವಿರ್ಭಾವ!’ ಯಾರನ್ನು ಜಗತ್ತು ಇಂದು ದೇವಮಾನವನೆಂದು ಅವತಾರವರಿಷ್ಠನೆಂದು ಸ್ತುತಿಸುತ್ತಿದೆಯೋ ಅಂತಹ ಶ್ರೀರಾಮಕೃಷ್ಣರು ಹೇಳುತ್ತಿದ್ದಾರೆ–ತಾವು ಈ ಅವತಾರವನ್ನು ತಾಳಲು ನರೇಂದ್ರನೇ ಕಾರಣ ಎಂದು. ತಮ್ಮ ಅವತಾರ ಕಾರ್ಯೋದ್ದೇಶ ಪೂರ್ಣವಾಗುವುದ ರಲ್ಲಿ ನರೇಂದ್ರನ ಮಹತ್ವದ ಪಾತ್ರವಿದೆ ಎಂಬುದನ್ನು ಈ ಮೂಲಕ ತಿಳಿಸಿಕೊಡುತ್ತಿದ್ದಾರೆ. ಇದು ಕೇವಲ ಹೊಗಳಿಕೆಯ ಮಾತಲ್ಲ. ಈ ಮಾತಿನ ಪೂರ್ಣ ಸತ್ಯವನ್ನು ಮುಂದೆ ನಾವು ನೋಡಲಿದ್ದೇವೆ.

ವಯಸ್ಸಿನಿಂದ ನರೇಂದ್ರ ಕಿರಿಯನಾದರೂ ತನ್ನ ಅಸಾಧಾರಣ ವಿವೇಚನಾ ಶಕ್ತಿಯಿಂದಲೂ, ಬುದ್ಧಿಮತ್ತೆಯಿಂದಲೂ ಉಳಿದೆಲ್ಲ ಶಿಷ್ಯರಿಗೆ ಹಾಗೂ ಭಕ್ತರಿಗೆ ಸಹಜವಾಗಿಯೇ ನಾಯಕನಾಗಿ, ಅವರಿಗೆ ಮಾರ್ಗದರ್ಶನ ನೀಡಲು ಸಮರ್ಥನಾಗಿದ್ದ. ಅಷ್ಟೇ ಅಲ್ಲದೆ, ಆವಶ್ಯಕತೆ ಕಂಡುಬಂದಾಗ ಅವರನ್ನು ತಿದ್ದಲು ಛೀಮಾರಿ ಹಾಕಿ ಬುದ್ಧಿ ಹೇಳುವ ಸಾಮರ್ಥ್ಯ-ಯೋಗ್ಯತೆ ಅವನಲ್ಲಿತ್ತು. ತನ್ನ ಆ ಸಾಮರ್ಥ್ಯವನ್ನು ಅವನು ಪ್ರಯೋಗಿಸಬೇಕಾದ ಪ್ರಸಂಗವೊಂದು ಒದಗಿಬಂತು.

ಈ ನಡುವೆ ಭಕ್ತರೆಲ್ಲ ತುಂಬ ಉತ್ಸಾಹ ಶ್ರದ್ಧೆಗಳಿಂದ ಶ್ರೀರಾಮಕೃಷ್ಣರ ಸೇವೆಯಲ್ಲಿ ನಿರತ ರಾಗಿಬಿಟ್ಟಿದ್ದರು. ಅವರ ದಿವ್ಯ ಸಾನ್ನಿಧ್ಯದ ಪರಿಣಾಮವಾಗಿ ಭಕ್ತರ ಭಕ್ತಿ ಭಾವಾವೇಶ ವೃದ್ಧಿಯಾಗ ತೊಡಗಿತ್ತು. ಭಕ್ತಿ-ಶ್ರದ್ಧೆಗಳು ಒಳ್ಳೆಯವೇ; ಆದರೆ ಅದರ ಆವೇಶ ಎಂಬುದು ಭಕ್ತರನ್ನು ಅವರ ಅರಿವಿಗೇ ಬಾರದಂತೆ ಅಪಾಯದ ಹಾದಿಯಲ್ಲಿ ನಡೆಸಿಕೊಂಡು ಹೋಗುತ್ತಿತ್ತು. ಶ್ರೀರಾಮಕೃಷ್ಣರ ಸತ್ಸಹವಾಸದ ಸಹಾಯದಿಂದ ಬಹುಬೇಗ ಆಧ್ಯಾತ್ಮಿಕ ಉನ್ನತಿಯನ್ನು ಹೊಂದಬೇಕೆಂಬ ಆವೇಶ ಅವರಲ್ಲುಂಟಾಗಿಬಿಟ್ಟಿತ್ತು. ಆದರೆ ಈ ಉತ್ಸಾಹದ ಭರದಲ್ಲಿ ಅವರಿಗರಿವಿಲ್ಲದಂತೆಯೇ ಅವರಲ್ಲಿ ಭಾವಾವೇಗ ಹೆಚ್ಚುತ್ತಿತ್ತೇ ಹೊರತು ನಿಜವಾದ ಪ್ರಪತ್ತಿಯಾಗಲಿ ಆತ್ಮಸಂಯಮವಾಗಲಿ ಕಂಡು ಬರುತ್ತಿರಲಿಲ್ಲ. ಆತ್ಮಸಂಯಮ ಮತ್ತು ಪ್ರೀತಿಪೂರ್ವಕ ಶರಣಾಗತಿ–ಇವೇ ಆಧ್ಯಾತ್ಮಿಕ ಜೀವನದ ಯಶಸ್ಸಿಗೆ ಪ್ರಬಲ ಅಡಿಗಲ್ಲುಗಳು. ಆದರೆ ಈ ಭಕ್ತರು ಭಕ್ತಿಯ ಹೆಸರಿನಲ್ಲಿ ಭಾವಾವೇಗಕ್ಕೆ ಬಲಿಯಾಗಲಾರಂಭಿಸಿದ್ದರು. ಆಧ್ಯಾತ್ಮಿಕ ಸಾಧನೆಯಲ್ಲಿ, ಅದರಲ್ಲೂ ಭಕ್ತಿಸಾಧನೆ ಯಲ್ಲಿ ಭಾವೋತ್ಕರ್ಷಕ್ಕೂ ಒಂದು ಸ್ಥಾನವಿದೆ, ಇಲ್ಲವೆಂದಲ್ಲ. ಆದರೆ ಈ ಭಾವಾವೇಗವನ್ನೇ ಪ್ರಧಾನವಾಗಿ ಮಾಡಿಕೊಂಡುಬಿಟ್ಟರೆ ಗುರಿ ತಪ್ಪುವುದೇ ಖಂಡಿತ. ಸಾಮಾನ್ಯವಾಗಿ ಸಾಧಕರೂ ಸಾಧಕರಲ್ಲದವರೂ ಈ ಬಗೆಯ ಭಾವಾವೇಗವನ್ನೇ ಆಧ್ಯಾತ್ಮಿಕತೆಯೆಂದು ಭ್ರಮಿಸುವ ಸಂಭವ ಹೆಚ್ಚು. ಸಾಮಾನ್ಯ ಜನರು, ಸಾಧನೆಗಿಳಿದಾಗಲೂ ಕೂಡ ಇಂದ್ರಿಯಗಳ ಬಯಕೆಗಳನ್ನು ನಿರೋ ಧಿಸಿ ಉದಾತ್ತೀಕರಿಸುವ ಕಷ್ಟವನ್ನು ಇಷ್ಟಪಡುವುದಿಲ್ಲ. ಅವರು ಇಂದ್ರಿಯಗಳ ಅಭಿಲಾಷೆಗೆ ಅನುಕೂಲಕರವಾದ ಮಾರ್ಗವನ್ನೇ ಬಯಸುತ್ತಾರೆ. ಆದ್ದರಿಂದ ಭಗವಂತ ಹಾಗೂ ಪ್ರಪಂಚ– ಇವೆರಡರ ನಡುವಿನ ಒಂದು ಮಾರ್ಗವನ್ನು ಕಂಡುಕೊಳ್ಳುತ್ತಾರೆ. ನಿಜವಾದ ತ್ಯಾಗಕ್ಕೆ ಅವರ ಮನಸ್ಸು ಒಪ್ಪದಿರುವುದರಿಂದ ಪ್ರಾಪಂಚಿಕ ಭೋಗ ಮತ್ತು ಆಧ್ಯಾತ್ಮಿಕ ಯೋಗಗಳ ಮಧ್ಯದ ಮಾರ್ಗವೊಂದು ಅವರಿಗೆ ಬೇಕು. ಈ ಭೋಗ-ಯೋಗಗಳ ನಡುವಿನ ಮಾರ್ಗವೇ ಭಾವಾವೇಗ! ಆದರೆ ಭಾವಾವೇಗದ ಆಧಿಕ್ಯದಿಂದ ಎಂತಹ ವಿಪರೀತ ಹಾಗೂ ವಿರುದ್ಧ ಪರಿಣಾಮವುಂಟಾ ದೀತೆಂಬ ಕಲ್ಪನೆಯೇ ಅವರಿಗಿರುವುದಿಲ್ಲ. ಈ ಭಾವಾವೇಗದ ಸಹಾಯದಿಂದ ಒಂದಿಷ್ಟು ತಾತ್ಕಾಲಿಕ ಸುಖ ಲಭಿಸಿದರೆ ಅಷ್ಟಕ್ಕೇ ತೃಪ್ತರಾಗಿಬಿಡುತ್ತಾರೆ. ಆದ್ದರಿಂದಲೇ ಶ್ರೀರಾಮಕೃಷ್ಣರು ತಮ್ಮ ಬಳಿಗೆ ಬರುತ್ತಿದ್ದ ಜಿಜ್ಞಾಸುಗಳನ್ನು, ಭಕ್ತರನ್ನು ಚೆನ್ನಾಗಿ ಪರೀಕ್ಷಿಸಿ ನೋಡುತ್ತಿದ್ದರು– ‘ಇವರೇನು ಅನುಕೂಲಕರ ವೇದಾಂತ ಬೇಕೆಂಬವರೋ ಅಥವಾ ಕಟ್ಟುನಿಟ್ಟನ ಸಾಧನೆ ಮಾಡಲು ಸಿದ್ಧರಾಗಿ ಬಂದವರೋ’ ಎಂದು. ತಮ್ಮ ಪ್ರಾಪಂಚಿಕ ಭೋಗಕ್ಕೆ ಧಕ್ಕೆ ಬಾರದಂತೆ ಯೋಗವನ್ನು ಕಲಿಸಿಕೊಡಬೇಕು ಎಂದು ಬಯಸುವವರಾದರೆ, ಅಂಥವರಿಗೆ ಶ್ರೀರಾಮಕೃಷ್ಣರು ಸಂಪೂರ್ಣ ತ್ಯಾಗವನ್ನು ಬೋಧಿಸುತ್ತಿರಲಿಲ್ಲ. ಅವರಿಂದ ಆಧ್ಯಾತ್ಮಿಕ ಆದರ್ಶಗಳನ್ನು ಎಷ್ಟರಮಟ್ಟಿಗೆ ಗ್ರಹಿಸಿ ಪಾಲಿಸಿಕೊಂಡು ಬರಲು ಸಾಧ್ಯವೋ ಅಷ್ಟನ್ನು ಮಾತ್ರ ಬೋಧಿಸುತ್ತಿದ್ದರು. ಆದ್ದರಿಂದಲೇ ಶ್ರೀರಾಮಕೃಷ್ಣರು ಕಾಮಕಾಂಚನದಿಂದ ಕಲುಷಿತರಾಗದ ಯುವಕಶಿಷ್ಯರಿಗೆ ನೀಡುತ್ತಿದ್ದ ಬೋಧನೆಗಿಂತ ಗೃಹಸ್ಥರಿಗೆ ನೀಡುತ್ತಿದ್ದ ಬೋಧನೆ ಭಿನ್ನವಾಗಿರುತ್ತಿತ್ತು. ಇದಲ್ಲದೆ ಸಾಮಾನ್ಯ ವಾಗಿ ಎಲ್ಲರೂ ಅನುಸರಿಸಬಹುದಾದಂತಹ ಬೋಧನೆಗಳನ್ನೂ ನೀಡುತ್ತಿದ್ದರು. ಉದಾಹರಣೆಗೆ, ಅವರು ಕೆಲವೊಮ್ಮೆ ಹೇಳುತ್ತಿದ್ದರು: “ಕಲಿಯುಗದಲ್ಲಿ ಭಕ್ತಿಯನ್ನು ಬೆಳೆಸಿಕೊಳ್ಳಲು ನಾಮ ಸಂಕೀರ್ತನೆಯೇ ಸುಲಭದ ದಾರಿ; ನಾರದೀಯ ಭಕ್ತಿಮಾರ್ಗವೇ ಸುಲಭೋಪಾಯ” ಎಂದು. ಈ ಮಾತಿನ ಸಂಪೂರ್ಣ ಅರ್ಥ-ಮಹತ್ವ ಹೆಚ್ಚಿನ ಭಕ್ತರಿಗೆ ತಿಳಿಯಲಿಲ್ಲವೆಂದೇ ಹೇಳಬೇಕಾಗು ತ್ತದೆ. ಏಕೆಂದರೆ ನಾರದೀಯ ಭಕ್ತಿಮಾರ್ಗವು ಬೋಧಿಸುವುದಾದರೂ ಭಗವತ್ಪ್ರೇಮದ ಮೂಲಕ ಸಂಪೂರ್ಣ ತ್ಯಾಗವನ್ನೇ. ನಾರದೀಯ ಭಕ್ತಿಮಾರ್ಗದ ಪ್ರಕಾರ, ಎಂತಹ ಭೋಗಿಗಳೇ ಆದರೂ ಭಗವನ್ನಾಮ ಸಂಕೀರ್ತನೆಯ ಮೂಲಕ ಕೊನೆಗೊಂದು ದಿನ ತ್ಯಾಗದ ಆದರ್ಶವನ್ನೇ ಮುಟ್ಟು ತ್ತಾರೆ. ನಾಮಸಂಕೀರ್ತನೆ ಮಾಡುತ್ತ ಹೋದಂತೆ ಭಗವಂತನಲ್ಲಿ ಅನುರಕ್ತಿಯುಂಟಾಗುತ್ತದೆ. ಭಗವಂತನಲ್ಲಿ ಅನುರಕ್ತಿಯುಂಟಾದಂತೆಲ್ಲ ಪ್ರಾಪಂಚಿಕ ಭೋಗಗಳಲ್ಲಿ ವಿರಕ್ತಿಯುಂಟಾಗು ತ್ತದೆ. ಆದ್ದರಿಂದ ನಾರದೀಯ ಭಕ್ತಿಮಾರ್ಗದಲ್ಲಿ ಪರ್ಯಾಯವಾಗಿ ತ್ಯಾಗವೈರಾಗ್ಯಗಳನ್ನೇ ಬೋಧಿಸಲಾಗಿದೆ.

ಈ ಭಕ್ತರು ಭಾವಾವೇಗದ ಆಕರ್ಷಣೆಗೆ ಬಲಿಯಾಗಲು ಇನ್ನೊಂದು ಕಾರಣವೂ ಇತ್ತು. ಇವರೆಲ್ಲ ಶ್ರೀರಾಮಕೃಷ್ಣರು ಭಾವಾವೇಶಭರಿತರಾಗಿ ನರ್ತಿಸುವುದನ್ನು ಕಂಡವರು; ಭಗವನ್ನಾಮ ಸಂಕೀರ್ತನೆಯನ್ನು ಕೇಳಿ ಕಣ್ಣೀರು ಸುರಿಸುವುದನ್ನು ಕಂಡವರು; ಅವರು ಆನಂದಭರಿತರಾಗಿ ನಗುವುದನ್ನು, ಸಮಾಧಿಗೇರುವುದನ್ನು ದಿನನಿತ್ಯವೆಂಬಂತೆ ಕಂಡವರು. ಆದರೆ ಶ್ರೀರಾಮಕೃಷ್ಣರ ತಪೋಮಯ ಜೀವನವನ್ನು ಸಮಗ್ರವಾಗಿ ಕಂಡವರಲ್ಲ, ಕಾಣುವ ಯತ್ನವನ್ನೂ ಮಾಡಿದವರಲ್ಲ. ಅವರ ಭಾವಾವೇಶವೆಂಬುದು ಇತರರ ಅಗ್ಗದ ಭಾವಾವೇಶದಂತಲ್ಲ, ಅದರ ಹಿನ್ನೆಲೆಯಲ್ಲಿ ಅದ್ಭುತ ತಪಸ್ಸು ಇತ್ತು, ಮಹಾ ತ್ಯಾಗವಿತ್ತು ಎಂಬುದು ಇವರಿಗೆ ತಿಳಿಯದು. ಆದರೆ ಶ್ರೀರಾಮಕೃಷ್ಣರು ನರ್ತಿಸುವುದನ್ನು ನೋಡಿ ತಾವೂ ನರ್ತಿಸಲಾರಂಭಿಸಿದರು; ಅವರು ನಗುವು ದನ್ನು ನೋಡಿ ತಾವೂ ನಗಲಾರಂಭಿಸಿದರು. ಆದರೆ ಇವರು ಅವರಂತೆ ಆಧ್ಯಾತ್ಮಿಕ ಜೀವನದ ಅಂತಿಮಾವಸ್ಥೆಯಾದ ಸಮಾಧಿಸ್ಥಿತಿಯನ್ನು ಮುಟ್ಟಬಲ್ಲರೆ! ಅದಕ್ಕೆ ವರ್ಷಾಂತರಗಳ, ಜನ್ಮಾಂತರಗಳ ತಪಸ್ಸು ಬೇಕು, ಶಿಸ್ತುಬದ್ಧ ಸಾಧನೆ ಬೇಕು. ಆದರೆ ಈ ಭಕ್ತರ ಹತ್ತಿರ ಇದ್ದುದೆಲ್ಲ ಬರಿಯ ಭಾವಾವೇಗ!

ಈ ಮಧ್ಯೆ, ಪರಮಭಕ್ತನಾದ ಗಿರೀಶಚಂದ್ರ ಘೋಷ್ ಶ್ರೀರಾಮಕೃಷ್ಣರನ್ನು ಭಗವಂತನ ಅವತಾರ ಎಂದು ಘಂಟಾಘೋಷವಾಗಿ ಸಾರಲಾರಂಭಿಸಿದ. ಅಲ್ಲದೆ ಇತರರನ್ನೂ ತನ್ನ ನಂಬಿಕೆಯಲ್ಲಿ ಸಹಭಾಗಿಗಳಾಗುವಂತೆ ಪ್ರಚೋದಿಸಿದ. ಇದರಿಂದ ಭಕ್ತಾದಿಗಳಲ್ಲಿ ಒಂದು ಹೊಸ ಆವೇಶವುಂಟಾಯಿತು. ಗಿರೀಶನಂತಹ ಮಹಾಭಕ್ತನೇ ಹೇಳುತ್ತಿರುವಾಗ ಇನ್ನೇನು ಎಂದು ತಾವೂ ಅಹುದಹುದೆಂದು ತಲೆದೂಗಿದರು. ಆದರೆ ಗಿರೀಶನ ವಿಷಯವೇ ಬೇರೆ. ಅವನ ಶ್ರದ್ಧೆಯ ಮಟ್ಟವೇ ಬೇರೆ. ಭಕ್ತರನೇಕರಲ್ಲಿ ಅವನಲ್ಲಿದ್ದಂತಹ ಪ್ರಾಮಾಣಿಕ ಶ್ರದ್ಧೆಯಿರಲಿಲ್ಲ. ಆದರೆ ಅವನಂತೆಯೇ ತಾವೂ ಶ್ರೀರಾಮಕೃಷ್ಣರಿಗೆ ತಮ್ಮ ಆಧ್ಯಾತ್ಮಿಕ ಜೀವನದ ವಕಾಲತ್ತನ್ನು ವಹಿಸಿ ಬಿಟ್ಟಿರುವುದಾಗಿಯೂ, ಆದ್ದರಿಂದ ತಾವಿನ್ನು ಯಾವ ಸಾಧನೆಯನ್ನೂ ಮಾಡಬೇಕಾಗಿಲ್ಲ ಎಂದೂ ಹೇಳಿಕೊಂಡು ತಿರುಗಲಾರಂಭಿಸಿದರು.

ಇದೆಲ್ಲದರ ಜೊತೆಗೆ ಬ್ರಾಹ್ಮಸಮಾಜದ ಮುಖಂಡರಲ್ಲೊಬ್ಬನಾದ ವಿಜಯಕೃಷ್ಣ ಗೋಸ್ವಾಮಿ ಶ್ರೀರಾಮಕೃಷ್ಣರ ಕುರಿತಾಗಿ ವ್ಯಕ್ತಪಡಿಸಿದ ಅಭಿಪ್ರಾಯದಿಂದಾಗಿ ಭಕ್ತಜನರಲ್ಲಿ ಒಂದು ಕೋಲಾಹಲವೇ ಎದ್ದಿತು. ಈತ ಅನೇಕ ವರ್ಷಗಳಿಂದಲೂ ಅವರ ವಿಶ್ವಾಸಿಗನಾಗಿ ದ್ದವನೇ. ಆದರೆ ಈಗ ಅವನಿಗೆ ಅವರ ವಿಚಾರದಲ್ಲಿದ್ದ ಭಕ್ತಿ-ಗೌರವ ಇದ್ದಕ್ಕಿದ್ದಂತೆ ತೀವ್ರವಾಗ ತೊಡಗಿತ್ತು. ಇದಕ್ಕೆ ಮುಖ್ಯ ಕಾರಣ ಆತ ದರ್ಶನವೊಂದರಲ್ಲಿ ಅವರನ್ನು ಕಂಡದ್ದು. ಆತ ಶ್ರೀರಾಮಕೃಷ್ಣರ ಬಳಿಗೆ ಬಂದು ತನ್ನ ದರ್ಶನಾನುಭವವನ್ನು ಬಣ್ಣಿಸಿ, “ನೀವು ನಿಜಕ್ಕೂ ಆಧ್ಯಾತ್ಮಿಕತೆಯ ತುದಿಗೇರಿದವರು ಎಂದು ನನಗೆ ಭರವಸೆಯುಂಟಾಗಿದೆ. ನಾನು ಕಂಡಿರುವ ಯಾವ ಸಾಧು-ಸಂತರೂ ನಿಮ್ಮ ಮಟ್ಟಕ್ಕೆ ಬರಲಾರರು. ಆದರೆ ನಿಮ್ಮ ನಿಜಸ್ವರೂಪವನ್ನು ಜನಗಳು ಅರಿತುಕೊಂಡಿಲ್ಲ, ಅಷ್ಟೇ” ಎಂದು ನುಡಿದ. ಸ್ವತಃ ಶ್ರೀರಾಮಕೃಷ್ಣರು ಅವನ ಮಾತನ್ನು ಸಮರ್ಥಿಸಲಿಲ್ಲವಾದರೂ ಭಕ್ತರಿಗೆಲ್ಲ ಅದು ಸಂಪೂರ್ಣ ಸಮ್ಮತವಾಯಿತು.

ಹೀಗೆ ಶ್ರೀರಾಮಕೃಷ್ಣರು ಅವತಾರಪುರುಷರೆಂಬ ನಂಬಿಕೆ ಬಹುಬೇಗ ಹರಡಿತು. ಕೆಲವರಂತೂ ಶ್ರೀರಾಮಕೃಷ್ಣರು ತಮ್ಮ ಅಲೌಕಿಕ ಶಕ್ತಿಯನ್ನು ವ್ಯಕ್ತಪಡಿಸುವ ಪವಾಡವನ್ನೇನಾದರೂ ಎಸಗ ಬಹುದೆಂಬ ನಿರೀಕ್ಷೆಯಲ್ಲೇ ಸದಾ ಇರತೊಡಗಿದರು. ಇನ್ನು ಕೆಲವರು ಭಗವತ್ಸಂಬಂಧವಾದ ಹಾಡುಗಳನ್ನು, ಮಾತುಗಳನ್ನು ಕೇಳಿದೊಡನೆ ಭಾವಾವೇಶಭರಿತರಾಗಿ ಅರೆಪ್ರಜ್ಞಾವಸ್ಥೆಯನ್ನು ತಾಳಲಾರಂಭಿಸಿದರು.

ನರೇಂದ್ರ ಪರಿಸ್ಥಿತಿಯನ್ನು ಸೂಕ್ಷ್ಮವಾಗಿ ಗಮನಿಸುತ್ತಿದ್ದ. ಭಕ್ತರೆಲ್ಲ ಆತ್ಮಸಂಯಮಪೂರ್ಣ ವಾದ ಸಾಧನೆಯನ್ನು ಬದಿಗೊತ್ತಿ ಕೇವಲ ಭಾವಾವೇಗಕ್ಕೆ ಬಲಿಯಾಗುತ್ತಿರುವುದರ ಅಪಾಯವನ್ನು ಅವನು ನಿಚ್ಚಳವಾಗಿ ಕಂಡ. ತಾನೀಗ ಪರಿಸ್ಥಿತಿಯನ್ನು ಕೈಗೆತ್ತಿಕೊಳ್ಳದಿದ್ದರೆ ಇದೆಲ್ಲ ವಿಕೋಪಕ್ಕೆ ಹೋಗಬಹುದು ಎಂದು ಊಹಿಸಿದ. ಹಿರಿಯರನ್ನು ತಿದ್ದುವುದು ಕಷ್ಟಸಾಧ್ಯವಾದ್ದರಿಂದ ಯುವಕ ರನ್ನಾದರೂ ಈ ಅಪಾಯದಿಂದ ಪಾರುಮಾಡಬೇಕು ಎಂದು ಆಲೋಚಿಸಿ ತನ್ನ ಯುವಮಿತ್ರರಿಗೆ ಎಚ್ಚರಿಕೆ ಹೇಳಿದ:

“ನೋಡಿ, ನಮ್ಮ ವ್ಯಕ್ತಿತ್ವದಲ್ಲಿ ಸತ್ಪರಿವರ್ತನೆಯನ್ನು ಉಂಟುಮಾಡಲಾಗದ ಈ ಕೇವಲ ಭಾವಾವೇಗದಿಂದ ಏನೂ ಪ್ರಯೋಜನವಿಲ್ಲ. ಮನುಷ್ಯನು ಭಾವಾವೇಗ ಕೆರಳಿದ ಕ್ಷಣದಲ್ಲಿ ಸಾಕ್ಷಾತ್ಕಾರಕ್ಕಾಗಿ ವ್ಯಾಕುಲತೆಯನ್ನು ವ್ಯಕ್ತಪಡಿಸಬಹುದು. ಆದರೆ, ಕಾಮಕಾಂಚನಾಸಕ್ತಿಯಿಂದ ಬಿಡುಗಡೆ ಹೊಂದಬಲ್ಲ ಸಾಮರ್ಥ್ಯವನ್ನು ತಂದುಕೊಡದಿದ್ದರೆ ಅದರಲ್ಲಿ ಏನೇನೂ ಸತ್ತ್ವವಿಲ್ಲ ಎಂಬುದನ್ನು ತಿಳಿದುಕೊಳ್ಳಿ. ಆದ್ದರಿಂದ ಇಂತಹ ಭಾವಾವೇಗದಿಂದ ಆಧ್ಯಾತ್ಮಿಕ ಜೀವನಕ್ಕೆ ಯಾವ ಸಹಾಯವೂ ಇಲ್ಲ, ಪ್ರಯೋಜನವೂ ಇಲ್ಲ. ಈ ಭಾವಾವೇಗದ ಪರಿಣಾಮವಾಗಿ ಕೆಲವರಿಗೆ ಕಣ್ಣೀರು ಸುರಿಯಬಹುದು; ರೋಮಾಂಚನದ ಅನುಭವವಾಗಬಹುದು; ಅಥವಾ ತಾತ್ಕಾಲಿಕವಾಗಿ ಶರೀರಪ್ರಜ್ಞೆ ಕೂಡ ತಪ್ಪಬಹುದು. ಆದರೆ ನಾನು ಸ್ಪಷ್ಟವಾಗಿ ಹೇಳುತ್ತೇನೆ ಕೇಳಿ, ಇದೆಲ್ಲ ಕೇವಲ ನರದೌರ್ಬಲ್ಯದ ಲಕ್ಷಣಗಳೇ ಹೊರತು ನಿಜವಾದ ಆಧ್ಯಾತ್ಮಿಕ ಅನುಭವ ಗಳಲ್ಲವೇ ಅಲ್ಲ, ಮತ್ತು ಇವು ಕೇವಲ ಕೃತಕ, ಅಸಹಜ. ಒಂದು ವೇಳೆ ಮನುಷ್ಯನಿಗೆ ತನ್ನ ಭಾವೋದ್ರೇಕವನ್ನು ಹತೋಟಿಯಲ್ಲಿಡಲು ಸಾಧ್ಯವಾಗದೆ ಹೋದರೆ ಅವನು ನಿಸ್ಸಂಕೋಚವಾಗಿ ವೈದ್ಯರ ಸಲಹೆ ಪಡೆಯಬೇಕು. ಪೌಷ್ಟಿಕ ಆಹಾರವನ್ನು ಸೇವಿಸಿ ಶರೀರವನ್ನು ಬಲಪಡಿಸಿಕೊಳ್ಳ ಬೇಕು. ನಮ್ಮ ಮನಸ್ಸಿನ ಮೇಲೆ ನಮಗೆ ಒಂದು ಬಲವಾದ ಹತೋಟಿ ಸಿಕ್ಕಾಗ ಮಾತ್ರ ನಮ್ಮಲ್ಲಿ ಉದಿಸುವ ಭಾವಗಳು ಪ್ರಾಮಾಣಿಕವಾದವುಗಳು ಎನ್ನಬಹುದು, ಅವುಗಳಲ್ಲಿ ಸತ್ತ್ವವಿದೆ ಎನ್ನ ಬಹುದು. ಎಲ್ಲೋ ಕೆಲವು ವ್ಯಕ್ತಿಗಳಲ್ಲಿ ಮಾತ್ರ ಅವರ ಪ್ರಬಲವಾದ ಆಧ್ಯಾತ್ಮಿಕ ಭಾವಲಹರಿಯು ಅವರೆಷ್ಟೇ ತಡೆಹಿಡಿದರೂ ಕೂಡ ಹೊರಹೊಮ್ಮಿ ಬಂದೀತು. ಆಗ ಅವರಲ್ಲಿ ರೋಮಾಂಚ, ಆನಂದಬಾಷ್ಟ, ಭಾವಪರವಶತೆ–ಇವುಗಳನ್ನೆಲ್ಲ ಕಾಣಬಹುದು, ತಾತ್ಕಾಲಿಕವಾಗಿ ಬಾಹ್ಯಪ್ರಜ್ಞೆ ತಪ್ಪುವುದನ್ನೂ ಕಾಣಬಹುದು, ಆದರೆ ಅಲ್ಪಬುದ್ಧಿಯ ಸಾಮಾನ್ಯ ಜನ ಇದನ್ನೇ ವಿಪರೀತವಾಗಿ ಅರ್ಥಮಾಡಿಕೊಂಡು, ಹೀಗೆಲ್ಲ ಆದರೆ ಮಾತ್ರವೇ ಅದು ಉನ್ನತ ಆಧ್ಯಾತ್ಮಿಕ ಸ್ಥಿತಿ ಎಂದು ಭಾವಿಸುತ್ತಾರೆ ಮತ್ತು ತಾವೂ ಅಂಥ ಲಕ್ಷಣಗಳನ್ನೆಲ್ಲ ತೋರಿಸಲು ತೊಡಗುತ್ತಾರೆ. ತಮ್ಮಲ್ಲಿ ಆಧ್ಯಾತ್ಮಿಕ ಉನ್ನತಿ ಆಗಿರುವುದನ್ನು ಇತರರು ತಿಳಿದುಕೊಳ್ಳಬೇಕು ಎನ್ನುವುದೇ ಅವರ ಉದ್ದೇಶ. ಕಾಲಕ್ರಮದಲ್ಲಿ ಅವರಿಗೆ ಈ ವರ್ತನೆಗಳೆಲ್ಲ ಒಗ್ಗಿಹೋಗುತ್ತದೆ; ಇದರ ಪರಿಣಾಮವಾಗಿ ಅವರ ನರಮಂಡಲ ದುರ್ಬಲಗೊಂಡು ಸ್ವಲ್ಪ ಭಾವಪ್ರಚೋದನೆಯಾದರೂ ಸಾಕು–ತನ್ನಷ್ಟಕ್ಕೇ ಕಣ್ಣಲ್ಲಿ ನೀರು ಬಂದುಬಿಡುತ್ತದೆ, ರೋಮಾಂಚನವಾಗಿಬಿಡುತ್ತದೆ! ಕ್ರಮೇಣ ಅವರು ಇನ್ನಷ್ಟು ದುರ್ಬಲರಾಗಿ ಬುದ್ಧಿಸ್ವಾಸ್ಥ್ಯವನ್ನು ಕಳೆದುಕೊಂಡು ನಿಷ್ಪ್ರಯೋಜಕರಾಗುತ್ತಾರೆ. ಹೀಗೆ ಭಗ ವಂತನ ಸಾಕ್ಷಾತ್ಕಾರಕ್ಕಾಗಿ ಪ್ರಯತ್ನ ಮಾಡುವವರಲ್ಲಿ ನೂರಕ್ಕೆ ಎಂಬತ್ತು ಜನ ಬೂಟಾಟಿಕೆ ದಾಸಯ್ಯಗಳಾಗಿಬಿಡುತ್ತಾರೆ, ಹದಿನೈದು ಜನ ತಲೆಕೆಡಿಸಿಕೊಂಡು ಹುಚ್ಚರಾಗುತ್ತಾರೆ. ಇನ್ನುಳಿದ ಐದು ಜನ ಮಾತ್ರ ಸಾಕ್ಷಾತ್ಕಾರ ಮಾಡಿಕೊಂಡು ಕೃತಾರ್ಥರಾಗುತ್ತಾರೆ, ಆದ್ದರಿಂದ ಎಚ್ಚರಿಕೆಯಿರಲಿ!”

ಮೊದಮೊದಲು ಉಳಿದ ತರುಣ ಶಿಷ್ಯರು ನರೇಂದ್ರನ ಈ ಅಭಿಪ್ರಾಯಗಳನ್ನು ಒಪ್ಪಲಿಲ್ಲ. ಇದೆಲ್ಲ ಅವನ ಕಲ್ಪನೆಗಳು ಮಾತ್ರ ಎಂದುಕೊಂಡರು. ಆದರೆ ಕೆಲವು ಭಕ್ತರನ್ನು ಸೂಕ್ಷ್ಮವಾಗಿ ಗಮನಿಸಿ ನೋಡಿದಾಗ ಗುಟ್ಟೆಲ್ಲ ಹೊರಗೆ ಬಂತು! ಇವರುಗಳು ಕಣ್ಣಲ್ಲಿ ನೀರು ಸುರಿಸುವುದನ್ನು, ಭಾವಪರವಶರಾದಂತೆ ಬೀಳುವುದನ್ನು ಮನೆಯಲ್ಲಿ ಚೆನ್ನಾಗಿ ಅಭ್ಯಾಸ ಮಾಡಿಕೊಂಡು ಬರು ತ್ತಿದ್ದರು! ಇನ್ನು ಕೆಲವರು ಸುಮ್ಮನೆ ಇತರರನ್ನು ಅನುಕರಿಸುತ್ತಿದ್ದರು. ನರೇಂದ್ರನ ಸಸ್ನೇಹ ಬುದ್ಧಿವಾದಕ್ಕೆ ಕಿವಿಗೊಟ್ಟು ಪೌಷ್ಟಿಕ ಆಹಾರವನ್ನು ಸೇವಿಸಿ, ಆತ್ಮಸಂಯಮವನ್ನು ಅಭ್ಯಾಸ ಮಾಡಿದಾಗ ಇವರಲ್ಲಿ ಒಂದಿಬ್ಬರ ಭಾವೋದ್ರೇಕವೆಲ್ಲ ಉಪಶಮನವಾದುದನ್ನು ಶಿಷ್ಯರು ಕಂಡರು. ಆದರೆ ಇನ್ನು ಕೆಲವರ ಬಳಿ ಈ ಉಪಾಯಗಳೂ ಫಲಿಸಲಿಲ್ಲ. ಅವರು ಭಾವೋದ್ರೇಕದ ಪ್ರದರ್ಶನವನ್ನು ನಿಲ್ಲಿಸಲೇ ಇಲ್ಲ. ಆಗ ನರೇಂದ್ರ ಅಂಥವರನ್ನು ಸ್ವಲ್ಪ ರೇಗಿಸುವುದರ ಮೂಲಕ ಸರಿದಾರಿಗೆ ತರಲು ಪ್ರಯತ್ನಿಸಿದ. ಈ ಸೋಗುಗಾರಿಕೆಯ ಬದಲಾಗಿ ತಮ್ಮ ಅಂತಶ್ಶಕ್ತಿಯನ್ನು ಬೆಳೆಸಿಕೊಳ್ಳುವಂತೆ ಅವರಿಗೆ ಬೋಧಿಸಿದ. ಆ ಯುವಕರನ್ನೆಲ್ಲ ಸೇರಿಸಿ ಜ್ಞಾನ-ಭಕ್ತಿ-ವಿವೇಕ- ವೈರಾಗ್ಯಪರವಾದ ಹಾಡುಗಳನ್ನು ಹೇಳಿಸುತ್ತಿದ್ದ. ತನ್ಮೂಲಕ ಅವರಲ್ಲಿ ಧ್ಯಾನ ಜಪ ತಪಗಳ ಆದರ್ಶಗಳನ್ನು ಜ್ವಲಂತಗೊಳಿಸಿದ. ಪ್ರಾಪಂಚಿಕ ಭೋಗಗಳ ವಿಚಾರದಲ್ಲಿ ವೈರಾಗ್ಯ ತಾಳುವಂತೆ ಪ್ರಚೋದಿಸಿದ, ಕೆಲವೊಮ್ಮೆ ಶ್ರೀರಾಮಕೃಷ್ಣರ ಅದ್ಭುತ ಭಕ್ತಿಯ ದ್ಯೋತಕವಾದ ಅವರ ಜೀವನದ ಘಟನೆಗಳನ್ನು ಅತ್ಯಂತ ವರ್ಣಮಯವಾಗಿ, ಮನನಾಟುವಂತೆ ಚಿತ್ರಿಸುತ್ತಿದ್ದ. ಅವರ ದಿವ್ಯತೆಯನ್ನು ಅವನು ವರ್ಣಿಸುವ ಕ್ರಮವನ್ನು ಕಂಡೇ ಅವನ ಸ್ನೇಹಿತರು ಬೆಕ್ಕಸ ಬೆರಗಾಗು ತ್ತಿದ್ದರು. ಇನ್ನು ಕೆಲವು ಬಾರಿ \eng{Imitation of Christ} ಎಂಬ ಪುಸ್ತಕದ ಸಾಲುಗಳನ್ನು ಉದಾಹರಿಸಿ ಹೇಳುತ್ತಿದ್ದ, “ಯಾವನು ಭಗವಂತನನ್ನು ನಿಜವಾಗಿಯೂ ಪ್ರೀತಿಸುತ್ತಾನೆಯೋ ಅಂಥವನ ಜೀವನವೂ ಸ್ವಯಂ ಭಗವಂತನ ರೀತಿಯಲ್ಲಿಯೇ ರೂಪುಗೊಳ್ಳುತ್ತದೆ. ಎಂದರೆ, ಆ ದೈವೀ ಗುಣಗಳು ಅವನಲ್ಲೂ ಪ್ರಕಟಗೊಳ್ಳುತ್ತವೆ. ಆದ್ದರಿಂದ, ನಾವು ಶ್ರೀರಾಮಕೃಷ್ಣರನ್ನು ನಿಜ ವಾಗಿಯೂ ಪ್ರೀತಿಸುತ್ತೇವೆಯೋ ಇಲ್ಲವೋ ಎಂಬುದು ಈ ಒಂದು ಪರೀಕ್ಷೆಯಿಂದಲೇ ಗೊತ್ತಾಗಿಬಿಡುತ್ತದೆ. ನಮಗೆ ಅವರಲ್ಲಿ ನಿಜವಾದ ಭಕ್ತಿ-ಪ್ರೀತಿ ಇರುವುದಾದಲ್ಲಿ, ನಾವು ಅವರನ್ನು ಸುಮ್ಮನೆ ಸ್ತುತಿಸಿದರೆ ಆಗಲಿಲ್ಲ; ಬದಲಾಗಿ ಅವರ ಆ ಪರಮಾದ್ಭುತ ಗುಣವಿಶೇಷಗಳನ್ನು ನಾವೂ ಕಿಂಚಿತ್ತಾದರೂ ಅಳವಡಿಸಿಕೊಂಡು ತೋರಿಸಬೇಕು.” ಮತ್ತು, ಶ್ರೀರಾಮಕೃಷ್ಣರು ಆಗಾಗ ಹೇಳುತ್ತಿದ್ದ ಈ ಅರ್ಥಗರ್ಭಿತ ಮಾತನ್ನು ನೆನಪಿಸಿಕೊಡುತ್ತಿದ್ದ: “ಮೊತ್ತಮೊದಲು ಅದ್ವೈತಜ್ಞಾನ ವನ್ನು ನಿಮ್ಮ ಬಟ್ಟೆಯ ತುದಿಗೆ ಗಂಟುಕಟ್ಟಿಕೊಳ್ಳಿ; ಅನಂತರ ನಿಮಗೆ ಇಷ್ಟಬಂದಂತೆ ಮಾಡಿ.” ಶ್ರೀರಾಮಕೃಷ್ಣರ ಭಾವಸಮಾಧಿಯ ಹಿನ್ನಲೆಯಲ್ಲಿ ಈ ಅದ್ವೈತಜ್ಞಾನವೆಂಬುದು ಹೇಗೆ ನೆಲೆ ಗೊಂಡಿದೆ ಎಂಬುದನ್ನು ಅವರಿಗೆ ವಿವರಿಸಿ, “ನಾವೂ ಆ ಬಗೆಯ ಜ್ಞಾನವನ್ನು ಮೊದಲು ಪಡೆದುಕೊಳ್ಳುವ ಪ್ರಯತ್ನ ಮಾಡಬೇಕು!” ಎಂದು ಸಹಶಿಷ್ಯರನ್ನು ಪ್ರೇರೇಪಿಸುತ್ತಿದ್ದ. 

