
\chapter{ಸತ್ಯಶೋಧಕ}

\noindent

ಭಾರತದಾದ್ಯಂತ ರಾಜಕೀಯ ಆಂದೋಲನಗಳು ಪ್ರಾರಂಭವಾಗಿದ್ದ ಸಮಯ ಅದು. ಜನರ ಲ್ಲೆಲ್ಲ ರಾಷ್ಟ್ರಪ್ರಜ್ಞೆ ಜಾಗೃತವಾಗುತ್ತಿತ್ತು. ಸ್ವತಂತ್ರ ಭಾರತದ ಕಲ್ಪನೆಯೊಂದು ಮೂಡಿಬರು ತ್ತಿತ್ತು. ಅನೇಕ ಧುರೀಣರು ಯುವಜನರ ಮನಸ್ಸಿನಲ್ಲಿ ರಾಷ್ಟ್ರಭಕ್ತಿಯ ಸ್ಫುರಣೆ ಮಾಡತೊಡಗಿ ದ್ದರು. ಅಂಗಸಾಧನೆ ಕೈಗೊಂಡು, ಶರೀರವನ್ನು ವಜ್ರಸಮವಾಗಿ ಬೆಳೆಸುವುದು ರಾಷ್ಟ್ರಪ್ರೇಮದ ಹೆಗ್ಗುರುತು ಎಂಬ ಅಭಿಪ್ರಾಯ ಪ್ರಚಲಿತವಾಗತೊಡಗಿತು. ಸುರೇಂದ್ರನಾಥ ಬ್ಯಾನರ್ಜಿ ಮತ್ತು ಆನಂದಮೋಹನ ಬಸು ಎಂಬ ಮುಖಂಡರು, ಬಂಗಾಳದ ಪ್ರತಿಯೊಬ್ಬ ಯುವಕನೂ ಶಾರೀರಿಕ ವಾಗಿ ಅತ್ಯಂತ ಬಲಶಾಲಿಯಾಗಿರಬೇಕು ಮತ್ತು ತೀವ್ರಗಾಮಿಯಾಗಬೇಕು ಎಂದು ಘೋಷಿಸು ತ್ತಿದ್ದರು. ನರೇಂದ್ರನಿಗೆ ಈ ಅಭಿಪ್ರಾಯ ಸಂಪೂರ್ಣ ಒಪ್ಪಿಗೆಯಾಯಿತು. ಆಯಾ ಕಾಲಸ್ಥಿತಿಗೆ, ಆವಶ್ಯಕತೆಗೆ ಸೂಕ್ತವಾದುದನ್ನು ಗ್ರಹಿಸಿ ನೆರವೇರಿಸುವಲ್ಲಿ ಅವನು ಯಾವಾಗಲೂ ಮುಂದು. ಸರಿ, ಸ್ವಲ್ಪವೂ ಕಾಲವಿಳಂಬ ಮಾಡದೆ ಅಂಬು ಗುಹ ಎಂಬವನ ವ್ಯಾಯಾಮಶಾಲೆಗೆ ಸೇರಿ ಕೊಂಡು, ಹೊಸ ಹುರುಪಿನಿಂದ ವ್ಯಾಯಾಮ-ಮಲ್ಲಯುದ್ಧಾದಿಗಳ ಅಭ್ಯಾಸದಲ್ಲಿ ನಿರತನಾದ. ಅದೇ ವ್ಯಾಯಾಮಶಾಲೆಗೆ ಅವನಂತೆಯೇ ಇನ್ನೂ ಎಷ್ಟೋ ಯುವಕರು ಧಾವಿಸಿಬಂದರು. ಅವರಲ್ಲಿ ಮುಂದೆ ಶ್ರೀರಾಮಕೃಷ್ಣರ ಆಶ್ರಯಕ್ಕೆ ಬಂದು ನರೇಂದ್ರನ ಸಹಶಿಷ್ಯನಾದ ರಾಖಾಲನೂ ಒಬ್ಬ.

ನರೇಂದ್ರನೀಗ ಅಂದಿನ ಕಾಲದ ಕ್ರಾಂತಿಕಾರೀ ವಿಚಾರಧಾರೆಯಲ್ಲಿ ವಿಶೇಷ ಆಸಕ್ತಿ ತಳೆದ. ಆ ದಿನಗಳಲ್ಲಿ ಬ್ರಾಹ್ಮಸಮಾಜ ಎಂಬ ಒಂದು ಹೊಸ ಧಾರ್ಮಿಕ ಪಂಥ ಬಂಗಾಳದ ಜನರ ಮೇಲೆ ಶಕ್ತಿಯುತವಾಗಿ ಪ್ರಭಾವ ಬೀರುತ್ತಿತ್ತು. ಇದು ತನ್ನದೇ ಆದ ನಿರ್ದಿಷ್ಟ ಸಿದ್ಧಾಂತಗಳನ್ನು ಹೊಂದಿದ್ದು, ತಾನು ಹಿಂದೂಧರ್ಮದಿಂದ ಪ್ರತ್ಯೇಕ ಎಂದು ಘೋಷಿಸಿಕೊಂಡಿತ್ತು. ಅತ್ತ ಸಾಂಪ್ರದಾಯಿಕ ಹಿಂದೂಸಮಾಜವು ಸೊರಗಿ ಮೃತಪ್ರಾಯವಾಗಿದ್ದರೆ, ಅದಕ್ಕೆ ಸರಿವಿರುದ್ಧವಾಗಿ ಬ್ರಾಹ್ಮಸಮಾಜ ಅತ್ಯಂತ ತೇಜಸ್ವಿಯಾಗಿ, ಕ್ರಿಯಾಶೀಲವಾಗಿ ಬೆಳೆಯುತ್ತಿತ್ತು. ಕೇಶವಚಂದ್ರ ಸೇನ ಎಂಬವನು ಬ್ರಾಹ್ಮ ಸಮಾಜದ ಮುಖಂಡನಾಗಿದ್ದ. ಅವನು ನೂರಾರು ವೇದಿಕೆಗಳ ವಾಗ್ಮಿವೀರ. ಹರಿತ ವಿಚಾರಗಳಿಂದ ಕೂಡಿದ ಅವನ ಪ್ರಚಂಡ ವಾಕ್ಪಟುತ್ವಕ್ಕೆ ಅಂದಿನ ವಿದ್ಯಾವಂತ ಜನಾಂಗ ಮಾರುಹೋಗಿತ್ತು. ಬಂಗಾಳದ ತರುಣರ ಆರಾಧ್ಯದೇವತೆಯೇ ಆಗಿ ಬಿಟ್ಟಿದ್ದ ಅವನು.

ಅನೇಕ ಶತಮಾನಗಳ ಬ್ರಿಟಿಷರ ದಬ್ಬಾಳಿಕೆ, ಕ್ರೈಸ್ತ ಮತೀಯರ ಆಕ್ರಮಣಶೀಲ ಮನೋ ಭಾವ, ಮುಸಲ್ಮಾನ ಸಂಸ್ಕೃತಿಯ ಸಂಮಿಶ್ರಣ ಮತ್ತು ಸ್ವತಃ ಹಿಂದೂಧರ್ಮದ ಲೋಪದೋಷ ಗಳು–ಈ ಎಲ್ಲವುಗಳ ಒಟ್ಟು ಪರಿಣಾಮದಿಂದಾಗಿ ಹದಿನೆಂಟು-ಹತ್ತೊಂಬತ್ತನೆಯ ಶತಮಾನದ ಹೊತ್ತಿಗೆ ಹಿಂದೂಧರ್ಮ ಅತ್ಯಂತ ಹೀನ ಸ್ಥಿತಿಯನ್ನು ತಲುಪಿದಂತೆ ಕಂಡುಬಂದಿತು. ಈ ಸಮಯದಲ್ಲಿ ಎಚ್ಚತ್ತುಕೊಂಡ ಕೆಲವು ವಿದ್ಯಾವಂತರು ಸೇರಿ, ಹೊಸದೊಂದು ಪಂಥವನ್ನು ಆವಿಷ್ಕರಿಸಿದರು. ಅದೇ ಬ್ರಾಹ್ಮಸಮಾಜ. ಸನಾತನ ಹಿಂದೂಧರ್ಮ ತನ್ನ ಸಂಪ್ರದಾಯಗಳನ್ನೆಲ್ಲ ಹಾಗೆಹಾಗೆಯೇ ಉಳಿಸಿಕೊಂಡು ಜೀವಿಸುವ ದುರ್ಬಲ ಪ್ರಯತ್ನ ಮಾಡುತ್ತಿದ್ದರೆ, ಈ ನೂತನ ಪಂಥವು ಹಿಂದೂಧರ್ಮಕ್ಕೆ ಬದಲಾಗಿ ನಿಲ್ಲುವ ಸನ್ನಾಹದಲ್ಲಿತ್ತು. ಆದರೆ ತತ್ತ್ವಶಃ ಇದು ಹಿಂದೂಧರ್ಮಕ್ಕೆ ವ್ಯತಿರಿಕ್ತವಾದ, ವಿಭಿನ್ನವಾದ ಪಂಥವೇನಾಗಿರಲಿಲ್ಲ. ಹಿಂದೂಧರ್ಮದ ಯಾವ ಅಂಶಗಳನ್ನು ಬ್ರಾಹ್ಮಸಮಾಜೀಯರು ಸಂಕುಚಿತವಾದವುಗಳು, ಕಂದಾಚಾರಗಳು, ಮೂಢ ನಂಬಿಕೆಗಳು ಎಂದು ತೀರ್ಮಾನಿಸಿದ್ದರೋ ಅಂಥವುಗಳನ್ನೆಲ್ಲ ಕತ್ತರಿಸಿ ತೆಗೆದು, ಅದರ ಕೆಲವು ಶ್ರೇಷ್ಠವಾದ–ವಿಚಾರಸಮ್ಮತವಾದ ಅಂಶಗಳನ್ನು ಮಾತ್ರ ಉಳಿಸಿಕೊಂಡರು. ಜೊತೆಗೆ, ಇತರ ಧರ್ಮಗಳಿಂದಲೂ ಅನೇಕ ಉತ್ತಮ ಅಂಶಗಳನ್ನು ಸೇರಿಸಿಕೊಂಡರು. ಈ ಸಂಸ್ಥೆಯ ಜನಕ ರಾಮಮೋಹನ ರಾಯ್. ಇವನ ಬುದ್ಧಿಶಕ್ತಿ ಅಪ್ರತಿಮ, ಇಚ್ಛಾಶಕ್ತಿ ಅದಮ್ಯ. ಇವನು ಕೆಚ್ಚೆದೆಯ ವೀರ, ಬಹುಜನ ಸಮ್ಮಾನಿತ, ಸಜ್ಜನ. ಹಿಂದೂಧರ್ಮದಲ್ಲಿ ಕಾಲಕ್ರಮದಿಂದ ಬೆಳೆದುಬಂದಿದ್ದ ಹಲವಾರು ಲೋಪದೋಷಗಳನ್ನೂ ರಾಷ್ಟ್ರಕ್ಕೇ ಮೃತ್ಯುಸ್ವರೂಪವಾದ ದುಷ್ಟಸಂಪ್ರದಾಯ ಗಳನ್ನೂ ನಿರ್ಮೂಲ ಮಾಡುವ, ರಾಷ್ಟ್ರನಾಡಿಯಲ್ಲಿ ಹೊಸ ರಕ್ತವನ್ನು ಹರಿಸುವ ಧೀರಪ್ರಯತ್ನ ಮಾಡಿದ ಈತ. ಇವನ ಕಾಲಾನಂತರ ಮಹರ್ಷಿ ದೇವೇಂದ್ರನಾಥ ಟಾಗೋರ್ ಹಾಗೂ ಕೆಲಕಾಲದ ಮೇಲೆ ಕೇಶವಚಂದ್ರ ಸೇನ–ಇವರು ಬ್ರಾಹ್ಮಸಮಾಜವನ್ನು ನಡಸಿಕೊಂಡು ಬಂದರು. ಬ್ರಾಹ್ಮ ಸಮಾಜವು ತೇಜಸ್ವಿಯಾಗಿ ಬೆಳಗಿ, ಚರಿತ್ರೆಯಲ್ಲಿ ಒಂದು ಪ್ರಮುಖ ಸ್ಥಾನವನ್ನು ಗಳಿಸಲು ಇವರಿಬ್ಬರ ನಾಯಕತ್ವವೇ ಕಾರಣ ಎನ್ನಬೇಕು.

ಬ್ರಾಹ್ಮಸಮಾಜವು ಹಿಂದೂಧರ್ಮದಲ್ಲಿ ಪ್ರಚಲಿತವಿದ್ದ ಹಲವಾರು ಸಿದ್ಧಾಂತಗಳನ್ನು ಬಲವಾಗಿ ವಿರೋಧಿಸುತ್ತಿತ್ತು. ಬಹುದೇವತಾ ವಾದವನ್ನು, ಎಂದರೆ ಹಲವು ದೇವ-ದೇವತೆಗಳ ಅಸ್ತಿತ್ವವನ್ನೂ ಆರಾಧನೆಯನ್ನೂ ಅದು ಒಪ್ಪುತ್ತಿರಲಿಲ್ಲ; ದೇವರು ನಾನಾ ರೂಪಗಳನ್ನು ತಾಳ ಬಲ್ಲನೆಂದೂ ಒಪ್ಪುತ್ತಿರಲಿಲ್ಲ. ಆದ್ದರಿಂದ ಮೂರ್ತಿಪೂಜೆಯನ್ನು ಮೌಢ್ಯವೆಂದು ಹೀಗಳೆಯು ತ್ತಿತ್ತು. ಅವತಾರತತ್ತ್ವವನ್ನೂ ಅದು ಮನ್ನಿಸುತ್ತಿರಲಿಲ್ಲ. ಆಧ್ಯಾತ್ಮಿಕ ಜೀವನ ನಡೆಸಲು ಒಬ್ಬ ಗುರುವಿನ ಆವಶ್ಯಕತೆ ಇದೆ ಎಂಬ ವಾದವನ್ನೂ ಅದು ತಿರಸ್ಕರಿಸುತ್ತಿತ್ತು. ಆದರೆ ಅದು ದೇವರ ಅಸ್ತಿತ್ವವನ್ನೇ ನಿರಾಕರಿಸಿತ್ತೆಂದಲ್ಲ; ಬದಲಾಗಿ ಏಕದೇವತಾ ವಾದವನ್ನು, ಎಂದರೆ ಇರುವವ ನೊಬ್ಬನೇ ದೇವರು ಎಂಬ ತತ್ತ್ವವನ್ನು ಪ್ರತಿಪಾದಿಸುತ್ತಿತ್ತು. ಅಲ್ಲದೆ ಬ್ರಾಹ್ಮಸಮಾಜವು ಸಮಾಜ ಸುಧಾರಣೆಯ ಕಡೆಗೆ ವಿಶೇಷ ಗಮನ ಕೊಟ್ಟಿತ್ತು. ಮೊದಲನೆಯದಾಗಿ ಜಾತಿಪದ್ಧತಿ ಯನ್ನು ನಿರ್ಮೂಲಗೊಳಿಸಲು ಅದು ಪ್ರಯತ್ನಿಸುತ್ತಿತ್ತು; ಪ್ರತಿಯೊಬ್ಬ ಮಾನವನ ಮೂಲ ಭೂತ ಹಕ್ಕುಗಳಿಗಾಗಿ ಅದು ಹೋರಾಡುತ್ತಿತ್ತು; ಸ್ತ್ರೀಯರ ವಿದ್ಯಾಭ್ಯಾಸ ಹಾಗೂ ಸರ್ವತೋ ಮುಖ ಅಭಿವೃದ್ಧಿಗಾಗಿ ಅದು ಪ್ರಯತ್ನ ನಡೆಸುತ್ತಿತ್ತು. ಇದರೊಂದಿಗೆ ಬಾಲ್ಯವಿವಾಹವೇ ಮೊದಲಾದ ಅನಿಷ್ಟ ಪದ್ಧತಿಗಳನ್ನು ತೊಡೆದುಹಾಕುವ ಹಾಗೂ ಸ್ತ್ರೀಯರ ವಿವಾಹದ ಕನಿಷ್ಠ ವಯೋಮಿತಿಯನ್ನು ಏರಿಸುವ ಪ್ರಯತ್ನ ಮಾಡುತ್ತಿತ್ತು. ಹೀಗೆ ಅದು ಮಹತ್ತರವಾದ ಕಾರ್ಯ ಪ್ರಣಾಳಿಯನ್ನೇನೋ ಹಾಕಿಕೊಂಡಿತ್ತು. ಆದರೆ ನಿರೀಕ್ಷಿಸಿದ ಯಶಸ್ಸು ಅದಕ್ಕೆ ಲಭಿಸಲಿಲ್ಲ. ಅಲ್ಲದೆ ಈ ಬ್ರಾಹ್ಮಸಮಾಜದವರು ಒಂದು ಪ್ರಮುಖವಾದ ಅಂಶವನ್ನು ಕಡೆಗಣಿಸಿಬಿಟ್ಟಿದ್ದರು–ಈ ಸುಧಾರಣೆಗಳೆಲ್ಲ ಸಮಾಜದ ಅಂತರಂಗದಿಂದ ಮೂಡಿಬರಬೇಕೇ ಹೊರತು, ಹೊರಗಿನಿಂದ ಹೇರುವುದರ ಮೂಲಕ ಅಥವಾ ಕೇವಲ ಕಾನೂನಿನ ಮೂಲಕ ಏನನ್ನೂ ಸಾಧಿಸಲಾಗುವುದಿಲ್ಲ; ಒಂದು ವೇಳೆ ಸುಧಾರಣೆ ಕಂಡುಬಂದರೂ ಅದು ಬಹುಕಾಲ ಉಳಿಯುವುದಿಲ್ಲ ಎಂಬ ಸತ್ಯವನ್ನು ಬ್ರಾಹ್ಮಸಮಾಜದ ಧುರೀಣರು ಅರಿತುಕೊಳ್ಳಲಿಲ್ಲ. ಆ ಪ್ರಣಾಳಿಗಳನ್ನು ಕಾರ್ಯಗತಗೊಳಿಸಲು ಬೇಕಾದ ಸಂಪನ್ಮೂಲಗಳಾಗಲಿ ಜನರ ಸಹಕಾರವಾಗಲಿ ಅಗತ್ಯವಿದ್ದಷ್ಟು ದೊರೆಯಲಿಲ್ಲ. ಜೊತೆಗೆ ನಾನಾ ಕಾರಣಗಳಿಂದ ಅದರ ಒಳಗೊಳಗೇ ಒಡಕುಗಳುಂಟಾಗುತ್ತಿದ್ದುವು.

ಈ ಲೋಪದೋಷಗಳೇನೇ ಇದ್ದರೂ ಆ ಕಾಲಕ್ಕೆ ಬ್ರಾಹ್ಮಸಮಾಜದ ವಿಚಾರಧಾರೆ ಹೊಚ್ಚ ಹೊಸದಾಗಿತ್ತು. ಅಲ್ಲದೆ, ವಿಚಾರವಂತರ ದೃಷ್ಟಿಗೆ ಅದು ತುಂಬ ಸೂಕ್ತವಾಗಿ ಕಂಡುಬರುತ್ತಿತ್ತು. ಆದ್ದರಿಂದ ಬಂಗಾಳದ ಎಷ್ಟೋ ಮಂದಿ ಬಿಸಿರಕ್ತದ ವಿದ್ಯಾವಂತ ಯುವಕರು ಈ ಸಮಾಜದ ಧ್ಯೇಯ ಧೋರಣೆಗಳಿಂದ ಆಕರ್ಷಿತರಾಗಿ ಅದರ ಸದಸ್ಯರಾಗಿದ್ದರು. ಹಾಗೆಯೇ, ಕಾಲೇಜು ಮೆಟ್ಟಿಲು ಹತ್ತಿದ ನರೇಂದ್ರನ ಮೇಲೂ ಅದರ ಸಿದ್ಧಾಂತಗಳು ತುಂಬ ಪ್ರಭಾವ ಬೀರಿದುವು. ಆದ್ದರಿಂದ ಅವನು ಅದರ ಸಭೆಗಳಲ್ಲಿ ಭಾಗವಹಿಸಲಾರಂಭಿಸಿದ. ಅದರ ಮುಖಂಡರ ಭಾವನೆ ಗಳೆಲ್ಲ ಅವನಿಗೆ ಒಪ್ಪಿಗೆಯಾದವು. ಜಾತಿಭೇದದಿಂದ ರಾಷ್ಟ್ರದ ಭಾವೈಕ್ಯಕ್ಕೆ ಹೇಗೆ ಹಾನಿ ಯುಂಟಾಗಿದೆ, ಜನ ಹೇಗೆ ನೊಂದಿದ್ದಾರೆ ಎನ್ನುವುದನ್ನು ಮನಗಂಡಿದ್ದ ನರೇಂದ್ರನಿಗೆ ಆ ದಿಸೆಯಲ್ಲಿ ಬ್ರಾಹ್ಮಸಮಾಜದ ಧೋರಣೆ-ಪ್ರಯತ್ನಗಳು ಬಹಳ ಮೆಚ್ಚುಗೆಯಾದುವು. ಹಾಗೆಯೇ, ರಾಷ್ಟ್ರದ ಅವನತಿಗೆ ಬಹುಮಟ್ಟಿಗೆ ಕಾರಣವಾಗಿದ್ದ ಅನಕ್ಷರತೆಯ ನಿರ್ಮೂಲನ ಮಾಡಲು ಅದು ಸ್ತ್ರೀವಿದ್ಯಾಭ್ಯಾಸವನ್ನು ಪ್ರೋತ್ಸಾಹಿಸುತ್ತಿದ್ದುದು ಅವನಿಗೆ ಇಷ್ಟವಾಯಿತು. ಜನರ ವೈಯಕ್ತಿಕ ಹಾಗೂ ಸಾಮಾಜಿಕ ಸಮಸ್ಯೆಗಳನ್ನು, ಮಾತ್ರವಲ್ಲ, ರಾಷ್ಟ್ರೀಯ ಸಮಸ್ಯೆಗಳನ್ನು ಕೂಡ ಪರಿ ಹರಿಸಬಲ್ಲ ಆದರ್ಶ ಸಂಸ್ಥೆಯಿದು ಎಂಬ ಭರವಸೆ ಅವನಲ್ಲಿ ಮೂಡಿತು. ಹಲವಾರು ಪಾಶ್ಚಾತ್ಯ ಹಾಗೂ ಭಾರತೀಯ ತತ್ತ್ವಶಾಸ್ತ್ರಜ್ಞರ ಸಿದ್ಧಾಂತಗಳನ್ನು ಓದಿಕೊಂಡು ಮನನ ಮಾಡಿದ್ದ ಅವನಿಗೆ, ದೇವರ ಕುರಿತಂತೆ ಬ್ರಾಹ್ಮಸಮಾಜದ ‘ಸಗುಣ ನಿರಾಕಾರ’ ಸಿದ್ಧಾಂತ ಬಹಳ ಒಪ್ಪಿಗೆಯಾಯಿತು. ಅಲ್ಲದೆ, ಬ್ರಾಹ್ಮಧುರೀಣನಾದ ಕೇಶವಚಂದ್ರ ಸೇನನಂಥವನ ಅಯಸ್ಕಾಂತೀಯ ವ್ಯಕ್ತಿತ್ವ, ಅದಮ್ಯ ಉತ್ಸಾಹ, ಪ್ರಖರ ವಿಚಾರಧಾರೆ–ಇವುಗಳನ್ನು ತಾನೂ ಮೈಗೂಡಿಸಿಕೊಳ್ಳಬೇಕು ಎಂಬ ತೀವ್ರ ಆಕಾಂಕ್ಷೆ ಅವನಲ್ಲುದಿಸಿತು.

೧೮೭೮ ರಲ್ಲಿ ಬ್ರಾಹ್ಮಸಮಾಜದಲ್ಲಿ ಅಭಿಪ್ರಾಯಭೇದವೇರ್ಪಟ್ಟು ಅದು ಒಡೆದು ಇಬ್ಭಾಗ ವಾಯಿತು. ಪಂಡಿತ ಶಿವನಾಥ ಶಾಸ್ತ್ರಿ ಹಾಗೂ ವಿಜಯಕೃಷ್ಣ ಗೋಸ್ವಾಮಿ ಎಂಬಿಬ್ಬರು ಮುಖಂಡರು ಕೇಶವಸೇನನಿಂದ ಬೇರೆಯಾಗಿ, ಇನ್ನೂ ಅನೇಕ ಸದಸ್ಯರನ್ನು ಸೇರಿಸಿಕೊಂಡು, ‘ಸಾಧಾರಣ ಬ್ರಾಹ್ಮಸಮಾಜ’ ಎಂಬ ಇನ್ನೊಂದು ಸಂಸ್ಥೆಯನ್ನು ಕಟ್ಟಿದರು. ಆಗ ನರೇಂದ್ರನೂ ಚೆನ್ನಾಗಿ ಆಲೋಚಿಸಿ ನೋಡಿ, ಈ ಹೊಸ ಸಂಸ್ಥೆಯನ್ನು ಸೇರಿದ. ಇದೇ ಸಮಯದಲ್ಲಿ ಬಂಗಾಳ ದಲ್ಲಿ ಇನ್ನೊಂದು ಆಂದೋಲನ ಹುಟ್ಟಿಕೊಂಡಿತ್ತು. ಜಾತಿ-ಕುಲ-ಗೋತ್ರ ಭೇದವಿಲ್ಲದೆ ಸರ್ವ ರಿಗೂ ವಿದ್ಯಾಭ್ಯಾಸ ದೊರಕುವಂತಾಗಬೇಕು ಎನ್ನುವುದು ಇದರ ಘೋಷಣೆಯಾಗಿತ್ತು. ನರೇಂದ್ರ ಈ ಸಂಚಲನಕ್ಕೂ ಸೇರಿಕೊಂಡ. ಜನತೆಯನ್ನು ಸತ್ವರಹಿತ ಕಂದಾಚಾರಗಳಿಂದ ಬಿಡಿಸಿ ಪ್ರಗತಿಪರ ಮಾರ್ಗದಲ್ಲಿ ಮುನ್ನಡೆಸುವ ಯಾವುದೇ ಸಂಸ್ಥೆಯಾಗಲಿ, ಚಳವಳಿಯಾಗಲಿ, ಅದಕ್ಕೆ ಸೇರಿಕೊಂಡು ಆ ಪ್ರಯತ್ನದಲ್ಲಿ ನೆರವಾಗಬೇಕು ಎನ್ನುವುದು ನರೇಂದ್ರನ ಅಭಿಲಾಷೆ. ರಾಷ್ಟ್ರದ ಪರಿಸ್ಥಿತಿ ಹೇಗಿದ್ದರೆ ತನಗೇನು ಎಂಬ ನಿರಾಸಕ್ತ-ಸಪ್ಪೆ ಮನೋಭಾವ ಅವನದಲ್ಲ. ಲೌಕಿಕ ವಿಷಯವೇ ಆಗಲಿ ಅಧ್ಯಾತ್ಮಕ್ಕೆ ಸಂಬಂಧಿಸಿದ್ದೇ ಆಗಲಿ, ತನ್ನೊಳಗೆ ಒಂದು ಸಮಸ್ಯೆ ಎದ್ದರೆ, ಅದಕ್ಕೆ ಸೂಕ್ತ ಪರಿಹಾರವನ್ನು ಕಂಡುಕೊಂಡ ಹೊರತು ಅವನಿಗೆ ಸಮಾಧಾನವಿಲ್ಲ. ಈ ದಿನಗಳಲ್ಲಿ ಅವನು ‘ಹಿಂದೂ ಮಹಾಮೇಳ’ ಎಂಬ ಇನ್ನೂ ಒಂದು ಸಂಸ್ಥೆಯ ಕಾರ್ಯಕಲಾಪಗಳಲ್ಲಿ ಆಗಾಗ ಭಾಗವಹಿಸುತ್ತಿದ್ದ. ಇದನ್ನು ಸ್ಥಾಪಿಸಿದ್ದವನು ನವಗೋಪಾಲ ಮಿತ್ರ. (ಈತನ ವ್ಯಾಯಾಮ ಶಾಲೆಗೇ ನರೇಂದ್ರ ಹಿಂದೆ ಹೋಗುತ್ತಿದ್ದುದು.) ಹಿಂದೂಗಳ ಜೀವನಕ್ರಮದ ಸುಧಾರಣೆ ಹಾಗೂ ಹಿಂದೂರಾಷ್ಟ್ರದ ಹಿತಸಾಧನೆ ಈ ಸಂಸ್ಥೆಯ ಮೂಲೋದ್ದೇಶ. ನರೇಂದ್ರ ಹೀಗೆ ಅನೇಕ ಸಂಘಸಂಸ್ಥೆಗಳನ್ನು ಸೇರಿಕೊಂಡ ಕ್ರಮವನ್ನು ವಿಶ್ಲೇಷಿಸಿ ನೋಡಿದಾಗ, ಅವನಲ್ಲಿ ತನ್ನ ಮಾತೃ ಭೂಮಿ, ಧರ್ಮ ಮತ್ತು ಸಂಸ್ಕೃತಿ ಇವುಗಳ ಶ್ರೇಯಸ್ಸಿಗಾಗಿ ಶ್ರಮಿಸಬೇಕೆನ್ನುವ ಮಹದಭಿಲಾಷೆ ತುಂಬಿರುವುದು ಕಂಡುಬರುತ್ತದೆ.

ಈಗ ನಾವು ಕಾಣುವಂತಹ ಯುವಕ ನರೇಂದ್ರನ ವ್ಯಕ್ತಿತ್ವ ಎಂಥದು?... ಹೂವಿನ ಮೊಗ್ಗೇ ಚಂದ; ಅದು ಅರಳಲು ಪ್ರಾರಂಭವಾದಾಗ ಇನ್ನೂ ಚಂದ; ಅರಳಿ ಸುಗಂಧವನ್ನು ಹೊರಸೂಸಲಾರಂಭಿಸಿದಾಗ ಚಂದವೋ ಚಂದ. ಅಂತೆಯೇ ನರೇಂದ್ರನ ಬಾಲ್ಯವೇ ಚಂದ; ಕೌಮಾರ್ಯ ಇನ್ನೂ ಚಂದ; ಈಗ ಯೌವನ ಮತ್ತೂ ಚಂದ; ಜೊತೆಗೆ ಶೀಲವೆಂಬ ಸುವಾಸನೆ ಕೂಡ ಸೇರಿಕೊಂಡಿದೆ! ನರೇಂದ್ರನ ವ್ಯಕ್ತಿತ್ವದ ಅತ್ಯಂತ ಪ್ರಾಮುಖ್ಯವಾದ ಅಂಶ ಯಾವುದೆಂದರೆ ಪವಿತ್ರತೆ-ಪರಿಶುದ್ಧತೆ. ಈ ಪವಿತ್ರತೆಯ ವಿಷಯದಲ್ಲಿ ನರೇಂದ್ರ ತುಂಬ ಕಠೋರ. ಸಾಮಾನ್ಯ ವಾಗಿ ಯುವಕ-ಯುವತಿಯರು ತಮ್ಮನ್ನು ತಪ್ಪುದಾರಿಗೆಳೆಯುವ ಹಲವಾರು ಪ್ರಭಾವಗಳಿಗೆ ಒಳಗಾಗುತ್ತಾರೆ. ಅಶ್ಲೀಲ ಸಾಹಸಗಳಲ್ಲಿ ತೊಡಗಿ ತಮ್ಮ ಪವಿತ್ರತೆಯನ್ನು ಕಳೆದುಕೊಳ್ಳಲು ಅವರಿಗೆ ನಾನಾ ಮಾರ್ಗಗಳು ಎದುರಾಗುತ್ತವೆ. ಆದರೆ ಭುವನೇಶ್ವರಿ ದೇವಿ ಮಗನಿಗೆ ಬಾಲ್ಯ ದಿಂದಲೂ ಸೂಕ್ತಮಾರ್ಗದರ್ಶನ ನೀಡಿದ್ದಳು. ಅವನೊಬ್ಬ ಸಜ್ಜನ-ಸುಚರಿತ್ರನೆನಿಸಿಕೊಳ್ಳಬೇಕಾ ದರೆ, ಅವನನ್ನು ಹೆತ್ತವಳಾದ ತನಗೂ ಅವನ ಕುಲಕ್ಕೂ ಅಪಖ್ಯಾತಿ ಬಾರದಂತೆ ನಡೆದುಕೊಳ್ಳ ಬೇಕು ಎನ್ನುವ ವಿಷಯವನ್ನು ಅವನಿಗೆ ಮನದಟ್ಟು ಮಾಡಿಸಿದ್ದಳು. ಅಲ್ಲದೆ, ಯಾವುದೇ ಅಶ್ಲೀಲ ಅನೈತಿಕ ಕಾರ್ಯದಲ್ಲಿ ತೊಡಗದಂತೆ ಯಾವುದೋ ಒಂದು ಅಗೋಚರ ಶಕ್ತಿ ತನ್ನನ್ನು ತಡೆಹಿಡಿಯುತ್ತಿತ್ತು ಎಂದು ಮುಂದೆ ಸ್ವಾಮಿ ವಿವೇಕಾನಂದರೇ ಹೇಳುತ್ತಾರೆ. ಯುವಕರಲ್ಲಿ ಸಾಹಸಪ್ರವೃತ್ತಿಯೆನ್ನುವುದು ಸಹಜವಾಗಿಯೇ ಇರುತ್ತದೆ. ಆದರೆ ಪ್ರಾಪಂಚಿಕ ಮನೋಭಾವದ ಸಾಮಾನ್ಯ ಯುವಕರು ಆ ಪ್ರವೃತ್ತಿಯನ್ನು ಯಾವ ರೀತಿಯಲ್ಲಿ ಮೆರೆಸಿಯಾರು ಎಂದು ಹೇಳಬರುವಂತಿಲ್ಲ. ಆದರೆ ನರೇಂದ್ರ ತನ್ನ ಸಾಹಸವನ್ನು ವ್ಯಕ್ತಪಡಿಸುವಾಗ ಅಥವಾ ತನ್ನ ವಯೋಗುಣಕ್ಕೆ ಸಹಜವಾದ ಆಮೋದಪ್ರಮೋದಗಳಲ್ಲಿ ತೊಡಗಿರುವಾಗ ಕೂಡ ಅದರಲ್ಲಿ ಒಂದು ಬಗೆಯ ಅಲೌಕಿಕತೆಯಿರುತ್ತಿತ್ತು. ಅಡ್ಡದಾರಿಗೆಳೆಯುವಂತಹ ಪ್ರಲೋಭನೆಗಳು ಎದುರಾ ದಾಗ ಅವನು ಕಿಂಚಿತ್ತೂ ವಿಚಲಿತನಾಗುತ್ತಿರಲಿಲ್ಲ. ಅವನ ಪವಿತ್ರತೆಯ ಸಂಬಂಧವಾಗಿ ಅವನ ಒಬ್ಬ ಸಹಪಾಠಿಯ ಮಾತುಗಳನ್ನು ಇಲ್ಲಿ ಉಲ್ಲೇಖಿಸಬಹುದು. ಸ್ವತಃ ಆ ಯುವಕನೂ ನೈತಿಕ ಜೀವನವನ್ನು ಅಷ್ಟು ಗಂಭೀರವಾಗಿ ಪರಿಗಣಿಸಿದವನೇನಲ್ಲ. ಆದರೆ ಅವನು ಮುಂದೆ ತನ್ನ ಜೀವನಕ್ರಮವನ್ನು ತಿದ್ದಿಕೊಂಡು ಸ್ವಾಮಿ ವಿವೇಕಾನಂದರ ಶಿಷ್ಯನಾಗುತ್ತಾನೆ. ಈತ ತನ್ನ ಕಾಲೇಜು ದಿನಗಳನ್ನು ನೆನಪಿಸಿಕೊಂಡು ಹೇಳುತ್ತಾನೆ: “ಆಗಲೂ ನರೇಂದ್ರ ಆಧ್ಯಾತ್ಮಿಕತೆಯ ಅಗ್ನಿಜ್ವಾಲೆ ಯಂತಿದ್ದ. ಅವನೆಂದೂ ನೀತಿಯ ಚೌಕಟ್ಟನ್ನು ಮೀರದಿರುವುದನ್ನು ಕಂಡು ಅವನನ್ನು ‘ಮಡಿ ಮನುಷ್ಯ’ ಎಂದು ತಮಾಷೆ ಮಾಡುತ್ತಿದ್ದೆ. ಆದರೆ ಅವನ ಜೊತೆಯಲ್ಲಿರುವಾಗ ಮಾತ್ರ ನನ್ನ ಮನಸ್ಸು ಅಳುಕಿ ಮುದುಡಿಹೋಗುತ್ತಿತ್ತು. ಏಕೆಂದರೆ ಅವನ ಮುಂದೆ ನನ್ನ ನೈತಿಕತೆಯ ಅಭಾವವೆನ್ನುವುದು ನನ್ನ ಕಣ್ಣನ್ನೇ ಕುಕ್ಕುತ್ತಿತ್ತು. ನೈತಿಕ-ಆಧ್ಯಾತ್ಮಿಕ ತೇಜಸ್ಸು ಅವನಿಂದ ಅಕ್ಷರಶಃ ಹೊರಸೂಸುತ್ತಿತ್ತು. ಅದು ಇತರರ ಮೇಲೂ ಪ್ರಭಾವ ಬೀರುವಷ್ಟು ಪ್ರಖರವಾಗಿತ್ತು.”

ನರೇಂದ್ರನ ಎಲ್ಲ ಸಂಗಾತಿಗಳು ಅವನ ಪವಿತ್ರತೆಯ ಪ್ರಕಾಶದಿಂದ ಹೀಗೆಯೇ ಪ್ರಭಾವಿತ ರಾಗಿದ್ದರು. ಅವನೀಗ ಒಬ್ಬ ಆದರ್ಶವಾದಿ; ತನ್ನ ಮನೋಸಾಮ್ರಾಜ್ಯದಲ್ಲಿ ಆದರ್ಶದ ಅರಮನೆ ಕಟ್ಟಿಕೊಂಡು ಆದರ್ಶದ ದೃಶ್ಯಗಳ ನಡುವೆ ವಿಹರಿಸುತ್ತಿದ್ದಾನೆ. ನಿರಾಶೆಯ ಕಹಿಯನ್ನು ಅವನು ಕಂಡವನೇ ಅಲ್ಲ. ಆದರೆ ಮುಂದೆ ಜೀವನರಂಗಕ್ಕೆ ಇಳಿದಾಗ ತನ್ನ ಕಲ್ಪನೆಯ ಗೋಪುರಗಳೆಲ್ಲ ಕುಸಿದು ಬೀಳಬಹುದೆಂಬುದು ಅವನ ಊಹೆಗೆ ಇನ್ನೂ ಸಿಕ್ಕಿಲ್ಲ. ಭವಿಷ್ಯವೆಲ್ಲ ಸುಖದ ಸುಪ್ಪತ್ತಿಗೆ ಯಂತೆ ಕಂಡುಬರುತ್ತಿದೆ. ಯೌವನದಲ್ಲಿ ಇದು ತೀರಾ ಸಾಮಾನ್ಯ. ಈ ಆದರ್ಶ ಎನ್ನುವುದು ಕನಸಿನ ಲೋಕದಿಂದ ಪ್ರತ್ಯಕ್ಷ ಜೀವನಕ್ಕಿಳಿದು, ಅದರ ಸತ್ಯದರ್ಶನ ಮಾಡಿಕೊಳ್ಳುವಂತಾಗ ಬೇಕಾದರೆ ವ್ಯಕ್ತಿಯು ಹಲವಾರು ವಿಷಮ ಪರಿಸ್ಥಿತಿಗಳನ್ನು ಎದುರಿಸಬೇಕಾಗುತ್ತದೆ; ಹಲವು ಬಾರಿ ಕಾಲು ಜಾರಿ ಸೋಲುಗಳನ್ನು ಎದುರಿಸಿಯೇ ಮೇಲೇರಿ ಬರಬೇಕಾಗುತ್ತದೆ. ಬಹಳಷ್ಟು ಆಲೋಚಿಸಿ-ವಿಮರ್ಶಿಸಿ ಮುಂದುವರಿಯಬೇಕಾಗುತ್ತದೆ.

ನರೇಂದ್ರನಲ್ಲಿ ಆಧ್ಯಾತ್ಮಿಕ ಪ್ರವೃತ್ತಿ ಕ್ರಮೇಣ ಬಲಗೊಂಡು ಅದು ಅವನ ವ್ಯಕ್ತಿತ್ವದ ಮೇಲೆ ಒತ್ತಡ ಬೀರಲಾರಂಭಿಸಿತ್ತು. ಯುವಕ ನರೇಂದ್ರನೀಗ ಎರಡು ವಿಭಿನ್ನ ಆದರ್ಶಗಳನ್ನು ಕಾಣು ತ್ತಿದ್ದಾನೆ. ಪ್ರತಿರಾತ್ರಿಯೂ ಮಲಗಿಕೊಳ್ಳುವಾಗ ಅವನ ಬಗೆಗಣ್ಣಿನ ಮುಂದೆ ಪರಸ್ಪರ ವಿರುದ್ಧ ಸ್ವಭಾವದ ಎರಡು ಅದ್ಭುತ ದೃಶ್ಯಗಳು ಮೂಡಿ ನಿಲ್ಲುತ್ತವೆ. ಒಂದು ದೃಶ್ಯದಲ್ಲಿ ಅವನು ಕಾಣುತ್ತಿದ್ದುದು ವಿಲಾಸಪೂರ್ಣವಾದ ಭೋಗ ಜೀವನ; ಹೇರಳವಾಗಿ ಹಣ ಸಂಪಾದಿಸಿ, ಹೆಸರು-ಕೀರ್ತಿಗಳನ್ನು ಮೆರೆದಾಡುತ್ತ ಅಧಿಕಾರವನ್ನು ಚಲಾಯಿಸುತ್ತ, ಆ ಮೂಲಕ ನಡೆಸುವ ವೈಭೋಗದ ಬಾಳ್ವೆ; ಇವುಗಳ ಜೊತೆಗೆ ಮನವೊಪ್ಪುವ ಮಡದಿ, ಮುದ್ದಿನ ಮಕ್ಕಳು– ಇವುಗಳಿಂದ ಕೂಡಿದ ಸುಖ ಸಂಸಾರದ ಜೀವನ! ಅಥವಾ ಒಂದೇ ಮಾತಿನಲ್ಲಿ ಹೇಳುವುದಾದರೆ, ‘ಪ್ರಾಪಂಚಿಕ ಜೀವನ’. ಎರಡನೆಯ ದೃಶ್ಯದಲ್ಲಿ ಅವನು ಕಾಣುವುದು ಸರ್ವಸಂಗ ಪರಿತ್ಯಾಗಿ ಯಾದ ಸಂನ್ಯಾಸಿಯ ಆದರ್ಶವನ್ನು. ಯಾವುದೇ ಬಗೆಯ ಸ್ವಂತದ ಸೊತ್ತೂ ಇಲ್ಲದ, ತನ್ನವ ರೆನ್ನುವ ಯಾರೂ ಇಲ್ಲದ, ಭಗವಂತನ ಇಚ್ಛೆಗೆ ಸರ್ವಸಮರ್ಪಣೆ ಮಾಡಿಕೊಂಡು ಭಗವದ್ಭಾವ ದಲ್ಲೇ ತನ್ಮಯನಾಗಿದ್ದುಕೊಂಡು, ಊರಿಂದೂರಿಗೆ ಸಂಚರಿಸುತ್ತ, ಭಗವಂತನ ಕೃಪೆಯಿಂದ ತಾನಾಗಿಯೇ ಬಂದ ಭಿಕ್ಷೆಯಿಂದ ಜೀವಿಸುತ್ತ, ಮರದ ಕೆಳಗೋ ಬೆಟ್ಟದ ತಪ್ಪಲಲ್ಲೋ ಎಲ್ಲೋ ಒಂದು ಕಡೆ ರಾತ್ರಿಯನ್ನು ಕಳೆಯುತ್ತ ಪರಿವ್ರಾಜಕ ಜೀವನ ನಡೆಸುವ ವಿರಾಗಿಯ ಆದರ್ಶ! ಎಂತಹ ವಿಭಿನ್ನ ಆದರ್ಶಗಳು! ಆದರೆ ನರೇಂದ್ರನಿಗೆ ದೃಢನಂಬಿಕೆಯಿದೆ; ಏನೆಂದರೆ ಈ ಎರಡರಲ್ಲಿ ತಾನು ಯಾವುದೇ ದಾರಿಯನ್ನು ಆರಿಸಿಕೊಂಡರೂ ಅದರಲ್ಲೇ ಯಶಸ್ವಿಯಾಗಿ ಮುನ್ನಡೆದು ಆ ಆದರ್ಶವನ್ನು ಸಿದ್ಧಿಸಿಕೊಳ್ಳಬಲ್ಲೆ ಎಂದು. ಆದ್ದರಿಂದ ಅವನು ಕೆಲವು ಸಲ ತನ್ನನ್ನು ಒಬ್ಬ ಗೃಹಸ್ಥನಂತೆ ಭಾವಿಸಿ ನೋಡುತ್ತಿದ್ದ. ಇನ್ನು ಕೆಲವು ಸಲ ಒಬ್ಬ ಪರಿವ್ರಾಜಕ ಸಂನ್ಯಾಸಿಯಾಗಿ ಭಾವಿಸಿಕೊಳ್ಳುತ್ತಿದ್ದ. ಅವನೊಳಗೆ ಇಬ್ಬರು ಕಲಾವಿದರು ಸೇರಿಕೊಂಡು ತಮ್ಮ ಕುಂಚಗಳಿಂದ ಎರಡು ಬಗೆಯ ಚಿತ್ರಗಳನ್ನು ರಚಿಸುತ್ತಿದ್ದಂತಿತ್ತು–ಒಬ್ಬ ಕಲಾವಿದ ಭೋಗ ಜೀವನದ ಸುಂದರ ದೃಶ್ಯವನ್ನು ಚಿತ್ರಿಸಿದರೆ ಇನ್ನೊಬ್ಬನು ತ್ಯಾಗಜೀವನದ ವೈಭವವನ್ನು ರಚಿಸು ತ್ತಾನೆ. ಆದರೆ ತನ್ನ ಮನಸ್ಸಿನ ಆಳಕ್ಕಿಳಿದು ನೋಡಿದಂತೆ, ಅವನಿಗೆ ತ್ಯಾಗಜೀವನದ ಚಿತ್ರವೇ ಹೆಚ್ಚು ಆಕರ್ಷಕವಾಗಿ, ಸ್ಪಷ್ಟವಾಗಿ ಕಾಣಲಾರಂಭಿಸಿತು. ನೋಡನೋಡುತ್ತಿದ್ದಂತೆ ಭೋಗ ಜೀವನದ ಚಿತ್ರ ಮಾಸಲಾಯಿತು; ಕೊನೆಗೆ ಸಂಪೂರ್ಣವಾಗಿ ಅಳಿಸಿಯೇ ಹೋಯಿತು! ಈ ಸಂಘರ್ಷ ಕೆಲಕಾಲ ನಡೆಯಿತು. ಕೊನೆಗೆ ನರೇಂದ್ರನ ಆಧ್ಯಾತ್ಮಿಕ ವ್ಯಕ್ತಿತ್ವ ಅವನ ಲೌಕಿಕ ವ್ಯಕ್ತಿತ್ವವನ್ನು ಮೆಟ್ಟಿನಿಂತು ಸಂಪೂರ್ಣ ಜಯಶಾಲಿಯಾಯಿತು. ಅವನು ಭಗವಂತನ ನೇರ ಸಾಕ್ಷಾತ್ಕಾರಕ್ಕೆ ನೆರವಾಗುವಂತಹ ತ್ಯಾಗಜೀವನವನ್ನೇ ಆರಿಸಿಕೊಂಡ.

ಈ ನಡುವೆ ಅವನು ಬ್ರಾಹ್ಮಸಮಾಜದ ನಿಕಟ ಸಂಪರ್ಕಕ್ಕೆ ಬಂದಿದ್ದ. ಅಲ್ಲಿನ ಬೌದ್ಧಿಕ ವಾತಾವರಣ ಅವನ ವಿಚಾರವಂತಿಕೆಗೆ ಬಹಳಷ್ಟು ತೃಪ್ತಿ ನೀಡುತ್ತಿತ್ತು. ಪ್ರಾರ್ಥನೆ-ಭಜನೆಯ ಕಾರ್ಯಕ್ರಮಗಳು, ಧಾರ್ಮಿಕ ಪ್ರವಚನಗಳು ಅಲ್ಲೊಂದು ಆಧ್ಯಾತ್ಮಿಕ ವಾತಾವರಣವನ್ನು ನಿರ್ಮಾಣ ಮಾಡಿದ್ದುವು. ಇದರಿಂದ ಅವನ ಆಧ್ಯಾತ್ಮಿಕ ತೃಷೆಗೂ ಒಂದಿಷ್ಟು ಸಮಾಧಾನ ಸಿಗುತ್ತಿತ್ತು.

ಸದಸ್ಯರು ಮಾಂಸಾಹಾರ ವರ್ಜಿಸಬೇಕೆಂಬುದು ಬ್ರಾಹ್ಮಸಮಾಜದ ನಿಯಮಗಳಲ್ಲೊಂದು. ಆದರೆ ನರೇಂದ್ರ ಹುಟ್ಟಿನಿಂದ ಕ್ಷತ್ರಿಯ; ಅವನಿಗೆ ಮಾಂಸಾಹಾರ ರೂಢಿಯಾಗಿತ್ತು. ಈಗ ಬ್ರಾಹ್ಮಸಮಾಜದ ನಿಯಮದ ಪ್ರಕಾರ ಅವನು ಮಾಂಸಾಹಾರವನ್ನು ವರ್ಜಿಸಬೇಕಾಯಿತು. ಆದರೆ ಆತ ಆದರ್ಶಪರಿಪಾಲನೆಗಾಗಿ ಏನನ್ನು ಬೇಕಾದರೂ ಮಾಡಬಲ್ಲ, ಯಾವುದನ್ನಾದರೂ ಬಿಡಬಲ್ಲ. ಆದ್ದರಿಂದ ಈಗ ಅವನು ಸಂಪೂರ್ಣವಾಗಿ ಸಸ್ಯಾಹಾರಿಯೇ ಆದ. ಈ ಸಂದರ್ಭದಲ್ಲಿ ಒಂದು ಸ್ವಾರಸ್ಯಕರ ಘಟನೆ ನಡೆಯಿತು. ನರೇಂದ್ರನ ಮನೆಯಲ್ಲಿ ಮೀನು-ಮಾಂಸ ಸರ್ವೇ ಸಾಮಾನ್ಯ. ಆದರೀಗ ಇದ್ದಕ್ಕಿದ್ದಂತೆ ಅವನೊಬ್ಬ ಸಸ್ಯಾಹಾರಿಯಾಗಿಬಿಟ್ಟಿದ್ದಾನೆ! ಈಗ ಅವನಿ ಗೋಸ್ಕರವೇ ಬೇರೆ ಅಡಿಗೆ ಮಾಡಬೇಕಾಯಿತು. ಮನೆಯಲ್ಲಿ ಒಂದು ಅಡಿಗೆ ಮಾಡಿ ಹಾಕುವು ದಕ್ಕೇ ಎಷ್ಟು ಕಷ್ಟ! ಹೀಗಿರುವಾಗ ಎರಡೆರಡು ಅಡಿಗೆ ಮಾಡಲು ಯಾರಿಗೆ ತಾನೆ ತಾಳ್ಮೆಯಿರು ತ್ತದೆ? ಒಂದು ದಿನ ನರೇಂದ್ರನ ಅಕ್ಕ ಸ್ವರ್ಣಮುಖಿ, ತರಕಾರಿ-ಮೀನು ಸೇರಿಸಿ ಪಲ್ಯ ಮಾಡಿ ಅದರಲ್ಲಿದ್ದ ಮೀನಿನ ಹೋಳುಗಳನ್ನು ತೆಗೆದು ತರಕಾರಿಯನ್ನು ಮಾತ್ರ ತಮ್ಮನಿಗೆ ಬಡಿಸಿದಳು. ನರೇಂದ್ರನಿಗೆ ಮೀನು ಇಲ್ಲದಿದ್ದರಾಯಿತು ತಾನೆ? ಪಲ್ಯದಲ್ಲಿರುವ ಮೀನಿನ ಹೋಳುಗಳನ್ನು ತೆಗೆದುಬಿಟ್ಟರೆ ಉಳಿದುದೆಲ್ಲ ಸಸ್ಯಾಹಾರವೇ ಎಂಬುದು ಅವಳ ಅಭಿಪ್ರಾಯ. ಆದರೆ ನರೇಂದ್ರನಿಗೆ ಕೋಪ ಬಂತು. ಏಕೆಂದರೆ ಪಲ್ಯದಿಂದ ಮೀನಿನ ಹೋಳುಗಳನ್ನು ತೆಗೆದರೂ ಅದರ ರಸ ತರಕಾರಿಯೊಂದಿಗೆ ಸೇರಿಕೊಂಡಿರುವುದಿಲ್ಲವೆ? ಅವನ ವ್ರತಕ್ಕೀಗ ಭಂಗ ಬಂದಿತಲ್ಲ! ಆದ್ದರಿಂದಲೇ ಕೋಪ. ಸೋದರ-ಸೋದರಿಯರಲ್ಲಿ ಬಿರುಸಾದ ವಾಗ್ವಾದ ನಡೆಯಿತು. ಅಲ್ಲಿಯೇ ಹೊರಗೆ ನಿಂತಿದ್ದ ತಂದೆ ವಿಶ್ವನಾಥ ಎಲ್ಲವನ್ನೂ ಕೇಳಿಸಿಕೊಂಡ. ಅವನು ಮಗನ ಮೇಲೆಯೇ ಕೋಪಿಸಿಕೊಂಡು, ಅಲ್ಲಿಂದಲೇ ಗಟ್ಟಿಯಾಗಿ ಹೇಳಿದ, “ಇವನ ತಾತ ಮುತ್ತಾತರಾದಿಯಾಗಿ ಹದಿನಾಲ್ಕು ಪೀಳಿಗೆಯವರೆಲ್ಲ ಮೀನು ತಿಂದೇ ಬದುಕಿದರು. ಆದರೆ ಇವನೀಗ ಬ್ರಹ್ಮದೈತ್ಯ ನಾಗಿಬಿಟ್ಟಿದ್ದಾನೆ!” ಇಲ್ಲಿ ‘ಬ್ರಹ್ಮದೈತ್ಯ’ ಎನ್ನುವ ಶಬ್ದಕ್ಕೆ ಎರಡರ್ಥ–ಒಂದನೆಯದಾಗಿ ಬ್ರಾಹ್ಮಣ ಪ್ರೇತ ಎನ್ನುವ ಅರ್ಥ; ಪ್ರೇತವಾದರೂ ‘ಬ್ರಾಹ್ಮಣ ಪ್ರೇತ’ವಾದ್ದರಿಂದ ಮೀನು ಮಾಂಸ ತಿನ್ನುವುದಿಲ್ಲ. ಎರಡನೆಯದಾಗಿ ನರೇಂದ್ರ ಈಗ ಬ್ರಾಹ್ಮಸಮಾಜಕ್ಕೆ ಸೇರಿದವನಾದ್ದ ರಿಂದ ‘ಬ್ರಹ್ಮದೈತ್ಯ’. ಆದರೆ ವಿಶ್ವನಾಥ ಹೀಗೆಲ್ಲ ಅಂದರೂ ಮಗನ ಸ್ವಾತಂತ್ರ್ಯಕ್ಕೆ ಅಡ್ಡ ಬರುತ್ತಿರಲಿಲ್ಲ. ಅಲ್ಲದೆ ಮಹರ್ಷಿ ದೇವೇಂದ್ರನಾಥ ಟಾಗೋರ್, ಶಿವನಾಥ ಶಾಸ್ತ್ರೀಯಂತಹ ಬ್ರಾಹ್ಮಸಮಾಜದ ಅನೇಕ ಮುಖಂಡರು ವಿಶ್ವನಾಥನ ಮನೆಗೆ ಆಗಾಗ ಬಂದು ಹೋಗುವುದಿತ್ತು.

ಆಹಾರದ ವಿಷಯ ಹಾಗಿರಲಿ; ಸಿದ್ಧಾಂತ ಹಾಗೂ ಸಾಧನೆಯ ವಿಚಾರದಲ್ಲಿ ಹೇಳುವುದಾದರೆ, ಬ್ರಾಹ್ಮಸಮಾಜದವರು ‘ಸಗುಣ-ನಿರಾಕಾರಬ್ರಹ್ಮ’ದ ಉಪಾಸಕರು. ಒಂದೊಂದು ಮತ ಧರ್ಮವೂ ದೇವರನ್ನು ಒಂದೊಂದು ಬಗೆಯಾಗಿ ಭಾವಿಸಿ ಬಣ್ಣಿಸುತ್ತದೆ. ಕೆಲವರು ಪರಮಾತ್ಮ ನನ್ನು ಸಗುಣ-ಸಾಕಾರನೆಂದು ಭಾವಿಸಿದರೆ ಇನ್ನು ಕೆಲವರು ಅವನನ್ನು ನಿರ್ಗುಣ-ನಿರಾಕಾರನೆಂದು ನಂಬುತ್ತಾರೆ. ಸಗುಣ-ಸಾಕಾರನೆಂದು ಬಣ್ಣಿಸುವವರು ಆತನನ್ನು ತಮ್ಮತಮ್ಮ ವಿಶಿಷ್ಟ ದೃಷ್ಟಿ ಕೋನಗಳಿಂದ ಕಂಡು ಅಸಂಖ್ಯಾತ ಗುಣವಿಶೇಷಗಳನ್ನು ಆರೋಪಿಸುತ್ತಾರೆ. ಆದರೆ ಈ ಬ್ರಾಹ್ಮ ಸಮಾಜದವರು ತಮ್ಮದೇ ಆದ ಒಂದು ಸಿದ್ಧಾಂತವನ್ನು ರೂಪಿಸಿಕೊಂಡಿದ್ದರು. ಭಗವಂತನು ಸಗುಣ ಹಾಗೂ ನಿರಾಕಾರ ಎನ್ನುವುದು ಅವರ ಸಿದ್ಧಾಂತ. ಎಂದರೆ, ಭಗವಂತನಲ್ಲಿ ಅನಂತ ಕಲ್ಯಾಣ ಗುಣಗಳಿವೆ, ಆದ್ದರಿಂದ ಅವನು ‘ಸಗುಣ’; ಆದರೆ ಅವನು ಆಕಾಶದಂತೆ ಆಕಾರರಹಿತ, ಆದ್ದರಿಂದ ಅವನು ‘ನಿರಾಕಾರ’–ಇದು ಬ್ರಾಹ್ಮಸಮಾಜದ ಸಿದ್ಧಾಂತ. ಈ ಸಿದ್ಧಾಂತವನ್ನು ನರೇಂದ್ರ ಹೃತ್ಪೂರ್ವಕವಾಗಿ ಒಪ್ಪಿಕೊಂಡು ಸ್ವೀಕರಿಸಿದ. ಆದರೆ ಅವನಿಗೂ ಇತರ ಬ್ರಾಹ್ಮ ಸದಸ್ಯರಿಗೂ ಒಂದು ಬಹುಮುಖ್ಯವಾದ ವ್ಯತ್ಯಾಸವಿತ್ತು. ಏನೆಂದರೆ, ಉಳಿದವರೆಲ್ಲ ಈ ಸಿದ್ಧಾಂತವನ್ನು ಅಂಗೀಕರಿಸಿ ಅದಕ್ಕನುಗುಣವಾಗಿಯೇ ಭಜನೆ-ಪ್ರಾರ್ಥನೆ-ಧ್ಯಾನಾದಿಗಳನ್ನು ಮಾಡಿದರು; ಆಮೇಲೆ ತಮ್ಮತಮ್ಮ ನಿತ್ಯದ ವ್ಯವಹಾರದಲ್ಲಿ ನಿರತರಾಗಿಬಿಟ್ಟರು. ಆ ಭಗವಂತ ನನ್ನು ನಿಜಕ್ಕೂ ಕಾಣಬಹುದು, ಸಾಕ್ಷಾತ್ಕರಿಸಿಕೊಳ್ಳಲೂಬಹುದು ಎಂಬ ನಂಬಿಕೆ ಯಾರಿಗೂ ಇದ್ದಂತಿರಲಿಲ್ಲ! ಆದರೆ ನರೇಂದ್ರ ಮಾತ್ರ ಇವುಗಳನ್ನೆಲ್ಲ ಮಾಡಿದ ಮೇಲೆ ಅಷ್ಟಕ್ಕೇ ಸುಮ್ಮ ನಾಗಿಬಿಡಲಿಲ್ಲ. ಏತಕ್ಕೋಸ್ಕರ ಭಜನೆ-ಪ್ರಾರ್ಥನೆ-ಧ್ಯಾನಗಳನ್ನು ಮಾಡಿದ್ದು? ಭಗವದ್ದರ್ಶನ ಕ್ಕಾಗಿಯೇ ಅಲ್ಲವೆ? ಆದರೆ ಭಗವಂತ ದರ್ಶನ ಕೊಟ್ಟನೆ? ಆತ ಇರುವುದೇ ನಿಜವಾದಲ್ಲಿ ಭಕ್ತನಾದವನು ಪ್ರಾಮಾಣಿಕ ಹೃದಯದಿಂದ ಪ್ರಾರ್ಥನೆ ಸಲ್ಲಿಸಿದಾಗ ಅವನು ಬಂದು ದರ್ಶನ ಕೊಡಲೇಬೇಕು ಎನ್ನುವುದು ನರೇಂದ್ರನ ದೃಢವಾದ ನಂಬಿಕೆ. ಭಗವಂತ ದರ್ಶನ ಕೊಡದೇ ಇದ್ದರೆ ಈ ಭಜನೆ-ಪ್ರಾರ್ಥನೆ-ಧ್ಯಾನ ಇವುಗಳನ್ನೆಲ್ಲ ಮಾಡಿ ಪ್ರಯೋಜನವಾದರೂ ಏನು! ಅಲ್ಲದೆ, ಆತನ ದರ್ಶನಲಾಭವಾಗದಿದ್ದ ಮೇಲೆ ಈ ಜೀವನವೇ ವ್ಯರ್ಥ–ಇದು ನರೇಂದ್ರನ ಭಾವನೆ. ಈಗೀಗ ಅವನಿಗೆ ಅನ್ನಿಸತೊಡಗಿದೆ–ಭಗವಂತನನ್ನು ಸಾಕ್ಷಾತ್ಕಾರ ಮಾಡಿಕೊಳ್ಳುವ ವಿಷಯದಲ್ಲಿ ಬ್ರಾಹ್ಮಸಮಾಜದಿಂದ ತನಗೇನೂ ಪ್ರಯೋಜನವಾಗಲಿಲ್ಲ. ತಾನು ಬ್ರಾಹ್ಮ ಸಮಾಜವನ್ನು ಸೇರುವ ಮೊದಲು ಯಾವ ಸ್ಥಿತಿಯಲ್ಲಿದ್ದೆನೋ ಈಗಲೂ ಹಾಗೆಯೇ ಇದ್ದೇನೆ, ಎಂದು.

ನರೇಂದ್ರನಲ್ಲಿ ಈಗ ಭಗವಂತನ ಸಾಕ್ಷಾತ್ಕಾರಕ್ಕಾಗಿ ಉತ್ಕಟ ವ್ಯಾಕುಲತೆಯುಂಟಾಗಿದೆ. ಅವನಿಗೆ ಈಗ ಪ್ರಾರ್ಥನೆ ಭಜನೆಗಳಷ್ಟರಿಂದಲೇ ಸಮಾಧಾನವಾಗುತ್ತಿಲ್ಲ. ಅವನಿಗೀಗ ಭಗ ವಂತನ ಸಾಕ್ಷಾತ್ಕಾರವೇ ಬೇಕು. ಆಹಾ! ಭಗವಂತನನ್ನು ತೋರಿಸಿಕೊಡುವವರಿದ್ದಿದ್ದರೆ! ಭಗವಂತನನ್ನು ತಮ್ಮ ಸಾಧನೆಗಳ ಮೂಲಕ ಈಗಾಗಲೇ ಕಂಡವರಿರಬಹುದಲ್ಲವೆ? ಅವರನ್ನು ಕೇಳಿದರೆ ಅವರು ತನಗೆ ಭಗವಂತನನ್ನು ತೋರಿಸಿಕೊಡಬಹುದಲ್ಲವೆ? ಆದರೆ, ಅವನು ದೇವಸ್ಥಾನಕ್ಕೆ ಹೋಗಿ ದೇವರನ್ನು ಕಾಣುವಂತಿಲ್ಲ, ಸಾಕ್ಷಾತ್ಕಾರ ನೀಡುವಂತೆ ಕೇಳುವಂತೆಯೂ ಇಲ್ಲ; ಏಕೆಂದರೆ ಅವನಿಗೆ ಮೂರ್ತಿಪೂಜೆಯಲ್ಲಿ ನಂಬಿಕೆಯಿಲ್ಲವಲ್ಲ! ಹಾಗಾದರೆ, ಅವನೀಗ ಬ್ರಾಹ್ಮಸಮಾಜಕ್ಕೆ ಸೇರಿಕೊಂಡು ಸಗುಣ ನಿರಾಕಾರ ಬ್ರಹ್ಮವನ್ನು ಉಪಾಸನೆ ಮಾಡುತ್ತಿದ್ದಾನಲ್ಲ, ಆ ನಿರಾಕಾರ ಬ್ರಹ್ಮವನ್ನೇ ತನಗೆ ಸಾಕ್ಷಾತ್ಕಾರ ಕೊಡುವಂತೆ ಕೇಳಿಕೊಳ್ಳಲಿ? ಇಲ್ಲ, ಅದೂ ಸಾಧ್ಯವಿಲ್ಲ. ಆಕಾರವೇ ಇಲ್ಲದ ದೇವರ ಹತ್ತಿರ ಹೇಗೆ ತಾನೆ ಮಾತನಾಡಿಯಾನು? ಆದ್ದರಿಂದ ಅವನಿಗೀಗ ದೇವರನ್ನು ತೋರಿಸಿಕೊಡಬಲ್ಲವರೊಬ್ಬರು ಬೇಕಾಗಿದ್ದಾರೆ. ನೆನಪು ಮಾಡಿಕೊಂಡ– ಯಾರಿದ್ದಾರೆ ದೇವರನ್ನು ತೋರಿಸಿಕೊಡಬಲ್ಲವರು? ಆಗ ಅವನಿಗೆ ಒಬ್ಬ ವ್ಯಕ್ತಿಯ ನೆನಪಾ ಯಿತು. ಅವರೇ ಮಹರ್ಷಿ ದೇವೇಂದ್ರನಾಥ ಟಾಗೋರರು. ಅವರು ಬ್ರಾಹ್ಮಸಮಾಜದ ಒಬ್ಬ ಪ್ರಾಧಾನ ಆಚಾರ್ಯರು; ಅಲ್ಲದೆ, ಅಂದಿನ ಕಾಲದ ಶ್ರೇಷ್ಠ ಆಧ್ಯಾತ್ಮಿಕ ವ್ಯಕ್ತಿ ಎಂದು ಜನರ ಮನ್ನಣೆಗೆ ಪಾತ್ರರಾಗಿದ್ದವರು. ಹಿಂದೊಮ್ಮೆ ನರೇಂದ್ರ ತನ್ನ ಸ್ನೇಹಿತರೊಡನೆ ಅವರನ್ನು ಭೇಟಿ ಮಾಡಿದ್ದ. ಟಾಗೋರರು ಆ ತರುಣರಿಗೆಲ್ಲ ತೀವ್ರವಾಗಿ ಧ್ಯಾನವನ್ನು ಅಭ್ಯಾಸ ಮಾಡುವಂತೆ ಹೇಳಿದ್ದರು. ಈಗ ಅವರು ಗಂಗಾನದಿಯ ಮೇಲೆ ಒಂದು ದೋಣಿಮನೆಯ ಏಕಾಂತದಲ್ಲಿ ವಾಸವಾಗಿದ್ದಾರೆ ಎಂದು ತಿಳಿದು ನರೇಂದ್ರ ಅವರ ಬಳಿಗೆ ಒಬ್ಬನೇ ಹೋದ; ಹೃದಯದಲ್ಲಿ ಭಗವತ್ಸಾಕ್ಷಾತ್ಕಾರದ ತೀವ್ರ ಹಂಬಲವನ್ನು ತುಂಬಿಕೊಂಡು ಹೋದ. ಅವನ ಆಕಸ್ಮಿಕ ಆಗಮನ ವನ್ನು ಕಂಡು ವೃದ್ಧ ಮಹರ್ಷಿಗಳು ಬೆರಗಾದರು. ಆದರೆ ಅವರಿನ್ನೂ “ಏನಪ್ಪಾ, ಏಕೆ ಬಂದೆ?” ಎಂದು ಕೇಳಲು ಬಾಯಿ ತೆರೆಯುವ ಮೊದಲೇ ನರೇಂದ್ರ ತಾನು ತಂದಿದ್ದ ಪ್ರಶ್ನೆಯನ್ನು ಮುಂದಿಟ್ಟ:

“ಸ್ವಾಮಿ, ನೀವು ದೇವರನ್ನು ಕಂಡಿದ್ದೀರಾ?”

ಆ ಪ್ರಶ್ನೆಗೆ ಉತ್ತರಿಸಲು ಮಹರ್ಷಿಗಳಿಗೆ ಸಾಧ್ಯವಾಗಲಿಲ್ಲ. ಕ್ಷಣಕಾಲ ವಿಚಲಿತರಾದರು. ಬಳಿಕ ಸುಧಾರಿಸಿಕೊಂಡು, ಅವನನ್ನೇ ದಿಟ್ಟಿಸುತ್ತ ಹೇಳಿದರು:

“ಮಗೂ, ನಿನಗೆ ಯೋಗಿಯ ಕಣ್ಣುಗಳಿವೆ.”

ಛೆ! ನರೇಂದ್ರ ಕೇಳಿದ ಪ್ರಶ್ನೆಯೇನು, ಅವನಿಗೆ ಸಿಕ್ಕಿದ ಉತ್ತರವೇನು? ಹಸಿವಿನಿಂದ ತೊಳಲುತ್ತಿರುವವನೊಬ್ಬ ಬಂದು, “ಹಸಿವು ಸ್ವಾಮೀ! ಅನ್ನ ಹಾಕಿ” ಎಂದು ಬೇಡಿದರೆ, “ನೋಡು, ನಿನ್ನಲ್ಲಿ ಉತ್ತಮ ಕೃಷಿಕನ ಲಕ್ಷಣಗಳಿವೆ” ಎಂದು ಯಾರಾದರೂ ಉತ್ತರಿಸಿಯಾ ರೇನು? ನರೇಂದ್ರನಿಗೆ ಸಿಕ್ಕ ಉತ್ತರ ಈ ಬಗೆಯದು. ಅವನಿಗೆ ತುಂಬ ನಿರಾಶೆಯಾಯಿತು. ಅಲ್ಲಿಂದ ಹೊರಟುಬಂದ. ಆದರೆ ಸುಮ್ಮನೆ ಕುಳಿತುಕೊಳ್ಳಲು ಅವನಿಂದ ಸಾಧ್ಯವಿಲ್ಲ. ತಾನು ಭಗವಂತನನ್ನು ಕಾಣಲೇಬೇಕು ಎಂಬ ತವಕ, ವ್ಯಾಕುಲತೆ ಅವನ ಹೃದಯವನ್ನು ಆಕ್ರಮಿಸಿಬಿಟ್ಟಿದೆ. ಹೃದಯದಲ್ಲಿ ಆ ಪ್ರಶ್ನೆಯನ್ನು ಹೊತ್ತುಕೊಂಡು ಇನ್ನೊಬ್ಬ ಧರ್ಮಗುರುವಿನ ಬಳಿಗೆ ಬಂದು ಕೇಳುತ್ತಾನೆ: “ಮಹಾಶಯರೆ, ನೀವು ಕಂಡಿದ್ದೀರಾ ದೇವರನ್ನು?” ಉತ್ತರವಿಲ್ಲ! ಆದರೆ ಅವನು ಅಷ್ಟಕ್ಕೆ ಬಿಡುವವನಲ್ಲ. ಇನ್ನೂ ಹಲವಾರು ಧರ್ಮಗುರುಗಳ ಬಳಿಗೆ, ಆಧ್ಯಾತ್ಮಿಕ ವ್ಯಕ್ತಿಗಳ ಬಳಿಗೆ ಹೋಗಿ ತನ್ನ ಪ್ರಶ್ನೆಯನ್ನು ಮುಂದಿಟ್ಟ. ಆದರೆ ಯಾರೊಬ್ಬರೂ ಆ ಪ್ರಶ್ನೆಗೆ ಸಮಾಧಾನಕರ ಉತ್ತರ ಕೊಡುವ ಸ್ಥಿತಿಯಲ್ಲಿರಲಿಲ್ಲ. ನರೇಂದ್ರನಿಗೂ ಈ ವೇಳೆಗೆ ಆಶ್ಚರ್ಯವಾಗಿರಬೇಕು. “ಎಲಾ! ಏನಿದು ವಿಚಿತ್ರ! ಇವರೆಲ್ಲ ಧರ್ಮಗುರುಗಳಂತೆ, ಆಧ್ಯಾತ್ಮಿಕ ವ್ಯಕ್ತಿಗಳಂತೆ. ಆದರೆ ‘ದೇವರನ್ನು ಕಂಡಿದ್ದೀರಾ?’ ಎಂದು ಕೇಳಿದರೆ, ‘ಹೌದು!’ ಎಂದು ಎದೆಮುಟ್ಟಿ ಉತ್ತರ ಹೇಳುವವರಿಲ್ಲವಲ್ಲ! ಹಾಗಾದರೆ ಇವರೆಲ್ಲ ಸಾಧನೆಯನ್ನೇ ಮಾಡಿಲ್ಲವೋ, ಅಥವಾ ದೇವರೇ ಇಲ್ಲವೋ? ಉಪನಿಷತ್ತು-ಪುರಾಣಗಳೇನೋ ಹೇಳುತ್ತಿದೆ, ‘ದೇವರಿದ್ದಾನೆ’ ಎಂದು. ಆದರೆ ಕಂಡವರಾರು? ಕಾಣಿಸಿಕೊಡಬಲ್ಲವರಾರು?” ಅವನಿಗೀಗ ದಕ್ಷಿಣೇಶ್ವರದ ಶ್ರೀರಾಮಕೃಷ್ಣರ ನೆನಪಾಯಿತು. ಅವರನ್ನೂ ಈ ಕುರಿತಾಗಿ ಒಂದು ಮಾತನ್ನು ಏಕೆ ಕೇಳಿನೋಡಬಾರದು?...

ಕೆಲತಿಂಗಳ ಹಿಂದೆಯಷ್ಟೆ, ಎಂದರೆ ೧೮೮೧ರ ನವೆಂಬರಿನಲ್ಲಿ, ಅವನಿಗೆ ತನ್ನ ನಂಟನಾದ ಸುರೇಂದ್ರನಾಥ ಮಿತ್ರ ಎಂಬಾತನ ಮನೆಯಲ್ಲಿ ಶ್ರೀರಾಮಕೃಷ್ಣರ ಪ್ರಥಮ ದರ್ಶನವಾಗಿತ್ತು. ಶ್ರೀರಾಮಕೃಷ್ಣರ ಭಕ್ತನಾದ ಸುರೇಂದ್ರ ಅವರನ್ನು ತನ್ನ ಮನೆಗೆ ಬರಮಾಡಿಕೊಂಡಿದ್ದ. ಆ ಸಂದರ್ಭದಲ್ಲಿ ಅಭ್ಯಾಗತರ ಸಂತೋಷಕ್ಕಾಗಿ ಹಾಡಲು ಅವನು ನರೇಂದ್ರನನ್ನು ಕರೆಸಿಕೊಂಡಿದ್ದ. ಅಂದು ನರೇಂದ್ರನ ದಿವ್ಯ ಮಧುರ ಗಾನವನ್ನು ಕೇಳಿ ಶ್ರೀರಾಮಕೃಷ್ಣರು ಬಹಳ ಸಂತಸಪಟ್ಟ ರಲ್ಲದೆ ಅವನೆಡೆಗೆ ಆಕರ್ಷಿತರಾದರು. ಅವನು ಯಾರು, ಅವನ ತಂದೆತಾಯಿಗಳು ಯಾರು, ಅವನ ಮನೆ ಎಲ್ಲಿ ಎನ್ನುವುದನ್ನೆಲ್ಲ ವಿವರವಾಗಿ ಕೇಳಿ ತಿಳಿದುಕೊಂಡರು. ಬಳಿಕ ತುಂಬ ವಿಶ್ವಾಸದಿಂದ, “ದಕ್ಷಿಣೇಶ್ವರಕ್ಕೊಮ್ಮೆ ಬಂದುಹೋಗು” ಎಂದು ಆಮಂತ್ರಣವಿತ್ತಿದ್ದರು.

ಆದರೆ, ನರೇಂದ್ರ ಇಷ್ಟೊಂದು ಆಧ್ಯಾತ್ಮಿಕ ಒಲವು ತೋರಿಸುತ್ತಿರುವಾಗ ಅವನ ನಡೆ ನುಡಿ ಭಾವನೆಗಳಲ್ಲೆಲ್ಲ ಆಧ್ಯಾತ್ಮಿಕ ಜೀವನದ ಹಂಬಲ ಕಂಡುಬರುತ್ತಿರುವಾಗ, ಅದು ಅವನ ತಂದೆ ತಾಯಂದಿರ ಗಮನಕ್ಕೆ ಬಾರದೆ ಹೋಗುತ್ತದೆಯೇ? ಬೆಳೆದುನಿಂತ ಮಗ ಎಲ್ಲರಂತೆ ಭೋಗ ವಿಲಾಸಗಳ ಹಿಂದೆ ಓಡಾಡಿಕೊಂಡಿದ್ದರೆ ಅವರೂ ಸುಮ್ಮನಿರುತ್ತಿದ್ದರು. ಆದರೆ ಅವನು ದೇವರು, ಆಧ್ಯಾತ್ಮಿಕ ಸಾಧನೆ, ಸಾಕ್ಷಾತ್ಕಾರ ಎನ್ನತೊಡಗಿದಾಗ ಅವನ ತಂದೆತಾಯಂದಿರು ಅವನನ್ನು ‘ಸ್ವಸ್ಥ’ ಸ್ಥಿತಿಗೆ ತರುವುದಕ್ಕಾಗಿ ಮದುವೆ ಮಾಡಿಬಿಡಲು ಆಲೋಚಿಸಿದರು. ಆದರೆ ನರೇಂದ್ರ ತನ್ನ ದೃಢನಿರ್ಧಾರವನ್ನು ಹೇಳಿಬಿಟ್ಟ–ತಾನು ಮದುವೆಯಾಗುವುದಿಲ್ಲ ಎಂದು. ಬೆಳೆದುನಿಂತ ಮಗ ತಾನು ಮದುವೆಯಾಗುವುದಿಲ್ಲ ಎಂದು ಮುಕ್ತಕಂಠದಿಂದ ಹೇಳಿಬಿಟ್ಟಾಗ ದೊಡ್ಡ ಗಲಾಟೆಗಿಟ್ಟು ಕೊಂಡಿತು. ಈಗ ಅವನಿಗೆ ಬಂಧುಗಳು ಸ್ನೇಹಿತರು ಹಿತೈಷಿಗಳು ಎಲ್ಲರೂ ಬಂದು ಬುದ್ಧಿ ಹೇಳಲಾರಂಭಿಸಿದರು–‘ಸಕಾಲದಲ್ಲಿ ಮದುವೆ ಮಾಡಿಕೊಂಡು ಸುಖವಾಗಿರುವುದೇ ಜಾಣತನ ವಪ್ಪ’ ಎಂದು. ವಿಶ್ವನಾಥ ದತ್ತನ ಸಂಬಂಧಿಯಾದ ರಾಮಚಂದ್ರ ದತ್ತನೂ ನರೇಂದ್ರನಿಗೆ ಮದುವೆ ಮಾಡಿಕೊಳ್ಳುವಂತೆ ಉಪದೇಶಿಸಿದ. ಆದರೆ ಅವನು ತನ್ನ ನಿರ್ಧಾರವನ್ನು ಬಿಡಲಿಲ್ಲ. ಆಗ ರಾಮಚಂದ್ರನಿಗೆ ಅವನ ಭಾವನೆ ಅರ್ಥವಾಯಿತು. ಏಕೆಂದರೆ ಸ್ವತಃ ಅವನೂ ಶ್ರೀ ರಾಮಕೃಷ್ಣರ ಶಿಷ್ಯ. ತ್ಯಾಗದ ಮೂಲಕ ಭಗವಂತನ ಸಾಕ್ಷಾತ್ಕಾರ ಗಳಿಸುವುದೇ ಜೀವನದ ಪರಮೋದ್ದೇಶ ಎಂಬ ಅವರ ಬೋಧನೆಯನ್ನು ಆತ ಬಹಳಷ್ಟು ಕೇಳಿದ್ದ. ಆದ್ದರಿಂದ, ಯಾವಾಗ ನರೇಂದ್ರ ತಾನು ತನ್ನ ಜೀವನವನ್ನು ಭಗವಂತನ ಸಾಕ್ಷಾತ್ಕಾರಕ್ಕಾಗಿ ಮುಡಿಪಾಗಿಟ್ಟಿರುವುದರಿಂದ ಖಂಡಿತ ಮದುವೆಯಾಗುವುದಿಲ್ಲವೆಂಬ ನಿರ್ಧಾರವನ್ನು ತಿಳಿಸಿಬಿಟ್ಟನೋ ಆಗ ರಾಮಚಂದ್ರ ಹೇಳಿದ:“ಹಾಗಾದರೆ ನೋಡಪ್ಪ, ಭಗವಂತನ ಸಾಕ್ಷಾತ್ಕಾರವನ್ನು ಮಾಡಿಕೊಳ್ಳಲೇಬೇಕು ಎನ್ನುವ ಆಕಾಂಕ್ಷೆ ಇರುವುದಾದರೆ ಸುಮ್ಮನೆ ಬ್ರಾಹ್ಮಸಮಾಜ, ಅಲ್ಲಿ-ಇಲ್ಲಿ ಅಂತ ಅಲೆಯಬೇಡ. ಸೀದಾ ದಕ್ಷಿಣೇಶ್ವರಕ್ಕೆ ಹೋಗು, ಅಲ್ಲಿ ಶ್ರೀರಾಮಕೃಷ್ಣರನ್ನು ನೋಡು.” ಈ ವೇಳೆಗೆ ನರೇಂದ್ರನೂ ಹಾಗೆಯೇ ಯೋಚಿಸಿದ್ದ. ದಕ್ಷಿಣೇಶ್ವರಕ್ಕೆ ಹೋಗಬೇಕು, ‘ನೀವು ದೇವರನ್ನು ಕಂಡಿದ್ದೀರಾ?’ ಎಂದು ಶ್ರೀರಾಮಕೃಷ್ಣರನ್ನು ಕೇಳಬೇಕು ಎಂದು ತೀರ್ಮಾನಿಸಿದ್ದ. ಭಗವಂತನನ್ನು ಕಾಣ ಬೇಕೆನ್ನುವ ನಿಜವಾದ ಕಾತರತೆ, ವ್ಯಾಕುಲತೆ ಇವು ಒಬ್ಬನ ಹೃದಯದಲ್ಲಿ ಉತ್ಪನ್ನವಾದಾಗ ಅವನ ಸ್ಥಿತಿ ಹೇಗಿರುತ್ತದೆಂದು ತಿಳಿಯಲು ನರೇಂದ್ರನನ್ನು ನೋಡಬೇಕು. ಅಂಥವನಿಗೆ ಭಗವತ್ಸಾಕ್ಷಾ ತ್ಕಾರದ ವಿಷಯವೊಂದನ್ನು ಬಿಟ್ಟು ಬೇರಾವ ವಿಷಯವೂ ಬೇಕಾಗುವುದಿಲ್ಲ. ಹಸಿವೆಯಿಂದ ಕಂಗಾಲಾದ ಮನುಷ್ಯ ಹೇಗೆ ಅನ್ನ ಹಾಕಬಲ್ಲವರನ್ನು ಅರಸುತ್ತ ಹೋಗುತ್ತಾನೋ ಹಾಗೆಯೇ ಭಗವಂತನನ್ನು ಕಾಣಬೇಕೆಂಬವನು ತನಗೆ ಆತನ ಸಾಕ್ಷಾತ್ಕಾರವನ್ನು ಮಾಡಿಸಿಕೊಡಬಲ್ಲವರನ್ನು ಹುಡುಕಿಕೊಂಡು ಹೊರಟುಬಿಡುತ್ತಾನೆ. ಹಾಗೆಯೇ ಈಗ ನರೇಂದ್ರ ದಕ್ಷಿಣೇಶ್ವರದ ಕಡೆಗೆ ಹೊರಟುಬಿಟ್ಟಿದ್ದಾನೆ. ನರೇಂದ್ರನೆಂಬ ಮರಿಸಿಂಹ ಶ್ರೀರಾಮಕೃಷ್ಣರೆಂಬ ಮಹಾಸಿಂಹದೆಡೆಗೆ ಹೊರಟುಬಿಟ್ಟಿದೆ!... 

ಯಾವುದು ಈ ಮಹಾಸಿಂಹ?

ಯಾರು ಈ ರಾಮಕೃಷ್ಣರು?

