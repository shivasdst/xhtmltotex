
\chapter{ಅತ್ಯಪೂರ್ವ ಆಧ್ಯಾತ್ಮಿಕ ಶಿಕ್ಷಣ}

\noindent

ಮೊತ್ತಮೊದಲಿನಿಂದಲೂ ಶ್ರೀರಾಮಕೃಷ್ಣರು ನರೇಂದ್ರನನ್ನು ಅಣಿಗೊಳಿಸುತ್ತಿದ್ದುದು ಅದ್ವೈತ ತತ್ತ್ವದ ಸಾಕ್ಷಾತ್ಕಾರಕ್ಕೆ. ಈ ಉದ್ದೇಶದಿಂದಲೇ ಅವರು ನರೇಂದ್ರನಿಗೆ ತಮ್ಮ ಬಳಿಯಲ್ಲಿದ್ದ ಅಷ್ಟಾವಕ್ರಸಂಹಿತೆಯೇ ಮೊದಲಾದ ಅದ್ವೈತತತ್ತ್ವವನ್ನು ಪ್ರತಿಪಾದಿಸುವ ಗ್ರಂಥಗಳನ್ನು ಗಟ್ಟಿ ಯಾಗಿ ಓದಲು ಹೇಳುತ್ತಿದ್ದರು. ಅವನಲ್ಲದೆ ಬೇರೆ ಯಾರಾದರೂ ಶಿಷ್ಯರು ಆ ಗ್ರಂಥಗಳನ್ನು ತೆಗೆದು ಓದಲು ಹವಣಿಸಿದರೆ ಶ್ರೀರಾಮಕೃಷ್ಣರು, “ಅವನ್ನು ತೆರೆಯಬೇಡ ಬಿಡು; ಅವು ನಿನಗೆ ಹೇಳಿಸಿದ್ದಲ್ಲ” ಎಂದುಬಿಡುತ್ತಿದ್ದರು. ಆದರೆ ನರೇಂದ್ರನಿಗೆ ಅವುಗಳ ಬೋಧನೆ ಒಪ್ಪಿಗೆಯಾಗು ತ್ತಿರಲಿಲ್ಲ. ಆದ್ದರಿಂದ ಅವನ್ನು ಓದಲೂ ಇಷ್ಟಪಡುತ್ತಿರಲಿಲ್ಲ. ಆದರೂ ಶ್ರೀರಾಮಕೃಷ್ಣರು ಒತ್ತಾಯ ಮಾಡಿ ಓದಿಸುತ್ತಿದ್ದರು. ಈ ಗ್ರಂಥಗಳಲ್ಲಿ ‘ಅಹಂ ಬ್ರಹ್ಮಾಸ್ಮಿ’ ‘ನಾನು ಪರಬ್ರಹ್ಮ’ ‘ನಾನೇ ಎಲ್ಲವೂ ಆಗಿದ್ದೇನೆ’ ಎನ್ನುವಂತಹ ತತ್ತ್ವಗಳನ್ನೇ ಎತ್ತಿಹಿಡಿಯಲಾಗಿದೆ. ನರೇಂದ್ರ ನಾದರೋ ಬ್ರಾಹ್ಮಸಮಾಜದ ‘ಸಗುಣ-ನಿರಾಕಾರ ಬ್ರಹ್ಮ’ದ ತತ್ತ್ವಗಳನ್ನೇ ನಂಬಿ ಸ್ವೀಕರಿಸಿ ದ್ದವನು. ಆದ್ದರಿಂದ ಅವನಿಗೆ ಈ ವೇದಾಂತದ ಭಾವನೆಗಳು ಒಪ್ಪಿಗೆಯಾಗುತ್ತಿರಲಿಲ್ಲ. ಒಮ್ಮೊಮ್ಮೆ ಅವನು ಸಹನೆತಪ್ಪಿ, “ಛೆ! ಇದೆಷ್ಟು ಅವಹೇಳನಕರ! ಈ ಬಗೆಯ ವಾದಕ್ಕೂ ನಾಸ್ತಿಕತೆಗೂ ಏನೇನೂ ವ್ಯತ್ಯಾಸವಿಲ್ಲ. ‘ನಾನೇ ಪರಬ್ರಹ್ಮ’ ಅಂತ ತಿಳಿದುಕೊಳ್ಳುವುದಕ್ಕಿಂತ ದೊಡ್ಡ ಪಾಪ ಜಗತ್ತಿನಲ್ಲಿ ಇನ್ನಾವುದೂ ಇಲ್ಲ. ನಾನೂ ದೇವರಂತೆ, ನೀವೂ ದೇವರಂತೆ, ಈ ಪ್ರಾಣಿಪಕ್ಷಿಗಳೆಲ್ಲವೂ ದೇವರಂತೆ! ಸೃಷ್ಟಿಸಲ್ಪಟ್ಟಿರುವ ಜೀವಿ ತಾನೇ ‘ಸೃಷ್ಟಿಕರ್ತ’ ಅಂತ ತಿಳಿದುಕೊಳ್ಳಲು ಹೇಗೆ ಸಾಧ್ಯ? ಇದಕ್ಕಿಂತ ಹಾಸ್ಯಾಸ್ಪದವಾದದ್ದೇನಾದರೂ ಉಂಟೆ! ಇಂಥದ ನ್ನೆಲ್ಲ ಬರೆದಿದ್ದಾದಲ್ಲ ಆ ಪುಷಿಗಳು, ಅವರಿಗೆ ತಲೆ ಸರಿಯಿರಲಿಲ್ಲ ಅಂತ ಕಾಣುತ್ತದೆ” ಎಂದು ಲೇವಡಿ ಮಾಡುತ್ತಿದ್ದ.

ಅವನ ಈ ಬಗೆಯ ಅಪಹಾಸ್ಯದ ಒರಟುಮಾತುಗಳನ್ನು ಕೇಳಿ ಶ್ರೀರಾಮಕೃಷ್ಣರಿಗೆ ಸಿಟ್ಟೇರಿರ ಬಹುದೆ? ಇಲ್ಲ; ಬದಲಾಗಿ ಅವರು ಇದನ್ನು ತಮಾಷೆಯಾಗಿಯೇ ನೋಡಿದರು. ಮತ್ತು ಇದ್ದಕ್ಕಿ ದ್ದಂತೆ ಅವನ ಭಾವಕ್ಕೆ ಆಘಾತವಾಗಬಾರದೆಂದು ನಯವಾಗಿಯೇ ಛೇಡಿಸಿದರು:

“ಆ ಮಾತುಗಳಲ್ಲಿ ನಿನಗೆ ನಂಬಿಕೆ ಬಾರದಿದ್ದರೆ ಬಿಟ್ಟುಬಿಡು. ನಿನಗೆ ಒಪ್ಪಿಗೆಯಾಗಲಿಲ್ಲ ಎನ್ನುವ ಕಾರಣಕ್ಕೆ ಆ ಮಹರ್ಷಿಗಳನ್ನೆಲ್ಲ ಹಾಗೆ ಹೀಯಾಳಿಸಬಹುದೆ? ಭಗವಂತನ ಮಹಿಮೆ ಅನಂತವಾದದ್ದು. ನೀನೇಕೆ ಅದನ್ನು ಸೀಮಿತಗೊಳಿಸಲು ಹೋಗುತ್ತೀಯೇ? ಆತ ಸತ್ಯಸ್ವರೂಪಿ. ಅವನನ್ನೇ ಪ್ರಾರ್ಥಿಸಿಕೊ. ಅವನು ತನ್ನನ್ನು ಯಾವ ರೂಪದಲ್ಲಿ ತೋರಿಸಿಕೊಡುತ್ತಾನೋ ಅದನ್ನೇ ನಂಬಿ ಸ್ವೀಕರಿಸು.”

ಆದರೆ ಬಡಪೆಟ್ಟಿಗೆ ಬಗ್ಗುವವನಲ್ಲ ನರೇಂದ್ರ. ತನ್ನ ವೈಜ್ಞಾನಿಕ ದೃಷ್ಟಿಗೆ ತಾಳೆಯಾಗದ ಯಾವುದನ್ನೂ ಅವನು ಒಪ್ಪುವವನೇ ಅಲ್ಲ. ಸುಳ್ಳು ಎನಿಸಿದ್ದನ್ನು ಶತಾಯಗತಾಯ ವಿರೋಧಿಸು ವುದು ಆತನ ಗುಣ. ಆದ್ದರಿಂದ ಅವಕಾಶ ಸಿಕ್ಕಾಗಲೆಲ್ಲ ಅದ್ವೈತವಾದವನ್ನು ಗೇಲಿಮಾಡುತ್ತಲೇ ಇದ್ದ. ಏನೇ ಆದರೂ ನರೇಂದ್ರನ ಮಾರ್ಗ ಜ್ಞಾನಮಾರ್ಗವೇ ಎಂಬುದು ಶ್ರೀರಾಮಕೃಷ್ಣರಿಗೆ ಖಂಡಿತವಾಗಿ ತಿಳಿದಿತ್ತು. ಆದ್ದರಿಂದ ಅವರು ಮಾತ್ರ ಅವನಿಗೆ ವೇದಾಂತತತ್ತ್ವಬೋಧನೆ ಮಾಡುವುದನ್ನು ಬಿಡಲಿಲ್ಲ.

ಸಾಮಾನ್ಯವಾಗಿ, ನರೇಂದ್ರ ಬರುತ್ತಿದ್ದಂತೆಯೇ ಶ್ರೀರಾಮಕೃಷ್ಣರು ಭಾವಪರವಶರಾಗಿಬಿಡುತ್ತಿ ದ್ದರು. ಕ್ರಮೇಣ ಅರ್ಧಪ್ರಜ್ಞೆ ಮರಳುತ್ತಿರುವಂತೆ, ಅತ್ಯಾನಂದದಿಂದ ಅವನೊಡನೆ ಆಧ್ಯಾತ್ಮಿಕ ಸಂಭಾಷಣೆಯಲ್ಲಿ ತೊಡಗುತ್ತಿದ್ದರು. ತನ್ಮೂಲಕ ಘನವಾದ ಆಧ್ಯಾತ್ಮಿಕ ರಹಸ್ಯಗಳನ್ನು ಅವನಿಗೆ ತಿಳಿಯಪಡಿಸುತ್ತಿದ್ದರು. ಕೆಲವೊಮ್ಮೆ ಅವರಿಗೆ ಅವನ ಬಾಯಿಂದ ಒಂದೆರಡು ಹಾಡುಗಳನ್ನು ಕೇಳುವ ಇಚ್ಛೆಯಾಗುತ್ತಿತ್ತು. ಆದರೆ ಆತನ ಗಂಧರ್ವಸ್ವರವನ್ನು ಕೇಳಿದೊಡನೆಯೇ ಸಮಾಧಿ! ಆದರೂ ನರೇಂದ್ರ ಹಾಡುವುದನ್ನು ನಿಲ್ಲಿಸದೆ ತನ್ಮಯಚಿತ್ತನಾಗಿ ಗಂಟೆಗಟ್ಟಲೆ ಒಂದಾದ ಮೇಲೊಂದು ಹಾಡು ಹೇಳುತ್ತಲೇ ಹೋಗುತ್ತಿದ್ದ. ಮಧ್ಯೆ ಶ್ರೀರಾಮಕೃಷ್ಣರಿಗೆ ಭಾಗಶಃ ಬಾಹ್ಯ ಪ್ರಜ್ಞೆ ಮರಳಿದರೆ ಆಗ ಅವರು ಅವನನ್ನು ಯಾವುದಾದರೊಂದು ನಿರ್ದಿಷ್ಟ ಹಾಡನ್ನು ಹೇಳುವಂತೆ ಕೇಳುತ್ತಿದ್ದರು. “ಜೋ ಕುಛ್ ಹೈ ತೂ ಹೀ ಹೈ”–‘ಏನೇನಿವೆಯೋ ಅವೆಲ್ಲ ನೀನೇ’ ಎಂಬ ಹಾಡನ್ನಂತೂ ಕೇಳದಿದ್ದರೆ ಅವರಿಗೆ ಸಮಾಧಾನವೇ ಇಲ್ಲ. ಅನಂತರ ಅವರು ಸ್ಪಲ್ಪ ಹೊತ್ತು ಅವನಿಗೆ ಅದ್ವೈತದ ವಿಚಾರಗಳನ್ನು ತಿಳಿಸಿಕೊಡುತ್ತಿದ್ದರು. ಜೀವನಿಗೂ ಈಶ್ವರನಿಗೂ ಇರುವ ಭೇದ, ಜೀವದ ಮತ್ತು ಬ್ರಹ್ಮದ ನಿಜಸ್ವರೂಪ–ಇಂಥವುಗಳ ಬಗ್ಗೆ ಹೇಳುತ್ತಿದ್ದರು. ಈ ನಡುವೆ ನರೇಂದ್ರನಿಗೆ ದಕ್ಷಿಣೇಶ್ವರದಲ್ಲೊಬ್ಬ ‘ಆಪ್ತ ಸಖ’ ಸಿಕ್ಕಿದ್ದ–ಅವನೇ ಹಾಜರಾ. ಹಿಂದೆಯೇ ಹೇಳಿದಂತೆ, ಅವನೊಬ್ಬ ಆಷಾಢಭೂತಿ. ಆದರೂ ಅವನೊಡನೆ ನರೇಂದ್ರನಿಗೆ ಅದೇನೋ ನಂಟು. ಅದೇನೋ ಗಂಟು! ಹಾಜರಾನದು ನರಿಬುದ್ಧಿಯಾದರೆ ನರೇಂದ್ರನದು ಸಿಂಹಬುದ್ಧಿ. ಅಂತೂ ಈ ಸಿಂಹರಾಜನೊಂದಿಗೆ ಆ ನರಿಮಂತ್ರಿಯ ಅಪೂರ್ವಜೋಡಿ! ಶ್ರೀರಾಮಕೃಷ್ಣರ ಬಳಿಗೆ ಬಂದಾಗಲೆಲ್ಲ ನರೇಂದ್ರ ಹಾಜರಾನೊಂದಿಗೆ, ಅನುಕೂಲವಿದ್ದಂತೆ ಒಂದೋ ಎರಡೋ ಗಂಟೆ ಕಾಲ ಹರಟೆ ಕೊಚ್ಚುತ್ತಿದ್ದ. ಅವನ ಮಾತುಗಳನ್ನೆಲ್ಲ ಹಾಜರಾ ಶ್ರದ್ಧೆಯಿಂದ ಆಲಿಸುತ್ತಿದ್ದ. ಅವನಿಗಾಗಿ ಹುಕ್ಕಾ ತಯಾರಿಸಿಕೊಡುತ್ತಿದ್ದ. ಇದನ್ನು ಕಂಡು ಉಳಿದ ತರುಣರು “ಹಾಜರಾ ಮಹಾಶಯ ನರೇಂದ್ರನ ‘ಫೆರೆಂಡು’” ಎಂದು ತಮಾಷೆ ಮಾಡುತ್ತಿದ್ದರು. ಇವನ ವಿಷಯದಲ್ಲಿ ನರೇಂದ್ರನಿಗೂ ಶ್ರೀರಾಮಕೃಷ್ಣರಿಗೂ ಯಾವಾಗಲೂ ಭಿನ್ನಾಭಿಪ್ರಾಯವೇ. ಒಂದು ದಿನ ಹೀಗೇ ಶ್ರೀರಾಮಕೃಷ್ಣರು ‘ಅದ್ವೈತ ವೇದಾಂತದ ದೃಷ್ಟಿಯಿಂದ ಜೀವಾತ್ಮನೂ ಪರಮಾತ್ಮನೂ ಒಂದೇ’ ಎಂಬುದನ್ನು ಅರ್ಥಪಡಿಸುವ ಪ್ರಯತ್ನ ನಡೆಸಿದ್ದಾರೆ. ನರೇಂದ್ರನೇನೋ ಅವರ ಮಾತನ್ನೆಲ್ಲ ಮನವಿಟ್ಟು ಕೇಳಿದ; ಆದರೆ ಅವನಿಗೆ ಅವು ಸ್ವಲ್ಪವೂ ಒಪ್ಪಿಗೆಯಾಗಲಿಲ್ಲ. ಕೆಲಹೊತ್ತಿನ ಮೇಲೆ ಕೋಣೆಯಿಂದ ಆಚೆ ಬಂದ. ಹಾಜರಾ ಅಲ್ಲೇ ಇದ್ದ. ನರೇಂದ್ರ ಹುಕ್ಕಾ ಎಳೆಯುತ್ತ ಅವನೊಂದಿಗೆ ಹರಟುತ್ತ ಕುಳಿತ. ಆಗತಾನೇ ಶ್ರೀರಾಮಕೃಷ್ಣರು ಹೇಳುತ್ತಿದ್ದ ವಿಷಯವನ್ನು ತೆಗೆದು, “ನೋಡಯ್ಯ, ಈ ಪಾತ್ರೆಯೂ ಬ್ರಹ್ಮವಂತೆ, ಲೋಟವೂ ಬ್ರಹ್ಮವಂತೆ! ಏನೇನು ಕಾಣು ತ್ತಿದೆಯೋ ಎಲ್ಲವೂ ಬ್ರಹ್ಮವಂತೆ! ಅಷ್ಟೇ ಅಲ್ಲ, ನಾವೆಲ್ಲರೂ ಬ್ರಹ್ಮವೇ ಅಂತೆ. ಹೇಗಿದೆ ತಮಾಷೆ!” ಎನ್ನುತ್ತ ಲೇವಡಿಮಾಡಿದ. “ಚೆನ್ನಾಗಿದೆ, ಚೆನ್ನಾಗಿದೆ” ಎನ್ನುತ್ತ ನರಿಮಂತ್ರಿ ದನಿಗೂಡಿಸಿದ. ಇಬ್ಬರೂ ಗಹಗಹಿಸಿ ನಕ್ಕರು.

ಈ ವೇಳೆಗೆ ಒಳಗಡೆ ಅರ್ಧಬಾಹ್ಯಾವಸ್ಥೆಯಲ್ಲಿದ್ದ ಶ್ರೀರಾಮಕೃಷ್ಣರು ಮೆಲ್ಲನೆದ್ದು, ಬಾಲಕ ನಂತೆ ತಮ್ಮ ಉಟ್ಟಬಟ್ಟೆಯನ್ನು ಕಂಕುಳಲ್ಲಿಟ್ಟುಕೊಂಡು ಅವರ ಬಳಿಗೆ ಬಂದರು. ಮತ್ತು ಅವರ ನಗುವನ್ನು ಕಂಡು ಮುಗುಳ್ನಗುತ್ತ, ವಾತ್ಸಲ್ಯದ ದನಿಯಲ್ಲಿ ಕೇಳಿದರು: “ಏನ್ರಪ್ಪಾ, ಏನು ಸಮಾಚಾರ?” ಹೀಗೆನ್ನುತ್ತ ಅವರ ಉತ್ತರಕ್ಕಾಗಿ ಕಾಯದೆ ಮುಂದೆ ಬಂದು ನರೇಂದ್ರನನ್ನು ಸ್ಪರ್ಶಿಸಿ ಸಮಾಧಿಸ್ಥರಾದರು.

ಆ ಸ್ಪರ್ಶದಿಂದ ನರೇಂದ್ರನ ಮೇಲುಂಟಾದ ಪರಿಣಾಮ ಮಾತ್ರ ಹಿಂದೆಂದಿಗಿಂತ ವಿಚಿತ್ರ, ಮಹತ್ತರ. ಅದನ್ನು ಅವನೇ ವರ್ಣಿಸಲಿ:

“ಕ್ಷಣಮಾತ್ರದಲ್ಲಿ ನನ್ನ ಮನಸ್ಥಿತಿಯಲ್ಲೇನೋ ಕ್ರಾಂತಿಯಾಗಿಬಿಟ್ಟಿತು. ಈ ಇಡೀ ವಿಶ್ವದಲ್ಲೇ ಪರಬ್ರಹ್ಮವನ್ನುಳಿದು ಬೇರೇನೂ ಇಲ್ಲವೆಂಬುದನ್ನು ಕಂಡು ಆಶ್ಚರ್ಯಾಘಾತಗೊಂಡೆ. ಆದರೂ ಸುಮ್ಮನೆಯೇ ಇದ್ದೆ–ನೋಡೋಣ, ಈ ಸ್ಥಿತಿ ಎಷ್ಟು ಹೊತ್ತಿರುತ್ತದೋ ಅಂತ. ಆದರೆ ಆ ದಿನ ವೆಲ್ಲ ಅದರ ತೀವ್ರತೆ ಕಡಮೆಯಾಗಲೇ ಇಲ್ಲ. ಮನೆಗೆ ಬಂದರೂ ಅದೇ ಸ್ಥಿತಿ. ಏನೇನು ಕಾಣು ತ್ತಿದ್ದೆನೋ ಅವೆಲ್ಲವು ಪರಬ್ರಹ್ಮ ಚೈತನ್ಯವೇ ಎಂಬಂತೆ ಭಾಸವಾಯಿತು. ಊಟಕ್ಕೆ ಕುಳಿತೆ. ಮುಂದಿದ್ದ ತಟ್ಟೆ, ಬಡಿಸಿದ ಅಡಿಗೆ, ಬಡಿಸುತ್ತಿರುವ ತಾಯಿ ಮತ್ತು ಸ್ವತಃ ನಾನು–ಹೀಗೆ ಪ್ರತಿ ಯೊಂದೂ ಅದೇ ಬ್ರಹ್ಮಮಯವಾಗಿ ಕಂಡಿತು. ಒಂದೆರಡು ತುತ್ತು ತಿಂದೆ, ಆಮೇಲೆ ಸುಮ್ಮನೆ ಕುಳಿತುಬಿಟ್ಟೆ. ‘ಯಾಕಪ್ಪ ಮಗು, ಸುಮ್ಮನೆ ಕುಳಿತೆಯಲ್ಲ? ಊಟ ಮಾಡೋದಿಲ್ಲವೆ?’ ಎಂಬ ತಾಯಿಯ ವಾತ್ಸಲ್ಯದ ದನಿ ಕೇಳಿ ಎಚ್ಚರವಾಯಿತು. ಮತ್ತೆ ತಿನ್ನತೊಡಗಿದೆ. ತಿನ್ನುತ್ತಿರಲಿ, ಕುಡಿಯುತ್ತಿರಲಿ, ವಿಶ್ರಮಿಸುತ್ತಿರಲಿ, ಅಡ್ಡಾಡುತ್ತಿರಲಿ–ಅದೇ ಅನುಭವ, ಅದೇ ಅವಸ್ಥೆ! ರಸ್ತೆ ಯಲ್ಲಿ ಹೋಗುವಾಗ ಎದುರಿನಿಂದ ಗಾಡಿ ಬರುತ್ತಿದ್ದರೆ, ಒಮ್ಮೊಮ್ಮೆ ನನಗೆ ಪಕ್ಕಕ್ಕೆ ಸರಿದು ಗಾಡಿಗೆ ದಾರಿ ಬಿಡಬೇಕೆಂದೇ ಅನ್ನಿಸುತ್ತಿರಲಿಲ್ಲ. ಏಕೆಂದರೆ ನಾನೂ ಆ ಗಾಡಿಯೂ ಒಂದೇ ಪರಬ್ರಹ್ಮವಸ್ತುವೆಂದು ಭಾಸವಾಗುತ್ತಿತ್ತು. ಅಂಗಾಂಗಗಳಲ್ಲಿ ಸ್ವರ್ಶಜ್ಞಾನವೇ ಇಲ್ಲವಾಗಿ ಬಿಟ್ಟಿತ್ತು. ಊಟ ಮಾಡಿದರೆ ತೃಪ್ತಿಯ ಅನುಭವವೇ ಇಲ್ಲ! ಉಣ್ಣುತ್ತಿರುವವನು ನಾನಲ್ಲ, ಬೇರೆ ಯಾರೋ ಎಂಬ ಭಾವನೆ ಬರುತ್ತಿತ್ತು. ಒಮ್ಮೊಮ್ಮೆ ಊಟದ ಮಧ್ಯೆ ಮಲಗಿಬಿಡುತ್ತಿದ್ದೆ. ಸ್ವಲ್ಪ ಹೊತ್ತಿಗೆ ಎದ್ದು ಪುನಃ ಉಣ್ಣುತ್ತಿದ್ದೆ. ಹೀಗೆ ಕೆಲವೊಮ್ಮೆ ಮೀತಿಮೀರಿ ತಿಂದುಬಿಡುತ್ತಿದ್ದೆ. ಆದರೂ ತೊಂದರೆಯೇನೂ ಆಗಲಿಲ್ಲ. ಆದರೆ ಈ ವಿಚಿತ್ರ ವರ್ತನೆಯನ್ನೆಲ್ಲ ಕಂಡು ತಾಯಿ ಗಾಬರಿಗೊಂಡು, ‘ನಿನಗೆ ಒಳಗೊಳಗೇ ಏನೋ ಕಾಯಿಲೆಯಾಗಿರಬೇಕು’ ಎಂದು ಉದ್ಗರಿಸು ತ್ತಿದ್ದಳು. ‘ಇವನಿನ್ನು ಹೆಚ್ಚು ದಿನ ಬದುಕಿರಲಾರ’ ಎಂದು ಭಾವಿಸಿ ತಲ್ಲಣಗೊಳ್ಳುತ್ತಿದ್ದಳು.

“ಕಡೆಗೊಮ್ಮೆ ಈ ಅನುಭವದ ಉತ್ಕಟತೆ ಇಳಿದಾಗ ಜಗತ್ತು ಒಂದು ಕನಸಿನಂತೆ ಕಾಣತೊಡ ಗಿತು. ಒಂದು ದಿನ ಕಾರ್ನ್ ವಾಲೀಸ್ ಚೌಕದ ಬದಿಯಲ್ಲಿ ನಡೆದುಕೊಂಡು ಹೋಗುವಾಗ ಅಲ್ಲಿ ಸುತ್ತಲೂ ಇದ್ದ ಕಬ್ಬಿಣದ ಕಂಬಿಗಳಿಗೆ ತಲೆಯನ್ನು ಚಚ್ಚಿಕೊಂಡು ನೋಡಿದೆ–ನಾನು ಕಾಣು ತ್ತಿರುವುದು ನಿಜವಾದ ಕಂಬಿಗಳೋ ಕನಸಿನ ಕಂಬಿಗಳೋ ಎಂದು! ಕೈಕಾಲುಗಳೆಲ್ಲ ಜೋಮು ಹಿಡಿದಂತಾಗಿದ್ದರಿಂದ ಪಾರ್ಶ್ವವಾಯು ಬಡಿಯುತ್ತದೇನೋ ಎಂದು ಭಯವಾಗುತ್ತಿತ್ತು. ಅಂತೂ ಎಷ್ಟೋ ದಿನ ಆ ಭಯಂಕರವಾದ ಭಾವಾವಸ್ಥೆಯಿಂದ ಬಿಡುಗಡೆ ಸಿಗಲೇ ಇಲ್ಲ. ಕಡೆಗೆ ಸಹಜಸ್ಥಿತಿಗೆ ಮರಳಿದ ಮೇಲೆ ನನಗೆ ತೋಚಿತು–‘ಇದು ಅದ್ವೈತಜ್ಞಾನದ ಲಕ್ಷಣವಾಗಿರ ಬೇಕು’ಎಂದು. ಹಾಗಾದರೆ, ಈ ಬಗ್ಗೆ ಶಾಸ್ತ್ರಗಳಲ್ಲಿ ಹೇಳಿರುವುದು ಸುಳ್ಳಲ್ಲವೆಂದಾಯಿತು. ಅಂದಿ ನಿಂದ ಅದ್ವೈತತತ್ತ್ವದ ವಿಷಯದಲ್ಲಿ ನಾನು ಮತ್ತೆ ಸಂದೇಹ ಪಡಲಿಲ್ಲ.”

ಶ್ರೀರಾಮಕೃಷ್ಣರು ಇನ್ನೂ ಯಾವಯಾವ ಪರಿಯಿಂದ ನರೇಂದ್ರನ ಸಂಶಯಗಳನ್ನೆಲ್ಲ ಹೋಗ ಲಾಡಿಸಿದರೋ, ಇನ್ನೆಷ್ಟೆಷ್ಟು ರೀತಿಯಲ್ಲಿ ಅವನಿಗೆ ಶಿಕ್ಷಣ ನೀಡಿದರೋ ಹೇಳುವುದು ಕಷ್ಟ. ಏಕೆಂದರೆ ಅವರಿಬ್ಬರ ಸಂವಾದ ಹೆಚ್ಚಾಗಿ ಮೌಖಿಕವಲ್ಲ; ಮಾನಸಿಕ. 

ಇದೇ ದಿನಗಳಲ್ಲಿ ನಡೆದ ಮತ್ತೊಂದು ಅಪೂರ್ವ ಪ್ರಸಂಗವನ್ನು ನರೇಂದ್ರ ಕೆಲವರ್ಷಗಳ ಬಳಿಕ ತನ್ನ ಸ್ನೇಹಿತ ಶರಚ್ಚಂದ್ರನ ಮುಂದೆ ಬಯಲುಮಾಡುತ್ತಾನೆ:

ಅದೊಂದು ದಿನ ಶರಚ್ಚಂದ್ರನೂ ಅವನ ಬಂಧು ಶಶಿಭೂಷಣನೂ ನರೇಂದ್ರನ ಮನೆಗೆ ಬಂದಿದ್ದಾರೆ. ಇಬ್ಬರೂ ಶ್ರೀರಾಮಕೃಷ್ಣರ ತರುಣಶಿಷ್ಯರು; ತಮ್ಮ ಗುರುವಿನ ಮಹಿಮೆಯನ್ನು ಅಲ್ಪಸ್ವಲ್ಪ ಅರಿತವರು. ಅಂದು ನರೇಂದ್ರ, ತನ್ನ ಉಜ್ವಲ ವಾಗ್ಧಾರೆಯಿಂದ ಅವರಿಬ್ಬರಲ್ಲೂ ಸ್ಫೂರ್ತಿಯ ಬುಗ್ಗೆಯನ್ನೇ ಚಿಮ್ಮಿಸಿದ. ಅವನ ಮಾತನ್ನು ಕೇಳುತ್ತಕೇಳುತ್ತ ಅವರು ರೊಮಾಂಚಿತ ರಾದರು. ಈ ಹಿಂದೆ ಅವರು ಆತನ ಯಾವ ದಿವ್ಯಶಕ್ತಿಯಿಂದ ಅವನೆಡೆಗೆ ಆಕರ್ಷಿತರಾಗಿದ್ದರೋ, ಆ ಶಕ್ತಿ ಅಂದು ಸಾವಿರಪಟ್ಟು ಗಾಢವಾಗಿದ್ದಂತೆ ತೋರಿತು. ಅಷ್ಟು ದಿನವೂ ಅವರು ಶ್ರೀರಾಮ ಕೃಷ್ಣರನ್ನು ಭಗವತ್ಸಾಕ್ಷಾತ್ಕಾರ ಪಡೆದುಕೊಂಡ ಸಂತನೋ ಅಥವಾ ಸಿದ್ಧಪುರುಷನೋ ಇರ ಬೇಕೆಂದು ಭಾವಿಸಿದ್ದರು. ಆದರೆ ಅಂದು ಶ್ರೀರಾಮಕೃಷ್ಣರ ಸಂಬಂಧವಾದ ನರೇಂದ್ರನ ಮನ ಮುಟ್ಟುವ ಮಾತುಗಳು ಅವರ ಮನಸ್ಸಿನಲ್ಲಿ ಹೊಸಬೆಳಕನ್ನು ಹೊತ್ತಿಸಿದುವು. ಮಹಾ ಮಹಿಮ ರಾದ ಏಸುಕ್ರಿಸ್ತ, ಚೈತನ್ಯ ಮೊದಲಾದವರ ಜೀವನಚರಿತ್ರೆಗಳನ್ನು ಅವರು ಓದಿದ್ದರು. ಅವುಗಳಲ್ಲಿ ಬರೆದಿದ್ದ ಅನೇಕ ಅಲೌಕಿಕ ಘಟನೆಗಳನ್ನು ಓದಿ, ಅವೆಲ್ಲ ನಂಬಲನರ್ಹವೆಂದು ತೀರ್ಮಾನಿಸಿ ದ್ದರು. ಆದರೆ ಆ ದಿನ ನರೇಂದ್ರನ ಸ್ಫೂರ್ತಿಯುತ ಮಾತುಗಳನ್ನು ಆಲಿಸಿದಾಗ ಅಂತಹ ಘಟನೆ ಗಳು ಶ್ರೀರಾಮಕೃಷ್ಣರ ಜೀವನದಲ್ಲಿ ಅನುದಿನವೂ ಸಂಭವಿಸುತ್ತಿವೆ ಎಂಬ ವಿಶ್ವಾಸ ಅವರ ಲ್ಲುಂಟಾಯಿತು. ಶ್ರೀರಾಮಕೃಷ್ಣರು ತಮ್ಮ ಸ್ಪರ್ಶಮಾತ್ರದಿಂದ ಅಥವಾ ಇಚ್ಛಾಮಾತ್ರದಿಂದ, ಶರಣಾಗತರಾದ ಭಕ್ತರ ಸಂಸ್ಕಾರಬಂಧನಗಳನ್ನು ಬಿಡಿಸಿ ಅವರಿಗೆ ಭಕ್ತಿಜ್ಞಾನಗಳನ್ನು ನೀಡು ತ್ತಿದ್ದಾರೆ, ಅಥವಾ ಅವರ ಜೀವನಗತಿಯನ್ನು ಆಧ್ಯಾತ್ಮಿಕ ಮಾರ್ಗಕ್ಕೆ ತಿರುಗಿಸುತ್ತಿದ್ದಾರೆ ಎಂದು ಅರಿತರು. 

ದಾರಿಯಲ್ಲಿ ಮಾತನಾಡುತ್ತ ಸ್ನೇಹಿತರು ಮೂವರೂ ಒಂದು ಕೆರೆಯ ಬಳಿಗೆ ಬಂದರು. ಶ್ರೀರಾಮಕೃಷ್ಣರ ಕೃಪೆಯಿಂದ ತನಗಾದ ದಿವ್ಯಾನುಭವಗಳನ್ನು ಹೇಳಿಕೊಳ್ಳುತ್ತ ನರೇಂದ್ರ ಭಾವ ಭರಿತನಾದ. ತನ್ನೊಳಗೆ ತಾನು ಮಗ್ನನಾಗಿ ತನ್ನ ಆಂತರ್ಯದ ಆನಂದವನ್ನು ಹೊರಸೂಸಲು, ಹಾಡೊಂದನ್ನು ಹಾಡತೊಡಗಿದ:

\begin{verse}
ಶ್ರೀಗೌರಾಂಗನು ಪ್ರೇಮಧನವನು ಸಕಲರಲ್ಲು ವಿತರಿಸುತಿಹನು;\\ನಿತ್ಯಾನಂದನು ‘ಬನ್ನಿರಿ’ ಎನ್ನುತ ಎಲ್ಲರನ್ನು ಕರೆಯುತಿಹನು!
\end{verse}

ಹಾಡು ಮುಗಿದ ಮೇಲೆ ನರೇಂದ್ರ ಸ್ವಗತವೆಂಬಂತೆ ಮೆಲ್ಲಮೆಲ್ಲನೆ ಹೇಳುತ್ತಾನೆ: “...ಹೌದು, ನಿಜವಾಗಿ ಹಂಚುತ್ತಿದ್ದಾನೆ. ಪ್ರೇಮವಾಗಲಿ ಭಕ್ತಿಯಾಗಲಿ ಜ್ಞಾನವಾಗಲಿ ಮುಕ್ತಿಯಾಗಲಿ– ಗೌರಾಂಗನು (ಶ್ರೀರಾಮಕೃಷ್ಣರು) ಬೇಕುಬೇಕೆಂದುದನ್ನು ಹಂಚುತ್ತಿದ್ದಾನೆ. ಆಹ್, ಎಂಥ ಶಕ್ತಿ! (ಸ್ವಲ್ಪ ಹೊತ್ತು ಮೌನವಾಗಿದ್ದು, ಬಳಿಕ) ಅಂದು ರಾತ್ರಿ, ಕೋಣೆಯ ಬಾಗಿಲಿಗೆ ಅಗಳಿ ಹಾಕಿಕೊಂಡು ಮಲಗಿದ್ದೆ. ಇದ್ದಕ್ಕಿದ್ದಂತೆ ಅವರು ನನ್ನ ಶರೀರದೊಳಗೆ ಇರುವವನನ್ನು ಸೆಳೆದು ದಕ್ಷಿಣೇಶ್ವರಕ್ಕೆ ಕರೆದೊಯ್ದರು. ಎಷ್ಟೋ ಬಗೆಯಾಗಿ ಉಪದೇಶ ಕೊಟ್ಟು ಮರಳಿ ಹೋಗಲು ಅನುಮತಿಯಿತ್ತರು! ದಕ್ಷಿಣೇಶ್ವರದ ಗೌರಾಂಗರು ಏನು ಬೇಕಾದರೂ ಮಾಡಬಲ್ಲರು!”

ಇದನ್ನೆಲ್ಲ ಕೇಳುತ್ತ ಗೆಳೆಯರಿಬ್ಬರೂ ವಿಸ್ಮಯಮೂಕರಾಗಿಬಿಟ್ಟರು.

ಈ ರೀತಿಯಾಗಿ ಶ್ರೀರಾಮಕೃಷ್ಣರು ನರೇಂದ್ರನನ್ನು ಮೆಲ್ಲಗೆ ಹೆಜ್ಜೆಹೆಜ್ಜೆಯಾಗಿ, ಮೆಟ್ಟಿಲು ಮೆಟ್ಟಿಲಾಗಿ, ಸಂಶಯದಿಂದ ಶ್ರದ್ಧೆಯ ಕಡೆಗೆ, ಕತ್ತಲಿನಿಂದ ಬೆಳಕಿನ ಕಡೆಗೆ, ಈ ಪ್ರಪಂಚದ ಸಂಕುಚಿತತೆಯಿಂದ ವಿಶ್ವಾತ್ಮಭಾವದ ವೈಶಾಲ್ಯದ ಕಡೆಗೆ ನಡೆಸಿಕೊಂಡು ಹೋದರು. ಅವರ ಅಲೌಕಿಕ ಶಕ್ತಿ ಆತನನ್ನು ಕ್ರಮಕ್ರಮವಾಗಿ ಲೌಕಿಕ ಬಂಧನದಿಂದ ಆಧ್ಯಾತ್ಮಿಕ ಸ್ವಾತಂತ್ರ್ಯದ ಕಡೆಗೆ, ಅಲ್ಪಬುದ್ಧಿಯ ಅರೆಜ್ಞಾನದಿಂದ ಸಮಗ್ರವಾದ ಸರ್ವಜ್ಞತ್ವದೆಡೆಗೆ, ಕೊನೆಗೆ ಸಚ್ಚಿದಾನಂದ ಸ್ವರೂಪದ ಬ್ರಹ್ಮಜ್ಞಾನದವರೆಗೆ ಮುನ್ನಡೆಸಿಕೊಂಡು ಹೋಯಿತು. ಶ್ರೀರಾಮಕೃಷ್ಣರ ಬೋಧನೆ ಗಳ ಸತ್ಯತೆಯನ್ನು ಮನಗಂಡು, ಹೊಸ ಅನುಭವಗಳನ್ನು ಗಳಿಸುತ್ತ ತನ್ನ ಸಂದೇಹಗಳನ್ನು ಪರಿಹರಿಸಿಕೊಳ್ಳುತ್ತ ಬಂದಂತೆ, ಅವರ ಮೇಲೆ ನರೇಂದ್ರನ ಪ್ರೀತಿ-ವಿಶ್ವಾಸ-ಗೌರವಗಳು ಸಾವಿರ ಮಡಿಯಾಗಿ ಹೆಚ್ಚಲಾರಂಭಿಸಿದುವು. ಕ್ರಮೇಣ ಅವನು ಅವರನ್ನು ಆಧ್ಯಾತ್ಮಿಕ ಸಾಮ್ರಾಜ್ಯದ ಚಕ್ರವರ್ತಿ ಎಂಬಂತೆ ಸ್ವೀಕರಿಸಲಾರಂಭಿಸಿದ.

ಆದರೆ ತಮ್ಮ ವ್ಯಕ್ತಿತ್ವವನ್ನಾಗಲಿ, ಬೋಧನೆಯನ್ನಾಗಲಿ ಶಿಷ್ಯರು ಅಷ್ಟು ಸುಲಭವಾಗಿ ಸ್ವೀಕರಿಸಬೇಕೆಂದು ನಿರೀಕ್ಷಿಸುವವರಲ್ಲ ಶ್ರೀರಾಮಕೃಷ್ಣರು. ಶಿಷ್ಯರಾಗುವವರು ತಮ್ಮನ್ನು ಚೆನ್ನಾಗಿ ಪರೀಕ್ಷೆ ಮಾಡಿನೋಡಿ ತೃಪ್ತಿಯಾದರೆ ಮಾತ್ರ ಸ್ವೀಕರಿಸಲಿ ಎನ್ನುವುದು ಅವರ ಧೀರ ಮನೋ ಭಾವ. ಯಾವ ಗುರುವು ಶರೀರ-ಮಾತು-ಮನಸ್ಸುಗಳಿಂದ ಪರಿಶುದ್ಧನಾಗಿರುತ್ತಾನೆಯೋ ಅಂಥ ವನು ಮಾತ್ರ ತನ್ನ ಶಿಷ್ಯರಿಂದ ಪರೀಕ್ಷೆ ಮಾಡಿಸಿಕೊಳ್ಳಲು ಒಪ್ಪುತ್ತಾನೆ. ಶ್ರೀರಾಮಕೃಷ್ಣರಾದರೋ ಒಳಗೂ ಹೊರಗೂ ಚೊಕ್ಕ ಬಂಗಾರ. ಅಲ್ಲದೆ, ಶಿಷ್ಯರು ತಾವೇ ಸ್ವತಃ ಪರೀಕ್ಷೆ ಮಾಡಿನೋಡಿ ತಮ್ಮ ಗುರುವಿನ ಪುಜುತ್ವವನ್ನು ಕಣ್ಣಾರೆ ಕಂಡಾಗ, ಆತನ ಬೋಧನೆಗಳು ಅವರ ಮನಸ್ಸಿಗೆ ಒಪ್ಪಿಗೆಯಾಗುವುದಲ್ಲದೆ ನೆಲೆನಿಲ್ಲುವಂತಾಗುತ್ತದೆ. ಆಗ ಗುರುವಿನಲ್ಲಿ, ಗುರುವಿನ ಬೋಧನೆ ಯಲ್ಲಿ ಶ್ರದ್ಧೆ ದೃಢವಾಗುತ್ತದೆ. ಆದ್ದರಿಂದ ಶ್ರೀರಾಮಕೃಷ್ಣರು ನರೇಂದ್ರನಿಗೆ ಹೇಳುತ್ತಾರೆ: “ನೋಡಯ್ಯ, ಸರಾಫು ವ್ಯಾಪಾರಿಗಳು ನಾಣ್ಯವನ್ನು ‘ಠಣ್’ ಅಂತ ಬಡಿದು, ಅದು ಖೋಟಾ ನಾಣ್ಯವೋ ತಾಜಾ ನಾಣ್ಯವೋ ಎನ್ನುವುದನ್ನು ಪರೀಕ್ಷೆ ಮಾಡುವುದಿಲ್ಲವೆ? ಹಾಗೆಯೇ ನೀನೂ ನನ್ನನ್ನು ಪರೀಕ್ಷೆ ಮಾಡಿನೋಡು. ಹಾಗಲ್ಲದೆ ಒಪ್ಪಿಕೊಳ್ಳಲೇ ಬೇಡ.” ನಿಜಕ್ಕೂ ಇದು ಶ್ರೀರಾಮ ಕೃಷ್ಣರ ಧೀರತನವೇ ಸರಿ. ಅದಕ್ಕೆ ಸರಿಯಾಗಿ ಶಿಷ್ಯ ನರೇಂದ್ರನ ಧೀರತನವೂ ಕಡಮೆಯೇನಲ್ಲ. ಅವನು ಹೆಜ್ಜೆಹೆಜ್ಜೆಗೂ ಅವರನ್ನು ಪರೀಕ್ಷೆಮಾಡಲು ಸಿದ್ಧನಾಗಿಯೇ ಇದ್ದ.

ಒಂದು ದಿನ ಅವನು ದಕ್ಷಿಣೇಶ್ವರಕ್ಕೆ ಬಂದಾಗ, ಶ್ರೀರಾಮಕೃಷ್ಣರು ಕಲ್ಕತ್ತಕ್ಕೆ ಹೋಗಿರುವ ವಿಚಾರ ತಿಳಿಯಿತು. ಆಗ ಅವನ ಮನಸ್ಸಿಗೆ ಹಠಾತ್ತನೆ ಒಂದು ಭಾವನೆ ಬಂದಿತು: ‘ಶ್ರೀರಾಮ ಕೃಷ್ಣರು ಹೇಳುತ್ತಾರೆ–ಕಾಮಕಾಂಚನವನ್ನು ತ್ರಿಕರಣಪೂರ್ವಕವಾಗಿ ತ್ಯಾಗ ಮಾಡಬೇಕು ಅಂತ; ಮತ್ತು ತಾವು ಹಣವನ್ನು ಸ್ಪರ್ಶ ಕೂಡ ಮಾಡಲಾರೆವು ಅಂತಲೂ ಹೇಳುತ್ತಾರೆ. ಆ ಮಾತಿನ ಸತ್ಯತೆಯನ್ನು ಪರೀಕ್ಷಿಸಿಯೇಬಿಡಬೇಕು.’ ಹೀಗೆ ಆಲೋಚಿಸಿ ಅವನು ಮೆಲ್ಲಗೆ ಅವರ ಹಾಸಿಗೆಯ ದುಪ್ಪಟದ ಕೆಳಗಡೆ ಒಂದು ರೂಪಾಯಿಯ ನಾಣ್ಯವನ್ನು ಅಡಗಿಸಿಟ್ಟು, ಅಲ್ಲಿಂದ ಮರೆಯಾದ. ಶ್ರೀರಾಮಕೃಷ್ಣರು ಸ್ವಲ್ಪ ಹೊತ್ತಿಗೆ ಹಿಂದಿರುಗಿದರು; ಎಂದಿನಂತೆ ತಮ್ಮ ಹಾಸಿಗೆಯ ಮೇಲೆ ಕುಳಿತರು. ತಕ್ಷಣವೇ ಏನೋ ಕುಟುಕಿದವರಂತೆ ನೋವಿನಿಂದ ಛಂಗನೆ ಹಾರಿದರು! ಏನಾಯಿ ತೆಂದು ಗೊತ್ತಾಗದೆ ಆಶ್ಚರ್ಯಪಡುತ್ತ ನಿಂತರು. ಅಲ್ಲಿದ್ದವನೊಬ್ಬ ಹಾಸಿಗೆಯನ್ನು ಕೊಡವಿ ನೋಡಿದ. ಆಗ ಹೊದಿಕೆಯ ಕೆಳಗೆ ಅಡಗಿಸಿಟ್ಟಿದ್ದ ರೂಪಾಯಿಯ ನಾಣ್ಯ ಕೆಳಗೆ ಬಿತ್ತು. ಈ ವೇಳೆಗೆ ಅಲ್ಲಿಗೆ ಬಂದು ನಿಂತಿದ್ದ ನರೇಂದ್ರ ಇದನ್ನೆಲ್ಲ ನೋಡಿದ. ಅವನಿಗೆ ನಾಚಿಕೆಯಾಯಿತು. ಒಂದು ಮಾತನ್ನೂ ಆಡದೆ ಕೋಣೆಯನ್ನು ಬಿಟ್ಟು ಹೊರಟುಹೋದ. ಅವನಿಟ್ಟಿದ್ದ ಪರೀಕ್ಷೆಯಲ್ಲಿ ಶ್ರಿರಾಮಕೃಷ್ಣರು ಯಶಸ್ವಿಯಾಗಿದ್ದರು. ಇದು ಇವನೇ ತಮ್ಮನ್ನು ಪರೀಕ್ಷೆಮಾಡಲು ನಡೆಸಿದ ತಂತ್ರ ಎಂದು ಗೊತ್ತಾದಾಗ ಶ್ರೀರಾಮಕೃಷ್ಣರು ಬಹಳವಾಗಿ ಸಂತೋಷಪಟ್ಟರು.

ನರೇಂದ್ರ ತನ್ನ ಗುರುವಿನ ಮೇಲೆ ನಡೆಸಿದ ಪರೀಕ್ಷೆಗಳಲ್ಲಿ ಇದು ಒಂದು ಮಾತ್ರ. ಅವರು ತನ್ನನ್ನು ಗಮನಿಸುತ್ತಿದ್ದಂತೆಯೇ ತಾನೂ ಅವರನ್ನು ಎಡೆಬಿಡದೆ ಸೂಕ್ಷ್ಮವಾಗಿ ಗಮನಿಸುತ್ತಿದ್ದ. ಅವರ ಒಂದೊಂದು ಮಾತನ್ನೂ ತನ್ನ ಬುದ್ಧಿಯ ಒರೆಗಲ್ಲಿಗೆ ಉಜ್ಜಿ, ಸರಿಯೆಂದು ಕಂಡುಬಂದು ದನ್ನು ಮಾತ್ರವೇ ಸ್ವೀಕರಿಸುತ್ತಿದ್ದ.

ಹಾಗೆಯೇ ಶ್ರೀರಾಮಕೃಷ್ಣರು ತಮ್ಮ ಶಿಷ್ಯರನ್ನು ಪರೀಕ್ಷೆಗೆ ಗುರಿಪಡಿಸದೆ ಬಿಡುತ್ತಿರಲಿಲ್ಲ. ನರೇಂದ್ರನೂ ಇಂತಹ ಹಲವಾರು ಪರೀಕ್ಷೆಗಳಿಗೆ ಗುರಿಯಾಗಬೇಕಾಯಿತು. ಒಮ್ಮೆ ಶ್ರೀರಾಮ ಕೃಷ್ಣರು ಅವನ ದೇಹವನ್ನು ಶರೀರಸಾಮುದ್ರಿಕ ದೃಷ್ಟಿಯಿಂದ ಪರೀಕ್ಷಿಸಿ, “ನಿನ್ನ ಶರೀರಲಕ್ಷಣ ವೆಲ್ಲ ಚೆನ್ನಾಗಿದೆ. ಆದರೆ ಒಂದೇ ಒಂದು ದೋಷವೆಂದರೆ ನೀನು ನಿದ್ರೆ ಮಾಡುವಾಗ ಭಾರವಾಗಿ ಉಸಿರಾಡುತ್ತೀಯೆ. ಯೋಗಿಗಳ ಪ್ರಕಾರ, ಭಾರವಾಗಿ ಉಸಿರಾಡುವವರ ಆಯುಷ್ಯ ಕಡಮೆ” ಎಂದರು. ಇನ್ನೊಂದು ಸಂದರ್ಭದಲ್ಲಿ ಅವರೆನ್ನುತ್ತಾರೆ: “ನಿನ್ನ ಕಣ್ಣುಗಳು ನೀನೊಬ್ಬ ಒಣಜ್ಞಾನಿ ಯಲ್ಲ ಎಂಬುದನ್ನು ಎತ್ತಿ ತೋರಿಸುತ್ತವೆ. ನಿನ್ನಲ್ಲಿ ಆಳವಾದ ಜ್ಞಾನವೂ ಮಧುರವಾದ ಭಕ್ತಿಯೂ ಬಹಳ ಚೆನ್ನಾಗಿ ಸಮರಸದಿಂದ ಬೆರೆತುಕೊಂಡಿವೆ.” ಇಂತಹ ಅನೇಕ ಪರೀಕ್ಷೆಗಳ ಮೂಲಕ ಶ್ರೀರಾಮಕೃಷ್ಣರು ಆತನ ಸಂಯಮ ಧೈರ್ಯ ಸ್ಥೈರ್ಯ ತ್ಯಾಗಬುದ್ದಿ–ಇವೆಲ್ಲ ಅತ್ಯುನ್ನತ ಮಟ್ಟ ದವು ಎಂದು ತೀರ್ಮಾನಿಸಿದ್ದರು. ಆದ್ದರಿಂದ, ‘ಎಷ್ಟೇ ಹಸಿದರೂ ಸಿಂಹ ಹುಲ್ಲು ತಿಂದೀತೆ?’ ಎಂಬಂತೆ ಎಂಥಾ ಸಂಕಟದ ಪರಿಸ್ಥಿತಿಯೇ ಒದಗಿಬಂದರೂ ನರೇಂದ್ರ ಮಾತ್ರ ಸಾಮಾನ್ಯರಂತೆ ವರ್ತಿಸುವುದಿಲ್ಲ ಎಂದು ಕಂಡುಕೊಂಡಿದ್ದರು.

ಒಮ್ಮೆ ಮಾತ್ರ ನರೇಂದ್ರ ತನಗರಿವಿಲ್ಲದಂತೆಯೇ ದೀರ್ಘ ಕಾಲದ, ಅತಿ ಕಠಿಣ ಪರೀಕ್ಷೆ ಯೊಂದಕ್ಕೆ ಈಡಾದ:

ನರೇಂದ್ರ ದಕ್ಷಿಣೇಶ್ವರಕ್ಕೆ ಬಂದ ದಿನಗಳಲ್ಲೆಲ್ಲ ಶ್ರೀರಾಮಕೃಷ್ಣರಿಗೆ ಉತ್ಸವ. ಅವನನ್ನು ಕಂಡರೆ ಅವರಿಗೆ ಹೇಳಿತೀರದಷ್ಟು ಆನಂದ. ಆದರೊಂದು ದಿನ ಅದೇಕೋ ಇದ್ದಕ್ಕಿದ್ದಂತೆ ಎಲ್ಲವೂ ಸಂಪೂರ್ಣ ವಿರುದ್ಧವಾಯಿತು. ನರೇಂದ್ರ ಬಂದ; ಎಂದಿನಂತೆ ನಮಸ್ಕರಿಸಿ ಕುಳಿತ. ಶ್ರೀರಾಮಕೃಷ್ಣರು ಮಿಸುಕಲಿಲ್ಲ, ಮುಗುಳ್ನಗೆ ಬೀರಲಿಲ್ಲ, ಅವನತ್ತ ತಿರುಗಿಯೂ ನೋಡಲಿಲ್ಲ. ಅಸಡ್ಡೆಯ ಮುಖಭಾವ, ವಿಲಕ್ಷಣ ಮೌನ! ನರೇಂದ್ರ ಬಹಳ ಹೊತ್ತು ಸುಮ್ಮನೆ ಕುಳಿತಿದ್ದ. ಕಡೆಗೆ ಬಹುಶಃ ಗುರುಗಳು ಭಾವಸ್ಥಿತಿಯಲ್ಲಿರಬೇಕೆಂದು ಭಾವಿಸಿ ಎದ್ದುಹೋದ. ಅಲ್ಲೇ ಹೊರಗೆ ಹಾಜರಾನಂತೂ ಇದ್ದೇ ಇದ್ದಾನಲ್ಲ. ಅವನ ಜೊತೆಯಲ್ಲಿ ಸ್ವಲ್ಪ ಹರಟೆಕೊಚ್ಚಿ, ಗುಡುಗುಡಿ ಸೇದಿದ. ಅಷ್ಟೊತ್ತಿಗೆ ಶ್ರೀರಾಮಕೃಷ್ಣರು ಯಾರೊಂದಿಗೋ ಮಾತನಾಡಿದ್ದು ಕೇಳಿಸಿತು. ಮತ್ತೆ ಒಳಗೆ ಬಂದ. ಆದರೆ ಇವನನ್ನು ನೋಡುತ್ತಿದ್ದಂತೆಯೇ ಅವರು ಮತ್ತೆ ಸುಮ್ಮನಾಗಿಬಿಟ್ಟರು. ಅಷ್ಟೇ ಅಲ್ಲ, ಗೋಡೆಗೆ ಮುಖ ಮಾಡಿ ಮಲಗಿಬಿಟ್ಟರು! ದಿನವೆಲ್ಲ ಹೀಗೇ ಕಳೆಯಿತು. ಆದರೆ ಇದನ್ನು ನರೇಂದ್ರನೇನೂ ತಪ್ಪಾಗಿ ತಿಳಿಯಲಿಲ್ಲ. ಸಂಜೆಯ ವೇಳೆಗೆ ಶ್ರೀರಾಮಕೃಷ್ಣರಿಗೆ ಪ್ರಣಾಮ ಸಲ್ಲಿಸಿ ನಿರ್ಗಮಿಸಿದ.

ಒಂದು ವಾರದೊಳಗಾಗಿ ಅವನು ಮತ್ತೊಮ್ಮೆ ಬಂದ. ಆದರೆ ಶ್ರೀರಾಮಕೃಷ್ಣರದು ಅದೇ ಮೌನ, ಅದೇ ಔದಾಸೀನ್ಯ! ಅಂದೂ ಅವನು ಹಾಜರಾ ಮತ್ತಿತರ ಭಕ್ತರೊಂದಿಗೆ ಹರಟಿ ಕೊಂಡಿದ್ದು ಹಿಂದಿರುಗಿದ. ಹೀಗೆಯೇ ಮೂರನೆಯ ಸಲವೂ ಬಂದ, ನಾಲ್ಕನೆಯ ಸಲವೂ ಬಂದ. ಶ್ರೀರಾಮಕೃಷ್ಣರ ವರ್ತನೆ ಮಾತ್ರ ಬದಲಾಗಲೇ ಇಲ್ಲ. ಅವನು ಎದುರಿಗಿದ್ದರೂ ಒಂದೇ ಇಲ್ಲ ದಿದ್ದರೂ ಒಂದೇ ಎಂಬಂಥ ಅಲಕ್ಷ್ಯ! ಇಲ್ಲವೆ, ಯಾವನೋ ಅಪರಿಚಿತನೆಂಬಂತೆ ನೋಡಿ ಅತ್ತ ತಿರುಗಿಬಿಡುತ್ತಾರೆ. ಆದರೂ ನರೇಂದ್ರ ಸ್ವಲ್ಪವೂ ಬೇಸರಿಸಿಕೊಳ್ಳಲಿಲ್ಲ. ಅಥವಾ ತನ್ನಿಂದ ಏನು ತಪ್ಪಾಯಿತೋ ಎಂದು ಕೊರಗಲೂ ಇಲ್ಲ. ಎಂದಿನ ವಿಶ್ವಾಸದಿಂದ ತನ್ನಷ್ಟಕ್ಕೆ ತಾನು ವಾರಕ್ಕೊಮ್ಮೆ ಬಂದು ಹೋಗುತ್ತಿದ್ದ. ಈ ನಡುವಿನ ಅವಧಿಯಲ್ಲಿ ಶ್ರೀರಾಮಕೃಷ್ಣರು ಯಾರ ನ್ನಾದರೂ ಕಲ್ಕತ್ತಕ್ಕೆ ಕಳಿಸಿ, ನರೇಂದ್ರನ ಅರಿವಿಗೆ ಬಾರದಂತೆ ಅವನ ಕ್ಷೇಮಸಮಾಚಾರವನ್ನು ತಿಳಿದುಕೊಳ್ಳುತ್ತಿದ್ದರು. ಹೀಗೇ ಒಂದು ತಿಂಗಳ ಮೇಲಾಯಿತು. ಒಂದು ದಿನ ಅವನು ಬಂದಾಗ ಶ್ರೀರಾಮಕೃಷ್ಣರು ಕೇಳಿದರು: “ನಾನು ನಿನ್ನ ಜೊತೆ ಮಾತುಕತೆಯೇನೂ ಆಡುವುದಿಲ್ಲ. ಆದರೂ ನೀನು ಮಾತ್ರ ಬರುತ್ತಲೇ ಇದ್ದೀಯಲ್ಲ, ಅದು ಹೇಗೆ?”

ನರೇಂದ್ರ ತನ್ನ ಎಂದಿನ ದಿಟ್ಟತನದಿಂದ ಸಹಜವಾಗಿ ಉತ್ತರಿಸಿದ: “ನೀವು ನನ್ನನ್ನು ಮಾತನಾಡಿಸಲಿ ಅಂತ ನಾನಿಲ್ಲಿಗೆ ಬರುತ್ತೇನೆ ಎಂದುಕೊಂಡಿರಾ? ನಾನು ನಿಮ್ಮನ್ನು ಪ್ರೀತಿಸು ತ್ತೇನೆ. ನಿಮ್ಮನ್ನು ನೋಡಲು ನನಗಿಷ್ಟ; ಅದಕ್ಕಾಗಿ ಬರುತ್ತೇನೆ.”

ಇದನ್ನು ಕೇಳಿದಾಗ ಶ್ರೀರಾಮಕೃಷ್ಣರಿಗಾದ ಸಂತಸ ಅಷ್ಟಿಷ್ಟಲ್ಲ. “ಆಹ್, ನರೇನ್! ಇಷ್ಟು ದಿನವೂ ನಾನು ನಿನ್ನನ್ನು ಸುಮ್ಮನೆ ಪರೀಕ್ಷಿಸುತ್ತಿದ್ದೆ ಅಷ್ಟೆ. ನಿನ್ನ ಮೇಲೆ ಪ್ರೀತಿ-ವಿಶ್ವಾಸ ತೋರದಿದ್ದರೆ ನೀನು ಹೇಗೆ ಪ್ರತಿಕ್ರಿಯಿಸುತ್ತೀಯೋ ನೋಡೋಣ ಎಂದೇ ಹಾಗೆ ಮಾಡಿದೆ. ಭಲೇ! ಈ ರೀತಿ ಅಸಡ್ಡೆ ತೋರಿದಾಗ ನಿನ್ನಂತಹ ಉನ್ನತ ಮಟ್ಟದ ಸಾಧಕರು ಮಾತ್ರ ತಡೆದು ಕೊಂಡಾರು. ಬೇರೆ ಯಾರೇ ಆಗಿದ್ದರೂ ಯಾವತ್ತೋ ನನ್ನಿಂದ ದೂರವಾಗಿಬಿಡುತ್ತಿದ್ದರು. ಮತ್ತೆ ಈ ಕಡೆ ತಿರುಗಿಯೂ ನೋಡುತ್ತಿರಲಿಲ್ಲ” ಎಂದು ಕೊಂಡಾಡಿದರು.

ನಿಜಕ್ಕೂ ಇದೊಂದು ಭಯಂಕರ ಅಗ್ನಿಪರೀಕ್ಷೆಯೇ ಸರಿ. ದಿವ್ಯ ಗುರು ನಡೆಸಿದ ಗುರುತರ ಪರೀಕ್ಷೆ. ಮೊಟ್ಟಮೊದಲ ಭೇಟಿಯಿಂದಲೂ ಅವರಿರ್ವರ ನಡುವೆ ಬೆಳೆದು ಬಂದಿದ್ದ ಬಾಂಧವ್ಯ ಎಷ್ಟು ಗಾಢವಾದದ್ದು, ಸುಮಧುರವಾದದ್ದು, ಅಲೌಕಿಕವಾದದ್ದು! ಶ್ರೀರಾಮಕೃಷ್ಣರು ನರೇಂದ್ರನ ಬಗ್ಗೆ ತೋರಿದ್ದ ಪ್ರೀತಿಯೇನು, ವಿಶ್ವಾಸವೇನು! ಅವನಿಗಾಗಿ ಸಿಹಿತಿಂಡಿಗಳನ್ನು ತೆಗೆದಿಟ್ಟಿದ್ದು ಅವನು ಬಂದ ಕೂಡಲೇ ಉಪಚರಿಸಿ, ತಮ್ಮ ಕೈಯಿಂದಲೇ ಅವನಿಗೆ ತಿನ್ನಿಸುವುದೇನು, ಮನೆಗೂ ಕಳಿಸಿಕೊಡುವುದೇನು! ಭಕ್ತರ ಮುಂದೆ ಅವನನ್ನು ಕೊಂಡಾಡುವುದೇನು! ಅವನನ್ನು ಹುಡುಕಿ ಕೊಂಡು ತಾವೇ ಕಲ್ಕತ್ತಕ್ಕೆ ಬರುವುದೇನು! ಈಗ ಇದ್ದಕ್ಕಿದ್ದಂತೆ, ಅದೂ ಅಕಾರಣವಾಗಿ ಈ ನಿರ್ಲಕ್ಷ್ಯವೇನು, ಔದಾಸೀನ್ಯವೇನು, ಅಸಹನೀಯ ಮೌನವೇನು! ಆದರೆ ನರೇಂದ್ರ ಮಾತ್ರ ಇದನ್ನೆಲ್ಲ ಸಹಿಸಿಕೊಂಡು ಶ್ರೀರಾಮಕೃಷ್ಣರ ಮೇಲೆ ಅದೇ ಭಕ್ತಿ-ಪ್ರೀತಿ-ಗೌರವವನ್ನಿಟ್ಟು ಕೊಂಡದ್ದು ದೊಡ್ಡ ಅದ್ಭುತವಲ್ಲವೆ? ಒಂದು ವೇಳೆ ಅವನು ಅವರ ಔದಾಸೀನ್ಯವನ್ನು ಕಂಡು ಅವರನ್ನು ನಿರಾಕರಿಸಿ ಹೊರಟುಹೋಗಿಬಿಟ್ಟಿದ್ದರೆ, ಅವನಿಗೆ ನಿಜಕ್ಕೂ ಬಹಳ ನಷ್ಟವಾಗುತ್ತಿತ್ತು. ಶ್ರೀರಾಮಕೃಷ್ಣರಂತಹ ಯುಗಪುರುಷರನ್ನು, ಪರಮಗುರುವನ್ನು ಸಂಧಿಸಲು ಅವನು ತೋರಿದ್ದ ಕಾತರತೆ ಅದೆಷ್ಟು! ಈ ರೀತಿ ತಾನು ಕಂಡುಕೊಂಡ ಅನರ್ಘ್ಯ ರತ್ನವನ್ನು ಬಿಟ್ಟುಬಿಟ್ಟಿದ್ದರೆ ನಿಜಕ್ಕೂ ಅವನಿಗುಂಟಾಗುತ್ತಿದ್ದುದು ಭರಿಸಲಾರದ ನಷ್ಟ.

ಭಗವಂತನು ನಮ್ಮನ್ನೇ ಆಗಲಿ, ನಾವು ಭಗವಂತನನ್ನೇ ಆಗಲಿ ನಿರಾಕರಿಸಿದರೆ ಅದರಿಂದ ನಷ್ಟ ಭಗವಂತನಿಗಲ್ಲ, ನಮಗೆ! ಗುರುಗಳು ನಮ್ಮನ್ನಾಗಲಿ, ನಾವು ಗುರುಗಳನ್ನಾಗಲಿ ನಿರಾಕರಿಸಿದಾಗ ನಷ್ಟವಾಗುವುದು ನಮಗೇ ಹೊರತು ಗುರುಗಳಿಗಲ್ಲ. ಈ ಬಗೆಯ ನಿರಾಕರಣೆ ಯಿಂದ ಬಂದೊದಗುವ ದುಃಖ ಮಾತ್ರ ನಿಜಕ್ಕೂ ಅಸಹನೀಯವಾಗುತ್ತದೆ. ಆದ್ದರಿಂದಲೇ ಉಪನಿಷತ್ತಿನಲ್ಲಿ ಶಿಷ್ಯ ಪ್ರಾರ್ಥಿಸಿಕೊಳ್ಳುತ್ತಾನೆ:

\begin{myquote}
ಮಾsಹಂ ಬ್ರಹ್ಮ ನಿರಾಕುರ್ಯಾಮ್ ಮಾ ಮಾ ಬ್ರಹ್ಮ ನಿರಾಕರೋತ್ ।\\ಅನಿರಾಕರಣಮಸ್ತು ಅನಿರಾಕರಣಂ ಮೇ ಅಸ್ತು \\ತದಾತ್ಮನಿ ನಿರತೇ ಯ ಉಪನಿಷತ್ಸು ಧರ್ಮಾಸ್ತೇ ಮಯಿ ಸಂತು \\ತೇ ಮಯಿ ಸಂತು ॥
\end{myquote}

\noindent

ಎಂದರೆ, “ಬ್ರಹ್ಮವನ್ನು (ಪರಮಾತ್ಮನನ್ನು) ನಾನು ಎಂದಿಗೂ ನಿರಾಕರಿಸದಂತಾಗಲಿ, ಬ್ರಹ್ಮವೂ ನನ್ನನ್ನು ನಿರಾಕರಿಸದಿರಲಿ. ನಿರಾಕರಣೆಯೇ ಇಲ್ಲದಿರಲಿ. ಕೊನೆಗೆ, ನನ್ನಿಂದಲಾದರೂ ನಿರಾಕರಣೆ ಇಲ್ಲದಿರಲಿ. ಉಪನಿಷತ್ತಿನ ಧರ್ಮಗಳೆಲ್ಲವೂ ಆತ್ಮನಿರತನಾದ ನನ್ನಲ್ಲಿ ಕಾಣಿಸಿಕೊಳ್ಳಲಿ” ಎಂದು. ಇಲ್ಲಿ ಶಿಷ್ಯ ಭಗವಂತನನ್ನು ಪ್ರಾರ್ಥಿಸಿಕೊಂಡದ್ದು ತುಂಬ ಮನನೀಯ. ಹಾಗೆಯೇ ಇಲ್ಲಿ ಶ್ರೀರಾಮಕೃಷ್ಣರು ತನ್ನನ್ನು ವಾರಗಟ್ಟಲೆ ಸಂಪೂರ್ಣವಾಗಿ ನಿರಾಕರಿಸಿದರೂ ನರೇಂದ್ರ ಮಾತ್ರ ಅವರನ್ನು ನಿರಾಕರಿಸದಿದ್ದುದು ಆ ಉಪನಿಷತ್ತಿನ ಪ್ರಾರ್ಥನೆಗೆ ಅನುಗುಣವಾಗಿದೆ.

ಆದರೆ ಇಲ್ಲೊಂದು ಪ್ರಶ್ನೆಯೇಳಬಹುದು–ಶ್ರೀರಾಮಕೃಷ್ಣರು ನರೇಂದ್ರನನ್ನು ನಿರಾಕರಿ ಸಿದ್ದು, ಅವನ ಕಡೆ ಔದಾಸೀನ್ಯ ತೋರಿದ್ದು ಇದೆಲ್ಲ ಕೇವಲ ತೋರಾಣಿಕೆಯಲ್ಲವೆ? ನಿಜವೇ. ಆದರೆ ಇದೆಲ್ಲ ಕೇವಲ ತೋರಾಣಿಕೆಯೆಂದು ಅವನಿಗೆ ಗೊತ್ತಾದುದು ಯಾವಾಗ? ನಾಟಕ ಮುಗಿದ ಮೇಲೆಯೇ! ಶ್ರೀರಾಮಕೃಷ್ಣರೇ ಅದನ್ನು ಹೇಳಿದ ಮೇಲೆ! ಆದ್ದರಿಂದ ನಿಜಕ್ಕೂ ಅದೊಂದು ಕಠಿಣ ಪರೀಕ್ಷೆ. ಅದಕ್ಕೇ ಶ್ರೀರಾಮಕೃಷ್ಣರು ಅವನಿಗೆ ಯೋಗ್ಯತಾಪತ್ರ ಕೊಡುತ್ತಾರೆ –‘ಈ ಬಗೆಯ ಔದಾಸೀನ್ಯವನ್ನು ನಿನ್ನಂಥ ಉನ್ನತ ಮಟ್ಟದ ಸಾಧಕರು ಮಾತ್ರ ತಡೆದುಕೊಳ್ಳ ಬಲ್ಲರು’ ಎಂದು.

ಹೀಗೆ ಶ್ರೀರಾಮಕೃಷ್ಣರು ನರೇಂದ್ರನನ್ನು ಪರಿಪರಿಯಾಗಿ ಪರೀಕ್ಷಿಸಿದರು. ಇನ್ನೊಂದು ದಿನ ಅವರು ಅವನನ್ನು ಪಂಚವಟಿಯ ಏಕಾಂತಕ್ಕೆ ಕರೆತಂದು ಹೇಳಿದರು:

“ನೋಡು, ಆಧ್ಯಾತ್ಮಿಕ ಸಾಧನೆಗಳ ಫಲವಾಗಿ ನನಗೆ ಅಣಿಮಾ ಮುಂತಾದ ಅಷ್ಟಸಿದ್ಧಿಗಳು\footnote{*ನೋಡಿ: ಅನುಬಂಧ ೧.} ಲಭ್ಯವಾಗಿವೆ. ಆದರೆ, ನೀನೇ ನೋಡುತ್ತಿದ್ದೀಯಲ್ಲ, ಮೈಮೇಲಿನ ಬಟ್ಟೆಯನ್ನೂ ಸರಿಯಾಗಿ ಇಟ್ಟುಕೊಳ್ಳಲಾರದವನು ನಾನು. ನನ್ನಂಥವನಿಗೆ ಆ ಸಿದ್ಧಿಗಳನ್ನೆಲ್ಲ ಕಟ್ಟಿಕೊಂಡು ಏನಾಗಬೇಕು! ಆದ್ದರಿಂದ ಅವನ್ನೆಲ್ಲ ನಿನಗೆ ಕೊಟ್ಟುಬಿಡಲು ಜಗನ್ಮಾತೆಯನ್ನು ಕೇಳಿಕೊಳ್ಳೋಣವೆಂದಿದ್ದೇನೆ. ಹೇಗಿದ್ದರೂ ಮುಂದೆ ನೀನು ಮಾತೆಯ ಕಾರ್ಯಗಳನ್ನು ಬಹಳಷ್ಟು ಸಾಧಿಸುವುದಕ್ಕಿದೆ ಅಂತ ಮಾತೆಯೇ ಹೇಳಿದ್ದಾಳೆ. ಆಗ ನೀನು ಈ ಸಿದ್ಧಿಗಳನ್ನು ಯಥೇಚ್ಛವಾಗಿ ಬಳಸಬಹುದು; ಏನೆನ್ನುವೆ?”

ನರೇಂದ್ರ (ಕ್ಷಣಕಾಲ ಯೋಚಿಸಿ): “ಅವುಗಳಿಂದ ನನಗೆ ಭಗವತ್ಸಾಕ್ಷಾರಕ್ಕೆ ಅನುಕೂಲವಾಗುತ್ತದೆಯೆ?”

ಶ್ರೀರಾಮಕೃಷ್ಣರು: “ಇಲ್ಲ ಇಲ್ಲ. ಆ ದಿಸೆಯಲ್ಲಿ ಅವುಗಳಿಂದ ನಿನಗೇನೂ ಉಪಯೋಗವಾಗ ಲಾರದು. ಆದರೆ ನೀನು ಸಾಕ್ಷಾತ್ಕಾರ ಪಡೆದುಕೊಂಡಾದಮೇಲೆ, ಜಗನ್ಮಾತೆಯ ಕಾರ್ಯಗಳನ್ನು ಸಾಧಿಸಲು ಸಹಾಯವಾದೀತು.”

ನರೇಂದ್ರ ತಕ್ಷಣ ಉತ್ತರಿಸಿದ: “ಹಾಗಾದರೆ ನನಗವೆಲ್ಲ ಬೇಡ. ಮೊದಲು ನನಗೆ ಭಗವತ್ಸಾಕ್ಷಾತ್ಕಾರ ಬೇಕು. ಆ ಸಿದ್ಧಿಗಳನ್ನೆಲ್ಲ ಪಡೆದುಕೊಳ್ಳುವುದೋ ಬಿಡುವುದೋ ಎಂದು ಆಮೇಲೆ ಆಲೋಚಿಸಬಹುದು. ಈಗಲೇ ಅವು ನನ್ನ ಕೈಸೇರಿದರೆ ನಾನು ಅಹಂಕಾರವಶನಾಗಿ ನನ್ನ ಧ್ಯೇಯವನ್ನು ಮರೆತು ಹಾಳಾಗಿಹೋದೇನು.”

ಈ ಉತ್ತರದಿಂದ ಶ್ರೀರಾಮಕೃಷ್ಣರಿಗೆ ಅತ್ಯಂತ ಆನಂದವಾಯಿತಷ್ಟೇ ಅಲ್ಲ, ಸಮಾಧಾನವೂ ಆಯಿತು–ಮುಂದೆ ಲೋಕಶಿಕ್ಷಣಕ್ಕಾಗಿ ನಿಯುಕ್ತವಾಗಲಿರುವ ತಮ್ಮ ನರೇಂದ್ರ ಎಂಥ ಪ್ರಲೋ ಭನೆಗೂ ಒಳಗಾಗುವವನಲ್ಲ ಎಂಬ ಸಮಾಧಾನ. ಈ ಜಗತ್ತಿನ ಸಕಲ ಪ್ರಲೋಭನೆಗಳನ್ನೂ ಮೀರಿಸುವಂಥವು ಈ ಅಣಿಮಾದಿ ಅಷ್ಟಸಿದ್ಧಿಗಳು. ಅವುಗಳಲ್ಲಿ ಒಂದು ದೊರೆತರೂ ಸಾಕು, ಮನುಷ್ಯ ಅದರ ಸಹಾಯದಿಂದ ಪವಾಡ ತೋರಿಸುತ್ತ, ಅದರಿಂದ ದೊರಕುವ ಹೆಸರು ಕೀರ್ತಿಗಳ ಜಾಲದಲ್ಲಿ ಸಿಲುಕಿಕೊಂಡು, ಧ್ಯೇಯದಿಂದ ತಾನು ಜಾರಿಬೀಳುವುದಲ್ಲದೆ ಜನರನ್ನೂ ದಾರಿತಪ್ಪಿಸು ತ್ತಾನೆ. ಹೀಗಿರುವಾಗ ಅಂತಹ ಸಿದ್ಧಿಗಳನ್ನು ತಾವಾಗಿಯೇ ಕೊಡಲು ಮುಂದಾದರೂ ಅವನ್ನು ಒಮ್ಮೆಗೇ ನಿರಾಕರಿಸಿದ ನರೇಂದ್ರ, ಶ್ರೀರಾಮಕೃಷ್ಣರೊಡ್ಡಿದ ಪರೀಕ್ಷೆಯಲ್ಲಿ ಮತ್ತೊಮ್ಮೆ ಉನ್ನತ ದರ್ಜೆಯಲ್ಲಿ ಉತ್ತೀರ್ಣನಾದ.

