
\chapter{ಸಿಡಿಲ ಕಿಡಿ}

\noindent

ನರೇಂದ್ರನನ್ನು ಆರನೆಯ ವಯಸ್ಸಿಗೆ ಗುರುಕುಲದಂತಹ ಪಾಠಶಾಲೆಗೆ ಸೇರಿಸಲಾಯಿತು. ಶಾಲೆಗೆ ಸೇರಿಸುವ ದಿನ ಅವನಿಗೆ ಹೊಸ ಧೋತಿ ಉಡಿಸಿದ್ದರು. ಒಂದು ಪುಟ್ಟ ಚಾಪೆಯನ್ನು ಸುರುಳಿ ಸುತ್ತಿ ಬಗಲಲ್ಲಿಟ್ಟಿದ್ದರು. ಏಕೆಂದರೆ ವಿದ್ಯಾರ್ಥಿಗಳು ತಂತಮ್ಮ ಆಸನಗಳನ್ನು ತಾವುತಾವೇ ತರಬೇಕಾಗಿತ್ತು. ಬಳಪದ ಕಡ್ಡಿಯನ್ನು ಒಂದು ದಾರದ ಸಹಾಯದಿಂದ ಅವನ ಸೊಂಟಕ್ಕೆ ಕಟ್ಟಲಾಗಿತ್ತು. ಕೈಯಲ್ಲಿ ಸ್ಲೇಟನ್ನು ಹಿಡಿದುಕೊಂಡು, ಬಗಲಲ್ಲಿ ಚಾಪೆ ಇಟ್ಟುಕೊಂಡು, ಸೊಂಟ ದಲ್ಲಿ ಕಡ್ಡಿಯನ್ನು ಜೋತಾಡಿಸಿಕೊಂಡು ಜಂಬದಿಂದ ಶಾಲೆಗೆ ಹೊರಟ ನರೇಂದ್ರ.

ಅದು ನರೇಂದ್ರನ ಬಾಲ್ಯದಲ್ಲಿ ಒಂದು ಮುಖ್ಯವಾದ ದಿನ. ಅಂದು ಬೆಳಗ್ಗೆ (ಶಾಲೆಗೆ ಹೊರಡುವ ಮೊದಲು) ಅವನಿಗೆ ಅಕ್ಷರಾಭ್ಯಾಸ ಮಾಡಿಸಲು ಕುಲಪುರೋಹಿತರು ಬಂದರು. ಮನೆಯವರೆಲ್ಲರೂ ಒಟ್ಟಿಗೆ ಸೇರಿದರು. ಪುರೋಹಿತರು ಕೆಲವು ಪ್ರಾರ್ಥನಾಶ್ಲೋಕಗಳನ್ನು ಹೇಳಿ, ಬಾಲಕ ನರೇಂದ್ರನಿಂದ ವಿದ್ಯಾಧಿದೇವತೆಯಾದ ಸರಸ್ವತಿಗೆ ನಮಸ್ಕಾರ ಮಾಡಿಸಿದರು. ಬಳಿಕ ಒಂದು ನಸುಗೆಂಪಿನ ಕಡ್ಡಿಯನ್ನು–ಅದರ ಹೆಸರು ರಾಮ್ಖಾಡಿ–ನರೇಂದ್ರನ ಬಲಗೈಯಲ್ಲಿ ಹಿಡಿಸಿ, ಆ ಕೈಯನ್ನು ತಾವು ಹಿಡಿದುಕೊಂಡು ನೆಲದ ಮೇಲೆ ಬಂಗಾಳೀ ಅಕ್ಷರಗಳನ್ನು ತಿದ್ದಿಸಿದರು. ತಿದ್ದಿಸುವಾಗ, “ಇದು ‘ಕ’, ಇದು ‘ಖ’ ” ಎಂದು ಹೇಳಿಕೊಟ್ಟರು. ಅವನೂ ಅದರಂತೆಯೇ ಹೇಳಿದ.

ನರೇಂದ್ರ ಶಾಲೆಗೆ ಹೋಗಲಾರಂಭಿಸಿದ. ಶಾಲೆ ಎನ್ನುವುದೊಂದು ವಿಚಿತ್ರ ಕ್ಷೇತ್ರ. ಇಲ್ಲಿಗೆ ಬೇರೆಬೇರೆ ಮನೆಗಳಿಂದ ನಾನಾ ಸಂಸ್ಕಾರಗಳ ಮಕ್ಕಳೆಲ್ಲ ಬರುತ್ತಾರಲ್ಲವೆ? ಕೆಲದಿನಗಳಲ್ಲೇ ನರೇಂದ್ರ ಈ ಶಾಲೆಯಲ್ಲಿ ಉಪಾಧ್ಯಾಯರಿಂದ ಪಾಠಗಳನ್ನು ಕಲಿತದ್ದಲ್ಲದೆ ತನ್ನ ಸಹವಿದ್ಯಾರ್ಥಿ ಗಳಿಂದಲೂ ಒಂದಷ್ಟು ಕಲಿತ–ಕೆಲವು ಕೆಟ್ಟ ಶಬ್ದಗಳನ್ನು! ಕಲಿತದ್ದಲ್ಲದೆ ಮನೆಗೆ ಬಂದು ಪ್ರಯೋಗಿಸಿಯೂಬಿಟ್ಟ. ಅದನ್ನು ಕೇಳಿದ ಮನೆಮಂದಿಯೆಲ್ಲ, “ಇದೇನಿದು? ಹುಡುಗ ಶಾಲೆಗೆ ಹೋಗಿ ಕುಲದ ಮರ್ಯಾದೆಯನ್ನೇ ಕಳೆಯುವ ಶಬ್ದಗಳನ್ನು ಕಲಿತುಕೊಂಡು ಬಂದುಬಿಟ್ಟಿ ದ್ದಾನಲ್ಲ!” ಎಂದು ತಬ್ಬಿಬ್ಬಾದರು. ಬಳಿಕ ಒಂದು ನಿಶ್ಚಯಕ್ಕೆ ಬಂದರು–ನರೇಂದ್ರನನ್ನು ಇನ್ನು ಮೇಲೆ ಪಾಠಶಾಲೆಗೆ ಕಳಿಸುವುದು ಬೇಡ; ಉಪಾಧ್ಯಾಯರನ್ನು ಮನೆಗೇ ಬರಮಾಡಿಕೊಂಡು ಮನೆಯಲ್ಲೇ ಹುಡುಗನಿಗೆ ವಿದ್ಯಾಭ್ಯಾಸ ಕೊಡಿಸೋಣ ಎಂದು ಸರಿ; ನರೇಂದ್ರನಿಗೆ ಶಾಲೆಗೆ ಹೋಗುವ ಕೆಲಸ ತಪ್ಪಿಹೋಯಿತು. ಆದರೆ ತಾನು ಶಾಲೆಯ ಹುಡುಗರಿಂದ ಕಲಿತದ್ದು ಉಚ್ಚರಿಸಬಾರದಂಥ ಕೆಟ್ಟ ಶಬ್ದಗಳು ಎನ್ನುವುದು ಅವನಿಗೇನು ಗೊತ್ತು? ಆದ್ದರಿಂದ ಅವನಿಗೆ ತನ್ನನ್ನು ಶಾಲೆಯಿಂದ ಏಕೆ ಬಿಡಿಸಿದರು ಎನ್ನುವುದು ಅರ್ಥವಾಗಲೇ ಇಲ್ಲ. ತಂದೆ ವಿಶ್ವನಾಥ ಅವನು ಮಾಡಿದ ತಪ್ಪನ್ನು ತಿಳಿಸಿಕೊಟ್ಟ. ಆದರೆ ಅಷ್ಟು ಹೊತ್ತಿಗಾಗಲೇ ಅವನು ಆ ಘಟನೆ ಗಳನ್ನೆಲ್ಲ ಮರೆತಾಗಿತ್ತು.

ಮನೆಯ ಪ್ರಾರ್ಥನಾಮಂದಿರದಲ್ಲಿ ನರೇಂದ್ರನಿಗಾಗಿ ತರಗತಿ ಆರಂಭವಾಯಿತು. ಇವನ ಜೊತೆಗೆ ಮನೆಯ ಇತರ ಮಕ್ಕಳೂ ವಿಶ್ವನಾಥನ ಸ್ನೇಹಿತರ ಮಕ್ಕಳೂ ಈ ತರಗತಿಗೇ ಸೇರಿ ಕೊಂಡರು.

ದಿನೇದಿನೇ ವಿದ್ಯಾಭ್ಯಾಸದಲ್ಲಿ ನರೇಂದ್ರನ ಪ್ರಚಂಡ ಬುದ್ಧಿಶಕ್ತಿ ವ್ಯಕ್ತವಾಗತೊಡಗಿತು. ಅವನ ಸಹಪಾಠಿಗಳೆಲ್ಲ ಇನ್ನೂ ಕಾಗುಣಿತದೊಂದಿಗೇ ಕುಸ್ತಿಯಾಡುತ್ತಿದ್ದರೆ ಅವನು ಮಾತ್ರ ಚೆನ್ನಾಗಿ ಓದಲೂ ಕಲಿತಾಯಿತು, ಬರೆಯಲೂ ಕಲಿತಾಯಿತು! ತರಗತಿಯಲ್ಲಿ ಅವನು ಕುಳಿತಿರುತ್ತಿದ್ದ ರೀತಿಯೇ ವಿಚಿತ್ರ. ಒಂದೋ ಕಣ್ಣು ಮುಚ್ಚಿಕೊಂಡು ಸ್ತಬ್ಧವಾಗಿ ಕುಳಿತಿರುತ್ತಿದ್ದ, ಇಲ್ಲವೆ ನೆಲದ ಮೇಲೆ ಒರಗಿಕೊಂಡಿರುತ್ತಿದ್ದ. ಅವನ ಈ ವಿಚಿತ್ರ ವರ್ತನೆ ಮೊದಮೊದಲು ಮಾಸ್ತರಿಗೆ ಅರ್ಥವಾಗಲಿಲ್ಲ. ಏನೋ ಹುಡುಗ ನಾಳೆ ಸರಿಹೋದಾನು, ನಾಡಿದ್ದು ಸರಿಹೋದಾನು ಎಂದು ಕೊಂಡಿದ್ದರೆ ಪ್ರತಿದಿನ ಅದೇ ವಿಚಿತ್ರ ವರ್ತನೆ! ಕೊನೆಗೆ ಇದನ್ನು ಕಂಡು ಮಾಸ್ತರರ ಕೋಪ ಕೆರಳಿತು. ಒಂದು ದಿನ ನರೇಂದ್ರ ಹಾಗೆ ಒರಗಿಕೊಂಡು ಕಣ್ಮುಮುಚ್ಚಿಕೊಂಡಿದ್ದಾನೆ; ಮಾಸ್ತರು ಕಣ್ಣುಕೆಂಪಗೆ ಮಾಡಿಕೊಂಡು ಬಂದು ಅವನನ್ನು ಹಿಡಿದು ಒರಟೊರಟಾಗಿ ಅಲುಗಾಡಿಸಿಬಿಟ್ಟರು. ನರೇಂದ್ರ ಬೆಚ್ಚಿಬಿದ್ದು ಎದ್ದು ಕುಳಿತ. ಅವನ ಅಭಿಮಾನಕ್ಕೆ ಧಕ್ಕೆಯಾಯಿತು. ಪಾಠ ಕೇಳುವುದನ್ನು ಬಿಟ್ಟು ‘ನಿದ್ರೆ’ ಮಾಡುತ್ತಿದ್ದುದಕ್ಕಾಗಿ ಮಾಸ್ತರು ಚೆನ್ನಾಗಿ ಬೈದರು. ಆದರೆ ಅವರಿಗಿನ್ನೂ ಅವನ ಬುದ್ಧಿಸಾಮರ್ಥ್ಯದ ಪರಿಚಯ ಸಾಕಷ್ಟು ಆಗಿರಲಿಲ್ಲ. ಈಗ ನರೇಂದ್ರ, ಅಂದು ಆ ತರಗತಿಯಲ್ಲಿ ಅವರು ಏನೇನು ಹೇಳಿದ್ದರೋ ಅವೆಲ್ಲವನ್ನೂ ಒಂದು ಶಬ್ದವನ್ನೂ ಬಿಡದೆ ಒಪ್ಪಿಸಿಬಿಟ್ಟ. ಉಪಾಧ್ಯಾಯರು ಬೆಕ್ಕಸ ಬೆರಗಾಗಿ ನಿಂತರು. ಆ ಬಗೆಯ ನೆನಪಿನ ಶಕ್ತಿಯಿರುವವರನ್ನು ಅವರೆಂದೂ ಕಂಡಿರಲಿಲ್ಲ. ಅಂದಿನಿಂದ ಅವರು ಅವನನ್ನು ವಿಶೇಷ ಪ್ರೀತಿ ವಿಶ್ವಾಸದಿಂದ ನೋಡಲಾರಂಭಿಸಿದರು.

ನರೇಂದ್ರ ಶಾಲಾ ವಿದ್ಯಾಭ್ಯಾಸವನ್ನು ಮಾಡುತ್ತಿದ್ದರೂ ಅವನು ತನ್ನ ತಾಯಿಯ ಆಶ್ರಯದಲ್ಲಿ ಕಲಿಯುವಂಥದು ನಿಂತಿರಲಿಲ್ಲ. ಭುವನೇಶ್ವರಿ ತಾನೇ ಮಗನಿಗೆ ಬಂಗಾಳೀ ಮತ್ತು ಇಂಗ್ಲಿಷ್ ವರ್ಣಮಾಲೆಯನ್ನು ಅಭ್ಯಾಸ ಮಾಡಿಸಿದಳು; ಹಲವಾರು ನೀತಿಕತೆಗಳನ್ನು ಹೇಳುತ್ತ ಅವನ ನೈತಿಕ ಜೀವನವನ್ನು ರೂಪಿಸಿದಳು; ಭಗವಂತನಿಗೆ ಶರಣಾಗಿದ್ದುಕೊಂಡು, ಜೀವನದಲ್ಲಿ ಎದುರಾಗುವ ಎಲ್ಲ ಕಷ್ಟಕಾರ್ಪಣ್ಯಗಳನ್ನು ಹೇಗೆ ಎದುರಿಸಬೇಕು ಎನ್ನುವುದನ್ನು ಬಾಲಭಾಷೆಯಲ್ಲಿ ತಿಳಿಸಿ ಕೊಡುತ್ತಿದ್ದಳು. ಅಲ್ಲದೆ ಆಗಾಗ ಹೇಳುತ್ತಿದ್ದಳು: “ಮಗೂ, ನೀನು ನಿನ್ನ ಜೀವನದುದ್ದಕ್ಕೂ ಪರಿಶುದ್ಧನಾಗಿರಬೇಕು, ಪವಿತ್ರನಾಗಿರಬೇಕು. ನಿನ್ನ ಆತ್ಮಗೌರವವನ್ನು ಯಾವಾಗಲೂ ಕಾಪಾಡಿಕೊ; ಹಾಗೆಯೇ ಇತರರ ಗೌರವಕ್ಕೂ ಧಕ್ಕೆ ತರದಂತೆ ನೋಡಿಕೊ. ಯಾವಾಗಲೂ ಸಮಾಧಾನದಿಂದಿರುವುದನ್ನು ಅಭ್ಯಾಸಮಾಡು. ಆದರೆ ಸಮಯ ಬಂದರೆ ಕಲ್ಲೆದೆಯವನಾಗಿ ರುವುದಕ್ಕೂ ಸಿದ್ಧನಾಗಿರು.” ತನ್ನನ್ನು ಹೀಗೆ ಸರ್ವಾಂಗಸುಂದರವಾಗಿ ರೂಪಿಸಲು ತನ್ನ ತಾಯಿ ಮಾಡಿದ ಪ್ರಯತ್ನವನ್ನು ನರೇಂದ್ರ ಚೆನ್ನಾಗಿ ಅರಿತುಕೊಂಡ; ಅವಳ ಆದೇಶಗಳನ್ನು ಯಥಾ ವತ್ತಾಗಿ ಪಾಲಿಸಿದ. ಮುಂದೆ ಸ್ವಾಮಿ ವಿವೇಕಾನಂದರು ಹೇಳುತ್ತಿದ್ದರು: “ಯಾವನು ತನ್ನ ತಾಯಿ ಯನ್ನು ಅಕ್ಷರಶಃ ಪೂಜಿಸಿ ಗೌರವಿಸುವುದಿಲ್ಲವೋ ಅವನು ಖಂಡಿತವಾಗಿಯೂ ದೊಡ್ಡ ವ್ಯಕ್ತಿ ಯಾಗಲಾರ.” ಅಲ್ಲದೆ, “ನನ್ನ ವಿವೇಕಬುದ್ಧಿಯನ್ನು ಪರಿಪೂರ್ಣವಾಗಿ ಅರಳಿಸಿದ ನನ್ನ ತಾಯಿಗೆ ನಾನು ಸದಾ ಪುಣಿ” ಎನ್ನುತ್ತಿದ್ದರು ಅವರು.

ನರೇಂದ್ರ ತನ್ನ ತಾಯಿಯಿಂದ ರಾಮಾಯಣದ ಬಹುಭಾಗವನ್ನು ಕಲಿತ ವಿಷಯವನ್ನು ನಾವು ನೋಡಿದ್ದೇವೆ. ಅವನು ಅದನ್ನು ಎಷ್ಟು ನಿಖರವಾಗಿ ಕಲಿತಿದ್ದನೆಂದರೆ, ಕೆಲವು ಕೀರ್ತನಕಾರರು ರಾಮಾಯಣವನ್ನು ಹಾಡಿಕೊಂಡು ಮನೆಯ ಮುಂದೆ ಬಂದಾಗ ಅದನ್ನು ಗಮನವಿತ್ತು ಕೇಳು ತ್ತಿದ್ದ; ಅವರು ಶ್ಲೋಕಗಳನ್ನು ಉಚ್ಚರಿಸುವಾಗ ಮಾಡುತ್ತಿದ್ದ ತಪ್ಪುಗಳನ್ನು ಅವರಿಗೆ ತೋರಿಸಿ ಕೊಡುತ್ತಿದ್ದ. ಆ ಕೀರ್ತನಕಾರರು ಹುಡುಗನ ಸಾಮರ್ಥ್ಯವನ್ನು ಕಂಡು ಬೆರಗಾಗುತ್ತಿದ್ದರು.

ಬಾಲಕ ನರೇಂದ್ರನ ವ್ಯಕ್ತಿತ್ವ ನಿರ್ಮಾಣದಲ್ಲಿ ಅವನ ತಂದೆ ವಿಶ್ವನಾಥ ದತ್ತನ ಕೊಡುಗೆಯೂ ಕಡಿಮೆಯದೇನಲ್ಲ. ಅವನು ನರೇಂದ್ರನಿಗೆ ಬಾಲ್ಯದಿಂದಲೇ ಸಂಗೀತ ಕಲಿಸುವ ವ್ಯವಸ್ಥೆ ಮಾಡಿದ. ಆ ಕಾಲದಲ್ಲಿ ಶಾಸ್ತ್ರೀಯ ಸಂಗೀತಕ್ಕೆ ಅಷ್ಟಾಗಿ ಮನ್ನಣೆಯಿರಲಿಲ್ಲ. ಆದರೆ ಸಂಗೀತವೆನ್ನುವುದು ಸರಳ-ಶುದ್ಧ ಆನಂದವನ್ನು ಕೊಡುವ ವಿದ್ಯೆ ಎಂಬುದು ವಿಶ್ವನಾಥನ ಅಭಿಮತವಾಗಿತ್ತು. ಆದ್ದರಿಂದ ಸಂಗೀತವನ್ನು ಬಹುವಾಗಿ ಪೋತ್ಸಾಹಿಸುತ್ತಿದ್ದ. ಮನೆಯಲ್ಲಿ ಸಂಗೀತದ ವಾತಾವರಣ ವಿದ್ದುದರಿಂದ ನರೇಂದ್ರ ಶ್ರೇಷ್ಠ ಹಾಡುಗಾರನಾಗಿ ನಿರ್ಮಾಣಗೊಳ್ಳುವುದು ಸುಲಭವಾಯಿತು.

ವಿಶ್ವನಾಥ ದತ್ತ ತನ್ನ ಮಕ್ಕಳನ್ನು ತಿದ್ದುವುದರಲ್ಲೊಂದು ವಿಶೇಷತೆಯಿತ್ತು. ಉದಾಹರಣೆಗೆ, ಮಕ್ಕಳು ಯಾರಾದರೂ ತಪ್ಪು ಮಾಡಿದರೆ ವಿಶ್ವನಾಥ ಬೈದು ಕೂಗಾಡುತ್ತಿರಲಿಲ್ಲ. ಬದಲಾಗಿ ಅವನ ಆ ತಪ್ಪು ಅವನ ಸ್ನೇಹಿತರಿಗೆ ತಿಳಿಯುವಂತೆ ಮಾಡುತ್ತಿದ್ದ. ಒಂದು ದಿನ ನರೇಂದ್ರ ತನ್ನ ತಾಯಿಯ ಮೇಲೆ ಕಟುಶಬ್ದಗಳನ್ನು ಪ್ರಯೋಗಿಸಿಬಿಟ್ಟ. ಇದು ವಿಶ್ವನಾಥನ ಕಿವಿಗೆ ಬಿತ್ತು. ಆದರೆ ಅವನು ಮಗನನ್ನು ಕರೆದು ಬೈಯಲಿಲ್ಲ. ಬದಲಾಗಿ ಅವನ ಕೋಣೆಯ ಬಾಗಿಲ ಮೇಲೆ, ‘ನರೇಂದ್ರ ಇಂದು ತನ್ನ ತಾಯಿಯನ್ನು ಇಂತಿಂಥ ಶಬ್ದಗಳಿಂದ ಬೈದಿದ್ದಾನೆ’ ಎಂದು ಬರೆದುಬಿಟ್ಟ. ಅದು ತಿಳಿಯದೆ ನರೇಂದ್ರ ತನ್ನ ಸ್ನೇಹಿತರೊಂದಿಗೆ ಬಂದ. ಬಾಗಿಲ ಮೇಲೆ ಬರೆದಿದ್ದನ್ನು ಎಲ್ಲರೂ ಓದಿಬಿಟ್ಟರು. ಆಗ ನರೇಂದ್ರನಿಗಾದ ನಾಚಿಕೆ, ಅಪಮಾನ ಅಷ್ಟಿಷ್ಟಲ್ಲ. ಮತ್ತೆಂದೂ ಅವನು ಅಂತಹ ತಪ್ಪು ಮಾಡಲಿಲ್ಲ.

ನರೇಂದ್ರನಿಗೆ ತನ್ನಲ್ಲಿ ಅಪಾರ ಆತ್ಮವಿಶ್ವಾಸವಿತ್ತು. ತನ್ನ ಬುದ್ಧಿಶಕ್ತಿಯಲ್ಲಿ ಅವನಿಗೆ ಬಲವಾದ ನಂಬಿಕೆ. ಆದ್ದರಿಂದ ತಾನೇ ತನ್ನ ಸಂಗಾತಿಗಳೆಲ್ಲರ ಮುಖಂಡನಂತೆ ವರ್ತಿಸುತ್ತಿದ್ದ. ಏನಾದರೂ ಕೆಲಸಕಾರ್ಯ ಮಾಡಬೇಕಾದರೆ ಅವನೇ ಅಲ್ಲಿ ಮುಂದಾಳು. ತಿರುಗಾಟಕ್ಕೆ ಹೊರಟರೆ ಅವನೇ ಮುಖಂಡ. ಆಡುವಾಗಲೂ ಅವನೇ ನಾಯಕ. ಅವನು ‘ರಾಜನ ಆಟ’ ಆಡುವುದನ್ನು ನೋಡ ಬೇಕು. ಸ್ನೇಹಿತರೊಂದಿಗೆ ಮನೆಯ ಮೆಟ್ಟಿಲುಗಳನ್ನು ಏರಿಕೊಂಡು ಹೋಗುತ್ತ “ನಾನು ರಾಜಾಧಿ ರಾಜ, ಸಾರ್ವಭೌಮ!” ಎಂದು ಗಟ್ಟಿಯಾಗಿ ಕೂಗಿ ಹೇಳುತ್ತಿದ್ದ. ಬಳಿಕ ಮೇಲಿನ ಮೆಟ್ಟಿಲ ಮೇಲೆ ಚಕ್ರವರ್ತಿಯ ಠೀವಿಯಿಂದ ಕುಳಿತುಬಿಡುತ್ತಿದ್ದ. ಅನಂತರ ಇಬ್ಬರು ಹುಡುಗರಿಗೆ ಅಲ್ಲೇ ಕೆಳಗಿನ ಮೆಟ್ಟಿಲ ಮೇಲೆ ತನ್ನ ಅಕ್ಕಪಕ್ಕದಲ್ಲಿ ನಿಂತುಕೊಳ್ಳುವಂತೆ ಹೇಳುತ್ತಿದ್ದ. ಅವರಲ್ಲಿ ಒಬ್ಬ ಪ್ರಧಾನ ಮಂತ್ರಿಯಂತೆ, ಇನ್ನೊಬ್ಬ ಸೇನಾಧಿಪತಿಯಂತೆ. ಬಳಿಕ ಅದಕ್ಕಿಂತ ಕೆಳಗಿನ ಮೆಟ್ಟಿಲ ಮೇಲೆ ಇನ್ನೈದು ಹುಡುಗರನ್ನು ನಿಲ್ಲಿಸುತ್ತಿದ್ದ. ಇವರು ಅವನಿಗೆ ಕಷ್ಟವನ್ನು ತರುವ ಸಾಮಂತ ರಾಜರು. ಇದಕ್ಕೂ ಕೆಳಗಿನ ಮೆಟ್ಟಿಲುಗಳ ಮೇಲೆ ಇತರ ಆಸ್ಥಾನಿಕರನ್ನು ಕೂಡಿಸಿ ನರೇಂದ್ರ ಕ್ರಮಪ್ರಕಾರ ದುರ್ಬಾರನ್ನು ಪ್ರಾರಂಭಿಸುತ್ತಿದ್ದ. ಆ ಬಳಿಕ ಸಾಮಂತ ರಾಜರು, ಉನ್ನತಾಧಿಕಾರಿ ಗಳು, ಪ್ರಜೆಗಳು ಎಲ್ಲರೂ ಒಬ್ಬೊಬ್ಬರಾಗಿ ಬಂದು ಚಕ್ರವರ್ತಿಗೆ ಕ್ರಮಪ್ರಕಾರವಾಗಿಯೇ ನಮಸ್ಕಾರ ಮಾಡಿ ಗೌರವ ಸಲ್ಲಿಸುತ್ತಿದ್ದರು. ಹಾಗೆ ಮಾಡುವಾಗ, “ಸೂರ್ಯವಂಶ ಪ್ರಸೂತ, ಧರ್ಮರಕ್ಷಣಕೋವಿದ, ರಾಜಾಧಿರಾಜನಿಗೆ ಜಯವಾಗಲಿ!” ಎಂದು ಹೇಳುತ್ತಿದ್ದರು. ಈ ಕಾರ್ಯ ಕ್ರಮವೆಲ್ಲ ಸಾಂಗವಾಗಿ ಮುಗಿದ ಮೇಲೆ ನರೇಂದ್ರ ಚಕ್ರವರ್ತಿ ತನ್ನ ಪ್ರಜೆಗಳ ಯೋಗಕ್ಷೇಮ ವನ್ನು ವಿಚಾರಿಸಿಕೊಳ್ಳುತಿದ್ದ. ಆಗ ಯಾರು ಬೇಕಾದರೂ ತಂತಮ್ಮ ಕಷ್ಟಗಳನ್ನು ಹೇಳಿಕೊಳ್ಳಬಹು ದಾಗಿತ್ತು. ಕೆಲವೊಮ್ಮೆ ಒಬ್ಬ ಕೊಲೆಗಡುಕನನ್ನು ತಂದು ದರ್ಬಾರಿನಲ್ಲಿ ನಿಲ್ಲಿಸಲಾಗುತ್ತಿತ್ತು. ತಳವಾರ, ಪ್ರಧಾನ ಮಂತ್ರಿಗಳು ಎಲ್ಲರೂ ಸೇರಿ ಅವನು ಮಾಡಿದ ಭಯಂಕರ ಅಪರಾಧಗಳನ್ನು ಸನ್ನಿಧಿಯಲ್ಲಿ ಸಾಬೀತುಪಡಿಸುತ್ತಿದ್ದರು. ಆಗ, ನರೇಂದ್ರಚಕ್ರವರ್ತಿ ಅಧಿಕಾರವಾಣಿಯಿಂದ ಆಜ್ಞೆ ಮಾಡುತ್ತಿದ್ದ: “ಯಾರಲ್ಲಿ ಭಟರು! ಇವನನ್ನು ಕರೆದುಕೊಂಡು ಹೋಗಿ ಇವನ ಶಿರಚ್ಛೇದನ ಮಾಡಿಬಿಡಿ!” ಎಂದು. ತಕ್ಷಣ ಹತ್ತು ಮಂದಿ ಭಟರು ನುಗ್ಗಿ ಕೊಲೆಗಡುಕನನ್ನು ಕರೆದೊಯ್ಯು ತ್ತಿದ್ದರು. ಹೀಗೆ ಅವನು ಚಕ್ರವರ್ತಿಯಾಗಿ ತುಂಬ ಗಾಂಭೀರ್ಯದಿಂದ ನ್ಯಾಯವಿತರಣೆ ಮಾಡುತ್ತಿದ್ದ. ಸಭೆಯಲ್ಲಿ ಯಾರಾದರೂ ಸ್ವಲ್ಪ ಅವಿಧೇಯರಾಗಿ ನಡೆದುಕೊಂಡರೆ, ನರೇಂದ್ರ ತನ್ನ ಹುಬ್ಬುಗಳನ್ನು ಸ್ವಲ್ಪಮಾತ್ರ ಗಂಟಿಕ್ಕುವುದರ ಮೂಲಕವೇ ಆ ಅವಿಧೇಯತೆಯನ್ನು ಹತೋಟಿಗೆ ತರುತ್ತಿದ್ದ! ಅಂತೂ ಒಳ್ಳೇ ಠೀವಿಯಿಂದ ನಡೆಯುತ್ತಿತ್ತು ಅವನ ‘ದರ್ಬಾರು’.

ವಿಶ್ವನಾಥ ದತ್ತ ವಕೀಲನಾದ್ದರಿಂದ ಅವನ ಮನೆಗೆ ಹಲವಾರು ಕಕ್ಷಿದಾರರು ಬರುತ್ತಿದ್ದರು. ಅವರೆಲ್ಲ ಬೇರೆಬೇರೆ ಜಾತಿಯವರೆಂದು ಹೇಳಲೇಬೇಕಾಗಿಲ್ಲ. ಇವರಿಗಾಗಿ ಮನೆಯ ಹಜಾರದಲ್ಲಿ ಹಲವಾರು ಹುಕ್ಕಾ ಕೊಳವೆ\footnote{ಬಂಗಾಳದಲ್ಲಿ ಧೂಮಪಾನ ಬಹಳ ವ್ಯಾಪಕವಾದದ್ದು. ಮನೆಗಳಲ್ಲಿ ಹುಕ್ಕ (ಗುಡುಗುಡಿ) ಇಟ್ಟಿರುವುದು ಸಾಮಾನ್ಯ ದೃಶ್ಯ. ಈ ಹುಕ್ಕಾದಲ್ಲಿ ತಂಬಾಕಿನ ಹೊಗೆಯನ್ನು ನೀರು ತುಂಬಿದ ಪಾತ್ರೆಯ ಮೂಲಕ ಎಳೆದುಕೊಂಡು ಸೇದಲಾಗುವಂತೆ ವ್ಯವಸ್ಥೆಯಿರುತ್ತದೆ. ಹೀಗೆ ಎಳೆದುಕೊಳ್ಳುವಾಗ `ಗುಳುಗುಳು' ಎಂಬ ಶಬ್ದವುಂಟಾಗುತ್ತದೆ. ಅತಿಥಿಗಳಿಗೆ ಹುಕ್ಕ ನೀಡಿ ಉಪಚರಿಸುವುದು ಅತಿಥಿಸತ್ಕಾರದ ಒಂದು ಅಂಗವೇ ಆಗಿದೆ.}ಗಳನ್ನು ಇರಿಸಲಾಗಿತ್ತು. ಒಂದೊಂದು ಜಾತಿಯವರಿಗೆ ಒಂದೊಂದು ಕೊಳವೆ. ಕಕ್ಷಿದಾರರು ತಮತಮಗೆ ಮೀಸಲಾದ ಕೊಳವೆಗಳಿಂದ ಸೇದಿ ಹೊಗೆ ಉಗುಳುತ್ತ ಸಲ್ಲಾಪದಲ್ಲಿ ನಿರತರಾಗಿರುತ್ತಿದ್ದರು. ಅವರಲ್ಲೊಬ್ಬ ಮುಸಲ್ಮಾನನೂ ಇದ್ದ. ಇವನು ಅಲ್ಲಿಗೆ ಬಂದಾಗಲೆಲ್ಲ ದಿಂಬುಗಳನ್ನು ಪೇರಿಸಿಕೊಂಡು ಒಳ್ಳೆ ನವಾಬನಂತೆ ಕುಳಿತುಕೊಳ್ಳುತ್ತಿದ್ದ. ಈತ ಹುಕ್ಕಾ ಎಳೆಯುತ್ತ ಮಧ್ಯೆಮಧ್ಯೆ “ಅಲ್ಲಾ ಹೋ ಅಕ್ಬರ್!” ಎಂದು ಉದ್ಗರಿಸಿ ಬಳಿಕ ಮತ್ತೆ “ಗುಳುಗುಳುಗುಳು” ಎಂದು ಹುಕ್ಕಾ ಎಳೆಯುತ್ತಿದ್ದ. ಒಮ್ಮೆ ಈ ವೇಳೆಗೆ ಸರಿಯಾಗಿ ನರೇಂದ್ರ ಒಳಗೆ ಬಂದ. ಸೊಂಟಕ್ಕೆ ಸುತ್ತಿದ ಪುಟ್ಟ ಧೋತಿ ಬಿಟ್ಟರೆ ಮೈಮೇಲೇನೂ ಇಲ್ಲ. ಅವನು ಒಳಗೆ ಬಂದ ಕೂಡಲೇ ಅವರೆಲ್ಲ “ಓ! ನರೇನ್ ನರೇನ್!” ಎಂದು ಅವನನ್ನು ಕರೆದರು. ಆ ಮುಸಲ್ಮಾನ ಕಕ್ಷಿದಾರ, “ಯಾ ಅಲ್ಲಾ! ಬಾ ಮಗೂ, ಇಲ್ಲಿ ಬಾ” ಎಂದ. ನರೇಂದ್ರ ಸೀದಾ ಅವನ ಬಳಿಗೆ ಹೋದ. ಆ ಕಕ್ಷಿದಾರರಿಗೆಲ್ಲ ನರೇಂದ್ರ ಬಹಳ ಪ್ರಿಯ. ಅದರಲ್ಲೂ ಈ ಮುಸಲ್ಮಾನ ಕಕ್ಷಿದಾರನಿಗೆ ಅವನನ್ನು ಕಂಡರೆ ಬಹಳ ಇಷ್ಟ. ನರೇಂದ್ರ ಅವನನ್ನು ಸಲಿಗೆಯಿಂದ, ‘ಮುಸಲ್ಮಾನ ಮಾಮ’ ಎಂದು ಕರೆಯುತ್ತಿದ್ದ. ಅವನು ನರೇಂದ್ರನಿಗೆ ಒಳ್ಳೊಳ್ಳೆಯ ಕೆರವಾನ್ ಕಥೆಗಳನ್ನು ಹೇಳುತ್ತಿದ್ದ. ಅರಬರು ಒಂಟೆಗಳ ಮೇಲೆ ಸರಕು ಹೇರಿಕೊಂಡು ಪಂಜಾಬಿನಿಂದ ಅಫಘಾನಿಸ್ತಾನದವರೆಗೆ ವಿಶಾಲವಾದ ಮೈದಾನ ಪ್ರದೇಶಗಳಲ್ಲಿ ಪ್ರಯಾಣ ಮಾಡುತ್ತ ಹೋಗುವ ರೋಮಾಂಚಕಾರಿ ಕಥೆಗಳನ್ನು ರಂಗುರಂಗಾಗಿ ಹೇಳುತ್ತಿದ್ದ. ಆ ದಾರಿ ಎಷ್ಟು ಉದ್ದವೋ ಮುಸಲ್ಮಾನ ಮಾಮನ ಕಥೆಗಳೂ ಅಷ್ಟೇ ಉದ್ದ! ಈ ಕಥೆಗಳನ್ನೆಲ್ಲ ನರೇಂದ್ರ ಕಣ್ಣರಳಿಸಿಕೊಂಡು ಕೇಳುತ್ತಿದ್ದ. “ಮಾಮಾ, ನನ್ನನ್ನೂ ಅಲ್ಲಿಗೆಲ್ಲ ಕರೆದುಕೊಂಡು ಹೋಗು” ಎಂದು ಕಾಡುತ್ತಿದ್ದ. ಆಗ ಮುಸಲ್ಮಾನ ಮಾಮಾ, “ಖಂಡಿತವಾಗಿ! ನೀನು ಇನ್ನು ಒಂದಿಂಚು ಬೆಳೆದ ಕೂಡಲೇ ಕರೆದುಕೊಂಡು ಹೋಗುತ್ತೇನೆ!” ಎನ್ನುತ್ತಿದ್ದ. ಒಮ್ಮೆ ಈ ಮುಸಲ್ಮಾನ ಮಾಮ ನರೇಂದ್ರನಗೆ ಒಂದು ಮಿಠಾಯಿ ತಂದುಕೊಟ್ಟ. ಇವನು ಅದನ್ನು ತಿಂದೇಬಿಟ್ಟ. ಇದನ್ನು ನೋಡಿ ಅಲ್ಲಿದ್ದವರೆಲ್ಲ “ಇದೇನಪ್ಪ ಇದೂ! ಛೀಛೀಛೀಛೀ!” ಎಂದುದ್ಗರಿಸುತ್ತ ಆವೇಶಭರಿತರಾಗಿ ಹುಕ್ಕಾವನ್ನು “ಗುಳುಗುಳುಗುಳುಗುಳು” ಎಂದು ಇನ್ನಷ್ಟು ಜೋರಾಗಿ ಸೇದಿದರು. ಆ ವೇಳೆಗೆ ವಿಶ್ವನಾಥ ಅಲ್ಲಿಗೆ ಬಂದ. ನೋಡುತ್ತಾನೆ, ಕಕ್ಷಿದಾರರೆಲ್ಲ ಮುಖ ಸಿಂಡರಿಸಿಕೊಂಡು ಒಮ್ಮೆ ಆ ಮುಸಲ್ಮಾನನನ್ನು, ಇನ್ನೊಮ್ಮೆ ಜಾತಿನಿಯಮವನ್ನು ಮುರಿದ ನರೇಂದ್ರನನ್ನು ದುರುಗುಟ್ಟಿ ಕೊಂಡು ನೋಡುತ್ತಿದ್ದಾರೆ. ವಿಶ್ವನಾಥ ಈ ದೃಶ್ಯವನ್ನು ನೋಡಿ ಸುಮ್ಮನೆ ನಸುನಕ್ಕ. ಮಗನಿ ಗೇನೂ ಹೇಳಲಿಲ್ಲ.

ಆದರೆ ಈ ಘಟನೆಯಿಂದಾಗಿ ನರೇಂದ್ರನ ತಲೆಯೊಳಗೊಂದು ಪ್ರಶ್ನೆ ಹುಟ್ಟಿಕೊಂಡಿತು. ತಾನು ಮುಸಲ್ಮಾನ ಮಾಮನ ಕೈಯಿಂದ ಒಂದು ತುಂಡು ಮಿಠಾಯಿ ತೆಗೆದುಕೊಂಡು ತಿಂದದ್ದಕ್ಕೆ ಇವರೆಲ್ಲ ಹೌಹಾರಿಬಿಟ್ಟರಲ್ಲ, ಅಂಥಾ ತಪ್ಪು ಏನಾಯಿತು ತನ್ನಿಂದ? ಅಲ್ಲದೆ, ಬೇರೆಬೇರೆ ಜಾತಿಯವರಿಗೆ ಅಂತ ಇಲ್ಲಿ ಬೇರೆಬೇರೆ ಹುಕ್ಕದ ವ್ಯವಸ್ಥೆ ಮಾಡಿದ್ದಾರಲ್ಲ ಯಾಕೆ?... ಅವನಿಗೆ ಈ ಜಾತಿಯೆನ್ನುವುದೊಂದು ವಿಚಿತ್ರ ರಹಸ್ಯವಾಗಿ ಕಾಣತೊಡಗಿತು. ‘ಒಂದು ಜಾತಿಯವರು ಇನ್ನೊಂದು ಜಾತಿಯವರ ಕೈಯಿಂದ ತಿಂದರೆ, ಅಥವಾ ಇನ್ನೊಂದು ಜಾತಿಯವರ ಹುಕ್ಕಾ ಸೇದಿದರೆ ಆಗುವುದೇನು? ಮನೆ ಮುರಿದುಕೊಂಡು ತಲೆಮೇಲೆ ಬೀಳುತ್ತದೆಯೇ? ಅಥವಾ ಸೇದಿದವರು ತಕ್ಷಣ ಸಾಯುತ್ತಾರೆಯೇ?’ ಎಂದು ಆಲೋಚಿಸತೊಡಗಿದ. ಕೊನೆಗೆ, ಇದನ್ನು ಪರೀಕ್ಷೆ ಮಾಡಿ ನೋಡೇಬಿಡುವುದು ಎಂಬ ತೀರ್ಮಾನಕ್ಕೆ ಬಂದ. ಸರಿ; ವ್ಯವಹಾರವನ್ನೆಲ್ಲ ಮುಗಿಸಿ ತಂದೆ ತನ್ನ ಕಕ್ಷಿದಾರರನ್ನು ಬೀಳ್ಗೊಳ್ಳಲು ರಸ್ತೆಯವರೆಗೆ ಹೋದಾಗ ಇವನು ಒಳಗೆ ಬಂದ; ಅಲ್ಲಿದ್ದ ಎಲ್ಲ ಹುಕ್ಕಾಕೊಳವೆಗಳನ್ನೂ ಒಂದೊಂದಾಗಿ ತನ್ನ ಮರಿತುಟಿಗಿಟ್ಟುಕೊಂಡು ಸೇದಿನೋಡಿದ. ಇಲ್ಲ! ತಾನು ಸಾಯಲಿಲ್ಲ. ಮನೆಯ ಮಾಡು ಮುರಿದುಕೊಂಡು ಬೀಳಲಿಲ್ಲ! ಎಲ್ಲ ಹೇಗಿತ್ತೋ ಹಾಗೆಯೇ ಇದೆ. ಅವನು ಹೀಗೆ ಪರೀಕ್ಷೆ ಮಾಡುತ್ತಿರುವ ಹೊತ್ತಿಗೆ ತಂದೆ ಹಿಂದಿರುಗಿ ಬಂದ; ಮಗನ ಕಾರುಬಾರನ್ನು ನೋಡಿ ಆಶ್ಚರ್ಯದಿಂದ ಕೇಳಿದ, “ಏನಪ್ಪಾ, ಏನು ಮಾಡುತ್ತಿದ್ದೀಯಾ?” ನರೇಂದ್ರ ಹೇಳುತ್ತಾನೆ, “ಏನಿಲ್ಲ ಅಪ್ಪ, ಜಾತಿಭೇದವನ್ನು ಮುರಿದರೆ ಏನಾಗುತ್ತೆ ಅಂತ ನೋಡುತ್ತಿದ್ದೇನೆ!” ಇದನ್ನು ಕೇಳಿ ವಿಶ್ವನಾಥ ಹೊಟ್ಟೆತುಂಬ ನಕ್ಕು ತನ್ನ ಕೋಣೆಗೆ ಹೊರಟುಹೋದ.

ನರೇಂದ್ರನ ಶರೀರದಲ್ಲೂ ಮನಸ್ಸಿನಲ್ಲೂ ಅಪರಿಮಿತ ಶಕ್ತಿ ತುಂಬಿಕೊಂಡಿತ್ತು. ಈ ಶಕ್ತಿ ನಾನಾ ಬಗೆಯಾಗಿ ವ್ಯಕ್ತವಾಗಲು ದಾರಿಯನ್ನು ನೋಡುತ್ತಿತ್ತು. ಇದರಿಂದಾಗಿ ಆತನ ಹುಡು ಗಾಟಿಕೆ-ತುಂಟತನಗಳ ತೀವ್ರತೆಯೂ ಹೆಚ್ಚಿತು. ಒಂದು ದಿನ ಅವನು ಮನೆಯಲ್ಲಿ ತನ್ನ ಸ್ನೇಹಿತರ ಜೊತೆಯಲ್ಲಿ ಕಣ್ಣಾಮುಚ್ಚಾಲೆ ಆಡುತ್ತಿದ್ದಾಗ, ವೇಗವಾಗಿ ಓಡುವ ರಭಸದಲ್ಲಿ ಜಗಲಿಯಿಂದ ಕೆಳಗೆ ಕಲ್ಲಿನ ಮೇಲೆ ಬಿದ್ದುಬಿಟ್ಟ. ಹಣೆ ಜಜ್ಜಿ ರಕ್ತ ಸೋರಿತು. ಅವನ ಬಲಹುಬ್ಬಿನ ಮೇಲ್ಭಾಗದಲ್ಲಿ ಒಂದು ದೊಡ್ಡ ಗಾಯವೇ ಆಯಿತು. ಈ ಗಾಯದ ಗುರುತು ಹಾಗೆಯೇ ಉಳಿದುಕೊಂಡುಬಿಟ್ಟಿತು. ಮುಂದೆ ಈ ಘಟನೆಯ ಕುರಿತಾಗಿ ತಿಳಿದಾಗ ಶ್ರೀರಾಮಕೃಷ್ಣರು ಹೇಳುತ್ತಾರೆ, “ಒಂದು ವೇಳೆ ಈ ಅಪಘಾತದಿಂದಾಗಿ ನರೇಂದ್ರನ ಅಮಿತ ಶಕ್ತಿ ಹತೋಟಿಗೆ ಬರದೇ ಹೋಗಿದ್ದರೆ ಅವನು ಇಡೀ ಜಗತ್ತನ್ನೇ ನುಚ್ಚುನೂರು ಮಾಡಿಬಿಡುತ್ತಿದ್ದ!” ಎಂದು.

ತನ್ನ ಓರಗೆಯವರಿಗೆಲ್ಲ ನರೇಂದ್ರನೇ ನಾಯಕನಾಗಿದ್ದ ಎನ್ನುವುದನ್ನು ಈಗಾಗಲೇ ನೋಡಿ ದ್ದೇವೆ. ನಾಯಕತ್ವ ಎನ್ನುವುದು ಅವನ ಹುಟ್ಟುಗುಣವೇ ಆಗಿತ್ತು. ನಾಯಕನಾಗುವುದೆಂದರೆ ರಾಜನಾಗಿ ದರ್ಬಾರು ಮಾಡುವುದು ಮಾತ್ರವಲ್ಲ, ಸಮಯ ಬಂದರೆ ತ್ಯಾಗಕ್ಕೂ ಸಿದ್ಧನಾಗಿರ ಬೇಕಾಗುತ್ತದೆ. ಈ ಬಗೆಯ ತ್ಯಾಗಬುದ್ಧಿಯನ್ನು ನರೇಂದ್ರ ತನ್ನ ಬಾಲ್ಯದಿಂದಲೇ ಪ್ರಕಟಿಸುವು ದನ್ನು ಕಾಣಬಹುದಾಗಿತ್ತು. ಅವನಿಗೆ ಆಗ ಸುಮಾರು ಆರು ವರ್ಷ ವಯಸ್ಸು; ಒಮ್ಮೆ ತನ್ನ ಸಂಬಂಧಿಯಾದ ಹುಡುಗನನ್ನು ಕರೆದುಕೊಂಡು ಹತ್ತಿರದ ಜಾತ್ರೆಗೆ ಹೋಗಿದ್ದ. ಜಾತ್ರೆಯಲ್ಲಿ ಅವನು ಶಿವನ ಬೇರೆಬೇರೆ ನಮೂನೆಯ ಗೊಂಬೆಗಳನ್ನು ಕೊಂಡಕೊಂಡ. ಬಳಿಕ ಇಬ್ಬರೂ ಮನೆಕಡೆ ನಡೆದರು. ಸಂಜೆಗತ್ತಲಾಗುತ್ತಿತ್ತು. ಅವರು ನಡೆದುಕೊಂಡು ಬರುತ್ತಿರುವಾಗ ಹುಡುಗ ಜನಸಂದಣಿಯಲ್ಲೆಲ್ಲೋ ತಪ್ಪಿಸಿಕೊಂಡು ಸ್ವಲ್ಪ ಹಿಂದುಳಿದ. ನರೇಂದ್ರ ಭಾವಿಸಿದ್ದ, ಅವನು ಹಿಂದೆಯೇ ಇದ್ದಾನೆ ಎಂದು. ಹೀಗಿರುವಾಗ, ಒಂದು ಸಾರೋಟು ವೇಗವಾಗಿ ಬರುತ್ತಿರುವ ಶಬ್ದ ಕೇಳಿಸಿತು ಅವನಿಗೆ. ಆ ಶಬ್ದ ಕೇಳಿ ಅಕಸ್ಮಾತ್ತಾಗಿ ಹಿಂದಿರುಗಿ ನೋಡುತ್ತಾನೆ, ಎದೆ ತಲ್ಲಣಿಸು ವಂತಹ ದೃಶ್ಯ. ತನ್ನ ಸಂಗಡಿಗ ನಡುರಸ್ತೆಯಲ್ಲಿ ಬೆಪ್ಪುಗಟ್ಟಿ ನಿಂತಿದ್ದಾನೆ, ಸಾರೋಟು ಇನ್ನೇನು ಅವನ ಮೈಮೇಲೆ ಹರಿಯುವುದರಲ್ಲಿದೆ! ತಕ್ಷಣ ನರೇಂದ್ರ ಬೊಂಬೆಗಳನ್ನು ಎಡದ ಬಗಲಿನಲ್ಲಿ ಇರಿಸಿಕೊಂಡು, ತನ್ನ ಜೀವದ ಹಂಗನ್ನೇ ತೊರೆದು ಮುನ್ನುಗ್ಗಿದ; ಸರ್ರನೆ ಹೋಗಿ ಕಣ್ಣೆವೆಯಿಕ್ಕು ವಷ್ಟರಲ್ಲಿ ಆ ಹುಡುಗನನ್ನು ಈ ಕಡೆಗೆ ಎಳೆದುಕೊಂಡುಬಿಟ್ಟ. ಬಾಲಕ ನರೇಂದ್ರನ ಈ ಸಾಹಸಕೃತ್ಯವನ್ನು ಕಂಡು ಸುತ್ತಲಿನ ಜನ ಬಿಲ್ಲಂಬೆರಗಾದರು. ಈ ಗಂಡಾಂತರ ಇದ್ದಕ್ಕಿದ್ದಂತೆ ಸಂಭವಿಸಿದ್ದರಿಂದ ಅದನ್ನು ಗಮನಿಸಿ ಸಹಾಯಕ್ಕೆ ಮುನ್ನುಗ್ಗಲು ಬೇರೆ ಯಾರಿಗೂ ವ್ಯವಧಾನವೇ ಇರಲಿಲ್ಲ. ಆದರೆ ಇವನು ಮಾತ್ರ ಮಿಂಚಿನ ವೇಗದಲ್ಲಿ ಪರಿಸ್ಥಿತಿಯನ್ನು ಅರ್ಥಮಾಡಿಕೊಂಡು ಆ ಹುಡುಗನನ್ನು ಮೃತ್ಯುಮುಖದಿಂದ ಬಿಡಿಸಿ ತಂದುಬಿಟ್ಟಿದ್ದ. ಇದನ್ನು ನೋಡಿದ ಕೆಲವರು ನರೇಂದ್ರನ ಬೆನ್ನು ತಟ್ಟಿದರು; ಕೆಲವರು ತಲೆಮುಟ್ಟಿ ಹರಸಿದರು. ಹುಡುಗರಿಬ್ಬರೂ ಮನೆಗೆ ಹಿಂದಿರುಗಿದರು. ಭುವನೇಶ್ವರಿಗೆ ಈ ವಿಷಯ ತಿಳಿದಾಗ ಹೆಮ್ಮೆ ಸಂತೋಷಗಳಿಂದ ಕಣ್ಣುಗಳಲ್ಲಿ ಕಂಬನಿದುಂಬಿತು. “ಮಗೂ, ಯಾವಾಗಲೂ ಹೀಗೆಯೇ ಧೀರನಾಗಿರು” ಎಂದು ಮಗನನ್ನು ಆಶೀರ್ವದಿಸಿದಳು.

ನರೇಂದ್ರನ ಬಾಲ್ಯದಲ್ಲೇ ಅವನ ಭವಿಷ್ಯಜೀವನದ ಉಜ್ವಲತೆಯನ್ನು ಗುರುತಿಸಿದವರುಂಟು. ಅವರಲ್ಲಿ ವಿಶ್ವನಾಥನ ಚಿಕ್ಕಪ್ಪನಾದ ಕಾಳೀಪ್ರಸಾದನೂ ಒಬ್ಬ. ಈಗ ಆತ ಮರಣಶಯ್ಯೆಯಲ್ಲಿ ದ್ದಾನೆ. ಒಂದು ದಿನ ತನ್ನ ಕೊನೆಗಾಲ ಸಮೀಪಿಸಿದೆ ಎನ್ನುವುದು ಅವನಿಗೆ ಗೊತ್ತಾಯಿತು. ಆದ್ದರಿಂದ ಮನೆಮಂದಿಯನ್ನೆಲ್ಲ ತನ್ನ ಬಳಿಗೆ ಕರೆಯಿಸಿಕೊಂಡು, “ಯಾರಾದರೊಬ್ಬರು ಮಹಾ ಭಾರತದ ಶ್ಲೋಕಗಳನ್ನು ಗಟ್ಟಿಯಾಗಿ ಓದಿ; ಅದನ್ನು ಕೇಳುತ್ತ ಕೇಳುತ್ತ ಪ್ರಾಣ ಬಿಡುತ್ತೇನೆ” ಎಂದು ಕೇಳಿಕೊಂಡ. ಆದರೆ ಆ ಸಾವಿನ ಬೀಭತ್ಸ ವಾತಾವರಣದಲ್ಲಿ ಏನನ್ನಾದರೂ ಓದುತ್ತ ಕುಳಿತುಕೊಳ್ಳುವ ಸ್ಥಿರತೆ ಯಾರಿಗಿರುತ್ತದೆ? ಇದನ್ನು ನರೇಂದ್ರ ನೋಡಿದ. ಯಾರೂ ಓದಲು ಮುಂದಾಗದಿದ್ದುದನ್ನು ಕಂಡು ತಾನೇ ಹೋಗಿ ಮಹಾಭಾರತವನ್ನು ತಂದು ತೆರೆದಿಟ್ಟುಕೊಂಡ. ಪುಟಗಳ ಮೇಲೆ ಪುಟಗಳನ್ನು ತಿರುವುತ್ತ ತನ್ನ ಸ್ವಷ್ಟ ಉಚ್ಚಕಂಠದಿಂದ ಪಾಂಡವರ ಹಾಗೂ ಶ್ರೀಕೃಷ್ಣನ ಮಹಿಮೆಯನ್ನು ವರ್ಣಿಸುವ ಶ್ಲೋಕಗಳನ್ನು ಓದುತ್ತ ಬಂದ. ಕಡೆಗೆ, ಗರುಡ ತನ್ನ ತಾಯಿಯಾದ ವಿನತೆಯನ್ನು ಬೆನ್ನಮೇಲೆ ಕೂರಿಸಿಕೊಂಡು ಹಾರಿ ಹೋಗುವ ವರ್ಣನೆ ಬಂತು. ಅದನ್ನು ಕೇಳುತ್ತಿದ್ದಂತೆ ಮುದುಕ ಕಾಳೀಪ್ರಸಾದನ ಶ್ವಾಸೋಚ್ಛ್ವಾಸ ನಿಧಾನವಾಗುತ್ತ ಬಂದಿತು. ಆಗ ಅವನು ಮೆಲುದನಿಯಲ್ಲಿ, ಆದರೆ ಸ್ಪಷ್ಟವಾಗಿ, ಒಂದು ಮಾತನ್ನು ಹೇಳುತ್ತಾನೆ: “ಮಗು, ನರೇಂದ್ರ, ನಿನ್ನ ಭವಿಷ್ಯ ಅತ್ಯಂತ ಉಜ್ವಲವಾಗಿದೆ!” ಕೆಲವರಿಗೆ ಮರಣಕಾಲದಲ್ಲಿ ಕೆಲವು ಸತ್ಯಾಂಶಗಳು ಗೋಚರಿಸುವುದುಂಟು. ಈಗ ಕಾಳೀಪ್ರಸಾದನೂ ನರೇಂದ್ರನ ಉಜ್ವಲ ಜೀವನದ ಮುಂಬೆಳಕನ್ನು ಸ್ಪಷ್ಟವಾಗಿ ಕಂಡಿರಬೇಕು. ಈ ಮಾತನ್ನು ಹೇಳಿದವನೇ ಅವನು ತನ್ನ ಜಡ ಶರೀರವನ್ನು ಬಿಟ್ಟು ಅತೀಂದ್ರಿಯ ಲೋಕಕ್ಕೆ ಹೊರಟುಬಿಟ್ಟ.

“ಬೆಳೆಯ ಗುಣ ಮೊಳಕೆಯಲ್ಲಿ” ಎಂಬ ಮಾತಿದೆ. ನಿಜಕ್ಕೂ ನರೇಂದ್ರನ ಬಾಲ್ಯವೆಂಬ ಮೊಳಕೆ ಅತ್ಯಂತ ದಷ್ಟಪುಷ್ಟವಾಗಿತ್ತು. ಅವನ ಸರ್ವಾಂಗಸುಂದರ ಶರೀರ, ತೇಜಸ್ವಿಯಾದ ಮೈಕಾಂತಿ, ಹೊಳೆಹೊಳೆಯುವ ವಿಶಾಲ ನಯನಗಳು, ಅರಳುತ್ತಿರುವ ಪ್ರತಿಭೆಯನ್ನು ಸೂಚಿಸುವ ಮುಖ ಮುದ್ರೆ–ಇವು ಎಲ್ಲರ ಗಮನ ಸೆಳೆಯುತ್ತಿದ್ದುವು. ಮನದಲ್ಲಿ ನೂರಾರು ಉನ್ನತ ಉದಾತ್ತ ಕಲ್ಪನೆಗಳು, ಹೃದಯದಲ್ಲಿ ಸದಾ ಜಿನುಗುವ ಪ್ರೇಮ, ಬುದ್ಧಿಯಲ್ಲಿ ಕತ್ತಿಯಲಗಿನ ಹರಿತ, ಸರಿಸಾಟಿಯಿಲ್ಲದ ಎದೆಗಾರಿಕೆ, ದಂಗು ಬಡಿಸುವಂಥ ಸಂಶೋಧನಾ ಪ್ರತಿಭೆ, ದಣಿವೆನ್ನದೆ ದುಡಿಯಬಲ್ಲ ಕಾರ್ಯಶೀಲತೆ, ಮತ್ತು ಒತ್ತಿದಷ್ಟೂ ಎತ್ತರಕ್ಕೆ ಪುಟಿಯಬಲ್ಲ ಉತ್ಸಾಹ ಅವನದು. ಬಾಲ್ಯದಿಂದಲೂ ಅವನಿಗೆ ಯಾರೂ ಸರಿಸಮರಿಲ್ಲ. ಇಷ್ಟಲ್ಲದೆ ಭಗವಂತನ ಕಡೆಗೆ ಹರಿಯುತ್ತಿದ್ದ ಅವನ ಭಕ್ತಿ ಎಂಥದು! ಕುಳಿತರೆ ಗಂಟೆಗಟ್ಟಲೆ ಧ್ಯಾನ ಮಾಡಬಲ್ಲ ಸಾಮರ್ಥ್ಯವೆಂಥದು! ಪೂಜೆ ಪ್ರಾರ್ಥನೆಗಳ ಸಂಬಂಧವಾದ ಅವನ ಅಭಿರುಚಿ ಎಂಥದು! ಇಂಥ ನರೇಂದ್ರನ ಇಡಿಯ ಜೀವನವನ್ನು ನಾವು ಆಳವಾಗಿ, ಆಮೂಲಾಗ್ರವಾಗಿ ಅಧ್ಯಯನ ಮಾಡಿದಾಗ ಈ ಬಣ್ಣನೆಯೆಲ್ಲ ಕೇವಲ ಉತ್ಪ್ರೇಕ್ಷೆಯಲ್ಲವೆಂಬ ಅರಿವು ನಮಗಾದೀತು. ಅಷ್ಟೇ ಅಲ್ಲ, ಅವನ ವ್ಯಕ್ತಿತ್ವದ ಪೂರ್ಣ ಪರಿಚಯ ಮಾಡಿಕೊಳ್ಳುವುದು ಇತರರಿಂದ ಸಾಧ್ಯವಿಲ್ಲ ಎಂಬ ಅರಿವೂ ನಮಗಾದೀತು. ಆದರೆ ಭವಿಷ್ಯದ ವಿಷಯ ಹಾಗಿರಲಿ, ಈಗ ನಾವು ಅವನ ಶಾಲಾ ದಿನಗಳನ್ನು ನೋಡಬೇಕು. 

ಹಿಂದೆ ನರೇಂದ್ರ ಪಾಠಶಾಲೆಯಲ್ಲಿ ಇತರ ಮಕ್ಕಳಿಂದ ಕೆಟ್ಟ ಶಬ್ದಗಳನ್ನು ಕಲಿತುಕೊಂಡು ಬಂದದ್ದರಿಂದ ಅವನನ್ನು ಶಾಲೆಯಿಂದ ಬಿಡಿಸಿ, ಮನೆಯಲ್ಲೇ ಪಾಠ ಹೇಳಿಸುವ ವ್ಯವಸ್ಥೆ ಮಾಡಿದ್ದು ನಮಗೆ ನೆನಪಿದೆ. ಆದರೆ ಈಗ ಅವನಿಗೆ ಮನೆಯಲ್ಲೇ ಹೇಳಿಸಿಕೊಳ್ಳುವ ವಯಸ್ಸು ದಾಟಿದೆ. ಆದ್ದರಿಂದ ಅವನನ್ನು ಈಶ್ವರಚಂದ್ರ ವಿದ್ಯಾಸಾಗರರಿಂದ ಸ್ಥಾಪಿತವಾದ ‘ಮೆಟ್ರೋ ಪಾಲಿಟನ್ ಶಾಲೆ’ಗೆ ಸೇರಿಸಲಾಯಿತು. ನರೇಂದ್ರ ಶಾಲೆಗೆ ಚಡ್ಡಿ ತೊಟ್ಟುಕೊಂಡು ಹೋಗುತ್ತಿದ್ದ, ಧೋತಿಯನ್ನಲ್ಲ. ಆದರೆ ಶಾಲೆ ಮುಗಿಸಿಕೊಂಡು ಮನೆಗೆ ಬರುವಾಗ ಪ್ರತಿದಿನವೂ ಅವನ ಚಡ್ಡಿ ಹರಿದಿರುತ್ತಿತ್ತು. ಏಕೆ? ಅಷ್ಟೊಂದು ತುಂಟತನ, ಅಷ್ಟೊಂದು ಚೇಷ್ಟೆ! ಅವನ ಚೇಷ್ಟೆಯನ್ನು ಮೀರಿಸುವ ಚೇಷ್ಟೆಯನ್ನು ಯಾರೂ ಕಂಡಿರಲಿಕ್ಕಿಲ್ಲ! ತರಗತಿಯಲ್ಲಿ ಬೆಂಚಿನ ಮೇಲೆ ಒಂದು ದಿನವಾದರೂ ಶಾಂತವಾಗಿ ಕುಳಿತುಕೊಂಡವನೇ ಅಲ್ಲ ಅವನು. ಒಮ್ಮೆ ಆ ಕಡೆ ತಿರುಗಿ ಕುಳಿತರೆ ಇನ್ನೊಮ್ಮೆ ಈ ಕಡೆ ತಿರುಗಿ ಕುಳಿತ! ಅಲ್ಲದೆ, ಸ್ವಲ್ಪ ಹೊತ್ತು ಕುಳಿತಿದ್ದರೆ ಇನ್ನೆಷ್ಟೋ ಹೊತ್ತು ನಿಂತಿರುತ್ತಿದ್ದ. ಅಷ್ಟೊಂದು ಶಕ್ತಿ ಪುಟಿಯುತ್ತಿತ್ತು ಅವನಲ್ಲಿ! ಅದು ಅವನನ್ನು ಸುಮ್ಮನಿರಲು ಬಿಟ್ಟರೆ ತಾನೆ! ಆಟಕ್ಕಿಳಿದರೆ ತುಂಬ ಬಿರುಸಿನಿಂದ ಆಟವಾಡುತ್ತಿದ್ದ. ಅವನ ಆಟಗಳಲ್ಲಿ ಹಾರುವುದು, ನೆಗೆಯುವುದು, ಕುಸ್ತಿ, ಕ್ರಿಕೆಟ್, ಗೋಲಿಯಾಟ–ಎಲ್ಲವೂ ಸೇರಿತ್ತು. ಕ್ರಿಕೆಟ್ನಲ್ಲಿ ಅವನು ಪ್ರವೀಣ. ಪ್ರತಿದಿನವೂ ಅದರ ಮರುದಿನ ಯಾವಯಾವ ಆಟಗಳನ್ನಾಡಬೇಕೆಂಬ ಪಟ್ಟಿ ತಯಾರಿಸುವವನು ಅವನೇ. ಕೆಲವೊಮ್ಮೆ ಹುಡುಗರಲ್ಲಿ ವಿರಸ ಹುಟ್ಟಿಕೊಳ್ಳುತ್ತಿತ್ತು. ಇದು ಮೊದಲು ಮಾತಿನಿಂದ ಪ್ರಾರಂಭವಾಗಿ ಬಳಿಕ ಮುಷ್ಟಾಮುಷ್ಟಿ ಕೇಶಾಕೇಶಿಗಳಿಗೆ ತಿರುಗಿಕೊಳ್ಳು ತ್ತಿತ್ತು. ನರೇಂದ್ರನಿಗೆ ಇವನ್ನೆಲ್ಲ ಕಂಡರಾಗದು. ಹುಡುಗರ ನಡುವೆ ಜಗಳ ಹತ್ತಿಕೊಂಡರೆ ಅದನ್ನು ಬಿಡಿಸಲು ತಾನು ಮಧ್ಯೆ ನುಗ್ಗುತ್ತಿದ್ದ. ಆಗ ಆ ಎರಡು ಕಡೆಯವರಿಂದಲೂ ಅವನಿಗೆ ಗುದ್ದು ಬೀಳುವ ಸಂಭವವಿತ್ತು. ಆದರೆ ಅವನು ಅದಕ್ಕೂ ಸಿದ್ಧ. ಅವನೂ ಅದರಲ್ಲಿ ಪಳಗಿದವನೇ ಆದ್ದರಿಂದ ಅವರ ಗುದ್ದನ್ನೂ ತಡೆದುಕೊಂಡು, ಪ್ರತಿಯಾಗಿ ನಾಲ್ಕು ಗುದ್ದು ಕೊಟ್ಟು ಅವರನ್ನು ಬಿಡಿಸುತ್ತಿದ್ದ.

ನರೇಂದ್ರ ತನ್ನ ಸ್ನೇಹಿತರನ್ನು ತುಂಬ ವಿಶ್ವಾಸದಿಂದ ನೋಡಿಕೊಳ್ಳುತ್ತಿದ್ದ. ಆಗಾಗ ಅವರ ನ್ನೆಲ್ಲ ಕರೆದುಕೊಂಡು ಹತ್ತಿರದ ಸ್ಥಳಗಳಿಗೆ ವನಭೋಜನಕ್ಕೆ ಹೊರಟುಬಿಡುತ್ತಿದ್ದ. ಒಂದು ದಿನ ಹೀಗೆ ಸುಮಾರು ಇಪ್ಪತ್ತು ಹುಡುಗರು ಸೇರಿಕೊಂಡು ಕಲ್ಕತ್ತದ ಐತಿಹಾಸಿಕ ಕೋಟೆಯನ್ನು ನೋಡಿಕೊಂಡುಬರಲು ಹೊರಟರು. ದಾರಿಯಲ್ಲಿ ಒಬ್ಬ ಹುಡುಗ ಇದ್ದಕ್ಕಿದ್ದಂತೆ ‘ಆಯಾಸ, ಸಂಕಟ’ ಎನ್ನುತ್ತ ಕುಳಿತುಬಿಟ್ಟ. ಇದನ್ನು ಕೇಳಿದ ಸಂಗಡಿಗರೆಲ್ಲ ಸುಮ್ಮನೆ ಅವನನ್ನು ಗೇಲಿ ಮಾಡುತ್ತ ಮುಂದೆ ನಡೆದೇಬಿಟ್ಟರು. ಬಹುಶಃ ಅವನು ಹೇಗಾದರೂ ಮಾಡಿ ಬಂದೇ ಬರುತ್ತಾನೆ ಎಂದು ಅವರೆಲ್ಲ ಭಾವಿಸಿರಬೇಕು. ಅದೂ ಅಲ್ಲದೆ ವಿಹಾರಕ್ಕೆ ಹೊರಟವರು ಒಬ್ಬನಿಗಾಗಿ ನಿಂತಾರೆಯೆ? ಆದರೆ ಅವರ ಜೊತೆಯಲ್ಲೇ ಸ್ವಲ್ಪ ದೂರ ಮುನ್ನಡೆದ ನರೇಂದ್ರ ಇದ್ದಕ್ಕಿದ್ದಂತೆ ನಿಂತುಬಿಟ್ಟ. “ಪಾಪ! ಅವನಿಗೆ ನಿಜವಾಗಿಯೂ ಏನೋ ತೊಂದರೆಯಾಗಿರಬೇಕು ಕಣ್ರೊ. ನೀವೆಲ್ಲ ಮುಂದೆ ನಡೆಯಿರಿ. ಆದರೆ ಅವನನ್ನು ನೋಡಿಕೊಳ್ಳಲು ಯಾರಾದರೂ ಹೋಗಲೇ ಬೇಕಲ್ಲವೆ? ಅದಕ್ಕೆ ನಾನೇ ಹೋಗುತ್ತೇನೆ” ಎಂದು ಸ್ನೇಹಿತರಿಗೆ ಹೇಳಿ ಆ ಹುಡುಗನ ಬಳಿಗೆ ಓಡಿದ. ಅಷ್ಟೊತ್ತಿಗೆ ಆ ಹುಡುಗನಿಗೆ ಜ್ವರ ಕೂಡ ಬಂದುಬಿಟ್ಟಿತ್ತು. ನರೇಂದ್ರ ಅವನನ್ನು ಮೇಲಕ್ಕೆತ್ತಿ, ಮೆಲ್ಲಗೆ ನಡೆಸಿಕೊಂಡು ಹತ್ತಿರವೇ ಇದ್ದ ಒಂದು ಜಟಕಾದಲ್ಲಿ ಅವನ ಮನೆಗೆ ಕರೆದುಕೊಂಡು ಹೋದ. ನರೇಂದ್ರನಿಗೆ ಅಂದಿನ ಪ್ರವಾಸದ ಖುಷಿ, ತಮಾಷೆ ಎಲ್ಲ ತಪ್ಪಿ ಹೋಯಿತು. ಆದರೆ ಅವನು ಈ ಬಗೆಯ ತ್ಯಾಗಕ್ಕೆ ಸದಾ ಸಿದ್ಧ. ಆದ್ದರಿಂದಲೇ ಆತ ತನ್ನ ಸ್ನೇಹಿತ ರಿಗೆ ಅಷ್ಟೊಂದು ಅಚ್ಚುಮೆಚ್ಚು.

ಸುಮಾರು ಇದೇ ಸಮಯದಲ್ಲಿ ಅವನು ಒಬ್ಬಳು ಹೆಂಗಸು ಹಾಗೂ ಆಕೆಯ ಮಗುವಿನ ಪ್ರಾಣಗಳನ್ನು ಉಳಿಸಿದ. ಆ ತಾಯಿ-ಮಕ್ಕಳು ಒಂದು ಸಾರೋಟಿಗೆ ಸಿಕ್ಕಿ ಅಪಾಯಕ್ಕೆ ಗುರಿಯಾಗು ವುದರಲ್ಲಿದ್ದರು. ಅದೃಷ್ಟಕ್ಕೆ ನರೇಂದ್ರ ಅದನ್ನು ಕಂಡ; ಕೂಡಲೇ ಓಡಿಹೋಗಿ ಆ ಮಗುವನ್ನು ಒಂದು ಕೈಯಲ್ಲಿ ಹಿಡಿದುಕೊಂಡು, ಇನ್ನೊಂದು ಕೈಯಿಂದ ಅದರ ತಾಯಿಯನ್ನು ಎಳೆದು, ಅಪಾಯದಿಂದ ಪಾರುಮಾಡಿದ. ಹೀಗೆ ತನ್ನ ಪ್ರಾಣದ ಹಂಗನ್ನೂ ತೊರೆದು ಇತರರನ್ನು ಅಪಾಯದ ಸುಳಿಯಿಂದ ತಪ್ಪಿಸುವ ಪರಹಿತದೃಷ್ಟಿಯನ್ನು ಅವನಲ್ಲಿ ಬಾಲ್ಯದಿಂದಲೇ ಕಾಣ ಬಹುದಾಗಿತ್ತು.

ನರೇಂದ್ರನಲ್ಲಿ ಎರಡು ವಿಶೇಷ ಸ್ವಭಾವಗಳು ಬೆಳೆದುಕೊಂಡು ಬರುತ್ತಿರವುದನ್ನು ಗಮನಿಸ ಬಹುದು. ಒಂದು–ತುಂಟತನ, ಚಟುವಟಿಕೆ, ಇನ್ನೊಂದು–ಕರುಣೆ. ಅವನ ಕರುಣೆಯ ಸ್ವಭಾವ ವೆನ್ನುವುದು ಬೆಳೆಯುತ್ತ ಬೆಳೆಯುತ್ತ ಪರಹಿತ, ಸಮಾಜಹಿತ, ರಾಷ್ಟ್ರಹಿತ, ಕೊನೆಗೆ ಲೋಕಹಿತ ಇವುಗಳನ್ನು ಸಾಧಿಸಲು ಕಾರಣವಾದರೆ, ಅವನ ತುಂಟತನದ ಸ್ವಭಾವವೆನ್ನುವುದು ಪ್ರಬಲ ಕ್ರಿಯಾತ್ಮಕ ಸಾಹಸವಾಗಿ ಪರಿಣಮಿಸಿ, ಬೃಹತ್ಕಾರ್ಯಗಳನ್ನು ಸಾಧಿಸಲು ಕಾರಣವಾಯಿತು. ಇವು ಗಳ ವೈಭವವನ್ನು ನಾವು ಮುಂದೆ ನೋಡಲಿದ್ದೇವೆ.

ಮೆಟ್ರೋಪಾಲಿಟನ್ ಶಾಲೆಗೆ ಸೇರಿದ ನರೇಂದ್ರನಿಗೆ, ತಾನು ಇಂಗ್ಲಿಷ್ ಭಾಷೆಯನ್ನು ಕಲಿಯಬೇಕಾಗುತ್ತದೆ ಎನ್ನುವುದು ತಿಳಿಯಿತು. ಇಂಗ್ಲಿಷ್ ಕಲಿಯುವುದೆಂದರೆ ಹೆಮ್ಮೆಯ ವಿಷಯ ವಲ್ಲವೆ? ಆದರೆ ಆಶ್ಚರ್ಯದ ಸಂಗತಿಯೆಂದರೆ, ಅದನ್ನು ಕಲಿಯಲು ಅವನಿಗೆ ಮಾತ್ರ ಇಷ್ಟವೇ ಇಲ್ಲ! “ಅದು ಪರಕೀಯರ ಭಾಷೆ, ನಾವೇಕೆ ಕಲಿಯಬೇಕು ಅದನ್ನು? ನಾವು ನಮ್ಮ ಮಾತೃಬಾಷೆ ಯನ್ನು ಮೊದಲು ಚೆನ್ನಾಗಿ ಕಲಿಯಬೇಡವೆ?” ಎನ್ನುತ್ತಾನವನು! ಹುಡುಗ ನರೇಂದ್ರ ಅಂದು ಆಡಿದ ಮಾತು ಎಷ್ಟು ನಿಜ, ಎಷ್ಟು ಅರ್ಥಪೂರ್ಣ! ಆದರೆ ಪಾಠಪಟ್ಟಿಯಲ್ಲಿರುವ ವಿಷಯವನ್ನು ತಾನೊಬ್ಬ ಕಲಿಯಲಾರೆ ಎಂದರೆ ನಡೆಯುತ್ತದೆಯೆ? ನರೇಂದ್ರ ಉಭಯಸಂಕಟಕ್ಕೆ ಸಿಲುಕಿದ. ಒಂದು ದಿನ ಅಳುತ್ತಾ ಮನೆಗೆ ಬಂದು, ತಾಯಿತಂದೆಯರ ಹತ್ತಿರ “ನನಗೆ ಇಂಗ್ಲಿಷ್ ಬೇಡ; ನಾನದನ್ನು ಕಲಿಯಲಾರೆ” ಎಂದ. ಅವರು, “ಇದೇನಪ್ಪ, ಹೀಗೆನ್ನುತ್ತಿದ್ದೀ? ಇಂಗ್ಲಿಷ್ ಕಲಿಯದಿದ್ದರೆ ಹೇಗೆ? ಮುಂದೆ ಕಾಲೇಜಿಗೆ ಹೋಗುವುದು ಬೇಡವೆ?” ಎಂದು ಬಗೆಬಗೆಯಾಗಿ ಹೇಳಿ ಪ್ರೋತ್ಸಾಹಿಸಿದರು. ಆದರೂ ಅವನು ಒಪ್ಪಲೇ ಇಲ್ಲ. ಈ ವಿಷಯವೆಲ್ಲ ಅವರ ವೃದ್ಧ ಸಂಬಂಧಿ ನೃಸಿಂಹದತ್ತನಿಗೆ ತಿಳಿಯಿತು. ಇವನಿಗೆ ನರೇಂದ್ರನ ಮೇಲೆ ವಿಶೇಷ ಮಮತೆ. ಇವನು ಹುಡುಗನನ್ನು ಬಳಿಯಲ್ಲಿ ಕುಳ್ಳಿರಿಸಿಕೊಂಡು, ವಿಶ್ವಾಸದಿಂದ ಮನವೊಲಿಸಿಕೊಳ್ಳುವ ಪ್ರಯತ್ನ ಮಾಡಿದ. ಆದರೆ ಅವನ ಮಾತೂ ಉಪಯೋಗವಾಗಲಿಲ್ಲ. ಇಷ್ಟಾದರೂ ಆತ ಬಿಡಲಿಲ್ಲ. ಹಲವಾರು ತಿಂಗಳವರೆಗೂ ಸಮಾಧಾನದಿಂದಲೇ ಹೇಳುತ್ತ ಬಂದ. ಕೊನೆಗೆ ಅವನನ್ನು ಗೆದ್ದು ಕೊಂಡೇಬಿಟ್ಟ ಮುದುಕ! ನರೇಂದ್ರ ಇಂಗ್ಲಿಷ್ ಕಲಿಯಲಾರಂಭಿಸಿದ. ಒಂದು ಗಮನಾರ್ಹ ಅಂಶವೇನೆಂದರೆ ಮೊದಲು ಅವನು ಅಷ್ಟು ಹಠ ಮಾಡಿದ್ದರೂ ಒಮ್ಮೆ ಒಪ್ಪಿಕೊಂಡು ಕಲಿಯತೊಡಗಿದ ಮೇಲೆ ಮಾತ್ರ ಅಲ್ಲಿ ಅರೆಮನಸ್ಸಿನ ಜಗ್ಗಾಟವಿರಲಿಲ್ಲ. ಪೂರ್ಣಶ್ರದ್ಧೆಯಿಂದ ಕಲಿತ. ಇದನ್ನು ಕಂಡು ಎಲ್ಲರಿಗೂ ಆಶ್ಚರ್ಯವಾಗದಿರಲಿಲ್ಲ. ಆದರೆ ಅವನು ಎಷ್ಟೇ ಹಠ ಮಾಡಿದರೂ ಇಂಗ್ಲಿಷನ್ನು ಕಲಿಯಲೇಬೇಕಾಗಿತ್ತು. ಏಕೆಂದರೆ ಮುಂದೆ ಅವನು ಅದೇ ಭಾಷೆ ಯಲ್ಲಿ ವಿಶ್ವವ್ಯಾಪಕವಾಗಿ ದಿವ್ಯಸಂದೇಶವನ್ನು ಮೊಳಗಬೇಕಾಗಿದೆಯಲ್ಲವೆ! ಇಡೀ ವಿಶ್ವದ ಆಧ್ಯಾತ್ಮಿಕ ಶಕ್ತಿಕೇಂದ್ರವೇ ತಾನಾಗಿ, ಘನ ತತ್ತ್ವಗಳನ್ನು ಧ್ವನಿಸಬೇಕಾಗಿರುವುದು ಆ ಪಾಶ್ಚಾತ್ಯ ಭಾಷೆಯಲ್ಲೇ ಅಲ್ಲವೆ?

ನರೇಂದ್ರ ಅಷ್ಟು ತುಂಟನಾದರೂ, ಸಾಧುಸಂನ್ಯಾಸಿಗಳನ್ನು ಕಂಡರೆ ಅವನಿಗೆ ಅಪಾರ ಭಕ್ತಿ, ಗೌರವ. ಅವರನ್ನು ನೋಡುವುದೆಂದರೇ ಅವನಿಗೊಂದು ವಿಶೇಷ ಆನಂದ. ಸಂನ್ಯಾಸಿಗಳೇ ನಾದರೂ ಮನೆಗೆ ಅತಿಥಿಗಳಾಗಿ ಬಂದರೆ ಅವರಿಗೇನು ಬೇಕೆಂಬುದನ್ನು ತಾನೇ ಕಂಡುಕೊಂಡು ಅಥವಾ ಕೇಳಿ ತಿಳಿದುಕೊಂಡು, ಅದನ್ನು ಒದಗಿಸಿಕೊಡುತ್ತಿದ್ದ. ಆ ಸಾಧುಗಳು ತುಂಬ ಸಂತುಷ್ಟ ರಾಗಿ ಅವನನ್ನು ಮನಸಾರೆ ಹರಸುತ್ತಿದ್ದರು. ಅವನ ತೇಜಸ್ವಿಯಾದ ಕಣ್ಣುಗಳೇ ಅವರಿಗೆ ಎಷ್ಟೋ ವಿಷಯಗಳನ್ನು ತಿಳಿಸಿಕೊಡುತ್ತಿದ್ದುವು. ಅವನಲ್ಲೊಬ್ಬ ಮಹಾತ್ಮನನ್ನೇ ಕಾಣುತ್ತಿದ್ದರು ಅವರು. ತಾನೂ ಸಂನ್ಯಾಸಿಯಾಗಬೇಕೆಂಬ ಆಸೆ ನರೇಂದ್ರನಲ್ಲಿ ಯಾವಾಗಲೂ ಇದ್ದೇ ಇತ್ತು. ಎಂದಿಗೆ ತಾನೂ ಸ್ವತಂತ್ರನಾಗಿ ಸಂನ್ಯಾಸಜೀವನವನ್ನು ಕೈಗೊಂಡೇನೋ ಎಂಬ ಆಲೋಚನೆ ಅವನ ಮನಸ್ಸಿನಲ್ಲಿ ಆಗಾಗ ಏಳುತ್ತಲೇ ಇತ್ತು. ತನ್ನ ಶಾಲೆಗೆ ಯಾರಾದರೊಬ್ಬ ಹೊಸ ಹುಡುಗ ಬಂದು ಸೇರಿಕೊಂಡರೆ ನರೇಂದ್ರ ಅವನ ಬಳಿಗೆ ಹೋಗಿ ಮಾತನಾಡಿಸುತ್ತ, “ನಿಮ್ಮ ಮನೆಯಲ್ಲಿ ಯಾರಾದರೂ ಸಂನ್ಯಾಸಿಗಳಾಗಿ ಹೋದವರಿದ್ದಾರೆಯೇ? ನಿಮ್ಮ ತಾತನೇನಾದರೂ ಸಂನ್ಯಾಸಿ ಯಾಗಿದ್ದರೆ?” ಎಂದು ಕೇಳುತ್ತಿದ್ದ. ಅದಕ್ಕೆ ಆ ಹುಡುಗ ಒಂದು ವೇಳೆ “ಹೌದು” ಎಂದುಬಿಟ್ಟರೆ ನರೇಂದ್ರನಿಗೆ ಅವನ ಮೇಲೆ ಒಂದು ವಿಶೇಷ ಆದರ, ಅಭಿಮಾನ! ಕೆಲಮೊಮ್ಮೆ ಹುಡುಗರೆಲ್ಲ ಸುತ್ತುಗಟ್ಟಿ ಕುಳಿತುಕೊಂಡು, ಪರಸ್ಪರ ಹಸ್ತಸಾಮುದ್ರಿಕ ನೋಡುವ ಆಟವನ್ನು ಆಡುತ್ತಿದ್ದರು– ಇವನ ಕೈನೋಡಿ ಅವನು ಅವನ ಕೈನೋಡಿ ಇವನು ಭವಿಷ್ಯ ಹೇಳುವುದು. ಇಲ್ಲಿಯೂ ನರೇಂದ್ರನೇ ಮುಖಂಡ. ಅವನು ತನ್ನ ಕೈಯನ್ನೇ ತೋರಿಸುತ್ತ ಹೇಳುತ್ತಿದ್ದ: “ಇಲ್ಲಿ ನೋಡಿ, ನನ್ನ ಕೈಯಲ್ಲಿ ಸಂನ್ಯಾಸರೇಖೆ ಇದೆ ಗೊತ್ತಾ! ಇಲ್ಲಿ ಕಾಣುತ್ತಲ್ಲ, ಇದೇ ಸಂನ್ಯಾಸರೇಖೆ. ನಾನು ಮುಂದೆ ಸಂನ್ಯಾಸಿಯಾಗುವುದರಲ್ಲಿ ಸಂಶಯವೇ ಇಲ್ಲ.” ಅವನ ಕೈಯಲ್ಲಿರುವ ಯಾವುದೋ ಒಂದು ರೇಖೆ ಸಂನ್ಯಾಸರೇಖೆಯಂತೆ! ಯಾವನೋ ಒಬ್ಬ ಮುದುಕಪ್ಪ ಅವನಿಗೆ ಅದನ್ನು ತೋರಿಸಿಕೊಟ್ಟನಂತೆ. ಯಾವಾಗ ನರೇಂದ್ರನ ಕೈಯಲ್ಲಿ ಸಂನ್ಯಾಸ ರೇಖೆ ಇದೆಯೆಂದು ಗೊತ್ತಾ ಯಿತೋ, ಇತರ ಹುಡುಗರು ತಮ್ಮ ಕೈಯಲ್ಲೂ ಅಂಥದೇನಾದರೂ ಇದೆಯೋ ಹೇಗೆ ಎಂದು ನೋಡಿಕೊಳ್ಳಲು ಹೊರಟರು. ಈಗ ಅವರಿಗೆಲ್ಲ ತಾವೂ ಸಂನ್ಯಾಸಿಗಳಾಗಬೇಕೆಂದು ಆಸೆಯಾಗಿ ಬಿಟ್ಟಿದೆ. ಏಕೆಂದರೆ ನರೇಂದ್ರ ಸಂನ್ಯಾಸಿಯಾಗುತ್ತಾನಂತಲ್ಲ! ಆದರೆ ಆ ಯೋಗ ಅವರ ಕೈಯಲ್ಲೂ ಬರೆದಿಲ್ಲ, ಹಣೆಯಲ್ಲೂ ಬರೆದಿಲ್ಲ.

ನರೇಂದ್ರ ಬಾಲ್ಯದಿಂದಲೂ ಧೀರತೆಯ ಮೂರ್ತಿ. ಇದಕ್ಕೊಂದು ನಿದರ್ಶನ ಈ ಮುಂದಿನ ಘಟನೆ. ಬೇಸರವಾದಾಗಲೆಲ್ಲ ಅವನು ಪಕ್ಕದಲ್ಲಿರುವ ತನ್ನ ಸ್ನೇಹಿತನ ಮನೆಗೆ ಆಡಲು ಹೋಗು ತ್ತಿದ್ದ. ಈ ಮನೆಯ ಕಾಂಪೌಡಿನಲ್ಲಿ ಒಂದು ದೊಡ್ಡ ಸಂಪಿಗೆ ಮರವಿತ್ತು. ಈ ಮರವೆಂದರೆ ನರೇಂದ್ರನಿಗೆ ತುಂಬ ಇಷ್ಟ. ಏಕೆಂದರೆ ಅದರ ಮೇಲೆ ಹತ್ತಿ, ಕೊಂಬೆಯಿಂದ ಕೊಂಬೆಗೆ ಜಿಗಿಯಬಹುದು; ಮರದ ಕೆಳಗೆ ದಟ್ಟವಾದ ನೆರಳಿರುವುದರಿಂದ ಮಧ್ಯಾಹ್ನದ ಬಿಸಿಲಿನಲ್ಲೂ ಅಲ್ಲಿ ಆಡಬಹುದು; ಇದೆಲ್ಲಕ್ಕಿಂತ ಹೆಚ್ಚಾಗಿ ಮರದ ಕೊಂಬೆಗೆ ಕಾಲು ತಗುಲಿಸಿಕೊಂಡು ತಲೆ ಕೆಳಗಾಗಿ ಜೋತಾಡಬಹುದು! ಈ ಆಟ ನರೇಂದ್ರನಿಗೆ ಅಚ್ಚುಮೆಚ್ಚು. ಅಲ್ಲದೆ ಚಂಪಕಪುಷ್ಪ ಶಿವನಿಗೆ ಪ್ರಿಯವಲ್ಲವೆ? ನರೇಂದ್ರನಿಗೂ ಅಷ್ಟೆ! ಅವನೂ ಅವನ ಸ್ನೇಹಿತರೂ ಆಗಾಗ ಇಲ್ಲಿ ಮರಕೋತಿಯಾಟ ಆಡುತ್ತಿದ್ದರು. ಆ ಮನೆಯಲ್ಲೊಬ್ಬ ತಾತಯ್ಯ. ಅವನಿಗೆ ಈ ಹುಡುಗರ ಗಲಾಟೆಯೆಲ್ಲ ಅಷ್ಟೇನೂ ಪ್ರಿಯವಲ್ಲ. ಆದರೇನು ಮಾಡುವುದು? ಅವರನ್ನು ಪಳಗಿಸುವ ಕೆಲಸ ಅಷ್ಟು ಸುಲಭವೆ? ಒಂದು ದಿನ ಹೀಗೇ ಹುಡುಗರೆಲ್ಲ ಸೇರಿದ್ದರು. ನರೇಂದ್ರ ಕೊಂಬೆಗೆ ಕಾಲು ತಗುಲಿಸಿಕೊಂಡು ಜೋತಾಡುತ್ತಿದ್ದ. ಆಗ ತಾತಾಯ್ಯ ಆಚೆಗೆ ಬಂದ. ಅವನದು ಮುದಿಕಣ್ಣು; ಸ್ವಷ್ಟವಾಗಿ ಕಾಣುತ್ತಿರಲಿಲ್ಲ. ಆದರೂ ಚೆನ್ನಾಗಿ ಅರಳಿಸಿಕೊಂಡು ನೋಡಿದ. ನೋಡಿದರೇ\eng{ss} ಅಷ್ಟೆತ್ತರದ ಕೊಂಬೆಯಲ್ಲಿ ನರೇಂದ್ರ ತಲೆಕೆಳಗಾಗಿ ಜೋತಾಡುತ್ತಿದ್ದಾನೆ! ಅವನಿಗೆ ಗಾಬರಿ ಯಾಯಿತು. ಈ ಹುಡುಗರು ಅಲ್ಲಿಂದ ಕೆಳಗೆ ಬಿದ್ದು ತಲೆಗಿಲೆ ಒಡೆದುಕೊಂಡರೇನಪ್ಪ ಗತಿ! ಅಲ್ಲದೆ ಇವರು ಹೂವುಗಳನ್ನೆಲ್ಲ ಉದುರಿಸಿಬಿಡುತ್ತಾರೆ... ತಾತಯ್ಯ ಕ್ಷಣಕಾಲ ಯೋಚಿಸಿದ. ಆಗ ಆತನ ತಲೆಗೆ ಹೊಳೆಯಿತು–ಮಕ್ಕಳನ್ನು ಓಡಿಸಲು ಒಳ್ಳೆಯ ಉಪಾಯ. ಮೆಲ್ಲಗೆ ಮರದ ಬಳಿಗೆ ಬಂದು, ನಡುಗುವ ಧ್ವನಿಯಲ್ಲಿ ಗಟ್ಟಿಯಾಗಿ ನುಡಿದ:

“ಅಯ್ಯಯ್ಯೋ! ಏ ಹುಡುಗ್ರಾ! ಈ ಮರವನ್ನು ಹತ್ತಬೇಡ್ರಪ್ಪ!”

“ಯಾಕೆ ತಾತ, ಈ ಮರವನ್ನು ಯಾಕೆ ಹತ್ತಬಾರದು?”–ನರೇಂದ್ರ ಕೇಳಿದ.

“ಯಾಕೆ ಅಂತ ಕೇಳ್ತೀಯೋ? ಈ ಮರದ ಮೇಲೊಂದು ದೊಡ್ಡ ಬ್ರಹ್ಮ ರಾಕ್ಷಸವಿದೆಯಪ್ಪ! ಅದು ಬಹಳ ಭಯಂಕರವಾದ್ದು. ಬಿಳೀ ಬಟ್ಟೆ ತೊಟ್ಟುಕೊಂಡು ರಾತ್ರಿಯಿಡೀ ಇಲ್ಲೆಲ್ಲ ಓಡಾ ಡ್ತಿರುತ್ತೆ!”

ನಿಜಕ್ಕೂ ಇದೊಂದು ವಿಚಿತ್ರ ಸಮಾಚಾರ! ನರೇಂದ್ರ ಇದನ್ನೆಲ್ಲ ಈ ಹಿಂದೆ ಕೇಳಿಯೇ ಇರಲಿಲ್ಲ. ಅವನಿಗೆ ಕುತೂಹಲ ಕೆರಳಿತು. 

“ಹೌದಾ ತಾತಾ? ಇನ್ನೇನ್ಮಾಡತ್ತೆ ತಾತ ಆ ಬ್ರಹ್ಮರಾಕ್ಷಸ?”

“ಏನ್ ಮಾಡತ್ತೆ ಗೊತ್ತಾ...? ಆ ಬ್ರಹ್ಮರಾಕ್ಷಸ ಈ ಮರ ಹತ್ತಿದವರ ಕತ್ತನ್ನ ಹೀಗ್ ಹಿಡಕೊಂಡು... ಹೀಗ್ ಹಿಡಕೊಂಡು... ಹೀಗ್ ಹಿಸುಕಿಹಾಕಿಬಿಡುತ್ತೆ!”

ಹುಡುಗರೆಲ್ಲ ತಾತಯ್ಯನ ಮಾತನ್ನು ಕಣ್ಣು ಕಿವಿತೆರೆದುಕೊಂಡು ಕೇಳುತ್ತಿದ್ದರು. ಯಾವಾಗ ಆತ, ಹೀಗೆ ಆ ಬ್ರಹ್ಮರಾಕ್ಷಸ ಮರ ಹತ್ತಿದವರ ಕತ್ತನ್ನು ಹಿಸುಕಿಹಾಕಿಬಿಡುತ್ತದೆ ಎಂದನೋ, ಎಲ್ಲರೂ ಅಲ್ಲಿಂದ ಪರಾರಿ! ನರೇಂದ್ರನೂ ದೂರಕ್ಕೆ ಹೊರಟು ಹೋದ. ತನ್ನ ಉಪಾಯ ಇಷ್ಟು ಸುಲಭವಾಗಿ ಫಲಿಸಿದ್ದನ್ನು ಕಂಡು ತಾತಯ್ಯ ವಿಜಯೋತ್ಸಾಹದಿಂದ ಬೊಚ್ಚುಬಾಯಿ ತೆರೆದುಕೊಂಡು ನಗುತ್ತ ಒಳಕ್ಕೆ ಹೋದ. ಆದರೆ ಅವನು ಆ ಕಡೆ ಹೋದನೋ ಇಲ್ಲವೋ, ಈ ಕಡೆ ನರೇಂದ್ರ ಮತ್ತೆ ಬಂದು ಮರ ಹತ್ತಿದ. ಅದನ್ನು ನೋಡಿ ಅವನ ಸ್ನೇಹಿತರು ದೂರದಿಂದಲೇ ಕೂಗಿಕೊಂಡರು: “ಏ ನರೇನ್! ಮರ ಹತ್ತಬೇಡವೋ! ಬ್ರಹ್ಮರಾಕ್ಷಸ ಹಿಡಿದುಕೊಳ್ಳುತ್ತೆ! ನಿನ್ನ ಕತ್ತನ್ನು ಹಿಸುಕಿಹಾಕಿಬಿಡುತ್ತೆ!” ನರೇಂದ್ರ ಅವರ ಅವಸ್ಥೆಯನ್ನು ಕಂಡು ಬಾಯ್ತುಂಬ ನಕ್ಕು ಹೇಳಿದ: “ಎಂಥಾ ಪುಕ್ಕಲ್ರಯ್ಯ ನೀವೆಲ್ಲ! ಯಾರೋ ಏನೋ ಹೇಳಿಬಿಟ್ಟರು ಅಂತ ನಂಬಿಬಿಡೋಣವೇ? ಆ ಬ್ರಹ್ಮರಾಕ್ಷಸ ಇಲ್ಲಿರುವುದು ನಿಜವಾಗಿದ್ದರೆ ನನ್ನ ಕತ್ತು ಹಾರಿ ಹೋಗಿ ಎಷ್ಟೋ ದಿನಗಳಾಗಬೇಕಿತ್ತು. ಆದರೆ ಇನ್ನೂ ಗಟ್ಟಿಯಾಗೇ ಇದೆ ನೋಡಿ!” ಆದರೆ ಹುಡುಗರಿ ಗೆಲ್ಲ ಮತ್ತೆ ಧೈರ್ಯ ಬರಬೇಕಾದರೆ ಎಷ್ಟೋ ಹೊತ್ತಾಯಿತೆನ್ನಿ.

ಮಕ್ಕಳನ್ನು ಹೆದರಿಸಲು ಈ ಬಗೆಯ ದೆವ್ವದ ಕಥೆಗಳನ್ನು ಹೇಳುವ ಪರಿಪಾಠವಿದೆ. ಅದನ್ನು ಕೇಳಿ ಮಕ್ಕಳು ಜೀವನದುದ್ದಕ್ಕೂ ಹೆದರಿಕೊಳ್ಳುವುದೂ ಇದೆ. ಆದರೆ ಇಲ್ಲಿ ಬಾಲಕ ನರೇಂದ್ರ ಕೊಟ್ಟ ಉತ್ತರ ಮಾತ್ರ ತುಂಬ ಅಪರೂಪದ್ದು. ನರೇಂದ್ರ ಹುಡುಗನಾಗಿರಬಹುದು; ಆದರೆ ಅವನಾಡಿದ ಮಾತಿನಲ್ಲಿ ಹುಡುಗುತನವಿಲ್ಲ. ಅಲ್ಲದೆ ತಾನು ಮುಂದೆ ಲೋಕವಿಖ್ಯಾತ ವಿವೇಕಾ ನಂದರಾಗಿ ಗುಡುಗಲಿರುವ ವಾಣಿಯನ್ನು ಅವನು ಇಂದೇ ಇಲ್ಲಿ ಮೊಳಗಿದ್ದಾನೆ ಎಂದರೂ ಆಶ್ಚರ್ಯವಿಲ್ಲ. ಮುಂದೆ ಸ್ವಾಮಿ ವಿವೇಕಾನಂದರು ಹೇಳುತ್ತಾರೆ: “ಒಂದು ವಿಷಯವನ್ನು ಯಾವುದೋ ಒಂದು ಪುಸ್ತಕದಲ್ಲಿ ಬರೆದಿದೆ, ಇಲ್ಲವೆ ಯಾವನೋ ಒಬ್ಬ ಮನುಷ್ಯ ಹಾಗೆಂದು ಹೇಳಿದ ಎಂಬಷ್ಟೇ ಕಾರಣಕ್ಕಾಗಿ ನಂಬಬೇಡಿ. ಅದರ ನಿಜತ್ವವನ್ನು ನೀವೇ ಕಂಡುಕೊಳ್ಳಿ. ಸತ್ಯಸಾಕ್ಷಾತ್ಕಾರವೆಂದರೆ ಅದೇ.”

ಅಂದು ನರೇಂದ್ರನನ್ನು ಹಾಗೆ ಹೆದರಿಸಿದರೂ ತಾತಯ್ಯನಿಗೆ ಅವನ ಮೇಲೆ ವಿಶೇಷ ಪ್ರೀತಿ. ಅಲ್ಲದೆ ನರೇಂದ್ರನಿಗೆ ಒಂದು ಒಳ್ಳೆಯ ಭವಿಷ್ಯವಿದೆ ಎಂಬುದನ್ನು ಅವನೂ ಕಂಡುಕೊಂಡಿದ್ದ. ಇನ್ನೊಂದು ದಿನ ನರೇಂದ್ರ ಅದೇ “ದೆವ್ವದ ಮರ”ದ ಮೇಲೆ ಹತ್ತಿ ಕೊಂಬೆಯಲ್ಲಿ ಜೋತಾಡು ತ್ತಿದ್ದಾಗ, ತಾತಯ್ಯ ಅವನನ್ನು ತನ್ನ ಮನೆಗೆ ಕರೆದುಕೊಂಡು ಹೋಗಿ ಕುಳ್ಳಿರಿಸಿಕೊಂಡು ಕೇಳಿದ: “ಏನಪ್ಪಾ ನೀನೇನು ಹೀಗೇ ದಿನವಿಡೀ ಮನೆಯಿಂದ ಮನೆಗೆ ಹೋಗುತ್ತ ಆಟವಾಡಿಕೊಂಡಿರು ತ್ತೀಯೋ ಅಥವಾ ಓದುಗೀದಿನ ಕಡೆಗೂ ಗಮನ ಕೊಡುತ್ತಿಯೋ?”

“ನಾನು ಆಟವನ್ನೂ ಆಡುತ್ತೇನೆ: ಪಾಠವನ್ನೂ ಓದುತ್ತೇನೆ”–ನರೇಂದ್ರನೆಂದ. ಆದರೆ ತಾತಯ್ಯ ಅಷ್ಟಕ್ಕೆ ಬಿಡಲಿಲ್ಲ. ಭೂಗೋಳದಲ್ಲಿ, ಗಣಿತದಲ್ಲಿ ಕೆಲವು ಪ್ರಶ್ನೆಗಳನ್ನು ಕೇಳಿದ. ನರೇಂದ್ರ ಸಲೀಸಾಗಿ ಉತ್ತರಿಸಿದ. ಅನಂತರ ತಾತಯ್ಯ ಪಾಠದ ಕೆಲವು ಪದ್ಯಗಳನ್ನು ಹೇಳುವಂತೆ ಕೇಳಿದ. ಅವನ್ನೂ ನರೇಂದ್ರ ಹಾಡಿ ತೋರಿಸಿದ. ಎಷ್ಟು ಕಷ್ಟದ ಪ್ರಶ್ನೆಗಳನ್ನು ಕೇಳಿದರೂ ಅವನು ಉತ್ತರಿಸಿದಾಗ ತಾತಯ್ಯನಿಗೆ ಬಹಳ ಆಶ್ಚರ್ಯವೂ ಆಯಿತು, ಸಂತೋಷವೂ ಆಯಿತು. ಅವನ ಉತ್ತರ, ಮುಖಭಾವ ಇವನ್ನೆಲ್ಲ ಗಮನಿಸಿದ ತಾತಯ್ಯ ಅವನಲ್ಲಿ ಏನೋ ಒಂದು ಬೆಳಕನ್ನು ಕಂಡ. ಮತ್ತು “ಮಗು, ನೀನೊಬ್ಬ ಧೀರನಾಗುವುದು ಖಂಡಿತ. ನಿನಗೆ ನನ್ನ ಸಂಪೂರ್ಣ ಆಶೀರ್ವಾದ ಇದೆ” ಎಂದು ಹೃತ್ಪೂರ್ವಕವಾಗಿ ಹರಸಿದ. ಮತ್ತು ಮುಂದೆ ಆ ಹಾರೈಕೆ ಫಲಿಸಿದ್ದನ್ನು ಕಣ್ಣಾರೆ ಕಾಣುವ ಭಾಗ್ಯವೂ ಅವನದಾಯಿತು.

ನರೇಂದ್ರನ ಧೈರ್ಯದೊಂದಿಗೆ ಅವನ ಬುದ್ಧಿಸಾಮರ್ಥ್ಯವನ್ನು ತೋರಿಸುವಂತಹ ಇನ್ನೊಂದು ಘಟನೆಯಿದೆ. ಅವನಿಗಾಗ ಹತ್ತು-ಹನ್ನೊಂದು ವರ್ಷ ವಯಸ್ಸು. ಆ ದಿನಗಳಲ್ಲಿ ಇಂಗ್ಲೆಂಡಿನ ಚಕ್ರವರ್ತಿಯಾದ ಏಳನೇ ಎಡ್ವರ್ಡ್ ಭಾರತಕ್ಕೆ ಬಂದಿದ್ದ. ಅವನು ತಂದಿದ್ದ ‘ಸಿರಾಪಿಸ್’ ಎಂಬ ಭಾರೀ ಯುದ್ಧದ ಹಡುಗು ಕಲ್ಕತ್ತದ ಬಂದರಿನಲ್ಲಿ ನಿಂತಿತ್ತು. ಅದನ್ನು ನೋಡಿಕೊಂಡು ಬರಬೇಕೆಂದು ನರೇಂದ್ರನ ಗೆಳೆಯರ ಆಸೆ. ಆದರೆ ಬಂದರಿಗೆ ಹೋಗಬೇಕಾದರೆ ಅನುಮತಿಪತ್ರ ಬೇಕು. ಅದಕ್ಕಾಗಿ ಹುಡುಗರೆಲ್ಲ ನರೇಂದ್ರನನ್ನು ಒತ್ತಾಯಪಡಿಸಿದರು. ಇಂಥ ಕೆಲಸಕ್ಕೆ ಅವನೇ ಸರಿ. ನರೇಂದ್ರ ಒಪ್ಪಿಕೊಂಡ. ಈ ಅನುಮತಿ ಪತ್ರವನ್ನು ಆಫೀಸಿನಲ್ಲಿ ‘ಬಡಾಸಾಹೇಬ’ರನ್ನು ಕಂಡೇ ಪಡೆಯಬೇಕಾಗಿತ್ತು. ಅಲ್ಲದೆ, ನಗರದ ಗಣ್ಯ ವ್ಯಕ್ತಿಗಳಿಗೆ ಮಾತ್ರ ಅನುಮತಿಪತ್ರಗಳು ಲಭ್ಯವಿದ್ದುವು–ಯಾರೆಂದರವರಿಗಲ್ಲ. ಆದರೂ ನರೇಂದ್ರ ತನ್ನ ಹಾಗೂ ಸ್ನೇಹಿತರ ಅರ್ಜಿಗಳನ್ನು ತೆಗೆದುಕೊಂಡು ಆಫೀಸಿಗೆ ಹೋದ. ಆದರೆ ಬಾಗಿಲಲ್ಲಿ ಭರ್ಜರಿ ಸಮವಸ್ತ್ರ ಧರಿಸಿದ್ದ ದ್ವಾರಪಾಲಕನೊಬ್ಬ ನಿಂತಿದ್ದ. ನರೇಂದ್ರನನ್ನು ಕಂಡು ಅವನು, “ಏಯ್, ಚೋಟುದ್ದ ಇದ್ದೀಯ, ನಿನಗ್ಯಾರೋ ಕೊಡ್ತಾರೆ ಅನುಮತಿಯನ್ನು? ನಡಿ ನಡಿ ಇಲ್ಲಿಂದ” ಎಂದು ಗದರಿಸಿ ಅಟ್ಟಿಬಿಟ್ಟ. ನರೇಂದ್ರನಿಗೆ ತುಂಬ ದುಃಖವಾಯಿತು. ಆದರೂ ಎದೆಗುಂದಲಿಲ್ಲ. ಆ ಕಟ್ಟಡವನ್ನೊಮ್ಮೆ ಪೂರ್ತಿ ಸುತ್ತುಹಾಕಿ ನೋಡಿದ. ಹಿಂಭಾಗದಲ್ಲಿ ಒಂದು ಕಿರಿದಾದ ಮೆಟ್ಟಿಲ ಸಾಲಿರುವುದು ಗೋಚರವಾಯಿತು. ಅವು ಜವಾನರು ಮಾತ್ರ ಉಪಯೋಗಿಸುತ್ತಿದ್ದ ಮೆಟ್ಟಿಲು ಗಳು, ಆ ಮಾರ್ಗವಾಗಿ ಹೋದರೆ ಬಡಾ ಸಾಹೇಬರ ಆಫೀಸಿಗೆ ಹೋಗಬಹುದೆಂದು ತೋರಿತು. ಸಿಕ್ಕಿಹಾಕಿಕೊಂಡರೆ ಮಾತ್ರ ತುಂಬ ಅವಮಾನ! ಆದರೂ ಧೈರ್ಯ ಮಾಡಿ ಮೆಲ್ಲನೆ ಮೆಟ್ಟಿ ಲೇರಿದ. ಅಲ್ಲಿ ಒಂದು ಬಾಗಿಲ ಮುಂದಿನ ಪರದೆ ಸರಿಸಿ ನೋಡುತ್ತಾನೆ, ಅದೇ ಆಫೀಸು. ಬಡಾ ಸಾಹೇಬರು ಕುಳಿತಿದ್ದರು. ನರೇಂದ್ರ ಜನರ ಸಾಲಿನಲ್ಲಿ ಸೇರಿಕೊಂಡ. ತನ್ನ ಸರದಿ ಬಂದಾಗ ತನ್ನ ಹಾಗೂ ಸ್ನೇಹಿತರ ಅರ್ಜಿಗಳನ್ನು ಸಾಹೇಬರ ಮುಂದಿಟ್ಟ. ಅವನ ಅದೃಷ್ಟ. ಸಾಹೇಬರು ತಲೆಯೆತ್ತಿ ನೋಡದೆ ಎಲ್ಲಕ್ಕೂ ರುಜು ಹಾಕಿಕೊಟ್ಟರು. ಹಿಂದಿರುಗುವಾಗ ನರೇಂದ್ರ ಆ ಪತ್ರವನ್ನು ಕೈಯಲ್ಲಿ ಹಿಡಿದು ಒಳ್ಳೇ ಜಂಬದಿಂದ ಅದೇ ದ್ವಾರಪಾಲಕನ ಮುಂದಿನಿಂದ ಬಂದ. ಅವನನ್ನು ಕಂಡು ದ್ವಾರಪಾಲಕ ಕಣ್ಕಣ್ಣು ಬಿಡುತ್ತ, “ನಿನಗೆ ಹೇಗೆ ಸಿಕ್ಕಿತೋ ಅನುಮತಿ ಪತ್ರ?” ಎಂದು ಕೇಳಿದ. ಅದಕ್ಕೆ ನರೇಂದ್ರ ತುಂಟನಗೆ ಬೀರುತ್ತ “ನಾನೊಬ್ಬ ಜಾದುಗಾರ ಗೊತ್ತಾ!” ಎಂದು ಹೇಳಿ ಠೀವಿಯಿಂದ ಮುನ್ನಡೆದ.

ನರೇಂದ್ರ ಆಟದಲ್ಲೂ ಮುಂದು, ಪಾಠದಲ್ಲೂ ಮುಂದು. ಅವನು ತನ್ನ ಪ್ರತಿ ದಿನದ ಪಾಠಗಳನ್ನು ತುಂಬ ಉತ್ಸಾಹದಿಂದ ಅಭ್ಯಾಸ ಮಾಡುತ್ತಿದ್ದ. ಕೇವಲ ಒಂದು ಗಂಟೆ ಸಾಕವನಿಗೆ, ಪುಸ್ತಕದ ಪಾಠಗಳನ್ನೆಲ್ಲ ಮಸ್ತಕಕ್ಕಿಳಿಸಿಬಿಡುತ್ತಿದ್ದ. ಆದ್ದರಿಂದಲೇ ತನ್ನ ಪಾಠಗಳನ್ನು ಅಧ್ಯಯನ ಮಾಡಿದ ಮೇಲೂ ಅವನಿಗೆ ಚೇಷ್ಟೆ ತಮಾಷೆ ತುಂಟತನಗಳಿಗೆ ತುಂಬ ಸಮಯ ಮಿಗುತ್ತಿತ್ತು. ಆಗಿನಿಂದಲೂ ಅವನಿಗೆ ಇಂಗ್ಲಿಷ್ ಭಾಷೆಯಲ್ಲಿ, ಇತಿಹಾಸದಲ್ಲಿ ಹಾಗೂ ಸಂಸ್ಕೃತದಲ್ಲಿ ವಿಶೇಷ ಆಸಕ್ತಿ. ಆದರೆ ಅವನಿಗೆ ಲೆಕ್ಕವೆಂದರಾಗದು. ಈ ವಿಷಯದಲ್ಲಿ ಅವನು ತನ್ನ ತಂದೆಯಂತೆ.

ಸಾಮಾನ್ಯವಾಗಿ ನರೇಂದ್ರನ ದೇಹಾರೋಗ್ಯ ಚೆನ್ನಾಗಿಯೇ ಇತ್ತಾದರೂ ಈ ಕೆಲವು ವರ್ಷ ಗಳಲ್ಲಿ ಅವನು ಅಗ್ನಿಮಾಂದ್ಯದಿಂದಾಗಿ–ಎಂದರೆ ಅಜೀರ್ಣ ರೋಗದಿಂದಾಗಿ ತುಂಬ ಸಣ್ಣಗಾಗಿ ಬಿಟ್ಟ. ಆದರೆ ತನ್ನ ಮೈಗೆ ಯಾವ ತಿಂಡಿ ಒಗ್ಗುವುದಿಲ್ಲವೋ ಅದನ್ನೇ ತಿನ್ನಬೇಕೆಂದು ಆಸೆಯಾಗು ತ್ತಿತ್ತು–ಎಲ್ಲ ಹುಡುಗರಂತೆ! ಆದ್ದರಿಂದ ಅವನ ತಾಯಿಗೆ ಈಗೊಂದು ಹೆಚ್ಚಿನ ಕೆಲಸ–ಆ ತಿಂಡಿಗಳನ್ನು ಅವನ ಕೈಗೆ ಎಟುಕದಂತೆ ಇಟ್ಟಿರುವುದು! ಆದರೆ ಹೀಗೆ ಅಜೀರ್ಣದಿಂದಾಗಿ ಸೊರಗಿದರೂ ಅವನೆಂದೂ ಜೋಲು ಮೋರೆ ಹಾಕಿಕೊಂಡಿರುತ್ತಿರಲಿಲ್ಲ. ಯಾವಾಗಲೂ ಗೆಲು ವಾಗಿರುತ್ತಿದ್ದ. ಅದೇ ಶಕ್ತಿ, ಅದೇ ಉತ್ಸಾಹ, ಅದೇ ತಮಾಷೆ, ಅದೇ ಎದೆಗಾರಿಕೆ–ಯಾವುದೂ ಕಡಿಮೆಯಾಗಿರಲಿಲ್ಲ.

ನರೇಂದ್ರ ತುಂಬ ಚೆನ್ನಾಗಿ ಚಿತ್ರ ಬರೆಯಬಲ್ಲವನಾಗಿದ್ದ. ಮತ್ತು ಅತ್ಯಂತ ಮಧುರವಾಗಿ ಹಾಡಬಲ್ಲವನಾಗಿದ್ದ. ಸಂಗೀತ ಕಛೇರಿಯಲ್ಲಿ ಹಾಡುಗಳನ್ನು ಒಂದು ಸಲ ಕೇಳಿದರೆ ಸಾಕು, ಎರಡನೆಯ ಸಲ ಅದು ಅವನ ಬಾಯಲ್ಲಿರುತ್ತಿತ್ತು.

ಅವನು ತನ್ನ ತರಗತಿಯ ಹುಡುಗರೊಂದಿಗೆ ತುಂಬ ಆತ್ಮೀಯವಾಗಿ ಬೆರೆಯುತ್ತಿದ್ದ. ಒಳ್ಳೊಳ್ಳೆಯ ಕಥೆಗಳನ್ನು ಹೇಳುತ್ತ ಅವರನ್ನು ರಮಿಸುತ್ತಿದ್ದ; ಹಾಡುಗಳನ್ನು ಹೇಳಿ ಸಂತೋಷ ಪಡಿಸುತ್ತಿದ್ದ; ತರತರದ ಅಣುಕುಚೇಷ್ಟೆಗಳನ್ನು ಮಾಡಿ ಹೊಟ್ಟೆ ಹುಣ್ಣಾಗುವಷ್ಟು ನಗಿಸುತ್ತಿದ್ದ. ಎಷ್ಟೋ ಸಲ ಅವನು ಕೀಟಲೆ ಮಾಡಿ ಹುಡುಗರನ್ನು ರೇಗಿಸುತ್ತಿದ್ದುದೂ ಉಂಟು. ಆದರೆ ಯಾರ ಮನಸ್ಸಿಗೂ ನೋವಾಗುವಂತೆ ಮಾತನಾಡುತ್ತಿರಲಿಲ್ಲ. ಹೀಗೆ ಅವನು ತನ್ನ ಸಂಗಡಿಗರನ್ನೆಲ್ಲ ತನ್ನೆಡೆಗೆ ಆಕರ್ಷಿಸಿಬಿಟ್ಟಿದ್ದ. ಎಂಥವರನ್ನೇ ಆಗಲಿ ಕೇವಲ ಐದೇ ನಿಮಿಷಗಳಲ್ಲಿ ಗೆದ್ದು ತನ್ನವರನ್ನಾಗಿಸಿಕೊಳ್ಳುವ ಸಾಮರ್ಥ್ಯ ಅವನಿಗಿತ್ತು. ಮಾತಿನಲ್ಲಿ ಅವನನ್ನು ಮೀರಿಸುವವರು ಯಾರೂ ಇರಲಿಲ್ಲ. ಅವನು ಒಳ್ಳೇ ಪ್ರತ್ಯುತ್ಪನ್ನಮತಿ; ಆಯಾ ಸನ್ನಿವೇಶಕ್ಕೆ ತಕ್ಕಂತೆ ಸೂಕ್ತವಾಗಿ ಮಾತನಾಡಬಲ್ಲ ಚತುರ. ಒಂದು ದಿನವಾದರೂ ಸಪ್ಪೆಮೋರೆ ಹಾಕಿಕೊಂಡು ಕುಳಿತವನಲ್ಲ. ತಾನೂ ನಕ್ಕು ಇತರರನ್ನೂ ನಗಿಸುತ್ತಿರುವ ನಿತ್ಯಾನಂದಮೂರ್ತಿ ಅವನು. ಅವನಲ್ಲಿ ಅಪರಿಮಿತ ಶಕ್ತಿ ತೇಜಸ್ಸು ಬಲ ಅಡಗಿಕೊಂಡಿತ್ತು. ಈ ವಿಷಯವಾಗಿ ಮುಂದೆ ಸ್ವಾಮಿ ವಿವೇಕಾನಂದರೇ ತಮ್ಮೊಬ್ಬ ಶಿಷ್ಯನ ಮುಂದೆ ಹೇಳುತ್ತಾರೆ: “ನೋಡು, ನನ್ನ ಬಾಲ್ಯದಲ್ಲಿ ನನ್ನೊಳಗೆ ಒಂದು ಅಪ್ರತಿಹತವಾದ ಶಕ್ತಿ ಉಕ್ಕುತ್ತಿರುವುದನ್ನು ನಾನು ಗಮನಿಸುತ್ತಿದ್ದೆ. ಆ ಶಕ್ತಿ ನನ್ನ ಶರೀರವನ್ನೆಲ್ಲ ತುಂಬಿ ಉಕ್ಕಿ ಹರಿಯುವಂತಿರುತ್ತಿತ್ತು. ಇದರಿಂದಾಗಿ ನಾನು ಸಮ್ಮನೆ ಚಡಪಡಿಸುತ್ತಿದ್ದೆ, ಮತ್ತು ಸುಮ್ಮನೆ ಕುಳಿತಿರಲು ನನ್ನಿಂದ ಸಾಧ್ಯವೇ ಆಗುತ್ತಿರಲಿಲ್ಲ. ಆದ್ದರಿಂದ ನಾನು ಒಂದಲ್ಲ ಒಂದು ಚೇಷ್ಟೆ ಮಾಡುತ್ತಲೇ ಇರುತ್ತಿದ್ದೆ. ಓದಿಬರೆಯುವ ಕೆಲಸವೇನೂ ಇಲ್ಲದೆ ಹೋದರೆ, ತುಂಟತನ ಮಾಡಲು ಹೊರಟುಬಿಡುತ್ತಿದ್ದೆ. ನನ್ನ ಅಂತರಂಗ ಸದಾ ಸ್ಪಂದಿಸುತ್ತ, ನಾನು ಏನನ್ನಾದರೂ ಮಾಡುತ್ತಿರುವಂತೆ ನನ್ನಲ್ಲಿ ಚಡಪಡಿಕೆಯನ್ನುಂಟುಮಾಡುತ್ತಿತ್ತು. ನನ್ನನ್ನೇನಾದರೂ ಒಂದು ಮೂರ್ನಾಲ್ಕು ದಿನ ಸುಮ್ಮನೆ ಕೂಡಿಸಿಬಿಟ್ಟರೆ ಒಂದೋ ನನಗೆ ಕಾಯಿಲೆಯಾಗಿಬಿಡುತ್ತಿತ್ತು, ಇಲ್ಲವೆ ಬುದ್ಧಿ ಕೆಟ್ಟಂತಾಗಿಬಿಡುತ್ತಿತ್ತು.”

ನರೇಂದ್ರ ಯಾವಾಗಲೂ ಇಷ್ಟೊಂದು ಚಟುವಟಿಕೆಯಿಂದಿರುತ್ತಿದ್ದುದೇನೋ ನಿಜ. ಆದರೆ ಯಾವಾಗಲಾದರೊಮ್ಮೆ ಅವನಲ್ಲಿ ಯಾವುದೋ ಒಂದು ವಿಚಿತ್ರ ಭಾವ ಎದ್ದೇಳುತ್ತಿತ್ತು. ಆಗ ಅವನ ಮನಸ್ಸು ಎಲ್ಲಿಗೋ ಬಹುದೂರ ಹೋಗಿದ್ದಂತೆ ಕಂಡು ಬರುತ್ತಿತ್ತು. ಸುಮ್ಮನೆ ಆಕಾಶ ವನ್ನು ನೆಟ್ಟನೋಟದಿಂದ ನಿಟ್ಟಿಸುತ್ತ ಕುಳಿತುಬಿಡುತ್ತಿದ್ದ. ಅವನ ಮನಸ್ಸು ತುಂಬ ಗಂಭೀರ ವಾಗಿರುತ್ತಿತ್ತು. ಅವನ ನಗು ತಮಾಷೆಗಳೆಲ್ಲ ಮಾಯವಾಗಿಬಿಟ್ಟಿರುತ್ತಿದ್ದುವು. ಕೆಲವೊಮ್ಮೆ ಈ ಬಗೆಯ ಸ್ಥಿತಿಯಿಂದ ಮೈತಿಳಿದು, “ನಾನು ರಾಜನಾಗುತ್ತೇನೆ; ನಾನು ಇದನ್ನು ಮಾಡುತ್ತೇನೆ, ನಾನು ಅದನ್ನು ಮಾಡುತ್ತೇನೆ” ಎಂದು ಏನೇನೋ ಹೇಳುತ್ತಿದ್ದ. ಇನ್ನು ಕೆಲವೊಮ್ಮೆ, “ಇದನ್ನು ಹಾಗೆ ಮಾಡಬೇಕು, ಈ ಸಲ ಇದನ್ನು ಮಾಡಿಬಿಡಬೇಕು” ಎನ್ನುತ್ತಿದ್ದ. ಅವನ ಆ ಮಾತಿನ ಅಭಿಪ್ರಾಯವೇನೆಂಬುದು ಮಾತ್ರ ಯಾರಿಗೂ ಅರ್ಥವಾಗುತ್ತಿರಲಿಲ್ಲ. ಏಕೆಂದರೆ ಅವನ ಈ ಮಾತಿಗೂ, ಸ್ವಲ್ಪಹೊತ್ತಿನ ಹಿಂದೆ ಅವನು ಸಹಜ ಸ್ಥಿತಿಯಲ್ಲಿದ್ದಾಗ ಆಡುತ್ತಿದ್ದ ಮಾತುಗಳಿಗೂ ಸಂಬಂಧವೇ ಇರುತ್ತಿರಲಿಲ್ಲ. ಕೆಲವು ಸಂದರ್ಭಗಳಲ್ಲಿ ಅವನು ತನ್ನಷ್ಟಕ್ಕೆ ತಾನು ಗಟ್ಟಿಯಾಗಿ ಮಾತನಾಡಿಕೊಳ್ಳುವುದೂ ಇತ್ತು. ಆಗ ಅವನೊಬ್ಬ ಬೇರೆಯೇ ಹುಡುಗನೋ ಎಂಬಂತೆ ಕಂಡುಬರುತ್ತಿದ್ದ. ಅವನ ಈ ಬಗೆಯ ಭಾವ ಕಳೆದು ಎಂದಿನಂತಾದಾಗ, “ನನಗೀಗ ಏನಾಗಿತ್ತು!” ಎಂಬಂತೆ ಕಣ್ಕಣ್ಣು ಬಿಡುತ್ತಿದ್ದ; ಪೇಚಿಗೆ ಸಿಕ್ಕಿಕೊಂಡವರಂತೆ ಮುಖ ಮಾಡಿಕೊಳ್ಳುತ್ತಿದ್ದ. ಆದರೆ ಇವೆಲ್ಲ ಮುಗಿದ ಬಳಿಕ ಅವನು ಎಂದಿನ ತುಂಟ ನರೇಂದ್ರನೇ ಆಗಿಬಿಡುತ್ತಿದ್ದ. ಇವನ್ನೆಲ್ಲ ಕಂಡು ಅವನ ಸ್ನೇಹಿತರು ಹೇಳುತ್ತಿದ್ದರು, “ನರೇನ್ ತುಂಬ ಒಳ್ಳೆಯವನು; ನಗುತ್ತಾನೆ, ನಗಿಸುತ್ತಾನೆ, ತಮಾಷೆ ಮಾಡುತ್ತಾನೆ–ಎಲ್ಲ ಸರಿ. ಆದರೆ ಕೆಲವು ಸಲ ಈ ತರಹ ಹುಚ್ಚುಚ್ಚಾ ಗಾಡುತ್ತಾನೆ. ಒಂದೊಂದು ಸಲವೆಂತೂ ಅವನು ಏನು ಮಾತನಾಡಿಯಾನು ಅಂತಲೇ ಹೇಳಲಾಗು ವುದಿಲ್ಲ!” ನಿಜ, ನರೇಂದ್ರನ ಭಾವದಂತರಂಗವನ್ನು ಯಾರುತಾನೆ ಬಲ್ಲರು?

ನರೇಂದ್ರ ಯಾವಾಗಲೂ ನಗುನಗುತ್ತ ಇರುತ್ತಿದ್ದವನು ಎನ್ನುವುದು ನಿಜವೇ. ಆದರೆ ಅವನಿಗೂ ಕಷ್ಟದ ಪರಿಸ್ಥಿತಿಗಳು ಬರಲಿಲ್ಲವೆಂದೇನಲ್ಲ. ಒಂದು ದಿನ ಅವನ ಶಾಲೆಯಲ್ಲೇ ಅವನಿಗೊಂದು ಕಷ್ಟ ಒದಗಿಬಂತು. ಆಗಿನ ಕಾಲದಲ್ಲಿ ಅಧ್ಯಾಪಕರು, ತಪ್ಪು ಮಾಡಿದ ಹುಡುಗರಿಗೆ ಬೆತ್ತದೇಟು ಕೊಡುವುದು, ಕಿವಿ ಹಿಂಡುವುದು, ಒಳಶುಂಠಿ ಕೊಡುವುದು ಇತ್ಯಾದಿ ಶಿಕ್ಷೆಗಳನ್ನು ವಿಧಿಸುವ ಪರಿಪಾಠ ಹೆಚ್ಚಾಗಿಯೇ ಇತ್ತು. ನರೇಂದ್ರನ ತರಗತಿಯಲ್ಲಿ ಒಬ್ಬ ಹುಡುಗನಿದ್ದ. ಅವನ ಮುಖ ಚಹರೆ, ನಡಿಗೆ, ಸ್ವಭಾವ ಎಲ್ಲ ಸ್ವಲ್ಪ ವಿಚಿತ್ರ. ಆದ್ದರಿಂದ ಎಲ್ಲರ ಹಾಸ್ಯಕ್ಕೊಂದು ವಸ್ತುವಾಗಿದ್ದ. ಅವನು ಆ ದಿನ ಏನೋ ತಂಟೆ ಮಾಡಿರಬೇಕು; ಅಧ್ಯಾಪಕರಿಗೆ ರೇಗಿತು. ಅವರೋ ಮೊದಲೇ ಕೆಟ್ಟ ಕೋಪಿಷ್ಟರು. ಸರಿ, ಆ ಹುಡುಗನನ್ನು ಒಂದೇ ಸಮನೆ ಬೆತ್ತದಿಂದ ಥಳಿಸತೊಡಗಿದರು. ಆ ದೃಶ್ಯ ಭಯಂಕರವಾಗಿತ್ತು. ಆದರೆ ಅಷ್ಟೇ ತಮಾಷೆಯೂ ಆಗಿತ್ತು. ಯಾರನ್ನಾದರೂ ಹಾಗೆ ಹೊಡೆಯುತ್ತಿದ್ದರೆ ಹುಡುಗರಿಗೆ ಭಯವಾಗದಿರುತ್ತದೆಯೆ? ಆದರೆ ಆ ತಂಟೆಮಾರಿ ಹುಡುಗನ ರೂಪ ಈ ಉಪಾಧ್ಯಾಯರ ಕೋಪ ಎರಡೂ ಸೇರಿ ಅಲ್ಲೊಂದು ತಮಾಷೆಯ ದೃಶ್ಯವನ್ನು ನಿರ್ಮಿಸಿದ್ದುವು. ಅದನ್ನು ಕಂಡು ನರೇಂದ್ರನಿಗೆ ನಗು ಉಕ್ಕಿಬಂತು. ಅವನು ನಗುವುದನ್ನು ಕಂಡಾಗ ಇತರ ಹುಡುಗರೂ ನಗು ತಡೆದುಕೊಳ್ಳಲಾರದೆ ಹೋದರು. ಈಗ ಉಪಾಧ್ಯಾಯರ ಕೋಪ ಈ ಹುಡುಗರ ಕಡೆಗೆ ತಿರುಗಿಬಿಟ್ಟಿತು. ನರೇಂದ್ರ ಮುಂದಿನ ಬೆಂಚಿನಲ್ಲೇ ಕುಳಿತಿದ್ದ. ಕೈಗೆ ಸುಲಭವಾಗಿ ಸಿಕ್ಕಿದ. ಉಪಾಧ್ಯಾಯರು ಅವನನ್ನು ಹಿಡುದುಕೊಂಡು ರಪರಪನೆ ಬಾರಿಸತೊಡಗಿದರು. ಹಾಗೆ ಬಾರಿಸುತ್ತಲೇ, “ಎಲ್ಲಿ, ನನಗೆ ಮಾತು ಕೊಡು, ಇನ್ನು ಮೇಲೆ ನನ್ನನ್ನು ಕಂಡು ನಗುವುದಿಲ್ಲ ಅಂತ” ಎಂದರು. ಆದರೆ ನರೇಂದ್ರ ಅದಕ್ಕೊಪ್ಪಲಿಲ್ಲ. ಏಕೆಂದರೆ ಅವನು ಆ ಉಪಾಧ್ಯಾಯರನ್ನು ಕಂಡು ನಕ್ಕದ್ದಲ್ಲ. ಅವನು ನಕ್ಕದ್ದು ಆ ಇಡೀ ಸನ್ನಿವೇಶವನ್ನು ಕಂಡು. ಈಗ ಉಪಾಧ್ಯಾಯರ ಕೋಪ ಎಲ್ಲೆ ಮೀರಿತು. ನರೇಂದ್ರನನ್ನು ಮತ್ತಷ್ಟು ಬಡಿದದ್ದಲ್ಲದೆ ಅವನ ಕಿವಿಗಳನ್ನು ಹಿಡಿದು ಬೆಂಚಿನಿಂದ ಮೇಲಕ್ಕೆತ್ತಿಬಿಟ್ಟರು! ಅವನ ಒಂದು ಕಿವಿಯಲ್ಲಿ ಗಾಯವಾಗಿ ರಕ್ತ ಸುರಿಯತೊಡಗಿತು. ಇಷ್ಟಾದರೂ ಅವನು ಉಪಾಧ್ಯಾಯರ ಷರತ್ತಿಗೆ ಒಪ್ಪಲೇ ಇಲ್ಲ. ಬದಲಾಗಿ ಕೋಪದಿಂದ ಕಣ್ಣೀರಿಳಿಸುತ್ತ, “ಬಿಡಿ ನನ್ನ ಕಿವಿಯನ್ನು. ನನ್ನನ್ನು ಹೊಡೆಯಲು ನೀವು ಯಾರು? ಇನ್ನೊಂದು ಸಲ ನನ್ನ ಮೈಮುಟ್ಟಿದಿರೋ ನೋಡಿಕೊಳ್ಳಿ!” ಎಂದ. ಅವನ ಮೈಗೂ ನೋವಾಯಿತು, ಅಭಿಮಾನಕ್ಕೂ ಧಕ್ಕೆಯಾಯಿತು. ಆ ವೇಳೆಗೆ ಸರಿಯಾಗಿ ಶಾಲೆಯ ಸ್ಥಾಪಕರಾದ ಈಶ್ವರಚಂದ್ರ ವಿದ್ಯಾಸಾಗರರು ಅಲ್ಲಿಗೆ ಬಂದರು. ನೋಡುತ್ತಾರೆ, ಹುಡುಗನ ಕಿವಿಯಿಂದ ರಕ್ತ ಇಳಿಯುತ್ತಿದೆ! ನರೇಂದ್ರ ಗಟ್ಟಿಯಾಗಿ ಅಳುತ್ತ ನಡೆದುದನ್ನೆಲ್ಲ ಬಿಡದೆ ಅವರೆದುರು ಹೇಳಿದ. ಬಳಿಕ, “ನಾನಿನ್ನು ನಿಮ್ಮ ಶಾಲೆಗೆ ಬರುವುದಿಲ್ಲ. ಈಗಿಂದಲೇ ನಿಮ್ಮ ಶಾಲೆಯನ್ನು ಬಿಟ್ಟುಬಿಟ್ಟೆ” ಎನ್ನುತ್ತ ಹೊರಟೇಬಿಟ್ಟ. ಆದರೆ ವಿದ್ಯಾಸಾಗರರು ಅವನ ಕೈಹಿಡಿದು ಆಫೀಸಿಗೆ ಕರೆದುಕೊಂಡು ಹೋಗಿ ಸಮಾಧಾನಪಡಿಸಿದರು. ಬಳಿಕ ಶಾಲೆಯ ಉಪಾಧ್ಯಾಯರನ್ನೆಲ್ಲ ಕರೆದು ವಿಚಾರಣೆ ನಡೆಸಿ, “ಇನ್ನು ಮುಂದೆ ಈ ಬಗೆಯ ಕ್ರೂರತನ ನಡೆಯಬಾರದು” ಎಂದು ಆಜ್ಞಾಪಿಸಿದರು. ನರೇಂದ್ರ ಮನೆಗೆ ಬಂದು ಈ ಘಟನೆಯನ್ನು ತಾಯಿಯ ಹತ್ತಿರ ಹೇಳಿದ. ಭುವನೇಶ್ವರಿಯ ಹೆತ್ತ ಕರುಳಿಗೆ ಉರಿಯಿಟ್ಟಂತಾಯಿತು. ಅವಳು “ನರೇನ್, ಇನ್ನು ಮೇಲೆ ಆ ಶಾಲೆಗೆ ಹೋಗಲೇಬೇಡ” ಎಂದು ಎಷ್ಟೋ ಹೇಳಿದಳು. ಆದರೆ ಅವನು ಆ ದುರ್ಘಟನೆಯನ್ನು ಮರೆತು ಮರುದಿನದಿಂದಲೇ ಶಾಲೆಗೆ ಹೋದ. ಕಿವಿಯ ಗಾಯ ವಾಸಿಯಾಗಲು ಮಾತ್ರ ಹಲವಾರು ದಿನಗಳೇ ಬೇಕಾದುವು.

ನರೇಂದ್ರನಿಗೆ ಇದು ಅಂತಹ ಮೊದಲ ಅನುಭವವೇನೂ ಅಲ್ಲ. ಹಿಂದೊಂದು ದಿನ ಭೂಗೋಳದ ಉಪಾಧ್ಯಾಯರು ಅವನಿಗೊಂದು ಪ್ರಶ್ನೆ ಹಾಕಿದ್ದರು. ಅದಕ್ಕೆ ಅವನು ಸರಿಯಾದ ಉತ್ತರವನ್ನೇ ಹೇಳಿದ. ಆದರೆ ಅದು ಸರಿಯಲ್ಲ ಎಂದು ಉಪಾಧ್ಯಾಯರು; ಇಲ್ಲ ತಾನು ಹೇಳಿದ್ದು ಸರಿಯಾಗಿಯೇ ಇದೆ ಎಂದು ನರೇಂದ್ರ. “ತಪ್ಪು ಉತ್ತರವನ್ನು ಕೊಟ್ಟದ್ದೂ ಅಲ್ಲದೆ ಅದೇ ಸರಿ ಅಂತ ಮೊಂಡಾಟ ಬೇರೆ ಮಾಡುತ್ತೀಯ?” ಎನ್ನುತ್ತ ಉಪಾಧ್ಯಾಯರು ಬೆತ್ತ ತೆಗೆದುಕೊಂಡು ಅವನ ಎಳೆಯ ಅಂಗೈ ಮೇಲೆ ರಪರಪನೆ ಹೊಡೆದರು, ಅವನು ಆ ನೋವನ್ನು ಸಹಿಸಿಕೊಂಡು ತುಟಿಪಿಟಕ್ಕೆನ್ನದೆ ಸುಮ್ಮನಿದ್ದುಬಿಟ್ಟ. ಉಪಾಧ್ಯಾಯರು ಮತ್ತೆ ಪಾಠ ಮಾಡ ಲಾರಂಭಿಸಿದರು. ಈ ಹೊತ್ತಿಗೆ ಅವರಿಗೆ ಅರಿವಾಯಿತು, ನರೇಂದ್ರ ಕೊಟ್ಟ ಉತ್ತರವೇ ಸರಿ ಎಂದು. ಅವರು ಪಶ್ಚಾತ್ತಾಪಗೊಂಡು ಕೂಡಲೇ ಆತನ ಕ್ಷೆಮೆ ಬೇಡಿದರು, ಮತ್ತು ಅಂದಿನಿಂದ ಅವನನ್ನು ತುಂಬ ಗೌರವಾದರಗಳಿಂದ ನೋಡಿಕೊಳ್ಳತೊಡಗಿದರು.

ಈ ಘಟನೆಯನ್ನೂ ನರೇಂದ್ರ ತಾಯಿಯ ಮುಂದೆ ಹೇಳಿದ. ಇಂಥ ಸಂದರ್ಭಗಳಲ್ಲೆಲ್ಲ ಹೇಳುತ್ತಿದ್ದಂತೆ ಆಕೆ ಮಗನನ್ನು ಸಮಾಧಾನಪಡಿಸುತ್ತ, “ಮಗು, ನಿನ್ನ ಕಡೆಯಿಂದ ನೀನು ಸರಿಯಾಗಿರುವಾಗ ಇನ್ನೇನು ಚಿಂತೆ? ನೋಡು, ನಿನಗೆ ಅನ್ಯಾಯವಾಗಬಹುದು, ಪರಿಸ್ಥಿತಿ ನಿನಗೆ ವಿರುದ್ಧವಾಗಬಹುದು; ಏನೇ ಆದರೂ ನೀನು ಮಾತ್ರ ಸರಿಯಾದದ್ದನ್ನೇ ಮಾಡುತ್ತ ಬಾ” ಎಂದಳು. ಈ ಬೋಧನೆ ನಿಜಕ್ಕೂ ಅವನ ಮೇಲೆ ತುಂಬ ಪ್ರಭಾವ ಬೀರಿತು. ಮುಂದೆ ಎಷ್ಟೋ ಸಲ ನರೇಂದ್ರನ ಕಾರ್ಯವಿಧಾನಗಳನ್ನು ಕಂಡು ಅವನ ಆಪ್ತರೇ ಅವನನ್ನು ಅಪಾರ್ಥಮಾಡಿ ಕೊಂಡು ಟೀಕಿಸಿದ್ದುಂಟು. ಆದರೆ ಅವನು ಮಾತ್ರ ಆ ಯಾವ ಮಾತಿಗೂ ಗಮನಕೊಡದೆ ತನಗೆ ಸೂಕ್ತವಾಗಿ ಕಂಡದ್ದನ್ನೇ ಮಾಡುತ್ತಿದ್ದ.

ನರೇಂದ್ರನಲ್ಲಿ ಬಾಲ್ಯದಿಂದಲೇ ಇನ್ನೊಂದು ಶ್ರೇಷ್ಠ ಗುಣ ಬೆಳೆದುಬರುತ್ತಿತ್ತು. ಅದು ಸತ್ಯ ಹೇಳುವ ಗುಣ. ಕಷ್ಟಗಳಿಂದ ತಪ್ಪಿಸಿಕೊಳ್ಳಲೇ ಆಗಲಿ, ಅಪ್ರಿಯ ಸನ್ನಿವೇಶಗಳಿಂದ ಪಾರಾಗಲೇ ಆಗಲಿ, ಸುಳ್ಳಿನ ಕಂತೆಯನ್ನು ಹೊಸೆಯುವ ದುರ್ಗುಣ ಅವನಿಗೆಂದೂ ಬರಲಿಲ್ಲ.

ಬೆಳೆದು ದೊಡ್ಡವನಾದಂತೆಲ್ಲ ನರೇಂದ್ರನಲ್ಲಿ ಧ್ಯಾನ ಮಾಡುವ ಅಭಿಲಾಷೆ ಇನ್ನಷ್ಟು ತೀವ್ರವಾಯಿತು. ರಾತ್ರಿ ಮಲಗಿಕೊಳ್ಳಲು ಕಣ್ಮುಚ್ಚಿದನೆಂದರೆ ಒಡನೆಯೇ ಜ್ಯೋತಿದರ್ಶನ! ಆದರೆ ಇವೆಲ್ಲ ಹೀಗಿದ್ದರೂ ಹಗಲಿನಲ್ಲಿ ಅವನ ತುಂಟಾಟ-ಚಟುವಟಿಕೆಗಳು ಎಂದಿನಂತೆಯೇ ನಡೆದುಕೊಂಡು ಬರುತ್ತಿದ್ದವು. ಮಾತ್ರವಲ್ಲದೆ, ತನ್ನ ವಯಸ್ಸಿಗೆ ತಕ್ಕಂತೆ ಅವನು ಹೊಸಹೊಸ ಕಾರ್ಯಕಲಾಪಗಳನ್ನು ಹಾಕಿಕೊಳ್ಳುತ್ತಿದ್ದ. ಉದಾಹರಣೆಗೆ, ಒಮ್ಮೆ ಅವನೊಂದು ನಾಟಕದ ತಂಡ ಕಟ್ಟಿದ. ತನ್ನ ಸ್ನೇಹಿತರಲ್ಲೇ ಕೆಲವರನ್ನು ಸೇರಿಸಿಕೊಂಡು ನಾಟಕಾಭ್ಯಾಸ ಮಾಡತೊಡಗಿದ. ಮನೆಯ ಭಜನ ಮಂದಿರದ ಒಂದು ಭಾಗ ರಂಗಮಂಟಪವಾಯಿತು. ಅಲ್ಲಿ ಈ ಹುಡುಗರು ಎಲ್ಲೋ ಕೆಲವೇ ನಾಟಕಗಳನ್ನು ಆಡಿರಬಹುದು. ಅಷ್ಟರಲ್ಲಿ ನರೇಂದ್ರನ ಸೋದರಮಾವ ಬಂದು ಪರದೆಗಿರದೆಗಳನ್ನೆಲ್ಲ ಕಿತ್ತೆಸೆದುಬಿಟ್ಟ. ಎಲ್ಲೋ ಹುಡುಗರ ಗಲಾಟೆ ಅತಿಯಾಗಿರಬೇಕು. ಬೇರೆ ಉಪಾಯಗಾಣದೆ ನರೇಂದ್ರ ನಾಟಕದ ಹವ್ಯಾಸವನ್ನು ಅಲ್ಲಿಗೆ ಬಿಟ್ಟ. ಈಗ ಮನೆಯ ಅಂಗಳ ದಲ್ಲೇ ಒಂದು ವ್ಯಾಯಾಮಶಾಲೆ ಸ್ಥಾಪಿಸಿದ. ಸ್ನೇಹಿತರೆಲ್ಲ ಸೇರಿ ಅಂಗಸಾಧನೆ ಶುರುಮಾಡಿದರು. ಇದು ಎಷ್ಟೋ ದಿನ ಚೆನ್ನಾಗಿಯೇ ನಡೆದುಕೊಂಡು ಬಂತು. ಹುಡುಗರ ಗಲಾಟೆ ಮನೆಯ ಹೊರಗಡೆಯೇ ಆದ್ದರಿಂದ ಸೋದರಮಾವ ಸುಮ್ಮನಿದ್ದ. ಆದರೆ ಈ ಮಧ್ಯೆ ಒಂದು ಅಪಘಾತ ಸಂಭವಿಸಿ ಒಬ್ಬ ಹುಡುಗ ಕೈಮೂಳೆ ಮುರಿದುಕೊಂಡ. ಇದನ್ನೇ ಕಾಯುತ್ತಿದ್ದವನಂತೆ ನರೇಂದ್ರನ ಸೋದರಮಾವ ಮತ್ತೆ ಬಂದ; ಉಪಕರಣಗಳನ್ನೆಲ್ಲ ಹೊರಗೆಸೆದು ವ್ಯಾಯಾಮಶಾಲೆ ಯನ್ನು ನಿರ್ನಾಮ ಮಾಡಿದ. ಇದಾದ ಮೇಲೆ ನರೇಂದ್ರ ತನ್ನ ಗೆಳೆಯರೊಂದಿಗೆ ಸಮೀಪದಲ್ಲೇ ಇದ್ದ ನವಗೋಪಾಲ ಮಿತ್ರ ಎನ್ನುವವನ ಗರಡಿಮನೆಗೆ ಸೇರಿಕೊಂಡ. ಈ ಹೊಸ ಸ್ಥಳ ತುಂಬ ಅನುಕೂಲಕರವಾಗಿತ್ತು. ನರೇಂದ್ರನೂ ಅವನ ಸ್ನೇಹಿತರೂ ಉತ್ಸಾಹ ಶ್ರದ್ಧೆಗಳಿಂದ ಅಂಗಸಾಧನೆ ಯಲ್ಲಿ ತೊಡಗಿದರು. ಕ್ರಮೇಣ ನರೇಂದ್ರ ಲಾಠಿ ಬೀಸುವುದರಲ್ಲಿ, ಮಲ್ಲಯುದ್ಧಗಳಲ್ಲಿ, ದೋಣಿ ನಡೆಸುವುದರಲ್ಲಿ, ಈಜುವುದರಲ್ಲಿ ಮತ್ತು ಇನ್ನೂ ಅನೇಕ ಕ್ರೀಡೆಗಳಲ್ಲಿ ಪ್ರವೀಣನಾದ. ಒಮ್ಮೆ ಪ್ರದರ್ಶನ ಪಂದ್ಯಾವಳಿಯೊಂದರಲ್ಲಿ ಅವನಿಗೆ ಮುಷ್ಟಿಯುದ್ಧದಲ್ಲಿ ಮೊದಲ ಬಹುಮಾನವಾಗಿ ಸುಂದರವಾದ ಬೆಳ್ಳಿಯ ಚಿಟ್ಟೆ ದೊರಕಿತ್ತು.

ಲಾಠಿ ಬೀಸುವುದರಲ್ಲಿ ನರೇಂದ್ರನಿಗೆ ವಿಶೇಷ ಉತ್ಸಾಹ, ಆಸಕ್ತಿ. ಇದನ್ನು ಅವನು ಹಲವಾರು ಮುಸಲ್ಮಾನ ಪೈಲ್ವಾನರಿಂದ ಕಲಿತು ಚೆನ್ನಾಗಿ ಸ್ವಾಧೀನಪಡಿಸಿಕೊಂಡಿದ್ದ. ಅವನು ಮೆಟ್ರೊ ಪಾಲಿಟನ್ ಶಾಲೆಯಲ್ಲೇ ವಿದ್ಯಾಭ್ಯಾಸ ಮಾಡುತ್ತಿದ್ದ ಸಮಯ. ಒಂದು ದಿನ ಶಾಲೆಯಲ್ಲಿ ಬಗೆಬಗೆಯ ಕಸರತ್ತು ಪ್ರದರ್ಶನಗಳನ್ನು ಏರ್ಪಡಿಸಿದ್ದರು. ಪ್ರೇಕ್ಷಕರಲ್ಲಿ ನರೇಂದ್ರನೂ ಒಬ್ಬ. ಒಂದಾದಮೇಲೊಂದು ಪಂದ್ಯಗಳು ನಡೆದುವು. ಈಗ ಲಾಠಿ ಬೀಸುವ ಪಂದ್ಯಗಳು ಪ್ರಾರಂಭ ವಾದುವು. ಪಂದ್ಯಗಳೇನೋ ನಡೆಯುತ್ತಲೇ ಇದ್ದರೂ ಅದರಲ್ಲಿ ಬಿರುಸೇ ಇರಲಿಲ್ಲ. ಇದನ್ನು ಕಂಡು ನರೇಂದ್ರನಿಗೆ ರೇಗಿತು. ಅವರೆಲ್ಲರಿಗಿಂತಲೂ ತಾನು ಎಷ್ಟೋ ಪಾಲು ಚೆನ್ನಾಗಿ ಲಾಠಿ ಬೀಸಬಲ್ಲೆ ಎಂದು ಅವನಿಗನ್ನಿಸಿತು. ಇದ್ದಕ್ಕಿದ್ದಂತೆ ಎದ್ದುಬಂದು ಆಖಾಡವನ್ನು ಪ್ರವೇಶಿಸಿ “ಈಗ ನನ್ನ ಜೊತೆಯಲ್ಲಿ ಲಾಠಿ ಬೀಸುವುದಕ್ಕೆ ಯಾರು ಬರುತ್ತೀರೋ ಬನ್ನಿ!” ಎಂದು ಕೂಗಿದ. ಆ ರಭಸಕ್ಕೆ ಹೆದರಿ ಅವನ ವಯಸ್ಸಿನ ಹುಡುಗರೆಲ್ಲ ಅಲ್ಲಲ್ಲೇ ಸುಮ್ಮನೆ ಕುಳಿತುಬಿಟ್ಟರು. ಆದರೆ ವಯಸ್ಸಿನಲ್ಲಿ ಅವನಿಗಿಂತ ಹಿರಿಯನಾದ ಒಬ್ಬ ಹುಡುಗ ಎದ್ದುಬಂದ. ಇವನೇ ಅಲ್ಲಿದ್ದವರಲ್ಲೆಲ್ಲ ಧಾಂಡಿಗ. ಸರಿ ಈಗ ಇಬ್ಬರ ಲಾಠಿಗಳೂ ತಿರುಗಿದುವು, ಒಂದಕ್ಕೊಂದು ಬಡಿದುವು. ಬರಬರುತ್ತ ಹೋರಾಟ ರೋಮಾಂಚಕಾರಿಯಾಯಿತು, ಪ್ರೇಕ್ಷಕರ ರಕ್ತ ಬಿಸಿಯೇರಿತು. ಆದರೆ ಏನೇ ಆದರೂ ಗೆಲ್ಲುವುದಂತೂ ಆ ಧಾಂಡಿಗನೇ ಎಂದು ಪ್ರೇಕ್ಷಕರೆಲ್ಲ ತೀರ್ಮಾನಿಸಿಬಿಟ್ಟಿದ್ದರು. ಆದರೂ ನರೇಂದ್ರ ಚುರುಕಾಗಿ ಲಾಠಿ ಬೀಸುತ್ತ, ವೀರಾವೇಶದಿಂದ ಹೋರಾಡುವುದನ್ನು ಕಂಡು ಪ್ರೇಕ್ಷಕ ರೆಲ್ಲ ನರೇಂದ್ರನ ಪರವಾದರು! ಗಟ್ಟಿಯಾಗಿ ಚಪ್ಪಾಳೆ ತಟ್ಟುತ್ತ ಹುರಿದುಂಬಿಸಿದರು. ನರೇಂದ್ರ ಮಾತ್ರ ಇದಾವುದಕ್ಕೂ ಗಮನಕೊಡದೆ ಪರಾಕ್ರಮದಿಂದ ಹೋರಾಡುತ್ತಿದ್ದಾನೆ–ಇದ್ದಕ್ಕಿದ್ದಂತೆ ಹಿಂದಕ್ಕೆ ಸರಿಯುತ್ತಾನೆ, ತಕ್ಷಣ ಮುಂದಕ್ಕೆ ನುಗ್ಗುತ್ತಾನೆ, ಮರುಕ್ಷಣ ಪಕ್ಕಕ್ಕೆ ಜಿಗಿಯುತ್ತಾನೆ– ಹೀಗೆ ತುಂಬ ಚಾಕಚಕ್ಯತೆಯಿಂದ ಲಾಠಿ ಬೀಸುತ್ತಿದ್ದಾನೆ. ಎದುರಾಳಿಯ ಸಾಮರ್ಥ್ಯ ಕೂಡ ಕಡಿಮೆಯೇನಿರಲಿಲ್ಲ. ಬಹುಶಃ ಆತ ಸ್ವಲ್ಪ ಹೆಚ್ಚಿನ ಕೌಶಲದಿಂದಲೇ ಲಾಠಿ ಬೀಸುತ್ತಿದ್ದಾನೆ. ಆದರೆ ಈಗ ನರೇಂದ್ರ ತನ್ನ ಲಾಠಿಯನ್ನು ಎಷ್ಟು ಚೆನ್ನಾಗಿ ಮತ್ತು ಜೋರಾಗಿ ಬೀಸಿದನೆಂದರೆ ಎದುರಾಳಿಯ ಲಾಠಿ ‘ಠಣಲ್’ ಎಂದು ಎರಡು ತುಂಡಾಗಿ ಬಿತ್ತು. ನರೇಂದ್ರ ಗೆದ್ದ, ಧಾಂಡಿಗನೇ ಸೋತ! ಪ್ರೇಕ್ಷಕರೆಲ್ಲ ಸಂತೋಷದಿಂದ ಹುಚ್ಚೆದ್ದುಬಿಟ್ಟಿದ್ದಾರೆ!

ನಾವು ಮೊದಲಿನಿಂಲೂ ನೋಡಿದಂತೆ, ಸುಮ್ಮನೆ ಕುಳಿತಿರುವುದು ಎಂದರಾಗದು ನರೇಂದ್ರನಿಗೆ; ಏನನ್ನಾದರೂ ಮಾಡುತ್ತಲೇ ಇರಬೇಕು. ಆದ್ದರಿಂದಲೇ ತನ್ನ ಸಮಯವನ್ನು ಸದಾ ಚಟುವಟಿಕೆಯಿಂದ ಕಳೆಯಲು ಹಲವಾರು ವಿದ್ಯೆಗಳನ್ನು ಕಲಿತಿದ್ದ. ಅವನಿಗೆ ಮಾಯಾಲಾಂದ್ರದ(ಸ್ಲೈಡ್ ಪ್ರೊಜೆಕ್ಟರ್) ಮೂಲಕ ಗೋಡೆಯ ಮೇಲೆ ಚಿತ್ರಗಳನ್ನು ಬಿಡುವುದಕ್ಕೂ ಬರುತ್ತಿತ್ತು. ಅಲ್ಲದೆ ಅವನ ತಂದೆ ಆತನಿಗೆ ಸಣ್ಣ ಕುದುರೆಯೊಂದನ್ನು ತೆಗೆಸಿಕೊಟ್ಟಿದ್ದ. ಇದರ ಮೇಲೆ ಸವಾರಿ ಮಾಡುವುದೆಂದರೆ ಅವನಿಗೆ ಬಹಳ ಪ್ರಿಯವಾದ ಮನರಂಜನೆ. ಇದರಲ್ಲಿ ಸಾಹಸ ಬೇರೆ ಇದೆಯಲ್ಲವೆ? ಕ್ರಮೇಣ ಆತ ಕುದುರೆಸವಾರಿಯಲ್ಲೂ ನಿಸ್ಸೀಮನಾದ.

ಇಷ್ಟೇ ಅಲ್ಲ ನರೇಂದ್ರನ ಸಾಹಸಗಳು. ಆ ದಿನಗಳಲ್ಲಿ ಕಲ್ಕತ್ತದ ರಸ್ತೆಗಳಲ್ಲಿ ಅನಿಲದ ದೀಪಗಳನ್ನು ಉರಿಸುತ್ತಿದ್ದರು. ಈ ಅನಿಲದ ಉತ್ಪಾದನೆಗಾಗಿ ದೊಡ್ಡ ಸ್ಥಾವರಗಳಿದ್ದವು. ತಾನೂ ಇಂತಹ ಆಟದ ಸ್ಥಾವರವೊಂದನ್ನು ನಿರ್ಮಿಸಿದ್ದ ನರೇಂದ್ರ. ಅದರಲ್ಲಿ ಒಣಹುಲ್ಲು, ಕಸಕಡ್ಡಿಗಳನ್ನು ಉರಿಸಿ ವಿಚಿತ್ರ ಬಣ್ಣದ ಹೊಗೆ ಬರಿಸುತ್ತಿದ್ದ. ಇದಲ್ಲದೆ ಆಟದ ರೈಲುಬಂಡಿ ಮತ್ತು ಇನ್ನೆಷ್ಟೆಷ್ಟೋ ತರಹದ ಯಂತ್ರೋಪಕರಣಗಳನ್ನೆಲ್ಲ ಮಾಡಿ ಸ್ನೇಹಿತರನ್ನು ರಂಜಿಸುತ್ತಿದ್ದ.

ಈಗ ನರೇಂದ್ರನಿಗೆ ಇನ್ನೊಂದು ಆಲೋಚನೆ ಬಂದುಬಿಟ್ಟಿದೆ – ಚೆನ್ನಾಗಿ ಅಡಿಗೆ ಮಾಡುವುದನ್ನು ಕಲಿಯಬೇಕು ಎಂದು ಆದರೆ ಅಡಿಗೆಮನೆಯೊಳಗೆ ಇವನ ಗಲಾಟೆಗೆಲ್ಲ ಪ್ರವೇಶವಿಲ್ಲ. ಆದ್ದರಿಂದ ಹೊರಗಡೆಯೇ ಒಂದು ಜಾಗ ಮಾಡಿಕೊಂಡು, ಒಲೆ-ಗುಂಡುಕಲ್ಲು ಹೂಡಬೇಕು, ಅಡಿಗೆ ಮಾಡಬೇಕು. ಅಕ್ಕಿ ಬೇಳೆಯಿಂದ ಹಿಡಿದು ತರಕಾರಿ ಸಾಸಿವೆ ಉಪ್ಪು ಎಣ್ಣೆ ಮೊದಲಾದ ಎಲ್ಲ ಸಾಮಾನುಗಳೂ ಸ್ವಂತದ್ದೇ ಆಗಬೇಕು ಎನ್ನುವುದು ಅವನಿಚ್ಛೆ. ಸ್ನೇಹಿತರನ್ನೆಲ್ಲ ಸೇರಿಸಿ ತನ್ನ ಆಲೋಚನೆಯನ್ನು ತಿಳಿಸಿದ. ಎಲ್ಲರೂ ಸಂತೋಷದಿಂದ ಒಪ್ಪಿಕೊಂಡರು. ರುಚಿಕರವಾದ ಅಡಿಗೆ ಮಾಡಿ ‘ಊಟ ಮಾಡುವ ಆಟ’ ಆಡುವುದೆಂದರೆ ಯಾರಿಗೆ ತಾನೆ ಇಷ್ಟವಿಲ್ಲ? ಆಗ ನರೇಂದ್ರ ಹೇಳಿದ: “ನೋಡ್ರಪ್ಪ, ಎಲ್ಲರೂ ಸ್ವಲ್ಪಸ್ವಲ್ಪ ವಂತಿಗೆ ಹಾಕಬೇಕಾಗುತ್ತೆ. ಆದರೆ ಹೆಚ್ಚಿನ ಭಾರವನ್ನೆಲ್ಲ ನಾನೇ ವಹಿಸಿಕೊಳ್ಳುತ್ತೇನೆ.” ಹುಡುಗರು ಅದಕ್ಕೂ ಒಪ್ಪಿದರು. ವಂತಿಗೆ ಸಂಗ್ರಹವಾಯಿತು. ಒಲೆ ಹೂಡಿಯೂ ಆಯಿತು. ಸಾಮಾನುಗಳನ್ನು ತಂದದ್ದೂ ಆಯಿತು. ಈಗ ಅಡಿಗೆ ಮಾಡಬೇಕು. ಆದರೆ ಏನಡಿಗೆ ಮಾಡುವುದು? ಖಿಚಡಿ(ಒಂದು ಬಗೆಯ ತೊವ್ವೆ-ಅನ್ನ) ಹಾಗೂ ಒಂದೆರಡು ಪಲ್ಯಗಳನ್ನು ಮಾಡುವುದು ಎಂದು ತೀರ್ಮಾನಿಸಿದರು. ನರೇಂದ್ರ ಇದನ್ನೆಲ್ಲ ಮಾಡುವ ವಿಧಾನವನ್ನು ಈಗಾಗಲೇ ತಾಯಿಯಿಂದ ಕಲಿತುಕೊಂಡಿದ್ದ. ಈಗ ಎಲ್ಲರೂ ಸೇರಿ ಅಡಿಗೆಗೆ ಕೈಹಚ್ಚಿದರು. ನರೇಂದ್ರನೇ ಮುಖ್ಯ ಬಾಣಸಿಗ. ಉಳಿದವರೆಲ್ಲ ಅವನಿಗೆ ತೈನಾತಿಗಳು. ಅಂತೂ ಅಡಿಗೆ ತಯಾರಾಯಿತು. ಎಲ್ಲರೂ ಅತ್ಯುತ್ಸಾಹದಿಂದ ಎಲೆಗೆ ಬಡಿಸಿಕೊಂಡರು. ಆದರೆ ಬಾಯಿಗಿಟ್ಟುಕೊಂಡರೆ ಖಾರವೋ ಖಾರ! ನರೇಂದ್ರನ ಕೈ ಖಾರಮುಂದು; ಪಲ್ಯಗಳಿಗೆ ಹಸಿರು ಮೆಣಸಿನಕಾಯಿಯನ್ನು ಧಾರಳವಾಗಿಯೇ ಹಾಕಿಬಿಟ್ಟಿದ್ದ. ಖಿಚಡಿಗೆ ಕೂಡ ಖಾರ ಹೆಚ್ಚೇ. ಆದರೂ ಅಡಿಗೆ ತುಂಬ ರುಚಿಯಾಗಿಯೇ ಇತ್ತು. ನರೇಂದ್ರನಿಗೇನೋ ಖಾರವೇ ಇಷ್ಟ;ಸರಿಯಾಗಿಯೇ ಊಟ ಮಾಡಿದ. ಆದರೆ ಉಳಿದವರೆಲ್ಲ ಕಣ್ಣು-ಮೂಗುಗಳನ್ನು ಆಗಾಗ ಒರೆಸಿಕೊಳ್ಳುತ್ತ ಊಟ ಮಾಡಿದರು; ಆದರೂ ‘ಚೆನ್ನಾಗಿದೆ’ ಅಂತಲೇ ಹೇಳಿದರು. ಕೆಲವೇ ದಿನಗಳಲ್ಲಿ ಎಲ್ಲರೂ ಅಡಿಗೆ ಮಾಡುವುದನ್ನು ಕಲಿತರು. ಈ ಪಾಕಶಾಸ್ತ್ರನಿಪುಣರೆಲ್ಲ ಸೇರಿ ಬಹಳ ದಿನಗಳವರೆಗೆ ಈ ಆಟವನ್ನು ಮುಂದುವರಿಸಿಕೊಂಡು ಬಂದರು. ಎಷ್ಟಾದರೂ ರುಚಿಕರವಾದ ಆಟವಲ್ಲವೆ? 

ನರೇಂದ್ರ ತನ್ನ ಸುತ್ತಮುತ್ತಲ ಮನೆಯವರೊಂದಿಗೆ ಒಂದಲ್ಲ ಒಂದು ರೀತಿಯ ಸಂಪರ್ಕವಿಟ್ಟಕೊಂಡಿದ್ದ. ಅವರು ಬಡವರಾಗಿರಬಹುದು, ಶ್ರೀಮಂತರಾಗಿರಬಹುದು, ಯಾವ ಜಾತಿಯವರಾದರೂ ಆಗಿರಬಹುದು– ಅವರೆಲ್ಲರ ಜೊತೆಯಲ್ಲೂ ಇವನು ಸ್ನೇಹ ಬೆಳೆಸಿದ್ದ. ಅವರೂ ಕೂಡ ನರೇಂದ್ರನನ್ನು ತಮ್ಮವನು ಎಂದೇ ಭಾವಿಸಿ ವಿಶ್ವಾಸವಿಟ್ಟಿದ್ದರು. ತನ್ನ ಸ್ನೇಹಿತರಲ್ಲಿ ಯಾರಾದರೂ ಕಷ್ಟದಲ್ಲಿರುವುದನ್ನು ಕಂಡರೆ ತಕ್ಷಣ ಅವನ ಹತ್ತಿರ ಓಡುತ್ತಿದ್ದ. ಅವನಿಗೆ ಸಹಾಯ ಮಾಡುತ್ತಿದ್ದ, ಸಮಾಧಾನ ಹೇಳುತ್ತಿದ್ದ. ಎಲ್ಲರೂ ಯಾವಗಲೂ ಖುಷಿಯಿಂದಿರಬೇಕು ಎನ್ನುವುದೇ ನರೇಂದ್ರನ ಆಸೆ, ನೆರೆಹೊರೆಯ ಮನೆಯವರನ್ನೆಲ್ಲ ದೊಡ್ಡವರು-ಚಿಕ್ಕವರು ಎನ್ನದೆ ತನ್ನ ಬಗೆಬಗೆಯ ತಮಾಷೆಗಳಿಂದ ನಗಿಸಿ ಖುಷಿಯಾಗಿಡುತ್ತಿದ್ದ. ತನ್ನ ಮುದ್ದಾದ ರೂಪ, ಮಧುರವಾದ ಕಂಠಸ್ವರ, ಹಾಸ್ಯಮಯ ಸ್ವಭಾವ, ಹಿತವಾದ ನಡವಳಿಕೆ– ಇವುಗಳಿಂದಾಗಿ ಅವನು ಪ್ರತಿಯೊಬ್ಬರಿಗೂ ಪ್ರಿಯನಾಗಿದ್ದ. ಎಲ್ಲರಿಗೂ ಅವನು ಬೇಕು. ಅವನ ಉತ್ಸಾಹಪೂಣನ್ ವ್ಯಕ್ತಿತ್ವ ಸುತ್ತಮುತ್ತಲ ಜನರಿಗೆಲ್ಲ ನಿರಂತರ ನವಚೇತನವನ್ನು ನೀಡುತ್ತಿತ್ತು ಎಂದರೆ ಅತಿಶಯೋಕ್ತಿಯಲ್ಲ. ಅಂತೆಯೇ ಅವನ ಮನೆಮಂದಿಯೂ ಕೂಡ ಅವನನ್ನು ವಿಶೇಷವಾಗಿ ಪ್ರೀತಿಸುತ್ತಿದ್ದುದರಲ್ಲಿ ಆಶ್ಚರ್ಯವೇನೂ ಇಲ್ಲ. ನರೇಂದ್ರ ಕಥೆ ಹೇಳುವುದಲ್ಲೂ ನಿಸ್ಸೀಮ. ಆದ್ದರಿಂದ ರಾತ್ರಿ ಊಟವಾದ ಬಳಿಕ ಅವನ ಇಬ್ಬರು ತಮ್ಮಂದಿರು ಅವನನ್ನು “ಕಥೆ ಹೇಳು” ಎಂದು ಕಾಡುತ್ತಿದ್ದರು. ನರೇಂದ್ರನ ನೆನಪಿನ ಚೀಲದಲ್ಲಿ ತುಂಬ ಕಥೆಗಳಿದ್ದುವು– ಅವನ ತಾಯಿತಂದೆ ಅಜ್ಜಿ ತಾತ ಇವರೆಲ್ಲ ಹೇಳಿದ್ದ ಕಥೆಗಳು. ಈ ಮಕ್ಕಳು ಪೀಡಿಸಿದಾಗಲೆಲ್ಲ ಒಂದೊಂದೇ ಕಥೆಯನ್ನು ಹೊರತೆಗೆದು ತುಂಬ ರಸವತ್ತಾಗಿ ಹೇಳುತ್ತಿದ್ದ.

ನರೇಂದ್ರನಿಗೆ ಬೆರಳ ನೆರಳಾಟ \eng{(Hand shadow play)} ಕೂಡ ಗೊತ್ತಿತ್ತು. ಅದನ್ನು ಮಾಡಿ ತೋರಿಸುವಂತೆ ತಮ್ಮಂದಿರು ಆಗಾಗ ಅವನನ್ನು ಪೀಡಿಸುವುದಿತ್ತು. ನರೇಂದ್ರ ಅದಕ್ಕೊಪ್ಪಿ ಒಂದು ಎತ್ತರದ ಪೀಠದ ಮೇಲೆ ದೀಪ ಹಚ್ಚಿ ಇಟ್ಟುಕೊಳ್ಳುತ್ತಿದ್ದ. ಆ ದೀಪದ ಮುಂದೆ ತನ್ನ ಹತ್ತೂ ಬೆರಳುಗಳನ್ನು ತಿರಿಚಿ-ಮುರುಚಿ, ಮಡಿಸಿ-ಬಿಡಿಸಿ ಗೋಡೆಯ ಮೇಲೆ ನಾನಾ ಆಕೃತಿಗಳು ಬೀಳುವಂತೆ ಮಾಡುತ್ತಿದ್ದ. ಹಾರುತ್ತಿರುವ ಬಾವಲಿ, ಓಡುತ್ತಿರುವ ಕುದುರೆ– ಇವನ್ನೆಲ್ಲ ತೋರಿಸುತ್ತಿದ್ದ. ಹಾಗೆಯೇ ದುರ್ಗೆ ಲಕ್ಷ್ಮಿಸರಸ್ವತಿ ಕಾರ್ತಿಕ ಗಣೇಶ ಇವರುಗಳನ್ನೂ ತೋರಿಸುತ್ತಿದ್ದ. ನರೇಂದ್ರನ ಕೈಚಳಕ ಅದ್ಭುತವಾದದ್ದೇ ಸರಿ. ಆದರೆ ನೋಡುತ್ತ ಕುಳಿತ ಚಿಕ್ಕಚಿಕ್ಕ ಮಕ್ಕಳು ಮಾತ್ರ ಆ ಆಕೃತಿಗಳನ್ನು ಗುರುತು ಹಿಡಿಯಲು ತಮ್ಮಊಹಾಶಕ್ತಿಯನ್ನು ಸ್ವಲ್ಪ ಹೆಚ್ಚಾಗಿಯೇ ಉಪಯೋಗಿಸಕೊಳ್ಳಬೇಕಾಗಿತ್ತು!

ನರೇಂದ್ರ ತನ್ನ ಸ್ನೇಹಿತರನ್ನು ಖುಷಿಪಡಿಸಲು ಅವರನ್ನು ಆಗಾಗ ಕಲ್ಕತ್ತದ ಉದ್ಯಾನಗಳಿಗೋ ಪ್ರದರ್ಶನಾಲಗಳಿಗೋ ಕರೆದೊಯ್ಯುತ್ತಿದ್ದ. ಒಂದು ಸಲ ಅವನು ಅವರನ್ನೆಲ್ಲ ಗಂಗಾನದಿಯ ಮೂಲಕ ದೋಣಿಯ ಮೇಲೆ ಪ್ರಾಣಿವನವೊಂದಕ್ಕೆ ಕರೆದುಕೊಂಡು ಹೋಗಿದ್ದ. ಅದೊಂದು ಅತ್ಯಂತ ವೈವಿಧ್ಯಪೂರ್ಣವಾದ ವಿಶಾಲ ಉದ್ಯಾನ. ಅದನ್ನು ನೋಡಿಕೊಂಡು ಹಿಂದಿರುಗುವಾಗ ಒಬ್ಬ ಹುಡುಗನಿಗೆ ಹೊಟ್ಟೆ ತೊಳಸಿಬಂದು ದೋಣಿಯಲ್ಲೇ ವಾಂತಿ ಮಾಡಿಕೊಂಡುಬಿಟ್ಟ. ದೋಣಿಯವರು ಕೋಪಗೊಂಡು, ತಕ್ಷಣ ಅವನ್ನೆಲ್ಲ ತೆಗೆದು ಶುಚಿ ಮಾಡುವಂತೆ ಹುಡುಗರಿಗೆ ಹೇಳಿದರು. ಆದರೆ ಹುಡುಗರು“ನಾವೆಲ್ಲ ಆ ಕೆಲಸ ಮಾಡುವುದಕ್ಕೆ ಸಾಧ್ಯವಿಲ್ಲ. ಬದಲಾಗಿ ನಿಮಗೆ ಎರಡರಷ್ಟು ಹಣ ಕೊಡುತ್ತೇವೆ. ನೀವೇ ಯಾರಕೈಲಾದರೂ ಮಾಡಿಸಿಕೊಳ್ಳಿ” ಎಂದರು. ಎಷ್ಟಾದರೂ ಚಿಕ್ಕ ಹುಡುಗರಲ್ಲವೆ ಎಂಬ ಕನಿಕರದಿಂದ, ಅಥವಾ ಅವರು ಎರಡರಷ್ಟು ಬಾಡಿಗೆ ಕೊಡಲು ಮುಂದಾದುದರಿಂದಲಾದರೂ ದೋಣಿಯವರು ಇದಕ್ಕೆ ಒಪ್ಪಿಕೊಳ್ಳಬಹುದಾಗಿತ್ತು. ಆದರೆ ಅವರು ಒಪ್ಪಲೇ ಇಲ್ಲ. ಅಂತೂ ಅದೇ ಸ್ಥಿತಿಯಲ್ಲಿ ದೋಣಿ ದಡದ ಹತ್ತಿರ ಬಂದು ನಿಂತಿತು. ಆದರೆ ಅಂಬಿಗರು ಹುಡುಗರನ್ನು ದಡಕ್ಕೆ ಇಳಿಯಲು ಬಿಡದೆ, “ ಮೊದಲು ದೋಣಿಯನ್ನು ಶುಚಿಮಾಡಿ, ಆಮೇಲೆ ಇಳಿಯಿರಿ” ಎಂದು ಗದರಿಸುತ್ತ ಅಡ್ಡಗಟ್ಟಿ ನಿಂತರು. ಹುಡುಗರು ಕೂಡ ತಮ್ಮಿಂದ ಆದೆಲ್ಲ ಸಾದ್ಯವೆ ಇಲ್ಲ ಎಂದು ವಾದಿಸುತ್ತಿದ್ದರು. ಇತ್ತ ನರೇಂದ್ರ ಯೋಚಿಸುತ್ತಿದ್ದ, ಏನು ಮಾಡುವುದೀಗ– ಎಂದು. ಹಾಗೆ ನೋಡಿದರೆ ಅವನೇ ಎಲ್ಲರಿಗಿಂತ ಚಿಕ್ಕವನು. ಆದರೂ ಅವನು ಕಂಗಾಲಾಗದೆ ಒಂದು ಉಪಾಯ ಹುಡುಕುತ್ತಿದ್ದ. ಅಷ್ಟೊತ್ತಿಗೆ ಇಬ್ಬರು ಆಂಗ್ಲ ಸೈನಿಕರು ದಡದ ಮೇಲೆ ಅಡ್ಡಾಡುತ್ತಿರುವುದು ಅವನ ಕಣ್ಣಿಗೆ ಬಿತ್ತು. ತಕ್ಷಣ ಅವನಿಗೊಂದು ಆಲೋಚನೆ ಹೊಳೆಯಿತು– ‘ ಈ ಸೈನಿಕರ ಸಹಾಯವನ್ನೇಕೆ ಪಡೆಯಬಾರದು?’ ಎಂದು. ಈಗ, ಹುಡುಗರೆಲ್ಲ ಓಡಿಹೋಗದಿರಲೆಂದು ಅಂಬಿಗರು ದೋಣಿಯನ್ನು ದಡದಿಂದ ದೂರಕ್ಕೆ ತೆಗೆದುಕೊಂಡು ಹೋಗಲಾರಂಭಿಸಿದ್ದರು. ನರೇಂದ್ರ ಈಗ ಸೈನಿಕರನ್ನು ಮಾತನಾಡಿಸಲೇ ಬೇಕಾಗಿದೆ. ಆದ್ದರಿಂದ ಅವನೀಗ ಕ್ಷಣಕಾಲವೂ ತಡಮಾಡದೆ ದೋಣಿಯಿಂದ ದಡಕ್ಕೆ ಜೋರಾಗಿ ಜಿಗಿದ. ಸ್ವಲ್ಪ ಹೆಚ್ಚುಕಡಿಮೆಯಾಗಿಬಿಟ್ಟಿದ್ದರೂ ಅವನು ನೀರೊಳಗೆ ಬಿದ್ದುಬಿಡುತ್ತಿದ್ದ. ಆದರೆ ಅವನಲ್ಲಿ ಹಟ ಹುಟ್ಟಿಬಿಟ್ಟಿತ್ತು. ಈ ದೋಣಿಯವರಿಗೆ ಪಾಠ ಕಲಿಸಲೇಬೇಕು ಎಂದು. ಆದ್ದರಿಂದ ಹಾರಿದ ರಭಸಕ್ಕೆ ಅವನು ದಡದ ಮೇಲೇ ಬಿದ್ದ. ಕೂಡಲೇ ಎದ್ದು ಆ ಸೈನಿಕರ ಬಳಿಗೆ ಓಡಿದ; ತನ್ನ ಮುರುಕಲು ಇಂಗ್ಲಿಷಿನಲ್ಲೇ ನಡೆದ ವಿಷಯವನ್ನೆಲ್ಲ ಹೇಳಿದ. ಅವನು ಹೇಳಿದ್ದನ್ನು ಆ ಸಿಪಾಯಿಗಳು ಇನ್ನೂ ಅರ್ಥಮಾಡಿಕೊಳ್ಳಲು ಪ್ರಯತ್ನ ಮಾಡುತ್ತಿರುವಾಗಲೇ ತನ್ನ ಪುಟ್ಟ ಕೈಯಿಂದ ಅವರಲ್ಲೊಬ್ಬನ ಕೈಯನ್ನು ಹಿಡಿದುಕೊಂಡು ಇತ್ತ ಕರೆದುಕೊಂಡು ಬಂದ. ಅವರಿಗೆ ಅರ್ಥವಾಯಿತು– ಹುಡುಗರಿಗೆ ಆ ಅಂಬಿಗರಿಂದ ಏನೋ ಕಷ್ಟವಾಗಿರಬೇಕು ಎಂದು. ಆದ್ದರಿಂದ ಅವರು ನರೇಂದ್ರನಿಗೆ,“ಆಗಲಪ್ಪ ಆಗಲಿ, ಏನೂ ಹೆದರಿಕೊಳ್ಳಬೇಡ” ಎಂದು ಧೈರ್ಯ ಹೇಳಿದರು. ಬಳಿಕ ಇನ್ನಷ್ಟು ಹತ್ತಿರ ಬಂದು, ತಮ್ಮ ಸೈನಿಕಸಹಜವಾದ ಗಡಸುದನಿಯಲ್ಲಿ ಗುಡುಗಿದರು: “ಏಯ್! ಯಾರೋ ಅದು? ಆ ಹುಡುಗರನ್ನು ಈಗಲೇ ಬಿಟ್ಟರೆ ಸರಿಹೋಯಿತು; ಇಲ್ಲದೆ ಹೋದರೆ ನಿಮ್ಮನ್ನು ಶೂಟ್ ಮಾಡಿಬಿಡುತ್ತೇವೆ ನೋಡಿಕೊಳ್ಳಿ! ” ಆ ಅಂಬಿಗರಿಗೋ ಬಿಳಿಯರನ್ನು ಕಂಡರೇ ಭಯ. ಅದರಲ್ಲೂ ಇವರು ಸಿಪಾಯಿಗಳು. ಅಲ್ಲದೆ ಅವರು ಅರ್ಧ ಇಂಗ್ಲಿಷ್ ಅರ್ಧ ಹಿಂದಿ ಬೆರೆಸಿ ಗಡಸುದನಿಯಲ್ಲಿ ಕೂಗಿದಾಗ ಅದು ಇನ್ನಷ್ಟು ಭಯಂಕರವಾಗಿತ್ತು. ಅಂಬಿಗರು ಹೆದರಿ, ಹಲ್ಲು ಕಿರಿಯುತ್ತ “ಆ.... ಆಗಲಿ ಸಾರ್, ಈಗಲೇ ಬಂದ್ ಬಟ್ವಿ ಸಾರ್” ಎಂದರು. ಸರಿ, ಒಂದೇ ನಿಮಿಷದಲ್ಲಿ ದೋಣಿ ದಡಸೇರಿತು. ಸೈನಿಕರು ತಮ್ಮ ಕೆಂಗಣ್ಣು ತೋರಿಸುತ್ತ ಅಂಬಿಗರನ್ನೇ ದುರುಗುಟ್ಟಿಕೊಂಡು ನೋಡಿದಾಗ ಅವರ ಅವಸ್ಥೆ ಬೇಡ! ಅಂತೂ ಹುಡುಗರೆಲ್ಲ ಸುರಕ್ಷಿತವಾಗಿ ದಡ ಸೇರಿದರು. ಅಂಬಿಗರಿಗೆ ಮಾತ್ರ ಎರಡರಷ್ಟು ಬಾಡಿಗೆ ಬರುವುದೂ ತಪ್ಪಿಹೋಯಿತು, ವಾಂತಿಯನ್ನು ತೆಗೆಯುವ ಕೆಲಸವೂ ಅಂಟಿಕೊಂಡಿತು.

ಆ ಸೈನಿಕರು ನರೇಂದ್ರನ ವ್ಯಕ್ತಿತ್ವದಿಂದ ತುಂಬ ಆಕರ್ಷಿತರಾಗಿಬಿಟ್ಟಿದ್ದರು. “ ನಡೆ ನಮ್ಮ ಜೊತೆ ನಾಟಕಕ್ಕೆ ಬಾ” ಎಂದು ಅವನನ್ನು ಕರೆದರು. ಆದರೆ ಅವನು ಸ್ನೇಹಿತರನ್ನು ಬಿಟ್ಟುಹೋಗಲು ಇಷ್ಟಪಡಲಿಲ್ಲ. ಸೈನಿಕರಿಗೆ ಧನ್ಯವಾದಗಳನ್ನರ್ಪಿಸಿ ಅವರನ್ನು ಬೀಳ್ಗೊಂಡ. ಈ ಕಡೆ ಬಂದಮೇಲೆ ಸ್ನೇಹಿತರೆಲ್ಲ ವಿಸ್ಮಯಮೂಕರಾಗಿ ಅವನನ್ನೇ ಕಣ್ಣಗಲಿಸಿಕೊಂಡು ನೋಡಲಾರಂಭಿಸಿದರು. ಅವರು ಆ ಕಡೆ ಅಂಬಿಗರಿಂದಲೂ ಹೆದರಿಬಿಟ್ಟಿದ್ದರು, ಈ ಕಡೆ ಸೈನಿಕರಿಂದಲೂ ಭಯಗೊಂಡಿದ್ದರು. ನರೇಂದ್ರ ಅಂತಹ ಸೈನಿಕರನ್ನೇ ಕರೆತಂದು ತಮ್ಮನ್ನು ಕಷ್ಟದ ಪರಿಸ್ಥಿತಿಯಿಂದ ಪಾರುಮಾಡಿಬಿಟ್ಟನಲ್ಲ! ಆ ಹುಡುಗರು ತನ್ನನ್ನೇ ಬೆಪ್ಪಾಗಿ ನೋಡುವುದನ್ನು ಕಂಡು ನರೇಂದ್ರ ಕೇಳಿದ: “ಏನು, ಏನಾಯಿತು ನಿಮಗೆಲ್ಲ! ” ಆಗ ಒಬ್ಬ ಹುಡುಗ ಮೆಲ್ಲಗೆ ಚೇತರಿಸಿಕೊಂಡವನಂತೆ, “ಏನೋ ನರೇನ್, ಆ ಸಿಪಾಯಿಗಳ ಹತ್ತಿರ ಮಾತನಾಡುವುದಕ್ಕೆ ನಿನಗೆ ಹೇಗೆ ಸಾಧ್ಯವಾಯಿತೋ! ನೀನು ಹಾಗೆಲ್ಲ ಮಾತನಾಡಬಲ್ಲೆ ಅಂತ ಊಹಿಸುವುದಕ್ಕೂ ಸಾಧ್ಯವಿಲ್ಲ. ಒಂದು ವೇಳೆ ಅವರು ನಮ್ಮ ಮೇಲೇ ಹಾಗೆ ಕಣ್ಣು ಬಿಟ್ಟಿದ್ದರೆ ನಮ್ಮ ಗತಿ ಏನಾಗುತ್ತಿತ್ತೋ!” ಎಂದು. ನರೇಂದ್ರ ಸುಮ್ಮನೆ ನಕ್ಕುಬಿಟ್ಟ.

ನಿಜಕ್ಕೂ ಈ ಘಟನೆಯ ಸ್ವಾರಸ್ಯವನ್ನು ನಾವು ಪೂರ್ಣವಾಗಿ ಅರಿತು ಆನಂದಿಸಬೇಕಾದರೆ, ಅಂದಿನ ಕಾಲದಲ್ಲಿ ಭಾರತದ ಜನಸಾಮಾನ್ಯರು ಆ ಬಿಳಿಯರನ್ನು ಅದೆಷ್ಟು ಭಯಗೌರವಗಳಿಂದ ನೋಡುತ್ತಿದ್ದರೆ ಎನ್ನುವುದನ್ನು ನೆನಪಿಸಿಕಳ್ಳಬೇಕು. ಆಗ ಮಾತ್ರ ನರೇಂದ್ರ ತೋರಿದ ಎದೆಗಾರಿಕೆಯನ್ನು ನಾವು ಮೆಚ್ಚಿ ಪ್ರಶಂಸಿಸಲು ಸಾಧ್ಯ. ಹಾಗೆಯೇ ಆ ಹುಡುಗರೆಲ್ಲ ನರೇಂದ್ರನನ್ನು ತಮ್ಮ ನಾಯಕನನ್ನಾಗಿ ಪರಿಭಾವಿಸಿಸದ್ದರಲ್ಲೂ ಆಶ್ಚರ್ಯವಿಲ್ಲ ಎನ್ನುವುದು ಅರ್ಥವಾಗುತ್ತದೆ.

ನರೇಂದ್ರನೂ ಅವನ ಸ್ನೇಹಿತರೂ ನವಗೋಪಾಲ ಮಿತ್ರನ ವ್ಯಾಯಾಮಶಾಲೆಗೆ ಸೇರಿ ಅಂಗಸಾಧನೆಯನ್ನು ಆರಂಭಿಸಿದ್ದನ್ನು ನೋಡಿದೆವು. ಅವರೆಲ್ಲ ಇನ್ನೂ ಹನ್ನೆರಡು-ಹದಿನಾಲ್ಕು ವರ್ಷದ ಹುಡುಗರಷ್ಟೆ. ಆದರೂ ಅವರ ಉತ್ಸಾಹ ಶ್ರದ್ಧೆಗಳನ್ನು ಮೆಚ್ಚಿಕೊಂಡ ನವಗೋಪಾಲ ತನ್ನ ವ್ಯಾಯಮಶಾಲೆಯ ಪಾರುಪತ್ಯವನ್ನು ಅವರ ಕೈಗೆ ಬಿಟ್ಟುಬಿಟ್ಟ. ಒಂದು ದಿನ ಈ ಹುಡುಗರೆಲ್ಲ ಸೇರಿ ಆ ವ್ಯಾಯಾಮಶಾಲೆಯ ಎದುರಿನಲ್ಲಿ ಒಂದು ಉಯ್ಯಾಲೆಯ ವ್ಯವಸ್ಥೆ ಮಾಡುವುದಕ್ಕಾಗಿ ಭಾರವಾದ ತೊಲೆಯೊಂದನ್ನು ಹಗ್ಗ-ರಾಟೆಗಳ ಸಹಾಯದಿಂದ ಎತ್ತುತ್ತಿದ್ದರು. ಇಷ್ಟು ಪುಟ್ಟ ತರಳರು ಅಂಥಾ ದೊಡ್ಡ ತೊಲೆಯನ್ನು ಎತ್ತುವ ಸಾಹಸ ಮಾಡುತ್ತಿರುವುದನ್ನು ಕಂಡು ಅಲ್ಲಿ ಜನ ಸೇರಿಬಿಟ್ಟರು. ಆ ಗುಂಪಿನಲ್ಲಿ ಒಬ್ಬ ಬ್ರಿಟಿಷ್ ನಾವಿಕನೂ ಇದ್ದ. ನರೇಂದ್ರ ಆ ನಾವಿಕನನ್ನು ನೋಡಿದ. ಆತ ಒಳ್ಳೆ ಕಟ್ಟುಮಸ್ತಾಗಿರುವುದನ್ನು ಕಂಡು ಈ ತೊಲೆಯನ್ನು ಎತ್ತಲು ಅವನ ಸಹಾಯವನ್ನು ಕೋರಿದ. ಅದಕ್ಕೆ ನಾವಿಕ ಸಂತೋಷದಿಂದ ಒಪ್ಪಿಕೊಂಡು ಬಂದು ಕೈಹಾಕಿದ. ಆದರೆ ದುರದೃಷ್ಟಕ್ಕೆ, ಆ ಭಾರವಾದ ತೊಲೆಯನ್ನು ಎತ್ತುವಾಗ ಒಂದು ಹಗ್ಗ ಕಿತ್ತುಹೋಗಿ ಆ ತೊಲೆ ನಾವಿಕನ ತಲೆಗೆ ಜೋರಾಗಿ ತಗುಲಿತು. ಆತ ಪ್ರಜ್ಞೆ ತಪ್ಪಿ ಬಿದ್ದುಬಿಟ್ಟ. ಅವನು ಸತ್ತೇಹೋದನೆಂದು ಭಾವಿಸಿ ಹುಡುಗರೆಲ್ಲ ಹೆದರಿ ಪರಾರಿಯಾದರು. ಸುತ್ತ ನೆರೆದಿದ್ದ ಜನರೂ ಮಂಗಮಾಯ! ಕಡೆಗೆ ಅಲ್ಲಿ ಉಳಿದುಕೊಂಡಿದ್ದವರು ನರೇಂದ್ರ ಹಾಗೂ ಅವನ ಒಂದಿಬ್ಬರು ಸ್ನೇಹಿತರು ಮಾತ್ರ. ನರೇಂದ್ರ ಸ್ವಲ್ಪವೂ ತಡಮಾಡದೆ ತನ್ನ ಧೋತಿಯನ್ನೇ ಹರಿದು, ಒಂದುತುಂಡನ್ನು ನಾವಿಕನ ಗಾಯಕ್ಕೆ ಕಟ್ಟಿದ. ಬಳಿಕ ಮುಖದ ಮೇಲೆ ತಣ್ಣೀರನ್ನು ಸಿಂಪಡಿಸಿ, ಮೆಲ್ಲನೆ ಗಾಳಿ ಬೀಸಿದ. ಮತ್ತು ವೈದ್ಯರಿಗೂ ವ್ಯಾಯಾಮಶಾಲೆಯ ಮಾಲಿಕನಾದ ನವಗೋಪಾಲ ಮಿತ್ರನಿಗೂ ಹೇಳಿಕಳಿಸಿದ. ನಾವಿಕ ಸ್ವಲ್ಪ ಚೇತರಿಸಿಕೊಂಡು ಎದ್ದು ಕುಳಿತಾಗ, ನರೇಂದ್ರ ತನ್ನ ಸ್ನೇಹಿತರ ಸಹಾಯದಿಂದ ಅವನನ್ನು ಹತ್ತಿರದಲ್ಲೇ ಇದ್ದ ಒಂದು ಶಾಲೆಯ ಕಟ್ಟಡಕ್ಕೆ ಕರೆತಂದ. ಒಂದು ವಾರದ ಕಾಲ ಸೇವೆ-ಶುಶ್ರೂಷೆಗಳನ್ನು ಮಾಡಿದ ಮೇಲೆ ಆತ ಎಂದಿನಂತಾದ. ಅನಂತರ ನರೇಂದ್ರ ತನ್ನ ಎಲ್ಲ ಸ್ನೇಹಿತರಿಂದಲೂ ವಂತಿಗೆ ವಸೂಲು ಮಾಡಿ ಒಂದಿಷ್ಟು ಹಣವನ್ನು ಸೇರಿಸಿ, ನಾವಿಕನಿಗೆ ಕೊಟ್ಟು ಆತನ ಸಹಾಯ-ಸಹಕಾರಗಳಿಗಾಗಿ ಕೃತಜ್ಞತೆಯನ್ನರ್ಪಿಸಿ ಕಳಿಸಿಕೊಟ್ಟ.

ಇಲ್ಲಿ, ತನ್ನ ಸಹಾಯಕ್ಕೆ ಒದಗಿಬಂದ ಆ ಮನುಷ್ಯ ಕಷ್ಟಕ್ಕೆ ಗುರಿಯಾದಾಗ ಅವನ ಕಷ್ಟ ಪರಿಹಾರವಾಗುವವರೆಗೂ ಅವನ ಜೊತೆಯಲ್ಲೇ ಇದ್ದು ಉಪಚರಿಸಿದ ನರೇಂದ್ರನ ಸೇವಾಪರತೆ ಹಾಗೂ ತ್ಯಾಗಬುದ್ಧಿ, ಧೈರ್ಯ-ಔದಾರ್ಯಗಳು ಭಾವೀ ಮಹಾಪುರುಷನ ಪೂರ್ವಸೂಚಿಗಳು.

ಇನ್ನೊಂದು ಸಂದರ್ಭ. ಒಂದು ನಾಟಕ ನಡೆಯುತ್ತಿದೆ; ಬಹಳ ಜನ ಕುಳಿತುಕೊಂಡು ತುಂಬ ಆಸಕ್ತಿಯಿಂದ ನೋಡುತ್ತಿದ್ದರೆ. ನರೇಂದ್ರನೂ ಪ್ರೇಕ್ಷಕರಲ್ಲೊಬ್ಬ. ನಾಟಕ ಭರದಿಂದ ಸಾಗಿದೆ. ಈ ಹೊತ್ತಿಗೆ ಇದ್ದಕ್ಕಿದ್ದಂತೆ ನ್ಯಾಯಾಲಯದ ಒಬ್ಬ ಅಮೀನ ನಾಟಕದ ಮುಖ್ಯ ಪಾತ್ರಧಾರಿಯನ್ನು ದಸ್ತಗಿರಿ ಮಾಡಲು ವಾರಂಟು ಹಿಡಿದುಕೊಂಡು ಬಂದ. ಆ ವ್ಯಕ್ತಿ ರಂಗಮಂಟಪದಲ್ಲಿರುವುದನ್ನು ಕಂಡು ನೇರವಾಗಿ ಅಲ್ಲಿಗೆ ನುಗ್ಗಿಬಿಟ್ಟ! ಬಹುಶಃ ಆತ ಏನೋ ಅಪರಾಧ ಮಾಡಿರಬೇಕು; ಆದ್ದರಿಂದ ಅವನನ್ನು ದಸ್ತಗಿರಿ ಮಾಡಬೇಕಾಗಿ ಬಂದಿರಬಹುದು. ಆದರೆ ನಾಟಕ ಅಷ್ಟುರಸವತ್ತಾಗಿ ನಡೆಯುತ್ತಿರುವಾಗ ಇವನು ಹಿಂದುಮುಂದು ನೋಡದೆ ಒಳಗೆ ನುಗ್ಗಿದನಲ್ಲದೆ, ಆ ಪಾತ್ರಧಾರಿಗೆ “ನಿನ್ನನ್ನು ದಸ್ತಗಿರಿ ಮಾಡಲಾಗಿದೆ” ಎಂದು ತಿಳಿಸಿದ. ಈಗ ತಾವೇನು ಮಾಡುವುದೆಂದು ತಿಳಿಯದೆ ಜನರೆಲ್ಲ ತಬ್ಬಿಬ್ಬಾದರು. ಆದರೆ ಇದ್ದಕ್ಕಿದಂತೆ ಸಭೆಯ ಮಧ್ಯದಿಂದ ಒಂದು ಧ್ವನಿ ಮೊಳಗಿತು: “ಹೋಯ್! ಬಾರಯ್ಯ ಈ ಕಡೆ! ನಾಟಕ ಮುಗಿಯುವ ತನಕ ಕಾಯುತ್ತಿರು. ಮಧ್ಯೆ ಬಂದು ಇಡೀ ಸಭೆಗೆ ತೊಂದರೆ ಕೊಡುವುದು ಯಾವ ನ್ಯಾಯ ಇದು!” ಅದೊಂದು ಕೀರಲು ಧ್ವನಿ; ಆದರೆ ಅದರಲ್ಲೊಂದು ಆಜ್ಞಾಪನೆ ಮಾಡುವ ಗತ್ತು ಇತ್ತು. ಯಾರದ್ದಪ್ಪಾ ಆ ಧ್ವನಿ? ಇನ್ನಾರದೂ ಅಲ್ಲ, ನಮ್ಮ ನರೇಂದ್ರನದೇ! ತಕ್ಷಣವೇ ಸಭೆಯಲ್ಲಿ ಹಲವಾರು ಧ್ವನಿಗಳು ಕೂಗಿ ಕೋಂಡವು– “ಏಯ್! ಹೋಗಾಕಡೆ. ಯಾವನೋ ನೀನು ಸ್ಟೇಜಿನ ಮಧ್ಯೆ ಬಂದು? ಹೋಗಾಕಡೆ!” ಅಮೀನ ಕಣ್ಕಣ್ಣು ಬಿಡುತ್ತ ಕೆಳಗೆ ಬಂದ. ನಾಟಕ ಈಗ ಮುಂದುವರಿಯಿತು. ಆಗ ನರೇಂದ್ರನ ಸುತ್ತ ಕುಳಿತಿದ್ದವರೆಲ್ಲ ಅವನ ಬೆನ್ನು ತಟ್ಟಿ, “ಭೇಷ್ ಭೇಷ್! ಬಹಳ ಒಳ್ಳೆಯ ಕೆಲಸ ಮಾಡಿದೆ ಕಣೊ. ನೀನು ಹಾಗೆ ಕೂಗದೇ ಇದ್ದಿದ್ದರೆ ಎಲ್ಲ ಹಾಳಾಗಿಹೋಗುತ್ತಿತ್ತು” ಎಂದು ಅಭಿನಂದಿಸಿದರು. ನಿಜ, ಯಾರಾದರೊಬ್ಬರು ಧೈರ್ಯಮಾಡಿ ಪರಿಸ್ಥಿತಿಯನ್ನು ಕೈಗೆ ತೆಗೆದುಕೊಳ್ಳದೆ ಹೋದರೆ ಎಲ್ಲವೂ ಹಾಳಾಗುವ ಸಂಭವ.

