
\chapter{ಕಣ್ದೆರೆಸಿದ ಅನುಭವ}

\noindent

ಮಠದಲ್ಲಿ ಸುಮಾರು ಒಂದು ವರ್ಷವನ್ನು ಕಳೆದ ಸ್ವಾಮೀಜಿ, ೧೮೮೯ ಡಿಸೆಂಬರ್ ತಿಂಗಳಲ್ಲಿಅ ಬಾರಾನಗೋರನ್ನು ಬಿಟ್ಟು, ವೈದ್ಯನಾಥದ ದಾರಿಯಾಗಿ ಉತ್ತರಭಾರತದ ಕಡೆಗೆ ತೀರ್ಥಾಟನೆಗಾಗಿ ಹೊರಟೇಬಿಟ್ಟರು. ಅವರ ಮನಸ್ಸು ವಾರಾಣಸಿಯ ಕಡೆಗೇ ವಿಶೇಷವಾಗಿ ವಾಲಿಕೊಂಡಿತ್ತು. ವೈದ್ಯನಾಥದಿಂದ ಅವರು ಪ್ರಮದದಾಸ ಮಿತ್ರರಿಗೊಂದು ಪತ್ರ ಬರೆಯುತ್ತಾರೆ: “ಕೆಲವು ದಿನ ವಾರಾಣಸಿಯಲ್ಲಿದ್ದುಕೊಂಡು, ನನ್ನ ವಿಷಯದಲ್ಲಿ ವಿಶ್ವನಾಥ ಅನ್ನಪೂರ್ಣೆಯರು ಏನು ಸಂಕಲ್ಪ ಮಾಡಿದ್ದಾರೋ ನೋಡಬೇಕು ಎಂಬುದು ನನ್ನ ಉದ್ದೇಶ. ನನ್ನ ದೃಢ ನಿರ್ಧಾರವೇನೆಂದರೆ, ಒಂದೋ–ನನ್ನ ಜೀವನದ ಧ್ಯೇಯವನ್ನು ಕಾಣಬೇಕು; ಇಲ್ಲವೆ ಈ ಶರೀರವನ್ನು ತ್ಯಾಗಮಾಡಿ ಬಿಡಬೇಕು. ಆದ್ದರಿಂದ ಹೇ ಕಾಶೀನಾಥ, ನನಗೆ ನೆರವಾಗು!”

ಆದರೆ ದೈವೇಚ್ಛೆ ಬೇರೆಯೇ ಆಗಿತ್ತು. ವೈದ್ಯನಾಥದಲ್ಲಿ ಸ್ವಾಮೀಜಿಗೆ, ಅಲಹಾಬಾದಿನಲ್ಲಿದ್ದ ಸ್ವಾಮಿ ಯೋಗಾನಂದರಿಗೆ ಸೀತಾಳೆ ಸಿಡುಬು ತಗಲಿದೆಯೆಂಬ ಸುದ್ದಿ ತಲುಪಿತು. ಒಡನೆಯೇ ಸ್ವಾಮೀಜಿ ಅಲಹಾಬಾದಿನ ಕಡೆಗೆ ಹೊರಟರು. ಆದರೆ ಅವರು ಅಲ್ಲಿಗೆ ತಲುಪುವಷ್ಟರಲ್ಲಿ ಯೋಗಾನಂದರು ಸಂಪೂರ್ಣ ಗುಣ ಹೊಂದಿದ್ದರು. ಅಲಹಾಬಾದಿನಲ್ಲಿ ಸ್ವಾಮೀಜಿಯ ಸಂಪರ್ಕಕ್ಕೆ ಬಂದ ಹಲವಾರು ಬಂಗಾಳೀ ಜನರು, ಅವರ ಅಸಾಮಾನ್ಯ ವ್ಯಕ್ತಿತ್ವವನ್ನೂ ಅಪಾರ ಜ್ಞಾನವನ್ನೂ ಕಂಡು ಬೆರಗಾದರು. ಶೀಘ್ರದಲ್ಲೇ ಅವರ ಕೀರ್ತಿ ಹರಡಿ, ಆ ಊರಿನ ಬಂಗಾಳಿ ಗಳೆಲ್ಲ ಬರಲಾರಂಭಿಸಿದರು. ಅವರೊಂದಿಗಿನ ಮಾತುಕತೆಗಳಲ್ಲಿ ಸ್ವಾಮೀಜಿ, ಹಿಂದೂಧರ್ಮ ದಲ್ಲಿ ಆಚರಣೆಯಲ್ಲಿದ್ದ ಹಲವಾರು ಅನ್ಯಾಯಗಳನ್ನು, ಪಕ್ಷಪಾತಗಳನ್ನು ಎತ್ತಿತೋರಿಸಿ, ತೀಕ್ಷ್ಣ ವಾಗಿ ಟೀಕಿಸಿದರು. ಆದರೆ ಸನಾತನ ಹಿಂದೂ ಧರ್ಮದ ಹಿರಿಮೆ-ಮಹಿಮೆಗಳನ್ನು ಅಷ್ಟೇ ಬಲವಾಗಿ, ಮನಮುಟ್ಟುವಂತೆ ಹೇಳಿದರು. ಅವರ ಅಯಸ್ಕಾಂತೀಯ ವ್ಯಕ್ತಿತ್ವದಿಂದ ಆಕರ್ಷಿತ ರಾದ ಆ ಬಂಗಾಳೀ ಜನರು ಅವರನ್ನು ಒಂದು ತಿಂಗಳ ಮಟ್ಟಿಗಾದರೂ ತಮ್ಮೊಂದಿಗೆ ಉಳಿದು ಕೊಳ್ಳುವಂತೆ ಒತ್ತಾಯಿಸಿ ಒಪ್ಪಿಸಿದರು.

ಈ ದಿನಗಳಲ್ಲೇ ಸ್ವಾಮೀಜಿ ಪವಾಹಾರಿ ಬಾಬಾ ಎಂಬ ಪ್ರಖ್ಯಾತ ಸಂತರ ಬಗ್ಗೆ ಕೇಳಿದರು. ಅವರಿದ್ದುದು ಘಾಜೀಪುರದಲ್ಲಿ. ಹಿಂದೆ ದಕ್ಷಿಣೇಶ್ವರದಲ್ಲಿದ್ದಾಗಲೇ ಸ್ವಾಮೀಜಿ ಅವರ ವಿಷಯ ವಾಗಿ ಕೇಳಿ ಅವರನ್ನು ನೋಡಲು ಬಯಸಿದ್ದರು. ಆದ್ದರಿಂದ ಈಗ ಸ್ವಾಮೀಜಿ ಘಾಜೀಪುರಕ್ಕೆ ಹೊರಟರು.

ಇಲ್ಲಿ ಅವರು ಸತೀಶಚಂದ್ರ ಮುಖರ್ಜಿ ಹಾಗೂ ಗಗನಚಂದ್ರ ರಾಯ್ ಎಂಬವರ ಮನೆಗಳಲ್ಲಿ ಉಳಿದುಕೊಂಡರು. ಸತೀಶಚಂದ್ರ ಸ್ವಾಮೀಜಿಯ ಹಳೆಯ ನಿಕಟ ಸ್ನೇಹಿತ; ಶ್ರೀ ರಾಮಕೃಷ್ಣರ ಗೃಹೀಭಕ್ತನಾದ ಈಶಾನಚಂದ್ರ ಮುಖರ್ಜಿಯ ಮಗ (ಈಶಾನನೀಗ ತನ್ನ ಮಗನೊಂದಿಗೇ ವಾಸಿಸುತ್ತಿದ್ದ.) ಸ್ವಾಮೀಜಿ ಇವನ ಮನೆಯಲ್ಲಿದ್ದಾಗ ಅವರನ್ನು ನೋಡಲು ಅನೇಕ ಜನ ಬರಲಾರಂಭಿಸಿದರು. ಆದರೆ ಆ ಜನರೆಲ್ಲ ಹಿಂದೂ ಸಂಸ್ಕೃತಿಯಿಂದ ಜಾರಿ ಪಾಶ್ಚಾತ್ಯರ ಥಳಕಿನ ಸಂಸ್ಕೃತಿಗೆ ಬಲಿಯಾಗಿರುವುದನ್ನು ಕಂಡು ಸ್ವಾಮೀಜಿಗೆ ತುಂಬ ಸಂಕಟ ವಾಯಿತು. ಅವರು ಪ್ರಮದದಾಸ ಮಿತ್ರರಿಗೆ ಇದನ್ನು ತಿಳಿಸುತ್ತ ಬೆರೆಯುತ್ತಾರೆ: “... ಇದೊಂದು ಶೋಚನೀಯ ಅಂಶ. ಭಾರತೀಯರು ಪಾಶ್ಚಾತ್ಯರ ಪ್ರಭಾವಕ್ಕೊಳಗಾಗುವುದನ್ನು ನಾನು ತೀವ್ರ ವಾಗಿ ಖಂಡಿಸುತ್ತೇನೆ. ಅದೃಷ್ಟಕ್ಕೆ ನನ್ನ ಸ್ನೇಹಿತ ಸತೀಶಚಂದ್ರ ಆ ಕಡೆಗೆ ಅಷ್ಟಾಗಿ ವಾಲಿಕೊಂಡಿಲ್ಲ. ಈ ಪಾಶ್ಚಾತ್ಯರು ನಿರ್ಮಾಣ ಮಾಡಿರುವ ನಾಗರಿಕತೆ ಎಷ್ಟು ತುಚ್ಛ-ಹೀನ! ಇವರು ಎಂತಹ ವಿಲಾಸ ಜೀವನದ ಭ್ರಮೆಯನ್ನು ತಮ್ಮೊಂದಿಗೆ ತಂದಿದ್ದಾರೆ! ನಮ್ಮ ದುರ್ಬಲ ಜನತೆಯನ್ನು ವಿಶ್ವನಾಥನೇ ರಕ್ಷಿಸಬೇಕು...!” ಪತ್ರದ ಕೊನೆಯಲ್ಲಿ ಮತ್ತೊಂದು ಸಾಲನ್ನು ಸೇರಿಸುತ್ತಾರೆ: “ಅಯ್ಯೋ ವಿಧಿವೈಚಿತ್ರ್ಯವೇ! ಯಾವ ರಾಷ್ಟ್ರದಲ್ಲಿ ಭಗವಾನ್ ಶುಕಮಹರ್ಷಿಗಳು ಜನ್ಮ ತಾಳಿದರೋ, ಆ ರಾಷ್ಟ್ರದ ಜನರೇ ಇಂದು ತ್ಯಾಗಜೀವನವನ್ನು ಹುಚ್ಚುತನ, ಪಾಪ ಎಂಬಂತೆ ನೋಡುತ್ತಿದ್ದಾರೆ!”

ಇದೇ ಊರಿನಲ್ಲಿ, ಪಾಶ್ಚಾತ್ಯ ಸಂಸ್ಕೃತಿಯ ಆಕ್ರಮಣವನ್ನು ತಡೆಗಟ್ಟುವ ಪ್ರಯತ್ನದಲ್ಲಿ ತೊಡಗಿದ್ದ ಕೆಲವು ಸಮಾಜ ಸುಧಾರಕರೂ ಇದ್ದರು. ಇವರನ್ನು ಉದ್ದೇಶಿಸಿ ಸ್ವಾಮೀಜಿ ಹೇಳು ತ್ತಾರೆ: “ಪಾಶ್ಚಾತ್ಯ ಸಂಸ್ಕೃತಿಯನ್ನು ಉಗ್ರವಾಗಿ ಖಂಡಿಸಲು ಹೋಗಬೇಡಿ. ಸಹನೆಯಿಂದ, ಪ್ರೀತಿ ಯಿಂದ ಮುಂದುವರಿಯಿರಿ. ಎಲ್ಲರಿಗೂ ಹಿಂದೂ ಸಂಸ್ಕೃತಿಯ ಕುರಿತು ಸರಿಯಾದ ಶಿಕ್ಷಣ ಕೊಡಿ; ಸರಿಯಾದ ತಿಳಿವಳಿಕೆಯನ್ನು ಉಂಟುಮಾಡಿ. ಇದರಿಂದಾಗಿ ಜನರ ಅಂತರಂಗದಲ್ಲೇ ಒಂದು ಪರಿವರ್ತನೆಯಾಗುತ್ತದೆ. ಇದೇ ಸಹಜವಾದ ಪರಿವರ್ತನೆ. ಏನು ಮಾಡುವುದು! ನಮ್ಮ ಜನ, ಹಿಂದೂ ನಾಗರಿಕತೆಯು ಹಿನ್ನೆಲೆಯಲ್ಲಿರುವ ಆಧ್ಯಾತ್ಮಿಕ ಮೌಲ್ಯಗಳ ಮಹಿಮೆಯನ್ನು ಅರಿಯಲು ಬೇಕಾದ ಅಂತರ್ದೃಷ್ಟಿಯನ್ನೇ ಕಳೆದುಕೊಂಡುಬಿಟ್ಟಿದ್ದಾರೆ.” ಇಲ್ಲಿ ಸ್ವಾಮೀಜಿ, ಸಮಾಜ ಸುಧಾರಣೆಯ ಉತ್ಸಾಹದಲ್ಲಿ ಅವಿವೇಕಕ್ಕೆ ಅವಕಾಶವಾಗದಂತೆ ಸುಧಾರಕರಿಗೆ ನೀಡಿದ ಸೂಚನೆಗಳು ವಿಚಾರಯೋಗ್ಯವಾಗಿವೆ. ಆಗಿನ್ನೂ ನಮ್ಮ ರಾಷ್ಟ್ರ ಪಾಶ್ಚಾತ್ಯರ ಅಧೀನದಲ್ಲಿತ್ತು. ಭಾರತೀಯರೆಲ್ಲ ಪಾಶ್ಚಾತ್ಯರ ಮೆರುಗಿಗೆ ಮಾರುಹೋಗಿಯಾಗಿತ್ತು. ಈ ಸಂದರ್ಭದಲ್ಲಿ ಸುಧಾರಕ ರೆನ್ನಿಸಿಕೊಂಡವರು ಪಾಶ್ಚಾತ್ಯ ಸಂಸ್ಕೃತಿಯನ್ನು ಕಟುವಾಗಿ ಟೀಕಿಸುತ್ತಿದ್ದರೆ ಜನತೆಯ ಅವ ಗ್ರಹಕ್ಕೂ ಪಾತ್ರರಾಗಬೇಕಾಗುತ್ತದೆ; ಕಾರ್ಯಸಿದ್ಧಿಯೂ ಆಗುವುದಿಲ್ಲ. ಆದ್ದರಿಂದಲೇ ಸ್ವಾಮೀಜಿ ಅವರಿಗೆ ಸಹನೆಯಿಂದ ಮುಂದುವರಿದು, ಜನಮನದಲ್ಲಿ ಸಹಜವಾದ ಪರಿವರ್ತನೆ ಯನ್ನು ಉಂಟುಮಾಡುವಂತೆ ತಿಳಿಯಹೇಳುತ್ತಾರೆ.

ನಾವಾಗಲೇ ನೋಡಿದಂತೆ, ಸ್ವಾಮೀಜಿ ಘಾಜೀಪುರಕ್ಕೆ ಬಂದದ್ದರ ಮುಖ್ಯ ಉದ್ದೇಶ ಪ್ರಖ್ಯಾತ ಸಂತ ಪವಾಹಾರಿ ಬಾಬಾರನ್ನು ಸಂದರ್ಶಿಸುವುದಾಗಿತ್ತು. ಈ ಪವಾಹಾರಿ ಬಾಬಾರ ಬಗ್ಗೆ ಇಲ್ಲಿ ಸಂಕ್ಷಿಪ್ತವಾಗಿ ಹೇಳಬಹುದು. ಇವರು ಒಬ್ಬ ಬಡ ಬ್ರಾಹ್ಮಣ ದಂಪತಿಗಳ ಪುತ್ರ. ಜನ್ಮಸ್ಥಳ ವಾರಾಣಸಿಯ ಬಳಿಯಿರುವ ಪ್ರೇಮಪುರ ಎಂಬ ಗ್ರಾಮ. ಇವರು ತಮ್ಮ ಬಾಲ್ಯದಲ್ಲೇ ಘಾಜೀಪುರಕ್ಕೆ ಹೋಗಿ, ಆಜನ್ಮ ಬ್ರಹ್ಮಚಾರಿಯಾಗಿದ್ದ ತಮ್ಮ ಚಿಕ್ಕಪ್ಪನ ಮಾರ್ಗದರ್ಶನದಲ್ಲಿ ವ್ಯಾಕರಣ, ನ್ಯಾಯ ಹಾಗೂ ವಿಶಿಷ್ಟಾದ್ವೈತ ವೇದಾಂತಗಳನ್ನು ಕಲಿತು, ಅವುಗಳಲ್ಲಿ ನಿಷ್ಣಾತ ರಾದರು. ಕಾಲಕ್ರಮದಲ್ಲಿ ಅವರ ಚಿಕ್ಕಪ್ಪ ಕಾಲವಶನಾದ. ಚಿಕ್ಕಪ್ಪನ ಮರಣದಿಂದ ಅವರ ಹೃದಯದಲ್ಲಿ ದೊಡ್ಡದೊಂದು ಶೂನ್ಯತೆಯೇ ಉಂಟಾಗಿಬಿಟ್ಟಿತು. ಬಳಿಕ, ಮನಸ್ಸನ್ನು ಗಟ್ಟಿ ಮಾಡಿಕೊಂಡು, ಆ ಶೂನ್ಯತೆಯನ್ನು ತುಂಬಿಕೊಳ್ಳಲು ಅವರು ನಿರ್ಧರಿಸಿದರು. ಪವಾಹಾರಿ ಬಾಬಾ ಬಾಲ್ಯದಿಂದಲೂ ಆಧ್ಯಾತ್ಮಿಕ ಪ್ರವೃತ್ತಿಯುಳ್ಳವರು; ವೈರಾಗ್ಯಶಾಲಿ. ಆದ್ದರಿಂದ, ಯಾವುದರ ಅನುಭವವಾದರೆ ಇನ್ನಾವ ಕೊರತೆಯೂ ಉಳಿಯುವುದಿಲ್ಲವೋ, ಯಾವುದು ಎಂದೆಂದಿಗೂ ಇರುವಂಥದೋ, ಯಾವುದು ಸಕಲ ಶೂನ್ಯವನ್ನೂ ನಿಜವಾಗಿಯೂ ‘ತುಂಬ’ಬಲ್ಲ ಪೂರ್ಣವೋ, ಅಂಥದನ್ನು ಸಾಕ್ಷಾತ್ಕರಿಸಿಕೊಳ್ಳುವುದರ ಮೂಲಕ ತಮ್ಮ ಹೃದಯದ ಶೂನ್ಯತೆಯನ್ನು ತುಂಬಿ ಕೊಳ್ಳಲು ಬಯಸಿದರು. ಈ ನಿರ್ಧಾರವನ್ನು ಮನದಲ್ಲಿ ಧಾರಣೆ ಮಾಡಿಕೊಂಡು ತಮ್ಮ ಆದರ್ಶ ಸಿದ್ಧಿಗಾಗಿ ನಾಡಿನ ತುಂಬೆಲ್ಲ ತಿರುಗಾಡಿದರು. ಕೊನೆಗೆ ಕಾಥೇವಾಡದ ಪ್ರಸಿದ್ಧ ತೀರ್ಥಕ್ಷೇತ್ರವಾದ ಗಿರಿನಾರ್ ಬೆಟ್ಟದ ಮೇಲೆ ವಾಸವಾಗಿದ್ದ ಒಬ್ಬ ಯೋಗಿಯಿಂದ ಯೋಗದೀಕ್ಷೆಯನ್ನು ಪಡೆದರು. ಬಳಿಕ ಅಲ್ಲಿಂದ ವಾರಾಣಸಿಗೆ ಬಂದರು. ಇಲ್ಲಿ ಅವರಿಗೆ ಗಂಗಾತೀರದ ಗವಿಯೊಂದರಲ್ಲಿ ವಾಸವಾಗಿದ್ದ ಸಂನ್ಯಾಸಿಯೊಬ್ಬನ ಭೇಟಿಯಾಯಿತು. ಆತನಿಂದ ಅವರು ಅದ್ವೈತ ವೇದಾಂತದಲ್ಲಿ ಪಾರಂಗತರಾದರು. ಅನಂತರ ಅವರು ಶಾಸ್ತ್ರಾಧ್ಯಯನ ಮಾಡುತ್ತ ಹಾಗೂ ತೀವ್ರತರವಾದ ತಪೋಜೀವನವನ್ನು ನಡೆಸುತ್ತ ದೇಶಸಂಚಾರವನ್ನು ಮುಗಿಸಿ, ತಮ್ಮ ಹಳೆಯ ಊರಾದ ಘಾಜೀಪುದಕ್ಕೆ ಹಿಂದಿರುಗಿ ಬಂದು ಅಲ್ಲಿಯೇ ನೆಲೆ ನಿಂತರು. ಕಾಶಿಯಲ್ಲಿನ ತಮ್ಮ ಗುರುವಿನ ಮೇಲ್ಪಂಕ್ತಿಯನ್ನು ಅನುಸರಿಸಿ, ತಾವೂ ನದೀತೀರದಲ್ಲಿ ಭೂಮಿಯನ್ನು ಅಗೆದು ಒಂದು ಗುಹೆಯನ್ನು ಮಾಡಿಕೊಂಡು ವಾಸಿಸಲಾರಂಭಿಸಿದರು. ದಿನದ ಹೆಚ್ಚಿನ ವೇಳೆಯೆಲ್ಲ ಅವರ ಪ್ರಿಯದೇವನಾದ ಶ್ರೀರಾಮಚಂದ್ರನ ಪೂಜೆಯಲ್ಲಿ ಕಳೆಯುತ್ತಿತ್ತು. ಇನ್ನು ರಾತ್ರಿಯೆಲ್ಲ ಕಠಿಣ ತಪಸ್ಸಾಧನೆಗಳಲ್ಲಿ ನಿರತರಾಗುತ್ತಿದ್ದರು. ಕ್ರಮೇಣ ಅವರು ತಮ್ಮ ಆಹಾರದ ಪ್ರಮಾಣವನ್ನು ತಗ್ಗಿಸಿಕೊಂಡು ಬಂದರು. ಕಡೆಗೆ ಅದು ಎಲ್ಲಿಯವರೆಗೆ ಬಂದಿತೆಂದರೆ ಒಂದಿಷ್ಟು ಬೇವಿನ ಸೊಪ್ಪು ಅಥವಾ ಒಂದು ಹಿಡಿಮೆಣಸನ್ನು ತಿಂದುಕೊಂಡು ದಿನವೆಲ್ಲ ಇರುತ್ತಿದ್ದರು! ಇದನ್ನು ಕಂಡೇ ಜನ ಅವರನ್ನು ‘ಪವಾಹಾರಿ ಬಾಬಾ ( ಎಂದರೆ ಗಾಳಿ ತಿಂದು ಬದುಕುವ ಸಂನ್ಯಾಸಿ) ಎಂದು ಕರೆಯಲಾರಂಭಿಸಿದರು. ದಿನ ಕಳೆದಂತೆಲ್ಲ ಅವರು ಹೆಚ್ಚುಹೆಚ್ಚು ಸಮಯ, ಕೆಲವೊಮ್ಮೆ ತಿಂಗಳುಗಟ್ಟಲೆ ಕಾಲ, ತಮ್ಮ ಗುಹೆಯಲ್ಲೇ ಧ್ಯಾನನಿರತರಾಗಿರಲಾರಂಭಿಸಿದರು. ಆ ಗುಹೆ ಕಪ್ಪೆ-ಹಾವುಗಳಿಗೂ ಆವಾಸಸ್ಥಾನವಾಗಿಬಿಟ್ಟಿತ್ತು. ಎಷ್ಟೋ ಸಲ ಜನರು ಬಾಬಾ ಸತ್ತುಹೋಗಿರ ಬೇಕು ಎಂದೇ ಬಾವಿಸಿದ್ದುಂಟು. ಆದರೆ ಕೆಲವು ದಿನಗಳ ಬಳಿಕ ಬಾಬಾ ಹೊರಕ್ಕೆ ಬಂದು ಕಾಣಿಸಿಕೊಳ್ಳುತ್ತಿದ್ದರು. ಧ್ಯಾನದಿಂದೆದ್ದು ಬಂದ ಸಮಯಗಳಲ್ಲಿ ಅವರು ತಮ್ಮನ್ನು ಕಾಣಬಂದ ಸಂದರ್ಶಕರನ್ನು ತಮ್ಮ ಗುಹೆಯ ಪ್ರವೇಶದ್ವಾರದ ಬಳಿಯಲ್ಲಿದ್ದ ಕೋಣೆಯೊಂದರಲ್ಲಿ ಭೇಟಿ ಮಾಡುತ್ತಿದ್ದರು. ಆದರೆ ಈ ಸಂದರ್ಶನಕಾಲದ ಬಳಿಕ ಅವರು ಯಾರನ್ನೂ ಭೇಟಿಮಾಡುತ್ತಿರ ಲಿಲ್ಲ. (ಈ ದಿನಗಳಲ್ಲೇ ಸ್ವಾಮೀಜಿ ಅವರನ್ನು ಭೇಟಿ ಮಾಡಿದ್ದು.) ಬಾಬಾಜಿ ಜ್ಞಾನಮಾರ್ಗಿ ಯಾದರೂ ಹವನ-ಹೋಮಗಳನ್ನೂ ಮಾಡುತ್ತಿದ್ದರು. ಮೇಲ್ಗಡೆಯ ಕೋಣೆಯಿಂದ ಹೊಗೆ ಬರುತ್ತಿದ್ದರೆ, ಅವರು ಸಮಾಧಿಯಿಂದೆದ್ದು ಹೋಮದಲ್ಲಿ ನಿರತರಾಗಿದ್ದಾರೆ ಎಂದು ಜನರು ತಿಳಿಯುತ್ತಿದ್ದರು. ಕೊನೆಗೊಂದು ದಿನ (೧೮೯೮ರಲ್ಲಿ) ಆ ಕೋಣೆಯಿಂದ ಮಾಂಸ ಸುಟ್ಟ ವಾಸನೆ ಬರುತ್ತಿರುವುದನ್ನು ಕಂಡ ಜನ ಒಳಗೆ ಹೋಗಿ ನೋಡುತ್ತಾರೆ–ಪವಾಹಾರಿ ಬಾಬಾ ಕೊನೆಯ ಆಹುತಿಯಾಗಿ ಬೆಂಕಿಯಲ್ಲಿ ತಮ್ಮ ದೇಹವನ್ನೇ ಭಗವಂತನಿಗೆ ಅರ್ಪಿಸಿಕೊಂಡುಬಿಟ್ಟಿದ್ದರು! ಅವರ ಆತ್ಮ ಚಿರಸಮಾಧಿಯಲ್ಲಿ ಲೀನವಾಗಿಬಿಟ್ಟಿತ್ತು. ಆದರೆ ಇದು ಒಂಬತ್ತು ವರ್ಷಗಳ ಅನಂತರದ ಮಾತು.

ಸ್ವಾಮೀಜಿ ಘಾಜೀಪುರದಲ್ಲಿದ್ದಾಗ ಮಹಾಪುರುಷರೆಂದು, ಹಠಯೋಗಿಯೆಂದು ಪ್ರಸಿದ್ಧ ರಾಗಿದ್ದ ಬಾಬಾಜಿಯವರ ಹೆಸರು ಮನೆಮಾತಾಗಿತ್ತು. ಅವರನ್ನು ಕಾಣಲು ಜನ ಧಾವಿಸಿ ಬರು ತ್ತಿದ್ದರು. ಆದ್ದರಿಂದ ಸ್ವಾಮೀಜಿಗೂ ಅವರನ್ನು ಕಾಣಬೇಕೆಂಬ ಆಸಕ್ತಿ ಹುಟ್ಟಿದ್ದರಲ್ಲಿ ಆಶ್ಚರ್ಯ ವೇನೂ ಇಲ್ಲ. ಈಗೇನೋ ಸ್ವಾಮಿ ವಿವೇಕಾನಂದರು ವಿಶ್ವವಿಖ್ಯಾತರು. ಅವರನ್ನು ಆಚಾರ್ಯ ನೆಂದು, ಶಿವಸ್ವರೂಪಿಯೆಂದು ಜನ ಪೂಜಿಸುತ್ತಾರೆ. ಆದರೆ ಆ ಸಮಯದಲ್ಲಿ ಅವರಿನ್ನೂ ೨೭ ವರ್ಷದ ಯುವಕ. ತಮ್ಮ ಮಹಿಮೆಯನ್ನು ತಾವೇ ಅರಿಯದಿರುವಷ್ಟು ಚಿಕ್ಕವರು. ಅಲ್ಲದೆ, ಮಹಾಪುರುಷರನ್ನು ಹುಡುಕಿಕೊಂಡು ಹೋಗುವ ಕುತೂಹಲ ಅವರಲ್ಲಿ ಸಹಜವಾಗಿಯೇ ಇತ್ತು. ಸ್ವಾಮೀಜಿ ಅನಂತರದ ದಿನಗಳಲ್ಲಿ ಹೇಳುತ್ತಿದ್ದರು–ತಾವು ಪವಾಹಾರಿ ಬಾಬಾರಿಗೆ ಬಹಳಷ್ಟು ಋಣಿ ಎಂದು. ಅಲ್ಲದೆ ತಾವು ತಮ್ಮ ಜೀವನದಲ್ಲಿ ಸಂದರ್ಶಿಸಿದ ಮಹಾಪುರುಷರಲ್ಲಿ ಇವರೊಬ್ಬರು ಎನ್ನುತ್ತಿದ್ದರು.

ಆದರೆ ಈ ಬಾಬಾರನ್ನು ಭೇಟಿ ಮಾಡುವುದೆಂದರೆ ಬಹಳ ಕಷ್ಟದ ಕೆಲಸವಾಗಿತ್ತು. ಅವರು ಯಾರೊಂದಿಗಾದರೂ ಮಾತನಾಡಿದರೂ ಕೇವಲ ಬಾಗಿಲವರೆಗೆ ಮಾತ್ರ ಬಂದು ಒಳಗಡೆ ಯಿಂದಲೇ ಮಾತನಾಡಿಸಿ, ಬಂದವರನ್ನು ಹೊರಗಿಂದ ಹೊರಗೇ ಕಳಿಸಿಬಿಡುತ್ತಿದ್ದರು. ಆದ್ದರಿಂದ ಬಹಳ ದಿನ ಕಳೆದರೂ ಸ್ವಾಮೀಜಿಗೆ ಅವರ ಭೇಟಿ ಆಗಲೇ ಇಲ್ಲ. ಇದರಿಂದ ಬೇಸತ್ತು, ಅವರನ್ನು ಭೇಟಿ ಮಾಡದೆಯೇ ಹೊರಟುಬಿಡುವ ಆಲೋಚನೆಯನ್ನೂ ಸ್ವಾಮೀಜಿ ಮಾಡಿದರು. ಆದರೆ ಕೊನೆಗೊಂದು ದಿನ ಸಂದರ್ಶನ ದೊರಕಿದಾಗ, ಅವರ ವ್ಯಕ್ತಿತ್ವ ಸ್ವಾಮೀಜಿಯ ಮೇಲೆ ಗಾಢ ಪ್ರಭಾವ ಬೀರಿಬಿಟ್ಟಿತು. ಬಾಬಾಜಿ ಕೂಡ ಸ್ವಾಮೀಜಿಯ ತೇಜೋಮಯ ವ್ಯಕ್ತಿತ್ವದಿಂದ ಆಕರ್ಷಿತರಾದರು. ಮತ್ತು ಅವರು ಶ್ರೀರಾಮಕೃಷ್ಣರ ಶಿಷ್ಯರೆಂಬುದನ್ನು ತಿಳಿದು ಬಹಳ ಸಂತಸ ಪಟ್ಟರು. ಈ ಸಂದರ್ಭದಲ್ಲಿ ಪ್ರಮದದಾಸ ಮಿತ್ರರಿಗೆ ಬರೆದ ಪತ್ರದಲ್ಲಿ ಸ್ವಾಮೀಜಿ ಹೇಳುತ್ತಾರೆ: “ನನ್ನ ಪರಮಾದೃಷ್ಟದ ಫಲವಾಗಿ ಬಾಬಾಜಿಯವರನ್ನು ಭೇಟಿ ಮಾಡಲು ಸಾಧ್ಯವಾಯಿತು. ನಿಜಕ್ಕೂ ಅವರೊಬ್ಬ ಮಹಾಮುನಿ. ಈ ನಾಸ್ತಿಕ ಯುಗದಲ್ಲಂತೂ ಇದೊಂದು ದೊಡ್ಡ ಅದ್ಭುತವೇ ಸರಿ. ಅವರನ್ನು ಭಕ್ತಿ-ಯೋಗಗಳಿಂದ ಒಡಮೂಡಿದ ಒಂದು ಮಹಾಶಕ್ತಿಯ ಶಿಖರ ಎನ್ನಬಹುದು. ಅವರ ಕೃಪೆಯಲ್ಲಿ ನಾನು ಶರಣಾಗಿದ್ದೇನೆ. ಅವರು ನನ್ನಲ್ಲಿ ಭರವಸೆ ಮೂಡಿಸಿ ದ್ದಾರೆ. ಕೆಲವೇ ಮಂದಿ ಅದೃಷ್ಟಶಾಲಿಗಳಿಗೆ ಮಾತ್ರ ದೊರಕಬಹುದಾದಂತಹ ಭರವಸೆ ಅದು. ನಾನು ಇನ್ನೂ ಕೆಲದಿನ ಇಲ್ಲೇ ಇರಬೇಕಂದು ಬಾಬಾಜಿಯವರ ಇಚ್ಛೆ. ಅವರು ನನ್ನ ಮೇಲೆ ಕೃಪೆ ಮಾಡಬಹುದು. ಆದ್ದರಿಂದ, ಅವರ ಆದೇಶದಂತೆ ನಾನು ಇನ್ನೂ ಕೆಲವು ಕಾಲ ಇಲ್ಲೇ ಉಳಿಯುತ್ತೇನೆ.”

ಗಗನ್ ಬಾಬುವಿನ ತೋಟದ ಮನೆಯಿದ್ದುದು ಪವಾಹಾರಿ ಬಾಬಾರ ಗುಹೆಯ ಹತ್ತಿರದಲ್ಲೇ. ಅಲ್ಲಿಗೆ ಹೋಗಿಬರಲು ಅನುಕೂಲವಾದ್ದರಿಂದ ಸ್ವಾಮೀಜಿ ಆ ಮನೆಯಲ್ಲೇ ಉಳಿದುಕೊಂಡು ಧ್ಯಾನಮಗ್ನರಾಗಿರತೊಡಗಿದರು. ಸ್ವಾಮೀಜಿ ಆಗ ಸುಮಾರು ಎರಡು ತಿಂಗಳಿಂದ ಸೊಂಟದ ವಾತರೋಗದಿಂದ ನರಳುತ್ತಿದ್ದರು. ಆದ್ದರಿಂದ ಬಾಬಾಜಿ ಅಷ್ಟು ಹತ್ತಿರದಲ್ಲಿದ್ದರೂ ಅವರನ್ನು ಭೇಟಿ ಮಾಡಲು ಕೆಲವೊಮ್ಮೆ ಸಾಧ್ಯವಾಗುತ್ತಿರಲಿಲ್ಲ. ಆದರೆ ಬಾಬಾಜಿ ಯಾರಾದರೊಬ್ಬರನ್ನು ಕಳಿಸಿ ಸ್ವಾಮೀಜಿಯ ದೇಹಸ್ಥಿತಿಯನ್ನು ವಿಚಾರಿಸಿ ತಿಳಿದುಕೊಳ್ಳುತ್ತಿದ್ದರು. ಈ ಸಮಯದಲ್ಲಿ ಸ್ವಾಮೀಜಿಗೆ ಅತಿಸಾರ ಬೇರೆ ಅಂಟಿಕೊಂಡು ಅವರ ಜೀರ್ಣಶಕ್ತಿಯೂ ಕ್ಷೀಣಗೊಂಡಿತು. ಆದರೂ ಬಾಬಾಜಿ ಮೂಡಿಸಿದ್ದ ಭರವಸೆಯಿಂದಾಗಿ ಅವರು ಅಲ್ಲೇ ಉಳಿದುಕೊಂಡರು.

ಪವಾಹಾರಿ ಬಾಬಾ ಒಬ್ಬ ಅದ್ಭುತ ವಿನಯಮೂರ್ತಿ. ಅವರು ಮಹಾಜ್ಞಾನಿಯೂ ಹೌದು. ಆದರೆ ಅವರೆಂದೂ ಧರ್ಮಬೋಧನೆ ಮಾಡುತ್ತಿರಲಿಲ್ಲ. ಸಂದರ್ಶಕರೇನಾದರೂ ಪ್ರಶ್ನೆಗಳನ್ನು ಕೇಳಿದರೆ. “ಈ ದಾಸನಿಗೇನು ಗೊತ್ತಿದೆ?” ಎನ್ನುತ್ತಿದ್ದರು. ಆದರೆ ಅವರು ಮಾತನಾಡಲಾರಂಭಿಸಿ ದರೆ ಮಾತಿನಲ್ಲಿ ಮಿಂಚು ಮಿನುಗುತ್ತಿತ್ತು. ಸ್ವಾಮೀಜಿ ಅವರನ್ನು ತಮಗೇನಾದರೂ ಹೇಳುವಂತೆ ಒತ್ತಾಯ ಮಾಡಿದಾಗಲೆಲ್ಲ, “ಇಲ್ಲಿ ಇನ್ನೂ ಕೆಲವು ದಿನ ಇರುವ ಕೃಪೆ ಮಾಡಿ” ಎನ್ನುತ್ತಿದ್ದರು. ಅವರ ಮಾತಿಗೆ ಮನ್ನಣೆಯಿತ್ತು ಸ್ವಾಮೀಜಿ ಅಲ್ಲಿನ ತಮ್ಮ ವಾಸ್ತವ್ಯವನ್ನು ಮುಂದುವರಿಸುತ್ತ ಬಂದರು.

ಈ ಸಂದರ್ಭದಲ್ಲಿ ಸ್ವಾಮೀಜಿ ಹಲವಾರು ದೈಹಿಕ ಹಾಗೂ ಮಾನಸಿಕ ಯಾತನೆಗಳಿಂದ ತೊಳಲುತ್ತಿದ್ದರು. ವಾತರೋಗದಿಂದ ಉಂಟಾದ ಸೊಂಟದ ನೋವು ಸಾಕಷ್ಟು ತೊಂದರೆ ಕೊಡುತ್ತಿತ್ತು. ಕೆಲವೊಮ್ಮೆ ಅದು ತೀರ ಅಸಹನೀಯವಾಗಿ ಚೀರುವಂತಾಗುತ್ತಿತ್ತು. ಇದೇ ಸಮಯದಲ್ಲಿ ಅವರಿಗೊಂದು ವರ್ತಮಾನ ತಲುಪಿತು– ಹೃಷೀಕೇಶದಲ್ಲಿ ಅಭೇದಾನಂದರು ಮತ್ತೆಮತ್ತೆ ಮಲೇರಿಯಾಗೆ ಗುರಿಯಾಗಿ ನರಳುತ್ತಿದ್ದಾರೆ ಎಂದು. ಈ ಸುದ್ದಿಯನ್ನು ಕೇಳಿ ಆಘಾತಗೊಂಡ ಸ್ವಾಮೀಜಿ ತಾವು ಅಲ್ಲಿಗೆ ಬರಬೇಕಾದ ಅಗತ್ಯವಿದೆಯೆ ಎಂದು ಕೇಳಿ ತಮ್ಮ ಸೋದರ ಸಂನ್ಯಾಸಿಗಳಿಗೆ ತಂತಿ ಕಳಿಸಿದರು. ಆದರೆ ಏಕೋ ಇದಕ್ಕೆ ಬಹಳ ದಿನಗಳವರೆಗೂ ಉತ್ತರ ಬರಲಿಲ್ಲ. ಈ ನಡುವೆ ಅವರು ಪ್ರಮದ ಬಾಬುವಿಗೆ ಬರೆಯುತ್ತಾರೆ: “ನಾನು ಈ ಮಾಯೆಯ ಬಲೆಯನ್ನು ನೇಯ್ದುಕೊಳ್ಳುತ್ತಿರುವುದನ್ನು ಕಂಡು ನಿಮಗೆ ನಗು ಬರಬಹುದು. ನಿಜಕ್ಕೂ ನನ್ನ ವರ್ತನೆ ಹಾಗೆಯೇ ಇದೆ. ಆದರೆ ಒಂದು ವಿಷಯವೇನೆಂದರೆ, ಸರಪಳಿಗಳಲ್ಲಿ ಕಬ್ಬಿಣದ ಸರಪಣಿ, ಚಿನ್ನದ ಸರಪಣಿ ಎಂದು ಎರಡು ವಿಧ. ಈ ಚಿನ್ನದ ಸರಪಣಿಯಿಂದ ಅನೇಕ ಸತ್ಕಾರ್ಯಗಳು ಸಾಧ್ಯವಾಗುತ್ತವೆ. ಅಲ್ಲದೆ ಸತ್ಫಲವನ್ನು ತಂದುಕೊಟ್ಟಮೇಲೆ ಅದು ತನ್ನಷ್ಟಕ್ಕೇ ಬಿದ್ದುಹೋಗುತ್ತದೆ. ನಿಜಕ್ಕೂ ನನ್ನ ಗುರುಪುತ್ರರು ನನಗೆ ಸೇವಾಪಾತ್ರರು. ಅವರ ವಿಷಯದಲ್ಲಿ ಮಾತ್ರವೇ ನನಗೆ ನನ್ನ ಕರ್ತವ್ಯ ಇನ್ನೂ ಸ್ವಲ್ಪ ಉಳಿದಿದೆ ಎಂದು ಅನ್ನಿಸುವುದು.” ಕೆಲವು ದಿನಗಳ ಬಳಿಕ ಸ್ವಾಮೀಜಿ ಅವರಿಗೆ ಮತ್ತೆ ಬರೆಯುತ್ತಾರೆ: “ನನ್ನದು ತೀಕ್ಷ್ಣ ವೇದಾಂತ ದೃಷ್ಟಿಕೋನ ವಾದರೂ ನನ್ನ ಹೃದಯ ಮಾತ್ರ ತೀರ ಮೃದು ಎಂಬುದು ನಿಮಗೆ ಗೊತ್ತಿರಲಾರದು; ಮತ್ತು ಇದೇ ನನ್ನಲ್ಲೊಂದು ದುರ್ಬಲ ಅಂಶ. ನನ್ನ ಹೃದಯವನ್ನು ಸ್ವಲ್ಪ ಮುಟ್ಟಿದರೂ ಸಾಕು, ಕರಗಿಹೋಗುತ್ತೇನೆ. ನನ್ನ ವೈಯಕ್ತಿಕ ಒಳಿತನ್ನು ಸಾಧಿಸಿಕೊಳ್ಳಬೇಕೆಂದು ನಾನೆಷ್ಟೇ ಆಲೋಚಿಸಿ ದರೂ, ನನ್ನ ಇಚ್ಛೆಯನ್ನೂ ಮೀರಿ ನಾನು ಇತರರ ಒಳಿತಿನ ಬಗೆಗೆ ಚಿಂತಿಸಲಾರಂಭಿಸುತ್ತೇನೆ.” ಸಿಂಹಗರ್ಜನೆ ಮಾಡುವ ಸ್ವಾಮಿ ವಿವೇಕಾನಂದರ ಹೃದಯ ವಾಸ್ತವವಾಗಿ ಎಷ್ಟು ಮೃದು ಎಂಬುದು ಇಲ್ಲಿ ಅವರ ಮಾತಿನಿಂದಲೇ ವ್ಯಕ್ತವಾಗುತ್ತದೆ. ಈ ಸಲ ಅವರು ಬಾರಾನಗೋರ್ ಮಠದಿಂದ ಹೊರಟಾಗ ತಮ್ಮ ಯೋಜನೆಗಳನ್ನು ಸಾಧಿಸಿಯೇ ತೀರುವುದು ಎಂದು ನಿಶ್ಚಯ ಮಾಡಿಕೊಂಡಿದ್ದರು. ಆದರೆ ಕಾಶಿಗೆ ಹೊರಟಿದ್ದ ಅವರು ಸ್ವಾಮಿ ಯೋಗಾನಂದರ ಅಸ್ವಸ್ಥತೆಯ ಸುದ್ದಿಕೇಳಿ ಅಲಹಾಬಾದಿಗೆ ಹೋಗಬೇಕಾಯಿತು. ಈಗ ಹೃಷೀಕೇಶದಿಂದ ಸ್ವಾಮಿ ಅಭೇದಾನಂದ ರಿಗೆ ಮಲೇರಿಯಾ ಆಗಿರುವ ಸುದ್ದಿ ಬಂದಿದೆ. ಸ್ವಾಮೀಜಿ ಈಗೊಂದು ಬಗೆಯ ಸಂದಿಗ್ಧ ಪರಿಸ್ಥಿತಿಯಲ್ಲಿ ಸಿಲುಕಿಕೊಂಡುಬಿಟ್ಟಿದ್ದಾರೆ–ಸಂನ್ಯಾಸಿಯ ವೈರಾಗ್ಯವೊಂದು ಕಡೆ. ಸೋದರ ಸಂನ್ಯಾಸಿಗಳ ಮೇಲಿನ ಪ್ರೀತಿಪೂರ್ವಕ ಹೊಣೆಗಾರಿಕೆಯೊಂದು ಕಡೆ. ಜೊತೆಗೆ ಅವರಿಗೇ ವಾತ, ಅತಿಸಾರ ಮುಂತಾದ ಕಾಯಿಲೆಗಳು! ಈ ಎಲ್ಲ ಜಂಜಡಗಳ ನಡುವೆ ಮನಸ್ಸಿನ ಸ್ಥಿರತೆಯನ್ನು ಕಾಪಾಡಿಕೊಂಡು ಪರಮಾತ್ಮನಲ್ಲಿ ಮನಸ್ಸನ್ನು ಏಕಾಗ್ರಗೊಳಿಸಲು ಯಾವ ಬಗೆಯ ಯೋಗಭ್ಯಾಸ ತಮಗೆ ನೆರವಾಗಬಲ್ಲುದು ಎಂದು ಅವರು ಚಿಂತಿಸುತ್ತಿದ್ದಾರೆ. ಇಂತಹ ಯೋಗವನ್ನು ಕಲಿಯ ಬೇಕು ಎಂದೇ ಅವರು ಪವಾಹಾರಿ ಬಾಬಾರ ಬಳಿಗೆ ಬಂದದ್ದು.

ಆದರೆ ಈ ಪವಾಹಾರಿ ಬಾಬಾ ಮಾತ್ರ ಯಾವುದನ್ನೂ ಅಷ್ಟು ಸುಲಭವಾಗಿ ಬಿಟ್ಟುಕೊಡುವವ ರಲ್ಲ. ಸ್ವಾಮೀಜಿ ಬಯಸಿದಂತಹ ಯೋಗವನ್ನು ಅವರಿಗೆ ಒಮ್ಮೆಗೇ ತಿಳಿಸಿಕೊಡಲು ಬಾಬಾಜಿ ಉತ್ಸುಕರಾಗಿದ್ದಂತೆ ತೋರಲಿಲ್ಲ. ಈ ಸಂದರ್ಭದಲ್ಲಿ ತಮ್ಮ ಮನಸ್ಸಿನಲ್ಲಿ ನಡೆಯುತ್ತಿದ್ದ ತುಮುಲವನ್ನು ಮತ್ತು ತಮಗಾದ ಒಂದು ಅಲೌಕಿಕ ಅನುಭವವನ್ನು ಮುಂದೊಮ್ಮೆ ಸ್ವಾಮೀಜಿ ಒಬ್ಬ ಶಿಷ್ಯನೆದುರು ಹೊರಗೆಡಹುತ್ತಾರೆ:

“ಪವಾಹಾರಿ ಬಾಬಾರ ಸಂಪರ್ಕಕ್ಕೆ ಬಂದಮೇಲೆ ನಾನು ಅವರನ್ನು ಬಹಳವಾಗಿ ಇಷ್ಟಪಡ ತೊಡಗಿದೆ. ಅವರಿಗೂ ನನ್ನ ಮೇಲೆ ಗಾಢ ವಾತ್ಸಲ್ಯ ಬೆಳೆಯಿತು. ಒಂದು ದಿನ ನನ್ನ ಮನಸ್ಸಿನ ಲ್ಲೊಂದು ಆಲೋಚನೆ ಎದ್ದಿತು–‘ಅಲ್ಲ, ನಾನು ಅಷ್ಟೊಂದು ವರ್ಷ ಶ್ರೀರಾಮಕೃಷ್ಣರ ಜೊತೆ ಯಲ್ಲಿದ್ದರೂ, ನನ್ನ ಈ ದುರ್ಬಲ ಶರೀರವನ್ನು ಬಲಪಡಿಸಿಕೊಳ್ಳುವ ಉಪಾಯವನ್ನು ಮಾತ್ರ ಕಲಿಯಲಿಲ್ಲವಲ್ಲ’ ಎಂದು. ಬಾಬಾಜಿಯವರಿಗೆ ಹಠಯೋಗ ತಿಳಿದಿದೆ ಎಂದು ಕೇಳಿದ್ದೆ. ಆದ್ದ ರಿಂದ ನಾನು ಅದನ್ನು ಅವರಿಂದ ಕಲಿತು, ಈ ದೇಹವನ್ನು ಬಲಪಡಿಸಿಕೊಳ್ಳಬೇಕು ಎಂದು ಆಲೋಚಿಸಿದೆ... ದೀಕ್ಷೆ ಸ್ವೀಕರಿಸುವುದೆಂದು ನಿಶ್ಚಯವಾಗಿದ್ದ ಹಿಂದಿನ ದಿನ ಸಂಜೆ ನಾನು ಮಂಚದ ಮೇಲೆ ಒರಗಿಕೊಂಡು ಏನೋ ಆಲೋಚಿಸುತ್ತಿದ್ದೆ. ಆಗ ಇದ್ದಕ್ಕಿದ್ದಂತೆ ಅಲ್ಲಿ ಶ್ರೀರಾಮಕೃಷ್ಣರು ಕಾಣಿಸಿಕೊಂಡರು! ಅವರು ನನ್ನ ಬಲಭಾಗದಲ್ಲಿ ನಿಂತುಕೊಂಡು ನನ್ನನ್ನೇ ನೆಟ್ಟ ದೃಷ್ಟಿಯಿಂದ ನೋಡುತ್ತಿದ್ದರು. ಅವರ ಆ ನೋಟದಲ್ಲಿ ನೋವಿನ ಛಾಯೆ ಕಾಣುತ್ತಿತ್ತು.... ನನ್ನನ್ನು ನಾನು ಅವರಿಗೆ ಸಂಪೂರ್ಣವಾಗಿ ಸಮರ್ಪಿಸಿಕೊಂಡವನು; ಆದರೆ ಈಗ ನಾನು ಮತ್ತೊಬ್ಬ ಗುರುವನ್ನು ಸ್ವೀಕರಿಸುವುದರಲ್ಲಿದ್ದೇನೆ!–ಈ ಯೋಚನೆ ಮನಸ್ಸಿಗೆ ಬಂದಾಗ, ನನಗೆ ಬಹಳ ನಾಚಿಕೆಯಾಯಿತು. ಶ್ರೀರಾಮಕೃಷ್ಣರನ್ನೇ ನೋಡುತ್ತ ಇದ್ದುಬಿಟ್ಟೆ. ಹೀಗೇ ಸುಮಾರು ಎರಡು-ಮೂರು ಗಂಟೆಗಳು ಕಳೆದಿರಬಹುದು; ಆದರೂ ನನ್ನ ಬಾಯಿಂದ ಒಂದು ಮಾತೂ ಹೊರಡಲಿಲ್ಲ. ಬಳಿಕ ಶ್ರೀರಾಮಕೃಷ್ಣರು ಇದ್ದಕ್ಕಿದ್ದಂತೆ ಅದೃಶ್ಯರಾದರು. ಈ ದರ್ಶನದ ಪರಿ ಣಾಮವಾಗಿ ನನ್ನ ಮನಸ್ಸು ಬಹಳ ತಳಮಳಗೊಂಡಿತು. ಆದ್ದರಿಂದ ನಾನು ಬಾಬಾರಿಂದ ಯೋಗದೀಕ್ಷೆ ಪಡೆಯುವ ವಿಚಾರವನ್ನು ಮುಂದೂಡಿದೆ. ಹೀಗೆ ಒಂದೆರಡು ದಿನಗಳು ಕಳೆದಿರ ಬಹುದು; ದೀಕ್ಷೆಯ ವಿಚಾರ ಮತ್ತೆ ಮನಸ್ಸಿಗೆ ಬಂತು. ಪುನಃ ಆ ರಾತ್ರಿ ಶ್ರೀರಾಮಕೃಷ್ಣರು ಹಿಂದಿನಂತೆಯೇ ಕಾಣಿಸಿಕೊಂಡರು! ಹೀಗೆ ಅನೇಕ ರಾತ್ರಿಗಳು ಅವರು ನಿರಂತರವಾಗಿ ಕಾಣಿಸಿಕೊಂಡರು. ಕಡೆಗೆ ಪವಾಹಾರಿ ಬಾಬಾರಿಂದ ದೀಕ್ಷೆ ಪಡೆಯುವ ಆಲೋಚನೆಯನ್ನು ನನ್ನ ಮನಸ್ಸಿನಿಂದ ತಳ್ಳಿಹಾಕಿಬಿಟ್ಟೆ. ಒಂದು ವೇಳೆ ದೀಕ್ಷೆಯನ್ನು ಪಡೆದದ್ದೇ ಆದರೂ ಅದರಿಂದ ನನಗೆ ಒಳ್ಳೆಯದೇನೂ ಆಗುವುದಿಲ್ಲ; ಏನಾದರೂ ಆಗುವುದಿದ್ದರೆ ಅದು ಕೆಡಕು ಮಾತ್ರ ಎಂದು ನನಗೆ ಮನವರಿಕೆಯಾಯಿತು.”

ಸ್ವಾಮೀಜಿಯ ಈ ಮಾತುಗಳು, ಅನುಭವಗಳು ನಿಜಕ್ಕೂ ಪರಮಾದ್ಭುತ. ಶ್ರೀರಾಮಕೃಷ್ಣ ಪರಮಹಂಸರ ಅಪಾರ ಕೃಪೆಗೆ ಪಾತ್ರರಾದಂತಹ ಸ್ವಾಮೀಜಿ, ಶ್ರೀರಾಮಕೃಷ್ಣರಿಂದ ಆಧ್ಯಾತ್ಮಿಕ ಸಿದ್ಧಿಸರ್ವಸ್ವವನ್ನೇ ಧಾರೆಯಾಗಿ ಪಡೆದಂತಹ ಸ್ವಾಮೀಜಿ, ಮತ್ತೊಬ್ಬ ಸಂತನಿಂದ ಪಡೆಯು ವಂಥದೇನಿದೆ! ಸ್ವತಃ ಸ್ವಾಮೀಜಿಗೆ ಈ ವಿಷಯ ತಿಳಿದಿಲ್ಲವೆ...? ತಿಳಿದಿದೆ, ಎಲ್ಲವೂ ತಿಳಿದಿದೆ. ಆದರೆ ಶ್ರೀರಾಮಕೃಷ್ಣರು ಅವರನ್ನು ಪರೀಕ್ಷೆ ಮಾಡುತ್ತಿದ್ದಾರೆ. ವಾತರೋಗ ಅತಿಸಾರವೇ ಮೊದಲಾದ ಕಾಯಿಲೆಗಳೆಲ್ಲ ಅವರ ಶರೀರವನ್ನು ಜರ್ಝರಿತಗೊಳಿಸಿಬಿಟ್ಟಿವೆ. ಇಂತಹ ಶರೀರ ದಿಂದ ಮಹಾಕಾರ್ಯಗಳನ್ನು ಹೇಗೆ ಸಾಧಿಸಿಯೇನು?–ಆದ್ದರಿಂದ ಈ ಹಠಯೋಗಿಯ ಸಹಾಯ ಪಡೆದು ದೇಹವನ್ನು ಬಲಪಡಿಸಿಕೊಳ್ಳೋಣ, ಎಂಬ ಆಲೋಚನೆ ಅವರದು. ಆದರೆ ಶ್ರೀರಾಮ ಕೃಷ್ಣರೇ ತಮಗೆ ಬೇಕಾದ ಎಲ್ಲ ಅನುಕೂಲತೆಗಳನ್ನು ಮಾಡಿಕೊಡಬಲ್ಲರು ಎನ್ನುವ ಯೋಚನೆ ಬರಲಿಲ್ಲ ಅವರಿಗೆ. ಇದೂ ಕೂಡ ಶ್ರೀರಾಮಕೃಷ್ಣರ ಇಚ್ಛೆಯೇ. ಏಕೆಂದರೆ ಪರಮ ಶಕ್ತಿಶಾಲಿ ಯಾದ ತಮ್ಮ ನರೇಂದ್ರ ಅತ್ಯಂತ ಅಸಹಾಯಕ ಪರಿಸ್ಥಿತಿಗೆ ಗುರಿಯಾದ ಸಮಯಕ್ಕೆ ಸರಿಯಾಗಿ ತಮ್ಮ ಸಹಾಯಹಸ್ತವನ್ನು ನೀಡುವಂತಾಗಬೇಕು; ತನ್ಮೂಲಕ ಆತನ ಶ್ರದ್ದೆಯನ್ನು ಸ್ಥಿರಗೊಳಿಸ ಬೇಕು ಎಂಬುದೇ ಶ್ರೀರಾಮಕೃಷ್ಣರ ಅಪೇಕ್ಷೆ. ತಮ್ಮ ಭಕ್ತರ ಹಿಂದೆ ತಾವು ಸದಾ ಬೆಂಗಾವಲಾಗಿ ಇದ್ದೇವೆ ಎನ್ನುವುದನ್ನು ಈ ಮೂಲಕ ಅವರು ದೃಢಪಡಿಸುತ್ತಿದ್ದಾರೆ. ಸ್ವಾಮೀಜಿಗೆ ಮತ್ತೆಮತ್ತೆ ದರ್ಶನ ಕೊಟ್ಚಿದ್ದೇ ಇದಕ್ಕೆ ಸಾಕ್ಷಿ. ಅಂತೂ ಈ ಘಟನೆಯ ಮೂಲಕ ಶ್ರೀರಾಮಕೃಷ್ಣರು ಜಗತ್ತಿನ ಜನರಿಗೆ ಒಂದು ಘನ ಸತ್ಯವನ್ನು ಪ್ರಕಟಪಡಿಸುತ್ತಾರೆ. ಏನದು? ತಾವು ಶರೀರ ಬಿಟ್ಟರೂ ತಮ್ಮ ಅವತಾರ ಕಾರ್ಯವನ್ನು ನಿಲ್ಲಿಸಿಲ್ಲ; ಅದೃಶ್ಯರಾಗಿ ಇಲ್ಲೇ ಇದ್ದುಕೊಂಡು ತಮ್ಮ ಭಕ್ತರ ಮೂಲಕ ಅವತಾರ ಕಾರ್ಯವನ್ನು ಸಾಧಿಸುತ್ತಿದ್ದೇವೆ ಎಂದು. ಈ ಅದ್ಭುತ ದರ್ಶನ ಸ್ವಾಮೀಜಿಯ ಮನಸ್ಸಿನ ಮೇಲೆ ಗಾಢ ಪರಿಣಾಮವನ್ನು ಬೀರಿತು. ತಾವು ಈ ಭೂಮಿಯ ಮೇಲೆ ಎಲ್ಲೇ ಸಂಚರಿಸುತ್ತಿ ದ್ದರೂ ಶ್ರೀರಾಮಕೃಷ್ಣರು ಅಲ್ಲಿಯೇ ಇದ್ದುಕೊಂಡು ತಮ್ಮನ್ನು ರಕ್ಷಿಸುತ್ತಾರೆ ಎನ್ನುವ ಭರವಸೆ ಸಿಕ್ಕಿತು. ಮುಂದೆ ಬಹುಕಾಲದ ಮೇಲೆ ಸ್ವಾಮೀಜಿ ಬಂಗಾಳಿಯಲ್ಲಿ, ‘ಗಾಯೆ ಗೀತ್ ಶುನಾತೆ ತೋ ಮಾಯ್’–‘ಹಾಡೊಂದು ಹಾಡುವೆ ನಾ ನಿನಗೆ’ ಎಂಬ ಕವನವೊಂದನ್ನು ರಚಿಸುತ್ತಾರೆ. ಈ ಕವನದಲ್ಲಿ, ಅವರಿಗೆ ಘಾಜೀಪುರದಲ್ಲಾದ ಶ್ರೀರಾಮಕೃಷ್ಣದರ್ಶನದ ದಿವ್ಯಾನುಭವ ಮೈದಾಳಿ ರುವುದನ್ನು ಗಮನಿಸಬಹುದು–

\begin{myquote}
ಅನವರತ ನೀನೆನ್ನ ಹಿಂದೆಯೇ ಇರುತಿರುವೆ\\ಮೃದುಮಧುರ ಮಂದಹಾಸವನು ಬೀರಿ;\\ಅಂಜಿಕೆಯದೆನಗೆಲ್ಲಿ, ಜನನ ಮರಣವು ಕೂಡ\\ಕಾಲಿನಡಿ ಬಿದ್ದಿಹುದು ಮೈಯ ಮುದುರಿ!
\end{myquote}

\begin{myquote}
ಜನುಮಜನುಮಂಗಳಲು ನಾ ನಿನ್ನ ಸೇವಕನು,\\ಹೇ ದಯಾನಿಧೆ, ನಿನ್ನ ಬಗೆಯನರಿಯೆ; \\ಇನಿತೆ ನನ್ನಿಚ್ಛೆ, ನಾನಿನಿತನ್ನೆ ಬೇಡುವೆನು, \\ದಡಕೆ ದಾಟಿಸು ಎನ್ನ, ನನ್ನ ಪ್ರಭುವೆ!
\end{myquote}

\begin{myquote}
ನಿನಗೆ ನಾ ಆಟದೊಳು ಮುಳುಗಿರುವ ಶಿಶುವಿನೊಲು–\\ನಿನ್ನೆದುರು ಆಟವಾಡುತ್ತಲಿರುವೆ;\\ಕೆಲವೊಮ್ಮೆ ನಿನ್ನಿಂದ ದೂರಾಗಿ ಅಲೆದು ಬಹೆ,\\ಕೆಲವೊಮ್ಮೆ ನಿನ್ನೊಡನೆ ಮುನಿಸು ತಾಳ್ವೆ!
\end{myquote}

\begin{myquote}
ಇರುಳಿನಾ ಕತ್ತಲಲಿ ಮೌನಾಶ್ರುಧಾರೆಯಲಿ\\ತುಂಬಿ ಬಹ ಕಂಗಳಿಂ ನೀ ನೋಡುವೆ;\\ಆ ನಿನ್ನ ಮುದ್ದು ಮುಖ ಮಣಿಸುವುದು ನನ್ನನ್ನು,\\ನಿನ್ನ ಪದದಡಿ ನಾನು ಶರಣು ಬರುವೆ!
\end{myquote}

\begin{myquote}
ನಿನ್ನ ಕ್ಷಮೆಯನ್ನೇಕೆ ನಾನು ಕೋರಲಿ ಪ್ರಭುವೆ,\\ಕೋಪ ನಿನಗಿಲ್ಲ ಈ ಸುತನ ಮೇಲೆ;\\ನೀನೆನ್ನ ತಾಯ್ತಂದೆ, ನಿನ್ನ ಚಿರಪುತ್ರ ನಾ,\\ನನ್ನ ನೀನೆಂದೆಂದು ಸಹಿಸುತಿರುವೆ!
\end{myquote}

\noindent

ಹೀಗೆ ಸ್ವಾಮೀಜಿ ತಮ್ಮ ಗುರುದೇವನನ್ನು ಇನ್ನೂ ಚೆನ್ನಾಗಿ ಅರಿತುಕೊಳ್ಳುವಂತಾಯಿತು; ಎಲ್ಲ ಆಧ್ಯಾತ್ಮಿಕತೆಯ ಅಂತಿಮ ಸಮಾವೇಶ ಹಾಗೂ ಅತ್ಯುನ್ನತ ಆಧ್ಯಾತ್ಮಿಕ ಶಿಖರವೇ ಶ್ರೀರಾಮ ಕೃಷ್ಣರು ಎಂಬುದನ್ನು ಸ್ಪಷ್ಟವಾಗಿ ಮನಗಾಣುವಂತಾಯಿತು. ಇಂತಹ ಶ್ರೀರಾಮಕೃಷ್ಣರ ಪದತಲ ದಲ್ಲಿ ಕುಳಿತು ಧನ್ಯನಾದವನಿಗೆ ಬೇರಾವ ಆಧ್ಯಾತ್ಮಿಕ ಗುರುವಿನ ಆವಶ್ಯಕತೆಯೂ ಇಲ್ಲವೆಂಬು ದನ್ನು ಅರಿತರು. ಈ ಸಂದರ್ಭದಲ್ಲಿ ಅವರು ಪ್ರಮದ ಬಾಬುವಿಗೆ ಒಂದು ಪತ್ರ ಬರೆಯುತ್ತಾರೆ–

“... ಈಗ ನೋಡಿದರೆ ಎಲ್ಲವೂ ನನ್ನ ನಿರೀಕ್ಷೆಗೆ ತದ್ವಿರುದ್ಧವಾಗಿಯೇ ಆಗಿಬಿಟ್ಟಿದೆ! ನಾನೇ ಒಬ್ಬ ಭಿಕ್ಷುಕನಂತೆ ಈ ಪವಾಹಾರಿ ಬಾಬಾರ ಬಾಗಿಲಿಗೆ ಬಂದಿದ್ದರೆ, ಇವರು ನನ್ನಿಂದಲೇ ಕಲಿಯಲು ಇಚ್ಛಿಸುತ್ತಾರೆ! ಬಹುಶಃ ಈ ಸಾಧುಗಳಿನ್ನೂ ಪರಿಪೂರ್ಣರಾಗಿಲ್ಲವೆಂದು ತೋರು ತ್ತದೆ. ವಿಪರೀತ ವಿಧಿ, ಆಚಾರಗಳು ಮತ್ತು ತುಂಬ ಮುಚ್ಚುಮರೆ ಇವರಲ್ಲಿ, ಸಾಗರ ಪೂರ್ಣ ವಾಯಿತೆಂದರೆ ಅದರ ದಡಗಳು ಅದನ್ನು ಹಿಡಿದಿಡಲಾರವು. ಆದ್ದರಿಂದ ನಾನು ನಿರ್ಧಾರ ಮಾಡಿಬಿಟ್ಟಿದ್ದೇನೆ–ಇವರಿಗೆ ಸುಮ್ಮನೆ ತೊಂದರೆ ಕೊಡುವುದು ಒಳ್ಳೆಯದಲ್ಲ, ಅದರಿಂದೇನೂ ಪ್ರಯೋಜನವಿಲ್ಲ ಎಂದು. ನಾನು ಇಷ್ಟರಲ್ಲೇ ಅವರಿಂದ ಬೀಳ್ಗೊಳ್ಳಲಿದ್ದೇನೆ...

“ಇನ್ನುಮೇಲೆ ನಾನು ಯಾವ ದೊಡ್ಡ ವ್ಯಕ್ತಿಯ ಬಳಿಗೂ ಹೋಗುವವನಲ್ಲ.” ಹೀಗೆ ಹೇಳಿ, ಬಂಗಾಳೀ ಹಾಡೊಂದನ್ನು ಉದ್ಧರಿಸುತ್ತಾರೆ–

\begin{verse}
ನಿನ್ನಲ್ಲಿಯೆ ನೀನಿರು ಮನವೆ\\ಹೊರಗೆಲ್ಲಿಯು ನೀನರಸದಿರು\\ಆವುದನೀತೆರ ಹುಡುಕುತಲಿಹೆಯೋ\\ಅದೆ ನಿನ್ನೆದಯೊಳಗಿರುತಿರಲು ॥
\end{verse}

\begin{verse}
ಹೃದಯದ ಕರೆಗೋಗೊಡುವನು ಅವನು\\ಸರ್ವಸಿದ್ಧಿಗಳ ಸ್ಪರುಷಮಣಿ\\ನಿನ್ನ ಎದೆಯೊಳೇ ಹುದುಗಿದೆ ಕಾಣೋ\\ಸಕಲೈಶ್ವರ್ಯದ ರನ್ನಗಣಿ ॥
\end{verse}

\begin{verse}
ಮುತ್ತುರತುನಗಳೊ ಪಚ್ಚೆ ಹರಳುಗಳೊ\\ಹೃದಯಮಂದಿರದ ಅಂಗಳದಿ\\ಚದುರಿ ಬಿದ್ದಿರಲು ಎಲ್ಲಿ ಹುಡುಕುತಿಹೆ\\ಮರುಳನಂತೆ ಅನ್ಯಾಶ್ರಯದಿ ॥
\end{verse}

“ಅದ್ದರಿಂದ ಈಗ ನನ್ನ ಅಂತಿಮ ಮಹಾನಿರ್ಣಯವೇನೆಂದರೆ ಶ್ರೀರಾಮಕೃಷ್ಣರಿಗೆ ಯಾರೂ ಸರಿಸಾಟಿಯಿಲ್ಲ ಎಂದು. ಶ್ರೀರಾಮಕೃಷ್ಣರಂತಹ ಅಭೂತಪೂರ್ವ ಪರಿಪೂರ್ಣತೆ ಈ ಜಗತ್ತಿನ ಲ್ಲೆಲ್ಲೂ ಇಲ್ಲ. ಸಕಲರನ್ನೂ ತನ್ನೆಡೆಗೆ ಸೆಳೆಯುವ ಅವರ ಅದ್ಭುತ ಅಹೈತುಕ ಕರುಣೆ, ಬದ್ಧಮಾನ ವನ ಮೇಲೆ ಅವರಿಗಿರುವ ತೀವ್ರ ಅನುಕಂಪೆ–ಇವನ್ನು ಇನ್ನೆಲ್ಲಿ ಕಾಣಲು ಸಾಧ್ಯ? ಅವರು, ತಾವೇ ಹೇಳುತ್ತಿದ್ದಂತೆ ಅವತಾರವೇ ಆಗಿರಬೇಕು, ಅಥವಾ ಮಾನವಕೋಟಿಯ ಉದ್ಧಾರಕ್ಕಾಗಿ ಮನುಷ್ಯದೇಹ ಧರಿಸಿ ಬರುವರೆಂದು ವೇದಾಂತದಲ್ಲಿ ಹೇಳಲ್ಪಟ್ಟಿರುವ ದೇವಮಾನವರೇ ಆಗಿರ ಬೇಕು. ಇದು ನನ್ನ ಖಂಡಿತವಾದ ಹಾಗೂ ಸುನಿಶ್ಚಿತವಾದ ಅಭಿಮತ. ಇಂತಹ ದೇವಮಾನವರ ಆರಾಧನೆಯನ್ನೇ ಪತಂಜಲಿ ಮಹರ್ಷಿಗಳು ತಮ್ಮ ಯೋಗಸೂತ್ರದಲ್ಲಿ ‘ಸಂತನೊಬ್ಬನ ಶುದ್ಧ ಹೃದಯದ ಮೇಲೆ ಧ್ಯಾನ ಮಾಡುವುದರ ಮೂಲಕ ನಮ್ಮ ಉದ್ದೇಶವನ್ನು ಸಿದ್ಧಿಸಿಕೊಳ್ಳಬಹುದು’ ಎಂದು ಉಲ್ಲೇಖಿಸಿದ್ದಾರೆ.”

ಹೀಗೆಂದ ಸ್ವಾಮೀಜಿ, ಈಗ ತಮ್ಮ ಅಂತರಾಳವನ್ನೇ ಪ್ರಮದಬಾಬುಗಳ ಮುಂದೆ ತೆರೆದಿಡುತ್ತಾರೆ.

“ಶ್ರೀರಾಮಕೃಷ್ಣರು ತಮ್ಮ ಜೀವಿತಾವಧಿಯಲ್ಲಿ ನನ್ನ ಒಂದೇ ಒಂದು ಪ್ರಾರ್ಥನೆಯನ್ನೂ ನಿರಾಕರಿಸಲಿಲ್ಲ. ನನ್ನ ಅನಂತಾನಂತ ಅಪರಾಧಗಳನ್ನು ಕ್ಷಮಿಸಿದ್ದಾರೆ ಅವರು. ಅಂತಹ ಅದ್ಭುತ ಪ್ರೀತಿ ನನಗೆ ನನ್ನ ತಂದೆತಾಯಿಯರಿಂದಲೂ ಸಿಗಲಿಲ್ಲ. ಇದರಲ್ಲಿ ಅತಿಶಯೋಕ್ತಿಯಾಗಲಿ ಕವಿತ್ವ ವಾಗಲಿ ಇಲ್ಲ. ಅವರ ಪ್ರತಿಯೊಬ್ಬ ಶಿಷ್ಯನಿಗೂ ತಿಳಿದಿರುವ ಸರಳ ಸತ್ಯ ಇದು. ಮಹಾ ಅಪಾಯ ಗಳ ಸಂದರ್ಭಗಳಲ್ಲಿ, ಘೋರ ಪ್ರಲೋಭನೆಗಳ ಸಮಯಗಳಲ್ಲಿ ನಾನು ತೀವ್ರ ತಳಮಳದಿಂದ ಅತ್ತು ಪ್ರಾರ್ಥಿಸಿದೆ–‘ಓ ದೇವರೆ, ಕಾಪಾಡು’ ಎಂದು. ಆದರೆ ಆಗ ನನಗೆ ಯಾವ ಉತ್ತರವೂ ಬರಲಿಲ್ಲ. ಆದರೆ ಈ ಅದ್ಭುತ ವ್ಯಕ್ತಿ–ನೀವು ಅವರನ್ನು ಅವತಾರ ಎಂದಾದರೂ ಕರೆಯಿರಿ, ಏನು ಬೇಕಾದರೂ ಕರೆಯಿರಿ–ಅವರು ತಮ್ಮ ಅಂತರ್ದೃಷ್ಟಿಯಿಂದ ನನ್ನ ಬೇಗೆಗಳೆಲ್ಲವನ್ನೂ ಅರಿತು ಪರಿಹರಿಸಿದ್ದಾರೆ. ನಾನೇ ಹಿಂಜರಿಯುತ್ತಿದ್ದರೂ ಕೂಡ ಅವರು ತಾವಾಗಿಯೇ ನನ್ನನ್ನು ತಮ್ಮೆಡೆಗೆ ಸೆಳೆದುಕೊಂಡು ರಕ್ಷಿಸಿದ್ದಾರೆ.”

ಆದರೆ ಸ್ವಾಮೀಜಿ ಪವಾಹಾರಿ ಬಾಬಾರನ್ನು ಮೆಚ್ಚಿಕೊಂಡಿದ್ದಾಗಲಿ, ಅವರಿಂದ ಯೋಗದೀಕ್ಷೆ ಯನ್ನು ಪಡೆಯಲು ಉತ್ಸುಕರಾದದ್ದಾಗಲಿ, ಅಥವಾ ಶ್ರೀರಾಮಕೃಷ್ಣರ ದಿವ್ಯದರ್ಶನವಾದ ಬಳಿಕ ತಮ್ಮ ನಿರ್ಧಾರವನ್ನು ಬದಲಿಸಿದ್ದಾಗಲಿ, ಅವರು ಶ್ರೀರಾಮಕೃಷ್ಣರಲ್ಲಿ ಶ್ರದ್ಧೆಯನ್ನು ಕಳೆದು ಕೊಂಡಿದ್ದರೆಂಬುದರ ಕುರುಹುಗಳೆಂದು ತಿಳಿಯಬೇಕಾಗಿಲ್ಲ. ಅವು ಬಹುಮುಖ ಜ್ಞಾನಾರ್ಜನೆ ಯಲ್ಲಿ ಅವರಿಗಿದ್ದ ಉತ್ಸಾಹವನ್ನು ತೋರಿಸುತ್ತವೆ. ಆದರೆ ಇದನ್ನು ಸ್ವತಃ ಅವರ ಗುರುಭಾಯಿ ಗಳೂ ಅಪಾರ್ಥ ಮಾಡಿಕೊಂಡಿದ್ದರು. ಆದ್ದರಿಂದ ಅದನ್ನು ಅಲ್ಲಗಳೆದು ಸ್ವಾಮೀಜಿ ಸ್ವಲ್ಪ ಖಾರವಾಗಿಯೇ ಅಖಂಡಾನಂದರಿಗೊಂದು ಪತ್ರ ಬರೆದಿದ್ದರು–“... ಒಳ್ಳೆಯದು ಎಲ್ಲೇ ಇದ್ದರೂ ಅದನ್ನು ಗುರುತಿಸಬೇಕೆಂಬುದೇ ನನ್ನ ಧ್ಯೇಯಮಂತ್ರ. ಇದರಿಂದ ಶ್ರೀರಾಮಕೃಷ್ಣರ ಮೇಲೆ ನನಗಿರುವ ಭಕ್ತಿ ಕಡಿಮೆಯಾಗಬಹುದೆಂದು ನನ್ನ ಗುರುಭಾಯಿಗಳು ಭಾವಿಸಿರುವಂತಿದೆ. ಇದು ಕೇವಲ ಮತಾಂಧರ ಮತ್ತು ಹಠಮಾರಿಗಳ ಅಭಿಪ್ರಾಯ. ಏಕೆಂದರೆ ಎಲ್ಲ ಗುರುಗಳೂ ಸಚ್ಚಿದಾನಂದಗುರುವಾದ ಒಬ್ಬನೇ ಭಗವಂತನ ಬೇರೆಬೇರೆ ಅಂಶಗಳು, ಬೇರೆಬೇರೆ ಮುಖ ಗಳು,” ಪವಾಹಾರಿ ಬಾಬಾರಿಂದ ರಾಜಯೋಗವನ್ನು ಕಲಿಯುವುದಷ್ಟೇ ಅವರ ಉದ್ದೇಶವಾಗಿ ತ್ತೆಂಬುದು ಇದೇ ಪತ್ರದ ಮೊದಲ ಕೆಲವು ಸಾಲುಗಳಿಂದ ಸ್ಪಷ್ಟವಾಗುತ್ತದೆ. ಅವರು ಬರೆಯು ತ್ತಾರೆ: “ನಮ್ಮ ಬಂಗಾಳವು ಭಕ್ತಿ-ಜ್ಞಾನಗಳ ನಾಡು. ಅಲ್ಲಿ ಯೋಗದ ಹೆಸರೇ ಇಲ್ಲ. ಇರುವ ಸ್ವಲ್ಪ ‘ಯೋಗ’ ಎಂಬುದಾದರೂ ಕೇವಲ ಹಠಯೋಗದ ಹೆಸರಿನ ಕಸರತ್ತುಗಳು, ಅಷ್ಟೆ. ಆದ್ದರಿಂದಲೇ ಈಗ ನಾನು ಈ ಅದ್ಭುತ ರಾಜಯೋಗಿಯ ಬಳಿಗೆ ಬಂದಿದ್ದೇನೆ... ” ಆದರೆ ಶ್ರೀರಾಮಕೃಷ್ಣರ ಇಚ್ಛೆ ಬೇರೆಯಾಗಿತ್ತು. ಸ್ವಾಮೀಜಿಯ ಪಾಲಿಗಂತೂ ಒಂದು ಅಮೂಲ್ಯ ಅನು ಭವ ಸಿಕ್ಕಿದಂತಾಯಿತು.

ಘಾಜೀಪುರದಲ್ಲಿದ್ದ ಈ ದಿನಗಳಲ್ಲಿ ಸ್ವಾಮೀಜಿ ಕೆಲವು ಐರೋಪ್ಯ ಅಧಿಕಾರಿಗಳೂ ಸೇರಿದಂತೆ ಹಲವಾರು ವ್ಯಕ್ತಿಗಳನ್ನು ಸಂಧಿಸಿದರು. ಗಗನ್ಬಾಬುವಿನ ಮೂಲಕ ಅವರಿಗೆ ರಾಸ್ ಎಂಬೊಬ್ಬ ಆಂಗ್ಲ ಸರ್ಕಾರೀ ಅಧಿಕಾರಿಯ ಪರಿಚಯವಾಯಿತು. ಈ ರಾಸ್ ಎಂಬುವನು ಸಾಕಷ್ಟು ಪಾಂಡಿತ್ಯ ವನ್ನು ಗಳಿಸಿದ್ದವನು. ಈತ ಸ್ವಾಮೀಜಿಯನ್ನು ಹಿಂದೂ ಹಬ್ಬ-ಉತ್ಸವಗಳ ಕುರಿತಾಗಿ, ಅದರಲ್ಲೂ ಮುಖ್ಯವಾಗಿ ಹೋಳಿ ಹಾಗೂ ರಾಮಲೀಲಾ ಉತ್ಸವಗಳ ಸಂಬಂಧವಾಗಿ, ಹಲವಾರು ಪ್ರಶ್ನೆ ಗಳನ್ನು ಕೇಳಿದ. ಅಲ್ಲದೆ ಹಿಂದೂಗಳ ಹಲವಾರು ಸಾಮಾಜಿಕ ಸಂಪ್ರದಾಯಗಳ ಬಗೆಗೂ ಪ್ರಶ್ನಿಸಿದ. ಈ ಪ್ರಶ್ನೆಗಳಿಂದಾಗಿ ಸ್ವಾಮೀಜಿಯಲ್ಲಿ ಸುಪ್ತವಾಗಿದ್ದ ಅನಂತ ಅಂತಸ್ಸತ್ವವೂ, ಅಪಾರ ವಿದ್ವತ್ತೂ ಜಾಗೃತಗೊಂಡು ಪ್ರಖರ ವಾಗ್ವಾಹಿನಿಯ ಮೂಲಕ ಹರಿಯಲಾರಂಭಿಸಿತು. ಧಾರ್ಮಿಕ ಬೆಳವಣಿಗೆಗೂ ಪ್ರಕೃತಿಪೂಜೆ ಹಾಗೂ ವ್ಯಕ್ತಿಪೂಜೆಗಳಿಗೂ ಇರುವ ಸಂಬಂಧವನ್ನು ಅವರು ಸೂಕ್ಷ್ಮವಾಗಿ ವಿಶ್ಲೇಷಿಸಿ ವಿವರಿಸಿದರು. ಹಿಂದೂ ಸಾಮಾಜಿಕ ಹಿನ್ನೆಲೆಯಲ್ಲಿರುವ ಶ್ರೇಷ್ಠ ಆಧ್ಯಾತ್ಮಿಕ ಭಾವನೆಗಳನ್ನು ತಮ್ಮ ವಿಶಿಷ್ಟ, ವೈಜ್ಞಾನಿಕ ದೃಷ್ಟಿಕೋನದ ಮೂಲಕ ವರ್ಣಿಸಿದರು. ಈ ಉತ್ತರಗಳನ್ನು ಕೇಳಿ ಆ ಪಾಶ್ಚಾತ್ಯ ವಿದ್ವಾಂಸನಿಗೆ ಅತ್ಯಾಶ್ಚರ್ಯವಾಯಿತು. ಸನಾತನ ಧರ್ಮದ ಅದ್ಭುತತೆ ಅವನಿಗೆ ಮನದಟ್ಟಾಯಿತು. ಹಿಂದೂಧರ್ಮದ ಆಧ್ಯಾತ್ಮಿಕ ತತ್ತ್ವಗಳು ಇಷ್ಟೊಂದು ಘನವಾಗಿದ್ದಾವೆಂದು ತಾನು ಕನಸಿನಲ್ಲೂ ಊಹಿಸಿರಲಿಲ್ಲ ಎಂದು ಆತ ಒಪ್ಪಿ ನುಡಿದ. ಸ್ವಾಮೀಜಿ ಅವನ ಕೋರಿಕೆಯಂತೆ ಹೋಳಿ ಹಬ್ಬದ ಕುರಿತಾಗಿ ಒಂದು ಲೇಖನವನ್ನು ಬರೆದುಕೊಟ್ಟರು. ಈ ರಾಸ್ ಮಹಾಶಯ ಅವರನ್ನು ಅಲ್ಲಿನ ಜಿಲ್ಲಾ ನ್ಯಾಯಾಧೀಶನಾದ ಪೆನ್ನಿಂಗ್ಟನ್ ಎಂಬವನಿಗೆ ಪರಿಚಯಿಸಿಕೊಟ್ಟ. ಈ ನ್ಯಾಯಾಧೀಶನು ಸ್ವಾಮೀಜಿಯೊಂದಿಗೆ ಸಂಭಾಷಣೆಯಲ್ಲಿ ತೊಡಗಿ ಹಿಂದೂಧರ್ಮದ ಹಲವಾರು ಅಂಶಗಳ ಬಗ್ಗೆ ಪ್ರಶ್ನಿಸಿದ. ಅವುಗಳಿಗೆಲ್ಲ ಸ್ವಾಮೀಜಿ ಅತ್ಯಂತ ಸಮರ್ಥವಾಗಿ ಉತ್ತರಿಸಿ, ಹಿಂದೂಧರ್ಮದ ಪುನರುತ್ಥಾನ ಹಾಗೂ ಭಾರತದ ನವಸಂಕ್ರಮಣ ಕಾಲ–ಇವುಗಳ ಕುರಿತಾದ ತಮ್ಮ ಭವ್ಯ ಕಲ್ಪನೆಗಳನ್ನು, ಯೋಜನೆಗಳನ್ನು ಸುದೀರ್ಘವಾಗಿ ವರ್ಣಿಸಿದರು. ಯೋಗದ ಶಾಸ್ತ್ರೀಯ ಹಾಗೂ ವೈಜ್ಞಾನಿಕ ತಳಹದಿ, ಹಿಂದೂ ಸಂನ್ಯಾಸಿಗಳ ಕ್ರಮ- ನಿಯಮಗಳು ಮತ್ತು ಇನ್ನೂ ಹಲವಾರು ವೈವಿಧ್ಯಮಯ ವಿಷಯಗಳ ಕುರಿತಾಗಿ ತಿಳಿಸಿಕೊಟ್ಟರು. ಯೋಗದ ಸೂಕ್ಷ್ಮವಿಚಾರಗಳನ್ನೆಲ್ಲ ಆಧುನಿಕ ಮನಶ್ಶಾಸ್ತ್ರದ ಬೆಳಕಿನಲ್ಲಿ ವಿವರಿಸಿದರು. ಸ್ವಾಮೀಜಿಯ ಈ ಅಸಾಮಾನ್ಯ ವಿಚಾರಧಾರೆಯಿಂದ ತೀವ್ರವಾಗಿ ಪ್ರಭಾವಿತನಾದ ಪೆನ್ನಿಂಗ್ ಟನ್, ಇಂಗ್ಲೆಂಡಿಗೆ ಹೋಗಿ ಸ್ವಾಮೀಜಿ ಈ ಸಂದೇಶಗಳನ್ನು ಪ್ರಚಾರ ಮಾಡಬೇಕು; ಅಲ್ಲಿ ಅವುಗಳಿಗೆ ಖಂಡಿತವಾಗಿ ಯೋಗ್ಯ ಮನ್ನಣೆ ದೊರಕುತ್ತದೆ ಎಂದು ಸಲಹೆ ಮಾಡಿದ. ಸ್ವಾಮೀಜಿ ಯವರು ಧರ್ಮಪ್ರಚಾರಕ್ಕಾಗಿ ಪಾಶ್ಚಾತ್ಯ ರಾಷ್ಟ್ರಗಳಿಗೆ ಹೋಗಬೇಕು ಎಂದು ಸೂಚಿಸಿದವರಲ್ಲಿ ಈತನೇ ಮೊದಲಿಗನೆಂಬಂತೆ ತೋರುತ್ತದೆ.

ಘಾಜೀಪುರದಲ್ಲಿ ಕರ್ನಲ್ ರಿವೆಟ್ ಕಾರ್ನಾಕ್ ಎಂಬ ಮತ್ತೊಬ್ಬ ಆಂಗ್ಲವ್ಯಕ್ತಿಗೆ ಸ್ವಾಮೀಜಿ ಯನ್ನು ಸಂಧಿಸುವ ಅವಕಾಶ ಲಭಿಸಿತು. ಆತ ಅವರೊಂದಿಗೆ ವೇದಾಂತದ ಬೋಧನೆಗಳ ಬಗ್ಗೆ ಹಾಗೂ ಅವುಗಳನ್ನು ತಮ್ಮ ನಿತ್ಯಜೀವನದಲ್ಲಿ ಅಳವಡಿಸಿಕೊಳ್ಳುವುದು ಹೇಗೆಂಬುದರ ಬಗ್ಗೆ ದೀರ್ಘ ಸಂಭಾಷಣೆಯನ್ನು ನಡೆಸಿದ. ಅವರ ಅದ್ಭುತ ಅಂತರ್ದೃಷ್ಟಿ, ಶಕ್ತಿಪೂರ್ಣ ವ್ಯಕ್ತಿತ್ವ, ಅವರ ಪರಿಪೂರ್ಣ ವೈರಾಗ್ಯಬುದ್ಧಿ–ಇವು ಅವರ ಮಾತುಗಳ ಮೂಲಕ ಹೊರಚಿಮ್ಮುತ್ತಿರುವು ದನ್ನು ಕಂಡು ಆತ ಗಾಢವಾಗಿ ಪ್ರಭಾವಿತನಾದ. ಅವರನ್ನು ಸಂದರ್ಶಿಸಿದ ಆ ಪಾಶ್ಚಾತ್ಯರಿಗೆಲ್ಲ ಸ್ವಾಮೀಜಿ ವೇದಾಂತವೇ ಮೈದಾಳಿ ಬಂದಂತೆ ಕಂಡರು.

ಈ ಅವಧಿಯಲ್ಲಿ ಸ್ವಾಮೀಜಿ ಟಿಬೆಟಿನಲ್ಲಿದ್ದ ಸ್ವಾಮಿ ಅಖಂಡಾನಂದರೊಂದಿಗೆ ಪತ್ರ ವ್ಯವಹಾರವನ್ನಿಟ್ಟುಕೊಂಡಿದ್ದರು. ತಾವೂ ಟಿಬೆಟಿಗೆ ಹೋಗಿ ಅಲ್ಲಿ ಲಭ್ಯವಿರುವ ಅಪೂರ್ವ ಬೌದ್ಧ ಶಾಸ್ತ್ರಗ್ರಂಥಗಳನ್ನು ಅಧ್ಯಯನ ಮಾಡುವ ಇಚ್ಛೆ ಸ್ವಾಮೀಜಿಯವರಿಗಿತ್ತು. ಆದರೆ ಟಿಬೆಟನ್ನು ಪ್ರವೇಶಿಸುವಲ್ಲಿ ತೀವ್ರ ಪ್ರತಿಬಂಧಗಳಿದ್ದುದರಿಂದ, ನೇಪಾಳಕ್ಕೆ ಹೋಗಿ, ಅಲ್ಲಿನ ಸ್ನೇಹಿತನೊ ಬ್ಬನ ಪ್ರಭಾವದ ಮೂಲಕ ಟಿಬೆಟಿಗೆ ಹೋಗಬಹುದು; ಹಾಗೇ ಚೀನಾಕ್ಕೂ ಹೋಗಿ, ಅಲ್ಲಿನ ಕೆಲವು ತೀರ್ಥಕ್ಷೇತ್ರಗಳನ್ನು ತಾವು ಸಂದರ್ಶಿಸಬಹುದು ಎಂದೆಲ್ಲ ಆಶಿಸಿ ಅವರು ಅಖಂಡಾನಂದ ರಿಗೆ ಪತ್ರ ಬರೆದರು. ಆದರೆ ಈ ಯಾವ ಸ್ಥಳಕ್ಕೂ ಭೇಟಿಕೊಡಲು ಅವರಿಗೆ ಸಾಧ್ಯವಾಗಲೇ ಇಲ್ಲ.

ಈ ನಡುವೆ ಹೃಷೀಕೇಶದಲ್ಲಿ ಮಲೇರಿಯಾದಿಂದ ನರಳುತ್ತಿದ್ದ ತಮ್ಮ ಗುರುಭಾಯಿ ಅಭೇದಾ ನಂದರ ವಿಷಯದಲ್ಲಿ ಸ್ವಾಮೀಜಿ ವ್ಯಾಕುಲಿತರಾಗಿದ್ದರು. ಅವರು ಇನ್ನಷ್ಟು ಉತ್ತಮ ಚಿಕಿತ್ಸೆ ಪಡೆದು ಸುಧಾರಿಸಿಕೊಳ್ಳಲಾಗುವಂತೆ ಸ್ವಾಮೀಜಿ ಅವರಿಗೆ ಕಾಶಿಗೆ ಬರಲು ಸೂಚಿಸಿ ದಾರಿ ಖರ್ಚನ್ನೂ ಕಳಿಸಿಕೊಟ್ಟರು. ಮತ್ತು ಅವರನ್ನು ಚೆನ್ನಾಗಿ ನೋಡಿಕೊಳ್ಳಬೇಕು ಎಂದು ವಿನಂತಿಸಿ ಕೊಂಡು ಸ್ವಾಮೀಜಿ ತಮ್ಮ ಸ್ನೇಹಿತ ಪ್ರಮದಬಾಬುವಿಗೆ ಪತ್ರ ಬರೆದರು. ಅಲ್ಲದೆ ತಮ್ಮನ್ನು ನೋಡಲೆಂದು ಘಾಜೀಪುರಕ್ಕೆ ಬಂದ ಪ್ರೇಮಾನಂದರನ್ನೂ ವಾರಾಣಸಿಗೆ ಕಳಿಸಿಕೊಟ್ಟರು.

ಆದರೆ ವಾರಾಣಸಿಯಲ್ಲಿ ಅಭೇದಾನಂದರ ದೇಹಸ್ಥಿತಿ ಸುಧಾರಿಸಲಿಲ್ಲವೆಂಬ ಸುದ್ದಿ ತಿಳಿದು ಬಂದಿತು. ಆದ್ದರಿಂದ ತಾವೇ ಅಲ್ಲಿಗೆ ಹೋಗಲು ನಿರ್ಧರಿಸಿ, ೧೮೯೦ರ ಏಪ್ರಿಲ್ ಮೊದಲ ವಾರ ದಲ್ಲಿ ಘಾಜೀಪುರದಿಂದ ಸ್ವಾಮೀಜಿ ಹೊರಟುಬಿಟ್ಟರು. ಅಲ್ಲದೆ, ಆ ಪವಿತ್ರ ನಗರದಲ್ಲಿ ತಪ ಶ್ಚರ್ಯೆಯಲ್ಲಿ ತೊಡಗಬೇಕು ಹಾಗೂ ಪ್ರಮದಬಾಬುಗಳ ಸಹವಾಸದಲ್ಲಿ ಶಾಸ್ತ್ರಾಧ್ಯಯನವನ್ನು ಮುಂದುವರಿಸಬೇಕು ಎಂಬುದು ಅವರ ಮತ್ತೊಂದು ಉದ್ದೇಶ. ವಾರಾಣಸಿಯನ್ನು ತಲುಪಿದ ಸ್ವಾಮೀಜಿ, ಅಭೇದಾನಂದರ ಶುಶ್ರೂಷೆಗೆ ಎಲ್ಲ ವ್ಯವಸ್ಥೆಯನ್ನು ಮಾಡಿ, ಅನಂತರ ತಾವು ಪ್ರಮದಬಾಬುಗಳ ಅತಿಥಿಯಾಗಿ ಅವರ ತೋಟದ ಮನೆಯಲ್ಲಿ ಉಳಿದುಕೊಂಡರು. ಆದರೆ ಕೆಲ ದಿನಗಳಲ್ಲೇ ಸ್ವಾಮೀಜಿಗೆ ಫ್ಲೂ ಜ್ವರ ತಗಲಿಕೊಂಡಿತು. ಪ್ರಮದಬಾಬುಗಳ ಹಾಗೂ ಅಭೇದಾ ನಂದರ ಆರೈಕೆಯ ಫಲವಾಗಿ ಸ್ವಾಮೀಜಿ ಕೆಲವೇ ದಿನಗಳಲ್ಲಿ ಚೇತರಿಸಿಕೊಂಡರು. ಆದರೆ ಅಭೇದಾನಂದರಿಗೆ ಮಲೇರಿಯಾ ಹಿಂದಿಗಿಂತಲೂ ತೀವ್ರವಾಗಿಯೇ ಮರುಕಳಿಸಿಬಿಟ್ಟಿತು. ಈಗ ಗುರುಭಾಯಿಯನ್ನು ಉಪಚರಿಸುವ ಸರದಿ ಮತ್ತೆ ಸ್ವಾಮೀಜಿಯದಾಯಿತು.

ಹೀಗೆ ಸ್ವಾಮೀಜಿ ವಾರಾಣಸಿಯಲ್ಲಿ ಸ್ವಲ್ಪ ನೆಲೆನಿಂತು, ಮುಂದಿನ ಕಾರ್ಯದ ಬಗ್ಗೆ ಆಲೋಚಿ ಸುವಷ್ಟರಲ್ಲೇ ಶ್ರೀರಾಮಕೃಷ್ಣರ ಮಹಾಭಕ್ತನಾದ ಬಲರಾಮ್ಬಾಬು ಇನ್ಫ್ಲೂಯೆಂಜಾ ಜ್ವರಕ್ಕೀಡಾಗಿ ನಿಧನನಾದನೆಂಬ ಅತಿ ದುಃಖದ ಸುದ್ದಿ ಬಂದಿತು. ಇದನ್ನು ಕೇಳಿ ಸ್ವಾಮೀಜಿ ತೀವ್ರ ಶೋಕಗ್ರಸ್ತರಾದರು. ಬಲರಾಮನ ಔದಾರ್ಯದ ಹಾಗೂ ಅವರ ಮಧುರ, ಸ್ನೇಹಪೂರ್ಣ ಸಂಬಂಧದಲ್ಲಿ ತಾವು ಕಳೆದಿದ್ದ ಆನಂದದ ದಿನಗಳ ನೆನಪುಗಳು ಒತ್ತರಿಸಿಬಂದು ಅವರು ಕಣ್ಣೀರ್ಗರೆದರು. ಸ್ವಾಮೀಜಿ ಹೀಗೆ ದುಃಖಿಸುವುದನ್ನು ಕಂಡ ಪ್ರಮದದಾಸ ಮಿತ್ರರಿಗೆ ಅಚ್ಚರಿ– ‘ಒಬ್ಬ ಸಂನ್ಯಾಸಿ, ಕಟ್ಟಾ ವೇದಾಂತಿ, ಗೃಹಸ್ಥನೊಬ್ಬನ ನಿಧನಕ್ಕಾಗಿ ಇಷ್ಟೊಂದು ದುಃಖಿಸು ತ್ತಿದ್ದಾರಲ್ಲ!’ ಎಂದು. ತಮ್ಮ ಈ ಅಭಿಪ್ರಾಯವನ್ನು ಸ್ವಾಮೀಜಿಯ ಮುಂದೆ ವ್ಯಕ್ತಪಡಿಸಿಯೂ ಬಿಟ್ಟರು. ಆಗ ಅವರಿಗೆ ಸ್ವಾಮೀಜಿ ಗಂಭೀರವಾಗಿ ಹೇಳುತ್ತಾರೆ: “ಏನು! ಒಬ್ಬ ವ್ಯಕ್ತಿ ಸಂನ್ಯಾಸಿ ಯಾದ ಮಾತ್ರಕ್ಕೆ ಅವನಿಗೆ ಹೃದಯವೇ ಇಲ್ಲ ಎಂದುಕೊಂಡಿರೇನು? ದಯವಿಟ್ಟು ಹಾಗೆ ಮಾತ ನಾಡಬೇಡಿ. ನಾವು ಭಾವಹೀನರಾದ ಶುಷ್ಕ ಸಂನ್ಯಾಸಿಗಳಲ್ಲ.”

ಬಲರಾಮಬಾಬುವಿನ ಕುಟುಂಬದವರೆಲ್ಲರೂ ಶ್ರೀರಾಮಕೃಷ್ಣರ ಭಕ್ತವರ್ಗಕ್ಕೆ ಸೇರಿದ್ದವರೇ. ಅಲ್ಲದೆ ಅವರು ಸಂನ್ಯಾಸೀಬಂಧುಗಳಿಗೆಲ್ಲ ಆತ್ಮೀಯರು. ಆದ್ದರಿಂದ ಈ ದುಃಖದ ಅವಧಿಯಲ್ಲಿ ಅವರೊಂದಿಗಿದ್ದು ಅವರಿಗೆ ಸಮಾಧಾನ ತರುವ ಹಾಗೂ ಮಠದ ಆಗುಹೋಗುಗಳನ್ನು ವಿಚಾರಿಸಿ ಕೊಳ್ಳುವ ಉದ್ದೇಶದಿಂದ ಸ್ವಾಮೀಜಿ ೧೮೯೦ರ ಏಪ್ರಿಲ್ನಲ್ಲಿ ಕಲ್ಕತ್ತದೆಡೆಗೆ ಹೊರಟರು.

