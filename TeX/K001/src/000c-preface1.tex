
\chapter*{ಪ್ರಕಾಶಕರ ಮಾತು}

ಸ್ವಾಮಿ ವಿವೇಕಾನಂದರು ಭರತಖಂಡಕ್ಕೆ ತನ್ನತನವನ್ನು ನೀಡಿದರು, ಹಿಂದೂಧರ್ಮಕ್ಕೆ ನಿರ್ದಿಷ್ಟ ರೂಪವನ್ನು ಕೊಟ್ಟರು, ವಿಶ್ವಧರ್ಮಕ್ಕೆ ಒಂದು ತಳಹದಿಯನ್ನು ಹಾಕಿದರು, ವಿವಿಧ ಜ್ಞಾನಕ್ಷೇತ್ರಗಳನ್ನು ಸಮನ್ವಯಗೊಳಿಸಿದರು, ಮಾನವನ ಶ್ರೇಷ್ಠತೆಯನ್ನು ಎತ್ತಿ ಸಾರಿ ಅವನ ಕೀಳರಿಮೆಯನ್ನು ಹೋಗಲಾಡಿಸಿದರು, ಮಾನವೀಯ ಭಾವನೆಗಳಿಗೆ ಹೆಚ್ಚು ಒತ್ತನ್ನು ನೀಡಿ ಮಾನವಸೇವೆಯ ಹೊಸ ಆದರ್ಶವನ್ನು ಬೆಳಕಿಗೆ ತಂದರು, ಆಧ್ಯಾತ್ಮಿಕ ತಳಹದಿಯನ್ನು ಆಧರಿಸಿದ ಹೊಸ ನೀತಿತತ್ತ್ವವನ್ನು ಪ್ರತಿಪಾದಿಸಿದರು, ಜಗತ್ತಿನ ಚಿಂತನಾಧಾರೆಯ ಮೇಲೆ ನಿರ್ದಿಷ್ಟ ಪ್ರಭಾವವನ್ನು ಬೀರಿದರು, ಯುವಜನತೆಗೆ ಒಂದು ಸ್ಫೂರ್ತಿಯ ಸೆಲೆಯಾಗಿ ನಿಂತರು ಮತ್ತು ಭವಿಷ್ಯತ್ತಿನ ಭವ್ಯ ಆದರ್ಶವನ್ನಿಟ್ಟುಕೊಂಡು ಭವಿಷ್ಯ ಜನಾಂಗವು ಮುಂದುವರಿಯುವಂತೆ ಮಾಡಿದರು. ಹೀಗೆ ಸ್ವಾಮಿ ವಿವೇಕಾನಂದರು ಎಂದೆಂದಿಗೂ ಹಳತಾಗದ ಚಿರನೂತನ ಸಂದೇಶದೊಂದಿಗೆ ಎಂದೆಂದಿಗೂ ಪ್ರಸ್ತುತವಾಗಿಯೇ ಉಳಿದಿದ್ದಾರೆ. ಅವರು ಆಶ್ರಯಿಸಿದ ಸನಾತನ ಸತ್ಯದಂತೆಯೇ ಚಿರಶಾಶ್ವತ ಆದರ್ಶಪ್ರಾಯ ವ್ಯಕ್ತಿಯಾಗಿ ಬೆಳಗುತ್ತಿದ್ದಾರೆ. ದಿನಗಳದಂತೆಲ್ಲ ಅವರ ವ್ಯಕ್ತಿತ್ವದ ಹೊಸ ಹೊಸ ಮುಖಗಳು, ಅವರ ಸಂದೇಶದ ಹೊಸ ಹೊಸ ಅಂಶಗಳು ಬೆಳಕಿಗೆ ಬರುತ್ತಾ, ಅವರು ಕಾಲಾತೀತ ವ್ಯಕ್ತಿಯಾಗಿ ಮೆರೆಯುತ್ತಿದ್ದಾರೆ.

ಇಂತಹ ಸ್ವಾಮಿ ವಿವೇಕಾನಂದರ ಜೀವನಚರಿತ್ರೆಯನ್ನು ಸ್ವಾಮಿ ಪುರುಷೋತ್ತಮಾನಂದರು ಮೂರು ಸಂಪುಟಗಳಲ್ಲಿ ಸವಿಸ್ತಾರವಾಗಿ ಅತ್ಯಂತ ಮನಮೋಹಕ ಪ್ರಬುದ್ಧ ಶೈಲಿಯಲ್ಲಿ ಬರೆದಿದ್ದಾರೆ, ಅವರ ವ್ಯಕ್ತಿತ್ವವನ್ನು ಪ್ರಸ್ಫುಟವಾಗಿ ಪ್ರತಿಬಿಂಬಿಸಿದ್ದಾರೆ. ೧೯೮೭ರಲ್ಲಿ ವರ್ಷದ ಅತ್ಯುತ್ತಮ ಜೀವನಚರಿತ್ರೆಗೆ ಮೈಸೂರಿನ ‘ದೇಜಗೌ ಟ್ರಸ್ಟ್’ ನೀಡುತ್ತಿರುವ ‘ವಿಶ್ವಮಾನವ ಪ್ರಶಸ್ತಿ’ಯ ಪುರಸ್ಕಾರವನ್ನು ಈ ಗ್ರಂಥವು ಪಡೆದಿದ್ದು ಕರ್ನಾಟಕ ಸಾರಸ್ವತ ಪ್ರಪಂಚದಲ್ಲಿ ಗಣ್ಯಸ್ಥಾನವನ್ನು ಗಳಿಸಿದೆ.

ಇದುವರೆಗೂ ಬೆಂಗಳೂರಿನ ರಾಮಕೃಷ್ಣ ಮಠದಿಂದ ಪ್ರಕಟವಾಗುತ್ತಿದ್ದ ಈ ಗ್ರಂಥವನ್ನು ಈಗ ನಾವು ಪ್ರಕಟಿಸಲು ಸಂತೋಷಿಸುತ್ತೇವೆ.

\bigskip

\noindent
ಶ್ರೀರಾಮಕೃಷ್ಣ ಆಶ್ರಮ\hfill\textbf{ಅಧ್ಯಕ್ಷರು}

\begin{flushleft}
ಮೈಸೂರು
\end{flushleft}

