
\chapter{ಆಶ್ಚರ್ಯಕರ ಗುರು–ಆಶ್ಚರ್ಯಕರ ಶಿಷ್ಯ}

\noindent

ಮನದಲ್ಲಿ ಆಳವಾಗಿ ಬೇರೂರಿರುವ ನಂಬಿಕೆಗಳನ್ನು ಬದಲಾಯಿಸಿಕೊಳ್ಳಬೇಕೆಂದರೆ ಎಂಥವರಿ ಗಾದರೂ ಕಷ್ಟವೇ. ದೃಢಚಿತ್ತರಾದವರ ವಿಷಯದಲ್ಲಿ ಇದು ಇನ್ನೂ ಸತ್ಯ. ಆಗ ಅಂಥವರ ಮನಸ್ಸು ತೀವ್ರವಾದ ಆಂತರಿಕ ಸಂಘರ್ಷವನ್ನು ಅನುಭವಿಸುತ್ತದೆ. ನರೇಂದ್ರನಿಗೊದಗಿದ ಸ್ಥಿತಿ ಇಂಥದು. ಅವನು ಶ್ರೀರಾಮಕೃಷ್ಣರ ಪವಾಡಶಕ್ತಿಯನ್ನು ಕಣ್ಣಾರೆ ಕಾಣುತ್ತಿದ್ದಾನೆ; ಆದರೂ ಅವರನ್ನು ಸಂಪೂರ್ಣವಾಗಿ ಸ್ವೀಕರಿಸಲಾರ. ಅವರ ಅಯಸ್ಕಾಂತೀಯ ವ್ಯಕ್ತಿತ್ವ ಒಂದೇ ಸಮನೆ ಸೆಳೆಯುತ್ತಿದೆ. ಆದರೆ ಆ ಸೆಳೆತಕ್ಕೆ ಒಳಗಾಗದೆ ದೂರವೇ ನಿಲ್ಲುವ ಪ್ರಯತ್ನ ನಡೆಸಿದ್ದಾನೆ.

ನರೇಂದ್ರನ ಎಲ್ಲ ಬಗೆಯ ಮಾನಸಿಕ ಹೋರಾಟವನ್ನು, ಸಂಶಯಗಳ ತುಮುಲವನ್ನು, ಭಾವಗಳ ಬಿರುಗಾಳಿಯನ್ನು ಶ್ರೀರಾಮಕೃಷ್ಣರು ಕಾಣುತ್ತಿದ್ದರು; ಅವುಗಳನ್ನು ಚೆನ್ನಾಗಿ ಅರ್ಥ ಮಾಡಿಕೊಳ್ಳುತ್ತಿದ್ದರು. ಏಕೆಂದರೆ ಅವರೂ ತಮ್ಮ ಸಾಧನಾಕಾಲದಲ್ಲಿ ಇಂತಹದೇ ಸ್ಥಿತಿಯನ್ನು ಹಾದುಹೋದವರಲ್ಲವೆ? ಈಗ ನರೇಂದ್ರನಲ್ಲಿ ಅದೇ ತೀವ್ರತೆಯನ್ನು ಕಂಡಾಗ ಅವರಿಗಾಗು ತ್ತಿದ್ದುದು ಸಂತೋಷವೇ. ಆತ್ಮಜಾಗೃತಿಯ ಸಮಯದಲ್ಲಿ ಸ್ವಾಭಾವಿಕವಾಗಿಯೇ ಉದ್ಭವಿಸುವ ಮಹಾ ಸಮರ ಇದು.

ಆದರೆ, ಸರ್ವಧರ್ಮಗಳನ್ನೂ ಸಮರ್ಥಿಸಿ, ಅವುಗಳ ಸಮನ್ವಯವನ್ನು ಬಿತ್ತರಿಸಲೆಂದು, ಮತ್ತು ಇಂದ್ರಿಯಭೋಗವೇ ಪರಮಪುರುಷಾರ್ಥವೆಂದು ಭ್ರಮಿಸುವ ಜನಕೋಟಿಗೆ ಭಗವ ದಾನಂದ ಪ್ರಾಪ್ತಿಯ ಮಾರ್ಗವನ್ನು ತೋರಿಸಕೊಡಲೆಂದು ಅವತಾರವೆತ್ತಿದವರು ಶ್ರೀರಾಮ ಕೃಷ್ಣರು. ಅವರ ಈ ಲೀಲಾಕಾರ್ಯವನ್ನು ಯಶಸ್ವಿಯಾಗಿ ಪೂರ್ಣಗೊಳಿಸುವುದೇ ನರೇಂದ್ರನ ಜನ್ಮೋದ್ದೇಶ. ಇದನ್ನು ಅರಿತಿದ್ದವರು ಶ್ರೀರಾಮಕೃಷ್ಣರೊಬ್ಬರೇ. ಈಗ ತಾವು ನರೇಂದ್ರನನ್ನು ಕೈಗೆ ತೆಗೆದುಕೊಂಡು ತಮಗೆ ಬೇಕಾದಂತೆ ರೂಪಿಸದಿದ್ದರೆ ಅವನು ದಾರಿತಪ್ಪಿಯಾನು ಎಂದು ಅವರು ಊಹಿಸಿದರು. ಏಕೆಂದರೆ ಅವನೊಳಗೆ ಪ್ರಚಂಡ ಶಕ್ತಿ ಮಹಾಪೂರವಾಗಿ ಹರಿಯುತ್ತಿತ್ತು. ಅವನೇನಾದರೂ ಅತ್ಯುನ್ನತ ಸಾಕ್ಷಾತ್ಕಾರವನ್ನು ಪಡೆದುಕೊಂಡು, ಸಕಲ ಮತಧರ್ಮಗಳ ಕುರಿತಾಗಿ ತಮ್ಮಂತೆಯೇ ಪರಮ ಉದಾರವೂ ಉದಾತ್ತವೂ ಆದ ದೃಷ್ಟಿಕೋನವನ್ನು ರೂಪಿಸಿಕೊಳ್ಳದಿದ್ದರೆ ತನ್ನೊಳಗಿನ ಆ ಶಕ್ತಿಯ ಒತ್ತಡದಿಂದ ಅವನು ಹೊಸದೊಂದು ಮತವನ್ನೋ ಪಂಥವನ್ನೋ ಸ್ಥಾಪಿಸುತ್ತಾನೆ; ಜಗತ್ತಿನ ಇತರ ಅನೇಕ ಧರ್ಮಗುರುಗಳಲ್ಲಿ ಒಬ್ಬನಾಗಿ ಪ್ರಸಿದ್ಧಿ ಪಡೆಯುತ್ತಾನೆ; ಆದರೆ ಅವನಿಂದ ಅದಕ್ಕಿಂತ ಘನತರವಾದ ಉದ್ದೇಶವಾವುದೂ ಈಡೇರುವುದಿಲ್ಲ ಎಂಬುದು ಶ್ರೀರಾಮಕೃಷ್ಣರಿಗೆ ತೋರಿತು. ಮತಪಂಥಗಳ ಉಗಮದ ಬಗ್ಗೆ ಅವರೇ ಹೇಳುತ್ತಿದ್ದರು: “ನೀರು ಹರಿಯದೆ ನಿಂತಿರುವ ಸಣ್ಣಪುಟ್ಟ ಕೆರೆಕುಂಟೆಗಳಲ್ಲೇ ಕಳೆಗಿಡಗಳೆಲ್ಲ ಬೆಳೆಯುವುದು. ಹರಿಯುವ ನದಿಗಳಲ್ಲಿ ಅವು ಕಾಣಸಿಗಲಾರವು. ಹಾಗೆಯೇ ಮನುಷ್ಯನೂ ಧಾರ್ಮಿಕ ಸತ್ಯದ ಯಾವುದೋ ಒಂದೊಂದು ಅಂಶವನ್ನು ಹಿಡಿದುಕೊಂಡು ಅಷ್ಟರಲ್ಲೇ ತೃಪ್ತನಾದರೆ ಅಲ್ಲಿಯೇ ಸಣ್ಣಪುಟ್ಟ ಮತಗಳು ಹುಟ್ಟಿಕೊಳ್ಳುತ್ತವೆ” ಎಂದು. ನರೇಂದ್ರನೂ ಇಂತಹ ಹೊಸದೊಂದು ಮತವನ್ನು ಹುಟ್ಟುಹಾಕಿದರೆ ಇಂದಿನ ಆಧುನಿಕ ಜಗತ್ತು ಯಾವ ಪರಮೋದಾರವಾದ ಆಧ್ಯಾತ್ಮಿಕ ತತ್ತ್ವಕ್ಕೆ ಹಸಿದಿದೆಯೋ ಅದನ್ನವನು ಜಗತ್ತಿಗೆ ನೀಡಲಾರ; ಹಾಗೇನಾದರೂ ಆದರೆ ಮಾತ್ರ ಅದು ಜಗದ ಜನರ ದೊಡ್ಡ ದೌರ್ಭಾಗ್ಯವಾಗುತ್ತದೆ ಎಂದು ಶ್ರೀರಾಮಕೃಷ್ಣರು ಕಂಡುಕೊಂಡರು. ಆದ್ದರಿಂದ ಹಾಗಾಗದಂತೆ ತಡೆದು, ಪರಿಪೂರ್ಣಜ್ಞಾನವನ್ನು ಪಡೆದುಕೊಳ್ಳಲು ಸಮರ್ಥನಾದ ವ್ಯಕ್ತಿಯಾಗು ವಂತೆ ನರೇಂದ್ರನಿಗೆ ಮಾರ್ಗದರ್ಶನ ನೀಡುವ ಕಾರ್ಯದಲ್ಲಿ ಅವರೀಗ ತೊಡಗಿದರು.

ಶ್ರೀರಾಮಕೃಷ್ಣರಿಂದ ಶಿಕ್ಷಣ ಪಡೆಯುತ್ತ, ಅವರ ಪವಿತ್ರ ಸನ್ನಿಧಿಯಲ್ಲಿ ನರೇಂದ್ರ ನಡೆಸಿದ ಜೀವನವೆನ್ನುವುದು ಆಧ್ಯಾತ್ಮಿಕ ಭಾವಜೀವನ; ಉನ್ನತ ಸಾಕ್ಷಾತ್ಕಾರಗಳ ಜೀವನ; ಆಧ್ಯಾತ್ಮಿಕ ಅನುಭವಪೂರ್ಣ ಜೀವನ. ಅವರಿಬ್ಬರ ನಡುವಣ ಬಾಂಧವ್ಯದ ಆಳ-ವೈಶಾಲ್ಯಗಳನ್ನು, ಜಟಿಲತೆ- ಸೂಕ್ಷ್ಮತೆಗಳನ್ನು ಬಣ್ಣಿಸಲಾಗದು. ಆ ಗುರುವೂ ಅದ್ಭುತ, ಈ ಶಿಷ್ಯನೂ ಅದ್ಭುತ. ಕಠೋಪ ನಿಷತ್ತು ಹೇಳುವಂತೆ, ‘ಆಶ್ಚರ್ಯೋ ವಕ್ತಾ ಕುಶಲೋsಸ್ಯ ಲಬ್ಧಾ ಆಶ್ಚರ್ಯೋ ಜ್ಞಾತಾ ಕುಶಲಾನುಶಿಷ್ಟ: –ಎಂದರೆ, ಆತ್ಮವಿದ್ಯೆಯನ್ನು ಬೋಧಿಸುವ ಗುರುವೂ ಆಶ್ಚರ್ಯಕರನಾಗಿರ ಬೇಕು; ಕಲಿಯುವ ಶಿಷ್ಯನೂ ಅತ್ಯಂತ ಕುಶಲಿಯಾಗಿರಬೇಕು. ಶ್ರಿರಾಮಕೃಷ್ಣ-ನರೇಂದ್ರರ ವಿಷಯದಲ್ಲಿ ಈ ಉಪನಿಷದ್ವಾಕ್ಯ ನಿಜಕ್ಕೂ ಅನ್ವರ್ಥವಾಗಿದೆ.

ಅತ್ಯುನ್ನತ ಸಾಕ್ಷಾತ್ಕಾರ ಮಾಡಿಕೊಂಡು ಹೃದಯದ ತುಂಬ ಅನುಭವಗಳನ್ನು ತುಂಬಿಕೊಂಡಿ ರುವ ಲೋಕಗುರುಗಳು, ಸಾಮರ್ಥ್ಯಶಾಲೀ ಶಿಷ್ಯನೊಬ್ಬನು ದೊರೆತಾಗ ಅವನಿಗೆ ತಮ್ಮ ಅಮೂಲ್ಯ ಅನುಭವಾಮೃತವನ್ನು ಧಾರೆಯೆರೆಯಲು ಕಾತರರಾಗಿರುತ್ತಾರೆ. ಶ್ರೀರಾಮಕೃಷ್ಣರೂ ಕಾದುಕುಳಿತಿದ್ದರು; ಅದಕ್ಕೆ ಸರಿಯಾಗಿ ಸಾಮರ್ಥ್ಯಶಾಲೀ ಶಿಷ್ಯ ನರೇಂದ್ರ ಬಂದೂಬಿಟ್ಟ; ಅವರು ತಮ್ಮ ಅನುಭವಾಮೃತವನ್ನು ಧಾರೆಯೆರೆಯಲೂ ಉತ್ಸಾಹದಿಂದ ಮುಂದಾದರು. ಆದರೆ ತಾವು ಅವನಿಗೆ ನಿರ್ವಿಕಲ್ಪ ಸಮಾಧಿಯ ಅನುಭವದ ರುಚಿಯನ್ನು ತೋರಿಸಹೋದಾಗ ಅವನು ಹೆದರಿ‘ನನಗೆ ಮನೆಯಲ್ಲಿ ತಾಯ್ತಂದೆಯರಿದ್ದಾರೆ, ಹೋಗಬೇಕು’ ಎಂದು ಕೂಗಿಕೊಂಡದ್ದನ್ನು ನೆನಪಿಸಿಕೊಂಡು ಶ್ರೀರಾಮಕೃಷ್ಣರು ಅವನನ್ನು ಆಗಾಗ ಛೇಡಿಸುತ್ತಿದ್ದರು: “ನೋಡಯ್ಯಾ, ಒಮ್ಮೆ ಒಬ್ಬ ಮನುಷ್ಯ ಸತ್ತು ಪ್ರೇತನಾದನಂತೆ. ಆ ಪ್ರೇತಕ್ಕೆ ಯಾರೂ ಜೊತೆ ಸಿಗದೆ ಒಂಟಿಯಾಗಿಯೇ ತಿರುಗುತ್ತಿತ್ತು. ಯಾರಾದರೂ ಸತ್ತರು ಅಂತ ಕೇಳಿ ಬಂದರೆ ಸಾಕು, ಅಲ್ಲಿಗೆ ಓಡುತ್ತಿತ್ತು– ತನಗೊಬ್ಬ ಜೊತೆಗಾರ ಸಿಕ್ಕಾನು ಅಂತ. ಆದರೆ, ಪಾಪ, ಅದಕ್ಕೆ ಪ್ರತಿಸಲವೂ ನಿರಾಸೆಯಾಗು ತ್ತಿತ್ತು. ಏಕೆಂದರೆ, ಆ ಸತ್ತ ವ್ಯಕ್ತಿ ಯಾವುದೋ ಒಂದು ಪುಣ್ಯಕಾರ್ಯ ಮಾಡಿದ್ದರಿಂದ ಮುಕ್ತನಾಗಿಹೋಗಿಬಿಡುತ್ತಿದ್ದ. ಆದ್ದರಿಂದ ಆ ಪ್ರೇತ ಪೆಚ್ಚುಮೋರೆ ಹಾಕಿಕೊಂಡು ಹಿಂದಿರುಗ ಬೇಕಾಗುತ್ತಿತ್ತು. ಈಗ, ನನ್ನ ಕಥೆಯೂ ಹಾಗೇ ಆಗಿದೆ. ನೀನು ಬಂದದ್ದನ್ನು ಕಂಡು ನಾನೆಣಿಸಿದೆ, ‘ಓ, ನನಗೊಬ್ಬ ಸ್ನೇಹಿತ ಸಿಕ್ಕಿದ!’ ಅಂತ. ಆದರೆ ನೀನು ಕೂಡ ‘ನನಗೆ ಮನೆಯಲ್ಲಿ ತಾಯ್ತಂದೆಯರಿದ್ದಾರೆ’ ಎಂದುಬಿಟ್ಟೆ. ನಾನೀಗ ಆ ಪ್ರೇತದ ಹಾಗೆ ಯಾರೂ ಜೊತೆಗಾರರೇ ಇಲ್ಲದೆ ಒಂಟಿಯಾಗಿಯೇ ಇರಬೇಕಾಗಿದೆ!”

ಆದರೆ ಶ್ರೀರಾಮಕೃಷ್ಣರು ಆ ಪ್ರೇತದಂತೆ ನಿರಾಶರಾಗಿ ಕುಳಿತುಕೊಳ್ಳುವವರಲ್ಲ. ಅವರು ನರೇಂದ್ರನನ್ನು ತಮ್ಮ ಕೈಯಿಂದ ತಪ್ಪಿಸಿಕೊಳ್ಳಲು ಬಿಟ್ಟರೆ ತಾನೆ? ಆತನನ್ನು ಅವರು ಬಲವಾಗಿ ಹಿಡಿದಿಟ್ಟುಕೊಂಡುಬಿಟ್ಟಿದ್ದರು!

ನರೇಂದ್ರನನ್ನು ಶ್ರೀರಾಮಕೃಷ್ಣರೊಂದಿಗೆ ಬಿಗಿದಿಟ್ಟ ಆ ಕಟ್ಟು ಯಾವುದಿರಬಹುದು? ಅದು ಅವರ ಪ್ರೇಮಬಂಧನ. ಆದರೆ, ಆ ಬಂಧನದ ಸ್ವರೂಪವನ್ನು ಮಾತ್ರ ಅರಿಯಬಲ್ಲವರಿಲ್ಲ, ಬಣ್ಣಿಸಬಲ್ಲವರಿಲ್ಲ. ಅದೊಂದು ಪ್ರವಾಹ; ನರೇಂದ್ರನನ್ನು ತರಗೆಲೆಯಂತೆ ಕೊಚ್ಚಿಕೊಂಡು ಹೋದ ಪ್ರೇಮಪ್ರವಾಹ. ಶ್ರೀರಾಮಕೃಷ್ಣರು ನರೇಂದ್ರನನ್ನು ಕಂಡ ಪ್ರಥಮ ಘಳಿಗೆಯಲ್ಲೇ ಅದರ ಉಗಮವಾಯಿತು. ಅದು ಭರತ-ಇಳಿತಗಳಿಲ್ಲದ ನಿರಂತರ ಮಹಾಪೂರ. ನರೇಂದ್ರ ಜೊತೆಗಿದ್ದಾಗ ಅವರಿಗಾಗುತ್ತಿದ್ದ ಆನಂದವನ್ನಾಗಲಿ, ಅವನು ಕೆಲಕಾಲ ಕಣ್ಮರೆಯಾಗಿದ್ದರೆ ಅವರು ಅನುಭವಿಸುತ್ತಿದ್ದಂತಹ ದುಗಡ-ದುಮ್ಮಾನವನ್ನಾಗಲಿ ಮಾನುಷ ಇತಿಹಾಸದಲ್ಲೇ ಯಾರೂ ಕಂಡಿರಲಾರರು.

ಮೊದಮೊದಲಿಗೆ ಅವನು ವಾರಕ್ಕೊಮ್ಮೆಯೋ ಪಕ್ಷಕ್ಕೊಮ್ಮೆಯೋ ದಕ್ಷಿಣೇಶ್ವರಕ್ಕೆ ಬಂದು ಹೋಗುತ್ತಿದ್ದ. ಕ್ರಮೇಣ ಈ ಅಂತರ ಕಡಿಮೆಯಾಗುತ್ತ ಬಂದಿತ್ತು. ವಾರಕ್ಕೆ ಎರಡು-ಮೂರು ಸಲ, ಅವಕಾಶವಾದಾಗ ಇನ್ನೂ ಹೆಚ್ಚು ಸಲ ಬರತೊಡಗಿದ. ರಜೆಯಲ್ಲಂತೂ ಮೂರ್ನಾಲ್ಕು ದಿನ ಅಲ್ಲಿಯೇ ಇದ್ದುಬಿಡುವುದೂ ಸರ್ವೇಸಾಮಾನ್ಯವಾಯಿತು. ಆದರೆ ಆಗಾಗ ಕಾರಣಾಂತರದಿಂದ ದಕ್ಷಿಣೇಶ್ವರಕ್ಕೆ ಚಕ್ಕರ್ ಕೊಡುವುದೂ ಇತ್ತು. ಆದರೆ ಶ್ರೀರಾಮಕೃಷ್ಣರು ಅದಕ್ಕೆ ಹೆಚ್ಚು ಅವಕಾಶ ಕೊಡುತ್ತಿರಲಿಲ್ಲ. ಯಾರೊಡನೆಯಾದರೂ ಮತ್ತೆಮತ್ತೆ ಹೇಳಿಕಳಿಸಿ ಅವನನ್ನು ಕರೆಸಿಕೊಳ್ಳುತ್ತಿದ್ದರು.

ಶ್ರೀರಾಮಕೃಷ್ಣರು ನರೇಂದ್ರನ ಮೇಲಿರಿಸಿದ್ದ ವಿಶ್ವಾಸ, ಪ್ರೇಮ, ಮೆಚ್ಚುಗೆ, ಆತ್ಮೀಯತೆ– ಇವು ಅವರ ಒಂದೊಂದು ಮಾತಿನಲ್ಲೂ ಒಂದೊಂದು ಕೃತಿಯಲ್ಲೂ ಎದ್ದೆದ್ದು ಕಾಣುತ್ತಿತ್ತು. ಅದನ್ನು ಕಂಡವರಿಗೆಲ್ಲ ಆಶ್ಚರ್ಯ, ಕುತೂಹಲ. ‘ಎಲಾ, ಈ ಕಾಲೇಜು ಹುಡುಗನಲ್ಲಿ ಅಂಥದೇ ನನ್ನು ಕಂಡರಪ್ಪ ಇವರು!’ ಎಂದು ಬೆರಗಾಗುವವರೇ ಎಲ್ಲ. ಅವರಿವರಿಗಿರಲಿ, ಸ್ವತಃ ನರೇಂದ್ರನಿಗೂ ಶ್ರೀರಾಮಕೃಷ್ಣರ ವರ್ತನೆಯಿಂದ ಇರುಸು ಮುರುಸಾಗುತ್ತಿತ್ತು. ಕೆಲವೊಮ್ಮೆ ಅವನು ‘ಇವೆಲ್ಲ ಆ ಮುದುಕನ ಭ್ರಮೆ’ ಎನ್ನುತ್ತಿದ್ದ. ಆದರೆ ಇನ್ನು ಕೆಲವು ಸಲ ಅವರ ನಿಸ್ವಾರ್ಥ ಪ್ರೇಮಕ್ಕೆ ಪ್ರತಿ ಸ್ಪಂದಿಸಿ ಓಗೊಡುತ್ತಿದ್ದ. ಒಮ್ಮೆ ಅನೇಕ ದಿನಗಳವರೆಗೆ ನರೇಂದ್ರ ಕಾಣಿಸಿ ಕೊಳ್ಳಲೇ ಇಲ್ಲ. ಶ್ರೀರಾಮಕೃಷ್ಣರು ಬಹಳ ವ್ಯಾಕುಲರಾಗಿ ಅವನನ್ನು ಇದಿರುನೋಡುತ್ತಿದ್ದರು. ಒಂದು ದಿನ ಸಂಜೆಯ ಹೊತ್ತಿಗೆ ರಾಮದಯಾಲ ಎಂಬೊಬ್ಬ ಭಕ್ತ, ರಾಖಾಲ ಮತ್ತು ಬಾಬುರಾಮ ಎಂಬೊಬ್ಬ ತರುಣ–ಇಷ್ಟು ಜನ ಬಂದರು. ಬಾಬುರಾಮ ದಕ್ಷಿಣೇಶ್ವರಕ್ಕೆ ಬಂದದ್ದು ಆ ದಿನವೇ ಮೊದಲು. ಶ್ರೀರಾಮಕೃಷ್ಣರು ಮೂವರನ್ನೂ ವಿಶ್ವಾಸದಿಂದ ಮಾತನಾಡಿ ಸಿದರು; ಬಳಿಕ ರಾಮದಯಾಲನನ್ನು ಕಾತರದಿಂದ ಕೇಳಿದರು: “ಏನಪ್ಪ, ನರೇಂದ್ರನನ್ನು ನೋಡಿದೆಯಾ? ಹೇಗಿದ್ದಾನೆ ಅವನು, ಆರೋಗ್ಯದಿಂದಿದ್ದಾನಷ್ಟೆ?”

“ಹೌದು, ನೋಡಿದೆ, ಚೆನ್ನಾಗಿಯೇ ಇದ್ದಾನೆ.”

“ನೋಡು, ಅವನಿಲ್ಲಿಗೆ ಬಂದು ಎಷ್ಟೊಂದು ದಿನ ಆಗಿಹೋಯಿತು! ಅವನನ್ನು ನೋಡಬೇಕು ಅಂತ ನನಗೆ ತುಂಬ ಇಚ್ಛೆಯಾಗುತ್ತಿದೆ. ದಯವಿಟ್ಟು ಅವನಿಗೆ ಇಲ್ಲಿಗೊಮ್ಮೆ ಬಂದುಹೋಗಲು ತಿಳಿಸುತ್ತೀಯಾ?”

ರಾಮದಯಾಲ ಸರಿಯೆಂದ, ಆ ಸಂಜೆ ಬಹಳ ಆನಂದದಿಂದ ಭಗವದ್ವಿಚಾರದಲ್ಲಿ ಕಳೆಯಿತು. ರಾತ್ರಿ ಹತ್ತುಗಂಟೆಗೆ ಮೂವರೂ ಅಲ್ಲೇ ಊಟ ಮಾಡಿದರು. ರಾಮದಯಾಲನೂ ಬಾಬು ರಾಮನೂ ಪಕ್ಕದ ವರಾಂಡದಲ್ಲಿ ಮಲಗಿದರು. ಒಂದು ತಾಸೂ ಕಳೆದಿಲ್ಲ, ಅಷ್ಟರಲ್ಲೇ ಶ್ರೀರಾಮ ಕೃಷ್ಣರು ಎದ್ದು ಅವರಿದ್ದಲ್ಲಿಗೆ ಬಂದು ಮೃದುವಾಗಿ ಕರೆದರು:“ಓ, ನಿದ್ದೆ ಹತ್ತಿತೇನು?” ತಕ್ಷಣ ಇಬ್ಬರೂ, “ಇಲ್ಲ, ಇಲ್ಲ” ಎನ್ನುತ್ತ ಗಡಿಬಿಡಿಯಿಂದ ಎದ್ದರು. ನೋಡುತ್ತಾರೆ, ಶ್ರೀರಾಮಕೃಷ್ಣರು ಪುಟ್ಟ ಹುಡುಗನಂತೆ ತಮ್ಮ ಮೈಮೇಲಿನ ಬಟ್ಟೆಯನ್ನು ಮುದ್ದೆಮಾಡಿ ಕಂಕುಳಲ್ಲಿಟ್ಟುಕೊಂಡು ನಿಂತಿದ್ದಾರೆ!

ಶ್ರೀರಾಮಕೃಷ್ಣರು (ರಾಮದಯಾಲನಿಗೆ): “ಏನಪ್ಪ, ನರೇಂದ್ರನನ್ನು ಬಹಳ ದಿನಗಳಿಂದಲೂ ನೋಡದೆ ನನ್ನ ಹೃದಯವನ್ನೇ ಯಾರೋ (ತಮ್ಮ ಕೈಯಲ್ಲಿದ್ದ ಬಟ್ಟೆಯನ್ನು ಹಿಂಡಿ ತೋರಿಸಿ) ಹೀಗೆ ಹಿಂಡುತ್ತಿದ್ದಾರೋ ಎನ್ನುವಷ್ಟು ನೋವಾಗುತ್ತಿದೆ. ನನ್ನನ್ನೊಮ್ಮೆ ನೋಡಿಹೋಗುವ ಹಾಗೆ ದಯವಿಟ್ಟು ಅವನಿಗೆ ಹೇಳುತ್ತೀಯಾ? ಅವನು ಪರಿಶುದ್ಧ ಸತ್ವದ ವ್ಯಕ್ತಿ. ಅವನು ಸ್ವಯಂ ನಾರಾಯಣ. ಅವನನ್ನು ಆಗಾಗ ನೋಡದಿದ್ದರೆ ನನಗೆ ನೆಮ್ಮದಿಯೇ ಇರುವುದಿಲ್ಲ.”

ರಾಮದಯಾಲನಿಗೆ ಶ್ರೀರಾಮಕೃಷ್ಣರ ಮುಗ್ಧಸ್ವಭಾವದ ಪರಿಚಯ ಚೆನ್ನಾಗಿಯೇ ಇತ್ತು. ಅವನು “ನೀವೇನೂ ಯೋಚನೆ ಮಾಡಬೇಡಿ, ಬೆಳಗಾಗುತ್ತಲೇ ನಾನು ಹೋಗಿ ಅವನಿಗೆ ಹೇಳಿ ಇಲ್ಲಿಗೆ ಕಳಿಸುತ್ತೇನೆ” ಎಂದು ಹೇಳಿ ಅವರನ್ನು ಸಂತೈಸಿದ. ಶ್ರೀರಾಮಕೃಷ್ಣರು ಕೋಣೆಗೆ ಮರಳಿದರು; ಇವರಿಬ್ಬರೂ ಮತ್ತೆ ಮಲಗಿದರು. ಆದರೆ ಅರ್ಧಗಂಟೆಯಾಗುವುದಕ್ಕಿಲ್ಲ, ಅವರು ಮತ್ತೆ ಬಂದುಬಿಟ್ಟಿದ್ದಾರೆ! “ ಓ ರಾಮದಯಾಲ್, ನರೇಂದ್ರ ಯಾಕೆ ಬಾರದೇ ಇದ್ದಿರಬಹುದು? ಆಹ್, ಅವನೆಂತಹ ಪರಿಶುದ್ಧಾತ್ಮ ಗೊತ್ತೇ... ” ಹೀಗೆನ್ನುತ್ತ ಅವನನ್ನು ಬಾಯ್ತುಂಬ ಕೊಂಡಾಡಿದರು. ಆದರೆ ಅಷ್ಟರಲ್ಲಿ ತಾನು ಇವರಿಬ್ಬರ ನಿದ್ರೆಗೆ ಭಂಗ ತರುತ್ತಿದ್ದೇನೆ ಎಂಬ ನೆನಪಾಗಿ ಹಿಂದಿರುಗಿದರು. ಮರುಘಳಿಗೆಯೇ ಅದನ್ನು ಮರೆತು ತಿರುಗಿ ಬಂದರು; ನರೇಂದ್ರನ ಗುಣಗಾನ ಮಾಡಿ ತಮ್ಮೆದೆಯ ಅಳಲನ್ನು ಕರುಣಾಜನಕವಾಗಿ ಬಣ್ಣಿಸಿದರು! ಆ ರಾತ್ರಿಯೆಲ್ಲ ಹೀಗೆಯೇ ಕಳೆಯಿತು.

ಇದನ್ನೆಲ್ಲ ನೋಡುತ್ತಿದ್ದ ಬಾಬುರಾಮ ಅತ್ಯಾಶ್ಚರ್ಯದಿಂದ ಸ್ತಬ್ಧನಾದ. ಅವನಿಗನ್ನಿಸಿತು: ‘ಎಲಾ, ಇದೆಂಥ ಅದ್ಭುತ ಪ್ರೇಮ! ಆದರೆ ಆ ನರೇಂದ್ರ ಎಂಥ ಕಲ್ಲೆದೆಯವನಿರಬೇಕು! ಇವರು ಅವನಿಗಾಗಿ ಅಷ್ಟೊಂದು ಅತ್ತುಕರೆಯುತ್ತಿದ್ದಾರೆ; ಆದರೂ ಅವನು ಬರದೆ ಸತಾಯಿಸುತ್ತಿದ್ದಾನಲ್ಲ!’

ಈ ರೀತಿ ಶ್ರೀರಾಮಕೃಷ್ಣರು ನರೇಂದ್ರನ ಬರವಿಗಾಗಿ ವ್ಯಾಕುಲಗೊಳ್ಳುತ್ತಿದ್ದುದು ತೀರ ಸಾಮಾನ್ಯ ದೃಶ್ಯ. ಇನ್ನೊಮ್ಮೆ ಅವನು ಹೀಗೆಯೇ ಬಹಳ ದಿನಗಳವರೆಗೆ ಕಾಣಿಸಿಕೊಳ್ಳದಿದ್ದಾಗ ಶ್ರೀರಾಮಕೃಷ್ಣರು ಚಡಪಡಿಸುತ್ತಿದ್ದುದನ್ನು ಕಂಡ ಅವರ ಭಕ್ತನಾದ ವೈಕುಂಠನಾಥ ಸನ್ಯಾಲ ಎಂಬವನು ಹೇಳುತ್ತಾನೆ: “ಆ ದಿನವೆಲ್ಲ ಶ್ರೀರಾಮಕೃಷ್ಣರು ನರೇಂದ್ರನ ವಿಷಯವಾಗೇ ಮಾತ ನಾಡುತ್ತಿದ್ದರು, ಅವನನ್ನು ಕೊಂಡಾಡುತ್ತಿದ್ದರು; ಹಾಗೇ ಮಾತನಾಡುತ್ತ ಆಡುತ್ತ ಇನ್ನಷ್ಟು ಉದ್ವಿಗ್ನರಾದರು. ಕಡೆಗೆ ಮೈಮರೆತರು. ಅವನ ವಿರಹವನ್ನು ತಡೆದುಕೊಳ್ಳಲಾರದೆ ಹೊರಗಡೆ ಜಗಲಿಗೆ ಓಡಿ, ‘ಅಮ್ಮಾ, ಜಗನ್ಮಾತೆ! ಅವನನ್ನು ಕಾಣದೆ ನಾನು ಬದುಕಲಾರೆ’ ಎಂದು ಗಟ್ಟಿಯಾಗಿ ಕೂಗಿಕೊಂಡರು. ಮತ್ತೆ ಕೋಣೆಗೆ ಬಂದು ‘ನಾನೆಷ್ಟೊಂದು ಅತ್ತಿದ್ದೇನೆ! ನರೇಂದ್ರ ಮಾತ್ರ ಇನ್ನೂ ಬರಲೇ ಇಲ್ಲವಲ್ಲಾ. ನನ್ನ ಎದೆಯನ್ನು ಯಾರೋ ಹಿಂಡಿದ ಹಾಗೆ ಯಾತನೆ ಯಾಗುತ್ತಿದೆ. ಆದರೆ ಅವನಿಗೆ ಇದೊಂದೂ ಲಕ್ಷ್ಯವೇ ಇಲ್ಲ’ ಎಂದರು. ಹೀಗೆ ಹೇಳಿ ಪುನಃ ಜಗಲಿಗೆ ಓಡಿದರು; ಮತ್ತೆ ಅಲ್ಲಿಂದ ಬೇಗ ಹಿಂದಿರುಗಿ ಬಂದು ತಮ್ಮ ಅಳಲನ್ನು ತೋಡಿ ಕೊಂಡರು: ‘ನಾನೋ ಮುದುಕ. ಆ ಹುಡುಗನಿಗಾಗಿ ತವಕಿಸುತ್ತಿದ್ದೇನೆ. ಇದನ್ನೆಲ್ಲ ನೋಡಿದರೆ ಜನ ಏನೆಂದುಕೊಂಡಾರು! ನೀವೆಲ್ಲ ನನ್ನ ಸ್ವಂತದವರು. ನಿಮ್ಮ ಮುಂದೆ ನನ್ನ ಮನಸ್ಥಿತಿಯನ್ನು ಹೇಳಿಕೊಳ್ಳಲು ನನಗೇನೂ ನಾಚಿಕೆಯಿಲ್ಲ. ಆದರೆ ಇತರರೆಲ್ಲ ಇದನ್ನು ಹೇಗೆ ತೆಗೆದುಕೊಳ್ಳು ತ್ತಾರೋ! ನನಗಂತೂ ತಡೆದುಕೊಳ್ಳುವುದಕ್ಕೆ ಆಗುತ್ತಿಲ್ಲ.’ ಹೀಗೆ ಅವರು ಹೇಳಿತೀರದಷ್ಟು ಸಂಕಟವನ್ನನುಭವಿಸಿದರು. ಆದರೆ ಕೊನೆಗೆ ನರೇಂದ್ರ ಬಂದಾಗ ಮಾತ್ರ ಅವರಿಗಾದ ಆನಂದವೂ ಅಷ್ಟೇ ತೀವ್ರ.”

ಒಮ್ಮೆ ಭಕ್ತರೆಲ್ಲ ಸೇರಿ ದಕ್ಷಿಣೇಶ್ವರದಲ್ಲಿ ಶ್ರೀರಾಮಕೃಷ್ಣರ ಜನ್ಮದಿನವನ್ನು ಆಚರಿಸು ತ್ತಿದ್ದರು. ಆದರೆ ಮಧ್ಯಾಹ್ನವಾದರೂ ಅಚ್ಚುಮೆಚ್ಚಿನ ಶಿಷ್ಯ ನರೇಂದ್ರ ಮಾತ್ರ ಇನ್ನೂ ಬರಲೇ ಇಲ್ಲ! “ ಅವನು ಬಂದನೆ? ಬಂದನೆ?” ಎಂದು ಶ್ರೀರಾಮಕೃಷ್ಣರು ಮತ್ತೆಮತ್ತೆ ಕೇಳುತ್ತಿದ್ದಾರೆ. ಕೊನೆಗೆ ನರೇಂದ್ರ ಬಂದ; ಬಂದು ಗುರುದೇವನಿಗೆ ನಮಸ್ಕರಿಸಿದ. ಅವನನ್ನು ನೋಡಿದ್ದೇ ತಡ, ಶ್ರೀರಾಮಕೃಷ್ಣರು ಅವನ ಭುಜದ ಮೇಲೊರಗಿಕೊಂಡು ಹಾಗೆಯೇ ಭಾವಸಮಾಧಿಸ್ಥರಾಗಿ ಬಿಟ್ಟರು! ಬಳಿಕ ಮೆಲ್ಲನೆ ಪ್ರಕೃತಿಸ್ಥರಾಗಿ ಶಿಷ್ಯನ ಮೈದಡುವುತ್ತ, ಕೈಯಾರೆ ಸಿಹಿತಿಂಡಿ ತಿನ್ನಿಸಿದರು. ಇನ್ನೊಮ್ಮೆ ನರೇಂದ್ರ ಹಲವಾರು ದಿನಗಳವರೆಗೆ ಬಾರದೆ ಕೊನೆಗೊಂದು ದಿನ ಬಂದ. ಅವನು ದಕ್ಷಿಣೇಶ್ವರದ ದಡದಲ್ಲಿ ದೋಣಿಯಿಂದ ಇಳಿಯುವುದನ್ನು ಕಂಡ ಶ್ರೀರಾಮಕೃಷ್ಣರು ಅಲ್ಲಿಗೇ ಓಡಿಬಂದರು; ವಾತ್ಸಲ್ಯದಿಂದ ಶಿಷ್ಯನ ಮುಖವನ್ನು ನೇವರಿಸುತ್ತ, ಓಂಕಾರವನ್ನುಚ್ಚರಿಸುತ್ತ ಹಾಗೆಯೇ ಸಮಾಧಿಸ್ಥರಾಗಿಬಿಟ್ಟರು!

ಮುಂದೆ ಎಷ್ಟೋ ಸಲ ನರೇಂದ್ರ ಮನೆಯಲ್ಲಿನ ತೊಂದರೆಗಳಿಂದಾಗಿ ದಕ್ಷಿಣೇಶ್ವರಕ್ಕೆ ಬಾರದಿದ್ದಾಗ, ಶ್ರೀರಾಮಕೃಷ್ಣರು ತಮಗೆ ತಾವೇ ಒಂದು ಬಗೆಯಲ್ಲಿ ಸಮಾಧಾನ ಹೇಳಿಕೊಳ್ಳು ತ್ತಿದ್ದುದೂ ಉಂಟು–“ಅವನು ಬರದಿದ್ದುದೇ ಒಳ್ಳೆಯದಾಯಿತು. ಅವನನ್ನು ನೋಡಿದರೆ ನನ್ನ ಭಾವಗಳಲ್ಲಿ ಅಲ್ಲೋಲಕಲ್ಲೋಲವೇ ನಡೆದುಬಿಡುತ್ತದೆ. ಅವನು ಇಲ್ಲಿಗೆ ಬಂದನೆಂದರೆ ಅದೇ ಒಂದು ದೊಡ್ಡ ಕಥೆಯಾಗಿಬಿಡುತ್ತದೆ”ಎಂದು.

ಶ್ರೀರಾಮಕೃಷ್ಣರ ನರೇಂದ್ರಪ್ರೇಮಕ್ಕೆ ಉದಾಹರಣೆಯಾದ ಒಂದು ಘಟನೆ:

೧೮೮೫ರ ಮೇ ತಿಂಗಳಿನಲ್ಲೊಂದು ಶನಿವಾರ. ಶ್ರೀರಾಮಕೃಷ್ಣರು ರಾಮಚಂದ್ರದತ್ತನ ಮನೆಗೆ ಬರಲಿದ್ದಾರೆ. ಒಂದು ದೊಡ್ಡ ಗುಂಪು ಅವರ ನಿರೀಕ್ಷೆಯಲ್ಲಿದೆ. ಶ್ರೀರಾಮಕೃಷ್ಣರು ಗಾಡಿಯಲ್ಲಿ ಬಂದಿಳಿದರು. ಬೈಠಕ್ಖಾನೆಗೆ ಪ್ರವೇಶಿಸಿ ಕುಳಿತೊಡನೆಯೇ ಯಾರನ್ನೋ ನಿರೀಕ್ಷಿಸುತ್ತ ಅತ್ತಿತ್ತ ನೋಡಿದರು. “ನರೇನ್ ಎಲ್ಲಿ, ಕಾಣುತ್ತಿಲ್ಲವಲ್ಲ?” ಎಂದು ಕಾತರದ ದನಿಯಲ್ಲಿ ಕೇಳಿದರು. ನರೇಂದ್ರನ ಮನೆ ಅಲ್ಲೇ ಹತ್ತಿರದಲ್ಲೇ. ಅವನು ಬರಲೇಬೇಕಾಗಿತ್ತು. ಆದರೆ ಏಕೆ ಬರಲಿಲ್ಲವೋ! ಆಗ ಯಾರೋ ಹೇಳಿದರು, “ ಅವನಿಗೆ ತಲೆಸಿಡಿತವಂತೆ, ಮಲಗಿಬಿಟ್ಟಿದ್ದಾನೆ.” ಇದನ್ನು ಕೇಳು ತ್ತಿದ್ದಂತೆ ಶ್ರೀರಾಮಕೃಷ್ಣರು ಉದ್ವಿಗ್ನರಾದರು. “ಹೌದೆ! ಹಾಗಾದರೆ ನಾನವನನ್ನು ನೋಡಲೇ ಬೇಕು. ಯಾರಾದರೂ ಹೋಗಿ ಅವನನ್ನು ಕರೆತನ್ನಿ! ಹೇಗಾದರೂ ಸರಿಯೆ!”

ನಿರಂಜನ, ಕಾಳೀಪ್ರಸಾದ ಮೊದಲಾದ ತರುಣಶಿಷ್ಯರು ನರೇಂದ್ರನ ಮನೆಗೆ ಹೊರಟರು. ಹೋಗಿ ನೋಡುತ್ತಾರೆ–ನರೇಂದ್ರ ಮಂಚದ ಮೇಲೆ ಮಲಗಿದ್ದಾನೆ; ಹಣೆಯ ಮೇಲೆ ಒದ್ದೆ ಪಟ್ಟಿ. ಯಾತನೆಯಿಂದ ಚಡಪಡಿಸುತ್ತಿದ್ದಾನೆ. ಇವನನ್ನು ಈ ಸ್ಥಿತಿಯಲ್ಲಿ ಎಬ್ಬಿಸುವುದಾದರೂ ಹೇಗೆ?... ಆದರೂ ಕಾಳೀಪ್ರಸಾದ ಬಾಯ್ದೆರೆದ:

“ನರೇನ್, ಠಾಕೂರರು ರಾಮ್ಬಾಬುವಿನ ಮನೆಗೆ ಬಂದಿದ್ದಾರೆ. ನಿನ್ನನ್ನು ನೋಡುವುದಕ್ಕೆ ತುಂಬ ಕಾತರರಾಗಿದ್ದಾರೆ. ನಿನ್ನನ್ನು ಅಲ್ಲಿಗೆ ಕರೆದುಕೊಂಡು ಹೋಗುವುದಕ್ಕೆ ಅಂತ ಬಂದಿದ್ದೇವೆ.”

“ನನಗೆ ಭಯಂಕರ ತಲೆನೋವು! ಎದ್ದೇಳುವುದಿರಲಿ, ಕಣ್ಣುಬಿಡುವುದಕ್ಕೂ ಆಗುತ್ತಿಲ್ಲ. ಬೆಳಕು ಕಂಡರೆ ಸಹಿಸುವುದಕ್ಕಾಗುವುದಿಲ್ಲ. ನಾನಲ್ಲಿಗೆ ಹೇಗೆ ಬರಲಪ್ಪಾ? ದಯವಿಟ್ಟು ಅವರಿಗೆ ನನ್ನ ನಮಸ್ಕಾರಗಳನ್ನು ತಿಳಿಸಿಬಿಟ್ಟು, ನನ್ನ ಪರವಾಗಿ ಕ್ಷಮೆ ಕೋರುತ್ತೀಯಾ?”–ನರೇಂದ್ರ ಬಹಳ ಕಷ್ಟಪಡುತ್ತ ನಿಧಾನವಾಗಿ ಹೇಳಿದ.

“ಆದರೆ ಠಾಕೂರರು ನಿನ್ನನ್ನು ನೋಡಲೇಬೇಕು ಅಂತ ಉತ್ಕಂಠಿತರಾಗಿ ಕಾಯುತ್ತಿದ್ದಾರೆ. ಹೇಗಾದರೂ ಮಾಡಿ ಬಂದುಬಿಡು. ನಾವು ಮೆಲ್ಲನೆ ನಡೆಸಿಕೊಂಡು ಹೋಗುತ್ತೇವೆ. ಇಲ್ಲವೆನ್ನ ಬೇಡ ನರೇನ್, ದಯವಿಟ್ಟು!”

ನರೇಂದ್ರ ಇದಕ್ಕೊಪ್ಪಿ ಎದ್ದುನಿಂತ. ತಲೆಯ ಮೇಲೆ ಒದ್ದೆ ಪಟ್ಟಿ ಹಾಗೆಯೇ ಇತ್ತು. ಕಾಳೀಪ್ರಸಾದ, ನಿರಂಜನ ಇಬ್ಬರೂ ಅತ್ತಿತ್ತ ನಿಂತು ಅವನ ಕೈಗಳನ್ನು ತಮ್ಮ ಭುಜದ ಮೇಲಿರಿಸಿಕೊಂಡು ಮೆಲ್ಲನೆ ಮುನ್ನಡೆಸಿಕೊಂಡು ಬಂದರು.

ಶ್ರೀರಾಮಕೃಷ್ಣರು ಭಕ್ತರಿಂದಾವೃತರಾಗಿ ಬೈಠಕ್ ಖಾನೆಯ ಮಧ್ಯದಲ್ಲಿ ಕುಳಿತಿದ್ದಾರೆ. ನರೇಂದ್ರನನ್ನು ನೋಡುತ್ತಲೇ ಅವರ ಮುಖ ಮಂದಹಾಸದಿಂದ ಬೆಳಗಿತು. ಎಲ್ಲರ ಕಣ್ಣುಗಳೂ ಅವರಿಬ್ಬರ ಮೇಲೆಯೇ ನೆಟ್ಚಿವೆ. ನರೇಂದ್ರನನ್ನು ಕಂಡು ಶ್ರೀರಾಮಕೃಷ್ಣರಿಗೆ ಎಷ್ಟು ಆನಂದ ವಾಗಿದೆಯೆಂಬುದನ್ನು ವಿಸ್ಮಯದಿಂದ ನೋಡುತ್ತಿದ್ದಾರೆ. ನರೇಂದ್ರ ಬಂದು ತಮಗೆ ಪ್ರಣಾಮ ಮಾಡುತ್ತಲೇ ಶ್ರೀರಾಮಕೃಷ್ಣರು ಅತ್ಯಂತ ವಾತ್ಸಲ್ಯದಿಂದ ತಮ್ಮೆರಡೂ ಕೈಗಳಿಂದ ಅವನ ತಲೆಯನ್ನು ಮುಟ್ಟಿ ಹರಸಿದರು. ಮರುಕ್ಷಣವೇ ಅವನು ಪುಟಿದೆದ್ದು ನಿಂತ. ಅವನೊಳಗೆ ಇದ್ದಕ್ಕಿದ್ದಹಾಗೆ ಏನೋ ಆಗಿಬಿಟ್ಟಿದೆ! ಅತ್ಯಾಶ್ಚರ್ಯದಿಂದ ಕೂಗಿಕೊಂಡ:“ಏನು ಮಾಡಿದಿರಿ ನೀವು! ನನ್ನ ತಲೆಶೂಲೆ ಹೊರಟುಹೋಯಿತಲ್ಲ!” ತನ್ನ ಮಾತಿನಲ್ಲಿ ತನಗೇ ನಂಬಿಕೆಯಿಲ್ಲದವ ನಂತೆ ಮತ್ತೆ ಕೂಗಿದ: “ಇದೇನೋ ಜಾದೂ ಮಾಡಿದಂತಿದೆ! ನಿಜಕ್ಕೂ ನನ್ನ ತಲೆನೋವು ಹೊರಟುಹೋಯಿತು!”

ಶ್ರೀರಾಮಕೃಷ್ಣರು ಪ್ರೀತಿಯಿಂದ ಅವನನ್ನು ದಿಟ್ಟಿಸುತ್ತ ಸುಮ್ಮನೆ ನಸುನಕ್ಕರು.

ಅಷ್ಟೊಂದು ಜನರ ಸಮ್ಮುಖದಲ್ಲೇ ಈ ಅದ್ಭುತ ನಡೆದುಹೋಯಿತು.

ಶ್ರೀರಾಮಕೃಷ್ಣರು ಇಂಥ ಅತೀಂದ್ರಿಯ ಶಕ್ತಿಯನ್ನು ವ್ಯಕ್ತಗೊಳಿಸುತ್ತಿದ್ದುದು ತೀರಾ ಅಪ ರೂಪ. ಅಷ್ಟೇ ಅಲ್ಲ, ಶಕ್ತಿಪ್ರದರ್ಶನ ಮಾಡುವ ಯೋಗಿ-ಯತಿಗಳನ್ನು ಆಷಾಢಭೂತಿಗಳೆಂದು ತಿಳಿದು ದೂರವಿರಬೇಕೆಂದು ಅವರು ಶಿಷ್ಯರಿಗೆ ಹೇಳುತ್ತಿದ್ದರು. ಆದರೆ ಇದೊಂದು ವಿಶೇಷ ಸಂದರ್ಭವೆಂದು ತಮ್ಮ ನಿಯಮವನ್ನು ಸಡಿಲಿಸಿದ್ದಿರಬಹುದು.

ಈಗ ಶ್ರೀರಾಮಕೃಷ್ಣರೆಂದರು: “ನರೇನ್, ಒಂದು ಹಾಡು ಹೇಳುತ್ತೀಯಾ?” ನರೇಂದ್ರ ತಂಬೂರಿ ಶ್ರುತಿ ಮಾಡಿಕೊಂಡು ಹಾಡಲಾರಂಭಿಸಿದ. ಭಕ್ತರೆಲ್ಲ ಆ ಮಧುರ ಗಾನಸುಧೆಯಲ್ಲಿ ಮುಳುಗಿಹೋದರು. ಶ್ರೀರಾಮಕೃಷ್ಣರಂತೂ ಒಮ್ಮೆಗೇ ಭಾವಸ್ಥರಾದರು. ಅವನ ಗಾಯನ ಮೂರು ತಾಸಿನ ಕಾಲ ಅವಿರತವಾಗಿ ಮುಂದುವರಿಯಿತು. ಆದರೆ ಅವನಲ್ಲಿ ದಣಿವಿನ ಸುಳಿವೇ ಇಲ್ಲದು ದನ್ನು ಕಂಡ ಭಕ್ತರು ಮತ್ತಷ್ಟು ಬೆರಗಾದರು. ಅನಂತರ ಸಾಮೂಹಿಕ ಸಂಕೀರ್ತನೆ ನರ್ತನಗಳು ನಡೆದು ಕಾರ್ಯಕ್ರಮ ಮುಕ್ತಾಯಗೊಂಡಿತು.

ನರೇಂದ್ರನಿಗೆ ಶ್ರೀರಾಮಕೃಷ್ಣರಲ್ಲಿನ ಅತಿದೊಡ್ಡ ಆಕರ್ಷಣೆಯೆಂದರೆ ಅವರ ಪರಿಪೂರ್ಣ ತ್ಯಾಗವೈರಾಗ್ಯ, ಅವರ ಸ್ಫಟಿಕಸದೃಶ ಪಾವಿತ್ರ್ಯ, ಅವರ ಭಗವತ್ಪ್ರೇಮ. ಶ್ರೀರಾಮಕೃಷ್ಣರನ್ನು ನರೇಂದ್ರನೆಡೆಗೆ ಸೆಳೆಯುತ್ತಿದ್ದ ಅತಿಮುಖ್ಯ ಅಂಶವೆಂದರೆ ಅವನ ಆತ್ಮಾವಲಂಬನ ಮನೋ ಭಾವ, ಪೌರುಷಪೂರ್ಣ ವ್ಯಕ್ತಿತ್ವ ಮತ್ತು ಸತ್ಯಸಾಕ್ಷಾತ್ಕಾರಕ್ಕಾಗಿ ಅವನ ಏಕನಿಷ್ಠೆಯ ಪ್ರಾಮಾಣಿಕ ಹಂಬಲ. ಶ್ರೀರಾಮಕೃಷ್ಣರು ನರೇಂದ್ರನ ಮೇಲಿಟ್ಟಿದ್ದ ವಿಶ್ವಾಸ ಅಪಾರವಾದದ್ದು, ಅದನ್ನು ಮಾತಿನಿಂದ ವರ್ಣಿಸಲು ಸಾಧ್ಯವಿಲ್ಲ. ಸಾಮಾನ್ಯ ಜನರು ಅವನ ವಿಶೇಷ ಗುಣಗಳನ್ನೆಲ್ಲ ತಪ್ಪಾಗಿ ಅರ್ಥಮಾಡಿಕೊಳ್ಳುತ್ತಿದ್ದುದೇ ಹೆಚ್ಚು. ಅವನು ಸ್ವಭಾವತಃ ಸ್ವಾವಲಂಬಿ, ಸ್ವತಂತ್ರ ಮನೋವೃತ್ತಿ ಯವನು. ಆದರೆ ಜನಗಳು ಅವನ ಈ ಸ್ವಾವಲಂಬಿ ಮನೋಭಾವವನ್ನು ಹುಚ್ಚುಸಾಹಸ ಎಂದು ಅರ್ಥೈಸಿದರು; ಅವನ ಪೌರುಷವನ್ನು ಹಟಮಾರಿತನ ಎಂದು ಅರ್ಥಮಾಡಿಕೊಂಡರು. ಅವನ ಸತ್ಯಶೋಧನೆಯ ಪ್ರಯತ್ನವೆನ್ನುವುದು ಅಪಕ್ವ ಬುದ್ಧಿಯಂತೆ ಕಾಣಿಸಿತು. ಇನ್ನು ಹೊಗಳಿಕೆ- ತೆಗಳಿಕೆಗಳ ವಿಷಯದಲ್ಲಂತೂ ನರೇಂದ್ರ ಸಂಪೂರ್ಣ ಉದಾಸೀನ. ಅವನ ಭಾವನೆ-ಮಾತು- ಕೃತಿಗಳಲ್ಲಿ ಸದಾ ಒಂದು ಅಸಾಮಾನ್ಯ ನಿರ್ಭಯತೆ ಎದ್ದು ಕಾಣುತ್ತಿತ್ತು. ಇದನ್ನು ಜನ ದುರಹಂಕಾರವೆಂದು ಭಾವಿಸುತ್ತಿದ್ದರು. ಹೀಗೆಲ್ಲ ಮಾತು ಹುಟ್ಟಿಕೊಳ್ಳಲು ಕಾರಣವಿಲ್ಲದಿರಲಿಲ್ಲ –ಏನೆಂದರೆ, ನರೇಂದ್ರ ಯಾರ ಮುಲಾಜನ್ನೂ ಇಟ್ಟುಕೊಳ್ಳುವವನಲ್ಲ. ಕಾರಣ, ಅವನಿಗೆ ಯಾರ ಹೊಗಳಿಕೆಯೂ ಬೇಕಿಲ್ಲ; ಇತರರಿಂದ ಮುಚ್ಚಿಡುವಂಥದು ಅವನಲ್ಲೇನೂ ಇಲ್ಲ. ಯಾವುದ ಕ್ಕಾದರೂ ಹೆದರಿಕೊಂಡು ಜೀವನ ನಡೆಸುವ ಅಭ್ಯಾಸ ಅವನದಲ್ಲ. ಹಿಂದೊಂದು ಮುಂದೊಂದು ಮಾಡುವವರನ್ನು ಕಂಡರೇ ಅವನಿಗೆ ಆಗುವುದಿಲ್ಲ. ಇಂಥವನನ್ನು ಎಷ್ಟು ಜನ ಸರಿಯಾಗಿ ಅರ್ಥಮಾಡಿಕೊಂಡಾರು?

ಆದರೆ ರತ್ನಕ್ಕೆ ಬೆಲೆಕಟ್ಟುವವನು ವಜ್ರದ ವ್ಯಾಪಾರಿಯಲ್ಲದೆ ತರಕಾರಿಯವನಲ್ಲವಲ್ಲ! ನರೇಂದ್ರನ ಯೋಗ್ಯತೆಯನ್ನರಿತಿದ್ದ ಶ್ರೀರಾಮಕೃಷ್ಣರು ಅವನನ್ನು ಅಪರಂಜಿ ಎನ್ನುತ್ತಿದ್ದರು. ಆದರೆ ಅವನ ವ್ಯಕ್ತಿತ್ವವಿನ್ನೂ ವಿಕಸನಗೊಳ್ಳಲಿಕ್ಕಿದೆ; ಮುಂದೆ ಅದು ಸಹಸ್ರದಳ ಕಮಲದಂತೆ ಅರಳಿ, ತನ್ನ ಅನುಪಮ ಪರಿಮಳವನ್ನು ಬೀರಲಿಕ್ಕಿದೆ ಎಂಬುದು ಅವರಿಗೆ ತಿಳಿದಿತ್ತು. ಲೋಕದ ಕಠೋರ ವಾಸ್ತವಿಕತೆಯ ಬಿಸಿ ಅವನಿಗಿನ್ನೂ ತಗಲಬೇಕಾಗಿದೆ; ಮಾನವನ ದಾರುಣ ದುಃಖ ಸಂಕಟಗಳ ಪರಿಚಯವಾಗಬೇಕಾಗಿದೆ. ಹಾಗಾದಾಗ ಅವನ ವ್ಯಕ್ತಿತ್ವ ಪರಿಪಾಕಗೊಂಡು ಬೇರೆಯೇ ರೂಪ ತಾಳಲಿದೆ. ಅವನ ಈಗಿನ ನಿರ್ಲಕ್ಷ್ಯ ನಿಷ್ಠುರತೆಗಳು ಕರಗಿ ಅಪಾರ ಅನುಕಂಪೆಯ ಸ್ರೋತವಾಗಲಿವೆ; ಆತನ ಅಮಿತ ಆತ್ಮವಿಶ್ವಾಸವು ಜೀವರೆದೆಗೆ ಹೊಸ ಭರವಸೆಯನ್ನೂ ಚೈತನ್ಯ ವನ್ನೂ ತುಂಬಬಲ್ಲ ಜೀವಸತ್ವದಂತೆ ಕೆಲಸ ಮಾಡಲಿದೆ; ಇಂದು ನಿರಂಕುಶ-ಸ್ವಚ್ಛಂದವೆಂಬಂತೆ ತೋರುತ್ತಿರುವ ನಡವಳಿಕೆಯೇ ಮುಂದೆ ಅಪೂರ್ವ ಸಂಯಮಪೂರ್ಣವಾದ ಆತ್ಮ ವಿಶ್ವಾಸವಾಗಿ ವಿಜೃಂಭಿಸಲಿದೆ ಎಂದು. ಭವಿಷ್ಯದ ಈ ಭವ್ಯಚಿತ್ರವನ್ನು ಶ್ರೀರಾಮಕೃಷ್ಣರು ನಿಚ್ಚಳವಾಗಿ ಕಾಣುತ್ತಿದ್ದರು. ಆದ್ದರಿಂದ ಅವನ ಸ್ವತಂತ್ರ ಮನೋವೃತ್ತಿಗೆ ಅವರು ಯಾವುದೇ ಬಗೆಯ ಕಡಿವಾಣ ತೊಡಿಸಲಿಲ್ಲ, ನಿರ್ಬಂಧ ಹೇರಲಿಲ್ಲ. ಅಷ್ಟೇ ಅಲ್ಲ, ಅವನ ಈ ಸದ್ಗುಣಗಳನ್ನು ಬಣ್ಣಿಸಿ ಕೊಂಡಾಡುವ ಒಂದು ಅವಕಾಶವನ್ನೂ ಬಿಡುತ್ತಿರಲಿಲ್ಲ.

ನರೇಂದ್ರನನ್ನು ಹೊಗಳಬೇಕೆಂದರೆ ಶ್ರೀರಾಮಕೃಷ್ಣರಿಗೆ ಒಂದು ನಾಲಿಗೆ ಸಾಲದು–ಅವನು ಎದುರಿಗಿದ್ದರೂ ಸರಿಯೆ, ಇಲ್ಲದಿದ್ದರೂ ಸರಿಯೆ. ಮುಖಸ್ತುತಿಯ ಕೆಡುಕನ್ನು ಅರಿಯದವ ರೇನಲ್ಲ ಅವರು. ಯಾವುದೇ ವ್ಯಕ್ತಿಯ–ವಿಶೇಷತಃ ಸಾಧಕನ–ಅವನತಿಗೆ ಈ ಹೊಗಳಿಕೆಯೇ ಜಾರುಬಂಡೆಯಾಗಿ ಪರಿಣಮಿಸಬಲ್ಲುದು. ಆದರೆ ನರೇಂದ್ರನ ವಿಷಯದಲ್ಲಿ ಮಾತ್ರ ಅವರು ಈ ಎಚ್ಚರಿಕೆಯನ್ನು ಗಾಳಿಗೆ ತೂರಿದ್ದರು. ಏಕೆಂದರೆ, ಸ್ತುತಿನಿಂದೆಗಳಾವುವೂ ತಲುಪಲಾರದಷ್ಟು ಎತ್ತರ ಅವನ ಬುದ್ಧಿಯ ಆವಾಸಸ್ಥಾನವೆಂಬುದು ಅವರಿಗೆ ನಿಚ್ಚಳವಾಗಿತ್ತು.

ಶ್ರೀರಾಮಕೃಷ್ಣರು ತಮ್ಮ ತರುಣಶಿಷ್ಯರ ಪೈಕಿ ನರೇಂದ್ರ, ರಾಖಾಲ, ಬಾಬುರಾಮ, ಯೋಗೀಂದ್ರ, ನಿರಂಜನ ಮತ್ತು ಪೂರ್ಣಚಂದ್ರ ಇವರುಗಳನ್ನು ಈಶ್ವರಕೋಟಿಗೆ ಸೇರಿದವರು ಎಂದು ಹೇಳುತ್ತಿದ್ದರು. ಈಶ್ವರಕೋಟಿಗಳು ಎಂದರೆ ನಿತ್ಯಸಿದ್ಧರು, ನಿತ್ಯಮುಕ್ತರು ಎಂದರ್ಥ. ಇವರೆಲ್ಲ ಆಧ್ಯಾತ್ಮಿಕ ಸಾಧನೆ ಮಾಡಬೇಕಾಗಿಯೇ ಇಲ್ಲ. ಅವತಾರಪುರುಷರ ಆಗಮನದೊಂದಿಗೆ ಅವರ ಲೀಲಾಸಹಚರರಾಗಿ ಬರುವವರು ಇವರು. ಶ್ರೀರಾಮಕೃಷ್ಣರು ನರೇಂದ್ರಾದಿಗಳ ಕುರಿತಾಗಿ ಕೆಲವೊಮ್ಮೆ ಹೇಳುತ್ತಿದ್ದರು: “ಇವರು ಇಲ್ಲಿ ಏನೇನು ಶಿಕ್ಷಣ ಪಡೆಯುತ್ತಿದ್ದಾರೋ, ಸಾಧನೆ ಮಾಡುತ್ತಿದ್ದಾರೋ ಅದೆಲ್ಲ ಅವರಿಗಾಗಿ ಅಲ್ಲವೇ ಅಲ್ಲ; ಅವರಿಗದರ ಆವಶ್ಯಕತೆಯೇ ಇಲ್ಲ. ಅವರ ಸಾಧನೆಯೆಲ್ಲ ಜಗತ್ತಿನ ಹಿತಕ್ಕಾಗಿ.” ನರೇಂದ್ರನನ್ನಂತೂ ಶ್ರೀರಾಮಕೃಷ್ಣರು ಅತ್ಯಂತ ಉನ್ನತ ಭಾವನೆಯಿಂದ ನೋಡುತ್ತಿದ್ದರು. ಅವನ ಕುರಿತಾಗಿ ಯಾರಾದರೂ ಟೀಕಿಸಿ ಮಾತನಾಡಿ ದರೆ, “ಏನು ಮಾತಾಡುತ್ತಿದ್ದೀಯೆ ನೀನು! ಶಿವನಿಂದೆ! ಶಿವನಿಂದೆ ಮಾಡುತ್ತಿದ್ದೀ!” ಎಂದು ಗದರಿಸುತ್ತಿದ್ದರು. ಒಮ್ಮೆ ಒಬ್ಬ ಭಕ್ತ ಬಂದು, “ಮಹಾಶಯರೆ, ನರೇಂದ್ರ ಈಚೀಚೆಗೆ ಕೆಲವು ಅಯೋಗ್ಯರ ಸಹವಾಸಕ್ಕೆ ಬಿದ್ದು ದಾರಿತಪ್ಪಿ ನಡೆಯುತ್ತಿದ್ದಾನೆ” ಎಂದು ಚಾಡಿ ಹೇಳಿದ. ಒಡನೆಯೇ ಶ್ರೀರಾಮಕೃಷ್ಣರು, “ನೋಡು, ಹಾಗಾಗುವುದಕ್ಕೆ ಸಾಧ್ಯವೇ ಇಲ್ಲ! ನರೇಂದ್ರ ಖಂಡಿತ ದಾರಿತಪ್ಪಿ ನಡೆಯಲಾರ ಎನ್ನುವುದನ್ನು ಜಗನ್ಮಾತೆಯೇ ನನಗೆ ತೋರಿಸಿಕೊಟ್ಟಿದ್ದಾಳೆ. ನೋಡಿಕೊ, ಇನ್ನೊಂದು ಸಲ ಅವನ ಮೇಲೇನಾದರೂ ಹೇಳಿದೆಯೊ, ನಾನು ಇನ್ನೆಂದಿಗೂ ನಿನ್ನ ಮುಖವನ್ನೇ ನೋಡುವುದಿಲ್ಲ!” ಎಂದು ರೇಗಿಬಿಟ್ಟರು. ಹೀಗೆ ನರೇಂದ್ರನ ಮೇಲೆ ಅಪವಾದ ಹೊರಿಸುವವರ ಬಾಯಿಮುಚ್ಚಿಸಿಬಿಡುತ್ತಿದ್ದರು. ಜೊತೆಗೆ, ಅವಕಾಶ ಸಿಕ್ಕಾಗಲೆಲ್ಲ ಭಕ್ತರ ಇದಿರಲ್ಲಿ ಅವನನ್ನು ಮನಸಾರೆ ಕೊಂಡಾಡುತ್ತಿದ್ದರು. ಆತನ ಸದ್ಗುಣಗಳನ್ನು ವರ್ಣಿಸುವಲ್ಲಿ ಅವರಿಗೆ ವಿಶೇಷ ಉತ್ಸಾಹ. ಸಾಮಾನ್ಯರಾದ ಅಲ್ಪಬುದ್ಧಿಯವರೇನಾದರೂ ಇಂತಹ ಮೆಚ್ಚುಗೆಯ ಮಾತುಗಳನ್ನು ಕೇಳಿದ್ದರೆ ಅವರ ತಲೆ ತಿರುಗಿಹೋಗುತ್ತಿತ್ತು. ಇದು ಶ್ರೀರಾಮಕೃಷ್ಣರಿಗೂ ಚೆನ್ನಾಗಿ ಗೊತ್ತಿತ್ತು. ಆದರೆ ನರೇಂದ್ರ ಆ ತರಹದ ಅಲ್ಪಬುದ್ಧಿಯವನಲ್ಲ ಎಂಬುದು ತಿಳಿದದ್ದರಿಂದಲೇ ಅವನನ್ನು ಹಾಗೆ ಎಲ್ಲರೆದುರು ಮುಕ್ತಕಂಠದಿಂದ ಹೊಗಳುತ್ತಿದ್ದರು.

ಒಂದು ದಿನ ಕೇಶವಚಂದ್ರ ಸೇನ, ವಿಜಯಕೃಷ್ಣ ಗೋಸ್ವಾಮಿಯೇ ಮೊದಲಾದ ಬ್ರಾಹ್ಮ ಸಮಾಜದ ಅನೇಕ ಪ್ರಖ್ಯಾತ ಮುಂದಾಳುಗಳು ಶ್ರೀರಾಮಕೃಷ್ಣರೆದುರು ಆಸೀನರಾಗಿದ್ದಾರೆ. ತರುಣ ನರೇಂದ್ರ ಕೂಡ ಅಲ್ಲಿದ್ದಾನೆ. ಮಾತಿನ ಮಧ್ಯೆ ಶ್ರೀರಾಮಕೃಷ್ಣರು ಭಾವಸ್ಥರಾದರು. ಹಾಗೆಯೇ ಎದುರು ಕುಳಿತಿದ್ದವರತ್ತ ದೃಷ್ಟಿ ಹಾಯಿಸಿದರು. ಬಳಿಕ ಅವರ ದೃಷ್ಟಿ ನರೇಂದ್ರನ ಮೇಲೆ ಬಿತ್ತು. ಒಡನೆಯೇ ಆತನ ಭವಿಷ್ಯದ ಘನವ್ಯಕ್ತಿತ್ವದ ಚಿತ್ರ ಅವರ ಕಂಗಳಿಗೆ ಗೋಚರ ವಾಯಿತು. ಅವನು ಮುಂದೆ ಎಂತಹ ಅದ್ಭುತಗಳನ್ನು ಎಸಗಲಿದ್ದಾನೆ ಎಂಬುದನ್ನು ಸ್ಪಷ್ಟವಾಗಿ ಕಂಡು ಅವರ ಹೃದಯದಲ್ಲಿ ಭಾವ ತುಂಬಿ ಬಂತು. ಆದರೆ ತಮ್ಮ ಮನಸ್ಸಿಗೆ ಬಂದದ್ದನ್ನು ಕೂಡಲೇ ಹೊರಹಾಕಲಿಲ್ಲ. ಸ್ವಲ್ಪ ಹೊತ್ತಿಗೆ ಬ್ರಾಹ್ಮಸದಸ್ಯರೆಲ್ಲ ಪ್ರಣಾಮ ಸಲ್ಲಿಸಿ ನಿರ್ಗಮಿಸಿ ದರು. ಆಗ ಉಳಿದ ಭಕ್ತರ ಹತ್ತಿರ ಶ್ರೀರಾಮಕೃಷ್ಣರು ಹೇಳುತ್ತಾರೆ:

“ಕೇಶವನಲ್ಲಿ ಶಕ್ತಿಯ ಒಂದು ಅಂಶ ಸಮೃದ್ಧವಾಗಿದೆ; ಅಷ್ಟರಿಂದಲೇ ಅವನು ಜಗತ್ಪ್ರಸಿದ್ಧ ನಾಗಿಬಿಟ್ಟಿದ್ದಾನೆ. ಆದರೆ ನರೇಂದ್ರನಲ್ಲಿ ಅಂತಹ ಹದಿನೆಂಟು ಅಂಶಗಳು ಪರಿಪೂರ್ಣವಾಗಿರುವು ದನ್ನು ನಾನೀಗ ಕಂಡೆ! ಕೇಶವ-ವಿಜಯಕೃಷ್ಣರಲ್ಲಿ ಜ್ಞಾನವು ಹಣತೆಯ ಹಾಗೆ ಉರಿಯುತ್ತಿದ್ದರೆ, ನರೇಂದ್ರನಲ್ಲಿ ಸ್ವಯಂ ಜ್ಞಾನಸೂರ್ಯನೇ ಬೆಳಗುತ್ತಿದ್ದಾನೆ; ಅದರ ಪ್ರಖರ ಪ್ರಕಾಶವು ಅಜ್ಞಾನ- ಮೋಹಗಳನ್ನು ಸುಟ್ಟುಹಾಕುತ್ತಿದೆ!”

ಬೇರಾರಾದರೂ ಇಂತಹ ಮೆಚ್ಚುಗೆಯ ಮಾತುಗಳನ್ನು ಕೇಳಿದ್ದರೆ ಅದೆಷ್ಟು ಉಬ್ಬಿಹೋಗಿರು ತ್ತಿದ್ದರೊ! ಆದರೆ ನರೇಂದ್ರನಿಗೆ ಈ ಮಾತು ಇಷ್ಟವೇ ಆಗಲಿಲ್ಲ. ಅವನು ಅಸಮಾಧಾನಗೊಂಡು ನುಡಿದ:

“ಮಹಾಶಯರೆ, ಇದೇನೆನ್ನುತ್ತಿದ್ದೀರಿ ನೀವು? ಇದನ್ನು ಕೇಳಿದರೆ ಜನ ನಿಮ್ಮನ್ನು ಹುಚ್ಚ ಎಂದಾರು, ಅಷ್ಟೆ. ಅಂತಹ ಜಗತ್ಪ್ರಸಿದ್ಧರಾದ ಕೇಶವಚಂದ್ರಸೇನರ, ಸಂತ ವಿಜಯಕೃಷ್ಣರ ಜೊತೆಯಲ್ಲಿ ನನ್ನಂತಹ ಸರ್ವೇಸಾಮಾನ್ಯನಾದ ಒಬ್ಬ ವಿದ್ಯಾರ್ಥಿಯನ್ನು ಹೋಲಿಸುತ್ತಿದ್ದೀರಲ್ಲ! ದಯವಿಟ್ಟು ಇನ್ನೆಂದಿಗೂ ಹೀಗೆಲ್ಲ ಮಾಡಬೇಡಿ.”

ನರೇಂದ್ರನ ಈ ವಿನಯವನ್ನು ಕಂಡು ಶ್ರೀರಾಮಕೃಷ್ಣರಿಗೆ ಇನ್ನಷ್ಟು ಸಂತೋಷವೇ ಆಯಿತು. ಅವರು ಮುಗುಳ್ನಗುತ್ತ ಉತ್ತರಿಸಿದರು:

“ಮಗು, ನಾನು ತಾನೆ ಏನು ಮಾಡಲಿ? ನಾನಾಗಿಯೇ ಹಾಗೆ ಹೇಳಿದೆ ಅಂತ ತಿಳಿದೆಯಾ? ಜಗನ್ಮಾತೆಯೇ ನನಗೆ ಹಾಗೆಂದು ತೋರಿಸಿಕೊಟ್ಟಳು. ಮತ್ತು ಆಕೆ ಎಂದಿಗೂ ಸುಳ್ಳುಪಳ್ಳನ್ನು ತೋರಿಸಿಲ್ಲ. ಅದಕ್ಕೆ ನಾನು ಹಾಗೆಂದೆ.”

ಆದರೆ ನರೇಂದ್ರ ಇದನ್ನೆಲ್ಲ ನಂಬುವವನಲ್ಲ. ಅವನು ಎಲ್ಲರೆದುರಿಗೇ ಕೇಳುತ್ತಾನೆ: “ಅವೆಲ್ಲ ಜಗನ್ಮಾತೆಯೇ ನಿಮಗೆ ತೋರಿಸಿಕೊಟ್ಟದ್ದೋ ಅಥವಾ ನಿಮ್ಮ ಮೆದುಳಿನ ಹುಚ್ಚುಕಲ್ಪನೆಯೋ ಯಾರಿಗೆ ಗೊತ್ತು! ನನಗೇನಾದರೂ ಅಂತಹ ದರ್ಶನ-ಗಿರ್ಶನಗಳಾಗಿದ್ದಿದ್ದರೆ ಅವನ್ನೆಲ್ಲ ಭ್ರಮೆ ಅಂತ ಬದಿಗೊತ್ತಿಬಿಡುತ್ತಿದ್ದೆ. ಪಾಶ್ಚಾತ್ಯ ವಿಜ್ಞಾನ ಸ್ಪಷ್ಟವಾಗಿ ಹೇಳುತ್ತಿದೆ, ಏನೆಂದರೆ ನಮ್ಮ ನಮ್ಮ ಮನಸ್ಸು-ಇಂದ್ರಿಯಗಳಿಂದಲೇ ನಾವು ಎಷ್ಟೋಸಲ ಮೋಸಹೋಗುತ್ತೇವೆ; ಅದರಲ್ಲೂ ನಮ್ಮ ಪೂರ್ವಗ್ರಹ ಪೀಡಿತವಾದ ವೈಯಕ್ತಿಕ ಅನಿಸಿಕೆಗಳಿಂದಾಗಿ ನಾವು ಭ್ರಮೆಗೀಡಾಗುವು ದಂತೂ ಇನ್ನೂ ಹೆಚ್ಚು ಅಂತ. ನೀವು ನನ್ನನ್ನು ಅಷ್ಟೊಂದಾಗಿ ಪ್ರೀತಿಸುತ್ತೀರಿ. ಎಲ್ಲ ವಿಷಯಗಳಲ್ಲೂ ನಾನೊಬ್ಬ ದೊಡ್ಡಮನುಷ್ಯ ಎನ್ನುವಂತೆ ಕಾಣಲು ಇಷ್ಟಪಡುತ್ತೀರಿ. ಅದಕ್ಕೆ ನನ್ನ ಕುರಿತಾಗಿ ನಿಮ್ಮ ಮೆದುಳಿನಲ್ಲಿ ಅಂಥ ಕಲ್ಪನೆಗಳೆಲ್ಲ ಹುಟ್ಟಿಕೊಳ್ಳುತ್ತವೆ.”

ಹೀಗೆ ಅವನು ಇಂತಹ ಸಂದರ್ಭಗಳಲ್ಲೆಲ್ಲ ಶ್ರೀರಾಮಕೃಷ್ಣರ ಮಾತುಗಳನ್ನು ನೇರವಾಗಿ, ಯುಕ್ತಿಯುಕ್ತವಾಗಿ ಪ್ರತಿಭಟಿಸುತ್ತಿದ್ದ. ಅವನು ಹೀಗೆ ತಮ್ಮನ್ನು ಖಂಡಿಸುವಾಗ ಶ್ರೀರಾಮಕೃಷ್ಣ ರೇನಾದರೂ ಭಾವಸ್ಥಿತಿಯಲ್ಲಿದ್ದರೆ ಅವನ್ನೆಲ್ಲ ಮನಸ್ಸಿಗೆ ಹಚ್ಚಿಕೊಳ್ಳುತ್ತಿರಲಿಲ್ಲ; ಸುಮ್ಮನೆ ಮುಗುಳ್ನಗುತ್ತ ಆಲಿಸುತ್ತಿದ್ದರು. ಆದರೆ ಅವರು ಸಹಜ ಸ್ಥಿತಿಯಲ್ಲಿರುವಾಗ ಅವನು ಹೀಗೆಲ್ಲ ಹೇಳಿದರೆ, ಅವರ ಮನಸ್ಸಿಗೆ ಬಹಳ ಗಲಿಬಿಲಿಯುಂಟಾಗಿಬಿಡುತ್ತಿತ್ತು. ಏಕೆಂದರೆ, ಮೇಲ್ನೋಟಕ್ಕೆ ಅವನ ಮಾತು ತರ್ಕಬದ್ಧವಾಗಿರುತ್ತಿತ್ತು. ಅಲ್ಲದೆ ಅವನನ್ನು ಶ್ರೀರಾಮಕೃಷ್ಣರು ನಾರಾಯಣ ಸ್ವರೂಪಿಯೆಂದು ತಿಳಿದಿದ್ದರು. ‘ಇಂಥವನಾಡುವ ಮಾತು ಸುಳ್ಳಾಗಲು ಸಾಧ್ಯವಿಲ್ಲವಲ್ಲ’– ಎಂಬ ಶಂಕೆ ಅವರಲ್ಲೆದ್ದು ತಮ್ಮ ದರ್ಶನದ ಬಗ್ಗೆಯೇ ಅನುಮಾನ ತಾಳಿಬಿಡುತ್ತಿದ್ದರು; ಮತ್ತು ಜಗನ್ಮಾತೆಯ ಬಳಿಗೋಡಿ, “ಅಮ್ಮಾ, ನರೇಂದ್ರ ಹೀಗೆಲ್ಲ ಹೇಳುತ್ತಾನಲ್ಲ, ಏನು ಮಾಡಲಿ?” ಎಂದು ಮೊರೆಯಿಡುತ್ತಿದ್ದರು. ಶ್ರೀರಾಮಕೃಷ್ಣರ ಪಾಲಿಗೆ ಆಕೆ ಸ್ಪಷ್ಟವಾಗಿ ಮಾತನಾಡುವ ಜೀವಂತ ದೇವಿಯಲ್ಲವೆ? ಅವಳೆನ್ನುತ್ತಿದ್ದಳು, “ಅವನು ಹೇಳಿದ್ದನ್ನೆಲ್ಲ ಏಕೆ ತಲೆಗೆ ಹಚ್ಚಿ ಕೊಳ್ಳುತ್ತಿ? ಇನ್ನು ಕೆಲವು ದಿನ, ಅಷ್ಟೆ. ಆಮೇಲೆ ಅವನು ನೀನು ಹೇಳಿದ ಪ್ರತಿಯೊಂದು ಮಾತನ್ನೂ ಒಪ್ಪಿಕೊಳ್ಳುತ್ತಾನೆ, ನೋಡುತ್ತಿರು.” ಆಗಲೇ ಅವರಿಗೆ ಸಮಾಧಾನ.

ಒಮ್ಮೆ ನರೇಂದ್ರ ಹಲವಾರು ದಿನಗಳವರೆಗೂ ದಕ್ಷಿಣೇಶ್ವರಕ್ಕೆ ಹೋಗಲೇ ಇಲ್ಲ. ಅವನನ್ನು ಕಾಣದೆ ಶ್ರೀರಾಮಕೃಷ್ಣರ ಜೀವ ತಲ್ಲಣಿಸಿಹೋಯಿತು. ಭಕ್ತರ ಮೂಲಕ ಹೇಳಿಕಳಿಸಿದರೂ ಅದೇಕೋ ಆತ ಬರಲೇ ಇಲ್ಲ. ಇನ್ನು ಶ್ರೀರಾಮಕೃಷ್ಣರಿಂದ ಸುಮ್ಮನಿರಲು ಸಾಧ್ಯವಾಗಲಿಲ್ಲ. ತಾವೇ ಹೋಗಿ ನೋಡಿಕೊಂಡು ಬರಲು ಮನಸ್ಸು ಮಾಡಿದರು. ಅಂದು ಭಾನುವಾರ. ಸಂಜೆಯ ಹೊತ್ತಿಗೆ ಅವನು ತಪ್ಪದೆ ‘ಸಾಧಾರಣ ಬ್ರಾಹ್ಮಸಮಾಜ’ದ ಪ್ರಾರ್ಥನಾಗೋಷ್ಠಿಗೆ ಹೋಗುತ್ತಾನೆ ಎನ್ನುವುದು ಅವರಿಗೆ ತಿಳಿದಿತ್ತು. ಆದ್ದರಿಂದ ತಾವು ಅಲ್ಲಿಗೇ ಹೋಗಿಬಿಡುವುದೆಂದು ಆಲೋಚಿಸಿ, ಒಂದಿಬ್ಬರು ಭಕ್ತರೊಂದಿಗೆ ನೇರವಾಗಿ ಅತ್ತ ಹೊರಟರು. ತಾವು ಹೀಗೆ ಕರೆಯಿಸಿಕೊಳ್ಳದೆ ಅಲ್ಲಿಗೆ ಹೋದರೆ ಯಾರೇನು ಭಾವಿಸಿಯಾರು ಎಂಬ ಶಂಕೆ ಅವರಿಗೆ ಬರಲೇ ಇಲ್ಲ. ಏಕೆಂದರೆ ಅವರು ಅದಾಗಲೇ ಹಲವಾರು ಬಾರಿ ಬ್ರಾಹ್ಮಸಮಾಜದ ಸಭೆಗಳಿಗೆ ಹೋದದ್ದುಂಟು. ಅಲ್ಲದೆ, ಬ್ರಾಹ್ಮ ಸಮಾಜದ ಎರಡು ಪಂಗಡಗಳ ಮುಖಂಡರಿಗೂ ಅವರು ಬೇಕಾದವರೇ. ಆದ್ದರಿಂದ ಆ ಬಗ್ಗೆ ಚಿಂತಿಸದೆ, ಬ್ರಾಹ್ಮಸಮಾಜಕ್ಕೆ ಬಂದು ತಲುಪಿದರು. ಸಭೆ ನಡೆಯುತ್ತಿತ್ತು. ಭಜನೆಯ ತಂಡದ ಇತರರೊಂದಿಗೆ ನರೇಂದ್ರ ವೇದಿಕೆಯ ಮೇಲೆ ಕುಳಿತಿದ್ದ. ಮುಖಂಡರೊಬ್ಬರು ಪ್ರವಚನ ನೀಡುತ್ತಿದ್ದರು. ಶ್ರೀರಾಮಕೃಷ್ಣರು ನರೇಂದ್ರನ ಕುರಿತಾಗಿಯೇ ಆಲೋಚಿಸುತ್ತ ಅರ್ಧಬಾಹ್ಯಾ ವಸ್ಥೆಯಲ್ಲಿದ್ದುದರಿಂದ, ಬೇರೇನನ್ನೂ ಗಮನಿಸದೆ ಸೀದಾ ಒಳಗೆ ಪ್ರವೇಶಮಾಡಿ ಮುಂದಕ್ಕೆ ಬಂದರು. ಸಭಿಕರೆಲ್ಲ ಅಚ್ಚರಿ-ಕುತೂಹಲಗಳಿಂದ ಅವರನ್ನೇ ದಿಟ್ಟಿಸಲಾರಂಭಿಸಿದರು. ಗುಜು ಗುಜು ಪ್ರಾರಂಭವಾಯಿತು. ಬ್ರಾಹ್ಮ ಧುರೀಣರೆಲ್ಲ ಶ್ರೀರಾಮಕೃಷ್ಣರನ್ನು ಅತ್ಯಂತ ಪೂಜ್ಯದೃಷ್ಟಿ ಯಿಂದ ನೋಡುತ್ತಿದ್ದರು. ಆದರೆ ಕೇಶವಸೇನ-ವಿಜಯಕೃಷ್ಣರಲ್ಲಿ ಭಿನ್ನಾಭಿಪ್ರಾಯವುಂಟಾಗಿ ಸಮಾಜವು ಒಡೆಯಲು ಶ್ರೀರಾಮಕೃಷ್ಣರೇ ಕಾರಣ ಎಂದು ಇತರ ಅನೇಕ ಸದಸ್ಯರು ತಪ್ಪುಕಲ್ಪನೆ ಹೊಂದಿದ್ದರು. ಆದ್ದರಿಂದ, ಅಂದು ಅಲ್ಲಿದ್ದ ಹಿರಿಯರು ಯಾರೂ ಅವರನ್ನು ಆದರಿಸುವ ಗೋಜಿಗೆ ಹೋಗದೆ ಸುಮ್ಮನಿದ್ದುಬಿಟ್ಟರು. ಆದರೆ ಶ್ರೀರಾಮಕೃಷ್ಣರು ಮಾತ್ರ ಅತ್ತಿತ್ತ ನೋಡದೆ, ಬೇರೇನನ್ನೂ ಚಿಂತಿಸದೆ ವೇದಿಕೆಯ ಬಳಿಗೇ ನಡೆದರು. ನರೇಂದ್ರನನ್ನು ಕಂಡು “ನರೇನ್, ನರೇನ್!” ಎನ್ನುತ್ತ ಪೂರ್ಣ ಸಮಾಧಿಸ್ಥರಾಗಿ ನಿಂತುಬಿಟ್ಟರು! ಇದು ಸಭಿಕರಿಗೆ ಇನ್ನಷ್ಟು ಕುತೂಹಲಕ್ಕೆ ಕಾರಣವಾಯಿತು. ಹಿಂದಿನ ಸಾಲುಗಳಲ್ಲಿ ಕುಳಿತಿದ್ದ ಅನೇಕರು ಅವರನ್ನು ಸರಿಯಾಗಿ ನೋಡುವ ಉದ್ದೇಶದಿಂದ ಬೆಂಚುಗಳ ಮೇಲೆ ಹತ್ತಿನಿಂತರು. ಈ ಗಲಭೆಯನ್ನೆಲ್ಲ ನಿಲ್ಲಿಸಲು ಕೆಲವರು ದೀಪಗಳನ್ನು ಆರಿಸಿಬಿಟ್ಟರು. ಇದರಿಂದ ಗಲಭೆ ನಿಲ್ಲುವುದರ ಬದಲಿಗೆ ಇನ್ನಷ್ಟು ಹೆಚ್ಚಿತ್ತು! ಒಳಗೆ ಕುಳಿತಿದ್ದವರೆಲ್ಲ ಬಾಗಿಲ ಕಡೆಗೆ ಓಡಿದರು. ಹೀಗೆ ನಿಮಿಷಾರ್ಧದಲ್ಲಿ ದೊಡ್ಡ ಕೋಲಾಹಲವುಂಟಾಯಿತು. ಇದನ್ನೆಲ್ಲ ನೋಡುತ್ತ ನರೇಂದ್ರ ಖೇದಗೊಂಡ. ಶ್ರೀರಾಮಕೃಷ್ಣರು ಈಗ ಇಲ್ಲಿಗೇಕೆ ಬಂದಿದ್ದಾರೆಂಬುದು ಅವನಿಗೆ ತಿಳಿದುಹೋಯಿತು. ಕೂಡಲೇ ಅವರ ನೆರವಿಗೆ ಧಾವಿಸಿದ. ಅವರ ಕೈಹಿಡಿದುಕೊಂಡು, ಮೆಲ್ಲಗೆ ಹಿಂಬಾಗಿಲ ಮೂಲಕ ಹೊರಗೆ ಕರೆತಂದ. ಬಳಿಕ ಒಂದು ಸಾರೋಟಿನಲ್ಲಿ ಅವರೊಂದಿಗೆ ದಕ್ಷಿಣೇಶ್ವರಕ್ಕೆ ಬಂದ.

ಬ್ರಾಹ್ಮಸಮಾಜೀಯರಿಂದ ಶ್ರೀರಾಮಕೃಷ್ಣರಿಗಾದ ಅವಮರ್ಯಾದೆಯನ್ನು ಕಂಡು ನರೇಂದ್ರನ ಮೈಯುರಿಯುತ್ತಿದೆ. ಒಬ್ಬ ಆಗಂತುಕನಿಗೆ ಕೊಡುವ ಮರ್ಯಾದೆಯನ್ನಾದರೂ ಕೊಡಬೇಡವೆ? ಅಲ್ಲದೆ ಇದಕ್ಕೆಲ್ಲ ಪರೋಕ್ಷವಾಗಿ ತಾನೇ ಕಾರಣನೆಂಬ ದುಗುಡ ಬೇರೆ. “ಇಂಥ ಜಾಗಕ್ಕೆ ನೀವೇಕೆ ಬಂದಿರಿ? ನನ್ನನ್ನು ಹುಡುಕಿಕೊಂಡು ಯಾಕೆ ಬರಬೇಕಾಗಿತ್ತು?” ಎಂದು ಶ್ರೀರಾಮಕೃಷ್ಣರನ್ನು ಚೆನ್ನಾಗಿ ದಬಾಯಿಸಿದ. ಆದರೆ ಅವರಿಗೆ ಮಾತ್ರ, ತಮಗೆ ಅಪಮಾನ ವಾಯಿತೆಂಬ ಭಾವನೆಯೂ ಇಲ್ಲ; ನರೇಂದ್ರನ ಛೀಮಾರಿಯಿಂದ ಬೇಸರವೂ ಇಲ್ಲ. ಅವರಿಗೆ ಬೇಕಾಗಿದ್ದುದೊಂದೇ–ನರೇಂದ್ರ. ಅವನು ಸಿಕ್ಕಿಬಿಟ್ಟಿದ್ದಾನೆ! ತಮ್ಮ ಸನಿಹದಲ್ಲೇ ಇದ್ದಾನೆ! ಇನ್ನಾವುದನ್ನು ಕಟ್ಟಿಕೊಂಡು ಏನಾಗಬೇಕು? ಆನಂದದಲ್ಲಿ ಮುಳುಗಿ ತಾವೇತಾವಾಗಿದ್ದಾರೆ.

ಶ್ರೀರಾಮಕೃಷ್ಣರು ತನ್ನ ಮೇಲಿನ ಪ್ರೀತಿಯಿಂದಾಗಿ, ತಮ್ಮ ಸ್ವಂತ ಆತ್ಮಗೌರವದ ಕಡೆಗೂ ಗಮನ ಕೊಡುತ್ತಿಲ್ಲವೆಂದು ಕಂಡಾಗ ನರೇಂದ್ರ ಅವರಿಗೊಂದು ಎಚ್ಚರಿಕೆಯ ಮಾತು ಹೇಳಿದ:

“ಪುರಾಣದಲ್ಲಿ ಬರುವ ರಾಜಾ ಭರತನ ಕಥೆಯನ್ನು ಜ್ಞಾಪಿಸಿಕೊಳ್ಳಿ. ಅವನು ತಪಸ್ಸಿನಲ್ಲಿ ತೊಡಗಿದ್ದಾಗಲೂ ತಾನು ಸಾಕಿದ್ದ ಜಿಂಕೆಯ ಬಗೆಗೇ ಯೋಚಿಸಿ ಯೋಚಿಸಿ ಸತ್ತದ್ದರಿಂದ ಮುಂದಿನ ಜನ್ಮದಲ್ಲಿ ಜಿಂಕೆಯಾಗಿಯೇ ಹುಟ್ಟಬೇಕಾಯಿತಂತೆ. ಈ ಕಥೆಯೇನಾದರೂ ನಿಜವೆನ್ನು ವುದಾದಲ್ಲಿ, ನನ್ನ ಕುರಿತಾಗಿ ಇಷ್ಟೊಂದು ಚಿಂತಿಸುತ್ತಿರುವ ನೀವೂ ಅದನ್ನು ನೆನಪಿಟ್ಟು ಕೊಳ್ಳುವುದು ಒಳ್ಳೆಯದು. ದಯವಿಟ್ಟು ಅದರ ಪರಿಣಾಮವೇನಾದೀತೆಂದು ಯೋಚಿಸಿನೋಡಿ.”

ಬಾಲಕ ಸ್ವಭಾವದ ಶ್ರೀರಾಮಕೃಷ್ಣರು ಈ ಮಾತನ್ನು ಕೇಳಿ ಅಲ್ಲಾಡಿಹೋದರು. ಸತ್ಯವ್ರತನಾದ ನರೇಂದ್ರನಲ್ಲವೆ ಹಾಗೆ ಹೇಳುತ್ತಿರುವುದು? ಎಂದಮೇಲೆ ಅದು ನಿಜವೇ ಇರಬೇಕು!

“ಆಹ್! ಸರಿಯಾಗೇ ಹೇಳಿದೆ ಮಗು! ಹಾಗೆಯೇ ಆಗಿಬಿಟ್ಟರೇನು ಗತಿ? ಆದರೆ ನನಗೆ ನಿನ್ನನ್ನು ನೋಡದೆ ಇರುವುದಕ್ಕೆ ಸಾಧ್ಯವೇ ಇಲ್ಲವಲ್ಲ! ಈಗೇನು ಮಾಡಲಪ್ಪ?” ಎಂದುದ್ಗರಿಸಿ ದರು. ಭಯಾಕ್ರಾಂತರಾಗಿ ಓಡಿದರು, ಮಾತೆಯ ಮಂದಿರಕ್ಕೆ! ಆದರೆ, ಒಂದೆರಡು ನಿಮಿಷದಲ್ಲಿ ಹಿಂದಿರುಗಿದಾಗ ಅವರ ಮುಖ ನಗೆಯಿಂದ ಬೆಳಗುತ್ತಿತ್ತು. ಬಂದವರೇ ಹುಸಿ ಮುನಿಸಿನಿಂದ ಗದರಿದರು:

“ಹೋಗೋ ಫಟಿಂಗ! ಇನ್ನೆಂದಿಗೂ ನಾನು ನಿನ್ನ ಮಾತಿಗೆ ಕಿವಿಗೊಡಲಾರೆ. ತಾಯಿ ಏನು ಹೇಳಿದಳು ಗೊತ್ತೆ? ‘ನೀನು ಅವನನ್ನು ಸ್ವಯಂ ನಾರಾಯಣನೆಂಬಂತೆ ನೋಡುತ್ತಿರುವುದ ರಿಂದಲೇ ಅಷ್ಟೊಂದು ಪ್ರೀತಿಸುತ್ತಿರುವುದು. ಯಾವಾಗ ಅವನಲ್ಲಿ ನಾರಾಯಣನನ್ನು ಕಾಣು ವುದಿಲ್ಲವೋ, ಆಗ ನೀನು ಅವನನ್ನು ಕಣ್ಣೆತ್ತಿಯೂ ನೋಡಲಾರೆ!’ ಅಂತ ಹೇಳಿದಳು, ತಿಳಿಯಿತೋ?”

ಹೀಗೆ ಶ್ರೀರಾಮಕೃಷ್ಣರು ತಮ್ಮ ಒಂದೇ ಮಾತಿನಿಂದ ಅವನ ತರ್ಕವನ್ನೆಲ್ಲ ತಳ್ಳಿ ಹಾಕಿಬಿಟ್ಟರು.

ಅವರ ಈ ಮಾತಿನ ಮರ್ಮವೇನು? ನರೇಂದ್ರನನ್ನು ಅವರು ಕೇವಲ ಒಬ್ಬ ವ್ಯಕ್ತಿಯೆಂಬಂತೆ ಕಂಡು ಪ್ರೀತಿಸುತ್ತಿಲ್ಲ; ಬದಲಾಗಿ ನರರೂಪಿಯಾದ ಸ್ವಯಂ ನಾರಾಯಣನನ್ನಾಗಿ ಕಂಡು ಪ್ರೀತಿಸುತ್ತಿದ್ದಾರೆ! ನಾವು ಮನುಷ್ಯರನ್ನು ಕೇವಲ ಮನುಷ್ಯರೆಂಬ ದೃಷ್ಟಿಯಿಂದ ಪ್ರೀತಿಸಿದರೆ ಅದು ಮೋಹವೆನ್ನಿಸುತ್ತದೆ; ಆದರೆ ಭಗವತ್ಸ್ವರೂಪಿಗಳಂತೆ ಕಂಡು ಪ್ರೀತಿಸಿದರೆ ಅದು ಪ್ರೇಮವೆನಿಸುತ್ತದೆ. ‘ಮೋಹ’ವೆನ್ನುವುದು ಬಂಧನಕ್ಕೆ ಕಾರಣವಾದರೆ, ‘ಪ್ರೇಮ’ವೆಂಬುವುದು ಬಿಡುಗಡೆಗೆ–ಎಂದರೆ ಮುಕ್ತಿಗೆ–ಕಾರಣವಾಗುತ್ತದೆ. ಇಲ್ಲಿ ಶ್ರೀರಾಮಕೃಷ್ಣರು ನರೇಂದ್ರನಲ್ಲಿ ಸಾಕ್ಷಾತ್ ಭಗವಂತನನ್ನು ಕಾಣುತ್ತಿದ್ದಾರೆ; ಆದ್ದರಿಂದ ಅವರಿಗೆ ಭರತನ ಗತಿ ಉಂಟಾಗಲಾರದು. ನರೇಂದ್ರ ಹಾಗೂ ಇತರ ಶಿಷ್ಯರ ಮೇಲೆ ಶ್ರೀರಾಮಕೃಷ್ಣರ ಪ್ರೀತಿ ಅಷ್ಟೊಂದು ಉಕ್ಕಿ ಹರಿಯುತ್ತಿದ್ದುದರ ಮೂಲಕಾರಣವನ್ನು ನಾವಿಲ್ಲಿ ಕಂಡುಕೊಳ್ಳಬಹುದಾಗಿದೆ. ಅವರ ಪ್ರೀತಿ ಯೆನ್ನುವುದು ಮಾನುಷವಾದ ಸಂಕುಚಿತ ಮೋಹವಲ್ಲ.

ದಕ್ಷಿಣೇಶ್ವರದಲ್ಲಿ ಶ್ರೀರಾಮಕೃಷ್ಣರ ಬಳಿಯಲ್ಲೇ ಇನ್ನೊಬ್ಬ ‘ಮಹಾತ್ಮ’ವಾಸವಾಗಿದ್ದ. ಅವನ ಹೆಸರು ಹಾಜರಾ. ಅವನೊಬ್ಬ ಆಷಾಢಭೂತಿ; ತಾನು ದೊಡ್ಡ ಆಧ್ಯಾತ್ಮಿಕ ವ್ಯಕ್ತಿಯೆಂಬಂತೆ ಸೋಗು ಹಾಕುತ್ತಿದ್ದ. ಅಲ್ಲಿಗೆ ಬರುವ ಭಕ್ತರನ್ನು–ಮುಖ್ಯವಾಗಿ ಯುವಕರನ್ನು–ತನ್ನೆಡೆಗೆ ಸೆಳೆದುಕೊಳ್ಳುವುದು; ಅವರಿಗೆ ಒಂದಿಷ್ಟು ‘ಬೋಧನೆ’ ಮಾಡುವುದು; ಸಾಧ್ಯವಾದರೆ ಒಂದಿಷ್ಟು ಹಣ ಕೀಳುವ ಪ್ರಯತ್ನ ಮಾಡುವುದು–ಇದೇ ಅವನ ಕಸಬು. ಶ್ರೀರಾಮಕೃಷ್ಣರು ನರೇಂದ್ರ- ರಾಖಾಲಾದಿ ಹುಡುಗರ ಜೊತೆ ಬಹಳವಾಗಿ ಬೆರೆಯುವುದುನ್ನು ಕಂಡು ಒಮ್ಮೆ ಟೀಕಿಸುತ್ತಾನೆ, “ಏನು, ನೀವು ಈ ಹುಡುಗರನ್ನೆಲ್ಲ ಅಷ್ಟೊಂದು ಹಚ್ಚಿಕೊಂಡುಬಿಟ್ಟಿದ್ದೀರಲ್ಲ, ನೀವು ಧ್ಯಾನಜಪ ಮಾಡುವುದು ಯಾವಾಗ? ” ಈತ ಸ್ವತಃ ತಾನೇ ದೇವರ ಧ್ಯಾನ ಮಾಡದೆ, ಇತರರು ಬರುವುದು-ಹೋಗುವುದನ್ನೇ ಗಮನಿಸುತ್ತ ಕುಳಿತುಕೊಂಡಿರುವವನು; ಇಂಥವನು ಶ್ರೀರಾಮಕೃಷ್ಣರಿಗೆ ಈ ಮಾತುಗಳನ್ನು ಹೇಳುತ್ತಿದ್ದಾನೆ! ಅಲ್ಲದೆ ಅವರು ಇನ್ನು ಧ್ಯಾನ ಮಾಡಿ ಸಾಧಿಸಬೇಕಾದ್ದೇನಿದೆ? ಸಾಕ್ಷಾತ್ಕಾರದ ಅತ್ಯುನ್ನತ ಶಿಖರವನ್ನೇ ಏರಿರುವವರು ಅವರು. ಆದರೆ ಹಿಂದೆಯೇ ನೋಡಿದಂತೆ ಅವರದು ಬಾಲಕಸ್ವಭಾವ. ಹೋಗಿ ಜಗನ್ಮಾತೆಯನ್ನು ಕೇಳುತಕ್ತಾರೆ–‘ಹಾಜರಾ ಹೀಗೆ ಹೇಳುತ್ತಾನಲ್ಲ, ಏನು ಮಾಡಲಿ?’ ಎಂದು. ತಕ್ಷಣ ಜಗನ್ಮಾತೆ ಅವರಿಗೆ ಸಮಾಧಿಸ್ಥಿತಿಯಲ್ಲಿ ನೆನಪಿಸಿಕೊಡುತ್ತಾಳೆ–“ಎಲ್ಲ ಮನುಷ್ಯರ ರೂಪದಲ್ಲೂ ವ್ಯಕ್ತವಾಗಿರುವವಳು ನಾನೇ; ಆದರೆ ಪರಿಶುದ್ಧಾತ್ಮರಲ್ಲಿ ವಿಶೇಷವಾಗಿ ಪ್ರಕಟವಾಗಿದ್ದೇನೆ” ಎಂದು. ಉತ್ತರ ಸ್ವಷ್ಟವಾಗಿದೆ. ನರೇಂದ್ರಾದಿಗಳೆಲ್ಲ ಪರಿಶುದ್ಧಾತ್ಮರು. ಆದ್ದರಿಂದ ಪರಮಾತ್ಮ ಅವರಲ್ಲಿ ವಿಶೇಷವಾಗಿ ಪ್ರಕಟಗೊಂಡಿದ್ದಾನೆ. ಅದಕ್ಕೆ ಶ್ರೀರಾಮಕೃಷ್ಣರಿಗೆ ಅವರೆಂದರೆ ವಿಶೇಷ ಪ್ರೀತಿ. ವಿಶೇಷ ಆನಂದ, ಜಗನ್ಮಾತೆಯಿಂದ ಆ ಉತ್ತರ ಪಡೆದು ಸಹಜಸ್ಥಿತಿಗೆ ಮರಳಿದಾಗ ಶ್ರೀರಾಮಕೃಷ್ಣರು ಹಾಜರಾನ ಮೇಲೆ ಸಿಟ್ಟುಗೊಂಡು ತಮ್ಮಷ್ಟಕ್ಕೆ ತಾವೇ ಅಂದುಕೊಳ್ಳುತ್ತಾರೆ:“ ಎಂಥ ದುಷ್ಟ ಇವನು! ನನ್ನ ಮನಸ್ಸನ್ನೆಲ್ಲ ಕದಡಿಬಿಟ್ಟನಲ್ಲ!” ಆದರೆ ಮತ್ತೆ ತಾವೇ ಸಮಾಧಾನ ಮಾಡಿಕೊಳ್ಳುತ್ತಾರೆ: “ಹೋಗಲಿ, ಆ ಬಡಪಾಯಿಯನ್ನೇಕೆ ಬೈಯುವುದು? ಅವನಿಗೆ ಇದೆಲ್ಲ ಹೇಗೆ ಅರ್ಥವಾಗಬೇಕು!”

ಶ್ರೀರಾಮಕೃಷ್ಣರು ಇನ್ನೊಮ್ಮೆ ಭಕ್ತರೆದುರು ಹೇಳುತ್ತಾರೆ:

“ಈ ಪರಿಶುದ್ಧ ಹೃದಯದ ಹುಡುಗರನ್ನೆಲ್ಲ ನಾನು ಸಾಕ್ಷಾತ್ ನಾರಾಯಣ ಸ್ವರೂಪರು ಅಂತಲೇ ಭಾವಿಸುತ್ತೇನೆ. ನರೇಂದ್ರ ಮೊದಲ ಸಲ ಇಲ್ಲಿಗೆ ಬಂದಾಗ ನೋಡುತ್ತೇನ–ಅವನಲ್ಲಿ ದೇಹಭಾವನೆಯೇ ಇರಲಿಲ್ಲ. ನಾನು ಅವನ ಎದೆಯನ್ನು ಮುಟ್ಟಿದ ತಕ್ಷಣವೇ ಅವನಿಗೆ ಬಾಹ್ಯಪ್ರಜ್ಞೆ ತಪ್ಪಿಹೋಯಿತು; ಅಂಥ ಪರಿಶುದ್ಧ ಆತ. ಕ್ರಮೇಣ ನನಗೆ ಅವನನ್ನು ಮತ್ತೆಮತ್ತೆ ನೋಡಬೇಕು ಎನ್ನುವ ಭಾವನೆ ಬೆಳೆಯುತ್ತ ಬಂತು. ಅವನನ್ನು ನೋಡದೆ ಇದ್ದರೆ ಮನಸ್ಸಿಗೆ ತುಂಬ ನೋವಾಗುತ್ತಿತ್ತು. ಆಗ ಒಮ್ಮೆ ನಾನು ಕಾಳೀದೇವಾಯದ ಅಧಿಕಾರಿಯಾದ ಭೋಲಾನಾಥನನ್ನು ಕೇಳಿದೆ –‘ಏನಪ್ಪ, ನನಗೆ ಈ ತರಹದ ಭಾವ ಏಕೆ ಬರುತ್ತಿದೆ? ಅದೂ ಒಬ್ಬ ಹುಡುಗನ ಮೇಲೆ! ಅದರಲ್ಲೂ ಅವನೊಬ್ಬ ಕಾಯಸ್ಥರ ಜಾತಿಯವನು!’ ಅದಕ್ಕೆ ಭೋಲಾನಾಥ ಹೇಳಿದ: ‘ಮಹಾಶಯರೆ; ಎಲ್ಲ ಸರಿಯಾಗಿಯೇ ಇದೆ. ಮಹಾಭಾರತದಲ್ಲಿ ಹೇಳಿದೆ–ಯಾರ ಮನಸ್ಸು ಸಮಾಧಿಸ್ಥಿತಿಗೆ ಮುಟ್ಟಿ ಸಹಜಸ್ಥಿತಿಗೆ ಹಿಂದಿರುಗುತ್ತದೆಯೋ, ಅಂಥವರು ಸತ್ವಗುಣ ತುಂಬಿರುವ ವ್ಯಕ್ತಿಗಳ ಜೊತೆಯಲ್ಲಿ ಮಾತ್ರವೇ ಸಂತೋಷವಾಗಿರಬಲ್ಲರು; ಉನ್ನು ಆಧ್ಯಾತ್ಮಶೀಲ ವ್ಯಕ್ತಿಗಳ ಜೊತೆಯಲ್ಲಿ ಮಾತ್ರವೇ ಅವರಿಗೆ ಆನಂದ ಅಂತ. ಭೋಲಾನಾಥ ಹೇಳಿದ ಈ ಮಾತನ್ನು ಕೇಳಿದ ಮೇಲೆಯೇ ನನಗೆ ಸ್ವಲ್ಪ ಸಮಾಧಾನವಾಯಿತು.”

ನರೇಂದ್ರನ ಗುಣಗಾನ ಮಾಡುದುವುದೆಂದರೆ ಶ್ರೀರಾಮಕೃಷ್ಣರಿಗೆ ಎಲ್ಲಿಲ್ಲದ ಉತ್ಸಾಹ. ಒಮ್ಮೆ ಯದುಮಲ್ಲಿಕನ ಉದ್ಯಾನಗೃಹದ ಅಧಿಕಾರಿಯಾದ ರತನ್ ಎಂಬವನ ಮುಂದೆ ನರೇಂದ್ರನನ್ನು ಕೊಂಡಾಡುತ್ತ ಹೇಳುತ್ತಾರೆ: “ನೋಡು, ಈ ಹುಡುಗರೆಲ್ಲರೂ ಜಾಣರೇ; ಬಹಳ ಒಳ್ಳೇಯವರೇ. ಆದರೆ ನರೇಂದ್ರನಂಥವನನ್ನು ನಾನೆಲ್ಲೂ ನೋಡಿಲ್ಲ. ಅವನು ಹಾಡುಗಾರಿಕೆಯಲ್ಲಿ, ವಾದ್ಯಗಳನ್ನು ನುಡಿಸುವುದರಲ್ಲಿ ಎಷ್ಟು ನಿಪುಣನೋ ಜ್ಞಾನಸಂಪಾದನೆಯಲ್ಲೂ ಅಷ್ಟೇ ಪ್ರಚಂಡ; ಸಂಭಾಷಣೆ ನಡೆಸುವುದರಲ್ಲಿ ಎಷ್ಟು ಗಟ್ಟಿಗನೋ ಆಧ್ಯಾತ್ಮಿಕವಾಗಿಯೂ ಅಷ್ಟೇ ಸಮರ್ಥ! ಧ್ಯಾನಕ್ಕೆ ಕುಳಿತನೆಂದರೆ ಇಡೀ ರಾತ್ರಿ ಮೈಮೇಲೆ ಪರಿವೆಯೇ ಇರುವುದಿಲ್ಲ. ನನ್ನ ನರೇಂದ್ರ ಇದ್ದಾನಲ್ಲ, ಅವನು ಖೋಟಾ ಅಲ್ಲ, ತಾಜಾ ನಾಣ್ಯ; ಬಡಿದರೆ ‘ಠಣಾರ್’ ಎಂಹ ಶುದ್ಧವಾದ ನಾದವೇ ಬರುತ್ತದೆ... ಇತರ ಹುಡುಗರೆಲ್ಲ ಹಾಗೂ ಹೀಗೂ ಕಷ್ಟಪಟ್ಟು ಎರಡುಮೂರು ಪರೀಕ್ಷೆಗಳನ್ನು ಪಾಸುಮಾಡಿದರೆ ಅವರ ಮಟ್ಟಿಗೆ ಅದೇ ಬಹಳ ದೊಡ್ಡದು. ಅಲ್ಲಿಗೆ ಅವರದ್ದೆಲ್ಲ ಮುಗಿಯಿತು. ಆದರೆ ನರೇಂದ್ರ ಹಾಗಲ್ಲ ಅವನಿಗೆ ಎಲ್ಲವೂ ಲೀಲಾಜಾಲ. ಪರೀಕ್ಷೆ ಪಾಸು ಮಾಡುವುದೆಲ್ಲ ಅವನಿಗೆ ನೀರು ಕುಡಿದಂತೆ. ಅಲ್ಲದೆ ಅವನು ಬ್ರಾಹ್ಮಸಮಾಜಕ್ಕೆ ಹೋಗುತ್ತಾನೆ. ಆದರೆ ಅವನು ಉಳಿದ ಬ್ರಾಹ್ಮ ಸದಸ್ಯರಂತಲ್ಲ; ಅವನು ನಿಜವಾದ ಬ್ರಹ್ಮಜ್ಞಾನಿ! ಅವನು ಧ್ಯಾನಕ್ಕೆ ಕುಳಿತಾಗ ಬೆಳಕು ಕಾಣುತ್ತಾನೆ! ನಾನು ನರೇಂದ್ರನನ್ನು ಅಷ್ಟೊಂದು ಪ್ರೀತಿಸುವುದು ಸುಮ್ಮಸುಮ್ಮನೆ ಎಂದು ತಿಳಿದುಕೊಂಡೆಯಾ?”

ಇನ್ನೊಂದು ಸಲ, ತಮ್ಮ ಇತರ ಭಕ್ತರೊಂದಿಗೆ ನರೇಂದ್ರನನ್ನು ಹೋಲಿಸುತ್ತ ಅವನ ಹಿರಿಮೆಯನ್ನು, ಶ್ರೇಷ್ಠತೆಯನ್ನು ವರ್ಣಿಸುತ್ತಾರೆ:

“ನರೇಂದ್ರ ಅತ್ಯಂತ ಉನ್ನತ ಮಟ್ಟಕ್ಕೆ –ಅಖಂಡ ಸಚ್ಚಿದಾನಂ ಸಾಮ್ರಾಜ್ಯಕ್ಕೆ ಸೇರಿದವನು. ಅವನು ಸ್ವಭಾವತಃ ಪೌರುಷವಂತ. ಇಲ್ಲಿಗೆ ಎಷ್ಟೊಂದು ಜನ ಬರುತ್ತಾರೆ, ಆದರೆ ನರೇಂದ್ರನಂಥವರು ಇನ್ನೊಬ್ಬರಿಲ್ಲ.

“ಆಗಾಗ ನಾನು ಭಕ್ತರ ಸಾಮರ್ಥ್ಯವನ್ನು ತೂಗಿ ನೋಡುತ್ತಿರುತ್ತೇನ. ಅವರಲ್ಲಿ ಕೆಲವರು ಹತ್ತು ದಳದ ಪದ್ಮದ ಹಾಗೆ ಕಾಣುತ್ತಾರೆ; ಇನ್ನು ಕೆಲವರು ಹದಿನಾರು ದಳದ ಪದ್ಮದಂತೆ; ಇನ್ನು ಕೆಲವರು ಶತದಳ ಪದ್ಮದಂತೆ ಕಾಣುತ್ತಾರೆ. ಆದರೆ ನರೇಂದ್ರ ಇದ್ದಾನಲ್ಲ, ಅವನು ಮಾತ್ರ ಸಹಸ್ರದಳ ಪದ್ಮ.ಇತರ ಭಕ್ತರೆಲ್ಲ ತಂಬಿಗೆ-ಬಿಂದಿಗೆಗಳಾದರೆ ನರೇಂದ್ರ ದೊಡ್ಡ ಗುಡಾಣ. ಇತರರೆಲ್ಲ ಕೆರೆಕುಂಟೆಗಳಾದರೆ ನರೇಂದ್ರ ಮಹಾಸರೋವರ.

“ನರೇಂದ್ರ ಯಾರ ಹಿಡತಕ್ಕೂ ನಿಕ್ಕುವವನಲ್ಲ ಅವನು ಇಂದ್ರಿಯಸುಖಗಳಿಗೆ ಅಂಟಿಕೊಂಡು ಅವುಗಳ ವಶನಾಗುವವನಲ್ಲ. ಅವನು ಗಂಡು ಪಾರಿವಾಳದಂತೆ; ಗಂಡುಪಾರಿವಾಳದ ಕೊಕ್ಕನ್ನು ಗಟ್ಟಿಯಾಗಿ ಹಿಡಿದುಕೊಂಡರೆ ಅದು ತಪ್ಪಿಸಿಕೊಂಡು ಹಾರಿಹೋಗುತ್ತದೆ, ಆದರೆ ಹೆಣ್ಣು ಪಾರಿವಾಳ ಹಾಗೆ ಮಾಡಲಾರದು. ನರೇಂದ್ರನದು ಪುರುಷಸ್ವಭಾವ. ಆದ್ದರಿಂದಲೇ ಅವನು ಗಾಡಿಯ ಬಲಭಾಗಕ್ಕೆ ಕುಳಿತುಕೊಳ್ಳುತ್ತಾನೆ. ಆದರೆ ಭವನಾಥನದ್ದು ಸ್ತ್ರೀಸ್ವಭವ. ಆದ್ದರಿಂದಲೇ ಅವನಿ ಗಾಡಿಯ ಎಡಭಾಗದಲ್ಲಿ ಕುಳಿತುಕೊಳ್ಳವುದು. ಯಾವುದಾದರೂ ಸಭೆಗೆ ಹೋದಾಗ ನರೇಂದ್ರ ನನ್ನ ಜೊತೆಯಲ್ಲಿದ್ದುಬಿಟ್ಟರೆ ನನಗೆ ತುಂಬ ಧೈರ್ಯ!” ಹೀಗೆ ನರೇಂದ್ರನ ಗುಣ ಸ್ವಭಾವಗಳನ್ನ ಎಷ್ಟು ಪರಿಯಾಗಿ ವರ್ಣಿಸಿದರೂ ಅವರಿಗೆ ತೃಫ್ತಿಯಿಲ್ಲ.

ಒಮ್ಮೆ ಕಾರಣಾಂತರದಿಂದ ಕೆಲದಿನಗಳವರೆಗೆ ನರೇಂದ್ರ ದಕ್ಷಿಣೇಶ್ವರಕ್ಕೆ ಹೋಗಿರಲಿಲ್ಲ. ಒಂದು ದಿನ ಅವನು ಮನೆಯಲ್ಲಿ ಮಹಡಿಯ ಮೇಲೆ ತನ್ನ ಸ್ನೇಹಿತರೊಂದಿಗೆ ಯಾವುದೋ ವಿಷಯವಾಗಿ ಚರ್ಚಿಸುತ್ತ ಕುಳಿತಿದ್ದಾನೆ. ಆಗ ಕೆಳಗಡೆಯಿಂದ “ನರೇನ್, ಓ ನರೇನ್!” ಎಂಬ ಮಧುರ ಸ್ವರವೊಂದು ಕೇಳಿಬಂತು. ಆ ದನಿ ಅವನಿಗೆ ಸುಪರಿಚಿ. ಕೂಡಲೇ ಕೆಳಗೆ ಓಡಿಬಂದು ನೋಡುತ್ತಾನೆ–ಶ್ರೀರಾಮಕೃಷ್ಣರು ಬಂದು ಬಿಟ್ಟಿದ್ದಾರೆ! ಅವನನ್ನು ಕಂಡು ಅವರು ಅಶ್ರುನಯನರಾಗಿ, “ನರೇನ್, ಯಾಕಪ್ಪ ಇಷ್ಟು ದಿವಸ ಬರಲೇ ಇಲ್ಲ!”ಎಂದು ವಾತ್ಸಲ್ಯದಿಂದ ವಿಚಾರಿಸಿದರು. ಆ ಪ್ರೀತಿಯನ್ನು ಕಂಡು ನರೇಂದ್ರ ಮೂಕನಾದ. ಬಳಿಕ ಅವರನ್ನು ಒಳಕ್ಕೆ ಬರಮಾಡಿಕೊಂಡು ಕುಳ್ಳಿರಿಸಿದ. ಶ್ರೀರಾಮಕೃಷ್ಣರು ಅವನಿಗಾಗಿ ಸಹಿತಿನಿಸು ತಂದಿದ್ದರೆ; ಈಗ ಅದನ್ನು ಅವನಿಗೆ ತಮ್ಮ ಕೈಗಳಿಂದಲೇ ತಿನ್ನಿಸಿ ಆನಂದಿಸಿದರು. ಬಳಿಕ, ಅವರ ಕೋರಿಕೆಯಂತೆ ನರೇಂದ್ರ ತಂಬೂರಿ ಮೀಟುತ್ತ ಭಾವಭರಿತನಾಗಿ ದೇವಿಯ ಒಂದು ಕೀರ್ತನೆಯನ್ನು ಹಾಡತೊಡಗಿದ. ಅವನ ಶುದ್ಧಹೃದಯದಿಂದ ಹೊರಹೊಮ್ಮಿದ ಮಧುರಗಾಯನವನ್ನು ಕೇಳುತ್ತ ಶ್ರೀರಾಮಕೃಷ್ಣರು ಸಮಾಧಿಮಗ್ನರಾಗಿಬಿಟ್ಟರು. ಅವರ ನರೇಂದ್ರ ಪ್ರೇಮವನ್ನು ಶಬ್ದಗಳಿಂದ ಬಣ್ಣಿಸಿ ಮುಗಿಸಲುಂಟೆ?

ಶ್ರೀರಾಮಕೃಷ್ಣರು ತಮ್ಮೆಲ್ಲ ಯುವಶಿಷ್ಯರೊಂದಿಗೂ ಅತ್ಯಂತ ಆತ್ಮೀಯ ಬಾಂಧವ್ಯವನ್ನಿಟ್ಟುಕೊಂಡಿದ್ದರು. ಆದರೆ ನರೇಂದ್ರನೊಂದಿಗಿನ ಅವರ ಸಂಬಂಧ ಮಾತ್ರ ಇತರ ಶಿಷ್ಯರೊಂದಿಗಿನದ್ದಕ್ಕಿಂತ ವಿಭಿನ್ನವಾದದ್ದು;ಅದೊಂದು ವಿಶೇಷ ಸಂಬಂಧ. ಅವನನ್ನು ಸಂಧಿಸಿದ ಹೊಸತರಲ್ಲಿ ಒಮ್ಮೆ ಅವರು ಹೇಳಿದ್ದರು, “ನೋಡು, ನಿನ್ನಲ್ಲಿ ಸಾಕ್ಷಾತ್ ಶಿವ ಇದ್ದಾನೆ; ನನ್ನಲ್ಲಿ ಶಕ್ತಿಯಿದ್ದಾಳೆ. ಆದರೆ ಶಿವ-ಶಕ್ತಿಯರು ಬೇರೆಯಲ್ಲ ಒಂದೇ! ಈ ಮಾತುಗಳಾವುವೂ ನರೇಂದ್ರನಿಗೆ ಆಗ ಅರ್ಥವಾಗಲಿಲ್ಲವೆನ್ನಿ. ಇನ್ನೊಂದು ಮುಖ್ಯ ಸಂಗತಿಯನ್ನು ಇಲ್ಲಿ ಸ್ಮರಿಸಬಹುದು. ಏನೆಂದರೆ, ಶ್ರೀರಾಮಕೃಷ್ಣರು ಅವನಿಗೆ ತಮ್ಮ ವೈಯಕ್ತಿಕ ಸೇವೆಯನ್ನು ಮಾಡಲು ಅವಕಾಶ ಕೊಡುತ್ತಿರಲಿಲ್ಲ. ಏಕೆಂದರೆ, ಅವರು ಅವನಲ್ಲಿ ಸಾಕ್ಷಾತ್ ಶಿವನನ್ನು ಅಷ್ಟು ಸ್ಪಷ್ಟವಾಗಿ ಕಾಣುತ್ತಿದ್ದರು. ಅಲ್ಲದೆ, ಸೇವೆ ಮಾಡುವುದರ ಉದ್ದೇಶವಾದರೂ ಏನು? ತನ್ನ ಅಂತಶ್ಶುದ್ಧಿಗಾಗಿ. ಆದರೆ ಯಾರ ಹೃದಯ ಅದಾಗಲೇ ಪರಮ ಪರಿಶುದ್ಧವಾಗಿದೆಯೋ ಅಂಥವರು ಸೇವೆ ಮಾಡಬೇಕಾದ ಅಗತ್ಯವಾದರೂ ಏನಿದೆ? ಆದರೆ ತಾನು ಮಾತ್ರ ಈ ಭಾಗ್ಯವಿಶೇಷದಿಂದ ವಂಚಿತನಾದೆನಲ್ಲ ಎನ್ನುವುದು ನರೇಂದ್ರನ ಪರಿತಾಪ. ಆದ್ದರಿಂದ ಅವನು ತುಂಬ ನಮ್ರತೆಯಿಂದ ಒಂದಲ್ಲ ಒಂದು ಬಗೆಯ ಸೇವೆ ಮಾಡಲು ಮುಂದಾಗುತ್ತಿದ್ದ. ಆದರೆ ಶ್ರೀರಾಮಕೃಷ್ಣರು ಅದಕ್ಕೆ ಅವಕಾಶ ಕೊಡುತ್ತಲೇ ಇರಲಿಲ್ಲ; “ನಿನ್ನ ದಾರಿ ಬೇರೆ” ಎಂದುಬಿಡುತ್ತಿದ್ದರು.

ಆಧ್ಯಾತ್ಮಿಕ ಜೀವನದಲ್ಲಿ ಶಿಸ್ತಿಗೆ ಅಗ್ರಸ್ಥಾನ. ಆದ್ದರಿಂದ ಶ್ರೀರಾಮಕೃಷ್ಣರು ತಮ್ಮ ಶಿಷ್ಯರ ನಿದ್ರಾಹಾರಗಳು, ಧ್ಯಾನ-ಜಪ-ಪ್ರಾರ್ಥನೆಗಳು, ದೈನಂದಿನ ಕಾರ್ಯಕಲಾಪಗಳು –ಈ ಎಲ್ಲ ವಿಷಯಗಳಲ್ಲೂ ಕಟ್ಟುಪಾಡುಗಳನ್ನು ಮಾಡಿಟ್ಟಿದ್ದರು. ಆದರೆ ಆಶ್ಚರ್ಯದ ಸಂಗತಿಯೆಂದರೆ, ನರೇಂದ್ರನ ವಿಷಯದಲ್ಲಿ ಮಾತ್ರ ಅವರು ಯಾವ ನಿಯಮ-ನಿರ್ಬಂಧವನ್ನೂ ಹೇರಿರಲಿಲ್ಲ. ಅವರು ಎಲ್ಲರೆದುರಿಗೂ ಹೇಳುತ್ತಿದ್ದರು: “ನರೇಂದ್ರ ನಿತ್ಯಸಿದ್ಧ, ಜನ್ಮತಃ ಧ್ಯಾನಿಸಿದ್ಧ. ಅವನಿನ್ನು ಹೊಸದಾಗಿ ಧ್ಯಾನವನ್ನು ಕಲಿತು ಅಭ್ಯಾಸ ಮಾಡಬೇಕಿಲ್ಲ. ಅವನಲ್ಲಿ ಜ್ಞಾನಾಗ್ನಿಯೆಂಬುದು ಪ್ರಜ್ವಲಿಸುತ್ತಿದೆ. ಅವನು ಎಂತಹ ಅಶುದ್ಧವಾದ ಆಹಾರವನ್ನು ತಿಂದರೂ ಅದರ ದುಷ್ಪರಿಣಾಮವನ್ನು ಆ ಅಗ್ನಿಯು ಸುಟ್ಟು ಭಸ್ಮಮಾಡಿಬಿಡಬಲ್ಲದು. ಆದ್ದರಿಂದ ಅಶುದ್ಧ ಆಹಾರಸೇವನೆಯಿಂದ ಅವನ ಮನಸ್ಸು ಎಂದಿಗೂ ಮಲಿನಗೊಳ್ಳುವುದಿಲ್ಲ. ತನ್ನ ಜ್ಞಾನಖಡ್ಗದಿಂದ ಅವನು ಮಾಯಾಪಾಶವನನು ಕತ್ತರಿಸಿಹಾಕಿಬಿಡುತ್ತಾನೆ. ಭುವನಮೋಹಿನೀ ಮಾಯೆ ಅವನನ್ನೆಂದಿಗೂ ಕಟ್ಟಿಹಾಕಲಾರಳು.” ಅವನ ಕುರಿತಾಗಿ ಇಂಥ ಮಾತುಗಲನ್ನೆಲ್ಲ ಕೇಳಿದರವರು ಬೆಕ್ಕಸಬೆರಗಾಗುತ್ತಿದ್ದರು, ಆ ಮಾತುಗಳ ಮರ್ಮವನ್ನರಿಯಲಾಗದೆ ಸ್ತಬ್ಧರಾಗುತ್ತಿದ್ದರು. ಶ್ರೀರಾಮಕೃಷ್ಣರ ದರ್ಶನಕ್ಕಾಗಿ ಬರುತ್ತಿದ್ದ ಅನೇಕ ಭಕ್ತರ –ಮುಖ್ಯವಾಗಿ ಶ್ರೀಮಂತರು–ಹಣ್ಣುಹಂಪಲು, ಸಿಹಿತಿಂಡಿ ಇವುಗಳನ್ನು ಕಾಣಿಕೆಯಾಗಿ ಅರ್ಪಿಸುತ್ತಿದ್ದರು. ಸಾಮಾನ್ಯವಾಗಿ ಶ್ರೀರಾಮಕೃಷ್ಣರು ಅವುಗಳಲ್ಲಿ ಒಂದು ತುಣುಕನ್ನು ಮಾತ್ರ ಪರಿಗ್ರಹಿಸಿ, ಉಳಿದುದನ್ನು ಭಕ್ತಿರಿಗೆ ಹಂಚಿಬಿಡುತ್ತಿದ್ದರು. ಆದರೆ ಆ ಕಾಣಿಕೆಗಳನ್ನು ಕೊಟ್ಟವರ ಚಾರಿತ್ರ್ಯ ದೋಷಪೂರ್ಣವಾಗಿದ್ದರೆ, ಇಲ್ಲವೆ ಸ್ವಾರ್ಥೋದ್ದೇಶದಿಂದ ಕೂಡಿದ್ದರೆ ಶ್ರೀರಾಮಕೃಷ್ಣರಿಗೆ ಅದು ಸ್ಪಷ್ಟವಾಗಿ ಗೋಚರಿಸಿಬಿಡುತ್ತಿತ್ತು. ಅಂಥವನ್ನು ಅವರು ತಾವೂ ತಿನ್ನುತ್ತಿರಲಿಲ್ಲ. ಭಕ್ತರಿಗೂ ಕೊಡುತ್ತಿರಲಿಲ್ಲ. ಅವು ಸಾಧಕರ ದೇಙ-ಮನೋಬುದ್ಧಿಗಳಿಗೆ ತುಂಬ ಹಾನಿಕಾರಕ ಎಂಬುದೇ ಅದಕ್ಕೆ ಕಾರಣ. ಹಾಗಾದರೆ ಅಂತಹ ಉಪಯುಕ್ತ ಪದಾರ್ಥಗಳನ್ನೆಲ್ಲ ಏನು ಮಾಡುವುದ? ಚಿಂತೆಯಿಲ್ಲ; ಅವುಗಳನ್ನು ಅರಗಿಸಿಕೊಳ್ಳಬಲ್ಲ ಧೀರನೊಬ್ಬನಿದ್ದಾನೆ–ನರೇಂದ್ರನಾಥ! ಆ ತಿನಿಸುಗಳನ್ನೆಲ್ಲ ಅವರು ನರೇಂದ್ರನಿಗೆ ಕೊಟ್ಟಬಿಡುತ್ತಿದ್ದರು. ಒಂದು ವೇಳೆ ಅವನು ಆಗ ಅಲ್ಲಿಲ್ಲದಿದ್ದರೆ ಅಥವಾ ಸದ್ಯದಲ್ಲೇ ಬರುವ ಸಂಭವವಿಲ್ಲದಿದ್ದರೆ, ಅವನ ಮನೆಗೇ ಕಳಿಸಿಕೊಡುತ್ತಿದ್ದರು. ಒಮ್ಮೊಮ್ಮೆ ಅವನು, “ನೀವು ನಿಷಿದ್ಧ ಎಂದು ಹೇಳು ಪದಾರ್ಥಗಳನ್ನೇ ಹೋಟೆಲಿನಲ್ಲಿ ನಾನಿವತ್ತು ತಿಂದೆ” ಎಂದು ಹೇಳುತ್ತಿದ್ದ. ಆದರೆ ಅವನು ‘ಸಾಹಸ ಪ್ರದರ್ಶನ’ಕ್ಕಾಗಿ ಹಾಗೆ ತಿಂದದ್ದಲ್ಲ, ಸ್ನೇಹಿತರ ಜೊತೆಯಲ್ಲಿ ಹೋಗಿದ್ದರಿಂದ ತಿನ್ನಬೇಕಾಯಿತು ಅಷ್ಟೆ, ಎಂಬುದನ್ನು ತಿಳಿದಿರುವ ಶ್ರೀರಾಮಕೃಷ್ಣರು ಹೇಳುತ್ತಿದ್ದರು: “ಅದರಿಂದ ನಿನಗೇನೂ ತೊಂದರೆಯಾಗುವುದಿಲ್ಲ. ಯಾರು ಭಗವಂತನಲ್ಲಿ ಮನಸ್ಸನ್ನು ದೃಢವಾಗಿ ನೆಲೆಗೊಳಿಸಿದ್ದಾರೋ ಅಂಥವರಿಗೆ ಹಂದಿಮಾಂಸ ಹಸುವಿನ ಮಾಂಸಗಳು ಕೂಡ ಹವಿಷ್ಯಾನ್ನಕ್ಕೆ ಸಮ. ಆದರೆ ಯಾರು ಕಾಮಕಾಂಚನದಲ್ಲಿ ಮನಸ್ಸಿಟ್ಟಿದ್ದಾರೋ ಅಂಥವರು ತರಕಾರಿಯನ್ನೇ ತಿಂದರೂ ಅದು ಹಂದಿಮಾಂಸ ಹಸುವಿನ ಮಾಂಸಕ್ಕೆ ಸಮ. ಆದ್ದರಿಂದ ನೀನು ನಿಷದ್ಧವೆನಿಸಿಕೊಂಡು ಆಹಾರವನ್ನೇ ತಿಂದಿದ್ದರೂ ನನ್ನಿಂದೇನೂ ಅಭ್ಯಂತರವಿಲ್ಲ. ಆದರೆ ಈ ಇತರ ಹುಡಗರೇನಾದರೂ ಹಾಗೆ ಮಾಡಿದರೆ, ಅವರು ನನ್ನನ್ನು ಮುಟ್ಟಿದರೂ ಸಹಿಸಿಕೊಳ್ಳಲಾರೆ.”

ಆದರೆ ಕಾಣಿಕೆಗಳನ್ನು ಸ್ವೀಕರಿಸುವ ವಿಷಯದಲ್ಲಿ ಶ್ರೀರಾಮಕೃಷ್ಣರು ತೋರಿಸುತ್ತಿದ್ದ ತಾರತಮ್ಯವನ್ನು ಕಂಡು ನರೇಂದ್ರನಿಗೆ ಆಶ್ಚರ್ಯ. ಇದು ಕೇವಲ ಅವರ ಒಂದು ವಿಚಿತ್ರ ಅಭ್ಯಾಸವೋ ಅಥವಾ ಅವರ ಅತಿ ಮಡಿವಂತಿಕೆಯೋ ಎಂದು ಆಲೋಚಿಸಿದ. ಆದರೆ ಶ್ರೀರಾಮಕೃಷ್ಣರು ಇದಕ್ಕೆ ವಿವರಣೆ ಕೊಡುತ್ತಾರೆ–ಯಾರ ಗುಣ-ನಡತೆ ಶುದ್ಧವಾಗಿರುವುದಿಲ್ಲವೋ ಅಂಥವರ ಕೈಯಿಂದ ಯಾವುದೇ ವಸ್ತುವನ್ನು ಪರಿಗ್ರಹಿಸಲು ತಮ್ಮಿಂದ ಸಾಧ್ಯವೇ ಆಗುವುದಿಲ್ಲ–ಎಂದು. ಇದನ್ನು ಕೇಳಿ ನರೇಂದ್ರನ ಕುತೂಹಲ ಇನ್ನಷ್ಟು ಕೆರಳಿತು. ‘ಇದೇನಾಶ್ಚರ್ಯ! ಅವರ ಶೀಲ ಚೆನ್ನಾಗಿದೆಯೋ ಇಲ್ಲವೋ ಇವರಿಗೆ ಹೇಗೆ ಗೊತ್ತಾಗಬೇಕು? ಎಲ್ಲರ ಚಾರಿತ್ರ್ಯವನ್ನೂ ಅಳೆದುನೋಡಲು ಇವರಿಂದ ಸಾಧ್ಯವೆ? ಆಗಲಿ, ಇದನ್ನು ಪರೀಕ್ಷೆ ಮಾಡಿ ನೋಡಿಯೇಬಿಡಬೇಕು’ ಎಂದು ತೀರ್ಮಾನಿಸಿದ. ಅದರಂತೆ ಯಾರ್ಯಾರು ತಂದುಕೊಟ್ಟ ತಿಂಡಿತಿನಿಸುಗಳನ್ನು ಅವರ ಶೀಲ ಚೆನ್ನಾಗಿಲ್ಲವೆಂಬ ಕಾರಣದಿಂದ ಶ್ರೀರಾಮಕೃಷ್ಣರು ಬದಿಗಿಟ್ಟಬಿಡುತ್ತಿದ್ದರೋ ಅಂಥವರ ಗುಣನಡತೆಗಳನ್ನೆಲ್ಲ ಕೆದಕಿ ಪರೀಕ್ಷಿಸಲು ಹೊರಟ. ಹಾಗೆ ನೋಡಿದಾಗ ಮೇಲ್ನೋಟಕ್ಕೆ ಆ ವ್ಯಕ್ತಿಗಳಲ್ಲಿ ಒಬ್ಬೊಬ್ಬನೂ ತುಂಬ ಸದ್ಗುಣವಂತನಂತೆ ಕಂಡುಬಂದರೂ ಅವರ ಶೀಲ ನಿಜಕ್ಕೂ ಚೆನ್ನಾಗಿಲ್ಲವೆಂಬುದು ದೃಢಪಟ್ಟಿತು. ಶ್ರೀರಾಮಕೃಷ್ಣರ ಪಾವಿತ್ರ್ಯವೆಂಬುದು ಅದೆಷ್ಟು ಉನ್ನತ ಮತ್ತು ಅವರ ಪರಿಶೀಲನಾ ಶಕ್ತಿ ಅದೆಷ್ಟು ಸೂಕ್ಷ್ಮ ಎಂಬುದನ್ನು ಭಾವಿಸಿ ನೋಡುತ್ತ ಅವನು ಆಶ್ಚರ್ಯಚಕಿತನಾದ.

ಎಷ್ಟೊ ಸಲ ನರೇಂದ್ರನಿಗೂ ಇತರ ಭಕ್ತರಿಗೂ ನಾನಾ ವಿಷಯಗಳ ಬಗ್ಗೆ ವಾದ-ವಿವಾದವೇರ್ಪಡುತ್ತಿತ್ತು. ಈ ದೃಶ್ಯವನ್ನು ನೋಡಲು ಶ್ರೀರಾಮಕೃಷ್ಣರಿಗೆ ತುಂಬ ಹಿಗ್ಗು. ಒಮ್ಮೊಮ್ಮೆ ಅವರೇ ಯಾರಿಗಾದರೂ ನರೇಂದ್ರನೊಂದಿಗೆ ವಾದ ಮಾಡುವಂತೆ ಹೇಳುವುದೂ ಇತ್ತು. ಏಕೆಂದರೆ ಅವನ ವಾದದ ವೈಖರಿ ಅಷ್ಟೊಂದು ಬಿರುಸು, ಸೊಗಸು. ಅವನು ವಾದಕ್ಕಿಳಿದರೆ ಒಂದು ಬಿರುಗಾಳಿಯನ್ನೇ ಎಬ್ಬಿಸಿಬಿಡುತ್ತಿದ್ದ. ಪಾಶ್ಚಾತ್ಯ ಹಾಗೂ ಪೌರ್ವಾತ್ಯ ತತ್ತ್ವಶಾಸ್ತ್ರಜ್ಞರ ವಾಕ್ಯಗಳನ್ನು ಉದಾಹರಿಸುತ್ತ ಅವನು ಮಾತನಾಡುತ್ತಿದ್ದರೆ ಅವನ ತಿಳಿವಳಿಕೆಯ ಆಳವನ್ನು ಕಂಡು ಎಲ್ಲರೂ ಬೆರಗಾಗುತ್ತಿದ್ದರು. ಅಲ್ಲದೆ ಅವನು ತತ್ತ್ವವಿಚಾರವಾಗಿ ಮಾತನಾಡುತ್ತಿದ್ದರೆ ಅದು ಕೇವಲ ಪುಸ್ತಕಪಾಂಡಿತ್ಯದಂತೆ ಕಾಣುತ್ತಿರಲಿಲ್ಲ. ಆ ಜ್ಞಾನವೇ ಅವನಲ್ಲಿ ಮೈದಳೆದು ವ್ಯಕ್ತವಾಗುತ್ತಿರುವಂತೆ ಕಂಡುಬರುತ್ತಿತ್ತು. ಅವನು ತನ್ನ ಅದ್ಭುತ ವಿಚಾರಶಕ್ತಿಯಿಂದ ತನಗಿಂತಲೂ ಎಷ್ಟೊ ಹಿರಿಯರಾದವರೊಡನೆ ವಾದ ಮಾಡಿ ಗೆಲ್ಲುವುದನ್ನು ಕಂಡು ಶ್ರೀರಾಮಕೃಷ್ಣರಿಗಾಗುತ್ತಿದ್ದ ಸಂತಸ ಅಷ್ಟಿಷ್ಟಲ್ಲ. ಆದರೆ ಯಾವಾಗಲಾದರೊಮ್ಮೆ ಶ್ರೀರಾಮಕೃಷ್ಣರೇ ಅವನ ಎದುರುವಾದಿಯ ಬೆಂಬಲಕ್ಕೆ ನಿಂತುಬಿಟ್ಟರೆ ನರೇಂದ್ರ ಸೋಲೊಪ್ಪ ಬೇಕಾಗುತ್ತಿತ್ತು, ಅದು ಬೇರೆ ವಿಷಯ.

ನರೇಂದ್ರ ಬ್ರಾಹ್ಮಸಮಾಜದ ‘ಸಗುಣ-ನಿರಾಕಾರ ಬ್ರಹ್ಮ’ದ ತತ್ತ್ವಕ್ಕೆ ಸಂಪೂರ್ಣ ಬದ್ಧನಾಗಿದ್ದ. ಆದ್ದರಿಂದ ಯಾವುದೇ ದೇವದೇವಿಯರ ಅಸ್ತಿತ್ವವನ್ನೂ ಅವನು ಒಪ್ಪಕೊಳ್ಳುತ್ತಿರಲಿಲ್ಲ. ಪ್ರತಿಮೆಗಳನ್ನಿಟ್ಟು ಪೂಜಿಸುವುದು, ಅವುಗಳ ಮುಂದೆ ಅಡ್ಡ ಬೀಳುವುದು–ಇದನ್ನೆಲ್ಲ ಮೌಢ್ಯವೆಂದು ಹೀಗಳೆಯುತ್ತಿದ್ದ. ಗರಡಿಮನೆಯಲ್ಲಿ ಅವನ ಸ್ನೇಹಿತನಾಗಿದ್ದ ರಾಖಾಲನೂ ಬ್ರಾಹ್ಮಸಮಾಜದ ಸದಸ್ಯನಾಗಿದ್ದ. ಆದರೆ ರಾಖಾಲ ಈಗ ಶ್ರೀರಾಮಕೃಷ್ಣರ ನಿಕಟ ಸಂಪರ್ಕಕ್ಕೆ ಬಂದಿದ್ದ. ಅಲ್ಲದೆ ಅವನದು ಭಕ್ತಿಪ್ರಧಾನವಾದ ಸ್ವಭಾವ. ಶ್ರೀರಾಮಕೃಷ್ಣರ ದಿವ್ಯ ಸನ್ನಿಧಿ-ನಹವಾಸದಿಂದಾಗಿ ಕ್ರಮೇಣ ಅವನ ಭಕ್ತಿ ಇನ್ನಷ್ಟು ಉಜ್ವಲಗೊಳ್ಲತೊಡಗಿತು. ಆದ್ದರಿಂದ ಅವನು ಶ್ರೀರಾಮಕೃಷ್ಣರೊಂದಿಗೆ ಕಾಳೀದೇವಸ್ಥಾನಕ್ಕೆ ಹೋಗುತ್ತಿದ್ದ; ದೇವಿಗೆ ಪ್ರಣಾಮ ಸಲ್ಲಿಸುತ್ತಿದ್ದ. ಆದರೆ ಇದೆಲ್ಲ ಬ್ರಾಹ್ಮಸಮಾಜದ ತತ್ತ್ವಗಳಿಗೆ ವಿರುದ್ಧ. ರಾಖಾಲ ಹೀಹೆ ಕಾಳಿಗೆ ನಮಸ್ಕರಿಸುವುದನ್ನು ಒಮ್ಮೆ ನರೇಂದ್ರ ನೋಡಿಬಿಟ್ಟ. ತಕ್ಷಣ ಅವನನ್ನು ತರಾಟೆಗೆ ತೆಗೆದುಕೊಂಡು– “ಎನೋ! ಬ್ರಾಹ್ಮಸಮಾಜದಲ್ಲಿ ನೀನು ಪ್ರತಿಜ್ಞೆ ಮಾಡಿಲ್ಲವೆ, ವಿಗ್ರಹಗಳ ಮುಂದೆ ಶಿರಬಾಗುವುದಿಲ್ಲ ಅಂತ! ಈಗ ಇದೇನು ಮಾಡುತ್ತಿರುವುದು ನೀನು?” ರಾಖಾಲ ತುಂಬ ಮೆದು ಸ್ವಭಾವದವನು. ಅವನು ನರೇಂದ್ರನ ಜೊತೆಯಲ್ಲಿ ಪ್ರತಿವಾದ ಮಾಡುವುದರ ಬದಲಾಗಿ ಅವನಿಂದ ತಪ್ಪಿಸಿಕೊಂಡು ತಿರಗಾಡಲಾರಂಭಿಸಿದ. ಇದು ಶ್ರಿರಾಮಕೃಷ್ಣರ ದೃಷ್ಟಿಗೆ ಬಿತ್ತು. ಅವರೇ ರಾಖಾಲನ ನೆರವಿಗೆ ಬಂದರು. ನರೇಂದ್ರನನ್ನು ಕರೆದೆ, “ನರೇನ್, ರಾಖಾಲನನ್ನು ಸುಮ್ಮನೆ ಏತಕ್ಕೆ ಹೆದರಿಸುತ್ತೀಯೋ? ಅವನು ನಿನ್ನನ್ನು ಕಂಡರೇ ಭಯಪಟ್ಟುಕೊಂಡು ಓಡುತ್ತಾನೆ! ನೋಡು, ಅವನಿಗೀಗ ಸಾಕಾರ ಭಗವಂತನಲ್ಲಿ ನಂಬಿಕೆ ಉಂಟಾಗಿದೆ. ನೀನೇಕೆ ಅವನ ಭಾವವನ್ನು ಬದಲಾಯಿಸಲು ಹೋಗುವುದು? ಒಂದೇ ಸಲಕ್ಕೆ ನಿರಾಕಾರ ಬ್ರಹ್ಮವನ್ನು ಸಾಕ್ಷಾತ್ಕಾರ ಮಾಡಿಕೊಳ್ಳುವುದು ಎಲ್ಲರಿಗೂ ಸಾಧ್ಯವಾಗುವಂಥದಲ್ಲ.” ಈ ಮಾತಿಗೆ ಮನ್ನಣೆ ಕೊಟ್ಟು, ಅವನು ರಾಖಾಲನ ಭಾವಸ್ವಾತಂತ್ರ್ಯದಲ್ಲಿ ಕೈಹಾಕುವುದನ್ನು ಬಿಟ್ಟುಬಿಟ್ಟ.

ಆದರೂ ತನ್ನ ನಂಬಿಕೆಯೇ ಸರಿಯಾದ್ದದು ಎನ್ನುವ ಧೋರೆ ನರೇಂದ್ರನನ್ನು ಬಿಟ್ಟಿರಲಿಲ್ಲ. ಅದನ್ನು ಕಂಡು ಶ್ರಿರಾಮಕೃಷ್ಣರು, “ನೋಡು ಸಾಕ್ಷಾತ್ಕಾರಕ್ಕಿರುವುದ ಒಂದೇ ದಾರಿ ಅಂತ ತಿಳಿಯಬೇಡ. ಎಲ್ಲ ದಾರಿಗಳಿಂದಲೂ, ಎಲ್ಲ ದಿಕ್ಕುಗಳಿಂದಲೂ ಅದೇ ಸತ್ಯವನ್ನು ತಲುಪಲು ಸಾಧ್ಯವಿದೆ. ಇದನ್ನು ಮನಗಾಣಲು ಪ್ರಯತ್ನಿಸು” ಎಂದು ತಿಳಿಯಹೇಳುತ್ತಿದ್ದರು. ಕ್ರಮೇಣ ಅವನು ಆಧ್ಯಾತ್ಮಿಕ ಸಾಧನೆಗಳ ಗುರಿ ಒಂದೇ ಎನ್ನುವುದನ್ನೇನೋ ಅರ್ಥಮಾಡಿಕೊಂಡರೂ, ಮೂರ್ತಿಪೂಜೆಯನ್ನು ಮಾತ್ರ ಸುತರಾಂ ಒಪ್ಪಿಕೊಳ್ಳಲಿಲ್ಲ. ಒಂದು ದಿನ ಶ್ರೀರಾಮಕೃಷ್ಣರು, ವಿಗ್ರಹಗಳು ಆಧ್ಯಾತ್ಮಿಕ ತತ್ತ್ವಗಳ ಮೂರ್ತರೂಪ ಎಂಬ ಸತ್ಯವನ್ನು ನರೇಂದ್ರನಿಗ ಮನಗಾಣಿಸಲು ಪ್ರಯತ್ನ ಪಟ್ಟು, ಸೋತು, ಕೊನೆಗೆ, “ನೀನು ನನ್ನ ತಾಯಿಯನ್ನು ಒಪ್ಪಿಕೊಳ್ಲದೆಯಿದ್ದಮೇಲೆ ಇಲ್ಲಿಗೇಕೆ ಬರುತ್ತಿರುವೆ ಮತ್ತೆ?” ಎಂದು ಕೆಣಕಿದರು. ನರೇಂದ್ರ ಇದಕ್ಕೆ ಕೊಟ್ಟ ಉತ್ತರವೆಂಥದು? ಅವನು ಕೇಳುತ್ತಾನೆ, “ಇಲ್ಲಿಗೆ ಬಂದಮಾತ್ರಕ್ಕೆ ನಾನು ಆಕೆಯನ್ನು ಒಪ್ಪಿಕೊಂಡುಬಿಡಬೇಕೋ?” ಇಂಥ ದಿಟ್ಟತನದ ಉತ್ತರವನ್ನು ಕೇಳಿ ಶ್ರೀರಾಮಕೃಷ್ಣರು ದಂಗಾಗಿಬಿಡಬೇಕಾಗಿತ್ತು. ಆದರೆ ಅವನ ಭೂತ-ಭವಿಷ್ಯಗಳೆಲ್ಲ ಅವರಿಗೆ ಅಂಗೈ ಮೇಲಿನ ನೆಲ್ಲಿಕಾಯಿಯಲ್ಲವೆ? ಮುಗುಳ್ನಕ್ಕು ನುಡಿದರು: “ ಆಗಲಿ, ಆಗಲಿ, ಹೀಗೇ ಹೇಳುತ್ತಿರು. ಇನ್ನು ಕೆಲವೇ ದಿನಗಳಲ್ಲಿ ನೀನು ನನ್ನ ತಾಯಿಯನ್ನು ಒಪ್ಪಿಕೊಳ್ಳವುದಷ್ಚೇ ಅಲ್ಲ, ಅವಳ ಹೆಸರಿನಲ್ಲಿ ಕಣ್ಣೀರು ಸುರುಸುತ್ತೀಯೆ, ನೋಡುತ್ತಿರು!” ಬಳಿಕ ಅಲ್ಲಿದ್ದ ಇತರ ಭಕ್ತರಿಗೆ ಹೇಳುತ್ತಾರೆ: “ಈ ಹುಡುಗನಿಗೆ ದೇವರ ಸಾಕಾರ ತತ್ತ್ವದಲ್ಲಿ ನಂಬಿಕೆಯಿಲ್ಲ. ನನ್ನ ಅತೀಂದ್ರಿಯ ಅನುಭವಗಳೆಲ್ಲ ಕೇವಲ ಭ್ರಾಂತಿ ಎನ್ನುತ್ತಾನೆ. ಆದರೆ ಇವನು ಶುದ್ಧ ಸಂಸ್ಕಾರಗಳಿಂದ ಕೂಡಿದವನು; ತುಂಬ ಒಳ್ಳೆಯ ಹುಡುಗ. ಪ್ರತ್ಯಕ್ಷ ಪ್ರಮಾಣವಿಲ್ಲದೆ ಯಾವುದನ್ನೂ ಸೀದಾಸೀದಾ ನಂಬುವವನೇ ಅಲ್ಲ. ಅಲ್ಲದೆ ತುಂಬ ಓದಿಕೊಂಡಿದ್ದಾನೆ; ಪ್ರಚಂಡವಾದ ಬುದ್ಧಿಶಕ್ತಿಯಿದೆ, ಒಳ್ಳೇ ವಿವೇಕ ಇದೆ.”

ನಿಜಕ್ಕೂ ಶ್ರೀರಾಮಕೃಷ್ಣ-ನರೇಂದ್ರರ ಸಂಬಂಧ ಅದ್ಭುತವೇ ಸರಿ. ನರೇಂದ್ರ ತಮ್ಮ ಅನುಭವಗಳನ್ನೇ ಪ್ರಶ್ನಿಸಿದರೂ, ತಮ್ಮ ಎಷ್ಟೋ ಅಭಿಪ್ರಾಯಗಳನ್ನು ಒಪ್ಪದೆ ಪ್ರತಿವಾದಿಸಿ ದರೂ, ಒಂದು ರೀತಿಯಲ್ಲಿ ಬಂಡಾಯಗಾರನಂತೆ ಕಾಣುತ್ತಿದ್ದರೂ ಶ್ರೀರಾಮಕೃಷ್ಣರು ಮಾತ್ರ ಅವನ ಮೇಲೆ ಪೂರ್ಣ ವಿಶ್ವಾಸವಿಟ್ಟಿದ್ದಾರೆ; ಪೂರ್ಣ ಪ್ರೀತಿ ಕೊಟ್ಟಿದ್ದಾರೆ. ಈ ಪ್ರೀತಿ ವಿಶ್ವಾಸ ಗಳು ಯಾವುದೇ ಕಾರಣದಿಂದಲೂ ಕಡಿಮೆಯಾಗುವಂಥದವಲ್ಲ. ಏಕೆಂದರೆ ಅವನು ಆ ರೀತಿ ಯಲ್ಲಿ ಏಕೆ ಪ್ರತಿಯೊಂದು ವಿಷಯದಲ್ಲೂ ತರ್ಕ ವಿತರ್ಕ ಮಾಡುತ್ತಾನೆ, ಏಕೆ ಎಲ್ಲವನ್ನೂ ಮತ್ತೆ ಮತ್ತೆ ಪ್ರಶ್ನಿಸಿ, ಪರೀಕ್ಷಿಸಿ ನೋಡುತ್ತಾನೆ ಎಂಬುದು ಶ್ರೀರಾಮಕೃಷ್ಣರಿಗೆ ಚೆನ್ನಾಗಿ ಗೊತ್ತಿತ್ತು. ಅವನ ಈ ವಿಚಾರಪೂರ್ಣ ಮನೋಭಾವವನ್ನು ಅವರು ಹೃತ್ಪೂರ್ವಕವಾಗಿ ಮೆಚ್ಚಿಕೊಂಡಿದ್ದರು.

ಅವರಿಬ್ಬರ ನಡುವೆ ಯಾವಾಗಲೂ ತೀವ್ರ ವಾದ ವಿವಾದಕ್ಕೆ ಒಳಗಾಗುತ್ತಿದ್ದ ಒಂದು ವಿಷಯ ವೆಂದರೆ ರಾಧಾ-ಕೃಷ್ಣರ ಪ್ರೇಮ ಮಿಲನದ ಮತ್ತು ಅವರ ವಿರಹ ವೇದನೆಯ ಕತೆ. ಆ ಇಡೀ ಘಟನೆಯ ಸತ್ಯತೆಯ ಬಗ್ಗೆಯೇ ನರೇಂದ್ರನಿಗೆ ಶಂಕೆ. ರಾಧಾ-ಕೃಷ್ಣರ ಕುರಿತಾದ ಎಲ್ಲ ಬಣ್ಣನೆ ಗಳೂ ಕೇವಲ ಕಟ್ಟುಕತೆ, ಕವಿಕಲ್ಪನೆ ಎಂದೇ ಅವನ ನಂಬಿಕೆ. ಮತ್ತು ಒಂದು ವೇಳೆ ಅವೆಲ್ಲ ನಿಜವೆಂದೇ ಇಟ್ಟುಕೊಂಡರೂ ರಾಧಾ-ಕೃಷ್ಣರ ನಡುವಿನ ಆ ಸಂಬಂಧವಂತೂ ಅನೈತಿಕ, ದೂಷಣೀಯ ಎಂದು ಹೀಗಳೆಯುತ್ತಿದ್ದ. ಸ್ವತಃ ಮಧುರಭಾವ ಸಾಧನೆ ಮಾಡಿ ರಾಧೆಯ ದಿವ್ಯ ಪ್ರೇಮವನ್ನು ತಮ್ಮೊಳಗೇ ತಂದುಕೊಂಡು, ಅನುಭವಿಸಿ, ಸಾಕ್ಷಾತ್ಕರಿಸಿಕೊಂಡಿದ್ದವರು ಶ್ರೀರಾಮ ಕೃಷ್ಣರು ಎಂಬ ಅದ್ಭುತ ವಿಚಾರವನ್ನು ಬಲ್ಲದವನೇನಲ್ಲ ನರೇಂದ್ರ. ಆದರೂ ಅವರ ರಾಧಾ-ಕೃಷ್ಣ ವರ್ಣನೆಯನ್ನು ಒಪ್ಪದೆ, ಅವರೊಂದಿಗೇ ನಿಶ್ಶಂಕೆಯಿಂದ ವಾದಿಸುತ್ತಾನೆ; ಲಕ್ಷಾಂತರ ಜನರ ಆರಾಧ್ಯದೈವರಾದ ರಾಧಾ-ಕೃಷ್ಣರ ಸಂಬಂಧವನ್ನು ಅಶ್ಲೀಲವೆಂದು ಖಂಡಿಸಿ ಈಡಾಡುತ್ತಾನೆ! ಶ್ರೀರಾಮಕೃಷ್ಣರು ಅವನಿಗೆ ಅನೇಕ ರೀತಿಗಳಲ್ಲಿ ತಿಳಿಯಹೇಳಿದರೂ ಅವನಿಗದು ಸಮ್ಮತವಾಗಲೇ ಇಲ್ಲ. ಕೊನೆಗೊಂದು ದಿನ ಅವರೆಂದರು:

“ಸರಿ, ನೀನೆನ್ನುವ ಹಾಗೆ ರಾಧೆ ಎನ್ನುವವಳೊಬ್ಬಳು ಇರಲೇ ಇಲ್ಲ ಅಂತಲೇ ಇಟ್ಟು ಕೊಳ್ಳೋಣ. ಅದು ಯಾವನೋ ಒಬ್ಬ ಪ್ರೇಮೀ ಸಾಧಕನ ಕಲ್ಪನೆ ಎಂದೇ ಭಾವಿಸೋಣ. ಆದರೆ ಆ ವ್ಯಕ್ತಿತ್ವವನ್ನು ಕಲ್ಪಿಸಿಕೊಂಡಾಗ ಆ ಸಾಧಕನು ರಾಧೆಯ ದಿವ್ಯಭಾವದಲ್ಲೇ ತನ್ಮಯನಾಗಿ, ತನ್ಮೂಲಕ ಆತ ರಾಧೆಯೇ ಆಗಿರಬೇಕು ಎಂದು ನೀನು ಒಪ್ಪುವೆಯಲ್ಲವೆ? ಅಂದಮೇಲೆ, ಬೃಂದಾವನಲೀಲೆಯು ಬಾಹ್ಯಪ್ರಪಂಚದಲ್ಲಿ ಕೂಡ ನಡೆದದ್ದು ನಿಜವೆಂದಾಗುತ್ತದೆಯಲ್ಲವೆ?” ಈ ಉತ್ತರವನ್ನು ನರೇಂದ್ರ ಒಪ್ಪಿಕೊಳ್ಳಲೇಬೇಕಾಯಿತು.

ಆದರೆ ಅವನ ಈ ಬಗೆಯ ಬಂಡಾಯದ ಬುದ್ಧಿಯನ್ನು ಕಂಡು ಶ್ರೀರಾಮಕೃಷ್ಣರು ಒಳ ಗೊಳಗೇ ಸಂತೋಷಿಸುತ್ತಿದ್ದರು. ಏಕೆಂದರೆ ನಿಜವಾದ ಬೋಧೆಯುಂಟಾಗಬೇಕಾದರೆ, ನಿಜವಾದ ಜ್ಞಾನದ ಬೆಳಕನ್ನು ಪಡೆಯಬೇಕಾದರೆ ಸಾಕಷ್ಟು ವಾದವಿವಾದಗಳನ್ನು ಮಾಡಲೇಬೇಕು, ಚೆನ್ನಾಗಿ ಹೋರಾಡಲೇಬೇಕು ಎಂದು ಅವರು ಅರಿತಿದ್ದರು. ಅಲ್ಲದೆ ಮುಂದೆ ನರೇಂದ್ರ ಇತರರ ಬೌದ್ಧಿಕ-ಆಧ್ಯಾತ್ಮಿಕ ಸಮಸ್ಯೆಗಳನ್ನು ಅರ್ಥಮಾಡಿಕೊಂಡು ಬಗೆಹರಿಸಲು ಸಮರ್ಥನಾಗಬೇಕಾದರೆ ಈ ಹಂತವನ್ನು ದಾಟಿ ಹೋಗಲೇಬೇಕು ಎಂದವರು ತಿಳಿದಿದ್ದರು. ಇಲ್ಲಿ ಇನ್ನೊಂದು ಗಮನೀಯ ವಾದ ವಿಷಯವಿದೆ. ಅದೇನೆಂದರೆ, ಅಂಥಾ ಪ್ರಚಂಡ ವಿಚಾರವಾದಿಯೂ ಸಂದೇಹಿಯೂ ಆದ ನರೇಂದ್ರ ತನ್ನೆಲ್ಲ ಸಮಸ್ಯೆಗಳನ್ನೂ ಬಗೆಹರಿಸಿಕೊಂಡು ಸತ್ಯಸಾಕ್ಷಾತ್ಕಾರದೆಡೆಗೆ ವೇಗವಾಗಿ ಮುಂದುವರಿಯುತ್ತಿರುವುದನ್ನು ಕಂಡಾಗ, ನಮಗೆ ಶ್ರೀರಾಮಕೃಷ್ಣರ ಅದ್ಭುತ ಶಿಕ್ಷಣ ಕೌಶಲದ ಪರಿಚಯವಾಗುತ್ತದೆ. ಅಲ್ಲದೆ, ಇಂದಿನ ನಾಸ್ತಿಕ್ಯದ ಯುಗದಲ್ಲಿ ಸ್ವಯಂ ಹಿಂದೂಧರ್ಮಸಾರವೇ ಶ್ರೀರಾಮಕೃಷ್ಣರ ರೂಪದಲ್ಲಿ ಮೈವೆತ್ತಿರುವಂತೆ ತೋರುತ್ತದೆ.

