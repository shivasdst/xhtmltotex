
\chapter{ದತ್ತ ಮನೆತನ}

ಹತ್ತೊಂಬತ್ತನೇ ಶತಮಾನದ ಪೂರ್ವಭಾಗ. ಭಾರತದ ಅಂದಿನ ರಾಜಧಾನಿ ಕಲ್ಕತ್ತಾ ಮಹಾನಗರ. ಕಲ್ಕತ್ತದಲ್ಲಿ ಸಿಮ್ಲಾ ಅಥವಾ ಸಿಮುಲಿಯಾ ಎನ್ನುವುದೊಂದು ಸ್ಥಳ; ಇಲ್ಲಿ ‘ದತ್ತ’ ಎಂಬುದೊಂದು ಕ್ಷತ್ರಿಯ ಮನೆತನ. ಈ ದತ್ತ ಮನೆತನದವರು ಶ್ರೀಮಂತರು, ವಿದ್ಯಾವಂತರು, ಘನತೆವೆತ್ತವರು. ದಾನಧರ್ಮಾದಿ ಕಾರ್ಯಗಳಲ್ಲಿ ಇವರು ತಲೆತಲಾಂತರದಿಂದಲೂ ಹೆಸರುವಾಸಿಯಾದವರು. ಇವರೆಲ್ಲ ಸ್ವತಂತ್ರ ಮನೋಭಾವದವರು, ಸ್ವಾತಂತ್ರ್ಯಪ್ರಿಯರು. ಈ ವಂಶದ ರಾಮಮೋಹನ ದತ್ತ ಎಂಬವನಿಗೆ ಇಬ್ಬರು ಮಕ್ಕಳು–ದುರ್ಗಾಪ್ರಸಾದ ಮತ್ತು ಕಾಳೀಪ್ರಸಾದ. ದುರ್ಗಾಪ್ರಸಾದ ತುಂಬ ಪ್ರತಿಭಾವಂತ; ಪಾರಸೀ—ಸಂಸ್ಕೃತ ಭಾಷೆಗಳಲ್ಲಿ ನುರಿತ ಪಂಡಿತ. ಇವನು ಕಾನೂನು ಪದವೀಧರ. ಆದ್ದರಿಂದ, ಸ್ವತಃ ವಕೀಲನಾದ ತಂದೆ ರಾಮಮೋಹನ ಇವನನ್ನು ತನ್ನೊಂದಿಗೆ ವಕೀಲಿ ವೃತ್ತಿಯಲ್ಲಿ ಮುಂದುವರಿಯುವಂತೆ ಪ್ರೋತ್ಸಾಹಿಸಿದ. ಆದರೆ ದುರ್ಗಾಪ್ರಸಾದನ ಮನೊಭಾವವೇ ಬೇರೆ. ಅವನ ಮನಸ್ಸೆಲ್ಲ ನಿವೃತ್ತಿಪರವಾದ ಸಂನ್ಯಾಸದ ಕಡೆಗೇ ವಾಲಿಕೊಂಡಿತ್ತು. ಅವನು ತನ್ನ ಇಪ್ಪತ್ತೈದನೆಯ ವಯಸ್ಸಿನವರೆಗೆ ಹೇಗೋ ಸಂಸಾರಜೀವನ ನಡೆಸಿದ. ಅವನಿಗೊಂದು ಗಂಡುಮಗುವೂ ಆಯಿತು. ಆದರೆ ಆ ವೇಳೆಗಾಗಲೇ ಅವನ ವೈರಾಗ್ಯ ತೀವ್ರವಾಗಿಬಿಟ್ಟಿತು. ೧೮೩೫ರಲ್ಲೊಂದು ದಿನ, ಮನೆ ಮಡದಿ ಮಗು ಎಲ್ಲವನ್ನೂ ಬಿಟ್ಟು ಹೊರಟುಹೋಗಿ ಸಂನ್ಯಾಸಿಯಾದ. ಬಳಿಕ ಮನೆಯವರಿಗೆ ಅವನ ಸುದ್ದಿಯೇ ಸಿಗಲಿಲ್ಲ. ಪತಿ ಹೀಗೆ ಸಂನ್ಯಾಸಿಯಾದರೂ ಪತ್ನಿ ಶ್ಯಾಮಸುಂದರಿ ತನ್ನ ಪುಟ್ಟ ಮಗು ವಿಶ್ವನಾಥನನ್ನು ಪಾಲಿಸಿ ಪೋಷಿಸಿ ಮೇಲಕ್ಕೆ ತರಬೇಕಾಗಿದೆ. ಶ್ಯಾಮಸುಂದರಿ ಸ್ವಭಾವತಃ ಧೈರ್ಯಶಾಲಿನಿ, ದೈವಭಕ್ತೆ. ಈಗ ಈ ಪುಟ್ಟ ಮಗುವನ್ನು ಸಾಕಿಸಲಹಿ ದೊಡ್ಡದು ಮಾಡುವ ಗುರುತರ ಜವಾಬ್ದಾರಿಯನ್ನೂ ನಿಭಾಯಿಸಬಲ್ಲ ಸಾಮರ್ಥ್ಯ ಅವಳಲ್ಲಿತ್ತು.

ಮಗು ವಿಶ್ವನಾಥ ಇನ್ನೂ ಮೂರು ವರ್ಷದವನಿರುವಾಗ ಶ್ಯಾಮಸುಂದರಿ ತನ್ನ ಮಗುವಿನೊಂದಿಗೆ ಕಾಶೀಯಾತ್ರೆ ಹೊರಟಳು. ಆಗಿನ್ನೂ ರೈಲುದಾರಿ ಆಗಿರಲಿಲ್ಲ. ಆದ್ದರಿಂದ ಇವಳ ಯಾತ್ರಿಕ ತಂಡ ದೋಣಿಯಲ್ಲಿ ಪ್ರಯಾಣ ಹೊರಟಿತು. ಕಲ್ಕತ್ತದಿಂದ ಕಾಶಿಗೆ ಸುಮಾರು ಐನೂರು ಮೈಲಿ. ಉದ್ದಕ್ಕೂ ಜಲಮಾರ್ಗವಾಗಿಯೇ ಪ್ರಯಾಣ. ಇದೊಂದು ರೋಮಾಂಚಕಾರೀ ಪ್ರವಾಸ. ಗಂಭೀರವಾಹಿನಿಯಾದ ಗಂಗೆಯ ಪವಿತ್ರ ಸಲಿಲದ ಮೇಲೆ ದೋಣಿ ತೇಲುತ್ತ ಹೋದಂತೆ ಪ್ರಯಾಣಿಕರಿಗೆ ನದಿಯ ಇಕ್ಕೆಲ\-ಗಳಲ್ಲಿನ ನೂತನ ನಗರಗಳು, ನೂತನ ದೃಶ್ಯಗಳು, ಜನವರ್ಗಗಳು, ನೂತನ ಭಾಷೆಗಳು–ಇವೆಲ್ಲ\-ದರ ಅನುಭವವಾಗುತ್ತಿತ್ತು.

ಒಂದು ದಿನ ಬೆಳಗ್ಗೆ ಎಂದಿನಂತೆ ದೋಣಿ ಸಾಗುತ್ತಿದೆ; ಮಗು ವಿಶ್ವಾನಾಥ ದೋಣಿಯ ಅಂಚಿನಲ್ಲಿ ಆಟವಾಡುತ್ತಿದ್ದವನು ದೊಪ್ಪನೆ ನದಿಯೊಳಗೆ ಬಿದ್ದುಬಿಟ್ಟ! ತಾಯಿ ಶ್ಯಾಮಸುಂದರಿ ಹಿಂದೆಮುಂದೆ ನೋಡದೆ ತಾನೂ ನದಿಯೊಳಗೆ ಹಾರಿಯೇ ಬಿಟ್ಟಳು! ಅವಳಿಗೇನೂ ಈಜು ಬರುತ್ತಿರಲಿಲ್ಲ. ಸಾಲದ್ದಕ್ಕೆ ಮೈತುಂಬ ಬಟ್ಟೆ ಬೇರೆ. ಆದರೆ ತನ್ನ ಮಗುವನ್ನು ರಕ್ಷಿಸಿಕೊಳ್ಳಬೇಕು ಎನ್ನುವ ಕಾತರದಲ್ಲಿ ಅವಳಿಗೆ ಆಗ ಅದೊಂದೂ ತೋಚಲಿಲ್ಲ. ಅದೃಷ್ಟವಶಾತ್ ಅವಳು ಸಕಾಲದಲ್ಲಿ ನೀರಿಗೆ ಧುಮುಕಿದ್ದರಿಂದ ಮಗುವಿನ ಕೈ ಸಿಕ್ಕಿತು; ಅದನ್ನೇ ಬಲವಾಗಿ ಹಿಡಿದುಕೊಂಡಳು. ಮಗುವನ್ನೇನೋ ಹಿಡಿದುಕೊಂಡಳು, ಆದರೆ ದೋಣಿಯನ್ನು ಸೇರುವ ಬಗೆ ಹೇಗೆ? ದೋಣಿ ಬೇರೆ ಮುಂದುಮುಂದಕ್ಕೆ ಸರಿಯುತ್ತಿದೆ! ಆದರೆ ಕುಶಲಿಗಳಾದ ಅಂಬಿಗರು ತಾಯಿ-ಮಗು ಇಬ್ಬರನ್ನೂ ರಕ್ಷಿಸಿ ಮೇಲೆ ತಂದರು. ಶ್ಯಾಮಸುಂದರಿ ಮಗುವನ್ನು ಎಷ್ಟೊಂದು ಬಲವಾಗಿ ಹಿಡಿದುಕೊಂಡಿದ್ದಳೆಂದರೆ, ಆ ಹಿಡಿತದಿಂದ ಮಗುವಿನ ಮೈಮೇಲೆ ಉಂಟಾದ ಗುರುತು ಹಲವಾರು ವರ್ಷಗಳವರೆಗೂ ಉಳಿದುಕೊಂಡಿತ್ತು.

ಅಂತೂ ತಾಯಿಮಕ್ಕಳು ವಾರಾಣಸಿಯನ್ನು ತಲುಪಿದರು. ಅಲ್ಲಿನ ಪವಿತ್ರ ವಾತಾವರಣದಿಂದ ಪುಳಕಿತಗೊಂಡ ಶ್ಯಾಮಸುಂದರಿ ಎಲ್ಲ ದೇವಾಲಯಗಳಿಗೂ ಹೋಗಿ ಪ್ರಣಾಮ ಸಲ್ಲಿಸಿದಳು. ಒಂದು ದಿನ ಅವಳು ಗಂಗಾನದಿಯಲ್ಲಿ ಸ್ನಾನ ಮುಗಿಸಿಕೊಂಡು ವಿಶ್ವನಾಥ ದೇವಾಲಯದ ಕಡೆಗೆ ಹೊರಟಿದ್ದಾಗ ದಾರಿಯಲ್ಲಿ ಕಾಲು ಜಾರಿ ಬಿದ್ದುಬಿಟ್ಟಳು. ಬಿದ್ದ ರಭಸಕ್ಕೆ ಪ್ರಜ್ಞೆಯೇ ತಪ್ಪಿ ಹೋದಂತಾಯಿತು. ಅದೇ ಸಮಯಕ್ಕೆ ಎದುರಿನಿಂದ ಬರುತ್ತಿದ್ದ ಒಬ್ಬ ಸಂನ್ಯಾಸಿ ಆಕೆಯ ನೆರವಿಗೆ ಒದಗಿಬಂದ. ಮೆಲ್ಲನೆ ಅವಳನ್ನು ಎತ್ತಿ ತಂದು ದೇವಸ್ಥಾನದ ಮೆಟ್ಟಿಲ ಮೇಲೆ ಮಲಗಿಸಿದ. ಶ್ಯಾಮಸುಂದರಿ ನಿಧಾನವಾಗಿ ಕಣ್ತೆರೆದು ನೋಡುತ್ತಾಳೆ–ಎದುರಿಗೆ ಒಬ್ಬ ಸಂನ್ಯಾಸಿ. ಪುನಃ ಕಣ್ಣುಜ್ಜಿಕೊಂಡು ನೋಡುತ್ತಾಳೆ–ಆ ಸಂನ್ಯಾಸಿ ಇನ್ನಾರೂ ಅಲ್ಲ, ತನ್ನ ಪತಿ! ಆತನಿಗೂ ಪತ್ನಿಯ ಗುರುತು ಸಿಕ್ಕಿತು. ತಕ್ಷಣ ಇಬ್ಬರ ಮನಸ್ಸಿನಲ್ಲೂ ಪ್ರಚಂಡ ಭಾವೋದ್ವೇಗ ಸಂಚಾರವಾಯಿತು. ಆದರೆ ಇಬ್ಬರೂ ಪ್ರಾಪಂಚಿಕತೆಯನ್ನು ತೊರೆದವರು. ಕಣ್ಮುಚ್ಚಿ ತೆರೆಯುವಷ್ಟರಲ್ಲಿ ಆ ಸಂನ್ಯಾಸಿ, “ಓ ಮಾಯೆ! ಮಾಯೆ!” ಎಂದು ಮಟಗುಟ್ಟುತ್ತ ಕಣ್ಮರೆಯಾಗಿಬಿಟ್ಟ. ಇತ್ತ ಶ್ಯಾಮಸುಂದರಿ ಏನೂ ನಡೆಯಲೇ ಇಲ್ಲವೆಂಬಂತೆ, ಸುಧಾರಿಸಿಕೊಂಡೆದ್ದು ತನ್ನ ದಾರಿಯಲ್ಲಿ ಮುಂದುವರಿದಳು.

ದುರ್ಗಾಪ್ರಸಾದ ಸಂನ್ಯಾಸಿಯಾಗಿ ಹೋದಾಗಿನಿಂದ ಅವನ ಸೋದರನಾದ ಕಾಳೀಪ್ರಸಾದನೇ ಮನೆಯ ಯಜಮಾನನಾದ. ಆದರೆ ಇವನು ಹೆಸರಿಗೆ ಮಾತ್ರ ಯಜಮಾನ ಅಷ್ಟೆ: ದುಡಿದು ತರಬಲ್ಲ ಸಾಮರ್ಥ್ಯ ಇರಲಿಲ್ಲ. ತನ್ನ ಪೂರ್ವಜರು ಕೂಡಿಟ್ಟ ಹಣವನ್ನೇ ಖರ್ಚು ಮಾಡುತ್ತ ಬಂದ. ‘ಕೂತುಂಬಗೆ ಕುಡಿಕೆ ಹಣವೂ ಸಾಲದು’ ಎಂಬಂತೆ ಇವನ ಸಂಪತ್ತು ಕರಗುತ್ತ ಬಂತು. ಸಾಲದ್ದಕ್ಕೆ ಸಂಸಾರ ಬೇರೆ ದೊಡ್ಡದು. ಬರಬರುತ್ತ ಬಡವರಾದರು. ಕಾಳೀಪ್ರಸಾದ ತನ್ನ ಅಣ್ಣನ ಸಂಸಾರವನ್ನಂತೂ ಬಹಳಷ್ಟು ನಿರ್ಲಕ್ಷಿಸಿಬಿಟ್ಟ. ಶ್ಯಾಮಸುಂದರಿ ಬಹಳಷ್ಟು ಕಷ್ಟಪಟ್ಟು ಮಗನನ್ನು ಬೆಳೆಸಬೇಕಾಯಿತು. ಹೋಗಲಿ, ಅವನನ್ನು ಪ್ರಾಯಪ್ರಬುದ್ಧನಾಗುವವರೆಗಾದರೂ ನೋಡಿ ಕೊಳ್ಳಲು ಅವಳಿಗೆ ಸಾಧ್ಯವಾಯಿತೇ ಎಂದರೆ ಅದೂ ಇಲ್ಲ. ಅಷ್ಟರೊಳಗೇ ಅವಳು ತೀರಿ ಹೋದಳು. ಹೀಗೆ ತಂದೆ ಸಂನ್ಯಾಸಿಯಾದ, ತಾಯಿ ತೀರಿಕೊಂಡಳು. ಮಗ ತಬ್ಬಲಿಯಾದ. ವಿಶ್ವನಾಥನಿಗಿನ್ನೂ ಹತ್ತು ವರ್ಷ. ಈಗ ಅವನನ್ನು ನೋಡಿಕೊಳ್ಳಬೇಕಾದವನು ಚಿಕ್ಕಪ್ಪನಾದ ಕಾಳೀಪ್ರಸಾದ. ಆದರೆ ಅವನು ಮೊದಲೇ ಅಣ್ಣನ ಸಂಸಾರದ ಕಡೆಗೆ ಉದಾಸೀನ. ಈಗಂತೂ ಈ ಅಸಹಾಯಕನಾದ ತಬ್ಬಲಿ ಹುಡುಗನನ್ನು ಅವನ ಪಾಡಿಗೆ ಬಿಟ್ಟುಬಿಟ್ಟ. ಬೆಳೆದು ದೊಡ್ಡವನಾದಂತೆಲ್ಲ ವಿಶ್ವನಾಥನಿಗೆ ಆಸ್ತಿಪಾಸ್ತಿಯ ವಿಷಯದಲ್ಲಿ ಚಿಕ್ಕಪ್ಪ ತನಗೆ ಮೋಸ ಮಾಡುತ್ತಿರುವುದು ಚೆನ್ನಾಗಿ ಗೊತ್ತಾಯಿತು. ಆದರೂ ಅವನು ಮಾತ್ರ ತನ್ನ ಚಿಕ್ಕಪ್ಪನನ್ನು ಕೊನೆಯವರೆಗೂ ಗೌರವದಿಂದಲೇ ನೋಡಿಕೊಂಡ.

ವಿಶ್ವನಾಥ ತಂದೆ-ತಾಯಿ ಇಲ್ಲದ ತಬ್ಬಲಿಯಾದರೂ ಅವನು ಬೆಳೆದುಬಂದದ್ದು ಅವಿಭಕ್ತ ಕುಟುಂಬದಲ್ಲಾದ್ದರಿಂದ ಅವನ ವಿಧ್ಯಾಭ್ಯಾಸ ನಿರಾತಂಕವಾಗಿ ನಡೆಯಿತು. ಬುದ್ಧಿವಂತನಾದ ವಿಶ್ವನಾಥನಲ್ಲಿ ಮೂಡಿಬರುತ್ತಿರುವ ಪ್ರತಿಭೆಯನ್ನು ಕಂಡರೆ ದತ್ತ ಮನೆತನದವರೆಲ್ಲ ಹೆಮ್ಮೆ ಪಡುವಂತಿತ್ತು. ಅವನು ತನ್ನ ವಿದ್ಯಾಭ್ಯಾಸದ ಅವಧಿಯಲ್ಲಿ ಹಲವಾರು ಭಾಷೆಗಳನ್ನು ಕಲಿತ, ಬಂಗಾಳಿ, ಇಂಗ್ಲಿಷ್, ಪರ್ಷಿಯನ್, ಅರೇಬಿಕ್, ಉರ್ದು, ಹಿಂದಿ–ಈ ಎಲ್ಲ ಭಾಷೆಗಳಲ್ಲಿ ಪರಿಣತಿ ಪಡೆದ. ಸಂಸ್ಕೃತ ಪಾಠಶಾಲೆಗೆ ಹೋಗಿ ಸಂಪ್ರದಾಯದ ಪ್ರಕಾರವೇ ಸಂಸ್ಕೃತ ಕಲಿತ. ವಿಶ್ವನಾಥನಿಗೆ ಇತಿಹಾಸದ ವಿಷಯದಲ್ಲಿ ಒಲವು. ಅವನು ಜ್ಯೋತಿಷವನ್ನೂ ಅಧ್ಯಯನ ಮಾಡಿದ್ದ. ಮುಂದೆ ತನ್ನ ಮಕ್ಕಳ ಜಾತಕವನ್ನು ಅವನೇ ಬರೆದ. ಅವನಿಗೆ ಸಂಗೀತದಲ್ಲೂ ವಿಶೇಷ ಆಸಕ್ತಿ. ತಂದೆ ದುರ್ಗಾಪ್ರಸಾದನ ಕಂಠದಂತೆ ಅವನ ಕಂಠವೂ ತುಂಬ ಮಧುರ. ಕೆಲವು ವರ್ಷ ಅವನು ಶಾಸ್ತ್ರೀಯ ಸಂಗೀತವನ್ನೂ ಅಭ್ಯಾಸ ಮಾಡಿದ್ದ. ಮುಂದೆ ಅವನು ಉದ್ಯೋಗಸ್ಥನಾಗಿ ಸ್ವತಂತ್ರನಾದ ಮೇಲೆ ಮನೆಯಲ್ಲಿ ಆಗಾಗ ಸಂಗೀತ ಕಛೇರಿಗಳನ್ನು ಏರ್ಪಡಿಸಿ ಆಹ್ವಾನಿತರಿಗೆಲ್ಲ ಪಲಾವಿನ ಔತಣ ಮಾಡಿಸುತ್ತಿದ್ದ.

ವಿದ್ಯಾಭ್ಯಾಸ ಮುಗಿಸಿದ ಮೇಲೆ ವಿಶ್ವನಾಥ ಕೈಗೊಂಡ ಮೊದಲ ವೃತ್ತಿ ವ್ಯಾಪಾರೋದ್ಯಮ. ಆದರೆ ಅದು ಅವನಿಗೆ ಕೈಹತ್ತಲಿಲ್ಲ. ಬಳಿಕ ವಕೀಲಿ ವೃತ್ತಿ ಹಿಡಿದ. ನುರಿತ ಆಂಗ್ಲ ವಕೀಲರಿಂದ ತರಬೇತಿ ಪಡೆದ. ಕ್ರಮೇಣ ಇವನ ವಕಾಲತ್ತು ಜನಪ್ರಿಯವಾಯಿತು. ಕಡೆಗೆ ಕಲ್ಕತ್ತದ ಹೈಕೋರ್ಟಿನ ವಕೀಲನಾಗಿ ಪ್ರತಿಷ್ಠಿತನಾದ. ಇವನ ವಕೀಲಿ ವೃತ್ತಿ ಉತ್ತರ ಹಿಂದೂಸ್ಥಾನದಲ್ಲೆಲ್ಲ ಪ್ರಸಿದ್ಧಿಗೊಂಡು ವ್ಯಾಪಿಸಿಕೊಂಡಿತು. ಈ ವೃತ್ತಿ ಅವನನ್ನು ದೂರದೂರದ ಸ್ಥಳಗಳಾದ ಲಕ್ನೋ, ಲಾಹೋರ್, ದೆಹಲಿ, ರಜಪುತಾನ, ಬಿಲಾಸಪುರ, ರಾಯಪುರಗಳಿಗೆಲ್ಲ ಕರೆದೊಯ್ದಿತು.

ವಿಶ್ವನಾಥ ವಕೀಲಿ ವೃತ್ತಿಯ ಮೂಲಕ ಅಪಾರ ಹಣ ಸಂಪಾದನೆ ಮಾಡಿದ. ಆದರೆ ಅಷ್ಟೇ ಧಾರಾಳವಾಗಿ ಖರ್ಚೂ ಮಾಡಿದ. ಮನೆಯ ತುಂಬ ಬಂಧುಗಳು, ಸ್ನೇಹಿತರು, ಮನೆಗೆಲಸಕ್ಕೆ ಹಲವಾರು ಜನ ಆಳುಕಾಳುಗಳು. ಸಂಚಾರಕ್ಕೆ ಸ್ವಂತ ಕುದುರೆ ಸಾರೋಟು. ಹೀಗೆ ಅವನದು ಸಮೃದ್ಧಿಯ ಜೀವನ, ಧಾರಾಳದ ಜೀವನ. ತನ್ನ ಬಳಿಯಿರುವವರೆಲ್ಲರೂ ತೃಪ್ತಿಯಿಂದಿರಬೇಕು ಎಂಬ ತುಂಬುಹೃದಯದ ಉಲ್ಲಾಸಮಯ ಸ್ವಭಾವ ಅವನದು. ಆತ ಒಳ್ಳೇ ರಸಜ್ಞ; ಪಾಕ ಶಾಸ್ತ್ರದಲ್ಲಿ ನುರಿತವನು. ಮನೆಗೆ ಬಂದ ಅತಿಥಿಗಳಿಗೆ ತಾನೇ ಸ್ವತಃ ರಸಭರಿತವಾದ ಭಕ್ಷ್ಯಗಳನ್ನು ತಯಾರಿಸಿ ಬಡಿಸುತ್ತಿದ್ದ.

ಬೆಳೆಯುವ ಮಕ್ಕಳಿಗೆ ಯೋಗ್ಯ ರೀತಿಯಲ್ಲಿ ಚೆನ್ನಾಗಿ ತಿನ್ನಿಸಬೇಕು, ಇಲ್ಲದಿದ್ದರೆ ಮಕ್ಕಳ ಬುದ್ಧಿಶಕ್ತಿ ವೃದ್ಧಿಯಾಗುವುದಿಲ್ಲ ಎನ್ನುವುದು ವಿಶ್ವನಾಥನ ನಂಬಿಕೆ. ಆದ್ದರಿಂದ ಮಕ್ಕಳಿಗೆ ಒಳ್ಳೆಯ ಪೌಷ್ಟಿಕಾಹಾರವನ್ನು ಒದಗಿಸಿಕೊಡುವಲ್ಲಿ ಖರ್ಚಿಗೆ ಸ್ವಲ್ಪವೂ ಹಿಂದೆ ಮುಂದೆ ನೋಡುತ್ತಿರಲಿಲ್ಲ. ಮಕ್ಕಳಿಗಾಗಿ ಅಪಾರವಾದ ಆಸ್ತಿ ಮಾಡಿಟ್ಟು ಹೋಗುವುದು ಅಗತ್ಯವೂ ಅಲ್ಲ, ಯೋಗ್ಯವೂ ಅಲ್ಲ; ಬದಲಾಗಿ ಅವರಿಗೆ ಶ್ರೇಷ್ಠ ಜೀವನಕ್ರಮವನ್ನು ಕಲಿಸಿಕೊಡಬೇಕು, ಒಳ್ಳೆಯ ವಿದ್ಯಾಭ್ಯಾಸ ಕೊಡಿಸಬೇಕು, ಅವರಲ್ಲಿ ಆರೋಗ್ಯಶಾಲಿಯಾದ ಶರೀರಸಂಪತ್ತನ್ನು ಬೆಳೆಸಬೇಕು; ಹೀಗೆ ಮಾಡಿ ಅವರನ್ನು ಅವರ ಪಾಡಿಗೆ ಬಿಟ್ಟುಬಿಟ್ಟರೆ ಮುಂದೆ ತಮ್ಮ ಜೀವನವನ್ನು ತಾವೇ ಚೆನ್ನಾಗಿ ನಿರ್ವಹಿಸಿಕೊಳ್ಳುತ್ತಾರೆ ಎನ್ನುವುದು ವಿಶ್ವನಾಥನ ಅಭಿಪ್ರಾಯ. ನಿಜಕ್ಕೂ ಇದೊಂದು ಗಮನಾರ್ಹವಾದ ಕ್ರಾಂತಿಕಾರೀ ಭಾವನೆಯೇ ಸರಿ. ಅವನು ಇಂತಹ ನಿರ್ಧಾರಕ್ಕೆ ಬರುವುದಕ್ಕೊಂದು ಪ್ರಬಲವಾದ ಕಾರಣವಿತ್ತು. ಹಿಂದೆ ಆ ದತ್ತ ಮನೆತನದ ಹಿರಿಯರು ಕೂಡಿಟ್ಟಿದ್ದ ಅಪಾರವಾದ ಸಂಪತ್ತನ್ನು ಮುಂದೆ ಹುಟ್ಟಿಬಂದ ಮಂದಿಯೆಲ್ಲ ಸುಮ್ಮನೆ ಕೂತುಂಡೇ ಕರಗಿಸಿಬಿಟ್ಟಿದ್ದರು. ಆದರೆ ವಿಶ್ವನಾಥ ಸ್ವತಃ ತಾನು ತಬ್ಬಲಿಯಾಗಿ ಕಷ್ಟದಲ್ಲೇ ಬೆಳೆದರೂ ಸ್ವಪ್ರಯತ್ನದಿಂದ ಶ್ರೀಮಂತನಾಗಿದ್ದ. ಇದನ್ನೆಲ್ಲ ಕಂಡೇ ಅವನು ಸರಿಯಾದ ಶಿಕ್ಷಣ ಮತ್ತು ಒಳ್ಳೆಯ ಆಹಾರ ಹಾಗೂ ವ್ಯಾಯಾಮಗಳ ಮೂಲಕ ಮಕ್ಕಳಿಗೆ ಆರೋಗ್ಯಶಾಲಿಯಾದ ಶರೀರಸಂಪತ್ತನ್ನೂ ತೇಜಸ್ವಿಯಾದ ವಿದ್ಯಾಬುದ್ಧಿಯನ್ನೂ ಕೊಡಬೇಕೇ ಹೊರತು ಅವರ ಹೆಸರಿನಲ್ಲಿ ಆಸ್ತಿ ಮಾಡಿಟ್ಟುಹೋಗುವುದಲ್ಲ ಎಂಬ ತೀರ್ಮಾನಕ್ಕೆ ಬಂದಿದ್ದ.

ವಿಶ್ವನಾಥ ತನ್ನ ಹಲವಾರು ಶ್ರೇಷ್ಠ ಗುಣಗಳಿಂದ ಬಹಳಷ್ಟು ಜನಪ್ರಿಯತೆ ಗಳಿಸಿದ್ದ. ಯಾವ ವ್ಯಕ್ತಿಗಳ ಜೊತೆಯಲ್ಲೇ ಇರಲಿ, ಎಂಥ ಪರಿಸ್ಥಿತಿಯಲ್ಲೇ ಆಗಲಿ ಅವನು ತನ್ನ ಉದಾತ್ತ-ಗಂಭೀರಭಾವವನ್ನು ಕಳೆದುಕೊಳ್ಳುತ್ತಿರಲಿಲ್ಲ. ಅವನದು ಸ್ವತಂತ್ರ ಮನೋಭಾವ, ಉದಾರ ಸ್ವಭಾವ. ಸ್ನೇಹಿತರೊಂದಿಗೆ ಅತ್ಯಂತ ಪ್ರಾಮಾಣಿಕ ಸ್ನೇಹ. ಅವನು ದೀನಬಂಧು. ದೀನದಲಿತರ ದೈನ್ಯದ ಕರೆಗೆ ಅವನ ಕಿವಿ ಯಾವಾಗಲೂ ತೆರೆದೇ ಇರುತ್ತಿತ್ತು. ಬಡವರ ಕಷ್ಟಕಾರ್ಪಣ್ಯಗಳನ್ನು ಕಂಡು ಅವನು ಕರಗಿಹೋಗುತ್ತಿದ್ದ. ಅವರಿಗಾಗಿ ಸಹಾನುಭೂತಿ ತೋರಿಸುತ್ತಿದ್ದ. ಕೇವಲ ಕಣ್ಣೊರೆಸುವ ಸಹಾನುಭೂತಿಯಲ್ಲ, ಅವರಿಗಾಗಿ ಎಷ್ಟಾದರೂ ಖರ್ಚು ಮಾಡಲು ಮುಂದಾಗುತ್ತಿದ್ದ. ಅವನ ಸಂಬಂಧಿಕರ ಮಕ್ಕಳೆಷ್ಟೋ ಜನ ಅವನ ಮನೆಯಲ್ಲೇ ವಾಸವಾಗಿದ್ದುಕೊಂಡು ಅವನ ಖರ್ಚಿನಲ್ಲೇ ವಿದ್ಯಾಭ್ಯಾಸ ಪಡೆದರು, ಮತ್ತು ಮುಂದೆ ಎಲ್ಲರೂ ಒಳ್ಳೊಳ್ಳೆಯ ಉದ್ಯೋಗ ಪಡೆದು ಯಶ್ವಸೀ ಜೀವನ ನಡೆಸುವಂತಾದರು. ಸಂಬಂಧಿಕರಿರಲಿ, ವಿಶ್ವನಾಥನ ನೆರೆಹೊರೆಯವರೂ ಕೂಡ ಅವನ ಸಹಾಯಹಸ್ತದಿಂದ ವಂಚಿತರಾಗಲಿಲ್ಲ. ಅವನಲ್ಲಿಗೆ ಸಹಾಯ ಬೇಡಿ ಬಂದವರು ಯಾರೂ ನಿರಾಶರಾಗಿ ಹಿಂದಿರುಗಲಿಲ್ಲ. ಆ ಪ್ರದೇಶದ ಜನರಲ್ಲಿ ಅವನು ‘ದಾತಾ ವಿಶ್ವನಾಥ’ ಎಂದೇ ಪ್ರಸಿದ್ಧನಾಗಿದ್ದ.

ಹೀಗೆ, ತಾನು ಹೇರಳವಾಗಿ ಸಂಪಾದಿಸಿದ್ದನ್ನು ಅವನು ಅಷ್ಟೇ ಧಾರಾಳವಾಗಿ ಖರ್ಚೂ ಮಾಡುತ್ತಿದ್ದ. ನಾಳೆಯ ಯೋಚನೆಯೇ ಇಲ್ಲದೆ ಕೇಳಿದವರಿಗೆಲ್ಲ ಕೊಡುತ್ತಿದ್ದ. ಅವನ ಔದಾರ್ಯ ಎಷ್ಟರಮಟ್ಟಿಗಿತ್ತೆಂದರೆ ಅವನು ಸೋಮಾರಿಗಳಾದ ಕೆಲವು ಸಂಬಂಧಿಗಳಿಗೆ ಆಶ್ರಯ ಕೊಟ್ಟದ್ದಲ್ಲದೆ ಅವರ ಕುಡಿತಕ್ಕೂ ಹಣ ಕೊಡುತ್ತಿದ್ದ! ಇದನ್ನು ಕಂಡ ಅವನ ಹಿರಿಯ ಮಗ ನರೇಂದ್ರ –ಇವನೇ ಮುಂದೆ ವೀರಸಂನ್ಯಾಸಿ ವಿವೇಕಾನಂದ!–ಈ ಅತಿರೇಕವನ್ನು ಟೀಕಿಸುತ್ತ ಒಮ್ಮೆ ಕೇಳುತ್ತಾನೆ: “ಅಪ್ಪಾ, ಇಂಥಾ ಸೋಂಬೇರಿಗಳನ್ನೆಲ್ಲ ಯಾಕೆ ಸಾಕುತ್ತಿದ್ದೀಯ ನೀನು? ” ಎಂದು. ಆಗ ಲೋಕಾನುಭವಿಯಾದ ಆ ತಂದೆ ಹೇಳುತ್ತಾನೆ: “ಮಗು, ಜನರು ಅನುಭವಿಸುವ ಕಷ್ಟಸಂಕಟಗಳನ್ನು ನೀನೇನು ಬಲ್ಲೆ ಹೇಳು? ಅದು ನಿನಗೆ ಅರ್ಥವಾದಾಗ ನೀನೂ ಅವರ ಅಸಹಾಯಕ ಪರಿಸ್ಥಿತಿಗಾಗಿ ಮರುಗುತ್ತೀ. ಇವರೆಲ್ಲ ತಮ್ಮ ಅಸಹನೀಯ ಕಷ್ಟಗಳನ್ನು ಒಂದಿಷ್ಟು ಮದ್ಯಪಾನದ ಮತ್ತಿನಲ್ಲಿ ತಾತ್ಕಾಲಿಕವಾಗಿಯಾದರೂ ಮರೆಯುವ ಹಾಗಿದ್ದರೆ ಯಾಕಾಗಬಾರದು?” ಇದಕ್ಕೇನೂ ಪ್ರತ್ಯುತ್ತರ ಕೊಡಲು ತೋಚದೆ ನರೇಂದ್ರ ಸುಮ್ಮನಾಗಿಬಿಟ್ಟ.

ವಿಶ್ವನಾಥ ದತ್ತ ಬಂಧುಗಳೊಂದಿಗೆ, ದೀನರೊಂದಿಗೆ ಕರುಣೆಯಿಂದಿರುತ್ತಿದ್ದ, ವಿಶ್ವಾಸದಿಂದಿರುತ್ತಿದ್ದ ಎನ್ನುವುದೇನೋ ನಿಜ. ಆದರೆ ಮೂಲತಃ ಅವನದು ಗಂಭೀರ ಮನೋಧರ್ಮ. ಯಾರೇ ಆಗಲಿ, ಅಸಂಬದ್ಧದ, ಅನ್ಯಾಯದ ಮಾತುಗಳನ್ನಾಡಿದರೆ ಅವನು ಅವುಗಳನ್ನೆಲ್ಲ ಖಡಕ್ಕಾಗಿ ಖಂಡಿಸಿಬಿಡುತ್ತಿದ್ದ.

ಅವನ ದೃಷ್ಟಿಕೋನ ವಿಶಾಲ; ತಿಳಿವಳಿಕೆ ಅಪಾರ. ಅವನು ಉದಾರ ಮನೋಭಾವದವನಾದ್ದರಿಂದ ಇತರ ಮತಗಳ ತತ್ತ್ವ-ಸಂದೇಶಗಳನ್ನು ತಿಳಿದುಕೊಳ್ಳಲು ಇಷ್ಟಪಡುತ್ತಿದ್ದ. ಬೈಬಲ್, ದಿವಾನ್-ಇ-ಹಫೀಜ್ ಈ ಗ್ರಂಥಗಳನ್ನು ಅವನು ಆಳವಾಗಿ ಅಧ್ಯಯನ ಮಾಡಿದ್ದ. ಬೈಬಲಿನ ಬಗ್ಗೆ ಅವನಿಗೆ ವಿಶೇಷ ಗೌರವಭಾವವಿತ್ತು. ಅವನು ಎಷ್ಟೋ ವೇಳೆ ಮಾತಿನ ಸಂದರ್ಭದಲ್ಲಿ ಬೈಬಲಿನ ಬೋಧನೆಗಳನ್ನು ಉದಾಹರಿಸುವುದಿತ್ತು.

ವಿಶ್ವನಾಥ ದತ್ತನ ಮನೋಭಾವ ಪುರೋಗಾಮಿ, ಪ್ರಗತಿಪರ. ಉತ್ತರ ಹಿಂದೂಸ್ಥಾನದ ಹಲವಾರು ಪ್ರದೇಶಗಳಲ್ಲಿ ಸಂಚರಿಸುತ್ತಿದ್ದಾಗ ಅವನು ಅಲ್ಲಿನ ಜನರ ನಾನಾ ಬಗೆಯ ಮೂಢನಂಬಿಕೆಗಳನ್ನೂ ಕಂದಾಚಾರಗಳನ್ನೂ ಕಂಡಿದ್ದ. ಅವುಗಳಿಂದ ಜನಜೀವನದ ಮೇಲಾಗಿದ್ದ ದುಷ್ಪರಿಣಾಮಗಳನ್ನೂ ಮನಗಂಡಿದ್ದ. ಆದ್ದರಿಂದ ಸಾಮಾನ್ಯರಿಗೆ ಸರಿಯಾದ ಜೀವನಕ್ರಮದ ಸ್ಪಷ್ಟಕಲ್ಪನೆ ಕೊಡುವುದಕ್ಕೋಸ್ಕರ ಹಿಂದೀ ಮತ್ತು ಬಂಗಾಳೀ ಭಾಷೆಗಳಲ್ಲಿ ‘ಶಿಷ್ಟಾಚಾರ ಪದ್ಧತಿ’ ಎಂಬ ಒಂದು ಗ್ರಂಥವನ್ನೇ ಬರೆದು ಪ್ರಚಾರ ಮಾಡಿದ. ಆ ದಿನಗಳಲ್ಲಿ ಬಂಗಾಳದಲ್ಲಿ ಬಾಲ್ಯ ವಿವಾಹ ತುಂಬ ಪ್ರಚಲಿತವಿತ್ತು. ಈ ಬಾಲ್ಯವಿವಾಹ ಪದ್ಧತಿಯ ಒಂದು ಪ್ರಧಾನ ದೋಷವೆಂದರೆ ಹುಡುಗಿ ಬಾಲವಿಧವೆಯಾಗುವ ಸಂಭವ. ಈ ಬಾಲವಿಧವೆ ಜೀವನದುದ್ದಕ್ಕೂ ಹೇಳಲಾರದ ಬವಣೆಗಳನ್ನು ಅನುಭವಿಸುವಂತಾಗುತ್ತದೆ. ಆದ್ದರಿಂದ, ಬಾಲ್ಯವಿವಾಹದ ಈ ಕ್ರೂರ ಪದ್ಧತಿಯನ್ನು ವಿಶ್ವನಾಥ ವಿರೋಧಿಸುತ್ತಿದ್ದ. ಮತ್ತು ಈ ಅನಿಷ್ಟ ಪದ್ಧತಿಯ ನಿರ್ಮೂಲನಕ್ಕಾಗಿ ಹಾಗೂ ವಿಧವೆಯರ ಪುನರ್ವಿವಾಹವನ್ನು ರೂಢಿಗೊಳಿಸುವುದಕ್ಕಾಗಿ ಈಶ್ವರಚಂದ್ರ ವಿದ್ಯಾಸಾಗರರು ನಡೆಸುತ್ತಿದ್ದ ಚಳವಳಿಯನ್ನು ಅವನು ಬೆಂಬಲಿಸುತ್ತಿದ್ದ. ಒಮ್ಮೆ ವಿಶ್ವನಾಥನ ಮನೆಯ ಬಳಿಯಲ್ಲೇ ಎರಡು ವಿಧವಾ ವಿವಾಹಗಳು ಜರುಗಿದಾಗ ಹಲವಾರು ಜನ ಅದನ್ನು ತೀವ್ರವಾಗಿ ಪ್ರತಿಭಟಿಸಿ ಗಲಭೆಯೆಬ್ಬಿಸಿದರು. ಆಗ ವಿಶ್ವನಾಥನೂ ಅವನ ಪತ್ನಿ ಭುವನೇಶ್ವರಿಯೂ ವಿವಾಹವನ್ನು ಬಲವಾಗಿ ಸಮರ್ಥಿಸಿದರು. ಅವನು ಹಿಂದೂಸಂಪ್ರದಾಯಗಳಲ್ಲಿ ಸಂಪೂರ್ಣ ಗೌರವಾದರಗಳನ್ನಿಟ್ಟಿದ್ದ. ಆದರೆ ಅದೇ ಸಂಪ್ರದಾಯಗಳಲ್ಲಿ ಕಂಡುಬರುವ ಸಂಕುಚಿತ ಮಡಿವಂತಿಕೆಗಳನ್ನು ದೂರದಲ್ಲೇ ಇಟ್ಟಿದ್ದ.

ಅದು ಭಾರತದ ಇತಿಹಾಸದಲ್ಲೊಂದು ಪರ್ವಕಾಲ. ೧೬ನೇ ಶತಮಾನದ ವೇಳೆಗೆ ನೂರಾರು ವರ್ಷಗಳ ಮುಸಲ್ಮಾನರ ಆಳ್ವಿಕೆಯ ಪರಿಣಾಮವಾಗಿ ಉತ್ತರ ಭಾರತದಲ್ಲಿ ಹಿಂದೂ-ಇಸ್ಲಾಂ ಸಂಸ್ಕೃತಿಗಳ ಮಿಶ್ರಣದಿಂದಾಗಿ ಹೊಸ ಬಗೆಯ ಸಂಸ್ಕೃತಿಯೊಂದು ಮೈದೋರಿತ್ತು. ಬಳಿಕ ಬಂದ ಬ್ರಿಟಿಷರ ಆಳ್ವಿಕೆಯ ಪರಿಣಾಮವಾಗಿ ಪಾಶ್ಚಾತ್ಯ ಸಂಸ್ಕೃತಿಯೂ ಹಿಂದೂ ಜನಜೀವನವನ್ನು ಪ್ರವೇಶಮಾಡಿತು. ಆಗಿನ ಕಾಲದ ವಿದ್ಯಾವಂತ ವರ್ಗದವರೆಲ್ಲ ಈ ಹೊಸ ಸಂಸ್ಕೃತಿಯ ಪ್ರಭಾವಕ್ಕೆ ಒಳಗಾದರು. ವಿಶ್ವನಾಥ ದತ್ತನೂ ಇದಕ್ಕೆ ಅಪವಾದವಾಗಿರಲಿಲ್ಲ. ಅವನ ಆಪ್ತವರ್ಗದಲ್ಲಿ ಮುಸಲ್ಮಾನರೂ ಆಂಗ್ಲರೂ ಅನೇಕರಿದ್ದರು. ಇವರೆಲ್ಲರ ನಿಕಟ ಸಂಪರ್ಕದಿಂದಾಗಿ ಅವನ ಸಾಮಾಜಿಕ ಹಾಗೂ ಕೌಟುಂಬಿಕ ಜೀವನದಲ್ಲಿ ಮುಸ್ಲಿಮರ ಮತ್ತು ಪಾಶ್ಚಾತ್ಯರ ಸಂಸ್ಕೃತಿ-ನಾಗರಿಕತೆಗಳು ಬಹಳಷ್ಟು ಪ್ರವೇಶ ಮಾಡಿದ್ದುವು. ಅವನ ಬಳಿಗೆ ಭಿಕ್ಷೆಗಾಗಿ ಹಿಂದೂ ಸಾಧುಗಳೂ ಬರುತ್ತಿದ್ದರು, ಮುಸಲ್ಮಾನ ಫಕೀರರೂ ಬರುತ್ತಿದ್ದರು. ಅವರೆಲ್ಲರ ಮೇಲೂ ವಿಶ್ವನಾಥ ಸಮಾನ ಮನ್ನಣೆ ತೋರುತ್ತಿದ್ದ. ದತ್ತ ವಂಶಸ್ಥರು ಶಾಕ್ತ ಸಂಪ್ರದಾಯದವರು; ಎಂದರೆ ದೇವಿಯ ಆರಾಧಕರು. ವಿಶ್ವನಾಥನು ಪರಮತಗಳ ವಿಷಯದಲ್ಲಿ ಉದಾರದೃಷ್ಟಿಯವನಾದರೂ ತನ್ನ ಮನೆತನದ ಧಾರ್ಮಿಕ ಸಂಪ್ರದಾಯಗಳನ್ನು ಬದಲಾಯಿಸುವ ಗೋಜಿಗೆ ಹೋಗಲಿಲ್ಲ.

