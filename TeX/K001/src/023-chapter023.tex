
\chapter{ಅನುಬಂಧಗಳು}

\section{೧. ಅಷ್ಟಸಿದ್ಧಿಗಳು (ಪುಟ ೧೧೧)}

ಅಷ್ಟಸಿದ್ಧಿಗಳು ಎಂದರೆ ಎಂಟು ಬಗೆಯ ಸಿದ್ಧಿಗಳು. ಅವು–ಅಣಿಮಾ (ಸೂಕ್ಷ್ಮರೂಪವನ್ನು ಧರಿಸುವ ಶಕ್ತಿ), ಮಹಿಮಾ (ಬೃಹತ್ ರೂಪವನ್ನು ಧರಿಸುವ ಶಕ್ತಿ), ಗರಿಮಾ (ಭಾರವಾಗುವಿಕೆ), ಲಘಿಮಾ (ಹಗುರವಾಗುವಿಕೆ), ಪ್ರಾಪ್ತಿ (ಪಡೆಯಲು ಸಾಧ್ಯವಿಲ್ಲದ್ದನ್ನು ಪಡೆಯುವಿಕೆ), ಪ್ರಾಕಾಮ್ಯ (ಸ್ವಲ್ಪವನ್ನು ವಿಪುಲವಾಗಿಸುವಿಕೆ), ಈಶಿತ್ವ (ಎಲ್ಲೆಲ್ಲೂ ಪ್ರಭಾವವನ್ನು ಬೀರುವಿಕೆ) ಮತ್ತು ವಶಿತ್ವ (ಇಂದ್ರಿಯನಿಗ್ರಹ). ಒಬ್ಬನು ಆಧ್ಯಾತ್ಮಿಕ ಸಾಧನೆಯಲ್ಲಿ ತೊಡಗಿ ಮುಂದುವರಿದಾಗ ವ್ಯಕ್ತ ವಾಗುವ ಶಕ್ತಿಗಳು ಇವು. ಆದರೆ ಸಾಕಷ್ಟು ಮನಶ್ಶುದ್ಧಿಯಿಲ್ಲದ ವ್ಯಕ್ತಿಗೆ ಈ ಸಿದ್ಧಿಗಳು ಪ್ರಾಪ್ತವಾ ದರೆ ಅದರಿಂದ ಅಪಾಯವೇ ಹೆಚ್ಚು. ಇದಕ್ಕೆ ಒಂದು ಉದಾಹರಣೆಯನ್ನು ಕೊಟ್ಟು ಹೇಳುವುದಾ ದರೆ, ಇದ್ದಕ್ಕಿದ್ದಂತೆ ಬಡವನೊಬ್ಬನಿಗೆ ಹತ್ತು ಕೋಟಿ ರೂಪಾಯಿಯಷ್ಟು ಹಣ ಸಿಕ್ಕಿಬಿಟ್ಟರೆ ಹೇಗಿರುತ್ತದೆಯೋ ಹಾಗಿರುತ್ತದೆ ಒಬ್ಬ ಸಾಧಕನ ಸ್ಥಿತಿ–ಇಂತಹ ಒಂದು ಸಿದ್ಧಿಯನ್ನು ಪಡೆದು ಕೊಂಡಾಗ; ಆ ಹತ್ತು ಕೋಟಿ ರೂಪಾಯಿ ಹಣವೂ ಉಪಯೋಗಿಸುತ್ತ ಹೋದಂತೆ ಹೇಗೆ ವ್ಯಯ ವಾಗಿಹೋಗುತ್ತದೆಯೋ ಹಾಗೆಯೇ ಈ ಸಿದ್ಧಿಯೂ ಕೂಡ ಬಳಸುತ್ತ ಹೋದಂತೆಲ್ಲ ನಶಿಸಿ ಹೋಗುತ್ತದೆ. ಈಗ, ಈ ಹತ್ತು ಕೋಟಿ ರೂಪಾಯಿ ಸಿಕ್ಕಿದವನು ಅಶಿಕ್ಷಿತ-ಅನೀತಿವಂತನಾಗಿದ್ದರೆ ಆ ಹಣವನ್ನು ಕುಡಿತ, ಜೂಜು, ವ್ಯಭಿಚಾರ ಮುಂತಾದ ಅಕ್ರಮ ರೀತಿಗಳಲ್ಲಿ ಬಳಸಿ ದುರ್ವ್ಯಯ ಮಾಡಿ ಬಿಡುತ್ತಾನೆ. ಆದರೆ ಅದೇ ಮನುಷ್ಯ ಸುಶಿಕ್ಷಿತ ಹಾಗೂ ನೀತಿವಂತನಾಗಿದ್ದಲ್ಲಿ ಅವನು ಆ ಹಣವನ್ನು ಸಾಹಿತ್ಯ, ಕಲೆ, ದಾನ-ಧರ್ಮಾದಿ ಸತ್ಕಾರ್ಯಗಳಿಗಾಗಿ ಬಳಸುತ್ತಾನೆ. ಅದೇ ರೀತಿಯಲ್ಲಿ ಮನುಷ್ಯನಿಗೆ ಭಗವಂತನ ಸಾಕ್ಷಾತ್ಕಾರವಾಗಿ, ಯುಕ್ತಾಯುಕ್ತತೆಯ ಪ್ರಜ್ಞೆ ಮೂಡುವ ಮೊದಲೇ ಈ ಸಿದ್ಧಿಗಳು ಲಭಿಸಿಬಿಟ್ಟರೆ, ಆತ ಅವುಗಳನ್ನು ದುರ್ವಿನಿಯೋಗ ಮಾಡಿ ದಾರಿ ತಪ್ಪುತ್ತಾನೆ, ದುಃಖಕ್ಕೀಡಾಗುತ್ತಾನೆ. ಆದರೆ ಭಗವಂತನ ಸಾಕ್ಷಾತ್ಕಾರದಿಂದಾಗಿ ವಿವೇಕೋದಯವಾದ ಮೇಲೆ ಆತ ಅದೇ ಸಿದ್ಧಿಗಳನ್ನು ಇಚ್ಛಾನುಸಾರ ಲೋಕಕಲ್ಯಾಣಕ್ಕಾಗಿ ಬಳಸಿ ವಿಶ್ವಶಾಂತಿಗೆ ಕಾರಣನಾಗು ತ್ತಾನೆ. ಆದ್ದರಿಂದಲೇ ನಾವು ನರೇಂದ್ರನ ಆ ಉತ್ತರ ಎಷ್ಟು ವಿವೇಕಪೂರ್ಣವಾಗಿದೆ, ಅವನ ವಿಚಾರಶಕ್ತಿ ಎಷ್ಟು ಪ್ರೌಢವಾಗಿದೆ ಎಂಬುದನ್ನು ಕಾಣಬಹುದಾಗಿದೆ.


\section{೨. ಗುಡುಗುಡಿ (ಪುಟ ೧೪೨)}

ದಕ್ಷಿಣಭಾರತೀಯರಾದ ನಮಗೆ ಸಾಧುಗಳು ಧೂಮಪಾನ ಮಾಡುತ್ತಾರೆಂದು ಕೇಳಿದರೆ ಆಶ್ಚರ್ಯವಾಗುವುದು ಸಹಜವೇ. ಆದರೆ ಉತ್ತರಭಾರತದಲ್ಲಿ ಸಾಧುಗಳು ಹುಕ್ಕಾ ಅಥವಾ ಗುಡುಗುಡಿಯನ್ನು ಸೇದುವುದು ಸರ್ವೇಸಾಮಾನ್ಯ. ಬಂಗಾಳದಲ್ಲಂತೂ ಮನೆಮನೆಗಳಲ್ಲೂ ಹುಕ್ಕಾ ಇರುತ್ತದೆ. ಅತಿಥಿಗಳಿಗೆ ಹುಕ್ಕಾ ತಯಾರಿಸಿ ಕೊಡುವುದು ಅಲ್ಲಿ ಅತಿಥಿ ಸತ್ಕಾರದ ಅವಿಭಾಜ್ಯ ಅಂಗ. ನಮ್ಮಲ್ಲಿ ಕಾಫಿ ಕುಡಿಯುವಷ್ಟೇ ಸಹಜ–ಅವರಲ್ಲಿ ಹುಕ್ಕಾ ಸೇದುವುದು. ಆದರೆ ಸಾಧುಗಳು ಹುಕ್ಕಾ ಮೊದಲಾದ ಭೋಗಗಳನ್ನೆಲ್ಲ ಅನುಭವಿಸಬಹುದೇ ಎಂಬ ಪ್ರಶ್ನೆ ಯೇಳುತ್ತದೆ. ಅದಕ್ಕೆ ಶ್ರೀರಾಮಕೃಷ್ಣರೇ ಹೇಳಿದ್ದುಂಟು–“ತ್ಯಾಗ ಮಾಡಬೇಕಾದದ್ದು ಈರುಳ್ಳಿ ಯನ್ನೂ ಅಲ್ಲ, ಎಲೆ ಅಡಿಕೆಯನ್ನೂ ಅಲ್ಲ; ತ್ಯಾಗಮಾಡಬೇಕಾದದ್ದು ಕಾಮಿನೀಕಾಂಚನವನ್ನು. ಕಾಮಕಾಂಚನತ್ಯಾಗವೇ ನಿಜವಾದ ತ್ಯಾಗ.”


\section{೩. ಧುನಿ (ಪುಟ ೧೭೯)}

ಸಂನ್ಯಾಸಿಗಳು ಧ್ಯಾನಕ್ಕೆ ಕುಳಿತುಕೊಳ್ಳುವಾಗ ತಮ್ಮ ಮುಂದೆ ಹಚ್ಚಿಕೊಳ್ಳುವ ಬೆಂಕಿಗೆ ‘ಧುನಿ’ ಎಂದು ಹೆಸರು. ಈ ಧುನಿಯನ್ನು ಹಚ್ಚಿಕೊಳ್ಳುವುದರಲ್ಲಿ ಎರಡು ಉದ್ದೇಶಗಳಿವೆ. ಮೊದಲನೆಯ ದಾಗಿ, ಉತ್ತರಭಾರತದಲ್ಲಿ, ಮುಖ್ಯವಾಗಿ ಹಿಮಾಲಯದಲ್ಲಿ ವಾಸಮಾಡುವ ಸಾಧು-ಸಂನ್ಯಾಸಿ ಗಳು ಕೊರೆಯುವ ಚಳಿಯಿಂದ ದೇಹವನ್ನು ರಕ್ಷಿಸಿಕೊಂಡು ಧ್ಯಾನ ಮಾಡಲು ಈ ಧುನಿಯನ್ನು ಹಚ್ಚಿಕೊಳ್ಳುತ್ತಾರೆ. ಆದರೆ ದಕ್ಷಿಣಭಾರತದಲ್ಲಿ ಅಂತಹ ಚಳಿ ಇಲ್ಲವಾದ್ದರಿಂದ, ಆ ಪದ್ಧತಿ ಕಂಡುಬರುವುದಿಲ್ಲ.

ಸಾಧಕನು ತನ್ನ ಮನಸ್ಸಿನ ಎಲ್ಲ ಕಾಮನೆಗಳನ್ನು, ಆಸೆ-ಆಕಾಂಕ್ಷೆಗಳನ್ನು, ಸಂಸ್ಕಾರರಾಶಿಯನ್ನು ಆ ಉರಿಯುವ ಅಗ್ನಿಯಲ್ಲಿ ಹಾಕಿ ಭಸ್ಮ ಮಾಡಿಬಿಟ್ಟೆ ಎಂಬ ಭಾವನೆಯನ್ನು ತಂದುಕೊಳ್ಳಲು ಈ ಧುನಿಯು ಸಹಾಯಮಾಡುತ್ತದೆ. ಇದೇ ಅದರ ಎರಡನೇ ಉದ್ದೇಶ. ಆ ಉರಿಯುವ ಧುನಿ ಒಂದು ಸಂಕೇತ. ಹೀಗೆ ತನ್ನೆಲ್ಲ ಆಸೆ-ಆಕಾಂಕ್ಷೆಗಳನ್ನು ಸುಟ್ಟುಬಿಟ್ಟೆನೆಂದು ಭಾವಿಸಿದಾಗ ಮನಸ್ಸು ನಿರ್ಮಲವಾಗಿ ಶಾಂತಗೊಳ್ಳುತ್ತದೆ. ಶಾಂತಗೊಂಡ ಮನಸ್ಸಿನಿಂದ ಧ್ಯಾನ ಮಾಡುವುದು ಸುಲಭ ವಾಗುತ್ತದೆ. ಚಳಿಯಿಂದ ದೇಹ ನಡುಗುತ್ತಿರುವಾಗಲೂ ಧ್ಯಾನ ಸಾಧ್ಯವಾಗುವುದಿಲ್ಲ; ಆಸೆಗಳಿಂದ ಮನಸ್ಸು ನಡುಗುತ್ತಿರುವಾಗಲೂ ಧ್ಯಾನ ಸಾಧ್ಯವಾಗುವುದಿಲ್ಲ. ಆದ್ದರಿಂದ ಸಂನ್ಯಾಸಿಗಳು ಧುನಿ ಉರಿಸಿಕೊಂಡು ಚಳಿಯನ್ನು ಓಡಿಸಿ, ಆಸೆಗಳನ್ನು ಸುಟ್ಟು, ಶಾಂತ ಮನಸ್ಸಿನಿಂದ ಧ್ಯಾನಮಗ್ನರಾಗಿ ಬಿಡಲು ಪ್ರಯತ್ನಿಸುತ್ತಾರೆ.


\section{೪. ಮಧುಕರೀ ಭಿಕ್ಷೆ (ಪುಟ ೧೮೫)}

‘ಮಧುಕರೀ’ ಎಂದರೆ ಜೇನ್ನೊಣ. ಜೇನ್ನೊಣ ಸಂಗ್ರಹಿಸಿದಂತಹ ಜೇನುತುಪ್ಪ ಎಷ್ಟು ಶುದ್ಧವಾಗಿರುತ್ತದೆಯೋ, ಸಂನ್ಯಾಸಿಗಳು ಮತ್ತು ವಟುಗಳು ನಾಲ್ಕಾರು ಮನೆಗಳಿಂದ ಸಂಗ್ರಹಿ ಸಿದ ಭಿಕ್ಷಾನ್ನವೂ ಅಷ್ಟೇ ಶುದ್ಧ ಎಂಬುದು ‘ಮಧುಕರೀ ಭಿಕ್ಷೆ’ ಅಥವಾ ‘ಮಧುಕರವೃತ್ತಿ’ ಎಂಬುದರ ಒಂದು ಅರ್ಥ. ಜೇನ್ನೊಣ ಹೇಗೆ ಹೂಗಳನ್ನು ಸ್ವಲ್ಪವೂ ನಲುಗಿಸದೆ, ನೋವು ಮಾಡದೆ ಮಧುವನ್ನು ಮಾತ್ರ ಹೀರುತ್ತದೆಯೋ ಹಾಗೆಯೇ ಸಂನ್ಯಾಸಿಗಳು ಮನೆಯವರ ಮನಸ್ಸನ್ನು ನೋಯಿಸದೆ ಭಿಕ್ಷೆಯನ್ನು ಸ್ವೀಕರಿಸುತ್ತಾರೆ. ಭಿಕ್ಷೆಗಾಗಿ ಬಂದ ಸಂನ್ಯಾಸಿಗಳು ‘ಇಷ್ಟೇ ಕೊಡಿ, ಅಷ್ಟೇ ಕೊಡಿ’ ಎಂದು ಕೇಳುವುದಿಲ್ಲ. ಅಥವಾ ‘ಇಂಥದೇ ಕೊಡಿ, ಅಂಥದೇ ಕೊಡಿ’ ಎಂದು ಕೇಳುವುದಿಲ್ಲ, ಕೇಳುವಂತಿಲ್ಲ. ಗೃಹಸ್ಥರು ಸಂತೋಷದಿಂದ ಎಷ್ಟು ಕೊಟ್ಟರೆ ಅಷ್ಟನ್ನು, ಏನು ಕೊಟ್ಟರೆ ಅದನ್ನು ಸ್ವೀಕರಿಸುತ್ತಾರೆ. ಸಂನ್ಯಾಸಿಗಳು ಭಿಕ್ಷೆ ಕೇಳುವಾಗ ನಾಲ್ಕು ಪ್ರಧಾನ ನಿಯಮಗಳನ್ನು ಅನುಸರಿಸಬೇಕು. (೧) ‘ನಾರಾಯಣ ಹರಿ!’ ಎಂದು ಭಗವನ್ನಾಮಸ್ಮರಣೆ ಮಾಡುತ್ತ ಕೇಳಬೇಕು. (೨) ಒಂದು ದಿನದ ಭಿಕ್ಷೆಗಾಗಿ ಏಳು ಮನೆಗಳವರೆಗೆ ಮಾತ್ರ ಹೋಗ ಬಹುದು. ಮೊದಲ ಮನೆಯಲ್ಲೇ ಅಂದಿಗಾಗುವಷ್ಟು ಭಿಕ್ಷೆ ಸಿಕ್ಕಿಬಿಟ್ಟರೆ, ಎರಡನೆಯ ಮನೆಗೆ ಹೋಗಬಾರದು. ಏಕೆಂದರೆ ಭಿಕ್ಷೆಯ ಹೆಸರಿನಲ್ಲಿ ಅಗತ್ಯಕ್ಕಿಂತ ಹೆಚ್ಚನ್ನು ಬೇಡಬಾರದು. ಹಾಗೆ ಬೇಡಿದರೆ ಅದು ‘ಅಪರಿಗ್ರಹ’ ನಿಯಮದ ಉಲ್ಲಂಘನೆಯಾಗುತ್ತದೆ; ಪರಿಗ್ರಹದ ದೋಷ ಬರುತ್ತದೆ. ಅಲ್ಲದೆ, ಬೇಕಾದುದಕ್ಕಿಂತ ಹೆಚ್ಚನ್ನು ಪಡೆದುಕೊಳ್ಳುವುದು ಲೋಭವಾಗುತ್ತದೆ. ಆದರೆ ಮೊದಲನೆಯ ಮನೆಯವರು ಭಿಕ್ಷೆ ಹಾಕದೆ ಹೋದರೆ, ಅಥವಾ ಹಾಕಿದ ಭಿಕ್ಷೆ ಸಾಲದೆ ಹೋದರೆ ಆಗ ಎರಡನೆಯ ಮನೆಗೆ ಹೋಗಬಹುದು. ಹೀಗೆ ಅವರು ಅಂದಿಗೆ ಸಾಕಾಗುವಷ್ಟು ಭಿಕ್ಷೆಯನ್ನು ಸಂಪಾದಿಸಲು ಏಳು ಮನೆಗಳವರೆಗೆ ಹೋಗಬಹುದು. ಆದರೆ ಈ ಏಳು ಮನೆಗಳಿಂದ ಸಂಗ್ರಹಿ ಸಿಯೂ ಸಾಕಷ್ಟು ಸಿಗದೆ ಹೋದರೆ ಎಂಟನೆಯ ಮನೆಗೆ ಹೋಗಬಾರದು. ಅಥವಾ ಒಂದು ವೇಳೆ ಈ ಏಳು ಮನೆಗಳವರೂ ಭಿಕ್ಷೆಯನ್ನೇ ಹಾಕದಿದ್ದರೂ ಎಂಟನೆಯ ಮನೆಗೆ ಹೋಗುವಂತಿಲ್ಲ. ಅಂದು ಭಗವಂತ ತಮಗೆ ಭಿಕ್ಷೆಯ ವ್ಯವಸ್ಥೆ ಮಾಡಿಲ್ಲವೆಂದು ತಿಳಿದುಕೊಂಡು ಉಪವಾಸವಿರ ಬೇಕು; ಉಪವಾಸವಿದ್ದುಕೊಂಡೇ ಭಗವದ್ಧ್ಯಾನ ಪರರಾಗಬೇಕು. (೩) ಭಿಕ್ಷೆ ಹಾಕದಿರುವವರ ಮೇಲೆ ಬೇಸರಿಸಿಕೊಳ್ಳಬಾರದು, ಅಥವಾ ಅವರನ್ನು ಬೈಯಲೂಬಾರದು. ಬದಲಾಗಿ, ಅವರಿಗೂ ಒಳ್ಳೆಯದಾಗಲಿ ಎಂದು ಮನಸ್ಸಿನಲ್ಲೇ ಪ್ರಾರ್ಥಿಸಿಕೊಳ್ಳಬೇಕು. (೪) ಸಂನ್ಯಾಸಿಗಳು ಭಿಕ್ಷೆಗಾಗಿ ಶ್ರೀಮಂತರ ಮನೆಗಳನ್ನೇ ಹುಡುಕಿಕೊಂಡು ಹೋಗಬಾರದು. ಭಗವಂತ ಯಾರ ಮನೆಯ ಮುಂದೆ ನಿಲ್ಲಿಸುತ್ತಾನೋ ಅಲ್ಲಿಗೆ ಹೋಗಬೇಕು. ಆದರೆ ಅದು ಮಾಂಸಾಹಾರ ಮೊದಲಾದ ವರ್ಜಿತ ಪದಾರ್ಥಗಳನ್ನು ತಿನ್ನುವವರ ಮನೆಯಾಗಿದ್ದು, ಅವರು ಅದನ್ನೇ ಕೊಡಲು ಬಂದರೆ ಬೇಡ ಎನ್ನಬಹುದು. ಸಂನ್ಯಾಸಿಗಳಿಗೆ ‘ಬೇಡ’ ಎನ್ನುವ ಅಧಿಕಾರವಿದೆ; ಆದರೆ ‘ಬೇಕು,’ ‘ಇಂಥದೇ ಬೇಕು’ ಎನ್ನುವ ಅಧಿಕಾರವಿಲ್ಲ.

ಶ್ರೀರಾಮಕೃಷ್ಣ ಮಹಾಸಂಘದ ಸದಸ್ಯರು ಬೇಲೂರು ಮಠದಲ್ಲಿ ಸಂನ್ಯಾಸದೀಕ್ಷೆ ಪಡೆದು ಕೊಂಡ ಮೇಲೆ ಮೂರುದಿನಗಳ ಮಟ್ಟಿಗೆ–ಶಾಸ್ತ್ರಕ್ಕೆ!–ಭಿಕ್ಷೆಗೆ ಹೋಗುತ್ತಾರೆ. (ಈ ಲೇಖಕ ಭಿಕ್ಷೆಗೆ ಹೋದಾಗ ಅವನನ್ನು ಯಾರೂ ಬೈಯಲೂ ಇಲ್ಲ. ಭಿಕ್ಷೆ ಹಾಕದೆ ಕಳಿಸಲೂ ಇಲ್ಲ!) ಪಡೆದ ಭಿಕ್ಷೆಯನ್ನು ಮಠಕ್ಕೆ ತಂದು ರಾಮಕೃಷ್ಣ ಮಠ ಮತ್ತು ಮಿಷನ್ನಿನ ಮಹಾಧ್ಯಕ್ಷರಿಗೆ ಅದರ ಅಗ್ರಭಾಗವನ್ನು ಸಮರ್ಪಿಸಿ, ಬಳಿಕ ಗಂಗೆಯ ದಡದ ಮರದ ಬುಡದಲ್ಲಿ ಕುಳಿತು ಅದನ್ನು ಊಟ ಮಾಡುತ್ತಾರೆ. ಸಂನ್ಯಾಸಿಗಳು ಅನಿಕೇತನರು–ಎಂದರೆ ಮನೆಯಿಲ್ಲದವರು. ಅದರ ಸಂಕೇತವಾಗಿ ಮರದ ಬುಡದಲ್ಲಿ ಊಟಮಾಡುವುದು. ಸಂನ್ಯಾಸಿಗಳ ಜೀವನಕ್ರಮ ಹೇಗೆಂದರೆ, ‘ಕರತಲ ಭಿಕ್ಷಾ, ತರುತಲ ವಾಸ’. ಆದರೆ ಈಗ ಸಂಘದ ಸದಸ್ಯರೆಲ್ಲ ಮಠಗಳಲ್ಲಿ ವಾಸವಾಗಿರುವಂತಹ ವ್ಯವಸ್ಥೆಯನ್ನು ಸ್ವಾಮಿ ವಿವೇಕಾನಂದರು ಮಾಡಿಹೋಗಿದ್ದಾರೆ. ಅಲ್ಲದೆ ಶ್ರೀಮಾತೆಯವರಿಗೂ ತಮ್ಮ ಸಂನ್ಯಾಸೀಪುತ್ರರು ಗೊತ್ತುಗುರಿಯಿಲ್ಲದೆ ತಿರುಗಿಕೊಂಡಿರುವುದು ಇಷ್ಟವಿರಲಿಲ್ಲ. ಭಿಕ್ಷೆಯ ಹೆಸರಿನಲ್ಲಿ ಸುತ್ತಾಡಿ ಸಮಯನಷ್ಟ-ಶಕ್ತಿನಷ್ಟ ಮಾಡಿಕೊಳ್ಳುವುದಕ್ಕಿಂತ ಮಠದಲ್ಲಿ ವಾಸವಾಗಿದ್ದು ಕೊಂಡು, ಸಾಧನೆಯನ್ನು ತೀವ್ರಗೊಳಿಸಿ ಆಧ್ಯಾತ್ಮಿಕ ಶಕ್ತಿಯನ್ನು ಬೆಳೆಸಿಕೊಂಡರೆ, ಅದರಿಂದ ಸಾಧಕರಿಗೂ ಶ್ರೇಯಸ್ಸು, ಸಮಾಜಕ್ಕೂ ಶ್ರೇಯಸ್ಸು.


\section{೫. ಶ್ರೀರಾಮಕೃಷ್ಣಾವತಾರ (ಪುಟ ೨೦೭)}

ತಾವು ಅವತಾರಪುರುಷರೆಂದು ಶ್ರೀರಾಮಕೃಷ್ಣರೇ ಅಷ್ಟು ಸ್ಪಷ್ಟವಾಗಿ ಹೇಳಿದ್ದರೂ, ಜನ ಇಂದಿಗೂ ಕೇಳುತ್ತಾರೆ–ಅವರು ಅವತಾರಪುರುಷರಾಗಿರಲು ಹೇಗೆ ಸಾಧ್ಯ? ಎಂದು. ಜನ ಸಾಮಾನ್ಯರು ಅವರನ್ನು ಕೇವಲ ಸಂತ ಅಥವಾ ಭಕ್ತನೇ ಹೊರತು ಅವತಾರಪುರುಷನಲ್ಲ ಎಂದು ಭಾವಿಸಿದರೆ ಅದರಲ್ಲೇನೂ ಆಶ್ಚರ್ಯವಿಲ್ಲ. ಕೆಲವರ ಸಮಸ್ಯೆಯೇನೆಂದರೆ–‘ಅವತಾರಗಳು ಕೇವಲ ಹತ್ತಲ್ಲವೆ? ಎಂದಮೇಲೆ ಶ್ರೀರಾಮಕೃಷ್ಣರು ಈ ಹತ್ತರೊಳಗೆ ಬರುವುದಿಲ್ಲವಲ್ಲ? ಅಲ್ಲದೆ, ದುಷ್ಟಶಿಕ್ಷಣ-ಶಿಷ್ಟರಕ್ಷಣೆಗಳನ್ನು ಮಾಡುವುದಕ್ಕಾಗಿ ಭಗವಂತ ಅವತರಿಸಿ ಬರುವುದು ಯುಗದ ಅಂತ್ಯದಲ್ಲಿ ಮಾತ್ರವೇ ಅಲ್ಲವೆ?’ ಎಂದು. ಆದರೆ ಈ ವಾದದ ಪ್ರಕಾರ, ಕೇವಲ ಯುಗಾಂತ್ಯದಲ್ಲಿ ಮಾತ್ರವೇ ಭಗವಂತ ಅವತಾರ ತಾಳುವನೆಂದಾದರೆ, ನಾಲ್ಕು ಯುಗಗಳಿಗೆ ನಾಲ್ಕು ಅವತಾರಗಳೇ ಹೊರತು ಹತ್ತಲ್ಲ. ಆದರೆ ಪ್ರಧಾನವಾಗಿ ಅವತಾರಗಳು ಹತ್ತು ಎಂದು ತೀರ್ಮಾನಿಸಿರುವುದರಿಂದ ಯುಗಾಂತ್ಯದಲ್ಲಿ ಮಾತ್ರವಲ್ಲದೆ ಯುಗಗಳ ಮಧ್ಯದಲ್ಲೂ ಭಗವಂತ ತನ್ನಿಚ್ಛೆಯಂತೆ ಅವತರಿಸಿ ಬರಬಲ್ಲ ಎಂದಂತಾಯಿತು.

ಭಗವದ್ಗೀತೆಯಲ್ಲಿ ಶ್ರೀಕೃಷ್ಣ, ‘ಧರ್ಮಸಂಸ್ಥಾಪನಾರ್ಥಾಯ ಸಂಭವಾಮಿ ಯುಗೇ ಯುಗೇ’ ಎಂದು ಹೇಳಿದುದನ್ನು ಕೆಲವರು ‘ಒಂದೊಂದು ಯುಗಕ್ಕೆ ಒಂದೊಂದು ಅವತಾರ’ ಎಂದು ಅರ್ಥೈಸಬಹುದು. ಆದರೆ ಅದು ಹಾಗಲ್ಲ. ಅವನೇ ಹೇಳುವಂತೆ–

\begin{verse}
ಯದಾ ಯದಾ ಹಿ ಧರ್ಮಸ್ಯ ಗ್ಲಾನಿರ್ಭವತಿ ಭಾರತ\\ಅಭ್ಯುತ್ಥಾನಮಧರ್ಮಸ್ಯ ತದಾತ್ಮಾನಂ ಸೃಜಾಮ್ಯಹಮ್ ॥
\end{verse}

\noindent

ಯಾವಯಾವಾಗ ಧರ್ಮ ಅವನತಿ ಹೊಂದುತ್ತದೆಯೋ, ಅಧರ್ಮ ತಲೆಯೆತ್ತಿ ನಿಲ್ಲುತ್ತದೆಯೋ ಆಗ ಅಧರ್ಮವನ್ನು ಮರ್ದಿಸಲು ಭಗವಂತ ಅವತರಿಸುತ್ತಾನೆ. ಒಂದು ಯುಗವೆಂದರೆ ಅಸಂಖ್ಯಾತ ವರ್ಷಗಳ ಒಂದು ಅವಧಿ. ಇಷ್ಟೊಂದು ದೀರ್ಘಾವಧಿಯ ಒಂದೊಂದು ಯುಗ ದಲ್ಲೂ ಎಷ್ಟೆಷ್ಟು ಸಲ ಧರ್ಮದ ಅವನತಿಯಾಗಿ ಅಧರ್ಮವು ಉಚ್ಛ್ರಾಯಕ್ಕೆ ಬರುತ್ತದೆಯೋ ಹೇಳಬಲ್ಲವರಾರು? ಆದ್ದರಿಂದ, ಆಗಾಗ ಅವನತಿಗಿಳಿಯುವ ಧರ್ಮವನ್ನು ಮೇಲೆತ್ತಲು ಭಗವಂತ ಆಗಾಗ ಬರುತ್ತಲೇ ಇರಬೇಕಾಗುತ್ತದೆ. ಎಂದಮೇಲೆ ಈ ಹತ್ತು ಅವತಾರಗಳೂ ಸಾಕಾಗದೆ ಹಲವಾರು ಅವತಾರಗಳೇ ಬೇಕಾಗುತ್ತವೆ. ಅಲ್ಲದೆ, ಶ್ರೀಮದ್ಭಾಗವತದಲ್ಲಿ ಹೇಳು ವಂತೆ, ‘ಅವತಾರಾ ಹಿ ಅಸಂಖ್ಯೇಯಾಃ ’–ಭಗವಂತನ ಅವತಾರಗಳು ಅಸಂಖ್ಯಾತ.


\section{೬. ಸಂನ್ಯಾಸವೋ ಸಂಸಾರವೋ? (ಪುಟ ೨೧೫)}

ಲೋಕಕ್ಕೆ ಶ್ರೀರಾಮಕೃಷ್ಣರು ನಿಜಕ್ಕೂ ಬೋಧಿಸಿದ್ದೇನು? ಅವರು ತೋರಿಸಿಕೊಟ್ಟ ಮಾರ್ಗ ಯಾವುದು–ಸಂನ್ಯಾಸಜೀವನವೆ ಅಥವಾ ಸಂಸಾರಜೀವನವೆ? ಸ್ವಯಂ ಶ್ರೀರಾಮಕೃಷ್ಣರ ಯುವಕ ಶಿಷ್ಯವರ್ಗ ಹಾಗೂ ಗೃಹಸ್ಥ ಶಿಷ್ಯವರ್ಗದಲ್ಲಿ ಇದೊಂದು ವಿವಾದದ ವಿಷಯವಾಗಿ ಪರಿಣಮಿಸಿತ್ತು. ನಿಜಕ್ಕೂ ಶ್ರೀರಾಮಕೃಷ್ಣರ ಜೀವನ-ಸಂದೇಶಗಳೆರಡೂ ಸ್ಫಟಿಕದಷ್ಟು ಸ್ಪಷ್ಟ. ಅವನ್ನು ಯಾರು ಬೇಕಾದರೂ ಅರ್ಥಮಾಡಿಕೊಳ್ಳಬಹುದು. ಅಲ್ಲಿ ಬಳಸುಮಾತುಗಳಾಗಲಿ ಪಾಂಡಿತ್ಯದ ವಾಕ್ಯಗಳಾಗಲಿ ಇಲ್ಲ. ಒಬ್ಬರು ಅವುಗಳನ್ನು ಓದಿದರೆ ಹೇಗೆ ಅರ್ಥವಾಗುತ್ತದೆಯೋ ಇನ್ನೊಬ್ಬರು ಓದಿದಾಗಲೂ ಹಾಗೇ ಅರ್ಥವಾಗಬೇಕು–ಹಾಗಿದೆ ಅವುಗಳ ನಿಖರತೆ. ಹೀಗಿದ್ದರೂ ಕೂಡ, ಅವರ ಸನ್ನಿಧಾನದಲ್ಲೇ ಕುಳಿತು ಹಲವಾರು ವರ್ಷಗಳ ಕಾಲ ಅವುಗಳನ್ನೆಲ್ಲ ಆಲಿಸಿದ, ಮತ್ತು ಅವರ ಜೀವನವನ್ನೇ ಕಣ್ಣಾರೆ ಕಂಡ ಭಕ್ತರು ಹಾಗೂ ಶಿಷ್ಯರಲ್ಲಿ ಅವರು ಏನು ಬೋಧಿಸಿದರೆಂಬುದರ ಬಗ್ಗೆ ಭಿನ್ನಾಭಿಪ್ರಾಯ ಹುಟ್ಟಿಕೊಂಡಿದೆ! ಯುವಶಿಷ್ಯರು ಸಂನ್ಯಾಸ ಸ್ವೀಕರಿಸಿ ಮಠಸ್ಥಾಪನೆ ಮಾಡುವುದೆಂದು ನಿಶ್ಚಯಿಸಿದರೆ, ಗೃಹಸ್ಥ ಭಕ್ತರು ‘ಶ್ರೀರಾಮಕೃಷ್ಣರು ಸಂನ್ಯಾಸದ ಆದರ್ಶವನ್ನು ಬೋಧಿಸಲೇ ಇಲ್ಲ’ ಎನ್ನುತ್ತಿದ್ದಾರೆ. ಗೃಹಸ್ಥರು ಆ ರೀತಿ ಹೇಳಿದ್ದು ಸಹಜವೇ ಎನ್ನಬೇಕು. ಏಕೆಂದರೆ, ಶ್ರೀರಾಮಕೃಷ್ಣರು ಅವರಿಗೆ ಬೋಧಿಸಿದ್ದು ಸಂಸಾರದಲ್ಲಿದ್ದು ಕೊಂಡೇ ಭವಸಾಗರವನ್ನು ದಾಟುವ ಮಾರ್ಗವನ್ನು. ಅಲ್ಲದೆ ಸ್ವಯಂ ಶ್ರೀರಾಮಕೃಷ್ಣರು– ಲೋಕದೃಷ್ಟಿಗೆ–ಸಂಸಾರಸ್ಥರೇ ಅಲ್ಲವೆ? ಸ್ವತಃ ಸಂಸಾರಿಯಾದ ಅವರು ಸಂನ್ಯಾಸಜೀವನವನ್ನು ಬೋಧಿಸಲಿಲ್ಲವೆಂದು ಗೃಹಸ್ಥರು ತಿಳಿದರು. ಆದರೆ ಶ್ರೀರಾಮಕೃಷ್ಣರ ಇಂಗಿತ-ಇಚ್ಛೆ ಸುಸ್ಪಷ್ಟ ವಾಗಿಯೇ ಇದೆ. ಅವರು ಯಾವಾಗಲೂ ಬೋಧಿಸಿದ್ದು ತ್ಯಾಗವನ್ನೇ. ಅದರಲ್ಲೂ ತಮ್ಮ ಯುವ ಶಿಷ್ಯರಿಗಂತೂ ತ್ಯಾಗದ ಮಹತ್ವವನ್ನು ಮತ್ತೆಮತ್ತೆ ಒತ್ತಿಹೇಳಿದ್ದಾರೆ. ಅವರಿಗೆ ತಮ್ಮ ಕೈಯಾರೆ ಕಾಷಾಯವಸ್ತ್ರವನ್ನು ನೀಡಿದ್ದಾರೆ. ಅಲ್ಲದೆ ಒಮ್ಮೆ ಸಂನ್ಯಾಸಧರ್ಮಕ್ಕನುಸಾರವಾಗಿ ಮನೆಮನೆಗೆ ಹೋಗಿ ಮಧುಕರೀ ಭಿಕ್ಷೆಯನ್ನು ಪಡೆದುತರುವಂತೆಯೂ ಮಾಡಿದ್ದಾರೆ. ನರೇಂದ್ರನಿಗಂತೂ ಈ ವಿಚಾರವಾಗಿ ವಿಶೇಷ ಸೂಚನೆಗಳನ್ನು ನೀಡಿದ್ದಾರೆ. ಆದ್ದರಿಂದ ಈ ಯುವಕರೆಲ್ಲ ಸಂನ್ಯಾಸಿಗಳಾಗಿ ಸಂಘಜೀವನ ನಡೆಸಬೇಕು ಎಂಬುದೇ ಅವರ ಇಚ್ಛೆಯಾಗಿತ್ತೆಂಬುದರಲ್ಲಿ ಸಂಶಯವಿಲ್ಲ. ಆದರೆ ಅವರು ತಮ್ಮ ಬಳಿಗೆ ಬಂದಬಂದವರೆಲ್ಲರನ್ನೂ ಸಂನ್ಯಾಸಜೀವನವನ್ನು ಕೈಗೊಳ್ಳುವಂತೆ ಪ್ರೇರೇ ಪಿಸಲಿಲ್ಲ ಎಂಬುದು ಗಮನಾರ್ಹ. ಯಾವ ಯುವಕರಲ್ಲಿ ಅಂತಸ್ಸತ್ತ್ವದಿಂದೊಡಗೂಡಿದ ತ್ಯಾಗ ಬುದ್ಧಿಯಿತ್ತೋ ಅಂಥವರನ್ನು ಮಾತ್ರ ಸಂನ್ಯಾಸಜೀವನಕ್ಕೆ ಅಣಿಗೊಳಿಸಿದರು. ಧರ್ಮದ ಪರಮೋಚ್ಚ ತತ್ತ್ವಗಳನ್ನು ಅವುಗಳ ಶುದ್ಧ ರೂಪದಲ್ಲಿ ಬೋಧಿಸಲು ಇಂತಹ ಮಹಾತ್ಯಾಗ ದಿಂದ ಪ್ರಜ್ವಲಿಸುತ್ತಿರುವ ಸಂನ್ಯಾಸಿಗಳೇ ಬೇಕು. ಆದರೆ ಇನ್ನೊಂದು ವಿಷಯವೂ ಅಷ್ಟೇ ಮುಖ್ಯ. ಏನೆಂದರೆ, ಈ ಸಂನ್ಯಾಸಿಗಳಿಗೆ ಕಾಲಕಾಲಕ್ಕೆ ನಿರಾತಂಕವಾಗಿ ಒಂದಿಷ್ಟು ಭಿಕ್ಷೆ ಸಿಗುವಂತಾಗಬೇಕಾ ದರೆ ಸದ್ಗೃಹಸ್ಥರ ನೆರವು ಬೇಕು! 

ಇದು ಗೃಹಸ್ಥರಿಗೂ ಯುವಶಿಷ್ಯರಿಗೂ ನಡುವೆ ಉಂಟಾದ ಭಿನ್ನಾಭಿಪ್ರಾಯವಾದರೆ, ಈ ಯುವಶಿಷ್ಯರಲ್ಲೇ ಮತ್ತೊಂದು ವಿವಾದ ಹುಟ್ಟಿಕೊಂಡಿದೆ. ಒಂದು ಗುಂಪಿನವರು ಹೇಳುತ್ತಾರೆ: ‘ಶ್ರೀರಾಮಕೃಷ್ಣರು ಬೋಧಿಸಿದ್ದು, ಮೊದಲು ಭಗವಂತನ ಸಾಕ್ಷಾತ್ಕಾರವನ್ನು ಸಿದ್ಧಿಸಿಕೊಂಡು ಬಳಿಕ ಜನಸೇವೆಯನ್ನು ಕೈಗೊಳ್ಳಬೇಕು ಅಂತ.’ ಮತ್ತೆ ಕೆಲವರು, ಮುಖ್ಯವಾಗಿ ನರೇಂದ್ರ, ಹೇಳುತ್ತಾರೆ–‘ಮೊದಲು ಜನಸೇವೆ; ಈ ಜನಸೇವೆಯ ಮೂಲಕವೇ ಸಾಕ್ಷಾತ್ಕಾರ’ ಎಂದು. ಇವೆರಡೂ ಪ್ರಬಲ ವಾದಗಳೇ. ಇವೆರಡು ವಾದಗಳಿಗೂ ಶ್ರೀರಾಮಕೃಷ್ಣರ ಜೀವನ ಹಾಗೂ ಬೋಧನೆಗಳಲ್ಲಿ ಆಧಾರ ಸಿಗುತ್ತದೆ. ಹಾಗಾದರೆ ಈ ಎರಡು ವಾದಗಳಲ್ಲಿ ಯಾವುದು ಸರಿ ಎಂದರೆ, ಎರಡೂ ಸರಿ! ಅದು ಹೇಗೆ ಸಾಧ್ಯ? ಶ್ರೀರಾಮಕೃಷ್ಣರು, ಮೊದಲು ಸಾಕ್ಷಾತ್ಕಾರ ಮಾಡಿಕೊಂಡು ಸೇವೆಗಿಳಿಯುವ ಮಾತನ್ನೂ ಹೇಳಿದ್ದಾರೆ; ಮೊದಲು ಲೋಕಸೇವೆ ಮಾಡಿ ತನ್ಮೂಲಕ ಸಾಕ್ಷಾತ್ಕಾರ ಮಾಡಿಕೊಳ್ಳುವ ಮಾತನ್ನೂ ಹೇಳಿದ್ದಾರೆ. ಆದರೆ ಇಲ್ಲಿ ನಾವು ಅರಿತು ಕೊಳ್ಳಬೇಕಾದ ವಿಷಯ ಇದು–ಶ್ರೀರಾಮಕೃಷ್ಣರು ಯಾವ ಮಾತನ್ನು ಯಾರಿಗೆ ಹೇಳಿದರು ಎಂಬುದು. ಎಲ್ಲ ಮಾತುಗಳೂ ಎಲ್ಲರಿಗೂ ಅನ್ವಯವಾಗುವುದಿಲ್ಲ. ಮನುಷ್ಯರೆಲ್ಲರೂ ಮೇಲ್ನೋಟಕ್ಕೆ ಒಂದೇ ರೀತಿಯಾಗಿ ಕಂಡುಬಂದರೂ, ಸಾಮರ್ಥ್ಯಗಳಲ್ಲಿ ಅಂತರವಿದ್ದೇ ಇದೆ. ಸತ್ತ್ವರಜಸ್ತಮೋಗುಣಗಳಿಗನುಸಾರವಾಗಿ ಜನರ ಸಾಮರ್ಥ್ಯಗಳು, ಸ್ವಭಾವಗಳು ಭಿನ್ನ ಭಿನ್ನ. ಭಗವಂತನ ಸಾಕ್ಷಾತ್ಕಾರ ಮಾಡಿಕೊಳ್ಳುವ ಹಂಬಲವಿರುವ ಪಾರಮಾರ್ಥಿಕರನ್ನು ಮುಖ್ಯವಾಗಿ ಎರಡು ವರ್ಗಗಳಾಗಿ ವಿಂಗಡಿಸಬಹುದು–ಸಮರ್ಥರು, ಸಾಮಾನ್ಯರು ಎಂದು. ಸಾತ್ತ್ವಿಕ, ರಾಜಸಿಕ ಹಾಗೂ ತಾಮಸಿಕ ಸ್ವಭಾವಗಳ ಜನರ ಗುಣಲಕ್ಷಣಗಳನ್ನು ಗೀತೆಯ ‘ಗುಣತ್ರಯ ವಿಭಾಗ ಯೋಗ’ದಲ್ಲಿ ಶ್ರೀಕೃಷ್ಣನು ಸ್ಪಷ್ಟವಾಗಿ ವಿವರಿಸಿದ್ದಾನೆ. ಯಾರಲ್ಲಿ ತಮೋಗುಣ ಲೇಶವೂ ಇಲ್ಲವೋ ಮತ್ತು ಸತ್ತ್ವಗುಣ ಜಾಗೃತವಾಗಿದೆಯೋ ಅವರನ್ನು ಸಮರ್ಥರೆನ್ನಬಹುದು. ಅವರು ನೇರವಾಗಿ ಭಗವಂತನ ಸಾಕ್ಷಾತ್ಕಾರಕ್ಕಾಗಿ ಪ್ರಯತ್ನಿಸಬಹುದು. ಅವರು ಜಪ, ಧ್ಯಾನಗಳೇ ಮೊದಲಾದ ಆಧ್ಯಾತ್ಮಿಕ ಸಾಧನೆಗಳನ್ನು ನಿರಾತಂಕವಾಗಿ ಮಾಡಬಲ್ಲರು; ಅಲ್ಲದೆ, ಅವರ ಸಾಧನೆಯೂ ಬಹು ಬೇಗ ಫಲಿಸುತ್ತದೆ. ರಜೋಗುಣ-ತಮೋಗುಣ ಅಧಿಕವಾಗಿರುವವರನ್ನು ಸಾಮಾನ್ಯರೆಂದು ವರ್ಗೀಕರಿಸಬಹುದು. ತಮೋಗುಣದವರಲ್ಲಿ ಆಲಸ್ಯ ಅಧಿಕ, ರಜೋಗುಣ ದವರಲ್ಲಿ ಚಟುವಟಿಕೆ ಅಧಿಕ. ಈ ಆಲಸ್ಯವನ್ನು ಕರಗಿಸಿ, ಮೈತುಂಬ ಕೆಲಸಕಾರ್ಯಗಳನ್ನು ಹಚ್ಚಿಕೊಳ್ಳುವ ಚಪಲ ಮನೋಭಾವವನ್ನು ಉದಾತ್ತೀಕರಿಸಿ, ಸಮಾಧಾನಚಿತ್ತದ ಸಾತ್ತ್ವಿಕ ಗುಣಕ್ಕೆ ಏರಿಸಿಕೊಳ್ಳುವಲ್ಲಿ ನಿಷ್ಕಾಮಕರ್ಮ ಅತ್ಯಂತ ಸಹಾಯಕಾರಿ. ಅಲ್ಲದೆ, ಯಾರೂ ಒಂದು ಕ್ಷಣ ವಾದರೂ ಯಾವ ಕರ್ಮವನ್ನೂ ಮಾಡದೆ ‘ಸುಮ್ಮನೆ’ ಇರಲಾರರು. ಹೆಚ್ಚಿನವರಿಗೆ, ದಿನಗಳಲ್ಲಿ ಎಷ್ಟು ಗಂಟೆಗಳ–ಅಥವಾ ನಿಮಿಷಗಳ–ಕಾಲವನ್ನು ಆಧ್ಯಾತ್ಮಿಕ ಸಾಧನೆಯಲ್ಲೇ ಕಳೆಯಲು ಸಾಧ್ಯ? ಆದ್ದರಿಂದ ಸಾಧಕರೆನ್ನಿಸಿಕೊಂಡವರು ಆಧ್ಯಾತ್ಮಿಕ ಸಾಧನೆಯ ಹೆಸರಿನಲ್ಲಿ ಸುಮ್ಮನೆ ಚಡಪಡಿಸುತ್ತಿರುವುದು ತೀರಾ ಸಾಮಾನ್ಯ. ಇಂತಹವರಿಗೆ, ತಮ್ಮ ರಜೋಗುಣ-ತಮೋಗುಣ ಗಳನ್ನು ಜಯಿಸಲು ಸುಲಭೋಪಾಯವೆಂದರೆ ನಿಷ್ಕಾಮಕರ್ಮ. ಇದನ್ನು ಶ್ರೀರಾಮಕೃಷ್ಣರು ಸ್ಪಷ್ಟವಾಗಿ ತಿಳಿಸಿದ ಸಂದರ್ಭಗಳಿವೆ. ಆದರೆ ಕೆಲವು ಯುವಶಿಷ್ಯರು ಶ್ರೀರಾಮಕೃಷ್ಣರ ಭಗವದಾನಂದವನ್ನು ಕಂಡು, ಹಾಗೂ ಅವರು ‘ಭಗವತ್ಸಾಕ್ಷಾತ್ಕಾರವೇ ಮನುಷ್ಯನ ಪ್ರಧಾನ ಕರ್ತವ್ಯ’ ಎಂದು ಹೇಳಿದ್ದನ್ನು ಕೇಳಿ, ಈಗ ತಾವೂ ಹೇಳುತ್ತಿದ್ದಾರೆ–‘ಮೊದಲು ಸಾಕ್ಷಾತ್ಕಾರ, ಬಳಿಕ ಸೇವೆಯ ಮಾತು’ ಎಂದು. ಆದರೆ ಬಹುತೇಕ ಜನರ ಅಸಾಮರ್ಥ್ಯವನ್ನು ಕಂಡು, ಹಾಗೂ ಶ್ರೀರಾಮಕೃಷ್ಣರು ಬೋಧಿಸಿದ ಜನಸೇವೆಯ ಪ್ರಾಶಸ್ತ್ಯವನ್ನು ಮನಗಂಡು ನರೇಂದ್ರ ಹೇಳುತ್ತಿ ದ್ದಾನೆ–‘ಜನಸೇವೆಯೇ ಮೊದಲು, ಬಳಿಕ ಭಗವತ್ಸಾಕ್ಷಾತ್ಕಾರದ ಮಾತು’ ಎಂದು. ನರೇಂದ್ರನ ಮಾತೇ ಗೆದ್ದಿತು ಎಂಬುದನ್ನು ಹೇಳಬೇಕಾಗಿಯೇ ಇಲ್ಲ. ಆದರೆ ಅದು ಅವನ ಇಚ್ಛೆಯಿಂದಲ್ಲ. ಅದು ಶ್ರೀರಾಮಕೃಷ್ಣರ ಇಚ್ಛೆಯಾದ್ದರಿಂದ; ಜಗನ್ಮಾತೆಯ ಇಚ್ಛೆಯಾದ್ದರಿಂದ.


\section{೭. ಒಂದು ಧರ್ಮಸೂಕ್ಷ್ಮ (ಪುಟ \textit{೨೬೬)}}

‘ಶ್ರೀರಾಮಕೃಷ್ಣರ ಶರೀರವನ್ನು ದಹನ ಮಾಡಿದ್ದು ಒಂದು ಆಕ್ಷೇಪಣೀಯ ಕಾರ್ಯವೇ ಸರಿ’ ಎಂದು ಬರೆಯುತ್ತಿದ್ದಾರೆ ಸ್ವಾಮೀಜಿ. ಇದಕ್ಕೆ ಕಾರಣವೇನೆಂದರೆ ಶ್ರೀರಾಮಕೃಷ್ಣರು ಸಂನ್ಯಾಸಿ ಗಳು ಎಂಬುದು. ಸಂನ್ಯಾಸಿಗಳು ನಿರಗ್ನಿಗಳಾದ್ದರಿಂದ–ಎಂದರೆ ಅಗ್ನಿಕಾರ್ಯಾದಿ ಕರ್ಮಗಳನ್ನು ತ್ಯಾಜ್ಯ ಮಾಡಿದವರಾದ್ದರಿಂದ–ಅವರ ಶರೀರಗಳನ್ನು ದಹನ ಮಾಡಬಾರದೆಂದು ಶಾಸ್ತ್ರವಿಧಿ. ಆದ್ದರಿಂದ, ಹಿಂದೂ ಸಂಪ್ರದಾಯಗಳ ಪ್ರಕಾರ, ಸಂನ್ಯಾಸಿಗಳ ಶರೀರವನ್ನು ಒಂದೋ ಭೂಗತ ಮಾಡಬೇಕು, ಇಲ್ಲವೆ ಭಾರವಾದ ಕಲ್ಲುಕಟ್ಟಿ ನದಿಯಲ್ಲಿ ವಿಸರ್ಜನೆ ಮಾಡಬೇಕು. ‘ಆದರೆ ಶ್ರೀರಾಮಕೃಷ್ಣರು ಸಂನ್ಯಾಸಿಯಲ್ಲವಲ್ಲ? ಅವರು ಪತ್ನಿಯೊಡಗೂಡಿದ ಗೃಹಸ್ಥರಲ್ಲವೆ? ಆದ್ದ ರಿಂದ ಅವರ ಶರೀರವನ್ನು ದಹನ ಮಾಡಿದ್ದೇ ಸರಿಯಲ್ಲವೆ?’ ಎಂದೂ ಕೇಳಬಹುದು. ಆದರೆ ಶ್ರೀರಾಮಕೃಷ್ಣರು ಸಂನ್ಯಾಸಿಗಳೆಂಬುದೇ ಸತ್ಯ. ಅವರು (ವಿವಾಹವಾದ ಮೇಲೆ) ತೋತಾಪುರಿ ಯಿಂದ ವಿಧಿವತ್ತಾಗಿ ಸಂನ್ಯಾಸದೀಕ್ಷೆಯನ್ನು ಪಡೆದವರು. ಸಂನ್ಯಾಸವನ್ನು ಸ್ವೀಕರಿಸಿ, ಅದ್ವೈತ ಸಾಧನೆ ಮಾಡಿ, ಅದರಲ್ಲಿ ಪರಿಪೂರ್ಣ ಸಿದ್ಧಿಯನ್ನು ಪಡೆದುಕೊಂಡವರು. ಅವರು ಸಪತ್ನೀಕರಾಗಿ ದ್ದರೂ, ಅವರ ಹಾಗೂ ಶ್ರೀಶಾರದಾದೇವಿಯವರ ನಡುವಣ ಸಂಬಂಧವು ಗೃಹಸ್ಥಧರ್ಮಕ್ಕನು ಸಾರವಾದದ್ದಲ್ಲ, ಅದೊಂದು ಅಭೂತಪೂರ್ವ ಅಲೌಕಿಕ ಸಂಬಂಧ; ಅವರದು ಲೀಲಾನಾಟಕ– ಇದು ಅವರ ಜೀವನವನ್ನು ಅಧ್ಯಯಿಸಿದವರೆಲ್ಲ ಕಂಡುಕೊಳ್ಳುವ ಅಂಶ. ಆದ್ದರಿಂದ ಅವರು ಸಂನ್ಯಾಸಧರ್ಮಕ್ಕೆ ವ್ಯತಿರಿಕ್ತವಾಗಿ ಜೀವಿಸಿದವರಲ್ಲ. ಹೀಗಿದ್ದರೂ ಅವರು ಸಂನ್ಯಾಸಿಗಳ ಉಡುಗೆ ಯಾದ ತ್ಯಾಗದ ಪ್ರತೀಕವಾದ ಕಾಷಾಯವಸ್ತ್ರವನ್ನು ಧರಿಸಲಿಲ್ಲವಲ್ಲ? ಇದಕ್ಕೆ ಎರಡು ಕಾರಣ ಗಳನ್ನು ಊಹಿಸಬಹುದು. ಮೊದಲನೆಯದಾಗಿ, ಅವರು ಸಂನ್ಯಾಸದೀಕ್ಷೆಯನ್ನು ಸ್ವೀಕರಿಸಿದ್ದವರೇ ಆದರೂ, ಅವರು ಕೇವಲ ಸಂನ್ಯಾಸಿಯಲ್ಲ–ಸಂನ್ಯಾಸವನ್ನೂ ಮೀರಿದ ಪರಮಹಂಸಾವಸ್ಥೆ ಯನ್ನು ತಲುಪಿದವರು, ಪರಮಹಂಸರು. ಈ ಅತ್ಯುನ್ನತ ಸ್ಥಿತಿಯಲ್ಲಿ ಒಬ್ಬನ ಆಹಾರ-ಬಟ್ಟೆಗಳೇ ಮೊದಲಾದ ಬಾಹ್ಯ ನಿಯಮಗಳಿಗೆ ಸಂಬಂಧಿಸಿದ ವಿಧಿನಿಷೇಧಗಳೆಲ್ಲ ಕಳಚಿಹೋಗಿರುತ್ತವೆ. ಆದ್ದರಿಂದ ಅವರು ಕಾಷಾಯವಸ್ತ್ರವನ್ನಾದರೂ ಧರಿಸಬಹುದು. ಬಿಳಿಯ ವಸ್ತ್ರವನ್ನಾದರೂ ಧರಿಸಬಹುದು ಅಥವಾ ಇನ್ನೆಂತಹ ಬಟ್ಟೆಯನ್ನಾದರೂ ಧರಿಸಬಹುದು–ಅದರಲ್ಲಿ ತಾರತಮ್ಯ ವಿಲ್ಲ. ಆದ್ದರಿಂದಲೇ ಅವರಲ್ಲಿ ಇಂತಹ ಬಾಹ್ಯಾಚರಣೆಗಳಿಗೆ ಸಂಬಂಧಿಸಿದುದೇನನ್ನೂ ನಾವು ಕಾಣುವಂತಿಲ್ಲ. ಎರಡನೆಯದಾಗಿ, ಶ್ರೀರಾಮಕೃಷ್ಣರು ಅವತಾರಪುರುಷರು. ಅವರು ಈ ಬಾರಿ ಜಗತ್ತಿನ ಎಲ್ಲ ಬಗೆಯ ಆಧ್ಯಾತ್ಮಿಕ ಸಾಧಕರಿಗೆ ಮಾರ್ಗದರ್ಶನ ನೀಡುವುದಕ್ಕಾಗಿ ಬಂದಿದ್ದಾರೆ. ಅವರ ಜೀವನ-ಸಂದೇಶಗಳಲ್ಲಿ ಹಿಂದೂ-ಮುಸಲ್ಮಾನ-ಕ್ರೈಸ್ತರಿಗೂ ಮಾರ್ಗದರ್ಶನವಿದೆ; ಜ್ಞಾನ- ಭಕ್ತಿ-ಕರ್ಮ-ರಾಜಯೋಗಿಗಳಿಗೂ ಮಾರ್ಗದರ್ಶನವಿದೆ; ದ್ವೈತ-ಅದ್ವೈತ-ವಿಶಿಷ್ಟಾದ್ವೈತ ಮಾರ್ಗಾವಲಂಬಿಗಳಿಗೂ ಮಾರ್ಗದರ್ಶನವಿದೆ; ಅಂತೆಯೇ, ಬ್ರಹ್ಮಚರ್ಯ-ಗೃಹಸ್ಥ-ಸಂನ್ಯಾಸಾ ಶ್ರಮಗಳವರಿಗೂ ಮಾರ್ಗದರ್ಶನವಿದೆ. ಗೃಹಸ್ಥರಾದರೂ ಕೂಡ ಪವಿತ್ರ ಜೀವನವನ್ನು ನಡೆಸಲು ಸಾಧ್ಯವಿದೆ ಎಂಬುದನ್ನು ತೋರಿಸುವುದಕ್ಕಾಗಿ, ಸಂಸಾರದಲ್ಲಿ ಇದ್ದೂ ಇಲ್ಲದವರಂತೆ ಹೇಗಿರ ಬಹುದು ಎಂಬುದನ್ನು ತೋರಿಸುವುದಕ್ಕಾಗಿ, ಶ್ರೀರಾಮಕೃಷ್ಣರು ಸಂನ್ಯಾಸಿಗಳಾದರೂ ಬಳಿಕ ಕಾಷಾಯವಸ್ತ್ರವನ್ನು ತ್ಯಜಿಸಿ ಬಿಳಿಯ ಬಟ್ಟೆಯಲ್ಲಿದ್ದರು. ಆದರೆ ಅವರು ತೋರಿಸಿಕೊಟ್ಟ ಈ ಆದರ್ಶವನ್ನು ತದ್ವತ್ತಾಗಿ ಪಾಲಿಸಲು ಎಷ್ಟು ಜನರಿಂದ ಸಾಧ್ಯ ಎಂಬ ವಿಚಾರ ಬೇರೆ. ಆದರೆ ಅವರು ಹೀಗೆ ಬಿಳಿಯಬಟ್ಟೆಯಲ್ಲಿದ್ದರೂ ವಿರಜಾಹೋಮದ ಮೂಲಕ ಪ್ರೈಷಮಂತ್ರವನ್ನು ಸ್ವೀಕಾರ ಮಾಡಿದವರು; ಅವರು ಹೀಗೆ ಬಿಳಿಯ ಬಟ್ಟೆಯಲ್ಲಿದ್ದರೂ ಅಪ್ಪಟ ಸಂನ್ಯಾಸಿಗಳಿ ಗಿಂತಲೂ ಕಟುತರ ಸಂನ್ಯಾಸಿಗಳು.

ಆದರೆ, “ಶ್ರೀರಾಮಕೃಷ್ಣರ ಶರೀರವನ್ನು ದಹನಮಾಡಿದ್ದು ಆಕ್ಷೇಪಣೀಯ ಕಾರ್ಯ ಎಂಬುದ ರಲ್ಲಿ ಸಂದೇಹವೇ ಇಲ್ಲ” ಎಂಬ ಈ ಮಾತನ್ನು ಸ್ವಾಮೀಜಿ ಪ್ರಮದದಾಸ ಮಿತ್ರರಿಗೆ ಈಗೇಕೆ ಬರೆಯಬೇಕಾಗಿತ್ತು? ಇದಕ್ಕೆ ಕಾರಣವಿಷ್ಟೇ: ಪ್ರಮದದಾಸ ಮಿತ್ರರು ಪ್ರತಿಭಾನ್ವಿತ ಪಂಡಿತರು, ಶಾಸ್ತ್ರಗಳನ್ನು ಚೆನ್ನಾಗಿ ಅರಿತವರು, ಮತ್ತು ವಾರಾಣಸಿಯಲ್ಲೇ ನೆಲೆಸಿರುವವರು. ಆದ್ದರಿಂದ ಅವರಿಗೆ ಹಿಂದೂ ಸಂಪ್ರದಾಯಗಳ ಎಲ್ಲ ವಿವರಗಳೂ ಚಿರಪರಿಚಿತ. ಅವರಿಗೆ ಶ್ರೀರಾಮಕೃಷ್ಣರು ಸಂನ್ಯಾಸ ಸ್ವೀಕರಿಸಿದವರು ಎಂಬ ವಿಚಾರ ತಿಳಿದಿತ್ತು. ಈಗ ಶ್ರೀರಾಮಕೃಷ್ಣರ ಸ್ಮಾರಕ ದೇವಾ ಲಯದ ಪ್ರಸ್ತಾಪವೆತ್ತಿದಾಗ ಅವರು ಆಕ್ಷೇಪಣೆಯೆತ್ತಬಹುದು–‘ಸಂನ್ಯಾಸಿಯ ಶರೀರವನ್ನು ಏಕೆ ದಹನ ಮಾಡಿದಿರಿ?’ಎಂದು. ಅಥವಾ ಅದಾಗಲೇ ಅವರಿಬ್ಬರ ನಡುವೆ ಈ ವಿಷಯದ ಪ್ರಸ್ತಾಪ ಬಂದಿದ್ದಿರಲೂ ಬಹುದು. ಆದ್ದರಿಂದಲೇ ಸ್ವಾಮೀಜಿ, ‘ಇದೊಂದು ಆಕ್ಷೇಪಣೀಯ ಕಾರ್ಯ’ ಎಂದು ಒಪ್ಪಿಕೊಳ್ಳುತ್ತಿದ್ದಾರೆ.

ಈಗ ಇನ್ನೊಂದು ಪ್ರಶ್ನೆ ಏಳಬಹುದು–ಸ್ವಾಮೀಜಿಗೇ ಈ ವಿಷಯ ತಿಳಿದಿದ್ದರೆ ಶ್ರೀರಾಮ ಕೃಷ್ಣರ ಶರೀರವನ್ನು ದಹನಮಾಡಿದ್ದೇಕೆ? ಪವಿತ್ರ ಗಂಗೆಯಲ್ಲಿ ವಿಸರ್ಜನೆ ಮಾಡಬಹುದಾಗಿ ತ್ತಲ್ಲ? ಎಂದು. ಆದರೆ ನರೇಂದ್ರನಿಗಾಗಲಿ, ಶ್ರೀರಾಮಕೃಷ್ಣರ ಭಕ್ತಾದಿಗಳಿಗಾಗಲಿ ಸಂನ್ಯಾಸಿಯ ಶರೀರವನ್ನು ದಹನಮಾಡಬಾರದೆಂಬ ವಿಷಯ ತಿಳಿದಿರಲಿಲ್ಲ, ಅಥವಾ ಆಗ ಹೊಳೆಯಲಿಲ್ಲ ಎನ್ನಬೇಕು. ಏಕೆಂದರೆ ತಂತ್ರಸಾಧನಾ ಪ್ರಧಾನವಾದ ಆ ಬಂಗಾಳದಲ್ಲಿ ಸಂನ್ಯಾಸದ ಸಂಪ್ರ ದಾಯವೇ ಇರಲಿಲ್ಲ. ಆದ್ದರಿಂದಲೇ ಶ್ರೀರಾಮಕೃಷ್ಣರ ಸಂನ್ಯಾಸೀಶರೀರಕ್ಕೆ ದಹನಕಾರ್ಯ ನಡೆದುಹೋಗಿಬಿಟ್ಟಿತು. ಈಗ ನರೇಂದ್ರನು ಸ್ವಾಮಿ ವಿವೇಕಾನಂದರಾಗಿ ಪರಿವ್ರಜನ ಕೈಗೊಂಡು, ಬಂಗಾಳದಿಂದ ಹೊರಗೆ ಬಂದಾಗ ತಿಳಿದುಬಂದಿದೆ–ತಾವು ಅಂದು ಮಾಡಿದ ಕಾರ್ಯ ಆಕ್ಷೇಪಣೀಯ ಎಂದು.

ಈಗ ಮತ್ತೊಂದು ಪ್ರಶ್ನೆ, ಸಂನ್ಯಾಸಿಯ ಶರೀರವನ್ನು ದಹನ ಮಾಡುವುದು ಸರಿಯಲ್ಲ ಎನ್ನುವುದಾದರೆ ಶ್ರೀರಾಮಕೃಷ್ಣ ಮಠಗಳಲ್ಲಿ ಇಂದಿಗೂ ಸಂನ್ಯಾಸಿಗಳ ಪಾರ್ಥಿವ ಶರೀರಗಳನ್ನು ದಹನ ಮಾಡುತ್ತಿದ್ದಾರಲ್ಲ ಏಕೆ? ಶಾಸ್ತ್ರಪ್ರಕಾರ ನೆಲ ಸಮಾಧಿ ಮಾಡಬೇಕಲ್ಲವೆ? ಇದಕ್ಕೆ ಉತ್ತರವಿಷ್ಟೆ: ಸಂನ್ಯಾಸಿಯ ಶರೀರವನ್ನು ನೆಲಸಮಾಧಿ ಮಾಡಿದರೆ ಅದರ ಮೇಲೊಂದು ಬೃಂದಾವನವನ್ನು ಕಟ್ಟಬೇಕಾಗುತ್ತದೆ. ಹೀಗೆ ಬೃಂದಾವನಗಳ ಸಂಖ್ಯೆ ಹೆಚ್ಚುತ್ತಲೇ ಹೋಗುತ್ತದೆ. ಕೊನೆಗೆ ಅದಕ್ಕೆ ಎಕರೆಗಟ್ಟಲೆ ಜಾಗ ಬೇಕಾಗುತ್ತದೆ. ಕ್ರೈಸ್ತ-ಮುಸಲ್ಮಾನರ ದೊಡ್ಡ ದೊಡ್ಡ ಸ್ಮಶಾನಗಳನ್ನು ಕಾಣುವುದಿಲ್ಲವೆ? ಈಗ ಹೇಗಿದ್ದರೂ ಸಂನ್ಯಾಸಿಗಳ ಶರೀರಗಳನ್ನು ದಹನ ಮಾಡುವ ಸಂಪ್ರದಾಯವನ್ನು ಸ್ವಾಮೀಜೀಯವರೇ ಪ್ರಾರಂಭಿಸಿಯಾಗಿದೆಯಲ್ಲ! ಅಲ್ಲದೆ ಇದರಿಂದ ಅನುಕೂಲವೂ ಇದೆಯಲ್ಲ? ಆದ್ದರಿಂದ ರಾಮಕೃಷ್ಣ ಮಠಗಳಲ್ಲಿ ಸಂನ್ಯಾಸಿಗಳ ಶರೀರಗಳನ್ನು ದಹನ ಮಾಡುವ ಸಂಪ್ರದಾಯವೇ ಬೆಳೆದುಕೊಂಡು ಬಂದಿದೆ.


\section{೮. ಸಂನ್ಯಾಸಿಗೀತೆ}

ಇದು ತ್ಯಾಗದ ಮತ್ತು ಸಂನ್ಯಾಸಜೀವನದ ಆದರ್ಶಗಳನ್ನು ಉಜ್ವಲವಾಗಿ ಬಣ್ಣಿಸುವ ಸುಂದರ ಕವನ. ಸ್ವಾಮಿ ವಿವೇಕಾನಂದರು ಅಮೆರಿಕೆಯಲ್ಲಿದ್ದಾಗ ಸ್ಫೂರ್ತಿಯ ಘಳಿಗೆಯೊಂದರಲ್ಲಿ (ಒಬ್ಬಾತನ ಟೀಕೆಯ ನುಡಿಗಳಿಗೆ ಉತ್ತರವಾಗಿ) ಇದನ್ನು ರಚಿಸಿದರು. ಆ ಸಂದರ್ಭವನ್ನು ‘ವಿಶ್ವವಿಜೇತ ವಿವೇಕಾನಂದ’ ಗ್ರಂಥದಲ್ಲಿ (ಪುಟ ೪೩೬) ಕಾಣಬಹುದು.

‘ಸಂನ್ಯಾಸಿಗೀತೆ’ಯ ತಾತ್ಪರ್ಯವನ್ನು ಕೆಲವು ವಿವರಣೆಗಳೊಂದಿಗೆ ಇಲ್ಲಿ ಕೊಡಲಾಗಿದೆ:

(೧) ಓ ಸಂನ್ಯಾಸಿಯೆ! ಜಾಗೃತನಾಗು. ತ್ಯಾಗಿ(ಚಾಗಿ)ಗಳ ಹಾಡಾದ ಓಂಕಾರವನ್ನು ಉದ್ಘೋಷಿಸು. ನಿದ್ರಾವಸ್ಥೆಯಲ್ಲಿರುವ ನಮ್ಮೀ ಭಾರತರಾಷ್ಟ್ರವನ್ನು ಆ ಹಾಡಿನಿಂದ ಎಚ್ಚರ ಗೊಳಿಸು. ತ್ಯಾಗಿಗಳ ಆ ಹಾಡೆಂಬುದು–ಓಂಕಾರವೆಂಬುದು–ಭೋಗದ ನಡುವೆ ಉದ್ಭವಿಸು ವಂಥದಲ್ಲ; ಎಲ್ಲಿ ಕಾಮ-ಮೋಹಗಳು ಸುಳಿಯವೋ ಮತ್ತು (ಮೇಣ್) ಎಲ್ಲಿ ಜೀವಾತ್ಮನು ಹಣ-ಹೆಸರು-ಕೀರ್ತಿಗಳೆಂಬ ಭ್ರಾಂತಿಯಿಂದ ಪಾರಾಗಿ (‘ಎಲ್ಲಿ ಜೀವವು ಕೀರ್ತಿಕಾಂಚನವೆಂಬು ವಾಸೆಗಳಿಂದ ಜನಿಸುವ ಭ್ರಾಂತಿಯ ತಿಳಿಯದೋ’) ನಿತ್ಯ(ನಿಚ್ಚ)ವಾದ ಶಾಂತಿಯನ್ನು, ಎಂದರೆ ಆತ್ಮಾನಂದವನ್ನು ಹೊಂದುವನೋ, ಎಲ್ಲಿ ಸತ್ಯದ (ನನ್ನಿಯ) ಅರಿವಿನ ಆನಂದವೆಂಬ ಚಿಲುಮೆ ಚಿಮ್ಮುತ್ತಿರುವುದೋ, ತೃಪ್ತಿಯೆಂಬ ಝರಿ(ತೊರೆ) ನಿರಂತರ ಹರಿಯುತ್ತಿರುವುದೋ ಅಂಥಲ್ಲಿ ಹುಟ್ಟಿಬರುವುದು ಈ ಓಂ ತತ್-ಸತ್ ಎಂಬ ಗಾನ. (ತತ್ = ಅದು, ಎಂದರೆ ಆತ್ಮ, ಪರಮಾತ್ಮ; ಸತ್ = ಎಂದೆಂದಿಗೂ ಇರುವಂಥದು) ಅದನ್ನು ನೀನು ಮೊಳಗು.

(೨) ನಿನ್ನನ್ನು ಬಂಧಿಸಿರುವ (ಈ ನಶ್ವರ ಜಗತ್ತಿನ ಸಂಬಂಧಗಳೆಂಬ) ಮಾಯಾಪಾಶವನ್ನು ಕತ್ತರಿಸಿಕೊ. ಹೆಣ್ಣು-ಹೊನ್ನು-ಮಣ್ಣುಗಳ ತೊಡಕನ್ನು ನಿವಾರಿಸಿಕೊ. (ಕಗ್ಗ = ಕೆಲಸಕ್ಕೆ ಬಾರದ್ದು. ಕಗ್ಗ ಎಂಬುದಕ್ಕೆ ಹರಟೆ ಎಂಬರ್ಥವೂ ಇದೆ ) ಗುಲಾಮನಾದವನು ಇನ್ನೊಬ್ಬನಿಂದ ಮುದ್ದಿಸಿ ಕೊಳ್ಳಲಿ ಅಥವಾ ಗುದ್ದಿಸಿಕೊಳ್ಳಲಿ, ಅವನ ಗುಲಾಮಗಿರಿಯೇನೂ ತಪ್ಪುವುದಿಲ್ಲ. ಹಾಗೆಯೇ, ಕಬ್ಬಿಣದ್ದಾದರೇನು, ಚಿನ್ನದ್ದಾದರೇನು–ಸರಪಳಿ ಸರಪಳಿಯೇ, ಬಂಧನಕಾರಿಯೇ (ಕಣ್ಣಿ = ಹಗ್ಗದ ಕುಣಿಕೆ). ಇಲ್ಲಿ ಸ್ವಾಮೀಜಿಯವರು ಎರಡು ಬಗೆಯ ಸರಪಳಿಗಳ ಬಗ್ಗೆ ಹೇಳುತ್ತಿದ್ದಾರೆ. (ಕಬ್ಬಿಣದ ಸರಪಳಿಯೆಂದರೆ ಹೆಂಡತಿ-ಮಕ್ಕಳು, ಆಸ್ತಿ-ಪಾಸ್ತಿ ಮೊದಲಾದುವುಗಳ ಬಂಧನ; ಚಿನ್ನದ ಸರಪಳಿಯೆಂದರೆ ಲೋಕಹಿತ ಕಾರ್ಯಗಳೇ ಮೊದಲಾದುವನ್ನು ಸಾಧಿಸಬೇಕೆಂಬ ಹೆಚ್ಚು ಉದಾತ್ತವಾದ ಆಕಾಂಕ್ಷೆಗಳು.) ಈ ಎರಡರಿಂದಲೂ ಬಿಡಿಸಿಕೊಳ್ಳಬೇಕಾದ್ದೇ ಯುಕ್ತ. ರಾಗ- ದ್ವೇಷ, ಪಾಪ-ಪುಣ್ಯ, ಸುಖ-ದುಃಖ ಇವೆಲ್ಲ ‘ದ್ವಂದ್ವ’ಗಳು. ಒಂದಿದ್ದಲ್ಲಿ ಇನ್ನೊಂದೂ ಇದ್ದೇ ಇರುತ್ತದೆ. ಇವುಗಳ ಕೈಗೆ ಸಿಲುಕಿಕೊಂಡು ತೊಂದರೆಗೀಡಾಗದಿರು.

(೩) ಕತ್ತಲು ತೊಲಗಲಿ; ಬುದ್ಧಿಗೆ ಭ್ರಮೆ ಕವಿಸುವಂತಹ, ಈ ಇಹಜೀವನದ ಮೇಲಿನ ಆಸೆ ಬತ್ತಿಹೋಗಲಿ, ಇದು ಮಾಯೆಯ ಸೃಷ್ಟಿ (ಪುತ್ಥಳಿ = ಸುಂದರವಾದ ಬೊಂಬೆ), ಕೇವಲ ಮರೀಚಿಕೆ. ಈ ದೇಹದ ಮೇಲಿನ ಮೋಹವೇ ನಮ್ಮನ್ನು ಮತ್ತೆಮತ್ತೆ ಹುಟ್ಟುಸಾವುಗಳ ಚಕ್ರಕ್ಕೆ ಸಿಲುಕಿ ನರಳುವಂತೆ ಮಾಡುವುದು. ಯಾವನು ತನ್ನನ್ನು ತಾನು ಗೆಲ್ಲಬಲ್ಲನೋ ಆತ ಇಡೀ ಜಗತ್ತನ್ನೇ ಗೆಲ್ಲಲು ಸಮರ್ಥ. ಇದನ್ನರಿತು, ಓ ಸಿದ್ಧ-ಪ್ರಬದ್ಧನಾದ ಸಂನ್ಯಾಸಿಯೇ, ಆ ವಿಷಜಾಲದಿಂದ ಪಾರಾಗು.

(೪) ತತ್ತ್ವಜ್ಞಾನಿಗಳು ಹೇಳುತ್ತಾರೆ: “ಬಿತ್ತಿದಂತೆ ಬೆಳೆ, ಬೀಜದಂತೆ ಮರ; ಹಾಗೆಯೇ ಕರ್ಮಕ್ಕೆ ತಕ್ಕಂತೆ ಪ್ರತಿಫಲ–ಕೆಟ್ಟ ಕರ್ಮಕ್ಕೆ ಪಾಪಫಲ, ಒಳ್ಳೆಯದಕ್ಕೆ ಪುಣ್ಯಫಲ; ಇದೊಂದು ಅನಿವಾರ್ಯ ನಿಯಮ; ಜೀವಾತ್ಮನು ದೇಹಧರಿಸಿ ಬಂದನೆಂದರೆ ಅವನು ಈ ಬಾಳೆಂಬ ಬಂಧನಕ್ಕೆ ಒಳಗಾಗಲೇ ಬೇಕು. ಒಂದು ಆಕಾರ ಇರುವವರೆಗೂ ಬಂಧನವು ತಪ್ಪಿದ್ದಲ್ಲ” ಎಂದು. (‘ಕಟ್ಟು ಕಟ್ಟನೆ ಹೆರುವುದು’ ಎಂದರೆ ಈ ಬಂಧನವೆಂಬುದೊಂದು ವಿಷ ವರ್ತುಲ; ಇದರಿಂದ ಬಿಡುಗಡೆ ಸಾಧ್ಯವಿಲ್ಲ ಎಂದು.) ಈ ಮಾತನ್ನು ಜನರೆಲ್ಲರೂ ಒಪ್ಪುತ್ತಾರೆ. ಅದೇನೋ ನಿಜವೆ. ಆದರೆ ನೀನೇನೂ ಈ ದೇಹವಲ್ಲ; ನೀನು ಸೂಕ್ಷ್ಮಾತಿಸೂಕ್ಷ್ಮವಾದ ಆತ್ಮ! ಆತ್ಮಕ್ಕೆ ಆಕಾರ ಗಾತ್ರಗಳೆಲ್ಲಿವೆ? ಅದು ಅನಂತ. ‘ನಾನು ಇಂತಹ ಆತ್ಮಸ್ವರೂಪಿ. ಆದ್ದರಿಂದ ಈ ನಿಯಮ- ನಿರ್ಬಂಧಗಳಾವುವೂ ನನಗಿಲ್ಲ. ನನಗೆ ಬಂಧನ-ಬಿಡುಗಡೆಗಳೆಂಬುದೇ ಇಲ್ಲ. ನಾನು ಚಿರಮುಕ್ತ’ ಎಂಬುದನ್ನು ಅರಿತುಕೊ. (ತತ್ತ್ವಮಸಿ = ಅವನು, ಎಂದರೆ ಆತ್ಮನು, ನೀನೇ ಆಗಿದ್ದೀಯೆ.) ಮತ್ತು ಓಂ ತತ್ ಸತ್ ಓಂ ಎಂಬ ಘೋಷಣೆಯಿಂದ ಈ ಸತ್ಯವನ್ನು ಜಗತ್ತಿಗೆಲ್ಲ ಸಾರಿ ಹೇಳು.

(೫) ‘ತಂದೆ-ತಾಯಿ, ಹೆಂಡತಿ-ಮಕ್ಕಳು’ ಎಂದು ಯಾರು ಪರದಾಡುತ್ತಾರೋ ಅವರು, ತಾವು ಆತ್ಮ ಎಂಬ ಸತ್ಯವನ್ನು ತಿಳಿಯರು. ಅವರು ಕನಸನ್ನೇ ನಿಜವೆಂದು ಭ್ರಮಿಸಿದ್ದಾರೆ. ಅನಿತ್ಯವಾದ ಈ ಸಂಬಂಧಗಳನ್ನೇ ಶಾಶ್ವತವೆಂದು ನಂಬಿದ್ದಾರೆ. ಆದರೆ ಆತ್ಮಕ್ಕೆ ಲಿಂಗಭೇದಗಳುಂಟೆ? ಜನನ-ವೃದ್ಧಾಪ್ಯ-ಮರಣಗಳುಂಟೆ? ಎಲ್ಲೆಲ್ಲೂ ವ್ಯಾಪಿಸಿರುವ, ಏಕಮೇವಾದ್ವಿತೀಯವಾದ (ಅದನ್ನು ಬಿಟ್ಟರೆ ಬೇರೊಂದಿಲ್ಲದ) ಆತ್ಮನು ಯಾರ ಮಗ, ಯಾರ ತಂದೆ! (ತಾತ = ತಂದೆ) ಆದರೆ ಹೀಗೆ ಮೇಲ್ನೋಟಕ್ಕೆ ವ್ಯಕ್ತಿ ವ್ಯಕ್ತಿಗಳಲ್ಲಿ ತೋರುವ ಭೇದವನ್ನು ಕಂಡು, ಬರಿಗಣ್ಣಿಗೆ ತೋರದ, ಎಲ್ಲರೊಳಗೂ ಇರುವ ಆತ್ಮನನ್ನು ಗುರುತಿಸದಿರಬೇಡ. ಈ ತೋರಿಕೆಯ ಭೇದಗಳೇ ನಾವು ಅಜ್ಞಾನದಲ್ಲಿರಲು ಕಾರಣ (‘ಹೇತು’). ಭೇದವನ್ನು ತೊರೆದು, ಇರುವವನೊಬ್ಬನೇ ಆತ್ಮ ಎಂದು ಅರಿತೆಯಾದರೆ, ಆ ಅರಿವೇ ಮುಕ್ತಿಗೆ ಸೇತುವೆಯಾಗುತ್ತದೆ. (‘ಜೀವನ್ಮುಕ್ತ’ =ದೇಹದಲ್ಲಿ ದ್ದರೂ ದೇಹಬಂಧನವನ್ನು ಕಡಿದುಕೊಂಡು ಮುಕ್ತನಾಗಿ ವಿಹರಿಸುತ್ತಿರುವವನು.) ಓ ಧೀರ ಸಂನ್ಯಾಸಿಯೇ, ಈ ತತ್ತ್ವವನ್ನು ಎಲ್ಲರಿಗೂ ಧೈರ್ಯದಿಂದ ಬೋಧಿಸು.

(೬) ನಿಜವಾಗಿಯೂ ಇರುವುದು ನಿತ್ಯಮುಕ್ತನೂ ಎಲ್ಲವನ್ನೂ ಅರಿತವನೂ ನಾಮರೂಪಗಳಿಗೆ ಅತೀತನೂ ಪಾಪಪುಣ್ಯಗಳನ್ನು ಮೀರಿದವನೂ ಆದ ಆತ್ಮನೊಬ್ಬನೇ. ಅವನು ತನ್ನದೇ ಮಾಯಾ ಶಕ್ತಿಯಿಂದ ಈ ಜಗತ್ತನ್ನು ಸೃಜಿಸಿದ್ದಾನೆ; ಈ ಎಲ್ಲ ‘ಕನಸು’ಗಳನ್ನೂ ನಿರ್ಮಿಸಿದ್ದಾನೆ. ತಾನೇ ಸಾಕ್ಷಿರೂಪದಿಂದ ನೋಡುತ್ತಿದ್ದಾನೆ. (ಸಾಕ್ಷಿ–ಅಕ್ಷಿ ಎಂದರೆ ಕಣ್ಣು; ಸಾಕ್ಷಿ ಎಂದರೆ ಕಣ್ಣುಳ್ಳ ವನು, ಅರ್ಥಾತ್ ಸಕಲವನ್ನೂ ನೋಡುತ್ತಿರುವವನು–ಇದು ಪರಮಾತ್ಮನ ಒಂದು ಹೆಸರು.) ಪ್ರಕೃತಿ-ಪುರುಷರ ರೂಪದಲ್ಲಿ ತೋರಿಬರುವವನು ಆ ಆತ್ಮನೇ.

ಮುಕ್ತಿಗಾಗಿ ಎಲ್ಲಿ ತಡಕಾಡುವೆ? ಅದಕ್ಕಾಗಿ ಏಕೆ ಸುಮ್ಮನೆ ವ್ಯರ್ಥ ಪ್ರಯತ್ನಗಳಲ್ಲಿ ತೊಡಗುವೆ? ಅದು ನಿನ್ನ ಸುತ್ತಲಿನ ಈ ಜಗತ್ತಿನಲ್ಲಾಗಲಿ, ಮೇಲೆ ಸ್ವರ್ಗದಲ್ಲಾಗಲಿ ಎಲ್ಲೋ ಸಿಗುವಂಥದಲ್ಲ; ಶಾಸ್ತ್ರಗ್ರಂಥಗಳಲ್ಲಾಗಲಿ, ಗುಡಿ-ಮಸೀದಿಗಳಲ್ಲಾಗಲಿ ಈ ಮುಕ್ತಿಯೆಂಬುದು ಅಡಗಿಲ್ಲ. ಅದಿರುವುದು ನಿನ್ನೊಳಗೇ. ನಿನ್ನನ್ನು ಬಂಧಿಸಿರುವ ಪಾಶದ ತುದಿ ನಿನ್ನ ಕೈಯಲ್ಲೇ ಇದೆ! ಇವನ್ನರಿಯದೆ, “ಅಯ್ಯೋ, ನಾನು ಬಂಧಿ” ಎಂದು ರೋದಿಸುತ್ತ ನಿನ್ನನ್ನು ನೀನೇ ವಂಚಿಸಿಕೊಳ್ಳ ಬೇಡ. ದೃಢಸಂಕಲ್ಪ ಮಾಡಿ, ಆ ಪಾಶವನ್ನು ಕತ್ತರಿಸಿಕೊ.

(೭) ಓ ಸಂನ್ಯಾಸಿಯೇ, ಹೀಗೆ ಹೇಳಿಕೊ (ಉಲಿ =ಹೇಳು ): “ಸಕಲರೂ ಶಾಂತಿಯಿಂದಿರಲಿ; ನನ್ನಿಂದಾಗಿ ಯಾವ ಜೀವಿಗೂ ಕಿಂಚಿತ್ತೂ ತೊಂದರೆಯಾಗದಿರಲಿ. ಏಕೆಂದರೆ ಆಗಸದಲ್ಲಿ ಹಾರಾಡುವ ಪಕ್ಷಿಗಳಾಗಲಿ, ನೆಲದ ಮೇಲೆ ಸಂಚರಿಸುವ ಪ್ರಾಣಿಗಳಾಗಲಿ, ಅವರೆಲ್ಲರೊಳಗೂ ಇರುವುದು ನಾನೇ–ನನ್ನ ಆತ್ಮವೇ! (ಈ ಪರಮಸತ್ಯವನ್ನು ಅರಿತುಕೊಂಡು,) ಸ್ವರ್ಗ-ನರಕ ಲೋಕಗಳ ಮೇಲಿನ ಆಸೆ-ಭಯಗಳನ್ನು ಮನದಿಂದಾಚೆಗೆ ದೂಡುತ್ತೇನೆ” ಎಂದು.

ಇನ್ನು ಈ ದೇಹವೆಂಬ ಹಂದರದ ಕೆಲಸವಾಯಿತು; ಅದು, ಬೇಕಾದರೆ ಬದುಕಿರಲಿ, ಇಲ್ಲವಾದರೆ ಸಾಯಲಿ! ಅದು ಕರ್ಮನದಿಯಲ್ಲಿ ಹೇಗೆ ಬೇಕಾದರೂ ತೇಲಿಕೊಂಡು ಹೋಗಲಿ (ಕರ್ಮನದಿ = ವ್ಯಕ್ತಿಯ ಕರ್ಮ-ಕರ್ಮಫಲಗಳ ಪರಂಪರೆ. ಇಲ್ಲಿ, ತ್ಯಾಗಿಯ ಶರೀರವನ್ನು ನೀರಿ ನಲ್ಲಿ ತೇಲಿಕೊಂಡು ಹೋಗುತ್ತಿರುವ ಒಂದು ಶವಕ್ಕೆ ಹೋಲಿಸಲಾಗಿದೆ). ಬೇಕಾದರೆ ಯಾರಾ ದರೂ ಅದನ್ನು ತಿರಸ್ಕರಿಸಿ ಒದೆಯಲಿ, ಅದರಿಂದೇನಂತೆ? ಮಣ್ಣಿನಿಂದ ಬಂದ ಈ ಶರೀರವು (ಹುಡಿ = ಮಣ್ಣು) ಮರಳಿ ಮಣ್ಣಿಗೇ ಹೋಗಲಿ, ಬಿಡು!ಹೊಗಳುವವರು-ಹೊಗಳಿಸಿಕೊಳ್ಳು ವವರು, ನಿಂದಿಸುವವರು-ನಿಂದಿಸಲ್ಪಡುವವರು ಎಲ್ಲರೂ ಒಂದೇ ಆತ್ಮವೆಂದ ಮೇಲೆ, ಆ ಹೊಗಳಿಕೆಯನ್ನಾಗಲಿ ನಿಂದೆಯನ್ನಾಗಲಿ ಅನುಭವಿಸುವವರು ತಾನೆ ಯಾರು! ಓ ಸಂನ್ಯಾಸಿಯೆ, (ಕಿತ್ತಡಿ = ಪಾದಗಳನ್ನು ಕದಲಿಸದೆ ಒಂದೆಡೆಯಲ್ಲೇ ನೆಟ್ಟವನು; ತಪಸ್ವಿ) ಈ ತತ್ತ್ವವನ್ನು ಅರಿತು ನೀನು ಪಾಶಮುಕ್ತನಾಗು.

(೮) ಯಾರಲ್ಲಿ ಕಾಮಕಾಂಚನಾಸಕ್ತಿಯು ಕಿಂಚಿತ್ತಾದರೂ ಇದೆಯೋ ಅಂಥವನು ಸತ್ಯವನ್ನು, ಎಂದರೆ ಪರಮ ಸತ್ಯಸ್ವರೂಪನಾದ ಭಗವಂತನನ್ನು ಕಾಣಲಾರನು; ಕಾಮಬುದ್ಧಿಯಿರುವವನಿಗೆ ಮುಕ್ತಿ ದೊರಕಲಾರದು; ಎಲ್ಲಿ ತ್ಯಾಗವೆಂಬುದು ನೆಲಸಿಲ್ಲವೋ ಅಲ್ಲಿ ಖಂಡಿತವಾಗಿ ಯೋಗವು ಕಾಣಸಿಗದು.

ಓ ಸಂನ್ಯಾಸಿಯೇ, ವಿಶ್ವಾತ್ಮನಾದ ನಿನಗೆ ಮನೆಯು ಸಾಕಾಗುವುದೆ? ಇಡೀ ಭೂಮಿಯೇ ನಿನ್ನ ಮನೆ; ಆಗಸವೇ ಛಾವಣಿ; ಬಯಲಿನ ಹುಲ್ಲುಹಾಸೇ ನಿನ್ನ ಹಾಸಿಗೆ! ಬೇಯಿಸಿದ್ದೋ ಬೆಂದಿಲ್ಲದ್ದೋ, ಎಂಥದೋ ಒಂದು, ಭಿಕ್ಷೆಯಿಂದ ತಾನಾಗಿಯೇ ಏನು ದೊರಕುತ್ತದೊ ಅದೇ ಅಂದಿನ ಆಹಾರ ಎಂದು ತಿಳಿದುಕೊ. ಏನು ತಿಂದರೇನು, ಏನು ಕುಡಿದರೇನು? ಆತ್ಮಕ್ಕೆ ಅದೇನು ಮಾಡೀತು? ಹೇಗೆ ಗಂಗಾನದಿಗೆ ಮಲಿನತೆಯೆಂಬುದಾಗಲಿ ನ್ಯೂನತೆಯೆಂಬುದಾಗಲಿ ಇಲ್ಲವೋ, ಹಾಗೆಯೇ ನೀನೂ ಆ ನದಿಯಂತೆ ಮುನ್ನಡೆ. ನೀನು ಮಿಂಚು-ಸಿಡಿಲುಗಳಂತೆ ತೇಜಸ್ವಿ, ಶಕ್ತಿಶಾಲಿ. ಇದನ್ನು ಮನಗಾಣುತ್ತ ಓಂಕಾರವನ್ನು ಮೊಳಗು.

(೯) ಹೇ ಮಹಾತ್ಮನೇ, (ನಿನ್ನ ವಿಷಯದಲ್ಲಿ) ವಸ್ತುಸ್ಥಿತಿಯೇನೆಂಬುದನ್ನು ಕೆಲವರು ಮಾತ್ರ ತಿಳಿಯಬಲ್ಲರು. ಉಳಿದವರೆಲ್ಲ ನಿನ್ನನ್ನು ಉಡಾಫೆಯಿಂದ ಕಂಡು ನಗಬಹುದು. (ಇಲ್ಲವೆ ಅಸಹನೆ ತಾಳಲೂಬಹುದು.) ಆಗಲಿ, ಅದರಿಂದೇನಂತೆ! ಅವರನ್ನೆಲ್ಲ ನಿರ್ಲಕ್ಷಿಸಿ ನೀನು ಸ್ವತಂತ್ರವಾಗಿ ಸಾಗು; ಊರಿಂದೂರಿಗೆ ಹೋಗು. ಯಾವ ಅಂಜಿಕೆಯನ್ನೂ ಪ್ರತಿಫಲಾಪೇಕ್ಷೆಯನ್ನೂ ಇಟ್ಟು ಕೊಳ್ಳದೆ, ಅಗತ್ಯವಿರುವವರಿಗೆಲ್ಲ ಆತ್ಮದ ಬೆಳಕನ್ನು ನೀಡು; ಮಾಯೆಯ ಕತ್ತಲಿನಿಂದ ಪಾರು ಮಾಡು (ಸಂಚಾರಿಗೆ =ಜಿಜ್ಞಾಸು-ಮುಮುಕ್ಷುಗಳಿಗೆ). ಹೀಗೆ, ನಿನ್ನೊಳಗಿನ ಶಕ್ತಿಯೆಲ್ಲ ಮುಗಿಯುವ ವರೆಗೂ ಮಾಡುತ್ತಹೋಗು. ಕೊನೆಗೆ ‘ನಾನು-ನೀನು’ ಎಂಬ ಭೇದಗಳೆಲ್ಲವನ್ನೂ ಮೀರಿ, ಸಚ್ಚಿದಾ ನಂದ ಆತ್ಮದಲ್ಲಿ ಲೀನನಾಗು. ‘ನಾನು ಪರಬ್ರಹ್ಮಸ್ವರೂಪಿ’ ಎಂದು ತಿಳಿದುಕೊಂಡು ಹಾಡು: ಓಂ ತತ್ ಸತ್ ಓಂ!

