
\chapter{ಮಹಾತ್ಯಾಗದ ಮುನ್ನೋಟ}

ಹೀಗೆ ನರೇಂದ್ರ ತನ್ನ ಯುವಸ್ನೇಹಿತರ ಶೀಲವನ್ನು ತಿದ್ದುತ್ತ ಅವರ ಜೀವನಗಳನ್ನು ರೂಪಿಸುತ್ತಿದ್ದ; ಜೊತೆಗೆ ತಾನೂ ಸಾಧನಾನಿರತನಾಗಿದ್ದು, ಗುರುಸೇವೆಯನ್ನೂ ಮಾಡುತ್ತ ಬಂದ. ಆದರೆ ಈ ಕಡೆ ಶ್ರೀರಾಮಕೃಷ್ಣರ ದೇಹಾರೋಗ್ಯ ಉತ್ತಮಗೊಳ್ಳುವುದರ ಬದಲಿಗೆ ದಿನದಿಂದ ದಿನಕ್ಕೆ ಕೆಡುತ್ತಲೇ ಬಂತು. ಔಷಧಗಳಾವೂವೂ ಪರಿಣಾಮಕಾರಿಯಾಗದೆ ಹೋದುವು. ಇದನ್ನು ಗಮನಿಸಿದ ಡಾಕ್ಟರ್ ಮಹೇಂದ್ರಲಾಲ್ ಸರ್ಕಾರ ಆಲೋಚಿಸಿದ–ಬಹುಶಃ ಕಲ್ಕತ್ತದ ಮಲಿನ ವಾತಾವರಣವೇ ಇದಕ್ಕೆ ಕಾರಣ; ಅವರನ್ನು ಯಾವುದಾದರೂ ಉದ್ಯಾನಗೃಹದಲ್ಲಿಟ್ಟರೆ ಅವರು ಗುಣಮುಖರಾಗಲು ಅನುಕೂಲವಾದೀತು ಎಂದು. ಭಕ್ತರಿಗೆಲ್ಲ ಇದು ಒಪ್ಪಿಗೆಯಾಯಿತು. ಸರಿ, ಕೂಡಲೇ ಎಲ್ಲರೂ ಅಂತಹ ಒಂದು ಜಾಗಕ್ಕಾಗಿ ಹುಡುಕಾಟ ನಡೆಸಿ ಉತ್ತರ ಕಲ್ಕತ್ತದ ಕಾಶೀಪುರ ಎಂಬಲ್ಲಿನ ಉದ್ಯಾನಗೃಹವೊಂದನ್ನು ಗೊತ್ತುಮಾಡಿದರು. ೧೮೮೫ರ ಡಿಸೆಂಬರ್ ತಿಂಗಳಲ್ಲಿ ಶ್ರೀರಾಮಕೃಷ್ಣರನ್ನು ಇಲ್ಲಿಗೆ ಕರೆತೆರಲಾಯಿತು. ಈ ಮನೆ, ಗಿಡಮರಬಳ್ಳಿಗಳ ನಡವೆ, ವಿಶಾಲವಾದ ಹಜಾರಗಳಿಂದ ಕೂಡಿದ್ದು ಬಹಳ ಸುಂದರವಾಗಿತ್ತು. ಇಲ್ಲಿನ ಶುದ್ಧವಾದ ಗಾಳಿ, ಪ್ರಶಾಂತವಾದ ವಾತಾವರಣ ಇವುಗಳಿಂದ ಅವರಿಗೆ ಎಷ್ಟೋ ಹಿತವೆನಿಸಿತು.

ಆದರೆ ಈ ಕಾಶೀಪುರದ ಉದ್ಯಾನಗೃಹವಾಸವೇ ಶ್ರೀರಾಮಕೃಷ್ಣರ ಜೀವನ ನಾಟಕದ ಕೊನೆಯ ದೃಶ್ಯ. ಅವರು ಗಂಟಲು ಬೇನೆಯಿಂದ ಅಧಿಕವಾಗಿ ನರಳಿದ್ದೂ ಇಲ್ಲೇ. ಹಾಗೆಯೇ ಭಕ್ತವೃಂದದ ಪಾಲಿಗೆ ಆನಂದಮಯವಾದ ಕೆಲವು ದಿನಗಳನ್ನು ಕಳೆದದ್ದೂ ಈ ಸ್ಥಳದಲ್ಲೇ. ಅವರು ತಮ್ಮ ಅವತಾರೋದ್ದೇಶದ ಕಾರ್ಯವನ್ನು ಪೂರ್ಣಗೊಳಿಸಿದ್ದೂ ಈ ದಿನಗಳಲ್ಲೇ. ಮತ್ತು ಮುಂದೆ ತಮ್ಮ ದಿವ್ಯಸಂದೇಶಗಳನ್ನು ಸಾರಲಿರುವಂತಹ ಸರ್ವಸಂಗ ಪರಿತ್ಯಾಗಿಗಳಾದ ಯುವಕರನ್ನು ಸೇರಿಸಿ ಸಂಘಟನೆ ಮಾಡಿದ್ದೂ ಈ ಸ್ಥಳದಲ್ಲೇ. ಈಗ ಅವರ ನಿರೀಕ್ಷೆ-ಭರವಸೆಗಳೆಲ್ಲ ನರೇಂದ್ರನಲ್ಲೇ ಕೇಂದ್ರೀಕೃತವಾಗಿವೆ. ತಾವು ಮರಣಶಯ್ಯೆಯಲ್ಲಿ ಮಲಗಿದ್ದರೂ ನರೇಂದ್ರ ಹಾಗೂ ಇತರ ಶಿಷ್ಯರನ್ನು ಮಹಾಕಾರ್ಯವೆಸಗಬಲ್ಲ ಮಹಿಮಾವಂತರನ್ನಾಗಿ ಸಿದ್ಧಗೊಳಿಸುವ ಕಾರ್ಯದಲ್ಲಿ ಅವರೀಗ ನಿರತರಾದರು.

ತಮ್ಮ ಅವತಾರಸಮಾಪ್ತಿಯ ಕಾಲದ ಬಗ್ಗೆ ಹಿಂದೆ ಅವರೇ ಅನೇಕ ಸೂಚನೆಗಳನ್ನು ಕೊಟ್ಟಿದ್ದರು: “ಯಾವಾಗ ಅಸಂಖ್ಯಾತ ಜನರು ಇಲ್ಲಿನ (ತಮ್ಮ) ಮಹಿಮೆಯ ಕುರಿತಾಗಿ ಮಾತನಾಡಿಕೊಳ್ಳಲು ತೊಡಗುತ್ತಾರೆಯೋ ಆಗ ಜಗನ್ಮಾತೆ ಈ ಶರೀರವನ್ನು ಹಿಂದೆಗೆದುಕೊಂಡುಬಿಡುತ್ತಾಳೆ.” “ನಾನು ಹೊರಡುವ ಮೊದಲು ಎಲ್ಲವನ್ನೂ (ಎಂದರೆ ತಮ್ಮ ದಿವ್ಯತೆಯನ್ನು) ಬಹಿರಂಗ ಪಡಿಸುತ್ತೇನೆ.” “ನನ್ನ ಜೀವನದ ಕೊನೆಯ ದಿನಗಳಲ್ಲಿ ಭಕ್ತರು ಎರಡು ಗುಂಪುಗಳಾಗಿ ಬೇರ್ಪಡೆಯಾಗುತ್ತಾರೆ–ಒಂದು ಗುಂಪು ನನ್ನ ಅಂತರಂಗಕ್ಕೆ ಸೇರಿದವರು, ಇನ್ನೊಂದು ಗುಂಪು ಬಹಿರಂಗ ವಲಯಕ್ಕೆ ಸೇರಿದವರದು.” ಈ ಬಗೆಯಲ್ಲಿ ಅವರೇ ನೀಡಿದ ಹಲವಾರು ಮುನ್ಸೂಚನೆಗಳು ಒಂದೊಂದಾಗಿ ನಿಜವಾಗುತ್ತ ಬರುವುದನ್ನು ಶಿಷ್ಯರು ಅಸಹಾಯಕರಾಗಿ ನೋಡುತ್ತಿದ್ದರು.

ತಮ್ಮ ಈ ಅಪಾಯ ಯಾತನೆಯ ನಡುವೆಯೂ ಶ್ರೀರಾಮಕೃಷ್ಣರು ನರೇಂದ್ರನಿಗೆ ವಿಶೇಷ ಆಧ್ಯಾತ್ಮಿಕ ತರಬೇತಿ ನೀಡುತ್ತಿದ್ದರು. ನರೇಂದ್ರನ ಆಧ್ಯಾತ್ಮಿಕ ಶಕ್ತಿ ವೃದ್ಧಿಯಾಗುತ್ತಿದ್ದಂತೆಯೇ ಅವರ ಶರೀರಶಕ್ತಿ ಕುಂದಿಹೋಗುತ್ತಿತ್ತು. ಆದರೆ ದೈವಿಕತೆ ಮಾತ್ರ ಇನ್ನಷ್ಟು ಹೆಚ್ಚಾಗಿಯೇ ಪ್ರಕಟವಾಗುತ್ತಿತ್ತು ಎನ್ನಬಹುದು. ಇಂತಹಘೋರ ವ್ಯಾಧಿಯ ಸಮಯದಲ್ಲೂ ಅವರಿಂದ ಅಪರಿಮಿತ ಆಧ್ಯಾತ್ಮಿಕ ಶಕ್ತಿ-ಆಧ್ಯಾತ್ಮಿಕ ಆನಂದ ಹೊರಸೂಸುತ್ತಿರುವುದನ್ನು ಪ್ರತ್ಯಕ್ಷವಾಗಿ ಕಂಡು ಅನುಭವಿಸುತ್ತ ಭಕ್ತರಲ್ಲಿ ಹೊಸ ಉತ್ಸಾಹ ಉಕ್ಕುತ್ತಿತ್ತು. ಯುವಶಿಷ್ಯರು ಗುರುವಿನ ಸೇವೆಯನ್ನು ಹೃತ್ಪೂರ್ವಕವಾಗಿ ಪೂಜೆಯೋ ಪಾದಿಯಲ್ಲಿ ಮಾಡಿಕೊಂಡು ಬರುತ್ತಿದ್ದರು. ಆದರೆ ದಿನದಿನಕ್ಕೆ ಉಲ್ಬಣಿಸುತ್ತಿದ್ದ ಗಂಟಲ ಬೇನೆಯನ್ನು ತಡೆಗಟ್ಟಲು ಮಾತ್ರ ಯಾರಿದಂಲೂ ಸಾಧ್ಯವಿರಲಿಲ್ಲ. ಶ್ರೀರಾಮಕೃಷ್ಣರು ಮಾತನಾಡಿ ಗಂಟಲಿಗೆ ಆಯಾಸ ಮಾಡಿಕೊಳ್ಳಬಾರದು ಎಂದು ಡಾಕ್ಟರು ಪುನಃ ಪುನಃ ಎಚ್ಚರಿಕೆ ಹೇಳುತ್ತಲೇ ಇದ್ದರು. ಆದರೆ ಹಗಲಿರುಳೆನ್ನದೆ ಧಾವಿಸಿ ಬಂದು ಮುತ್ತುತ್ತಿರುವ ಮುಮುಕ್ಷುಗಳಿಗೆ, ತಮ್ಮ ಅನಂತ ಆಧ್ಯಾತ್ಮಿಕ ಜ್ಞಾನವನ್ನು, ಆನಂದವನ್ನು ಹಂಚಿಕೊಡದೆ ಸುಮ್ಮನಿರಲು ಶ್ರೀರಾಮಕೃಷ್ಣರಿಂದ ಹೇಗೆ ತಾನೆ ಸಾಧ್ಯವಾದೀತು?

ನರೇಂದ್ರನ ನೇತೃತ್ವದಲ್ಲಿ ನಿರಂತರ ಸೇವೆಯಲ್ಲಿ ತೊಡಗಿದ್ದ ಯುವಶಿಷ್ಯರು ಕಾಶೀಪುರದ ಮನೆಯಲ್ಲೇ ಉಳಿದುಕೊಳ್ಳಬೇಕಾಗಿ ಬಂತು. ಇದರಿಂದಾಗಿ ಸಹಜವಾಗಿಯೇ ಅವರ ತಾಯ್ತಂದೆಯರಿಗೆ ಬಹಳ ಅಸಮಾಧಾನವಾಯಿತು. ಹುಡುಗರು ಓದುಬರಹ ಬಿಟ್ಟು ಗುರುಸೇವೆ ಮಾಡುತ್ತೇವೆಂದು ಅಲ್ಲಿಯೇ ಕುಳಿತುಬಿಟ್ಟದ್ದು ಅವರಿಗೆ ಇಷ್ಟವಾಗಲಿಲ್ಲ; ಪ್ರತಿಭಟಿಸಿದರು. ಆದರೂ ಯುವಕರು ಜಗ್ಗಲಿಲ್ಲ. ಈ ನಡುವೆ ನರೇಂದ್ರನಿಗೂ ಕಾನೂನು ಪರೀಕ್ಷೆ ಸನ್ನಿಹಿತವಾಗುತ್ತಿತ್ತು. ಜೊತೆಗೆ ಮನೆಗೆ ಸಂಬಂಧಿಸಿದ ಕೋರ್ಟ ವ್ಯವಹಾರ ಬೇರೆ ನಡೆಯುತ್ತಿತ್ತು. ಆದ್ದರಿಂದ ಅವನು ಆಗಾಗ ಕಲ್ಕತ್ತಕ್ಕೆ ಹೋಗಿಬರಬೇಕಾಗುತ್ತಿತ್ತು. ಹೀಗಿದ್ದರೂ ಅವನು ಶ್ರೀಗುರುಸೇವೆಯನ್ನೇ ತನ್ನ ಆದ್ಯ ಕರ್ತವ್ಯವೆಂದು ಭಾವಿಸಿ ಕಾಶೀಪುರದಲ್ಲಿ ದೃಢಮನಸ್ಕನಾಗಿ ನಿಂತುಬಿಟ್ಟ. ಬಿಡುವಿನ ವೇಳೆಯಲ್ಲಿ ಓದಿ ಪರೀಕ್ಷೆಗೆ ತಯಾರಿ ಮಾಡಿಕೊಳ್ಳುತ್ತಿದ್ದ.

ಈಗ ಶ್ರೀರಾಮಕೃಷ್ಣರು ಕಾಶೀಪುರದ ಉದ್ಯಾನಗೃಹದಲ್ಲಿ ಕೇವಲ ತಮ್ಮ ಅಂತರಂಗ ಶಿಷ್ಯರ ನಡುವೆ ವಾಸವಾಗಿದ್ದಾರೆ. ಎಲ್ಲರೂ ಬಿಸಿರಕ್ತದ ಯುವಕರು. ಅವರಲ್ಲಿ ಮುಖ್ಯರಾದ ಹನ್ನೆರಡು ಜನರೆಂದರೆ ನರೇಂದ್ರ, ರಾಖಾಲ, ಬಾಬುರಾಮ್, ನಿರಂಜನ, ಯೋಗೀಂದ್ರ, ಲಾಟು, ತಾರಕನಾಥ, ಕಾಳೀಪ್ರಸಾದ, ಹಿರಿಯ ಗೋಪಾಲ(ಇವರೊಬ್ಬರೇ ವಯಸ್ಕರು), ಶಶಿಭೂಷಣ, ಶರಚ್ಚಂದ್ರ ಮತ್ತು ಕಿರಿಯ ಗೋಪಾಲ. ಶ್ರೀರಾಮಕೃಷ್ಣರ ಸೇವೆಗಾಗಿ ತಮ್ಮ ಮನೆಮಂದಿಯನ್ನೂ ಶಾಲಾಕಾಲೇಜುಗಳನ್ನೂ ತೊರೆದುಬರುವಂತೆ ಇವರನ್ನೆಲ್ಲ ಪ್ರೋತ್ಸಾಹಿಸಿದ್ದು ನರೇಂದ್ರನೇ. ಅತ್ತ ಶ್ರೀರಾಮಕೃಷ್ಣರ ಸೆಳೆತ ಇದ್ದೇ ಇದೆ, ಜೊತೆಗೆ ಇತ್ತ ನರೇಂದ್ರನಿಂದಲೂ ಸ್ಫೂರ್ತಿ ಸಿಗುತ್ತಿದೆ. ಎಲ್ಲರೂ ಬಿಡುವಾಗಿದ್ದಾಗಲೆಲ್ಲ ಅವನು ಅವರನ್ನು ಒಟ್ಟಗೂಡಿಸಿ ವೇದಶಾಸ್ತ್ರಗಳ ಅಧ್ಯಯನ ನಡೆಸುತ್ತಿದ್ದ. ಸಂಕೀರ್ತನೆ ಮಾಡಿಸುತ್ತಿದ್ದ. ಅವರು ಶ್ರೀರಾಮಕೃಷ್ಣರ ಜೀವನ-ವ್ಯಕ್ತಿತ್ವ-ಬೋಧನೆಗಳ ಕುರಿತಾಗಿ ಸಮಾಲೋಚಿಸುತ್ತಿದ್ದರು. ಹೀಗೆ ತಮ್ಮ ಬೌದ್ಧಿಕ-ಆಧ್ಯಾತ್ಮಿಕ ಉನ್ನತಿಗಾಗಿ ಶ್ರಮಿಸುತ್ತಿದ್ದರು. ಆ ಯುವಕರ ಕೌಟುಂಬಿಕ ಹಿನ್ನೆಲೆಗಳು, ಸಂಸ್ಕಾರಗಳು, ಅಭಿರುಚಿಗಳು ಎಲ್ಲ ಬೇರಬೇರೆ, ಆದರೆ ನರೇಂದ್ರ ತನ್ನ ಉಜ್ವಲ-ಪ್ರಖರ ವ್ಯಕ್ತಿತ್ವದ ಕಾವಿನಿಂದ ಇಂತಹ ವಿಭಿನ್ನ ವ್ಯಕ್ತಿತ್ವಗಳನ್ನು ಬೆಸೆದು, ಅವರಲ್ಲೊಂದು ಮಧುರ ಸಮರಸತೆಯನ್ನು ತರಲೆತ್ನಿಸಿದ. ತ್ಯಾಗದಲ್ಲಾಗಲಿ, ಶ್ರದ್ಧಾಭಕ್ತಿಗಳಲ್ಲಾಗಲಿ ಒಬ್ಬರಿಗಿಂತ ಒಬ್ಬರು ಶ್ರೇಷ್ಠರು. ಇಂತಹ ಆ ತೇಜಸ್ವೀ ಯುವಕರನ್ನೆಲ್ಲ ಒಟುಟಗೂಡಿಸಿ, ಅವರೆಲ್ಲ ಒಂದೇ ಶರೀರ ಒಂದೇ ಆತ್ಮ ಎಂಬಂತೆ ಮಾಡಿಬಿಟ್ಟ ನರೇಂದ್ರ. ಮುಂದೆ ಇವರೊಂದಿಗೇ ಸಂನ್ಯಾಸ ಸ್ವೀಕರಿಸಿದ ಇತರ ಮೂರ್ನಾಲ್ಕು ಯುವಶಿಷ್ಯರು ಮನೆಯವರ ಒತ್ತಡ ಹೆಚ್ಚಾದುದರಿಂದ ಈಗ ಮನೆಗಳಿಗೆ ಹಿಂದಿರುಗಿದ್ದರಾದರೂ ಆಗಾಗ ಬಂದುಹೋಗುತ್ತಿದ್ದರು.

ಶ್ರೀರಾಮಕೃಷ್ಣರ ಇಹಜೀವನ ಕೊನೆಗೊಳ್ಳುತ್ತದ್ದಂತೆ ನರೇಂದ್ರನಲ್ಲಿ ಭಗವತ್ಸಾಕ್ಷಾತ್ಕಾರದ ಹಂಬಲ ಭುಗಿಲೆದ್ದುಬಿಟ್ಟಿತು. ಒಂದು ದಿನ, ರಾತ್ರಿಯ ಕೆಲಸಕಾರ್ಯಗಳನ್ನೆಲ್ಲ ಮುಗಿಸಿ ಮಲಗಿಕೊಳ್ಳಲು ಸಿದ್ಧನಾದ; ಬೆಳಗಾಗೆದ್ದು ಮನೆಗೆ ಹಿಂದಿರುಗಿ ಒಂದೆರಡು ದಿನಗಳ ಮಟ್ಟಿಗೆ ಇದ್ದು, ಕೆಲವು ತುರ್ತು ಕೆಲಸಗಳನ್ನು ಮುಗಿಸಿಕೊಂಡು ಬರಬೇಕೆಂದು ನಿಶ್ಚಯಿಸಿದ್ದ. ಆದರೆ ಮಲಗಿಕೊಂಡರೆ ನಿದ್ರೆಯೇ ಬರುತ್ತಿಲ್ಲ. ಆಗ ಅವನು, ಇನ್ನೂ ಎಚ್ಚರವಾಗಿದ್ದ ಶರತ್, ಕಿರಿಯ ಗೋಪಾಲ ಮತ್ತಿತರ ಗೆಳೆಯರನ್ನು ಎಬ್ಬಿಸಿ, “ಬನ್ನಿ, ಇಲ್ಲೇ ತೋಟದಲ್ಲಿ ಸ್ವಲ್ಪ ಸುತ್ತಾಡೋಣ” ಎಂದು ಕರೆದ. ಹಾಗೇ ಅಡ್ಡಾಡುತ್ತ ಮಾತನಾಡಲಾರಂಭಿಸಿದ: “ಶ್ರೀರಾಮಕೃಷ್ಣರ ಕಾಯಿಲೆ ನಿಜಕ್ಕೂ ತುಂಬ ಗಂಭೀರವಾಗಿದೆ. ಅವರು ತಮ್ಮ ದೇಹವನ್ನು ತ್ಯಜಿಸಿಬಿಡುವ ನಿರ್ಧಾರ ಮಾಡದಿರಲಿ! ಆದರೆ ಪರಿಸ್ಥಿತಿ ಕೈಮೀರುವ ಮೊದಲೇ ನಾವು ಅವರ ಸೇವೆಯನ್ನು ಮಾಡುವುದರ ಮೂಲಕ ಮತ್ತು ಪ್ರಾರ್ಥನೆ-ಧ್ಯಾನಗಳನ್ನು ತೀವ್ರಗೊಳಿಸುವುದರ ಮೂಲಕ, ಭಗವಂತನ ಸಾಕ್ಷಾತ್ಕಾರವನ್ನು ಮಾಡಿಕೊಂಡುಬಿಡಬೇಕು. ಶ್ರೀರಾಮಕೃಷ್ಣರು ಶರೀರ ಬಿಟ್ಟು ಹೋದ ಮೇಲೆ ಎಷ್ಟೇ ಪಶ್ಚಾತ್ತಾಪ-ಪರಿತಾಪ ಪಟ್ಟರೂ ಪ್ರಯೋಜನವಿಲ್ಲ.‘ಈ ಕೆಲಸವನ್ನು ಮಾಡಿ ಮುಗಿಸಿಬಿಡುತ್ತೇನೆ; ಅದೊಂದು ಕೆಲಸ ಬಾಕಿ ಇದೆ, ಅದನ್ನು ಮುಗಿಸಿದ ಮೇಲೆ ಸಮಯ ಮಾಡಿಕೊಂಡು ಜಪ-ಧ್ಯಾನ ಮಾಡೋಣ’ ಅಂತ ಯೋಸಿಸುತ್ತಿದೇವೆ. ಆದರೆ ಇದು ಕೇವಲ ಭ್ರಮೆ. ಹೀಗೆ ಭ್ರಮಿಸುತ್ತ ಅತ್ಯಮೂಲ್ಯವಾದ ಸಮಯವನ್ನು ವ್ಯರ್ಥ ಮಾಡುತ್ತಿದ್ದೇವೆ. ಅಲ್ಲದೆ ಆಶಾಪಾಶಗಳು ನಮ್ಮನ್ನು ಇನ್ನಷ್ಟು ಬಲವಾಗಿ ಬಿಗಿಯುತ್ತವೆ. ಈ ಆಸೆಗಳೇ ಮೃತ್ಯು. ಆದ್ದರಿಂದ ಈ ಆಸೆಗಳನ್ನು ಬೇರು ಸಹಿತ ಕ್ತಿತುಹಾಕಿ ಮುಕ್ತರಾಗೋಣ.”

ಅದೊಂದು ಚಳಿಗಾಲದ ರಾತ್ರಿ; ಆಗಸ ನಿರ್ಮಲವಾಗಿದೆ; ನಕ್ಷತ್ರಗಳು ಮಿನುಗುತ್ತಿದವೆ. ಆ ಪ್ರಶಾಂತ ವಾತಾವರಣದಲ್ಲಿ, ಧ್ಯಾನಮಗ್ನರಾಗಿ ಕುಳಿತುಬಿಡಬೇಕೆಂಬ ತೀವ್ರ ಬಯಕೆ ಅವರಲ್ಲಿ ಜಾಗೃತವಾಯಿತು. ಹತ್ತಿರದಲ್ಲಿ ಬಿದ್ದಿದ್ದ ಒಣ ಕಡ್ಡಿ-ಕೊಂಬೆಗಳನ್ನು ನೋಡಿದಾಗ ನರೇಂದ್ರನಿಗೊಂದು ಆಲೋಚನೆ ಹೊಳೆಯಿತು. ತನ್ನ ಮಾತನ್ನು ಮುಂದುವರಿಸುತ್ತ ಅವನೆಂದ: “ಈ ಒಣ ಕಡ್ಡಿಗಳಿಗೆ ಬೆಂಕಿ ಹೊತ್ತಿಸಿ ಕುಳಿತುಕೊಳ್ಳೋಣ. ಸಂನ್ಯಾಸಿಗಳು ಧುನಿಯನ್ನು\footnote{*ನೋಡಿ: ಅನುಬಂಧ ೩.} ಹಚ್ಚಿಕೊಂಡು ಧ್ಯಾನ ಮಾಡುವುದೂ ಈ ಹೊತ್ತಿನಲ್ಲೇ. ನಾವೂ ಹಾಗೆಯೇ ಧುನಿ ಹೊತ್ತಿಸಿಕೊಂಡು ನಮ್ಮ ಆಸೆ-ಆಕಾಂಕ್ಷೆಗಳನ್ನು ಅದರೊಳಗೆಸೆದು ಸುಟ್ಟುಬಿಡೋಣ!” ಅಂತೆಯೇ ಧುನಿಯನ್ನು ಹಚ್ಚಿಕೊಂಡು ಯುವಶಿಷ್ಯರೆಲ್ಲರೂ ಅದರ ಸುತ್ತ ಕುಳಿತರು. ಧಗಧಗಿಸುತ್ತಿದ್ದ ಆ ಬೆಂಕಿಯನ್ನು ನೋಡುತ್ತ, ‘ನಮ್ಮ ಸಮಸ್ತ ಆಸೆಗಳನ್ನು ಅಗ್ನಿಯಲ್ಲಿ ಆಹುತಿಕೊಟ್ಟು ಹೋಮ ಮಾಡಿಬಿಟ್ಟೆವು; ಈಗ ನಮ್ಮ ಮನಸ್ಸಿನಿಂದ ಎಲ್ಲ ಕುಸಂಸ್ಕಾರಗಳು, ಎಲ್ಲ ಪಾಪಗಳು ಸುಟ್ಟುಹೋಗಿ ಮನಸ್ಸು ಪರಮ ಪರಿಶುದ್ಧವಾಗಿಬಿಟ್ಟಿದೆ’ ಎಂಬ ಭಾವನೆ ಅವರಲ್ಲಿ ತಾನೇತಾನಾಗಿ ಉಂಟಾಯಿತು. ಇದು ಅವರ ಪಾಲಿಗೊಂದು ಹೊಸ, ಉತ್ತೇಜನಕರ ಅನುಭವ.

ನರೇಂದ್ರನಿಗೆ ಶ್ರೀರಾಮಕೃಷ್ಣರು ಹಲವಾರು ಬೋಧನೆಗಳನ್ನು ನೀಜಿ ಹಲವಾರು ಆಧ್ಯಾತ್ಮಿಕ ಅನುಭವಗಳನ್ನು ಮಾಡಿಸಿಕೊಟ್ಟಿದ್ದರೂ, ಅವನಿಗೆ ಮಂತ್ರದೀಕ್ಷೆ ಕೊಟ್ಟಿರಲಿಲ್ಲ. ಪ್ರಾಯಶಃ ಈಗ ಅದಕ್ಕೆ ಸೂಕ್ತ ಕಾಲ ಸನ್ನಿಹಿತವಾಗಿತ್ತು. ಈ ದಿನಗಳಲ್ಲೊಮ್ಮೆ ಅವರು ನರೇಂದ್ರನನ್ನು ಕುಳ್ಳಿರಿಸಿಕೊಂಡು ರಾಮಮಂತ್ರವನ್ನು ಉಪದೇಶ ಮಾಡುತ್ತ, “ಇಗೋ, ಇದು ನಿನ್ನ ಇಷ್ಟಮಂತ್ರ. ಇದು ನನಗೆ ನನ್ನ ಗುರುವಿನಿಂದ ಸಿಕ್ಕಿದ ಮಂತ್ರ” ಎಂದು ತಿಳಿಸಿದರು. ಆ ಮಂತ್ರವನ್ನು ಸ್ವೀಕರಿಸಿದಾಗ ಅವನಲ್ಲುಂಟಾದ ಪರಿಣಾಮ ಮಹಾದ್ಭುತವಾಗಿತ್ತು. ಸ್ವತಃ ಆ ಮಂತ್ರವೋ ಸಿದ್ಧಮಂತ್ರ; ಅದನ್ನು ಉಪದೇಶಿಸಿದವರೋ ಮಂತ್ರಸಿದ್ಧರು; ಆ ಮಂತ್ರವನ್ನು ಪಡೆದವನೋ ನಿತ್ಯಸಿದ್ಧ! ಹೀಗಿರುವಾಗ ಅಲ್ಲಿ ಅದ್ಭುತವಾದ ಪರಿಣಾಮವುಂಟಾದುದರಲ್ಲಿ ಆಶ್ಚರ್ಯವೇನಿದೆ? ಮಂತ್ರವನ್ನು ಪಡೆದುಕೊಂಡ ಕೂಡಲೇ ಅವನಲ್ಲಿ ಆಧ್ಯಾತ್ಮಿಕ ಭಾವವನ್ನು ಬಡಿದೆಬ್ಬಿಸಿದಂತಾಯಿತು. ಅಂದು ಸಂಜೆಯೆಲ್ಲ ಅವನು ರಾಮನಾಮವನ್ನು ಗಟ್ಟಿಯಾಗಿ ಉಚ್ಚರಿಸುತ್ತ ಉದ್ಯಾನ ಗೃಹದ ಸುತ್ತ ಪ್ರದಕ್ಷಿಣೆ ಹಾಕಲಾರಂಭಿಸಿದ! ಒಂದೇ ಸಮನೆ ‘ರಾಮ, ರಾಮ’ ಎನ್ನುತ್ತ ಆನಂದೋತ್ಸಾಹಭರಿತನಾಗಿ ಮತ್ತೆಮತ್ತೆ ಸುತ್ತುಬರತೊಡಗಿದ. ನೋಡಿದರೆ ಬಾಹ್ಯಪ್ರಜ್ಞೆಯನ್ನೇ ಕಳೆದುಕೊಂಡಂತೆ ತೋರುತ್ತಿತ್ತು. ಇದನ್ನು ಕಂಡ ಗುರುಭಾಯಿಗಳಿಗೆ ಅತ್ಯಾಶ್ಚರ್ಯ. ಅವರಿಗೆ ಅವನನ್ನು ಸಮೀಪಿಸಲೂ ಧೆರ್ಯ ಸಾಲದು. ಶ್ರೀರಾಮಕೃಷ್ಣರಿಗೆ ಈ ವಿಷಯವನ್ನು ತಿಳಿಸಿದಾಗ ಅವರು,“ಇರಲಿ, ಎಲ್ಲಸರಿಹೋಗುತ್ತದೆ” ಎಂದುತ್ತರಿಸಿದರು. ನರೇಂದ್ರ ಅದೇ ಆನಂದಭರಿತ ಸ್ಥಿತಿಯಲ್ಲಿ ರಾಮಮಂತ್ರವನ್ನುಚ್ಚರಿಸುತ್ತ ಕೆಲವು ಗಂಟೆಗಳ ಕಾಲ ಮತ್ತೆಮತ್ತೆ ಪ್ರದಕ್ಷಿಣೆ ಬರುತ್ತಲೇ ಇದ್ದ. ಆಮೇಲೆ ಅವನ ಉನ್ಮತ್ತತೆ ನಿಧಾನವಾಗಿ ಇಳಿಯುತ್ತ ಬಂತು. ಬಳಿಕ ಅವನು ಎಂದಿನಂತಾದ.

ಕಾಶೀಪುರದ ಉದ್ಯಾನಗೃಹ ಈ ಶಿಷ್ಯರ ಪಾಲಿಗೆ ಒಂದು ದೇವಾಲಯವೂ ಆಗಿದೆ; ಒಂದು ವಿಶ್ವವಿದ್ಯಾನಿಲಯವೂ ಆಗಿದೆ. ಕೆಲವೊಮ್ಮೆ ಅಲ್ಲಿ ತತ್ತ್ವಶಾಸ್ತ್ರಗಳ ಅಧ್ಯಯನ ನಡೆದರೆ, ಇನ್ನು ಕೆಲವೊಮ್ಮೆ ಭಕ್ತಿಭಾವದ ಹೊನಲು ಹರಿಯುತ್ತದೆ. ನರೇಂದ್ರನೂ ಅವನ ಸ್ನೇಹಿತರೂ ಸೇರಿ ಭಜನೆ-ಪ್ರಾರ್ಥನೆ-ಧ್ಯಾನ ಮಾಡುತ್ತಾರೆ; ವೇದೋಪನಿಷತ್ತುಗಳ ಕುರಿತಾಗಿ ಸಂಭಾಷಣೆ ನಡೆಸುತ್ತಾರೆ. ಶಿಷ್ಯರೆಲ್ಲ ಹೀಗೆ ಸಾಧನೆ ಮಾಡಿ ಭಗವಂತನ ಕೃಪೆಯನ್ನು ಬೇಡಿದಾಗ, ಆ‘ದೇವಲಾಯ’ದ ದೇವಮೂರ್ತಿ ಶ್ರೀರಾಮಕೃಷ್ಣರು ಅನುಗ್ರಹ ಮಾಡುವ ರೀತಿ ಅಪೂರ್ವ! ನೋಡುವವರ ಕಣ್ಣಿಗೆ ರೋಗಗ್ರಸ್ತ ದೀನ ವ್ಯಕ್ತಿಯಾಗಿ ಕಂಡುಬಂದರೂ ಅರು ಸಕಲರಿಗೂ ಆನಂದವನ್ನು ಕರುಣಿಸಬಲ್ಲ ಆನಂದಮೂರ್ತಿ. ಆ ಕರುಣಾಮಯನ ಕೃಪಾಪ್ರವಾಹ ವಿಶೇಷವಾಗಿ ಹರಿದು, ಕೃಪಾಕಾಂಕ್ಷಿಗಳಾದ ಭಕ್ತರೆಲ್ಲ ಆ ಪ್ರವಾಹದಲ್ಲಿ ಮಿಂದು ಪುನೀತರಾದಂತಹ ರೋಮಾಂಚಕಾರಿಯಾದ ಘಟನೆಯೊಂದು ಕಾಶೀಪುರದ ತೋಟದ ಮನೆಯಲ್ಲಿ ನಡೆಯಿತು.

ಅಂದು ೧೮೮೬ನೇ ಇಸವಿಯ ಜನವರಿ ಒಂದನೇ ತಾರೀಕು. ಕೆಲದಿನಗಳ ಹಿಂದೆ ತಾನೆ ಕಲ್ಕತ್ತದ ಪ್ರಸಿದ್ಧ ಹೋಮಿಯೋಪತಿ ವೈದ್ಯರಾದ ರಾಜೇಂದ್ರಲಾಲ್ ದತ್ತ ಎಂಬುವರು ಶ್ರೀರಾಮಕೃಷ್ಣರ ದೇಹಸ್ಥಿತಿಯನ್ನು ಚೆನ್ನಾಗಿ ಪರೀಕ್ಷೆ ಮಾಡಿ ಹೊಸದೊಂದು ಔಷಧಿಯನ್ನು ಕೊಟ್ಟಿದ್ದರು. ಇದರ ಪರಿಣಾಮವಾಗಿ ಕಾಯಿಲೆ ಎಷ್ಟೋ ಉಪಶಮನವಾದಂತಿತ್ತು. ಹೀಗೆ ದೇಹಸ್ಥಿತಿ ಸ್ವಲ್ಪ ಸುಧಾರಿಸಿದ್ದರಿಂದ ಶ್ರೀರಾಮಕೃಷ್ಣರು ಉದ್ಯಾನವನದಲ್ಲಿ ಮೂರು ಗಂಟೆಯ ಸಮಯ. ಯುವಶಿಷ್ಯರು ರಾತ್ರಿಯ ವೇಳೆಯಲ್ಲಿ ಸೇವಾಕಾರ್ಯಗಳಲ್ಲೂ ಸಾಧನೆಗಳಲ್ಲೂ ನಿರತರಾಗಿ ಬಹಳ ಹೊತ್ತು ಎದ್ದಿರುತ್ತಿದ್ದುದರಿಂದ ಈಗ ಕೆಳಗೆ ಹಜಾರದಲ್ಲಿ ವಿಶ್ರಮಿಸುತ್ತಿದ್ದರು. ಜನವರಿ ಒಂದನೇ ತಾರೀಕು ಇಂಗ್ಲಿಷ್ ವರ್ಷದ ಮೊದಲನೇ ದಿನ; ಆದ್ದರಿಂದ ಅಂದು ಎಲ್ಲರಿಗೂ ರಜಾದಿಸ. ಹಲವಾರು ಜನ ಗೃಹಸ್ಥರು ಶ್ರೀರಾಮಕೃಷ್ಣರ ದರ್ಶನಕ್ಕಾಗಿ ಕಾಶೀಪುರಕ್ಕೆ ಬಂದಿದ್ದರು. ಗಿರೀಶಚಂದ್ರನೂ ಬಂದಿದ್ದ; ಅಲ್ಲೇ ಮಾವಿನಮರದ ಬುಡದಲ್ಲಿ ಕುಳಿತುಕೊಂಡು ರಾಮಚಂದ್ರ ದತ್ತ ಹಾಗೂ ಇನ್ನಿತರ ಭಕ್ತರೊಂದಿಗೆ ಮಾತನಾಡುತ್ತಿದ್ದ. ಆಗ ಅಲ್ಲಿಗೆ ಶ್ರೀರಾಮಕೃಷ್ಣರು ಕೆಲವು ಭಕ್ತರೊಡನೆ ಮೆಲ್ಲಗೆ ನಡೆದು ಬಂದರು. ಗಿರೀಶನನ್ನು ಕಂಡು ಇದ್ದಕ್ಕಿದ್ದಂತೆ ನುಡಿದರು:

“ಗಿರೀಶ್, ನೀನು ನನ್ನ ವಿಷಯವಾಗಿ ಎಲ್ಲರೆದುರಿನಲ್ಲೂ ಏನೇನೂ (ಅವರು ಅವತಾರ ಪುರುಷರು ಎಂದು ) ಸಾರುತ್ತಿದ್ದೀಯಲ್ಲ, ನೀನು ಅಂಥಾದ್ದೇನನ್ನು ಕಂಡೆ ನನ್ನಲ್ಲಿ?”

ಶ್ರೀರಾಮಕೃಷ್ಣರ ದೈವತ್ವದಲ್ಲಿ ಅಪಾರ ಶ್ರದ್ಧೆಯಿಟ್ಟಿದ್ದ ಗಿರೀಶ ತಕ್ಷಣ ಕೈ ಜೋಡಿಸಿ, ಭಾವಭರಿತನಾಗಿ ಗದ್ಗದ ಸ್ವರದಿಂದ ಉದ್ಗರಿಸಿದ:

“ಯಾರ ಮಹಿಮೆಯನ್ನು ವರ್ಣಿಸುವಲ್ಲಿ ವ್ಯಾಸ-ವಾಲ್ಮೀಕಿಗಳೂ ಅಸಮರ್ಥರಾಗಿದ್ದಾರೋ ಅಂತಹ ಮಹಾಮಹಿಮನ ವಿಚಾರವಾಗಿ ನನ್ನಂಥವನು ಏನುತಾನೆ ಹೇಳಬಲ್ಲ!

ಗಿರೀಶಚಂದ್ರನ ಮಾತು ನಿಜಕ್ಕೂ ಎಷ್ಟು ಅದ್ಭುತ! ಹಿಂದೆ ವಾಲ್ಮೀಕಿ-ವ್ಯಾಸ ಮಹರ್ಷಿಗಳು ರಾಮಾಯಣ-ಮಹಾಭಾರತಗಳೆಂಬ ಅಮರಗ್ರಂಥಗಳಲ್ಲಿ ಭಗವಾನೇ ವಿಷ್ಣುವಿನ ಅವತಾರಗಳಾದ ಶ್ರೀರಾಮನ ಮತ್ತು ಶ್ರೀಕೃಷ್ಣನ ಗುಣ-ಮಹಿಮೆಗಳನ್ನು ನಾನಾವಿಧವಾಗಿ ವರ್ಣಿಸಿ ದಣಿದರೂ ಆ ಗುಣಕಥನ ಮುಗಿಯಲಿಲ್ಲ. ಅಂತಹ ರಾಮ-ಕೃಷ್ಣರೇ ಈಗ ಶ್ರೀರಾಮಕೃಷ್ಣರಾಗಿ ಅವತಾರ ತಾಳಿರುವಾಗ ಅವರ ಮಹಿಮೆಯನ್ನು ತನ್ನಂಥ ಪಾಮರನೇನು ವರ್ಣಿಸಬಲ್ಲ?– ಎಂಬುದು ಅವನ ಮಾತಿನ ಅಭಿಪ್ರಾಯ. ಅದನ್ನು ಕೇಳಿದ ತಕ್ಷಣ ಶ್ರೀರಾಮಕೃಷ್ಣರಲ್ಲಿ ಭಾವವುಕ್ಕಿ ಬಂದು ಗಾಢ ಸಮಾಧಿಸ್ಥರಾಗಿ ಬಿಟ್ಟರು. ಗಿರೀಶ ಇದನ್ನು ಕಂಡು ಆನಂದಭರಿತನಾಗಿ “ಜೈ ರಾಮಕೃಷ್ಣ! ಜೈ ರಾಮಕೃಷ್ಣ!” ಎಂದು ಘೋಷಿಸುತ್ತ ಮತ್ತೆಮತ್ತೆ ಅವರ ಪಾದಧೂಳಿಯನ್ನು ತೆಗೆದುಕೊಂಡು ಹಣೆಗಿಟ್ಟುಕೊಂಡ. ಆಗ ಶ್ರೀರಾಮಕೃಷ್ಣರು ಅರೆಸಹಜಾವಸ್ಥೆಗಿಳಿದು ಮಂದಸ್ಮಿತವದನರಾಗಿ “ ನಾನಿನ್ನೇನು ತಾನೆ ಹೇಳಲಿ! ನಿಮಗೆಲ್ಲರಿಗೂ ಆಧ್ಯಾತ್ಮಿಕ ಜಾಗೃತಿಯುಂಟಾಗಲಿ!” ಎಂದುದ್ಗರಿಸಿದರು. ಸುತ್ತ ನೆರೆದಿದ್ದ ಭಕ್ತಾದಿಗಳೆಲ್ಲ ಶ್ರೀರಾಮಕೃಷ್ಣರ ಆಶೀರ್ವಾದರೂಪದ ಈ ವಾಣಿಯನ್ನು ಕೇಳಿ ಆನಂದೋದ್ರೇಕದಿಂದ “ಜೈ ರಾಮಕೃಷ್ಣ, ಜೈ ರಾಮಕೃಷ್ಣ!” ಎಂದು ಉದ್ಗಾರ ಮಾಡುತ್ತ ಅವರಿಗೆ ಸಾಷ್ಟಾಂಗ ಪ್ರಣಾಮ ಮಾಡಿದರು. ಕೆಲವರು ಉದ್ಯಾನದಿಂದ ಹೂವುಗಳನ್ನು ಕಿತ್ತುತಂದು ಅವರ ಪಾದಗಳಿಗೆ ಸಮರ್ಪಿಸಿ ಪೂಜಿಸಿದರು. ಇನ್ನು ಕೆಲವರು ಸ್ತೋತ್ರ-ಮಂತ್ರಗಳನ್ನು ಉಚ್ಚರಿಸುತ್ತ ಅರ ಪಾಧೂಳಿಯನ್ನು ತೆಗೆದುಕೊಂಡು. ಶ್ರೀರಾಮಕೃಷ್ಣರು ತಮ್ಮ ಬಳಿಗೆ ಬಂದ ಪ್ರತಿಯೊಬ್ಬನ ಎದೆಯನ್ನೂ ಮುಟ್ಟಿ ‘ನಿನಗೆ ಆತ್ಮಜಾಗೃತಿಯುಂಟಾಗಲಿ” ಎಂದು ಹರಸಿದರು. ಅವರ ಆ ದಿವ್ಯಸ್ಪರ್ಶದಿಂದಾಗಿ ಪ್ರತಿಯೊಬ್ಬರಿಗೂ ಒಂದೊಂದು ಬಗೆಯ ಅದ್ಭುತ ಆಧ್ಯಾತ್ಮಿಕಾನುಭವಾಯಿತು! ಕೆಲವರು ಆ ಆನಂದಾನುಭವದ ಉತ್ಕಟತೆಯನ್ನು ತಾಳಲಾರದೆ ಕುಸಿದುಬಿದ್ದರೆ, ಇನ್ನು ಕೆಲವರು ಹಾಡುತ್ತ ನರ್ತಿಸಲಾರಂಭಿಸಿದರು! ನಿಮಿಷಾರ್ಧದಲ್ಲಿ ಅಲ್ಲೊದು ಆನಂದದ ಸಂತೆಯೇ ನಿರ್ಮಾಣವಾಯಿತು. ಹೀಗೆ ಭಕ್ತರೆಲ್ಲ ತಮ್ಮ ಮನೋರಥಗಳನ್ನು ಪೂರೈಸುವ ಕಲ್ಪತರುವಿನ ಕೃಪೆಗೆ ಪಾತ್ರರಾದರು. (ಇಂದಿಗೂ ಜನವರಿ ಒಂದರಂದು ಕಾಶೀಪುರದ ಆ ದಿವ್ಯ ಉದ್ಯಾನದಲ್ಲಿ ಲಕ್ಷಗಟ್ಟಲೆ ಜನ ಸೇರಿ ಅತ್ಯುತ್ಸಾಹದಿಂದ ‘ಕಲ್ಪತರು ದಿನಾಚರಣೆ’ಯನನು ನೆರವೇರಿಸುತ್ತಾರೆ.)

ತಾನು ಕೆಲವು ವರ್ಷಗಳವರೆಗಾದರೂ ವಕೀಲನಾಗಿ ದುಡಿದು, ಚೆನ್ನಾಗಿ ಹಣ ಸಂಪಾದಿಸಿ, ತನ್ನ ಮನೆಮಂದಿಯ ಜೀವನೋಪಾಯಕ್ಕಾಗಿ ವ್ಯವಸ್ಥೆ ಮಾಡಿಟ್ಟು, ಅನಂತರ ಸಂಸಾರತ್ಯಾಗ ಮಾಡಿ ಸಂನ್ಯಾಸಿಯಾಗಬೇಕು ಎಂಬ ಆಕಾಂಕ್ಷೆ ನರೇಂದ್ರನದಾಗಿತ್ತು. ಆದರೆ ಶ್ರೀರಾಮಕೃಷ್ಣರ ದಿವ್ಯ ಸಾನ್ನಿಧ್ಯದಲ್ಲಿ ಬದುಕುತ್ತ, ದಿನ ಕಳೆದಂತೆ ಅವನ ಮನೋಭಾವದಲ್ಲಿ ತೀವ್ರ ಬದಲಾವಣೆಯುಂಟಾಗಿಬಿಟ್ಟಿತು. ಪ್ರಾಪಂಚಿಕ ಬಂಧನಗಳನ್ನೆಲ್ಲ ಕಡಿದುಕೊಂಡು ಧ್ಯಾನ-ಸಮಾಧಿಯ ಆನಂದದಲ್ಲಿ ಮುಳುಗಿಬಿಡಲು ಹಾತೊರೆಯಲಾರಂಭಿಸಿದ. ಅದನ್ನು ಗುರುತಿಸಿದ ಶ್ರೀರಾಮಕೃಷ್ಣರು ಒಂದು ದಿನ (ಜನವರಿ ನಾಲ್ಕರಂದು) ಅವನನ್ನು ಅರ್ಧಹಾಸ್ಯವಾಗಿ ಕೇಳುತ್ತಾರೆ: “ಏನು, ನಿನ್ನ ಅಧ್ಯಯನವನ್ನು ಮುಂದುವರಿಸುವುದಿಲ್ಲವೆ?” ಆಗ ಅವನು ಗಂಭೀರವಾಗಿ ಉತ್ತರಿಸುತ್ತಾನೆ: “ಮಹಾಶಯರೇ, ನಾನು ಈಗಾಗಲೇ ಓದಿರುವುದನ್ನೆಲ್ಲ ಮರೆತುಬಿಡುವಂತೆ ಮಾಡುವ ಔಷಧಿಯೇನಾದರೂ ಇದ್ದಿದ್ದರೆ ಅದನ್ನು ತೆಗೆದುಕೊಂಡು ನೆಮ್ಮದಿಯಿಂದಿರುತ್ತಿದ್ದೆ!”

ಅದೇ ದಿನ ಸಾಯಂಕಾಲ ಮಹೇಂದ್ರನಾಥನ ಗುಪ್ತನ ಜೊತೆಯಲ್ಲಿ ಏಕಾಂತದಲ್ಲಿ ಮಾತನಾಡುತ್ತಿರುವಾಗ ಅವನು ತನ್ನ ಚಡಪಡಿಕೆಯನ್ನು ತೋಡಿಕೊಳ್ಳುತ್ತಾನೆ:

ನರೇಂದ್ರ:“ ಕಳೆದ ಶನಿವಾರ (ಜನವರಿ ೨, ೧೮೮೬) ಧ್ಯಾನ ಮಾಡುತ್ತಿದ್ದಾಗ ನನ್ನ ಎದೆಯಲ್ಲಿ ಏನೋ ಒಂದು ವಿಚಿತ್ರ ಸಂವೇದನೆಯ ಅನುಭವವಾಯಿತು.”

ಮಹೇಂದ್ರನಾಥ: “ಕುಂಡಲಿನಿ ಜಾಗೃತವಾಗಿದ್ದಿರಬೇಕು. ”

ನರೇಂದ್ರ:“ ಇರಬಹುದೇನೋ. ನಾನಾ ಇಡಾ-ಪಿಂಗಳಾ ನಾಡಿಗಳ ಮಿಡಿತವನ್ನು ಸ್ಪಷ್ಟವಾಗಿ ಅನುಭವಿಸಿದೆ. ಆಗ ಅಲ್ಲಿದ್ದ ಹಾಜರಾನಿಗೆ ಹೇಳಿದೆ–ನನ್ನ ಎದೆಯನ್ನು ಮುಟ್ಟಿನೋಡು ಅಂತ. ನಿನ್ನೆ ಶ್ರೀರಾಮಕೃಷ್ಣರಿಗೂ ಈ ವಿಷಯವನ್ನೂ ತಿಳಿಸಿದೆ. ಬಳಿಕ ಹೇಳಿದೆ–‘ಅಂದು ಒಂದನೇ ತಾರೀಖಿನಂದು, ಎಲ್ಲರಿಗೂ ಸಾಕ್ಷಾತ್ಕಾರದ ಅನುಭವವಾಯಿತು. ಈಗ ನನಗೂ ಸ್ವಲ್ಪ ಸಾಕ್ಷಾತ್ಕಾರ ಮಾಡಿಸಿಕೊಡಿ. ಎಲ್ಲರಿಗೂ ಸಿದ್ಧಿಯಾಯಿತು. ನನಗೆ ಮಾತ್ರ ಏನೂ ಇಲ್ಲವೆ?’ ಎಂದು. ಅದಕ್ಕೆ ಅವರು, ‘ಮೊದಲು ನಿನ್ನ ಮನೆಯವರಿಗೆ ಒಂದು ವ್ಯವಸ್ಥೆ ಮಾಡಿಟ್ಟು ಬಾ. ಆಮೇಲೆ ನಿನಗೇನೇನು ಬೇಕೋ ಎಲ್ಲವೂ ದೊರೆಯುತ್ತದೆ. ನಿನಗೀಗ ಬೇಕಾದುದಾದರೂ ಏನು ಹೇಳು?’ ಎಂದರು. ನಾನೆಂದೆ–‘ನನಗೆ ಒಟ್ಟಿಗೆ ಮೂರು-ನಾಲ್ಕು ದಿನಗಳಾದರೂ ಸತತ ಸಮಾಧಿಸ್ಥಿತಿಯಲ್ಲೇ ಇದ್ದುಬಿಡಬೇಕೆನ್ನುವ ಆಸೆ. ಯಾವಾಗಲಾದರೂ ಒಂದೊಂದು ಸಲ ಶರೀರರಕ್ಷಣೆಗಾಗಿ ಸ್ವಲ್ಪ ಆಹಾರ ತೆಗೆದುಕೊಳ್ಳುವುದಕ್ಕೋಸ್ಕರ ಮಾತ್ರ ಸಮಾಧಿಯಿಂದ ಇಳಿದು ಬರುವಂತಾದರೆ ಸಾಕು’ ಅಂತ. ಆಗ ಅವರು,‘ಎಂಥಾ ಸಣ್ಣ ಬುದ್ಧಿ ನಿನ್ನದು! ಆ ಸಮಾಧಿಸ್ಥಿತಿಗಿಂತಲೂ ಉನ್ನತವಾದ ಒಂದು ಸ್ಥಿತಿ ಇದೆ. “ಜಗತ್ತಿನಲ್ಲಿರುವುದೆಲ್ಲ ಆ ಭಗವಂತನೊಬ್ಬನೇ” ಎಂಬರ್ಥದ ಹಾಡನ್ನು ಯಾವಾಗಲೂ ಹೇಳುತ್ತಿರುತ್ತೀಯ! ಮೊದಲು ನಿನ್ನ ಮನೊಯ ವ್ಯವಹಾರಗಳನ್ನೆಲ್ಲ ಮುಗಿಸಿಕೊಂಡು, ಅಮೇಲೆ ಇಲ್ಲಿಗೆ ಬಾ. ಸಮಾಧಿಗಿಂತಲೂ ಎತ್ತರದ ಸ್ಥಿತಿಗೇರುವಂತೆ’ ಎಂದು ಛೇಡಿಸಿ ಬುದ್ಧಿ ಹೇಳಿದರು.

“ಇವತ್ತು ಬೆಳಗ್ಗೆ ಮನೆಗೆ ಹೋಗಿದ್ದೆ. ಮನೆಯಲ್ಲಿ ಎಲ್ಲರೂ ನನ್ನನ್ನು ಬೈದರು. ‘ಯಾಕೆ ಸುಮ್ಮನೆ ನಿಷ್ಪ್ರಯೋಜಕನಂತೆ ಅಲೆದಾಡುತ್ತಿದ್ದೀಯ? ನಿನ್ನ ‘ಲಾ’ ಪರೀಕ್ಷೆ ಸಮೀಪಿಸುತ್ತಿದೆ; ಆದರೆ ನೀನು ಮಾತ್ರ ಓದಿನ ಕಡೆಗೆ ಸ್ವಲ್ಪವೂ ಗಮನ ಕೊಡುತ್ತಿಲ್ಲವಲ್ಲ? ಸುಮ್ಮನೆ ಗೊತ್ತುಗುರಿಯಿಲ್ಲದೆ ಸುತ್ತಾಡುತ್ತಿದ್ದೀಯ!” ಅಂತ ಛೀಮಾರಿ ಹಾಕಿದರು. ನಾನು ಪುಸ್ತಕಗಳನ್ನು ಹಿಡಿದುಕೊಂಡು ಓದುವುದಕ್ಕೆ ಅಂತ ಅಜ್ಜಿಯ ಮನೆಗೆ ಹೋದೆ. ಪುಸ್ತಕ ತೆರೆದು ಓದಲು ಪ್ರಯತ್ನಪಟ್ಟೆ. ಆದರೆ ಯಾಕೋ ನನ್ನನ್ನು ಒಂದು ವಿಚಿತ್ರವಾದ ಭಯ ಅವರಿಸಿಕೊಂಡಿತು. ಇದನ್ನು ಓದುವುದೇ ಒಂದು ಭಯಂಕರ ಅಪರಾಧ ಅಂತ ಅನ್ನಿಸಿಬಿಟ್ಟಿತು. ನನ್ನ ಎದೆಯೊಳಗೆಲ್ಲ ಏನೋ ತಳಮಳ! ಒಂದೇ ಸಮನೆ ಆಳತೊಡಗಿದೆ. ನನ್ನ ಜೀವಮಾನದಲ್ಲೇ ಎಂದೂ ಆ ರೀತಿ ಅತ್ತಿರಲಿಲ್ಲ ನಾನು. ಪುಸ್ತಕಗಳನ್ನೆಲ್ಲ ಅಲ್ಲೇ ಎಸೆದು ಅಲ್ಲಿಂದೆದ್ದು ಓಡಿದೆ. ಆ ರಭಸಕ್ಕೆ ಪಾದರಕ್ಷೆಗಳು ದಾರಿಯಲ್ಲೆಲ್ಲೋ ಕಳಚಿ ಬಿದ್ದುಹೋದುವು. ಹುಲ್ಲುಬಣವೆಗಳ ಸಂದಿಗೊಂದಿಯಲ್ಲೆಲ್ಲ ನುಗ್ಗಿದ್ದರಿಂದ ತಲೆ ಮೈ ಕೈಗಲ್ಲ, ಹುಲ್ಲುಗರಿಗಳು ಸಿಕ್ಕಿಕೊಂಡುಬಿಟ್ಟವು. ಒಂದೇ ಸಮನೆ ಓಡಿ, ಇಲ್ಲಿಗೆ ಬಂದು ನಿಂತೆ.”

ನರೇಂದ್ರನ ಚಡಪಡಿಕೆಯನ್ನೂ ವೈರಾಗ್ಯದ ತೀವ್ರತೆಯನ್ನೂ ಗಮನಿಸಿ ಮಹೇಂದ್ರನಾತ ಅವಾಕ್ಕಾಗಿ ನಿಂತಿದ್ದ. ನರೇಂದ್ರ ತನ್ನ ಮಾತನ್ನು ಮುಂದುವರಿಸಿದ: “ವಿವೇಕಚೂಡಾಮಣಿಯನ್ನು ಓದಿದ ಮೇಲಂತೂ ನನ್ನ ಮನಸ್ಸು ಪೂರ್ತಿ ಕಲಕಿಹೋಗಿ ಬಿಟ್ಟಿದೆ. ಅದರಲ್ಲಿ ಶಂಕರಾಚಾರ್ಯರು ಹೇಳಿದ್ದಾರೆ: ಮನುಷ್ಯಜನ್ಮ, ಮುಮುಕ್ಷುತ್ವ ಮತ್ತು ಮಹಾಪುರುಷರ ಸಂಸರ್ಗ–ಇವು ಮೂರು ಲಭಿಸಬೇಕಾದರೆ, ಕಠಿಣ ತಪಸ್ಸಿನ ಮತ್ತು ಪೂರ್ವಜನ್ಮದ ಸತ್ಕರ್ಮಗಲ ಫಲದಿಂದ ಮಾತ್ರವೇ ಸಾಧ್ಯ ಅಂತ. ನನಗೆ ನಾನೇ ಹೇಳಿಕೊಂಡೆ: ‘ನನಗೆ ಇವು ಮೂರೂ ದೊರೆತಿವೆ–ಮಹಾತಪಸ್ಸಿನ ಬಲದಿಂದ ನನಗೆ ಮನುಷ್ಯಜನ್ಮ ದೊರೆತಿದೆ; ನನ್ನ ಹೃದಯ ಮುಕ್ತಿಗಾಗಿ ತವಕಿಸುತ್ತಿದೆ. ಇಂಥ ಮಹಾಪುರುಷರ ಸಂಶ್ರಯವೂ ದೊರಕಿದೆ’ ಅಂತ... ಸಂಸಾರಜೀವನ ನನಗೆ ರುಚಿಸುತ್ತಿಲ್ಲ. ಒಬ್ಬಿಬ್ಬರನ್ನು ಬಿಟ್ಟರೆ ಉಳಿದ ಸಂಸಾರಿಗಳ ಸಹವಾಸವೂ ನನಗೆ ಹಿಡಿಸುತ್ತಿಲ್ಲ... ನನ್ನ ಜೀವ ಒಂದೇ ಸಮನೆ ತಳಮಳಗುಟ್ಟುತ್ತಿದೆ. ”

ಅದೇ ದಿನ ರಾತ್ರಿ ಸುಮಾರು ಒಂಬತ್ತು ಗಂಟೆಯ ಹೊತ್ತಿಗೆ ನಿರಂಜನ ಮತ್ತು ಶಶಿಭೂಷಣ ಶ್ರೀರಾಮಕೃಷ್ಣರ ಬಳಿ ಕುಳಿತಿದ್ದರು. ಶ್ರೀರಾಮಕೃಷ್ಣರು ಮಾತುಮಾತಿಗೂ ನರೇಂದ್ರನ ವಿಷಯವಾಗಿ ಹೇಳುತ್ತಿದ್ದರು:“ಈಗೀಗ ಅವನ ಮನಸ್ಥಿತ ಎಷ್ಟು ಅದ್ಭುತವಾಗಿದೆ ನೋಡಿ! ಇದೇ ನರೇಂದ್ರ ಹಿಂದೆ ದೇವರ ಸಾಕಾರತ್ವವನ್ನು ಸುತರಾಂ ನಂಬುತ್ತಿರಲಿಲ್ಲ. ಆದರೆ ಈಗ ನೋಡಿ, ದೇವರಿಗಾಗಿ ಎಷ್ಟು ವ್ಯಾಕುಲನಾಗಿದ್ದಾನೆ! ಭಗವಂತನಿಗಾಗಿ ನಿಮ್ಮ ಜೀವ ತೀವ್ರ ವ್ಯಾಕುಲಗೊಂಡಾಗ, ಆತನ ದರ್ಶನಕ್ಕಾಗಿ ನೀವಿನ್ನು ಹೆಚ್ಚು ದಿನ ಕಾಯಬೇಕಾಗಿಲ್ಲ. ದಿಗಂತದಲ್ಲಿ ಅರುಣೋದಯವಾದ ಮೇಲೆ ಸೂರ್ಯ ಉದಯಿಸಲು ಇನ್ನು ಹೆಚ್ಚು ಹೊತ್ತೇನೂ ಇಲ್ಲ. ”

ಈ ಮಾತಿನ ಮೂಲಕ ಅವರು, ನರೇಂದ್ರ ಶೀಘ್ರದಲ್ಲೇ ತನ್ನ ಗುರಿಯನ್ನು ಸಿದ್ಧಿಸಿಕೊಳ್ಳುವವನಿದ್ದಾನೆ ಎಂಬುದನ್ನು ಸೂಚಿಸುತ್ತಿದ್ದಾರೆ. 

ಸಮಾಧಿಸ್ಥಿತಿಗಾಗಿ ನರೇಂದ್ರ ತೀವ್ರವಾಗಿ ಹಂಬಲಿಸುತ್ತಿರುವಾಗ ಶ್ರೀರಾಮಕಷ್ಣರು ಅವನ ಆ ಬಯಕೆಯನ್ನು ಪ್ರೋತ್ಸಾಹಿಸದಿರುವುದನ್ನು ಕಂಡು ನಮಗೆ ಅಚ್ಚರಿಯೆನಿಸಬಹುದು. ನಿಜಕ್ಕೂ ಅವನ ಹಂಬಲಿಕೆಯಲ್ಲಿ ಅಸಹಜವಾದದ್ದೇನೂ ಇಲ್ಲ. ಪ್ರಾಮಾಣಿಕರಾದ ಆಧ್ಯಾತ್ಮಸಾಧಕರು ಸಮಾಧಿಯ ಮೂಲಕ ಭಗವದಾನಂದದಲ್ಲಿ ಮುಳುಗಿರಲು ಹಂಬಲಿಸುವುದನ್ನು ಯುಗಯುಗಗಳಿಂದಲೂ ಕಾಣಬಹುದು. ಆದರೆ ನರೇಂದ್ರ ಜನ್ಮ ತಾಳಿರುವುದು ಕೇವಲ ಸಮಾಧಿಯನ್ನು ಗಳಿಸುವುದಕ್ಕಲ್ಲ. ಅವನು ಕೇವಲ ಸಿದ್ಧಪುರುಷನಾಗಿಬಿಟ್ಟರೆ ಸಾಲದು. ಅವನು ಜಗದೋದ್ಧಾರಕನೂ ಆಗಬೇಕಾಗಿದೆ. ಮಾಯಾಸಾಗರವನ್ನು ತಾನೊಬ್ಬನೇ ದಾಟಿಬಿಟ್ಟರೆ ಸಾಲದು, ಇತರರೂ ದಾಟಲು ಅವನು ನೆರವಾಗಬೇಕಾಗಿದೆ. ಆದ್ದರಿಂದ ಅವನು ತಾನೊಬ್ಬ ಮಾತ್ರ ಮುಕ್ತಿಯನ್ನು ಸಂಪಾದಿಸಿಕೊಂಡುಬಿಟ್ಟರೆ ಅದರಲ್ಲೇನೂ ಹಿಚ್ಚುಗಾರಿಕೆಯಿಲ್ಲ. ಅದರಲ್ಲೇನೂ ಮಹತ್ವವಿಲ್ಲ. ನರೇಂದ್ರ ಮಾಡಿಕೊಳ್ಳಬೇಕಾದ ಆಧ್ಯಾತ್ಮಿಕ ಅನುಭವವು ಎಷ್ಟು ಉನ್ನತವಾದುದೋ, ಅವನು ಮಾಡಬೇಕಾಗಿರುವ ಕಾರ್ಯವೂ ಅಷ್ಟೇ ವಿಶಾಲವಾದುದು. ಅಲ್ಲದೆ ತನಗೆ ಸಮಾಧಿಸ್ಥಿತಿಯನ್ನು ದೊರಕಿಸಿಕೊಡುವಂತೆ ಶ್ರೀರಾಮಕೃಷ್ಣರನ್ನು ಕೇಳಿಕೊಂಡಾಗ ಅವರು ಅವನಿಗೆ ಅದಕ್ಕಿಂತಲೂ ಉನ್ನತವಾದ ಸ್ಥಿತಿಯನ್ನು ಸಿದ್ಧಸಿಕೊಡುವುದಾಗಿ ಹೇಳಿದ್ದನ್ನು ನೋಡಿದೆವು. ಯಾವುದಿರಬಹುದು ಅದು? ಅದು ವಿಜ್ಞಾನಸ್ಥಿತಿ. ಶ್ರೀರಾಮಕೃಷ್ಣರೇ ಹೇಳುವಂತೆ, ಹಾಲನ್ನು ಕಣ್ಣಿನಿಂದ ನೋಡಿ ಅದರ ಬಾಹ್ಯ ಗುಣಗಳ ಪರಿಚಯ ಮಾಡಿಕೊಳ್ಳುವುದು ಜ್ಞಾನವಾದರೆ, ಹಾಲನ್ನು ಕುಡಿದು ಪುಷ್ಟಿಯಾಗಿ ಬೆಳೆಯುವುದು ವಿಜ್ಞಾನ. ಅಂತೆಯೇ ಸಮಾಧಿಗೇರಿ ತನ್ನೊಳಗೆ ಭಗವಂತನನ್ನು ಕಾಣುವುದ ಜ್ಞಾನ; ಸಮಾಧಿಯಿಂದಿಳಿದು, ಕಣ್ತೆರೆದು, ಸಕಲ ಜೀವಜಂತುಗಳಲ್ಲೂ ಅದೇ ಭಗವಂತನನ್ನು ಕಾಣುವುದೇ ವಿಜ್ಞಾನಸ್ಥಿತಿ. ಆದ್ದರಿಂದಲೇ ನರೇಂದ್ರ ಸಮಾಧಿಸ್ಥಿತಿಯನ್ನು ಬಯಸಿದಾಗ ಅವರು ಅವನಿಗೆ ಅದಕ್ಕಿಂತಲೂ ಉನ್ನತವಾದ ಸ್ಥಿತಿಯ ಅನುಭವ ಮಾಡಿಸಿಕೊಡುವುದಾಗಿ ಹೇಳುತ್ತಾರೆ.

ಈ ನಡುವೆ ನರೇಂದ್ರ ಜಪ-ತಪ-ಧ್ಯಾನಗಳನ್ನು ಇನ್ನಷ್ಟು ತೀವ್ರಗೊಳಿಸಿದ್ದ. ಪ್ರತಿರಾತ್ರಿಯೂ ದಕ್ಷೀಣೇಶ್ವರಕ್ಕೆ ಹೋಗಿ ಪಂಚವಟಿಯಲ್ಲಿ ಧುನಿ ಹೊತ್ತಿಸಿಕೊಂಡು ಧ್ಯಾನನಿರತನಾಗಿಬಿಡುತ್ತಿದ್ದ. ಶ್ರೀರಾಮಕೃಷ್ಣರು ಅವನಿಗೆ ಹಲವಾರು ಆಧ್ಯಾತ್ಮಿಕ ಸಾಧನಾ ವಿಧಾನಗಳನ್ನು ಬೋಧಿಸುತ್ತಿದ್ದರು. ಆ ಸೂಚನೆಗಳನ್ನು ಅನುಸರಿಸುತ್ತ ಅವನು ಯಶಸ್ವಿಯಾಗಿ ಮುಂದುವರಿಯುತ್ತಿದ್ದ.

ಇದರೊಂದಿಗೆ ಅವರು ಮೆಲ್ಲನೆ ನರೇಂದ್ರನನ್ನು ಉಳಿದೆಲ್ಲ ಯುವಶಿಷ್ಯರ ನಾಯಕನನ್ನಾಗಿ ಸಿದ್ಧಗೊಳಿಸುತ್ತಿದ್ದರು. ತಾವೆಲ್ಲರೂ ಭವ್ಯ ಆಧ್ಯಾತ್ಮಿಕ ವ್ಯಕ್ತಿತ್ವಗಳನ್ನು ರೂಪಿಸಿಕೊಂಡು ಸರ್ವಸಂಗ ಪರಿತ್ಯಾಗ ಮಾಡಿ ಶ್ರೀರಾಮಕೃಷ್ಣರ ಉದಾತ್ತ ಸಂದೇಶಗಳನ್ನು ಜಗತ್ತಿನಲ್ಲಿ ಪ್ರಸಾರ ಮಾಡಬೇಕಾಗಿದೆ ಎಂಬುದು ನರೇಂದ್ರನೂ ಸೇರಿದಂತೆ ಆ ಯುವಕರಾರಿಗೂ ಗೊತ್ತಿರಲಿಲ್ಲ. ಶ್ರೀರಾಮಕೃಷ್ಣರ ದಿವ್ಯ ದೃಷ್ಟಿಗೆ ಮಾತ್ರ ಇವೆಲ್ಲವೂ ಹಗಲಿನಷ್ಟು ಸ್ಪಷ್ಟ. ಆದರೆ ಆ ಯುವಕರೇನಾದರೂ ಹರಿಹಂಚಿಹೋದರೆ, ಬಳಿಕ ಅವರೆಲ್ಲ ತಮ್ಮ ಪಾಡಿಗೆ ತಾವು ಸಾಧನೆ ಮಾಡಿ ಎಂತಹ ಮಹಾತ್ಮರೇ ಆದರೂ, ಅವರಿಂದ ಲೋಕಕಲ್ಯಾಣಕಾರ್ಯವೆಂಬುದು ಸಾಧ್ಯವಾಗುವುದಿಲ್ಲ; ಅವರೆಲ್ಲರೂ ಒಗ್ಗಟ್ಟಿನಿಂದಿರಬೇಕಾದರೆ ಅವರೆಲ್ಲರನ್ನೂ ಮನವರಿತು ನಡೆಸಿಕೊಂಡು ಹೋಗಬಲ್ಲ ಒಬ್ಬ ನಾಯಕ ಬೇಕೇಬೇಕು; ಮತ್ತು ಆ ಶಕ್ತಿ-ಸಾಧ್ಯತೆಗಳನ್ನುಳ್ಳವನು ನರೇಂದ್ರನೇ ಎಂಬುದೂ ಅವರಿಗೆ ತಿಳಿದಿತ್ತು. ಆದ್ದರಿಂದ ಮುಂದೆ ಅವನು ಆ ಶಿಷ್ಯರನ್ನೆಲ್ಲ ಸಮರ್ಥ ರೀತಿಯಲ್ಲಿ ಒಂದುಗೂಡಿಸಿ ಮುನ್ನಡೆಸಿ, ಅವರಿಂದ ಮಹಾಕಾರ್ಯಗಳನ್ನು ಸಾಧಿಸುವಂತೆ ಶ್ರೀರಾಮಕೃಷ್ಣರು ಅವನನ್ನು ತಯಾರುಗೊಳಿಸುತ್ತಿದ್ದರು. ಒಂದು ದಿನ ಅವರು ನರೇಂದ್ರನನ್ನು ಕರೆದು, “ನೋಡು, ನಾನು ಇವರನ್ನೆಲ್ಲ ನಿನ್ನ ರಕ್ಷಣೆಯಲ್ಲಿ ಬಿಟ್ಟು ಹೋಗುತ್ತಿದ್ದೇನೆ. ನಾನು ಶರೀರತ್ಯಾಗ ಮಾಡಿದ ಮೇಲೂ ಇವರು ಆಧ್ಯಾತ್ಮಿಕ ಸಾಧನೆ ಮಾಡುವಂತೆ ಮತ್ತು ಮನೆಗಳಿಗೆ ಹಿಂದಿರುಗಿ ಹೋಗದಂತೆ ನೋಡಿಕೋ” ಎಂಬ ಸ್ಪಷ್ಟ ಸೂಚನೆಯನ್ನಿತ್ತರು.

ಅವರೆಲ್ಲರೂ ಮುಂದೆ ಸರ್ವಸಂಗ ಪರಿತ್ಯಾಗ ಮಾಡಿ ಸಂನ್ಯಾಸಿಗಳಾಗಬೇಕು ಎಂಬುದೇ ಶ್ರೀರಾಮಕೃಷ್ಣರ ಉದ್ದೇಶ. ಧರ್ಮಪ್ರಸಾರ ಮಾಡಬೇಕಾದವರಲ್ಲಿ ಇರಬೇಕಾದ ಪ್ರಧಾನ ಯೋಗ್ಯತೆಯೇ ತ್ಯಾಗ. ಜನಸಾಮಾನ್ಯರು ತಾವೇ ಸ್ವತಃ ತ್ಯಾಗ ಮಾಡಲು ಸಿದ್ಧರಾಗಿರದಿದ್ದರೂ ತ್ಯಾಗ ಮಾಡಿದವರನ್ನು ಕಂಡು ಮೆಚ್ಚಿಕೊಳ್ಳುವ ಗುಣ ಅವರಲ್ಲಿರುತ್ತದೆ. ಸ್ವತಃ ತಾವೇ ಭೋಗಿಗಳಾಗಿ ವಿಲಾಸದಲ್ಲೇ ಮುಳುಗಿರುವವರೂ ಕೂಡ ತಮಗೆ ಧರ್ಮವನ್ನು ಬೋಧಿಸುವವರು ತ್ಯಾಗಿಗಳಾಗಿರಬೇಕು ಎಂದು ನಿರೀಕ್ಷಿಸುತ್ತಾರೆ. ಇದು ತೀರ ಸಹಜ. ಭೋಗವನ್ನು ಅನುಭವಿಸುವವನು ಯೋಗದ ಮಾತನಾಡಿದರೆ ಯಾರು ಕೇಳುತ್ತಾರೆ? ಆದ್ದರಿಂದ ಯೋಗವನ್ನು ಬೋಧಿಸುವವರು ತ್ಯಾಗಿಗಳಾಗಿರಬೇಕಾಗುತ್ತದೆ. ಅಲ್ಲದೆ ಜನಸೇವೆ ಮಾಡಬೇಕಾದರೂ ತ್ಯಾಗಿಗಳಾಗಿರಬೇಕಾಗುತ್ತದೆ. ಹೀಗೆ, ತ್ಯಾಗವಿಲ್ಲದೆ ಯಾವ ಮಹಾಕಾರ್ಯವನ್ನೂ ಸಾಧಿಸಲಾಗುವುದಿಲ್ಲ. ಅದರಲ್ಲೂ ಧರ್ಮಪ್ರಸಾರಕ್ಕಂತೂ ತ್ಯಾಗಮಯ ವ್ಯಕ್ತಿತ್ವವೇ ಬೇಕು. ಆದ್ದರಿಂದಲೇ ಶ್ರೀರಾಮಕೃಷ್ಣರು ಪರಿಶುದ್ಧ ಹೃದಯದ ಆ ಯುವಕರನ್ನು ಮನೆಗಳಿಗೆ ಹಿಂದುರುಗಗೊಡದೆ ತ್ಯಾಗಿಗಳನ್ನಾಗಿ ಮಾಡುತ್ತಿದ್ದಾರೆ. 

ಇಲ್ಲಿ ಇನ್ನೊಂದು ಬಹುಮಖ್ಯವಾದ ಅಂಶವನ್ನು ಗಮನಿಸಬೇಕಾದದ್ದಿದೆ. ಏನೆಂದರೆ, ಅಂತಹ ಯೌವನಭರಿತರಾದ, ಶಕ್ತಿಶಾಲಿಗಳಾದ ಯುವಕರನ್ನು ಭೋಗವಿಮುಕ್ತರನ್ನಾಗಿಸಿ ತ್ಯಾಗಿಗಳನ್ನಾಗಿ ಪರಿವರ್ತಿಸಬೇಕಾದರೆ ಶ್ರೀರಾಮಕೃಷ್ಣರ ಆಕರ್ಷಣಾ ಶಕ್ತಿ ಎಷ್ಟು ಪ್ರಬಲವಾಗಿದ್ದಿರಬೇಕು!... ಅಲ್ಲದೆ, ಅವರ ನಿರ್ಯಾಣಾನಂತರ ಆ ಯುವಕರನ್ನು ತ್ಯಾಗಿಗಳನ್ನಾಗಿಯೇ ಉಳಿಸಿ ಹಿಡಿದಿಟ್ಟುಕೊಳ್ಳಬೇಕಾದರೆ ನರೇಂದ್ರನಲ್ಲಿ ಎಂತಹ ಸಾಮರ್ಥ್ಯ ಇದ್ದಿರಬೇಕು! ಪ್ರಾಯಪ್ರಬುದ್ಧರಾದ ಯುವಕರು ತಮ್ಮ ಭವಿಷ್ಯ ಜೀವನದ ಸುಂದರ ಸ್ವಪ್ನಗಳನ್ನು ಕಾಣುತ್ತಿರುವುದೇ ಸಾಮಾನ್ಯ. ಅಲ್ಲದೆ ಅವರಿಗೆ ತಮ್ಮ ತಾಯ್ತಂದೆಯರ, ಕುಟುಂಬವರ್ಗದವರ ಪ್ರೀತಿ-ವಿಶ್ವಾಸದ ಬಂಧನ ಸಾಕಷ್ಟು ಬಲವಾಗಿಯೇ ಇರುತ್ತದೆ. ಹೀಗಿರುವಾಗ, ಶ್ರೀರಾಮಕೃಷ್ಣರು ಆ ಯುವಕರು ಸುಂದರ ಸ್ವಪ್ನಗಳನ್ನೆಲ್ಲ ಚೆದರಿಸಿ, ಸ್ವಜನಪ್ರೇಮದ ಬಂಧನದಿಂದ ಬಿಡಿಸಿ, ತಮ್ಮ ಧರ್ಮಪ್ರಸಾರ ಕಾರ್ಯಕ್ಕೆ ಅವರನ್ನು ನಿಯೋಜಿಸಬೇಕಾದರೆ ಅದೆಂತಹ ಆಕರ್ಷಣೆ, ಅದೆಂತಹ ವ್ಯಕ್ತಿತ್ವ ಶ್ರೀರಾಮಕೃಷ್ಣರದ್ದಾಗಿರಬೇಕು!

ಶ್ರೀರಾಮಕೃಷ್ಣರು ತಮ್ಮ ಶಿಷ್ಯರನ್ನು ಮೆಲ್ಲನೆ ಸಂನ್ಯಾಸಿಜೀವನಕ್ಕೆ ಅಣಿಗೊಳಿಸುತ್ತಿದ್ದರು. ಅವರಲ್ಲಿ ಸೂಕ್ಷ್ಮ ರೂಪದಲ್ಲಿರಬಹುದಾದ ದೊಡ್ಡಸ್ತಿಕೆ, ಜಂಬ ಇವುಗಳನ್ನು ಹೋಗಲಾಡಿಸಿ, ಸಂನ್ಯಾಸದ ಕಷ್ಟ-ಕಾರ್ಪಣ್ಯಗಳ ಪರಿಚಯ ಮಾಡಿಕೊಡುವ ಉದ್ದೇಶದಿಂದ, ಒಂದು ದಿನ ಅವರನ್ನೆಲ್ಲ ಕರೆದು, ಮನೆಮನೆಗೆ ಹೋಗಿ ಭಿಕ್ಷೆ ಪಡೆದುಕೊಂಡು ಬರುವಂತೆ ಆದೇಶಿಸಿದರು.(ಸಾಧು-ಸಂನ್ಯಾಸಿಗಳು ಹೀಗೆ ಭಿಕ್ಷೆ ಬೇಡುವುದು ಭಾರತದಲ್ಲಿ ಅನಾದಿಕಾಲದಿಂದ ಬಂದಿರುವ ಪದ್ಧತಿ. ಇದಕ್ಕೆ ಶಾಸ್ತ್ರಗಳಲ್ಲಿ ‘ಮಧುಕರೀ ಭಿಕ್ಷೆ’\footnote{*ನೋಡಿ: ಅನುಬಂಧ ೪.} ಎಂಬ ಹೆಸರಿದೆ. ) ಕೂಡಲೇ ಆ ಶಿಷ್ಯರೆಲ್ಲ ಉತ್ಸಾಹದಿಂದ ಸಿದ್ಧರಾದರು. ಅವರೆಲ್ಲ ಒಳ್ಳೇ ವಿದ್ಯಾವಂತರು; ಅನುಕೂಲಸ್ಥ ಕುಟುಂಬಗಳಿಂದ ಬಂದವರು. ಆದರೂ ನಾಚಿಕೆ-ಕಸಿವಿಸಿ ಪಟ್ಟುಕೊಳ್ಳದೆ, ಭಿಕ್ಷಾಪಾತ್ರೆ ಹಿಡಿದು ಹೊರಟು ಕಾಶೀಪುರದ ಕೆಲವು ಮನೆಗಳ ಮುಂದೆ ನಿಂತು ‘ನಾರಾಯಣ ಹರಿ!’ ಎನ್ನುತ್ತ ಭಿಕ್ಷೆ ಯಾಚಿಸಿದರು. ಆದರೆ ಒಂದನೆಯದಾಗಿ, ಅವರಾರೂ ಕಾವಿ ಧರಿಸಿರಲಿಲ್ಲ. ಎರಡನೆಯದಾಗಿ, ಗಟ್ಟಿ ಮುಟ್ಟಾದ ಯುವಕರು! ಆದ್ದರಿಂದ ಕೆಲವು ಮನೆಯವರು,“ಏನ್ರಯ್ಯ, ಒಳ್ಳೆ ಕಟ್ಟುಮಸ್ತಾಗಿ ಬೆಳೆದಿದ್ದೀರಿ, ಭಿಕ್ಷೆ ಕೇಳಲು ಬರುತ್ತೀರಲ್ಲ! ನಾಚಿಕೆಯಾಗುವುದಿಲ್ಲವೇ? ಹೋಗಿ ಹೋಗಿ, ದುಡಿದು ತಿನ್ನಿ” ಎಂದು ಬೈದು ಅಟ್ಟಿದರು. ಇನ್ನು ಕೆಲವು ತಾಯಂದಿರು ಇವರನ್ನು ಕಂಡು “ಅಯ್ಯೋ! ಇಂತಹ ಕುಲವಂತ ಹುಡುಗರಿಗೆ ಭಿಕ್ಷೆ ಬೇಡುವ ಗತಿ ಏಕಾದರೂ ಬಂತೋ!” ಎಂದು ಕಣ್ಣೀರು ಸುರಿಸಿದರೂ. ಅಂತೂ ಈ ಶಿಷ್ಯರೆಲ್ಲ ಸೇರಿ ಒಂದಿಷ್ಟು ಅಕ್ಕಿ-ಬೇಳೆಕಾಳುಗಳು ಭಿಕ್ಷೆ ಸಂಗ್ರಹಿಸಿ ತಂದರು. ಅದನ್ನು ಬೇಯಿಸಿ ಅಡಿಗೆ ಮಾಡಿ ಶ್ರೀರಾಮಕೃಷ್ಣರಿಗೆ ಸಮರ್ಪಿಸಿದರು. ಅವರು ಬಹಳ ಆನಂದ ಪಡುತ್ತ ಒಂದೆರಡು ಅಗುಳನ್ನು ಬಾಯಿಗೆ ಹಾಕಿಕೊಂಡು “ಆಹ್! ಬಹಳ ಒಳ್ಳೆಯ ಕೆಲಸ ಮಾಡಿದಿರಿ; ಇದು ಅತ್ಯಂತ ಪರಿಶುದ್ಧವಾದ ಆಹಾರ ” ಎಂದು ಉದ್ಗರಿಸಿದರು. ಬಳಿಕ ಯುವಕರೆಲ್ಲ ಆ ಭಿಕ್ಷಾನ್ನವನ್ನು ಹಂಚಿಕೊಂಡರು.

ಕಾಶೀಪುರದ ತೋಟದ ಮನೆಯ ವಾಸದ ಈ ಸಂದರ್ಭದಲ್ಲಿ ತನಗುಂಟಾದ ಒಂದು ಅಲೌಕಿಕ ಅನುಭವವನ್ನು ತಾರಕನಾಥ (ಮುಂದೆ ಸ್ವಾಮಿ ಶಿವಾನಂ) ಅನೇಕ ವರ್ಷಗಳ ಬಳಿಕ ಹೊರಗೆಡಹುತ್ತಾನೆ. ಶಿಷ್ಯರೆಲ್ಲ ಪ್ರತಿರಾತ್ರಿ ಒಟ್ಟಾಗಿಯೇ ಮಲಗುತ್ತಿದ್ದರು. ಹಾಸಿಗೆ ಬಟ್ಟೆಗಳ ಅನುಕೂಲತೆಯೂ ಹೆಚ್ಚಾಗಿರಲಿಲ್ಲ. ಆದ್ದರಿಂದ ಒಂದೇ ಸೊಳ್ಳೆಪರದೆಯೊಳಗೆ ಅನೇಕರು ಒಟ್ಟಾಗಿ ಮಲಗುತ್ತಿದ್ದರು. ಒಂದು ರಾತ್ರಿ ಎಂದಿನಂತೆ ತಾರಕ. ಶಶಿ, ನರೇಂದ್ರ ಮತ್ತಿತರು ಹೀಗೇ ಒಟ್ಟಾಗಿ ಮಲಗಿದ್ದಾರೆ. ಮಧ್ಯರಾತ್ರಿಯಲ್ಲಿ ತಾರಕನಿಗೆ ಅದೇಕೋ ಇದ್ದಕ್ಕಿದ್ದಂತೆ ಎಚ್ಚರವಾಯಿತು. ಕಣ್ತೆರೆದು ನೋಡುತ್ತಾನೆ, ಪರದೆಯ ಒಳಗೆಲ್ಲ ಏನೋ ಒಂದು ಅಲೌಕಿಕ ಬೆಳಕು! ಪಕ್ಕಕ್ಕೆ ನೋಡಿದರೆ ಅಲ್ಲೇ ಮಲಗಿದ್ದ ನರೇಂದ್ರ ಈಗ ಕಾಣುತ್ತಲೇ ಇಲ್ಲ;ಬದಲಾಗಿ ಶಿವನ ತದ್ರೂಪಿಗಳಾದ ಏಳೆಂಟು ವರ್ಷ ವಯಸ್ಸಿನ ಹಲವಾರು ಪುಟ್ಟಪುಟ್ಟ ಬಾಲಕರು ಮಲಗಿ ನಿದ್ರಿಸುತ್ತಿದ್ದಾರೆ! ಎಲ್ಲರೂ ದಿಗಂಬರರಾಗಿ, ಜಟಾಜೂಟಧಾರಿಗಳಾಗಿ, ಶ್ವೇತವರ್ಣದಿಂದ ವಿರಾಜಿಸುತ್ತಿದ್ದಾರೆ! ಪರದೆಯ ಒಳಗೆ ಹರಡಿಕೊಂಡಿರುವ ಬೆಳಕು ಇವರಿಂದಲೇ ಹೊಮ್ಮುತ್ತಿದೆ! ತಾರಕನಿಗೆ ಮೊದಲು ತನ್ನ ನಿದ್ದೆಗಣ್ಣಿಗೆ ಏನೋ ಕಾಣಿಸುತ್ತಿರಬೇಕು ಎನ್ನಿಸಿತು. ಆದರೆ ಕಣ್ಣುಜ್ಜಿಕೊಂಡು ಪೂರ್ತಿ ಎಚ್ಚರಿಕೆಯಿಂದ ದಿಟ್ಟಿಸಿ ನೋಡಿದಾಗಲೂ ಅದೇ ದೃಶ್ಯ ಮತ್ತಷ್ಟು ಸ್ಪಷ್ಟವಾಗಿ ಕಾಣುತ್ತಿದೆ! ತಾರಕ ಕಿಂಕರ್ತವ್ಯವಿಮೂಢನಾಗಿ ಹಾಗೇ ಕುಳಿತುಬಿಟ್ಟ. ಮತ್ತೆ ಮಲಗಲು ಮನಸ್ಸಾಗಲಿಲ್ಲ. ನಿದ್ರೆ ಆವರಿಸಿ, ಎಚ್ಚರ ತಪ್ಪಿ ಅವರಿಗೆ ತನ್ನ ಕಾಲು ತಗಲಿಬಿಟ್ಟರೆ!–ಎಂಬ ಭಯವುಂಟಾಯಿತು. ಬಳಿಕ ಅವನು ಆ ರಾತ್ರಿಯೆಲ್ಲ ಧ್ಯಾನ ಮಾಡುತ್ತ ಕುಳಿತಿದ್ದುಬಿಟ್ಟ. ಅರುಣೋದಯದ ವೇಳೆಗೆ ಕಣ್ತೆರೆದು ನೋಡುತ್ತಾನೆ–ಪಕ್ಕದಲ್ಲಿ ನರೇಂದ್ರ ರಾತ್ರಿ ಹೇಗೆ ಮಲಗಿದ್ದನೋ ಹಾಗೇ ಮಲಗಿದ್ದಾನೆ! ಆದರೆ ತಾನು ಕಂಡದ್ದು ಶತಪಾಲು ಸತ್ಯವೆಂದು ತಾರಕನಿಗೆ ದೃಢವಾಗಿ ಬಿಟ್ಟಿತ್ತು. ನರೇಂದ್ರ ಎದ್ದಮೇಲೆ ತನ್ನ ಅನುಭವವನ್ನು ಅವನಿಗೆ ತಿಳಿಸಿದ. ಎಲ್ಲವನ್ನೂ ಕೇಳಿ ನರೇಂದ್ರ ಸುಮ್ಮನೆ ಗಟ್ಟಿಯಾಗಿ ನಕ್ಕುಬಿಟ್ಟ.

ಈ ನಡುವೆ ಶ್ರೀರಾಮಕೃಷ್ಣರ ಅನಾರೋಗ್ಯ ಕ್ರಮೇಣ ಉಲ್ಬಣವಾಗುತ್ತಲೇ ಬಂತು. ಕ್ಯಾನ್ಸರ್ ರೋಗವನ್ನು ಗುಣಪಡಿಸಬಲ್ಲ ಚಿಕಿತ್ಸೆ ಇಂದಿಗೂ ಲಭ್ಯವಿಲ್ಲ. ಆಗಂತೂ ಇರಲೇ ಇಲ್ಲ. ಜೊತೆಗೆ ರೋಗದ ಬಗೆಗಿನ ಜನರ ತಿಳಿವಳಿಕೆಯೂ ಅಷ್ಟಕ್ಕಷ್ಟೇ. ಹೀಗಿರುವಾಗ ಒಂದು ದಿನ ಇದ್ದಕ್ಕಿದ್ದಂತೆ ವದಂತಿಯೊಂದು ಕೇಳಿಬಂತು–ಶ್ರೀರಾಮಕೃಷ್ಣರ ವ್ಯಾಧಿ ಸಾಂಕ್ರಾಮಿಕವಾದದ್ದು ಎಂದು! ಈ ಸುದ್ದಿಯನ್ನು ಹುಟ್ಟು ಹಾಕಿದವರು ಯಾರೋ ತಿಳಿಯದು. ಆದರೆ ಅದರ ಪರಿಣಾಮ ಮಾತ್ರ ಭಯಂಕರವಾಗಿತ್ತು. ತಮ್ಮ ಪ್ರಿಯ ಗುರುದೇವನ ಕಾಯಿಲೆ ವಾಸಿಯಾಗುವುದಿರಲಿ, ತಮಗೇ ಆ ಕಾಯಿಲೆ ಅಂಟಿಕೊಂಡರೇನು ಗತಿ ಎಂದು ಅನೇಕರು ಹೆದರಿ, ಮೆಲ್ಲನೆ ಹಿಂಜರಿಯಲಾರಂಭಿಸಿ ದರು. ಇದನ್ನು ಕಂಡ ನರೇಂದ್ರನ ಎದೆಯಲ್ಲಿ ಭಾವಗಳ ಜ್ವಾಲಾಮುಖಿ ಸ್ಫೋಟಿಸಿತು. ಕೂಡಲೇ ಸಹಶಿಷ್ಯರನ್ನೆಲ್ಲ ಒಟ್ಟುಗೂಡಿಸಿಕೊಂಡು ಶ್ರೀರಾಮಕೃಷ್ಣರ ಕೋಣೆಗೆ ಹೋದ. ಶ್ರೀರಾಮಕೃಷ್ಣರು ಕುಡಿದು ಉಳಿಸಿದ್ದ, ಅವರ ಎಂಜಲು ಬೆರೆತಿದ್ದ ಗಂಜಿಯ ಬಟ್ಟಲು ಅಲ್ಲಿತ್ತು. ನರೇಂದ್ರ ಒಂದು ಮಾತನ್ನೂ ಆಡದೆ, ಬಟ್ಟಲನ್ನೆತ್ತಿ, ಅದರಲ್ಲಿದ್ದ ಗಂಜಿಯನ್ನೆಲ್ಲ ಕುಡಿದುಬಿಟ್ಟ! ‘ಒಂದು ವೇಳೆ ಶ್ರೀರಾಮಕೃಷ್ಣರ ಕಾಯಿಲೆ ಅಂಟುರೋಗವೇ ಆಗಿದ್ದು, ನಮಗೂ ತಗಲುವ ಸಂಭವವಿದ್ದರೆ, ಅದಕ್ಕೆ ಮೊತ್ತಮೊದಲು ಬಲಿಯಾಗಲು ಇದೋ, ನಾನು ಸಿದ್ಧನಿದ್ದೇನೆ’ ಎಂದು ಅವನ ವರ್ತನೆ ಸಾರಿ ಹೇಳುವಂತಿತ್ತು. ಇದನ್ನು ಕಂಡ ಶಿಷ್ಯರೆಲ್ಲ ಸ್ತಂಭೀಭೂತರಾಗಿ ನಿಂತುಬಿಟ್ಟರು. ಬಳಿಕ ನಾಚಿಕೆಯಿಂದ ತಲೆತಗ್ಗಿಸಿ ಪಶ್ಚಾತ್ತಾಪದ ಕಂಬನಿ ಹರಿಸಿದರು. 

ನರೇಂದ್ರನು ತಮ್ಮೆಲ್ಲರಿಗಿಂತ ಹೇಗೆ ಶ್ರೇಷ್ಠನಾದವನು ಎಂಬುದನ್ನು ಮನಗಾಣಲು ಇತರ ಯುವಕರಿಗೆಲ್ಲ ಇದೊಂದು ಉದಾಹರಣೆಯಷ್ಟೆ. ಅವನು ಬುದ್ಧಿಶಕ್ತಿ, ವಿಚಾರಶಕ್ತಿ, ಜ್ಞಾನಶಕ್ತಿ ಹಾಗೂ ಶ್ರದ್ಧಾಭಕ್ತಿಗಳಲ್ಲಿ ತಮ್ಮೆಲ್ಲರನ್ನೂ ಮೀರಿಸಿದವನು ಎಂಬುದನ್ನು ಅವರೆಲ್ಲ ಅರಿತಿದ್ದರು. ಅಲ್ಲದೆ, ಅವನು ತನ್ನನ್ನು ಮೇಲೆತ್ತಿಕೊಳ್ಳುವುದರ ಜೊತೆಗೆ, ತನ್ನ ಸೋದರಶಿಷ್ಯರನ್ನೂ ತಮ್ಮೆಲ್ಲರ ಜೀವನಾದರ್ಶವನ್ನೂ ಎತ್ತಿಹಿಡಿಯುತ್ತಿದ್ದ. ತನ್ನ ಧೀರ ಗಂಭೀರ ವ್ಯಕ್ತಿತ್ವದಿಂದ ಇತರ ಶಿಷ್ಯರ ಆಧ್ಯಾತ್ಮಿಕ ಜ್ವಾಲೆಯನ್ನೂ ಇನ್ನಷ್ಟು ಪ್ರಖರಗೊಳಿಸುತ್ತಿದ್ದ. ಆ ಯುವಕರ ತ್ಯಾಗ-ವೈರಾಗ್ಯ ಪ್ರವೃತ್ತಿಯ ಕುರಿತಾಗಿ ಇತರರು ಯಾರಾದರೂ ಹಗುರವಾಗಿ ಮಾತನಾಡಿದರೆ, ಅಥವಾ ಶ್ರೀರಾಮ ಕೃಷ್ಣರ ಬೋಧನೆಗಳನ್ನು ಟೀಕಿಸಿದರೆ, ಅವನು ತನ್ನ ಪ್ರಖರವಾದ ವಾಗ್ಸಾಮರ್ಥ್ಯದಿಂದ ಅಂಥವರ ಬಾಯಿ ಮುಚ್ಚಿಸಿಬಿಡುತ್ತಿದ್ದ.

ಹೀಗೆ ಅವನು ಶಿಷ್ಯಸಮುದಾಯದ ಸಮರ್ಥ ನಾಯಕನಾಗಿ ವ್ಯಕ್ತವಾಗುತ್ತಿದ್ದ. ಶ್ರೀರಾಮ ಕೃಷ್ಣರೂ ಅವನ ನಾಯಕತ್ವವನ್ನು ನಾನಾ ಬಗೆಯಲ್ಲಿ ಪ್ರೋತ್ಸಾಹಿಸಿ ಬಲಪಡಿಸುತ್ತಿದ್ದರು. “ನರೇಂದ್ರನೇ ನಿಮ್ಮ ನಾಯಕ. ಅವನ ಆಧ್ಯಾತ್ಮಿಕ ಶಕ್ತಿ ಹಾಗೂ ಆಧ್ಯಾತ್ಮಿಕ ಜ್ಞಾನವೇ ಮುಂದೆ ನಿಮಗೆ ಮಾರ್ಗದರ್ಶಕವಾಗಿ ನಿಲ್ಲುತ್ತವೆ” ಎಂದು ಅವರು ಇತರ ಯುವಕರಿಗೆ ಎಷ್ಟೋ ಸಲ ಹೇಳಿದ್ದರು. ನಿಜಕ್ಕೂ ಈ ಯುವಕರು ಶ್ರೀರಾಮಕೃಷ್ಣರನ್ನು ಸರಿಯಾಗಿ ಅರ್ಥ ಮಾಡಿಕೊಳ್ಳಲು ಸಾಧ್ಯವಾದದ್ದು ಕೂಡ ನರೇಂದ್ರನ ಮೂಲಕವೇ.

ಈ ಎಲ್ಲ ವಿವಿಧ ಚಟುವಟಿಕೆ-ಸಾಧನೆಗಳ ಮಧ್ಯದಲ್ಲೂ ಅವನು, ತಾನು ಭಗವಂತನ ಸಾಕ್ಷಾತ್ಕಾರ ಪಡೆದುಕೊಳ್ಳಬೇಕಾದರೆ ಶ್ರೀರಾಮಕೃಷ್ಣರ ಕೃಪೆಯಿಂದ ಮಾತ್ರವೇ ಸಾಧ್ಯ ಎಂಬ ಸತ್ಯವನ್ನು ಯಾವಾಗಲೂ ನೆನಪಿಟ್ಟುಕೊಂಡಿದ್ದ. ತನ್ನನ್ನು ಸಾಕ್ಷಾತ್ಕಾರದ ಹಾದಿಯಲ್ಲಿ ನಿರಂತರ ವಾಗಿ ಮುನ್ನಡೆಸುತ್ತಿರುವವರು ಅವರೇ ಎಂಬುದನ್ನು ಆತ ಚೆನ್ನಾಗಿ ಮನಗಂಡಿದ್ದ. ತನ್ನ ಸರ್ವಸ್ವವೂ ಆಗಿದ್ದ ಶ್ರೀರಾಮಕೃಷ್ಣರ ಕಾಯಿಲೆಯ ವಿಚಾರ ಅವನ ಮನಸ್ಸನ್ನು ಸದಾ ಕಾಡುತ್ತಲೇ ಇತ್ತು. ಒಂದು ದಿನ, ಶ್ರೀರಾಮಕೃಷ್ಣರ ಒಬ್ಬ ಮಹಾಭಕ್ತನಾದ ಶಶಧರ ತರ್ಕಚೂಡಾಮಣಿ ಎಂಬ ಪಂಡಿತ ಕಾಶೀಪುರಕ್ಕೆ ಬಂದಿದ್ದ. ಆತ ಶ್ರೀರಾಮಕೃಷ್ಣರ ರೋಗದ ತೀವ್ರತೆಯನ್ನು ಕಂಡು ಬಹಳ ಸಂಕಟಪಟ್ಟ. ಜೊತೆಗೆ ಅವನಿಗೆ ತುಂಬ ಆಶ್ಚರ್ಯ ಕೂಡ. ಅವನು ಕೇಳಿದ:

“ಮಹಾಶಯರೆ, ನಿಮ್ಮಂತಹ ಮಹಾತ್ಮರು ಇಂತಹ ಕಾಯಿಲೆಗಳನ್ನೆಲ್ಲ ಕೇವಲ ಇಚ್ಛಾಮಾತ್ರ ದಿಂದ ವಾಸಿ ಮಾಡಿಕೊಳ್ಳಬಹುದು ಎಂದು ಶಾಸ್ತ್ರಗಳಲ್ಲಿ ಬರೆದಿದೆ. ನೀವು ಆ ಕಾಯಿಲೆಯ ಜಾಗದಲ್ಲಿ ನಿಮ್ಮ ಮನಸ್ಸನ್ನು ಏಕಾಗ್ರಗೊಳಿಸಿಕೊಂಡು, ‘ಇದು ವಾಸಿಯಾಗಿಬಿಡಲಿ’ ಎಂದು ದೃಢವಾಗಿ ಸಂಕಲ್ಪಿಸಿದರಾಯಿತು–ಕಾಯಿಲೆ ವಾಸಿಯಾಗಿ ಬಿಡುತ್ತದೆ. ನೀವೇಕೆ ಹಾಗೆ ಮಾಡಲೊಲ್ಲಿರಿ?”

ಆಗ ಶ್ರೀರಾಮಕೃಷ್ಣರು ಕೂಡ ಅಷ್ಟೇ ಆಶ್ಚರ್ಯದಿಂದ ಮರುಪ್ರಶ್ನೆ ಹಾಕಿದರು: “ಏನಯ್ಯ, ನೀನು ಇಂಥಾ ಮಹಾಪಂಡಿತ! ಆದರೂ ಇಂತಹ ಅವಿವೇಕದ ಸಲಹೆ ಕೊಡುತ್ತೀಯಲ್ಲ! ನಾನು ನನ್ನ ಮನಸ್ಸನ್ನು ಒಮ್ಮೆ ಭಗವಂತನಿಗೆ ಕೊಟ್ಟುಬಿಟ್ಟಾಗಿದೆ. ಈಗ ಆ ಮನಸ್ಸನ್ನು ಹಿಂದೆಗೆದುಕೊಂಡು, ಮತ್ತೆ ಅದನ್ನು ಈ ಕೊಳೆತು ನಾರುವ ರಕ್ತಮಾಂಸಗಳ ಗೂಡಿನ ಮೇಲಿಡಲೇ?” ಪಂಡಿತನಿಗೆ ಆ ಮಾತಿನ ಸತ್ಯತೆ ಮನವರಿಕೆಯಾಗಿರಬೇಕು; ಆದ್ದರಿಂದ ಮರು ಮಾತಾಡಲಿಲ್ಲ. ಆದರೆ ನರೇಂದ್ರಾದಿ ಶಿಷ್ಯರಿಗೆ ಆತನ ಸಲಹೆ ತುಂಬ ಮೆಚ್ಚಿಗೆಯಾಗಿತ್ತು. ಆತ ಅಲ್ಲಿಂದ ನಿರ್ಗಮಿಸುತ್ತಿದ್ದಂತೆಯೇ ಇವರೆಲ್ಲ ಶ್ರೀರಾಮಕೃಷ್ಣರನ್ನು ಒತ್ತಾಯಪಡಿಸಲಾರಂಭಿಸಿ ದರು, “ಮಹಾಶಯರೆ, ಕನಿಷ್ಠಪಕ್ಷ ನಮಗೋಸ್ಕರವಾದರೂ ನೀವು ಈ ಕಾಯಿಲೆಯನ್ನು ವಾಸಿಮಾಡಿಕೊಳ್ಳಬೇಕು” ಎಂದು. 

ಶ್ರೀರಾಮಕೃಷ್ಣರು (ನರೇಂದ್ರನಿಗೆ): “ಏನಪ್ಪ, ಈ ಕಾಯಿಲೆಯನ್ನು ನಾನಾಗಿಯೇ ನನ್ನ ಮೇಲೆ ತಂದುಕೊಂಡೆ ಅಂತ ತಿಳಿದೆಯಾ? ನನಗೇನೋ ಇದು ವಾಸಿಯಾಗಲಿ ಅಂತಲೇ ಇಷ್ಟ. ಆದರೂ ಅದು ವಾಸಿಯಾಗುತ್ತಿಲ್ಲ. ಇದೆಲ್ಲ ಜಗನ್ಮಾತೆಯ ಇಚ್ಛೆ–ಅವಳ ಲೀಲೆ.”

ನರೇಂದ್ರ: “ಹಾಗಾದರೆ ಈ ಕಾಯಿಲೆಯನ್ನು ಗುಣಪಡಿಸುವಂತೆ ಜಗನ್ಮಾತೆಯನ್ನೇ ಕೇಳಿ ಕೊಳ್ಳಿ. ಅವಳು ನಿಮ್ಮ ಪ್ರಾರ್ಥನೆಯನ್ನು ಕೇಳದಿರಲು ಸಾಧ್ಯವೇ ಇಲ್ಲ.”

ಶ್ರೀರಾಮಕೃಷ್ಣರು: “ನೀನೇನೋ ಸುಲಭವಾಗಿ ಹಾಗೆ ಹೇಳಿಬಿಡಬಹುದು. ಆದರೆ ನಾನು ಇಂಥ ವಿಷಯಗಳನ್ನೆಲ್ಲ ತಾಯಿಯ ಬಳಿ ಕೇಳಿಕೊಳ್ಳಲಾರೆ.”

ನರೇಂದ್ರ: “ಅದೆಲ್ಲ ಆಗುವುದಿಲ್ಲ. ನೀವು ಆಕೆಯನ್ನು ಕೇಳಿಕೊಳ್ಳಲೇ ಬೇಕು. ಕಡೆಯಪಕ್ಷ, ನಮಗೋಸ್ಕರವಾದರೂ ಕೇಳಿಕೊಳ್ಳಲೇ ಬೇಕು.”

ಕೊನೆಗೆ ಶ್ರೀರಾಮಕೃಷ್ಣರು ತಮ್ಮ ಪ್ರಿಯಶಿಷ್ಯನ ಒತ್ತಾಯಕ್ಕೆ ಕಟ್ಟುಬಿದ್ದು, “ಆಗಲಿ, ಏನಾಗುತ್ತದೋ ನೋಡೋಣ” ಎಂದರು.

ನರೇಂದ್ರಾದಿಗಳು ಅಲ್ಲಿಂದೆದ್ದು ಹೊರಟುಹೋದರು. ಒಂದೆರಡು ಗಂಟೆಗಳ ಬಳಿಕ ನರೇಂದ್ರ ಹಿಂದಿರುಗಿ ಬಂದವನೇ ಕಾತರದಿಂದ ಕೇಳಿದ: “ಕಾಯಿಲೆ ಹೋಗಲಾಡಿಸುವಂತೆ ಜಗನ್ಮಾತೆಯನ್ನು ಕೇಳಿಕೊಂಡಿರಾ? ಆಕೆ ಏನು ಹೇಳಿದಳು?”

ಶ್ರೀರಾಮಕೃಷ್ಣರು: “ಜಗನ್ಮಾತೆಗೆ ನನ್ನ ಗಂಟಲನ್ನು ತೋರಿಸಿ ಹೇಳಿದೆ–‘ಅಮ್ಮಾ ಈ ಕಾಯಿಲೆಯ ದೆಸೆಯಿಂದ ನಾನು ಏನನ್ನೂ ತಿನ್ನುವುದಕ್ಕೆ ಸಾಧ್ಯವಾಗುತ್ತಿಲ್ಲ; ಸ್ವಲ್ಪ ಆಹಾರ ತೆಗೆದುಕೊಳ್ಳುವುದಕ್ಕಾದರೂ ಅನುಕೂಲ ಮಾಡಿಕೊಡು’ ಅಂತ. ಆದರೆ ಅವಳು ನಿಮ್ಮೆಲ್ಲರ ಕಡೆ ಬೆರಳುಮಾಡಿ ತೋರಿಸಿ, ‘ಏಕೆ! ನೀನು ಅಷ್ಟೊಂದು ಬಾಯಿಗಳ ಮೂಲಕ ಈಗ ತಿನ್ನುತ್ತಿಲ್ಲವೇ?’ ಅಂತ ಕೇಳಿಬಿಟ್ಟಳು. ನನಗೆ ಅದನ್ನು ಕೇಳಿ ಎಷ್ಟು ನಾಚಿಕೆಯಾಯಿತೆಂದರೆ ಇನ್ನೊಂದು ಮಾತನ್ನೂ ಆಡಲು ಸಾಧ್ಯವಾಗಲಿಲ್ಲ.”

ಇದನ್ನು ಕೇಳಿ ನರೇಂದ್ರ ಆಶ್ಚರ್ಯಚಕಿತನಾದ. ಕಿಂಚಿತ್ತೂ ದೇಹಬುದ್ಧಿಯಾಗಲಿ ಸ್ವಾರ್ಥತೆ ಯಾಗಲಿ ಇಲ್ಲದ ಎಂತಹ ಅದ್ಭುತ ಸ್ಥಿತಿ ಇದು! ನಿಜವಾದ ಅದ್ವೈತ ಭಾವವೆಂದರೆ ಇದೇ ಅಲ್ಲವೆ!‘ಶ್ರೀರಾಮಕೃಷ್ಣರ ಸಾಕ್ಷಾತ್ಕಾರಗಳೆಲ್ಲ ಎಷ್ಟು ಪರಿಪೂರ್ಣವಾದವುಗಳು!’ ಎಂದು ಆಲೋಚಿಸುತ್ತ ಅವನ ಬಾಯಿ ಕಟ್ಟಿಹೋಯಿತು.

ದಿನ ಕಳೆದಂತೆ ನರೇಂದ್ರನಲ್ಲಿ ಧ್ಯಾನಭಾವ ಹೆಚ್ಚು ಆಳವಾಗುತ್ತ, ಹೆಚ್ಚು ಸಹಜವಾಗುತ್ತ ಬಂದಿತ್ತು. ಅವನು ಯಾವುದೇ ವಿಷಯದ ಕುರಿತಾಗಿ ಚಿಂತಿಸುವಾಗ ತೀವ್ರ ಏಕಾಗ್ರತೆಯಿಂದ ಅದರ ಆಳಕ್ಕೆ ಮುಳುಗುವುದು ಕಂಡುಬರುತ್ತಿತ್ತು. ಧ್ಯಾನಕ್ಕೆ ಕುಳಿತನೆಂದರೆ ಗಾಢ ಧ್ಯಾನದಲ್ಲಿ ಮಗ್ನನಾಗಿಬಿಡುತ್ತಿದ್ದ. ಕೆಲವೊಮ್ಮೆ ಅವನು ಧ್ಯಾನ ಮುಗಿದ ಮೇಲೆ ತನ್ನ ಬಳಿಯಲ್ಲಿ ತನ್ನಂತೆಯೇ ಇರುವ ಇನ್ನೊಬ್ಬ ವ್ಯಕ್ತಿಯನ್ನು ಕಂಡು, ‘ಯಾರಿರಬಹುದು ಈತ!’ ಎಂದು ಅಚ್ಚರಿಗೊಳ್ಳುತ್ತಿದ್ದ. ಕನ್ನಡಿಯೊಳಗಿನ ಪ್ರತಿಬಿಂಬ ನಮ್ಮ ಪ್ರತಿಯೊಂದು ಚಲನವಲನವನ್ನೂ ಅನುಕರಿಸುವಂತೆ, ತನ್ನ ಈ ಪ್ರತಿರೂಪವೂ ತನ್ನಂತೆಯೇ ನಡೆದುಕೊಳ್ಳುವುದನ್ನು ನೋಡುತ್ತಿದ್ದ. ಕೆಲವು ಸಲವಂತೂ ಈ ‘ವ್ಯಕ್ತಿ’ ಒಂದು ಗಂಟೆಗಿಂತಲೂ ಹೆಚ್ಚು ಕಾಲ ತನ್ನೊಂದಿಗೇ ಇದ್ದಂತೆ ಅನುಭವವಾಗುತ್ತಿತ್ತು. ಅವನು ಈ ವಿಷಯವನ್ನು ಶ್ರೀರಾಮಕೃಷ್ಣರಿಗೆ ತಿಳಿಸಿದಾಗ ಅವರು, “ಇವೆಲ್ಲ ಧ್ಯಾನ ಜೀವನದಲ್ಲಿ ಪ್ರಗತಿ ಹೊಂದುತ್ತ ಬರುವಾಗ ಕಂಡುಬರುವಂತಹ ಸಂಗತಿಗಳಷ್ಟೆ. ಇದಕ್ಕೆಲ್ಲ ವಿಶೇಷ ಗಮನ ಕೊಡಬೇಕಾಗಿಲ್ಲ” ಎಂದು ಹೇಳಿ ಆ ವಿಷಯವನ್ನು ಅಲ್ಲೇ ತೇಲಿಸಿಬಿಟ್ಟರು.

ಬರಬರುತ್ತ ತನ್ನಲ್ಲಿ ಆಧ್ಯಾತ್ಮಿಕ ಶಕ್ತಿ ಅಭಿವೃದ್ಧಿ ಹೊಂದುತ್ತಿರುವುದನ್ನು ನರೇಂದ್ರ ತಾನೇ ಕಂಡುಕೊಳ್ಳುತ್ತಿದ್ದ. ಅವನ ಆಲೋಚನಾಶಕ್ತಿಯೆಂಬುದು ಇತರರನ್ನೂ ಪರಿವರ್ತಿಸಬಲ್ಲ ಒಂದು ಪ್ರಬಲ ಶಕ್ತಿಯಾಗಿತ್ತು. ಈ ಶಕ್ತಿಯನ್ನು ಪ್ರಯೋಗಿಸಿ ನೋಡುವ ಬಾಲಿಶ ಕುತೂಹಲದಲ್ಲಿ, ಒಮ್ಮೆ ಅದನ್ನು ವ್ಯಕ್ತಪಡಿಸಿಯೂಬಿಟ್ಟ.

ಅಂದು ೧೮೮೬ರ ಶಿವರಾತ್ರಿಯ ದಿನ. ಆ ದಿನವೆಲ್ಲ ವಿಶೇಷ ಪೂಜೆ ಪ್ರಾರ್ಥನೆ ಧ್ಯಾನಾದಿಗಳಲ್ಲಿ ಕಳೆಯಬೇಕೆಂದು ಶಿಷ್ಯರು ಸಂಕಲ್ಪಿಸಿದ್ದರು. ನರೇಂದ್ರ ಮೂರು-ನಾಲ್ಕು ಜನ ಸಹಶಿಷ್ಯರೊಂದಿಗೆ ಉದ್ಯಾನಗೃಹದ ಒಂದು ಕೋಣೆಯಲ್ಲಿ ಮಾತನಾಡುತ್ತಿದ್ದ. ಸ್ವಲ್ಪ ಹೊತ್ತಿನ ಮೇಲೆ ಕಾಳೀಪ್ರಸಾದ ನೊಬ್ಬನನ್ನುಳಿದು ಮಿಕ್ಕವರೆಲ್ಲ ಆ ಕೋಣೆಯಿಂದಾಚೆಗೆ ಹೊರಟುಹೋದರು. ಈಗ ನರೇಂದ್ರ ಧ್ಯಾನಕ್ಕೆ ಕುಳಿತ ಸ್ವಲ್ಪ ಹೊತ್ತಿನಲ್ಲೇ ತನ್ನಲ್ಲೇನೋ ಒಂದು ಶಕ್ತಿಯ ಆವಿರ್ಭಾವವಾಗುತ್ತಿರುವಂತೆ ಭಾಸವಾಯಿತು. ಆಗ ಅವನು ಬಳಿಯಲ್ಲೇ ಕುಳಿತಿದ್ದ ಕಾಳೀಪ್ರಸಾದನಿಗೆ, “ನೋಡು, ಇನ್ನು ಸ್ವಲ್ಪ ಹೊತ್ತಿನಲ್ಲಿ ನನ್ನನ್ನು ಮುಟ್ಟು” ಎಂದು ಹೇಳಿ ಧ್ಯಾನಮಗ್ನನಾಗಿ ಕುಳಿತ. ಸ್ವಲ್ಪ ಹೊತ್ತಿನ ಮೇಲೆ, ಹೊರಗಡೆ ಹೋಗಿದ್ದ ಮತ್ತೊಬ್ಬ ಶಿಷ್ಯ ಬಂದು ನೋಡುತ್ತಾನೆ–ನರೇಂದ್ರ ಗಾಢ ಧ್ಯಾನಲೀನ ನಾಗಿ ಸ್ಥಿರವಾಗಿ ಕುಳಿತಿದ್ದಾನೆ; ಕಾಳೀಪ್ರಸಾದ ತನ್ನ ಬಲಗೈಯಿಂದ ಆತನ ಬಲ ಮೊಳಕಾಲನ್ನು ಸ್ಪರ್ಶಿಸುತ್ತ ಕಣ್ಮುಚ್ಚಿ ಕುಳಿತಿದ್ದಾನೆ, ಅವನ ಕೈ ಒಂದೇ ಸಮನೆ ನಡುಗುತ್ತಿದೆ. ಒಂದೆರಡು ನಿಮಿಷಗಳಾದ ಮೇಲೆ ನರೇಂದ್ರ ಕಣ್ದೆರೆದು ನುಡಿದ: “ಸಾಕು, ಬಿಡು. ನಿನಗೆ ಹೇಗೆನ್ನಿಸಿತು?” ಆಗ ಕಾಳೀಪ್ರಸಾದ, “ನನಗೆ ವಿದ್ಯುತ್ ಆಘಾತವಾಗುತ್ತಿದ್ದ ಹಾಗಿತ್ತು! ನನ್ನ ಕೈ ಒಂದೇ ಸಮನೆ ನಡುಗುತ್ತಿತ್ತು. ಕೈಯನ್ನು ಸ್ಥಿರವಾಗಿಟ್ಟುಕೊಳ್ಳಲು ಎಷ್ಟೇ ಪ್ರಯತ್ನಪಟ್ಟರೂ ಸಾಧ್ಯವಾಗಲಿಲ್ಲ” ಎಂದುತ್ತರಿಸಿದ. 

ರಾತ್ರಿಯ ಎರಡನೆ ಜಾವದ ಮೇಲೆ ಶಿಷ್ಯರೆಲ್ಲ ಸೇರಿ ಶಿವನ ಪೂಜೆ, ಧ್ಯಾನಗಳಲ್ಲಿ ತೊಡಗಿದರು. ಆ ಸಮಯದಲ್ಲಿ ಕಾಳೀಪ್ರಸಾದ ಅತ್ಯಂತ ಗಾಢವಾದ ಧ್ಯಾನದಲ್ಲಿ ತಲ್ಲೀನನಾಗಿಬಿಟ್ಟ. ಅವನು ಇಲ್ಲಿಯವರೆಗೆ ಅಂತಹ ಗಾಢ ಧ್ಯಾನವನ್ನು ಅನುಭವಿಸಿದವನೇ ಅಲ್ಲ. ಅವನ ಇಡೀ ಶರೀರ ಸ್ಥಿರವಾಗಿಬಿಟ್ಟಿತ್ತು. ಸ್ವಲ್ಪವೂ ಬಾಹ್ಯಪ್ರಜ್ಞೆ ಇರುವಂತೆ ಕಾಣುತ್ತಿರಲಿಲ್ಲ. ಇದನ್ನು ಕಂಡವರಿಗೆಲ್ಲ ಆಶ್ಚರ್ಯವಾಯಿತು. ಸ್ವಲ್ಪ ಹೊತ್ತಿನ ಹಿಂದೆ ಅವನು ನರೇಂದ್ರನನ್ನು ಸ್ಪರ್ಶಿಸಿದ್ದೇ ಅವನ ಈ ಸ್ಥಿತಿಗೆ ಕಾರಣವಿರಬೇಕೆಂದು ಅವರು ಊಹಿಸಿದರು. ನರೇಂದ್ರನೂ ಇದನ್ನು ಗಮನಿಸಿದ.

ಬೆಳಗ್ಗೆ ನಾಲ್ಕು ಗಂಟೆಯ ಹೊತ್ತಿಗೆ ಪೂಜಾದಿಗಳೆಲ್ಲ ಮುಗಿದುವು. ಆ ಸಮಯಕ್ಕೆ ಶಶಿ ಭೂಷಣ ಅಲ್ಲಿಗೆ ಬಂದು, “ಶ್ರೀರಾಮಕೃಷ್ಣರು ನಿನ್ನನ್ನು ಕರೆಯುತ್ತಿದ್ದಾರೆ. ಈಗಲೇ ಹೋಗಿ ನೋಡಬೇಕಂತೆ” ಎಂದು ನರೇಂದ್ರನಿಗೆ ಹೇಳಿದ. ಕೂಡಲೇ ನರೇಂದ್ರ ಶಶಿಯ ಜೊತೆಯಲ್ಲಿ ಮಹಡಿಯ ಮೇಲಕ್ಕೆ ಹೋದ. ಅವನನ್ನು ನೋಡಿದ ತಕ್ಷಣ ಶ್ರೀರಾಮಕೃಷ್ಣರು ಗಂಭೀರವಾಗಿ ನುಡಿದರು: “ಏನಿದೆಲ್ಲ? ಸಂಪಾದಿಸುವ ಮೊದಲೇ ಖರ್ಚು ಮಾಡಲು ಶುರುಮಾಡಿಬಿಟ್ಟೆಯಾ? ಆಧ್ಯಾತ್ಮಿಕ ಶಕ್ತಿಯನ್ನು ಮೊದಲು ನೀನು ಸಾಕಷ್ಟು ಸಂಪಾದಿಸಿ ಶೇಖರಿಸಿಕೊ. ಆಮೇಲೆ ಅದನ್ನು ಎಲ್ಲಿ, ಹೇಗೆ ವಿನಿಯೋಗಿಸಬೇಕು ಎನ್ನುವುದು ನಿನಗೇ ತಿಳಿಯುತ್ತದೆ. ಸ್ವತಃ ಜಗನ್ಮಾತೆಯೇ ನಿನಗೆ ಅದನ್ನೆಲ್ಲ ತಿಳಿಸಿಕೊಡುತ್ತಾಳೆ. ನೀನೀಗ ನಿನ್ನ ಮನೋಭಾವವನ್ನು ಕಾಳೀಪ್ರಸಾದನಲ್ಲಿ ಪ್ರಚೋದಿಸಿ ಬಿಟ್ಟೆಯಲ್ಲ, ಅದರಿಂದ ಅವನಿಗೆ ಎಂತಹ ಅಪಾಯ ಸಂಭವಿಸಿದೆ ಎಂಬುದು ನಿನಗೆ ಗೊತ್ತಿ ದೆಯೆ? ಇಲ್ಲಿಯವರೆಗೆ ಅವನಲ್ಲಿ ಆಧ್ಯಾತ್ಮಿಕ ಭಾವ ಯಾವುದೋ ಒಂದು ನಿರ್ದಿಷ್ಟ ಕ್ರಮದಿಂದ ಬೆಳೆದುಕೊಂಡು ಬರುತ್ತಿತ್ತು. ಈಗ ಅದೆಲ್ಲ ನಾಶವಾಗಿಹೋಯಿತು. ಹೋಗಲಿ, ಆದದ್ದಾಯಿತು. ಇನ್ನು ಮೇಲಾದರೂ ಇಂತಹ ಅವಿವೇಕದ ಕೆಲಸ ಮಾಡಬೇಡ. ಹುಡುಗನ ಅದೃಷ್ಟ ಚೆನ್ನಾಗಿತ್ತು. ಆದ್ದರಿಂದ ಹೆಚ್ಚೇನೂ ಹಾನಿಯಾಗಲಿಲ್ಲ.” ತಾನೆಸಗಿದ ತಪ್ಪಿನ ಅರಿವಾಗಿ ನರೇಂದ್ರ ತಲೆತಗ್ಗಿಸಿದ. ಅದಕ್ಕಿಂತ ಹೆಚ್ಚಾಗಿ, ತಾವು ಮಾಡಿದ್ದನ್ನೆಲ್ಲ ಶ್ರೀರಾಮಕೃಷ್ಣರು ಯಾರೂ ಹೇಳದೆಯೇ ತಿಳಿದು ಕೊಂಡುಬಿಟ್ಟಿರುವುದನ್ನು ಕಂಡು ಅವನು ಆಶ್ಚರ್ಯಚಕಿತನಾದ.

ಈ ಘಟನೆಯಲ್ಲಿ ನಾವು ಹಲವಾರು ವಿಚಾರಗಳನ್ನು ಗಮನಿಸಬಹುದಾಗಿದೆ. ಆಧ್ಯಾತ್ಮಿಕ ಶಕ್ತಿಯನ್ನು ಸಂಪೂರ್ಣವಾಗಿ ಸಿದ್ಧಿಸಿಕೊಂಡ ಮೇಲೆ ಅದನ್ನು ಸರಿಯಾದ ರೀತಿಯಲ್ಲಿ ವಿನಿ ಯೋಗಿಸಬೇಕಾದರೂ ಸಾಕಷ್ಟು ವಿವೇಕ ಇರಬೇಕಾಗುತ್ತದೆ. ಆಧ್ಯಾತ್ಮಿಕ ಶಕ್ತಿಯನ್ನು ಪಡೆದು ಕೊಳ್ಳುವುದಕ್ಕೂ ಭಗವಂತನ ಅನುಗ್ರಹ ಬೇಕು; ಹಾಗೆಯೇ ಅದನ್ನು ಸರಿಯಾದ ರೀತಿಯಲ್ಲಿ ವಿನಿಯೋಗಿಸುವುದಕ್ಕೂ ಭಗವಂತನ ಮಾರ್ಗದರ್ಶನ-ನಿರ್ದೇಶನ ಬೇಕು.

\delimiter

ಶ್ರೀರಾಮಕೃಷ್ಣರ ಶರೀರ ದಿನೇದಿನೇ ಕುಗ್ಗಿಹೋಗುತ್ತಿತ್ತು. ಇದನ್ನು ಕಂಡು ಭಕ್ತರು ಅತೀವ ಕಳವಳಗೊಂಡು ಇನ್ನಷ್ಟು ಶ್ರದ್ಧೆಯಿಂದ, ಹೆಚ್ಚಿನ ಪರಿಶ್ರಮ ವಹಿಸಿ ಸೇವೆ ಮಾಡುತ್ತ ಬಂದರು. ಯುವಕ ಶಿಷ್ಯರಂತೂ ತಮ್ಮ ಮನೆಯವರ ಒತ್ತಾಯವನ್ನು ಸ್ವಲ್ಪವೂ ಲೆಕ್ಕಿಸದೆ ಕಾಶೀಪುರದ ತೋಟದ ಮನೆಯಲ್ಲೇ ನೆಲೆನಿಂತುಬಿಟ್ಟಿದ್ದರು. ಗೃಹಸ್ಥಭಕ್ತರು ಸ್ವಲ್ಪವೂ ಹಿಂಜರಿಯದೆ, ಲೆಕ್ಕಾಚಾರ ಹಾಕದೆ, ಔಷಧೋಪಚಾರಗಳ ಸಮಸ್ತ ವೆಚ್ಚವನ್ನೂ ವಹಿಸಿಕೊಂಡಿದ್ದರು. ಆದರೆ ತಮ್ಮ ಜೀವನಾಧಾರ ಕಣ್ಮರೆಯಾಗುವುದೇ ಖಂಡಿತ ಎಂದು ಎಲ್ಲರಿಗೂ ಅನ್ನಿಸುತ್ತಿತ್ತು. ರೋಗ ಉಲ್ಬಣಿಸಿ ಶ್ರೀರಾಮಕೃಷ್ಣರ ಗಂಟಲಿನಿಂದ ರಕ್ತಸ್ರಾವವಾಗುವ ದೃಶ್ಯವನ್ನು ಕಂಡಾಗಲಂತೂ ಭಕ್ತರ ಎದೆ ತಲ್ಲಣಿಸಿ ಹೋಗುತ್ತಿತ್ತು.

ಶ್ರೀರಾಮಕೃಷ್ಣರು ಮಾತ್ರ ಆ ಅಸಹನೀಯ ಯಾತನೆಯ ನಡುವೆಯೂ ಆನಂದಮೂರ್ತಿ ಯಾಗಿಯೇ ತೋರುತ್ತಿದ್ದರು. ಆದರೆ ಕೆಲವು ಸಲ ನೋವು ವಿಪರೀತವಾಗಿ ಬಿಡುತ್ತಿತ್ತು. ಆಗ ಅವರು ಮುಗುಳ್ನಗುತ್ತ ತಮಗೆ ತಾವೇ ಹೇಳಿಕೊಳ್ಳುತ್ತಿದ್ದರು: “ಈ ಶರೀರವೂ ನೋವೂ ಪರಸ್ಪರ ಕಿತ್ತಾಡಿಕೊಳ್ಳುತ್ತಿರಲಿ; ಓ ಮನಸ್ಸೇ, ನಿನ್ನ ಪಾಡಿಗೆ ನೀನು ಆನಂದದಿಂದಿದ್ದುಬಿಡು!” ಅವರಿಗೆ ಆ ಭಯಂಕರ ಯಾತನೆಯ ಅನುಭವ ಆಗುತ್ತಿರಲಿಲ್ಲವೆಂದಲ್ಲ; ಆದರೆ ಸಹಿಸಿಕೊಳ್ಳುತ್ತಿದ್ದರು. ನೋವನ್ನು ಶರೀರಕ್ಕೆ ಬಿಟ್ಟು ತಾವು ನಿಶ್ಚಿಂತರಾಗಿದ್ದುಬಿಟ್ಟಿದ್ದರು! ಒಂದು ದಿನ ರಾತ್ರಿ ಅವರು ಮಹೇಂದ್ರನಾಥನಿಗೆ ಹೇಳುತ್ತಾರೆ: “ನೋಡು, ನಾನಿದನ್ನೆಲ್ಲ ಸಹಿಸಿಕೊಳ್ಳುತ್ತಿರುವುದು ಏಕೆ ಗೊತ್ತೇನು? ಇಲ್ಲದಿದ್ದರೆ ನೀವೆಲ್ಲ ಅಳುತ್ತ ಕುಳಿತುಬಿಡುತ್ತೀರಿ. ನೀವೆಲ್ಲರೂ ಈಗ, ಈ ಶರೀರ ಹೀಗೆ ಸುಮ್ಮನೆ ಯಾತನೆಪಡುವುದಕ್ಕಿಂತ ಹೊರಟುಹೋದರೇ ಒಳ್ಳೆಯದು ಎನ್ನುವುದಾದರೆ ಅದು ಹೊರಟುಹೋಗಿಬಿಡುತ್ತದೆ.” ಮಹೇಂದ್ರನಾಥ ಈ ಮಾತಿಗೆ ಏನೆಂದು ಉತ್ತರಿಸಿಯಾನು? ಸುಮ್ಮನೆ ತಲೆತಗ್ಗಿಸಿಕೊಂಡು ನಿಂತ.

ಅಂದು ಮಾರ್ಚ್ ತಿಂಗಳ ಹದಿನೈದನೆಯ ದಿನ, ಬೆಳಗ್ಗೆ ಸುಮಾರು ಏಳು ಗಂಟೆ. ಹಿಂದಿನ ರಾತ್ರಿಯಂತೂ ಅವರ ಯಾತನೆ ತುಂಬ ತೀವ್ರವಾಗಿಬಿಟ್ಟಿತ್ತು. ಅವರೇನೋ ಅದನ್ನು ಬ್ರಹ್ಮಭಾವ ದಿಂದ ಸಹಿಸಿಕೊಂಡಿರಬಹುದು, ಆದರೆ ಯಾತನೆ ಯಾತನೆಯೇ. ಈಗ ಅವರು ಸ್ವಲ್ಪ ಸುಧಾರಿಸಿ ದಂತೆ ಕಂಡುಬರುತ್ತಿದ್ದರು. ನರೇಂದ್ರ, ರಾಖಾಲ, ಲಾಟು, ಮಹೇಂದ್ರ, ಹಿರಿಯ ಗೋಪಾಲ ಮತ್ತಿತರರು ಅಲ್ಲಿದ್ದರು. ಎಲ್ಲರ ಮನಸ್ಸೂ ಖಿನ್ನವಾಗಿತ್ತು. ಶ್ರೀರಾಮಕೃಷ್ಣರು ಮಾತ್ರ ತಮ್ಮ ದಿವ್ಯ ಭಾವದಲ್ಲೇ ಇದ್ದರು. ಸುತ್ತಲಿದ್ದ ಯುವಶಿಷ್ಯರ ಕಡೆಗೆ ನೋಡುತ್ತ ಮೆಲುದನಿಯಲ್ಲಿ, ಕೆಲವೊಮ್ಮೆ ಕೇವಲ ಸಂಜ್ಞೆಗಳ ಮೂಲಕ ಮಾತನಾಡಲಾರಂಭಿಸಿದರು:

“ನಾನೀಗ ಈ ಕ್ಷಣದಲ್ಲಿ ಏನು ಕಾಣುತ್ತಿದ್ದೇನೆ ಗೊತ್ತೆ? ಸಕಲವೂ ನನಗೆ ಸಾಕ್ಷಾತ್ ಭಗವಂತನಂತೆಯೇ ಕಾಣುತ್ತಿದೆ! ಜನರೂ ಜೀವಕೋಟಿಗಳೆಲ್ಲರೂ ಕೇವಲ ಚರ್ಮದ ಚೀಲ ಗಳು; ಆ ಚೀಲಗಳ ಒಳಗೆ ಭಗವಂತ ವಾಸವಾಗಿದ್ದುಕೊಂಡು, ಕೈ ಕಾಲು ತಲೆಗಳನ್ನೆಲ್ಲ ಆಡಿಸುತ್ತಿದ್ದಾನೆ ಎಂಬಂತೆ ನನಗೆ ಕಾಣಿಸುತ್ತದೆ. ಹಿಂದೊಮ್ಮೆ ನನಗೆ ಇದೇ ರೀತಿಯ ದರ್ಶನ ವಾಗಿತ್ತು. ಆಗ ನನಗೆ ಮನೆಗಳು ಉದ್ಯಾನಗಳು ರಸ್ತೆಗಳು ಜನಗಳು ದನಕರುಗಳು–ಎಲ್ಲವೂ ಒಂದೇ ರೀತಿಯ ವಸ್ತುವಿನಿಂದ ಮಾಡಲ್ಪಟ್ಟಂತೆ ಕಂಡುಬಂದುವು; ಅವುಗಳೆಲ್ಲವೂ ಮೇಣ ದಿಂದ ತಯಾರಾದವೋ ಎಂಬಂತಿತ್ತು.

“ನ್ಯಾಯಾಧೀಶ, ಅಪರಾಧಿ, ಗಲ್ಲುಗಂಬ ಎಲ್ಲವೂ ಭಗವಂತನೇ ಆಗಿದ್ದಾನೆ ಎನ್ನುವುದು ನನಗೆ ಸ್ಪಷ್ಟವಾಗಿ ಕಾಣುತ್ತಿದೆ.”

ಹೀಗೆ ಶಿಷ್ಯರ ಮುಂದೆ ತಮ್ಮ ಅಲೌಕಿಕ ಅನುಭವಗಳನ್ನು ವರ್ಣಿಸುತ್ತಿದ್ದಂತೆ, ಸರ್ವವ್ಯಾಪಿ ಯಾದ ಭಗವಂತನ ಲೀಲೆಯನ್ನು ನೆನೆದು ಶ್ರೀರಾಮಕೃಷ್ಣರ ಮನಸ್ಸು ಅದೇ ಭಾವದಿಂದ ರಂಜಿತವಾಯಿತು; “ಆಹಾ! ಎಂಥ ದರ್ಶನ!” ಎಂದು ಉದ್ಗರಿಸುತ್ತ ಗಾಢ ಸಮಾಧಿಮಗ್ನರಾಗಿ ಬಿಟ್ಟರು. ಅವರಿಗೀಗ ಶರೀರದ ಪರಿವೆಯೂ ಇಲ್ಲ, ಬಾಹ್ಯ ಜಗತ್ತಿನ ಪರಿವೆಯೂ ಇಲ್ಲ. ಇಂತಹ ದಾರುಣ ಯಾತನೆಯ ಸ್ಥಿತಿಯಲ್ಲೂ ಅವರು ಹೀಗೆ ಸಮಾಧಿಸ್ಥರಾದುದನ್ನು ಕಂಡ ಭಕ್ತರೆಲ್ಲ ಆಶ್ಚರ್ಯದಿಂದ ದಿಙ್ಮೂಢರಾಗಿ ಕುಳಿತುಬಿಟ್ಟರು. ಸ್ವಲ್ಪ ಹೊತ್ತಿನಲ್ಲೇ ಶ್ರೀರಾಮಕೃಷ್ಣರಿಗೆ ಅರ್ಧಬಾಹ್ಯಪ್ರಜ್ಞೆ ಮರಳಿತು. “ಈಗ ನನಗೆ ಯಾವ ನೋವೂ ಇಲ್ಲ. ನಾನೀಗ ಮೊದಲಿನಂತೆಯೇ ಇದ್ದೇನೆ” ಎಂದು ಉದ್ಗರಿಸಿದರು. ಅಲ್ಲೇ ಕುಳಿತಿದ್ದ ಲಾಟುವಿನ ಕಡೆಗೆ ದೃಷ್ಟಿ ಹರಿಸಿ ನುಡಿದರು: “ಅಗೋ ಅಲ್ಲಿ ಲಾಟು ತಲೆಗೆ ಕೈಕೊಟ್ಟುಕೊಂಡು ಕುಳಿತಿದ್ದಾನೆ. ಹಾಗೆ ಕುಳಿತಿರುವುದು ಸಾಕ್ಷಾತ್ ಭಗವಂತನೇ ಎಂಬಂತೆ ನನಗೆ ಕಾಣುತ್ತಿದೆ.”

ಈಗ ಅವರು ತಮ್ಮ ಸುತ್ತ ಕುಳಿತಿದ್ದ ಭಕ್ತರೆಡೆಗೆ ದೃಷ್ಟಿ ಹರಿಸಿದರು. ಆ ನೋಟದಲ್ಲಿ ಅವರ ಪ್ರೀತಿ ಪ್ರವಾಹದೋಪಾದಿಯಲ್ಲಿ ಹರಿದು ಬರುವಂತಿತ್ತು. ತಾಯಿ ಅಕ್ಕರೆಯಿಂದ ಮಕ್ಕಳನ್ನು ಮುದ್ದಿಸುವಂತೆ, ಪರಮವಾತ್ಸಲ್ಯದಿಂದ ನರೇಂದ್ರ ಹಾಗೂ ರಾಖಾಲರ ಮುಖಗಳನ್ನು ನೇವರಿಸಿ ದರು. ಕೆಲವು ನಿಮಿಷಗಳಾದ ಮೇಲೆ ಮಹೇಂದ್ರನನ್ನುದ್ದೇಶಿಸಿ ನುಡಿದರು: “ಈ ಶರೀರ ಇನ್ನೂ ಕೆಲವು ದಿನ ಹೆಚ್ಚಿಗೆ ಉಳಿದಿದ್ದರೆ, ಇನ್ನೂ ಹಲವಾರು ಜನ ಇದರ ಮೂಲಕ ಆಧ್ಯಾತ್ಮಿಕ ಜಾಗೃತಿ ಮಾಡಿಕೊಳ್ಳಬಹುದಾಗಿತ್ತು.”

ಸ್ವಲ್ಪ ಹೊತ್ತು ಸುಮ್ಮನಿದ್ದು, ಮತ್ತೆ: “... ಆದರೆ ಅದು ಹಾಗಾಗುವುದಿಲ್ಲ. ಈ ಸಲ ಇನ್ನು ಇದು ಉಳಿಯುವುದಿಲ್ಲ.”

ಶ್ರೀರಾಮಕೃಷ್ಣರು ಈಗ ಇನ್ನೇನು ಹೇಳಬಹುದು ಎಂದು ಭಕ್ತರೆಲ್ಲ ಕತ್ತು ಚಾಚಿಕೊಂಡು ಕಾತರರಾಗಿ ಕುಳಿತಿದ್ದಾರೆ. 

ಶ್ರೀರಾಮಕೃಷ್ಣರು: “ಇಲ್ಲ, ಈ ಶರೀರ ಉಳಿಯುವುದು ಭಗವಂತನಿಗೆ ಇಷ್ಟವಿಲ್ಲ. ಇನ್ನವನು ಇದನ್ನು ಉಳಿಯಗೊಡುವುದಿಲ್ಲ. ಏಕೆಂದರೆ ನನ್ನ ಮುಗ್ಧತೆಯನ್ನು, ಸರಳತೆಯನ್ನು ಕಂಡು ಜನ ನನ್ನನ್ನು ದುರುಪಯೋಗಪಡಿಸಿಕೊಳ್ಳುವಂತಾಗದಿರಲಿ, ಅಲ್ಲದೆ ನಾನೂ ನನ್ನ ಸರಳ ಸ್ವಭಾವ ದಿಂದಾಗಿ ಯಾರೆಂದರೆ ಅವರಿಗೆ ಎಲ್ಲವನ್ನೂ ಕೊಟ್ಟುಬಿಡದಂತಾಗಲಿ ಅಂತ... ಈ ಕಲಿಯುಗ ದಲ್ಲಿ ಯಾರಿಗೂ ಜಪಧ್ಯಾನ ಮಾಡಲು ಇಷ್ಟವಿಲ್ಲ.”

ರಾಖಾಲ (ಕಾತರತೆಯಿಂದ): “ನಿಮ್ಮ ಶರೀರವನ್ನು ಇನ್ನೂ ಕೆಲಕಾಲ ಉಳಿಸುವಂತೆ ದಯವಿಟ್ಟು ಭಗವಂತನಿಗೆ ಹೇಳಿ.”

ಶ್ರೀರಾಮಕೃಷ್ಣರು: “ಅದು ಉಳಿಯುವುದೂ ಬಿಡುವುದೂ ಅವನ ಇಚ್ಛೆಯನ್ನು ಅವಲಂಬಿಸಿದೆ.”

ನರೇಂದ್ರ: “ಆದರೆ ನಿಮ್ಮ ಇಚ್ಛೆಯೂ ಭಗವಂತನ ಇಚ್ಛೆಯೂ ಒಂದೇ ಆಗಿಬಿಟ್ಟಿದೆಯಲ್ಲ?”

ಶ್ರೀರಾಮಕೃಷ್ಣರು ಈಗ ಏನೋ ಆಲೋಚನೆ ಮಾಡುವವರಂತೆ ಮೌನ ತಾಳಿದರು. ಬಳಿಕ ನುಡಿದರು: “ಈಗ ನಾನು ಭಗವಂತನಿಗೆ ಹೇಳಿದರೂ ಅದರಿಂದೇನೂ ಪ್ರಯೋಜನವಾಗು ವಂತಿಲ್ಲ. ಏಕೆಂದರೆ ಈಗ ನಾನೂ ಆತನೂ ಒಂದಾಗಿಬಿಟ್ಟಿರುವಂತೆ ಸ್ಪಷ್ಟವಾಗಿ ಕಾಣುತ್ತಿದ್ದೇನೆ.... ”

ಹೀಗೆ ಹೇಳಿ ಮೌನ ತಾಳಿದರು. ಅರೆ ಘಳಿಗೆ ಬಿಟ್ಟು ಮತ್ತೆ ನುಡಿದರು:

“ಈ ಶರೀರದಲ್ಲಿ ಇಬ್ಬರು ವ್ಯಕ್ತಿಗಳಿದ್ದಾರೆ. ಒಬ್ಬಳು ಜಗನ್ಮಾತೆ... ಹೌದು, ಒಬ್ಬಳು ಆಕೆ; ಇನ್ನೊಬ್ಬ ಆಕೆಯ ಭಕ್ತ. ಹಿಂದೆ ಕೈಮುರಿದುಕೊಂಡಿದ್ದವನು\footnote{*ಆಗ ಎರಡು ವರ್ಷಗಳ ಕೆಳಗೆ ಶ್ರೀರಾಮಕೃಷ್ಣರೊಮ್ಮೆ ಭಾವಾವಸ್ಥೆಯಲ್ಲಿದ್ದಾಗ ಬಿದ್ದು ಕೈಮೂಳೆ ಮುರಿದುಕೊಂಡಿದ್ದರು.} ಆ ಭಕ್ತನೇ. ಈಗ ಕಾಯಿಲೆ ಅನುಭವಿಸುತ್ತಿರುವವನೂ ಆ ಭಕ್ತನೇ. ಏನು ಅರ್ಥವಾಯಿತೇ?”

ಭಕ್ತಾದಿಗಳೆಲ್ಲ ಉಸಿರು ಹಿಡಿದು ಕುಳಿತಿದ್ದಾರೆ.

ಶ್ರೀರಾಮಕೃಷ್ಣರು: “ಆಹಾ! ನಾನು ಇದನ್ನೆಲ್ಲ ನೀವಲ್ಲದೆ ಇನ್ಯಾರ ಹತ್ತಿರ ಹೇಳಲಿ? ಯಾರು ನನ್ನನ್ನು ಅರ್ಥ ಮಾಡಿಕೊಳ್ಳಬಲ್ಲರು? ನೋಡಿ, ಭಗವಂತ ತನ್ನ ಭಕ್ತರ ಜೊತೆಗೂಡಿ ಕೊಂಡು ಮನುಷ್ಯರೂಪದಿಂದ ಈ ಭೂಮಿಗೆ ಬರುತ್ತಾನೆ. ಅವನು ಹೊರಟುಹೋಗುವಾಗ ಈ ಭಕ್ತರು ಅವನ ಜೊತೆಯಲ್ಲೇ ಹೊರಟುಬಿಡುತ್ತಾರೆ.”

ತಕ್ಷಣ ರಾಖಾಲನೆಂದ: “ಆದ್ದರಿಂದಲೇ ನಾವು ನಿಮ್ಮನ್ನು ಕೇಳಿಕೊಳ್ಳುತ್ತಿರುವುದು– ನಮ್ಮನ್ನೆಲ್ಲ ಬಿಟ್ಟು ನೀವು ಹೊರಟುಹೋಗಬಾರದು ಅಂತ.”

ಶ್ರೀರಾಮಕೃಷ್ಣರು (ಮುಗುಳ್ನಕ್ಕು): “ಎಲ್ಲಿಂದಲೋ ಒಂದು ಗಾಯಕರ ಮೇಳದವರು ಬಂದರು, ಹಾಡಿದರು, ಕುಣಿದರು; ಬಳಿಕ ಹೇಗೆ ಬಂದರೋ ಹಾಗೇ ಇದ್ದಕ್ಕಿದ್ದಂತೆ ಮಾಯ ವಾದರು. ಅವರು ಬರುತ್ತಾರೆ, ಹೊರಟುಹೋಗುತ್ತಾರೆ. ಆದರೆ ಯಾರೂ ಅವರನ್ನು ಗುರುತಿಸುವುದಿಲ್ಲ.”

ಇದನ್ನು ಕೇಳಿ ಭಕ್ತರೂ ಮುಗುಳ್ನಕ್ಕರು.

ನರೇಂದ್ರ: “ಮಹಾಶಯರೆ, ನಾನು ತ್ಯಾಗದ ಮಾತೆತ್ತಿದರೆ ಕೆಲವರು ಕೋಪಿಸಿಕೊಳ್ಳುತ್ತಾರಲ್ಲ?”

ಶ್ರೀರಾಮಕೃಷ್ಣರು: “ತ್ಯಾಗ ತುಂಬ ಆವಶ್ಯಕವಾದದ್ದು. ಒಂದು ವಸ್ತುವಿನ ಮೇಲೆ ಇನ್ನೊಂದು ವಸ್ತು ಇದೆ ಅಂತ ಇಟ್ಟುಕೊ. ಈಗ ನಿನಗೆ ಆ ಕೆಳಗಿನ ವಸ್ತು ಬೇಕಿದ್ದರೆ, ಅದರ ಮೇಲಿರುವ ವಸ್ತುವನ್ನು ಎತ್ತಿ ತೆಗೆಯಲೇಬೇಕು. ಮೇಲಿರುವ ವಸ್ತುವನ್ನು ತೆಗೆಯದೆಯೇ ಕೆಳಗಿರುವ ವಸ್ತುವನ್ನು ತೆಗೆದುಕೊಳ್ಳಲು ಸಾಧ್ಯವೇ?”

ನರೇಂದ್ರ: “ಹೌದು, ನೀವೆನ್ನುವುದು ನಿಜ.”

ತ್ಯಾಗದ ಅವಶ್ಯಕತೆಯನ್ನು ಮನಗಾಣಿಸಲು ಶ್ರೀರಾಮಕೃಷ್ಣರು ಕೊಟ್ಟ ಉದಾಹರಣೆ ತುಂಬ ಅದ್ಬುತವಾಗಿದೆ. ಭಗವಂತನನ್ನು ಕಾಣಲು ನಮಗೆ ಬೇಕಾದ ವಸ್ತು ಯೋಗ. ಆದರೆ ಈ ಯೋಗದ ಮೆಲೆ ಭೋಗ ಹೇರಿಕೊಂಡಿದೆ. ಈಗ ನಮಗೆ ಬೇಕಾದ ಯೋಗವನ್ನು ಪಡೆದುಕೊಳ್ಳಬೇಕಾದರೆ ಮೇಲಿರುವ ಭೋಗವನ್ನು ಎತ್ತಿ ಸರಸಲೇಬೇಕಾಗುತ್ತದೆ. ಹೀಗೆ ಭೋಗವನ್ನು ಎತ್ತಿ ಸರಿಸುವುದಕ್ಕೆ ತ್ಯಾಗ ಎಂದು ಹೆಸರು. ಭೋಗವನ್ನು ತ್ಯಾಗ ಮಾಡಿದಾಗ ಯೋಗ ಲಭಿಸುತ್ತದೆ. ಎಂಬುದನ್ನೇ ಶ್ರೀರಾಮಕೃಷ್ಣರು ಈ ಎರಡು ವಸ್ತುಗಳ ಉದಾಹರಣೆಯನ್ನಿತ್ತು ವಿವರಿಸುತ್ತಾರೆ.

ಶ್ರೀರಾಮಕೃಷ್ಣರು ವಾತ್ಸಲ್ಯ ತುಂಬಿದ ದೃಷ್ಟಿಯಿಂದ ನರೇಂದ್ರನನ್ನೊಮ್ಮೆ ದಿಟ್ಟಿಸಿದರು;\\ಬಳಿಕ ಇತರೆಡೆಗೆ ತಿರುಗಿ ಉದ್ಗರಿಸಿದರು: “ಅದ್ಭುತ!”

ನರೇಂದ್ರ: “ಏನುದು ಅದ್ಭುತ?”

ಶ್ರೀರಾಮಕೃಷ್ಣರು: “ಒಳ್ಳೇ ಭರದಿಂದ ತಯಾರಿ ನಡೆಯುತ್ತಿದೆ ಅದ್ಭುತವಾದ ತ್ಯಾಗಕ್ಕೆ!”

ನರೇಂದ್ರಾದಿಗಳು ಅದ್ಭುತವಾದ ಮಹಾತ್ಯಾಗಕ್ಕೆ ಮನಸ್ಸು ಮಾಡಿರುವುದನ್ನು ಅವರು ಸ್ಫುಟವಾಗಿ ಕಾಣುತ್ತಿದ್ದರು. ಎಲ್ಲರೂ ಮೌನವಾಗಿ ಶ್ರೀರಾಮಕೃಷ್ಣರನ್ನೇ ದಿಟ್ಟಿಸುತ್ತಿದ್ದಾರೆ.

ರಾಖಾಲ: “ನರೇಂದ್ರ ಈಗೀಗ ನಿಮ್ಮನ್ನು ಚೆನ್ನಾಗಿ ಅರ್ಥಮಾಡಿಕೊಳ್ಳುತ್ತಿದ್ದಾನೆ.”

ಶ್ರೀರಾಮಕೃಷ್ಣರು (ನಗುತ್ತ): “ಹೌದು ಮತ್ತೆ! ಅವನು ಮಾತ್ರವೇ ಅಲ್ಲ, ಇನ್ನೂ ಹಲವರೆಲ್ಲ ಅರ್ಥ ಮಾಡಿಕೊಳ್ಳುತ್ತಿದ್ದಾರೆ!”

ಸ್ವಲ್ಪ ಹೊತ್ತಿನ ಮೇಲೆ ಶ್ರೀರಾಮಕೃಷ್ಣರೆಂದರು: “ಪ್ರತಿಯೊಂದು ವಸ್ತುವೂ–ಈ ಜಗತ್ತಿ ನಲ್ಲಿ ಏನೇನಿದೆಯೋ ಅವೆಲ್ಲವೂ–ಇದರೊಳಗಿನಿಂದಲೇ ಹೊರಬಂದಿದೆ.”

‘ಇದರೊಳಗಿನಿಂದಲೇ’ ಎನ್ನುವಾಗ ಅವರು ತೋರಿಸಿದ್ದು ತಮ್ಮ ಶರೀರವನ್ನು. ಬಳಿಕ ನರೇಂದ್ರನನ್ನು ಸಂಜ್ಞೆಯಿಂದಲೇ ಕೇಳಿದರು: “ಏನು, ನಿನಗೇನು ಅರ್ಥವಾಯಿತು?”

ನರೇಂದ್ರ: “ಸೃಷ್ಟಿಯ ಪ್ರತಿಯೊಂದು ವಸ್ತುವೂ ನಿಮ್ಮೊಳಗಿಂದಲೇ ಬಂದಿದೆ.”

ನರೇಂದ್ರನ ಈ ಉತ್ತರವನ್ನು ಕೇಳಿ ಶ್ರೀರಾಮಕೃಷ್ಣರ ಮುಖ ಆನಂದದಿಂದ ಬೀಗಿತು. ರಾಖಾಲನಿಗೆಂದರು: “ಕೇಳಿಸಿಕೊಂಡೆಯಾ ಅವನು ಹೇಳಿದ್ದನ್ನು?”

ಶ್ರೀರಾಮಕೃಷ್ಣರ ಈ ನುಡಿಗಳನ್ನು ಕೇಳುತ್ತಿದ್ದಂತೆ ನರೇಂದ್ರನಿಗೆ ಅನೇಕ ಹೊಸ ವಿಚಾರಗಳು ತಿಳಿದುಬಂದುವು. ಅವನು ಶ್ರೀರಾಮಕೃಷ್ಣರ ಬಳಿಗೆ ಬರುವಾಗಲೇ ಹಲವಾರು ಸಂಶಯಗಳನ್ನೆಲ್ಲ ಹೊತ್ತು ತಂದಿದ್ದ. ಅವನಿಗೆ ಭಗವಂತನ ಅಸ್ತಿತ್ವದಲ್ಲಿಯೇ ತೀವ್ರ ಸಂಶಯ; ಒಂದು ವೇಳೆ ಭಗವಂತ ಇರುವುದೇ ನಿಜವಾದರೂ ಅವನ ಸ್ವಭಾವ-ಸ್ವರೂಪಗಳ ವಿಷಯದಲ್ಲಿ ಸಂಶಯ– ಹೀಗೆ ಸಂಶಯಗಳ ರಾಶಿಯನ್ನೇ ಹೊತ್ತು ತಂದಿದ್ದ. ಶ್ರೀರಾಮಕೃಷ್ಣರು ಮೊತ್ತಮೊದಲು ತಮ್ಮ ಸ್ಪರ್ಶಮಾತ್ರದಿಂದ ಅವನ ಮನಸ್ಸನ್ನು ಸಮಾಧಾನಗೊಳಿಸಿದರು. ಬಳಿಕ ಬೌದ್ಧಿಕವಾಗಿ ಪರ ಮಾತ್ಮನ ವಿಷಯವನ್ನು ತಿಳಿಯಹೇಳಿದರು. ಸೂಕ್ಷ್ಮಬುದ್ಧಿಯ ನರೇಂದ್ರ ಅವೆಲ್ಲವನ್ನೂ ಅರಗಿಸಿ ಕೊಂಡ. ಆದರೆ ಕೇವಲ ಬೌದ್ಧಿಕವಾಗಿ ತಿಳಿಯುವುದಕ್ಕಿಂತ ಹೆಚ್ಚಾಗಿ, ಆ ಪರಬ್ರಹ್ಮವಸ್ತುವನ್ನು ಸಾಕ್ಷಾತ್ಕರಿಸಿಕೊಳ್ಳುವುದರ ಮೂಲಕ, ನೇರ ಅನುಭವ ಮಾಡಿಕೊಳ್ಳಲು ಹಾತೊರೆದ. ತನ್ನಾತ್ಮ ದಲ್ಲೇ ತಾನು ಮುಳುಗಿ ಸಮಾಧಿಮಗ್ನನಾಗಿದ್ದುಬಿಡಬೇಕೆಂದು ಹಂಬಲಿಸಿದ. ಆದರೆ ಶ್ರೀರಾಮ ಕೃಷ್ಣರು ಅವನಿಗೆ ಅದಕ್ಕಿಂತಲೂ ಉನ್ನತವಾದ ಸ್ಥಿತಿಯ ಅನುಭವ ಮಾಡಿಸಿಕೊಡುವುದಾಗಿ ಹೇಳಿದರು. ‘ಸರ್ವಂ ಖಲ್ವಿದಂ ಬ್ರಹ್ಮ’–ಎಂದರೆ, ಏನೇನಿದೆಯೋ ಅದೆಲ್ಲ ಬ್ರಹ್ಮವೇ ಆಗಿದೆ; ಯಾವ ಪರಬ್ರಹ್ಮವಸ್ತುವು ಎಲ್ಲಕ್ಕೂ ಅತೀತವಾಗಿದೆಯೋ ಅದೇ ಪರಬ್ರಹ್ಮವಸ್ತುವೇ ಅಂತರ್ಯಾಮಿಯಾಗಿಯೂ ಇದೆ–ಎಂಬ ಅನುಭವವಾದಾಗ, ಅದೇ ಪರಮಾರ್ಥಸತ್ಯವೇ ಈ ವ್ಯಾವಹಾರಿಕ ಜಗತ್ತಾಗಿ ವ್ಯಕ್ತವಾಗಿದೆ ಎನ್ನುವ ಅರಿವಾಗುತ್ತದೆ. ಯಾರಿಗೆ ಈ ಬಗೆಯ ಅತ್ಯುನ್ನತ ಅನುಭವವಾಗುತ್ತದೆಯೋ ಅವರು ಆ ಪಾರಮಾರ್ಥಿಕ ಸತ್ಯದ ಎತ್ತರಕ್ಕೂ ಏರಬಲ್ಲರು, ಈ ವ್ಯಾವಹಾರಿಕ ಜಗತ್ತಿಗೂ ಇಳಿದುಬರಬಲ್ಲರು. ಶ್ರೀರಾಮಕೃಷ್ಣರು ನರೇಂದ್ರನಿಗೆ ಇಂತಹ ಅತ್ಯುನ್ನತ ಆಧ್ಯಾತ್ಮಿಕ ಅನುಭವವನ್ನು ಮಾಡಿಸಿಕೊಡುವುದಾಗಿ ಹೇಳಿದರು. ನರೇಂದ್ರ ಈಗೀಗ ಆಲೋಚಿಸತೊಡಗಿದ–‘ನಿರಾಕಾರವಾದ ಪರಬ್ರಹ್ಮವೇ ಈ ಜಗತ್ತಾಗಿ ವ್ಯಕ್ತಗೊಂಡಿದೆಯೆಂದ ಮೇಲೆ, ಅದು ಮಾನವರೂಪದಿಂದ ವ್ಯಕ್ತವಾಗಲು ಸಾಧ್ಯವಿಲ್ಲವೆ?’ ಎಂದು. ಈ ಜಗದ್ಭಾವವನ್ನು ಅತಿಕ್ರಮಿಸಿ ಬ್ರಹ್ಮಭಾವವನ್ನು ತಲುಪಿದ ಮೇಲೆ ಪುನಃ ಕೆಳಗಿಳಿದು ಜಗದಾತ್ಮಭಾವದಿಂದಿರು ವುದೇ ಉನ್ನತವಾದ ಸ್ಥಿತಿ ಎಂದು ಅವನು ಕಂಡುಕೊಂಡ. ಮತ್ತೂ ಮುಂದುವರಿದು, ಭಕ್ತಿ ಮತ್ತು ಜ್ಞಾನ ಇವೆರಡೂ ಒಂದೇ ಗುರಿಗೆ ಒಯ್ಯುವ ಎರಡು ಮಾರ್ಗಗಳು ಎಂಬ ಬಹುಮುಖ್ಯ ಅಂಶವನ್ನು ಅವನು ಮನಗಂಡ. ಅಲ್ಲದೆ, ಅತ್ಯುನ್ನತ ಸಾಕ್ಷಾತ್ಕಾರದ ಪರಾಕಾಷ್ಠೆಯೇ ಪರಮ ಪ್ರೇಮ ಎಂಬುದೂ ಅವನಿಗೆ ತಿಳಿಯಿತು.

ಈಗ ಸುಮಾರು ಒಂದು ವರ್ಷದಿಂದಲೂ ಅವನ ಮನಸ್ಸು ಭಗವಾನ್ ಬುದ್ಧನಿಂದ ತೀವ್ರವಾಗಿ ಆಕರ್ಷಿತವಾಗಿಬಿಟ್ಟಿತ್ತು. ಬುದ್ಧನ ಜೀವನ-ಸಂದೇಶಗಳ ಬಗ್ಗೆ ಅವನೂ ಇತರ ಯುವಶಿಷ್ಯರೂ ಆಳವಾಗಿ ಚರ್ಚಿಸುತ್ತ ಹೊಸ ಸ್ಫೂರ್ತಿ ಗಳಿಸುತ್ತಿದ್ದರು. ನರೇಂದ್ರನಂತೂ ಒಂದು ವಿಧದಲ್ಲಿ ಬುದ್ಧನ ಅನುಯಾಯಿಯೇ ಆಗಿಬಿಟ್ಟಿದ್ದ ಎನ್ನಬೇಕು–ಅಷ್ಟರ ಮಟ್ಟಿಗೆ ಆತ ಬುದ್ಧನ, ಬೌದ್ಧಧರ್ಮದ ಚಿಂತನೆಯಲ್ಲಿ ಮುಳುಗಿ ಹೋಗಿದ್ದ. ಬುದ್ಧನ ಪರ್ವತೋಪಮವಾದ ಬುದ್ಧಿಮತ್ತೆ, ಸುಸಂಬದ್ಧ ವಿಚಾರದೃಷ್ಟಿ, ಅವನ ಸತ್ಯಸಾಕ್ಷಾತ್ಕಾರದ ಅದಮ್ಯ ಆಕಾಂಕ್ಷೆ, ಉಜ್ವಲ ತ್ಯಾಗಬುದ್ಧಿ, ಕರುಣಾಪೂರ್ಣ ಹೃದಯ, ಅವನ ಮಧುರ-ಗಂಭೀರ ತೇಜೋಮಯ ವ್ಯಕ್ತಿತ್ವ, ಉದಾತ್ತ ನಿಷ್ಕಳಂಕ ಚಾರಿತ್ರ್ಯ, ಇಂದ್ರಿಯಗ್ರಾಹ್ಯವಲ್ಲದ ಆಧ್ಯಾತ್ಮಜೀವನದ ಹಾಗೂ ಮಾನುಷ ಜೀವನದ ಮಧ್ಯೆ ಅವನು ಸಮತೋಲ-ಸಾಮರಸ್ಯಗಳನ್ನು ತೋರಿಸಿದ ರೀತಿ ಇವುಗಳೆಲ್ಲ ನರೇಂದ್ರನಲ್ಲಿ ಒಂದು ಪ್ರಚಂಡ ಭಾವತರಂಗವನ್ನೇ ಎಬ್ಬಿಸಿಬಿಟ್ಟಿದ್ದುವು. ಈ ಭಾವೋತ್ಸಾಹ ವೆಂಬುದು ಅವನ ಇತರ ಸ್ನೇಹಿತರಲ್ಲೂ ಆ ಭಾವವನ್ನು ಜಾಗೃತಗೊಳಿಸಿಬಿಟ್ಟಿತ್ತು. ಬುದ್ಧ ಭಗವಂತನಂತೆಯೇ ತಾವೂ ಪ್ರಾಣದ ಹಂಗು ತೊರೆದಾದರೂ ಸಾಕ್ಷಾತ್ಕಾರ ಮಾಡಿಕೊಳ್ಳಲೇ ಬೇಕು ಎಂಬ ಛಲ ಅವರಲ್ಲುಂಟಾಗಿತ್ತು. ತಾವು ಧ್ಯಾನಮಾಡುತ್ತಿದ್ದ ಕೋಣೆಯ ಗೋಡೆಯ ಮೇಲೆ ಬುದ್ಧನ ಈ ಮಾತನ್ನು ದೊಡ್ಡ ಅಕ್ಷರಗಳಲ್ಲಿ ಬರೆದಿಟ್ಟುಕೊಂಡಿದ್ದರು: \textit{“ಈ ದೇಹವು ಇದೇ ಆಸನದ ಮೇಲೆ ಒಣಗಿಹೋಗಲಿ, ಮೂಳೆಮಾಂಸಗಳು ಕರಗಿಹೋಗಲಿ, ಒಂದು ಯುಗವೇ ಕಳೆದುಹೋಗಲಿ–ಸತ್ಯಸಾಕ್ಷಾತ್ಕಾರವಾಗುವವರಗೆ ಈ ದೇಹ ಮಾತ್ರ ತನ್ನ ಆಸನವನ್ನು ಬಿಟ್ಟು ಏಳುವುದಿಲ್ಲ!”} ಗೌತಮ ಬುದ್ಧನು ಈ ದೃಢಸಂಕಲ್ಪವನ್ನು ಮಾಡಿ ಕುಳಿತು ಸತ್ಯಸಾಕ್ಷಾತ್ಕಾರ ಮಾಡಿಕೊಂಡದ್ದು ಬುದ್ಧಗಯೆಯಲ್ಲಿ. ಆದ್ದರಿಂದ ನರೇಂದ್ರನ ಮನಸ್ಸು ಬುದ್ಧಗಯೆಯ ಕಡೆಗೆ ವಿಶೇಷವಾಗಿ ಆಕರ್ಷಿತವಾಗುತ್ತಿತ್ತು. ತಾನೂ ಒಮ್ಮೆ ಆ ಪವಿತ್ರ ಸ್ಥಳಕ್ಕೆ ಹೋಗಿ, ಬೋಧಿವೃಕ್ಷ ದಡಿಯಲ್ಲಿ ಧ್ಯಾನನಿರತನಾಗಬೇಕು ಎಂಬ ಹಂಬಲ ಅವನಲ್ಲಿದ್ದೇ ಇತ್ತು. ಈ ವಿಷಯವನ್ನು ತಾರಕನಾಥ ಹಾಗೂ ಕಾಳೀಪ್ರಸಾದ ಇವರಿಬ್ಬರಿಗೆ ಮಾತ್ರ ತಿಳಿಸಿದ್ದ.

ಈಗ ಭಗವತ್ಸಾಕ್ಷಾತ್ಕಾರದ ಹಂಬಲ ನರೇಂದ್ರನಲ್ಲಿ ಭುಗಿಲೆದ್ದುಬಿಟ್ಟಿದೆ. ಆದ್ದರಿಂದ ೧೮೮೬ರ ಏಪ್ರಿಲ್ನಲ್ಲಿ ತಾರಕನಾಥ ಮತ್ತು ಕಾಳೀಪ್ರಸಾದರನ್ನು ಜೊತೆಗೂಡಿಕೊಂಡು ಬುದ್ಧಗಯೆಗೆ ಹೊರಟೇಬಿಟ್ಟ. ಗಂಗಾನದಿಯನ್ನು ದಾಟಿ, ಬಾಲಿ ಎಂಬಲ್ಲಿ ಗಯೆಗೆ ಹೋಗುವ ಟ್ರೈನು ಹತ್ತಿ ದರು. ಮೂವರ ಪ್ರಯಾಣದ ವೆಚ್ಚವನ್ನೂ ತಾರಕನಾಥನೇ ವಹಿಸಿಕೊಂಡಿದ್ದ. ತಾವು ಎಲ್ಲಿಗೆ ಹೋಗುತ್ತಿದ್ದೇವೆಂಬುದನ್ನು ಮಾತ್ರ ಅವರು ಯಾರಿಗೂ ಹೇಳಿರಲಿಲ್ಲ. ಆದರೆ ಆ ಮೂವರೂ ‘ಕಾವಿ ಬಟ್ಟೆ ಧರಿಸಿ, ತಪಸ್ಸು ಮಾಡಲು ಬುದ್ಧಗಯೆಗೆ ಹೋಗಿದ್ದಾರೆ’ ಎಂಬ ಸುದ್ದಿ ಉಳಿದವರಿಗೆ ಹೇಗೋ ತಿಳಿದುಬಂತು. ಆಗ ಅವರೆಲ್ಲ, ಮೂವರೂ ಪರಿವ್ರಾಜಕ ಸಂನ್ಯಾಸಿಗಳಾಗಿ ಹೊರಟು ಬಿಟ್ಟರೇನೋ, ಇನ್ನು ಅವರು ಹಿಂದಿರುಗಿ ಬರುವುದೇ ಇಲ್ಲವೇನೋ ಎಂದು ಭಾವಿಸಿ ಕಳವಳಗೊಂಡರು.

ಇತ್ತ ಈ ಮೂವರು ಗೆಳೆಯರು ರೈಲಿನಲ್ಲಿ ಗಯೆಗೆ ಬಂದು ತಲುಪಿದರು. ಅಲ್ಲಿಂದ ಬುದ್ಧನಿಗೆ ಜ್ಞಾನೋದಯವಾದ ಸ್ಥಳಕ್ಕೆ ಏಳು ಮೈಲಿ ದೂರ; ಅಲ್ಲಿಗೆ ನಡೆದುಕೊಂಡೇ ಹೋದರು. ಆ ಸ್ಥಳವನ್ನು ತಲುಪಿದಾಗ ಅವರಿಗಾದ ಆನಂದ ಅಪಾರ. ತುಂಬ ಪ್ರಶಾಂತವಾದ ಪ್ರದೇಶ; ಪರಮ ಪವಿತ್ರ ವಾತಾವರಣ. ಪ್ರತಿ ನಿಮಿಷವೂ ಬುದ್ಧನ ಪವಿತ್ರ ಸ್ಮರಣೆಯನ್ನು ತಂದುಕೊಡುವ ಆ ದಿವ್ಯಧಾಮದ ಸಾನ್ನಿಧ್ಯದಲ್ಲಿ ಅವರ ಮನಸ್ಸು ಭಾವದುಂಬಿ ಮೌನ ತಾಳಿತು. ಕೆಲವು ದಿನವಾದರೂ ಅಲ್ಲೇ ಇದ್ದು ಆನಂದವನ್ನು ಸವಿಯಲು ಉದ್ದೇಶಿಸಿ, ಅಲ್ಲಿನ ದೇವಸ್ಥಾನದ ಮಹಂತರ ಮನೆಯಲ್ಲಿ ಇಳಿದುಕೊಂಡರು. 

ಒಂದು ದಿನ ಸಂಜೆ; ಮೊದಲೇ ಪ್ರಶಾಂತವಾಗಿದ್ದ ಆ ಸ್ಥಳ ಇನ್ನಷ್ಟು ನೀರವವಾಗಿದೆ. ಮೂವರೂ ಬೋಧಿವೃಕ್ಷದ ಕೆಳಗೆ ಕಲ್ಲಿನ ಆಸನದ ಮೇಲೆ ಕುಳಿತು ಧ್ಯಾನಮಗ್ನರಾಗಿದ್ದಾರೆ. ನರೇಂದ್ರನ ಮನಸ್ಸು ವಿಶೇಷ ಭಾವರಂಜಿತವಾಗಿದೆ. ಬುದ್ಧಭಗವಂತನ ಗುಣ-ಮಹಿಮೆಗಳನ್ನು ಭಾವಿಸುತ್ತ ಧ್ಯಾನಿಸುತ್ತ ಅವನ ಹೃದಯಾಂತರಾಳದಿಂದ ಭಾವದಲೆಗಳು ರಭಸದಿಂದ ಉಕ್ಕಿ ಬಂದುವು. ಆ ಭಾವದುಬ್ಬರವನ್ನು ತಾಳಲಾರದೆ ಅವನು ಅಶ್ರುಧಾರೆಯನ್ನು ಹರಿಸುತ್ತ, ಪಕ್ಕದಲ್ಲೇ ಕುಳಿತಿದ್ದ ತಾರಕನನ್ನು ಬಳಸಿ ತಬ್ಬಿಕೊಂಡುಬಿಟ್ಟ. ಚಕಿತಗೊಂಡ ತಾರಕ, “ನರೇನ್, ಏಕೆ, ಏನಾಯಿತು?” ಎನ್ನುತ್ತ ಅವನನ್ನು ಸಾಂತ್ವನಗೊಳಿಸಲೆತ್ನಿಸಿದ. ಆಗ ನರೇಂದ್ರನೆಂದ: “ನಾನು ಧ್ಯಾನ ಮಾಡುತ್ತಿದ್ದಂತೆಯೇ ನನ್ನ ಕಣ್ಣಮುಂದೆ ಬುದ್ಧನ ಉದಾತ್ತ ವ್ಯಕ್ತಿತ್ವ, ಅವನ ಅದ್ಭುತ ಕರುಣೆ, ಅವನ ಮಾನವೀಯವಾದ ಬೋಧನೆಗಳು, ಅವನಿಂದಾಗಿ ಭಾರತದಲ್ಲಿ ಉಂಟಾದ ಅದ್ಭುತ ಪರಿವರ್ತನೆ–ಇವೆಲ್ಲ ಸ್ಪಷ್ಟವಾಗಿ, ತೇಜೋಮಯವಾಗಿ ಮೂಡಿಬರಲಾರಂಭಿಸಿದುವು. ಆದ್ದರಿಂದ ನನ್ನೊಳಗೆ ಒತ್ತಿಬಂದ ಭಾವಲಹರಿಯನ್ನು ನನ್ನಿಂದ ತಡೆದುಕೊಳ್ಳಲಾಗಲಿಲ್ಲ.”

ಬುದ್ಧನ ವ್ಯಕ್ತಿತ್ವ ನರೇಂದ್ರನ ಮೇಲೆ ಇಂತಹ ಅಗಾಧವಾದ ಪ್ರಭಾವ ಬೀರಿದೆ! ಇದೇಕಿರ ಬಹುದು? ಬಹುಶಃ ಬುದ್ಧ ಅಂದು ಹರಿಸಿದ ಕರುಣಾಪ್ರವಾಹ ಇಂದು ಇವನ ಮೂಲಕ ಮತ್ತೆ ಧರೆಯಲ್ಲಿ ಹರಿದು ಜೀವರೆದೆಗಳನ್ನು ತಂಪುಗೊಳಿಸಬೇಕಾದ್ದಿದೆ ಎಂಬಂತೆ ತೋರುತ್ತದೆ. ನರೇಂದ್ರ ಹರಿಸಿದ ಕಣ್ಣೀರು ಅದರ ಪೂರ್ವ ಸೂಚನೆಯಾಗಿರುವಂತಿದೆ.

ಇತ್ತ ಕಾಶೀಪುರದಲ್ಲಿ ಉಳಿದ ಯುವಶಿಷ್ಯರಿಗೆ ನರೇಂದ್ರನಿಲ್ಲದ ದಿನಗಳು ಅಸಹನೀಯವಾಗಿ ತೋರುತ್ತಿದ್ದುವು. ನರೇಂದ್ರ ಅವರ ಮನಸ್ಸನ್ನು ಎಷ್ಟು ಸೂರೆಗೊಂಡಿದ್ದನೆಂದರೆ ಅವರಿಗೆ ಕ್ಷಣವೊಂದೊಂದೂ ಯುಗವಾಗಿ ತೋರುತ್ತಿತ್ತು. ಅವರಲ್ಲಿ ಕೆಲವರಂತೂ ತಾವೂ ಅವನನ್ನು ಹಿಂಬಾಲಿಸಿ ಬುದ್ಧಗಯೆಗೆ ಹೋಗುವ ವಿಚಾರ ಮಾಡುತ್ತಿದ್ದರು. ಅಲ್ಲದೆ, ಅವನು ಪರಿವ್ರಾಜಕ ನಾಗಿ ಹೊರಟೇಹೋಗಿಬಿಡಬಹುದು ಎಂಬ ಭಯ ಎಲ್ಲರೊಳಗೂ. ಈ ವಿಷಯವೆಲ್ಲ ಶ್ರೀರಾಮ ಕೃಷ್ಣರ ಕಿವಿಗೂ ಬಿತ್ತು. ಹಿಂದೆ, ಕೆಲವು ದಿನಗಳವರೆಗೆ ನರೇಂದ್ರ ಕಾಣಿಸಿಕೊಳ್ಳದಿದ್ದರೆ ಶ್ರೀರಾಮ ಕೃಷ್ಣರು ಹೇಗೆ ಚಡಪಡಿಸುತ್ತಿದ್ದರು ಎಂಬುದನ್ನು ನಾವು ಬಲ್ಲೆವು. ಈಗ, ನರೇಂದ್ರ ಮತ್ತೆ ಬಾರದೆಯೇ ಇದ್ದುಬಿಟ್ಟಾನು ಎಂದು ಶಂಕಿಸುವಂತಹ ಸಂದರ್ಭವೊದಗಿದೆ. ಎಂದಮೇಲೆ, ಅವರೆಷ್ಟು ಕಾತರರಾಗಿರಬೇಕಿತ್ತು! ಆದರೆ ಹಾಗಾಗಲಿಲ್ಲ. ಅವರು ಶಿಷ್ಯರನ್ನೆಲ್ಲ ಕರೆದು ನುಡಿದರು: “ಅವನಿಗೋಸ್ಕರ ನೀವು ಅಷ್ಟೊಂದೇಕೆ ಕಾತರರಾಗಿದ್ದೀರಿ? ಅವನು ಎಲ್ಲಿಗೆ ತಾನೆ ಹೋದಾನು? ಎಷ್ಟು ದಿನ ಅಂತ ನಮ್ಮಿಂದ ದೂರವಿದ್ದಾನು? ನೋಡುತ್ತಿರಿ, ಅವನು ಖಂಡಿತವಾಗಿಯೂ ಬೇಗನೆ ಹಿಂದಿರುಗಿ ಬಂದುಬಿಡುತ್ತಾನೆ.” ಬಳಿಕ ಸ್ವಲ್ಪ ತಡೆದು ಹೇಳಿದರು: “ನೋಡಿ, ನೀವು ಜಗತ್ತಿನ ನಾಲ್ಕು ಮೂಲೆಗಳಿಗೂ ಹೋಗಬಹುದು. ಆದರೆ ನಿಮಗೆ ಎಲ್ಲೂ ಏನೂ ಸಿಗುವುದಿಲ್ಲ. ಅಲ್ಲೆಲ್ಲ ಏನನ್ನು ಪಡೆಯಬಹುದೋ ಅದು ಇಲ್ಲಿಯೇ ಇದೆ.” ‘ಇಲ್ಲಿಯೇ’ ಎಂಬುದನ್ನು ಶ್ರೀರಾಮಕೃಷ್ಣರು ತಮ್ಮ ಕಡೆ ಬೆರಳು ಮಾಡಿ ತೋರಿಸುತ್ತ ಹೇಳಿದರು. ಅದನ್ನು ಎರಡು ರೀತಿಯಲ್ಲಿ ಅರ್ಥೈಸಬಹುದಾಗಿತ್ತು. ಮೊದಲನೆಯದು: ಆಧ್ಯಾತ್ಮಿಕ ಶಕ್ತಿ, ಆಧ್ಯಾತ್ಮಿಕ ಪ್ರಕಾಶ ಗಳು ಆಗ ಶ್ರೀರಾಮಕೃಷ್ಣರಲ್ಲಿ ಅಭಿವ್ಯಕ್ತವಾದಷ್ಟು ಪ್ರಮಾಣದಲ್ಲಿ ಇನ್ನೆಲ್ಲಿಯೂ ಪ್ರಕಟ ವಾಗಿರಲಿಲ್ಲ. ಆದ್ದರಿಂದ ಆಧ್ಯಾತ್ಮಿಕತೆಯನ್ನು ಹುಡುಕಿಕೊಂಡು ಯಾರೂ ಹಲವು ಹದಿನೆಂಟು ಕಡೆ ಹೋಗಬೇಕಾದದ್ದಿಲ್ಲ ಎಂಬರ್ಥದಲ್ಲಿ ‘ಇಲ್ಲಿ’ ಎಂದು ನಿರ್ದೇಶಿಸಿರಬಹುದು. ಎರಡನೆಯ ಅರ್ಥವೇನೆಂದರೆ–ಭಗವಂತ ಪ್ರತಿಯೊಬ್ಬನ ಹೃದಯದಲ್ಲೂ ವಾಸವಾಗಿದ್ದಾನೆ, ಭಕ್ತಿಯೊಂದು ಜಾಗೃತವಾಗಿಬಿಟ್ಟರೆ ಅವನನ್ನು ಇಲ್ಲೇ–ತಮ್ಮತಮ್ಮ ಹೃದಯದಲ್ಲೇ–ಕಾಣಬಹುದು; ಭಕ್ತಿ ಉದಯಿಸದೆಹೋದರೆ ಹೊರಗೆ ಎಲ್ಲೆಲ್ಲಿ ಸುತ್ತಿದರೂ ಅದರಿಂದೇನೂ ಪ್ರಯೋಜನವಿಲ್ಲ –ಎಂದು. ಶಿಷ್ಯರು ಶ್ರೀರಾಮಕೃಷ್ಣರ ಈ ಮಾತನ್ನು ಮೊದಲನೆಯ ರೀತಿಯಲ್ಲಿ ಅರ್ಥಮಾಡಿ ಕೊಂಡರು. ಶ್ರೀರಾಮಕೃಷ್ಣರಲ್ಲಿ ವಿಜೃಂಭಿಸುತ್ತಿರುವಂತೆ ಆಧ್ಯಾತ್ಮಿಕಶಕ್ತಿಯು ಇನ್ನೆಲ್ಲೂ ಪ್ರಕಾಶ ಗೊಂಡಿಲ್ಲ; ಆ ಶಕ್ತಿಯ ಆಕರ್ಷಣೆಯನ್ನು ಮೀರಿ ಹೋಗಲಾರದೆ ನರೇಂದ್ರ ಹಿಂದಿರುಗಿ ಬಂದೇ ಬರುತ್ತಾನೆ ಎಂಬುದು ಅವರಿಗೆ ಮತ್ತಷ್ಟು ದೃಢವಾಯಿತು.

ನಿಜಕ್ಕೂ ನರೇಂದ್ರ ಹಾಗೆ ಹೋದದ್ದರಿಂದ ಶ್ರೀರಾಮಕೃಷ್ಣರಿಗೆ ಒಂದು ರೀತಿಯಲ್ಲಿ ಸಂತೋಷವೇ ಆಗಿತ್ತು. ಏಕೆಂದರೆ, ಈ ಅನುಭವವನ್ನು ಪಡೆದಮೇಲೆ ಅವನು ತಮ್ಮನ್ನು ಇನ್ನಷ್ಟು ಚೆನ್ನಾಗಿ ಅರ್ಥಮಾಡಿಕೊಂಡು, ತನ್ಮೂಲಕ ಇನ್ನಷ್ಟು ಹೆಚ್ಚು ಮೆಚ್ಚಿಕೊಳ್ಳುತ್ತಾನೆ ಎಂದು ಅವರು ಮನಗಂಡಿದ್ದರು.

ಇನ್ನೂ ಒಂದೆರಡು ದಿನ ಕಳೆಯಿತು. ಈ ಮಧ್ಯೆ ಯುವಶಿಷ್ಯರು, ನರೇಂದ್ರ ಇನ್ನೂ ಹಿಂದಿರುಗಿಬಾರದಿದ್ದುದನ್ನು ಕಂಡು, ತಮ್ಮ ಕಾತರತೆಯನ್ನು ಇನ್ನು ತಾಳಲಾರದೆ, “ದಯವಿಟ್ಟು ನರೇಂದ್ರನನ್ನು ಹಿಂದಿರುಗುವ ಹಾಗೆ ಮಾಡಿ” ಎಂದು ಶ್ರೀರಾಮಕೃಷ್ಣರ ಬಳಿ ಮೊರೆಯಿಟ್ಟರು. ಆಗ ಶ್ರೀರಾಮಕೃಷ್ಣರು ನೆಲದ ಮೇಲೆ ವೃತ್ತಾಕಾರವಾಗಿ ಒಂದು ಗೆರೆ ಎಳೆದು, “ಅವನು ಇದರಿಂದಾಚೆಗೆ ಹೋಗಲಾರ” ಎಂದು ಸ್ಪಷ್ಟವಾಗಿ ಹೇಳಿಬಿಟ್ಟರು. ಶ್ರೀರಾಮಕೃಷ್ಣರ ಸಂಕಲ್ಪ ವನ್ನು ಮೀರಿ ನರೇಂದ್ರ ದೂರ ಹೋಗಲಾರ ಎಂದು ಅರ್ಥ ಮಾಡಿಕೊಂಡು ಶಿಷ್ಯರು ಸಮಾಧಾನ ಗೊಂಡರು. ಅದು ನಿಜಕ್ಕೂ ಹಾಗೆಯೇ ಆಯಿತು.

ಇತ್ತ ನರೇಂದ್ರನೂ ಅವನ ಇಬ್ಬರು ಸಂಗಡಿಗರೂ ಗಯೆಯಲ್ಲಿ ಮೂರ್ನಾಲ್ಕು ದಿನಗಳನ್ನು ಕಳೆದಿರಬಹುದು. ಅಷ್ಟರಲ್ಲೇ ಅವರ ಮನಸ್ಸಿಗೆ ಶ್ರೀರಾಮಕೃಷ್ಣರನ್ನು ನೋಡಬೇಕೆಂಬ ಕಾತರ ಹತ್ತಿಕೊಂಡಿತು. ಮೂವರೂ ವಾಪಸು ಹೊರಡುವ ಮನಸ್ಸು ಮಾಡಿದರು. ಆದರೆ ದಾರಿಖರ್ಚಿಗೆ ಯಾರ ಕೈಯಲ್ಲೂ ಹಣವಿರಲಿಲ್ಲ. ಆಗ ಅವರು ಉಳಿದುಕೊಂಡಿದ್ದ ದೇವನಸ್ಥಾನದ ಪಾರುಪತ್ಯ ಗಾರ ಸ್ವಲ್ಪಮಟ್ಟಿನ ಖರ್ಚನ್ನು ವಹಿಸಿಕೊಂಡ. ಅವರು ಅಲ್ಲಿಂದ ಗಯಾಪಟ್ಟಣಕ್ಕೆ ಬಂದರು. ಅಲ್ಲಿ ನರೇಂದ್ರನಿಗೆ ಅವನ ತಂದೆಯ ಹಳೆಯ ಸ್ನೇಹಿತನೊಬ್ಬನೊಂದಿಗೆ ಭೇಟಿಯಾಯಿತು. ಅಂದು ಆತನ ಆತಿಥ್ಯ ಸ್ವೀಕರಿಸಿ, ಮರುದಿನವೇ ಕಲ್ಕತ್ತದ ಕಡೆಗೆ ಹೊರಟರು. ಉಳಿದ ಖರ್ಚನ್ನು ಆ ಸ್ನೇಹಿತನೇ ತುಂಬಿಕೊಟ್ಟ. ಅಂತೂ, ಮೂವರೂ ಕಾಶೀಪುರಕ್ಕೆ ಬಂದು ತಲುಪಿದರು. 

ಶ್ರೀರಾಮಕೃಷ್ಣರಿಗೆ ತಮ್ಮ ಪ್ರೀತಿಯ ನರೇಂದ್ರನನ್ನು ನೋಡಿ ಅತ್ಯಾನಂದವಾಯಿತು. ಅವನು ಬುದ್ಧಗಯೆಯಲ್ಲಿ ಏನೇನು ಕಂಡ, ಏನೇನು ಭಾವಿಸಿದ ಎಂಬುದನ್ನೆಲ್ಲ ವಿವರವಾಗಿ ಕೇಳಿ ತಿಳಿದುಕೊಂಡರು. ಅವನು ಹಿಂದಿರುಗಿ ಬಂದದ್ದರಿಂದ ಇತರ ಯುವಶಿಷ್ಯರಿಗಂತೂ ಹಿಡಿಸ ಲಾರದ ಹಿಗ್ಗು. ಬುದ್ಧಗಯೆಯ ಅನುಭವಗಳು ನರೇಂದ್ರನ ಮೇಲೆ ಎಷ್ಟು ಆಳವಾದ ಪ್ರಭಾವ ವನ್ನುಂಟುಮಾಡಿದ್ದವೆಂದರೆ, ಅನೇಕ ದಿನಗಳವರೆಗೆ ಅವನು ಬುದ್ಧನ ವಿಚಾರವಾಗಿ ಮಾತನಾಡುತ್ತಿದ್ದ.

ನರೇಂದ್ರ ಗಯೆಯಿಂದ ಹಿಂದಿರುಗಿ ಬಂದು ಕೆಲವೇ ದಿನಗಳಾಗಿವೆ. ಅವನೊಂದಿಗೆ ಲಾಟು, ಮಹೇಂದ್ರನಾಥ, ಶಶಿ, ರಾಖಾಲ ಹಾಗೂ ಇತರರು ಶ್ರೀರಾಮಕೃಷ್ಣರ ಸಮ್ಮುಖದಲ್ಲಿ ಕುಳಿತಿದ್ದಾರೆ.

ಶ್ರೀರಾಮಕೃಷ್ಣರು (ಮುಗುಳ್ನಗುತ್ತ) ಮಹೇಂದ್ರನಿಗೆ: “ಇವನು ಅಲ್ಲಿಗೆ (ಬುದ್ಧ ಗಯೆಗೆ) ಹೋಗಿದ್ದ.”

ಮಹೇಂದ್ರ (ನರೇಂದ್ರನಿಗೆ): “ಬುದ್ಧನ ಮೂಲಭೂತ ತತ್ತ್ವಗಳೇನು?”

ನರೇಂದ್ರ: “ನಿಜಕ್ಕೂ, ತಾನು ಆಧ್ಯಾತ್ಮಿಕ ಸಾಧನೆ ಮಾಡಿ ಯಾವ ಸಾಕ್ಷಾತ್ಕಾರ ಮಾಡಿ ಕೊಂಡನೋ, ಅದನ್ನು ಮಾತಿನಿಂದ ವರ್ಣಿಸಲು ಅವನಿಗೆ ಸಾಧ್ಯವಾಗಲಿಲ್ಲ. ಆದ್ದರಿಂದ ಜನ ಅವನನ್ನು ನಾಸ್ತಿಕ ಎಂದು ಪರಿಗಣಿಸಿದರು.”

ಶ್ರೀರಾಮಕೃಷ್ಣರು ಈ ಮಾತನ್ನು ಅನುಮೋದಿಸುತ್ತ ನುಡಿದರು: “ಅವನೇಕೆ ನಾಸ್ತಿಕನಾದಾನು? ಅವನು ನಾಸ್ತಿಕನಲ್ಲ. ಅವನಿಗೆ ತನ್ನ ಅನುಭವಗಳನ್ನು ಶಬ್ದಗಳ ಮೂಲಕ ವ್ಯಕ್ತಪಡಿಸುವುದ ಕ್ಕಾಗಲಿಲ್ಲ, ಅಷ್ಟೆ, ‘ಬುದ್ಧ’ ಎಂಬುದರ ಅರ್ಥ ಗೊತ್ತೇನು ನಿನಗೆ? ಯಾರಿಗೆ ಶುದ್ಧಚೈತನ್ಯದ ಬೋಧೆಯಾಗಿದೆಯೋ ಅವನೇ ಬುದ್ಧ. ಅಂತಹ ವ್ಯಕ್ತಿ ತಾನೂ ಶುದ್ಧಚೈತನ್ಯವೇ ಆಗಿಬಿಡುತ್ತಾನೆ.”

ಹೀಗೇ ಸ್ವಲ್ಪಹೊತ್ತು ಮಾತುಕತೆ ನಡೆಯಿತು. ಬಳಿಕ ಶ್ರೀರಾಮಕೃಷ್ಣರು ನರೇಂದ್ರನನ್ನು ದ್ದೇಶಿಸಿ ಕೇಳಿದರು: “ಬುದ್ಧ ಏನೇನು ಬೋಧಿಸಿದ?”

ನರೇಂದ್ರ ಉತ್ಸಾಹದಿಂದ ಹೇಳತೊಡಗಿದ: “ಅವನು ಭಗವಂತನ ಅಸ್ತಿತ್ವ-ನಾಸ್ತಿತ್ವದ ವಿಷಯವಾಗಿ ಏನೂ ಮಾತನಾಡಲೇ ಇಲ್ಲ. ಅವನು ತನ್ನ ಜೀವನವಿಡೀ ಸುಮ್ಮನೆ ಎಲ್ಲರಿಗೂ ಕರುಣೆ ತೋರುತ್ತ ಬಂದ. ಅವನದು ಎಂಥ ಅದ್ಭುತವಾದ ತ್ಯಾಗ! ರಾಜಕುಮಾರನಾಗಿ ಹುಟ್ಟಿಯೂ ಎಲ್ಲವನ್ನೂ ತ್ಯಾಗ ಮಾಡಿಬಿಟ್ಟ. ಕೈಯಲ್ಲಿ ಕಾಸಿಲ್ಲದವನು ತ್ಯಾಗ ಮಾಡಿದ್ದೇನೆಂದರೆ ಅದಕ್ಕೇನರ್ಥ?

“ಆತ್ಮಸಾಕ್ಷಾತ್ಕಾರ ಮಾಡಿಕೊಂಡಮೇಲೆ ಒಮ್ಮೆ ಅವನು ತನ್ನ ಹಿಂದಿನ ಅರಮನೆಗೆ ಬಂದಿದ್ದ. ಆಗ ಅವನು ತನ್ನ ಹೆಂಡತಿ, ಮಗ ಹಾಗೂ ರಾಜಮನೆತನದ ಇನ್ನೂ ಅನೇಕರನ್ನು ಕೂಡ ತ್ಯಾಗಜೀವನವನ್ನು ಸ್ವೀಕರಿಸುವಂತೆ ಪ್ರೋತ್ಸಾಹಿಸುತ್ತಾನೆ! ಅವನ ತ್ಯಾಗದಲ್ಲಿ ಎಂತಹ ತೀವ್ರತೆ ಯಿತ್ತು! ಆದರೆ ವ್ಯಾಸನ ಬುದ್ಧಿಯನ್ನು ನೋಡಿ. ತ್ಯಾಗ ಮಾಡಲು ಹೊರಟ ಮಗನನ್ನು ತಡೆದು ಹೇಳುತ್ತಾನೆ–‘ಬೇಡಪ್ಪ, ಗೃಹಸ್ಥನಾಗಿದ್ದುಕೊಂಡೇ ಸಾಧನೆ ಮಾಡು’ಅಂತ.”

ಸಂದರ್ಭವೊದಗಿದಾಗಲೆಲ್ಲ ನರೇಂದ್ರ ಬುದ್ಧನ ಗುಣ-ವ್ಯಕ್ತಿತ್ವ-ತತ್ತ್ವಗಳನ್ನು ಹೀಗೆಯೇ ಮುಕ್ತಕಂಠದಿಂದ ವರ್ಣಿಸುತ್ತಿದ್ದ.

ತಮ್ಮ ಯುವಶಿಷ್ಯರ ವೈರಾಗ್ಯವನ್ನು ಜ್ವಲಂತವಾಗಿರಿಸಲು ಮತ್ತು ತಮ್ಮ ನಿರ್ಯಾಣಾನಂತರ ಅವರೆಲ್ಲ ಸಂನ್ಯಾಸ ಸ್ವೀಕರಿಸುವಂತಾಗಲು ಶ್ರೀರಾಮಕೃಷ್ಣರು ನಾನಾ ರೀತಿಯಲ್ಲಿ ಯತ್ನಿಸು ತ್ತಿದ್ದರು. ಈ ಯುವಕರಿಗೆ ಅವರು ತಾವಾಗಿಯೇ ಸಾಂಕೇತಿಕವಾಗಿ ಸಂನ್ಯಾಸ ಕೊಡಲು ಒಂದು ಸಂದರ್ಭವೊದಗಿತು.

ಬಂಗಾಳದಲ್ಲಿ ಗಂಗೆ ಸಾಗರವನ್ನು ಸೇರುವ ಒಂದು ಸ್ಥಳವಿದೆ. ಇದೊಂದು ತೀರ್ಥಕ್ಷೇತ್ರ. ಇಲ್ಲಿ ವಾರ್ಷಿಕೋತ್ಸವ ನಡೆಯುತ್ತದೆ. ಈ ಉತ್ಸವಕ್ಕೆ ಹೋಗುವ ಅನೇಕ ಸಾಧು-ಸಂನ್ಯಾಸಿಗಳು ಕಲ್ಕತ್ತದ ಮೂಲಕ ಹೋಗುತ್ತಾರೆ. ಹಿರಿಯ ಗೋಪಾಲ ಈ ಸಾಧುಗಳಿಗೆ ಕಾಷಾಯವಸ್ತ್ರವನ್ನು ಹಾಗೂ ರುದ್ರಾಕ್ಷಿಮಾಲೆಯನ್ನು ಕೊಡಬೇಕೆಂಬ ತನ್ನ ಇಚ್ಛೆಯನ್ನು ವ್ಯಕ್ತಪಡಿಸಿದ. ಶ್ರೀರಾಮ ಕೃಷ್ಣರಿಗೆ ಈ ವಿಷಯ ತಿಳಿದಾಗ ಅವನನ್ನು ಕರೆದು ನರೇಂದ್ರಾದಿ ಯುವಶಿಷ್ಯರನ್ನು ತೋರಿಸುತ್ತ ಹೇಳಿದರು: “ಇಲ್ಲಿದ್ದಾರೆ ನೋಡು, ತ್ಯಾಗಭರಿತರಾದ ಯುವಕರು. ಇವರಿಗಿಂತ ಪರಿಶುದ್ಧ ವ್ಯಕ್ತಿಗಳನ್ನು ಹುಡುಕಿಕೊಂಡು ಎಲ್ಲಿಗೆ ಹೋಗುತ್ತಿ? ಆ ವಸ್ತುಗಳನ್ನು ಇವರಿಗೇ ಕೊಡು. ನಿನಗೆ ಶ್ರೇಯಸ್ಸುಂಟಾಗುತ್ತದೆ.” ಅದಕ್ಕೊಪ್ಪಿ ಆತ ಹನ್ನೆರಡು ಜೊತೆ ಕಾವಿಬಟ್ಟೆ ಹಾಗೂ ರುದ್ರಾಕ್ಷಿ ಮಾಲೆಗಳನ್ನು ತಂದು ಶ್ರೀರಾಮಕೃಷ್ಣರ ಕೈಯಲ್ಲಿಟ್ಟ. ಅವರು ಅದನ್ನು ಆಗ ಅಲ್ಲಿದ್ದ ನರೇಂದ್ರ, ರಾಖಾಲ, ಬಾಬುರಾಮ, ನಿರಂಜನ, ಯೋಗೀಂದ್ರ, ತಾರಕನಾಥ, ಶಶಿಭೂಷಣ, ಶರಚ್ಚಂದ್ರ, ಕಾಳೀಪ್ರಸಾದ, ಲಾಟು ಹಾಗೂ ಹಿರಿಯ ಗೋಪಾಲ (ಶಿಷ್ಯರ ಪೈಕಿ ಇವರೊಬ್ಬರೇ ವಯಸ್ಕರು ) ಇವರಿಗೆ ತಮ್ಮ ಕೈಯಾರೆ ಹಂಚಿದರು. ಇನ್ನೊಂದು ಜೊತೆಯನ್ನು ಗಿರೀಶನಿಗಾಗಿ ಎತ್ತಿರಿಸಿದರು. ನಿಜಕ್ಕೂ, ಈ ಮೂಲಕ ಭವಿಷ್ಯದ ಶ್ರೀರಾಮಕೃಷ್ಣ ಮಹಾಸಂಘದ ಸಂಸ್ಥಾಪನೆಯೇ ಆಯಿತು ಎನ್ನಬಹುದು.

ಈ ದಿನಗಳಲ್ಲೇ ನರೇಂದ್ರನ ಆಧ್ಯಾತ್ಮಿಕ ಸಾಧನೆಯ ತೀವ್ರತೆ ಹೆಚ್ಚುತ್ತ ಪರಾಕಾಷ್ಠೆಯನ್ನು ಮುಟ್ಟುತ್ತಿತ್ತು. ಈಗ ಒಂದು ಬಯಕೆ ಅವನೊಳಗೆ ಭೂತದಂತೆ ಹೊಕ್ಕುಬಿಟ್ಟಿದೆ; ಹಗಲಿರುಳೂ ಕಾಳ್ಗಿಚ್ಚಿನಂತೆ ಉರಿಯುತ್ತಿದೆ. ಏನದು? ಅದ್ವೈತ ಸಾಧನೆಯ ಅತ್ಯುನ್ನತ ಸ್ಥಿತಿಯಾದ ನಿರ್ವಿಕಲ್ಪ ಸಮಾಧಿಯ ಆನಂದವನ್ನು ಅನುಭವಿಸಬೇಕೆಂಬ ಹಂಬಲ! ಈ ಜಗತ್ತು, ತನ್ನ ದೇಹ, ಮನಸ್ಸು, ಬುದ್ಧಿಗಳೆಲ್ಲವನ್ನೂ ಮರೆತು, ಆತ್ಮದಲ್ಲಿ ಲೀನನಾಗಿ ಸುಖಿಸುತ್ತಿರಬೇಕೆಂಬ ಆಸೆ! ಮತ್ತು ಚಿರ ಮುಕ್ತನಾಗಿಬಿಡಬೇಕು ಎನ್ನುವ ಕಾತರ. ತನಗೆ ಅದನ್ನು ಶ್ರೀರಾಮಕೃಷ್ಣರು ಸಂಕಲ್ಪಮಾತ್ರದಿಂದ ಕೊಡಿಸಬಲ್ಲರು ಮತ್ತು ಆ ಶಕ್ತಿ ಇರುವುದು ಅವರೊಬ್ಬರಿಗೇ ಎಂಬ ಅರಿವು ಅವನಿಗೆ ಚೆನ್ನಾ ಗಿತ್ತು. ಅವರ ಮುಂದೆ ಎಷ್ಟೋ ಸಲ ತನ್ನ ಬಯಕೆಯನ್ನು ತೋಡಿಕೊಂಡಿದ್ದ. ಆಗೆಲ್ಲ ಅವರು ನಾನಾ ಬಗೆಯಿಂದ ಸಮಾಧಾನ ನೀಡಿ ಸುಮ್ಮನಾಗಿಸಿದ್ದರು. ಆದರೆ ಇನ್ನು ಅವನಿಗೆ ತಡೆಯಲು ಸಾಧ್ಯವಿಲ್ಲವಾಯಿತು. ಒಂದು ದಿನ ಇದ್ದಕ್ಕಿದ್ದಂತೆ ಶ್ರೀರಾಮಕೃಷ್ಣರ ಕೋಣೆಯೊಳಕ್ಕೆ ನುಗ್ಗಿ, ಅವರೆದುರು ನಿಂತು ನುಡಿದ:

“ಸ್ವಾಮಿ, ನೀವು ನನಗೆ ನಿರ್ವಿಕಲ್ಪ ಸಮಾಧಿಯನ್ನು ಕೊಡಿಸಲೇಬೇಕು.”

ಶ್ರೀರಾಮಕೃಷ್ಣರು ಒಂದು ಕ್ಷಣ ವಿಚಲಿತರಾದರೂ ಶಾಂತವಾಗಿಯೇ ನುಡಿದರು:

“ಆಗಲಿ, ನಾನು ಗುಣಹೊಂದಿದ ಮೇಲೆ ನೀನೇನು ಕೇಳುತ್ತೀಯೋ ಅದನ್ನು ಕೊಡಿಸುತ್ತೇನೆ.”

ಆದರೆ ಅವನು ಹಟ ಬಿಡದೆ ಮತ್ತೆಮತ್ತೆ ಕಾಡಿದ. ನಿರಂತರವಾಗಿ ಸಮಾಧಿಯಲ್ಲಿ ಮುಳು ಗಿದ್ದು, ಎಲ್ಲೋ ಆಗೊಮ್ಮೆ ಈಗೊಮ್ಮೆ ಮಾತ್ರ, ದೇಹರಕ್ಷಣೆಗಾಗಿ ಪ್ರಕೃತಿಸ್ಥನಾಗಬೇಕು ಎಂಬ ತನ್ನ ಬಯಕೆಯನ್ನು ಪುನರುಚ್ಚರಿಸಿದ. ಆಗ ಶ್ರೀರಾಮಕೃಷ್ಣರು ಅಸಹನೆಗೊಂಡು ಅವನಿಗೆ ಛೀಮಾರಿ ಹಾಕಿದರು:

“ಛಿ! ನಾಚಿಗೆಗೇಡು! ನೀನೊಂದು ಮಹಾ ಆಶ್ರಯಸ್ಥಾನವಾಗಿ ಇರಬೇಕಾದವನು. ನಿನ್ನ ಯೋಗ್ಯತೆಗೆ ಇಂಥ ಮಾತು ಒಪ್ಪುತ್ತದೆಯೆ? ನಾನೆಣಿಸಿದ್ದೆ–ನೀನೊಂದು ದೊಡ್ಡ ಆಲದ ಮರದ ಹಾಗಾಗುತ್ತೀ ಮತ್ತು ಅಸಂಖ್ಯಾತ ದೀನಾರ್ತರಿಗೆ ನಿನ್ನ ನೆರಳಿನಲ್ಲಿ ಆಶ್ರಯ ಕೊಡುತ್ತೀ ಅಂತ. ಈಗ ನೋಡಿದರೆ ನೀನು ನಿನ್ನ ಸ್ವಂತ ಸುಖವನ್ನು, ಸ್ವಂತ ಮುಕ್ತಿಯನ್ನು ಬಯಸುತ್ತಿದ್ದೀ ಯಲ್ಲ!... ಮಗೂ, ಈ ಮುಕ್ತಿ ಎನ್ನುವುದು ಅತ್ಯಂತ ಕ್ಷುದ್ರವಾದುದು. ಬೇಡ, ಇಂತಹ ಸಂಕುಚಿತ ದೃಷ್ಟಿಯಿಟ್ಟುಕೊಳ್ಳಬೇಡ. ಭಗವಂತನನ್ನು ಎಲ್ಲೆಡೆಗಳಲ್ಲೂ ಎಲ್ಲ ಬಗೆಗಳಿಂದಲೂ ನೋಡುವುದೇ ನನ್ನ ವೈಶಿಷ್ಟ್ಯ. ಯಾವ ಭಗವಂತನನ್ನು ನಾನು ಸಮಾಧಿಸ್ಥಿತಿಯಲ್ಲಿ ಪರಬ್ರಹ್ಮ ನಂತೆ ಕಾಣುತ್ತೇನೋ ಅದೇ ಭಗವಂತನೇ ವಿವಿಧ ಮಾನವರೂಪಗಳಲ್ಲಿ ವಿರಾಜಿಸುತ್ತಿರುವುದನ್ನು ಜಾಗೃತಸ್ಥಿತಿಯಲ್ಲೂ ಕಂಡು ಆನಂದಿಸುತ್ತೇನೆ. ಇಂತಹ ಅತ್ಯುನ್ನತ ಅನುಭವವನ್ನು ನಿನಗೆ ಮಾಡಿಸಿ ಕೊಡಲು ಸಿದ್ಧನಿರುವಾಗ, ನೀನು ಅಂತಹ ಕ್ಷುದ್ರ ವಸ್ತುವಿಗಾಗಿ ಹಾತೊರೆಯುತ್ತಿದ್ದೀಯಲ್ಲ!”

ಏನೇ ಹೇಳಿದರೂ ನರೇಂದ್ರನಿಗೆ ಪೂರ್ತಿ ಒಪ್ಪಿಗೆಯಾಗಲಿಲ್ಲ.

ಇದಾಗಿ ಕೆಲದಿನಗಳಾಗಿರಬಹುದು. ಒಂದು ಅಪರಾಹ್ನ ಅವನು ಎಂದಿನಂತೆ ಮಲಗಿರುವ ಸ್ಥಿತಿಯಲ್ಲೇ ಧ್ಯಾನಮಗ್ನನಾಗಿದ್ದ. ಆಗ ಅವನಿಗೆ ಅತ್ಯಂತ ಅನಿರೀಕ್ಷಿತವಾಗಿ ಒಂದು ಅನಿರ್ ವಚನೀಯ ಅನುಭವವಾಯಿತು. ಮೊದಲು ಅವನು ತನ್ನ ತಲೆಯ ಹಿಂಭಾಗದಲ್ಲಿ ಉಜ್ವಲ ಜ್ಯೋತಿಯೊಂದನ್ನು ಕಂಡ. ಒಡನೆಯೇ ಅವನನ್ನು ಭೌತಿಕ ಪ್ರಪಂಚಕ್ಕೆ ಬಂಧಿಸಿದ್ದ ಕಟ್ಟುಗಳೆಲ್ಲ ಹರಿದುಹೋದಂತಾಗಿ, ಅವನ ಮನಸ್ಸು ಅತ್ಯುನ್ನತ ಸ್ಥಿತಿಗೆ ಏರಿಹೋಯಿತು. ಆ ಅನುಭವವನ್ನು ಅವನಿಂದಲೇ ಕೇಳಿ ತಿಳಿದ ಗೆಳೆಯ ಶರಚ್ಚಂದ್ರ ಮುಂದೆ ಅದನ್ನು ಬಣ್ಣಿಸುತ್ತಾನೆ:

“ನರೇಂದ್ರನ ಆತ್ಮ ಪ್ರಚಂಡ ಭಾವಾವೇಶದ ಬಿರುಗಾಳಿಗೆ ಸಿಲುಕಿಕೊಂಡಿತು. ಅಲ್ಲಿಂದ ಅದು ಒಂದು ವಿಧವಾದ ದಿಗ್ಭ್ರಮೆಯ ದಿವ್ಯಾವಸ್ಥೆಯನ್ನು ಮುಟ್ಟಿತು. ಆ ಅವಸ್ಥೆಯಲ್ಲಿ ಸಕಲ ನಾಮರೂಪಗಳೂ ಭಾವನೆಗಳೂ ತನ್ನ ಮನಸ್ಸಿನಿಂದ ಕಳಚಿಬೀಳುತ್ತಿರುವಂತೆ ಅವನಿಗೆ ಭಾಸ ವಾಯಿತು. ಬಳಿಕ ಅವನ ಆತ್ಮ ಆ ಸ್ಥಿತಿಯಿಂದಲೂ ಮೇಲೇರಿ ಅತ್ಯಾನಂದದ ಹಂತವೊಂದನ್ನು ತಲುಪಿತು. ಆಗ ಅವನಿಗೆ ತನ್ನ ಸಂಪೂರ್ಣ ವ್ಯಕ್ತಿತ್ವವೇ ದಿವ್ಯತೆಯ ದೀಪ್ತಿಯಿಂದ ತುಂಬಿ ಹೋದಂತಹ ಅನುಭವವಾಯಿತು. ಕಡೆಗೆ ಆ ವ್ಯಕ್ತಿತ್ವವೂ ಕರಗಿಹೋಗಿ ದಿವ್ಯತೆಯೊಂದೇ ಉಳಿಯಿತು. ಈ ಎಲ್ಲ ಅನುಭವಗಳ ಪರಾಕಾಷ್ಠೆಯೆಂಬಂತೆ ಅವನ ಆತ್ಮ ಒಂದು ಮಹಾ ಜ್ಯೋತಿಯ ರೂಪದಿಂದ ಆ ಉತ್ಕಟ ಆನಂದದ ಸ್ಥಿತಿಯನ್ನೂ ಅತಿಕ್ರಮಿಸಿ ಅನಂತ ಶಾಂತಿಯ ನೀರವತೆಯನ್ನು ಪ್ರವೇಶಿಸಿತು. ”

ಶ್ರೀರಾಮಕೃಷ್ಣರ ಇಚ್ಛಾಮಾತ್ರದಿಂದ ನರೇಂದ್ರ ಇಂತಹ ಅತ್ಯಪೂರ್ವ ನಿರ್ವಿಕಲ್ಪ ಸಮಾಧಿಯ ಅನುಭವವನ್ನು ಹೊಂದಿ ಬೇರೆಯೇ ಲೋಕದಲ್ಲಿ ವಿಹರಿಸುತ್ತಿದ್ದಾನೆ. ಆದರೆ ಇತರರಿಗೆ ಮಾತ್ರ, ಅವನು ಅಂಗಾತ ಮಲಗಿದ್ದಾನೆ ಎನ್ನುವುದಲ್ಲದೆ ಬೇರೇನೂ ತಿಳಿಯುವಂತಿಲ್ಲ. ಆ ಕೋಣೆಯಲ್ಲಿ ಹಿರಿಯ ಗೋಪಾಲನೊಬ್ಬನೇ ಧ್ಯಾನನಿರತನಾಗಿ ಕುಳಿತಿದ್ದಾನೆ. ಉಳಿದವರು ಅಲ್ಲಿ-ಇಲ್ಲಿ ಬೇರೆಬೇರೆ ಕೆಲಸಕಾರ್ಯಗಳಲ್ಲಿ ತೊಡಗಿದ್ದಾರೆ. ಬಹಳ ಹೊತ್ತಿನ ಮೇಲೆ ಈಗ ಶ್ರೀರಾಮಕೃಷ್ಣರ ಸಂಕಲ್ಪದಿಂದ ಮತ್ತೆ ನರೇಂದ್ರನ ಪ್ರಜ್ಞೆ ಕೆಳಸ್ತರಕ್ಕೆ ಇಳಿದು ಬರಲಾರಂಭಿಸಿತು. ಸ್ವಲ್ಪ ಎಚ್ಚರಗೊಂಡ ನರೇಂದ್ರ ಮೆಲ್ಲನೆ ಕಣ್ತೆರೆದು ನೋಡಿದ; ಇಂದ್ರಿಯ ಪ್ರಪಂಚಕ್ಕೆ ಹೊಂದಿ ಕೊಳ್ಳುವ ಪ್ರಯತ್ನ ಮಾಡಿದ. ಆದರೆ ಅವನಿಗೆ ತನ್ನ ಮುಖದ ಅಸ್ತಿತ್ವ ಮಾತ್ರ ಗೊತ್ತಾಗುತ್ತಿದೆ; ಇತರ ಅವಯವಗಳ ಮೇಲಿನ ಪ್ರಜ್ಞೆ ಮರಳಿಲ್ಲ. ಎಂಥ ಭಯಂಕರ ಸ್ಥಿತಿ! ಸ್ವಲ್ಪ ದೂರದಲ್ಲಿ ಕುಳಿತಿದ್ದ ಹಿರಿಯ ಗೋಪಾಲನನ್ನು ಕಂಡು ಅವನು ಜೋರಾಗಿ ಕೂಗಿಕೊಳ್ಳುವ ಪ್ರಯತ್ನ ಮಾಡಿದ, “ಓ ಗೋಪಾಲ್ ದಾ! ನನ್ನ ದೇಹ ಎಲ್ಲಿ? ನನ್ನ ದೇಹ ಎಲ್ಲಿ?” ಎಂದು. ಆದರೆ ಆ ಸ್ವರ ಗುಹೆಯೊಳಗಿಂದ ಬಂದಂತೆ ಕ್ಷೀಣವಾಗಿತ್ತು. ಗೋಪಾಲನಿಗೆ ಆಶ್ಚರ್ಯವಾಯಿತು. ಆತ ಕೂಡಲೇ ಎದ್ದು ಬಂದು, “ನರೇನ್, ಇಲ್ಲೇ ಇದೆಯಲ್ಲ!” ಎಂದು ಗಟ್ಟಿಯಾಗಿ ಹೇಳುತ್ತ, ಅವನ ಮೈಮುಟ್ಟಿ ನೋಡುತ್ತಾನೆ–ಅದು ಮರದ ಕೊರಡಿನಂತಾಗಿಬಿಟ್ಟಿದೆ! ಒಂದೆರಡು ಬಾರಿ ಮಾತನಾಡಿದ ನರೇಂದ್ರ ಮತ್ತೆ ಪ್ರಜ್ಞೆ ಕಳೆದುಕೊಂಡುಬಿಟ್ಟ. ಗೋಪಾಲ ಕಂಗಾಲದ. “ಬೇಗ ಬನ್ನಿ, ಬೇಗ ಬನ್ನಿ! ನರೇಂದ್ರನಿಗೆ ಏನೋ ಆಗಿಬಿಟ್ಟಿದೆ!” ಎಂದು ಕೂಗಿಕೊಂಡ. ಗುರುಭಾಯಿ ಗಳೆಲ್ಲ ಓಡೋಡಿ ಬಂದರು. ನರೆಂದ್ರ ನಿಶ್ಚೇಷ್ಟಿತನಾಗಿ ಬಿದ್ದಿದ್ದಾನೆ; ಉಸಿರಾಟವಿಲ್ಲ, ಅಂಗಾಂಗ ಗಳ ಮೇಲೆ ನಿಯಂತ್ರಣವಿಲ್ಲ! ಎಲ್ಲೋ ಕುಟುಕು ಜೀವ ಮಾತ್ರ ಇದ್ದಂತಿದೆ. ಆ ಅವಸ್ಥೆಯನ್ನು ನೋಡಿ ಎಲ್ಲರೂ ಭಯವಿಹ್ವಲರಾದರು. ಕೆಲವರು ಅವನ ಕೈಕಾಲುಗಳನ್ನು ತಿಕ್ಕಿದರು; ಕೃತಕ ಉಸಿರಾಟದ ಪ್ರಯೋಗದ ಮೂಲಕ ಶರೀರವನ್ನು ಚೇತನಗೊಳಿಸುವ ಪ್ರಯತ್ನ ಮಾಡಿದರು. ಏನೂ ಪ್ರಯೋಜನವಾಗಲಿಲ್ಲ. ಆತಂಕ ಮಿತಿಮೀರಿತು. “ನನಗೆ ಯಾಕೋ ಭಯವಾಗುತ್ತಿದೆ...” ಎಂದೊಬ್ಬ ಪಿಸುಗುಟ್ಟಿದ. ಇನ್ನೊಬ್ಬನೆಂದ, “ಬನ್ನಿ, ಠಾಕೂರರಿಗೆ ಹೇಳೋಣ!” ತಕ್ಷಣವೇ ಒಂದಿಬ್ಬರನ್ನು ಬಿಟ್ಟು ಉಳಿದವರೆಲ್ಲ ಮೇಲಕ್ಕೋಡಿದರು.

ಅಲ್ಲಿ ಹೋಗಿ ನೋಡಿದರೆ, ಶ್ರೀರಾಮಕೃಷ್ಣರು ತಮಗೆ ಎಲ್ಲ ವಿಷಯವೂ ತಿಳಿದೇ ಇದೆ ಎನ್ನುವಂತೆ ಗಂಭೀರ ಮುಖಮುದ್ರೆ ಧರಿಸಿ ಕುಳಿತಿದ್ದಾರೆ! ಯುವಶಿಷ್ಯರು ಕಳವಳದಿಂದ ‘ನರೇಂದ್ರನಿಗೆ ಏನೋ ಆಗಿಬಿಟ್ಟಿರುವುದನ್ನು’ ಬಣ್ಣಿಸಿದಾಗ ಅವರು, “ಇರಲಿ, ಇರಲಿ, ಸ್ವಲ್ಪ ಹೊತ್ತು ಹಾಗೇ ಇರಲಿ, ಅದಕ್ಕಾಗಿ ಅವನು ನನಗೆ ಸಾಕಷ್ಟು ತೊಂದರೆ ಕೊಟ್ಟಿದ್ದಾನೆ” ಎಂದು ಶಾಂತವಾಗಿ ನುಡಿದರು.

ಈ ಮಾತನ್ನು ಕೇಳಿದ ಮೇಲೆ ಶಿಷ್ಯರಿಗೆ ಎಷ್ಟೋ ಸಮಾಧಾನವಾಯಿತು. ನರೇಂದ್ರನಿದ್ದ ಕೋಣೆಗೆ ಮರಳಿ ಬಂದು, ಅವನನ್ನೇ ಗಮನಿಸತೊಡಗಿದರು. ಸುಮಾರು ಒಂದೆರಡು ಗಂಟೆಗಳ ಬಳಿಕ ಅವನಿಗೆ ಮೆಲ್ಲನೆ ಪ್ರಜ್ಞೆ ಮರಳತೊಡಗಿತು. ಕಡೆಗೆ ಯಾವುದೋ ಒಂದು ಶಕ್ತಿ ತನ್ನನ್ನು ಬಲವಾಗಿ ಹೊರದೂಡಿದಂತಾಗಿ ಇದ್ದಕ್ಕಿದ್ದಂತೆ ಎಚ್ಚರಗೊಂಡ. ಆದರೆ ಇನ್ನೂ ಮಬ್ಬು ಕವಿದಂತಿತ್ತು. ಬಳಿಕ ನಿಧಾನವಾಗಿ ತಾನು ಸವಿದ ದಿವ್ಯಾನಂದದ ಮಧುರ ಸ್ಮೃತಿ ಮರುಕಳಿಸ ಲಾರಂಭಿಸಿತು. ಒಳಗೆ ತುಂಬಿಕೊಂಡಿದ್ದ ಪ್ರಚಂಡ ಭಾವೋದ್ವೇಗ ಸ್ಫೋಟಗೊಂಡಂತಾಗಿ, ಧನ್ಯತೆಯ ಕಣ್ಣೀರು ಸುರಿಸುತ್ತ “ಆಹಾ! ಏನಾಗಿಹೋಯಿತು!” ಎಂದು ಉದ್ಗರಿಸಿದ. 

ಈಗ ಅವನು ಮೇಲೆದ್ದ; ತನಗರಿವಿಲ್ಲದಂತೆಯೇ ಮಹಡಿಯ ಮೇಲಕ್ಕೆ ಹೋಗಿ, ಮೌನವಾಗಿ ಶ್ರೀರಾಮಕೃಷ್ಣರ ಮುಂದೆ ನಿಂತ. ಅವನ ಕಂಗಳಲ್ಲಿ ತಮ್ಮ ದೃಷ್ಟಿಯನ್ನು ನೆಟ್ಟು ಶ್ರೀರಾಮಕೃಷ್ಣರು ಗಂಭೀರವಾಗಿ ನುಡಿದರು:

“ಜಗನ್ಮಾತೆ ನಿನಗೆ ಎಲ್ಲವನ್ನೂ ತೋರಿಸಿದಳಷ್ಟೆ? ಆದರೆ ನೋಡು, ಅಮೂಲ್ಯ ವಸ್ತು ವೊಂದನ್ನು ಸಂದೂಕದಲ್ಲಿಟ್ಟು ಭದ್ರಪಡಿಸಿದಂತೆ, ನಿನ್ನ ಈ ಅನುಭವಕ್ಕೆ ಬೀಗಮುದ್ರೆ ಹಾಕಿಡ ಲಾಗುತ್ತದೆ, ಮತ್ತು ಅದರ ಕೀಲಿಕೈ ನನ್ನ ಬಳಿಯಿರುತ್ತದೆ. ನೀನು ಸಾಧಿಸಬೇಕಾದ ಕೆಲಸಗಳು ಇನ್ನೂ ಬಹಳಷ್ಟಿವೆ. ಅವುಗಳನ್ನೆಲ್ಲ ನೀನು ಮಾಡಿ ಮುಗಿಸಿದ ಮೇಲೆ ಮತ್ತೆ ಆ ಬೀಗವನ್ನು ತೆರೆಯಲಾಗುತ್ತದೆ. ಆಗ ಆ ನಿಧಿ ಮತ್ತೆ ನಿನ್ನದಾಗುತ್ತದೆ. ಎಲ್ಲವೂ ಪುನಃ ನಿನಗೆ ಈಗ ಕಂಡಂತೆಯೇ ಅನುಭವಕ್ಕೆ ಬರುತ್ತದೆ.”

ಆದರೆ ತಾನು ಒಮ್ಮೆ ಸವಿದ ಆನಂದವನ್ನು ಮೆಲಕು ಹಾಕುತ್ತ ನರೇಂದ್ರ ಗೋಗರೆದ: “ನಾನು ಸಮಾಧಿಯಲ್ಲಿ ತುಂಬ ಆನಂದದಿಂದಿದ್ದೆ. ನನ್ನನ್ನು ಅದೇ ಸ್ಥಿತಿಯಲ್ಲಿರಿಸುವ ಕೃಪೆ ಮಾಡಬೇಕು.”

ಅವನ ಸ್ವಾರ್ಥಬುದ್ಧಿಯನ್ನು ಶ್ರೀರಾಮಕೃಷ್ಣರು ಪುನಃ ಖಂಡಿಸಿದರು. ಆದರೆ ಬಳಿಕ ಅತಿ ಮಹತ್ವಪೂರ್ಣವಾದ ಈ ಮಾತುಗಳನ್ನು ನುಡಿದರು:

“ಜಗನ್ಮಾತೆಯ ಕೃಪೆಯಿಂದ, ಮುಂದೆ ನಿನಗೆ ಈ ಅನುಭವ ಎಷ್ಟು ಸಹಜವಾಗಿಬಿಡುತ್ತದೆ ಯೆಂದರೆ, ಜಾಗೃದವಸ್ಥೆಯಲ್ಲೇ ನೀನು ಸಕಲ ಜೀವಿಗಳಲ್ಲೂ ಆ ದಿವ್ಯತೆಯನ್ನು ಕಾಣಲು ಸಮರ್ಥನಾಗುವೆ. ಜಗತ್ತಿನಲ್ಲಿ ನೀನು ಮಹಾಕಾರ್ಯಗಳನ್ನು ಸಾಧಿಸುತ್ತೀಯೆ. ಜನಕೋಟಿಯಲ್ಲಿ ಆಧ್ಯಾತ್ಮಿಕ ಜಾಗೃತಿಯನ್ನುಂಟುಮಾಡಿ, ದೀನದರಿದ್ರರ ಕಷ್ಟಕೋಟಲೆಗಳನ್ನು ದೂರ ಮಾಡುವೆ.”

ಬಳಿಕ, ಸಕಲ ಆಧ್ಯಾತ್ಮಿಕ ರಹಸ್ಯಗಳಲ್ಲೂ ಪರಿಣತರಾದ ಶ್ರೀರಾಮಕೃಷ್ಣರು ಅವನಿಗೆ, ಇನ್ನು ಕೆಲವು ಕಾಲ ತನ್ನ ಶರೀರದ ಕಡೆಗೆ ಎಚ್ಚರದಿಂದಿರುವಂತೆ ಮತ್ತು ಮನೆಯಲ್ಲೇ ತಯಾರಾದ ಪರಿಶುದ್ಧ ಆಹಾರವನ್ನು ಮಾತ್ರ ಸೇವಿಸುವಂತೆ ಸೂಚನೆಯಿತ್ತರು.

ತರುವಾಯ ಅವರು ಇತರ ಆಪ್ತ ಶಿಷ್ಯರೆದುರು ನರೇಂದ್ರನ ಬಗೆಗಿನ ಅನೇಕ ಗುಪ್ತ ವಿಚಾರಗಳನ್ನು ಬಯಲುಪಡಿಸುತ್ತಾರೆ: “ಅವನು ಇಚ್ಛಾಮರಣಿ–ಯಾವಾಗ ತನ್ನ ನಿಜಸ್ವರೂಪ ವನ್ನು ಕಂಡುಕೊಳ್ಳುತ್ತಾನೋ ಆಮೇಲೆ ಕ್ಷಣಕಾಲವೂ ಶರೀರದಲ್ಲಿರಲು ಇಷ್ಟಪಡದೆ ಸ್ವೇಚ್ಛೆ ಯಿಂದ ದೇಹವನ್ನು ಬಿಟ್ಟುಬಿಡುತ್ತಾನೆ. ಮುಂದೊಂದು ಕಾಲ ಬರುತ್ತದೆ, ಆಗ ಅವನು ತನ್ನ ಅಪರಿಮಿತ ಬುದ್ಧಿಶಕ್ತಿ, ಆಧ್ಯಾತ್ಮಿಕ ಶಕ್ತಿಗಳಿಂದ ಜಗತ್ತನ್ನೇ ಬುಡಸಹಿತ ಅಲುಗಾಡಿಸಿಬಿಡುತ್ತಾನೆ! ಅವನಿಂದ ಬಹಳಷ್ಟು ಕೆಲಸವಾಗಬೇಕಾದ್ದಿದೆ. ಆದ್ದರಿಂದ ಅವನ ಅದ್ವೈತ ಸಾಕ್ಷಾತ್ಕಾರದ ಮೇಲೆ ಪರದೆಯನ್ನು ಎಳೆದುಬಿಡುವಂತೆ ಜಗನ್ಮಾತೆಯನ್ನು ಬಹಳವಾಗಿ ಪ್ರಾರ್ಥಿಸಿಕೊಂಡಿದ್ದೇನೆ. ಆದರೆ ಆ ಪರದೆ ತುಂಬ ಸೂಕ್ಷ್ಮ; ಎಷ್ಟರ ಮಟ್ಟಿಗೆಂದರೆ ಅದು ಯಾವಾಗ ಬೇಕಾದರೂ ಹರಿದುಹೋಗಬಹುದು.”

ತಮ್ಮ ಸಂಗಡಿಗನೂ ಪ್ರಿಯನಾಯಕನೂ ಆದ ನರೇಂದ್ರನ ಕುರಿತಾಗಿ ಶ್ರೀರಾಮಕೃಷ್ಣರ ಬಾಯಿಂದ ಇಂಥ ಪರಮಾದ್ಭುತ ಭವಿಷ್ಯವಾಣಿಯನ್ನು ಆಲಿಸಿದ ಶಿಷ್ಯರೆಲ್ಲ ಮೂಕವಿಸ್ಮಿತ ರಾದರು. ಆದರೆ ಅನೇಕ ವರ್ಷಗಳ ಬಳಿಕ ಆ ಪ್ರತಿಯೊಂದು ಮಾತೂ ಅಕ್ಷರಶಃ ಸತ್ಯವಾಗಿ ಪರಿಣಮಿಸಿದುದನ್ನು ಕಣ್ಣಾರೆ ನೋಡುವವರೆಗೂ ಅವುಗಳ ಬಗ್ಗೆ ಅವರ ಶಂಕೆ ಪೂರ್ಣವಾಗಿ ದೂರವಾಗಿರಲಿಲ್ಲ.

ಶ್ರೀರಾಮಕೃಷ್ಣರು ಹೀಗೆ ಬೀಗ ಹಾಕಿಟ್ಟರೂ, ನರೇಂದ್ರನಿಗೆ ಮುಂದೆ ಅನೇಕ ಬಾರಿ ನಿರ್ವಿಕಲ್ಪ ಸಮಾಧಿ ಪ್ರಾಪ್ತವಾಗಿತ್ತೆಂಬ ವಿಷಯ ತಿಳಿದುಬರುತ್ತದೆ. ಮುಂದೆ ಸ್ವಾಮಿ ವಿವೇಕಾನಂದ ಅಮೆರಿಕೆಯ ಶಿಕಾಗೋದ ಮಿಚಿಗನ್ ಸರೋವರದ ತೀರದಲ್ಲಿ, ನ್ಯೂಹ್ಯಾಂಪ್ಷೈರಿನ ಕ್ಯಾಂಪ್ ಪರ್ಸಿ ಎಂಬಲ್ಲಿ ಮತ್ತು ಸಹಸ್ರದ್ವೀಪೋದ್ಯಾನ ಎಂಬಲ್ಲಿ ಇಂತಹ ಅನುಭವವಾದುದನ್ನು ಕಂಡವರಿದ್ದಾರೆ.\footnote{*`ವಿಶ್ವಮಾನವ ವಿವೇಕಾನಂದ' (ಸಂ.೩) ಗ್ರಂಥದಲ್ಲಿ ಈ ಬಗ್ಗೆ ಇನ್ನಷ್ಟು ವಿವರವಾಗಿ ಓದಬಹುದು.} ಶ್ರೀರಾಮಕೃಷ್ಣರೇ ಹೇಳಿದಂತೆ, ನರೇಂದ್ರನ ಮೇಲಿದ್ದ ಮಾಯೆಯ ಪರದೆ ಎಷ್ಟು ತೆಳುವೆಂದರೆ ಅದು ಯಾವಾಗಲಾದರೂ ಹರಿದುಹೋಗಬಹುದಾಗಿತ್ತು. ಹೀಗೆ ಇಂತಹ ಸಂದರ್ಭ ಗಳಲ್ಲಿ ಅದು ಆಗಾಗ ಹರಿದುಹೋಗುತ್ತಿತ್ತು ಎಂದು ನಾವು ಊಹಿಸಬಹುದು. ಆದರೆ ಈ ಮೂರು ಸ್ಥಳಗಳಲ್ಲಿ ಮಾತ್ರವಲ್ಲದೆ ಅವರಿಗೆ ಇನ್ನೆಲ್ಲೆಲ್ಲಿ ಎಷ್ಟೆಷ್ಟು ಸಲ ಈ ಅನುಭವವಾಗಿತ್ತೋ ಬಲ್ಲವರಾರು?

ಶ್ರೀರಾಮಕೃಷ್ಣರ ದರ್ಶನಕ್ಕಾಗಿ ಜನ ಗುಂಪುಗುಂಪಾಗಿ ಧಾವಿಸಿ ಬರಲಾರಂಭಿಸಿದ್ದರು. ಅದನ್ನು ನೋಡಿದವರಿಗೆಲ್ಲ ಅವರ ಲೀಲಾಸಮಾಪ್ತಿಯ ದಿನ ಬಹಳ ವೇಗವಾಗಿ ಸಮೀಪಿಸುತ್ತಿದೆ ಎಂದೆನಿಸುತ್ತಿತ್ತು. ನರೇಂದ್ರನಂತೂ ಅದನ್ನು ಸ್ಪಷ್ಟವಾಗಿ ಕಾಣುತ್ತಿದ್ದ. ಅಗಲುವಿಕೆಯ ದುಃಖ ದೊಂದಿಗೆ, ತಾನು ಶ್ರೀರಾಮಕೃಷ್ಣರಿಂದ ಪಡೆದ ಅಮೂಲ್ಯ ಆಧ್ಯಾತ್ಮಿಕ ಅನುಭವಗಳ ಮಧುರ ಸ್ಮೃತಿಯೂ ಉಕ್ಕಿಬರುತ್ತಿತ್ತು. ಅಲ್ಲದೆ, ಅವರ ನಿರ್ಯಾಣದ ಬಳಿಕ ಅವರ ಆಧ್ಯಾತ್ಮಿಕ ಸಂಪತ್ತಿಗೆ ಅಧಿಕಾರಿಯಾಗಬೇಕಾದವನು ತಾನೇ ಎಂಬ ಅರಿವಿನಿಂದ ಆ ಹೊಣೆಗಾರಿಕೆಯನ್ನು ಹೊರಲು ಬೇಕಾದ ಹೊಸ ಶಕ್ತಿಯೊಂದು ಅವನಲ್ಲಿ ಉದ್ಭವಿಸುತ್ತಿತ್ತು.

ಇಲ್ಲಿ ನಾವು ಕುತೂಹಲಕರ ಅಂಶವೊಂದನ್ನು ಗಮನಿಸಬಹುದು; ಏನೆಂದರೆ ನರೇಂದ್ರನು ಜಗನ್ಮಾತೆಯನ್ನು ಒಪ್ಪಿಕೊಂಡು, ಜಗನ್ಮಾತೆಗೆ ಅವನ ಜವಾಬ್ದಾರಿಯನ್ನು ವಹಿಸಿಕೊಡು ವಂತಾಗಲು ಶ್ರೀರಾಮಕೃಷ್ಣರು ಮೂರ್ನಾಲ್ಕು ವರ್ಷಗಳೇ ಕಾಯಬೇಕಾಯಿತು. ಹೀಗೆ ಮಾಡಲು ಸಾಧ್ಯವಾದ ಮೇಲೆಯೇ ಅವರು ತಮ್ಮ ಆಧ್ಯಾತ್ಮಿಕ ಸಂಪತ್ತನ್ನು ಅವನಿಗೆ ಧಾರೆಯೆರೆದು ಶರೀರತ್ಯಾಗ ಮಾಡಲು ಸಿದ್ಧರಾಗಿರುವುದು. ಈ ವಿಷಯವಾಗಿ ನರೇಂದ್ರ ಮುಂದೆ ಹೇಳುತ್ತಾನೆ: “ಶ್ರೀರಾಮಕೃಷ್ಣರು ನನ್ನನ್ನು ಜಗನ್ಮಾತೆಗೆ ಒಪ್ಪಿಸಿಕೊಟ್ಟಮೇಲೆ ಅವರು ಆರೋಗ್ಯದಿಂದಿದ್ದುದು ಆರು ತಿಂಗಳು ಮಾತ್ರ. ಆಮೇಲೆ ಅವರು ಕಾಯಿಲೆ ಬಿದ್ದುಬಿಟ್ಟರು.” ನಿಜಕ್ಕೂ ಇನ್ನು ಮುಂದೆ ಶ್ರೀರಾಮಕೃಷ್ಣರ ಶಕ್ತಿತರಂಗವು ಹೊಸ ಕಾಲುವೆಯೊಂದರ ಮೂಲಕ ಹರಿದುಬರಲಿತ್ತು. ನರೇಂದ್ರನೇ ಈ ಹೊಸ ಕಾಲುವೆ. ನಾಲ್ಕು ವರ್ಷಗಳ ಸುದೀರ್ಘ ಪರಿಶ್ರಮದಿಂದ ಈ ಕಾಲುವೆಯನ್ನು ಶ್ರೀರಾಮಕೃಷ್ಣರು ಸಿದ್ಧಪಡಿಸಿದ್ದರು.

ಶ್ರೀರಾಮಕೃಷ್ಣರು ನಿರ್ವಿಕಲ್ಪ ಸಮಾಧಿಯ ಅನುಭವವನ್ನು ಮಾಡಿಸಿಕೊಟ್ಟ ಮೇಲೆ ನರೇಂದ್ರನ ಧ್ಯಾನದ ತೀವ್ರತೆ ಇನ್ನೂ ಹೆಚ್ಚಾಗಿಬಿಟ್ಟಿತ್ತು. ಒಂದು ದಿನ ಅವನೂ ಗಿರೀಶಚಂದ್ರ ಘೋಷನೂ ಒಂದು ಮರದ ಕೆಳಗೆ ಧ್ಯಾನಕ್ಕೆ ಕುಳಿತಿದ್ದರು. ಆದರೆ ಅಲ್ಲಿ ಅಸಂಖ್ಯಾತ ಸೊಳ್ಳೆಗಳು, ಸೊಳ್ಳೆಗಳ ಕಾಟ ತಡೆಯಲಾರದೆ ಗಿರೀಶ ಎದ್ದುಬಿಟ್ಟ. ಆ ಸೊಳ್ಳೆಗಳು ನರೇಂದ್ರನನ್ನು ಮಾತ್ರ ಕಚ್ಚಲೇ ಇಲ್ಲವೆ? ಗಿರೀಶ ನೋಡುತ್ತಾನೆ–ಸೊಳ್ಳೆಗಳು ನರೇಂದ್ರನ ಮೈಯನ್ನು ಹೇಗೆ ಮುತ್ತಿಕೊಂಡಿದ್ದುವೆಂದರೆ, ಅವನೊಂದು ಕರೀಕಂಬಳಿಯನ್ನು ಹೊದ್ದು ಕುಳಿತಿರುವಂತೆ ಕಾಣು ತ್ತಿತ್ತು! ಆದರೆ ಅವನು ಮಾತ್ರ ಅದರ ಪರಿವೆಯೇ ಇಲ್ಲದೆ ಗಾಢ ಧ್ಯಾನದಲ್ಲಿ ಮುಳುಗಿ ಹೋಗಿದ್ದ.

ಯುವಶಿಷ್ಯರೆಲ್ಲ ತನುಮನಪೂರ್ವಕವಾಗಿ ಗುರುಸೇವೆಯನ್ನು ಮಾಡಿಕೊಂಡು ಬರುತ್ತಿದ್ದಾರೆ. ಆದರೆ ತನುಮನಗಳೊಂದಿಗೆ ಧನವೂ ಬೇಕಲ್ಲ! ವಿದ್ಯಾರ್ಥಿಗಳಾದ ಆ ಯುವಕರ ಕೈಯಲ್ಲಿ ಧನವೆಲ್ಲಿಂದ ಬರಬೇಕು? ಶ್ರೀರಾಮಕೃಷ್ಣರ ಶುಶ್ರೂಷೆಯ ಹಾಗೂ ಅಲ್ಲಿರುವವರೆಲ್ಲರ ಊಟೋಪಚಾರಗಳ ವೆಚ್ಚಕ್ಕೆ ಮತ್ತು ಮನೆ ಬಾಡಿಗೆಗೆ ಹಣ ಕೊಡುತ್ತಿದ್ದವರು ಗೃಹೀಭಕ್ತರು. ಅವರಲ್ಲಿ ರಾಮಚಂದ್ರ ದತ್ತ, ಕಾಳೀಪದಘೋಷ್ ಹಾಗೂ ಸುರೇಂದ್ರನಾಥ ಮಿತ್ರ–ಇವರು ಪ್ರಧಾನವಾಗಿ ವಂತಿಗೆ ಕೊಡುತ್ತಿದ್ದವರು. ಈ ಯುವಕರು ಹಣವನ್ನು ಹದವರಿತು, ಬೇಕಾದಷ್ಟನ್ನು ಮಾತ್ರವೇ ವೆಚ್ಚಮಾಡುತ್ತಿದ್ದರು. ಹಿರಿಯ ಗೋಪಾಲ ಈ ಹಣದ ಲೆಕ್ಕವಿಡುತ್ತಿದ್ದ, ರಾಮಚಂದ್ರ ದತ್ತ ಆಗಾಗ ಆ ಲೆಕ್ಕವನ್ನು ಪರಿಶೋಧಿಸಿ ನೋಡುತ್ತಿದ್ದ. ಒಂದು ಸಲ ಈ ಖರ್ಚು ಸ್ವಲ್ಪ ಹೆಚ್ಚಾಗಿ ಕಂಡುಬಂದಿರಬೇಕು. ಗೃಹೀಭಕ್ತರು ಅಸಮಾಧಾನಗೊಂಡು ಖಾರವಾಗಿ ಮಾತಾಡಿ ಬಿಟ್ಟರು. ಗುರುಭಕ್ತಿ ಎಷ್ಟೇ ಇದ್ದಿರಬಹುದು. ಆದರೆ ಹಣ ಹೆಚ್ಚಾಗಿ ಖರ್ಚಾಗುತ್ತಿದೆ ಎಂದು ಅನ್ನಿಸಿದಾಗ ಅವರಿಗೆ ಹೊಟ್ಟೆ ಚುರುಗುಟ್ಟಿದ್ದು ಸಹಜವೇ. ಆದರೆ ಸ್ವಾಭಿಮಾನಿಗಳಾದ ಯುವಕರ ಪಾಲಿಗೆ ಈ ಮಾತು ಸ್ವಲ್ಪ ಹೆಚ್ಚಾಯಿತು. ಅವರಾದರೂ ತಮ್ಮ ಸ್ವಂತಕ್ಕೋಸ್ಕರ ಖರ್ಚುಮಾಡಿ ಕೊಳ್ಳುತ್ತಿದ್ದಾರೆಯೆ! ನರೇಂದ್ರ ಬೇರೆ ದಾರಿಗಾಣದೆ ಶ್ರೀರಾಮಕೃಷ್ಣರ ಬಳಿಗೇ ಹೋಗಿ ವಿಷಯ ವನ್ನೆಲ್ಲ ತಿಳಿಸಿದ. ಅವನ ಮಾತನ್ನು ಸಹಾನುಭೂತಿಯಿಂದ ಆಲಿಸಿದ ಶ್ರೀರಾಮಕೃಷ್ಣರು ತಕ್ಷಣವೇ ನುಡಿದರು: “ನರೇನ್, ನೀನು ನನ್ನನ್ನು ನಿನ್ನ ಹೆಗಲ ಮೇಲೆ ಕುಳ್ಳಿರಿಸಿಕೊಂಡು ಎಲ್ಲಿಗೆ ಕರೆದು ಕೊಂಡು ಹೋಗುತ್ತೀಯೋ ಅಲ್ಲಿಗೆ ಬರುತ್ತೇನೆ; ನೀನು ನನ್ನನ್ನು ಹೇಗಿರಿಸಿದರೆ ಹಾಗಿರುತ್ತೇನೆ.” ಅಂಥ ಅಸಹಾಯಕ ಪರಿಸ್ಥಿತಿಯಲ್ಲೂ ಶ್ರೀರಾಮಕೃಷ್ಣರ ಮಾತಿನಲ್ಲಿ ಎಂತಹ ದೃಢತೆ, ಯುವ ಶಿಷ್ಯರ ಬಗ್ಗೆ ಎಂತಹ ವಿಶ್ವಾಸ ಪ್ರೀತಿ! ಆದರೆ ಅವರ ಆ ಮಾತನ್ನು ಕೇಳಿ ಯುವಕರಿಗೆಲ್ಲ ಚಿಂತೆ ಹುಟ್ಟಿತು. ಶ್ರೀರಾಮಕೃಷ್ಣರೇನೋ ಹೃತ್ಪೂರ್ವಕವಾಗಿಯೇ ಹಾಗೆಂದಿದ್ದರು. ಆದರೆ ಈ ಯುವಕರು ಅವರನ್ನು ಎಲ್ಲಿಗೆ ಕರೆದೊಯ್ದಾರು? ಅಷ್ಟೊಂದು ಖರ್ಚನ್ನು ಹೇಗೆ ನಿಭಾಯಿಸಿ ಯಾರು? ಈ ವಿಚಾರವೆಲ್ಲ ಲಕ್ಷ್ಮೀನಾರಾಯಣ ಮಾರ್ವಾಡಿ ಎಂಬ ಶ್ರೀಮಂತ ಭಕ್ತನಿಗೆ ತಿಳಿಯಿತು. ಅವನು ತಕ್ಷಣ ಧನ ಸಹಾಯ ನೀಡಲು ಮುಂದಾದ. ಆದರೆ ಶ್ರೀರಾಮಕೃಷ್ಣರು ಅದನ್ನು ಸ್ವೀಕರಿಸಲು ಒಪ್ಪಲಿಲ್ಲ. ಬದಲಾಗಿ ಗಿರೀಶನನ್ನು ಕರೆದು, “ನೋಡು, ನೀನೊಬ್ಬನೇ ಎಲ್ಲ ಖರ್ಚನ್ನೂ ವಹಿಸಿಕೋ” ಎಂದು ಹೇಳಿಬಿಟ್ಟರು. ಗಿರೀಶ ಅಂಥ ಶ್ರೀಮಂತನೇನೂ ಅಲ್ಲ. ಆದರೆ ಶ್ರೀರಾಮಕೃಷ್ಣರು ಕೇವಲ ಹಣವನ್ನು ನೋಡುವವರಲ್ಲ. ಹಣದೊಂದಿಗೆ ಭಕ್ತಿ-ವಿಶ್ವಾಸ ಇರ ಬೇಕು. ಆದ್ದರಿಂದ ಗಿರೀಶ ಏನು ಕೊಟ್ಟರೆ ಅದೇ ಸಾಕು. ಶ್ರೀರಾಮಕೃಷ್ಣರು ತನ್ನಂಥವನನ್ನು ವಿಶ್ವಾಸಕ್ಕೆ ತೆಗೆದುಕೊಂಡು ಕೇಳುವ ಕೃಪೆ ಮಾಡಿದರಲ್ಲ ಎಂದು ಗಿರೀಶನಿಗೆ ಹೃದಯದುಂಬಿ ಬಂತು. ಕೂಡಲೇ ಅವನು ಭಾವಭರಿತನಾಗಿ, “ಅಗತ್ಯ ಬಿದ್ದರೆ ನನ್ನ ಆಸ್ತಿ-ಮನೆಯನ್ನೇ ಮಾರಿ ಬಿಡುತ್ತೇನೆ” ಎಂದು ನುಡಿದ. ಆದರೆ ಅಂತಹ ಪರಿಸ್ಥಿತಿ ಒದಗಲಿಲ್ಲ, ಹೇಗೋ ನಿಭಾವಣೆ ಯಾಯಿತು. ಅಂದಿನಿಂದ ಆ ಯುವಕರು ಮಾತ್ರ ಆ ಮೂವರು ಗೃಹಸ್ಥಭಕ್ತರಿಂದ ಯಾವ ಧನ ಸಹಾಯವನ್ನೂ ಪಡೆಯಲಿಲ್ಲ. ಅಷ್ಟೇ ಅಲ್ಲ, ಅವರನ್ನು ಶ್ರೀರಾಮಕೃಷ್ಣರ ದರ್ಶನ ಮಾಡಲೂ ಕೆಲದಿನ ಬಿಡಲಿಲ್ಲ! ಆದರೆ ಈ ವೈಮನಸ್ಯ ಹೆಚ್ಚು ದಿನ ಮುಂದುವರಿಯಲಿಲ್ಲ. ಶ್ರೀರಾಮ ಕೃಷ್ಣರು ಎರಡು ಕಡೆಯವರನ್ನೂ ಸಮಾಧಾನಪಡಿಸಿ ಹೊಂದಿಕೊಳ್ಳುವಂತೆ ಮಾಡಿದರು. ಏಕೆಂದರೆ, ಯಾವ ಮನುಷ್ಯನೂ ಪರಿಪೂರ್ಣನಲ್ಲ; ಪ್ರತಿಯೊಬ್ಬನಲ್ಲೂ ಒಂದಲ್ಲ ಒಂದು ಬಗೆಯ ದೋಷ ಇದ್ದೇ ಇರುತ್ತದೆ ಎಂದು ಅವರಿಗೆ ಚೆನ್ನಾಗಿ ಗೊತ್ತಿತ್ತು.

ತಮ್ಮ ಜೀವನದ ಅಂತ್ಯಕ್ಕೆ ಕೆಲವೇ ದಿನಗಳು ಉಳಿದಿರುವುದನ್ನು ಕಂಡು ಶ್ರೀರಾಮಕೃಷ್ಣರು ಇನ್ನಷ್ಟು ಉತ್ಸಾಹದಿಂದ ಶಿಷ್ಯರಿಗೆ ಮಾರ್ಗದರ್ಶನ ನೀಡತೊಡಗಿದ್ದರು; ಅವರ ವ್ಯಕ್ತಿತ್ವವನ್ನು ತಿದ್ದುತ್ತಿದ್ದರು. ಅದರಲ್ಲೂ ನರೇಂದ್ರನನ್ನು ಪ್ರತಿದಿನ ಸಂಜೆಯ ಹೊತ್ತು ತಮ್ಮ ಕೋಣೆಗೆ ಬರಮಾಡಿಕೊಂಡು, ಸುಮಾರು ಎರಡು-ಮೂರು ಗಂಟೆಗಳ ಕಾಲ ವೈಯಕ್ತಿಕವಾಗಿ ಬೋಧನೆ ಗಳನ್ನು ನೀಡುತ್ತಿದ್ದರು; ಹಲವಾರು ಬಗೆಯ ಆಧ್ಯಾತ್ಮಿಕ ಸಾಧನೆಗಳ ಮರ್ಮವನ್ನು ತಿಳಿಸಿಕೊಡು ತ್ತಿದ್ದರು. ಜೊತೆಗೆ, ಅವನು ಇತರ ಶಿಷ್ಯರನ್ನು ಮುಂದೆ ಹೇಗೆ ನೋಡಿಕೊಳ್ಳಬೇಕು, ಸಾಧನೆಯಲ್ಲಿ ಮುಂದುವರಿದು ತ್ಯಾಗಮಯ ಜೀವನವನ್ನು ನಡೆಸುವಂತೆ ಅವರಿಗೆ ಹೇಗೆ ತರಬೇತಿ ಕೊಡಬೇಕು ಎಂಬುದನ್ನು ವಿವರಿಸುತ್ತಿದ್ದರು.

ಒಂದು ದಿನ ಅವರು, ನರೇಂದ್ರನನ್ನು ಬಿಟ್ಟು ಉಳಿದ ಯುವಶಿಷ್ಯರನ್ನೆಲ್ಲ ತ್ವರಿತದ ಕರೆ ಕೊಟ್ಟು ಬಳಿಗೆ ಕರೆಯಿಸಿಕೊಂಡು, “ಮುಂದೆ ನೀವು ಯಾವಾಗಲೂ ನರೇಂದ್ರನ ಕಡೆಗೆ ವಿಶೇಷ ಗಮನ ಕೊಡಬೇಕು. ಅವನ ಆರೋಗ್ಯಕ್ಕೆ, ಅನುಕೂಲತೆಗೆ ಏನೇನು ಮಾಡಲು ಸಾಧ್ಯವೋ ಅದೆಲ್ಲವನ್ನೂ ನೀವು ಮಾಡಿಕೊಡಬೇಕು” ಎಂದು ಸ್ಪಷ್ಟವಾಗಿ ಆದೇಶ ನೀಡಿದರು. ಬಳಿಕ ನರೇಂದ್ರನೊಬ್ಬನನ್ನೇ ಕರೆದು, “ನರೇನ್! ನಾನು ಈ ಹುಡುಗರನ್ನೆಲ್ಲ ನಿನ್ನ ವಶಕ್ಕೆ ಬಿಟ್ಟು ಹೋಗುತ್ತಿದ್ದೇನೆ. ಅವರನ್ನೆಲ್ಲ ರಕ್ಷಿಸುವ ಜವಾಬ್ದಾರಿ ನಿನ್ನದು” ಎಂದು ನೆನಪಿಸಿದರು. ‘ನರೇಂದ್ರನೇ ನಾಯಕ, ಉಳಿದವರೆಲ್ಲ ಅವನನ್ನು ಅನುಸರಿಸಬೇಕು’ ಎಂಬ ಮಾತನ್ನು ಶ್ರೀರಾಮ ಕೃಷ್ಣರು ಹಿಂದೆ ಅನೇಕ ಸಲ ಹೇಳಿದ್ದುಂಟು. ಆದರೆ ಈ ಸಂದರ್ಭದಲ್ಲಿ ಅದನ್ನೇ ಮತ್ತೆ ಒತ್ತಿ ಹೇಳಿದ್ದಕ್ಕೆ ಒಂದು ವಿಶೇಷ ಕಾರಣವಿದೆ. ಏನೆಂದರೆ, ಮರಣಕಾಲದಲ್ಲಿ ಹೇಳಿದ ಮಾತುಗಳು ಮನಸ್ಸಿನಲ್ಲಿ ವಿಶೇಷವಾಗಿ ನಿಲ್ಲುತ್ತವೆ. ಅಲ್ಲದೆ, ಮನುಷ್ಯರು ಎಷ್ಟೇ ಒಳ್ಳೆಯವರಾದರೂ ಅವರನ್ನು ಒಂದುಗೂಡಿಸಿ ಕೆಲಸ ಮಾಡಿಸುವುದು ಸುಲಭವಲ್ಲ. ಅದರಲ್ಲೂ ದೀರ್ಘಕಾಲದವರೆಗೆ ಅವರನ್ನೆಲ್ಲ ಒಂದುಗೂಡಿಸಿ ಒಗ್ಗಟ್ಟಾಗಿಡುವುದು ಇನ್ನೂ ಕಷ್ಟದ ಕೆಲಸ. ಬುದ್ಧಿವಂತರಾದವರು ಒಗ್ಗಟ್ಟಾಗಿರುವುದು ಮತ್ತೂ ಕಷ್ಟ! ಅವರೆಲ್ಲ ಒಗ್ಗಟ್ಟಾಗಿರದಿದ್ದರೆ ಅವರ ಮೂಲಕ ಮಹ ತ್ಕಾರ್ಯಗಳನ್ನು ಸಾಧಿಸುವುದು ಅಸಂಭವನೀಯ. ಶ್ರೀರಾಮಕೃಷ್ಣರು ತಮ್ಮ ಅಸಾಮಾನ್ಯ ಜೀವನದ ಮೂಲಕ ಇಡೀ ಜಗತ್ತಿಗೇ ಮಹತ್ತರವಾದ ಸಂದೇಶಗಳನ್ನು ನೀಡಿ, ಜಗತ್ತಿನ ವಿಚಾರಧಾರೆಗೆ ಒಂದು ಹೊಸ ಚೈತನ್ಯವನ್ನು ಕೊಡುವುದಕ್ಕಾಗಿ ಬಂದಿದ್ದಾರೆ. ಈ ಕೆಲಸ ಅವರ ನಿರ್ಯಾಣದ ಬಳಿಕ ಈ ನವಯುವಕರ ಮೂಲಕವೇ ಆಗಬೇಕಾಗಿದೆ. ಆದ್ದರಿಂದಲೇ ಮರಣ ಶಯ್ಯೆಯಲ್ಲೂ ಈ ಯುವಕರನ್ನೆಲ್ಲ ಕರೆಸಿಕೊಂಡು, ಅವರೆಲ್ಲ ನರೇಂದ್ರನ ಮುಂದಾಳ್ತನದಲ್ಲಿ ಒಗ್ಗಟ್ಟಾಗಿರಬೇಕೆಂಬ ಮಾತನ್ನು ಮನದಟ್ಟು ಮಾಡಿಸುತ್ತಿದ್ದಾರೆ.

ತಮ್ಮ ಜೀವಿತಾವಧಿಯ ಈ ಕೊನೆಯ ದಿನಗಳಲ್ಲೇ ಒಮ್ಮೆ ಶ್ರೀರಾಮಕೃಷ್ಣರು ಇನ್ನೊಂದು ಬಹು ಮುಖ್ಯವಾದ ವಿಷಯವನ್ನು ವ್ಯಕ್ತಪಡಿಸಿದರು. ಗಂಟಲನೋವಿನಿಂದಾಗಿ ಅವರಿಗೆ ಮಾತ ನಾಡಲೂ ಸಾಧ್ಯವಿರದ ಪರಿಸ್ಥಿತಿಯಲ್ಲೂ ಒಂದು ದಿನ ನರೇಂದ್ರನನ್ನು ಕರೆದು ಬಳಿಯಲ್ಲಿ ಕುಳ್ಳಿರಿಸಿಕೊಂಡು, ಒಂದು ತುಂಡು ಕಾಗದದಲ್ಲಿ ಒಂದು ಚಿತ್ರ ಬರೆದು, ಅದರ ಪಕ್ಕದಲ್ಲಿ ‘ನರೇನ್ ಸ್ವದೇಶದಲ್ಲೂ ವಿದೇಶಗಳಲ್ಲೂ ಧರ್ಮಬೋಧನೆ ಮಾಡುತ್ತಾನೆ’ ಎಂದು ಬರೆದು ತೋರಿಸಿದರು. (ಈ ಚಿತ್ರವನ್ನೂ ಅದರ ವಿವರಣೆಯನ್ನೂ ಗ್ರಂಥದ ಪ್ರಾರಂಭದಲ್ಲಿ ನೋಡ ಬಹುದು.) ಈ ಭವಿಷ್ಯವಾಣಿಯನ್ನು ಕೇಳಿ ಎಲ್ಲರಿಗೂ ಅದೆಷ್ಟು ಆಶ್ಚರ್ಯವಾಗಿರಬೇಕು! ಆದರೆ ನರೇಂದ್ರನಿಗೆ ತುಂಬ ಸಂಕೋಚವಾಯಿತು. “ನಾನದನ್ನು ಮಾಡಲಾರೆ” ಎಂದುಬಿಟ್ಟ. ಆದರೆ ಶ್ರೀರಾಮಕೃಷ್ಣರು ಬಿಟ್ಟಾರೆಯೆ! ಮುಗುಳ್ನಕ್ಕು, “ನೀನು ಮಾಡಲೇಬೇಕಾಗುತ್ತದೆ” ಎಂದರು. ಆಗ ಅವನು ಸುಮ್ಮನಾಗಬೇಕಾಯಿತು. ಈ ಮೂಲಕ ಶ್ರೀರಾಮಕೃಷ್ಣರು ಅವನನ್ನು ಅಧಿಕೃತ ಧರ್ಮಪ್ರಸಾರಕನನ್ನಾಗಿ ನೇಮಿಸುತ್ತಿದ್ದಾರೆ. “ಒಬ್ಬ ಮನುಷ್ಯ ಧರ್ಮಪ್ರಸಾರ ಮಾಡಬೇಕಾದರೆ ಭಗವಂತನಿಂದ ಅಧಿಕೃತನಾಗಿರಬೇಕು; ಭಗವಂತ ‘ಇವನಿಂದ ಧರ್ಮಪ್ರಸಾರ ನಡೆಯಲಿ’ ಎಂದು ಇಚ್ಛಿಸಿದರೆ ಮಾತ್ರ ಅವನು ಮಾಡುವ ಬೋಧನೆಯನ್ನು ಜನ ಕೇಳುತ್ತಾರೆ, ಕೇಳಿ ಪ್ರಭಾವಿತರಾಗುತ್ತಾರೆ, ಅನುಷ್ಠಾನ ಮಾಡುತ್ತಾರೆ. ಭಗವಂತನ ಕೃಪಾಶೀರ್ವಾದ ಇಲ್ಲದೆ ಹೋದರೂ ಮನುಷ್ಯ ಗಂಟಲು ಕಿತ್ತುಕೊಳ್ಳಬಹುದು; ಆದರೆ ಜನ ಅದಕ್ಕೆ ಮನ್ನಣೆ ಕೊಡುವು ದಿಲ್ಲ” ಎಂದು ಅವರು ಪದೇಪದೇ ಹೇಳುತ್ತಿದ್ದರು. ಈಗ ಅವರು ನರೇಂದ್ರನನ್ನು ಧರ್ಮ ದೂತನನ್ನಾಗಿ ತಾವೇ ಸ್ವತಃ ನೇಮಕ ಮಾಡುತ್ತಿರುವುದರಿಂದ, ಅದಕ್ಕೆ ಭಗವಂತನ ಸಮ್ಮತಿ- ಅನುಗ್ರಹ ಇದ್ದೇ ಇದೆಯೆಂದೇ ತಾತ್ಪರ್ಯ. ಇದಲ್ಲದೆ, ಹಿಂದೆಯೇ ಅವರೊಮ್ಮೆ ನರೇಂದ್ರನಿಗೆ, “ನನ್ನ ಸಿದ್ಧಿಗಳೆಲ್ಲ ಮುಂದೆ ಸಕಾಲದಲ್ಲಿ ನಿನ್ನ ಮೂಲಕ ಪ್ರಕಟಗೊಳ್ಳುತ್ತವೆ” ಎಂದು ಹೇಳಿದ್ದರು. ಮುಂದೆ ಆತ ಸ್ವಾಮಿ ವಿವೇಕಾನಂದರಾಗಿ ಧರ್ಮಪ್ರಸಾರ ಕಾರ್ಯವನ್ನು ಪ್ರಾರಂಭಿಸಿದಾಗ ಅವರ ಮೂಲಕ ಅಪಾರ ಆಧ್ಯಾತ್ಮಿಕ ಶಕ್ತಿಯೂ ಸಿದ್ಧಿಗಳೂ ನಾನಾ ವಿಧದಲ್ಲಿ ಪ್ರಕಟಗೊಳ್ಳಲಿರುವುದನ್ನು ನಾವು ಕಾಣಲಿದ್ದೇವೆ.

ಶ್ರೀರಾಮಕೃಷ್ಣರ ಮಹಾಸಮಾಧಿಗೆ ಮೂರ್ನಾಲ್ಕು ದಿನಗಳಿರುವಾಗ ಅತ್ಯಂತ ಮಹತ್ತರವಾದ ಒಂದು ಘಟನೆ ನಡೆಯಿತು. ಅವರು ನರೇಂದ್ರನನ್ನು ಕರೆದು ಎದುರಿಗೆ ಕುಳ್ಳಿರಿಸಿಕೊಂಡರು. ಮಾತನ್ನೇನೂ ಆಡಲಿಲ್ಲ; ಆದರೆ ಅವನನ್ನೇ ನೆಟ್ಟನೋಟದಿಂದ ದಿಟ್ಟಿಸಲಾರಂಭಿಸಿದರು. ನೋಡ ನೋಡುತ್ತ ಹಾಗೇ ಗಾಢಸಮಾಧಿಸ್ಥರಾಗಿಬಿಟ್ಟರು. ಆಗ ಅವನಿಗೆ ವಿದ್ಯುತ್ ಪ್ರವಾಹದಂತೆ ಏನೋ ಒಂದು ಸೂಕ್ಷ್ಮಶಕ್ತಿ ತನ್ನೊಳಗೆ ಪ್ರವೇಶ ಮಾಡುತ್ತಿರುವಂತೆ ಅನುಭವವಾಯಿತು; ಬಾಹ್ಯಪ್ರಜ್ಞೆ ತಪ್ಪಿಹೋಯಿತು. ಬಳಿಕ ಮೈ ತಿಳಿದಾಗ ಅವನು ನೋಡುತ್ತಾನೆ, ಶ್ರೀರಾಮಕೃಷ್ಣರು ಕಣ್ಣೀರ್ ಗರೆಯುತ್ತಿದ್ದಾರೆ! ನರೇಂದ್ರ ಆಶ್ಚರ್ಯಪಡುತ್ತ ಕಣ್ಣೀರಿಗೆ ಕಾರಣವೇನೆಂದು ಕೇಳಿದ. ಶ್ರೀರಾಮಕೃಷ್ಣರೆನ್ನುತ್ತಾರೆ: “ಓ ನರೇನ್, ಇಂದು ನಾನು ನಿನಗೆ ನನ್ನ ಸರ್ವಸ್ವವನ್ನೂ ಕೊಟ್ಟುಬಿಟ್ಟೆ. ಈಗ ನಾನು ಕೇವಲ ಫಕೀರನಾಗಿಬಿಟ್ಟೆ, ಬರಿಗೈ ಭಿಕಾರಿಯಾಗಿಬಿಟ್ಟೆ! ಇರಲಿ, ನಾನೀಗ ನಿನ್ನೊಳಗೆ ಹರಿಯಿಸಿದ ಶಕ್ತಿಯಿಂದ ಮುಂದೆ ನಿನ್ನ ಮೂಲಕ ದೊಡ್ಡದೊಡ್ಡ ಕಾರ್ಯ ಗಳೆಲ್ಲ ನಡೆಯುತ್ತವೆ. ಅವೆಲ್ಲ ಮುಗಿದ ಮೇಲೆಯೇ ನೀನು ನಿನ್ನ ಸ್ವಸ್ಥಳಕ್ಕೆ ಹಿಂದಿರುಗುವುದು.”

ಈ ಮಾತುಗಳ ಮರ್ಮವನ್ನು ಪೂರ್ಣವಾಗಿ ಗ್ರಹಿಸಲಾರದೆ ನರೇಂದ್ರ ಆಶ್ಚರ್ಯಚಕಿತನಾದ.

೧೩ನೇ ಆಗಸ್ಟ್ ೧೮೮೬; ಶ್ರೀರಾಮಕೃಷ್ಣರು ರುಗ್ಣಶಯ್ಯೆಯಲ್ಲಿ ಮಲಗಿದ್ದಾರೆ. ಅವರ ಶರೀರ ಸಂಪೂರ್ಣವಾಗಿ ಜರ್ಝರಿತವಾಗಿಬಿಟ್ಟಿದೆ. ಅಲುಗಾಡಲೂ ಶಕ್ತಿಯಿಲ್ಲದಂತಾಗಿದೆ. ಶಿಷ್ಯರೂ ಭಕ್ತರೂ ಸುತ್ತ ಮೌನವಾಗಿ ನಿಂತಿದ್ದಾರೆ. ನರೇಂದ್ರನ ಮನಸ್ಸಿನಲ್ಲಿ ಮಾತ್ರ ಮಂಥನ ನಡೆಯು ತ್ತಿದೆ; ಸಂದೇಹ ಸುಳಿದಾಡುತ್ತಿದೆ. ‘ನಮ್ಮೆಲ್ಲರ ಕಣ್ಣೆದುರಿಗೆ ಒಬ್ಬ ಸಾಮಾನ್ಯ ಮರ್ತ್ಯನಂತೆ ಮೃತ್ಯುಶಯ್ಯೆಯಲ್ಲಿ ಮಲಗಿರುವ ಈ ಶ್ರೀರಾಮಕೃಷ್ಣರು ಯಾರು? ಇವರೇನೋ ಹಲವಾರು ಬಾರಿ ಹೇಳಿದ್ದಾರೆ, ತಾವು ಭಗವಂತನ ಅವತಾರವೆಂದು. ಇನ್ನೂ ಎಷ್ಟೋ ಜನ ಹಾಗೆಯೇ ಹೇಳಿದ್ದಾರೆ. ನನಗೆ ಮಾತ್ರ ಅದಿನ್ನೂ ಒಪ್ಪಿಗೆಯಾಗಿಲ್ಲ. ಆದರೆ ಈ ಪ್ರಾಣೋತ­ಮಣದ ಸಮಯದಲ್ಲಿ, ಶರೀರ ಸಂಪೂರ್ಣ ದುರ್ಬಲಗೊಂಡು ಯಾತನೆಯನ್ನು ಅನುಭವಿಸುತ್ತಿರುವ ಇಂತಹ ಸಮಯದಲ್ಲಿ ಇವರು ಹೇಳಲಿ, “ನಾನು ಭಗವಂತನ ಅವತಾರ” ಅಂತ. ಆಗ ನಾನದನ್ನು ನಂಬಿಯೇನು.’ ಅವನ ಮನಸ್ಸಿನಲ್ಲಿ ಈ ಆಲೋಚನೆ ಸುಳಿದದ್ದೇ ತಡ, ಶ್ರೀರಾಮಕೃಷ್ಣರು ತಮ್ಮ ಸರ್ವಶಕ್ತಿಯನ್ನೂ ಒಗ್ಗೂಡಿಸಿಕೊಂಡು ಅವನೆಡೆಗೆ ತಿರುಗಿ ಉದ್ಗರಿಸಿದರು: “ಓ ನನ್ನ ನರೇನ್, ನಿನಗಿನ್ನೂ ನಂಬಿಕೆಯಾಗಲಿಲ್ಲವೆ? ಹಿಂದೆ ಯಾರು ರಾಮನಾಗಿದ್ದನೋ ಯಾರು ಕೃಷ್ಣನಾಗಿ ದ್ದನೋ ಅವನೇ ಈಗ ಈ ಶರೀರದಲ್ಲಿ ರಾಮಕೃಷ್ಣನಾಗಿ ಅವತರಿಸಿದ್ದಾನೆ. ಆದರೆ ಇದು ಕೇವಲ ನಿನ್ನ ವೇದಾಂತದ ದೃಷ್ಟಿಯಿಂದಲ್ಲ, \textit{ನಿಜವಾಗಿಯೂ}.”

ನರೇಂದ್ರ ಸಿಡಿಲು ಬಡಿದವನಂತೆ ಬೆಚ್ಚಿಬಿದ್ದ. ಶ್ರೀರಾಮಕೃಷ್ಣರ ಈ ಹಠಾತ್ ಉದ್ಗಾರವನ್ನು ಕೇಳಿ ಅಲ್ಲಿದ್ದ ಇನ್ನಿತರರೂ ಆಶ್ಚರ್ಯಗೊಂಡರು. ಆದರೆ ಅದು ನರೇಂದ್ರನ ಮನದಾಳ ದಲ್ಲೆದ್ದಿದ್ದ ಪ್ರಶ್ನೆಗೆ ಉತ್ತರ ಎನ್ನುವುದು ಅವರು ಹೇಗೆ ಊಹಿಸಿಯಾರು! ನರೇಂದ್ರನಿಗೆ ಅತ್ಯಾಶ್ಚರ್ಯ, ಜೊತೆಗೆ ಅಷ್ಟೇ ನಾಚಿಕೆ. ಇಲ್ಲಿಯವರೆಗೂ ತಾನು ಶ್ರೀರಾಮಕೃಷ್ಣರನ್ನು ಪರೀಕ್ಷಿ ಸಿದ ಪರಿಗಳೆಷ್ಟು! ಅವರು ತನಗೆ ಮಾಡಿಸಿಕೊಟ್ಟ ಅನುಭವಗಳೆಷ್ಟು! ದರ್ಶನಗಳೆಷ್ಟು! ಇಷ್ಟೆಲ್ಲ ಆದಮೇಲೂ, ಈ ಘಳಿಗೆಯಲ್ಲಿ ತಾನು ಅವರ ದೈವತ್ವವನ್ನು ಸಂದೇಹಿಸಿದೆನಲ್ಲ–ಎಂಬ ಪರಿತಾಪ. ಆದರೆ ಅವರು ತನ್ನ ಸಂದೇಹಕ್ಕೆ, ದ್ವಂದ್ವಕ್ಕೆಡೆಯಿಲ್ಲದಂತಹ ಉತ್ತರ ನೀಡಿದ್ದರಿಂದ ಅಷ್ಟೇ ಸಮಾಧಾನ ಮತ್ತು ಸಂತೋಷ ಕೂಡ.

ಒಂದು ದೃಷ್ಟಿಯಲ್ಲಿ ನರೇಂದ್ರನ ಮನಸ್ಸಿನಲ್ಲಿ ಅಂತಹ ಸಂಶಯವೆದ್ದುದು ನಿರರ್ಥಕ ವೇನಲ್ಲ. ಅದರಿಂದಾಗಿಯೇ ಈ ಸಂದರ್ಭದಲ್ಲಿ ಶ್ರೀರಾಮಕೃಷ್ಣರು ತಮ್ಮ ಅವತಾರತ್ವವನ್ನು ಘಂಟಾಘೋಷವಾಗಿ ಬಹಿರಂಗಪಡಿಸುವಂತಾಯಿತಲ್ಲವೆ? ವೀರಾಧಿವೀರನಾದ ಅರ್ಜುನ ಯುದ್ಧಭೂಮಿಯಲ್ಲಿ ಸಂದೇಹದಿಂದ ಗಲಿಬಿಲಿಗೊಳಗಾಗಿರದಿದ್ದರೆ ಶ್ರೀಕೃಷ್ಣನು ಗೀತಾಮೃತ ವನ್ನು ವರ್ಷಿಸಲು ಅವಕಾಶವೆಲ್ಲಿರುತ್ತಿತ್ತು!

ಅಲ್ಲದೆ, ಇಲ್ಲಿ ಶ್ರೀರಾಮಕೃಷ್ಣರಾಡುವ ಮಾತು ತುಂಬ ಗಹನವಾದ ಅರ್ಥದಿಂದ ಕೂಡಿದೆ. ಅವರು ‘ನಾನೊಬ್ಬ ಅವತಾರಪುರುಷ’ ಎಂದಷ್ಟೇ ಹೇಳಲಿಲ್ಲ. ಅಥವಾ, ‘ಹೇಗೆ ರಾಮನೂ ಕೃಷ್ಣನೂ ಅವತಾರಪುರುಷರೋ ಹಾಗೆಯೇ ನಾನೂ ಒಬ್ಬ ಅವತಾರಪುರುಷ’ ಎಂದೂ ಹೇಳ ಲಿಲ್ಲ. ಬದಲಾಗಿ, ‘ಹಿಂದೆ ಯಾರು ರಾಮನಾಗಿದ್ದನೋ ಯಾರು ಕೃಷ್ಣನಾಗಿದ್ದನೋ ಅವನೇ ಈಗ ರಾಮಕೃಷ್ಣನಾಗಿ ಅವತರಿಸಿದ್ದಾನೆ’ ಎಂದು ಸಂಶಯಕ್ಕೆಡೆಯಿಲ್ಲದಂತೆ ಹೇಳುತ್ತಿದ್ದಾರೆ. ಜೊತೆಗೆ, ಮತ್ತೆ ಸ್ಪಷ್ಟಪಡಿಸುತ್ತಿದ್ದಾರೆ, ‘ಇದು ಕೇವಲ ನಿನ್ನ ವೇದಾಂತದ ದೃಷ್ಟಿಯಿಂದಲ್ಲ’ ಎಂದು. ಇದೂ ಕೂಡ ಅಷ್ಟೇ ಅರ್ಥಪೂರ್ಣವಾದದ್ದು. ಹೇಗೆಂದರೆ, ಅದ್ವೈತವೇದಾಂತದ ಪ್ರಕಾರ ಎಲ್ಲ ಜೀವಿಗಳೂ ಸತ್-ಚಿತ್-ಆನಂದಸ್ವರೂಪನಾದ ಪರಬ್ರಹ್ಮನ ಆವಿರ್ಭಾವಗಳೇ. ಎಂದರೆ ಮನುಷ್ಯರೆಲ್ಲರೂ ಭಗವತ್ಸ್ವರೂಪಿಗಳೇ, ಅವತಾರರೇ. ಆದರೆ ತಾವು ಅವತಾರವೆಂಬುದು ಈ ಅರ್ಥದಲ್ಲಲ್ಲ ಎಂಬುದನ್ನು ಶ್ರೀರಾಮಕೃಷ್ಣರು ತಾವಾಗಿಯೇ ಸ್ಪಷ್ಟಪಡಿಸುತ್ತಿದ್ದಾರೆ.

ಇನ್ನೊಂದು ಗಮನಾರ್ಹ ಅಂಶವಿದೆ ಇಲ್ಲಿ. ಶ್ರೀರಾಮಕೃಷ್ಣರು ಸಾಮಾನ್ಯ ಭಾವದಲ್ಲಿದ್ದಾಗ, ಇತರರು, ತಮ್ಮನ್ನು ‘ಗುರು’ ‘ಭಗವಾನ್’ ಇತ್ಯಾದಿ ವಿಶೇಷಣಗಳಿಂದ ಸಂಬೋಧಿಸುವುದನ್ನು ಸಹಿಸುತ್ತಿರಲಿಲ್ಲ. ತಾವೊಬ್ಬ ಅತಿ ಸಾಮಾನ್ಯ ವ್ಯಕ್ತಿಯೆಂಬಂತೆ ಅವರ ಪ್ರತಿಯೊಂದು ನಡೆ ನುಡಿಯೂ ಕೂಡ. ಆದ್ದರಿಂದ ಉನ್ನತ ದೈವೀಭಾವದಲ್ಲಿರುವಾಗಲಲ್ಲದೆ ಬೇರೆ ಸಮಯದಲ್ಲಿ ಅವರ ಬಾಯಿಂದ ಇಂತಹ ಅದ್ಭುತ ನುಡಿಗಳು ಹೊರಬರಲು ಸಾಧ್ಯವಿರಲಿಲ್ಲ. ಈಗಿನ ಅವರ ಸ್ಥಿತಿಯೋ, ಕರುಣಾಜನಕ! ಪ್ರಾಣೋತ­ಮಣ ಕಾಲ ಸನ್ನಿಹಿತವಾಗಿದೆ. ಶರೀರವೆಲ್ಲ ಸೊರಗಿ ಹೋಗಿದೆ. ಗಂಟಲಲ್ಲಿ ಅಪಾರ ವೇದನೆ. ಇನ್ನೆರಡೇ ದಿನಗಳಲ್ಲಿ ಅವರ ಶರೀರ ಒಬ್ಬ ಸಾಮಾನ್ಯ ಮರ್ತ್ಯನ ಶರೀರದಂತೆ ಭಸ್ಮವಾಗಲಿದೆ. ಇಂತಹ ಸ್ಥಿತಿಯಲ್ಲಿ ಅವರು ತಮ್ಮ ಅವತಾರತ್ವವನ್ನು ಘೋಷಿಸುತ್ತಿದ್ದಾರೆ! ಮತ್ತೂ ಕುತೂಹಲದ ಸಂಗತಿಯೆಂದರೆ, ಇತರರ ದೃಷ್ಟಿಗೆ ಅತ್ಯಂತ ಅಸಹಾಯಕ ಸ್ಥಿತಿಯಲ್ಲಿರುವಂತೆ ಕಾಣುತ್ತಿದ್ದ ಅವರು ಆ ಕ್ಷಣದಲ್ಲೂ ನರೇಂದ್ರನ ಮನದಲ್ಲಿ ಹಾದುಹೋದ ಭಾವನೆಯನ್ನು ಗ್ರಹಿಸಿ, ಒಡನೆಯೇ ಸಂದೇಹ ನಿವಾರಣೆ ಮಾಡಲು ಸಮರ್ಥರಾಗಿ ದ್ದುದು! ಶ್ರೀರಾಮಕೃಷ್ಣರ ನಿಜಸ್ವರೂಪವನ್ನು ಅರಿತುಕೊಳ್ಳಲು ಈ ಅಂಶಗಳು ನಮಗೆ ನೆರವಾಗುತ್ತವೆ.\footnote{*ನೋಡಿ: ಅನುಬಂಧ ೫.}

ಶ್ರೀರಾಮಕೃಷ್ಣರೀಗ ಶರೀರತ್ಯಾಗ ಮಾಡಲು ಸಿದ್ಧರಾಗಿದ್ದಾರೆ. ಅದಕ್ಕೆ ಸೂಕ್ತವಾದ ದಿನವನ್ನು ಅವರು ಮೊದಲೇ ಗೊತ್ತುಮಾಡಿಟ್ಟುಕೊಂಡಿದ್ದರು. ಆಗಸ್ಟ್ ತಿಂಗಳ ಆರಂಭದಲ್ಲೇ ಒಂದು ದಿನ ಶಿಷ್ಯನೊಬ್ಬನನ್ನು ಕರೆದು ಬಂಗಾಳಿ ಪಂಚಾಂಗವನ್ನು ಓದಲು ಹೇಳಿದ್ದರು. ಶಿಷ್ಯ ಪ್ರತಿದಿನದ ತಿಥಿ, ನಕ್ಷತ್ರ, ವಾರಗಳನ್ನು ಓದುತ್ತ ಬಂದ. ಶ್ರಾವಣಮಾಸದ ಕೊನೆಯ ದಿನದ, ಎಂದರೆ ಆಗಸ್ಟ್ ೧೪ನೇ ತಾರೀಖಿನ, ವಿವರಗಳನ್ನು ಹೇಳಿದಾಗ ಅವನಿಗೆ ಓದುವುದನ್ನು ನಿಲ್ಲಿಸಲು ಹೇಳಿದ್ದರು. ಈಗ ಆ ದಿನ ಸನ್ನಿಹಿತವಾಗಿದೆ.

ಸಂಜೆಯ ಹೊತ್ತಿಗೆ ಶ್ರೀರಾಮಕೃಷ್ಣರು ತಮಗೆ ಉಸಿರಾಡಲು ಕಷ್ಟವಾಗುತ್ತಿದೆ ಎಂದರು; ಬಳಿಕ ತಕ್ಷಣವೇ ಸಮಾಧಿಸ್ಥರಾಗಿಬಿಟ್ಟರು. ಆದರೆ ಈ ಸಲ ಅವರ ಸಮಾಧಿ ಎಂದಿನಂತಿರಲಿಲ್ಲ. ಇದನ್ನು ಕಂಡು ಎಲ್ಲರೂ ಆತಂಕಗೊಂಡರು. ಅವರನ್ನು ಎಚ್ಚರಗೊಳಿಸಲು ಎಲ್ಲರೂ ‘ಹರಿ ಓಂ ತತ್ ಸತ್’ ಮಂತ್ರವನ್ನು ಉಚ್ಚರಿಸುವಂತೆ ನರೇಂದ್ರ ಹೇಳಿದ. ಬಹಳ ಹೊತ್ತು ಎಲ್ಲರೂ ಅದನ್ನು ಘೋಷಿಸಿದರು. ಮಧ್ಯರಾತ್ರಿಯ ವೇಳೆಗೆ ಶ್ರೀರಾಮಕೃಷ್ಣರಿಗೆ ಬಾಹ್ಯಪ್ರಜ್ಞೆ ಮರಳಿತು. ‘ಹಸಿವು’ ಎಂದರು. ಸ್ವಲ್ಪ ರವೆಗಂಜಿಯನ್ನು ಕುಡಿಸಲಾಯಿತು. ಆಗ ಸ್ವಲ್ಪ ಸುಧಾರಿಸಿದಂತೆ ಕಂಡುಬಂದರು. ಅವರನ್ನು ಐದಾರು ದಿಂಬುಗಳ ಮೇಲೆ ಒರಗಿಸಿ ಕುಳ್ಳಿರಿಸಲಾಯಿತು. ಆ ಸ್ಥಿತಿಯಲ್ಲೇ ಅವರು ಪಿಸುದನಿಯಲ್ಲಿ ತಮ್ಮ ಕೊನೆಯ ಕ್ಷಣದವರೆಗೂ ಸೂಚನೆ-ಸಂದೇಶಗಳನ್ನು ನೀಡಿದರು. ಹಲವಾರು ನಿಮಿಷಗಳವರೆಗೆ ಇದು ನಡೆಯಿತು. ಬಳಿಕ ‘ಕಾಳಿ! ಕಾಳಿ! ಕಾಳಿ!’ ಎಂದು ಮೂರು ಸಲ ಉಚ್ಚರಿಸುತ್ತ ದಿಂಬಿನ ಮೇಲೆ ನಿಧಾನವಾಗಿ ಒರಗಿಕೊಂಡರು. ಆಗ ಸಮಯ ರಾತ್ರಿ ಒಂದು ಗಂಟೆ ಎರಡು ನಿಮಿಷ. ಇದ್ದಕ್ಕಿದ್ದಂತೆ ಅವರ ಶರೀರದಲ್ಲಿ ಒಂದು ಕಂಪನವುಂಟಾಯಿತು. ಕಣ್ಣುಗಳೆರಡೂ ನಾಸಿಕಾಗ್ರದಲ್ಲಿ ನೆಟ್ಟುವು. ರೋಮಗಳೆಲ್ಲ ನಿಮಿರಿ ನಿಂತುವು. ಮುಖದಲ್ಲಿ ದಿವ್ಯವಾದ ಮುಗುಳ್ನಗೆ ಬೆಳಗಿತು. ಅವರು ಮತ್ತೊಮ್ಮೆ ಗಾಢ ಸಮಾಧಿಗೇರಿದರು. ಅದು ಹಿಂದಿರುಗಿ ಬರಲು ಸಾಧ್ಯವಿಲ್ಲದಂತಹ ಮಹಾಸಮಾಧಿಯಾಗಿತ್ತು.

ಆದರೆ, ಅವರು ಎಂದಿನಂತೆಯೇ ಸಮಾಧಿಯಿಂದ ಇಳಿದು ಬರಬಹುದು ಎಂಬ ಆಶಾಪೂರ್ಣ ನಿರೀಕ್ಷೆ ಭಕ್ತರದು. ನರೇಂದ್ರ ಉಚ್ಚ ಕಂಠದಿಂದ ‘ಹರಿ ಓಂ ತತ್ ಸತ್’ ಎಂದು ಉಚ್ಚರಿಸ ಲಾರಂಭಿಸಿದ. ಇತರರೂ ಅವನ ಜೊತೆ ಸೇರಿಕೊಂಡರು. ಮಂತ್ರೋಚ್ಚಾರಣೆ ರಾತ್ರಿಯುದ್ದಕ್ಕೂ ನಡೆಯಿತು. ಆದರೆ ಶ್ರೀರಾಮಕೃಷ್ಣರಲ್ಲಿ ಮಾತ್ರ ಬಾಹ್ಯಪ್ರಜ್ಞೆ ಮರಳುವ ಸೂಚನೆಗಳು ಕಾಣದೆ ಭಕ್ತರ ಕಳವಳ ಹೆಚ್ಚಿತು. ಶ್ರೀರಾಮಕೃಷ್ಣರ ಶರೀರದಲ್ಲಿ ಬಿಸಿಯೇರುವಂತೆ ಮಾಡಲು ಅವರ ಬೆನ್ನಿಗೆ ಹಸುವಿನ ತುಪ್ಪವನ್ನು ತಿಕ್ಕಿದರು. ಯಾರೇನು ಮಾಡಿದರೂ ಪ್ರಯೋಜನವಾಗಲಿಲ್ಲ. ಬೆಳಗಾದ ಮೇಲೆ, ಸುದ್ದಿ ತಿಳಿದು ಬಂದ ಡಾ. ಸರ್ಕಾರ ಶರೀರವನ್ನು ಪರೀಕ್ಷೆ ಮಾಡಿ ನೋಡಿ, “ಶ್ರೀರಾಮಕೃಷ್ಣರು ಶರೀರವನ್ನು ಬಿಟ್ಟುಬಿಟ್ಟಿದ್ದಾರೆ” ಎಂದು ಪ್ರಕಟಿಸಿದ. ಮಹಾನ್ ಆಧ್ಯಾತ್ಮಿಕ ಜೀವನವೊಂದರ ಮೇಲೆ ಅಂಕದ ಪರದೆ ಇಳಿಬಿದ್ದಿತ್ತು. ಅನಂತವಾದ ಅಮರ ಆತ್ಮ ಇಲ್ಲಿಯ ವರೆಗೆ ಯಾವ ನಾಮ-ರೂಪಗಳ ಕೋಶದೊಳಗೆ ನಿಬದ್ಧವಾಗಿತ್ತೋ, ಅದೀಗ ಸಕಲ ಬಂಧನ ಗಳನ್ನೂ ಭೇದಿಸಿಕೊಂಡು ಅನಂತಾತ್ಮದಲ್ಲಿ ಲಯವಾಗಿಬಿಟ್ಟಿತ್ತು. ದೇಶಕಾಲಗಳೆಂಬ ಕಟ್ಟುಗಳು ಕತ್ತರಿಸಿ ಹೋದುವು. ಇಲ್ಲಿಯವರೆಗೆ ಬಂಗಾಳ ರಾಜ್ಯಕ್ಕೆ ಮಾತ್ರ ಸೇರಿದ್ದ ಅವರು ಈಗ ಎಲ್ಲೆಡೆಗೂ ಸೇರಿದವರಾದರು. ಇಲ್ಲಿಯವರೆಗೆ ಅವರು ಕೇವಲ ಹತ್ತೊಂಬತ್ತನೆಯ ಶತಮಾನಕ್ಕೆ ಸೇರಿದವರಾಗಿದ್ದರು; ಈಗ ಎಲ್ಲ ಶತಮಾನಗಳಿಗೂ ಸೇರಿದವರಾಗಿಬಿಟ್ಟರು. ಅವರು ಶರೀರ ದಲ್ಲಿರುವವರೆಗೂ ಕಲ್ಕತ್ತದ ಕೆಲವರಿಗೆ ಮಾತ್ರ ಮಾರ್ಗದರ್ಶಕರಾಗಿದ್ದರು; ಈಗ ಸರ್ವರಿಗೂ ಬೆಳಕು ತೋರುವ ವಿಶ್ವಜ್ಯೋತಿಯಾದರು. ಆ ಭಕ್ತ-ಶಿಷ್ಯರಿಗೆ ಇವೆಲ್ಲ ಗೊತ್ತಿಲ್ಲವೆಂದಲ್ಲ. ಆದರೆ ಉಕ್ಕಿಬರುವ ದುಃಖವನ್ನು ಹತ್ತಿಕ್ಕಲಾದೀತೆ? ಇಲ್ಲಿಯವರೆಗೆ ಅವರನ್ನೆಲ್ಲ ಬಂಧಿಸಿಟ್ಟಿದ್ದ ಆ ಮನಮೋಹಕ ಮಂದಹಾಸ ಮರೆಯಾಗಿದೆ; ಅವರಿಗೆಲ್ಲ ಸ್ಫೂರ್ತಿಯ ಸೆಲೆಯಾಗಿದ್ದ ಆ ದಿವ್ಯ ಮುಖಮಂಡಲ ಇನ್ನಿಲ್ಲ. ಅವರೊಂದಿಗೆ ಅತ್ಯುನ್ನತ ಜ್ಞಾನ ಭಕ್ತಿ ಪ್ರೇಮಗಳ ಮಧುರನುಡಿಗಳ ನ್ನಾಡುತ್ತಿದ್ದ ಆ ತುಟಿಗಳಿಗೆ ಮೃತ್ಯುವಿನ ಬೀಗಮುದ್ರೆ ಬಿದ್ದಿದೆ.

ಶ್ರೀಶಾರದಾದೇವಿಯವರನ್ನು ಮೆಲ್ಲನೆ ಮಹಡಿಯ ಮೇಲಕ್ಕೆ ಕರೆತರಲಾಯಿತು. ಪತಿಯ ನಿಶ್ಚೇಷ್ಟಿತ ಶರೀರವನ್ನು ಕಂಡು ಅವರು, “ಅಮ್ಮ ಕಾಳಿ! ಎಲ್ಲಿ ಹೋದೆ ನೀನು?” ಎಂದು ಎದೆಬಿರಿದು ಕೂಗಿದರು. ಈ ಹೃದಯ ವಿದ್ರಾವಕ ದೃಶ್ಯ ಅಲ್ಲಿದ್ದವರೆಲ್ಲರ ಅಂತಃಕರಣವನ್ನು ಕಲುಕಿತು. ಆದರೆ ಶಾರದಾದೇವಿಯವರು ಬೇಗನೆ ತಮ್ಮನ್ನು ನಿಯಂತ್ರಿಸಿಕೊಂಡು ಶಾಂತಚಿತ್ತ ರಾದರು. ಸಂಜೆಯ ಹೊತ್ತಿಗೆ ಅವರು ಹಿಂದೂ ಸಂಪ್ರದಾಯದಂತೆ ತಮ್ಮ ಕಂಕುಮವನ್ನು ಅಳಿಸಿಕೊಂಡು, ಸೀರೆಯ ಕೆಂಪು ಅಂಚನ್ನು ಹರಿದು, ಬಳೆಗಳನ್ನೂ ತೆಗೆಯಲು ಹೊರಟರು. ಆಗ ಶ್ರೀರಾಮಕೃಷ್ಣರು ಅವರೆದುರಿಗೆ ಪ್ರಕಟವಾಗಿ “ನಾನೆಲ್ಲಿಗೆ ಹೋದೆ ಅಂತ ತಿಳಿದುಕೊಂಡೆ? ನಾನಿಲ್ಲೇ ಇದ್ದೇನೆ. ಒಂದು ಕೋಣೆಯಿಂದ ಇನ್ನೊಂದು ಕೋಣೆಗೆ ಬಂದಿದ್ದೇನೆ ಅಷ್ಟೇ” ಎಂದರು. ಈ ಸ್ಪಷ್ಟದರ್ಶನಾನುಭವದಿಂದ ಶಾರದಾದೇವಿಯವರು ಅತ್ಯಾಶ್ಚರ್ಯಗೊಂಡರು. ಅಂದಿಗೆ ಅವರು ಬಳೆಗಳನ್ನು ತೆಗೆಯಲಿಲ್ಲ. ಆದರೆ ತಮಗಾದ ಈ ಅನುಭವ ತಮ್ಮ ಮನಸ್ಸಿನ ಭ್ರಮೆಯಿರಬಹುದು ಎಂದು ಭಾವಿಸಿ ಮತ್ತೆ ಎರಡು ಸಲ ಅವುಗಳನ್ನು ತೆಗೆದಿಡುವ ಪ್ರಯತ್ನ ಮಾಡಿದರು. ಆದರೆ ಎರಡು ಸಲವೂ ಮತ್ತೆ ಶ್ರೀರಾಮಕೃಷ್ಣರು ಪ್ರಕಟರಾಗಿ ಅವರನ್ನು ತಡೆದರು. ಕಡೆಗೆ ಶಾರದಾದೇವಿಯವರು ತಮ್ಮ ಆ ಪ್ರಯತ್ನವನ್ನು ಕೈಬಿಡಬೇಕಾಯಿತು. ಅನಂತರ ಅವರು ತಮ್ಮ ಉಳಿದ ಜೀವಿತಾವಧಿಯಿಡೀ ಬಳೆಗಳನ್ನು ಧರಿಸುತ್ತಿದ್ದರು ಮತ್ತು ಅಂಚಿನ ಸೀರೆಯನ್ನೇ ಉಡುತ್ತಿದ್ದರು. 

ಡಾ. ಸರ್ಕಾರನು ಬಂದು ತನ್ನ ನಿರ್ಧಾರವನ್ನು ಹೇಳುವ ಮೊದಲೇ ಶ್ರೀರಾಮಕೃಷ್ಣರ ನಿರ್ಯಾಣದ ವಾರ್ತೆ ಎಲ್ಲೆಡೆ ಹರಡಿತ್ತು. ಭಕ್ತರೆಲ್ಲರೂ ಕಾಶೀಪುರದ ಉದ್ಯಾನಗೃಹಕ್ಕೆ ಬಂದು ಸೇರಿದರು. ಶ್ರೀರಾಮಕೃಷ್ಣರ ಶರೀರವನ್ನು ಹೂಹಾರಗಳಿಂದ ಅಲಂಕರಿಸಿದರು, ಭಜನೆ ಮಾಡಿ ದರು. ಸಂಜೆ ಸುಮಾರು ಐದು ಗಂಟೆಯ ಹೊತ್ತಿಗೆ ಮೆರವಣಿಗೆ ಹೊರಟಿತು. ಶ್ಮಶಾನಘಟ್ಟದಲ್ಲಿ ಶ್ರೀರಾಮಕೃಷ್ಣರ ಪವಿತ್ರ ಶರೀರಕ್ಕೆ ಗಂಗಾಜಲದಿಂದ ಸ್ನಾನ ಮಾಡಿಸಿ, ಕಾಷಾಯವಸ್ತ್ರವನ್ನು ತೊಡಿಸಲಾಯಿತು. ಮತ್ತೊಮ್ಮೆ ಹೂಹಾರಗಳಿಂದ ಅಲಂಕರಿಸಿ ಪೂಜಿಸಲಾಯಿತು. ಅನಂತರ ಶ್ರೀರಾಮಕೃಷ್ಣರ ಆ ಸುಕೋಮಲ ಶರೀರವನ್ನು ನಿಧಾನವಾಗಿ ಚಿತೆಯ ಮೇಲಿರಿಸಲಾಯಿತು. ಚಿತೆಗೆ ಅಗ್ನಿಸ್ಪರ್ಶ ಮಾಡಿದರು. ತುಪ್ಪ, ಗಂಧ, ಧೂಪಗಳನ್ನು ಅಗ್ನಿಗೆ ಸುರಿದರು. ಒಂದೆರಡು ಗಂಟೆಗಳಲ್ಲಿ ಎಲ್ಲ ಮುಗಿದುಹೋಯಿತು.

ಅಸ್ಥಿ-ಭಸ್ಮಗಳನ್ನು ತಾಮ್ರದ ಪಾತ್ರೆಯೊಂದರಲ್ಲಿ ಶೇಖರಿಸಿ ಕಾಶೀಪುರದ ಉದ್ಯಾನಗೃಹದ ಲ್ಲಿರಿಸಲಾಯಿತು. ಬೇರೊಂದು ಸ್ಥಳಾಂತರಗೊಳಿಸುವವರೆಗೂ ಪ್ರತಿದಿನ ಅದನ್ನು ಅಲ್ಲೇ ಪೂಜಿಸುತ್ತಿದ್ದರು. 

ಕ್ರಮೇಣ ಶಿಷ್ಯರ ಹಾಗೂ ಭಕ್ತರ ಹೃದಯದಲ್ಲಿ ಒಂದು ಬಗೆಯ ಶಾಂತ-ಶರಣಾಗತಿ ಭಾವ ತಾನೇತಾನಾಗಿ ಆವರಿಸಿತು; ಶ್ರೀರಾಮಕೃಷ್ಣರ ಮಧುರಸ್ಮೃತಿ ಅವರ ಮನಸ್ಸನ್ನು ತುಂಬಿಬಿಟ್ಟಿತು. ಇದರ ಪರಿಣಾಮವಾಗಿ ಅವರ ಶೋಕ-ದುಃಖಗಳು ಶಾಂತಿ-ಆನಂದಗಳಾಗಿ ಪರಿವರ್ತನೆಗೊಂಡುವು.

