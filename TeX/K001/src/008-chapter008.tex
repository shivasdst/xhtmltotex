
\chapter{ಮರಿಸಿಂಹ–ಮಹಾಸಿಂಹ}

\noindent

ಜಗತ್ತಿನ ಜನರು ಅಜ್ಞಾನದಿಂದಾಗಿ ಯಾತನೆಗಳಿಗೀಡಾಗಿ ಕಂಗಾಲಾಗುವುದನ್ನು ಕಂಡ ಭಗ ವಂತ, ಕರುಣೆಯಿಂದ ಕರಗಿ ಆಯಾ ಕಾಲಕ್ಕೆ ಆವಶ್ಯಕವಾದ ಸಂದೇಶವನ್ನೀಯಲು ಧರೆಗಿಳಿದು ಬರುತ್ತಾನೆ. ಅವನು ತನ್ನದೇ ಸೃಷ್ಟಿಯಾದ ಮಾಯೆಯ ಮುಸುಕು ಹಾಕಿಕೊಂಡು ಮೈದಳೆದು ಬರುತ್ತಾನೆ. ಆದರೆ ಅವನಿಗೆ ತನ್ನ ಸ್ವಸ್ವರೂಪದ ಅರಿವು ಜನ್ಮದಾರಭ್ಯದಿಂದಲೇ ಇರುತ್ತದೆ. ತಾನು ಈ ಜಗತ್ತಿಗೆ ಅವತರಿಸಿಬಂದ ಉದ್ದೇಶದ ಅರಿವು ಅವನಿಗಿರುತ್ತದೆ. ಕೆಲವೇ ವರ್ಷಗಳ ತೀವ್ರ ಆಧ್ಯಾತ್ಮಿಕ ಸಾಧನೆಯ ಮೂಲಕ ತನ್ನ ಮಾಯೆಯ ತೆಳುವಾದ ಪರದೆಯನ್ನು ಹರಿದು ಆತ ಸಚ್ಚಿದಾನಂದ ಸ್ವರೂಪದಿಂದ ಬೆಳಗುತ್ತಾನೆ. ಆದರೆ, ಸಾಮಾನ್ಯ ಜೀವಿಯೊಬ್ಬನು ತನ್ನ ಮಾಯೆಯ ಪರದೆಯನ್ನು ಹರಿಯಬೇಕಾದರೆ ಅಸಂಖ್ಯಾತ ಜನ್ಮಗಳೇ ಬೇಕಾಗಬಹುದು. ಅವತಾರಪುರುಷರಿಗೂ ಸಾಮಾನ್ಯರಿಗೂ ಇದೊಂದು ಪ್ರಮುಖ ವ್ಯತ್ಯಾಸ. ಅವತಾರಪುರುಷನು ಸಕಾಲದಲ್ಲಿ ತನ್ನ ಮಾಯೆಯ ಪರದೆಯನ್ನು ಹರಿದು, ಪ್ರಬಲ ಆಧ್ಯಾತ್ಮಿಕ ಶಕ್ತಿಕೇಂದ್ರವಾಗಿ ವಿರಾಜಿಸುತ್ತಾನೆ. ಆಗ ಆ ಶಕ್ತಿಯ ತೇಜಸ್ಸನ್ನು ಮುಸುಕಬಲ್ಲ ಶಕ್ತಿ ಇನ್ನೊಂದಿಲ್ಲ. ಅವನು ತನ್ನ ಪ್ರಖರ ತೇಜಸ್ಸಿನಿಂದ ಪ್ರಪಂಚದ ವಿಚಾರಧಾರೆಯನ್ನೇ ಬದಲಿಸಿಬಿಡುತ್ತಾನೆ; ಜಗತ್ತಿನಲ್ಲೊಂದು ಕ್ರಾಂತಿಯನ್ನೇ ಮಾಡಿಬಿಡುತ್ತಾನೆ; ತನ್ನ ಸಾನ್ನಿಧ್ಯದಿಂದಲೇ ಆಧ್ಯಾತ್ಮಿಕತೆಯನ್ನು ಪಸರಿಸುತ್ತಾನೆ. ಅವನ ಸ್ಪರ್ಶಮಾತ್ರದಿಂದಲೇ ಪವಾಡಗಳು ನಡೆಯಬಹುದು. ಆತನ ವೀಕ್ಷಣಮಾತ್ರದಿಂದಲೇ ಅದ್ಭುತಗಳು ಘಟಿಸಬಹುದು. ಆದರೆ ಅವತಾರಪುರುಷನ ಶರೀರವು ಅತ್ಯಂತ ಸಾತ್ವಿಕವಾದ್ದರಿಂದ ಅವನಿಂದ ಹೆಚ್ಚಿನ ಶರೀರಶ್ರಮ ಸಾಧ್ಯವಿಲ್ಲ. ಆದ್ದರಿಂದ ಅವನು ತಾನೇ ಸ್ವತಃ ಓಡಾಡಿ ಧರ್ಮವನ್ನು ಬಿತ್ತರಿಸಲಾರ. ಆ ಕಾರ್ಯಕ್ಕೆ ಸಮೃದ್ಧ ರಜೋಗುಣದಿಂದ ಕೂಡಿದ ವ್ಯಕ್ತಿ ಯೊಬ್ಬನು ಬೇಕಾಗುತ್ತಾನೆ. ಇಲ್ಲಿ ಇನ್ನೊಂದು ಸ್ವಾರಸ್ಯಕರ ವಿಷಯವಿದೆ: ಅದೇನೆಂದರೆ, ಭಗವಂತ ಅವತಾರ ತಾಳುವಾಗ ತನ್ನ ದಿವ್ಯ ಸಂದೇಶಗಳನ್ನು ಸಾರಲು ಅಥವಾ ತನ್ನ ಕಾರ್ಯವನ್ನು ನಿರ್ವಹಿಲು ತನ್ನದೇ ಆದ ಇನ್ನೊಂದು ಅಂಶವೂ ಧರೆಯಲ್ಲಿ ಮೈದಳೆಯುವಂತೆ ಮಾಡುತ್ತಾನೆ. ಮತ್ತು ಸಕಾಲದಲ್ಲಿ ಆ ವ್ಯಕ್ತಿಯನ್ನು ತನ್ನ ಬಳಿಗೆ ಆಕರ್ಷಿಸಿಕೊಂಡು ತನ್ನ ಧರ್ಮಸಂಸ್ಥಾಪನೆಯ ಕಾರ್ಯವನ್ನು ಅವನ ಮೂಲಕ ಮಾಡಿಸುತ್ತಾನೆ. ಭಗವಾನ್ ಕ್ರಿಸ್ತನಿಗೆ ಹಲವಾರು ಶಿಷ್ಯರು-ಭಕ್ತರು ಇದ್ದರಾದರೂ ಅವನು ತನ್ನ ಧರ್ಮಸಂಸ್ಥೆಯನ್ನು ಸ್ಥಾಪಿಸಿದ್ದು ಪೀಟರ್ ಎಂಬ ತನ್ನ ಶಿಷ್ಯ ನೊಬ್ಬನ ಮೂಲಕ. ಶ್ರೀಕೃಷ್ಣಪರಮಾತ್ಮನು ತನ್ನ ದುಷ್ಟಶಿಕ್ಷಣ-ಶಿಷ್ಟರಕ್ಷಣ ರೂಪವಾದ ಧರ್ಮ ಸಂಸ್ಥಾಪನಾ ಕಾರ್ಯವನ್ನು ನಡೆಸಿದ್ದು ಅರ್ಜುನನ ಮೂಲಕ. ಭಗವಾನ್ ಬುದ್ಧನು ತನ್ನ ಕಾರ್ಯವನ್ನು ನಡೆಸಿದ್ದು ಆನಂದನ ಮೂಲಕ. ಹಾಗೆಯೇ ಶ್ರೀಕೃಷ್ಣಚೈತನ್ಯನ ನೆರವಿಗೊದಗಿ ದವನು ಆತನ ಶಿಷ್ಯ-ಸಖನಾದ ನಿತ್ಯಾನಂದ. ಈಗ ಶ್ರೀರಾಮಕೃಷ್ಣರ ಧರ್ಮ ಸಂಸ್ಥಾಪನಾ ಕಾರ್ಯಕ್ಕೆ ನೆರವಾಗವುದಕ್ಕಾಗಿಯೇ ಆಗಮಿಸಿದ್ದಾನೆ, ನರೇಂದ್ರನಾಥ–ಮುಂದೆ ಸ್ವಾಮಿ ವಿವೇಕಾ ನಂದ.

ಶ್ರೀರಾಮಕೃಷ್ಣರು ಮೊದಲ ಸಲ ನರೇಂದ್ರನನ್ನು ನೋಡಿದಾಗಲೇ ಗುರುತಿಸಿಬಿಟ್ಟರು, ತಮ್ಮ ದಿವ್ಯ ಸಂದೇಶಗಳನ್ನು ಹರಡಬೇಕಾದವನು ಇವನೇ ಎಂದು. ಅವರು ತಮ್ಮ ನಿರ್ವಿಕಲ್ಪ ಸಮಾಧಿಯ ಮೂಲಕ ಭೂತ-ವರ್ತಮಾನ-ಭವಿಷ್ಯತ್ಕಾಲಗಳ ಎಲ್ಲ ವಿಷಯಗಳನ್ನು, ವಿದ್ಯಮಾನ ಗಳನ್ನು ತಿಳಿದುಕೊಳ್ಳಬಲ್ಲವರಾಗಿದ್ದರು. ಹಾಗೆಯೇ, ತಮ್ಮ ಬಳಿಗೆ ಬಂದ ಭಕ್ತರಲ್ಲಿ ಹಾಗೂ ಶಿಷ್ಯರಲ್ಲಿ ಯಾರ್ಯಾರಿಂದ ಏನೇನು ಆಗಬೇಕಾಗಿದೆ ಎಂಬುದನ್ನೂ ತಿಳಿದೇ ಇದ್ದರು. ಶ್ರೀರಾಮ ಕೃಷ್ಣರು ಆಧ್ಯಾತ್ಮಿಕ ಜ್ಞಾನದ ಹಾಗೂ ತಪೋಮಯ ಪುಷಿಪರಂಪರೆಯ ಭಾರತವನ್ನು ಪ್ರತಿನಿಧಿಸಿ ದರೆ, ನರೇಂದ್ರನು ಎಲ್ಲ ಬಗೆಯ ಸಂಶಯ-ಅಪನಂಬಿಕೆಗಳಿಂದ ಹಾಗೂ ನಾಸ್ತಿಕತೆಯಿಂದ ಕೂಡಿದ ಆಧುನಿಕ ಯುಗದ ಪ್ರತಿನಿಧಿಯಂತಿದ್ದ. ಸತ್ಯಸಾಕ್ಷಾತ್ಕಾರ ಮಾಡಿಕೊಳ್ಳುವ ಪ್ರಾಮಾಣಿಕ ಹಂಬಲ ಅವನಿಗಿತ್ತು. ಎಂತಹ ಘನವಾದ ಆಧ್ಯಾತ್ಮಿಕ ಸತ್ಯಗಳನ್ನೇ ಆಗಲಿ, ಚೆನ್ನಾಗಿ ಪರೀಕ್ಷೆ ಮಾಡಿ ರುಜುವಾತುಪಡಿಸಿಕೊಳ್ಳದೆ ಅವನ್ನು ಸ್ವೀಕರಿಸುತ್ತಿರಲಿಲ್ಲ. ಭೌತಿಕ ಜಗತ್ತಿನಲ್ಲಿ ಜ್ಞಾನಾರ್ ಜನೆಗೆ ವಿಚಾರವಂತಿಕೆ ಎನ್ನುವುದು ಅತ್ಯುಪಯುಕ್ತ ಸಾಧನವೇ ನಿಜ; ಆದರೆ ಭೌತಿಕ ಜಗತ್ತಿ ನಿಂದಾಚೆ ಅದು ಕೆಲಸಕ್ಕೆ ಬಾರದು, ಪಾರಮಾರ್ಥಿಕ ಪ್ರಪಂಚವನ್ನು ಪ್ರವೇಶಿಸಿ ಭಗವಂತನ ಅನಂತತೆಯನ್ನು ಅದು ಅಳೆಯಲಾರದು ಎನ್ನುವುದನ್ನು ನರೇಂದ್ರ ತಿಳಿದುಕೊಳ್ಳಬೇಕಾಗಿದೆ. ಶ್ರೀರಾಮಕೃಷ್ಣರೆಂಬ ಈ ಮಹಾಶಕ್ತಿಯೊಂದಿಗಿನ ಸಮಾಗಮದಿಂದ ಅವನು ನವಭಾರತದ ಹೃದಯ-ಬುದ್ಧಿಗಳ ಕೇಂದ್ರವಾಗಿ ನಿರ್ಮಾಣಗೊಳ್ಳಲಿದ್ದಾನೆ. ಈ ಎರಡು ಪ್ರಾಚೀನ-ಅರ್ವಾಚೀನ ಬೃಹತ್ ಚೈತನ್ಯಗಳ ಸಮ್ಮೇಳನದಿಂದ, ಜಗತ್ತಿಗೇ ಆದರ್ಶಪ್ರಾಯವಾಗಬಲ್ಲ ನೂತನ ಹಿಂದೂ ಧರ್ಮದ ಜನನವಾಗಲಿದೆ.

ನರೇಂದ್ರ ತಮ್ಮ ಬಳಿಗೆ ಬಂದ ಸಂದರ್ಭವನ್ನು ಶ್ರೀರಾಮಕೃಷ್ಣರೇ ತುಂಬ ಸ್ವಾರಸ್ಯಕರವಾಗಿ ಬಣ್ಣಿಸುತ್ತಾರೆ:

“ಗಂಗೆಗೆ ಇದಿರಾಗಿರುವ ಪಶ್ಚಿಮದ ದ್ವಾರದಿಂದ ನರೇಂದ್ರ ಕೋಣೆಯನ್ನು ಪ್ರವೇಶಿಸಿದ. ತನ್ನ ಜೊತೆಯಲ್ಲಿ ಬಂದಿದ್ದ ಇತರರಂತಲ್ಲದೆ ಅವನು ತನ್ನ ಶರೀರದ ಬಗ್ಗೆ, ಬಟ್ಟೆಬರೆಯ ಬಗ್ಗೆ ಅಲಕ್ಷ್ಯದಿಂದಿರುವಂತೆ ತೋರಿತು. ಬಾಹ್ಯಜಗತ್ತಿನ ಕಡೆಗೆ ಅವನಿಗೆ ಗಮನವೇ ಇರಲಿಲ್ಲ. ಆತನ ಮನಸ್ಸು ಸದಾ ಅಂತರ್ಮುಖಿಯಾಗಿರುತ್ತಿದ್ದುದನ್ನು ಕಣ್ಣುಗಳೇ ಹೇಳುತ್ತಿದ್ದುವು... ‘ಕಲ್ಕತ್ತ ದಂತಹ ಭೋಗಭರಿತ ನಗರದಿಂದ ಇಂತಹ ಅಧ್ಯಾತ್ಮಶೀಲ ವ್ಯಕ್ತಿಯೊಬ್ಬ ಬರುತ್ತಿದ್ದಾನಲ್ಲ!’ ಎಂದು ಆಶ್ಚರ್ಯಪಟ್ಟೆ... ನನ್ನ ಅಪೇಕ್ಷೆಯಂತೆ ಅವನೊಂದು ಬಂಗಾಳೀ ಹಾಡು ಹೇಳಿದ: ‘ಓ ಮನಸ್ಸೇ! ನಡೆ, ನಾವು ನಮ್ಮ ಮನೆಗೆ ಹೋಗೋಣ. ಈ ಜಗತ್ತಿನಲ್ಲಿ ನಾವೇಕೆ ಪರದೇಸಿ ಗಳಂತೆ ಅಲೆಯಬೇಕು?’ ಎಂಬರ್ಥದ ಹಾಡು ಅದು. ಪರಮಾತ್ಮನೊಡನೆ ಮಿಲನಗೊಳ್ಳಲು ಕಾತರಿಸುತ್ತಿರುವ ಹೃದಯವೊಂದರಿಂದ ತಾನಾಗಿಯೇ ಉಕ್ಕಿಬಂದಂತಿತ್ತು ಆ ಕರೆ. ಅದನ್ನು ಕೇಳಿ ನನಗೆ ತಡೆದುಕೊಳ್ಳಲಾಗಲಿಲ್ಲ, ಸಮಾಧಿಸ್ಥನಾಗಿಬಿಟ್ಟೆ... ”

ಮುಂದೆ ನಡೆದ ವಿಚಿತ್ರವನ್ನು ನರೇಂದ್ರನ ಬಾಯಿಂದ ಕೇಳೋಣ:

“ಸರಿ, ನಾನು ಹಾಡಿದೆ. ಬಳಿಕ ಅವರು ಇದ್ದಕ್ಕಿದ್ದಂತೆ ಎದ್ದರು. ನನ್ನ ಕೈಹಿಡಿದು ಪಕ್ಕದ ವರಾಂಡಕ್ಕೆ ಕರೆದೊಯ್ದು, ಕೋಣೆಯ ಬಾಗಿಲು ಮುಚ್ಚಿಬಿಟ್ಟರು. ವರಾಂಡದ ಇನ್ನೊಂದು ಪಕ್ಕಕ್ಕೆ ತಡಿಕೆ ಹಾಕಿತ್ತು. ಈಗ ಅಲ್ಲಿದ್ದುದು ನಾವಿಬ್ಬರೇ. ಇವರೇನೋ ನನಗೆ ವೈಯಕ್ತಿಕವಾದ ಬೋಧನೆ- ಗೀದನೆ ಕೊಟ್ಟಾರು ಅಂದುಕೊಂಡೆ. ಆದರೆ ಅಲ್ಲಿ ಅವರ ವರ್ತನೆಯನ್ನು ಕಂಡು ನನಗೆ ಕಕ್ಕಾಬಿಕ್ಕಿಯಾಯಿತು. ಅವರು ನನ್ನ ಕೈಗಳನ್ನು ಹಿಡಿದು ಧಾರಾಕಾರವಾಗಿ ಆನಂದದ ಕಣ್ಣೀರು ಹರಿಸತೊಡಗಿದರು! ನಾನವರಿಗೆ ಬಹುಕಾಲದ ಆತ್ಮೀಯನೋ ಎಂಬಂತೆ ಪರಮಾದರದ ದನಿ ಯಲ್ಲಿ ಮಾತನಾಡಿಸಿದರು. ‘ಯಾಕಪ್ಪಾ ಇಷ್ಟು ತಡವಾಗಿ ಬಂದೆ? ನನ್ನನ್ನು ಇಷ್ಟುಕಾಲ ಸತಾಯಿಸುವಷ್ಟು ನಿಷ್ಕರುಣಿಯಾದೆಯಾ ನೀನು? ಈ ಪ್ರಾಪಂಚಿಕ ಜನರ ಗಳಹುವಿಕೆಯನ್ನು ಕೇಳಿಕೇಳಿ ನನ್ನ ಕಿವಿಯೆಲ್ಲ ಸೀದು ಹೋದಂತಾಗಿದೆ. ಅಯ್ಯೋ, ನನ್ನನ್ನು ಅರ್ಥಮಾಡಿಕೊಳ್ಳ ಬಲ್ಲವರ ಮುಂದೆ ನನ್ನ ಅಂತರಂಗದ ಅನುಭವಗಳನ್ನೆಲ್ಲ ತೋಡಿಕೊಂಡು ಎದೆಯನ್ನು ಹಗುರ ಮಾಡಿಕೊಳ್ಳೋಣ ಅಂತ ಎಷ್ಟು ಒದ್ದಾಡುತ್ತಿದ್ದೇನೆ’ ಎಂದು ಹೀಗೇ ಏನೇನೋ ಹೇಳುತ್ತ ಬಿಕ್ಕಿದರು.

“ಆಮೇಲೆ, ಇದ್ದಕ್ಕಿದ್ದಂತೆ ನನ್ನೆದರು ಕೈಜೋಡಿಸಿ ನಿಂತು ಹೇಳತೊಡಗಿದರು: ‘ಪ್ರಭು, ನಾ ಬಲ್ಲೆ! ನೀನು ಆ ಪುರಾತನ ಮಹರ್ಷಿ, ನಾರಾಯಣನ ಅವತಾರನಾದ ನರಪುಷಿ; ಮಾನವ ಕೋಟಿಯ ದುಃಖನಿವಾರಣೆ ಮಾಡಲು ನೀನು ಇಳೆಗೆ ಅವತರಿಸಿ ಬಂದಿರುವೆಯೆಂದು ನಾಬಲ್ಲೆ!’ ನನಗೆ ದಿಗ್ಭ್ರಾಂತಿಯಾಯಿತು. ‘ಇದ್ಯಾವ ವಿಚಿತ್ರ ಮನುಷ್ಯನನ್ನು ನೋಡಲು ಬಂದೆನಪ್ಪ! ಇವನೆಲ್ಲೋ ಪೂರ್ತಿ ಹುಚ್ಚನಿರಬೇಕು. ನನ್ನನ್ನು ನರಪುಷಿ, ಅವತಾರ ಅಂತ ಏನೇನೋ ಕರೆಯುತ್ತಿದ್ದಾನಲ್ಲ!’ ಎಂದುಕೊಂಡೆ. ಆದರೂ ನಾನು ಸುಮ್ಮನಿದ್ದುಬಿಟ್ಟೆ–ಹೇಳಿಕೊಂಡು ಹೋಗಲಿ ಎಂದು. ಅಷ್ಟರಲ್ಲಿ ಅವರು ಪುನಃ ಕೋಣೆಗೆ ಹೋಗಿ ಒಂದಿಷ್ಟು ಮಿಠಾಯಿ, ಕಲ್ಲುಸಕ್ಕರೆ, ಬೆಣ್ಣೆ, ಇವನ್ನೆಲ್ಲ ತಂದು ತಮ್ಮ ಕೈಯಾರೆ ನನಗೆ ತಿನ್ನಿಸಲು ಮುಂದಾದರು. ‘ಅಯ್ಯೋ, ಬೇಡಬೇಡ! ಅದನ್ನು ನನ್ನ ಕೈಗೆ ಕೊಡಿ. ಗೆಳೆಯರೊಂದಿಗೆ ಹಂಚಿಕೊಂಡು ತಿನ್ನುತ್ತೇನೆ’ ಎಂದು ನಾನು ಮತ್ತೆಮತ್ತೆ ಹೇಳಿದರೂ ಕೇಳದೆ, ‘ಅವರಿಗೆಲ್ಲ ಆಮೇಲೆ ಕೊಟ್ಟ ರಾಯಿತು’ ಎನ್ನುತ್ತ ಬಲವಂತದಿಂದ ಅಷ್ಟನ್ನೂ ತಿನ್ನಿಸಿಯೇಬಿಟ್ಟರು! ಬಳಿಕ ಮತ್ತೆ ನನ್ನ ಕೈಗಳನ್ನು ಬಿಗಿಯಾಗಿ ಹಿಡಿದುಕೊಂಡು, ‘ಎಲ್ಲಿ, ನಾನೊಬ್ಬನೇ ಇಲ್ಲಿಗೆ ಬೇಗ ಇನ್ನೊಮ್ಮೆ ಬರುತ್ತೇನೆ ಅಂತ ಮಾತುಕೊಡು!’ ಎಂದು ಒತ್ತಾಯಪಡಿಸಿದರು. ಅವರ ಒತ್ತಾಯಕ್ಕೆ ಮಣಿದು ನಾನು ‘ಸರಿ, ಬರುತ್ತೇನೆ’ಎನ್ನಬೇಕಾಯಿತು.”

ಬಳಿಕ ಇಬ್ಬರೂ ಕೋಣೆಗೆ ಹಿಂದಿರುಗಿದರು. ಈಗಾಗಲೇ ಅಲ್ಲಿದ್ದ ಭಕ್ತರೆಲ್ಲ ಯೋಚಿಸುತ್ತಿರ ಬಹುದು–‘ಇದೇನು, ಶ್ರೀರಾಮಕೃಷ್ಣರು ನಮ್ಮ ನರೇಂದ್ರನ ಮೇಲೆ ಇಷ್ಟೊಂದು ವಿಶ್ವಾಸ ತೋರುತ್ತಿದ್ದಾರಲ್ಲ; ಅವನಲ್ಲಿ ಅವರು ಅಂಥದೇನು ಕಂಡಿರಬಹುದು?’ ಎಂದು. ಈಗ ಶ್ರೀರಾಮಕೃಷ್ಣರು ಮತ್ತೆ ತಮ್ಮ ಸಣ್ಣ ಮಂಚದ ಮೇಲೆ ಕುಳಿತರು. ತಮ್ಮೆದುರು ಕುಳಿತಿದ್ದ ಭಕ್ತರಿಗೆ ನರೇಂದ್ರನನ್ನು ತೋರಿಸುತ್ತ ಉದ್ಗರಿಸಿದರು: “ನೋಡಿ ಇಲ್ಲಿ, ಇವನ ಮುಖದಲ್ಲಿ ಹೇಗೆ ಸರಸ್ವತಿ ಬೆಳಗುತ್ತಿದ್ದಾಳೆ!” ಹೊಸಬನಾದ ಒಬ್ಬ ತರುಣನ ಬಗ್ಗೆ ಶ್ರೀರಾಮಕೃಷ್ಣರು ಇಂಥ ಅದ್ಭುತವಾದ ಮಾತು ಹೇಳಿದ್ದನ್ನು ಕೇಳಿ ಭಕ್ತರೆಲ್ಲ ಅವನನ್ನೇ ಬೆರಗುಗಣ್ಣಿನಿಂದ ದಿಟ್ಟಿಸಿದರು. 

ಈಗ ಶ್ರೀರಾಮಕೃಷ್ಣರು ನರೇಂದ್ರನನ್ನು ಕೇಳಿದರು: “ನೀನು ದಿನಾಲೂ ರಾತ್ರಿ ಮಲಗಿಕೊಳ್ಳು ವಾಗ ಒಂದು ಬೆಳಕು ಕಾಣುತ್ತೀಯೇನು?” ನರೇಂದ್ರ ಹೌದೆಂದ. ಒಡನೆಯೇ ಶ್ರೀರಾಮಕೃಷ್ಣರು ಗಟ್ಟಿಯಾಗಿ, “ಆ್ಹ! ಎಲ್ಲವೂ ತಾಳೆಯಾಗುತ್ತಿದೆ; ಇವನು ಜನ್ಮತಃ ಧ್ಯಾನಸಿದ್ಧ” ಎಂದರು. ನರೇಂದ್ರ ಮಾತ್ರ ಸುಮ್ಮನೆ ನೋಡುತ್ತ ಕುಳಿತಿದ್ದಾನೆ. ಅವನ ಮನಸ್ಸಿಗಿನ್ನೂ ತಾಳೆಯಾಗ ಬೇಕಾಗಿದೆ. ಅವನು ಒಂದೇ ಸಮನೆ ಶ್ರೀರಾಮಕೃಷ್ಣರನ್ನು ಪರೀಕ್ಷಾದೃಷ್ಟಿಯಿಂದ ನೋಡುತ್ತಿದ್ದಾನೆ. ಅವರು ಮಾತನಾಡುವ ಕ್ರಮವನ್ನು ನೋಡುತ್ತಾನೆ, ಎಲ್ಲವು ಸರಿಯಾಗಿದೆ. ಏನೂ ತಪ್ಪಿಲ್ಲ. ಅವರು ಭಕ್ತರೊಂದಿಗೆ ನಡೆದುಕೊಳ್ಳುವ ಕ್ರಮವನ್ನು ನೋಡುತ್ತಾನೆ; ಎಲ್ಲ ಸರಿಯಾಗಿದೆ, ಏನೂ ಎಡವಟ್ಟಿಲ್ಲ! ಅವರ ವರ್ತನೆ ಸುಸಂಬದ್ಧವಾಗಿದೆ, ಏನೂ ದೋಷವಿಲ್ಲ. ಅವರ ಆಧ್ಯಾತ್ಮಿಕ ಮಾತುಕತೆಯನ್ನು ಕೇಳಿದಾಗ, ಅವರು ಭಾವಸಮಾಧಿಗೇರುವ ಪರಿಯನ್ನು ನೋಡಿದಾಗ, ನೈಜ ತ್ಯಾಗಿಯಂತೆ, ವಿರಾಗಿಯಂತೆ ಕಂಡುಬರುತ್ತಿದ್ದಾರೆ. ಅವರ ಬಾಹ್ಯ ವ್ಯಕ್ತಿತ್ವವನ್ನೂ ಮಾತುಗಳನ್ನೂ ಹೋಲಿಸಿ ನೋಡಿದಾಗ ಒಂದಕ್ಕೊಂದು ನಿಕಟವಾದ ಹೊಂದಾಣಿಕೆಯಿರುವುದು ಸ್ವಷ್ಟವಾಗಿ ಗೋಚರಿಸುತ್ತಿದೆ. ಅವರ ಭಾಷೆ ತುಂಬ ಸರಳ; ತೀರ ಸಾಮಾನ್ಯರೂ ಅರ್ಥಮಾಡಿಕೊಳ್ಳುವಷ್ಟು ಸರಳ, ಅಕೃತ್ರಿಮ. ಆದರೂ... ಇವರು ನಿಜಕ್ಕೂ ಅಂಥಾ ದೊಡ್ಡ ಆಧ್ಯಾತ್ಮಿಕ ವ್ಯಕ್ತಿಯೇ? ಹಾಗಾದರೆ ನೋಡೋಣ, ತನ್ನ ಪ್ರಶ್ನೆಗೆ ಇವರೇನು ಉತ್ತರಿಸುತ್ತಾರೆಂದು. ನರೇಂದ್ರ ಈಗ ಅವರ ಬಳಿಗೆ ಸರಿದು, ತನ್ನ ಯಕ್ಷಪ್ರಶ್ನೆಯನ್ನು ಮುಂದಿಟ್ಟ:

“ಮಹಾಶಯರೆ, ನೀವು ದೇವರನ್ನು ಕಂಡಿದ್ದೀರಾ?”

ಸುತ್ತ ಕುಳಿತವರಿಗೆಲ್ಲ ಆಶ್ಚರ್ಯ, ಕುತೂಹಲ. ಈ ಹಿಂದೆ ಇಂತಹ ಪ್ರಶ್ನೆಯನ್ನು ಕೇಳಿದವರೇ ಇಲ್ಲ. ಆದರೆ ಶ್ರೀರಾಮಕೃಷ್ಣರು ಮಾತ್ರ ಸ್ವಲ್ಪವೂ ಕಸಿವಿಸಿಗೊಳ್ಳಲಿಲ್ಲ, ತಬ್ಬಿಬ್ಬಾಗಲಿಲ್ಲ. ಮುಗುಳ್ನಗುತ್ತಲೇ, ಬಾಣದಂತೆ ಬಂದ ಅವನ ಪ್ರಶ್ನೆಗೆ ಅಷ್ಟೇ ನೇರವಾದ ಉತ್ತರವಿತ್ತರು:

“ಹೌದಪ್ಪ, ಕಂಡಿದ್ದೇನೆ! ಈಗ ನಿನ್ನನ್ನು ಕಾಣುತ್ತಿರುವಂತೆಯೇ, ಆದರೆ ಇನ್ನೂ ಹೆಚ್ಚು ಸ್ವಷ್ಟವಾಗಿ ಕಂಡಿದ್ದೇನೆ!”

ನರೇಂದ್ರ ಸ್ತಬ್ಧನಾಗಿ ಕುಳಿತ. ಶ್ರೀರಾಮಕೃಷ್ಣರು ಮತ್ತೆ ನುಡಿದರು:

“ನಾನು ನಿನ್ನನ್ನು ಕಾಣುವಷ್ಟೇ, ನಿನ್ನೊಡನೆ ಮಾತನಾಡುವಷ್ಟೇ ಸ್ವಷ್ಟವಾಗಿ ಭಗವಂತನನ್ನು ಕಾಣಬಹುದು, ಕಂಡು ಮಾತನಾಡಬಹುದು. ಆದರೆ ದೇವರು ಯಾರಿಗೆ ಬೇಕಾಗಿದೆ, ಹೇಳು? ಜನಗಳು ಹೆಂಡತಿ-ಮಕ್ಕಳು, ಹಣ-ಆಸ್ತಿಗಾಗಿ ಕೊಡಗಟ್ಟಲೆ ಕಣ್ಣೀರು ಸುರಿಸುತ್ತಾರೆ; ಆದರೆ ದೇವರಿಗಾಗಿ ಅಳುವವರು ಯಾರಿದ್ದಾರೆ ಹೇಳು? ದೇವರಿಗಾಗಿ ಪ್ರಾಮಾಣಿಕವಾಗಿ ಅತ್ತದ್ದೇ ಆದರೆ ಅವನು ದರ್ಶನ ಕೊಟ್ಟೇ ಕೊಡುತ್ತಾನೆ.”

ಈ ಉತ್ತರ ನರೇಂದ್ರನ ಮನಸ್ಸಿನಲ್ಲಿ ಆಳವಾಗಿ ನಾಟಿತು; ಗಾಢ ಪ್ರಭಾವ ಬೀರಿತು. ‘ನಾನು ದೇವರನ್ನು ನೋಡಿದ್ದೇನೆ’ ಎಂದು ಎದೆತಟ್ಟಿ ಹೇಳಬಲ್ಲ ವ್ಯಕ್ತಿಯನ್ನು ಅವನು ಜೀವನದಲ್ಲಿ ಇದೇ ಮೊದಲ ಸಲ ನೋಡುತ್ತಿದ್ದಾನೆ. ಧರ್ಮವೆನ್ನುವುದು ಅನುಭವಗಮ್ಯವಾದ ಸತ್ಯ, ಹಾಗೂ ನಾವು ಈ ಜಗತ್ತನ್ನು ಕಾಣುವುದಕ್ಕಿಂತಲೂ ಎಷ್ಟೋ ಪಾಲು ಹೆಚ್ಚು ಸ್ಪಷ್ಟವಾಗಿ ಭಗವಂತನನ್ನು ಕಾಣಬಹುದು ಎಂಬ ಮಾತನ್ನು ಇದೇ ಮೊಟ್ಟಮೊದಲ ಬಾರಿಗೆ ಕೇಳುತ್ತಿದ್ದಾನೆ. ಶ್ರೀರಾಮಕೃಷ್ಣ ರಾದರೂ ಅದನ್ನು ಹೇಗೆ ಹೇಳುತ್ತಿದ್ದಾರೆ? ಸಾಮಾನ್ಯ ಗುರುಗಳಂತೆ ಅನುಮಾನಾಸ್ಪದವಾದ ದನಿಯಲ್ಲಿ ಹೇಳುತ್ತಿದ್ದಾರೇನು? ಇಲ್ಲ, ತಮ್ಮ ಸಾಕ್ಷಾತ್ಕಾರದ ಆಧಾರದ ಮೇಲೆ, ಎದೆತಟ್ಟಿ, ದ್ವಂದ್ವಕ್ಕೆಡೆಯಿಲ್ಲದಂತೆ ಹೇಳುತ್ತಿದ್ದಾರೆ! ಆ ಮಾತು ಅಷ್ಟು ಶಕ್ತಿಪೂರ್ಣವಾಗಿದೆ.

ನರೇಂದ್ರನಿಗಾದ ಅಚ್ಚರಿ-ಆನಂದ ಎಷ್ಟೆಂದು ಹೇಳೋಣ! ಅವನ ತಲೆಯಲ್ಲಿ ನೂರೆಂಟು ಭಾವಗಳು, ಆಲೋಚನೆಗಳು ನಿಮಿಷಮಾತ್ರದಲ್ಲಿ ಹಾದುಹೋದುವು. “ಅಂತೂ ಕಡೆಗೊಬ್ಬ ಸಿಕ್ಕಿದನಲ್ಲ–ದೇವರನ್ನು ನೋಡಿದ್ದೇನೆ, ನೀನೂ ನೋಡಬಹುದು ಎಂದು ಎದೆತಟ್ಟಿ ಹೇಳಬಲ್ಲ ಗಂಡು! ಇವರಲ್ಲಿ ಎಂಥ ಆತ್ಮವಿಶ್ವಾಸವಿದೆ! ಅದು ಕೇವಲ ಬಾಯಿಮಾತಾಗಿರಲು ಸಾಧ್ಯವಿಲ್ಲ. ಇವರ ಅನುಭವದ ಆಳದಿಂದಲೇ ಆ ಮಾತು ಹೊಮ್ಮಿಬಂದಂತೆ ಭಾಸವಾಗುತ್ತಿದೆ. ಆದರೆ ಮೊದಲು ಹಾಗೇಕೆ ವಿಚಿತ್ರವಾಗಿ ನಡೆದುಕೊಂಡರು?... ನನ್ನನ್ನು ‘ನರಪುಷಿ’ ‘ನಾರಾಯಣ’ ಎಂದೆಲ್ಲ ಕರೆದರಲ್ಲ, ಅದೇಕೆ? ಈ ಎರಡು ಅತಿರೇಕಗಳನ್ನು ಹೇಗೆ ಅರ್ಥಮಾಡಿಕೊಳ್ಳಲಿ? ಬಹುಶಃ ಅದೊಂದು ಹುಚ್ಚೆ ಇರಬೇಕು–ಇಪ್ಪತ್ತನಾಲ್ಕು ಗಂಟೆಯೂ ಒಂದೇ ಆಲೋಚನೆ ಯಲ್ಲಿ ತೊಡಗಿರುವಂತಹ ಹುಚ್ಚು. ಆದರೆ ಒಂದಂತೂ ನಿಜ–ಇವರ ತ್ಯಾಗವೈರಾಗ್ಯಗಳು ಮಾತ್ರ ಎಲ್ಲೋ ಕೆಲವೇ ಮಂದಿ ಭಾಗ್ಯಶಾಲಿಗಳಲ್ಲಿ ಕಂಡುಬರುವಂಥವುಗಳು. ಒಂದು ವೇಳೆ ಹುಚ್ಚರೇ ಆಗಿದ್ದರೂ ಇವರು ಪರಮಪವಿತ್ರರೇ ಸರಿ, ಇವರು ನಿಜವಾದ ಸಂತ. ಈ ಒಂದು ಗುಣಕ್ಕಾದರೂ ಮನುಕುಲ ಇವರಿಗೆ ತಲೆಬಾಗಲೇ ಬೇಕು... ” ಹೀಗೆಲ್ಲ ಆಲೋಚಿಸುತ್ತ ನರೇಂದ್ರ ಇಬ್ಬಂದಿಯ ಮನಸ್ಥಿತಿಯಲ್ಲೇ ಶ್ರೀರಾಮಕೃಷ್ಣರಿಗೆ ಪ್ರಣಾಮ ಸಲ್ಲಿಸಿ ಹಿಂದಿರುಗಿದ.

ಮನೆಗೆ ಬಂದಮೇಲೂ ನರೇಂದ್ರನ ಮನಸ್ಸು ಶ್ರೀರಾಮಕೃಷ್ಣರ ಕುರಿತಾದ ಆಲೋಚನೆಗಳನ್ನೇ ತುಂಬಿಕೊಂಡ ಜೇನುಗೂಡಿನಂತಾಗಿತ್ತು. ತಾನೇನೋ ಶ್ರೀರಾಮಕೃಷ್ಣರನ್ನು ಹುಚ್ಚರೆಂದು ಕಡೆ ಗಣಿಸಬಹುದು; ಆದರೆ ಅವರ ಸನ್ನಿಧಿಯಲ್ಲಿರುವವರೆಗೂ ಒಂದು ಅಪೂರ್ವವಾದ ಮಂಗಳಕರ ಭಾವನೆ ತನ್ನ ಇಡೀ ವ್ಯಕ್ತಿತ್ವವನ್ನೇ ಆವರಿಸಿದ ಹಾಗೆ ಭಾಸವಾಗುತ್ತಿತ್ತಲ್ಲ, ಅದು ಹೇಗೆ ಸಾಧ್ಯ ವಾಯಿತು? ಒಟ್ಟಿನಲ್ಲಿ ಅವನಿಗೆ ದಕ್ಷಿಣೇಶ್ವರದ ಕೋಣೆಯಲ್ಲಿನ ಎಲ್ಲವೂ ಅದ್ಭುತ-ಅಪೂರ್ವ ವಾಗಿ ತೋರುತ್ತಿತ್ತು. ಶ್ರೀರಾಮಕೃಷ್ಣರನ್ನು ಅಪಾರ ಭಕ್ತಿಗೌರವಗಳಿಂದ ಕಾಣುವ ಭಕ್ತವೃಂದ ವೇನು! ಶ್ರೀರಾಮಕೃಷ್ಣರು ಆಗಾಗ ಏರಿಹೋಗುತ್ತಿದ್ದ ಭಾವಸಮಾಧಿಯ ವೈಚಿತ್ರ್ಯವೇನು! ಮತ್ತೆ ಆ ಸ್ಥಿತಿಯಿಂದ ಇಳಿದುಬರುವ ಅದ್ಭುತವೇನು! ಹಾಗೆ ಇಳಿದುಬಂದಾಗ ಅಲ್ಲಿ ಉಂಟಾ ಗುವ ಅತ್ಯುನ್ನತ ಆಧ್ಯಾತ್ಮಿಕ ವಾತಾವರಣವೇನು! ಅವರ ಅಲೌಕಿಕವಾದ ಮಾತುಗಳೇನು! ಅವರ ಪವಿತ್ರ ಸನ್ನಿಧಿಯಲ್ಲಿ ತನ್ನ ಮನಸ್ಸು ಅತಿಸಹಜವಾಗಿ ಉನ್ನತ ಸ್ಥಿತಿಗೇರುವ ಪರಿಯೇನು! ಹೀಗೆ ಇವುಗಳನ್ನೆಲ್ಲ ಭಾವಿಸುತ್ತ ನರೇಂದ್ರ ನಿಬ್ಬೆರಗಾದ. ಬೇಗ ಹೋಗಿ ಅವರನ್ನು ಮತ್ತೊಮ್ಮೆ ಕಾಣಬೇಕೆಂಬ ಉತ್ಕಟೇಚ್ಛೆ ಉದಿಸಿತು. ಅಲ್ಲದೆ, ಏಕಾಂಗಿಯಾಗಿ ಶೀಘ್ರದಲ್ಲೇ ಮತ್ತೆ ಬರುತ್ತೇ ನೆಂದು ಮಾತು ಬೇರೆ ಕೊಟ್ಟಿದ್ದ. ಆದರೆ ತಾನು ಅವರ ಪ್ರಭಾವಕ್ಕೆ ಒಳಗಾಗಬಾರದು ಎಂಬ ಮನಸ್ಸು. ಆದ್ದರಿಂದ ಒಂದಲ್ಲ ಒಂದು ಕೆಲಸದ ನೆಪವನ್ನು ಮುಂದಿಟ್ಟುಕೊಂಡು ದಕ್ಷಿಣೇಶ್ವರಕ್ಕೆ ಹೋಗುವುದನ್ನು ಮುಂದೂಡುತ್ತಲೇ ಬಂದ. ಸುಮಾರು ಒಂದು ತಿಂಗಳ ಬಳಿಕ ಕೊನೆಗೊಂದು ದಿನ ಅತ್ತ ಹೊರಟ. ಈ ಸಲ ಅವನೊಬ್ಬನೇ ನಡೆದುಕೊಂಡು ಹೋದ. ದಕ್ಷಿಣೇಶ್ವರದ ಮಂದಿರೋದ್ಯಾನಕ್ಕೆ ಬಂದವನೇ ಶ್ರೀರಾಮಕೃಷ್ಣರ ಕೋಣೆಯನ್ನು ಪ್ರವೇಶಿಸಿದ. ಶ್ರೀರಾಮ ಕೃಷ್ಣರು ತಮ್ಮ ಸಣ್ಣಮಂಚದ ಮೇಲೆ ಕುಳಿತಿದ್ದರು. ನರೇಂದ್ರನನ್ನು ಕಾಣುತ್ತಲೇ ಅವರಿಗೆ ಅತ್ಯಾನಂದವಾಗಿಬಿಟ್ಟಿತು. ಅವನನ್ನು ತುಂಬ ವಿಶ್ವಾಸದಿಂದ ಬರಮಾಡಿಕೊಂಡು, ಮಂಚದ ಮೇಲೆ ತಮ್ಮ ಪಕ್ಕದಲ್ಲೇ ಕುಳ್ಳಿರಿಸಿಕೊಂಡರು. ಆದರೆ ಮರುಕ್ಷಣಕ್ಕೆ ಅವರಲ್ಲೇನೋ ಒಂದು ಬಗೆಯ ಭಾವಾವೇಶವುಂಟಾದುದನ್ನು ಕಂಡ ನರೇಂದ್ರ. ಈಗ ಅವರು ತಮ್ಮಷ್ಟಕ್ಕೆ ಏನನ್ನೋ ಮಟಗುಟ್ಟುತ್ತ ಅವನೆಡೆಗೆ ಸರಿದರು. ದೃಷ್ಟಿ ಅವನ ಮೇಲೇ ನೆಟ್ಟಿದೆ. ಇದನ್ನು ಕಂಡು ನರೇಂದ್ರ, ‘ಕಳೆದ ಸಲ ಬಂದಾಗ ಏನೋ ಒಂದು ತರಹ ವಿಚಿತ್ರವಾಗಿ ನಡೆದುಕೊಂಡರಲ್ಲ, ಹಾಗೆಯೇ ಈಗಲೂ ಏನೋ ಮಾಡುತ್ತಾರೆ ಅಂತ ಕಾಣುತ್ತದೆ’ ಎಂದು ಭಾವಿಸಿದ. ಆದರೆ ಕಣ್ರೆಪ್ಪೆ ಹೊಡೆಯುವಷ್ಟರಲ್ಲಿ ಶ್ರೀರಾಮಕೃಷ್ಣರು ತಮ್ಮ ಬಲಪಾದವನ್ನೆತ್ತಿ ಆತನ ಮೈಮೇಲಿಟ್ಟುಬಿಟ್ಟರು! ಆಹ್! ಅದೆಂಥ ದಿವ್ಯ ಸ್ಪರ್ಶ! ಒಡನೆಯೇ ನರೇಂದ್ರನಿಗೆ ಅತಿ ವಿಚಿತ್ರವಾದ, ಭಯಂಕರ ಅನುಭವವಾಯಿತು. ಅವನು ನೋಡನೋಡುತ್ತಿದ್ದಂತೆಯೇ ಆ ಕೋಣೆಯ ಗೋಡೆಗಳೂ ಕೋಣೆಯೊಳಗಿನ ಸಮಸ್ತ ವಸ್ತುಗಳೂ ರಭಸದಿಂದ ಗಿರಗಿರನೆ ತಿರುಗಿ ಅವ್ಯಕ್ತದಲ್ಲಿ ಲೀನವಾಗಿ ಹೋಗಿಬಿಟ್ಟುವು. ಅವನ ವ್ಯಕ್ತಿತ್ವವೂ ಸೇರಿದಂತೆ ಸಮಸ್ತ ವಿಶ್ವವೇ ಕರಕರಗಿ ಯಾವುದೋ ವರ್ಣನಾತೀತ ರಹಸ್ಯ ಶೂನ್ಯದೊಳಗೆ ಲಯವಾಗಿಹೋಯಿತು! ನರೇಂದ್ರನ ಹೃದಯ ಭಯ ದಿಂದ ನಡುಗಿತು. ಮಹಾಸಿಂಹದ ಸ್ಪರ್ಶಮಾತ್ರದಿಂದ ಮರಿಸಿಂಹ ತಲ್ಲಣಿಸಿತು! ಅವನಿಗೆ ತಾನು ಭಯಂಕರ ಮೃತ್ಯುವಿನ ಬಾಯಿಗೀಡಾಗುತ್ತಿರುವಂತೆ ತೋರಿತು. ಆ ವಿಚಿತ್ರ ಅನುಭವದ ತೀವ್ರತೆಯನ್ನು ತಾಳಲಾರದೆ, “ನನಗಿದೇನು ಮಾಡುತ್ತಿದ್ದೀರಿ ನೀವು! ನನಗೆ ಮನೆಯಲ್ಲಿ ತಾಯ್ತಂದೆಯರಿದ್ದಾರೆ!” ಎಂದು ಕೂಗಿಕೊಂಡ. ಅವನ ಈ ಅಸಹಾಯಕ ಚೀರಾಟವನ್ನು ಕೇಳಿ ಶ್ರೀರಾಮಕೃಷ್ಣರು ನಕ್ಕುಬಿಟ್ಟರು. ಆದರೆ ಅವರಿಗೆ ಸ್ವಲ್ಪ ಅಚ್ಚರಿಯೂ ಆಯಿತು. ಏಕೆಂದರೆ ಅವರು ಅವನಿಂದ ಆ ಬಗೆಯ ಪ್ರತಿಕ್ರಿಯೆಯನ್ನು ನಿರೀಕ್ಷಿಸಿರಲಿಲ್ಲ. ಬಳಿಕ ಅವನ ಎದೆಯನ್ನು ನೇವರಿಸುತ್ತ “ಆಗಲಿ, ಆಗಲಿ; ಇಂದಿಗಿಷ್ಟೇ ಸಾಕು. ಮುಂದೆ ಸಕಾಲದಲ್ಲಿ ಎಲ್ಲವೂ ಆಗುತ್ತದೆ” ಎಂದರು. ಅತ್ಯಾಶ್ಚರ್ಯ! ಶ್ರೀರಾಮಕೃಷ್ಣರು ಹೀಗೆ ಅಂದದ್ದೇ ತಡ, ನರೇಂದ್ರನ ಆ ವಿಚಿತ್ರ ಅನುಭವ ದೂರವಾಯಿತು. ಅವನು ಮತ್ತೆ ಮೊದಲಿನಂತಾದ. ಈಗ ಕೋಣೆಯ ಗೋಡೆಗಳು, ಒಳಗಿರುವ ವಸ್ತುಗಳು ಎಲ್ಲ ಮೊದಲಿನಂತೆಯೇ ಕಾಣತೊಡಗಿದುವು. ಆದರೆ ಇವಿಷ್ಟೂ ನಡೆದುಹೋದದ್ದು ಕೆಲವೇ ಕ್ಷಣಗಳಲ್ಲಿ! ಈ ವಿಚಿತ್ರ ಪ್ರಸಂಗದ ತಲೆಬುಡ ಅರ್ಥವಾಗದೆ ನರೇಂದ್ರ ದಿಗ್ಭ್ರಾಂತನಾಗಿ ಕುಳಿತ.

ನರೇಂದ್ರನ ಆ ವಿಚಿತ್ರ ಅವಸ್ಥೆ ದೂರವಾದ ಮೇಲೆ ಶ್ರೀರಾಮಕೃಷ್ಣರೂ ಪುನಃ ಸಹಜಸ್ಥಿತಿಗೆ ಮರಳಿದರು. ಈಗ ಆತನೊಂದಿಗೆ, ಅವನು ತಮಗೆ ಚಿರಪರಿಚಿತನಾದ ಹಳೆಯ ಸ್ನೇಹಿತನೋ ಎಂಬಂತೆ ನಡೆದುಕೊಳ್ಳತ್ತ ಅವನನ್ನು ಉಪಚರಿಸತೊಡಗಿದರು. ಅವನನ್ನು ಎಷ್ಟು ಉಪಚರಿಸಿ ದರೂ ಅವರಿಗೆ ತೃಪ್ತಿಯಿಲ್ಲ! ಇಂತಹ ನಿಃಸ್ವಾರ್ಥ, ನಿಷ್ಕಲ್ಮಷ ಪ್ರೀತಿಯನ್ನು ಕಂಡು ನರೇಂದ್ರ ಸಂಪೂರ್ಣ ಮಾರುಹೋದ. ಆ ದಿನವಿಡೀ ಅವನು ದಕ್ಷಿಣೇಶ್ವರದಲ್ಲೇ ಇದ್ದ. ಸಾಯಂಕಾಲ ವಾಯಿತು, ತಾನಿನ್ನು ಹೋಗಿಬರುವುದಾಗಿ ಶ್ರೀರಾಮಕೃಷ್ಣರಿಗೆ ಹೇಳಿದ. ಅವರಿಗೆ ತುಂಬ ಖೇದ ವಾಯಿತು. ಕೊನೆಗೆ ಸಾಧ್ಯವಾದಷ್ಟು ಬೇಗನೆ ಇನ್ನೊಮ್ಮೆ ಬರುವುದಾಗಿ ಅವನಿಂದ ಭಾಷೆ ತೆಗೆದುಕೊಂಡು, ಒಲ್ಲದ ಮನಸ್ಸಿನಿಂದಲೇ ಅನುಮತಿಯಿತ್ತರು.

ನರೇಂದ್ರನ ತಲೆಯಲ್ಲಿ ನೂರೆಂಟು ಆಲೋಚನೆಗಳು. ಈ ಆಶ್ಚರ್ಯಕರ ವ್ಯಕ್ತಿಯ ಇಚ್ಛಾ ಮಾತ್ರದಿಂದ ತನಗೆ ಅಂತಹ ನಿಗೂಢ ಅವಸ್ಥೆ ಬಂದೊದಗಿತು; ಮತ್ತು ಅವರು ಇಚ್ಛಿಸಿದೊಡನೆ ಅದು ಹೊರಟುಹೋಯಿತು! ಏನಿರಬಹುದು ಇದರ ಮರ್ಮ? ಇದೊಂದು ವಶೀಕರಣವೇ? ಅಥವಾ ಸಮ್ಮೋಹಿನೀ ವಿದ್ಯೆಯ ಪ್ರಯೋಗವೆ? ಛೆ, ಇವೆಲ್ಲ ಸಾಧ್ಯವೇ ಇಲ್ಲ. ಏಕೆಂದರೆ ಸಮ್ಮೋಹಿನಿ-ವಶೀಕರಣ ಇವೆಲ್ಲ ಪ್ರಭಾವ ಬೀರುವುದು ದುರ್ಬಲ ಮನಸ್ಸಿನವರ ಮೇಲೆ ಮಾತ್ರ. ತಾನಾದರೋ ಅಸಾಧರಾಣ ಗಟ್ಟಿಮನಸ್ಸಿನ ಯುವಕ! ಅಲ್ಲದೆ ತಾನು ಇವರನ್ನು ಕೇವಲ ಒಬ್ಬ ಮರುಳ ಎಂದು ಪರಿಗಣಿಸಿದವನು. ಹೀಗಿರುವಾಗ, ಈಗ ತನ್ನಲ್ಲಿ ಇದ್ದಕ್ಕಿದ್ದಂತೆ ಈ ಬಗೆಯ ಪರಿಣಾಮ ಹೇಗೆ ಆಗಿರಬಹುದು? ಯೋಚನೆ ಮಾಡಿಮಾಡಿ, ಕೊನೆಗೆ ಬೇರೇನೂ ತೋಚದೆ, ‘ಇದೊಂದು ಬಿಡಿಸಲಾಗದ ಒಗಟು. ಇದನ್ನು ಬಿಡಿಸುವ ಪ್ರಯತ್ನವನ್ನು ಬಿಟ್ಟುಬಿಡುವುದೊಂದೇ ಮಾರ್ಗ’ ಎಂದು ತೀರ್ಮಾನಿಸಿದ. ಆದರೆ ಮತ್ತಿನ್ನೆಂದೂ ತಾನು ಅವರ ಪ್ರಭಾವಕ್ಕೆ ಬಲಿ ಬೀಳದಂತೆ ಜಾಗರೂಕತೆಯಿಂದ ಇರಬೇಕು ಎಂದು ತನಗೆ ತಾನೇ ಹೇಳಿಕೊಂಡ. ಮತ್ತೆ ಮರುಕ್ಷಣಕ್ಕೆ ಅವನಿಗೆ ಅನ್ನಿಸಿತು–‘ಅಲ್ಲ, ನನ್ನ ಮನಸ್ಸು ಎಷ್ಟು ಗಟ್ಟಿ, ಎಷ್ಟು ದೃಢ! ಇಂತಹ ನನ್ನ ಮನಸ್ಸನ್ನೇ ಕ್ಷಣಾರ್ಧದಲ್ಲಿ ತಮ್ಮ ಹಿಡಿತಕ್ಕೆ ತೆಗೆದುಕೊಂಡುಬಿಟ್ಟರಲ್ಲ–ಅಂತಹ ಮನುಷ್ಯ ನನ್ನು ಕೇವಲ ಮರುಳ, ಹುಚ್ಚ ಅಂತ ಹೇಗೆ ತಾನೆ ಕರೆಯಲು ಸಾಧ್ಯ...? ಆದರೆ ಇವರು ಮೊದಲು ನನ್ನನ್ನು ಕಂಡಾಗ ನಡೆದುಕೊಂಡ ರೀತಿಯನ್ನು ನೋಡಿದರೆ, ಯಾರಾದರೂ ಇವರನ್ನು ಹುಚ್ಚನೆಂದು ತಿಳಿಯುವುದೇ ನಿಜ. ಹೋಗಲಿ, ಇವರೇನಾದರೂ ಅವತಾರ ಪುರುಷರಾಗಿದ್ದರೆ ಇದನ್ನೆಲ್ಲ ಸಹಜ ಅಂತ ಒಪ್ಪಿಕೊಳ್ಳಬಹುದಾಗಿತ್ತು. ಆದರೆ ಅವತಾರ-ಗಿವತಾರ ಎನ್ನುವುದೆಲ್ಲ ಬಹಳ ದೂರದ ಮಾತು.’ಆ ದಿನವೆಲ್ಲ ಇಂತಹ ಆಲೋಚನೆಗಳೇ ಅವನ ಮನಸ್ಸನ್ನು ಕೊರೆಯುತ್ತಿದ್ದುವು. ಆ ಒಂದು ವಿಚಿತ್ರ ಘಟನೆಯ ಮರ್ಮವನ್ನು ಅರಿಯಲು ಅಸಮರ್ಥನನ್ನಾಗಿಸಿದ ಪರಿಸ್ಥಿತಿಯೇ ಅವನ ಬುದ್ಧಿಶಕ್ತಿಗೊಂದು ಬಲವಾದ ಏಟು! ಆದರೂ ಸ್ಥೈರ್ಯಗೆಡದೆ ‘ ಈ ನಿಗೂಢ ರಹಸ್ಯವನ್ನು ತಾನು ಭೇದಿಸಿಯೇ ಸಿದ್ಧ’ ಎಂದು ದೃಢನಿಶ್ಚಯ ಮಾಡಿದ. ಎಷ್ಟಾದರೂ ಸಿಂಹದ ಮರಿಯಲ್ಲವೆ!

ಕೆಲದಿನಗಳು ಕಳೆದುವು. ನರೇಂದ್ರನಿಗೆ ಈಗ ಮತ್ತೊಮ್ಮೆ ದಕ್ಷಿಣೇಶ್ವರಕ್ಕೆ ಹೋಗಿ ಬರುವ ಮನಸ್ಸಾಗಿದೆ. ಆದರೆ, ತಾನು ಮತ್ತೆ ಆ ವಿಚಿತ್ರ ಮನುಷ್ಯನ ಸಮ್ಮೋಹಿನಿಗೆ ಒಳಗಾಗಲೇಬಾರದು ಎಂದು ಹಿಂದೆಯೇ ಮಾಡಿಕೊಂಡ ನಿರ್ಧಾರವನ್ನು ಈಗ ಮತ್ತೊಮ್ಮೆ ದೃಢಪಡಿಸಿಕೊಂಡ. ಮತ್ತು ಅದೇ ಭಾವದಲ್ಲೇ ದಕ್ಷಿಣೇಶ್ವರಕ್ಕೆ ಬಂದ. ಅವನನ್ನು ಕಂಡೊಡನೆಯೇ ಶ್ರಿರಾಮಕೃಷ್ಣರು ಉಲ್ಲಸಿತರಾದರು. ಅವನೊಡನೆ ಏಕಾಂತದಲ್ಲಿ ಮಾತನಾಡಬಯಸಿ, ಜನಜಂಗುಳಿಯಿದ್ದ ಆ ಜಾಗವನ್ನು ಬಿಟ್ಟು, ಸನಿಹದಲ್ಲಿ ಇದ್ದ ಯದುಮಲ್ಲಿಕ ಎಂಬ ಭಕ್ತನ ತೋಟಕ್ಕೆ ಅವನನ್ನು ಕರೆತಂದರು. ಅಲ್ಲಿ ಸ್ವಲ್ಪ ಹೊತ್ತು ಅಡ್ಡಾಡುತ್ತಿದ್ದು. ಬಳಿಕ ಒಂದೆಡೆ ಕುಳಿತರು. ನರೇಂದ್ರನೂ ಪಕ್ಕದಲ್ಲಿ ಕುಳಿತ. ಈಗ ಇವರು ಇನ್ನೇನು ಮಾಡಿಯಾರು ಎಂದು ಅವನು ಎಚ್ಚರಿದಿಂದ ಗಮನಿಸುತ್ತಿದ್ದಾನೆ. ಶ್ರೀರಾಮಕೃಷ್ಣರು ಮೆಲ್ಲನೆ ಭಾವಸಮಾಧಿಗೇರಿದರು. ಮತ್ತು ಆ ಸ್ಥಿತಿಯಲ್ಲೇ ಇದ್ದಕ್ಕಿದ್ದಂತೆ ಅವನನ್ನು ಸ್ಪರ್ಶಿಸಿದರು. ತಾನು ಈ ಬಾರಿ ಅವರ ಸ್ವರ್ಶದಿಂದೆಲ್ಲ ಪ್ರಭಾವಿತನಾಗಬಾರದು ಎಂದು ದೃಢನಿರ್ಧಾರ ಮಾಡಿಕೊಂಡು ಬಂದಿದ್ದವನಲ್ಲವೆ ಅವನು? ಆದರೆ ಶ್ರೀರಾಮಕೃಷ್ಣರು ಸ್ಪರ್ಶಿಸಿದಾಗ ಅದೇನೂ ಕೆಲಸಕ್ಕೆ ಬರಲಿಲ್ಲ.ಎಷ್ಟು ಪ್ರಯತ್ನಪಟ್ಟರೂ ತನ್ನನ್ನು ನಿಯಂತ್ರಿಸಿಕೊಳ್ಳಲಾರದಾದ; ಬಾಹ್ಯಪ್ರಜ್ಞೆ ತಪ್ಪಿಯೇ ಹೋಯಿತು! ಹಾಗೆಯೇ ಎಷ್ಟುಹೊತ್ತು ಕಳೆಯಿತೋ ಅವನಿಗೆ ತಿಳಿಯಲಿಲ್ಲ. ಆಗ ಏನಾಯಿತೆಂಬುದೂ ಗೊತ್ತಾಗಲಿಲ್ಲ. ಯಾವಾಗಲೋ ಮೈತಿಳಿದಾಗ ಶ್ರೀರಾಮಕೃಷ್ಣರು ತನ್ನ ಎದೆಯನ್ನು ನೇವರಿಸುತ್ತಿರುವುದನ್ನು ಕಂಡ. ಅವರು ತಮ್ಮ ಸ್ಪರ್ಶಮಾತ್ರದಿಂಂದ ಅವನ ಕುಂಡಲಿನೀಶಕ್ತಿಯನ್ನು ಜಾಗೃತಗೊಳಿಸಿ ಸಮಾಧಿಸ್ಥಿತಿಗೇರಿಸಿದ್ದರು; ಈಗ ಮತ್ತೆ ಅವನ ಮನಸ್ಸನ್ನು ಸಹಜಸ್ಥಿತಿಗೆ ತರುತ್ತಿದ್ದಾರೆ. ಆದರೆ ಆ ಸಿಂಹದ ಮರಿಗೆ ಇವೆಲ್ಲ ಇನ್ನೂ ತಿಳಿದಿಲ್ಲ. ಶ್ರೀರಾಮಕೃಷ್ಣರಿಗೆ ಮಾತ್ರ ಅಂದು ಅವನ ಸಂಬಂಧವಾಗಿ ಹಲವಾರ ವಿಷಯಗಳು ತಿಳಿದು ಬಂದುವು. ಈ ಕುರಿತಾಗಿ ಮುಂದೆ ಅವರೇ ಆಪ್ತ ಶಿಷ್ಯರೆದುರು ಹೇಳುತ್ತಾರೆ: “ನರೇಂದ್ರ ಆ ಸ್ಥಿತಿಯಲ್ಲಿದ್ದಾಗ ನಾನು ಅವನಿಗೆ ಹಲವಾರು ಪ್ರಶ್ನೆಗಳನ್ನು ಹಾಕಿದೆ. ಅವನ ಪೂರ್ವವೃತ್ತಾಂತವೇನು, ಅವನು ಎಲ್ಲಿ ವಾಸವಾಗಿದ್ದವನು, ಅವನು ಈ ಮರ್ತ್ಯಲೋಕದ ತನ್ನ ಜೀವಿತಾವಧಿಯಲ್ಲಿ ಏನೇನು ಕಾರ್ಯಗಳನ್ನು ಸಾಧಿಸಲಿದ್ದಾನೆ– ಎಂಬುದನ್ನೆಲ್ಲ ಕೇಳಿದೆ. ಅವನು ತನ್ನ ಅಂತರಾಳದೊಳಗೆ ಮುಳುಗಿ ಆ ಎಲ್ಲ ಪ್ರಶ್ನೆಗಳಿಗೂ ಸಮರ್ಪಕವಾದ ಉತ್ತರ ಕೊಟ್ಟ. ಆದರೆ ಆ ವಿಷಯಗಳನ್ನೆಲ್ಲ ನಾನು ಮೊದಲೇ ದರ್ಶನಗಳಲ್ಲಿ ಕಂಡು ತಿಳಿದುಕೊಂಡಿದ್ದೆ; ಈಗ ಅವನಿಂದಲೇ ಕೇಳಿ ದೃಢಪಡಿಸಿಕೊಂಡೆ ಅಷ್ಟೆ. ಅವೆಲ್ಲ ಬಹಳ ರಹಸ್ಯವಾದ ವಿಚಾರಗಳು. ಆದರೆ ಇಷ್ಟು ಮಾತ್ರ ಹೇಳಬಲ್ಲೆ: ಅವನೊಬ್ಬ ನಿತ್ಯಸಿದ್ಧನಾದ ಪುಷಿ, ಧ್ಯಾನಸಿದ್ಧ; ತನ್ನ ನಿಜಸ್ವರೂಪವನ್ನು ತಿಳಿದೊಡನೆಯೇ, ಕೇವಲ ತನ್ನ ಇಚ್ಛಾಮಾತ್ರದಿಂದ ಯೋಗದ ಮೂಲಕ ಶರೀರವನ್ನು ತ್ಯಜಿಸಿಬಿಡುತ್ತಾನೆ.”

ನರೇಂದ್ರನ ಕುರಿತಾಗಿ– ಅವನನ್ನು ಭೇಟಿಯಾಗುವ ಮೊದಲೂ ಅನಂತರವೂ–ಶ್ರೀರಾಮಕೃಷ್ಣರು ಕಂಡಿದ್ದ ದರ್ಶನಗಳು ಅನೇಕ; ಕೌತುಕಮಯ. ಅವುಗಳಲ್ಲೊಂದನ್ನು ಅವರೇ ಹೀಗೆ ಬಣ್ಣಿಸುತ್ತಾರೆ: \footnote{*``ಇವನ್ನೆಲ್ಲ ಅವರು ತಮ್ಮ ಅಪೂರ್ವ ಸರಳ ಶೈಲಿಯಲ್ಲಿ ತಿಳಿಸಿದ್ದಾರಾದರೂ ಅವನ್ನು ತದ್ವತ್ ಹಾಗೆಯೇ ಪುನರುದ್ಧರಿಸಲು ನಮ್ಮಿಂದ ಸಾಧ್ಯವೇ ಇಲ್ಲ" ಎಂದು ಸ್ವಾಮಿ ಶಾರದಾನಂದರು `ಶ್ರೀರಾಮಕೃಷ್ಣ ಲೀಲಾಪ್ರಸಂಗ'ದಲ್ಲಿ ಬರೆಯುತ್ತಾರೆ.}

“ಒಂದು ದಿನ ನನ್ನ ಮನಸ್ಸು ಸಮಾಧಿಯಲ್ಲಿ ಜ್ಯೋತಿರ್ಮಯ ಮಾರ್ಗವಾಗಿ ಎತ್ತರೆತ್ತರಕ್ಕೆ ಏರುತ್ತ ಹೋಗುವುದನ್ನು ಕಂಡೆ. ಶೀಘ್ರದಲ್ಲೇ ಅದು ಸೂರ್ಯ-ಚಂದ್ರ-ನಕ್ಷತ್ರಗಳಿಂದ ಕೂಡಿದ ಈ ಸ್ಥೂಲಜಗತ್ತನ್ನು ಅತಿಕ್ರಮಿಸಿ ಸೂಕ್ಷ್ಮವಾದ ಭಾವಜಗತ್ತನ್ನು ಪ್ರವೇಶಿಸಿತು. ಆ ಜಗತ್ತಿನ ಮತ್ತೂ ಸೂಕ್ಷ್ಮತರವಾದ ಸ್ತರಗಳ ಮೂಲಕ ಹಾದು ಹೋದಂತೆ, ಮಾರ್ಗದ ಇಕ್ಕೆಲಗಳಲ್ಲೂ ದೇವದೇವಿಯರ ವಿಚಿತ್ರ ರೂಪಗಳನ್ನು ಕಂಡೆ. ಕ್ರಮೇಣ ನನ್ನ ಮನಸ್ಸು ಇನ್ನೂ ಮೇಲೇರಿ, ಆ ಲೋಕದ ಹೊರಗಣ ಗಡಿಯನ್ನು ಮುಟ್ಟಿತು. ಅಲ್ಲೊಂದು ಜ್ಯೋತಿಯ ಆವರಣ ಹಬ್ಬಿಕೊಂಡು ಖಂಡ-ಅಖಂಡ ಜಗತ್ತುಗಳನ್ನು ಪ್ರತ್ಯೇಕಿಸುತ್ತದೆ. ಆ ಆವರಣವನ್ನೂ ದಾಟಿಕೊಂಡು ನನ್ನ ಮನಸ್ಸು ನಿರ್ಗುಣ-ನಿರಾಕಾರವಾದ ಅಖಂಡರಾಜ್ಯವನ್ನು ಪ್ರವೇಶಿಸಿತು. ದೇವದೇವಿಯರಿಗೂ ಸಹ ಇಲ್ಲಿ ಪ್ರವೇಶವಿಲ್ಲ. ಇಲ್ಲಿಗೆ ಅಡಿಯಿಡಲು ಭಯವೋ ಎಂಬಂತೆ ಅವರೆಲ್ಲ ಇಲ್ಲಿಗಿಂತ ಎಷ್ಟೋ ಕೆಳಗಿನ ಸ್ತರದಲ್ಲಿ ತಮ್ಮ ಆಧಿಪತ್ಯವನ್ನು ಸ್ಥಾಪಿಸಿದ್ದಾರೆ. ಅಲ್ಲಿ ನಾನು ದಿವ್ಯ ಜ್ಯೋತಿರ್ಮಯ ಶರೀರಧಾರಿಗಳಾದ ಏಳು ಜನ ಪುಷಿಗಳು ಸಾಮಧಿ ಮಗ್ನರಾಗಿ ಕುಳಿತಿರುವುದನ್ನು ಕಂಡೆ. ಜ್ಞಾನದಲ್ಲಾಗಲಿ ಪವಿತ್ರತೆಯಲ್ಲಾಗಲಿ ತ್ಯಾಗದಲ್ಲಾಗಲಿ ಪ್ರೇಮದಲ್ಲಾಗಲಿ ಅವರು, ಮಾನವಮಾತ್ರ ವಿಚಾರ ಹಾಗಿರಲಿ, ದೇವತೆಗಳನ್ನೂ ಮೀರಿಸಿದ್ದಾರೆ ಎಂದು ನನಗನಿಸಿತು. ಆ ಸಪ್ತಪರ್ಷಿಗಳ ಮಹಿಮೆ-ಮಹತ್ತುಗಳನ್ನು ಮನಗಾಣುತ್ತ ವಿಸ್ಮಿತನಾಗಿ ನಿಂತೆ. ಆಗ ಇದ್ದಕ್ಕಿದ್ದಂತೆ ಆ ಅಖಂಡ ಜ್ಯೋತಿರ್ಮಯ ರಾಜ್ಯದ ಒಂದಂಶ ಹಾಗಯೇ ಘನೀಭೂತವಾಗಿ ಒಂದು ದಿವ್ಯ ಶಿಶುವಿನ ರೂಪ ತಾಳಿತು. ಆ ಶಿಶು ಮೆಲ್ಲನೆ ಒಬ್ಬ ಪುಷಿಯ ಬಳಿಗೆ ಬಂದಿತು. ತನ್ನ ಸುಂದರ ಸುಕೋಮಲ ಕರಗಳಿಂದ ಆತನ ಕೊರಳನ್ನು ಬಳಸಿ ಅಪ್ಪಿಕೊಂಡಿತು. ಮಾಧುರ್ಯದಲ್ಲಿ ವೀಣಾನಾದವನ್ನು ನಾಚಿಸುವ ತನ್ನ ಮಧುರ ಜೇನು ಕಂಠದಿಂದ ಅವನನ್ನು ಕರೆಯುತ್ತ, ಸಮಾಧಿಮಗ್ನವಾದ ಅವನ ಮನಸ್ಸನ್ನು ತನ್ನೆಡೆಗೆ ಸೆಳೆಯುವ ಪ್ರಯತ್ನ ಮಾಡಿತು. ಶಿಶುವಿನ ಪ್ರೇಮ ಪೂರ್ಣ, ಕೋಮಲಸ್ಪರ್ಶ ಪುಷಿಯನ್ನು ತುರೀಯ ಸ್ಥಿತಿಯಿಂದ ಕೆಳಗೆ ತಂದಿತು. ಅವನ ಕಣ್ರೆಪ್ಪೆಗಲು ಮೆಲ್ಲನೆ ಅರೆಬಿರಿದುವು. ಆ ದಿವ್ಯಶಿಶುವನ್ನೇ ಅವನು ಎವೆಯಿಕ್ಕದೆ ದೃಷ್ಟಿಸಿದ. ನೋಡುತ್ತಿದ್ದಂತೆಯೇ ಅವನ ಮುಖ ಪ್ರಫುಲ್ಲವಾಯಿತು. ಅದನ್ನು ಕಂಡು, ಈ ಶಿಶುವೇ ಆ ಪುಷಿಯ ಹೃದಯದ ಸಿರಿ, ಮತ್ತು ಅವರಿಬ್ಬರದು ಅನಾದಿಕಾಲದ ಅವಿನಾ ಸಂಬಂಧ ಎಂದು ನಾನರಿತೆ. ಆ ದಿವ್ಯಶಿಶು ಅತ್ಯಂತ ಆನಂದಪಡುತ್ತ, ‘ನಾನು ಭೂಮಿಗೆ ಹೋಗುತ್ತಿದ್ದೇನೆ, ನೀನೂ ನನ್ನೊಂದಿಗೆ ಬರಬೇಕು’ ಎಂದು ಅವನಿಗೆಂದಿತು. ಪುಷಿ ಮೌನವಾಗಿಯೇ ಇದ್ದ. ಆದರೆ ಅವನ ಪ್ರೇಮಪೂರಿತ ನೋಟವೇ ಅವನ ಸಮ್ಮತಿಯನ್ನು ಸೂಚಿಸುತ್ತಿತ್ತು. ಹಾಗೆಯೇ ನೆಟ್ಟದೃಷ್ಟಿಯಿಂದ ಆ ಶಿಶುವನ್ನು ನೋಡುತ್ತ ಅವನು ಪುನಃ ಸಮಾಧಿಸ್ಥನಾಗಿಬಿಟ್ಟ. ತರುವಾಯ, ಆತನ ದೇಹ-ಮನಸ್ಸುಗಳ ಒಂದಂಶವು ಉಜ್ವಲ ಜ್ಯೋತಿ ರೂಪ ತಾಳಿ ಭೂಮಿಯತ್ತ ಇಳಿಯುವುದನ್ನು ಕಂಡು ನಾನು ವಿಸ್ಮಿತನಾದೆ. ನರೇಂದ್ರನನ್ನು ನೋಡುತ್ತಿದ್ದಂತೆಯೇ ಗುರುತುಹಿಡಿದುಬಿಟ್ಟೆ– ಇವನೇ ಆ ಪುಷಿ ಎಂದು.”

“ಹಾಗಾದರೆ, ಆ ದಿವ್ಯಶಿಶು ಯಾರೂ? ಎಂದು ಒಬ್ಬ ಭಕ್ತ ಕೇಳಿದಾಗ ಶ್ರೀರಾಮಕೃಷ್ಣರು ಮುಗುಳ್ನಗುತ್ತ ನುಡಿಯುತ್ತಾರೆ: “ನಾನೇ!

ಶ್ರೀರಾಮಕೃಷ್ಣರ ಈ ದರ್ಶನ ಅದೆಷ್ಟು ಭವ್ಯ! ಇಲ್ಲಿ ನಮಗೆ ನರೇಂದ್ರ ಯಾರು ಎನ್ನುವುದರ ಪರಿಚಯವಾದಂತೆ ಶ್ರೀರಾಮಕೃಷ್ಣರ ನಿಜಸ್ವರೂಪದ ಪರಿಚಯವೂ ಆಗುತ್ತದೆ. ಅಖಂಡ ಸಚ್ಚಿದಾನಂದವೇ ಘನೀಭೂತವಾಗಿ ಅಲ್ಲಿ ಶಿಶುರೂಪ ತಾಳಿದರೆ. ಇಲ್ಲಿ ಶ್ರೀರಾಮಕೃಷ್ಣ ರೂಪ ತಾಳಿದೆ. ಆ ಪುಷಿಯ ವ್ಯಕ್ತಿತ್ವದ ಒಂದು ಭಾಗವೇ ಈಗ ಈ ನರೇಂದ್ರ; ಮುಂದೆ ವೀರಸಂನ್ಯಾಸಿ ಸ್ವಾಮಿ ವಿವೇಕಾನಂದ!

ಇನ್ನೊಮ್ಮೆ ಶ್ರೀರಾಮಕೃಷ್ಣರು ಒಂದು ದರ್ಶನದ್ಲಲಿ ದೇದೀಪ್ಯಮಾನವಾದ ಜ್ಯೋತಿಯ ರೇಖೆಯೊಂದು ಕಾಶಿಯಿಂದ ಕಲ್ಕತ್ತಕ್ಕೆ ಆಕಾಶಮಾರ್ಗವಾಗಿ ಹೋದುದನ್ನು ಕಂಡಿದ್ದರು. ಅದರ ಗೂಢಾರ್ಥವನ್ನು ಗ್ರಹಿಸಿದ ಅವರು, “ನನ್ನ ಪ್ರಾರ್ಥನೆ ನೆರವೇರಿತು! ನನ್ನವನಾದ ಆ ವ್ಯಕ್ತಿ ನನ್ನ ಬಳಿಗೆ ಬಂದೇ ಬರುತ್ತಾನೆ!” ಎಂದು ಉದ್ಗರಿಸಿದ್ದರು. ಯಾವುದು ಆ ಜ್ಯೋತಿ? ಯಾರು ಆ ವ್ಯಕ್ತಿ? ಕಾಶೀ ವಿಶ್ವೇಶ್ವರನೇ ಆಜ್ಯೋತಿ, ನರೇಂದ್ರನೇ ಆ ವ್ಯಕ್ತಿ! ಹೀಗೆ ನರೇಂದ್ರನ ಆಗಮನದ ಕುರುಹುಗಳನ್ನು ಶ್ರೀರಾಮಕೃಷ್ಣರು ಮೊದಲೇ ಕಂಡುಕೊಂಡಿದ್ದರು.

ನರೇಂದ್ರ ಮೂರನೇ ಬಾರಿ ತಮ್ಮಲ್ಲಿಗೆ ಬಂದಾಗ ಅವರು ಆತನನ್ನು ಸಮಾಧಿ ಸ್ಥಿತಿಗೇರಿಸಿ ಅವನ ಪೂರ್ವಾಪರಗಳನ್ನೆಲ್ಲ ತಿಳಿದುಕೊಂಡರೂ ಅವನಿಗೆ ಮಾತ್ರ ಅದಾವುದನ್ನೂ ತಿಳಿಯಗೊಡಲೇ ಇಲ್ಲ! ಏಕಿರಬಹುದು? ಅದು ಶ್ರೀರಾಮಕೃಷ್ಣರ ಇಚ್ಛೆ. ಅವನು ಈಗಲೇ ತನ್ನ ಸ್ವಸ್ವರೂಪವನ್ನು ತಿಳಿಯುವಂತಾಗಬಾರದು ಎನ್ನುವುದು ಅವರ ಉದ್ದೇಶ. ಏಕೆಂದರೆ ಅವನಿನ್ನೂ ಅದಕ್ಕೆ ಸಿದ್ಧನಾಗಿಲ್ಲ. ಒಂದು ವೇಳೆ ಅಂತಹ ತಿಳಿವಳಿಕೆಯೇನಾದರೂ ಉಂಟಾದರೆ ಅವನು ಭಯಭೀತನಾಗಿ ತಲ್ಲಣಿಸಿಹೋಗುತ್ತಾನೆ ಎನ್ನುವುದನ್ನು ಅವರು ಮನಗಂಡಿದ್ದರು. ಅವನು ಎರಡನೆಯ ಸಲ ತಮ್ಮಲ್ಲಿಗೆ ಬಂದಿದ್ದಾಗಲೇ ಅಂತಹ ಅನುಭವವೊಂದು ಕಿಂಚಿತ್ ಆದಾಗ ಅವನು ಕಂಗಾಲಾಗಿ, ‘ಇದೇನು ಮಾಡುತ್ತಿದ್ದೀರಿ! ನನಗೆ ಮನೆಯಲ್ಲಿ ತಾಯ್ತಂದೆಯರಿದ್ದಾರೆ’ ಎಂದು ಗಟ್ಟಿಯಾಗಿ ಕೂಗಿಕೊಳ್ಳಲಿಲ್ಲವೆ? ಅವನು ಹಾಗೆ ತಲ್ಲಣಿಸಿದ್ದನ್ನು ಕಂಡಾಗ ಶ್ರೀರಾಮಕೃಷ್ಣರಿಗೇ ಸಂದೇಹವುಂಟಾಗಿಬಿಟ್ಟಿತ್ತು–ತಾವು ಹಿಂದೆ ದರ್ಶನದಲ್ಲಿ ಕಂಡಿದ್ದ ವ್ಯಕ್ತಿ ಇವನೆಯೇ?– ಎಂದು. ಆದ್ದರಿಂದಲೇ ಮೂರನೆಯ ಭೇಟಿಯಲ್ಲಿ ಅವರು ಅವನನ್ನು ಸಮಾಧಿಸ್ಥಿತಿಗೇರಿಸಿ, ಅವನ ವೃತ್ತಾಂತವನ್ನೆಲ್ಲ ಕೇಳಿ ಖಚಿತಪಡಿಸಿಕೊಂಡದ್ದು.

ನರೇಂದ್ರನಂತೂ ಈ ಅನುಭವದಿಂದ ಸಂಪೂರ್ಣ ತತ್ತರಿಸಿಹೋದ. ಈ ಅತಿಮಾನುಷ ಅಲೌಕಿಕ ಶಕ್ತಿಯ ಮುಂದೆ ತನ್ನ ಬದ್ಧಿಶಕ್ತಿ-ಇಚ್ಛಾಶಕ್ತಿಗಳೆಲ್ಲ ತೀರಾ ದುರ್ಬಲವೆಂದು ಅವನಿಗೆ ಮನದಟ್ಟಾಯಿತು. ಹಾಗಲ್ಲದಿದ್ದರೆ ತಾನು ಅಷ್ಟೊಂದು ಎಚ್ಚರ ವಹಿಸಿಯೂ, ಅವರ ಪ್ರಭಾವಕ್ಕೆ ಒಳಗಾಗಲೇಬಾರದೆಂದು ದೃಢನಿಶ್ಚಯ ಮಾಡಿಯೂ ಏನೇನೂ ಪ್ರಯೋಜನವಾಗಲಿಲ್ಲವಲ್ಲ, ಹೇಗದು! ಅಧ್ಯಾತ್ಮದ ಬಗೆಗಿನ ಅವನ ಹಿಂದಿನ ಕಲ್ಪನೆಗಳ ಕಟ್ಟಡ ಬಿರುಕುಬಿಟ್ಟಿತು. ಅವನೀಗ ಆಲೋಚಿಸತೊಡಗಿದ–‘ನಾನೆಂದುಕೊಂಡಿದ್ದಂತೆ ಶ್ರೀರಾಮಕೃಷ್ಣರು ಹುಚ್ಚರಂತೂ ಆಗಿರಲು ಸಾಧ್ಯವಿಲ್ಲ. ಮಾತ್ರವಲ್ಲ, ಅವರು ದೈವೀಶಕ್ತಿಸಂಪನ್ನರಾದ ಮಹಾಪುರುಷರೇ ಆಗಿರಬೇಕು. ನನ್ನಂತಹ ಯಾವ ವ್ಯಕ್ತಿಯ ಮನಸ್ಸನ್ನಾದರೂ ಇಚ್ಛಾಮಾತ್ರದಿಂದ ಪರಿವರ್ತಿಸಿ, ಉನ್ನತ ಸ್ತರಕ್ಕೇರಿಸುವ ಶಕ್ತಿ ಅವರಿಗಿರುವುದು ನಿಸ್ಸಂಶಯ. ಇಂತಹ ಮಹಾತ್ಮರ ಕೃಪೆಗೊಳಗಾಗಬೇಕಾದರೆ ಅದು ನನ್ನ ಪರಮಸೌಭಾಗ್ಯವೇ ಸರಿ.’

ಕಾಮಕಾಂಚನಗಳ ಬೆಡಗಿಗೆ ಮರುಳಾಗಿ ತಲೆಕೆಟ್ಟು ನಡೆದುಕೊಳ್ಳುವ ಲೋಕದ ಕೋಟಿಕೋಟಿ ಹುಚ್ಚು ಜನಗಳ ಮಧ್ಯೆ ತಲೆ ಸರಿಯಾಗಿರುವವರು ಇವರೊಬ್ಬರೇ ಎಂದು ಅವನಿಗನ್ನಿಸತೊಡಗಿತು. ಆದರೆ ಶ್ರೀರಾಮಕೃಷ್ಣರು ಆ ಮೊದಲ ಭೇಟಿಯಲ್ಲಿ ತನ್ನೊಡನೆ ವರ್ತಿಸಿದ ರೀತಿ ಮಾತ್ರ ಅವನಿಗೆ ಒಗಟಾಗಿ ತೋರುತ್ತಿತ್ತು. ತಾನು ಶ್ರೀರಾಮಕೃಷ್ಣರ ಲೀಲಾಸಹಚರನಾಗಿದ್ದಕೊಂಡು, ಅವರ ಅದ್ಭುತಕಾರ್ಯಗಳಲ್ಲಿ ವಹಿಸಬೇಕಾದ ಮಹತ್ತರ ಪಾತ್ರವನ್ನು ಅವನು ಕಾಲಾಂತರದಲ್ಲಿ –ಇನ್ನೂ ಹಲವಾರು ಪರೀಕ್ಷೆ-ನಿರೀಕ್ಷೆಗಳಾದ ಮೇಲಷ್ಟೇ–ತಿಳಿಯಬೇಕಾಗಿದೆ.

ಅಂತೂ ಶ್ರೀರಾಮಕೃಷ್ಣರ ಆ ದಿವ್ಯ ಸ್ವರ್ಶದ ಮುಹೂರ್ತದಿಂದಲೇ ನರೇಂದ್ರ ಅವರಿಗೆ ಸೇರಿದವನಾಗಿಬಿಟ್ಟ. ಅವನೀಗ ಶ್ರೀರಾಮಕೃಷ್ಣರ ಸ್ವತ್ತು. ಆದರೆ ಆ ಬಗೆಯ ಪರಾಧೀನತೆ ಬಂಧನಕಾರಿಯಲ್ಲ. ಶ್ರೀರಾಮಕೃಷ್ಣರ ಸಂಸ್ವರ್ಶವೆನ್ನುವುದು ಮುಕ್ತಿದಾಯಕವಾದದ್ದು. ಅವನು ಪೂರ್ವಗ್ರಹಗಳೆಂಬ ಬಂಧನಗಳಿಂದ ಮುಕ್ತನಾಗುವ ಸಂದರ್ಭವೊದಗಿದೆ.

ಹಿಂದೂ ಪರಂಪರೆಯಲ್ಲಿ ಗುರು-ಶಿಷ್ಯ ಬಾಂಧವ್ಯಕ್ಕೆ ಮಹತ್ವದ ಸ್ಥಾನ. ಆದರೆ ಹಿಂದೆ, ಆಧ್ಯಾತ್ಮಿಕ ಜೀವನದಲ್ಲಿ ಗುರುವಿಗೊಂದು ಪಾತ್ರವಿದೆಯೆಂದು ಒಪ್ಪಿದವನೇ ಅಲ್ಲ ನರೇಂದ್ರ.‘ಎಷ್ಟಾದರೂ ಮನುಷ್ಯ ಮನುಷ್ಯನೇ; ಅಧ್ಯಾತ್ಮವೋ ಬಹಳ ಸೂಕ್ಷ್ಮ. ಇಂತಹ ಸೂಕ್ಷ್ಮ ವಿಷಯದಲ್ಲಿ ಮನುಷ್ಯ ಸಹಜವಾದ ಸಕಲ ಲೋಪದೋಷಗಳಿಂದಲೂ ಕೂಡಿರುವಂಥವನೊಬ್ಬನು ಇನ್ನೊಬ್ಬ ವ್ಯಕ್ತಿಗೆ ಮಾರ್ಗದರ್ಶನ ನೀಡುವುದೆಂದರೇನು!\\ಕುರುಡ ಕುರುಡನಿಗೆ ದಾರಿತೋರಬಲ್ಲನೆ? ಅಂತಹ ಒಬ್ಬ ‘ಗುರು’ವಿನ ಕೈಯಲ್ಲಿ ಸಿಕ್ಕಿಹಾಕಿಕೊಂಡು, ಅವನು ಗಳಹಿದ್ದನ್ನೆಲ್ಲ ಮರುಮಾತಿಲ್ಲದೆ ಒಪ್ಪಿಕೊಂಡು, ಅವನ ಗುಲಾಮನಾಗಿರುವುದಕ್ಕಿಂತ ಮೂರ್ಖತನ ಇನ್ನೊಂದಿಲ್ಲ’–ಇದು ನರೇಂದ್ರನ ಅಭಿಮತ. ಬ್ರಾಹ್ಮಸಮಾಜದ ಸಂಪರ್ಕದಿಂದಾಗಿ ಅವನ ಈ ನಂಬಿಕೆ ಮತ್ತಷ್ಟು ದೃಢವಾಗಿತ್ತು. ಆದರೆ ಈಗ ಎರಡು ಸಲ ಶ್ರೀರಾಮಕೃಷ್ಣರ ದಿವ್ಯ ಸ್ವರ್ಶವಾಯಿತೋ ಇಲ್ಲವೋ, ಅದು ಬದಲಾಗಿಹೋಯಿತು. ದೇವರೊಬ್ಬನೇ ದಾರಿತೋರಬಲ್ಲವನು ಎಂದು ಅವನು ಹಿಂದೆ ವಾದಿಸುತ್ತಿದ್ದ. ಆದರೆ ಈಗ ಅವನಿಗೆ ಅನಿಸಿತು–ಸಾಧಾರಣ ಜನಗಳ ಬುದ್ಧಿಗೆ ಎಟುಕಬಲ್ಲ ದೇವರ ಕಲ್ಪನೆಯಾದರೂ ಎಷ್ಟು ಸಂಕುಚಿತ! ಪ್ರೇಮ-ಪಾವಿತ್ರ್ಯಗಳಲ್ಲೂ ತ್ಯಾಗ-ತಪಸ್ಸುಗಳಲ್ಲೂ ಅಂತಹ ‘ದೇವರ’ನ್ನೇ ಮೀರಿಸಬಲ್ಲ ವ್ಯಕ್ತಿಗಳು ಈ ಭೂಮಿಯಲ್ಲೇ ಇರಬಲ್ಲರು. ಅಂಥವರ ಸಂಖ್ಯೆ ತೀರ ಅಲ್ಪವಾದರೂ, ಅಂಥವರು ಜನ್ಮತಾಳುವುದಂತೂ ನಿಜ. ಅಂಥ ಮಹಾತ್ಮರು ನಿಜಕ್ಕೂ ಇತರ ಸಾಮಾನ್ಯರಿಗೆ ಗುರುಗಳಾಗಬಲ್ಲರು. ಮತ್ತು ಅವರನ್ನು ಗುರುವಾಗಿ ಸ್ವೀಕರಿಸುವುದರ ಮೂಲಕ ಇತರರು ಧನ್ಯರಾಗುವುದರಲ್ಲಿ ಅಚ್ಚರಿಯಿಲ್ಲ’ ಎಂದು. ಆದ್ದರಿಂದ ಅವನೀಗ ಶ್ರೀರಾಮಕೃಷ್ಣರನ್ನು ಗುರುವೆಂದು ಸ್ವೀಕರಿಸಲು ಮಾನಸಿಕವಾಗಿ ಸಿದ್ಧನಾದ, ಆದರೆ ಒಂದು ಷರತ್ತಿನ ಮೇಲೆ–ತಾನು ತನ್ನ ವಿಚಾರಸ್ವಾತಂತ್ರ್ಯವನ್ನು ಮಾತ್ರ ಎಂದೆಂದಿಗೂ ಬಿಟ್ಟುಕೊಡಲಾರೆ ಎಂದು.

ಈ ವಿಚಾರಸ್ವಾತಂತ್ರ್ಯದ ಚಿಂತೆ ನರೇಂದ್ರನನ್ನು ಒಂದು ಗೀಳಿನಂತೆಯೇ ಹಿಡಿದುಕೊಂಡಿತ್ತು. ಅದಕ್ಕೆ ಪ್ರಬಲ ಕಾರಣವೂ ಇಲ್ಲದಿರಲಿಲ್ಲ. ಅವನು ಮನುಷ್ಯರ ವರ್ತನೆಯನ್ನು ಚೆನ್ನಾಗಿ ಗಮನಿಸಿದ್ದವನು. ಅವನು ತನ್ನೊಳಗೇ ಹೇಳಿಕೊಂಡ–‘ಪ್ರಬಲ ವ್ಯಕ್ತಿತ್ವದ ಇಂತಹ ಮಹಾಪುರುಷರ ಸಂಪರ್ಕಕ್ಕೆ ಬಂದಾಗ ಇತರ ಸಾಮಾನ್ಯರು ತಮ್ಮ ವಿವೇಚನಾಶಕ್ತಿಯನ್ನು ಉಪಯೋಗಿಸಹೋಗದೆ, ಅವರು ಹೇಳುವುದನ್ನು ಅರ್ಧ ಪರೀಕ್ಷೆ ಮಾಡಿ, ಇಲ್ಲವೆ ಪರೀಕ್ಷೆಯನ್ನೇ ಮಾಡದೆ ನಂಬಿಬಿಡುತ್ತಾರೆ. ನಾನು ಮಾತ್ರ ಎಷ್ಟೇ ಕಷ್ಟವಾಗಲಿ, ಇಂತಹ ಅವಿವೇಕಕ್ಕೆ ಎಡೆಗೊಡಲಾರೆ. ಶ್ರಿರಾಮಕೃಷ್ಣರು ಪಡೆದುಕೊಳ್ಳುತ್ತಾರೆಂದು ಹೇಳುವ ದರ್ಶನಗಳನ್ನಾಗಲಿ, ಅನುಭವಗಳನ್ನಾಗಲಿ– ನನಗೆ ಅವು ಲಭ್ಯವಾದಹೊರತು, ಇಲ್ಲವೆ ಆ ಮಾತುಗಳನ್ನು ಅತಿ ತೀಕ್ಷ್ಣವಾದ ಪರೀಕ್ಷೆಗೆ ಒಳಪಡಿಸಿದಹೊರತು–ನಂಬಲಾರೆ. ಇದರಿಂದಾಗಿ ಅವರ ಅಸಮಾಧಾನಕ್ಕೆ ಗುರಿಯಾಗಬೇಕಾದರೂ ಸರಿಯೆ... ’

