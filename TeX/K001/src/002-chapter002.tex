
\chapter{ಧರೆಗಿಳಿದ ನರೇಂದ್ರ}

\noindent

ಭುವನೇಶ್ವರೀ ದೇವಿ ಮದುವೆಯಾಗಿ ಪತಿಗೃಹ ಪ್ರವೇಶ ಮಾಡಿದಾಗ ಅವಳಿಗಿನ್ನೂ ಹತ್ತು ವರ್ಷವಷ್ಟೆ. ಪತಿ ವಿಶ್ವನಾಥ ದತ್ತ ಹದಿನಾರರ ತರುಣ. ಭುವನೇಶ್ವರಿ ತನ್ನ ತಾಯ್ತಂದೆಯರಿಗೆ ಒಬ್ಬಳೇ ಮಗಳು. ಆದ್ದರಿಂದ ಮುಂದೆ ಅವಳ ಪಿತ್ರಾರ್ಜಿತ ಆಸ್ತಿಯೆಲ್ಲ ಅವಳ ಪಾಲಿಗೇ ಬಂದು ಸೇರಿತು. ಹುಡುಗಿ ಭುವನೇಶ್ವರಿ ಬೆಳೆದು ದೊಡ್ಡವಳಾದಂತೆ ಮನೆವಾರ್ತೆಯ ಎಲ್ಲ ಕೆಲಸ ಕಾರ್ಯಗಳನ್ನೂ ತುಂಬ ದಕ್ಷತೆಯಿಂದ ನಿರ್ವಹಿಸಿಕೊಂಡು ಬಂದಳು. ಅದು ಅವಿಭಕ್ತ ಕುಟುಂಬವಾದುದರಿಂದ ಅವಳಿಗೆ ಕೈತುಂಬ ಕೆಲಸ. ಆದರೆ ಅವಳ ಸಾಮರ್ಥ್ಯದ ಮುಂದೆ ಅದೊಂದು ತೀರಾ ಸಣ್ಣ ವಿಷಯ. ಅವಳು ತುಂಬ ಚಾತುರ್ಯದಿಂದ ಕೆಲಸ ಕಾರ್ಯಗಳನ್ನು ನಿರ್ವಹಿಸುತ್ತಿದ್ದುದರಿಂದ ಸಾಕಷ್ಟು ಸಮಯ ಮಿಗುತ್ತಿತ್ತು. ಈ ಬಿಡುವಿನ ವೇಳೆಯಲ್ಲಿ ಅವಳು ಹೊಲಿಗೆಯ ಕೆಲಸ, ಸಂಗೀತಾಭ್ಯಾಸ, ರಾಮಾಯಣ-ಮಹಾಭಾರತಗಳ ಅಧ್ಯಯನ ಮಾಡುತ್ತಿದ್ದಳು. ಅವಳದು ಭವ್ಯ, ಪ್ರಸನ್ನ-ಗಂಭೀರ ವ್ಯಕ್ತಿತ್ವ. ಆದ್ದರಿಂದ ಜನ ಅವಳನ್ನು ಗೌರವಾದರಗಳಿಂದ ಕಾಣದಿರಲು ಸಾಧ್ಯವೇ ಇರಲಿಲ್ಲ. ಸುತ್ತಮುತ್ತಲ ಜನ ಹಲವಾರು ವಿಷಯಗಳಲ್ಲಿ ಅವಳ ಸಹಾಯವನ್ನು ಪಡೆಯುತ್ತಿದ್ದರು. ಇವೆಲ್ಲಕ್ಕಿಂತ ಹೆಚ್ಚಾಗಿ ಆಕೆ ತುಂಬ ಭಕ್ತಿಮತಿ. ತಪ್ಪದೆ ಪ್ರತಿದಿನ ಶಿವಪೂಜೆ ಮಾಡುತ್ತಿದ್ದಳು. ಅವಳು ಅತಿ ಮಾತಿನವಳಲ್ಲ; ಭಗವದಿಚ್ಛೆಗೆ ಮಣಿದು ಶಾಂತ-ಗಂಭೀರ ಜೀವನ ನಡೆಸುತ್ತಿದ್ದ ಉದಾತ್ತ ಮಹಿಳೆ. ಬಿಡುವಿನ ವೇಳೆಯಲ್ಲಿ ದೇವರನಾಮಗಳನ್ನು ಹಾಡಿಕೊಳ್ಳುತ್ತಿದ್ದಳು. ಪತಿ ವಿಶ್ವನಾಥನಂತೆಯೇ ಅವಳಿಗೂ ಮಧುರವಾದ ಕಂಠಶ್ರೀ ಇತ್ತು. ಅವಳ ನೆನಪಿನ ಶಕ್ತಿ ಅಗಾಧವಾದುದು. ರಾಮಾಯಣ-ಮಹಾಭಾರತಗಳ ನೂರಾರು ಶ್ಲೋಕಗಳನ್ನು ಕಂಠಸ್ಥ ಮಾಡಿಬಿಟ್ಟಿದ್ದಳು. ಅದಕ್ಕಿಂತ ಹೆಚ್ಚಾಗಿ, ಈ ಗ್ರಂಥಗಳ ಸಾರವನ್ನು ಚೆನ್ನಾಗಿ ಗ್ರಹಿಸಿದ್ದಳು. ಇವಳ ಇನ್ನೊಂದು ಗುಣವಿಶೇಷವೆಂದರೆ, ಪತಿ ವಿಶ್ವನಾಥನಂತೆಯೇ ಬಡಬಗ್ಗರ ಮೇಲಿನ ಸಹಾನುಭೂತಿ.

ಇವೆಲ್ಲ ಸರಿಯೆ. ಉತ್ತಮ ಪತಿ, ಹೇರಳ ಐಶ್ವರ್ಯ, ಉನ್ನತ ಗುಣನಡತೆ–ಈ ಎಲ್ಲ ಸೌಭಾಗ್ಯಗಳೂ ಭುವನೇಶ್ವರಿಯ ಪಾಲಿಗಿದ್ದುವು. ಆದರೆ ಒಂದೇ ಒಂದು ಚಿಂತೆ ಒಳಗೇ ಕೊರೆಯುತ್ತಿತ್ತು–ಒಂದಾದರೂ ಗಂಡು ಮಗುವಿಲ್ಲವೆಂಬ ಚಿಂತೆ. ಅವಳಿಗಾಗಲೇ ಮೂರು ಮಕ್ಕಳಾಗಿದ್ದುವು, ಆದರೆ ಮೂರೂ ಹೆಣ್ಣು–ಒಬ್ಬಳು ಹರಮೋಹಿನಿ, ಇನ್ನೂಬ್ಬಳು ಸ್ವರ್ಣಮಯಿ; ಮೂರನೆಯವಳು ಬಾಲ್ಯದಲ್ಲೇ ತೀರಿಹೋಗಿದ್ದಳು. ಭುವನೇಶ್ವರಿಗೆ ಹೆಣ್ಣುಮಕ್ಕಳೆಂದು ತಿರಸ್ಕಾರವೇನಿಲ್ಲ; ಆದರೆ ಒಂದು ಪುತ್ರಸಂತಾನ ಬೇಕು ಎನ್ನುವುದು ಬಹುತೇಕ ಮಾತೆಯರ ಹಂಬಲವಲ್ಲವೆ? ತಮ್ಮ ಬಯಕೆಗಳನ್ನು ಕುಲದೇವತೆಯ ಮುಂದಿಟ್ಟು, ಇಲ್ಲವೆ ಆ ವಿಶೇಷ ಬಯಕೆಗಳನ್ನು ಸುಲಭವಾಗಿ ಈಡೇರಿಸಿಕೊಡಬಲ್ಲ ಇನ್ನಾವುದೇ ದೇವತೆಯ ಮುಂದಿಟ್ಟು ಪ್ರಾರ್ಥಿಸಿಕೊಳ್ಳುವುದು, ಅದಕ್ಕೆ ಸಂಬಂಧಪಟ್ಟ ವ್ರತಗಳನ್ನು ಆಚರಿಸುವುದು–ಇದು ಹಿಂದೂಗಳಲ್ಲಿ ಅನಾದಿಯಿಂದ ನಡೆದುಕೊಂಡು ಬಂದ ಪದ್ಧತಿ. ಅಂತೆಯೇ ಭುವನೇಶ್ವರೀ ದೇವಿಯೂ ಈಗ, ತನ್ನ ಇಷ್ಟದೇವತೆಯಾದ ಶಿವನನ್ನು ಪ್ರಾರ್ಥಿಸಿಕೊಳ್ಳಲಾರಂಭಿಸಿದಳು. ಪೂಜೆಯ ವೇಳೆಯಲ್ಲಿ ಮಾತ್ರವಲ್ಲ, ಮನೆವಾರ್ತೆಯ ಕೆಲಸಕಾರ್ಯಗಳನ್ನು ಮಾಡುತ್ತಿರುವಾಗಲೂ ಮನಸ್ಸಿನಲ್ಲೇ, “ ಹೇ ದೇವ, ನನಗೊಂದು ಗಂಡುಸಂತಾನವನ್ನು ದಯಪಾಲಿಸು” ಎಂದು ಪ್ರಾರ್ಥಿಸಿಕೊಳ್ಳುತ್ತಿದ್ದಳು. ಹೀಗಿರುವಾಗ, ಅವಳಿಗೆ ಕಾಶಿಯ ವೀರೇಶ್ವರ ಶಿವನ ನೆನಪಾಯಿತು. ಅವನಿಗೆ ಹರಕೆ ಹೊತ್ತು ಪೂಜೆ ಸಲ್ಲಿಸಿದರೆ ತನ್ನ ಮನೋರಥ ಈಡೇರಬಹುದೆಂಬ ಆಸೆ ಮೂಡಿತು. ಆದರೆ ವಾರಾಣಸಿಗೆ ಹೋಗುವುದೆಂದರೆ ಅಷ್ಟು ಸುಲಭವೆ? ಆದರೆ ಆ ದಿನಗಳಲ್ಲಿ ಒಂದು ಸಂಪ್ರದಾಯ ಪ್ರಚಲಿತವಿತ್ತು. ಏನೆಂದರೆ ದೂರದ ಊರುಗಳಲ್ಲಿರುವವರು ತಮ್ಮ ಸಂಕಲ್ಪಗಳನ್ನು ಸಿದ್ಧಿಸಿಕೊಳ್ಳಲು, ಕಾಶಿಯಲ್ಲಿ ತಮ್ಮ ಪರಿಚಿತರು ಯಾರಾದರೂ ಇದ್ದರೆ ಅವರ ಮೂಲಕ ಕಾಶೀ ವಿಶ್ವನಾಥನಿಗೆ ಹರಕೆ ಸಲ್ಲಿಸಿ ಪೂಜೆ ಕೊಡಿಸಬಹುದಾಗಿತ್ತು. ಕಾಶಿಯಲ್ಲಿ ಭುವನೇಶ್ವರಿಯ ಮುದಿ ಚಿಕ್ಕಮ್ಮ ಒಬ್ಬಳಿದ್ದಳು. ತನ್ನ ಪರವಾಗಿ ವೀರೇಶ್ವರ ಶಿವನಿಗೆ ಪೂಜೆ ಮಾಡಿಸಿ ಪ್ರಾರ್ಥನೆ ಸಲ್ಲಿಸಬೇಕು ಎಂದು ಕೇಳಿಕೊಂಡು ಭುವನೇಶ್ವರಿ ತನ್ನ ಚಿಕ್ಕಮ್ಮನಿಗೊಂದು ಪತ್ರ ಬರೆದಳು. ಅದಕ್ಕೆ ಅವಳ ಚಿಕ್ಕಮ್ಮ ಒಪ್ಪಿಕೊಂಡು, “ಒಂದು ವರ್ಷಕಾಲ ನಾನಿಲ್ಲಿ ಪ್ರತಿ ಸೋಮವಾರ ಶಿವನಿಗೆ ಪೂಜೆ ಸಲ್ಲಿಸುತ್ತೇನೆ, ನೀನೂ ಆ ಸಮಯಕ್ಕೆ ಸರಿಯಾಗಿ ಮನೆಯಲ್ಲೇ ವಿಶೇಷ ವ್ರತನಿಯಮಗಳನ್ನಾಚರಿಸಿ ಪ್ರಾರ್ಥನೆ ಮಾಡುತ್ತಿರು” ಎಂದು ಉತ್ತರ ಬರೆದಳು. ಭುವನೇಶ್ವರಿಯ ಆನಂದಕ್ಕೆ ಪಾರವೇ ಇಲ್ಲವಾಯಿತು. ಸೋಮವಾರ ಬಂದಿತೆಂದರೆ ತನ್ನ ಚಿಕ್ಕಮ್ಮ ಅಲ್ಲಿ ವಾರಾಣಸಿಯಲ್ಲಿ ವೀರೇಶ್ವರ ಶಿವನಿಗೆ ಪುಷ್ಪಾರ್ಚನೆ ಮಾಡಿಸಿ ಪ್ರಾರ್ಥನೆ ಸಲ್ಲಿಸುತ್ತಿರುವಾಗ ಇಲ್ಲಿ ಭುವನೇಶ್ವರಿಯೂ ಶಿವಪೂಜೆಯಲ್ಲಿ ನಿರತಳಾಗುತ್ತಿದ್ದಳು. ಅವಳ ಮನಸ್ಸೆಲ್ಲ ವಾರಾಣಸಿಗೆ ಹಾರಿಹೋಗಿ ಅಲ್ಲಿ ಚಿಕ್ಕಮ್ಮನ ಜೊತೆ ಸೇರಿ ವೀರೇಶ್ವರ ಶಿವನ ಪೂಜೆಯಲ್ಲಿ ಭಾಗಿಯಾಗುತ್ತಿತ್ತು. ಸೋಮವಾರಗಳಂದು ಮಾತ್ರವಲ್ಲದೆ ಅವಳು ಪ್ರತಿದಿನವೂ ಪೂಜೆ ಮಾಡುತ್ತಿದ್ದಳು. ಉಪವಾಸ ವ್ರತಗಳನ್ನು ಆಚರಿಸುತ್ತಿದ್ದಳು. ಹೀಗೆಯೇ ಕಳೆಯಿತು ಒಂದು ವರ್ಷ.

ಅಂದು ಆ ವರ್ಷದ ಕೊನೆಯ ದಿನ. ಆ ದಿನ ಬೆಳಗ್ಗೆ ಭುವನೇಶ್ವರಿ ಇನ್ನಷ್ಟು ವಿಶೇಷವಾಗಿ ವೀರೇಶ್ವರ ಶಿವನ ಧ್ಯಾನದಲ್ಲಿ ತಲ್ಲೀನಳಾದಳು. ಒಂದು ವರ್ಷದ ನಿರಂತರ ಭಕ್ತಿಸಾಧನೆಯ ಪರಿಣಾಮವಾಗಿ ಅವಳ ಮನಸ್ಸು ಶಿವಮಯವೇ ಆಗಿ ಧ್ಯಾನಲೀನವಾಗಿಬಿಟ್ಟಿತು. ಅನಂತ ಶಕ್ತಿಶಾಲಿಯಾದ ವೀರೇಶ್ವರ ಶಿವ ಅವಳ ಹೃದಯದಲ್ಲಿ ತಾನೇತಾನಾಗಿ ಬೆಳಗುತ್ತಿದ್ದ. ಅಂದು ಹಗಲೆಲ್ಲ ಶಿವಪೂಜೆ-ಶಿವಧ್ಯಾನಗಳಲ್ಲೇ ಕಳೆಯಿತು. ಹಗಲು ಕಳೆದು ರಾತ್ರಿಯಾಯಿತು. ಭುವನೇಶ್ವರಿಯ ಜೀವನದ ಅತ್ಯಂತ ಶುಭರಾತ್ರಿ ಅದು. ಅವಳು ಶಿವಸ್ಮರಣೆ ಮಾಡುತ್ತಲೇ ಮಲಗಿಕೊಂಡಳು; ನಿದ್ರೆಯ ಮೌನದಲ್ಲಿ ಮುಳುಗಿದಳು. ಆಗ ದೇವಲೋಕದಲ್ಲಿ ಒಂದು ದಿವ್ಯ ಶುಭಮುಹೂರ್ತ ಸಂಘಟಿಸಿತು. ಆಗಲೇ ಭುವನೇಶ್ವರಿ ಶಿವಕೃಪೆ ಹೊಂದುವ ಕಾಲವೂ ಪ್ರಾಪ್ತವಾಯಿತು. ಆಕೆಯ ಭಕ್ತಿಯುತ ಪೂಜೆಗೆ ಒಲಿದ ವೀರೇಶ್ವರ ಶಿವ ತನ್ನ ಅನಂತ ಧ್ಯಾನದಿಂದೆದ್ದು ಬಂದು ಅವಳಿಗೆ ದರ್ಶನ ನೀಡಿದ. ಅವಳು ಆ ಜ್ಯೋತಿರ್ಮಯ ಶಿವನ ಪಾದವನ್ನು ಮುಟ್ಟಿ ನಮಸ್ಕರಿಸಿದಳು. ತಕ್ಷಣ ಶಿವನ ಆ ಆಕೃತಿ ಒಂದು ಗಂಡು ಮಗುವಿನ ರೂಪಕ್ಕೆ ಪರಿವರ್ತಿತವಾಯಿತು. ತಾನು ಅವಳ ಮಗನಾಗಿ ಜನಿಸುತ್ತೇನೆ ಎಂಬುದರ ಸಂಕೇತವೇ ಇದು! ತನ್ನ ಹೃತ್ಪೂರ್ವಕ ಪ್ರಾರ್ಥನೆಗೆ ಸ್ವಯಂ ವೀರೇಶ್ವರ ಶಿವನೇ ಈ ಸ್ವಪ್ನದರ್ಶನದ ಮೂಲಕ ಉತ್ತರವನ್ನು ಪ್ರಕಟಪಡಿಸಿದ್ದಾನೆ ಎಂದು ಭುವನೇಶ್ವರಿಗೆ ನಂಬಿಕೆಯಾಯಿತು. ಅವಳ ಶ್ರದ್ಧೆ ಫಲಿಸಿತು. ಸಕಾಲದಲ್ಲಿ ಆಕೆ ಪುತ್ರಪ್ರಸವಕ್ಕೆ ಸಿದ್ಧಳಾದಳು.

೧೮೬೩ನೇ ಇಸವಿ ಜನವರಿ ತಿಂಗಳ ೧೨, ಸೋಮವಾರದಂದು ಭುವನೇಶ್ವರೀ ದೇವಿ ಸಮಸ್ತ ಭುವನವನ್ನೇ ಬೆಳಗಲಿರುವ ದಿವ್ಯಜ್ಯೋತಿಯನ್ನು ಧರೆಗೆ ನೀಡಿದಳು. ಸ್ವಾಮಿ ವಿವೇಕಾನಂದ ಎಂಬ ಪ್ರಖ್ಯಾತ ನಾಮದಿಂದ ಜ್ವಲಿಸಬೇಕಾದ ಆ ಜ್ಯೋತಿ, ಸೂರ್ಯೋದಯಕ್ಕೆ ಸ್ವಲ್ಪ ಮೊದಲು ಭುವಿಯನ್ನು ಸೋಂಕಿತು. ಸ್ವಯಂ ಜ್ಯೋತಿಷ ಬಲ್ಲ ವಿಶ್ವನಾಥ ಜಾತಕ ನೋಡಿದ–ಅತ್ಯಂತ ಶ್ರೇಷ್ಠ ತಾರಾಬಲ, ಗ್ರಹಬಲ. ಅಂದು ಮಕರಸಂಕ್ರಾಂತಿಯ ಶುಭದಿನ, ಹಿಂದುಗಳ ಪರ್ವದಿನ. ಕೋಟಿಗಟ್ಟಲೆ ಜನ ಅಂದು ಆ ಶುಭದಿನದಲ್ಲಿ ಪೂಜೆ-ಪ್ರಾರ್ಥನೆಗಳನ್ನು ಸಲ್ಲಿಸುತ್ತಿದ್ದರು. ನಿಜಕ್ಕೂ ತಮಗರಿವಿಲ್ಲದೆಯೇ ಅವರೆಲ್ಲ ತಮ್ಮ ಪೂಜೆ-ಪ್ರಾರ್ಥನೆಗಳ ಮೂಲಕ ಆ ನವಜಾತ ಶಿಶುವನ್ನು ಸ್ವಾಗತಿಸಿದರೆನ್ನಬಹುದು. ಹಾಗೆಯೇ ಇತ್ತಕಡೆ ಕಲ್ಕತ್ತದಿಂದ ಕೆಲವೇ ಮೈಲಿ ದೂರದಲ್ಲಿರುವ ದಕ್ಷಿಣೇಶ್ವರದ ಕಾಳೀದೇವಾಲಯದಲ್ಲಿ, ಭೂತ-ವರ್ತಮಾನ-ಭವಿಷ್ಯತ್ಕಾಲಗಳನ್ನೆಲ್ಲ ಏಕಕಾಲದಲ್ಲಿ ಕಾಣುತ್ತಿರುವ ಪರಮಪುಷಿಯೊಬ್ಬನು, ತಾನು ಈ ಜಗತ್ತಿನಲ್ಲಿ ಮಾಡುತ್ತಿರುವ ಮಹಾಕಾರ್ಯವನ್ನು ಮುಂದುವರಿಸಿಕೊಂಡು ಹೋಗುವುದಕ್ಕಾಗಿ ಅವತರಿಸಿದ ಈ ನವಜಾತ ಶಿಶುವಿನ ಆಗಮನವನ್ನೇ ನಿರೀಕ್ಷಿಸಿಕೊಂಡು ಕುಳಿತಿದ್ದ.

ಮಗುವಿನ ರೂಪಲಾವಣ್ಯವನ್ನು, ಅದರ ಮೈಕಟ್ಟನ್ನು ನೋಡಿ ಮನೆಮಂದಿಯೆಲ್ಲ ವಿಸ್ಮಯಾನಂದಭರಿತರಾದರು. ಮಗು ಹಲವು ವಿಧದಲ್ಲಿ ತನ್ನ ತಾತನಾದ ದುರ್ಗಾಪ್ರಸಾದನನ್ನೇ ಹೋಲುತ್ತಿರುವುದನ್ನು ಕಂಡು ಅವನೇ ಈಗ ಮತ್ತೊಮ್ಮೆ ಜನ್ಮ ತಾಳಿ ಬಂದಿರಬಹುದೇ ಹೇಗೆ, ಎಂಬ ಆಶ್ಚರ್ಯ ಅವರಿಗೆಲ್ಲ. ನಾಮಕರಣದ ಸಂದರ್ಭ ಬಂದಾಗ ಮನೆಮಂದಿಯೊಳಗೆ ಒಂದು ದೊಡ್ಡ ಚರ್ಚೆಯೇ ನಡೆಯಿತು. ಮಗುವಿಗೆ ‘ದುರ್ಗಾಪ್ರಸಾದ’ ಎಂದು ಅದರ ತಾತನ ಹೆಸರನ್ನೇ ಇಡುವುದೆಂಬ ಅಭಿಪ್ರಾಯಮೂಡಿತು. ಆದರೆ ಭುವನೇಶ್ವರಿಗೆ ಇದು ಒಪ್ಪಿಗೆಯಾಗಲಿಲ್ಲ. ಅವಳು ಮಗುವಿನ ಮುಖವನ್ನೇ ನಿಟ್ಟಿಸಿದಳು; ಅದರ ಕಣ್ಣೊಳಗೆ ಕಣ್ಣಿಟ್ಟು ನೋಡಿದಳು; “ಇವನು ವೀರೇಶ್ವರ ಶಿವನ ವರಪ್ರಸಾದವಲ್ಲವೇನು? ಆದ್ದರಿಂದ ಇವನ ಹೆಸರು ‘ವೀರೇಶ್ವರ’ ಎಂದೇ ಆಗಬೇಕು.” ಎಲ್ಲರಿಗೂ ಇದು ಒಪ್ಪಿಗೆಯಾಯಿತು. ವೀರೇಶ್ವರ ಎಂದು ನಾಮಕರಣವಾಯಿತು. ಮುದ್ದಿಗೆ ಈ ಹೆಸರು ‘ಬಿಲೇ’ ಎಂದಾಯಿತು. ಆದರೆ ಮುಂದೆ ‘ನರೇಂದ್ರನಾಥ’ ಎಂಬ ಹೆಸರು ರೂಢಿಗೆ ಬಂತು.

ಈ ಮುದ್ದಾದ, ಗುಂಡುಗುಂಡಾದ ಮಗುವನ್ನು ತೋಳಲ್ಲೆತ್ತಿಕೊಂಡಾಗ ತಾಯಿ ಭುವನೇಶ್ವರಿಯ ಎದೆ ಹೆಮ್ಮೆಯಿಂದ ಬೀಗುತ್ತಿತ್ತು. ಕಂಗಳು ತೇವವಾಗುತ್ತಿದ್ದುವು. ಸುದೀರ್ಘ ಪ್ರಾರ್ಥನಾದಿಗಳ ಫಲವಾಗಿ ಮಡಿಲಿಗೆ ಬಂದ ಮುದ್ದು ಮಗುವಲ್ಲವೇ ಅದು! ಆದ್ದರಿಂದಲೇ ಆ ಹೆಮ್ಮೆ ಧನ್ಯತೆಯ ಆನಂದಾಶ್ರು. ಆದರೆ ಈತ ಸಾಮಾನ್ಯ ಮಗುವಲ್ಲ. ಒಳಗೆ ಅಸಾಧಾರಣ ಶಕ್ತಿ ತುಂಬಿಕೊಂಡಿದೆ. ಆದ್ದರಿಂದಲೇ ವಿಪರೀತ ಚೇಷ್ಟೆ. ಮನೆಯ ತುಂಬ ಅವನದೇ ಕಾರುಬಾರು. ಆ ದೊಡ್ಡ ಮನೆಯಲ್ಲಿ ಅವನು ಮುಟ್ಟದ ಮೂಲೆಗಳೇ ಇಲ್ಲ. ಒಟ್ಟಿನಲ್ಲಿ ಅವನು ಮನೆ ತುಂಬಿಬಿಟ್ಟಿದ್ದಾನೆ. ಅವನಿನ್ನೂ ಮೂರು ವರ್ಷದ ಬಾಲನಾಗಿರುವಾಗಲೇ ಮನೆಯ ನಾರೀಕುಲದಿಂದ ಭುವನೇಶ್ವರಿಗೆ ದೂರುಗಳು ಬರಲಾರಂಭವಾಯಿತು. ನಿಜ, ಗಂಡು ಮಗು, ಕೇವಲ ಗಂಡಲ್ಲ. ಮುಂದೆ ಗಂಡುಗಲಿಯಾಗಲಿರುವ ಧೀರ ಗಂಡು! ಆದ್ದರಿಂದ ಬಾಲ್ಯದಲ್ಲಿ ಅವನ ತಂಟೆ ಕೂಡ ಅದೇ ಪರಿಮಾಣದಲ್ಲಿದ್ದರೆ ಆಶ್ಚರ್ಯವೇನಿದೆ?

ಆದರೆ ಅವನ ತಂಟೆಯ ಭರಾಟೆಯನ್ನು ನಿಭಾಯಿಸುವವರಿಗೆ ಮಾತ್ರ ಸಾಕು ಸಾಕಾಗುತ್ತಿತ್ತು. ಕೆಲವು ಸಲವಂತೂ ಅವನ ತಂಟೆ ಮಿತಿಮೀರುತ್ತಿತ್ತು. ಆಗ ಮನೆಮಂದಿಯೆಲ್ಲ ಸುಸ್ತಾಗಿಬಿಡುತ್ತಿದ್ದರು. ಹೆದರಿಸಿಯೂ ನೋಡುತ್ತಿದ್ದರು. ಆದರೆ ಯಾವುದಕ್ಕೂ ಜಗ್ಗುವವನಲ್ಲ ನರೇಂದ್ರ. ಇಂತಹ ಮಗನನ್ನು ಪಡೆಯಲು ಭುವನೇಶ್ವರಿ ಅಷ್ಟೇ ತಪಸ್ಸು ಮಾಡಬೇಕಾಯಿತು! ಕೊನೆಗೆ ಅವಳೊಂದು ಉಪಾಯ ಕಂಡುಕೊಂಡಳು. ನರೇಂದ್ರ ರಂಪಾಟ ಮಾಡಿದಾಗಲೆಲ್ಲ ‘ಶಿವ, ಶಿವ!’ ಎಂದು ಉಚ್ಚರಿಸುತ್ತ ಅವನ ತಲೆಯ ಮೇಲೊಂದು ಕೊಡ ತಣ್ಣೀರು ಸುರಿಯುತ್ತಿದ್ದಳು. ಆಶ್ಚರ್ಯ! ತಕ್ಷಣವೇ ಅವನ ರಂಪಾಟ ನಿಲ್ಲುತ್ತಿತ್ತು! ನಿಜ; ಅವನು ವೀರೇಶ್ವರ ಶಿವನಲ್ಲವೇ? ಆದ್ದರಿಂದ ಜಪ ಮಾಡುತ್ತ ಕುಂಭಾಭಿಷೇಕ ಮಾಡಿದಾಗಲೇ ಅವನು ಶಾಂತನಾಗುವುದು! ಇನ್ನು ಕೆಲವು ಸಲ ಅವಳು, “ನೋಡು, ನೀನು ಇದೇ ರೀತಿ ರಂಪ ಮಾಡುತ್ತಿದ್ದರೆ ಶಿವ ನಿನ್ನನ್ನು ಕೈಲಾಸಕ್ಕೆ ಸೇರಿಸುವುದಿಲ್ಲ!” ಎಂದು ಹೆದರಿಸುತ್ತಿದ್ದಳು. ಈ ಮಾತು ಮಂತ್ರದಂತೆ ಕೆಲಸ ಮಾಡುತ್ತಿತ್ತು; ಅವನು ಕೂಡಲೇ ತಣ್ಣಗಾಗಿಬಿಡುತ್ತಿದ್ದ. ಆದರೂ ಈ ತುಂಟ ನರೇಂದ್ರನನ್ನು ನೋಡಿಕೊಳ್ಳಲು ಭುವನೇಶ್ವರಿ ಒಬ್ಬರಲ್ಲ, ಇಬ್ಬರು ದಾದಿಯರನ್ನು ನೇಮಕ ಮಾಡಿಕೊಳ್ಳಬೇಕಾಯಿತಂತೆ! ಆದರೆ ಅವನು ಅಷ್ಟು ತುಂಟನಾದರೂ ಎಲ್ಲರಿಗೂ ಪ್ರಿಯನಾಗಿದ್ದ. ಅವನಲ್ಲೊಂದು ಮಾಧುರ್ಯವಿತ್ತು; ಆಕರ್ಷಣೆಯಿತ್ತು. ಹಳಬರು-ಹೊಸಬರು, ಪರಿಚಿತರು-ಅಪರಿಚಿತರು ಎಂಬ ಭೇದವೇ ಇಲ್ಲದೆ ಯಾರು ಕರೆದರೆ ಅವರ ಬಳಿಗೆ ಹೋಗಿರುತ್ತಿದ್ದ.

ಸಾಧುಗಳನ್ನು, ದಾಸಯ್ಯಗಳನ್ನು ಕಂಡರೆ ನರೇಂದ್ರನಿಗೆ ಅದೇನೋ ವಿಶೇಷ ಆಸಕ್ತಿ. ಮನೆಯ ಬಾಗಿಲಿಗೆ ಸಾಧುಗಳು ಬಂದರೆ ಸಾಕು, ಉತ್ಸಾಹದಿಂದ ಅವರ ಬಳಿಗೋಡುತ್ತಿದ್ದ; ಕೈಗೆ ಸಿಕ್ಕಿದ್ದನ್ನು ತೆಗೆದು ಕೊಟ್ಟುಬಿಡುತ್ತಿದ್ದ. ಒಂದು ದಿನ ಹೀಗೇ ಒಬ್ಬ ಸಾಧು ಬಂದು ಭಿಕ್ಷೆ ಕೇಳಿದ. ನರೇಂದ್ರ ನೋಡಿದ, ಆಗ ಅವನ ಹತ್ತಿರ ಇದ್ದುದೆಂದರೆ ಅವನು ಸುತ್ತಿಕೊಂಡಿದ್ದ ಪುಟ್ಟ ಧೋತಿ ಮಾತ್ರ (ಬಂಗಾಳದಲ್ಲಿ ಪುಟ್ಟ ಹುಡುಗರಿಗೂ ಧೋತಿ ಉಡಿಸುತ್ತಾರೆ). ಅದು ಕಸೂತಿ ಹಾಕಿದ್ದ ಒಳ್ಳೆಯ ಧೋತಿ. ಅವನಿಗೆ ಅದರ ಮೇಲೆ ತುಂಬ ಅಭಿಮಾನ: ಏಕೆಂದರೆ ಅವನು ಈಚೆಗೆ ಬಾಲ್ಯವನ್ನು ದಾಟಿ ಕೌಮಾರ್ಯಕ್ಕೆ ಕಾಲಿಟ್ಟಾಗ ಉಡಿಸಿದ್ದ ಧೋತಿ ಅದು. ಆದರೆ ಸಾಧು ಭಿಕ್ಷೆ ಕೇಳಿದಾಗ ಹಿಂದೆಮುಂದೆ ನೋಡದೆ, ಅದನ್ನೇ ಬಿಚ್ಚಿ ಕೊಟ್ಟುಬಿಟ್ಟ! ಸಾಧು ಸಂತೋಷದಿಂದ ಆ ಧೋತಿಯನ್ನು ತಲೆಗೆ ಸುತ್ತಿಕೊಂಡು ಹುಡುಗನನ್ನು ಆಶೀರ್ವದಿಸುತ್ತ ಹೊರಟು ಹೋದ. ಹುಡುಗ ಕೈಬೀಸಿಕೊಂಡು ತಾಯಿಯ ಹತ್ತಿರಕ್ಕೆ ಬಂದ–ಬರಿ ಮೈ! “ಬಿಲೇ, ಧೋತಿ ಎಲ್ಲಿ?” ಎಂದು ತಾಯಿ ಕೇಳಿದಳು. ನರೇಂದ್ರ ಸಲೀಸಾಗಿ ಹೇಳಿಬಿಟ್ಟ: “ಸಾಧು ಬಂದು ಭಿಕ್ಷೆ ಕೇಳಿದ, ನಾನದನ್ನು ಕೊಟ್ಟುಬಿಟ್ಟೆ!” ಆಗ ಆ ತಾಯಿಯ ಮನಃಸ್ಥಿತಿ ಏನಾಗಿರಬಹುದೆಂಬುದನ್ನು ನಾವೇ ಊಹಿಸಿ ನೋಡಬೇಕು.

ವಿಶ್ವನಾಥ ದತ್ತನ ಮನೆಗೆ ಎಷ್ಟೋ ಜನ ಸಾಧುಗಳು-ಭಿಕ್ಷುಕರು ಬರುತ್ತಿದ್ದರು. ಏಕೆಂದರೆ ವಿಶ್ವನಾಥ ತಮ್ಮನ್ನೆಂದಿಗೂ ಬರಿಗೈಯಲ್ಲಿ ಹಿಂದಿರುಗಿಸುವುದಿಲ್ಲ ಎಂದು ಅವರಿಗೆಲ್ಲ ಗೊತ್ತಾಗಿಬಿಟ್ಟಿತ್ತು. ವಿಶ್ವನಾಥ, ಭುವನೇಶ್ವರಿ ಇಬ್ಬರೂ ಉದಾರಿಗಳೇ. ಆದರೆ ಯಾವಾಗ ನರೇಂದ್ರ ಹೀಗೆ ಸಿಕ್ಕಿದ್ದನ್ನೆಲ್ಲ ಎತ್ತೆತ್ತಿ ಕೊಡುವುದಕ್ಕೆ ಶುರುಮಾಡಿದನೋ ಆಗ ಅವನನ್ನು ಸ್ವಲ್ಪ ಎಚ್ಚರದಿಂದ ನೋಡಿಕೊಳ್ಳಲು ತಾಯಿ ಕಣ್ಣಿಟ್ಟಿರಬೇಕಾಯಿತು. ಯಾರಾದರೂ ಸಾಧು ಮನೆಬಾಗಿಲಿಗೆ ಬಂದರೆ ಭುವನೇಶ್ವರಿ ಮಾಡುತ್ತಿದ್ದ ಮೊದಲ ಕೆಲಸವೆಂದರೆ ನರೇಂದ್ರನನ್ನು ಕೋಣೆಯೊಳಗೆ ಕೂಡಿಟ್ಟು ಅಗಳಿ ಹಾಕುವುದು! ಆದರೆ ಅವನು ಮಾತ್ರ ತಾನು ಮಾಡುವುದನ್ನು ಬಿಡಲೇ ಇಲ್ಲ. ಒಂದು ದಿನ ಒಬ್ಬ ಬೈರಾಗಿ ಬಂದು ಭಿಕ್ಷೆ ಕೇಳಿದ. ಕೂಡಲೇ ನರೇಂದ್ರನನ್ನು ಅವನ ತಾಯಿ ಮಹಡಿಯ ಮೇಲೆ ಒಂದು ಕೊಠಡಿಯಲ್ಲಿ ಕೂಡಿಹಾಕಿದಳು. ನರೇಂದ್ರ ಚಡಪಡಿಸಿದ. ಬೈರಾಗಿಗೆ ಏನನ್ನಾದರೂ ಕೊಟ್ಟ ಹೊರತು ಅವನಿಗೆ ಸಮಾಧಾನವಿಲ್ಲ. ಅತ್ತಿತ್ತ ನೋಡಿದ. ತನ್ನ ತಾಯಿಯ ಸೀರೆಯೊಂದು ಕಂಡಿತು–ಅದೂ ಒಳ್ಳೇ ಬೆಲೆಬಾಳುವ ಸೀರೆ. ಸರಿ ಅದನ್ನೇ ತೆಗೆದು ಕಿಟಕಿಯಿಂದ ಹೊರಕ್ಕೆ ಹಾಕಿಬಿಟ್ಟ ಮಹಾರಾಯ! ಬೈರಾಗಿ ಅದನ್ನು ತೆಗೆದುಕೊಂಡು ಹೊರಟ. ಇಂಥ ಘಟನೆಗಳು ಅದೆಷ್ಟೋ! ಆಗೆಲ್ಲ ಮನೆಯವರು ಯಾರಾದರೂ, ಆ ಭಿಕ್ಷುಕರನ್ನು ಹುಡುಕಿಕೊಂಡು ಹೋಗಿ ಅವರಿಗೆ ಇನ್ನೇನಾದರೂ ಭಿಕ್ಷೆ ಹಾಕಿ ಆ ವಸ್ತುವನ್ನು ಬಿಡಿಸಿಕೊಂಡು ಬರುತ್ತಿದ್ದರು.

ತುಂಟ ನರೇಂದ್ರನನ್ನು ಪಳಗಿಸುವ ಕೆಲಸ ಭುವನೇಶ್ವರಿಗೆ ಸುಲಭವಾಗಿರಲಿಲ್ಲ. ತನ್ನ ಅಕ್ಕಂದಿರನ್ನು ಗೋಳುಹೊಯ್ದುಕೊಳ್ಳುವುದೆಂದರೆ ಅವನಿಗೆ ಬಲು ಪ್ರೀತಿ. ಸಂದರ್ಭ ಸಿಕ್ಕಾಗಲೆಲ್ಲ ಅಕ್ಕಂದಿರನ್ನು ಛೇಡಿಸಿ ಛೇಡಿಸಿ ಇಡುತ್ತಿದ್ದ. ಅಕ್ಕಂದಿರು ರೇಗಿ ಅಟ್ಟಿಸಿಕೊಂಡು ಬಂದರೆ ಓಡಿಹೋಗಿ ರಸ್ತೆ ಪಕ್ಕದ ಮೋರಿಯೊಳಗೆ ನಿಂತುಬಿಡುತ್ತಿದ್ದ. ಆ ಮೈಲಿಗೆಯ ಜಾಗಕ್ಕೆ ಈ ಅಕ್ಕಂದಿರು ಹೋಗಲಾರರು. ನರೇಂದ್ರ ಅಲ್ಲಿಂದಲೇ ಹಲ್ಲು ಕಿರಿದು ಅಣಕಿಸುತ್ತ “ಎಲ್ಲಿ ಹಿಡೀರಿ ನೋಡೋಣ ನನ್ನನ್ನು!” ಎನ್ನುತ್ತಿದ್ದ. ಇದರಿಂದ ಅವನ ಅಕ್ಕಂದಿರಿಗೆ ಇನ್ನಷ್ಟು ಕೋಪ ಬರುತ್ತಿತ್ತು. ಆದರೆ ಏನು ಮಾಡಲೂ ಸಾಧ್ಯವಾಗದೆ ಕೋಪವನ್ನು ನುಂಗಿಕೊಂಡು ಹಿಂದಿರುಗಬೇಕಾಗುತ್ತಿತ್ತು. ಹೀಗೆ ಕೊನೆಯಲ್ಲಿ ಯಾವಾಗಲೂ ನರೇಂದ್ರನೇ ಜಯಶಾಲಿ!

ಬಾಲಕ ನರೇಂದ್ರನ ಸ್ನೇಹಿತರಲ್ಲಿ ಕೆಲವು ಪ್ರಾಣಿ ಪಕ್ಷಿಗಳೂ ಇದ್ದುವು. ಒಂದು ಕೋತಿ, ಒಂದು ಆಡು, ಒಂದು ನವಿಲು, ಕೆಲವು ಪಾರಿವಾಳಗಳು, ಗಿಳಿಗಳು ಮತ್ತು ಒಂದೆರಡು ಬಿಳಿ ಇಲಿಗಳು–ಇವೆಲ್ಲ ಅವನ ಮುದ್ದಿನ ಸಂಗಾತಿಗಳು. ಬಿಳಿ ಇಲಿಗಳ ಕುತ್ತಿಗೆಗೆ ಅವನು ಪುಟ್ಟ ಗಂಟೆಗಳನ್ನು ಕಟ್ಟಿದ್ದ. ಅವನ ಮನೆಯಲ್ಲೊಂದು ಹಸುವಿತ್ತು. ಅದರ ಮೇಲೆ ಅವನಿಗೆ ವಿಶೇಷ ಪ್ರೀತಿ. ಅದನ್ನು ತನ್ನ ಪುಟ್ಟಪುಟ್ಟ ಕೈಗಳಿಂದ ಸವರಿ ಪ್ರೀತಿಯಿಂದ ಮಾತನಾಡಿಸುತ್ತಿದ್ದ. ಹಬ್ಬದ ದಿನಗಳಲ್ಲಿ ತನ್ನ ಅಕ್ಕಂದಿರು ಅದನ್ನು ಅಲಂಕರಿಸಿ ಅದಕ್ಕೆ ತಿಂಡಿ ಕೊಡುವಾಗ ತಾನೂ ಹೋಗಿ ಅದಕ್ಕೆ ಹಾರ ಹಾಕಿ ಹಣೆಗೆ ಅರಸಿನ-ಕುಂಕುಮ ಹಚ್ಚಿ ಬಗ್ಗಿ ನಮಸ್ಕರಿಸುತ್ತಿದ್ದ.

ವಿಶ್ವನಾಥ ದತ್ತನ ಮನೆಯಲ್ಲಿ ಹಲವಾರು ಜನ ಆಳುಕಾಳುಗಳಿದ್ದರು. ಅವರಲ್ಲೊಬ್ಬ ಸಾರೋಟು ಸವಾರ. ಅವನೆಂದರೆ ನರೇಂದ್ರನಿಗೆ ತುಂಬ ಅಚ್ಚುಮೆಚ್ಚು. ಅವನು ತಲೆಗೆ ಪೇಟ ಸುತ್ತಿಕೊಂಡು ಮೈತುಂಬ ಗರಿಮುರಿ ಬಟ್ಟೆ ತೊಟ್ಟುಕೊಂಡು, ಎತ್ತರದಲ್ಲಿ ಕುಳಿತು ಕೈಯಲ್ಲಿ ಚಾವಟಿ ಹಿಡಿದುಕೊಂಡು “ಹೇಯ್, ಹೇಯ್!” ಎನ್ನುತ್ತ ಠೀವಿಯಿಂದ ಕುದುರೆಗಳನ್ನು ಓಡಿಸಿಕೊಂಡು ಹೋಗುತ್ತಿದ್ದರೆ....ಹುಡುಗ ನರೇಂದ್ರನ ಕಣ್ಣಿಗೆ ಅವನೊಬ್ಬ ಮಹಾರಾಜನೇ ಸರಿ!

ಮಕ್ಕಳಿಗೆ ಮೊದಲ ವಿದ್ಯಾಭ್ಯಾಸ ಮಾತೆಯ ಮಡಿಲಲ್ಲಿ. ಭುವನೇಶ್ವರಿ ತನ್ನ ಮಗುವನ್ನು ಮಡಿಲಲ್ಲಿ ಕೂಡಿಸಿಕೊಂಡು, ದೇವದೇವಿಯರ ಕಥೆಗಳನ್ನು, ಮಹಿಮೆಗಳನ್ನು ಬಣ್ಣಿಸುತ್ತಿದ್ದಳು, ಭಾರತದ ಪುಷಿಮುನಿಗಳ ಹಾಗೂ ದತ್ತ ವಂಶದ ಹಿರಿಯರ ಕತೆಗಳನ್ನು ಹೇಳುತ್ತಿದ್ದಳು. ಸಂನ್ಯಾಸಿಯಾಗಿ ಹೋದ ಅವನ ತಾತ ದುರ್ಗಾಪ್ರಸಾದನ ವಿಷಯವನ್ನೂ ಹೇಳುತ್ತಿದ್ದಳು. ತಾಯಿಯ ಮಡಿಲಲ್ಲಿ ಕುಳಿತೇ ನರೇಂದ್ರ ರಾಮಾಯಣ-ಮಹಾಭಾರತಗಳ ಕತೆಯನ್ನು ಆಲಿಸಿದ. ಪ್ರತಿದಿನವೂ ಮಧ್ಯಾಹ್ನ ಮನೆಯ ಮಹಿಳೆಯರೆಲ್ಲ ತಮ್ಮತಮ್ಮ ಕೆಲಸಕಾರ್ಯಗಳನ್ನು ಮುಗಿಸಿಕೊಂಡು ಒಟ್ಟಾಗಿ ಕುಳಿತು, ಅವರಲ್ಲೇ ಯಾರಾದರೊಬ್ಬರಿಂದ ಆ ಮಹಾಕಾವ್ಯಗಳನ್ನು ಓದಿಸಿ ಕೇಳುತ್ತಿದ್ದರು. ಈ ಗುಂಪಿನಲ್ಲಿ ನರೇಂದ್ರನೂ ತನ್ನ ತುಂಟಾಟ ಚೇಷ್ಟೆಗಳನ್ನೆಲ್ಲ ಬಿಟ್ಟು ಶಾಂತವಾಗಿ ಕುಳಿತುಕೊಂಡು ಕೊನೆಯವರೆಗೂ ಏಕಾಗ್ರತೆಯಿಂದ ಕೇಳುತ್ತಿದ್ದ. ಅಲ್ಲದೆ ಅವನ ಅಜ್ಜಿಯೂ ಮುತ್ತಜ್ಜಿಯೂ ಶ್ರೀಮದ್ಭಾಗವತವನ್ನು ಚೆನ್ನಾಗಿ ತಿಳಿದುಕೊಂಡವರು. ಇವರೂ ನರೇಂದ್ರನಿಗೆ ಭಾಗವತದ ಕತೆಗಳನ್ನು ಸ್ವಾರಸ್ಯವಾಗಿ ಹೇಳುತ್ತಿದ್ದರು. ಬಾಲ ನರೇಂದ್ರನ ಎಳೆಯ ಮನಸ್ಸಿನ ಮೇಲೆ ಈ ಎಲ್ಲ ಕತೆಗಳೂ ಅಚ್ಚಳಿಯದ ಸತ್ಪರಿಣಾಮವನ್ನುಂಟುಮಾಡಿದವು.

ಬಂಗಾಳದ ಜನಪದ ಗೀತೆಗಳು ತುಂಬ ಅರ್ಥಗರ್ಭಿತ, ಭಾವಭರಿತ. ಅಲ್ಲಿನ ಒಂದು ಪಂಗಡದ ಭಿಕ್ಷುಕರು–ಇವರನ್ನು ಬಾವುಲ್ ಗಳೆನ್ನುತ್ತಾರೆ–ಈ ಹಾಡುಗಳನ್ನು ಸುಮಧುರವಾಗಿ ಹಾಡಬಲ್ಲರು. ಇಂತಹ ಹಾಡುಗಳನ್ನು ನರೇಂದ್ರ ಅತ್ಯಾಸಕ್ತಿಯಿಂದ ಆಲಿಸುತ್ತಿದ್ದ. ಮತ್ತು ಅವುಗಳಿಂದ ಪ್ರಭಾವಿತನಾಗಿ ಅನೇಕಾನೇಕ ವಿಷಯಗಳನ್ನು ಕಲಿತಿದ್ದ. ಭುವನೇಶ್ವರಿ ಈ ಬಾವುಲ್ ಹಾಡುಗಾರರು ಬಂದರೆ ಆದರದಿಂದ ಸ್ವಾಗತಿಸುತ್ತಿದ್ದಳು. ತನ್ನ ಮಗ ಆ ನೀತಿಪರ ಹಾಗೂ ಭಕ್ತಿಪರ ಹಾಡುಗಳನ್ನು ಕೇಳಿ ರಾಷ್ಟ್ರೀಯ ಸಂಸ್ಕೃತಿಯನ್ನು ಮೈಗೂಡಿಸಿಕೊಳ್ಳುವಂತಾಗಲಿ ಎನ್ನುವುದು ಅವಳ ಉದ್ದೇಶ.

ನಿಜಕ್ಕೂ ಈ ಸಂದರ್ಭಗಳಲ್ಲೇ ಬಾಲಕ ನರೇಂದ್ರನ ಹೃದಯದಲ್ಲಿ ಆಧ್ಯಾತ್ಮಿಕತೆಯ ಬೀಜ ಬಿತ್ತಿದಂತಾಯಿತು ಎನ್ನಬಹುದು. ತನ್ನ ತಾಯಿ ಹೇಳುತ್ತಿದ್ದ ರಾಮಾಯಣದ ಕಥೆಗಳ ಮೂಲಕ ಮತ್ತು ಬಾವುಲ್ ಗಳು ಹಾಡುತ್ತಿದ್ದ ದೇವರನಾಮಗಳ ಮೂಲಕ ಮತ್ತೆ ಮತ್ತೆ ಕಿವಿಯ ಮೇಲೆ ಬೀಳುತ್ತಿದ್ದ ರಾಮ-ಸೀತೆಯರ ಹೆಸರುಗಳು ಅವನ ಮನಃಪಟಲದ ಮೇಲೆ ಅಚ್ಚಳಿಯದ ಮುದ್ರೆಯನ್ನೊತ್ತಿದುವು. ಶ್ರೀರಾಮನ ಗುಣಮಾಧುರ್ಯ, ದಿವ್ಯ ಚಾರಿತ್ರ್ಯ, ಅದ್ಭುತ ಪರಾಕ್ರಮ ಇವೆಲ್ಲ ಆತನ ಮನಸೂರೆಗೊಂಡಿದ್ದುವು. ಅಲ್ಲದೆ, ಮನೆಯಲ್ಲಿ ಹಿರಿಯರು ಕ್ರಮಪ್ರಕಾರ ದೇವರಪೂಜೆ, ಜಪಧ್ಯಾನಗಳನ್ನು ಮಾಡುವುದನ್ನು ಅವನು ಗಮನಕೊಟ್ಟು ನೋಡುತ್ತಿದ್ದ. ತಾನೂ ಹಾಗೆಯೇ ದೇವರ ಪೂಜೆ ಜಪ ಧ್ಯಾನ ಮಾಡಬೇಕು ಎಂಬ ಇಚ್ಛೆ ಅವನಲ್ಲಿ ಬಲಿಯುತ್ತಿತ್ತು.

ಒಂದು ದಿನ ನರೇಂದ್ರ ಪಕ್ಕದ ಮನೆಯ ಸ್ನೇಹಿತ ಹರಿಯನ್ನು ಕರೆದುಕೊಂಡು ಅಂಗಡಿಗೆ ಹೋಗಿ ಜೇಡಿಮಣ್ಣಿನಿಂದ ತಯಾರಿಸಿದ ಸೀತಾರಾಮರ ವಿಗ್ರಹಗಳನ್ನು ಕೊಂಡು ತಂದ; ಮನೆಯವರಾರಿಗೂ ಕಾಣದಂತೆ ಸ್ನೇಹಿತನೊಂದಿಗೆ ಮಹಡಿಯನ್ನೇರಿ ಒಂದು ಕೋಣೆಯನ್ನು ಹೊಕ್ಕ. ಬಾಗಿಲು ಮುಚ್ಚಿ ಅಗಳಿ ಹಾಕಿಕೊಂಡ. ಅಲ್ಲಿ ಆ ವಿಗ್ರಹಗಳನ್ನು ಪ್ರತಿಷ್ಠಾಪಿಸಿಕೊಂಡು ಇಬ್ಬರೂ ಸೀತಾರಾಮರ ‘ಧ್ಯಾನ’ದಲ್ಲಿ ತೊಡಗಿದರು. ಎಷ್ಟೋ ಹೊತ್ತಾಗಿಹೋಯಿತು. ಕಡೆಗೆ ಇಬ್ಬರು ಮಕ್ಕಳೂ ಬಹಳ ಹೊತ್ತಿನಿಂದ ಕಾಣದಿದ್ದುದರಿಂದ ಇಬ್ಬರ ಅಮ್ಮಂದಿರೂ ಆತಂಕಗೊಂಡರು. ಮತ್ತು ಎರಡು ಮನೆಯವರೂ ಅವರನ್ನು ಹುಡುಕತೊಡಗಿದರು. ಆದರೆ ಮನೆಯ ಮೂಲೆಮೂಲೆಯನ್ನೆಲ್ಲ ಹುಡುಕಾಡಿ ನೋಡಿದರೂ ಹುಡುಗರಿಲ್ಲ! ಕೊನೆಗೆ ಹಿರಿಯರು ಹಾಗೇ ಮೆಟ್ಟಿಲೇರಿ ತಾರಸಿಯ ಮೇಲೆ ಬಂದರು. ಕೋಣೆಯ ಬಾಗಿಲಿಗೆ ಒಳಗಿಂದ ಅಗಳಿಹಾಕಿತ್ತು. ‘ಓಹೋ, ಹುಡುಗರು ಇಲ್ಲೇ ಇರಬೇಕು’ ಎಂದು ಊಹಿಸಿದರು. ಆದರೆ ಬಾಗಿಲು ತಟ್ಟಿ ನೋಡಿದರು, ಕೂಗಿಕೊಂಡರು, ಉತ್ತರವೇ ಇಲ್ಲ! ಕೊನೆಗೆ ಬಾಗಿಲನ್ನು ಜೋರಾಗಿ ಒದ್ದಾಗ ಅಗಳಿ ಬಿಚ್ಚಿಕೊಂಡು ಬಾಗಿಲು ತೆರೆದುಕೊಂಡಿತು. ಹರಿಯೇನೋ ಮೊದಲ ಸಲ ಬಾಗಿಲು ತಟ್ಟಿದಾಗಲೇ ‘ಧ್ಯಾನ’ ಭಂಗವಾಗಿ ಎದ್ದು ನಿಂತಿದ್ದ. ಆದರೆ ಬಾಗಿಲು ತೆರೆಯಲು ಭಯವಾಗಿ ಸುಮ್ಮನೆ ಬೆಪ್ಪಾಗಿ ನಿಂತಿದ್ದ. ಈಗ ಬಾಗಿಲು ತೆರೆದುಕೊಂಡ ಕೂಡಲೇ ಅಲ್ಲಿಂದ ಓಟ ಕಿತ್ತ! ಆದರೆ ಪರಮಾಶ್ಚರ್ಯದ ಸಂಗತಿಯೆಂದರೆ, ಇಷ್ಟು ಗಲಾಟೆಯಾದರೂ ನರೇಂದ್ರನಿಗೆ ಮಾತ್ರ ಏನೂ ಕೇಳಿಸಿಯೇ ಇರಲಿಲ್ಲ. ಅವನು ಆ ಸೀತಾರಾಮರ ವಿಗ್ರಹದ ಮುಂದೆ ತಾನು ಇನ್ನೊಂದು ವಿಗ್ರಹವಾಗಿ ಕುಳಿತುಬಿಟ್ಟದ್ದ. ಭುವನೇಶ್ವರಿ “ನರೇನ್! ಓ ನರೇನ್!” ಎಂದು ಕೂಗಿ ಕರೆದಳು. ಆದರೂ ಉತ್ತರವಿಲ್ಲ. ಕೊನೆಗೆ ಆಕೆ ಮಗನ ಮೈಯನ್ನು ಜೋರಾಗಿ ಅಲುಗಾಡಿಸಿದಳು. ಆಗ ಅವನು ತನ್ನ ಗಾಢ ಧ್ಯಾನದಿಂದ ಎಚ್ಚರಗೊಂಡ. ಆದರೆ ಅವನಿಗೆ ಧ್ಯಾನದಿಂದ ಏಳುವ ಮನಸ್ಸೇ ಇಲ್ಲ. “ನಾನು ಬರುವುದಿಲ್ಲ, ನನ್ನನ್ನು ಹೀಗೇ ಬಿಟ್ಟು ಬಿಡಿ” ಎಂದುಬಿಟ್ಟ! ಬೇರೆ ದಾರಿಗಾಣದೆ ಹಿರಿಯರು ಅವನನ್ನು ಅವನ ಪಾಡಿಗೆ ಬಿಡಬೇಕಾಯಿತು. ಇದನ್ನೆಲ್ಲ ಏನೆಂದು ಅರ್ಥಮಾಡಿಕೊಳ್ಳಬೇಕೆಂದೇ ತಿಳಿಯಲಿಲ್ಲ ಅವರಿಗೆ. ‘ಈ ಎಳೆಯ ವಯಸ್ಸಿನಲ್ಲಿ ಇಷ್ಟೆಲ್ಲ ಗಾಢ ಧ್ಯಾನ ಸಾಧ್ಯವೆ? ಇದೆಂಥ ವಿಚಿತ್ರವಪ್ಪಾ!’ ಎಂದು ಮಾತಾಡಿಕೊಂಡರು.

ನರೇಂದ್ರ ಸೀತಾ-ರಾಮರ ವಿಗ್ರಹಗಳೆದುರಿನಲ್ಲಿ ಧ್ಯಾನಲೀನನಾಗಿ ಕುಳಿತ ಆ ದೃಶ್ಯ ನಿಜಕ್ಕೂ ಅದ್ಭುತವಾದದ್ದೇ ಸರಿ. ಆದರೆ ಆ ದೃಶ್ಯ ಬಹಳ ಕಾಲ ಉಳಿಯಲಿಲ್ಲ. ಇದಕ್ಕೆ ಕಾರಣ ಆ ಮನೆಯ ಸಾರೋಟುವಾಲನೇ ಎನ್ನಬೇಕು. ನಾವಾಗಲೇ ನೋಡಿದಂತೆ ಬಾಲಕ ನರೇಂದ್ರನ ಮನಸ್ಸನ್ನು ಆಕರ್ಷಿಸಿದ ಮಹಾಪುರುಷರಲ್ಲಿ ಈ ಸಾರೋಟುವಾಲ ಬಹಳ ಪ್ರಮುಖನಾದವನು. ಹಿಂದೊಮ್ಮೆ ವಿಶ್ವನಾಥ ದತ್ತ ಹುಡುಗ ನರೇಂದ್ರನನ್ನು “ಏನಪ್ಪಾ, ಮುಂದೆ ದೊಡ್ಡವನಾದ ಮೇಲೆ ಏನಾಗಬೇಕು ಅಂತ ನಿನ್ನಿಷ್ಟ?” ಎಂದು ಕೇಳಿದಾಗ ಅವನು ತಕ್ಷಣ, “ನಾನು ಸಾರೋಟುವಾಲಾ ಆಗ್ತೀನಪ್ಪಾ” ಎಂದಿದ್ದ! ಅವನು ತನ್ನ ಬಿಡುವಿನ ವೇಳೆಯಲ್ಲೆಲ್ಲ ಇರುತ್ತಿದ್ದುದು ಆ ಸಾರೋಟುವಾಲನ ಜೊತೆಯಲ್ಲೇ, ಅವನಿಗೆ ಯಾವುದಾದರೂ ವಿಷಯದಲ್ಲಿ ಹೆಚ್ಚಿನ ಮಾಹಿತಿ ಬೇಕಿದ್ದರೆ ತಕ್ಷಣ ಓಡುತ್ತಿದ್ದುದು ಆ ಸಾರೋಟುವಾಲನ ಬಳಿಗೆ. ಅವನ ಪಾಲಿಗೆ ಆ ಸಾರೋಟುವಾಲನೇ ಮಹಾಜ್ಞಾನಿ, ಸರ್ವಜ್ಞ. ಅಂದೂ ಕೂಡ ನರೇಂದ್ರ ಯಾವುದೋ ಒಂದು ವಿಷಯಕ್ಕಾಗಿ ಅವನ ಬಳಿಗೆ ಹೋದ. ಆ ವಿಷಯ ಮಾತಾಡಿ ಮುಗಿದ ಮೇಲೆ, ಮಾತು ಮದುವೆಯ ವಿಷಯಕ್ಕೆ ತಿರುಗಿಕೊಂಡಿತು. ಸಾರೋಟುವಾಲ ಮದುವೆಯ ವಿಷಯವಾಗಿ ಮಾತನಾಡುತ್ತ ಮದುವೆಯನ್ನು ಒಂದೇ ಸಮನೆ ನಿಂದಿಸಿ ಟೀಕೆ ಮಾಡಲಾರಂಭಿಸಿದ. ಅದೇಕೆ ಅವನು ಆ ರೀತಿ ಮದುವೆಯನ್ನು ಹಳಿದನೋ ಗೊತ್ತಿಲ್ಲ. ಬಹುಶಃ ಅವನೂ ಒಂದು ಮದುವೆಮಾಡಿಕೊಂಡು ಕೈಸುಟ್ಟುಕೊಂಡಿರಬಹುದು. ಅಂತೂ ಈ ಮದುವೆಯ ಅನಾಹುತದ ಸಂಬಂಧವಾಗಿ ಅತ್ಯಂತ ಕಟುವಾದ ಭಾಷೆಯಿಂದ ಒಂದು ಭಾಷಣವನ್ನೇ ಬಿಗಿದ: “ ಅಯ್ಯೋ! ಮದುವೆ ಅಂದರೆ ಒಂದು ದೊಡ್ಡ ತಾಪತ್ರಯ ಕಣಪ್ಪಾ! ಹೆಂಡತಿಯಾಗಿ ಬರುವವಳ ಸ್ವಭಾವಕ್ಕೆ ಸರಿಯಾಗಿ ಸರ್ಕಸ್ ಮಾಡಬೇಕಾಗುತ್ತದೆ. ಆಮೇಲೆ ಒಂದಿಷ್ಟು ತರಲೆ ಮಕ್ಕಳನ್ನು ಕಟ್ಟಿಕೊಂಡು ಒದ್ದಾಡಬೇಕಾಗುತ್ತದೆ. ಅದೇನು ಗೋಳು ಅನ್ನುತ್ತಿ! ಸಾಕಪ್ಪ, ಈ ಮದುವೆಯ ಮೋಜೂ ಸಾಕು, ಅದರ ತಾಪತ್ರಯವೂ ಸಾಕು.” ಇಷ್ಟು ಹೇಳುವಷ್ಟರಲ್ಲಿ ಅವನ ಶಬ್ದಕೋಶವೆಲ್ಲ ಬರಿದಾಯಿತು. ಬಾಲಕ ನರೇಂದ್ರ ಆ ಉದ್ರೇಕಕಾರೀ ಭಾಷಣವನ್ನು ಉಸಿರು ಬಿಗಿಹಿಡಿದುಕೊಂಡು ಕೇಳಿದ. ಅವನ ಮುಖ ಗಂಭೀರವಾಗಿತ್ತು. ಮದುವೆಯಾಗುವುದರಿಂದ ಉಂಟಾಗುವ ಪಾಡನ್ನು ಭಾವಿಸಿ ನೋಡಿತು ಅವನ ಮನಸ್ಸು. ಹೆದರಿಕೆಯಾಯಿತು. ತನಗೇ ಈಗ ಮದುವೆ ನಿಶ್ಚಯವಾಗಿ ಆ ಮದುವೆಯ ದಿನ ಓಡೋಡಿ ಬರುತ್ತಿದೆಯೇನೋ ಎನ್ನುವಂತೆ ಬೆಚ್ಚಿಬಿದ್ದ. ಅವನ ಮನಸ್ಸಿನಲ್ಲಿ ಮದುವೆಯ ಸಂಬಂಧವಾಗಿ ಬಲವಾದ ಜುಗುಪ್ಸೆ ಹುಟ್ಟಿಬಿಟ್ಟಿತು. ಈ ಸಮಯದಲ್ಲಿ ಅವನ ಮನಃಪಟಲದ ಮೇಲೆ ಮೊದಲು ಕಾಣಿಸಿಕೊಂಡ ವ್ಯಕ್ತಿಗಳೆಂದರೆ ಸೀತಾ-ರಾಮರು. ‘ಅರೆ! ಸೀತಾ-ರಾಮರು ಮದುವೆಯಾದವರಲ್ಲವೆ?... ಅಂತಹ ದೊಡ್ಡ ದೇವರಾದ ಶ್ರೀರಾಮನು ಮದುವೆಯ ಬಂಧನಕ್ಕೊಳಗಾದವನೇ ಅಲ್ಲವೆ?...’ ಸೀತಾ-ರಾಮರ ವೈವಾಹಿಕ ಜೀವನದ ವೈಭವಪೂರ್ಣ ವರ್ಣನೆಯನ್ನು ಅವನು ಬಹಳಷ್ಟು ಕೇಳಿದ್ದ. ಅವರಿಬ್ಬರ ಕಾವ್ಯಮಯ ಪ್ರೇಮಸಂಬಂಧವನ್ನು, ಪರಸ್ಪರ ಪ್ರೇಮದ ಪ್ರಾಮಾಣಿಕ ನಿಷ್ಠೆಯನ್ನು ಕೇಳಿದ್ದ. ಆದರೆ ಈಗ ಆ ಕಾವ್ಯವೆಲ್ಲ ನೀರಸ ಗದ್ಯವಾಗಿ ತೋರಲಾರಂಭಿಸಿತು. ಏಕೆಂದರೆ ಸರ್ವಜ್ಞನಾದ ಸಾರೋಟುಸವಾರ ಈ ಮದುವೆಯ ಭಯಂಕರ ನಿಜಸ್ವರೂಪವನ್ನು ತಿಳಿಸಿಕೊಟ್ಟುಬಿಟ್ಟಿದ್ದಾನಲ್ಲ! ಇನ್ನು ಅವರನ್ನು ತಾನು ಹೇಗೆ ಧ್ಯಾನಿಸಲಿ? ಹೇಗೆ ಆರಾಧಿಸಲಿ? ಆದರೆ ತಾಯಿಯ ಬಾಯಿಂದ ಸೀತಾ-ರಾಮರ ವರ್ಣನೆಯನ್ನು ಕೇಳಿ ಕೇಳಿ ಅವರನ್ನು ತನ್ನ ಆರಾಧ್ಯ ದೇವತೆಗಳೆಂದು ಸ್ವೀಕರಿಸಿಯಾಗಿಬಿಟ್ಟಿದೆ. ಈಗ ಅವರನ್ನು ತನ್ನ ಮನಸ್ಸಿನಿಂದ ಹೊರದೂಡುವುದಾದರೂ ಹೇಗೆ? ನರೇಂದ್ರನ ಮನಸ್ಸು ಉಭಯಸಂಕಟದಿಂದ ಒದ್ದಾಡಿತು. ಏನು ಮಾಡುವುದಕ್ಕೂ ತೋಚದೆ ಒಂದೇ ಸಮನೆ ಅಳತೊಡಗಿದ. ದುಃಖದಿಂದ ತಾಯಿಯ ಬಳಿಗೋಡಿದ. ಭುವನೇಶ್ವರಿ ನೋಡುತ್ತಾಳೆ–ಮಗನ ಕಣ್ಣಲ್ಲಿ ನೀರು! ಎಂದೂ ಅಳದಿರುವವನು ಇಂದೇಕೆ ಹೀಗೆ ಕಣ್ಣೀರು ಸುರಿಸುತ್ತಿದ್ದಾನೆ! ವಾತ್ಸಲ್ಯದಿಂದ ಮಗನನ್ನು ತಬ್ಬಿಕೊಂಡು “ಏನಾಯಿತಪ್ಪ, ಏಕೆ ಅಳುತ್ತಿದ್ದೀ?” ಎಂದು ಕೇಳಿದಳು. ಹುಡುಗ ಒಂದು ಕ್ಷಣ ಶಾಂತನಾದ. ಆದರೆ ಮತ್ತೆ ಗಟ್ಟಿಯಾಗಿ ಬಿಕ್ಕಿಬಿಕ್ಕಿ ಅಳಲಾರಂಭಿಸಿದ. ಅಳುತ್ತಳುತ್ತಲೇ ತಾಯಿಯನ್ನು ಕೇಳಿದ: “ಅಮ್ಮಾ ನಾನಿನ್ನು ಮೇಲೆ ಸೀತಾ-ರಾಮರನ್ನು ಹೇಗೆ ಪೂಜೆ ಮಾಡಲಿ? ಸೀತೆ ರಾಮನ ಹೆಂಡತಿಯಲ್ಲವೆ?” ಬಳಿಕ ತಾನು ಸಾರೋಟುವಾಲನ ಹತ್ತಿರ ಕೇಳಿ ತಿಳಿದುಕೊಂಡ ಹೊಸ ವಿಚಾರಗಳನ್ನೆಲ್ಲ ತಾಯಿಗೆ ಹೇಳಿದ. ಭುವನೇಶ್ವರಿಗೆ ವಿಷಯ ಅರ್ಥವಾಯಿತು. ಓಹೋ, ಹೀಗೋ ಸಮಾಚಾರ! ಆದರೆ ಈಗ ಮಗನನ್ನು ಸಮಾಧಾನ ಪಡಿಸುವ ಬಗೆ ಹೇಗೆ? ಕ್ಷಣಕಾಲ ಆಲೋಚಿಸಿದ ಮೇಲೆ ಮಿಂಚಿನಂತೆ ಒಂದು ಉಪಾಯ ಹೊಳೆಯಿತು. ಹೇಳಿದಳು, “ನೋಡು, ನೀನು ಇನ್ನು ಮೇಲೆ ರಾಮನನ್ನು ಬಿಟ್ಟು ಧ್ಯಾನಮಗ್ನನಾದ ಶಿವನನ್ನು (ಎಂದರೆ, ಅವಿವಾಹಿತನಾದ ತಾಪಸಿ ಶಿವನನ್ನು) ಪೂಜೆ ಮಾಡು, ಅಷ್ಟೆ!” ಎಂದು. ಈ ಸಲಹೆ ನರೇಂದ್ರನಿಗೆ ಕೂಡಲೇ ಒಪ್ಪಿಗೆಯಾಯಿತು, ಸಮಾಧಾನ ತಂದಿತು. ಹಾಗೇ ಮಾಡಲು ಆ ಕ್ಷಣವೇ ನಿರ್ಧರಿಸಿಬಿಟ್ಚ.

ಅಂದು ಸಂಜೆಗತ್ತಲು ಕವಿಯುತ್ತಿದ್ದ ಸಮಯದಲ್ಲಿ ಅವನು ಯಾರ ಕಣ್ಣಿಗೂ ಬೀಳದಂತೆ ತಾರಸಿಯ ಮೇಲೇರಿದ. ತಾನು ಸೀತಾ-ರಾಮರ ವಿಗ್ರಹಗಳನ್ನಿಟ್ಟಿದ್ದ ಕೋಣೆಯ ಬಾಗಿಲುಗಳನ್ನು ತೆರೆದು ಒಳಗೆ ಹೋದ. ಆ ವಿಗ್ರಹಗಳನ್ನು ಕ್ಷಣಕಾಲ ದಿಟ್ಟಿಸಿದ. ಅವನು ಭಕ್ತಿಯಿಂದ ತಂದು ಪ್ರತಿಷ್ಠಾಪಿಸಿ ಧ್ಯಾನಿಸಿದ ವಿಗ್ರಹಗಳಲ್ಲವೆ ಅವು? ಅವನೆದೆಯೊಳಗಿಂದ ಬಿಸಿ ನಿಟ್ಟುಸಿರೊಂದು ತಾನೇತಾನಾಗಿ ಹೊರಹೊಮ್ಮಿತು, ಮರುಕ್ಷಣವೇ ಮನಸ್ಸು ಗಟ್ಟಿ ಮಾಡಿಕೊಂಡು ಆ ವಿಗ್ರಹಗಳನ್ನು ಕೈಗೆತ್ತಿಕೊಂಡ. ಬಳಿಕ ತಾರಸಿಯ ಅಂಚಿಗೆ ಬಂದು ಆ ವಿಗ್ರಹಗಳನ್ನು ರೊಯ್ಯನೆ ಕೆಳಕ್ಕೆಸೆದುಬಿಟ್ಟ! ವಿಗ್ರಹಗಳು ರಸ್ತೆಯ ಮೇಲೆ ಬಿದ್ದು ಚೂರುಚೂರಾದುವು. ಜೊತೆಗೆ ನರೇಂದ್ರನ ಮನಸ್ಸಿನಲ್ಲಿದ್ದ ಮದುವೆಯ ಮೇಲಿನ ಆದರವೂ ಪುಡಿಪುಡಿಯಾಯಿತು.

ಮರುದಿನವೇ ಅವನು ಅಂಗಡಿಗೆ ಹೋಗಿ ಒಂದು ಧ್ಯಾನಮಗ್ನ ಶಿವನ ವಿಗ್ರಹವನ್ನು ಕೊಂಡುತಂದ. ಅದನ್ನು ತಾನು ಹಿಂದೆ ಸೀತಾ-ರಾಮರನ್ನು ಪ್ರತಿಷ್ಠಾಪಿಸಿದ್ದ ಜಾಗದಲ್ಲಿ ಪ್ರತಿಷ್ಠಾಪಿಸಿದ. ಬಳಿಕ ಅದರ ಮುಂದೆ ಧ್ಯಾನಕ್ಕೆ ಕುಳಿತ. ಹೊರಪ್ರಪಂಚದ ಪರಿವೆ ಕಳಚಿತು. ಅಲ್ಲೀಗ ಇರುವುದು ಎರಡು ವಿಗ್ರಹಗಳು ಅಷ್ಟೆ; ಒಂದು–ಧ್ಯಾನಸ್ಥನಾದ ಭಗವಾನ್ ಶಿವ; ಇನ್ನೊಂದು–ಧ್ಯಾನಸ್ಥ ನಾದ ಬಾಲಕ ಶಿವ!

ನರೇಂದ್ರ ಹಾಗೆ ಸೀತಾರಾಮರ ವಿಗ್ರಹಗಳನ್ನು ಅಂದು ಗಟ್ಟಿ ಮನಸ್ಸಿನಿಂದ ಎಸೆದುಬಿಟ್ಟನಾದರೂ ಆಗ ಅವನ ಬಾಲಹೃದಯಕ್ಕೆ ಅದೆಷ್ಟು ದುಃಖವಾಯಿತೆಂಬುದನ್ನು ನಾವು ಭಾವಿಸಿಯೇ ತಿಳಿಯಬೇಕು. ಏಕೆಂದರೆ ಮಕ್ಕಳ ಹೊಂಗನಸು ಮುರಿದು ಬಿದ್ದಾಗ ಅವು ಸಕಲೈಶ್ವರ್ಯವನ್ನೂ ಕಳೆದುಕೊಂಡವನಿಗಿಂತ ಹೆಚ್ಚು ಬಡವಾಗುತ್ತವೆ. ಮಕ್ಕಳ ಹೊಂಗನಸಿಗೆ ಸರಿದೂಗುವ ಐಶ್ವರ್ಯವೇ ಇಲ್ಲ. ಆದರೆ ತನ್ನ ಹೊಂಗನಸು ಮುರಿದುಬಿದ್ದರೂ ತಾನು ಮೆಚ್ಚಿದ್ದ ಸೀತಾ-ರಾಮರ ವಿಗ್ರಹಗಳನ್ನು ಎಸೆಯುವಲ್ಲಿ ನರೇಂದ್ರ ತೋರಿದ ದಿಟ್ಟತನವನ್ನು ಗಮನಿಸಬೇಕು. ತನ್ನ ಜೀವನತತ್ತ್ವಕ್ಕೆ ಸರಿ ಹೊಂದದ ವಿಷಯವನ್ನು ತಕ್ಷಣ ತ್ಯಜಿಸಿಬಿಡುವ ಪ್ರಾಮಾಣಿಕತೆ ಕಂಡುಬರುತ್ತದೆ ಇಲ್ಲಿ. ಅಲ್ಲದೆ, ಮದುವೆಯ ಮೂಲಕ ಉಂಟಾಗುವ ಇಂದ್ರಿಯಜೀವನದ ಬಂಧನದಿಂದ ಬಿಡಿಸಿಕೊಂಡು ಮುಕ್ತನಾಗಿರಬೇಕೆಂಬ ಸ್ವತಂತ್ರ ಮನೋಭಾವ ಬಾಲಕ ನರೇಂದ್ರನ ಹೃದಯದಲ್ಲಾಗಲೇ ಜಾಗೃತವಾಗಿತ್ತೆನ್ನುವುದು ಸ್ವಷ್ಟವಾಗುತ್ತದೆ.

ಆದರೆ ಇಷ್ಟಾದರೂ ಅವನಿಗೆ ರಾಮ-ಸೀತೆಯರ ಮೇಲಿನ ಭಕ್ತಿಯಾಗಲಿ ರಾಮಾಯಣದ ಮೇಲಿನ ಶ್ರದ್ಧೆಯಾಗಲಿ ಅಳಿಸಿಹೋಗಿರಲಿಲ್ಲ; ಹೋಗುವುದು ಸಾಧ್ಯವೂ ಇರಲಿಲ್ಲ. ರಾಮಾಯಣದ ಕಥೆ ಅವನನ್ನು ಅಷ್ಟು ಗಾಢವಾಗಿ ಆಕರ್ಷಿಸಿ ಪ್ರಭಾವ ಬೀರಿಬಿಟ್ಟಿತ್ತು. ಆದ್ದರಿಂದ ನೆರೆಕೆರೆಯಲ್ಲಿ ಯಾರಾದರೂ ರಾಮಾಯಣವನ್ನು ಓದುತ್ತಿದ್ದರೆ ತಪ್ಪದೆ ಅದನ್ನು ಕೇಳಲು ಹೋಗುತ್ತಿದ್ದ. ರಾಮನ ಅದ್ಭುತ ಕಥಾಪ್ರಸಂಗಗಳನ್ನು ಗಮನಕೊಟ್ಟು ಆಲಿಸುತ್ತಿದ್ದ. ಅದರಲ್ಲೂ ರಾಮಾಯಣ ಮಹಾಮಾಲೆಯ ಶ್ರೇಷ್ಠ ರತ್ನವಾದ ಭಕ್ತವರ ಹನುಮಂತನ ಮೇಲೆ ನರೇಂದ್ರನಿಗೆ ವಿಶೇಷ ಆಕರ್ಷಣೆ. ಒಮ್ಮೆ, ಹನುಮಂತನು ಬಾಳೆಯ ತೋಟದಲ್ಲಿ ವಾಸವಾಗಿರುತ್ತಾನೆ ಎಂದು ರಾಮಾಯಣದ ಕಥನಕಾರರೊಬ್ಬರು ಹೇಳಿದ್ದನ್ನು ಅವನು ಕೇಳಿದ. ತಕ್ಷಣ ಅವನಿಗೆ ಆ ಕಪಿ ಶ್ರೇಷ್ಠನಾದ ಹನುಮಂತನನ್ನು ಕಾಣಲೇಬೇಕೆಂಬ ಉತ್ಸಾಹ ಕೆರಳಿತು. ಆದರೆ ತನ್ನಂತಹ ಹುಡುಗರು ಹೋದರೆ ಅವನು ಸಿಗುವನೋ ಇಲ್ಲವೋ? ಕಥಾಕಾಲಕ್ಷೇಪ ಮುಗಿದ ಕೂಡಲೇ ಆ ಕಥನಕಾರರ ಹತ್ತಿರ ಹೋಗಿ ಕೇಳಿದ: “ಅಲ್ಲಿಗೆ ನಾನು ಹೋದರೆ ಅವನನ್ನು ನೋಡಬಹುದೆ?” ಎಂದು. ಬಾಲಕ ನರೇಂದ್ರ ತುಂಬ ಮುಗ್ಧಭಾವದಿಂದ ಹಾಗೆ ಕೇಳಿದರೆ, ಆ ಕಥನಕಾರರು ಮುಗುಳ್ನಗುತ್ತ, “ಓ ಬೇಕಾದರೆ ಹೋಗಿ ನೋಡು!” ಎಂದರು. ನರೇಂದ್ರ ಮನೆಯ ಕಡೆ ಹೊರಟ. ಮನೆಯ ಹತ್ತಿರವೇ ಒಂದು ಬಾಳೆಯ ತೋಟವಿತ್ತು. ನರೇಂದ್ರ ನೇರವಾಗಿ ಅದರೊಳಕ್ಕೆ ಹೋದ. ಒಂದು ಬಾಳೆಯ ಮರದ ಬುಡದಲ್ಲಿ ಕುಳಿತುಕೊಂಡು ಹನುಮಂತನ ಬರವನ್ನೇ ನಿರೀಕ್ಷಿಸುತ್ತಾ ಇದ್ದುಬಿಟ್ಟ. ಆದರೆ ಹನುಮಂತನೇಕೋ ಬರಲೇ ಇಲ್ಲ. ನರೇಂದ್ರ ಕಾದುಕಾದು ಬಳಲಿದ. ನಿರಾಶೆಯಾಯಿತು. ಮನೆಗೆ ಬಂದು ವಿಷಯವನ್ನೆಲ್ಲ ಹೇಳಿದ. ಮನೆಯವರಿಗೆಲ್ಲ ನಗುವೋ ನಗು. ಆದರೆ ನಗೆಯನ್ನು ತೋರಗೊಡದೆ ಸಮಾಧಾನ ಹೇಳಿದರು: “ನೋಡು, ಹನುಮಂತ ರಾಮನ ಬಂಟ ಅಲ್ಲವೆ? ರಾಮ ಅವನಿಗೇನೋ ತ್ವರಿತದ ಕೆಲಸ ಹೇಳಿರಬಹುದು, ಅವನದನ್ನು ಮಾಡಿಕೊಂಡು ಬರಲು ಹೋಗಿರಬಹುದು” ಎಂದು. ನರೇಂದ್ರ ನಿಗೆ ಈ ಮಾತು ಒಪ್ಪಿಗೆಯಾಯಿತು.

ಆದರೆ ಈಗ ಅವನ ಅಚ್ಚುಮೆಚ್ಚಿನ ದೇವರೆಂದರೆ ಶಿವ. ಅವನು ಪ್ರತಿದಿನ ಶಿವನ ಮುಂದೆ ಕುಳಿತುಕೊಂಡು ಧ್ಯಾನ ಮಾಡುತ್ತಿದ್ದ. ಶಿವ ಸಂನ್ಯಾಸಿಗಳ ರಾಜ–ಯತಿರಾಜ. ನರೇಂದ್ರನಿಗೂ ತಾನು ಸಂನ್ಯಾಸಿಯಾಗಬೇಕೆಂಬ ಬಾಲಕಲ್ಪನೆ. ಒಂದು ದಿನ ಅವನು ಸೊಂಟಕ್ಕೊಂದು ತುಂಡು ಕಾವಿಬಟ್ಟೆ ಸುತ್ತಿಕೊಂಡು ಮನೆಯ ತುಂಬೆಲ್ಲ ಓಡಾಡುತ್ತಿದ್ದ. ಆ ಕಾವಿಬಟ್ಟೆ ಬಿಟ್ಟರೆ ಮೈಮೇಲೆ ಇನ್ನೇನೂ ಇಲ್ಲ! ತಾಯಿ ಭುವನೇಶ್ವರಿ ನೋಡಿದಳು ಈ ದೃಶ್ಯವನ್ನು. ಆಶ್ಚರ್ಯಕುತೂಹಲಗಳಿಂದ ಕೇಳಿದಳು: “ಇದೆಲ್ಲ ಏನಪ್ಪ ನರೇನ್? ಏನು ಮಾಡಬೇಕು ಅಂತ ಹೊರಟಿದ್ದೀ!” ಆಗ ಅವನು ತಾನೇನೋ ಅದ್ಭುತವನ್ನು ಸಾಧಿಸಿದವನ ಹಾಗೆ ಗಟ್ಟಿಯಾಗಿ ನುಡಿದ: “ಅಮ್ಮಾ ನಾನು ಶಿವ ಕಣಮ್ಮಾ! ನಾನು ಶಿವ!”

ಅವನು ಈ ರೀತಿ ಪ್ರತಿದಿನವೂ ಧ್ಯಾನ ಮಾಡುವುದನ್ನು ಕಂಡು ಮನೆಯ ಮಹಿಳೆಯರು, “ನೋಡು ನರೇನ್, ಧ್ಯಾನ ಮಾಡುತ್ತಿದ್ದರೆ ಪುಷಿಮುನಿಗಳ ಹಾಗೆ ತಲೆಯ ಮೇಲೆ ಉದ್ದುದ್ದದ ಜಟೆಗಳು ಬೆಳೆಯುತ್ತವೆ. ಆಮೇಲೆ ಆ ಜಟೆಗಳೆಲ್ಲ ಆಲದ ಮರದ ಬೇರಿನಂತೆ ನೆಲದೊಳಗೆ ಇಳಿಯುತ್ತವೆ” ಎಂದು ತಮಾಷೆ ಮಾಡಿದರು. ಮುಗ್ಧ ಮನಸ್ಸಿನ ನರೇಂದ್ರ ಇದನ್ನು ಕೇಳಿಕೊಂಡು ತಕ್ಷಣ ಹೋಗಿ ಧ್ಯಾನಕ್ಕೆ ಕುಳಿತ. ಧ್ಯಾನ ಮಾಡುತ್ತ ಮಧ್ಯೆಮಧ್ಯೆ ಕಣ್ಣು ಬಿಟ್ಟು ನೋಡುತ್ತಿದ್ದ –ತನ್ನ ಕೂದಲು ಬೆಳೆದು ಜಟೆಯಾಗಿದೆಯೇ ಹೇಗೆ ಎಂದು. ಆದರೆ ಎಷ್ಟು ಹೊತ್ತಾದರೂ ಕೂದಲು ಬೆಳೆಯಲೂ ಇಲ್ಲ, ಜಟೆಯಾಗಲೂ ಇಲ್ಲ. ತುಂಬ ನಿರಾಶೆಯಾಯಿತು. ತಾಯಿಯ ಬಳಿಗೋಡಿ ತನ್ನ ದೊಡ್ಡ ಕಣ್ಣುಗಳನ್ನು ಅರಳಿಸಿಕೊಂಡು ಕೇಳಿದ, “ಅಮ್ಮಾ ನಾನು ಎಷ್ಟೋ ಹೊತ್ತು ಧ್ಯಾನ ಮಾಡಿದೆ. ಆದರೂ ನನ್ನ ಕೂದಲು ಜಟೆಗಟ್ಟಲೇ ಇಲ್ಲವಲ್ಲ ಯಾಕೆ?” ಆಗ ಭುವನೇಶ್ವರಿ ಸಮಾಧಾನ ಹೇಳಿದಳು: “ಮಗು, ಅದು ಹಾಗೆ ಒಂದು ಗಂಟೆ, ಒಂದು ದಿನದೊಳಗೆಲ್ಲ ಬೆಳೆಯುವುದಿಲ್ಲಪ್ಪ. ಅಷ್ಟುದ್ದದ ಜಟೆಗಟ್ಟಬೇಕಾದರೆ ಎಷ್ಟೋ ತಿಂಗಳುಗಳು ವರ್ಷಗಳೇ ಬೇಕಾಗುತ್ತವೆ.”

ಸರಿ; ಆ ವಿಷಯ ಅಲ್ಲಿಗೆ ನಿಂತಿತು. ಈಗ ಇನ್ನೊಂದು ವಿಷಯ ಎದ್ದಿತು. ಆ ದಿನ ಬೆಳಗ್ಗೆ ಧ್ಯಾನ ಮಾಡುವಾಗ ಅವನಿಗೆ, ತನ್ನ ಅಮ್ಮ ಈ ಹಿಂದೆ, “ ನರೇನ್, ನೀನು ಇದೇ ರೀತಿ ರಂಪ ಮಾಡುತ್ತಿದ್ದರೆ ಶಿವ ನಿನ್ನನ್ನು ಕೈಲಾಸಕ್ಕೆ ಸೇರಿಸುವುದಿಲ್ಲ” ಎಂದು ಹೇಳಿದ್ದು ನೆನಪಾಯಿತಂತೆ. ಆದ್ದರಿಂದ ಕೇಳಿದ, “ಅಮ್ಮ ನಾನು ಹಿಂದೆ ಸಾಧುವಾಗಿದ್ದೆ ಅಂತ ನನಗನಿಸುತ್ತಿದೆ. ನಾನು ಒಳ್ಳೆಯವನಾದರೆ ಶಿವ ನನ್ನನ್ನು ಕೈಲಾಸಕ್ಕೆ ಮತ್ತೆ ಸೇರಿಸುತ್ತಾನೇನಮ್ಮಾ?” ಭುವನೇಶ್ವರಿ, “ಹ್ಞೂ ಕಣಪ್ಪಾ” ಎಂದಳು. ಆದರೆ ಮರುಕ್ಷಣದಲ್ಲೇ ಅವಳೆದೆ ಬೆಚ್ಚಿತು, ಎಲ್ಲಿ ಇವನೂ ತನ್ನ ತಾತನಂತೆ ಸಂನ್ಯಾಸಿಯಾಗಿ ಶಿವಧಾಮವನ್ನು ಸೇರಿಬಿಡುತ್ತಾನೋ ಎಂದು. ಆದರೆ ಬಳಿಕ ತನಗೆ ತಾನೇ ಸಮಾಧಾನ ತಂದುಕೊಂಡಳು. ಅವನಿನ್ನೂ ಬೆಳೆದು ದೊಡ್ಡವನಾಗಿ, ವಿವೇಕ ಪ್ರಜ್ಞೆ ಬೆಳೆದು, ಸರ್ವಸಂಗ ಪರಿತ್ಯಾಗ ಮಾಡಬೇಕಾದರೆ ಇನ್ನೂ ಎಷ್ಟೋ ವರ್ಷಗಳು ಕಳೆಯಬೇಕು; ಈಗಲೇ ಈ ವಿಷಯವನ್ನೆಲ್ಲ ತಲೆಗೆ ಹಚ್ಚಿಕೊಂಡು ಚಿಂತಿಸುತ್ತ ಕುಳಿತುಕೊಳ್ಳುವುದೇಕೆ ಎಂದುಕೊಳ್ಳುತ್ತ ಆ ಆಲೋಚನೆಗಳನ್ನೆಲ್ಲ ಬದಿಗೊತ್ತಿದಳು.

ಆದರೆ ಈಚೀಚೆಗಂತೂ ನರೇಂದ್ರ ಆಟದ ಹೆಸರಿನಲ್ಲಿ ಬಹಳವಾಗಿ ಧ್ಯಾನ ಮಾಡುತ್ತಕುಳಿತಿರುವುದನ್ನು ಮನೆಮಂದಿ ಕಾಣುತ್ತಿದ್ದರು. ಅವನು ಸ್ಥಿರವಾಗಿ ಕುಳಿತಿರುತ್ತಿದ್ದ ಕ್ರಮವನ್ನು ನೋಡಿ ಅವನೇನು ಧ್ಯಾನದ ಆಟವಾಡುತ್ತಿದ್ದಾನೆಯೋ ಅಥವಾ ನಿಜವಾದ ಧ್ಯಾನದಲ್ಲೇ ಮಗ್ನನಾಗಿಬಿಟ್ಟಿದ್ದಾನೆಯೋ ಎಂದು ಬೆರಗಾಗಿ ನಿಲ್ಲುತ್ತಿದ್ದರು. ಆ ಧ್ಯಾನ ಅದೆಷ್ಟು ಗಾಢವಾಗಿರುತ್ತಿತ್ತೆಂದರೆ ಕೆಲವೊಮ್ಮೆ ಅವನ ಮೈಯನ್ನು ಕುಲುಕಿ ಅಲ್ಲಾಡಿಸಿ ಎಬ್ಬಿಸಬೇಕಾಗುತ್ತಿತ್ತು. ಇನ್ನು ಕೆಲವು ಸಲವಂತೂ ಮೈ ಕುಲಕುಕಿದರೂ ಎಬ್ಬಿಸಲು ಸಾಧ್ಯವಾಗುತ್ತಿರಲಿಲ್ಲ. ಒಂದು ದಿನ ಅವನು ಈ ಧ್ಯಾನದ ಆಟದಲ್ಲಿ ತನ್ನ ಪಕ್ಕದ ಮನೆಯ ಹುಡುಗರನ್ನೂ ಸೇರಿಸಿಕೊಂಡಿದ್ದ. ಸಂಜೆಗತ್ತಲು; ಹುಡುಗರೆಲ್ಲ ಮನೆಯ ಪ್ರಾರ್ಥನಾ ಮಂದಿರದಲ್ಲಿ ಧ್ಯಾನನಿರತರಾಗಿದ್ದರು. ಆದರೆ ಕೆಲವು ಹುಡುಗರು ಮಧ್ಯೆಮಧ್ಯೆ ಅರೆಗಣ್ಣು ತೆರೆದು ಇತರರೆಲ್ಲ ಹೇಗೆ ಧ್ಯಾನ ಮಾಡುತ್ತಿದ್ದಾರೆ ಎನ್ನುವುದನ್ನು ಕದ್ದು ನೋಡುತ್ತಿದ್ದರು. ಆ ಹೊತ್ತಿಗೆ ಅಲ್ಲೊಂದು ದೊಡ್ಡ ಸರ್ಪ ಆ ಕಲ್ಲುಚಪ್ಪಡಿ ಹಾಸಿದ ನೆಲದ ಮೇಲೆ ಹರಿದಾಡಿಕೊಂಡು ಬಂತು. ಒಬ್ಬ ಹುಡುಗ ಅದನ್ನು ನೋಡಿಬಿಟ್ಟ. ನೋಡಿದವನೇ, “ಅಯ್ಯೋ, ಹಾವು, ಹಾವು!” ಎಂದು ಗಟ್ಟಿಯಾಗಿ ಕಿರುಚಿಕೊಂಡ. ಮರುಕ್ಷಣವೇ ಹುಡುಗರೆಲ್ಲ ದಡಬಡ ಎದ್ದು ಪರಾರಿ. ಆದರೆ ನರೇಂದ್ರ ಮಾತ್ರ ಗಾಢ ಧ್ಯಾನದಲ್ಲಿ ಕುಳಿತೇ ಇದ್ದ. ಹುಡುಗರೆಲ್ಲ ನರೇಂದ್ರನನ್ನು ದೂರದಿಂದಲೇ ಗಟ್ಟಿಯಾಗಿ ಕೂಗಿ ಕರೆದರು. ಆದರೆ ಅವನು ಮಿಸುಕಲೇ ಇಲ್ಲ. ಹುಡುಗರು ಓಡಿಹೋಗಿ ನರೇಂದ್ರನ ತಾಯ್ತಂದೆಯರಿಗೆ ಸುದ್ದಿ ತಿಳಿಸಿದರು. ಅವರು ಗಾಬರಿಯಾಗಿ ಓಡಿಬಂದು ನೋಡುತ್ತಾರೆ–ಎದೆ ನಡುಗಿಸುವ ದೃಶ್ಯ! ಆ ಸರ್ಪ ತನ್ನ ಹೆಡೆ ಬಿಚ್ಚಿಕೊಂಡು ಓಳ್ಳೇ ಠೀವಿಯಿಂದ ನೋಡುತ್ತಿದೆ! ಏನು ಮಾಡುವುದೀಗ? ಆ ಕಡೆ ನರೇಂದ್ರನಿಗೋ ಬಾಹ್ಯಪ್ರಜ್ಞೆಯೇ ಇದ್ದಂತಿಲ್ಲ. ಈ ಕಡೆ ಸರ್ಪವೋ ಹೆಡೆಬಿಚ್ಚಿಕೊಂಡು ನಿಂತಿದೆ! ಅದನ್ನು ಹೆದರಿಸಿ ಓಡಿಸೋಣವೆಂದರೆ ಇವರೇ ಭಯದಿಂದ ತಲ್ಲಣಿಸುತ್ತಿದ್ದಾರೆ. ಆದರೆ ಅದೃಷ್ಟವಶಾತ್ ಅದು ತಾನಾಗಿಯೇ ಹೊರಟು ಹೋಯಿತು. ಆಮೇಲೆ ಹುಡುಕಿನೋಡಿದರೂ ಅದು ಸಿಗಬೇಕಲ್ಲ! ಬಳಿಕ ತಾಯ್ತಂದೆಯರು ನರೇಂದ್ರನನ್ನು ಕೇಳಿದರು, “ಅಲ್ಲವೋ, ಹಾವು ಬಂತು ಅಂತ ಹುಡುಗರೆಲ್ಲ ಅಷ್ಟು ಗಟ್ಟಿಯಾಗಿ ಕೂಗಿದರೂ ನೀನೇಕೆ ಎದ್ದು ಬರದೆ ಅಲ್ಲೇ ಕುಳಿತಿದ್ದೆ?” ಅದಕ್ಕೆ ಅವನೆಂದ, “ಏನು! ಹಾವು ಬಂದಿತ್ತೆ? ನನಗೆ ಹಾವು ಬಂದದ್ದೂ ಗೊತ್ತಾಗಲಿಲ್ಲ, ಏನಾಯಿತು ಅಂತಲೂ ಗೊತ್ತಾಗಲಿಲ್ಲ. ನನ್ನಷ್ಟಕ್ಕೆ ನಾನು ತುಂಬ ಆನಂದದಿಂದಿದ್ದೆ.”

ನರೇಂದ್ರನ ಧ್ಯಾನಾನಂದದ ಕುರಿತಾಗಿ ಅವನ ಹೆತ್ತವರಿಗೆ ಅದೆಷ್ಟು ಅರ್ಥವಾಯಿತೋ ಗೊತ್ತಿಲ್ಲ. ಆದರೆ ಅವನಂತೂ ಆನಂದವನ್ನು ಅನುಭವಿಸುತ್ತಿದ್ದುದು ನಿಜವೇ. ಅವನಿಗೆ ಪ್ರತಿ ರಾತ್ರಿಯೂ ಒಂದು ದರ್ಶನಾನುಭವವಾಗುತ್ತಿತ್ತು. ಅವನು ಮಲಗಿ ನಿದ್ರೆಹೋಗುತ್ತಿದ್ದ ಕ್ರಮವೇ ಒಂದು ವಿಶೇಷ. ರಾತ್ರಿ ಮಲಗಿ ಕಣ್ಣುಮುಚ್ಚಿಕೊಳ್ಳುವ ಹೊತ್ತಿಗೆ ಸರಿಯಾಗಿ ಅವನಿಗೆ ಭ್ರೂಮಧ್ಯದಲ್ಲಿ ಒಂದು ಅದ್ಭುತವಾದ ಜ್ಯೋತಿಯ ಕಿಡಿ ಕಾಣುತ್ತಿತ್ತು. ಆ ಜ್ಯೋತಿಯ ವರ್ಣ ಆಗಾಗ ಬದಲಾಗುತ್ತಿತ್ತು. ಅದು ಕ್ರಮೇಣ ಬೆಳೆಯುತ್ತ ಬೆಳೆಯುತ್ತ ಕೊನೆಗೆ ಸ್ಫೋಟಗೊಳ್ಳುತ್ತಿತ್ತು. ಆಗ ವಿಶಾಲವಾಗಿ ಹರಡಿದ ಆ ಶುಭ್ರ ಜ್ಯೋತಿಯ ಅಲೆಗಳು ಇಡೀ ಶರೀರವನ್ನು ಆವರಿಸಿಬಿಡುತ್ತಿದ್ದುವು. ಹೀಗೆ ಅವನ ಮನಸ್ಸು ಈ ದಿವ್ಯಾನುಭವದಲ್ಲಿ ಮಗ್ನವಾಗಿದ್ದರೆ, ಶರೀರ ನಿದ್ರೆಯಲ್ಲಿ ಮಗ್ನವಾಗುತ್ತಿತ್ತು. ಇದು ಅವನ ಪಾಲಿಗೆ ದಿನಂಪ್ರತಿಯ ಘಟನೆಯಾಯಿತು. ಕೆಲವೊಮ್ಮೆ ಅವನು ಆ ಜ್ಯೋತಿಯ ಕಡೆಗೆ ಮನಸ್ಸನ್ನು ಏಕಾಗ್ರಗೊಳಿಸಿ ಇನ್ನಷ್ಟು ಸ್ಪಷ್ಟವಾಗಿ ಅದರ ಅನುಭವ ಪಡೆಯಲು ಬೋರಲಾಗಿ ಮಲಗಿಕೊಳ್ಳುತ್ತಿದ್ದ. ತನಗೆ ಈ ಅನುಭವ ಇಷ್ಟೊಂದು ಸಹಜವಾಗಿ ಆಗುತ್ತಿದ್ದುದರಿಂದ, ಬಹುದಿನಗಳವರೆಗೆ ಅವನು, ಇದು ಎಲ್ಲರಿಗೂ ಆಗುವ ಸರ್ವೇಸಾಮಾನ್ಯ ಅನುಭವ ಎಂದೇ ಭಾವಿಸಿದ್ದ. ಆದ್ದರಿಂದಲೇ ಅವನು ಈ ವಿಷಯವಾಗಿ ಯಾರ ಹತ್ತಿರವೂ ಹೇಳಹೋಗಿರಲಿಲ್ಲ. ಆದರೆ ಮುಂದೊಮ್ಮೆ ಅವನು ತನ್ನ ಸಂಗಡಿಗನೊಬ್ಬನನ್ನು ಸಾಂದರ್ಭಿಕವಾಗಿ “ರಾತ್ರಿ ಮಲಗಿಕೊಂಡಾಗ ನಿನಗೆ ನಿನ್ನ ಹುಬ್ಬುಗಳ ಮಧ್ಯದಲ್ಲಿ ಒಂದು ಬೆಳಕು ಕಾಣಿಸುತ್ತದೆಯೆ?” ಎಂದು ಕೇಳಿದಾಗ ಸ್ನೇಹಿತ “ಇಲ್ಲವಲ್ಲ!” ಎಂದ. ಆಗ ನರೇಂದ್ರ ನುಡಿದ, “ಮತ್ತೆ ನನಗೆ ಕಾಣಿಸುತ್ತದೆಯಲ್ಲಾ?... ನೋಡು, ಇವತ್ತು ನೀನು ಮಲಗಿಕೊಂಡ ತಕ್ಷಣ ನಿದ್ರೆ ಮಾಡಬೇಡ. ಸ್ವಲ್ಪ ಹೊತ್ತು ಹಾಗೇ ಎಚ್ಚರವಾಗಿಯೇ ಇರು; ಆಗ ನಿನಗೇ ಕಾಣಿಸುತ್ತದೆ ಬೆಳಕು.” ಆದರೆ ಹಾಗೆ ಎಚ್ಚತ್ತಿದ್ದರೂ ಇತರರಿಗೆ ಜ್ಯೋತಿ ಕಾಣಿಸಲು ಸಾಧ್ಯವೆ? ನರೇಂದ್ರನ ಸಂಸ್ಕಾರವೇ ಬೇರೆ. ಅನೇಕ ಜನ್ಮಗಳ ಸಾಧನೆಯ ಹಿನ್ನಲೆ ಇದ್ದವರಿಗೆ ಮಾತ್ರ ಈ ಬಗೆಯ ದಿವ್ಯಾನುಭವಗಳಾದಾವು.

