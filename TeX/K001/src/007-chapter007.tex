
\chapter{ಶ್ರೀರಾಮಕೃಷ್ಣ}

ಬಂಗಾಳದಲ್ಲಿ ಕಲ್ಕತ್ತದಿಂದ ಸುಮಾರು ಅರವತ್ತು ಮೈಲಿ ದೂರಕ್ಕೆ ಕಾಮರಪುಕುರ ಎನ್ನುವ ಒಂದು ಹಳ್ಳಿ. ಅಲ್ಲಿ ಚಂದ್ರಮಣೀ ದೇವಿ, ಕ್ಷುದಿರಾಮ ಚಟ್ಟೋಪಾಧ್ಯಾಯರೆಂಬ ಬಡ ಬ್ರಾಹ್ಮಣ ದಂಪತಿಗಳು. ಬಡವರಾದರೂ ಅವರ ಸಾಧುತ್ವವನ್ನು ಗುರುತಿಸಿ ಹಳ್ಳಿಯ ಜನರೆಲ್ಲ ಅವರನ್ನು ತುಂಬ ಗೌರವದಿಂದ ನೋಡುತ್ತಿದ್ದರು. ಅವರ ಕಿರಿಮಗನಾಗಿ ಜನಿಸಿದವನು ಗದಾಧರ. ಅವನು ಬೆಳೆದುಬಂದದ್ದು ಹಳ್ಳಿಯ ಸರಳ ವಾತಾವರಣದಲ್ಲಿ. ಆದರೆ, ಅವನಲ್ಲಿ ಬಾಲ್ಯದಿಂದಲೇ ಅನೇಕ ಅದ್ಭುತ ಲಕ್ಷಣಗಳು ಕಾಣತೊಡಗಿದ್ದುವು. ಭಗವನ್ನಾಮ ಸಂಕೀರ್ತನೆಯನ್ನು ಕೇಳಿದರೆ ಅವನು ಭಾವಪರವಶನಾಗುತಿದ್ದ; ಶ್ಯಾಮಲ ಮೇಘಗಳಿಗೆ ಇದಿರಾಗಿ ಬೆಳ್ವಕ್ಕಿಗಳು ಹಾರುವ ರಮ್ಯ ದೃಶ್ಯವನ್ನು ಕಂಡೇ ಮೈಮರೆಯುತ್ತಿದ್ದ. ಹುಡುಗ ಸಣ್ಣ ವಯಸ್ಸಿನವನಾಗಿರುವಾಗಲೇ ತಂದೆ ತೀರಿಕೊಂಡ. ಮೊದಲೇ ಬಡ ಸಂಸಾರ, ಈಗ ಇನ್ನಷ್ಟು ಕಷ್ಟಗಳಿಗೆ ಗುರಿಯಾಯಿತು. ಗದಾಧರನ ಹಿರಿಯಣ್ಣ ರಾಮಕುಮಾರ ಕಲ್ಕತ್ತಕ್ಕೆ ಬಂದು ಸಂಸ್ಕೃತ ಪಾಠಶಾಲೆಯೊಂದನ್ನು ಆರಂಭಿಸಿದ. ಉದರನಿಮಿತ್ತವಾಗಿ ತರುಣ ಗದಾಧರನೂ ಅಣ್ಣನ ಜೊತೆಗೆ ಬರಬೇಕಾಯಿತು. ಇಲ್ಲಿ ರಾಮಕುಮಾರ ತಮ್ಮನನ್ನು ವಿದ್ಯಾಭ್ಯಾಸಕ್ಕೆ ಹಾಗಿದ. ಆದರೆ ಗದಾಧರ ತನ್ನ ಜೀವನದ ಗುರಿಯ ಬಗ್ಗೆ, ದಾರಿಯ ಬಗ್ಗೆ ಬಹಳಷ್ಟು ಆಲೋಚಿಸಿ ಒಂದು ಇತ್ಯರ್ಥಕ್ಕೆ ಬರಲಾರಂಭಿಸಿದ. ಅವನು ತನ್ನನ್ನೇ ಕೇಳಿಕೊಳ್ಳುತ್ತಲಿದ್ದ: “ಈ ಲೌಕಿಕ ವಿದ್ಯಾಭ್ಯಾಸದಿಂದ ಭಗವಂತನನ್ನು ಸಾಕ್ಷಾತ್ಕರಿಸಿಕೊಳ್ಳಲಾಗುವುದೆ?ಅದಕ್ಕೆ ಅಗತ್ಯವಾದ ಜ್ಞಾನವನ್ನು-ಭಕ್ತಿಯನ್ನು ಬೆಳೆಸಿಕೊಳ್ಳಬಹುದೆ?” ‘ಇಲ್ಲ, ಸಾಧ್ಯವಿಲ್ಲ!’ಎಂದು ಅವನ ಮನಸ್ಸೇ ಸ್ಪಷ್ಟವಾಗಿ ಹೇಳುತ್ತಿತ್ತು. ‘ಈ ಶಿಕ್ಷಣವನ್ನು ಪಡೆಯುವುದರ ಮೂಲಕ ನಾನು ಜೀವನಸಾರ್ಥಕ್ಯವನ್ನು ಹೊಂದಬಲ್ಲೆನೆ? ಅಜ್ಞಾನದ ಕತ್ತಲಿನಿಂದ ಪಾರಾಗಬಲ್ಲೆನೆ? ಭೋಗಾಸ್ಕತಿಯಿಂದ ಮುಕ್ತನಾಗಬಲ್ಲೆನೆ? ’ ‘ಇಲ್ಲ ಸಾಧ್ಯವಿಲ್ಲ!’ ಎನ್ನುವ ಅದೇ ಉತ್ತರ ಪ್ರತಿಧ್ವನಿಸುತ್ತಿತ್ತು ಅವನ ಹೃದಯದಾಳದಿಂದ. ‘ಹಾಗಾದರೆ ಈ ವಿದ್ಯೆಯಿಂದ ನನಗೇನಾಗಬೇಕಾಗಿದೆ? ಯಾವ ವಿದ್ಯೆ ಭಗವಂತನ ಸಾಕ್ಷಾತ್ಕಾರವನ್ನು ದೊರಕಿಸಿಕೊಡಲಾರದೋ, ಲೋಕದ ದುಃಖಪರಂಪರೆಯಿಂದ ಪಾರುಮಾಡಲಾರದೋ ಆ ವಿದ್ಯೆಯಿಂದ ಏನು ಪ್ರಯೋಜನ? ಆ ವಿದ್ಯೆಯನ್ನು ಪಡೆಯುವುದರ ಬದಲು ನಾನು ಅನಕ್ಷರಸ್ಥನಾಗಿಯೇ ಉಳಿದುಕೊಂಡು ನೇರವಾಗಿ ಭಗವಂತನ ಸಾಕ್ಷಾತ್ಕಾರಕ್ಕಾಗಿ ಪ್ರಯತ್ನಿಸುತ್ತ ಮುಂದುವರಿಯುತ್ತೇನೆ’ ಎಂದೆನ್ನುವ ತೀರ್ಮಾನಕ್ಕೆ ಬಂದ ಗದಾಧರ. ಅಣ್ಣ ತನ್ನನ್ನು ಶಾಲೆಗೆ ಹೋಗಿ ವಿದ್ಯಾಭ್ಯಾಸ ಮಾಡುವಂತೆ ಪ್ರೀತಿಯಿಂದ ಬಲವಂತಪಡಿಸಿದಾಗ ಗದಾಧರ ತನ್ನ ತೀರ್ಮಾನವನ್ನು ಹೇಳಿಬಿಟ್ಟ: “ಅಣ್ಣ, ಈ ಹೊಟ್ಟೆಪಾಡಿನ ವಿದ್ಯೆಯನ್ನು ಕಟ್ಟಿಕೊಂಡು ನಾನೇನು ಮಾಡಲಿ? ಇದರ ಬದಲಿಗೆ ಯಾವ ಜ್ಞಾನವನ್ನು ಪಡೆಯುವುದರ ಮೂಲಕ ಹೃದಯ ಬೆಳಗುತ್ತದೆಯೋ ಮತ್ತು ನಿತ್ಯತೃಪ್ತಿ ಸಿಗುತ್ತದೆಯೋ ಅಂತಹ ಜ್ಞಾನವನ್ನು ಪಡೆಯುತ್ತೇನೆ.” ಕೀಶೋರನ ಬಾಯಲ್ಲಿ ಎಂತಹ ಅದ್ಭುತ ಮಾತಿದು! 

ಇದೇ ಸಮಯದಲ್ಲಿ ಕಲ್ಕತ್ತಕ್ಕೆ ಅನತಿದೂರದ ದಕ್ಷಿಣೇಶ್ವರದಲ್ಲಿ ಒಂದು ಭವ್ಯ ಕಾಳೀ ದೇವಾಲಯದ ನಿರ್ಮಾಣವಾಯಿತು. ಇದನ್ನು ಕಟ್ಟಿಸಿದವಳು ರಾಣಿ ರಾಸಮಣಿ ಎಂಬವಳು. ಈ ದೇವಾಲಯದ ಅರ್ಚಕನಾಗಿ ಬಂದು ಅನುಗ್ರಹಿಸುವಂತೆ, ವಿಪ್ರಶ್ರೇಷ್ಠನೂ ವಿಶಾಲಬುದ್ಧಿಯವನೂ ಆದ ರಾಮಕುಮಾರನನ್ನು ರಾಸಮಣಿ ಕೇಳಿಕೊಂಡಳು. ತನಗೆ ಅದರಲ್ಲಿ ವಿಶೇಷ ಆಸಕ್ತಿಯೇನಿಲ್ಲದಿದ್ದರೂ, ಅದೇ ದೈವೇಚ್ಛೆಯೆಂದರಿತು ರಾಮಕುಮಾರ ಒಪ್ಪಿಕೊಂಡ; ತಮ್ಮ ಗದಾಧರನನ್ನು ಕರೆದುಕೊಂಡು ದಕ್ಷಿಣೇಶ್ವರಕ್ಕೆ ಬಂದುಬಿಟ್ಟ. ಕಾಳೀದೇವಾಲಯದ ಅರ್ಚಕನಾಗಿ ಕಾರ್ಯಾರಂಭ ಮಾಡಿದ. ಈ ದೇವಾಲಯವಿರುವುದು ಪರಮಪಾವನಕರ ಗಂಗಾತೀರದಲ್ಲಿ. ಸುತ್ತ ಮುತ್ತೆಲ್ಲ ದಟ್ಟವಾದ ವನರಾಶಿ. ಪರಮ ಪ್ರಶಾಂತ ವಾತಾವರಣ, ಕಾಳಿಕಾದೇವಿಯ ಸನ್ನಿಧಿ– ಇವೆಲ್ಲವುಗಳಿಂದಾಗಿ ಗದಾಧರನ ಮನಸ್ಸಿನಲ್ಲಿ ಉನ್ನತ ಬಾವ ಜಾಗೃತವಾಗಿ ಬಿಟ್ಟಿತು. ಮೊದಲೇ ಅವನಲ್ಲಿ ಕಿಡಿಯ ರೂಪದಲ್ಲಿದ್ದ ಭಗವತ್ಸಾಕ್ಷಾತ್ಕಾರದ ಹಂಬಲ ಈಗ ಭುಗಿಲ್ಲೆದ್ದಿತು. ಆಸ್ಥೆ ವಹಿಸಿ ಅಣ್ಣನಿಂದ ಪೂಜಾಕಾರ್ಯಗಳನ್ನು ಕಲಿತು ಆತನಿಗೆ ನೆರವಾಗತೊಡಗಿದ. ಮತ್ತು ತಾನು ಸಾಕ್ಷಾತ್ ದೇವತೆಯ ಎದುರು ಕುಳಿತು ಪೂಜೆ ನೈವೇದ್ಯಾದಿಗಳನ್ನು ಸಲ್ಲಿಸುತ್ತಿರುವಂತೆ ಭಾವಿಸುತ್ತಿದ್ದ. ಕ್ರಮೇಣ ಅರ್ಚಕಕಾರ್ಯದ ಹೊಣೆಗಾರಿಕೆ ಅವನ ಮೇಲೆಯೇ ಬಂದಿತು. ಇದೇ ವೇಳೆಗೆ ರಾಮಕುಮಾರ ತನ್ನ ಕರ್ತವ್ಯವನ್ನು ಈಡೇರಿಸಿದ ಸಂತೃಪ್ತಿಯಿಂದಲೋ ಎಂಬಂತೆ, ಇದ್ದಕ್ಕಿದ್ದಂತೆ ಇಹಲೋಕ ಯಾತ್ರೆಯನ್ನು ಮುಗಿಸಿದ.

ಈಗ ಗದಾಧರ ಶ್ರೀರಾಮಕೃಷ್ಣರೆಂಬ ಹೆಸರಿನಿಂದ ಭವತಾರಿಣೀ ಕಾಳಿಯ ಅರ್ಚಕನಾದ, ಭಕ್ತವರೇಣ್ಯನಾದ, ಸಾಧಕನಾದ, ಸಿದ್ಧನಾದ, ಪ್ರಸಿದ್ಧನಾದ–ಪರಮಹಂಸನಾದ. ಅದು ಹನ್ನೆರಡು ವರ್ಷಗಳ ಅವಿರತ, ಅಭೂತಪೂರ್ವ, ಪರಮಾದ್ಭುತ ಸಾಧನೆಯ ಫಲ. ಶ್ರೀರಾಮಕೃಷ್ಣರ ಆಧ್ಯಾತ್ಮಿಕ ಸಾಧನೆಯನ್ನು, ಅವರ ಭಗವತ್ಸಾಕ್ಷಾತ್ಕಾರದ ಹಂಬಲವನ್ನು, ತ್ಯಾಗವನ್ನು, ಸತ್ಯಸಂಧತೆಯನ್ನು, ಏಕನಿಷ್ಠೆಯ ಭಕ್ತಿಯನ್ನು, ಅವರಿಗೆ ದೊರಕಿದ ಆನಂದಾನುಭವವನ್ನು ಮಾತುಗಳ ಮೂಲಕ ವರ್ಣಿಸಲಸಾಧ್ಯ. ಶ್ರೀರಾಮಕೃಷ್ಣರಿಗೆ ಶಾಸ್ತ್ರಗಳ ಪರಿಚಯವೇ ಇರಲಿಲ್ಲ. ಆದ್ದರಿಂದ ಜಟಿಲ ಯೋಗಾಭ್ಯಾಸಗಳ ಸೂಕ್ಷವೂ ತಿಳಿದಿರಲಿಲ್ಲ. ಅವರು ತಮ್ಮ ಸಾಧನೆಯ ಕಾಲದಲ್ಲಿ ಪಡೆದ ಗುರುಸಹಾಯ, ಗುರುಮಾರ್ಗದರ್ಶನ ಅತ್ಯಲ್ಪ. ಸಾಧನೆಯ ಪ್ರಾರಂಭದಲ್ಲಿ ಅವರಲ್ಲಿದ್ದುದು ಮಗು ತನ್ನ ತಾಯಿಗಾಗಿ ಹಂಬಲಿಸುವಂತಹ ವ್ಯಾಕುಲತೆ ಮತ್ತು ಇಂದ್ರಿಯ ಸುಖದ ವಿಚಾರದಲ್ಲಿ ಸಂಪೂರ್ಣ ವೈರಾಗ್ಯ ಮಾತ್ರ ಹಗಲೆಲ್ಲ ಕಾಳಿಕಾದೇವಿಯ ಪೂಜೆ-ಪ್ರಾರ್ಥನೆ-ಕೀರ್ತನೆಗಳಲ್ಲಿ ಕಲೆದರೆ, ಪ್ರಾತಃಕಾಲ-ಸಾಯಂಕಾಲಗಳು ಗಂಗೆಯ ತೀರದಲ್ಲಿ ಕಾಳಿಯ ಸ್ಮರಣ-ಮನನದಲ್ಲಿ ಕಳೆಯುತ್ತಿದ್ದುವು; ಇನ್ನು ರಾತ್ರಿಯೆಲ್ಲ ಆಕೆಯ ಧ್ಯಾನ. ನಿಜಕ್ಕೂ ಆ ಕಾಳೀಮಾತೆ–ತಮ್ಮ ಆರಾಧ್ಯದೈವ–ಇರುವಳೆ? ಇರುವಳಾದರೆ ಅವಳೆಲ್ಲಿ? ಆಕೆ ಕಣ್ಣಿಗೇಕೆ ಕಾಣದಿದ್ದಾಳೆ? ಇಷ್ಟು ಕರೆದರೂ ಮೈದೋರದಿದ್ದಾಳೆ?... ನಿರಂತರವೂ ಇದೊಂದೇ ತುಡಿತ. ಈ ವೇಳೆಗೆ ಅವರ ನೆಂಟನಾದ ಹೃದಯರಾಮ ಎಂಬ ತರುಣನೊಬ್ಬ ಅವರ ಸಹಾಯಕನಾಗಿ ಕೆಲಸಕ್ಕೆ ಸೇರಿದ್ದ. ಈತ ಅವರಲ್ಲಿ ತುಂಬ ವಿಶ್ವಾಸದಿಂದಿದ್ದು ಅವರ ದೈಹಿಕ ಯೋಗಕ್ಷೆಮವನ್ನು ನೋಡಿಕೊಳ್ಳುತ್ತಿದ್ದ.

ಶ್ರೀರಾಮಕೃಷ್ಣರು ಹಗಲಿರುಳೂ ಕಾಳೀದೇವಿಯ ಸಾಕ್ಷಾತ್ಕಾರಕ್ಕಾಗಿ ಕಾತರರಾಗಿರುತ್ತಿದ್ದರು. ಅವರ ಮನಸ್ಸಿನ ಒಂದೇ ಒಂದು ಮಹಾತ್ವಾಕಾಂಕ್ಷೆಯೆಂದರೆ ಕಾಳಿಕಾದೇವಿಯ ಸಾಕ್ಷಾತ್ಕಾರ. ದಿವಸಗಳು ಕಳೆದುವು, ಮಾಸಗಳು ಉರುಳಿದುವು. ಆದರೆ ಇನ್ನೂ ಅವಳ ದರ್ಶನವಾಗುವ ಸೂಚನೆಯೇ ಇಲ್ಲ! ಇಷ್ಟಾದರೂ ರಾಮಕೃಷ್ಣರ ಉತ್ಸಾಹ ಕಡಿಮೆಯಾಗಲಿಲ್ಲ; ಬದಲಿಗೆ‘ಇನ್ನೂ ಸಾಕ್ಷಾತ್ಕಾರವಾಗಲಿಲ್ಲವಲ್ಲಾ’ ಎನ್ನುವ ಹೃದಯದ ಬೇಗೆ ಅಧಿಕವಾಗುತ್ತ ಬಂದಿತು. ಸಂಜೆಯ ಹೊತ್ತು ಗಂಗೆಯ ದಡದಲ್ಲಿ ನಿಂತು ಗಟ್ಟಿಯಾಗಿ ಅಳುತ್ತ ಜಗನ್ಮಾತೆಯಲ್ಲಿ ಮೊರೆಯಿಡುತ್ತಾರೆ: “ಅಮ್ಮ, ಇನ್ನೂ ಒಂದು ದಿನ ವ್ಯರ್ಥವಾಗಿ ಹೋಯಿತು. ಆದರೆ ನಿನ್ನ ದರ್ಶನ ಮಾತ್ರ ಆಗಲಿಲ್ಲವಲ್ಲ! ಅಯ್ಯೋ! ಈ ಕ್ಷಣಿಕ ಜೀವನದ ಇನ್ನೂ ಒಂದು ದಿನ ಕಳೆದುಹೋಯಿತಲ್ಲಮ್ಮಾ!” ಕೆಲವೊಮ್ಮೆ ಅವರ ಮನಸ್ಸಿನಲ್ಲಿ ಸಂಶಯ ಸುಳಿಯುವುದೂ ಉಂಟು. ಆಗ ಜಗನ್ಮಾತೆಯನ್ನು ಕೇಳುತ್ತಾರೆ: “ಅಮ್ಮಾ, ನೀನು ನಿಜವಾಗಿಯೂ ಇರುವುದು ಹೌದೇನಮ್ಮ? ಅಥವಾ ಎಲ್ಲವೂ ಕಟ್ಟುಕತೆಯೇ? ಪುರಾಣದ ಕಂತೆಯೇ? ಅಮ್ಮಾ, ನೀನು ಇರುವುದೇ ಹೌದಾದಲ್ಲಿ, ನನಗೇಕೆ ನೀನು ಕಾಣಿಸುತ್ತಿಲ್ಲ? ಈ ಅಧ್ಯಾತ್ಮವೆನ್ನುವುದೆಲ್ಲ ಕೇವಲ ಭ್ರಾಂತಿಯೇ ಹಾಗಾದರೆ? ಆಕಾಶಗೋಪುರ ಮಾತ್ರವೆ?” ಆದರೆ ಅವರ ಮನದಲ್ಲೆದ್ದ ಈ ಬಗೆಯ ಸಂಶಯ-ನಾಸ್ತಿಕತೆಯೆಲ್ಲ ಕೇವಲ ಕ್ಷಣಕಾಲದವರೆಗೆ ಮಾತ್ರ. ಮರುಕ್ಷಣವೇ, ಹಿಂದೆ ಸಾಕ್ಷಾತ್ಕಾರ ಮಾಡಿಕೊಂಡವರೆಲ್ಲ ಅವರ ನೆನಪಿಗೆ ಬರುತ್ತಾರೆ. ಆಗ ಅವರ ವ್ಯಾಕುಲತೆ ದ್ವಿಗುಣವಾಗುತ್ತದೆ, ಇನ್ನಷ್ಟು ತೀವ್ರವಾದ ಸಾಧನೆಯಲ್ಲಿ ನಿರತರಾಗಿಬಿಡುತ್ತಾರೆ.

ಒಂದು ದಿನವಂತೂ ಅವರ ದುಃಖ ಅಸಹನೀಯವಾಯಿತು. ಹೃದಯದಲ್ಲಿ ತಾಳಲಾರದ ವೇದನೆ. ಭಗವತಿಯ ಸಾಕ್ಷಾತ್ಕಾರವಾಗದಿದ್ದ ಮೇಲೆ ಬದುಕಿಯಾದರೂ ಏನು ಪ್ರಯೋಜನ ಎನ್ನುವ ಭಾವನೆಯಿಂದ ಅವರು ತಮ್ಮ ಜೀವನವನ್ನೇ ಕೊನೆಗೊಳಿಸಿಕೊಂಡುಬಿಡುವ ನಿರ್ಧಾರ ಮಾಡಿದರು. ಕಾಳಿಕಾದೇವಿಯ ಪಕ್ಕದಲ್ಲಿಯೇ ಗೋಡೆಗೆ ತೂಗುಹಾಕಿದ್ದ ಖಡ್ಗ ಕಂಡಿತು. ಮುನ್ನುಗ್ಗಿ ಅದನ್ನು ಕೈಗೆತ್ತಿಕೊಂಡರು. ಇನ್ನೇನು ಅವರ ಶಿರ ಉರುಳಿಬೀಳಬೇಕು... ಅದ್ಭುತ! ಆ ಕ್ಷಣದಲ್ಲೇ ಕಾಳಿಕಾದೇವಿ ಕೋಟಿಸೂರ್ಯಪ್ರಕಾಶದಿಂದ ಪ್ರತ್ಯಕ್ಷಳಾದಳು! ಶ್ರೀರಾಮಕೃಷ್ಣರು ಬಾಹ್ಯಪ್ರಜ್ಞೆ ತಪ್ಪಿ ಬಿದ್ದುಬಿಟ್ಟರು. ಆ ಬಳಿಕ ಏನಾಯಿತೆಂಬುದು ಅವರಿಗೆ ತಿಳಿಯಲಿಲ್ಲ. ಆದರೆ ಅವರ ಹೃದಯದೊಳಗೆ ಈಗ ಅಪಾರವಾದ ಅಲೌಕಿಕ ಆನಂದ ನಿರಂತರವಾಗಿ ಹರಿಯುತ್ತಿತ್ತು. ಅಂತಹ ದಿವ್ಯಾನಂದವನ್ನು ಅವರು ಹಿಂದೆಂದೂ ಅನುಭವಿಸಿರಲಿಲ್ಲ–ಅಂತಹ ಅವರ್ಣನೀಯ ಆನಂದ!

ಇದಾದ ನಂತರವೂ ಅವರಿಗೆ ಹಲವಾರು ಅದ್ಭುತ ದರ್ಶನಗಳಾಗುತ್ತಲೇ ಇದ್ದುವು. ತಾವು ಯಾವುದೋ ಬೇರೆಯೇ ಲೋಕದಲ್ಲಿರುವಂತೆ ಭಾಸವಾಗುತ್ತಿತ್ತು. ಸಮಾಧಿಸ್ಥಿತಿಯಲ್ಲೂ ತೆರೆಗಣ್ಣಿಗೂ ನಾನಾ ಬಗೆಯ ಹಲವು ದರ್ಶನಗಾಳಾದುವು. ಆದರೆ ಅವರ ವರ್ತನೆಯನ್ನು ಮಾತುಕತೆಯನ್ನು ಜನ ಹುಚ್ಚು ಎಂದು ಅರ್ಥಮಾಡಿಕೊಂಡರು.

ಹೀಗೆ ಜಗನ್ಮಾತೆಯ ದರ್ಶನವೇ ಮೊದಲಾದ ದಿವ್ಯಾನುಭವಗಳೆಲ್ಲ ಆದರೂ ಶ್ರೀರಾಮಕೃಷ್ಣರಿಗೆ ಮಾತ್ರ ಸಮಾಧಾನವಿಲ್ಲ. ಏಕೆಂದರೆ, ಜಗನ್ಮಾತೆಯನ್ನು ಸದಾ ಸರ್ವದಾ ಕಾಣುತ್ತಿರಲು ಸಾಧ್ಯವಾಗಿಲ್ಲವಲ್ಲ! ಆಕೆ ಹೀಗೆ ಒಮ್ಮೆ ದರ್ಶನ ಕೊಟ್ಟು ಓಡಿಹೋಗಲು ಬಿಡುವವರಲ್ಲ ಅವರು. ಈಗ ಮತ್ತೊಮ್ಮೆ ಸಾಧನೆಯಲ್ಲಿ ತೊಡಗಿದರು. ಈ ಸಲ ಇನ್ನಷ್ಟು ತೀವ್ರವಾದ ಸಾಧನೆ: ಇನ್ನಷ್ಟು ವ್ಯಾಕುಲ ಹೃದಯದ ಪ್ರಾರ್ಥನೆ! ಇದರ ಪರಿಣಾಮವಾಗಿ ಅವರ ಆಧ್ಯಾತ್ಮಿಕ ಅರಿವು ಇನ್ನೂ ಆಳವಾಗುತ್ತ ಬಂದಿತು; ನಿರಂತರವೂ ಜಗನ್ಮಾತೆಯ ದರ್ಶನ ದೊರಕುವಂತಾಯಿತು. ಈಗ ಅವರ ಪಾಲಿಗೆ ದೇವಾಲಯದ ಗರ್ಭಗುಡಿಯಲ್ಲಿ ಕಾಳಿಯ ಪ್ರತಿಮೆ ಇಲ್ಲವಾಗಿಬಿಟ್ಟಿದೆ; ಅಲ್ಲಿ ಈಗ ಸಾಕ್ಷಾತ್ ಚಿನ್ಮಯೀ ಕಾಳಿಯೇ ನಿಂತಿದ್ದಾಳೆ! ಅವರನ್ನು ಸದಾ ಆಶೀರ್ವದಿಸುತ್ತ ಮಂದಸ್ಮಿತಳಾಗಿ ತಾಯಿ ಕಾಳಿಯೇ ಅಲ್ಲಿ ನಿಂತಿದ್ದಾಳೆ! ಅವಳು ಅಲ್ಲಿ ಎಷ್ಟು ಜೀವಂತಳಾಗಿ ನಿಂತಿರುವಳೆಂದರೆ ಶ್ರೀರಾಮಕೃಷ್ಣರಿಗೆ ಅವಳ ಉಸಿರಾಟವೂ ಸ್ಪಷ್ಟವಾಗಿ ಗೋಚರಿಸುತ್ತಿದೆ. ಅವಳ ಕಾಲಂದುಗೆಯ ಝಣತ್ಕಾರ ಕೇಳಿಸುತ್ತಿದೆ. ತಾನು ಜಗನ್ಮಾತೆಯಿಂದ ದೂರದಲ್ಲಿದ್ದೇನೆ ಎನ್ನುವ ಭಾವನೆಯೇ ಅವರ ಮನಸ್ಸನಿಂದ ಹೊರಟುಹೋಗಿದೆ.

ಈ ಬಗೆಯ ನಿರಂತರ ಆನಂದಾನುಭವದಿಂದ ಶ್ರೀರಾಮಕೃಷ್ಣರ ಶರೀರವೆಲ್ಲ ತೀರ ಸೂಕ್ಷ್ಮವಾಗಿ, ಅದು ಕೇವಲ ಆಧ್ಯಾತ್ಮಿಕ ಸುಖಾನುಭವಕ್ಕೆ ಮಾತ್ರ ಹೊಂದಿಕೊಳ್ಳುವಂತಾಗಿಬಿಟ್ಟಿತ್ತು. ಲೈಂಗಿಕ ಸುಖಾಭಿಲಾಷೆಯನ್ನೊಳಗೊಂಡಂತೆ ಇಂದ್ರಿಯ ಭೋಗಾಸಕ್ತಿ ಅವರ ಮನಸ್ಸಿನಿಂದ ಸಂಪೂರ್ಣ ಅಳಿಸಿಹೋಗಿಬಿಟ್ಟಿತ್ತು. ಆದರೆ ಅದೆಂದು ಅಸಹಜ-ಅನಿಷ್ಟ ಸ್ಥಿತಿಯೆಂದು ಭಾವಿಸಿದ ರಾಸಮಣಿಯ ಅಳಿಯ ಮಥುರಾನಾಥ ಅವರ ಮನಸ್ಸನ್ನು ವಿಷಯಸುಖದ ಕಡೆಗೆ ಸೆಳೆದು, ಅವರ ದೇಹ-ಮನಸ್ಸುಗಳನ್ನು ಸುಸ್ಥಿತಿಗೆ ತರುವೆನೆಂಬ ಭ್ರಮೆಯಿಂದ ಗುಟ್ಟಾಗಿ ವೇಶ್ಯಾಸಂಪರ್ಕಕ್ಕೆ ವ್ಯವಸ್ಥೆ ಮಾಡಿಸಿದ. ಪವಿತ್ರತೆಯ, ಆತ್ಮಸಂಯಮದ ಮೂರ್ತಿಯಾದ ಶ್ರೀರಾಮಕೃಷ್ಣರು ಈ ಪರೀಕ್ಷೆಯಲ್ಲಿ ಉತ್ತೀರ್ಣರಾದರೆಂದು ಹೇಳಬೇಕಾಗಿಯೇ ಇಲ್ಲ. ಅವರು ಕನಸಿನಲ್ಲೂ ಕೂಡ ಸ್ತ್ರಿಯನ್ನು ಸಾಕ್ಷಾತ್ ಜಗನ್ಮಾತೆಯೆಂದೇ ತಿಳಿದವರೇ ಹೊರತು ಅನ್ಯಥಾ ಅಲ್ಲ.

ಅತ್ಯಂತ ತೀವ್ರವಾದ ಬಗೆಬಗೆಯ ಸಾಧನೆಗಳ ಪರಿಣಾಮವಾಗಿ ಶ್ರೀರಾಮಕೃಷ್ಣರ ಶರೀರ ಬಳಲಿತ್ತು; ಅಲ್ಲದೆ ಅವರ ವಿಚಿತ್ರ ಸಾಧನೆ, ವಿಚಿತ್ರ ನಡವಳಿಕೆಯಗಳನ್ನು ಕಂಡ ಜನ ಅವರಿಗೆ ಹುಚ್ಚು ಹಿಡಿದಿದೆಯೆಂದೇ ನಂಬಿದರು. ತಾಯಿ ಚಂದ್ರಮಣಿದೇವಿಗೂ ಈ ಸುದ್ಧಿ ತಲುಪಿತು. ಅವಳು ಕಳವಳಗೊಂಡು ಮಗನನ್ನು ಊರಿಗೆ ಕರೆಸಿಕೊಂಡಳು. ಮತ್ತು ಅವರನ್ನು‘ಸ್ವಾಸ್ಥ್ಯ’ಕ್ಕೆ ತರಲು ಮದುವೆ ಮಾಡಿಬಿಡುವ ಪ್ರಯತ್ನದಲ್ಲಿ ತೊಡಗಿದಳು. ಶ್ರೀರಾಮಕೃಷ್ಣರು ಇದಕ್ಕೆ ತಮ್ಮ ಅನುಮತಿಯನ್ನು ಕೊಟ್ಟದ್ದಲ್ಲದೆ ತಮಗೆ ಮಡದಿಯಾಗಿ ಬರಬೇಕಾದ ಹುಡುಗಿ ಯಾರು ಎನ್ನುವುದನ್ನು ತಾವೇ ಸೂಚಿಸಿದರು. ಹುಡುಗಿ, ಸಮೀಪದ ಜಯರಾಂಬಾಟಿ ಎಂಬ ಹಳ್ಳಿಯ ರಾಮಚಂದ್ರ ಮುಖರ್ಜೀ ಮತ್ತು ಶ್ಯಾಮಸುಂದರೀ ದೇವಿ ಎಂಬ ದಂಪತಿಗಳ ಮಗಳು–ಶಾರದೆ. ಶ್ರೀರಾಮಕೃಷ್ಣರ ವಯಸ್ಸು ೨೩ ವರ್ಷವಾದರೆ, ಆ ಹುಡುಗಿಗಿನ್ನೂ ಐದು ವರ್ಷ! ಆದರೂ ಮದುವೆ ನಡೆದೇಹೋಯಿತು. ಶ್ರೀರಾಮಕೃಷ್ಣರು ಕೆಲಕಾಲ ಕಾಮರಪುರಕುರದಲ್ಲಿದ್ದು ಆರೋಗ್ಯ ಸುಧಾರಿಸಿಕೊಂಡು ಮತ್ತೆ ದಕ್ಷಿಣೇಶ್ವರಕ್ಕೆ ಹೊರಟರು. ಶಾರದೆ ತನ್ನ ತವರಿಗೆ ಮರಳಿದಳು.

ಈಗ ಶ್ರೀರಾಮಕೃಷ್ಣರಲ್ಲಿ ಇನ್ನೂ ತೀವ್ರವಾದ ಆಧ್ಯಾತ್ಮಿಕ ತೃಷೆ ಉತ್ಪನ್ನವಾಯಿತು. ಆಧ್ಯಾ ತ್ಮಿಕತೆಯೆನ್ನುವುದು ಅನಂತ. ಆ ಅನಂತತೆಯ ಸ್ಪಷ್ಟ ಅನುಭವ ಮಾಡಿಕೊಳ್ಳಬೇಕೆಂಬ ಹಂಬಲ ಶ್ರೀರಾಮಕೃಷ್ಣರಿಗೆ. ಸರಿ, ಮತ್ತೊಮ್ಮೆ ತಾಯಿ, ಮಡದಿ, ಬಂಧುಗಳು ಇವರನ್ನೆಲ್ಲ ಮರೆತು ತೀವ್ರ ಸಾಧನೆಯಲ್ಲಿ ತೊಡಗಿದರು. ಈ ಸಂದರ್ಭದಲ್ಲಿ ದಕ್ಷಿಣೇಶ್ವರಕ್ಕೆ ಭೈರವಿ ಬ್ರಾಹ್ಮಣಿ ಎಂಬ ವೈಷ್ಣವ ಸಾಧಕಿ ಬಂದಳು. ಇವಳು ಭಕ್ತಿಶಾಸ್ತ್ರ, ತಂತ್ರಶಾಸ್ತ್ರಗಳಲ್ಲಿ ಚೆನ್ನಾಗಿ ನುರಿತವಳು. ಈಕೆ ಶ್ರೀರಾಮಕೃಷ್ಣರ ಶರೀರ ಲಕ್ಷಣಗಳನ್ನೂ ಮನಃಸ್ಥಿತಿಯನ್ನೂ ಚೆನ್ನಾಗಿ ಪರೀಕ್ಷಿಸಿ, “ಇದು ವೈಷ್ಣವ ಶಾಸ್ತ್ರದಲ್ಲಿ ಬಣ್ಣಿಸಲ್ಪಟ್ಟಿರುವ ‘ಮಹಾಭಾವ’ ಎಂಬ ಅದ್ಭುತ ಸ್ಥಿತಿ” ಎಂದು ಸಾರಿದಳು. ಮಹಾಭಾವ ಎನ್ನುವುದು ಭಕ್ತಿಭಾವದ ಅತ್ಯುನ್ನತ ಸ್ಥಿತಿ; ಈಶ್ವರಕೋಟಿಗಳಿಗೆ ಮಾತ್ರ ಲಭ್ಯವಾಗು ವಂತಹದು. ಸಾಮಾನ್ಯ ಜನರೆಲ್ಲ ಶ್ರೀರಾಮಕೃಷ್ಣರ ಭಾವವನ್ನು ಕಂಡು ಅದೊಂದು ಬಗೆಯ ಹುಚ್ಚು ಎಂದು ತಿಳಿದರೆ, ಭೈರವಿ ಬ್ರಾಹ್ಮಣಿ ಅದನ್ನು ‘ಮಹಾಭಾವ’ ಎಂದು ಕರೆಯುತ್ತಿದ್ದಾಳೆ! ಶ್ರೀರಾಮಕೃಷ್ಣರು ಆಧ್ಯಾತ್ಮಿಕ ಅನುಭವದ ಉತ್ತುಂಗ ಸ್ಥಿತಿಗೇರಿದವರು ಎನ್ನುವುದನ್ನು ಅವಳು ಕಂಡುಕೊಂಡಳು. ಮಾತ್ರವಲ್ಲ, ಶಾಸ್ತ್ರೋಕ್ತಿಗಳನ್ನು ಆಳವಾಗಿ ಅಧ್ಯಯಿಸಿ, ದೀರ್ಘಾಲೋಚನೆ ನಡೆಸಿ ಕಡೆಗೆ ಶ್ರೀರಾಮಕೃಷ್ಣರು ಸಾಕ್ಷಾತ್ ಭಗವಂತನ ಅವತಾರ ಎಂಬುದನ್ನು ಆಕೆ ಸ್ಪಷ್ಟವಾಗಿ ಗುರುತಿಸಿದಳು. ಅಷ್ಟೇ ಅಲ್ಲ, ಆ ಕಾಲದ ಅತಿ ಪ್ರಸಿದ್ಧ ಪಂಡಿತರನ್ನೆಲ್ಲ ಕರೆಸಿ ಸಭೆ ಸೇರಿಸಿ, ಅವರ ಮುಂದೆ ಆಧಾರಗಳನ್ನು ತೋರಿಸಿಕೊಟ್ಟು, ಶ್ರೀರಾಮಕೃಷ್ಣರು ಅವತಾರಪುರುಷರು ಎಂಬುದನ್ನು ಸಾಬೀತುಪಡಿಸಿದಳು. ಪಂಡಿತರೆಲ್ಲರೂ ಅದನ್ನು ಮರುಮಾತಿಲ್ಲದೆ ಸರ್ವಾನುಮತ ದಿಂದ ಒಪ್ಪಿಕೊಳ್ಳಬೇಕಾಯಿತು.

ಶ್ರೀರಾಮಕೃಷ್ಣರು ಭೈರವಿ ಬ್ರಾಹ್ಮಣಿಯನ್ನು ಗುರುವಾಗಿ ಸ್ವೀಕರಿಸಿ, ವೈಷ್ಣವಶಾಸ್ತ್ರಗಳಿಗೆ ಅನುಸಾರವಾದ ತಂತ್ರಸಾಧನೆಯನ್ನು ಕೈಗೊಂಡರು, ಮತ್ತು ಅತ್ಯಲ್ಪ ಕಾಲದಲ್ಲಿ ಅನೇಕ ಬಗೆಯ ಸಾಧನೆಗಳಲ್ಲಿ ಸಿದ್ಧರಾದರು.

ಇದಾದ ಮೇಲೆ ದಕ್ಷಿಣೇಶ್ವರಕ್ಕೆ ವೈಷ್ಣವ ಸಾಧುವೊಬ್ಬನು ಬಂದ. ಇವನು ರಾಮಭಕ್ತ. ಸುದೀರ್ಘ ಸಾಧನೆಯಿಂದ ತನ್ನ ಇಷ್ಟದೇವತೆಯಾದ ರಾಮಲಾಲನ ಸಾಕ್ಷಾತ್ಕಾರ ಮಾಡಿಕೊಂಡ ವನು. ಇವನ ಹತ್ತಿರ ಬಾಲರಾಮನ ಒಂದು ಪುಟ್ಟ ಪ್ರತಿಮೆ ಇತ್ತು. ಅದರಲ್ಲಿ ಅವನು ಜೀವಂತ ರಾಮನನ್ನು ಕಣ್ಣಾರೆ ಕಾಣುತ್ತಿದ್ದ. ಆ ಹಸುಳೆ ರಾಮನನ್ನು ತನ್ನ ಕಂದನೆಂಬಂತೆ ಲಾಲಿಸಿ ಪಾಲಿಸುತ್ತಿದ್ದ. ಯಾವಾಗ ಶ್ರೀರಾಮಕೃಷ್ಣರು ಆ ಪ್ರತಿಮೆಯನ್ನು ಕಂಡರೋ, ಒಡನೆಯೇ ಅದರಲ್ಲಿ ಬಾಲರಾಮನನ್ನು ಗುರುತಿಸಿ ಬಿಟ್ಟರು! ಅವರಿಗೂ ರಾಮಲಾಲನಿಗೂ ಬಹುಬೇಗ ಗಾಢ ಸ್ನೇಹಬಾಂಧವ್ಯ ಬೆಳೆಯಿತು. ಅವರೀಗ ಇತರರನ್ನು ಕಾಣುವಷ್ಟೇ ಸ್ಪಷ್ಟವಾಗಿ ರಾಮಲಾಲನನ್ನು ಕಾಣತೊಡಗಿದರು. ರಾಮಲಾಲ ನರ್ತನ ಮಾಡುತ್ತಾನೆ; ಅವರ ಬೆನ್ನೇರಿ ಕುಣಿದು ಕುಪ್ಪಳಿಸು ತ್ತಾನೆ; ತೋಳಿನಲ್ಲೆತ್ತಿಕೊಳ್ಳುವಂತೆ ಕಾಡುತ್ತಾನೆ! ಆ ದಿವ್ಯಬಾಲಕನು ಶ್ರೀರಾಮಕೃಷ್ಣರ ಜೊತೆಗೆ ಎಷ್ಟು ಆತ್ಮೀಯವಾಗಿ ಹೊಂದಿಕೊಂಡುಬಿಟ್ಟನೆಂದರೆ ತನ್ನನ್ನು ಕರೆತಂದಿದ್ದ ವೈಷ್ಣವ ಸಾಧುವಿ ನೊಂದಿಗೆ ತಾನು ಹೋಗಲೊಲ್ಲೆ ಎಂದುಬಿಟ್ಟ! ರಾಮಲಾಲ ಹೇಗಿದ್ದರೂ ಶ್ರೀರಾಮಕೃಷ್ಣರ ಜೊತೆಗೆ ಆನಂದದಿಂದ ಇರುವುದನ್ನು ಕಂಡು, ಮತ್ತು ತಾನು ತನ್ನ ಸಾಧನೆಯ ಚರಮಾವಸ್ಥೆ ಯನ್ನು ಮುಟ್ಟಿದ್ದೇನೆಂದರಿತು, ಆ ಸಾಧು ಅವನನ್ನು–ರಾಮಲಾಲ ವಿಗ್ರಹವನ್ನು–ಅಲ್ಲೇ ಬಿಟ್ಟು ಹೊರಟುಹೋದ.

ಈಗ ಶ್ರೀರಾಮಕೃಷ್ಣರು ವೈಷ್ಣವ ಶಾಸ್ತ್ರಕ್ಕೆ ಅನುಸಾರವಾದ ಇನ್ನೊಂದು ವಿಶಿಷ್ಟ ಸಾಧನೆಯನ್ನು ಕೈಗೊಳ್ಳುತ್ತಾರೆ. ಇದೇ ಮಧುರಭಾವ ಸಾಧನೆ. ಪ್ರಿಯೆಯು ತನ್ನ ಪ್ರಿಯತಮನ ಮೇಲಿಡುವ ಭಾವವೇ ಮಧುರಭಾವ. ಭಗವಂತನನ್ನೇ ಈ ಭಾವದಿಂದ ಕಂಡು ಆರಾಧಿಸಿದಾಗ ಅದು ಸಾಕ್ಷಾತ್ಕಾರಕ್ಕೆ ಸಾಧಕವಾಗುತ್ತದೆ. ವೈಷ್ಣವ ಸಾಧನೆಗಳಲ್ಲಿ ಭಕ್ತಿಗೆ ಪ್ರಾಧಾನ್ಯ. ಅದರಲ್ಲೂ ಈ ಮಧುರಭಾವ ಸಾಧನೆಯಲ್ಲಿ ಭಕ್ತ-ಭಗವಂತರಲ್ಲಿ ಉಂಟಾಗುವ ತಾದಾತ್ಮ್ಯಭಾವವು ಅತ್ಯುನ್ನತ ಮಟ್ಟದ್ದು. ಭಗವಂತನೊಂದಿಗಿನ ಮಧುರತರ ನಿಕಟ ಸಂಬಂಧವೇ ಈ ಸಾಧನೆಯ ವೈಶಿಷ್ಟ್ಯ. ಇದಕ್ಕೊಂದು ಶ್ರೇಷ್ಠ ಉದಾಹರಣೆ ರಾಧೆಯ ಜೀವನ. ಈ ಭಾವದ ಅನುಸಾರ ಸಾಧನೆ ಮಾಡುವ ಭಕ್ತ, ತನ್ನ ವೈಯಕ್ತಿಕ ಸುಖ ಸಂತೋಷ ಅನುಕೂಲಗಳನ್ನು ಕಿಂಚಿತ್ತೂ ಲೆಕ್ಕಿಸದೆ, ತನ್ನ ಪ್ರೀತಿ ಪಾತ್ರನಾದ ಭಗವಂತನನ್ನು ಸಂತುಷ್ಟಿಪಡಿಸುವುದರ ಕಡೆಗಷ್ಟೇ ಗಮನ ಕೊಡುತ್ತಾನೆ. ಮಧುರ ಭಾವ ಸಾಧನೆಯ ಒಂದು ಪ್ರಮುಖ ಅಂಶವೆಂದರೆ, ಸಾಧನೆಯನ್ನು ಕೈಗೊಂಡ ವ್ಯಕ್ತಿ ಸ್ತ್ರೀಯಾಗಲಿ ಪುರುಷನಾಗಲಿ, ತನ್ನ ಲೈಂಗಿಕ ಪ್ರವೃತ್ತಿಯನ್ನು ಸಂಪೂರ್ಣವಾಗಿ ಅತಿಕ್ರಮಿಸಿದ ಮೇಲಷ್ಟೇ ಮುಂದುವರಿಯಬಹುದಾದ ಸಾಧನಾವಿಧಾನ ಅದು. ಇಂತಹ ಸಾಧನೆಯನ್ನು ಶ್ರೀರಾಮಕೃಷ್ಣರು ತಮ್ಮ ಸ್ವಭಾವಸಹಜ ಉತ್ಸಾಹದಿಂದ ಪ್ರಾರಂಭಿಸಿದರು. ತಮ್ಮನ್ನು ಒಬ್ಬಳು ಸ್ತ್ರೀಯೆಂದೇ ಭಾವಿಸಿಕೊಂಡರು; ಸ್ತ್ರೀಯರಂತೆಯೇ ಉಡಿಗೆತೊಡಿಗೆ ಧರಿಸುತ್ತಿದ್ದರು; ಸ್ತ್ರೀಯರಂತೆಯೇ ಮಾತನಾಡುತ್ತಿದ್ದರು. ತಮ್ಮ ಭಕ್ತನಾದ ಮಥುರಾನಾಥನ ಮನೆಯ ಮಹಿಳೆಯರ ಸಂಗಡವೇ ಇರುತ್ತಿದ್ದರು. ಅವರ ಭಾವನೆ ಮಾತುಕತೆ ನಡಿಗೆ ಎಲ್ಲವೂ ಹೇಗೆ ಸ್ತ್ರೀಯರಂತೆಯೇ ಆಗಿಬಿಟ್ಟಿ ತ್ತೆಂದರೆ, ಮಥುರಾನಾಥನ ಮನೆಯ ಮಹಿಳೆಯರಿಗೆ ಇವರೊಬ್ಬ ಪುರುಷ ಎನ್ನುವ ಭಾವನೆಯೇ ಬರುತ್ತಿರಲಿಲ್ಲ. ಹೀಗೆ ತಮ್ಮನ್ನು ಭಗವಂತನ ಪ್ರಿಯೆಯೆಂಬಂತೆ ಭಾವಿಸಿಕೊಂಡು ಮಾಡಿದ ಈ ಮಧುರಭಾವ ಸಾಧನೆಯ ಮೂಲಕ ಶ್ರೀರಾಮಕೃಷ್ಣರು ಅದೇ ಭಗವಂತನನ್ನು ಸಾಕ್ಷಾತ್ಕರಿಸಿ ಕೊಂಡರು. ಪುರುಷನೊಬ್ಬ ಸ್ತ್ರೀಭಾವದಿಂದಲೂ ಭಗವಂತನ ಸಾಕ್ಷಾತ್ಕಾರ ಮಾಡಿಕೊಳ್ಳಬಹುದು ಎಂಬ ಸತ್ಯವನ್ನು ಈ ಮೂಲಕ ಅವರು ಕಂಡುಕೊಂಡರು.

ಈ ಸಮಯಕ್ಕೆ ದಕ್ಷಿಣೇಶ್ವರಕ್ಕೆ ಅದ್ವೈತವೇದಾಂತದ ಸಾಧಕಶ್ರೇಷ್ಠನೊಬ್ಬ ಆಗಮಿಸಿದ. ಇವನೇ ನಗ್ನಸಂನ್ಯಾಸಿ ತೋತಾಪುರಿ. ಪ್ರಥಮ ನೋಟದಲ್ಲೇ ಈತ ಶ್ರೀರಾಮಕೃಷ್ಣರ ವರ್ಚಸ್ಸನ್ನು ಕಂಡು, ಅವರಿಗೆ ಅದ್ವೈತ ತತ್ತ್ವದ ರಹಸ್ಯವನ್ನು ಬೋಧಿಸಲು ಮುಂದಾದ. ಶ್ರೀರಾಮಕೃಷ್ಣರು ಒಪ್ಪಿಕೊಂಡರು. ಆತನ ಆದೇಶದಂತೆ ಸಂನ್ಯಾಸವನ್ನು ಸ್ವೀಕರಿಸಿದರು. ಅವನ ಮಾರ್ಗದರ್ಶನ ದಲ್ಲಿ ಮುಂದುವರಿದು, ಅತ್ಯಲ್ಪ ಅವಧಿಯಲ್ಲಿ ಅದ್ವೈತ ಸಾಕ್ಷಾತ್ಕಾರವನ್ನು ಪಡೆದುಕೊಂಡರು. ಅದ್ವೈತ ಸಾಕ್ಷಾತ್ಕಾರವೆಂದರೆ ನಿರ್ವಿಕಲ್ಪ ಸಮಾಧಿಸ್ಥಿತಿಯಲ್ಲಿ, ಆತ್ಯಂತಿಕ ಸತ್ಯಸ್ವರೂಪನಾದ ಪರಬ್ರಹ್ಮನೊಂದಿಗೆ ಸಾಧಕನ ಆತ್ಮವು ಐಕ್ಯಹೊಂದುವಂತಹ ತುರೀಯ ಸ್ಥಿತಿ. ಈ ಅನುಭವ ವನ್ನು ಪಡೆದುಕೊಳ್ಳಲು ತೋತಾಪುರಿಗೆ ನಲವತ್ತು ವರ್ಷಗಳ ಸಾಧನೆ ಬೇಕಾಗಿತ್ತು. ಆದರೆ ಇದೇ ಅನುಭವವನ್ನು ಹೊಂದಲು ಶ್ರೀರಾಮಕೃಷ್ಣರಿಗೆ ಕೇವಲ ಒಂದೆ ಒಂದು ದಿನ ಸಾಕಾಯಿತು! ಈ ಅನುಭವದ ಕುರಿತಾಗಿ ಮುಂದೆ ಅವರು ಹೇಳುತ್ತಾರೆ: “ಈ ನಿರ್ವಿಕಲ್ಪ ಸಮಾಧಿ ಸ್ಥಿತಿಯಲ್ಲಿ ನಾನು, (ಗುರು ತೋತಾಪುರಿ ಇಲ್ಲಿಂದ ನಿರ್ಗಮಿಸಿದ ಮೇಲೆ) ಆರು ತಿಂಗಳ ಕಾಲ ಇದ್ದೆ. ಈ ಸ್ಥಿತಿಯನ್ನು ಮುಟ್ಟಿದ ಸಾಮಾನ್ಯ ಜೀವರಾದರೆ ಅಲ್ಲಿಂದ ಹಿಂದಿರುಗಿ ಬರಲಾರರು. ಇಪ್ಪ ತ್ತೊಂದು ದಿನಗಳ ಬಳಿಕ ಅವರ ಶರೀರವು ಒಣಗಿದ ಎಲೆಯಂತೆ ಬಿದ್ದುಹೋಗುತ್ತದೆ, ಮತ್ತು ಅವರ ಆತ್ಮ ಪರಬ್ರಹ್ಮವಸ್ತುವಿನಲ್ಲಿ ಸೇರಿಹೋಗಿಬಿಡುತ್ತದೆ.”

ಆ ಪರಿವ್ರಾಜಕ ಸಂನ್ಯಾಸಿಯಾದ ತೋತಾಪುರಿ ಯಾವ ಸ್ಥಳದಲ್ಲೂ ಮೂರು ದಿನಗಳಿಗಿಂತ ಹೆಚ್ಚು ಇದ್ದವನೇ ಅಲ್ಲ. ಅಂಥವನು ಶ್ರೀರಾಮಕೃಷ್ಣರ ಪರಮಾದ್ಭುತ ವ್ಯಕ್ತಿತ್ವದಿಂದ ಆಕರ್ಷಿತ ನಾಗಿ ಹನ್ನೊಂದು ತಿಂಗಳ ಕಾಲ ಅವರೊಂದಿಗೆ ಇದ್ದುಬಿಟ್ಟ. ಈ ಅವಧಿಯಲ್ಲಿ ಅವನು ತನ್ನ ಶಿಷ್ಯನಿಂದಲೇ ಒಂದು ಹೊಸ ಪಾಠವನ್ನು ಕಲಿತುಕೊಂಡ. ಆತ ಕಟ್ಟಾ ಅದ್ವೈತಿ. ದ್ವೈತತತ್ತ್ವವನ್ನು ಒಪ್ಪದೆ, ಅದನ್ನು ಮೌಢ್ಯವೆಂದು ಹೀಗಳೆಯುತ್ತಿದ್ದವನು. ಇಂಥವನಿಗೆ ಶ್ರೀರಾಮಕೃಷ್ಣರು ಪರಮಾತ್ಮನ ಸಗುಣ-ಸಾಕಾರ ರೂಪವನ್ನು ಅತ್ಯಂತ ವಿಶಿಷ್ಟವಾದ ರೀತಿಯಲ್ಲಿ ತೋರಿಸಿ ಕೊಟ್ಟರು.

ಈಗ ಶ್ರೀರಾಮಕೃಷ್ಣರು ಇತರ ಧರ್ಮಗಳ ತತ್ತ್ವಸಾಕ್ಷಾತ್ಕಾರ ಮಾಡಿಕೊಳ್ಳಲು ಕಂಕಣಬದ್ಧ ರಾದರು. ಆಗ ಕ್ರೈಸ್ತ, ಇಸ್ಲಾಂ ಮೊದಲಾದ ಧರ್ಮಗಳ ಅನುಯಾಯಿಗಳು ಅವರ ಮಾರ್ಗ ದರ್ಶಕರಾಗಿ ತಾವಾಗಿಯೇ ಒದಗಿಬಂದರು. ಇವರ ನೆರವಿನಿಂದ ಇತರ ಧರ್ಮಗಳನ್ನು ಅತ್ಯಂತ ಪ್ರಾಮಾಣಿಕ ಭಾವದಿಂದ ಅಭ್ಯಾಸ ಮಾಡಿದ ಶ್ರೀರಾಮಕೃಷ್ಣರು ಸ್ವತಃ ಕಂಡುಕೊಂಡರು –ಹಿಂದೂ ಧರ್ಮವು ನಮ್ಮನ್ನು ಯಾವ ಗುರಿಗೆ ಕೊಂಡೊಯ್ಯುತ್ತದೆಯೋ, ಅದೇ ಗುರಿಗೆ ಇತರ ಧರ್ಮಗಳೂ ಕೊಂಡೊಯ್ಯುತ್ತವೆ ಎಂದು. ಬೇರೆಬೇರೆ ಮತಧರ್ಮಗಳ ಸಾಧಕರೊಂದಿಗೆ ತಮ್ಮ ಅನುಭವವನ್ನು ಹೋಲಿಸಿ ನೋಡಿಕೊಂಡಾಗ ಅವುಗಳೆಲ್ಲ ತಾಳೆಯಾಗುವುದು ಅವರಿಗೆ ಸ್ಪಷ್ಟವಾಗಿ ಕಂಡು ಬಂದಿತು. ಆಗ ಅವರು ಎಲ್ಲ ಧರ್ಮದವರೂ ಒಬ್ಬ ಭಗವಂತನನ್ನೇ ಬೇರೆ ಬೇರೆ ರೂಪಗಳಿಂದ ದರ್ಶನ ಮಾಡುತ್ತಾರೆ ಎಂಬ ನಿಶ್ಚಿತ ತೀರ್ಮಾನಕ್ಕೆ ಬಂದರು. 

ಈ ಹೊತ್ತಿಗೆ ಅತ್ತ ಜಯರಾಂಬಾಟಿಯಲ್ಲಿ, ಶ್ರೀರಾಮಕೃಷ್ಣರಿಗೆ ಸಂಪೂರ್ಣ ಹುಚ್ಚು ಹಿಡಿದು ಬಿಟ್ಟಿದೆ ಎಂದು ಪ್ರಚಾರವಾಗಿಬಿಟ್ಟಿತ್ತು. ಪತ್ನಿ ಶಾರದಾದೇವಿಯವರು (ಆಗ ಅವರು ಹದಿನೆಂಟು- ಹತ್ತೊಂಬತ್ತು ವರ್ಷದ ಯುವತಿ) ಪತಿಯನ್ನು ಕಣ್ಣಾರೆ ಕಾಣುವ ಉದ್ದೇಶದಿಂದ ದಕ್ಷಿಣೇಶ್ವರಕ್ಕೆ ಧಾವಿಸಿದರು. ಆದರೆ ಇಲ್ಲಿ ಅವರು ಕಂಡದ್ದೇನು? ಹುಚ್ಚನನ್ನಲ್ಲ, ಭಗವತ್ಸ್ವರೂಪಿಯಾಗಿ ಕಂಗೊಳಿಸುವ ಪರಮಪ್ರೀತಿಯ ಪತಿಯನ್ನು! ಶ್ರೀರಾಮಕೃಷ್ಣರು ಪತ್ನಿಯನ್ನು ಅತ್ಯಾದರದಿಂದ ಬರಮಾಡಿಕೊಂಡರು. ಆಕೆಯ ವಸತಿಗೂ ಊಟೋಪಚಾರಕ್ಕೂ ವ್ಯವಸ್ಥೆ ಮಾಡಿದರು. ಹೀಗೆ ತಮ್ಮನ್ನು ಅತ್ಯಂತ ಪ್ರೀತಿಯಿಂದ ನೋಡಿಕೊಂಡ ಪತಿದೇವರನ್ನು ಕಂಡಾಗ ಶಾರದಾದೇವಿ ಯವರ ಮನಸ್ಸಿನ ಆತಂಕ ಮಾಯವಾಗಿ ಶಾಂತಿ-ಸಂತೋಷ ತುಂಬಿತು.

ಈಗ ಶ್ರೀರಾಮಕೃಷ್ಣರು ಒಬ್ಬ ಆದರ್ಶ ಪತಿಯಂತೆ ತಮ್ಮ ಸರಳ ಸ್ವಭಾವದ ಪತ್ನಿಗೆ ಹಲವಾರು ವಿಷಯಗಳಲ್ಲಿ ಶಿಕ್ಷಣ ನೀಡಲಾರಂಭಿಸಿದರು. ಮನೆವಾರ್ತೆಯ ಸಣ್ಣಪುಟ್ಟ ವಿಚಾರ ಗಳಿಂದ ಹಿಡಿದು, ಬ್ರಹ್ಮಜ್ಞಾನದವರೆಗಿನ ಸಮಸ್ತ ವಿಷಯಗಳನ್ನೂ ಪ್ರೀತಿಯಿಂದ ತಿಳಿಸಿಕೊಟ್ಟರು. ಒಂದು ದಿನ ಮಡದಿಯನ್ನು ಕುಳ್ಳಿರಿಸಿಕೊಂಡು ಹೇಳುತ್ತಾರೆ: “ನೋಡು, ಚಂದಮಾಮ ಲೋಕದ ಎಲ್ಲ ಮಕ್ಕಳಿಗೂ ಮಾಮ; ಹಾಗೆಯೇ ಭಗವಂತ ಕೂಡ ಎಲ್ಲಿರಿಗೂ ಸೇರಿದವನು. ಯಾರು ಆತನನ್ನು ಕೂಗಿ ಕರೆಯುತ್ತಾರೋ ಅವರಿಗೆ ಅವನು ದರ್ಶನ ಕೊಡುತ್ತಾನೆ. ನೀನು ಅವನನ್ನು ಕರೆ, ನಿನಗೂ ಅವನು ದರ್ಶನ ಕೊಡುತ್ತಾನೆ.” ಇನ್ನೊಮ್ಮೆ ತಮ್ಮ ಪತ್ನಿಯ ಮನೋಭಾವವನ್ನು ತಿಳಿದುಕೊಳ್ಳಲು ಕೇಳುತ್ತಾರೆ: “ನೀನು ಇಲ್ಲಿಗೆ ಬಂದಿದ್ದೀಯಲ್ಲ, ನನ್ನನ್ನು ಪ್ರಪಂಚಕ್ಕೆ ಎಳೆಯ ಬೇಕು ಅಂತಲೆ?” ಪರಮಪವಿತ್ರ ಹೃದಯದ ಶಾರದಾದೇವಿಯವರು ಪತಿಯ ಇಂಗಿತವನ್ನು ಅರ್ಥಮಾಡಿಕೊಂಡು ತಕ್ಷಣ ಹೇಳುತ್ತಾರೆ: “ಇಲ್ಲ ಇಲ್ಲ; ನಾನೇಕೆ ನಿಮ್ಮನ್ನು ಪ್ರಪಂಚಕ್ಕೆ ಎಳೆಯಲಿ? ನಾನು ಬಂದಿರುವುದು ನಿಮ್ಮ ಆಧ್ಯಾತ್ಮಿಕ ಜೀವನದಲ್ಲಿ ನೆರವಾಗುವುದಕ್ಕೋಸ್ಕರ.” ಇನ್ನೊಂದು ದಿನ ಪತಿಯ ಕಾಲೊತ್ತುತ್ತ ಕುಳಿತಿದ್ದ ಆತ್ಮೀಯ ಘಳಿಗೆಯಲ್ಲಿ ಶಾರದಾದೇವಿ ಯವರು ಪತಿಯನ್ನು ಕೇಳುತ್ತಾರೆ, “ನೀವು ನನ್ನನ್ನು ಯಾವ ದೃಷ್ಟಿಯಿಂದ ಕಾಣುತ್ತಿದ್ದೀರಿ?” ಒಡನೆಯೇ ಶ್ರೀರಾಮಕೃಷ್ಣರೆನ್ನುತ್ತಾರೆ: “ಮಂದಿರದಲ್ಲಿ ಪೂಜಿಸಲ್ಪಡುತ್ತಿರುವ ಜಗನ್ಮಾತೆ, ನಹಬತ್ ಖಾನೆಯಲ್ಲಿ ವಾಸಿಸುತ್ತಿರುವ ನನ್ನ ಜನ್ಮದಾತೆ–ಅವಳೇ ಈಗ ಇಲ್ಲಿ ನನ್ನ ಕಾಲುಗಳನ್ನು ಒತ್ತುತ್ತ ಕುಳಿತಿದ್ದಾಳೆ ಎಂಬುದೇ ನನ್ನ ಭಾವನೆ. ಸತ್ಯವಾಗಿಯೂ ನಾನು ನಿನ್ನಲ್ಲಿ ಯಾವಾಗಲೂ ಆನಂದಮಯಿಯಾದ ಜಗನ್ಮಾತೆಯೇ ಮೈದಳೆದು ಬಂದಿರುವಂತೆ ಕಾಣುತ್ತಿದ್ದೇನೆ.” ಮತ್ತೊಂದು ದಿನ ತಮ್ಮ ಮಗ್ಗುಲಲ್ಲಿ ಮಲಗಿರುವ ಹರೆಯದ ಪತ್ನಿಯನ್ನು ಕಂಡು ಶ್ರೀರಾಮ ಕೃಷ್ಣರು ತಮ್ಮಲ್ಲಿಯೇ ವಿವೇಚಿಸುತ್ತ ತಮ್ಮ ಮನಸ್ಸಿಗೆ ಹೇಳುತ್ತಾರೆ: “ಓ ಮನಸ್ಸೇ, ಹೆಣ್ಣಿನ ದೇಹ ಎಂದೆನ್ನುವುದು ಇದಕ್ಕೆ. ಜಗದ ಜನರೆಲ್ಲ ಪರಮಭೋಗದ ವಸ್ತುವೆಂದು ಭಾವಿಸಿ ಚಪಲಚಿತ್ತರಾಗಿ ಧಾವಿಸುವುದು ಈ ದೇಹದ ಹಿಂದೆಯೇ. ಆದರೆ ಇದನ್ನು ಅಂಗೀಕರಿಸಿದರೆ ದೇಹ ಬುದ್ಧಿಯ ಬಲೆಯಲ್ಲಿ ಸದಾ ನಿಬದ್ಧರಾಗಿರಬೇಕಾಗುತ್ತದೆ. ಸತ್ಯಜ್ಞಾನಾನಂದಸ್ವರೂಪನಾದ ಪರ ಮಾತ್ಮನನ್ನು ಎಂದಿಗೂ ಪಡೆಯುವುದಕ್ಕಾಗುವುದಿಲ್ಲ. ಈಗ ನೀ ಹೇಳು, ಓ ಮನಸ್ಸೇ, ವಂಚಕನಾಗಬೇಡ. ಒಳಗೊಂದೆಣಿಸಿ ಹೊರಗಿನ್ನೊಂದೆನಬೇಡ. ಸತ್ಯವಾಗಿ ಹೇಳು, ನಿನಗೆ ಈ ಹೆಣ್ಣಿನ ದೇಹ ಬೇಕೋ, ಅಥವಾ ದೇವರು ಬೇಕೋ? ನಿನಗೆ ಈ ಸ್ತ್ರೀಶರೀರವೇ ಬೇಕಿದ್ದರೆ, ಇಗೋ ನೋಡು, ಅದು ನಿನ್ನೆದುರಿಗೇ ಇದೆ. ನೀನದನ್ನು ಅನುಭವಿಸಲು ಸ್ವತಂತ್ರ.” ಹೀಗೆ ವಿವೇಚಿಸುತ್ತ ತಮ್ಮ ಮಡದಿಯ ಮೈಯನ್ನು ಮುಟ್ಟಲು ಕೈಚಾಚಿದರು; ಅಷ್ಟರಲ್ಲಿ ಅವರ ಮನಸ್ಸು-ದೇಹಗಳು ಎಷ್ಟೊಂದು ರಭಸದಿಂದ ಹಿಮ್ಮೆಟ್ಟಿದುವೆಂದರೆ ಅವರು ಇದ್ದಕ್ಕಿದ್ದಂತೆ ಗಾಢಸಮಾಧಿಯಲ್ಲಿ ಲೀನವಾಗಿಬಿಟ್ಟರು. ಅವರ ಮನಸ್ಸು ಇಡೀ ರಾತ್ರಿಯೆಲ್ಲ ವ್ಯಾವಹಾರಿಕ ಜಗತ್ತಿಗೆ ಇಳಿಯಲೇ ಇಲ್ಲ. ಮರುದಿನ ಬೆಳಗ್ಗೆ ಅವರ ಮನಸ್ಸನ್ನು ಸಮಾಧಿಸ್ಥಿತಿಯಿಂದ ಇಳಿಸಿ ಬಾಹ್ಯಪ್ರಪಂಚಕ್ಕೆ ತರಬೇಕಾದರೆ ಬಹಳ ಹೊತ್ತು ಅವರ ಕಿವಿಯಲ್ಲಿ ಭಗವನ್ನಾಮವನ್ನು ಉಚ್ಚರಿಸಬೇಕಾಯಿತು.

ಮುಂದೊಮ್ಮೆ ಶ್ರೀರಾಮಕೃಷ್ಣರು ತಮ್ಮ ಪತ್ನಿಯ ಬಗ್ಗೆ ಹೇಳುತ್ತಾರೆ: “ಅವಳು ಅಷ್ಟು ಪವಿತ್ರಳಾಗಿಲ್ಲದೆ ಹೋಗಿದ್ದರೆ ನಾನು ಅವಳ ಆಕರ್ಷಣೆಗೆ ಒಳಗಾಗಿ ನನ್ನ ಆತ್ಮ ಸಂಯಮವನ್ನು ಕಳೆದುಕೊಂಡುಬಿಡುತ್ತಿದ್ದೆನೋ ಏನೋ, ನನ್ನ ದೇಹಬುದ್ಧಿ ಕೆರಳುತ್ತಿತ್ತೋ ಏನೋ ಯಾರಿಗೆ ಗೊತ್ತು! ನನಗೆ ಮದುವೆಯಾದಾಗ ನಾನು ಜಗನ್ಮಾತೆಯನ್ನು ಅತ್ತುಕರೆದು ಪ್ರಾರ್ಥಿಸಿಕೊಂಡೆ: ‘ನನ್ನ ಹೆಂಡತಿಯ ಮನಸ್ಸಿನಲ್ಲಿ ಕಿಂಚಿತ್ತೂ ಕಾಮಭಾವನೆ ಇಲ್ಲದಹಾಗೆ ಮಾಡಿಬಿಡು’ ಅಂತ. ಬಳಿಕ ಅವಳೊಂದಿಗೆ ಜೀವನ ನಡೆಸಿದಾಗ ನಾನು ಅರಿತೆ–ಜಗನ್ಮಾತೆ ನನ್ನ ಪ್ರಾರ್ಥನೆಯನ್ನು ಈಡೇರಿಸಿದ್ದಾಳೆ ಅಂತ.”

ಶಾರದಾದೇವಿಯವರು ದಕ್ಷಿಣೇಶ್ವರಕ್ಕೆ ಬಂದ ಕೆಲವು ತಿಂಗಳ ಮೇಲೆ, ಶ್ರೀರಾಮಕೃಷ್ಣರಿಗೆ ತಮ್ಮ ಪತ್ನಿಯನ್ನೇ ಕುಳ್ಳಿರಿಸಿ, ಅವಳಲ್ಲಿ ಜಗನ್ಮಾತೆಯನ್ನು ಆವಾಹನೆ ಮಾಡಿ, ‘ಷೋಡಶೀ ಪೂಜೆ’ ಮಾಡಬೇಕೆಂಬ ಪ್ರೇರೇಪಣೆಯುಂಟಾಯಿತು. ಈ ಮೂಲಕ ತಮ್ಮ ಸಮಸ್ತ ಸಾಧನೆ ಹಾಗೂ ಸಿದ್ಧಿಗಳೊಂದಿಗೆ ತಮ್ಮನ್ನೇ ಸಮರ್ಪಿಸಿಕೊಂಡು ಬಿಡುವುದೆಂದು ನಿಶ್ಚಯಿಸಿದರು. ಅಂತೆಯೇ ಫಲಹಾರಿಣೀ ಕಾಳೀಪೂಜೆಯ ದಿನ ವಿಧಿಯುಕ್ತವಾಗಿ ಆ ಪೂಜೆಯನ್ನು ನೆರವೇರಿಸಿದರು. ಪೂಜೆಯ ವೇಳೆಗಾಗಲೇ ಶಾರದಾದೇವಿಯವರು ಸಮಾಧಿಸ್ಥರಾಗಿಬಿಟ್ಟಿದ್ದರು. ಪೂಜೆ ಮುಗಿ ಯುವ ಹೊತ್ತಿಗೆ ಶ್ರೀರಾಮಕೃಷ್ಣರು ಸಮಾಧಿಸ್ಥಿತಿಗೇರಿದರು. ಹೀಗೆ ಪೂಜ್ಯ-ಪೂಜಕರಿಬ್ಬರೂ ಇಂದ್ರಿಯರಾಜ್ಯವನ್ನು ಅತಿಕ್ರಮಿಸಿ ಆತ್ಮಭಾವದಲ್ಲಿ ಲೀನರಾದರು; ಸತ್-ಚಿತ್-ಆನಂದ ಸ್ವರೂಪದಲ್ಲಿ ತಾದಾತ್ಮ್ಯ ಹೊಂದಿದರು. ಅರ್ಧರಾತ್ರಿ ಕಳೆದಮೇಲಷ್ಟೇ ಶ್ರೀರಾಮಕೃಷ್ಣರಿಗೆ ಸ್ವಲ್ಪ ಬಾಹ್ಯಪ್ರಜ್ಞೆ ಬಂದಿತು. ಆಗ ಅವರು ತಮ್ಮ ಮುಂದೆ ವಿರಾಜಿಸುತ್ತಿದ್ದ ಜಗನ್ಮಾತೆಗೆ ಅನನ್ಯ ಶರಣಾಗತಿಭಾವದಿಂದ ತಮ್ಮನ್ನೂ ತಮ್ಮ ಸಕಲ ಸಾಧನೆಗಳ ಫಲವನ್ನೂ ಸಮರ್ಪಣೆ ಮಾಡಿ ಕೊಂಡರು. ಈಗ ಶ್ರೀರಾಮಕೃಷ್ಣರಿಗೆ ವಿಶ್ವದ ಪ್ರತಿಯೊಂದು ವಸ್ತುವೂ ಜಗನ್ಮಾತೆಯ ಪ್ರತೀಕ ವಾಯಿತು.

ಯೋಗಶಾಸ್ತ್ರಗಳು ಉದ್ಘೋಷಿಸುವ ಅತ್ಯುನ್ನತ ಸ್ಥಿತಿ, ನಿರ್ವಿಕಲ್ಪ ಸಮಾಧಿ. ಶ್ರೀರಾಮಕೃಷ್ಣರು ಸದಾ ಈ ಸ್ಥಿತಿಯಲ್ಲೇ ಇರುವಂತಾಗಿಬಿಟ್ಟಿತ್ತು. ಆದರೆ ಅವರು ಯಾವಾಗಲೂ ಈ ಸ್ಥಿತಿಯಲ್ಲೇ ಇರುವಂತಾದರೆ, ತಮ್ಮ ಅವತರಣದ ಉದ್ದೇಶವಾದ ಲೋಕಕಲ್ಯಾಣ ಕಾರ್ಯವನ್ನು ನೆರವೇರಿ ಸಲು ಸಾಧ್ಯವಾಗುವುದಿಲ್ಲ. ಆದ್ದರಿಂದ ಭಗವದಿಚ್ಛೆಯಂತೆ ಅವರ ಮನಸ್ಸು ಸದಾ ‘ಭಾವಮುಖ’ ದಲ್ಲಿರುವಂತಾಯಿತು. ಭಾವಮುಖವೆಂದರೆ ಅದೊಂದು ಅದ್ಭುತ ಮಾನಸಿಕ ಸ್ಥಿತಿ. ಈ ಸ್ಥಿತಿ ಯಲ್ಲಿ ಮನಸ್ಸು ಸಗುಣ-ನಿರ್ಗುಣಗಳ ನಡುವೆ, ಜಗತ್ತು-ಪರಬ್ರಹ್ಮಗಳ ನಡುವೆ ನಿಂತಿರುತ್ತದೆ. ಈ ಸ್ಥಿತಿಯಲ್ಲಿ ಶ್ರೀರಾಮಕೃಷ್ಣರು ಜಗತ್ತನ್ನು ಜಗನ್ಮಾತೆಯ ದಿವ್ಯಲೀಲೆಯಾಗಿ ಕಾಣುತ್ತಿದ್ದರು. ಹೀಗೆ ಕಾಣುತ್ತ, ತಮ್ಮನ್ನು ಜಗನ್ಮಾತೆಯ ಶಿಶುವೆಂದು ಭಾವಿಸಿ, ಸದಾ ಅವಳ ಮಾರ್ಗದರ್ಶನ ದಲ್ಲೇ ಇರುವಂತಹ ಶಿಶುಭಾವವನ್ನು ಆಶ್ರಯಿಸಿಕೊಂಡಿದ್ದರು.

ಹೀಗೆ ಅವರು ತಮ್ಮ ಸಮಸ್ತ ಸಾಧನೆಗಳನ್ನು ಮುಗಿಸಿ, ಜಗನ್ಮಾತೆಯೊಂದಿಗೆ ಅತ್ಯಂತ ನಿಕಟ ಸಂಪರ್ಕ-ಸಂಬಂಧವನ್ನಿಟ್ಟುಕೊಂಡು ಜೀವಿಸಲಾರಂಭಿಸಿದರು. ಈ ಸಂದರ್ಭದಲ್ಲಿ ಅವರಿಗೆ ತಮ್ಮ ಅವತಾರದ ಸಂಬಂಧವಾಗಿ ಹಲವಾರು ದರ್ಶನಗಳಾದವು. ಮುಂದೆ ತಮ್ಮಿಂದ ಏನೇನು ಆಗಲಿಕ್ಕಿದೆ ಎನ್ನುವುದನ್ನೆಲ್ಲ ಅವರು ಸ್ಪಷ್ಟವಾಗಿ ಕಂಡರು. ಅಲ್ಲದೆ, ತಮಗೆ ಇತರ ಜೀವರಂತೆ ಮುಕ್ತಿ ಎನ್ನುವುದಿಲ್ಲ; ತಾವು ಮುಕ್ತಿಗೂ ಅತೀತರಾದವರು; ಬದಲಾಗಿ ತಾವು ಮತ್ತೆಮತ್ತೆ ಅವತಾರವೆತ್ತಿ ಬಂದು ಇತರ ಜೀವರ ಮುಕ್ತಿಗಾಗಿ ಮಾರ್ಗದರ್ಶನ ನೀಡುವವರು ಎಂಬುದನ್ನು ಕಂಡುಕೊಂಡರು. ಅಲ್ಲದೆ, ಜಗನ್ಮಾತೆ ತಮ್ಮ ಮೂಲಕ ಈ ಜಗತ್ತಿನಲ್ಲಿ ಒಂದು ಮಹಾಸಂಸ್ಥೆ ಯನ್ನು ಸ್ಥಾಪಿಸುವವಳಿದ್ದಾಳೆ; ಆ ಸಂಸ್ಥೆಯಲ್ಲಿ ಅಸಂಖ್ಯಾತ ಮುಮುಕ್ಷುಗಳು ಬಂದು ಸೇರಿ ಕೊಂಡು, ತಾವು ಮಾಡಿದ ಸಾಧನಾಕ್ರಮವನ್ನನುಸರಿಸಿ ಆಧ್ಯಾತ್ಮಿಕ ಸಾಧನೆ ಮಾಡುತ್ತಾರೆ, ವಿಶ್ವಾತ್ಮಭಾವವನ್ನು ಅಭ್ಯಾಸ ಮಾಡುತ್ತಾರೆ ಎಂಬುದನ್ನು ಮುಂಗಂಡರು. ಅಲ್ಲದೆ ಈ ಜಗತ್ತಿ ನಲ್ಲಿ, ಯಾರುಯಾರು ಭಗವಂತನನ್ನು ಹೃತ್ಪೂರ್ವಕವಾಗಿ ಕರೆದಿದ್ದಾರೋ, ಕರೆಯುತ್ತಿದ್ದಾರೋ ಅಂತಹವರೆಲ್ಲ ತಮ್ಮ ಬಳಿಗೆ ಬರಲಿದ್ದಾರೆ ಎಂಬುದನ್ನೂ ಜಗನ್ಮಾತೆ ಅವರಿಗೆ ತೋರಿಸಿ ಕೊಟ್ಟಳು.

ಈಗ ಅವರಲ್ಲಿ ಇನ್ನೊಂದು ಬಗೆಯ ವ್ಯಾಕುಲತೆ ಉಂಟಾಗಿಬಿಟ್ಟಿದೆ. ಹಿಂದೆ ಯಾವ ಭಗವತ್ಸಾಕ್ಷಾತ್ಕಾರಕ್ಕಾಗಿ ವ್ಯಾಕುಲಗೊಂಡು ತಹತಹಿಸಿದರೋ, ಅದೇ ಸಾಕ್ಷಾತ್ಕಾರದ ಅನುಭವ ಗಳನ್ನು ಪ್ರಾಮಾಣಿಕರಾದ ಆಧ್ಯಾತ್ಮಿಕ ಸಾಧಕರಿಗೆ ತಿಳಿಸಿಕೊಡಬೇಕು ಎಂಬ ತೀವ್ರ ಕಾತರ ಅವರಲ್ಲಿ ಆರಂಭವಾಗಿಬಿಟ್ಟಿದೆ! ಶುದ್ಧ ಹೃದಯದ ಸಾಧಕರಲ್ಲಿ ತಮ್ಮ ದಿವ್ಯಾನುಭವಗಳನ್ನು ಯಾವಾಗ ಹಂಚಿಕೊಂಡೇನು ಎಂದು ಚಡಪಡಿಸಲಾರಂಭಿಸಿದ್ದಾರೆ. ಈ ವ್ಯಾಕುಲತೆಯ ಕುರಿತಾಗಿ ಶ್ರೀರಾಮಕೃಷ್ಣರು ಮುಂದೆ ಹೇಳುತ್ತಾರೆ: “ಓಹ್! ಆಗ ನನ್ನ ಕಾತರತೆಗೆ ಕೊನೆಯೇ ಇರಲಿಲ್ಲ. ಹಗಲುಹೊತ್ತು ಹೇಗೋ ತಾಳಿಕೊಳ್ಳುತ್ತಿದ್ದೆ. ಆದರೆ ಪ್ರಾಪಂಚಿಕರ ಕಾಡುಹರಟೆಯನ್ನು ಕೇಳಿ ತುಂಬ ವೇದನೆಯಾಗುತ್ತಿತ್ತು. ನನ್ನ ಪ್ರೀತಿಯ ಶುದ್ಧಹೃದಯದ ತ್ಯಾಗಿಗಳು, ಆಧ್ಯಾತ್ಮಿಕ ಸಾಧಕರು ಎಂದಿಗೆ ನನ್ನ ಬಳಿ ಬಂದಾರು ಎನ್ನುವುದನ್ನೇ ಆಸೆಯ ಕಣ್ಣುಗಳಿಂದ ಇದಿರು ನೋಡುತ್ತಿದ್ದೆ. ನನ್ನ ಅಂತರಂಗದ ಅನುಭವಗಳನ್ನೆಲ್ಲ ಅವರ ಮುಂದೆ ತೋಡಿಕೊಂಡು ನನ್ನ ಹೃದಯವನ್ನು ಹಗುರಮಾಡಿಕೊಳ್ಳುವ ನಿರೀಕ್ಷೆಯಲ್ಲೇ ಇರುತ್ತಿದ್ದೆ. ಸುತ್ತಮುತ್ತ ಯಾವೊಂದು ಸಂಗತಿ ನಡೆ ದರೂ ನನಗೆ ತಕ್ಷಣ ಆ ನನ್ನ ಭಾವೀಶಿಷ್ಯರ ಕುರಿತಾಗಿಯೇ ಆಲೋಚಿಸುವಂತಾಗುತ್ತಿತ್ತು. ಅವರು ಬಂದಾಗ ಯಾರಿಗೆ ಯಾವುದನ್ನು ಹೇಳಬೇಕು, ಯಾರಿಗೆ ಏನನ್ನು ಕೊಡಬೇಕು ಎನ್ನುವುದನ್ನೆಲ್ಲ ಮನಸ್ಸಿನಲ್ಲೇ ಗುರುತುಹಾಕಿಕೊಂಡು, ಜೋಡಿಸಿಟ್ಟುಕೊಳ್ಳುತ್ತಿದ್ದೆ.

“ಹೀಗೆ ಹಗಲುಹೊತ್ತು ಹೇಗೋ ಕಳೆಯುತ್ತಿತ್ತು. ಆದರೆ ದಿನ ಕಳೆದು ಸೂರ್ಯಾಸ್ತಮಾನ ವಾದಾಗ ಮಾತ್ರ ನನ್ನಿಂದ ತಡೆದುಕೊಳ್ಳಲಾಗುತ್ತಿರಲಿಲ್ಲ. ‘ಇನ್ನೂ ಒಂದು ದಿನ ಕಳೆದು ಹೋಯಿತು, ಅವರಾರೂ ಇನ್ನೂ ಬರಲೇ ಇಲ್ಲವಲ್ಲ’ ಎಂದು ಕಳವಳಿಸುತ್ತಿದ್ದೆ. ದೇವಾಲಯ ದಲ್ಲಿ ಘಂಟೆ-ಶಂಖ-ಜಾಗಟೆಗಳ ನಾದದೊಂದಿಗೆ ಜಗನ್ಮಾತೆಗೆ ಆರತಿಯಾಗುತ್ತಿದ್ದರೆ ನಾನು ಏಕಾಂತಕ್ಕೆ ಹೋಗಿ, ಛಾವಣಿಯ ಮೇಲೇರಿ ಹಿಂಡುವ ಹೃದಯದಿಂದ, ಗಟ್ಟಿಯಾಗಿ ಅತ್ತುಕರೆಯು ತ್ತಿದ್ದೆ–‘ಓ ನನ್ನ ಮಕ್ಕಳಿರಾ, ಬನ್ನಿ! ಎಲ್ಲಿರುವಿರಿ ನೀವೆಲ್ಲ! ನೀವಿರದೆ ನಾನೊಂದು ಕ್ಷಣವೂ ಬದುಕಿರಲಾರೆ!’ ಎಂದು. ಹಡೆದ ತಾಯಿಯೂ ಕೂಡ ತನ್ನ ಮಗುವಿಗಾಗಿ ಅಷ್ಟೊಂದು ಕಾತರ ಳಾಗಿರಲಿಕ್ಕಿಲ್ಲ. ಯಾವ ಸ್ನೇಹಿತನೂ ತನ್ನ ಸ್ನೇಹಿತನಿಗಾಗಿ, ಯಾವ ಪ್ರಿಯನೂ ತನ್ನ ಪ್ರಿಯತಮೆ ಗಾಗಿ ಅಷ್ಟೊಂದು ವ್ಯಾಕುಲನಾಗಿರಲಿಕ್ಕಿಲ್ಲ! ಓಹ್, ಆಗಿನ ನನ್ನ ವ್ಯಾಕುಲತೆ ಅವರ್ಣನೀಯ. ಇದಾದಮೇಲೆಯೇ ಅವರೆಲ್ಲ ಒಬ್ಬೊಬ್ಬರಾಗಿ ಬರಲಾರಂಭಿಸಿದರು.”

ಶ್ರೀರಾಮಕೃಷ್ಣರೆಂಬ ಸಹಸ್ರದಳದ ಕಮಲ ಅರಳಿನಿಂತಾಗ ಅದರ ಮಧುವನ್ನು ಹೀರಲು ನಾನಾ ಬಗೆಯ ಜನರು ದುಂಬಿಗಳಂತೆ ಬರಲಾರಂಭಿಸಿದರು–ಗೌರೀಪಂಡಿತ, ಪದ್ಮಲೋಚನ, ವೈಷ್ಣವಚರಣ, ಶಶಧರ ತರ್ಕಚೂಡಾಮಣಿ ಇವರೇ ಮೊದಲಾದ ಸಾಧಕರು ಬರುತ್ತಾರೆ. ಕೇಶವಚಂದ್ರ ಸೇನ, ಪ್ರತಾಪಚಂದ್ರ ಮಜುಮದಾರ, ವಿಜಯಕೃಷ್ಣ ಗೋಸ್ವಾಮಿ ಮುಂತಾದ ಖ್ಯಾತ ವ್ಯಕ್ತಿಗಳು ಬರುತ್ತಾರೆ. ಕ್ರೈಸ್ತರು, ಮುಸಲ್ಮಾನರು, ಸಿಕ್ಖರು, ಹಿಂದೂಗಳು–ಇವರೆಲ್ಲ ಬರುತ್ತಾರೆ. ಪಂಡಿತರು ಬರುತ್ತಾರೆ, ಕವಿಗಳು ಬರುತ್ತಾರೆ, ಸಾಹಿತಿಗಳು ಬರುತ್ತಾರೆ, ವಾಗ್ಮಿಗಳು ಬರುತ್ತಾರೆ, ಸಿದ್ಧಾಂತಿಗಳು ಬರುತ್ತಾರೆ, ಉಪನ್ಯಾಸಕರು ಬರುತ್ತಾರೆ, ಸಮಾಜದ ಮುಖಂಡರು ಬರುತ್ತಾರೆ, ಭಕ್ತಶ್ರೇಷ್ಠರು ಬರುತ್ತಾರೆ. ಹೀಗೆ ಹಲವುಹತ್ತು ವಿಧದ ಜನರು ಅವರ ಬಳಿಗೆ ಬರಲಾರಂಭಿಸಿದ್ದಾರೆ. ಆದರೆ ಇವರೆಲ್ಲರ ನಡುವೆ ತಾವು ಅತಿಮುಖ್ಯನೆಂದು ಪರಿಗಣಿಸಿರುವ ಒಬ್ಬಾತನ ಬರವನ್ನೇ ಶ್ರೀರಾಮಕೃಷ್ಣರು ಕಾತರದಿಂದ ಇದಿರುನೋಡುತ್ತಿದ್ದಾರೆ... 

ಇದೇ ಸಮಯದಲ್ಲೇ ನಮ್ಮ ಕಥಾನಾಯಕನಾದ ನರೇಂದ್ರನ ಹೃದಯದಲ್ಲೂ ಶ್ರೀರಾಮಕೃಷ್ಣ ರನ್ನು ಕಾಣಬೇಕು; ಕಂಡು, ‘ಮಹಾಶಯರೆ, ನೀವು ದೇವರನ್ನು ಕಂಡಿದ್ದೀರಾ?’ ಎಂದು ಕೇಳಬೇಕು ಎನ್ನುವ ತವಕ ಹುಟ್ಟಿದೆ. ಆತ ತನಗರಿವಿಲ್ಲದೆಯೇ ಅವರೆಡೆಗೆ ಆಕರ್ಷಿತನಾಗಿದ್ದಾನೆ. ಮರಿಸಿಂಹ ಮಹಾಸಿಂಹದೆಡೆಗೆ ಆಕರ್ಷಿತವಾಗುವುದು ಸಹಜವೇ!

