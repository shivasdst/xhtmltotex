\sethyphenation{kannada}{
ಅ
ಅ-ವರು
ಅಂಅ
ಅಂಕ
ಅಂಕ-ಗಳು
ಅಂಕ-ವನ್ನು
ಅಂಕೆ-ಯ-ಲ್ಲಿ-ಟ್ಟು-ಕೊ-ಳ್ಳದೆ
ಅಂಕೆ-ಯ-ಲ್ಲಿರುವ
ಅಂಗ-ಗಳು
ಅಂಗ-ವೆಂದೇ
ಅಂಗ-ಸಾ-ಧನೆ
ಅಂಗ-ಸಾಧ-ನೆಗೆ
ಅಂಗಕ್ಕೆ
ಅಂಗಲಾಚಿ
ಅಂಗಾಂಗ-ಗಳು
ಅಂಗಿ
ಅಂಗಿ-ಯಂತೇ
ಅಂಗೀ-ಕಾರ-ವನ್ನು
ಅಂಗೀಕ-ರಿಸಿದ
ಅಂಗು-ಲವೂ
ಅಂಗುಲ-ವನ್ನೂ
ಅಂಗೈ
ಅಂಗೋಪಾಂಗ-ದ-ಲ್ಲಿ
ಅಂಚಿ-ನ-ಲ್ಲಿ
ಅಂಚಿ-ನ-ಲ್ಲಿ-ದ್ದಾಗ
ಅಂಚೆ
ಅಂಚೆಯ
ಅಂಜ-ತೊಡಗಿದಳು
ಅಂಜ-ದಿ-ರುವ
ಅಂಜ-ಬೇಕಾಗಿ-ಲ್ಲ
ಅಂಜ-ಬೇಕೆ
ಅಂಜ-ಬೇಡ
ಅಂಜ-ಬೇಡಿ
ಅಂಜಿ
ಅಂಜಿ-ಕೆ-ಗ-ಳನ್ನು
ಅಂಜಿ-ಕೆ-ಯನ್ನು
ಅಂಜಿ-ಕೆ-ಯಾ-ಯಿತು
ಅಂಜಿ-ಕೆ-ಯಾಗಲಿ
ಅಂಜಿ-ಕೆ-ಯಾಗುವಂತಹ
ಅಂಜಿ-ಕೆ-ಯಿ-ಲ್ಲ
ಅಂಜಿ-ಕೆ-ಯಿಂದ
ಅಂಜಿ-ಕೆ-ಯೆ-ಲ್ಲ
ಅಂಜಿ-ಕೆ-ಯೇಕೆ
ಅಂಜಿ-ಕೆಗೆ
ಅಂಜಿ-ಕೆಯು
ಅಂಜಿ-ಕೆಯೂ
ಅಂಜಿ-ಕೆಯೆ
ಅಂಜಿ-ಕೆಯೇ
ಅಂಜಿ-ದರು
ಅಂಜಿ-ಸ-ಬೇಕೆಂದು
ಅಂಜಿ-ಸಿ-ದನು
ಅಂಜಿ-ಸಿದ
ಅಂಜಿಕೆ
ಅಂಜು-ವ-ವ-ನ-ಲ್ಲ
ಅಂಜು-ವ-ವ-ರ-ಲ್ಲ
ಅಂಜು-ವನು
ಅಂಜು-ವರು
ಅಂಜು-ವರೊ
ಅಂಜು-ವು-ದಿ-ಲ್ಲ
ಅಂಜು-ವುದೆ
ಅಂಜುಕು-ಳಿಗೆ
ಅಂಜುಕುಳಿ-ಗಳು
ಅಂಜುವೆನೆ
ಅಂಟಿ-ಕೊಂಡವು
ಅಂಟಿ-ಕೊಂಡಿ-ರುವ
ಅಂಟಿ-ಕೊಂಡಿ-ರುವುದು
ಅಂಟಿ-ಕೊಂಡು
ಅಂತ
ಅಂತ-ರ-ಗಂಗೆ-ಯಂತೆ
ಅಂತ-ರ-ತಮ
ಅಂತ-ರಂಗದ
ಅಂತ-ರಾರ್ಥ
ಅಂತ-ರಾಳ-ದ-ಲ್ಲಿ
ಅಂತ-ರಾಳ-ದ-ಲ್ಲಿ-ರುವ
ಅಂತ-ರಾಳ-ದ-ಲ್ಲಿಯೂ
ಅಂತ-ರಾಳ-ದಿಂದ
ಅಂತ-ರಾಳ-ವನ್ನೆ-ಲ್ಲ
ಅಂತ-ರಾಳಕ್ಕೂ
ಅಂತ-ರಾಳಾ-ದ-ಲ್ಲಿ
ಅಂತ-ರಿಕ-ವಾ-ದುದು
ಅಂತ-ಸ್ತಿ-ನ-ಲ್ಲಿ
ಅಂತ-ಸ್ತಿನ-ವರಿ-ದ್ದಾರೆ
ಅಂತ-ಸ್ತು
ಅಂತ-ಸ್ತು-ಗ-ಳನ್ನು
ಅಂತ-ಸ್ತು-ಗಳುಳ್ಳ
ಅಂತ-ಸ್ಥಿನ
ಅಂತ-ಹ-ವ-ನಿಗೆ
ಅಂತ-ಹ-ವ-ರ-ನ್ನು
ಅಂತ-ಹ-ವ-ರಿಂದ
ಅಂತ-ಹ-ವ-ರಿಗೆ
ಅಂತ-ಹ-ವನ
ಅಂತ-ಹ-ವರ
ಅಂತಹ
ಅಂತಿಮ
ಅಂತೂ
ಅಂಥ
ಅಂಥ-ವ-ನಿಗೆ
ಅಂಥ-ವ-ರ-ನ್ನು
ಅಂಥ-ವ-ರ-ಲ್ಲಿ
ಅಂಥ-ವ-ರಿಗೂ
ಅಂಥ-ವನ
ಅಂಥ-ವನು
ಅಂಥ-ವನೆ
ಅಂದ-ಮೇಲೆ
ಅಂದ-ವಾಗಿ
ಅಂದ-ವಾದ
ಅಂದಾಜಿನ
ಅಂದು
ಅಂದು-ಕೊಂಡರೆ
ಅಂದು-ಬಿ-ಟ್ಟೆ
ಅಂದೆ
ಅಂದೇ
ಅಂಧ
ಅಂಧ-ಕಾರ
ಅಂಧ-ಕಾರ-ದ-ಲ್ಲಿ
ಅಂಧ-ಕಾರ-ದಿಂ-ದಾವೃತ-ವಾಗಿವೆ
ಅಂಧ-ಕಾರ-ವನ್ನು
ಅಂಧ-ಕಾರ-ವನ್ನೇ
ಅಂಧ-ಕಾರದ
ಅಂಧ-ರಾಗಿ
ಅಂಧ-ಳಾದ
ಅಂಧ-ಶ್ರದ್ಧೆ
ಅಂಧವೇ
ಅಂಬಾ-ದ-ತ್ತನ
ಅಂಬಾ-ದ-ತ್ತರ
ಅಂಬೆ-ಗಾಲಿ-ಡುವ
ಅಂಶ
ಅಂಶ-ಗಳ-ನ್ನೆ-ಲ್ಲ
ಅಂಶ-ಗಳಿವೆ
ಅಂಶ-ಗಳು
ಅಂಶ-ಗಳೆಂದು
ಅಂಶ-ದಿಂದ
ಅಂಶ-ವನ್ನು
ಅಂಶ-ವಾಗ-ಬೇಕು
ಅಂಶ-ವಿದ್ದರೆ
ಅಂಶ-ವೆಂದು
ಅಂಶವೇ
ಅಂಶಾ-ವತಾರ
ಅಕೌಂಟೆಂ-ಟ್
ಅಕ್ಕ
ಅಕ್ಕ-ಪಕ್ಕದ-ಲ್ಲೆ-ಲ್ಲ
ಅಕ್ಕ-ಸಾಲಿಗ
ಅಕ್ಕಂದಿ-ರ-ನ್ನು
ಅಕ್ಕಂದಿರ
ಅಕ್ಕಿ
ಅಕ್ಟೋಬರ್
ಅಕ್ಬರ-ನಿಗೆ
ಅಕ್ಬರ್
ಅಕ್ಷಯ
ಅಕ್ಷಯ-ವಿದ್ಯೆ
ಅಕ್ಷರ
ಅಕ್ಷರ-ಗ-ಳನ್ನು
ಅಕ್ಷರ-ಗಳ-ನ್ನೆ-ಲ್ಲ
ಅಕ್ಷರ-ದ-ಲ್ಲಿಯೂ
ಅಕ್ಷರಶಃ
ಅಗ-ತ್ಯ
ಅಗ-ಲರಿರಿ
ಅಗ-ಸ್ಟೈ-ನನ
ಅಗಲ-ಬೇಕೆಂದು
ಅಗಲ-ವಾಗಿ
ಅಗಲ-ವಿದೆ
ಅಗಲಬೇ-ಕ-ಲ್ಲ
ಅಗಲಿ
ಅಗಲಿ-ಕೆಗೆ
ಅಗಲಿ-ದರು
ಅಗಾಧ
ಅಗಾಧ-ವಾ-ಗಿದೆ
ಅಗಾಧ-ವಾಗಿ-ತ್ತು
ಅಗಾಧ-ವಾದ
ಅಗಿದು
ಅಗೆ-ಯು-ವುದು
ಅಗೋಚರ
ಅಗ್ಗಿಷ್ಠಿ-ಕೆಯ
ಅಗ್ನಿ
ಅಗ್ನಿ-ಕುಂಡ-ದಿಂದ
ಅಗ್ನಿ-ದೇವನ
ಅಗ್ನಿ-ಪರೀಕ್ಷೆಗೆ
ಅಗ್ನಿ-ಮಯ
ಅಗ್ನಿ-ಶಿಖೆ-ಯಂತಿ-ರುವ
ಅಗ್ನಿಯ
ಅಗ್ರ-ಗಣ್ಯ
ಅಗ್ರ-ಗಣ್ಯ-ನಾಗು-ವನು
ಅಗ್ರ-ಗಣ್ಯರು
ಅಘಟಿತ
ಅಚಲ
ಅಚಲ-ವಾದ
ಅಚ್ಚ
ಅಚ್ಚ-ರಿ-ಪ-ಟ್ಟರು
ಅಚ್ಚ-ರಿ-ಯ-ನ್ನುಂಟು-ಮಾಡಿತು
ಅಚ್ಚ-ಳಿ-ಯದೆ
ಅಚ್ಚರಿ
ಅಚ್ಚಿಗೆ
ಅಚ್ಬ-ಲ್
ಅಚ್ಯುತಾ-ನಂ-ದರು
ಅಚ್ಯುತಾ-ನಂದ
ಅಚ್ಯುತಾ-ನಂದ-ರಿಗೆ
ಅಚ್ಯುತಾ-ನಂದ-ರೊ-ಡನೆ
ಅಚ್ವಾ-ಲ್
ಅಜ್ಜ
ಅಜ್ಜ-ನಂತೆ
ಅಜ್ಜಿ
ಅಜ್ಜಿಯ
ಅಜ್ಞಾತ-ರಾಗಿ-ದ್ದರು
ಅಜ್ಞಾತ-ವಾಸ-ವನ್ನು
ಅಜ್ಮೀ-ರಕ್ಕೆ
ಅಜ್ಮೀರ್
ಅಡ-ಕ-ವಾಗಿ
ಅಡಗಿವೆ
ಅಡಚಣೆ-ಗ-ಳನ್ನು
ಅಡಿ
ಅಡಿ-ಕೆ-ಯನ್ನು
ಅಡಿ-ಕೆಯ
ಅಡಿ-ಗಳ
ಅಡಿ-ಗಳ-ಲ್ಲದೆ
ಅಡಿ-ಗಳಷ್ಟು
ಅಡಿ-ಗಳಿ-ಗಿಂತ
ಅಡಿ-ಗೆ-ಗ-ಳನ್ನು
ಅಡಿ-ಗೆ-ಗಳ-ನ್ನೆ-ಲ್ಲ
ಅಡಿ-ಗೆ-ಮನೆ-ಯ-ಲ್ಲಿದೆ
ಅಡಿ-ಗೆ-ಮಾಡಿ
ಅಡಿ-ಗೆ-ಯ-ಲ್ಲಿ
ಅಡಿ-ಗೆ-ಯ-ವ-ನನ್ನು
ಅಡಿ-ಗೆ-ಯ-ವನು
ಅಡಿ-ಗೆ-ಯನ್ನು
ಅಡಿ-ಗೆಗೆ
ಅಡಿ-ಗೆಯ
ಅಡಿ-ದಾವ-ರೆ-ಯ-ಲ್ಲಿ
ಅಡಿ-ದಾವ-ರೆ-ಯ-ಲ್ಲಿಡಿ
ಅಡಿ-ಯ-ಲ್ಲಿರುವ
ಅಡಿ-ಯಾ-ಳಾಗು-ವುದು
ಅಡಿ-ಯೊಂದು
ಅಡಿಕೆ
ಅಡಿಗೆ
ಅಡುಗೂಲಜ್ಜಿಯ
ಅಡ್ಡ
ಅಡ್ಡ-ದಾ-ರಿಗೆ
ಅಡ್ಡ-ಬ-ರಲಿ
ಅಡ್ಡ-ಬಿದ್ದರು
ಅಡ್ಡ-ಬೀಳು-ವು-ದಿ-ಲ್ಲ
ಅಡ್ಡ-ಲಾ-ಗಿರಿಸಿ
ಅಡ್ಡ-ಲಾಗಿ-ತ್ತು
ಅಡ್ಡ-ಲಾಗಿದ್ದ
ಅಡ್ಡ-ಹಾದಿಗೆ
ಅಡ್ಡಾಡಿ-ಕೊಂಡು
ಅಡ್ಡಿ
ಅಡ್ಡಿ-ಯನ್ನೂ
ಅಡ್ಡಿ-ಯಿ-ಲ್ಲ
ಅಣ-ತಿ-ಯಂತೆ
ಅಣಕಿ-ಸಿದ್ದರು
ಅಣಕಿಸಿ-ರು-ವೆನು
ಅಣಕಿಸು-ವಂತಹ
ಅಣತಿ-ಯನ್ನೂ
ಅಣಿ
ಅಣಿ-ಗೊಳಿಸಿ-ಕೊಂಡು
ಅಣಿ-ಮಾಡಲಾ-ಯಿತು
ಅಣಿ-ಮಾಡಲು
ಅಣಿ-ಮಾಡಿ
ಅಣಿ-ಮಾಡಿ-ಕೊಂಡ
ಅಣಿ-ಮಾಡಿ-ಕೊಂಡು
ಅಣಿ-ಮಾಡಿ-ದರು
ಅಣಿ-ಮಾಡಿ-ದ್ದರು
ಅಣಿ-ಮಾಡಿ-ರುವರು
ಅಣಿ-ಮಾಡಿದ
ಅಣಿ-ಮಾಡು-ತ್ತಾ-ರೆಂದೂ
ಅಣಿ-ಮಾಡುವ
ಅಣಿ-ಮಾದಿ
ಅಣಿ-ಯಾಗಿ
ಅಣಿ-ಯಾಗಿ-ತ್ತು
ಅಣಿ-ಯಾಗಿ-ರು-ವು-ದಾಗಿಯೂ
ಅಣಿ-ಯಾಗು-ತ್ತಿ-ತ್ತು
ಅಣಿ-ಯಾಗು-ತ್ತಿದ್ದರು
ಅಣಿ-ಯಾಗು-ತ್ತಿದ್ದೆ
ಅಣಿ-ಯಾಗು-ವು-ದಕ್ಕೆ
ಅಣಿ-ಯಾಗುವ
ಅಣಿ-ಯಾಗುವೆ
ಅಣು-ವಿನ
ಅಣುರಣಿತ-ವಾಗು-ತ್ತಿ-ತ್ತು
ಅಣುರೇಣುವಿ-ನ-ಲ್ಲೂ
ಅಣ್ಣ
ಅಣ್ಣ-ತಮ್ಮಂದಿ-ರಂತೆ
ಅಣ್ಣ-ತಮ್ಮಂದಿರು
ಅಣ್ಣ-ನಂತೆ
ಅಣ್ಣ-ನನ್ನು
ಅಣ್ಣನೂ
ಅತಂರ್-ಜ್ಞಾನ-ದ-ಲ್ಲೇ
ಅತವಾ
ಅತಿ
ಅತಿ-ಕ್ರಮಿಸಿ
ಅತಿ-ಗಳಿಂದ
ಅತಿ-ಪ್ರಜ್ಞೆ
ಅತಿ-ಮಧು-ರ-ವಾದ
ಅತಿ-ಮುಖ್ಯ-ವಾ-ದರೂ
ಅತಿ-ಯಾಗಿ
ಅತಿ-ಶಯ-ವ-ಲ್ಲ
ಅತಿ-ಶಯ-ವಾಗಿ-ರ-ಬೇಕು
ಅತಿ-ಶಯೋಕ್ತಿ-ಯ-ಲ್ಲ
ಅತಿ-ಶೋ-ಚನೀಯ-ವಾ-ಗಿದೆ
ಅತಿ-ಸಾರ-ದಿಂದ
ಅತೀಂದ್ರಿಯ
ಅತುಲನೀಯ-ವಾದ
ಅಥರ್ವ-ವೇದ
ಅಥವಾ
ಅಥೆ-ನ್ಸ್
ಅದ-ಕ್ಕಾಗಿಯೇ
ಅದ-ಕ್ಕೆ-ಲ್ಲ
ಅದ-ಕ್ಕೋ-ಸ್ಕರವೇ
ಅದ-ಕ್ಕೋಸುಗವೇ
ಅದ-ನ್ನೆ-ಲ್ಲ
ಅದ-ನ್ನೆ-ಲ್ಲಾ
ಅದ-ಮ್ಯ-ನಾಗಿ-ರ-ಬೇಕು
ಅದ-ರ-ಲ್ಲಿ
ಅದ-ರ-ಲ್ಲಿ-ರು-ತ್ತಿ-ತ್ತು
ಅದ-ರ-ಲ್ಲಿ-ರುವ
ಅದ-ರ-ಲ್ಲಿದ
ಅದ-ರ-ಲ್ಲಿದೆ
ಅದ-ರ-ಲ್ಲಿಯೂ
ಅದ-ರ-ಲ್ಲಿಯೇ
ಅದ-ರ-ಲ್ಲೂ
ಅದ-ರ-ಲ್ಲೆ
ಅದ-ರಂತೆ
ಅದ-ರಂತೆಯೆ
ಅದ-ರಂತೆಯೇ
ಅದ-ರಿಂದಲೇ
ಅದ-ರೊ-ಡನೆ
ಅದ-ರೊಂದಿಗೆ
ಅದಕ್ಕಂತೂ
ಅದಕ್ಕಿಂ-ತಲೂ
ಅದಕ್ಕಿಂತ
ಅದಕ್ಕೇ
ಅದು
ಅದು-ರು-ತಿ-ತ್ತು
ಅದೂ
ಅದೆ
ಅದೆ-ಲ್ಲ-ವನ್ನು
ಅದೆ-ಲ್ಲ-ವನ್ನೂ
ಅದೆಂದಿಗೂ
ಅದೆಷ್ಟು
ಅದೇ
ಅದೇ-ನಾ-ದರೂ
ಅದೇ-ನೆಂ-ದರೆ
ಅದೇ-ನೆಂದು
ಅದೇಕೆ
ಅದೇನು
ಅದೇನೂ
ಅದೊ
ಅದೊಂದಿಷ್ಟು
ಅದೊಂದೂ
ಅದೊಂದೇ
ಅದೋ
ಅದ್ಕ್ಕಾಗಿ
ಅದ್ದ-ಲಿ-ಲ್ಲ
ಅದ್ವಿ-ತೀಯ
ಅದ್ವಿ-ತೀಯನೆ
ಅದ್ವಿ-ತೀಯರು
ಅಧಃ-ಪ-ತನ
ಅಧಃ-ಪತ-ನಕ್ಕೆ
ಅಧಃ-ಪತಿ-ತ-ನಾದ
ಅಧಃಪಾ-ತಾಳಕ್ಕೆ
ಅಧಿ-ಕಾರ
ಅಧಿ-ಕಾರ-ಗ-ಳಿದ್ದ-ರೇನು
ಅಧಿ-ಕಾರ-ವನ್ನು
ಅಧಿ-ಕಾರ-ವನ್ನೂ
ಅಧಿ-ಕಾರ-ವನ್ನೆ-ಲ್ಲಾ
ಅಧಿ-ಕಾರ-ವಾಣಿ-ಯಿಂದ
ಅಧಿ-ಕಾರ-ವಿ-ಲ್ಲ
ಅಧಿ-ಕಾರ-ವಿ-ಲ್ಲ-ವೆ-ನ್ನುವರು
ಅಧಿ-ಕಾರ-ವಿದೆ
ಅಧಿ-ಕಾರ-ವಿದೆಯೋ
ಅಧಿ-ಕಾರದ
ಅಧಿ-ಕಾರವೂ
ಅಧಿ-ಕಾರಿ
ಅಧಿ-ಕಾರಿ-ಗ-ಳನ್ನು
ಅಧಿ-ಕಾರಿ-ಗ-ಳಾದ
ಅಧಿ-ಕಾರಿ-ಗಳಿ-ಗೆ-ಲ್ಲ
ಅಧಿ-ಕಾರಿ-ಗಳು
ಅಧಿ-ಕಾರಿ-ಗಳೂ
ಅಧಿ-ಕಾರಿ-ಗಳೊ-ಡನೆ
ಅಧಿ-ಕಾರಿ-ಯೊಬ್ಬ
ಅಧಿ-ಕಾರಿ-ಯೊಬ್ಬನ
ಅಧಿಕ
ಅಧಿಕ-ವಾ-ಗಿದೆ
ಅಧಿಕ-ವಾ-ಗು-ತ್ತಿದೆ
ಅಧಿಕ-ವಾಗಿ
ಅಧಿಕ-ವಾಗಿ-ತ್ತು
ಅಧಿಕಾ-ರಿಗೆ
ಅಧಿಕೃತ-ವಾದ
ಅಧಿಕ್ಯ-ದಿಂದ
ಅಧಿಪ-ತ್ಯ
ಅಧಿಪ-ತ್ಯ-ವನ್ನು
ಅಧಿವೇ-ಶನ
ಅಧಿವೇಶ-ನದ
ಅಧೀನ
ಅಧೀನ-ದ-ಲ್ಲಿ-ಟ್ಟು-ಕೊಂಡು
ಅಧೋ-ಗ-ತಿಗೆ
ಅಧೋ-ಗತಿ
ಅಧೋ-ಗತಿ-ಗಿ-ಳಿದ
ಅಧೋ-ಗತಿ-ಗಿ-ಳಿದಿದೆ
ಅಧೋ-ಗತಿ-ಗಿಳಿ-ದಿ-ದ್ದಾರೆ
ಅಧೋ-ಗತಿ-ಗಿಳಿ-ಯು-ವುದು
ಅಧೋ-ಗತಿ-ಗೆ-ಲ್ಲ
ಅಧೋ-ಗತಿಯ
ಅಧ್ಯ-ಯನ
ಅಧ್ಯ-ಯನ-ಗ-ಳನ್ನು
ಅಧ್ಯ-ಯನ-ದ-ಲ್ಲಿ
ಅಧ್ಯ-ಯನ-ದಿಂದ
ಅಧ್ಯ-ಯನ-ಮಾಡಿ
ಅಧ್ಯ-ಯನ-ವನ್ನು
ಅಧ್ಯಕ್ಷ-ನಾಗಿದ್ದ
ಅಧ್ಯಕ್ಷ-ರ-ನ್ನಾಗಿ
ಅಧ್ಯಕ್ಷ-ರ-ನ್ನಾಗಿಯೂ
ಅಧ್ಯಕ್ಷ-ರಾ-ದರು
ಅಧ್ಯಕ್ಷ-ರಾಗಿ
ಅಧ್ಯಕ್ಷ-ರಾಗಿ-ದ್ದರು
ಅಧ್ಯಕ್ಷ-ರಾದ
ಅಧ್ಯಕ್ಷ-ರಿಗೆ
ಅಧ್ಯಕ್ಷತೆ
ಅಧ್ಯಕ್ಷತೆ-ವಹಿಸಿ
ಅಧ್ಯಕ್ಷರು
ಅಧ್ಯಾ-ತ್ಮ
ಅಧ್ಯಾ-ತ್ಮ-ದೊಂದಿಗೆ
ಅಧ್ಯಾ-ತ್ಮಿಕ
ಅಧ್ಯಾ-ತ್ಮಿಕತೆ
ಅಧ್ಯಾ-ಯದ
ಅಧ್ಯಾಪಕ-ನಾದ
ಅಧ್ಯಾಯ
ಅಧ್ಯಾಯ-ದಂತೆ
ಅನ-ವರ-ತವೂ
ಅನ-ವರತ
ಅನಂ-ತನೂ
ಅನಂತ
ಅನಂತ-ಕೃಪೆ-ಯನ್ನು
ಅನಂತ-ಜ್ಞಾನ
ಅನಂತ-ತೆ-ಯನ್ನು
ಅನಂತ-ತೆ-ಯೊ-ಡನೆ
ಅನಂತ-ದಿಂದ
ಅನಂತ-ನಾಗಕ್ಕೆ
ಅನಂತ-ನಾರಾ-ಯಣ
ಅನಂತ-ವಾದ
ಅನಂತ-ವೆಂದು
ಅನಂತ-ಸಾ-ಗರ
ಅನಂತವೆ
ಅನಂತಾ-ತ್ಮರು
ಅನಂತಾಕಾಶವೇ
ಅನಾ-ಚಾರ-ಗಳು
ಅನಾ-ರೋಗ್ಯ-ದಿಂದ
ಅನಾ-ರೋಗ್ಯ-ವಾ-ಗಿದೆ
ಅನಾ-ರೋಗ್ಯ-ವಾ-ಯಿತು
ಅನಾ-ರೋಗ್ಯ-ವಾಗಲಿ
ಅನಾ-ರೋಗ್ಯ-ವಾದ
ಅನಾ-ವರಣ
ಅನಾಸಕ್ತ-ರಾಗಿ
ಅನಾಸಕ್ತ-ರಾಗಿ-ರು-ವುದು
ಅನಾಸಕ್ತಿ-ಯಿಂದ
ಅನಿ-ಬೆ-ಸೆಂ-ಟ್
ಅನಿ-ವಾರ್ಯ
ಅನಿಬೆಸಂಟರು
ಅನಿಬೆಸೆಂಟರ
ಅನಿಬೆಸೆಂಟರು
ಅನಿರೀಕ್ಷಿ-ತ-ವಾಗಿ
ಅನಿರೀಕ್ಷಿ-ತ-ವಾಗಿ-ತ್ತು
ಅನಿರ್ವ-ಚನೀಯ
ಅನಿರ್ವ-ಚನೀಯ-ವಾದ
ಅನಿಲ-ಗಳು
ಅನು-ಕರ-ಣ-ವನ್ನು
ಅನು-ಕರ-ಣಕ್ಕೆ
ಅನು-ಕರ-ಣೆ-ಯ-ಲ್ಲಿ
ಅನು-ಕರ-ಣೆ-ಯನ್ನೂ
ಅನು-ಕರ-ಣೆಗೆ
ಅನು-ಕರಣೆ
ಅನು-ಕರಿ-ಸಲಾರೆವು
ಅನು-ಕರಿ-ಸು-ವುದು
ಅನು-ಕರಿ-ಸುವರು
ಅನು-ಕರಿಸ-ಹೋಗಿ
ಅನು-ಕರಿಸು-ವು-ದ-ಲ್ಲ
ಅನು-ಗಾಲ
ಅನು-ಗಾಲವೂ
ಅನು-ಗ್ರಹಿ-ಸು-ವುದು
ಅನು-ಗ್ರಹಿಸಿ
ಅನು-ಗ್ರಹಿಸಿ-ದರು
ಅನು-ಗ್ರಹಿಸಿ-ದರೆ
ಅನು-ಗ್ರಹಿಸಿದ
ಅನು-ಭೂತಿ
ಅನು-ಭೂತಿ-ಯನ್ನು
ಅನು-ಮತಿ
ಅನು-ಮತಿ-ಯನ್ನು
ಅನು-ಮತಿ-ಯೊಂದಿಗೆ
ಅನು-ಮಾನ
ಅನು-ಮಾನ-ಗ-ಳನ್ನು
ಅನು-ಮಾನ-ಗಳ
ಅನು-ಮಾನ-ದಿಂದ
ಅನು-ಮಾನ-ವನ್ನು
ಅನು-ಮಾನ-ವವೂ
ಅನು-ಮಾನ-ವಿ-ರ-ಲಿ-ಲ್ಲ
ಅನು-ಮಾನ-ವಿ-ಲ್ಲ
ಅನು-ಮಾನ-ವೂ-ಅನು-ಮಾನಾ-ಸ್ಪದ-ವಾದ
ಅನು-ಮಾನದ
ಅನು-ಮಾನವೂ
ಅನು-ಮಾನಿ-ಸ-ಲಿ-ಲ್ಲ
ಅನು-ಮಾನಿ-ಸಿ-ದನು
ಅನು-ಮಾನಿ-ಸಿದ
ಅನು-ಮಾನಿ-ಸಿದೆ
ಅನು-ಮಾನಿ-ಸಿದೆ-ನ-ಲ್ಲ
ಅನು-ಮಾನಿ-ಸು-ತ್ತಿ-ದ್ದರೂ
ಅನು-ಮಾನಿ-ಸು-ತ್ತಿ-ರ-ಲಿ-ಲ್ಲ
ಅನು-ಮಾನಿ-ಸು-ತ್ತಿ-ರು-ವಿರಿ
ಅನು-ಮಾನಿ-ಸುವ-ವನು
ಅನು-ಮಾನಿ-ಸುವನು
ಅನು-ಮಾನಿಸಿ-ರ-ಲಿ-ಲ್ಲ
ಅನು-ಮಾನಿಸು-ವಂತ-ಹ-ವ-ನನ್ನು
ಅನು-ರಕ್ತ-ನಾಗಿ
ಅನು-ವಾದ
ಅನು-ವಾದ-ದ-ಲ್ಲಿ
ಅನು-ಸ-ರಿಸಿ
ಅನು-ಸರಿ-ಸ-ಕೂಡ-ದೆಂದು
ಅನು-ಸರಿ-ಸ-ದು-ದ-ಕ್ಕಾಗಿ
ಅನು-ಸರಿ-ಸ-ಬೇಕಾಗಿ-ಲ್ಲ
ಅನು-ಸರಿ-ಸ-ಬೇಕಾಗಿದೆ
ಅನು-ಸರಿ-ಸ-ಬೇಕು
ಅನು-ಸರಿ-ಸ-ಬೇಕೆ
ಅನು-ಸರಿ-ಸ-ಬೇಕೆಂದು
ಅನು-ಸರಿ-ಸಬೆಕಾಗಿ-ಲ್ಲ
ಅನು-ಸರಿ-ಸಲಿ
ಅನು-ಸರಿ-ಸಿ-ಕೊಂಡು
ಅನು-ಸರಿ-ಸಿ-ತ್ತೋ
ಅನು-ಸರಿ-ಸಿ-ದರು
ಅನು-ಸರಿ-ಸಿ-ದರೆ
ಅನು-ಸರಿ-ಸಿ-ದುದು
ಅನು-ಸರಿ-ಸಿ-ರುವರು
ಅನು-ಸರಿ-ಸಿದ
ಅನು-ಸರಿ-ಸು-ತ್ತಿ-ದ್ದರು
ಅನು-ಸರಿ-ಸು-ತ್ತಿ-ರು-ವು-ದ-ನ್ನು
ಅನು-ಸರಿ-ಸು-ತ್ತಿದ್ದ
ಅನು-ಸರಿ-ಸು-ತ್ತೇನೆ
ಅನು-ಸರಿ-ಸು-ತ್ತೇವೆ
ಅನು-ಸರಿ-ಸು-ವಂತೆ
ಅನು-ಸರಿ-ಸು-ವು-ದ-ನ್ನು
ಅನು-ಸರಿ-ಸು-ವು-ದಕ್ಕೆ
ಅನು-ಸರಿ-ಸು-ವು-ದಿ-ಲ್ಲ
ಅನು-ಸರಿ-ಸು-ವುದು
ಅನು-ಸರಿ-ಸು-ವೆನು
ಅನು-ಸರಿ-ಸುವ
ಅನು-ಸರಿ-ಸುವರು
ಅನುಕ-ರಿಸಿ-ಕೊಂಡು
ಅನುಕಂಪ
ಅನುಕಂಪ-ದಿಂದ
ಅನುಕಂಪ-ವನ್ನು
ಅನುಕಂಪ-ವಿ-ತ್ತು
ಅನುಕಂಪ-ವೆ-ಲ್ಲ
ಅನುಕಂಪಕ್ಕೆ
ಅನುಕಂಪೆ-ಯಿಂದ
ಅನುಕಂಪೆಯ
ಅನುಕೂಲ
ಅನುಕೂಲ-ವನ್ನು
ಅನುಕೂಲ-ವಾ-ಗಿದೆ
ಅನುಕೂಲ-ವಾ-ದರೆ
ಅನುಕೂಲ-ವಾ-ದಾಗ
ಅನುಕೂಲ-ವಾಗು-ವಂತೆ
ಅನುಕೂಲ-ವಾಗು-ವು-ದಿ-ಲ್ಲ
ಅನುಕೂಲ-ವಾಗು-ವುದು
ಅನುಕೂಲ-ವಾದ
ಅನುಕೂಲ-ವಿ-ದ್ದ-ವರು
ಅನುಕೂಲ-ವಿ-ರ-ಲಿ-ಲ್ಲ
ಅನುಗ್ರಹ
ಅನುಗ್ರಹ-ದಿಂದ
ಅನುಗ್ರಹ-ದಿಂದಲೇ
ಅನುಗ್ರಹಿ-ಸ-ಲಿ-ಲ್ಲ
ಅನುಗ್ರಹಿ-ಸಲಿ
ಅನುಗ್ರಹಿಸ-ಬೇಕೆಂದು
ಅನುಗ್ರಹಿಸು
ಅನುಗ್ರಹಿಸು-ತ್ತಾನೆ
ಅನುಗ್ರಹಿಸು-ವಳು
ಅನುಗ್ರಹಿಸು-ವು-ದಾದರೆ
ಅನುಗ್ರಹಿಸು-ವೆನು
ಅನುಚಿತ
ಅನುತಾಪ-ಕ್ಕಾಗಿ
ಅನುತಾಪ-ವನ್ನು
ಅನುತಾಪಕ್ಕೆ
ಅನುಪಮ
ಅನುಪಮ-ವಾದ
ಅನುಭ-ವವೂ
ಅನುಭವ
ಅನುಭವ-ಗ-ಳನ್ನು
ಅನುಭವ-ಗ-ಳಾಗು-ತ್ತವೆ
ಅನುಭವ-ಗ-ಳಾದವು
ಅನುಭವ-ಗ-ಳಿದ್ದುವು
ಅನುಭವ-ಗಳ
ಅನುಭವ-ಗಳ-ನ್ನೆ-ಲ್ಲ
ಅನುಭವ-ಗಳ-ಲ್ಲದೆ
ಅನುಭವ-ಗಳಾಗ-ಬಹುದು
ಅನುಭವ-ಗಳಾಗಿ
ಅನುಭವ-ಗಳಾಗಿ-ದ್ದರೂ
ಅನುಭವ-ಗಳಿಂದ
ಅನುಭವ-ಗಳು
ಅನುಭವ-ಗಳೆ-ಲ್ಲ
ಅನುಭವ-ಗಳೆಂದು
ಅನುಭವ-ಗಳೇ
ಅನುಭವ-ದ-ಲ್ಲಿ
ಅನುಭವ-ವ-ನ್ನಾಗಿ
ಅನುಭವ-ವನ್ನು
ಅನುಭವ-ವನ್ನೂ
ಅನುಭವ-ವಾ-ಗು-ತ್ತಿದೆ
ಅನುಭವ-ವಾ-ದರೂ
ಅನುಭವ-ವಾ-ಯಿತು
ಅನುಭವ-ವಾಗಿ-ತ್ತು
ಅನುಭವ-ವಾಗು-ವುದು
ಅನುಭವ-ವಿ-ತ್ತು
ಅನುಭವ-ವೆಂದೂ
ಅನುಭವಕ್ಕೂ
ಅನುಭವಕ್ಕೆ
ಅನುಭವದ
ಅನುಭವಿ-ಸದ
ಅನುಭವಿ-ಸಲಾರದ
ಅನುಭವಿ-ಸಲು
ಅನುಭವಿ-ಸಲೇ-ಬೇಕಾಗುವುದು
ಅನುಭವಿ-ಸಿ-ಲ್ಲ
ಅನುಭವಿ-ಸಿದ
ಅನುಭವಿ-ಸಿದೆ
ಅನುಭವಿ-ಸು-ತ್ತ
ಅನುಭವಿ-ಸು-ತ್ತಾ
ಅನುಭವಿ-ಸು-ತ್ತಿದ್ದರು
ಅನುಭವಿ-ಸು-ತ್ತಿರು-ವಾಗಲೂ
ಅನುಭವಿ-ಸು-ತ್ತಿರು-ವೆನು
ಅನುಭವಿ-ಸು-ತ್ತಿರುವ
ಅನುಭವಿ-ಸು-ತ್ತೇನೆ
ಅನುಭವಿ-ಸು-ವಂತೆ
ಅನುಭವಿ-ಸು-ವು-ದ-ನ್ನು
ಅನುಭವಿ-ಸು-ವುದು
ಅನುಭವಿ-ಸುವ-ವ-ರೆಗೆ
ಅನುಭವಿ-ಸುವ-ವರು
ಅನುಭವಿ-ಸುವರು
ಅನುಭವಿ-ಸುವರೊ
ಅನುಭವಿ-ಸುವುದ-ರ-ಲ್ಲಿ
ಅನುಭವಿಸ-ತೊಡಗಿದರು
ಅನುಭವಿಸ-ಬ-ಲ್ಲ-ವ-ರಾಗಿ-ದ್ದರು
ಅನುಭವಿಸ-ಬಹು-ದಾಗಿ-ತ್ತು
ಅನುಭವಿಸ-ಬಹುದು
ಅನುಭವಿಸ-ಬೇ-ಕಾದ
ಅನುಭವಿಸ-ಬೇಕಾಗುವುದು
ಅನುಭವಿಸ-ಬೇಕು
ಅನುಭವಿಸ-ಬೇಕೆಂಬ
ಅನುಭವಿಸಿ
ಅನುಭವಿಸಿ-ದ-ವ-ನೊಬ್ಬ-ನಿಂದ
ಅನುಭವಿಸಿ-ದ-ವನು
ಅನುಭವಿಸಿ-ದ-ವರು
ಅನುಭವಿಸಿ-ದನು
ಅನುಭವಿಸಿ-ದರು
ಅನುಭವಿಸಿ-ದ್ದರು
ಅನುಮಂಶಿ-ಕತೆ
ಅನುಮೋ-ದಿ-ಸಿದರು
ಅನುಮೋದಿ-ಸು-ತ್ತೀರಾ
ಅನು-ಯಾಯಿ-ಗಳ
ಅನು-ಯಾಯಿ-ಗಳು
ಅನು-ಯಾಯಿ-ಗಳೆ-ಲ್ಲಾ
ಅನು-ಯಾಯಿ-ಯಾದ
ಅನು-ಯಾಯಿಯ
ಅನುರಣಿತ-ವಾ-ಗು-ತ್ತಿದೆಯೋ
ಅನುರಣಿತ-ವಾಗು-ತ್ತಿ-ತ್ತು
ಅನುರಣಿತ-ವಾಗು-ವುದು
ಅನುರಾಗ
ಅನುರಾಧಾ-ಪು-ರಕ್ಕೆ
ಅನುರಾಧಾ-ಪು-ರದಿಂದ
ಅನುಷ್ಠಾ-ನಕ್ಕೆ
ಅನುಷ್ಠಾನ
ಅನುಷ್ಠಾನ-ದ-ಲ್ಲಿ
ಅನುಷ್ಠಾನ-ಮಾಡಿ
ಅನುಷ್ಠಾನ-ವನ್ನು
ಅನೃತ-ವಿ-ಲ್ಲ
ಅನೆ-ಕ-ರಿಗೆ
ಅನೇಕ
ಅನೇಕ-ರಿಗೆ
ಅನೇಕ-ವಾಗಿ
ಅನೇಕ-ವೇಳೆ
ಅನೇಕರ
ಅನೇಕರು
ಅನೈಕ-ಮತ್ಯ-ದಿಂದ
ಅನೌಪ-ಚಾರಿ-ಕ-ವಾದ
ಅಪ-ನಂಬಿಕೆ
ಅಪ-ಪ್ರ-ಚಾರ
ಅಪ-ಪ್ರ-ಚಾರ-ಗ-ಳನ್ನು
ಅಪ-ಪ್ರ-ಚಾರ-ಗಳ-ಲ್ಲಿ
ಅಪ-ಪ್ರ-ಚಾರ-ವನ್ನು
ಅಪ-ಪ್ರ-ಚಾರದ
ಅಪ-ಮಾನ
ಅಪ-ಯಶ-ಸ್ಸು
ಅಪ-ರೂಪ
ಅಪ-ವಾದ
ಅಪ-ವಾದ-ಗ-ಳಿಗೆ
ಅಪ-ಹರಿ-ಸಿ-ಕೊಂಡು
ಅಪ-ಹರಿ-ಸು-ವುದು
ಅಪಮಾ-ನಕ್ಕೆ
ಅಪಾ-ಯದ
ಅಪಾ-ಯವೂ
ಅಪಾಯ
ಅಪಾಯ-ಕಾರಿ
ಅಪಾಯ-ಗ-ಳಿಗೆ
ಅಪಾಯ-ವನ್ನೂ
ಅಪಾಯ-ವಿ-ಲ್ಲವೆ
ಅಪಾಯ-ವಿದೆ
ಅಪಾರ
ಅಪೀ-ಲ್
ಅಪೆರಾ
ಅಪೇಕ್ಷಿ-ಸಿ-ದರು
ಅಪೇಕ್ಷಿ-ಸುವರು
ಅಪ್ಪಟ
ಅಪ್ಪಳಿ-ಸಲು
ಅಪ್ಪಳಿ-ಸಿ-ದಂತೆ
ಅಪ್ಪಿ-ಕೊ-ಳ್ಳುವ
ಅಪ್ಪಿ-ಕೊಂಡನು
ಅಪ್ಪಿ-ಕೊಂಡಿತು
ಅಪ್ಪಿ-ರು-ವೆನು
ಅಪ್ಪಿಕೊಂಂಡರು
ಅಪ್ರಾಪ್ಯ
ಅಬು
ಅಬ್ಬ
ಅಭ-ಯಾ-ನಂದ
ಅಭಕ್ಷ್ಯ
ಅಭಿ-ನಂದ-ನೆ-ಗ-ಳನ್ನು
ಅಭಿ-ನಂದ-ನೆ-ಗಳು
ಅಭಿ-ನಂದ-ನೆಗೆ
ಅಭಿ-ನಂದನ
ಅಭಿ-ನಂದಿಸಿ
ಅಭಿ-ನಂದಿಸಿ-ಕೊಂಡರೂ
ಅಭಿ-ನಂದಿಸಿ-ದರು
ಅಭಿ-ಪ್ರಾಯ
ಅಭಿ-ಪ್ರಾಯ-ಗ-ಳನ್ನು
ಅಭಿ-ಪ್ರಾಯ-ಗ-ಳಿಗೆ
ಅಭಿ-ಪ್ರಾಯ-ಗಳ
ಅಭಿ-ಪ್ರಾಯ-ಗಳ-ನ್ನೆ-ಲ್ಲ
ಅಭಿ-ಪ್ರಾಯ-ಗಳ-ನ್ನೇ
ಅಭಿ-ಪ್ರಾಯ-ಗಳ-ಲ್ಲಿ
ಅಭಿ-ಪ್ರಾಯ-ಗಳು
ಅಭಿ-ಪ್ರಾಯ-ಗಳೆ-ಲ್ಲ
ಅಭಿ-ಪ್ರಾಯ-ದ-ಲ್ಲಿ
ಅಭಿ-ಪ್ರಾಯ-ದಂತೆ
ಅಭಿ-ಪ್ರಾಯ-ಪ-ಟ್ಟರು
ಅಭಿ-ಪ್ರಾಯ-ಪಡು-ತ್ತಾರೆ
ಅಭಿ-ಪ್ರಾಯ-ವ-ನ್ನೆ-ಲ್ಲ
ಅಭಿ-ಪ್ರಾಯ-ವನ್ನು
ಅಭಿ-ಪ್ರಾಯ-ವನ್ನೂ
ಅಭಿ-ಪ್ರಾಯ-ವನ್ನೆ
ಅಭಿ-ಪ್ರಾಯ-ವಾಗಿ-ರಲಿ-ಲ್ಲ-ವೆಂದು
ಅಭಿ-ಪ್ರಾಯ-ವೆಂ-ದರೆ
ಅಭಿ-ಪ್ರಾಯ-ವೇ-ನೆಂ-ದರೆ
ಅಭಿ-ಪ್ರಾಯ-ವೇನು
ಅಭಿ-ಪ್ರಾಯ-ವೊಂ-ದ-ನ್ನೇ
ಅಭಿ-ಪ್ರಾಯಕ್ಕೆ
ಅಭಿ-ಪ್ರಾಯವೇ
ಅಭಿ-ಮಂ-ತ್ರಿಸಿ
ಅಭಿ-ಮಾನ
ಅಭಿ-ಮಾನ-ವನ್ನು
ಅಭಿ-ಮುಖ-ವಾಗಿದ್ದ
ಅಭಿ-ರುಚಿ
ಅಭಿ-ರುಚಿ-ಯಿ-ರ-ಲಿ-ಲ್ಲ
ಅಭಿ-ರುಚಿ-ಯಿ-ರು-ವುದು
ಅಭಿ-ವೃದ್ಧಿ
ಅಭಿ-ವೃದ್ಧಿ-ಗಾಗಿ
ಅಭಿ-ವೃದ್ಧಿ-ಯಾಗು-ತ್ತ
ಅಭಿ-ವೃದ್ಧಿಗೆ
ಅಭಿ-ವ್ಯಕ್ತ-ವಾಗು-ತ್ತಿ-ಲ್ಲ
ಅಭಿ-ಶಾಪವು
ಅಭಿನ-ಯದ
ಅಭಿನಯಿ-ಸಿ-ದರು
ಅಭಿನಯಿ-ಸು-ತ್ತಿ-ದ್ದರು
ಅಭಿಲಾಷೆ
ಅಭಿಲಾಷೆ-ಯಿ-ದ್ದರೆ
ಅಭಿಲಾಷೆ-ಯಿಂದ
ಅಭೀಃ
ಅಭೀಪ್ಸೆ
ಅಭೀಪ್ಸೆ-ಗೆ-ಲ್ಲ
ಅಭೀಪ್ಸೆಯ
ಅಭೀಷ್ಟ
ಅಭೇ-ದಾನಂದ
ಅಭೇ-ದಾನಂದ-ರ-ನ್ನು
ಅಭೇ-ದಾನಂದ-ರಿಗೂ
ಅಭೇ-ದಾನಂದ-ರಿಗೆ
ಅಭೇ-ದಾನಂದರ
ಅಭೇ-ದಾನಂದರು
ಅಭೇದ-ಸ್ಥಾನ-ದಿಂದ
ಅಭೇದ್ಯ-ಕೋಟೆ-ಯಂತೆ
ಅಭೇದ್ಯ-ವ-ನ್ನಾಗಿ
ಅಭೇದ್ಯ-ವಾದ
ಅಭ್ಯಂ-ತರ
ಅಭ್ಯಂ-ತರ-ವಿ-ಲ್ಲದೇ
ಅಭ್ಯಂ-ತರ-ವೇನೂ
ಅಭ್ಯಂತ-ರವೂ
ಅಭ್ಯಸಿ-ಸು-ತ್ತಿ-ರುವೆ-ಯ-ಲ್ಲ
ಅಭ್ಯುದ-ಯದ
ಅಮಲಿ-ನ-ಲ್ಲಿ
ಅಮಲೇರಿದವಳಂತೆ
ಅಮಿತ
ಅಮೃ-ತ-ತ್ವಕ್ಕೆ
ಅಮೃ-ತ-ತ್ವದ
ಅಮೃ-ತಕ್ಕಿಂತ
ಅಮೃ-ತದ
ಅಮೃತ
ಅಮೃತ-ದಂತೆ
ಅಮೃತ-ಪು-ತ್ರ
ಅಮೃತ-ಪ್ರವಾಹ
ಅಮೃತ-ವನ್ನು
ಅಮೃತ-ವಾಹಿನಿ
ಅಮೃತ-ಶಿಲೆ-ಯ-ಲ್ಲಿ
ಅಮೃತ-ಶಿಲೆ-ಯಿಂದ
ಅಮೃತ-ಶಿಲೆಯ
ಅಮೃತ-ಸ-ರಕ್ಕೆ
ಅಮೃತಾ-ತ್ಮರು
ಅಮೆಂಪಿ-ಸ್
ಅಮೆರಿ-ಕನ್
ಅಮೆರಿಕ-ದಿಂದ
ಅಮೆರಿಕಾ
ಅಮೊಘ-ವಾದ
ಅಮೋಘ-ವಾ-ದುದು
ಅಮೋಘ-ವಾಗಿ-ತ್ತು
ಅಮೋಘ-ವಾದ
ಅಯತಿಃ
ಅಯ್ಯಂ-ಗಾರ್
ಅಯ್ಯರ್
ಅಯ್ಯಾ
ಅಯ್ಯೊ
ಅಯ್ಯೋ
ಅರ-ಚಿ-ಕೊಂಡ
ಅರ-ಚಿ-ಕೊಂಡೆ
ಅರ-ಮ-ನೆಗೆ
ಅರ-ಮನೆ
ಅರ-ಮನೆ-ಗ-ಳನ್ನು
ಅರ-ಮನೆ-ಗಳ-ಲ್ಲಿ-ದ್ದರು
ಅರ-ಮನೆ-ಗಳ-ಲ್ಲಿಯೂ
ಅರ-ಮನೆ-ಗಳು
ಅರ-ಮನೆ-ಯ-ಲ್ಲಿ
ಅರ-ಮನೆ-ಯ-ಲ್ಲಿ-ದ್ದರೂ
ಅರ-ಮನೆ-ಯ-ಲ್ಲಿ-ದ್ದಾಗ
ಅರ-ಮನೆ-ಯ-ಲ್ಲಿ-ರುವುದಕ್ಕಿಂತ
ಅರ-ಮನೆ-ಯ-ಲ್ಲಿಯೇ
ಅರ-ಮನೆ-ಯ-ವ-ರೆಗೆ
ಅರ-ಮನೆ-ಯ-ವರು
ಅರ-ಮನೆ-ಯಂತಹ
ಅರ-ಮನೆ-ಯನ್ನು
ಅರ-ಮನೆ-ಯೊಂದ-ರ-ಲ್ಲಿ
ಅರ-ಮನೆಯ
ಅರ-ಸರು
ಅರ-ಸರು-ಗಳು
ಅರ-ಸು-ತ್ತಿ-ರುವ
ಅರ-ಸು-ವು-ದಿ-ಲ್ಲ
ಅರ-ಸುವುದ-ರ-ಲ್ಲಿ
ಅರಗಿಳಿ-ಯಂತೆ
ಅರಗಿಸಿ-ಕೊ-ಳ್ಳದೆ
ಅರಗಿಸಿ-ಕೊಂಡರೊ
ಅರಗಿಸಿ-ಕೊಂಡಿದ್ದೇನೆ
ಅರಣ್ಯ
ಅರಣ್ಯ-ಗಳ
ಅರಣ್ಯ-ದ-ಲ್ಲಿ
ಅರಣ್ಯ-ದಿಂದ
ಅರಣ್ಯಾರಣ್ಯ-ಗಳ
ಅರಬ್ಬಿ
ಅರಬ್ಬಿ-ದೇಶ
ಅರಬ್ಬಿ-ಯ-ವರೂ
ಅರಬ್ಬಿ-ಸಮುದ್ರ
ಅರಳಿ-ರುವ
ಅರಳಿತು
ಅರಳು
ಅರಸರ
ಅರಸಿ
ಅರಸಿ-ಕೊಂಡು
ಅರಿ
ಅರಿ-ತ-ಮೇಲೆ
ಅರಿ-ತ-ವನ
ಅರಿ-ತನು
ಅರಿ-ತರು
ಅರಿ-ತರೆ
ಅರಿ-ತಾಗ
ಅರಿ-ತಿ-ದ್ದ-ವರು
ಅರಿ-ತಿ-ದ್ದರು
ಅರಿ-ತಿ-ರುವಾಗ
ಅರಿ-ತಿ-ಲ್ಲ
ಅರಿ-ತು-ಕೊಂಡಿತು
ಅರಿ-ತೊ-ಡನೆ
ಅರಿ-ತೊ-ಡನೆಯೆ
ಅರಿ-ತೊ-ಡನೆಯೇ
ಅರಿ-ಯ-ತೊಡಗಿದ
ಅರಿ-ಯ-ಬ-ಲ್ಲೆಯಾ
ಅರಿ-ಯ-ಬೇ-ಕಾ-ದರೆ
ಅರಿ-ಯ-ಬೇಕಾಗಿದೆ
ಅರಿ-ಯ-ಬೇಕು
ಅರಿ-ಯ-ಬೇಕೆಂದು
ಅರಿ-ಯದ
ಅರಿ-ಯದೆ
ಅರಿ-ಯಲಾ-ರದ
ಅರಿ-ಯಲು
ಅರಿ-ಯು-ತ್ತಿರುವ
ಅರಿ-ಯು-ವು-ದ-ಕ್ಕಾಗಿ
ಅರಿ-ಯು-ವು-ದಕ್ಕೆ
ಅರಿ-ಯು-ವುದೇ
ಅರಿ-ಯುವ-ವ-ರೆಗೆ
ಅರಿ-ಯುವರೋ
ಅರಿ-ಯುವೆ
ಅರಿ-ಯೆವು
ಅರಿ-ವನ್ನು
ಅರಿ-ವಾ-ಗಿದೆ
ಅರಿ-ವಾ-ಯಿತು
ಅರಿ-ವಾ-ಯಿತೆಂ-ದರೆ
ಅರಿ-ವಾಗ-ತೊಡಗಿತು
ಅರಿ-ವಾಗ-ಲಿ-ಲ್ಲ
ಅರಿ-ವಾಗಿ
ಅರಿ-ವಾಗಿ-ದೆಯೆ
ಅರಿ-ವಾಗಿ-ದ್ದವು
ಅರಿ-ವಾಗು-ವಂತೆ
ಅರಿ-ವಾಗು-ವುದು
ಅರಿ-ವಿ-ಲ್ಲದ
ಅರಿ-ವಿ-ಲ್ಲದೆ
ಅರಿ-ವಿ-ಲ್ಲದೇ
ಅರಿತ
ಅರಿತು
ಅರಿತೆ
ಅರಿವೇ
ಅರುಂಧ-ತಿಯರ
ಅರುಣಾಚಲಂ
ಅರುಣೋದ-ಯ-ದಂತೆ
ಅರುಣೋದಯ-ದ-ಲ್ಲಿ
ಅರುಣೋದಯ-ವನ್ನು
ಅರೆ
ಅರೆ-ಗ-ಳಿಗೆಯೂ
ಅರೆ-ನಾ-ಗರಿಕ-ರಾಗಿ
ಅರೆ-ನಾ-ಗರೀ-ಕರು
ಅರೆ-ನಿದ್ರೆ-ಯ-ಲ್ಲಿರು-ವಂತೆ
ಅರೆ-ನಿದ್ರೆ-ಯ-ಲ್ಲಿರು-ವಾಗ
ಅರ್ಚಕ-ರೊ-ಡನೆ
ಅರ್ಜಿ-ಯನ್ನು
ಅರ್ಜು-ನನ
ಅರ್ಜುನ
ಅರ್ಥ
ಅರ್ಥ-ಗ-ಳನ್ನು
ಅರ್ಥ-ದ-ಲ್ಲಿ
ಅರ್ಥ-ಮಾಡಿ-ಕೊ-ಳ್ಳದೇ
ಅರ್ಥ-ಮಾಡಿ-ಕೊ-ಳ್ಳಲು
ಅರ್ಥ-ಮಾಡಿ-ಕೊ-ಳ್ಳು-ತ್ತಿದ್ದರು
ಅರ್ಥ-ಮಾಡಿ-ಕೊ-ಳ್ಳುವ
ಅರ್ಥ-ಮಾಡಿ-ಕೊ-ಳ್ಳುವರು
ಅರ್ಥ-ಮಾಡಿ-ಕೊ-ಳ್ಳುವೆ
ಅರ್ಥ-ಮಾಡಿ-ಕೊಂಡಷ್ಟು
ಅರ್ಥ-ಮಾಡಿ-ಕೊಂಡಿ-ರ-ಲಿ-ಲ್ಲ
ಅರ್ಥ-ಮಾಡಿ-ಕೊಂಡಿ-ಲ್ಲ
ಅರ್ಥ-ಮಾಡಿ-ಕೊಂಡಿ-ಲ್ಲವೊ
ಅರ್ಥ-ಮಾಡಿ-ಕೊಂಡಿದ್ದರೆ
ಅರ್ಥ-ಮಾಡಿ-ಕೊಂಡಿದ್ದೇನೆ
ಅರ್ಥ-ಮಾಡಿ-ಕೊಂಡು
ಅರ್ಥ-ಮಾಡಿ-ಕೊಂಡೆ
ಅರ್ಥ-ಮಾಡಿ-ಕೊಳ್ಳ-ಬ-ಲ್ಲ
ಅರ್ಥ-ಮಾಡಿ-ಕೊಳ್ಳ-ಬ-ಲ್ಲ-ವನೋ
ಅರ್ಥ-ಮಾಡಿ-ಕೊಳ್ಳ-ಬ-ಲ್ಲರು
ಅರ್ಥ-ಮಾಡಿ-ಕೊಳ್ಳ-ಬಹು-ದಾಗಿ-ತ್ತು
ಅರ್ಥ-ಮಾಡಿ-ಕೊಳ್ಳಲಾ-ರರು
ಅರ್ಥ-ವನ್ನು
ಅರ್ಥ-ವನ್ನೆ-ಲ್ಲ
ಅರ್ಥ-ವಾ-ಗು-ತ್ತಿದೆ
ಅರ್ಥ-ವಾ-ಯಿತು
ಅರ್ಥ-ವಾ-ಯಿತೆ
ಅರ್ಥ-ವಾಗ-ಬೇಕು
ಅರ್ಥ-ವಾಗ-ಲಿ-ಲ್ಲ
ಅರ್ಥ-ವಾಗದೆ
ಅರ್ಥ-ವಾಗು-ತ್ತ
ಅರ್ಥ-ವಾಗು-ತ್ತದೆ
ಅರ್ಥ-ವಾಗು-ತ್ತಿ-ರಲಿ-ಲ್ಲ
ಅರ್ಥ-ವಾಗು-ವಂತಹು-ದ-ಲ್ಲ
ಅರ್ಥ-ವಾಗು-ವಂತೆ
ಅರ್ಥ-ವಾಗು-ವು-ದಿ-ಲ್ಲ
ಅರ್ಥ-ವಾಗುವ
ಅರ್ಥ-ವಾಯಿತೇ
ಅರ್ಥ-ವಿ-ದೆಯೆ
ಅರ್ಥ-ವಿ-ಲ್ಲ
ಅರ್ಥ-ವೇ-ನೆಂಬು-ದ-ನ್ನು
ಅರ್ಥ-ವೇನು
ಅರ್ಥ-ವೇನೋ
ಅರ್ಥದ
ಅರ್ಧ
ಅರ್ಧ-ಗಂಟೆ
ಅರ್ಧ-ಗಂಟೆ-ಗಿಂತ
ಅರ್ಧ-ದಷ್ಟೂ
ಅರ್ಧ-ದೂರ
ಅರ್ಧ-ನಗ್ನ-ರಾದ
ಅರ್ಧ-ನಿಮೀಲಿತ
ಅರ್ಧ-ರಾ-ತ್ರಿ
ಅರ್ಧ-ರಾ-ತ್ರಿಯ
ಅರ್ಪಣ
ಅರ್ಪಣೆ
ಅರ್ಪಿ-ಸಲಾ-ಯಿತು
ಅರ್ಪಿ-ಸಲು
ಅರ್ಪಿ-ಸಿದ
ಅರ್ಪಿ-ಸಿದ್ದ
ಅರ್ಪಿ-ಸಿದ್ದೆ
ಅರ್ಪಿ-ಸು-ತ್ತ
ಅರ್ಪಿ-ಸುವ
ಅರ್ಪಿಸ-ಬ-ಲ್ಲ-ವರ
ಅರ್ಪಿಸ-ಬಹುದು
ಅರ್ಪಿಸ-ಬೇಕಾಗಿದೆ
ಅರ್ಪಿಸ-ಬೇಕು
ಅರ್ಪಿಸ-ಲಾಗು-ವು-ದಿ-ಲ್ಲವೆ
ಅರ್ಪಿಸಿ
ಅರ್ಪಿಸಿ-ದರು
ಅರ್ಪಿಸಿ-ದರೆ
ಅರ್ಪಿಸಿ-ರು-ವಳು
ಅರ್ಪಿಸಿರು
ಅರ್ಪಿಸು
ಅರ್ಪಿಸು-ತ್ತೇನೆ
ಅರ್ಪಿಸು-ತ್ತೇವೆ
ಅರ್ಪಿಸು-ವ-ದ-ಕ್ಕ-ಲ್ಲ
ಅರ್ಪಿಸು-ವು-ದ-ಕ್ಕಾಗಿ
ಅರ್ಪಿಸು-ವು-ದ-ನ್ನು
ಅರ್ಪಿಸು-ವು-ದೆಂದು
ಅರ್ಪಿಸು-ವು-ವಂತೆ
ಅರ್ಪಿಸು-ವೆನು
ಅರ್ಹ
ಅರ್ಹ-ತಾ-ಪ-ತ್ರ
ಅರ್ಹ-ತಾ-ಪ-ತ್ರ-ದೊ-ಡನೆ
ಅರ್ಹ-ತೆ-ಯನ್ನು
ಅರ್ಹ-ತ್
ಅರ್ಹ-ನಾಗು-ವ-ರೆಂದು
ಅರ್ಹ-ರಾದ
ಅರ್ಹ-ವಾಗಿ-ದ್ದರೆ
ಅರ್ಹರು
ಅರ್ಹರೇ
ಅಲ-ಕಾ-ನಂದಾ
ಅಲ-ನ್
ಅಲಂ-ಕರಿ-ಸಿ-ದರು
ಅಲಂ-ಕರಿ-ಸಿ-ರುವರು
ಅಲಂ-ಕರಿ-ಸಿದ್ದರು
ಅಲಂ-ಕರಿಸ-ಬೇಕೆಂದು
ಅಲಂ-ಕರಿಸ-ಲ್ಪಟ್ಟಿ-ತ್ತು
ಅಲಂ-ಕಾ-ರಕ್ಕೆ
ಅಲಂ-ಕಾರ
ಅಲಂ-ಕಾರ-ಗ-ಳನ್ನು
ಅಲಂ-ಕಾರ-ಗಳು
ಅಲಂ-ಕಾರ-ಮಯ-ವಾಗು-ವುದು
ಅಲಂ-ಕಾರ-ವ-ಲ್ಲ
ಅಲಂಕ-ರಿಸಿದ
ಅಲಂಕೃ-ತ-ರಾಗಿ
ಅಲಂಕೃ-ತ-ವಾಗಿ-ತ್ತು
ಅಲಂಕೃತ-ವಾದ
ಅಲಂಕೃತಳಾಗಿ
ಅಲಕೃಂತ
ಅಲಗಿನ
ಅಲಭ್ಯ
ಅಲಹಾಬಾದಿ-ನ-ಲ್ಲಿ
ಅಲಹಾಬಾದಿ-ನ-ಲ್ಲಿ-ರುವ
ಅಲಹಾಬಾದ್
ಅಲಿ-ಯ-ವ-ರೆಗೂ
ಅಲುಗಿ-ಸಲು
ಅಲೆ
ಅಲೆ-ಅಲೆ-ಯಾಗಿ
ಅಲೆ-ಗ-ಳನ್ನು
ಅಲೆ-ಗಳು
ಅಲೆ-ಗ್ಸಾಂಡ್ರಿ-ಯಾವ-ರೆಗೂ
ಅಲೆ-ದರು
ಅಲೆ-ದಲೆದು
ಅಲೆ-ದಾ-ಡು-ತ್ತಿದ್ದನು
ಅಲೆ-ದಾ-ಡುವ
ಅಲೆ-ದಾಟ-ದ-ಲ್ಲಿ
ಅಲೆ-ದಾಡ-ಬೇಕಾ-ಯಿತು
ಅಲೆ-ದಾಡಿ
ಅಲೆ-ದಾಡಿ-ದರೆ
ಅಲೆ-ದಾಡು-ತ್ತಿದ್ದಾಗ
ಅಲೆ-ದಿವೆ
ಅಲೆ-ಯ-ಬೇಕಾ-ಯಿತು
ಅಲೆ-ಯಂತೆ
ಅಲೆ-ಯನ್ನು
ಅಲೆ-ಯಾಗಿ-ರ-ಬಹುದು
ಅಲೆ-ಯು-ತ್ತಿ-ರು-ವಿರಿ
ಅಲೆ-ಯು-ತ್ತಿದ್ದರು
ಅಲೆ-ಯು-ತ್ತಿದ್ದರೆ
ಅಲೆ-ಯು-ತ್ತಿದ್ದಾಗ
ಅಲೆ-ಯು-ತ್ತಿರು-ವರು
ಅಲೆ-ಯು-ತ್ತಿರು-ವುದು
ಅಲೆ-ಯು-ತ್ತಿರು-ವೆನು
ಅಲೆ-ಯು-ತ್ತಿರುವ
ಅಲೆ-ಯು-ತ್ತಿರುವೆ
ಅಲೆ-ಯು-ವುದು
ಅಲೆಯ
ಅಲೋ-ಚನಾ
ಅಲೋ-ಚನೆ-ಗ-ಳನ್ನು
ಅಲೋ-ಚಿಸಿ-ದರು
ಅಲೋ-ಚಿಸಿ-ದರೆ
ಅಲೋಚ-ನೆ-ಯಿಂದ
ಅಳ-ಲ-ನ್ನು
ಅಳ-ಸಿಂಗ್
ಅಳಲು
ಅಳವಡಿಸಿ-ಕೊಂಡಿ-ರುವೆವು
ಅಳಸಿಂಗ
ಅಳಿಯಂದಿ-ರಿ-ದ್ದರು
ಅಳಿಸಿ-ಬಿ-ಟ್ಟು
ಅಳಿಸಿ-ಹೋ-ಗಿದೆ
ಅಳಿಸಿ-ಹೋಗು-ವು-ದಿ-ಲ್ಲ
ಅಳು-ತ್ತ
ಅಳು-ತ್ತಾ
ಅಳು-ತ್ತಾರೆ
ಅಳು-ತ್ತಿ-ದ್ದು-ದ-ರಿಂದ
ಅಳು-ತ್ತಿ-ರು-ವು-ದ-ನ್ನು
ಅಳು-ತ್ತಿದ್ದರು
ಅಳು-ತ್ತಿದ್ದಾಗ
ಅಳು-ತ್ತಿರು-ವರು
ಅಳು-ತ್ತಿು-ರು-ವನು
ಅಳು-ನಗು-ಗಳೊ-ಡನೆ
ಅಳುವೆ
ಅಳೆ-ಯಲೆ-ತ್ನಿ-ಸುವೆಯಾ
ಅಳೆ-ಯು-ವು-ದಕ್ಕೆ
ಅಳೆಯ-ಬೇಡಿ
ಅಳೆಯು-ವಂತಹ
ಅವ-ತ-ರಿ-ಸಿದ್ದು
ಅವ-ತ-ರಿಸಿ-ದರು
ಅವ-ತ-ರಿಸಿ-ದ್ದ-ನೆಂದು
ಅವ-ನ-ಲ್ಲಿ
ಅವ-ನ-ಲ್ಲಿ-ರು-ವುದು
ಅವ-ನ-ಲ್ಲಿ-ರುವ
ಅವ-ನ-ಲ್ಲಿದೆ
ಅವ-ನ-ಲ್ಲಿಯೂ
ಅವ-ನದು
ಅವ-ನದೇ
ಅವ-ನನ್ನು
ಅವ-ನನ್ನೇ
ಅವ-ನನ್ನೇಕೆ
ಅವ-ನಿ-ಗಾ-ದರೋ
ಅವ-ನಿ-ಗಿಂತ
ಅವ-ನಿ-ಲ್ಲದ
ಅವ-ನಿಂದ
ಅವ-ನಿಗೂ
ಅವ-ನಿಗೇ
ಅವ-ನಿಗೇ-ತಕ್ಕೆ
ಅವ-ನೊಂದಿಗೆ
ಅವ-ನೊಬ್ಬ
ಅವ-ನೊಬ್ಬನೇ
ಅವ-ರ-ನ್ನೇ
ಅವ-ರ-ಲ್ಲಿ
ಅವ-ರ-ಲ್ಲಿ-ತ್ತು
ಅವ-ರ-ಲ್ಲಿ-ದ್ದಂತೆ
ಅವ-ರ-ಲ್ಲಿ-ರು-ವು-ದ-ನ್ನು
ಅವ-ರ-ಲ್ಲಿ-ರುವ
ಅವ-ರ-ಲ್ಲಿದೆ
ಅವ-ರ-ಲ್ಲಿದ್ದ
ಅವ-ರ-ಲ್ಲಿಯೂ
ಅವ-ರಂತೆ
ಅವ-ರಿ-ಗಾಗಿ
ಅವ-ರಿ-ಗಿದೆ
ಅವ-ರಿ-ಗೋ-ಸ್ಕರ
ಅವ-ರಿಗಿ-ತ್ತು
ಅವ-ರಿಗಿ-ರುವ
ಅವ-ರಿಗಿದ್ದ
ಅವ-ರಿಗೇ
ಅವ-ರಿಗೇ-ನಾ-ದರೂ
ಅವ-ರಿಗೇನೂ
ಅವ-ರಿದ್ದ
ಅವ-ರಿಬ್ಬರೂ
ಅವ-ರೇನು
ಅವ-ರೊಬ್ಬರೆ
ಅವ-ಲೋಕಿ-ಸು-ತ್ತಿ-ದ್ದರು
ಅವ-ಳನ್ನು
ಅವ-ಳಿಗೆ
ಅವ-ಶ್ಯಕ
ಅವ-ಶ್ಯಕ-ತೆ-ಗ-ಳನ್ನು
ಅವ-ಶ್ಯಕ-ತೆಗೆ
ಅವ-ಶ್ಯಕ-ವಾಗಿ
ಅವ-ಶ್ಯಕ-ವಾದ
ಅವ-ಶ್ಯಕ-ವೇನೂ
ಅವ-ಶ್ಯಕತೆ
ಅವ-ಸ್ಥಾ-ಭೇದ-ವನ್ನು
ಅವ-ಸ್ಥಾನ-ವೆಂದು
ಅವ-ಸ್ಥಿತ-ವಾದ
ಅವ-ಸ್ಥೆ
ಅವ-ಸ್ಥೆ-ಗಳ-ಲ್ಲಿಯೂ
ಅವ-ಸ್ಥೆ-ಯ-ಲ್ಲಿ
ಅವ-ಸ್ಥೆ-ಯ-ಲ್ಲೇ
ಅವ-ಸ್ಥೆ-ಯನ್ನು
ಅವ-ಸ್ಥೆ-ಯಿಂದ
ಅವ-ಸ್ಥೆಗೆ
ಅವ-ಸ್ಥೆಯ
ಅವ-ಸ್ಥೆಯು
ಅವ-ಸ್ಥೆಯೇ
ಅವಗಾಹ-ನೆಗೆ
ಅವಲಂ-ಬಿಸಿ-ಕೊಂಡು
ಅವಲಂ-ಬಿಸಿದ
ಅವಲಂ-ಬಿಸಿದೆ
ಅವಲಂಬಿ-ಸು-ವಂತೆ
ಅವಲಂಬಿ-ಸುವರು
ಅವಲಾಂಚ್
ಅವಳ
ಅವಳದೇ
ಅವಳಿ-ಗೊಂದು
ಅವಳಿ-ಲ್ಲದ
ಅವಳೆ
ಅವಳೇ
ಅವಳೊ-ಡನೆ
ಅವಳೊಂದು
ಅವಶೇಷ
ಅವಶೇಷ-ಗ-ಳನ್ನು
ಅವಶೇಷ-ಗ-ಳಿಗೆ
ಅವಶೇಷ-ಗಳು
ಅವಶೇಷ-ದ-ಲ್ಲಿ
ಅವಶೇಷ-ವ-ನ್ನಿ-ಟ್ಟಿ-ರುವ
ಅವಶೇಷ-ವನ್ನು
ಅವಶೇಷಕ್ಕೂ
ಅವಶೇಷದ
ಅವಸರ-ದ-ಲ್ಲಿ
ಅವಸರ-ಪಡು-ವು-ದಿ-ಲ್ಲ
ಅವಹೇ-ಳನ
ಅವಹೇ-ಳನಾ-ಸ್ಪದ-ವಾದ
ಅವಹೇ-ಳನೀಯ-ವಾದ
ಅವಹೇಳ-ನದ
ಅವಿ-ಕಾರಿ-ಯಾದುದು
ಅವಿಚ್ಛಿ-ನ್ನ
ಅವಿಚ್ಛಿ-ನ್ನ-ವಾಗಿ
ಅವಿಚ್ಛಿ-ನ್ನ-ವಾದ
ಅವಿತಿ-ರು-ವುದು
ಅವಿತು-ಕೊಂಡಿತು
ಅವಿಧೇಯ-ವಾಗಿ-ರ-ಕೂ-ಡದು
ಅವಿರ್ಭಾವ-ನೆಯ
ಅವಿಶ್ರಾಂ-ತ-ರಾಗಿ
ಅವು
ಅವು-ಗ-ಳನ್ನು
ಅವು-ಗ-ಳಿಗೆ
ಅವು-ಗಳ
ಅವು-ಗಳ-ನ್ನೆ-ಲ್ಲ
ಅವು-ಗಳ-ನ್ನೆ-ಲ್ಲಾ
ಅವು-ಗಳ-ನ್ನೇ
ಅವು-ಗಳ-ಲ್ಲಿ
ಅವು-ಗಳಿಂದ
ಅವು-ಗಳಿಂದೆಷ್ಟು
ಅವು-ಗಳು
ಅವು-ಗಳೆ-ಲ್ಲ
ಅವು-ಗಳೇ
ಅವು-ಗಳೊ-ಡನೆ
ಅವು-ಗಳೊ-ಳಗೆ
ಅವೆ-ನ್ಯೂ
ಅವೆ-ರಡ-ನ್ನೂ
ಅವೇ
ಅವೇ-ನಾ-ದರೂ
ಅವ್ಯಾಹ-ತ-ವಾಗಿ
ಅವ್ಯಾಹತ-ವಾದ
ಅಶಿ-ಸು-ತ್ತೆನೆ
ಅಶೀರ್ವ-ದಿ-ಸಿದರು
ಅಶ್ರು
ಅಶ್ರು-ಲೋ-ಚನ-ನಾಗಿ
ಅಶ್ಲೀಲ
ಅಶ್ಲೀಲ-ತೆ-ಯಿಂದ
ಅಶ್ವ-ಸ್ಥ-ವೃಕ್ಷದ
ಅಶ್ವತ್ಥ
ಅಶ್ವಿನೀ-ಕುಮಾರ
ಅಶ್ವಿನೀ-ಕುಮಾರ-ದ-ತ್ತರು
ಅಶ್ವಿನೀ-ಬಾಬು
ಅಶ್ವಿನೀ-ಬಾಬು-ಗ-ಳನ್ನು
ಅಷ್ಟ-ನ್ನು
ಅಷ್ಟ-ಪದಿ-ಯನ್ನು
ಅಷ್ಟ-ಸಿದ್ಧಿ-ಗಳ-ಲ್ಲಿ
ಅಷ್ಟಷ್ಟು
ಅಷ್ಟಾಧ್ಯಾಯಿ-ಯನ್ನು
ಅಷ್ಟಾವಕ್ರ
ಅಷ್ಟು
ಅಷ್ಟು-ದೂರ
ಅಷ್ಟು-ಹೊ-ತ್ತಿನಿಂದಲೂ
ಅಷ್ಟೆ
ಅಷ್ಟೇ
ಅಷ್ಟೇನು
ಅಷ್ಟೇನೂ
ಅಷ್ಟೊಂದು
ಅಷ್ವತ್ಥ-ವೃಕ್ಷದ
ಅಸ-ಮನ್
ಅಸಂಖ್ಯ
ಅಸಂಖ್ಯಾತ
ಅಸಂತುಷ್ಟ-ನಾಗಿ-ರ-ಬೇಕು
ಅಸಡ್ಡೆ
ಅಸಡ್ಡೆ-ಯನ್ನು
ಅಸಹ್ಯ
ಅಸಹ್ಯ-ಕರ-ವಾದ
ಅಸಿ
ಅಸಿ-ಸ್ಟೆಂ-ಟ್
ಅಸೀಮ
ಅಸೀಮ-ವಾದ
ಅಸುರೀ
ಅಸೂ-ಯೆಗೆ
ಅಸೂಯೆ
ಅಸೋ-ಷಿಯೇ-ಶನ್ನಿನ
ಅಸೋಸಿಯೇಷ-ನ್
ಅಸೌ-ಜನ್ಯ-ವನ್ನು
ಅಹ-ನ್ಯಾಸ
ಅಹ-ಮ್ಮದಾಬಾ-ದಿನ
ಅಹಂ
ಅಹಂ-ಕಾರ
ಅಹಂ-ಕಾರ-ದಿಂ-ದಾಗಿ
ಅಹಂ-ಕಾರ-ವನ್ನು
ಅಹಂ-ಕಾರ-ವಿ-ರ-ಲಿ-ಲ್ಲ
ಅಹಂ-ಕಾರದ
ಅಹಂ-ಕಾರವೇ
ಅಹಂ-ಭಾವ-ವಿ-ರು-ವುದೋ
ಅಹಮದಾಬಾ-ದಿ-ನ-ಲ್ಲಿ
ಅಹಮದಾಬಾ-ದಿನ
ಅಹಮದಾಬಾದ್
ಅಹಮದ್
ಅಹುರಮಾಜ್ದನೊ
ಅಹೋ-ರಾ-ತ್ರಿ
ಆ
ಆಂಆ
ಆಂಗ್ಲ
ಆಂಗ್ಲ-ಪದ್ಯ-ದ-ಲ್ಲಿ
ಆಂಗ್ಲ-ಭಾಷೆ-ಯ-ಲ್ಲಿ
ಆಂಗ್ಲ-ಭಾಷೆಯ
ಆಂಗ್ಲ-ಸರ್ಕಾರ-ವನ್ನು
ಆಂಗ್ಲಿ-ಕನ್
ಆಂಗ್ಲೇ-ಯನ
ಆಂಗ್ಲೇ-ಯನು
ಆಂಗ್ಲೇ-ಯರ
ಆಂಗ್ಲೇ-ಯರು
ಆಂಗ್ಲೇಯ
ಆಂಗ್ಲೇಯ-ನಿಗೆ
ಆಂಗ್ಲೇಯ-ರ-ಲ್ಲಿ
ಆಂಗ್ಲೇಯ-ರ-ಲ್ಲಿಯೂ
ಆಂಗ್ಲೇಯ-ರಿಗೆ
ಆಂದೋ-ಳನ-ವನ್ನು
ಆಂದೋ-ಳನ-ವೆ-ದ್ದಿತು
ಆಂದೋಲ-ನದ
ಆಂದೋಲನವೆ-ದ್ದಿತು
ಆಂದೋಳ-ನಕ್ಕೆ
ಆಂದೋಳ-ನವೇ
ಆಆ-ವಿಷಯದ
ಆಕಾ-ಶಕ್ಕೆ
ಆಕಾಶ
ಆಕಾಶ-ದ-ಲ್ಲಿ
ಆಕಾಶ-ದಂತೆ
ಆಕಾಶ-ದಷ್ಟು
ಆಕಾಶ-ವನ್ನು
ಆಕಾಶ-ಸ-ಮಾನ-ವಾದ
ಆಕಾಶವೆ
ಆಕೆ
ಆಕೆ-ಯನ್ನು
ಆಕೆ-ಯಿಂದ
ಆಕೆ-ಯೊ-ಡನೆ
ಆಕೆಗೂ
ಆಕ್ಟಿನ
ಆಕ್ಷೇಪಣೆ
ಆಕ್ಷೇಪಿ-ಸಿದ
ಆಕ್ಸಫರ್ಡ್ನ-ಲ್ಲಿ
ಆಕ್ಸಿ-ಜನ್
ಆಕ್ಸ್ಫರ್ಡ್
ಆಗ
ಆಗ-ತಾನೆ
ಆಗ-ದ-ವರು
ಆಗ-ದಂತೆ
ಆಗ-ದಷ್ಟು
ಆಗ-ಬ-ಲ್ಲದು
ಆಗ-ಬಹುದು
ಆಗ-ಬೇ-ಕಾ-ದರೆ
ಆಗ-ಬೇ-ಕಾದ
ಆಗ-ಬೇಕಾಗಿ-ತ್ತು
ಆಗ-ಬೇಕಾಗಿ-ಲ್ಲ
ಆಗ-ಬೇಕು
ಆಗ-ಬೇಕೆಂದು
ಆಗ-ಮಾ-ತ್ರ
ಆಗ-ಲಂತೂ
ಆಗ-ಲಾ-ದರೂ
ಆಗ-ಲಾ-ರದು
ಆಗ-ಲಿ-ಲ್ಲ
ಆಗ-ಲಿದೆ
ಆಗ-ಸ್ಟ್
ಆಗಂತುಕ
ಆಗದಾ-ಗಿದೆ
ಆಗದು
ಆಗದೆ
ಆಗದೇ
ಆಗಬೇ-ಕೇನು
ಆಗಲಾರ-ದ-ವನು
ಆಗಲಿ
ಆಗಲೂ
ಆಗಲೆ
ಆಗಸ
ಆಗಾಗ
ಆಗಾಗ್ಗೆ
ಆಗಾಗ್ಯೆ
ಆಗಿ
ಆಗಿ-ತ್ತು
ಆಗಿ-ತ್ತೇ
ಆಗಿ-ದೆಯೆ
ಆಗಿ-ದ್ದನು
ಆಗಿ-ದ್ದರು
ಆಗಿ-ದ್ದರೆ
ಆಗಿ-ದ್ದವು
ಆಗಿ-ದ್ದಿ-ತೆಂದೂ
ಆಗಿ-ದ್ದೀರಿ
ಆಗಿ-ದ್ದು-ದ-ನ್ನು
ಆಗಿ-ದ್ದು-ದ-ರಿಂದ
ಆಗಿ-ದ್ದೇವೆ
ಆಗಿ-ನ್ನೂ
ಆಗಿ-ಬಿ-ಟ್ಟಿ-ದ್ದೇನೆ
ಆಗಿ-ಬಿಡು-ತ್ತಾರೆ
ಆಗಿ-ಬಿಡು-ತ್ತಿದ್ದರು
ಆಗಿ-ರ-ಬಹು-ದೆಂದು
ಆಗಿ-ರ-ಬಹುದೆ
ಆಗಿ-ರ-ಬೇಕು
ಆಗಿ-ರ-ಲಿ-ಲ್ಲ
ಆಗಿ-ರಲಿ
ಆಗಿ-ರು-ವಂತೆಯೂ
ಆಗಿ-ರು-ವಳು
ಆಗಿ-ರು-ವು-ದಾಗಿ
ಆಗಿ-ರು-ವು-ದಿ-ಲ್ಲ
ಆಗಿ-ರು-ವುದು
ಆಗಿ-ರುವ
ಆಗಿ-ರುವರು
ಆಗಿ-ರುವಾಗ
ಆಗಿ-ರುವೆ
ಆಗಿ-ಲ್ಲ
ಆಗಿ-ಹೋ-ಗಿದೆ
ಆಗಿ-ಹೋ-ದು-ದ-ನ್ನು
ಆಗಿ-ಹೋಗಿ-ತ್ತು
ಆಗಿ-ಹೋಗಿ-ರು-ವೆವು
ಆಗಿ-ಹೋಗು-ತ್ತಿ-ತ್ತು
ಆಗಿ-ಹೋಗೆ-ದೆಯೊ
ಆಗಿ-ಹೋದವು
ಆಗಿಂ-ದಾಗ್ಗೆ
ಆಗಿದ್ದ
ಆಗಿನ
ಆಗಿಯೂ
ಆಗಿವೆ
ಆಗು-ತ್ತದೆ
ಆಗು-ತ್ತಾ
ಆಗು-ತ್ತಾನೆ
ಆಗು-ತ್ತಿ-ತ್ತು
ಆಗು-ತ್ತಿ-ರಲಿ-ಲ್ಲ
ಆಗು-ತ್ತಿ-ರಲಿ-ಲ್ಲವೋ
ಆಗು-ತ್ತಿ-ರು-ವು-ದ-ನ್ನು
ಆಗು-ತ್ತಿದ್ದ
ಆಗು-ತ್ತಿದ್ದನು
ಆಗು-ತ್ತಿದ್ದಾಗ
ಆಗು-ತ್ತಿರುವ
ಆಗು-ತ್ತಿವೆ
ಆಗು-ತ್ತೀರಿ
ಆಗು-ವಂ-ತಿ-ಲ್ಲ
ಆಗು-ವನು
ಆಗು-ವರು
ಆಗು-ವು-ದ-ನ್ನು
ಆಗು-ವು-ದ-ಲ್ಲ
ಆಗು-ವು-ದಕ್ಕೆ
ಆಗು-ವು-ದಾದರೆ
ಆಗು-ವು-ದಿ-ಲ್ಲ
ಆಗು-ವು-ದಿ-ಲ್ಲ-ವ-ಲ್ಲ
ಆಗು-ವು-ದಿ-ಲ್ಲ-ವೆಂದು
ಆಗು-ವು-ದಿ-ಲ್ಲವೆ
ಆಗು-ವುದು
ಆಗು-ವುದೇನು
ಆಗು-ವುದೊ
ಆಗು-ವುವು
ಆಗು-ವೆವು
ಆಗುವ
ಆಗುವ-ವ-ರೆಗೂ
ಆಗುವಿಕೆ
ಆಗುವೆ
ಆಗ್ನೇಯಾ-ಸ್ರ-ಗಳು
ಆಗ್ರ
ಆಗ್ರಾ-ದಿಂದ
ಆಘಾ-ತ-ದಿಂದ
ಆಚ-ಮನಾ-ನಂ-ತರ-ದ-ಲ್ಲಿ
ಆಚೆ
ಆಚೆಗೆ
ಆಚೆಯ
ಆಚ್ಛಾಧಿತ-ವಾದ
ಆಜ್ಞಾ-ವರ್ತಿ-ಗಳು
ಆಜ್ಞೆ
ಆಜ್ಞೆ-ಯನ್ನು
ಆಜ್ಞೆಗೆ
ಆಜ್ಞೋ-ಲ್ಲಂಘ-ನೆಗೆ
ಆಜ್ಯ-ಗ-ಳನ್ನು
ಆಟ
ಆಟ-ಕ್ಕಾಗಿ
ಆಟ-ಗ-ಳನ್ನು
ಆಟ-ಗಳು
ಆಟ-ಗಾರ
ಆಟ-ಗಾರ-ನ-ವನು
ಆಟ-ಗಾರರ
ಆಟ-ದ-ಲ್ಲಿ
ಆಟ-ದಂತೆ
ಆಟ-ಪಾಠ-ಗಳು
ಆಟ-ಪಾಠ-ಗಳೆ-ರಡೂ
ಆಟ-ವನ್ನು
ಆಟ-ವಾ-ದರೂ
ಆಟ-ವೆಂದು
ಆಟಕ್ಕೂ
ಆಟದ
ಆಟವಾ-ಡು-ತ್ತಿದೆವು
ಆಟವಾ-ಡು-ವುದು
ಆಟವಾ-ಡುವ
ಆಟವಾ-ಡುವಾಗ
ಆಟವಾ-ಡುವುದ-ರ-ಲ್ಲಿ
ಆಟವಾಡು-ವನು
ಆಟಾರ್ನಿ
ಆಡ-ಕೂಡ-ದೆಂದು
ಆಡ-ಬೇಕು
ಆಡ-ಬೇಡಿ
ಆಡ-ಲಿ-ಲ್ಲ
ಆಡಂ-ಬ-ರದ
ಆಡಂ-ಬರ-ದಿಂದ
ಆಡಚಣೆ-ಯಾಗು-ವು-ದಿ-ಲ್ಲ
ಆಡಳಿತ
ಆಡಿ
ಆಡಿ-ಟೋ-ರಿಯಂ-ನ-ಲ್ಲಿ
ಆಡಿ-ದ-ಮೇಲೆ
ಆಡಿ-ದರು
ಆಡಿ-ದರೆ
ಆಡಿ-ಸು-ತ್ತಿ-ರು-ವಳು
ಆಡಿದ
ಆಡು-ತ್ತಿ-ರಲಿ-ಲ್ಲ
ಆಡು-ತ್ತಿದ್ದ
ಆಡು-ತ್ತಿದ್ದಾಗ
ಆಡು-ತ್ತಿದ್ದೆ
ಆಡು-ತ್ತಿರು-ತ್ತೀರಿ
ಆಡು-ತ್ತಿರು-ವಳು
ಆಡು-ತ್ತಿರುವ
ಆಡು-ವಂತೆ
ಆಡು-ವನು
ಆಡು-ವು-ದಕ್ಕೆ
ಆಣ-ತಿ-ಯಂತೆ
ಆಣತಿ-ಯನ್ನು
ಆಣೆ-ಗಳ
ಆತ
ಆತ-ನ-ಲ್ಲಿ
ಆತ-ನನ್ನು
ಆತ-ನಿಂದ
ಆತ-ನಿಗೆ
ಆತ-ನೊ-ಡನೆ
ಆತಂ-ಕ-ವಾಗಿ
ಆತಂಕ-ಗ-ಳನ್ನು
ಆತಂಕ-ಗಳಿವೆ
ಆತಂಕ-ಗಳು
ಆತಂಕ-ಗಳೇ
ಆತಂಕ-ವನ್ನು
ಆತಂಕ-ವಿ-ಲ್ಲದ್ದೇ
ಆತಿಥ್ಯ-ವನ್ನು
ಆತುರ-ದಿಂದ
ಆದ
ಆದ-ಕಾರ-ಣವೆ
ಆದ-ಕಾರ-ಣವೇ
ಆದ-ಕಾರಣ
ಆದ-ದ್ದೇನು
ಆದ-ಮೇಲೂ
ಆದ-ಮೇಲೆ
ಆದ-ರಕ್ಕೆ
ಆದ-ರೇನು
ಆದ-ವರ
ಆದಾಯ-ದಿಂದ
ಆದಾಯ-ವೆ-ಲ್ಲವೂ
ಆದಿ
ಆದಿ-ಪಾಪ-ವನ್ನು
ಆದಿ-ಯ-ಲ್ಲೆ
ಆದಿ-ಯನ್ನು
ಆದಿ-ಯಿಂದ
ಆದಿ-ಶಕ್ತಿ-ಯೊಬ್ಬಳೇ
ಆದಿ-ಶಕ್ತಿಯ
ಆದಿಯೇ
ಆದುದ-ನ್ನೆ-ಲ್ಲ
ಆದುವು
ಆದ್ಯ
ಆದ್ಯಂ-ತ-ವಾಗಿ
ಆಧಾ-ತ್ಮಿಕ
ಆಧಾ-ರದ
ಆಧಾ-ರವೂ
ಆಧಾರ
ಆಧಿಕ್ಯ-ದಿಂದ
ಆಧುನಿಕ
ಆಧುನಿಕ-ರ-ಲ್ಲಿ
ಆಧುನಿಕ-ರಿಗೆ
ಆಧುನೈನ
ಆಧ್ಯಾ
ಆಧ್ಯಾ-ತ್ಮ
ಆಧ್ಯಾ-ತ್ಮ-ದಿಂದ
ಆಧ್ಯಾ-ತ್ಮ-ದೊಂದಿಗೆ
ಆಧ್ಯಾ-ತ್ಮ-ವನ್ನು
ಆಧ್ಯಾ-ತ್ಮ-ವಿದ್ಯೆ
ಆಧ್ಯಾ-ತ್ಮ-ವೋದೇ
ಆಧ್ಯಾ-ತ್ಮಿಕ
ಆಧ್ಯಾ-ತ್ಮಿಕ-ಕ್ಕಿಂತ
ಆಧ್ಯಾ-ತ್ಮಿಕ-ಜೀವಿ-ಗಳು
ಆಧ್ಯಾ-ತ್ಮಿಕ-ತೆ-ಯನ್ನು
ಆಧ್ಯಾ-ತ್ಮಿಕ-ತೆ-ಯಿಂದ
ಆಧ್ಯಾ-ತ್ಮಿಕ-ತೆ-ಯೆ-ಲ್ಲ
ಆಧ್ಯಾ-ತ್ಮಿಕ-ತೆಯ
ಆಧ್ಯಾ-ತ್ಮಿಕತೆ
ಆನಂ-ದಕ್ಕೆ
ಆನಂ-ದಿ-ಸಿದರು
ಆನಿ-ಸ್ಕ್ವಾಂ-ಮ್
ಆನಿ-ಸ್ಕ್ವಾಂ-ಮ್ನ-ಲ್ಲಿ
ಆನೆ
ಆನೆ-ಗಳು
ಆನೆ-ಯ-ನ್ನಾ-ದರೂ
ಆನೆ-ಯನ್ನು
ಆನೆ-ಯನ್ನೆ
ಆನೆ-ಯೊಂದು
ಆಪ-ತ್ಕಾಲ-ದ-ಲ್ಲಿ
ಆಪಾದಮ-ಸ್ತ-ಕದ-ವ-ರೆಗೆ
ಆಪೂರಣ-ದಿಂದ
ಆಪ್
ಆಪ್ತ
ಆಪ್ತ-ಕಾರ್ಯ-ದರ್ಶಿ-ಗ-ಳಾದ
ಆಪ್ತ-ರಿಗೆ
ಆಫೀ-ಸಿಗೆ
ಆಫೀಸಿ-ನ-ಲ್ಲಿ
ಆಫೀಸಿ-ನಿಂದ
ಆಭರಣ-ಗಳಿಂದ
ಆಮದು
ಆಮೀರ್
ಆಮೆ
ಆಮೆ-ಯನ್ನು
ಆಮೆ-ಯಾಗಿ
ಆಮೋದ
ಆಯಾ
ಆಯಾ-ಸ-ಗೊಂಡಿ-ದ್ದರು
ಆಯಾ-ಸ-ಗೊಂಡಿ-ರು-ವು-ದ-ರಿಂದ
ಆಯಾ-ಸ-ಗೊಂಡಿದೆ
ಆಯಾ-ಸ-ವಾಗಲಿ
ಆಯಾ-ಸ-ವಾಗು-ವುದೋ
ಆಯಾಯಾ
ಆಯಿತ-ಲ್ಲ
ಆಯುಧ
ಆಯುಧ-ದಿಂದ
ಆಯುಧ-ವನ್ನು
ಆಯುಧ-ವಾದ
ಆಯುರ್ವೇದ
ಆಯುಷ್ಯ-ವನ್ನೆ-ಲ್ಲ
ಆಯ್ಕೆ
ಆಯ್ದ
ಆರ-ನೆಯ
ಆರಂಭ-ಮಾಡು-ವು-ದಕ್ಕೆ
ಆರಂಭ-ವಾ-ದವು
ಆರಂಭ-ವಾ-ಯಿತು
ಆರಂಭ-ವಾಗು-ವುದು
ಆರಂಭಿ-ಸಿದ
ಆರಂಭಿ-ಸಿದೆ
ಆರಂಭಿಸಿ
ಆರಂಭಿಸಿ-ದನೊ
ಆರಂಭಿಸಿ-ದರು
ಆರಂಭಿಸಿ-ದರೋ
ಆರತಿ-ಗಾಗಿ
ಆರಾಧ-ನೆ-ಯನ್ನು
ಆರಾಧ-ನೆ-ಯಿಂದ
ಆರಾಧ-ನೆಯ
ಆರಾಧಿ-ಸಲು
ಆರಾಧಿ-ಸಿದ
ಆರಾಧಿ-ಸು-ತ್ತೀರಿ
ಆರಾಧಿಸಿ
ಆರಾವಳಿ
ಆರಿ-ಸು-ತ್ತದೆ
ಆರಿ-ಸು-ತ್ತಾನೆ
ಆರಿ-ಹೋಗಿ-ದ್ದರೆ
ಆರು
ಆರೇಳು
ಆರೈಕೆ-ಯ-ಲ್ಲಿ
ಆರೋಪ-ಮಾಡಿ
ಆರೋಪ-ಮಾಡಿ-ಕೊಂಡು
ಆರೋಪಿಸಿ-ಕೊಂಡಿದ್ದರು
ಆರೋಪಿಸು-ವುದೂ
ಆರ್
ಆರ್ಕಿಯಾಲಜಿ
ಆರ್ಟ
ಆರ್ತರೂ
ಆರ್ಥಿಕ
ಆರ್ಥಿಕ-ದೃಷ್ಟಿ-ಯಿಂದ
ಆರ್ನೋ
ಆರ್ಬಿ-ಸ್ನೋಡ-ನ್
ಆರ್ಭಟಿ-ಸು-ವುದು
ಆರ್ಯ
ಆರ್ಯ-ಮಹರ್ಷಿ-ಗಳ
ಆರ್ಯ-ರದು
ಆರ್ಯ-ರಿಗೆ
ಆರ್ಯ-ರು-ಗಳ
ಆರ್ಯ-ವೈಶ್ಯ
ಆರ್ಯ-ಸಮಾಜ
ಆರ್ಯ-ಸಮಾಜ-ದ-ಲ್ಲಿ
ಆರ್ಯ-ಸಮಾಜ-ದ-ಲ್ಲಿ-ರುವ
ಆರ್ಯ-ಸಮಾಜ-ದ-ವರು
ಆರ್ಯ-ಸಮಾಜದ
ಆರ್ಯರ
ಆರ್ಯರು
ಆರ್ಯಾ-ವರ್ತದ
ಆರ್ಯಾ-ವರ್ತದ-ಲ್ಲಿ
ಆಲ-ಸ್ಯ-ದಿಂದ
ಆಲಂ-ಬ-ಜಾರಿ-ನ-ಲ್ಲಿದ್ದ
ಆಲಂ-ಬ-ಜಾರಿ-ನಿಂದ
ಆಲದ
ಆಲಯ
ಆಲಯ-ದ-ಲ್ಲಿ
ಆಲಯಕ್ಕೆ
ಆಲಾಪ-ನೆಯೂ
ಆಲಿ
ಆಲಿ-ಸ-ಬ-ಲ್ಲ
ಆಲಿ-ಸಿದ
ಆಲಿ-ಸು-ತ್ತಿ-ದ್ದರೆ
ಆಲಿ-ಸು-ತ್ತಿರಿ
ಆಲೂ-ಗೆಡ್ಡೆ-ಯನ್ನು
ಆಲೂಗಡ್ಡೆ
ಆಲೋ
ಆಲೋ-ಚ-ನೆಯೂ
ಆಲೋ-ಚ-ನೆಯೆ
ಆಲೋ-ಚ-ನೆಯೇ
ಆಲೋ-ಚನಾ
ಆಲೋ-ಚನಾ-ಪ-ರರು
ಆಲೋ-ಚನಾ-ಪರ-ರ-ನ್ನಾಗಿ
ಆಲೋ-ಚನಾ-ಪರ-ರಿಗೆ
ಆಲೋ-ಚನಾ-ಪ್ರವಾಹ
ಆಲೋ-ಚನಾ-ಮಗ್ನ-ನಾಗಿ
ಆಲೋ-ಚನೆ
ಆಲೋ-ಚನೆ-ಗ-ಳನ್ನು
ಆಲೋ-ಚನೆ-ಗಳು
ಆಲೋ-ಚನೆ-ಗಳೇ
ಆಲೋ-ಚನೆ-ಗೆ-ಲ್ಲ
ಆಲೋ-ಚನೆ-ಯ-ಲ್ಲಿ
ಆಲೋ-ಚನೆ-ಯ-ಲ್ಲಿಯೋ
ಆಲೋ-ಚನೆ-ಯನ್ನು
ಆಲೋ-ಚಿ-ಸದೆ
ಆಲೋ-ಚಿ-ಸಲಿ
ಆಲೋ-ಚಿ-ಸು-ತ್ತ
ಆಲೋ-ಚಿಸ-ತೊಡಗಿದರು
ಆಲೋ-ಚಿಸ-ತೊಡಗಿದೆ
ಆಲೋ-ಚಿಸ-ಬಹುದು
ಆಲೋ-ಚಿಸ-ಬೇಕು
ಆಲೋ-ಚಿಸಿ
ಆಲೋ-ಚಿಸಿ-ದಂತೆ
ಆಲೋ-ಚಿಸಿ-ದರು
ಆಲೋ-ಚಿಸಿ-ದ್ದರೊ
ಆಲೋ-ಚಿಸಿ-ದ್ದೆವು
ಆಲೋ-ಚಿಸಿದ
ಆಲೋ-ಚಿಸಿದೆ
ಆಲೋ-ಚಿಸಿಯೇ
ಆಲೋ-ಚಿಸು
ಆಲೋ-ಚಿಸು-ತ್ತಿ-ಲ್ಲ-ವಾ-ದರೂ
ಆಲೋ-ಚಿಸು-ತ್ತಿದ್ದರು
ಆಲೋ-ಚಿಸು-ತ್ತಿದ್ದೆ
ಆಲೋ-ಚಿಸು-ತ್ತಿರು-ವಂತೆ
ಆಲೋ-ಚಿಸು-ವ-ದಕ್ಕೆ
ಆಲೋ-ಚಿಸು-ವಂತೆ
ಆಲೋ-ಚಿಸು-ವನೊ
ಆಲೋ-ಚಿಸು-ವಿರೋ
ಆಲೋ-ಚಿಸು-ವುದು
ಆಲೋ-ಚಿಸುವ
ಆಳ-ಆಳಕ್ಕೆ
ಆಳ-ದ-ಲ್ಲಿ
ಆಳ-ವನ್ನು
ಆಳ-ವಾ-ಗಿದೆ
ಆಳ-ವಾಗಿ
ಆಳ-ವಾಗಿ-ತ್ತು
ಆಳ-ವಾದ
ಆಳಕ್ಕೆ
ಆಳಲು
ಆಳಸಿಂಗ
ಆಳು
ಆಳು-ಕಾಳು-ಗಳು
ಆಳು-ಗಳು
ಆಳು-ಗಳೊ-ಡನೆ
ಆಳು-ತ್ತಿರು-ವ-ವರ
ಆಳು-ತ್ತಿರು-ವನು
ಆಳು-ತ್ತಿರುವ
ಆಳು-ವಂತೆ
ಆಳ್ವಾ-ರಿ-ನಿಂದ
ಆಳ್ವಾ-ರಿಗೆ
ಆಳ್ವಾರಿ-ನ-ಲ್ಲಿ
ಆಳ್ವಾರಿ-ನ-ಲ್ಲೆ-ಲ್ಲಾ
ಆಳ್ವಾರಿನ
ಆಳ್ವಿ-ಕೆಯ
ಆಳ್ವಿಕೆ
ಆಳ್ವಿಕೆ-ಯ-ನ್ನೆ-ಲ್ಲ
ಆಳ್ವಿಕೆ-ಯ-ಲ್ಲಿರು-ವಾಗ
ಆವ-ಶ್ಯಕ
ಆವಶ್ಯ
ಆವಶ್ಯ-ಕ-ತೆ-ಯನ್ನು
ಆವಶ್ಯ-ಕ-ತೆ-ಯೇ-ನಿ-ಲ್ಲ
ಆವಶ್ಯ-ಕ-ವಾಗಿ
ಆವಶ್ಯ-ಕ-ವಿ-ರು-ವು-ದ-ರಿಂದ
ಆವಶ್ಯ-ಕ-ವಿದ್ದಾಗ
ಆವಶ್ಯ-ಕ-ವೇನೋ
ಆವಶ್ಯ-ಕತೆ
ಆವಾಹನೆ
ಆವಿ
ಆವಿರ್ಭವಿಸ-ಲಿ-ರುವ
ಆವಿರ್ಭಾವ
ಆವಿರ್ಭಾವ-ಗ-ಳನ್ನು
ಆವಿರ್ಭಾವ-ವಾಗು-ವುದು
ಆವಿರ್ಭಾವ-ವಿದೆ
ಆವಿರ್ಭೂತ-ವಾ-ಗಿದೆ
ಆವೃ-ತ-ವಾಗಿ-ತ್ತು
ಆವೃತ-ರಾಗಿ-ರುವಿರಿ
ಆವೃತ-ವಾದ
ಆಶಾ-ದಾಯಕ-ವಾ-ಗಿದೆ
ಆಶಿ-ಸಲೂ
ಆಶಿ-ಸು-ತ್ತಿ-ದ್ದರು
ಆಶಿ-ಸು-ತ್ತೇನೆ
ಆಶಿ-ಸು-ವು-ದ-ರಿಂದ
ಆಶಿ-ಸು-ವುದು
ಆಶಿ-ಸುವರು
ಆಶಿಷ್ಠನೂ
ಆಶಿಸಿ
ಆಶಿಸಿ-ದರು
ಆಶಿಸಿ-ದಾಗ
ಆಶಿಸಿ-ದೆವು
ಆಶಿಸಿ-ದ್ದರು
ಆಶಿಸಿ-ದ್ದರೋ
ಆಶೀರ್ವ-ಚನ-ವನ್ನು
ಆಶೀರ್ವ-ದಿಸು-ವು-ವರು
ಆಶೀರ್ವದಿ-ಸಲಿ
ಆಶೀರ್ವದಿ-ಸು-ತ್ತ
ಆಶೀರ್ವದಿಸಿ
ಆಶೀರ್ವದಿಸಿ-ದರು
ಆಶೀರ್ವದಿಸಿ-ದರೆ
ಆಶೀರ್ವದಿಸಿ-ದಿ-ದರು
ಆಶೀರ್ವಾದ
ಆಶೀರ್ವಾದ-ಗಳೂ
ಆಶೀರ್ವಾದ-ಗಳೇ
ಆಶೀರ್ವಾದ-ದಂತೆ
ಆಶೀರ್ವಾದ-ದಿಂದ
ಆಶೀರ್ವಾದ-ವನ್ನು
ಆಶ್ಚರ್ಯ
ಆಶ್ಚರ್ಯ-ಕ-ರವೆ-ನಿ-ಸುವ
ಆಶ್ಚರ್ಯ-ಕರ-ವಾಗಿ
ಆಶ್ಚರ್ಯ-ಕರ-ವಾಗು-ವಂತೆ
ಆಶ್ಚರ್ಯ-ಕರ-ವಾದ
ಆಶ್ಚರ್ಯ-ಚಕಿ-ತ-ನಾದರು
ಆಶ್ಚರ್ಯ-ಚಕಿತ-ನಾಗಿ
ಆಶ್ಚರ್ಯ-ಚಕಿತ-ನಾಗಿ-ರು-ವೆನು
ಆಶ್ಚರ್ಯ-ದಿಂದ
ಆಶ್ಚರ್ಯ-ಪ-ಟ್ಟರು
ಆಶ್ಚರ್ಯ-ಭರಿತ-ವಾ-ಗಿದೆ
ಆಶ್ಚರ್ಯ-ವಾ-ಗಿದೆ
ಆಶ್ಚರ್ಯ-ವಾ-ಯಿತು
ಆಶ್ಚರ್ಯ-ವಾಗ-ಬಹುದು
ಆಶ್ಚರ್ಯ-ವಾಗಿ
ಆಶ್ಚರ್ಯ-ವಾಗು-ತ್ತದೆ
ಆಶ್ಚರ್ಯ-ವಾಗು-ವುದು
ಆಶ್ಚರ್ಯ-ವೇ-ನೆಂ-ದರೆ
ಆಶ್ಚರ್ಯ-ವೇನೂ
ಆಶ್ಚರ್ಯವೇ
ಆಶ್ರಯ
ಆಶ್ರಯ-ದ-ಲ್ಲಿ
ಆಶ್ರಯ-ವನ್ನು
ಆಶ್ರಯ-ವಿ-ತ್ತು
ಆಶ್ರಯಿಸಿ
ಆಶ್ರಯಿಸಿ-ದ್ದರು
ಆಶ್ರಿತ
ಆಶ್ರಿತ-ನಾಗು-ವುದು
ಆಷಾಢ-ಭೂತಿ-ಗಳೆಂದೂ
ಆಷಾಢ-ಭೂತಿ-ಗಳೇ
ಆಷಾಢ-ಭೂತಿ-ತನ
ಆಷ್ಚರ್ಯ-ವಾ-ಯಿತು
ಆಸ-ನದ
ಆಸಕ್ತ-ನಾಗಿ
ಆಸಕ್ತ-ರಾದ
ಆಸಕ್ತ-ರಾದ-ವ-ರಿಗೆ
ಆಸಕ್ತ-ರಾದ-ವರು
ಆಸಕ್ತಿ
ಆಸಕ್ತಿ-ಯಂತೆ
ಆಸಕ್ತಿ-ಯನ್ನು
ಆಸಕ್ತಿ-ಯಿ-ತ್ತು
ಆಸಕ್ತಿ-ಯಿಂದ
ಆಸಕ್ತಿ-ಯಿಂದಲೂ
ಆಸಕ್ತಿ-ಯಿದೆಯೋ
ಆಸಕ್ತಿಯೂ
ಆಸರೆ
ಆಸಿಡ್
ಆಸೆ
ಆಸೆ-ಗ-ಳನ್ನು
ಆಸೆ-ಗಳ-ನ್ನೆ-ಲ್ಲ
ಆಸೆ-ಗಳಿ-ರುವ-ವ-ರೆಗೂ
ಆಸೆ-ಯನ್ನು
ಆಸೆ-ಯಾಗುವುದು
ಆಸೆ-ಯಿ-ಲ್ಲ
ಆಸೆ-ಯಿದೆ
ಆಸೆ-ಯೆಂ-ದರೆ
ಆಸೆ-ಯೊಂದು
ಆಸೆಗೂ
ಆಸೆಗೆ
ಆಸೆಯ
ಆಹುತಿ
ಆಹುತಿ-ಗಳ-ನ್ನೂ
ಆಹ್ಲಾದ-ಕರ-ವಾಗಿ
ಆಹ್ಲಾದ-ಕರ-ವಾಗಿ-ದ್ದವು
ಆಹ್ವಾನ-ವನ್ನು
ಆಹ್ವಾನಿಸಿ
ಆಹ್ವಾನಿಸಿ-ದರು
ಇ
ಇಂಇ
ಇಂಗ-ಬೇಕು
ಇಂಗೀಷ-ನ್ನು
ಇಂಗ್ಲಿಷಿನವ
ಇಂಗ್ಲಿಷ್
ಇಂಗ್ಲಿಷ್-ಬ್ಯಾಂಡ್
ಇಂಗ್ಲೀಷ-ನ್ನು
ಇಂಗ್ಲೀಷ-ರಾಗಲಿ
ಇಂಗ್ಲೀಷರ
ಇಂಗ್ಲೀಷರು
ಇಂಗ್ಲೀಷಿ-ನ-ಲ್ಲಿ
ಇಂಗ್ಲೀಷಿಗೆ
ಇಂಗ್ಲೀಷಿನ
ಇಂಗ್ಲೀಷಿನ-ಲ್ಲಿಯೂ
ಇಂಗ್ಲೀಷಿನ-ಲ್ಲಿಯೇ
ಇಂಗ್ಲೀಷಿನ-ವ-ರ-ನ್ನು
ಇಂಗ್ಲೀಷಿನ-ವ-ರ-ಲ್ಲಿ
ಇಂಗ್ಲೀಷಿನ-ವ-ರಾ-ದರೊ
ಇಂಗ್ಲೀಷಿನ-ವ-ರಿಂದ
ಇಂಗ್ಲೀಷಿನ-ವ-ರಿಗೆ
ಇಂಗ್ಲೀಷಿನ-ವರ
ಇಂಗ್ಲೀಷಿನ-ವರ-ಲ್ಲಿನ
ಇಂಗ್ಲೀಷಿನ-ವರು
ಇಂಗ್ಲೀಷ್
ಇಂಗ್ಲೀಷ್ನ-ಲ್ಲಿ
ಇಂಗ್ಲೆಂಡ-ನ್ನು
ಇಂಗ್ಲೆಂಡಾಗಲಿ
ಇಂಗ್ಲೆಂಡಿ-ನ-ಲ್ಲಿ
ಇಂಗ್ಲೆಂಡಿ-ನಿಂದ
ಇಂಗ್ಲೆಂಡಿಗೆ
ಇಂಗ್ಲೆಂಡಿನ
ಇಂಗ್ಲೆಂಡಿನ-ಲ್ಲಿ-ದ್ದಾಗ
ಇಂಗ್ಲೆಂಡಿನ-ಲ್ಲಿದೆ
ಇಂಗ್ಲೆಂಡಿನ-ಲ್ಲಿಯೇ
ಇಂಗ್ಲೆಂಡಿನ-ವ-ರೆಗೆ
ಇಂಗ್ಲೆಂಡಿನ-ವರು
ಇಂಗ್ಲೆಂಡು-ಗಳ
ಇಂಗ್ಲೆಂಡು-ಗಳಿಂದ
ಇಂಗ್ಲೆಂಡ್
ಇಂಚರ
ಇಂಜಿನಿ-ಯರ-ನನ್ನು
ಇಂಜಿನಿ-ಯರ್
ಇಂಜಿನೀ-ಯರ್
ಇಂಡಿ-ಯಾದ
ಇಂಡಿಯ
ಇಂಡಿಯ-ನ್ನ-ರ-ನ್ನು
ಇಂಡಿಯಾ
ಇಂಡಿಯಾ-ದ-ಲ್ಲಿ
ಇಂಡಿಯಾ-ದೇ-ಶಕ್ಕೆ
ಇಂಡಿಯಾ-ದೇಶ-ದ-ಲ್ಲಿ
ಇಂಡಿಯಾ-ದೇಶ-ದ-ವ-ರಿಗಾಗಲಿ
ಇಂಡಿಯಾ-ದೇಶ-ದಿಂದ
ಇಂಡಿಯಾ-ದೇಶ-ವನ್ನು
ಇಂಡಿಯಾ-ದೇಶದ
ಇಂತಹ
ಇಂತಹ-ವ-ರಿಂದ
ಇಂತಹ-ವನು
ಇಂತಹ-ವರ
ಇಂತಹ-ವರು
ಇಂಥ
ಇಂಥ-ವ-ರಿಂದ
ಇಂಥ-ವ-ರಿಗೆ
ಇಂಥ-ವನು
ಇಂಥಾ-ದ-ನ್ನು
ಇಂಥಾ-ದ್ದ-ನ್ನು
ಇಂದ
ಇಂದಿ-ಗಿಂತಲೂ
ಇಂದಿ-ನಿಂದ
ಇಂದಿಗೂ
ಇಂದಿಗೆ
ಇಂದು
ಇಂದೂ
ಇಂದೇ
ಇಂದೊ
ಇಂದೋ
ಇಂದೋರ್
ಇಂದ್ರ
ಇಂದ್ರ-ಜಾಲ
ಇಂದ್ರ-ಜಿ-ತ್ತು
ಇಂದ್ರನು
ಇಂದ್ರಿಯ
ಇಂದ್ರಿಯ-ಗ-ಳನ್ನು
ಇಂದ್ರಿಯ-ಗಳ-ಲ್ಲಿ
ಇಂದ್ರಿಯ-ಗಳು
ಇಂದ್ರಿಯಾ-ತೀತ
ಇಂದ್ರಿಯಾ-ತೀತ-ವಾದ
ಇಂಪಾ-ಗಿದೆ
ಇಂಪಾಗಿ-ತ್ತು-ಅ-ದೊಂದು
ಇಂಪಿನ-ಲ್ಲೆ
ಇಕ-ಬೀರ್
ಇಕ್ಕ-ಟ್ಟು
ಇಗಿನ
ಇಚ್ಛಾ-ಮರ-ಣ-ವನ್ನು
ಇಚ್ಛಾ-ಮಾ-ತ್ರ-ದಿಂದ
ಇಚ್ಛಾ-ಶಕ್ತಿ
ಇಚ್ಛಾ-ಶಕ್ತಿ-ಯನ್ನು
ಇಚ್ಛಾ-ಶಕ್ತಿಯ
ಇಚ್ಛಾನು-ಸಾರ
ಇಚ್ಛಿ-ಸ-ಲಿ-ಲ್ಲ
ಇಚ್ಛಿ-ಸದೆ
ಇಚ್ಛಿ-ಸದೇ
ಇಚ್ಛಿ-ಸಿ-ದನು
ಇಚ್ಛಿ-ಸಿ-ದರು
ಇಚ್ಛಿ-ಸಿ-ದರೂ
ಇಚ್ಛಿ-ಸಿ-ದರೆ
ಇಚ್ಛಿ-ಸಿದ
ಇಚ್ಛಿ-ಸು-ವು-ದಿ-ಲ್ಲ
ಇಚ್ಛಿ-ಸು-ವೆನು
ಇಚ್ಛಿ-ಸುವ
ಇಚ್ಛಿ-ಸುವರು
ಇಚ್ಛಿಸ-ಬಹುದು
ಇಚ್ಛಿಸಿ-ದಾಗ
ಇಚ್ಛಿಸಿ-ದು-ದ-ರಿಂದ
ಇಚ್ಛಿಸು-ವೆಯೋ
ಇಚ್ಛೆ
ಇಚ್ಛೆ-ಪ-ಟ್ಟರೆ
ಇಚ್ಛೆ-ಪಟ್ಟ-ಲ್ಲಿ
ಇಚ್ಛೆ-ಪಡ-ಲಿ-ಲ್ಲ
ಇಚ್ಛೆ-ಪಡು-ವು-ದ-ರಿಂದ
ಇಚ್ಛೆ-ಯಂತೆ
ಇಚ್ಛೆ-ಯಂತೆಯೆ
ಇಚ್ಛೆ-ಯನ್ನು
ಇಚ್ಛೆ-ಯಾ-ದರೆ
ಇಚ್ಛೆ-ಯಾಗ-ಲಿ-ಲ್ಲ
ಇಚ್ಛೆ-ಯಾಗಿ-ತ್ತು
ಇಚ್ಛೆ-ಯಾದ
ಇಚ್ಛೆ-ಯಿ-ದ್ದರೆ
ಇಚ್ಛೆ-ಯಿ-ರ-ಲಿ-ಲ್ಲ
ಇಚ್ಛೆ-ಯಿ-ಲ್ಲ
ಇಚ್ಛೆ-ಯಿ-ಲ್ಲದೆ
ಇಚ್ಛೆ-ಯಿ-ಲ್ಲದೇ
ಇಚ್ಛೆ-ಯಿಂದ
ಇಚ್ಛೆ-ಯಿಂದಲೇ
ಇಚ್ಛೆ-ಯಿಂದಾದುದು
ಇಚ್ಛೆ-ಯುಂಟಾಗು-ತ್ತದೆ
ಇಚ್ಛೆ-ಯೆ-ಲ್ಲ
ಇಚ್ಛೆಗೆ
ಇಚ್ಛೆಯ
ಇಚ್ಛೆಯೇ
ಇಟ-ಲಿಯ
ಇಟಲಿ
ಇಟಿ
ಇಡ
ಇಡ-ಲ್ಪಟ್ಟಿ-ದ್ದವು
ಇಡಲಾರೆ
ಇಡಲು
ಇಡಿ
ಇಡಿಯ
ಇಡೀ
ಇಡು
ಇಡು-ತ್ತಿ-ತ್ತು
ಇಡು-ತ್ತಿದ್ದರು
ಇಡು-ತ್ತಿದ್ದಳು
ಇಡು-ತ್ತಿರು-ವಳು
ಇಡು-ವಂತೆ
ಇಡು-ವರು
ಇಡು-ವು-ದಿ-ಲ್ಲ
ಇಡು-ವುದು
ಇಡು-ವುದೋ
ಇಡು-ವೆನು
ಇಡೋಣ
ಇಣಕಿ
ಇತ-ರರ
ಇತ-ರರು
ಇತಿ-ಹಾಸ
ಇತಿ-ಹಾಸ-ಗಳ
ಇತಿ-ಹಾಸ-ದ-ಲ್ಲಿ
ಇತಿ-ಹಾಸ-ವೆ-ಲ್ಲ
ಇತಿ-ಹಾಸಕ್ಕೆ
ಇತಿ-ಹಾಸದ
ಇದ-ಕ್ಕೆ-ಲ್ಲ
ಇದ-ಕ್ಕೋ-ಸ್ಕರ-ವಾಗಿಯೇ
ಇದ-ನ್ನರಿತ
ಇದ-ನ್ನೆ-ಲ್ಲ
ಇದ-ನ್ನೆ-ಲ್ಲಾ
ಇದ-ರ-ಲ್ಲಿ
ಇದ-ರ-ಲ್ಲಿ-ರುವ
ಇದ-ರ-ಲ್ಲೆ
ಇದ-ರಂತೆ
ಇದ-ರಂತೆಯೆ
ಇದ-ರಂತೆಯೇ
ಇದ-ರಷ್ಟು
ಇದ-ರಿಂದಲೇ
ಇದ-ರೊ-ಡನೆ
ಇದ-ರೊಂದಿಗೆ
ಇದ-ಲ್ಲದೇ
ಇದಕ್ಕಿಂತ
ಇದಕ್ಕೇ
ಇದು
ಇದು-ವ-ರೆಗೂ
ಇದು-ವ-ರೆಗೆ
ಇದು-ವರೆ-ವಿಗೂ
ಇದು-ವರೆಗು
ಇದೂ
ಇದೆ
ಇದೆ-ಯ-ಲ್ಲ
ಇದೆ-ಯೆ-ಇದೆ-ಯೇನು
ಇದೆ-ಲ್ಲವೂ
ಇದೆಂತಹ
ಇದೇ
ಇದೇನು
ಇದೇನೋ
ಇದೊಂ-ದರ
ಇದೊಂದೇ
ಇದ್ದ
ಇದ್ದ-ಕಡೆ
ಇದ್ದ-ಕಡೆಯೇ
ಇದ್ದ-ಕ್ಕಿ-ದ್ದಂತೆ
ಇದ್ದ-ಕ್ಕಿ-ದ್ದಂತೆಯೇ
ಇದ್ದ-ದ್ದ-ರಿಂದ
ಇದ್ದ-ನ್ನೇ
ಇದ್ದ-ರೆಂದು
ಇದ್ದ-ರೆಷ್ಟು
ಇದ್ದ-ರೇ-ನಂತೆ
ಇದ್ದ-ರೇನು
ಇದ್ದ-ವ-ನ-ಲ್ಲ
ಇದ್ದ-ವನು
ಇದ್ದ-ವರ
ಇದ್ದ-ವರು
ಇದ್ದ-ವಳು
ಇದ್ದ-ಹಾಗೆ
ಇದ್ದ-ಹಾಗೆಯೇ
ಇದ್ದಂತಹ
ಇದ್ದದ್ದು
ಇದ್ದನೆ
ಇದ್ದಷ್ಟು
ಇದ್ದಿತೊ
ಇದ್ದಿರ-ಬಹುದು
ಇದ್ದೀತು
ಇದ್ದು
ಇದ್ದು-ಕೊಂಡು
ಇದ್ದು-ದ-ಕ್ಕಾಗಿ
ಇದ್ದು-ದ-ನ್ನು
ಇದ್ದು-ದ-ರಿಂದ
ಇದ್ದು-ಬಿಡಿ
ಇದ್ದು-ಬಿಡು
ಇದ್ದು-ಬಿಡು-ವು-ದ-ರಿಂದ
ಇದ್ದು-ರಿಂದ
ಇದ್ದುವೋ
ಇದ್ದೆ
ಇದ್ದೆ-ವೆಂದೂ
ಇದ್ದೇ
ಇಪ್ಪ-ತ್ತ-ನಾ-ಲ್ಕು
ಇಪ್ಪ-ತ್ತ-ನೆಯ
ಇಪ್ಪ-ತ್ತನೆ
ಇಪ್ಪ-ತ್ತನೇ
ಇಪ್ಪ-ತ್ತು
ಇಪ್ಪ-ತ್ತೈದ-ನೆಯ
ಇಪ್ಪ-ತ್ತೈದು
ಇಬ್ಬ-ರಿಗೂ
ಇಬ್ಬ-ರಿಗೆ
ಇಬ್ಬರ
ಇಬ್ಬರು
ಇಬ್ಬರೂ
ಇಬ್ಭಾಗ-ವಾಗಿ
ಇರ-ಕೂ-ಡದು
ಇರ-ಕೂಡ-ದೆಂದು
ಇರ-ದಿದ್ದರೆ
ಇರ-ಬ-ಲ್ಲ-ವ-ನಾ-ದರೆ
ಇರ-ಬ-ಲ್ಲದು
ಇರ-ಬ-ಲ್ಲನೆ
ಇರ-ಬಹು-ದೆಂದು
ಇರ-ಬಹುದು
ಇರ-ಬೇ-ಕಾ-ದರೆ
ಇರ-ಬೇಕಾ-ಯಿತು
ಇರ-ಬೇಕಾಗಿ-ತ್ತು
ಇರ-ಬೇಕಾಗಿ-ಲ್ಲ
ಇರ-ಬೇಕಾಗಿದೆ
ಇರ-ಬೇಕಾಗು-ತ್ತದೆ
ಇರ-ಬೇಕಾಗುವುದು
ಇರ-ಬೇಕು
ಇರ-ಬೇಕೆ
ಇರ-ಬೇಕೆಂದು
ಇರ-ಬೇಕೆಂಬ
ಇರ-ವನ್ನೇ
ಇರಬೆಕಾಗಿ-ತ್ತು
ಇರಬೆಕು
ಇರಲು
ಇರಲೇ
ಇರಲೇ-ಬೇಕು
ಇರಿ
ಇರಿ-ಸ-ಬಹುದು
ಇರು
ಇರು-ತ್ತ-ದ-ಲ್ಲ
ಇರು-ತ್ತದೆ
ಇರು-ತ್ತದೆಯೊ
ಇರು-ತ್ತವೆ
ಇರು-ತ್ತಾನೆ
ಇರು-ತ್ತಾರೊ
ಇರು-ತ್ತಿ-ತ್ತು
ಇರು-ತ್ತಿ-ರಲಿ-ಲ್ಲ
ಇರು-ತ್ತಿದ್ದ
ಇರು-ತ್ತಿದ್ದರು
ಇರು-ತ್ತಿದ್ದವು
ಇರು-ತ್ತಿದ್ದಿರಿ
ಇರು-ತ್ತಿದ್ದುವು
ಇರು-ತ್ತಿದ್ದೆ
ಇರು-ತ್ತೇನೆ
ಇರು-ತ್ತೇವೆ
ಇರು-ವ-ರೆಂದು
ಇರು-ವ-ರೆಂದೂ
ಇರು-ವ-ರೆಂಬು-ದ-ನ್ನು
ಇರು-ವ-ವ-ರ-ನ್ನು
ಇರು-ವ-ವ-ರಿ-ಗೆ-ಲ್ಲ
ಇರು-ವ-ವ-ರಿಗೂ
ಇರು-ವ-ವ-ರಿಗೆ
ಇರು-ವ-ವ-ರೆಗೆ
ಇರು-ವ-ವನು
ಇರು-ವ-ವನೇ
ಇರು-ವ-ವರು
ಇರು-ವ-ಹಾಗಿ-ಲ್ಲ
ಇರು-ವಂತಾಗಲಿ
ಇರು-ವಂತೆ
ಇರು-ವಂತೆಯೇ
ಇರು-ವಂಥಾ
ಇರು-ವದು
ಇರು-ವನು
ಇರು-ವನೆ
ಇರು-ವರು
ಇರು-ವರೆ
ಇರು-ವರೋ
ಇರು-ವಳು
ಇರು-ವಷ್ಟೇ
ಇರು-ವಾಗ
ಇರು-ವಾಗಲೂ
ಇರು-ವಾಗಲೇ
ಇರು-ವಿ-ರೆಂದು
ಇರು-ವಿರಾ
ಇರು-ವಿರಿ
ಇರು-ವು-ದ-ನ್ನು
ಇರು-ವು-ದ-ನ್ನೇ
ಇರು-ವು-ದ-ರಿಂದ
ಇರು-ವು-ದ-ರಿಂದಲೇ
ಇರು-ವು-ದ-ಲ್ಲ
ಇರು-ವು-ದಕ್ಕೆ
ಇರು-ವು-ದಾಗಿ
ಇರು-ವು-ದಾಗಿಯೂ
ಇರು-ವು-ದಾದರೆ
ಇರು-ವು-ದಿ-ಲ್ಲ
ಇರು-ವು-ದಿ-ಲ್ಲವೋ
ಇರು-ವು-ದೆ-ಲ್ಲ
ಇರು-ವುದಕ್ಕಿಂತ
ಇರು-ವುದು
ಇರು-ವುದೆ
ಇರು-ವುದೇ
ಇರು-ವುದೇನು
ಇರು-ವುದೇನೋ
ಇರು-ವುದೊ
ಇರು-ವುದೊಂದೇ
ಇರು-ವುದೋ
ಇರು-ವುವು
ಇರು-ವುವೋ
ಇರು-ವೆ-ಯಂತೆ
ಇರು-ವೆ-ಯೋಪಾದಿ-ಯ-ಲ್ಲಿ
ಇರು-ವೆಗೂ
ಇರು-ವೆನು
ಇರು-ವೆಯ
ಇರು-ವೆವು
ಇರೋಣ
ಇಲ
ಇಲಾಖೆ
ಇಲಾಖೆ-ಯ-ಲ್ಲಿ
ಇಲಾಖೆ-ಯ-ಲ್ಲಿದ್ದ
ಇಲಾಖೆ-ಯ-ವರು
ಇಳಿ
ಇಳಿ-ಜಾರಿ-ನ-ಲ್ಲಿ
ಇಳಿ-ಜಾರಿನ
ಇಳಿ-ದ-ಮೇಲೆ
ಇಳಿ-ದ-ವ-ರ-ನ್ನು
ಇಳಿ-ದರು
ಇಳಿ-ದಾಗ
ಇಳಿ-ದಿ-ರುವರು
ಇಳಿ-ದಿ-ರುವೆ
ಇಳಿ-ದು-ಕೊ-ಳ್ಳು-ವು-ದ-ಕ್ಕಾಗಿ
ಇಳಿ-ದು-ಕೊ-ಳ್ಳು-ವು-ದಕ್ಕೆ
ಇಳಿ-ದು-ಕೊ-ಳ್ಳೋಣ
ಇಳಿ-ದು-ಕೊಂಡ-ರು-ಅ-ವ-ರ-ನ್ನು
ಇಳಿ-ದು-ಕೊಂಡರು
ಇಳಿ-ದು-ಕೊಂಡಿದ್ದ
ಇಳಿ-ದು-ಕೊಂಡಿದ್ದರು
ಇಳಿ-ದು-ಬ-ರು-ವುದು
ಇಳಿ-ದು-ಬ-ರುವ
ಇಳಿ-ದು-ಬಿ-ಟ್ಟರು
ಇಳಿ-ದು-ಹೋಗಿ-ರು-ವೆನು
ಇಳಿ-ದು-ಹೋಗು-ತ್ತಿ-ರು-ವು-ದ-ನ್ನು
ಇಳಿ-ದು-ಹೋಗು-ತ್ತಿದ್ದರು
ಇಳಿ-ಬಿ-ಟ್ಟಿ-ದ್ದರು
ಇಳಿ-ಬಿ-ಟ್ಟಿ-ರುವ
ಇಳಿ-ಬಿ-ಟ್ಟು-ಕೊಂಡು
ಇಳಿ-ಯ-ಬೇ-ಕಾ-ದರೆ
ಇಳಿ-ಯ-ಬೇಕು
ಇಳಿ-ಯ-ಬೇಕೆಂದು
ಇಳಿ-ಯಿತು
ಇಳಿ-ಯು-ತ್ತಾರೆ
ಇಳಿ-ಯು-ತ್ತಿದೆ
ಇಳಿ-ಯು-ತ್ತಿದ್ದ
ಇಳಿ-ಯು-ತ್ತಿದ್ದರು
ಇಳಿ-ಯು-ತ್ತಿದ್ದರೊ
ಇಳಿ-ಯು-ತ್ತಿದ್ದಾಗ
ಇಳಿ-ಯು-ವು-ದಕ್ಕೆ
ಇಳಿ-ಯು-ವುದು
ಇಳಿ-ಯುವ
ಇಳಿ-ಯುವನು
ಇಳಿ-ಯುವರು
ಇಳಿ-ಯುವಾಗ
ಇಳಿ-ಸ-ಬೇಕಾಗಿ-ತ್ತು
ಇಳಿ-ಸ-ಬೇಕೆಂದು
ಇಳಿ-ಸಲಾ-ಯಿತು
ಇಳಿ-ಸಲು
ಇಳಿ-ಸಿ-ರುವರು
ಇಳಿ-ಸು-ವು-ದ-ಕ್ಕಾಗಿ
ಇಳಿ-ಸು-ವು-ದಕ್ಕೆ
ಇಳಿಸಿ
ಇವ
ಇವ-ನನ್ನು
ಇವ-ನಿಂದ
ಇವ-ನಿಗೂ
ಇವ-ನೆ-ಲ್ಲ
ಇವ-ನೊಬ್ಬ
ಇವ-ನೊಬ್ಬನೇ
ಇವ-ರ-ನ್ನೇ
ಇವ-ರ-ಲ್ಲಿ
ಇವ-ರ-ಲ್ಲಿದೆ
ಇವ-ರಿ-ಗಾಗಿ
ಇವ-ರಿ-ಗಿಂತ
ಇವ-ರಿಬ್ಬರೂ
ಇವ-ಳನ್ನು
ಇವ-ಳಿಗೆ
ಇವಳ
ಇವಳೊ-ಡನೆ
ಇವಾ-ನ್ಸ್ಟ-ನ್
ಇವಾವುವೂ
ಇವು
ಇವು-ಗ-ಳನ್ನು
ಇವು-ಗ-ಳಿಗೆ
ಇವು-ಗ-ಳಿದ್ದರೂ
ಇವು-ಗಳ
ಇವು-ಗಳ-ನ್ನೆ
ಇವು-ಗಳ-ನ್ನೆ-ಲ್ಲ
ಇವು-ಗಳ-ನ್ನೆ-ಲ್ಲಾ
ಇವು-ಗಳ-ಲ್ಲಿ
ಇವು-ಗಳ-ಲ್ಲಿಯೂ
ಇವು-ಗಳ-ಲ್ಲೆ
ಇವು-ಗಳ-ಲ್ಲೇ
ಇವು-ಗಳಾವುದ-ರ-ಲ್ಲಿಯೂ
ಇವು-ಗಳಿ-ಗೆ-ಲ್ಲ
ಇವು-ಗಳಿ-ಲ್ಲದೆ
ಇವು-ಗಳಿಂದ
ಇವು-ಗಳು
ಇವು-ಗಳೂ
ಇವು-ಗಳೆ-ರಡರ
ಇವು-ಗಳೆ-ಲ್ಲ
ಇವು-ಗಳೆ-ಲ್ಲ-ದರ
ಇವು-ಗಳೆ-ಲ್ಲ-ದರ-ಲ್ಲಿಯೂ
ಇವು-ಗಳೆ-ಲ್ಲ-ವನ್ನೂ
ಇವು-ಗಳೆ-ಲ್ಲಾ
ಇವು-ಗಳೇ
ಇವು-ಗಳೊಂದಿಗೆ
ಇವು-ಗಳೊಂದೂ
ಇವೆ
ಇವೆ-ಯೆಂದೂ
ಇವೆ-ರಡ-ನ್ನೂ
ಇವೆ-ರಡ-ನ್ನೇ
ಇವೆ-ರಡಕ್ಕೂ
ಇವೆ-ರಡರ
ಇವೆ-ರಡು
ಇವೆ-ರಡೂ
ಇವೆ-ರಡೇ
ಇವೆ-ಲ್ಲ-ವನ್ನು
ಇವೆ-ಲ್ಲವೂ
ಇವೇ
ಇಷ್ಟ
ಇಷ್ಟ-ದೇವ-ತೆ-ಗಳು
ಇಷ್ಟ-ದೇವ-ನಾದ
ಇಷ್ಟ-ದೈವ-ವಾಗ-ಬೇಕು
ಇಷ್ಟ-ನ್ನೆ-ಲ್ಲ
ಇಷ್ಟ-ಪ-ಟ್ಟರೆ
ಇಷ್ಟ-ಪಟ್ಟರೂ
ಇಷ್ಟ-ಪಟ್ಟಿರಿ
ಇಷ್ಟ-ಪಟ್ಟೆ
ಇಷ್ಟ-ಪಡು-ವರೊ
ಇಷ್ಟ-ರ-ಲ್ಲಿ
ಇಷ್ಟ-ವಿ-ಲ್ಲ
ಇಷ್ಟ-ವಿ-ಲ್ಲದೇ
ಇಷ್ಟ-ವಿದ್ದರೆ
ಇಷ್ಟ-ವಿದ್ದು
ಇಷ್ಟಾರ್ಥ
ಇಷ್ಟು
ಇಷ್ಟೆ
ಇಷ್ಟೇ
ಇಷ್ಟೊಂದು
ಇಷ್ತೊಂದು
ಇಹ-ಲೋಕದ
ಇಹವೂ
ಈ
ಈಕೆ
ಈಕೆ-ಯನ್ನು
ಈಗ
ಈಗ-ತಾನೆ
ಈಗ-ತಾನೇ
ಈಗಲೂ
ಈಗಲೆ
ಈಗಲೋ
ಈಗಿ-ನಂತಹ
ಈಗಿ-ನದು
ಈಗಿ-ರು-ವಂತೆ
ಈಗಿ-ರು-ವು-ದಿ-ಲ್ಲ
ಈಗಿ-ರುವ
ಈಗಿ-ರುವುದ-ರ-ಲ್ಲಿ
ಈಗಿನ
ಈಗಿನಷ್ಟು
ಈಚೀಚೆಗೆ
ಈಜಿ-ಕೊಂಡು
ಈಜಿ-ದರು
ಈಜಿಪ್ಟಿಗೆ
ಈಜಿಪ್ಟಿನ
ಈಜಿಪ್ಟಿನ-ಲ್ಲಿರುವ
ಈಜು
ಈಜು-ತ್ತಾನೆ
ಈಜು-ತ್ತಿ-ರು-ವು-ದ-ನ್ನು
ಈಜು-ಬ-ಲ್ಲ
ಈಜು-ವು-ದಕ್ಕೆ
ಈಜು-ವುದು
ಈಜುವ
ಈಡಾ-ದರೂ
ಈಡಾಗಿ-ರು-ವೆನು
ಈಡಾದ
ಈಡು-ಮಾಡು-ತ್ತಿ-ತ್ತು
ಈಡೇ-ರ-ಲಿ-ಲ್ಲ
ಈಡೇ-ರಿಸಿದ
ಈಡೇ-ರು-ತ್ತದೆ
ಈಡೇರಿ-ಸು-ವು-ದಿ-ಲ್ಲ
ಈಡೇರಿ-ಸುವ
ಈಡೇರಿಸ-ಬೇಕೆಂದು
ಈತ
ಈತ-ನನ್ನು
ಈತ-ನಿಗೂ
ಈತ-ನಿಗೆ
ಈತ-ನಿದ್ದ
ಈರ್ವರ
ಈಶ್ವ-ರಾರ್ಥ-ವಾಗಿ
ಈಶ್ವರ
ಈಶ್ವರ-ಚಂದ್ರ
ಈಶ್ವರ-ನ-ಲ್ಲಿ
ಈಶ್ವರ-ನನ್ನು
ಈಶ್ವರ-ನಿ-ರು-ವನು
ಈಶ್ವರ-ನಿಂದೆ
ಈಶ್ವರ-ನಿಂದೆ-ಯನ್ನು
ಈಶ್ವರ-ನಿಗೆ
ಈಶ್ವರ-ನಿದ್ದಾನೆ
ಈಶ್ವರನ
ಈಶ್ವರನು
ಈಶ್ವರೀ
ಈ-ಪ-ತ್ರಿಕೆಯ
ಈ-ರೀತಿ
ಉ
ಉಂಗುರ-ವನ್ನೂ
ಉಂಟಾ-ಗಿದೆ
ಉಂಟಾ-ದರೂ
ಉಂಟಾ-ಯಿತು
ಉಂಟಾಗಿ-ಬಿಡು-ತ್ತಿ-ತ್ತು
ಉಂಟಾಗು-ತ್ತದೆ
ಉಂಟಾಗು-ತ್ತಿ-ರಲಿ-ಲ್ಲ
ಉಂಟಾಗು-ವುದು
ಉಂಟಾಗುವು-ದೆಂಬು-ದ-ನ್ನು
ಉಂಟಾಗುವುದ-ರ-ಲ್ಲಿ
ಉಂಟಾದ
ಉಂಟಾದೀತು
ಉಂಟು
ಉಂಟು-ಮಾಡ-ಬ-ಲ್ಲುದು
ಉಂಟು-ಮಾಡ-ಬೇಕು
ಉಂಟು-ಮಾಡಿ
ಉಂಟು-ಮಾಡಿ-ತೆಂದೂ
ಉಂಟು-ಮಾಡಿ-ದವು
ಉಂಟು-ಮಾಡಿತು
ಉಂಟು-ಮಾಡು-ವುದೋ
ಉಂಟು-ಮಾಡುವ
ಉಂಟೆ
ಉಂಡ-ವರೇ
ಉಕ್ಕಿ
ಉಕ್ಕಿ-ನಂತಹ
ಉಕ್ಕಿತು
ಉಕ್ಕು-ತ್ತಿದ್ದ
ಉಗಮ-ಸ್ಥಾನ
ಉಗಮಿ-ಸುವ
ಉಗಿಯ
ಉಗುಳಿ
ಉಗುಳು-ವರು
ಉಗ್ರ
ಉಗ್ರ-ದೈ-ತ್ಯ-ನಿ-ಲ್ಲದೇ
ಉಗ್ರ-ಭಾವ-ವನ್ನು
ಉಗ್ರ-ವಾದ
ಉಗ್ರ-ಸಾ-ಧನೆ
ಉಗ್ರ-ಸಾಧ-ನೆ-ಯನ್ನು
ಉಗ್ರತೆ
ಉಗ್ರಾಣ-ದ-ಲ್ಲಿ
ಉಗ್ರಾಣಕ್ಕೆ
ಉಚಿ-ತ-ವಾಗಿ
ಉಚಿತ-ವಾದ
ಉಚ್ಚ
ಉಚ್ಚ-ಕಂಠ-ದಿಂದ
ಉಚ್ಚ-ಕುಲ-ಪ್ರಸೂತ-ರೆಂದು
ಉಚ್ಚ-ಧರ್ಮ
ಉಚ್ಚ-ರಿ-ಸಲಾರದೆ
ಉಚ್ಚ-ರಿ-ಸಿದ್ದು
ಉಚ್ಚ-ರಿ-ಸು-ತ್ತಿ-ದ್ದಿರ-ಬಹುದು
ಉಚ್ಚ-ರಿ-ಸು-ತ್ತಿ-ದ್ದು-ದ-ನ್ನು
ಉಚ್ಚ-ರಿಸಿ-ದೊಡ-ನೆಯೇ
ಉಚ್ಚ-ರೀತಿ-ಯ-ಲ್ಲಿ
ಉಚ್ಚ-ವರ್ಗ-ದ-ವ-ರೆಂದು
ಉಚ್ಚ-ವರ್ಗ-ದ-ವರ
ಉಚ್ಚ-ಸಾಧ-ನೆ-ಯನ್ನು
ಉಚ್ಚಾರ
ಉಚ್ಚಾರ-ಗ-ಳನ್ನು
ಉಚ್ಚಾರಣೆ
ಉಚ್ಛ
ಉಚ್ಛೃಂಖಲತೆ-ಯಾಗಲಿ
ಉಚ್ಛ್ರಾ-ಯದ
ಉಚ್ಛ್ರಾಯ-ಸ್ಥಿತಿಗೆ
ಉಛ್ವಾಸ
ಉಜ್ಜಿ-ಕೊ-ಳ್ಳಲು
ಉಜ್ವಲ
ಉಜ್ವಲ-ವಾಗಿ
ಉಜ್ವಲ-ವಾದ
ಉಡಲು
ಉಡಿಗೆ
ಉಡಿಗೆ-ತೊಡಿಗೆ
ಉಡಿಗೆ-ತೊಡಿಗೆ-ಯಂತೆ
ಉಡಿಸಿ
ಉಡಿಸಿ-ದ್ದಾ-ಯಿತು
ಉಡು-ತ್ತಾನೆ
ಉಡುಗೆ-ಯ-ಲ್ಲಿ-ದ್ದು-ದ-ನ್ನು
ಉಡುಪಿ-ನ-ಲ್ಲಿಯೂ
ಉಡುಪು
ಉಣಿ-ಸು-ವು-ದ-ನ್ನು
ಉಣ್ಣ-ಬಹು-ದ-ಲ್ಲವೆ
ಉದ-ಹರಿ-ಸದೆ
ಉದ-ಹರಿ-ಸಿದ
ಉದ-ಹರಿ-ಸು-ತ್ತಿ-ದ್ದರು
ಉದಕ-ಮಂಡಲಕ್ಕೆ
ಉದಯ
ಉದಯಿ-ಸಿದ
ಉದಯಿ-ಸು-ತ್ತಿ-ರುವಾಗ
ಉದಯಿ-ಸು-ತ್ತಿದೆ
ಉದಯಿ-ಸುವ
ಉದಯಿಸ-ಬೇಕಾಗಿದೆ
ಉದಾ-ತ್ತ
ಉದಾ-ಹರಿ-ಸ-ತೊಡಗಿದ-ರು-ದುಃಖ
ಉದಾ-ಹರಿ-ಸ-ತೊಡಗಿದರು
ಉದಾ-ಹರಿ-ಸ-ಬೇಕಾ-ಯಿತು
ಉದಾ-ಹರಿ-ಸ-ಬೇಕೋ
ಉದಾ-ಹರಿ-ಸಿ-ದರು
ಉದಾ-ಹರಿ-ಸಿ-ಲ್ಲ
ಉದಾ-ಹರಿ-ಸಿದೆ
ಉದಾ-ಹರಿ-ಸು-ತ್ತಿ-ದ್ದರು
ಉದಾ-ಹರಿ-ಸು-ತ್ತೇನೆ
ಉದಾ-ಹರಿ-ಸು-ತ್ತೇವೆ
ಉದಾ-ಹರಿ-ಸು-ವು-ದ-ಕ್ಕಾಗಿದೆ
ಉದಾ-ಹರಿ-ಸು-ವು-ದಕ್ಕೆ
ಉದಾ-ಹರಿ-ಸು-ವೆವು
ಉದಾ-ಹರಿ-ಸುವ
ಉದಾರ
ಉದಾರ-ಬುದ್ಧಿಯ
ಉದಾರ-ಭಾ-ವನೆ
ಉದಾರ-ಭಾವ-ವನ್ನು
ಉದಾರ-ಮನ-ಸ್ಸಿನ-ವರು
ಉದಾರ-ವಾ-ದುದು
ಉದಾರ-ವಾಗಿ
ಉದಾರ-ವಾಗಿ-ತ್ತೋ
ಉದಾರ-ವಾಗಿವೆ
ಉದಾರ-ವಾಣಿ-ಯನ್ನು
ಉದಾರ-ವಾದ
ಉದಾರತೆ
ಉದಿ-ಸಿತು
ಉದಿಸಿ
ಉದು-ರು-ವಂತೆ
ಉದು-ರು-ವು-ದ-ನ್ನು
ಉದುರಿ
ಉದ್ಘಾ-ಟನಾ
ಉದ್ಘಾಟನೆ
ಉದ್ಘಾಟಿ-ಸಿ-ದರು
ಉದ್ಘೋಷಿ-ಸು-ತ್ತ
ಉದ್ದ-ವಾದ
ಉದ್ದ-ವಿದೆ
ಉದ್ದಕ್ಕೂ
ಉದ್ದೀಪನೆ
ಉದ್ದೇ-ಶಕ್ಕೆ
ಉದ್ದೇಶ
ಉದ್ದೇಶ-ಗ-ಳನ್ನು
ಉದ್ದೇಶ-ಗಳಿವೆ
ಉದ್ದೇಶ-ಗಳೇ
ಉದ್ದೇಶ-ದಿಂದ
ಉದ್ದೇಶ-ದಿಂದಲೇ
ಉದ್ದೇಶ-ವನ್ನಿ-ಟ್ಟು-ಕೊಂಡು
ಉದ್ದೇಶ-ವನ್ನು
ಉದ್ದೇಶ-ವಾಗಿ-ದ್ದವು
ಉದ್ದೇಶ-ವಿದೆ
ಉದ್ದೇಶ-ವೂ-ಉದ್ದೇಶ-ವೇನು
ಉದ್ದೇಶ-ವೇ-ನೆಂಬುದು
ಉದ್ದೇಶದ
ಉದ್ದೇಶವೂ
ಉದ್ದೇಶಿ-ಸುವಿರೊ
ಉದ್ದೇಶಿಸಿ
ಉದ್ಧ-ರಿ-ಸಲು
ಉದ್ಧ-ರಿಸಿ-ಕೊಳ್ಳ-ಬೇಕು
ಉದ್ಧಟ-ತನ-ವೆಂದು
ಉದ್ಧರಿ-ಸು-ತ್ತಿ-ದ್ದರು
ಉದ್ಧರಿ-ಸುವ
ಉದ್ಧರಿಸ-ಬ-ಲ್ಲದು
ಉದ್ಧರಿಸ-ಬೇಕು
ಉದ್ಧರೇದಾ-ತ್ಮ-ನಾ-ತ್ಮಾನಂ-ತನ್ನನ್ನು
ಉದ್ಧಾ-ರಕ್ಕೆ
ಉದ್ಧಾಮ
ಉದ್ಧಾರ
ಉದ್ಧಾರ-ಮಾಡ-ಬೇ-ಕಾ-ದರೆ
ಉದ್ಧಾರ-ಮಾಡಲು
ಉದ್ಧಾರ-ಮಾಡಿ
ಉದ್ಧಾರ-ಮಾಡಿ-ದನು
ಉದ್ಧಾರ-ಮಾಡು
ಉದ್ಧಾರ-ಮಾಡೆಂದು
ಉದ್ಧಾರ-ವನ್ನು
ಉದ್ಧಾರ-ವಾಗ-ಬೇ-ಕಾ-ದರೆ
ಉದ್ಧಾರ-ವಾಗ-ಲೆಂದು
ಉದ್ಧಾರ-ವಾಗಲಿ
ಉದ್ಬೋ-ಧನ
ಉದ್ಬೋ-ಧನ-ಕ್ಕಾಗಿ
ಉದ್ಬೋ-ಧನ-ವನ್ನು
ಉದ್ಬೋ-ಧನವು
ಉದ್ಬೋಧ-ನಕ್ಕೆ
ಉದ್ಭೋ-ಧನ
ಉದ್ಭೋ-ಧನ-ಕ್ಕಾಗಿ
ಉದ್ಯಮ-ವನ್ನು
ಉದ್ಯಾ-ನಕ್ಕೆ
ಉದ್ಯಾನ
ಉದ್ಯಾನ-ದ-ಲ್ಲಿ
ಉದ್ಯಾನ-ಮನೆ-ಯ-ಲ್ಲಿ
ಉದ್ಯಾನ-ಮನೆಯ
ಉದ್ಯಾನ-ವ-ನ-ದ-ಲ್ಲಿ
ಉದ್ಯಾನ-ವ-ನ-ದಿಂದ
ಉದ್ಯಾನ-ವನ-ಗ-ಳಿಗೆ
ಉದ್ಯಾನ-ವನ-ಗಳು
ಉದ್ಯಾನ-ವಿದೆ
ಉದ್ಯಾನದ
ಉದ್ಯುಕ್ತ-ರಾಗಿ-ರು-ತ್ತಿದ್ದರು
ಉದ್ಯೋಗ
ಉದ್ಯೋಗ-ವಿದ್ದರೂ
ಉದ್ಯೋಗ-ಸ್ಥ-ರೆ-ಲ್ಲ
ಉದ್ಯೋಗ-ಸ್ಥರು
ಉದ್ರೇಕ
ಉದ್ರೇಕ-ಪರ
ಉದ್ರೇಕಿ-ಸು-ತ್ತಿದೆ
ಉದ್ರೇಕಿ-ಸು-ವುದು
ಉದ್ವಿಗ್ನ-ತೆಗೂ
ಉದ್ವಿಗ್ನರಾಗು-ತ್ತಿ-ರಲಿ-ಲ್ಲ
ಉದ್ವೇಗ
ಉದ್ವೇಗ-ಗೊ-ಳ್ಳದೆ
ಉದ್ವೇಗ-ದ-ಲ್ಲಿ
ಉದ್ವೇಗ-ದಿಂದ
ಉದ್ವೇಗ-ದಿಂದಲೂ
ಉದ್ವೇಗ-ದೈ-ತ್ಯ
ಉದ್ವೇಗ-ಪ-ರವ-ಶತೆ
ಉದ್ವೇಗ-ಪರರೋ
ಉದ್ವೇಗ-ವನ್ನು
ಉದ್ವೇಗ-ವಾಗಲಿ
ಉದ್ವೇಗ-ವಿ-ಲ್ಲ
ಉದ್ವೇಗಕ್ಕೆ
ಉದ್ವೇಗವೇ
ಉದ್-ಗ್ರಂಥ-ಗಳ-ನ್ನೇ
ಉಪ-ಕರ-ಣ-ಗಳ
ಉಪ-ಕರ-ಣ-ಗಳಿ-ಗಾಗಿ
ಉಪ-ಕರ-ಣ-ಗಳು
ಉಪ-ಕಾ-ರಕ್ಕೆ
ಉಪ-ಕಾರ
ಉಪ-ಕಾರ-ವನ್ನು
ಉಪ-ಕಾರ-ವಾಗು-ತ್ತ-ದೆ-ಯೆಂದು
ಉಪ-ಕಾರಿ-ಗಳು
ಉಪ-ಕಾರ್ಯ-ದರ್ಶಿ-ಗಳಾಗಿಯೂ
ಉಪ-ಕ್ರಮಿ-ಸು-ತ್ತಾರೆ
ಉಪ-ಕ್ರಮಿಸಿ-ದನು
ಉಪ-ಕ್ರಮಿಸಿ-ದರು
ಉಪ-ಚಾರ-ಗ-ಳಾದ
ಉಪ-ಚಾರ-ಗಳ
ಉಪ-ಚಾರ-ವನ್ನು
ಉಪ-ದೇಶ-ಗ-ಳನ್ನು
ಉಪ-ದೇಶ-ಗಳು
ಉಪ-ದೇಶ-ವನ್ನು
ಉಪ-ನ್ಯಾ-ಸದ
ಉಪ-ನ್ಯಾಸ
ಉಪ-ನ್ಯಾಸ-ಕ-ರ-ನ್ನಾಗಿ
ಉಪ-ನ್ಯಾಸ-ಕ-ರ-ನ್ನು
ಉಪ-ನ್ಯಾಸ-ಕ-ರ-ಲ್ಲಿ
ಉಪ-ನ್ಯಾಸ-ಕ-ರಂತೆ
ಉಪ-ನ್ಯಾಸ-ಕ-ರಿಗೂ
ಉಪ-ನ್ಯಾಸ-ಕರು
ಉಪ-ನ್ಯಾಸ-ಕರೂ
ಉಪ-ನ್ಯಾಸ-ಗ-ಳನ್ನು
ಉಪ-ನ್ಯಾಸ-ಗ-ಳಿಗೆ
ಉಪ-ನ್ಯಾಸ-ಗಳ
ಉಪ-ನ್ಯಾಸ-ಗಳ-ನ್ನಿ-ಡು-ವುದು
ಉಪ-ನ್ಯಾಸ-ಗಳ-ನ್ನೂ
ಉಪ-ನ್ಯಾಸ-ಗಳ-ನ್ನೊಳ-ಗೊಂಡ
ಉಪ-ನ್ಯಾಸ-ಗಳ-ಲ್ಲದೆ
ಉಪ-ನ್ಯಾಸ-ಗಳ-ಲ್ಲಿ
ಉಪ-ನ್ಯಾಸ-ಗಳಂತೆಯೇ
ಉಪ-ನ್ಯಾಸ-ಗಳು
ಉಪ-ನ್ಯಾಸ-ಗಳೆ-ಲ್ಲ
ಉಪ-ನ್ಯಾಸ-ದ-ಲ್ಲಿ
ಉಪ-ನ್ಯಾಸ-ದ-ಲ್ಲಿ-ರುವ
ಉಪ-ನ್ಯಾಸ-ದಂ-ತಿ-ಲ್ಲ
ಉಪ-ನ್ಯಾಸ-ದಿಂದ
ಉಪ-ನ್ಯಾಸ-ದಿಂದಲೇ
ಉಪ-ನ್ಯಾಸ-ಮಾಡಿ
ಉಪ-ನ್ಯಾಸ-ಮಾಡಿ-ದರು
ಉಪ-ನ್ಯಾಸ-ವನ್ನು
ಉಪ-ನ್ಯಾಸ-ವನ್ನೂ
ಉಪ-ನ್ಯಾಸ-ವನ್ನೆ-ಲ್ಲ
ಉಪ-ನ್ಯಾಸ-ವನ್ನೆ-ಲ್ಲಾ
ಉಪ-ನ್ಯಾಸ-ವಾ-ಗಿದೆ
ಉಪ-ನ್ಯಾಸ-ವಾ-ದರೂ
ಉಪ-ನ್ಯಾಸ-ವಾದ
ಉಪ-ನ್ಯಾಸ-ವಾದ-ಮೇಲೆ
ಉಪ-ನ್ಯಾಸ-ವೆ-ಲ್ಲ
ಉಪ-ನ್ಯಾಸಕ
ಉಪ-ನ್ಯಾಸಕ್ಕೆ
ಉಪ-ನ್ಯಾಸವೆ
ಉಪ-ನ್ಯಾಸವೇ
ಉಪ-ನ್ಯಾಸಾ-ದಿ-ಗ-ಳನ್ನು
ಉಪ-ನ್ಯಾಸಾ-ದಿ-ಗಳು
ಉಪ-ನ್ಯಾಸಾದಿ-ಗಳ
ಉಪ-ಮಾನ
ಉಪ-ಮಾನ-ಗ-ಳನ್ನು
ಉಪ-ಮಾನದ
ಉಪ-ಯುಕ್ತ-ರಾದ
ಉಪ-ಯೊಗಿ-ಸುವರು
ಉಪ-ಯೊಗಿಸ-ಬಹುದು
ಉಪ-ಯೊಗಿಸಿ
ಉಪ-ಯೋಗ-ವನ್ನು
ಉಪ-ಯೋಗ-ವಾಗ-ಬಹುದು
ಉಪ-ಯೋಗ-ವಾಗದ
ಉಪ-ಯೋಗ-ವಾಗು-ತ್ತದೆ
ಉಪ-ಯೋಗ-ವಾಗು-ವಂತೆ
ಉಪ-ಯೋಗ-ವುಳ್ಳದ್ದು
ಉಪ-ಯೋಗಿ-ಸ-ಕೂಡ-ದೆಂದು
ಉಪ-ಯೋಗಿ-ಸ-ಬಹುದು
ಉಪ-ಯೋಗಿ-ಸ-ಬೇಕು
ಉಪ-ಯೋಗಿ-ಸ-ಲಿ-ಲ್ಲ
ಉಪ-ಯೋಗಿ-ಸ-ಲಿ-ಲ್ಲವೆ
ಉಪ-ಯೋಗಿ-ಸ-ಲ್ಪ-ಡು-ತ್ತವೆ
ಉಪ-ಯೋಗಿ-ಸ-ಲ್ಪಟ್ಟಿವೆ
ಉಪ-ಯೋಗಿ-ಸದೆ
ಉಪ-ಯೋಗಿ-ಸದೇ
ಉಪ-ಯೋಗಿ-ಸಿ-ಕೊಳ್ಳ-ಬ-ಲ್ಲೆವು
ಉಪ-ಯೋಗಿ-ಸಿ-ಕೊಳ್ಳ-ಬಹುದು
ಉಪ-ಯೋಗಿ-ಸಿ-ಕೊಳ್ಳು-ತ್ತೀರಿ
ಉಪ-ಯೋಗಿ-ಸಿ-ದರೆ
ಉಪ-ಯೋಗಿ-ಸಿ-ದಾಗ
ಉಪ-ಯೋಗಿ-ಸಿ-ರ-ಲಿ-ಲ್ಲ
ಉಪ-ಯೋಗಿ-ಸಿ-ರು-ವೆವು
ಉಪ-ಯೋಗಿ-ಸಿ-ರುವ
ಉಪ-ಯೋಗಿ-ಸಿ-ರುವರೇ
ಉಪ-ಯೋಗಿ-ಸಿ-ಲ್ಲ
ಉಪ-ಯೋಗಿ-ಸಿದ
ಉಪ-ಯೋಗಿ-ಸು-ತ್ತಾನೆ
ಉಪ-ಯೋಗಿ-ಸು-ತ್ತಾರೆ
ಉಪ-ಯೋಗಿ-ಸು-ತ್ತಿ-ದ್ದರು
ಉಪ-ಯೋಗಿ-ಸು-ತ್ತಿ-ರು-ವಿರೋ
ಉಪ-ಯೋಗಿ-ಸು-ತ್ತಿದ್ದ
ಉಪ-ಯೋಗಿ-ಸು-ವು-ದಿ-ಲ್ಲ
ಉಪ-ಯೋಗಿ-ಸುವ
ಉಪ-ಯೋಗಿ-ಸುವಂತೆ
ಉಪ-ಯೋಗಿ-ಸುವರು
ಉಪ-ಯೋಗಿ-ಸುವಳು
ಉಪ-ಯೋಗಿ-ಸುವಷ್ಟು
ಉಪ-ಯೋಗಿಸಿ
ಉಪ-ವಾಸ
ಉಪ-ವಾಸ-ದಿಂದ
ಉಪ-ವಾಸ-ವಿ-ರು-ವೆನು
ಉಪ-ವಾಸ-ವಿದ್ದರು
ಉಪ-ವಾಸ-ವಿದ್ದು
ಉಪ-ವಾಸಕ್ಕೆ
ಉಪ-ವಾಸದ
ಉಪ-ಶಮ-ನಕ್ಕೆ
ಉಪ-ಹಾರ-ವನ್ನು
ಉಪಕಾ-ರಾರ್ಥ
ಉಪಚ-ರಿಸಿ
ಉಪನ-ಯನ
ಉಪನ-ಯನ-ವನ್ನು
ಉಪನಯ-ನಕ್ಕೆ
ಉಪನಯ-ನದ
ಉಪನಿಷ-ತ್ತ-ಲ್ಲದೆ
ಉಪನಿಷ-ತ್ತನ್ನು
ಉಪನಿಷ-ತ್ತಿ-ನ-ಲ್ಲಿ
ಉಪನಿಷ-ತ್ತಿ-ನಿಂದ
ಉಪನಿಷ-ತ್ತಿನ
ಉಪನಿಷ-ತ್ತಿನ-ಲ್ಲಿ-ರು-ವು-ದೆ-ಲ್ಲ
ಉಪನಿಷ-ತ್ತಿನ-ಲ್ಲೆ-ಲ್ಲ
ಉಪನಿಷ-ತ್ತು
ಉಪನಿಷ-ತ್ತು-ಗಳ-ಲ್ಲದೆ
ಉಪನಿಷ-ತ್ತು-ಗಳ-ಲ್ಲಿ
ಉಪನಿಷ-ತ್ತು-ಗಳು
ಉಪನಿಷ-ತ್ತೆಂದೂ
ಉಪಶ-ಮನ
ಉಪಾ-ದಾನ-ಗಳಿಂದ
ಉಪಾ-ಸನೆ
ಉಪಾ-ಹಾರದ
ಉಪಾಧಿ
ಉಪಾಧಿ-ಗಳೆ-ಲ್ಲ
ಉಪಾಧ್ಯಾ-ಯರ
ಉಪಾಧ್ಯಾ-ಯರು
ಉಪಾಧ್ಯಾಯ
ಉಪಾಧ್ಯಾಯ-ನನ್ನು
ಉಪಾಧ್ಯಾಯ-ನಾಗಿ
ಉಪಾಧ್ಯಾಯ-ನಾಗಿ-ರು-ವು-ದ-ರಿಂದ
ಉಪಾಧ್ಯಾಯ-ನಾದ
ಉಪಾಧ್ಯಾಯ-ನಿಗೆ
ಉಪಾಧ್ಯಾಯ-ರ-ನ್ನು
ಉಪಾಧ್ಯಾಯ-ರಾದ
ಉಪಾಧ್ಯಾಯ-ರಿಗೆ
ಉಪಾಧ್ಯಾಯ-ರು-ಗ-ಳನ್ನು
ಉಪಾಧ್ಯಾಯ-ರು-ಗಳಿ-ಗೆ-ಲ್ಲ
ಉಪಾಧ್ಯಾಯ-ರೊ-ಡನೆ
ಉಪಾಯ
ಉಪಾಯ-ದಿಂದ
ಉಪಾಯ-ವನ್ನು
ಉಪಾಯ-ವನ್ನೆ-ಲ್ಲ
ಉಪಾಸಕ-ರಾಗಿ-ದ್ದರು
ಉಪೇ-ನ್
ಉಪೇಕ್ಷೆ
ಉಪ್ಪ-ನ್ನು
ಉಪ್ಪ-ನ್ನೇಕೆ
ಉಪ್ಪು
ಉಬ್ಬಿ-ದರೂ
ಉಮಾ-ಕುಮಾರಿ
ಉಮಾ-ಶಂ-ಕರ್
ಉರಿ-ಬಿಸಿ-ಲಿ-ನ-ಲ್ಲಿ
ಉರಿ-ಯು-ತ್ತಿ-ತ್ತು
ಉರಿ-ಯು-ತ್ತಿದೆ
ಉರಿ-ಯು-ತ್ತಿದ್ದ
ಉರಿ-ಯು-ತ್ತಿದ್ದರೆ
ಉರಿ-ಯು-ತ್ತಿರುವ
ಉರಿ-ಯು-ವುದು
ಉರಿದು
ಉರಿದು-ಹೋಗು-ತ್ತಿದ್ದೇನೆ
ಉರುಳಿ-ಕೊ-ಳ್ಳು-ತ್ತಿ-ತ್ತು
ಉರುಳಿ-ಸುವುದ-ರ-ಲ್ಲಿ-ದ್ದರು
ಉರುಳಿಸಿ-ಬಿ-ಟ್ಟರು
ಉರ್ದು
ಉಲಾರ್
ಉಲು
ಉಳಲು
ಉಳಿ-ಗಾಲ-ವಿ-ಲ್ಲ
ಉಳಿ-ದದ್ದು
ಉಳಿ-ದಿ-ರು-ವಿರಿ
ಉಳಿ-ದಿ-ರು-ವು-ದ-ನ್ನು
ಉಳಿ-ದಿ-ರು-ವುದು
ಉಳಿ-ದಿ-ರು-ವುದೇನು
ಉಳಿ-ದಿದೆ
ಉಳಿ-ದಿವೆ
ಉಳಿ-ದು-ದ-ನ್ನು
ಉಳಿ-ದುದು
ಉಳಿ-ದೆ-ಲ್ಲವೂ
ಉಳಿ-ಯಿತು
ಉಳಿ-ಯು-ವುದು
ಉಳಿ-ಯು-ವುವು
ಉಳಿ-ಯುವರೆ
ಉಳಿ-ಸಲು
ಉಳಿ-ಸು-ವು-ದ-ಕ್ಕಾಗಿ
ಉಳಿ-ಸು-ವು-ದಿ-ಲ್ಲ
ಉಳಿತಾಯ-ವಾಗು-ವುದು
ಉಳಿಯು-ತ್ತವೆ
ಉಳಿಸ-ಬೇಕು
ಉಳಿಸಿ-ಕೊಂಡಿರಲು
ಉಳಿಸಿ-ಕೊಂಡು
ಉಳಿಸಿ-ಕೊಡು-ವು-ದಿ-ಲ್ಲ
ಉಳ್ಳ
ಉಸಿ-ರ-ಲ್ಲಿ
ಉಸಿ-ರಿ-ನಿಂದ
ಉಸಿ-ರಿಗೆ
ಉಸಿ-ರಿನ
ಉಸಿ-ರಿನ-ಲ್ಲೆ
ಉಸಿ-ರಿನಂ-ತಿ-ತ್ತು
ಉಸಿರಾ-ಡದ
ಉಸಿರಾ-ಡುವ-ವ-ರೆಗೂ
ಉಸಿರಾ-ಡುವಿಕೆ
ಉಸಿರಾಗ-ಬೇಕು
ಉಸಿರಾಡಲು
ಉಸಿರು-ಬಿ-ಟ್ಟರು
ಉಸಿರೆಳೆದು
ಊ
ಊಟ
ಊಟ-ಕ್ಕಾಗಿ
ಊಟ-ಮಾಡ-ಬ-ಲ್ಲೆ
ಊಟ-ಮಾಡಲು
ಊಟ-ಮಾಡಿ-ರು-ವು-ದಾಗಿ
ಊಟ-ಮಾಡಿ-ಸಿ-ದಳು
ಊಟ-ಮಾಡಿ-ಸು-ತ್ತಾ
ಊಟ-ಮಾಡು-ತ್ತಿದ್ದಾಗ
ಊಟ-ಮಾಡು-ತ್ತಿದ್ದೆ
ಊಟ-ಮಾಡು-ತ್ತಿರು-ವಂತೆ
ಊಟ-ಮಾಡು-ವಂತೆ
ಊಟ-ಮಾಡು-ವಾಗ
ಊಟ-ಮಾಡು-ವಿರಾ
ಊಟ-ಮಾಡು-ವುದು
ಊಟ-ಮಾಡುವ
ಊಟ-ವ-ನ್ನಾ-ದರೂ
ಊಟ-ವನ್ನರ್ಪಿಸಿ-ದ್ದೇನೆ
ಊಟ-ವನ್ನು
ಊಟ-ವನ್ನೆ-ಲ್ಲ
ಊಟ-ವಾದ
ಊಟ-ವಾದ-ಮೇಲೆ
ಊಟ-ವಿ-ರ-ಲಿ-ಲ್ಲ
ಊಟ-ವಿ-ಲ್ಲದೆ
ಊಟ-ವೆ-ಲ್ಲ
ಊಟಕ್ಕೆ
ಊಟದ
ಊಟವಿ-ಡು-ತ್ತಿದ್ದರು
ಊದಿ-ಕೊಂಡಿ-ತ್ತು
ಊದಿ-ದ್ದುವು
ಊನ-ವಾ-ಯಿತು
ಊಪ-ನ್ಯಾಸ-ದ-ಲ್ಲಿ
ಊರಿ-ನ-ಲ್ಲಿ
ಊರಿ-ನ-ಲ್ಲಿ-ದ್ದಾಗಲೇ
ಊರಿ-ನ-ಲ್ಲಿ-ರುವ
ಊರಿ-ನ-ಲ್ಲಿದ್ದ
ಊರಿ-ನಿಂದಲೂ
ಊರು
ಊರು-ಊರು-ಗಳ-ಲ್ಲಿ
ಊರು-ಗ-ಳನ್ನು
ಊರು-ಗಳ-ನ್ನೆ-ಲ್ಲ
ಊರು-ಗಳಿಂದ
ಊರು-ಗಳು
ಊರು-ಗೋಲು
ಊರೇ
ಊರ್ಜಿತ-ಗೊಳಿಸಿ-ಕೊಳ್ಳ-ಬಹು-ದೆಂದು
ಊರ್ಧ್ವ-ಮುಖ-ವಾಗಿ
ಊಹಿ-ಸದ
ಊಹಿ-ಸಲಾರ-ದವ-ನಾಗಿದ್ದೆ
ಊಹಿ-ಸಲೂ
ಊಹಿ-ಸಲೇ-ಬೇಕಾಗುವುದು
ಊಹಿ-ಸಿದ್ದು
ಊಹಿ-ಸು-ವೆವು
ಊಹಿಸಿ
ಊಹಿಸಿ-ದರು
ಊಹಿಸಿ-ದವ-ನಿಗೂ
ಊಹೆ-ಗ-ಳನ್ನು
ಊಹೆ-ಯನ್ನು
ಋ
ಋಜು-ತ್ವ
ಋಣಿ
ಋಣಿ-ಗಳಾಗಿ-ರು-ತ್ತಾರೆ
ಋಣಿ-ಗಳೋ
ಋಣಿ-ಯಾಗಿ-ದ್ದನು
ಋಣಿ-ಯಾಗಿ-ದ್ದರೂ
ಋಣಿ-ಯಾಗಿ-ರು-ವುದು
ಋತ
ಋತು-ವಿನ
ಋಷಿ
ಋಷಿ-ಗ-ಳಾಗು-ವಂತೆ
ಋಷಿ-ಗಳ
ಋಷಿ-ಗಳ-ಲ್ಲಿ
ಋಷಿ-ಗಳಂತೆ
ಋಷಿ-ಗಳಾಗ-ಬೇಕು
ಋಷಿ-ಗಳು
ಋಷಿ-ಗಳೆ-ಲ್ಲ
ಋಷಿ-ಗಳೊ-ಡನೆ
ಋಷಿ-ಬೀರ್
ಋಷಿ-ಯಂತೆ
ಋಷಿ-ಯೊಬ್ಬ
ಋಷಿ-ವತ್
ಎ
ಎಂ
ಎಂಎ
ಎಂಜಿ-ನ್ನಿ-ನಂತೆ
ಎಂಟನೇ
ಎಂಟು
ಎಂಡಿನ
ಎಂತಲೇ
ಎಂತಲೇನೋ
ಎಂತಹ
ಎಂತಹ-ವನು
ಎಂತಹ-ವರೋ
ಎಂತಹದು
ಎಂತಹುದು
ಎಂಥ
ಎಂದ
ಎಂದ-ಮೇಲೆ
ಎಂದ-ರೇನು
ಎಂದಿ-ಗಿಂತ
ಎಂದಿ-ಗಿಂತಲೂ
ಎಂದಿಗೂ
ಎಂದಿಗೆ
ಎಂದಿತು
ಎಂದು
ಎಂದು-ಕೊ-ಳ್ಳು-ತ್ತೇನೆ
ಎಂದು-ಕೊಂಡೆ
ಎಂದು-ಕೊಂಡೆಯಾ
ಎಂದು-ಬಿಡು-ತ್ತಿದ್ದೆ
ಎಂದೂ
ಎಂದೆ
ಎಂದೆ-ನಿ-ಸಿತು
ಎಂದೆ-ನಿಸಿ-ಕೊಂಡಿ-ರುವ
ಎಂದೆಂದಿಗೂ
ಎಂದೋ
ಎಂಬ
ಎಂಬಂತೇ
ಎಂಬಷ್ಟು
ಎಂಬಾ-ಕೆಯ
ಎಂಬಾಕೆ
ಎಂಬಾಕೆ-ಯನ್ನು
ಎಂಬಾತ
ಎಂಬು-ದ-ನ್ನು
ಎಂಬು-ದ-ನ್ನೇ
ಎಂಬು-ದ-ರ-ಲ್ಲಿ
ಎಂಬು-ದ-ರ-ಲ್ಲೇ
ಎಂಬು-ದ-ಲ್ಲ
ಎಂಬು-ದ-ಲ್ಲದೆ
ಎಂಬು-ದಕ್ಕೂ
ಎಂಬು-ದಕ್ಕೆ
ಎಂಬು-ದರ
ಎಂಬು-ದರಿ-ವಾ-ಯಿತು
ಎಂಬು-ದಾಗಿ
ಎಂಬು-ದೊಂದು
ಎಂಬು-ವಂತ-ಹ-ವ-ರಿ-ದ್ದರು
ಎಂಬು-ವನ
ಎಂಬು-ವನು
ಎಂಬು-ವನೂ
ಎಂಬು-ವರ
ಎಂಬು-ವರು
ಎಂಬು-ವಳು
ಎಂಬುದ-ನ್ನೆ-ಲ್ಲ
ಎಂಬುದ-ನ್ನೆ-ಲ್ಲಾ
ಎಂಬುದು
ಎಂಬುದೇ
ಎಂಬುವ
ಎಂಬುವ-ನನ್ನು
ಎಂಬುವ-ನೊ-ಡನೆ
ಎಂಬುವ-ರ-ನ್ನು
ಎಂಬುವ-ರ-ನ್ನೆ-ಲ್ಲಾ
ಎಂಬುವ-ರಿಗೆ
ಎಂಬುವ-ರೊ-ಡನೆ
ಎಂಬುವ-ರೊಬ್ಬ-ರಿಗೆ
ಎಂಬುವ-ವರು
ಎಂಬುವಳ
ಎಂಭ-ತ್ತು
ಎಕ್ಸ್
ಎಗ್ಮೋರ್
ಎಗ್ಮೋರ್
ಎಚ್ಎ-ಲ್
ಎಚ್ಚ-ರಿ-ಸು-ವು-ದಕ್ಕೆ
ಎಚ್ಚರ-ವಾಗಿದ್ದು
ಎಚ್ಚರಗೊಳಿ-ಸಲೆ-ತ್ನಿ-ಸಿ-ದರು
ಎಚ್ಚರಗೊಳ್ಳಿ
ಎಚ್ಚರಗೊಳ್ಳು
ಎಚ್ಚರಗೊಳ್ಳು-ವು-ದಿ-ಲ್ಲ
ಎಚ್ಚರಿ-ಕೆ-ಯನ್ನು
ಎಚ್ಚರಿ-ಕೆ-ಯಿಂದ
ಎಚ್ಚರಿ-ಕೆ-ಯಿಂದಿರ-ಬೇಕು
ಎಚ್ಚರಿ-ಸ-ಬೇಕಾಗಿದೆ
ಎಚ್ಚರಿ-ಸಿತು
ಎಚ್ಚರಿಕೆ
ಎಚ್ಚೆ-ತ್ತರು
ಎಚ್ಚೆ-ತ್ತಿರು-ವಾಗ
ಎಚ್ಚೆ-ತ್ತು
ಎಡ
ಎಡ-ಬಲ-ಗಳ-ಲ್ಲಿ
ಎಡ-ಬಿ-ಡದೆ
ಎಡಗಣ್ಣಿ-ನ-ಲ್ಲಿ
ಎಡಗಣ್ಣಿನ
ಎಡಗೈ-ಯ-ಲ್ಲಿ
ಎಡಗೈ-ಯನ್ನು
ಎಡಗೈ-ಯಿಂದ
ಎಡವಿ
ಎಡಾ-ಬೀರ್
ಎಡಿ-ನ್ಬರೋ
ಎಡೆ-ಬಿ-ಡದೆ
ಎಡೆ-ಯಿ-ಲ್ಲ
ಎಡೆಯೇ
ಎಡ್ವರ್ಡ್ಸ್
ಎಡ್-ವರ್ಡ್
ಎಣಿ-ಸಿದ್ದೇ
ಎಣಿಸಿ
ಎಣಿಸಿಯೇ
ಎಣ್ಣೆ
ಎಣ್ಣೆಯ
ಎಥಿ-ಕ-ಲ್
ಎದು-ರಾಗಿ
ಎದು-ರಿ-ಸು-ವು-ದಕ್ಕೆ
ಎದು-ರಿಗಿ-ದ್ದರು
ಎದು-ರಿಗಿ-ರುವ
ಎದು-ರಿಸಿ
ಎದು-ರಿಸಿ-ದನು
ಎದು-ರಿಸಿ-ದರು
ಎದು-ರಿಸಿ-ದರೂ
ಎದು-ರಿಸಿ-ದಳು
ಎದು-ರಿಸಿ-ವನು
ಎದುರಿ-ಗೊಂದು
ಎದುರಿ-ನ-ಲ್ಲಿ
ಎದುರಿ-ಸ-ಬೇಕಾ-ಯಿತು
ಎದುರಿ-ಸ-ಬೇಕಾಗಿ
ಎದುರಿ-ಸ-ಬೇಕಾಗಿ-ರು-ವುದೇ
ಎದುರಿ-ಸು-ತ್ತಿವೆ
ಎದುರಿ-ಸು-ವುದು
ಎದುರಿ-ಸುವ
ಎದುರಿಸ-ಬೇಕು
ಎದುರಿಸ-ಬೇಕೆ
ಎದುರು
ಎದುರು-ಗೊ-ಳ್ಳಲು
ಎದುರು-ಗೊ-ಳ್ಳು-ವು-ದ-ಕ್ಕಾಗಿ
ಎದುರು-ಗೊ-ಳ್ಳು-ವು-ದಕ್ಕೆ
ಎದುರು-ಗೊ-ಳ್ಳುವರು
ಎದುರು-ಗೊಂಡು
ಎದೆ
ಎದೆ-ಕರ-ಗುವ
ಎದೆ-ಕರಗಿ
ಎದೆ-ಗಾರಿಕೆ
ಎದೆ-ಗೆ-ಡ-ಬೇಡ
ಎದೆ-ಯ-ಲ್ಲಿ
ಎದೆ-ಯನ್ನು
ಎದೆ-ಹಾ-ಲ-ನ್ನು
ಎದೆಗೆ
ಎದೆಯ
ಎದ್ದ
ಎದ್ದು
ಎದ್ದು-ನಿಂತು
ಎದ್ದೇಳಿ
ಎದ್ದೇಳು
ಎನಿ-ಸಿ-ದರೆ
ಎನಿಸಿ-ಕೊ-ಳ್ಳುವ
ಎನು
ಎಪ್ಪ-ತ್ತನೇ
ಎಪ್ಪ-ತ್ತು
ಎಬ್ಬಿ-ಸು-ತ್ತ
ಎಬ್ಬಿಸ-ಬ-ಲ್ಲದೊ
ಎಬ್ಬಿಸ-ಬೇಕು
ಎಬ್ಬಿಸಿ
ಎಬ್ಬಿಸಿ-ಕೊಂಡು
ಎಬ್ಬಿಸಿ-ದರು
ಎಬ್ಬಿಸು
ಎಬ್ಬಿಸು-ತ್ತಿದೆ
ಎರ-ಕ-ದ-ಲ್ಲಿ
ಎರ-ಡನೆ
ಎರ-ಡನೆ-ಯ-ದ-ನ್ನು
ಎರ-ಡನೆ-ಯ-ದಾಗಿ
ಎರ-ಡನೆ-ಯ-ವ-ರೆಂದು
ಎರ-ಡನೆ-ಯ-ವರು
ಎರ-ಡನೆ-ಯ-ವರೇ
ಎರ-ಡನೆ-ಯದು
ಎರ-ಡನೆ-ಯದೆ
ಎರ-ಡನೆಯ
ಎರಡ-ನ್ನು
ಎರಡನೇ
ಎರಡೆ-ರಡು
ಎರಿಕ್
ಎರೆ-ದಳು
ಎರೆ-ದಿದ್ದರೋ
ಎರೆ-ಯು-ತ್ತೇವೆ
ಎರೆ-ಯು-ವು-ದಕ್ಕೆ
ಎರೆದಿ-ರು-ವೆನು
ಎರೆದು
ಎಲುಬಿನ
ಎಲೆ
ಎಲೆ-ಗಳಿಂದ
ಎಲೆ-ಫೆ-ನ್ಸ್ಟೋ-ನ್
ಎಲೆ-ಹಾಕಿ-ಕೊಂಡು
ಎಲೈ
ಎಳ-ನೀ-ರ-ನ್ನು
ಎಳ-ನೀರು
ಎಳ-ನೀರು-ಗಳು
ಎಳೆ
ಎಳೆ-ದ-ಮೇಲೆ
ಎಳೆ-ದ-ವ-ರ-ಲ್ಲಿ
ಎಳೆ-ದರು
ಎಳೆ-ದಾಟ
ಎಳೆ-ದು-ಕೊಂಡು
ಎಳೆ-ದು-ಕೊಳ್ಳ-ಬ-ಲ್ಲ
ಎಳೆ-ಯ-ದಾಗಿ
ಎಳೆ-ಯ-ಬೇಕೆಂದು
ಎಳೆ-ಯ-ಲ್ಪಟ್ಟ
ಎಳೆ-ಯದೆ
ಎಳೆ-ಯಲು
ಎಳೆ-ಯು-ತ್ತಿದ್ದರು
ಎಳೆ-ಯು-ತ್ತಿರು-ವ-ವ-ರಿಗೆ
ಎಳೆ-ಯು-ವ-ವ-ರ-ಲ್ಲ
ಎಳೆ-ಯು-ವುವೋ
ಎಳೆದ
ಎಳೆದು
ಎಳೆಯ
ಎಳ್ಳಷ್ಟೂ
ಎಳ್ಳು
ಎಷ್ಟ-ನ್ನು
ಎಷ್ಟರ
ಎಷ್ಟರ-ಮ-ಟ್ಟಿಗೆ
ಎಷ್ಟಿ-ತ್ತೆಂ-ದರೆ
ಎಷ್ಟು
ಎಷ್ಟು-ಕಾಲ
ಎಷ್ಟು-ಮ-ಟ್ಟಿಗೆ
ಎಷ್ಟೂ
ಎಷ್ಟೆಷ್ಟು
ಎಷ್ಟೇ
ಎಷ್ಟೊಂದು
ಎಷ್ಟೋ
ಎಷ್ಟೋ-ಎಸ-ಳನ್ನು
ಎಸೆ-ದನು
ಎಸೆ-ದಿದ್ದ
ಎಸೆದು
ಎಸೆದು-ಬಿಡು-ತ್ತಿದ್ದ
ಎಸೆಯ-ಬ-ಲ್ಲದು
ಎಸೆಯ-ಬಹುದು
ಎಸೆಯು-ತ್ತಾರೆ
ಏ
ಏಂಏ
ಏಕ
ಏಕ-ಪ್ರ-ಕಾರ-ವಾಗಿ
ಏಕ-ಪ್ರ-ಕಾರ-ವಾದ
ಏಕ-ಮನ-ಸ್ಕ-ರಾದ-ರೆಂದು
ಏಕ-ಮಾ-ತ್ರ
ಏಕ-ಮುಖ-ವಾದ
ಏಕ-ರೂಪ
ಏಕ-ಸೂ-ತ್ರ-ದ-ಲ್ಲಿ
ಏಕಾ-ಸ-ನ-ದ-ಲ್ಲಿ
ಏಕಾಗ್ರ
ಏಕಾಗ್ರ-ತೆ-ಯನ್ನು
ಏಕಾಗ್ರ-ತೆ-ಯಿಂದ
ಏಕಾಗ್ರ-ತೆಗೆ
ಏಕಾಗ್ರ-ಮಾಡಲು
ಏಕಾಗ್ರ-ಮಾಡು
ಏಕಾಗ್ರ-ವಾ-ದಾಗ
ಏಕಾಗ್ರ-ವಾಗಿ
ಏಕಾಗ್ರತೆ
ಏಕಿ-ನ್ನೂ
ಏಕಿರ-ಬಾ-ರದು
ಏಕೀ-ಕರಿ-ಸಿ-ದರು
ಏಕೆ
ಏಕೆಂ-ದರೆ
ಏಕೆಂದ-ದರೆ
ಏಗ
ಏಜಂಟಿಗೆ
ಏಟಿ-ನಿಂದ
ಏಡ-ನ್
ಏಡ-ನ್ನಿಗೆ
ಏಣಿ
ಏಣಿ-ಯ-ಲ್ಲಿ
ಏಣಿ-ಯಿಂದ
ಏತ-ಕ್ಕೆಂ-ದರೆ
ಏತ-ರಿಂದ
ಏನಾ-ಗಿದೆಯೊ
ಏನಾ-ಗು-ತ್ತಿದೆ-ಯೆಂಬುದೂ
ಏನಾ-ದರು
ಏನಾಗ-ಬಹುದು
ಏನಿ-ತ್ತೋ
ಏನಿ-ರು-ವುದು
ಏನಿ-ರು-ವುದೋ
ಏನಿದು
ಏನಿದೆ
ಏನಿರ-ಬೇಕು
ಏನು
ಏನು-ತಾನೆ
ಏನೂ
ಏನೆಂ-ದಳು
ಏನೇ
ಏನೇ-ನನ್ನು
ಏನೇನೊ
ಏನೊಂದು
ಏನೋ
ಏಪ್ರಿ-ಲ್
ಏರಲು
ಏರಿ-ದರು
ಏರಿ-ದರೆ
ಏರಿ-ದಾಗ
ಏರಿ-ಸ-ಲಿ-ಲ್ಲ
ಏರಿ-ಸು-ವೆವು
ಏರಿಸ-ಬೇಕಾಗಿದೆ
ಏರ್ಪಡಿ-ಸಿದ
ಏರ್ಪಡಿ-ಸಿದ್ದ
ಏರ್ಪಡಿ-ಸಿದ್ದರು
ಏರ್ಪಡಿ-ಸು-ವುದು
ಏರ್ಪಾಡ-ನ್ನು
ಏರ್ಪಾಡಾದ
ಏರ್ಪಾಡು
ಏರ್ಪಾಡೆ-ಲ್ಲ
ಏಳ-ಬಹುದು
ಏಳ-ಬೇಕೆ
ಏಳಲಾ-ರವು
ಏಳಿ
ಏಳಿ-ಗೆಗೆ
ಏಳಿ-ಸು-ತ್ತಿ-ದ್ದರು
ಏಳು
ಏಳು-ತ್ತಿದ್ದವು
ಏಳು-ತ್ತಿದ್ದೆ
ಏಳು-ತ್ತಿವೆ
ಏಳು-ವಂ-ತಿ-ಲ್ಲ
ಏಳು-ವನು
ಏಳು-ವು-ದ-ರಿಂದಲೇ
ಏಳು-ವು-ದಕ್ಕೆ
ಏಳು-ವು-ದಿ-ಲ್ಲ
ಏಳು-ವುದು
ಏಳುವ
ಏಳೂ-ವರೆ
ಏಷಾ
ಏಷ್ಯಾ
ಏಸು-ಕ್ರಿ-ಸ್ತ
ಏಸೆಛೆ
ಐ
ಐಂದ್ರಜಾ-ಲಿ-ಕರು
ಐಕ್ಯ-ವಾ-ಯಿತು
ಐಕ್ಯ-ವಾಗಿ
ಐಕ್ಯ-ವಾಗಿ-ದ್ದೆವು
ಐಕ್ಯ-ವಾಗು-ವುದು
ಐಕ್ಯ-ವಾಗು-ವುದೋ
ಐಕ್ಯತೆ
ಐಕ್ಯತೆ-ಯನ್ನು
ಐತಿ-ಹಾಸಿಕ
ಐದು
ಐದು-ಗಂಟೆಗೆ
ಐದು-ನೂರು
ಐದು-ವರ್ಷದ
ಐರೋಪ್ಯ
ಐರೋಪ್ಯ-ರ-ನ್ನು
ಐರೋಪ್ಯ-ರಿಂದ
ಐರೋಪ್ಯ-ರೊ-ಡನೆ
ಐರೋಪ್ಯ-ರೊಂದಿಗೆ
ಐರೋಪ್ಯರ
ಐರೋಪ್ಯರು
ಐರೋಪ್ಯರೂ
ಐರ್ಲೆಂಡಿ-ನ-ಲ್ಲಿದ್ದ
ಐಶ್ವರ್ಯ
ಐಶ್ವರ್ಯ-ಗ-ಳನ್ನು
ಐಶ್ವರ್ಯ-ಗ-ಳಾದರೊ
ಐಶ್ವರ್ಯ-ಗ-ಳಿದ್ದರೂ
ಐಶ್ವರ್ಯ-ದ-ಲ್ಲಿ
ಐಶ್ವರ್ಯ-ದಿಂದ
ಐಶ್ವರ್ಯ-ಮಂಜೂಷೆಯಿ-ನ್ನೂ
ಐಶ್ವರ್ಯ-ವನ್ನು
ಐಶ್ವರ್ಯ-ವಾಗಿ-ದ್ದ-ನೆಂಬು-ದ-ನ್ನು
ಐಶ್ವರ್ಯಕ್ಕೆ
ಐಸಿ-ಎ-ಸ್
ಐಹಿಕ
ಐಹಿಕ-ಸುಖ
ಐಹಿಕ-ಸುಖ-ವನ್ನು
ಒ
ಒಂಒ
ಒಂಟಿ-ಯಾಗಿ
ಒಂಟೆ
ಒಂಟೆಯ
ಒಂದ-ನ್ನೊಂದು
ಒಂದ-ರ-ಲ್ಲಿ
ಒಂದ-ರ-ಲ್ಲಿಯೇ
ಒಂದ-ರ-ಲ್ಲೆ
ಒಂದ-ರಂತೆ
ಒಂದ-ರದೋ
ಒಂದ-ರಿಂದಲೇ
ಒಂದಿಗೆ
ಒಂದಿಬ್ಬರು
ಒಂದಿಷ್ಟು
ಒಂದು
ಒಂದು-ಗೂ-ಡು-ತ್ತವೆ
ಒಂದು-ಗೂಡ-ಬೇ-ಕಾ-ದರೆ
ಒಂದು-ಗೂಡ-ಬೇಕು
ಒಂದು-ಗೂಡಿ-ಸುವ
ಒಂದು-ಗೂಡಿ-ಸುವಂತೆ
ಒಂದು-ಗೂಡಿಸ-ಬೇಕೆ-ನ್ನಿ-ಸು-ವುದು
ಒಂದು-ಗೂಡಿಸಿ
ಒಂದು-ಗೂಡಿಸಿದ
ಒಂದು-ದಿನ
ಒಂದು-ಮಾಡಿದ
ಒಂದು-ಸಲ
ಒಂದೂ
ಒಂದೂ-ವರೆ
ಒಂದೆ
ಒಂದೆ-ರಡು
ಒಂದೆಡೆ
ಒಂದೇ
ಒಂದೊಂ-ದ-ನ್ನು
ಒಂದೊಂ-ದಕ್ಕೆ
ಒಂದೊಂ-ದಾಗಿ
ಒಂಬ-ತ್ತು
ಒಂಭ-ತ್ತು
ಒಂಭ-ತ್ತೂ-ವರೆ
ಒಕ್ಕಲು-ತನ
ಒಗೆದು
ಒಗ್ಗದೆ
ಒಗ್ಗದೇ
ಒಗ್ಗಿ-ಕೊಂಡು
ಒಗ್ಗಿ-ದರೆ
ಒಗ್ಗು-ತ್ತದೆ
ಒಗ್ಗು-ವು-ದಿ-ಲ್ಲ
ಒಗ್ಗು-ವು-ದೆಂದು
ಒಡ-ನೆಯೆ
ಒಡಂಬಡ-ಲಿ-ಲ್ಲ
ಒಡಕು
ಒಡನಾಡಿ-ಯಾಗಿ-ದ್ದಂಥ-ವರ
ಒಡವೆ-ಗಳು
ಒಡೆ-ಯ-ರಿಗೆ
ಒಡೆ-ಯರ-ವ-ರಿಗೆ
ಒಡೆ-ಯಲು
ಒಡೆ-ಯಿತು
ಒಡೆದ
ಒಡೆದು
ಒಡೆದು-ಹಾಕಿ
ಒಡೆದು-ಹಾಕು-ವು-ದಕ್ಕೆ
ಒಡ್ಡು-ವುದು
ಒಣ
ಒಣ-ಗಿದೆ
ಒಣ-ತ-ತ್ತ್ವ
ಒಣಗಲು
ಒಣಗಿ
ಒಣಗಿ-ಸ-ಲಾ-ರದು
ಒಣಗಿ-ಸಿದ
ಒಣಗಿತು
ಒಣಗಿದ
ಒದಗಿ
ಒದಗಿ-ತ್ತು
ಒದಗಿ-ದ್ದರೂ
ಒದಗಿ-ಬ-ರುವ
ಒದಗಿ-ಸ-ಬಹುದು
ಒದಗಿ-ಸ-ಬೇಕಾಗಿ-ತ್ತು
ಒದಗಿ-ಸಿ-ಕೊ-ಟ್ಟಳು
ಒದಗಿ-ಸಿ-ಕೊ-ಡು-ತ್ತ
ಒದಗಿ-ಸಿ-ಕೊಡು-ವುದು
ಒದಗಿ-ಸಿ-ದನು
ಒದಗಿ-ಸಿ-ರ-ಲಾ-ರದು
ಒದಗಿ-ಸಿ-ರು-ವುದು
ಒದಗಿ-ಸಿದ
ಒದಗಿ-ಸು-ತ್ತವೆ
ಒದಗಿ-ಸು-ತ್ತಿ-ರುವರು
ಒದಗಿ-ಸು-ತ್ತಿದೆ
ಒದಗಿ-ಸುವ
ಒದಗಿ-ಸುವನು
ಒದಗಿತು
ಒದಗು-ತ್ತಿ-ತ್ತೋ
ಒದು
ಒದು-ವು-ದಕ್ಕೆ
ಒದೆದು
ಒದ್ದಾ-ಡು-ತ್ತ
ಒದ್ದಾಡು-ತ್ತಿ-ತ್ತು
ಒದ್ದೆ
ಒಪ್ಪ-ಬೇಕಾ-ಯಿತು
ಒಪ್ಪ-ಲಿ-ಲ್ಲ
ಒಪ್ಪದೇ
ಒಪ್ಪಿ-ಕೊ-ಳ್ಳದೆ
ಒಪ್ಪಿ-ಕೊ-ಳ್ಳದೇ
ಒಪ್ಪಿ-ಕೊ-ಳ್ಳಲಿ-ಲ್ಲ
ಒಪ್ಪಿ-ಕೊ-ಳ್ಳಲು
ಒಪ್ಪಿ-ಕೊ-ಳ್ಳು-ತ್ತಿ-ರಲಿ-ಲ್ಲ
ಒಪ್ಪಿ-ಕೊ-ಳ್ಳು-ತ್ತಿ-ರುವರು
ಒಪ್ಪಿ-ಕೊ-ಳ್ಳು-ತ್ತಿ-ಲ್ಲ
ಒಪ್ಪಿ-ಕೊ-ಳ್ಳು-ತ್ತೇವೆ
ಒಪ್ಪಿ-ಕೊ-ಳ್ಳು-ವು-ದ-ಕ್ಕಾಗಿ
ಒಪ್ಪಿ-ಕೊ-ಳ್ಳು-ವು-ದಕ್ಕೆ
ಒಪ್ಪಿ-ಕೊ-ಳ್ಳು-ವು-ದಿ-ಲ್ಲ
ಒಪ್ಪಿ-ಕೊ-ಳ್ಳುವ
ಒಪ್ಪಿ-ಕೊ-ಳ್ಳುವಂತಹ-ವ-ರ-ಲ್ಲ
ಒಪ್ಪಿ-ಕೊ-ಳ್ಳುವಂತಹ-ವರು
ಒಪ್ಪಿ-ಕೊ-ಳ್ಳುವಂತೆ
ಒಪ್ಪಿ-ಕೊ-ಳ್ಳುವನು
ಒಪ್ಪಿ-ಕೊ-ಳ್ಳುವುದು
ಒಪ್ಪಿ-ಕೊಂಡ
ಒಪ್ಪಿ-ಕೊಂಡನು
ಒಪ್ಪಿ-ಕೊಂಡರು
ಒಪ್ಪಿ-ಕೊಂಡರೂ
ಒಪ್ಪಿ-ಕೊಂಡರೆ
ಒಪ್ಪಿ-ಕೊಂಡಾಗ
ಒಪ್ಪಿ-ಕೊಂಡಾಗಲೇ
ಒಪ್ಪಿ-ಕೊಂಡಿ-ರುವುದು
ಒಪ್ಪಿ-ಕೊಂಡಿದ್ದು
ಒಪ್ಪಿ-ಕೊಂಡು
ಒಪ್ಪಿ-ಕೊಳ್ಳ-ದಿ-ರು-ವುದು
ಒಪ್ಪಿ-ಕೊಳ್ಳ-ಬೇಕಾಗಿ-ತ್ತು
ಒಪ್ಪಿ-ಕೊಳ್ಳ-ಬೇಕು
ಒಪ್ಪಿ-ಕೊಳ್ಳಬೇ-ಕೇನು
ಒಪ್ಪಿ-ಕೊಳ್ಳಲೇ-ಬೇಕಾ-ಯಿತು
ಒಪ್ಪಿ-ಕೊಳ್ಳಲೇ-ಬೇಕು
ಒಪ್ಪಿ-ಕೊಳ್ಳಿ
ಒಪ್ಪಿ-ದರು
ಒಪ್ಪಿ-ದರೂ
ಒಪ್ಪಿ-ದರೆ
ಒಪ್ಪಿ-ಸಲು
ಒಪ್ಪಿ-ಸು-ವುದು
ಒಪ್ಪಿ-ಸು-ವುದೇ
ಒಪ್ಪಿಕೋ
ಒಪ್ಪಿಗೆ
ಒಪ್ಪಿಗೆ-ಯನ್ನು
ಒಪ್ಪಿಗೆ-ಯಾ-ಯಿತು
ಒಪ್ಪಿಸಿ
ಒಪ್ಪಿಸಿ-ದ-ನಂ-ತರ
ಒಪ್ಪಿಸಿ-ದನು
ಒಪ್ಪಿಸಿ-ದರು
ಒಪ್ಪು-ತ್ತಿ-ರಲಿ-ಲ್ಲ
ಒಪ್ಪು-ತ್ತಿ-ಲ್ಲ
ಒಪ್ಪು-ತ್ತಿದ್ದರು
ಒಪ್ಪು-ವು-ದಿ-ಲ್ಲ
ಒಪ್ಪು-ವು-ದಿ-ಲ್ಲ-ವೆಂದೂ
ಒಪ್ಪು-ವುದು
ಒಪ್ಪುವ
ಒಬ
ಒಬ್ಬ
ಒಬ್ಬ-ನ-ಲ್ಲಿ
ಒಬ್ಬ-ನನ್ನು
ಒಬ್ಬ-ನಿ-ರುವಾಗ
ಒಬ್ಬ-ನಿಂದ
ಒಬ್ಬ-ನಿಗೆ
ಒಬ್ಬ-ನೆಂದು
ಒಬ್ಬ-ರ-ನ್ನಾಗಿ
ಒಬ್ಬ-ರ-ನ್ನು
ಒಬ್ಬ-ರ-ನ್ನೊಬ್ಬರು
ಒಬ್ಬ-ರ-ಲ್ಲಿ
ಒಬ್ಬ-ರಂತೆ
ಒಬ್ಬ-ರಾಗಿ-ದ್ದರು
ಒಬ್ಬ-ರಾದ
ಒಬ್ಬ-ರಿಗೂ
ಒಬ್ಬ-ರಿಗೆ
ಒಬ್ಬನ
ಒಬ್ಬನು
ಒಬ್ಬನೇ
ಒಬ್ಬರ
ಒಬ್ಬರು
ಒಬ್ಬರೆ
ಒಬ್ಬರೇ
ಒಬ್ಬಳು
ಒಬ್ಬಳೇ
ಒಬ್ಬಾಕೆ
ಒಬ್ಬಿಬ್ಬರು
ಒಬ್ಬೊಬ್ಬ
ಒಬ್ಬೊಬ್ಬ-ನದು
ಒಬ್ಬೊಬ್ಬ-ರ-ನ್ನಾಗಿ
ಒಬ್ಬೊಬ್ಬರು
ಒಬ್ಬೊಬ್ಬರೂ
ಒಯ್ದನು
ಒಯ್ದರು
ಒಯ್ದಿತು
ಒಯ್ದಿದೆ
ಒಯ್ದಿದ್ದರು
ಒಯ್ದು
ಒಯ್ಯಲು
ಒಯ್ಯಿ
ಒಯ್ಯು-ತ್ತವೆ
ಒಯ್ಯು-ವುದು
ಒಯ್ಯು-ವುದೇ
ಒಯ್ಯು-ವುದೊ
ಒಯ್ಯು-ವುದೋ
ಒಯ್ಯು-ವುವು
ಒಯ್ಯುವ
ಒಯ್ಯುವ-ನೆಂದು
ಒರ-ಸಿ-ದರು
ಒರಗಿ-ಕೊಂಡು
ಒರಟು
ಒರೆ-ಗ-ಲ್ಲಿಗೆ
ಒರೆಗ-ಲ್ಲ-ನ್ನು
ಒಲಿ-ಯು-ವಂತೆ
ಒಲಿ-ಯು-ವುದು
ಒಲಿ-ಯುವನು
ಒಲಿಸಿ-ಕೊ-ಳ್ಳಲು
ಒಲಿಸಿ-ಕೊ-ಳ್ಳು-ವು-ದಕ್ಕೆ
ಒಲಿಸಿ-ಕೊಂಡನು
ಒಲಿಸಿ-ಕೊಂಡರು
ಒಲೆ
ಒಲೆ-ಯ-ಮೇಲೆ
ಒಲೆ-ಯ-ಲ್ಲಿ
ಒಲೆಗೆ
ಒಳ
ಒಳ-ಕೊ-ಳ್ಳಲು
ಒಳ-ಗಡೆ
ಒಳ-ಗಾ-ಯಿತು
ಒಳ-ಗಾಗಿ-ರುವೆಯಾ
ಒಳ-ಗಿ-ಳಿದು
ಒಳ-ಗೆ-ಲ್ಲ
ಒಳ-ಗೆ-ಲ್ಲಾ
ಒಳ-ಗೊ-ಳ್ಳುವುದು
ಒಳ-ಗೊಂಡಿ-ದ್ದವು
ಒಳ-ಗೊಂಡಿ-ರು-ವುದು
ಒಳ-ಗೊಂಡಿ-ರುವ-ವೆಂದು
ಒಳ-ಗೊಂಡಿವೆ
ಒಳ-ಗೊಂಡು
ಒಳ-ಭಾಗ
ಒಳ-ಭಾಗ-ವೇಕೆ
ಒಳ-ಹೊಕ್ಕು
ಒಳಕ್ಕೆ
ಒಳಗಾ-ದು-ದಕ್ಕೆ
ಒಳಗಾಗ-ಕೂ-ಡದು
ಒಳಗಾಗ-ಬೇಕಾಗುವುದು
ಒಳಗಾಗು-ತ್ತಿ-ತ್ತು
ಒಳಗಿ-ನಿಂದ
ಒಳಗಿ-ನಿಂದಲೇ
ಒಳಗಿ-ರುವ
ಒಳಗಿನ
ಒಳಗೆ
ಒಳಗೇ
ಒಳಗೊ-ಳ್ಳುವ
ಒಳಗೊಳಗೇ
ಒಳಹೊಕ್ಕಿ-ಲ್ಲ
ಒಳೆಯ-ದೆಂದು
ಒಳ್ಳೆ
ಒಳ್ಳೆ-ಯ-ದ-ನ್ನು
ಒಳ್ಳೆ-ಯ-ದ-ನ್ನೂ
ಒಳ್ಳೆ-ಯ-ದ-ನ್ನೆ-ಲ್ಲ
ಒಳ್ಳೆ-ಯ-ದ-ನ್ನೇ
ಒಳ್ಳೆ-ಯ-ದ-ನ್ನೋ
ಒಳ್ಳೆ-ಯ-ದ-ಲ್ಲ
ಒಳ್ಳೆ-ಯ-ದಕ್ಕೊ
ಒಳ್ಳೆ-ಯ-ದಾ-ಗಿದೆಯೆ
ಒಳ್ಳೆ-ಯ-ದಾ-ಯಿತು
ಒಳ್ಳೆ-ಯ-ದಾ-ವುದೂ
ಒಳ್ಳೆ-ಯ-ದಾಗ-ಬೇ-ಕಾ-ದರೆ
ಒಳ್ಳೆ-ಯ-ದಾಗ-ಲೆಂದು
ಒಳ್ಳೆ-ಯ-ದಾಗಿ-ದ್ದರೆ
ಒಳ್ಳೆ-ಯ-ದಾಗಿ-ರು-ವುದು
ಒಳ್ಳೆ-ಯ-ದಾಗು-ತ್ತದೆ
ಒಳ್ಳೆ-ಯ-ದಾಗು-ವಂತಹ
ಒಳ್ಳೆ-ಯ-ದಾಗು-ವು-ದೆಂದು
ಒಳ್ಳೆ-ಯ-ದಾಗು-ವುದು
ಒಳ್ಳೆ-ಯ-ದಿವೆ
ಒಳ್ಳೆ-ಯ-ದೆಂದು
ಒಳ್ಳೆ-ಯ-ವ-ನಾಗು
ಒಳ್ಳೆ-ಯ-ವ-ರೆಂದು
ಒಳ್ಳೆ-ಯ-ವ-ರೊ-ಡನೆ
ಒಳ್ಳೆ-ಯ-ವ-ಳಾಗು
ಒಳ್ಳೆ-ಯ-ವನು
ಒಳ್ಳೆ-ಯ-ವರು
ಒಳ್ಳೆ-ಯ-ವರೇ
ಒಳ್ಳೆ-ಯದು
ಒಳ್ಳೆ-ಯದೂ
ಒಳ್ಳೆ-ಯದೆ
ಒಳ್ಳೆ-ಯದೇ
ಒಳ್ಳೆ-ಯದೋ
ಒಳ್ಳೆ-ಯವೂ
ಒಳ್ಳೆ-ಯವೆ
ಒಳ್ಳೆಯ
ಒಳ್ಳೊಳ್ಳೆಯ
ಒಸಾಕ
ಓ
ಓಂ
ಓಔತ-ಣದ
ಓಕಾಕುರ
ಓಕ್ಲೆಂಡಿಗೆ
ಓಜ-ಸ್ಸನ್ನು
ಓಜ-ಸ್ಸು
ಓಜೋ
ಓಜೋ-ಮಯಿ
ಓಡ-ತೊಡಗಿದರು
ಓಡ-ಬೇಡಿ
ಓಡ-ಲಿ-ಲ್ಲ
ಓಡಾ
ಓಡಾ-ಡು-ತ್ತ
ಓಡಾ-ಡು-ತ್ತಿ-ರು-ವಿರಿ
ಓಡಾ-ಡು-ತ್ತಿದೆ
ಓಡಾ-ಡುವ
ಓಡಿ
ಓಡಿ-ದರು
ಓಡಿ-ಬ-ರು-ತ್ತಿದ್ದ
ಓಡಿ-ಬ-ರು-ತ್ತಿದ್ದರು
ಓಡಿ-ಬ-ರು-ತ್ತಿದ್ದೆ
ಓಡಿ-ಬಂದ
ಓಡಿ-ಬಂದು
ಓಡಿ-ಬಂದೆ
ಓಡಿ-ಯಾ-ಡು-ತ್ತ
ಓಡಿ-ಯಾ-ಡು-ತ್ತಿದೆ
ಓಡಿ-ಸಿ-ಕೊಂಡು
ಓಡಿ-ಸಿ-ಬಿಡ-ಬಹುದು
ಓಡಿ-ಹೋ-ಯಿತು
ಓಡಿ-ಹೋಗ-ಬಹುದು
ಓಡಿ-ಹೋಗ-ಬೇಡಿ
ಓಡಿ-ಹೋಗಿ
ಓಡಿ-ಹೋಗು-ವನು
ಓಡಿ-ಹೋಗು-ವುದ-ರ-ಲ್ಲಿದ್ದ
ಓಡಿ-ಹೋದಳು
ಓಡಿದ
ಓಡಿದೆ
ಓಡಿಸಿ
ಓಡು
ಓಡು-ತ್ತಿದೆ
ಓಡು-ತ್ತಿರು-ವುದು
ಓಡು-ತ್ತಿರು-ವೆವು
ಓತಪ್ರೋ-ತ-ವಾಗಿ
ಓತಪ್ರೋ-ತ-ವಾಗಿ-ತ್ತು
ಓತಪ್ರೋತ-ನಾಗಿ-ರು-ವ-ನೆಂದೂ
ಓತಪ್ರೋತ-ನಾದೆ
ಓತಪ್ರೋತ-ರಾಗು-ವು-ದಕ್ಕೆ
ಓತಪ್ರೋತ-ವಾಗು-ವುದು
ಓತಪ್ರೋತ-ವಾದ
ಓದ-ತೊಡಗಿದರು
ಓದ-ಬಹುದು
ಓದ-ಬೇಕಾಗಿ-ದ್ದು-ದ-ರಿಂದ
ಓದ-ಬೇಕೆಂದು
ಓದ-ಲ್ಪ-ಡು-ತ್ತದೆ
ಓದದೆ
ಓದಲು
ಓದಲು-ಪ-ಕ್ರಮಿಸಿದ
ಓದಲೇ-ಬೇಕೆಂದೂ
ಓದಿ
ಓದಿ-ದ-ವ-ರ-ಲ್ಲ
ಓದಿ-ದ-ವರು
ಓದಿ-ದನು
ಓದಿ-ದರು
ಓದಿ-ದರೂ
ಓದಿ-ದರೆ
ಓದಿ-ದರೋ
ಓದಿ-ದಾಗ
ಓದಿ-ದು-ದ-ನ್ನು
ಓದಿ-ದ್ದೀರಿ
ಓದಿ-ದ್ದು-ದ-ರಿಂದ
ಓದಿ-ದ್ದೇನೆ
ಓದಿ-ಬಿ-ಟ್ಟರೆ
ಓದಿ-ಯಾ-ದರೂ
ಓದಿ-ರ-ಲಿ-ಲ್ಲ
ಓದಿ-ರು-ವನು
ಓದಿ-ರು-ವಿರಾ
ಓದಿ-ರು-ವಿರಿ
ಓದಿ-ರು-ವು-ದಾಗಿಯೂ
ಓದಿ-ರು-ವೆವು
ಓದಿ-ರುವ
ಓದಿ-ರುವರು
ಓದಿ-ರುವು-ದ-ನ್ನೂ
ಓದಿ-ಲ್ಲವೆ
ಓದಿದ
ಓದಿಯೇ
ಓದಿಸಿ
ಓದು
ಓದು-ಗ-ರ-ನ್ನು
ಓದು-ಗ-ರಿಗೆ
ಓದು-ಗರ
ಓದು-ಗರು
ಓದು-ತ್ತ
ಓದು-ತ್ತಿದ್ದ
ಓದು-ತ್ತಿದ್ದ-ವರು
ಓದು-ತ್ತಿದ್ದನು
ಓದು-ತ್ತಿದ್ದರು
ಓದು-ತ್ತಿದ್ದರೆ
ಓದು-ತ್ತಿದ್ದಾಗ
ಓದು-ತ್ತಿದ್ದೆ-ವೆಂದೂ
ಓದು-ತ್ತಿರು-ವೆವು
ಓದು-ತ್ತಿರುವೆ
ಓದು-ತ್ತೇ-ವ-ಲ್ಲ
ಓದು-ತ್ತೇನೆ
ಓದು-ವಂತೆ
ಓದು-ವರು
ಓದು-ವಾಗ
ಓದು-ವು-ದ-ನ್ನು
ಓದು-ವು-ದ-ರಿಂದ
ಓದು-ವು-ದ-ರಿಂದಲೇ
ಓದು-ವು-ದಕ್ಕೆ
ಓದು-ವು-ದಿ-ಲ್ಲ-ವೆಂದೂ
ಓದು-ವುದ-ರ-ಲ್ಲಿ
ಓದು-ವುದು
ಓದು-ವುದೆಂ-ದರೆ
ಓದು-ವುದೇ
ಓದು-ವುದೇಕೆ
ಓದು-ವೆನು
ಓದುವ
ಓದುವೆ
ಓಪಿಯಂ
ಓರಗೆ
ಓರಗೆ-ಯ-ವ-ರೊಂದಿಗೆ
ಓರೇ
ಓಲಗದ-ವರ
ಓಲೆಗ-ರಿಯ
ಓಷ್ಠಾ-ಧರ
ಔ
ಔತಣ-ವನ್ನು
ಔಪ-ಮಾನಿ-ಕ-ವಾಗಿ
ಔಷಧ
ಔಷಧಿ
ಔಷಧಿ-ಗ-ಳನ್ನು
ಔಷಧಿ-ಗಳು
ಔಷಧಿ-ಯನ್ನು
ಔಷಧೋಪ-ಚಾರ
ಕ
ಕಂ
ಕಂಕಣ-ನಾಗು
ಕಂಕುಳ-ಲ್ಲಿ
ಕಂಕುಳಿ-ನ-ಲ್ಲಿ
ಕಂಟಕ
ಕಂಟಕ-ಗಳಿವೆ
ಕಂಟಕ-ಪ್ರಾಯ-ರಂತೆ
ಕಂಟಕ-ಮ-ಯದ
ಕಂಠ-ದ-ಲ್ಲಿ
ಕಂಠ-ದಿಂದ
ಕಂಠ-ಪಾಠ
ಕಂಠ-ಪಾಠ-ಮಾಡಿ
ಕಂಠ-ವಿದ್ದುದು
ಕಂಠವೂ
ಕಂಠವೋ
ಕಂಡ
ಕಂಡ-ಕೂಡಲೆ
ಕಂಡ-ದ್ದ-ನ್ನು
ಕಂಡ-ಮೇಲೆ
ಕಂಡ-ರೇನೇ
ಕಂಡ-ವ-ರೆ-ಲ್ಲ
ಕಂಡ-ವನು
ಕಂಡ-ವರು
ಕಂಡಂತೆ
ಕಂಡರು
ಕಂಡರೂ
ಕಂಡರೆ
ಕಂಡರೋ
ಕಂಡಳು
ಕಂಡವು
ಕಂಡಾಗ
ಕಂಡಾದ
ಕಂಡಾದ-ಮೇಲೆ
ಕಂಡಿ
ಕಂಡಿ-ಗಿಂತ
ಕಂಡಿ-ದ್ದರು
ಕಂಡಿ-ದ್ದರೋ
ಕಂಡಿ-ದ್ದೇನೆಂದು
ಕಂಡಿ-ರ-ಬೇಕು
ಕಂಡಿ-ರ-ಲಿ-ಲ್ಲ
ಕಂಡಿ-ರು-ವಿರಾ
ಕಂಡಿ-ರು-ವೆನು
ಕಂಡಿ-ಲ್ಲ
ಕಂಡಿ-ಲ್ಲದೇ
ಕಂಡಿತು
ಕಂಡಿತೆ
ಕಂಡಿಯ
ಕಂಡಿರಿ
ಕಂಡು
ಕಂಡು-ಕೊಂಡಿದ್ದೇನೆ
ಕಂಡು-ದ-ನ್ನು
ಕಂಡು-ದ-ನ್ನೆ-ಲ್ಲ
ಕಂಡು-ಬ-ರು-ತ್ತದೆ
ಕಂಡು-ಬ-ರು-ತ್ತವೆ
ಕಂಡು-ಬ-ರು-ವು-ದಿ-ಲ್ಲ
ಕಂಡು-ಬ-ರು-ವುದು
ಕಂಡು-ಬಂ-ದವು
ಕಂಡು-ಬಂದರು
ಕಂಡು-ಬಂದರೆ
ಕಂಡು-ಬಂದಳು
ಕಂಡು-ಬಂದಿತು
ಕಂಡು-ಬರು-ತ್ತಿದೆ
ಕಂಡು-ಬರು-ತ್ತಿದ್ದಾನೆ
ಕಂಡು-ಹಿಡಿ-ದ-ವ-ರ-ನ್ನು
ಕಂಡು-ಹಿಡಿ-ದಂತೆ
ಕಂಡು-ಹಿಡಿ-ದರು
ಕಂಡು-ಹಿಡಿ-ದಿ-ಲ್ಲ-ವೆಂದೂ
ಕಂಡು-ಹಿಡಿ-ಯ-ಬೇಕೆಂದು
ಕಂಡು-ಹಿಡಿ-ಯಲು
ಕಂಡು-ಹಿಡಿ-ಯು-ತ್ತ
ಕಂಡು-ಹಿಡಿ-ಯು-ತ್ತಾನೆ
ಕಂಡು-ಹಿಡಿ-ಯು-ತ್ತಿದ್ದರು
ಕಂಡು-ಹಿಡಿ-ಯು-ತ್ತಿರು-ವರು
ಕಂಡು-ಹಿಡಿ-ಯು-ತ್ತಿರುವೆ
ಕಂಡು-ಹಿಡಿ-ಯು-ವು-ದ-ನ್ನು
ಕಂಡು-ಹಿಡಿ-ಯು-ವು-ದಕ್ಕೆ
ಕಂಡು-ಹಿಡಿ-ಯು-ವುದು
ಕಂಡು-ಹಿಡಿ-ಯು-ವುದೇಕೆ
ಕಂಡು-ಹಿಡಿ-ಯುವ
ಕಂಡು-ಹಿಡಿ-ಯುವನು
ಕಂಡು-ಹಿಡಿ-ಯುವುದ-ರ-ಲ್ಲಿ
ಕಂಡು-ಹಿಡಿದ
ಕಂಡು-ಹಿಡಿದು
ಕಂಡೆ
ಕಂಡೆಯಾ
ಕಂಡೇ
ಕಂಡೊ-ಡನೆ
ಕಂಡೊಡ-ನೆಯೆ
ಕಂಡೊಡ-ನೆಯೇ
ಕಂತೆ
ಕಂತೆ-ಯ-ಲ್ಲ
ಕಂದಕ-ವನ್ನು
ಕಂದಾ-ಚಾರ
ಕಂದು-ಬಣ್ಣ
ಕಂದು-ಬಣ್ಣದ
ಕಂಪನಿ
ಕಂಪನಿಯ
ಕಂಪಿ-ಸು-ತ್ತಿ-ತ್ತು
ಕಂಪಿ-ಸು-ವುದು
ಕಂಬ-ಗ-ಳನ್ನು
ಕಂಬ-ಗಳ
ಕಂಬ-ಗಳು
ಕಂಬ-ಗಳುಳ್ಳ
ಕಂಬ-ವನ್ನು
ಕಂಬಕ್ಕೆ
ಕಂಬನಿ
ಕಂಬನಿ-ಗ-ರೆಯು-ತ್ತಿ-ದ್ದು-ದ-ನ್ನು
ಕಂಬನಿ-ಗರೆ-ದರೂ
ಕಂಬನಿ-ದುಂಬಿ
ಕಂಬನಿ-ಯಿಂದ
ಕಂಬಿ-ಗ-ಳನ್ನು
ಕಂಬಿ-ಯನ್ನು
ಕಂಸೂ-ಲ್
ಕಕ್ಕ-ಸನ್ನು
ಕಕ್ಕಾಬಿಕ್ಕಿ-ಯಾ-ದರು
ಕಕ್ಕಾಬಿಕ್ಕಿ-ಯಾಗಿ
ಕಕ್ಕಾಬಿಕ್ಕಿ-ಯಾಗು-ತ್ತಾನೋ
ಕಕ್ಕಾಬಿಕ್ಕಿ-ಯಾದ
ಕಕ್ಷಿ
ಕಕ್ಷಿ-ಗಳು
ಕಕ್ಷಿ-ಯವ-ನನ್ನು
ಕಚ್ಚ-ಕೂ-ಡದು
ಕಚ್ಚಲಿ-ಲ್ಲ-ವೆಂದು
ಕಚ್ಚಾಡು-ತ್ತಿ-ರು-ವು-ದ-ನ್ನು
ಕಚ್ಚಿ-ದರೂ
ಕಚ್ಚಿನ್ನೋ-ಭಯ-ವಿಭ್ರಷ್ಟಃಛಿ-ನ್ನಾ-ಭ್ರಮಿವ
ಕಛೇ-ರಿಗೆ
ಕಛೇರಿ
ಕಛೇರಿ-ಗಳ-ಲ್ಲಿ
ಕಟ-ಬ-ಲ್ಲೆ-ಯಾ-ದರೆ
ಕಟು
ಕಟು-ಕ-ರಂತೆ
ಕಟು-ಕ-ರಿಂದ
ಕಟು-ಕನ
ಕಟು-ಕರ
ಕಟು-ನಿಂದೆ-ಯನ್ನು
ಕಟು-ವಾಗಿ
ಕಟು-ವಾದ
ಕಠಿಣ
ಕಠಿಣ-ತಮ-ವಾದುದೆ
ಕಠಿಣ-ವಾದ
ಕಠಿಣವೋ
ಕಠೊರ-ತಮ
ಕಠೋರ
ಕಠೋರ-ವಾಗಿ
ಕಡ-ಲಿ-ನಿಂದ
ಕಡ-ಲಿನ
ಕಡಮೆ
ಕಡಲೇ-ಬೇಳೆ
ಕಡಿ-ದಾದ
ಕಡಿಮೆ
ಕಡಿಮೆ-ಮಾಡ-ಬ-ಲ್ಲಿರಾ
ಕಡಿಮೆ-ಯ-ಲ್ಲಿದೆ
ಕಡಿಮೆ-ಯಾ-ಗು-ತ್ತಿದೆ
ಕಡಿಮೆ-ಯಾಗ-ಬೇಕೆಂದು
ಕಡಿಮೆ-ಯಾಗ-ಲಿ-ಲ್ಲ
ಕಡಿಮೆ-ಯಾಗ-ಲಿ-ಲ್ಲವೋ
ಕಡಿಮೆ-ಯಾಗಿ
ಕಡಿಮೆ-ಯಾಗಿ-ತ್ತು
ಕಡಿಮೆ-ಯಾಗಿ-ರ-ಲಿ-ಲ್ಲ
ಕಡಿಮೆ-ಯಾಗಿ-ರು-ವು-ದ-ನ್ನು
ಕಡಿಮೆ-ಯಾಗು-ತ್ತ
ಕಡಿಮೆ-ಯಾಗುವುದು
ಕಡಿಮೆ-ಯಾಗುವುದೋ
ಕಡಿಮೆ-ಯೇನೂ
ಕಡು-ಕಷ್ಟ-ವೆಂಬ
ಕಡುಗೆಂಪಿನ
ಕಡೆ
ಕಡೆ-ಗಳ-ಲ್ಲಿ
ಕಡೆ-ಗಳ-ಲ್ಲಿಯೂ
ಕಡೆ-ಗಳ-ಲ್ಲೂ
ಕಡೆ-ಗಳಂತೆ
ಕಡೆ-ಗಳಿ-ಗಿಂತಲೂ
ಕಡೆ-ಗಳಿಂದ
ಕಡೆ-ಗಿಂತ
ಕಡೆ-ಗಿಂತಲೂ
ಕಡೆ-ಗಿಂದು
ಕಡೆ-ಯ-ಪಕ್ಷ
ಕಡೆ-ಯ-ಲ್ಲಿ
ಕಡೆ-ಯ-ಲ್ಲಿಯೂ
ಕಡೆ-ಯ-ಲ್ಲಿಯೇ
ಕಡೆ-ಯ-ಲ್ಲೆ
ಕಡೆ-ಯ-ವರ
ಕಡೆ-ಯ-ವರು
ಕಡೆ-ಯಾಗಿ
ಕಡೆ-ಯಿಂದ
ಕಡೆ-ಯಿಂದಲೂ
ಕಡೆಗೇ
ಕಡೆಯ
ಕಡೆಯೂ
ಕಡೆಯೇ
ಕಡ್ಡಿ
ಕಡ್ಡಿ-ಯನ್ನು
ಕಡ್ಡಿಯ
ಕಣಿ-ವೆಯ
ಕಣಿವೆ
ಕಣಿವೆ-ಯ-ಲ್ಲಿ
ಕಣಿವೆ-ಯೊಳಗೆ
ಕಣು-ಗ-ಳನ್ನು
ಕಣ್ಕಡೆ-ಯ-ಲ್ಲಿ
ಕಣ್ಣ
ಕಣ್ಣ-ನ್ನು
ಕಣ್ಣ-ನ್ನೂ
ಕಣ್ಣ-ಮುಂದೆ
ಕಣ್ಣಾರೆ
ಕಣ್ಣಿ-ಟ್ಟಿದ್ದರು
ಕಣ್ಣಿ-ದೆಯೋ
ಕಣ್ಣಿ-ನ-ಲ್ಲಿ
ಕಣ್ಣಿ-ನಿಂದ
ಕಣ್ಣಿ-ನೊಳಗೆ
ಕಣ್ಣಿಗೂ
ಕಣ್ಣಿಗೆ
ಕಣ್ಣಿದೆ
ಕಣ್ಣಿನ
ಕಣ್ಣೀ-ರ-ನ್ನು
ಕಣ್ಣೀರು
ಕಣ್ಣು
ಕಣ್ಣು-ಕ-ಟ್ಟಿನ
ಕಣ್ಣು-ಗ-ಳನ್ನು
ಕಣ್ಣು-ಗಳ
ಕಣ್ಣು-ಗಳ-ನ್ನೂ
ಕಣ್ಣು-ಗಳ-ನ್ನೇ
ಕಣ್ಣು-ಗಳ-ಲ್ಲಿ
ಕಣ್ಣು-ಗಳಿಂದ
ಕಣ್ಣು-ಗಳಿವೆ
ಕಣ್ಣು-ಗಳು
ಕಣ್ಣು-ಗಳೂ
ಕಣ್ಣೆ-ತ್ತಿ
ಕಣ್ಣೆ-ದು-ರಿಗೆ
ಕಣ್ಣೆ-ದು-ರಿಗೇ
ಕಣ್ತೆರೆ-ಯು-ವಂತೆ
ಕಣ್ತೆರೆ-ಯುವ
ಕಣ್ಮರೆ-ಯಾಗ-ತೊಡಗಿತು
ಕಣ್ಮರೆ-ಯಾಗಲಿ
ಕಣ್ಮರೆ-ಯಾಗಿವೆ
ಕಣ್ಮರೆ-ಯಾದ
ಕತೆ-ಗಾರರು
ಕತೆ-ಯನ್ನು
ಕಥನ
ಕಥಾ-ಪ್ರ-ಸಂಗ-ದ-ಲ್ಲಿಯೂ
ಕಥಾ-ರೂಪ-ದ-ಲ್ಲಿ
ಕಥೆ
ಕಥೆ-ಗ-ಳನ್ನು
ಕಥೆ-ಗ-ಳಿಗೆ
ಕಥೆ-ಗಳ
ಕಥೆ-ಗಳ-ನ್ನೆ-ಲ್ಲ
ಕಥೆ-ಗಳ-ಲ್ಲಿ
ಕಥೆ-ಗಳು
ಕಥೆ-ಗಳೂ
ಕಥೆ-ಗಳೆ-ಲ್ಲಾ
ಕಥೆ-ಯ-ಲ್ಲಿ
ಕಥೆ-ಯನ್ನು
ಕಥೆ-ಯಾ-ಯಿತು
ಕಥೆಯ
ಕದ
ಕದ-ನ-ಗಳ-ಲ್ಲಿ
ಕದ-ಲದ
ಕನ-ಸನ್ನು
ಕನ-ಸಾ-ಯಿತು
ಕನ-ಸಾದ
ಕನ-ಸಿ-ನ-ಲ್ಲಿ
ಕನ-ಸಿ-ನ-ಲ್ಲಿ-ದೆಯೇ
ಕನ-ಸಿ-ನ-ಲ್ಲಿಯೂ
ಕನ-ಸಿ-ನಂತೆ
ಕನ-ಸಿ-ನಿಂದ
ಕನ-ಸಿನ
ಕನ-ಸು-ಗ-ಳನ್ನು
ಕನ-ಸು-ಗಳೆ-ಲ್ಲ
ಕನ-ಸೆಂದು
ಕನ-ಸ್ಸಿ-ನ-ಲ್ಲಿಯೂ
ಕನ-ಹರಿ
ಕನಸು
ಕನಸೇ
ಕನಿ-ಕರ
ಕನಿ-ಕರ-ಗೊಂಡು
ಕನಿ-ಕರ-ದಿಂದ
ಕನಿ-ಕರ-ವನ್ನು
ಕನಿಷ್ಟ
ಕನಿಷ್ಟ-ಪಕ್ಷ
ಕನಿಷ್ಠ-ತಮ-ನಾಗಿ-ದ್ದರೂ
ಕನ್ನಡ
ಕನ್ನಡಿ-ಯ-ಲ್ಲಿ
ಕನ್ನಡಿ-ಯನ್ನು
ಕನ್ನಡಿ-ಯೊಳಗೆ
ಕನ್ಯಾ-ಕುಮಾರಿ
ಕನ್ಯಾ-ಕುಮಾರಿ-ಯ-ಲ್ಲಿ
ಕನ್ಯಾ-ಕುಮಾರಿ-ಯಿಂದ
ಕನ್ಯಾ-ಕುಮಾರಿಗೆ
ಕನ್ಯಾ-ಕುಮಾರಿಯ
ಕನ್ಯಾ-ರಾಶಿ-ಯ-ಲ್ಲಿದ್ದ
ಕಪಿ
ಕಪಿ-ಗ-ಳನ್ನು
ಕಪಿ-ಗಳು
ಕಪಿ-ಗಳೆ-ಲ್ಲ
ಕಪಿ-ಮುಷ್ಟಿ-ಯ-ಲ್ಲಿ
ಕಪಿ-ಯಂತೆ
ಕಪಿ-ಯೊಂದು
ಕಪಿಯ
ಕಪ್ಪಾಗಿ-ತ್ತು
ಕಪ್ಪಾಗಿ-ಲ್ಲ
ಕಪ್ಪು
ಕಪ್ಪೆ
ಕಪ್ಪೆ-ಗ-ಳನ್ನು
ಕಪ್ಪೆ-ಗ-ಳಾಗು-ವೆವು
ಕಪ್ಪೆ-ಗಳಂತೆ
ಕಪ್ಪೆ-ಗಳೊ-ಡನೆ
ಕಪ್ಪೆ-ಯಂತೆ
ಕಪ್ಪೆ-ಯಾಗಿ-ತ್ತು
ಕಪ್ಪೆಯ
ಕಬ್ಬಿ-ಣದ
ಕಬ್ಬಿ-ಣದ್ದು
ಕಬ್ಬಿಣ-ದಂತಹ
ಕಬ್ಬಿಣ-ವೊಂದು
ಕಮಲ
ಕಮಲ-ಗ-ಳಾದರೆ
ಕಮಾ-ನನ್ನು
ಕಮಿಟಿ-ಯ-ಲ್ಲಿ
ಕಮಿಟಿ-ಯ-ವರು
ಕಮಿಟಿ-ಯನ್ನು
ಕಮೀಷ-ನರ್
ಕರ-ಗ-ಬೇಕು
ಕರ-ಗಿ-ದಾಗ
ಕರ-ಗಿ-ಸ-ಲಾ-ರದು
ಕರ-ಗಿ-ಸಿ-ಕೊ-ಳ್ಳುವ
ಕರ-ಗಿ-ಸು-ವು-ದ-ಕ್ಕಾಗಿ
ಕರ-ಗಿ-ಹೋಗಿ-ರ-ಬೇಕು
ಕರ-ಗಿ-ಹೋಗು-ತ್ತಾ-ರೇನು
ಕರ-ಗಿತು
ಕರ-ಗು-ತ್ತ
ಕರ-ಗು-ತ್ತಿ-ತ್ತು
ಕರ-ಗುವಂತೆ
ಕರ-ತಾಡ-ನದ
ಕರ-ತಾಡನ
ಕರ-ತಾಡನ-ಗ-ಳನ್ನು
ಕರ-ತಾಡನ-ಗಳಿಂದ
ಕರ-ತಾಡನ-ವಾ-ಯಿತು
ಕರ-ವ-ಸ್ತ್ರ-ದ-ಲ್ಲಿ
ಕರಂ-ಡದ-ಲ್ಲಿ-ಟ್ಟು
ಕರಗಿ
ಕರಾಚಿಗೆ
ಕರಾರಿ-ನಿಂದ
ಕರಾರು
ಕರಾಳ-ವಾಗಿ
ಕರಾಳಿ
ಕರಾವಳಿಯ
ಕರು
ಕರು-ಗ-ಳನ್ನು
ಕರು-ಗಳೊಂದಿಗೆ
ಕರು-ಣಾ-ಪೂರಿತ
ಕರು-ಣಿ-ಸೆಂದು
ಕರು-ಣಿಸು
ಕರು-ಣೆ-ಯನ್ನು
ಕರು-ಬು-ವುದು
ಕರುಣೆ
ಕರೆ
ಕರೆ-ಗಳು
ಕರೆ-ತಂ-ದಿ-ರು-ವುದು
ಕರೆ-ದ-ಲ್ಲದೆ
ಕರೆ-ದನು
ಕರೆ-ದರು
ಕರೆ-ದರೆ
ಕರೆ-ದಾಗ
ಕರೆ-ದಿ-ರ-ಲಿ-ಲ್ಲ
ಕರೆ-ದಿ-ರುವನು
ಕರೆ-ದಿ-ರುವರು
ಕರೆ-ದಿದ್ದ
ಕರೆ-ದಿದ್ದರು
ಕರೆ-ದಿರೋ
ಕರೆ-ದು-ಕೊ-ಳ್ಳು-ತ್ತಿದ್ದರು
ಕರೆ-ದು-ಕೊಂಡು
ಕರೆ-ದು-ಕೊಂಡು-ಬ-ರು-ತ್ತವೆ
ಕರೆ-ದು-ಕೊಂಡು-ಹೋಗ-ಬೇಕು
ಕರೆ-ದು-ಕೊಂಡು-ಹೋಗಿ
ಕರೆ-ದು-ಕೊಂಡು-ಹೋದನು
ಕರೆ-ದು-ಕೊಂಡು-ಹೋದರು
ಕರೆ-ದು-ದ-ರ-ಲ್ಲಿ
ಕರೆ-ದೆವು
ಕರೆ-ದೊಯ್ದು
ಕರೆ-ದೊಯ್ಯಿ
ಕರೆ-ದೊಯ್ಯು-ವುವು
ಕರೆ-ಯ-ತೊಡಗಿದರು
ಕರೆ-ಯ-ಬಹುದು
ಕರೆ-ಯ-ಬೇ-ಕಾ-ದರೆ
ಕರೆ-ಯ-ಬೇಕೆಂದು
ಕರೆ-ಯದೆ
ಕರೆ-ಯನ್ನು
ಕರೆ-ಯಲು
ಕರೆ-ಯಲೆ
ಕರೆ-ಯಿ-ಸು-ತ್ತಿ-ದ್ದರು
ಕರೆ-ಯು-ತ್ತ
ಕರೆ-ಯು-ತ್ತಾರೆ
ಕರೆ-ಯು-ತ್ತಾರೆಂಬುದು
ಕರೆ-ಯು-ತ್ತಿದೆ
ಕರೆ-ಯು-ತ್ತಿದ್ದನು
ಕರೆ-ಯು-ತ್ತಿದ್ದರು
ಕರೆ-ಯು-ತ್ತಿದ್ದರೂ
ಕರೆ-ಯು-ತ್ತಿದ್ದರೊ
ಕರೆ-ಯು-ತ್ತಿದ್ದವು
ಕರೆ-ಯು-ತ್ತಿರು-ವ-ರೆಂದೂ
ಕರೆ-ಯು-ತ್ತೇವೆ
ಕರೆ-ಯು-ವಿ-ರೇನು
ಕರೆ-ಯು-ವು-ದಾದರೆ
ಕರೆ-ಯು-ವು-ದಿ-ಲ್ಲ
ಕರೆ-ಯು-ವುದು
ಕರೆ-ಯು-ವುದೇ
ಕರೆ-ಯುವ
ಕರೆ-ಯುವರು
ಕರೆ-ಯುವಳು
ಕರೆ-ಯೋಣ
ಕರೆ-ಸ-ಬೇಕೆಂದು
ಕರೆ-ಸಲಾ-ಯಿತು
ಕರೆ-ಸಿ-ಕೊ-ಳ್ಳಲು
ಕರೆ-ಸಿ-ಕೊ-ಳ್ಳು-ತ್ತಿ-ರುವ
ಕರೆ-ಸಿ-ಕೊ-ಳ್ಳು-ತ್ತಿದ್ದ-ವರು
ಕರೆ-ಸಿ-ಕೊಂಡರು
ಕರೆ-ಸಿ-ದರು
ಕರೆ-ಸಿ-ದ್ದರು
ಕರೆ-ಸಿದ್ದು
ಕರೆದು
ಕರೆಸಿ
ಕರೇ
ಕರ್ಣನ
ಕರ್ಣಪ್ರ-ಯಾಗ-ವನ್ನು
ಕರ್ಣಾಟ
ಕರ್ಣಾಟ-ಕ-ದ-ಲ್ಲಿ
ಕರ್ತ-ವ್ಯ-ಗ-ಳನ್ನು
ಕರ್ತ-ವ್ಯ-ಗಳ-ನ್ನೆ-ಲ್ಲ
ಕರ್ತ-ವ್ಯ-ವನ್ನು
ಕರ್ತ-ವ್ಯ-ವಾ-ಗಿದೆ
ಕರ್ತ-ವ್ಯ-ವೆಂ-ದರೆ
ಕರ್ತ-ವ್ಯ-ವೆಂದು
ಕರ್ತ-ವ್ಯ-ವೆಂದೂ
ಕರ್ತ-ವ್ಯದ
ಕರ್ತವ್ಯ
ಕರ್ತೇಕರ್ನನ್
ಕರ್ನ-ಲ್
ಕರ್ನನ್ಗೆ
ಕರ್ನನ್ನ-ವ-ರೆಗೆ
ಕರ್ನಾಟಕದ
ಕರ್ಪೂ-ರದಾರತಿ-ಯನ್ನು
ಕರ್ಪೊಗೆ-ಯಂತೆ
ಕರ್ಮ
ಕರ್ಮ-ಕಾಂಡ
ಕರ್ಮ-ಕೋಟೆ-ಯೊಳಗೆ
ಕರ್ಮ-ಗ-ಳನ್ನು
ಕರ್ಮ-ಗಳ-ನ್ನೆ-ಲ್ಲ
ಕರ್ಮ-ಗಳಿಂದಲೂ
ಕರ್ಮ-ಗಳೊ-ಡನೆ
ಕರ್ಮ-ದ-ಲ್ಲಿ
ಕರ್ಮ-ದ-ಲ್ಲಿ-ರುವ-ವ-ರೆಗೆ
ಕರ್ಮ-ದ-ಲ್ಲಿಯೂ
ಕರ್ಮ-ದ-ಲ್ಲೆ
ಕರ್ಮ-ದಿಂದ
ಕರ್ಮ-ದಿಂದಲೇ
ಕರ್ಮ-ಪ್ರ-ವೃ-ತ್ತಿ
ಕರ್ಮ-ಫಲ-ದಾಸೆ-ಯನ್ನೆ-ಲ್ಲಾ
ಕರ್ಮ-ಫಲ-ದಿಂದ
ಕರ್ಮ-ಫಲ-ದಿಂದಲೇ
ಕರ್ಮ-ಫಲ-ವನ್ನು
ಕರ್ಮ-ಭೂಮಿ
ಕರ್ಮ-ಭೂಮಿಗೆ
ಕರ್ಮ-ಮಾಡಲು
ಕರ್ಮ-ಮಾಡು-ವಾಗ
ಕರ್ಮ-ಮಾಡು-ವುದು
ಕರ್ಮ-ಯೋಗ
ಕರ್ಮ-ಯೋಗ-ವೆಂದು
ಕರ್ಮ-ಯೋಗದ
ಕರ್ಮ-ಯೋಗಿ
ಕರ್ಮ-ವನ್ನು
ಕರ್ಮ-ವನ್ನೂ
ಕರ್ಮ-ವೀ-ರರು
ಕರ್ಮ-ವೀರ
ಕರ್ಮ-ಸಿದ್ಧಾಂತ
ಕರ್ಮಕ್ಕೆ
ಕರ್ಮದ
ಕರ್ಮವೇ
ಕರ್ಮವೋ
ಕರ್ಮ್ಯ-ತ್ಯಾಗ
ಕಲಹ
ಕಲಾ
ಕಲಾ-ದೃಷ್ಟಿ-ಯನ್ನು
ಕಲಾ-ಭಾವ-ನೆ-ಗಳು
ಕಲಾ-ಮಯ-ವಾಗಿ
ಕಲಾ-ಮಯ-ವಾದ
ಕಲಾ-ವಿ-ಜ್ಞಾನ
ಕಲಾ-ವಿದ
ಕಲಾ-ವಿದ-ರೊ-ಡನೆ
ಕಲಾ-ವಿದರ
ಕಲಾ-ವಿದರು
ಕಲಾ-ಶಾಲೆಯ
ಕಲಿ-ತರು
ಕಲಿ-ತಿ-ರುವಿ
ಕಲಿ-ಯ-ಬೇಕಾ-ಯಿತು
ಕಲಿ-ಯಿರಿ
ಕಲಿ-ಯು-ತ್ತಿದ್ದ
ಕಲಿ-ಯು-ತ್ತಿದ್ದನು
ಕಲಿ-ಯು-ತ್ತಿದ್ದರು
ಕಲಿ-ಯು-ತ್ತಿರು-ವರು
ಕಲಿ-ಯು-ತ್ತಿರು-ವೆ-ನೆಂಬು-ದ-ನ್ನು
ಕಲಿ-ಯು-ತ್ತೇವೆ
ಕಲಿ-ಯು-ವಂತೆ
ಕಲಿ-ಯು-ವು-ದಕ್ಕೆ
ಕಲಿ-ಯು-ವುದು
ಕಲಿ-ಯುಗ-ದ-ಲ್ಲಿ
ಕಲಿ-ಯುವ
ಕಲಿ-ಯುವ-ವ-ರೆಗೆ
ಕಲಿ-ಯುವರು
ಕಲಿ-ಯುವರೇ
ಕಲಿ-ಯುವುದಕ್ಕಿಂತ
ಕಲಿ-ಯುವೆವೆ
ಕಲಿ-ಸಿ-ದರು
ಕಲಿ-ಸಿದೆ
ಕಲಿ-ಸು-ತ್ತಿ-ದ್ದರು
ಕಲಿ-ಸು-ತ್ತೇನೆ
ಕಲಿ-ಸು-ವು-ದ-ರಿಂದ
ಕಲಿ-ಸು-ವು-ದಕ್ಕೆ
ಕಲಿ-ಸು-ವುದು
ಕಲಿ-ಸುವ
ಕಲಿತ
ಕಲಿತಿ-ದ್ದರೋ
ಕಲಿತಿ-ರು-ವುದು
ಕಲಿತಿ-ರು-ವೆನು
ಕಲಿತಿ-ರುವುದ-ನ್ನೆ-ಲ್ಲಾ
ಕಲಿತಿರಿ
ಕಲಿತಿರು-ವೆಯೋ
ಕಲಿತು
ಕಲಿತು-ಕೊ-ಳ್ಳ-ಬಹುದು
ಕಲಿತು-ಕೊ-ಳ್ಳು-ತ್ತ
ಕಲಿತು-ಕೊ-ಳ್ಳು-ವು-ದಿ-ಲ್ಲ
ಕಲಿತು-ಕೊ-ಳ್ಳುವುದ-ರ-ಲ್ಲಿ
ಕಲಿತು-ಕೊ-ಳ್ಳುವುದು
ಕಲಿತು-ಕೊಂಡಿದ್ದಾರೆ
ಕಲಿತು-ಕೊಳ್ಳ-ಬೇಕು
ಕಲಿತು-ಕೊಳ್ಳಿ
ಕಲಿತು-ದು-ದ-ನ್ನು
ಕಲಿತುಕೊ
ಕಲಿಸ-ಬಹುದು
ಕಲಿಸ-ಬೇಕು
ಕಲಿಸಿ-ಕೊಡ-ಬೇಕೆಂದು
ಕಲಿಸಿ-ದಳು
ಕಲೆ
ಕಲೆ-ಕ್ಟರ್
ಕಲೆ-ಗಳ-ಲ್ಲಿ
ಕಲೆ-ಗಾರ
ಕಲೆ-ಗಾರರು
ಕಲೆ-ತರೆ
ಕಲೆ-ಯ-ಲ್ಲಿ
ಕಲೆ-ಯನ್ನು
ಕಲೆ-ಸ್
ಕಲೆಗೆ
ಕಲೆತು
ಕಲೆಯ
ಕಲೆಯೂ
ಕಳಂಕ
ಕಳಚಿ
ಕಳಚಿ-ಬಿ-ಟ್ಟ
ಕಳಚಿದ
ಕಳಪೆ
ಕಳವಳ
ಕಳವಳ-ಗೊಂಡರು
ಕಳವಳ-ಗೊಂಡು
ಕಳವಳ-ವ-ನ್ನುಂಟು-ಮಾಡು-ವುದು
ಕಳವಳ-ವನ್ನು
ಕಳಿ-ಸು-ತ್ತಿ-ರ-ಲಿ-ಲ್ಲ
ಕಳಿ-ಸು-ವು-ದಕ್ಕೆ
ಕಳಿ-ಸೆಂದು
ಕಳಿಸಿ
ಕಳಿಸಿ-ದರು
ಕಳಿಸಿ-ದಿ-ರ-ಲ್ಲ
ಕಳುಹಿ
ಕಳುಹಿ-ಸ-ಬೇಕು
ಕಳುಹಿ-ಸ-ಬೇಕೆಂಬ
ಕಳುಹಿ-ಸಿ-ಕೊ-ಟ್ಟ
ಕಳುಹಿ-ಸಿ-ಕೊ-ಟ್ಟರು
ಕಳುಹಿ-ಸಿ-ಕೊ-ಟ್ಟು
ಕಳುಹಿ-ಸಿ-ಕೊಡ-ಬಹು-ದೆಂದು
ಕಳುಹಿ-ಸಿ-ಕೊಡ-ಬೇಕೆಂದು
ಕಳುಹಿ-ಸಿ-ಕೊಡು-ತ್ತಾರೆ
ಕಳುಹಿ-ಸಿ-ತ್ತು
ಕಳುಹಿ-ಸಿ-ದ-ನ-ಲ್ಲ
ಕಳುಹಿ-ಸಿ-ದ-ರೆಂಬುದು
ಕಳುಹಿ-ಸಿ-ದಂತೆ
ಕಳುಹಿ-ಸಿ-ದನು
ಕಳುಹಿ-ಸಿ-ದರು
ಕಳುಹಿ-ಸಿ-ದಳು
ಕಳುಹಿ-ಸಿ-ದ್ದರು
ಕಳುಹಿ-ಸಿ-ದ್ದು-ದ-ರಿಂದ
ಕಳುಹಿ-ಸಿ-ಬಿ-ಟ್ಟರು
ಕಳುಹಿ-ಸಿ-ಬಿ-ಡು-ವುದು
ಕಳುಹಿ-ಸಿ-ರ-ಬೇಕೆಂದು
ಕಳುಹಿ-ಸಿ-ರ-ಲಾ-ರದು
ಕಳುಹಿ-ಸಿ-ರು-ವನು
ಕಳುಹಿ-ಸಿ-ರು-ವೆನು
ಕಳುಹಿ-ಸಿ-ರುವರು
ಕಳುಹಿ-ಸಿ-ರುವೆವೆ
ಕಳುಹಿ-ಸಿದ
ಕಳುಹಿ-ಸು-ತ್ತಿ-ದ್ದರು
ಕಳುಹಿ-ಸು-ತ್ತಿ-ದ್ದಳು
ಕಳುಹಿ-ಸು-ತ್ತಿ-ದ್ದಾಗ
ಕಳುಹಿ-ಸು-ತ್ತಿದ್ದ
ಕಳುಹಿ-ಸು-ತ್ತಿರು-ವಳು
ಕಳುಹಿ-ಸು-ತ್ತೀರಿ
ಕಳುಹಿ-ಸು-ತ್ತೇನೆ
ಕಳುಹಿ-ಸು-ವಂತೆ
ಕಳುಹಿ-ಸು-ವರು
ಕಳುಹಿ-ಸು-ವು-ದಕ್ಕೂ
ಕಳುಹಿ-ಸು-ವು-ದಕ್ಕೆ
ಕಳುಹಿ-ಸು-ವು-ದಿ-ಲ್ಲ
ಕಳುಹಿ-ಸು-ವುದು
ಕಳುಹಿಸಿ
ಕಳುಹಿಸು
ಕಳೆ
ಕಳೆ-ಕೊಡ-ಬ-ಲ್ಲೆ-ಯಾ-ದರೆ
ಕಳೆ-ಗಳೇ
ಕಳೆ-ದ-ರೆಂದು
ಕಳೆ-ದ-ಲ್ಲದೆ
ಕಳೆ-ದ-ವರ
ಕಳೆ-ದ-ವರು
ಕಳೆ-ದಂತೆ
ಕಳೆ-ದನು
ಕಳೆ-ದರು
ಕಳೆ-ದರೆ
ಕಳೆ-ದವು
ಕಳೆ-ದಾದ
ಕಳೆ-ದಾದ-ಮೇಲೆ
ಕಳೆ-ದಿ-ದ್ದಾ-ಯಿತು
ಕಳೆ-ದಿದ್ದ
ಕಳೆ-ದಿದ್ದರು
ಕಳೆ-ದು-ಕೊ-ಳ್ಳದೆ
ಕಳೆ-ದು-ಕೊ-ಳ್ಳಲಿ-ಲ್ಲ
ಕಳೆ-ದು-ಕೊ-ಳ್ಳಲು
ಕಳೆ-ದು-ಕೊ-ಳ್ಳು-ತ್ತ
ಕಳೆ-ದು-ಕೊ-ಳ್ಳು-ತ್ತದೆ
ಕಳೆ-ದು-ಕೊ-ಳ್ಳು-ತ್ತಿ-ರುವಾಗ
ಕಳೆ-ದು-ಕೊ-ಳ್ಳು-ತ್ತಿದ್ದ
ಕಳೆ-ದು-ಕೊ-ಳ್ಳು-ವು-ದಕ್ಕೆ
ಕಳೆ-ದು-ಕೊ-ಳ್ಳುವರು
ಕಳೆ-ದು-ಕೊ-ಳ್ಳುವುದ-ರ-ಲ್ಲಿ
ಕಳೆ-ದು-ಕೊ-ಳ್ಳುವುದು
ಕಳೆ-ದು-ಕೊಂಡ
ಕಳೆ-ದು-ಕೊಂಡ-ಮೇಲೆ
ಕಳೆ-ದು-ಕೊಂಡಂತೆ
ಕಳೆ-ದು-ಕೊಂಡರೆ
ಕಳೆ-ದು-ಕೊಂಡಿ-ರುವರು
ಕಳೆ-ದು-ಕೊಂಡಿ-ರುವೆವು
ಕಳೆ-ದು-ಕೊಂಡಿ-ಲ್ಲ
ಕಳೆ-ದು-ಕೊಂಡು
ಕಳೆ-ದು-ಕೊಳ್ಳ-ಬೇಕಾಗಿ-ತ್ತು
ಕಳೆ-ದು-ಹೋ-ಯಿತು
ಕಳೆ-ದು-ಹೋಗಿ
ಕಳೆ-ದು-ಹೋಗು-ತ್ತಿ-ತ್ತು
ಕಳೆ-ಯ-ತೊಡಗಿದರು
ಕಳೆ-ಯ-ಬೇಕಾ-ಯಿತು
ಕಳೆ-ಯ-ಬೇಕೆ-ನ್ನಿ-ಸು-ವುದು
ಕಳೆ-ಯ-ಬೇಕೆಂದು
ಕಳೆ-ಯ-ಬೇಕೆಂದೂ
ಕಳೆ-ಯನ್ನು
ಕಳೆ-ಯಿತು
ಕಳೆ-ಯು-ತ್ತಿದೆ
ಕಳೆ-ಯು-ತ್ತಿದ್ದ
ಕಳೆ-ಯು-ತ್ತಿದ್ದರು
ಕಳೆ-ಯು-ತ್ತಿರು-ವರು
ಕಳೆ-ಯು-ತ್ತೇವೆ
ಕಳೆ-ಯು-ವ-ರೆಂದೂ
ಕಳೆ-ಯು-ವಿರಿ
ಕಳೆ-ಯು-ವು-ದಕ್ಕೆ
ಕಳೆ-ಯು-ವುದು
ಕಳೆ-ಯುವನು
ಕಳೆ-ಯುವರು
ಕಳೆ-ಯುವುದಕ್ಕಿಂತ
ಕಳೆದ
ಕಳೆದು
ಕಳೆದೆ
ಕಳೇ-ಬ-ರದ
ಕಳೇಬರ-ವನ್ನು
ಕಳೇಬರ-ವಿ-ತ್ತು
ಕಳ್ಳ-ತನ
ಕಳ್ಳ-ರಾಗಿ-ದ್ದರು
ಕವಚ
ಕವಲಾಗಿ
ಕವಿ
ಕವಿ-ಗಳ
ಕವಿ-ಗಳ-ಲ್ಲಿ
ಕವಿ-ಗಳು
ಕವಿ-ದಿ-ರುವಾಗ
ಕವಿ-ದಿದೆ
ಕವಿ-ದು-ಕೊಂಡಿ-ತ್ತು
ಕವಿ-ದು-ಕೊಂಡಿ-ರುವ
ಕವಿ-ಭಾವ
ಕವಿ-ರಾಜನ
ಕವಿ-ರಾಜರ
ಕವಿತೆ
ಕವಿಯ
ಕಶ್ಚಿ-ತ್
ಕಷ್ಟ
ಕಷ್ಟ-ಕಾರ್ಪಣ್ಯ-ಗಳ
ಕಷ್ಟ-ಕಾರ್ಪಣ್ಯ-ಗಳ-ಲ್ಲಿ
ಕಷ್ಟ-ಕಾರ್ಪಣ್ಯ-ಗಳು
ಕಷ್ಟ-ಕ್ಕಾಗಿ
ಕಷ್ಟ-ಗ-ಳನ್ನು
ಕಷ್ಟ-ಗ-ಳಿಗೆ
ಕಷ್ಟ-ಗಳ
ಕಷ್ಟ-ಗಳ-ಲ್ಲಿ
ಕಷ್ಟ-ಗಳು
ಕಷ್ಟ-ದ-ಲ್ಲಿ
ಕಷ್ಟ-ದ-ಲ್ಲಿ-ರು-ವನು
ಕಷ್ಟ-ದಿಂದ
ಕಷ್ಟ-ಪ-ಡು-ತ್ತಿದ್ದರು
ಕಷ್ಟ-ಪ-ಡು-ವುವು
ಕಷ್ಟ-ಪಟ್ಟಿ-ರು-ವೆನು
ಕಷ್ಟ-ಪಟ್ಟು
ಕಷ್ಟ-ಪಟ್ಟು-ಕೊಂಡು
ಕಷ್ಟ-ಪಡ-ಬೇಕಾಗಿ-ತ್ತು
ಕಷ್ಟ-ಪಡ-ಬೇಕೊ
ಕಷ್ಟ-ವ-ನ್ನಾ-ದರೂ
ಕಷ್ಟ-ವನ್ನು
ಕಷ್ಟ-ವಾ-ಗಿದೆ
ಕಷ್ಟ-ವಾ-ದರೂ
ಕಷ್ಟ-ವಾ-ಯಿತು
ಕಷ್ಟ-ವಾಗದೆ
ಕಷ್ಟ-ವಾಗಿ
ಕಷ್ಟ-ವಾಗಿ-ತ್ತು
ಕಷ್ಟ-ವಾಗು-ವುದು
ಕಷ್ಟ-ವಾದ
ಕಷ್ಟ-ವಿರ-ಬಹುದು
ಕಷ್ಟ-ವೆಂದು
ಕಷ್ಟ-ವೆಂದೂ
ಕಷ್ಟದ
ಕಸ-ಗುಡಿ-ಸುವ
ಕಸ-ಗುಡಿ-ಸುವ-ವನು
ಕಸ-ವನ್ನೂ
ಕಸಬು-ಗಳ-ಲ್ಲಿಯೂ
ಕಸರ-ತ್ತನ್ನು
ಕಸರ-ತ್ತಿ-ನಿಂದ
ಕಸುಬ-ನ್ನು
ಕಸುಬಾದ
ಕಸೂ-ತಿಯ
ಕಹಳೆ-ಗ-ಳನ್ನು
ಕಹಿ-ನೆನಪ-ನ್ನು
ಕಹಿ-ವೇದನೆ
ಕಹೆ
ಕಾ
ಕಾಂ
ಕಾಂತಿ
ಕಾಕೂರಗಾಚಿ
ಕಾಗದ
ಕಾಗದ-ಗ-ಳನ್ನು
ಕಾಗದ-ದ-ಲ್ಲಿ
ಕಾಗದ-ವನ್ನು
ಕಾಗೆ
ಕಾಗೆ-ಗಳಂತೆ
ಕಾಟ-ದಿಂದ
ಕಾಟ-ನ್
ಕಾಟ-ವನ್ನು
ಕಾಟೀ-ಸ್ಯಾನ್
ಕಾಡ
ಕಾಡ-ತೊಡಗಿತು
ಕಾಡಲಿ
ಕಾಡಿ
ಕಾಡಿ-ದರು
ಕಾಡಿ-ನ-ಲ್ಲಿ
ಕಾಡಿ-ನ-ಲ್ಲೊ
ಕಾಡಿ-ನೊಳಗೆ
ಕಾಡಿಗೆ
ಕಾಡಿನ
ಕಾಡು
ಕಾಡು-ಗಳ-ಲ್ಲಿ
ಕಾಡು-ಗಳು
ಕಾಡು-ಜ-ನರು
ಕಾಡು-ಜೇನು
ಕಾಡು-ತ್ತಲೇ
ಕಾಡು-ತ್ತವೆ
ಕಾಡು-ತ್ತಿದ್ದ
ಕಾಡು-ತ್ತಿರುವ
ಕಾಡು-ಮನುಷ್ಯ-ರಾಗಿ-ದ್ದರು
ಕಾಡು-ಮೇಡು
ಕಾಡು-ಮೇಡು-ಗಳ-ಲ್ಲಿ
ಕಾಡು-ವುದೆಂ-ದರೆ
ಕಾಡು-ಹರಟೆ-ಯ-ಲ್ಲಿ
ಕಾಡು-ಹರಟೆ-ಯಿಂದ
ಕಾಡ್ಗಿಚ್ಚಿ-ನಂತೆ
ಕಾಡ್ಗಿಚ್ಚಿಗೆ
ಕಾಣ-ತೊಡಗಿದರು
ಕಾಣ-ತೊಡಗಿದವು
ಕಾಣ-ದಂತೆ
ಕಾಣ-ಬ-ರ-ಲಿ-ಲ್ಲ
ಕಾಣ-ಬ-ರು-ವು-ದಿ-ಲ್ಲ-ವೆಂದೂ
ಕಾಣ-ಬ-ರು-ವುದು
ಕಾಣ-ಬರು-ತ್ತಿದೆ
ಕಾಣ-ಬಹುದೊ
ಕಾಣ-ಬೇಕು
ಕಾಣ-ಬೇಕೆಂದು
ಕಾಣ-ಲಿ-ಲ್ಲ
ಕಾಣರು
ಕಾಣಲು
ಕಾಣಿ-ತ್ತಿ-ತ್ತು
ಕಾಣಿ-ಸದ
ಕಾಣಿ-ಸಿ-ಕೊಂಡಿತು
ಕಾಣಿ-ಸು-ತ್ತಿ-ತ್ತು
ಕಾಣಿ-ಸು-ತ್ತಿ-ದೆಯೋ
ಕಾಣಿ-ಸು-ತ್ತಿದೆ
ಕಾಣಿ-ಸು-ವಂತೆ
ಕಾಣಿ-ಸು-ವು-ದಿ-ಲ್ಲ
ಕಾಣಿ-ಸು-ವುದು
ಕಾಣಿಕೆ
ಕಾಣಿಕೆ-ಯನ್ನು
ಕಾಣಿಸ-ಬಹುದೊ
ಕಾಣಿಸಿ-ಕೊಂಡ
ಕಾಣಿಸಿ-ಕೊಂಡ-ನೆಂಬುದು
ಕಾಣಿಸಿ-ಕೊಂಡಂತೆ
ಕಾಣಿಸಿ-ಕೊಂಡು
ಕಾಣು
ಕಾಣು-ತ್ತದೆ
ಕಾಣು-ತ್ತವೆ
ಕಾಣು-ತ್ತಾನೆ
ಕಾಣು-ತ್ತಿ-ತ್ತು
ಕಾಣು-ತ್ತಿ-ರಲಿ-ಲ್ಲ
ಕಾಣು-ತ್ತಿದೆ
ಕಾಣು-ತ್ತಿದ್ದ
ಕಾಣು-ತ್ತಿದ್ದರೇ
ಕಾಣು-ತ್ತಿದ್ದವು
ಕಾಣು-ತ್ತಿದ್ದವೆ
ಕಾಣು-ತ್ತಿವೆ
ಕಾಣು-ತ್ತೇವೆ
ಕಾಣು-ವ-ವ-ರೆಗೆ
ಕಾಣು-ವಂತೆ
ಕಾಣು-ವದು
ಕಾಣು-ವರು
ಕಾಣು-ವರೋ
ಕಾಣು-ವಷ್ಟು
ಕಾಣು-ವಾಗ
ಕಾಣು-ವಿರಿ
ಕಾಣು-ವು-ದ-ಕ್ಕ-ಲ್ಲ
ಕಾಣು-ವು-ದ-ರಿಂದ
ಕಾಣು-ವು-ದಕ್ಕೆ
ಕಾಣು-ವು-ದಿ-ಲ್ಲ
ಕಾಣು-ವು-ದಿ-ಲ್ಲವೇ
ಕಾಣು-ವು-ದಿ-ಲ್ಲವೋ
ಕಾಣು-ವುದು
ಕಾಣು-ವುದೆ-ನ್ನು-ತ್ತಿ-ದ್ದು-ದ-ನ್ನು
ಕಾಣು-ವುವು
ಕಾಣು-ವೆವು
ಕಾಣುವ
ಕಾಣೆ
ಕಾತ-ರರಾಗು-ವಿ-ರ-ಲ್ಲ
ಕಾತು-ರದ
ಕಾತುರ-ರಾಗಿ-ರುವರು
ಕಾಥ್ಗೊಡಾ-ಮ್
ಕಾದ
ಕಾದಂ-ಬರಿ-ಗ-ಳನ್ನು
ಕಾದಾ-ಡು-ತ್ತಿದ್ದರು
ಕಾದಾಡು-ತ್ತಾರೆ
ಕಾದಿ-ದ್ದುವು
ಕಾದಿ-ರಿಸಿ-ಕೊಂಡಿರಲು
ಕಾದಿ-ರು-ತ್ತಿ-ತ್ತೊ
ಕಾದಿದೆ
ಕಾದು
ಕಾದು-ಕೊಂಡಿ-ತ್ತು
ಕಾದು-ಕೊಂಡಿ-ರಲಾರೆ
ಕಾದು-ಕೊಂಡಿ-ರು-ತ್ತಿ-ತ್ತು
ಕಾದು-ಕೊಂಡಿ-ರು-ತ್ತಿದ್ದರು
ಕಾದು-ಕೊಂಡಿದೆಯೊ
ಕಾದು-ಕೊಂಡಿದ್ದರು
ಕಾದು-ಕೊಂಡಿರ-ಬೇಕೆಂದು
ಕಾದೆ
ಕಾನ-ನ-ದಿಂದ
ಕಾನಿಷ್ಕನ
ಕಾನೂ-ನಿನ
ಕಾನೂನು
ಕಾನೂನು-ಗಳು
ಕಾಪಾಡ-ಬೇಕು
ಕಾಪಾಡಲು
ಕಾಪಾಡಲೆಂದು
ಕಾಪಾಡಿ-ಕೊ-ಳ್ಳು-ವು-ದ-ಕ್ಕಾಗಿ
ಕಾಪಾಡಿ-ಕೊಂಡು
ಕಾಪಾಡಿ-ಕೊಳ್ಳ-ಬೇಕು
ಕಾಪಾಡು-ತ್ತೇವೆ
ಕಾಪಾಡು-ವು-ದಕ್ಕೆ
ಕಾಪಿ-ಯನ್ನು
ಕಾಮ
ಕಾಮ-ಕಾಂ-ಚ-ನಕ್ಕೆ
ಕಾಮ-ಕಾಂ-ಚ-ನದ
ಕಾಮ-ಕಾಂ-ಚ-ನವೇ
ಕಾಮ-ಕಾಂ-ಚನ
ಕಾಮ-ಕಾಂ-ಚನ-ವನ್ನು
ಕಾಮ-ಕಾಂ-ಚನಾ-ಸಕ್ತಿ
ಕಾಮ-ವನ್ನು
ಕಾಮ-ವಾಸ-ನಾ-ದಿ-ಗಳು
ಕಾಮ-ವಿ-ರುವ
ಕಾಮ-ವಿ-ಲ್ಲ
ಕಾಮದ
ಕಾಮವೇ
ಕಾಮಾಖ್ಯ
ಕಾಮಾಖ್ಯ-ದ-ಲ್ಲೇನು
ಕಾಮಾರಪುಕುರ-ದ-ಲ್ಲಿದ್ದ
ಕಾಮಾರಪುಕುರ-ವೆಂಬ
ಕಾಮಾಸಕ್ತಿ
ಕಾಮಿನಿ
ಕಾಮಿನಿ-ಕಾಂ-ಚ-ನದ
ಕಾಮಿನಿಯ
ಕಾಯ-ಬೇಕು
ಕಾಯ-ಭಂಗಿ
ಕಾಯ-ಸ್ತ
ಕಾಯ-ಸ್ಥ
ಕಾಯಿ-ಸಿದೆ-ಯ-ಲ್ಲ
ಕಾಯು-ತ್ತಿದ್ದಳು
ಕಾರ-ಣ-ಕರ್ತ-ನ-ಲ್ಲ-ವೆಂ-ದರು
ಕಾರ-ಣ-ಕರ್ತ-ರಾದ
ಕಾರ-ಣ-ಕರ್ತರು
ಕಾರ-ಣ-ಕ್ಕಾಗಿಯೇ
ಕಾರ-ಣ-ಗ-ಳನ್ನು
ಕಾರ-ಣ-ಗಳಿಂದ
ಕಾರ-ಣ-ಗಳೂ
ಕಾರ-ಣ-ಗಳೇನು
ಕಾರ-ಣ-ದಿಂದ
ಕಾರ-ಣ-ವ-ಲ್ಲ
ಕಾರ-ಣ-ವನ್ನು
ಕಾರ-ಣ-ವಾ-ಗಿದೆ
ಕಾರ-ಣ-ವಾ-ದಳು
ಕಾರ-ಣ-ವಾ-ಯಿತು
ಕಾರ-ಣ-ವಾಗಿ-ದ್ದರೆ
ಕಾರ-ಣ-ವಾಗು-ತ್ತಾನೆ
ಕಾರ-ಣ-ವಾಗು-ವುದು
ಕಾರ-ಣ-ವಾದ
ಕಾರ-ಣ-ವೆಂ-ದರೆ
ಕಾರ-ಣ-ವೆಂದು
ಕಾರ-ಣ-ವೇ-ನಿರ-ಬಹುದು
ಕಾರ-ಣ-ವೇ-ನೆಂದು
ಕಾರ-ಣ-ವೇನು
ಕಾರ-ಣವೆ
ಕಾರ-ಣವೇ
ಕಾರ-ಣಾಂ-ತರ-ಗಳಿಂದ
ಕಾರ-ನ್ವಾಲೀ-ಸ್
ಕಾರಣ
ಕಾರು-ತ್ತ
ಕಾರ್ಖಾನೆ
ಕಾರ್ಖಾನೆ-ಗಳು
ಕಾರ್ಖಾನೆ-ಯಿಂದ
ಕಾರ್ಗೋಬೋಟೆ
ಕಾರ್ನ್ಯಾಕ್
ಕಾರ್ಪಣ್ಯ-ದ-ಲ್ಲಿ
ಕಾರ್ಪಣ್ಯ-ವಾ-ದರೊ
ಕಾರ್ಪಣ್ಯದ
ಕಾರ್ಬಾಲಿಕ್
ಕಾರ್ಮನ್
ಕಾರ್ಮೋಡ
ಕಾರ್ಮೋಡ-ಗಳ
ಕಾರ್ಯ
ಕಾರ್ಯ-ಕಲಾ-ಪ-ಗಳ-ಲ್ಲಿ
ಕಾರ್ಯ-ಕಾರ-ಣ-ಗಳ
ಕಾರ್ಯ-ಕಾರ-ಣ-ಗಳು
ಕಾರ್ಯ-ಕಾರಣ
ಕಾರ್ಯ-ಕಾರಿ-ಯಾಗುವಂತೆ
ಕಾರ್ಯ-ಕಾರಿ-ಯಾದ
ಕಾರ್ಯ-ಕ್ರಮ
ಕಾರ್ಯ-ಕ್ರಮ-ಗ-ಳನ್ನು
ಕಾರ್ಯ-ಕ್ರಮ-ಗಳಿ-ರು-ವು-ದ-ರಿಂದ
ಕಾರ್ಯ-ಕ್ರಮ-ದ-ಲ್ಲಿ
ಕಾರ್ಯ-ಕ್ರಮ-ವನ್ನು
ಕಾರ್ಯ-ಕ್ರಮದ
ಕಾರ್ಯ-ಕ್ಷೇ-ತ್ರ-ಕ್ಕಿಂತ
ಕಾರ್ಯ-ಕ್ಷೇ-ತ್ರ-ಗಳ-ಲ್ಲಿ
ಕಾರ್ಯ-ಕ್ಷೇ-ತ್ರ-ಗಳಿಂದಲೂ
ಕಾರ್ಯ-ಕ್ಷೇ-ತ್ರ-ದ-ಲ್ಲಿ
ಕಾರ್ಯ-ಕ್ಷೇ-ತ್ರ-ದ-ಲ್ಲಿ-ರಲಿ
ಕಾರ್ಯ-ಕ್ಷೇ-ತ್ರ-ದ-ಲ್ಲಿ-ರುವಿರೊ
ಕಾರ್ಯ-ಕ್ಷೇ-ತ್ರ-ದ-ಲ್ಲಿಯೂ
ಕಾರ್ಯ-ಕ್ಷೇ-ತ್ರ-ದ-ಲ್ಲೆ-ಲ್ಲ
ಕಾರ್ಯ-ಕ್ಷೇ-ತ್ರ-ದಿಂದ
ಕಾರ್ಯ-ಕ್ಷೇ-ತ್ರಕ್ಕೆ
ಕಾರ್ಯ-ಗ-ಳನ್ನು
ಕಾರ್ಯ-ಗತ
ಕಾರ್ಯ-ಗತ-ಮಾಡಲು
ಕಾರ್ಯ-ಗತ-ಮಾಡುವ
ಕಾರ್ಯ-ಗತ-ವಾಗು-ವು-ದರೊ-ಳಗೆ
ಕಾರ್ಯ-ಗತ-ವಾಗು-ವು-ದಿ-ಲ್ಲ
ಕಾರ್ಯ-ಗಳ-ಲ್ಲಿ
ಕಾರ್ಯ-ಗಳಾಗಿವೆ
ಕಾರ್ಯ-ಗಳು
ಕಾರ್ಯ-ತ-ತ್ಪರ-ರಾಗಲು
ಕಾರ್ಯ-ತ-ತ್ಪರಾ-ಗಿ-ದ್ದರು
ಕಾರ್ಯ-ದ-ಲ್ಲಿ
ಕಾರ್ಯ-ದಕ್ಷತೆ
ಕಾರ್ಯ-ದರ್ಶಿ-ಗಳಾಗಿಯೂ
ಕಾರ್ಯ-ದರ್ಶಿ-ಗಳಿಂದ
ಕಾರ್ಯ-ದರ್ಶಿ-ಗಳೊ-ಡನೆ
ಕಾರ್ಯ-ದರ್ಶಿಯ
ಕಾರ್ಯ-ದರ್ಶಿಯಾದ
ಕಾರ್ಯ-ದರ್ಶಿಯೇ
ಕಾರ್ಯ-ಪ-ರರು
ಕಾರ್ಯ-ಭಾರ-ವನ್ನು
ಕಾರ್ಯ-ರೂಪ-ದ-ಲ್ಲಿ
ಕಾರ್ಯ-ರೂಪಕ್ಕೆ
ಕಾರ್ಯ-ವ-ಲ್ಲ
ಕಾರ್ಯ-ವನ್ನು
ಕಾರ್ಯ-ವಿ-ಭಾಗ
ಕಾರ್ಯ-ವಿ-ಲ್ಲ
ಕಾರ್ಯ-ವೆ-ಲ್ಲ
ಕಾರ್ಯ-ವೇನೊ
ಕಾರ್ಯ-ವೇನೋ
ಕಾರ್ಯ-ಸಿದ್ಧಿ
ಕಾರ್ಯಕ್ಕೆ
ಕಾರ್ಯತಃ
ಕಾರ್ಯವು
ಕಾರ್ಯವೂ
ಕಾರ್ಯಾರಂಭ-ವಾ-ಯಿತು
ಕಾರ್ಯಾಲಯಕ್ಕೆ
ಕಾರ್ಯೋ-ತ್ಸಾಹ
ಕಾರ್ಯೋ-ನ್ಮುಖ-ರಾಗಿ
ಕಾರ್ಯೋ-ನ್ಮುಖ-ರಾಗಿ-ರು-ವ-ರೆಂದೂ
ಕಾರ್ಯೋ-ನ್ಮುಖ-ವಾಗ-ಬೇಕು
ಕಾರ್ಯೋ-ನ್ಮುಖ-ವಾದ
ಕಾರ್ಲೈ-ಲ್ನ
ಕಾರ್ಸಿಕಾ
ಕಾರ್ಪೆಂ-ಟರ್
ಕಾಲ
ಕಾಲ-ಕಳೆ-ದರು
ಕಾಲ-ಕಾಲಕ್ಕೆ
ಕಾಲ-ಕ್ಕಿಂತ
ಕಾಲ-ಕ್ರಮ-ದ-ಲ್ಲಿ
ಕಾಲ-ಕ್ರಮ-ದಿಂದ
ಕಾಲ-ಕ್ರಮೇಣ
ಕಾಲ-ಗ-ಳಿದ್ದು
ಕಾಲ-ಗರ್ಭ-ದ-ಲ್ಲಿ
ಕಾಲ-ಗಳ-ಲ್ಲಿ
ಕಾಲ-ಗಳಿಂದ
ಕಾಲ-ದ-ದ-ವ-ರೆಗೆ
ಕಾಲ-ದ-ಮೇಲೆ
ಕಾಲ-ದ-ಲ್ಲಿ
ಕಾಲ-ದ-ಲ್ಲಿ-ತ್ತು
ಕಾಲ-ದ-ಲ್ಲಿ-ಯೇನೋ
ಕಾಲ-ದ-ಲ್ಲಿ-ರಲಿ
ಕಾಲ-ದ-ಲ್ಲಿಯೂ
ಕಾಲ-ದ-ಲ್ಲಿಯೇ
ಕಾಲ-ದ-ಲ್ಲೆ
ಕಾಲ-ದ-ಲ್ಲೆ-ಲ್ಲ
ಕಾಲ-ದ-ಲ್ಲೇ
ಕಾಲ-ದ-ವ-ರೆಗೂ
ಕಾಲ-ದ-ವ-ರೆಗೆ
ಕಾಲ-ದಂತೆ
ಕಾಲ-ದಿಂದ
ಕಾಲ-ದಿಂದಲೂ
ಕಾಲ-ದಿಂದಲೇ
ಕಾಲ-ದೇಶ-ನಿಮಿ-ತ್ತದ
ಕಾಲ-ದೇಶದ
ಕಾಲ-ದೊಂದಿಗೆ
ಕಾಲ-ಮೇಲೆ
ಕಾಲ-ಯಾಪನೆ
ಕಾಲ-ವನ್ನು
ಕಾಲ-ವನ್ನೆ-ಲ್ಲ
ಕಾಲ-ವನ್ನೆ-ಲ್ಲಾ
ಕಾಲ-ವನ್ನೇ
ಕಾಲ-ವಾ-ದರು
ಕಾಲ-ವಾ-ದರೂ
ಕಾಲ-ವಾ-ದರೆ
ಕಾಲ-ವಾಗಿ
ಕಾಲ-ವಾಗಿ-ದ್ದರು
ಕಾಲ-ವಾಗು-ವು-ದರೊ-ಳಗೆ
ಕಾಲ-ವಾದ
ಕಾಲ-ವಾದ-ಮೇಲೆ
ಕಾಲ-ವಿ-ತ್ತು
ಕಾಲ-ವಿ-ಲ್ಲ-ವೆಂದೂ
ಕಾಲ-ವಿದೆ
ಕಾಲ-ವೆ-ಲ್ಲ
ಕಾಲ-ವೆಂದು
ಕಾಲ-ವೆಂಬ
ಕಾಲಕ್ಕೆ
ಕಾಲದ
ಕಾಲರಾ
ಕಾಲವೇ
ಕಾಲಾ-ತೀತ-ವಾದ
ಕಾಲಾ-ನಂ-ತರ
ಕಾಲಾತೀ-ತ-ವಾಗಿ
ಕಾಲಾನು-ಕಾಲಕ್ಕೆ
ಕಾಲಿ-ಟ್ಟ
ಕಾಲಿ-ಟ್ಟರು
ಕಾಲಿ-ಟ್ಟಳು
ಕಾಲಿ-ಟ್ಟಾಗ
ಕಾಲಿ-ಟ್ಟಿತು
ಕಾಲಿ-ಟ್ಟೆವು
ಕಾಲಿ-ಟ್ಟೊಡ-ನೆಯೆ
ಕಾಲಿ-ನ-ಲ್ಲಿ
ಕಾಲಿ-ನ-ಲ್ಲಿಯೂ
ಕಾಲೀ-ಪ್ರಸಾದ-ದ-ತ್ತ
ಕಾಲು
ಕಾಲು-ಗಳ
ಕಾಲು-ಗಳು
ಕಾಲು-ಗಳೇ
ಕಾಲು-ಚಾಚಿ-ಕೊಂಡು
ಕಾಲು-ಜಾರಿ
ಕಾಲು-ತನಕ
ಕಾಲು-ನಡಿಗೆ-ಯ-ಲ್ಲಿ
ಕಾಲುವೆ
ಕಾಲೇ-ಜನ್ನು
ಕಾಲೇ-ಜಿಗೆ
ಕಾಲೇಜಿ-ನ-ಲ್ಲಿ
ಕಾಲೇಜಿ-ನಿಂದ
ಕಾಲೇಜಿನ
ಕಾಲೇಜು
ಕಾಲೇಜು-ಗಳ-ಲ್ಲಿ
ಕಾಲೇಜು-ಗಳಿಗೂ
ಕಾಲೇಜು-ಗಳು
ಕಾಲೇನಾ-ತ್ಮನಿ
ಕಾಲ್ಪನಿಕ
ಕಾಲ್ಪನಿಕವೆ
ಕಾಲ್ವಿ
ಕಾಲ್ವಿಗೆ
ಕಾಳ-ಪಾನಿ
ಕಾಳಿ
ಕಾಳಿ-ಅಭೇ-ದಾನಂದ
ಕಾಳಿ-ಕಾ-ದೇವಾಲಯ-ದ-ಲ್ಲಿ
ಕಾಳಿ-ಕಾ-ಮಾತೆ
ಕಾಳಿ-ಕಾ-ಮಾತೆ-ಯನ್ನು
ಕಾಳಿ-ಕಾ-ಮಾತೆಗೆ
ಕಾಳಿ-ಕಾ-ಮಾತೆಯ
ಕಾಳಿ-ಕಾ-ಮೂರ್ತಿಯ
ಕಾಳಿ-ಘಾಟಿನ
ಕಾಳಿ-ಚರಣ
ಕಾಳಿ-ದಾಸ
ಕಾಳಿ-ದಾಸನ
ಕಾಳಿ-ಬಾಬು-ಕುಂಜ
ಕಾಳಿ-ಯನ್ನು
ಕಾಳಿ-ಯೊ-ಡನೆ
ಕಾಳಿಕಾ
ಕಾಳಿಗೆ
ಕಾಳಿಯ
ಕಾಳೀ
ಕಾಳೀ-ಘಾಟಿ-ನ-ಲ್ಲಿ-ರುವ
ಕಾಳು
ಕಾವ-ಲಿನ-ವರು
ಕಾವಲು-ಗಾರ-ನನ್ನು
ಕಾವಿ
ಕಾವಿ-ಬ-ಟ್ಟೆ
ಕಾವಿ-ಬ-ಟ್ಟೆ-ಯನ್ನು
ಕಾವಿ-ಯ-ಬ-ಟ್ಟೆ-ಯನ್ನು
ಕಾವಿ-ಯನ್ನು
ಕಾವಿ-ಯನ್ನು-ಟ್ಟ
ಕಾವಿಯ
ಕಾವು
ಕಾವೇರಿದ
ಕಾವ್ಯ
ಕಾವ್ಯ-ಗಳು
ಕಾವ್ಯ-ಗಳೊ
ಕಾವ್ಯ-ದಂತೆ
ಕಾವ್ಯ-ಧಾರೆ-ಯಂತೆ
ಕಾವ್ಯ-ಮಯ-ವಾಗಿ
ಕಾವ್ಯ-ಮಯ-ವಾದ
ಕಾವ್ಯ-ವನ್ನು
ಕಾವ್ಯದ
ಕಾಶಿ
ಕಾಶಿ-ನಾಥ
ಕಾಶಿ-ಯ-ಲ್ಲಿ
ಕಾಶಿ-ಯ-ಲ್ಲಿ-ರುವ
ಕಾಶಿ-ಯ-ಲ್ಲಿದ್ದ
ಕಾಶಿ-ಯಂತೆ
ಕಾಶಿ-ಯನ್ನು
ಕಾಶಿ-ಯಾ-ತ್ರೆಗೆ
ಕಾಶಿ-ಯಿಂದ
ಕಾಶಿಗೆ
ಕಾಶಿಯ
ಕಾಶೀ-ಪು-ರಕ್ಕೆ
ಕಾಶೀ-ಪು-ರದ
ಕಾಶೀ-ಪು-ರದಿಂದ
ಕಾಶೀ-ಪುರ-ದ-ಲ್ಲಿ
ಕಾಶ್ಮಿರ-ದಿಂದ
ಕಾಶ್ಮೀ-ರಕ್ಕೆ
ಕಾಶ್ಮೀ-ರದ
ಕಾಶ್ಮೀರ
ಕಾಶ್ಮೀರ-ಗಳಿಂದ
ಕಾಶ್ಮೀರ-ದ-ಲ್ಲಿ
ಕಾಶ್ಮೀರ-ದ-ಲ್ಲಿ-ದ್ದಾಗ
ಕಾಶ್ಮೀರ-ದ-ಲ್ಲಿ-ರುವ-ವ-ರೆಗೆ
ಕಾಶ್ಮೀರ-ದಿಂದ
ಕಾಶ್ಮೀರಿಯ
ಕಾಷಾಯ
ಕಾಸು
ಕಾಸೂ
ಕಿ
ಕಿಂ
ಕಿಕ್ಕಿರಿ-ದಿದೆ
ಕಿಕ್ಕಿರಿದ
ಕಿಕ್ಕಿರಿದು
ಕಿಕ್ಕಿರಿದು-ಹೋಗಿ-ದ್ದರು
ಕಿಕ್ಕಿರಿದು-ಹೋಗಿ-ದ್ದಿತು
ಕಿಚಡಿ
ಕಿಚ್ಚನ್ನು
ಕಿಟಕಿ-ಗಳ
ಕಿಟಕಿಯ
ಕಿಡಿ
ಕಿಡಿ-ಗಳ
ಕಿಡಿ-ಗಳಂತಿವೆ
ಕಿಡಿ-ಗಳು
ಕಿಡಿ-ಗಳೂ
ಕಿಡಿ-ಯಂತೆ
ಕಿಡಿ-ಯನ್ನು
ಕಿಡ್ಡರ್ಪೂರ್
ಕಿಡ್ಡಿ
ಕಿಯೋಟೋ
ಕಿರ-ಣದ
ಕಿರಣ
ಕಿರಿಚು-ವರು
ಕಿರಿದು
ಕಿರೀಟ
ಕಿರೀಟದ
ಕಿರು
ಕಿರು-ಕುಳ-ಗಳಿಂದ
ಕಿರು-ಕುಳದ
ಕಿರು-ಕುಳದಿಂದ
ಕಿರು-ಚಿ-ಕೊ-ಳ್ಳು-ತ್ತಿದ್ದ
ಕಿರು-ಚಿ-ಕೊಂಡ
ಕಿರು-ಚಿ-ಕೊಂಡಿದ್ದ
ಕಿರು-ಚಿ-ಕೊಂಡೊಡ-ನೆಯೇ
ಕಿರು-ಚಿ-ದರೂ
ಕಿರು-ಹೊ-ತ್ತಿಗೆಯ
ಕಿವಿ
ಕಿವಿ-ಕೊಡ-ಲಿ-ಲ್ಲ
ಕಿವಿ-ಗ-ಳನ್ನು
ಕಿವಿ-ಗಳೇ
ಕಿವಿ-ಗೊ-ಟ್ಟು
ಕಿವಿ-ಗೊ-ಡದಿ-ರಲಾರೆ
ಕಿವಿ-ಗೊ-ಡುವ-ನೇನು
ಕಿವಿ-ಗೊಂದು
ಕಿವಿ-ಗೊಡಿ
ಕಿವಿ-ಗೊಡು-ವಂತೆ
ಕಿವಿ-ಗೊಡು-ವು-ದಿ-ಲ್ಲ
ಕಿವಿ-ಯ-ಲ್ಲಿ
ಕಿವಿ-ಯನ್ನಿರಿ-ಯು-ತ್ತದೆ
ಕಿವುಡಾಗು-ವಂತೆ
ಕಿಶ್ಟ
ಕಿಷ್ಟ
ಕಿಷ್ಟ-ನನ್ನು
ಕಿಷ್ಟ-ನಿಗೆ
ಕಿಷ್ಟನು
ಕಿಸೆ-ಯ-ಲ್ಲಿ-ಟ್ಟು-ಕೊಂಡು
ಕಿಸೆಗೆ
ಕೀ
ಕೀಟ
ಕೀಟ-ಕ್ಕಿಂತ
ಕೀಟ-ದಂತೆ
ಕೀಟ-ವನ್ನು
ಕೀಟಕ್ಕೂ
ಕೀಟವೂ
ಕೀರ್ತನೆ
ಕೀರ್ತನೆ-ಗ-ಳನ್ನು
ಕೀರ್ತನೆ-ಯನ್ನು
ಕೀರ್ತಿ
ಕೀರ್ತಿ-ಕೊಡ-ಬೇಕು
ಕೀರ್ತಿ-ಗಳ
ಕೀರ್ತಿ-ಯನ್ನು
ಕೀರ್ತಿ-ಯಿಂದ
ಕೀರ್ತಿ-ಸಿ-ದರೂ
ಕೀರ್ತಿಗೆ
ಕೀರ್ತಿಗೋ
ಕೀರ್ತಿಯ
ಕೀಲಿಕೈ
ಕೀಳ-ಬೇಕು
ಕೀಳ-ಲ್ಲ
ಕೀಳು
ಕೀಳು-ಮ-ಟ್ಟದ್ದು
ಕು
ಕುಂ
ಕುಂಠಿತ
ಕುಂಠಿತ-ವಾಗಿ-ಲ್ಲ
ಕುಂಡ-ಲಿ-ಯನ್ನೇ
ಕುಂಡ-ಲಿನಿ
ಕುಂಡ-ಲಿನಿ-ಯನ್ನು
ಕುಂಡ-ಲಿನಿ-ಯನ್ನೇ
ಕುಂಡಕ್ಕೂ
ಕುಂಡಲ
ಕುಂಡಲ-ವನ್ನೂ
ಕುಂದು
ಕುಂದು-ಕೊರತೆ-ಗ-ಳನ್ನು
ಕುಂದು-ಕೊರತೆ-ಗಳು
ಕುಂಭ-ಕೋ-ಣದ
ಕುಂಭ-ಕೋಣ-ದ-ಲ್ಲಿ
ಕುಂಭ-ಕೋಣ-ವನ್ನು
ಕುಕ್ಕಿದ
ಕುಕ್ಷಿ-ಯಾಗಿ-ದ್ದರೆ
ಕುಗ್ಗದ
ಕುಗ್ಗಿ
ಕುಗ್ಗಿ-ಹೋಗಲಿ-ಲ್ಲ
ಕುಗ್ಗಿ-ಹೋಗು-ವ-ವ-ರ-ಲ್ಲ
ಕುಗ್ಗು-ತ್ತಿದೆ
ಕುಚ್
ಕುಚ್ಚನ್ನು
ಕುಟೀ-ರಕ್ಕೆ
ಕುಟೀರ
ಕುಟೀರ-ಗ-ಳನ್ನು
ಕುಟೀರ-ದ-ಲ್ಲಿ
ಕುಟೀರ-ವನ್ನು
ಕುಟುಂಬ
ಕುಟುಂಬಕ್ಕೂ
ಕುಟುಂಬಕ್ಕೆ
ಕುಟುಕಿ-ದಂತೆ
ಕುಠಾರ-ಪ್ರಾಯ-ವಾಗು-ವುದು
ಕುಡಿ
ಕುಡಿ-ಕೆ-ಗಳು
ಕುಡಿ-ದಂತೆ
ಕುಡಿ-ದರು
ಕುಡಿ-ದರೆ
ಕುಡಿ-ದಿ-ರುವರೆ
ಕುಡಿ-ದೆ-ನೆಂದು
ಕುಡಿ-ದೆನೆಂ-ದರು
ಕುಡಿ-ನೋಟ
ಕುಡಿ-ಯ-ಕೂಡ-ದೆಂದು
ಕುಡಿ-ಯ-ಬಾ-ರ-ದೆಂದು
ಕುಡಿ-ಯ-ಬೇ-ಕಾ-ದರೆ
ಕುಡಿ-ಯ-ಬೇಕೆಂಬ
ಕುಡಿ-ಯಲು
ಕುಡಿ-ಯು-ತ್ತ
ಕುಡಿ-ಯು-ತ್ತಿ-ರಲಿ-ಲ್ಲ
ಕುಡಿ-ಯು-ತ್ತಿದ್ದಿ-ರ-ಲ್ಲ
ಕುಡಿ-ಯು-ತ್ತೇನೆ
ಕುಡಿ-ಯು-ವು-ದ-ನ್ನು
ಕುಡಿ-ಯು-ವು-ದಕ್ಕೆ
ಕುಡಿ-ಯು-ವುದು
ಕುಡಿ-ಯು-ವುದೇ
ಕುಡಿ-ಯುವೆ
ಕುಡಿದ
ಕುಡಿದು
ಕುಡಿದೇ
ಕುಡು-ಕನೊ
ಕುಡುಕ
ಕುಣಿ-ದಾ-ಡುವ
ಕುಣಿ-ಯು-ತ್ತ
ಕುಣಿ-ಯು-ತ್ತಾ
ಕುಣಿ-ಯು-ತ್ತಾರೆ
ಕುಣಿ-ಯು-ವುದು
ಕುಣಿತ
ಕುಣಿತ-ಕ್ಕೂಕುಣಿತವೆ
ಕುಣಿದಿ-ರು-ವೆವು
ಕುತೂಹಲ
ಕುತೂಹಲ-ದಿಂದ
ಕುತೂಹಲ-ದಿಂದಲೂ
ಕುತೂಹಲ-ವನ್ನು
ಕುತೂಹಲ-ವಿ-ತ್ತು
ಕುತೂಹಲ-ವೆ-ಲ್ಲ
ಕುತೂಹಲಿ-ಗ-ಳಾದ
ಕುತೂಹಲಿ-ಗಳಾಗಿ
ಕುತೂಹಲಿ-ಗಳಾಗಿ-ರು-ವ-ರೆಂದು
ಕುತೂಹಲಿ-ಗಳಾಗಿ-ರು-ವಾಗಲೇ
ಕುತೂಹಲಿ-ಗಳಾಗಿ-ರುವರು
ಕುದಿ-ಯು-ತ್ತಿ-ತ್ತು
ಕುದಿ-ಯು-ತ್ತಿದ್ದ
ಕುದಿ-ಯು-ವು-ದಿ-ಲ್ಲವೆ
ಕುದಿಯ-ಬೇಕು
ಕುದಿಯು-ತ್ತಿ-ರು-ವು-ದ-ನ್ನು
ಕುದು-ರೆಯ
ಕುದುರೆ
ಕುದುರೆ-ಗ-ಳನ್ನು
ಕುದುರೆ-ಗಳಿಂದ
ಕುದುರೆ-ಯ-ವ-ನೊ-ಡನೆ
ಕುದುರೆ-ಯನ್ನು
ಕುದುರೆ-ಯಿಂದ
ಕುಪಿ-ತ-ರಾಗಿ
ಕುಪಿತ-ರಾಗು-ತ್ತಿದ್ದರು
ಕುಪ್ಪಳಿ-ಸು-ತ್ತಾರೆ
ಕುಬೇ-ರನ
ಕುಮಾರ
ಕುಮಾರ-ಸ್ವಾ-ಮಿ-ಯ-ವರು
ಕುಮಾರ-ಸ್ವಾಮಿ
ಕುರಿ-ಗಳಂತಿದ್ದ-ವ-ರ-ನ್ನು
ಕುರಿ-ಗಳಂತೆಯೇ
ಕುರಿ-ಗಳು
ಕುರಿ-ತದ್ದು
ಕುರಿ-ಮರಿ
ಕುರಿ-ಮರಿ-ಯನ್ನು
ಕುರಿ-ಯಂತೆ
ಕುರಿತು
ಕುರುಚಲ
ಕುರುಡ-ನಾಗು-ವನು
ಕುರುಡ-ರಿಗೆ
ಕುರುಡು
ಕುರುಬರ
ಕುರುಹಾಗಿ-ತ್ತು
ಕುರ್ಚಿ-ಯ-ಲ್ಲಿ
ಕುರ್ಚಿ-ಯನ್ನು
ಕುರ್ಚಿಯ
ಕುಲ
ಕುಲ-ಗೋ-ತ್ರ
ಕುಲ-ಗೋ-ತ್ರ-ಗ-ಳಿಗೆ
ಕುಲ-ದ-ಲ್ಲಿ
ಕುಲ-ದ-ವರು
ಕುಲ-ದಿಂದ
ಕುಲ-ಸಂ-ಭೂತ-ರೆಂದು
ಕುಲಕ್ಕೆ
ಕುಲದ
ಕುಲಾವಿ
ಕುಲುಮೆ-ಯ-ಲ್ಲಿ
ಕುಳಿ-ತನು
ಕುಳಿ-ತರು
ಕುಳಿ-ತರೆ
ಕುಳಿ-ತಾಗ
ಕುಳಿ-ತಿ-ತ್ತು
ಕುಳಿ-ತಿ-ದ್ದರು
ಕುಳಿ-ತಿದ್ದೆ
ಕುಳಿ-ತಿದ್ದೆನೋ
ಕುಳಿ-ತೊ-ಡನೆ
ಕುಳಿತ
ಕುಳಿತ-ವ-ರಿ-ಗೆ-ಲ್ಲ
ಕುಳಿತ-ವ-ರಿಗೆ
ಕುಳಿತ-ವ-ರೆ-ಲ್ಲ
ಕುಳಿತ-ವರ
ಕುಳಿತ-ವರು
ಕುಳಿತಿ-ದ್ದಳು
ಕುಳಿತಿ-ದ್ದಾಗ
ಕುಳಿತಿ-ದ್ದಾನೆ
ಕುಳಿತಿ-ದ್ದಾರೆ
ಕುಳಿತಿ-ದ್ದು-ದ-ನ್ನು
ಕುಳಿತಿ-ದ್ದೆವು
ಕುಳಿತಿ-ರು-ತ್ತಿದ್ದ
ಕುಳಿತಿ-ರು-ವನು
ಕುಳಿತಿ-ರು-ವಾಗಲೂ
ಕುಳಿತಿ-ರುವ
ಕುಳಿತಿ-ರುವರು
ಕುಳಿತಿ-ರುವೆ
ಕುಳಿತಿದ್ದ
ಕುಳಿತಿದ್ದ-ವ-ರ-ನ್ನು
ಕುಳಿತಿದ್ದ-ವರು
ಕುಳಿತಿದ್ದೀಯೋ
ಕುಳಿತಿರಲು
ಕುಳಿತು
ಕುಳಿತು-ಕೊ-ಳ್ಳ-ಬೇಕೆಂದು
ಕುಳಿತು-ಕೊ-ಳ್ಳ-ಬೇಡ
ಕುಳಿತು-ಕೊ-ಳ್ಳಲು
ಕುಳಿತು-ಕೊ-ಳ್ಳು-ತ್ತದೆ
ಕುಳಿತು-ಕೊ-ಳ್ಳು-ತ್ತಿ-ತ್ತು
ಕುಳಿತು-ಕೊ-ಳ್ಳು-ತ್ತಿದ್ದ
ಕುಳಿತು-ಕೊ-ಳ್ಳು-ತ್ತಿದ್ದರು
ಕುಳಿತು-ಕೊ-ಳ್ಳು-ವು-ದ-ಕ್ಕಾಗಿ
ಕುಳಿತು-ಕೊ-ಳ್ಳು-ವು-ದಕ್ಕೆ
ಕುಳಿತು-ಕೊ-ಳ್ಳುವ
ಕುಳಿತು-ಕೊ-ಳ್ಳುವಂತೆ
ಕುಳಿತು-ಕೊ-ಳ್ಳುವನು
ಕುಳಿತು-ಕೊ-ಳ್ಳುವರು
ಕುಳಿತು-ಕೊ-ಳ್ಳುವಷ್ಟು
ಕುಳಿತು-ಕೊ-ಳ್ಳುವಾಗ
ಕುಳಿತು-ಕೊ-ಳ್ಳುವುದು
ಕುಳಿತು-ಕೊ-ಳ್ಳುವೆ
ಕುಳಿತು-ಕೊಂಡ
ಕುಳಿತು-ಕೊಂಡ-ಮೇಲೆ
ಕುಳಿತು-ಕೊಂಡ-ರೆಂದೂ
ಕುಳಿತು-ಕೊಂಡರು
ಕುಳಿತು-ಕೊಂಡರೂ
ಕುಳಿತು-ಕೊಂಡಳು
ಕುಳಿತು-ಕೊಂಡಿ-ತ್ತು
ಕುಳಿತು-ಕೊಂಡಿ-ರು-ವು-ದ-ನ್ನು
ಕುಳಿತು-ಕೊಂಡಿ-ರುವಂತೆ
ಕುಳಿತು-ಕೊಂಡಿ-ರುವೆವು
ಕುಳಿತು-ಕೊಂಡಿತು
ಕುಳಿತು-ಕೊಂಡಿದ್ದ
ಕುಳಿತು-ಕೊಂಡಿದ್ದರು
ಕುಳಿತು-ಕೊಂಡು
ಕುಳಿತು-ಕೊಂಡೆ
ಕುಳಿತು-ಕೊಂಡೆವು
ಕುಳಿತು-ಕೊಳ್ಳ-ಬೇಕು
ಕುಳಿತುಕೊ
ಕುಳಿತೆ
ಕುಳಿತೆವು
ಕುಳ್ಳರು
ಕುಳ್ಳಿ-ರಿಸಿ
ಕುಳ್ಳಿ-ರಿಸಿ-ಕೊಂಡರು
ಕುಳ್ಳಿ-ರಿಸಿ-ಕೊಂಡು
ಕುಳ್ಳಿರಲು
ಕುವೆಂಪು
ಕುಶಲ
ಕುಶಲ-ಕಲೆ-ಗಳ-ಲ್ಲಿಯೂ
ಕುಶಲಿ-ಗ-ಳಾದ
ಕುಶಾಗ್ರ
ಕೂ
ಕೂಗಿ
ಕೂಗಿ-ಕೊ-ಳ್ಳುವ
ಕೂಗಿ-ಕೊಳ್ಳು-ತ್ತೀರಿ
ಕೂಗು-ತ್ತಿದ್ದೇವೆ
ಕೂಜಂತ
ಕೂಟ-ಕ್ಕೆ-ಲ್ಲ
ಕೂಟ-ಗಳ-ಲ್ಲಿ
ಕೂಟ-ಗಳ-ಲ್ಲಿಯೂ
ಕೂಟ-ವನ್ನು
ಕೂಟಕ್ಕೆ
ಕೂಡ
ಕೂಡ-ದೆಂದು
ಕೂಡ-ನಿ-ಮ್ಮ
ಕೂಡಲು
ಕೂಡಲೆ
ಕೂಡಲೇ
ಕೂಡಾ
ಕೂಡಿ
ಕೂಡಿ-ಕೊಂಡು
ಕೂಡಿ-ಡ-ಬೇಕಾಗಿ-ಲ್ಲ
ಕೂಡಿ-ತ್ತು
ಕೂಡಿ-ದರು
ಕೂಡಿ-ದೆಯೋ
ಕೂಡಿ-ದ್ದರು
ಕೂಡಿ-ದ್ದವು
ಕೂಡಿ-ರು-ವುದು
ಕೂಡಿ-ರು-ವುದೋ
ಕೂಡಿ-ರುವ
ಕೂಡಿ-ಸಿ-ಕೊಂಡರು
ಕೂಡಿ-ಸಿ-ದರು
ಕೂಡಿ-ಸುವ
ಕೂಡಿ-ಹಾಕು-ತ್ತಿದ್ದರು
ಕೂಡಿ-ಹಾಕುವ
ಕೂಡಿ-ಹಾಕುವುದು
ಕೂಡಿದ
ಕೂಡಿದೆ
ಕೂಡಿಸಿ
ಕೂದ-ಲಿನ
ಕೂದಲು-ಗಳಿಂದ
ಕೂಪ-ದ-ಲ್ಲಿ
ಕೂಪಕ್ಕೆ
ಕೂಪಮಂಡೂಕ-ಗಳ-ನ್ನೂ
ಕೂರ್ಮ
ಕೂಲಂಕುಷ-ವಾಗಿ
ಕೂಲಿ
ಕೂಲಿ-ಕಾರರು
ಕೂಲಿ-ಗಳ
ಕೂಲಿ-ಗಳಿಗೂ
ಕೂಲಿ-ಯ-ವನು
ಕೂಲಿ-ಯನ್ನು
ಕೂಳಿ-ಗಾಗಿ
ಕೂಳಿ-ಲ್ಲದೆ
ಕೂಸು
ಕೃ
ಕೃತ-ಜ್ಞರು
ಕೃತ-ನಿಶ್ಚಯ-ರಾಗಿ-ದ್ದರು
ಕೃತ-ವಿದ್ಯ-ರಾದ
ಕೃತಘ್ನರು
ಕೃತಜ್ಞ-ನಾಗು-ತ್ತಾನೆ
ಕೃತಜ್ಞತೆ
ಕೃತಜ್ಞತೆ-ಗ-ಳನ್ನು
ಕೃತಜ್ಞತೆ-ಗಳು
ಕೃತಜ್ಞತೆ-ಯನ್ನು
ಕೃತಜ್ಞತೆ-ಯಿಂದ
ಕೃತಿ
ಕೃತಿ-ಗಳು
ಕೃತಿ-ಯನ್ನು
ಕೃತಿ-ಶ್ರೇಣಿ-ಯನ್ನು
ಕೃತಿಗೆ
ಕೃಪಾ-ಸಿಂಧು
ಕೃಪಾಕಟಾಕ್ಷ-ದಿಂದ
ಕೃಪಾಶ್ರಯ-ದ-ಲ್ಲಿ
ಕೃಪಾಹ-ಸ್ತ
ಕೃಪೆ
ಕೃಪೆ-ದೋರಿ
ಕೃಪೆ-ಮಾಡಿ
ಕೃಪೆ-ಯನ್ನು
ಕೃಪೆ-ಯಿಂದ
ಕೃಪೆ-ಯೊ-ಡನೆ
ಕೃಪೆಗೆ
ಕೃಷಿ-ಕನ
ಕೃಷ್ಣ
ಕೃಷ್ಣ-ಗರ್
ಕೃಷ್ಣ-ನನ್ನು
ಕೃಷ್ಣ-ನಾಗಿ-ದ್ದನೊ
ಕೃಷ್ಣ-ಲಾಲ
ಕೃಷ್ಣರ
ಕೆ
ಕೆಂ
ಕೆಂಡ-ವನ್ನು
ಕೆಂದ್ರ-ದಿಂದ
ಕೆಂಪಾ-ಯಿತು
ಕೆಂಪಾಗಿ
ಕೆಂಪು
ಕೆಂಪು-ಬಣ್ಣದ
ಕೆಂಪು-ವ-ಸ್ತ್ರ-ದಿಂದ
ಕೆಂಪು-ಸಮುದ್ರ-ದ-ಲ್ಲಿ
ಕೆಂಪು-ಸಮುದ್ರದ
ಕೆಂಪೇರಿ-ದು-ದ-ನ್ನು
ಕೆಂಪೇರಿದೆ
ಕೆಚ್ಚದೆ
ಕೆಚ್ಚನ್ನು
ಕೆಚ್ಚು
ಕೆಚ್ಚೆದೆ
ಕೆಚ್ಚೆದೆ-ಯುಳ್ಳ-ವರು
ಕೆಚ್ಚೆದೆಯ
ಕೆಡಹಿ
ಕೆಡಿ-ಸು-ವಂತೆ
ಕೆಡಿಸಿ-ಬಿ-ಟ್ಟಿ-ರುವರು
ಕೆಣಕಿ-ದಾಗ
ಕೆರಳ-ತೊಡಗಿತು
ಕೆರಳಿ-ತ್ತು
ಕೆರಳಿ-ದ್ದರು
ಕೆರಳಿ-ಸಿ-ದರು
ಕೆರಳಿ-ಸುವ
ಕೆರಳಿತು
ಕೆರಳು-ವುದು
ಕೆರೆ-ಯು-ತ್ತಿ-ರಲಿ-ಲ್ಲ
ಕೆಲ-ವ-ನ್ನಾ-ದರೂ
ಕೆಲ-ವ-ರ-ನ್ನು
ಕೆಲ-ವ-ರಾ-ದರೊ
ಕೆಲ-ವ-ರಿಗೆ
ಕೆಲ-ವನ್ನು
ಕೆಲ-ವರ
ಕೆಲ-ವರಿ-ಗಂತೂ
ಕೆಲ-ವರು
ಕೆಲ-ಸದ
ಕೆಲವೆಡೆ
ಕೆಲವೇ
ಕೆಲಸ
ಕೆಲಸ-ಕ್ಕಾಗಿ
ಕೆಲಸ-ಕ್ಕಿಂತ
ಕೆಲಸ-ಕ್ಕೆ-ಎಂ-ದರೆ
ಕೆಲಸ-ಗ-ಳನ್ನು
ಕೆಲಸ-ಗ-ಳಿಗೆ
ಕೆಲಸ-ಗಳ
ಕೆಲಸ-ಗಳ-ನ್ನೆ-ಲ್ಲ
ಕೆಲಸ-ಗಳ-ಲ್ಲಿ
ಕೆಲಸ-ಗಳು
ಕೆಲಸ-ಗಳೊ-ಡನೆ
ಕೆಲಸ-ಗಳೊಂದಿಗೆ
ಕೆಲಸ-ಗಾರರು
ಕೆಲಸ-ದ-ಲ್ಲಾಗಲಿ
ಕೆಲಸ-ದ-ಲ್ಲಿ
ಕೆಲಸ-ದಂತೆ
ಕೆಲಸ-ದಿಂದ
ಕೆಲಸ-ಮಾ-ಡದೆ
ಕೆಲಸ-ಮಾಡ-ತೊಡಗಿದನು
ಕೆಲಸ-ಮಾಡ-ಬೇ-ಕಾ-ದರೆ
ಕೆಲಸ-ಮಾಡ-ಬೇಕು
ಕೆಲಸ-ಮಾಡಲಿ
ಕೆಲಸ-ಮಾಡಲು
ಕೆಲಸ-ಮಾಡಿ
ಕೆಲಸ-ಮಾಡಿ-ಕೊಂಡು
ಕೆಲಸ-ಮಾಡಿ-ದರೆ
ಕೆಲಸ-ಮಾಡು
ಕೆಲಸ-ಮಾಡು-ತ್ತಿದ್ದರೆ
ಕೆಲಸ-ಮಾಡು-ತ್ತಿರು-ವಾಗ
ಕೆಲಸ-ಮಾಡು-ವ-ವ-ರ-ನ್ನು
ಕೆಲಸ-ಮಾಡು-ವಂತೆ
ಕೆಲಸ-ಮಾಡು-ವರೋ
ಕೆಲಸ-ಮಾಡು-ವಳು
ಕೆಲಸ-ಮಾಡು-ವು-ದ-ನ್ನು
ಕೆಲಸ-ಮಾಡುವ
ಕೆಲಸ-ವ-ಲ್ಲ
ಕೆಲಸ-ವನ್ನು
ಕೆಲಸ-ವನ್ನೂ
ಕೆಲಸ-ವನ್ನೆ-ಲ್ಲ
ಕೆಲಸ-ವನ್ನೆ-ಲ್ಲಾ
ಕೆಲಸ-ವನ್ನೇ
ಕೆಲಸ-ವಾ-ಯಿತು
ಕೆಲಸ-ವಾಗಲಿ
ಕೆಲಸ-ವಾಗಿ-ರ-ಲಿ-ಲ್ಲ
ಕೆಲಸ-ವಿ-ತ್ತು
ಕೆಲಸ-ವಿ-ದ್ದ-ಲ್ಲಿ
ಕೆಲಸ-ವಿ-ಲ್ಲ
ಕೆಲಸ-ವಿ-ಲ್ಲದೆ
ಕೆಲಸ-ವಿದು
ಕೆಲಸ-ವಿದೆ
ಕೆಲಸ-ವೆಂದು
ಕೆಲಸಕ್ಕೂ
ಕೆಲಸಕ್ಕೆ
ಕೆಲಸವೂ
ಕೆಲಸವೆ
ಕೆಲಸವೇ
ಕೆಳ-ಕಂಡಂತೆ
ಕೆಳ-ಗಡೆ
ಕೆಳ-ಗಿ-ಟ್ಟು
ಕೆಳ-ಗಿ-ಳಿದಂ-ತಾ-ಯಿತು
ಕೆಳ-ಗಿ-ಳಿದರು
ಕೆಳ-ಗಿ-ಳಿದು-ಬಂದು
ಕೆಳ-ಮ-ಟ್ಟದ
ಕೆಳ-ಮ-ಟ್ಟದ-ಲ್ಲಿಯೇ
ಕೆಳಕ್ಕೂ
ಕೆಳಕ್ಕೆ
ಕೆಳಗಣ
ಕೆಳಗಾ-ದರೊ
ಕೆಳಗಿ-ನದು
ಕೆಳಗಿ-ರುವ
ಕೆಳಗಿನ
ಕೆಳಗಿನ-ವ-ರ-ನ್ನು
ಕೆಳಗಿನ-ವ-ರಿಗೆ
ಕೆಳಗಿನ-ವರ
ಕೆಳಗೂ
ಕೆಳಗೆ
ಕೆಳಗೆ-ಎ-ಲ್ಲಾ
ಕೆಳಗೇ
ಕೆಳಿ
ಕೆಳಿ-ಬರು-ತ್ತಿದೆ
ಕೆಳು-ತ್ತೇನೆಂದೂ
ಕೆಳೆ-ದರು
ಕೆಸ-ರ-ನ್ನು
ಕೇ
ಕೇಂದ್ರ
ಕೇಂದ್ರ-ಗ-ಳನ್ನು
ಕೇಂದ್ರ-ದ-ಲ್ಲಿ
ಕೇಂದ್ರ-ದ-ಲ್ಲಿಯೇ
ಕೇಂದ್ರ-ದ-ವರು
ಕೇಂದ್ರ-ವ-ನ್ನಾಗಿ
ಕೇಂದ್ರ-ವನ್ನು
ಕೇಂದ್ರ-ವಾ-ಗಿ-ಟ್ಟು-ಕೊಂಡು
ಕೇಂದ್ರ-ವಾಗಿ-ತ್ತು
ಕೇಂದ್ರ-ವಾಗು-ವುದು
ಕೇಂದ್ರ-ವಾಗು-ವುದೆಂದು
ಕೇಂದ್ರ-ವಾದ
ಕೇಂದ್ರ-ವೆನಿ-ಸಿ-ರುವ
ಕೇಂದ್ರ-ಸ್ಥಾನ-ವಾಗು-ತ್ತದೆ
ಕೇಂದ್ರಕ್ಕೆ
ಕೇಂದ್ರದ
ಕೇಂದ್ರವೂ
ಕೇಂದ್ರವೇ
ಕೇಂದ್ರೀಕ-ರಿ-ಸಲು
ಕೇಂದ್ರೀಕ-ರಿ-ಸು-ವು-ದಕ್ಕೆ
ಕೇಂದ್ರೀಕೃತ-ವಾಗಿ-ರು-ತ್ತದೆ
ಕೇಂದ್ರೀಕೃತ-ವಾಗಿ-ರುವ
ಕೇಂದ್ರೀಕೃತ-ವಾಗು-ವಂತೆ
ಕೇಂಬ್ರಿಡ್ಜ್
ಕೇದಾರ
ಕೇದಾರ-ನಾಥ
ಕೇದಾರ-ನಾಥಕ್ಕೆ
ಕೇನ
ಕೇಪಿ
ಕೇರೆ
ಕೇಳ-ತೊಡಗಿತು
ಕೇಳ-ತೊಡಗಿದರು
ಕೇಳ-ದಂತೆ
ಕೇಳ-ಬರು-ತ್ತಿ-ಲ್ಲ
ಕೇಳ-ಬಹು-ದಾಗಿ-ತ್ತು
ಕೇಳ-ಬಹು-ದೆಂದೂ
ಕೇಳ-ಬಹುದು
ಕೇಳ-ಬಹುದೆ
ಕೇಳ-ಬಾ-ರದು
ಕೇಳ-ಬೇಕು
ಕೇಳ-ಬೇಕೆಂ-ದಿದ್ದೀಯೋ
ಕೇಳ-ಬೇಕೆಂದು
ಕೇಳ-ಬೇಕೆಂದೆ
ಕೇಳ-ಲಿ-ಲ್ಲ
ಕೇಳದ
ಕೇಳದೆ
ಕೇಳಬೇ-ಕೇನು
ಕೇಳಲಾಗಿ
ಕೇಳಲಿ
ಕೇಳಲು
ಕೇಳಲೆಂದು
ಕೇಳಿ
ಕೇಳಿ-ಕೊಂಡ
ಕೇಳಿ-ಕೊಂಡನು
ಕೇಳಿ-ಕೊಂಡರು
ಕೇಳಿ-ಕೊಂಡರೆ
ಕೇಳಿ-ಕೊಂಡಳು
ಕೇಳಿ-ಕೊಂಡಾಗ
ಕೇಳಿ-ಕೊಂಡಿದ್ದರು
ಕೇಳಿ-ಕೊಂಡು-ದ-ರಿಂದ
ಕೇಳಿ-ಕೊಂಡೆ
ಕೇಳಿ-ಕೊಂಡೆಯಾ
ಕೇಳಿ-ಕೊಳ್ಳ-ಬೇಕು
ಕೇಳಿ-ಕೊಳ್ಳಿ
ಕೇಳಿ-ದ-ಮೇಲೆ
ಕೇಳಿ-ದ-ವ-ರಿ-ಗೆ-ಲ್ಲ
ಕೇಳಿ-ದ-ವ-ರಿಗೆ
ಕೇಳಿ-ದ-ವರ
ಕೇಳಿ-ದ-ವರು
ಕೇಳಿ-ದ-ವರೆ-ದೆ-ಯನ್ನು
ಕೇಳಿ-ದ-ವರೆದೆ
ಕೇಳಿ-ದಂತೆ-ಲ್ಲ
ಕೇಳಿ-ದನು
ಕೇಳಿ-ದರು
ಕೇಳಿ-ದರೂ
ಕೇಳಿ-ದರೆ
ಕೇಳಿ-ದಳು
ಕೇಳಿ-ದಳೊ
ಕೇಳಿ-ದಾಗ
ಕೇಳಿ-ದಾಗ-ಲಂತೂ
ಕೇಳಿ-ದಿರಾ
ಕೇಳಿ-ದಿರೋ
ಕೇಳಿ-ದು-ದ-ನ್ನು
ಕೇಳಿ-ದು-ದ-ರಿಂದಲೇ
ಕೇಳಿ-ದು-ದಕ್ಕೆ
ಕೇಳಿ-ದೆನು
ಕೇಳಿ-ದೆನೊ
ಕೇಳಿ-ದೆಯಾ
ಕೇಳಿ-ದೆಯೋ
ಕೇಳಿ-ದೆವು
ಕೇಳಿ-ದೊ-ಡನೆ
ಕೇಳಿ-ದೊ-ಡನೆಯೆ
ಕೇಳಿ-ದೊ-ಡನೆಯೇ
ಕೇಳಿ-ದ್ದ-ಕ್ಕೆ-ಲ್ಲ
ಕೇಳಿ-ದ್ದಕ್ಕೆ
ಕೇಳಿ-ದ್ದರು
ಕೇಳಿ-ದ್ದರೂ
ಕೇಳಿ-ದ್ದೇ-ವ-ಲ್ಲ
ಕೇಳಿ-ದ್ದೇನೆ
ಕೇಳಿ-ದ್ದೇವೆ
ಕೇಳಿ-ಬ-ರು-ತ್ತಿ-ತ್ತು
ಕೇಳಿ-ಬಂತು
ಕೇಳಿ-ಯಾದ
ಕೇಳಿ-ರ-ಲಿ-ಲ್ಲ
ಕೇಳಿ-ರು-ವು-ದೆ-ಲ್ಲ
ಕೇಳಿ-ರು-ವೆನು
ಕೇಳಿ-ರುವರು
ಕೇಳಿ-ಲ್ಲವೆ
ಕೇಳಿ-ಸಿ-ಕೊ-ಳ್ಳಲಿ-ಲ್ಲ
ಕೇಳಿ-ಸಿತು
ಕೇಳಿಕೊ
ಕೇಳಿತು
ಕೇಳಿಯೂ
ಕೇಳಿಯೇ
ಕೇಳು
ಕೇಳು-ತ್ತ
ಕೇಳು-ತ್ತಿ-ತ್ತು
ಕೇಳು-ತ್ತಿ-ರಲಿ-ಲ್ಲ
ಕೇಳು-ತ್ತಿ-ಲ್ಲ
ಕೇಳು-ತ್ತಿದ್ದ
ಕೇಳು-ತ್ತಿದ್ದ-ವ-ರಿಗೆ
ಕೇಳು-ತ್ತಿದ್ದ-ವನ
ಕೇಳು-ತ್ತಿದ್ದ-ವರೆ
ಕೇಳು-ತ್ತಿದ್ದ-ವರೊ-ಬ್ಬ-ರಿಗೆ
ಕೇಳು-ತ್ತಿದ್ದಂತೆ
ಕೇಳು-ತ್ತಿದ್ದರು
ಕೇಳು-ತ್ತಿದ್ದರೂ
ಕೇಳು-ತ್ತಿದ್ದಾಗ
ಕೇಳು-ತ್ತಿರು-ವ-ರೆಂದೂ
ಕೇಳು-ತ್ತಿರು-ವಂತೆ
ಕೇಳು-ತ್ತಿರು-ವಾಗ
ಕೇಳು-ತ್ತಿರು-ವೆನು
ಕೇಳು-ತ್ತಿರು-ವೆವು
ಕೇಳು-ತ್ತಿರುವ
ಕೇಳು-ತ್ತೇವೆ
ಕೇಳು-ವ-ವ-ನ-ಲ್ಲ
ಕೇಳು-ವ-ವ-ರೆಗೆ
ಕೇಳು-ವ-ವ-ರೆದೆ-ಯ-ಲ್ಲಿ
ಕೇಳು-ವ-ವರ
ಕೇಳು-ವಂತೆ
ಕೇಳು-ವನು
ಕೇಳು-ವನೆ
ಕೇಳು-ವರು
ಕೇಳು-ವು-ದ-ರಿಂದ
ಕೇಳು-ವು-ದಕ್ಕೆ
ಕೇಳು-ವು-ದಿ-ಲ್ಲ
ಕೇಳು-ವುದ-ರ-ಲ್ಲೆ
ಕೇಳು-ವುದು
ಕೇಳು-ವುದೆಂ-ದರೆ
ಕೇಳು-ವುದೇ
ಕೇಳು-ವೆವು
ಕೇಳುವ
ಕೇಳೆಂದು
ಕೇಳೋಣ
ಕೇವಲ
ಕೇಶವ
ಕೇಶವ-ಚಂದ್ರ-ಸೇನ
ಕೇಶವ-ಚಂದ್ರ-ಸೇನ-ನ-ಲ್ಲಿ
ಕೇಶವ-ಚಂದ್ರ-ಸೇನರು
ಕೇಶವ-ಚಂದ್ರನ
ಕೇಶವ-ನ-ಲ್ಲಿ
ಕೇಶವ-ಸೇನನು
ಕೇಶವ-ಸೇನನೇ
ಕೇಸು
ಕೇಸು-ಗಳು
ಕೈ
ಕೈಗೂ-ಡು-ತ್ತದೆ
ಕೈಗೂಡಲಿ
ಕೈಗೆ
ಕೈಮಗ್ಗ-ದಿಂದ
ಕೈಯ-ಲ್ಲೆ-ಲ್ಲ
ಕೈಯ್ಯಲಾಗು-ತ್ತಿದ್ದರೂ
ಕೈರೋ
ಕೈರೋ-ದ-ಲ್ಲಿ
ಕೈಲಾ-ದಷ್ಟು
ಕೈಲಾ-ಸದ
ಕೈಲಾಗದ
ಕೈಲಾಗು-ತ್ತದೆಯೋ
ಕೈಲಾದ
ಕೈಲಾದ-ಮ-ಟ್ಟಿಗೆ
ಕೈಲಾದುದ-ನ್ನೆ-ಲ್ಲ-ವನ್ನು
ಕೈಲಾಸಕ್ಕೆ
ಕೈಸಡಿಲಿ-ಸಿ-ದರು
ಕೈಹಾಕ-ಬೇ-ಕಾದ-ವರು
ಕೈಹಾಕ-ಬೇಡ
ಕೊ
ಕೊಂ
ಕೊಂಚ
ಕೊಂಚ-ಕಾಲ
ಕೊಂಚ-ದೂರ
ಕೊಂಚವೂ
ಕೊಂಡ-ಮೇಲೆ
ಕೊಂಡರು
ಕೊಂಡಾ-ಡು-ತ್ತಿದ್ದರು
ಕೊಂಡಾ-ಡು-ವುದು
ಕೊಂಡಾ-ಡುವರು
ಕೊಂಡಾ-ಯಿತು
ಕೊಂಡಾಡ-ಲಿ-ಲ್ಲ
ಕೊಂಡಾಡಿ
ಕೊಂಡಾಡಿ-ದರು
ಕೊಂಡಾಡು-ತ್ತಿದ್ದಾಗ
ಕೊಂಡಾಡು-ವಂತಹ
ಕೊಂಡಾಡು-ವಾಗಲೂ
ಕೊಂಡಿದ್ದು
ಕೊಂಡು
ಕೊಂಡು-ಕೊಂಡು
ಕೊಂಡು-ಕೊಳ್ಳ-ಬೇಕಾಗಿ-ತ್ತು
ಕೊಂಡು-ಕೊಳ್ಳ-ಬೇಕು
ಕೊಂಡೊಡ-ನೆಯೇ
ಕೊಂಡೊಯ್ಯು-ವುವು
ಕೊಂಡೊಯ್ಯುವ
ಕೊಂಡೊಯ್ಯುವರು
ಕೊಂಡೊಯ್ಯುವಾಗ
ಕೊಂಬಿ-ನಂತೆ
ಕೊಂಬೆ
ಕೊಂಬೆ-ಗ-ಳನ್ನು
ಕೊಂಬೆ-ಗಳಿಂದ
ಕೊಂಬೆ-ಗಳು
ಕೊಂಬೆ-ಯನ್ನು
ಕೊಂಬೆ-ಯಿಂದ
ಕೊಂಬೆಗೆ
ಕೊಕ್ಕರೆ
ಕೊಕ್ಕರೆ-ಗಂತೂ
ಕೊಚ್ಚಿ
ಕೊಚ್ಚಿ-ಕೊ-ಳ್ಳು-ತ್ತಿ-ರುವರೋ
ಕೊಚ್ಚಿ-ಕೊ-ಳ್ಳು-ತ್ತಿದ್ದ
ಕೊಚ್ಚಿ-ಕೊ-ಳ್ಳು-ತ್ತಿದ್ದರು
ಕೊಚ್ಚಿ-ಕೊ-ಳ್ಳು-ತ್ತಿದ್ದೆ
ಕೊಚ್ಚಿ-ಕೊ-ಳ್ಳುವನು
ಕೊಚ್ಚಿ-ಕೊ-ಳ್ಳುವುದು
ಕೊಚ್ಚಿ-ಕೊಂಡರೂ
ಕೊಚ್ಚಿ-ಕೊಂಡು
ಕೊಚ್ಚಿ-ಯೊಳಿ-ರು-ವುದು
ಕೊಚ್ಚಿತು
ಕೊಚ್ಚೆ-ಯನ್ನು
ಕೊಠಡಿ
ಕೊಠಡಿ-ಗಳು
ಕೊಠಡಿ-ಯ-ಲ್ಲಿ
ಕೊಠಡಿ-ಯ-ಲ್ಲೇ
ಕೊಠಡಿ-ಯನ್ನು
ಕೊಠಡಿಗೆ
ಕೊಠಡಿಯ
ಕೊಡ
ಕೊಡ-ತಕ್ಕದ್ದು
ಕೊಡ-ತೊಡಗಿದರು
ಕೊಡ-ದ-ವನು
ಕೊಡ-ದಂತೆ
ಕೊಡ-ದಷ್ಟು
ಕೊಡ-ದಿ-ದ್ದ-ಲ್ಲಿ
ಕೊಡ-ಬ-ಲ್ಲ-ವ-ರಾಗಿ-ರ-ಲಿ-ಲ್ಲ
ಕೊಡ-ಬ-ಲ್ಲದು
ಕೊಡ-ಬ-ಲ್ಲರು
ಕೊಡ-ಬ-ಲ್ಲಿರಾ
ಕೊಡ-ಬ-ಲ್ಲೆ
ಕೊಡ-ಬಹುದು
ಕೊಡ-ಬೆಡ
ಕೊಡ-ಬೇ-ಕಾ-ದರೆ
ಕೊಡ-ಬೇಕಾ-ಯಿತು
ಕೊಡ-ಬೇಕಾಗಿ-ತ್ತು
ಕೊಡ-ಬೇಕಾಗಿದೆ
ಕೊಡ-ಬೇಕು
ಕೊಡ-ಬೇಕೆಂದು
ಕೊಡ-ಬೇಕೆಂದೂ
ಕೊಡ-ಬೇಕೆಂಬುದೇ
ಕೊಡ-ಬೇಕೇ
ಕೊಡ-ಬೇಕೋ
ಕೊಡ-ಬೇಡ
ಕೊಡ-ಲಾ-ರದು
ಕೊಡ-ಲಾ-ರದೊ
ಕೊಡ-ಲಿ-ಲ್ಲ
ಕೊಡ-ಲಿಯ
ಕೊಡ-ಲ್ಪ-ಡು-ತ್ತವೆ
ಕೊಡ-ವಿ-ಕೊಂಡು
ಕೊಡ-ವಿ-ದೊಡ-ನೆಯೇ
ಕೊಡಲಿ
ಕೊಡಲು
ಕೊಡಲೆ
ಕೊಡಲೇ
ಕೊಡವಿ
ಕೊಡಿ
ಕೊಡಿ-ಸು-ವಂತೆ
ಕೊಡಿಸಿ
ಕೊಡು
ಕೊಡು-ತ್ತದೆ
ಕೊಡು-ತ್ತಾ-ರೆ-ಯ-ಲ್ಲ
ಕೊಡು-ತ್ತಾ-ರೆಯೋ
ಕೊಡು-ತ್ತಾರೆ
ಕೊಡು-ತ್ತಾಳೆ
ಕೊಡು-ತ್ತಿ-ತ್ತು
ಕೊಡು-ತ್ತಿ-ರಲಿ-ಲ್ಲ
ಕೊಡು-ತ್ತಿ-ರು-ವು-ದ-ನ್ನು
ಕೊಡು-ತ್ತಿದ್ದ
ಕೊಡು-ತ್ತಿದ್ದ-ನೆಂದೂ
ಕೊಡು-ತ್ತಿದ್ದನು
ಕೊಡು-ತ್ತಿದ್ದರು
ಕೊಡು-ತ್ತಿದ್ದಳು
ಕೊಡು-ತ್ತಿದ್ದವು
ಕೊಡು-ತ್ತಿದ್ದಾಗ
ಕೊಡು-ತ್ತಿದ್ದಾನೆ
ಕೊಡು-ತ್ತಿದ್ದುದು
ಕೊಡು-ತ್ತಿದ್ದೆನೋ
ಕೊಡು-ತ್ತಿರು-ವರು
ಕೊಡು-ತ್ತಿರು-ವಳು
ಕೊಡು-ತ್ತಿರು-ವೆ-ಯ-ಲ್ಲ
ಕೊಡು-ತ್ತಿರುವೆ
ಕೊಡು-ತ್ತೀರಿ
ಕೊಡು-ತ್ತೇನೆ
ಕೊಡು-ತ್ತೇನೆಂದೂ
ಕೊಡು-ತ್ತೇವೆ
ಕೊಡು-ವ-ರೆಂದೂ
ಕೊಡು-ವ-ವನು
ಕೊಡು-ವ-ವಳು
ಕೊಡು-ವಂತಹ
ಕೊಡು-ವಂತಾಗು-ವಷ್ಟು
ಕೊಡು-ವಂತೆ
ಕೊಡು-ವಂತೆಯೇ
ಕೊಡು-ವನು
ಕೊಡು-ವರು
ಕೊಡು-ವರೆ
ಕೊಡು-ವರೋ
ಕೊಡು-ವವು
ಕೊಡು-ವಷ್ಟು
ಕೊಡು-ವಾಗ
ಕೊಡು-ವಾಗಲೆ
ಕೊಡು-ವಿರಿ
ಕೊಡು-ವು-ದ-ಕ್ಕಾಗಿ
ಕೊಡು-ವು-ದ-ಕ್ಕಾಗಿದೆ
ಕೊಡು-ವು-ದ-ನ್ನು
ಕೊಡು-ವು-ದ-ರ-ಲ್ಲಿ
ಕೊಡು-ವು-ದ-ಲ್ಲ
ಕೊಡು-ವು-ದಕ್ಕೆ
ಕೊಡು-ವು-ದರ
ಕೊಡು-ವು-ದರೊ-ಳಗೆ
ಕೊಡು-ವು-ದಿ-ಲ್ಲ
ಕೊಡು-ವು-ದಿ-ಲ್ಲ-ವೆಂದೂ
ಕೊಡು-ವು-ದಿ-ಲ್ಲ-ವೋಕೊಡು-ವುದು
ಕೊಡು-ವುದ-ಕ್ಕೋ-ಸ್ಕರ
ಕೊಡು-ವುದಕ್ಕಿಂತ
ಕೊಡು-ವುದಕ್ಕೇ
ಕೊಡು-ವುದು-ಚೇ-ತನ-ವನ್ನು
ಕೊಡು-ವುದೆ
ಕೊಡು-ವುದೇ
ಕೊಡು-ವೆವು
ಕೊಡುವ
ಕೊಡುವೆ
ಕೊಡೆಂದು
ಕೊಡೊಯ್ಯು-ವಂತ-ಹದು
ಕೊನೆ
ಕೊನೆ-ಗ-ಳಿಗೆ-ಯ-ವ-ರೆಗೂ
ಕೊನೆ-ಗಾಣ-ಬೇಕಾಗು-ತ್ತದೆ
ಕೊನೆ-ಗಾಣಲಿ
ಕೊನೆ-ಗಾಣಿ-ಸಿ-ದರು
ಕೊನೆ-ಗಾಣಿ-ಸು-ತ್ತಿ-ದ್ದ-ರೆಂದು
ಕೊನೆ-ಗಾಣಿ-ಸುವೆ
ಕೊನೆ-ಗಾಣು-ತ್ತಿದೆ
ಕೊನೆ-ಗಾಣು-ವುದು
ಕೊನೆ-ಗಾಲ-ದ-ಲ್ಲಿ
ಕೊನೆ-ಗಾಲ-ವನ್ನು
ಕೊನೆ-ಗಾಲವು
ಕೊನೆ-ಗೊ-ಮ್ಮೆ
ಕೊನೆ-ಗೊ-ಳ್ಳು-ತ್ತಿ-ತ್ತು
ಕೊನೆ-ಗೊ-ಳ್ಳುವುದು
ಕೊನೆ-ಗೊಂಡವು
ಕೊನೆ-ಗೊಂಡಿತು
ಕೊನೆ-ಗೊಂಡಿದೆ
ಕೊನೆ-ಗೊಂದು
ಕೊನೆ-ಮುಟು-ವು-ದಕ್ಕೆ
ಕೊನೆ-ಮೊದ-ಲಿ-ಲ್ಲ
ಕೊನೆ-ಯ-ದಕ್ಕೆ
ಕೊನೆ-ಯ-ದಾಗಿ
ಕೊನೆ-ಯ-ಲ್ಲ
ಕೊನೆ-ಯ-ಲ್ಲಿ
ಕೊನೆ-ಯ-ವ-ರೆಗೂ
ಕೊನೆ-ಯ-ವ-ರೆಗೆ
ಕೊನೆ-ಯದು
ಕೊನೆ-ಯಿಂದ
ಕೊಬ್ಬ-ನ್ನು
ಕೊಯ್ಯಲು
ಕೊರ-ಗು-ತ್ತಿದೆ
ಕೊರ-ಳನ್ನು
ಕೊರ-ಳಿಗೆ
ಕೊರಕಲು
ಕೊರಗಿ
ಕೊರಗುವೆ
ಕೊರತೆ
ಕೊರತೆ-ಗಳೆ-ಲ್ಲ-ವನ್ನೂ
ಕೊರತೆ-ಯನ್ನು
ಕೊರಳ
ಕೊರಾ-ನ್
ಕೊರೆ-ದಿ-ರುವ
ಕೊರೆ-ದಿ-ಲ್ಲ
ಕೊರೆತ
ಕೊರೆದ
ಕೊರೈಸಿ
ಕೊಲಂ-ಬಿಯ
ಕೊಲಂ-ಬಿಯಾ-ದ-ಲ್ಲಿರುವ
ಕೊಲಂಬೊ
ಕೊಲಂಬೊ-ದಿಂದ
ಕೊಲಂಬೊ-ಯಿಂದ
ಕೊಲಂಬೋಗೆ
ಕೊಲೆ-ಪಾ-ತಕ-ನನ್ನು
ಕೊಲೊಂಬೊ-ಯಿಂದ
ಕೊಳ
ಕೊಳ-ಗಳು
ಕೊಳ-ವಿ-ತ್ತು
ಕೊಳ-ವೆಯ
ಕೊಳಕು
ಕೊಳಕ್ಕೆ
ಕೊಳದ
ಕೊಳವೆ
ಕೊಳೆ
ಕೊಳೆ-ತು-ಹೋಗು-ತ್ತಿರುವ
ಕೊಳೆತು
ಕೊಳ್ಳ-ಬಹು-ದಾಗಿ-ತ್ತು
ಕೊಳ್ಳ-ಬೇಕು
ಕೊಳ್ಳೆ
ಕೋ
ಕೋಚು-ಗಾಡಿ
ಕೋಚ್
ಕೋಚ್ಗಾಡಿ
ಕೋಚ್ಮ್ಯಾನ್
ಕೋಟ-ವಾಗ-ಬೇಕೆಂಬುದೇ
ಕೋಟಲೆಗೆ
ಕೋಟಾ
ಕೋಟಿ
ಕೋಟಿ-ಯ-ಲ್ಲಿ
ಕೋಟಿ-ಯಷ್ಟೇ
ಕೋಟಿಗೆ
ಕೋಟಿಯ
ಕೋಟು
ಕೋಟೆ
ಕೋಟೆ-ಯನ್ನು
ಕೋಟೆ-ಯಿಂದ
ಕೋಣೆ
ಕೋಣೆ-ಗ-ಳನ್ನು
ಕೋಣೆ-ಗಳ
ಕೋಣೆ-ಗಳು
ಕೋಣೆ-ಯ-ನ್ನೆ-ಲ್ಲ
ಕೋಣೆ-ಯ-ಲ್ಲಿ
ಕೋಣೆ-ಯ-ಲ್ಲಿದ್ದ
ಕೋಣೆ-ಯ-ಲ್ಲಿಯೇ
ಕೋಣೆ-ಯ-ಲ್ಲೆ
ಕೋಣೆ-ಯನ್ನು
ಕೋಣೆ-ಯನ್ನೇ
ಕೋಣೆ-ಯಿಂದ
ಕೋಣೆ-ಯೊಂದ-ರ-ಲ್ಲಿ
ಕೋಣೆ-ಯೊಳಗೆ
ಕೋಣೆಯ
ಕೋಣೆಯೂ
ಕೋಪ
ಕೋಪ-ಗೊಂಡರು
ಕೋಪ-ಗೊಂಡರೂ
ಕೋಪ-ಗೊಂಡು
ಕೋಪ-ದಿಂದ
ಕೋಪ-ಬಂದು
ಕೋಪ-ಮಾಡಿ-ಕೊಂಡಿದ್ದು
ಕೋಪ-ವನ್ನು
ಕೋಪಕ್ಕೆ
ಕೋಪವೇ
ಕೋಪಿನ-ವನ್ನು
ಕೋಪಿಷ್ಠರು
ಕೋಪಿಸಿ-ಕೊಂಡು
ಕೋಪಿಸಿ-ಕೊಳ್ಳು-ತ್ತಾರೆ
ಕೋಬ್ಲೆಂಜ್ಗೆ
ಕೋಮಲ
ಕೋಮಲ-ಸ್ವ-ಭಾವದ
ಕೋಮಿ-ನ-ಲ್ಲಿಯೂ
ಕೋಮಿಗೆ
ಕೋಮು
ಕೋಮು-ಗ-ಳಿಗೆ
ಕೋರಿ
ಕೋರಿ-ಕೆ-ಯಂತೆ
ಕೋರಿ-ಕೆ-ಯನ್ನು
ಕೋರಿ-ಕೆ-ಯಿಂದ
ಕೋರಿ-ಕೆಯ
ಕೋರಿ-ಕೊಂಡ
ಕೋರಿ-ಕೊಂಡನು
ಕೋರಿ-ಕೊಂಡರು
ಕೋರಿ-ಕೊಂಡಾಗ
ಕೋರಿ-ಕೊಂಡಿದ್ದ
ಕೋರಿ-ಕೊಂಡಿದ್ದರು
ಕೋರಿ-ಕೊಂಡು
ಕೋರಿ-ಕೊಂಡು-ದ-ರಿಂದ
ಕೋರಿ-ದರು
ಕೋರಿ-ದರೂ
ಕೋರಿಕೆ
ಕೋರಿದೆ
ಕೋರೈ-ಸು-ತ್ತಿ-ರುವ
ಕೋರೈ-ಸು-ತ್ತಿ-ರುವಾಗ
ಕೋರ್ಟಿ-ನ-ಲ್ಲಿ
ಕೋರ್ಟಿ-ನ-ಲ್ಲಿ-ತ್ತು
ಕೋರ್ಟಿ-ನ-ಲ್ಲಿದ್ದ
ಕೋರ್ಟಿ-ನಿಂದ
ಕೋರ್ಟು
ಕೋರ್ನೀರ್ಗಾ-ಟ್
ಕೋಲು
ಕೋಲೇ
ಕೋಲೋಗ್ನಿ-ಯಿಂದ
ಕೋಲೋಗ್ನಿಗೆ
ಕೋಳವೆ-ಯನ್ನು
ಕೌ
ಕೌಪೀನ
ಕೌಶ-ಲ್ಯ
ಕ್ಕೆ
ಕ್ತ
ಕ್ಯಾಂಟ-ನ್
ಕ್ಯಾಂಡ-ಲ್
ಕ್ಯಾಂಡಿಗೆ
ಕ್ಯಾಂಡಿಯ
ಕ್ಯಾಥೊಲಿಕ್
ಕ್ಯಾಥೋಲಿ-ಕರು
ಕ್ಯಾಥೋಲಿಕ್
ಕ್ಯಾನ-ನ್
ಕ್ಯಾನ್ಸರ್
ಕ್ಯಾಪಿಡಿಲೈ-ನ್
ಕ್ಯಾಪ್ಟ-ನ್
ಕ್ಯಾಲಿಫೋರ್ನಿಯಾ
ಕ್ಯಾಲಿಫೋರ್ನಿಯಾ-ದಿಂದ
ಕ್ಯಾಲಿಫೋರ್ನಿಯಾಕ್ಕೆ
ಕ್ಯಾಸ-ಲ್
ಕ್ರಮ
ಕ್ರಮ-ಕ್ರಮ-ವಾಗಿ
ಕ್ರಮ-ಪರಿಣತಿ
ಕ್ರಮ-ಬದ್ಧ-ವಾದ
ಕ್ರಮ-ವನ್ನು
ಕ್ರಮ-ವಾಗಿ
ಕ್ರಮ-ವಿ-ತ್ತು
ಕ್ರಮ-ವಿಕಾಸ
ಕ್ರಮ-ವಿಕಾಸದ
ಕ್ರಮೇಣ
ಕ್ರಮೇಣ
ಕ್ರಮೋ-ನ್ನತಿ
ಕ್ರಾ-ಸ್
ಕ್ರಾಂ
ಕ್ರಾಂತಿ-ಯನ್ನು
ಕ್ರಿ
ಕ್ರಿಯೆ
ಕ್ರಿಶ
ಕ್ರೀ
ಕ್ರೀಡಾ
ಕ್ರೀಡೆ
ಕ್ರೀಡೆಯೆ
ಕ್ರೂ
ಕ್ರೂರ
ಕ್ರೂರ-ತ-ನ-ದಿಂದ
ಕ್ರೂರ-ನಾದ
ಕ್ರೂರಿ
ಕ್ರೈ
ಕ್ರೌರ್ಯ
ಕ್ರೌರ್ಯ-ಕೃ-ತ್ಯ-ಗಳು
ಕ್ರೌರ್ಯ-ಗ-ಳನ್ನು
ಕ್ಲಬ್
ಕ್ಲಬ್ಬಿ-ನ-ಲ್ಲಿ
ಕ್ಲಬ್ಬಿಗೆ
ಕ್ಲಬ್ಬು-ಗಳ-ಲ್ಲಿ
ಕ್ಲಾ
ಕ್ಲಿಷ್ಟ
ಕ್ಲಿಷ್ಟ-ವಾದ
ಕ್ಲೈಬ್ಯ-ವನ್ನು
ಕ್ವ
ಕ್ಷ
ಕ್ಷಂತವ್ಯಮೇತ-ತ್
ಕ್ಷಣ
ಕ್ಷಣ-ಕಾಲ
ಕ್ಷಣ-ಕಾಲವೂ
ಕ್ಷಣ-ಗ-ಳಾದ
ಕ್ಷಣ-ಗಳು
ಕ್ಷಣ-ದ-ಲ್ಲಿ
ಕ್ಷಣ-ಭಂಗುರ-ಗಳೆಂದು
ಕ್ಷಣ-ಮಾ-ತ್ರ-ದ-ಲ್ಲಿ
ಕ್ಷಣಕ್ಕೆ
ಕ್ಷಣವೇ
ಕ್ಷಣಾರ್ಧ-ದ-ಲ್ಲಿ
ಕ್ಷಣಾರ್ಧ-ದ-ಲ್ಲಿಯೇ
ಕ್ಷಣಿಕ
ಕ್ಷಮಾಪಣೆ
ಕ್ಷಮಾಪಣೆ-ಯನ್ನು
ಕ್ಷಮಾಪಣೆಯ
ಕ್ಷಮಿಸ-ಬೇಕು
ಕ್ಷಮಿಸಿ
ಕ್ಷಮಿಸಿ-ರುವರು
ಕ್ಷಮಿಸು
ಕ್ಷಮಿಸು-ತ್ತೇನೆ
ಕ್ಷಯ-ವನ್ನು
ಕ್ಷಯ-ವಾಗು-ವು-ದಕ್ಕೆ
ಕ್ಷಾಮ
ಕ್ಷಾಮ-ದಿಂದ
ಕ್ಷಾಮ-ನಿ-ವಾರಣಾ
ಕ್ಷಾಮ-ಪೀಡಿ-ತ-ರಾಗಿ
ಕ್ಷೀ
ಕ್ಷೀಣ-ವಾಗಿ
ಕ್ಷೀಣ-ವಾಗು-ತ್ತ
ಕ್ಷೀಣಿ-ಸು-ತ್ತಿ-ರುವ
ಕ್ಷೀರ-ಭಾವಾನಿ
ಕ್ಷೀರಭವಾ-ನಿಗೆ
ಕ್ಷೀರಭವಾನಿ
ಕ್ಷೀರಭವಾನಿ-ಯ-ಲ್ಲಿ
ಕ್ಷೀರಭವಾನಿಯ
ಕ್ಷುದ್ರ
ಕ್ಷುಧೆ
ಕ್ಷೇ
ಕ್ಷೇಮ-ಕ್ಕಾಗಿ
ಕ್ಷೇಮ-ವೆಂದು
ಕ್ಷೋ
ಕ್ಷೌ
ಕ್ಷೌರ
ಖ
ಖಂ
ಖಂಡ
ಖಂಡ-ದ-ಲ್ಲಿ
ಖಂಡದ
ಖಂಡಿ-ಸಲೂ-ಬಹುದು
ಖಂಡಿ-ಸು-ತ್ತಿ-ದ್ದರು
ಖಂಡಿ-ಸು-ತ್ತಿದ್ದೆ
ಖಂಡಿ-ಸು-ವು-ದಕ್ಕೆ
ಖಂಡಿತ
ಖಂಡಿತ-ವಾಗಿ
ಖಂಡಿತ-ವಾಗಿಯೂ
ಖಂಡಿತ-ವಾದ
ಖಂಡಿಸ-ತೊಡಗಿದರು
ಖಂಡಿಸಿ
ಖಂಡಿಸಿ-ದರು
ಖಂಡ್ವಾ-ದ-ಲ್ಲಿ
ಖಂಡ್ವಾಕ್ಕೆ
ಖಂಡ್ವಾದ-ಲ್ಲಿದ್ದು
ಖಚೋರಿ-ಯನ್ನು
ಖಡ್ಗದ
ಖಡ್ಗದಂಚಿ-ನಿಂದ
ಖನಿ
ಖರ್ಚನ್ನು
ಖರ್ಚನ್ನೆ-ಲ್ಲ
ಖರ್ಚಾಗಿ
ಖರ್ಚಿ-ಗೆಂದು
ಖರ್ಚಿ-ನಿಂದ
ಖರ್ಚಿಗೆ
ಖರ್ಚಿನ
ಖರ್ಚು
ಖರ್ಚು-ಮಾಡು-ತ್ತಿದ್ದನು
ಖರ್ಜೂ-ರದ
ಖಾ
ಖಾಂಡ್ವ
ಖಾಂಡ್ವ-ದ-ಲ್ಲಿ
ಖಾಂಡ್ವಾ
ಖಾನೆ
ಖಾನೆಯ
ಖಾಯಾಲಿ-ಯಂತೆ
ಖಾಯಿಲ
ಖಾಯಿಲೆ
ಖಾಯಿಲೆ-ಗ-ಳಿಗೆ
ಖಾಯಿಲೆ-ಯ-ಲ್ಲಿದ್ದರೂ
ಖಾಯಿಲೆ-ಯನ್ನು
ಖಾಯಿಲೆ-ಯಾಗಿ
ಖಾಯಿಲೆ-ಯಿಂದ
ಖಾಯಿಲೆಯ
ಖಾರ
ಖಾರ-ವಾಗಿ-ತ್ತು
ಖಾಲಿ
ಖಾಲಿ-ಮಾಡು
ಖಾಲಿ-ಯಾಗಿ-ರ-ಲಿ-ಲ್ಲ
ಖಿ
ಖು
ಖುದೀ-ರಾಮ
ಖುರ್ಶಿದ್
ಖೂ
ಖೇ
ಖ್ಯಾತಿ
ಖ್ಯಾತಿ-ಗೊಂಡ
ಖ್ಯಾತಿ-ಯನ್ನು
ಗ
ಗಂ
ಗಂಗಾ
ಗಂಗಾ-ಜಲ
ಗಂಗಾ-ಜಲ-ದ-ಲ್ಲಿ
ಗಂಗಾ-ಜಲ-ದಿಂದ
ಗಂಗಾ-ಜಲ-ವನ್ನು
ಗಂಗಾ-ತೀ-ರದ
ಗಂಗಾ-ತೀರ-ದ-ಲ್ಲಿ
ಗಂಗಾ-ತೀರ-ದ-ಲ್ಲಿ-ರುವ
ಗಂಗಾ-ತೀರ-ದ-ಲ್ಲಿದ್ದ
ಗಂಗಾ-ದ್ವೀ-ಪದ
ಗಂಗಾ-ಧರ
ಗಂಗಾ-ಧರ-ಅ-ಖಂಡಾ-ನಂದ
ಗಂಗಾ-ಧರ್
ಗಂಗಾ-ನದಿ
ಗಂಗಾ-ನದಿ-ಯ-ಲ್ಲಿ
ಗಂಗಾ-ನದಿ-ಯ-ಲ್ಲೇ
ಗಂಗಾ-ನದಿ-ಯನ್ನು
ಗಂಗಾ-ನದಿಗೆ
ಗಂಗಾ-ನದಿಯ
ಗಂಗಾ-ನೀರು
ಗಂಗಾ-ಮ-ಹಿಮೆ
ಗಂಗಾ-ವಾರಿ
ಗಂಗಾ-ವಾರಿ-ಯನ್ನು
ಗಂಗಾ-ವಾರಿಯ
ಗಂಗಾ-ಸ್ನಾನ
ಗಂಗಾ-ಸ್ನಾನ-ಕ್ಕಾಗಿ
ಗಂಗೆ
ಗಂಗೆ-ಯ-ಲ್ಲಿ
ಗಂಗೆ-ಯಂತೆ
ಗಂಗೆ-ಯನ್ನು
ಗಂಜಿ-ಯನ್ನು
ಗಂಟ-ಲ-ನ್ನು
ಗಂಟ-ಲಿ-ನ-ಲ್ಲಿ
ಗಂಟ-ಲಿನ
ಗಂಟಲು
ಗಂಟುಮೂಟೆ
ಗಂಟೆ
ಗಂಟೆ-ಗ-ಟ್ಟಲೆ
ಗಂಟೆ-ಗ-ಳನ್ನು
ಗಂಟೆ-ಗ-ಳಾದ
ಗಂಟೆ-ಗಳ
ಗಂಟೆ-ಗಳ-ಲ್ಲಿ
ಗಂಟೆ-ಗಳು
ಗಂಟೆ-ಯ-ವ-ರೆಗೆ
ಗಂಟೆ-ಯಂತೆ
ಗಂಟೆ-ಯನ್ನು
ಗಂಟೆ-ಯಾ-ದರೂ
ಗಂಟೆ-ಯಾಗಿ
ಗಂಟೆ-ಯಾಗಿ-ತ್ತು
ಗಂಟೆ-ಯಾದ
ಗಂಟೆ-ಯಾದ-ಮೇಲೆ
ಗಂಟೆ-ಯಿಂದ
ಗಂಟೆಗೆ
ಗಂಟೆಯ
ಗಂಡ
ಗಂಡ-ನಂತೆಯೇ
ಗಂಡ-ನನ್ನು
ಗಂಡ-ನಾಗಿ
ಗಂಡ-ನಿಗೆ
ಗಂಡ-ನೊ-ಡನೆ
ಗಂಡ-ಸ-ರಿ-ಗಿಂತ
ಗಂಡ-ಸನ್ನು
ಗಂಡ-ಸರು
ಗಂಡ-ಸಿಗೂ
ಗಂಡಂದಿರ
ಗಂಡನ
ಗಂಡನು
ಗಂಡರ
ಗಂಡಸು
ಗಂಡಾಂ-ತರ-ವಾ-ಗಿ-ದ್ದರು
ಗಂಡು
ಗಂಡು-ಮಕ್ಕಳಿ-ದ್ದರು
ಗಂಡು-ಮಗು
ಗಂಡು-ಮಗು-ವಾ-ಯಿತು
ಗಂಡು-ಮಗು-ವಾಗು-ವಂತೆ
ಗಂಡು-ಶಿಶು-ವಿನ
ಗಂಧ
ಗಂಧವೇ
ಗಂಭೀರ
ಗಂಭೀರ-ತಮ-ವಾ-ದುದು
ಗಂಭೀರ-ತೆ-ಯನ್ನು
ಗಂಭೀರ-ತೆಯ
ಗಂಭೀರ-ನಿ-ನಾದ
ಗಂಭೀರ-ವಾಗಿ
ಗಂಭೀರ-ವಾಗಿ-ದ್ದಿತು
ಗಂಭೀರ-ವಾಗಿಯೇ
ಗಂಭೀರ-ವಾಗು-ತ್ತ
ಗಂಭೀರ-ವಾದ
ಗಂಭೀರಾ-ಕೃತಿಯ
ಗಗ-ನ್ಚಂದ್ರ
ಗಗನ
ಗಗನ-ದ-ಲ್ಲಿ
ಗಗನ-ವನ್ನು
ಗಚ್ಛತಿ
ಗಜ-ಕಡ್ಡಿ-ಯಿಂದ
ಗಜಗ-ಮನೆ
ಗಡಿ
ಗಡಿ-ಬಿಡಿ-ಯಿಂದ
ಗಡಿ-ಬಿಡಿಯ
ಗಡಿ-ಬಿಡಿಯೂ
ಗಡಿ-ಯಾರ-ವನ್ನು
ಗಡಿಯ
ಗಣ-ನೆಗೆ
ಗಣನೆಗೇ
ಗಣಿ
ಗಣಿ-ತ-ವನ್ನೇ
ಗಣಿ-ತ-ಶಾ-ಸ್ತ್ರದ
ಗಣಿ-ತದ
ಗಣಿ-ಯನ್ನು
ಗಣಿ-ಸದೆ
ಗಣಿ-ಸುವ
ಗಣಿತ
ಗಣೇಶಪ್ರ-ಯಾಗ
ಗಣೇಶಪ್ರ-ಯಾಗಕ್ಕೆ
ಗಣ್ಯ
ಗಣ್ಯ-ವ್ಯಕ್ತಿ-ಗಳು
ಗಣ್ಯರು
ಗತ-ಕಾಲದ
ಗತ-ಪ್ರಾಣ-ಳಾದ
ಗತ-ವೈಭವ-ವಾ-ಗಿದೆ
ಗತಂ
ಗತಿ
ಗತಿ-ಯನ್ನು
ಗತಿ-ಯಿ-ಲ್ಲದೇ
ಗದ್ಗದ
ಗದ್ದಲ
ಗದ್ದಲ-ದಿಂದ
ಗದ್ದಲ-ವಂತೂ
ಗದ್ದಲ-ವೆ-ಲ್ಲ
ಗದ್ದಲದ
ಗದ್ದು-ಗೆಯ
ಗಯಾ
ಗಯೆ-ಯನ್ನು
ಗಯೆಯ
ಗರ-ಡಿಯ
ಗರಡಿ
ಗರಿ-ಗಳು
ಗರಿ-ಯನ್ನು
ಗರ್ಜನೆ
ಗರ್ಜನೆ-ಯಿಂದ
ಗರ್ಜಿ-ಸಿ-ದರು
ಗರ್ಜಿ-ಸು-ತ್ತಿ-ರು-ವು-ದ-ನ್ನು
ಗರ್ಜಿ-ಸು-ತ್ತಿದೆ
ಗರ್ಜಿ-ಸು-ವು-ದ-ನ್ನು
ಗರ್ಭ-ಗುಡಿ
ಗರ್ಭ-ಗುಡಿ-ಯ-ಲ್ಲಿ
ಗರ್ಭ-ಗುಡಿ-ಯನ್ನು
ಗರ್ಭ-ಗುಡಿಗೆ
ಗರ್ಭ-ಗುಡಿಯ
ಗರ್ಭ-ದಿಂದ
ಗರ್ಭಿಣಿ
ಗರ್ವಾ-ಲ್
ಗರ್ವಿಷ್ಠ
ಗಲೀಜು
ಗಳ-ಲ್ಲಿ
ಗಳಿ-ಸು-ತ್ತಾನೆ
ಗಳಿ-ಸು-ವು-ದ-ಕ್ಕಾಗಿ
ಗಳಿ-ಸುವುದ-ರ-ಲ್ಲಿ
ಗಳಿಸ-ಬಹುದು
ಗಳಿಸ-ಬೇಕಾಗಿ-ರುವುದ-ನ್ನೆ-ಲ್ಲ
ಗಳಿಸಿ
ಗಳಿಸಿ-ಕೊಂಡು
ಗಳಿಸಿ-ದನು
ಗಳಿಸಿ-ದ್ದರು
ಗಳಿಸಿ-ರುವ
ಗಳಿಸಿದ
ಗಳಿಸು-ವುದೆಂದ-ರೇನು
ಗಹನ
ಗಹನ-ಭಾವ-ನೆ-ಯನ್ನು
ಗಹನ-ವಾದ
ಗಹ್ವರ-ಗಳ-ಲ್ಲಿ
ಗಾ
ಗಾಂಧಿ
ಗಾಂಭೀರ್ಯ
ಗಾಂಭೀರ್ಯ-ದ-ಲ್ಲಿ
ಗಾಂಭೀರ್ಯ-ದಿಂದ
ಗಾಂಭೀರ್ಯ-ವನ್ನು
ಗಾಜಿನ
ಗಾಜೀ-ಪು-ರಕ್ಕೆ
ಗಾಜೀ-ಪು-ರದ
ಗಾಜೀ-ಪುರ-ದ-ಲ್ಲಿ
ಗಾಡಾಂಧ-ಕಾರ
ಗಾಡಿ
ಗಾಡಿ-ಗಳ-ನ್ನೆ-ಲ್ಲ
ಗಾಡಿ-ಗಳ-ಲ್ಲಿ
ಗಾಡಿ-ಗಳು
ಗಾಡಿ-ಗಳೂ
ಗಾಡಿ-ಮಾಡಿ-ಕೊಂಡು
ಗಾಡಿ-ಯ-ಲ್ಲಿ
ಗಾಡಿ-ಯ-ಲ್ಲಿಯೇ
ಗಾಡಿ-ಯ-ಲ್ಲೆ
ಗಾಡಿ-ಯನ್ನು
ಗಾಡಿ-ಯಿಂದ
ಗಾಡಿ-ಯಿಂದಿ-ಳಿದು
ಗಾಡಿಗೆ
ಗಾಡಿಯ
ಗಾಢ
ಗಾಢ-ಧ್ಯಾನ
ಗಾಢ-ಧ್ಯಾನ-ದ-ಲ್ಲಿ
ಗಾಢ-ನಿದ್ರೆ-ಯ-ಲ್ಲಿ
ಗಾಢ-ವಾಗ-ತೊಡಗಿತು
ಗಾಢ-ವಾಗಿ
ಗಾಢ-ವಾಗಿ-ರು-ತ್ತಿ-ತ್ತು
ಗಾಢ-ವಾಗು-ತ್ತಾ
ಗಾಢ-ವಾದ
ಗಾಢಾಂಧ-ಕಾರ-ದಿಂದ
ಗಾಣ-ದ-ಲ್ಲಿ
ಗಾದೆ
ಗಾನ-ಮಾಡು-ತ್ತಲೂ
ಗಾನ-ಮಾಡು-ವು-ದಕ್ಕೆ
ಗಾನ-ವನ್ನು
ಗಾಯ
ಗಾಯ-ಕ-ರಿ-ದ್ದರು
ಗಾಯ-ಕವಾ-ಡದ
ಗಾಯ-ಕಿ-ಯೊಬ್ಬ-ಳನ್ನು
ಗಾಯ-ತ್ರಿ
ಗಾಯ-ತ್ರೀ
ಗಾಯ-ವನ್ನು
ಗಾಯಕಿ
ಗಾಯಿ-ಕೆಯಾದ
ಗಾರ್ಡ-ನ್
ಗಾರ್ಡ-ನ್ನ-ಲ್ಲಿ
ಗಾರ್ವಾ-ಲ್
ಗಾಲಿ-ಯನ್ನು
ಗಾಳಿ
ಗಾಳಿ-ಯ-ಲ್ಲಿ
ಗಾಳಿ-ಯ-ಲ್ಲಿ-ದ್ದಾಗ
ಗಾಳಿ-ಯನ್ನೇ
ಗಾಳಿ-ಯಾಗಿ
ಗಾಳಿ-ಯಿ-ಲ್ಲದ
ಗಾಳಿಯ
ಗಿ
ಗಿಡ
ಗಿಡ-ಗ-ಳನ್ನು
ಗಿಡ-ಗಳಿಂದ
ಗಿಡ-ಗಳು
ಗಿಡ-ದಂತೆ
ಗಿಡ-ಮರ-ಗಳು
ಗಿಡುಗ
ಗಿಡ್ಸಿ-ಸ್
ಗಿರಿ
ಗಿರಿ-ಕಂ-ದರ-ಗಳಿಂದ
ಗಿರಿ-ಝರಿ-ಗಳು
ಗಿರಿ-ತೊರೆ-ಗಳು
ಗಿರಿ-ರಾಜನ
ಗಿರಿ-ಶಿಖ-ರದ
ಗಿರಿಯ
ಗಿರೀಂದ್ರ-ನಾಥ
ಗಿರೀಶ
ಗಿರೀಶ-ಘೋಷ-ರಿಗೆ
ಗಿರೀಶ-ಘೋಷರು
ಗಿರೀಶ-ಚಂದ್ರ
ಗಿರೀಶ-ಬಾಬು
ಗಿರೀಶ-ಬಾಬು-ಗ-ಳಿಗೆ
ಗಿರೀಶ-ಬಾಬು-ಗಳ
ಗಿರೀಶ-ಬಾಬು-ಗಳು
ಗಿರೀಶ್
ಗಿರ್ನಾರ್
ಗಿರ್ನಾರ್ಗೆ
ಗೀತಾ
ಗೀತಾ-ಕರ್ತೃವು
ಗೀತಾ-ತ-ತ್ತ್ವ-ಗಳು
ಗೀತೆ
ಗೀತೆ-ಯ-ನ್ನೇನೋ
ಗೀತೆ-ಯ-ಲ್ಲಿ
ಗೀತೆ-ಯನ್ನು
ಗೀತೆಯ
ಗೀಳು
ಗು
ಗುಂಡಿ-ನಿಂದ
ಗುಂಡಿ-ಯನ್ನು
ಗುಂಪಿ-ನ-ಲ್ಲಿ
ಗುಂಪಿ-ನಿಂದ
ಗುಂಪಿಗೆ
ಗುಂಪಿನ
ಗುಂಪಿನ-ವ-ರಿಗೂ
ಗುಂಪಿನ-ವರು
ಗುಂಪು
ಗುಂಪು-ಕೂಡಿದ
ಗುಂಪು-ಗುಂಪಾಗಿ
ಗುಜು-ಗುಂಪು
ಗುಜುಗುಜು
ಗುಟುಕು
ಗುಡಿ
ಗುಡಿ-ಗ-ಳನ್ನು
ಗುಡಿ-ಗುಡಿ-ಯನ್ನು
ಗುಡಿ-ಗುಡಿ-ಯಿಂದ
ಗುಡಿ-ಗೋಪುರ-ಗಳು
ಗುಡಿ-ಯ-ಲ್ಲಿ
ಗುಡಿ-ಯ-ಲ್ಲಿಯೂ
ಗುಡಿ-ಯ-ವರು
ಗುಡಿ-ಯಿಂದ
ಗುಡಿ-ಯೊಳು
ಗುಡಿ-ಸ-ಲ-ಲ್ಲಿ
ಗುಡಿ-ಸಲು
ಗುಡಿ-ಸಲು-ಗಳ-ಲ್ಲಿ-ದ್ದರು
ಗುಡಿ-ಸಲು-ಗಳು
ಗುಡಿ-ಸಲೇ
ಗುಡಿ-ಸಿ-ದರು
ಗುಡಿ-ಸಿ-ಲಿ-ನ-ಲ್ಲಿ
ಗುಡಿ-ಸಿ-ಲಿನ
ಗುಡಿ-ಸುವ-ವರು
ಗುಡಿಗೆ
ಗುಡಿಯೂ
ಗುಡು-ಗುಡಿ
ಗುಡುಗು
ಗುಡುಗು-ಡಿ-ಯನ್ನು
ಗುಡ್ಇ-ಯರ್
ಗುಡ್ಡ
ಗುಡ್ಡ-ಗಳ-ಲ್ಲಿ
ಗುಡ್ಡದ
ಗುಡ್ವಿನ್
ಗುಡ್-ವಿನ್
ಗುಣ
ಗುಣ-ಕಥನ-ವನ್ನು
ಗುಣ-ಗ-ಳನ್ನು
ಗುಣ-ಗಳ
ಗುಣ-ಗಳ-ನ್ನೆ-ಲ್ಲ
ಗುಣ-ಗಳ-ಲ್ಲಿ
ಗುಣ-ಗಳಿ-ಲ್ಲವೆ
ಗುಣ-ಗಳಿಂದಲೂ
ಗುಣ-ಗಳು
ಗುಣ-ಗಳೆ-ರಡೂ
ಗುಣ-ಗಾನ-ದ-ಲ್ಲಿಯೂ
ಗುಣ-ಮಾಡ-ಬೇಡ
ಗುಣ-ಮಾಡು
ಗುಣ-ಮಾಡುವ
ಗುಣ-ಮುಖ-ರಾ-ದರು
ಗುಣ-ಮುಖ-ರಾ-ದಾಗ
ಗುಣ-ಮುಖ-ರಾದ
ಗುಣ-ಮುಖ-ವಾ-ಗಿದೆ
ಗುಣ-ಮುಖ-ವಾಗಿ-ಲ್ಲ
ಗುಣ-ಮುಖ-ವಾಗು-ವು-ದರೊ-ಳ-ಗಾಗಿ
ಗುಣ-ವಾ-ಗಿದೆ
ಗುಣ-ವಾ-ಯಿತು
ಗುಣ-ವಾಗ-ಬ-ಲ್ಲುದು
ಗುಣ-ವಾಗ-ಬೇಕೆಂದು
ಗುಣ-ವಾಗಲಿ
ಗುಣ-ವಾಗಿ
ಗುಣ-ವಾಗು-ತ್ತಿರಬೇಕ-ಲ್ಲವೆ
ಗುಣ-ವಾಗುವ
ಗುಣ-ವಾಚಕ-ಗ-ಳನ್ನು
ಗುಣ-ವಾಚಕ-ಗಳೆ-ಲ್ಲ
ಗುಣ-ವಾದ
ಗುಣಕ್ಕೆ
ಗುಣಾ-ತೀತ
ಗುಣಾ-ತೀತ-ವಾ-ಗಿದೆಯೋ
ಗುದ್ದಲಿ-ಯನ್ನು
ಗುದ್ದಲಿ-ಯಿಂದ
ಗುದ್ದು
ಗುಪ್ತ
ಗುಪ್ತ-ವಾದ
ಗುಪ್ತರು
ಗುರಿ
ಗುರಿ-ಮಾಡಿ-ದರು
ಗುರಿ-ಮಾಡು-ತ್ತಿದ್ದರು
ಗುರಿ-ಯ-ಲ್ಲ
ಗುರಿ-ಯ-ವ-ರೆಗೆ
ಗುರಿ-ಯನ್ನು
ಗುರಿ-ಯನ್ನೇ
ಗುರಿ-ಯಾ-ಗಿ-ಟ್ಟು-ಕೊಂಡಿ-ರುವರೋ
ಗುರಿ-ಯಾ-ಗಿ-ಟ್ಟು-ಕೊಂಡು
ಗುರಿ-ಯಾ-ಗಿ-ಟ್ಟು-ಕೊಳ್ಳ-ಬೇಕು
ಗುರಿ-ಯಾಗ-ಬೇಕು
ಗುರಿ-ಯೆ-ಡೆಗೆ
ಗುರಿ-ಯೆ-ಲ್ಲ
ಗುರಿ-ಸೇ-ರುವ-ವ-ರೆಗೂ
ಗುರಿಯ
ಗುರಿಯೂ
ಗುರಿಯೋ
ಗುರು
ಗುರು-ಗ-ಳಾದ
ಗುರು-ಗ-ಳಿಗೆ
ಗುರು-ಗಳ
ಗುರು-ಗಳಾಗಿ-ದ್ದರೆ
ಗುರು-ಗಳಾಗಿ-ರುವರು
ಗುರು-ಗಳಾಣತಿ-ಯನ್ನು
ಗುರು-ಗಳಿಂದ
ಗುರು-ಗಳು
ಗುರು-ಗಳೂ
ಗುರು-ಗಳೆ-ದು-ರಿಗೆ
ಗುರು-ಗಳೇ
ಗುರು-ಗೀ-ತ-ದಿಂದ
ಗುರು-ಚರಣ
ಗುರು-ತನ್ನು
ಗುರು-ತ್ವಾ-ಕರ್ಷಣ
ಗುರು-ದರ್ಶನ
ಗುರು-ದೇವ
ಗುರು-ದೇವ-ನಿಗೂ
ಗುರು-ದೇವ-ನಿಗೆ
ಗುರು-ದೇವ-ರ-ನ್ನು
ಗುರು-ದೇವನ
ಗುರು-ದೇವರ
ಗುರು-ದೇವರು
ಗುರು-ಪ-ತ್ನಿ-ಗ-ಳಾದ
ಗುರು-ಭಕ್ತಿ
ಗುರು-ಭಕ್ತಿಯೇ
ಗುರು-ಭಾಯಿ
ಗುರು-ಭಾಯಿ-ಗ-ಳನ್ನು
ಗುರು-ಭಾಯಿ-ಗ-ಳಾದ
ಗುರು-ಭಾಯಿ-ಗ-ಳಿಗೆ
ಗುರು-ಭಾಯಿ-ಗಳ
ಗುರು-ಭಾಯಿ-ಗಳ-ನ್ನೆ-ಲ್ಲ
ಗುರು-ಭಾಯಿ-ಗಳ-ಲ್ಲಿ
ಗುರು-ಭಾಯಿ-ಗಳ-ಲ್ಲೆ-ಲ್ಲಾ
ಗುರು-ಭಾಯಿ-ಗಳಿ-ಗೆ-ಲ್ಲ
ಗುರು-ಭಾಯಿ-ಗಳಿಂದ
ಗುರು-ಭಾಯಿ-ಗಳು
ಗುರು-ಭಾಯಿ-ಗಳೂ
ಗುರು-ಭಾಯಿ-ಗಳೆ-ಲ್ಲ
ಗುರು-ಭಾಯಿ-ಗಳೊ-ಡನೆ
ಗುರು-ಭಾಯಿ-ಗಳೊ-ಬ್ಬ-ರಿಗೆ
ಗುರು-ಭಾಯಿ-ಯನ್ನು
ಗುರು-ಭಾಯಿ-ಯೊಬ್ಬ-ನಿಗೆ
ಗುರು-ಭಾಯಿ-ಯೊಬ್ಬರು
ಗುರು-ಭಾಯಿಯ
ಗುರು-ಮಹಾ-ರಾಜ-ರಿಗೆ
ಗುರು-ರಾದಿರ-ನಾದಿಶ್ಚ
ಗುರು-ರೇವ
ಗುರು-ವನ್ನು
ಗುರು-ವರ್ಯನ
ಗುರು-ವಾಗಿ
ಗುರು-ವಾಗಿ-ದ್ದರು
ಗುರು-ವಾರ
ಗುರು-ವಿ-ನ-ಲ್ಲಿ
ಗುರು-ವಿ-ನಂತೆ
ಗುರು-ವಿ-ನಂತೆಯೇ
ಗುರು-ವಿ-ನಿಂದ
ಗುರು-ವಿ-ನೊಂದಿಗೆ
ಗುರು-ವಿಗೆ
ಗುರು-ವಿನ
ಗುರು-ಶಿಷ್ಯ
ಗುರು-ಶಿಷ್ಯ-ರಿಬ್ಬರೂ
ಗುರು-ಶಿಷ್ಯರು
ಗುರು-ಶುಶ್ರೂಷೆ
ಗುರು-ಹಿರಿ-ಯರು
ಗುರುಃ-ಪರ-ಮ-ದೈ-ವತಂ
ಗುರುತು
ಗುರುರ್ದೇವೋ
ಗುರುರ್ಬ್ರಹ್ಮಾ
ಗುರುರ್ವಿಷ್ಣುಃ
ಗುರುವೆ
ಗುರುವೇ
ಗುರೋಃ
ಗುಲಾ-ಮನೆ
ಗುಲಾ-ಮರು
ಗುಲಾಬಿ
ಗುಲಾಮ-ಗಿರಿ
ಗುಲಾಮ-ಗಿರಿ-ಯ-ಲ್ಲಿ-ತ್ತು
ಗುಲಾಮ-ಗಿರಿಯ
ಗುಲಾಮ-ರ-ನ್ನಾಗಿ
ಗುಲಾಮ-ರಾಗಿ-ರುವರು
ಗುಳ್ಳೆ
ಗುಳ್ಳೆ-ಗಳಂತೆ
ಗುಳ್ಳೆ-ಯಂತೆ
ಗುಸೂರಿ
ಗುಹ-ನನ್ನು
ಗುಹೆ
ಗುಹೆ-ಯ-ಲ್ಲಿ
ಗುಹೆ-ಯ-ಲ್ಲಿ-ರು-ತ್ತಿದ್ದ
ಗುಹೆ-ಯ-ಲ್ಲಿ-ರು-ವುವೋ
ಗುಹೆ-ಯ-ಲ್ಲಿಯೇ
ಗುಹೆ-ಯಂತಹ
ಗುಹೆ-ಯನ್ನು
ಗುಹೆ-ಯಿಂದ
ಗುಹೆ-ಯೊಂದ-ರ-ಲ್ಲಿ
ಗುಹೆ-ಯೊಳಗೆ
ಗುಹೆಗೆ
ಗುಹೆಯ
ಗೂಡ-ನ್ನು
ಗೂಡಾರ್ಥ-ವನ್ನು
ಗೂಡಿ-ನಂತೆ
ಗೂಢ-ತಮ
ಗೂಢತೆ
ಗೂರ್ಖಾ
ಗೃಹ-ವನ್ನು
ಗೃಹ-ಸ್ಥ
ಗೃಹ-ಸ್ಥ-ನ-ನ್ನಾಗಿ
ಗೃಹ-ಸ್ಥ-ನನ್ನು
ಗೃಹ-ಸ್ಥ-ನಾಗಿ
ಗೃಹ-ಸ್ಥ-ನಾಗಿರು
ಗೃಹ-ಸ್ಥ-ನಿಗೂ
ಗೃಹ-ಸ್ಥ-ನಿಗೆ
ಗೃಹ-ಸ್ಥ-ಭಕ್ತ-ನಾದ
ಗೃಹ-ಸ್ಥ-ಭಕ್ತನ
ಗೃಹ-ಸ್ಥ-ಭಕ್ತರು
ಗೃಹ-ಸ್ಥ-ರ-ನ್ನೂ
ಗೃಹ-ಸ್ಥ-ರಾ-ದರೊ
ಗೃಹ-ಸ್ಥ-ರಿಗೆ
ಗೃಹ-ಸ್ಥನು
ಗೃಹ-ಸ್ಥನೂ
ಗೃಹ-ಸ್ಥರ
ಗೃಹ-ಸ್ಥರು
ಗೃಹಿ-ಣಿಯ
ಗೃಹಿಣಿಗೆ
ಗೃಹೀ
ಗೃಹೀ-ಭಕ್ತ-ನಿಗೆ
ಗೃಹೀ-ಭಕ್ತ-ರಾದ
ಗೆ
ಗೆಜ್ಜೆ
ಗೆಡ್ಡೆ
ಗೆಡ್ಡೆ-ಗೆಣಸು-ಗ-ಳನ್ನು
ಗೆಣಸು-ಗ-ಳನ್ನು
ಗೆದ್ದ
ಗೆದ್ದ-ವನು
ಗೆದ್ದೆವೊ
ಗೆಲಿ-ಲಿಯೇ
ಗೆಲುವು-ಗ-ಳನ್ನು
ಗೆಳೆಯ
ಗೇಟ-ನ್ನ
ಗೇಟ-ನ್ನು
ಗೇಟಿ-ನ-ಲ್ಲಿ-ರುವ
ಗೇಟಿನ
ಗೇಲಿ
ಗೈರಿಕ
ಗೈರಿಕ-ವ-ಸನ
ಗೈರಿಕ-ವ-ಸನ-ಗಳಿಂದ
ಗೈರಿಕ-ವ-ಸನ-ಧಾರಿ-ಗ-ಳಾದ
ಗೈರಿಕ-ವ-ಸನ-ಧಾರಿ-ಯಾಗಿ
ಗೈರಿಕ-ವ-ಸನದ
ಗೊ
ಗೊಂಚ-ಲಿನ
ಗೊಂಚಲು
ಗೊಂದಲ-ದ-ಲ್ಲಿ
ಗೊಂದಲ-ದ-ಲ್ಲಿಯೇ
ಗೊಂದಲವೆಬ್ಬಿ-ಸು-ತ್ತಿ-ದ್ದರು
ಗೊಂಬೆ-ಗಳಂತೆ
ಗೊಚರ-ವಾಗದ
ಗೊಡ್ಡು
ಗೊಡ್ಡು-ಸಾರು
ಗೊಣಗಾ-ಡದೆ
ಗೊಣಗು-ತ್ತಿದ್ದರು
ಗೊಳಿ-ಸುವುವು
ಗೊಳ್ಳೆಂದು
ಗೋ
ಗೋಗ-ರಿಯು-ವ-ವ-ನ-ಲ್ಲ
ಗೋಚ-ರಕ್ಕೆ
ಗೋಚರ-ವಾಗು-ತ್ತಿ-ತ್ತು
ಗೋಚರ-ವಾಗು-ವು-ದಿ-ಲ್ಲ
ಗೋಚರ-ವಾಗು-ವುವು
ಗೋಜಿಗೂ
ಗೋಡೆ
ಗೋಡೆ-ಗ-ಳನ್ನು
ಗೋಡೆ-ಗಳ
ಗೋಡೆ-ಗಳು
ಗೋಡೆ-ಯನ್ನು
ಗೋಡೆಯ
ಗೋಧೂಳಿಯ
ಗೋಪಾಲ
ಗೋಪಾಲ-ದಾದ
ಗೋಪಾಲ-ದಾದ-ಅ-ದ್ವೈತಾ-ನಂದ
ಗೋಪಾಲ-ದಾದ-ನಿಗೆ
ಗೋಪಾಲ-ಲಾಲ-ಶೀಲರ
ಗೋಪಾಲರ
ಗೋಪುರ
ಗೋಪುರ-ವ-ನ್ನೊಳ-ಗೊಂಡ
ಗೋಪ್ಯ-ವಾಗಿ
ಗೋಪ್ಯ-ವಾಗಿ-ಡುವರು
ಗೋಪ್ಯ-ವಾಗಿ-ರ-ಬೇಕೆಂದು
ಗೋಪ್ಯ-ವಾಗಿ-ರುವ
ಗೋಪ್ಯ-ವಾಗಿ-ರುವರು
ಗೋರ-ಕ್ಷಣೀ
ಗೋರಕ್ಷಾ
ಗೋರಿ-ಗಳು
ಗೋಲಿ
ಗೋಳ
ಗೋಳ-ವನ್ನು
ಗೋಳಿ-ನಿಂದ
ಗೋಳು
ಗೋಳು-ಮುಖವೇ
ಗೋವಾಕ್ಕೆ
ಗೌ
ಗೌಣ
ಗೌಣ-ವಾಗಿ-ದ್ದರೂ
ಗೌರ-ವಾರ್ಥ-ವಾಗಿ
ಗೌರ-ವಿ-ಸಿ-ದರು
ಗೌರಮೋಹನ
ಗೌರೀ-ಶಂ-ಕರ
ಗೌರ್ನಮೆಂ-ಟ್
ಗೌರ್ನರ್
ಗೌರ್ನಿಂಗ್
ಗ್ಯಾ
ಗ್ರಂಥ
ಗ್ರಂಥ-ಕರ್ತ
ಗ್ರಂಥ-ಗ-ಳನ್ನು
ಗ್ರಂಥ-ಗಳ
ಗ್ರಂಥ-ಗಳ-ನ್ನೂ
ಗ್ರಂಥ-ಗಳ-ಲ್ಲಿ
ಗ್ರಂಥ-ಗಳ-ಲ್ಲೆ-ಲ್ಲಾ
ಗ್ರಂಥ-ಗಳು
ಗ್ರಂಥ-ದ-ಲ್ಲಿ
ಗ್ರಂಥ-ವ-ಲ್ಲ
ಗ್ರಂಥ-ವನ್ನು
ಗ್ರಂಥ-ವನ್ನೆ-ಲ್ಲ
ಗ್ರಂಥ-ವಾಗಿ-ತ್ತು
ಗ್ರಂಥವೇ
ಗ್ರಹ-ಗ-ಳನ್ನು
ಗ್ರಹ-ಣದ
ಗ್ರಹ-ದ-ಲ್ಲಿಯೇ
ಗ್ರಹಣ
ಗ್ರಹಣ-ಕಾಲ-ದ-ಲ್ಲಿ
ಗ್ರಹಣ-ವಾಗು-ವು-ದಕ್ಕೆ
ಗ್ರಹಣ-ವಾಗುವ
ಗ್ರಹಿ-ಸ-ಲಾ-ರದು
ಗ್ರಹಿ-ಸಲಾರದೆ
ಗ್ರಹಿ-ಸಲಾರರು
ಗ್ರಹಿ-ಸಲು
ಗ್ರಹಿ-ಸು-ತ್ತ-ದೆಯೆ
ಗ್ರಹಿ-ಸು-ವಂತೆ
ಗ್ರಹಿ-ಸು-ವುದು
ಗ್ರಹಿಸಿ
ಗ್ರಹಿಸಿ-ದರು
ಗ್ರಹಿಸಿ-ದ್ದರು
ಗ್ರಹಿಸಿ-ಬಿಡು-ತ್ತಿದ್ದ
ಗ್ರಹಿಸಿ-ಬಿಡು-ತ್ತಿದ್ದರು
ಗ್ರಹಿಸಿ-ರು-ವು-ದಾಗಿಯೂ
ಗ್ರಹಿಸಿದ
ಗ್ರಹಿಸಿದೆ
ಗ್ರಹಿಸು-ವಂತಹ
ಗ್ರಾಂಡ್
ಗ್ರಾಮ
ಗ್ರಾಮ-ಗ್ರಾಮ-ಗಳಿಗೂ
ಗ್ರಾಮ-ದ-ಲ್ಲಿ
ಗ್ರಾಮ-ವನ್ನು
ಗ್ರಾಮ-ಸ್ಥ-ರಿಗೆ
ಗ್ರಾಮದ
ಗ್ರಾಸ-ವಾಗಿ
ಗ್ರೀ
ಗ್ರೀಕ್
ಗ್ರೀಷ್ಮ
ಗ್ರೀಷ್ಮ-ಕಾಲದ
ಗ್ಲಾ
ಗ್ವಾ
ಘ
ಘಂಟಾ-ಘೋಷ-ದಿಂದ
ಘಂಟಾನಾ-ದವು
ಘಂಟೆ
ಘಂಟೆ-ಗಳ
ಘಂಟೆ-ಯನ್ನು
ಘಂಟೆಯ
ಘಟ-ನೆಗೆ
ಘಟ-ನೆಯ
ಘಟ-ನೆಯೂ
ಘಟ-ಸರ್ಪ-ವನ್ನು
ಘಟನೆ
ಘಟನೆ-ಗ-ಳನ್ನು
ಘಟನೆ-ಗ-ಳಿಗೆ
ಘಟನೆ-ಗಳ
ಘಟನೆ-ಗಳ-ನ್ನೆ-ಲ್ಲ
ಘಟನೆ-ಗಳ-ಲ್ಲಿ
ಘಟನೆ-ಗಳಿವೆಯೋ
ಘಟನೆ-ಗಳು
ಘಟನೆ-ಗಳೇ
ಘಟನೆ-ಗಳೋ
ಘಟನೆ-ಯ-ಲ್ಲದೆ
ಘಟನೆ-ಯಂತೆ
ಘಟನೆ-ಯನ್ನು
ಘನ
ಘನೀ-ಭೂತ-ವಾಗಿ
ಘನೀ-ಭೂತ-ವಾದ
ಘರೆ
ಘರ್ಷಣೆ
ಘರ್ಷಣೆ-ಗಳಿಂದ
ಘರ್ಷಣೆ-ಗಳು
ಘರ್ಷಣೆ-ಯ-ಲ್ಲಿ
ಘರ್ಷಣೆಯ
ಘಾಟಿ-ನ-ಲ್ಲಿ
ಘಾಟಿ-ನಿಂದ
ಘಾಟಿಗೆ
ಘೋರ-ಧರ್ಮ
ಘೋರ-ವಾ-ಗಿದೆ
ಘೋಷ
ಘೋಷ-ಜರೆ
ಘೋಷರು
ಘೋಷರೇ
ಘೋಷಾ
ಘೋಷಿ-ಸಿದುವು
ಘೋಷಿ-ಸು-ತ್ತಾ
ಚ
ಚಂಗ-ಲ್-ಪೇಟೆಯ
ಚಂಚ-ಲರಾಗು-ತ್ತಾರೆಯೋ
ಚಂಚಲ-ರಾಗಿ-ರುವಿರಾ
ಚಂಚಲ-ವಾಗೇ
ಚಂಚಲ-ವಾದ
ಚಂಡ-ಮಾರು-ತ-ದಿಂದ
ಚಂಡಾಲ
ಚಂಡಾಲ-ನ-ವ-ರೆಗೆ
ಚಂಡಾಲ-ನನ್ನೂ
ಚಂಡಾಲ-ನಾಗಿ-ರಲಿ
ಚಂಡಾಲ-ರ-ವ-ರೆಗೆ
ಚಂಡಾಲನ
ಚಂಡಿಕೇಶ್ವರ
ಚಂದ-ನಾದಿ
ಚಂದ-ನಾದಿ-ಗ-ಳನ್ನು
ಚಂದಾ
ಚಂದ್ರ
ಚಂದ್ರ-ಘೋಷ್ನೊ-ಡನೆ
ಚಂದ್ರ-ದ-ತ್ತ
ಚಂದ್ರ-ನಾಥ
ಚಂದ್ರ-ನಿ-ಗಿಂತ
ಚಂದ್ರ-ಭ-ಟ್ಟಾ-ಚಾರ್ಯ
ಚಂದ್ರ-ಮಣಿ-ದೇವಿಯೇ
ಚಂದ್ರ-ಮಣೀ-ದೇ-ವಿಗೆ
ಚಂದ್ರ-ಮನ
ಚಂದ್ರ-ವದನ-ವನ್ನು
ಚಂದ್ರೋದಯ-ವಾ-ಯಿತು
ಚಂಬಿ-ನ-ಲ್ಲಿ
ಚಕ-ವರ್ತಿ-ಯ-ವ-ರದೂ
ಚಕ್ರ
ಚಕ್ರ-ದಿಂದ
ಚಕ್ರ-ವರ್ತಿ
ಚಕ್ರ-ವರ್ತಿ-ಗ-ಳಿದ್ದ
ಚಕ್ರ-ವರ್ತಿ-ಯಂತೆ
ಚಕ್ರ-ವರ್ತಿ-ಯದು
ಚಕ್ರ-ವರ್ತಿ-ಯನ್ನು
ಚಕ್ರ-ವರ್ತಿಗೂ
ಚಕ್ರಕ್ಕೆ
ಚಕ್ರದ
ಚಕ್ರಾ-ಕಾರ-ವಾಗಿ
ಚಕ್ರಾಧಿಪ-ತ್ಯ
ಚಕ್ರಾಧಿಪ-ತ್ಯ-ಗಳ
ಚಕ್ರಾಧಿಪ-ತ್ಯ-ಗಳು
ಚಕ್ರಾಧಿಪ-ತ್ಯ-ಗಳೆ-ಲ್ಲ
ಚಕ್ರಾಧಿಪ-ತ್ಯ-ದ-ಲ್ಲೆ-ಲ್ಲಾ
ಚಕ್ರಾಧಿಪ-ತ್ಯದ
ಚಚ್ಚಿ-ಕೊ-ಳ್ಳು-ತ್ತ
ಚಟುಲ
ಚಟುವಟಿ-ಕೆ-ಗ-ಳನ್ನು
ಚಟುವಟಿ-ಕೆಯ
ಚಟುವಟಿಕೆ-ಯ-ಲ್ಲಿ
ಚಟುವಟಿಕೆ-ಯುಳ್ಳ-ವ-ನಾಗಿ
ಚತುರ-ತೆ-ಯಿಂದ
ಚದುರ
ಚಪಾತಿ
ಚಪ್ಪ-ರಕ್ಕೆ
ಚಪ್ಪ-ರದ
ಚಪ್ಪರ
ಚಪ್ಪರ-ಗಳು
ಚಪ್ಪರ-ದ-ಲ್ಲಿ
ಚಪ್ಪರ-ದ-ವ-ರೆಗೆ
ಚಪ್ಪರ-ದಿಂದ
ಚಪ್ಪರ-ದೆ-ಡೆಗೆ
ಚಪ್ಪರ-ದೊಳಗೆ
ಚಪ್ಪರ-ವನ್ನು
ಚಪ್ಪಾಳೆ
ಚಪ್ಪಿ-ನ-ಲ್ಲಿ
ಚಬೀ-ಲ್ದಾಸರು
ಚಬೀ-ಲ್ದಾಸ್
ಚಮಕಿ-ತ-ರಾ-ದರು
ಚಮೋನಿಕ್ಸ್
ಚಮೋನಿಕ್ಸ್-ನಿಂದ
ಚರಂಡಿ
ಚರಂಡಿ-ಯಂತೆ
ಚರಂಡಿಯ
ಚರಕ-ದಿಂದ
ಚರಣ-ಗಳ
ಚರಣ-ಧೂಳಿ-ಯನ್ನು
ಚರಮ
ಚರಮ-ಗುರಿ
ಚರಮ-ಸಿದ್ಧಾಂತ-ವೆಂದು
ಚರಮ-ಸಿದ್ಧಾಂತ-ವೆಂಬುದು
ಚರಾಚರ
ಚರಿ-ತಾಮೃತ
ಚರಿ-ತ್ರಾರ್ಹ-ವಾದ
ಚರಿ-ತ್ರೆ
ಚರಿ-ತ್ರೆ-ಯ-ಲ್ಲಿ
ಚರಿ-ತ್ರೆ-ಯನ್ನು
ಚರಿ-ತ್ರೆ-ಯನ್ನೆ-ಲ್ಲ
ಚರಿ-ತ್ರೆಗೂ
ಚರಿ-ತ್ರೆಗೆ
ಚರ್ಚನ್ನು
ಚರ್ಚಿ-ಗ-ಳಾಗು-ವರು
ಚರ್ಚಿ-ನ-ಲ್ಲಿ
ಚರ್ಚಿ-ಸು-ತ್ತ
ಚರ್ಚಿ-ಸು-ತ್ತಾ
ಚರ್ಚಿ-ಸು-ತ್ತಿದ್ದ
ಚರ್ಚಿ-ಸು-ತ್ತಿದ್ದೇನೆ
ಚರ್ಚಿ-ಸು-ವು-ದ-ನ್ನು
ಚರ್ಚಿ-ಸು-ವುದು
ಚರ್ಚಿಗೆ
ಚರ್ಚಿನ
ಚರ್ಚಿನ-ಲ್ಲಿ-ರು-ವ-ವರು
ಚರ್ಚಿನ-ವ-ರೆ-ದು-ರಿಗೆ
ಚರ್ಚಿಸ-ತೊಡಗಿದರು
ಚರ್ಚಿಸ-ಬೇಕಾಗಿ-ತ್ತು
ಚರ್ಚಿಸ-ಬೇಕು
ಚರ್ಚಿಸ-ಬೇಕೆಂದು
ಚರ್ಚಿಸಿ
ಚರ್ಚಿಸಿ-ದರು
ಚರ್ಚಿಸಿ-ದಳು
ಚರ್ಚಿಸಿದ
ಚರ್ಚು
ಚರ್ಚು-ಗ-ಳನ್ನು
ಚರ್ಚು-ಗಳ
ಚರ್ಚೆ
ಚರ್ಚೆ-ಮಾಡಲು
ಚರ್ಚೆ-ಮಾಡು-ವಾಗ
ಚರ್ಚೆ-ಮಾಡು-ವುದು
ಚರ್ಚೆ-ಯ-ಲ್ಲಿ
ಚರ್ಚೆ-ಯನ್ನು
ಚರ್ಚೆ-ಯಾಗು-ತ್ತಿದ್ದಾಗ
ಚರ್ಚೆಗೆ
ಚರ್ಚ್
ಚರ್ಮ-ದಷ್ಟು
ಚರ್ಮ-ದಿಂದ
ಚರ್ಮ-ದೊಂದಿಗೆ
ಚಲ-ಮಾನ
ಚಲನ-ದೊಂದಿಗೆ
ಚಲನವಲನ-ಗ-ಳಿಗೆ
ಚಲನವಲನ-ಗಳ-ಲ್ಲಿ
ಚಲನೆ
ಚಲಾಯಿ-ಸಲು
ಚಲಿ-ಸ-ಲಿ-ಲ್ಲ
ಚಲಿ-ಸದೆ
ಚಲಿ-ಸಲೂ
ಚಲಿ-ಸಿತು
ಚಲಿ-ಸು-ತ್ತಿ-ರು-ವಂತೆ
ಚಲಿ-ಸು-ತ್ತಿ-ರು-ವನು
ಚಲಿ-ಸು-ತ್ತಿ-ರು-ವುದು
ಚಲಿ-ಸು-ವಂತೆ
ಚಲಿ-ಸು-ವುದು
ಚಲಿ-ಸುವ
ಚಳಿ
ಚಳಿ-ಗಾಲ-ವಾಗಿ-ದ್ದು-ದ-ರಿಂದ
ಚಳಿ-ಗಾಲದ
ಚಳಿ-ಗಾಳಿ
ಚಳಿ-ಯಿಂದ
ಚಳಿ-ಯೇನು
ಚಳುವಳಿ
ಚಳುವಳಿಯ
ಚಾ
ಚಾಂಡಾಲ
ಚಾಚಿ
ಚಾಡಿ
ಚಾತುರ್ಯ-ವುಳ್ಳ
ಚಾಪೆ
ಚಾಪೆ-ಯ-ಲ್ಲದೆ
ಚಾಪೆ-ಯನ್ನು
ಚಾಮ-ರಾಜ
ಚಾಮ-ರಾಜೇಂದ್ರ
ಚಾರಿ-ತ್ರಿಕ
ಚಾರಿ-ತ್ರಿಕ-ವಾಗಿ
ಚಾರಿ-ತ್ರಿಕವೇ
ಚಾರಿ-ತ್ರ್ಯ
ಚಾರಿ-ತ್ರ್ಯ-ವಿ-ತ್ತು
ಚಾರಿ-ತ್ರ್ಯ-ವಿ-ಲ್ಲದೇ
ಚಾರು-ಚೂರು
ಚಾರ್ಲೆ-ಸ್
ಚಾರ್ವಾಕ
ಚಾಲಕ
ಚಾಳಿ-ಯಾಗಿ
ಚಾವ-ಣಿಯೂ
ಚಾವಟಿ
ಚಾವಟಿ-ಯಿಂದ
ಚಾವಣಿ
ಚಾವಣಿ-ಯನ್ನು
ಚಾವಣಿಗೆ
ಚಿ
ಚಿಂ
ಚಿಂತಾ-ಭಾರ
ಚಿಂತಾ-ಮಗ್ನ-ರಾಗಿ-ದ್ದರು
ಚಿಂತಾ-ಮಗ್ನ-ರಾಗಿ-ರು-ವುದು
ಚಿಂತಾ-ಕ್ರಾಂತ-ರಾಗಿ-ದ್ದರು
ಚಿಂತಿ-ಸಲೂ
ಚಿಂತಿ-ಸು-ತ್ತ
ಚಿಂತಿಸ-ತೊಡಗಿದರು
ಚಿಂತಿಸ-ತೊಡಗಿದೆ
ಚಿಂತಿಸಿ
ಚಿಂತಿಸಿ-ದರೆ
ಚಿಂತಿಸು
ಚಿಂತಿಸು-ತಿ-ತ್ತು
ಚಿಂತಿಸು-ತ್ತಿ-ತ್ತು
ಚಿಂತಿಸು-ತ್ತಿ-ರಲಿ-ಲ್ಲ
ಚಿಂತಿಸು-ತ್ತಿದ್ದಂತೆ
ಚಿಂತಿಸು-ತ್ತಿದ್ದರೆ
ಚಿಂತಿಸು-ತ್ತಿರು-ವಾಗ
ಚಿಂತಿಸು-ವು-ದಕ್ಕೆ
ಚಿಂತಿಸು-ವು-ದಿ-ಲ್ಲ
ಚಿಂತಿಸು-ವುದು
ಚಿಂತಿಸು-ವುದೂ
ಚಿಂತೆ-ಯಿ-ಲ್ಲ
ಚಿಂದಿಯ
ಚಿಂದಿಯ-ಲ್ಲಿ-ರು-ವುದು
ಚಿಕಾಗೊ
ಚಿಕಾಗೋ
ಚಿಕಾಗೋ-ದಿಂದ
ಚಿಕಾಗೋ-ನ-ಗರ-ದ-ಲ್ಲಿ
ಚಿಕಿ-ತ್ಸೆ
ಚಿಕಿ-ತ್ಸೆ-ಗಾಗಿ
ಚಿಕಿ-ತ್ಸೆ-ಯ-ಲ್ಲಿ
ಚಿಕಿ-ತ್ಸೆ-ಯನ್ನು
ಚಿಕಿ-ತ್ಸೆ-ಯಿಂದ
ಚಿಕಿ-ತ್ಸೆಗೆ
ಚಿಕಿ-ತ್ಸೆಯ
ಚಿಕ್ಕ
ಚಿಕ್ಕ-ದಾದ
ಚಿಕ್ಕ-ಮ್ಮ
ಚಿಕ್ಕಪ್ಪ
ಚಿತಾವಣೆ
ಚಿತೆ-ಯ-ಲ್ಲಿ
ಚಿನ್ನ
ಚಿನ್ನ-ವ-ನ್ನಾಗಿ
ಚಿನ್ನದ
ಚಿನ್ನದ್ದು
ಚಿನ್ಮಯ
ಚಿನ್ಮಯಾ-ದೇವಿ-ಯನ್ನು
ಚಿರ-ಋಣಿ-ಗಳಾಗಿ-ರ-ಬೇಕು
ಚಿರ-ಕಾಲ
ಚಿರ-ಪರಿಚಿ-ತ-ನಾದ
ಚಿರ-ಸ್ಥಾಯಿ-ಯಾಗಿ
ಚಿರ-ಸ್ನೇಹಿ-ತ-ರಾ-ದರು
ಚಿರ-ಸ್ಮರ-ಣೀಯ
ಚಿರ-ಸ್ಮರ-ಣೀಯ-ವ-ನ್ನಾಗಿ
ಚಿರ-ಸ್ಮರ-ಣೀಯ-ವಾಗಿ
ಚಿರಂ-ಜೀವಿ-ಯಾಗಿ-ದ್ದರೆ
ಚಿರಂ-ತರ
ಚಿರಂಜೀ-ವಿ-ಯ-ನ್ನಾಗಿ
ಚಿರಕೃತಜ್ಞ-ನಾಗಿ-ರು-ವಂತೆ
ಚಿರಮುದ್ರಿತ-ವಾ-ಗಿದೆ
ಚಿರಮುದ್ರಿತ-ವಾಗಿವೆ
ಚಿಲುಮೆ-ಯನ್ನು
ಚಿಲುಮೆ-ಯಿಂದ
ಚಿಲುಮೆಯೂ
ಚಿಹ್ನೆ
ಚಿಹ್ನೆ-ಗಳಿವೆ
ಚಿಹ್ನೆ-ಗಳೆಂದು
ಚಿಹ್ನೆ-ಯಂತೆ
ಚಿಹ್ನೆ-ಯನ್ನು
ಚಿಹ್ನೆ-ಯಾಗಿ
ಚಿಹ್ನೆ-ಯಾಗಿ-ರು-ವಳು
ಚಿಹ್ನೆಯ
ಚಿಹ್ನೆಯೇ
ಚೀಟಿ-ಯ-ಲ್ಲಿ
ಚೀಟಿ-ಯನ್ನು
ಚೀಣ
ಚೀಣರು
ಚೀಣಾ
ಚೀಣಾದ
ಚೀಣಿ
ಚೀಣೀ
ಚೀಣೀ-ಯನ
ಚೀಣೀ-ಯನು
ಚೀಣೀಯ
ಚೀಫ್
ಚೀಲ-ದ-ಲ್ಲಿ
ಚೀಲದ-ಲ್ಲಿ-ಟ್ಟು
ಚು
ಚುಂಬಿ-ಸಲು
ಚುಕ್ಕಾ-ಣಿಯ
ಚುಕ್ಕಾಣಿ-ಗಳುಳ್ಳ
ಚುಚುಂದ-ರವಧ
ಚುಚ್ಚಿ
ಚುಚ್ಚಿ-ದಂ-ತಾಗು-ತ್ತಿ-ತ್ತು
ಚುರುಕಾಗಿ-ತ್ತು
ಚುರುಕು
ಚುರುಕೋ
ಚೂ
ಚೂರು
ಚೂರ್ಣೀ-ಭೂತ
ಚೂರ್ಣೀ-ಭೂತ-ವಾಗ-ಬೇಕು
ಚೆ
ಚೆಂಡಿ-ನಂತೆ
ಚೆಂಬೊ-ನ್ನಾಗಿ
ಚೆಕ್
ಚೆಕ್ಕು
ಚೇ
ಚೇತೊ-ಹಾರಿ-ಯಾ-ಗಿದೆ
ಚೇತೋ-ಹಾರಿ-ಯಾ-ಗಿದೆ
ಚೇಪಾಕ್
ಚೇಳು
ಚೇಷ್ಟೆ
ಚೇಷ್ಟೆ-ಮಾಡು-ವುದು
ಚೈ
ಚೈನಾ
ಚೈನಾ-ವನ್ನು
ಚೈನೀ
ಚೈನೀ-ಯ-ರ-ನ್ನು
ಚೈನೀ-ಯ-ರೆ-ಲ್ಲ
ಚೊಂಬು
ಚೊಕ್ಕಟ-ವಾ-ಗಿದೆ
ಚೊಕ್ಕಟ-ವಾಗಿ-ದ್ದುವು
ಚೋಟ-ಗೋಪಾಲ
ಚೌ
ಚೌದುರಿ-ಯೊ-ಡನೆ
ಚ್ಯುತ-ನಾಗಿ-ರು-ವೆ-ನೆಂದು
ಚ್ಯುತ-ರಾದ
ಚ್ಯುತಿ
ಛ
ಛಲ
ಛಳಿ
ಛಳಿ-ಗಾಲ-ಗಳ
ಛಳಿ-ಯಿಂದ
ಛಾಯಾ
ಛಾಯಾ-ಚಿ-ತ್ರ-ಗಳ
ಛಾಯಾ-ಮೂರ್ತಿ-ಗಳಿರಾ
ಛಾಯಾ-ರೂಪ-ವಾಗಿ
ಛಾಯೆ
ಛಾಯೆ-ಗಳಿ-ಲ್ಲ
ಛಾಯೆ-ಗಳು
ಛಾಯೆ-ಯಂತೆ
ಛಾಯೆ-ಯನ್ನು
ಛಾಯೆಯ
ಛಾವಣಿ-ಯನ್ನೇ
ಛಿ
ಛೀ
ಛೇದಿ-ಸ-ಲಾ-ರದು
ಜ
ಜಂಗ-ಮ-ಗಳೆ-ಲ್ಲವೂ
ಜಂತು-ಗಳೂ
ಜಗ-ತ್
ಜಗ-ತ್ಕ-ಲ್ಯಾಣ
ಜಗ-ತ್ತ-ನ್ನೆ-ಲ್ಲ
ಜಗ-ತ್ತನ್ನು
ಜಗ-ತ್ತಿ-ಗೆ-ಲ್ಲ
ಜಗ-ತ್ತಿ-ನ-ಲ್ಲಿ
ಜಗ-ತ್ತಿ-ನಿಂದ
ಜಗ-ತ್ತಿಗೂ
ಜಗ-ತ್ತಿಗೆ
ಜಗ-ತ್ತಿಗೇ
ಜಗ-ತ್ತಿನ
ಜಗ-ತ್ತಿನ-ಲ್ಲಿ-ದ್ದವು
ಜಗ-ತ್ತಿನ-ಲ್ಲಿ-ರುವ
ಜಗ-ತ್ತಿನ-ಲ್ಲೆ-ಲ್ಲ
ಜಗ-ತ್ತಿನ-ಲ್ಲೆ-ಲ್ಲಾ
ಜಗ-ತ್ತು
ಜಗ-ತ್ತೆಂಬ
ಜಗ-ತ್ತೇ
ಜಗ-ತ್ಪಾಲಕ-ನಾದ
ಜಗ-ದಾದಿ
ಜಗ-ದಾದಿ-ಯಿಂದಲೂ
ಜಗ-ದಿಂದ
ಜಗ-ನ್ನಾಥ
ಜಗ-ನ್ನಾಥ-ರಥದ
ಜಗ-ನ್ನಾಥನ
ಜಗ-ನ್ಮಯಿ
ಜಗ-ನ್ಮಯಿ-ಯ-ಲ್ಲಿ
ಜಗ-ನ್ಮಯಿ-ಯನ್ನು
ಜಗ-ನ್ಮಯಿಗೆ
ಜಗ-ನ್ಮಯಿಯ
ಜಗ-ನ್ಮಯಿಯು
ಜಗ-ನ್ಮಾತೆ
ಜಗ-ನ್ಮಾತೆ-ಯಂತೆ
ಜಗ-ನ್ಮಾತೆ-ಯನ್ನು
ಜಗ-ನ್ಮಾತೆಗೆ
ಜಗ-ನ್ಮಾತೆಯ
ಜಗ-ನ್ಮಾತೆಯೇ
ಜಗಜ್ಜ-ನನಿಯ
ಜಗಜ್ಜಾಲಾ-ತ್
ಜಗದ
ಜಗದ-ಮೇಲೆ
ಜಗದೀ-ಶನ
ಜಗದೀಶ
ಜಗದ್ಗುರು-ಗ-ಳಾದ
ಜಗದ್ವಿಖ್ಯಾತ
ಜಗದ್ವಿಖ್ಯಾತ-ರಾ-ದರು
ಜಗದ್ವಿಖ್ಯಾತ-ರಾ-ದರೋ
ಜಗದ್ವಿಖ್ಯಾತ-ರಾದ
ಜಗದ್-ವಿಜಯಿ
ಜಗಮೋಹ-ನ್
ಜಗಮೋಹ-ನ್ಲಾಲರು
ಜಗಮೋಹ-ನ್ಲಾಲ್
ಜಗಮೋಹ-ನ್ಲಾಲ್ರೊಂದಿಗೆ
ಜಗಮೋಹನ-ಲಾಲರ
ಜಗಮೋಹನ-ಲಾಲ್
ಜಗು-ಲಿಯ
ಜಟಾಜೂಟ
ಜಟಿಲ
ಜಟಿಲ-ವಾದ
ಜಟೆ-ಗಳು
ಜಡ
ಜಡ-ನಾಗಿ
ಜಡ-ನಿದ್ರೆ-ಯನ್ನು
ಜಡ-ವ-ಸ್ತು-ವ-ಲ್ಲ
ಜಡ-ವ-ಸ್ತು-ವಿನ
ಜಡ-ವನ್ನು
ಜಡ-ವಾದ
ಜಡ-ವಾದದ
ಜಡ-ಸಮಾಧಿ-ಯ-ಲ್ಲಿ
ಜಡರಂ-ತಾಗಿ-ಬಿ-ಟ್ಟಿ-ದ್ದರು
ಜಡೆ-ಗಳಿಂದ
ಜಡೆ-ಯನ್ನೂ
ಜಡೆಸ-ಹಿತ
ಜಡ್ಜ್
ಜಡ್ಡಿ-ಗಳು
ಜತೆ-ಗಾರರು
ಜತೆ-ಯ-ಲ್ಲಿ
ಜನ
ಜನ-ಕ-ನಂತೆ
ಜನ-ಕ-ರಾಜ-ನಂತೆ
ಜನ-ಕ-ರಾಜನ
ಜನ-ಕನ
ಜನ-ಗ-ಳಿಗೆ
ಜನ-ಗಳ
ಜನ-ಗಳು
ಜನ-ಗಳೂ
ಜನ-ಗಳೆ-ಲ್ಲ
ಜನ-ಜೀವನ
ಜನ-ತೆಗೆ
ಜನ-ನದ
ಜನ-ರ-ನ್ನು
ಜನ-ರ-ಲ್ಲಿ
ಜನ-ರ-ಲ್ಲಿಯೇ
ಜನ-ರಂತೆ
ಜನ-ರಿ-ಗಿಂತ
ಜನ-ರಿ-ಗೆ-ಲ್ಲ
ಜನ-ರಿ-ದ್ದರು
ಜನ-ರಿಂದ
ಜನ-ರಿಗೂ
ಜನ-ರಿಗೆ
ಜನ-ರೆ-ದು-ರಿಗೆ
ಜನ-ರೆ-ಲ್ಲ
ಜನ-ರೆ-ಲ್ಲಾ
ಜನ-ರೊ-ಡನೆ
ಜನ-ರೊಂದಿಗೆ
ಜನ-ವ-ರಿಯ
ಜನ-ವ-ಲ್ಲ
ಜನ-ವರಿ
ಜನ-ವರಿ-ಯಿಂದ
ಜನ-ವಾರಿಧಿ
ಜನ-ಶೂ-ನ್ಯ
ಜನ-ಸಂಖ್ಯೆ
ಜನ-ಸಂಖ್ಯೆ-ಯ-ಲ್ಲಿ
ಜನ-ಸಂಖ್ಯೆ-ಯುಳ್ಳ
ಜನ-ಸಂಘ
ಜನ-ಸಂದ-ಣಿ-ಯನ್ನು
ಜನ-ಸಂದ-ಣಿಯ
ಜನ-ಸಂದ-ಣಿಯು
ಜನ-ಸಂದಣಿ
ಜನ-ಸಮುದಾಯ-ದ-ಲ್ಲಿ
ಜನ-ಸಾ-ಧಾರಣ-ದ-ಲ್ಲಿ
ಜನ-ಸಾ-ಧಾರಣ-ರ-ನ್ನು
ಜನ-ಸಾ-ಧಾರಣ-ರ-ಲ್ಲಿ
ಜನ-ಸಾ-ಧಾರಣ-ರಿಗೆ
ಜನ-ಸಾ-ಧಾರಣ-ರೆ-ಲ್ಲ
ಜನ-ಸಾ-ಧಾರಣರ
ಜನ-ಸಾ-ಧಾರಣರು
ಜನ-ಸಾ-ಮಾನ್ಯ-ರ-ನ್ನು
ಜನ-ಸಾ-ಮಾನ್ಯ-ರಿಗೆ
ಜನ-ಸಾ-ಮಾನ್ಯರ
ಜನ-ಸಾ-ಮಾನ್ಯರು
ಜನ-ಸಾ-ಮಾನ್ಯರೇ
ಜನ-ಸೇವೆ-ಯನ್ನು
ಜನ-ಸ್ತೋಮ
ಜನವೂ
ಜನಾಂಗ
ಜನಾಂಗ-ಗ-ಳನ್ನು
ಜನಾಂಗ-ಗ-ಳಿಗೆ
ಜನಾಂಗ-ಗಳಿ-ಗಿಂತ
ಜನಾಂಗ-ಗಳು
ಜನಾಂಗ-ಗಳೆ-ದು-ರಿಗೆ
ಜನಾಂಗ-ದ-ಲ್ಲಿ
ಜನಾಂಗ-ದ-ಲ್ಲಿ-ರುವ
ಜನಾಂಗ-ದ-ಲ್ಲಿಯೂ
ಜನಾಂಗ-ದ-ವರು
ಜನಾಂಗ-ದ-ವರೂ
ಜನಾಂಗ-ದ-ವರೋ
ಜನಾಂಗ-ದಷ್ಟು
ಜನಾಂಗ-ದಿಂದಲೇ
ಜನಾಂಗ-ವನು
ಜನಾಂಗ-ವನ್ನು
ಜನಾಂಗ-ವಾ-ದರೂ
ಜನಾಂಗ-ವಾಗಿ-ಲ್ಲ
ಜನಾಂಗ-ವೇನು
ಜನಾಂಗಕ್ಕೆ
ಜನಾಂಗದ
ಜನಾಂಗವು
ಜನಾಂಗವೂ
ಜನಾಭಿ-ಪ್ರಾಯ-ವನ್ನು
ಜನಿ-ವಾರ
ಜನಿ-ವಾರ-ವನ್ನು
ಜನಿ-ಸಿ-ದರು
ಜನಿ-ಸಿ-ದರೆ
ಜನಿ-ಸಿದ
ಜನಿ-ಸಿದೆ
ಜನಿ-ಸುವ-ವನು
ಜನಿಸಿ-ದು-ದ-ರಿಂದ
ಜನ್ಮ
ಜನ್ಮ-ಗ-ಳನ್ನು
ಜನ್ಮ-ಗಳ
ಜನ್ಮ-ಗಳು
ಜನ್ಮ-ತಾ-ಳಿದ
ಜನ್ಮ-ತಾ-ಳಿದ್ದು
ಜನ್ಮ-ತಿಥಿಯ
ಜನ್ಮ-ದ-ಲ್ಲಿ
ಜನ್ಮ-ದ-ಲ್ಲಿಯೇ
ಜನ್ಮ-ದ-ಲ್ಲೇ
ಜನ್ಮ-ದಿ-ನಕ್ಕೆ
ಜನ್ಮ-ದಿನ
ಜನ್ಮ-ಭೂಮಿ
ಜನ್ಮ-ಭೂಮಿಯ
ಜನ್ಮ-ವನ್ನು
ಜನ್ಮ-ವಿ-ತ್ತ
ಜನ್ಮ-ವಿ-ತ್ತೆ
ಜನ್ಮ-ವಿದೆ
ಜನ್ಮ-ವೆ-ತ್ತ-ಬೇಕು
ಜನ್ಮ-ವೆ-ತ್ತಿ
ಜನ್ಮ-ವೆ-ತ್ತಿ-ದು-ದ-ರಿಂದ
ಜನ್ಮ-ವೆ-ತ್ತಿ-ಬಂದನೋ
ಜನ್ಮ-ವೆ-ತ್ತಿ-ರು-ವಿರಿ
ಜನ್ಮ-ವೆ-ತ್ತಿ-ರು-ವುದು
ಜನ್ಮ-ವೆ-ತ್ತಿ-ಲ್ಲ
ಜನ್ಮ-ವೆ-ತ್ತಿದ
ಜನ್ಮ-ಸ್ಥಳ-ವನ್ನು
ಜನ್ಮ-ಸ್ಥಳಕ್ಕೆ
ಜನ್ಮ-ಸ್ಥಾನ-ವಾದ
ಜನ್ಮಕ್ಕೆ
ಜನ್ಮತಃ
ಜನ್ಮದ
ಜನ್ಮೋ-ತ್ಸವ
ಜನ್ಮೋ-ತ್ಸವ-ಕ್ಕಾಗಿ
ಜನ್ಮೋ-ತ್ಸವಕ್ಕೆ
ಜನ್ಯ
ಜಪ
ಜಪ-ಗ-ಳನ್ನು
ಜಪ-ಧ್ಯಾನ
ಜಪ-ಮಾಡಿ-ದರು
ಜಪ-ಮಾಲೆ-ಗ-ಳನ್ನು
ಜಪ-ವನ್ನು
ಜಪತಪಾ-ದಿ-ಗ-ಳನ್ನು
ಜಪಾ-ದಿ-ಗ-ಳನ್ನು
ಜಪಾ-ನನ್ನು
ಜಪಿ-ಸು-ತ್ತಿ-ರುವಾಗ
ಜಬ್ಬ-ಲ್ಪುರ
ಜಯ
ಜಯ-ಕಾರ
ಜಯ-ಕಾರ-ಗಳು
ಜಯ-ದೇವ
ಜಯ-ಪು-ರಕ್ಕೆ
ಜಯ-ಪು-ರದ
ಜಯ-ಪು-ರದ-ವ-ರೆಗೆ
ಜಯ-ಪು-ರದಿಂದ
ಜಯ-ಪುರ-ದ-ಲ್ಲಿ
ಜಯ-ಪ್ರದ-ವಾಗಿ
ಜಯ-ರಾಮ್
ಜಯ-ವನ್ನು
ಜಯ-ವಾಗಲಿ
ಜಯ-ವೊಂ-ದ-ನ್ನೇ
ಜಯ-ಶಂ-ಕರ
ಜಯ-ಶಾಲಿ-ಗ-ಳಾದರು
ಜಯ-ಶಾಲಿ-ಯಾಗುವಂತೆ
ಜಯ-ಶೀಲ-ನಾಗದೆ
ಜಯ-ಶೀಲ-ನಾಗಿ-ರು-ವೆನು
ಜಯ-ಶೀಲ-ನಾಗಿ-ರು-ವೆನೊ
ಜಯ-ಶೀಲ-ನಾಗು-ವನು
ಜಯ-ಶೀಲ-ನಾಗು-ವೆ-ನೆಂದು
ಜಯ-ಶೀಲ-ರಾಗಿಯೇ
ಜಯ-ಹೊಂದುವೆ
ಜಯಕ್ಕೆ
ಜಯತೇ
ಜಯಬೋಲೋ
ಜಯಾಪಜಯ-ಗಳ-ಲ್ಲಿ
ಜಯಿ-ಗಳು
ಜಯಿ-ಸಲಾರರು
ಜಯಿ-ಸಿ-ದರೆ
ಜಯಿ-ಸುವೆ
ಜಯಿಸ-ಬೇಕು
ಜರಿ-ದ-ವ-ರ-ಲ್ಲ
ಜರಿ-ಯು-ತ್ತಿದ್ದ
ಜರೂ-ರಾದ
ಜರೆ-ದರು
ಜರೆ-ಯು-ತ್ತಿದ್ದ
ಜರೆ-ಯುವ
ಜರೆದು
ಜರ್ಝರಿ-ತ-ವಾಗಿ-ತ್ತು
ಜರ್ಝರಿ-ತ-ವಾಗು-ವುದೋ
ಜರ್ಝರಿ-ತನಾ-ದಾಗಲೂ
ಜರ್ನ-ಲ್
ಜರ್ಮನಿಯ
ಜರ್ಮನ್
ಜಲ
ಜಲ-ಚರ
ಜಲ-ಜನ-ಕ-ಗಳ
ಜಲ-ಪಾತ-ವನ್ನು
ಜಲ-ವನ್ನು
ಜಲ-ಸಂಧಿ-ಯನ್ನು
ಜಲ-ಸಂಧಿಯ
ಜಲಕ್ಕೆ
ಜಲಾ-ಶಯ-ಗಳು
ಜವಾಬ್ದಾರ-ನ-ನ್ನಾಗಿ
ಜವಾಬ್ದಾರಿ
ಜವಾಬ್ದಾರಿ-ಯನ್ನು
ಜವಾಬ್ದಾರಿ-ಯನ್ನೆ-ಲ್ಲ
ಜವಾಬ್ದಾರಿ-ಯು-ನ್ನು
ಜವಾಬ್ದಾರಿ-ಯೆ-ಲ್ಲ
ಜಹ-ಜನ್ನು
ಜಹಜಿನ
ಜಹಜು
ಜಹಜು-ಗಳ
ಜಾ
ಜಾಗ
ಜಾಗ-ರಣೆ-ಗೆಂದು
ಜಾಗ-ರೂಕ-ನಾಗಿದ್ದು
ಜಾಗ-ವನ್ನು
ಜಾಗ-ವಿ-ರು-ತ್ತದೆಯೆ
ಜಾಗಟೆ
ಜಾಗವೂ
ಜಾಗೃ-ತಿಗೆ
ಜಾಗೃತ-ಗೊ-ಳ್ಳುವುದು
ಜಾಗೃತ-ಗೊಳಿಸಿ-ಕೊಂಡಿ-ರುವನು
ಜಾಗೃತ-ವಾಗು-ವುದೆಂದು
ಜಾಗೃತಗೊಳಿಸ-ಬಹುದು
ಜಾಗೃತಿ
ಜಾಗೃತಿ-ಗೊ-ಳ್ಳುವುದು
ಜಾಗ್ರ-ತಿಯ
ಜಾಗ್ರತ
ಜಾಗ್ರತ-ಗೊ-ಳ್ಳುವುದು
ಜಾಗ್ರತ-ಗೊಳಿ-ಸು-ವು-ದಕ್ಕೆ
ಜಾಗ್ರತ-ಗೊಳಿ-ಸುವುದು
ಜಾಗ್ರತ-ಗೊಳಿಸ-ಬ-ಲ್ಲ-ವ-ರಾಗಿ-ದ್ದರು
ಜಾಗ್ರತ-ಗೊಳಿಸ-ಬ-ಲ್ಲೆ-ನಾ-ದರೆ
ಜಾಗ್ರತ-ಗೊಳಿಸ-ಬಹುದು
ಜಾಗ್ರತ-ಗೊಳಿಸ-ಬೇಕು
ಜಾಗ್ರತ-ಗೊಳಿಸ-ಲೋಸುಗ
ಜಾಗ್ರತ-ಗೊಳಿಸಿ-ದರೆ
ಜಾಗ್ರತ-ನಾಗಿ
ಜಾಗ್ರತ-ನಾಗಿ-ರುವ
ಜಾಗ್ರತ-ನಾಗಿ-ರುವ-ನೆಂದು
ಜಾಗ್ರತ-ನಾಗು-ವ-ವ-ರೆಗೆ
ಜಾಗ್ರತ-ಮಾಡುವ
ಜಾಗ್ರತ-ರ-ನ್ನಾಗಿ
ಜಾಗ್ರತ-ರಾಗಿ
ಜಾಗ್ರತ-ರಾಗಿ-ರುವರು
ಜಾಗ್ರತ-ವಾ-ಗಿದೆ
ಜಾಗ್ರತ-ವಾ-ಗು-ತ್ತಿದೆ
ಜಾಗ್ರತ-ವಾಗ-ಬೇಕು
ಜಾಗ್ರತ-ವಾಗಿವೆ
ಜಾಗ್ರತ-ವಾಗು-ವಂತೆ
ಜಾಗ್ರತ-ವಾಗು-ವು-ದಕ್ಕೆ
ಜಾಗ್ರತ-ವಾದ
ಜಾಗ್ರತಾವ-ಸ್ಥೆ-ಯ-ಲ್ಲಿ
ಜಾಗ್ರತಾವ-ಸ್ಥೆಯೋ
ಜಾಗ್ರತಿ
ಜಾಗ್ರತಿ-ಯನ್ನು
ಜಾಗ್ರತೆ-ಗೊಳಿ-ಸುವುದು
ಜಾಜ್ವಾ-ಲ್ಯ-ಮಾನ-ವಾಗಿ
ಜಾಡಮಾ-ಲಿಯ
ಜಾಡಮಾಲಿ-ಗಳು
ಜಾಡಿ
ಜಾಡಿ-ಯ-ಲ್ಲಿ-ಟ್ಟು
ಜಾಡ್ಯ
ಜಾಣತ-ನ-ದ-ಲ್ಲಿ
ಜಾಣೆ-ಯಂತೆ
ಜಾತಿ
ಜಾತಿ-ಕುಲ-ಗ-ಳನ್ನು
ಜಾತಿ-ಕುಲ-ಗ-ಳಿಗೆ
ಜಾತಿ-ಗ-ಳನ್ನು
ಜಾತಿ-ಗಳ-ನ್ನೂ
ಜಾತಿ-ಗಿಂತ
ಜಾತಿ-ಬಂಧ-ನ-ದ-ಲ್ಲಿ
ಜಾತಿ-ಭಾ-ವನೆ
ಜಾತಿ-ಭೇದ-ವಿದೆ
ಜಾತಿ-ಮತ-ಗ-ಳನ್ನು
ಜಾತಿ-ಮತ-ಗಳ
ಜಾತಿ-ಯ-ನ್ನಾಗಿ
ಜಾತಿ-ಯ-ಲ್ಲಿ
ಜಾತಿ-ಯ-ವ-ರಿಗೂ
ಜಾತಿ-ಯ-ವನ-ಲ್ಲ-ವೆಂದೂ
ಜಾತಿ-ಯನ್ನು
ಜಾತಿ-ಯಿಂದ
ಜಾತಿ-ಯಿದೆ
ಜಾತಿ-ಯೊಂದಿಗೂ
ಜಾಫ್ನ
ಜಾಫ್ನಕ್ಕೆ
ಜಾಫ್ನದ
ಜಾರತೂಷ್ಟ್ರ
ಜಾರಿ
ಜಾರಿ-ಬಿದ್ದು
ಜಾರಿ-ಹೋ-ಯಿತು
ಜಾರಿ-ಹೋಗು-ವಾಗ
ಜಾರಿ-ಹೋದಾಗ
ಜಾರಿತು
ಜಾರುತುಷ್ಟ
ಜಾರ್ಜ್
ಜಾಲಾಡಿ
ಜಾಲ್
ಜಾವ
ಜಾವದ
ಜಾಹಿರಾ-ತಿನ
ಜಾಹೀರಾತು
ಜಿ
ಜಿಂ
ಜಿಂಕೆ
ಜಿಂಕೆ-ಮ-ರಿಗೆ
ಜಿತೇಂದ್ರಿಯ-ನಾಗಿ-ರ-ಬೇಕು
ಜಿನೀವ
ಜಿನೀವಾಕ್ಕೆ
ಜಿನುಗು-ವಂತೆ
ಜಿಬ್ರಾ-ಲ್ಟರ್
ಜಿವ-ನಕ್ಕೆ
ಜೀ
ಜೀನಾರ್
ಜೀರ್ಣ
ಜೀರ್ಣ-ವಾದ
ಜೀರ್ಣಿಸಿ-ಕೊ-ಳ್ಳು-ವು-ದಕ್ಕೆ
ಜೀರ್ಣಿಸಿ-ಕೊ-ಳ್ಳುವುದು
ಜೀರ್ಣಿಸಿ-ಕೊಂಡು
ಜೀಲಂ
ಜೀವ
ಜೀವ-ಗ-ಳನ್ನು
ಜೀವ-ಗಳ
ಜೀವ-ಗಳು
ಜೀವ-ಚ್ಛವ-ಗಳು
ಜೀವ-ಜಂತು-ಗ-ಳನ್ನು
ಜೀವ-ಜಂತು-ಗಳೆ-ಲ್ಲವೂ
ಜೀವ-ಜಗ-ತ್ತಿನ
ಜೀವ-ದಾನ
ಜೀವ-ನ-ಗತಿ
ಜೀವ-ನ-ಗತಿಯ
ಜೀವ-ನ-ಗಳ-ಲ್ಲಿ
ಜೀವ-ನ-ದ-ಲ್ಲಿ
ಜೀವ-ನ-ದ-ಲ್ಲಿದ್ದು
ಜೀವ-ನ-ದ-ಲ್ಲಿಯೂ
ಜೀವ-ನ-ದ-ಲ್ಲೇ
ಜೀವ-ನ-ದಲಿ
ಜೀವ-ನ-ದಿ-ಗ-ಳನ್ನು
ಜೀವ-ನ-ದಿಂದ
ಜೀವ-ನ-ದಿಂದಲೇ
ಜೀವ-ನ-ದೊಂದಿಗೆ
ಜೀವ-ನ-ಪ್ರದವೊ
ಜೀವ-ನ-ವನ್ನು
ಜೀವ-ನ-ವನ್ನೆ-ಲ್ಲ
ಜೀವ-ನ-ವನ್ನೆ-ಲ್ಲಾ
ಜೀವ-ನ-ವನ್ನೇ
ಜೀವ-ನ-ವಿ-ಲ್ಲ
ಜೀವ-ನ-ವೂಜೀವ-ನ-ವೆ-ಲ್ಲ
ಜೀವ-ನ-ವೆ-ಲ್ಲಾ
ಜೀವ-ನಕ್ಕೂ
ಜೀವ-ನಕ್ಕೆ
ಜೀವ-ನದ
ಜೀವ-ನದ್ದು
ಜೀವ-ನವೇ
ಜೀವ-ನಾಡಿ-ಯಾಗಿ-ರು-ತ್ತಾನೆ
ಜೀವ-ನಾದರ್ಶ
ಜೀವ-ನಾದರ್ಶವೋ
ಜೀವ-ನಿಗೂ
ಜೀವ-ನಿಗೆ
ಜೀವ-ನೋಪಾಯ-ಕ್ಕಾಗಿ
ಜೀವ-ನೋಪಾಯ-ವನ್ನು
ಜೀವ-ನೋಪಾಯಕ್ಕೆ
ಜೀವ-ನ್ಮುಕ್ತ-ನಾಗ-ಬಾ-ರದೇಕೆ
ಜೀವ-ನ್ಮುಕ್ತ-ರಾದ
ಜೀವ-ನ್ಮುಕ್ತಿ
ಜೀವ-ನ್ಮುಕ್ತಿ-ಯನ್ನು
ಜೀವ-ನ್ಮುಕ್ತಿಯ
ಜೀವ-ಪೋಷಕ-ವಾಗುವ
ಜೀವ-ಬ್ರಹ್ಮೈಕ್ಯ-ವನ್ನು
ಜೀವ-ಮಾ-ನ-ದ-ಲ್ಲಿ
ಜೀವ-ಮಾನ-ದ-ಲ್ಲೆ-ಲ್ಲ
ಜೀವ-ಮಾನ-ದ-ಲ್ಲೇ
ಜೀವ-ಮಾನ-ವ-ನ್ನೆ-ಲ್ಲ
ಜೀವ-ಮಾನ-ವನ್ನು
ಜೀವ-ರ-ನ್ನು
ಜೀವ-ರಿಗೆ
ಜೀವ-ರೆ-ಲ್ಲ
ಜೀವ-ವನ್ನು
ಜೀವ-ವಿ-ಲ್ಲ
ಜೀವ-ಸ-ಹಿತ
ಜೀವ-ಸೇವೆ
ಜೀವ-ಸೇವೆ-ಗಾಗಿ
ಜೀವಂ-ತ-ವಾ-ಯಿತು
ಜೀವಕ್ಕೆ
ಜೀವನ
ಜೀವನು
ಜೀವರ
ಜೀವರು
ಜೀವವೂ
ಜೀವಾಣು-ಗ-ಳನ್ನು
ಜೀವಾಳ
ಜೀವಾವಧಿ
ಜೀವಿ
ಜೀವಿ-ಗ-ಳನ್ನು
ಜೀವಿ-ಗ-ಳಾದರೋ
ಜೀವಿ-ಗ-ಳಿಗೆ
ಜೀವಿ-ಗಳ
ಜೀವಿ-ಗಳ-ನ್ನಾಗಿ
ಜೀವಿ-ಗಳ-ಲ್ಲೂ
ಜೀವಿ-ಗಳಿ-ಗೆ-ಲ್ಲ
ಜೀವಿ-ಗಳಿಗೂ
ಜೀವಿ-ಗಳು
ಜೀವಿ-ಗಳೆ-ಲ್ಲ
ಜೀವಿ-ಗಳೆಂದಿಗೂ
ಜೀವಿ-ತದ
ಜೀವಿ-ಯ-ಲ್ಲಿ
ಜೀವಿ-ಯ-ಲ್ಲಿಯೂ
ಜೀವಿ-ಯನ್ನು
ಜೀವಿ-ಸಿ-ಕೊಂಡಿ-ರುವರು
ಜೀವಿ-ಸಿದ್ದರೆ
ಜೀವಿ-ಸಿರು-ತ್ತಾರೆ
ಜೀವಿ-ಸು-ತ್ತ
ಜೀವಿ-ಸು-ತ್ತಾ
ಜೀವಿ-ಸು-ತ್ತಿದ್ದ
ಜೀವಿ-ಸು-ತ್ತಿದ್ದರು
ಜೀವಿ-ಸು-ತ್ತಿದ್ದರೂ
ಜೀವಿ-ಸು-ತ್ತಿದ್ದರೋ
ಜೀವಿ-ಸು-ತ್ತಿದ್ದಾಗಲೇ
ಜೀವಿ-ಸು-ತ್ತಿರು-ವರು
ಜೀವಿ-ಸು-ತ್ತಿರುವ
ಜೀವಿ-ಸು-ವೆಯೋ
ಜೀವಿಯ
ಜೀವಿಯು
ಜೀವಿಯೂ
ಜೀವಿಸು
ಜೀಸ-ಸ್
ಜೀಸ-ಸ್ಸನ
ಜೀಹೋ-ವನೋ
ಜು
ಜುಗುಪ್ಸಾ-ಕಾರಕ
ಜುಗುಪ್ಸೆ
ಜುಗುಪ್ಸೆ-ಯಿಂದ
ಜುನಗ-ಡ-ದ-ಲ್ಲಿ
ಜುನಗ-ಡದ
ಜುನಗಡ-ದಿಂದ
ಜುನಗಡಕ್ಕೆ
ಜುಲೈ
ಜುಲೈಗೆ
ಜೂ
ಜೆ
ಜೆಜೆ-ಗುಡ್ವಿನ್
ಜೆರ-ಮಾ-ಟ್
ಜೆರು-ಸಲೆಂ
ಜೆರೂ-ಸಲಂ
ಜೆಸುಇ-ಸ್ಟ್
ಜೇ
ಜೇಡರ
ಜೇಡಿ-ಮಣ್ಣಿ-ನಿಂದಲೇ
ಜೇಡಿ-ಮಣ್ಣಿನ
ಜೇನು
ಜೇನು-ತುಪ್ಪ
ಜೇಬಿ-ನ-ಲ್ಲಿ
ಜೇಬಿ-ನ-ಲ್ಲಿದ್ದ
ಜೇಬಿ-ನಿಂದ
ಜೇಬಿನೊಳ-ಗಿ-ಟ್ಟು
ಜೈ
ಜೈನ
ಜೈನ-ಧರ್ಮ
ಜೈನ-ಧರ್ಮದ
ಜೈಲಿ-ನ-ಲ್ಲಿ-ಟ್ಟರು
ಜೈಲಿ-ನಿಂದ
ಜೊ
ಜೊಂಪೆ-ಗಳು
ಜೊತೆ
ಜೊತೆ-ಗಾರ-ರ-ನ್ನೆ-ಲ್ಲ
ಜೊತೆ-ಗೂಡಿ-ದರು
ಜೊತೆ-ಯ-ಲ್ಲಿ
ಜೊತೆ-ಯ-ಲ್ಲಿ-ದ್ದ-ವ-ರೊಬ್ಬರು
ಜೊತೆ-ಯ-ಲ್ಲಿ-ದ್ದರೆ
ಜೊತೆ-ಯ-ಲ್ಲಿ-ದ್ದು-ದ-ರಿಂದ
ಜೊತೆ-ಯ-ಲ್ಲಿ-ರು-ವುದು
ಜೊತೆ-ಯ-ಲ್ಲಿ-ರುವ
ಜೊತೆ-ಯ-ಲ್ಲಿದ್ದ
ಜೊತೆ-ಯ-ಲ್ಲಿಯೇ
ಜೊತೆ-ಯ-ವ-ರಿಂದ
ಜೊಳ್ಳು
ಜೋ
ಜೋಕೆ
ಜೋಗೇಶ್
ಜೋಗೇಶ್ಗೆ
ಜೋಡಣೆ
ಜೋಡಣೆ-ಗ-ಳನ್ನು
ಜೋಡಣೆ-ಯ-ಲ್ಲಿ
ಜೋಡಣೆಯ
ಜೋಡಿ-ರ-ಲಿ-ಲ್ಲ
ಜೋಡಿ-ಸುವ
ಜೋಡಿಸಿ
ಜೋಡಿಸಿ-ದ್ದರು
ಜೋಡು
ಜೋತಾಡು-ತ್ತಿರು-ವುದು
ಜೋದ್ಪುರ
ಜೋದ್ಪುರ-ದ-ಲ್ಲಿ-ದ್ದಾಗ
ಜೋರಾ-ಯಿತು
ಜೋಲಾ-ಡುವ
ಜೋಲು-ತ್ತಿದ್ದ
ಜೋಶಿ
ಜೋಷಿ
ಜೋಸಿ-ಪೈ-ನ್
ಜ್ಝೀವ-ನದ
ಜ್ಞಾನ
ಜ್ಞಾನ-ಕಾಂ-ಡ-ಗಳಿವೆ
ಜ್ಞಾನ-ಕಾಂ-ಡ-ದ-ಲ್ಲಿ
ಜ್ಞಾನ-ದ-ಲ್ಲಿ
ಜ್ಞಾನ-ದಂತೆ
ಜ್ಞಾನ-ದಾನ
ಜ್ಞಾನ-ದಾಹ-ವನ್ನು
ಜ್ಞಾನ-ದಿಂದ
ಜ್ಞಾನ-ನಿಧಿ
ಜ್ಞಾನ-ಭಂಡಾರ
ಜ್ಞಾನ-ಭಂಡಾರ-ದಿಂದ
ಜ್ಞಾನ-ಭಾ-ಸ್ಕರನೇ
ಜ್ಞಾನ-ಮಾರ್ಗವೂ
ಜ್ಞಾನ-ಯೋಗ
ಜ್ಞಾನ-ಯೋಗ-ಗಳು
ಜ್ಞಾನ-ಯೋಗಕ್ಕೆ
ಜ್ಞಾನ-ಯೋಗದ
ಜ್ಞಾನ-ಲಾಭ-ವಾಗು-ವುದು
ಜ್ಞಾನ-ಲಾಭ-ವಾಗು-ವುದೆ
ಜ್ಞಾನ-ಲಾಭವೇ
ಜ್ಞಾನ-ವನ್ನು
ಜ್ಞಾನ-ವನ್ನೂ
ಜ್ಞಾನ-ವಾ-ದರೂ
ಜ್ಞಾನ-ವಾಗು-ತ್ತದೆ
ಜ್ಞಾನ-ವಿದೆ
ಜ್ಞಾನ-ವಿದೆಯೋ
ಜ್ಞಾನ-ವೈರಾಗ್ಯ-ಗಳು
ಜ್ಞಾನ-ಶಕ್ತಿಯ
ಜ್ಞಾನದ
ಜ್ಞಾನವೂ
ಜ್ಞಾನವೇ
ಜ್ಞಾನಾ-ಕಾಂ-ಕ್ಷಿ-ಯಾಗಿ-ರುವ
ಜ್ಞಾನಾ-ತೀತ-ವಾದ
ಜ್ಞಾನಾಗ್ನಿ
ಜ್ಞಾನಿ
ಜ್ಞಾನಿ-ಗಳೇ
ಜ್ಞಾನಿ-ಯಂತೆ
ಜ್ಞಾನಿ-ಯನ್ನು
ಜ್ಞಾನಿ-ಯೋರ್ವ-ನನ್ನು
ಜ್ಞಾನಿಗೆ
ಜ್ಞಾನೋದಯ-ವಾಗು-ವಂತೆ
ಜ್ಞಾಪ-ಕಾಂಡದ
ಜ್ಞಾಪ-ಕಾರ್ಥ-ವಾಗಿ
ಜ್ಞಾಪಕ
ಜ್ಞಾಪಕ-ದ-ಲ್ಲಿ
ಜ್ಞಾಪಕ-ದ-ಲ್ಲಿ-ಡ-ಬೇ-ಕಾ-ದುದೇ
ಜ್ಞಾಪಕ-ದ-ಲ್ಲಿಡು
ಜ್ಞಾಪಕ-ದ-ಲ್ಲಿದ್ದ
ಜ್ಞಾಪಕ-ವಿ-ತ್ತು
ಜ್ಞಾಪಕ-ವಿ-ದೆಯೆ
ಜ್ಞಾಪಕ-ವಿ-ಲ್ಲವೆ
ಜ್ಞಾಪಕ-ವಿದೆ
ಜ್ಞಾಪಕ-ಶಕ್ತಿ
ಜ್ಞಾಪಕಕ್ಕೆ
ಜ್ಞಾಪಕವೇ
ಜ್ಞಾಪಿಸಿ-ಕೊ-ಳ್ಳಲು
ಜ್ಞಾಪಿಸಿ-ಕೊ-ಳ್ಳು-ವು-ದಕ್ಕೆ
ಜ್ಯೂನಿ-ಪರ್
ಜ್ಯೂಬಿಲಿ
ಜ್ಯೋತಿ
ಜ್ಯೋತಿ-ಯನ್ನು
ಜ್ಯೋತಿ-ಷ್ಯರು
ಜ್ಯೋತಿಯ
ಜ್ಯೋತಿಯೇ
ಜ್ಯೋತಿ-ರ್ಮಯ
ಜ್ವರ
ಜ್ವರ-ದಿಂದ
ಜ್ವಲಂತ
ಜ್ವಲಂತ-ಗೊಳಿಸ-ಬೇಕು
ಜ್ವಲಿ-ಸುವ
ಜ್ವಾಜ್ವ-ಲ್ಯ-ಮಾನ-ವಾಗಿ
ಜ್ವಾಲಾ-ದ-ತ್
ಜ್ವಾಲಾ-ಮಯ-ವಾದ
ಜ್ವಾಲಾ-ಮುಖ-ದಿಂದ
ಜ್ವಾಲಾ-ಮುಖಿ
ಜ್ವಾಲಾ-ಮುಖಿ-ಯಂತೆ
ಜ್ವಾಲಾ-ಮುಖಿ-ಯನ್ನು
ಜ್ವಾಲಾ-ಮುಖಿ-ಯಿಂದ
ಜ್ವಾಲಾ-ಮುಖಿಯ
ಜ್ವಾಲೆ
ಜ್ವಾಲೆ-ಯಂತಿ-ರುವ
ಜ್ವಾಲೆಯ
ಝಣಝಣ
ಝರಿ
ಝರಿ-ಗಳ-ಲ್ಲಿ
ಝರಿ-ಗಳು
ಝರಿ-ಯಂತೆ
ಝರಿ-ಯೊಂದು
ಝಳ
ಝಾವ
ಟಬ್ಬು
ಟರ್ಕಿ-ಸ್ಥಾನದ
ಟವ-ಲ-ನ್ನು
ಟವಲಿ-ನಿಂದ
ಟಾಂಗಾ-ದ-ಲ್ಲಿ
ಟಿ
ಟಿಕೆ-ಟ್
ಟಿಕೆ-ಟ್ಟು
ಟಿಕೇಟ-ನ್ನು
ಟಿಕೇಟಿನ
ಟಿಕೇಟೆ-ಲ್ಲ
ಟಿಕ್
ಟಿಪ್ಪ-ಣಿಯೂ
ಟಿಪ್ಪಣಿ
ಟೀ
ಟೀಕಿ-ಸಲು
ಟೀಕಿ-ಸಿದ
ಟೀಕಿ-ಸು-ತ್ತಿ-ದ್ದದು
ಟೀಕಿ-ಸು-ತ್ತಿ-ದ್ದರು
ಟೀಕಿ-ಸು-ತ್ತಿ-ದ್ದಾಗ
ಟೀಕಿ-ಸು-ತ್ತಿ-ರ-ಲಿ-ಲ್ಲ
ಟೀಕಿ-ಸು-ತ್ತಿ-ರುವ
ಟೀಕಿ-ಸು-ತ್ತಿದ್ದ
ಟೀಕಿ-ಸು-ತ್ತೀರಿ
ಟೀಕಿ-ಸು-ತ್ತೇವೆ
ಟೀಕಿ-ಸು-ವು-ದಿ-ಲ್ಲ
ಟೀಕಿ-ಸುವ-ವರು
ಟೀಕಿ-ಸುವ-ವರೇ
ಟೀಕಿ-ಸುವಾಗ
ಟೀಕಿಸಿ
ಟೀಕೆ
ಟೀಕೆ-ಗ-ಳನ್ನು
ಟೀಕೆ-ಯನ್ನು
ಟೀಕೆ-ಯಿಂದ
ಟೀಕೆಗೂ
ಟೀಗೆ
ಟೆ
ಟೆನಿ-ಸ್
ಟೆರೈ
ಟೇಪ್ಹೌ-ಸಿನ
ಟೇಬ-ಲ್
ಟೈ
ಟೈಟ-ಸ್
ಟೈಫಾಯಿಡ್
ಟೋಕಿಯೋ
ಟೋಪಿ-ಯನ್ನು
ಟ್ರ-ಸ್ಟಿ-ಗ-ಳನ್ನು
ಟ್ರ-ಸ್ಟಿ-ಗಳು
ಟ್ರ-ಸ್ಟ್
ಟ್ರಂಕ್
ಟ್ರಪೀಜಿನ
ಟ್ರಾಂ
ಟ್ರಾಮ್
ಟ್ರಾವಂಕೂರ್
ಟ್ರಿಬ್ಯೂ-ನ್
ಟ್ರೈ-ನ್
ಟ್ರೋ-ಜನ್
ಠಕ್ಕಾಬಿಕ್ಕಿ
ಠರಾವು-ಗ-ಳನ್ನು
ಠಾಕೂ-ರರ
ಠಾಕೂರ್
ಠೀವಿ-ಯಿಂದ
ಡ
ಡಂಬ-ಲ್
ಡಂಭ-ವಿದೆ
ಡಬ್ಲ್ಯು
ಡಮ-ರು-ಗಳು
ಡಾ
ಡಾಕಾಕ್ಕೆ
ಡಾಯ-ಸನ್
ಡಾಯ್ಸನ್
ಡಾಯ್ಸನ್ಗೆ
ಡಾಯ್ಸನ್ರ
ಡಾರ್ಜಿ-ಲಿಂಗಿಗೆ
ಡಾರ್ಜಿ-ಲಿಂಗ್ನ-ಲ್ಲಿ
ಡಾರ್ವಿನ್
ಡಾರ್ವಿನ್ನನ
ಡಿ
ಡಿಕ್ಕಿ
ಡಿಪಾರ್ಟ್ಮೆಂಟಿನ-ವರು
ಡಿಪಾಸಿ-ಟ್
ಡಿಸಂಬರ್
ಡಿಸೆಂಬರ್
ಡೀಡಿನ
ಡೆ
ಡೆಕ್ಕಿನ
ಡೆಕ್ಕಿನ-ಮೇಲೆ
ಡೆಚರ್
ಡೆಹರಾಡೂ-ನಿಗೆ
ಡೆಹರಾಡೂ-ನ್ನ-ಲ್ಲಿ
ಡೆಹರಾಡೂನಿ-ನ-ಲ್ಲಿದ್ದು
ಡೆಹರಾಡೂನಿ-ನ-ಲ್ಲೇ
ಡೆಹರಾಡೂನಿ-ನಿಂದ
ಡೇ
ಡೇರೆ
ಡೇರೆ-ಗ-ಳನ್ನು
ಡೇರೆ-ಯಂತೂ
ಡೇರೆ-ಯನ್ನು
ಡೈಮಂಡ್
ಡೈರೆ-ಕ್ಟರ್
ಡೈಲಿ
ಡೊಂಕು
ಡೊಂಕು-ಡೊಂಕಾಗಿ
ಡೋ
ಡ್ರೆ
ಢಾ
ತ
ತಂ
ತಂಗ-ಬೇಕಾಗಿ
ತಂಗ-ಬೇಕೋ
ತಂಗ-ಲಿ-ಲ್ಲ
ತಂಗಲು
ತಂಗಿ
ತಂಗಿ-ದರು
ತಂಗಿ-ದ್ದನು
ತಂಗಿ-ದ್ದರು
ತಂಗಿ-ದ್ದಾಗ
ತಂಗಿ-ಯನ್ನು
ತಂಗಿದ
ತಂಗಿದ್ದ
ತಂಗಿದ್ದು
ತಂಗಿಯ
ತಂಗು-ತ್ತಿದ್ದ
ತಂಗು-ವರು
ತಂಗು-ವು-ದಕ್ಕೆ
ತಂಗುವ
ತಂಟೆ-ಯನ್ನು
ತಂಟೆಗೆ
ತಂಟೆಗೇ
ತಂಟೆಯ
ತಂಡ
ತಂಡ-ದ-ಲ್ಲಿ
ತಂಡ-ದ-ವರು
ತಂಡ-ವನ್ನು
ತಂಡ-ವಾಗಿ
ತಂಡವೇ
ತಂಡವೋ
ತಂಡೋಪ-ತಂಡ-ವಾಗಿ
ತಂತಿ
ತಂತಿ-ಗಳು
ತಂತಿ-ಯನ್ನು
ತಂತು-ವಿ-ನಿಂದ
ತಂತು-ವಿನ
ತಂದ
ತಂದ-ನೆ-ನ್ನು-ವುದು
ತಂದ-ಮೇಲೆ
ತಂದಿ-ದ್ದೇನೆ
ತಂದಿ-ರು-ವಂತೆ
ತಂದಿ-ರು-ವೆನು
ತಂದಿತು
ತಂದು
ತಂದು-ಕೊ-ಟ್ಟಿತು
ತಂದು-ಕೊ-ಟ್ಟು
ತಂದು-ಕೊ-ಳ್ಳಲು
ತಂದು-ಕೊಡ-ಬೇಕು
ತಂದು-ಕೊಡು-ತ್ತಿದ್ದ
ತಂದು-ಕೊಡುವೆ
ತಂದು-ಕೊಳ್ಳಿ
ತಂದು-ದ-ನ್ನು
ತಂದೆ
ತಂದೆ-ತಾಯಿ-ಗ-ಳಿಗೆ
ತಂದೆ-ತಾಯಿ-ಗಳು
ತಂದೆ-ಯ-ವರು
ತಂದೆ-ಯನ್ನು
ತಂದೆ-ಯರು
ತಂದೆ-ಯಾಗಿ
ತಂದೆ-ಯಾದ
ತಂದೆ-ಯೊ-ಡನೆ
ತಂದೆಗೂ
ತಂದೆಗೆ
ತಂದೆಯ
ತಂದೇ
ತಂಪಾಗಿ
ತಂಪಾದ
ತಂಬಾ-ಕನ್ನು
ತಂಬೂ-ರಿಯ
ತಕ್ಕ
ತಕ್ಕಂತೆ
ತಕ್ಕಷ್ಟು
ತಗಲಿ
ತಗಲು
ತಗಲು-ವುದೋ
ತಗಲುವ
ತಗುಲುವ
ತಗ್ಗ-ದಂತೆ
ತಗ್ಗಿ
ತಗ್ಗಿ-ಸಿ-ಕೊಂಡು
ತಗ್ಗಿ-ಸು-ವು-ದ-ಕ್ಕಾಗಿಯೇ
ತಗ್ಗಿ-ಸೆಂದು
ತಗ್ಗು
ತಗ್ಗು-ತ್ತಿದೆ
ತಗ್ಗು-ವ-ವನೇ
ತಟ
ತಡ
ತಡ-ವಾ-ಯಿತು
ತಡ-ವಾಗು-ವುದು
ತಡೆ-ದರು
ತಡೆ-ಯಲಾ-ರದೆ
ತಡೆ-ಯಲು
ತಡೆ-ಯಿ-ಲ್ಲದೆ
ತಡೆ-ಯಿರಿ
ತಡೆ-ಯು-ವು-ದಕ್ಕೆ
ತಡೆ-ಯುವ-ವರಾರು
ತಡೆಯ-ಬ-ಲ್ಲ-ವ-ರಿ-ಲ್ಲ
ತಡೆಯ-ಬೇ-ಕಾ-ದರೆ
ತಡೆಯಲಾ-ರರು
ತಡೆಯು-ವರಾರು
ತಣಿಸಿ-ಕೊ-ಳ್ಳು-ವು-ದಕ್ಕೆ
ತಣ್ಣ-ಗಾ-ಯಿತು
ತಣ್ಣಗಿ-ರುವ
ತನ-ಗಾಗಿ
ತನ-ಗಾದ
ತನ-ಗಿಂತ
ತನಕ
ತನಗೂ
ತನಗೆ
ತನ್ನ
ತನ್ನ-ದ-ನ್ನು
ತನ್ನ-ದ-ನ್ನೆ-ಲ್ಲ
ತನ್ನ-ದೆಂದು
ತನ್ನ-ದೆಂಬುದೇನೂ
ತನ್ನ-ಲ್ಲಿ
ತನ್ನ-ಲ್ಲಿ-ರು-ವು-ದ-ನ್ನು
ತನ್ನ-ಲ್ಲಿ-ರುವ
ತನ್ನ-ಷ್ಟಕ್ಕೆ
ತನ್ನಂತಹ
ತನ್ನದು
ತನ್ನದೇ
ತನ್ನನ್ನ-ನು-ಸರಿ-ಸು-ವಂತೆ
ತನ್ನನ್ನು
ತನ್ನನ್ನೇ
ತನ್ನಾ-ತ್ಮ
ತನ್ನಿ
ತನ್ನಿಂದಲೇ
ತನ್ನೆ-ಡೆಗೆ
ತನ್ನೊ-ಡನೆ
ತನ್ಮ-ಯರಾಗ-ಬೇಕೆಂದೂ
ತನ್ಮ-ಯರಾಗು-ವು-ದ-ನ್ನು
ತನ್ಮ-ಯರಾಗು-ವು-ದಾಗಿಯೂ
ತನ್ಮಯ-ನಾಗ-ಬಹುದು
ತನ್ಮಯ-ನಾಗಿ
ತನ್ಮಯ-ನಾಗಿ-ರು-ವನು
ತನ್ಮಯ-ರಾಗಿ
ತನ್ಮಯ-ರಾಗಿ-ದ್ದರು
ತನ್ಮಯ-ರಾಗಿ-ದ್ದಾರೆ
ತನ್ಮಯ-ರಾಗಿ-ದ್ದು-ದ-ರಿಂದ
ತನ್ಮಯ-ರಾಗಿ-ಹೋದರು
ತನ್ಮಯ-ರಾದರು
ತನ್ಮಯ-ರಾದರೆ
ತನ್ಮಯತೆ
ತಪ-ಸ್
ತಪ-ಸ್ವಿ-ಗ-ಳಾದ-ವರ
ತಪ-ಸ್ಸ-ಲ್ಲ-ವೆಂಬುದು
ತಪ-ಸ್ಸನ್ನು
ತಪ-ಸ್ಸಾಗು-ತ್ತದೆ
ತಪ-ಸ್ಸಿ-ಗಾಗಿ
ತಪ-ಸ್ಸಿ-ನ-ಲ್ಲಿ
ತಪ-ಸ್ಸಿ-ನಿಂದ
ತಪ-ಸ್ಸಿ-ಲ್ಲದೆ
ತಪ-ಸ್ಸಿಗೆ
ತಪ-ಸ್ಸಿನ
ತಪ-ಸ್ಸು
ತಪಃಫಲ-ವನ್ನೆ-ಲ್ಲ
ತಪಿ-ಸುವ
ತಪೋ-ಶಕ್ತಿ-ಯ-ನ್ನೆ-ಲ್ಲ
ತಪೋ-ಶಕ್ತಿ-ಯನ್ನು
ತಪೋ-ಶಕ್ತಿಯ
ತಪ್ಪ-ನ್ನು
ತಪ್ಪ-ಲ್ಲ
ತಪ್ಪ-ಲ್ಲ-ವೆಂದು
ತಪ್ಪದು
ತಪ್ಪದೆ
ತಪ್ಪದೇ
ತಪ್ಪಲ
ತಪ್ಪಲ-ಲ್ಲಿರುವ
ತಪ್ಪಲ-ಲ್ಲೊ
ತಪ್ಪಲಿ-ನ-ಲ್ಲಿದೆ
ತಪ್ಪಲೆ
ತಪ್ಪಾಗಿ-ದ್ದರೂ
ತಪ್ಪಿ
ತಪ್ಪಿ-ತ-ಸ್ಥ-ರ-ನ್ನು
ತಪ್ಪಿ-ತೆಂ-ದರೆ
ತಪ್ಪಿ-ದಂ-ತಾಗಿ
ತಪ್ಪಿ-ದಂತೆ
ತಪ್ಪಿ-ದಾಗ
ತಪ್ಪಿ-ರು-ವನೋ
ತಪ್ಪಿ-ಸಿ-ಕೊ-ಳ್ಳಲು
ತಪ್ಪಿ-ಸಿ-ಕೊ-ಳ್ಳು-ವು-ದಕ್ಕೆ
ತಪ್ಪಿ-ಸಿ-ಕೊ-ಳ್ಳುವರು
ತಪ್ಪಿ-ಸಿ-ಕೊಂಡರೆ
ತಪ್ಪಿ-ಸಿ-ಕೊಂಡು
ತಪ್ಪಿ-ಸಿ-ಕೊಂಡು-ಹೋಗಲು
ತಪ್ಪಿ-ಸಿ-ಕೊಂಡೆ-ವೆಂದೂ
ತಪ್ಪಿ-ಸಿ-ಕೊಡರು
ತಪ್ಪಿ-ಸಿದ
ತಪ್ಪಿ-ಸು-ವು-ದಾದರೆ
ತಪ್ಪಿ-ಹೋ-ಯಿತು
ತಪ್ಪಿ-ಹೋಗಿ-ತ್ತು
ತಪ್ಪಿ-ಹೋಗು-ವು-ದೆಂದು
ತಪ್ಪಿತು
ತಪ್ಪಿಸಿ
ತಪ್ಪು
ತಪ್ಪು-ಗ-ಳನ್ನು
ತಪ್ಪು-ವ-ವ-ರ-ಲ್ಲ
ತಪ್ಪು-ವರೊ
ತಪ್ಪೆ
ತಪ್ಪೆ-ಲ್ಲ-ವನ್ನೂ
ತಪ್ಪೊ
ತಬ್ಬಿ-ಕೊಂಡು
ತಬ್ಬಿ-ಕೊಳ್ಳ-ಬ-ಲ್ಲರೊ
ತಮ-ಗಾಗಿ
ತಮ-ಗಾಗಿ-ಯಾ-ದರೂ
ತಮ-ಗಾಗಿ-ರುವ
ತಮ-ಗಾದ
ತಮ-ಗಿಂತ
ತಮ-ಗಿಷ್ಟ
ತಮ-ಗೆ-ಲ್ಲ
ತಮ-ಟೆ-ಗ-ಳನ್ನು
ತಮ-ಸ್ಸಿ-ನ-ಲ್ಲಿ
ತಮ-ಸ್ಸಿ-ನಿಂದ
ತಮ-ಸ್ಸು
ತಮಗೂ
ತಮಗೆ
ತಮಗೇ
ತಮಟೆ
ತಮಾಮಾಷೆ-ಯಿಂದ
ತಮಾಷೆ
ತಮಾಷೆ-ಗಾಗಿ
ತಮಾಷೆ-ಯಂತೆ
ತಮಾಷೆ-ಯಾಗಿ
ತಮಾಷೆ-ಯಾದ
ತಮಾಷೆ-ಯಿಂದ
ತಮಾಷೆಯ
ತಮಿಳು
ತಮ್ಮ
ತಮ್ಮ-ನನ್ನು
ತಮ್ಮ-ನೊ-ಡನೆ
ತಮ್ಮ-ಲ್ಲಿ
ತಮ್ಮ-ಲ್ಲಿ-ರುವ
ತಮ್ಮ-ಲ್ಲಿದೆ
ತಮ್ಮ-ಲ್ಲಿದ್ದ
ತಮ್ಮ-ಲ್ಲಿಯೇ
ತಮ್ಮ-ಲ್ಲೆ
ತಮ್ಮ-ಷ್ಟಕ್ಕೆ
ತಮ್ಮಂ-ತೆಯೇ
ತಮ್ಮಂತಹ
ತಮ್ಮಂಥ-ವರು
ತಮ್ಮಂದಿ-ರಾದ
ತಮ್ಮಂದಿರು
ತಮ್ಮದು
ತಮ್ಮದೇ
ತಮ್ಮನ
ತಮ್ಮನ್ನು
ತಮ್ಮನ್ನೂ
ತಮ್ಮನ್ನೇ
ತಮ್ಮಿಬ್ಬರ
ತಮ್ಮೆದುರು
ತಮ್ಮೊ-ಡನೆ
ತಮ್ಮೊಂದಿಗೆ
ತಮ್ಮೊಡ-ನಿದ್ದ
ತಮ್ಮೊಳಗೆ
ತಮ್ಮೊಳಗೇ
ತರ-ಕಾರಿ
ತರ-ಕೂ-ಡದು
ತರ-ಗ-ತಿಗೆ
ತರ-ಗತಿ-ಗಳ-ನ್ನೇರ್ಪಡಿ-ಸು-ವುದು
ತರ-ಗತಿ-ಯ-ಲ್ಲಿದ್ದ
ತರ-ಗತಿ-ಯ-ಲ್ಲಿರು-ವರು
ತರ-ಗತಿ-ಯನ್ನು
ತರ-ಗತಿಯ
ತರ-ತ-ಮ-ಗಳು
ತರ-ತಮ-ದ-ಲ್ಲಿದೆ
ತರ-ತಮಕ್ಕೆ
ತರ-ಬಹುದು
ತರ-ಬೆ-ಕಾ-ದರೆ
ತರ-ಬೇ-ಕಾ-ದರೆ
ತರ-ಬೇ-ತನ್ನು
ತರ-ಬೇ-ತಾಗದೆ
ತರ-ಬೇಕಾಗಿ-ತ್ತು
ತರ-ಬೇಕು
ತರ-ಬೇಕೆಂದು
ತರ-ಬೇತು
ತರ-ವ-ಲ್ಲ
ತರ-ಹದ
ತರಂಗ-ಗಳಾಗಿ
ತರಂಗ-ಗಳು
ತರಂಗಮಾಲೆ-ಗಳು
ತರಂಗವೂ
ತರಂಗೋ-ತ್ತ
ತರಲು
ತರಹ
ತರೀಖು
ತರು-ಣ-ರಿ-ದ್ದರೆ
ತರು-ಣನು
ತರು-ಣರು
ತರು-ಣರೇ
ತರು-ತ್ತಿದ್ದನು
ತರು-ತ್ತೀರೊ
ತರು-ಲತೆ-ಗಳು
ತರು-ವಂತಹ
ತರು-ವಾಯ
ತರುಣ
ತರುವ
ತರ್ಕ
ತರ್ಕ-ಚೂಡಾ-ಮಣಿ
ತರ್ಕ-ದಿಂದ
ತರ್ಕ-ಬದ್ಧ-ವ-ಲ್ಲ
ತರ್ಕ-ಬದ್ಧ-ವಾಗಿ-ತ್ತು
ತರ್ಕ-ಬದ್ಧ-ವಾದ
ತರ್ಕದ
ತರ್ಕವೂ
ತರ್ಕಾವ-ಸ್ಥೆ-ಯ-ಲ್ಲೇ
ತರ್ಕಿ-ಸಲು
ತರ್ಕಿ-ಸಿ-ದರು
ತರ್ಕಿ-ಸು-ತ್ತಿ-ದ್ದನು
ತರ್ಕಿ-ಸು-ವು-ದಕ್ಕೆ
ತರ್ಕಿಸ-ತೊಡಗಿದರು
ತರ್ಪಣ-ಗ-ಳನ್ನು
ತಲಪು-ತ್ತದೆ
ತಲುಪಿ-ದರು
ತಲುಪಿ-ಸಿದ
ತಲುಪಿ-ಸು-ವಂತೆ
ತಲುಪಿತು
ತಲುಪಿದ
ತಲುಪಿದೆ
ತಲೆ
ತಲೆ-ಕೆಳ-ಗಾಗಿ
ತಲೆ-ಕೆಳಕಾಗಿ
ತಲೆ-ಕೆಳಗೆ
ತಲೆ-ಕೊಡು-ವ-ವನೇ
ತಲೆ-ಗಳಿಂದ
ತಲೆ-ಗಳು
ತಲೆ-ಜು-ಟ್ಟ-ನ್ನು
ತಲೆ-ತಲಾಂ-ತ-ರದ-ವ-ರೆಗೂ
ತಲೆ-ತಿ-ರುಗಿ
ತಲೆ-ದೂಗು-ವಂತಹ
ತಲೆ-ದೂಗು-ವಂತೆ
ತಲೆ-ದೋರಿತು
ತಲೆ-ದೋರಿದೆ
ತಲೆ-ನೋವು
ತಲೆ-ಮಾ-ರಿನ
ತಲೆ-ಮಾರಿ-ನ-ವ-ರೆ-ಲ್ಲ
ತಲೆ-ಮಾರು-ಗಳಿಂದ
ತಲೆ-ಯ-ಮೇಲೆ
ತಲೆ-ಯ-ಲ್ಲಿ
ತಲೆ-ಯ-ಲ್ಲಿ-ರುವ
ತಲೆ-ಯನ್ನು
ತಲೆ-ಯಿ-ಲ್ಲದ
ತಲೆ-ಯಿಂದ
ತಲೆ-ಯೆ-ತ್ತಿ-ಕೊಂಡಿವೆ
ತಲೆ-ಸು-ತ್ತಿ
ತಲೆಗೆ
ತಲೆಯ
ತಳ-ದಿಂದ
ತಳತಳಿ-ಸದೆ
ತಳಪಾ-ಯವೆ
ತಳಪಾಯ
ತಳಪಾಯ-ವನ್ನು
ತಳಮಳ
ತಳಮಳ-ಗೊ-ಳ್ಳು-ತ್ತಿದ್ದರು
ತಳಮಳ-ಗೊಳ್ಳು-ತ್ತಿರುವೆ
ತಳಹದಿ
ತಳಹದಿ-ಯನ್ನು
ತಳಹದಿ-ಯಾಗ-ಬ-ಲ್ಲದು
ತಳಹದಿ-ಯಾಗಿ
ತಳಹದಿ-ಯಾಗಿವೆ
ತಳಹದಿಯ
ತಳಹದಿಯೇ
ತಳಿರು
ತಳಿರು-ಗಳಿಂದ
ತಳಿರು-ತೋರಣ-ಗಳಿಂದ
ತಳಿರು-ತೋರಣ-ಗಳು
ತಳ್ಳ-ಬಹುದು
ತಳ್ಳಿ
ತಳ್ಳಿ-ರು-ವುವು
ತವಕ
ತವಕ-ಪಡು-ತ್ತಿರು-ವುದು
ತವಕ-ಪಡು-ತ್ತಿರು-ವೆನು
ತಹ-ಶೀಲ್ದಾ-ರರು
ತಹ-ಶೀಲ್ದಾರ್
ತಹ್ಲ
ತಾ
ತಾಂ
ತಾಂಡವ-ಗಳೊಂದಿಗೆ
ತಾಂಡವ-ನೃ-ತ್ಯ
ತಾಂಡವ-ವಾ-ಡು-ತ್ತಿವೆ
ತಾಂಡವವಾಡು-ವಳು
ತಾಂಬೂಲಸೇ-ವನೆ
ತಾಕಿ
ತಾಕಿ-ದಂತೆ
ತಾಕಿ-ದೆ-ದೆ-ಯನ್ನು
ತಾಕಿ-ದ್ದರೂ
ತಾಕಿ-ಲ್ಲ-ವೆಂದು
ತಾಕಿ-ಸಿ-ದರು
ತಾಕಿತು
ತಾಕಿದೆ
ತಾಕು-ತ್ತಿ-ತ್ತು
ತಾಕು-ವಂತಿರ-ಬೇಕು
ತಾಕು-ವುದು
ತಾಗಿ-ದವು
ತಾಜ್
ತಾಜ್ಮಹ-ಲ್
ತಾಜ್ಮಹ-ಲ್ಲ-ನ್ನು
ತಾಜ್ಮಹ-ಲ್ಲಿನ
ತಾಡಿತ-ವಾದ
ತಾತ
ತಾನಾರು
ತಾನು
ತಾನೂ
ತಾನೆ
ತಾನೇ
ತಾಪ-ಗಳು
ತಾಮಸ
ತಾಮಸ-ದ-ಲ್ಲಿ
ತಾಯಂ-ದಿರಾ-ಗ-ಬೇಕು
ತಾಯಂದಿ-ರೊಂದಿಗೆ
ತಾಯಿ
ತಾಯಿ-ಗ-ಳಾದರೋ
ತಾಯಿ-ಗ-ಳಿಗೆ
ತಾಯಿ-ಗಳು
ತಾಯಿ-ತಂದೆ-ಗಳು
ತಾಯಿ-ಯ-ಲ್ಲದೆ
ತಾಯಿ-ಯ-ವ-ರ-ನ್ನು
ತಾಯಿ-ಯಂತೆ
ತಾಯಿ-ಯನ್ನು
ತಾಯಿ-ಯಿಂದ
ತಾಯಿ-ಯಿಂದಾದ
ತಾಯಿ-ಯೊ-ಡನೆ
ತಾಯಿ-ಯೊಬ್ಬಳೆ
ತಾಯಿಗೆ
ತಾಯಿಯ
ತಾಯಿಯೂ
ತಾಯಿಯೆ
ತಾಯಿಯೇ
ತಾಯೊಲುಮೆ
ತಾಯ್ನಾಡಾದ
ತಾಯ್ನಾಡಿಗೆ
ತಾಯ್ನಾಡಿನ
ತಾರ-ತಮ್ಯ-ಭಾವ-ವನ್ನು
ತಾರಕ
ತಾರಕ-ನನ್ನು
ತಾರಕ-ಶಿವಾ-ನಂದ
ತಾರಾ
ತಾರಾ-ಕಾಂ-ತಿಯ
ತಾರಾ-ಖಚಿತ
ತಾರಾ-ಖಚಿತ-ನಭ
ತಾರಾ-ಡು-ತ್ತಿದ್ದನು
ತಾರಾ-ಮಂಡಲ-ಗ-ಳನ್ನು
ತಾರಾ-ವಳಿ-ಗಳು
ತಾರೀಖಿ-ನಿಂದ
ತಾರೀಖಿನ
ತಾರೀಖಿನ-ವ-ರೆಗೆ
ತಾರೀಖು
ತಾರೆ
ತಾರೆ-ಗಳಂತೆ
ತಾರೆ-ಯನ್ನು
ತಾರ್ಕಿ-ಕರು
ತಾರ್ಕಿಕ-ವಾಗಿ-ಲ್ಲ
ತಾಳ-ಬೇಕು
ತಾಳ-ಮೇಳ-ಗಳು
ತಾಳಕ್ಕೆ
ತಾಳಿ
ತಾಳಿ-ದಂತೆ
ತಾಳಿ-ರು-ವುದು
ತಾಳಿತು
ತಾಳು-ವರು
ತಾಳು-ವುದು
ತಾಳೆ
ತಾಳೇ
ತಾಳ್ಮೆ-ಯ-ಲ್ಲಿ
ತಾಳ್ಮೆ-ಯಾಗಲಿ
ತಾಳ್ಮೆ-ಯಿಂದ
ತಾಳ್ಮೆ-ಯಿಂದಿರಲು
ತಾವು
ತಾವು-ಗಳೆ-ಲ್ಲಾ
ತಾವೂ
ತಾವೆ
ತಾವೆ-ಲ್ಲರೂ
ತಾವೆಂದೆಂದಿಗೂ
ತಾವೇ
ತಾವೇ-ನಾ-ದರೂ
ತಾವೊಬ್ಬರೆ
ತಾವೊಬ್ಬರೇ
ತಿ
ತಿಂ
ತಿಂಡಿ
ತಿಂಡಿ-ಕೊ-ಟ್ಟು
ತಿಂಡಿ-ಗ-ಳನ್ನು
ತಿಂಡಿ-ಗಳ-ನ್ನೆ-ಲ್ಲ
ತಿಂಡಿ-ಗಾ-ದರೂ
ತಿಂಡಿ-ಗಾಗಿ
ತಿಂಡಿ-ತೀರ್ಥ-ಗ-ಳನ್ನು
ತಿಂಡಿ-ತೀರ್ಥ-ಗಳು
ತಿಂಡಿ-ಯನ್ನು
ತಿಂಡಿ-ಯಾದ
ತಿಂದ
ತಿಂದಿರ-ಬೇಕು
ತಿಂದು
ತಿಂದು-ಬಿಡು-ತ್ತಿದ್ದು-ದಾಗಿ
ತಿಂದು-ಹಾಕಿ
ತಿಂದೇ
ತಿಕ್ಕಿ
ತಿಥಿ
ತಿದ್ದಿ-ಕೊ-ಳ್ಳುವುದು
ತಿದ್ದಿ-ಕೊಂಡರು
ತಿದ್ದಿ-ಕೊಳ್ಳಲಾ-ರರು
ತಿದ್ದಿ-ದರು
ತಿದ್ದು-ತ್ತಾರೆ
ತಿದ್ದು-ತ್ತಿದ್ದನು
ತಿದ್ದು-ತ್ತಿದ್ದರು
ತಿದ್ದು-ವು-ದ-ಕ್ಕಾಗಿ
ತಿನ್ನ-ಬಹುದು
ತಿನ್ನ-ಬಹುದೆ
ತಿನ್ನ-ಬೇಕು
ತಿನ್ನದೆ
ತಿನ್ನಲಾರೆ
ತಿನ್ನಲಿ
ತಿನ್ನಲಿಕ್ಕೆ
ತಿನ್ನಲು
ತಿನ್ನಿಸ-ತೊಡಗಿದರು
ತಿನ್ನಿಸಿ
ತಿನ್ನು-ತ್ತಾನೆ
ತಿನ್ನು-ತ್ತಾರೆ
ತಿನ್ನು-ತ್ತಿ-ತ್ತು
ತಿನ್ನು-ತ್ತಿ-ಲ್ಲವೆ
ತಿನ್ನು-ತ್ತಿದ್ದ
ತಿನ್ನು-ತ್ತಿದ್ದ-ರೆಂದೂ
ತಿನ್ನು-ತ್ತಿದ್ದರು
ತಿನ್ನು-ತ್ತೀರ-ಲ್ಲವೆ
ತಿನ್ನು-ವಂತೆ
ತಿನ್ನು-ವು-ದ-ರಿಂದ
ತಿನ್ನು-ವು-ದಕ್ಕೆ
ತಿನ್ನು-ವು-ದಿ-ಲ್ಲ
ತಿನ್ನುವ
ತಿನ್ನುವರು
ತಿನ್ನುವು-ದ-ರ-ಲ್ಲೇ-ಅ-ವರ
ತಿನ್ನುವುದ-ರ-ಲ್ಲಿ-ತ್ತು
ತಿನ್ನುವೆ
ತಿನ್ಪಾನಿ
ತಿರ-ಸ್ಕ-ರಿಸಿ
ತಿರ-ಸ್ಕ-ರಿಸಿ-ದರೋ
ತಿರ-ಸ್ಕ-ರಿಸಿದೆ
ತಿರ-ಸ್ಕರಿ-ಸಿತು
ತಿರ-ಸ್ಕರಿಸು-ವು-ದ-ಕ್ಕ-ಲ್ಲ
ತಿರು-ಳನ್ನು
ತಿರು-ಳಾದ
ತಿರುಗ-ಬೇಕೆಂದು
ತಿರುಗ-ಲಿ-ಲ್ಲ
ತಿರುಗಾಡಲು
ತಿರುಗು-ತ್ತದೆ
ತಿರುಗು-ತ್ತಿವೆ
ತಿರುಗು-ವಂತೆ
ತಿರುಗು-ವು-ದಕ್ಕೆ
ತಿರುಗುವ
ತಿರುಪ-ತ್ತೂರಿ-ನಿಂದ
ತಿರುಪ್ಪುಲಾನಿ
ತಿರುಳ-ಲ್ಲ
ತಿರ್ಮಾ-ನಕ್ಕೆ
ತಿಲ-ಕ-ರಿಗೆ
ತಿಲ-ಕರು
ತಿಲ-ತರ್ಪಣ
ತಿಲ-ತರ್ಪಣ-ವನ್ನು
ತಿಳಿ
ತಿಳಿ-ಗೇಡಿ-ತನ
ತಿಳಿ-ದ-ಕೊಂಡ-ರೇನೆ
ತಿಳಿ-ದ-ಕೊಂಡು
ತಿಳಿ-ದ-ವರ
ತಿಳಿ-ದಂತೆ
ತಿಳಿ-ದರೆ
ತಿಳಿ-ದಿ-ರ-ಲಿ-ಲ್ಲ
ತಿಳಿ-ದಿ-ರದ
ತಿಳಿ-ದಿ-ರು-ವಿರಾ
ತಿಳಿ-ದಿ-ರು-ವು-ದ-ನ್ನು
ತಿಳಿ-ದಿ-ರು-ವುದು
ತಿಳಿ-ದಿ-ರುವ
ತಿಳಿ-ದಿ-ರುವರು
ತಿಳಿ-ದಿ-ರುವೆಯಾ
ತಿಳಿ-ದಿ-ಲ್ಲ
ತಿಳಿ-ದಿದ್ದನು
ತಿಳಿ-ದಿದ್ದರು
ತಿಳಿ-ದಿದ್ದರೆ
ತಿಳಿ-ದಿದ್ದೆವೊ
ತಿಳಿ-ದಿರು-ವಿ-ರೇನು
ತಿಳಿ-ದು-ಕೊ-ಳ್ಳ-ದ-ವರು
ತಿಳಿ-ದು-ಕೊ-ಳ್ಳ-ಬ-ಲ್ಲ
ತಿಳಿ-ದು-ಕೊ-ಳ್ಳ-ಬ-ಲ್ಲ-ದ್ದಾಗಿ-ತ್ತು
ತಿಳಿ-ದು-ಕೊ-ಳ್ಳ-ಬ-ಲ್ಲರು
ತಿಳಿ-ದು-ಕೊ-ಳ್ಳ-ಬ-ಲ್ಲೆಯಾ
ತಿಳಿ-ದು-ಕೊ-ಳ್ಳ-ಬೇ-ಕಾ-ದರೆ
ತಿಳಿ-ದು-ಕೊ-ಳ್ಳ-ಬೇ-ಕಾದ
ತಿಳಿ-ದು-ಕೊ-ಳ್ಳ-ಬೇಕಾ-ಯಿತು
ತಿಳಿ-ದು-ಕೊ-ಳ್ಳ-ಬೇಕಾಗಿದೆ
ತಿಳಿ-ದು-ಕೊ-ಳ್ಳ-ಬೇಕೆಂದು
ತಿಳಿ-ದು-ಕೊ-ಳ್ಳ-ಬೇಕೆಂದೂ
ತಿಳಿ-ದು-ಕೊ-ಳ್ಳ-ಬೇಕೆಂಬ
ತಿಳಿ-ದು-ಕೊ-ಳ್ಳ-ಬೇಡ
ತಿಳಿ-ದು-ಕೊ-ಳ್ಳಲಾ-ರದೆ
ತಿಳಿ-ದು-ಕೊ-ಳ್ಳಲಾ-ರದೆ-ಹೋದೆ
ತಿಳಿ-ದು-ಕೊ-ಳ್ಳಲಾ-ರನು
ತಿಳಿ-ದು-ಕೊ-ಳ್ಳಲಾಗ-ಲಿ-ಲ್ಲ
ತಿಳಿ-ದು-ಕೊ-ಳ್ಳಲಾರ-ನೆಂಬು-ದ-ನ್ನು
ತಿಳಿ-ದು-ಕೊ-ಳ್ಳಲಾರೆ
ತಿಳಿ-ದು-ಕೊ-ಳ್ಳಲು
ತಿಳಿ-ದು-ಕೊ-ಳ್ಳು-ತ್ತಿ-ತ್ತು
ತಿಳಿ-ದು-ಕೊ-ಳ್ಳು-ವು-ದ-ನ್ನು
ತಿಳಿ-ದು-ಕೊ-ಳ್ಳು-ವು-ದಕ್ಕೆ
ತಿಳಿ-ದು-ಕೊ-ಳ್ಳು-ವುದ-ಕ್ಕೋ-ಸ್ಕರ
ತಿಳಿ-ದು-ಕೊ-ಳ್ಳುವ
ತಿಳಿ-ದು-ಕೊ-ಳ್ಳುವಂತಹ
ತಿಳಿ-ದು-ಕೊ-ಳ್ಳುವುದ-ರ-ಲ್ಲಿ
ತಿಳಿ-ದು-ಕೊ-ಳ್ಳುವುದು
ತಿಳಿ-ದು-ಕೊ-ಳ್ಳೋಣ-ವಾ-ಯಿತು
ತಿಳಿ-ದು-ಕೊಂಡ
ತಿಳಿ-ದು-ಕೊಂಡ-ವರ
ತಿಳಿ-ದು-ಕೊಂಡ-ಹಾ-ಗಿದೆ
ತಿಳಿ-ದು-ಕೊಂಡನು
ತಿಳಿ-ದು-ಕೊಂಡರು
ತಿಳಿ-ದು-ಕೊಂಡರೆ
ತಿಳಿ-ದು-ಕೊಂಡಳು
ತಿಳಿ-ದು-ಕೊಂಡಿ-ರುವ-ವರು
ತಿಳಿ-ದು-ಕೊಂಡಿ-ರುವಂತೆ
ತಿಳಿ-ದು-ಕೊಂಡಿ-ರುವಿ-ರೇನು
ತಿಳಿ-ದು-ಕೊಂಡಿ-ರುವುದು
ತಿಳಿ-ದು-ಕೊಂಡಿ-ಲ್ಲ
ತಿಳಿ-ದು-ಕೊಂಡಿದ್ದನು
ತಿಳಿ-ದು-ಕೊಂಡಿದ್ದರು
ತಿಳಿ-ದು-ಕೊಂಡಿದ್ದಾರೆಯೋ
ತಿಳಿ-ದು-ಕೊಂಡಿದ್ದೀಯ
ತಿಳಿ-ದು-ಕೊಂಡಿದ್ದೀರಿ
ತಿಳಿ-ದು-ಕೊಂಡಿರ-ಬೇಕೆಂದು
ತಿಳಿ-ದು-ಕೊಂಡು
ತಿಳಿ-ದು-ಕೊಂಡು-ಬಿ-ಟ್ಟಿ-ದ್ದರು
ತಿಳಿ-ದು-ಕೊಂಡು-ಬಿ-ಟ್ಟೆ
ತಿಳಿ-ದು-ಕೊಂಡೆ
ತಿಳಿ-ದು-ಕೊಂಡೆಯಾ
ತಿಳಿ-ದು-ಕೊಳ್ಳ-ಬೇಕು
ತಿಳಿ-ದು-ಕೊಳ್ಳಿ
ತಿಳಿ-ದುಕೊ
ತಿಳಿ-ದೆ-ಯೇನು
ತಿಳಿ-ದೆಯಾ
ತಿಳಿ-ದೊಡ-ನೆಯೆ
ತಿಳಿ-ನೀರಿ-ನಂತೆ
ತಿಳಿ-ಯ-ದಂತೆ
ತಿಳಿ-ಯ-ದಿ-ರಲಿ
ತಿಳಿ-ಯ-ದಿ-ರುವ-ವ-ನೆಂಬು-ದ-ನ್ನು
ತಿಳಿ-ಯ-ದೆಯೋ
ತಿಳಿ-ಯ-ಪಡಿ-ಸ-ಬೇಕಾಗಿದೆ
ತಿಳಿ-ಯ-ಪಡಿ-ಸಿ-ರುವರು
ತಿಳಿ-ಯ-ಬ-ಲ್ಲನು
ತಿಳಿ-ಯ-ಬಂದಿತು
ತಿಳಿ-ಯ-ಬೇ-ಕಾ-ದರೆ
ತಿಳಿ-ಯ-ಬೇಕಾಗಿದೆ
ತಿಳಿ-ಯ-ಬೇಕು
ತಿಳಿ-ಯ-ಬೇಕೆಂದು
ತಿಳಿ-ಯ-ಲಿ-ಲ್ಲ
ತಿಳಿ-ಯದ
ತಿಳಿ-ಯದ-ವ-ನಿಗೆ
ತಿಳಿ-ಯದು
ತಿಳಿ-ಯದೆ
ತಿಳಿ-ಯಲಿ
ತಿಳಿ-ಯಲು
ತಿಳಿ-ಯಲೆ-ತ್ನಿ-ಸಿ-ದನು
ತಿಳಿ-ಯಿತು
ತಿಳಿ-ಯಿತೆ
ತಿಳಿ-ಯಿರಿ
ತಿಳಿ-ಯು-ತ್ತಾನೆ
ತಿಳಿ-ಯು-ತ್ತೇನೆ
ತಿಳಿ-ಯು-ವಂತೆ
ತಿಳಿ-ಯು-ವು-ದಿ-ಲ್ಲವೊ
ತಿಳಿ-ಯು-ವುದು
ತಿಳಿ-ಯುವ
ತಿಳಿ-ಯುವಿರೊ
ತಿಳಿ-ಯುವೆ
ತಿಳಿ-ಸ-ಬೇಕು
ತಿಳಿ-ಸದೆ
ತಿಳಿ-ಸಲು
ತಿಳಿ-ಸಿ-ಕೊಡಿ
ತಿಳಿ-ಸಿ-ದನು
ತಿಳಿ-ಸಿ-ದರು
ತಿಳಿ-ಸಿ-ದರೆ
ತಿಳಿ-ಸಿ-ದ್ದರು
ತಿಳಿ-ಸಿ-ರ-ಲಿ-ಲ್ಲ
ತಿಳಿ-ಸಿದ
ತಿಳಿ-ಸು-ತ್ತಿ-ದ್ದರು
ತಿಳಿ-ಸು-ತ್ತೇನೆ
ತಿಳಿ-ಸು-ವಂತೆ
ತಿಳಿ-ಸು-ವು-ದಕ್ಕೆ
ತಿಳಿ-ಸು-ವು-ದಾಗಿ
ತಿಳಿ-ಸುವ
ತಿಳಿಸಿ
ತಿಳಿಸು
ತಿಳುವಳಿ-ಕೆಯೂ
ತಿಳುವಳಿಕೆ
ತಿಳುವಳಿಕೆ-ಗೆಂದು
ತಿಳುವಳಿಕೆ-ಯನ್ನು
ತಿವಿ-ದಾಗ
ತೀ
ತೀಕ್ಷ್ಣ
ತೀರ
ತೀರ-ದ-ಲ್ಲಿ
ತೀರ-ದ-ಲ್ಲಿ-ರುವ
ತೀರ-ದ-ಲ್ಲಿದೆ
ತೀರ-ದ-ಲ್ಲಿದ್ದ
ತೀರ-ದ-ವ-ರೆಗೆ
ತೀರ-ಬೇಕು
ತೀರಿ
ತೀರಿ-ಕೊಂಡಿರ-ಬಹುದು
ತೀರಿ-ದೊಡ-ನೆಯೆ
ತೀರಿ-ಸಿ-ಕೊ-ಳ್ಳು-ತ್ತೇನೆ
ತೀರಿ-ಸಿ-ಕೊ-ಳ್ಳು-ವು-ದಕ್ಕೆ
ತೀರಿ-ಸಿ-ಕೊ-ಳ್ಳುವ
ತೀರಿ-ಸಿ-ಕೊ-ಳ್ಳುವನು
ತೀರಿ-ಹೋಗಿ-ದ್ದರು
ತೀರಿ-ಹೋದ-ದ್ದಕ್ಕೆ
ತೀರಿ-ಹೋದ-ಮೇಲೆ
ತೀರಿ-ಹೋದರು
ತೀರಿ-ಹೋದಳು
ತೀರಿದ
ತೀರ್ಥ
ತೀರ್ಥ-ಕ್ಷೇ-ತ್ರ
ತೀರ್ಥ-ಕ್ಷೇ-ತ್ರ-ದ-ಲ್ಲಿ
ತೀರ್ಥ-ಕ್ಷೇ-ತ್ರ-ವಾ-ಗಿದೆ
ತೀರ್ಥ-ಕ್ಷೇ-ತ್ರ-ವಾ-ಯಿತು
ತೀರ್ಥ-ಕ್ಷೇ-ತ್ರವೂ
ತೀರ್ಥ-ಗ-ಳಿಗೆ
ತೀರ್ಥ-ಗಳೂ
ತೀರ್ಥ-ದಲಿ
ತೀರ್ಥ-ದಿಂದ
ತೀರ್ಥ-ಯಾ-ತ್ರೆ
ತೀರ್ಥ-ಯಾ-ತ್ರೆಗೆ
ತೀರ್ಥ-ರಾಮ
ತೀರ್ಥ-ರಾಮ-ರಿಗೂ
ತೀರ್ಥ-ರಾಮರ
ತೀರ್ಥ-ರಾಮರು
ತೀರ್ಥ-ವನ್ನು
ತೀರ್ಥ-ಸ್ಥಳ
ತೀರ್ಥ-ಸ್ಥಳ-ಗ-ಳನ್ನು
ತೀರ್ಥ-ಸ್ಥಳ-ಗ-ಳಿಗೆ
ತೀರ್ಥ-ಸ್ಥಳ-ಗಳ-ನ್ನೆ-ಲ್ಲ
ತೀರ್ಥ-ಸ್ಥಳ-ಗಳು
ತೀರ್ಥ-ಸ್ಥಳಕ್ಕೆ
ತೀರ್ಥ-ಸ್ಥಳ್ದ-ಲ್ಲಿ
ತೀರ್ಥಕ್ಕೆ
ತೀರ್ಪ-ನ್ನು
ತೀರ್ಪು
ತೀರ್ಮಾನ-ಮಾಡಿ-ದರು
ತೀರ್ಮಾನಿ-ಸಿದೆ
ತೀವ್ರ
ತೀವ್ರ-ಕೋಪ
ತೀವ್ರ-ವಾಗ-ತೊಡಗಿತು
ತೀವ್ರ-ವಾಗಿ
ತೀವ್ರ-ವಾಗಿ-ದ್ದರೆ
ತೀವ್ರ-ವಾಗಿ-ದ್ದವು
ತೀವ್ರ-ವಾದ
ತೀವ್ರತೆ
ತು
ತುಂಟ
ತುಂಟ-ತನ
ತುಂಟ-ತನ-ವನ್ನು
ತುಂಟ-ತನ-ವನ್ನೇ
ತುಂಡ-ನ್ನು
ತುಂಡಿನ
ತುಂಡು
ತುಂಡು-ಗ-ಳನ್ನು
ತುಂಬ
ತುಂಬ-ಬ-ಲ್ಲನು
ತುಂಬ-ಬೇಕು
ತುಂಬ-ಬೇಕೋ
ತುಂಬಾ
ತುಂಬಿ
ತುಂಬಿ-ಕೊಂಡಂತಿದೆ
ತುಂಬಿ-ಕೊಂಡು
ತುಂಬಿ-ತು-ಳು-ಕಾಡು-ತ್ತಿದ್ದಾರೆ
ತುಂಬಿ-ತುಳು-ಕಾಡು-ತ್ತಿದೆ
ತುಂಬಿ-ದ-ವರು
ತುಂಬಿ-ದರೆ
ತುಂಬಿ-ದ್ದವು
ತುಂಬಿ-ರು-ವುವು
ತುಂಬಿ-ರುವ
ತುಂಬಿ-ಸಲು
ತುಂಬಿ-ಸು-ವು-ದಕ್ಕೆ
ತುಂಬಿ-ಹೋ-ಗಿದೆ
ತುಂಬಿ-ಹೋಗಿ-ತ್ತು
ತುಂಬಿ-ಹೋಗಿ-ರುವ
ತುಂಬಿ-ಹೋಗಿದ್ದೆ
ತುಂಬಿ-ಹೋಗಿವೆ
ತುಂಬಿ-ಹೋಗು-ತ್ತದೆ
ತುಂಬಿತು
ತುಂಬಿದ
ತುಂಬು
ತುಂಬು-ತ್ತಿದ್ದರು
ತುಂಬು-ವುದು
ತುಂಬುವ
ತುಚ್ಛೀ-ಕರಿ-ಸುವರು
ತುಚ್ಛೀಕ-ರಿಸಿ
ತುಟಿ-ಗಳಿಂದ
ತುದಿ-ಯನ್ನು
ತುಪ್ಪ-ದ-ಲ್ಲಿ
ತುಮುಲ
ತುರಾಯಿ-ಗಳ-ನ್ನರ್ಪಿ-ಸಿ-ದರು
ತುರಿ-ಯಾ-ನಂದ
ತುರಿ-ಯಾ-ನಂದ-ರ-ನ್ನು
ತುರಿ-ಯಾ-ನಂದ-ರಿಗೆ
ತುರಿ-ಯಾ-ನಂದ-ರೊ-ಡನೆ
ತುರಿ-ಯಾ-ನಂದರು
ತುರುಕಿ-ದನು
ತುರ್ತು
ತುರ್ತು-ಕರೆ
ತುಲನೆ
ತುಳಸಿ-ಗಳೊ-ಡನೆ
ತುಳಿ-ದಿದೆ
ತುಳಿ-ಯು-ವಂತೆ
ತುಳಿಯು-ತ್ತಿರು-ವೆವು
ತುಳು-ಕಾ-ಡುವ
ತುಳು-ಕಾಡಿತು
ತುಳು-ಕಾಡು-ತ್ತದೆ
ತುಳು-ಕಾಡು-ತ್ತಿ-ತ್ತು
ತುಳು-ಕಾಡು-ತ್ತಿದೆ
ತುಳು-ಕಾಡು-ತ್ತಿದ್ದವು
ತುಳು-ಕಾಡು-ತ್ತಿರು-ವರು
ತುಳು-ಕಾಡು-ವಂತೆ
ತುಳುಕು-ತ್ತಿ-ತ್ತು
ತುಳುಕು-ತ್ತಿದ್ದ
ತುಷ್ಟಿ
ತೂ
ತೂಕ-ಮಾಡಿ-ದ್ದರೆ
ತೂಕ-ವಾಗಿ-ರು-ವು-ದ-ನ್ನು
ತೂಕ-ವುಳ್ಳ
ತೂಕಡಿ-ಸು-ತ್ತಿ-ದ್ದರು
ತೂಗುಯ್ಯಾಲೆ-ಯಂತೆ
ತೂಗುವ
ತೂರಾಡು-ತ್ತಿ-ತ್ತು
ತೂರಾಡು-ವಂತೆ
ತೂರಿ-ಕೊಂಡು
ತೂರಿ-ಹೋಗಿ
ತೂರ್ಯ-ವಾಣಿ-ಯನ್ನು
ತೂರ್ಯ-ವಾಣಿ-ಯೊಂದು
ತೃಣ-ದಂತೆ
ತೃಣಕ್ಕೆ
ತೃಪ್ತ-ನಾಗು-ವನು
ತೃಪ್ತ-ನಾದೆ
ತೃಪ್ತ-ರಾ-ದರು
ತೃಪ್ತ-ರಾಗಿ-ರುವರು
ತೃಪ್ತನಾಗಲಾರ
ತೃಪ್ತರಾಗ-ಬೇಕು
ತೃಪ್ತಿ
ತೃಪ್ತಿ-ಕರ-ವಾ-ಗಿದೆ
ತೃಪ್ತಿ-ಕರ-ವಾಗಿ
ತೃಪ್ತಿ-ಕರ-ವಾಗಿ-ತ್ತು
ತೃಪ್ತಿ-ಕರ-ವಾಗಿ-ರ-ಲಿ-ಲ್ಲ
ತೃಪ್ತಿ-ಕರ-ವಾದ
ತೃಪ್ತಿ-ದಾಯಕ-ವಾ-ಗಿದೆ
ತೃಪ್ತಿ-ದಾಯಕ-ವಾಗಿ-ದ್ದವು
ತೃಪ್ತಿ-ದಾಯಕ-ವಾಗು-ವಂತೆ
ತೃಪ್ತಿ-ಪಡಿ-ಸ-ಬೇಕಾಗಿ-ತ್ತು
ತೃಪ್ತಿ-ಪಡಿ-ಸ-ಬೇಕು
ತೃಪ್ತಿ-ಪಡಿ-ಸು-ವು-ದ-ಕ್ಕಾಗಿ
ತೃಪ್ತಿ-ಯನ್ನು
ತೃಪ್ತಿ-ಯಾ-ಯಿತು
ತೃಪ್ತಿ-ಯಾಗು-ತ್ತ-ದೆಯೆ
ತೃಪ್ತಿ-ಯಿ-ಲ್ಲ
ತೃಪ್ತಿಯೂ
ತೃಷೆ
ತೃಷೆ-ಯನ್ನು
ತೆ
ತೆಂಗಿನ
ತೆಂಗಿನ-ಮರ-ಗಳು
ತೆಂಗು
ತೆಗೆ-ದರೆ
ತೆಗೆ-ದಾರು
ತೆಗೆ-ದಿ-ಟ್ಟಿ-ರು-ವೆನು
ತೆಗೆ-ದಿ-ಡದೆ
ತೆಗೆ-ದಿರಿಸಿ
ತೆಗೆ-ದು-ಕೊ-ಟ್ಟನು
ತೆಗೆ-ದು-ಕೊ-ಟ್ಟರೆ
ತೆಗೆ-ದು-ಕೊ-ಟ್ಟಿದ್ದರು
ತೆಗೆ-ದು-ಕೊ-ಟ್ಟು
ತೆಗೆ-ದು-ಕೊ-ಳ್ಳ-ಬಹು-ದೆಂದು
ತೆಗೆ-ದು-ಕೊ-ಳ್ಳ-ಬಹುದು
ತೆಗೆ-ದು-ಕೊ-ಳ್ಳ-ಬಾ-ರದು
ತೆಗೆ-ದು-ಕೊ-ಳ್ಳ-ಬೇಕಾ-ಯಿತು
ತೆಗೆ-ದು-ಕೊ-ಳ್ಳ-ಬೇಕಾಗಿ-ಲ್ಲ
ತೆಗೆ-ದು-ಕೊ-ಳ್ಳ-ಬೇಕಾಗಿದೆ
ತೆಗೆ-ದು-ಕೊ-ಳ್ಳ-ಬೇಕೆಂದು
ತೆಗೆ-ದು-ಕೊ-ಳ್ಳ-ಬೇಕೆಂದೂ
ತೆಗೆ-ದು-ಕೊ-ಳ್ಳದೆ
ತೆಗೆ-ದು-ಕೊ-ಳ್ಳದೇ
ತೆಗೆ-ದು-ಕೊ-ಳ್ಳಲಿ
ತೆಗೆ-ದು-ಕೊ-ಳ್ಳಲಿ-ಲ್ಲ
ತೆಗೆ-ದು-ಕೊ-ಳ್ಳಲು
ತೆಗೆ-ದು-ಕೊ-ಳ್ಳು-ತ್ತ
ತೆಗೆ-ದು-ಕೊ-ಳ್ಳು-ತ್ತಲೇ
ತೆಗೆ-ದು-ಕೊ-ಳ್ಳು-ತ್ತಾರೆ
ತೆಗೆ-ದು-ಕೊ-ಳ್ಳು-ತ್ತಿ-ರಲಿ-ಲ್ಲ
ತೆಗೆ-ದು-ಕೊ-ಳ್ಳು-ತ್ತಿದ್ದನು
ತೆಗೆ-ದು-ಕೊ-ಳ್ಳು-ತ್ತಿದ್ದರು
ತೆಗೆ-ದು-ಕೊ-ಳ್ಳು-ತ್ತಿದ್ದೆ
ತೆಗೆ-ದು-ಕೊ-ಳ್ಳು-ತ್ತಿರು-ವ-ರೆಂದೂ
ತೆಗೆ-ದು-ಕೊ-ಳ್ಳು-ತ್ತೀರಿ
ತೆಗೆ-ದು-ಕೊ-ಳ್ಳು-ತ್ತೇನೆ
ತೆಗೆ-ದು-ಕೊ-ಳ್ಳು-ತ್ತೇನೆಂದು
ತೆಗೆ-ದು-ಕೊ-ಳ್ಳು-ತ್ತೇವೆ
ತೆಗೆ-ದು-ಕೊ-ಳ್ಳು-ವು-ದ-ನ್ನು
ತೆಗೆ-ದು-ಕೊ-ಳ್ಳು-ವು-ದಕ್ಕೆ
ತೆಗೆ-ದು-ಕೊ-ಳ್ಳು-ವು-ದಿ-ಲ್ಲ
ತೆಗೆ-ದು-ಕೊ-ಳ್ಳುವ
ತೆಗೆ-ದು-ಕೊ-ಳ್ಳುವ-ರೆಂಬುದು
ತೆಗೆ-ದು-ಕೊ-ಳ್ಳುವಂತೆ
ತೆಗೆ-ದು-ಕೊ-ಳ್ಳುವನೋ
ತೆಗೆ-ದು-ಕೊ-ಳ್ಳುವರು
ತೆಗೆ-ದು-ಕೊ-ಳ್ಳುವರೋ
ತೆಗೆ-ದು-ಕೊ-ಳ್ಳುವಾಗ
ತೆಗೆ-ದು-ಕೊ-ಳ್ಳುವು-ದರ
ತೆಗೆ-ದು-ಕೊ-ಳ್ಳುವು-ದಾಗಿ-ತ್ತು
ತೆಗೆ-ದು-ಕೊ-ಳ್ಳುವುದು
ತೆಗೆ-ದು-ಕೊ-ಳ್ಳುವುದೇ
ತೆಗೆ-ದು-ಕೊ-ಳ್ಳುವುದೊ
ತೆಗೆ-ದು-ಕೊ-ಳ್ಳುವೆನು
ತೆಗೆ-ದು-ಕೊಂಡ
ತೆಗೆ-ದು-ಕೊಂಡ-ಮೇಲೆ
ತೆಗೆ-ದು-ಕೊಂಡ-ವರೆ
ತೆಗೆ-ದು-ಕೊಂಡರು
ತೆಗೆ-ದು-ಕೊಂಡರೂ
ತೆಗೆ-ದು-ಕೊಂಡರೆ
ತೆಗೆ-ದು-ಕೊಂಡಾಗ
ತೆಗೆ-ದು-ಕೊಂಡಿ-ರುವ
ತೆಗೆ-ದು-ಕೊಂಡಿ-ರುವ-ರೆಂದೂ
ತೆಗೆ-ದು-ಕೊಂಡಿ-ರುವೆ
ತೆಗೆ-ದು-ಕೊಂಡಿ-ಲ್ಲ
ತೆಗೆ-ದು-ಕೊಂಡು
ತೆಗೆ-ದು-ಕೊಂಡು-ಬಿ-ಟ್ಟರು
ತೆಗೆ-ದು-ಕೊಳ್ಳ-ಬೇಕು
ತೆಗೆ-ದು-ಕೊಳ್ಳಿ
ತೆಗೆ-ದು-ಬಿ-ಟ್ಟಿತು
ತೆಗೆ-ದು-ಬಿಡು-ತ್ತಿದ್ದರು
ತೆಗೆ-ದು-ಹಾಕಲು
ತೆಗೆ-ದು-ಹಾಕಿ-ದರು
ತೆಗೆ-ದು-ಹಾಕಿ-ದೆವು
ತೆಗೆ-ದು-ಹಾಕಿ-ಬಿ-ಟ್ಟರೆ
ತೆಗೆ-ದು-ಹಾಕು-ವು-ದಕ್ಕೆ
ತೆಗೆ-ದುಕೊ
ತೆಗೆ-ದುಕೋ
ತೆಗೆ-ಯ-ಬೇಕು
ತೆಗೆ-ಯ-ಲಿ-ಲ್ಲ
ತೆಗೆ-ಯಲು
ತೆಗೆ-ಯಿರಿ
ತೆಗೆ-ಯು-ತ್ತಿದ್ದಳು
ತೆಗೆ-ಯು-ವಂತೆ
ತೆಗೆ-ಯು-ವು-ದ-ನ್ನು
ತೆಗೆ-ಯುವನು
ತೆಗೆದ
ತೆಗೆದು
ತೆದು-ಕೊಂಡು
ತೆಪ್ಪಗಾ-ದರು
ತೆಪ್ಪಗಾಗು-ತ್ತಿದ್ದ
ತೆಪ್ಪಗಾಗು-ತ್ತಿದ್ದರು
ತೆಪ್ಪಗಾಗು-ತ್ತಿದ್ದೆ
ತೆಪ್ಪೋ-ತ್ಸವ
ತೆಯ್ಯು-ವು-ದಕ್ಕೆ
ತೆರ-ಬೇಕು
ತೆರ-ಳಿದರು
ತೆರಳಿತು
ತೆರಳು-ತ್ತೇನೆ
ತೆರೆ
ತೆರೆ-ಗ-ಳನ್ನು
ತೆರೆ-ದ-ದ್ದಾ-ಯಿತು
ತೆರೆ-ದ-ಮೇಲೆ
ತೆರೆ-ದಂ-ತಾ-ಯಿತು
ತೆರೆ-ದರು
ತೆರೆ-ದಳು
ತೆರೆ-ದವು
ತೆರೆ-ದಾಗ
ತೆರೆ-ದಿ-ರು-ವಿರಿ
ತೆರೆ-ದಿದೆ
ತೆರೆ-ಯ-ಬಹುದು
ತೆರೆ-ಯ-ಬೇಕೆಂದೂ
ತೆರೆ-ಯ-ಲ್ಲಿ
ತೆರೆ-ಯನ್ನು
ತೆರೆ-ಯಲು
ತೆರೆ-ಯು-ವಂತೆ
ತೆರೆ-ಯು-ವು-ದ-ಕ್ಕಾಗಿ
ತೆರೆ-ಯು-ವು-ದಕ್ಕೆ
ತೆರೆ-ಯು-ವುದು
ತೆರೆ-ಯುವ
ತೆರೆ-ಯೋಣ
ತೆರೆದ
ತೆರೆದು
ತೆರೆಸಿ
ತೆಲುಗಿ-ನ-ಲ್ಲಿ
ತೆಳು-ವಾಗಿ-ರುವುದು
ತೆಳ್ಳಗೆ
ತೇ
ತೇಜ-ಸ್ವಿ-ಯಾದ
ತೇಜ-ಸ್ಸಿ-ನಿಂದ
ತೇಜ-ಸ್ಸಿನ
ತೇಜ-ಸ್ಸು
ತೇಜಃಪುಂಜ
ತೇಜಃಪುಂಜ-ವಾದ
ತೇಜೋ-ಮಯ-ವಾದ
ತೇಜೋ-ರಾಶಿ
ತೇಜೋ-ರಾಶಿ-ಯ-ಲ್ಲಿ
ತೇಜೋ-ರಾಶಿ-ಯಿಂದ
ತೇದಿ
ತೇಲಿ
ತೇಲಿ-ಕೊಂಡು
ತೇಲಿ-ಬಿಡು-ತ್ತಿದ್ದರು
ತೇಲಿ-ಹೋಗಿ
ತೇಲಿ-ಹೋಗು-ತ್ತಿದ್ದರು
ತೇಲು-ತ್ತ
ತೇಲು-ತ್ತಿದ್ದರು
ತೇಲು-ತ್ತಿದ್ದರೆ
ತೇಲು-ತ್ತಿದ್ದಾಗ
ತೇಲು-ತ್ತಿರ-ವೆನು
ತೇಲು-ತ್ತಿರು-ವೆನು
ತೇಲು-ತ್ತಿರುವ
ತೇಲು-ತ್ತೇನೆ
ತೇಲುವ
ತೇಷು
ತೈಲ-ಚಿ-ತ್ರ
ತೈಲ-ಚಿ-ತ್ರ-ಗ-ಳನ್ನು
ತೈಲ-ಚಿ-ತ್ರ-ಗಳು
ತೈಲ-ಚಿ-ತ್ರ-ದ-ಲ್ಲೂ
ತೊ
ತೊಂ
ತೊಂದ-ರೆಗೆ
ತೊಂಬ-ತ್ತು
ತೊಡ-ಬೇಕೆಂದೂ
ತೊಡಕು-ಗಳೆ-ಲ್ಲ
ತೊಡಗಿದರು
ತೊಡಗಿದರೊ
ತೊಡಿ-ಸಿದ
ತೊಡಿ-ಸು-ವು-ದಕ್ಕೆ
ತೊಡಿಗೆ
ತೊಡಿಗೆ-ಗಾಗಿ
ತೊಡಿಗೆಯ
ತೊಡೆ-ಗಳ
ತೊಡೆದುಹಾಕು
ತೊಡೆಯ
ತೊಡೆಯ-ಮೇಲೆ
ತೊಯ್ದು-ಹೋಗಿ-ದ್ದವು
ತೊರಿ-ರು-ವುವು
ತೊರೆದು
ತೊರೆದು-ಹಾಕಿ
ತೊಲಗಿ
ತೊಳಲಾಡಿ
ತೊಳೆ-ದನು
ತೊಳೆ-ದರೂ
ತೊಳೆ-ಯಲು
ತೊಳೆ-ಯು-ವು-ದಕ್ಕೆ
ತೊಳೆ-ಯು-ವುದು
ತೊಳೆ-ಯು-ವುದೇ
ತೊಳೆದು
ತೊಳೆದು-ಕೊ-ಳ್ಳದೆ
ತೊಳೆದು-ಕೊ-ಳ್ಳು-ವು-ದಕ್ಕೆ
ತೊಳೆದು-ಕೊಂಡು
ತೊಳೆದು-ಕೊಳ್ಳಿ
ತೊಳೆಯು-ತ್ತಿವೆ
ತೊವ್ವೆ
ತೋ
ತೋಚ-ಲಿ-ಲ್ಲ
ತೋಚಿ-ದಂತೆ
ತೋಚಿ-ದು-ದ-ನ್ನು
ತೋಚು-ವು-ದಿ-ಲ್ಲವೋ
ತೋಟ
ತೋಟ-ದ-ಲ್ಲಿ
ತೋಟ-ದ-ಲ್ಲಿದ್ದ
ತೋಟ-ವನ್ನು
ತೋಟ-ವಿ-ತ್ತು
ತೋಟ-ವಿದೆ
ತೋಟಕ್ಕೆ
ತೋಟದ
ತೋಡಿ
ತೋಡಿ-ಕೊ-ಳ್ಳು-ತ್ತಿದ್ದರು
ತೋತಾಪು-ರಿಗೆ
ತೋತಾಪುರಿ
ತೋಯಿ-ಸಿ-ರುವರು
ತೋಯಿಸಿ-ರು-ವುವು
ತೋರ-ದು-ದರ
ತೋರ-ಬ-ಲ್ಲ
ತೋರ-ಬ-ಲ್ಲರೆ
ತೋರ-ಬಹುದು
ತೋರ-ಬೇಕೆಂದು
ತೋರ-ಬೇಕೆಂದೂ
ತೋರ-ಬೇಕೆಂಬು-ದ-ನ್ನು
ತೋರಣ-ಗ-ಳನ್ನು
ತೋರಣ-ಗಳಿಂದ
ತೋರಣ-ವಿ-ತ್ತು
ತೋರಲು
ತೋರಿ
ತೋರಿ-ಕೆಗೆ
ತೋರಿ-ಕೆಯ
ತೋರಿ-ದ-ವರು
ತೋರಿ-ದನು
ತೋರಿ-ದರು
ತೋರಿ-ದರೂ
ತೋರಿ-ದರೆ
ತೋರಿ-ದರೊ
ತೋರಿ-ದಳು
ತೋರಿ-ದಷ್ಟು
ತೋರಿ-ದು-ದ-ನ್ನು
ತೋರಿ-ದ್ದರು
ತೋರಿ-ದ್ದರೆ
ತೋರಿ-ಬಂದರೆ
ತೋರಿ-ರು-ವ-ರೆಂದೂ
ತೋರಿ-ರು-ವನು
ತೋರಿ-ರು-ವಳು
ತೋರಿ-ರುವರು
ತೋರಿ-ರುವೆ
ತೋರಿ-ಸ-ಬ-ಲ್ಲ
ತೋರಿ-ಸ-ಬಹುದು
ತೋರಿ-ಸ-ಬೇಕಾಗಿದೆ
ತೋರಿ-ಸ-ಲಿ-ಲ್ಲ
ತೋರಿ-ಸದೆ
ತೋರಿ-ಸಲು
ತೋರಿ-ಸಿ-ಕೊ-ಟ್ಟು
ತೋರಿ-ಸಿ-ಕೊ-ಳ್ಳು-ತ್ತಿದ್ದ
ತೋರಿ-ಸಿ-ಕೊಂಡು
ತೋರಿ-ಸಿ-ದರು
ತೋರಿ-ಸಿ-ದರೆ
ತೋರಿ-ಸಿ-ದ್ದರು
ತೋರಿ-ಸಿ-ದ್ದಾರೆ
ತೋರಿ-ಸಿ-ರು-ವನು
ತೋರಿ-ಸಿ-ಲ್ಲ
ತೋರಿ-ಸಿತು
ತೋರಿ-ಸಿದ
ತೋರಿ-ಸಿದೆ
ತೋರಿ-ಸು-ತ್ತ
ತೋರಿ-ಸು-ತ್ತವೆ
ತೋರಿ-ಸು-ತ್ತಾ
ತೋರಿ-ಸು-ತ್ತಿದ್ದ
ತೋರಿ-ಸು-ತ್ತಿದ್ದರು
ತೋರಿ-ಸು-ತ್ತಿರುವ
ತೋರಿ-ಸು-ತ್ತೇನೆ
ತೋರಿ-ಸು-ವು-ದ-ಕ್ಕಾಗಿ
ತೋರಿ-ಸು-ವು-ದಕ್ಕೆ
ತೋರಿ-ಸು-ವುದ-ರ-ಲ್ಲೆ
ತೋರಿ-ಸು-ವುದು
ತೋರಿ-ಸುವ
ತೋರಿಕೆ
ತೋರಿತು
ತೋರಿದ
ತೋರಿಸು
ತೋರು-ತ್ತ
ತೋರು-ತ್ತಿದೆ
ತೋರು-ತ್ತಿದ್ದವು
ತೋರು-ತ್ತಿರು-ವರು
ತೋರು-ವಂತಹ
ತೋರು-ವಂತಿದೆ
ತೋರು-ವನೆ
ತೋರು-ವುದೆಂಬ
ತೋರು-ವುದೊ
ತೋಳು-ಗಳ-ಲ್ಲಿ
ತೋಳು-ಗಳ-ಲ್ಲಿಯೂ
ತೌರು-ಮನೆ-ಯಂತಿ-ರುವ
ತೌರೂರು
ತ್ಮ
ತ್ಮಿಕ
ತ್ಯಜ
ತ್ಯಜಿ-ಸಲಾರರು
ತ್ಯಜಿ-ಸಲು
ತ್ಯಜಿ-ಸಲೂ
ತ್ಯಜಿ-ಸಿದ
ತ್ಯಜಿ-ಸಿದೆ
ತ್ಯಜಿ-ಸು-ತ್ತಿ-ರುವರು
ತ್ಯಜಿಸ-ಕೂ-ಡದು
ತ್ಯಜಿಸ-ಬೇಕು
ತ್ಯಜಿಸ-ಬೇಕೆಂದು
ತ್ಯಜಿಸಿ
ತ್ಯಜಿಸಿ-ದ-ವನು
ತ್ಯಜಿಸಿ-ದ-ವರು
ತ್ಯಜಿಸಿ-ದರು
ತ್ಯಜಿಸಿ-ಬಿಡ-ಬಹುದು
ತ್ಯಜಿಸಿ-ಬಿಡಿ
ತ್ಯಜಿಸಿ-ರುವರು
ತ್ಯಜಿಸು
ತ್ಯಜಿಸು-ತ್ತೇನೆ
ತ್ಯಜಿಸು-ವಂತೆ
ತ್ಯಜಿಸು-ವು-ದಕ್ಕೆ
ತ್ಯಜಿಸು-ವುದು
ತ್ಯಾಗ
ತ್ಯಾಗ-ಗಳು
ತ್ಯಾಗ-ಜೀವ-ನದ
ತ್ಯಾಗ-ಜೀವನ
ತ್ಯಾಗ-ದ-ಲ್ಲಿ
ತ್ಯಾಗ-ದಿಂದ
ತ್ಯಾಗ-ಭೂಮಿ-ಯಿಂದ
ತ್ಯಾಗ-ಮಾಡಲೇ-ಬೇಕಾಗು-ತ್ತದೆ
ತ್ಯಾಗ-ಮಾಡಿದ
ತ್ಯಾಗ-ಮಾಡು-ವಂತಹ
ತ್ಯಾಗ-ಮಾಡು-ವುದೇ
ತ್ಯಾಗ-ವನ್ನು
ತ್ಯಾಗ-ವನ್ನೂ
ತ್ಯಾಗ-ವಾಗಿ-ರ-ಬಹುದು
ತ್ಯಾಗ-ವಿ-ಲ್ಲದೆ
ತ್ಯಾಗ-ವೆಂ-ದರೆ
ತ್ಯಾಗದ
ತ್ಯಾಗವೂ
ತ್ಯಾಗವೇ
ತ್ಯಾಗಿ
ತ್ಯಾಗಿ-ಗ-ಳಾದ
ತ್ಯಾಗಿ-ಗ-ಳಾದಂಥ
ತ್ಯಾಗಿ-ಗಳಂತೆ
ತ್ಯಾಗಿ-ಗಳಾಗಿ
ತ್ಯಾಗಿ-ಗಳಾಗಿ-ದ್ದರು
ತ್ಯಾಗಿ-ಗಳೆ-ಲ್ಲ
ತ್ಯಾಗಿ-ಯಾಗಿ-ರುವೆಯೊ
ತ್ಯಾಗಿಗೆ
ತ್ಯಾಗಿಯೇ
ತ್ಯಾಗೀ
ತ್ರಯೀ
ತ್ರಯೋದಶಿಯ
ತ್ರಾಣ-ವಿ-ರ-ಲಿ-ಲ್ಲ
ತ್ರಾಸ-ವಾ-ಯಿತು
ತ್ರಿ
ತ್ರಿ-ಕರ-ಣ-ಪೂರ್ವಕ
ತ್ರಿಶೂಲ
ತ್ರೈ
ತ್ವರೆ-ಯಿಂದ
ಥರ್ಸ
ಥಳಕಿನ
ಥಾಮ-ಸ್
ಥಿಯಸಫಿ-ಸ್ಟ-ರಿಗೆ
ಥಿಯಾಸಫಿ
ಥಿಯಾಸಫಿ-ಕ-ಲ್
ಥಿಯಾಸಫಿ-ಯ-ವರು
ಥಿಯಾಸಫಿಗೆ
ಥಿಯೇಟರಿ-ನ-ಲ್ಲಿ
ಥೆರಪುಟ
ದ
ದಂ
ದಂಡ
ದಂಡ-ಕ-ಮಂಡಲು-ಧಾರಿ-ಗಳಾಗಿ
ದಂಡ-ಕ-ಮಂಡಲು-ಧಾರಿ-ಯಾಗಿ
ದಂಡ-ದಂತಿ-ರ-ಲಿ-ಲ್ಲ
ದಂಡ-ಪ್ರಣ-ಮ-ಗ-ಳನ್ನು
ದಂಡ-ಪ್ರಣಾಮ
ದಂಡ-ಯಾ-ತ್ರೆ
ದಂಡ-ವಾಗಿ-ರ-ಲಿ-ಲ್ಲ
ದಂಡಿ-ಯ-ಲ್ಲಿ
ದಂಡೆ-ಯಾ-ತ್ರೆ
ದಂತಹ
ದಂಪಂತಿ-ಗಳು
ದಕ್ಕಿ-ರ-ಲಿ-ಲ್ಲ
ದಕ್ಷಿ-ಣದ
ದಕ್ಷಿಣ
ದಕ್ಷಿಣ-ದ-ಲ್ಲಿ
ದಕ್ಷಿಣ-ದ-ಲ್ಲಿ-ರುವ
ದಕ್ಷಿಣ-ಭಾಗದ
ದಕ್ಷಿಣ-ವೆ-ನ್ನದೆ
ದಕ್ಷಿಣಕ್ಕೆ
ದಕ್ಷಿಣಾ-ತ್ಯ
ದಕ್ಷಿಣಾ-ತ್ಯರು
ದಕ್ಷಿಣೆ-ಗಾಗಿ
ದಕ್ಷಿಣೆ-ಯನ್ನೂ
ದಕ್ಷಿಣೇಶ್ವ-ರಕ್ಕೆ
ದಕ್ಷಿಣೇಶ್ವ-ರದ
ದಕ್ಷಿಣೇಶ್ವರ
ದಕ್ಷಿಣೇಶ್ವರ-ದ-ಲ್ಲಿ
ದಕ್ಷಿಣೇಶ್ವರ-ದ-ಲ್ಲಿ-ರುವ
ದಕ್ಷಿಣೇಶ್ವರ-ದ-ಲ್ಲಿಯೇ
ದಕ್ಷಿಣೇಶ್ವರ-ವನ್ನು
ದಕ್ಷಿಣೇಶ್ವರಕೆ
ದಡ-ಗಳ
ದಡ-ಮೀರಿ
ದಡಕ್ಕೆ
ದಡಾರ-ದಿಂದ
ದಣಿದು
ದಧೀಚಿ
ದನ
ದನಕಾಯು-ವ-ವರು
ದನಿ-ಯನ್ನು
ದಪ್ಪ-ನಾದ
ದಬ್ಬ-ಬೇಕು
ದಬ್ಬಾಳಿ-ಕೆಗೆ
ದಬ್ಬಾಳಿಕೆ
ದಬ್ಬು-ವರು
ದಯ-ಪಾಲಿಸಿ-ದರು
ದಯ-ವಿ-ಟ್ಟು
ದಯಪಾಲಿ-ಸಲಿ
ದಯಪಾಲಿ-ಸೆಂದು
ದಯವಿಟು
ದಯಾ
ದಯಾ-ದಾಕ್ಷಿಣ್ಯ-ವಿ-ಲ್ಲದೆ
ದಯಾ-ಮಯ
ದಯಾ-ಮಯ-ನಾದ
ದಯಾ-ಮಯ-ರಾದ
ದಯಾ-ಮಯಿ
ದಯಾ-ಮೂರ್ತಿ-ಯಾಗಿ
ದಯಾ-ಳು-ಗಳು
ದಯಾ-ಳು-ವಾದ
ದಯಾ-ಸಿಂಧು
ದಯಾ-ಸಿಂಧುವೂ
ದಯೆ
ದಯೆ-ಗಾಗಿ
ದಯೆ-ಯ-ಲ್ಲಿ
ದಯೆ-ಯನ್ನು
ದಯೆ-ಯಿಂದ
ದಯೆಯೇ
ದರ
ದರಿದ್ರ
ದರಿದ್ರ-ನ-ವ-ರೆಗೆ
ದರಿದ್ರ-ನಾರಾ-ಯಣ-ನಿಗೆ
ದರಿದ್ರ-ರ-ನ್ನು
ದರಿದ್ರ-ರ-ಲ್ಲಿ
ದರಿದ್ರ-ರಿಗೆ
ದರಿದ್ರ-ಲಕ್ಷ್ಮಿ
ದರಿದ್ರದು
ದರಿದ್ರರು
ದರಿದ್ರರೂ
ದರೋ-ಡೆ-ಗಾರನೋ
ದರೋ-ಡೆ-ಯನ್ನು
ದರ್ಜಿ-ಯ-ವರ
ದರ್ಜೆ-ಯ-ಲ್ಲಿ
ದರ್ಜೆಗೆ
ದರ್ಜೆಯ
ದರ್ಪ
ದರ್ಬಾರಿ-ನ-ಲ್ಲಿ
ದರ್ಬಾರು
ದರ್ಬಾರ್
ದರ್ಭಾಂಗದ
ದರ್ಮ-ವನ್ನೇ
ದರ್ಮದ
ದರ್ವಾ-ನ್
ದರ್ಶ-ನಕ್ಕೆ
ದರ್ಶನ
ದರ್ಶನ-ಕ್ಕಾಗಿ
ದರ್ಶನ-ಗ-ಳನ್ನು
ದರ್ಶನ-ಗಳ
ದರ್ಶನ-ಗಳ-ಲ್ಲಿ
ದರ್ಶನ-ದ-ಲ್ಲಿ
ದರ್ಶನ-ದಿಂದ
ದರ್ಶನ-ಮಾಡಿ
ದರ್ಶನ-ವನ್ನು
ದರ್ಶನ-ವಾ-ಯಿತು
ದರ್ಶನ-ವಾ-ಯಿತೆಂದು
ದರ್ಶನ-ವಾಗು-ತ್ತದೆ
ದರ್ಶನದ
ದರ್ಶನವೂ
ದರ್ಶನಾ-ಕಾಂ-ಕ್ಷಿ-ಗಳಾಗಿ
ದರ್ಶನಾ-ಕಾಂ-ಕ್ಷಿ-ಗಳಾಗಿ-ದ್ದರು
ದರ್ಶನಾರಭ್ಯ
ದರ್ಶನಾರ್ಥ-ವಾಗಿ
ದರ್ಶನಾರ್ಥಿ-ಗಳು
ದಲಿ-ತ-ರ-ನ್ನು
ದಲಿ-ತ-ರಿಗೆ
ದಲಿ-ತರೋ
ದಳದ
ದಳ್ಳು-ರಿಯ
ದವ-ಡೆಗೆ
ದವಡೆಯೊಳಗಿಂದ
ದವಸ-ಕೊ-ಟ್ಟ
ದಶ-ದಿಕ್ಕು-ಗಳಿಂದಲೂ
ದಶ-ದಿಕ್ಕು-ಗಳೂ
ದಶಕಂಠ
ದಶಮ-ಸ್ಥಾನ-ದ-ಲ್ಲಿದ್ದ
ದಹಿ-ಸ-ಬೇಕಾ-ಯಿತು
ದಹಿ-ಸು-ತ್ತಿ-ತ್ತು
ದಹಿ-ಸು-ತ್ತಿ-ರುವ
ದಾ
ದಾಕ್ಷಿಣ್ಯ-ವನ್ನು
ದಾಕ್ಷಿಣ್ಯ-ವಿ-ಲ್ಲದೆ
ದಾಟ-ಬ-ಲ್ಲ
ದಾಟ-ಬಾ-ರದು
ದಾಟ-ಲಾಗದ
ದಾಟದೆ
ದಾಟಿ
ದಾಟಿ-ಕೊಂಡು
ದಾಟಿ-ದ-ವ-ರ-ನ್ನು
ದಾಟಿ-ಸು-ತ್ತಾನೆ
ದಾಟಿದ
ದಾಟು-ವರು
ದಾನ
ದಾನ-ಕ್ಕಿಂತ
ದಾನ-ಗಳಿವೆ
ದಾನ-ಪ-ತ್ರ-ವನ್ನು
ದಾನ-ಮಾಡ-ಬೇಕಾಗಿದೆ
ದಾನ-ಮಾಡ-ಬೇಕು
ದಾನ-ಮಾಡಿ
ದಾನ-ಮಾಡು
ದಾನ-ಮಾಡು-ತ್ತವೆಯೋ
ದಾನ-ಮಾಡು-ವು-ದಕ್ಕೆ
ದಾನ-ಮಾಡುವ
ದಾನ-ವನ್ನು
ದಾನ-ವಾ-ದರೋ
ದಾನ-ವಿ-ಲ್ಲ
ದಾನಿ
ದಾನಿ-ಯಾಗ-ಬೇಕೆಂದು
ದಾನಿಯ
ದಾಬೂ-ಲ್
ದಾರ-ದಂತೆ
ದಾರ-ವನ್ನು
ದಾರಿ
ದಾರಿ-ಕಾ-ಣದೆ
ದಾರಿ-ಗಳ-ಲ್ಲಿ
ದಾರಿ-ಗಳೆ-ಲ್ಲ
ದಾರಿ-ಗಾಗಿ
ದಾರಿ-ತೋ-ರುವ-ವನು
ದಾರಿ-ದ್ಯ್ರ-ದ-ಲ್ಲಿ
ದಾರಿ-ದ್ರ್ಯ-ದಿಂದ
ದಾರಿ-ದ್ರ್ಯ-ವನ್ನು
ದಾರಿ-ದ್ರ್ಯದ
ದಾರಿ-ಬಿಡ-ಬೇಕೆಂದು
ದಾರಿ-ಯ-ಲ್ಲಿ
ದಾರಿ-ಯ-ಲ್ಲಿ-ರುವುದ-ನ್ನೆ-ಲ್ಲ
ದಾರಿ-ಯ-ಲ್ಲಿದ್ದ
ದಾರಿ-ಯ-ಲ್ಲಿಯೇ
ದಾರಿ-ಯ-ಲ್ಲೆ
ದಾರಿ-ಯ-ಲ್ಲೇ
ದಾರಿ-ಯನ್ನು
ದಾರಿ-ಯೆ-ಲ್ಲ
ದಾರಿ-ಹೋಕ-ನಿಗೂ
ದಾರಿದ್ಯ್ರ
ದಾರಿದ್ರ್ಯ
ದಾರಿಯ
ದಾರಿಯೆ
ದಾರಿಯೇ
ದಾರು-ಣ-ಯಾ-ತನೆ
ದಾರು-ಣ-ವಾ-ದು-ದ-ನ್ನು
ದಾರು-ಣ-ವಾಗಿ-ತ್ತು
ದಾರು-ಣ-ವಾದ
ದಾರುಣ
ದಾರ್ಢ್ಯ-ವಿ-ದೆಯೆ
ದಾರ್ಶನಿ-ಕ-ರಾಗಿ-ದ್ದರೂ
ದಾರ್ಶನಿ-ಕ-ರಿಗೆ
ದಾರ್ಶನಿ-ಕನೂ
ದಾರ್ಶನಿ-ಕರ
ದಾರ್ಶನಿಕ
ದಾವ
ದಾವಾನಲ-ದಂತೆ
ದಾಸ
ದಾಸ-ನ-ನ್ನಾಗಿ
ದಾಸ-ನಾ-ದರೆ
ದಾಸ-ನಾಗದೆ
ದಾಸ-ನಿಗೆ
ದಾಸ-ನೇನು
ದಾಸ-ರಿಗೆ
ದಾಸರ
ದಾಸರು
ದಾಸಾನು-ದಾಸ-ರಾಗಿ
ದಾಸಿ-ಯ-ನ್ನಾಗಿ
ದಾಸಿ-ಯ-ರ-ನ್ನಾಗಿ
ದಾಸಿ-ಯ-ರ-ನ್ನು
ದಾಸ್ಯ
ದಾಹ
ದಾಹ-ವನ್ನು
ದಿ
ದಿಂಬಿನ
ದಿಂಬು-ಗ-ಳನ್ನು
ದಿಕ್ಕಿ-ನಿಂದ
ದಿಕ್ಕಿ-ಲ್ಲದ
ದಿಕ್ಕಿ-ಲ್ಲದ-ವರ
ದಿಕ್ಕಿಗೆ
ದಿಕ್ಕು
ದಿಕ್ಕು-ಗಳ-ಲ್ಲಿಯೂ
ದಿಕ್ಕು-ಗಳಿಂದ
ದಿಕ್ಕು-ಗಳೆ-ಲ್ಲವೂ
ದಿಕ್ಕು-ದಿಕ್ಕಿಗೂ
ದಿಕ್ತಟ-ಗಳಿಂದ
ದಿಕ್ತಟ-ಗ-ಳನ್ನು
ದಿಗಂ-ತಕ್ಕೆ
ದಿಗಂ-ತದ-ಲ್ಲಿ
ದಿಗಂ-ತದ-ವ-ರೆಗೆ
ದಿಗಂ-ತದಿಂದ
ದಿಗಂತ-ವನ್ನು
ದಿಗಂಬರ
ದಿಗ್ಭ್ರಮೆ
ದಿಗ್ಭ್ರಾಂತ-ನಾಗಿ
ದಿಗ್ಭ್ರಾಂತ-ನಾದೆ
ದಿಗ್ವಿಜಯ-ಗ-ಳಾದಿರಿ
ದಿಗ್ವಿಜಯಾ-ನಂ-ತರ
ದಿಗ್ವಿಜಯಿಗೂ
ದಿಣ್ಣೆ-ಗ-ಳಿದ್ದ
ದಿನ
ದಿನ-ಕರ
ದಿನ-ಕಳೆ-ದಂತೆ
ದಿನ-ಗ-ಳನ್ನು
ದಿನ-ಗ-ಳಾದ
ದಿನ-ಗ-ಳಾದ-ಮೇಲೆ
ದಿನ-ಗ-ಳಾದರೂ
ದಿನ-ಗ-ಳಾದವು
ದಿನ-ಗ-ಳಿಗೆ
ದಿನ-ಗ-ಳಿದ್ದರು
ದಿನ-ಗ-ಳಿದ್ದರೆ
ದಿನ-ಗ-ಳಿದ್ದಾದ
ದಿನ-ಗ-ಳಿದ್ದು
ದಿನ-ಗಳ
ದಿನ-ಗಳ-ಲ್ಲಿ
ದಿನ-ಗಳ-ಲ್ಲಿಯೇ
ದಿನ-ಗಳ-ಲ್ಲೊಂದು
ದಿನ-ಗಳ-ವ-ರೆಗೆ
ದಿನ-ಗಳಾ-ದು-ದ-ರಿಂದ
ದಿನ-ಗಳಿ-ಗಿಂತ
ದಿನ-ಗಳಿಂದ
ದಿನ-ಗಳಿಂದಲೂ
ದಿನ-ಗಳಿವೆ
ದಿನ-ಗಳು
ದಿನ-ಚರಿ
ದಿನ-ದ-ಲ್ಲಿ
ದಿನ-ದಂದು
ದಿನ-ದಿ-ನಕ್ಕೂ
ದಿನ-ದಿಂದ
ದಿನ-ದಿಂದಲೂ
ದಿನ-ದಿಂದಲೇ
ದಿನ-ವಂತೂ
ದಿನ-ವನ್ನು
ದಿನ-ವನ್ನೂ
ದಿನ-ವೆ-ಲ್ಲ
ದಿನ-ವೆ-ಲ್ಲಾ
ದಿನದ
ದಿನವೂ
ದಿನವೆ
ದಿನವೇ
ದಿನಾಜ್
ದಿಬ್ಬ-ಗಳ
ದಿಬ್ಬದ
ದಿವಸ
ದಿವಾ-ನರ
ದಿವಾ-ನರು
ದಿವಾ-ನರೇ
ದಿವಾ-ನ್
ದಿವಾ-ನ್ಜಿ
ದಿವಾ-ನ್ಜಿ-ಯ-ವ-ರಿಗೆ
ದಿವಾ-ನ್ಜಿ-ಯ-ವರು
ದಿವಾನ-ಖಾನೆ-ಯ-ಲ್ಲಿಯೂ
ದಿವಾನ-ರಾ-ದರೊ
ದಿವಾನ-ರಾಗಿದ್ದ
ದಿವಾನ-ರಾದ
ದಿವಾನ-ರಿಂದ
ದಿವಾನ-ರಿಗೆ
ದಿವಾನ-ರೊ-ಡನೆ
ದಿವ್ಯ
ದಿವ್ಯ-ಚಕ್ಷು-ಸ್ಸಿಗೆ
ದಿವ್ಯ-ಪ್ರ-ಭೆಯ
ದಿವ್ಯ-ವಾದ
ದಿವ್ಯ-ಶಕ್ತಿ
ದಿಶಿ
ದೀಕ್ಷಾ-ಬದ್ಧ-ರಾ-ದರು
ದೀಕ್ಷೆ
ದೀಕ್ಷೆ-ಯನ್ನು
ದೀನ
ದೀನ-ರ-ನ್ನು
ದೀನ-ರ-ಲ್ಲಿ
ದೀನ-ರಿಗೆ
ದೀನರ
ದೀನರೂ
ದೀನರೋ
ದೀನ್
ದೀಪ
ದೀಪ-ಗ-ಳನ್ನು
ದೀಪ-ಗಳ
ದೀಪ-ಗಳು
ದೀಪ-ಗಳೊಂದಿಗೆ
ದೀಪ-ದ-ಲ್ಲಿ
ದೀಪ-ದಿಂದ
ದೀಪ-ವನ್ನು
ದೀಪ-ವನ್ನೆ-ಲ್ಲ
ದೀಪಕ್ಕೆ
ದೀಪದ
ದೀಪಾಲಂ-ಕಾರ-ವನ್ನು
ದೀಪಾವಳಿಯ
ದೀಪೋದ್ಯಾನ-ದ-ಲ್ಲಿ
ದೀಪ್ತಿ-ಗೊಂಡು
ದೀರ್ಘ
ದೀರ್ಘ-ಕಾಲ
ದೀರ್ಘ-ಕಾಲದ
ದೀರ್ಘ-ದೃಷ್ಟಿ-ಯಿಂದ
ದೀರ್ಘ-ಧ್ಯಾನ-ದ-ಲ್ಲಿ
ದೀರ್ಘ-ಧ್ಯಾನ-ದ-ಲ್ಲಿಯೂ
ದೀರ್ಘ-ನಿದ್ರೆ
ದೀರ್ಘ-ಲಿಪಿ-ಯ-ಲ್ಲಿಯೂ
ದೀರ್ಘ-ವಾಗಿ
ದೀರ್ಘ-ವಾದ
ದೀರ್ಘಾಲೋ-ಚನೆ
ದು
ದುಂದುಭಿ-ಯನ್ನು
ದುಂಬಿ
ದುಂಬಿ-ಗಳು
ದುಂಬಿ-ಯಂತೆ
ದುಃ
ದುಃಖ
ದುಃಖ-ಕರ-ವಾದ
ದುಃಖ-ಕ್ಕಾಗಿ
ದುಃಖ-ಗ-ಳಿಗೆ
ದುಃಖ-ತಾಡಿತ
ದುಃಖ-ದ-ಲ್ಲಿ
ದುಃಖ-ದಿಂದ
ದುಃಖ-ದಿಂದಲೂ
ದುಃಖ-ಪಡು-ವರೊ
ದುಃಖ-ಪಡು-ವು-ದಿ-ಲ್ಲ
ದುಃಖ-ಭಾ-ಜನ-ರಾ-ದು-ದ-ರಿಂದ
ದುಃಖ-ಮಯ
ದುಃಖ-ವನ್ನು
ದುಃಖ-ವನ್ನೆ-ಲ್ಲ
ದುಃಖ-ವನ್ನೇ
ದುಃಖ-ವಿರ-ಬ-ಲ್ಲದು
ದುಃಖ-ವೆ-ಲ್ಲ
ದುಃಖಕ್ಕೆ
ದುಃಖದ
ದುಃಖನೀ
ದುಃಖವೂ
ದುಃಖವೇ
ದುಃಖಾ-ತ್ಮರ
ದುಃಖಿ
ದುಃಖಿ-ಗಳ
ದುಃಖಿ-ಗಳೋ
ದುಃಖಿ-ತ-ರಾ-ದರು
ದುಃಖಿ-ಸದ
ದುಗು-ಡದ
ದುಗುಡ-ವನ್ನು
ದುಡಿ-ತ-ದಿಂದ
ದುಡಿ-ತಕ್ಕೆ
ದುಡಿ-ದರೂ
ದುಡಿ-ದರೆ
ದುಡಿ-ಯಲು
ದುಡಿ-ಯಿರಿ
ದುಡಿ-ಯು-ವುದು
ದುಡಿದು
ದುಡಿಯ-ಬೇಕು
ದುಡಿಯ-ಬೇಕೆಂದು
ದುಡಿಯು-ತ್ತಲೆ
ದುಡಿಸಿ-ಕೊಂಡರು
ದುಡ್ಡ-ನ್ನು
ದುಡ್ಡ-ನ್ನೆ-ಲ್ಲ
ದುಡ್ಡಿ-ನಿಂದ-ಲೇದುಡ್ಡಿ-ಲ್ಲದೆ
ದುಡ್ಡಿಗೆ
ದುಡ್ಡು
ದುರ-ದೃ-ಷ್ಟಕ್ಕೆ
ದುರ-ದೃಷ್ಟ
ದುರ-ದೃಷ್ಟ-ವಶ-ದಿಂದ
ದುರ-ದೃಷ್ಟ-ವಶಾ-ತ್
ದುರ-ಭ್ಯಾಸ
ದುರಭಿ-ಮಾನ
ದುರಾ-ಚಾರಿ-ಗಳ-ನ್ನಾಗಿ
ದುರಾ-ತ್ಮ-ನಾ-ದರೂ
ದುರುಗು-ಟ್ಟಿ-ಕೊಂಡು
ದುರ್ಗಂಧ
ದುರ್ಗಂಧ-ವನ್ನು
ದುರ್ಗತಿಂ
ದುರ್ಗಾ
ದುರ್ಗಾ-ಚರಣ-ದ-ತ್ತ
ದುರ್ಗಾ-ಚರಣ-ದ-ತ್ತ-ನಿಗೆ
ದುರ್ಗಾ-ಚರಣ-ದ-ತ್ತನ
ದುರ್ಗಾ-ಚರಣ-ನಾ-ದರೊ
ದುರ್ಗಾ-ಚರಣನ
ದುರ್ಗಾ-ಚರಣನು
ದುರ್ಗಾ-ಪೂಜೆ-ಯನ್ನು
ದುರ್ಜ್ಞೇಯ-ವಾದ
ದುರ್ದ-ಮ್ಯ
ದುರ್ದೆಸೆ-ಯನ್ನು
ದುರ್ದೈವ-ವಷ್ಟೆ
ದುರ್ಬ-ಲರು
ದುರ್ಬಲ
ದುರ್ಬಲ-ಗೊಳಿ-ಸುವ
ದುರ್ಬಲ-ತೆ-ಯನ್ನು
ದುರ್ಬಲ-ನಾಗಿ-ರುವೆ
ದುರ್ಬಲ-ರ-ನ್ನು
ದುರ್ಬಲ-ರ-ಲ್ಲಿ
ದುರ್ಬಲ-ವಾ-ಗಿದೆ
ದುರ್ಬಲ-ವಾಗಲು
ದುರ್ಬಲ-ವಾಗಿ
ದುರ್ಬಲ-ವಾಗಿ-ದ್ದರೆ
ದುರ್ಬಲ-ವಾಗಿ-ರು-ವು-ದೆ-ಲ್ಲ
ದುರ್ಬಲ-ವಾಗು-ತ್ತಿದ್ದರೂ
ದುರ್ಬಲ-ವಾದ
ದುರ್ಬಲತೆ
ದುರ್ಬಲನ
ದುರ್ಲಭ-ವಾದ
ದುಶ್ಚರಿ-ತ್ರ-ವಾಗಲಿ
ದುಷ್ಕಾಲ-ದ-ಲ್ಲಿ
ದುಷ್ಟ
ದುಷ್ಟ-ನನ್ನು
ದೂ
ದೂತ-ನ-ನ್ನಾಗಿ
ದೂತನು
ದೂರ
ದೂರ-ಕೂ-ಡದು
ದೂರ-ತ್ತಾರೊ
ದೂರ-ದ-ಲ್ಲಿ
ದೂರ-ದ-ಲ್ಲಿ-ದ್ದು-ದ-ರಿಂದ
ದೂರ-ದ-ಲ್ಲಿ-ರು-ವನು
ದೂರ-ದ-ಲ್ಲಿ-ರುವ
ದೂರ-ದ-ಲ್ಲಿದೆ
ದೂರ-ದ-ಲ್ಲಿದ್ದ
ದೂರ-ದ-ವ-ರೆಗೂ
ದೂರ-ದ-ವ-ರೆಗೆ
ದೂರ-ದರ್ಶಕ
ದೂರ-ದಿಂದ
ದೂರ-ದೂರದ
ದೂರ-ದೇ-ಶಕ್ಕೆ
ದೂರ-ದೇಶ-ದಿಂದ
ದೂರ-ಬೇಡಿ
ದೂರ-ಲಿ-ಲ್ಲ
ದೂರ-ವಾ-ದುದು
ದೂರ-ವಾಗಿ
ದೂರ-ವಾಗಿದ್ದ
ದೂರ-ವಾದ
ದೂರ-ವಿ-ರ-ಲಿ-ಲ್ಲ
ದೂರ-ವಿ-ರುವ
ದೂರ-ವಿದೆ
ದೂರ-ವಿರ-ಬೇಕು
ದೂರ-ಹೋದರೂ
ದೂರದ
ದೂರದೆ
ದೂರಿ
ದೂರಿ-ದರು
ದೂರಿ-ದರೂ
ದೂರಿ-ದರೆ
ದೂರು-ತ್ತೀರಿ
ದೂರುವ
ದೂರುವ-ವ-ರಿಗೆ
ದೃಡಿಷ್ಠನೂ
ದೃಢ
ದೃಢ-ಕಾ-ಯರು
ದೃಢ-ಕಾಯ-ವಾದ
ದೃಢ-ಚಿ-ತ್ತ
ದೃಢ-ನಂಬಿಕೆ
ದೃಢ-ಪಡಿಸಿ
ದೃಢ-ಮನ-ಸ್ಸಿ-ನಿಂದ
ದೃಢ-ಮಾಡಿ-ಕೊಂ-ದಿದ್ದರು
ದೃಢ-ವಾಗಿ
ದೃಢ-ವಾಗಿ-ತ್ತು
ದೃಢ-ವಾಗಿ-ಲ್ಲ
ದೃಢ-ವಾಗು-ತ್ತದೆ
ದೃಢ-ವಾದ
ದೃಢ-ಸಂ-ಕ-ಲ್ಪ
ದೃಢೀಕ-ರಿಸಿ-ದವು
ದೃಶ್ಯ
ದೃಶ್ಯ-ಗ-ಳನ್ನು
ದೃಶ್ಯ-ಗಳ
ದೃಶ್ಯ-ಗಳ-ನ್ನೆ-ಲ್ಲ
ದೃಶ್ಯ-ಗಳು
ದೃಶ್ಯ-ದ-ಲ್ಲಿ
ದೃಶ್ಯ-ದಂತೆ
ದೃಶ್ಯ-ವನ್ನು
ದೃಶ್ಯ-ವಾಗಲಿ
ದೃಶ್ಯ-ವಿ-ಲ್ಲ
ದೃಷ್ಟ
ದೃಷ್ಟಂ
ದೃಷ್ಟಾಂತ
ದೃಷ್ಟಾಂತವು
ದೃಷ್ಟಿ
ದೃಷ್ಟಿ-ಕೋ-ನ-ದಿಂದ
ದೃಷ್ಟಿ-ಕೋಣ-ದಿಂದ
ದೃಷ್ಟಿ-ಕೋನ-ಗಳಿಂದ
ದೃಷ್ಟಿ-ಕೋನ-ಗಳಿಂದಲೂ
ದೃಷ್ಟಿ-ಗ-ಳನ್ನು
ದೃಷ್ಟಿ-ಗಳ-ನ್ನೂ
ದೃಷ್ಟಿ-ಗಳ-ಲ್ಲಿ
ದೃಷ್ಟಿ-ಗಳಿಂದ
ದೃಷ್ಟಿ-ಯ-ಲ್ಲಿ
ದೃಷ್ಟಿ-ಯ-ಲ್ಲಿ-ರ-ಬೇಕು
ದೃಷ್ಟಿ-ಯನ್ನು
ದೃಷ್ಟಿ-ಯನ್ನೆ
ದೃಷ್ಟಿ-ಯನ್ನೇ
ದೃಷ್ಟಿ-ಯಾ-ದರೋ
ದೃಷ್ಟಿ-ಯಿ-ಲ್ಲ
ದೃಷ್ಟಿ-ಯಿಂದ
ದೃಷ್ಟಿ-ಯಿಂದ-ಲ್ಲ
ದೃಷ್ಟಿ-ಯಿಂದಲೂ
ದೃಷ್ಟಿ-ಯಿಂದಲೇ
ದೃಷ್ಟಿ-ಯಿಂದಲೋ
ದೃಷ್ಟಿ-ಯಿಂದಾಚೆಗೆ
ದೃಷ್ಟಿ-ಯೊಂದು
ದೃಷ್ಟಿಗೂ
ದೃಷ್ಟಿಗೆ
ದೃಷ್ಯ-ಗಳು
ದೆವ್ವ
ದೆವ್ವ-ವೆಂದು
ದೆವ್ವದ
ದೆಸೆ-ಯ-ಲ್ಲಿದ್ದ
ದೆಸೆ-ಯ-ಲ್ಲಿರುವ
ದೆಸೆ-ಯಿಂದ
ದೆಹ-ಲಿಗೆ
ದೆಹ-ಲಿಯ
ದೆಹ-ಲಿಯ-ಲ್ಲಿ-ರುವ
ದೆಹಲಿ-ಯ-ಲ್ಲಿ
ದೆಹಲಿ-ಯನ್ನು
ದೇ
ದೇಗುಲ-ದ-ಲ್ಲೇ
ದೇಗುಲ-ದಂತೆ
ದೇವ
ದೇವ-ಘ-ರಕ್ಕೆ
ದೇವ-ತಾ-ತ್ಮ
ದೇವ-ತಾ-ತ್ಮಾ
ದೇವ-ತಾ-ಸ್ವ-ರೂಪ-ವಾದ
ದೇವ-ತೆ-ಗ-ಳನ್ನು
ದೇವ-ತೆ-ಗಳ
ದೇವ-ತೆ-ಗಳ-ನ್ನೂ
ದೇವ-ತೆ-ಗಳು
ದೇವ-ತೆ-ಗಳೂ
ದೇವ-ತೆ-ಗಳೆ-ಲ್ಲ
ದೇವ-ತೆ-ಗಳೆ-ಲ್ಲಾ
ದೇವ-ತೆ-ಗಳೇ
ದೇವ-ತೆ-ಗಿಂತ
ದೇವ-ತೆಯ
ದೇವ-ತೆಯೂ
ದೇವ-ತ್ವಕ್ಕೆ
ದೇವ-ತ್ವವೇ
ದೇವ-ದಾರು
ದೇವ-ದೂತ
ದೇವ-ದೂತ-ನನ್ನೂ
ದೇವ-ದೂತ-ನಿಗೂ
ದೇವ-ದೇವಿ-ಯ-ವರ
ದೇವ-ದೇವಿ-ಯರ
ದೇವ-ದೇವಿ-ಯರು
ದೇವ-ಮಾನವ
ದೇವ-ರ-ನ್ನು
ದೇವ-ರ-ನ್ನೂ
ದೇವ-ರ-ನ್ನೇ
ದೇವ-ರ-ಮ-ನೆಗೆ
ದೇವ-ರ-ಲ್ಲ
ದೇವ-ರ-ಲ್ಲದೆ
ದೇವ-ರ-ಲ್ಲಿ
ದೇವ-ರಂತೆ
ದೇವ-ರಂತೆಯೇ
ದೇವ-ರಾಗಿ-ರುವನೋ
ದೇವ-ರಾಗೋಣ
ದೇವ-ರಾದ
ದೇವ-ರಿ-ಗಾಗಿ
ದೇವ-ರಿ-ರು-ವನು
ದೇವ-ರಿ-ಲ್ಲ
ದೇವ-ರಿ-ಲ್ಲದ
ದೇವ-ರಿಂದ
ದೇವ-ರಿಗೂ
ದೇವ-ರಿಗೆ
ದೇವ-ರಿಗೇ
ದೇವ-ರಿರು-ವನೆ
ದೇವ-ರು-ಗ-ಳನ್ನು
ದೇವ-ರು-ಗಳ
ದೇವ-ರು-ಗಳ-ಲ್ಲಿ
ದೇವ-ರೆ-ಡೆಗೆ
ದೇವ-ರೆಂದು
ದೇವ-ರೆಡೆ-ಯ-ಲ್ಲಿ
ದೇವ-ರೊಂದಿಗೆ
ದೇವ-ರೊಬ್ಬನು
ದೇವ-ಸಂತಾ-ನರು
ದೇವ-ಸ್ಥಾನ
ದೇವ-ಸ್ಥಾನ-ಗ-ಳನ್ನು
ದೇವ-ಸ್ಥಾನ-ಗಳ-ನ್ನೆ-ಲ್ಲ
ದೇವ-ಸ್ಥಾನ-ಗಳು
ದೇವ-ಸ್ಥಾನ-ದ-ಲ್ಲಿ
ದೇವ-ಸ್ಥಾನ-ದ-ವರು
ದೇವ-ಸ್ಥಾನ-ದಿಂದ
ದೇವ-ಸ್ಥಾನ-ವನ್ನು
ದೇವ-ಸ್ಥಾನ-ವಿ-ತ್ತು
ದೇವ-ಸ್ಥಾನ-ವಿ-ರು-ವುದು
ದೇವ-ಸ್ಥಾನ-ವಿದೆ
ದೇವ-ಸ್ಥಾನಕ್ಕೆ
ದೇವ-ಸ್ಥಾನದ
ದೇವ-ಸ್ಥಾನವು
ದೇವತೆ
ದೇವನ
ದೇವರ
ದೇವರು
ದೇವರೇ
ದೇವಾಲ-ಯದ
ದೇವಾಲಯ
ದೇವಾಲಯ-ಗ-ಳನ್ನು
ದೇವಾಲಯ-ಗ-ಳಿಗೆ
ದೇವಾಲಯ-ಗಳ-ನ್ನೂ
ದೇವಾಲಯ-ಗಳಿಂದ
ದೇವಾಲಯ-ವನ್ನು
ದೇವಾಲಯಕ್ಕೆ
ದೇವಿ
ದೇವಿ-ಯ-ರ-ಲ್ಲೆ
ದೇವಿ-ಯ-ಲ್ಲಿ
ದೇವಿ-ಯನ್ನು
ದೇವಿ-ಯರ
ದೇವಿ-ಯರು
ದೇವಿ-ಯಾ-ದರೊ
ದೇವಿ-ಯೊಬ್ಬಳೇ
ದೇವಿಯ
ದೇವೇಂದ್ರ-ನಾಥ
ದೇವೇಂದ್ರ-ನಾಥರು
ದೇಶ
ದೇಶ-ಕಾಲ
ದೇಶ-ಕಾಲ-ನಿಮಿ-ತ್ತಾ-ತೀತ-ವಾದ
ದೇಶ-ಕ್ಕಾಗಿ
ದೇಶ-ಕ್ಕಿಂತ
ದೇಶ-ಗ-ಳನ್ನು
ದೇಶ-ಗ-ಳಿಗೆ
ದೇಶ-ಗಳ
ದೇಶ-ಗಳ-ಲ್ಲಿ
ದೇಶ-ಗಳ-ಲ್ಲಿಯೂ
ದೇಶ-ಗಳಿಂದ
ದೇಶ-ಗಳಿಂದಲೂ
ದೇಶ-ಗಳು
ದೇಶ-ಗಳೊ-ಡನೆ
ದೇಶ-ದ-ಲ್ಲಾ-ದರೋ
ದೇಶ-ದ-ಲ್ಲಿ
ದೇಶ-ದ-ಲ್ಲಿ-ದ್ದಾಗ
ದೇಶ-ದ-ಲ್ಲಿ-ರಲಿ
ದೇಶ-ದ-ಲ್ಲಿ-ರು-ವಷ್ಟು
ದೇಶ-ದ-ಲ್ಲಿ-ರುವ
ದೇಶ-ದ-ಲ್ಲಿಯೂ
ದೇಶ-ದ-ಲ್ಲಿಯೇ
ದೇಶ-ದ-ಲ್ಲೂ
ದೇಶ-ದ-ಲ್ಲೆ-ಲ್ಲ
ದೇಶ-ದ-ಲ್ಲೇ
ದೇಶ-ದ-ವ-ರ-ನ್ನು
ದೇಶ-ದ-ವ-ರಿಗೆ
ದೇಶ-ದ-ವನು
ದೇಶ-ದ-ವರು
ದೇಶ-ದಂತೆ
ದೇಶ-ದಲಿ
ದೇಶ-ದಿಂದ
ದೇಶ-ಪ್ರೇಮ
ದೇಶ-ಪ್ರೇಮದ
ದೇಶ-ಭಕ್ತ-ರಾಗು-ವು-ದಕ್ಕೆ
ದೇಶ-ಭಕ್ತರೆ
ದೇಶ-ಭಕ್ತಿ
ದೇಶ-ಭಕ್ತಿ-ಯ-ಲ್ಲಿ
ದೇಶ-ಭಕ್ತಿ-ಯನ್ನು
ದೇಶ-ಭಕ್ತಿ-ಯೆ-ಲ್ಲಾ
ದೇಶ-ಭಕ್ತಿ-ಯೆ-ಲ್ಲಿ
ದೇಶ-ಭಕ್ತಿಯ
ದೇಶ-ಭಾಷೆ-ಗಳ
ದೇಶ-ಭಾಷೆ-ಯ-ಲ್ಲಿಯೂ
ದೇಶ-ಭಾಷೆಯ
ದೇಶ-ಮಾ-ತ್ರ
ದೇಶ-ವನ್ನು
ದೇಶ-ವನ್ನೆ-ಲ್ಲ
ದೇಶ-ವನ್ನೆ-ಲ್ಲಾ
ದೇಶ-ವಾಗಿ
ದೇಶ-ವಿದೆ
ದೇಶ-ವೆ-ಲ್ಲ
ದೇಶ-ವೆ-ಲ್ಲಾ
ದೇಶಕ್ಕೂ
ದೇಶದ
ದೇಶವೂ
ದೇಶವೇ
ದೇಶವೋ
ದೇಶಾ-ಚಾರ
ದೇಶಾ-ತೀತ-ವಾ-ದುದು
ದೇಶಿ-ಯರು
ದೇಶಿಯ-ರಂತೆಯೇ
ದೇಶೀಯ
ದೇಶೀಯ-ನಿಗೆ
ದೇಶೀಯ-ರಿಗೂ
ದೇಶೀಯ-ರಿಗೆ
ದೇಶೀಯನು
ದೇಶೀಯರ
ದೇಶೀಯರು
ದೇಶೀಯರೆ
ದೇಶ್ದ-ಲ್ಲಿ
ದೇಹ
ದೇಹ-ಗಳಂತೆ
ದೇಹ-ದ-ಮೇಲೆ
ದೇಹ-ದ-ಲ್ಲಿ
ದೇಹ-ದ-ಲ್ಲಿ-ರುವ
ದೇಹ-ದ-ಲ್ಲಿ-ರುವು-ದೆಂದೂ
ದೇಹ-ದ-ಲ್ಲೆ-ಲ್ಲಾ
ದೇಹ-ದ-ಲ್ಲೇ
ದೇಹ-ದಿಂದ
ದೇಹ-ಧಾರಣೆ
ದೇಹ-ಪ್ರವೇಶ
ದೇಹ-ಭಾ-ವನೆ
ದೇಹ-ವ-ಲ್ಲ
ದೇಹ-ವನ್ನು
ದೇಹ-ವನ್ನೇ
ದೇಹ-ವಿ-ರು-ವುದೋ
ದೇಹ-ವಿ-ರುವ-ವ-ರೆಗೂ
ದೇಹ-ವೆಂಬ
ದೇಹ-ಶ್ರಮ-ಪಡು
ದೇಹಕ್ಕೆ
ದೇಹದ
ದೇಹವೂ
ದೇಹವೆ
ದೇಹವೇ
ದೇಹಾ-ತ್ಮ-ವಾದಿ-ಗಳೋ
ದೇಹಾದ್ಯಂತವೂ
ದೇಹಾಭಿ-ಮಾನ-ವ-ನ್ನೆ-ಲ್ಲ
ದೇಹಾವಸಾನ-ವಾದ
ದೇಹಿ
ದೈ
ದೈವ-ತ್ವದ
ದೈವ-ಭಕ್ತಿ
ದೈವ-ವನ್ನು
ದೈವ-ವಶಾ-ತ್
ದೈವ-ಸಾಕ್ಷಾ-ತ್ಕಾರ-ವಾಗದ
ದೈವದಂತಿ-ಗಳು
ದೈವೀ
ದೈವೀ-ಭಾವ-ದ-ಲ್ಲಿ
ದೈವೀ-ಶಕ್ತಿ
ದೈವೀ-ಶಕ್ತಿ-ಯೆಂದು
ದೈವೀ-ಶಕ್ತಿಯ
ದೈಹಿಕ
ದೈಹಿಕ-ವಾಗಿ
ದೊ
ದೊಂ
ದೊಂಬಿ
ದೊಡ್ಡ
ದೊಡ್ಡ-ದ-ಲ್ಲ
ದೊಡ್ಡ-ದಾಗಿ
ದೊಡ್ಡ-ದಾಗಿ-ದ್ದುದು
ದೊಡ್ಡ-ದಾಗಿ-ರ-ಲಿ-ಲ್ಲ
ದೊಡ್ಡ-ದಾಗಿ-ರು-ತ್ತಿ-ತ್ತು
ದೊಡ್ಡ-ದೊಂದು
ದೊಡ್ಡ-ಮನು-ಷ್ಯರು
ದೊಡ್ಡ-ವ-ನಾ-ದರೆ
ದೊಡ್ಡ-ವ-ನಾಗಿ-ದ್ದರೂ
ದೊಡ್ಡ-ವ-ರಾದ
ದೊಡ್ಡ-ವ-ರಿಗೆ
ದೊಡ್ಡ-ವ-ರೆ-ಲ್ಲ
ದೊಡ್ಡ-ವನಾಗ-ಬೇಕೆಂದು
ದೊಡ್ಡ-ವನು
ದೊಡ್ಡ-ವರಾಗ-ಕೂ-ಡದು
ದೊಡ್ಡ-ವರಾಗು-ತ್ತಾರೆ
ದೊಡ್ಡ-ವರೇ
ದೊಡ್ಡ-ಶಿ-ಲ್ಪಿ
ದೊಡ್ಡ-ಸ್ತಿಕೆ
ದೊಡ್ಡ-ಸ್ತಿಕೆ-ಯನ್ನು
ದೊಡ್ಡದು
ದೊಡ್ಡದೆ
ದೊಡ್ಡದೇ
ದೊಡ್ಡದೋ
ದೊಡ್ಡಾ
ದೊಡ್ಡಿ-ಗ-ಳನ್ನು
ದೊಡ್ದ
ದೊಣಿ
ದೊಣ್ಣೆ
ದೊರ-ಕದೆ
ದೊರ-ಕದೊ
ದೊರ-ಕಲಾ-ರದು
ದೊರ-ಕಲಾ-ರರು
ದೊರಕ-ದುದೇ
ದೊರಕಿ-ದಂ-ತಾಗಿದೆ
ದೊರಕಿ-ದವು
ದೊರಕಿ-ದಾಗ
ದೊರಕಿ-ಲ್ಲ
ದೊರಕಿತು
ದೊರಕಿದ
ದೊರಕಿದೆ
ದೊರಕು-ತ್ತದೆ
ದೊರಕು-ತ್ತವೆ
ದೊರಕು-ವು-ದ-ರಿಂದ
ದೊರಕು-ವು-ದಿ-ಲ್ಲ-ವೆಂದು
ದೊರಕು-ವು-ದೆಂದು
ದೊರಕು-ವುದು
ದೊರಕು-ವುದೆ
ದೊರಕು-ವುದೇ
ದೊರಕುವ-ವ-ರೆಗೂ
ದೊರಕುವ-ವ-ರೆಗೆ
ದೊರಕುವುದ-ರ-ಲ್ಲಿ
ದೊರೆ-ಗ-ಳಾದ
ದೊರೆ-ಗಳೇ
ದೊರೆ-ತರು
ದೊರೆ-ತರೂ
ದೊರೆ-ತರೆ
ದೊರೆ-ಯ-ಲಿ-ಲ್ಲ
ದೊರೆ-ಯದೇ
ದೊರೆ-ಯನ್ನು
ದೊರೆ-ಯಿತು
ದೊರೆ-ಯು-ತ್ತದೆ
ದೊರೆ-ಯು-ವು-ದಿ-ಲ್ಲ
ದೊರೆ-ಯು-ವು-ದಿ-ಲ್ಲವೋ
ದೊರೆ-ಯು-ವುದು
ದೊರೆ-ಯು-ವುವು
ದೊರೆತ
ದೊರೆತಿದೆ
ದೊರೆತು-ದಾಗಿ
ದೊರೆತುದು
ದೋ
ದೋಣಿ
ದೋಣಿ-ಗ-ಳನ್ನು
ದೋಣಿ-ಗಳ
ದೋಣಿ-ಗಳು
ದೋಣಿ-ಗಳೆಷ್ಟು
ದೋಣಿ-ಯ-ನ್ನೆ-ಲ್ಲ
ದೋಣಿ-ಯ-ಲ್ಲಿ
ದೋಣಿ-ಯ-ಲ್ಲಿ-ದ್ದನು
ದೋಣಿ-ಯ-ಲ್ಲೆ
ದೋಣಿ-ಯ-ಲ್ಲೇ
ದೋಣಿ-ಯ-ವ-ನನ್ನು
ದೋಣಿ-ಯ-ವ-ನಿಗೆ
ದೋಣಿ-ಯ-ವನ
ದೋಣಿ-ಯ-ವನು
ದೋಣಿ-ಯ-ವರು
ದೋಣಿ-ಯನ್ನು
ದೋಣಿ-ಯಿಂದ
ದೋಣಿ-ಯೊಳಗೆ
ದೋಣಿಗೆ
ದೋಷ
ದೋಷ-ಗ-ಳನ್ನು
ದೋಷ-ಗಳಿವೆ
ದೋಷ-ವನ್ನು
ದೋಷ-ವೆಂದು
ದೋಷದ
ದೋಷವ
ದೋಷವೂ
ದೌರಾ-ತ್ಮ್ಯ
ದೌರ್ಜನ್ಯ-ದಿಂದ
ದೌರ್ಜನ್ಯ-ವಾ-ಗಿದೆ
ದೌರ್ಬ-ಲ್ಯ
ದೌರ್ಬ-ಲ್ಯ-ಗ-ಳನ್ನು
ದೌರ್ಬ-ಲ್ಯ-ಗಳೆ-ಲ್ಲಾ
ದೌರ್ಬ-ಲ್ಯ-ದಿಂದ
ದೌರ್ಬ-ಲ್ಯ-ವನ್ನು
ದೌರ್ಬ-ಲ್ಯ-ವಿ-ತ್ತು
ದೌರ್ಬ-ಲ್ಯ-ವಿ-ರ-ಲಿ-ಲ್ಲ
ದೌರ್ಬ-ಲ್ಯ-ವೆ-ಲ್ಲ
ದೌರ್ಬ-ಲ್ಯ-ವೆಂತಲೇ
ದೌರ್ಬ-ಲ್ಯ-ವೇನು
ದೌರ್ಬ-ಲ್ಯದ
ದೌರ್ಬ-ಲ್ಯವೂ
ದೌರ್ಬ-ಲ್ಯವೇ
ದೌರ್ಭಾಗ್ಯ-ಕ್ಕೆ-ಲ್ಲ
ದ್ರವ್ಯ
ದ್ರವ್ಯ-ಗಳ-ಲ್ಲಿ
ದ್ರವ್ಯ-ಗಳಿಂದ
ದ್ರವ್ಯ-ರೂಪ-ದ-ಲ್ಲಿ
ದ್ರವ್ಯ-ವನ್ನು
ದ್ರವ್ಯ-ಸಹಾಯ
ದ್ರವ್ಯ-ಸಹಾಯ-ವನ್ನೂ
ದ್ರವ್ಯದ
ದ್ರವ್ಯಾರ್ಜನೆ
ದ್ರಷ್ಟರು
ದ್ರಾಕ್ಷಿ
ದ್ರಾಕ್ಷಿಯ
ದ್ರು
ದ್ರೋಹ-ವನ್ನು
ದ್ರೋಹಿ-ಗಳಾರೂ
ದ್ವಂದ್ವ-ಗಳ-ಲ್ಲಿ
ದ್ವಂದ್ವ-ಗಳಿಂದ
ದ್ವಾರ
ದ್ವಾರ-ಕ-ದಾಸರ
ದ್ವಾರ-ಕ-ದಾಸರು
ದ್ವಾರ-ಕೆ-ಯ-ಲ್ಲಿ
ದ್ವಾರ-ಕೆ-ಯಿಂದ
ದ್ವಾರ-ಕೆಗೆ
ದ್ವಾರ-ಗಳಿವೆ
ದ್ವಾರಕೆ
ದ್ವಾರದ
ದ್ವಿ
ದ್ವಿಜ-ರಿ-ಗೆ-ಲ್ಲ
ದ್ವಿಜ-ರಿಗೆ
ದ್ವೀ
ದ್ವೀಪ
ದ್ವೀಪ-ಗಳಿವೆ
ದ್ವೀಪ-ಗಳು
ದ್ವೀಪ-ದ-ಲ್ಲಿ
ದ್ವೀಪ-ದ-ಲ್ಲಿ-ರುವೆ
ದ್ವೀಪ-ದ-ಲ್ಲಿದ್ದ
ದ್ವೀಪ-ದಂತೆ
ದ್ವೀಪ-ದಂತೆಯೇ
ದ್ವೀಪ-ವನ್ನು
ದ್ವೀಪ-ವೆಂದೂ
ದ್ವೀಪೋ-ಧ್ಯಾನ-ದ-ಲ್ಲಿ
ದ್ವೀಪೋದ್ಯಾ-ನಕ್ಕೆ
ದ್ವೀಪೋದ್ಯಾನ-ದ-ಲ್ಲಿ
ದ್ವೀಪೋದ್ಯಾನದ
ದ್ವೇಷ-ಗಳ-ಲ್ಲಿ
ದ್ವೇಷ-ದಿಂದಲೋ
ದ್ವೇಷ-ಪರಾ-ಯಣ-ರ-ಲ್ಲ
ದ್ವೇಷದ
ದ್ವೇಷವೇ
ದ್ವೇಷಿ-ಸು-ವು-ದಕ್ಕೆ
ದ್ವೇಷಿ-ಸು-ವು-ದಿ-ಲ್ಲ
ದ್ವೇಷಿ-ಸು-ವೆನು
ದ್ವೇಷಿಸ-ಬಾ-ರದು
ದ್ವೇಷಿಸಿ
ದ್ವೇಷಿಸಿ-ದೆವೊ
ದ್ವೇಷಿಸಿಯೇ
ದ್ವೈತ
ದ್ವೈತಿ-ಗಳು
ದ್ವೈತಿ-ಗಳೆ
ಧ
ಧಗಧಗಿ-ಸು-ತ್ತಿದೆ
ಧನ-ಕನ-ಕಾದಿ
ಧನ-ವನ್ನು
ಧನಿ
ಧನ್ಯ
ಧನ್ಯ-ನಾಗಿ-ದ್ದೇನೆ
ಧನ್ಯ-ನಾಗಿ-ರು-ವನು
ಧನ್ಯ-ನಾದೆ
ಧನ್ಯ-ನೆಂದು
ಧನ್ಯ-ರ-ನ್ನಾಗಿ
ಧನ್ಯ-ರಾ-ದರು
ಧನ್ಯ-ರಾಗದ-ವರ
ಧನ್ಯ-ರಾಗಿ
ಧನ್ಯ-ರಾಗಿ-ಬಿಡು-ತ್ತಿದ್ದೆವು
ಧನ್ಯ-ರಾಗು-ತ್ತಿದ್ದರು
ಧನ್ಯ-ರಾದ
ಧನ್ಯ-ವಾ-ಯಿತು
ಧನ್ಯ-ವಾಗು-ವುದು
ಧನ್ಯ-ವಾದ
ಧನ್ಯ-ವಾದ-ಗಳು
ಧನ್ಯ-ವಾದ-ವನ್ನು
ಧನ್ಯರು
ಧನ್ಯಳು
ಧನ್ಯಾ-ತ್ಮ-ಳಾವಳೊ
ಧರಿ-ಸಿತು
ಧರಿ-ಸಿದ್ದ
ಧರಿ-ಸು-ತ್ತಾನೆ
ಧರಿ-ಸು-ತ್ತೀರಿ
ಧರಿ-ಸು-ತ್ತೇನೆ
ಧರಿ-ಸು-ವು-ದ-ರಿಂದ
ಧರಿ-ಸುವಾಗ
ಧರಿಸು-ವು-ದ-ಲ್ಲ
ಧರೆ-ಗಿ-ಳಿದು
ಧರ್ಮ
ಧರ್ಮ-ಕಾರಣಂ
ಧರ್ಮ-ಕಾರ್ಯ-ಗಳ-ಲ್ಲಿ
ಧರ್ಮ-ಕ್ಕಾಗಿ
ಧರ್ಮ-ಕ್ಷೇ-ತ್ರ
ಧರ್ಮ-ಗ-ಳನ್ನು
ಧರ್ಮ-ಗ-ಳಿಗೆ
ಧರ್ಮ-ಗಳ
ಧರ್ಮ-ಗಳ-ನ್ನೂ
ಧರ್ಮ-ಗಳ-ಲ್ಲಿ
ಧರ್ಮ-ಗಳ-ಲ್ಲಿಯೂ
ಧರ್ಮ-ಗಳಾವುವೂ
ಧರ್ಮ-ಗಳಿ-ಗೆ-ಲ್ಲಾ
ಧರ್ಮ-ಗಳಿಗೂ
ಧರ್ಮ-ಗಳು
ಧರ್ಮ-ಗಳೂ
ಧರ್ಮ-ಗಳೆ-ದು-ರಿಗೆ
ಧರ್ಮ-ಗಳೆ-ಲ್ಲ
ಧರ್ಮ-ಗಳೆ-ಲ್ಲಾ
ಧರ್ಮ-ಗಳೇ
ಧರ್ಮ-ಗ್ರಂಥ
ಧರ್ಮ-ಗ್ರಂಥ-ಗ-ಳನ್ನು
ಧರ್ಮ-ಗ್ರಂಥ-ಗಳ
ಧರ್ಮ-ಗ್ರಂಥ-ಗಳ-ನ್ನೇ
ಧರ್ಮ-ಗ್ರಂಥ-ಗಳ-ಲ್ಲಿ
ಧರ್ಮ-ಗ್ರಂಥ-ಗಳ-ಲ್ಲಿ-ರುವ
ಧರ್ಮ-ಗ್ರಂಥ-ಗಳ-ಲ್ಲೂ
ಧರ್ಮ-ಘರ್
ಧರ್ಮ-ಜಾಗ್ರ-ತಿಯ
ಧರ್ಮ-ಜಾಗ್ರತ-ವಾ-ದರೆ
ಧರ್ಮ-ಜ್ಯೋತಿ
ಧರ್ಮ-ತ-ತ್ತ್ವದ
ಧರ್ಮ-ದ-ಲ್ಲಿ
ಧರ್ಮ-ದ-ಲ್ಲಿ-ರುವ
ಧರ್ಮ-ದ-ಲ್ಲಿ-ಲ್ಲ
ಧರ್ಮ-ದ-ಲ್ಲಿದೆ
ಧರ್ಮ-ದ-ಲ್ಲಿಯೂ
ಧರ್ಮ-ದ-ವರು
ಧರ್ಮ-ದಾನ-ಗ-ಳನ್ನು
ಧರ್ಮ-ದಿಂದ
ಧರ್ಮ-ಪ-ತ್ನಿ
ಧರ್ಮ-ಪರಿಪಾಲ-ನೆಗೆ
ಧರ್ಮ-ಪಾ-ಲರ
ಧರ್ಮ-ಪಾ-ಲರು
ಧರ್ಮ-ಪಾಲ-ರ-ನ್ನು
ಧರ್ಮ-ಪಿಪಾಸೆ
ಧರ್ಮ-ಪ್ರ-ಚಾರ-ವೆಂ-ದರೆ
ಧರ್ಮ-ಪ್ರ-ಭಾವಿಭಾಷಿತ-ವಾದ
ಧರ್ಮ-ಬೋ-ಧನೆ-ಯೊ-ಡನೆ
ಧರ್ಮ-ಬೋಧ-ಕರು
ಧರ್ಮ-ಬೋಧಕ-ರಂತೆ
ಧರ್ಮ-ಭಾವ-ಗಳು
ಧರ್ಮ-ರ-ಕ್ಷಣೆ
ಧರ್ಮ-ರ-ಕ್ಷಣೆ-ಗಾಗಿ
ಧರ್ಮ-ರಕ್ಷ-ಕ-ರೆಂದು
ಧರ್ಮ-ವ-ನ್ನಾ-ದರೂ
ಧರ್ಮ-ವ-ನ್ನಾಗಲಿ
ಧರ್ಮ-ವನ್ನು
ಧರ್ಮ-ವನ್ನೂ
ಧರ್ಮ-ವನ್ನೆ-ಲ್ಲಾ
ಧರ್ಮ-ವನ್ನೇ
ಧರ್ಮ-ವಾ-ಗಿದೆ
ಧರ್ಮ-ವಾಗಲೀ
ಧರ್ಮ-ವಾಗು-ವುವು
ಧರ್ಮ-ವಿ-ಲ್ಲದೆ
ಧರ್ಮ-ವೆ-ಲ್ಲ
ಧರ್ಮ-ವೆಂ-ದರೆ
ಧರ್ಮ-ವೆಂಬ
ಧರ್ಮ-ವೇನು
ಧರ್ಮ-ಶಾ-ಸ್ತ್ರ
ಧರ್ಮ-ಶಾ-ಸ್ತ್ರ-ಗಳ-ನ್ನೆ-ಲ್ಲಾ
ಧರ್ಮ-ಶಾ-ಸ್ತ್ರ-ಗಳ-ನ್ನೋದು-ವು-ದ-ರಿಂದ
ಧರ್ಮ-ಶಾ-ಸ್ತ್ರ-ಗಳೇ
ಧರ್ಮ-ಶಾ-ಸ್ತ್ರ-ವನ್ನು
ಧರ್ಮ-ಶಾಲಾ
ಧರ್ಮ-ಶಾಲೆ-ಯಿಂದ
ಧರ್ಮ-ಶ್ರದ್ಧೆ-ಯೆ-ಲ್ಲಿ
ಧರ್ಮ-ಸಂ-ದೇಶದ
ಧರ್ಮ-ಸಂ-ಸ್ಥಾಪನಾ-ಚಾರ್ಯರು
ಧರ್ಮ-ಸಂ-ಸ್ಥೆಗೂ
ಧರ್ಮ-ಸಂಬಂಧ-ವಾಗಿ
ಧರ್ಮ-ಸಭೆ-ಯ-ಲ್ಲಿ
ಧರ್ಮ-ಸಮ್ಮೇಳ-ನಕ್ಕೆ
ಧರ್ಮ-ಸಮ್ಮೇಳ-ನದ
ಧರ್ಮ-ಸಮ್ಮೇಳನ
ಧರ್ಮ-ಸ್ಥ-ರಿಂದ
ಧರ್ಮ-ಸ್ಯ
ಧರ್ಮಕ್ಕೂ
ಧರ್ಮಕ್ಕೆ
ಧರ್ಮದ
ಧರ್ಮವೂ
ಧರ್ಮವೇ
ಧರ್ಮವೋ
ಧರ್ಮಾ-ಚಾರ
ಧರ್ಮಾಂಧ-ನಾಗಿದ್ದೆ
ಧರ್ಮಾನುಯಾಯಿ-ಗಳ-ಲ್ಲಿಯೂ
ಧರ್ಮಾಮೃತ-ವನ್ನು
ಧರ್ಮೋದ್ಧಾ-ರದ
ಧಾ
ಧಾತು-ಗರ್ಭ
ಧಾತು-ಗರ್ಭ-ದಂತೆ
ಧಾತು-ಗರ್ಭ-ದೊಳಗೆ
ಧಾರಣ-ವನ್ನು
ಧಾರಣೆ
ಧಾರಣೆ-ಮಾಡಿ
ಧಾರಾ-ಕಾರ-ವಾಗಿ
ಧಾರಾಳ-ವಾಗಿ
ಧಾರೆ
ಧಾರೆ-ಯೆ-ರೆಯ-ಬ-ಲ್ಲ
ಧಾರೆ-ಯೆ-ರೆಯ-ಬ-ಲ್ಲಿರಿ
ಧಾರೆ-ಯೆ-ರೆಯ-ಬೇಕಾಗಿದೆ
ಧಾರೆ-ಯೆರೆ-ಯಲು
ಧಾರೆ-ಯೆರೆ-ಯುವ
ಧಾರೆ-ಯೆರೆದು
ಧಾರೆ-ಯೆರೆದು-ಕೊಂಡಿ-ರುವುದು
ಧಾರ್ಮಿ-ಕರ
ಧಾರ್ಮಿಕ
ಧಾರ್ಮಿಕ-ಭಾ-ವನೆ
ಧಾರ್ಮಿಕ-ವಾಗಿ
ಧಾಳಿ
ಧಾಳಿ-ಯನ್ನು
ಧಾಳಿ-ಯಿಂದ
ಧಾವಿ-ಸು-ತ್ತಿ-ದ್ದರು
ಧಾವಿ-ಸು-ತ್ತಿ-ದ್ದು-ದ-ನ್ನು
ಧಾವಿ-ಸು-ವೆವು
ಧಾವಿ-ಸುವ
ಧಾವಿ-ಸುವರು
ಧಾವಿಸಿ
ಧಾವಿಸಿ-ದಂತೆ
ಧಾವಿಸಿ-ದರು
ಧಿಃ
ಧಿರ-ನಾಗಿ-ದ್ದರೆ
ಧೀ
ಧೀರ
ಧೀರ-ನಂತೆ
ಧೀರ-ಮಂ-ತ್ರ
ಧೀರ-ರಾಗಿ
ಧೀರ-ರಿಗೆ
ಧೀರಳು
ಧೀರಾಧಿ
ಧುಮು-ಧುಮುಕಿ
ಧುಮುಕಿ
ಧುಮುಕಿ-ಕೊಂಡು
ಧೂ
ಧೂನಿ
ಧೂನಿ-ಯನ್ನು
ಧೂಪ-ದಾನಿ
ಧೂಪ-ದಿಂದ
ಧೂಪ-ದೀಪಾದಿ-ಗಳ
ಧೂಪ-ವನ್ನು
ಧೂಮ
ಧೂಮ-ಪಾನ
ಧೂಳಿ-ನ-ಲ್ಲಿ
ಧೂಳಿ-ನಿಂದ
ಧೂಳೀ-ಪಟ
ಧೂಳೀಕಣ-ಗಳ-ನ್ನಾಗಿ
ಧೂಳು
ಧೂಸರ
ಧೃವ-ತಾರೆ-ಯಂತೆ
ಧೃವ-ದಷ್ಟು
ಧೈರ್ಯ
ಧೈರ್ಯ-ಗ-ಳನ್ನು
ಧೈರ್ಯ-ದಿಂದ
ಧೈರ್ಯ-ಧ್ವಜ-ವನ್ನು
ಧೈರ್ಯ-ವನ್ನು
ಧೈರ್ಯ-ವಾಗಲಿ
ಧೈರ್ಯ-ವಾಗಿ
ಧೈರ್ಯ-ವಿ-ಲ್ಲ-ದವ-ರಾಗಿ-ದ್ದೇವೆ
ಧೈರ್ಯ-ವಿದೆ
ಧೈರ್ಯ-ವಿದ್ದರೆ
ಧೈರ್ಯ-ಶಾಲಿ
ಧೈರ್ಯ-ಹೀನ-ವಾ-ಗಿದೆ
ಧ್ಯಾ-ನಕ್ಕೆ
ಧ್ಯಾ-ನಾದಿ-ಗಳ-ಲ್ಲಿ
ಧ್ಯಾನ
ಧ್ಯಾನ-ಕಾಲ-ದ-ಲ್ಲಿ
ಧ್ಯಾನ-ಗಳಿಂದ
ಧ್ಯಾನ-ದ-ಲ್ಲಿ
ಧ್ಯಾನ-ದ-ಲ್ಲಿ-ದ್ದರು
ಧ್ಯಾನ-ದ-ಲ್ಲಿ-ರುವಾಗ
ಧ್ಯಾನ-ದ-ಲ್ಲಿಯೂ
ಧ್ಯಾನ-ದ-ಲ್ಲೇ
ಧ್ಯಾನ-ದಿಂದ
ಧ್ಯಾನ-ಧಾರಣ
ಧ್ಯಾನ-ಪ-ರವ-ಶ-ರಾಗಲು
ಧ್ಯಾನ-ಮಗ್ನ
ಧ್ಯಾನ-ಮಗ್ನ-ನಾಗಿ-ದ್ದು-ದ-ನ್ನು
ಧ್ಯಾನ-ಮಗ್ನ-ರಾಗ-ತೊಡಗಿದರು
ಧ್ಯಾನ-ಮಗ್ನ-ರಾಗಿ
ಧ್ಯಾನ-ಮಗ್ನ-ರಾಗಿ-ದ್ದಾಗ
ಧ್ಯಾನ-ಮಗ್ನ-ರಾಗಿ-ರು-ವಂತೆ
ಧ್ಯಾನ-ಮಾಡಲು
ಧ್ಯಾನ-ಮಾಡಿ
ಧ್ಯಾನ-ಮಾಡಿದ
ಧ್ಯಾನ-ಮಾಡು-ವು-ದ-ಕ್ಕಾಗಿ
ಧ್ಯಾನ-ಮಾಡು-ವು-ದಕ್ಕೆ
ಧ್ಯಾನ-ಮಾಡುವ
ಧ್ಯಾನ-ವನ್ನು
ಧ್ಯಾನ-ವಾಗಿ
ಧ್ಯಾನ-ವಾದ
ಧ್ಯಾನ-ಸಿಂಗ್
ಧ್ಯಾನ-ಸಿದ್ಧ
ಧ್ಯಾನ-ಸಿದ್ಧ-ನಾಗು-ತ್ತಾನೆಯೊ
ಧ್ಯಾನ-ಸ್ಥ-ರಾ-ದರು
ಧ್ಯಾನ-ಸ್ಥ-ವಾಗು-ತ್ತದೆ
ಧ್ಯಾನ-ಸ್ಥರಾಗಿ
ಧ್ಯಾನದ
ಧ್ಯಾನಾರೂಢ-ರಾಗಿ-ದ್ದರು
ಧ್ಯಾನಾವ-ಸ್ಥೆ-ಯ-ಲ್ಲಿ
ಧ್ಯಾನಾವ-ಸ್ಥೆ-ಯ-ಲ್ಲಿ-ದ್ದ-ರೆಂಬು-ದ-ನ್ನು
ಧ್ಯಾನಾವ-ಸ್ಥೆ-ಯನ್ನು
ಧ್ಯಾನಾಸಕ್ತ-ರಾ-ದರು
ಧ್ಯಾನಿಸಿ
ಧ್ಯಾನಿಸು
ಧ್ಯಾನಿಸು-ತ್ತಿದ್ದಾಗ
ಧ್ಯಾನಿಸು-ತ್ತಿರ-ಬಹುದು
ಧ್ಯಾನಿಸು-ವರೋ
ಧ್ಯೇ
ಧ್ಯೇಯ
ಧ್ಯೇಯ-ಗಳಾ-ವುವೆಂಬು-ದರ
ಧ್ಯೇಯ-ದಿಂದ
ಧ್ಯೇಯ-ವಾಗಿ-ತ್ತು
ಧ್ರುವ-ತಾರೆ-ಯೊಂದು
ಧ್ವಂಸ
ಧ್ವಂಸ-ಕಾರಕ
ಧ್ವಂಸ-ಕಾರ್ಯ-ವಿ-ರ-ಲಿ-ಲ್ಲ
ಧ್ವಂಸ-ಮಾಡಿ
ಧ್ವಂಸ-ಮಾಡಿ-ಬಿಡು-ತ್ತಿದ್ದರು
ಧ್ವಂಸ-ವ-ಲ್ಲ
ಧ್ವಂಸ-ವಾಗಿ
ಧ್ವಂಸಕ್ಕೆ
ಧ್ವಜ-ಪತಾಕೆ-ಯನ್ನು
ಧ್ವಜ-ವನ್ನು
ಧ್ವಜದ
ಧ್ವನಿ
ಧ್ವನಿ-ಗಳ
ಧ್ವನಿ-ತ-ವಾ-ಗು-ತ್ತಿದೆ
ಧ್ವನಿ-ಪೂರ್ಣ-ವಾದುವು
ಧ್ವನಿ-ಯ-ಲ್ಲಿ
ಧ್ವನಿ-ಯಂತೆ
ಧ್ವನಿ-ಯನ್ನು
ಧ್ವನಿ-ಯಾಗ-ಲೆಂಬುದೆ
ಧ್ವನಿ-ಯಿಂದ
ಧ್ವನಿ-ಯಿಂದಲೂ
ಧ್ವನಿ-ಯಿದೆ
ಧ್ವನಿ-ಯೊಂದೇ
ಧ್ವನಿ-ಸ್ಪಂದನ
ಧ್ವನಿತ
ಧ್ವನಿಯ
ಧ್ವನಿಯು
ಧ್ವನಿಯೋ
ನ
ನಂ
ನಂಟ
ನಂಟ-ರಿಷ್ಟ-ರ-ನ್ನು
ನಂಬ-ಬೇಕು
ನಂಬ-ಲಾಗು-ವು-ದಿ-ಲ್ಲ
ನಂಬ-ಲಿ-ಲ್ಲ
ನಂಬದ
ನಂಬದ-ವನು
ನಂಬದ-ವರೂ
ನಂಬದೆ
ನಂಬದೇ
ನಂಬಲಿ
ನಂಬಲೇ-ಬೇಕಾಗುವುದು
ನಂಬಿ
ನಂಬಿ-ಕೆ-ಗ-ಳನ್ನು
ನಂಬಿ-ಕೆ-ಗ-ಳಿಗೆ
ನಂಬಿ-ಕೆ-ಗಳ-ಲ್ಲಿ
ನಂಬಿ-ಕೆ-ಗಳು
ನಂಬಿ-ಕೆ-ಯ-ನ್ನೆ-ಲ್ಲ
ನಂಬಿ-ಕೆ-ಯನ್ನು
ನಂಬಿ-ಕೆ-ಯಿ-ಲ್ಲ
ನಂಬಿ-ಕೆ-ಯಿ-ಲ್ಲವೆ
ನಂಬಿ-ಕೆ-ಯುಂಟಾಗಿ
ನಂಬಿ-ಕೆ-ಯುಂಟಾಗು-ವು-ದಿ-ಲ್ಲ
ನಂಬಿ-ಕೆಯ
ನಂಬಿ-ಕೆಯೇ
ನಂಬಿ-ದ-ವ-ರ-ನ್ನು
ನಂಬಿ-ದ-ವ-ರಿಗೆ
ನಂಬಿ-ದರೂ
ನಂಬಿ-ದರೆ
ನಂಬಿ-ಬಿಡುವ
ನಂಬಿ-ರ-ಲಿ-ಲ್ಲ
ನಂಬಿ-ರುವಾಗ
ನಂಬಿಕೆ
ನಂಬಿದೆ
ನಂಬು-ತ್ತಾ-ರ-ಲ್ಲ
ನಂಬು-ತ್ತಾನೆ
ನಂಬು-ತ್ತಾರೆ
ನಂಬು-ತ್ತಿ-ರಲಿ-ಲ್ಲ
ನಂಬು-ತ್ತಿದ್ದುದು
ನಂಬು-ತ್ತೇನೆ
ನಂಬು-ತ್ತೇವೆ
ನಂಬು-ವ-ವ-ನ-ಲ್ಲ
ನಂಬು-ವಂತೆಯೇ
ನಂಬು-ವನು
ನಂಬು-ವರು
ನಂಬು-ವು-ದ-ರಿಂದ
ನಂಬು-ವು-ದಕ್ಕೆ
ನಂಬು-ವು-ದಿ-ಲ್ಲ
ನಂಬು-ವುದೇ
ನಂಬುಗೆ
ನಂಬುವ-ವ-ರಿ-ದ್ದರು
ನಂಬುವವ-ನಿಗೂ
ನಕಾಶೆ-ಯನ್ನು
ನಕ್ಕರು
ನಕ್ಕಿ-ದ್ದಕ್ಕೆ
ನಕ್ಕು
ನಕ್ಕು-ಬಿ-ಟ್ಟ
ನಕ್ಷ-ತ್ರ-ಗ-ಳನ್ನು
ನಕ್ಷ-ತ್ರ-ಗಳ-ಲ್ಲಿ
ನಕ್ಷ-ತ್ರ-ಗಳ-ಲ್ಲಿ-ರು-ವೆವು
ನಕ್ಷ-ತ್ರ-ಗಳಂತೆ
ನಖ-ಗಳಿಂದ
ನಖ-ದಿಂದ
ನಗ
ನಗ-ತೊಡಗಿದ
ನಗ-ತೊಡಗಿದರು
ನಗ-ರಕ್ಕೆ
ನಗಾಧಿ-ರಾಜಃ
ನಗಿ-ಸು-ವು-ದ-ನ್ನು
ನಗು
ನಗು-ತ್ತ
ನಗು-ತ್ತಾ
ನಗು-ತ್ತಾರೆ
ನಗು-ನಗು-ತ್ತ
ನಗು-ಮುಖ
ನಗು-ಮುಖ-ದಿಂದ
ನಗು-ವನ್ನು
ನಗು-ವರು
ನಗು-ವಿ-ಲ್ಲ
ನಗು-ವು-ದಕ್ಕೆ
ನಗು-ವು-ದಿ-ಲ್ಲ
ನಗುವೋ
ನಗೆ-ಯನ್ನು
ನಗೆಗೀಡಾಗಿವೆ
ನಗ್ನ-ಸ-ತ್ಯ-ವನ್ನು
ನಚಿಕೇ-ತನ
ನಚಿಕೇ-ತನ-ಗಿ-ದ್ದಂತೆ
ನಚಿಕೇ-ತನಿಗಿದ್ದಂತಹ
ನಚಿಕೇತ-ನಿಗೆ
ನಟ
ನಟ-ಕೃಷ್ಣ
ನಟನೆ
ನಟರು
ನಟಿ
ನಟಿ-ಸು-ವುದು
ನಟಿ-ಸುವರು
ನಟಿಗೆ
ನಟೇಶ
ನಡ-ತೆಗೆ
ನಡ-ತೆಯ
ನಡ-ತೆಯೇ
ನಡಗುವ
ನಡತೆ
ನಡತೆ-ಯ-ಲ್ಲಾ-ದರೂ
ನಡತೆ-ಯ-ಲ್ಲಿ
ನಡತೆ-ಯನ್ನು
ನಡವಳಿ-ಕೆ-ಗಳು
ನಡವಳಿ-ಕೆಗೆ
ನಡವಳಿಕೆ-ಯ-ಲ್ಲಿ
ನಡವಳಿಕೆ-ಯನ್ನೂ
ನಡಿ-ಗೆಯ
ನಡಿ-ಗೆಯೆ
ನಡಿಗೆ-ಯ-ಲ್ಲಿ
ನಡು-ಮನೆ-ಯ-ಲ್ಲಿ
ನಡುಗ-ತೊಡಗಿದವು
ನಡುಗಿಸು-ವು-ದಕ್ಕೂ
ನಡೆ
ನಡೆ-ದ-ಮೇಲೆ
ನಡೆ-ದಂತೆಯೇ
ನಡೆ-ದದ್ದು
ನಡೆ-ದರು
ನಡೆ-ದರೆ
ನಡೆ-ದವು
ನಡೆ-ದಾದ-ಮೇಲೆ
ನಡೆ-ದಿ-ರುವುದ-ನ್ನೆ-ಲ್ಲ
ನಡೆ-ದಿ-ಲ್ಲ
ನಡೆ-ದು-ಕೊಂಡರೆ
ನಡೆ-ದು-ಕೊಂಡು
ನಡೆ-ದು-ಕೊಂಡು-ಹೋಗು-ತ್ತಿದ್ದರು
ನಡೆ-ದು-ಕೊಂಡೇ
ನಡೆ-ದು-ದ-ನ್ನೆ-ಲ್ಲ
ನಡೆ-ನುಡಿ-ಗಳು
ನಡೆ-ನುಡಿ-ಯ-ಲ್ಲಿ
ನಡೆ-ಯ-ಬ-ಲ್ಲದು
ನಡೆ-ಯ-ಬಹುದು
ನಡೆ-ಯ-ಬೇ-ಕಾ-ದರೆ
ನಡೆ-ಯ-ಬೇಕು
ನಡೆ-ಯಲಾ-ರದು
ನಡೆ-ಯಲಾರಂಭಿ-ಸಿ-ದರು
ನಡೆ-ಯಲು
ನಡೆ-ಯಿತು
ನಡೆ-ಯಿರಿ
ನಡೆ-ಯಿರಿ-ಹಣವಿ-ರಲಿ
ನಡೆ-ಯು-ತ್ತ
ನಡೆ-ಯು-ತ್ತದೆ
ನಡೆ-ಯು-ತ್ತಾರೆ
ನಡೆ-ಯು-ತ್ತಿ-ತ್ತು
ನಡೆ-ಯು-ತ್ತಿ-ರಲಿ-ಲ್ಲ
ನಡೆ-ಯು-ತ್ತಿ-ಲ್ಲ-ವ-ಲ್ಲ
ನಡೆ-ಯು-ತ್ತಿದೆ
ನಡೆ-ಯು-ತ್ತಿದ್ದ
ನಡೆ-ಯು-ತ್ತಿದ್ದರೂ
ನಡೆ-ಯು-ತ್ತಿದ್ದಾಗ
ನಡೆ-ಯು-ತ್ತಿರುವ
ನಡೆ-ಯು-ತ್ತೇನೆ
ನಡೆ-ಯು-ವಂತೆ
ನಡೆ-ಯು-ವು-ದಕ್ಕೆ
ನಡೆ-ಯು-ವು-ದಿ-ಲ್ಲ
ನಡೆ-ಯು-ವು-ದೆಂದು
ನಡೆ-ಯು-ವುದು
ನಡೆ-ಯು-ವುದೊಂದೇ
ನಡೆ-ಯು-ವುವು
ನಡೆ-ಯುವ
ನಡೆ-ಯುವ-ವ-ನಿಗೆ
ನಡೆ-ಯುವ-ವ-ರೆಗೆ
ನಡೆ-ಯುವಾಗ
ನಡೆ-ಸ-ಬೇ-ಕಾ-ದರೆ
ನಡೆ-ಸ-ಬೇಕೆಂ-ದಾಗ
ನಡೆ-ಸ-ಬೇಕೆಂದು
ನಡೆ-ಸ-ಬೇಕೆಂಬ
ನಡೆ-ಸಲು
ನಡೆ-ಸಿ-ಕೊಂಡು
ನಡೆ-ಸಿ-ಕೊಂಡು-ಹೋಗಲು
ನಡೆ-ಸಿ-ದಂತೆ
ನಡೆ-ಸಿ-ದರು
ನಡೆ-ಸಿ-ರು-ವೆನು
ನಡೆ-ಸಿದ
ನಡೆ-ಸು-ತ್ತಿ-ದ್ದರು
ನಡೆ-ಸು-ತ್ತಿ-ದ್ದರೆ
ನಡೆ-ಸು-ತ್ತಿ-ದ್ದಳು
ನಡೆ-ಸು-ತ್ತಿ-ರುವ-ವನು
ನಡೆ-ಸು-ತ್ತಿದ್ದ
ನಡೆ-ಸು-ತ್ತ್ತಿದ್ದರು
ನಡೆ-ಸು-ವ-ವರ
ನಡೆ-ಸು-ವರು
ನಡೆ-ಸು-ವಳು
ನಡೆ-ಸು-ವು-ದ-ಕ್ಕಾಗಿ
ನಡೆ-ಸು-ವು-ದಕ್ಕೆ
ನಡೆ-ಸು-ವುದ-ಕ್ಕೋ-ಸ್ಕರ
ನಡೆ-ಸು-ವುದ-ರ-ಲ್ಲಿ
ನಡೆ-ಸು-ವುದು
ನಡೆ-ಸುವ
ನಡೆದ
ನಡೆದು
ನಡೆದೇ
ನಡೆವ
ನಡೆಸಿ
ನಡೆಸು
ನದಿ
ನದಿ-ಗಳ
ನದಿ-ಗಳು
ನದಿ-ಗಳೆ-ಲ್ಲ
ನದಿ-ಯ-ಲ್ಲಿ
ನದಿ-ಯನ್ನು
ನದಿ-ಯೊಂದು
ನದಿಗೆ
ನದಿಯ
ನದಿಯು
ನದೀ
ನನ-ಗಂತೂ
ನನ-ಗಾ-ವು-ದರ
ನನ-ಗಾಗಿ
ನನ-ಗಾಗುವಿ-ದಿ-ಲ್ಲ
ನನ-ಗಿ-ರುವ
ನನ-ಗಿಂದು
ನನ-ಗಿದ್ದ
ನನ-ಗೀಗ
ನನ-ಗೇನು
ನನ-ಗೇನೂ
ನನ-ಗೇನೋ
ನನ-ಗೊಂದು
ನನ-ಗೊಬ್ಬ-ನಿಗೆ
ನನ-ಗೊಸುಗ
ನನಗೂ
ನನಗೆ
ನನಗೇ
ನನ್ನ
ನನ್ನ-ದ-ಲ್ಲ
ನನ್ನ-ಲ್ಲಿ
ನನ್ನ-ಲ್ಲಿ-ತ್ತು
ನನ್ನ-ಲ್ಲಿ-ರುವ
ನನ್ನ-ಲ್ಲಿದೆ
ನನ್ನ-ವರು
ನನ್ನಂತಹ
ನನ್ನಂತೆ
ನನ್ನದು
ನನ್ನದೇ
ನನ್ನನು
ನನ್ನನ್ನು
ನನ್ನನ್ನೂ
ನನ್ನನ್ನೇ
ನನ್ನಾಸೆ
ನನ್ನಿಂದ
ನನ್ನಿಂದ-ಲೇ-ಆ-ದ-ವರು
ನನ್ನೆ-ಡೆಗೆ
ನನ್ನೊ-ಡನೆ
ನನ್ನೊ-ಡನೆಯೇ
ನಮ-ಗಂತೂ
ನಮ-ಗಾಗಿ
ನಮ-ಗಿಂತ
ನಮ-ಗೆ-ಲ್ಲ
ನಮ-ಗೊಂದು
ನಮ-ಸ್ಕ-ರಿಸಿ
ನಮ-ಸ್ಕ-ರಿಸಿ-ದರು
ನಮ-ಸ್ಕರಿ-ಸು-ವುದು
ನಮ-ಸ್ಕಾರ
ನಮ-ಸ್ಕಾರ-ಗ-ಳನ್ನು
ನಮಃ
ನಮಗಿ-ರುವ
ನಮಗೀಗ
ನಮಗೂ
ನಮಗೆ
ನಮಗೇನು
ನಮಿಸ-ಬೇಕು
ನಮಿಸಿ
ನಮಿಸಿ-ದರು
ನಮೋ
ನಯ-ವಾಗಿ
ನರ
ನರ-ಕ-ದಂತೆ
ನರ-ಕ-ಯಾ-ತನೆ
ನರ-ಕ-ಸದೃಶ-ವ-ನ್ನಾಗಿ
ನರ-ಕಕ್ಕೆ
ನರ-ಕಕ್ಕೇ
ನರ-ಕದ
ನರ-ಗ-ಳನ್ನು
ನರ-ಗಳ
ನರ-ಗಳು
ನರ-ದೇಹದ
ನರ-ನಾರಿ-ಯರು
ನರ-ಭ-ಕ್ಷಣ
ನರ-ಭಕ್ಷಕ
ನರ-ಮಹರ್ಷಿಯೇ
ನರ-ಮಾಂಸ
ನರ-ರಕ್ತ-ದಿಂದ
ನರ-ಳ-ದಂತೆ
ನರ-ಳ-ಬೇಕಾ-ಯಿತು
ನರ-ಳ-ಬೇಕು
ನರ-ಳು-ತ್ತಲೇ
ನರ-ಳು-ತ್ತಿ-ರಲಿ
ನರ-ಳು-ತ್ತಿ-ರು-ವಿರಿ
ನರ-ಳು-ತ್ತಿ-ರು-ವು-ದ-ನ್ನು
ನರ-ಳು-ತ್ತಿದ್ದ
ನರ-ಳು-ತ್ತಿದ್ದರು
ನರ-ಳು-ತ್ತಿದ್ದಾಗ
ನರ-ಳು-ತ್ತಿರು-ವ-ರೆಂಬು-ದ-ನ್ನು
ನರ-ಳು-ತ್ತಿರು-ವ-ವ-ರಿಗೆ
ನರ-ಳು-ತ್ತಿರು-ವ-ವರು
ನರ-ಳು-ತ್ತಿರು-ವರು
ನರ-ಳು-ತ್ತಿರು-ವರೋ
ನರ-ಳು-ತ್ತಿರುವ
ನರ-ಳು-ತ್ತಿರುವೆ
ನರ-ಳುವ
ನರ-ವುಳ್ಳ-ವ-ನಿಗೆ
ನರ-ಶುದ್ಧಿ
ನರ-ಸಿಂಹ
ನರ-ಸಿಂಹಾ-ಚಾರಿ
ನರಕ
ನರಳಿ
ನರೇ-ನನು
ನರೇ-ನ್
ನರೇಂದ್ರ
ನರೇಂದ್ರ-ದ-ತ್ತರು
ನರೇಂದ್ರ-ನ-ಲ್ಲಿ
ನರೇಂದ್ರ-ನಂತಹ
ನರೇಂದ್ರ-ನಂಥ-ವ-ನಿಗೆ
ನರೇಂದ್ರ-ನದು
ನರೇಂದ್ರ-ನನ್ನು
ನರೇಂದ್ರ-ನನ್ನೂ
ನರೇಂದ್ರ-ನಾ-ದರೊ
ನರೇಂದ್ರ-ನಾ-ದರೋ
ನರೇಂದ್ರ-ನಾಥ
ನರೇಂದ್ರ-ನಾಥ-ದ-ತ್ತ
ನರೇಂದ್ರ-ನಾಥ-ನ-ದಾದರೊ
ನರೇಂದ್ರ-ನಾಥ-ನನ್ನು
ನರೇಂದ್ರ-ನಾಥ-ನಿಗೆ
ನರೇಂದ್ರ-ನಾಥನ
ನರೇಂದ್ರ-ನಾಥರು
ನರೇಂದ್ರ-ನಿ-ಗಾ-ದರೋ
ನರೇಂದ್ರ-ನಿಂದ
ನರೇಂದ್ರ-ನಿಗಾ-ದರೊ
ನರೇಂದ್ರ-ನಿಗೂ
ನರೇಂದ್ರ-ನಿಗೆ
ನರೇಂದ್ರ-ನಿಗೇ
ನರೇಂದ್ರ-ನೆಂಬ
ನರೇಂದ್ರ-ನೊ-ಡನೆ
ನರೇಂದ್ರ-ನೊಂದಿಗೆ
ನರೇಂದ್ರ-ನೊಬ್ಬ
ನರೇಂದ್ರ-ನೊಬ್ಬ-ನಿಗೇ
ನರೇಂದ್ರನ
ನರೇಂದ್ರನು
ನರೇಂದ್ರನೂ
ನರೇಂದ್ರನೆ
ನರೇಂದ್ರನೇ
ನರೇಂದ್ರಾ-ದಿ-ಗ-ಳನ್ನು
ನರೇಂದ್ರಾದಿ-ಗ-ಳಿಗೆ
ನರೇನ-ನ-ಲ್ಲಿ
ನರ್ತಕಿ
ನರ್ತಕಿ-ಯನ್ನು
ನರ್ತನ
ನರ್ತನ-ಮಾಡು-ತ್ತಿದೆ
ನರ್ತಿ-ಸಿತು
ನರ್ಮ-ದೇಶ್ವರ
ನರ್ಮದಾ
ನರ್ಮದಾ-ತೀರ-ದ-ಲ್ಲಿ
ನಲ-ವತ್ತನೇ
ನಲ-ವತ್ತು
ನಲ-ವತ್ತೈದು
ನಲಿ-ದರು
ನಲಿ-ಯು-ತ್ತಿದ್ದ
ನಲಿ-ಯು-ತ್ತಿರುವ
ನಲಿದಾಡು-ತ್ತಿದ್ದ
ನವ
ನವ-ಗೋಪಾಲ
ನವ-ಭಾರ-ತದ
ನವ-ರ-ತ್ನ-ಗಳಂತೆ
ನವ-ಶೈಲಿ-ಯ-ಲ್ಲಿ
ನವಂಬರ್
ನವವಿಧಾನ
ನವಾಬರ
ನವಾಬರು
ನವಿಲು
ನವೀನ
ನವ್ಯತೆ-ಯಿ-ಲ್ಲ
ನಶಿಸಿ
ನಶಿಸಿ-ಹೋಗು-ತ್ತಿರುವ
ನಶ್ಯತಿ
ನಶ್ವರ-ತೆಯ
ನಷ್ಟ
ನಷ್ಟ-ಗ-ಳನ್ನು
ನಷ್ಟ-ಗಳ
ನಷ್ಟ-ವಾ-ದರೂ
ನಷ್ಟ-ವಾಗು-ವುದು
ನಷ್ಟ-ವಾಗು-ವುದೆಂದು
ನಸು-ನಗು-ತ್ತ
ನಾ
ನಾಗ-ಮಹಾ-ಶ-ಯರ
ನಾಗ-ಮಹಾ-ಶ-ಯರು
ನಾಗ-ಮಹಾ-ಶ-ಯರೂ
ನಾಗ-ಮಹಾ-ಶ-ಯರೆ
ನಾಗ-ಮಹಾ-ಶಯ
ನಾಗ-ಮಹಾ-ಶಯ-ರ-ನ್ನು
ನಾಗ-ಮಹಾ-ಶಯ-ರಂಥ
ನಾಗ-ಮಹಾ-ಶಯ-ರಿಂದ
ನಾಗ-ಮಹಾ-ಶಯ-ರಿಗೂ
ನಾಗ-ಮಹಾ-ಶಯ-ರಿಗೆ
ನಾಗ-ಮಹಾ-ಶಯ--ಆ-ತನ
ನಾಗ-ಸಾಕಿ
ನಾಗ-ಸಾಕಿ-ಯಿಂದ
ನಾಗ-ಸ್ವ-ರದ
ನಾಗಿ-ದ್ದಾಗ
ನಾಚಿ
ನಾಚಿ-ಕೆ-ಯಾ-ಯಿತು
ನಾಚಿ-ಕೆ-ಯಿಂದ
ನಾಚಿಕೆ
ನಾಟಕ
ನಾಟಕ-ಕರ್ತೃ-ಗ-ಳಿಗೆ
ನಾಟಕ-ಗ-ಳನ್ನು
ನಾಟಕ-ಗಳ-ಲ್ಲಿ
ನಾಟಕ-ಗಳಿ-ರುವ
ನಾಟಕ-ಗಳು
ನಾಟಕ-ದ-ಲ್ಲಿ
ನಾಟಕ-ವನ್ನು
ನಾಟಕಕ್ಕೆ
ನಾಟಕದ
ನಾಟಕವೇ
ನಾಟಕೀಯ-ವಾಗಿ
ನಾಟಿಂಗ್ಹಾ-ಮ್
ನಾಟು-ವಂತೆ
ನಾಡಿ-ನ-ಲ್ಲಿ
ನಾಡಿ-ಯನ್ನು
ನಾಡಿದ್ದು
ನಾಡಿನ
ನಾಡು
ನಾಣ್ಣುಡಿ
ನಾಣ್ನುಡಿ-ಯೊಂದು
ನಾಣ್ಯ-ಗ-ಳನ್ನು
ನಾಣ್ಯ-ಗಳು
ನಾಣ್ಯ-ವನ್ನು
ನಾಣ್ಯ-ವಿದ್ದರೆ
ನಾಣ್ಯದ
ನಾಥ-ಸೇನ್
ನಾದ-ವಾಗಲಿ
ನಾದ-ಸ್ವರ-ಗಳೆ-ಲ್ಲವೂ
ನಾನ-ದ-ನ್ನು
ನಾನ-ರಿಯೆ
ನಾನಾ
ನಾನಾ-ಕಾರ-ಗ-ಳನ್ನು
ನಾನಾ-ಗಲೇ
ನಾನಾ-ದರೋ
ನಾನಾ-ದರೋ-ನಾನಾರು
ನಾನಾ-ವಿಧದ
ನಾನಾ-ಸಿದ್ಧಿ-ಗ-ಳನ್ನು
ನಾನಾಗೆ
ನಾನಿ-ದ್ದು-ದ-ನ್ನು
ನಾನೀ
ನಾನೀ-ಜನ್ಮ-ದ-ಲ್ಲೇ
ನಾನೀ-ರೀತಿ
ನಾನೀಗ
ನಾನು
ನಾನೂ
ನಾನೇ
ನಾನೇ-ನಾ-ದರೂ
ನಾನೇಕೆ
ನಾನೇನೊ
ನಾನೊಂದು
ನಾಮ
ನಾಮ-ಧೇಯ-ವನ್ನು
ನಾಮ-ರೂಪ-ಗ-ಳನ್ನು
ನಾಮ-ರೂಪ-ಗಳ
ನಾಮ-ರೂಪ-ಗಳ-ಲ್ಲಿ
ನಾಮ-ರೂಪದ
ನಾಮ-ರೂಪಾ-ತೀತ-ವಾದ
ನಾಮ-ವನ್ನು
ನಾಮ-ವರಣ
ನಾಮು-ತ್ರ
ನಾಮೋಚ್ಚಾರ-ಣೆ-ಯನ್ನು
ನಾಮೋಚ್ಚಾರಣೆ
ನಾಯಕ
ನಾಯಕ-ತ್ವ-ದ-ಲ್ಲಿ
ನಾಯಕ-ತ್ವಕ್ಕೆ
ನಾಯಕ-ತ್ವದ
ನಾಯಕ-ನ-ನ್ನಾಗಿ
ನಾಯಕ-ನಾಗಲು
ನಾಯಕ-ನಾದ
ನಾಯಕ-ನಿಗೆ
ನಾಯಕ-ರಾ-ದರು
ನಾಯಕನ
ನಾಯಕನು
ನಾಯಕರು
ನಾಯಮಾ-ತ್ಮಾ
ನಾಯಿ
ನಾಯಿ-ಕೆಯರು
ನಾಯಿ-ಗ-ಳಿಗೆ
ನಾಯಿ-ಗಳ
ನಾಯಿ-ಗಳಂತೆ
ನಾಯಿ-ಗಳು
ನಾಯಿ-ಮರಿ-ಗಳು
ನಾಯಿ-ಯಂತೆ
ನಾಯಿಗೆ
ನಾಯಿಯ
ನಾರಾ-ಯಣ
ನಾರಾ-ಯಣ-ನಿಗೆ
ನಾರಾ-ಯಣನು
ನಾರಾ-ಯಣರು
ನಾರಾ-ಯಣಿ
ನಾರಿ-ಯ-ರಿಗೆ
ನಾರಿ-ಯರು
ನಾರು-ತ್ತಿರುವ
ನಾರ್ವೆ
ನಾಲ-ಗೆಯ
ನಾಲೆಯ
ನಾಳೆ
ನಾಳೆ-ಯೋ-ನಾವ-ದ-ನ್ನು
ನಾಳೆಯೇ
ನಾಳೆಯೋ
ನಾವಿ-ಕನ
ನಾವಿಕ-ನಿಗೆ
ನಾವಿಬ್ಬರೇ
ನಾವೀಗ
ನಾವು
ನಾವು-ಗಳೂ
ನಾವೂ
ನಾವೆ-ಯಂತೆ
ನಾವೇ
ನಾವೊ
ನಾಶ
ನಾಶ-ಪಡಿ-ಸ-ಬ-ಲ್ಲದೊ
ನಾಶ-ಪಡಿ-ಸ-ಲಿ-ಲ್ಲ
ನಾಶ-ಮಾ-ಡದೆ
ನಾಶ-ಮಾಡಲು
ನಾಶ-ಮಾಡಿ-ರು-ವುವು
ನಾಶ-ಮಾಡಿದೆ
ನಾಶ-ಮಾಡು-ವ-ವ-ರೆಗೆ
ನಾಶ-ಮಾಡು-ವು-ದ-ರಿಂದ
ನಾಶ-ಮಾಡು-ವುದು
ನಾಶ-ಮಾಡುವ
ನಾಶ-ವ-ಲ್ಲ
ನಾಶ-ವನ್ನು
ನಾಶ-ವಾ-ದರೂ
ನಾಶ-ವಾ-ದವು
ನಾಶ-ವಾಗ-ದಿ-ರಲಿ
ನಾಶ-ವಾಗ-ಬಹುದು
ನಾಶ-ವಾಗ-ಬೇಕು
ನಾಶ-ವಾಗ-ಲೇ-ಬೇಕು
ನಾಶ-ವಾಗದ
ನಾಶ-ವಾಗದೆ
ನಾಶ-ವಾಗಿ
ನಾಶ-ವಾಗಿ-ಹೋಗಿ-ದ್ದರು
ನಾಶ-ವಾಗು
ನಾಶ-ವಾಗು-ವು-ದಿ-ಲ್ಲ
ನಾಶ-ವಾಗು-ವುದು
ನಾಶ-ವಾಗು-ವುದೆಂ-ದರೆ
ನಿ
ನಿಂ
ನಿಂತ
ನಿಂತ-ಮೇಲೆ
ನಿಂತಳು
ನಿಂತವು
ನಿಂತಿ-ದ್ದನು
ನಿಂತಿ-ದ್ದರೂ
ನಿಂತಿ-ದ್ದಾರೆ
ನಿಂತಿ-ದ್ದುದು
ನಿಂತಿ-ರು-ತ್ತದೆ
ನಿಂತಿ-ರು-ವಂತೆ
ನಿಂತಿ-ರು-ವು-ದ-ನ್ನು
ನಿಂತಿ-ರು-ವುದು
ನಿಂತಿ-ರುವ
ನಿಂತಿತು
ನಿಂತಿದೆ
ನಿಂತಿದೆ-ನಿಂತಿದ್ದ
ನಿಂತಿವೆ
ನಿಂತು
ನಿಂತು-ಕೊಂಡ
ನಿಂತು-ಕೊಂಡರು
ನಿಂತು-ಕೊಂಡಾಗ
ನಿಂತು-ಕೊಂಡಿತು
ನಿಂತು-ಕೊಂಡಿದೆ
ನಿಂತು-ಕೊಂಡಿದ್ದರು
ನಿಂತು-ಕೊಂಡು
ನಿಂತು-ಕೊಂಡೇ
ನಿಂತು-ಕೊಳ್ಳ-ಬೇಕಾಗಿ-ಲ್ಲ
ನಿಂತು-ಕೊಳ್ಳ-ಬೇಕು
ನಿಂತು-ಹೋ-ಯಿತು
ನಿಂತು-ಹೋಗಲಿ
ನಿಂತು-ಹೋಗಿ-ತ್ತು
ನಿಂತು-ಹೋಗು-ವಂತೆ
ನಿಂತೊಡ-ನೆಯೆ
ನಿಂದ
ನಿಂದಾ-ಸ್ತುತಿ-ಗ-ಳನ್ನು
ನಿಂದಾ-ಸ್ಪದ-ವಾದ
ನಿಂದಾ-ಸ್ಪದ-ವಾದುದೆಂಬುದ-ರ-ಲ್ಲಿ
ನಿಂದಿ-ಸಲಿ
ನಿಂದಿ-ಸು-ತ್ತಿ-ದ್ದರು
ನಿಂದಿ-ಸು-ತ್ತಿ-ರುವ
ನಿಂದಿ-ಸು-ತ್ತೇವೆ
ನಿಂದಿ-ಸು-ವು-ದಿ-ಲ್ಲ
ನಿಂದಿ-ಸು-ವುದು
ನಿಂದಿ-ಸ್ದು-ತ್ತಿದ್ದ
ನಿಂದಿಸಿ
ನಿಂದಿಸಿ-ಕೊ-ಳ್ಳು-ತ್ತಿದ್ದರು
ನಿಂದಿಸಿ-ದರೆ
ನಿಂದೆ
ನಿಂದೆ-ಯನ್ನು
ನಿಂದೆಗೆ
ನಿಂದೆಯ
ನಿಃ
ನಿಃಶ್ವಾಸ-ಗಳೊಂದಿಗೆ
ನಿಕಟ
ನಿಕಟ-ತೆ-ಯಿಂದ
ನಿಕಟ-ವಾ-ಯಿತು
ನಿಕಟ-ವಾಗಿ
ನಿಕಟ-ವಾಗಿ-ತ್ತು
ನಿಕೃಷ್ಟ
ನಿಗದಿ-ಯಾದ
ನಿಗೂ-ಢ-ಗಹ್ವರ-ದ-ಲ್ಲಿ
ನಿಗೂ-ಢದ
ನಿಗ್ರಹ-ವನ್ನು
ನಿಗ್ರಹಕ್ಕೆ
ನಿಗ್ರಹದ
ನಿಗ್ರಹಿ-ಸದೆ
ನಿಗ್ರಹಿಸ-ಬೇಕಾಗಿದೆ
ನಿಗ್ರಹಿಸ-ಬೇಕು
ನಿಗ್ರಹಿಸು
ನಿಜ
ನಿಜ-ವ-ಲ್ಲ-ವೇನು
ನಿಜ-ವ-ಲ್ಲದೆ
ನಿಜ-ವ-ಲ್ಲವೆ
ನಿಜ-ವಾ-ದರೂ
ನಿಜ-ವಾ-ದರೆ
ನಿಜ-ವಾ-ದುದು
ನಿಜ-ವಾಗಿ
ನಿಜ-ವಾಗಿ-ದ್ದಿರ-ಬಹುದೆ
ನಿಜ-ವಾಗಿ-ರು-ವುದೇ
ನಿಜ-ವಾಗಿ-ರುವ
ನಿಜ-ವಾಗಿಯು
ನಿಜ-ವಾಗಿಯೂ
ನಿಜ-ವಾದ
ನಿಜ-ವಿರ-ಬಹುದು
ನಿಜ-ವಿರ-ಬೇಕು
ನಿಜ-ವೆ-ನ್ನು-ವು-ದ-ನ್ನು
ನಿಜ-ವೆಂದು
ನಿಜ-ಸ-ತ್ತೆ-ಯನ್ನು
ನಿಜವೆ
ನಿಜವೇ
ನಿಡುಸುಯ್ದರು
ನಿತಾಯ್
ನಿದ್ದೆ
ನಿದ್ದೆ-ಯನ್ನು
ನಿದ್ದೆ-ಯಾಗಿ-ರ-ಬಹುದು
ನಿದ್ದೆ-ಯಿಂದ
ನಿದ್ದೆಗೆ
ನಿದ್ರಾ
ನಿದ್ರಾ-ಭಂಗ-ವಾಗು-ತ್ತಿ-ತ್ತು
ನಿದ್ರಿ-ಸು-ತ್ತಿ-ದ್ದರು
ನಿದ್ರಿ-ಸು-ತ್ತಿ-ರ-ಬಹುದು
ನಿದ್ರಿ-ಸು-ತ್ತಿ-ರು-ವಳು
ನಿದ್ರಿ-ಸು-ತ್ತಿ-ರುವ
ನಿದ್ರಿ-ಸು-ತ್ತಿ-ರುವರು
ನಿದ್ರಿ-ಸು-ತ್ತಿದ್ದ
ನಿದ್ರಿಸ-ತೊಡಗಿದರು
ನಿದ್ರೆ
ನಿದ್ರೆ-ಯಿಂದ
ನಿದ್ರೆ-ಯಿಂದೆಚ್ಚೆ-ತ್ತ
ನಿದ್ರೆಗೆ
ನಿಧಾನ
ನಿಧಾನ-ವಾಗಿ
ನಿಧಾನ-ವಾಗಿಯೇ
ನಿಧಾನ-ವಾಗು-ತ್ತ
ನಿಧಾನ-ವಾಗು-ತ್ತಾ
ನಿಧಿ-ಯ-ಲ್ಲಿ
ನಿಧಿ-ಯನ್ನು
ನಿಧಿ-ಯಿಂದ
ನಿಧಿಗೆ
ನಿಧಿಯ
ನಿನ-ಗಾಗಿ
ನಿನ-ಗಿಂತ
ನಿನ-ಗೇಕೆ
ನಿನಗೂ
ನಿನಗೆ
ನಿನ್ನ
ನಿನ್ನ-ಲ್ಲಿ
ನಿನ್ನ-ಲ್ಲಿದೆ
ನಿನ್ನ-ವ-ರೆಂದು
ನಿನ್ನಂತಹ
ನಿನ್ನಂತೆ
ನಿನ್ನದು
ನಿನ್ನನ್ನು
ನಿನ್ನನ್ನೇ
ನಿನ್ನಿಂದ
ನಿನ್ನೂ-ರಿಗೆ
ನಿನ್ನೆ
ನಿನ್ನೆ-ಡೆಗೆ
ನಿನ್ನೊ-ಡನೆ
ನಿಪುಣ
ನಿಪುಣ-ರ-ನ್ನಾಗಿ
ನಿಪುಣ-ರಂತೆ
ನಿಪುಣ-ರಾದ-ರ-ವರು
ನಿಪುಣನೋ
ನಿಪುಣರು
ನಿಬಿಡ-ವಾದ
ನಿಮ-ಗಿಂತ
ನಿಮ-ಗೆ-ಲ್ಲ
ನಿಮ-ಗೆ-ಲ್ಲಾ
ನಿಮಗೂ
ನಿಮಗೆ
ನಿಮಿ-ತ್ತ
ನಿಮಿ-ತ್ತ-ಗಳ
ನಿಮಿ-ತ್ತ-ಗಳ-ನ್ನೂ
ನಿಮಿ-ತ್ತ-ನಾ-ದನೋ
ನಿಮಿ-ತ್ತ-ಮಾ-ತ್ರ
ನಿಮಿ-ತ್ತ-ವಾ-ಯಿತು
ನಿಮಿ-ತ್ತ-ವಾಗಿ
ನಿಮಿ-ತ್ತ-ವೆಂಬ
ನಿಮಿ-ತ್ತದ
ನಿಮಿ-ತ್ತಾತೀ-ತ-ವಾಗಿ
ನಿಮಿರಿ
ನಿಮಿಷ
ನಿಮಿಷ-ಗ-ಳಾದ
ನಿಮಿಷ-ಗ-ಳಾದರೂ
ನಿಮಿಷ-ಗಳ
ನಿಮಿಷ-ಗಳಾಗಿ-ತ್ತು
ನಿಮಿಷ-ಗಳಿವೆ
ನಿಮಿಷ-ಗಳು
ನಿಮಿಷ-ವಾಗಿ-ತ್ತು
ನಿಮಿಷ-ವಿದೆ
ನಿಮಿಷದ
ನಿಮಿಷಾರ್ಧ-ದ-ಲ್ಲಿ
ನಿಯ-ಮ-ಗ-ಳನ್ನು
ನಿಯ-ಮವೂ
ನಿಯ-ಮಾನು-ಸಾರ-ವಾದ
ನಿಯಮ
ನಿಯಮ-ಗ-ಳಾದ
ನಿಯಮ-ಗಳ
ನಿಯಮ-ಗಳು
ನಿಯಮ-ಗಳೂ
ನಿಯಮ-ವಾಗಿ-ತ್ತು
ನಿಯಮ-ವಾಗು-ವುದು
ನಿಯಮ-ವೆಂದು
ನಿಯಮಕ್ಕೆ
ನಿಯಮದ
ನಿಯಮವು
ನಿಯಮವೋ
ನಿಯಮಿ-ಸಿದ
ನಿರ-ತ-ರಾ-ದರು
ನಿರ-ತ-ರಾಗಿ
ನಿರ-ತ-ರಾಗಿ-ದ್ದ-ವರು
ನಿರ-ತ-ರಾಗಿ-ದ್ದರು
ನಿರ-ತ-ರಾಗಿ-ರು-ತ್ತಿದ್ದರು
ನಿರ-ತ-ರಾಗಿ-ರು-ವ-ರೆಂದೂ
ನಿರ-ತ-ರಾಗಿ-ರು-ವು-ದ-ನ್ನು
ನಿರ-ತ-ರಾಗಿ-ರು-ವುದು
ನಿರ-ತ-ರಾಗಿ-ರುವ-ರೆಂಬುದು
ನಿರ-ತನಾಗ-ಬೇಕೆಂದು
ನಿರ-ತರಾಗ-ಬಹು-ದೆಂದೂ
ನಿರ-ತರಾಗ-ಬೇಕೆಂದು
ನಿರ-ತರಾಗು-ತ್ತಿ-ದ್ದು-ದ-ನ್ನು
ನಿರಂ-ಜನ
ನಿರಂ-ಜನ-ನಿಗೆ
ನಿರಂ-ಜನ-ನಿರಂ-ಜನಾ-ನಂದ
ನಿರಂ-ಜನಾ-ನಂದ
ನಿರಂ-ಜನಾ-ನಂದ-ರೊ-ಡನೆ
ನಿರಂ-ಜನಾ-ನಂದರ
ನಿರಂ-ತರ
ನಿರಂ-ತರ-ವಾಗಿ
ನಿರಂ-ತರವೂ
ನಿರಂಜ-ನನ
ನಿರಕ್ಷರ
ನಿರತ-ನಾ-ದರೆ
ನಿರತ-ನಾಗಿ-ರ-ಬೇಕು
ನಿರತ-ನಾಗಿ-ರು-ತ್ತಿದ್ದ
ನಿರತ-ನಾಗಿ-ರು-ವುದು
ನಿರತ-ನಾಗಿ-ರು-ವೆನು
ನಿರತ-ರಾ-ದರೆ
ನಿರತ-ರಾಗಲು
ನಿರತ-ರಾಗಿದ್ದ
ನಿರತ-ರಾಗು-ತ್ತಿದ್ದರು
ನಿರತ-ರಾಗುವಂತೆ
ನಿರತ-ರಾದ
ನಿರತ-ಳಾದಳು
ನಿರತ-ವಾಗಿ-ರು-ವು-ದ-ನ್ನು
ನಿರಪೇಕ್ಷ
ನಿರಪೇಕ್ಷ-ವಾಗಿ
ನಿರಪೇಕ್ಷ-ವಾದ
ನಿರಪೇಕ್ಷೆ
ನಿರಾ-ಕರಿ-ಸದೆ
ನಿರಾ-ಕರಿ-ಸಲಾರೆ
ನಿರಾ-ಕರಿ-ಸಿದ್ದ
ನಿರಾ-ಕರಿಸು-ವ-ವ-ರ-ಲ್ಲ
ನಿರಾ-ಕಾರ
ನಿರಾ-ಕಾರ-ದ-ಲ್ಲಿ
ನಿರಾ-ಕಾರ-ವನ್ನು
ನಿರಾ-ಕಾರ-ವಾಗಿ
ನಿರಾ-ಕಾರನೂ
ನಿರಾ-ಕಾರನೆ
ನಿರಾ-ಶನಾಗದಿರು
ನಿರಾಕ-ರಿಸಿ
ನಿರಾಕ-ರಿಸಿ-ದರು
ನಿರಾಕ-ರಿಸಿ-ದರೆ
ನಿರಾಕ-ರಿಸಿ-ರುವರು
ನಿರಾತಂ-ಕ-ವಾಗಿ
ನಿರಾಧಾರ-ವಾ-ದುದು
ನಿರಾಶ-ರಾ-ದರು
ನಿರಾಶ-ರಾಗಿ
ನಿರಾಶರಾಗ-ಲಿ-ಲ್ಲವೋ
ನಿರಾಶಾ-ಜನ-ಕ-ವ-ಲ್ಲ
ನಿರಾಶೆ
ನಿರಾಶೆ-ಗಾಗಿ
ನಿರಾಶೆ-ಗೊಂಡ
ನಿರಾಶೆ-ಯನ್ನು
ನಿರಾಶೆ-ಯಾಗುವುದು
ನಿರಾಶೆಯ
ನಿರಾಶ್ರಿತ-ರಾ-ದ-ವ-ರಿಗೆ
ನಿರಾಸೆ-ಯಿಂದ
ನಿರೀಕ್ಷಿ-ಸಿದುದಕ್ಕಿಂತ
ನಿರೀಕ್ಷಿ-ಸು-ತ್ತ
ನಿರೀಕ್ಷಿ-ಸು-ತ್ತಾರೆ
ನಿರೀಕ್ಷಿ-ಸು-ತ್ತಿದೆ
ನಿರೀಕ್ಷಿ-ಸು-ವಂತೆ
ನಿರೀಕ್ಷಿ-ಸುವುದಕ್ಕಿಂತ
ನಿರೀಕ್ಷಿಸ-ಬ-ಲ್ಲಿರಿ
ನಿರೀಕ್ಷಿಸ-ಬೇಕೆಂದು
ನಿರೀಶ್ವರ
ನಿರೀಶ್ವರ-ವಾದ
ನಿರೀಶ್ವರ-ವಾದ-ವ-ಲ್ಲ
ನಿರೀಶ್ವರ-ವಾದಿ
ನಿರು-ತ್ತರ-ನಾ-ದಾಗ
ನಿರು-ತ್ತರ-ನಾದ
ನಿರುದ್ಧ-ವಾ-ದರೆ
ನಿರುದ್ಧ-ವಾಗಿ-ಬಿಡು-ತ್ತಿ-ತ್ತು
ನಿರೋಧ
ನಿರೋಧ-ನ-ವನ್ನೇ
ನಿರ್ಗಚ್ಛತಿ
ನಿರ್ಗತಿ-ಕ-ನಾ-ದರೂ
ನಿರ್ಗತಿ-ಕ-ರಾಗಿ-ದ್ದಾರೆ
ನಿರ್ಗತಿ-ಕ-ರಾದ
ನಿರ್ಗುಣ-ವಾಗಿ
ನಿರ್ಗುಣನೆ
ನಿರ್ಜನ
ನಿರ್ಜನ-ಪ್ರ-ದೇಶ-ದ-ಲ್ಲಿ
ನಿರ್ಜನ-ವಾದ
ನಿರ್ಜೀವ
ನಿರ್ಜೀವ-ದಂತಿದ್ದ
ನಿರ್ಣಯ
ನಿರ್ಣಯ-ಗ-ಳನ್ನು
ನಿರ್ಣಯ-ಗಳು
ನಿರ್ಣಯ-ವನ್ನು
ನಿರ್ಣಯ-ವಾ-ದರೂ
ನಿರ್ಣಯಕ್ಕೂ
ನಿರ್ಣಯಕ್ಕೆ
ನಿರ್ಣಯಿ-ಸುವ
ನಿರ್ದಯ
ನಿರ್ದಯ-ನಾಗಿ-ರು-ವನು
ನಿರ್ದಯ-ನಾಗು-ತ್ತಾನೆ
ನಿರ್ದಯ-ನಾದ
ನಿರ್ದಯ-ರಾದ
ನಿರ್ದಯ-ವಾದ
ನಿರ್ದಾಕ್ಷಿಣ್ಯ-ವಾಗಿ
ನಿರ್ದಿಷ್ಟ
ನಿರ್ದಿಷ್ಟ-ವಾಗಿ-ರ-ಬೇಕು
ನಿರ್ಧ-ರಿಸಿ-ದರು
ನಿರ್ಧ-ರಿಸಿ-ದೆವು
ನಿರ್ಧ-ರಿಸಿ-ದ್ದರು
ನಿರ್ಧ-ರಿಸಿ-ದ್ದಾನೆ
ನಿರ್ಧ-ರಿಸಿದೆ
ನಿರ್ಧರಿ-ಸಲಾರದ
ನಿರ್ಧರಿಸ-ಬಹುದು
ನಿರ್ಧಾ-ರಕ್ಕೆ
ನಿರ್ಧಾ-ರವೇ
ನಿರ್ಧಾರ
ನಿರ್ಧಾರ-ಗ-ಳನ್ನು
ನಿರ್ಧಾರ-ಗಳು
ನಿರ್ಧಾರ-ದಿಂದ
ನಿರ್ಧಾರ-ವನ್ನು
ನಿರ್ಧಾರ-ವಾ-ಯಿತು
ನಿರ್ಧಾರಕ್ಕೂ
ನಿರ್ನಾಮ-ವಾಗ-ಬೇಕು
ನಿರ್ನಾಮ-ವಾಗಿ
ನಿರ್ನಾಮ-ವಾಗಿ-ಬಿಡು-ವು-ದಿ-ಲ್ಲ
ನಿರ್ನಾಮ-ವಾಗಿದ್ದು
ನಿರ್ನಾಮ-ವಾಗಿವೆ
ನಿರ್ನಾಮ-ವಾಗು-ವರು
ನಿರ್ನಾಮ-ವಾಗು-ವೆವು
ನಿರ್ಭ-ಯಾ-ನಂದ
ನಿರ್ಭಯ
ನಿರ್ಭಯ-ತೆಯ
ನಿರ್ಭಯ-ದಿಂದ
ನಿರ್ಭಯ-ನಾಗಿ-ರ-ಬೇಕು
ನಿರ್ಭಯ-ನಾಗು
ನಿರ್ಭಯ-ವಾಗಿ
ನಿರ್ಭಯಾವ-ಸ್ಥೆ-ಯನ್ನು
ನಿರ್ಭಾಗ್ಯರ
ನಿರ್ಭೀತ-ನ-ನ್ನಾಗಿ
ನಿರ್ಭೀತಿ-ಯಿಂದ
ನಿರ್ಮಲ
ನಿರ್ಮಲ-ವಾಗಿ
ನಿರ್ಮಲ-ಶೀಲ
ನಿರ್ಮಾ-ಣಕ್ಕೂ
ನಿರ್ಮಾ-ಣದ
ನಿರ್ಮಾಣ
ನಿರ್ಮಾಣ-ವಾಗಿ
ನಿರ್ಮಾಣ-ವಾದ
ನಿರ್ಮಾಣಕ್ಕೆ
ನಿರ್ಮಾಪ-ಕ-ರಂತೆ
ನಿರ್ಮಿ-ಸಲಿ
ನಿರ್ಮಿ-ಸಲು
ನಿರ್ಮಿ-ಸು-ತ್ತಿ-ದ್ದರು
ನಿರ್ಮಿ-ಸು-ವು-ದ-ಕ್ಕಾಗಿ
ನಿರ್ಮಿ-ಸು-ವುದು
ನಿರ್ಮಿತ
ನಿರ್ಮಿತ-ವಾ-ದಂತೆ
ನಿರ್ಮಿಸಿ
ನಿರ್ಮಿಸಿ-ದರು
ನಿರ್ಮೂಲ
ನಿರ್ಮೂಲ-ಮಾಡ-ಬೇಕಾಗಿ-ರುವುದು
ನಿರ್ಯಾಣ
ನಿರ್ಯಾಣ-ವಾಗಲು
ನಿರ್ಯಾಣ-ವಾದ-ಮೇಲೆ
ನಿರ್ಯಾಣಾ-ನಂ-ತರ
ನಿರ್ಲಕ್ಷಿಸಿ
ನಿರ್ಲಕ್ಷಿಸಿ-ದ-ವ-ರ-ಲ್ಲ
ನಿರ್ಲಕ್ಷಿಸಿದ್ದ
ನಿರ್ಲಕ್ಷ್ಯ-ದಿಂದ
ನಿರ್ಲಕ್ಷ್ಯ-ನಾಗಿದ್ದ
ನಿರ್ಲಿಪ್ತ
ನಿರ್ಲಿಪ್ತ-ರಾಗ-ತೊಡಗಿದರು
ನಿರ್ಲಿಪ್ತ-ವಾಗ-ಬೇಕು
ನಿರ್ವಂಚ-ನೆ-ಯಿಂದ
ನಿರ್ವಹಿ-ಸಲು
ನಿರ್ವಹಿಸಿ-ಕೊಂಡು
ನಿರ್ವಾ-ಣಕ್ಕೂ
ನಿರ್ವಾಣ
ನಿರ್ವಾಣ-ದ-ಲ್ಲಿ
ನಿರ್ವಾಣ-ವನ್ನು
ನಿರ್ವಾಣಕ್ಕೆ
ನಿರ್ವಾಹ
ನಿರ್ವಿ-ಕ-ಲ್ಪ
ನಿರ್ವಿಷಯ
ನಿಲು-ಕದ
ನಿಲು-ಕದ-ವರು
ನಿಲು-ವಂಗಿ
ನಿಲು-ವಂಗಿ-ಗಳ-ನ್ನೂ
ನಿಲು-ವಂಗಿ-ಯನ್ನು
ನಿಲುಕು-ವಂತಹ
ನಿವಾಳಿ-ಸಿ-ದರು
ನಿವು
ನಿವೃ-ತ್ತ
ನಿವೃ-ತ್ತ-ರಾಗುವ
ನಿವೇ
ನಿವೇ-ದಿ-ತೆ-ಯನ್ನು
ನಿವೇ-ದಿ-ತೆಗೆ
ನಿವೇ-ದಿ-ತೆಯ
ನಿವೇ-ದಿ-ತೆಯೂ
ನಿವೇ-ದಿ-ಸಿದ
ನಿವೇ-ದಿ-ಸಿದರು
ನಿವೇ-ದಿತಾ
ನಿವೇ-ದಿತಾ-ಳನ್ನು
ನಿವೇ-ದಿಸು
ನಿವೇ-ಶ-ನಕ್ಕೆ
ನಿವೇ-ಶನ
ನಿವೇ-ಶನ-ದ-ಲ್ಲಿ
ನಿವೇ-ಶನ-ವನ್ನು
ನಿವೇ-ಶನವೂ
ನಿಶ್ಚಯ
ನಿಶ್ಚಯ-ಮಾಡಿ-ರುವಾಗ
ನಿಶ್ಚಯ-ವಾಗಿ
ನಿಶ್ಚಯ-ವಾಗಿ-ರುವಾಗ
ನಿಶ್ಚಯ-ವೇ-ನೆಂ-ದರೆ
ನಿಶ್ಚಯಿ-ಸಿದ
ನಿಶ್ಚಯಿ-ಸಿದೆ
ನಿಶ್ಚಯಿಸ-ಬಹುದು
ನಿಶ್ಚಯಿಸಿ
ನಿಶ್ಚಯಿಸಿ-ದರು
ನಿಶ್ಚಲ
ನಿಶ್ಚಲ-ಚಿ-ತ್ತರು
ನಿಶ್ಚಲ-ರಾಗಿ-ಬಿಡು-ತ್ತಿದ್ದರು
ನಿಶ್ಚಿಂ-ತ-ರಾಗಿ
ನಿಶ್ಚಿಂ-ತೆ-ಯಿಂದ
ನಿಶ್ಚೇಷ್ಟಿತ-ನಾಗಿ
ನಿಶ್ಚೇಷ್ಟಿತ-ವಾಗಿ-ರುವಂತೆ
ನಿಶ್ಯಬ್ದ-ವಾ-ಗಿದೆ
ನಿಶ್ಯಬ್ದ-ವಾದ
ನಿಶ್ಯೇಷ-ವಾಗಿ
ನಿಶ್ವಾಸ-ಗಳ-ಲ್ಲಿ
ನಿಷಿದ್ಧ
ನಿಷೇ-ದಾರ್ಥ-ವನ್ನು
ನಿಷೇಧ
ನಿಷೇಧ-ಗಳು
ನಿಷೇಧ-ಗಳೆ-ಲ್ಲಾ
ನಿಷೇಧ-ಗಳೆ-ಲ್ಲಿವೆ
ನಿಷೇಧ-ವನ್ನು
ನಿಷೇಧಾ-ತ್ಮ-ಕ-ವಾ-ದುದು
ನಿಷೇಧಾ-ತ್ಮ-ಕ-ವಾದ
ನಿಷೇಧಿಸ-ತೊಡಗಿದರು
ನಿಷೇಧಿಸಿ
ನಿಷ್ಕ-ಪಟಿ-ಗ-ಳಾದ
ನಿಷ್ಕ-ಪಟಿಯೇ
ನಿಷ್ಕರ್ಷಿ-ಸಿ-ದರು
ನಿಷ್ಕಾ-ಪಟ್ಯದ
ನಿಷ್ಕಾಮ-ಕರ್ಮ
ನಿಷ್ಟುರ-ತೆ-ಯಿಂದ
ನಿಷ್ಠಾ-ವಂತ
ನಿಷ್ಠುರ-ವಾಗಿ
ನಿಷ್ಠುರ-ವಾದ
ನಿಷ್ಣಾತ
ನಿಷ್ಪಲ-ವಾಗು-ವು-ದಿ-ಲ್ಲ
ನಿಷ್ಪ್ರಯೋ-ಜನ-ವಾ-ದರೆ
ನಿಷ್ಪ್ರಯೋ-ಜನ-ವಾಗಿ
ನಿಷ್ಪ್ರಯೋ-ಜನ-ವೆಂದು
ನಿಷ್ಪ್ರಯೋಜ-ಕ-ರಾಗಿ-ದ್ದೇವೆ
ನಿಷ್ಪ್ರಯೋಜಕ-ವಾ-ಯಿತು
ನಿಷ್ಪ್ರಾಪಂಚಿಕ-ತೆ-ಯನ್ನೂ
ನಿಷ್ಫಲ-ವೆಂದು
ನೀ
ನೀಗ್ರೋ
ನೀಗ್ರೋ-ಕುಲ-ದ-ವ-ರಿಗೆ
ನೀಗ್ರೋ-ಗ-ಳನ್ನು
ನೀಗ್ರೋ-ಗ-ಳಿಗೆ
ನೀಗ್ರೋ-ಗಳ-ಲ್ಲ
ನೀಗ್ರೋ-ದ್ವೇಷಿ-ಗ-ಳಾದ
ನೀಗ್ರೋ-ನೀಚ
ನೀಚ-ನತ
ನೀಚ-ನಿಂದಲೂ
ನೀಚ-ವಾದ
ನೀಡ-ಬ-ಲ್ಲದು
ನೀಡ-ಬ-ಲ್ಲರು
ನೀಡ-ಲಿ-ಲ್ಲ
ನೀಡಲಾ-ಯಿತು
ನೀಡಲಿ
ನೀಡಲು
ನೀಡಿ
ನೀಡಿ-ದನು
ನೀಡಿ-ದರು
ನೀಡಿ-ದಳು
ನೀಡಿ-ರ-ಲಿ-ಲ್ಲ
ನೀಡಿದ
ನೀಡಿದೆ
ನೀಡು
ನೀಡು-ತ್ತಿದ್ದ
ನೀಡು-ತ್ತಿದ್ದರು
ನೀಡು-ತ್ತಿದ್ದಳು
ನೀಡು-ವ-ವರೇ
ನೀಡು-ವರು
ನೀಡು-ವು-ದಕ್ಕೆ
ನೀಡು-ವುದು
ನೀಡು-ವುದೇ
ನೀಡೆನಗೆ
ನೀತಂ
ನೀತಿ
ನೀತಿ-ಗ-ಳನ್ನು
ನೀತಿ-ಯನ್ನು
ನೀತಿ-ವಂತರು
ನೀನ-ಲ್ಲವೆ
ನೀನಾರು
ನೀನಿ-ಡುವ
ನೀನಿ-ದ-ನ್ನು
ನೀನೀ
ನೀನೀಗ
ನೀನು
ನೀನೂ
ನೀನೆ
ನೀನೇ
ನೀನೇ-ನಾ-ದರೂ
ನೀನೇನೂ
ನೀನೊಂದು
ನೀನೊಳ್ಳೆ
ನೀನೋರ್ವನೇ
ನೀರ-ನ್ನೆ-ಲ್ಲ
ನೀರಸ
ನೀರಸ-ವಾಗಿ
ನೀರಸ-ವಾದ
ನೀರಿ-ನ-ಲ್ಲಿ
ನೀರಿ-ನಂತೆ
ನೀರು
ನೀರು-ಕೊ-ಟ್ಟು
ನೀರು-ಗುಳ್ಳೆ-ಯಂತೆ
ನೀರೆರೆ-ದಂತೆ
ನೀರ್ಗ-ಲ್ಲ
ನೀರ್ಪನಿ-ಯೊಂದು
ನೀಲಕಂಠ
ನೀಲಕಂಠ-ಮಹಾ-ದೇವ
ನೀಲಾಂಬರ
ನೀಲಿ
ನೀಳ-ವಾದ
ನೀವೀಗ
ನೀವು
ನೀವು-ಗಳೆ-ಲ್ಲ
ನೀವೂ
ನೀವೆ
ನೀವೆ-ನಿತು
ನೀವೆ-ಲ್ಲರೂ
ನೀವೆಂ-ದಾದರೂ
ನೀವೇ
ನೀವೇ-ನಾ-ದರೂ
ನೀವೊಬ್ಬರೇ
ನೀವೋ
ನುಂಗಿ
ನುಗ್ಗಲು
ನುಗ್ಗಾಟವೆಷ್ಟು
ನುಗ್ಗಿ
ನುಗ್ಗಿ-ದರೆ
ನುಗ್ಗು-ತ್ತಿದ್ದಾರೆ
ನುಚ್ಚು-ನೂ-ರಾಗಿ
ನುಚ್ಚು-ನೂ-ರಾಗಿ-ಸುವ
ನುಚ್ಚು-ನೂರು
ನುಡಿ
ನುಡಿ-ಗ-ಳನ್ನು
ನುಡಿ-ಗಳಿವು
ನುಡಿ-ಗಳು
ನುಡಿ-ದನು
ನುಡಿ-ದರು
ನುಡಿ-ದಿದ್ದರೆ
ನುಡಿ-ಯನ್ನು
ನುಡಿ-ಯನ್ನೂ
ನುಡಿ-ಯೊಂದು
ನುಡಿ-ಸಿ-ದರು
ನುಡಿಯೂ
ನುಣುಪಾದ
ನುರಿತ
ನೂ
ನೂಕಾಟ
ನೂಕಿ
ನೂಕಿ-ಕೊಂಡು
ನೂಕು
ನೂಕು-ನುಗ್ಗಲ
ನೂಕು-ನುಗ್ಗಲ-ಲ್ಲಿ
ನೂಕು-ನುಗ್ಗಲುಗ-ಳಾಗು-ತ್ತಿದ್ದವೋ
ನೂಕು-ವಾಗಲೂ
ನೂಕುವ
ನೂರ
ನೂರ-ರ-ಲ್ಲಿ
ನೂರ-ರಷ್ಟು
ನೂರಾರು
ನೂರು
ನೂರ್ಮಡಿ-ಯಾದ
ನೃ
ನೃಣಾಂ
ನೃಣಾಮೇಕೋ
ನೆ
ನೆಂಟ-ರೊಬ್ಬರ
ನೆಂಟ-ರೊಬ್ಬರಿಗೆ
ನೆಂಟರಿಷ್ಟ-ರೆ-ಲ್ಲ
ನೆಂಟರಿಷ್ಟರು
ನೆಂಟರು
ನೆಂಟರು-ಗಳೆ-ಲ್ಲ
ನೆಂದದ್ದೂ
ನೆಂಬೊಬ್ಬ-ನನ್ನು
ನೆಕ್ಕು-ತ್ತಿರು-ವಂತೆ
ನೆಗೆ-ದಾಗ
ನೆಗೆ-ಯಿತು
ನೆಗೆ-ಯು-ತ್ತ
ನೆಗೆ-ಯು-ವಾಗಲೂ
ನೆಗೆ-ಯು-ವುದು
ನೆಗೆದು
ನೆಚ್ಚಿ
ನೆಚ್ಚಿ-ಕೊಂಡಿ-ರುವುದು
ನೆಚ್ಚಿ-ಕೊಂಡು
ನೆಚ್ಚಿ-ದ-ವನು
ನೆಚ್ಚಿ-ದವ-ನನ್ನು
ನೆಚ್ಚಿಗೆ
ನೆಚ್ಚಿನ
ನೆಚ್ಚುಗೆ-ಟ್ಟ-ವರು
ನೆನಪ-ನ್ನು
ನೆನಪಾ-ದರೂ
ನೆನಪಾಗು-ತ್ತದೆ
ನೆನಪಾಗು-ವುದು
ನೆನಪಿ-ರ-ಲಿ-ಲ್ಲ
ನೆನಪಿಗೆ
ನೆನಪಿದೆ
ನೆನಪಿನ
ನೆನಪಿನ-ಲ್ಲಿ-ಟ್ಟಿರ-ಬೇಕು
ನೆನಪಿನ-ಲ್ಲಿಡ-ಬೇಕು
ನೆನಪಿನ-ಲ್ಲಿಡಿ
ನೆನಪಿನ-ಲ್ಲಿಡು
ನೆನಪಿನ-ಲ್ಲಿಡು-ವಂತ-ಹ-ವರು
ನೆನಪು
ನೆನಪೇ
ನೆನಸಿ
ನೆನಹಿನ
ನೆನೆ
ನೆನೆ-ಯು-ತ್ತಿದ್ದರೆ
ನೆನೆ-ಸಿ-ದುದು
ನೆನೆದು
ನೆಪೋ-ಲಿಯ-ನ್
ನೆಪೋ-ಲಿಯ-ನ್ನಿಗೆ
ನೆಪೋ-ಲಿಯ-ನ್-ನನ್ನು
ನೆಯ
ನೆರ-ಳನ್ನು
ನೆರ-ವಾಗ-ಬ-ಲ್ಲೆಯಾ
ನೆರ-ವಾಗದ
ನೆರ-ವಾಗಲು
ನೆರ-ವಿಗೆ
ನೆರಳಿ-ನ-ಲ್ಲಿ-ತ್ತು
ನೆರಳಿ-ನಂತೆ
ನೆರಳು
ನೆರೆ-ದರು
ನೆರೆ-ದಿ-ರುವರು
ನೆರೆ-ದಿದ್ದ
ನೆರೆ-ದಿದ್ದ-ವರು
ನೆರೆ-ದಿದ್ದರು
ನೆರೆ-ಯ-ವ-ರಿಗೆ
ನೆರೆ-ಯ-ವರ
ನೆರೆ-ಯಲು
ನೆರೆ-ಯು-ತ್ತಿದ್ದರು
ನೆರೆ-ಯುವರು
ನೆರೆ-ಹೊ-ರೆಯ
ನೆರೆ-ಹೊರೆ-ಯ-ವ-ರ-ನ್ನು
ನೆರೆ-ಹೊರೆ-ಯ-ವ-ರ-ಲ್ಲಿ
ನೆರೆತೊರೆ-ಗಳಿಂದ
ನೆರೆದ
ನೆರೆದ-ವ-ರ-ಲ್ಲಿ
ನೆರೆದ-ವ-ರಿ-ಗೆ-ಲ್ಲ
ನೆರೆದ-ವ-ರಿಗೆ
ನೆರೆದ-ವ-ರೆ-ಲ್ಲ
ನೆರೆದ-ವ-ರೆ-ಲ್ಲರೂ
ನೆರೆದ-ವ-ರೊ-ಡನೆ
ನೆರೆದ-ವರು
ನೆರೆದು
ನೆಲ
ನೆಲ-ದ-ಲ್ಲಿ
ನೆಲ-ಮು-ಟ್ಟಿ
ನೆಲ-ವನ್ನು
ನೆಲ-ವೆ-ಲ್ಲ
ನೆಲ-ಸ-ತೊಡಗಿದರು
ನೆಲ-ಸಮ
ನೆಲ-ಸಲಿ
ನೆಲ-ಸಲು
ನೆಲ-ಸಿ-ತ್ತೋ
ನೆಲ-ಸಿ-ರುವ
ನೆಲ-ಸಿ-ರುವನು
ನೆಲ-ಸಿತು
ನೆಲ-ಸಿದ
ನೆಲ-ಸಿದೆ
ನೆಲ-ಸಿದ್ದರು
ನೆಲ-ಸು-ವಂತೆ
ನೆಲಕ್ಕೆ
ನೆಲದ
ನೆಲೆ
ನೆಲೆ-ಬೀಡಾಗಿದ್ದ
ನೆಲೆ-ಸಿ-ತ್ತು
ನೆಲೆ-ಸಿ-ಬಿಡು-ತ್ತೇನೆ
ನೆಲೆ-ಸಿತು
ನೆಲೆ-ಸಿದೆ
ನೆಳ-ಲ-ನ್ನು
ನೆವ-ವನ್ನು
ನೇ
ನೇಣು-ಹಾಕಿ-ಬಿಡು-ತ್ತಿದ್ದರು
ನೇತು-ಹಾ-ಕಿ-ರುವ
ನೇತು-ಹಾಕಿ-ದ್ದರು
ನೇತೃ-ತ್ವ-ದ-ಲ್ಲಿ
ನೇತೃ-ತ್ವ-ದ-ಲ್ಲಿಯೇ
ನೇಪ-ಲ್ಸ್
ನೇಪ-ಲ್ಸ್-ನ-ಲ್ಲಿ
ನೇಪ-ಲ್ಸ್-ನಿಂದ
ನೇಮಕ
ನೇಮಕ-ಮಾಡಿದ
ನೇಮಿ-ಸು-ತ್ತಿ-ದ್ದರು
ನೇಯ್ದರು
ನೇರ
ನೇರ-ಮಾಡಲು
ನೇರ-ವಾಗಿ
ನೇರ-ವಾಗಿಯೋ
ನೇರ-ವಾದ
ನೇರ-ವೇರಿತು
ನೇರಳೆ
ನೈ
ನೈಜ
ನೈಜ-ತ್ವ-ವನ್ನು
ನೈಜ-ಸ್ವ-ಭಾವ-ವನ್ನು
ನೈಜಾ-ಮರ
ನೈತಿಕ
ನೈತೃ-ತ್ವ-ದ-ಲ್ಲಿಯೇ
ನೈನಿತಾ-ಲ-ನ್ನು
ನೈನಿತಾ-ಲಿಗೆ
ನೈನಿತಾಲಿ-ನ-ಲ್ಲಿ
ನೈನಿತಾಲಿ-ನಿಂದ
ನೈವೇಹ
ನೈಷ್ಠಿಕ
ನೊಂ
ನೊಡ-ಬಹುದು
ನೊಣ
ನೊಣೆ-ದಿ-ರುವ
ನೊಣೆ-ಯಲು
ನೊರೆತ-ಗಳಿಂದ
ನೋ
ನೋಟ
ನೋಟ-ಕ-ರ-ನ್ನು
ನೋಟ-ನ್ನು
ನೋಟ-ವೊಂದೇ
ನೋಟಕ್ಕೆ
ನೋಡ-ತೊಡಗಿದರು
ನೋಡ-ದಿರಿ
ನೋಡ-ಬ-ಲ್ಲ
ನೋಡ-ಬ-ಲ್ಲ-ವ-ರಾಗಿ-ದ್ದರು
ನೋಡ-ಬ-ಲ್ಲೆ
ನೋಡ-ಬಯಸಿ-ದರು
ನೋಡ-ಬಹು-ದಾಗಿ-ತ್ತು
ನೋಡ-ಬಹುದು
ನೋಡ-ಬೇ-ಕಾ-ದರೆ
ನೋಡ-ಬೇಕಾಗುವುದು
ನೋಡ-ಬೇಕು
ನೋಡ-ಬೇಕೆ
ನೋಡ-ಬೇಕೆ-ನಿ-ಸಿತು
ನೋಡ-ಬೇಕೆಂದು
ನೋಡ-ಬೇಕೆಂದೂ
ನೋಡ-ಬೇಡಿ
ನೋಡ-ಲಾಗು-ವು-ದಿ-ಲ್ಲ
ನೋಡ-ಲಿ-ಲ್ಲ
ನೋಡಬಯ-ಸು-ತ್ತೇನೆ
ನೋಡಲಪೇಕ್ಷಿ-ಸು-ವೆನು
ನೋಡಲಾ-ರದಿ-ದ್ದ-ರಿಂದ
ನೋಡಲಾ-ರದೆ
ನೋಡಲಾಗಿ
ನೋಡಲಾರೆ
ನೋಡಲಿ
ನೋಡಲು
ನೋಡಲು-ಪ-ಕ್ರಮಿಸಿ-ದನು
ನೋಡಲೂ
ನೋಡಿ
ನೋಡಿ-ಕೊ-ಳ್ಳ-ಬೇಕಾಗಿ
ನೋಡಿ-ಕೊ-ಳ್ಳ-ಬೇಕಾಗುವುದು
ನೋಡಿ-ಕೊ-ಳ್ಳಲು
ನೋಡಿ-ಕೊ-ಳ್ಳು-ತ್ತ
ನೋಡಿ-ಕೊ-ಳ್ಳು-ತ್ತಾನೆ
ನೋಡಿ-ಕೊ-ಳ್ಳು-ತ್ತಾರೆ
ನೋಡಿ-ಕೊ-ಳ್ಳು-ತ್ತಿದ್ದ
ನೋಡಿ-ಕೊ-ಳ್ಳು-ತ್ತಿದ್ದರು
ನೋಡಿ-ಕೊ-ಳ್ಳು-ತ್ತಿರು-ವರು
ನೋಡಿ-ಕೊ-ಳ್ಳು-ತ್ತೇನೆಂದು
ನೋಡಿ-ಕೊ-ಳ್ಳು-ವು-ದಕ್ಕೆ
ನೋಡಿ-ಕೊ-ಳ್ಳುವ-ವರು
ನೋಡಿ-ಕೊ-ಳ್ಳುವುದು
ನೋಡಿ-ಕೊ-ಳ್ಳುವುದೇ
ನೋಡಿ-ಕೊಂಡರು
ನೋಡಿ-ಕೊಂಡಿ-ಲ್ಲ
ನೋಡಿ-ಕೊಂಡು
ನೋಡಿ-ಕೊಂಡೆ
ನೋಡಿ-ಕೊಳ್ಳಿ
ನೋಡಿ-ದ-ಮೇಲೆ
ನೋಡಿ-ದ-ರಂತೂ
ನೋಡಿ-ದ-ಲ್ಲದೆ
ನೋಡಿ-ದ-ಲ್ಲದೇ
ನೋಡಿ-ದ-ವ-ರಿಗೆ
ನೋಡಿ-ದ-ವ-ರೆ-ಲ್ಲ
ನೋಡಿ-ದ-ವನು
ನೋಡಿ-ದ-ವರು
ನೋಡಿ-ದಂದಿ-ನಿಂದ
ನೋಡಿ-ದನು
ನೋಡಿ-ದನೊ
ನೋಡಿ-ದರು
ನೋಡಿ-ದರೂ
ನೋಡಿ-ದರೆ
ನೋಡಿ-ದರೊ
ನೋಡಿ-ದಳು
ನೋಡಿ-ದಳೋ
ನೋಡಿ-ದಾಗ
ನೋಡಿ-ದಾಗ-ಲಂತೂ
ನೋಡಿ-ದಿ-ರೇನು
ನೋಡಿ-ದಿರಿ
ನೋಡಿ-ದಿರೊ
ನೋಡಿ-ದೀ-ಯೇನು
ನೋಡಿ-ದು-ದ-ನ್ನು
ನೋಡಿ-ದು-ದರ-ಲ್ಲೆ-ಲ್ಲ
ನೋಡಿ-ದು-ದೆ-ಲ್ಲವೂ
ನೋಡಿ-ದುದ-ನ್ನೆ-ಲ್ಲ
ನೋಡಿ-ದೆ-ನೆಂದೂ
ನೋಡಿ-ದೆನು
ನೋಡಿ-ದೆಯಾ
ನೋಡಿ-ದೆಯೋ
ನೋಡಿ-ದೆವು
ನೋಡಿ-ದೊ-ಡನೆ
ನೋಡಿ-ದೊ-ಡನೆಯೆ
ನೋಡಿ-ದೊ-ಡನೆಯೇ
ನೋಡಿ-ದ್ದಂತೆ
ನೋಡಿ-ದ್ದನು
ನೋಡಿ-ದ್ದರು
ನೋಡಿ-ದ್ದರೂ
ನೋಡಿ-ದ್ದೀರಾ
ನೋಡಿ-ದ್ದೇನೆ
ನೋಡಿ-ಯಾದ-ಮೇಲೆ
ನೋಡಿ-ರ-ಲಿ-ಲ್ಲ
ನೋಡಿ-ರು-ವಿರಾ
ನೋಡಿ-ರು-ವು-ದ-ನ್ನು
ನೋಡಿ-ರು-ವು-ದಾಗಿ
ನೋಡಿ-ರು-ವುದ-ರ-ಲ್ಲೆ-ಲ್ಲ
ನೋಡಿ-ರು-ವೆನು
ನೋಡಿ-ರು-ವೆವು
ನೋಡಿ-ರುವ
ನೋಡಿ-ರುವ-ವ-ರೆ-ಲ್ಲ-ರ-ನ್ನೂ
ನೋಡಿ-ರುವರು
ನೋಡಿ-ಲ್ಲ
ನೋಡಿ-ಲ್ಲವೆ
ನೋಡಿ-ಲ್ಲವೋ
ನೋಡಿ-ವೆವು
ನೋಡಿಕೊ
ನೋಡಿಕೋ
ನೋಡಿದ
ನೋಡಿದೆ
ನೋಡಿದ್ದ
ನೋಡಿದ್ದು
ನೋಡಿಯೂ
ನೋಡು
ನೋಡು-ತ್ತಲೂ
ನೋಡು-ತ್ತಾನೆ
ನೋಡು-ತ್ತಾರೆ
ನೋಡು-ತ್ತಿ-ರಲಿ-ಲ್ಲ
ನೋಡು-ತ್ತಿ-ಲ್ಲವೆ
ನೋಡು-ತ್ತಿದ್ದ
ನೋಡು-ತ್ತಿದ್ದ-ವರು
ನೋಡು-ತ್ತಿದ್ದಂತೆ
ನೋಡು-ತ್ತಿದ್ದನು
ನೋಡು-ತ್ತಿದ್ದರು
ನೋಡು-ತ್ತಿದ್ದವು
ನೋಡು-ತ್ತಿದ್ದಾಗ
ನೋಡು-ತ್ತಿದ್ದೀ-ಯ-ಲ್ಲಾ
ನೋಡು-ತ್ತಿದ್ದುದು
ನೋಡು-ತ್ತಿದ್ದೆ
ನೋಡು-ತ್ತಿದ್ದೆವು
ನೋಡು-ತ್ತಿರ-ಬೇಕು
ನೋಡು-ತ್ತಿರು
ನೋಡು-ತ್ತಿರು-ವಂತೆ
ನೋಡು-ತ್ತಿರು-ವರು
ನೋಡು-ತ್ತಿರು-ವು-ದ-ರಿಂದ
ನೋಡು-ತ್ತಿರು-ವುದೇ
ನೋಡು-ತ್ತಿರು-ವೆ-ಯೇನು
ನೋಡು-ತ್ತಿರು-ವೆನು
ನೋಡು-ತ್ತಿರುವೆ
ನೋಡು-ತ್ತೀಯ
ನೋಡು-ತ್ತೀಯಾ-ಎಂದು
ನೋಡು-ತ್ತೀರಿ
ನೋಡು-ತ್ತೇನೆ
ನೋಡು-ತ್ತೇನೆಯೋ
ನೋಡು-ತ್ತೇವೆ
ನೋಡು-ವಂತಹ
ನೋಡು-ವಂತೆ
ನೋಡು-ವನು
ನೋಡು-ವರು
ನೋಡು-ವರೊ
ನೋಡು-ವರೋ
ನೋಡು-ವಷ್ಟು
ನೋಡು-ವಾಗ
ನೋಡು-ವಿ-ಯೇನು
ನೋಡು-ವಿರಿ
ನೋಡು-ವು-ದ-ಕ್ಕಾಗಿ
ನೋಡು-ವು-ದ-ನ್ನು
ನೋಡು-ವು-ದಕ್ಕೂ
ನೋಡು-ವು-ದಕ್ಕೆ
ನೋಡು-ವು-ದಾದರೆ
ನೋಡು-ವು-ದಿ-ಲ್ಲ
ನೋಡು-ವು-ದಿ-ಲ್ಲ-ವೇನು
ನೋಡು-ವು-ದಿ-ಲ್ಲವೆ
ನೋಡು-ವುದ-ರ-ಲ್ಲಿಯೇ
ನೋಡು-ವುದು
ನೋಡು-ವುದೇ
ನೋಡು-ವುವು
ನೋಡು-ವೆಯೋ
ನೋಡು-ವೆವು
ನೋಡು-ವೆವೊ
ನೋಡುವ
ನೋಡುವೆ
ನೋಡೋಣ
ನೋಬ-ಲ್
ನೋಬ-ಲ್-ಳನ್ನು
ನೋಯಿ-ಸು-ವು-ದ-ನ್ನು
ನೋವಿ-ನಿಂದ
ನೋವು
ನೋವುಂಟು-ಮಾಡ-ಬೇಡಿ
ನೋವೂ
ನೌ
ನೌಕಾ-ಪಡೆ-ಯನ್ನು
ನೌಕಾ-ಸಭಾಂಗಣ-ದ-ಲ್ಲಿ
ನೌಕೆ-ಯ-ಲ್ಲಿ
ನೌಬ-ತ್ತಿನ
ನ್ಯಾಯ
ನ್ಯಾಯ-ರ-ತ್ನ
ನ್ಯಾಯ-ವಾದ
ನ್ಯಾಯ-ಶಾ-ಸ್ತ್ರ
ನ್ಯಾಯವೋ
ನ್ಯಾಯಾಧಿ-ಪತಿ
ನ್ಯಾಯಾಧಿ-ಪತಿ-ಗ-ಳಾದ
ನ್ಯಾಯಾಧಿ-ಪತಿ-ಗಳು
ನ್ಯಾಷನ-ಲ್
ನ್ಯೂ
ಪ
ಪಂ
ಪಂಕ್ತಿ-ಗಳಾಗಿ-ದ್ದವು
ಪಂಗ-ಡದ
ಪಂಗಡ
ಪಂಗಡ-ಗ-ಳನ್ನು
ಪಂಗಡ-ಗಳ
ಪಂಗಡ-ದ-ವರು
ಪಂಗಡ-ವ-ಲ್ಲ
ಪಂಗಡ-ವನ್ನು
ಪಂಗಡ-ವನ್ನೂ
ಪಂಗಡ-ವಿದೆ
ಪಂಗಡಕ್ಕೆ
ಪಂಚ-ಭೂತ-ಗ-ಳನ್ನು
ಪಂಚ-ಭೂತ-ಗ-ಳಿಗೆ
ಪಂಚ-ಭೂತ-ಗಳ
ಪಂಚತಪ-ವನ್ನು
ಪಂಚವಟಿ
ಪಂಚವಟಿಯ
ಪಂಚಾಂಗ-ವನ್ನು
ಪಂಚೆ
ಪಂಚೆ-ಯಂ-ಚನ್ನು
ಪಂಚೆಯ
ಪಂಚೇಂದ್ರಿಯ-ಗಳ
ಪಂಜಾಬಿ
ಪಂಜಾಬಿನ
ಪಂಜಾಬು
ಪಂಜಾಬ್
ಪಂಡ-ರಿನಾಥ
ಪಂಡಿ-ತನ
ಪಂಡಿ-ತರ
ಪಂಡಿ-ತರು
ಪಂಡಿ-ತೆಯರ
ಪಂಡಿ-ತ್
ಪಂಡಿತ
ಪಂಡಿತ-ನಾಗಲಿ
ಪಂಡಿತ-ನಾಗಿ-ರುವೆ
ಪಂಡಿತ-ನಾಗಿದ್ದ
ಪಂಡಿತ-ನಿಗೆ
ಪಂಡಿತ-ನಿದ್ದ
ಪಂಡಿತ-ರಾಗಿ-ದ್ದರು
ಪಂಡಿತ-ರಾದ
ಪಂಡಿತ-ರಿ-ದ್ದರು
ಪಂಡಿತ-ರಿಗೆ
ಪಂಡಿತ-ರೆ-ಲ್ಲ
ಪಂಡಿತ-ರೆ-ಲ್ಲರೂ
ಪಂಡಿತ-ರೊ-ಡನೆ
ಪಂಡಿತಾನಾಂ
ಪಂಥ
ಪಂಥ-ಗ-ಳಿಗೆ
ಪಂಥ-ಗಳ-ಲ್ಲಿಯೂ
ಪಂಥ-ದ-ವ-ರಾಗಿ-ರ-ಬಹುದು
ಪಂಥ-ದ-ವರು
ಪಂಥ-ವನ್ನು
ಪಂಥಕ್ಕೆ
ಪಂಥದ
ಪಂಥವೂ
ಪಕ್ಕ-ದ-ಲ್ಲಿ
ಪಕ್ಕ-ದ-ಲ್ಲೆ
ಪಕ್ಕಕ್ಕೆ
ಪಕ್ಕದ
ಪಕ್ಕದ-ಲ್ಲಿ-ರುವ
ಪಕ್ಕದ-ಲ್ಲಿದ್ದ
ಪಕ್ಕದ-ಲ್ಲಿಯೂ
ಪಕ್ಕದ-ಲ್ಲಿಯೇ
ಪಕ್ಷ-ಗ-ಳಿದ್ದವು
ಪಕ್ಷ-ದ-ಲ್ಲಿ
ಪಕ್ಷ-ಪಾತ
ಪಕ್ಷ-ಪಾತಿ
ಪಕ್ಷ-ಪಾತಿ-ಯ-ನ್ನಾಗಿ
ಪಕ್ಷ-ವನ್ನು
ಪಕ್ಷದ
ಪಕ್ಷಿ-ಗಳ
ಪಚ್ಚೈಯಪ್ಪ
ಪಟ-ಗ-ಳನ್ನು
ಪಟ-ಗಳ-ಲ್ಲಿ
ಪಟು-ಗಳು
ಪಟ್ಟ
ಪಟ್ಟ-ಣ-ಗ-ಳನ್ನು
ಪಟ್ಟ-ಣ-ಗಳ-ಲ್ಲೆ-ಲ್ಲ
ಪಟ್ಟ-ಣ-ಗಳು
ಪಟ್ಟ-ಣ-ವನ್ನು
ಪಟ್ಟ-ಣಕ್ಕೂ
ಪಟ್ಟ-ಣದ
ಪಟ್ಟ-ಣವೇ
ಪಟ್ಟ-ದ್ದಾಗಿ-ದ್ದವು
ಪಟ್ಟ-ಶಿಷ್ಯ-ನಾದ
ಪಟ್ಟಂತೆ
ಪಟ್ಟಣ
ಪಟ್ಟದ
ಪಟ್ಟರೂ
ಪಟ್ಟಿ
ಪಟ್ಟಿ-ದ್ದೆ-ಲ್ಲ
ಪಠಿ-ಸು-ತ್ತಿ-ದ್ದರೂ
ಪಠಿ-ಸುವ
ಪಠಿಸಿ
ಪಠ್ಯ
ಪಡ-ಬೇಕಾಗುವುದು
ಪಡ-ಬೇಕಾಗುವುದೆಂ-ದಿದ್ದರು
ಪಡ-ಬೇಕು
ಪಡಬೇಕಾಯಿತ-ಲ್ಲ
ಪಡಿ-ಸಿ-ದರು
ಪಡಿ-ಸಿ-ದು-ದ-ರಿಂದ
ಪಡು-ತ್ತಾ-ರೆಂದು
ಪಡು-ತ್ತಿ-ತ್ತು
ಪಡು-ತ್ತಿದ್ದರೂ
ಪಡು-ತ್ತಿರು-ವನು
ಪಡು-ತ್ತಿರು-ವೆನು
ಪಡು-ತ್ತಿರುವ
ಪಡು-ವಂತಹ
ಪಡು-ವಂತಾಗು-ವುದು
ಪಡು-ವಿರಿ
ಪಡು-ವು-ದಕ್ಕೆ
ಪಡು-ವು-ದಿ-ಲ್ಲ
ಪಡೆ
ಪಡೆ-ದ-ಮೇಲೆ
ಪಡೆ-ದ-ವ-ರಿ-ದ್ದರು
ಪಡೆ-ದ-ವ-ರಿಗೆ
ಪಡೆ-ದ-ವನು
ಪಡೆ-ದ-ವರು
ಪಡೆ-ದ-ವರೇ
ಪಡೆ-ದಂತೆ
ಪಡೆ-ದಂದಿ-ನಿಂದ
ಪಡೆ-ದನು
ಪಡೆ-ದರು
ಪಡೆ-ದರೂ
ಪಡೆ-ದರೆ
ಪಡೆ-ದರೊ
ಪಡೆ-ದಿ-ದ್ದಾರೆ
ಪಡೆ-ದಿ-ರುವರು
ಪಡೆ-ದಿದ್ದ
ಪಡೆ-ದಿದ್ದರೆ
ಪಡೆ-ದಿದ್ದರೋ
ಪಡೆ-ದಿದ್ದೆ
ಪಡೆ-ದು-ಕೊ-ಳ್ಳು-ತ್ತಿದ್ದರು
ಪಡೆ-ದು-ಕೊ-ಳ್ಳುವ
ಪಡೆ-ದು-ಕೊಂಡ
ಪಡೆ-ದು-ಕೊಂಡದ್ದು
ಪಡೆ-ದು-ಕೊಂಡರು
ಪಡೆ-ದು-ಕೊಂಡರೊ
ಪಡೆ-ದು-ಕೊಂಡು
ಪಡೆ-ದು-ದ-ನ್ನು
ಪಡೆ-ದುದು
ಪಡೆ-ದೆ-ನೆಂದೂ
ಪಡೆ-ಯ-ಬ-ಲ್ಲ
ಪಡೆ-ಯ-ಬ-ಲ್ಲೆ-ಯಾ-ದರೆ
ಪಡೆ-ಯ-ಬಹು-ದೆಂದು
ಪಡೆ-ಯ-ಬಹುದು
ಪಡೆ-ಯ-ಬೇ-ಕಾ-ದರೆ
ಪಡೆ-ಯ-ಬೇಕು
ಪಡೆ-ಯ-ಬೇಕೆಂದು
ಪಡೆ-ಯ-ಬೇಕೆಂಬ
ಪಡೆ-ಯ-ಲೋಸುಗ
ಪಡೆ-ಯದೇ
ಪಡೆ-ಯಲು
ಪಡೆ-ಯಿತು
ಪಡೆ-ಯಿರಿ
ಪಡೆ-ಯು-ತ್ತಾನೆ
ಪಡೆ-ಯು-ತ್ತಾರೆ
ಪಡೆ-ಯು-ತ್ತಿ-ರಲಿ-ಲ್ಲ
ಪಡೆ-ಯು-ತ್ತಿದ್ದರು
ಪಡೆ-ಯು-ವು-ದ-ನ್ನು
ಪಡೆ-ಯು-ವು-ದಕ್ಕೆ
ಪಡೆ-ಯು-ವುದ-ಕ್ಕೋ-ಸ್ಕರ
ಪಡೆ-ಯು-ವುದಕ್ಕಾಗಲಿ
ಪಡೆ-ಯು-ವುದು
ಪಡೆ-ಯು-ವುದೆಂದ-ರೇನು
ಪಡೆ-ಯು-ವುದೇ
ಪಡೆ-ಯುವ
ಪಡೆ-ಯುವ-ರೆಂದು
ಪಡೆ-ಯುವ-ವ-ರೆಗೆ
ಪಡೆ-ಯುವನು
ಪಡೆ-ಯುವೆ
ಪಡೆ-ಯುವೆನೋ
ಪಡೆ-ಯೆ-ಲ್ಲಾ
ಪಡೆ-ಯೋಣ
ಪಡೆದ
ಪಡೆದು
ಪಡೆದೆ
ಪಡೇ-ಯು-ತ್ತೇನೆ
ಪತಂ-ಜಲಿ
ಪತಂಗ
ಪತಂಜ-ಲಿಯ
ಪತಾಕೆ-ಯನ್ನು
ಪತಿ
ಪತಿ-ತ-ರಾಗಿ-ಬಿ-ಟ್ಟಿ-ರುವರು
ಪತಿ-ತನು
ಪತಿ-ತಳಾಗಿ-ರಲಿ
ಪತಿ-ಭಾ-ವಂತ
ಪತಿ-ಯನ್ನು
ಪತಿ-ವ್ರ-ತೆಯ-ರಾದ
ಪತಿಯ
ಪಥ
ಪಥ-ಗಳು
ಪಥ-ಗಳೆಷ್ಟೋ
ಪಥ-ದ-ಲ್ಲಿ
ಪಥದ
ಪಥದ-ಲ್ಲಿಯೂ
ಪಥಿ
ಪಥಿ-ಕರು
ಪಥ್ಯ
ಪಥ್ಯ-ಮಿತಿ
ಪಥ್ಯದ
ಪದ
ಪದ-ಗ-ಳನ್ನು
ಪದ-ಗ-ಳಿಗೆ
ಪದ-ಗಳ
ಪದ-ಗಳ-ನ್ನೆ-ಲ್ಲ
ಪದ-ಗಳ-ಲ್ಲಿ
ಪದ-ಗಳಿ-ರು-ವುವು
ಪದ-ಗಳು
ಪದ-ಗಳೇ
ಪದ-ತಲ-ದ-ಲ್ಲಿ
ಪದ-ತಳ-ದ-ಲ್ಲಿ
ಪದ-ತಳಕ್ಕೆ
ಪದ-ದ-ಲ್ಲಿ
ಪದ-ದಲಿ-ತ-ರಾಗಿ-ದ್ದಾರೆ
ಪದ-ದಳಿ-ತರ
ಪದ-ದಳಿತ-ರ-ನ್ನು
ಪದ-ದಳಿತ-ರಿಂದ
ಪದ-ದಳಿತ-ರಿಗೆ
ಪದ-ದಿಂದ
ಪದ-ಪ್ರ-ಯೋಗ
ಪದ-ವನ್ನು
ಪದ-ವಾಗಲೀ
ಪದ-ವಿ-ಯ-ಲ್ಲಿದ್ದ-ರೇನು
ಪದ-ವಿಗೆ
ಪದ-ವೀಧ-ರರು
ಪದ-ಸಂಘಾ-ತದ
ಪದವಿ
ಪದಾಘಾತ-ವನ್ನು
ಪದೇ
ಪದ್ಧತಿ
ಪದ್ಧತಿ-ಯನ್ನು
ಪದ್ಮ
ಪದ್ಮ-ದಂತೆ
ಪದ್ಮಾ-ಸ-ನ-ದ-ಲ್ಲಿ
ಪದ್ಯ-ಗಳು
ಪದ್ಯ-ದಿಂದಲೇ
ಪದ್ಯ-ವನ್ನು
ಪದ್ಯ-ವನ್ನೋದಿದ
ಪದ್ಯದ
ಪಯ-ಣದ
ಪಯಸಾ-ಮರ್ಣವ
ಪರ
ಪರ-ಚರ್ಚೆ
ಪರ-ತರಂ
ಪರ-ದೇ-ಶಕ್ಕೆ
ಪರ-ದೇಶ-ಗ-ಳಿಗೆ
ಪರ-ದೇಶ-ಗಳ-ಲ್ಲಿ
ಪರ-ದೇಶ-ದ-ಲ್ಲಿ
ಪರ-ದೇಶ-ದ-ವರು
ಪರ-ದೇಶ-ದಿಂದ
ಪರ-ದೇಶದ
ಪರ-ದೇಶಿ-ಯರಾ-ರ-ನ್ನೂ
ಪರ-ದೇಶಿಯ-ರಿಗೂ
ಪರ-ದೇಶಿಯೂ
ಪರ-ದೇಶೀಯ
ಪರ-ದೇಶೀಯ-ರಿಗಾಗಲಿ
ಪರ-ದೇಶೀಯರ
ಪರ-ದೇಶೀಯರು
ಪರ-ನಿಂದೆ
ಪರ-ಬ್ರಹ್ಮ
ಪರ-ಬ್ರಹ್ಮ-ನ-ಲ್ಲಿ
ಪರ-ಬ್ರಹ್ಮ-ನನ್ನೇ
ಪರ-ಬ್ರಹ್ಮ-ನೆ-ಡೆಗೆ
ಪರ-ಬ್ರಹ್ಮನ
ಪರ-ಮ-ಕುಡಿಗೆ
ಪರ-ಮ-ಗುರು-ಗ-ಳಿಗೆ
ಪರ-ಮ-ಪವಿ-ತ್ರ
ಪರ-ಮ-ಪ್ರಿಯ-ವಾದ
ಪರ-ಮ-ಪ್ರೀತಿ
ಪರ-ಮ-ಪ್ರೇಮ
ಪರ-ಮ-ಪ್ರೇಮ-ಸ್ವ-ರೂಪ
ಪರ-ಮ-ಶಾಂತಿ
ಪರ-ಮ-ಸ-ತ್ಯ-ವನ್ನು
ಪರ-ಮ-ಹಂಸ
ಪರ-ಮ-ಹಂಸ-ರ-ನ್ನು
ಪರ-ಮ-ಹಂಸ-ರ-ಲ್ಲಿ
ಪರ-ಮ-ಹಂಸ-ರಿಂದ
ಪರ-ಮ-ಹಂಸ-ರಿಗೆ
ಪರ-ಮ-ಹಂಸರ
ಪರ-ಮ-ಹಂಸರು
ಪರ-ಮ-ಹಂಸರೆ
ಪರ-ಮಾ-ತ್ಮ
ಪರ-ಮಾ-ತ್ಮ-ನನ್ನು
ಪರ-ಮಾ-ತ್ಮ-ನನ್ನೂ
ಪರ-ಮಾ-ತ್ಮನ
ಪರ-ಮಾಣುವಿ-ನ-ಲ್ಲಿ
ಪರ-ಮಾಧಿ-ಕಾರವೂ
ಪರ-ಮಾನಂದ
ಪರ-ಮಾನಂದ-ವಾ-ಯಿತು
ಪರ-ಮಾನ್ನ-ವನ್ನು
ಪರ-ಮಾಪ್ತ
ಪರ-ಮಾರ್ಥ-ದಿಂದ
ಪರ-ಮಾರ್ಥವೂ
ಪರ-ಮಾರ್ಥಿಕ
ಪರ-ಮಾವಧಿ
ಪರ-ಮಾಶ್ಚರ್ಯ-ವಾ-ಯಿತು
ಪರ-ಮಿದಮದಃ
ಪರ-ಮೋ-ಧರ್ಮ
ಪರ-ಮೋಚ್ಚ
ಪರ-ರ-ನ್ನು
ಪರ-ರಿ-ಗೋ-ಸ್ಕರ
ಪರ-ರಿಂದ
ಪರ-ರಿಗೆ
ಪರ-ಲೋಕ-ದ-ಲ್ಲಿ
ಪರ-ಲೋಕದ
ಪರ-ವಶ-ರಾಗಿ
ಪರ-ವಾಗಿ
ಪರ-ಸ್ಪರ
ಪರ-ಹಿ-ತ-ಕ್ಕಾಗಿ
ಪರ-ಹಿ-ತ-ಕ್ಕಾಗಿಯೇ
ಪರ-ಹಿತ
ಪರ-ಹಿತ-ವನ್ನು
ಪರ-ಹಿತ-ವೇ-ತಕ್ಕೆ
ಪರ-ಹಿತೇಚ್ಛೆಯು
ಪರಂ
ಪರಂ-ಪ-ರೆ-ಯ-ಲ್ಲಿ
ಪರಂ-ಪ-ರೆಯ
ಪರಂ-ಪ-ರೆಯಿಂದ
ಪರಂ-ಪರೆ-ಯನ್ನು
ಪರಮ
ಪರರ
ಪರಾ
ಪರಾ-ಕಾಷ್ಠತೆ-ಯನ್ನು
ಪರಾ-ಕಾಷ್ಠೆ
ಪರಾ-ಕಾಷ್ಠೆ-ಯ-ಲ್ಲಿ
ಪರಾ-ಕಾಷ್ಠೆ-ಯನ್ನು
ಪರಾ-ಕ್ರಮ-ವನ್ನು
ಪರಾ-ಕ್ರಮ-ವಾಗಿ-ರ-ಬಹುದು
ಪರಾ-ಜಿ-ತರು
ಪರಾ-ಯಣ
ಪರಾ-ಯಣ-ರಾದ
ಪರಾ-ಯಣತೆ
ಪರಾ-ಯಣರ
ಪರಾ-ರಿ-ಯಾ-ದರು
ಪರಾ-ರಿಯಾ-ದನು
ಪರಾ-ವಾಗಿ
ಪರಾಗ
ಪರಿ-ಕ್ರಮದ
ಪರಿ-ಗಣಿ-ಸು-ತ್ತಿ-ದ್ದರು
ಪರಿ-ಗಣಿ-ಸು-ವಂತೆಯೇ
ಪರಿ-ಗಣಿ-ಸುವರು
ಪರಿ-ಚಾರ-ಕ-ಗಳಾಗಿ-ರ-ಬಹು-ದಾಗಿ-ತ್ತು
ಪರಿ-ತ್ಯಾಗ
ಪರಿ-ತ್ಯಾಗ-ಮಾಡಿದ
ಪರಿ-ನಿರ್ಯಾಣಕ್ಕೆ
ಪರಿ-ಪೂರ್ಣ-ನಾಗಿ
ಪರಿ-ಪೂರ್ಣ-ವಾ-ದುದು
ಪರಿ-ಪೂರ್ಣ-ವಾಗಿ-ರು-ವು-ದ-ನ್ನು
ಪರಿ-ಪೂರ್ಣ-ವಾದ
ಪರಿ-ಪೂರ್ಣತೆ
ಪರಿ-ಪೂರ್ಣಾ-ತ್ಮ
ಪರಿ-ಮಿತಿ
ಪರಿ-ವರ್ತಿ-ಸಿತು
ಪರಿ-ವಾರ
ಪರಿ-ವಾರ-ದ-ವ-ರೆ-ಲ್ಲರೂ
ಪರಿ-ವಾರ-ದ-ವರು
ಪರಿ-ವಾರ-ದೊ-ಡನೆ
ಪರಿ-ವಾರ-ವನ್ನೆ-ಲ್ಲ
ಪರಿ-ವೆಯೂ
ಪರಿ-ವೆಯೆ
ಪರಿ-ವೆಯೇ
ಪರಿ-ಶೀಲ-ನೆ-ಮಾಡಿ
ಪರಿ-ಶೀಲಿ-ಸುವು-ದಾಗಿ-ತ್ತು
ಪರಿ-ಶೀಲಿಸಿ
ಪರಿ-ಶೀಲಿಸಿ-ದಾಗ
ಪರಿ-ಶುದ್ಧ
ಪರಿ-ಶುದ್ಧ-ನಾದ
ಪರಿ-ಶುದ್ಧ-ರಾಗುವರು
ಪರಿ-ಶುದ್ಧ-ರಾದ
ಪರಿ-ಶುದ್ಧ-ವಾ-ದದ್ದು
ಪರಿ-ಶುದ್ಧ-ವಾಗಲು
ಪರಿ-ಶುದ್ಧ-ವಾಗಿ-ದ್ದಂತಹ
ಪರಿ-ಶುದ್ಧ-ವಾಗಿ-ರ-ಬೇಕು
ಪರಿ-ಶುದ್ಧ-ವಾದ
ಪರಿ-ಶುದ್ಧ-ವಾದು-ದ-ನ್ನೇ
ಪರಿ-ಶುದ್ಧಾ-ತ್ಮ-ನಿಂದ
ಪರಿ-ಶುದ್ಧಾ-ತ್ಮರು
ಪರಿ-ಶ್ರಮ
ಪರಿ-ಶ್ರಮ-ಪಟ್ಟ-ರೆಂಬು-ದ-ನ್ನು
ಪರಿ-ಸ್ಥಿತಿ
ಪರಿ-ಸ್ಥಿತಿ-ಗಳು
ಪರಿ-ಸ್ಥಿತಿ-ಯ-ಲ್ಲಿ-ದ್ದಾಗ
ಪರಿ-ಸ್ಥಿತಿ-ಯ-ಲ್ಲಿಯೂ
ಪರಿ-ಸ್ಥಿತಿ-ಯನ್ನು
ಪರಿ-ಸ್ಥಿತಿ-ಯಿಂದಲೂ
ಪರಿ-ಹರಿ-ಸಿ-ಕೊ-ಳ್ಳಲು
ಪರಿ-ಹರಿ-ಸಿ-ಕೊ-ಳ್ಳು-ವು-ದಕ್ಕೆ
ಪರಿ-ಹರಿ-ಸಿ-ಕೊ-ಳ್ಳುವುದು
ಪರಿ-ಹರಿ-ಸಿ-ಕೊಳ್ಳ-ಬೇಕೆಂದು
ಪರಿ-ಹರಿ-ಸಿ-ದರು
ಪರಿ-ಹರಿ-ಸು-ತ್ತಿ-ದ್ದನು
ಪರಿ-ಹರಿ-ಸು-ತ್ತಿದೆ
ಪರಿ-ಹರಿ-ಸು-ವು-ದಕ್ಕೆ
ಪರಿ-ಹರಿ-ಸು-ವುದು
ಪರಿ-ಹರಿ-ಸುವ
ಪರಿ-ಹಾ-ರಕ್ಕೆ
ಪರಿ-ಹಾರ
ಪರಿ-ಹಾರ-ಗಳ
ಪರಿ-ಹಾರ-ನಿಧಿಗೆ
ಪರಿ-ಹಾರ-ವನ್ನು
ಪರಿ-ಹಾರ-ವಾ-ದವು
ಪರಿ-ಹಾರ-ವಾಗು-ವು-ದಿ-ಲ್ಲ
ಪರಿ-ಹಾರ-ವಾದಂ-ತಾ-ಯಿತು
ಪರಿ-ಹಾರದ
ಪರಿ-ಹಾರೋಪಾಯ-ವನ್ನು
ಪರಿ-ಹಾರೋಪಾಯ-ವಿ-ತ್ತರು
ಪರಿ-ಹಾಸ್ಯ
ಪರಿ-ಹಾಸ್ಯ-ಗ-ಳನ್ನು
ಪರಿ-ಹಾಸ್ಯ-ಗಳು
ಪರಿಚಯ
ಪರಿಚಯ-ಮಾಡಿ-ಸಿ-ದರು
ಪರಿಚಯ-ವನ್ನು
ಪರಿಚಯ-ವಾ-ದಂತೆ
ಪರಿಚಯ-ವಾ-ದದ್ದು
ಪರಿಚಯ-ವಾ-ದುದು
ಪರಿಚಯ-ವಾ-ಯಿತು
ಪರಿಚಯ-ವಾಗ-ಬೇಕಾಗಿ-ರುವುದೊ
ಪರಿಚಯ-ವಾಗ-ಬೇಕೆಂದು
ಪರಿಚಯ-ವಾಗಿ
ಪರಿಚಯ-ವಾಗಿ-ತ್ತು
ಪರಿಚಯ-ವಾಗಿ-ದ್ದರು
ಪರಿಚಯ-ವಾಗಿ-ರುವ
ಪರಿಚಯ-ವಾಗು-ತ್ತ
ಪರಿಚಯ-ವಾದ
ಪರಿಚಯ-ವಾದ-ಮೇಲೆ
ಪರಿಚಯ-ವಿ-ತ್ತು
ಪರಿಚಯ-ವಿ-ರ-ಲಿ-ಲ್ಲ
ಪರಿಚಯ-ವಿ-ರುವ
ಪರಿಚಯ-ವಿ-ಲ್ಲ-ದಾಗ
ಪರಿಚಯ-ವಿದೆ
ಪರಿಚಯ-ವಿದ್ದ
ಪರಿಚಯ-ವಿದ್ದ-ವ-ರೊಬ್ಬರು
ಪರಿಚಯ-ವಿರ-ಬೇಕು
ಪರಿಚಯ-ಸ್ಥರ
ಪರಿಚಯ-ಸ್ಥರು
ಪರಿಚಯವೂ
ಪರಿಚಯವೇ
ಪರಿಚಾಲ-ನವೇ
ಪರಿಚಿ-ತ-ನಾದ
ಪರಿಚಿ-ತರಾಗು-ತ್ತ
ಪರಿಚಿ-ತರಾಗು-ವ-ರೆಂದು
ಪರಿಚಿತ-ನೆಂದು
ಪರಿಚಿತ-ನೊಬ್ಬ
ಪರಿಚಿತ-ರಾದ
ಪರಿಚಿತ-ರಾದ-ವರು
ಪರಿಚಿತ-ವಾದ
ಪರಿಣತ-ಗೊಳಿ-ಸು-ವು-ದಕ್ಕೆ
ಪರಿಣತ-ರಾದ
ಪರಿಣತ-ರಾದ-ಮೇಲೆ
ಪರಿಣತ-ವಾ-ಗಿದೆ
ಪರಿಣತ-ವಾದ
ಪರಿಣತಿಯ
ಪರಿಣತಿಯು
ಪರಿಣಮಿ-ಸಿ-ರುವೆ
ಪರಿಣಮಿ-ಸಿದೆ
ಪರಿಣಮಿ-ಸು-ವುದು
ಪರಿಣಾಮ
ಪರಿಣಾಮ-ಕಾರಿ-ಯಾ-ಯಿತು
ಪರಿಣಾಮ-ಕಾರಿ-ಯಾಗಿ
ಪರಿಣಾಮ-ಕಾರಿ-ಯಾಗಿ-ತ್ತು
ಪರಿಣಾಮ-ಕಾರಿ-ಯಾಗು-ತ್ತವೆ
ಪರಿಣಾಮ-ಕಾರಿ-ಯಾಗುವ
ಪರಿಣಾಮ-ಕಾರಿ-ಯಾಗುವುದು
ಪರಿಣಾಮ-ಕಾರಿ-ಯಾದ
ಪರಿಣಾಮ-ಗಳು
ಪರಿಣಾಮ-ದಿಂದ
ಪರಿಣಾಮ-ವ-ನ್ನುಂಟು-ಮಾಡಿತು
ಪರಿಣಾಮ-ವನ್ನು
ಪರಿಣಾಮ-ವಾಗಿ
ಪರಿಣಾಮ-ವಾಗಿಯೇ
ಪರಿಣಾಮದ
ಪರಿಣಾಮವು
ಪರಿಣಾಮವೇ
ಪರಿತಪಿ-ಸು-ತ್ತೇನೆ
ಪರಿಧಿ
ಪರಿಪಾಲನೆ
ಪರಿಪಾಲಿ-ಸು-ತ್ತಿ-ಲ್ಲ-ವೆಂದೂ
ಪರಿಪಾಲಿ-ಸುವುದ-ರ-ಲ್ಲಿ
ಪರಿಮಳದ್ದು
ಪರಿವೃ-ತ-ವಾಗಿ-ತ್ತು
ಪರಿವ್ರಾಜ-ಕ-ನಾಗಿ
ಪರಿವ್ರಾಜ-ಕ-ರಾಗಿ
ಪರಿವ್ರಾಜಕ
ಪರೀಕ್ಷಾ
ಪರೀಕ್ಷಿ-ಸಲು
ಪರೀಕ್ಷಿ-ಸು-ತ್ತಾ-ಹೋದ
ಪರೀಕ್ಷಿ-ಸು-ತ್ತಾಳೆ
ಪರೀಕ್ಷಿ-ಸು-ತ್ತಿದ್ದೆ
ಪರೀಕ್ಷಿ-ಸು-ವು-ದ-ಕ್ಕಾಗಿ
ಪರೀಕ್ಷಿಸ-ಬೇಕು
ಪರೀಕ್ಷಿಸ-ಬೇಕೆ-ನಿ-ಸಿತು
ಪರೀಕ್ಷಿಸ-ಬೇಕೆಂದು
ಪರೀಕ್ಷಿಸಿ
ಪರೀಕ್ಷಿಸಿ-ದಾಗ
ಪರೀಕ್ಷಿಸಿ-ದ್ದಕ್ಕೆ
ಪರೀಕ್ಷೆ
ಪರೀಕ್ಷೆ-ಗಳಿ-ಗಾಗಿ
ಪರೀಕ್ಷೆ-ಯ-ಲ್ಲಿ
ಪರೀಕ್ಷೆ-ಯನ್ನು
ಪರೀಕ್ಷೆಗೆ
ಪರೀಕ್ಷೆಯ
ಪರುಸ-ವೇದಿ
ಪರೋಕ್ಷ-ವಾಗಿ
ಪರೋಕ್ಷ-ವಾದ
ಪರೋದ್ದೇಶ-ದಿಂದ
ಪರೋಪ-ಕಾರ
ಪರ್ಯಂ-ತರ
ಪರ್ಯಂತ-ರವೂ
ಪರ್ಯಟ-ನ-ದ-ಲ್ಲಿ
ಪರ್ಯಟ-ನದ
ಪರ್ಯಟನೆ
ಪರ್ಯಟನೆ-ಯ-ಲ್ಲಿ
ಪರ್ಯವಸಾನ-ವಾಗು-ವುದು
ಪರ್ಯವಸಾನ-ವಾಗು-ವುದೆಂದೂ
ಪರ್ಯಾಯ-ದ್ವೀಪಕ್ಕೆ
ಪರ್ಯಾಲೋ-ಚಿಸಿ-ದರು
ಪರ್ಯಾಲೋಚಿಸ-ಬೇಕಾಗಿದೆ
ಪರ್ಯಾವ-ಸಾನ-ವಾಗಿ-ತ್ತು
ಪರ್ವ-ಗಳ-ಲ್ಲಿ
ಪರ್ವ-ತಕ್ಕೂ
ಪರ್ವ-ತಕ್ಕೆ
ಪರ್ವತ
ಪರ್ವತ-ಗಳ
ಪರ್ವತ-ಗಳ-ಲ್ಲಿರು-ವು-ವೋಪರ್ವತದ
ಪರ್ವತ-ದಷ್ಟೇ
ಪರ್ವತ-ದಿಂದ
ಪರ್ವತ-ನ-ಗ-ರಿಗೆ
ಪರ್ವತ-ವನ್ನು
ಪರ್ವತ-ವನ್ನೇ
ಪರ್ವತ-ವಿ-ರು-ವು-ದ-ರಿಂದ
ಪರ್ವತ-ಶಿಖರ-ಗಳಿಂದ
ಪರ್ವತ-ಶಿಖರ-ಗಳು
ಪರ್ವತಾರೋಹಿ-ಗ-ಳಿಗೆ
ಪರ್ವತೋ-ಮಯ
ಪರ್ವತೋಪಮ
ಪರ್ಸಿಯ
ಪಲಟೈ-ನ್
ಪಲಾ-ಯನ
ಪಲಾ-ಯನ-ವಾ-ಯಿತು
ಪವಾ-ಹಾರಿ
ಪವಾ-ಹಾರಿ-ಬಾಬರ
ಪವಾ-ಹಾರಿ-ಬಾಬಾ
ಪವಾ-ಹಾರಿ-ಬಾಬಾ-ರ-ನ್ನು
ಪವಾ-ಹಾರಿ-ಬಾಬಾ-ರ-ವರು
ಪವಾ-ಹಾರಿ-ಬಾಬಾ-ರಿಂದ
ಪವಾ-ಹಾರಿ-ಬಾಬಾ-ರೆಂಬ
ಪವಾ-ಹಾರಿ-ಬಾಬಾರ
ಪವಾಡ-ಗ-ಳನ್ನು
ಪವಾಡ-ಗಳೂ
ಪವಿ-ತ್ರ
ಪವಿ-ತ್ರ-ತ-ಮರು
ಪವಿ-ತ್ರ-ತಮ
ಪವಿ-ತ್ರ-ತೆ-ಯ-ನ್ನಾಗಲೀ
ಪವಿ-ತ್ರ-ತೆ-ಯನ್ನು
ಪವಿ-ತ್ರ-ತೆ-ಯಾ-ಗಿದೆಯೋ
ಪವಿ-ತ್ರ-ತೆಗೆ
ಪವಿ-ತ್ರ-ತೆಯ
ಪವಿ-ತ್ರ-ನಾಗು-ವು-ದಕ್ಕೆ
ಪವಿ-ತ್ರ-ನಾದ
ಪವಿ-ತ್ರ-ಭೂಮಿ
ಪವಿ-ತ್ರ-ವಾ-ಗಿದೆ
ಪವಿ-ತ್ರ-ವಾ-ದುದು
ಪವಿ-ತ್ರ-ವಾಗಿ
ಪವಿ-ತ್ರ-ವಾಗಿ-ತ್ತು
ಪವಿ-ತ್ರ-ವಾಗಿ-ರು-ವು-ದೆ-ಲ್ಲ
ಪವಿ-ತ್ರ-ವಾಗಿ-ರು-ವುದೋ
ಪವಿ-ತ್ರ-ವಾಗಿ-ರುವ
ಪವಿ-ತ್ರ-ವಾಗು-ವುದು
ಪವಿ-ತ್ರ-ವಾದ
ಪವಿ-ತ್ರತೆ
ಪವಿ-ತ್ರಾ-ತ್ಮ
ಪವಿ-ತ್ರಾ-ತ್ಮ-ರಾಗುವುದು
ಪವಿ-ತ್ರಾ-ತ್ಮನ
ಪವಿ-ತ್ರಾ-ತ್ಮರು
ಪವಿ-ತ್ರೋ-ತ್ತ-ಮರು
ಪಶು
ಪಶು-ಗಳಂತೆ
ಪಶು-ಪಕ್ಷಿ
ಪಶು-ಪಕ್ಷಿ-ಗಳ
ಪಶು-ಪಕ್ಷಿ-ಗಳು
ಪಶು-ಪತಿ-ನಾಥ-ಬೋ-ಸ್
ಪಶು-ಪತಿ-ಮತಂ
ಪಶು-ಸಮಾ-ನಕ್ಕೆ
ಪಶ್ಚಾ-ತ್ತಾಪ
ಪಶ್ಚಾ-ತ್ತಾಪ-ಕರ-ವಾ-ದುದು
ಪಶ್ಚಾ-ತ್ತಾಪ-ವಾಗಿ
ಪಶ್ಚಿಮ
ಪಶ್ಚಿಮ-ದ-ಲ್ಲಿರುವ
ಪಶ್ಚಿಮ-ದಿಕ್ಕಿಗೆ
ಪಶ್ಚಿಮ-ದೇಶ-ಗ-ಳಿಗೆ
ಪಶ್ಚಿಮ-ದ್ವಾರದ
ಪಶ್ಚಿಮಕ್ಕೆ
ಪಶ್ಚಿಮದ
ಪಹ-ರೆಯ-ವ-ರಾಗಿ-ದ್ದಾಗ
ಪಹರೆ-ಯ-ವನು
ಪಾ
ಪಾಂಚ-ಭೌತಿಕ
ಪಾಂಡ-ವರ
ಪಾಂಡಿ-ತ್ಯ
ಪಾಂಡಿ-ತ್ಯ-ದಿಂದ
ಪಾಂಡಿ-ತ್ಯ-ಪೂರಿತ
ಪಾಂಡಿ-ತ್ಯ-ವನ್ನು
ಪಾಂಡಿ-ತ್ಯ-ವಿ-ತ್ತು
ಪಾಂಡಿ-ತ್ಯ-ವಿ-ರುವ
ಪಾಂಡಿ-ತ್ಯ-ವೆ-ಲ್ಲ-ವನ್ನೂ
ಪಾಂಡಿ-ತ್ಯಕ್ಕೆ
ಪಾಂಡಿ-ತ್ಯವು
ಪಾಂಡಿಚೇ-ರಿಗೆ
ಪಾಂಡುಪ-ಲ್
ಪಾಂಡುರಂಗ
ಪಾಂಡೆ
ಪಾಂಡ್ಯ-ರಿಗೆ
ಪಾಂಪುರ
ಪಾಂಪೆಯ
ಪಾಂಬ-ನ್
ಪಾಂಬ-ನ್ನಿ-ನ-ಲ್ಲಿ
ಪಾಂಬ-ನ್ನಿ-ನಿಂದ
ಪಾಂಬ-ನ್ನಿಗೆ
ಪಾಂಬ-ನ್ಗೆ
ಪಾಕ-ಶಾ-ಸ್ತ್ರ-ದ-ಲ್ಲಿ
ಪಾಕ-ಶಾ-ಸ್ತ್ರ-ದ-ಲ್ಲಿಯೂ
ಪಾಚಿ
ಪಾಚೀ
ಪಾಟಲೀ-ಪು-ತ್ರ-ವೆಂಬ
ಪಾಠ
ಪಾಠ-ಕ-ರಿಗೆ
ಪಾಠ-ಕ್ಕಿಂತ
ಪಾಠ-ಗ-ಳನ್ನು
ಪಾಠ-ಗಳ-ನ್ನೆ-ಲ್ಲ
ಪಾಠ-ದ-ಲ್ಲಿ
ಪಾಠ-ಪ್ರ-ವ-ಚನ-ಗ-ಳನ್ನು
ಪಾಠ-ವನ್ನು
ಪಾಠ-ವನ್ನೆ-ಲ್ಲ
ಪಾಠ-ಶಾಲೆ
ಪಾಠ-ಶಾಲೆ-ಗ-ಳಿಗೆ
ಪಾಠ-ಶಾಲೆ-ಯ-ಲ್ಲಿ
ಪಾಠ-ಶಾಲೆ-ಯನ್ನು
ಪಾಠ-ಶಾಲೆಗೆ
ಪಾಠ-ಶಾಲೆಯ
ಪಾಠಕ್ಕೆ
ಪಾಠದ
ಪಾಠವೇ
ಪಾಡ-ನ್ನು
ಪಾಡುಪ-ಲ್
ಪಾಡೇಕೆ
ಪಾಡೇನು
ಪಾಡ್ಯಮಿ
ಪಾಣಿನಿಯ
ಪಾತಿ-ವ್ರ-ತ್ಯದ
ಪಾದ-ಕಮಲ-ಗಳ-ಲ್ಲಿ
ಪಾದ-ಕಮಲ-ಗಳಂತಿ-ರುವ
ಪಾದ-ಗ-ಳನ್ನು
ಪಾದ-ಗ-ಳಿಗೆ
ಪಾದ-ಗಳು
ಪಾದ-ಗಳುಳ್ಳ
ಪಾದ-ಚಾರಿ-ಗಳಾಗಿ
ಪಾದ-ಚಾರಿ-ಗಳಾಗಿಯೇ
ಪಾದ-ಚಾರಿ-ಯಾಗಿ
ಪಾದ-ಪದ್ಮ-ಗ-ಳನ್ನು
ಪಾದ-ಪದ್ಮ-ಗಳ-ಲ್ಲಿ
ಪಾದ-ಪದ್ಮ-ದ-ಲ್ಲಿಯೂ
ಪಾದ-ಪೂಜೆ
ಪಾದ-ಪೂಜೆ-ಯನ್ನು
ಪಾದ-ಸೇವೆ
ಪಾದ-ಸೇವೆಯೆ
ಪಾದ-ಸ್ಪರ್ಶ-ದಿಂದ
ಪಾದದಡಿ
ಪಾದದಡಿ-ಯ-ಲ್ಲಿರುವ
ಪಾದಧೂಳಿ-ಯನ್ನು
ಪಾದಧೂಳಿ-ಯಿಂದ
ಪಾದಾ-ರವಿಂದ-ದ-ಲ್ಲಿ
ಪಾದು-ಕೆಯ
ಪಾದುಕೆ-ಗಳ
ಪಾದ್ರಿ
ಪಾದ್ರಿ-ಗ-ಳನ್ನು
ಪಾದ್ರಿ-ಗ-ಳಾದ
ಪಾದ್ರಿ-ಗ-ಳಿಗೆ
ಪಾದ್ರಿ-ಗಳ
ಪಾದ್ರಿ-ಗಳು
ಪಾದ್ರಿ-ಗಳೂ
ಪಾದ್ರಿ-ಗಳೊಂದಿಗೆ
ಪಾದ್ರಿಯ
ಪಾದ್ರಿಯು
ಪಾನ
ಪಾನ-ಕ-ವನ್ನು
ಪಾನ-ಕದ
ಪಾನ-ಮಾಡಿ
ಪಾಪ
ಪಾಪ-ಕರ-ವಾದ
ಪಾಪ-ಗ-ಳಿಗೆ
ಪಾಪ-ಗಳಿಂದ
ಪಾಪ-ದಿಂದ
ಪಾಪ-ಪುಣ್ಯ-ಗ-ಳೆಂಬ
ಪಾಪ-ವ-ಲ್ಲ
ಪಾಪ-ವ-ಲ್ಲದೆ
ಪಾಪ-ವನ್ನು
ಪಾಪ-ವನ್ನೇ
ಪಾಪ-ವೆಂಬುದು
ಪಾಪಕ್ಕೆ
ಪಾಪದ
ಪಾಪವು
ಪಾಪವೂ
ಪಾಪವೆ
ಪಾಪವೇ
ಪಾಪಿ
ಪಾಪಿ-ಗ-ಳನ್ನು
ಪಾಪಿ-ಗ-ಳಿಗೆ
ಪಾಪಿ-ಗಳು
ಪಾಪಿ-ಗಳೆ
ಪಾಪಿ-ಗಳೆಂದೂ
ಪಾಪಿ-ಯಾಗಲಿ
ಪಾಪಿ-ಯಾಗುವ
ಪಾಪಿಯೇ
ಪಾಮ-ರರು
ಪಾಯಸ-ವನ್ನು
ಪಾರ-ವಿ-ರ-ಲಿ-ಲ್ಲ
ಪಾರಂ-ಗತ-ನಾಗು-ವನು
ಪಾರಮಾರ್ಥಿಕ
ಪಾರಾ-ದಾರು
ಪಾರಾ-ಯಣ
ಪಾರಾ-ಯಣ-ದ-ಲ್ಲಿ
ಪಾರಾ-ಯಣ-ಮಾಡಿ
ಪಾರಾ-ಯಿತು
ಪಾರಾಗ-ಬಹು-ದೆಂದು
ಪಾರಾಗ-ಬಹುದು
ಪಾರಾಗ-ಬೇಕು
ಪಾರಾಗ-ಬೇಕೆಂ-ದಾಗಲಿ
ಪಾರಾಗ-ಬೇಕೆಂದು
ಪಾರಾಗಲೆ-ತ್ನಿಸಿ-ದಳು
ಪಾರಾಗು
ಪಾರಾಗು-ವಂತೆ
ಪಾರಾಗು-ವನು
ಪಾರಾಗು-ವರು
ಪಾರಾಗು-ವು-ದಕ್ಕೆ
ಪಾರಾಗು-ವುದು
ಪಾರಿವಾಳ
ಪಾರಿವಾಳ-ಗ-ಳನ್ನು
ಪಾರಿವಾಳ-ಗಳು
ಪಾರು
ಪಾರು-ಮಾಡಿ-ರುವರು
ಪಾರು-ಮಾಡು-ವುದೋ
ಪಾರು-ಮಾಡುವ
ಪಾರು-ಮಾಡೆಂದು
ಪಾರ್ಕಿ-ನ-ಲ್ಲಿ-ರುವ
ಪಾರ್ಟಿ
ಪಾರ್ಟಿ-ಗ-ಳನ್ನು
ಪಾರ್ಟಿ-ಗ-ಳಿದ್ದುವು
ಪಾರ್ಟಿ-ಯನ್ನು
ಪಾರ್ಟಿಯ
ಪಾರ್ಥ
ಪಾರ್ಥ-ಸಾರ-ಥಿಯೂ
ಪಾರ್ಥ-ಸಾರಥಿ
ಪಾರ್ಥ-ಸಾರಿಥಿ
ಪಾರ್ವತಿ
ಪಾರ್ಶ್ವ-ವಾಯು
ಪಾರ್ಸಿ
ಪಾರ್ಸಿ-ಭಾಷೆಯ
ಪಾರ್ಸಿ-ಮತ
ಪಾಲ-ನೆ-ಯ-ಲ್ಲಿ
ಪಾಲಾ-ಗಿದೆ
ಪಾಲಾಗು-ವುವು
ಪಾಲಿ-ಸು-ತ್ತಿ-ದ್ದೇನೆ
ಪಾಲಿ-ಸು-ವು-ದ-ನ್ನು
ಪಾಲಿ-ಸು-ವು-ದಕ್ಕೆ
ಪಾಲಿ-ಸು-ವುದು
ಪಾಲಿಗೂ
ಪಾಲಿತಾನ
ಪಾಲಿಸ-ದ-ವ-ರಿಗೆ
ಪಾಲಿಸಿ
ಪಾಲಿಸಿ-ದರು
ಪಾಲಿಸು-ವ-ವ-ನ-ಲ್ಲ
ಪಾಲು
ಪಾಲೂ
ಪಾಳ
ಪಾಳಿ
ಪಾಳು
ಪಾಳು-ಮನೆ-ಯ-ಲ್ಲಿ
ಪಾಳು-ಮನೆಯ
ಪಾಳೆ-ಗಾರನೋ
ಪಾಶ-ದಿಂದ
ಪಾಶ್ಚಾ-ತ್ಯ
ಪಾಶ್ಚಾ-ತ್ಯ-ದೇ-ಶಕ್ಕೆ
ಪಾಶ್ಚಾ-ತ್ಯ-ದೇಶ-ಗ-ಳಿಗೆ
ಪಾಶ್ಚಾ-ತ್ಯ-ದೇಶ-ಗಳ
ಪಾಶ್ಚಾ-ತ್ಯ-ದೇಶ-ಗಳ-ಲ್ಲಿದ್ದಾಗ
ಪಾಶ್ಚಾ-ತ್ಯ-ದೇಶ-ಗಳಿಂದ
ಪಾಶ್ಚಾ-ತ್ಯ-ದೇಶ-ದ-ಲ್ಲಿ
ಪಾಶ್ಚಾ-ತ್ಯ-ದೇಶದ
ಪಾಶ್ಚಾ-ತ್ಯ-ರ-ನ್ನು
ಪಾಶ್ಚಾ-ತ್ಯ-ರ-ಲ್ಲಿ
ಪಾಶ್ಚಾ-ತ್ಯ-ರದು
ಪಾಶ್ಚಾ-ತ್ಯ-ರಾದ
ಪಾಶ್ಚಾ-ತ್ಯ-ರಿ-ಗಿಂತ
ಪಾಶ್ಚಾ-ತ್ಯ-ರಿ-ಗೆ-ಲ್ಲ
ಪಾಶ್ಚಾ-ತ್ಯ-ರಿಂದ
ಪಾಶ್ಚಾ-ತ್ಯ-ರಿಂದಲೂ
ಪಾಶ್ಚಾ-ತ್ಯ-ರಿಗೂ
ಪಾಶ್ಚಾ-ತ್ಯ-ರಿಗೆ
ಪಾಶ್ಚಾ-ತ್ಯದ
ಪಾಶ್ಚಾ-ತ್ಯರ
ಪಾಶ್ಚಾ-ತ್ಯರು
ಪಾಶ್ಚಾ-ತ್ಯರೇ
ಪಾಶ್ಯಾ-ತ್ಯರ
ಪಾಸ-ದೀನ
ಪಾಸಾದ-ಮೇಲೆ
ಪಾಸಿಟಿ-ವಿ-ಸ್ಟ್
ಪಾಸು-ಮಾಡಿ-ದ-ಮೇಲೆ
ಪಾಸು-ಮಾಡಿ-ದರೆ
ಪಾಸು-ಮಾಡುವ
ಪಿ
ಪಿಂ
ಪಿಂಜ-ರಾದಿವ
ಪಿಂಡಾಂಡ
ಪಿಂಡಾಂಡ-ಗಳೆ-ರಡು
ಪಿಂಡಾಂಡ-ವಾದ
ಪಿಟೀಲಿ-ನ-ಲ್ಲಿ
ಪಿಟೀಲು-ವಾದ-ಕನ
ಪಿತೃ-ಪೂಜೆ
ಪಿನಾಂಗಿ-ನಿಂದ
ಪಿನಾಂಗ್
ಪಿಪಾಯಿ-ಯನ್ನೇ
ಪಿಪಾಸು-ಗಳಿ-ಗಾಗಿ
ಪಿಪಾಸು-ಗಳು
ಪಿರಮಿಡ್ಡು-ಗಳು
ಪಿರಾವಿ
ಪಿಳ್ಳೈ
ಪಿಶಾಚಿ
ಪಿಶಾಚಿ-ಗಳಂತೆ
ಪಿಶಾಚಿ-ಯಂತೆ
ಪಿಶಾಚಿ-ಯಾಗಿ
ಪಿಶಾಚಿಯ
ಪಿಸು-ಮಾ-ತಿನಂತಿದೆ
ಪೀ
ಪೀಠ-ದಿಂದೆದ್ದು
ಪೀಠ-ದೊಂದಿಗೆ
ಪೀಠದ
ಪೀಠಿಕೆ
ಪೀಠಿಕೆ-ಯನ್ನು
ಪೀಡಿ-ಸ-ತೊಡಗಿತು
ಪೀಡಿ-ಸು-ತ್ತಿದೆ
ಪೀಡಿ-ಸುವುದ-ರ-ಲ್ಲಿ
ಪೀಡೆ-ಗಳ
ಪೀಡೆ-ಯಿಂದ
ಪೀಡೆಯೇ
ಪೀಪ-ಲ್ಸ್
ಪೀಸಾ
ಪೀಸಾ-ನ-ಗರ-ಗಳ-ಲ್ಲಿ-ರುವ
ಪು
ಪುಂಗಿ-ನಾದ-ವನ್ನು
ಪುಕ್ಕ-ಗಳು
ಪುಜೆ-ಯ-ನ್ನಾ-ದರೂ
ಪುಜೆ-ಯನ್ನು
ಪುಟ-ಗ-ಳನ್ನು
ಪುಟ-ವನ್ನೇ
ಪುಡಿ
ಪುಡಿ-ಕಾಸು
ಪುಡಿ-ಕಾಸು-ಗ-ಳನ್ನು
ಪುಡಿ-ಮಾಡ-ಬ-ಲ್ಲದು
ಪುಣ್ಯ
ಪುಣ್ಯ-ಕಥೆ-ಗ-ಳಿಗೆ
ಪುಣ್ಯ-ಕಥೆ-ಗಳ
ಪುಣ್ಯ-ಕಾಲ-ದ-ಲ್ಲಿ
ಪುಣ್ಯ-ಕೆಲಸ-ಗ-ಳನ್ನು
ಪುಣ್ಯ-ಕ್ಷೇ-ತ್ರ-ದ-ಲ್ಲಿ
ಪುಣ್ಯ-ಗಳಿಸಿ
ಪುಣ್ಯ-ಗಳು
ಪುಣ್ಯ-ಗ್ರಂಥ-ಗಳ-ಲ್ಲಿ
ಪುಣ್ಯ-ಚಾರಿ-ತ್ರ್ಯದ
ಪುಣ್ಯ-ದಿಂದ
ಪುಣ್ಯ-ದಿನ
ಪುಣ್ಯ-ದಿನ-ದ-ಲ್ಲಿ
ಪುಣ್ಯ-ನದಿ-ಯ-ಲ್ಲಿ
ಪುಣ್ಯ-ನಾ-ಮವೂ
ಪುಣ್ಯ-ಭೂಮಿ
ಪುಣ್ಯ-ಭೂಮಿ-ಯನ್ನು
ಪುಣ್ಯ-ಭೂಮಿ-ಯಾ-ಗಿದೆ
ಪುಣ್ಯ-ವಂತ
ಪುಣ್ಯ-ವತಿ-ಯಾಗುವಳು
ಪುಣ್ಯ-ವನ್ನು
ಪುಣ್ಯ-ವಿದೆ
ಪುಣ್ಯ-ವಿಶೇಷ-ವಿರ-ಬೇಕು
ಪುಣ್ಯ-ಸ್ಥಳಕ್ಕೆ
ಪುಣ್ಯ-ಸ್ಮೃ-ತಿಯ
ಪುಣ್ಯದ
ಪುಣ್ಯಾ-ತ್ಮರಿ-ದ್ದರು
ಪುಣ್ಯಾ-ತ್ಮರು
ಪುಣ್ಯಾ-ವಕಾಶ
ಪುನಃ
ಪುನೀತ-ನಾದೆ
ಪುರ-ಜ-ನರ
ಪುರ-ಜ-ನರು
ಪುರ-ಜನ-ರಿಗೆ
ಪುರ-ಜನ-ರೆ-ಲ್ಲ
ಪುರ-ದ-ಲ್ಲಿ
ಪುರ-ಪ್ರ-ಮುಖ-ರ-ನ್ನೆ-ಲ್ಲ
ಪುರ-ಪ್ರ-ಮುಖರು
ಪುರ-ವಾಸಿ-ಗಳೆ-ಲ್ಲ
ಪುರ-ಸ್ಕಾರ
ಪುರಾ-ತನ
ಪುರಾ-ತನ-ಕಾಲ-ದಿಂದಲೂ
ಪುರಾ-ತನ-ಕಾಲದ
ಪುರಾ-ತನ-ವಾ-ದುದು
ಪುರಾ-ತನ-ವಾಗಿ-ಲ್ಲವೋ
ಪುರಾ-ತನ-ವಾದ
ಪುರಾಣ
ಪುರಾಣ-ಗಳ
ಪುರಾಣ-ಗಳ-ಲ್ಲಿ
ಪುರಾಣ-ಗಳಿಂದ
ಪುರಾಣ-ಗಳು
ಪುರಾಣ-ದ-ಲ್ಲಿ
ಪುರಾಣ-ದ-ಲ್ಲಿ-ರುವ
ಪುರಾಣ-ವನ್ನು
ಪುರಾಣದ
ಪುರಾವೆ-ಗಳ
ಪುರುಷ
ಪುರುಷ-ಕಾರ
ಪುರುಷ-ಕಾರದ
ಪುರುಷ-ತ್ವ-ವೆ-ಲ್ಲ
ಪುರುಷ-ನನ್ನು
ಪುರುಷ-ನಾಗಿ
ಪುರುಷ-ನಿಗೆ
ಪುರುಷ-ನೊಂದಿಗೆ
ಪುರುಷ-ರ-ನ್ನು
ಪುರುಷ-ರ-ನ್ನೂ
ಪುರುಷ-ರಾಗಿ-ದ್ದರು
ಪುರುಷ-ರಿ-ದ್ದರು
ಪುರುಷ-ರಿ-ರುವರು
ಪುರುಷ-ರಿಗೆ
ಪುರುಷ-ರೆ-ಲ್ಲರೂ
ಪುರುಷ-ರೆಂದು
ಪುರುಷ-ರೆಂಬುದೇನೋ
ಪುರುಷ-ಸಿಂಹ
ಪುರುಷ-ಸಿಂಹ-ರ-ನ್ನು
ಪುರುಷ-ಸಿಂಹ-ರಾಗಿ
ಪುರುಷ-ಸಿಂಹ-ರಾಗುವಂತಹ
ಪುರುಷ-ಸಿಂಹ-ರಾಗುವುದೇ
ಪುರುಷ-ಸಿಂಹರು
ಪುರುಷನ
ಪುರುಷನೂ
ಪುರುಷರು
ಪುರುಷರೂ
ಪುರುಷರೆ
ಪುರೈ-ಸಿದ
ಪುರೈಸಿ
ಪುರೋ-ಹಿತ
ಪುರೋ-ಹಿತ-ರ-ನ್ನಾಗಿ
ಪುರೋ-ಹಿತ-ರ-ನ್ನು
ಪುರೋ-ಹಿತನು
ಪುರೋ-ಹಿತರ
ಪುರೋ-ಹಿತರು
ಪುರೋಭಿ-ವೃದ್ಧಿಗೆ
ಪುರ್ಣ-ವಾದ
ಪುರ್ವಾ-ಶ್ರಮದ
ಪುಲಕಿ-ತ-ವಾ-ಯಿತು
ಪುಳಕಿ-ತ-ರಾ-ದರು
ಪುಷ್ಕಳ
ಪುಷ್ಟವೃಷ್ಟಿ-ಯನ್ನು
ಪುಷ್ಟಿ
ಪುಷ್ಟಿ-ಕರ-ವಾದ
ಪುಷ್ಟಿ-ಗೊಳಿಸಿ-ದರು
ಪುಷ್ಪ-ಗ-ಳನ್ನು
ಪುಷ್ಪ-ವನ್ನು
ಪುಷ್ಪವೃಷ್ಟಿ-ಯನ್ನು
ಪುಷ್ಪವೃಷ್ಟಿ-ಯನ್ನೇ
ಪುಷ್ಪಾ-ದಿ-ಗ-ಳನ್ನು
ಪುಷ್ಪಾಂ-ಜಲಿ-ಯನ್ನು
ಪುಷ್ಯಮಾ-ಸದ
ಪೂ
ಪೂಜ-ಕರ
ಪೂಜಾ
ಪೂಜಾ-ಗೃಹ-ದ-ಲ್ಲಿ
ಪೂಜಾ-ಗೃಹದ
ಪೂಜಾ-ಗೃಹವು
ಪೂಜಾ-ದಿ-ಗ-ಳನ್ನು
ಪೂಜಾ-ನಂ-ತರ
ಪೂಜಾ-ಮಂದಿ-ರ-ದಿಂದ
ಪೂಜಾ-ಮಂದಿ-ರ-ವಾಗಿ
ಪೂಜಾ-ಮಂದಿ-ರಕ್ಕೆ
ಪೂಜಾ-ಮಂದಿ-ರದ
ಪೂಜಾ-ರಿ-ಗ-ಳಿಗೆ
ಪೂಜಾ-ರಿ-ಗಳ
ಪೂಜಾ-ರಿ-ಗಳೊ-ಡನೆ
ಪೂಜಾ-ರಿ-ಗಾಗಿ
ಪೂಜಾ-ರಿ-ಯಾಗಿ
ಪೂಜಾ-ರಿಯ
ಪೂಜಾ-ಸ-ನ-ದ-ಲ್ಲಿ
ಪೂಜಿ-ಸಲಾರೆ
ಪೂಜಿ-ಸಿದ
ಪೂಜಿ-ಸು-ತ್ತಿ-ದ್ದರು
ಪೂಜಿ-ಸು-ತ್ತಿ-ದ್ದಾಗ
ಪೂಜಿ-ಸು-ತ್ತಿ-ರು-ವೆವು
ಪೂಜಿ-ಸು-ತ್ತಿ-ರುವರು
ಪೂಜಿ-ಸು-ತ್ತಿದ್ದ
ಪೂಜಿ-ಸು-ವು-ದಕ್ಕೆ
ಪೂಜಿ-ಸು-ವು-ದಿ-ಲ್ಲ
ಪೂಜಿ-ಸು-ವುದು
ಪೂಜಿ-ಸುವ
ಪೂಜಿ-ಸುವ-ರೆಂದು
ಪೂಜಿ-ಸುವ-ವರು
ಪೂಜಿ-ಸುವರು
ಪೂಜಿ-ಸುವರೋ
ಪೂಜಿ-ಸುವುದಕ್ಕಿಂತ
ಪೂಜಿ-ಸುವೆ
ಪೂಜಿಸ-ಬಹುದು
ಪೂಜಿಸಿ
ಪೂಜಿಸಿ-ದರು
ಪೂಜೆ
ಪೂಜೆ-ಗ-ಳನ್ನು
ಪೂಜೆ-ಗ-ಳಾದ
ಪೂಜೆ-ಗಿಂತ
ಪೂಜೆ-ಗೋ-ಸ್ಕರ
ಪೂಜೆ-ಮಾ-ಡದೆ
ಪೂಜೆ-ಮಾಡ-ಬೇಕು
ಪೂಜೆ-ಮಾಡು-ತ್ತಿದ್ದರು
ಪೂಜೆ-ಯ-ಲ್ಲಿ
ಪೂಜೆ-ಯಂತೆಯೇ
ಪೂಜೆ-ಯನ್ನ
ಪೂಜೆ-ಯನ್ನು
ಪೂಜೆ-ಯಾದ
ಪೂಜೆ-ಯಿಂದ
ಪೂಜೆ-ಯೆ-ಲ್ಲ
ಪೂಜೆಗೆ
ಪೂಜೆಯ
ಪೂಜೆಯೂ
ಪೂಜ್ಯ
ಪೂಜ್ಯ-ದೃಷ್ಟಿ-ಯಿಂದ
ಪೂಜ್ಯ-ಭಾವ-ದಿಂದ
ಪೂಜ್ಯ-ರೊ-ಡನೆ
ಪೂಜ್ಯನ್ತೇ
ಪೂಜ್ಯರೆ
ಪೂಜ್ಯರೇ
ಪೂನಾ-ದ-ಲ್ಲಿ
ಪೂನಾ-ದಿಂದ
ಪೂರಕ-ವಾಗಿ-ವೆಯೇ
ಪೂರಿ
ಪೂರಿ-ಗೋ-ವರ್ಧನ
ಪೂರಿ-ತ-ವಾಗಿಯೂ
ಪೂರೈ-ಸ-ಲಿ-ಲ್ಲ
ಪೂರೈ-ಸಿತು
ಪೂರೈ-ಸಿದ
ಪೂರೈ-ಸು-ತ್ತಿ-ದ್ದರು
ಪೂರೈಕೆ
ಪೂರೈಸ-ಬೇಕೆಂದು
ಪೂರೈಸಿ
ಪೂರೈಸಿ-ಕೊ-ಳ್ಳುವರು
ಪೂರೈಸಿ-ಕೊಂಡು
ಪೂರೈಸಿ-ದರು
ಪೂರೈಸಿ-ದಳು
ಪೂರೈಸಿ-ಬಿಡ-ಬೇಕೆಂದೂ
ಪೂರೈಸು
ಪೂರೈಸು-ವು-ದಕ್ಕೆ
ಪೂರ್ಣ
ಪೂರ್ಣ-ಕುಂಭ
ಪೂರ್ಣ-ಚಂದ್ರ
ಪೂರ್ಣ-ಜೀವನ
ಪೂರ್ಣ-ಜ್ಞಾನ
ಪೂರ್ಣ-ತೆ-ಯ-ನ್ನಾಗಲಿ
ಪೂರ್ಣ-ತೆ-ಯ-ಲ್ಲಿ
ಪೂರ್ಣ-ತೆ-ಯನ್ನು
ಪೂರ್ಣ-ತೆಗೆ
ಪೂರ್ಣ-ತೆಯ
ಪೂರ್ಣ-ದೃಷ್ಟಿಯೂ
ಪೂರ್ಣ-ನಾಗ-ಬ-ಲ್ಲ
ಪೂರ್ಣ-ಪ್ರಕಾಶ-ಸ್ವ-ರೂಪರು
ಪೂರ್ಣ-ಮಾಡಿ-ಕೊ-ಳ್ಳಲು
ಪೂರ್ಣ-ವಾ-ದುದು
ಪೂರ್ಣ-ವಾ-ಯಿತು
ಪೂರ್ಣ-ವಾಗಿ
ಪೂರ್ಣ-ವಾಗು-ವ-ವ-ರೆಗೂ
ಪೂರ್ಣ-ವಾದ
ಪೂರ್ಣ-ವಿಕಾಸ
ಪೂರ್ಣತೆ
ಪೂರ್ಣಾ-ನಂದ
ಪೂರ್ಣಾವ-ಸ್ಥೆ
ಪೂರ್ತಿ
ಪೂರ್ತಿ-ಗೊಳಿ-ಸಲು
ಪೂರ್ತಿ-ಯಾಗಿ
ಪೂರ್ವ
ಪೂರ್ವ-ಕ-ಲ್ಪಿತ
ಪೂರ್ವ-ಜನ್ಮ-ಗ-ಳನ್ನು
ಪೂರ್ವ-ಜನ್ಮ-ಗಳ
ಪೂರ್ವ-ಜರ
ಪೂರ್ವ-ದ-ಲ್ಲಿ
ಪೂರ್ವ-ದ-ಲ್ಲಿದ್ದ
ಪೂರ್ವ-ದಿಂದಲೂ
ಪೂರ್ವ-ದಿಗಂ-ತದ
ಪೂರ್ವ-ಪಕ್ಷ-ಪರ-ವಾಗಿ
ಪೂರ್ವ-ಪಕ್ಷ-ವನ್ನು
ಪೂರ್ವ-ಪರಿಚಿತ-ವಾದ
ಪೂರ್ವ-ಬಂಗಾಳ
ಪೂರ್ವ-ಬಂಗಾಳ-ದ-ಲ್ಲಿ
ಪೂರ್ವ-ಬಂಗಾಳ-ದಿಂದ
ಪೂರ್ವ-ಬಂಗಾಳ-ವನ್ನು
ಪೂರ್ವ-ಬಂಗಾಳಕ್ಕೆ
ಪೂರ್ವ-ಬಂಗಾಳದ
ಪೂರ್ವ-ಭಾವಿ-ಯಾಗಿ
ಪೂರ್ವ-ಮೀ-ಮಾಂಸಾ
ಪೂರ್ವ-ಮುಖ-ವಾಗಿ
ಪೂರ್ವ-ಸಂ-ಸ್ಕಾರವೇ
ಪೂರ್ವಾ-ಚಾರ
ಪೂರ್ವಾ-ಚಾರ-ದ-ಲ್ಲಿ
ಪೂರ್ವಾ-ಚಾರ-ಪರಾ-ಯಣ-ರಿಗೆ
ಪೂರ್ವಾ-ಚಾರ-ಪ್ರಿಯರು
ಪೂರ್ವಾ-ಪರ
ಪೂರ್ವಾ-ಪರೌ
ಪೂರ್ವಾ-ಶ್ರಮದ
ಪೂರ್ವಾಭಿ-ಮುಖ-ವಾಗಿ
ಪೂರ್ವಾರ್ಜಿತ
ಪೂರ್ವಿ-ಕ-ರ-ನ್ನು
ಪೂರ್ವಿ-ಕ-ರಿಗೆ
ಪೂರ್ವಿ-ಕರ
ಪೂರ್ವಿ-ಕರಿ-ತ್ತ
ಪೂರ್ವಿ-ಕರು
ಪೂರ್ವಿಕ-ರ-ಲ್ಲಿ
ಪೃಥಿವ್ಯಾಮಿವ
ಪೃಥ್ವಿ
ಪೃಥ್ವಿ-ಯನ್ನೆ-ಲ್ಲ
ಪೃಥ್ವಿ-ರಾಜನ
ಪೆ
ಪೆದ್ದ
ಪೆನಂಸುಲಾರ್
ಪೆರಿ
ಪೆರಿ-ಪಾಟಿ
ಪೆರಿ-ಯನ್ನು
ಪೆರುಮಾಳ್
ಪೇ
ಪೇಜ್
ಪೇಟ-ವಂತೂ
ಪೇಟ-ವನ್ನು
ಪೇಟೆ
ಪೇಟೆಗೆ
ಪೇಟೆಯ
ಪೇಪ-ರಿಗೆ
ಪೇಶ್ಕಾರ-ರಾದ
ಪೇಷಾ-ವರ್
ಪೈ
ಪೈಕಿ
ಪೈಕ್ರಾಪ್ಟ್
ಪೈಪೋಟಿ-ಯ-ಲ್ಲಿ
ಪೈಪೋಟಿಯ
ಪೈಶಾಚಿಕ-ವಾ-ದುದು
ಪೊದೆ-ಗಳ-ಲ್ಲಿ
ಪೊರೆ
ಪೊಲಿಟಿ-ಕ-ಲ್
ಪೋಟಾಪೋಟಿ
ಪೋಪು-ಗಳು
ಪೋಪ್
ಪೋರಂ
ಪೋರ್ಟಿಕೊ
ಪೋರ್ಟಿನ
ಪೋರ್ಬಂದ-ರ-ನ್ನು
ಪೋರ್ಬಂದ-ರಿ-ನಿಂದ
ಪೋರ್ಬಂದ-ರಿಗೆ
ಪೋರ್ಬಂದ-ರಿನ
ಪೋರ್ಬಂದ-ರಿನ-ಲ್ಲಿರು-ವಾಗ
ಪೋರ್ಬಂದರ್
ಪೋರ್ಸಿಲೇ-ನಿನ
ಪೋಲೀಸಿ-ನ-ವನು
ಪೋಲು
ಪೋಷಕ
ಪೋಷಕ-ರಾಗಿ-ದ್ದಾರೆ
ಪೋಷಕ-ವಾದ
ಪೋಷಾ-ಕನ್ನು
ಪೋಷಾಕಿ-ನ-ಲ್ಲಿ
ಪೋಷಾಕು
ಪೋಷಾಕೆ-ಲ್ಲ
ಪೋಷಿ-ತರು
ಪೋಷಿ-ಸು-ತ್ತದೆ
ಪೋಷಿ-ಸು-ತ್ತಿ-ರುವ
ಪೋಷಿ-ಸುವ
ಪೋಷಿತ-ವಾದ
ಪೌ
ಪೌಂಡು
ಪೌರ
ಪೌರ-ಪಾಶ್ಚಾ-ತ್ಯ-ರಿಬ್ಬರೂ
ಪೌರ-ಶಾ-ಸ್ತ್ರ
ಪೌರ-ಸ್ತ್ಯ-ದೇಶ-ದ-ಲ್ಲಿ
ಪೌರಾ-ತ್ಯ
ಪೌರಾ-ತ್ಯ-ದೇಶ-ಗಳಿಂದ
ಪೌರಾ-ತ್ಯ-ದೇಶ-ದ-ಲ್ಲಿ
ಪೌರಾ-ತ್ಯನ
ಪೌರುಷ
ಪೌರುಷ-ದ-ಲ್ಲಿರು-ವಂತೆ
ಪೌರುಷ-ವನ್ನು
ಪೌಳಿಯ
ಪ್ಪ್ರಜ್ವಲನ
ಪ್ಯಾ
ಪ್ಯಾ-ರಿಸಿ-ನ-ಲ್ಲಿ
ಪ್ಯಾರಾಗ್ರಾಫಿನ
ಪ್ಯಾರಿ-ಸ್
ಪ್ಯಾರಿ-ಸ್ನ-ಲ್ಲಿ
ಪ್ಯಾರಿ-ಸ್ನಷ್ಟೆ
ಪ್ಯಾರಿ-ಸ್ಸನ್ನು
ಪ್ಯಾರಿ-ಸ್ಸಿ-ನ-ಲ್ಲಿ
ಪ್ಯಾರಿ-ಸ್ಸಿಗೆ
ಪ್ಯಾರಿ-ಸ್ಸಿನ
ಪ್ಯಾರಿ-ಸ್ಸಿನ-ಲ್ಲಿ-ದ್ದಾಗ
ಪ್ಯಾಲೇ-ಸ್
ಪ್ರ
ಪ್ರಕಟ-ಗೊಳಿಸಿತು
ಪ್ರಕಟ-ಪಡಿ-ಸಿದ
ಪ್ರಕಟ-ಮಾಡಿದ
ಪ್ರಕಟ-ವಾ-ಗಿದೆ
ಪ್ರಕಟ-ವಾಗಿ-ದ್ದವು
ಪ್ರಕಟಗೊಳಿಸ-ಬೇಕು
ಪ್ರಕಟಣೆ
ಪ್ರಕಟಿ-ಸಿ-ದರು
ಪ್ರಕಟಿ-ಸಿ-ರುವರು
ಪ್ರಕಟಿ-ಸು-ತ್ತಾ
ಪ್ರಕಟಿಸಿ-ರು-ವೆವು
ಪ್ರಕಾಶ
ಪ್ರಕಾಶ-ದಿಂದ
ಪ್ರಕಾಶ-ಮಾನ-ವಾಗಿ
ಪ್ರಕಾಶ-ಮಾನ-ವಾಗಿದೆ
ಪ್ರಕಾಶ-ವಾಗಲು
ಪ್ರಕಾಶ-ವಾಗಿ-ದ್ದವು
ಪ್ರಕಾಶ-ವಾಗು-ವುದು
ಪ್ರಕಾಶ-ವೆಂದು
ಪ್ರಕಾಶಕ್ಕೆ
ಪ್ರಕಾಶಾ-ನಂದ
ಪ್ರಕಾಶಿ-ಸು-ತ್ತಿದೆ
ಪ್ರಕಾಶಿ-ಸು-ವುದು
ಪ್ರಕಾಶಿತ-ವಾಗು-ವುದು
ಪ್ರಕ್ಷುಬ್ಧ
ಪ್ರಖ್ಯಾ-ತಿಗೆ
ಪ್ರಖ್ಯಾತ
ಪ್ರಖ್ಯಾತ-ಗೊಳಿಸಿದ
ಪ್ರಖ್ಯಾತ-ನಾಗಿ-ದ್ದನು
ಪ್ರಖ್ಯಾತ-ನಾಗಿ-ರು-ವೆನು
ಪ್ರಖ್ಯಾತ-ನಾಗಿದ್ದ
ಪ್ರಖ್ಯಾತ-ನಾಗು-ತ್ತ
ಪ್ರಖ್ಯಾತ-ರಾ-ದರು
ಪ್ರಖ್ಯಾತ-ರಾಗಿ-ದ್ದರು
ಪ್ರಖ್ಯಾತ-ರಾಗು-ವು-ದಕ್ಕೆ
ಪ್ರಖ್ಯಾತ-ರಾಗುವಿರಿ
ಪ್ರಖ್ಯಾತ-ರಾದ
ಪ್ರಖ್ಯಾತ-ರಾದ-ವರೊ-ಡ-ನೆ-ಲ್ಲ
ಪ್ರಖ್ಯಾತ-ವಾ-ಗಿದೆ
ಪ್ರಖ್ಯಾತ-ವಾಗು-ವುದು
ಪ್ರಖ್ಯಾತ-ವಾದ
ಪ್ರಚಂಡ
ಪ್ರಚಲಿತ-ವಾದ
ಪ್ರಚೋ-ದಿ-ಸಿದ-ವನು
ಪ್ರಚೋ-ದಿ-ಸಿದರು
ಪ್ರಚೋ-ದಿಸು-ವಂತಹ
ಪ್ರಚೋದಿ-ಸಲು
ಪ್ರಚೋದಿ-ಸಿದ್ದರೆ
ಪ್ರಚೋದಿ-ಸು-ತ್ತಿ-ತ್ತು
ಪ್ರಚೋದಿಸ-ಬೇಕು
ಪ್ರಜಾ-ಹಿ-ತಕ್ಕೆ
ಪ್ರಜೆ-ಗ-ಳಿಗೆ
ಪ್ರಜೆ-ಗಳ
ಪ್ರಜೆ-ಗಳಿ-ಗಾಗಿ
ಪ್ರಜೆ-ಗಳಿಗೂ
ಪ್ರಜೆ-ಗಳೂ
ಪ್ರಜೆ-ಗಳೆ-ಲ್ಲ
ಪ್ರಜೋ-ತ್ಪ-ತ್ತಿಯೇ
ಪ್ರಜ್ಞೆ
ಪ್ರಜ್ಞೆ-ಗಿಂತ
ಪ್ರಜ್ಞೆ-ಯ-ಲ್ಲಿ
ಪ್ರಜ್ಞೆ-ಯನ್ನು
ಪ್ರಜ್ಞೆ-ಯಿ-ಲ್ಲ
ಪ್ರಜ್ಞೆ-ಯಿ-ಲ್ಲದೆ
ಪ್ರಜ್ಞೆ-ಯಿ-ಲ್ಲದೇ
ಪ್ರಜ್ಞೆ-ಯಿದೆ
ಪ್ರಜ್ಞೆ-ಯೆ-ಲ್ಲ
ಪ್ರಜ್ಞೆಗೆ
ಪ್ರಜ್ಞೆಯ
ಪ್ರಜ್ಞೆಯೇ
ಪ್ರಜ್ವಲಿ-ಸು-ತ್ತಿ-ದ್ದವು
ಪ್ರಜ್ವಲಿ-ಸು-ತ್ತಿ-ರುವ
ಪ್ರಣಾ-ಮ-ಗ-ಳನ್ನು
ಪ್ರಣಾಮ
ಪ್ರಣಾಮ-ಮಾಡಿ-ದರು
ಪ್ರಣಾಮ-ಮಾಡು
ಪ್ರಣಾಯಾಮದ
ಪ್ರತಾಪ-ಚಂದ್ರ
ಪ್ರತಾಪಸಿಂಗ
ಪ್ರತಿ
ಪ್ರತಿ-ಕ್ರಿ-ಯೆ-ಯನ್ನು
ಪ್ರತಿ-ಕ್ರಿ-ಯೆ-ಯಾಗಿ
ಪ್ರತಿ-ಕ್ರಿ-ಯೆ-ಯಿಂದ
ಪ್ರತಿ-ಕ್ರಿ-ಯೆ-ಯುಂಟಾಗು-ವುದು
ಪ್ರತಿ-ಕ್ರಿ-ಯೆಯೆ
ಪ್ರತಿ-ಕ್ರಿಯೆ
ಪ್ರತಿ-ಕ್ಷಣವೂ
ಪ್ರತಿ-ಗಳು
ಪ್ರತಿ-ಜೀವಿ-ಯದೂ
ಪ್ರತಿ-ಜ್ಞೆ-ಯನ್ನು
ಪ್ರತಿ-ಜ್ಞೆಯ
ಪ್ರತಿ-ದಿ-ಧ್ವನಿ
ಪ್ರತಿ-ದಿನ
ಪ್ರತಿ-ದಿನದ
ಪ್ರತಿ-ದಿನವೂ
ಪ್ರತಿ-ದ್ವನಿ
ಪ್ರತಿ-ಧ್ವನಿ-ತ-ವಾಗು-ತ್ತಿ-ತ್ತು
ಪ್ರತಿ-ಧ್ವನಿ-ತ-ವಾಗು-ವಂತೆ
ಪ್ರತಿ-ನಮ-ಸ್ಕಾರ
ಪ್ರತಿ-ನಿ-ತ್ಯವೂ
ಪ್ರತಿ-ನಿಧಿ
ಪ್ರತಿ-ನಿಧಿ-ಗ-ಳನ್ನು
ಪ್ರತಿ-ನಿಧಿ-ಗ-ಳಿಗೆ
ಪ್ರತಿ-ನಿಧಿ-ಗಳಿ-ಗಿಂತಲು
ಪ್ರತಿ-ನಿಧಿ-ಗಳು
ಪ್ರತಿ-ನಿಧಿ-ಗಳೂ
ಪ್ರತಿ-ನಿಧಿ-ಯನ್ನು
ಪ್ರತಿ-ನಿಧಿ-ಯಾಗಿ
ಪ್ರತಿ-ನಿಧಿ-ಯಾದ
ಪ್ರತಿ-ನಿಧಿ-ಯೊಬ್ಬರು
ಪ್ರತಿ-ಪಕ್ಷದ
ಪ್ರತಿ-ಪಾದನೆ
ಪ್ರತಿ-ಪಾದಿ-ಸು-ವುದು
ಪ್ರತಿ-ಫಲ-ವನ್ನು
ಪ್ರತಿ-ಫಲ-ವನ್ನೂ
ಪ್ರತಿ-ಫಲವೇ
ಪ್ರತಿ-ಫಲಾಪೇಕ್ಷೆ
ಪ್ರತಿ-ಬಂದ-ಕ-ವೆಂಬು-ದ-ನ್ನು
ಪ್ರತಿ-ಬಂಧ-ಕ-ವಾಗಿ
ಪ್ರತಿ-ಬಂಧಕ-ಗ-ಳನ್ನು
ಪ್ರತಿ-ಬಂಧಕ-ಗಳ
ಪ್ರತಿ-ಬಂಧಕ-ಗಳಿಂ-ದಾಗಿ
ಪ್ರತಿ-ಬಂಧಕ-ಗಳು
ಪ್ರತಿ-ಬಿಂಬ
ಪ್ರತಿ-ಬಿಂಬ-ಗಳಿ-ಗೆ-ಲ್ಲ
ಪ್ರತಿ-ಬಿಂಬ-ಗಳು
ಪ್ರತಿ-ಬಿಂಬ-ದಂತೆ
ಪ್ರತಿ-ಬಿಂಬ-ವನ್ನೂ
ಪ್ರತಿ-ಬಿಂಬ-ವಾಗಿ-ರ-ಬೇಕು
ಪ್ರತಿ-ಬಿಂಬ-ವಾಗಿ-ರುವಂತೆ
ಪ್ರತಿ-ಬಿಂಬಿ-ಸು-ತ್ತಾನೆ
ಪ್ರತಿ-ಬಿಂಬಿ-ಸು-ತ್ತಿ-ರು-ವುದು
ಪ್ರತಿ-ಬಿಂಬಿ-ಸು-ತ್ತಿದ್ದ
ಪ್ರತಿ-ಬಿಂಬಿಸಿ
ಪ್ರತಿ-ಭಟನೆ
ಪ್ರತಿ-ಭಟನೆ-ಯನ್ನೂ
ಪ್ರತಿ-ಭಟಿ-ಸದೆ
ಪ್ರತಿ-ಭಟಿಸಿ-ದು-ದ-ರಿಂದ
ಪ್ರತಿ-ಭಾ-ವಂ-ತ-ರಾಗಿ
ಪ್ರತಿ-ಭಾ-ಶಾಲಿ
ಪ್ರತಿ-ಭಾ-ಶಾಲಿ-ಗಳು
ಪ್ರತಿ-ಭೆ-ಯ-ಲ್ಲಿ
ಪ್ರತಿ-ಭೆ-ಯನ್ನು
ಪ್ರತಿ-ಭೆ-ಯಿಂದ
ಪ್ರತಿ-ಭೆಗೆ
ಪ್ರತಿ-ಭೆಯ
ಪ್ರತಿ-ಮೆ-ಯಂತೆ
ಪ್ರತಿ-ಮೆ-ಯನ್ನು
ಪ್ರತಿ-ಮೆಯೋ
ಪ್ರತಿ-ಯಾಗಿ
ಪ್ರತಿ-ಯೊಂ-ದ-ನ್ನೂ
ಪ್ರತಿ-ಯೊಂದ-ರ-ಲ್ಲಿಯೂ
ಪ್ರತಿ-ಯೊಂದು
ಪ್ರತಿ-ಯೊಂದೂ
ಪ್ರತಿ-ಯೊಬ್ಬ
ಪ್ರತಿ-ಯೊಬ್ಬ-ರ-ನ್ನೂ
ಪ್ರತಿ-ಯೊಬ್ಬ-ರ-ಲ್ಲಿಯೂ
ಪ್ರತಿ-ಯೊಬ್ಬ-ರಿಗೂ
ಪ್ರತಿ-ಯೊಬ್ಬನ
ಪ್ರತಿ-ಯೊಬ್ಬನೂ
ಪ್ರತಿ-ಯೊಬ್ಬರ
ಪ್ರತಿ-ಯೊಬ್ಬರು
ಪ್ರತಿ-ಯೊಬ್ಬರೂ
ಪ್ರತಿ-ವರು-ಷವೂ
ಪ್ರತಿ-ವರ್ಷವೂ
ಪ್ರತಿ-ವಾದ-ಗಳ
ಪ್ರತಿ-ವಾರವೂ
ಪ್ರತಿ-ವೇಳೆ
ಪ್ರತಿ-ಷ್ಠಾಪನೆ
ಪ್ರತಿ-ಷ್ಠಿ-ತ-ರಾಗಿ
ಪ್ರತಿ-ಷ್ಠಿ-ತ-ವಾ-ಯಿತು
ಪ್ರತಿ-ಷ್ಠಿತ-ನಾಗುವೆ
ಪ್ರತಿ-ಷ್ಠಿತ-ವಾ-ಗಿದೆ
ಪ್ರತಿ-ಷ್ಠಿತ-ವಾಗಿ-ರುವ
ಪ್ರತಿ-ಷ್ಠಿತಾ
ಪ್ರತಿ-ಷ್ಠೆ-ಯನ್ನು
ಪ್ರತಿ-ಸ-ಲವೂ
ಪ್ರತಿಜ್ಞೆ
ಪ್ರತಿಭಾ
ಪ್ರತಿಭೆ
ಪ್ರತಿಮೆ
ಪ್ರತಿಷ್ಠೆ
ಪ್ರತೀ-ಕಾರ-ದ-ಲ್ಲಾ-ದರೂ
ಪ್ರತೀ-ಕಾರ-ದ-ಲ್ಲಿ
ಪ್ರತೀ-ಕಾರ-ವನ್ನು
ಪ್ರಥಮ
ಪ್ರಥಮ-ದ-ಲ್ಲಿ
ಪ್ರಥಮ-ದಿಂದಲೂ
ಪ್ರಥಮ-ಬಾರಿ
ಪ್ರಥಮ-ಭಿಕ್ಷೆ-ಯನ್ನು
ಪ್ರಥಮ-ರಾ-ದರು
ಪ್ರದ-ಕ್ಟಿಣೆ
ಪ್ರದ-ಕ್ಷಣೆ
ಪ್ರದರ್ಶಿ-ಸಿದ್ದಳು
ಪ್ರದರ್ಶಿ-ಸು-ತ್ತಿ-ದ್ದರು
ಪ್ರದರ್ಶಿ-ಸು-ವುದು
ಪ್ರಧಾನ
ಪ್ರಧಾನ-ವಾ-ದುದು
ಪ್ರಪಂಚ
ಪ್ರಪಂಚ-ಕ್ಕೆ-ಲ್ಲ
ಪ್ರಪಂಚ-ದ-ಲ್ಲಿ
ಪ್ರಪಂಚ-ದ-ಲ್ಲಿ-ರ-ಬೇಕಾಗಿ
ಪ್ರಪಂಚ-ದ-ಲ್ಲಿ-ರುವ
ಪ್ರಪಂಚ-ದ-ಲ್ಲಿ-ರುವ-ವ-ರೆ-ಲ್ಲ
ಪ್ರಪಂಚ-ದ-ಲ್ಲಿಯೂ
ಪ್ರಪಂಚ-ದ-ಲ್ಲೆ
ಪ್ರಪಂಚ-ದ-ಲ್ಲೆ-ಲ್ಲ
ಪ್ರಪಂಚ-ದ-ಲ್ಲೆ-ಲ್ಲಾ
ಪ್ರಪಂಚ-ದಿಂದ
ಪ್ರಪಂಚ-ವ-ನ್ನೆ-ಲ್ಲ
ಪ್ರಪಂಚ-ವ-ನ್ನೆ-ಲ್ಲಾ
ಪ್ರಪಂಚ-ವನ್ನು
ಪ್ರಪಂಚ-ವನ್ನೆ
ಪ್ರಪಂಚ-ವನ್ನೇ
ಪ್ರಪಂಚ-ವಿದೆ
ಪ್ರಪಂಚ-ವೆ-ಲ್ಲ
ಪ್ರಪಂಚ-ವೆಂಬ
ಪ್ರಪಂಚ-ವೆಂಬುದು
ಪ್ರಪಂಚ-ವೇನೂ
ಪ್ರಪಂಚಂ-ದ-ಲ್ಲಿ
ಪ್ರಪಂಚಕ್ಕೂ
ಪ್ರಪಂಚಕ್ಕೆ
ಪ್ರಪಂಚದ
ಪ್ರಪಂಚವೇ
ಪ್ರಪಾತ-ಗಳು
ಪ್ರಪು-ಲ್ಲ-ವಾಗಿ-ತ್ತು
ಪ್ರಪು-ಲ್ಲ-ವಾದ
ಪ್ರಬಲ
ಪ್ರಬಲ-ವಾದ
ಪ್ರಭಾತ
ಪ್ರಭಾಸ-ಗ-ಳಿಗೆ
ಪ್ರಭಾಸಕ್ಕೆ
ಪ್ರಭು-ದೇವನ
ಪ್ರಮದ
ಪ್ರಮದ-ದಾಸ
ಪ್ರಮದ-ದಾಸ-ಬಾಬು
ಪ್ರಮದ-ದಾಸ-ಮಿ-ತ್ರ
ಪ್ರಮದ-ದಾಸ-ಮಿ-ತ್ರ-ರಿಗೆ
ಪ್ರಮದ-ಬಾಬು-ಗ-ಳಿಗೆ
ಪ್ರಮದ-ಬಾಬು-ಗಳ
ಪ್ರಮದ-ಬಾಬು-ಗಳು
ಪ್ರಮಾ-ಣದ
ಪ್ರಮಾ-ಣವೇ
ಪ್ರಮಾಣ
ಪ್ರಮಾಣ-ದ-ಲ್ಲಿ
ಪ್ರಮಾಣ-ದ-ಲ್ಲಿ-ರು-ವುದು
ಪ್ರಮಾಣ-ದಂತೆ
ಪ್ರಮಾಣ-ಪಡಿ-ಸುವ-ವ-ರೆಗೆ
ಪ್ರಮಾಣ-ವನ್ನು
ಪ್ರಮಾಣ-ವಾ-ದರೂ
ಪ್ರಮಾಣ-ವಾದಿ
ಪ್ರಮಾಣ-ಸಿದ್ಧ
ಪ್ರಮಾದ
ಪ್ರಮಾದ-ಗಳ-ನ್ನೇ
ಪ್ರಮೋದ-ಗಳ-ಲ್ಲಿ
ಪ್ರಯತಿ
ಪ್ರಯಾ-ಣದ
ಪ್ರಯಾಣ
ಪ್ರಯಾಣ-ಮಾಡಿ
ಪ್ರಯಾಣ-ಮಾಡಿ-ಕೊಂಡು
ಪ್ರಯಾಣ-ಮಾಡಿ-ರು-ವು-ದಾಗಿಯೂ
ಪ್ರಯಾಣ-ಮಾಡು-ತ್ತಿದ್ದರು
ಪ್ರಯಾಣ-ಮಾಡು-ತ್ತಿರುವ
ಪ್ರಯಾಣ-ವನ್ನು
ಪ್ರಯಾಣಕ್ಕೆ
ಪ್ರಯಾಣಿ-ಕರ
ಪ್ರಯಾಣಿಕ-ರೊ-ಡನೆ
ಪ್ರಯಾಸ
ಪ್ರಯಾಸ-ವನ್ನು
ಪ್ರಯೊ-ಜನ-ವಾ-ಯಿತು
ಪ್ರಯೊ-ಜನ-ವಿ-ಲ್ಲ
ಪ್ರಯೋ-ಜನ
ಪ್ರಯೋ-ಜನ-ವನ್ನು
ಪ್ರಯೋ-ಜನ-ವನ್ನೆ-ಲ್ಲಾ
ಪ್ರಯೋ-ಜನ-ವಾ-ಗಿದೆ
ಪ್ರಯೋ-ಜನ-ವಾಗ-ಬಹು-ದೆಂದು
ಪ್ರಯೋ-ಜನ-ವಾಗ-ಲಿ-ಲ್ಲ
ಪ್ರಯೋ-ಜನ-ವಾಗು-ವು-ದಿ-ಲ್ಲ-ವೆಂದು
ಪ್ರಯೋ-ಜನ-ವಾಗು-ವು-ದಿ-ಲ್ಲ-ವೆಂದೂ
ಪ್ರಯೋ-ಜನ-ವಾಗು-ವುದು
ಪ್ರಯೋ-ಜನ-ವಿ-ದೆಯೆಂದು
ಪ್ರಯೋ-ಜನ-ವಿ-ಲ್ಲ
ಪ್ರಯೋ-ಜನ-ವಿ-ಲ್ಲ-ವೆಂಬುದು
ಪ್ರಯೋ-ಜನ-ವೆಂಬುದಿ-ಲ್ಲ-ವೆಂದು
ಪ್ರಯೋ-ಜನ-ವೇನು
ಪ್ರಯೋ-ಜನವೂ
ಪ್ರಯೋಜ-ನಕ್ಕೆ
ಪ್ರಲಾಪ-ವನ್ನು
ಪ್ರಲೋಭ-ನದ
ಪ್ರಲೋಭ-ನೆಯ
ಪ್ರಲೋಭನ-ಗಳು
ಪ್ರಲೋಭನೆ-ಗಳು
ಪ್ರಳಯ
ಪ್ರಳಯ-ಗ-ಳನ್ನು
ಪ್ರಳಯ-ಗಳಿಗೂ
ಪ್ರಳಯ-ಗಳೆ-ಲ್ಲ
ಪ್ರವ-ಚ-ನಕ್ಕೂ
ಪ್ರವ-ಚ-ನದ
ಪ್ರವ-ಚ-ನಾ-ದಿ-ಗಳು
ಪ್ರವ-ಚ-ನಾದಿ-ಗ-ಳಿಗೆ
ಪ್ರವ-ಚ-ನಾದಿ-ಗಳ
ಪ್ರವ-ಚನ
ಪ್ರವ-ಚನ-ಗ-ಳಿಗೆ
ಪ್ರವ-ಚನ-ಗಳ
ಪ್ರವ-ಚನ-ಗಳ-ಲ್ಲಿ
ಪ್ರವ-ಚನ-ಗಳು
ಪ್ರವ-ಚನಾ-ದಿ-ಗ-ಳನ್ನು
ಪ್ರವ-ಚಾನಾ-ದಿ-ಗ-ಳನ್ನು
ಪ್ರವಾಹ
ಪ್ರವಾಹ-ದ-ಲ್ಲಿ
ಪ್ರವಾಹ-ದ-ಲ್ಲಿ-ರುವ
ಪ್ರವಾಹ-ದ-ಲ್ಲೇ
ಪ್ರವಾಹ-ದಂತೆ
ಪ್ರವಾಹ-ದಿಂದ
ಪ್ರವಾಹ-ವನ್ನು
ಪ್ರವಾಹ-ವೊಂದು
ಪ್ರವಾಹಕ್ಕೆ
ಪ್ರವಾಹದ
ಪ್ರವಾಹವೂ
ಪ್ರವೀಣ
ಪ್ರವೀಣ-ನಾ-ದು-ದ-ರಿಂದ
ಪ್ರವೀಣ-ನಾಗಿ
ಪ್ರವೀಣ-ನಾದ
ಪ್ರವೇ-ಶ-ಮಾಡಿ-ದ-ಲ್ಲದೆ
ಪ್ರವೇ-ಶ-ಮಾಡಿ-ದರೆ
ಪ್ರವೇ-ಶ-ಮಾಡಿ-ಸ-ಬೇಕು
ಪ್ರವೇ-ಶಿ-ಸಿತು
ಪ್ರವೇ-ಶಿ-ಸಿದ
ಪ್ರವೇ-ಶಿ-ಸಿದೆ
ಪ್ರವೇ-ಶಿ-ಸು-ವು-ದಕ್ಕೆ
ಪ್ರವೇ-ಶಿ-ಸುವರು
ಪ್ರವೇ-ಶಿ-ಸುವಾಗ
ಪ್ರವೇ-ಶಿಸ-ಲಾಗು-ವು-ದಿ-ಲ್ಲ
ಪ್ರವೇ-ಶಿಸಿ
ಪ್ರವೇ-ಶಿಸಿ-ದಂತೆ
ಪ್ರವೇ-ಶಿಸಿ-ದರು
ಪ್ರವೇ-ಶಿಸಿ-ದರೆ
ಪ್ರವೇ-ಶಿಸಿ-ದವು
ಪ್ರವೇ-ಶಿಸಿ-ದಾಗ
ಪ್ರವೇ-ಶಿಸಿ-ದು-ದ-ನ್ನು
ಪ್ರವೇ-ಶಿಸಿ-ದ್ದರು
ಪ್ರವೇಶ
ಪ್ರಶ-ಸ್ತ-ವಾದ
ಪ್ರಶ-ಸ್ತಿ-ಯನ್ನು
ಪ್ರಶಂಸಿ-ಸಲಿ
ಪ್ರಶಂಸಿಸಿ
ಪ್ರಶಂಸೆ-ಯನ್ನೂ
ಪ್ರಶಂಸೆಯೇ
ಪ್ರಶ್ನಿ-ಸಲು
ಪ್ರಶ್ನಿ-ಸಿ-ದರು
ಪ್ರಶ್ನಿ-ಸಿದ
ಪ್ರಶ್ನಿ-ಸು-ತ್ತಿ-ದ್ದ-ರೆಂದು
ಪ್ರಶ್ನಿ-ಸು-ತ್ತೇನೆ
ಪ್ರಶ್ನಿ-ಸು-ವು-ದ-ನ್ನು
ಪ್ರಶ್ನಿ-ಸು-ವುದು
ಪ್ರಶ್ನಿಸಿ-ದಾಗ
ಪ್ರಶ್ನಿಸಿ-ದು-ದಕ್ಕೆ
ಪ್ರಶ್ನಿಸಿ-ರ-ಲಿ-ಲ್ಲ
ಪ್ರಶ್ನೆ
ಪ್ರಶ್ನೆ-ಗ-ಳನ್ನು
ಪ್ರಶ್ನೆ-ಗ-ಳಾದ
ಪ್ರಶ್ನೆ-ಗ-ಳಿಗೆ
ಪ್ರಶ್ನೆ-ಗಳ-ನ್ನೆ
ಪ್ರಶ್ನೆ-ಗಳ-ನ್ನೇ
ಪ್ರಶ್ನೆ-ಗಳ-ಲ್ಲಿ
ಪ್ರಶ್ನೆ-ಗಳಿ-ಗೆ-ಲ್ಲ
ಪ್ರಶ್ನೆ-ಗಳು
ಪ್ರಶ್ನೆ-ಗಳೇ
ಪ್ರಶ್ನೆ-ಮಾಡಿ
ಪ್ರಶ್ನೆ-ಮಾಡಿ-ದರು
ಪ್ರಶ್ನೆ-ಮಾಡಿದ
ಪ್ರಶ್ನೆ-ಯನ್ನು
ಪ್ರಶ್ನೆ-ಯನ್ನೇ
ಪ್ರಶ್ನೆ-ಯಾದ-ನಂ-ತರ
ಪ್ರಶ್ನೆಗೆ
ಪ್ರಶ್ನೆಯ
ಪ್ರಶ್ನೆಯೇ
ಪ್ರಶ್ನೋ-ತ್ತ-ರದ
ಪ್ರಶ್ನೋ-ತ್ತರ
ಪ್ರಸಾದ
ಪ್ರಸಾದ-ವನ್ನು
ಪ್ರಸಾದ-ವನ್ನೆ-ಲ್ಲ
ಪ್ರಸಾದದ
ಪ್ರಸೂತ
ಪ್ರಾ
ಪ್ರಾಂ
ಪ್ರಾಂಗಣ
ಪ್ರಾಂಗಣ-ದ-ಲ್ಲಿ
ಪ್ರಾಂಗಣ-ವನ್ನು
ಪ್ರಾಂಗಣ-ವನ್ನೇ
ಪ್ರಾಕ್ತನ
ಪ್ರಾಚೀನ
ಪ್ರಾಚೀನ-ಕಾಲದ
ಪ್ರಾಚೀನ-ತಮ
ಪ್ರಾಚೀನ-ತಮ-ವಾದ
ಪ್ರಾಚ್ಯ
ಪ್ರಾಚ್ಯ-ಸಂ-ಸ್ಕೃತಿಯ
ಪ್ರಾಟಿ-ಸ್ಟೆಂಟ-ರಿ-ಗಿಂತ
ಪ್ರಾಣ
ಪ್ರಾಣ-ದುಸಿರು
ಪ್ರಾಣ-ಪಕ್ಷಿ
ಪ್ರಾಣ-ವ-ನ್ನಾ-ದರೂ
ಪ್ರಾಣ-ವನ್ನು
ಪ್ರಾಣ-ವನ್ನೆ-ಲ್ಲ
ಪ್ರಾಣ-ವನ್ನೇ
ಪ್ರಾಣ-ವಿದ್ದುದು
ಪ್ರಾಣ-ವೆ-ಲ್ಲ
ಪ್ರಾಣ-ವೆ-ಲ್ಲಿದೆ
ಪ್ರಾಣ-ಸ್ಥಾನ-ದ-ಲ್ಲಿ
ಪ್ರಾಣ-ಹೋದರೂ
ಪ್ರಾಣಕ್ಕೆ
ಪ್ರಾಣಾಯಾಮ
ಪ್ರಾಣಾಯಾಮದ
ಪ್ರಾಣಿ
ಪ್ರಾಣಿ-ಗ-ಳನ್ನು
ಪ್ರಾಣಿ-ಗಳ
ಪ್ರಾಣಿ-ಗಳ-ನ್ನಿ-ಟ್ಟಿ-ರುವ
ಪ್ರಾಣಿ-ಗಳಂತೆ
ಪ್ರಾಣಿ-ಗಳಿಗೂ
ಪ್ರಾಣಿ-ಗಳೂ
ಪ್ರಾಣಿ-ಜಗ-ತ್ತಿ-ನ-ಲ್ಲಿ
ಪ್ರಾಣಿ-ಯ-ಲ್ಲಿಯೂ
ಪ್ರಾತಃ-ಕಾಲ
ಪ್ರಾದ್ರಿ-ಗಳು
ಪ್ರಾಧಾ-ನ್ಯ
ಪ್ರಾಧಾ-ನ್ಯ-ತೆ-ಯನ್ನು
ಪ್ರಾಧ್ಯಾಪ-ಕ-ರ-ನ್ನು
ಪ್ರಾಧ್ಯಾಪ-ಕ-ರಾಗಿದ್ದ
ಪ್ರಾಧ್ಯಾಪ-ಕ-ರಾದ
ಪ್ರಾಧ್ಯಾಪ-ಕರ
ಪ್ರಾಧ್ಯಾಪ-ಕರು
ಪ್ರಾಧ್ಯಾಪಕ-ರಾಗಿ-ದ್ದರು
ಪ್ರಾಧ್ಯಾಪಕ-ರೆ-ಲ್ಲಾ
ಪ್ರಾಧ್ಯಾಪಕ-ರೊಬ್ಬರು
ಪ್ರಾಪಂಚಿ-ಕತೆ
ಪ್ರಾಪಂಚಿ-ಕರು
ಪ್ರಾಪಂಚಿಕ
ಪ್ರಾಪ್ತ
ಪ್ರಾಪ್ತ-ವಯ-ಸ್ಸಾಗಿ-ಲ್ಲದೆ
ಪ್ರಾಪ್ತ-ವಯ-ಸ್ಸಾದ
ಪ್ರಾಪ್ತ-ವಯ-ಸ್ಸಿಗೆ
ಪ್ರಾಪ್ತ-ವಾ-ಗಿದೆ
ಪ್ರಾಪ್ತ-ವಾಗಲಿ
ಪ್ರಾಪ್ತ-ವಾಗಿ-ತ್ತೆಂದು
ಪ್ರಾಪ್ತ-ವಾಗು-ವುದು
ಪ್ರಾಪ್ತಿ-ಯಾಗು-ತ್ತದೆ
ಪ್ರಾಪ್ಯ-ವರಾ-ನ್ನಿಬೋಧತ
ಪ್ರಾಬ-ಲ್ಯ
ಪ್ರಾಬ-ಲ್ಯ-ದ-ಲ್ಲಿರು-ವರು
ಪ್ರಾಬ-ಲ್ಯ-ದಿಂದ
ಪ್ರಾಮಾಣಿ-ಕರು
ಪ್ರಾಮಾಣಿಕ-ತನ-ವಿ-ದ್ದ-ಲ್ಲಿ
ಪ್ರಾಯ
ಪ್ರಾಯ-ರಾದ
ಪ್ರಾಯ-ವಿ-ಳಿದ
ಪ್ರಾಯ-ಶ್ಚಿ-ತ್ತ
ಪ್ರಾಯ-ಶ್ಚಿ-ತ್ತ-ವಿದೆ
ಪ್ರಾಯದ
ಪ್ರಾಯೋಪವೇಶ
ಪ್ರಾರಂಭ
ಪ್ರಾರಂಭ-ದ-ಲ್ಲಿ
ಪ್ರಾರಂಭ-ದ-ಲ್ಲಿಯೇ
ಪ್ರಾರಂಭ-ದಿಂದಲೂ
ಪ್ರಾರಂಭ-ಮಾಡಿ
ಪ್ರಾರಂಭ-ಮಾಡಿ-ದರು
ಪ್ರಾರಂಭ-ಮಾಡಿ-ದಾಗ
ಪ್ರಾರಂಭ-ವಾ-ಗಿದೆ
ಪ್ರಾರಂಭ-ವಾ-ದದ್ದು
ಪ್ರಾರಂಭ-ವಾ-ದರೂ
ಪ್ರಾರಂಭ-ವಾ-ದರೆ
ಪ್ರಾರಂಭ-ವಾ-ದವು
ಪ್ರಾರಂಭ-ವಾ-ದುದು
ಪ್ರಾರಂಭ-ವಾ-ಯಿತು
ಪ್ರಾರಂಭ-ವಾ-ಯಿತೆಂದು
ಪ್ರಾರಂಭ-ವಾಗಿ
ಪ್ರಾರಂಭ-ವಾಗಿ-ರ-ಬಹುದು
ಪ್ರಾರಂಭ-ವಾಗು-ವು-ದಕ್ಕೆ
ಪ್ರಾರಂಭ-ವಾಗು-ವುದು
ಪ್ರಾರಂಭ-ವಾದ
ಪ್ರಾರಂಭಿ-ಸಿದ
ಪ್ರಾರಂಭಿ-ಸಿದ್ದು
ಪ್ರಾರಂಭಿ-ಸು-ವು-ದಕ್ಕೆ
ಪ್ರಾರಂಭಿಸ-ಬೇಕು
ಪ್ರಾರಂಭಿಸಿ
ಪ್ರಾರಂಭಿಸಿ-ದನು
ಪ್ರಾರಂಭಿಸಿ-ದರು
ಪ್ರಾರಂಭಿಸಿ-ದರೆ
ಪ್ರಾರಂಭಿಸಿ-ದ್ದಾರೆಂದು
ಪ್ರಾರಬ್ಧ
ಪ್ರಾರ್ಥ-ನಾ-ದಿ-ಗಳು
ಪ್ರಾರ್ಥ-ನಾ-ಶ್ಲೋಕ-ವನ್ನು
ಪ್ರಾರ್ಥ-ನಾ-ಶ್ಲೋಕ-ವೊಂ-ದ-ನ್ನು
ಪ್ರಾರ್ಥ-ನೆ-ಗಳಿಂದ
ಪ್ರಾರ್ಥ-ನೆ-ಗಳು
ಪ್ರಾರ್ಥ-ನೆ-ಗಳೇ
ಪ್ರಾರ್ಥ-ನೆ-ಗಾಗಿ
ಪ್ರಾರ್ಥ-ನೆ-ಯ-ನ್ನೇನೋ
ಪ್ರಾರ್ಥ-ನೆ-ಯನ್ನು
ಪ್ರಾರ್ಥ-ನೆ-ಯಿಂದ
ಪ್ರಾರ್ಥ-ನೆಗೆ
ಪ್ರಾರ್ಥನಾ
ಪ್ರಾರ್ಥನೆ
ಪ್ರಾರ್ಥಿ-ಸಿದ
ಪ್ರಾರ್ಥಿ-ಸಿದೆ
ಪ್ರಾರ್ಥಿಸ-ತೊಡಗಿದರು
ಪ್ರಾರ್ಥಿಸ-ಬೇಕು
ಪ್ರಾರ್ಥಿಸಿ
ಪ್ರಾರ್ಥಿಸಿ-ಕೊಂಡನು
ಪ್ರಾರ್ಥಿಸಿ-ಕೊಂಡರು
ಪ್ರಾರ್ಥಿಸಿ-ಕೊಂಡಳು
ಪ್ರಾರ್ಥಿಸಿ-ಕೊಂಡಿದ್ದ-ರಿಂದ
ಪ್ರಾರ್ಥಿಸಿ-ಕೊಂಡಿದ್ದರು
ಪ್ರಾರ್ಥಿಸಿ-ಕೊಂಡು
ಪ್ರಾರ್ಥಿಸಿ-ದರು
ಪ್ರಾರ್ಥಿಸಿ-ದರೂ
ಪ್ರಾರ್ಥಿಸಿ-ದರೆ
ಪ್ರಾರ್ಥಿಸಿ-ರು-ವೆನು
ಪ್ರಾರ್ಥಿಸಿ-ರುವೆ
ಪ್ರಾರ್ಥಿಸು
ಪ್ರಾರ್ಥಿಸು-ತಿದ್ದೆ
ಪ್ರಾರ್ಥಿಸು-ತ್ತಿದ್ದರು
ಪ್ರಾರ್ಥಿಸು-ತ್ತಿದ್ದುದು
ಪ್ರಾರ್ಥಿಸು-ತ್ತೇನೆ
ಪ್ರಾರ್ಥಿಸು-ವು-ದ-ರಿಂದ
ಪ್ರಾರ್ಥಿಸು-ವುದೊಂದೆ
ಪ್ರಾವೀಣ್ಯತೆ-ಯನ್ನು
ಪ್ರಾಸವೂ
ಪ್ರಿಂ
ಪ್ರಿತಿ
ಪ್ರಿನ್ಸಿ-ಪಾ-ಲರಾದ
ಪ್ರಿನ್ಸಿ-ಪಾ-ಲ್
ಪ್ರಿನ್ಸಿಪಾಲ-ರಿಂದ
ಪ್ರಿನ್ಸಿಪಾಲ-ಳಾಗಿ-ದ್ದಳು
ಪ್ರಿನ್ಸೆ
ಪ್ರಿನ್ಸೆ-ಸ್
ಪ್ರಿಮೇ-ಸನ್
ಪ್ರಿಯ
ಪ್ರಿಯ-ತ-ಮನೇ
ಪ್ರಿಯ-ತ-ಮರು
ಪ್ರಿಯ-ತಮ
ಪ್ರಿಯ-ತಮೆ
ಪ್ರಿಯ-ಳಾದ
ಪ್ರಿಯ-ವಾ-ದುದು
ಪ್ರಿಯ-ವಾಗಿ-ದ್ದವು
ಪ್ರಿಯ-ವಾಗಿ-ದ್ದುವು
ಪ್ರಿಯ-ವಾಗಿ-ರು-ವು-ದಕ್ಕೆ
ಪ್ರಿಯ-ವಾಗು-ವು-ದಕ್ಕೆ
ಪ್ರಿಯ-ವಾದ
ಪ್ರಿಯೆ
ಪ್ರಿಯೆ-ಯನ್ನು
ಪ್ರೀ-ತಿಗೆ
ಪ್ರೀತಿ
ಪ್ರೀತಿ-ಗೋಸುಗ
ಪ್ರೀತಿ-ಪೂರ್ವ-ಕ-ವಾಗಿ
ಪ್ರೀತಿ-ಯ-ಲ್ಲ
ಪ್ರೀತಿ-ಯ-ಲ್ಲಿ
ಪ್ರೀತಿ-ಯನ್ನು
ಪ್ರೀತಿ-ಯಿ-ದ್ದರೂ
ಪ್ರೀತಿ-ಯಿ-ದ್ದರೆ
ಪ್ರೀತಿ-ಯಿಂದ
ಪ್ರೀತಿ-ಸ-ಬ-ಲ್ಲ-ವ-ರಾಗಿ-ದ್ದರು
ಪ್ರೀತಿ-ಸ-ಬೇಕು
ಪ್ರೀತಿ-ಸದೆ
ಪ್ರೀತಿ-ಸಲಾರ
ಪ್ರೀತಿ-ಸಲು
ಪ್ರೀತಿ-ಸಿ-ದಂತೆ
ಪ್ರೀತಿ-ಸಿ-ದನೊ
ಪ್ರೀತಿ-ಸಿ-ದರು
ಪ್ರೀತಿ-ಸಿ-ರಲಾರ
ಪ್ರೀತಿ-ಸಿ-ರಲಾರಳು
ಪ್ರೀತಿ-ಸಿದ
ಪ್ರೀತಿ-ಸು-ತ್ತಾರೆ
ಪ್ರೀತಿ-ಸು-ತ್ತಿ-ದ್ದಂತೆ
ಪ್ರೀತಿ-ಸು-ತ್ತಿ-ದ್ದರು
ಪ್ರೀತಿ-ಸು-ತ್ತಿ-ದ್ದಳು
ಪ್ರೀತಿ-ಸು-ತ್ತಿ-ರ-ಲಿ-ಲ್ಲ
ಪ್ರೀತಿ-ಸು-ತ್ತಿ-ರು-ವೆವು
ಪ್ರೀತಿ-ಸು-ತ್ತಿದ್ದ
ಪ್ರೀತಿ-ಸು-ತ್ತಿದ್ದೆ
ಪ್ರೀತಿ-ಸು-ತ್ತೇನೆ
ಪ್ರೀತಿ-ಸು-ತ್ತೇವೆ
ಪ್ರೀತಿ-ಸು-ವಿ-ರೇನು
ಪ್ರೀತಿ-ಸು-ವು-ದ-ನ್ನು
ಪ್ರೀತಿ-ಸು-ವು-ದ-ರಿಂದ
ಪ್ರೀತಿ-ಸು-ವು-ದಕ್ಕೆ
ಪ್ರೀತಿ-ಸು-ವುದು
ಪ್ರೀತಿ-ಸು-ವೆನು
ಪ್ರೀತಿ-ಸುವ
ಪ್ರೀತಿ-ಸುವನು
ಪ್ರೀತಿ-ಸುವನೊ
ಪ್ರೀತಿ-ಸುವರೊ
ಪ್ರೀತಿಯ
ಪ್ರೀತಿಯೂ
ಪ್ರೀತಿಸಿ
ಪ್ರೆ
ಪ್ರೆಸಿಡೆ-ನ್ಸಿ
ಪ್ರೆಸಿಡೆಂಟ-ರಾದ
ಪ್ರೇ
ಪ್ರೇಕ್ಷ-ಕ-ರ-ನ್ನು
ಪ್ರೇಕ್ಷ-ಕ-ರಿಂದ
ಪ್ರೇಕ್ಷ-ಕ-ರಿಗೆ
ಪ್ರೇಕ್ಷ-ಕ-ರೆ-ಲ್ಲ
ಪ್ರೇಕ್ಷ-ಕರ
ಪ್ರೇಕ್ಷ-ಕರು
ಪ್ರೇಕ್ಷಕ-ರ-ಲ್ಲಿ
ಪ್ರೇಕ್ಷಕ-ರೆ-ಲ್ಲರೂ
ಪ್ರೇಕ್ಷಕ-ವೃಂದ
ಪ್ರೇತ
ಪ್ರೇತ-ಗ-ಳನ್ನು
ಪ್ರೇತ-ಗಳ
ಪ್ರೇತ-ಗಳು
ಪ್ರೇತ-ಗಳೇನೊ
ಪ್ರೇತ-ಶರೀ-ರಿಗೆ
ಪ್ರೇತ-ಸ್ವ-ರೂಪ-ದಿಂದ
ಪ್ರೇತಾ-ತ್ಮ-ಗಳ
ಪ್ರೇತಾ-ತ್ಮಕ್ಕೆ
ಪ್ರೇಮ
ಪ್ರೇಮ-ದ-ಲ್ಲಿ
ಪ್ರೇಮ-ದ-ಲ್ಲಿ-ರುವ
ಪ್ರೇಮ-ದಷ್ಟು
ಪ್ರೇಮ-ದಿಂದ
ಪ್ರೇಮ-ಪುರ-ವೆಂಬ
ಪ್ರೇಮ-ಪೂರ್ಣ-ವಾದ
ಪ್ರೇಮ-ವ-ಲ್ಲ
ಪ್ರೇಮ-ವನ್ನು
ಪ್ರೇಮ-ವನ್ನೂ
ಪ್ರೇಮ-ವಿ-ತ್ತು
ಪ್ರೇಮ-ವೆ-ಲ್ಲ
ಪ್ರೇಮ-ವೆ-ಲ್ಲಿರು-ವುದೊ
ಪ್ರೇಮ-ವೆಂದ-ರೇನು
ಪ್ರೇಮ-ಸ್ವ-ರೂಪ
ಪ್ರೇಮಕ್ಕೆ
ಪ್ರೇಮದ
ಪ್ರೇಮವೆ
ಪ್ರೇಮವೇ
ಪ್ರೇಮಾ-ನಂದ
ಪ್ರೇಮಾ-ನಂದ-ರ-ನ್ನು
ಪ್ರೇಮಾ-ನಂದ-ರಿಗೆ
ಪ್ರೇಮಾ-ನಂದ-ರೊ-ಡನೆ
ಪ್ರೇಮಾ-ನಂದರು
ಪ್ರೇಮಾ-ಲಿಂಗ-ನ-ದ-ಲ್ಲಿ
ಪ್ರೇಮಾಮೃತ-ವನ್ನೇ
ಪ್ರೇಮಿ
ಪ್ರೇಮಿ-ಗಳ-ಲ್ಲ
ಪ್ರೇಮಿ-ಗಳು
ಪ್ರೇಮಿಯೂ
ಪ್ರೇಮೇಶ್ವರನೇ
ಪ್ರೇಮೋ-ನ್ಮಾದ-ದ-ಲ್ಲಿ
ಪ್ರೇಯಸಿ-ಯನ್ನು
ಪ್ರೇರಣೆ
ಪ್ರೇರಣೆ-ಯಾಗಿ
ಪ್ರೇರಣೆ-ಯಿಂದ
ಪ್ರೇರಿ-ತ-ರಾಗಿ
ಪ್ರೇರಿತ-ನಾಗಿ
ಪ್ರೇರೇಪಿ-ಸಿದ
ಪ್ರೇರೇಪಿ-ಸಿದ-ವರು
ಪ್ರೇರೇಪಿ-ಸು-ತ್ತಿ-ದ್ದನು
ಪ್ರೇರೇಪಿ-ಸು-ತ್ತಿದೆ
ಪ್ರೇರೇಪಿ-ಸು-ವುದು
ಪ್ರೇರೇಪಿತ-ರಾಗಿ-ದ್ದರು
ಪ್ರೇರೇಪಿತ-ರಾದ
ಪ್ರೇರೇಪಿಸಿ-ವು-ದಕ್ಕೆ
ಪ್ರೈ
ಪ್ರೊಫೆ-ಸ-ರಿಗೆ
ಪ್ರೊಫೆ-ಸರ್
ಪ್ರೊಫೆ-ಸರ್ಗಳೆ-ಲ್ಲ-ರ-ನ್ನು
ಪ್ರೋ
ಪ್ರೌಢ
ಪ್ಲಾ
ಪ್ಲೇಗಿನ
ಪ್ಲೇಗು
ಪ್ಲೇಗು-ಗ-ಳನ್ನು
ಫಂಕಿ
ಫಕೀ-ರನ
ಫಕೀ-ರರು
ಫಕೀರ
ಫಕೀರ-ನನ್ನು
ಫಕೀರ-ಬಾಬು
ಫರ್ಮಿ-ನ-ಲ್ಲಿ
ಫಲ
ಫಲ-ಕಾರಿ
ಫಲ-ಕಾರಿ-ಯಾ-ಗು-ತ್ತಿದೆ
ಫಲ-ಕಾರಿ-ಯಾ-ಯಿತು
ಫಲ-ಕಾರಿ-ಯಾಗ-ಬೇ-ಕಾ-ದರೆ
ಫಲ-ಕಾರಿ-ಯಾಗ-ಲಿ-ಲ್ಲ
ಫಲ-ಕಾರಿ-ಯಾಗಲು
ಫಲ-ಕಾರಿ-ಯಾಗಿ
ಫಲ-ಕಾರಿ-ಯಾಗಿ-ಲ್ಲ
ಫಲ-ಕಾರಿ-ಯಾಗು-ವು-ದಕ್ಕೆ
ಫಲ-ಕಾರಿ-ಯಾಗು-ವು-ದಿ-ಲ್ಲ
ಫಲ-ಕಾರಿ-ಯಾಗುವುದು
ಫಲ-ಕಾರಿ-ಯಾದುವು
ಫಲ-ಗಳಿಗ-ಲ್ಲ
ಫಲ-ದಿಂದ
ಫಲ-ಪುಷ್ಪಾ-ದಿ-ಗ-ಳನ್ನು
ಫಲ-ವ-ಲ್ಲ
ಫಲ-ವನ್ನು
ಫಲ-ವಾ-ಗಿದೆ
ಫಲ-ವಾ-ದರೆ
ಫಲ-ವಾಗಿ
ಫಲ-ವಿ-ಲ್ಲ
ಫಲಕ್ಕೆ
ಫಲಾ-ಕಾಂಕ್ಷೆ
ಫಲಾ-ಹಾರ-ವನ್ನು
ಫಲಾಪೇಕ್ಷೆ
ಫಲಾಪೇಕ್ಷೆ-ಯನ್ನು
ಫಲಾಪೇಕ್ಷೆ-ಯಿ-ಲ್ಲದೆ
ಫಲಿ-ಸಿತು
ಫಲಿತಾಂಶ
ಫಲಿಸಿ-ರ-ಲಿ-ಲ್ಲ
ಫಲಿಸಿತ-ಲ್ಲ
ಫಾ
ಫಾಕ್ಸ್
ಫಿರಂಗಿ-ಗಳಂತೆ
ಫಿರಂಗಿ-ಗಳಿಂದ
ಫಿರಂಗಿ-ಗಳು
ಫಿಲಿಪ್
ಫಿಲಿಪ್ಸ್
ಫೀ
ಫೀಜು
ಫೆ
ಫೆಬ್ರವ-ರಿ-ಯ-ಲ್ಲಿ
ಫೆಬ್ರವರಿ
ಫೇನ-ಮಯ-ವಾದ
ಫೋಟೊ-ವನ್ನು
ಫೋಟೋ
ಫೋಟೋ-ವನ್ನು
ಫೋನೋ-ಗ್ರಾಮಿ-ನ-ಲ್ಲಿ
ಫೋನೋಗ್ರಾಫಿನ
ಫ್ಯಾಷನಬ-ಲ್
ಫ್ರಾ
ಫ್ರೀ
ಫ್ರೂ
ಫ್ರೆಂಚ್
ಫ್ರೆಡರಿಕ್
ಫ್ಲಾರೆ-ನ್ಸ್
ಫ್ಲೋರ-ಲ್ನ-ಲ್ಲಿ
ಬ
ಬಂ
ಬಂಕು
ಬಂಗಲೆ-ಯ-ಲ್ಲಿ
ಬಂಗಾಳ
ಬಂಗಾಳ-ಕೊ-ಲ್ಲಿ
ಬಂಗಾಳ-ಕೊ-ಲ್ಲಿ-ಯ-ಲ್ಲಿ
ಬಂಗಾಳ-ದ-ಲ್ಲಿ
ಬಂಗಾಳ-ದ-ಲ್ಲಿಯೂ
ಬಂಗಾಳ-ದೇಶ
ಬಂಗಾಳ-ದೇಶ-ದ-ಲ್ಲಿ
ಬಂಗಾಳ-ವೊಂ-ದ-ನ್ನು
ಬಂಗಾಳಕ್ಕೆ
ಬಂಗಾಳದ
ಬಂಗಾಳಾದ
ಬಂಗಾಳಿ
ಬಂಗಾಳಿ-ಗ-ಳನ್ನು
ಬಂಗಾಳಿ-ಗ-ಳಿಗೆ
ಬಂಗಾಳಿ-ಗಳಿ-ಗೆ-ಲ್ಲ
ಬಂಗಾಳಿ-ಗಳು
ಬಂಗಾಳಿ-ಯ-ವರ
ಬಂಗಾಳಿ-ಯಿಂದ
ಬಂಗಾಳಿ-ಯೆಂದು
ಬಂಗಾಳಿಯ
ಬಂಗಾಳೀ
ಬಂಜೆಯ
ಬಂಡವಾಳ-ಗಾರ-ರಿಗೂ
ಬಂಡಿ-ಯ-ಲ್ಲಿ
ಬಂಡಿಗೆ
ಬಂಡಿಯ
ಬಂಡೆ
ಬಂಡೆ-ಯಿಂದ
ಬಂಡೆಯ
ಬಂತು
ಬಂತೆ
ಬಂದ
ಬಂದ-ದ-ನ್ನು
ಬಂದ-ದ್ದ-ನ್ನು
ಬಂದ-ನಂ-ತರ
ಬಂದ-ನೆಂದು
ಬಂದ-ಮೇಲೆ
ಬಂದ-ರ-ನ್ನು
ಬಂದ-ರಿನ
ಬಂದ-ರಿನ-ಲ್ಲಿ-ತ್ತು
ಬಂದ-ವ-ರ-ಲ್ಲ
ಬಂದ-ವ-ರ-ಲ್ಲಿ
ಬಂದ-ವ-ರಿ-ಗೆ-ಲ್ಲ
ಬಂದ-ವ-ರಿಗೆ
ಬಂದ-ವ-ರೆ-ಲ್ಲ
ಬಂದ-ವ-ರೆಂದೂ
ಬಂದ-ವ-ರೊ-ಡನೆ
ಬಂದ-ವ-ರೊಂದಿಗೆ
ಬಂದ-ವನು
ಬಂದ-ವನೆ
ಬಂದ-ವರು
ಬಂದ-ವರೇ
ಬಂದ-ವು-ಗಳ-ಲ್ಲ
ಬಂದದು
ಬಂದದ್ದೇ
ಬಂದರು
ಬಂದರೂ
ಬಂದರೆ
ಬಂದರೋ
ಬಂದಳು
ಬಂದಾದ
ಬಂದಾದ-ಮೇಲೆ
ಬಂದಿ-ತೆಂದೂ
ಬಂದಿ-ತೆಂಬುದು
ಬಂದಿ-ದೆಯೆ
ಬಂದಿ-ದ್ದೇನೆ
ಬಂದಿ-ರು-ವಿರೋ
ಬಂದಿ-ರು-ವು-ದಾಗಿ
ಬಂದಿ-ರು-ವು-ದಿ-ಲ್ಲ
ಬಂದಿ-ರು-ವೆನು
ಬಂದಿ-ರು-ವೆವು
ಬಂದಿ-ರುವು-ದೆಂದೂ
ಬಂದಿತು
ಬಂದಿತು-ನರೇಂದ್ರನ
ಬಂದಿತೆಂ-ತಲೂ
ಬಂದಿತೆಂದು
ಬಂದಿರ-ಬಹುದು
ಬಂದಿರ-ಬೇ-ಕಾ-ದರೆ
ಬಂದಿರ-ಬೇಕು
ಬಂದಿರ-ಬೇಕೆಂದು
ಬಂದೀತು
ಬಂದೀತೆಂದು
ಬಂದು
ಬಂದು-ದ-ನ್ನು
ಬಂದು-ದ-ರಿಂದ
ಬಂದು-ದ-ಲ್ಲ
ಬಂದು-ಬಿ-ಟ್ಟ
ಬಂದು-ಬಿ-ಟ್ಟೆ
ಬಂದು-ವೆಂದು
ಬಂದು-ಹೋ-ಗಿದೆ
ಬಂದು-ಹೋಗಿವೆ
ಬಂದು-ಹೋಗುವ
ಬಂದು-ಹೋದುವು
ಬಂದುದು
ಬಂದೂ-ಕದ
ಬಂದೂಕ-ವನ್ನು
ಬಂದೂಕಿ-ನಿಂದ
ಬಂದೆ
ಬಂದೆ-ಡೆಗೆ
ಬಂದೆ-ಯ-ಲ್ಲ
ಬಂದೆ-ವೆಂದೂ
ಬಂದೆನು
ಬಂದೆಯಾ
ಬಂದೆಯೋ
ಬಂದೆವು
ಬಂದೇ
ಬಂದೇ-ಬಂ
ಬಂದೇನು
ಬಂದೊದಗಿತು
ಬಂದೊದಗುವ
ಬಂಧ-ನ-ದ-ಲ್ಲಿ
ಬಂಧ-ನ-ದಿಂದ
ಬಂಧ-ನಕ್ಕೆ
ಬಂಧ-ನದ
ಬಂಧ-ನದ-ಲ್ಲಿ-ದ್ದಾನೋ
ಬಂಧ-ನದ-ಲ್ಲಿ-ರುವೆ
ಬಂಧ-ನವೇ
ಬಂಧಿ-ಸಲು
ಬಂಧಿ-ಸಿತು
ಬಂಧಿ-ಸಿದ್ದ
ಬಂಧಿತ-ವ-ಲ್ಲ
ಬಂಧಿಸ-ಬ-ಲ್ಲದು
ಬಂಧಿಸಿ
ಬಂಧಿಸಿ-ರು-ವುದು
ಬಂಧು-ಗಳ
ಬಂಧು-ಗಳು
ಬಂಧು-ಗಳೊ-ಬ್ಬರು
ಬಂಧು-ಬಳಗ-ಗಳಿ-ಲ್ಲವೋ
ಬಂಧು-ಬಳಗ-ದ-ವರ-ನ್ನೆ-ಲ್ಲ
ಬಂಧು-ವಂತೆ
ಬಂಧು-ವನ್ನೋ
ಬಂಧುವಾತ
ಬಗೆ
ಬಗೆ-ಗಣ್ಣಿಗೆ
ಬಗೆ-ಬ-ಗೆಯ
ಬಗೆ-ಯ-ಲ್ಲಿ
ಬಗೆ-ಯದು
ಬಗೆ-ಯಾಗಿ-ರುವರು
ಬಗೆ-ಯಿತು
ಬಗೆ-ಯು-ವಂತೆ
ಬಗೆ-ಹರಿ-ಯಿತು
ಬಗೆ-ಹರಿ-ಸ-ಬ-ಲ್ಲಂತಹ
ಬಗೆ-ಹರಿ-ಸ-ಬ-ಲ್ಲದು
ಬಗೆ-ಹರಿ-ಸದೆ
ಬಗೆ-ಹರಿ-ಸಲು
ಬಗೆ-ಹರಿ-ಸಿ-ಕೊಳ್ಳ-ಬ-ಲ್ಲರು
ಬಗೆ-ಹರಿ-ಸಿ-ದರು
ಬಗೆ-ಹರಿ-ಸಿ-ರುವರು
ಬಗೆ-ಹರಿ-ಸು-ತ್ತೇನೆ
ಬಗೆ-ಹರಿ-ಸು-ವು-ದಕ್ಕೆ
ಬಗೆ-ಹರಿ-ಸು-ವುದು
ಬಗೆದು
ಬಗ್ಗ-ಲಿ-ಲ್ಲ
ಬಗ್ಗದೇ
ಬಗ್ಗಿ-ದವು
ಬಗ್ಗಿ-ಸ-ಲಾ-ರದು
ಬಗ್ಗಿ-ಸು-ತ್ತಿ-ದ್ದಳು
ಬಗ್ಗು-ವ-ವ-ನ-ಲ್ಲ
ಬಗ್ಗುವ-ವರ-ಲ್ಲ-ವೆಂದೂ
ಬಗ್ಗೆ
ಬಚ್ಚಲ
ಬಚ್ಚಿ-ಟ್ಟು
ಬಚ್ಚಿಡು-ವಂತೆ
ಬಡ
ಬಡ-ಕುಟೀರ-ದ-ಲ್ಲಿ
ಬಡ-ಗುಡಿ-ಸಲಿ-ನಿಂದ
ಬಡ-ಜ-ನರ
ಬಡ-ಜ-ನರು
ಬಡ-ಜನ-ರಿಗೆ
ಬಡ-ಜೀವಿ-ಗಳು
ಬಡ-ತ-ನದ
ಬಡ-ತನ
ಬಡ-ತನ-ದ-ಲ್ಲಿ
ಬಡ-ತನ-ದ-ಲ್ಲಿ-ದ್ದರೂ
ಬಡ-ತನ-ದಿಂದ
ಬಡ-ತನ-ವನ್ನು
ಬಡ-ದೇಶ
ಬಡ-ಪೆ-ಟ್ಟಿಗೆ
ಬಡ-ವ-ರಿಗೆ
ಬಡ-ವರ
ಬಡ-ವರಾ-ದ-ವರು
ಬಡ-ವರು
ಬಡ-ವರೋ
ಬಡ-ಹುಡುಗ
ಬಡ-ಹುಡುಗನು
ಬಡ-ಹುಡುಗರು
ಬಡಗಿ-ಯ-ವನು
ಬಡತ-ನವೇ
ಬಡಪಾಯಿ-ಯಂತೆ
ಬಡಬಗ್ಗ-ರಿಗೆ
ಬಡಬಗ್ಗರ
ಬಡವ-ರ-ಲ್ಲಿ
ಬಡಿ-ದಂತೆ
ಬಡಿ-ದಿದ್ದ
ಬಡಿ-ಸಲು
ಬಡಿ-ಸಿದ
ಬಡಿ-ಸು-ತ್ತಾನೆ
ಬಡಿ-ಸು-ತ್ತಿ-ದ್ದ-ವರು
ಬಡಿ-ಸು-ತ್ತಿ-ದ್ದನು
ಬಡಿ-ಸು-ವಂತೆ
ಬಡಿ-ಸುವಾಗ
ಬಡಿ-ಸುವೆ
ಬಡಿದು
ಬಡಿಸ-ಬೇಕೆಂ-ದಿದ್ದರು
ಬಡಿಸಿ
ಬಡಿಸಿ-ದ-ಮೇಲೆ
ಬಡಿಸಿ-ದನು
ಬಡಿಸಿ-ದರು
ಬಡೋ
ಬಡ್ಡಿ
ಬಣ-ವೆಯ
ಬಣ್ಣ
ಬಣ್ಣ-ಗಳ
ಬಣ್ಣ-ಗಳದ್ದು
ಬಣ್ಣ-ದಿಂದ
ಬಣ್ಣ-ದಿಂದಲೇ
ಬಣ್ಣ-ವನ್ನು
ಬಣ್ಣದ
ಬಣ್ಣಿ-ಸ-ಲಾ-ರದು
ಬಣ್ಣಿ-ಸಿ-ದರು
ಬಣ್ಣಿ-ಸಿದ್ದರು
ಬಣ್ಣಿ-ಸು-ತ್ತಿ-ದ್ದರು
ಬಣ್ಣಿ-ಸು-ವುದು
ಬಣ್ಣಿ-ಸುವರು
ಬಣ್ಣಿ-ಸುವಳು
ಬದ-ರಿಗೆ
ಬದ-ರಿಯ
ಬದ-ಲಾಗು-ವು-ದಿ-ಲ್ಲ
ಬದನೆಕಾಯಿ-ಯನ್ನು
ಬದಲಾಗ-ಬೇಕು
ಬದಲಾಗದೆ
ಬದಲಾಗಿ
ಬದಲಾಗು-ತ್ತಿದ್ದೆ
ಬದಲಾಗು-ತ್ತಿವೆ
ಬದಲಾಯಿ-ಸದೆ
ಬದಲಾಯಿ-ಸಲು
ಬದಲಾಯಿ-ಸಿತು
ಬದಲಾಯಿ-ಸಿದೆ
ಬದಲಾಯಿ-ಸು-ವಂತೆ
ಬದಲಾಯಿ-ಸು-ವಷ್ಟು
ಬದಲಾಯಿ-ಸು-ವು-ದಿ-ಲ್ಲ
ಬದಲಾಯಿ-ಸು-ವುದು
ಬದಲಾಯಿ-ಸುವರು
ಬದಲಾಯಿಸ-ಬೇಕೆಂದು
ಬದಲಾಯಿಸಿ
ಬದಲಾಯಿಸಿ-ಕೊ-ಳ್ಳು-ತ್ತಿದ್ದರು
ಬದಲಾಯಿಸಿ-ಕೊಂಡರು
ಬದಲಾಯಿಸಿ-ತೆಂದೂ
ಬದಲಾಯಿಸಿ-ದರು
ಬದಲಾಯಿಸಿ-ರು-ವುದು
ಬದಲಾಯಿಸಿ-ರು-ವೆವು
ಬದಲಾವ-ಣೆಗೆ
ಬದಲಾವಣೆ
ಬದಲಾವಣೆ-ಗಳು
ಬದಲಾವಣೆ-ಗಳೆ-ಲ್ಲಾ
ಬದಲಾವಣೆ-ಗಾಗಿ
ಬದಲಾವಣೆ-ಯನ್ನು
ಬದಲಾವಣೆ-ಯಾ-ದರು
ಬದಲಾವಣೆ-ಯಾಗು-ತ್ತ
ಬದಲಾವಣೆ-ಯಿಂದ
ಬದಲಾವಣೆಯ
ಬದಲು
ಬದಿ-ಗಿಡು
ಬದಿ-ಗಿರಿ-ಸಿ-ರುವರು
ಬದಿ-ಯ-ಲ್ಲಿ
ಬದಿಗೆ
ಬದು-ಕನ್ನು
ಬದು-ಕಲಾರೆ
ಬದು-ಕಿ-ರುವ
ಬದು-ಕಿರು-ವ-ವ-ರೆಗೆ
ಬದು-ಕಿರು-ವನು
ಬದು-ಕಿರು-ವುದು
ಬದು-ರಿಗಿ-ರುವ
ಬದುಕ-ಬ-ಲ್ಲದು
ಬದುಕ-ಬೇ-ಕಾ-ದರೆ
ಬದುಕಲಿ
ಬದುಕಿ-ದರೆ-ಏನು
ಬದುಕಿ-ದ್ದರೆ
ಬದುಕಿ-ದ್ದಾಗ
ಬದುಕಿ-ನಿಂದ
ಬದುಕಿದ್ದ
ಬದುಕಿದ್ದ-ರೇನು
ಬದುಕಿನ
ಬದುಕಿರ-ಬೇ-ಕಾ-ದರೆ
ಬದುಕಿಸಿ
ಬದುಕಿಸಿ-ಕೊಳ್ಳ-ಬೇಕು
ಬದುಕು
ಬದುಕು-ವರೋ
ಬದುಕು-ವು-ದಿ-ಲ್ಲ
ಬದ್ಧ
ಬದ್ಧ-ಕಂಕಣ-ರಾ-ದರು
ಬದ್ಧ-ಕಂಕಣ-ರಾಗಿ-ರುವಂತಹ
ಬದ್ಧ-ಕಂಕಣ-ಳಾಗು-ವಂತೆ
ಬದ್ಧ-ಜೀವಿ
ಬದ್ಧ-ನಾಗು-ತ್ತಾನೆ
ಬದ್ಧ-ರಾಗಿ
ಬದ್ಧ-ರಾಗಿ-ರುವರು
ಬದ್ಧ-ಳಾಗಿ-ರು-ವಳು
ಬದ್ಧ-ವಾಗಿ-ದ್ದರೆ
ಬದ್ಧ-ಸೇವ-ಕರು
ಬದ್ಧರು
ಬನಿಯ
ಬನಿಯ-ನಿಗೆ
ಬನಿಯ-ನ್
ಬಫೆಲೋ
ಬಯ-ಸ-ಲಿ-ಲ್ಲ
ಬಯ-ಸಿದ
ಬಯ-ಸಿದ್ದ
ಬಯ-ಸು-ತ್ತಿ-ರ-ಲಿ-ಲ್ಲ
ಬಯ-ಸು-ವಂತೆಯೇ
ಬಯ-ಸು-ವು-ದಿ-ಲ್ಲ
ಬಯ-ಸು-ವೆನು
ಬಯ-ಸು-ವೆವು
ಬಯ-ಸುವ
ಬಯ-ಸುವನು
ಬಯ-ಸುವನೇ
ಬಯ-ಸುವರೊ
ಬಯ-ಸುವೆ
ಬಯಕೆ
ಬಯಕೆ-ಗ-ಳನ್ನು
ಬಯಕೆ-ಯನ್ನು
ಬಯಸಿ
ಬಯಸಿ-ದ-ವರು
ಬಯಸಿ-ದನು
ಬಯಸಿ-ದರು
ಬಯಸಿ-ದಾಗ
ಬಯಸಿ-ದ್ದಳು
ಬಯುಕೆ
ಬಯ್ಯು-ತ್ತಿದ್ದರೆ
ಬಯ್ಯು-ವು-ದ-ನ್ನು
ಬರ-ಕೂ-ಡದು
ಬರ-ಕೂಡ-ದೆಂದು
ಬರ-ಗಾಲ
ಬರ-ಗಾಲ-ಗಳ-ಲ್ಲಿ
ಬರ-ಗಾಲ-ದ-ಲ್ಲಿ
ಬರ-ಗಾಲ-ವನ್ನ-ಲ್ಲ
ಬರ-ತೊಡಗಿದವು
ಬರ-ದಂತೆ
ಬರ-ಬರು-ತ್ತ
ಬರ-ಬಹು-ದಾಗಿ-ತ್ತು
ಬರ-ಬಹುದು
ಬರ-ಬೇ-ಕಾ-ದರೆ
ಬರ-ಬೇ-ಕಾದ
ಬರ-ಬೇಕಾ-ಯಿತು
ಬರ-ಬೇಕಾಗಿ-ತ್ತು
ಬರ-ಬೇಕಾಗಿದೆ
ಬರ-ಬೇಕು
ಬರ-ಬೇಕೆಂ-ದಿದ್ದೇನೆ
ಬರ-ಬೇಕೆಂದು
ಬರ-ಬೇಕೆಂದೂ
ಬರ-ಬೇಡ
ಬರ-ಬೇಡಿ
ಬರ-ಮಾಡಿ-ಕೊ-ಳ್ಳಲು
ಬರ-ಮಾಡಿ-ಕೊ-ಳ್ಳು-ತ್ತಿದ್ದರು
ಬರ-ಮಾಡಿ-ಕೊ-ಳ್ಳು-ವು-ದಕ್ಕೆ
ಬರ-ಮಾಡಿ-ಕೊಂಡ
ಬರ-ಮಾಡಿ-ಕೊಂಡರು
ಬರ-ಮಾಡಿ-ಕೊಂಡಿ-ರುವರು
ಬರ-ಮಾಡಿ-ಕೊಂಡು
ಬರ-ಹೇಳಿ
ಬರಲು
ಬರಲೇ
ಬರಹ-ಗಳಿ-ಗಿಂತ
ಬರಹ-ಗಾರರು
ಬರಿ
ಬರಿ-ಸು-ವು-ದರ
ಬರೀ
ಬರು-ತ್ತ
ಬರು-ತ್ತ-ದೆ-ಯ-ಲ್ಲ
ಬರು-ತ್ತ-ದೆಯೋ
ಬರು-ತ್ತ-ವೆಂಬು-ದ-ನ್ನು
ಬರು-ತ್ತಲೂ
ಬರು-ತ್ತಾ-ರ-ಲ್ಲ
ಬರು-ತ್ತಾ-ರೆಯೆ
ಬರು-ತ್ತಾ-ರೆಯೊ
ಬರು-ತ್ತಾ-ರೆಯೋ
ಬರು-ತ್ತಾನೆ
ಬರು-ತ್ತಾರೆ
ಬರು-ತ್ತಾರೊ
ಬರು-ತ್ತಿ-ದ್ದು-ದ-ನ್ನು
ಬರು-ತ್ತಿ-ರು-ತ್ತವೆ
ಬರು-ತ್ತಿ-ರು-ವಿರಿ
ಬರು-ತ್ತಿ-ರು-ವು-ದ-ನ್ನು
ಬರು-ತ್ತಿದೆ
ಬರು-ತ್ತಿದ್ದಾರೆ
ಬರು-ತ್ತಿದ್ದುದು
ಬರು-ತ್ತಿದ್ದುವು
ಬರು-ತ್ತಿರು-ತ್ತಾರೆ
ಬರು-ತ್ತಿರು-ವ-ನೇನೋ
ಬರು-ತ್ತಿರು-ವರ
ಬರು-ತ್ತಿರು-ವರು
ಬರು-ತ್ತಿರು-ವಾಗ
ಬರು-ತ್ತಿರು-ವೆನು
ಬರು-ತ್ತಿರುವ
ಬರು-ತ್ತೀಯಾ
ಬರು-ತ್ತೀಯೆ
ಬರು-ತ್ತೀಯೋ
ಬರು-ತ್ತೀರಿ
ಬರು-ತ್ತೇನೆಂದು
ಬರು-ತ್ತೇನೆಂದೂ
ಬರು-ತ್ತೇವೆ
ಬರು-ತ್ತೇವೆಂದೂ
ಬರು-ಬರು-ತ್ತಾ
ಬರು-ವಂ-ತಿ-ಲ್ಲ
ಬರು-ವು-ದ-ಕ್ಕಾಗಿ
ಬರು-ವು-ದ-ನ್ನೇ
ಬರು-ವು-ದ-ರ-ಲ್ಲಿ
ಬರು-ವು-ದ-ಲ್ಲ
ಬರು-ವು-ದಕ್ಕೂ
ಬರು-ವು-ದರ
ಬರು-ವು-ದರ-ಲ್ಲಿ-ತ್ತು
ಬರು-ವು-ದರ-ಲ್ಲಿ-ದ್ದವು
ಬರು-ವು-ದೆಂದು
ಬರು-ವುದೆ
ಬರೆ
ಬರೆ-ದದ್ದು
ಬರೆ-ದರು
ಬರೆ-ದರೆ
ಬರೆ-ದಳು
ಬರೆ-ದಾದ
ಬರೆ-ದಾದ-ಮೇಲೆ
ಬರೆ-ದಿ-ಟ್ಟು
ಬರೆ-ದಿ-ತ್ತು
ಬರೆ-ದಿ-ದ್ದಾನೆ
ಬರೆ-ದಿ-ದ್ದಾನೆಂಬುದು
ಬರೆ-ದಿ-ರ-ಲಿ-ಲ್ಲ
ಬರೆ-ದಿ-ರು-ವುದು
ಬರೆ-ದಿ-ರುವ
ಬರೆ-ದಿ-ರುವನು
ಬರೆ-ದಿ-ರುವರು
ಬರೆ-ದಿ-ರುವೆ
ಬರೆ-ದಿಡ-ಬೇಕೆಂದು
ಬರೆ-ದಿಡು
ಬರೆ-ದಿಡು-ವು-ದ-ರಿಂದ
ಬರೆ-ದಿದ್ದ
ಬರೆ-ದಿದ್ದರು
ಬರೆ-ದಿದ್ದು
ಬರೆ-ದು-ಕೊ-ಟ್ಟನು
ಬರೆ-ದು-ಕೊ-ಟ್ಟರು
ಬರೆ-ದು-ಕೊ-ಟ್ಟು
ಬರೆ-ದು-ಕೊಡ-ಬಹು-ದಾಗಿ-ತ್ತು
ಬರೆ-ದು-ಕೊಡು-ವಂತೆ
ಬರೆ-ದು-ದ-ರಿಂದ
ಬರೆ-ಯ-ತೊಡಗಿದರು
ಬರೆ-ಯ-ಬೇಕು
ಬರೆ-ಯ-ಬೇಕೆಂದು
ಬರೆ-ಯ-ಬೇಕೆಂದೂ
ಬರೆ-ಯ-ಲಾಗು-ವು-ದಿ-ಲ್ಲ
ಬರೆ-ಯ-ಲ್ಪಟ್ಟಿತು
ಬರೆ-ಯಲಾಗು-ವುದು
ಬರೆ-ಯಲಾರೆ
ಬರೆ-ಯಲಿ
ಬರೆ-ಯಿತು
ಬರೆ-ಯಿರಿ
ಬರೆ-ಯು-ತ್ತ
ಬರೆ-ಯು-ತ್ತಾರೆ
ಬರೆ-ಯು-ತ್ತಿದ್ದ
ಬರೆ-ಯು-ತ್ತಿದ್ದರು
ಬರೆ-ಯು-ತ್ತಿದ್ದಾನೆ
ಬರೆ-ಯು-ತ್ತಿರು-ವೆನು
ಬರೆ-ಯು-ತ್ತಿರುವ
ಬರೆ-ಯು-ವಂತೆಯೂ
ಬರೆ-ಯು-ವು-ದಕ್ಕೆ
ಬರೆ-ಯು-ವು-ದಾಗಿ
ಬರೆ-ಯು-ವುದು
ಬರೆ-ಯುವ
ಬರೆ-ಯುವ-ವ-ರಿಂದ
ಬರೆ-ಯುವ-ವ-ರೆಗೆ
ಬರೆ-ಯುವರು
ಬರೆ-ಯುವಾಗ
ಬರೆ-ಯುವೆ
ಬರೆ-ಸಿ-ಕೊಂಡಿತು
ಬರೆ-ಸಿ-ದರು
ಬರೆದ
ಬರೆದು
ಬರೇ-ಲಿಗೆ
ಬರೋ-ಸ್
ಬರೋಡಕ್ಕೆ
ಬರ್ದವಾ-ನ್
ಬರ್ಲಿನ್
ಬರ್ಲಿನ್ಗೆ
ಬರ್ನಾರ್ಡ್
ಬಲ
ಬಲ-ಗಳ-ಲ್ಲಿ
ಬಲ-ದಿಂದ
ಬಲ-ದಿಂದಲೇ
ಬಲ-ಪಡಿ-ಸಿ-ಕೊ-ಳ್ಳುವು-ದರ
ಬಲ-ಪ್ರದ-ವಾಗುವ
ಬಲ-ಭುಜದ
ಬಲ-ಮಸಿ
ಬಲ-ರಾಮ
ಬಲ-ರಾಮ-ಬಾಬು-ಗಳ
ಬಲ-ರಾಮ-ಬೋ-ಸ್
ಬಲ-ರಾಮ-ಬೋಸರ
ಬಲ-ರಾಮ-ಬೋಸರು
ಬಲ-ರಾಮನ
ಬಲ-ವಂತ
ಬಲ-ವಂತ-ದಿಂದ
ಬಲ-ವಂತ-ವಾಗಿ
ಬಲ-ವತ್ತರ-ವಾಗು-ತ್ತ
ಬಲ-ವನ್ನು
ಬಲ-ವನ್ನೆ-ಲ್ಲ
ಬಲ-ವಾ-ಗಿದೆ
ಬಲ-ವಾ-ದಾಗ
ಬಲ-ವಾಗಿ
ಬಲ-ವಾಗಿ-ದ್ದರೂ
ಬಲ-ವಾಗಿ-ದ್ದರೇನೆ
ಬಲ-ವಾಗಿ-ದ್ದು-ದ-ರಿಂದ
ಬಲ-ವಾಗಿ-ರು-ವಂತೆ
ಬಲ-ವಾಗಿ-ರು-ವುದು
ಬಲ-ವಾಗಿಯೇ
ಬಲ-ವಾಗು-ವುದು
ಬಲ-ವಾದ
ಬಲ-ಶಾಲಿ-ಯಾದ
ಬಲ-ಹೀನ-ರಿಗೆ
ಬಲ-ಹೀನೇನ
ಬಲಂ
ಬಲಗಾ-ಲ-ನ್ನು
ಬಲಗೈ-ನಂತೆ
ಬಲಗೈ-ಯಿಂದ
ಬಲವಂ-ತಕ್ಕೆ
ಬಲಾ-ತ್ಕ-ರಿಸಿ-ದಾಗ
ಬಲಾ-ತ್ಕರಿ-ಸಿ-ದನು
ಬಲಾ-ತ್ಕರಿ-ಸು-ತ್ತಿದೆ
ಬಲಾ-ತ್ಕರಿಸ-ಬೇಡಿ
ಬಲಾ-ತ್ಕಾ-ರಕ್ಕೆ
ಬಲಾ-ತ್ಕಾರ
ಬಲಾ-ತ್ಕಾರ-ದಿಂದ
ಬಲಾ-ತ್ಕಾರ-ವಾಗಿ
ಬಲಾ-ತ್ಕಾರ-ವಿ-ಲ್ಲದೆ
ಬಲಾ-ತ್ಕಾರವು
ಬಲಾ-ತ್ಕಾರಿಸಿ
ಬಲಾಢ್ಯ-ನ-ನ್ನಾಗಿ
ಬಲಾಢ್ಯ-ನಾಗಿ-ರ-ಬೇಕು
ಬಲಾಢ್ಯ-ವಾದ
ಬಲಿ-ಕೊ-ಟ್ಟು
ಬಲಿ-ಕೊ-ಡದೆ
ಬಲಿ-ಕೊಡ-ಕೂ-ಡದು
ಬಲಿ-ಕೊಡ-ತಕ್ಕ-ವ-ನಾಗಿಯೂ
ಬಲಿ-ಕೊಡು-ವು-ದಕ್ಕೆ
ಬಲಿ-ದಾನ-ಮಾಡಿ
ಬಲಿ-ಯಾ-ದರೂ
ಬಲಿ-ಯಾಗಿಯೂ
ಬಲಿ-ಯಾಗುವುದು
ಬಲಿಪೀಠ-ವಾಗಿಯೂ
ಬಲಿಷ್ಠ-ನಾಗಿ-ದ್ದರೆ
ಬಲಿಷ್ಠ-ರಾಗಿ
ಬಲಿಷ್ಠ-ವಾಗಲಿ
ಬಲಿಷ್ಠ-ವಾಗು-ವುದು
ಬಲಿಷ್ಠನೂ
ಬಲಿಷ್ಠರಾಗ-ಬೇ-ಕಾ-ದರೆ
ಬಲು
ಬಲು-ಬೇಗ
ಬಲೆ
ಬಲೆ-ಗಿಂತ
ಬಲೆ-ಯ-ಲ್ಲಿರುವ
ಬಲೆ-ಯನ್ನು
ಬಲೆ-ಯಿಂದ
ಬಲೆ-ಯೆ-ಲ್ಲ
ಬಲೆಗೆ
ಬಳ-ಕೆಯ-ಲ್ಲಿವೆ
ಬಳ-ಸನ್ನು
ಬಳ-ಸುವರು
ಬಳ-ಸುವುದ-ರ-ಲ್ಲಿ
ಬಳಕೆ-ಯ-ಲ್ಲಿ-ರು-ವುದು
ಬಳಗ
ಬಳಗ-ವನ್ನೇ
ಬಳಲಿ
ಬಳಸಿ
ಬಳಸಿ-ಕೊಂಡರೆ
ಬಳಸಿ-ರುವ
ಬಳಿ
ಬಳಿ-ದಂತೆ
ಬಳಿ-ದರು
ಬಳಿ-ದರೂ
ಬಳಿ-ದು-ಕೊಂಡು
ಬಳಿ-ಯ-ಬೇಕೊ
ಬಳಿಗೇ
ಬಳ್ಳಿ-ಯನ್ನು
ಬಸವ-ಳಿದು
ಬಸಾ-ಕರ
ಬಸಿರೊಳಗೆ
ಬಹದ್ದೂರ್
ಬಹಳ
ಬಹಳ-ಕಾಲ
ಬಹಳ-ವಾಗಿ
ಬಹಳ-ವೆ-ನ್ನು-ವುದೇನೋ
ಬಹಿಃ-ಪ್ರಕಾಶ
ಬಹಿರಂಗ
ಬಹಿರಂಗ-ದ-ಲ್ಲಿ
ಬಹಿರಂಗ-ವಾಗಿ
ಬಹಿರಂಗ-ವಾಗಿಯೇ
ಬಹಿರ್ದೃಷ್ಟಿ-ಯಾ-ದಂತೆ
ಬಹಿರ್ಮುಖ
ಬಹಿರ್ಮುಖಿ-ಯ-ನ್ನಾಗಿ
ಬಹಿಷ್ಕರಿ-ಸಿ-ದರು
ಬಹು
ಬಹು-ಕಾಲ
ಬಹು-ಕಾಲ-ದಿಂದಲೂ
ಬಹು-ಕಾಲದ
ಬಹು-ಜ-ನರ
ಬಹು-ಜನ
ಬಹು-ಜನ-ರಿ-ದ್ದರು
ಬಹು-ಜನ-ರೊ-ಡನೆ
ಬಹು-ತೀಕ್ಷ್ಣ-ವಾಗಿ-ರು-ತ್ತವೆ
ಬಹು-ದೂರ-ದಿಂದ
ಬಹು-ಪಾ-ಲ-ನ್ನು
ಬಹು-ಪಾಲು
ಬಹು-ಬೇಗ
ಬಹು-ಭಾಗ
ಬಹು-ಭಾಗ-ವನ್ನು
ಬಹು-ಮ-ಟ್ಟಿಗೆ
ಬಹು-ಮಾನ
ಬಹು-ಮಾನ-ವನ್ನು
ಬಹು-ಮಾನ-ವಾಗಿ
ಬಹು-ಮಾನವೂ
ಬಹು-ಮುಖ-ವಾ-ದುದು
ಬಹು-ಮುಖ-ವಾದ
ಬಹು-ಮುಖದ
ಬಹು-ವಾಗಿ
ಬಹು-ವಾಗಿವೆ
ಬಹುಶಃ
ಬಹೆಗ-ರಿ-ಸು-ವು-ದಕ್ಕೆ
ಬಹ್ಮ-ಸೂ-ತ್ರ
ಬಾ
ಬಾಂಗ್ಲಾ-ನಿಂದ
ಬಾಂಧವ್ಯ
ಬಾಂಬಿ-ನಂತೆ
ಬಾಕ್ಸಿಂಗ್
ಬಾಗ-ಬ-ಜಾರಿ-ನ-ಲ್ಲಿ
ಬಾಗ-ಲಿ-ಲ್ಲ
ಬಾಗದ-ವರೇ
ಬಾಗಿ
ಬಾಗಿ-ದರು
ಬಾಗಿ-ಲ-ನ್ನು
ಬಾಗಿ-ಲ-ಲ್ಲಿಯೇ
ಬಾಗಿ-ಲಿ-ನ-ಲ್ಲಿ
ಬಾಗಿ-ಲಿ-ನಿಂದ
ಬಾಗಿ-ಲಿ-ನಿಂದಲೇ
ಬಾಗಿ-ಲಿಗೆ
ಬಾಗಿ-ಲಿನ
ಬಾಗಿ-ಲಿನ-ವ-ರೆಗೆ
ಬಾಗಿ-ಲು-ಗ-ಳನ್ನು
ಬಾಗಿದ್ದ
ಬಾಗಿಲು
ಬಾಗು-ವು-ದಿ-ಲ್ಲ
ಬಾಗ್
ಬಾಗ್ಬ-ಜಾರಿ-ನ-ಲ್ಲಿ-ರುವ
ಬಾಗ್ಬ-ಜಾರಿನ
ಬಾಚಿ
ಬಾಡಿ
ಬಾಡಿ-ಗೆ-ಯನ್ನು
ಬಾಡಿ-ಗೆಗೆ
ಬಾಡಿ-ಗೆಯ
ಬಾಡಿ-ದ್ದಂತೆ
ಬಾಡಿ-ಯ-ಲ್ಲಿದ್ದರು
ಬಾಡಿ-ಹೋದ
ಬಾಡಿಗೆ
ಬಾಡ್ಲಿಯ-ನ್
ಬಾಣ
ಬಾಣ-ಬಿರುಸು-ಗ-ಳನ್ನು
ಬಾಣ-ವನ್ನು
ಬಾತು-ಗಳು
ಬಾತೊಂದು
ಬಾದಾಮಿ-ಯನ್ನು
ಬಾಧಕ-ವಿ-ಲ್ಲ-ವೆಂದು
ಬಾಧಿ-ಸಿತು
ಬಾಧಿ-ಸು-ತ್ತಿ-ತ್ತು
ಬಾಧಿಸಿ-ದವು
ಬಾಧೆ-ಯಿಂದ
ಬಾಧ್ಯತೆ-ಗ-ಳನ್ನು
ಬಾಬ-ರ-ನ್ನು
ಬಾಬರ-ವ-ರಿಂದ
ಬಾಬಾ
ಬಾಬಾ-ಜಿ-ಯ-ವ-ರ-ನ್ನು
ಬಾಬಾ-ಜಿ-ಯ-ವರ
ಬಾಬಾ-ಜಿ-ಯ-ವರೆ
ಬಾಬಾ-ಜಿಗೆ
ಬಾಬಾಜಿ
ಬಾಬಾರ
ಬಾಬು
ಬಾಬು-ಗ-ಳಿಗೆ
ಬಾಬು-ಗಳ
ಬಾಬು-ಗಳಂತೆ
ಬಾಬು-ಗಳು
ಬಾಬು-ಗಳೂ
ಬಾಬು-ಗಳೊಂದಿಗೆ
ಬಾಬು-ರಾಮ
ಬಾಬು-ರಾಮ-ಪ್ರೇಮಾ-ನಂದ
ಬಾಬು-ರಾಮನ
ಬಾಬು-ವನ್ನು
ಬಾಬು-ವಿಗೆ
ಬಾಬು-ವಿನ
ಬಾಬುವೇ
ಬಾಯಿ
ಬಾಯಿ-ಗಳ
ಬಾಯಿ-ತಪ್ಪಿ
ಬಾಯಿ-ನ-ಲ್ಲಿ
ಬಾಯಿ-ನಿಂದ
ಬಾಯಿ-ನಿಂದಲೇ
ಬಾಯಿ-ಮಾತಿ-ನ-ಲ್ಲಿ
ಬಾಯಿ-ಯ-ಲ್ಲಿಯೂ
ಬಾಯಿ-ಯ-ಲ್ಲೂ
ಬಾಯಿ-ಯಿಂದ
ಬಾಯಿಂದ
ಬಾಯಿಗೆ
ಬಾಯಿಯ
ಬಾಯಿಯೂ
ಬಾಯ್ಸ್
ಬಾರ-ನ್
ಬಾರಯ್ಯ
ಬಾರಾ-ನ-ಗರ
ಬಾರಾ-ನ-ಗರ-ದ-ಲ್ಲಿ
ಬಾರಾ-ನ-ಗರದ
ಬಾರಿ
ಬಾರಿ-ಯ-ಲ್ಲ
ಬಾರಿ-ಸ-ಬೇಕು
ಬಾರಿ-ಸಲು
ಬಾರಿ-ಸಿ-ದಂತೆ
ಬಾರಿ-ಸಿ-ದರು
ಬಾರಿ-ಸಿದ
ಬಾರಿ-ಸು-ತ್ತ
ಬಾರಿ-ಸು-ವು-ದ-ನ್ನು
ಬಾರಿ-ಸುವುದ-ರ-ಲ್ಲಿಯೂ
ಬಾರಿಯೂ
ಬಾರ್ನ-ಸ್
ಬಾರ್ನಾರ್ಡ್
ಬಾಲ
ಬಾಲ-ಕ-ರಂತೆ
ಬಾಲ-ಕನ
ಬಾಲ-ಕನೇ
ಬಾಲ-ಕಿ-ಯ-ರಿಗೆ
ಬಾಲ-ಕಿ-ಯರ
ಬಾಲ-ಕಿಯ-ರಿ-ಗಾಗಿ
ಬಾಲ-ಕ್ರಿ-ಸ್ತನ
ಬಾಲ-ಗಂಗಾ-ಧರ-ತಿಲ-ಕರು
ಬಾಲ-ಗೊಪಾಲ-ನಂತೆ
ಬಾಲ-ದಂತೆ
ಬಾಲ-ಬ್ರಹ್ಮ-ಚಾರಿ-ಗಳು
ಬಾಲ-ರಿ-ದ್ದರು
ಬಾಲಿಕಾ
ಬಾಳ-ಬೇ-ಕಾ-ದರೆ
ಬಾಳ-ಬೇಕು
ಬಾಳ-ಬೇಕೆಂಬ
ಬಾಳ-ಲಾ-ರದು
ಬಾಳ-ಲಿ-ಲ್ಲ
ಬಾಳಲಾಗ-ಲಿ-ಲ್ಲ-ವೆಂ-ದರೆ
ಬಾಳಲು
ಬಾಳಿ
ಬಾಳಿ-ದರು
ಬಾಳಿ-ನಿಂದ
ಬಾಳಿ-ರುವರು
ಬಾಳಿನ
ಬಾಳು
ಬಾಳು-ತ್ತದೆಯೊ
ಬಾಳು-ವ-ರೆಂದೂ
ಬಾಳು-ವಂತೆ
ಬಾಳು-ವಷ್ಟು
ಬಾಳು-ವು-ದಕ್ಕೆ
ಬಾಳು-ವುದ-ರ-ಲ್ಲಿ
ಬಾಳು-ವುದು
ಬಾಳೆ
ಬಾಳೆ-ಗಿಡ-ವನ್ನು
ಬಾಳೆ-ಹಣ್ಣ-ನ್ನು
ಬಾಳೆಯ
ಬಾಳೇನು
ಬಾವಿ-ಯ-ಲ್ಲಿ
ಬಾವಿಯ
ಬಾವಿಯ-ಲ್ಲಿ-ರುವ
ಬಾವಿಯಷ್ಟು
ಬಾವುಟ
ಬಾವುಟ-ಗ-ಳನ್ನು
ಬಾವುಟ-ಗಳು
ಬಾವೋದ್ರೇಕಕ್ಕೆ
ಬಾಹಿರ-ರ-ಲ್ಲ
ಬಾಹ್ಯ
ಬಾಹ್ಯ-ಜಗ-ತ್ತನ್ನು
ಬಾಹ್ಯ-ಜೀ-ನ-ದ-ಲ್ಲಿ
ಬಾಹ್ಯ-ಜ್ಞಾನ
ಬಾಹ್ಯ-ಜ್ಞಾನ-ವಿ-ಲ್ಲದೆ
ಬಾಹ್ಯ-ದ-ಲ್ಲಿ
ಬಾಹ್ಯ-ದಿಂದ
ಬಾಹ್ಯ-ನಡವಳಿಕೆಗೂ
ಬಾಹ್ಯ-ಪೂಜೆ
ಬಾಹ್ಯ-ಪೂಜೆ-ಗಳೂ
ಬಾಹ್ಯ-ಪೂಜೆಗೂ
ಬಾಹ್ಯ-ಪ್ರಜ್ಞೆ
ಬಾಹ್ಯ-ಪ್ರಪಂಚ-ವನ್ನು
ಬಾಹ್ಯ-ರೂಪ
ಬಾಹ್ಯ-ವಿಧಿ-ಗಳ-ನ್ನನು-ಸರಿ-ಸ-ಬೇಕು
ಬಾಹ್ಯ-ವೇಷ
ಬಾಹ್ಯ-ವೇಷವೇ
ಬಾಹ್ಯದ
ಬಾಹ್ಯಾ-ಕಾರ-ವನ್ನು
ಬಾಹ್ಯಾ-ಚರಣೆ-ಗಳ-ನ್ನನು-ಸರಿ-ಸ-ದಿ-ರುವ
ಬಾಹ್ಯಾ-ಚರಣೆ-ಗಳ-ನ್ನಾ-ಚರಿ-ಸದೆ
ಬಾಹ್ಯಾ-ಚಾರ
ಬಾಹ್ಯಾ-ಚಾರ-ಗಳೇ
ಬಾಹ್ಯಾ-ಚಾರ-ದ-ಲ್ಲಿ
ಬಾಹ್ಯಾಡಂಬ-ರಕ್ಕೆ
ಬಾಹ್ಯಾರ್ಥ-ವನ್ನು
ಬಾಹ್ಯಾಲಂಬ-ನವೇ
ಬಾಹ್ಯಾಲಂಬನ
ಬಾಹ್ಯಾಲಂಬನ-ವನ್ನೇ
ಬಿ
ಬಿಂಕಿ
ಬಿಂಕಿಗೆ
ಬಿಂದು
ಬಿಂದು-ವನ್ನು
ಬಿಂದು-ವನ್ನೂ
ಬಿಂದು-ವಿಗೆ
ಬಿಂದುವೂ
ಬಿಎ
ಬಿಗಿ
ಬಿಗಿ-ಯಾಗಿ
ಬಿಗಿ-ಹಿಡಿ-ದು-ಕೊಂಡು
ಬಿಚ್ಚಿ
ಬಿಚ್ಚು-ತ್ತಾರೆ
ಬಿಚ್ಚು-ವರು
ಬಿಟಿಷ್
ಬಿಡ-ಕೂಡ-ದೆಂದು
ಬಿಡ-ಬಹು-ದೆಂದು
ಬಿಡ-ಬೇಕಾ-ಯಿತು
ಬಿಡ-ಬೇಕಾಗಿ
ಬಿಡ-ಬೇಕು
ಬಿಡ-ಬೇಕೆ-ನ್ನಿ-ಸು-ವುದು
ಬಿಡ-ಬೇಕೆಂದು
ಬಿಡ-ಬೇಡ
ಬಿಡ-ಬೇಡಿ
ಬಿಡ-ಲಿ-ಲ್ಲ
ಬಿಡಲಿ
ಬಿಡಲು
ಬಿಡಲೇ
ಬಿಡಾ-ರಕ್ಕೆ
ಬಿಡಾರ-ಮಾಡಿ-ದ್ದರು
ಬಿಡಾರ-ವನ್ನು
ಬಿಡಾರ-ವಾಗಿ
ಬಿಡಿ
ಬಿಡಿ-ಸಲು
ಬಿಡಿ-ಸಿ-ಕೊಂಡರು
ಬಿಡಿ-ಸಿ-ಕೊಂಡು
ಬಿಡಿ-ಸಿ-ದರು
ಬಿಡಿ-ಸಿದೆ
ಬಿಡಿ-ಸು-ತ್ತಿ-ದ್ದರು
ಬಿಡಿ-ಸು-ವು-ದ-ಕ್ಕಾಗಿ
ಬಿಡಿ-ಸು-ವು-ದಕ್ಕೆ
ಬಿಡಿಸಿ
ಬಿಡು-ಗಡೆ
ಬಿಡು-ಗಡೆಯ
ಬಿಡು-ತ್ತಾನೆ
ಬಿಡು-ತ್ತಾರೆ
ಬಿಡು-ತ್ತಾರೆಯೆ
ಬಿಡು-ತ್ತಿ-ರಲಿ-ಲ್ಲ
ಬಿಡು-ತ್ತಿದ್ದರು
ಬಿಡು-ತ್ತೇನೆ
ಬಿಡು-ವ-ವ-ರೆಗೆ
ಬಿಡು-ವ-ವರ
ಬಿಡು-ವಂತೆ
ಬಿಡು-ವಾ-ದಾಗ-ಲೆ-ಲ್ಲ
ಬಿಡು-ವಾಗ
ಬಿಡು-ವಾಗಿ-ತ್ತು
ಬಿಡು-ವಿ-ಲ್ಲ
ಬಿಡು-ವಿ-ಲ್ಲದ
ಬಿಡು-ವಿ-ಲ್ಲದೆ
ಬಿಡು-ವು-ದಕ್ಕೆ
ಬಿಡು-ವು-ದಿ-ಲ್ಲ
ಬಿಡು-ವು-ದಿ-ಲ್ಲ-ವೆಂಬ
ಬಿಡು-ವುದ-ರ-ಲ್ಲಿ
ಬಿಡು-ವೆನು
ಬಿಡುವ
ಬಿಡುವೆ
ಬಿಡುವೇ
ಬಿದ್ದ
ಬಿದ್ದ-ಮೇಲೆ
ಬಿದ್ದ-ರೆಂದು
ಬಿದ್ದ-ವ-ನನ್ನು
ಬಿದ್ದ-ವ-ರ-ನ್ನು
ಬಿದ್ದ-ವ-ರಿಗೆ
ಬಿದ್ದದ್ದು
ಬಿದ್ದನು
ಬಿದ್ದರು
ಬಿದ್ದರೆ
ಬಿದ್ದರೋ
ಬಿದ್ದಳು
ಬಿದ್ದಾಗ
ಬಿದ್ದಿ-ತ್ತು
ಬಿದ್ದಿ-ರು-ತ್ತಿದ್ದೆನೊ
ಬಿದ್ದಿ-ರುವ
ಬಿದ್ದಿ-ಲ್ಲ
ಬಿದ್ದಿದ್ದ
ಬಿದ್ದಿದ್ದಾಗ
ಬಿದ್ದಿದ್ದು
ಬಿದ್ದಿರು-ತ್ತಿದ್ದವು
ಬಿದ್ದಿರು-ವು-ದೆಂದು
ಬಿದ್ದಿವೆ
ಬಿದ್ದು
ಬಿದ್ದು-ದ-ನ್ನು
ಬಿದ್ದು-ದ-ರಿಂದ
ಬಿದ್ದು-ಬಿ-ಟ್ಟರು
ಬಿದ್ದು-ಬಿ-ಟ್ಟೆ
ಬಿದ್ದು-ಬಿಡು-ವರೊ
ಬಿದ್ದು-ಹೋಗದೆ
ಬಿದ್ದು-ಹೋಗಿ-ರುವ
ಬಿದ್ದು-ಹೋಗು-ವುದು
ಬಿದ್ದು-ಹೋಗುವ
ಬಿದ್ದೊಡ-ನೆಯೆ
ಬಿಪಿ-ನ್-ಚಂದ್ರ-ಪಾ-ಲ್
ಬಿರು-ಗಾಳಿ-ಯಂತೆ
ಬಿರು-ಗಾಳಿಗೆ
ಬಿರು-ಸಾಗಿ
ಬಿರುಕಿ-ನ-ಲ್ಲಿ
ಬಿರುಕು
ಬಿರುಸಿ-ನಿಂದ
ಬಿರುಸು
ಬಿಲ-ಗಳಿಂದ
ಬಿಲೆ
ಬಿಳಿ-ಗಿರಿ
ಬಿಳಿಯ
ಬಿಷಪ್
ಬಿಸಾಡಿ
ಬಿಸಾಡಿ-ದ-ರೆಂದೂ
ಬಿಸಾಡಿದ
ಬಿಸಾಡು
ಬಿಸಾಡು-ವರು
ಬಿಸಿ
ಬಿಸಿ-ರಕ್ತ
ಬಿಸಿ-ಲ-ಲ್ಲಿ
ಬಿಸಿ-ಲಿ-ನ-ಲ್ಲಿ
ಬಿಸಿ-ಲು-ಮಳೆ-ಯೆಂಬ
ಬಿಸಿಲ
ಬಿಸಿಲು
ಬಿಸುಟ
ಬಿಸುಟರು
ಬಿಸುಟು
ಬೀ
ಬೀಗ
ಬೀಗ-ಮುದ್ರೆ
ಬೀಗ-ವನ್ನು
ಬೀಗದ
ಬೀಜ
ಬೀಜ-ಗ-ಳನ್ನು
ಬೀಜ-ಗ-ಳಿಗೆ
ಬೀಜ-ಗಳ-ನ್ನೆ-ಲ್ಲ
ಬೀಜ-ಗಳು
ಬೀಜ-ಗಳೂ
ಬೀಜ-ಗಳೆ-ಲ್ಲ
ಬೀಜ-ನಿಗೆ
ಬೀಜ-ವನ್ನು
ಬೀಜ-ವಾ-ಗಿದೆ
ಬೀಜಾಂಕುರ
ಬೀಡಿ
ಬೀಡೋ
ಬೀದಿ-ಗಳ-ನ್ನೆ-ಲ್ಲ
ಬೀದಿ-ಗಳ-ಲ್ಲಿ
ಬೀದಿ-ಬೀದಿ-ಗಳ-ಲ್ಲಿ
ಬೀದಿ-ಯ-ಲ್ಲಿ
ಬೀದಿಗೆ
ಬೀದಿಯ
ಬೀದಿಯೊಂದ-ರ-ಲ್ಲಿ
ಬೀರ-ತೊಡಗಿದರು
ಬೀರ-ತೊಡಗಿದುವು
ಬೀರ-ಬ-ಲ್ಲದು
ಬೀರಿ
ಬೀರಿ-ದಂತೆ
ಬೀರಿ-ದನು
ಬೀರಿ-ದರು
ಬೀರಿ-ದವು
ಬೀರಿ-ದ್ದರು
ಬೀರಿ-ದ್ದರೂ
ಬೀರಿ-ದ್ದವು
ಬೀರಿ-ದ್ದಾಳೆ
ಬೀರಿ-ರುವರು
ಬೀರು-ತ್ತ
ಬೀರು-ತ್ತಿದೆ
ಬೀರು-ತ್ತಿರುವ
ಬೀಳ-ದಂತೆ
ಬೀಳ-ಬಹುದು
ಬೀಳ-ಬೇಕು
ಬೀಳ-ಲಿ-ಲ್ಲ
ಬೀಳದ
ಬೀಳದೆ
ಬೀಳಿ
ಬೀಳಿ-ಸಲು
ಬೀಳು-ತ್ತದೆ
ಬೀಳು-ತ್ತಿ-ತ್ತು
ಬೀಳು-ತ್ತಿ-ದ್ದು-ದ-ನ್ನು
ಬೀಳು-ತ್ತಿ-ರು-ವು-ದ-ನ್ನು
ಬೀಳು-ತ್ತಿದೆ
ಬೀಳು-ತ್ತಿದೆಯೋ
ಬೀಳು-ತ್ತಿವೆ
ಬೀಳು-ತ್ತೇನೆ
ಬೀಳು-ವಂತೆ
ಬೀಳು-ವಂತೆಯೇ
ಬೀಳು-ವನು
ಬೀಳು-ವರು
ಬೀಳು-ವು-ದಕ್ಕೆ
ಬೀಳು-ವು-ದಿ-ಲ್ಲ
ಬೀಳು-ವುದು
ಬೀಳು-ವುದೋ
ಬೀಳು-ವುವು
ಬೀಳು-ವೆನು
ಬೀಳು-ವೆವು
ಬೀಳುವ
ಬೀಳುವುದ-ರ-ಲ್ಲಿ
ಬೀಳ್ಕೊ-ಟ್ಟ
ಬೀಳ್ಕೊ-ಟ್ಟರು
ಬೀಳ್ಕೊ-ಟ್ಟಳು
ಬೀಳ್ಕೊ-ಳ್ಳುವಾಗ
ಬೀಳ್ಕೊಂಡ-ಮೇಲೆ
ಬೀಳ್ಕೊಂಡರು
ಬೀಳ್ಕೊಂಡಳು
ಬೀಳ್ಕೊಂಡು
ಬೀಳ್ಕೊಂಡೆ
ಬೀಳ್ಕೊಡಲು
ಬೀಳ್ಕೊಡು-ವಾಗ
ಬೀಳ್ಕೊಡು-ವು-ದಕ್ಕೆ
ಬೀಳ್ಕೊಡುವ
ಬೀಸಣಿಗೆ-ಯಂತಿ-ರುವ
ಬೀಸಣಿಗೆ-ಯನ್ನು
ಬೀಸಣಿಗೆ-ಯಿಂದ
ಬೀಸಿ
ಬೀಸಿ-ಕೊ-ಳ್ಳು-ತ್ತ
ಬೀಸಿ-ದರೆ
ಬೀಸಿ-ದಳು
ಬೀಸು
ಬೀಸು-ವಂತೆ
ಬೀಸು-ವು-ದಕ್ಕೆ
ಬೀಸು-ವುದು
ಬು
ಬುಕ್
ಬುಡ-ವಿ-ಲ್ಲದ
ಬುಡ್ಡಿ-ಗಳ
ಬುಡ್ಡಿ-ಯನ್ನು
ಬುಡ್ಡಿಯ
ಬುದ್ಧ
ಬುದ್ಧ-ಗ-ಯೆಗೆ
ಬುದ್ಧ-ದೇವ
ಬುದ್ಧ-ದೇವನು
ಬುದ್ಧ-ನಂತೆ
ಬುದ್ಧ-ನದು
ಬುದ್ಧ-ನನ್ನು
ಬುದ್ಧ-ನಾದ
ಬುದ್ಧ-ನಿ-ಗಿಂತ
ಬುದ್ಧ-ನಿ-ಲ್ಲದೆ
ಬುದ್ಧ-ನೆಂದು
ಬುದ್ಧ-ರಿಗೂ
ಬುದ್ಧನ
ಬುದ್ಧನೋ
ಬುದ್ಧರು
ಬುದ್ಧಾ-ವತಾರ-ದ-ಲ್ಲಿ
ಬುದ್ಧಿ
ಬುದ್ಧಿ-ಗೇನೋ
ಬುದ್ಧಿ-ಯ-ಲ್ಲಿ
ಬುದ್ಧಿ-ಯಾ-ದರೋ
ಬುದ್ಧಿ-ಯಿಂದ
ಬುದ್ಧಿ-ವಂ-ತ-ನಾದ
ಬುದ್ಧಿ-ವಂ-ತ-ರಾಗಿ
ಬುದ್ಧಿ-ವಂತ-ನಾಗಿದ್ದ
ಬುದ್ಧಿ-ವಂತ-ನಾಗು-ತ್ತ
ಬುದ್ಧಿ-ವಂತ-ನೆನಿಸಿ-ಕೊಂಡಿ-ರ-ಲಿ-ಲ್ಲ
ಬುದ್ಧಿ-ವಂತ-ರ-ನ್ನು
ಬುದ್ಧಿ-ವಂತ-ರಾದ
ಬುದ್ಧಿ-ವಂತ-ರಾದ-ವರು
ಬುದ್ಧಿ-ವಂತರು
ಬುದ್ಧಿ-ವಂತಿಕೆ-ಯನ್ನು
ಬುದ್ಧಿ-ವಾದ-ದಂತೆ
ಬುದ್ಧಿ-ವಾದ-ವನ್ನು
ಬುದ್ಧಿ-ಶಕ್ತಿ
ಬುದ್ಧಿ-ಶಕ್ತಿ-ಯನ್ನು
ಬುದ್ಧಿ-ಶಕ್ತಿ-ಯನ್ನೂ
ಬುದ್ಧಿ-ಶಕ್ತಿ-ಯಾ-ದರೋ
ಬುದ್ಧಿ-ಶಕ್ತಿಗೆ
ಬುದ್ಧಿ-ಹೇಳಿ
ಬುದ್ಧಿಗೆ
ಬುದ್ಧಿಯ
ಬುದ್ಧಿಯೂ
ಬುದ್ಧಿಯೆ
ಬುದ್ಧಿಯೋ
ಬುಧ-ವಾರ
ಬುಧಾಷ್ಟ-ಮಿಯ
ಬುಲೆ-ಟ್ಗ-ಳನ್ನು
ಬೂಟಾಟಿಕೆ-ಗಳಿಂದ
ಬೂದಿ
ಬೂದಿ-ಯ-ಲ್ಲದೆ
ಬೂದಿ-ಯನ್ನು
ಬೂದಿ-ಯಾ-ಯಿತು
ಬೂದಿ-ಯಾಗಿ-ತ್ತು
ಬೂದಿ-ಯಾಗುವುದು
ಬೂದಿ-ಯಿಂದ
ಬೃಂ
ಬೃಹ-ತ್
ಬೃಹ-ತ್ಕಥೆ
ಬೃಹ-ತ್ತಾ-ದುದು
ಬೃಹದಾ-ಕಾರ-ವುಳ್ಳ-ದ್ದ-ನ್ನು
ಬೃಹದಾ-ಕಾರದ
ಬೆ
ಬೆಂ
ಬೆಂಕಿ
ಬೆಂಕಿ-ಯಂತಹ
ಬೆಂಕಿ-ಯನ್ನು
ಬೆಂಕಿಯ
ಬೆಂಚಿ-ನ-ಲ್ಲಿ
ಬೆಂಜಮಿ-ನ್
ಬೆಂಡಾಗಿ
ಬೆಂದ
ಬೆಂದ-ವರು
ಬೆಕ್ಕನ್ನು
ಬೆಕ್ಕಿನ
ಬೆಕ್ಕು
ಬೆಕ್ಕು-ಗಳಂತೆ
ಬೆಚ್ಚಗಿ-ರುವ
ಬೆಚ್ಚು
ಬೆಣ್ಣೆ
ಬೆನಾರ-ಸ್
ಬೆರ-ಗಾ-ಯಿತು
ಬೆರ-ಗಾಗಿ
ಬೆರ-ಳು-ಗಳು
ಬೆರಗಾ-ದರು
ಬೆರಗಾಗು-ವಂತಹ
ಬೆರಗಾಗು-ವರು
ಬೆರಳಿ-ನ-ಲ್ಲಿ
ಬೆರಸಿ
ಬೆರಾ-ವು-ದಕ್ಕೂ
ಬೆರೆ-ತರೆ
ಬೆರೆ-ಯ-ತೊಡಗಿದ
ಬೆರೆ-ಯು-ತ್ತಿದ್ದ
ಬೆರೆತ-ಮೇಲೆ
ಬೆರೆತಿ-ದ್ದುವು
ಬೆರೆತಿ-ಲ್ಲ-ವೆಂದು
ಬೆರೆತು
ಬೆಲೂರು
ಬೆಲೆ
ಬೆಲೆ-ಕೊ-ಟ್ಟು
ಬೆಲೆ-ಬಾಳುವ
ಬೆಲೆ-ಯನ್ನು
ಬೆಲೆ-ಯಿ-ಲ್ಲ
ಬೆಳ-ಕನ್ನು
ಬೆಳ-ಗಾ-ಯಿತು
ಬೆಳ-ಗು-ತ್ತಿದೆ
ಬೆಳ-ಸಿದ
ಬೆಳಕಾಗಿ-ರುವರು
ಬೆಳಕಿ-ನ-ಲ್ಲಿ
ಬೆಳಕಿ-ನಂತೆ
ಬೆಳಕಿ-ಲ್ಲ
ಬೆಳಕಿಗೆ
ಬೆಳಕಿನ
ಬೆಳಕಿನ-ಲ್ಲಿಯೂ
ಬೆಳಕು
ಬೆಳಕು-ಗ-ಳನ್ನು
ಬೆಳಗಲು
ಬೆಳಗಾಂ-ನ-ಲ್ಲಿ-ದ್ದಾಗ
ಬೆಳಗಾಂ-ನ-ಲ್ಲಿ-ರುವ
ಬೆಳಗಾಂ-ವಿಗೆ
ಬೆಳಗಾಂವಿ-ನ-ಲ್ಲಿ
ಬೆಳಗಾಂವಿ-ನ-ಲ್ಲಿ-ರುವ
ಬೆಳಗಾಗುವ
ಬೆಳಗಿ-ದರು
ಬೆಳಗಿ-ನಿಂದ
ಬೆಳಗಿತು
ಬೆಳಗಿನ
ಬೆಳಗು-ತ್ತ
ಬೆಳಗು-ತ್ತಿ-ತ್ತು
ಬೆಳಗು-ತ್ತಿ-ದ್ದ-ವರು
ಬೆಳಗು-ತ್ತಿರು-ವನು
ಬೆಳಗು-ವಂತೆ
ಬೆಳಗುವ
ಬೆಳಗ್ಗೆ
ಬೆಳದಿಂ-ಗಳ
ಬೆಳವಣಿ-ಗೆಗೆ
ಬೆಳವಣಿಗೆ
ಬೆಳಸ-ಬೇ-ಕಾ-ದರೆ
ಬೆಳಸಿ
ಬೆಳಸಿ-ಕೊ-ಳ್ಳು-ವು-ದಕ್ಕೆ
ಬೆಳಸಿ-ಕೊಂಡ
ಬೆಳಸಿ-ಕೊಳ್ಳ-ಬೇಕು
ಬೆಳಸಿ-ದರು
ಬೆಳಿಗ್ಗೆ
ಬೆಳುದಿಂ-ಗ-ಳಿಗೆ
ಬೆಳೆ-ದಂತೆ
ಬೆಳೆ-ದರು
ಬೆಳೆ-ದವು
ಬೆಳೆ-ದಿದ್ದ
ಬೆಳೆ-ಯ-ತೊಡಗಿತು
ಬೆಳೆ-ಯದ
ಬೆಳೆ-ಯಿತು
ಬೆಳೆ-ಯು-ತ್ತ
ಬೆಳೆ-ಯು-ತ್ತಾ
ಬೆಳೆ-ಯು-ತ್ತಾರೆ
ಬೆಳೆ-ಯು-ತ್ತಿ-ತ್ತು
ಬೆಳೆ-ಯು-ತ್ತಿರುವ
ಬೆಳೆ-ಯು-ವಂತೆ
ಬೆಳೆ-ಯು-ವು-ದಿ-ಲ್ಲ
ಬೆಳೆ-ಯು-ವುದು
ಬೆಳೆ-ಯುವ
ಬೆಳೆ-ಯುವನು
ಬೆಳೆ-ಸು-ತ್ತಿ-ದ್ದರು
ಬೆಳೆ-ಸು-ತ್ತಿ-ದ್ದಳು
ಬೆಳೆ-ಸುವ
ಬೆಳೆದ
ಬೆಳೆದ-ವನು
ಬೆಳೆದು
ಬೆಳೆದು-ಕೊಂಡು
ಬೆಳೆಯ-ಬೇ-ಕಾ-ದರೆ
ಬೆಳೆಯು-ತ್ತಿ-ರು-ವು-ದ-ನ್ನು
ಬೆಳೆಸಿ-ಕೊ-ಳ್ಳುವುದಕ್ಕಿಂತ
ಬೆಳೆಸು-ವುದ-ಕ್ಕೋ-ಸ್ಕರ
ಬೆಳ್ಳಿ
ಬೆಳ್ಳಿಯ
ಬೆಳ್ಳುಳ್ಳಿ
ಬೇ
ಬೇಕಾಗ-ಬಹು-ದೆಂದು
ಬೇಕಾಗ-ಬಹುದು
ಬೇಕಾಗಿ-ತ್ತು
ಬೇಕಾಗಿ-ದ್ದ-ವ-ರಾಗಿ-ದ್ದರು
ಬೇಕಾಗಿ-ದ್ದಾರೆ
ಬೇಕಾಗಿ-ದ್ದು-ದ-ರಿಂದ
ಬೇಕಾಗಿ-ದ್ದುದು
ಬೇಕಾಗಿ-ರ-ಲಿ-ಲ್ಲ
ಬೇಕಾಗಿ-ರುವ
ಬೇಕಾಗಿ-ರುವರು
ಬೇಕಾಗಿ-ರುವುದು
ಬೇಕಾಗಿ-ಲ್ಲ
ಬೇಕಾಗಿದೆ
ಬೇಕಾಗುವ
ಬೇಕಾಗುವುದು
ಬೇಕಿ-ಲ್ಲ
ಬೇಕು
ಬೇಕೆ
ಬೇಕೆಂತಲೇ
ಬೇಕೆಂದು
ಬೇಕೆಂದೂ
ಬೇಕೊ
ಬೇಕೋ
ಬೇಗ
ಬೇಗಂ
ಬೇಗನೆ
ಬೇಗೆ
ಬೇಜಾ-ರಾ-ದರೆ
ಬೇಜಾ-ರಾಗಿ
ಬೇಜಾರಾ-ಯಿತು
ಬೇಜಾರಾಗು-ತ್ತಿ-ತ್ತು
ಬೇಜಾರು
ಬೇಡ
ಬೇಡ-ದಂತೆ
ಬೇಡ-ಬಹುದು
ಬೇಡ-ಬೇಕು
ಬೇಡ-ವೆಂ-ದರೂ
ಬೇಡ-ವೆಂದು
ಬೇಡ-ವೆಂದೂ
ಬೇಡದೆ
ಬೇಡಲು
ಬೇಡವೆ
ಬೇಡವೊ
ಬೇಡಿ
ಬೇಡಿ-ಕೆ-ಯನ್ನು
ಬೇಡಿ-ಕೊ-ಳ್ಳು-ತ್ತಿ-ತ್ತು
ಬೇಡಿ-ಕೊ-ಳ್ಳು-ತ್ತಿದ್ದರು
ಬೇಡಿ-ಕೊಂಡ
ಬೇಡಿ-ಕೊಂಡ-ಮೇಲೆ
ಬೇಡಿ-ಕೊಂಡನು
ಬೇಡಿ-ಕೊಂಡರು
ಬೇಡಿ-ಕೊಂಡಿತು
ಬೇಡಿ-ಕೊಂಡಿದ್ದ
ಬೇಡಿ-ಕೊಂಡಿದ್ದರು
ಬೇಡಿ-ಕೊಂಡು
ಬೇಡಿ-ಕೊಂಡೆ
ಬೇಡಿ-ಕೊಳ್ಳ-ಬೇಕೆಂದು
ಬೇಡಿ-ದಂತೆ
ಬೇಡಿ-ದನು
ಬೇಡಿ-ದರು
ಬೇಡಿ-ದರೆ
ಬೇಡಿ-ದೆನು
ಬೇಡಿ-ರು-ವೆನು
ಬೇಡಿದೆ
ಬೇಡು
ಬೇಡು-ತ್ತಲೇ
ಬೇಡು-ತ್ತಿದ್ದರು
ಬೇಡು-ವನು
ಬೇಡು-ವು-ದಕ್ಕೆ
ಬೇಡು-ವುದ-ಕ್ಕೆಂದು
ಬೇಡು-ವೆನು
ಬೇಡುವ
ಬೇಣೀ-ಪಾ-ಲ್
ಬೇಯಿ-ಸಿದ
ಬೇಯಿ-ಸು-ವು-ದಕ್ಕೆ
ಬೇಯಿಸಿ
ಬೇರಾ-ವುದೂ
ಬೇರಾ-ವುದೋ
ಬೇರಾವ
ಬೇರಾವು-ದ-ನ್ನೂ
ಬೇರು
ಬೇರು-ತ್ತ
ಬೇರು-ಬಿ-ಟ್ಟಿದೆ
ಬೇರು-ಬಿಡಲು
ಬೇರೂ-ರು-ವು-ದಕ್ಕೆ
ಬೇರೂ-ರು-ವು-ದಿ-ಲ್ಲವೋ
ಬೇರೂ-ರುವ
ಬೇರೂರಲು
ಬೇರೂರಿ
ಬೇರೂರಿದೆ
ಬೇರೂರು-ವುದೊ
ಬೇರೆ
ಬೇರೆ-ಯ-ದ-ನ್ನು
ಬೇರೆ-ಯಾಗಲು
ಬೇರೆ-ಯಾಗಿ
ಬೇರೆ-ಯಾಗಿ-ತ್ತು
ಬೇರೆ-ಯಾಗಿ-ರ-ಬಹುದು
ಬೇರೆ-ಯಾಗಿ-ರು-ವುದು
ಬೇರೆ-ಯಾಗಿಯೇ
ಬೇರೆ-ಲ್ಲಿಯೂ
ಬೇರೇ-ನನ್ನೂ
ಬೇರೇನೂ
ಬೇರೇನೋ
ಬೇರೊಂದು
ಬೇರೊಬ್ಬ
ಬೇರ್ಪಡಿ-ಸುವ
ಬೇಲಿ-ಯನ್ನು
ಬೇಲಿ-ಯಿಂದ
ಬೇಲಿ-ಯೊಳಗೆ
ಬೇಲೂರ
ಬೇಲೂರ-ಮಠ-ದ-ಲ್ಲಿ-ದ್ದಾಗ
ಬೇಲೂರು
ಬೇಳೆ
ಬೇಸ-ಗೆಯ
ಬೇಸಗೆ
ಬೇಸರ
ಬೇಸರ-ಗೊಂಡಿ-ದ್ದನು
ಬೇಸರ-ವಾಗಿ-ತ್ತು
ಬೈ
ಬೈಗುಳ
ಬೈಟಕ್
ಬೈಠಕ್
ಬೈಠಕ್ಖಾನೆ-ಯ-ಲ್ಲಿ
ಬೈದು
ಬೈಬ-ಲ್
ಬೈಬ-ಲ್ಲ-ನ್ನು
ಬೈಬ-ಲ್ಲಿ-ನ-ಲ್ಲಿ
ಬೈಬ-ಲ್ಲಿ-ನಿಂದ
ಬೈಬ-ಲ್ಲಿಗೆ
ಬೈಲು
ಬೊ
ಬೊಂ
ಬೊಂಬಿನ
ಬೊಗಸೆ
ಬೊಟಾನಿ-ಕ-ಲ್
ಬೊಧ-ನೆಗೆ
ಬೊಬ್ಬೆ
ಬೋ
ಬೋಂ
ಬೋಧ-ಕರು
ಬೋಧ-ಗ-ಯೆಗೆ
ಬೋಧ-ಪ್ರದ-ವಾ-ದು-ದ-ನ್ನು
ಬೋಧ-ಸ್ವ-ರೂಪ-ನಾಗು-ವಿಕೆ
ಬೋಧ-ಸ್ವ-ರೂಪ-ವನ್ನು
ಬೋಧಕ
ಬೋಧಕ-ರ-ನ್ನು
ಬೋಧಕ-ರಾಗಲು
ಬೋಧಿ-ಸ-ಕೂ-ಡದು
ಬೋಧಿ-ಸ-ತ್ವ
ಬೋಧಿ-ಸ-ಬೇ-ಕಾ-ದರೆ
ಬೋಧಿ-ಸ-ಬೇಕಾಗಿದೆ
ಬೋಧಿ-ಸ-ಬೇಕು
ಬೋಧಿ-ಸ-ಬೇಕೆಂದೂ
ಬೋಧಿ-ಸ-ಬೇಕೆಂಬುದು
ಬೋಧಿ-ಸ-ಲ್ಪ-ಡು-ವುವು
ಬೋಧಿ-ಸಲು
ಬೋಧಿ-ಸಿ-ದನು
ಬೋಧಿ-ಸಿ-ದರು
ಬೋಧಿ-ಸಿ-ದರೆ
ಬೋಧಿ-ಸಿ-ದಾಗ
ಬೋಧಿ-ಸಿ-ದು-ದರ
ಬೋಧಿ-ಸಿ-ದ್ದರು
ಬೋಧಿ-ಸಿ-ರು-ವಿರಿ
ಬೋಧಿ-ಸಿ-ರುವರು
ಬೋಧಿ-ಸಿದ
ಬೋಧಿ-ಸು-ತ್ತದೆ
ಬೋಧಿ-ಸು-ತ್ತಾ
ಬೋಧಿ-ಸು-ತ್ತಾರೆ
ಬೋಧಿ-ಸು-ತ್ತಿ-ದೆಯೊ
ಬೋಧಿ-ಸು-ತ್ತಿ-ದ್ದರು
ಬೋಧಿ-ಸು-ತ್ತಿ-ದ್ದಾರೆ
ಬೋಧಿ-ಸು-ತ್ತಿ-ರ-ಲಿ-ಲ್ಲ
ಬೋಧಿ-ಸು-ತ್ತಿ-ರುವರು
ಬೋಧಿ-ಸು-ತ್ತಿ-ರುವಾಗ
ಬೋಧಿ-ಸು-ತ್ತೇನೆ
ಬೋಧಿ-ಸು-ವಂತೆ
ಬೋಧಿ-ಸು-ವು-ದ-ಕ್ಕಾಗಿ
ಬೋಧಿ-ಸು-ವು-ದ-ನ್ನು
ಬೋಧಿ-ಸು-ವು-ದ-ರಿಂದ
ಬೋಧಿ-ಸು-ವು-ದಕ್ಕೆ
ಬೋಧಿ-ಸು-ವು-ದಿ-ಲ್ಲ
ಬೋಧಿ-ಸು-ವುದು
ಬೋಧಿ-ಸು-ವುದೂ
ಬೋಧಿ-ಸುವ
ಬೋಧಿ-ಸುವನು
ಬೋಧಿ-ಸುವರು
ಬೋಧಿ-ಸುವರೋ
ಬೋಧಿ-ಸುವೆ
ಬೋಧಿಸಿ
ಬೋಧೆ-ಗ-ಳನ್ನು
ಬೋಧೆ-ಯಾಗು-ತ್ತವೆ
ಬೋಧೆ-ಯಾದ
ಬೋಧೆಯೂ
ಬೋರ್ಡಿಂಗ್
ಬೋರ್ನಿಯೋ
ಬೋಳು
ಬೋಸರ
ಬೋಸರು
ಬೌತ-ಶಾ-ಸ್ತ್ರ
ಬೌದ್ಧ
ಬೌದ್ಧ-ಕಾಲ
ಬೌದ್ಧ-ಧರ್ಮ
ಬೌದ್ಧ-ಧರ್ಮ-ದಂತೆ
ಬೌದ್ಧ-ಧರ್ಮ-ವನ್ನು
ಬೌದ್ಧ-ಧರ್ಮ-ವೆಂಬ
ಬೌದ್ಧ-ಧರ್ಮಕ್ಕೆ
ಬೌದ್ಧ-ಧರ್ಮದ
ಬೌದ್ಧ-ಭಿಕ್ಷು-ಗಳು
ಬೌದ್ಧ-ರಿಗೆ
ಬೌದ್ಧರ
ಬೌದ್ಧರು
ಬೌದ್ಧಿಕ
ಬೌದ್ಧಿಕ-ವಾ-ದುದು
ಬೌದ್ಧಿಕ-ವಾಗಿ
ಬ್ಯಕ್ತಿ-ಗಳು
ಬ್ಯಾಂ
ಬ್ಯಾಂಕಿ-ನ-ಲ್ಲಿ-ಡು-ತ್ತಿದ್ದರು
ಬ್ಯಾಂಕ್
ಬ್ಯಾಂಡು-ವಾದನ
ಬ್ಯಾಂಡೇ-ಜನ್ನು
ಬ್ಯಾಂಡೇಜ್
ಬ್ಯಾಂಡ್
ಬ್ಯಾಗ್ಲಿ
ಬ್ಯಾನರ್ಜಿ
ಬ್ಯಾನರ್ಜಿ-ಯಂತಹ
ಬ್ಯೂರೊ-ವನ್ನು
ಬ್ಯೂರೋ
ಬ್ರ
ಬ್ರಜೇ-ನ್ಬಾಬು
ಬ್ರಜೇ-ನ್ಬಾಬು-ವಿನ
ಬ್ರಷ್ಟರಾಗು-ತ್ತಾರೆ
ಬ್ರಹ್ಮ
ಬ್ರಹ್ಮ-ಋಷಿ-ಗಳು
ಬ್ರಹ್ಮ-ಚರ್ಯ
ಬ್ರಹ್ಮ-ಚರ್ಯ-ವನ್ನು
ಬ್ರಹ್ಮ-ಚರ್ಯ-ವಿ-ಲ್ಲದೇ
ಬ್ರಹ್ಮ-ಚರ್ಯದ
ಬ್ರಹ್ಮ-ಚರ್ಯಾ-ಭ್ಯಾಸ
ಬ್ರಹ್ಮ-ಚರ್ಯಾ-ಭ್ಯಾಸ-ದಿಂದ
ಬ್ರಹ್ಮ-ಚಾ-ರಿಗೆ
ಬ್ರಹ್ಮ-ಚಾರಿ
ಬ್ರಹ್ಮ-ಚಾರಿ-ಗ-ಳನ್ನು
ಬ್ರಹ್ಮ-ಚಾರಿ-ಗ-ಳಿಗೆ
ಬ್ರಹ್ಮ-ಚಾರಿ-ಗಳ
ಬ್ರಹ್ಮ-ಚಾರಿ-ಗಳ-ನ್ನೂ
ಬ್ರಹ್ಮ-ಚಾರಿ-ಗಳ-ನ್ನೆ-ಲ್ಲಾ
ಬ್ರಹ್ಮ-ಚಾರಿ-ಗಳ-ಲ್ಲಿ
ಬ್ರಹ್ಮ-ಚಾರಿ-ಗಳಾಗಿ-ರ-ಬೇಕು
ಬ್ರಹ್ಮ-ಚಾರಿ-ಗಳು
ಬ್ರಹ್ಮ-ಚಾರಿ-ಗಳೆ-ಲ್ಲಾ
ಬ್ರಹ್ಮ-ಚಾರಿ-ಗಳೇ
ಬ್ರಹ್ಮ-ಚಾರಿ-ಗಳೊ-ಡನೆ
ಬ್ರಹ್ಮ-ಚಾರಿ-ಣಿಯ
ಬ್ರಹ್ಮ-ಚಾರಿ-ಯಾಗಿ-ರು-ವಂತೆ
ಬ್ರಹ್ಮ-ಚಿಂ-ತನೆ-ಯ-ಲ್ಲಿಯೇ
ಬ್ರಹ್ಮ-ಜ್ಞಾನ
ಬ್ರಹ್ಮ-ಜ್ಞಾನ-ಕ್ಕಾ-ದರೂ
ಬ್ರಹ್ಮ-ಜ್ಞಾನ-ವನ್ನು
ಬ್ರಹ್ಮ-ಜ್ಞಾನ-ವನ್ನೂ
ಬ್ರಹ್ಮ-ಜ್ಞಾನದ
ಬ್ರಹ್ಮ-ಜ್ಞಾನಿ
ಬ್ರಹ್ಮ-ಜ್ಞಾನಿ-ಗಳಾಗಿ
ಬ್ರಹ್ಮ-ದರ್ಶನ
ಬ್ರಹ್ಮ-ದರ್ಶನ-ವನ್ನು
ಬ್ರಹ್ಮ-ನ-ಲ್ಲಿ
ಬ್ರಹ್ಮ-ನನ್ನು
ಬ್ರಹ್ಮ-ನಾಗಿ-ರು-ವನೊ
ಬ್ರಹ್ಮ-ನಿಂದ
ಬ್ರಹ್ಮ-ನಿಗೇಕೆ
ಬ್ರಹ್ಮ-ನಿದ್ದಾನೆ
ಬ್ರಹ್ಮ-ನೆ-ಡೆಗೆ
ಬ್ರಹ್ಮ-ನೊ-ಡನೆ
ಬ್ರಹ್ಮ-ನೊಂದಿಗೆ
ಬ್ರಹ್ಮ-ಪು-ತ್ರ
ಬ್ರಹ್ಮ-ಪುತಾ-ನದಿಯ
ಬ್ರಹ್ಮ-ಭಾವದ
ಬ್ರಹ್ಮ-ಮಯ
ಬ್ರಹ್ಮ-ರಾಕ್ಷಸ
ಬ್ರಹ್ಮ-ರಾಕ್ಷಸನೂ
ಬ್ರಹ್ಮ-ವ-ಲ್ಲದೇ
ಬ್ರಹ್ಮ-ವಾಗಿಯೇ
ಬ್ರಹ್ಮ-ವಾದಿನಿ
ಬ್ರಹ್ಮ-ವಿದ್ಯೆ-ಯನ್ನು
ಬ್ರಹ್ಮ-ಸಮಾಜ
ಬ್ರಹ್ಮ-ಸಮಾಜ-ದ-ಲ್ಲಿ
ಬ್ರಹ್ಮ-ಸಮಾಜ-ದ-ಲ್ಲಿ-ರ-ಬಹು-ದೆಂದು
ಬ್ರಹ್ಮ-ಸಮಾಜ-ದ-ವರು
ಬ್ರಹ್ಮ-ಸಮಾಜಕ್ಕೆ
ಬ್ರಹ್ಮ-ಸಮಾಜದ
ಬ್ರಹ್ಮ-ಸಾಕ್ಷಾ-ತ್ಕಾರ-ವಾಗು-ವ-ವ-ರೆಗೂ
ಬ್ರಹ್ಮ-ಸಾಕ್ಷಾ-ತ್ಕಾರ-ವಾದ
ಬ್ರಹ್ಮ-ಸೂ-ತ್ರ
ಬ್ರಹ್ಮ-ಸ್ವ-ರೂಪ-ವೆಂದು
ಬ್ರಹ್ಮಣಃ
ಬ್ರಹ್ಮದ
ಬ್ರಹ್ಮನ
ಬ್ರಹ್ಮನು
ಬ್ರಹ್ಮನೆ
ಬ್ರಹ್ಮನೇ
ಬ್ರಹ್ಮನೋ
ಬ್ರಹ್ಮರೇ
ಬ್ರಹ್ಮಾ-ನಂ-ದರು
ಬ್ರಹ್ಮಾ-ನಂ-ದರೂ
ಬ್ರಹ್ಮಾ-ನಂದ
ಬ್ರಹ್ಮಾ-ನಂದ-ರಿಗೂ
ಬ್ರಹ್ಮಾ-ನಂದ-ರಿಗೆ
ಬ್ರಹ್ಮಾ-ನಂದರ
ಬ್ರಹ್ಮಾ-ನುಭವ
ಬ್ರಹ್ಮಾ-ವಾದಿ-ನಿಗೆ
ಬ್ರಹ್ಮಾ-ಸ್ರ-ವನ್ನು
ಬ್ರಹ್ಮಾಂಡ
ಬ್ರಹ್ಮಾಂಡ-ಗಳು
ಬ್ರಹ್ಮಾಂಡ-ದ-ಲ್ಲಿ-ರು-ವು-ದೆ-ಲ್ಲ
ಬ್ರಹ್ಮೀ-ಭೂತ-ನಾಗಿ-ರು-ವುದು
ಬ್ರಾಂಡಿ
ಬ್ರಾಹ್ಮಣ
ಬ್ರಾಹ್ಮಣ-ಧರ್ಮ
ಬ್ರಾಹ್ಮಣ-ಧರ್ಮದ
ಬ್ರಾಹ್ಮಣ-ನ-ಲ್ಲ
ಬ್ರಾಹ್ಮಣ-ನಂತೆ
ಬ್ರಾಹ್ಮಣ-ನನ್ನೂ
ಬ್ರಾಹ್ಮಣ-ನಾಗಿ-ರ-ಬೇಕು
ಬ್ರಾಹ್ಮಣ-ನಾದ
ಬ್ರಾಹ್ಮಣ-ನಿಗೂ
ಬ್ರಾಹ್ಮಣ-ರ-ನ್ನಾಗಿ
ಬ್ರಾಹ್ಮಣ-ರ-ಲ್ಲ-ವೆಂದೂ
ಬ್ರಾಹ್ಮಣ-ರ-ಲ್ಲಿ
ಬ್ರಾಹ್ಮಣ-ರ-ಲ್ಲಿಯೂ
ಬ್ರಾಹ್ಮಣ-ರಾಗುವ-ವ-ರೆಗೆ
ಬ್ರಾಹ್ಮಣ-ರಾದ
ಬ್ರಾಹ್ಮಣ-ರಿಂದ
ಬ್ರಾಹ್ಮಣ-ರೊ-ಡನೆ
ಬ್ರಾಹ್ಮಣನ
ಬ್ರಾಹ್ಮಣನು
ಬ್ರಾಹ್ಮಣನೇ
ಬ್ರಾಹ್ಮಣರ
ಬ್ರಾಹ್ಮಣರು
ಬ್ರಾಹ್ಮಣೀ
ಬ್ರಾಹ್ಮಣೇ-ತರ
ಬ್ರಾಹ್ಮೋ
ಬ್ರಿಟಾ-ನಿಗೆ
ಬ್ರಿಟಾನಿ-ಯಿಂದ
ಬ್ರಿಟಾನಿಕ
ಬ್ರಿಟಿಷ್
ಬ್ರಿಟೀಷ್
ಬ್ರುಕ್ಲಿನ್
ಬ್ರೂಕ್ಲಿನ್
ಬ್ರೆಡ್
ಬ್ಲಾಡ್ಗೆ-ಟ್
ಭ
ಭಂಗ
ಭಂಗ-ವಾಗಿ-ಲ್ಲ
ಭಂಗ-ವಾಗು-ವುದೆಂದು
ಭಂಗ-ವಿ-ಲ್ಲ
ಭಂಗಿ
ಭಂಗಿ-ಯ-ವ-ನಿಗೆ
ಭಂಡಾರ-ವನ್ನು
ಭಂಡಾರ-ವಾಗಿ-ರುವನು
ಭಕ್ತ
ಭಕ್ತ-ಗಣ-ವನ್ನು
ಭಕ್ತ-ನನ್ನು
ಭಕ್ತ-ನಾ-ದನು
ಭಕ್ತ-ನಿಗೆ
ಭಕ್ತ-ನೊಬ್ಬ
ಭಕ್ತ-ಮಂಡಲಿ-ಯನ್ನು
ಭಕ್ತ-ರ-ನ್ನು
ಭಕ್ತ-ರ-ನ್ನೊಳ-ಗೊಂಡ
ಭಕ್ತ-ರ-ಲ್ಲೆ-ಲ್ಲ
ಭಕ್ತ-ರಾ-ದರು
ಭಕ್ತ-ರಾದ
ಭಕ್ತ-ರಿ-ಗಾಗಿ
ಭಕ್ತ-ರಿ-ಗೆ-ಲ್ಲ
ಭಕ್ತ-ರಿಂದ
ಭಕ್ತ-ರಿಗೂ
ಭಕ್ತ-ರಿಗೆ
ಭಕ್ತ-ರೆ-ದು-ರಿಗೆ
ಭಕ್ತ-ರೆ-ಲ್ಲ-ರೂ-ಭಕ್ತರೇ
ಭಕ್ತ-ರೆ-ಲ್ಲರೂ
ಭಕ್ತ-ರೊ-ಡನೆ
ಭಕ್ತ-ರೊಬ್ಬರ
ಭಕ್ತ-ರೊಬ್ಬರಿಗೆ
ಭಕ್ತ-ರೊಬ್ಬರು
ಭಕ್ತ-ವೃಂದ
ಭಕ್ತ-ವೃಂದ-ದೊ-ಡನೆ
ಭಕ್ತನ
ಭಕ್ತನು
ಭಕ್ತರ
ಭಕ್ತರು
ಭಕ್ತರೂ
ಭಕ್ತಾ-ದಿ-ಗಳು
ಭಕ್ತಾದಿ-ಗ-ಳಿಗೆ
ಭಕ್ತಾದಿ-ಗಳ
ಭಕ್ತಾದಿ-ಗಳೆ-ಲ್ಲ
ಭಕ್ತಾದಿ-ಗಳೊ-ಡನೆ
ಭಕ್ತಿ
ಭಕ್ತಿ-ಗ-ಳಿದ್ದರೂ
ಭಕ್ತಿ-ಗಳ
ಭಕ್ತಿ-ಗಳಿಂದ
ಭಕ್ತಿ-ಗೀತೆ-ಗ-ಳನ್ನು
ಭಕ್ತಿ-ಗೀತೆ-ಯನ್ನು
ಭಕ್ತಿ-ಭಾವ
ಭಕ್ತಿ-ಭಾವ-ಗಳಿಂದ
ಭಕ್ತಿ-ಭಾವ-ದಿಂದ
ಭಕ್ತಿ-ಮಾರ್ಗವೂ
ಭಕ್ತಿ-ಯ-ಲ್ಲ
ಭಕ್ತಿ-ಯ-ಲ್ಲಿ-ಲ್ಲ
ಭಕ್ತಿ-ಯನ್ನು
ಭಕ್ತಿ-ಯನ್ನೂ
ಭಕ್ತಿ-ಯಿ-ದ್ದರೆ
ಭಕ್ತಿ-ಯಿ-ಲ್ಲ-ದಿ-ದ್ದ-ಲ್ಲಿ
ಭಕ್ತಿ-ಯಿಂದ
ಭಕ್ತಿ-ಯುಳ್ಳ-ವ-ನಾಗಿ
ಭಕ್ತಿ-ಯುಳ್ಳ-ವ-ನಾಗು-ವನು
ಭಕ್ತಿ-ಯೆಂ-ದರೆ
ಭಕ್ತಿ-ಯೋಗ
ಭಕ್ತಿ-ಯೋಗ-ವನ್ನು
ಭಕ್ತಿ-ಯೋಗದ
ಭಕ್ತಿ-ರೂಪ-ದ-ಲ್ಲಿ
ಭಕ್ತಿ-ಶಾ-ಸ್ತ್ರ
ಭಕ್ತಿ-ಸ್ವ-ಭಾವ-ದ-ವನು
ಭಕ್ತಿ-ಹೀನ-ನಾದ
ಭಕ್ತಿಯ
ಭಕ್ತಿಯೇ
ಭಕ್ತೆಯಾ-ದಳು
ಭಕ್ಷ-ಕನಿರ-ಬೇಕು
ಭಕ್ಷ್ಯ-ಗ-ಳನ್ನು
ಭಗ-ವಂತ
ಭಗ-ವಂತ-ನನ್ನು
ಭಗ-ವಂತ-ನನ್ನೇ
ಭಗ-ವಂತ-ನಿಂದ
ಭಗ-ವಂತ-ನಿಗೆ
ಭಗ-ವಂತ-ನೆ-ಡೆಗೆ
ಭಗ-ವಂತ-ನೊಂದಿಗೆ
ಭಗ-ವಂತ-ನೊಬ್ಬನೇ
ಭಗ-ವಂತನ
ಭಗ-ವಂತನು
ಭಗ-ವಂತನೇ
ಭಗ-ವತಿ
ಭಗ-ವತಿಯೂ
ಭಗ-ವತ್
ಭಗ-ವತ್ಕೃಪೆ-ಯಿಂದ
ಭಗ-ವತ್ಸಾಕ್ಷಾ-ತ್ಕಾ-ರಕ್ಕೆ
ಭಗ-ವತ್ಸಾಕ್ಷಾ-ತ್ಕಾರ
ಭಗ-ವನ್ಮಯ-ವಾಗಿ
ಭಗವ-ದರ್ಪ-ಣೆಗೆ
ಭಗವ-ದಾನಂದ-ವನ್ನು
ಭಗವದಿಚ್ಛೆ
ಭಗವದಿಚ್ಛೆ-ಯಿಂದ
ಭಗವದ್ಗೀತೆ-ಯನ್ನು
ಭಗವಾ-ನ್
ಭಗ್ನ-ವಾಗಿ
ಭಗ್ನ-ವಾಗಿ-ರು-ವು-ದ-ನ್ನು
ಭಗ್ನ-ವಾಗು-ವುದೋ
ಭಗ್ನ-ವಾದ
ಭಜದ-ಮೇಲೆ
ಭಜಾ-ಮ್ಯಹಂ
ಭದ್ರ
ಭದ್ರ-ಮಹಿಳೆ
ಭದ್ರ-ವಾಗಿ
ಭದ್ರಾ-ಕಾರವೂ
ಭಯ
ಭಯ-ಕ್ಕಾಗಿ
ಭಯ-ಚಕಿತ
ಭಯ-ದಿಂದ
ಭಯ-ಪಡ-ಬೇಡ
ಭಯ-ಪಡ-ಬೇಡಿ
ಭಯ-ಭಕ್ತಿ-ಯನ್ನು
ಭಯ-ಭಕ್ತಿ-ಯಿಂದ
ಭಯ-ವನ್ನೇ
ಭಯ-ವೇನು
ಭಯಂ
ಭಯಂ-ಕರ
ಭಯಂ-ಕರ-ವಾಗಿ
ಭಯಂ-ಕರ-ವಾದ
ಭಯವೇ
ಭರ-ಖಂಡಕ್ಕೆ
ಭರತ
ಭರತ-ಖಂಡ
ಭರತ-ಖಂಡ-ದ-ಲ್ಲಿ
ಭರತ-ಖಂಡ-ದ-ಲ್ಲಿ-ರುವ
ಭರತ-ಖಂಡ-ದ-ಲ್ಲಿದ್ದ
ಭರತ-ಖಂಡ-ದ-ಲ್ಲೆ-ಲ್ಲ
ಭರತ-ಖಂಡ-ದ-ಲ್ಲೇ
ಭರತ-ಖಂಡ-ದ-ವ-ರೆ-ಲ್ಲರೂ
ಭರತ-ಖಂಡ-ದಿಂದ
ಭರತ-ಖಂಡ-ವನ್ನು
ಭರತ-ಖಂಡ-ವನ್ನೆ-ಲ್ಲ
ಭರತ-ಖಂಡ-ವನ್ನೇ
ಭರತ-ಖಂಡ-ವಾ-ದರೊ
ಭರತ-ಖಂಡ-ವಿ-ರ-ಲಿ-ಲ್ಲ
ಭರತ-ಖಂಡಕ್ಕೆ
ಭರತ-ಖಂಡದ
ಭರತ-ಖಂಡವೂ
ಭರತ-ಖಂಡವೆ
ಭರತ-ಖಂಡವೇ
ಭರತ-ಚಂದ್ರ
ಭರತ-ಚಂದ್ರ-ರ-ನ್ನು
ಭರತ-ಚಂದ್ರರ
ಭರತ-ನಂತೆ
ಭರತ-ಮಾತೆಯು
ಭರ್ಜಿ-ಯಂತೆ
ಭರ್ತಿ-ಯಾಗಿ-ತ್ತು
ಭರ್ತ್ಸಿ-ಸಿ-ದರು
ಭವ
ಭವ-ಜೀವಿ-ಗ-ಳಿಗೆ
ಭವ-ಜೀವಿ-ಗಳ
ಭವ-ಜೀವಿ-ಗಳಿ-ಗೆ-ಲ್ಲ
ಭವ-ಜೀವಿ-ಗಳು
ಭವ-ನಾಥ-ನೆಂಬ
ಭವ-ಸಾ-ಗರ-ದ-ತ್ತ
ಭವ-ಸಾ-ಗರ-ದ-ಲ್ಲಿ
ಭವ-ಸಾ-ಗರದ
ಭವಿತವ್ಯದ
ಭವಿಷ್ಯ
ಭವಿಷ್ಯ-ಕಾಲ-ವನ್ನು
ಭವಿಷ್ಯ-ತ್ತು
ಭವಿಷ್ಯ-ದ-ಲ್ಲಿ
ಭವಿಷ್ಯ-ವನ್ನು
ಭವಿಷ್ಯ-ವಿದೆ
ಭವಿಷ್ಯ-ವೆ-ಲ್ಲ
ಭವಿಷ್ಯದ
ಭವಿಷ್ಯವೂ
ಭವ್ಯ
ಭವ್ಯ-ಜೀವನ
ಭವ್ಯ-ತೆ-ಗಳೆ-ರಡೂ
ಭವ್ಯ-ತೆಗೆ
ಭವ್ಯ-ವಾಗಿ
ಭವ್ಯ-ವಾದ
ಭವ್ಯವೂ
ಭಾ
ಭಾಗ
ಭಾಗ-ಗ-ಳನ್ನು
ಭಾಗ-ಗಳ-ನ್ನೆ-ಲ್ಲ
ಭಾಗ-ಗಳ-ಲ್ಲಿಯೂ
ಭಾಗ-ಗಳು
ಭಾಗ-ಗಳೂ
ಭಾಗ-ದ-ಲ್ಲಿ
ಭಾಗ-ದ-ಲ್ಲಿಯೂ
ಭಾಗ-ಲ್ಪು-ರಕ್ಕೆ
ಭಾಗ-ಲ್ಪು-ರದ
ಭಾಗ-ಲ್ಪುರ-ವನ್ನು
ಭಾಗ-ವತ
ಭಾಗ-ವತ-ದ-ಲ್ಲಿ
ಭಾಗ-ವತ-ವನ್ನು
ಭಾಗ-ವನ್ನು
ಭಾಗ-ವಹಿ-ಸಲು
ಭಾಗ-ವಹಿ-ಸು-ತ್ತಾರೆ
ಭಾಗ-ವಹಿ-ಸು-ತ್ತಿ-ದ್ದರು
ಭಾಗ-ವಹಿ-ಸು-ತ್ತಿದ್ದ
ಭಾಗ-ವಹಿ-ಸು-ವಂತೆ
ಭಾಗ-ವಹಿ-ಸು-ವು-ದ-ಕ್ಕಾಗಿ
ಭಾಗ-ವಹಿ-ಸು-ವುದು
ಭಾಗ-ವಹಿಸಿ-ದರು
ಭಾಗ-ವಹಿಸಿದ
ಭಾಗ-ವಾ-ದರೂ
ಭಾಗ-ವಾಗಿ-ತ್ತು
ಭಾಗಕ್ಕೆ
ಭಾಗದ
ಭಾಗಿ-ಗ-ಳಾಗು-ವು-ದಕ್ಕೆ
ಭಾಗಿ-ಗ-ಳಾದ
ಭಾಗಿ-ಗ-ಳಾದರು
ಭಾಗಿ-ಗಳಾಗಿ-ರುವರು
ಭಾಗಿ-ಗಳು
ಭಾಗಿ-ಯಾ-ದರು
ಭಾಗಿ-ಯಾಗ-ಬ-ಲ್ಲರೊ
ಭಾಗಿ-ಯಾಗು-ತ್ತಿದ್ದಳು
ಭಾಗ್ಯ
ಭಾಟಿ
ಭಾಟೆ-ಯ-ವ-ರ-ನ್ನು
ಭಾದಕ-ವಿ-ಲ್ಲ
ಭಾನು-ವಾರ
ಭಾನು-ವಾರದ
ಭಾನು-ವಾರವೂ
ಭಾಯ್
ಭಾರ
ಭಾರ-ತ-ಖಂಡಕ್ಕೆ
ಭಾರ-ತ-ದಿಂದ
ಭಾರ-ತ-ಮಾತೆ
ಭಾರ-ತ-ಮಾತೆಯ
ಭಾರ-ತ-ವರ್ಷ
ಭಾರ-ತ-ವರ್ಷ-ದ-ಲ್ಲಿ
ಭಾರ-ತ-ವರ್ಷ-ದ-ಲ್ಲಿ-ರುವ
ಭಾರ-ತ-ವರ್ಷ-ದೋಪಾದಿ-ಯ-ಲ್ಲಿ
ಭಾರ-ತ-ವರ್ಷದ
ಭಾರ-ತ-ವೇಳಲಿ
ಭಾರ-ತಕ್ಕಿಂತ
ಭಾರ-ತಕ್ಕೆ
ಭಾರ-ತದ
ಭಾರ-ತಾಂಬೆ
ಭಾರ-ತೀಯ
ಭಾರ-ತೀಯ-ನಾಗ-ಬೇಕು
ಭಾರ-ತೀಯ-ನಿ-ಗಾ-ದರೋ
ಭಾರ-ತೀಯ-ನಿಗೂ
ಭಾರ-ತೀಯ-ರ-ನ್ನು
ಭಾರ-ತೀಯ-ರ-ಲ್ಲಿ
ಭಾರ-ತೀಯ-ರಾಗ-ಬ-ಲ್ಲಿರಾ
ಭಾರ-ತೀಯ-ರಿ-ಗಾಗಿ
ಭಾರ-ತೀಯ-ರಿ-ಗಿಂತ
ಭಾರ-ತೀಯ-ರಿಂದ
ಭಾರ-ತೀಯ-ರಿಗೂ
ಭಾರ-ತೀಯ-ರಿಗೂ-ಭಾರ-ತೀಯ-ರಿಗೆ
ಭಾರ-ತೀಯ-ರಿರಾ
ಭಾರ-ತೀಯ-ಳಾದಳು
ಭಾರ-ತೀಯನ
ಭಾರ-ತೀಯನು
ಭಾರ-ತೀಯರ
ಭಾರ-ತೀಯರು
ಭಾರ-ವನ್ನು
ಭಾರ-ವನ್ನೆ-ಲ್ಲ
ಭಾರ-ವಾ-ಯಿತು
ಭಾರ-ವಾಗಿ
ಭಾರ-ವಾದ
ಭಾರತ
ಭಾರೀ
ಭಾವ
ಭಾವ-ಗ-ಳನ್ನು
ಭಾವ-ಗ-ಳಿಗೆ
ಭಾವ-ಗಳ
ಭಾವ-ಗಳ-ನ್ನೆ-ಲ್ಲ
ಭಾವ-ಗಳಿಗೂ
ಭಾವ-ಗಳಿವೆ
ಭಾವ-ಗಳು
ಭಾವ-ಚಿ-ತ್ರ
ಭಾವ-ಚಿ-ತ್ರ-ವನ್ನಿ-ಟ್ಟು
ಭಾವ-ಚಿ-ತ್ರ-ವನ್ನು
ಭಾವ-ಚಿ-ತ್ರವೆ
ಭಾವ-ದ-ಲ್ಲಿ
ಭಾವ-ದ-ಲ್ಲಿ-ದ್ದರು
ಭಾವ-ದಿಂದ
ಭಾವ-ಧಾರೆ-ಯನ್ನು
ಭಾವ-ನ-ಗರ
ಭಾವ-ನಾ-ತರಂಗ-ಗಳ
ಭಾವ-ನಾ-ಮೃತ-ವನ್ನೂ
ಭಾವ-ನೆ-ಗ-ಳನ್ನು
ಭಾವ-ನೆ-ಗ-ಳಾದರೊ
ಭಾವ-ನೆ-ಗ-ಳಿಗೆ
ಭಾವ-ನೆ-ಗಳ
ಭಾವ-ನೆ-ಗಳ-ನ್ನೆ-ಲ್ಲ
ಭಾವ-ನೆ-ಗಳ-ನ್ನೇ
ಭಾವ-ನೆ-ಗಳ-ನ್ನೊಳ-ಗೊಂಡ
ಭಾವ-ನೆ-ಗಳಿಂದ
ಭಾವ-ನೆ-ಗಳಿಗೂ
ಭಾವ-ನೆ-ಗಳು
ಭಾವ-ನೆ-ಗಳೆ-ಲ್ಲ
ಭಾವ-ನೆ-ಗಳೆ-ಲ್ಲ-ದರ
ಭಾವ-ನೆ-ಗಳೆ-ಲ್ಲಾ
ಭಾವ-ನೆ-ಯ-ನ್ನೆ-ಲ್ಲ
ಭಾವ-ನೆ-ಯ-ನ್ನೊಳ-ಗೊಂಡ
ಭಾವ-ನೆ-ಯ-ಲ್ಲಿ
ಭಾವ-ನೆ-ಯ-ಲ್ಲಿಯೇ
ಭಾವ-ನೆ-ಯನ್ನು
ಭಾವ-ನೆ-ಯಾ-ಗಿದೆ
ಭಾವ-ನೆ-ಯಾಗಲಿ
ಭಾವ-ನೆ-ಯಿಂದ
ಭಾವ-ನೆ-ಯಿಂದಲೇ
ಭಾವ-ನೆಗೂ
ಭಾವ-ನೆಗೆ
ಭಾವ-ನೆಯ
ಭಾವ-ನೆಯೂ
ಭಾವ-ನೆಯೇ
ಭಾವ-ಪ-ರವ-ಶ-ನಾಗು-ವೆನೋ
ಭಾವ-ಪೂರ್ಣ-ವಾಗಿ-ತ್ತೊ
ಭಾವ-ಮಯ-ರಾದ
ಭಾವ-ಮಯ-ರಾದ-ವರು
ಭಾವ-ಮುಖ-ದ-ಲ್ಲಿ
ಭಾವ-ಮುಖ-ವಾ-ಯಿತು
ಭಾವ-ಯೇ-ತ್
ಭಾವ-ರ-ತ್ನ-ಗಳು
ಭಾವ-ವನ್ನು
ಭಾವ-ವನ್ನೂ
ಭಾವ-ವನ್ನೇ
ಭಾವ-ವಶ-ರಾಗು-ತ್ತಿದ್ದರು
ಭಾವ-ವಾಗಲಿ
ಭಾವ-ವಾದ-ರೋಭಾವ-ವಿದೆ
ಭಾವ-ಶೂ-ನ್ಯ
ಭಾವ-ಸಮಾಧಿ-ಯ-ಲ್ಲಿ
ಭಾವ-ಸಮಾಧಿಗೆ
ಭಾವಂದಿರೂ
ಭಾವಕ್ಕೆ
ಭಾವದ
ಭಾವನಾ
ಭಾವಾನು-ವಾದ-ವನ್ನು
ಭಾವಾವ-ಸ್ಥೆ-ಯ-ಲ್ಲಿ-ರು-ವೆನು
ಭಾವಾವ-ಸ್ಥೆಗೆ
ಭಾವಾವಿಷ್ಟ-ರಾ-ದರು
ಭಾವಾವಿಷ್ಟಳಾಗಿ
ಭಾವಿ-ಸ-ಲಿ-ಲ್ಲ
ಭಾವಿ-ಸದ
ಭಾವಿ-ಸಿದ
ಭಾವಿ-ಸಿದೆ
ಭಾವಿ-ಸಿದ್ದ
ಭಾವಿ-ಸಿದ್ದೆ
ಭಾವಿ-ಸು-ತ್ತಿ-ತ್ತು
ಭಾವಿ-ಸು-ತ್ತಿ-ದ್ದರು
ಭಾವಿ-ಸು-ತ್ತಿ-ದ್ದರೋ
ಭಾವಿ-ಸು-ತ್ತಿ-ದ್ದುದೇ
ಭಾವಿ-ಸು-ತ್ತೀಯೊ
ಭಾವಿ-ಸು-ತ್ತೇನೆ
ಭಾವಿ-ಸು-ತ್ತೇವೆ
ಭಾವಿ-ಸು-ವಂತೆ
ಭಾವಿ-ಸು-ವಿ-ರೇನು
ಭಾವಿ-ಸು-ವುದು
ಭಾವಿ-ಸು-ವೆನು
ಭಾವಿ-ಸು-ವೆವು
ಭಾವಿ-ಸುವ
ಭಾವಿ-ಸುವನು
ಭಾವಿ-ಸುವರು
ಭಾವಿ-ಸುವುದ-ರ-ಲ್ಲಿಯೇ
ಭಾವಿ-ಸುವುದಕ್ಕಿಂತ
ಭಾವಿ-ಸುವೆ
ಭಾವಿ-ಸುವೆಯಾ
ಭಾವಿ-ಸುವೆವೂ
ಭಾವಿಸ-ಕೂ-ಡದು
ಭಾವಿಸ-ತೊಡಗಿದನು
ಭಾವಿಸ-ತೊಡಗಿದರು
ಭಾವಿಸ-ತೊಡಗಿದೆ
ಭಾವಿಸ-ಬಹುದು
ಭಾವಿಸ-ಬಾ-ರದು
ಭಾವಿಸ-ಬೇಕು
ಭಾವಿಸ-ಬೇಡಿ
ಭಾವಿಸಿ
ಭಾವಿಸಿ-ಕೊ-ಳ್ಳು-ತ್ತ
ಭಾವಿಸಿ-ಕೊಂಡೆವು
ಭಾವಿಸಿ-ದನು
ಭಾವಿಸಿ-ದರು
ಭಾವಿಸಿ-ದರೂ
ಭಾವಿಸಿ-ದರೆ
ಭಾವಿಸಿ-ದಳು
ಭಾವಿಸಿ-ದಿ-ರೇನು
ಭಾವಿಸಿ-ದಿರಾ
ಭಾವಿಸಿ-ದೆವು
ಭಾವಿಸಿ-ದ್ದರು
ಭಾವಿಸಿ-ದ್ದರೆ
ಭಾವಿಸಿ-ದ್ದೆನು
ಭಾವಿಸಿ-ದ್ದೇನೆ
ಭಾವಿಸಿ-ಯಾರು
ಭಾವಿಸಿ-ರ-ಬೇಕು
ಭಾವಿಸಿ-ರ-ಲಿ-ಲ್ಲ
ಭಾವಿಸು-ವಿರಾ
ಭಾವಿಸು-ವಿರಿ
ಭಾವಿಸು-ವೆವೋ
ಭಾವೀ
ಭಾವುಕ
ಭಾವೋ-ನ್ಮತ್ತ-ನಾದ
ಭಾವೋದ್ರೇಕ-ವನ್ನು
ಭಾಷಣ
ಭಾಷಣ-ಗ-ಳನ್ನು
ಭಾಷಣ-ಗಳ-ಲ್ಲಿ
ಭಾಷಣ-ಗಳು
ಭಾಷಣ-ದ-ಲ್ಲಿ
ಭಾಷಣ-ದಿಂದ
ಭಾಷಣ-ದೊಂದಿಗೆ
ಭಾಷಣ-ಮಾಡಲು
ಭಾಷಣ-ವನ್ನು
ಭಾಷಣದ
ಭಾಷಾ-ಶಾ-ಸ್ತ್ರ-ಗಳ
ಭಾಷಾ-ಶಾ-ಸ್ತ್ರ-ದ-ಲ್ಲಿ
ಭಾಷಾಂ-ತರ
ಭಾಷಾಂ-ತರ-ಗ-ಳನ್ನು
ಭಾಷಾಂ-ತರ-ಮಾಡಿ
ಭಾಷಾಂ-ತರ-ಮಾಡು-ತ್ತಿದ್ದರು
ಭಾಷಾಂ-ತರಿ-ಸು-ವುದು
ಭಾಷಾಂತ-ರವೇ
ಭಾಷೆ
ಭಾಷೆ-ಗ-ಳನ್ನು
ಭಾಷೆ-ಗಳ
ಭಾಷೆ-ಗಳ-ಲ್ಲಿ
ಭಾಷೆ-ಯ-ಲ್ಲಿ
ಭಾಷೆ-ಯ-ಲ್ಲಿ-ಡು-ವುದು
ಭಾಷೆ-ಯ-ಲ್ಲಿ-ರುವ
ಭಾಷೆ-ಯ-ಲ್ಲಿಯೇ
ಭಾಷೆ-ಯ-ಲ್ಲೆ
ಭಾಷೆ-ಯನ್ನು
ಭಾಷೆಗೆ
ಭಾಷೆಯ
ಭಾಷ್ಯ
ಭಾಷ್ಯ-ಗಳ
ಭಾಷ್ಯ-ದಂತೆ
ಭಾಷ್ಯ-ವನ್ನು
ಭಾಷ್ಯಂ
ಭಾಷ್ಯದ
ಭಾಸ-ವಾ-ಗು-ತ್ತಿದೆ
ಭಾಸ-ವಾ-ಯಿತು
ಭಾಸ-ವಾಗು-ತ್ತದೆ
ಭಾಸ-ವಾಗು-ತ್ತಿ-ತ್ತು
ಭಾಸ-ವಾಗು-ತ್ತಿ-ರಲಿ-ಲ್ಲ
ಭಾಸ-ವಾಗು-ವುದು
ಭಿ
ಭಿಂಗ
ಭಿಕ್ಕು
ಭಿಕ್ಷ-ಕರಾಗ-ಬೇಕಾಗಿ-ಲ್ಲ
ಭಿಕ್ಷ-ದಿಂದ
ಭಿಕ್ಷವೆ-ತ್ತಲು
ಭಿಕ್ಷಾ-ನ್ನ-ದಿಂದ
ಭಿಕ್ಷಾ-ಪಾ-ತ್ರೆ
ಭಿಕ್ಷಾ-ಪಾ-ತ್ರೆ-ಯನ್ನು
ಭಿಕ್ಷು
ಭಿಕ್ಷು-ಕ-ನಂತೆ
ಭಿಕ್ಷು-ಕ-ನಾ-ದರೊ
ಭಿಕ್ಷು-ಕ-ನಾಗಿ-ರು-ವೆನು
ಭಿಕ್ಷು-ಕ-ನೊಬ್ಬ
ಭಿಕ್ಷು-ಕನ
ಭಿಕ್ಷು-ಕರ
ಭಿಕ್ಷು-ಕರು
ಭಿಕ್ಷು-ಗ-ಳಾದ
ಭಿಕ್ಷು-ಗಳಿಗೂ
ಭಿಕ್ಷು-ವಿಗೆ
ಭಿಕ್ಷುಕ
ಭಿಕ್ಷೆ
ಭಿಕ್ಷೆ-ಯನ್ನು
ಭಿಕ್ಷೆ-ಯಿಂದ
ಭಿಕ್ಷೆಗೆ
ಭಿಕ್ಷೆಯ
ಭೀ
ಭೀತಿ-ಯಾ-ಯಿತು
ಭೀತಿ-ಯಿಂದ
ಭೀಮ
ಭೀಮ-ವ್ಯಕ್ತಿ
ಭೀಮಾ-ಕಾರದ
ಭೀರುತೆ
ಭು
ಭುಜದ
ಭುತ-ಗಳ-ಲ್ಲಿ
ಭುವಿ
ಭುಸುಗು-ಟ್ಟ-ಬೇಕು
ಭೂ
ಭೂಜ್ಗೆ
ಭೂತ
ಭೂತ-ಕಾಲ
ಭೂತ-ಕಾಲ-ದ-ಲ್ಲೆ
ಭೂತ-ಕಾಲದ
ಭೂತ-ಗಳ
ಭೂತ-ಗಳಾಗಿ-ದ್ದೀರಿ
ಭೂತ-ಗಳಿ-ಗಿಂತ
ಭೂತ-ಪ್ರೇತ-ಗ-ಳನ್ನು
ಭೂತ-ಪ್ರೇತ-ಗಳ-ನ್ನೂ
ಭೂತ-ಪ್ರೇತ-ಗಳಿಂದ
ಭೂತ-ಪ್ರೇತವೂ
ಭೂತದ
ಭೂಮ
ಭೂಮ-ವನ್ನು
ಭೂಮ-ವಾ-ದುದು
ಭೂಮ-ವಾದ
ಭೂಮದ
ಭೂಮಾ-ಕಾರದ
ಭೂಮಿ
ಭೂಮಿ-ಕೆ-ಯ-ಲ್ಲಿ-ರಲು
ಭೂಮಿ-ಕೆಗೆ
ಭೂಮಿ-ಕೆಯೇ
ಭೂಮಿ-ಯ-ಲ್ಲಿ
ಭೂಮಿ-ಯನ್ನು
ಭೂಮಿಗೆ
ಭೂಮಿಯ
ಭೂಷಣ-ಗಳ-ಲ್ಲಿ
ಭೂಷಣ-ಗಳ-ಲ್ಲಿಯೂ
ಭೂಷಣ-ಪ್ರಾಯ-ವಾಗಿ-ದ್ದವು
ಭೂಷಣವ-ಲ್ಲದ
ಭೂಷಿತ
ಭೂಷಿತಾಂಗ-ರಾದ
ಭೇಟಿ
ಭೇಟಿ-ಕೊ-ಟ್ಟರು
ಭೇಟಿ-ಗಳ-ಲ್ಲಿ
ಭೇಟಿ-ಗಳು
ಭೇಟಿ-ಮಾಡಿ-ಕೊಂಡು-ಹೋಗಿ
ಭೇಟಿ-ಮಾಡಿ-ದರು
ಭೇಟಿ-ಮಾಡು-ವುದು
ಭೇಟಿ-ಯನ್ನು
ಭೇಟಿಗೆ
ಭೇದ-ಗಳ
ಭೇದ-ಭಾ-ವನೆ
ಭೇದ-ಭಾವ
ಭೇದ-ಭಾವ-ದಿಂದ
ಭೇದ-ಭಾವ-ವಿ-ಲ್ಲದೆ
ಭೇದ-ಮಾಡಲು
ಭೇದ-ವನ್ನು
ಭೇದ-ವಿ-ಲ್ಲದೆ
ಭೇದವೂ
ಭೇದಿ-ಸಲಾಗದ
ಭೇದಿ-ಸು-ತ್ತಿದ್ದ
ಭೇದಿ-ಸು-ವು-ದಕ್ಕೆ
ಭೇದಿಸ-ಬೇಕೆಂದು
ಭೇದಿಸ-ಬೇಕೆಂಬ
ಭೇದಿಸಿ
ಭೇದಿಸಿ-ಕೊಂಡು
ಭೇದಿಸಿ-ದರು
ಭೇಷ್
ಭೈ
ಭೈರ-ವನ
ಭೈರ-ವನೇ-ಭೈ
ಭೋ
ಭೋಗ
ಭೋಗ-ಭೂಮಿ-ಯೆನಿ-ಸಿದ
ಭೋಗ-ಭೂಮಿಯ
ಭೋಗ-ವಿಲಾಸ-ಗ-ಳಿಗೆ
ಭೋಗಕ್ಕೆ
ಭೋಗದ
ಭೋಗವೇ
ಭೋಗಿ-ಸು-ವು-ದಕ್ಕೆ
ಭೋಗ್ಯ
ಭೋದ-ಗಯೆಯ-ಲ್ಲಿದ್ದ
ಭೋನಿಪ್ಯಾಟಿಯೊ
ಭೋರ್ಗರೆದು
ಭೌತ-ಶಾ-ಸ್ತ್ರ
ಭೌತಿಕ
ಭೌದ್ಧಿ-ಕ-ವಾಗಿ
ಭ್ರಮ
ಭ್ರಮಿ-ಸು-ವು-ದಕ್ಕೆ
ಭ್ರಮೆ
ಭ್ರಮೆ-ಯ-ಲ್ಲಿ
ಭ್ರಮೆ-ಯನ್ನು
ಭ್ರಮೆ-ಯಿಂದ
ಭ್ರಮೆ-ಯುಂಟಾಗು-ವುದು
ಭ್ರಷ್ಟ-ತನ
ಭ್ರಷ್ಟ-ನ-ನ್ನಾಗಿ
ಭ್ರಷ್ಟ-ರಾಗಿ
ಭ್ರಷ್ಟ-ರೆಂದೂ
ಭ್ರಷ್ಟನಾದ
ಭ್ರಷ್ಟರಾ-ದ-ವರು
ಭ್ರಷ್ಟರಾಗು-ತ್ತಾರೆಯೇ
ಭ್ರಷ್ಟವೃಕ್ತಿ-ಗಳು
ಭ್ರಹ್ಮ-ನನ್ನು
ಭ್ರಾಂ
ಭ್ರಾಂತಿ
ಭ್ರಾಂತಿ-ಪುರುಷ-ರ-ಲ್ಲ
ಭ್ರಾಂತಿ-ಯ-ಲ್ಲದೇ
ಭ್ರಾಂತಿ-ಯನ್ನು
ಭ್ರಾಂತಿಯ
ಭ್ರಾಂತಿಯೋ
ಭ್ರಾತೃ-ಗಳ
ಭ್ರಾತೃ-ವರ್ಗ-ದ-ವ-ರಾದ
ಭ್ರಾತೃ-ವರ್ಯರೇ
ಭ್ರಾಮ್ಯಕ
ಭ್ರೂ
ಮ
ಮಂ
ಮಂಕು
ಮಂಗ
ಮಂಗ-ನಿಂದ
ಮಂಗ-ಳ-ದಾಯಕನೂ
ಮಂಗ-ಳ-ಧ್ವನಿ
ಮಂಗ-ಳ-ರವ
ಮಂಗ-ಳ-ವಾಗವೆ
ಮಂಗ-ಳ-ವಾರ
ಮಂಗ-ಳ-ಸಿಂ-ಗರು
ಮಂಗ-ಳ-ಸಿಂ-ಗರೂ
ಮಂಗ-ಳ-ಸಿಂಗ್
ಮಂಗ-ಳಾರತಿ
ಮಂಗ-ಳಾರತಿಗೆ
ಮಂಚ-ವನ್ನು
ಮಂಚದ
ಮಂಜಿ-ನಿಂದ
ಮಂಜೂಷೆ-ಯನ್ನೂ
ಮಂಟಪ-ಗಳು
ಮಂಟಪ-ವನ್ನು
ಮಂಡಿಯ-ವ-ರೆಗೆ
ಮಂಡೋ-ದರಿ
ಮಂದ
ಮಂದ-ಗ-ಮ-ನ-ದಿಂದ
ಮಂದ-ಗ-ಮನೆ
ಮಂದ-ಮಧುರ
ಮಂದ-ವಾಗಿ
ಮಂದ-ವಾಗಿ-ತ್ತು
ಮಂದ-ವಾಗಿ-ರು-ವುದು
ಮಂದ-ವಾದ
ಮಂದ-ಹಾಸ
ಮಂದ-ಹಾಸ-ದಿಂದ
ಮಂದ-ಹಾಸ-ವನ್ನು
ಮಂದಿ
ಮಂದಿ-ಗ-ಳನ್ನು
ಮಂದಿ-ಗ-ಳಿಗೆ
ಮಂದಿ-ಗಳ
ಮಂದಿ-ಗಳ-ಲ್ಲಿ
ಮಂದಿ-ಯ-ವ-ರೆಗೂ
ಮಂದಿ-ರ-ದ-ಲ್ಲಿದ್ದ
ಮಂದಿ-ರ-ದ-ಲ್ಲಿಯೂ
ಮಂದಿ-ರ-ದಿಂದ
ಮಂದಿ-ರ-ವನ್ನು
ಮಂದಿ-ರಕ್ಕೆ
ಮಂದಿ-ರದ
ಮಂದಿ-ರವು
ಮಂದಿರ
ಮಂದೆ-ಯಂತೆ
ಮಂದೆಯ
ಮಂದೆಯ-ಲ್ಲಿದ್ದ
ಮಂದೆಯ-ಲ್ಲಿಯೇ
ಮಂದೆಯು
ಮಕ್ಕಳ
ಮಕ್ಕಳ-ನ್ನು
ಮಕ್ಕಳ-ನ್ನೇ
ಮಕ್ಕಳ-ವರು
ಮಕ್ಕಳಂತೆ
ಮಕ್ಕಳಾಗಿ-ರ-ಲಿ-ಲ್ಲ
ಮಕ್ಕಳಾಟ
ಮಕ್ಕಳಾಟ-ವ-ಲ್ಲ
ಮಕ್ಕಳಾದ
ಮಕ್ಕಳಿ-ರ-ಲಿ-ಲ್ಲ
ಮಕ್ಕಳಿಗೆ
ಮಕ್ಕಳಿರಾ
ಮಕ್ಕಳಿರಾ-ಮಕ್ಕಳು
ಮಕ್ಕಳು-ಗಳಾಗಿ-ದ್ದರೂ
ಮಕ್ಕಳೆ
ಮಕ್ಕಳೇ
ಮಕ್ಕಳೊಂದಿಗೆ
ಮಗ
ಮಗ-ನಂತೆ
ಮಗ-ನನ್ನು
ಮಗ-ನನ್ನೂ
ಮಗ-ನಾ-ದು-ದ-ರಿಂದ
ಮಗ-ನಿಗೂ
ಮಗ-ನಿಗೆ
ಮಗ-ನೊ-ಡನೆ
ಮಗನ
ಮಗನೆ
ಮಗನೇ
ಮಗು
ಮಗು-ಚಿ-ಕೊಂಡಿತು
ಮಗು-ವನ್ನು
ಮಗು-ವಾ-ದಾಗಿಲಿ-ನಿಂದಲೂ
ಮಗು-ವಿ-ಗಿಂತಲೂ
ಮಗು-ವಿ-ನಂತಹ
ಮಗು-ವಿ-ನಂತೆ
ಮಗು-ವಿ-ನಂತೆಯೇ
ಮಗು-ವಿ-ನೊ-ಡನೆ
ಮಗು-ವಿಗೂ
ಮಗು-ವಿಗೆ
ಮಗು-ವಿನ
ಮಗು-ವಿನ-ಲ್ಲಿ-ರು-ವುದು
ಮಗುವೇ
ಮಗ್ನ-ನಾ-ದನು
ಮಗ್ನ-ನಾಗಿ
ಮಗ್ನ-ರಾ-ದರು
ಮಗ್ನ-ರಾ-ದು-ದ-ರಿಂದ
ಮಗ್ನ-ರಾಗಿ
ಮಗ್ನ-ರಾಗಿ-ದ್ದರು
ಮಗ್ನ-ರಾಗಿ-ರ-ಬೇಕೆಂದು
ಮಗ್ನ-ರಾಗಿ-ರುವರು
ಮಗ್ನ-ರಾಗುವರು
ಮಗ್ನ-ರಾದ
ಮಚ್ಚೆ
ಮಚ್ಚೆ-ಯನ್ನು
ಮಜುಂ-ದಾ-ರರು
ಮಠ
ಮಠ-ಗಳ-ನ್ನೆ-ಲ್ಲ
ಮಠ-ಗಳಿ-ಲ್ಲದೆ
ಮಠ-ಗಳು
ಮಠ-ದ-ಲ್ಲಿ
ಮಠ-ದ-ಲ್ಲಿ-ದ್ದರು
ಮಠ-ದ-ಲ್ಲಿ-ದ್ದಾಗ
ಮಠ-ದ-ಲ್ಲಿ-ರುವ-ವ-ರಿಗೆ
ಮಠ-ದ-ಲ್ಲಿ-ರುವ-ವ-ರಿಗೆ-ಲ್ಲ
ಮಠ-ದ-ಲ್ಲಿ-ರುವ-ವ-ರೆ-ಲ್ಲ
ಮಠ-ದ-ಲ್ಲಿ-ರುವಾಗ
ಮಠ-ದ-ಲ್ಲಿದೆ
ಮಠ-ದ-ಲ್ಲಿದ್ದ
ಮಠ-ದ-ಲ್ಲಿದ್ದು
ಮಠ-ದ-ಲ್ಲಿಯೇ
ಮಠ-ದ-ಲ್ಲೆ-ಲ್ಲ
ಮಠ-ದ-ಲ್ಲೇ
ಮಠ-ದ-ವ-ರೆ-ಲ್ಲ
ಮಠ-ದಿಂದ
ಮಠ-ಪ್ರ-ದೇಶ-ದ-ಲ್ಲಿ
ಮಠ-ವನ್ನು
ಮಠ-ವನ್ನೆ-ಲ್ಲ
ಮಠ-ವನ್ನೆ-ಲ್ಲಾ
ಮಠ-ವಿ-ತ್ತು
ಮಠ-ವಿ-ದ್ದಿದ್ದರೆ
ಮಠ-ವಿದೆ
ಮಠ-ವಿದೆ-ಯ-ಲ್ಲ
ಮಠಕ್ಕೆ
ಮಠದ
ಮಠವೂ
ಮಠವೇ
ಮಠಾಧಿ-ಪತಿ-ಗ-ಳನ್ನು
ಮಠಾಧಿ-ಪತಿ-ಗಳ
ಮಠಾಧಿ-ಪತಿ-ಗಳು
ಮಠಾಧ್ಯಕ್ಷರು
ಮಠಾಧ್ಯಕ್ಷರೂ
ಮಡಿ
ಮಡಿ-ಕೆ-ಯನ್ನು
ಮಡಿ-ದ-ರೇ-ನಂತೆ
ಮಡಿ-ದಿದ್ದಾನೆ
ಮಡಿ-ಯ-ಬೇಕು
ಮಡಿಕೆ
ಮಡಿಸಿ
ಮಡು
ಮಡು-ತ್ತಿದ್ದುವು
ಮಣ
ಮಣಿ-ಗ-ಳನ್ನು
ಮಣಿ-ಯು-ತ್ತಿದ್ದ
ಮಣಿ-ಯು-ವಂತೆ
ಮಣಿ-ಯುವನು
ಮಣಿದು
ಮಣ್ಣ-ನ್ನು
ಮಣ್ಣಿ-ನಿಂದ
ಮಣ್ಣು
ಮಣ್ಣು-ಪಾಲು
ಮಣ್ಣೆ
ಮಣ್ಣೇ-ನಾ-ದರೂ
ಮತ
ಮತ-ಗ-ಳನ್ನು
ಮತ-ಗ-ಳಿಗೆ
ಮತ-ಗಳ
ಮತ-ಗಳ-ನ್ನೂ
ಮತ-ಗಳ-ಲ್ಲಿ
ಮತ-ಗಳಷ್ಟು
ಮತ-ಗಳೇ
ಮತ-ದ-ಲ್ಲೊ
ಮತ-ದಂತೆ
ಮತ-ಪಂಥದ
ಮತ-ಪ್ರ-ಚಾರವೂ
ಮತ-ಭಾವ-ಗ-ಳನ್ನು
ಮತ-ಭ್ರಾಂ-ತ-ರಾಗಿ-ರುವರು
ಮತ-ಭ್ರಾಂ-ತರು
ಮತ-ಭ್ರಾಂತ-ರ-ನ್ನು
ಮತ-ಭ್ರಾಂತಿ
ಮತ-ವನ್ನು
ಮತ-ವೇ-ನೆಂ-ದರೆ
ಮತ-ಸ್ಥರಾದ
ಮತ-ಸ್ಥಾಪ-ಕರ
ಮತವು
ಮತವೆ
ಮತಾಂ-ತರ-ಗೊಳಿ-ಸು-ವು-ದಕ್ಕೆ
ಮತಾಂ-ತರ-ಗೊಳಿ-ಸುವು-ದ-ಲ್ಲ
ಮತಾಂ-ತರ-ಗೊಳಿಸಿ-ರು-ವಿರಿ
ಮತಾಂ-ತರ-ಗೊಳ್ಳ-ಬೇಕಾಗಿ-ತ್ತು
ಮತಾಂ-ತರ-ವನ್ನು
ಮತಾಂ-ತರ-ವಾಗು-ವು-ದ-ನ್ನು
ಮತಾಂಧ-ತೆಯ
ಮತಾಂಧತೆ
ಮತಾಂಧತೆ-ಯನ್ನು
ಮತಿ-ಗತಿ-ಗ-ಳನ್ನು
ಮತಿ-ಯಾಗಿ-ದ್ದನು
ಮತು
ಮತು-ಕತೆ-ಯನ್ನು
ಮತ್ತ-ರಾಗಿ-ಬಿ-ಟ್ಟಿ-ದ್ದರು
ಮತ್ತತೆ-ಯ-ಲ್ಲಿ
ಮತ್ತಷ್ಟು
ಮತ್ತಾ-ರಿಗೆ
ಮತ್ತಾ-ವಾಗಲೂ
ಮತ್ತಾ-ವು-ದ-ರಿಂದಲೂ
ಮತ್ತಾ-ವು-ದಕ್ಕೂ
ಮತ್ತಾ-ವುದು
ಮತ್ತಾ-ವುದೂ
ಮತ್ತಾರ
ಮತ್ತಾರ-ನ್ನೋ
ಮತ್ತಾರಿಗೋ
ಮತ್ತಾರೂ
ಮತ್ತಾರೋ
ಮತ್ತಾವ
ಮತ್ತಾವು-ದಾದರೂ
ಮತ್ತಾವು-ದಿ-ರು-ವುದು
ಮತ್ತಿಳಿ-ಯು-ವುದೋ
ಮತ್ತು
ಮತ್ತೂ
ಮತ್ತೆ
ಮತ್ತೆ-ಲ್ಲಿ
ಮತ್ತೆ-ಲ್ಲಿಗೊ
ಮತ್ತೆ-ಲ್ಲಿಯೂ
ಮತ್ತೆ-ಲ್ಲೂ
ಮತ್ತೆಂದಿಗೂ
ಮತ್ತೇ-ನನ್ನು
ಮತ್ತೇ-ನನ್ನೂ
ಮತ್ತೇನಿದೆ
ಮತ್ತೇನು
ಮತ್ತೇನೂ
ಮತ್ತೇರಿ-ದೆಯೋ
ಮತ್ತೇರಿದ
ಮತ್ತೊ-ಮ್ಮೆ
ಮತ್ತೊಂ-ದ-ನ್ನು
ಮತ್ತೊಂ-ದ-ರ-ಲ್ಲಿ
ಮತ್ತೊಂ-ದರ
ಮತ್ತೊಂ-ದಿ-ಲ್ಲ
ಮತ್ತೊಂದು
ಮತ್ತೊಂದೇ
ಮತ್ತೊಬ್ಬ
ಮತ್ತೊಬ್ಬ-ನನ್ನು
ಮತ್ತೊಬ್ಬ-ನಿಂದ
ಮತ್ತೊಬ್ಬ-ನಿಗೆ
ಮತ್ತೊಬ್ಬ-ನೊಂದಿಗೆ
ಮತ್ತೊಬ್ಬ-ರ-ದ-ರ-ಲ್ಲಿ
ಮತ್ತೊಬ್ಬ-ರ-ನ್ನು
ಮತ್ತೊಬ್ಬ-ರ-ಲ್ಲಿ
ಮತ್ತೊಬ್ಬ-ರದು
ಮತ್ತೊಬ್ಬ-ರಿಂದ
ಮತ್ತೊಬ್ಬ-ರಿಗೆ
ಮತ್ತೊಬ್ಬ-ರೊ-ಡನೆ
ಮತ್ತೊಬ್ಬನ
ಮತ್ತೊಬ್ಬನು
ಮತ್ತೊಬ್ಬರ
ಮತ್ತೊಬ್ಬರು
ಮತ್ತೊಬ್ಬರೇ
ಮತ್ಸ್ಯ
ಮದು-ವೆಯ
ಮದುವೆ
ಮದುವೆ-ಮಾಡಿ
ಮದುವೆ-ಯನ್ನು
ಮದುವೆ-ಯನ್ನೂ
ಮದುವೆ-ಯಾಗ-ಬೇಕಾಗಿ-ತ್ತು
ಮದುವೆ-ಯಾಗದೆ
ಮದುವೆ-ಯಾಗಿ
ಮದುವೆ-ಯಾಗು-ತ್ತಾನೆ
ಮದುವೆ-ಯಾದ
ಮದ್ದು
ಮದ್ರಾ-ಸನ್ನು
ಮದ್ರಾ-ಸಿ-ನ-ಲ್ಲಿ
ಮದ್ರಾ-ಸಿ-ನಿಂದ
ಮದ್ರಾ-ಸಿಗೆ
ಮದ್ರಾ-ಸಿನ
ಮದ್ರಾ-ಸಿನ-ಲ್ಲಿ-ದಾಗ
ಮದ್ರಾ-ಸಿನ-ಲ್ಲಿ-ದ್ದಾಗ
ಮದ್ರಾ-ಸಿನ-ಲ್ಲಿ-ರು-ವು-ದ-ನ್ನು
ಮದ್ರಾ-ಸಿನ-ಲ್ಲಿ-ರುವ
ಮದ್ರಾ-ಸಿನ-ಲ್ಲಿ-ರುವಾಗ
ಮದ್ರಾ-ಸಿನ-ಲ್ಲಿದ್ದ
ಮದ್ರಾ-ಸಿನ-ಲ್ಲಿಯೇ
ಮದ್ರಾ-ಸಿನ-ಲ್ಲೆ
ಮದ್ರಾ-ಸಿನ-ವ-ರೆಗೆ
ಮದ್ರಾ-ಸಿನ-ವರು
ಮದ್ರಾ-ಸ್
ಮದ್ರಾಸು
ಮಧು
ಮಧು-ಕರೀ
ಮಧು-ಪರ್ಕ-ದ-ಲ್ಲಿ
ಮಧು-ರ-ಮಧು-ರ-ವಾಗಿ
ಮಧು-ರ-ವಾ-ಯಿತು
ಮಧು-ರ-ವಾಗಿ
ಮಧು-ರ-ವಾಗಿ-ರುವ
ಮಧು-ರ-ವಾದ
ಮಧು-ರೆ-ಯ-ಲ್ಲಿ
ಮಧು-ರೆ-ಯ-ಲ್ಲಿ-ದ್ದಾಗ
ಮಧು-ರೆಗೆ
ಮಧು-ಸೂ-ಧನ
ಮಧು-ಸೂದ-ನನ
ಮಧುರ
ಮಧ್ಯ
ಮಧ್ಯ-ಕಾಲದ
ಮಧ್ಯ-ದ-ಲ್ಲಿ
ಮಧ್ಯ-ದ-ಲ್ಲಿ-ದ್ದರೂ
ಮಧ್ಯ-ದ-ಲ್ಲಿ-ದ್ದಾಗ
ಮಧ್ಯ-ದ-ಲ್ಲಿ-ರು-ವಿರಿ
ಮಧ್ಯ-ದ-ಲ್ಲಿ-ರು-ವುದು
ಮಧ್ಯ-ದ-ಲ್ಲಿ-ರುವ
ಮಧ್ಯ-ದ-ಲ್ಲಿಯೇ
ಮಧ್ಯ-ದ-ವ-ರೆಗೆ
ಮಧ್ಯ-ದಲಿ
ಮಧ್ಯ-ಪ್ರಾಂ-ತ್ಯದ-ಲ್ಲಿ-ರುವ
ಮಧ್ಯ-ಭಾಗ
ಮಧ್ಯ-ಭಾಗ-ದ-ಲ್ಲಿ
ಮಧ್ಯ-ಮ-ವರ್ಗ-ದ-ವ-ರಿಂದ
ಮಧ್ಯ-ವರ್ತಿ
ಮಧ್ಯ-ವರ್ತಿ-ಯ-ನ್ನಾಗಿ
ಮಧ್ಯ-ವರ್ತಿ-ಯಾಗ-ಬೇ-ಕಾ-ದರೆ
ಮಧ್ಯಮ
ಮಧ್ಯಾಹ್ನ
ಮಧ್ಯಾಹ್ನ-ದ-ಲ್ಲಿ
ಮಧ್ಯಾಹ್ನ-ದ-ವ-ರೆಗೂ
ಮಧ್ಯಾಹ್ನ-ನದ
ಮಧ್ಯಾಹ್ನದ
ಮಧ್ಯಾಹ್ನವೇ
ಮಧ್ಯೆ
ಮಧ್ಯೆ-ಮಧ್ಯೆ
ಮಧ್ವ
ಮನ-ಕರ-ಗಿತು
ಮನ-ಕರ-ಗುವಂತೆ
ಮನ-ಗಂಡ
ಮನ-ಗಂಡರು
ಮನ-ಗಂಡಿ-ದ್ದರು
ಮನ-ಗಂಡಿ-ರುವರು
ಮನ-ಗಂಡಿರ-ಬೇಕು
ಮನ-ಗಂಡು
ಮನ-ದ-ಟ್ಟಾ-ಯಿತು
ಮನ-ನ-ದ-ಲ್ಲಿ
ಮನ-ನ-ದಿಂದ
ಮನ-ನ-ಮಾ-ಡು-ತ್ತ
ಮನ-ನ-ಮಾಡ-ತೊಡಗಿದರು
ಮನ-ನ-ಮಾಡಿ
ಮನ-ನ-ಮಾಡಿಕೊ
ಮನ-ನ-ಮಾಡು-ತ್ತಿದ್ದರು
ಮನ-ಮ-ರುಗಿ
ಮನ-ಮಧು-ರೆಗೆ
ಮನ-ಮಧು-ರೆಯ
ಮನ-ಮು-ಟ್ಟು-ವಂತೆ
ಮನ-ರಂ-ಜನೆ
ಮನ-ರಂಜ-ನೆಗೆ
ಮನ-ರಂಜಕ
ಮನ-ವಿ-ಮಾಡಿ-ಕೊಂಡು
ಮನ-ಸೋ-ತರು
ಮನ-ಸ್ಕ-ನಾಗಿ
ಮನ-ಸ್ಕನು
ಮನ-ಸ್ತಾಪ-ಗ-ಳಿದ್ದರೆ
ಮನ-ಸ್ತಾಪ-ಗ-ಳೆಂಬ
ಮನ-ಸ್ಸನ್ನು
ಮನ-ಸ್ಸನ್ನೆ-ಲ್ಲ
ಮನ-ಸ್ಸನ್ನೆ-ಲ್ಲಾ
ಮನ-ಸ್ಸಾ-ದರೋ
ಮನ-ಸ್ಸಾಗ-ಲಿ-ಲ್ಲ
ಮನ-ಸ್ಸಿ-ಟ್ಟ-ವ-ರಿಗೆ
ಮನ-ಸ್ಸಿ-ಟ್ಟು
ಮನ-ಸ್ಸಿ-ನ-ಲ್ಲಿ
ಮನ-ಸ್ಸಿ-ನಿಂದ
ಮನ-ಸ್ಸಿ-ನೊ-ಡನೆ
ಮನ-ಸ್ಸಿ-ನೊಳಗೆ
ಮನ-ಸ್ಸಿ-ರ-ಲಿ-ಲ್ಲ
ಮನ-ಸ್ಸಿ-ಲ್ಲದ
ಮನ-ಸ್ಸಿಗೆ
ಮನ-ಸ್ಸಿದೆ
ಮನ-ಸ್ಸಿದ್ದರೆ
ಮನ-ಸ್ಸಿನ
ಮನ-ಸ್ಸಿನ-ಲ್ಲಿ-ಟ್ಟು-ಕೊಂಡು
ಮನ-ಸ್ಸಿನ-ಲ್ಲಿ-ದ್ದುದ-ನ್ನೆ-ಲ್ಲ
ಮನ-ಸ್ಸಿನ-ಲ್ಲಿ-ರು-ವು-ದ-ನ್ನು
ಮನ-ಸ್ಸಿನ-ಲ್ಲಿ-ರು-ವುದು
ಮನ-ಸ್ಸಿನ-ಲ್ಲಿ-ರುವ
ಮನ-ಸ್ಸಿನ-ಲ್ಲಿಡಿ
ಮನ-ಸ್ಸಿನ-ಲ್ಲಿದ್ದ
ಮನ-ಸ್ಸಿನ-ಲ್ಲಿಯೂ
ಮನ-ಸ್ಸಿನ-ಲ್ಲಿಯೇ
ಮನ-ಸ್ಸು
ಮನ-ಸ್ಸು-ಮಾಡಿ-ದರು
ಮನ-ಸ್ಸೂ
ಮನ-ಸ್ಸೆ
ಮನ-ಸ್ಸೇ
ಮನ-ಸ್ಸೇನೂ
ಮನಃ-ಕ-ಲ್ಪಿತವೆ
ಮನಃ-ಪೂರ್ವ-ಕ-ವಾಗಿ
ಮನನ
ಮನವಿ
ಮನಸಾ
ಮನಿ-ಭಾಯಿ
ಮನು-ಷ್ಯರು
ಮನುಷ್ಯ
ಮನುಷ್ಯ-ದೇಹ-ಧಾರಣೆ
ಮನುಷ್ಯ-ನ-ಲ್ಲಿ
ಮನುಷ್ಯ-ನ-ಲ್ಲಿ-ರುವ
ಮನುಷ್ಯ-ನ-ಲ್ಲಿದ್ದ
ಮನುಷ್ಯ-ನ-ಲ್ಲಿಯೂ
ಮನುಷ್ಯ-ನ-ವ-ರೆಗೆ
ಮನುಷ್ಯ-ನಂತೇ
ಮನುಷ್ಯ-ನದು
ಮನುಷ್ಯ-ನನ್ನು
ಮನುಷ್ಯ-ನಾ-ದರೂ
ಮನುಷ್ಯ-ನಾ-ದರೊ
ಮನುಷ್ಯ-ನಿ-ದ್ದರೆ
ಮನುಷ್ಯ-ನಿಂದ
ಮನುಷ್ಯ-ನಿಗೂ
ಮನುಷ್ಯ-ನಿಗೆ
ಮನುಷ್ಯ-ನೇನು
ಮನುಷ್ಯ-ನೇನೋ
ಮನುಷ್ಯ-ರ-ನ್ನು
ಮನುಷ್ಯ-ರ-ಲ್ಲ
ಮನುಷ್ಯ-ರ-ಲ್ಲಿ
ಮನುಷ್ಯ-ರಂತೆ
ಮನುಷ್ಯ-ರಾಗಿ
ಮನುಷ್ಯ-ರಿಗೂ
ಮನುಷ್ಯ-ರೆ-ಲ್ಲ
ಮನುಷ್ಯ-ರೆಂದು
ಮನುಷ್ಯನ
ಮನುಷ್ಯನು
ಮನುಷ್ಯನೆ
ಮನುಷ್ಯನೋ
ಮನುಷ್ಯರ
ಮನುಷ್ಯರೆ
ಮನುಷ್ಯರೇ
ಮನುಷ್ಯಾ-ಕೃತಿ-ಗಳು
ಮನುಷ್ಯಾಃ
ಮನುಷ್ಯೇ-ತರ
ಮನೆ
ಮನೆ-ಗ-ಳನ್ನು
ಮನೆ-ಗ-ಳಿಗೆ
ಮನೆ-ಗಳ
ಮನೆ-ಗಳ-ಲ್ಲಿ
ಮನೆ-ಗಳ-ಲ್ಲಿಯೂ
ಮನೆ-ಗಳಂತಿವೆ
ಮನೆ-ಗಳಿಂದ
ಮನೆ-ಗಳು
ಮನೆ-ಗಳೂ
ಮನೆ-ಗೆ-ಲ್ಲ
ಮನೆ-ತ-ನ-ದಿಂದ
ಮನೆ-ತನ-ಗಳಿಂದ
ಮನೆ-ದೇವ-ರಾದ
ಮನೆ-ಬಿ-ಟ್ಟು
ಮನೆ-ಮ-ನೆಗೆ
ಮನೆ-ಮಠ
ಮನೆ-ಮನೆ-ಯನ್ನು
ಮನೆ-ಮಾ-ತಾ-ಯಿತು
ಮನೆ-ಯ-ಲ್ಲಿ
ಮನೆ-ಯ-ಲ್ಲಿ-ದ್ದನು
ಮನೆ-ಯ-ಲ್ಲಿ-ದ್ದರು
ಮನೆ-ಯ-ಲ್ಲಿ-ದ್ದರೆ
ಮನೆ-ಯ-ಲ್ಲಿ-ದ್ದರೋ
ಮನೆ-ಯ-ಲ್ಲಿ-ದ್ದಾಗ
ಮನೆ-ಯ-ಲ್ಲಿ-ರಿಸಿ-ಕೊಂಡು
ಮನೆ-ಯ-ಲ್ಲಿ-ರು-ತ್ತಿದ್ದರು
ಮನೆ-ಯ-ಲ್ಲಿ-ರು-ವಿರಿ
ಮನೆ-ಯ-ಲ್ಲಿ-ರುವ
ಮನೆ-ಯ-ಲ್ಲಿ-ರುವ-ವ-ರೆ-ಲ್ಲ
ಮನೆ-ಯ-ಲ್ಲಿ-ರುವ-ವ-ರೆಗೆ
ಮನೆ-ಯ-ಲ್ಲಿ-ರುವ-ವರ
ಮನೆ-ಯ-ಲ್ಲಿ-ರುವ-ವರು
ಮನೆ-ಯ-ಲ್ಲಿದ್ದ
ಮನೆ-ಯ-ಲ್ಲಿಯೇ
ಮನೆ-ಯ-ಲ್ಲೆ
ಮನೆ-ಯ-ಲ್ಲೇ
ಮನೆ-ಯ-ವ-ರ-ನ್ನು
ಮನೆ-ಯ-ವ-ರಷ್ಟು
ಮನೆ-ಯ-ವ-ರಿಗೆ
ಮನೆ-ಯ-ವ-ರೆ-ಲ್ಲ
ಮನೆ-ಯ-ವರ
ಮನೆ-ಯ-ವರ-ನ್ನೆ-ಲ್ಲಾ
ಮನೆ-ಯ-ವರು
ಮನೆ-ಯಂತಿ-ರುವ
ಮನೆ-ಯನ್ನು
ಮನೆ-ಯಾಕೆ
ಮನೆ-ಯಾದ
ಮನೆ-ಯಿಂದ
ಮನೆ-ಯೆಂದೂ
ಮನೆ-ಯೊಂದ-ರ-ಲ್ಲಿ
ಮನೆ-ಯೊಳಗೆ
ಮನೆಯ
ಮನೆಯು
ಮನೆಯೂ
ಮನೊ-ಭಾವ
ಮನೊ-ಭಾವದ
ಮನೋ-ದೌರ್ಬ-ಲ್ಯ
ಮನೋ-ನಿಗ್ರಹ-ದಿಂದ
ಮನೋ-ಭಾವ
ಮನೋ-ಭಾವ-ನೆಯ
ಮನೋ-ಭಾವ-ವುಳ್ಳ-ವರು
ಮನೋ-ರೋಗ
ಮನೋ-ವಾಕ್ಕಾಯ-ವಾಗಿ
ಮನೋ-ವೃ-ತ್ತಿ
ಮನೋ-ಶಕ್ತಿ-ಯನ್ನು
ಮನೋ-ಹಾರಿ
ಮನೋಭಿ-ಪ್ರಾಯ-ವುಳ್ಳ-ವ-ರೊ-ಡನೆ
ಮನೋರಂ-ಜನಿಯ
ಮನೋವಕ್ಕಾಯ-ಗಳಿಂದ
ಮನೋವಿ-ಕಾರ-ವಿರ-ಬಹುದು
ಮನೋವಿ-ಕಾರವೋ
ಮನೋಹರ
ಮನೋಹರ-ವಾ-ಗಿದೆ
ಮನೋಹರ-ವಾಗಿ-ತ್ತು
ಮನೋಹರ-ವಾದ
ಮನ್ಮಥ-ನಾಥ
ಮನ್ಮಥ-ನಿಗೆ
ಮನ್ಮಥ-ಬಾಬು
ಮನ್ಮಥ-ಬಾಬು-ವಿಗೆ
ಮಮ
ಮಯ-ವಾಗಿ-ತ್ತು
ಮಯಾ
ಮಯಿ
ಮರ
ಮರ-ಕೋತಿ
ಮರ-ಗಳ
ಮರ-ಗಳ-ನ್ನೊಳ-ಗೊಂಡ
ಮರ-ಗಳಿಂದ
ಮರ-ಗಳು
ಮರ-ಗಿಡ-ಗಳ-ಲ್ಲಿ
ಮರ-ಗಿಡ-ಗಳು
ಮರ-ಣ-ಕಾಲ-ದ-ಲ್ಲಿ
ಮರ-ಣ-ಗಳ
ಮರ-ಣ-ದ-ಲ್ಲಿ
ಮರ-ಣ-ದಿಂದ
ಮರ-ಣ-ವನ್ನು
ಮರ-ಣ-ವನ್ನೈದು-ವರೋ
ಮರ-ಣಕ್ಕೆ
ಮರ-ಣದ
ಮರ-ಣಾ-ನಂ-ತರ
ಮರ-ಣೋ-ತ್ತರ
ಮರ-ದ-ಲ್ಲಿ
ಮರ-ದಿಂದ
ಮರ-ಳನ್ನು
ಮರ-ಳಿ-ದರು
ಮರ-ಳು-ಕಾಡಿ-ನ-ಲ್ಲಿ
ಮರ-ವನ್ನು
ಮರ-ವಿನ
ಮರಣ
ಮರದ
ಮರಳ
ಮರಳಿ
ಮರಳು
ಮರಿ
ಮರಿ-ಗಳು
ಮರಿ-ಯಂತೆ
ಮರಿ-ಯನ್ನು
ಮರೀಚಿಕೆ
ಮರೀಚಿಕೆ-ಯಂತೆ
ಮರು-ಕ್ಷಣ
ಮರು-ಕ್ಷಣವೇ
ಮರು-ದಿನ
ಮರು-ಪ್ರಶ್ನೆ
ಮರು-ಮರೀಚಿಕೆ-ಗಳಿ-ಲ್ಲ
ಮರು-ಮರೀಚಿಕೆ-ಗಳು
ಮರು-ಮಾ-ತನಾ-ಡದೆ
ಮರು-ಮಾ-ತಿ-ಲ್ಲದೆ
ಮರು-ವನ-ಗಳು
ಮರುಕ
ಮರುಗ-ಬೇಕು
ಮರುಗು-ವಂತೆ
ಮರುಗುವೆ
ಮರುದನಿ-ಯಾಗಿ
ಮರುದನಿಗೈ-ಯು-ತ್ತಿ-ತ್ತು
ಮರುದನಿಯ
ಮರುಳು
ಮರೆ-ತನು
ಮರೆ-ತರು
ಮರೆ-ತರೂ
ಮರೆ-ತರೆ
ಮರೆ-ತಿ-ದ್ದರು
ಮರೆ-ತಿ-ರು-ವಿರಾ
ಮರೆ-ಮಾಡಿ-ಲ್ಲ
ಮರೆ-ಮಾಡಿದ
ಮರೆ-ಯ-ದಂತೆ
ಮರೆ-ಯ-ಲಿ-ಲ್ಲ
ಮರೆ-ಯದೆ
ಮರೆ-ಯಿ-ಲ್ಲದೆ
ಮರೆ-ಯಿರಿ
ಮರೆ-ಯು-ತ್ತ
ಮರೆ-ಯು-ತ್ತಿ-ರಲಿ-ಲ್ಲ
ಮರೆ-ಯು-ತ್ತಿದ್ದೆ
ಮರೆ-ಯು-ವಂತೆ
ಮರೆ-ಯು-ವು-ದಕ್ಕೆ
ಮರೆ-ಯು-ವು-ದಿ-ಲ್ಲ
ಮರೆ-ಯು-ವುದು
ಮರೆ-ಯುವ
ಮರೆ-ಯುವನು
ಮರೆ-ಯುವುದಕ್ಕಾಗು-ವು-ದಿ-ಲ್ಲ
ಮರೆ-ಸಲು
ಮರೆ-ಸಿ-ತ್ತು
ಮರೆ-ಸುವ
ಮರೆತ
ಮರೆತಿ-ರ-ಲಿ-ಲ್ಲ
ಮರೆತಿ-ರು-ವೆವು
ಮರೆತಿ-ರುವರು
ಮರೆತು
ಮರೆತು-ಬಿ-ಟ್ಟಿ-ದ್ದನು
ಮರೆತು-ಬಿ-ಟ್ಟಿ-ರು-ವ-ರೆಂದೂ
ಮರೆತು-ಬಿ-ಟ್ಟೆ
ಮರೆತು-ಬಿಡು-ತ್ತಿದ್ದರು
ಮರೆತು-ಬಿಡು-ವಂತಹ
ಮರೆತು-ಹೋ-ಗಿದೆ
ಮರೆತು-ಹೋ-ಯಿತು
ಮರೆತೇ
ಮರೆತೇ-ಬಿ-ಟ್ಟಿದ್ದ
ಮರೆಮಾಚಿ-ಕೊಂಡು
ಮರೆಮಾಚು-ವುವು
ಮರೆವು
ಮರೆಸಿ-ಬಿ-ಟ್ಟಿ-ದ್ದವು
ಮರ್ಮಭೇದಿ-ಯಾದ
ಮರ್ಮರ
ಮರ್ಮಾಂ-ತಕ
ಮರ್ಯಾದೆ-ಗಳು
ಮರ್ಯಾದೆ-ಗಳೊ-ಡನೆ
ಮರ್ಯಾದೆ-ಯೊ-ಡನೆ
ಮರ್ಡಿಗ್ಲೇ-ಸ್
ಮಲಗಲು
ಮಲಗಿ
ಮಲಗಿ-ಕೊ-ಳ್ಳು-ತ್ತೇನೆ
ಮಲಗಿ-ಕೊಂಡ
ಮಲಗಿ-ಕೊಂಡರು
ಮಲಗಿ-ಕೊಂಡು
ಮಲಗಿ-ದ-ಮೇಲೆ
ಮಲಗಿ-ದರು
ಮಲಗಿ-ದ್ದ-ವನು
ಮಲಗಿ-ದ್ದನು
ಮಲಗಿ-ದ್ದರು
ಮಲಗಿ-ದ್ದರೋ
ಮಲಗಿ-ದ್ದಾಗ
ಮಲಗಿ-ಬಿಟರು
ಮಲಗಿ-ರು-ವ-ರೆಂದು
ಮಲಗಿ-ರು-ವುದು
ಮಲಗಿ-ರುವಾಗ
ಮಲಗಿ-ಸಿದ
ಮಲಗಿ-ಸು-ತ್ತಿ-ರು-ವನು
ಮಲಗಿ-ಸುವಾಗ
ಮಲಗಿದ್ದ
ಮಲಗು-ತ್ತಿ-ತ್ತು
ಮಲಗು-ತ್ತಿದ್ದರು
ಮಲಗು-ತ್ತಿದ್ದರೊ
ಮಲಗು-ತ್ತಿದ್ದಾರೆ
ಮಲಗು-ವರು
ಮಲಗು-ವು-ದಕ್ಕೆ
ಮಲಗು-ವುದು
ಮಲಯ
ಮಲಯ-ರ-ನ್ನು
ಮಳೆ
ಮಳೆ-ಗರೆ-ಯು-ವುವು
ಮಳೆ-ಗಾಲ
ಮಳೆ-ಗಾಲ-ದ-ಲ್ಲಿರು-ವಂತೆ
ಮಳೆ-ಯ-ಲ್ಲಿ
ಮಳೆ-ಯಿಂದ
ಮಳೆಗೆ
ಮಳೆಯ
ಮಸಾಚುಸೆ-ಟ್ಸ್
ಮಸಾಲೆ
ಮಸಿ
ಮಸೀದಿ
ಮಸೀದಿ-ಗ-ಳನ್ನು
ಮಸೀದಿ-ಗಳು
ಮಸುಕಾಗು-ತ್ತ
ಮಸುಕು
ಮಸೂ-ರಿಯ
ಮಸೂದೆಯ
ಮಹ-ತ್
ಮಹ-ತ್ಕಾರ್ಯ-ಗ-ಳನ್ನು
ಮಹ-ತ್ಕಾರ್ಯ-ವನ್ನು
ಮಹ-ತ್ತರ-ವಾ-ದುದು
ಮಹ-ತ್ತಾದ
ಮಹ-ತ್ಭಾಗ್ಯ-ದಿಂದ
ಮಹ-ತ್ವ
ಮಹ-ತ್ವ-ಪೂರ್ಣ-ವಾದ
ಮಹ-ತ್ವ-ವನ್ನು
ಮಹ-ತ್ವ-ವಾ-ದುದು
ಮಹ-ತ್ವ-ವಾಗಲಿ
ಮಹ-ದಾದರ್ಶ-ವಿದೆ
ಮಹ-ದಾನಂದ-ವನ್ನು
ಮಹ-ದೇಶ್ವರ-ನಷ್ಟೇ
ಮಹ-ದ್ಭುತಂ
ಮಹ-ಮ್ಮ-ದನು
ಮಹ-ಮ್ಮದ-ನಿ-ಲ್ಲದೆ
ಮಹ-ಮ್ಮದನ
ಮಹ-ಮ್ಮದೀ-ಯನ
ಮಹ-ಮ್ಮದೀ-ಯರ
ಮಹ-ಮ್ಮದೀ-ಯರು
ಮಹ-ಮ್ಮದೀಯ
ಮಹ-ಮ್ಮದೀಯ-ರ-ಲ್ಲಿ
ಮಹ-ಮ್ಮದೀಯ-ರಾ-ದು-ದ-ರಿಂದ
ಮಹ-ಮ್ಮದೀಯ-ರಾಗಿ-ದ್ದರು
ಮಹ-ಮ್ಮದೀಯ-ರಾಗಿ-ದ್ದರೂ
ಮಹ-ಮ್ಮದೀಯ-ರಿ-ದ್ದರು
ಮಹ-ಮ್ಮದೀಯ-ರೊಬ್ಬ-ರಿಂದ
ಮಹ-ರಾಜ್
ಮಹಂ-ತರು
ಮಹಡಿ-ಯ-ಲ್ಲಿ
ಮಹಡಿ-ಯಿಂದ
ಮಹಡಿಗೆ
ಮಹಡಿಯ
ಮಹದಾಲೋ-ಚನೆ-ಗ-ಳನ್ನು
ಮಹದಾಲೋ-ಚನೆ-ಗಳು
ಮಹದ್ವ್ಯಕ್ತಿ-ಯನ್ನು
ಮಹನೀ-ಯ-ರಿಗೆ
ಮಹಮದ್
ಮಹರ್ಷಿ
ಮಹರ್ಷಿ-ಗ-ಳಿಗೆ
ಮಹರ್ಷಿ-ಗಳಿಂದ
ಮಹರ್ಷಿ-ಗಳು
ಮಹರ್ಷಿ-ಯಾಗಲಿ
ಮಹಾ
ಮಹಾ-ಋಷಿ-ಗ-ಳಿಗೆ
ಮಹಾ-ಕಾರ್ಯ
ಮಹಾ-ಕಾರ್ಯ-ಗ-ಳನ್ನು
ಮಹಾ-ಕಾರ್ಯ-ಗಳು
ಮಹಾ-ಕಾರ್ಯ-ಗಳೂ
ಮಹಾ-ಕಾರ್ಯ-ವನ್ನು
ಮಹಾ-ಕಾರ್ಯ-ಸಾಧ-ನೆಗೆ
ಮಹಾ-ಕಾರ್ಯಕ್ಕೆ
ಮಹಾ-ಕಾರ್ಯಕ್ಕೇ
ಮಹಾ-ಕಾಳಿ
ಮಹಾ-ಕಾವ್ಯ
ಮಹಾ-ಗಣಿ-ಯಿಂದ
ಮಹಾ-ಗುರು-ಗಳಾಗ-ಬ-ಲ್ಲರು
ಮಹಾ-ಗುರು-ಗಳಿರ-ಬೇಕು
ಮಹಾ-ಜ್ಞಾನಿ-ಗಳು
ಮಹಾ-ತಪ-ಸ್ವಿ-ಗಳ
ಮಹಾ-ತಪ-ಸ್ಸು
ಮಹಾ-ತಾಯಿ
ಮಹಾ-ತೇಜ-ಸ್ವಿ-ಗಳಾಗಿ
ಮಹಾ-ತ್ಕಾರ್ಯ-ಸಾಧ-ನೆಗೆ
ಮಹಾ-ತ್ಮ
ಮಹಾ-ತ್ಮ-ನ-ಲ್ಲಿ
ಮಹಾ-ತ್ಮ-ನಾಗಿ-ಲ್ಲ
ಮಹಾ-ತ್ಮ-ನಾಗು-ವನು
ಮಹಾ-ತ್ಮ-ನಾದ
ಮಹಾ-ತ್ಮ-ನಿರ-ಬೇಕು
ಮಹಾ-ತ್ಮ-ರ-ನ್ನು
ಮಹಾ-ತ್ಮ-ರ-ನ್ನೂ
ಮಹಾ-ತ್ಮ-ರ-ಲ್ಲೊಬ್ಬರ
ಮಹಾ-ತ್ಮ-ರಿ-ಲ್ಲ
ಮಹಾ-ತ್ಮ-ರಿಗೆ
ಮಹಾ-ತ್ಮ-ರು-ಗಳ-ಲ್ಲಿ
ಮಹಾ-ತ್ಮ-ರು-ಗಳು
ಮಹಾ-ತ್ಮನ
ಮಹಾ-ತ್ಮರ
ಮಹಾ-ತ್ಮರು
ಮಹಾ-ತ್ಮೆ
ಮಹಾ-ತ್ಮೆ-ಯ-ನ್ನರಿತು
ಮಹಾ-ತ್ಮೆ-ಯನ್ನು
ಮಹಾ-ತ್ಮೆಯ
ಮಹಾ-ತ್ಮ್ಯೆ-ಯನ್ನು
ಮಹಾ-ತ್ಯಾಗ-ದಿಂದ
ಮಹಾ-ತ್ಯಾಗ-ಮಾಡು
ಮಹಾ-ತ್ಯಾಗಕ್ಕೆ
ಮಹಾ-ದುಷ್ಟ-ನಾಗು
ಮಹಾ-ದೇವ
ಮಹಾ-ದೇವನ
ಮಹಾ-ದೇವನು
ಮಹಾ-ದೋಷ
ಮಹಾ-ದ್ಭುತ
ಮಹಾ-ದ್ವಾರ-ಗಳಂತೆ
ಮಹಾ-ನಂದ-ದ-ಲ್ಲಿ
ಮಹಾ-ನಂದ-ವನ್ನು
ಮಹಾ-ನದಿ-ಗಳಿಂದ
ಮಹಾ-ನದಿ-ಗಳು
ಮಹಾ-ನೀ-ಯರು
ಮಹಾ-ನು-ಭಾ-ವನೆ
ಮಹಾ-ನು-ಭಾವ-ನಾದ
ಮಹಾ-ನ್
ಮಹಾ-ಪರ್ವತ-ದಂತೆ
ಮಹಾ-ಪವಿ-ತ್ರ-ವಾದ
ಮಹಾ-ಪಾ-ತಕ-ವಾಗಿ
ಮಹಾ-ಪಾ-ತಕವೆ
ಮಹಾ-ಪಾಪ
ಮಹಾ-ಪಾಪಿ
ಮಹಾ-ಪುರುಷ
ಮಹಾ-ಪುರುಷ-ನಾ-ದರೋ
ಮಹಾ-ಪುರುಷ-ನಾಗು-ವುದ-ರ-ಲ್ಲಿ
ಮಹಾ-ಪುರುಷ-ರ-ನ್ನು
ಮಹಾ-ಪುರುಷ-ರಾಗಲು
ಮಹಾ-ಪುರುಷ-ರಿಗೂ
ಮಹಾ-ಪುರುಷ-ರಿಗೆ
ಮಹಾ-ಪುರುಷನ
ಮಹಾ-ಪುರುಷರ
ಮಹಾ-ಪುರುಷರು
ಮಹಾ-ಪ್ರವಾಹ
ಮಹಾ-ಪ್ರವಾಹ-ದ-ಲ್ಲಿ-ರುವ
ಮಹಾ-ಪ್ರವಾಹ-ದಂತೆ
ಮಹಾ-ಪ್ರವಾಹದ
ಮಹಾ-ಪ್ರಾರಂಭ-ಗಾನ
ಮಹಾ-ಫಲ-ವನ್ನು
ಮಹಾ-ಬಲೇಶ್ವರ
ಮಹಾ-ಬಾಹೋ
ಮಹಾ-ಬೋಧಿ
ಮಹಾ-ಭ-ವ-ನ-ದ-ಲ್ಲಿ
ಮಹಾ-ಭಕ್ತ-ಳ-ವಳು
ಮಹಾ-ಭಾರ-ತ-ಗ-ಳಿಗೆ
ಮಹಾ-ಭಾರ-ತ-ಗಳ
ಮಹಾ-ಭಾರ-ತ-ಗಳ-ಲ್ಲಿ
ಮಹಾ-ಭಾರ-ತ-ದ-ಲ್ಲಿ
ಮಹಾ-ಭಾರ-ತಕ್ಕಿಂತ
ಮಹಾ-ಭಾರ-ತದ
ಮಹಾ-ಭಾರತ
ಮಹಾ-ಭಾವ-ನೆ-ಗ-ಳನ್ನು
ಮಹಾ-ಭಾವ-ನೆಯ
ಮಹಾ-ಭಾಷಣ
ಮಹಾ-ಭಾಷ್ಯ-ವನ್ನು
ಮಹಾ-ಮಾಯೆ
ಮಹಾ-ಮಾಯೆಯೇ
ಮಹಾ-ಯುಗಾ-ವತಾರ
ಮಹಾ-ಯೋಗಿ-ಗಳು
ಮಹಾ-ರಣ್ಯ-ಗಳ-ಲ್ಲಿ
ಮಹಾ-ರಾಜ
ಮಹಾ-ರಾಜ-ನನ್ನು
ಮಹಾ-ರಾಜ-ರ-ನ್ನು
ಮಹಾ-ರಾಜ-ರಿಗೆ
ಮಹಾ-ರಾಜ-ರು-ಗಳು
ಮಹಾ-ರಾಜ-ರೊ-ಡನೆ
ಮಹಾ-ರಾಜರ
ಮಹಾ-ರಾಜರು
ಮಹಾ-ರಾಜರೆ
ಮಹಾ-ರಾಜರೇ
ಮಹಾ-ರಾಜ್
ಮಹಾ-ರಾಣಿ
ಮಹಾ-ರಾಷ್ಟ್ರ
ಮಹಾ-ರಾಷ್ಟ್ರದ
ಮಹಾ-ಲೀಲೆಗೆ
ಮಹಾ-ವಾಕ್ಯ-ಗಳು
ಮಹಾ-ವಾಗ್ಮಿ-ಗಳು
ಮಹಾ-ವಾಯು
ಮಹಾ-ವಿದ್ವಾಂ-ಸರು
ಮಹಾ-ವೀರ
ಮಹಾ-ವೀರ-ನಂತೆ
ಮಹಾ-ವೀರ-ನನ್ನು
ಮಹಾ-ವೀರ-ನಾದ-ವನ
ಮಹಾ-ವ್ಯಕ್ತಿ-ಗ-ಳನ್ನು
ಮಹಾ-ವ್ಯಕ್ತಿ-ಗ-ಳಿಗೆ
ಮಹಾ-ವ್ಯಕ್ತಿ-ಗಳ
ಮಹಾ-ವ್ಯಕ್ತಿ-ಗಳು
ಮಹಾ-ವ್ಯಕ್ತಿ-ಗಳೇ
ಮಹಾ-ಶ-ಯನು
ಮಹಾ-ಶ-ಯರೆ
ಮಹಾ-ಶಕ್ತಿ
ಮಹಾ-ಶಕ್ತಿ-ಯಾಗುವನು
ಮಹಾ-ಶೂ-ನ್ಯ-ದ-ಲ್ಲಿ
ಮಹಾ-ಶೂ-ರರು
ಮಹಾ-ಸ-ತ್ಯ-ವನ್ನು
ಮಹಾ-ಸ-ತ್ಯದ
ಮಹಾ-ಸ-ಭೆಯ
ಮಹಾ-ಸ-ಮನ್ವಯ
ಮಹಾ-ಸಂ-ಸ್ಥೆ-ಯನ್ನು
ಮಹಾ-ಸಭೆ
ಮಹಾ-ಸಭೆ-ಯ-ಲ್ಲಿ
ಮಹಾ-ಸಮಾಧಿ
ಮಹಾ-ಸಮಾಧಿ-ಯನ್ನು
ಮಹಾ-ಸಮಾಧಿಗೆ
ಮಹಾ-ಸಮಾಧಿಯ
ಮಹಾ-ಸಾ-ಗರ
ಮಹಾ-ಸಾ-ಗರ-ದ-ಲ್ಲಿ
ಮಹಾ-ಸಾ-ಗರವೇ
ಮಹಾ-ಸಾಹಸ-ದ-ಲ್ಲಿ
ಮಹಾ-ಸುಧಾರಕ-ರಾಗುವಂತೆ
ಮಹಾ-ಸೇವೆ-ಯ-ಲ್ಲಿ
ಮಹಾ-ಸ್ತ್ರ-ದಂತೆ
ಮಹಾ-ಸ್ತ್ರವೇ
ಮಹಿಳೆ
ಮಹಿಳೆ-ಯ-ರಿಗೆ
ಮಹಿಳೆ-ಯ-ಲ್ಲಿ
ಮಹಿಳೆ-ಯರ
ಮಹಿಳೆ-ಯರು
ಮಹಿಳೆಗೆ
ಮಹೇಂದ್ರ
ಮಹೇಂದ್ರ-ನಾಥ
ಮಹೇಂದ್ರ-ನಾಥರು
ಮಹೇಂದ್ರ-ನಿಗೂ
ಮಹೇಂದ್ರ-ನಿಗೆ
ಮಹೇಂದ್ರನ
ಮಹೇಶ
ಮಹೇಶ-ಚಂದ್ರ
ಮಹೇಶ್ವ-ರರು
ಮಹೇಶ್ವರ
ಮಹೇಶ್ವರಃ
ಮಹೇಶ್ವರನ
ಮಹೋ-ತ್ಸವ
ಮಹೋ-ನ್ನತ
ಮಾ
ಮಾಂ
ಮಾಂಸ
ಮಾಕ್ಲಿಯಾಡ್
ಮಾಕ್ಸಿ-ಮ್
ಮಾಗು-ವು-ದಕ್ಕೆ
ಮಾಘ
ಮಾಟ
ಮಾಟ-ಗಾತಿ-ಯ-ರ-ನ್ನು
ಮಾಡ-ಕೂ-ಡದು
ಮಾಡ-ಕೂಡ-ದೆಂದು
ಮಾಡ-ಗೊಡಿ-ಸದೆ
ಮಾಡ-ತಕ್ಕ
ಮಾಡ-ತೊಡಗಿದ
ಮಾಡ-ತೊಡಗಿ-ದನು
ಮಾಡ-ತೊಡಗಿ-ದರು
ಮಾಡ-ಬ-ಲ್ಲ
ಮಾಡ-ಬ-ಲ್ಲ-ವ-ರಾಗಿ-ದ್ದರು
ಮಾಡ-ಬ-ಲ್ಲಂತಹ
ಮಾಡ-ಬ-ಲ್ಲರು
ಮಾಡ-ಬ-ಲ್ಲರೊ
ಮಾಡ-ಬ-ಲ್ಲರೋ
ಮಾಡ-ಬ-ಲ್ಲಿರಿ
ಮಾಡ-ಬ-ಲ್ಲೆ
ಮಾಡ-ಬ-ಲ್ಲೆವು
ಮಾಡ-ಬಯ-ಸುವೆ
ಮಾಡ-ಬಹುದು
ಮಾಡ-ಬಹುದೆ
ಮಾಡ-ಬಾ-ರದು
ಮಾಡ-ಬೇ-ಕಾ-ದರೂ
ಮಾಡ-ಬೇ-ಕಾ-ದರೆ
ಮಾಡ-ಬೇ-ಕಾ-ದುದೇ-ನೆಂಬುದ-ನ್ನೆ-ಲ್ಲ
ಮಾಡ-ಬೇ-ಕಾದ
ಮಾಡ-ಬೇಕಾಗಿ
ಮಾಡ-ಬೇಕಾಗಿ-ತ್ತು
ಮಾಡ-ಬೇಕಾಗಿ-ರ-ಲಿ-ಲ್ಲ
ಮಾಡ-ಬೇಕಾಗಿ-ರುವ
ಮಾಡ-ಬೇಕಾಗಿ-ಲ್ಲ
ಮಾಡ-ಬೇಕಾಗಿದೆ
ಮಾಡ-ಬೇಕಾಗುವುದು
ಮಾಡ-ಬೇಕು
ಮಾಡ-ಬೇಕೆಂ-ದಿ-ರು-ವು-ದ-ನ್ನು
ಮಾಡ-ಬೇಕೆಂ-ದಿ-ರು-ವುದು
ಮಾಡ-ಬೇಕೆಂ-ದಿ-ರುವರು
ಮಾಡ-ಬೇಕೆಂ-ದಿ-ರುವೆ
ಮಾಡ-ಬೇಕೆಂ-ದಿ-ರುವೆನು
ಮಾಡ-ಬೇಕೆಂದು
ಮಾಡ-ಬೇಕೆಂದೂ
ಮಾಡ-ಬೇಕೆಂದೂ-ಮಾಡ-ಬೇಕೆಂಬ
ಮಾಡ-ಬೇಕೆಂಬು-ದ-ನ್ನು
ಮಾಡ-ಬೇಕೆಂಬು-ದೊಂದು
ಮಾಡ-ಬೇಕೊ
ಮಾಡ-ಬೇಕೋ
ಮಾಡ-ಬೇಡ
ಮಾಡ-ಬೇಡವೆ
ಮಾಡ-ಬೇಡಿ
ಮಾಡ-ಲಾ-ರದು
ಮಾಡ-ಲಾಗು-ವು-ದಿ-ಲ್ಲ
ಮಾಡ-ಲಾಗು-ವು-ದಿ-ಲ್ಲ-ವೆಂದೂ
ಮಾಡ-ಲೋಸುಗ
ಮಾಡ-ಲ್ಪಟ್ಟಿತು
ಮಾಡ-ಲ್ಪಟ್ಟಿದೆ
ಮಾಡ-ಲ್ಪಟ್ಟಿವೆ
ಮಾಡ-ವುದು
ಮಾಡಲ-ನರ್ಹವಾ-ಗು-ತ್ತಿದೆ
ಮಾಡಲಾ-ಯಿತು
ಮಾಡಲಾ-ರದೆ
ಮಾಡಲಾ-ರರು
ಮಾಡಲಾ-ರರೊ
ಮಾಡಲಾ-ರವು
ಮಾಡಲಾಗ-ಲಿ-ಲ್ಲ
ಮಾಡಲಾದೀತೆ
ಮಾಡಲಾರಂಭಿ-ಸಿ-ದರು
ಮಾಡಲಾರೆ
ಮಾಡಲಿ
ಮಾಡಲಿ-ರುವ
ಮಾಡಲಿ-ಲ್ಲ
ಮಾಡಲು
ಮಾಡಲೂ
ಮಾಡಲೆ
ಮಾಡಲೆ-ತ್ನಿ-ಸಿ-ದರು
ಮಾಡಲೇ
ಮಾಡಲೇ-ಬೇಕಾಗುವುದು
ಮಾಡಲೇ-ಬೇಕು
ಮಾಡಿ
ಮಾಡಿ-ಕೊ-ಟ್ಟನು
ಮಾಡಿ-ಕೊ-ಟ್ಟರು
ಮಾಡಿ-ಕೊ-ಟ್ಟು
ಮಾಡಿ-ಕೊ-ಳ್ಳಲಿ
ಮಾಡಿ-ಕೊ-ಳ್ಳಲು
ಮಾಡಿ-ಕೊ-ಳ್ಳು-ತ್ತಾರೋ
ಮಾಡಿ-ಕೊ-ಳ್ಳು-ವು-ದಕ್ಕೆ
ಮಾಡಿ-ಕೊ-ಳ್ಳುವ
ಮಾಡಿ-ಕೊ-ಳ್ಳುವ-ವರು
ಮಾಡಿ-ಕೊ-ಳ್ಳುವರು
ಮಾಡಿ-ಕೊ-ಳ್ಳುವರೆ
ಮಾಡಿ-ಕೊ-ಳ್ಳುವಿರಿ
ಮಾಡಿ-ಕೊ-ಳ್ಳುವುದಕ್ಕಿಂತ
ಮಾಡಿ-ಕೊ-ಳ್ಳುವುದು
ಮಾಡಿ-ಕೊ-ಳ್ಳುವುದೇ
ಮಾಡಿ-ಕೊಂಡ
ಮಾಡಿ-ಕೊಂಡ-ಳೆಂಬ
ಮಾಡಿ-ಕೊಂಡಂತೆ
ಮಾಡಿ-ಕೊಂಡನು
ಮಾಡಿ-ಕೊಂಡನೋ
ಮಾಡಿ-ಕೊಂಡರು
ಮಾಡಿ-ಕೊಂಡರೆ
ಮಾಡಿ-ಕೊಂಡಿ-ರುವನು
ಮಾಡಿ-ಕೊಂಡಿ-ರುವೆ
ಮಾಡಿ-ಕೊಂಡಿ-ರುವೆನು
ಮಾಡಿ-ಕೊಂಡಿದ್ದ
ಮಾಡಿ-ಕೊಂಡಿದ್ದರು
ಮಾಡಿ-ಕೊಂಡಿದ್ದುವು
ಮಾಡಿ-ಕೊಂಡಿದ್ದೆ
ಮಾಡಿ-ಕೊಂಡಿದ್ದೆವೊ
ಮಾಡಿ-ಕೊಂಡಿದ್ದೇನೆಂ-ದರೆ
ಮಾಡಿ-ಕೊಂಡು
ಮಾಡಿ-ಕೊಂಡೆವು
ಮಾಡಿ-ಕೊಡಲು
ಮಾಡಿ-ಕೊಡಿ-ದ್ದರೆ
ಮಾಡಿ-ಕೊಡು-ತ್ತಾ-ರಂತೆ
ಮಾಡಿ-ಕೊಡು-ತ್ತಿದ್ದರು
ಮಾಡಿ-ಕೊಡುವ
ಮಾಡಿ-ಕೊಡೆಂದು
ಮಾಡಿ-ಕೊಳ್ಳ-ಬ-ಲ್ಲ-ವ-ಳಾಗಿ-ದ್ದಳು
ಮಾಡಿ-ಕೊಳ್ಳ-ಬಹು-ದೆಂದು
ಮಾಡಿ-ಕೊಳ್ಳ-ಬಹುದು
ಮಾಡಿ-ಕೊಳ್ಳ-ಬೇ-ಕಾ-ದರೆ
ಮಾಡಿ-ಕೊಳ್ಳ-ಬೇಕಾಗಿ-ಲ್ಲ-ವೆಂಬು-ದ-ನ್ನು
ಮಾಡಿ-ಕೊಳ್ಳ-ಬೇಕಾಗಿದೆ
ಮಾಡಿ-ಕೊಳ್ಳ-ಬೇಕಾಗುವ
ಮಾಡಿ-ಕೊಳ್ಳ-ಬೇಕು
ಮಾಡಿ-ಕೊಳ್ಳ-ಬೇಕೆ
ಮಾಡಿ-ಕೊಳ್ಳ-ಬೇಕೆಂ-ತಲೂ
ಮಾಡಿ-ಕೊಳ್ಳ-ಲಾಗು-ವು-ದಿ-ಲ್ಲ
ಮಾಡಿ-ಕೊಳ್ಳಿ
ಮಾಡಿ-ಕೊಳ್ಳು-ವು-ದಾಗಿದೆ
ಮಾಡಿ-ದ-ಮೇಲೆ
ಮಾಡಿ-ದ-ರೆಂದು
ಮಾಡಿ-ದ-ರೆಂಬುದು
ಮಾಡಿ-ದ-ವ-ರ-ಲ್ಲ
ಮಾಡಿ-ದ-ವ-ರಿಗೆ
ಮಾಡಿ-ದ-ವ-ರೆ-ಲ್ಲರೂ
ಮಾಡಿ-ದ-ವನು
ಮಾಡಿ-ದ-ವರು
ಮಾಡಿ-ದಂತೆ
ಮಾಡಿ-ದನು
ಮಾಡಿ-ದನೊ
ಮಾಡಿ-ದನೋ
ಮಾಡಿ-ದರು
ಮಾಡಿ-ದರೂ
ಮಾಡಿ-ದರೆ
ಮಾಡಿ-ದರೇ
ಮಾಡಿ-ದಳು
ಮಾಡಿ-ದವು
ಮಾಡಿ-ದಷ್ಟೂ
ಮಾಡಿ-ದಾಗ
ಮಾಡಿ-ದಾಗಲೂ
ಮಾಡಿ-ದಾದ
ಮಾಡಿ-ದಿರಿ
ಮಾಡಿ-ದು-ದ-ಕ್ಕಾಗಿ
ಮಾಡಿ-ದು-ದ-ನ್ನು
ಮಾಡಿ-ದು-ದ-ರಿಂದ
ಮಾಡಿ-ದು-ದ-ಲ್ಲ
ಮಾಡಿ-ದು-ದರ
ಮಾಡಿ-ದುದು
ಮಾಡಿ-ದೆನು
ಮಾಡಿ-ದೆಯೆ
ಮಾಡಿ-ದೆವು
ಮಾಡಿ-ದೆವೊ
ಮಾಡಿ-ದೆವೋ
ಮಾಡಿ-ದೊ-ಡನೆ
ಮಾಡಿ-ದೊ-ಡನೆಯೇ
ಮಾಡಿ-ದ್ದ-ಕ್ಕಾಗಿ
ಮಾಡಿ-ದ್ದರು
ಮಾಡಿ-ದ್ದರೆ
ಮಾಡಿ-ದ್ದಳು
ಮಾಡಿ-ದ್ದಾಗಿ-ತ್ತು
ಮಾಡಿ-ದ್ದಾರೆ
ಮಾಡಿ-ದ್ದಾರೆಯೆ
ಮಾಡಿ-ದ್ದೆವು
ಮಾಡಿ-ದ್ದೇನೆ
ಮಾಡಿ-ದ್ದೇವೆ
ಮಾಡಿ-ಬಂದು
ಮಾಡಿ-ಬಿ-ಟ್ಟ
ಮಾಡಿ-ಬಿ-ಟ್ಟರೆ
ಮಾಡಿ-ಬಿ-ಡು-ತ್ತದೆ
ಮಾಡಿ-ಬಿ-ಡು-ವುದು
ಮಾಡಿ-ಬಿಡ-ಬಹುದು
ಮಾಡಿ-ಬಿಡಿ
ಮಾಡಿ-ಬಿಡು-ತ್ತಿದ್ದರು
ಮಾಡಿ-ಬಿಡು-ತ್ತಿದ್ದೆ
ಮಾಡಿ-ಬಿಡು-ವುದೋ
ಮಾಡಿ-ಬಿಡು-ವೆವು
ಮಾಡಿ-ರ-ಬೇಕು
ಮಾಡಿ-ರ-ಬೇಕೆಂದು
ಮಾಡಿ-ರ-ಲಾ-ರದು
ಮಾಡಿ-ರ-ಲಿ-ಲ್ಲ
ಮಾಡಿ-ರು-ತ್ತಾರೆ
ಮಾಡಿ-ರು-ವನು
ಮಾಡಿ-ರು-ವಿರಾ
ಮಾಡಿ-ರು-ವಿರಿ
ಮಾಡಿ-ರು-ವುದು
ಮಾಡಿ-ರು-ವುದೇ
ಮಾಡಿ-ರು-ವೆನು
ಮಾಡಿ-ರು-ವೆವು
ಮಾಡಿ-ರುವ
ಮಾಡಿ-ರುವರು
ಮಾಡಿ-ರುವರೋ
ಮಾಡಿ-ರುವೆ
ಮಾಡಿ-ರುವೆ-ನೆಂದೂ
ಮಾಡಿ-ರುವೆಯಾ
ಮಾಡಿ-ರೆಂದು
ಮಾಡಿ-ಲ್ಲ
ಮಾಡಿ-ಲ್ಲವೆ
ಮಾಡಿ-ಸ-ಲಿ-ಲ್ಲ
ಮಾಡಿ-ಸಲಾರವು
ಮಾಡಿ-ಸಲು
ಮಾಡಿ-ಸಿ-ಕೊಂಡರು
ಮಾಡಿ-ಸಿ-ಕೊಡು
ಮಾಡಿ-ಸಿ-ದರು
ಮಾಡಿ-ಸಿ-ದರೆ
ಮಾಡಿ-ಸಿ-ದಳು
ಮಾಡಿ-ಸಿ-ದಾಗ
ಮಾಡಿ-ಸಿದ
ಮಾಡಿ-ಸು-ತ್ತ
ಮಾಡಿ-ಸು-ತ್ತಾರೆ
ಮಾಡಿ-ಸು-ತ್ತಿ-ದ್ದಾನೆ
ಮಾಡಿ-ಸು-ತ್ತಿದ್ದರು
ಮಾಡಿ-ಸು-ತ್ತಿದ್ದಾರೆ
ಮಾಡಿ-ಸು-ತ್ತೇನೆ
ಮಾಡಿ-ಸು-ವು-ದ-ಕ್ಕಾಗಿ
ಮಾಡಿ-ಸುತೇನೆ
ಮಾಡಿ-ಸೋಣ
ಮಾಡಿತು
ಮಾಡಿದ
ಮಾಡಿದೆ
ಮಾಡಿದ್ದ
ಮಾಡಿದ್ದು
ಮಾಡಿಯೇ
ಮಾಡಿವೆ
ಮಾಡಿಸಿ
ಮಾಡೀತು
ಮಾಡು
ಮಾಡು-ತ್ತಲೂ
ಮಾಡು-ತ್ತಲೇ
ಮಾಡು-ತ್ತಾ-ರಂತೆ
ಮಾಡು-ತ್ತಾ-ರೆಯೊ
ಮಾಡು-ತ್ತಾ-ರೆಯೋ
ಮಾಡು-ತ್ತಾನೆ
ಮಾಡು-ತ್ತಾರೆ
ಮಾಡು-ತ್ತಾರೊ
ಮಾಡು-ತ್ತಿ-ತ್ತು
ಮಾಡು-ತ್ತಿ-ದ್ದು-ದ-ನ್ನು
ಮಾಡು-ತ್ತಿ-ರಲಿ
ಮಾಡು-ತ್ತಿ-ರಲಿ-ಲ್ಲ
ಮಾಡು-ತ್ತಿ-ರು-ವಿರಿ
ಮಾಡು-ತ್ತಿ-ರು-ವು-ದ-ನ್ನು
ಮಾಡು-ತ್ತಿ-ಲ್ಲ
ಮಾಡು-ತ್ತಿದೆ
ಮಾಡು-ತ್ತಿದ್ದ
ಮಾಡು-ತ್ತಿದ್ದ-ನೇನು
ಮಾಡು-ತ್ತಿದ್ದ-ರೆಂದು
ಮಾಡು-ತ್ತಿದ್ದ-ವ-ರಿಗೆ
ಮಾಡು-ತ್ತಿದ್ದ-ವರು
ಮಾಡು-ತ್ತಿದ್ದಂತೆ
ಮಾಡು-ತ್ತಿದ್ದನು
ಮಾಡು-ತ್ತಿದ್ದರು
ಮಾಡು-ತ್ತಿದ್ದರೂ
ಮಾಡು-ತ್ತಿದ್ದರೆ
ಮಾಡು-ತ್ತಿದ್ದಳು
ಮಾಡು-ತ್ತಿದ್ದಾಗ
ಮಾಡು-ತ್ತಿದ್ದಾಗಲೂ
ಮಾಡು-ತ್ತಿದ್ದೀರಿ
ಮಾಡು-ತ್ತಿದ್ದು-ದ-ಲ್ಲದೆ
ಮಾಡು-ತ್ತಿದ್ದುದು
ಮಾಡು-ತ್ತಿದ್ದೆ
ಮಾಡು-ತ್ತಿದ್ದೆವು
ಮಾಡು-ತ್ತಿದ್ದೇನೆ
ಮಾಡು-ತ್ತಿರ-ಬಹು-ದೆಂದು
ಮಾಡು-ತ್ತಿರ-ಬಹುದು
ಮಾಡು-ತ್ತಿರು-ವ-ನೆಂದೂ
ಮಾಡು-ತ್ತಿರು-ವ-ರೆಂದು
ಮಾಡು-ತ್ತಿರು-ವ-ವ-ರೆ-ಲ್ಲ
ಮಾಡು-ತ್ತಿರು-ವನು
ಮಾಡು-ತ್ತಿರು-ವರು
ಮಾಡು-ತ್ತಿರು-ವರೋ
ಮಾಡು-ತ್ತಿರು-ವಾಗ
ಮಾಡು-ತ್ತಿರು-ವಾಗಲೇ
ಮಾಡು-ತ್ತಿರು-ವುದು
ಮಾಡು-ತ್ತಿರು-ವುದೊ
ಮಾಡು-ತ್ತಿರು-ವೆ-ನೆಂದೂ
ಮಾಡು-ತ್ತಿರು-ವೆನು
ಮಾಡು-ತ್ತಿರುವ
ಮಾಡು-ತ್ತಿರುವೆ
ಮಾಡು-ತ್ತಿವೆ
ಮಾಡು-ತ್ತಿವೆಯೋ
ಮಾಡು-ತ್ತೀರಿ
ಮಾಡು-ತ್ತೇನೆ
ಮಾಡು-ತ್ತೇವೆ
ಮಾಡು-ತ್ತೇವೆಂದೂ
ಮಾಡು-ವ-ನೆಂದು
ಮಾಡು-ವ-ನೆಂಬು-ದ-ನ್ನು
ಮಾಡು-ವ-ರೆಂಬುದು
ಮಾಡು-ವ-ವ-ರಿಗೆ
ಮಾಡು-ವ-ವನು
ಮಾಡು-ವ-ವರಿ-ದ್ದರೇ
ಮಾಡು-ವ-ವರು
ಮಾಡು-ವ-ವಳು
ಮಾಡು-ವಂತಹ
ಮಾಡು-ವಂತೆ
ಮಾಡು-ವಂತೆಯೂ
ಮಾಡು-ವನು
ಮಾಡು-ವನೋ
ಮಾಡು-ವರು
ಮಾಡು-ವರೋ
ಮಾಡು-ವಳು
ಮಾಡು-ವಷ್ಟು
ಮಾಡು-ವಾಗ
ಮಾಡು-ವಿ-ರ-ಲ್ಲ
ಮಾಡು-ವಿ-ರೇನು
ಮಾಡು-ವಿಯೋ
ಮಾಡು-ವಿರಿ
ಮಾಡು-ವು-ದ-ಕ್ಕಾಗಿ
ಮಾಡು-ವು-ದ-ನ್ನು
ಮಾಡು-ವು-ದ-ರ-ಲ್ಲಿ
ಮಾಡು-ವು-ದ-ರಿಂದ
ಮಾಡು-ವು-ದ-ರಿಂದಲೂ
ಮಾಡು-ವು-ದ-ಲ್ಲ
ಮಾಡು-ವು-ದಕ್ಕೂ
ಮಾಡು-ವು-ದಕ್ಕೆ
ಮಾಡು-ವು-ದರ
ಮಾಡು-ವು-ದಾಗಿ
ಮಾಡು-ವು-ದಾಗಿದೆ
ಮಾಡು-ವು-ದಾಗಿಯೂ
ಮಾಡು-ವು-ದಿ-ರಲಿ
ಮಾಡು-ವು-ದಿ-ಲ್ಲ
ಮಾಡು-ವು-ದಿ-ಲ್ಲ-ವ-ಲ್ಲ
ಮಾಡು-ವು-ದಿ-ಲ್ಲ-ವೆಂ-ದರು
ಮಾಡು-ವು-ದಿ-ಲ್ಲ-ವೆಂದು
ಮಾಡು-ವು-ದಿ-ಲ್ಲವೋ
ಮಾಡು-ವು-ದೆಂದು
ಮಾಡು-ವು-ದೊಂದು
ಮಾಡು-ವುದ-ಕ್ಕೆ-ಮಾಡು-ವುದಕ್ಕೇ
ಮಾಡು-ವುದಕ್ಕಾಗು-ವು-ದಿ-ಲ್ಲ-ವೆಂದೂ
ಮಾಡು-ವುದಕ್ಕಿಂತ
ಮಾಡು-ವುದು
ಮಾಡು-ವುದೆಂ-ದರೆ
ಮಾಡು-ವುದೇ
ಮಾಡು-ವುದೋ
ಮಾಡು-ವುವು
ಮಾಡು-ವುವೆ
ಮಾಡು-ವೆ-ನೆಂದು
ಮಾಡು-ವೆ-ನೆಂದೂ
ಮಾಡು-ವೆನು
ಮಾಡು-ವೆವು
ಮಾಡು-ವೆವೋ
ಮಾಡುವ
ಮಾಡುವೆ
ಮಾಡೇ
ಮಾಡೋಣ
ಮಾತ-ಗ-ಳನ್ನು
ಮಾತ-ನಾ-ಡು-ತ್ತ
ಮಾತ-ನಾ-ಡು-ತ್ತಾ
ಮಾತ-ನಾ-ಡುವ
ಮಾತ-ನಾಡು-ತ್ತದೆ
ಮಾತ-ನಾಡು-ತ್ತಲೇ
ಮಾತ-ನಾಡು-ತ್ತವೆ
ಮಾತ-ನಾಡು-ತ್ತಾನೆ
ಮಾತ-ನಾಡು-ತ್ತಾರೆ
ಮಾತ-ನಾಡು-ತ್ತಿ-ದ್ದು-ದ-ನ್ನು
ಮಾತ-ನಾಡು-ತ್ತಿ-ರಲಿ-ಲ್ಲ
ಮಾತ-ನಾಡು-ತ್ತಿದ್ದ
ಮಾತ-ನಾಡು-ತ್ತಿದ್ದ-ವನು
ಮಾತ-ನಾಡು-ತ್ತಿದ್ದಂತೆ
ಮಾತ-ನಾಡು-ತ್ತಿದ್ದನು
ಮಾತ-ನಾಡು-ತ್ತಿದ್ದರು
ಮಾತ-ನಾಡು-ತ್ತಿದ್ದಳು
ಮಾತ-ನಾಡು-ತ್ತಿದ್ದಾಗ
ಮಾತ-ನಾಡು-ತ್ತಿದ್ದಿರಿ
ಮಾತ-ನಾಡು-ತ್ತಿದ್ದು-ದೆ-ಲ್ಲ
ಮಾತ-ನಾಡು-ತ್ತಿದ್ದುದು
ಮಾತ-ನಾಡು-ತ್ತಿದ್ದೆ
ಮಾತ-ನಾಡು-ತ್ತಿದ್ದೆವು
ಮಾತ-ನಾಡು-ತ್ತಿರು
ಮಾತ-ನಾಡು-ತ್ತಿರು-ವಂತೆ
ಮಾತ-ನಾಡು-ತ್ತಿರು-ವಾಗಲೂ
ಮಾತ-ನಾಡು-ತ್ತಿರು-ವಿರಿ
ಮಾತ-ನಾಡು-ತ್ತಿರು-ವುದು
ಮಾತ-ನಾಡು-ತ್ತಿರುವೆ
ಮಾತ-ನಾಡು-ತ್ತೀಯಾ
ಮಾತ-ನಾಡು-ತ್ತೀರಿ
ಮಾತ-ನಾಡು-ತ್ತೇನೆಂದು
ಮಾತ-ನಾಡು-ವಂತೆ
ಮಾತ-ನಾಡು-ವರು
ಮಾತ-ನಾಡು-ವಾಗ
ಮಾತ-ನಾಡು-ವಾಗಲೂ
ಮಾತ-ನಾಡು-ವು-ದ-ನ್ನು
ಮಾತ-ನಾಡು-ವು-ದ-ಲ್ಲ
ಮಾತ-ನಾಡು-ವು-ದಕ್ಕೆ
ಮಾತ-ನಾಡು-ವು-ದರೊಳ-ಗಾಗಿಯೇ
ಮಾತ-ನಾಡು-ವು-ದಿ-ಲ್ಲ
ಮಾತ-ನಾಡು-ವುದ-ರ-ಲ್ಲಿ
ಮಾತ-ನಾಡು-ವುದಕ್ಕಿಂತ
ಮಾತ-ನಾಡು-ವುದು
ಮಾತ-ಲ್ಲ
ಮಾತ-ಲ್ಲದೆ
ಮಾತಂತೂ
ಮಾತಾ-ಡುವಾಗ
ಮಾತಾಡಲು
ಮಾತಾಡು-ವಿರಿ
ಮಾತಿ-ಗಿಂತ
ಮಾತಿ-ನ-ಲ್ಲಿ
ಮಾತಿ-ನಿಂದಲೇ
ಮಾತು
ಮಾತು-ಕತೆ
ಮಾತು-ಕತೆ-ಗ-ಳನ್ನು
ಮಾತು-ಕತೆ-ಗ-ಳಾದ
ಮಾತು-ಕತೆ-ಗಳ-ನ್ನಾಡಲು
ಮಾತು-ಕತೆ-ಗಳ-ಲ್ಲಿ
ಮಾತು-ಕತೆ-ಗಳಾಡಿದ
ಮಾತು-ಕತೆ-ಗಳಾಡಿದರು
ಮಾತು-ಕತೆ-ಗಳೂ
ಮಾತು-ಕತೆ-ಯ-ಲ್ಲದೆ
ಮಾತು-ಕತೆ-ಯನ್ನು
ಮಾತು-ಕತೆ-ಯಾ-ಡು-ತ್ತಿದ್ದರು
ಮಾತು-ಕತೆ-ಯಾ-ಡು-ತ್ತೇನೆ
ಮಾತು-ಕತೆ-ಯಾ-ಡು-ವುದು
ಮಾತು-ಕತೆ-ಯಾಡ-ಬೇಕೆಂದು
ಮಾತು-ಕತೆ-ಯಾಡಲು
ಮಾತು-ಕತೆ-ಯಾಡಿ-ದರು
ಮಾತು-ಕತೆ-ಯಾದ
ಮಾತು-ಕಥೆ
ಮಾತು-ಕೊಡು
ಮಾತು-ಕೊಡುವ
ಮಾತು-ಗ-ಳನ್ನು
ಮಾತು-ಗ-ಳಾದರೋ
ಮಾತು-ಗಳ-ನ್ನಾ-ಡು-ತ್ತಾ
ಮಾತು-ಗಳ-ನ್ನೆ-ಲ್ಲ
ಮಾತು-ಗಳ-ಲ್ಲಿ
ಮಾತು-ಗಳಾಗಿ-ರ-ಲಿ-ಲ್ಲ
ಮಾತು-ಗಳಿಂದ
ಮಾತು-ಗಳು
ಮಾತು-ಗಾರಿಕೆ
ಮಾತೂ
ಮಾತೃ
ಮಾತೃ-ತ್ವದ
ಮಾತೃ-ಪ್ರೇಮದ
ಮಾತೃ-ಭಾಷೆ-ಯಾದ
ಮಾತೃ-ಭೂಮಿ-ಯ-ನ್ನಾಗಲಿ
ಮಾತೃ-ಭೂಮಿ-ಯ-ಲ್ಲಿ
ಮಾತೃ-ಭೂಮಿ-ಯನ್ನು
ಮಾತೃ-ಭೂಮಿ-ಯಿಂದ
ಮಾತೃ-ಭೂಮಿ-ಯೆಂಬ
ಮಾತೃ-ಭೂಮಿಯೆ
ಮಾತೆ
ಮಾತೆ-ಯನ್ನು
ಮಾತೆಗೆ
ಮಾತೆಯ
ಮಾತೆಯು
ಮಾತೆಯೂ
ಮಾತೇ
ಮಾದ-ಬಹುದು
ಮಾದಿ-ದರೆ
ಮಾಧುರ್ಯ
ಮಾಧುರ್ಯ-ದಿಂದ
ಮಾಧುರ್ಯ-ವನ್ನು
ಮಾಧ್ವ-ರಂತಹ
ಮಾನ-ದಂಡ
ಮಾನ-ದಂಡಃ
ಮಾನ-ಭಂಗ
ಮಾನ-ವ-ಕುಲ-ಶಾ-ಸ್ತ್ರ
ಮಾನ-ವ-ಕೋಟಿ
ಮಾನ-ವ-ಕೋಟಿ-ಯನ್ನು
ಮಾನ-ವ-ಕೋಟಿಗೆ
ಮಾನ-ವ-ಕೋಟಿಯ
ಮಾನ-ವ-ತೆಯ
ಮಾನ-ವ-ತ್ವ
ಮಾನ-ವ-ನ-ಲ್ಲಿ
ಮಾನ-ವ-ನ-ಲ್ಲಿಯೇ
ಮಾನ-ವ-ನನ್ನು
ಮಾನ-ವ-ನಿ-ಗಿಂತ
ಮಾನ-ವ-ನಿಗೂ
ಮಾನ-ವ-ನಿಗೆ
ಮಾನ-ವ-ಪ್ರೇಮ
ಮಾನ-ವ-ರ-ನ್ನಾಗಿ
ಮಾನ-ವ-ರ-ನ್ನು
ಮಾನ-ವ-ರಿ-ಗಿಂತ
ಮಾನ-ವ-ರಿಗೆ
ಮಾನ-ವ-ರೆ-ಲ್ಲ
ಮಾನ-ವ-ರೆಂಬು-ದ-ನ್ನು
ಮಾನ-ವನ
ಮಾನ-ವನು
ಮಾನ-ವನೇ
ಮಾನ-ವರ
ಮಾನ-ವರು
ಮಾನ-ವಾ-ಕಾರ
ಮಾನ-ವಾ-ತೀತ
ಮಾನ-ವೇ-ತರ
ಮಾನ-ಸಚಕ್ಷು-ಗಳಿಂದ
ಮಾನ-ಸಿಕ
ಮಾನ-ಸಿಕ-ವಾಗಿ
ಮಾನ-ಸಿಕ-ವಾದುವು
ಮಾನದ
ಮಾನವ
ಮಾನುಷ
ಮಾನ್ಯ
ಮಾನ್ಯ-ತೆ-ಯನ್ನು
ಮಾನ್ಯಿ-ಯರ್
ಮಾನ್ಕ್ಯೂರ್ಡಿಕಾ-ನ್ಯೆ
ಮಾಫಿ
ಮಾಯ
ಮಾಯ-ಮಂ-ತ್ರ
ಮಾಯ-ವಾ-ದವು
ಮಾಯ-ವಾ-ದಾಗ
ಮಾಯ-ವಾ-ಯಿತು
ಮಾಯ-ವಾಗ-ತೊಡಗಿದವು
ಮಾಯ-ವಾಗದೆ
ಮಾಯ-ವಾಗಿ
ಮಾಯ-ವಾಗಿ-ತ್ತು
ಮಾಯ-ವಾಗಿವೆ
ಮಾಯ-ವಾಗು-ತ್ತದೆ
ಮಾಯ-ವಾಗು-ತ್ತಿ-ತ್ತು
ಮಾಯ-ವಾಗು-ತ್ತಿವೆ
ಮಾಯ-ವಾಗು-ವುದು
ಮಾಯ-ವಾಗು-ವುವು
ಮಾಯ-ವಾದ
ಮಾಯಾ
ಮಾಯಾ-ಜಾಲ-ವನ್ನು
ಮಾಯಾ-ತೀತ
ಮಾಯಾ-ದಂಡ-ಗ-ಳನ್ನು
ಮಾಯಾ-ದೀಪ
ಮಾಯಾ-ದೇವಿಯ
ಮಾಯಾ-ಪ್ರಪಂಚ
ಮಾಯಾ-ಬಂ-ಧನ-ವನ್ನೆ-ಲ್ಲ
ಮಾಯಾ-ಮಂ-ತ್ರ-ವನ್ನು
ಮಾಯಾ-ರಾ-ಹಿತ್ಯ
ಮಾಯಾ-ರಾಜ್ಯ-ದ-ಲ್ಲಿವೆ
ಮಾಯಾ-ವ-ತಿಗೆ
ಮಾಯಾ-ವ-ರದ
ಮಾಯಾ-ವತಿ
ಮಾಯಾ-ವತಿ-ಯ-ಲ್ಲಿ
ಮಾಯಾ-ವತಿ-ಯನ್ನು
ಮಾಯಾ-ವತಿಯ
ಮಾಯಾ-ವರ-ಣಕ್ಕೆ
ಮಾಯಾ-ವರ-ಣದ
ಮಾಯಾ-ವಾ-ದಿ-ಗಳು
ಮಾಯಾ-ಶಕ್ತಿ
ಮಾಯಾ-ಸಿದ್ಧಾಂತದ
ಮಾಯೆ
ಮಾಯೆ-ಗಳೆಂ-ದರೆ
ಮಾಯೆ-ಯ-ಲ್ಲಿ
ಮಾಯೆಯ
ಮಾಯೆಯೂ
ಮಾಯೆಯೇ
ಮಾರ-ನೆಂ-ದರೆ
ಮಾರ-ನೆಯ
ಮಾರಿ
ಮಾರಿ-ಕೊಂಡಿದ್ದಾರೆ
ಮಾರು
ಮಾರು-ತ್ತರ-ವನ್ನು
ಮಾರು-ತ್ತಿದ್ದ
ಮಾರು-ತ್ತಿದ್ದರು
ಮಾರು-ಹೋಗಿ
ಮಾರು-ಹೋಗಿ-ತ್ತು
ಮಾರು-ಹೋಗಿ-ದ್ದನು
ಮಾರು-ಹೋಗಿ-ರುವೆ
ಮಾರು-ಹೋದರು
ಮಾರ್ಗ
ಮಾರ್ಗ-ಕ್ಕಾಗಿ
ಮಾರ್ಗ-ಗಳ-ನ್ನೆ-ಲ್ಲಾ
ಮಾರ್ಗ-ಗಳ-ಲ್ಲಿ
ಮಾರ್ಗ-ಗಳು
ಮಾರ್ಗ-ದ-ಲ್ಲಿ
ಮಾರ್ಗ-ದರ್ಶಕ-ನನ್ನು
ಮಾರ್ಗ-ದರ್ಶಕನೂ
ಮಾರ್ಗ-ದರ್ಶನ
ಮಾರ್ಗ-ರೆ-ಟ್
ಮಾರ್ಗ-ರೇ-ಟ್
ಮಾರ್ಗ-ವ-ನ್ನಾ-ದರೂ
ಮಾರ್ಗ-ವನ್ನು
ಮಾರ್ಗ-ವನ್ನೇ
ಮಾರ್ಗ-ವಾಗಿ
ಮಾರ್ಗ-ವಿ-ದೆಯೆ
ಮಾರ್ಗ-ವಿ-ಲ್ಲ
ಮಾರ್ಗ-ವಿರ-ಬೇಕೆಂಬ
ಮಾರ್ಗ-ವೆಂದು
ಮಾರ್ಗ-ವೇಕೆ
ಮಾರ್ಗ-ವೊಂದೆ
ಮಾರ್ಗಕ್ಕೆ
ಮಾರ್ಗದ
ಮಾರ್ಗವೇ
ಮಾರ್ಚಿ
ಮಾರ್ಚ್
ಮಾರ್ತಾಂಡ
ಮಾರ್ತಾಂಡ-ವರ್ಮ-ರಿಗೆ
ಮಾರ್ಪಟ್ಟಿ-ರು-ತ್ತವೆ
ಮಾರ್ಪಡಿ-ಸಲಾಗದ
ಮಾರ್ಪಡಿಸು
ಮಾರ್ಮಿಕ-ವಾದ
ಮಾರ್ವಾಡಿ
ಮಾರ್ಶ್ಮ್ಯಾನ್
ಮಾರ್ಸೆ-ಲ್ಸ್
ಮಾಲಿಕೆ-ಯ-ಲ್ಲಿ
ಮಾಲೀ-ಕನು
ಮಾಲೆ-ಗ-ಳನ್ನು
ಮಾವ
ಮಾವು
ಮಾಸ-ಪ-ತ್ರಿಕೆ
ಮಾಸ-ಪ-ತ್ರಿಕೆ-ಗಳು
ಮಾಸ-ಪ-ತ್ರಿಕೆ-ಯ-ಲ್ಲಿ
ಮಾಸ-ಪ-ತ್ರಿಕೆಗೆ
ಮಿ
ಮಿಂ
ಮಿಂಚಿ-ದರೆ
ಮಿಂಚಿ-ನಂತೆ
ಮಿಂಚು-ಗಳು
ಮಿಂಚೊಂದು
ಮಿಂದು
ಮಿಕ್ಕ
ಮಿಕ್ಕ-ಕೆಲಸ-ವನ್ನು
ಮಿಕ್ಕ-ವ-ರ-ನ್ನು
ಮಿಕ್ಕ-ವ-ರ-ಲ್ಲಿ
ಮಿಕ್ಕ-ವರೂ
ಮಿಕ್ಕರೆ
ಮಿಕ್ಕಿ-ರು-ವು-ದೆ-ಲ್ಲ
ಮಿಗಿಲಾಗಿ
ಮಿಗಿಲಾಗಿ-ರು-ವುದು
ಮಿಗಿಲಾದ
ಮಿಗುವ
ಮಿಟುಕಿ-ಸುವುದ-ರ-ಲ್ಲಿ
ಮಿಠಾಯಿ
ಮಿಠಾಯಿ-ಗ-ಳನ್ನು
ಮಿಠಾಯಿ-ಯನ್ನು
ಮಿಡಿ-ದಿ-ರು-ವಿರಿ
ಮಿಡಿಯು-ವಂತಹ
ಮಿತಾ-ಹಾರಿ-ಯಾಗಿ
ಮಿತಿ
ಮಿತಿ-ಗಳಿವೆ
ಮಿತಿ-ಮೀರಿ
ಮಿತಿ-ಯನ್ನು
ಮಿಥ್ಯಕ್ಕಿಂತ
ಮಿಥ್ಯಾ
ಮಿಥ್ಯಾ-ಚಾರ
ಮಿಥ್ಯಾ-ಪ್ರಪಂಚಕ್ಕೆ
ಮಿಥ್ಯೆ
ಮಿನಿಯಪೋಲೀ-ಸ್
ಮಿನಿಯಪೋಲೀ-ಸ್ನ-ಲ್ಲಿ-ರುವ
ಮಿರುಗು-ತ್ತಿ-ತ್ತು
ಮಿಲನ-ವಿದೆ
ಮಿಲಾ-ನ್
ಮಿಳಿತ-ವಾಗಿ-ರುವ
ಮಿಶಿ-ನ್ಗ-ನ್ನನ್ನು
ಮಿಶಿನ-ರಿಗೆ
ಮಿಶಿನೆರಿ-ಯಷ್ಟೇ
ಮಿಶ್ರ-ದಂತೆ
ಮಿಶ್ರಣ-ಗಳಿಂದಲೂ
ಮಿಶ್ರಣ-ವಾ-ಗಿದೆ
ಮಿಷ-ನರಿ
ಮಿಷ-ನರಿ-ಗಳು
ಮಿಷ-ನರಿ-ಯಾಗಲಿ
ಮಿಷ-ನ್
ಮಿಷ-ನ್ನನ್ನು
ಮಿಷ-ನ್ನಿನ
ಮಿಷನ-ರಿಯ
ಮೀ
ಮೀನಾ-ಕ್ಷಿಯ
ಮೀನಿ-ನಂತೆ
ಮೀನು
ಮೀರ-ತ್
ಮೀರ-ತ್ತಿ-ನ-ಲ್ಲಿ
ಮೀರ-ತ್ತಿ-ನಿಂದ
ಮೀರ-ತ್ತಿಗೆ
ಮೀರ-ಬ-ಲ್ಲದು
ಮೀರ-ಬಹು-ದೆಂದು
ಮೀರಿ
ಮೀರಿ-ದ-ವ-ರಿಗೆ
ಮೀರಿ-ದ-ವ-ರೊಬ್ಬರು
ಮೀರಿ-ದಾಗ
ಮೀರಿ-ದು-ದ-ನ್ನು
ಮೀರಿ-ರು-ವು-ದ-ನ್ನು
ಮೀರಿ-ಸು-ವುದು
ಮೀರಿ-ಸುವ-ವರ
ಮೀರಿ-ಸುವ-ವರು
ಮೀರಿ-ಹೋಗಿ
ಮೀರಿ-ಹೋಗು-ವೆಯೋ
ಮೀರಿ-ಹೋಗುವ
ಮೀರಿತು
ಮೀರಿದ
ಮೀರಿಸಿ
ಮೀಸ-ಲಾಗಿ-ತ್ತು
ಮೀಸ-ಲಾಗಿದ್ದ
ಮು
ಮುಂ
ಮುಂಚಿ-ತ-ವಾಗಿ
ಮುಂಚಿ-ನಂತೆಯೇ
ಮುಂಚೆ
ಮುಂಚೆಯೆ
ಮುಂಚೆಯೇ
ಮುಂತಾ-ದವು
ಮುಂದಾಲೋ-ಚನೆ-ಯಾಗಲಿ
ಮುಂದಾಳಾಗಿ
ಮುಂದಾಳಾಗಿ-ರು-ವು-ದ-ನ್ನು
ಮುಂದಾಳು
ಮುಂದಾಳು-ಗ-ಳನ್ನು
ಮುಂದಾಳು-ಗಳು
ಮುಂದಾಳು-ವನ್ನು
ಮುಂದಿಟಿ-ಲ್ಲ
ಮುಂದಿಡಿ
ಮುಂದು
ಮುಂದು-ಗಡೆ
ಮುಂದು-ವ-ರಿ-ಸು-ವು-ದಕ್ಕೆ
ಮುಂದು-ವ-ರಿಯ-ಬೇ-ಕಾ-ದರೆ
ಮುಂದು-ವ-ರಿಯ-ಬೇಕೆಂದು
ಮುಂದು-ವ-ರಿಯ-ಬೇಕೆಂಬ
ಮುಂದು-ವ-ರಿಯ-ಬೇಕೆಂಬು-ದ-ನ್ನು
ಮುಂದು-ವ-ರಿಯ-ಲಾಗ-ಲಿ-ಲ್ಲ
ಮುಂದು-ವ-ರಿಯು-ವುದೊ
ಮುಂದು-ವ-ರಿಸಿ
ಮುಂದು-ವ-ರಿಸಿ-ಕೊಂಡು
ಮುಂದು-ವ-ರಿಸಿ-ದರು
ಮುಂದು-ವ-ರಿಸಿ-ರುವರೋ
ಮುಂದು-ವರಿ-ದಂತೆ
ಮುಂದು-ವರಿ-ದಂತೆ-ಲ್ಲ
ಮುಂದು-ವರಿ-ದಷ್ಟೂ
ಮುಂದು-ವರಿ-ದಿ-ರುವರು
ಮುಂದು-ವರಿ-ಯಿತು
ಮುಂದು-ವರಿ-ಯು-ತ್ತ
ಮುಂದು-ವರಿ-ಯು-ತ್ತದೆ
ಮುಂದು-ವರಿ-ಯು-ತ್ತಿದೆ
ಮುಂದು-ವರಿ-ಯು-ತ್ತಿದ್ದನು
ಮುಂದು-ವರಿ-ಯು-ತ್ತಿವೆ
ಮುಂದು-ವರಿ-ಯು-ವಂತೆ
ಮುಂದು-ವರಿ-ಯು-ವಿರಿ
ಮುಂದು-ವರಿ-ಯು-ವು-ದಕ್ಕೆ
ಮುಂದು-ವರಿ-ಯು-ವುದು
ಮುಂದು-ವರಿ-ಯುವರು
ಮುಂದು-ವರಿ-ಸು-ವಂತೆ
ಮುಂದು-ವರಿ-ಸು-ವು-ದ-ನ್ನು
ಮುಂದು-ವರಿದ
ಮುಂದು-ವರಿದ-ವರು
ಮುಂದು-ವರಿದನು
ಮುಂದು-ವರಿದರು
ಮುಂದು-ವರಿದವು
ಮುಂದು-ವರಿದಿ-ರು-ವು-ದಕ್ಕೆ
ಮುಂದು-ವರಿದಿವೆ
ಮುಂದು-ವರಿದು
ಮುಂದು-ವರಿಸ-ತೊಡಗಿದರು
ಮುಂದು-ವರಿಸ-ಬಾ-ರದು
ಮುಂದು-ವರಿಸ-ಬೇಕೆಂದು
ಮುಂದು-ವರಿಸು-ವಿರಾ
ಮುಂದು-ವರೆ-ಸು-ತ್ತಾರೆ
ಮುಂದೆ
ಮುಂದೆ-ಹೋದರು
ಮುಂದೆಯೂ
ಮುಂದೆಯೇ
ಮುಂದೇ-ನಾಗು-ವುದು
ಮುಂದೇನು
ಮುಕ್ಕಳಿ-ಸು-ವುದೇ
ಮುಕ್ಕಳಿಸು-ವುದೆ
ಮುಕ್ಕಾಲು
ಮುಕ್ಕಾಲು-ಪಾಲು
ಮುಕ್ತ
ಮುಕ್ತ-ಕಂಠ-ದಿಂದ
ಮುಕ್ತ-ನಾಗಿ-ಬಿಡುವೆ
ಮುಕ್ತ-ಪದ್ಮಾ-ಸ-ನ-ದ-ಲ್ಲಿ
ಮುಕ್ತ-ಪುರುಷ-ರಾದ
ಮುಕ್ತ-ಪುರುಷರು
ಮುಕ್ತ-ರಾ-ದ-ವ-ರ-ಲ್ಲ
ಮುಕ್ತ-ರಾ-ದಂತಹ
ಮುಕ್ತ-ರಾಗಿ
ಮುಕ್ತಾ-ತ್ಮ
ಮುಕ್ತಾ-ತ್ಮ-ರಾಗುವುದು
ಮುಕ್ತಾ-ತ್ಮ-ರು-ಗಳು
ಮುಕ್ತಾ-ತ್ಮರೂ
ಮುಕ್ತಾಯ
ಮುಕ್ತಾಯ-ಗೊಂಡ
ಮುಕ್ತಾಯ-ಗೊಂಡು
ಮುಕ್ತಾಯ-ಗೊಳಿಸಿ-ದರು
ಮುಕ್ತಾಯ-ಮಾಡಿ-ದಾಗ
ಮುಕ್ತಾಯ-ವಾ-ಯಿತೆಂದು
ಮುಕ್ತಾಯ-ವಾಗು-ವ-ವ-ರೆಗೆ
ಮುಕ್ತಿ
ಮುಕ್ತಿ-ಕೊಡಲು
ಮುಕ್ತಿ-ಗಾಗಿ
ಮುಕ್ತಿ-ಗಾಗಿಯೂ
ಮುಕ್ತಿ-ಯ-ಲ್ಲಿ
ಮುಕ್ತಿ-ಯ-ಲ್ಲಿ-ಲ್ಲ
ಮುಕ್ತಿ-ಯನ್ನು
ಮುಕ್ತಿ-ಯಾ-ದ-ಲ್ಲದೆ
ಮುಕ್ತಿ-ಯಾಗುವ-ವ-ರೆಗೂ
ಮುಕ್ತಿ-ಯಿ-ಲ್ಲ
ಮುಕ್ತಿಗೂ
ಮುಕ್ತಿಗೆ
ಮುಕ್ತಿಯ
ಮುಖ
ಮುಖ-ಗಳಂತೆ
ಮುಖ-ಗಳು
ಮುಖ-ದ-ಲ್ಲಿ
ಮುಖ-ದಿಂದ
ಮುಖ-ಭಾವ
ಮುಖ-ಮಂಡಲ
ಮುಖ-ಮಾಡಿ
ಮುಖ-ರಾಗ-ಬೇಕು
ಮುಖ-ರಿತ-ವಾ-ಗು-ತ್ತಿದೆ
ಮುಖ-ರಿತ-ವಾಗು-ತ್ತಿ-ತ್ತು
ಮುಖ-ವನ್ನು
ಮುಖ-ವನ್ನೇ
ಮುಖ-ವೆ-ಲ್ಲ
ಮುಖಕ್ಕೆ
ಮುಖತಃ
ಮುಖದ
ಮುಖರ್ಜಿ
ಮುಖರ್ಜಿಯ
ಮುಖಾ-ರವಿಂದ
ಮುಖಾಂ-ತರ
ಮುಖೇನ
ಮುಖ್ಯ
ಮುಖ್ಯ-ಕೇಂದ್ರ
ಮುಖ್ಯ-ರಾದ-ವ-ರ-ಲ್ಲಿ
ಮುಖ್ಯ-ವಾ-ಗಿದೆ
ಮುಖ್ಯ-ವಾ-ದುದು
ಮುಖ್ಯ-ವಾ-ದುದೇ-ನೆಂ-ದರೆ
ಮುಖ್ಯ-ವಾಗಿ
ಮುಖ್ಯ-ವಾಗಿ-ತ್ತೊ
ಮುಖ್ಯ-ವಾದ
ಮುಖ್ಯ-ವಾದುವು
ಮುಖ್ಯ-ವೆಂದು
ಮುಖ್ಯ-ಸ್ಥ
ಮುಖ್ಯ-ಸ್ಥ-ರ-ನ್ನಾಗಿ
ಮುಖ್ಯ-ಸ್ಥ-ರಾದ
ಮುಖ್ಯ-ಸ್ಥ-ರೆ-ಲ್ಲ
ಮುಖ್ಯೋಪಾಧ್ಯಾಯ
ಮುಗಿ-ದಿ-ರ-ಲಿ-ಲ್ಲ
ಮುಗಿ-ಯದೆ
ಮುಗಿ-ಯಿತು
ಮುಗಿ-ಯು-ತ್ತದೆ
ಮುಗಿ-ಯು-ವು-ದಿ-ಲ್ಲ
ಮುಗಿ-ಯುವ
ಮುಗಿ-ಯುವ-ವ-ರೆಗೆ
ಮುಗಿ-ಸಿದ
ಮುಗಿದ
ಮುಗಿದ-ಮೇಲೆ
ಮುಗಿದಂತೆ
ಮುಗಿದರೆ
ಮುಗಿದವು
ಮುಗಿದು
ಮುಗಿದು-ಹೋಗಿವೆ
ಮುಗಿಲಿ-ನಂತೆ
ಮುಗಿಸಿ
ಮುಗಿಸಿ-ಕೊಂಡು
ಮುಗಿಸಿ-ದರು
ಮುಗಿಸಿ-ಬಿ-ಟ್ಟಿ-ದ್ದರು
ಮುಗ್ಧ-ನಾಗಿ
ಮುಗ್ಧ-ರ-ನ್ನಾಗಿ
ಮುಗ್ಧ-ರಾಗಿ
ಮುಗ್ಧಬೋಧ-ವೆಂಬ
ಮುಚ್ಚಿ
ಮುಚ್ಚಿ-ಕೊಂಡಂತೆ
ಮುಚ್ಚಿ-ಕೊಂಡಳು
ಮುಚ್ಚಿ-ಕೊಂಡಿ-ರುವ
ಮುಚ್ಚಿ-ಕೊಂಡು
ಮುಚ್ಚಿ-ಕೊಳ್ಳಿ
ಮುಚ್ಚಿ-ದರು
ಮುಚ್ಚಿ-ದರೂ
ಮುಚ್ಚಿ-ಬಿಡು-ವಂತೇ
ಮುಚ್ಚಿ-ಹೋಗಿ-ತ್ತು
ಮುಚ್ಚಿದ
ಮುಚ್ಚಿದೆ
ಮುಚ್ಚು
ಮುಚ್ಚು-ಮರೆ-ಯಿ-ಲ್ಲದೆ
ಮುಚ್ಚು-ವುದು
ಮುಡುಪಾ-ಯಿತು
ಮುದು-ಕ-ರಿಗೆ
ಮುದು-ಕರು
ಮುದು-ಕರೇ
ಮುದುಕ-ನಾದ
ಮುದುಕ-ನಿಗೆ
ಮುದುಕ-ನೊಬ್ಬನು
ಮುದುಕಿ-ಯೊಬ್ಬಳು
ಮುದುಕಿಯ
ಮುದೆ
ಮುದ್ದಿಸು-ವಳು
ಮುದ್ದು
ಮುದ್ದೆ-ಯಂತೆ
ಮುದ್ದೆ-ಯನ್ನು
ಮುದ್ರಣ
ಮುದ್ರಣಾಲಯ-ದ-ಲ್ಲಿ
ಮುದ್ರಾಲಯ-ವನ್ನು
ಮುದ್ರಿಸ-ಬೇಕೆಂದು
ಮುದ್ರೆ
ಮುನಿಯ
ಮುನಿಸಿ-ಕೊಂಡಂತೆ
ಮುನಿಸಿಪ-ಲ್
ಮುನುಗು-ತ್ತಿ-ರು-ವು-ದ-ನ್ನು
ಮುರಿ-ಯು-ವಂತೆ
ಮುರಿದ
ಮುರಿದು
ಮುರಿದು-ಬಿ-ತ್ತು
ಮುರ್ರಿ-ಯ-ಲ್ಲಿ
ಮುರ್ರಿ-ಯಿಂದ
ಮುರ್ರಿಗೆ
ಮುರ್ರಿಯ
ಮುರ್ರಿಯ-ಲ್ಲಿ-ದ್ದಾಗ
ಮುರ್ಷಿದಾಬಾದಿ-ನ-ಲ್ಲಿ
ಮುರ್ಷಿದಾಬಾದಿ-ನಿಂದ
ಮುರ್ಷಿದಾಬಾದ್
ಮುಲಕ
ಮುಳು-ಗು-ತ್ತಿದೆ
ಮುಳುಗಿ
ಮುಳುಗಿ-ದರೆ
ಮುಳುಗಿ-ದೆ-ನೆಂ-ದರೆ
ಮುಳುಗಿ-ದ್ದರೆ
ಮುಳುಗಿ-ದ್ದೆಯೋ
ಮುಳುಗಿ-ರುವ
ಮುಳುಗಿ-ರುವ-ವ-ರ-ನ್ನು
ಮುಳುಗಿ-ಸಿ-ಬಿ-ಟ್ಟಿತು
ಮುಳುಗಿ-ಸು-ವಂತೆ
ಮುಳುಗಿ-ಸು-ವಷ್ಟು
ಮುಳುಗಿ-ಹೋ-ಗಿದೆ
ಮುಳುಗಿ-ಹೋಗಿ-ದ್ದರು
ಮುಳುಗಿ-ಹೋಗಿದ್ದ
ಮುಳುಗಿದ್ದ
ಮುಳುಗು-ತ್ತಾನೆ
ಮುಳುಗು-ತ್ತಿ-ರು-ವು-ದ-ನ್ನು
ಮುಳುಗು-ತ್ತಿದ್ದರು
ಮುಳುಗು-ತ್ತಿರುವ
ಮುಳುಗುವ
ಮುಷ್ಟಿ
ಮುಸ-ಲ್ಮಾನ
ಮುಸ-ಲ್ಮಾನರು
ಮುಸ-ಲ್ಮಾನಳು
ಮುಸ-ಲ್ಮಾನಿ
ಮುಸ-ಲ್ಮಾನ್
ಮುಸು-ಕನ್ನು
ಮುಸು-ಕಿ-ರುವ
ಮುಸುಡಿ-ಯನ್ನು
ಮೂ
ಮೂಕ
ಮೂಕ-ನಂ-ತಾದ
ಮೂಕ-ನಂ-ತಾದನು
ಮೂಕ-ನನ್ನು
ಮೂಗಿನ
ಮೂಟೆ-ಕ-ಟ್ಟಿ
ಮೂಟೆ-ಯನ್ನು
ಮೂಡಲಿ
ಮೂಡಿ-ದುದು
ಮೂಡಿ-ದ್ದರೆ
ಮೂಡಿತು
ಮೂಡಿದ
ಮೂಡಿದಂ-ತಿ-ತ್ತು
ಮೂಡಿದೊಡ-ನೆಯೆ
ಮೂಡಿದೊಡ-ನೆಯೇ
ಮೂಡಿರ-ಬೇಕು
ಮೂಡು-ತ್ತಿ-ತ್ತು
ಮೂಡು-ತ್ತಿರುವ
ಮೂಡು-ವಂತೆ
ಮೂಡು-ವು-ದಿ-ಲ್ಲವೆ
ಮೂಢ
ಮೂಢ-ನಂಬಿ-ಕೆ-ಗ-ಳನ್ನು
ಮೂಢ-ನಂಬಿ-ಕೆ-ಗಳ-ಲ್ಲಿ
ಮೂಢ-ನಂಬಿ-ಕೆ-ಗಳಿಂದ
ಮೂಢ-ನಂಬಿ-ಕೆ-ಗಳಿಂದಲೂ
ಮೂಢ-ನಂಬಿಕೆ
ಮೂಢ-ರ-ನ್ನು
ಮೂಢ-ವಾದ
ಮೂಢನ
ಮೂರು
ಮೂರು-ಗಂಟೆ
ಮೂರು-ಗಂಟೆ-ಗಳು
ಮೂರು-ಬಾರಿ
ಮೂರ್ಖ
ಮೂರ್ಖ-ತನ-ವೆಂದು
ಮೂರ್ಖ-ರ-ನ್ನು
ಮೂರ್ಖ-ರ-ಲ್ಲ
ಮೂರ್ಖ-ರಿಗೆ
ಮೂರ್ಖತೆ
ಮೂರ್ಖರ
ಮೂರ್ಖರು
ಮೂರ್ಛೆ-ಗೊಂಡ
ಮೂರ್ತಿ
ಮೂರ್ತಿ-ಗಳ
ಮೂರ್ತಿ-ಗಳು
ಮೂರ್ತಿ-ಭವಿ-ಸಿದ್ದಾನೆ
ಮೂರ್ತಿ-ಯಂತೆ
ಮೂರ್ತಿ-ಯನ್ನು
ಮೂರ್ತಿ-ಯನ್ನೇ
ಮೂರ್ತಿ-ಯಾಗಿ
ಮೂರ್ತಿ-ಯಾದ
ಮೂರ್ತಿ-ವತ್ತಾಗಿ
ಮೂರ್ತಿ-ವೆ-ತ್ತಂತೆ
ಮೂರ್ತಿಯ
ಮೂರ್ತಿಯೋ
ಮೂಲ
ಮೂಲ-ಕ-ವ-ಲ್ಲದೆ
ಮೂಲ-ಕ-ವಾಗಿ
ಮೂಲ-ಕ-ವಾಗಿಯೂ
ಮೂಲ-ಕ-ವಾಗಿಯೇ
ಮೂಲ-ಕವೂ
ಮೂಲ-ಕವೊ
ಮೂಲ-ಕಾರ-ಣ-ವನ್ನು
ಮೂಲ-ದ-ಲ್ಲಿ
ಮೂಲ-ದ-ಲ್ಲಿ-ರು-ವುದು
ಮೂಲ-ದಿಂದ
ಮೂಲ-ಧನ
ಮೂಲ-ಧನ-ವನ್ನು
ಮೂಲ-ಪುರುಷ-ರಾ-ದರು
ಮೂಲ-ಮಂ-ತ್ರ-ವಾಗಲಿ
ಮೂಲ-ವನ್ನು
ಮೂಲ-ವನ್ನೇ
ಮೂಲ-ವಾದ
ಮೂಲ-ವೆ-ಲ್ಲ
ಮೂಲ-ಶಾ-ಸ್ತ್ರ-ಗ-ಳನ್ನು
ಮೂಲ-ಸ್ಥಾನ-ವಾಗಿ
ಮೂಲ-ಸ್ಫೂರ್ತಿ
ಮೂಲಕ
ಮೂಲಕ್ಕೆ
ಮೂಲದ
ಮೂಲವೂ
ಮೂಲವೆ
ಮೂಲವೋ
ಮೂಲಾಧಾರ
ಮೂಲಾಧಾರ-ದ-ಲ್ಲಿರುವ
ಮೂಲಿ-ಕೆಯ
ಮೂಲೆ
ಮೂಲೆ-ಗಳ-ಲ್ಲಿ
ಮೂಲೆ-ಗೊ-ತ್ತು-ವುದ-ರ-ಲ್ಲಿ
ಮೂಲೆ-ಯ-ಲ್ಲಿ
ಮೂಲೆ-ಯಿಂದ
ಮೂಲೆ-ಯೊಂ-ದ-ರಿಂದ
ಮೂಲೆಗೆ
ಮೂಲೆಯ
ಮೂಲ್ಚಂದ್
ಮೂಳೆ
ಮೂಳೆ-ಗಳ
ಮೂಳೆ-ಗಳು
ಮೂಳೆ-ಗಳೆ-ಲ್ಲ
ಮೂಳೆ-ಗಳೇ
ಮೂಳೆ-ಮಾಂ-ಸ-ವನ್ನು
ಮೂಳೆ-ಯನ್ನು
ಮೃ
ಮೃಗ
ಮೃಗ-ಗಳು
ಮೃಗ-ಪಕ್ಷಿ-ಗಳೂ
ಮೃಗ-ವಾಗಿ
ಮೃಗ-ವಾಗು-ವುದು
ಮೃಗಕ್ಕೆ
ಮೃಗೀಯತೆ-ಯನ್ನು
ಮೃತ-ಗತ-ಭಾರ-ತದ
ಮೃತ-ದೇಹ-ವನ್ನು
ಮೃದಂಗ
ಮೃದಂಗ-ವನ್ನು
ಮೃದಂಗ-ವಾದನ
ಮೃದಂಗದ
ಮೃದು-ಮಧುರ
ಮೃದು-ವಾ-ಗಿದೆ
ಮೃದು-ವಾಗಿ
ಮೃದು-ವಾಗಿ-ತ್ತು
ಮೃದು-ವಾಗಿ-ದ್ದವು
ಮೃದು-ವಾದ
ಮೃದು-ಸ್ವರ-ದಿಂದ
ಮೃದು-ಹೃದಯ-ವನ್ನು
ಮೃಷ್ಟಾ-ನ್ನ
ಮೆ
ಮೆಂಪಿ-ಸ್ಗೆ
ಮೆಂಪಿ-ಸ್ನ
ಮೆಕ್ಕ
ಮೆಚ್ಚ-ತೊಡಗಿದರು
ಮೆಚ್ಚ-ಲಿ-ಲ್ಲ
ಮೆಚ್ಚಿ
ಮೆಚ್ಚಿ-ಗೆ-ಯಾಗಿ
ಮೆಚ್ಚಿ-ಗೆಗೆ
ಮೆಚ್ಚಿ-ದ-ವರು
ಮೆಚ್ಚಿ-ದನು
ಮೆಚ್ಚಿ-ದರು
ಮೆಚ್ಚಿ-ದಳು
ಮೆಚ್ಚಿ-ದವ-ರೆಂ-ದರೆ
ಮೆಚ್ಚಿ-ರುವರು
ಮೆಚ್ಚಿಗೆ
ಮೆಚ್ಚು-ಗಾರ-ನಾಗಿದ್ದ
ಮೆಚ್ಚು-ಗೆಯ
ಮೆಚ್ಚು-ತ್ತ
ಮೆಚ್ಚು-ತ್ತಿ-ತ್ತು
ಮೆಚ್ಚು-ತ್ತಿ-ರಲಿ-ಲ್ಲ
ಮೆಚ್ಚು-ತ್ತಿದ್ದರು
ಮೆಚ್ಚು-ತ್ತೀರಾ
ಮೆಚ್ಚು-ತ್ತೇನೆ
ಮೆಚ್ಚು-ವಂತ-ಹ-ವರು
ಮೆಚ್ಚು-ವರು
ಮೆಚ್ಚು-ವು-ದಿ-ಲ್ಲ
ಮೆಚ್ಚು-ವುದು
ಮೆಚ್ಚು-ವುದೋ
ಮೆಚ್ಚುಗೆ-ಯನ್ನು
ಮೆಚ್ಚುಗೆ-ಯನ್ನೂ
ಮೆಚ್ಚುವ
ಮೆಚ್ಚುವ-ರೆಂದೂ
ಮೆಟಫಿಸಿ-ಕ-ಲ್
ಮೆಡಿಟರೇನಿ-ಯನ್
ಮೆಣ-ಸಿನ
ಮೆದು-ಳನ್ನೂ
ಮೆದು-ಳಿಗೆ
ಮೆದು-ಳಿದೆ
ಮೆದುಳಿ-ದೆಯೇ
ಮೆದುಳಿ-ನ-ಲ್ಲಿ
ಮೆದುಳಿನ
ಮೆದುಳು
ಮೆನೆ
ಮೆರೆ-ಯು-ತ್ತಿದ್ದವು
ಮೆರೆ-ಯು-ವುದೇನೂ
ಮೆರೆ-ಸ-ಲಿ-ಲ್ಲ
ಮೆರೆದಾಡಲಿ
ಮೆಲಕು
ಮೆಲಕು-ತ್ತ
ಮೆಲುಕು
ಮೆಲುಕು-ತ್ತಿರ-ಬೇಡ
ಮೆಲುಕು-ತ್ತಿರ-ಬೇಡಿ
ಮೆಲುದನಿ-ಯ-ಲ್ಲಿ
ಮೆಲೆ
ಮೆಹ-ದಿಂದ
ಮೆಹಬೂಬ್
ಮೇ
ಮೇಕಳಿ
ಮೇಕ್ಲಿಯಾಡ್
ಮೇಘ
ಮೇಘ-ನಾದ-ವಧ
ಮೇಘ-ಮಾಲೆ-ಯಂತೆ
ಮೇಘ-ವನ್ನೂ
ಮೇಜಿನ
ಮೇಡ-ಮ್
ಮೇಡಂ
ಮೇಣ-ದಿಂದ
ಮೇಧಾವಿ
ಮೇಧಾವಿ-ಗ-ಳಾದ
ಮೇಧಾವಿ-ಗಳಿಂದ
ಮೇಧಾವಿ-ಗಳು
ಮೇಧಾವಿಯೂ
ಮೇರಿ
ಮೇರು
ಮೇರು-ಸದೃಶ
ಮೇರೆ
ಮೇರೆ-ಯನ್ನು
ಮೇರೆ-ಯಿ-ರ-ಲಿ-ಲ್ಲ
ಮೇರೆ-ಯಿ-ಲ್ಲದ
ಮೇರೆಗೆ
ಮೇರೆಯೇ
ಮೇಲ-ಲ್ಲ
ಮೇಲಕೆ-ತ್ತು-ವುದೋ
ಮೇಲಕ್ಕೆ
ಮೇಲಣ
ಮೇಲಾಗಲೀ
ಮೇಲಾಗಿರ-ಬೇಕಾಗಿ-ತ್ತು
ಮೇಲಿ-ಟ್ಟರು
ಮೇಲಿ-ಟ್ಟು
ಮೇಲಿ-ತ್ತು
ಮೇಲಿ-ನಂತಹ
ಮೇಲಿ-ರು-ವು-ದ-ನ್ನು
ಮೇಲಿ-ರು-ವುದೋ
ಮೇಲು
ಮೇಲು-ಕೀ-ಳೆಂಬ
ಮೇಲು-ಗಡೆ-ಯಿಂದ
ಮೇಲೂ
ಮೇಲೆ
ಮೇಲೆ-ತ್ತ-ಬ-ಲ್ಲಿರಾ
ಮೇಲೆ-ತ್ತ-ಬೇ-ಕಾ-ದರೆ
ಮೇಲೆ-ತ್ತಲು
ಮೇಲೆ-ತ್ತಿ
ಮೇಲೆ-ತ್ತಿತು
ಮೇಲೆ-ತ್ತು-ತ್ತಿದ್ದರು
ಮೇಲೆ-ತ್ತು-ವು-ದಕ್ಕೆ
ಮೇಲೆ-ತ್ತು-ವುದು
ಮೇಲೆ-ದ್ದರು
ಮೇಲೆ-ದ್ದಿತು
ಮೇಲೆ-ದ್ದಿದೆ
ಮೇಲೆಂದು
ಮೇಲೆಂದೂಮೇಲೆ-ಣಿ-ಸುವರು
ಮೇಲೆದ್ದ
ಮೇಲೆದ್ದು
ಮೇಲೆಯು
ಮೇಲೆಯೂ
ಮೇಲೆಯೆ
ಮೇಲೆಯೇ
ಮೇಲೇ-ರಲಾರನು
ಮೇಲೇ-ರಲಾರರು
ಮೇಲೇ-ರು-ವುದೇನೋ
ಮೇಲೇಳ-ಬೇ-ಕಾ-ದರೆ
ಮೇಲೇಳು-ತ್ತಿರು-ವಳು
ಮೇಲೇಳು-ವನು
ಮೇಲೇಳು-ವು-ದಕ್ಕೆ
ಮೇಲೊಂದು
ಮೇಳ
ಮೇಳ-ಗೀತೆ-ಗಳೆ-ಲ್ಲಿ
ಮೈ
ಮೈಕೇ-ಲ್
ಮೈಕೇ-ಲ್-ನಿಗೆ
ಮೈಕೇಲನ
ಮೈಕೇಲನು
ಮೈಗೆ
ಮೈಥುನ
ಮೈದಡ-ವಿದರು
ಮೈದಡವು-ತ್ತ
ಮೈದಡವು-ತ್ತಿರು-ವಳು
ಮೈದೋ-ರಲಿ
ಮೈದೋ-ರು-ವನು
ಮೈದೋ-ರು-ವಳು
ಮೈಲಿ
ಮೈಲಿ-ಗ-ಳನ್ನು
ಮೈಲಿ-ಗ-ಳಾದರೂ
ಮೈಲಿ-ಗಳ
ಮೈಲಿ-ಗಳಷ್ಟು
ಮೈಲಿ-ಗಳಾಚೆ
ಮೈಲಿ-ಗಳಿವೆ
ಮೈಲಿ-ಗಳು
ಮೈಲಿ-ಗೆ-ಯಾದ
ಮೈಸೂ-ರಿಗೆ
ಮೈಸೂರಿ-ನ-ಲ್ಲಿ
ಮೈಸೂರಿ-ನ-ಲ್ಲಿ-ದ್ದರೂ
ಮೈಸೂರಿ-ನಿಂದ
ಮೈಸೂರು
ಮೊ
ಮೊಂಡ-ನಾದ-ವನ
ಮೊಗ
ಮೊಗ-ದ-ಲ್ಲಿ
ಮೊಗ-ಲರ
ಮೊಗದ
ಮೊದ-ಮೊದಲು
ಮೊದ-ಲ-ಲ್ಲಿ
ಮೊದ-ಲಿ-ನಿಂದ
ಮೊದ-ಲಿ-ನಿಂದಲೂ
ಮೊದ-ಲಿ-ಲ್ಲ
ಮೊದ-ಲಿನ
ಮೊದಲ
ಮೊದಲ-ನೆ-ಯ-ವರು
ಮೊದಲ-ನೆ-ಯದು
ಮೊದಲ-ನೆ-ಯದೇ
ಮೊದಲ-ನೆಯ
ಮೊದಲನೆ
ಮೊದಲನೇ
ಮೊದಲಾ-ಗಿದೆ
ಮೊದಲಾ-ಯಿತು
ಮೊದಲಾದ
ಮೊದಲಾದ-ವು-ಗಳ-ಲ್ಲಿ
ಮೊದಲಾದುವು-ಗ-ಳನ್ನು
ಮೊದಲು
ಮೊದಲು-ಮಾಡಿ-ದರು
ಮೊದಲೇ
ಮೊಬ್ಬಾಗಿ
ಮೊಬ್ಬು
ಮೊರೆ
ಮೊರೆ-ಯನ್ನು
ಮೊರೆ-ಯು-ತ್ತಿರ-ಬೇಕು
ಮೊರೆ-ಹೋಗು
ಮೊಲ-ವನ್ನು
ಮೊಲದ
ಮೊಳಗಿ-ದರು
ಮೊಳಗಿ-ದರೆ
ಮೊಳೆತು
ಮೊಳೆಯು-ತ್ತಿ-ರು-ವು-ದ-ನ್ನು
ಮೊಸಳೆ
ಮೊಸಳೆ-ಗಳ
ಮೊಸಳೆ-ಗಳು
ಮೊಸಳೆಗೆ
ಮೊಸಳೆಯ
ಮೊಹರು-ಗ-ಳನ್ನು
ಮೋಕ್ಷ
ಮೋಕ್ಷ-ದ-ಲ್ಲಿ
ಮೋಕ್ಷ-ವ-ಲ್ಲ
ಮೋಕ್ಷ-ವನ್ನು
ಮೋಕ್ಷ-ವನ್ನೂ
ಮೋಕ್ಷ-ವಿ-ಲ್ಲ
ಮೋಕ್ಷಕ್ಕೆ
ಮೋಕ್ಷದ
ಮೋಕ್ಷವೇ
ಮೋಡ
ಮೋಡ-ಗಳು
ಮೋಡ-ದ-ಲ್ಲಿದೆ
ಮೋಡ-ದಿಂದ
ಮೋಡದ
ಮೋತಿ
ಮೋತಿ-ಲಾಲ್
ಮೋರ್ಗ-ನ್
ಮೋಸ
ಮೋಸ-ಗೊಂಡು
ಮೋಸ-ಮಾಡು-ತ್ತವೆ
ಮೋಸ-ವಿ-ಲ್ಲ
ಮೋಹ-ನನ
ಮೋಹ-ನರಾಯ-ನಿಂದ
ಮೋಹಕ
ಮೋಹಕ-ರ-ವಾದ
ಮೋಹನ-ರಾಯ್
ಮೋಹಿನಿ-ಬಾಬು-ಗಳ
ಮೌ
ಮೌಂ
ಮೌಢ್ಯ
ಮೌಢ್ಯ-ತೆ-ಗ-ಳನ್ನು
ಮೌಢ್ಯ-ದಿಂದ
ಮೌಢ್ಯತೆ
ಮೌನ
ಮೌನ-ದಿಂದ
ಮೌನ-ಮುಗ್ಧ-ರಾಗಿ
ಮೌನ-ವಾ-ದರು
ಮೌನ-ವಾ-ದರೂ
ಮೌನ-ವಾಗಿ
ಮೌನ-ವಾಗಿ-ದ್ದನು
ಮೌನ-ವಾಗಿ-ದ್ದರು
ಮೌನ-ವಾಗಿದ್ದೆ
ಮೌನ-ವಾಗಿಯೇ
ಮೌನ-ವೆ-ಲ್ಲಿ
ಮೌನಿ-ಯಾದ
ಮ್ಯಾಂಗೋ-ಸ್ಪೀ-ನ್
ಮ್ಯಾಕ್ಲಿಯಾಡ್
ಮ್ಯಾಕ್ಲೆಯಾಡ್
ಮ್ಯಾಕ್ಸಿಮಾಕ್
ಮ್ಯಾಕ್ಸ್
ಮ್ಯಾಕ್ಸ್-ಮು-ಲ್ಲರ್
ಮ್ಯಾಡ-ಮ್
ಮ್ಯಾಡಿ-ಸನ್
ಮ್ಯಾನೇ-ಜರ್
ಮ್ಯೂಸಿಯಂ
ಮ್ಯೂಸಿಯಂ-ಗ-ಳನ್ನು
ಮ್ಲಾನ-ರಾಗಿ
ಮ್ಲಾನ-ವಾಗದೆ
ಮ್ಲೆಚ್ಛಾ
ಮ್ಲೇಚ್ಛ-ರ-ನ್ನು
ಮ್ಲೇಚ್ಛರ
ಮ್ಲೇಚ್ಛರ-ಲ್ಲಿಯೂ
ಮ್ಲೇಚ್ಛರು
ಯ
ಯಂ
ಯಜ-ಮಾನ
ಯಜ-ಮಾನ-ರ-ನ್ನು
ಯಜ-ಮಾನನ
ಯಜ-ಮಾನರು
ಯಜ-ಮಾನಿ
ಯಜ-ಮಾನಿಯ
ಯಜ-ಮಾನಿಯು
ಯಜುರ್ವೇದ-ವನ್ನು
ಯಜ್ಞ
ಯಜ್ಞ-ಗ-ಳನ್ನು
ಯಜ್ಞಕ್ಕೆ
ಯಜ್ಞದ
ಯಜ್ಞೇಶ್ವರ
ಯಜ್ಞೇಶ್ವರ-ಬಾಬು-ವಿನ
ಯಜ್ಞೋಪವೀತ-ಗ-ಳನ್ನು
ಯಜ್ಞೋಪವೀತ-ವನ್ನು
ಯತಿ-ಗಳಿಂದ
ಯತಿ-ಯನ್ನು
ಯತಿ-ಯಾಗುವಂತಹ
ಯತಿ-ವೃಂದ-ವನ್ನು
ಯಥಾ
ಯಥಾ-ಪ್ರ-ಕಾರ
ಯಥಾರ್ಥ
ಯಥೇಚ್ಛ-ವಾಗಿ
ಯಥೇಚ್ಛ-ವಾದ
ಯಥೇಷ್ಟ-ವಾಗಿ
ಯಥೋಚಿ-ತ-ವಾಗಿ
ಯದು-ನಾಥನ
ಯನ್ನು
ಯಮ-ನನ್ನು
ಯಮ-ಯಾ-ತನೆ-ಯನ್ನು
ಯಮ-ಲೋಕಕ್ಕೆ
ಯಮು-ನೆ-ಯ-ಲ್ಲಿ
ಯಮು-ನೆಯ-ಲಿ-ರು-ವುದು
ಯಶ-ಸ್ವಿ
ಯಶ-ಸ್ವಿ-ಯಾ-ದರು
ಯಶ-ಸ್ವಿ-ಯಾ-ಯಿತು
ಯಶ-ಸ್ವಿ-ಯಾಗ-ಬೇ-ಕಾ-ದರೆ
ಯಶ-ಸ್ವಿ-ಯಾಗ-ಲಿ-ಲ್ಲ
ಯಶ-ಸ್ವಿ-ಯಾಗಿ
ಯಶ-ಸ್ವಿ-ಯಾಗಿ-ದ್ದರು
ಯಶ-ಸ್ವಿ-ಯಾಗಿ-ರು-ವೆನು
ಯಶ-ಸ್ವಿ-ಯಾಗಿ-ರುವರು
ಯಶ-ಸ್ವಿ-ಯಾಗುವುದೊ
ಯಶ-ಸ್ಸಿಗೆ
ಯಶ-ಸ್ಸು
ಯಶೋ-ಧರೆ-ಯರು
ಯಹೂದ್ಯ
ಯಹೂದ್ಯ-ರಿಗೆ
ಯಹೂದ್ಯರ
ಯಹೂದ್ಯರು
ಯಾ
ಯಾಂ
ಯಾಕೋಹಾಮಕ್ಕೆ
ಯಾಗ
ಯಾಗ-ಯಜ್ಞ-ಗ-ಳನ್ನು
ಯಾಗಿ
ಯಾಗಿ-ರ-ಬಹುದು
ಯಾನ
ಯಾರ
ಯಾರ-ದ-ನ್ನೋ
ಯಾರ-ನ್ನಾ-ದರೂ
ಯಾರ-ನ್ನೊ
ಯಾರ-ನ್ನೋ
ಯಾರ-ಭಾವ-ವನ್ನೂ
ಯಾರಾ-ದರೂ
ಯಾರಿ-ಗಾ-ದರೂ
ಯಾರಿಗೋ
ಯಾರಿರ-ಬಹುದು
ಯಾರು
ಯಾರೂ
ಯಾರೇ
ಯಾರೊ
ಯಾರೊ-ಡನೆ
ಯಾರೊ-ಬ್ಬ-ರಿಗೂ
ಯಾರೋ
ಯಾರೋ-ಯಾರ್ಕಾಡ್
ಯಾವ
ಯಾವ-ದಾರಿಯೂ
ಯಾವ-ಭಾಗ-ದ-ಲ್ಲಿ
ಯಾವ-ರೀತಿ
ಯಾವಳೋ
ಯಾವು
ಯಾವು-ದ-ಕ್ಕಾಗಿ
ಯಾವು-ದ-ನ್ನಾ-ದರೂ
ಯಾವು-ದ-ನ್ನು
ಯಾವು-ದ-ನ್ನೂ
ಯಾವು-ದ-ನ್ನೇ
ಯಾವು-ದ-ನ್ನೋ
ಯಾವು-ದ-ರ-ಲ್ಲಿ
ಯಾವು-ದ-ರಿಂದಲೂ
ಯಾವು-ದಕ್ಕೂ
ಯಾವು-ದಕ್ಕೆ
ಯಾವು-ದರ
ಯಾವು-ದರ-ಲ್ಲಿಯೂ
ಯಾವು-ದಾದ-ರೊಂದು
ಯಾವು-ದಾದರೂ
ಯಾವು-ದಿ-ತ್ತು
ಯಾವು-ದಿ-ರು-ವುದು
ಯಾವು-ದಿ-ರು-ವುದೋ
ಯಾವು-ದಿದೆ
ಯಾವು-ದೆಂದು
ಯಾವು-ದೆಂಬು-ದ-ನ್ನು
ಯಾವು-ದೊಂದು
ಯಾವು-ಯಾ-ವುದೋ
ಯಾವು-ವೆಂ-ದರೆ
ಯಾವುವೂ
ಯು
ಯುಕ್ತಾ-ಯುಕ್ತ
ಯುಕ್ತಿ
ಯುಕ್ತಿ-ಪೂರ್ವ-ಕ-ವಾಗಿ
ಯುಕ್ತಿ-ಯುಕ್ತ-ವಾಗಿ-ರುವುದು
ಯುಕ್ತಿಗೆ
ಯುಕ್ತಿಯ
ಯುಗ
ಯುಗ-ಗಳ-ಲ್ಲಿಯೂ
ಯುಗ-ದ-ಲ್ಲಿ
ಯುಗ-ಯುಗ-ಗಳ
ಯುಗಕ್ಕೆ
ಯುಗಾ-ವತಾರ
ಯುದ್ಧ
ಯುದ್ಧ-ದ-ಲ್ಲಿ
ಯುದ್ಧ-ವಿ-ಮುಖ-ನಾಗು-ವಂತೆ
ಯುದ್ಧಕ್ಕೆ
ಯುದ್ಧದ
ಯುವ-ಕ-ರ-ನ್ನು
ಯುವ-ಕ-ರಾಗಿ
ಯುವ-ಕ-ರಾಗಿ-ದ್ದಾಗ
ಯುವ-ಕ-ರಿಗೆ
ಯುವ-ಕ-ರಿರಾ
ಯುವ-ಕ-ರೆ-ಲ್ಲ
ಯುವ-ಕ-ರೇ-ನಾಗಿ
ಯುವ-ಕನ
ಯುವ-ಕರ
ಯುವ-ಕರು
ಯುವ-ಕರೂ
ಯುವ-ರಾಜ-ರ-ನ್ನು
ಯುವ-ಶಿಷ್ಯರು
ಯುವಕ
ಯೂನಿ-ವರ್ಸಿಟಿಯ
ಯೂನಿಟೆ-ರಿಯ-ನ್
ಯೂನಿಟೇ-ರಿಯ-ನ್
ಯೂಪ-ಸ್ತಂಭ
ಯೂಪ-ಸ್ತಂಭ-ವನ್ನು
ಯೂರೋಪ-ನ್ನು
ಯೂರೋಪಿ-ನ-ಲ್ಲಿ
ಯೂರೋಪಿ-ಯನ್ನರ
ಯೂರೋಪಿಗೆ
ಯೂರೋಪಿನ
ಯೂರೋಪಿನ-ಲ್ಲೆ-ಲ್ಲ
ಯೂರೋಪ್
ಯೆ
ಯೇ
ಯೊ
ಯೊಗ್ಯ
ಯೊಗ್ಯ-ವ-ಲ್ಲ
ಯೊಚಿ-ಸು-ತ್ತಿದ್ದ
ಯೋ
ಯೋಕೋಹಾಮ
ಯೋಕೋಹಾಮ-ದಿಂದ
ಯೋಗ
ಯೋಗ-ಕ್ಷೇಮ-ವನ್ನು
ಯೋಗ-ಗಳ
ಯೋಗ-ಗಳು
ಯೋಗ-ದಿಂದ
ಯೋಗ-ಭ್ರಷ್ಟ-ರಾ-ದರೆ
ಯೋಗ-ಮಗ್ನ
ಯೋಗ-ಶಕ್ತಿ-ಯಿಂದ
ಯೋಗ-ಶಾ-ಸ್ತ್ರ-ವನ್ನು
ಯೋಗ-ಸಂ-ಸಿದ್ಧಿಂ
ಯೋಗಃ
ಯೋಗಾ-ನಂದ
ಯೋಗಾ-ನಂದ-ರ-ನ್ನು
ಯೋಗಾ-ನಂದ-ರೊ-ಡನೆ
ಯೋಗಾ-ನಂದ-ಸ್ವಾ-ಮಿ-ಗಳು
ಯೋಗಾ-ನಂದರು
ಯೋಗಾ-ಭ್ಯಾಸ
ಯೋಗಾಚ್ಚಲಿತ-ಮಾನಸಃ
ಯೋಗಿ-ಗಳಾದ
ಯೋಗಿ-ಗಳು
ಯೋಗಿ-ಗಳೋ
ಯೋಗಿ-ನ್
ಯೋಗಿ-ನ್ಯೋಗಾ-ನಂದ
ಯೋಗಿಯ
ಯೋಗೋದ್ಯಾನ-ದ-ಲ್ಲಿ
ಯೋಗೋದ್ಯಾನ-ವೆಂಬ
ಯೋಗ್ಯ
ಯೋಗ್ಯ-ಕಾಲ-ದ-ಲ್ಲಿ
ಯೋಗ್ಯ-ತಮ
ಯೋಗ್ಯ-ತಮ-ವಾ-ದದ್ದು
ಯೋಗ್ಯ-ತಮವೇ
ಯೋಗ್ಯ-ತೆ-ಯನ್ನು
ಯೋಗ್ಯ-ತೆ-ಯನ್ನೂ
ಯೋಗ್ಯ-ನಾಗು-ತ್ತಿರು-ವನು
ಯೋಗ್ಯ-ನಾದ
ಯೋಗ್ಯ-ರ-ನ್ನಾಗಿ
ಯೋಗ್ಯ-ರ-ಲ್ಲ
ಯೋಗ್ಯ-ರಾಗಿ-ರ-ಬೇಕು
ಯೋಗ್ಯ-ರಾಗು-ವು-ದಿ-ಲ್ಲ
ಯೋಗ್ಯ-ರಾಗುವರು
ಯೋಗ್ಯ-ರಾದ
ಯೋಗ್ಯ-ರಾದ-ವ-ರಿಗೆ
ಯೋಗ್ಯ-ವ-ಲ್ಲ
ಯೋಗ್ಯ-ವಾ-ದಾಗ
ಯೋಗ್ಯ-ವಾ-ದುದು
ಯೋಗ್ಯ-ವಾಗಿ
ಯೋಗ್ಯ-ವಾಗಿ-ತ್ತು
ಯೋಗ್ಯ-ವಾಗಿ-ರು-ವು-ದ-ನ್ನು
ಯೋಗ್ಯ-ವಾಗಿ-ರು-ವುದು
ಯೋಗ್ಯ-ವಾಗಿ-ರುವಂಥದು
ಯೋಗ್ಯ-ವಾದ
ಯೋಗ್ಯತಾ
ಯೋಗ್ಯತೆ
ಯೋಗ್ಯವೂ
ಯೋಗ್ಯವೇ
ಯೋಗ್ಯವೋ
ಯೋಚ-ನೆಗೂ
ಯೋಚಿ-ಸ-ತೊಡಗಿತು
ಯೋಚಿ-ಸಿದ್ದು
ಯೋಚಿಸ-ತೊಡಗಿದ
ಯೋಚಿಸ-ತೊಡಗಿದರು
ಯೋಚಿಸ-ಬಹುದು
ಯೋಜ-ನೆಯೂ
ಯೋಧ
ಯೋಧ-ನಿಗ-ಲ್ಲ
ಯೋಧಾಗ್ರಣಿ
ಯೋನಿ
ಯೋನಿ-ಪೂಜೆ
ಯೋನಿ-ಪೂಜೆಯ
ಯೋನಿಯ
ಯೌ
ರ
ರಂಗ-ಭೂಮಿ
ರಂಗ-ಭೂಮಿ-ಯಾದ
ರಂಗ-ಭೂಮಿಯ
ರಂಗಾ-ಚಾರ್
ರಂಗಿನ
ರಂಗೋಲೆ
ರಂಜಿ-ಸು-ತ್ತಿದೆ
ರಂದು
ರಂಧ್ರ-ವನ್ನು
ರಕ್ತ
ರಕ್ತ-ಗತ-ಮಾಡಿ-ಕೊ-ಳ್ಳಲು
ರಕ್ತ-ಗತ-ವಾಗು-ವಂತೆ
ರಕ್ತ-ಗತ-ವಾಗು-ವುದು
ರಕ್ತ-ದ-ಲ್ಲಿ
ರಕ್ತ-ದಿಂದ
ರಕ್ತ-ಮಾಂಸ
ರಕ್ತ-ವ-ನ್ನಾ-ದರೂ
ರಕ್ತ-ವನ್ನು
ರಕ್ತ-ವನ್ನೂ
ರಕ್ತ-ವಾಗ-ಬೇಕು
ರಕ್ತ-ವಾಗಲಿ
ರಕ್ತದ
ರಕ್ತ್ತದ
ರಕ್ಷಿ-ಸಲಿ
ರಕ್ಷಿ-ಸುವ
ರಕ್ಷಿಸ-ಲ್ಪ-ಡು-ವುದು
ರಕ್ಷಿಸಿ
ರಕ್ಷಿಸಿ-ರು-ವೆವು
ರಕ್ಷಿಸು
ರಕ್ಷಿಸು-ತ್ತೇವೆ
ರಕ್ಷಿಸು-ವಿರಿ
ರಕ್ಷಿಸು-ವು-ದಕ್ಕೆ
ರಘು-ನಂದನ
ರಘು-ನಾಥ
ರಘು-ನಾಥ-ದಾಸ
ರಘು-ನಾಥ-ದಾಸ-ರ-ನ್ನು
ರಘು-ನಾಥ-ದಾಸ-ರಿಗೆ
ರಘು-ನಾಥ-ದಾಸರ
ರಘು-ನಾಥ-ದಾಸ್
ರಘು-ನಾಥ-ರ-ನ್ನು
ರಘುರಾಈ
ರಚ-ನೆ-ಯ-ಲ್ಲಿ
ರಚ-ನೆ-ಯೊಂದಿಗೆ
ರಚ-ನೆಗೆ
ರಚಿ-ತ-ವಾಗಿ-ತ್ತು
ರಚಿ-ಸು-ತ್ತಿ-ದ್ದರು
ರಚಿಸ-ಬ-ಲ್ಲೆ-ಯಾ-ದರೆ
ರಚಿಸ-ಬೇಕೆಂದು
ರಜ
ರಜ-ಸ್ಸು
ರಜಪುಟಾಣ
ರಜಾ
ರಜೋ-ಗುಣೋ-ನ್ಮಾದ
ರಣರಂಗ-ದ-ಲ್ಲಿ
ರಥ-ವನ್ನು
ರಥದ
ರದ್ದು-ಗೊಳಿ-ಸುವ
ರಬ್ಬಿಯಾದ
ರಬ್ಬಿಯೂ
ರಭಸ-ದಿಂದ
ರಭಸ-ವನ್ನು
ರಮಣೀಯ
ರಮಣೀಯ-ತೆ-ಯನ್ನು
ರಮಾ-ಬಾಯಿ
ರಮಾ-ಬಾಯಿಗೆ
ರವಾನಿಸ-ಬೇಕೆಂ-ದಿ-ರುವೆನು
ರವಾನೆ
ರಷ್ಯಾದ
ರಸ
ರಸ-ಭಂಗ-ವಾ-ದಂತೆ
ರಸ-ವತ್ತಾಗಿ
ರಸ-ವನ್ನು
ರಸ-ವನ್ನೆ-ಲ್ಲಾ
ರಸಾ-ಯನ
ರಸಾ-ಯನ-ಶಾ-ಸ್ತ್ರ
ರಹ-ದಾರಿ-ಯಿ-ಲ್ಲದೆ
ರಹ-ಸ್ಯ
ರಹ-ಸ್ಯ-ಕ್ಕೆ-ಲ್ಲ
ರಹ-ಸ್ಯ-ಗ-ಳನ್ನು
ರಹ-ಸ್ಯ-ಗಳ-ನ್ನೆ-ಲ್ಲ
ರಹ-ಸ್ಯ-ಮತ
ರಹ-ಸ್ಯ-ವನ್ನು
ರಹ-ಸ್ಯ-ವನ್ನೂ
ರಹ-ಸ್ಯ-ವನ್ನೆ-ಲ್ಲ
ರಹ-ಸ್ಯ-ವಾಗಿ
ರಹ-ಸ್ಯ-ವಾಗಿ-ರುವ
ರಹ-ಸ್ಯ-ವಾದ
ರಾ
ರಾಕ್
ರಾಕ್ಫೆ-ಲ್ಲರ್
ರಾಕ್ಷ-ಸಾ-ಕಾರ-ನನ್ನು
ರಾಖಾ-ಲ್
ರಾಖಾ-ಲ್ಬ್ರಹ್ಮಾ-ನಾಂದ
ರಾಖಾಲ
ರಾಖಾಲ-ನನ್ನು
ರಾಖಾಲ-ನಿಗೆ
ರಾಖಾಲ-ನೊಂದಿಗೆ
ರಾಗ-ದಂತೆ
ರಾಜ
ರಾಜ-ಋಷಿ-ಗಳು
ರಾಜ-ಕಾರ-ಣಿಗ-ಲ್ಲ
ರಾಜ-ಕಾರಣ
ರಾಜ-ಕೀಯ
ರಾಜ-ಕೀಯ-ದ-ಲ್ಲಿ
ರಾಜ-ಕೀಯ-ದಿಂದಲೂ
ರಾಜ-ಕೀಯಕ್ಕೆ
ರಾಜ-ಕುಮಾರ-ನಾಗಿ-ದ್ದರೂ
ರಾಜ-ಕುಮಾರ-ರಾದ
ರಾಜ-ಕುಮಾರರ
ರಾಜ-ಗೋಪಾಲಾ-ಚಾರಿ-ಯ-ವರೂ
ರಾಜ-ತಂ-ತ್ರ-ಗಳು
ರಾಜ-ತಂ-ತ್ರ-ದ-ಲ್ಲಿಯೂ
ರಾಜ-ದಾನಿ-ಯಿಂದ
ರಾಜ-ಧಾನಿ
ರಾಜ-ಧಾನಿ-ಯ-ಲ್ಲಿ
ರಾಜ-ಧಾನಿ-ಯಾಗಿ
ರಾಜ-ಧಾನಿ-ಯಾಗಿ-ತ್ತು
ರಾಜ-ಧಾನಿ-ಯಾದ
ರಾಜ-ನನ್ನು
ರಾಜ-ನಾರಾ-ಯಣ-ಬೋ-ಸ್
ರಾಜ-ನಿಂದ
ರಾಜ-ನಿಗೆ
ರಾಜ-ನಿದ್ದ
ರಾಜ-ನೀತಿ
ರಾಜ-ನೀತಿ-ಯ-ಲ್ಲ
ರಾಜ-ಪು-ತ್ರ
ರಾಜ-ಪು-ತ್ರರ
ರಾಜ-ಪು-ರಕ್ಕೆ
ರಾಜ-ಮನೆ-ತ-ನದ
ರಾಜ-ಮರ್ಯಾದೆ-ಯೊ-ಡನೆ
ರಾಜ-ಮಹೇಂದ್ರಿಯ
ರಾಜ-ಮಾರ್ಗ-ಗಳು
ರಾಜ-ಮಾರ್ಗ-ದ-ಲ್ಲಿ
ರಾಜ-ಮಾರ್ಗ-ವನ್ನು
ರಾಜ-ಮ್
ರಾಜ-ಯೊಗ
ರಾಜ-ಯೊಗ-ವನ್ನು
ರಾಜ-ಯೋಗ
ರಾಜ-ಯೋಗ-ವನ್ನು
ರಾಜ-ರ-ನ್ನು
ರಾಜ-ರ-ಲ್ಲ
ರಾಜ-ರ-ವರು
ರಾಜ-ರಂತೆ
ರಾಜ-ರಾಗಿ
ರಾಜ-ರಾದ
ರಾಜ-ರಾಮ
ರಾಜ-ರಿ-ಲ್ಲ
ರಾಜ-ರಿ-ಲ್ಲದೇ
ರಾಜ-ರಿಗೂ
ರಾಜ-ರಿಗೆ
ರಾಜ-ರು-ಗ-ಳನ್ನು
ರಾಜ-ರು-ಗಳಿಂದ
ರಾಜ-ರೊ-ಡನೆ
ರಾಜ-ರೊಂದಿಗೆ
ರಾಜ-ರೊಬ್ಬರು
ರಾಜ-ಸಿಕ
ರಾಜನ
ರಾಜರ
ರಾಜರು
ರಾಜರೆ
ರಾಜರೇ
ರಾಜರ್ಷಿ
ರಾಜಾ
ರಾಜಾ-ರಾಮ
ರಾಜಾ-ರಾಮ-ಮೋಹನ-ರಾಯ್
ರಾಜಾ-ರಾಮ-ಸಿಂಗ್
ರಾಜಾ-ಸ್ಥ-ನ-ದ-ಲ್ಲಿ
ರಾಜಾ-ಸ್ಥಾನ
ರಾಜಾ-ಸ್ಥಾನ-ದ-ಲ್ಲಿ
ರಾಜಾ-ಸ್ಥಾನದ
ರಾಜಿ-ನಾಮೆ
ರಾಜಿ-ಮಾಡಲು
ರಾಜಿ-ಯಾಗದ
ರಾಜೀ-ನಾಮೆ
ರಾಜ್ಯ
ರಾಜ್ಯ-ದ-ಲ್ಲಿ
ರಾಜ್ಯ-ದಿಂದಲೇ
ರಾಜ್ಯ-ವನ್ನು
ರಾಜ್ಯ-ವಿ-ಲ್ಲದೆ
ರಾಜ್ಯಕ್ಕೆ
ರಾಣಿ
ರಾಣಿ-ಯನ್ನೇ
ರಾಧಾ-ಕಾಂತ
ರಾಧಾ-ಕೃಷ್ಣ-ದೇವ
ರಾಧಾ-ಕೃಷ್ಣರ
ರಾಧಾ-ಪ್ರೇಮ-ವನ್ನು
ರಾಧಾಕುಂ-ಡದ
ರಾಧೆ
ರಾಧೆ-ಯಂತಾ-ದರು
ರಾಧೆ-ಯನ್ನು
ರಾಧೆಗೆ
ರಾಧೆಯ
ರಾಬ-ಸ್ಪಿಯ-ರಿನ
ರಾಬರ್ಟ್
ರಾಮ
ರಾಮ-ಕೃಷ್ಣ
ರಾಮ-ಕೃಷ್ಣ-ನಾಗಿ
ರಾಮ-ಕೃಷ್ಣ-ಪು-ರದ
ರಾಮ-ಕೃಷ್ಣ-ರ-ಲ್ಲಿ
ರಾಮ-ಕೃಷ್ಣ-ರಿಗೂ
ರಾಮ-ಕೃಷ್ಣ-ರಿಗೆ
ರಾಮ-ಕೃಷ್ಣ-ರೆಂಬ
ರಾಮ-ಕೃಷ್ಣರ
ರಾಮ-ಕೃಷ್ಣರು
ರಾಮ-ಕೃಷ್ಣಾ-ನಂದ
ರಾಮ-ಕೃಷ್ಣಾ-ನಂದ-ರಿಗೆ
ರಾಮ-ಕೃಷ್ಣಾ-ನಂದರು
ರಾಮ-ಕೃಷ್ಣಾಯ
ರಾಮ-ಚಂದ್ರ
ರಾಮ-ಚಂದ್ರ-ದ-ತ್ತ
ರಾಮ-ಚಂದ್ರ-ದ-ತ್ತನ
ರಾಮ-ಚಂದ್ರಕಿ
ರಾಮ-ಚಂದ್ರಜಿ
ರಾಮ-ಚಂದ್ರನ
ರಾಮ-ತೀರ್ಥ-ರಾ-ದರು
ರಾಮ-ದಯಾಳು
ರಾಮ-ದಾಸ್
ರಾಮ-ನಂತಹ
ರಾಮ-ನನ್ನ-ಲ್ಲದೆ
ರಾಮ-ನಾಗಿ-ದ್ದನೊ
ರಾಮ-ನಾಡ-ನ್ನು
ರಾಮ-ನಾಡಿ-ನ-ಲ್ಲಿ-ದ್ದಾಗ
ರಾಮ-ನಾಡಿಗೆ
ರಾಮ-ನಾಡಿನ
ರಾಮ-ನಾಡಿನಿಂದ
ರಾಮ-ನಾಮ
ರಾಮ-ನಾಮ-ವನ್ನು
ರಾಮ-ನಾಮಾಮೃತ-ವನ್ನು
ರಾಮ-ನಿ-ರುವ
ರಾಮ-ನಿ-ಲ್ಲ
ರಾಮ-ನೇ-ತಕ್ಕೆ
ರಾಮ-ಪ್ರ-ಸನ್ನ
ರಾಮ-ಪ್ರಸಾದನ
ರಾಮ-ಪ್ರಸಾದರ
ರಾಮ-ಬ್ರಹ್ಮ
ರಾಮ-ಬ್ರಹ್ಮ-ಬಾಬು
ರಾಮ-ಭ-ಜನೆ
ರಾಮ-ಮೋಹನ-ದ-ತ್ತ
ರಾಮ-ರಾಮೇತಿ
ರಾಮ-ರಾವ್
ರಾಮ-ಲಾಲ
ರಾಮ-ಲಾಲ್
ರಾಮ-ಸ್ನೇಹಿ
ರಾಮನ
ರಾಮನೂ
ರಾಮೇಶ್ವ-ರಕ್ಕೆ
ರಾಮೇಶ್ವರ
ರಾಯ-ಪು-ರಕ್ಕೆ
ರಾಯ-ಪುರ-ದ-ಲ್ಲಿ
ರಾಯಪುರ-ವನ್ನು
ರಾಯ್
ರಾಯ್ಬಹದ್ದೂರ್
ರಾವ-ಲ್ಪಿಂಡಿ-ಯಿಂದ
ರಾವ-ಲ್ಪಿಂಡಿಗೆ
ರಾವಣ
ರಾವಣ-ನನ್ನು
ರಾವಣಾಸುರ-ನನ್ನು
ರಾಶಿ
ರಾಶಿ-ಗಳೆ-ಲ್ಲ
ರಾಶಿ-ನಕ್ಷ-ತ್ರದ
ರಾಶಿ-ಯ-ನ್ನ-ಲ್ಲದೆ
ರಾಶಿ-ಯನ್ನೇ
ರಾಶಿ-ಯಿಂದ
ರಾಶಿಯ
ರಾಷ್ಟ್ರ
ರಾಷ್ಟ್ರ-ಗ-ಳಿಗೆ
ರಾಷ್ಟ್ರ-ಗಳ
ರಾಷ್ಟ್ರ-ಗಳಿಂದ
ರಾಷ್ಟ್ರ-ಗಳೊ-ಡನೆ
ರಾಷ್ಟ್ರ-ದ-ಲ್ಲೆ-ಲ್ಲ
ರಾಷ್ಟ್ರ-ಪ್ರೇಮ
ರಾಷ್ಟ್ರ-ವನ್ನು
ರಾಷ್ಟ್ರ-ವನ್ನೇ
ರಾಷ್ಟ್ರದ
ರಾಷ್ಟ್ರೀಯ
ರಿ
ರಿಗಿ
ರಿಚಿಲಿ
ರಿಜಿ-ಸ್ಟರ್
ರಿಡ್ಜ್
ರಿತಿ-ಯ-ಲ್ಲಿ
ರಿಪೇರಿ-ಯಾಗ-ಬೇಕಾಗಿ-ತ್ತು
ರಿಪ್ಪ-ನ್
ರಿಯಾಯಿತಿ-ಗ-ಳನ್ನು
ರಿವೆ-ಟ್
ರೀತಿ
ರೀತಿ-ಗ-ಳನ್ನು
ರೀತಿ-ನೀತಿ-ಗ-ಳನ್ನು
ರೀತಿ-ನೀತಿ-ಗಳು
ರೀತಿ-ಯ-ಲ್ಲಿ
ರೀತಿ-ಯ-ಲ್ಲಿ-ತ್ತು
ರೀತಿ-ಯ-ಲ್ಲಿ-ದ್ದು-ದ-ನ್ನು
ರೀತಿ-ಯ-ಲ್ಲೆ
ರೀತಿ-ಯಂತೆ
ರೀತಿ-ಯನ್ನು
ರೀತಿ-ಯಾ-ಗು-ತ್ತಿದೆ
ರೀತಿ-ಯಾಗಿ
ರೀತಿಯ
ರೀತಿಯೇ
ರು
ರುಂಡ-ವನ್ನು
ರುಂಡಮಾಲೆ-ಯನ್ನು
ರುಚಿ
ರುಚಿ-ಕರ-ವಾಗ
ರುಚಿ-ಕರ-ವಾಗಿ-ರುವುದು
ರುಚಿ-ಕರ-ವಾದ
ರುಚಿ-ಯನ್ನು
ರುಚಿ-ಯಾಗಿ-ತ್ತೆಂದು
ರುಚಿ-ಯಾಗಿ-ದ್ದಿ-ತೆಂದೂ
ರುಚಿ-ಯಾದ
ರುಚಿ-ರುಚಿ-ಕರ-ವಾದ
ರುಚಿ-ಸ-ಬಹುದು
ರುಚಿ-ಸ-ಲಿ-ಲ್ಲ
ರುಚಿ-ಸದು
ರುಚಿ-ಸಿತು
ರುಚೀನಾಂ
ರುದ್ರಪ್ರ-ಯಾಗಕ್ಕೆ
ರುದ್ರಾಣಿ
ರುದ್ರೋಪಾಸ-ನೆಯೆ
ರುಮಾ-ಲ-ನ್ನು
ರುಮಾ-ಲಿಗೆ
ರುಮಾಲಿ-ನಂತೆ
ರುಮಾಲು
ರುಮೇನಿಯ
ರೂ
ರೂಢಿ
ರೂಢಿ-ಯ-ಲ್ಲಿ-ತ್ತು
ರೂಢಿ-ಯ-ಲ್ಲಿ-ರು-ವುದು
ರೂಢಿ-ಯ-ಲ್ಲಿದ್ದ
ರೂಢಿ-ಯ-ಲ್ಲಿರುವ
ರೂಢಿ-ಯಂತೆ
ರೂಢಿ-ಯಿದೆ
ರೂಢಿ-ಸಿ-ಕೊಳ್ಳ-ಬೇಕಾಗಿದೆ
ರೂಢಿ-ಸಿ-ದರೆ
ರೂಢಿಗೆ
ರೂಪ
ರೂಪ-ಗ-ಳನ್ನು
ರೂಪ-ಗಳು
ರೂಪ-ದ-ಲ್ಲಿ
ರೂಪ-ದ-ಲ್ಲಿಯೂ
ರೂಪ-ವತಿ-ಯ-ರಾದ
ರೂಪ-ವನ್ನು
ರೂಪ-ವಾಗಿ-ತ್ತೆ
ರೂಪ-ವಾದ
ರೂಪ-ವೆ-ತ್ತಂತೆ
ರೂಪಕ
ರೂಪವೊ
ರೂಪಾಂ-ತರ
ರೂಪಾಯಿ
ರೂಪಾಯಿ-ಗ-ಳನ್ನು
ರೂಪಾಯಿ-ಗಳು
ರೂಪಾಯಿ-ನಿಂದ
ರೂಪಾಯಿ-ಯಂತೆ
ರೂಪಾಯಿ-ಯನ್ನು
ರೂಪಾಯಿಯೆ
ರೂಪಿ-ನ-ಲ್ಲಿ
ರೂಪಿ-ಸಿದೆ
ರೂಪಿ-ಸು-ವು-ದರ
ರೂಪಿ-ಸುವ
ರೂಪಿ-ಸುವರು
ರೂಪಿಸ-ತೊಡಗಿದರು
ರೂಪಿಸ-ಬೇಕೆಂ-ದಿದ್ದೇನೆ
ರೂಪಿಸಿ-ಕೊ-ಳ್ಳೋಣ
ರೂಪು
ರೂಪು-ಗೊ-ಳ್ಳಲಿ
ರೂಪು-ಗೊಂಡಿ-ರು-ವು-ದ-ನ್ನು
ರೂಪು-ಗೊಳಿಸಿ-ಕೊಂಡ-ವ-ರಾಗಿ-ರು-ತ್ತಾರೆ
ರೂಮಿ-ನ-ಲ್ಲಿ
ರೂಮಿ-ನ-ಲ್ಲಿ-ದ್ದಾಗ
ರೂಮಿ-ನ-ಲ್ಲಿ-ರು-ವಾಗಲೂ
ರೂಮಿ-ನಿಂದ
ರೂಮಿನ
ರೂಮು
ರೆ
ರೆಂಬೆ
ರೆಂಬೆ-ಕೊಂಬೆ-ಗಳ-ನ್ನೇ
ರೆಂಬೆ-ಯಿಂದ
ರೆಂಬೆಗೆ
ರೆಕ್ಕೆ
ರೆಜಿಮೆಂಟಿನ
ರೆಡ್
ರೆಸಿಡೆಂ-ಟ್
ರೇ
ರೇಖಾ
ರೇಖಾ-ಗಣಿತ
ರೇಖೆ-ಯ-ಲ್ಲಿ
ರೇಖೆ-ಯನ್ನು
ರೇಗಿ
ರೇವು
ರೇವು-ಪಟ್ಟ-ಣ-ದ-ಲ್ಲಿ
ರೇವು-ಪಟ್ಟ-ಣ-ವನ್ನು
ರೇವು-ಪಟ್ಟ-ಣವೇ
ರೇಶ್ಮೆಯ
ರೇಷ್ಮೆಯ
ರೈ
ರೈಲಿ-ನ-ಲ್ಲಿ
ರೈಲು
ರೈಲು-ಗಾಡಿ-ಗ-ಳನ್ನು
ರೊ
ರೊಚ್ಚಿಗೇಳು-ವುದು
ರೊಚ್ಚು
ರೊಯ್ಯನೆ
ರೋ
ರೋಗ
ರೋಗ-ಗ-ಳನ್ನು
ರೋಗ-ಗಳೆ-ಲ್ಲ
ರೋಗ-ದಿಂದ
ರೋಗ-ವನ್ನು
ರೋಗ-ವನ್ನೇ
ರೋಗ-ವೆಂದು
ರೋಗಕ್ಕೂ
ರೋಗಕ್ಕೆ
ರೋಗದ
ರೋಗಿ
ರೋಗಿ-ಗಳ-ಲ್ಲಿ
ರೋಗಿಗೆ
ರೋಗಿಯ
ರೋಡಿ-ನ-ಲ್ಲಿ
ರೋಡಿ-ನ-ಲ್ಲಿ-ರುವ
ರೋಡಿನ
ರೋಡು
ರೋಡ್
ರೋಡ್ನ-ಲ್ಲಿ
ರೋಮ
ರೋಮ-ನ್
ರೋಮ-ನ್ನರ
ರೋಮ-ನ್ನರೂ
ರೋಮಿ-ನ-ಲ್ಲಿ
ರೋಲಾ
ರೋಸಿ
ರೋಸಿ-ಹೋಗಿ
ರೌದ್ರ-ಕ್ಕಾಗಿ
ರೌದ್ರ-ಭಾವ-ಗಳ
ರೌದ್ರ-ವನ್ನು
ರೌದ್ರೋಪಾಸ-ಕ-ರಾಗಿ
ರ್ಯಾಫಿ-ಲ್
ಲ
ಲಂ
ಲಂಗ
ಲಂಚ-ಕೊಡು-ವುದು
ಲಂಡ-ನ್
ಲಂಡ-ನ್ನಿ-ನ-ಲ್ಲಿ
ಲಂಡ-ನ್ನಿ-ನಿಂದ
ಲಂಡ-ನ್ನಿಗೆ
ಲಂಡ-ನ್ನಿನ
ಲಂಡ-ನ್ನಿನ-ಲ್ಲಿ-ರುವ
ಲಕ್ನೊ
ಲಕ್ನೋ
ಲಕ್ಷ
ಲಕ್ಷ-ಗ-ಟ್ಟಲೆ
ಲಕ್ಷಕ್ಕೆ
ಲಕ್ಷಾಂ-ತರ
ಲಕ್ಷಿ-ಸು-ವುದೇ
ಲಕ್ಷಿಸಿ-ದರು
ಲಕ್ಷೋಪ-ಲಕ್ಷ
ಲಕ್ಷ್ಮಿ
ಲಕ್ಷ್ಯ-ವಿ-ಟ್ಟು
ಲಕ್ಷ್ಯಕ್ಕೆ
ಲಕ್ಷ್ಯವೇ
ಲಘು
ಲಘು-ಕೌಮುದಿಯ
ಲಘು-ವಾದ
ಲಜ್ಜಾವ-ತಿ-ಯಂತೆ
ಲಜ್ಜೆ
ಲಟು
ಲಟು-ಅ-ದ್ಭುತಾ-ನಂದ
ಲಟು-ವನ್ನು
ಲಬೆಕ್
ಲಭಿ-ಸ-ಲಿ-ಲ್ಲ
ಲಭಿ-ಸಿ-ತ್ತು
ಲಭಿ-ಸು-ತ್ತದೆ
ಲಭಿ-ಸು-ವು-ದಿ-ಲ್ಲ
ಲಭಿಸಿ
ಲಭಿಸಿ-ದರೆ
ಲಭಿಸಿ-ರ-ಲಿ-ಲ್ಲ
ಲಭಿಸಿ-ರುವಾಗ
ಲಭ್ಯಃ
ಲಯ-ವಾ-ಯಿತು
ಲಯ-ವಾಗು-ತ್ತಿ-ರು-ವು-ದ-ನ್ನು
ಲಲಿತ-ಕಲಾ-ವಿದರು
ಲವಲವಿ-ಕೆಯಿಂ-ದಿ-ದ್ದಾರೆ
ಲವಲೇಶ-ವಾ-ದರೂ
ಲವಲೇಶವೂ
ಲಾ
ಲಾಟು-ವನ್ನು
ಲಾಟೂನ
ಲಾಟೆ
ಲಾಭ
ಲಾಭ-ದಾಯಕ-ವಾಗಿ-ತ್ತು
ಲಾಭ-ದಿಂದ
ಲಾಭ-ವಾಗು-ವು-ದಿ-ಲ್ಲ
ಲಾಭಕ್ಕೆ
ಲಾಮಾ-ಗಳು
ಲಾಯ-ರ-ನ್ನೂ
ಲಾಯ-ರು-ಗಳು
ಲಾಯ-ಸನ್
ಲಾರೆ-ನ್ಸ್
ಲಾಲನೆ
ಲಾಲಾ
ಲಾಲ್ಶಂ-ಕರ್
ಲಾವಣಿ-ಗ-ಳನ್ನು
ಲಾಸಾಕ್ಕೆ
ಲಿ
ಲಿಂಗ
ಲಿಂಗ-ಪುರಾಣ-ದ-ಲ್ಲಿ
ಲಿಂಗ-ಪೂಜೆ
ಲಿಂಗ-ಬೇಧ-ಗಳಿಂದ
ಲಿಂಗ-ಬೇಧ-ವಿದೆ-ಯೇನು
ಲಿಂಗ-ವನ್ನು
ಲಿಂಗಂ
ಲಿಂಗಕ್ಕೆ
ಲಿಂಗದ
ಲಿಂಗವೂ
ಲಿಂಬ್ಡಿ
ಲಿಂಬ್ಡಿ-ಯನ್ನು
ಲಿಂಬ್ಡಿಗೆ
ಲಿಕ್ಕಿ-ಸಲಾರ-ದಷ್ಟು
ಲಿಖಿತ-ಗಳು
ಲಿಚಿ
ಲಿಪಿ-ಯ-ಲ್ಲಿ
ಲಿಪಿ-ಯಿಂದ
ಲಿಪ್ತ-ವಾಗಿ-ಲ್ಲ
ಲಿಯ-ನಾರ್ಡೊ
ಲಿಯ-ನ್
ಲಿಯ-ನ್ಸ್
ಲೀನ-ವಾಗಿ
ಲೀನ-ವಾಗು-ವುದು
ಲೀನ-ವಾಗು-ವುದೋ
ಲೀಲಾ
ಲೀಲಾ-ಜಾಲ-ವಾಗಿ
ಲೀಲಾ-ದರ್ಶನ
ಲೀಲಾ-ನಾಟಕ
ಲೀಲಾ-ಭೂಮಿ
ಲೀಲಾ-ಸ್ಥಾನ-ಗಳ-ನ್ನೂ
ಲೀಲೆ
ಲೀಲೆಗೆ
ಲೀಲೆಯ
ಲೂಯಿ-ಸ್
ಲೆ
ಲೆಕ್ಕ-ಮಾಡಿ-ಕೊಂಡು
ಲೆಕ್ಕಕ್ಕೆ
ಲೆಕ್ಕಾ-ಚಾರ
ಲೆಕ್ಕಾ-ಚಾರ-ದಂತೆ
ಲೆಕ್ಕಿ-ಸದೆ
ಲೆಕ್ಕಿ-ಸದೇ
ಲೆಕ್ಕಿ-ಸಿ-ದರು
ಲೆಕ್ಕಿ-ಸು-ವು-ದಿ-ಲ್ಲ
ಲೆಕ್ಕಿ-ಸುವರು
ಲೆಕ್ಕಿಸ-ಬೇಡಿ
ಲೆಗ-ಟ್
ಲೆಗೆ-ಟ್
ಲೆಪ್ಟರ್
ಲೇಖ-ನದ
ಲೇಖ-ನವೇ
ಲೇಖನ
ಲೇಖನ-ಗ-ಳನ್ನು
ಲೇಖನ-ಗಳೊ
ಲೇಖನ-ದ-ಲ್ಲಿ
ಲೇಖನ-ದಿಂದಲೇ
ಲೇಖನ-ವನ್ನು
ಲೇಖನಿ
ಲೇಖನಿಯ
ಲೇಪಿ-ಸಿ-ದರು
ಲೇಪಿತ-ವಾ-ದಂತೆ
ಲೇಪಿಸಿ-ಕೊ-ಳ್ಳು-ತ್ತಿದ್ದರು
ಲೇಪಿಸಿ-ಕೊಂಡು
ಲೈ
ಲೈಬ್ರರಿ
ಲೈಬ್ರರಿ-ಯ-ಲ್ಲಿ
ಲೈಬ್ರರಿ-ಯ-ವನು
ಲೈಬ್ರರಿ-ಯನ್
ಲೈಬ್ರರಿ-ಯನ್ನೆ-ಲ್ಲ
ಲೋ
ಲೋಕ
ಲೋಕ-ಕ-ಲ್ಯಾಣ
ಲೋಕ-ಕ-ಲ್ಯಾಣ-ಕ್ಕಾಗಿ
ಲೋಕ-ಕ-ಲ್ಯಾಣಕ್ಕೆ
ಲೋಕ-ಗಳಿಗೂ
ಲೋಕ-ದ-ಲ್ಲಿದ್ದಂತೆ
ಲೋಕ-ದಿಂದ
ಲೋಕ-ಪ್ರಖ್ಯಾತ-ರಾಗಿ-ದ್ದರು
ಲೋಕ-ವನ್ನು
ಲೋಕ-ವನ್ನೇ
ಲೋಕ-ಶಿ-ಕ್ಷಣ-ಕ್ಕಾಗಿಯೂ
ಲೋಕಂ
ಲೋಕಕ್ಕೆ
ಲೋಕದ
ಲೋಕಾ-ಚಾರ
ಲೋಕಾ-ಲೋಕ-ಗಳ
ಲೋಕಾಭಿ-ರಾಮ-ವಾಗಿ
ಲೋಟ
ಲೋಡಿಯಾ-ದ-ಲ್ಲಿ
ಲೋನಿ
ಲೋಪ
ಲೋಪ-ದೋಷ-ಗ-ಳನ್ನು
ಲೋಪ-ದೋಷ-ಗಳಿಂದ
ಲೋಪ-ದೋಷ-ಗಳು
ಲೋಪ-ದೋಷ-ವನ್ನು
ಲೋಪ-ವಾದ
ಲೋಹ-ಗ-ಳನ್ನು
ಲೋಹ-ಗಳೂ
ಲೋಹವೆ
ಲೌಕಿ-ಕದ
ಲೌಕಿ-ಕವೂ
ಲೌಕಿಕ
ಲೌಕಿಕ-ವಾಗಿ
ಲೌಕಿಕ-ವಾದುದ-ನ್ನೆ-ಲ್ಲಾ
ಲ್ಯಾಂಪ-ನ್ನು
ವ
ವಂ
ವಂಗ
ವಂಗ-ದೇಶ-ದ-ಲ್ಲಿ
ವಂಗ-ದೇಶದ
ವಂಗ-ದೇಶವೇ
ವಂಗ-ಭೂಮಿ-ಯ-ಲ್ಲಿ
ವಂಗ-ಸಾ-ಹಿತಿ
ವಂದನೆ-ಗ-ಳನ್ನು
ವಂದನೆ-ಗಳು
ವಂದಿಸ-ಬೇಕು
ವಂಶ-ದ-ಲ್ಲಿ
ವಂಶ-ವನ್ನು
ವಂಶದ
ವಂಶದ-ವ-ರದು
ವಂಶಾನುಗ-ತ-ವಾಗಿ
ವಕೀ-ಲರು
ವಕೀ-ಲಿಯ
ವಕೀಲ
ವಕೀಲ-ನ-ನ್ನಾಗಿ
ವಕೀಲ-ನ-ಲ್ಲ
ವಕೀಲ-ನಾಗಿ
ವಕೀಲ-ನಾಗಿ-ದ್ದನು
ವಕೀಲ-ನಾಗಿದ್ದ
ವಕೀಲ-ನಾಗು-ವಂತೆ
ವಕೀಲ-ನಾಗು-ವುದು
ವಕೀಲ-ನಾದ
ವಕೀಲ-ನಿಗೆ
ವಕೀಲ-ರಾದ
ವಕೀಲ-ರೊ-ಡನೆ
ವಕೀಲನ
ವಕೀಲನು
ವಕೀಲನೂ
ವಕೀಲಿ
ವಕ್ರ-ತೆಯೂ
ವಕ್ರ-ವಾಗಿಯೋ
ವಕ್ಷ-ಸ್ಥಳ-ದ-ಲ್ಲಿ
ವಜಾ
ವಜ್ರ
ವಜ್ರ-ಪಡಿ
ವಜ್ರದ
ವಜ್ರಾಯುಧ-ದಂತೆ
ವಜ್ರೋಪಮ
ವಜ್ರೋಪಮ-ಯ-ವಾಗಿ-ರುವ
ವತ್ಸ
ವನ-ಪರ್ವ
ವನ-ಸ್ಪತಿಯ
ವನ್ನು
ವನ್ಯ
ವಪೀಂಗಣ
ವಯ-ಸ್ಸಾ-ಯಿತು
ವಯ-ಸ್ಸಾಗಿ-ದ್ದರೂ
ವಯ-ಸ್ಸಾಗಿ-ಲ್ಲದೇ
ವಯ-ಸ್ಸಾಗು-ತ್ತ
ವಯ-ಸ್ಸಾಗುವ
ವಯ-ಸ್ಸಾದ
ವಯ-ಸ್ಸಾದಂತೆ
ವಯ-ಸ್ಸಾದಾಗ
ವಯ-ಸ್ಸಿ-ನ-ಲ್ಲಿ
ವಯ-ಸ್ಸಿ-ನ-ಲ್ಲೇ
ವಯ-ಸ್ಸಿಗೆ
ವಯ-ಸ್ಸಿಗೇ
ವಯ-ಸ್ಸಿನ
ವಯ-ಸ್ಸಿನ-ಲ್ಲಿಯೇ
ವಯ-ಸ್ಸಿನ-ಲ್ಲೆ
ವಯ-ಸ್ಸು
ವರ-ದಂತೆ
ವರ-ದಕ್ಷಿಣೆ-ಯನ್ನು
ವರ-ದಿ-ಗಳು
ವರ-ನೊಬ್ಬ
ವರ-ಮಾನ
ವರ-ಮಾನ-ವಿ-ರ-ಲಿ-ಲ್ಲ
ವರ-ವನ್ನು
ವರ-ಹಾ-ನ-ಗರ-ದ-ಲ್ಲಿ
ವರಾಂ-ಡದ
ವರಾಂಡ
ವರಾಂಡ-ದ-ಲ್ಲಿ
ವರಾಂಡ-ವಾ-ದರೋ
ವರಾಹ
ವರಾಹ-ನ-ಗರ-ದ-ಲ್ಲಿ-ರುವ
ವರು-ಷ-ಗ-ಳಾದ
ವರು-ಷ-ಗ-ಳಾದರೂ
ವರು-ಷ-ಗಳ
ವರು-ಷ-ಗಳ-ಲ್ಲಿ
ವರು-ಷ-ಗಳ-ವ-ರೆಗೆ
ವರು-ಷ-ಗಳಿ-ಗಿಂತ
ವರು-ಷ-ಗಳಿಂದ
ವರು-ಷ-ಗಳಿಂದಲೂ
ವರು-ಷ-ಗಳು
ವರು-ಷ-ದ-ಲ್ಲಿಯೇ
ವರು-ಷ-ದ-ವ-ರೆಗೆ
ವರು-ಷ-ವಾ-ದಾಗ
ವರು-ಷಕ್ಕೆ
ವರು-ಷದ
ವರು-ಷವೂ
ವರು-ಷವೇ
ವರುಷ
ವರ್ಗ
ವರ್ಗ-ದ-ಲ್ಲಿ
ವರ್ಗ-ದ-ಲ್ಲಿ-ರು-ವ-ವರು
ವರ್ಗ-ದ-ವ-ರಿಂದ
ವರ್ಗ-ದ-ವ-ರಿಗೆ
ವರ್ಗ-ದ-ವರ
ವರ್ಗ-ದ-ವರು
ವರ್ಗಕ್ಕೆ
ವರ್ಗದ
ವರ್ಗಾಯಿ-ಸಿತು
ವರ್ಗಾಯಿಸಿ
ವರ್ಗಾಯಿಸಿ-ದ್ದರು
ವರ್ಚ-ಸ್ಸಿಗೆ
ವರ್ಜಿಸ-ಬೇಕು
ವರ್ಡ್ಸ-ವರ್ತ್
ವರ್ಣ
ವರ್ಣ-ಗ-ಳನ್ನು
ವರ್ಣ-ಗಳ
ವರ್ಣ-ಗಳ-ಲ್ಲಿ
ವರ್ಣ-ಗಳು
ವರ್ಣ-ಚಿ-ತ್ರ
ವರ್ಣ-ನಾತೀ-ತ-ವಾಗಿ-ತ್ತು
ವರ್ಣ-ನೆ-ಯನ್ನು
ವರ್ಣ-ನೆ-ಯಿಂದಲೇ
ವರ್ಣ-ವನ್ನು
ವರ್ಣಕ್ಕೂ
ವರ್ಣನೆ
ವರ್ಣಾ-ಶ್ರಮದ
ವರ್ಣಾಶ್ರ-ಮ-ಗ-ಳನ್ನು
ವರ್ಣಿ-ಸಲು
ವರ್ಣಿ-ಸು-ತ್ತಾರೆ
ವರ್ಣಿ-ಸು-ತ್ತಿ-ದ್ದರು
ವರ್ಣಿ-ಸು-ತ್ತಿದ್ದ
ವರ್ಣಿ-ಸುವರು
ವರ್ಣಿ-ಸುವಾಗ
ವರ್ತ-ಕನ
ವರ್ತ-ಕರ
ವರ್ತ-ಕರು
ವರ್ತ-ಮಾನ
ವರ್ತ-ಮಾನ-ಕಾಲ-ದ-ಲ್ಲಿ
ವರ್ತ-ಮಾನ-ವನ್ನು
ವರ್ತ-ಮಾನತೆ
ವರ್ತ-ಮಾನದ
ವರ್ತಂತೇ
ವರ್ತಕ
ವರ್ತಕ-ನೊಬ್ಬ
ವರ್ತನೆ-ಯನ್ನು
ವರ್ತಿ-ಸ-ಬಹುದು
ವರ್ತಿ-ಸಿ-ದರೂ
ವರ್ತಿ-ಸಿದ
ವರ್ತಿ-ಸಿದ-ರೆಂದೂ
ವರ್ತಿ-ಸು-ತ್ತಿ-ದ್ದರು
ವರ್ತಿ-ಸು-ತ್ತಿದ್ದೆ
ವರ್ತಿ-ಸು-ವುದು
ವರ್ತಿ-ಸುವ
ವರ್ತ್ಮಾನು
ವರ್ನ
ವರ್ಮರು
ವರ್ಷ
ವರ್ಷ-ಕ್ಕಿಂತ
ವರ್ಷ-ಗ-ಳಾದ
ವರ್ಷ-ಗ-ಳಾದ-ಮೇಲೆ
ವರ್ಷ-ಗಳ
ವರ್ಷ-ಗಳ-ಮೇಲೆ
ವರ್ಷ-ಗಳ-ಲ್ಲಿ
ವರ್ಷ-ಗಳ-ವ-ರೆಗೆ
ವರ್ಷ-ಗಳಷ್ಟು
ವರ್ಷ-ಗಳಾಗಿ-ತ್ತು
ವರ್ಷ-ಗಳಿಂದ
ವರ್ಷ-ಗಳಿಂದಲು
ವರ್ಷ-ಗಳಿಂದಲೂ
ವರ್ಷ-ಗಳಿರ-ಬೇಕು
ವರ್ಷ-ಗಳು
ವರ್ಷ-ದಿಂದ
ವರ್ಷ-ವರ್ಷಕ್ಕೂ
ವರ್ಷ-ವಾ-ದರೂ
ವರ್ಷಕ್ಕೆ
ವರ್ಷದ
ವರ್ಷವೂ
ವರ್ಷಾದ್ಯಂತವೂ
ವರ್ಷಿ-ಸಿದ್ದರೋ
ವರ್ಷಿ-ಸುವನು
ವಲಯ
ವಲಯ-ಗಳ-ನ್ನೂ
ವಲಯ-ದೊಳಗೆ
ವಲಯ-ವನ್ನು
ವಶ-ನಾಗಿ-ರಲಿ-ಲ್ಲ
ವಶ-ನಾಗಿ-ರು-ವೆಯಾ
ವಶ-ಪಡಿ-ಸಿ-ಕೊಂಡ-ವರು
ವಶ-ಮಾಡಿ-ಕೊ-ಳ್ಳು-ತ್ತಿ-ರಲಿ-ಲ್ಲ
ವಶ-ಮಾಡಿ-ಕೊ-ಳ್ಳು-ತ್ತಿ-ರುವರು
ವಶ-ಮಾಡಿ-ಕೊ-ಳ್ಳು-ವು-ದಕ್ಕೆ
ವಶ-ಮಾಡಿ-ಕೊ-ಳ್ಳುವುದ-ರ-ಲ್ಲಿ
ವಶ-ಮಾಡಿ-ಕೊಂಡಿತು
ವಶ-ರಾಗಿ
ವಶ-ರಾಗುವುದು
ವಶ-ವಾಗಿ
ವಶ-ವಾಗಿದ್ದು
ವಶದ-ಲ್ಲಿ-ಟ್ಟು-ಕೊಂಡಿದ್ದ
ವಶರಾಗು-ತ್ತಿ-ರಲಿ-ಲ್ಲ
ವಸಿಷ್ಠ
ವಸಿಷ್ಠ-ನಿಗೆ
ವಸಿಷ್ಠನ
ವಸು
ವಸು-ಗಳ
ವಸೂಲಿ
ವಸೂಲಿ-ಮಾಡ-ಬೇಕೆಂದು
ವಸೂಲಿ-ಮಾಡಿ
ವಹಿ-ಸಲು
ವಹಿ-ಸು-ತ್ತಿ-ದ್ದನು
ವಹಿ-ಸು-ತ್ತಿ-ದ್ದರು
ವಹಿಸ-ಬೇಕೆಂದು
ವಹಿಸಿ
ವಹಿಸಿ-ಕೊ-ಳ್ಳು-ತ್ತಿದ್ದಳು
ವಹಿಸಿ-ಕೊ-ಳ್ಳು-ತ್ತೇನೆ
ವಹಿಸಿ-ಕೊಂಡಳು
ವಹಿಸಿ-ಕೊಂಡಿದ್ದರು
ವಹಿಸಿ-ಕೊಂಡು
ವಹಿಸಿ-ಕೊಳ್ಳ-ಬೇಕು
ವಹಿಸಿ-ಕೊಳ್ಳ-ಬೇಕೆಂದು
ವಹಿಸಿ-ದರು
ವಹಿಸಿ-ರುವರು
ವಾ
ವಾಂಕೂ-ವರ್
ವಾಂತಿ
ವಾಕ್
ವಾಕ್ಚಾತುರ್ಯ
ವಾಕ್ಚಾತುರ್ಯ-ವನ್ನು
ವಾಕ್ಯ
ವಾಕ್ಯ-ಗ-ಳನ್ನು
ವಾಕ್ಯ-ಗ-ಳಿಗೆ
ವಾಕ್ಯ-ಗಳ
ವಾಕ್ಯ-ಗಳ-ನ್ನೇ
ವಾಕ್ಯ-ಗಳು
ವಾಕ್ಯ-ವನ್ನು
ವಾಕ್ಯಾರ್ಥ
ವಾಗ್
ವಾಗ್ಧಾರೆ-ಯಿಂದ
ವಾಗ್ಧೋರಣೆ
ವಾಗ್ಮಿ
ವಾಗ್ಮಿ-ಗ-ಳಾದ
ವಾಗ್ಮಿ-ಗ-ಳಿದ್ದರು
ವಾಗ್ಮಿ-ಗಳ
ವಾಗ್ಮಿ-ತ್ವ
ವಾಗ್ಮಿಯ
ವಾಗ್ವೈಖರಿ
ವಾಗ್ವೈಖರಿಯ
ವಾಚಾಳಿ-ಗಳೇ
ವಾಚಾಳಿ-ತನ
ವಾಚಾಳಿ-ಯ-ನ್ನಾಗಿ
ವಾಡಿಕೆ
ವಾಣಿ
ವಾಣಿ-ಯ-ಲ್ಲಿ
ವಾಣಿ-ಯಗಿ-ತ್ತು
ವಾಣಿ-ಯನ್ನು
ವಾಣಿ-ಯೊಂದು
ವಾಣಿ-ಯೊಂದೇ
ವಾತ-ವರ-ಣ-ದಿಂದ
ವಾತಾ-ವರ-ಣ-ಗಳ
ವಾತಾ-ವರ-ಣ-ದ-ಲ್ಲಿ
ವಾತಾ-ವರ-ಣ-ದಿಂದ
ವಾತಾ-ವರ-ಣ-ದೊಂದಿಗೆ
ವಾತಾ-ವರ-ಣ-ವನ್ನು
ವಾತಾ-ವರ-ಣ-ವಿ-ತ್ತು
ವಾತಾ-ವರ-ಣ-ವೆ-ಲ್ಲ
ವಾತಾ-ವರ-ಣಕ್ಕೆ
ವಾತಾ-ವರ-ಣವೇ
ವಾತಾ-ವರಣ
ವಾದ
ವಾದ-ಗಳೂ
ವಾದ-ಗಳೇ
ವಾದ-ದ-ಲ್ಲಿ
ವಾದ-ಮಾಡ-ತೊಡಗಿದರು
ವಾದ-ಮಾಡಿ
ವಾದ-ಮಾಡು-ವಾಗ
ವಾದ-ಮಾಡು-ವು-ದ-ನ್ನು
ವಾದ-ಸರಣಿ-ಯನ್ನು
ವಾದದ
ವಾದಿ-ಸಿ-ದರು
ವಾದಿ-ಸಿ-ದರೂ
ವಾದಿ-ಸಿ-ದರೆ
ವಾದಿ-ಸು-ತ್ತ
ವಾದಿ-ಸು-ತ್ತಿದ್ದ
ವಾದಿ-ಸು-ತ್ತಿದ್ದರು
ವಾದಿ-ಸು-ವುದು
ವಾದಿಸಿ
ವಾದ್ಯ
ವಾದ್ಯ-ಗ-ಳನ್ನು
ವಾದ್ಯ-ಗಳು
ವಾದ್ಯ-ಧ್ವನಿ-ಯಿಂದ
ವಾದ್ವಾ-ನ್ಗೆ
ವಾನಪ್ರ-ಸ್ಥರಾದ
ವಾಪ-ಸ್
ವಾಮಾ-ಚಾರ-ವನ್ನು
ವಾಮಾಚಾ-ರಕ್ಕೆ
ವಾಯು
ವಾಯು-ವಿ-ನಂತೆ
ವಾಯುವ್ಯ
ವಾರ
ವಾರ-ಕ್ಕಿಂತ
ವಾರ-ಗ-ಳನ್ನು
ವಾರ-ಗ-ಳಿದ್ದರು
ವಾರ-ಗಳ
ವಾರ-ಗಳ-ವ-ರೆಗೆ
ವಾರ-ಗಳು
ವಾರ-ಣಾಸಿ
ವಾರ-ದ-ಲ್ಲಿ
ವಾರ-ದ-ವ-ರಿಗೆ
ವಾರ-ವಾದ
ವಾರ-ವಾದ-ಮೇಲೆ
ವಾರದ
ವಾರ್ತಾ
ವಾರ್ತೆ
ವಾರ್ತೆ-ಗಳು
ವಾರ್ತೆ-ಯನ್ನು
ವಾರ್ಷಿಕ-ವ-ರದಿ-ಯ-ಲ್ಲಿ
ವಾರ್ಷಿಕೋ-ತ್ಸವ
ವಾರ್ಷಿಕೋ-ತ್ಸವ-ದ-ಲ್ಲಿ
ವಾರ್ಷಿಕೋ-ತ್ಸವದ
ವಾಲಗ
ವಾಲಾಜ
ವಾಲಿ-ತ್ತು
ವಾಲು-ವು-ದಕ್ಕೆ
ವಾಷಿಂಗ್ಟ-ನ್ಗ-ಳನ್ನು
ವಾಸ
ವಾಸ-ಕ್ಕಾಗಿ
ವಾಸ-ಗೃಹಕ್ಕೆ
ವಾಸ-ದಂತೆ
ವಾಸ-ನೆ-ಗಳ-ಲ್ಲಿ
ವಾಸ-ಮಾ-ಡು-ತ್ತ
ವಾಸ-ಮಾ-ಡು-ತ್ತಾ
ವಾಸ-ಮಾಡ-ಬ-ಲ್ಲೆ
ವಾಸ-ಮಾಡ-ಬೇಕೆಂದು
ವಾಸ-ಮಾಡಿ-ಕೊಂಡಿದ್ದುದು
ವಾಸ-ಮಾಡಿ-ಕೊಂಡು
ವಾಸ-ಮಾಡಿದ
ವಾಸ-ಮಾಡು-ತ್ತಿದ್ದ
ವಾಸ-ಮಾಡು-ತ್ತಿದ್ದರು
ವಾಸ-ಮಾಡು-ತ್ತಿರು-ವರು
ವಾಸ-ಮಾಡು-ವಂತೆ
ವಾಸ-ಮಾಡು-ವರು
ವಾಸ-ಮಾಡು-ವಾಗ
ವಾಸ-ಮಾಡು-ವು-ದಕ್ಕೆ
ವಾಸ-ಮಾಡು-ವುದು
ವಾಸ-ಯೋಗ್ಯ
ವಾಸ-ವಾ-ಗಿ-ದ್ದರು
ವಾಸ-ವಾಗಿ-ತ್ತು
ವಾಸ-ವಾಗಿ-ರುವ
ವಾಸ-ವಾಗಿದ್ದ
ವಾಸ-ಸ್ಥಾನ-ವಾಗಿ-ತ್ತೆಂದು
ವಾಸಕ್ಕೆ
ವಾಸನೆ
ವಾಸಿ
ವಾಸಿ-ಗಳೆ-ಲ್ಲರೂ
ವಾಸಿ-ಮಾಡಿ-ರುವ
ವಾಸಿ-ಯಾಗು-ವು-ದ-ನ್ನು
ವಾಸಿ-ಸ-ತೊಡಗಿದರು
ವಾಸಿ-ಸ-ಬೇಕು
ವಾಸಿ-ಸು-ತ್ತಿ-ದ್ದರು
ವಾಸಿ-ಸು-ತ್ತಿ-ರುವ
ವಾಸಿ-ಸು-ತ್ತಿ-ರುವ-ವನು
ವಾಸಿ-ಸು-ತ್ತಿದ್ದ
ವಾಸಿ-ಸು-ತ್ತೇನೆ
ವಾಸಿ-ಸು-ವು-ದಕ್ಕೆ
ವಾಸಿ-ಸುವ
ವಾಸಿ-ಸುವ-ರೆಂದೂ
ವಾಸಿ-ಸುವರು
ವಾಸಿ-ಸುವುದಕ್ಕಿಂತ
ವಾಸ್ತವಿ-ಕ-ವಾಗಿ
ವಾಸ್ತವಿ-ಕಾಂಶ
ವಾಸ್ತು-ಶಿ-ಲ್ಪ
ವಾಸ್ತು-ಶಿ-ಲ್ಪ-ಗಳು
ವಾಹ-ನಾದಿ-ಗಳ-ಲ್ಲಿ
ವಿ
ವಿಂಬ-ಲ್ಡ-ನ್
ವಿಂಬ-ಲ್ಡ-ನ್ನ-ಲ್ಲಿ-ರುವ
ವಿಂಬ-ಲ್ಡ-ನ್ನಿ-ನ-ಲ್ಲಿ
ವಿಕ-ಸನ
ವಿಕಸಿತ
ವಿಕಸಿಸು-ವುವು
ವಿಕಾಸ
ವಿಕಾಸ-ಗೊ-ಳ್ಳಲು
ವಿಕಾಸ-ಗೊಂಡು
ವಿಕಾಸ-ಗೊಳಿಸಿ
ವಿಕಾಸ-ದಿಂದ
ವಿಕಾಸ-ವಾ-ದಂತೆ
ವಿಕಾಸ-ವಾ-ದಷ್ಟು
ವಿಕಾಸ-ವಾ-ಯಿತು
ವಿಕಾಸ-ವಾಗ-ಬೇ-ಕಾ-ದರೆ
ವಿಕಾಸ-ವಾಗ-ಬೇಕು
ವಿಕಾಸ-ವಾಗಿ
ವಿಕಾಸ-ವಾಗು-ತ್ತಾ
ವಿಕಾಸ-ವಾಗು-ತ್ತಿರು-ವೆನೋ
ವಿಕಾಸ-ವಾಗು-ತ್ತಿರುವ
ವಿಕಾಸ-ವಾಗು-ವು-ದಕ್ಕೆ
ವಿಕಾಸ-ವಾಗು-ವುದು
ವಿಕಾಸ-ವಾದ-ವನ್ನು
ವಿಕಾಸ-ವುಳ್ಳ
ವಿಕಾಸಕ್ಕೆ
ವಿಕಾಸದ
ವಿಕಾಸವೇ
ವಿಕ್ಟೋ-ರಿಯ
ವಿಕ್ಟೋ-ರಿಯಾ
ವಿಗ್ರ-ಹಾರಾ-ಧನೆ
ವಿಗ್ರ-ಹಾರಾ-ಧನೆ-ಯ-ಲ್ಲಿ
ವಿಗ್ರ-ಹಾರಾ-ಧನೆ-ಯನ್ನು
ವಿಗ್ರ-ಹಾರಾಧ-ಕರ
ವಿಗ್ರ-ಹಾರಾಧ-ಕರು
ವಿಗ್ರ-ಹಾರಾಧ-ಕರೆ
ವಿಗ್ರ-ಹಾರಾಧ-ನೆಗೆ
ವಿಗ್ರ-ಹಾರಾಧ-ನೆಯ
ವಿಗ್ರ-ಹಾರಾಧಕ-ರಾಗಿ-ರುವರು
ವಿಗ್ರಗಹಳೇ
ವಿಗ್ರಹ
ವಿಗ್ರಹ-ಗಳು
ವಿಗ್ರಹ-ಗಳೆ-ಲ್ಲ
ವಿಗ್ರಹ-ದ-ಲ್ಲಿ
ವಿಗ್ರಹ-ದಂತೆ
ವಿಗ್ರಹ-ವನ್ನು
ವಿಗ್ರಹ-ವಾಗಿ-ರ-ಲಿ-ಲ್ಲ
ವಿಗ್ರಹ-ವೇನೋ
ವಿಗ್ರಹಕ್ಕೆ
ವಿಗ್ರಹದ
ವಿಗ್ರಹೋಪಾ-ಸನೆ-ಯನ್ನು
ವಿಗ್ರಾ-ಹಾರಾಧ-ನೆ-ಯಿಂದಲೇ
ವಿಚಾ-ರಕ್ಕೆ
ವಿಚಾ-ರಿಸಿ-ಕೊ-ಳ್ಳು-ತ್ತಿದ್ದರು
ವಿಚಾ-ರಿಸಿ-ಕೊಂಡು
ವಿಚಾ-ರಿಸಿ-ದಳು
ವಿಚಾ-ರಿಸಿ-ದಾಗ
ವಿಚಾ-ರಿಸಿದ
ವಿಜ-ಯರು
ವಿಜಯ
ವಿಜಯ-ಕೃಷ್ಣ
ವಿಜಯ-ದ್ವಾರ-ವನ್ನು
ವಿಜಯ-ರ-ಲ್ಲಿ
ವಿಜಯ-ವಾಗು-ವು-ದಕ್ಕೆ
ವಿಜೃಂಭಣೆ-ಯಿಂದ
ವಿಟ-ಮನ್
ವಿಡಂಬನ
ವಿಡಂಬನೆ
ವಿದಂತಿ
ವಿದ್ಯ-ತ್
ವಿದ್ಯತೇ
ವಿದ್ಯಾ
ವಿದ್ಯಾ-ದಾನ
ವಿದ್ಯಾ-ಧಿ-ದೇವತೆ
ವಿದ್ಯಾ-ಪ್ರ-ಚಾರ
ವಿದ್ಯಾ-ಪ್ರ-ಚಾರ-ವಾಗು-ತ್ತಿರುವ
ವಿದ್ಯಾ-ಪ್ರ-ಚಾರದ
ವಿದ್ಯಾ-ಬುದ್ಧಿ
ವಿದ್ಯಾ-ಭ್ಯಾಸ
ವಿದ್ಯಾ-ಭ್ಯಾಸ-ವನ್ನು
ವಿದ್ಯಾ-ಭ್ಯಾಸ-ವಿ-ಲ್ಲ
ವಿದ್ಯಾ-ಭ್ಯಾಸಕ್ಕೆ
ವಿದ್ಯಾ-ಭ್ಯಾಸದ
ವಿದ್ಯಾ-ಮಾಯೆ
ವಿದ್ಯಾ-ಮಾಯೆ-ಯನ್ನು
ವಿದ್ಯಾ-ಮಾಯೆಗೆ
ವಿದ್ಯಾ-ಮಾಯೆಯ
ವಿದ್ಯಾ-ಮಾಯೆಯೇ
ವಿದ್ಯಾ-ಲಯ-ಗಳಿಂದ
ವಿದ್ಯಾ-ವಂ-ತ-ನಾದ
ವಿದ್ಯಾ-ವಂ-ತ-ನಾದರೂ
ವಿದ್ಯಾ-ವಂ-ತ-ರಾಗಿ
ವಿದ್ಯಾ-ವಂತ-ನನ್ನು
ವಿದ್ಯಾ-ವಂತ-ನಾ-ದು-ದ-ರಿಂದ
ವಿದ್ಯಾ-ವಂತ-ನಾಗಿ
ವಿದ್ಯಾ-ವಂತ-ನಾಗಿ-ದ್ದರೂ
ವಿದ್ಯಾ-ವಂತ-ರ-ನ್ನಾಗಿ
ವಿದ್ಯಾ-ವಂತ-ರ-ನ್ನೆ-ಲ್ಲ
ವಿದ್ಯಾ-ವಂತ-ರಾಗುವ-ವ-ರೆಗೆ
ವಿದ್ಯಾ-ವಂತ-ರಾದ
ವಿದ್ಯಾ-ವಂತ-ರಿ-ದ್ದರು
ವಿದ್ಯಾ-ವಂತ-ರು-ಗಳೆ-ಲ್ಲರೂ
ವಿದ್ಯಾ-ವಂತ-ಳಾದ
ವಿದ್ಯಾ-ವಂತರು
ವಿದ್ಯಾ-ವಂತರೂ
ವಿದ್ಯಾ-ವಂತರೆ
ವಿದ್ಯಾ-ವ್ಯಾ-ಸಂಗಕ್ಕೆ
ವಿದ್ಯಾ-ಶ್ರಮ-ದ-ಲ್ಲಿ
ವಿದ್ಯಾ-ಸಂ-ಸ್ಥೆಗೆ
ವಿದ್ಯಾ-ಸಾ-ಗರ
ವಿದ್ಯಾ-ಸಾ-ಗರನ
ವಿದ್ಯಾಂ-ಸ-ರಿಗೆ
ವಿದ್ಯಾಂಸ-ರೊಬ್ಬರು
ವಿದ್ಯಾರ್ಥಿ
ವಿದ್ಯಾರ್ಥಿ-ಗ-ಳನ್ನು
ವಿದ್ಯಾರ್ಥಿ-ಗ-ಳಿಗೆ
ವಿದ್ಯಾರ್ಥಿ-ಗಳ
ವಿದ್ಯಾರ್ಥಿ-ಗಳಾ-ಗಿ-ದ್ದರು
ವಿದ್ಯಾರ್ಥಿ-ಗಳು
ವಿದ್ಯಾರ್ಥಿ-ಗಳೆ-ಲ್ಲ
ವಿದ್ಯಾರ್ಥಿ-ದೆಸೆ-ಯ-ಲ್ಲಿರು-ವಾಗಲೇ
ವಿದ್ಯಾರ್ಥಿ-ನಿ-ಯೊಬ್ಬಳು
ವಿದ್ಯಾರ್ಥಿ-ಯಾಗಿದ್ದ
ವಿದ್ಯಾರ್ಥಿ-ಯಾದ
ವಿದ್ಯಾರ್ಥಿ-ವರ್ಗ-ದ-ಲ್ಲಿ
ವಿದ್ಯಾರ್ಥಿ-ವರ್ಗ-ದ-ವ-ರಿ-ಗೆ-ಲ್ಲ
ವಿದ್ಯಾರ್ಥಿ-ವೃಂದದ
ವಿದ್ಯಾರ್ಥಿಗೆ
ವಿದ್ಯಾರ್ಥಿಯ
ವಿದ್ಯು-ತ್
ವಿದ್ಯು-ತ್ಶಕ್ತಿ-ಯಾಗಲಿ
ವಿದ್ಯು-ತ್ಶಕ್ತಿಯ
ವಿದ್ಯು-ನ್ಮಯ
ವಿದ್ಯುಚ್ಛಕ್ತಿ-ಯನ್ನು
ವಿದ್ಯೆ
ವಿದ್ಯೆ-ಗಳಿಗ-ಲ್ಲ
ವಿದ್ಯೆ-ಯ-ನ್ನೆ-ಲ್ಲಾ
ವಿದ್ಯೆ-ಯ-ಲ್ಲಿ
ವಿದ್ಯೆ-ಯನ್ನು
ವಿದ್ಯೆ-ಯನ್ನೂ
ವಿದ್ಯೆ-ಯನ್ನೇ
ವಿದ್ಯೆ-ಯಾಗಲಿ
ವಿದ್ಯೆ-ಯಿಂದ
ವಿದ್ಯೆಗೂ
ವಿದ್ಯೆಗೆ
ವಿದ್ಯೆಯ
ವಿದ್ಯೆಯೇ
ವಿದ್ರಾವಕ-ವಾಗು-ವಂತೆ
ವಿದ್ವಜ್ಜ-ನರು
ವಿದ್ವಜ್ಜ-ನರೆದು-ರಿ-ಗಿ-ಟ್ಟು
ವಿದ್ವತ್
ವಿದ್ವತ್ತನ್ನು
ವಿದ್ವತ್ತಿ-ನ-ಲ್ಲಿ
ವಿದ್ವತ್ತಿಗೆ
ವಿದ್ವತ್ತು
ವಿದ್ವತ್ಪೂರ್ಣ-ವಾಗಿ
ವಿದ್ವನ್
ವಿದ್ವನ್ಮಣಿ-ಗಳ
ವಿದ್ವನ್ಮಣಿ-ಗಳೆ-ಲ್ಲ
ವಿದ್ವಾಂ-ಸನೂ
ವಿದ್ವಾಂ-ಸರು
ವಿದ್ವಾಂಸ
ವಿದ್ವಾಂಸ-ನಂತೆ
ವಿದ್ವಾಂಸ-ನಾ-ದರೊ
ವಿದ್ವಾಂಸ-ನೊಬ್ಬ
ವಿದ್ವಾಂಸ-ರ-ನ್ನು
ವಿದ್ವಾಂಸ-ರ-ನ್ನೆ-ಲ್ಲ
ವಿದ್ವಾಂಸ-ರಾಗಿ-ದ್ದರು
ವಿದ್ವಾಂಸ-ರಾಗಿ-ದ್ದರೂ
ವಿದ್ವಾಂಸ-ರಾದ
ವಿದ್ವಾಂಸ-ರಿಗೆ
ವಿದ್ವಾಂಸ-ರು-ಗಳು
ವಿದ್ವಾಂಸ-ರೆಂದು
ವಿದ್ವಾಂಸ-ರೊ-ಡನೆ
ವಿದ್ವಾಂಸರ
ವಿದ್ವಾಂಸರೂ
ವಿಧ-ಗಳಗೆ
ವಿಧ-ದ-ಲ್ಲಿ
ವಿಧ-ದಿಂದಲೂ
ವಿಧ-ವಾಗಿ
ವಿಧ-ವಾಗಿ-ದ್ದು-ದ-ನ್ನು
ವಿಧ-ವಾದ
ವಿಧ-ವಿಧ-ವಾಗಿ
ವಿಧ-ವೆಯರ
ವಿಧ-ವೆಯಾದ
ವಿಧದ
ವಿಧದ-ಲ್ಲಿಯೂ
ವಿಧವೆ-ಯ-ರ-ನ್ನು
ವಿಧವೆ-ಯ-ರಿಗೆ
ವಿಧವೆ-ಯರು
ವಿಧವೆ-ಯಾಗಿದ್ದು
ವಿಧಾ-ತನು
ವಿಧಾತ
ವಿಧಿ
ವಿಧಿ-ಯನ್ನು
ವಿಧಿ-ಯಾಗಿ-ತ್ತು
ವಿಧಿ-ಯಿ-ಲ್ಲ
ವಿಧಿ-ಯಿ-ಲ್ಲದೆ
ವಿಧಿ-ಯುಕ್ತ
ವಿಧಿ-ವತ್ತಾಗಿ
ವಿಧಿ-ವಿಲಾಸ
ವಿಧಿ-ಸು-ತ್ತಿದ್ದ
ವಿಧಿಯೇ
ವಿಧೇಯ-ರಾಗಿ
ವಿಧೇಯತೆ
ವಿನ-ಮ್ರ-ವಾಗಿ-ದ್ದವು
ವಿನ-ಯ-ಕೃಷ್ಣ-ದೇವ್
ವಿನ-ಯ-ಸಂಪ-ನ್ನರು
ವಿನಃ
ವಿನಯ
ವಿನಾ
ವಿನಾ-ಶ-ಸ್ತ-ಸ್ಯ
ವಿನಾ-ಶಹೇತು
ವಿನಿ-ಯೋಗಿ-ಸು-ತ್ತೇನೆ
ವಿನಿ-ಯೋಗಿ-ಸುವೆನು
ವಿನೋದ
ವಿನೋದ-ಪ್ರಿಯ
ವಿನೋದ-ಪ್ರಿಯರು
ವಿನೋದ-ವಾ-ಗಿದೆ
ವಿನೋನಿಯ
ವಿನ್ಯಾಸವೂ
ವಿನ್ಸಿ
ವಿಪ-ತ್ತಿಗೂ
ವಿಪುಲ-ವಾದ
ವಿಫಲ-ಳಾದಗ
ವಿಫಲ-ವಾ-ಯಿತು
ವಿಮೋ-ಚನೆ-ಯನ್ನು
ವಿಮೋಚ-ನೆಗೆ
ವಿರ-ಸನ-ವನ್ನೆ-ಲ್ಲ
ವಿರಳ
ವಿರಾ-ಟ್
ವಿರಾ-ಟ್-ಚೈ-ತನ್ಯದ
ವಿರಾ-ಡ್ರೂಪದ
ವಿರುದ್ಧ
ವಿರುದ್ಧ-ವಾಗಿ
ವಿರುದ್ಧ-ವಾದ
ವಿರೋ-ಧ-ದಿಂದ
ವಿರೋ-ಧ-ವಾ-ಗಿದೆ
ವಿರೋ-ಧ-ವಾಗಿ
ವಿರೋ-ಧ-ವಾಗಿ-ದ್ದರು
ವಿರೋ-ಧ-ವಾಗಿ-ರು-ವುದೇ
ವಿರೋ-ಧ-ವಾಗಿ-ರುವ
ವಿರೋ-ಧ-ವಾಗಿವೆ
ವಿರೋ-ಧ-ವಾದ
ವಿರೋ-ಧವೇ
ವಿರೋ-ಧಾಬಾಸ-ಗಳು
ವಿರೋ-ಧಾಭಾಸ-ದಂತೆ
ವಿರೋ-ಧಿ-ಗ-ಳಾದ
ವಿರೋ-ಧಿ-ಗಳೊ-ಬ್ಬ-ರ-ನ್ನು
ವಿರೋ-ಧಿ-ಸಿ-ದರು
ವಿರೋ-ಧಿ-ಸಿ-ದರೂ
ವಿರೋ-ಧಿ-ಸಿ-ದರೆ
ವಿರೋ-ಧಿ-ಸು-ತ್ತಿ-ರ-ಲಿ-ಲ್ಲ
ವಿರೋ-ಧಿ-ಸು-ತ್ತಿ-ರುವರು
ವಿರೋ-ಧಿ-ಸು-ತ್ತೇನೆ
ವಿರೋ-ಧಿ-ಸು-ವು-ದಿ-ಲ್ಲ
ವಿರೋ-ಧಿಸ-ಬೇಕು
ವಿರೋಧ
ವಿಲ-ಕ್ಷಣಾರ್ಥ-ದ-ಲ್ಲಿ
ವಿಲಾ-ಯ-ತಿಗೆ
ವಿಲಾಸ-ಭೂಮಿ
ವಿಳಂಬ
ವಿಳಾಸ-ವನ್ನು
ವಿಳಾಸಕ್ಕೆ
ವಿವ-ರಿ-ಸಲು
ವಿವ-ರಿ-ಸುವಳು
ವಿವಾ-ಕಾ-ನಂದ-ರ-ನ್ನು
ವಿವಾಹ
ವಿವಾಹ-ಪದ್ಧತಿ-ಗಳು
ವಿವಿಧ
ವಿವೇ-ಕದ
ವಿವೇ-ಚನಾ-ಶಕ್ತಿ-ಯಿಂದ
ವಿವೇ-ಚನೆ
ವಿವೇಕ
ವಿವೇಕಾ-ನಂ-ದರು
ವಿವೇಕಾ-ನಂದ
ವಿವೇಕಾ-ನಂದ-ರ-ನ್ನು
ವಿವೇಕಾ-ನಂದ-ರ-ಲ್ಲಿ
ವಿವೇಕಾ-ನಂದ-ರಂತಹ
ವಿವೇಕಾ-ನಂದ-ರಂತಹ-ವರು
ವಿವೇಕಾ-ನಂದ-ರಾ-ದರು
ವಿವೇಕಾ-ನಂದ-ರಾ-ದರೊ
ವಿವೇಕಾ-ನಂದ-ರಾ-ದರೋ
ವಿವೇಕಾ-ನಂದ-ರಾದ
ವಿವೇಕಾ-ನಂದ-ರಿ-ರುವರು
ವಿವೇಕಾ-ನಂದ-ರಿಗೆ
ವಿವೇಕಾ-ನಂದ-ರೊ-ಡನೆ
ವಿವೇಕಾ-ನಂದರ
ವಿವೇಕಾನಂ-ದರೂ
ವಿವೇಕಾನಂ-ದರೇ
ವಿವೇಕಿ
ವಿಶದ-ವಾಗಿ
ವಿಶದ-ವಾದ
ವಿಶಾದ-ವಾ-ಗಿದೆ
ವಿಶಾಲ
ವಿಶಾಲ-ಗೊಳಿಸ-ಬೇಕೆಂ-ದಾಗ
ವಿಶಾಲ-ವ-ನ್ನಾಗಿ
ವಿಶಾಲ-ವಾ-ಗಿದೆ
ವಿಶಾಲ-ವಾಗ-ಲಾ-ರದು
ವಿಶಾಲ-ವಾಗಿ
ವಿಶಾಲ-ವಾಗಿ-ತ್ತು
ವಿಶಾಲ-ವಾಗಿ-ರ-ಬೇಕು
ವಿಶಾಲ-ವಾಗಿವೆ
ವಿಶಾಲ-ವಾಗು-ತ್ತ
ವಿಶಾಲ-ವಾದ
ವಿಶಿಷ್ಟಾ-ದ್ವೈತ
ವಿಶಿಷ್ಟಾ-ದ್ವೈತ-ಗಳ-ನ್ನೆ-ಲ್ಲ
ವಿಶಿಷ್ಟಾ-ದ್ವೈತಿ-ಗಳು
ವಿಶಿಷ್ಟಾ-ದ್ವೈತಿ-ಗಳೆ
ವಿಶೇಷ
ವಿಶೇಷ-ಗಳು
ವಿಶೇಷ-ವನ್ನು
ವಿಶೇಷ-ವಾಗಿ
ವಿಶೇಷ-ವಾದ
ವಿಶ್ರಾಂ-ತಿಗೂ
ವಿಶ್ರಾಂ-ತಿಗೆ
ವಿಶ್ರಾಂತಿ
ವಿಶ್ರಾಂತಿ-ಗಾಗಿ
ವಿಶ್ರಾಂತಿ-ಯನ್ನು
ವಿಶ್ವ
ವಿಶ್ವ-ಕಾರುಣ್ಯ-ವನ್ನು
ವಿಶ್ವ-ಕೋಟಿ
ವಿಶ್ವ-ಕೋಶ-ದಂತೆ
ವಿಶ್ವ-ಜನ-ತೆಯ
ವಿಶ್ವ-ದ-ಲ್ಲಿ
ವಿಶ್ವ-ದ-ಲ್ಲಿ-ರುವ
ವಿಶ್ವ-ದ-ಲ್ಲೆ-ಲ್ಲ
ವಿಶ್ವ-ದ-ಲ್ಲೆ-ಲ್ಲಾ
ವಿಶ್ವ-ದೊಂದಿಗೆ
ವಿಶ್ವ-ಧರ್ಮ
ವಿಶ್ವ-ಧರ್ಮ-ವನ್ನು
ವಿಶ್ವ-ಧರ್ಮ-ಸಮ್ಮೇಳ-ನ-ಕ್ಕಾಗಿ
ವಿಶ್ವ-ಧರ್ಮ-ಸಮ್ಮೇಳ-ನ-ದ-ಲ್ಲಿ
ವಿಶ್ವ-ಧರ್ಮ-ಸಮ್ಮೇಳ-ನ-ವನ್ನು
ವಿಶ್ವ-ಧರ್ಮ-ಸಮ್ಮೇಳ-ನದ
ವಿಶ್ವ-ಧರ್ಮದ
ವಿಶ್ವ-ನಾಥ
ವಿಶ್ವ-ನಾಥ-ದ-ತ್ತ
ವಿಶ್ವ-ನಾಥ-ದ-ತ್ತ-ನನ್ನು
ವಿಶ್ವ-ನಾಥ-ದ-ತ್ತ-ನಿಗೆ
ವಿಶ್ವ-ನಾಥ-ದ-ತ್ತನ
ವಿಶ್ವ-ನಾಥ-ನನ್ನು
ವಿಶ್ವ-ನಾಥ-ನಿಗೆ
ವಿಶ್ವ-ನಾಥನ
ವಿಶ್ವ-ನಾಥನು
ವಿಶ್ವ-ಪ್ರೇಮಕ್ಕೆ
ವಿಶ್ವ-ಮೇಳ
ವಿಶ್ವ-ಮೇಳ-ವನ್ನು
ವಿಶ್ವ-ವನ್ನು
ವಿಶ್ವ-ವನ್ನೇ
ವಿಶ್ವ-ವಿದ್ಯಾ-ನಿಲ-ಯದ
ವಿಶ್ವ-ವಿದ್ಯಾ-ನಿಲಯ-ಗ-ಳನ್ನು
ವಿಶ್ವ-ವಿದ್ಯಾ-ನಿಲಯ-ಗಳ-ಲ್ಲಿ
ವಿಶ್ವ-ವಿದ್ಯಾ-ನಿಲಯ-ದ-ಲ್ಲಿ
ವಿಶ್ವ-ವೆ-ಲ್ಲ
ವಿಶ್ವ-ವ್ಯಾಪಿ-ಯಾ-ಗಿದೆ
ವಿಶ್ವ-ವ್ಯಾಪಿ-ಯಾಗ-ಲಾ-ರದು
ವಿಶ್ವ-ಶಕ್ತಿ
ವಿಶ್ವಕ್ಕೆ
ವಿಶ್ವದ
ವಿಶ್ವವೇ
ವಿಶ್ವಾ-ತ್ಮನ
ವಿಶ್ವಾ-ಮಿ-ತ್ರ
ವಿಶ್ವಾ-ಮಿ-ತ್ರನ
ವಿಶ್ವಾ-ಮಿ-ತ್ರರ
ವಿಶ್ವಾನುಕಂಪ
ವಿಶ್ವಾಸ
ವಿಶ್ವಾಸ-ಗ-ಳನ್ನು
ವಿಶ್ವಾಸ-ಗ-ಳಿಗೆ
ವಿಶ್ವಾಸ-ಗ-ಳಿದ್ದರೆ
ವಿಶ್ವಾಸ-ಗಳ-ನ್ನೆ-ಲ್ಲ
ವಿಶ್ವಾಸ-ಗಳಿಂದ
ವಿಶ್ವಾಸ-ಗಳಿರ-ಬೇಕು
ವಿಶ್ವಾಸ-ದಿಂದ
ವಿಶ್ವಾಸ-ವನ್ನು
ವಿಶ್ವಾಸ-ವಿ-ಟ್ಟು
ವಿಶ್ವಾಸ-ವಿದೆ
ವಿಶ್ವಾಸಕ್ಕೂ
ವಿಶ್ವಾಸಕ್ಕೆ
ವಿಶ್ವಾಸದ
ವಿಶ್ವೇಶ್ವರ
ವಿಶ್ವೇಶ್ವರ-ದೇವ-ಸ್ಥಾನ
ವಿಷ-ಕ್ರಿ-ಮಿ-ಗಳು
ವಿಷ-ಯಾನ್
ವಿಷ-ವತ್
ವಿಷ-ವನ್ನು
ವಿಷ-ವನ್ನೂ
ವಿಷ-ವಿ-ರ-ಲಿ-ಲ್ಲ
ವಿಷಯ
ವಿಷಯ-ಗ-ಳನ್ನು
ವಿಷಯ-ಗ-ಳಿಗೆ
ವಿಷಯ-ಗಳ
ವಿಷಯ-ಗಳ-ನ್ನಾ-ದರೂ
ವಿಷಯ-ಗಳ-ನ್ನೂ
ವಿಷಯ-ಗಳ-ನ್ನೆ-ಲ್ಲ
ವಿಷಯ-ಗಳ-ಲ್ಲಿ
ವಿಷಯ-ಗಳಾಗಿ-ರ-ಲಿ-ಲ್ಲ
ವಿಷಯ-ಗಳಿ-ಗೆ-ಲ್ಲ
ವಿಷಯ-ಗಳಿವೆ
ವಿಷಯ-ಗಳು
ವಿಷಯ-ದ-ಲ್ಲಿ
ವಿಷಯ-ದ-ಲ್ಲಿಂದು
ವಿಷಯ-ದ-ಲ್ಲಿಯೂ
ವಿಷಯ-ವ-ಲ್ಲದೆ
ವಿಷಯ-ವನ್ನರಿ-ತಿ-ದ್ದರು
ವಿಷಯ-ವನ್ನು
ವಿಷಯ-ವನ್ನೂ
ವಿಷಯ-ವನ್ನೆ-ಲ್ಲ
ವಿಷಯ-ವನ್ನೆ-ಲ್ಲಾ
ವಿಷಯ-ವನ್ನೇ
ವಿಷಯ-ವನ್ನೇಕೆ
ವಿಷಯ-ವಾಗದೇ
ವಿಷಯ-ವಾಗಿ
ವಿಷಯ-ವಾಗಿ-ತ್ತು
ವಿಷಯ-ವಾಗಿಯೂ
ವಿಷಯ-ವಾಗಿಯೇ
ವಿಷಯ-ವೆಂದು
ವಿಷಯ-ವೆಂಬು-ದ-ನ್ನು
ವಿಷಯ-ವೇ-ನೆಂ-ದರೆ
ವಿಷಯ-ಸಂಗ್ರಹದ
ವಿಷಯದ
ವಿಷಯವೂ
ವಿಷಯವೇ
ವಿಷಯಾ-ಕಾರ-ವನ್ನು
ವಿಷಯಾ-ಕಾರ-ವಾಗಿ-ರುವುದೇನೋ
ವಿಷುವದ್ರೇಖೆಗೆ
ವಿಷ್ಣು
ವಿಷ್ಣುವೊ
ವಿಷ್ವ-ನಾಥ-ದ-ತ್ತ
ವಿಹಂಗಮ
ವಿಹಂಗಮದ
ವಿಹ್ವಲ-ಗೊಂಡಿ-ದ್ದರು
ವೀ
ವೀರ-ನಂತೆ
ವೀರ-ನಾಗ-ಬೇಕು
ವೀರ-ಪುರುಷ
ವೀರ-ರ-ನ್ನು
ವೀರ-ವಾಣಿ-ಯ-ಲ್ಲಿ
ವೀರ-ವೆ-ಲ್
ವೀರೇಶ್ವರ
ವೀರೇಶ್ವರ-ನಿಗೆ
ವೀರೇಶ್ವರ-ನೆಂದು
ವೀರೇಶ್ವರನ
ವೀರೋದ್ಯಮದ
ವೀರ್ಯ-ಮಸಿ
ವೀರ್ಯ-ವತ್ತಾದ
ವೀರ್ಯ-ಹೀನ-ವಾ-ಗಿದೆ
ವೀರ್ಯಂ
ವೀಳ್ಯದೆಲೆ
ವುಡ್
ವೃ
ವೃಂ
ವೃಂದ
ವೃಂದ-ದ-ಲ್ಲಿ
ವೃಂದ-ದ-ವರು
ವೃಂದ-ವನ್ನು
ವೃಂದದ
ವೃಕ್ಷ
ವೃಕ್ಷ-ಗ-ಳನ್ನು
ವೃಕ್ಷ-ಗಳು
ವೃಕ್ಷ-ಚ್ಛಾಯೆ
ವೃಕ್ಷ-ದ-ಲ್ಲಿ
ವೃಕ್ಷ-ವನ್ನು
ವೃಕ್ಷ-ವಾಗಿ
ವೃಕ್ಷದ
ವೃಥಾ
ವೃದ್ಧ
ವೃದ್ಧ-ನಾ-ದರೋ
ವೃದ್ಧ-ಳಾದ
ವೃದ್ಧ-ಸಚ್ಚಿ-ದಾನಂದ
ವೃದ್ಧರ
ವೃದ್ಧರು
ವೃದ್ಧಾ-ಚಾರ-ಗಳ-ಲ್ಲೇ
ವೃದ್ಧಾಪ್ಯ-ದ-ಲ್ಲಿ
ವೃದ್ಧಾಪ್ಯಕ್ಕೆ
ವೃದ್ಧಿ
ವೃದ್ಧಿ-ಮಾಡಿ-ಕೊ-ಳ್ಳು-ವು-ದ-ಕ್ಕಾಗಿ
ವೃದ್ಧಿ-ಯಾ-ಯಿತು
ವೃದ್ಧಿ-ಯಾಗು-ತ್ತ
ವೃದ್ಧಿ-ಯಾಗುವಂತೆ
ವೃದ್ಧಿ-ಯಾಗುವುದು
ವೃಷ-ಭಾರೂಢ
ವೆ
ವೆಚ್ಚ
ವೆಚ್ಚ-ವನ್ನು
ವೆಚ್ಚ-ವನ್ನೆ-ಲ್ಲ
ವೆಚ್ಚ-ವಾಗು-ತ್ತಿ-ತ್ತು
ವೆಚ್ಚ-ವಾಗುವ
ವೆಚ್ಚ-ವೆ-ಲ್ಲ
ವೆನಿ-ಗರ್
ವೆಸೂವಿಯ-ಸ್
ವೇ
ವೇಗ-ದ-ಲ್ಲಿ
ವೇಗ-ದಿಂದ
ವೇಗ-ವನ್ನೆ-ಲ್ಲ
ವೇಗ-ವಾಗಿ
ವೇಗ-ವಾದ
ವೇದ
ವೇದ-ಗ-ಳನ್ನು
ವೇದ-ಗಳ
ವೇದ-ಗಳ-ಲ್ಲಿ
ವೇದ-ಗಳ-ಲ್ಲಿ-ರುವ
ವೇದ-ಗಳು
ವೇದ-ಗಳೇ
ವೇದ-ಘೋಷದ
ವೇದ-ಜ್ಞಾನ
ವೇದ-ದ-ಲ್ಲಿ
ವೇದ-ದ-ಲ್ಲಿ-ರುವ
ವೇದ-ದ-ಲ್ಲಿಯೂ
ವೇದ-ನೆಯ
ವೇದ-ನೆಯ-ನ್ನುಂಟು
ವೇದ-ನೆಯುಂಟಾ-ಯಿತು
ವೇದ-ನೆಯೂ
ವೇದ-ಮಂ-ತ್ರ-ಗಳು
ವೇದ-ವನ್ನು
ವೇದ-ವೇದಾಂತ-ಗಳ
ವೇದ-ಶಾ-ಸ್ತ್ರ-ಜ್ಞರಾದ
ವೇದದ
ವೇದಾಂ-ತಕ್ಕೆ
ವೇದಾಂ-ತದ
ವೇದಾಂತ
ವೇದಾಂತ-ಗಳ
ವೇದಾಂತ-ದ-ಲ್ಲಿ
ವೇದಾಂತ-ದ-ಲ್ಲಿ-ರುವ
ವೇದಾಂತ-ದಂತೆಯೆ
ವೇದಾಂತ-ದರ್ಶನ
ವೇದಾಂತ-ದಿಂದ
ವೇದಾಂತ-ಬೋ-ಧನೆ
ವೇದಾಂತ-ಭಾವ-ನೆಯ
ವೇದಾಂತ-ವನ್ನು
ವೇದಾಂತ-ವಿದ್ದರೆ
ವೇದಾಂತವೇ
ವೇದಾಂತಿ
ವೇದಾಂತಿ-ಗ-ಳಾಗು-ವರು
ವೇದಾಂತಿ-ಗಳ
ವೇದಾಂತಿ-ಗಳ-ಲ್ಲಿ
ವೇದಾಂತಿ-ಗಳು
ವೇದಾಂತಿ-ಗಳೂ
ವೇದಾಂತಿ-ಗಳೆ-ಲ್ಲಾ
ವೇದಾಂತಿ-ಯಾದ
ವೇದಾಧ್ಯಯ-ನಕ್ಕೆ
ವೇದಿ-ಕೆ-ಯ-ಮೇಲೆ
ವೇದಿ-ಕೆಯ
ವೇದಿಕೆ
ವೇದೋಕ್ತಿ-ಯಿಂದ
ವೇದೋಕ್ತಿಗೆ
ವೇದೋಪನಿಷ-ತ್ತು
ವೇದ್ಯ
ವೇದ್ಯ-ವಾ-ಯಿತು
ವೇದ್ಯ-ವಾಗಿ-ತ್ತು
ವೇದ್ಯ-ವಾಗು-ತ್ತದೆ
ವೇದ್ಯ-ವಾಗು-ವಂತೆ
ವೇದ್ಯ-ವಾಗು-ವುದು
ವೇಳೆ
ವೇಳೆ-ಯ-ಲ್ಲಿ
ವೇಳೆಗೆ
ವೇಳೆಯೂ
ವೇಶ್ಯಾ
ವೇಶ್ಯಾಂಗ-ನೆ-ಯರು
ವೇಶ್ಯಾಂಗ-ನೆಯ
ವೇಶ್ಯೆ
ವೇಶ್ಯೆ-ಯರಿ-ರುವ
ವೇಷ
ವೇಷ-ದ-ಲ್ಲಿ
ವೇಷ-ದ-ಲ್ಲಿಯೇ
ವೇಷ-ಭೂಷಣ-ಗ-ಳನ್ನು
ವೇಷ-ಭೂಷಣ-ಗಳ-ನ್ನೆ-ಲ್ಲ
ವೇಷ-ಭೂಷಣ-ಗಳಿಂದ
ವೇಷ-ಭೂಷಾಣಾ-ದಿ-ಗಳು
ವೇಷ-ವೆಂದು
ವೈ
ವೈಕುಂಠ
ವೈಕುಂಠ-ನಾಥ
ವೈಕುಂಠ-ನಾಥ-ರಿಗೆ
ವೈಕುಂಠ-ನಾಥ್
ವೈಖರಿ
ವೈದಿಕ
ವೈದಿಕ-ಧರ್ಮವು
ವೈದ್ಯ
ವೈದ್ಯ-ಕೀಯ
ವೈದ್ಯ-ನನ್ನು
ವೈದ್ಯ-ನಾಥ-ದ-ಲ್ಲಿದ್ದಾಗ
ವೈದ್ಯ-ನಾಥ-ದಿಂದ
ವೈದ್ಯ-ನಾಥಕ್ಕೆ
ವೈದ್ಯ-ನಾದ
ವೈದ್ಯ-ನಿ-ದ್ದನು
ವೈದ್ಯ-ನಿಗೆ
ವೈದ್ಯ-ರ-ನ್ನು
ವೈದ್ಯ-ರಾದ
ವೈದ್ಯ-ರಿಗೆ
ವೈದ್ಯ-ರು-ಗಳ-ಲ್ಲಿ
ವೈದ್ಯನ
ವೈದ್ಯನು
ವೈದ್ಯರ
ವೈದ್ಯರು
ವೈಭ-ವವೂ
ವೈಭವ
ವೈಭವ-ಗ-ಳಿದ್ದ-ರೇನು
ವೈಭವ-ಗಳ-ಲ್ಲಿ
ವೈಭವ-ದಿಂದ
ವೈಭವ-ಯುಕ್ತ-ವಾ-ಯಿತು
ವೈಭವ-ಯುಕ್ತ-ವಾದ
ವೈಭವ-ಯುತ-ವಾದ
ವೈಭವದ
ವೈಮೇ-ನ್
ವೈರಾಗ್ಯ
ವೈರಾಗ್ಯ-ಗಳ-ಲ್ಲದೇ
ವೈರಾಗ್ಯ-ದಿಂದ
ವೈರಾಗ್ಯ-ಪ್ರದ
ವೈರಾಗ್ಯ-ಮೇವಾ
ವೈರಾಗ್ಯ-ಮೇವಾ-ಭಯಂ
ವೈರಾಗ್ಯ-ವನ್ನು
ವೈರಾಗ್ಯ-ವೊಂದೇ
ವೈರಾಗ್ಯದ
ವೈರಿ
ವೈರಿ-ಗ-ಳನ್ನು
ವೈವಿಧ್ಯ-ತೆಯ
ವೈವಿಧ್ಯ-ಪೂರ್ಣ-ವಾದ
ವೈವಿಧ್ಯತೆ
ವೈವಿಧ್ಯತೆ-ಯ-ಲ್ಲಿ
ವೈಶಾ-ಲ್ಯದ
ವೈಶಿಷ್ಟ್ಯ
ವೈಶಿಷ್ಟ್ಯ-ದೊಂದಿಗೆ
ವೈಶಿಷ್ಟ್ಯ-ವನ್ನು
ವೈಶಿಷ್ಟ್ಯ-ವಿದೆ
ವೈಶ್ಯ-ಕುಲಕ್ಕೆ
ವೈಶ್ಯ-ರಿ-ಗೆ-ಲ್ಲ
ವೈಷ್ಣ-ವರ
ವೈಷ್ಣ-ವರು
ವೈಷ್ಣವ
ವೈಷ್ಣವ-ಮಿತಿ
ವೈಷ್ಣವ-ರ-ಲ್ಲಿ
ವೈಷ್ಣವ-ರಿಗೆ
ವೈಷ್ಣವ-ರೆ-ಲ್ಲರೂ
ವೈಷ್ಣವ-ರೆಂದು
ವ್ಯ
ವ್ಯಂಗ್ಯ-ವಾಗಿ
ವ್ಯಂಗ್ಯ-ವಾದ
ವ್ಯಂಗ್ಯೋಕ್ತಿ
ವ್ಯಂಜಕ-ವಾದ
ವ್ಯಕಿ-ಗ-ಳನ್ನು
ವ್ಯಕ್ತ
ವ್ಯಕ್ತ-ಗೊಳಿ-ಸದೆ
ವ್ಯಕ್ತ-ಗೊಳಿ-ಸು-ತ್ತಿದೆ
ವ್ಯಕ್ತ-ಗೊಳಿ-ಸು-ತ್ತಿವೆ
ವ್ಯಕ್ತ-ಗೊಳಿ-ಸು-ವು-ದಾಗಿದೆ
ವ್ಯಕ್ತ-ಗೊಳಿಸ-ಬೇಕೆಂಬುದೇ
ವ್ಯಕ್ತ-ಪ-ಡು-ತ್ತದೆ
ವ್ಯಕ್ತ-ಪಡಿ-ಸ-ಬ-ಲ್ಲೆ
ವ್ಯಕ್ತ-ಪಡಿ-ಸ-ಬ-ಲ್ಲೆಯಾ
ವ್ಯಕ್ತ-ಪಡಿ-ಸ-ಬೇಕು
ವ್ಯಕ್ತ-ಪಡಿ-ಸ-ಬೇಡ
ವ್ಯಕ್ತ-ಪಡಿ-ಸ-ಲಿ-ಲ್ಲ
ವ್ಯಕ್ತ-ಪಡಿ-ಸಲು
ವ್ಯಕ್ತ-ಪಡಿ-ಸಿ-ದನು
ವ್ಯಕ್ತ-ಪಡಿ-ಸಿ-ದರು
ವ್ಯಕ್ತ-ಪಡಿ-ಸಿ-ದಾಗ
ವ್ಯಕ್ತ-ಪಡಿ-ಸಿ-ದ್ದರು
ವ್ಯಕ್ತ-ಪಡಿ-ಸಿ-ದ್ದಳು
ವ್ಯಕ್ತ-ಪಡಿ-ಸಿ-ರಲಾರಳು
ವ್ಯಕ್ತ-ಪಡಿ-ಸಿ-ರು-ವಳು
ವ್ಯಕ್ತ-ಪಡಿ-ಸಿ-ರುವರು
ವ್ಯಕ್ತ-ಪಡಿ-ಸಿದ
ವ್ಯಕ್ತ-ಪಡಿ-ಸು-ತ್ತಿ-ತ್ತು
ವ್ಯಕ್ತ-ಪಡಿ-ಸು-ತ್ತಿ-ದ್ದರು
ವ್ಯಕ್ತ-ಪಡಿ-ಸು-ತ್ತಿ-ರ-ಲಿ-ಲ್ಲ
ವ್ಯಕ್ತ-ಪಡಿ-ಸು-ತ್ತಿದ್ದ
ವ್ಯಕ್ತ-ಪಡಿ-ಸು-ವಂತೆ
ವ್ಯಕ್ತ-ಪಡಿ-ಸು-ವು-ದ-ನ್ನು
ವ್ಯಕ್ತ-ಪಡಿ-ಸು-ವು-ದಿ-ಲ್ಲ
ವ್ಯಕ್ತ-ಪಡಿ-ಸು-ವುದು
ವ್ಯಕ್ತ-ಪಡಿ-ಸುವ
ವ್ಯಕ್ತ-ಪಡಿ-ಸುವ-ವರು
ವ್ಯಕ್ತ-ಪಡಿ-ಸುವರು
ವ್ಯಕ್ತ-ಪಡಿ-ಸುವುದ-ರ-ಲ್ಲಿ
ವ್ಯಕ್ತ-ಪಡಿ-ಸುವುದೂ
ವ್ಯಕ್ತ-ಪಡಿಸಿ
ವ್ಯಕ್ತ-ವಾ-ಗು-ತ್ತಿದೆ
ವ್ಯಕ್ತ-ವಾ-ದಾಗ
ವ್ಯಕ್ತ-ವಾಗ-ಬೇಕು
ವ್ಯಕ್ತ-ವಾಗ-ಲಾ-ರದು
ವ್ಯಕ್ತ-ವಾಗದ
ವ್ಯಕ್ತ-ವಾಗಲಿ
ವ್ಯಕ್ತ-ವಾಗಲು
ವ್ಯಕ್ತ-ವಾಗಿ-ದ್ದರೆ
ವ್ಯಕ್ತ-ವಾಗಿ-ರು-ತ್ತದೆ
ವ್ಯಕ್ತ-ವಾಗಿ-ಲ್ಲ
ವ್ಯಕ್ತ-ವಾಗಿ-ವೆ-ಯ-ಲ್ಲ
ವ್ಯಕ್ತ-ವಾಗಿವೆ
ವ್ಯಕ್ತ-ವಾಗು-ತ್ತವೆ
ವ್ಯಕ್ತ-ವಾಗು-ತ್ತಾನೆ
ವ್ಯಕ್ತ-ವಾಗು-ತ್ತಿ-ತ್ತು
ವ್ಯಕ್ತ-ವಾಗು-ತ್ತಿರುವ
ವ್ಯಕ್ತ-ವಾಗು-ತ್ತಿರುವೆ
ವ್ಯಕ್ತ-ವಾಗು-ವಂತೆ
ವ್ಯಕ್ತ-ವಾಗು-ವುದು
ವ್ಯಕ್ತ-ವಾಗು-ವುವು
ವ್ಯಕ್ತ-ಸ್ವ-ರೂಪ
ವ್ಯಕ್ತಿ
ವ್ಯಕ್ತಿ-ಗ-ಳನ್ನು
ವ್ಯಕ್ತಿ-ಗ-ಳಾದ
ವ್ಯಕ್ತಿ-ಗ-ಳಾದರೋ
ವ್ಯಕ್ತಿ-ಗ-ಳಿಗೆ
ವ್ಯಕ್ತಿ-ಗಳ
ವ್ಯಕ್ತಿ-ಗಳ-ಲ್ಲಿ
ವ್ಯಕ್ತಿ-ಗಳಂತೆ
ವ್ಯಕ್ತಿ-ಗಳಿ-ರ-ಲಿ-ಲ್ಲ
ವ್ಯಕ್ತಿ-ಗಳಿ-ರುವರು
ವ್ಯಕ್ತಿ-ಗಳು
ವ್ಯಕ್ತಿ-ಗಳೂ
ವ್ಯಕ್ತಿ-ಗಳೆ-ಲ್ಲ
ವ್ಯಕ್ತಿ-ಗಳೆ-ಲ್ಲ-ಕ್ಕಿಂ-ತಲೂ
ವ್ಯಕ್ತಿ-ಗಳೆಂದು
ವ್ಯಕ್ತಿ-ತ್ತ್ವ-ದ-ಲ್ಲಿ
ವ್ಯಕ್ತಿ-ತ್ತ್ವ-ವನ್ನು
ವ್ಯಕ್ತಿ-ತ್ವ
ವ್ಯಕ್ತಿ-ತ್ವ-ಕ್ಕ-ಲ್ಲ
ವ್ಯಕ್ತಿ-ತ್ವ-ದ-ಲ್ಲಿ
ವ್ಯಕ್ತಿ-ತ್ವ-ದ-ಲ್ಲೆ
ವ್ಯಕ್ತಿ-ತ್ವ-ದಿಂದ
ವ್ಯಕ್ತಿ-ತ್ವ-ದೊ-ಡನೆ
ವ್ಯಕ್ತಿ-ತ್ವ-ವನ್ನು
ವ್ಯಕ್ತಿ-ತ್ವ-ವನ್ನೆ-ಲ್ಲ
ವ್ಯಕ್ತಿ-ತ್ವ-ವಿ-ತ್ತು
ವ್ಯಕ್ತಿ-ತ್ವ-ವೆ-ಲ್ಲ
ವ್ಯಕ್ತಿ-ತ್ವಕ್ಕೆ
ವ್ಯಕ್ತಿ-ತ್ವದ
ವ್ಯಕ್ತಿ-ತ್ವವೇ
ವ್ಯಕ್ತಿ-ತ್ವವೋ
ವ್ಯಕ್ತಿ-ಯ-ಮೇಲೆ
ವ್ಯಕ್ತಿ-ಯ-ಲ್ಲಿ
ವ್ಯಕ್ತಿ-ಯ-ಲ್ಲಿ-ರುವ
ವ್ಯಕ್ತಿ-ಯಂತಿದ್ದ
ವ್ಯಕ್ತಿ-ಯಂತೆ
ವ್ಯಕ್ತಿ-ಯನ್ನು
ವ್ಯಕ್ತಿ-ಯನ್ನೇ
ವ್ಯಕ್ತಿ-ಯಾಗಿ
ವ್ಯಕ್ತಿ-ಯಾಗಿ-ರ-ಬೇಕು
ವ್ಯಕ್ತಿ-ಯೆಂದು
ವ್ಯಕ್ತಿ-ಯೆಂದೂ
ವ್ಯಕ್ತಿ-ಯೊ-ಡನೆ
ವ್ಯಕ್ತಿ-ಯೊಂ-ದ-ನ್ನು
ವ್ಯಕ್ತಿ-ಯೊಬ್ಬ-ನಿಗೆ
ವ್ಯಕ್ತಿಗೂ
ವ್ಯಕ್ತಿಗೆ
ವ್ಯಕ್ತಿಯ
ವ್ಯಕ್ತಿಯು
ವ್ಯಕ್ತಿಯೆ
ವ್ಯಥಿತ-ರಾಗುವಂತೆ
ವ್ಯಥೆ
ವ್ಯಥೆ-ಪ-ಟ್ಟನು
ವ್ಯಥೆ-ಪ-ಟ್ಟರು
ವ್ಯಥೆ-ಪ-ಡು-ತ್ತಿದ್ದರು
ವ್ಯಥೆ-ಪಟ್ಟಾಗ
ವ್ಯಥೆ-ಪಡ-ಲಿ-ಲ್ಲ
ವ್ಯಥೆ-ಪಡಲೇ
ವ್ಯಥೆ-ಪಡು-ತ್ತಿರುವ
ವ್ಯಥೆ-ಯ-ನ್ನುಂಟು-ಮಾಡಲು
ವ್ಯಥೆ-ಯಾ-ಯಿತು
ವ್ಯಥೆ-ಯಾಗಿ-ರ-ಬೇಕು
ವ್ಯಥೆ-ಯಾಗು-ವು-ದೆಂದು
ವ್ಯಥೆ-ಯಿಂದ
ವ್ಯಥೆಗೆ
ವ್ಯಭಿ-ಚಾರಿ-ಗಳು
ವ್ಯಯ
ವ್ಯಯ-ಮಾಡಲು
ವ್ಯಯ-ಮಾಡಿ-ದನು
ವ್ಯಯ-ಮಾಡು-ತ್ತಿರುವೆ
ವ್ಯಯ-ವಾ-ಗಿದೆ
ವ್ಯಯ-ವಾಗ-ಬೇಕು
ವ್ಯರ್ಥ
ವ್ಯರ್ಥ-ಮಾಡಿ-ಕೊಂಡಿ-ರುವೆ
ವ್ಯರ್ಥ-ಮಾಡಿ-ದಂತೆ
ವ್ಯರ್ಥ-ಮಾಡು-ತ್ತಿರು-ವೆವು
ವ್ಯರ್ಥ-ವಾಗ-ಕೂ-ಡದು
ವ್ಯರ್ಥ-ವಾಗ-ಲಿ-ಲ್ಲ
ವ್ಯರ್ಥ-ವಾಗ-ಲಿ-ಲ್ಲ-ವೆಂದು
ವ್ಯರ್ಥ-ವಾಗಿ
ವ್ಯರ್ಥ-ವಾಗು-ವು-ದಿ-ಲ್ಲ
ವ್ಯವ-ಸ್ಥಿತ
ವ್ಯವ-ಸ್ಥೆ
ವ್ಯವ-ಸ್ಥೆ-ಗೊಂಡಿ-ರ-ಲಿ-ಲ್ಲ
ವ್ಯವ-ಸ್ಥೆ-ಗೊಳಿಸಿ
ವ್ಯವ-ಸ್ಥೆ-ಗೊಳಿಸಿ-ದರು
ವ್ಯವ-ಹರಿ-ಸ-ತೊಡಗಿತು
ವ್ಯವ-ಹರಿ-ಸ-ಬೇಕು
ವ್ಯವ-ಹರಿ-ಸಿ-ದಿರಿ
ವ್ಯವ-ಹರಿ-ಸಿದ
ವ್ಯವ-ಹರಿ-ಸು-ತ್ತಿ-ರುವರು
ವ್ಯವ-ಹರಿ-ಸು-ವು-ದಕ್ಕೆ
ವ್ಯವ-ಹರಿ-ಸುವರು
ವ್ಯವ-ಹರಿ-ಸುವಾಗ
ವ್ಯವ-ಹರಿ-ಸುವೆ
ವ್ಯವ-ಹಾ-ರಕ್ಕೆ
ವ್ಯವ-ಹಾರ
ವ್ಯವ-ಹಾರ-ಗಳ
ವ್ಯವ-ಹಾರ-ಗಳಂತೂ
ವ್ಯವ-ಹಾರ-ಗಳಿಗೂ
ವ್ಯವ-ಹಾರ-ಗಳು
ವ್ಯವ-ಹಾರ-ಗಳೆ-ಲ್ಲ
ವ್ಯವ-ಹಾರ-ಗಳೊಂದಿಗೆ
ವ್ಯವ-ಹಾರ-ದ-ಲ್ಲಿ
ವ್ಯವ-ಹಾರ-ದ-ಲ್ಲೆ-ಲ್ಲ
ವ್ಯವ-ಹಾರ-ವನ್ನೆ-ಲ್ಲ
ವ್ಯವ-ಹಾರದ
ವ್ಯವಸಾ-ಯದ
ವ್ಯಾ
ವ್ಯಾಖ್ಯಾನ
ವ್ಯಾಖ್ಯಾನ-ವನ್ನು
ವ್ಯಾಘ್ರ-ಭಯ-ಗಳೆ-ಲ್ಲ
ವ್ಯಾಜ್ಯ
ವ್ಯಾಜ್ಯ-ವನ್ನು
ವ್ಯಾಧಿ-ಗ-ಳನ್ನು
ವ್ಯಾಧಿ-ಯಿಂದ
ವ್ಯಾಧಿಗೆ
ವ್ಯಾಪಾ-ರಕ್ಕೆ
ವ್ಯಾಪಾ-ರದ
ವ್ಯಾಪಾ-ರಿಗೆ
ವ್ಯಾಪಾರ
ವ್ಯಾಪಾರ-ಗಳೆ-ರಡೂ
ವ್ಯಾಪಾರ-ವನ್ನು
ವ್ಯಾಪಾರಿ
ವ್ಯಾಪಾರಿ-ಗಳು
ವ್ಯಾಪಿ-ಸ-ತೊಡಗಿತು
ವ್ಯಾಪಿ-ಸಿ-ಕೊ-ಳ್ಳು-ತ್ತಿ-ತ್ತು
ವ್ಯಾಪಿ-ಸಿ-ಕೊಂಡು
ವ್ಯಾಪಿ-ಸಿ-ದೆಯೆ
ವ್ಯಾಪಿ-ಸಿತು
ವ್ಯಾಪಿ-ಸಿದೆ
ವ್ಯಾಪಿ-ಸು-ವಂತೆ
ವ್ಯಾಪಿಸಿ
ವ್ಯಾವ-ಹಾರಿಕ
ವ್ಯಾಸ-ದೇವನ
ವ್ಯೂಹ
ವ್ರಣ-ದಿಂದ
ವ್ರಣ-ವಾಗಿ-ರು-ವು-ದ-ರಿಂದ
ವ್ರತ
ವ್ರತ-ಧಾರಿ-ಗ-ಳಾಗು-ವಂತೆ
ವ್ರತ-ಭ್ರಷ್ಟ-ರಾಗುವರೋ
ವ್ರತ-ವನ್ನು
ವ್ರಾ
ಶ
ಶಂ
ಶಂಖ
ಶಂಖ-ಧ್ವನಿಯೂ
ಶಂಖದ
ಶಂಭು-ನಾಥ-ನಿಗೆ
ಶಂಭು-ನಾಥಜಿ
ಶಕ್ತ-ರಾಗಿ-ಲ್ಲ
ಶಕ್ತಿ
ಶಕ್ತಿ-ಗಳಿವೆ
ಶಕ್ತಿ-ಗಳು
ಶಕ್ತಿ-ಗಳೊ-ಡನೆ
ಶಕ್ತಿ-ಗಿಂತ
ಶಕ್ತಿ-ಗುಂದುವ-ವ-ರೆಗೂ
ಶಕ್ತಿ-ಪಾತ-ವನ್ನು
ಶಕ್ತಿ-ಪೂಜೆಯ
ಶಕ್ತಿ-ಬಾಹುಳ್ಯ
ಶಕ್ತಿ-ಯ-ನ್ನಿ-ತ್ತು
ಶಕ್ತಿ-ಯ-ನ್ನೆ-ಲ್ಲ
ಶಕ್ತಿ-ಯ-ಲ್ಲ
ಶಕ್ತಿ-ಯ-ಲ್ಲಿ
ಶಕ್ತಿ-ಯ-ಲ್ಲಿ-ರುವ
ಶಕ್ತಿ-ಯನ್ನು
ಶಕ್ತಿ-ಯನ್ನೂ
ಶಕ್ತಿ-ಯಾಗಿ
ಶಕ್ತಿ-ಯಾಗುವಂತೆ
ಶಕ್ತಿ-ಯಿಂದ
ಶಕ್ತಿ-ಯಿಂದಲೆ
ಶಕ್ತಿ-ಯಿದೆ
ಶಕ್ತಿ-ಯೆ-ಲ್ಲ
ಶಕ್ತಿ-ಶಾಲಿ-ಗಳಾಗಿ-ದ್ದಾರೆ
ಶಕ್ತಿ-ಹೀನರು
ಶಕ್ತಿಗೆ
ಶಕ್ತಿಯ
ಶಕ್ತಿಯೂ
ಶಕ್ತಿಯೇ
ಶತ
ಶತ-ಮಾನ
ಶತ-ಮಾನ-ಗಳ
ಶತ-ಮಾನ-ಗಳ-ವ-ರೆಗೆ
ಶತ-ಮಾನ-ಗಳಿಂದ
ಶತ-ಮಾನ-ಗಳಿಂದಲೂ
ಶತ-ಮಾನ-ಗಳು
ಶತ-ಮಾನ-ದಿಂದ
ಶತ-ಮಾನದ
ಶತಪಥ
ಶತೃಂಜಯ
ಶನಿ
ಶನಿ-ವಾರ
ಶಪಥ
ಶಪಥ-ತೊ-ಟ್ಟರು
ಶಪಥ-ಮಾಡು
ಶಪಥ-ವನ್ನು
ಶಬ್ದ
ಶಬ್ದ-ಮಾಡಿ-ಕೊಂಡು
ಶಬ್ದ-ಮಾಡು-ವು-ದಕ್ಕೆ
ಶಬ್ಧದ
ಶಮ-ನಕ್ಕೆ
ಶಯ್ಯೆಯ
ಶರ-ಗಳಂತೆ
ಶರ-ತ್
ಶರ-ತ್ಗೆ
ಶರ-ತ್ಚಂದ್ರ
ಶರ-ತ್ಚಂದ್ರ-ಗುಪ್ತ
ಶರ-ತ್ಚಂದ್ರ-ಗುಪ್ತ-ನಿಗೆ
ಶರ-ತ್ಚಂದ್ರ-ನಿ-ಗಾಗಿ
ಶರ-ತ್ಚಂದ್ರ-ನಿಗೆ
ಶರ-ತ್ಚಂದ್ರ-ನೊ-ಡನೆ
ಶರ-ತ್ಚಂದ್ರ-ರ-ನ್ನು
ಶರ-ತ್ಚಂದ್ರ-ಸರ್ಕಾರರು
ಶರ-ತ್ಚಂದ್ರನು
ಶರ-ತ್ತನ್ನೂ
ಶರ-ತ್ಶಾ-ರದಾ-ನಂದ
ಶರ-ವನ್ನು
ಶರಚ್ಚಂದ್ರ-ನಿಗೆ
ಶರಣಾ-ಗತ-ನಾ-ದರೆ
ಶರಣಾ-ಗತ-ನಾಗಿ-ರು-ವೆನು
ಶರಣಾ-ಗತ-ರಾಗಲು
ಶರಣಾ-ಗತಿ
ಶರಣಾ-ದರು
ಶರಣಾ-ದರೆ
ಶರಣಾಗಿ-ದ್ದರೂ
ಶರಣಾಗಿ-ರು-ವೆನು
ಶರಣಾಗು
ಶರಣಾಗು-ವುದು
ಶರಾಯಿ
ಶರಾಯಿ-ಯನ್ನು
ಶರೀ-ರಕ್ಕೆ
ಶರೀ-ರದ
ಶರೀ-ರವೇ
ಶರೀರ
ಶರೀರ-ಗ-ಳನ್ನು
ಶರೀರ-ದ-ಲ್ಲಿ
ಶರೀರ-ದ-ಲ್ಲೇ
ಶರೀರ-ದಿಂದ
ಶರೀರ-ಧಾರಣೆ
ಶರೀರ-ವನ್ನು
ಶರೀರ-ವಿ-ರು-ವುದು
ಶರೀರ-ವಿಂದು
ಶರೀರಾಲ-ಸ್ಯ-ವಾಗಿ-ರು-ವು-ದ-ರಿಂದ
ಶರೀರಿ-ಗಳಿಗೂ
ಶವ-ಗಳಾಗಿ-ದ್ದೀರಿ
ಶವ-ದಂತೆ
ಶವ-ದೊ-ಡನೆ
ಶವಕ್ಕೆ
ಶಶಿ
ಶಶಿ-ಧರ
ಶಶಿ-ಭೂಷಣ-ಘೋಷರು
ಶಶಿ-ಯನ್ನು
ಶಶಿ-ರಾಮ-ಕೃಷ್ಣಾ-ನಂದ
ಶಹರಾ-ನ್ಪು-ರಕ್ಕೆ
ಶಾ
ಶಾಂತ
ಶಾಂತ-ಚಿ-ತ್ತ-ರಾಗಿ
ಶಾಂತ-ಜೀವ-ನೋಪಾಯ-ವನ್ನು
ಶಾಂತ-ನಾಗಲಾರೆ
ಶಾಂತ-ನಾಗಿ
ಶಾಂತ-ರಾಗಿ
ಶಾಂತ-ಳಾಗು
ಶಾಂತ-ವಾ-ಗಿದೆ
ಶಾಂತ-ವಾಗಿ
ಶಾಂತ-ವಾಗಿ-ತ್ತು
ಶಾಂತಿ
ಶಾಂತಿ-ಗಾಗಿ
ಶಾಂತಿ-ಗಿಂತ
ಶಾಂತಿ-ದಾಯಕ-ವಾದ
ಶಾಂತಿ-ಮಯ-ವಾದ
ಶಾಂತಿ-ಯನ್ನು
ಶಾಂತಿ-ಯಿಂದ
ಶಾಂತಿಗೆ
ಶಾಂತಿಯ
ಶಾಂತಿಯೇ
ಶಾಕ್ತರು
ಶಾಕ್ತರೂ
ಶಾಖ-ವಾದ
ಶಾಖವೂ
ಶಾಖಾ-ಹಾರಿ-ಗಳೇ
ಶಾಖೆ-ಗ-ಳನ್ನು
ಶಾಖೆ-ಗಳ-ನ್ನೂ
ಶಾಖೆಗೆ
ಶಾಖೆಯ
ಶಾಖೋಪಶಾಖೆ-ಗಳಿ-ಗೆ-ಲ್ಲ
ಶಾಪ
ಶಾಪ-ಕೊ-ಟ್ಟನು
ಶಾಪ-ದಂತೆ
ಶಾಪ-ವನ್ನು
ಶಾಮಲ-ದಾಸ
ಶಾಮಿಗೆ-ಯಿಂದ
ಶಾರಾ-ದಾನಂದರು
ಶಾರೀ-ರದ
ಶಾಲ-ನ್ನೇ
ಶಾಲಾ
ಶಾಲಿ-ಗಳು
ಶಾಲೆ
ಶಾಲೆ-ಗ-ಳನ್ನು
ಶಾಲೆ-ಯ-ಲ್ಲಿ
ಶಾಲೆ-ಯನ್ನು
ಶಾಲೆ-ಯಿಂದ
ಶಾಲೆಗೆ
ಶಾಲೆಯ
ಶಾಲೆಯೂ
ಶಾಶ್ವತ
ಶಾಶ್ವತ-ವ-ಲ್ಲ
ಶಾಶ್ವತ-ವಾಗಿ
ಶಾಶ್ವತ-ವಾದ
ಶಾಶ್ವತ-ವಾದ-ದ್ದ-ಲ್ಲ
ಶಿ
ಶಿಂ
ಶಿಕಾ-ರಿಗೆ
ಶಿಕ್ಷಕ-ರ-ನ್ನಾಗಿ
ಶಿಕ್ಷಾಲಯ
ಶಿಕ್ಷಿ-ತರಾಗದೆ
ಶಿಕ್ಷಿತ-ರ-ನ್ನಾಗಿ
ಶಿಕ್ಷಿಸ-ಬೇಕು
ಶಿಕ್ಷೆ
ಶಿಕ್ಷೆ-ಯನ್ನು
ಶಿಕ್ಷೆಗೆ
ಶಿಕ್ಷೆಯೇ
ಶಿಖ-ರಕ್ಕೆ
ಶಿಖ-ರದ
ಶಿಖರ
ಶಿಖರ-ಗಳು
ಶಿಖರ-ಗಳೆ-ದುರು
ಶಿಖರ-ದ-ವ-ರೆಗೆ
ಶಿಖರ-ದಿಂದ
ಶಿಖರ-ದೆ-ಡೆಗೆ
ಶಿಖರ-ವನ್ನು
ಶಿಖರ-ವನ್ನೇ-ರಿ-ದ್ದರು
ಶಿಖರ-ವನ್ನೇರಿದ
ಶಿಥಿಲ-ವಾಗ-ತೊಡಗಿತು
ಶಿಬಿ-ರದ
ಶಿಬಿರ
ಶಿಬಿರ-ವಾ-ಯಿತು
ಶಿಬಿರಾಣನ
ಶಿಯಾ
ಶಿಯಾ-ದೇವಿ
ಶಿರ
ಶಿರ-ವನ್ನು
ಶಿರ-ವನ್ನೇ
ಶಿರ-ವಾ-ದರೋ
ಶಿರ-ಸಾ-ವಹಿಸಿ
ಶಿರ-ಸಾ-ವಹಿಸಿ-ದನು
ಶಿರ-ಸ್ತೇದಾರ-ರಾದ
ಶಿರೋ-ನಾಮೆ-ಯ-ಲ್ಲಿ
ಶಿರೋ-ರ-ತ್ನ-ವಾದ
ಶಿಲಾ-ಪ್ರವಾಹ-ದಂತೆ
ಶಿಲಾ-ವಿಗ್ರಹ-ದಂತೆ
ಶಿಲುಬೆ
ಶಿಲುಬೆಯ
ಶಿಲೆ
ಶಿಲೆ-ಗಿಂತ
ಶಿಲೆ-ಯ-ಲ್ಲಿ
ಶಿಲೆ-ಯನ್ನು
ಶಿಲೆ-ಯಿಂದ
ಶಿಲೆಯ
ಶಿವ
ಶಿವ-ಎ-ನ್ನು-ತ್ತಿದ್ದಳು
ಶಿವ-ದರ್ಶನ-ದಿಂದ
ಶಿವ-ದರ್ಶನ-ವನ್ನು
ಶಿವ-ದೇವಾಲಯ-ದ-ಲ್ಲಿ
ಶಿವ-ನಂತೆ
ಶಿವ-ನನ್ನು
ಶಿವ-ನಾಥ
ಶಿವ-ನಾಮ
ಶಿವ-ನಾಮ-ವನ್ನು
ಶಿವ-ನಿಂದ
ಶಿವ-ನಿಗೆ
ಶಿವ-ಭಾವ
ಶಿವ-ಭಾವ-ದ-ಲ್ಲಿ
ಶಿವ-ಯೋಗಿ-ಗಳ
ಶಿವ-ರಾ-ತ್ರಿ
ಶಿವ-ರಾ-ತ್ರಿಯ
ಶಿವ-ರಾಜ
ಶಿವ-ಲಿಂಗ
ಶಿವ-ಲಿಂಗ-ವನ್ನು
ಶಿವ-ಲಿಂಗದ
ಶಿವಾ-ನಂದ
ಶಿವಾ-ನಂದ-ರ-ನ್ನು
ಶಿವಾ-ನಂದರು
ಶಿವಾಂಶಸಂ-ಭೂತ-ನಾಗು-ವನು
ಶಿವಾಯ
ಶಿವೋಽಹಂ
ಶಿಶು
ಶಿಶು-ವಿ-ಹಾರ-ಗಳಿ-ದ್ದಂತೆ
ಶಿಶ್ನ-ಪೂಜೆ
ಶಿಶ್ನದ
ಶಿಷ್ಟಾ-ಚಾರ
ಶಿಷ್ಯ
ಶಿಷ್ಯ-ನ-ನ್ನಾಗಿ
ಶಿಷ್ಯ-ನ-ನ್ನು-ದ್ದೇಶಿಸಿ
ಶಿಷ್ಯ-ನ-ನ್ನೆಬ್ಬಿಸಿ
ಶಿಷ್ಯ-ನ-ಲ್ಲಿ
ಶಿಷ್ಯ-ನನ್ನು
ಶಿಷ್ಯ-ನಾ-ದರೆ
ಶಿಷ್ಯ-ನಾದ
ಶಿಷ್ಯ-ನಿಂದ
ಶಿಷ್ಯ-ನಿಗೂ
ಶಿಷ್ಯ-ನಿಗೆ
ಶಿಷ್ಯ-ನೆದುರು
ಶಿಷ್ಯ-ನೊ-ಡನೆ
ಶಿಷ್ಯ-ನೊಂದಿಗೆ
ಶಿಷ್ಯ-ನೊಬ್ಬ
ಶಿಷ್ಯ-ನೊಬ್ಬ-ನಿಗೆ
ಶಿಷ್ಯ-ರ-ನ್ನಾಗಿ
ಶಿಷ್ಯ-ರ-ನ್ನು
ಶಿಷ್ಯ-ರ-ನ್ನೆ-ಲ್ಲ
ಶಿಷ್ಯ-ರ-ಲ್ಲಿ
ಶಿಷ್ಯ-ರ-ಲ್ಲೊಬ್ಬರಾದ
ಶಿಷ್ಯ-ರಂತೆ
ಶಿಷ್ಯ-ರದು
ಶಿಷ್ಯ-ರಾ-ದರು
ಶಿಷ್ಯ-ರಾ-ದರೆ
ಶಿಷ್ಯ-ರಾಗ-ಬೇಕೆಂದು
ಶಿಷ್ಯ-ರಾಗಿ
ಶಿಷ್ಯ-ರಾಗಿ-ರುವರು
ಶಿಷ್ಯ-ರಾದ
ಶಿಷ್ಯ-ರಿ-ಗೆ-ಲ್ಲ
ಶಿಷ್ಯ-ರಿಂದಲೇ
ಶಿಷ್ಯ-ರಿಗೆ
ಶಿಷ್ಯ-ರಿರ-ಬೇಕು
ಶಿಷ್ಯ-ರು-ಗಳೂ
ಶಿಷ್ಯ-ರೆ-ಲ್ಲ
ಶಿಷ್ಯ-ರೆ-ಲ್ಲರೂ
ಶಿಷ್ಯ-ರೆದೆ-ಯ-ಲ್ಲಿ
ಶಿಷ್ಯ-ರೊ-ಡನೆ
ಶಿಷ್ಯ-ರೊಂದಿಗೆ
ಶಿಷ್ಯ-ರೊಬ್ಬ-ರಿಗೆ
ಶಿಷ್ಯ-ಳಾದ
ಶಿಷ್ಯ-ಳಿಗೆ
ಶಿಷ್ಯ-ವರ್ಗ
ಶಿಷ್ಯ-ವರ್ಗ-ದ-ವ-ರಿಗೆ
ಶಿಷ್ಯ-ವರ್ಗ-ದವ-ರಿದ್ದ
ಶಿಷ್ಯ-ವರ್ಗಕ್ಕೆ
ಶಿಷ್ಯ-ವೃಂ-ದಕ್ಕೆ
ಶಿಷ್ಯ-ಆ
ಶಿಷ್ಯನ
ಶಿಷ್ಯನು
ಶಿಷ್ಯನೂ
ಶಿಷ್ಯರ
ಶಿಷ್ಯರು
ಶಿಷ್ಯರೂ
ಶಿಷ್ಯಳ
ಶಿಷ್ಯಳು
ಶಿಷ್ಯಾಗ್ರಣಿ
ಶಿಷ್ಯೆ
ಶಿಷ್ಯೆ-ಯ-ರ-ನ್ನು
ಶಿಷ್ಯೆ-ಯಾದ
ಶಿಷ್ಯ್ತ-ರಿಗೆ
ಶೀಘ್ರ-ದ-ಲ್ಲಿ
ಶೀಘ್ರ-ದ-ಲ್ಲಿಯೇ
ಶೀಘ್ರ-ವಾದ
ಶೀಘ್ರಲಿ-ಪಿ-ಕಾರ
ಶೀಘ್ರಲಿ-ಪಿ-ಕಾರ-ನನ್ನು
ಶೀಘ್ರಲಿ-ಪಿ-ಕಾರ-ನಾದ
ಶೀಘ್ರಲಿಪಿ
ಶೀತಲ-ವಾರಿಯ
ಶೀಲ
ಶೀಲ-ಗ-ಳನ್ನು
ಶೀಲ-ದ-ಲ್ಲಿ
ಶೀಲ-ದ-ವ-ರಿಗೂ
ಶೀಲ-ವಂತ-ರಾದ
ಶೀಲ-ವನ್ನು
ಶೀಲ-ವನ್ನೇ
ಶೀಲದ
ಶೀಲರ
ಶು
ಶುಕ
ಶುಕ-ದೇವ
ಶುಕ-ದೇವ-ನಿಗೆ
ಶುಕ-ದೇವನ
ಶುಕ್ರ-ವಾರ
ಶುಕ್ಲ
ಶುಚಿ-ಯಾ-ಗಿದೆ
ಶುಚಿ-ಯಾದ
ಶುದ್ಧ
ಶುದ್ಧ-ಚಾರಿ-ತ್ರ-ವೆಂದೂ
ಶುದ್ಧ-ಜಲ
ಶುದ್ಧ-ತ-ತ್ತ್ವ
ಶುದ್ಧ-ರಾಗಿ-ರುವುದು
ಶುದ್ಧ-ವಾ-ಗಿ-ಲ್ಲದೆ
ಶುದ್ಧ-ವಾ-ದರೆ
ಶುದ್ಧ-ವಾ-ದುದು
ಶುದ್ಧ-ವಾಗು-ವುದು
ಶುದ್ಧ-ವಾದ
ಶುದ್ಧಾ
ಶುದ್ಧಾ-ನಂದ
ಶುದ್ಧಾ-ನಂದ-ರಿಗೆ
ಶುದ್ಧಿ
ಶುದ್ಧಿ-ಮಾಡು-ವನು
ಶುದ್ಧಿ-ಮಾಡು-ವುದು
ಶುದ್ಧಿಯ
ಶುಭ
ಶುಭ-ವಾರ್ತೆ-ಯಿಂದ
ಶುಭದ
ಶುಭಾ-ಶಯ
ಶುಭಾ-ಶಯ-ಗ-ಳನ್ನು
ಶುಭಾ-ಶಯ-ವನ್ನು
ಶುಭ್ರ
ಶುಭ್ರ-ವಾದ
ಶುಭ್ರತೆ
ಶುಯೇಛೆ
ಶುರು-ವಾ-ಯಿತು
ಶುಶ್ರೂಷೆ
ಶುಷ್ಕ
ಶೂ
ಶೂದ್ರ-ನಾ-ದರೆ
ಶೂದ್ರ-ನಾಗಲಿ
ಶೂದ್ರ-ನಿಗೆ
ಶೂದ್ರರು
ಶೃಂ
ಶೃಂಖಲೆ-ಗಳೆ-ಲ್ಲ
ಶೃಂಖಲೆ-ಯನ್ನು
ಶೇಕಡ
ಶೇಕಡಾ
ಶೇಕ್ಸ್
ಶೇಖ-ರಿಸಿ-ಟ್ಟು-ಕೊಂಡಿದ್ದರು
ಶೇಖರಿ-ಸಿದ್ದನೋ
ಶೇಖರಿ-ಸು-ವು-ದ-ರಿಂದ
ಶೇರ್ಷಾ
ಶೇಷಾದ್ರಿ
ಶೈ
ಶೈತಾ-ನ್
ಶೈರಿ-ನ-ಲ್ಲಿ
ಶೈಲಿ
ಶೈಲಿ-ಗಳು
ಶೈಲಿ-ಯ-ಲ್ಲಿ
ಶೈಲಿ-ಯ-ಲ್ಲಿ-ಡ-ಬೇಕು
ಶೈಲಿ-ಯಂತೆಯೇ
ಶೈವ
ಶೋ
ಶೋಕ
ಶೋಕ-ಭರಿತಳಾಗಿ-ದ್ದರೂ
ಶೋಕಿ
ಶೋಭಾ
ಶೋಭಾ-ಯ-ಮಾನ-ವಾಗಿ-ರು-ವಂತಹ
ಶೋಭಿ-ಸು-ತ್ತಿ-ರು-ವನು
ಶೋಭಿ-ಸು-ತ್ತಿದ್ದ
ಶೋಭಿ-ಸು-ತ್ತಿವೆ
ಶೌರ್ಯ
ಶೌರ್ಯ-ವಿದೆ
ಶ್ಯಾಮ-ಬ-ಜಾರಿನ
ಶ್ಯಾಮಪುಕುರ-ದ-ಲ್ಲಿ
ಶ್ರದ್ದೆ
ಶ್ರದ್ಧಯೋಪೇತೋ
ಶ್ರದ್ಧಾ
ಶ್ರದ್ಧಾ-ಪೂರ್ವ-ಕ-ವಾದ
ಶ್ರದ್ಧಾ-ಭಕ್ತಿ-ಗಳು
ಶ್ರದ್ಧಾ-ವಂತ
ಶ್ರದ್ಧಾ-ವಂತ-ರಾಗಿ
ಶ್ರದ್ಧಾ-ಹೀನ-ರಾದ
ಶ್ರದ್ಧೆ
ಶ್ರದ್ಧೆ-ಬೇಕು
ಶ್ರದ್ಧೆ-ಯನ್ನು
ಶ್ರದ್ಧೆ-ಯಾಗು-ವು-ದಿ-ಲ್ಲ
ಶ್ರದ್ಧೆ-ಯಿಂದ
ಶ್ರದ್ಧೆ-ಯುಳ್ಳ
ಶ್ರದ್ಧೆಗೆ
ಶ್ರದ್ಧೆಯ
ಶ್ರದ್ಧೆಯೂ
ಶ್ರಮ
ಶ್ರಮ-ದಿಂದ
ಶ್ರಮ-ಪಟ್ಟು
ಶ್ರಮ-ವನ್ನು
ಶ್ರಮ-ವನ್ನೂ
ಶ್ರಮ-ವಿ-ಲ್ಲದೆ
ಶ್ರಾದ್ಧ
ಶ್ರಾದ್ಧಾದಿ
ಶ್ರಾದ್ಧಾದಿ-ಗಳಿಂದ
ಶ್ರಿ
ಶ್ರೀ
ಶ್ರು
ಶ್ರೇಣಿಗೆ
ಶ್ರೇಯ-ಸ್ಸಿಗೆ
ಶ್ರೇಯ-ಸ್ಸಿನ
ಶ್ರೇಷ್ಠ
ಶ್ರೇಷ್ಠ-ಗುರು
ಶ್ರೇಷ್ಠ-ತಮ
ಶ್ರೇಷ್ಠ-ತಮ-ಕೃತಿ
ಶ್ರೇಷ್ಠ-ತಮ-ವಾ-ದು-ದ-ನ್ನು
ಶ್ರೇಷ್ಠ-ತೆ-ಯನ್ನು
ಶ್ರೇಷ್ಠ-ದರ್ಜೆಗೆ
ಶ್ರೇಷ್ಠ-ಭಾವ-ನೆ-ಗ-ಳನ್ನು
ಶ್ರೇಷ್ಠ-ರಾಗುವರು
ಶ್ರೇಷ್ಠ-ರಾದ
ಶ್ರೇಷ್ಠ-ವ-ಲ್ಲ
ಶ್ರೇಷ್ಠ-ವ-ಸ್ತುವೇ
ಶ್ರೇಷ್ಠ-ವಾದ
ಶ್ರೇಷ್ಥ-ವಾದ
ಶ್ಲಾಘನೆ
ಶ್ಲಾಘಿ-ಸಿ-ದರು
ಶ್ಲಾಘಿಸ-ತೊಡಗಿದರು
ಶ್ಲೋಕ
ಶ್ಲೋಕ-ಗ-ಳನ್ನು
ಶ್ಲೋಕ-ಗ-ಳಿದ್ದವು
ಶ್ಲೋಕ-ದ-ಲ್ಲಿ
ಶ್ಲೋಕ-ವನ್ನು
ಶ್ಲೋಕ-ವೆಂದು
ಶ್ಲೋಕದ
ಶ್ವಾಸ-ಕೋಶ
ಶ್ವಾಸ-ಕೋಶ-ಗಳು
ಶ್ವೇ
ಶ್ವೇತಚ್ಛ-ತ್ರದ
ಷಡ್ದರ್ಶನ-ಗ-ಳನ್ನು
ಷರ-ತ್ತಿನ
ಷಿ
ಷಿಲಾಂಗ್ಗೆ
ಷೇಕ್ಸ್
ಸ
ಸಂ
ಸಂಕಟ
ಸಂಕಟ-ಗ-ಳನ್ನು
ಸಂಕಟ-ದ-ಲ್ಲಿ
ಸಂಕಟ-ಪಡು-ತ್ತಿರುವ
ಸಂಕಟ-ವನ್ನು
ಸಂಕಟ-ವಾ-ದರೂ
ಸಂಕಟಕ್ಕೆ
ಸಂಕಟದ
ಸಂಕಲನ
ಸಂಕೀರ್ತನ
ಸಂಕೀರ್ತನೆ-ಗಳೊ-ಡನೆ
ಸಂಕುಚಿತ
ಸಂಕೇತ
ಸಂಕೇತ-ವನ್ನೊ
ಸಂಕೋಚ
ಸಂಕೋಚ-ವಾ-ದಂತೆ
ಸಂಕೋಚ-ವಿ-ಲ್ಲ
ಸಂಕೋಚ-ವಿ-ಲ್ಲದೆ
ಸಂಕೋಚವೇ
ಸಂಕ್ರಾಂ-ತಿಯ
ಸಂಕ್ಷಿಪ್ತ
ಸಂಕ್ಷೇಪ
ಸಂಕ್ಷೇಪ-ವಾಗಿ
ಸಂಖ್ಯಾ-ತರ
ಸಂಖ್ಯೆ
ಸಂಖ್ಯೆ-ಯ-ಲ್ಲಿ
ಸಂಖ್ಯೆಯ
ಸಂಗ-ತ-ವಾಗು-ತ್ತವೆ
ಸಂಗ-ದಿಂದಾದ
ಸಂಗ-ಮ-ವಾ-ದರೆ
ಸಂಗಮ
ಸಂಗಾತಿ-ಗಳು
ಸಂಗಾತಿ-ಯಾದ
ಸಂಗೀ-ತಕ್ಕೆ
ಸಂಗೀ-ತದ
ಸಂಗೀತ
ಸಂಗೀತ-ಗಾರ-ನನ್ನು
ಸಂಗೀತ-ಗಾರ-ರೊಬ್ಬರು
ಸಂಗೀತ-ಗಾರರ
ಸಂಗೀತ-ಗಾರರು
ಸಂಗೀತ-ಗಾರಳು
ಸಂಗೀತ-ದ-ಲ್ಲಿ
ಸಂಗೀತ-ದ-ಲ್ಲಿ-ರು-ವಂತೆ
ಸಂಗೀತ-ದ-ಲ್ಲಿಯೂ
ಸಂಗೀತ-ದಂತೆ
ಸಂಗೀತ-ದಿಂದಲೇ
ಸಂಗೀತ-ವನ್ನು
ಸಂಗೀತ-ವಾಗಲಿ
ಸಂಗೀತ-ವಾಗು-ತ್ತಿದ್ದಾಗ
ಸಂಗೀತ-ಶಾ-ಸ್ತ್ರ-ದ-ಲ್ಲಿ
ಸಂಗೀತಾ-ದಿ-ಗಳು
ಸಂಗ್ರಹ
ಸಂಗ್ರಹ-ವಾಗು-ತ್ತಿರು-ವುದು
ಸಂಗ್ರಹ-ವಾಗುವ
ಸಂಗ್ರಹ-ಶಾಲೆ-ಯ-ಲ್ಲಿ
ಸಂಗ್ರಹ-ಶಾಲೆಯ
ಸಂಗ್ರಹಾಲಯ-ದಿಂದ
ಸಂಗ್ರಹಿಸು
ಸಂಗ್ರಹಿಸು-ತ್ತಿದ್ದ
ಸಂಗ್ರಹಿಸು-ತ್ತಿದ್ದಳು
ಸಂಗ್ರಹಿಸು-ವು-ದಕ್ಕೆ
ಸಂಗ್ರಹಿಸು-ವು-ದೊಂದು
ಸಂಗ್ರಹಿಸು-ವುದ-ರ-ಲ್ಲಿ
ಸಂಘ
ಸಂಘ-ಗಳ
ಸಂಘ-ಟನಾ
ಸಂಘ-ಟನಾ-ಶಕ್ತಿ-ಯನ್ನು
ಸಂಘ-ದ-ಲ್ಲಿ
ಸಂಘ-ದ-ಲ್ಲಿ-ರುವ
ಸಂಘ-ದ-ವರು
ಸಂಘ-ದಿಂದ
ಸಂಘ-ರೂಪ-ವಾಗಿ
ಸಂಘ-ವನ್ನು
ಸಂಘ-ವಿ-ಲ್ಲದೆ
ಸಂಘ-ಸ್ಥಾಪನೆ
ಸಂಘ-ಸ್ಥಾಪನೆಯ
ಸಂಘಕ್ಕೆ
ಸಂಘದ
ಸಂಘವು
ಸಂಚ-ಕಾರ
ಸಂಚ-ರಿ-ಸಲು
ಸಂಚ-ರಿ-ಸಿದ್ದು
ಸಂಚ-ರಿಸಿ
ಸಂಚ-ರಿಸಿ-ಕೊಂಡು
ಸಂಚ-ರಿಸಿ-ದಂತಾಯ್ತು
ಸಂಚ-ರಿಸಿ-ದ್ದರು
ಸಂಚ-ರಿಸಿದ
ಸಂಚ-ರಿಸಿದೆ
ಸಂಚಾ-ರಕ್ಕೆ
ಸಂಚಿಕೆ
ಸಂಜೆ
ಸಂಜೆ-ಯಾಗಿ-ತ್ತು
ಸಂಜೆಗೆ
ಸಂಜೆಯ
ಸಂಜ್ಞೆ
ಸಂಜ್ಞೆ-ಗಳ
ಸಂತ-ಜಾ-ನ್
ಸಂತತಿ-ಯ-ವರು
ಸಂತಾ-ನರು
ಸಂತಾ-ಲರ
ಸಂತಾ-ಲರು
ಸಂತಾನ-ದ-ವ-ರಿಗೆ
ಸಂತಾನ-ದ-ವರು
ಸಂತಾನ-ರ-ಲ್ಲವೆ
ಸಂತಾನ-ರಾಗಿ-ರು-ವ-ರೆಂದು
ಸಂತಾನ-ರೆ-ಲ್ಲ
ಸಂತಾಲ
ಸಂತಾಲ-ರ-ನ್ನು
ಸಂತಾಲ-ರ-ಲ್ಲಿ
ಸಂತಾಲ-ರೊಡಾನೆ
ಸಂತುಷ್ಟ-ರಾ-ದರು
ಸಂತೆ
ಸಂತೆ-ಯ-ಲ್ಲಿ
ಸಂತೆ-ಯಿಂದ
ಸಂತೈ-ಸಿ-ದರು
ಸಂತೈ-ಸುವ-ವ-ರಿ-ಲ್ಲ
ಸಂತೊಷವೇ
ಸಂತೋಷ
ಸಂತೋಷ-ಚಿ-ತ್ತ-ದವ-ನಂತೆ
ಸಂತೋಷ-ಚಿ-ತ್ತ-ದವಳಾಗ-ಬೇಕು
ಸಂತೋಷ-ಚಿ-ತ್ತ-ನಾಗಿರು
ಸಂತೋಷ-ದ-ಲ್ಲಿದ್ದರು
ಸಂತೋಷ-ದಾಯಕ-ವಾಗಿ-ತ್ತು
ಸಂತೋಷ-ದಾಯಕ-ವಾದ
ಸಂತೋಷ-ದಿಂದ
ಸಂತೋಷ-ಪ-ಟ್ಟರು
ಸಂತೋಷ-ಪ-ಡು-ತ್ತಿದ್ದರು
ಸಂತೋಷ-ಪ-ಡುವರು
ಸಂತೋಷ-ಪಟ್ಟು
ಸಂತೋಷ-ಪಡ-ಲಿ-ಲ್ಲ
ಸಂತೋಷ-ಪಡಿ-ಸು-ತ್ತಿದ್ದ
ಸಂತೋಷ-ಪಡು-ತ್ತಿದ್ದ
ಸಂತೋಷ-ಭರಿತ-ಳಾದೆ
ಸಂತೋಷ-ಲೋ-ಲರು
ಸಂತೋಷ-ವನ್ನು
ಸಂತೋಷ-ವಾ-ದರೆ
ಸಂತೋಷ-ವಾ-ಯಿತು
ಸಂತೋಷ-ವಾಗಿ
ಸಂತೋಷ-ವಾಗಿ-ದ್ದೇನೆ
ಸಂತೋಷ-ವಾಗಿ-ರ-ಬಹುದು
ಸಂತೋಷ-ವಾಗು-ವುದು
ಸಂತೋಷದ
ಸಂತೋಷವೇ
ಸಂತೋಷಿ-ಸು-ತ್ತಿ-ದ್ದರು
ಸಂತೋಷಿಸ-ಬಹುದು
ಸಂದ
ಸಂದಿಗ್ಧ
ಸಂದೇಹ
ಸಂದೇಹ-ಗಳು
ಸಂದೇಹ-ಗಳೆ-ಲ್ಲ
ಸಂದೇಹ-ವಿ-ಲ್ಲ
ಸಂದೇಹಾ-ದಿ-ಗ-ಳನ್ನು
ಸಂಧಿ-ಕಾಲ
ಸಂಧಿ-ಕಾಲ-ವೆಂದು
ಸಂಧಿ-ಸಲು
ಸಂಧಿ-ಸಿದ
ಸಂಧಿ-ಸು-ವು-ದ-ನ್ನು
ಸಂಧಿ-ಸು-ವು-ದಕ್ಕೆ
ಸಂಧಿಸಿ
ಸಂಧಿಸಿ-ದಂತೆ
ಸಂಧಿಸಿ-ದರೂ
ಸಂಧಿಸಿ-ದಾಗ
ಸಂಧಿಸಿ-ದು-ದ-ನ್ನು
ಸಂಧಿಸಿ-ದೆವು
ಸಂಧಿಸಿ-ದ್ದುವು
ಸಂಧ್ಯಾವಂದ-ನಾದಿ
ಸಂಧ್ಯಾವಂದನೆ
ಸಂಧ್ಯಾವಂದನೆ-ಯನ್ನು
ಸಂಧ್ಯೆಯ
ಸಂಪ-ತ್ತು
ಸಂಪಾ-ಸದಿಸ-ಬೇಕಾಗಿದೆ
ಸಂಪಾದ-ಕರು
ಸಂಪಾದಕ
ಸಂಪಾದಕ-ರ-ನ್ನಾಗಿ
ಸಂಪಾದಕ-ರಾದ
ಸಂಪಾದಿ-ಸಲು
ಸಂಪಾದಿ-ಸು-ತ್ತ
ಸಂಪಾದಿ-ಸು-ತ್ತಿದ್ದ
ಸಂಪಾದಿ-ಸು-ವು-ದಕ್ಕೆ
ಸಂಪಾದಿಸ-ಬಹುದು
ಸಂಪಾದಿಸ-ಬೇಕು
ಸಂಪಾದಿಸಿ
ಸಂಪಾದಿಸಿ-ಕೊಂಡು
ಸಂಪಾದಿಸಿ-ದನು
ಸಂಪಾದಿಸಿ-ದುದ-ನ್ನೆ-ಲ್ಲ
ಸಂಪುಟ-ಗಳ-ಲ್ಲಿ
ಸಂಪುಟ-ಗಳು
ಸಂಪುಟ-ವನ್ನು
ಸಂಬಂಧ
ಸಂಬಂಧ-ಗ-ಳನ್ನು
ಸಂಬಂಧ-ಗಳ
ಸಂಬಂಧ-ದ-ಲ್ಲಿ
ಸಂಬಂಧ-ಪ-ಟ್ಟಿದ್ದ-ನ್ನೂ
ಸಂಬಂಧ-ಪಟ್ಟ
ಸಂಬಂಧ-ಪಟ್ಟ-ದ್ದ-ನ್ನೆ-ಲ್ಲ
ಸಂಬಂಧ-ಪಟ್ಟಂತೆ
ಸಂಬಂಧ-ಪಟ್ಟದ್ದು
ಸಂಬಂಧ-ಪಟ್ಟಿ-ರು-ವುದು
ಸಂಬಂಧ-ಮೂರ್ತಿಯ
ಸಂಬಂಧ-ವ-ನ್ನುಂಟು
ಸಂಬಂಧ-ವನ್ನು
ಸಂಬಂಧ-ವಾಗಿ
ಸಂಬಂಧ-ವಾದ
ಸಂಬಂಧ-ವಿ-ಲ್ಲ
ಸಂಬಂಧ-ವಿ-ಲ್ಲ-ವೆಂದೂ
ಸಂಬಂಧ-ವೇನು
ಸಂಬಂಧದ
ಸಂಬಂಧವೂ
ಸಂಬಳ
ಸಂಬಳ-ವನ್ನೇ
ಸಂಬಳವೂ
ಸಂಭ-ವವೂ
ಸಂಭಂದ-ಪಟ್ಟ
ಸಂಭವ
ಸಂಭವ-ವನ್ನು
ಸಂಭವಿ-ಸು-ವು-ದಿ-ಲ್ಲ
ಸಂಭವಿಸ-ಬಹುದೆ
ಸಂಯಮ
ಸಂಯೋ-ಜನಾ
ಸಂರಕ್ಷಿಸ-ತೊಡಗಿದೆ
ಸಂವಿಭಾತಿ
ಸಂಶ-ಯವೂ
ಸಕ-ಲರ
ಸಕಲ
ಸಕ್ಕರೆ
ಸಕ್ಕರೆಯ
ಸಖ-ನಾದ
ಸಖನೆ
ಸಖಿ
ಸಚ್ಚಿ-ದಾನಂದ
ಸಚ್ಚಿ-ದಾನಂದ-ದ-ಲ್ಲಿ
ಸಚ್ಚಿ-ದಾನಂದ-ರ-ಲ್ಲಿ
ಸಚ್ಚಿ-ದಾನಂದ-ವೆಂಬ
ಸಡ-ಗರ-ದಿಂದ
ಸಡಿಲ-ವಾಗಿ-ರ-ಲಿ-ಲ್ಲ
ಸಡಿಲಿ-ಸು-ವೆವು
ಸಣ್ಣ
ಸಣ್ಣ-ತನ-ವಿ-ರ-ಲಿ-ಲ್ಲ
ಸಣ್ಣ-ದ-ನ್ನು
ಸಣ್ಣ-ದಾಗಿ
ಸಣ್ಣ-ದಾಗಿ-ದ್ದರೂ
ಸಣ್ಣ-ದಾಗಿ-ರುವ
ಸಣ್ಣ-ದೊಂದು
ಸಣ್ಣ-ಪು-ಟ್ಟ
ಸಣ್ಣ-ಪು-ಟ್ಟ-ದ-ನ್ನು
ಸಣ್ಣ-ಮನೆ
ಸಣ್ಣ-ವ-ರಿಗೆ
ಸಣ್ಣದು
ಸತತ
ಸತಿ
ಸತಿ-ಯನ್ನು
ಸತಿ-ಯರು
ಸತಿ-ಯೊ-ಡನೆ
ಸತೀಶ-ಚಂದ್ರ-ಮುಖರ್ಜಿ
ಸದ-ಸ್ಯ-ನ-ನ್ನಾಗಿ
ಸದ-ಸ್ಯ-ನಾಗಿ
ಸದ-ಸ್ಯ-ನಾಗಿ-ದ್ದನು
ಸದ-ಸ್ಯ-ರ-ನ್ನು
ಸದ-ಸ್ಯ-ರಾಗಿ-ದ್ದರು
ಸದ-ಸ್ಯ-ರಿಗೆ
ಸದ-ಸ್ಯರ-ನೇ-ಕರು
ಸದ-ಸ್ಯರಾಗ-ಬಹು-ದಿ-ತ್ತು
ಸದ-ಸ್ಯರಾಗ-ಬೇ-ಕಾ-ದರೆ
ಸದ-ಸ್ಯರು
ಸದ-ಸ್ಯರು-ಗಳ-ನ್ನೆ-ಲ್ಲ
ಸದ-ಸ್ಯರೇ
ಸದಾ
ಸದಾ-ಚಾರ
ಸದಾ-ನಂ-ದರು
ಸದಾ-ನಂದ
ಸದಾ-ವಕಾಶ
ಸದೃಶ
ಸದೆ
ಸದೆ-ಬಡಿ-ರು-ವೆವು
ಸದೆ-ಬಡೆಯು-ವಂತಹ
ಸದ್ಗುಣ-ಸಂಪ-ನ್ನ-ರಾದ
ಸದ್ದು-ಗದ್ದ-ಲವೂ
ಸದ್ದು-ಗದ್ದಲ-ವಿ-ಲ್ಲದೆ
ಸದ್ಯ-ದ-ಲ್ಲಿ
ಸದ್ಯಃ
ಸದ್ಯಕ್ಕೆ
ಸದ್ವಿನಿ-ಯೋಗ-ವಾಗು-ತ್ತದೆ
ಸದ್ವ್ಯವ-ಹಾರ
ಸಧ್ಯ-ದ-ಲ್ಲಿ
ಸಧ್ಯಕ್ಕೆ
ಸನಾ-ತನ
ಸನಾ-ತನ-ತ-ತ್ವದ
ಸನಾ-ತನ-ಧರ್ಮ
ಸನಾ-ತನ-ಧರ್ಮದ
ಸನಾ-ತನ-ವಾ-ದುದು
ಸನಾ-ತನ-ವಾಗಿ
ಸನಾ-ತನ-ವಾಗಿ-ರು-ವುದು
ಸನಾ-ತನ-ವಾದ
ಸನ್ನದ್ಧ-ನಾಗಿ-ರು-ವಂತೆ
ಸನ್ನದ್ಧ-ರಾ-ದರು
ಸನ್ನಾಹ-ಗಳು
ಸನ್ನಾಹ-ದ-ಲ್ಲಿ
ಸನ್ನಿ-ಹಿತ-ವಾ-ಗಿದೆ
ಸನ್ನಿ-ಹಿತ-ವಾ-ಗು-ತ್ತಿದೆ
ಸನ್ನಿ-ಹಿತ-ವಾ-ಯಿತು
ಸನ್ನಿ-ಹಿತ-ವಾ-ಯಿತೆಂದು
ಸನ್ನಿ-ಹಿತ-ವಾಗು-ತ್ತಿರು-ವಾಗ
ಸನ್ನಿಧಿಗೆ
ಸನ್ನಿವೇ-ಶ-ದ-ಲ್ಲಿ
ಸನ್ನಿವೇ-ಶಕ್ಕೆ
ಸನ್ನಿವೇಶ
ಸನ್ನೆಯ
ಸನ್ಮಾನ-ಪ-ತ್ರ-ವನ್ನು
ಸನ್ಮಾನ-ವನ್ನು
ಸನ್ಮಾನದ
ಸನ್ಮಾರ್ಗ-ದಿಂದ
ಸನ್ಯಾ-ಲರು
ಸನ್ಯಾಲ
ಸನ್ಯಾಸ-ಧರ್ಮದ
ಸನ್ಯಾಸಿ-ಗ-ಳಿಗೆ
ಸಪ್ತ
ಸಪ್ತ-ಋಷಿ-ಗಳು
ಸಪ್ತಮಿ
ಸಪ್ಪಳ-ವನ್ನು
ಸಪ್ಪೆ-ಯಾಗಿ
ಸಪ್ಪೆ-ಯಾದ
ಸಫಲ-ವಾ-ದವು
ಸಫಲ-ವಾಗು-ವುದ-ಕ್ಕೋ-ಸ್ಕರ
ಸಬಲತೆ
ಸಬ್
ಸಬ್ಜಡ್ಜರ
ಸಭಾ
ಸಭಾ-ಪತಿ-ಗ-ಳಾದರು
ಸಭಾ-ಪತಿ-ಗಳೂ
ಸಭಿ-ಕ-ರ-ನ್ನು
ಸಭಿ-ಕ-ರಿಗೆ
ಸಭಿ-ಕ-ರೆ-ಲ್ಲ
ಸಭಿ-ಕರು
ಸಭಿ-ಕರೆ-ದು-ರಿಗೆ
ಸಭಿ-ಕರೆ-ಲ್ಲರೂ
ಸಭಿಕ-ನಾಗಿ
ಸಭಿಕ-ರ-ಲ್ಲಿ
ಸಭೆ
ಸಭೆ-ಗಳ-ಲ್ಲಿ
ಸಭೆ-ಗಾಗಿ
ಸಭೆ-ಗೋ-ಸುಗ-ವಾಗಿ
ಸಭೆ-ಯ-ಲ್ಲಿ
ಸಭೆ-ಯನ್ನು
ಸಭೆ-ಯೊಂದೇ
ಸಭ್ಯ
ಸಭ್ಯ-ರ-ನ್ನು
ಸಭ್ಯ-ಸಮಾಜ
ಸಮ
ಸಮ-ಕ್ಷಮದ-ಲ್ಲಿ-ಟ್ಟು
ಸಮ-ಚಿ-ತ್ತ-ದಿಂದ
ಸಮ-ತಾ-ವಾದ
ಸಮ-ತೂ-ಕ-ವಾಗಿ
ಸಮ-ತ್ವ-ವನ್ನು
ಸಮ-ನ-ಲ್ಲ
ಸಮ-ನಾಗಿ
ಸಮ-ನಾಗಿ-ದ್ದರು
ಸಮ-ನಾಗಿ-ರು-ವುದು
ಸಮ-ವೆಂದು
ಸಮ-ಷ್ಟಿ-ಯ-ಲ್ಲಿರುವ
ಸಮ-ಷ್ಟಿಯ
ಸಮ-ಸ್ತ
ಸಮ-ಸ್ತ-ವನ್ನೂ
ಸಮ-ಸ್ತರ
ಸಮ-ಸ್ಯೆ
ಸಮ-ಸ್ಯೆ-ಗ-ಳನ್ನು
ಸಮ-ಸ್ಯೆ-ಗ-ಳಿಗೆ
ಸಮ-ಸ್ಯೆ-ಗಳ
ಸಮ-ಸ್ಯೆ-ಯನ್ನು
ಸಮ-ಸ್ಯೆ-ಯಿಂದ
ಸಮ-ಸ್ಯೆಗೆ
ಸಮ-ಸ್ಯೆಯ
ಸಮ-ಸ್ಯೆಯು
ಸಮ-ಸ್ಯೆಯೇ
ಸಮ-ಸ್ವರ-ದ-ಲ್ಲಿ
ಸಮಕ್ಕೆ
ಸಮತಾ
ಸಮಷ್ಟಿ
ಸಮಾ-ಗ-ಮನ
ಸಮಾ-ಚಾರ
ಸಮಾ-ಚಾರ-ವನ್ನು
ಸಮಾ-ಚಾರ-ವೇನು
ಸಮಾ-ಚಾರವೇ
ಸಮಾಜ
ಸಮಾಜ-ದ-ಲ್ಲಿ
ಸಮಾಜ-ದ-ವ-ರ-ನ್ನು
ಸಮಾಜ-ದ-ವರು
ಸಮಾಜ-ದಿಂದ
ಸಮಾಜ-ವನ್ನು
ಸಮಾಜ-ವನ್ನೂ
ಸಮಾಜ-ವಾಗು-ವುದು
ಸಮಾಜ-ವೆಂಬ
ಸಮಾಜ-ಸುಧಾರ-ಕರ
ಸಮಾಜಕ್ಕೆ
ಸಮಾಜದ
ಸಮಾಜವು
ಸಮಾಜವೂ
ಸಮಾಧಾನ
ಸಮಾಧಾನ-ಚಿ-ತ್ತ-ರಾಗಿ
ಸಮಾಧಾನ-ದ-ಲ್ಲಿ-ರು-ವುದು
ಸಮಾಧಾನ-ಪ-ಟ್ಟರು
ಸಮಾಧಾನ-ವನ್ನು
ಸಮಾಧಾನ-ವಾ-ಯಿತು
ಸಮಾಧಿ
ಸಮಾಧಿ-ಗಳ-ನ್ನೂ
ಸಮಾಧಿ-ಗಳು
ಸಮಾಧಿ-ಗಳೂ
ಸಮಾಧಿ-ಗಿಂತ
ಸಮಾಧಿ-ಮಗ್ನ-ರಾ-ದರು
ಸಮಾಧಿ-ಮಗ್ನ-ರಾಗಿ
ಸಮಾಧಿ-ಮಗ್ನ-ರಾಗುವುದು
ಸಮಾಧಿ-ಮಗ್ನ-ವಾಗಿ-ಬಿಡು-ತ್ತಿ-ತ್ತು
ಸಮಾಧಿ-ಯ-ಲ್ಲಿ
ಸಮಾಧಿ-ಯ-ಲ್ಲಿ-ದ್ದಾಗ
ಸಮಾಧಿ-ಯ-ಲ್ಲಿ-ರು-ವು-ದ-ನ್ನು
ಸಮಾಧಿ-ಯ-ಲ್ಲಿದ್ದ
ಸಮಾಧಿ-ಯನ್ನು
ಸಮಾಧಿ-ಯನ್ನೂ
ಸಮಾಧಿ-ಯಿಂದ
ಸಮಾಧಿ-ಸ್ಥ-ನಾಗಲು
ಸಮಾಧಿ-ಸ್ಥ-ರಾ-ದರು
ಸಮಾಧಿ-ಸ್ಥ-ಳಾದಳು
ಸಮಾಧಿ-ಸ್ಥಳಕ್ಕೆ
ಸಮಾಧಿಗೆ
ಸಮಾಧಿಯ
ಸಮಾರಂಭ
ಸಮಾರಂಭ-ದ-ಲ್ಲಿ
ಸಮಾರಂಭ-ವನ್ನು
ಸಮಾರಂಭ-ವಾ-ದರೊ
ಸಮಾರಂಭಕ್ಕೂ
ಸಮಾರಂಭಕ್ಕೆ
ಸಮೀ-ಪದ
ಸಮೀ-ಪದ-ಲ್ಲಿ-ಟ್ಟು
ಸಮೀ-ಪದ-ಲ್ಲಿ-ತ್ತು
ಸಮೀ-ಪದ-ಲ್ಲಿ-ದ್ದರೂ
ಸಮೀ-ಪದ-ಲ್ಲಿ-ದ್ದರೆ
ಸಮೀ-ಪದ-ಲ್ಲಿ-ರು-ವನು
ಸಮೀ-ಪದ-ಲ್ಲಿ-ರುವ
ಸಮೀ-ಪದ-ಲ್ಲಿದ್ದು
ಸಮೀ-ಪದ-ಲ್ಲೆಯೇ
ಸಮೀಪ-ದ-ಲ್ಲಿ
ಸಮೀಪ-ದ-ಲ್ಲೆ
ಸಮೀಪ-ದ-ಲ್ಲೇ
ಸಮೀಪ-ವಾಗಿ
ಸಮೀಪ-ವಿದ್ದ
ಸಮೀಪಕ್ಕೂ
ಸಮೀಪಕ್ಕೆ
ಸಮೀಪಿ-ಸಿ-ದರು
ಸಮೀಪಿ-ಸಿ-ರುವ
ಸಮೀಪಿ-ಸಿತು
ಸಮೀಪಿ-ಸಿದೆ
ಸಮೀಪಿ-ಸು-ತ್ತಿ-ರುವಾಗ
ಸಮೀಪಿ-ಸು-ತ್ತಿದೆ
ಸಮೀಪಿಸಿ-ದಂತೆ-ಲ್ಲ
ಸಮೀಪಿಸಿ-ದಾಗ
ಸಮೀಪಿಸಿ-ದಾಗ-ಲೆ-ಲ್ಲ
ಸಮೀರಣ-ದಂತೆ
ಸಮುದಾ-ಯದ
ಸಮುದಾಯ-ಜೀವಿ-ಗಳ
ಸಮುದ್ರ
ಸಮುದ್ರ-ಕಳ್ಳರ
ಸಮುದ್ರ-ಗಳು
ಸಮುದ್ರ-ತೀ-ರಕ್ಕೆ
ಸಮುದ್ರ-ತೀರ-ದ-ಲ್ಲಿ
ಸಮುದ್ರ-ತೀರ-ದ-ಲ್ಲಿ-ರುವ
ಸಮುದ್ರ-ದ-ಲ್ಲಿ
ಸಮುದ್ರ-ದಾಚೆ-ಯಿಂದ
ಸಮುದ್ರ-ದಿಂದ
ಸಮುದ್ರ-ಯಾನ
ಸಮುದ್ರ-ಯಾನದ
ಸಮುದ್ರ-ವನ್ನು
ಸಮುದ್ರಕ್ಕೆ
ಸಮುದ್ರದ
ಸಮುದ್ರವೇ
ಸಮೇತ
ಸಮ್ಮತ-ವಿ-ಲ್ಲ
ಸಮ್ಮತಿ-ಯನ್ನು
ಸಮ್ಮಿಲನ-ವಾಗಿ-ರ-ಬೇಕು
ಸಮ್ಮಿಳಿತ-ವಾಗಿ-ದ್ದವು
ಸಮ್ಮುಖ-ದ-ಲ್ಲಿ
ಸಮ್ಮೆಳ-ನದ
ಸಮ್ಮೇಳ-ನ-ದ-ಲ್ಲಿ
ಸಮ್ಮೇಳ-ನ-ದಂತೆ
ಸಮ್ಮೇಳ-ನ-ದಿಂದ
ಸಮ್ಮೇಳ-ನ-ವನ್ನು
ಸಮ್ಮೇಳ-ನ-ವಾಗಿ-ತ್ತು
ಸಮ್ಮೇಳ-ನಕ್ಕೆ
ಸಮ್ಮೇಳ-ನದ
ಸಮ್ಮೇಳನ
ಸಮ್ಮೋಹ
ಸಮ್ಮೋಹ-ನಾ-ಸ್ತ್ರ-ದಿಂದ
ಸಮ್ಮೋಹ-ನಾ-ಸ್ತ್ರ-ವನ್ನು
ಸಮ್ಮೋಹ-ನಾ-ಸ್ತ್ರಕ್ಕೆ
ಸಮ್ಮೋಹನ
ಸಮ್ಮೋಹಿನಿ
ಸಮ್ಮೋಹಿನೀ
ಸರ-ದಾರ್
ಸರ-ಳತೆ
ಸರ-ಸ್ವತಿ
ಸರ-ಸ್ವತಿಯ
ಸರ-ಸ್ವತೀ
ಸರಣಿ
ಸರಪಳಿ
ಸರಪಳೆಯೇ
ಸರಳ
ಸರಳ-ಜೀವ-ನ-ದ-ಲ್ಲಿದ್ದ
ಸರಳ-ವಾಗಿ
ಸರಳ-ವಾಗಿ-ದ್ದರು
ಸರಳ-ವಾಗಿಯೂ
ಸರಳ-ವಾಗಿಯೇ
ಸರಳ-ವಾದ
ಸರಹದ್ದಿನ
ಸರಾಗ-ವಾಗಿ
ಸರಾಗ-ವಾಗಿ-ತ್ತು
ಸರಿ
ಸರಿ-ತೋರಿ-ದರೆ
ಸರಿ-ದಂತೆ
ಸರಿ-ದು-ಹೋಗು-ವುವು
ಸರಿ-ಪಡಿ-ಸಿ-ಕೊಳ್ಳ-ಬೇಕು
ಸರಿ-ಪಡಿ-ಸಿ-ದರೆ
ಸರಿ-ಮಾಡಿ-ದರೆ
ಸರಿ-ಮಾಡು-ವುದು
ಸರಿ-ಯ-ಲ್ಲ
ಸರಿ-ಯಾ-ಗಿದೆ
ಸರಿ-ಯಾಗ-ಬಹು-ದೆಂದು
ಸರಿ-ಯಾಗಿ
ಸರಿ-ಯಾಗಿ-ತ್ತು
ಸರಿ-ಯಾಗಿ-ದ್ದರೂ
ಸರಿ-ಯಾಗಿ-ರು-ವುದು
ಸರಿ-ಯಾಗಿ-ಲ್ಲ
ಸರಿ-ಯಾಗಿ-ಲ್ಲ-ವೆಂದೂ
ಸರಿ-ಯಾಗಿ-ಲ್ಲದೇ
ಸರಿ-ಯಾಗುವುದು
ಸರಿ-ಯಾಗುವುವು
ಸರಿ-ಯಾದ
ಸರಿ-ಯಿ-ಲ್ಲ
ಸರಿ-ಯಿ-ಲ್ಲದ
ಸರಿ-ಯಿತು
ಸರಿ-ಯು-ವುದು
ಸರಿ-ಸ-ಬೇಡವೇ
ಸರಿ-ಸ-ಮಾನ
ಸರಿ-ಸ-ಮಾನ-ರಾಗಿ-ರ-ಬೇಕು
ಸರಿ-ಸ-ಮಾನ-ರಾಗಿ-ರುವರು
ಸರಿ-ಸ-ಮಾನ-ರಿ-ಲ್ಲ
ಸರಿ-ಸಮ-ನಾಗಿ
ಸರಿ-ಸಮ-ವಾ-ಗಿದೆ
ಸರಿ-ಸಮ-ವಾಗಿ
ಸರಿ-ಹೋಗು-ತ್ತಿ-ತ್ತು
ಸರಿ-ಹೋಗು-ವು-ದಿ-ಲ್ಲ-ವೆಂದೂ
ಸರಿ-ಹೋಗು-ವುದು
ಸರಿದು
ಸರೋ-ವ-ರದ
ಸರೋ-ವರ
ಸರೋ-ವರ-ದ-ಲ್ಲಿ
ಸರೋವ-ರಕ್ಕೆ
ಸರೋವ-ರವೂ
ಸರ್
ಸರ್ಕ-ಸ್
ಸರ್ಕ-ಸ್ಸಿನ
ಸರ್ಕಾರ-ದ-ಲ್ಲಿ
ಸರ್ಕಾರ-ದ-ವರು
ಸರ್ಕಾರ-ದೊಂದಿಗೆ
ಸರ್ಕಾರ-ವನ್ನು
ಸರ್ಕಾರದ
ಸರ್ಕಾರಿ
ಸರ್ಕ್ಯು-ಲರ್
ಸರ್ಜ-ನ್ನ-ರ-ನ್ನು
ಸರ್ಜಾ-ನ್
ಸರ್ದಾರ್
ಸರ್ಪ
ಸರ್ಪ-ಗಳೇ
ಸರ್ಪ-ದಂತೆ
ಸರ್ಪದ
ಸರ್ಪವೂ
ಸರ್ಬಿಯ
ಸರ್ವ
ಸರ್ವ-ಜ-ನರೂ
ಸರ್ವ-ಜೀವಿ-ಯನ್ನೂ
ಸರ್ವ-ತೋ-ಮುಖ-ವಾಗಿ
ಸರ್ವ-ತೋ-ಮುಖ-ವಾದ
ಸರ್ವ-ಧರ್ಮ-ಗಳ
ಸರ್ವ-ಧರ್ಮ-ಸ್ವ-ರೂಪಿಣೇ
ಸರ್ವ-ಧರ್ಮದ
ಸರ್ವ-ನಾಶ
ಸರ್ವ-ನಾಶ-ವಾ-ಗು-ತ್ತಿದೆ
ಸರ್ವ-ನಾಶ-ವಾ-ದಂತೆ
ಸರ್ವ-ನಾಶ-ವಾಗು-ವುವು
ಸರ್ವ-ನಾಶದ
ಸರ್ವ-ಭಕ್ಷಕಳು
ಸರ್ವ-ಭಾಗ-ಗಳ-ನ್ನೂ
ಸರ್ವ-ಭಾವ-ಗಳ
ಸರ್ವ-ಭೂತ-ಗಳ-ಲ್ಲಿಯೂ
ಸರ್ವ-ಭೂತೇಶು
ಸರ್ವ-ಮತ
ಸರ್ವ-ಮತ-ಗಳೂ
ಸರ್ವ-ರ-ನ್ನೂ
ಸರ್ವ-ರ-ಲ್ಲಿಯೂ
ಸರ್ವ-ರಿಗೂ
ಸರ್ವ-ವನ್ನೂ
ಸರ್ವ-ವೃ-ತ್ತಿ
ಸರ್ವ-ವ್ಯಾಪಿ
ಸರ್ವ-ವ್ಯಾಪಿ-ಯಾಗಿ-ರು-ವನು
ಸರ್ವ-ವ್ಯಾಪಿ-ಯಾದ
ಸರ್ವ-ವ್ಯಾಪಿ-ಯೆಂದೂ
ಸರ್ವ-ಶಕ್ತ
ಸರ್ವ-ಶಕ್ತನ
ಸರ್ವ-ಶಕ್ತಿ
ಸರ್ವ-ಶ್ರಮ
ಸರ್ವ-ಶ್ರೇಷ್ಠ
ಸರ್ವ-ಸಂಗ
ಸರ್ವ-ಸಾ-ಮಾನ್ಯವೂ
ಸರ್ವ-ಸುಲಭ-ಗ್ರಾಹ್ಯವೂ
ಸರ್ವ-ಸ್ವ
ಸರ್ವ-ಸ್ವ-ವ-ಲ್ಲ
ಸರ್ವ-ಸ್ವ-ವನ್ನು
ಸರ್ವ-ಸ್ವ-ವನ್ನೂ
ಸರ್ವಂ
ಸರ್ವಜ್ಞ
ಸರ್ವದಾ
ಸರ್ವವೂ
ಸರ್ವಶಃ
ಸರ್ವಾಂ-ತರ್ಯಾಮಿ
ಸರ್ವಾವಯವ
ಸರ್ವಿ-ಸ್
ಸರ್ವೋ-ತ್ಕೃಷ್ಟ-ವಾದ
ಸಲ
ಸಲ-ಕರ-ಣೆ-ಗಳು
ಸಲ-ಹಿ-ದಳು
ಸಲ-ಹೆ-ಗ-ಳನ್ನು
ಸಲ-ಹೆ-ಗಳ
ಸಲ-ಹೆ-ಯಂತೆ
ಸಲ-ಹೆ-ಯನ್ನು
ಸಲ-ಹೆ-ಯನ್ನೇ
ಸಲ-ಹೆಯ
ಸಲಕ್ಕೆ
ಸಲದ
ಸಲಹೆ
ಸಲಾ-ಕೆ-ಗ-ಳನ್ನು
ಸಲು-ವಾಗಿ
ಸಲೂ-ನಿಗೆ
ಸವ-ರು-ತ್ತಿದ್ದರು
ಸವಾ-ಲ-ನ್ನು
ಸವಿ
ಸವಿ-ದಿನ-ಗ-ಳನ್ನು
ಸವಿ-ಯನ್ನು
ಸವಿ-ಯುವ
ಸವೆ-ಸಿ-ದರೆ
ಸವೆ-ಸು-ತ್ತಾನೆ
ಸವೆದು-ಹೋಗಿ-ರ-ಬಹುದು
ಸಸಿ-ಗಳು
ಸಹ-ಕರಿ-ಸಲಾಗದು
ಸಹ-ಕರಿ-ಸಿ-ದರೆ
ಸಹ-ಕಾರಿ
ಸಹ-ಕಾರಿ-ಗ-ಳಾದ
ಸಹ-ಕಾರಿ-ಗಳೂ
ಸಹ-ಕಾರಿ-ಯಾಗುವುದು
ಸಹ-ಪಾಠಿ-ಗಳೊ-ಡನೆ
ಸಹ-ಸ್ರ
ಸಹ-ಸ್ರ-ದಳದ
ಸಹ-ಸ್ರ-ದ್ವೀಪೋದ್ಯಾ-ನಕ್ಕೆ
ಸಹ-ಸ್ರ-ದ್ವೀಪೋದ್ಯಾನ
ಸಹ-ಸ್ರ-ದ್ವೀಪೋದ್ಯಾನ-ದಿಂದ
ಸಹ-ಸ್ರಾರ-ವನ್ನು
ಸಹ-ಸ್ರಾರು
ಸಹಜ
ಸಹಜ-ವಾಗಿಯೇ
ಸಹಜ-ವಾದ
ಸಹಜ-ಸ್ಥಿತಿ-ಯ-ಲ್ಲಿ-ದ್ದರು
ಸಹಜ-ಸ್ಥಿತಿಗೆ
ಸಹಜ-ಸ್ವ-ಭಾವ
ಸಹನೆ
ಸಹನೆ-ಗೆಡ-ಲಿ-ಲ್ಲ
ಸಹನೆ-ಯಿಂದ
ಸಹಪಂಕ್ತಿ-ಯ-ಲ್ಲಿ
ಸಹಾ-ಯಾರ್ಥ-ವಾಗಿ
ಸಹಾನು-ಭೂತಿ
ಸಹಾನು-ಭೂತಿ-ಗಳು
ಸಹಾನು-ಭೂತಿ-ಯ-ಲ್ಲಿ
ಸಹಾನು-ಭೂತಿ-ಯನ್ನು
ಸಹಾನು-ಭೂತಿ-ಯಿ-ಲ್ಲ
ಸಹಾನು-ಭೂತಿ-ಯಿಂದ
ಸಹಾನು-ಭೂತಿ-ಯಿದೆ
ಸಹಾನು-ಭೂತಿಯ
ಸಹಾನು-ಭೂತಿಯೂ
ಸಹಾಯ
ಸಹಾಯ-ಕ-ನಾ-ದನು
ಸಹಾಯ-ಕ-ನಾಗುವೆ
ಸಹಾಯ-ಕ-ರಾಗಿ
ಸಹಾಯ-ಕ-ರಾಗಿರಿ
ಸಹಾಯ-ಕ-ರಿ-ಲ್ಲ
ಸಹಾಯ-ಕ-ವಾಗು-ವುದು
ಸಹಾಯ-ಕರಾಗ-ದಿ-ದ್ದ-ಲ್ಲಿ
ಸಹಾಯ-ಕಳಾಗ-ಬೇಕೆಂದು
ಸಹಾಯ-ಕ್ಕಾಗಿ
ಸಹಾಯ-ಗಳಿ-ಲ್ಲದೆ
ಸಹಾಯ-ದಿಂದ
ಸಹಾಯ-ಮಾಡ-ಬ-ಲ್ಲವು
ಸಹಾಯ-ಮಾಡಲು
ಸಹಾಯ-ಮಾಡಿ
ಸಹಾಯ-ಮಾಡಿ-ದರೆ
ಸಹಾಯ-ಮಾಡಿದ
ಸಹಾಯ-ಮಾಡಿದೆ
ಸಹಾಯ-ಮಾಡು-ವರು
ಸಹಾಯ-ಮಾಡು-ವುದು
ಸಹಾಯ-ಮಾಡು-ವುವು
ಸಹಾಯ-ವನ್ನು
ಸಹಾಯ-ವನ್ನೂ
ಸಹಾಯ-ವಾಗ-ಬಹು-ದೆಂದು
ಸಹಾಯ-ವಾಗ-ಬೇ-ಕಾ-ದರೆ
ಸಹಾಯ-ವಾಗು-ವಂತಹ
ಸಹಾಯ-ವಾಗು-ವುದು
ಸಹಾಯ-ವಾದೀತು
ಸಹಾಯ-ವಿ-ಲ್ಲದೆ
ಸಹಾಯಕ್ಕೆ
ಸಹಾಯವೂ
ಸಹಿ-ಸ-ಲಿ-ಲ್ಲ
ಸಹಿ-ಸಲ-ಸದ-ಳ-ವಾ-ಯಿತು
ಸಹಿ-ಸಲ-ಸದ-ಳ-ವಾಗಿ-ರು-ತ್ತಿ-ತ್ತು
ಸಹಿ-ಸಲ-ಸದ-ಳ-ವಾದ
ಸಹಿ-ಸಲ-ಸಾಧ್ಯ-ವಾದ
ಸಹಿ-ಸಲಾಗ-ಲಿ-ಲ್ಲ
ಸಹಿ-ಸಲಾರದೆ
ಸಹಿ-ಸಲಾರರು
ಸಹಿ-ಸಲಾರೆ
ಸಹಿ-ಸಲು
ಸಹಿ-ಸು-ತ್ತಿ-ರ-ಲಿ-ಲ್ಲ
ಸಹಿ-ಸು-ತ್ತಿ-ರುವೆ
ಸಹಿ-ಸು-ತ್ತೇವೆ
ಸಹಿ-ಸು-ವು-ದಕ್ಕೆ
ಸಹಿಷ್ಣು-ಗ-ಳಾದ
ಸಹಿಷ್ಣುತೆ
ಸಹಿಸ-ಬ-ಲ್ಲದು
ಸಹಿಸ-ಬ-ಲ್ಲರು
ಸಹಿಸ-ಬ-ಲ್ಲುದು
ಸಹಿಸ-ಲಾಗು-ವು-ದಿ-ಲ್ಲ
ಸಹಿಸಿ
ಸಹಿಸು-ವು-ದಕ್ಕೂ
ಸಹೊ-ದರ
ಸಹೊ-ದರನು
ಸಹೊ-ದರಿ-ಯನ್ನು
ಸಹೊದ-ರರೇ
ಸಹೊದ-ರಿಯರೆ
ಸಹೋ
ಸಾ
ಸಾಂ
ಸಾಂಕ್ರಾಮಿಕ
ಸಾಂಖ್ಯ
ಸಾಂಖ್ಯ-ತ-ತ್ತ್ವದ
ಸಾಂಖ್ಯ-ದರ್ಶನ-ದ-ಲ್ಲಿ
ಸಾಂಖ್ಯಂ
ಸಾಂಡರ್ಸ
ಸಾಂದ್ರ-ತೆಯ
ಸಾಂಧ್ಯಾ-ಕಾಲ
ಸಾಕ-ತೊಡಗಿದರು
ಸಾಕ-ಲ್ಲವೆ
ಸಾಕಾ-ಗಿದೆ
ಸಾಕಾ-ಯಿತು
ಸಾಕಾಗಿ
ಸಾಕಾಗಿ-ತ್ತು
ಸಾಕಾಗಿ-ಹೋಗಿ-ತ್ತು
ಸಾಕಾಗು-ವಷ್ಟು
ಸಾಕಿ
ಸಾಕಿದ್ದ
ಸಾಕು
ಸಾಕು-ತಾಯಿ-ಯಂತೆ
ಸಾಕು-ಮ-ಗಳು
ಸಾಕು-ಮಾಡು
ಸಾಕೆ
ಸಾಕ್ಷಾ-ತ್
ಸಾಕ್ಷಾ-ತ್ಕ-ರಿಸಿ-ಕೊ-ಳ್ಳಲಾ-ರದೆ
ಸಾಕ್ಷಾ-ತ್ಕ-ರಿಸಿ-ಕೊಂಡಿ-ರುವ
ಸಾಕ್ಷಾ-ತ್ಕ-ರಿಸಿ-ಕೊಳ್ಳ-ಬ-ಲ್ಲರು
ಸಾಕ್ಷಾ-ತ್ಕ-ರಿಸಿ-ಕೊಳ್ಳಿ
ಸಾಕ್ಷಾ-ತ್ಕ-ರಿಸಿ-ಕೊಳ್ಳು-ವು-ದಾಗಿದೆ
ಸಾಕ್ಷಾ-ತ್ಕಾ-ರಕ್ಕೆ
ಸಾಕ್ಷಾ-ತ್ಕಾರ
ಸಾಕ್ಷಾ-ತ್ಕಾರ-ಗ-ಳನ್ನು
ಸಾಕ್ಷಾ-ತ್ಕಾರ-ಗೊಳಿಸಿ-ಕೊಳ್ಳ-ಬೇಕು
ಸಾಕ್ಷಾ-ತ್ಕಾರ-ದಿಂದ
ಸಾಕ್ಷಾ-ತ್ಕಾರ-ಮಾಡಿ-ಕೊ-ಳ್ಳಲು
ಸಾಕ್ಷಾ-ತ್ಕಾರ-ವನ್ನು
ಸಾಕ್ಷಾ-ತ್ಕಾರ-ವಾಗಲಿ
ಸಾಕ್ಷಾ-ತ್ಕಾರ-ವಾಗಿ-ಲ್ಲ
ಸಾಕ್ಷಾ-ತ್ಕಾರ-ವಾಗು-ವುದೆಂಬ
ಸಾಕ್ಷಾ-ತ್ಕಾರ-ವಾದ
ಸಾಕ್ಷಾ-ತ್ಕಾರದ
ಸಾಕ್ಷಾ-ತ್ಕಾರವೂ
ಸಾಕ್ಷಾ-ತ್ಕಾರವೇ
ಸಾಕ್ಷಾ-ತ್ಕಾರಾವೇ
ಸಾಕ್ಷಾ-ತ್ಕಾರಿ-ಸಿ-ಕೊ-ಳ್ಳುವರೋ
ಸಾಕ್ಷಾ-ತ್ತಾಗಿ
ಸಾಕ್ಷಿ
ಸಾಕ್ಷಿ-ಯಾ-ಗಿದೆ
ಸಾಕ್ಷಿ-ಯಾಗಿ
ಸಾಕ್ಷಿ-ಯಾಗಿ-ದ್ದರು
ಸಾಕ್ಷಿ-ಯಾಗಿ-ದ್ದವು
ಸಾಕ್ಷೀ-ಭೂತ-ರಾಗಿ
ಸಾಗ-ರಕ್ಕೆ
ಸಾಗಿ
ಸಾಗಿ-ದರು
ಸಾಗಿ-ಬ-ರು-ವಂತೆ
ಸಾಗಿ-ಸಿ-ಕೊಂಡು
ಸಾಗಿ-ಸಿ-ದಂತೆ
ಸಾಗಿ-ಸಿ-ರುವರು
ಸಾಗಿ-ಹೋಗು-ತ್ತಾನೋ
ಸಾಗಿತು
ಸಾಗಿಸು
ಸಾಗು-ತ್ತಿರು-ತ್ತೇವೆ
ಸಾಗು-ತ್ತಿರುವ
ಸಾಗು-ವು-ದಿ-ಲ್ಲ
ಸಾಗುವ
ಸಾಗುವುದು
ಸಾಚಾ
ಸಾಧ
ಸಾಧ-ಕ-ನಿಗೆ
ಸಾಧ-ಕ-ರ-ಲ್ಲಿ
ಸಾಧ-ಕನ
ಸಾಧ-ಕನು
ಸಾಧ-ನ-ವಾಗ-ಬೇಕು
ಸಾಧ-ನೆ-ಗ-ಳನ್ನು
ಸಾಧ-ನೆ-ಗ-ಳಾದ
ಸಾಧ-ನೆ-ಗ-ಳೆಂಬ
ಸಾಧ-ನೆ-ಗಳಿವೆಯೋ
ಸಾಧ-ನೆ-ಗಾಗಿಯೇ
ಸಾಧ-ನೆ-ಮಾಡಿದ
ಸಾಧ-ನೆ-ಯ-ಲ್ಲಿ
ಸಾಧ-ನೆ-ಯನ್ನು
ಸಾಧ-ನೆ-ಯನ್ನೂ
ಸಾಧ-ನೆ-ಯಾದ
ಸಾಧ-ನೆ-ಯಿಂದ
ಸಾಧ-ನೆ-ಯಿಂದಲೆ
ಸಾಧ-ನೆಗೆ
ಸಾಧ-ನೆಯ
ಸಾಧ-ನೆಯೇ
ಸಾಧಕ
ಸಾಧಾರಾಣ
ಸಾಧಿ-ಸದ
ಸಾಧಿ-ಸಲಾರರು
ಸಾಧಿ-ಸಲು
ಸಾಧಿ-ಸಲೇ-ಬೇಕು
ಸಾಧಿ-ಸಿ-ಲ್ಲ
ಸಾಧಿ-ಸಿತು
ಸಾಧಿ-ಸಿದೆ
ಸಾಧಿ-ಸು-ವು-ದಕ್ಕೆ
ಸಾಧಿ-ಸು-ವು-ದಿ-ಲ್ಲ
ಸಾಧಿ-ಸುವ
ಸಾಧಿತ-ವಾಗ-ಬೇಕು
ಸಾಧಿಸ-ಬ-ಲ್ಲ
ಸಾಧಿಸ-ಬ-ಲ್ಲದು
ಸಾಧಿಸ-ಬಹು-ದೆಂದು
ಸಾಧಿಸ-ಬಹು-ದೆಂಬು-ದ-ನ್ನು
ಸಾಧಿಸ-ಬಹುದು
ಸಾಧಿಸ-ಬೇಕಾಗಿದೆ
ಸಾಧಿಸ-ಬೇಕು
ಸಾಧಿಸಿ
ಸಾಧಿಸಿ-ದಂತೆ
ಸಾಧಿಸಿ-ದನು
ಸಾಧಿಸಿ-ದರು
ಸಾಧಿಸಿ-ದರೆ
ಸಾಧಿಸಿ-ದು-ದ-ನ್ನು
ಸಾಧಿಸಿ-ರು-ವೆವು
ಸಾಧಿಸಿ-ರುವರು
ಸಾಧಿಸಿವೆ
ಸಾಧು
ಸಾಧು-ಗ-ಳನ್ನು
ಸಾಧು-ಗ-ಳಿಗೆ
ಸಾಧು-ಗಳ
ಸಾಧು-ಗಳ-ನ್ನೂ
ಸಾಧು-ಗಳ-ಲ್ಲದೆ
ಸಾಧು-ಗಳ-ಲ್ಲಿ
ಸಾಧು-ಗಳಾಗಲಾ-ರರು
ಸಾಧು-ಗಳು
ಸಾಧು-ಗಳೆ-ಲ್ಲ
ಸಾಧು-ಗಳೊ-ಡನೆ
ಸಾಧು-ಗಳೊ-ಬ್ಬರು
ಸಾಧು-ಜೀವನ
ಸಾಧು-ಪುರುಷ
ಸಾಧು-ಪುರುಷ-ರ-ನ್ನು
ಸಾಧು-ವನ್ನು
ಸಾಧು-ವಿಗೆ
ಸಾಧು-ವಿನ
ಸಾಧು-ಸಂ-ತರ
ಸಾಧು-ಸಂ-ತರು
ಸಾಧು-ಸಂತ-ರ-ನ್ನು
ಸಾಧ್ಯ
ಸಾಧ್ಯ-ತೆ-ಗಳಾವುವೂ
ಸಾಧ್ಯ-ವಾ-ದರೆ
ಸಾಧ್ಯ-ವಾ-ದಷ್ಟು
ಸಾಧ್ಯ-ವಾ-ಯಿ-ತೆಂದೂ
ಸಾಧ್ಯ-ವಾ-ಯಿತು
ಸಾಧ್ಯ-ವಾಗ-ಬಹು-ದೆಂದೂ
ಸಾಧ್ಯ-ವಾಗ-ಬಹುದು
ಸಾಧ್ಯ-ವಾಗ-ಲಾ-ರದು
ಸಾಧ್ಯ-ವಾಗ-ಲಿ-ಲ್ಲ
ಸಾಧ್ಯ-ವಾಗ-ಲಿ-ಲ್ಲ-ವ-ಲ್ಲ
ಸಾಧ್ಯ-ವಾಗ-ಲಿ-ಲ್ಲವೋ
ಸಾಧ್ಯ-ವಾಗದೆ
ಸಾಧ್ಯ-ವಾಗಿ-ತ್ತೋ
ಸಾಧ್ಯ-ವಾಗಿ-ರು-ವು-ದ-ರಿಂದ
ಸಾಧ್ಯ-ವಾಗು-ತ್ತಿ-ತ್ತು
ಸಾಧ್ಯ-ವಾಗು-ತ್ತಿ-ರಲಿ-ಲ್ಲ
ಸಾಧ್ಯ-ವಾಗು-ವಂತಿರ-ಬೇಕು
ಸಾಧ್ಯ-ವಾಗು-ವಂತೆ
ಸಾಧ್ಯ-ವಾಗು-ವು-ದಿ-ಲ್ಲ
ಸಾಧ್ಯ-ವಾಗು-ವುದು
ಸಾಧ್ಯ-ವಾಗು-ವುದೆ
ಸಾಧ್ಯ-ವಾಗುವ
ಸಾಧ್ಯ-ವಾದ
ಸಾಧ್ಯ-ವಾದ-ಲ್ಲದೇ
ಸಾಧ್ಯ-ವಿ-ರ-ಲಿ-ಲ್ಲ
ಸಾಧ್ಯ-ವಿ-ಲ್ಲ
ಸಾಧ್ಯ-ವಿ-ಲ್ಲ-ವೆಂ-ದರು
ಸಾಧ್ಯ-ವಿ-ಲ್ಲ-ವೆಂದು
ಸಾಧ್ಯ-ವಿ-ಲ್ಲ-ವೆಂದೂ
ಸಾಧ್ಯ-ವಿ-ಲ್ಲದೆ
ಸಾಧ್ಯ-ವಿ-ಲ್ಲದೇ
ಸಾಧ್ಯ-ವಿ-ಲ್ಲವೆ
ಸಾಧ್ಯ-ವಿ-ಲ್ಲವೊ
ಸಾಧ್ಯ-ವಿ-ಲ್ಲವೋ
ಸಾಧ್ಯ-ವೆಂ-ದರು
ಸಾಧ್ಯ-ವೆಂದು
ಸಾಧ್ಯ-ವೆಂದೂ
ಸಾಧ್ಯತೆ
ಸಾಧ್ಯವೆ
ಸಾಧ್ಯವೇ
ಸಾಧ್ಯವೋ
ಸಾಧ್ಯಾ-ಸಾಧ್ಯ-ತೆ-ಗಳ-ನ್ನೆ-ಲ್ಲ
ಸಾಧ್ಯು-ಗಳು
ಸಾಧ್ವಿ
ಸಾಧ್ವೀ-ಮಣಿ
ಸಾಪೇಕ್ಷ
ಸಾಪೇಕ್ಷ-ವ-ಸ್ತು-ವಿಗೆ
ಸಾಪೇಕ್ಷ-ವಾ-ಯಿತು
ಸಾಪೇಕ್ಷ-ವಾದ
ಸಾಮಂಜ-ಸ್ಯವು
ಸಾಮಗ್ರಿ
ಸಾಮಗ್ರಿ-ಗ-ಳನ್ನು
ಸಾಮಗ್ರಿ-ಗಳ
ಸಾಮಾ-ನ-ನ್ನೆ-ಲ್ಲ
ಸಾಮಾ-ನನ್ನು
ಸಾಮಾಜಿಕ
ಸಾಮಾಧಿ-ಗಳು
ಸಾಮಾಧಿ-ಸ್ಥ-ರಾ-ದರು
ಸಾಮೀ-ಜಿಗೆ
ಸಾಮೀಜಿ
ಸಾಮೀಜಿ-ಯ-ವ-ರಿಗೆ
ಸಾಮೀಜಿ-ಯ-ವರು
ಸಾಮೂಹಿಕ
ಸಾಯ-ಬಹುದು
ಸಾಯ-ಬೇಕು
ಸಾಯ-ಬೇಕೆಂದು
ಸಾಯಂ-ಕಾಲ
ಸಾಯಂ-ಕಾಲ-ದ-ವ-ರೆಗೆ
ಸಾಯಂ-ಕಾಲ-ವಾದ
ಸಾಯಂ-ಕಾಲದ
ಸಾಯಂ-ಕಾಲವೂ
ಸಾಯಂ-ಕಾಲವೆ
ಸಾಯಂ-ಕಾಲವೇ
ಸಾಯಲೇ
ಸಾಯಿ-ಸಿ-ದರೂ
ಸಾಯು-ತ್ತಾರೆ
ಸಾಯು-ತ್ತಿದ್ದಾರೆ
ಸಾಯು-ತ್ತಿದ್ದಾರೆಂದು
ಸಾಯು-ವಾಗಲೂ
ಸಾಯು-ವೆವು
ಸಾರ
ಸಾರ-ತೊಡಗಿದರು
ಸಾರ-ಬೇ-ಕಾ-ದರೆ
ಸಾರ-ಬೇಕು
ಸಾರ-ಬೇಕೆಂದು
ಸಾರ-ವನ್ನು
ಸಾರ-ವನ್ನೆ-ಲ್ಲ
ಸಾರ-ವನ್ನೇ
ಸಾರ-ವಾದ
ಸಾರ-ವೆ-ಲ್ಲ
ಸಾರಥಿ
ಸಾರಲು
ಸಾರವೇ
ಸಾರಾ
ಸಾರಾ-ನಾಥ-ದ-ಲ್ಲಿ
ಸಾರಾಂಶ-ವನ್ನು
ಸಾರಾಂಶ-ವನ್ನೆ-ಲ್ಲ
ಸಾರಾಯಿ
ಸಾರಿ
ಸಾರಿ-ದ-ವ-ರೆ-ಲ್ಲಾ
ಸಾರಿ-ದಂ-ತಾಗು-ವು-ದಿ-ಲ್ಲ
ಸಾರಿ-ದನು
ಸಾರಿ-ದರು
ಸಾರಿ-ದರೂ
ಸಾರಿ-ದಾಗ
ಸಾರಿ-ಬಿ-ಟ್ಟೆ
ಸಾರಿ-ರು-ವನು
ಸಾರಿತು
ಸಾರಿದ
ಸಾರು
ಸಾರು-ತ್ತದೆ
ಸಾರು-ತ್ತಾನೆ
ಸಾರು-ತ್ತಿ-ತ್ತು
ಸಾರು-ತ್ತಿದ್ದವು
ಸಾರು-ತ್ತಿವೆ
ಸಾರು-ತ್ತೇನೆ
ಸಾರು-ವ-ವರ
ಸಾರು-ವಂತೆ
ಸಾರು-ವರು
ಸಾರು-ವು-ದಕ್ಕೆ
ಸಾರು-ವು-ದಿ-ಲ್ಲ
ಸಾರು-ವು-ದಿ-ಲ್ಲವೆ
ಸಾರು-ವುದು
ಸಾರುವ
ಸಾರ್ಥ-ಕ-ವಾಗಿ-ಸಿ-ದರು
ಸಾರ್ಥಕ-ವಾ-ಯಿತು
ಸಾರ್ಥಕ-ವಾ-ಯಿತೆಂದು
ಸಾರ್ಥಕ-ವಾಗ-ಲಿ-ಲ್ಲ-ವೆಂದೂ
ಸಾರ್ಥಕ-ವಾಗ-ಲಿ-ಲ್ಲವೆ
ಸಾರ್ಥಕ-ವಾಗು-ವಂತೇ
ಸಾರ್ಥಕ-ವಿ-ಲ್ಲ
ಸಾರ್ಥಕ-ವೆಂದು
ಸಾರ್ವ-ಜನಿಕ
ಸಾರ್ವ-ಜನಿಕ-ರ-ಲ್ಲಿ
ಸಾರ್ವ-ಜನಿಕ-ರಿಗೆ
ಸಾರ್ವತ್ರಿಕ
ಸಾರ್ವಭೌ-ಮರು
ಸಾಲ-ಗಾರರು
ಸಾಲ-ದಾ-ಯಿತು
ಸಾಲ-ದಾಗು-ತ್ತದೆ
ಸಾಲ-ದಾಗು-ವುದು
ಸಾಲ-ಮನ್ನನ
ಸಾಲ-ವನ್ನು
ಸಾಲ-ವಾಗಿ
ಸಾಲಂಕೃತ
ಸಾಲದಾ-ಗಿದೆ
ಸಾಲದು
ಸಾಲದೆ
ಸಾಲಿ-ಗ್ರಾಮ
ಸಾಲಿ-ಗ್ರಾಮ-ದಂತೆ
ಸಾಲಿ-ನ-ಲ್ಲಿ
ಸಾಲು
ಸಾಲು-ತ್ತದೆ
ಸಾಲು-ತ್ತಿ-ರಲಿ-ಲ್ಲ
ಸಾವಧಾ-ನ-ದಿಂದ
ಸಾವಿರ
ಸಾವಿರ-ದ-ವರೆ-ಗಾ-ದರೂ
ಸಾವಿರ-ಪಾಲು
ಸಾವಿರದ
ಸಾವಿರಾರು
ಸಾವು
ಸಾವು-ಗಳಿಗೀಡಾ-ದರೂ
ಸಾವೇ
ಸಾಷ್ಟಾಂಗ
ಸಾಸಿವೆ-ಕಾಳಿಗೂ
ಸಾಹಸ
ಸಾಹಸ-ಗಳಿವೆ
ಸಾಹಸ-ಗಳು
ಸಾಹಸ-ದಿಂದ
ಸಾಹಸ-ವನ್ನು
ಸಾಹಸ-ವನ್ನೆ-ಲ್ಲ
ಸಾಹಸ-ವನ್ನೆ-ಲ್ಲಾ
ಸಾಹಸ-ವಿದೆಯೊ
ಸಾಹಸವೇ
ಸಾಹಿಸಿ-ಗಳೊ
ಸಾಹೇಬನು
ಸಾಹೇಬರು
ಸಾಹೇಬ್
ಸಿ
ಸಿಂ
ಸಿಂಗ-ಪು-ರಕ್ಕೆ
ಸಿಂಗ-ಪುರ-ದ-ಲ್ಲಿ
ಸಿಂಗ-ರಿಸಿ-ಕೊಂಡಂತೆ
ಸಿಂಗಪುರ
ಸಿಂಧು-ವಿ-ನ-ಲ್ಲಿ
ಸಿಂಧೂ-ನದಿಯ
ಸಿಂಧ್
ಸಿಂಹ
ಸಿಂಹ-ಗಳ-ನ್ನಾಗಿ
ಸಿಂಹ-ಗಳು
ಸಿಂಹ-ದಂತೆ
ಸಿಂಹ-ನಿರಂ-ತರ
ಸಿಂಹ-ಳ-ದ್ವೀ-ಪದ
ಸಿಂಹ-ವನ್ನು
ಸಿಂಹ-ಸದೃಶ
ಸಿಂಹ-ಸದೃಶ-ರಾಗಿ
ಸಿಂಹ-ಸದೃಶ-ರಾಗುವಿರಿ
ಸಿಂಹದ
ಸಿಂಹಳ
ಸಿಂಹವೇ
ಸಿಂಹಾ-ಸನ
ಸಿಂಹಾ-ಸನ-ವಿ-ಲ್ಲದೆ
ಸಿಂಹಾಸ-ನಕ್ಕೆ
ಸಿಂಹಾಸ-ನದ
ಸಿಂಹಿಣಿ-ಯಂತಿದ್ದ
ಸಿಎಚ್
ಸಿಐಡಿ
ಸಿಕ್ಕ-ದಿದ್ದಾಗ
ಸಿಕ್ಕ-ಬಹು-ದೆಂದು
ಸಿಕ್ಕ-ಬಹು-ದೇನೊ
ಸಿಕ್ಕ-ಬೇಕು
ಸಿಕ್ಕ-ಲಿ-ಲ್ಲ
ಸಿಕ್ಕದೆ
ಸಿಕ್ಕದೇ
ಸಿಕ್ಕರ
ಸಿಕ್ಕಲಾ-ರ-ದೆಂದು
ಸಿಕ್ಕಲಿ
ಸಿಕ್ಕಲಿ-ಲ್ಲ-ವ-ಲ್ಲ
ಸಿಕ್ಕಾ-ಪಟ್ಟೆ
ಸಿಕ್ಕಿ
ಸಿಕ್ಕಿ-ಕೊ-ಳ್ಳು-ವು-ದಕ್ಕೆ
ಸಿಕ್ಕಿ-ಕೊಂಡಿ-ರುವಂತಿದೆ
ಸಿಕ್ಕಿ-ಕೊಂಡಿದೆ
ಸಿಕ್ಕಿ-ಕೊಂಡಿದ್ದಾರೆ
ಸಿಕ್ಕಿ-ಕೊಂಡು
ಸಿಕ್ಕಿ-ದ-ನ-ಲ್ಲ
ಸಿಕ್ಕಿ-ದ-ವ-ರ-ನ್ನು
ಸಿಕ್ಕಿ-ದ-ವರ
ಸಿಕ್ಕಿ-ದರು
ಸಿಕ್ಕಿ-ದರೂ
ಸಿಕ್ಕಿ-ದರೆ
ಸಿಕ್ಕಿ-ದಾಗ
ಸಿಕ್ಕಿ-ದಾಗ-ಲೆ-ಲ್ಲ
ಸಿಕ್ಕಿ-ದು-ದ-ಕ್ಕಾಗಿ
ಸಿಕ್ಕಿ-ದೊಡ-ನೆಯೇ
ಸಿಕ್ಕಿ-ದ್ದರೆ
ಸಿಕ್ಕಿ-ಬಿ-ಟ್ಟ-ನೆಂದು
ಸಿಕ್ಕಿ-ಬಿಡು-ವನು
ಸಿಕ್ಕಿ-ಬೀಳು-ವರು
ಸಿಕ್ಕಿ-ರ-ಲಿ-ಲ್ಲ
ಸಿಕ್ಕಿ-ರು-ವು-ದ-ನ್ನು
ಸಿಕ್ಕಿ-ರುವ
ಸಿಕ್ಕಿ-ರುವರು
ಸಿಕ್ಕಿ-ರುವಾಗ
ಸಿಕ್ಕಿ-ಲ್ಲ
ಸಿಕ್ಕಿ-ಸಿ-ಕೊಂಡು
ಸಿಕ್ಕಿ-ಹಾಕಿ-ಕೊ-ಳ್ಳುವಂತಹ
ಸಿಕ್ಕಿ-ಹಾಕಿ-ಕೊ-ಳ್ಳುವೆವು
ಸಿಕ್ಕಿ-ಹಾಕಿ-ಕೊಂಡು
ಸಿಕ್ಕಿತು
ಸಿಕ್ಕಿದ
ಸಿಕ್ಕಿದೆ
ಸಿಕ್ಕಿದ್ದ
ಸಿಕ್ಕಿದ್ದು
ಸಿಕ್ಕೀತೇನೊ
ಸಿಕ್ಕೀತೇನೋ
ಸಿಕ್ಕು-ತ್ತವೆ
ಸಿಕ್ಕು-ತ್ತಾ-ನೇನು
ಸಿಕ್ಕು-ತ್ತಾನೆ
ಸಿಕ್ಕು-ತ್ತಿ-ರಲಿ-ಲ್ಲ
ಸಿಕ್ಕು-ತ್ತಿ-ಲ್ಲ
ಸಿಕ್ಕು-ವಂ-ತಿ-ಲ್ಲ
ಸಿಕ್ಕು-ವಂತ-ಹು-ದ-ಲ್ಲ
ಸಿಕ್ಕು-ವಂತೆ
ಸಿಕ್ಕು-ವನು
ಸಿಕ್ಕು-ವರು
ಸಿಕ್ಕು-ವಾಗ
ಸಿಕ್ಕು-ವು-ದಕ್ಕೆ
ಸಿಕ್ಕು-ವು-ದಿ-ಲ್ಲ
ಸಿಕ್ಕು-ವು-ದಿ-ಲ್ಲ-ವೆಂದು
ಸಿಕ್ಕು-ವು-ದಿ-ಲ್ಲವೇ
ಸಿಕ್ಕು-ವು-ದೆಂದು
ಸಿಕ್ಕು-ವುದು
ಸಿಕ್ಕು-ವುದೆ
ಸಿಕ್ಕು-ವುದೋ
ಸಿಕ್ಕು-ವುವು
ಸಿಕ್ಕುವ
ಸಿಕ್ಕುವ-ವ-ರೆಗೆ
ಸಿಗ-ದಂತೆ
ಸಿಗ-ಬಹುದು
ಸಿಡಿ-ಯು-ತ್ತಿದ್ದರೆ
ಸಿಡಿದ
ಸಿಡಿದು-ಬಂದ
ಸಿಡಿಮದ್ದಿ-ನಂತೆ
ಸಿಡಿಲಾಳ್ಮೆ
ಸಿಡಿಲಿ-ನಂತಹ
ಸಿಡಿಲಿ-ನಂತೆ
ಸಿಡಿಲು
ಸಿಡಿಲು-ಗಳಿಗೂ
ಸಿಡ್ನಿಯ
ಸಿದ್ಧ
ಸಿದ್ಧ-ತೆ-ಗ-ಳಾಗು-ತ್ತಿ-ತ್ತು
ಸಿದ್ಧ-ನಾ-ದರೆ
ಸಿದ್ಧ-ನಾಗ-ಬೇಕು
ಸಿದ್ಧ-ನಾಗಿ-ದ್ದನೇ
ಸಿದ್ಧ-ನಾಗಿ-ದ್ದಾನೆ
ಸಿದ್ಧ-ನಾಗಿ-ರ-ಬೇಕು
ಸಿದ್ಧ-ನಾಗಿ-ರು-ವನು
ಸಿದ್ಧ-ನಾಗಿ-ರು-ವೆನು
ಸಿದ್ಧ-ನಾಗಿ-ರುವೆ
ಸಿದ್ಧ-ನಾಗು
ಸಿದ್ಧ-ನಾಗು-ವು-ದಕ್ಕೆ
ಸಿದ್ಧ-ನಿ-ರುವೆ
ಸಿದ್ಧ-ಪಡಿ-ಸಲಾ-ಯಿತು
ಸಿದ್ಧ-ಪಡಿ-ಸಲು
ಸಿದ್ಧ-ಪಡಿ-ಸಿ-ಕೊಂಡದ್ದು
ಸಿದ್ಧ-ಪಡಿ-ಸಿ-ಕೊಂಡು
ಸಿದ್ಧ-ಪಡಿಸಿ
ಸಿದ್ಧ-ಮಾಡಿ-ಕೊಂಡು
ಸಿದ್ಧ-ರಾ-ದರು
ಸಿದ್ಧ-ರಾಗ-ಬೇಕು
ಸಿದ್ಧ-ರಾಗಿ
ಸಿದ್ಧ-ರಾಗಿ-ದ್ದ-ವರು
ಸಿದ್ಧ-ರಾಗಿ-ದ್ದರು
ಸಿದ್ಧ-ರಾಗಿ-ದ್ದರೇ
ಸಿದ್ಧ-ರಾಗಿ-ದ್ದಾರೆ
ಸಿದ್ಧ-ರಾಗಿ-ರ-ಬೇಕು
ಸಿದ್ಧ-ರಾಗಿ-ರುವರು
ಸಿದ್ಧ-ಳಾಗಿ-ದ್ದಳು
ಸಿದ್ಧ-ವಾ-ಗಿ-ದ್ದರು
ಸಿದ್ಧ-ವಾ-ಗಿದೆ
ಸಿದ್ಧ-ವಾ-ದರು
ಸಿದ್ಧ-ವಾ-ಯಿತು
ಸಿದ್ಧ-ವಾಗಿ-ತ್ತು
ಸಿದ್ಧ-ವಾಗಿ-ದ್ದವು
ಸಿದ್ಧ-ವಾಗಿ-ದ್ದೇವೆ
ಸಿದ್ಧ-ವಾಗಿ-ರು-ವು-ದ-ನ್ನು
ಸಿದ್ಧ-ವಾಗಿ-ರುವೆ
ಸಿದ್ಧ-ವಾಗು-ತ್ತದೆ
ಸಿದ್ಧ-ವಾಗು-ತ್ತಿ-ದ್ದು-ದ-ನ್ನು
ಸಿದ್ಧ-ವಿ-ರು-ವು-ದಾಗಿ
ಸಿದ್ಧ-ವಿ-ರು-ವು-ದಾಗಿಯೂ
ಸಿದ್ಧತೆ
ಸಿದ್ಧಾಂತ
ಸಿದ್ಧಾಂತ-ಕಾರನೂ
ಸಿದ್ಧಾಂತ-ಗ-ಳನ್ನು
ಸಿದ್ಧಾಂತ-ಗ-ಳಿಗೆ
ಸಿದ್ಧಾಂತ-ಗಳ-ಲ್ಲಿ
ಸಿದ್ಧಾಂತ-ಗಳಿವೆ
ಸಿದ್ಧಾಂತ-ಗಳೊಂದಿಗೆ
ಸಿದ್ಧಾಂತ-ದಂತೆ
ಸಿದ್ಧಾಂತ-ಪಕ್ಷ-ವನ್ನು
ಸಿದ್ಧಾಂತ-ವನ್ನು
ಸಿದ್ಧಾಂತ-ವನ್ನೂ
ಸಿದ್ಧಾಂತ-ವಾ-ದರೊ
ಸಿದ್ಧಾಂತ-ವಾದಿ
ಸಿದ್ಧಾಂತಕ್ಕೆ
ಸಿದ್ಧಾಂತದ
ಸಿದ್ಧಾಂತವೇ
ಸಿದ್ಧಾಂತಿ-ಗಳು
ಸಿದ್ಧಾರ್ಥ
ಸಿದ್ಧಿ-ಗಳು
ಸಿದ್ಧಿ-ಗಳೂ
ಸಿದ್ಧಿ-ಯನ್ನು
ಸಿದ್ಧಿ-ಸಲಿ
ಸಿದ್ಧಿ-ಸು-ತ್ತಿದೆ
ಸಿದ್ಧಿ-ಸು-ವುದು
ಸಿದ್ಧಿ-ಸು-ವುವು
ಸಿನಿ-ಯರ್
ಸಿನ್ಹ
ಸಿಪಾಯಿ-ಗ-ಳನ್ನು
ಸಿಪಾಯಿ-ಗಳಿಬ್ಬರೂ
ಸಿಪಾಯಿ-ಗಳು
ಸಿಪಾಯಿದಂಗೆ-ಯ-ಲ್ಲಿ
ಸಿಯಾ-ಲ್ಕೋಟೆಗೆ
ಸಿಯೊ-ಲ್ಡಾ
ಸಿರ್ದಾರ್
ಸಿಲಬೆ-ಸ್
ಸಿಲು-ಕದ
ಸಿಲುಕಿ-ದ್ದಾರೆ
ಸಿಲೋ-ನ್
ಸಿವಿ-ಲ್
ಸಿಸಿಲಿ-ಯನ್ನು
ಸಿಹಿ
ಸೀ
ಸೀಟಿ-ನ-ಲ್ಲಿ
ಸೀತಾ
ಸೀತಾ-ಪತಿ
ಸೀತೆ-ಯನ್ನು
ಸೀನಿ-ಸ್
ಸೀನು-ವುದು
ಸೀಮೆ-ಯ-ಲ್ಲಿ
ಸೀಮೆ-ಯನ್ನು
ಸೀಮೆಗೆ
ಸೀಳಿ
ಸೀಳಿ-ಬ-ರು-ವನು
ಸೀಸೆ-ಮ್
ಸು
ಸುಂ
ಸುಂಕ
ಸುಂಕ-ವನ್ನು
ಸುಂಟರ
ಸುಂಟರ-ಗಾಳಿ
ಸುಂಟರ-ಗಾಳಿ-ಯ-ಲ್ಲಿಯೇ
ಸುಕ್ಕಿ-ಹೋಗಿ-ರ-ಲಿ-ಲ್ಲ
ಸುಖ
ಸುಖ-ಕೊಡಲು
ಸುಖ-ಕ್ಕಾಗಿ
ಸುಖ-ಗಳ-ಲ್ಲಿಯೇ
ಸುಖ-ದ-ಲ್ಲಿ
ಸುಖ-ದುಃಖ
ಸುಖ-ದುಃಖ-ಗ-ಳನ್ನು
ಸುಖ-ದುಃಖ-ದ-ಲ್ಲಿ
ಸುಖ-ದುಃಖ-ದಿಂದ
ಸುಖ-ದೆ-ಡೆಗೆ
ಸುಖ-ನಿದ್ರೆ-ಯ-ಲ್ಲಿ
ಸುಖ-ಭೋಗ-ಗಳ-ಲ್ಲಿ
ಸುಖ-ಭೋಗ-ಗಳೆ-ಲ್ಲ
ಸುಖ-ವನ್ನು
ಸುಖ-ವಾಗಿ
ಸುಖ-ವಾಗಿ-ರ-ಬೇಕೆಂದು
ಸುಖ-ವಾದ
ಸುಖ-ವೆಂಬುದಿ-ಲ್ಲ-ವೆಂದು
ಸುಖ-ವೇನಿದೆ
ಸುಖಕ್ಕೂ
ಸುಖಕ್ಕೆ
ಸುಖವೂ
ಸುಖಾಭಿಲಾಷಿತ-ರಾದ
ಸುಖೀ
ಸುಖೇಚ್ಛೆ-ಯನ್ನು
ಸುಗ-ವಾಗಿ
ಸುಗಮ-ವಾಗು-ವುದು
ಸುಡ-ತ-ದಿಂದ
ಸುಡು-ತ್ತಾರೆ-ಇ-ತ್ಯಾದಿ
ಸುಡು-ತ್ತಾರೆಯೇ
ಸುಡು-ತ್ತಿದ್ದುದು
ಸುಡು-ತ್ತಿರು-ವಾಗ
ಸುಡು-ವನು
ಸುಡು-ವು-ದಿ-ಲ್ಲ
ಸುತು-ಮು-ತ್ತ
ಸುದ್ದಿ
ಸುದ್ದಿ-ಕಾರ-ರೊ-ಡನೆ
ಸುದ್ದಿ-ಗ-ಳನ್ನು
ಸುದ್ದಿ-ಗಳು
ಸುದ್ದಿ-ಗಳೆ-ಲ್ಲ
ಸುದ್ದಿ-ಯನ್ನು
ಸುದ್ದಿಯೆ
ಸುಧಾ-ಮನ
ಸುಧಾ-ರಿಸಿ-ಕೊಂಡು
ಸುಧಾಮಪುರಿ
ಸುಧಾರ-ಕರ
ಸುಧಾರ-ಕರು
ಸುಧಾರ-ಕರೆ
ಸುಧಾರಕ
ಸುಧಾರಕ-ರೆ-ಲ್ಲ
ಸುಧಾರಕ-ರೆಂ-ದರೆ
ಸುಪ್ತ-ಪ್ರಾಣವು
ಸುಪ್ತ-ವಾ-ಗಿದೆ
ಸುಪ್ತ-ವಾಗಿ-ರುವ
ಸುಪ್ತ-ವಾಗಿ-ರುವುದು
ಸುಪ್ತ-ವಾಗಿದ್ದ
ಸುಪ್ತ-ವಾಗಿವೆ
ಸುಪ್ತ-ವಾದ
ಸುಪ್ತ-ಶಕ್ತಿ-ಗ-ಳನ್ನು
ಸುಪ್ತ-ಸ್ಥಿತಿ-ಯ-ಲ್ಲಿವೆ
ಸುಪ್ತಾವ-ಸ್ಥೆಯ
ಸುಪ್ಪ-ತ್ತಿಗೆ-ಯ-ಲ್ಲಿ
ಸುಬೋಧ್ಸುಬೋಧಾ-ನಂದ
ಸುಬ್ಬ-ರಾವ್
ಸುಬ್ರಮಣ್ಯ
ಸುಭಿಕ್ಷ-ವನ್ನು
ಸುಮಾ-ರಾಗಿ
ಸುರ-ತರಿಂಗಿನಿ
ಸುರಂಗ-ವನ್ನು
ಸುರಕ್ಷಿ-ತ-ವಾಗಿ
ಸುರಾ-ಪಾನಾ-ದಿ-ಗ-ಳನ್ನು
ಸುರಿ-ಮಳೆ
ಸುರಿ-ಮಳೆ-ಯನ್ನು
ಸುರಿ-ಯಿತು
ಸುರಿ-ಯು-ತ್ತಿ-ತ್ತು
ಸುರಿ-ಯು-ತ್ತಿದ್ದರೂ
ಸುರಿ-ಸು-ತ್ತ
ಸುರಿ-ಸು-ತ್ತಾರೆ
ಸುರಿ-ಸು-ತ್ತಿರು-ವೆನು
ಸುರಿ-ಸು-ವುದು
ಸುರಿ-ಸುವ
ಸುರಿದು
ಸುರಿಸ-ತೊಡಗಿದರು
ಸುರೇಂದ್ರ
ಸುರೇಂದ್ರ-ನಾಥ
ಸುರೇಂದ್ರ-ನಾಥ-ಮಿ-ತ್ರ
ಸುರೇಂದ್ರ-ನಾಥ-ಮಿ-ತ್ರ-ನೆಂಬ
ಸುರೇಂದ್ರ-ನಿಗೆ
ಸುರೇಶ-ಚಂದ್ರ-ಮಿ-ತ್ರ
ಸುರೇಶ-ಬಾಬು
ಸುರೇಶ್ವರಾ-ನಂದ
ಸುರೇಶ್ವರಾ-ನಂದ-ರಿಗೆ
ಸುಲಭ
ಸುಲಭ-ವ-ಲ್ಲ
ಸುಲಭ-ವಗಿ
ಸುಲಭ-ವಾಗಿ
ಸುಲಭ-ವಾಗಿ-ದ್ದು-ದ-ರಿಂದ-ಲೇನೋ
ಸುಲಭ-ವಾದ
ಸುಲಭ-ವೆಂದು
ಸುಲಭವೂ
ಸುಲಭವೇ
ಸುಳಿ-ಗಳಿಂದ
ಸುಳಿ-ಯ-ಲ್ಲಿದ್ದರು
ಸುಳಿ-ಯು-ತ್ತಿ-ರಲಿ-ಲ್ಲ
ಸುಳಿ-ಯು-ತ್ತಿದ್ದವು
ಸುಳಿ-ಯು-ವಂತೆ
ಸುಳಿ-ವಿ-ಲ್ಲದೇ
ಸುಳಿಯ-ದಿ-ದ್ದ-ಲ್ಲಿ
ಸುಳಿಯಲಾ-ರವು
ಸುಳಿವು
ಸುಳಿವೂ
ಸುಳು
ಸುಳ್ಳ-ಲ್ಲ
ಸುಳ್ಳನ್ನು
ಸುಳ್ಳು
ಸುಳ್ಳು-ಗಾರ
ಸುಳ್ಳೆಂದೂ
ಸುಶಿಕ್ಷಿತ
ಸುಶಿಕ್ಷಿತ-ರಾಗಿ-ದ್ದಾರೆ
ಸುಶಿಕ್ಷಿತ-ರಾದ
ಸುಶು-ಮ್ನಃ
ಸುಶ್ರಾವ್ಯ-ವಾದ
ಸುಷು-ಮ್ನಾ-ವನ್ನು
ಸುಸಜ್ಜಿತ-ರಾಗಿ-ರುವರು
ಸುಸಜ್ಜಿತ-ವಾದ
ಸೂ
ಸೂಕ್ತ-ವಾ-ಗಿದೆ
ಸೂಕ್ತ-ವಾಗಿ
ಸೂಕ್ತ-ವಾದ
ಸೂಕ್ಷ್ಮ
ಸೂಕ್ಷ್ಮ-ವಲಯ-ವನ್ನು
ಸೂಕ್ಷ್ಮ-ವಾಗಿ
ಸೂಕ್ಷ್ಮ-ವಾಗಿ-ತ್ತು
ಸೂಕ್ಷ್ಮ-ವಾಗಿ-ರು-ವು-ದ-ರಿಂದ
ಸೂಕ್ಷ್ಮ-ವಾದ
ಸೂಕ್ಷ್ಮಾವ-ಸ್ಥೆಗೆ
ಸೂಚಿ-ಸು-ತ್ತವೆ
ಸೂಚಿ-ಸು-ತ್ತಿ-ತ್ತು
ಸೂಚಿ-ಸು-ತ್ತಿ-ದ್ದರು
ಸೂಚಿಸ-ಬಹುದೇ
ಸೂಜಿ
ಸೂಜಿ-ಯನ್ನು
ಸೂಫಿ
ಸೂಫಿ-ಗಳ
ಸೂಯೇಜ್
ಸೂರ-ದಾಸನ
ಸೂರಜ್
ಸೂರೆ-ಗೊಂಡಿ-ರುವರು
ಸೂರೆ-ಗೊಂಡಿವೆ
ಸೂರೆ-ಮಾಡು-ವುದು
ಸೂರೆಗೊ-ಳ್ಳುವ
ಸೂರ್ಯ
ಸೂರ್ಯ-ಗ್ರಹ-ಣದ
ಸೂರ್ಯ-ಚಂದ್ರರು
ಸೂರ್ಯ-ನ-ಲ್ಲಿರು-ವೆವು
ಸೂರ್ಯ-ನಂತೆ
ಸೂರ್ಯ-ನನ್ನು
ಸೂರ್ಯ-ನಿಗೂ
ಸೂರ್ಯ-ನಿಗೆ
ಸೂರ್ಯ-ನೆ-ದು-ರಿಗೆ
ಸೂರ್ಯ-ನೊ-ಡನೆ
ಸೂರ್ಯ-ಮಂಡಲ-ವನ್ನು
ಸೂರ್ಯ-ರಶ್ಮಿಶ್ಚ
ಸೂರ್ಯನ
ಸೂರ್ಯನು
ಸೂರ್ಯನೇ
ಸೂರ್ಯಾ-ಸ್ತ-ಮ-ಯದ-ಸ-ಮಯ-ದ-ಲ್ಲಿ
ಸೂರ್ಯೋದಯಕ್ಕೆ
ಸೃಜಿ-ಸಿತು
ಸೃಜಿ-ಸು-ವುದು
ಸೃಜಿಸು-ವುದೋ
ಸೃಷ್ಟಿ
ಸೃಷ್ಟಿ-ಕರ್ತ
ಸೃಷ್ಟಿ-ಕರ್ತ-ನನ್ನು
ಸೃಷ್ಟಿ-ಮಾಡಿ-ದರು
ಸೃಷ್ಟಿ-ಮಾಡು-ತ್ತಾನೆ
ಸೃಷ್ಟಿ-ಯ-ನ್ನೆ-ಲ್ಲ
ಸೃಷ್ಟಿ-ಯ-ಲ್ಲಿ
ಸೃಷ್ಟಿ-ಯ-ಲ್ಲಿ-ರುವ
ಸೃಷ್ಟಿ-ಯಷ್ಟೇ
ಸೃಷ್ಟಿ-ಯಾಗಿ-ರು-ವು-ದೆ-ಲ್ಲ
ಸೃಷ್ಟಿ-ಸ-ಬ-ಲ್ಲರು
ಸೃಷ್ಟಿ-ಸ-ಬೇಕಾಗಿದೆ
ಸೃಷ್ಟಿ-ಸ-ಬೇಕಾಗುವುದು
ಸೃಷ್ಟಿ-ಸ-ಲ್ಪಟ್ಟುವು
ಸೃಷ್ಟಿ-ಸಲಾರಿರಿ
ಸೃಷ್ಟಿ-ಸಿ-ರುವನು
ಸೃಷ್ಟಿ-ಸಿ-ರುವರು
ಸೃಷ್ಟಿ-ಸಿ-ರುವಳು
ಸೃಷ್ಟಿ-ಸಿ-ಲ್ಲ
ಸೃಷ್ಟಿ-ಸಿದ
ಸೃಷ್ಟಿ-ಸಿದ-ವನು
ಸೃಷ್ಟಿ-ಸಿದೆ
ಸೃಷ್ಟಿ-ಸು-ವು-ದಾಗಿದೆ
ಸೃಷ್ಟಿ-ಸುವ
ಸೃಷ್ಟಿ-ಸ್ಥಿತಿ-ಗಳ-ಲ್ಲಿ
ಸೃಷ್ಟಿಗೆ
ಸೃಷ್ಟಿಯ
ಸೃಷ್ಟಿಯೇ
ಸೆ
ಸೆಂ
ಸೆಖೆ
ಸೆಪ್ಟೆಂಬರ್
ಸೆರೆ
ಸೆರೆ-ಮನೆ
ಸೆರೆ-ಮನೆ-ಯಿಂದ
ಸೆರೆ-ಮನೆಯ
ಸೆರೆ-ಯಿ-ಟ್ಟ
ಸೆರೆ-ಯೊಳಗೆ
ಸೆಲೂ-ನ್ನ-ಲ್ಲಿ
ಸೆಳೆ-ಯ-ದಿ-ರಲಿ
ಸೆಳೆದು
ಸೆಳೆದು-ಕೊ-ಳ್ಳುವ
ಸೇ
ಸೇಂ
ಸೇಟರ
ಸೇಡ-ನ್ನು
ಸೇಡು
ಸೇತು-ಪತಿ
ಸೇತು-ಪತಿ-ಗಳು
ಸೇತು-ಪತಿ-ಯನ್ನು
ಸೇತು-ವೆಯ
ಸೇತುವೆ
ಸೇತುವೆ-ಗಳು
ಸೇತುವೆ-ಯನ್ನು
ಸೇದ-ತೊಡಗಿದರು
ಸೇದದೆ
ಸೇದಲು
ಸೇದಿ
ಸೇದಿ-ದರೆ
ಸೇದಿದ
ಸೇದು
ಸೇದು-ತ್ತಾ
ಸೇದು-ತ್ತಿ-ದ್ದು-ದ-ನ್ನು
ಸೇದು-ತ್ತಿದ್ದ
ಸೇದು-ತ್ತಿದ್ದನು
ಸೇದು-ತ್ತಿದ್ದರು
ಸೇದು-ವಾಗ
ಸೇದು-ವು-ದಕ್ಕೆ
ಸೇದು-ವುದ-ರ-ಲ್ಲಿ-ದ್ದರು
ಸೇದು-ಹೋಗಿ-ಬಿಡು-ತ್ತಿ-ತ್ತು
ಸೇದುವ
ಸೇನಾ-ನಾಯಕ
ಸೇನಾ-ಶಕ್ತಿ-ಯಾಗಲಿ
ಸೇನಾನಿ
ಸೇರ-ಬ-ಲ್ಲೆ
ಸೇರ-ಬೇಕು
ಸೇರ-ಬೇಕೆಂಬ
ಸೇರಲು
ಸೇರಿ
ಸೇರಿ-ಕೊಂಡರು
ಸೇರಿ-ಕೊಂಡಳು
ಸೇರಿ-ಕೊಂಡಿತು
ಸೇರಿ-ಕೊಂಡು
ಸೇರಿ-ದ-ಮೇಲೆ
ಸೇರಿ-ದ-ವ-ನೆಂದು
ಸೇರಿ-ದ-ವ-ನೆಂಬುದು
ಸೇರಿ-ದ-ವ-ರ-ಲ್ಲ
ಸೇರಿ-ದ-ವ-ರ-ಲ್ಲ-ವೆಂದೂ
ಸೇರಿ-ದ-ವ-ರಿ-ದ್ದರು
ಸೇರಿ-ದ-ವ-ರೆಂದು
ಸೇರಿ-ದ-ವ-ರೆಂದೂ
ಸೇರಿ-ದ-ವನು
ಸೇರಿ-ದ-ವನೋ
ಸೇರಿ-ದ-ವರು
ಸೇರಿ-ದ-ವರೂ
ಸೇರಿ-ದ-ವರೇ
ಸೇರಿ-ದ-ವಳ-ಲ್ಲ
ಸೇರಿ-ದ-ವಳಾ-ದು-ದ-ರಿಂದ
ಸೇರಿ-ದ-ವಳು
ಸೇರಿ-ದನು
ಸೇರಿ-ದರು
ಸೇರಿ-ದರೆ
ಸೇರಿ-ದವು
ಸೇರಿ-ದಾಗ
ಸೇರಿ-ದುದು
ಸೇರಿ-ದ್ದ-ಲ್ಲ
ಸೇರಿ-ದ್ದರು
ಸೇರಿ-ದ್ದರೂ
ಸೇರಿ-ದ್ದರೊ
ಸೇರಿ-ದ್ದಾರೆ
ಸೇರಿ-ದ್ದೆಂದು
ಸೇರಿ-ರಲಿ
ಸೇರಿ-ರು-ವಿರಿ
ಸೇರಿ-ರು-ವೆನು
ಸೇರಿ-ಲ್ಲವೆ
ಸೇರಿ-ಸ-ಬಾರ-ದೆಂದೂ
ಸೇರಿ-ಸ-ಬೇಕೆಂ-ದಾಗ
ಸೇರಿ-ಸರು
ಸೇರಿ-ಸಿ-ಕೊಂಡು
ಸೇರಿ-ಸಿ-ದನು
ಸೇರಿ-ಸಿ-ದರು
ಸೇರಿ-ಸಿ-ದರೂ
ಸೇರಿ-ಸಿ-ದ್ದರು
ಸೇರಿ-ಸು-ತ್ತಿ-ದ್ದನು
ಸೇರಿ-ಸು-ವು-ದ-ನ್ನು
ಸೇರಿ-ಸು-ವುದು
ಸೇರಿ-ಸು-ವುದೇ
ಸೇರಿ-ಸುವ
ಸೇರಿತು
ಸೇರಿದ
ಸೇರಿವೆ
ಸೇರು-ತ್ತಾನೆ
ಸೇರು-ತ್ತಿವೆ
ಸೇವ-ಕರು
ಸೇವಕ
ಸೇವಕ-ನಂತಿರ-ಬೇಕು
ಸೇವಕ-ನಾಗ-ಬೇಕಾಗಿ
ಸೇವಕ-ಮಾ-ತ್ರ
ಸೇವಕ-ರ-ಲ್ಲ
ಸೇವಾ-ಕಾರ್ಯ
ಸೇವಾ-ಕಾರ್ಯ-ಗಳ-ನ್ನೂ
ಸೇವಾ-ಕಾರ್ಯ-ದ-ಲ್ಲಿ
ಸೇವಾ-ತ-ತ್ತ್ವ-ಗ-ಳನ್ನು
ಸೇವಾ-ಧರ್ಮ-ವನ್ನು
ಸೇವಾ-ಪರ-ನಾಗಿ
ಸೇವಾ-ಭಾವ-ದಿಂದ
ಸೇವಾ-ಭಾವ-ನೆ-ಯಿಂದ
ಸೇವಾ-ಶ್ರಮ
ಸೇವಾ-ಶ್ರಮ-ವನ್ನು
ಸೇವಾ-ಶ್ರಮದ
ಸೇವಾ-ಸ-ತ್ರ-ದ-ಲ್ಲಿ
ಸೇವಿ-ಯರ್
ಸೇವಿ-ಯರ್ಸ
ಸೇವಿ-ಯರ್ಸ್
ಸೇವಿ-ಸದೆ
ಸೇವಿ-ಸಲಾರಿರಿ
ಸೇವಿ-ಸಿದ
ಸೇವಿ-ಸು-ತ್ತ
ಸೇವಿ-ಸು-ತ್ತಾನೋ
ಸೇವಿ-ಸು-ತ್ತಿದ್ದ
ಸೇವಿ-ಸು-ತ್ತಿದ್ದಾರೆ
ಸೇವಿ-ಸು-ವು-ದ-ರಿಂದ
ಸೇವಿ-ಸು-ವು-ದಕ್ಕೆ
ಸೇವಿ-ಸು-ವುದು
ಸೇವಿ-ಸುವರು
ಸೇವಿಸ-ಬೇ-ಕಾ-ದರೆ
ಸೇವಿಸಿ
ಸೇವಿಸಿ-ದಂತೆ
ಸೇವಿಸಿ-ದರೂ
ಸೇವಿಸು-ವಿರಿ
ಸೇವೆ
ಸೇವೆ-ಗಾಗಿ
ಸೇವೆ-ಗೆಂದು
ಸೇವೆ-ಮಾ-ಡು-ತ್ತ
ಸೇವೆ-ಮಾಡಿ
ಸೇವೆ-ಮಾಡು
ಸೇವೆ-ಮಾಡುವ
ಸೇವೆ-ಯ-ಲ್ಲಿ
ಸೇವೆ-ಯನ್ನು
ಸೇವೆ-ಯನ್ನೂ
ಸೇವೆಗೆ
ಸೇವೆಯ
ಸೇವೆಯೇ
ಸೈ
ಸೈಂ
ಸೈತಾ-ನ್
ಸೈತಾನ-ನನ್ನು
ಸೈತಾನ-ನಿಂದ
ಸೊ
ಸೊಂಟ
ಸೊಂಟ-ಕ-ಟ್ಟಿ
ಸೊಂಟ-ದ-ಮೇಲೆ
ಸೊಂಟ-ದ-ಲ್ಲಿ
ಸೊಂಟದ
ಸೊಗ-ಸಾಗಿ
ಸೊಗಸಾದ
ಸೊಡರಿ-ನಂತೆ
ಸೊಪ್ಪು-ಸದೆ-ಗಳಿಂದ
ಸೊಪ್ಪು-ಹಾ-ಕದೆ
ಸೊಬಗ-ನ್ನು
ಸೊಳ್ಳೆ-ಗಳು
ಸೊಳ್ಳೆಗೆ
ಸೊಳ್ಳೆಯ
ಸೊಸೆ-ಯರು
ಸೊಸೈಟಿ
ಸೊಸೈಟಿ-ಗಳಿಂದ
ಸೊಸೈಟಿ-ಯ-ಲ್ಲಿದ್ದ
ಸೊಸೈಟಿ-ಯ-ಲ್ಲಿಯೂ
ಸೊಸೈಟಿ-ಯ-ವರು
ಸೊಸೈಟಿ-ಯನ್ನು
ಸೊಸೈಟಿಯ
ಸೋ
ಸೋಂಕಿ-ನಿಂದಲೇ
ಸೋಕಿ-ದರೆ-ರಡೂ
ಸೋಗು-ಹಾಕಲು
ಸೋಜಿಗ-ವಾ-ದರೂ
ಸೋಡಾ
ಸೋತ
ಸೋತ-ವನು
ಸೋತು
ಸೋತು-ಹೋಗಿ
ಸೋತು-ಹೋಗು-ವರು
ಸೋತೆವೊ
ಸೋಮ-ನಾಥ
ಸೋಮ-ನಾಥ-ದ-ಲ್ಲಿ
ಸೋಮ-ವಾರ
ಸೋಮಬಳ್ಳಿ
ಸೋಮಬಳ್ಳಿಯ
ಸೋರು-ತ್ತಿರ-ಬಹುದು
ಸೋರೇಬುರುಡೆ-ಯನ್ನು
ಸೋಲಿ-ಸಿ-ದರೂ
ಸೋಲಿ-ಸಿದೆ
ಸೋಲಿ-ಸು-ವಂತೆ
ಸೋಲಿಸಿ-ಬಿಡು-ತ್ತಿದ್ದರು
ಸೋಲು
ಸೋಲು-ತ್ತಿದ್ದರೂ
ಸೋಲು-ವ-ವ-ನ-ಲ್ಲ
ಸೋಲು-ವೆನು
ಸೋಲುವ
ಸೋಽಹಂ
ಸೌ
ಸೌಂ
ಸೌಖ್ಯ-ಗ-ಳನ್ನು
ಸೌಖ್ಯ-ವನ್ನು
ಸೌದೆ
ಸೌಧ
ಸೌರ-ವ್ಯೂಹ-ದ-ಲ್ಲಿ
ಸೌಲಭ್ಯ-ಗಳಿಂದ
ಸೌಲಭ್ಯ-ಗಳು
ಸ್
ಸ್ಕಂಬ
ಸ್ಕಂಭ-ವಾ-ಯಿತು
ಸ್ಕಾಟಿಶ್
ಸ್ಕುಟಾರಿ
ಸ್ಕೂ-ಲ-ನ್ನು
ಸ್ಕೂ-ಲಿಗೆ
ಸ್ಕೂ-ಲಿನ
ಸ್ಕೂ-ಲ್
ಸ್ಕೂಲಿ-ನ-ಲ್ಲಿ
ಸ್ಕೂಲು
ಸ್ಕ್ರೀ-ನ್
ಸ್ಕ್ವೇರಿ-ನ-ಲ್ಲಿ
ಸ್ಕ್ವೇರ್
ಸ್ಖಲನಂ
ಸ್ಟರ್ಗಿ-ಸ್
ಸ್ಟರ್ಡಿ
ಸ್ಟರ್ಡಿ-ಯ-ವರ
ಸ್ಟಾರ್
ಸ್ಟೀ
ಸ್ಟೀ-ಮರ್
ಸ್ಟೀ-ಮರ್-ನ-ಲ್ಲಿ
ಸ್ಟೀಮ-ರ-ನ್ನು
ಸ್ಟೇ-ಶನ್
ಸ್ಟೇ-ಶನ್ನಿ-ನ-ಲ್ಲಿ
ಸ್ಟೇ-ಶನ್ನಿನ-ಲ್ಲೆ
ಸ್ಟೇಶ-ನಿನ್ನ-ಲ್ಲಿ
ಸ್ಟೇಷ-ನ್
ಸ್ಟೇಷ-ನ್ಗೆ
ಸ್ಟೇಷ-ನ್ನಿ-ನ-ಲ್ಲಿ
ಸ್ಟೇಷ-ನ್ನಿ-ನಿಂದ
ಸ್ಟೇಷ-ನ್ನಿಗೂ
ಸ್ಟೇಷ-ನ್ನಿಗೆ
ಸ್ಟೇಷ-ನ್ನಿನ
ಸ್ಟೇಷ-ನ್ನೇ
ಸ್ಟೇಷ-ನ್-ಮಾ-ಸ್ಟರ್
ಸ್ಟೇಷ-ನ್ಮಾ-ಸ್ಟ-ರಿಗೆ
ಸ್ಟ್ರೀ-ಟ್
ಸ್ಟ್ರೈ-ಟ್ಸ್
ಸ್ತಂಬದ
ಸ್ತಂಭ-ವನ್ನು
ಸ್ತಂಭಿ-ತ-ನಾದ
ಸ್ತಂಭೀ-ಭೂತ-ನಾದ
ಸ್ತಂಭೀ-ಭೂತ-ರಾ-ದೆವು
ಸ್ತಬ್ಧ-ರಾಗಿ
ಸ್ತಿ
ಸ್ತು-ತಿಯ
ಸ್ತುತಿ
ಸ್ತುತಿ-ನಿಂದೆ-ಗಳ-ಲ್ಲಿ
ಸ್ತುತಿ-ಪಾಠ-ಕರೇ
ಸ್ತುತಿ-ಸು-ತ್ತಿ-ರುವ
ಸ್ತುತಿ-ಸು-ವುದು
ಸ್ತೂಪ-ಗ-ಳನ್ನು
ಸ್ತೂಪ-ಗಳ
ಸ್ತೂಪ-ವನ್ನು
ಸ್ತೋ-ತ್ರ
ಸ್ತೋ-ತ್ರ-ಮಾಡು-ವರು
ಸ್ತೋ-ತ್ರ-ಮಾಡುವ
ಸ್ತೋ-ತ್ರ-ವನ್ನು
ಸ್ತೋ-ತ್ರ-ವಾಗು-ವುದು
ಸ್ತ್ರೀ
ಸ್ತ್ರೀಯ
ಸ್ತ್ರೀಯೂ
ಸ್ಥ-ನ-ದ-ಲ್ಲಿ
ಸ್ಥಂಭೀ-ಭೂತ-ರಾಗಿ
ಸ್ಥಳ
ಸ್ಥಳ-ಗ-ಳನ್ನು
ಸ್ಥಳ-ಗ-ಳಿಗೆ
ಸ್ಥಳ-ಗಳ
ಸ್ಥಳ-ಗಳ-ನ್ನೆ-ಲ್ಲ
ಸ್ಥಳ-ಗಳ-ಲ್ಲಿ
ಸ್ಥಳ-ಗಳಿ-ಗೆ-ಲ್ಲ
ಸ್ಥಳ-ಗಳು
ಸ್ಥಳ-ದ-ಲ್ಲಿ
ಸ್ಥಳ-ದ-ಲ್ಲಿ-ಯಾ-ದರೂ
ಸ್ಥಳ-ದ-ಲ್ಲಿದ್ದ
ಸ್ಥಳ-ದ-ಲ್ಲಿಯೂ
ಸ್ಥಳ-ದ-ಲ್ಲಿಯೇ
ಸ್ಥಳ-ದಂತೆ
ಸ್ಥಳ-ದಿಂದ
ಸ್ಥಳ-ಪುರಾಣ
ಸ್ಥಳ-ವನ್ನು
ಸ್ಥಳ-ವನ್ನೆ-ಲ್ಲ
ಸ್ಥಳ-ವಾಗಿ-ತ್ತು
ಸ್ಥಳ-ವಾಗು-ವುದು
ಸ್ಥಳ-ವಾದ
ಸ್ಥಳ-ವಿ-ತ್ತೊ
ಸ್ಥಳ-ವಿ-ದೆಯೆ
ಸ್ಥಳ-ವಿ-ರ-ಲಿ-ಲ್ಲ
ಸ್ಥಳ-ವಿ-ರ-ಲಿ-ಲ್ಲವೊ
ಸ್ಥಳ-ವಿ-ಲ್ಲ
ಸ್ಥಳ-ವಿ-ಲ್ಲದೇ
ಸ್ಥಳ-ವಿದೆ
ಸ್ಥಳ-ವಿದೆ-ಯ-ಲ್ಲ
ಸ್ಥಳ-ವೊಂ-ದಿದ್ದರೆ
ಸ್ಥಳಕ್ಕೆ
ಸ್ಥಳದ
ಸ್ಥಳವು
ಸ್ಥಳವೂ
ಸ್ಥಳವೇ
ಸ್ಥಳೀಯ
ಸ್ಥಾ-ವರ
ಸ್ಥಾನ
ಸ್ಥಾನ-ದ-ಲ್ಲಿ
ಸ್ಥಾನ-ದ-ಲ್ಲಿದ್ದ
ಸ್ಥಾನ-ವನ್ನು
ಸ್ಥಾನ-ವಾಗು-ವುದು
ಸ್ಥಾನ-ವಿ-ತ್ತು
ಸ್ಥಾನ-ವಿದೆ
ಸ್ಥಾನ-ವೆಂದು
ಸ್ಥಾನಕ್ಕೆ
ಸ್ಥಾನದ
ಸ್ಥಾಪಕ
ಸ್ಥಾಪಕಾಯ
ಸ್ಥಾಪನೆ
ಸ್ಥಾಪನೆ-ಮಾಡಿ
ಸ್ಥಾಪಿ-ತ-ವಾ-ಯಿತು
ಸ್ಥಾಪಿ-ಸಲು
ಸ್ಥಾಪಿ-ಸಿತು
ಸ್ಥಾಪಿ-ಸಿದ
ಸ್ಥಾಪಿ-ಸು-ತ್ತಿ-ದ್ದರು
ಸ್ಥಾಪಿ-ಸು-ವು-ದ-ಕ್ಕಾಗಿ
ಸ್ಥಾಪಿ-ಸು-ವು-ದಕ್ಕೆ
ಸ್ಥಾಪಿ-ಸು-ವುದು
ಸ್ಥಾಪಿ-ಸು-ವೆವು
ಸ್ಥಾಪಿ-ಸುವ
ಸ್ಥಾಪಿತ-ವಾ-ಗಿದೆ
ಸ್ಥಾಪಿತ-ವಾ-ದುದು
ಸ್ಥಾಪಿಸ-ಬೇಕೆಂದು
ಸ್ಥಾಪಿಸ-ಬೇಕೆಂದೂ
ಸ್ಥಾಪಿಸ-ಲ್ಪ-ಡು-ವುದು
ಸ್ಥಾಪಿಸ-ಲ್ಪಟ್ಟ
ಸ್ಥಾಪಿಸ-ಲ್ಪಟ್ಟವು
ಸ್ಥಾಪಿಸ-ಲ್ಪಟ್ಟಿ-ರು-ವುದು
ಸ್ಥಾಪಿಸತ-ವಾದ
ಸ್ಥಾಪಿಸಿ
ಸ್ಥಾಪಿಸಿ-ದರು
ಸ್ಥಾಪಿಸಿ-ದರೆ
ಸ್ಥಾಪಿಸಿ-ದು-ದಕ್ಕೆ
ಸ್ಥಾಪಿಸಿ-ದ್ದರು
ಸ್ಥಾಯಿ-ಯಾಗಿ
ಸ್ಥಿ-ತ-ರಾ-ದರು
ಸ್ಥಿತಃ
ಸ್ಥಿತಿ
ಸ್ಥಿತಿ-ಗಳು
ಸ್ಥಿತಿ-ಯ-ಲ್ಲಿ
ಸ್ಥಿತಿ-ಯ-ಲ್ಲಿ-ತ್ತು
ಸ್ಥಿತಿ-ಯ-ಲ್ಲಿ-ದ್ದಾದ
ಸ್ಥಿತಿ-ಯ-ಲ್ಲಿ-ದ್ದಿದ್ದರೆ
ಸ್ಥಿತಿ-ಯ-ಲ್ಲಿ-ದ್ದೆವು
ಸ್ಥಿತಿ-ಯ-ಲ್ಲಿ-ರ-ಲಿ-ಲ್ಲ
ಸ್ಥಿತಿ-ಯ-ಲ್ಲಿ-ರುವ
ಸ್ಥಿತಿ-ಯ-ಲ್ಲಿದೆ
ಸ್ಥಿತಿ-ಯ-ಲ್ಲಿದ್ದ
ಸ್ಥಿತಿ-ಯ-ಲ್ಲಿಯೇ
ಸ್ಥಿತಿ-ಯ-ಲ್ಲೇ
ಸ್ಥಿತಿ-ಯನ್ನು
ಸ್ಥಿತಿ-ಯನ್ನೇ
ಸ್ಥಿತಿ-ಯಿಂದ
ಸ್ಥಿತಿ-ಯೊಂದಿಗೆ
ಸ್ಥಿತಿಗೂ
ಸ್ಥಿತಿಗೆ
ಸ್ಥಿತಿಯ
ಸ್ಥಿತಿಯೇ
ಸ್ಥಿರ
ಸ್ಥಿರ-ಪಡಿ-ಸಲು
ಸ್ಥಿರ-ಪಡಿ-ಸಿ-ಕೊಳ್ಳ-ಬೇಕೆಂಬ
ಸ್ಥಿರ-ಬುದ್ಧಿ-ಯಿಂದ
ಸ್ಥಿರ-ವ-ಲ್ಲ
ಸ್ಥಿರ-ವಾ-ದರು
ಸ್ಥಿರ-ವಾ-ದುದು
ಸ್ಥಿರ-ವಾಗಿ
ಸ್ಥಿರ-ವಾಗಿ-ದ್ದರು
ಸ್ಥಿರ-ವಾಗು-ತ್ತಿ-ತ್ತು
ಸ್ಥಿರ-ವಾದ
ಸ್ಥೂಲ
ಸ್ಥೂಲ-ದೇಹ-ವನ್ನು
ಸ್ಥೂಲ-ದೇಹಧಾರಿ-ಗ-ಳಿಗೆ
ಸ್ಥೂಲ-ವಾಗಿ-ರುವ
ಸ್ಥೂಲ-ವಾಗು-ವುದು
ಸ್ಥೂಲದ
ಸ್ಥೈರ್ಯ
ಸ್ಥ್ಯ
ಸ್ನಾನ
ಸ್ನಾನ-ಕ್ಕಾಗಿ
ಸ್ನಾನ-ಘ-ಟ್ಟಕ್ಕೆ
ಸ್ನಾನ-ಮಾಡಿ
ಸ್ನಾನ-ಮಾಡಿ-ದರೂ
ಸ್ನಾನ-ವೆ-ಲ್ಲ
ಸ್ನಾನಕ್ಕೆ
ಸ್ನಾನಾ-ದಿ-ಗ-ಳನ್ನು
ಸ್ನಾಯುಜಾ-ಲಕ್ಕೂ
ಸ್ನಿಗ್ಧ
ಸ್ನಿಗ್ಧ-ಕಾಂ-ತಿ-ಯ-ಲ್ಲಿ
ಸ್ನಿಗ್ಧೋಜ್ವಲ
ಸ್ನೆ
ಸ್ನೇಹ
ಸ್ನೇಹಾಶೀರ್ವಾದ-ಗ-ಳನ್ನು
ಸ್ನೇಹಿ-ತ-ನಂತೆಯೇ
ಸ್ನೇಹಿ-ತ-ನನ್ನು
ಸ್ನೇಹಿ-ತ-ನನ್ನೋ
ಸ್ನೇಹಿ-ತ-ನಾದ
ಸ್ನೇಹಿ-ತ-ನಿ-ದ್ದನು
ಸ್ನೇಹಿ-ತ-ನಿಗೆ
ಸ್ನೇಹಿ-ತ-ನೊಂದಿಗೆ
ಸ್ನೇಹಿ-ತ-ನೊಬ್ಬ
ಸ್ನೇಹಿ-ತ-ನೊಬ್ಬ-ನನ್ನು
ಸ್ನೇಹಿ-ತ-ನೊಬ್ಬನು
ಸ್ನೇಹಿ-ತ-ರ-ನ್ನು
ಸ್ನೇಹಿ-ತ-ರ-ನ್ನೆ-ಲ್ಲ
ಸ್ನೇಹಿ-ತ-ರ-ಲ್ಲಿ
ಸ್ನೇಹಿ-ತ-ರಾ-ದರು
ಸ್ನೇಹಿ-ತ-ರಾ-ದರೋ
ಸ್ನೇಹಿ-ತ-ರಾಗಿ
ಸ್ನೇಹಿ-ತ-ರಾಗಿ-ದ್ದರು
ಸ್ನೇಹಿ-ತ-ರಾಗು-ವು-ದಕ್ಕೆ
ಸ್ನೇಹಿ-ತ-ರಾಗುವರು
ಸ್ನೇಹಿ-ತ-ರಾದ
ಸ್ನೇಹಿ-ತ-ರಿ-ಗೆ-ಲ್ಲ
ಸ್ನೇಹಿ-ತ-ರಿ-ದ್ದರು
ಸ್ನೇಹಿ-ತ-ರಿ-ಲ್ಲ
ಸ್ನೇಹಿ-ತ-ರಿಂದ
ಸ್ನೇಹಿ-ತ-ರಿಗೂ
ಸ್ನೇಹಿ-ತ-ರಿಗೆ
ಸ್ನೇಹಿ-ತ-ರು-ಗಳು
ಸ್ನೇಹಿ-ತ-ರೆ-ಲ್ಲ
ಸ್ನೇಹಿ-ತ-ರೊ-ಡನೆ
ಸ್ನೇಹಿ-ತ-ರೊಂದಿಗೆ
ಸ್ನೇಹಿ-ತ-ರೊಬ್ಬರು
ಸ್ನೇಹಿ-ತನ
ಸ್ನೇಹಿ-ತನೆ
ಸ್ನೇಹಿ-ತನೇ
ಸ್ನೇಹಿ-ತರ
ಸ್ನೇಹಿ-ತರು
ಸ್ನೇಹಿ-ತರೂ
ಸ್ನೇಹಿ-ತೆಯ
ಸ್ನೇಹಿತ
ಸ್ನೇಹಿತೆ
ಸ್ಪಂ
ಸ್ಪಂದನವೂ
ಸ್ಪಂದಿ-ಸು-ತ್ತಿ-ದೆಯೆ
ಸ್ಪಂದಿ-ಸು-ತ್ತಿದೆ
ಸ್ಪಂದಿ-ಸು-ತ್ತಿದ್ದ
ಸ್ಪಂದಿತ
ಸ್ಪರ್ಧಿ-ಯಾದ
ಸ್ಪರ್ಧಿ-ಸು-ತ್ತಿ-ತ್ತು
ಸ್ಪರ್ಧಿ-ಸು-ತ್ತಿ-ದ್ದರು
ಸ್ಪರ್ಧಿಸ-ಬೇ-ಕಾ-ದರೆ
ಸ್ಪರ್ಧೆ
ಸ್ಪರ್ಧೆಯೇ
ಸ್ಪರ್ಶ
ಸ್ಪರ್ಶ-ದಿಂದ
ಸ್ಪರ್ಶ-ಮಣಿ
ಸ್ಪರ್ಶ-ಮಣಿ-ಯಂತೆ
ಸ್ಪರ್ಶ-ಮಾ-ತ್ರ-ದಿಂದ
ಸ್ಪರ್ಶ-ಮಾ-ತ್ರ-ದಿಂದಲೇ
ಸ್ಪರ್ಶ-ಮಾ-ತ್ರವೆ
ಸ್ಪರ್ಶ-ಮಾಡಿ
ಸ್ಪರ್ಶದ
ಸ್ಪರ್ಶನ
ಸ್ಪರ್ಶಿ-ಯಾಗುವಂತೆ
ಸ್ಪರ್ಶಿ-ಸಿದ
ಸ್ಪರ್ಶಿಸಿ-ದಾಗ
ಸ್ಪರ್ಶಿಸಿ-ರು-ವೆನು
ಸ್ಪಷ್ಟ
ಸ್ಪಷ್ಟ-ಪಡಿ-ಸು-ವು-ದಾಗಿದೆ
ಸ್ಪಷ್ಟ-ವಾ-ಯಿತು
ಸ್ಪಷ್ಟ-ವಾಗ-ತೊಡಗಿತು
ಸ್ಪಷ್ಟ-ವಾಗಿ
ಸ್ಪಷ್ಟ-ವಾಗಿ-ರ-ಬೇಕು
ಸ್ಪಷ್ಟ-ವಾಗಿ-ರು-ವಷ್ಟು
ಸ್ಪಷ್ಟ-ವಾಗು-ವುದು
ಸ್ಪಷ್ಟ-ವಾದ
ಸ್ಪಿನಿ-ಸ್
ಸ್ಪೂರ್ತಿ-ಯನ್ನು
ಸ್ಪೆ-ನ್ಸರ್
ಸ್ಪೆಯಿ-ನ್
ಸ್ಪೆಷ-ಲ್
ಸ್ಪೇನಿಷ್
ಸ್ಫು-ರದ್ರೂಪಿಯು
ಸ್ಫುಟಗೊ-ಳ್ಳುವುವು
ಸ್ಫುರಿಸು-ವುವು
ಸ್ಫೂರ್ತಿ
ಸ್ಫೂರ್ತಿ-ಗೊಂಡ
ಸ್ಫೂರ್ತಿ-ಗೊಂಡಂತೆ
ಸ್ಫೂರ್ತಿ-ಗೊಂಡು
ಸ್ಫೂರ್ತಿ-ದಾಯಕ
ಸ್ಫೂರ್ತಿ-ದಾಯಕ-ವಾದ
ಸ್ಫೂರ್ತಿ-ಯನ್ನು
ಸ್ಫೂರ್ತಿ-ಯಾಗಿ
ಸ್ಫೂರ್ತಿ-ಯಿಂದ
ಸ್ಫೂರ್ತಿ-ಯುತ-ವಾದ
ಸ್ಫೂರ್ತಿ-ವಾಣಿ
ಸ್ಫೂರ್ತಿಯ
ಸ್ಮ-ರಿಸಿ
ಸ್ಮರಣೆ
ಸ್ಮರಿ-ಸಿ-ಕೊಂಡರು
ಸ್ಮರಿ-ಸು-ತ್ತ
ಸ್ಮರಿಸು
ಸ್ಮಶಾ-ನಕ್ಕೆ
ಸ್ಮಶಾನ-ಗಳೆಂದು
ಸ್ಮಶಾನ-ಸದೃಶ-ವಾಗ-ಬೇಕು
ಸ್ಮಶಾನಘ-ಟ್ಟ-ದ-ಲ್ಲಿ
ಸ್ಮಾರಕ
ಸ್ಮಾರಕ-ಗಳು
ಸ್ಮಾರಕ-ವನ್ನು
ಸ್ಮಾರಕ-ವಾಗಿ
ಸ್ಮಾರಕ-ಸ್ತಂಭ-ವನ್ನು
ಸ್ಮಿ-ತ್
ಸ್ಮೃತಿ
ಸ್ಮೃತಿ-ಗಳಿಂದ
ಸ್ಮೃತಿ-ಗಳು
ಸ್ಮೃತಿ-ಚಿ-ತ್ರ-ಗಳು
ಸ್ಮೃತಿ-ಪಥ-ದ-ಲ್ಲಿ
ಸ್ಮೃತಿ-ಯ-ಲ್ಲಿ
ಸ್ಮೃತಿ-ಯ-ಲ್ಲಿಯೂ
ಸ್ಮೃತಿ-ಯಿಂದ
ಸ್ಮೃತಿ-ಸ್ತಂಭ-ವನ್ನು
ಸ್ಯಾನ್ಫ್ರಾ-ನ್ಸಿ-ಸ್ಕೋಗೆ
ಸ್ರೀ-ಯರು
ಸ್ರೋತ
ಸ್ರ್ತೀ
ಸ್ಲೇ-ಟರ್
ಸ್ವ
ಸ್ವ-ಜ-ನರ
ಸ್ವ-ಜನ
ಸ್ವ-ತಂ-ತ್ರ
ಸ್ವ-ತಂ-ತ್ರ-ರಾಗ-ಬೇಕೆ
ಸ್ವ-ತಂ-ತ್ರ-ವಾಗ-ಲಾ-ರದು
ಸ್ವ-ತಂ-ತ್ರ-ವಾಗಿ
ಸ್ವ-ತಂ-ತ್ರ-ವಾಗಿ-ದ್ದರೆ
ಸ್ವ-ತಂ-ತ್ರ-ವಾಗು-ವುದು
ಸ್ವ-ತಂ-ತ್ರ-ವಾದ
ಸ್ವ-ತಂ-ತ್ರರು
ಸ್ವ-ಮತಾಭಿ-ಮಾನ
ಸ್ವ-ರವೂ
ಸ್ವ-ರೂಪ-ನಾದ
ಸ್ವ-ರೂಪ-ವೇನು
ಸ್ವ-ರೂಪರೂ
ಸ್ವ-ರೂಪಿ
ಸ್ವ-ರೂಪಿ-ಗಾಗಿ
ಸ್ವ-ಲ್ಪ
ಸ್ವ-ಲ್ಪ-ಕಾಲ
ಸ್ವ-ಲ್ಪ-ಕಾಲ-ದ-ಲ್ಲಿಯೇ
ಸ್ವ-ಲ್ಪ-ಕಾಲ-ವಾದ
ಸ್ವ-ಲ್ಪ-ಕಾಲದ
ಸ್ವ-ಲ್ಪ-ದ-ರ-ಲ್ಲೇ
ಸ್ವ-ಲ್ಪ-ದೂರ
ಸ್ವ-ಲ್ಪ-ನೋಡು
ಸ್ವ-ಲ್ಪ-ಭಾಗ-ವನ್ನು
ಸ್ವ-ಲ್ಪ-ಮ-ಟ್ಟಿಗೆ
ಸ್ವ-ಲ್ಪ-ವನ್ನು
ಸ್ವ-ಲ್ಪ-ವಾ-ದರೂ
ಸ್ವ-ಲ್ಪ-ವಾಗಿ
ಸ್ವ-ಲ್ಪ-ಸ್ವ-ಲ್ಪ
ಸ್ವ-ಲ್ಪ-ಹೊ-ತ್ತಿಗೆ
ಸ್ವ-ಲ್ಪ-ಹೊ-ತ್ತಿನ
ಸ್ವ-ಲ್ಪ-ಹೊ-ತ್ತು
ಸ್ವ-ಲ್ಪವೂ
ಸ್ವ-ಸ್ತಿ
ಸ್ವಂ-ತಂ-ತ್ರ-ವಾದ
ಸ್ವಂತ
ಸ್ವಂತ-ವಾಗಿ
ಸ್ವಚ್ಛ-ವಾದ
ಸ್ವಚ್ಛಂದ
ಸ್ವಚ್ಛಂದ-ವಾಗಿ-ರ-ಬೇಕು
ಸ್ವತಃ
ಸ್ವಧಾ
ಸ್ವಪ್ನ-ದಂತಿದೆ
ಸ್ವಪ್ನ-ಲೋಕದ
ಸ್ವಪ್ನ-ಲೋಕವೇ
ಸ್ವಪ್ನದ
ಸ್ವಪ್ನವೋ
ಸ್ವಪ್ನಾವ-ಸ್ಥೆ-ಯ-ಲ್ಲಿ
ಸ್ವಮೀಜಿ
ಸ್ವಮೀಜಿ-ಯ-ವ-ರ-ನ್ನು
ಸ್ವಮೀಜಿ-ಯ-ವ-ರೊ-ಡನೆ
ಸ್ವಮೀಜಿ-ಯ-ವರ
ಸ್ವಮೀಜಿ-ಯ-ವರೂ
ಸ್ವಯಂ
ಸ್ವಯಂ-ಪ್ರಕಾಶ
ಸ್ವಯಂ-ಪ್ರಕಾಶ-ಮಾನ-ವಾಗಿ
ಸ್ವರ
ಸ್ವರ-ಗಳ-ಲ್ಲಿ
ಸ್ವರ-ವಿದೆ
ಸ್ವರೂ-ಪ-ರಾದ
ಸ್ವರೂ-ಪಾ-ನಂ-ದರು
ಸ್ವರೂ-ಪಾನಂದ-ರ-ನ್ನು
ಸ್ವರ್ಗ
ಸ್ವರ್ಗ-ದ-ಲ್ಲಿ-ದ್ದರೂ
ಸ್ವರ್ಗ-ದ-ಲ್ಲಿ-ರುವ
ಸ್ವರ್ಗ-ವಿದೆ
ಸ್ವರ್ಗ-ವೆ-ನ್ನು-ವ-ಲ್ಲಿ
ಸ್ವರ್ಗ-ಸ್ಥರಾ-ದುದು
ಸ್ವರ್ಗದ
ಸ್ವರ್ಗಾದಿ
ಸ್ವರ್ಣ-ಸ್ವಪ್ನ
ಸ್ವರ್ಣ-ಸ್ವಪ್ನ-ದಂತೆ
ಸ್ವಷ್ಟ-ವಾಗಿ
ಸ್ವಷ್ಟ-ವಾದ
ಸ್ವಾ
ಸ್ವಾಮಿ
ಸ್ವಾರ್ಥ
ಸ್ವಿ-ಕರಿ-ಸಿ-ದರು
ಸ್ವಿ-ಕರಿ-ಸು-ವು-ದಿ-ಲ್ಲ
ಸ್ವಿ-ಟ್ಜರ್ಲೆಂಡಿ-ನ-ಲ್ಲಿ
ಸ್ವಿ-ಟ್ಜರ್ಲೆಂಡಿಗೆ
ಸ್ವಿ-ಟ್ಸರ್ಲೆಂಡ್
ಸ್ವೀ-ಕರಿ-ಸ-ಬೇಕಾ-ಯಿತು
ಸ್ವೀ-ಕರಿ-ಸದೆ
ಸ್ವೀ-ಕರಿ-ಸದೇ
ಸ್ವೀ-ಕರಿ-ಸಲಿ
ಸ್ವೀ-ಕರಿ-ಸಲೇ-ಬೇಕು
ಸ್ವೀ-ಕರಿ-ಸು-ತ್ತಾರೆ
ಸ್ವೀ-ಕರಿ-ಸು-ತ್ತಿ-ದ್ದಳು
ಸ್ವೀ-ಕರಿ-ಸು-ತ್ತಿ-ರ-ಲಿ-ಲ್ಲ
ಸ್ವೀ-ಕರಿ-ಸು-ವಂತೆ
ಸ್ವೀ-ಕರಿ-ಸು-ವುದು
ಸ್ವೀ-ಕರಿ-ಸುವ
ಸ್ವೀ-ಕರಿ-ಸುವ-ವನ
ಸ್ವೀ-ಕರಿ-ಸುವನು
ಸ್ವೀ-ಕರಿ-ಸುವರು
ಸ್ವೀ-ಕರಿ-ಸುವರೆ
ಸ್ವೀ-ಕರಿ-ಸುವೆಯೊ
ಸ್ವೀ-ಕರಿ-ಸೋಣ
ಸ್ವೀ-ಕರಿಸ-ಬ-ಲ್ಲಂತಹ
ಸ್ವೀ-ಕರಿಸ-ಬೇಕು
ಸ್ವೀ-ಕರಿಸ-ಬೇಕೆಂದು
ಸ್ವೀ-ಕರಿಸ-ಲ್ಪ-ಡು-ವುದು
ಸ್ವೀ-ಕರಿಸು-ವು-ದೆಂದು
ಸ್ವೀ-ಕಾರ
ಸ್ವೀಕ-ರಿ-ಸಲು
ಸ್ವೀಕ-ರಿ-ಸು-ವು-ದಕ್ಕೆ
ಸ್ವೀಕ-ರಿ-ಸು-ವೆನು
ಸ್ವೀಕ-ರಿಸಿ
ಸ್ವೀಕ-ರಿಸಿ-ಕೊಂಡು
ಸ್ವೀಕ-ರಿಸಿ-ದರು
ಸ್ವೀಕ-ರಿಸಿ-ದರೂ
ಸ್ವೀಕ-ರಿಸಿ-ದಳು
ಸ್ವೀಕ-ರಿಸಿ-ದ್ದರು
ಸ್ವೀಕ-ರಿಸಿದ
ಸ್ವೀಡ-ನ್ಬರ್ಗನ
ಸ್ವೇಚ್ಛಾ-ಜೀವನ
ಸ್ವೇಚ್ಛೆ-ಯಾಗಿ
ಸ್ವೇಚ್ಛೆ-ಯಿಂದ
ಸ್ವೇಚ್ಛೆ-ಯಿಂದಲೇ
ಸ್ವ್ಬಾಮೀ-ಜಿಗೆ
ಸ್ವ್ವಾಮೀಜಿ
ಹ
ಹಂ
ಹಂಗಿ-ಸು-ತ್ತಿದ್ದ
ಹಂಗಿ-ಸು-ವು-ದಕ್ಕೆ
ಹಂಗೇರಿ
ಹಂಚದೆ
ಹಂಚಲು
ಹಂಚಿ
ಹಂಚಿ-ಕೆಯ
ಹಂಚಿ-ಕೆಯ-ನ್ನೆ-ಲ್ಲ
ಹಂಚಿ-ಕೊಳ್ಳ-ಬೇಕೆಂದು
ಹಂಚಿ-ದರು
ಹಂಚಿ-ಬಿಡು-ತ್ತಿದ್ದರು
ಹಂಚಿ-ಲ್ಲ
ಹಂಚಿ-ಹೋಗಿ-ರು-ತ್ತದೆ
ಹಂಚು
ಹಂಚು-ವು-ದ-ನ್ನು
ಹಂತ-ದ-ಲ್ಲಿ
ಹಂಬಲ-ವಿದೆ
ಹಂಬಲಿ-ಸಿ-ದರು
ಹಂಬಲಿ-ಸಿದ್ದ
ಹಂಬಲಿ-ಸು-ತ್ತದೆ
ಹಂಬಲಿ-ಸು-ತ್ತಿ-ದ್ದೆನೊ
ಹಂಸ
ಹಂಸ-ಗತಿ
ಹಂಸ-ಗೀತೆ
ಹಂಸ-ತೂಲಿಕಾತ-ಲ್ಪ-ಗಳು
ಹಂಸರು
ಹಂಸವೂ
ಹಂಸಿ
ಹಕ್ಕನ್ನು
ಹಕ್ಕಿ
ಹಕ್ಕಿ-ಗ-ಳನ್ನು
ಹಕ್ಕಿ-ಗಳ
ಹಕ್ಕಿ-ಯಂತೆ
ಹಕ್ಕಿ-ಯನ್ನು
ಹಕ್ಕಿ-ಲ್ಲ
ಹಕ್ಕಿಗೆ
ಹಕ್ಕಿದೆ
ಹಕ್ಕಿಯ
ಹಕ್ಕಿಯೋ
ಹಕ್ಕು
ಹಕ್ಕು-ದಾ-ರರು
ಹಕ್ಕು-ದಾರ-ನಾಗಿ
ಹಕ್ಕು-ದಾರ-ರ-ನ್ನಾಗಿ
ಹಕ್ಕು-ದಾರಿಕೆ
ಹಗ-ಲೆ-ಲ್ಲ
ಹಗಲಿ-ನಷ್ಟೇ
ಹಗಲಿರುಳು
ಹಗಲು
ಹಗಲು-ರಾ-ತ್ರಿ
ಹಗಲೂ
ಹಗುರ-ವಾ-ಯಿತು
ಹಗ್ಗ
ಹಗ್ಗ-ದ-ಲ್ಲಿ
ಹಗ್ಗ-ದಿಂದಲೂ
ಹಗ್ಗ-ಮಾಡಿ-ದರೆ
ಹಗ್ಗ-ವನ್ನು
ಹಚ್ಚನೆ
ಹಚ್ಚಿ
ಹಚ್ಚಿ-ಕೊ-ಳ್ಳಲಿ-ಲ್ಲ
ಹಚ್ಚಿ-ಕೊಂಡರೆ
ಹಚ್ಚಿ-ಕೊಂಡಾಗ
ಹಚ್ಚಿ-ಕೊಂಡಿ-ರುವುದು
ಹಚ್ಚಿ-ಕೊಳ್ಳು-ತ್ತೀರಿ
ಹಚ್ಚಿ-ದರು
ಹಚ್ಚಿ-ಸಿದ್ದರು
ಹಚ್ಚಿ-ಸು-ವಾಗ
ಹಚ್ಚಿಸು
ಹಚ್ಚು-ವುದು
ಹಟ
ಹಟ-ತೊ-ಟ್ಟರು
ಹಟ-ಮಾಡಿ
ಹಟ-ಹಿಡಿ-ದರು
ಹಟ-ಹಿಡಿದ
ಹಟತೊ-ಟ್ಟಿ-ರುವರು
ಹಟಾ-ತ್ತಾಗಿ
ಹಠಾ-ತ್
ಹಡಗ-ನ್ನು
ಹಡಗಿ-ನ-ಲ್ಲಿ
ಹಡಗಿಗೆ
ಹಡಗಿನ
ಹಡಗಿನ-ಲ್ಲಿ-ರುವ
ಹಡಗಿನ-ಲ್ಲಿದ್ದ
ಹಡಗು
ಹಡಗು-ಗಳ
ಹಡುಗು-ಗ-ಳಿಗೆ
ಹಡ್ಸನ್
ಹಣ
ಹಣ-ಕ್ಕಾಗಿ
ಹಣ-ಕ್ಕೋ-ಸ್ಕರ
ಹಣ-ಗಳಿ-ಸು-ತ್ತಿ-ರು-ವನು
ಹಣ-ದ-ಲ್ಲಿ
ಹಣ-ದಾಸೆಗೆ
ಹಣ-ದಿಂದ
ಹಣ-ವಂತ-ರಾದ
ಹಣ-ವಂತ-ರಿಗೆ
ಹಣ-ವಂತರ
ಹಣ-ವನ್ನು
ಹಣ-ವನ್ನೆ-ಲ್ಲ
ಹಣ-ವನ್ನೆ-ಲ್ಲಾ
ಹಣ-ವಿದ್ದರೆ
ಹಣ-ಸಂಗ್ರಹ-ಕ್ಕಾಗಿ
ಹಣತೆ-ಗಳಂತೆ
ಹಣತೆ-ಯನ್ನು
ಹಣಾ-ಭಾವ-ದಿಂದ
ಹಣೆ
ಹಣೆ-ಯ-ಲ್ಲಿ
ಹಣೆ-ಯಂತೆ
ಹಣ್ಣ-ನ್ನು
ಹಣ್ಣಿ-ನಂತೆ
ಹಣ್ಣಿನ
ಹಣ್ಣು
ಹಣ್ಣು-ಗ-ಳನ್ನು
ಹಣ್ಣೆಂದು
ಹತಾಶ-ನಾಗಲಿ-ಲ್ಲ
ಹತಾಶ-ಳಾದಳು
ಹತಿ-ರವೇ
ಹದಗೆ-ಟ್ಟಿತು
ಹದಗೆಡಲು
ಹದಿ-ನಾ-ಲ್ಕನೇ
ಹದಿ-ನಾ-ಲ್ಕು
ಹದಿ-ಮೂ-ರನೆ
ಹನಿ
ಹನಿ-ಗ-ಳನ್ನು
ಹನಿ-ಯಂತೆ
ಹನುಮಂ-ತನ
ಹನುಮಂ-ತನೂ
ಹನುಮಂ-ತರ
ಹನುಮಂತ
ಹನುಮಂತ-ನಿಗೆ
ಹನುಮಂತ-ರಾ-ಯನ
ಹಬ್ಬ
ಹಬ್ಬ-ದಂತೆ
ಹಬ್ಬ-ವನ್ನಾ-ಚರಿ-ಸು-ವುದು
ಹಬ್ಬದ
ಹಬ್ಬಿ
ಹಬ್ಬಿ-ರು-ವುದು
ಹಬ್ಬಿತು
ಹಬ್ಬಿದೆ
ಹಬ್ಬು-ತ್ತಿದೆ
ಹಬ್ಬು-ತ್ತಿರುವ
ಹಯಾಸಿಂ-ತ್
ಹರ
ಹರ-ಡು-ತ್ತ
ಹರ-ಡು-ತ್ತಿದೆ
ಹರ-ಡು-ತ್ತಿದ್ದರು
ಹರ-ಡು-ವುದು
ಹರ-ಡು-ವುವು
ಹರ-ಡುವ
ಹರ-ಡುವುದ-ರ-ಲ್ಲಿ
ಹರಕೆ
ಹರಕೆ-ಯನ್ನು
ಹರಟು-ತ್ತಿದ್ದನು
ಹರಟೆ
ಹರಟೆ-ಯ-ಲ್ಲಿ
ಹರಡ-ಬೇಕು
ಹರಡಲು
ಹರಡಿ
ಹರಡಿ-ದರು
ಹರಡಿ-ದ್ದವು
ಹರಡಿ-ರುವ
ಹರಡಿತು
ಹರಡಿದ
ಹರಡಿದೆ
ಹರಮೋಹನ
ಹರಳು
ಹರಸ-ಬೇಕೆ-ನ್ನಿ-ಸು-ವುದು
ಹರಸಿ
ಹರಸಿ-ಕಳುಹಿ-ಸಿ-ದರು
ಹರಸಿ-ಕೊಂಡಿದ್ದರು
ಹರಸಿ-ದ್ದರು
ಹರಹರ
ಹರಿ
ಹರಿ-ಜ-ನರು
ಹರಿ-ತ-ವಾಗಿ-ದ್ದರೂ
ಹರಿ-ತುರಿ-ಯಾ-ನಂದ
ಹರಿ-ದಾಡು-ತ್ತಿದ್ದವು
ಹರಿ-ದಾಡು-ತ್ತಿರು-ವನು
ಹರಿ-ದಾಸ
ಹರಿ-ದಾಸ-ಬಾಬು-ಗಳ
ಹರಿ-ದಾಸಿ
ಹರಿ-ದಾಸ್
ಹರಿ-ದಾಸ್ಬಾಬು
ಹರಿ-ದಿ-ತ್ತು
ಹರಿ-ದು-ಕೊಂಡು
ಹರಿ-ದು-ಬಂದ
ಹರಿ-ದು-ಬಂದು
ಹರಿ-ದು-ಹೋಗ-ಬಹುದು
ಹರಿ-ದು-ಹೋಗಿ-ತ್ತು
ಹರಿ-ದು-ಹೋಗು-ವಂತೆ
ಹರಿ-ದ್ವಾ-ರಕ್ಕೆ
ಹರಿ-ದ್ವಾರ
ಹರಿ-ಪದ
ಹರಿ-ಪ್ರ-ಸನ್ನ-ವಿ-ಜ್ಞಾನಾ-ನಂದ
ಹರಿ-ಪ್ರಸಾದ
ಹರಿ-ಯ-ತೊಡಗಿತು
ಹರಿ-ಯಲಿ
ಹರಿ-ಯಿತು
ಹರಿ-ಯು-ತ್ತದೆ
ಹರಿ-ಯು-ತ್ತಿ-ತ್ತು
ಹರಿ-ಯು-ತ್ತಿ-ರು-ತ್ತದೆ
ಹರಿ-ಯು-ತ್ತಿದೆ
ಹರಿ-ಯು-ತ್ತಿದ್ದ
ಹರಿ-ಯು-ತ್ತಿದ್ದರೂ
ಹರಿ-ಯು-ತ್ತಿರುವ
ಹರಿ-ಯು-ವಂತೆ
ಹರಿ-ಯು-ವುದು
ಹರಿ-ಯುವ
ಹರಿ-ರಾಮ್
ಹರಿ-ಸ-ಲಿ-ಲ್ಲ
ಹರಿ-ಸಿ-ದರು
ಹರಿ-ಸಿಂಗ್
ಹರಿ-ಸು-ವು-ದಕ್ಕೆ
ಹರಿಃ
ಹರಿದ
ಹರಿದು
ಹರಿಯೆ
ಹರಿವ
ಹರ್ಷ-ದಿಂದ
ಹರ್ಷ-ಪ್ರದ
ಹರ್ಷೋ-ತ್ಕರ್ಷ-ದಿಂದ
ಹರ್ಷೋದ್ಗಾರದ
ಹಲ-ವ-ರ-ನ್ನು
ಹಲ-ವ-ರಿಂದ
ಹಲ-ವ-ರಿಗೆ
ಹಲ-ವ-ರೊ-ಡನೆ
ಹಲ-ವರ
ಹಲ-ವರು
ಹಲ-ವಾರು
ಹಲ-ಸಿನ
ಹಲವ-ರ-ಲ್ಲಿ
ಹಲಹ-ಸ್ತೆ-ಯಾಗಿ
ಹಳದಿ
ಹಳದಿಯ
ಹಳಿ-ದಿದ್ದ
ಹಳಿ-ಯು-ತ್ತಿದ್ದರು
ಹಳಿ-ಯು-ವು-ದಕ್ಕೆ
ಹಳಿ-ಯು-ವುದ-ಕ್ಕೋ-ಸ್ಕರವೇ
ಹಳುಹಿ-ಸಿ-ದರು
ಹಳೆ-ಯದು
ಹಳೆಯ
ಹಳೆಯ-ದ-ರ-ಲ್ಲಿ
ಹಳೆಯ-ದರಷ್ಟೇ
ಹಳೆಯ-ದರೊ-ಳಗೆ
ಹಳ್ಳಕ್ಕೂ
ಹಳ್ಳಿ-ಗ-ಳಿಗೆ
ಹಳ್ಳಿ-ಗಳ
ಹಳ್ಳಿ-ಗಳು
ಹಳ್ಳಿ-ಯ-ಲ್ಲಿ
ಹಳ್ಳಿ-ಯ-ಲ್ಲೇ
ಹಳ್ಳಿ-ಯನ್ನು
ಹಳ್ಳಿ-ಯಾದ
ಹಳ್ಳಿ-ಯಿಂದ
ಹಳ್ಳಿಗೆ
ಹಳ್ಳಿಯ
ಹಳ್ಳಿಯ-ಲ್ಲಿ-ದ್ದಾಗ
ಹಳ್ಳಿಯ-ಲ್ಲಿದೆ
ಹಳ್ಳಿಯ-ಲ್ಲಿದ್ದ
ಹಳ್ಳಿಯ-ಲ್ಲಿಯೂ
ಹವಣಿ-ಸು-ತ್ತ
ಹವಣಿ-ಸು-ತ್ತಿ-ರುವರು
ಹವಾ
ಹವಾ-ಗುಣ
ಹವಾ-ಗುಣದ
ಹಸಹ್ರಾರು
ಹಸಿ-ದಿ-ತ್ತು
ಹಸಿ-ಯು-ತ್ತಿ-ತ್ತು
ಹಸಿ-ವನ್ನು
ಹಸಿ-ವಾ-ಗು-ತ್ತಿದೆ
ಹಸಿ-ವಿ-ತ್ತು
ಹಸಿ-ವಿ-ನಿಂದ
ಹಸಿ-ವಿನ
ಹಸಿ-ವೇನು
ಹಸಿರಿ-ನೊಂದಿಗೆ
ಹಸಿರು
ಹಸಿರು-ಮಯ
ಹಸಿವು
ಹಸಿವೇ
ಹಸು
ಹಸು-ಕರು-ಗಳು
ಹಸು-ಗಳ
ಹಸು-ರಾಗಿ-ದ್ದುವು
ಹಸು-ರಾಗಿ-ರುವ
ಹಸು-ರಿನ
ಹಸು-ಳೆ-ಗಳು
ಹಸು-ವಿನ
ಹಾ
ಹಾಂ
ಹಾಕ-ಕೂಡ-ದೆಂದೂ
ಹಾಕ-ತೊಡಗಿದರು
ಹಾಕ-ಬಹುದೇ
ಹಾಕ-ಬಾ-ರದು
ಹಾಕ-ಬೇಕು
ಹಾಕ-ಬೇಕೆಂದು
ಹಾಕ-ಬೇಡಿ
ಹಾಕಲು
ಹಾಕಿ
ಹಾಕಿ-ಕೊ-ಳ್ಳು-ವು-ದಕ್ಕೆ
ಹಾಕಿ-ಕೊ-ಳ್ಳು-ವು-ದಿ-ಲ್ಲ
ಹಾಕಿ-ಕೊಂಡಿ-ತ್ತು
ಹಾಕಿ-ಕೊಂಡಿದ್ದ
ಹಾಕಿ-ಕೊಂಡಿದ್ದರು
ಹಾಕಿ-ಕೊಂಡಿದ್ದರೂ
ಹಾಕಿ-ಕೊಂಡು
ಹಾಕಿ-ಕೊಳ್ಳ-ಕೂ-ಡದು
ಹಾಕಿ-ಕೊಳ್ಳಿ
ಹಾಕಿ-ಟ್ಟಿದ್ದ
ಹಾಕಿ-ದ-ವಳು
ಹಾಕಿ-ದಂತೆ
ಹಾಕಿ-ದರು
ಹಾಕಿ-ದರೆ
ಹಾಕಿ-ದಳು
ಹಾಕಿ-ದಾಗ
ಹಾಕಿ-ದೊಡ-ನೆಯೇ
ಹಾಕಿ-ದ್ದರು
ಹಾಕಿ-ದ್ದರೂ
ಹಾಕಿ-ದ್ದು-ದ-ನ್ನು
ಹಾಕಿ-ಬಿಡು
ಹಾಕಿ-ಬಿಡು-ತ್ತೇನೆ
ಹಾಕಿ-ರ-ಲಿ-ಲ್ಲ
ಹಾಕಿ-ಸ-ಬೇಕು
ಹಾಕಿ-ಸಿ-ಕೊಂಡು
ಹಾಕಿ-ಸಿ-ದರು
ಹಾಕಿ-ಸು-ತ್ತೇನೆ
ಹಾಕಿದ
ಹಾಕಿದೆ
ಹಾಕಿದ್ದ
ಹಾಕಿದ್ದು
ಹಾಕು-ತ್ತಾನೆ
ಹಾಕು-ತ್ತಾರೆ
ಹಾಕು-ತ್ತಾರೆಯೇ
ಹಾಕು-ತ್ತಾರೆಯೊ
ಹಾಕು-ತ್ತಿ-ತ್ತು
ಹಾಕು-ತ್ತಿ-ರು-ವಿರಿ
ಹಾಕು-ತ್ತಿದ್ದ
ಹಾಕು-ತ್ತಿದ್ದನು
ಹಾಕು-ತ್ತಿದ್ದರು
ಹಾಕು-ತ್ತೇನೆ
ಹಾಕು-ವು-ದಕ್ಕೆ
ಹಾಕುವ
ಹಾಕುವ-ರೆಂದು
ಹಾಕುವ-ವ-ನ-ಲ್ಲ
ಹಾಕುವನು
ಹಾಕುವರು
ಹಾಕುವುದು
ಹಾಕುವುದೇ
ಹಾಗ-ಲ್ಲ
ಹಾಗ-ಲ್ಲದೆ
ಹಾಗ-ಲ್ಲದೇ
ಹಾಗಾ-ದರೆ
ಹಾಗಾಗು-ವು-ದಿ-ಲ್ಲ
ಹಾಗಿ-ದ್ದ-ಲ್ಲಿ
ಹಾಗಿ-ದ್ದರೂ
ಹಾಗಿ-ದ್ದರೆ
ಹಾಗಿ-ದ್ದರೇ
ಹಾಗಿ-ರಲಿ
ಹಾಗಿ-ರುವಾಗ
ಹಾಗಿ-ಲ್ಲ
ಹಾಗಿ-ಲ್ಲದೆ
ಹಾಗಿ-ಲ್ಲದೇ
ಹಾಗಿದ್ದಿರ-ಬಹುದು
ಹಾಗೂ
ಹಾಗೆ
ಹಾಗೆ-ಮಾಡು
ಹಾಗೆಯೆ
ಹಾಗೆಯೇ
ಹಾಗೇ
ಹಾಗೇ-ನಾ-ದರೂ
ಹಾಗೇನೇ
ಹಾಜ-ರಾಗಿ
ಹಾಜ-ರಿ-ದ್ದರು
ಹಾಡ-ತೊಡಗಿದರು
ಹಾಡ-ನ್ನು
ಹಾಡ-ಬ-ಲ್ಲ
ಹಾಡ-ಬೇಕೆಂಬ
ಹಾಡ-ಲಿ-ಲ್ಲ
ಹಾಡಲಾರೆ
ಹಾಡಲು
ಹಾಡಿ
ಹಾಡಿ-ದನು
ಹಾಡಿ-ದರು
ಹಾಡಿ-ದಳು
ಹಾಡಿ-ದಷ್ಟು
ಹಾಡಿ-ನ-ಲ್ಲಿರುವ
ಹಾಡಿ-ಬಿಡು-ತ್ತಿದ್ದರು
ಹಾಡಿ-ಯನ್ನು
ಹಾಡಿ-ರಲಿ-ಲ್ಲ-ವೆಂದು
ಹಾಡಿ-ಸಿ-ದರು
ಹಾಡಿತು
ಹಾಡಿದ
ಹಾಡಿದೆ
ಹಾಡಿನ
ಹಾಡು
ಹಾಡು-ಗ-ಳನ್ನು
ಹಾಡು-ಗ-ಳಿಗೆ
ಹಾಡು-ಗಳ-ನ್ನೆ-ಲ್ಲ
ಹಾಡು-ಗಳ-ಲ್ಲಿ
ಹಾಡು-ಗಳು
ಹಾಡು-ಗಾರಿಕೆಯೂ
ಹಾಡು-ತ್ತಲೇ
ಹಾಡು-ತ್ತಾರೆ
ಹಾಡು-ತ್ತಿದ್ದ
ಹಾಡು-ತ್ತಿದ್ದ-ವ-ನನ್ನು
ಹಾಡು-ತ್ತಿದ್ದರು
ಹಾಡು-ತ್ತಿದ್ದಳು
ಹಾಡು-ತ್ತಿದ್ದಾಗ
ಹಾಡು-ಮುಗಿದ
ಹಾಡು-ವ-ವನು
ಹಾಡು-ವ-ವರು
ಹಾಡು-ವಂತೆ
ಹಾಡು-ವಾಗ
ಹಾಡು-ವು-ದ-ನ್ನು
ಹಾಡು-ವು-ದಕ್ಕೆ
ಹಾಡು-ವುದ-ರ-ಲ್ಲಿ
ಹಾಡು-ವುದಕ್ಕೇ
ಹಾಡು-ವುದು
ಹಾಡೆ
ಹಾಡೊಂ-ದ-ನ್ನು
ಹಾತೊ-ರೆಯು-ತ್ತಾನೆ
ಹಾತೊರೆ-ಯು-ತ್ತಿದ್ದರು
ಹಾತೊರೆ-ಯು-ತ್ತಿರುವೆ
ಹಾತೊರೆ-ಯು-ವಂತೆ
ಹಾತೊರೆ-ಯುವರು
ಹಾದಿ
ಹಾದಿ-ಯ-ಲ್ಲಿ
ಹಾದಿ-ಯಿಂದ
ಹಾದಿಗೆ
ಹಾದಿಯ
ಹಾದಿಯೆ
ಹಾದಿಯೇ
ಹಾದು
ಹಾದು-ಹೋಗು-ವಾಗ
ಹಾಪ್ಸ್
ಹಾರ
ಹಾರ-ಗ-ಳನ್ನು
ಹಾರ-ಗಳು
ಹಾರ-ಗಳೇ
ಹಾರ-ವನ್ನು
ಹಾರಾಡ-ಲಾ-ರದು
ಹಾರಾಡು-ತ್ತಿದ್ದ
ಹಾರಾಡು-ತ್ತಿರು-ವೆವು
ಹಾರಿ
ಹಾರಿ-ಬಂದು
ಹಾರಿ-ಸಿ-ದರು
ಹಾರಿ-ಹೋ-ಯಿತು
ಹಾರಿ-ಹೋಗ-ಬೇಕೆಂದು
ಹಾರಿ-ಹೋಗು-ವಂತೆ
ಹಾರಿಯೇ
ಹಾರೈಕೆ-ಯೆ-ಲ್ಲಾ
ಹಾರ್ಡ್ಫರ್ಡ್
ಹಾರ್ಡ್ಮನ್
ಹಾರ್ಡ್ಮೆ-ನ್
ಹಾರ್ದಿಕ
ಹಾರ್ಬರ್
ಹಾರ್ವರ್ಡ್
ಹಾಲಿ-ನ-ಲ್ಲಿ
ಹಾಲಿ-ನ-ಲ್ಲೇ
ಹಾಲಿಗೂ
ಹಾಲು
ಹಾಲು-ಕೊಡು
ಹಾಳಾ-ಗಿದೆ
ಹಾಳಾ-ಯಿತು
ಹಾಳಾಗಗೊಡ-ಲಿ-ಲ್ಲ
ಹಾಳಾಗಿ
ಹಾಳಾಗಿ-ಹೋಗಿ-ರು-ವೆವು
ಹಾಳು
ಹಾಳು-ಮಡು-ವರು
ಹಾಳು-ಮಾಡಿ-ರುವರು
ಹಾಳು-ಮಾಡಿದೆ
ಹಾಳು-ಮಾಡು-ವರು
ಹಾಳೆಯ
ಹಾವ-ಭಾವ-ನೆ-ಗ-ಳನ್ನು
ಹಾವಳಿ-ಯ-ಲ್ಲಿ
ಹಾವು
ಹಾವು-ಗಳು
ಹಾವೇ
ಹಾಸ-ಗೆ-ಯಂತೆ
ಹಾಸಿದ
ಹಾಸಿದಂತಿದೆ
ಹಾಸುಹೊ-ಕ್ಕಾಗಿ
ಹಾಸುಹೊ-ಕ್ಕಾಗಿ-ರ-ಬೇಕೆಂದು
ಹಾಸುಹೊ-ಕ್ಕಾಗಿವೆ
ಹಾಸ್ಯ
ಹಾಸ್ಯ-ದಿಂದ
ಹಾಸ್ಯ-ಪ್ರಿಯರು
ಹಾಸ್ಯ-ಮಯ-ವಾಗಿ
ಹಾಸ್ಯ-ಮಾ-ಡು-ತ್ತ
ಹಾಸ್ಯ-ಮಾಡಿ
ಹಾಸ್ಯ-ಮಾಡಿ-ಕೊಂಡು
ಹಾಸ್ಯ-ಮಾಡು-ತ್ತಿ-ಲ್ಲ
ಹಾಸ್ಯ-ಮಾಡು-ವರು
ಹಾಸ್ಯ-ವನ್ನು
ಹಾಸ್ಯ-ವಾಗಿ
ಹಾಸ್ಯ-ವಾಗಿಯೇ
ಹಾಸ್ಯದ
ಹಾಸ್ಯಾ-ಸ್ಪದ
ಹಾಸ್ಯಾ-ಸ್ಪದ-ವಾಗಿ
ಹಾಸ್ಯಾ-ಸ್ಪದ-ವಾಗಿ-ತ್ತು
ಹಾಸ್ಯಾ-ಸ್ಪದ-ವಾಗಿ-ರು-ವುದು
ಹಾಸ್ಯಾ-ಸ್ಪದ-ವಾದ
ಹಾಹಾ-ಕಾರ-ವೇ-ಳು-ವುದು
ಹಿ
ಹಿಂ
ಹಿಂಗಲಾ-ಜಿಗೆ
ಹಿಂಗಲಾಜ್
ಹಿಂಗಿ
ಹಿಂಗಿ-ಸಲಾರವು
ಹಿಂಗಿ-ಸಿ-ಕೊ-ಳ್ಳಲಿ
ಹಿಂಜ-ರಿಯದೆ
ಹಿಂಡಿ
ಹಿಂಡಿ-ತೋರಿ-ದರು
ಹಿಂಡಿ-ದಂ-ತಾ-ಯಿತು
ಹಿಂಡು
ಹಿಂಡು-ತ್ತಿರು-ವಂತೆ
ಹಿಂಡು-ವೆನು
ಹಿಂತಿರುಗ-ದಿರಿ
ಹಿಂತಿರುಗ-ಬೇಕು
ಹಿಂತಿರುಗ-ಬೇಡಿ
ಹಿಂತಿರುಗಲು
ಹಿಂತಿರುಗು-ತ್ತಿದ್ದೆ
ಹಿಂದಿ
ಹಿಂದಿ-ಗಿಂತ
ಹಿಂದಿ-ಗಿಂತಲೂ
ಹಿಂದಿ-ನ-ಕಾಲದ
ಹಿಂದಿ-ನ-ದ-ನ್ನು
ಹಿಂದಿ-ನ-ದ-ನ್ನೂ
ಹಿಂದಿ-ನ-ದ-ನ್ನೆ-ಲ್ಲ
ಹಿಂದಿ-ನ-ದ-ನ್ನೇ
ಹಿಂದಿ-ನಂತೆ
ಹಿಂದಿ-ನಂತೆಯೇ
ಹಿಂದಿ-ನದು
ಹಿಂದಿ-ನದೂ
ಹಿಂದಿ-ನಷ್ಟೇ
ಹಿಂದಿ-ನಿಂದ
ಹಿಂದಿ-ನಿಂದಲೂ
ಹಿಂದಿ-ಭಾಷೆ
ಹಿಂದಿ-ಯ-ಲ್ಲಿ
ಹಿಂದಿ-ಯ-ಲ್ಲೇ
ಹಿಂದಿ-ರು-ಗಿದೆ
ಹಿಂದಿ-ರುಗಿ
ಹಿಂದಿ-ರುಗಿ-ದರು
ಹಿಂದಿ-ರುಗು-ತ್ತಿದ್ದರು
ಹಿಂದಿ-ರುಗು-ತ್ತಿದ್ದಾಗ
ಹಿಂದಿ-ರುಗು-ವಾಗ
ಹಿಂದಿ-ರುಗು-ವು-ದಿ-ಲ್ಲ
ಹಿಂದಿ-ರುವ
ಹಿಂದೀ
ಹಿಂದೀ-ಭಾಷೆ-ಯ-ಲ್ಲಿಯೇ
ಹಿಂದು
ಹಿಂದು-ಗ-ಳಿಗೆ
ಹಿಂದು-ಗ-ಳಿದ್ದರು
ಹಿಂದು-ಗಳು
ಹಿಂದು-ತ್ವ-ವಿದೆ
ಹಿಂದು-ಧರ್ಮದ
ಹಿಂದು-ಳಿ-ದ-ವರು
ಹಿಂದು-ವಾಗಿ
ಹಿಂದು-ವಿನ
ಹಿಂದೂ
ಹಿಂದೂ-ಗ-ಳನ್ನು
ಹಿಂದೂ-ಗ-ಳಾಗು-ವು-ದಕ್ಕೆ
ಹಿಂದೂ-ಗ-ಳಾದ
ಹಿಂದೂ-ಗ-ಳಾದರೋ
ಹಿಂದೂ-ಗ-ಳಿಗೆ
ಹಿಂದೂ-ಗಳ
ಹಿಂದೂ-ಗಳ-ಲ್ಲ
ಹಿಂದೂ-ಗಳ-ಲ್ಲ-ದ-ವ-ರಿಂದ
ಹಿಂದೂ-ಗಳ-ಲ್ಲ-ದ-ವರು
ಹಿಂದೂ-ಗಳ-ಲ್ಲಿ
ಹಿಂದೂ-ಗಳ-ಲ್ಲಿ-ರುವ
ಹಿಂದೂ-ಗಳ-ಲ್ಲೆ
ಹಿಂದೂ-ಗಳಿ-ಗೆ-ಲ್ಲ
ಹಿಂದೂ-ಗಳಿಂದ
ಹಿಂದೂ-ಗಳಿಗೂ
ಹಿಂದೂ-ಗಳು
ಹಿಂದೂ-ಗಳೂ
ಹಿಂದೂ-ಗಳೆ-ಲ್ಲರೂ
ಹಿಂದೂ-ತ-ತ್ವ-ಗ-ಳನ್ನು
ಹಿಂದೂ-ದೇಶ-ದ-ಲ್ಲಿ
ಹಿಂದೂ-ದೇಶ-ದ-ಲ್ಲಿದ್ದ
ಹಿಂದೂ-ದೇಶದ
ಹಿಂದೂ-ಧರ್ಮ
ಹಿಂದೂ-ಧರ್ಮ-ದ-ಲ್ಲಿ
ಹಿಂದೂ-ಧರ್ಮ-ದಷ್ಟು
ಹಿಂದೂ-ಧರ್ಮ-ವನ್ನು
ಹಿಂದೂ-ಧರ್ಮಕ್ಕೂ
ಹಿಂದೂ-ಧರ್ಮಕ್ಕೆ
ಹಿಂದೂ-ಧರ್ಮದ
ಹಿಂದೂ-ವಾಗಿ-ದ್ದರೂ
ಹಿಂದೂ-ವಿಗೆ
ಹಿಂದೂ-ಶಾ-ಸ್ತ್ರ-ಗ-ಳನ್ನು
ಹಿಂದೂ-ಶಾ-ಸ್ತ್ರ-ದ-ಲ್ಲಿ
ಹಿಂದೂ-ಸಂ-ಸ್ಕೃತಿ
ಹಿಂದೂ-ಸಾ-ಗರ
ಹಿಂದೂ-ಸ್ತಾನ-ದಷ್ಟೇ
ಹಿಂದೂ-ಸ್ತಾನಿ
ಹಿಂದೂ-ಸ್ಥಾನ-ದ-ಲ್ಲಿ
ಹಿಂದೂ-ಸ್ಥಾನ-ದ-ವ-ನೆಂದೂ
ಹಿಂದೂ-ಸ್ಥಾನ-ವನ್ನು
ಹಿಂದೂ-ಸ್ಥಾನಿ
ಹಿಂದೂ-ಹಿಂದೆ
ಹಿಂದೂವು
ಹಿಂದೆ-ಗೆ-ಯದೆ
ಹಿಂದೆ-ಗೆ-ಯದೇ
ಹಿಂದೆ-ನಿಂತು
ಹಿಂದೆಯೂ
ಹಿಂದೊ-ಮ್ಮೆ
ಹಿಂಬದಿಯ
ಹಿಂಸಾ
ಹಿಂಸಾ-ಜನ-ಕರೂ
ಹಿಂಸಿ-ಸಿ-ದರೆ
ಹಿಂಸಿ-ಸು-ವುದು
ಹಿಂಸಿಸಿ-ದಷ್ಟೂ
ಹಿಂಸೆ
ಹಿಂಸೆ-ಗ-ಳನ್ನು
ಹಿಂಸೆ-ಯನ್ನು
ಹಿಂಸೆ-ಯಿಂದ
ಹಿಂಸೆಗೆ
ಹಿಂಸೆಯ
ಹಿಗಿ-ನ್ಸ್
ಹಿಗಿ-ರುವಾಗ
ಹಿಗೆ
ಹಿಗ್ಗಿ
ಹಿಗ್ಗಿ-ದರೂ
ಹಿಗ್ಗು-ವರು
ಹಿಡಿ
ಹಿಡಿ-ದ-ರೆಂ-ದರೆ
ಹಿಡಿ-ದ-ವರು
ಹಿಡಿ-ದಂತೆ
ಹಿಡಿ-ದರು
ಹಿಡಿ-ದರೆ
ಹಿಡಿ-ದರೋ
ಹಿಡಿ-ದಿ-ರು-ವಿರಾ
ಹಿಡಿ-ದಿ-ರು-ವುದು
ಹಿಡಿ-ದಿ-ರು-ವುದೇನು
ಹಿಡಿ-ದಿದ್ದರು
ಹಿಡಿ-ದಿದ್ದರೂ
ಹಿಡಿ-ದಿದ್ದಳು
ಹಿಡಿ-ದಿರ-ಬೇಕು
ಹಿಡಿ-ದು-ಕೊ-ಳ್ಳು-ವು-ದಕ್ಕೆ
ಹಿಡಿ-ದು-ಕೊಂಡಳು
ಹಿಡಿ-ದು-ಕೊಂಡಿ-ರುವನು
ಹಿಡಿ-ದು-ಕೊಂಡಿದ್ದ-ವರು
ಹಿಡಿ-ದು-ಕೊಂಡಿದ್ದರು
ಹಿಡಿ-ದು-ಕೊಂಡು
ಹಿಡಿ-ದು-ಕೊಂಡೆ
ಹಿಡಿ-ದು-ಕೊಂಡೇ
ಹಿಡಿ-ದು-ದ-ನ್ನು
ಹಿಡಿ-ದು-ನಿಂ-ತರು
ಹಿಡಿ-ದೆನು
ಹಿಡಿ-ದೆಳೆದು
ಹಿಡಿ-ಯ-ಬಹುದು
ಹಿಡಿ-ಯ-ಬೇಕಾಗಿ-ದ್ದು-ದ-ರಿಂದ
ಹಿಡಿ-ಯ-ಬೇಕು
ಹಿಡಿ-ಯ-ಬೇಕೆಂದು
ಹಿಡಿ-ಯದಿರಲೆಂದು
ಹಿಡಿ-ಯಲು
ಹಿಡಿ-ಯಲೇ-ಬೇಕು
ಹಿಡಿ-ಯಿ-ತೆಂದೂ
ಹಿಡಿ-ಯಿತು
ಹಿಡಿ-ಯಿತೋ
ಹಿಡಿ-ಯಿರಿ
ಹಿಡಿ-ಯು-ತ್ತಿ-ತ್ತು
ಹಿಡಿ-ಯು-ವು-ದ-ನ್ನು
ಹಿಡಿ-ಯು-ವು-ದಕ್ಕೆ
ಹಿಡಿ-ಯು-ವುದು
ಹಿಡಿ-ಸ-ಬಹುದೋ
ಹಿಡಿ-ಸ-ಲಿ-ಲ್ಲ
ಹಿಡಿ-ಸಿತು
ಹಿಡಿ-ಸು-ತ್ತಿ-ರ-ಲಿ-ಲ್ಲ
ಹಿಡಿ-ಸು-ವಷ್ಟು
ಹಿಡಿ-ಸು-ವುದು
ಹಿಡಿದ
ಹಿಡಿದು
ಹಿತ
ಹಿತ-ಕರ-ವಾದ
ಹಿತ-ಕಾರಿ
ಹಿತ-ಕಾರಿ-ಯ-ಲ್ಲ
ಹಿತ-ಕಾರಿ-ಯ-ಲ್ಲವೋ
ಹಿತ-ಕಾರಿ-ಯಾದ
ಹಿತ-ದ-ಲ್ಲಿ
ಹಿತ-ವ-ಚನ-ವಿದು
ಹಿತ-ವನ್ನು
ಹಿತ-ವನ್ನೇ
ಹಿತದ
ಹಿತವೂ
ಹಿತಾರ್ಥ-ವಾಗಿ
ಹಿಮ
ಹಿಮ-ದಿಂದ
ಹಿಮ-ಪಂಕ್ತಿ-ಗ-ಳನ್ನು
ಹಿಮ-ಮಣಿ-ಗಳು
ಹಿಮ-ರಾಶಿ
ಹಿಮ-ರಾಶಿ-ಯಿಂದ
ಹಿಮ-ರಾಶಿಯು
ಹಿಮ-ಶೀತಲ
ಹಿಮದ
ಹಿಮಾಲ-ಯದ
ಹಿಮಾಲಯ
ಹಿಮಾಲಯ-ಗಳ
ಹಿಮಾಲಯ-ದ-ಲ್ಲಿ
ಹಿಮಾಲಯ-ದ-ಲ್ಲಿ-ರುವಾಗ-ಲಂತೂ
ಹಿಮಾಲಯ-ದ-ಲ್ಲಿಯೂ
ಹಿಮಾಲಯ-ದ-ಲ್ಲೆ-ಲ್ಲ
ಹಿಮಾಲಯ-ದ-ವ-ರೆಗೆ
ಹಿಮಾಲಯ-ದಂತೆ
ಹಿಮಾಲಯ-ದಿಂದ
ಹಿಮಾಲಯ-ವನ್ನು
ಹಿಮಾಲಯ-ವಾಹಿನಿ
ಹಿಮಾಲಯಕ್ಕೆ
ಹಿಮಾಲಯೋ
ಹಿಮಾವೃತ
ಹಿರಿ-ಯರು
ಹಿರಿದ
ಹಿರಿಮೆ-ಯನ್ನು
ಹಿಸುಕಲಿ
ಹಿಸುಕಿ-ಬಿ-ಡು-ತ್ತದೆ
ಹಿಸುಕು
ಹೀ
ಹೀಗಾಗು-ವುದು
ಹೀಗಿ-ದ್ದರೆ
ಹೀಗಿ-ದ್ದುವು
ಹೀಗಿ-ರುವಾಗ
ಹೀಗಿದ್ದ
ಹೀಗೂ
ಹೀಗೆ
ಹೀಗೆಂ-ದನು
ಹೀಗೆಂ-ದರು
ಹೀಗೆಂ-ದರೆ
ಹೀಗೆಂದೆ
ಹೀಗೇ
ಹೀನ
ಹೀನ-ನಾಗಿದ್ದೆ
ಹೀನ-ವಾಗಿ
ಹೀನ-ವಾಗಿ-ರು-ವುದು
ಹೀನ-ವಾದ
ಹೀನ-ಸ್ಥಿತಿ-ಯ-ಲ್ಲಿ
ಹೀನ-ಸ್ಥಿತಿ-ಯ-ಲ್ಲಿ-ರುವ
ಹೀನ-ಸ್ಥಿತಿಗೆ
ಹೀನ-ಸ್ಥಿತಿಯೊ
ಹೀನತೆ
ಹೀನಾಯ-ವಾದ
ಹೀಯಾಳಿ-ಸಿ-ದರೆ
ಹೀರ-ಬೇಕೆಂ-ದಿ-ರುವೆನು
ಹೀರಿ
ಹೀರಿ-ಕೊ-ಳ್ಳುವು-ದರಿಂದ
ಹೀರಿ-ಕೊ-ಳ್ಳುವುದು
ಹೀರಿ-ಕೊಂಡಿದ್ದರು
ಹೀರಿ-ಕೊಂಡು
ಹೀರಿ-ಕೊಳ್ಳ-ಬೇಕು
ಹೀರಿ-ದನು
ಹೀರಿ-ರು-ವೆವು
ಹೀರಿ-ರುವರು
ಹೀರಿತು
ಹೀರು-ತ್ತಿರ-ಬೇಕು
ಹೀರು-ತ್ತಿರು-ವರು
ಹು
ಹುಕ್ಕ-ದಿಂದ
ಹುಕ್ಕ-ವನ್ನು
ಹುಕ್ಕದ
ಹುಚ್ಚ-ನಂತೆ
ಹುಚ್ಚ-ನಾಗಿ-ರ-ಬೇಕೆಂದು
ಹುಚ್ಚ-ರಂತೆ
ಹುಚ್ಚ-ರಂತೆಯೇ
ಹುಚ್ಚ-ರಾಗಿ
ಹುಚ್ಚ-ರಾಗಿ-ದ್ದರೂ
ಹುಚ್ಚ-ರಾಗಿ-ದ್ದರೆ
ಹುಚ್ಚ-ರಾಗಿ-ಬಿ-ಡುವರು
ಹುಚ್ಚ-ರಾಗಿ-ರ-ಬೇಕು
ಹುಚ್ಚ-ರಾಗಿ-ರುವ-ರೇನೊ
ಹುಚ್ಚ-ರಾಗಿ-ಹೋಗು-ತ್ತೇವೋ
ಹುಚ್ಚ-ರಾಗುವರು
ಹುಚ್ಚರ
ಹುಚ್ಚರ-ನ್ನಾಗಿ
ಹುಚ್ಚರಿ-ರ-ಬಹುದು
ಹುಚ್ಚರಿ-ರ-ಬೇಕು
ಹುಚ್ಚರು
ಹುಚ್ಚರೇ
ಹುಚ್ಚಿಯಂತಾ-ದಳು
ಹುಚ್ಚು
ಹುಚ್ಚು-ತನ
ಹುಚ್ಚೆ-ಲ್ಲ
ಹುಟಿದ್ದು
ಹುಡಿ-ಯ-ಲ್ಲಿ
ಹುಡಿ-ಯಾಗಿ
ಹುಡು-ಕಾಡಿದ
ಹುಡು-ಕು-ವಂತೆ
ಹುಡುಕ-ಬೇ-ಕಾ-ದರೆ
ಹುಡುಕಿ
ಹುಡುಕಿ-ಕೊ-ಟ್ಟರು
ಹುಡುಕಿ-ಕೊಂಡು
ಹುಡುಕಿ-ದಂತೆ
ಹುಡುಕಿ-ದರೂ
ಹುಡುಕಿದೆ
ಹುಡುಕಿಸಿ
ಹುಡುಕು-ತ್ತಿದ್ದಾಗ
ಹುಡುಕು-ತ್ತಿದ್ದುದು
ಹುಡುಕು-ತ್ತಿರು-ವಂತಿದೆ-ಯ-ಲ್ಲ
ಹುಡುಕು-ತ್ತಿರು-ವನು
ಹುಡುಕು-ವು-ದ-ಕ್ಕಾಗಿ
ಹುಡುಕು-ವು-ದಕ್ಕೆ
ಹುಡುಕು-ವುದು
ಹುಡುಗ
ಹುಡುಗ-ನ-ಲ್ಲಿ
ಹುಡುಗ-ನಂತೆ
ಹುಡುಗ-ನನ್ನು
ಹುಡುಗ-ನಯ್ಯ
ಹುಡುಗ-ನಾಗಿ-ದ್ದಾಗ
ಹುಡುಗ-ನಾದ
ಹುಡುಗ-ನಿ-ಗಾಗಿ
ಹುಡುಗ-ನಿಗೆ
ಹುಡುಗ-ರ-ನ್ನು
ಹುಡುಗ-ರ-ನ್ನೆ-ಲ್ಲ
ಹುಡುಗ-ರ-ಲ್ಲಿ
ಹುಡುಗ-ರಾಗಿ-ದ್ದ-ರೆಂದೂ
ಹುಡುಗ-ರಾಗಿದ್ದ
ಹುಡುಗ-ರಿ-ಗೆ-ಲ್ಲ
ಹುಡುಗ-ರಿಂದ
ಹುಡುಗ-ರಿಂದಲೇ
ಹುಡುಗ-ರಿಗೂ
ಹುಡುಗ-ರಿಗೆ
ಹುಡುಗ-ರಿಗೇ
ಹುಡುಗ-ರಿರಾ
ಹುಡುಗ-ರೆ-ಲ್ಲ
ಹುಡುಗ-ರೆ-ಲ್ಲಾ
ಹುಡುಗ-ರೊ-ಡನೆ
ಹುಡುಗ-ರೊಂದಿಗೆ
ಹುಡುಗ-ರೊಂದಿಗೆ-ಲ್ಲ
ಹುಡುಗನ
ಹುಡುಗರ
ಹುಡುಗರು
ಹುಡುಗರೆ
ಹುಡುಗರೇ
ಹುಡುಗಾಟ-ದ-ಲ್ಲಿ
ಹುಡುಗಾಟವೆ
ಹುಡುಗಿ
ಹುಡುಗಿ-ಯ-ರ-ನ್ನು
ಹುಡುಗಿ-ಯ-ರಿಗೆ
ಹುಡುಗಿ-ಯನ್ನು
ಹುಡುಗಿ-ಯರು
ಹುಡುಗಿ-ಯೊ-ಡನೆ
ಹುಡುಗಿ-ಯೊಂದಿಗೆ
ಹುಣಸೆ-ಹಣ್ಣ-ನ್ನು
ಹುಣಸೇ-ಹಣ್ಣು
ಹುಣ್ಣಾಗು-ವಂತೆ
ಹುಣ್ಣಿಮೆ
ಹುಣ್ಣಿಮೆಯ
ಹುದು-ಗಿದೆ
ಹುದುಗಿ-ಕೊ-ಳ್ಳುವುದಕ್ಕಾ-ದರೂ
ಹುದುಗಿ-ರುವ
ಹುದುಗಿ-ರುವರೊ
ಹುದುಗಿಸಿ
ಹುದ್ದೆ-ಯ-ಲ್ಲೇ
ಹುದ್ದೆಗೆ
ಹುದ್ದೆಯ-ಲ್ಲಿ-ದ್ದ-ವರು
ಹುಬ್ಬಿನ-ಮೇಲೆ
ಹುರಿ-ದುಂಬಿ-ಸಲು
ಹುರಿ-ದುಂಬಿ-ಸಿ-ದನು
ಹುರಿ-ದುಂಬಿ-ಸು-ತ್ತಿ-ದ್ದನು
ಹುರಿ-ದುಂಬಿ-ಸುವ
ಹುರಿ-ದುಂಬಿಸಿ
ಹುಲಿ
ಹುಲಿ-ಗ-ಳಿಗೆ
ಹುಲಿ-ಯಾ-ದರೂ
ಹುಲಿ-ಯೊಂದು
ಹುಲಿ-ಸಿಂಹ-ಗ-ಳಿಗೆ
ಹುಲು-ಸಾಗಿ
ಹುಳಹುಪ್ಪಟೆ-ಗಳು
ಹುಳು
ಹುಳು-ಕು-ಗಳು
ಹುಳು-ಗಳು
ಹುಳು-ವಿಗೂ
ಹುಳು-ವಿಗೆ
ಹುಸಿಗನಸು
ಹೂ
ಹೂಜಿ
ಹೂಡಿ
ಹೂಡಿ-ದರು
ಹೂಡಿ-ದೆವು
ಹೂಡಿ-ರು-ವು-ದ-ನ್ನು
ಹೂಣರ
ಹೂತು-ಕೊಂಡಿತು
ಹೂವು
ಹೂವು-ಗಳ
ಹೂವು-ಗಳಿಂದ
ಹೂವೂ
ಹೂವೋ
ಹೃ
ಹೃದ-ಯದ
ಹೃದ-ಯವೂ
ಹೃದಯ
ಹೃದಯ-ಕ್ಕಾಗಿ
ಹೃದಯ-ಗಳು
ಹೃದಯ-ಗಹ್ವರ-ದ-ಲ್ಲಿ
ಹೃದಯ-ದ-ಲ್ಲಿ
ಹೃದಯ-ದ-ಲ್ಲಿ-ತ್ತು
ಹೃದಯ-ದ-ಲ್ಲಿ-ರುವ
ಹೃದಯ-ದ-ಲ್ಲಿದ್ದ
ಹೃದಯ-ದ-ಲ್ಲಿಯೂ
ಹೃದಯ-ದ-ವರು
ಹೃದಯ-ದಿಂದ
ಹೃದಯ-ದಿಂದಲೂ
ಹೃದಯ-ವನ್ನಾಳುವ
ಹೃದಯ-ವನ್ನು
ಹೃದಯ-ವನ್ನೆ-ಲ್ಲ
ಹೃದಯ-ವನ್ನೆ-ಲ್ಲಾ
ಹೃದಯ-ವನ್ನೇ
ಹೃದಯ-ವಿ-ರು-ವುದು
ಹೃದಯ-ಸ್ತಂಭ-ನ-ದಿಂದ
ಹೃದಯ-ಸ್ಪರ್ಶಿ-ಯಾದ
ಹೃದಯಂಗಮ-ವಾಗಿ
ಹೃದಯಕ್ಕೆ
ಹೃದಯಳು
ಹೃದಯಾಂ-ತರ-ದ-ಲ್ಲೆ
ಹೃದಯಾಂ-ತರಾಳ
ಹೃದಯಾಂ-ತರಾಳ-ದಿಂದ
ಹೃದಯಾಂ-ತರಾಳದ
ಹೃದಯಿ-ಗಳು
ಹೃದಯಿ-ಯಾ-ದರು
ಹೃದಯೇಶ್ವ-ರಿಯೇ
ಹೃಷಿಕೇಶದ
ಹೃಷೀಕೇ-ಶಕ್ಕೆ
ಹೃಷೀಕೇಶ
ಹೃಷೀಕೇಶ-ದ-ಲ್ಲಿ
ಹೃಷೀಕೇಶ-ದ-ಲ್ಲಿ-ದ್ದರು
ಹೃಷೀಕೇಶದ
ಹೆ
ಹೆಂಗ-ಸನ್ನು
ಹೆಂಗ-ಸರು
ಹೆಂಗ-ಸಾಗಿ
ಹೆಂಗ-ಸಿ-ನಿಂದ
ಹೆಂಗ-ಸಿಗೂ
ಹೆಂಗ-ಸಿನ
ಹೆಂಗಸ-ರ-ನ್ನು
ಹೆಂಗಸ-ರಾ-ದರೋ
ಹೆಂಗಸ-ರೊ-ಡನೆ
ಹೆಂಗಸರ
ಹೆಂಗಸರೂ
ಹೆಂಗಸರೇ
ಹೆಂಗಸಾ-ದರೋ
ಹೆಂಗಸು
ಹೆಂಗಸೊಬ್ಬಳು
ಹೆಂಡ-ತಿಗೆ
ಹೆಂಡ-ತಿಯ
ಹೆಂಡತಿ
ಹೆಂಡತಿ-ಯನ್ನು
ಹೆಂಡತಿ-ಯನ್ನೇ
ಹೆಂಡತಿ-ಯಾಗಿ
ಹೆಂಡಿರು
ಹೆಗ-ಲ-ನ್ನು
ಹೆಗ-ಲಿನ
ಹೆಗಲ
ಹೆಗಲ-ನ್ನೇರಿ
ಹೆಗಲ-ಮೇಲೆ
ಹೆಗಲು
ಹೆಗಿ-ತ್ತೊ
ಹೆಚ್
ಹೆಚ್ಆರ್
ಹೆಚ್ಚ-ತೊಡಗಿತು
ಹೆಚ್ಚನ್ನು
ಹೆಚ್ಚಾ-ದರು
ಹೆಚ್ಚಾ-ದಾಗ
ಹೆಚ್ಚಾಗಲು
ಹೆಚ್ಚಾಗಿ
ಹೆಚ್ಚಾಗಿ-ತ್ತು
ಹೆಚ್ಚಾಗಿ-ದ್ದರೆ
ಹೆಚ್ಚಾಗಿ-ದ್ದಿದ್ದರೆ
ಹೆಚ್ಚಾಗಿ-ರು-ವು-ದ-ರಿಂದ
ಹೆಚ್ಚಾಗು-ತ್ತ
ಹೆಚ್ಚಾಗು-ತ್ತದೆ
ಹೆಚ್ಚಾಗು-ತ್ತಿ-ತ್ತೆಂದು
ಹೆಚ್ಚಾದ
ಹೆಚ್ಚಿ-ದಾಗ
ಹೆಚ್ಚಿ-ರುವ
ಹೆಚ್ಚಿ-ಸು-ತ್ತಲೆ
ಹೆಚ್ಚಿ-ಸು-ವುದು
ಹೆಚ್ಚಿತು
ಹೆಚ್ಚಿನ
ಹೆಚ್ಚಿಸುವ
ಹೆಚ್ಚು
ಹೆಚ್ಚು-ತ್ತಾ
ಹೆಚ್ಚು-ತ್ತಿ-ತ್ತು
ಹೆಚ್ಚು-ಮಂದಿ
ಹೆಚ್ಚು-ವುದು
ಹೆಚ್ಚೆಂ-ದರೆ
ಹೆಚ್ಚೇನೂ
ಹೆಜ್ಜೆ
ಹೆಜ್ಜೆ-ಗಳಷ್ಟು
ಹೆಜ್ಜೆ-ಗಳು
ಹೆಜ್ಜೆ-ಯನ್ನು
ಹೆಜ್ಜೆ-ಯನ್ನೂ
ಹೆಜ್ಜೆ-ಹೆಜ್ಜೆಗೆ
ಹೆಜ್ಜೆಗೂ
ಹೆಜ್ಜೆಗೆ
ಹೆಜ್ಜೆಯೆ
ಹೆಡೆ
ಹೆಣ-ಗಳಾಗಿ-ದ್ದೀರಿ
ಹೆಣ್ಣ-ನ್ನು
ಹೆಣ್ಣಿಗೆ
ಹೆಣ್ಣಿನ
ಹೆಣ್ಣಿನ-ವರು
ಹೆಣ್ಣು
ಹೆಣ್ಣು-ಮಕ್ಕಳ
ಹೆಣ್ಣು-ಮಕ್ಕಳ-ನ್ನು
ಹೆಣ್ಣು-ಮಕ್ಕಳು
ಹೆದ-ರು-ತ್ತೇನೆ
ಹೆದ-ರುವರು
ಹೆದೆಗೆ
ಹೆಬ್ಬಾಗಿಲು
ಹೆರು-ತ್ತಾ
ಹೆಳ-ವನ
ಹೆಸ-ರ-ನ್ನು
ಹೆಸ-ರ-ನ್ನೂ
ಹೆಸ-ರ-ಲ್ಲ
ಹೆಸರಾಂತ
ಹೆಸರೆ
ಹೆಸರೇ
ಹೇ
ಹೇಗಾಗು-ವುದೊ
ಹೇಗಿ-ತ್ತು
ಹೇಗಿ-ದ್ದೀರಿ
ಹೇಗಿ-ರು-ವನೋ
ಹೇಗಿ-ರುವರೋ
ಹೇಗಿರ-ಬೇಕೆಂದು
ಹೇಗೆ
ಹೇಗೆ-ತಾನೆ
ಹೇಗೊ
ಹೇಗೋ
ಹೇಗೋ-ಹೇಡಿ-ಗಳಂತೆ
ಹೇಡಿ-ಗಳೊಂದಿಗೆ
ಹೇಡಿ-ತನ
ಹೇಡಿ-ತನ-ವನ್ನು
ಹೇಡಿ-ತನ-ವೆಂದು
ಹೇಡಿ-ತನವೂ
ಹೇಡಿಯ
ಹೇಮ-ಚಂದ್ರ-ಸೇನ-ನೆ-ಡೆಗೆ
ಹೇಯ
ಹೇರಳ-ವಾ-ಗಿದೆ
ಹೇರು-ವು-ದ-ಲ್ಲ
ಹೇಳ-ಕೂಡ-ದೆಂದು
ಹೇಳ-ತೀ-ರದು
ಹೇಳ-ತೊಡಗಿದನು
ಹೇಳ-ತೊಡಗಿದರು
ಹೇಳ-ದಿದ್ದರೆ
ಹೇಳ-ಬ-ಲ್ಲರು
ಹೇಳ-ಬ-ಲ್ಲಿರಾ
ಹೇಳ-ಬ-ಲ್ಲೆಯಾ
ಹೇಳ-ಬಯ-ಸು-ವು-ದಿ-ಲ್ಲ
ಹೇಳ-ಬಹುದು
ಹೇಳ-ಬಾರ-ದಿ-ತ್ತು
ಹೇಳ-ಬೇ-ಕಾ-ದರೆ
ಹೇಳ-ಬೇಕಾಗಿ-ಲ್ಲ
ಹೇಳ-ಬೇಕಾಗು-ತ್ತದೆ
ಹೇಳ-ಬೇಕಾಗುವುದು
ಹೇಳ-ಬೇಕು
ಹೇಳ-ಬೇಕೆಂ-ದರೆ
ಹೇಳ-ಬೇಕೆಂ-ದಿ-ರುವನೋ
ಹೇಳ-ಬೇಕೆಂದು
ಹೇಳ-ಲಾ-ರದು
ಹೇಳ-ಲಾಗು-ವು-ದಿ-ಲ್ಲ
ಹೇಳ-ಲಿ-ಲ್ಲ
ಹೇಳ-ಲ್ಪಟ್ಟಿ-ದೆ-ಯ-ಲ್ಲಾ
ಹೇಳದ
ಹೇಳದಂ-ತಾ-ಯಿತು
ಹೇಳದೆ
ಹೇಳದೇ
ಹೇಳಬೇ-ಕೇನು
ಹೇಳಲಾರೆ
ಹೇಳಲಿ
ಹೇಳಲಿ-ಲ್ಲ-ವೆಂ-ದರು
ಹೇಳಲಿ-ಲ್ಲವೆ
ಹೇಳಲು
ಹೇಳಲು-ಪ-ಕ್ರಮಿ-ಸುವರು
ಹೇಳಲೆ
ಹೇಳಲೇ-ಬೇಕಾಗಿ-ಲ್ಲ
ಹೇಳಿ
ಹೇಳಿ-ಕಳುಹಿ-ಸಿ-ದರು
ಹೇಳಿ-ಕಳುಹಿ-ಸು-ತ್ತಿ-ದ್ದರು
ಹೇಳಿ-ಕೊ-ಟ್ಟರು
ಹೇಳಿ-ಕೊ-ಟ್ಟಳು
ಹೇಳಿ-ಕೊ-ಟ್ಟಿ-ರುವರು
ಹೇಳಿ-ಕೊ-ಟ್ಟು
ಹೇಳಿ-ಕೊ-ಳ್ಳಲಿ
ಹೇಳಿ-ಕೊ-ಳ್ಳಲು
ಹೇಳಿ-ಕೊ-ಳ್ಳು-ತ್ತಿ-ರಲಿ-ಲ್ಲ
ಹೇಳಿ-ಕೊ-ಳ್ಳು-ತ್ತಿದ್ದ
ಹೇಳಿ-ಕೊ-ಳ್ಳು-ತ್ತಿದ್ದರು
ಹೇಳಿ-ಕೊ-ಳ್ಳು-ತ್ತೇನೆ
ಹೇಳಿ-ಕೊ-ಳ್ಳು-ವು-ದ-ನ್ನು
ಹೇಳಿ-ಕೊ-ಳ್ಳು-ವು-ದಕ್ಕೆ
ಹೇಳಿ-ಕೊ-ಳ್ಳು-ವು-ದಿ-ಲ್ಲ
ಹೇಳಿ-ಕೊ-ಳ್ಳುವರು
ಹೇಳಿ-ಕೊ-ಳ್ಳುವಷ್ಟು
ಹೇಳಿ-ಕೊ-ಳ್ಳುವುದು
ಹೇಳಿ-ಕೊಂಡರು
ಹೇಳಿ-ಕೊಂಡಳು
ಹೇಳಿ-ಕೊಂಡು
ಹೇಳಿ-ಕೊಡ-ಬೇಕು
ಹೇಳಿ-ಕೊಡಲು
ಹೇಳಿ-ಕೊಡು-ತ್ತಿದ್ದರು
ಹೇಳಿ-ಕೊಡು-ತ್ತಿದ್ದಾಗ
ಹೇಳಿ-ಕೊಡು-ತ್ತೇನೆ
ಹೇಳಿ-ಕೊಡು-ವು-ದಕ್ಕೆ
ಹೇಳಿ-ಕೊಳ್ಳು-ತ್ತೀರಿ
ಹೇಳಿ-ತ್ತಿದ್ದೆನು
ಹೇಳಿ-ದ-ಮೇಲೆ
ಹೇಳಿ-ದ-ರ-ಲ್ಲವೆ
ಹೇಳಿ-ದ-ರೆಂ-ದರೆ
ಹೇಳಿ-ದ-ರೆಂದು
ಹೇಳಿ-ದ-ರೇನು
ಹೇಳಿ-ದ-ವ-ರಿಗೂ
ಹೇಳಿ-ದ-ವನೇ
ಹೇಳಿ-ದ-ವರು
ಹೇಳಿ-ದಂತೆ
ಹೇಳಿ-ದನು
ಹೇಳಿ-ದನೊ
ಹೇಳಿ-ದರು
ಹೇಳಿ-ದರೂ
ಹೇಳಿ-ದರೆ
ಹೇಳಿ-ದಳು
ಹೇಳಿ-ದಾಗ
ಹೇಳಿ-ದಿರಿ
ಹೇಳಿ-ದು-ದ-ನ್ನು
ಹೇಳಿ-ದು-ದ-ರಿಂದ
ಹೇಳಿ-ದು-ದರ
ಹೇಳಿ-ದುದ-ನ್ನೆ-ಲ್ಲ
ಹೇಳಿ-ದುದು
ಹೇಳಿ-ದುದೇ
ಹೇಳಿ-ದೆ-ನ-ಲ್ಲ
ಹೇಳಿ-ದೆನು
ಹೇಳಿ-ದೆಯೋ
ಹೇಳಿ-ದೆವು
ಹೇಳಿ-ದೊಡ-ನೆಯೇ
ಹೇಳಿ-ದ್ದ-ನ್ನು
ಹೇಳಿ-ದ್ದ-ಲ್ಲದೆ
ಹೇಳಿ-ದ್ದರು
ಹೇಳಿ-ದ್ದರೋ
ಹೇಳಿ-ದ್ದಾರೆ
ಹೇಳಿ-ದ್ದೀರಿ
ಹೇಳಿ-ದ್ದೇ-ನೆಂ-ದರೆ
ಹೇಳಿ-ದ್ದೇನು
ಹೇಳಿ-ಬಿ-ಟ್ಟ
ಹೇಳಿ-ಬಿ-ಟ್ಟರು
ಹೇಳಿ-ಬಿ-ಟ್ಟಿ-ರು-ವು-ದಾಗಿಯೂ
ಹೇಳಿ-ಬಿಡ-ಬಹು-ದಿ-ತ್ತು
ಹೇಳಿ-ಬಿಡ-ಬಹುದು
ಹೇಳಿ-ಬಿಡಿ
ಹೇಳಿ-ಬಿಡು-ತ್ತಿದ್ದರು
ಹೇಳಿ-ಬಿಡು-ತ್ತೇನೆ
ಹೇಳಿ-ಬಿಡು-ವುದಕ್ಕಿಂತ
ಹೇಳಿ-ರ-ಬಹುದು
ಹೇಳಿ-ರ-ಲಿ-ಲ್ಲ
ಹೇಳಿ-ರು-ತ್ತೇನೆ
ಹೇಳಿ-ರು-ವಂತೆ
ಹೇಳಿ-ರು-ವು-ದ-ನ್ನು
ಹೇಳಿ-ರು-ವು-ದಾಗಿಯೂ
ಹೇಳಿ-ರು-ವುದು
ಹೇಳಿ-ರು-ವುದೇ-ನೆಂ-ದರೆ
ಹೇಳಿ-ರು-ವೆನು
ಹೇಳಿ-ರು-ವೆವು
ಹೇಳಿ-ರುವ
ಹೇಳಿ-ರುವ-ರೆಂದು
ಹೇಳಿ-ರುವರು
ಹೇಳಿ-ರುವೆ
ಹೇಳಿ-ವರು
ಹೇಳಿ-ಸಲು
ಹೇಳಿ-ಸಿ-ಕೊಂಡ-ವ-ನ-ಲ್ಲ
ಹೇಳಿ-ಸಿ-ದರು
ಹೇಳಿ-ಹೋದ
ಹೇಳಿತು
ಹೇಳಿದ
ಹೇಳಿದೆ
ಹೇಳಿದ್ದ
ಹೇಳಿದ್ದು
ಹೇಳಿದ್ದೇ
ಹೇಳೀ-ದನು
ಹೇಳೀ-ದರು
ಹೇಳು
ಹೇಳು-ತಿ-ದ್ದರು
ಹೇಳು-ತ್ತ
ಹೇಳು-ತ್ತದೆ
ಹೇಳು-ತ್ತವೆ
ಹೇಳು-ತ್ತಾ
ಹೇಳು-ತ್ತಾ-ನೆಂಬು-ದ-ನ್ನು
ಹೇಳು-ತ್ತಾ-ರೆಂ-ದಾಗಲಿ
ಹೇಳು-ತ್ತಾನೆ
ಹೇಳು-ತ್ತಾರೆ
ಹೇಳು-ತ್ತಾಳೆ
ಹೇಳು-ತ್ತಿ-ದ್ದು-ದ-ನ್ನು
ಹೇಳು-ತ್ತಿ-ದ್ದು-ದ-ರಿಂದ
ಹೇಳು-ತ್ತಿ-ರಲಿ-ಲ್ಲ
ಹೇಳು-ತ್ತಿ-ರು-ವಿರಿ
ಹೇಳು-ತ್ತಿದೆಯೋ
ಹೇಳು-ತ್ತಿದ್ದ
ಹೇಳು-ತ್ತಿದ್ದನು
ಹೇಳು-ತ್ತಿದ್ದರು
ಹೇಳು-ತ್ತಿದ್ದರೂ
ಹೇಳು-ತ್ತಿದ್ದರೆ
ಹೇಳು-ತ್ತಿದ್ದರೊ
ಹೇಳು-ತ್ತಿದ್ದರೋ
ಹೇಳು-ತ್ತಿದ್ದಳು
ಹೇಳು-ತ್ತಿದ್ದಾಗ
ಹೇಳು-ತ್ತಿದ್ದಾರೆ
ಹೇಳು-ತ್ತಿದ್ದೀ-ರೇನು
ಹೇಳು-ತ್ತಿದ್ದು-ದ-ಲ್ಲದೆ
ಹೇಳು-ತ್ತಿದ್ದುದು
ಹೇಳು-ತ್ತಿದ್ದೆ
ಹೇಳು-ತ್ತಿರ-ಬೇಕು
ಹೇಳು-ತ್ತಿರು-ವ-ರೆಂದೂ
ಹೇಳು-ತ್ತಿರು-ವರು
ಹೇಳು-ತ್ತಿರು-ವಾಗಲೇ
ಹೇಳು-ತ್ತಿರು-ವುದು
ಹೇಳು-ತ್ತಿರುವೆ
ಹೇಳು-ತ್ತೀಯೆ
ಹೇಳು-ತ್ತೀರಾ
ಹೇಳು-ತ್ತೀರಿ
ಹೇಳು-ತ್ತೇನೆ
ಹೇಳು-ತ್ತೇವೆ
ಹೇಳು-ವ-ವ-ರ-ನ್ನು
ಹೇಳು-ವ-ವರು
ಹೇಳು-ವಂತೆ
ಹೇಳು-ವನು
ಹೇಳು-ವರು
ಹೇಳು-ವರೋ
ಹೇಳು-ವಳು
ಹೇಳು-ವಷ್ಟು
ಹೇಳು-ವಾಗ
ಹೇಳು-ವು-ದ-ಕ್ಕಾಗಿ
ಹೇಳು-ವು-ದ-ನ್ನು
ಹೇಳು-ವು-ದ-ರ-ಲ್ಲಿ
ಹೇಳು-ವು-ದಕ್ಕೂ
ಹೇಳು-ವು-ದಕ್ಕೆ
ಹೇಳು-ವು-ದರ
ಹೇಳು-ವು-ದಾಗಿದೆ
ಹೇಳು-ವು-ದಾದರೆ
ಹೇಳು-ವು-ದಿ-ಲ್ಲ
ಹೇಳು-ವು-ದಿ-ಲ್ಲವೆ
ಹೇಳು-ವು-ದೆ-ಲ್ಲ
ಹೇಳು-ವುದ-ನ್ನೆ-ಲ್ಲ
ಹೇಳು-ವುದ-ರ-ಲ್ಲೆ
ಹೇಳು-ವುದು
ಹೇಳು-ವುದೇ
ಹೇಳು-ವುದೇನೋ
ಹೇಳು-ವುರಿ
ಹೇಳು-ವುವು
ಹೇಳು-ವೆನು
ಹೇಳುವ
ಹೇಳುವೆ
ಹೈ
ಹೈಕೋರ್ಟಿನ
ಹೈಡ-ಲ್ಬರ್ಗಿಗೆ
ಹೈಡ್ರೋ-ಜನ್
ಹೊ
ಹೊಂ
ಹೊಂದ-ಬೇಕು
ಹೊಂದಲು
ಹೊಂದಿ
ಹೊಂದಿ-ಕೊಂಡು
ಹೊಂದಿ-ಕೊಂಡು-ಹೋಗ-ಬೇಕು
ಹೊಂದಿ-ದ-ಮೇಲೆ
ಹೊಂದಿ-ದ-ಮೇಲೆಯೆ
ಹೊಂದಿ-ದ್ದಾರೆ
ಹೊಂದಿ-ರ-ಬಹುದು
ಹೊಂದಿ-ರು-ವ-ವರು
ಹೊಂದಿ-ರು-ವುದು
ಹೊಂದಿ-ರುವೆ
ಹೊಂದಿ-ಸಿ-ಕೊ-ಳ್ಳಲಾ-ರದೆ
ಹೊಂದಿ-ಸಿ-ಕೊಳ್ಳ-ಬೇಕು
ಹೊಂದು-ತ್ತದೆ
ಹೊಂದು-ತ್ತಾನೆ
ಹೊಂದು-ವಂತೆ
ಹೊಂದು-ವರು
ಹೊಂದು-ವಿರಿ
ಹೊಂದು-ವುದು
ಹೊಂದು-ವುದೋ
ಹೊಂದುವೆ
ಹೊಂದೇ
ಹೊಕ್ಕು
ಹೊಗ-ದಂತೆ
ಹೊಗ-ಬ-ಲ್ಲ-ವ-ರಾಗಿ-ದ್ದರು
ಹೊಗ-ಬೇಕೆಂದು
ಹೊಗ-ಲಿ-ಲ್ಲ
ಹೊಗಿ
ಹೊಗಿ-ಬಿ-ಟ್ಟಿ-ದ್ದರೆ
ಹೊಗಿ-ರು-ವನು
ಹೊಗಿ-ರುವ
ಹೊಗು-ತ್ತಿ-ರು-ವು-ದ-ರಿಂದಲೇ
ಹೊಗು-ತ್ತಿದ್ದ
ಹೊಗು-ತ್ತಿದ್ದರು
ಹೊಗು-ತ್ತಿದ್ದಳು
ಹೊಗು-ತ್ತಿದ್ದಾಗ
ಹೊಗು-ವಾಗ
ಹೊಗು-ವು-ದ-ನ್ನು
ಹೊಗು-ವು-ದಿ-ಲ್ಲ-ವೆಂದು
ಹೊಗೆ
ಹೊಗೆ-ಸೊಪ್ಪ-ನ್ನು
ಹೊಡೆ-ದರು
ಹೊಡೆ-ದಾಗ
ಹೊಡೆ-ದಾದ
ಹೊಡೆ-ಯಲು
ಹೊಡೆ-ಯು-ತ್ತ
ಹೊಡೆ-ಯು-ತ್ತಿದ್ದರು
ಹೊಡೆ-ಯು-ವು-ದ-ನ್ನು
ಹೊಡೆ-ಯು-ವು-ದಕ್ಕೆ
ಹೊಡೆ-ಯು-ವುದೇ
ಹೊಡೆ-ಯುವ-ವ-ನಿಗೆ
ಹೊಡೆ-ಯುವ-ವನು
ಹೊಡೆದ
ಹೊಡೆದ-ಟ್ಟಿ
ಹೊಡೆದು
ಹೊಡೆದು-ಕೊಂಡು
ಹೊಡೆದೆಬ್ಬಿಸು
ಹೊಡೆಯ
ಹೊಡೆಯ-ತೊಡಗಿದ
ಹೊಡೆಯು-ತ್ತಿ-ದ್ದು-ದ-ನ್ನು
ಹೊಡೆಸಿ
ಹೊಡೆಸಿ-ಕೊಂಡ
ಹೊಣೆ
ಹೊಣೆ-ಗಳುಳ್ಳ
ಹೊದಿಕೆ
ಹೊದೆ
ಹೊದೆ-ಯು-ವು-ದಕ್ಕೆ
ಹೊದ್ದ
ಹೊದ್ದಿ-ಕೆಯ
ಹೊದ್ದು-ಕೊಂಡಿದ್ದ
ಹೊದ್ದು-ಕೊಂಡಿದ್ದನು
ಹೊದ್ದು-ಕೊಂಡಿರು-ತ್ತಾರೆ
ಹೊರ-ಗಡೆ
ಹೊರ-ಗಡೆ-ಯಿಂದ
ಹೊರ-ಗಡೆಯ
ಹೊರ-ಗಡೆಯೆ
ಹೊರ-ಗೆ-ಲ್ಲ
ಹೊರ-ಡದಂ-ತಾ-ಯಿತು
ಹೊರ-ಡುವ
ಹೊರ-ಪಡಿ-ಸು-ತ್ತದೆ
ಹೊರ-ಬಂದು
ಹೊರ-ಬಿ-ತ್ತು
ಹೊರ-ಬಿದ್ದ
ಹೊರ-ಬಿದ್ದಿ-ರ-ಲಿ-ಲ್ಲ
ಹೊರ-ವಲಯ-ದ-ಲ್ಲಿ
ಹೊರಗಿ-ನದು
ಹೊರಗಿ-ನಿಂದ
ಹೊರಗಿ-ರುವ
ಹೊರಗಿ-ಲ್ಲ
ಹೊರಗಿದ್ದ
ಹೊರಗಿನ
ಹೊರಗಿನ-ವರು
ಹೊರಗಿನ-ವರೂ
ಹೊರಗೆ
ಹೊರಗೇಕೆ
ಹೊರಟ
ಹೊರಟ-ವರು
ಹೊರಟನು
ಹೊರಟರು
ಹೊರಟಳು
ಹೊರಟಿ-ದ್ದರೆ
ಹೊರಟಿ-ದ್ದೇನೆ
ಹೊರಟಿತು
ಹೊರಟಿದ್ದ
ಹೊರಟು
ಹೊರಟು-ಬಂದರು
ಹೊರಟು-ಹೋ-ಯಿತು
ಹೊರಟು-ಹೋಗಲು
ಹೊರಟು-ಹೋಗಿದ್ದ
ಹೊರಟು-ಹೋಗು-ತ್ತಿದ್ದನು
ಹೊರಟು-ಹೋಗು-ತ್ತಿದ್ದರು
ಹೊರಟು-ಹೋಗು-ತ್ತೇನೆ
ಹೊರಟು-ಹೋಗು-ವುದು
ಹೊರಟು-ಹೋದ
ಹೊರಟು-ಹೋದ-ಮೇಲೆ
ಹೊರಟು-ಹೋದನು
ಹೊರಟು-ಹೋದರು
ಹೊರಟು-ಹೋದವು
ಹೊರಟು-ಹೋದೆ
ಹೊರಟೆ
ಹೊರಟೆವು
ಹೊರಟೇ
ಹೊರಡ-ಲಿ-ಲ್ಲ
ಹೊರಡಲು
ಹೊರಡಿ
ಹೊರಡಿ-ಸ-ಬೇಕೆಂದು
ಹೊರಡಿ-ಸುವರು
ಹೊರತು
ಹೊರಳಾಡಿ-ದರು
ಹೊರಳಾಡು-ತ್ತಿರುವ
ಹೊರಳಿ
ಹೊರಳಿ-ಸಿ-ದರೆ
ಹೊರಳು-ತ್ತಾನೆ
ಹೊರಹೊ-ಮ್ಮಲಿ
ಹೊರಹೊ-ಮ್ಮು-ವುದು
ಹೊರಿ-ಸಿ-ರುವರು
ಹೊರೆ
ಹೊರೆ-ಯನ್ನು
ಹೊರೆ-ಯಿಂದ
ಹೊರೆ-ಯು-ತ್ತಿರುವೆ
ಹೊರೆ-ಹೊ-ರುವ
ಹೊಲ
ಹೊಲ-ಗಳ-ಲ್ಲಿ
ಹೊಲ-ದ-ಲ್ಲಿ
ಹೊಲ-ವನ್ನು
ಹೊಲಕ್ಕೆ
ಹೊಲಸು
ಹೊಲಿ-ಸಿ-ದರು
ಹೊಲಿದ
ಹೊಳೆ-ದವು
ಹೊಳೆ-ಯಿತು
ಹೊಳೆ-ಯು-ತ್ತಿ-ತ್ತು
ಹೊಳೆ-ಯು-ತ್ತಿದ್ದ
ಹೊಳೆ-ಯು-ತ್ತಿರು-ವನು
ಹೊಳೆ-ಯು-ತ್ತಿರುವ
ಹೊಳೆ-ಯು-ವಂತೆ
ಹೊಳೆ-ಯು-ವು-ದಿ-ಲ್ಲ
ಹೊಳೆ-ಯು-ವುದು
ಹೊಳೆ-ಯು-ವುದೇ
ಹೊಳೆ-ಯುವು-ದೆಂದೂ
ಹೊಳ್ಳೆ-ಗಳ
ಹೊಸ
ಹೊಸ-ತಾಗಿ
ಹೊಸ-ದ-ನ್ನು
ಹೊಸ-ದ-ನ್ನೆ-ಲ್ಲ
ಹೊಸ-ದಾಗಿ
ಹೊಸ-ದಾಗಿ-ತ್ತು
ಹೊಸ-ದಾಗಿ-ರ-ಲಿ-ಲ್ಲ
ಹೊಸ-ದೊಂದು
ಹೊಸ-ಬೆಳಕು
ಹೊಸ-ಮನುಷ್ಯ-ರ-ನ್ನಾಗಿ
ಹೊಸ-ಮನೆ-ಯನ್ನು
ಹೊಸತು
ಹೋ
ಹೋಗ-ಉ-ತ್ತಿದ್ದರು
ಹೋಗ-ಕೂ-ಡದು
ಹೋಗ-ಕೂಡ-ದೆಂದು
ಹೋಗ-ಕೂಡ-ದೆಂದೂ
ಹೋಗ-ತಕ್ಕ-ವನು
ಹೋಗ-ತೊಡಗಿದ
ಹೋಗ-ದಂತೆ
ಹೋಗ-ಬ-ಲ್ಲರು
ಹೋಗ-ಬ-ಲ್ಲೆವು
ಹೋಗ-ಬಯ-ಸುವನು
ಹೋಗ-ಬಯಸಿ-ದರು
ಹೋಗ-ಬಹು-ದಾಗಿ-ತ್ತು
ಹೋಗ-ಬಹು-ದೆಂದು
ಹೋಗ-ಬಹು-ದೆಂದೂ
ಹೋಗ-ಬಹುದು
ಹೋಗ-ಬಾ-ರದು
ಹೋಗ-ಬೇ-ಕಾ-ದರೂ
ಹೋಗ-ಬೇ-ಕಾ-ದರೆ
ಹೋಗ-ಬೇಕಾ-ಯಿತು
ಹೋಗ-ಬೇಕಾಗಿ
ಹೋಗ-ಬೇಕಾಗಿ-ತ್ತು
ಹೋಗ-ಬೇಕಾಗಿ-ರ-ಲಿ-ಲ್ಲ
ಹೋಗ-ಬೇಕಾಗಿ-ರು-ವು-ದ-ರಿಂದ
ಹೋಗ-ಬೇಕಾಗಿ-ಲ್ಲ
ಹೋಗ-ಬೇಕಾಗಿದೆ
ಹೋಗ-ಬೇಕಾಗುವುದು
ಹೋಗ-ಬೇಕು
ಹೋಗ-ಬೇಕೆ
ಹೋಗ-ಬೇಕೆ-ನ್ನು-ವುದು
ಹೋಗ-ಬೇಕೆಂ-ದಿ-ರು-ವು-ದ-ನ್ನು
ಹೋಗ-ಬೇಕೆಂ-ದಿದ್ದ
ಹೋಗ-ಬೇಕೆಂದು
ಹೋಗ-ಬೇಕೆಂದು-ದ-ನ್ನು
ಹೋಗ-ಬೇಕೆಂಬ
ಹೋಗ-ಬೇಕೆಂಬು-ದ-ನ್ನೂ
ಹೋಗ-ಬೇಡ
ಹೋಗ-ಬೇಡಿ
ಹೋಗ-ಲಾಗು-ವು-ದಿ-ಲ್ಲ
ಹೋಗದೆ
ಹೋಗದೇ
ಹೋಗಬೆಕಾ-ಗಿದೆ
ಹೋಗಬೆಕಿ-ಲ್ಲ
ಹೋಗಬೇ-ಕ-ಲ್ಲ
ಹೋಗಲಾ-ರರು
ಹೋಗಲಾಗು-ತ್ತಿ-ಲ್ಲ
ಹೋಗಲಾಡಿ-ಸುವ
ಹೋಗಲಾಡಿ-ಸುವು-ದೆಂದೂ
ಹೋಗಲಾಡಿಸಿ
ಹೋಗಲಾರೆವು
ಹೋಗಲಿ
ಹೋಗಲಿ-ಲ್ಲ
ಹೋಗಲಿ-ಲ್ಲವೆ
ಹೋಗಲು
ಹೋಗಿ
ಹೋಗಿ-ತ್ತು
ಹೋಗಿ-ದ್ದ-ವನು
ಹೋಗಿ-ದ್ದನು
ಹೋಗಿ-ದ್ದರು
ಹೋಗಿ-ದ್ದಳು
ಹೋಗಿ-ದ್ದಾಗ
ಹೋಗಿ-ದ್ದಾರೆ
ಹೋಗಿ-ದ್ದಿ-ರೆಂದು
ಹೋಗಿ-ಬ-ರು-ತ್ತಿದ್ದರು
ಹೋಗಿ-ಬ-ರು-ವು-ದಕ್ಕೆ
ಹೋಗಿ-ಬಂ-ದಿದ್ದರು
ಹೋಗಿ-ಬಂದರು
ಹೋಗಿ-ಬಿ-ಟ್ಟಿ-ರ-ಬಹು-ದೆಂದು
ಹೋಗಿ-ಬಿ-ಡು-ವುದು
ಹೋಗಿ-ಬಿಡು-ತ್ತಿದ್ದರು
ಹೋಗಿ-ಬಿಡು-ತ್ತಿದ್ದರೋ
ಹೋಗಿ-ಬಿಡು-ವಳು
ಹೋಗಿ-ಬಿಡು-ವು-ದಿ-ಲ್ಲ
ಹೋಗಿ-ಬಿಡು-ವುದ-ರ-ಲ್ಲಿ-ತ್ತು
ಹೋಗಿ-ರ-ಲಿ-ಲ್ಲ
ಹೋಗಿ-ರು-ತ್ತದೆ
ಹೋಗಿ-ರು-ತ್ತಿದ್ದರು
ಹೋಗಿ-ರು-ವನು
ಹೋಗಿ-ರು-ವನೊ
ಹೋಗಿ-ರು-ವು-ದ-ರಿಂದ
ಹೋಗಿ-ರು-ವುದು
ಹೋಗಿ-ರು-ವೆನು
ಹೋಗಿ-ರುವ
ಹೋಗಿ-ರುವ-ವರ-ನ್ನೆ-ಲ್ಲ
ಹೋಗಿ-ರುವ-ವರು
ಹೋಗಿ-ರುವರು
ಹೋಗಿ-ರುವೆ-ನೆಂದೂ
ಹೋಗಿ-ರೆಂದು
ಹೋಗಿದ್ದ
ಹೋಗಿದ್ದೆ
ಹೋಗಿವೆ
ಹೋಗು
ಹೋಗು-ತ್ತ
ಹೋಗು-ತ್ತದೆ
ಹೋಗು-ತ್ತಲೂ
ಹೋಗು-ತ್ತವೆ
ಹೋಗು-ತ್ತಾ-ರೆಂದು
ಹೋಗು-ತ್ತಾನೆ
ಹೋಗು-ತ್ತಾರೆ
ಹೋಗು-ತ್ತಿ-ತ್ತು
ಹೋಗು-ತ್ತಿ-ರಲಿ-ಲ್ಲ
ಹೋಗು-ತ್ತಿ-ರು-ವು-ದ-ನ್ನು
ಹೋಗು-ತ್ತಿದೆ
ಹೋಗು-ತ್ತಿದ್ದ
ಹೋಗು-ತ್ತಿದ್ದನು
ಹೋಗು-ತ್ತಿದ್ದರು
ಹೋಗು-ತ್ತಿದ್ದರೂ
ಹೋಗು-ತ್ತಿದ್ದರೆ
ಹೋಗು-ತ್ತಿದ್ದಳು
ಹೋಗು-ತ್ತಿದ್ದವು
ಹೋಗು-ತ್ತಿದ್ದಾಗ
ಹೋಗು-ತ್ತಿದ್ದಾಗಲೇ
ಹೋಗು-ತ್ತಿದ್ದಾನೆ
ಹೋಗು-ತ್ತಿದ್ದುದು
ಹೋಗು-ತ್ತಿದ್ದುವು
ಹೋಗು-ತ್ತಿದ್ದೆ
ಹೋಗು-ತ್ತಿದ್ದೆನು
ಹೋಗು-ತ್ತಿದ್ದೇನೆ
ಹೋಗು-ತ್ತಿರ-ಬೇಕು
ಹೋಗು-ತ್ತಿರಿ
ಹೋಗು-ತ್ತಿರು-ವಂತೆ
ಹೋಗು-ತ್ತಿರು-ವರು
ಹೋಗು-ತ್ತಿರು-ವರೆ
ಹೋಗು-ತ್ತಿರು-ವಾಗ
ಹೋಗು-ತ್ತಿರು-ವುದು
ಹೋಗು-ತ್ತಿರು-ವೆ-ನೆಂದೂ
ಹೋಗು-ತ್ತಿರು-ವೆನು
ಹೋಗು-ತ್ತಿರು-ವೆವು
ಹೋಗು-ತ್ತಿರುವ
ಹೋಗು-ತ್ತಿರುವೆ
ಹೋಗು-ತ್ತೀಯೆ
ಹೋಗು-ತ್ತೀರಿ
ಹೋಗು-ತ್ತೇ-ನೆಂಬು-ದ-ನ್ನು
ಹೋಗು-ತ್ತೇನೆ
ಹೋಗು-ತ್ತೇವೆ
ಹೋಗು-ತ್ತೇವೆಂದು
ಹೋಗು-ವ-ವ-ನ-ಲ್ಲ
ಹೋಗು-ವ-ವ-ರಾಗಿ-ದ್ದರೊ
ಹೋಗು-ವ-ವ-ರಾಗಿ-ರುವ-ವರು
ಹೋಗು-ವ-ವ-ರೆಗೂ
ಹೋಗು-ವ-ವ-ರೆಗೆ
ಹೋಗು-ವ-ವನು
ಹೋಗು-ವಂತಹ
ಹೋಗು-ವಂತೆ
ಹೋಗು-ವನು
ಹೋಗು-ವರು
ಹೋಗು-ವಳು
ಹೋಗು-ವಷ್ಟು
ಹೋಗು-ವಾಗ
ಹೋಗು-ವಾಗಲೂ
ಹೋಗು-ವು-ದ-ನ್ನು
ಹೋಗು-ವು-ದ-ರಿಂದ
ಹೋಗು-ವು-ದಕ್ಕೆ
ಹೋಗು-ವು-ದರೊ-ಳಗೆ
ಹೋಗು-ವು-ದಾಗಿ
ಹೋಗು-ವು-ದಾಗಿಯೂ
ಹೋಗು-ವು-ದಾದರೆ
ಹೋಗು-ವು-ದಿ-ಲ್ಲ
ಹೋಗು-ವು-ದಿ-ಲ್ಲವೋ
ಹೋಗು-ವು-ದೆಂದು
ಹೋಗು-ವುದ-ರ-ಲ್ಲಿ-ದ್ದರು
ಹೋಗು-ವುದ-ರ-ಲ್ಲಿಯೂ
ಹೋಗು-ವುದ-ರ-ಲ್ಲೆ
ಹೋಗು-ವುದಕ್ಕಾಗಲಿ
ಹೋಗು-ವುದು
ಹೋಗು-ವುದೇ
ಹೋಗು-ವುದೇನು
ಹೋಗು-ವುದೋ
ಹೋಗು-ವುವೋ
ಹೋಗು-ವೆ-ನೆಂದೂ
ಹೋಗು-ವೆವು
ಹೋಗುವ
ಹೋಗುವೆ
ಹೋಗೋಣ
ಹೋಗೋಣ-ವೆಂದು
ಹೋಟ-ಲಿಗೆ
ಹೋಟ-ಲ್
ಹೋಟಲಿ-ನ-ಲ್ಲಿ
ಹೋಟೆಲಿ-ನ-ಲ್ಲಿ
ಹೋಟೆಲಿ-ನ-ಲ್ಲಿ-ದ್ದರು
ಹೋಟೆಲಿ-ನ-ಲ್ಲಿ-ಯಾ-ದರೂ
ಹೋಟೆಲಿ-ನ-ಲ್ಲಿ-ರುವ
ಹೋಟೇಲು-ಗ-ಳಿಗೆ
ಹೋದ
ಹೋದ-ನೆಂದು
ಹೋದ-ಮೇಲೆ
ಹೋದ-ರೆಷ್ಟು
ಹೋದ-ರೇ-ನಂತೆ
ಹೋದ-ರೇನು
ಹೋದ-ವ-ನಿಗೆ
ಹೋದ-ವ-ರ-ಲ್ಲ
ಹೋದ-ವ-ರ-ಲ್ಲಿ
ಹೋದ-ವನು
ಹೋದ-ವರು
ಹೋದ-ವರೆ
ಹೋದನು
ಹೋದನೋ
ಹೋದರು
ಹೋದರೂ
ಹೋದರೆ
ಹೋದಳು
ಹೋದವು
ಹೋದಾಗ
ಹೋದೆ
ಹೋದೆ-ನೆಂದು
ಹೋದೆ-ಯೇನು
ಹೋದೆ-ವು-ಹೋಮ
ಹೋದೆವು
ಹೋಮ-ರನ
ಹೋಮ-ವನ್ನು
ಹೋಮಾ-ದಿ-ಗ-ಳನ್ನು
ಹೋಮಾಗ್ನಿ-ಯನ್ನು
ಹೋರಾ-ಡು-ವುದು
ಹೋರಾ-ಡುವ
ಹೋರಾಟ
ಹೋರಾಟ-ದ-ಲ್ಲಿ
ಹೋರಾಟ-ದ-ಲ್ಲಿ-ದ್ದಾಗ
ಹೋರಾಟ-ವ-ಲ್ಲ
ಹೋರಾಟ-ವನ್ನೆ-ಲ್ಲ
ಹೋರಾಟ-ವೆ-ಲ್ಲ
ಹೋರಾಟಕ್ಕೆ
ಹೋರಾಟದ
ಹೋರಾಟವೂ
ಹೋರಾಡ-ಬ-ಲ್ಲರು
ಹೋರಾಡ-ಬೇ-ಕಾ-ದರೆ
ಹೋರಾಡ-ಬೇಕಾ-ಯಿತು
ಹೋರಾಡ-ಬೇಕಾಗಿ
ಹೋರಾಡ-ಬೇಕಾಗಿ-ರ-ಲಿ-ಲ್ಲ
ಹೋರಾಡ-ಬೇಕಾಗಿ-ಲ್ಲ
ಹೋರಾಡ-ಬೇಕಾಗಿದೆ
ಹೋರಾಡ-ಬೇಕಾಗುವುದು
ಹೋರಾಡಲು
ಹೋರಾಡಲೇ
ಹೋರಾಡಿ
ಹೋರಾಡಿ-ದರು
ಹೋರಾಡು-ತ್ತಿ-ರು-ವಿರಿ
ಹೋರಾಡು-ತ್ತಿ-ರು-ವು-ದ-ನ್ನು
ಹೋರಾಡು-ತ್ತಿರು-ವನು
ಹೋರಾಡು-ತ್ತಿರು-ವಾಗ
ಹೋರಾಡು-ತ್ತಿರು-ವೆವು
ಹೋರಾಡು-ತ್ತೇವೆ
ಹೋರಾಡು-ವು-ದ-ಲ್ಲ
ಹೋರಾಡು-ವು-ದಕ್ಕೆ
ಹೋರಿ
ಹೋರಿ-ಯೊಂದು
ಹೋರ್
ಹೋಲಿ-ಸು-ತ್ತಿ-ದ್ದರು
ಹೋಲಿ-ಸು-ತ್ತೀರಿ
ಹೋಲಿ-ಸು-ವು-ದಕ್ಕೆ
ಹೋಲಿ-ಸುವೆ-ಯ-ಲ್ಲ
ಹೋಲಿಸ-ಬ-ಲ್ಲೆ
ಹೋಲಿಸಿ
ಹೋಲಿಸಿ-ಕೊ-ಳ್ಳು-ವು-ದಿ-ಲ್ಲ
ಹೋಲಿಸಿ-ದರೆ
ಹೋಲು-ತ್ತದೆಯೇ
ಹೋಲು-ತ್ತವೆ
ಹೋಲು-ತ್ತಿ-ತ್ತು
ಹೋಲು-ವುದು
ಹೋಲುವ
ಹೋಳಿ
ಹೌ
ಹೌದು
ಹೌದೂ
ಹೌದೇನೋ
ಹೌದೊ
ಹೌರಾ
ಹೌಸಿ-ನ-ಲ್ಲಿ
ಹ್ಯಾ
ಹ್ಯಾಮಂಡ್
ಹ್ಯಾರಿ-ಸನ್
}
