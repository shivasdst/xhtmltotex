
\chapter{ಕ್ಷೀರಭವಾನಿ}

 ಸ್ವಾಮೀಜಿಯವರು ನೇರವಾದ ಹತ್ತಿರದ ದಾರಿಯಲ್ಲಿ ಅಮರನಾಥದಿಂದ\break ಪಹಿಲ್‍ಗಾಮ್‍ಗೆ ಹಿಂತಿರುಗಿದರು. ಇಲ್ಲಿ ಪಾಶ್ಚಾತ್ಯ ಶಿಷ್ಯರೆಲ್ಲರೂ ಉತ್ಸಾಹದಿಂದ ಸ್ವಾಮೀಜಿಯವರಿಗೆ ಕಾದುಕೊಂಡಿದ್ದರು. ಅಲ್ಲಿಂದ ಆಗಸ್ಟ್ ೮ನೇ ತಾರೀಖು ಶ‍್ರೀನಗರಕ್ಕೆ ಹೊರಟರು. ಅಲ್ಲಿ ಸೆಪ್ಟೆಂಬರ್ ೩೦ನೇ ತಾರೀಖಿನವರೆಗೆ ಇದ್ದರು. ಪದೇ ಪದೇ ಸ್ವಾಮೀಜಿ ನಿರ್ಜನ ಪ್ರದೇಶಕ್ಕೆ ಹೋಗಿ ಧ್ಯಾನದಲ್ಲಿ ತಲ್ಲೀನರಾಗುತ್ತಿದ್ದರು. ಸ್ವಾಮೀಜಿ ಅಮರನಾಥದಿಂದ ಬಂದಮೇಲೆ ಹೆಚ್ಚು ಹೆಚ್ಚು ಅಂತರ್ಮುಖಿಗಳಾಗುತ್ತ ಬಂದರು. 

 ಸ್ವಾಮೀಜಿಯವರು ಈ ಸಲ ಕಾಶ್ಮೀರ ಮಹಾರಾಜರ ಕೋರಿಕೆಯ ಮೇಲೆ ಒಂದು ಕೇಂದ್ರವನ್ನು ಅಲ್ಲಿ ಪ್ರಾರಂಭಿಸುವುದಕ್ಕೆ ಸೂಕ್ತವಾದ ಸ್ಥಳವನ್ನು ಆರಿಸಿಕೊಳ್ಳಲು ಶ‍್ರೀನಗರಕ್ಕೆ ಬಂದರು. ಒಂದು ಸುಂದರವಾದ ಸ್ಥಳವನ್ನು ಜೀಲಂ ನದಿ ತೀರದಲ್ಲಿ ಆರಿಸಿ ಆಶ್ರಮವನ್ನು ಸ್ಥಾಪಿಸಲು ಸಿದ್ಧರಾದರು. ಆದರೆ ಅಲ್ಲಿ ಬ್ರಿಟೀಷ್ ರೆಸಿಡೆಂಟ್ ಅದಕ್ಕೆ ಅವಕಾಶ ಕೊಡಲಿಲ್ಲ. ಆದಕಾರಣ ಕಾಶ್ಮೀರದಲ್ಲಿ ಮಠವನ್ನು ಸ್ಥಾಪಿಸುವ ಉದ್ದೇಶವನ್ನು ಬಿಡಬೇಕಾಯಿತು. 

 ಸ್ವಾಮೀಜಿ ಅಮರನಾಥವನ್ನು ನೋಡಿಕೊಂಡು ಶ‍್ರೀನಗರಕ್ಕೆ ಬಂದ ಮೇಲೆ ಅವರ ಮನಸ್ಸು ಶಿವನಿಂದ ಆದಿಶಕ್ತಿಯ ಕಡೆ ಹರಿಯತೊಡಗಿತು. ಯಾವಾಗಲೂ ರಾಮಪ್ರಸಾದರ ಹಾಡನ್ನು ಹಾಡುತ್ತಿದ್ದರು. ತಾವು ಜಗನ್ಮಾತೆಯ ಮಗು ಎಂಬ ಭಾವನೆಯಲ್ಲಿ ಓತಪ್ರೋತರಾಗುವುದಕ್ಕೆ ಯತ್ನಿಸುತ್ತಿದ್ದರು. ಕೆಲವರಿಗೆ ಅವರೊಮ್ಮೆ, ಎಲ್ಲಿ ನೋಡಲಿ ಅಲ್ಲಿ ತಾಯಿ ಇರುವಂತೆ, ಒಂದು ಕೋಣೆಯಲ್ಲಿ ಕುಳಿತುಕೊಂಡಿರುವಂತೆ ಭಾಸವಾಗುತ್ತಿದೆ ಎಂದರು. ಯಾವಾಗಲೂ ತಾಯಿ ತಾಯಿ ಎನ್ನುವುದು ಅವರ ಸ್ವಭಾವವಾಗಿ ಹೋಗಿತ್ತು. 

 ಕ್ರಮೇಣ ಸ್ವಾಮೀಜಿಯವರ ತನ್ಮಯತೆ ಮತ್ತೂ ಗಾಢವಾಗತೊಡಗಿತು. ನಿದ್ರೆ ವಿಶ್ರಾಂತಿಗೂ ಅವಕಾಶ ಕೊಡದಂತೆ ಈ ಅಲೋಚನಾ ಜ್ವರ ಹೇಗೆ ತಮ್ಮನ್ನು ಪೀಡಿಸುತ್ತಿದೆ, ಕೆಲವು ವೇಳೆ ಮಾನವನ ಧ್ವನಿಯಂತೆ ಅದು ತಮ್ಮನ್ನು ಹೇಗೆ ಬಲಾತ್ಕರಿಸುತ್ತಿದೆ ಎಂಬುದನ್ನು ಹೇಳಿಕೊಳ್ಳುತ್ತಿದ್ದರು. ಸುಖದುಃಖ ಪಾಪಪುಣ್ಯಗಳೆಂಬ ದ್ವಂದ್ವಗಳಿಂದ ಪಾರಾಗುವ ಆದರ್ಶವನ್ನು ಶಿಷ್ಯರಿಗೆ ಹೇಳುತ್ತಿದ್ದರು. ಪಾಪ ಭಾವನೆಯಿಂದ\break ಪಾರಾಗುವುದಕ್ಕೆ ಹಿಂದೂಗಳು ನೀಡುವ ಒಂದು ಪರಿಹಾರ ಇದು. ಆದರೆ ಈಗ ಅವರು ತಮ್ಮ ಮನಸ್ಸನ್ನೆಲ್ಲ ಅಂಧಕಾರದ ಕಡೆ, ದುಃಖದ ಕಡೆ, ನಿಗೂಢದ ಕಡೆ ಇಟ್ಟರು. ಇದರ ಮೂಲಕ ಪ್ರತಿಕ್ರಿಯೆ ಹಿಂದೆ ಇರುವ ಸತ್ಯವನ್ನು ಅರಿಯಬೇಕೆಂದು ಸಂಕಲ್ಪಿಸಿಕೊಂಡರು. 

 ರುದ್ರೋಪಾಸನೆಯೆ ಅವರ ಉಪದೇಶದ ಪಲ್ಲವಿ ಆಯಿತು. ಅನಾರೋಗ್ಯವಾಗಲಿ, ಯಾತನೆಯಾಗಲಿ “ಅವಳೇ ದೇಹ, ಈ ನೋವೂ ಅವಳದೇ, ಇದನ್ನು ಕೊಡುವವಳು ಅವಳೇ ಕಾಳಿ!” ಎನ್ನುತ್ತಿದ್ದರು. 

 ಒಂದು ದಿನ ಅವರು ತಮ್ಮ ಮೆದುಳು ಅಲೋಚನೆಯಿಂದ ತುಂಬಿ ತುಳುಕಾಡುತ್ತಿದೆ; ಅದನ್ನು ಬರೆಯುವವರೆಗೆ ತಮ್ಮ ಕೈ ಸುಮ್ಮನಿರಲಾರದು ಎಂದರು. ಸಾಯಂಕಾಲ ಶಿಷ್ಯರೆಲ್ಲ ಎಲ್ಲೊ ಹೊರಗೆ ಹೋಗಿದ್ದರು. ಆಗ ದೋಣಿಯ ಮನೆಗೆ ಬಂದು ‘ತಾಯಿ ಕಾಳಿ’ ಎಂಬ ಕವನವನ್ನು ರಚಿಸಿದರು. ಸ್ಫೂರ್ತಿಯ ಉದ್ವೇಗದಲ್ಲಿ ಬರೆದಾದ ಮೇಲೆ ಅವರಿಗೇ ಸಾಕಾಗಿ ನೆಲದ ಮೇಲೆ ಬಿದ್ದುಬಿಟ್ಟರು. ತಮ್ಮ ಭಾವದ ಉದ್ವೇಗದಲ್ಲಿ ತಾವೇ ದಣಿದು ಹೋಗಿದ್ದರು ಎಂದು ಅನಂತರ ಹೇಳಿದರು. ಅವರ ಇಂಗ್ಲೀಷಿನ ಪದ್ಯದ ಭಾವಾನುವಾದವನ್ನು ಕೆಳಗೆ ಕೊಡುವೆವು. ಅವರು ಬರೆದ ಕವನದಲ್ಲಿ ಇದೊಂದು ಶ್ರೇಷ್ಠದರ್ಜೆಗೆ ಸೇರಿದ ಕವನ.

\begin{verse}
“ತಾರಾ ರಾಶಿಗಳೆಲ್ಲ ನಿಶ್ಯೇಷವಾಗಿ ಕಣ್ಮರೆಯಾಗಿವೆ.\\ಮೇಘ ಮೇಘವನ್ನೂ ಆವರಿಸಿದೆ.\\ಎಲ್ಲೆಲ್ಲೂ ಸ್ಪಂದಿತ ಧ್ವನಿತ ಅಂಧಕಾರ.\\ವಾಯು ವೇಗದಿಂದ ಗರ್ಜಿಸುತ್ತಿದೆ.\\ಲಕ್ಷೋಪಲಕ್ಷ ಉನ್ಮಾದ ಜೀವಿಗಳು ಸೆರೆಮನೆಯಿಂದ ಬಂದಂತಿವೆ,\\ವೃಕ್ಷವನ್ನು ಅಮೂಲಾಗ್ರವಾಗಿ ಕಿತ್ತು ದಾರಿಯಲ್ಲಿರುವುದನ್ನೆಲ್ಲ ಕೊಚ್ಚಿ ಕೊಚ್ಚಿಕೊಂಡು ಹೋಗುತ್ತಿದೆ.\\ಫೇನಮಯವಾದ ಸಮುದ್ರ ಗಾಢಾಂಧಕಾರದಿಂದ ಆವೃತವಾದ ಗಗನವನ್ನು ಚುಂಬಿಸಲು\\ಪರ್ವತೋಮಯ ಅಲೆಗಳನ್ನು ಎಬ್ಬಿಸುತ್ತಿದೆ.\\ಗಗನದಲ್ಲಿ ಕೋರೈಸುತ್ತಿರುವ ಮಿಂಚಿನ ಗೊಂಚಲಿನ ಕಾಂತಿ\\ಲಕ್ಷೋಪಲಕ್ಷ ಛಾಯಾ ಶರೀರಗಳನ್ನು ವ್ಯಕ್ತಗೊಳಿಸುತ್ತಿದೆ.\\ಮೃತ್ಯುವಿನಂತೆ ಕಠೋರವಾಗಿ ಕರಾಳವಾಗಿ, ದುಃಖ, ವ್ಯಸನ, ಪ್ಲೇಗುಗಳನ್ನು ಸುತ್ತಲೂ ಹರಡುತ್ತ\\ಸಂತೋಷದ ಉನ್ಮಾದದಿಂದ ನಾಟ್ಯವಾಡುತ್ತ ತಾಯಿ ನನ್ನೆಡೆಗೆ ಬಾ.\\ನಿನ್ನ ಹೆಸರು ಕರಾಳಿ! ನಿನ್ನ ಉಛ್ವಾಸ ನಿಶ್ವಾಸಗಳಲ್ಲಿ ಮೃತ್ಯುವಿದೆ.\\ನಿನ್ನ ಭೀಮ ಚರಣಗಳ ಒಂದು ಅಡಿ ಒಂದು ಪ್ರಪಂಚವನ್ನೇ ನಾಶಮಾಡುವುದು.\\ಕಾಳಿ, ಸರ್ವಭಕ್ಷಕಳು ನೀನು, ಬಾ ತಾಯಿ ನನ್ನೆಡೆಗೆ ಬಾ!\\ಯಾರಿಗೆ ದುಃಖವನ್ನು ಪ್ರೀತಿಸಿ ಮೃತ್ಯುವನ್ನೇ ಅಪ್ಪಿಕೊಳ್ಳುವ ಸಾಹಸವಿದೆಯೊ,\\ಯಾರು ಅವಳ ಪ್ರಳಯ ನಾಟ್ಯದಲ್ಲಿ ಭಾಗಿಯಾಗಬಲ್ಲರೊ\\ಅವರ ಸಮೀಪಕ್ಕೆ ತಾಯಿ ಬರುವಳು.”
\end{verse}

\vskip 2pt

 ಸ್ವಾಮೀಜಿ ಆ ಸಮಯದಲ್ಲಿ ದೋಣಿಯನ್ನು ಬೇರೆ ಕಡೆ ಒಯ್ದಿದ್ದರು. ಆ ಬೇಸಿಗೆಯಲ್ಲಿ ಸ್ವಾಮೀಜಿಯವರೊಡನೆ ಇರುತ್ತಿದ್ದ ಒಬ್ಬ ಬ್ರಾಹ್ಮೋ ವೈದ್ಯನಿಗೆ ಮಾತ್ರ ಸ್ವಾಮೀಜಿ ಎಲ್ಲಿರುವರು, ಅವರಿಗೆ ಏನು ಬೇಕಾಗಿದೆ ಎಂಬುದು ಗೊತ್ತಾಗುತ್ತಿತ್ತು. ಈತನಿಗೆ ಸ್ವಾಮೀಜಿ ಮೇಲೆ ಇದ್ದ ಭಕ್ತಿ ವಿಶ್ವಾಸಗಳಿಗೆ ಒಂದು ಮೇರೆ ಇರಲಿಲ್ಲ. ಮಾರನೆ ದಿನ ಸಾಯಂಕಾಲ ಡಾಕ್ಟರ್ ಎಂದಿನಂತೆ ಸ್ವಾಮೀಜಿಯವರನ್ನು ನೋಡಲು ಹೋದರು. ಚಿಂತನೆಯಲ್ಲಿ ತನ್ಮಯರಾಗಿದ್ದುದರಿಂದ ಅವರೊಡನೆ ಮಾತನಾಡದೆ ಹಾಗೆಯೆ ಬಂದರು. ಮಾರನೆ ದಿನ ಸೆಪ್ಟೆಂಬರ್ ಮೂವತ್ತನೆ ತಾರೀಖು ಸ್ವಾಮೀಜಿ ಕ್ಷೀರಭವಾನಿಗೆ ಹೋಗಿದ್ದರು. ಯಾರೂ ಅವರನ್ನು ಅನುಸರಿಸಕೂಡದೆಂದು ತಿಳಿಸಿದ್ದರು. ಸ್ವಾಮೀಜಿ ಅಂದಿನಿಂದ ಅಕ್ಟೋಬರ್ ೬ನೇ ತಾರೀಖಿನವರೆಗೆ ಹೊರಗೆ ಹೋಗಿದ್ದರು. 

\vskip 2pt

 ಕ್ಷೀರಭಾವಾನಿ ಕಾಶ್ಮೀರದಲ್ಲಿ ಪ್ರಖ್ಯಾತವಾದ ದೇವಿಗೆ ಸಂಬಂಧಪಟ್ಟ ಯಾತ್ರಾಸ್ಥಳ. ಇತ್ತೀಚೆಗೆ ಸ್ವಾಮೀಜಿ ದೇವಿಯ ಧ್ಯಾನದಲ್ಲಿ ತಲ್ಲೀನರಾಗಿರುತ್ತಿದ್ದರು. ಆ ಮಹಾತಾಯಿ, ಆದಿಶಕ್ತಿಯೊಬ್ಬಳೇ ಸೃಷ್ಟಿ, ಸ್ಥಿತಿ, ಪ್ರಳಯಗಳನ್ನು ಮಾಡುತ್ತಿರುವುದು. ತಾವು ಅವಳ ಏನೂ ಅರಿಯದ ಮಗು ಎಂಬ ಭಾವವನ್ನು ಆರೋಪಿಸಿಕೊಂಡಿದ್ದರು. 

\vskip 2pt

 ಸ್ವಾಮೀಜಿ ಕ್ಷೀರಭವಾನಿಯಲ್ಲಿ ಉಪಾಸನೆ, ಪ್ರತಿದಿನ ಕ್ಷೀರಭವಾನಿಯ ತೀರ್ಥಕ್ಕೆ ಹಾಲು ಅನ್ನ ಬಾದಾಮಿಯನ್ನು ನೈವೇದ್ಯ ಮಾಡುವುದು, ಒಬ್ಬ ಬ್ರಾಹ್ಮಣನ ಕಿರಿಯ ಮಗಳನ್ನು ಉಮಾಕುಮಾರಿ ಎಂದು ಪೂಜಿಸುವುದು ಇವುಗಳನ್ನೆಲ್ಲಾ ಮಾಡಿದರು. ಸ್ವಾಮೀಜಿ ಇಲ್ಲಿದ್ದಾಗಲೆ ಎರಡು ಅನುಭವಗಳಾಗಿ ಅವರ ದೃಷ್ಟಿಯನ್ನೇ ಬದಲಾಯಿಸಿತು. ಮುಸ್ಲಿಮರ ಧಾಳಿಯಿಂದ ದೇವಿಯ ದೇವಸ್ಥಾನವು ಭಗ್ನವಾಗಿ ಹೋಗಿತ್ತು. ಅದನ್ನು ನೋಡಿ ಸ್ವಾಮೀಜಿ ಆ ಸಮಯದಲ್ಲಿ ತಾವು ಇದ್ದರೆ ತಮ್ಮ ಪ್ರಾಣವನ್ನೇ ಕೊಟ್ಟು ಅದನ್ನು ಸಂರಕ್ಷಿಸುತ್ತಿದೆ ಎಂದು ಭಾವಿಸಿದರು. ತಕ್ಷಣವೇ ಅವರಿಗೆ “ನಾನು ನಿನ್ನನ್ನು ಸಂರಕ್ಷಿಸುವೆನೊ, ನೀನು ನನ್ನನ್ನು ಸಂರಕ್ಷಿಸುವೆನೊ?” ಎಂಬ ದೇವಿಯ ವಾಣಿ ಕೇಳಿಸಿತು. ಸ್ವಾಮೀಜಿ ಅಲ್ಲಿರುವ ಭಗ್ನವಾದ ದೇವಸ್ಥಾನದ ಸ್ಥಳದಲ್ಲಿ ತಾವೊಂದು ಅವಳಿಗೆ ಸುಂದರವಾದ ದೇವಸ್ಥಾನವನ್ನು ಕಟ್ಟಬೇಕೆಂದು ಮನಸ್ಸು ಮಾಡಿದರು. ಆಗ ತಾಯಿ “ನನಗೆ ಇಚ್ಛೆಯಾದರೆ ಈ ಕ್ಷಣದಲ್ಲಿ ಒಂದು ಏಳು ಅಂತಸ್ಥಿನ ಚಿನ್ನದ ದೇವಸ್ಥಾನವನ್ನು ಕಟ್ಟಿಸಬಲ್ಲೆ. ಆದರೆ ನನಗೆ ಈ ಪಾಳ ದೇಗುಲದಲ್ಲೇ ಇರಲು ಆಸೆ” ಎಂದಂತೆ ಆಯಿತು. ಅನಂತರ ಸಂಪೂರ್ಣವಾಗಿ ಶರಣಾಗತಿ ಭಾವವನ್ನು ಸ್ವಾಮೀಜಿಯವರ ಜೀವನದಲ್ಲಿ ನೋಡುವೆವು. 

 ಅಕ್ಟೋಬರ್ ೬ನೇ ತಾರೀಖು ಮಧ್ಯಾಹ್ನ ಸ್ವಾಮೀಜಿ ಶಿಷ್ಯರನ್ನು ನೋಡಲು ನದಿ ಮೇಲುಗಡೆಯಿಂದ ಬರುತ್ತಿದ್ದರು. ಅವರು ದೋಣಿಯ ಮುಂದೆ ನಿಂತಿದ್ದರು. ಒಂದು ಕೈಯಲ್ಲಿ ದೋಣಿಯ ಚಾವಣಿಯನ್ನು ಹಿಡಿದುಕೊಂಡಿದ್ದರು. ಮತ್ತೊಂದು ಕೈಯಲ್ಲಿ ಹಳದಿ ಬಣ್ಣದ ಹೂವುಗಳ ಗೊಂಚಲು ಇದ್ದುವು. ಅವರು ಶಿಷ್ಯರ ದೋಣಿಯಲ್ಲಿ ಮನೆಯೊಳಗೆ ಹೋದರು. ಅವರ ವ್ಯಕ್ತಿತ್ವ ಸಂಪೂರ್ಣವಾಗಿ ಮಾರುಹೋಗಿತ್ತು. ಮೌನದಿಂದ ಎಲ್ಲರನ್ನೂ ಆಶೀರ್ವದಿಸುತ್ತ ಹೂವನ್ನು ಶಿಷ್ಯರ ತಲೆಯ ಮೇಲೆ ಇಡುತ್ತ ಬಂದರು. ಕೊನೆಗೆ ಹೂವಿನ ಹಾರವನ್ನು ಶಿಷ್ಯರೊಬ್ಬರಿಗೆ ಕೊಟ್ಟು ಇವುಗಳನ್ನೆಲ್ಲಾ ತಾವು ಜಗನ್ಮಾತೆಗೆ ಅರ್ಪಿಸಿದ್ದೆ ಎಂದು ಹೇಳಿ ಕುಳಿತರು. ಇನ್ನು ಮೇಲೆ “ಹರಿಃ ಓಂ ಇಲ್ಲ. ಎಲ್ಲಾ ತಾಯಿ” ಎಂದರು ನಸುನಗುತ್ತ. ಅನಂತರ “ನನ್ನ ದೇಶಭಕ್ತಿಯೆಲ್ಲಾ ಹೋಯಿತು. ಈಗ ತಾಯಿಯೊಬ್ಬಳೆ ನಿಜ” ಎಂದರು. 

 ಒಂದು ದಿನ ಒಬ್ಬನು ಸ್ವಾಮೀಜಿಯವರನ್ನು ಪ್ರಶ್ನಿಸಲು ಬಂದನು. ಆತ “ಧರ್ಮರಕ್ಷಣೆಗಾಗಿ ಒಬ್ಬ ಸಾವನ್ನು ಎದುರಿಸಬೇಕೆ ಅಥವಾ ಪ್ರತಿಭಟಿಸದೆ ಸುಮ್ಮನೆ ಇರುವುದೆ?” ಎಂದು ಪ್ರಶ್ನಿಸಿದ. ಸ್ವಾಮೀಜಿ ತಾವು ಯಾವ ಪ್ರತಿಭಟನೆಯನ್ನೂ ಮಾಡುವುದಿಲ್ಲವೆಂದರು. ಆದರೆ ಇದು ಸಂನ್ಯಾಸಿಗಳ ಆದರ್ಶ, ಗೃಹಸ್ಥರು ಆತ್ಮ ರಕ್ಷಣೆ ಮಾಡಿಕೊಳ್ಳಬೇಕು ಎಂದರು. 

 ಸ್ವಾಮೀಜಿಯವರ ಸ್ವಭಾವ ದಿನಕಳೆದಂತೆ ಗಂಭೀರವಾಗುತ್ತ ಬಂದಿತು. ಇದು ಒಂದು ತಮ್ಮ ಜೀವನದ ಸಂಧಿಕಾಲವೆಂದು ಸ್ವಾಮೀಜಿ ಅನಂತರ ಹೇಳಿದರು. “ತಾಯಿ ತೊಡೆಯಮೇಲೆ ಕುಳಿತ ಮಗು ನಾನು. ಅವಳು ನನ್ನ ಮೈದಡವುತ್ತಿರುವಳು” ಎಂದು ಹೇಳಿದರು. 

 ಸ್ವಾಮೀಜಿ ಇಲ್ಲಿದ್ದಾಗ ಒಬ್ಬ ಮಹಮ್ಮದೀಯ ಫಕೀರನ ಶಿಷ್ಯನೊಬ್ಬ ಸ್ವಾಮೀಜಿ ಬಳಿ ಬಂದು ಮಾತುಕತೆ ಆಡಿ ಹೋಗುತ್ತಿದ್ದ. ಆತ ಸ್ವಾಮಿಜಿಯವರನ್ನು ಪ್ರೀತಿಸುತ್ತಿದ್ದ. ಆತನಿಗೆ ಒಂದು ಸಲ ಜ್ವರ ಬಂದು ಮನೆಯಲ್ಲಿ ನರಳುತ್ತಿದ್ದ. ಸ್ವಾಮೀಜಿಯವರನ್ನು ನೋಡಲು ಬರುವುದಕ್ಕೆ ಅವನಿಗೆ ಆಗಲಿಲ್ಲ. ಸ್ವಾಮೀಜಿಯೇ ಅವನ ಮನೆಗೆ ಹೋದರು. ರೋಗಿಯ ಶಯ್ಯೆಯ ಪಕ್ಕದಲ್ಲಿ ಕುಳಿತುಕೊಂಡು ಮೈದಡವಿದರು. ಅನಂತರ ಆತನಿಗೆ ಖಾಯಿಲೆ ಗುಣವಾಯಿತು. ಆತನ ಪ್ರೀತಿ ಅನಂತರ ಹಿಂದಿಗಿಂತ ಹೆಚ್ಚಾಗಿತ್ತು. ಸ್ವಾಮೀಜಿ ಬಳಿಗೆ ಹೆಚ್ಚು ಹೋಗತೊಡಗಿದ. ಆತನ ಮಹಮ್ಮದೀಯ ಗುರುವಿಗೆ ಇದನ್ನು ಸಹಿಸಲು ಆಗಲಿಲ್ಲ. ಆತ ಸ್ವಾಮೀಜಿಯವರ ಮೇಲೆ ಏನೇನೋ ಹೇಳಿ ತನ್ನ ಶಿಷ್ಯನಿಗೆ ಅವರ ಬಳಿಗೆ ಹೋಗಬೇಡ ಎಂದನು. ಆದರೆ ಅವನ ಮಾತನ್ನು ಕೇಳಲಿಲ್ಲ. ಆತ ಒಂದು ದಿನ ಸ್ವಾಮೀಜಿಯವರಿಗೆ ತನ್ನ ಶಿಷ್ಯನೆದುರು ಅವರು ಕಾಶ್ಮೀರ ಬಿಡುವುದಕ್ಕೆ ಮೊದಲು ತಲೆಸುತ್ತಿ ವಾಂತಿ ಮಾಡಿಕೊಳ್ಳುವರು ಎಂದು ಶಾಪಕೊಟ್ಟನು.\break ಸ್ವಾಮೀಜಿಗೆ ಒಂದು ದಿನ ಹಾಗೆ ಆಯಿತು. ಅವರು ಅನಂತರ ಆಲೋಚಿಸತೊಡಗಿದರು, ಶ‍್ರೀರಾಮಕೃಷ್ಣರಂತಹವರಲ್ಲಿ ನಾನು ಶರಣಾಗಿದ್ದರೂ ಈ ಫಕೀರನ ಶಾಪ ನನ್ನ ಮೇಲೆ ಫಲಿಸಿತಲ್ಲ, ಇದು ಹೇಗೆ? ಎಂದು. ಅವರೇ ಒಂದು ಸಲ ಕಲ್ಕತ್ತೆಗೆ ಹೋದಮೇಲೆ ಶಾರದಾದೇವಿಯವರನ್ನು ಈ ಪ್ರಶ್ನೆ ಕೇಳಿದರು. ಅದಕ್ಕೆ ಶಾರದಾದೇವಿಯವರು, “ಶ‍್ರೀರಾಮಕೃಷ್ಣರೇ ತಮಗೆ ಕೊಟ್ಟ ಶಾಪವನ್ನು ಸ್ವೀಕರಿಸಿದರು. ಅವರು ಬಂದದ್ದು ಯಾವುದನ್ನೂ ತಿರಸ್ಕರಿಸುವುದಕ್ಕಲ್ಲ” ಎಂದರು. 

\vskip 2pt

 ಸ್ವಾಮೀಜಿಯವರ ವೃಂದ ಬಾರಾಮುಲ್ಲಕ್ಕೆ ಹೊರಟಿತು. ಅಕ್ಟೋಬರ್ ೧೧ನೇ ತಾರೀಖು ಈ ಊರನ್ನು ಸೇರಿದರು. ಮಾರನೆ ದಿನ ಮಧ್ಯಾಹ್ನ ಸ್ವಾಮೀಜಿ ಲಾಹೋರಿಗೆ ಹೋಗುವುದಾಗಿಯೂ, ಮತ್ತು ಅವರ ಶಿಷ್ಯರು ಇನ್ನೂ ಕೆಲವು ದಿನ ಇಲ್ಲೇ ಇರುವುದಾಗಿಯೂ ಗೊತ್ತಾಯಿತು. ಬಾರಾಮುಲ್ಲಕ್ಕೆ ದೋಣಿಯಲ್ಲಿ ಹೋಗುತ್ತಿದ್ದಾಗ ಶಿಷ್ಯರಿಗೆ ಸ್ವಾಮೀಜಿ ಭೇಟಿ ಅಷ್ಟು ಆಗುತ್ತಿರಲಿಲ್ಲ. ಸಂಜೆ ಹೊತ್ತು ದೋಣಿ ಯಾವುದಾದರೊಂದು ಕಡೆ ತಂಗುವುದಕ್ಕೆ ನಿಂತಾಗ ಸ್ವಾಮೀಜಿ ಮೌನವಾಗಿಯೇ ಇರುತ್ತಿದ್ದರು. ನದೀ ತೀರದಲ್ಲಿ ಒಬ್ಬರೇ ದೂರ ನಡೆದುಕೊಂಡುಹೋಗುತ್ತಿದ್ದರು. ಭರತಖಂಡಕ್ಕೆ ಹಿಂತಿರುಗಿ ಬಂದ ಮೇಲೆ ಮಾಡಿದ ಕೆಲಸ, ಮತ್ತು ಈಗ ತಾನೆ ಅವರಿಗೆ ಆದ ಆಧ್ಯಾತ್ಮಿಕ ಅನುಭವ ಇವುಗಳಿಂದ ಶರೀರದ ಮೇಲೆ ಪ್ರತಿಕ್ರಿಯೆಯಾಗಿ ಆರೋಗ್ಯ ಕೆಟ್ಟಿತ್ತು. 

\vskip 2pt

 ಸ್ವಾಮೀಜಿ ತಮ್ಮ ಶಿಷ್ಯರಿಗೆ ತಾವು ಹೋಗುತ್ತೇವೆ ಎಂಬುದನ್ನು ಹೇಳಲು ಬುಧವಾರ ಬೆಳಿಗ್ಗೆ ಬಂದರು. ಶಿಷ್ಯರು ಸ್ವಾಮೀಜಿಯವರಿಗೆ ತಮ್ಮ ದೋಣಿಯಲ್ಲೆ ಕುಳಿತು ಮಾತನಾಡುವಂತೆ ಹೇಳಿದರು. ಅಂದಿನ ಬೆಳಿಗ್ಗೆ ಹಲವು ವಿಷಯಗಳನ್ನು ಹೇಳಿದರು. ಕೆಲವು ವೇಳೆ ಯಾವುದಾದರೊಂದು ಭಕ್ತಿಗೀತೆಯನ್ನು ಹೇಳಿ ಅದನ್ನು ಭಾಷಾಂತರ ಮಾಡುತ್ತಿದ್ದರು. ಯಾವಾಗಲೂ ಹಾಡುಗಳು ತಾಯಿಗೆ ಸಂಬಂಧ ಪಟ್ಟದ್ದಾಗಿದ್ದವು. ತನ್ನ ಭಕ್ತರ ಹೃದಯದ ಮೇಲೆ ನಿಂತು ನಲಿಯುತ್ತಿರುವ ಕಾಳಿ ಶಿಷ್ಯವರ್ಗದವರಿಗೆ ಬರಬರುತ್ತ ಸ್ಪಷ್ಟವಾಗತೊಡಗಿತು. ಅವರು ಬೇರೆ ವಿಷಯಗಳನ್ನು ಹೇಳಿದರು. ತಾಯಿ ಜಗದ ಸಂತೆಯಲ್ಲಿ ಕುಳಿತು ಆಡುವವರೊಡನೆ ಆಡುತ್ತಿರುವಳು, ಅವಳು ಪಟಗಳನ್ನು ಆಡಿಸುತ್ತಿರುವಳು. ಸಹಸ್ರಾರು ಪಟಗಳಲ್ಲಿ ಒಂದರದೋ ಎರಡರದೋ ದಾರವನ್ನು ಕತ್ತರಿಸುವಳು. 

\vskip 2pt

 ಸ್ವಾಮೀಜಿ ತಮ್ಮ ಪದ್ಯದಿಂದಲೇ ಉದಾಹರಿಸತೊಡಗಿದರು:“ದುಃಖ ವ್ಯಾಧಿಗಳನ್ನು ಸುತ್ತ ಹರಡುತ್ತ, ಆನಂದೋನ್ಮಾದದಲ್ಲಿ ಕುಣಿಯುತ್ತ ಬಾ ತಾಯಿ, ಬಾ, ನಿನ್ನ ಹೆಸರೇ ರುದ್ರಾಣಿ. ನಿನ್ನ ಉಸಿರಿನಲ್ಲೆ ಮೃತ್ಯು ಇರುವುದು. ನೀನಿಡುವ ಅಡಿಯೊಂದು ಲೋಕವನ್ನು ಚೂರ್ಣೀಭೂತ ಮಾಡುವುದು.” ಅವರೇ ಅನಂತರ ಹೀಗೆ ಹೇಳಿದರು: “ಇದರಲ್ಲಿರುವ ಪ್ರತಿಯೊಂದೂ ಸತ್ಯವಾಗತೊಡಗಿತು. ಯಾರು ದುಃಖವನ್ನು ಪ್ರೀತಿಸುವರೊ, ಪ್ರಳಯ ತಾಂಡವಗಳೊಂದಿಗೆ ನಾಟ್ಯವಾಡುವರೊ, ಮೃತ್ಯುವನ್ನೇ ಬಾಚಿ ತಬ್ಬಿಕೊಳ್ಳಬಲ್ಲರೊ ಅವರಿಗೆ ತಾಯಿ ನಿಜವಾಗಿ ಪ್ರತ್ಯಕ್ಷಳಾಗುವಳು. ನಾನು ಇದನ್ನು ಸಮರ್ಥಿಸಬಲ್ಲೆ. ಏಕೆಂದರೆ ನಾನೇ ಮೃತ್ಯುವನ್ನು ಅಪ್ಪಿರುವೆನು.” 

 ಅವರು ತಮ್ಮ ಭವಿಷ್ಯವನ್ನು ಕುರಿತು ಹೇಳಿದರು. ಅವರಿಗೆ ಇನ್ನೇನೂ ಬೇಕಾಗಿರಲಿಲ್ಲ. ಗಂಗಾ ತೀರದಲ್ಲಿ ಮೌನವಾಗಿ ಬೆತ್ತಲೆ ಅಲೆಯುತ್ತಿದ್ದರೆ ಸಾಕು. “ಸ್ವಾಮೀಜಿ” ನಾಶವಾಗಿಹೋಗಿದ್ದರು. “ಪ್ರಪಂಚಕ್ಕೆ ಬೋಧಿಸುವ ಜವಾಬ್ದಾರಿ ತನ್ನದು ಎನ್ನುವುದಕ್ಕೆ ನಾನಾರು? ಇದೆಲ್ಲ ಬರೀ ಗಲಾಟೆ, ಭ್ರಮೆ, ತಾಯಿಗೆ ನಾನು ಬೇಕಾಗಿಲ್ಲ. ಆದರೆ ನನಗೆ ಅವಳು ಬೇಕು. ಇದನ್ನರಿತ ಮೇಲೆ ಕರ್ಮ ಕೂಡ ಒಂದು ಕನಸಿನಂತೆ ತೋರುವುದು.” 

 ಪ್ರೀತಿ ವಿನಾ ಬೇರೆ ದಾರಿಯೇ ಇಲ್ಲ. ಇತರರು ನಮಗೆ ವಿರೋಧವಾಗಿ ವರ್ತಿಸಿದರೂ ನಾವು ಅವರನ್ನು ಪ್ರೀತಿಸಬೇಕು. ಆ ಪ್ರೀತಿಯನ್ನು ತಡೆಯುವುದಕ್ಕೆ ಅವರಿಗೆ ಆಗದಂತೆ ಇರಬೇಕು. ಇದೊಂದೇ ದಾರಿ. ಸ್ವಾಮೀಜಿ ಆಗ ವಸಿಷ್ಠ ವಿಶ್ವಾಮಿತ್ರರ ಕಥೆಯನ್ನು ಹೇಳಿದರು. ವಸಿಷ್ಠನ ನೂರು ಮಕ್ಕಳು ಕಾಲವಾಗಿದ್ದರು. ರಾಜ್ಯವಿಲ್ಲದೆ ಸಿಂಹಾಸನವಿಲ್ಲದೆ ಅವನೊಬ್ಬನೇ ಉಳಿದುಕೊಂಡಿದ್ದನು. ಅರಣ್ಯದಲ್ಲಿ ಒಂದು ಕುಟೀರದಲ್ಲಿ ವಾಸಿಸುತ್ತಿದ್ದ ವಸಿಷ್ಠನ ಚಿತ್ರವನ್ನು ಸ್ವಾಮೀಜಿ ಬಣ್ಣಿಸಿದರು. ಅಂದು ಬೆಳದಿಂಗಳ ರಾತ್ರಿ. ವಸಿಷ್ಠ ತನ್ನ ವೈರಿ ವಿಶ್ವಾಮಿತ್ರ ಬರೆದಿದ್ದ ಒಂದು ಗ್ರಂಥವನ್ನು ಓದುವುದರಲ್ಲಿ ತಲ್ಲೀನನಾಗಿದ್ದ. ಆಗ ಅವನ ಹೆಂಡತಿ ಬಂದು ಗಂಡನ ಭುಜದ ಮೇಲೆ ಕೈಯಿಟ್ಟು “ನೋಡಿ! ಹೊರಗೆ ಚಂದ್ರ ಎಷ್ಟು ಪ್ರಕಾಶಮಾನವಾಗಿ ಹೊಳೆಯುತ್ತಿರುವನು” ಎಂದಳು. ವಸಿಷ್ಠ ಅತ್ತ ಕಡೆ ನೋಡದೆ “ಆದರೆ ಪ್ರಿಯೆ, ಚಂದ್ರನಿಗಿಂತ ಹತ್ತು ಸಾವಿರಪಾಲು ಪ್ರಕಾಶಮಾನವಾಗಿದೆ ವಿಶ್ವಾಮಿತ್ರನ ಬುದ್ಧಿ!” ಎಂದನು. 

 ಎಲ್ಲಾ ಮರೆತುಬಿಟ್ಟಿದ್ದನು! ತನ್ನ ನೂರು ಮಕ್ಕಳ ಸಾವು, ಕಷ್ಟ, ನಷ್ಟ, ಎಲ್ಲವನ್ನೂ ಮರೆತುಬಿಟ್ಟಿದ್ದನು. ತನ್ನ ವೈರಿಯ ಪ್ರತಿಭೆಗೆ ಮಾರುಹೋಗಿದ್ದನು! ನಮ್ಮ ಪ್ರೀತಿಯೂ ಕೂಡ ವಸಿಷ್ಠನಿಗೆ ವಿಶ್ವಾಮಿತ್ರನ ಮೇಲೆ ಇದ್ದಂತೆ ಇರಬೇಕು ಎಂದರು. ಸ್ವಲ್ಪವಾದರೂ ದ್ವೇಷದ ಛಾಯೆ ಇರಲಿಲ್ಲ ಅಲ್ಲಿ. 

 ಅಂದೇ ಸ್ವಾಮೀಜಿ ಲಾಹೋರಿಗೆ ಹೋದರು. ಬೇಲೂರು ಮಠಕ್ಕೆ ಅಕ್ಟೋಬರ್ ೧೮ನೇ ತಾರೀಖು ತಲುಪಿದರು. 

