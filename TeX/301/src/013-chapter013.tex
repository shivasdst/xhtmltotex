
\chapter{ಹಿಮಾಲಯ ಪರ್ಯಟನೆ}

ಸ್ವಾಮೀಜಿ ಮೊದಲು ಅಖಂಡಾನಂದರೊಂದಿಗೆ ಭಾಗಲ್‍ಪುರಕ್ಕೆ ಹೋದರು. ಅಲ್ಲಿಂದ ಇನ್ನೂ ಸುಮಾರು ಒಂದು ವರುಷದವರೆಗೆ ಸ್ವಾಮೀಜಿ ಅಖಂಡಾನಂದರೊಡನೆ ಇರುವುದನ್ನು ನೋಡುವೆವು. ಮೊದಲು ಕುಮಾರ ನಿತ್ಯಾನಂದ ಸಿನ್ಹ ಅವರ ಮನೆಯಲ್ಲಿ ಅತಿಥಿಗಳಾಗಿದ್ದರು. ಅನಂತರ ಅಲ್ಲಿಯ ರಾಜಕುಮಾರರ ಉಪಾಧ್ಯಾಯನಾದ ಮನ್ಮಥನಾಥ ಚೌದುರಿಯೊಡನೆ ಇದ್ದರು. ಆತ ಬ್ರಹ್ಮಸಮಾಜದ ನಿಷ್ಠಾವಂತ ಸದಸ್ಯನಾಗಿದ್ದನು. ಮೊದಲು ಮನ್ಮಥಬಾಬು ಸ್ವಾಮೀಜಿ ಅವರನ್ನು ಅಷ್ಟು ಗಮನಿಸಲಿಲ್ಲ. ಅವರು ಅಷ್ಟು ವಿದ್ಯಾವಂತರು ಎಂಬುದನ್ನು ಆಗ ತಿಳಿಯದೆ ಹೋದ. ಆತ ಒಂದು ದಿನ ಬೌದ್ಧಧರ್ಮದ ಒಂದು ಇಂಗ್ಲೀಷ್ ಪುಸ್ತಕವನ್ನು ಓದುತ್ತಿದ್ದ. ಸ್ವಾಮೀಜಿ ಆತನನ್ನು “ನೀನು ಯಾವ ಪುಸ್ತಕವನ್ನು ಓದುತ್ತಿರುವೆ?” ಎಂದು ಕೇಳಿದರು. ಆತ ಅದಕ್ಕೆ “ಇದು ಬೌದ್ಧಧರ್ಮಕ್ಕೆ ಸಂಬಂಧಪಟ್ಟಿರುವುದು. ನಿಮಗೆ ಇಂಗ್ಲಿಷ್ ಗೊತ್ತೆ?” ಎಂದು ಕೇಳಿದ. ಸ್ವಾಮೀಜಿ “ಎಲ್ಲೋ ಸ್ವಲ್ಪ ಗೊತ್ತು” ಎಂದರು. ಅನಂತರ ಅವರೊಡನೆ ಬೌದ್ಧಧರ್ಮದ ವಿಷಯವಾಗಿ ಇಂಗ್ಲೀಷಿನಲ್ಲಿ ಸಂಭಾಷಣೆ ಮಾಡತೊಡಗಿದರು. ಸ್ವಲ್ಪ ಹೊತ್ತು ಮಾತನಾಡಿದಮೇಲೆ ಮನ್ಮಥಬಾಬುವಿಗೆ ಸ್ವಾಮೀಜಿಯವರಿಗೆ ಇಂಗ್ಲಿಷ್ ಭಾಷೆಯ ಮೇಲಿರುವ ಸ್ವಾಮಿತ್ವ ಮತ್ತು ಅವರಿಗೆ ಬೌದ್ಧಧರ್ಮದ ವಿಷಯದ ಮೇಲಿರುವ ಪರಿಶ್ರಮ ಅಸಾಧಾರಣವಾದುದು ಎಂಬುದು ಪರಿಚಯವಾಯಿತು. 

ಒಂದು ದಿನ ಸ್ವಾಮೀಜಿ ಒಂದು ಆಧ್ಯಾತ್ಮಿಕ ಸಾಧನೆಯ ವಿಷಯವನ್ನು ಕುರಿತು ಅವನಿಗೆ ಹೇಳತೊಡಗಿದರು. ಆತ ಆ ಸಾಧನೆಯನ್ನು ಮಾಡುತ್ತಿದ್ದ. ತಾನು ಏನು ಮಾಡುತ್ತಿರುವುದು ಎಂಬುದು ಸ್ವಾಮೀಜಿಗೆ ಗೊತ್ತಾಗಿ ಆ ವಿಷಯದ ಮೇಲೆ ಅಷ್ಟು ವಿದ್ವತ್‍ಪೂರ್ಣವಾಗಿ ಮಾತನಾಡಿದುದನ್ನು ಕೇಳಿ ಮನ್ಮಥನಿಗೆ ಆಷ್ಚರ್ಯವಾಯಿತು. ಅವರ ಸಂಸ್ಕೃತ ಭಾಷೆಯ ಪಾಂಡಿತ್ಯವು ಎಷ್ಟು ಮಟ್ಟಿಗೆ ಇದೆ ಎಂಬುದನ್ನು ಅರಿಯುವುದಕ್ಕಾಗಿ ಉಪನಿಷತ್ತುಗಳಲ್ಲಿ ಕೆಲವು ಭಾಗವನ್ನು ತೆಗೆದು ಅದನ್ನು ವಿವರಿಸಲು ಕೇಳಿದನು. ಆಗ ಸ್ವಾಮೀಜಿಯವರು ಕೊಟ್ಟ ಸ್ವಂತಂತ್ರವಾದ ವಿಚಾರಪೂರಿತವಾದ ವಿವರಣೆಗಳನ್ನು ಕೇಳಿ ಆತ ಅವಾಕ್ಕಾದನು. 

ಸ್ವಾಮೀಜಿ ಒಂದು ದಿನ ತಮಗೆ ತಾವೇ ಒಂದು ಹಾಡನ್ನು ಹೇಳಿಕೊಳ್ಳುತ್ತಿದ್ದರು. ಮನ್ಮಥಬಾಬು “ಸ್ವಾಮೀಜಿಯವರಿಗೆ ಹಾಡಲು ಬರುವುದೆ?” ಎಂದು ಕೇಳಿದಾಗ “ಎಲ್ಲೊ ಸ್ವಲ್ಪ” ಎಂದರು. ಅವರನ್ನು ಬಲಾತ್ಕರಿಸಿದಾಗ ಅವರು ಹಾಡಿದರು. ಅವರ ಸಂಗೀತವನ್ನು ಕೇಳಿದ ಮೇಲೆ ಸ್ವಾಮೀಜಿಗೆ ಶಾಸ್ತ್ರದಲ್ಲಿ ಇರುವಷ್ಟೇ ಅಗಾಧವಾಗಿದೆ ಪಾಂಡಿತ್ಯ ಸಂಗೀತದ ಮೇಲೂ ಎಂಬುದು ಅರಿವಾಯಿತು. ಒಂದು ದಿನ ಸಂಜೆ ಅವನ ಮನೆಗೆ ಭಾಗಲ್‍ಪುರದ ಹಲವು ಪುರಪ್ರಮುಖರು ಮತ್ತು ಸಂಗೀತವನ್ನು ಬಲ್ಲವರನ್ನು ಕರೆಸಲಾಯಿತು. ಸಂಜೆ ಸ್ವಾಮೀಜಿ ಹಾಡುವುದಕ್ಕೆ ಪ್ರಾರಂಭ ಮಾಡಿದರು. ಅವರು ಒಂದಾದ ಮೇಲೆ ಒಂದನ್ನು ಹಾಡುತ್ತಲೇ ಹೋದರು. ಕುಳಿತವರು ಕಾಲವನ್ನೇ ಮರೆತರು, ಸ್ವಾಮೀಜಿಯವರ ಗಾನಕ್ಕೆ ಮನಸೋತರು. ಹಾಡುವುದನ್ನು ನಿಲ್ಲಿಸಿದಾಗ ಅರ್ಧರಾತ್ರಿ ಸುಮಾರು ಎರಡು ಗಂಟೆ. ಕೆಲವು ಗಂಟೆಗಳ ಮುಂಚೆಯೇ ಮೃದಂಗವನ್ನು ಹೊಡೆಯುವವನಿಗೆ ಬೆರಳುಗಳು ಸೋತುಹೋಗಿ ಬಾರಿಸುವುದನ್ನು ಬಿಟ್ಟುಬಿಟ್ಟಿದ್ದನು. ಮಾರನೆಯ ದಿನ ಸಾಯಂಕಾಲವೂ ಸ್ವಾಮೀಜಿಯವರ ಗಾನವನ್ನು ಕೇಳಲು ಜನರೆಲ್ಲ ಬಂದರು. ಆದರೆ ಸ್ವಾಮೀಜಿ ಅಂದು ಹಾಡಲಿಲ್ಲ. 

ಒಂದು ದಿನ ಮನ್ಮಥಬಾಬು ಭಾಗಲ್‍ಪುರದ ಶ‍್ರೀಮಂತರ ಮನೆಗೆಲ್ಲ ತಾನು ಗಾಡಿಯಲ್ಲೆ ಸ್ವಾಮೀಜಿಯವರನ್ನು ಕರೆದುಕೊಂಡು ಹೋಗಿ ಪರಿಚಯ ಮಾಡಿಸುತ್ತೇನೆ ಎಂದು ಹೇಳಿದ. ಅದಕ್ಕೆ ಸ್ವಾಮೀಜಿ, ಏನೂ ಕೆಲಸವಿಲ್ಲದೆ ಸುಮ್ಮನೇ ಶ‍್ರೀಮಂತರನ್ನು ನೋಡಲು ಹೋಗುವುದು ಸಂನ್ಯಾಸ ಧರ್ಮಕ್ಕೆ ವಿರುದ್ಧ ಎಂದು ಹೇಳಿ ಆತನೊಡನೆ ಹೋಗಲು ಒಪ್ಪಲಿಲ್ಲ. ಬಾಲ್ಯದಿಂದಲೂ ಮನ್ಮಥಬಾಬುವಿಗೆ ಯಾವುದಾದರೂ ಒಂದು ಪುಣ್ಯಸ್ಥಳಕ್ಕೆ ಹೋಗಿ ಸಾಧನೆ ಭಜನೆಯಲ್ಲಿ ನಿರತನಾಗಬೇಕೆಂದು ಆಸೆ ಇತ್ತು. ಆದರೆ ಅದು ಫಲಕಾರಿ ಆಗಿರಲಿಲ್ಲ. ಸ್ವಾಮೀಜಿ ಅವರನ್ನು ಕಂಡಮೇಲೆ ಪುನಃ ಹಿಂದಿನ ಅಭಿಲಾಷೆ ಮೇಲೆದ್ದಿತು. ಆತ ಸ್ವಾಮೀಜಿಗೆ, ತಮ್ಮಿಬ್ಬರ ಹೆಸರಿನಲ್ಲಿ ಬೃಂದಾವನದ ದೇವಸ್ಥಾನದಲ್ಲಿ, ಮುನ್ನೂರು ರೂಪಾಯಿಗಳನ್ನು ಡಿಪಾಸಿಟ್ ಮಾಡುವೆನೆಂದೂ, ಅದರಿಂದ ಬದುಕಿರುವ ಪರಿಯಂತವೂ ದೇವಸ್ಥಾನದಿಂದ ಇಬ್ಬರಿಗೂ ಪ್ರಸಾದದ ಊಟ ಬರುವುದೆಂದೂ, ಇಬ್ಬರೂ ಹತ್ತಿರ ಯಾವುದಾದರೂ ಒಂದು ಸ್ಥಳದಲ್ಲಿ ತಪಸ್ಸಿನಲ್ಲಿ ನಿರತರಾಗಬಹುದೆಂದೂ ಹೇಳಿದನು. ಸ್ವಾಮೀಜಿ ಅದಕ್ಕೆ “ಅದು ನನ್ನ ಸ್ವಭಾವಕ್ಕೆ ಒಗ್ಗುವುದಿಲ್ಲ” ಎಂದರು. ಅವರ ಸ್ವಭಾವ ಭಗವಂತನನ್ನೇ ನೆಚ್ಚಿಕೊಂಡು ಪರಿವ್ರಾಜಕರಾಗಿ ಅಲೆಯುವುದು. 

ಮನ್ಮಥಬಾಬುವಿಗೆ ಸ್ವಾಮೀಜಿ ಹಿಂದೂಧರ್ಮದ ವಿಷಯಗಳನ್ನು ಪರಿಚಯ ಮಾಡಿಸಿದರು ಮತ್ತು ಶ‍್ರೀಕೃಷ್ಣನ ಜೀವನದ ಕೆಲವು ಘಟನೆಗಳ ಮಹತ್ವವನ್ನು ಕುರಿತು ಮಾತನಾಡಿದರು. ಇದರಿಂದ ಆತನಿಗೆ ಹಿಂದೂಧರ್ಮದ ಮೇಲೆ ಇದ್ದ ಅಸಡ್ಡೆ ಜಾರಿ ಉದಾರ ದೃಷ್ಟಿಯಿಂದ ಅದನ್ನು ನೋಡಲು ಪ್ರಯತ್ನಿಸಿದನು. ಮನ್ಮಥಬಾಬು ಸ್ವಾಮೀಜಿ ಮೇಲೆ ತುಂಬಾ ಆಸಕ್ತನಾಗಿ ಹೋಗಿದ್ದ. ಆತನಿಗೆ ಹೇಳಿ ಒಪ್ಪಿಗೆ ತೆಗೆದುಕೊಂಡು ಹೋಗುವುದು ಅಸಾಧ್ಯವಾಗಿ ತೋರಿತು. ಆದಕಾರಣ ಒಂದು ಸಲ ಆತ ಮನೆಯಲ್ಲಿ ಇಲ್ಲದಾಗ ಸ್ವಾಮೀಜಿ ಮತ್ತು ಅಖಂಡಾನಂದರು ಮನೆಯವರಿಗೆ ಹೇಳಿ ಹೊರಟು ಹೋದರು. ಮನ್ಮಥಬಾಬು ಮನೆಗೆ ಬಂದಮೇಲೆ ಸ್ವಾಮೀಜಿ ಅವರಿಂದ ಆದ ಅಗಲಿಕೆಗೆ ಬಹಳ ವ್ಯಥೆಪಟ್ಟನು. ಹಲವು ಕಡೆಗೆ ಹೋಗಿ ವಿಚಾರಿಸಿದನು ಅವರು ಎಲ್ಲಿಗೆ ಹೋಗಿರುವರು ಎಂದು. ಆದರೆ ಎಲ್ಲಿಯೂ ಅವನಿಗೆ ಅವರ ಸುಳು ಸಿಕ್ಕಲಿಲ್ಲ. 

 ಸ್ವಾಮೀಜಿ ಭಾಗಲ್‍ಪುರವನ್ನು ಬಿಟ್ಟಮೇಲೆ ಅಖಂಡಾನಂದರೊಡನೆ ವೈದ್ಯನಾಥಕ್ಕೆ\break ಹೋದರು. ಅಲ್ಲಿ ಬಾಬು ರಾಜನಾರಾಯಣಬೋಸ್ ಎಂಬ ಬ್ರಹ್ಮಸಮಾಜದ ಅನುಯಾಯಿಯ ಮನೆಯಲ್ಲಿದ್ದರು. ಅನಂತರ ಅಲ್ಲಿಂದ ಇಬ್ಬರೂ ಕಾಶಿಗೆ ಹೋದರು. ಕಾಶಿಯಲ್ಲಿ ಸ್ವಾಮೀಜಿ ವಿದ್ವಾಂಸರಾದ ಪ್ರಮದಬಾಬುಗಳ ಮನೆಯಲ್ಲಿ ಇದ್ದರು. ಶಾಸ್ತ್ರದ ಮೇಲೆ ಬೇಕಾದಷ್ಟು ಇಬ್ಬರೂ ಚರ್ಚೆಯನ್ನು ಮಾಡಿದರು. ಸ್ವಾಮೀಜಿಯವರಿಗೆ ಹಿಮಾಲಯಕ್ಕೆ ಹೋಗಬೇಕೆಂಬ ಆಸೆ ಬಲವಾಗಿದ್ದುದರಿಂದ ಅಲ್ಲಿ ಬಹಳ ಕಾಲ ತಂಗಲಿಲ್ಲ. ತಾವು ಹೊರಡುವುದಕ್ಕೆ ಪ್ರಮದಬಾಬುಗಳ ಅಪ್ಪಣೆ ತೆಗೆದುಕೊಳ್ಳುವ ವೇಳೆ, “ನಾನು ಪುನಃ ಹಿಂತಿರುಗುವಾಗ ಸಮಾಜದ ಮೇಲೆ ಒಂದು ಬಾಂಬಿನಂತೆ ಬೀಳುವೆನು. ಅದು ನನ್ನನ್ನು ನಾಯಿ ಅನುಸರಿಸುವಂತೆ ಅನುಸರಿಸುವುದು” ಎಂದು ಸ್ವಾಮೀಜಿ ಹೇಳಿದರು. ನಿಜವಾಗಿಯೂ ಸ್ವಾಮೀಜಿ ಮತ್ತೊಮ್ಮೆ ಕಾಶಿಗೆ ಬಂದಾಗ ಲೋಕಪ್ರಖ್ಯಾತರಾಗಿದ್ದರು. ಅವರಾಡಿದ ಮಾತನ್ನು ಕೇಳುವುದಕ್ಕೆ, ಅದನ್ನು ಅನುಸರಿಸುವುದಕ್ಕೆ, ಇಡೀ ಸಮಾಜ ಸಿದ್ಧವಾಗಿರುವುದನ್ನು ನೋಡುವೆವು. 

ಅನಂತರ ಸ್ವಾಮೀಜಿ ಮತ್ತು ಅವರ ಗುರುಭಾಯಿಗಳು ನೈನಿತಾಲಿಗೆ ಹೋದರು. ಅಲ್ಲಿ ಸುಮಾರು ಹದಿನೈದು ದಿನಗಳು ರಾಮಪ್ರಸನ್ನ ಭಟ್ಟಾಚಾರ‍್ಯ ಎಂಬುವರ ಮನೆಯಲ್ಲಿ ಇದ್ದರು. ಅನಂತರ ಬದರಿಕಾಶ್ರಮದ ಕಡೆ ಹೋಗಬೇಕೆಂದು ನಿರ್ಧರಿಸಿದರು. ದಾರಿಯನ್ನು ಬರೀ ನಡೆದುಕೊಂಡೇ ಹೋಗಿ, ಊಟಕ್ಕೆ ಭಿಕ್ಷೆ ಮಾಡಿಕೊಂಡು ಹೋಗುತ್ತೇವೆ ಎಂದು ಹಟತೊಟ್ಟರು. ಮೂರು ದಿನಗಳು ನಡೆದಾದಮೇಲೆ ಒಂದು ದಿನ ನದಿಯ ಸಮೀಪವಿದ್ದ ಅಷ್ವತ್ಥವೃಕ್ಷದ ಕೆಳಗೆ ಸ್ವಾಮೀಜಿ ಧ್ಯಾನಕ್ಕೆ ಕುಳಿತರು. ಆ ಸ್ಥಳದ ಹೆಸರು ಕಾಕ್ರಿಘಾಟ್ ಎಂದು. ಗಾಢಧ್ಯಾನದಲ್ಲಿ ಅವರಿಗೊಂದು ನೂತನ ಭಾವನೆ ಹೊಳೆಯಿತು. ಅನಂತರ ಇದನ್ನು ಕುರಿತು ಅಖಂಡಾನಂದರಿಗೆ ಹೇಳಿದರು: “ಓ ಗಂಗಾಧರ್! ಇಂದು ನನ್ನ ಜೀವನದ ಒಂದು ಅವಿಸ್ಮರಣೀಯ ದಿನ. ಈ ಅಶ್ವತ್ಥ ವೃಕ್ಷದ ಕೆಳಗೆ ನನ್ನ ಜೀವನದ ಒಂದು ದೊಡ್ಡ ಸಮಸ್ಯೆ ಪರಿಹಾರವಾದಂತಾಯಿತು. ಬ್ರಹ್ಮಾಂಡ ಪಿಂಡಾಂಡಗಳೆರಡು ಒಂದೇ ಎಂಬುದನ್ನು ಕಂಡೆ. ಈ ಪಿಂಡಾಂಡವಾದ ದೇಹದಲ್ಲಿ ಬ್ರಹ್ಮಾಂಡದಲ್ಲಿರುವುದೆಲ್ಲ ಇದೆ. ಒಂದು ಪರಮಾಣುವಿನಲ್ಲಿ ಇಡೀ ವಿಶ್ವವನ್ನು ಕಂಡೆ.” ಈ ಭಾವನೆಗಳನ್ನೇ ಅನಂತರ ಸ್ವಾಮೀಜಿ ಜ್ಞಾನಯೋಗದ ಉಪನ್ಯಾಸಗಳಲ್ಲಿ ವಿಸ್ತಾರ ಮಾಡುವರು. 

ಸ್ವಾಮೀಜಿ ಆಲ್ಮೋರಕ್ಕೆ ಬಂದಮೇಲೆ ಅಖಂಡಾನಂದರು ಅವರನ್ನು ತಮಗೆ ಪೂರ್ವಪರಿಚಿತವಾದ ಅಂಬಾದತ್ತರ ತೋಟದ ಮನೆಗೆ ಕರೆದುಕೊಂಡು ಹೋದರು. ಅನಂತರ ಅದೇ ಊರಿನಲ್ಲಿದ್ದ ಶಾರದಾನಂದ ಮತ್ತು ವೈಕುಂಠನಾಥರಿಗೆ ತಾವು ಬಂದಿರುವ ವಿಷಯವನ್ನು ತಿಳಿಸಲು ಹೋದರು. ಈ ಇಬ್ಬರು ಗುರುಭಾಯಿಗಳು ಆಲ್ಮೋರದ ಲಾಲಾ ಬದರಿಶಾ ಮನೆಯಲ್ಲಿ ಕೆಲವು ಕಾಲಗಳಿಂದ ಇದ್ದರು. ಸ್ವಾಮೀಜಿ ಬಂದಿರುವರೆಂಬ ಸಮಾಚಾರ ಅವರಿಗೆ ಗೊತ್ತಾದ ಕೂಡಲೆ ಅಂಬಾದತ್ತನ ಉದ್ಯಾನವನದಲ್ಲಿ ಅವರನ್ನು ನೋಡಲು ಹೊರಟರು. ಅವರು ಅಲ್ಲಿಗೆ ಹೋಗುತ್ತಿದ್ದಾಗ ಸ್ವಾಮೀಜಿಯವರೇ ಅವರನ್ನು ನೋಡಲು ಬರುತ್ತಿದ್ದರು. ಈಗ ಎಲ್ಲರೂ ಬದರಿಶಾ ಅವರ ಮನೆಗೆ ಹೋದರು. ಆ ಊರಿನ ಶಿರಸ್ತೇದಾರರಾದ ಶ‍್ರೀಕೃಷ್ಣಜೋಶಿ ಎಂಬುವರ ಹತ್ತಿರ ತ್ಯಾಗದ ಮೇಲೆ ಒಂದು ದೊಡ್ಡ ಚರ್ಚೆಯನ್ನು ನಡೆಸಿದರು. ಸ್ವಾಮೀಜಿಯವರ ಪಾಂಡಿತ್ಯವನ್ನು ನೋಡಿ ನೆರೆದವರೆಲ್ಲರೂ ಬೆರಗಾದರು. ಈ ಸ್ಥಳದಲ್ಲಿಯೇ ಇವರ ಸಹೋದರಿ ಜೀವನದಲ್ಲಿ ಯಾವುದೋ ಕಷ್ಟಕ್ಕೆ ಸಿಕ್ಕಿ ಆತ್ಮಹತ್ಯೆಯನ್ನು ಮಾಡಿಕೊಂಡಳೆಂಬ ತಂತಿ ಕಲ್ಕತ್ತಾದಿಂದ ಬಂದಿತು. ಸ್ವಾಮೀಜಿ ಈ ಸುದ್ದಿಯನ್ನು ಕೇಳಿ ದುಃಖಿತರಾದರು. ಆಕೆಯ ಸಂಕಟ ಕಷ್ಟಗಳು ಹಲವು ನಾರಿಯರಿಗೂ ಒಂದಲ್ಲ ಒಂದು ವಿಧದಲ್ಲಿ ಇದ್ದೇ ಇವೆ. ಅದರ ಪರಿಹಾರೋಪಾಯವನ್ನು ಕಂಡುಹಿಡಿಯಲು ಯತ್ನಿಸಿದರು. ಅವರ ತಂಗಿಯ ದೆಸೆಯಿಂದ ಭರತಖಂಡದ ಸ್ತ್ರೀಯರ ಸಮಸ್ಯೆಯ ಕಡೆ ಅವರ ಮನಸ್ಸು ಹರಿಯಿತು. ಸಮಾಜವೆಂಬ ಹಕ್ಕಿಯ ಒಂದು ರೆಕ್ಕೆಯಂತೆ ಸ್ತ್ರೀ. ಹಕ್ಕಿಯ ಒಂದು ರೆಕ್ಕೆ (ಪುರುಷ) ಎಷ್ಟೇ ಬಲವಾಗಿದ್ದರೂ ಇನ್ನೊಂದು ದುರ್ಬಲವಾಗಿದ್ದರೆ ಬಹಳ ಕಾಲ ಹಾರಾಡಲಾರದು. ಮಹಾಪುರುಷರು ಕೇವಲ ಒಂದು ವ್ಯಕ್ತಿಯ ಮೂಲಕ ತಮಗೆ ಬರುವ ಕಷ್ಟದಿಂದ ಸಮಷ್ಟಿಯಲ್ಲಿರುವ ನ್ಯೂನತೆಯ ಪರಿಹಾರದ ಕಡೆ ಗಮನ ಕೊಡುವರು. ಆಲ್ಮೋರದಿಂದ ಶಾರದಾನಂದ ಅಖಂಡಾನಂದ ವೈಕುಂಠನಾಥ್ ಇವರೊಡನೆ ಗಾರ್‍ವಾಲ್ ಪ್ರಾಂತ್ಯದ ಕಡೆ ಹೊರಟರು. 

ಗಾರ್‍ವಾಲ್ ಪ್ರಾಂತ್ಯದಲ್ಲಿ ಸಂಚಾರ ಮಾಡುತ್ತಿದ್ದಾಗ ಅಖಂಡಾನಂದರು ಖಾಯಿಲೆ ಬಿದ್ದರು. ಆದರೂ ನಡೆದುಕೊಂಡು ಕರ್ಣಪ್ರಯಾಗವನ್ನು ತಲುಪಿದರು. ಮೂರು ದಿನಗಳು ಅಲ್ಲಿ ತಂಗಿದ್ದು ಸುಧಾರಿಸಿಕೊಂಡು ಮುಂದುವರಿದರು. ದಾರಿಯಲ್ಲಿ ಒಂದು ಛತ್ರದಲ್ಲಿ ತಂಗಿದ್ದಾಗ ಅಖಂಡಾನಂದರಿಗೆ ಪುನಃ ಆರೋಗ್ಯ ಕೆಟ್ಟಿತು. ಒಂದು ವಾರ ಅಲ್ಲಿ ತಂಗಿದ್ದು ವಿಶ್ರಾಂತಿಯನ್ನು ತೆಗೆದುಕೊಂಡು ಅಲ್ಲಿಂದ ಮುಂದುವರಿದು ರುದ್ರಪ್ರಯಾಗಕ್ಕೆ ಬಂದರು. ಅಲ್ಲಿ ಹಿಮಾಲಯ ದೃಶ್ಯ ಬಹಳ ರಮ್ಯವಾಗಿದೆ. ಸುತ್ತಲೂ ಪರ್ವತಶಿಖರಗಳು, ಮಧ್ಯದಲ್ಲಿ ಹರಿಯುತ್ತಿರುವ ಗಿರಿ ಝರಿಗಳು, ಸಣ್ಣ ಸಣ್ಣ ಪ್ರಪಾತಗಳು ಮತ್ತು ಪ್ರಶಾಂತಿಯನ್ನು ಬೀರುತ್ತಿರುವ ಕಾನನಗಳು ಒಬ್ಬನನ್ನು ಅಂತರ್ಮುಖನನ್ನಾಗಿ ಮಾಡುವುದು. ಇಲ್ಲಿ ಪೂರ್ಣಾನಂದ ಎಂಬ ಬಂಗಾಳಿ ಸಾಧುಗಳೊಡನೆ ಒಂದು ದಿನ ಇದ್ದು ಮುಂದಕ್ಕೆ ಹೊರಟರು. ಸ್ವಲ್ಪ ಮೈಲಿಗಳು ನಡೆದುಕೊಂಡು ಹೋದಮೇಲೆ ಇಬ್ಬರು ಸ್ವಾಮಿಗಳಿಗೂ ಜ್ವರ ಬಂತು. ದಾರಿಯಲ್ಲಿದ್ದ ಒಂದು ಛತ್ರದಲ್ಲಿ ತಂಗಿದರು. ಅಕಸ್ಮಾತ್ತಾಗಿ ಗಾರ್‍ವಾಲ್ ಜಿಲ್ಲೆಯ ಅಧಿಕಾರಿ ಕೆಲಸದ ಮೇಲೆ ಬಂದಿದ್ದವನು ಅಲ್ಲಿ ತಂಗಿದ್ದನು. ಆತನ ಹೆಸರು ಬದರೀ ದತ್ತ ಜೋಶಿ ಎಂದು. ಆತ ಖಾಯಿಲೆ ಬಿದ್ದಿದ್ದ ಸಾಧುಗಳಿಗೆ ತನ್ನ ಹತ್ತಿರ ಇದ್ದ ಆಯುರ್ವೇದ ಔಷಧಿಯನ್ನು ಕೊಟ್ಟು ಕೆಲವು ದಿನಗಳ ಮೇಲೆ, ಅವರ ಆರೋಗ್ಯ ಉತ್ತಮಗೊಂಡ ಮೇಲೆ ಅವರನ್ನು ಒಂದು ದಂಡಿಯಲ್ಲಿ ಒಂಭತ್ತು ಮೈಲಿಗಳು ದೂರದಲ್ಲಿರುವ ಶ‍್ರೀನಗರಕ್ಕೆ ಕಳುಹಿಸಿಕೊಟ್ಟ. ಶ‍್ರೀನಗರದಲ್ಲಿ ಅಲಕಾನಂದಾ ನದೀ ತೀರದಲ್ಲಿದ್ದ ಒಂದು ಸಣ್ಣ ಕುಟೀರದಲ್ಲಿ ಅವರು ವಾಸಿಸತೊಡಗಿದರು. ಭಿಕ್ಷೆಯಿಂದ ಊಟ ಮಾಡುತ್ತಿದ್ದರು. ತಮ್ಮ ಕಾಲವನ್ನು ಸ್ವಾಮಿಗಳು ಧ್ಯಾನದಲ್ಲಿ ಕಳೆದರು. ಪ್ರತಿದಿನ ಅಖಂಡಾನಂದರಿಗೆ ಸ್ವಾಮೀಜಿ ಉಪನಿಷತ್ತನ್ನು ಹೇಳಿಕೊಡುತ್ತಿದ್ದರು. ಶ‍್ರೀನಗರದಲ್ಲಿ ವೈಶ್ಯಕುಲಕ್ಕೆ ಸೇರಿದ ಒಬ್ಬ ಶಾಲಾ ಉಪಾಧ್ಯಾಯನನ್ನು ಕಂಡರು. ಆತ ಇತ್ತೀಚೆಗೆ ಕೈಸ್ತಮತಕ್ಕೆ ಸೇರಿದ್ದ. ಸ್ವಾಮೀಜಿ ಹಿಂದೂಧರ್ಮದ ಮಹಾತ್ಮ್ಯೆಯನ್ನು ವಿವರಿಸಿದಾಗ ಆ ಉಪಾಧ್ಯಾಯ ಹೊಸದಾಗಿ ಅಂಗೀಕರಿಸಿದ ಕ್ರೈಸ್ತಧರ್ಮವನ್ನು ತ್ಯಜಿಸಿ, ಸನಾತನ ಹಿಂದೂಧರ್ಮಕ್ಕೆ ಮರಳಿ ಬಂದನು. ಕ್ರಿಸ್ತನ ವ್ಯಕ್ತಿತ್ವದ ಮೇಲೆ ಪ್ರೀತಿ ಇದ್ದರೆ, ಹಿಂದುವಾಗಿ ಇದೇ ಎಲ್ಲಾ ಮತಕ್ಕೆ ಸೇರಿದ ಮಹಾತ್ಮರನ್ನೂ ಪೂಜಿಸಬಹುದು, ಅದಕ್ಕಾಗಿ ಒಬ್ಬ ಮತಾಂತರವನ್ನು ಮಾಡಿಕೊಳ್ಳಬೇಕಾಗಿಲ್ಲವೆಂಬುದನ್ನು ಅರಿತನು. 

ಶ‍್ರೀನಗರದಿಂದ ಟೆಹರಿಗೆ ಹೊರಟರು. ಅಲ್ಲಿ ಗಂಗಾತೀರದಲ್ಲಿದ್ದ ಒಂದು ತೋಟದಲ್ಲಿ ಸಾಧುಗಳಿಗೆ ಎಂದು ಕುಟೀರಗಳನ್ನು ಮಾಡಿದ್ದರು. ಅದು ಖಾಲಿ ಇದ್ದುದರಿಂದ ಇಬ್ಬರೂ ಅಲ್ಲಿ ವಾಸಿಸತೊಡಗಿದರು. ಭಿಕ್ಷಾನ್ನದಿಂದ ಜೀವಿಸುತ್ತ ಪ್ರಾರ್ಥನೆ ಮತ್ತು ಧ್ಯಾನದಲ್ಲಿ ಕಾಲವನ್ನು ಕಳೆಯತೊಡಗಿದರು. ಕೆಲವು ಕಾಲದ ಮೇಲೆ ಟೆಹರಿರಾಜರ ದಿವಾನರಾಗಿದ್ದ ಬಾಬು ರಘುನಾಥ ಭಟ್ಟಾಚಾರ‍್ಯ ಎಂಬುವರ ಪರಿಚಯವಾಯಿತು. ಆತ ಬಂಗಾಳೀ ದೇಶದವರು. ಕಲ್ಕತ್ತೆಯಲ್ಲಿದ್ದ ಹರಿಪ್ರಸಾದ ಶಾಸ್ತ್ರಿ ಎಂಬುವರ ಅಣ್ಣ. ಸ್ವಾಮೀಜಿ ಅವರೊಡನೆ ಕೆಲವು ಕಾಲ ಇದ್ದರು. ಗಂಗಾ ನದಿಯ ತೀರದಲ್ಲಿ ಸುಂದರವಾದ ನಿರ್ಜನವಾದ ಯಾವುದಾದರೊಂದು ಸ್ಥಳದಲ್ಲಿ ತಪಸ್ಸಿನಲ್ಲಿ ಕಾಲ ಕಳೆಯಬೇಕೆಂದು ಸ್ವಾಮೀಜಿಗೆ ಆಸಕ್ತಿಯಿತ್ತು. ಇದನ್ನು ಬಾಬು ರಘುನಾಥ ಭಟ್ಟಾಚಾರ‍್ಯರಿಗೆ ವ್ಯಕ್ತಪಡಿಸಿದಾಗ ಅವರು ಇದಕ್ಕೆ ಒಂದು ಯೋಗ್ಯವಾದ ಸ್ಥಳವನ್ನು ಹುಡುಕಿಕೊಟ್ಟರು. ಅದೇ ಗಂಗಾ ಮತ್ತು ವಿಲಂಗನ ನದಿಗಳ ಸಂಗಮ ಸ್ಥಳವಾದ ಗಣೇಶಪ್ರಯಾಗ ಎಂಬುದು. ಆದರೆ ವಿಧಿ ಅದಕ್ಕೆ ಅವಕಾಶ ಕೊಡಲಿಲ್ಲ. ಅಂದೇ ಅಖಂಡಾನಂದರು ಖಾಯಿಲೆ ಬಿದ್ದರು. ಸ್ಥಳ ವೈದ್ಯರು ಇವರನ್ನು ಪರೀಕ್ಷಿಸಿ ಇಲ್ಲಿ ಅವರ ಆರೋಗ್ಯ ಸರಿಹೋಗುವುದಿಲ್ಲವೆಂದೂ, ಇನ್ನು ಮೇಲೆ ಮತ್ತೂ ಛಳಿ ಹೆಚ್ಚುತ್ತಾ ಬರುವುದೆಂದೂ ಅವರನ್ನು ಬಯಲು ಸೀಮೆಗೆ ಕರೆದುಕೊಂಡುಹೋಗಿ ಚಿಕಿತ್ಸೆ ಮಾಡುವುದು ಮೇಲು ಎಂದೂ ಹೇಳಿದರು. ಸ್ವಾಮೀಜಿ ದಿವಾನ್‍ಜಿಯವರಿಗೆ ಪರಿಸ್ಥಿತಿಯನ್ನು ವಿವರಿಸಿ ಸದ್ಯಕ್ಕೆ ಗಣೇಶಪ್ರಯಾಗಕ್ಕೆ ಹೋಗುವುದನ್ನು ನಿಲ್ಲಿಸಬೇಕಾಗುವುದೆಂದೂ, ಮುಂದೆ ಯಾವಾಗಲಾದರು ಅನುಕೂಲವಾದರೆ ಪುನಃ ಬರುತ್ತೇನೆಂದೂ ಹೇಳಿದರು. ದಿವಾನ್‍ಜಿಯವರು ಡೆಹರಾಡೂನ್‍ನಲ್ಲಿ ಇರುವ ವೈದ್ಯರಿಗೆ ಒಂದು ಪತ್ರವನ್ನು ಬರೆದುಕೊಟ್ಟು, ಇಬ್ಬರು ಸ್ವಾಮಿಗಳಿಗೂ ಹಿಂತಿರುಗಿ ಹೋಗುವುದಕ್ಕೆ ಕುದುರೆಯನ್ನು ಮಾಡಿಕೊಟ್ಟು ದಾರಿಯ ಖರ್ಚಿನ ಭಾರವನ್ನೆಲ್ಲ ವಹಿಸಿದರು. 

ಟೆಹರಿಯಿಂದ ಹೊರಟವರು ಮಸೂರಿಯ ಮೂಲಕ ರಾಜಪುರಕ್ಕೆ ಹೊರಟರು. ಇಲ್ಲಿ ಸ್ವಾಮಿ ತುರಿಯಾನಂದರು ಅವರನ್ನು ಸೇರಿಕೊಂಡರು. ಅನಂತರ ಮೂರು ಜನರೂ ಡೆಹರಾಡೂನಿಗೆ ಹೋಗಿ ಸ್ವಾಮೀಜಿಯವರು ತಂದಿದ್ದ ಪತ್ರವನ್ನು ಅಲ್ಲಿಯ ಆಸ್ಪತ್ರೆಯ ವೈದ್ಯರಾದ ಡಾಕ್ಟರ್ ಮೇಕಲಾರೆನ್ ಅವರಿಗೆ ಕೊಟ್ಟರು. ಅವರು ಅಖಂಡಾನಂದರನ್ನು ಪರೀಕ್ಷಿಸಿ ಅವರ ಶ್ವಾಸಕೋಶಗಳು ಸರಿಯಾಗಿಲ್ಲವೆಂದೂ ಅವರನ್ನು ಕೆಳಗಿನ ಬಯಲು ಸೀಮೆಗೆ ಕರೆದುಕೊಂಡು ಹೋಗಬೇಕು ಎಂದೂ ಹೇಳಿದರು. ಆಗ ಅಖಂಡಾನಂದರು ತುಂಬಾ ನಿತ್ರಾಣರಾಗಿದ್ದರು. ಮುಂದಕ್ಕೆ ಹೊರಡುವ ಸ್ಥಿತಿಯಲ್ಲಿರಲಿಲ್ಲ. ಕೆಲವು ಕಾಲ ಡೆಹರಾಡೂನಿನಲ್ಲೇ ತಂಗಬೇಕಾಗಿ ಬಂದಿತು. ಸ್ವಾಮೀಜಿ ಅಖಂಡಾನಂದರಿಗೆ ತಂಗುವುದಕ್ಕೆ ಒಂದು ಯೋಗ್ಯವಾದ ಸ್ಥಳ ಮತ್ತು ಅವರಿಗೆ ಬೇಕಾಗುವ ಪಥ್ಯದ ಊಟಕ್ಕೆ ಎಲ್ಲಿಯಾದರೂ ಸಾಧ್ಯವೆ ಎಂದು ಹಲವರನ್ನು ಕೇಳಿದರು. ರೋಗಿ ಸ್ವಾಮಿಯ ಜವಾಬ್ದಾರಿಯನ್ನು ತೆಗೆದುಕೊಳ್ಳುವುದಕ್ಕೆ ಯಾವ ಗೃಹಸ್ಥನೂ ಮುಂದೆ ಬರಲಿಲ್ಲ. ಕೊನೆಗೆ ಕಾಶ್ಮೀರಿಯ ಬ್ರಾಹ್ಮಣನಾದ ಪಂಡಿತ ಅನಂತನಾರಾಯಣ ಎಂಬ ವಕೀಲನು ಅವರನ್ನು ನೋಡಿಕೊಳ್ಳಲು ಒಪ್ಪಿಕೊಂಡನು. ಆತ ಒಂದು ಸಣ್ಣ ಮನೆಯನ್ನು ಅವರಿಗಾಗಿ ಬಾಡಿಗೆಗೆ ಗೊತ್ತುಮಾಡಿ, ಶಾಖವಾದ ಬಟ್ಟೆಗಳನ್ನು ಕೊಟ್ಟು ಊಟಕ್ಕೆ ಅಣಿಮಾಡಿದ.. ಸ್ವಾಮೀಜಿ ಮೂರು ವಾರಗಳು ಡೆಹರಾಡೂನಿನಲ್ಲಿದ್ದು ಅಖಂಡಾನಂದರು ಸ್ವಲ್ಪ ಉತ್ತಮಗೊಂಡಮೇಲೆ ಅಲಹಾಬಾದಿನಲ್ಲಿರುವ ಒಬ್ಬ ಸ್ನೇಹಿತನ ಮನೆಗೆ ಅವರನ್ನು ಹೋಗುವಂತೆ ಹೇಳಿ ಅವರು ಹೃಷೀಕೇಶಕ್ಕೆ ಹೊರಟರು. 

ಸಾಧುಗಳಿಗೆ ಯೋಗ್ಯವಾದ ಸ್ಥಳ ಹೃಷೀಕೇಶ. ಆಗಿನ ಕಾಲದಲ್ಲಿ ಅದು ನಿರ್ಜನ ಪ್ರದೇಶವಾಗಿತ್ತು. ಸಾಧುಗಳಲ್ಲದೆ ಬೇರೆ ಯಾರೂ ಅಲ್ಲಿರುತ್ತಿರಲಿಲ್ಲ. ಹೃಷೀಕೇಶ ಹಿಮಾಲಯದ ತಪ್ಪಲಿನಲ್ಲಿದೆ. ಗಂಗಾನದಿ ಒಂದು ಕಣಿವೆಯ ಒಳಗೆ ಅಲ್ಲಿ ಹರಿದುಕೊಂಡು ಹೊಗುತ್ತಿದೆ. ಗಂಗಾನದಿಯ ಅಕ್ಕಪಕ್ಕದಲ್ಲೆಲ್ಲ ದಟ್ಟವಾದ ಕಾಡು. ಆ ಕಾಡಿನ ಮಧ್ಯದಲ್ಲಿ ಸಾಧುಗಳು ತಪಸ್ಸಿಗಾಗಿ ಸಣ್ಣ ಕುಟೀರಗಳನ್ನು ಕಟ್ಟಿಕೊಂಡು ವಾಸಿಸುತ್ತಿದ್ದರು. ಹೃಷೀಕೇಶದ ವಾತಾವರಣವೇ ತಪಸ್ಸಿನಿಂದ ಓತಪ್ರೋತವಾಗಿತ್ತು. ಸ್ವಾಮೀಜಿ ಇಂತಹ ರಮ್ಯವಾದ ಸ್ಥಳದಲ್ಲಿ ಚಂಡಿಕೇಶ್ವರ ಮಹಾದೇವ ಎಂಬ ದೇವಸ್ಥಾನದ ಪಕ್ಕದಲ್ಲಿ ಭಿಕ್ಷೆಯಿಂದ ಜೀವಿಸುತ್ತ ಧ್ಯಾನದಲ್ಲಿ ಕಾಲಕಳೆದರು. ಉಗ್ರವಾದ ತಪಸ್ಸಿನಲ್ಲಿ ಇಲ್ಲೆ ಕೆಲವು ಕಾಲ ಇರಬೇಕೆಂದು ಸ್ವಾಮೀಜಿಯವರ ಬಯಕೆ ಆಗಿತ್ತು. ಆದರೆ ಅವರಿಗೆ ದಾರುಣವಾದ ಜ್ವರ ಬಂದಿತು, ಪಜ್ಞೆಯೂ ಕೂಡಾ ತಪ್ಪಿಹೋಯಿತು. ಹತ್ತಿರದಲ್ಲಿ ಯಾವ ವೈದ್ಯಕೀಯ ಸಹಾಯವೂ ಇರಲಿಲ್ಲ. ಹುಲ್ಲಿನ ಮೇಲೆ ಹಾಸಿದ ಒಂದು ಕಂಬಳಿಯ ಮೇಲೆ ಅವರು ಮಲಗಿದ್ದರು. ಆ ಸಮಯದಲ್ಲಿ ಆ ಸ್ಥಳದ ಒಬ್ಬ ಮನುಷ್ಯ ಯಾವುದೋ ಒಂದು ಮೂಲಿಕೆಯ ರಸವನ್ನು ಜೇನು ತುಪ್ಪದಲ್ಲಿ ಬೆರಸಿ ಪ್ರಜ್ಞೆಯಿಲ್ಲದೆ ಮಲಗಿದ್ದ ಸ್ವಾಮೀಜಿಯವರ ಬಾಯಿಗೆ ಬಲವಂತದಿಂದ ತುರುಕಿದನು. ಕ್ರಮೇಣ ಸ್ವಾಮೀಜಿಗೆ ಪ್ರಜ್ಞೆ ಬಂದಿತು. ಹತ್ತಿರವಿದ್ದ ಗುರುಭಾಯಿಗಳಿಗೆ ಸಂತೋಷವಾಯಿತು. ಸ್ವಾಮೀಜಿ ಅವರು ಸ್ವಲ್ಪ ಚೇತರಿಸಿಕೊಂಡಮೇಲೆ ಅವರನ್ನು ಹರಿದ್ವಾರಕ್ಕೆ ಕರೆದುಕೊಂಡು ಹೋದರು. ಸ್ವಾಮಿ ಬ್ರಹ್ಮಾನಂದರು ಅಲ್ಲಿ ಸ್ವಾಮೀಜಿಯವರೊಡನೆ ಸೇರಿಕೊಂಡರು. ಎಲ್ಲರೂ ಅನಂತರ ಶಹರಾನ್‍ಪುರಕ್ಕೆ ಹೋದರು. ಅಲ್ಲಿಯ ವಕೀಲರಾದ ಬಂಕು ಬಿಹಾರಿಬಾಬು ಎಂಬುವರ ಮನೆಯಲ್ಲಿ ತಂಗಿದ್ದು ಅಲ್ಲಿಂದ ಮೀರತ್ತಿಗೆ ಹೋದರು. ಮೀರತ್ತಿನಲ್ಲಿ ಸ್ವಾಮಿ ಅಖಂಡಾನಂದರು ಡಾಕ್ಟರ್ ತ್ರೈಲೋಕನಾತ್ ಬಾಬು ಅವರ ಮನೆಯಲ್ಲಿದ್ದರು. ಬಹಳ ಕ್ಷೀಣವಾಗಿ ಹೊಗಿರುವ ಸ್ವಾಮೀಜಿಯವರನ್ನು ವೈದ್ಯರು ಮನೆಗೆ ಕರೆದುಕೊಂಡು ಹೋಗಿ ಅಲ್ಲಿಯೇ ಹದಿನೈದು ದಿನಗಳು ಇಟ್ಟುಕೊಂಡು ಚಿಕಿತ್ಸೆ ಮಾಡಿದರು. ಇತರ ಸಾಧ್ಯುಗಳು ಯಜ್ಞೇಶ್ವರ ಬಾಬು ಎಂಬುವರ ಮನೆಯಲ್ಲಿದ್ದರು. ಅನಂತರ ಎಲ್ಲರೂ ಯಜ್ಞೇಶ್ವರಬಾಬುವಿನ ಸ್ನೇಹಿತರಾದ ಒಬ್ಬ ಸೇಟರ ತೋಟದಲ್ಲಿ ವಾಸಿಸತೊಡಗಿದರು. ಸ್ವಾಮೀಜಿಯವರು ಇನ್ನೂ ಔಷಧಿಯನ್ನು ತೆಗೆದುಕೊಳ್ಳುತ್ತಲೇ ಇದ್ದರು. ಕ್ರಮೇಣ ಅವರು ಗುಣಮುಖರಾದರು. 

ಸುತ್ತಮುತ್ತ ಸಂಚಾರಮಾಡುತ್ತಿದ್ದ ಗುರುಭಾಯಿಗಳೆಲ್ಲ ಸೇಟ್‍ಜಿ ಉದ್ಯಾನದಲ್ಲಿ ಒಟ್ಟಿಗೆ ಸೇರಿದರು. ಸ್ವಾಮಿ ಬ್ರಹ್ಮಾನಂದ, ಅಖಂಡಾನಂದ, ತುರಿಯಾನಂದ, ಶಾರದಾನಂದ, ವೈಕುಂಠನಾಥ ಇವರುಗಳೆಲ್ಲ ಇದ್ದರು. ಜೊತೆಗೆ ಕೆಲವು ದಿನಗಳು ಆದಮೇಲೆ ಅದ್ವೈತಾನಂದರೂ ಇವರ ಜೊತೆಗೆ ಸೇರಿದರು. ಸ್ವಾಮೀಜಿ ಇಲ್ಲಿ ಸಂಸ್ಕೃತ ಸಾಹಿತ್ಯದಿಂದ ಹಲವು ಸ್ವಾರಸ್ಯವಾದ ಭಾಗಗಳನ್ನು ಗುರುಭಾಯಿಗಳಿಗೆ ಓದಿ ಹೇಳುತ್ತಿದ್ದರು. ಬೆಳಗ್ಗೆ ಸಾಯಂಕಾಲ ಭಜನೆ ಧ್ಯಾನಾದಿಗಳಲ್ಲಿ ಕಳೆದರು. ಸ್ವಾಮೀಜಿ ಅವರು ವಿಚಾರಪೂರಿತವಾದ ಪುಸ್ತಕಗಳನ್ನು ಓದಬೇಕೆಂದು ಬಯಸಿ ಅಖಂಡಾನಂದರಿಗೆ ಮೀರತ್ ಸಾರ್ವಜನಿಕ ಪುಸ್ತಕಾಲಯದಿಂದ ಸರ್‍ಜಾನ್ ಲಬೆಕ್ ಬರೆದ ಪುಸ್ತಕಗಳನ್ನು ತರಲು ಹೇಳಿದರು. ಆ ಪುಸ್ತಕಗಳನ್ನು ಸ್ವಾಮೀಜಿ ಒಂದು ದಿನದಲ್ಲಿ ಓದಿ ಪೂರೈಸಿ ಅಖಂಡಾನಂದರಿಗೆ ಅದನ್ನು ಹಿಂತಿರುಗಿ ಕೊಡುವುದಕ್ಕೆ ಹೇಳಿದರು. ಲೈಬ್ರರಿಯವನು ಅಖಂಡಾನಂದರು ಅಷ್ಟು ಬೇಗ ಪುಸ್ತಕಗಳನ್ನು ಹಿತಿರುಗಿ ತಂದು ಕೊಡುವುದನ್ನು ನೋಡಿ, “ನೀವು ಯಾರಿಗೆ ತೆಗೆದುಕೊಂಡು ಹೋದಿರೊ ಅವರು ಇಷ್ಟು ಬೇಗ ಪುರೈಸಿ ಬಿಟ್ಟರೆ?” ಎಂದು ವಿಸ್ಮಯವನ್ನು ತೋರಿಸಿದ. ಅಖಂಡಾನಂದರು ಸ್ವಾಮೀಜಿಗೆ ಈ ಸಮಾಚಾರವನ್ನು ಕೊಟ್ಟಾಗ ಅವರು ಲೈಬ್ರರಿಯನ್ ಬಳಿಗೆ ಹೊಗಿ, ತಾವು ಆ ಪುಸ್ತಕಗಳನ್ನು ಓದಿರುವುದಾಗಿಯೂ, ಬೇಕಾದರೆ ಆ ಪುಸ್ತಕದಲ್ಲಿ ಯಾವುದಾದರೂ ವಿಷಯವನ್ನು ಕೇಳಬಹುದೆಂದೂ ಹೇಳೀದರು. ಆತ ಪ್ರಶ್ನೆಯನ್ನು ಕೇಳಿದ. ಸ್ವಾಮೀಜಿ ಉತ್ತರವನ್ನು ಹೇಳುವಾಗ ಆ ಪುಸ್ತಕದ ವಾಕ್ಯ ವಾಕ್ಯಗಳನ್ನೇ ಉದ್ಧರಿಸುತ್ತಿದ್ದರು. ಲೈಬ್ರರಿಯನ್ ವಿಸ್ಮಿತನಾಗಿ “ಇದು ಹೇಗೆ ಸಾಧ್ಯ” ಎಂದು ಕೇಳಿದಾಗ ಸ್ವಾಮೀಜಿ, ತಾವು ಪುಸ್ತಕವನ್ನು ಸಾಧಾರಣ ಮನುಷ್ಯರು ಓದುವಂತೆ ಓದುವುದಿಲ್ಲವೆಂದೂ, ಕೆಲವು ವಾಕ್ಯಗಳನ್ನು ಓದಿಬಿಟ್ಟರೆ ಗ್ರಂಥಕರ್ತ ಯಾವುದನ್ನು ಹೇಳಬೇಕೆಂದಿರುವನೋ ಅದೆಲ್ಲ ಮನಸ್ಸಿನಲ್ಲಿ ಹೊಳೆಯುವುದೆಂದೂ ಹೇಳಿದರು. ಯಾರಿಗೆ ಪ್ರಥಮ ಬಾರಿ ಭಾವನೆಗಳು ಪರಿಚಯವಾಗಬೇಕಾಗಿರುವುದೊ ಅವರಿಗೆ ಅಸಾಧ್ಯವಾದ ಕೆಲಸ ಇದು. ಆದರೆ ಸ್ವಾಮೀಜಿ ಬುದ್ಧಿಶಕ್ತಿಗೆ ವಿಶ್ವದ ಭಾವನೆಗಳೆಲ್ಲದರ ಪರಿಚಯ ಆಗಲೆ ಆಗಿದೆ. ಅಲ್ಲಿ ಆಗಲೇ ಒಂದು ದೊಡ್ಡ ಜ್ಞಾನನಿಧಿ ಇದೆ. ಯಾವುದಾದರೂ ಒಂದನ್ನು ಓದಿದಾಗ ಅದಕ್ಕೆ ಸಂಬಂಧಪಟ್ಟ ಆಗಲೇ ಅವರಲ್ಲಿರುವ ಭಾವನೆಗಳೆಲ್ಲ ತಕ್ಷಣವೇ ಸ್ಫುರಿಸುವುವು. 

ಮೀರತ್ತಿನಲ್ಲಿ ಸ್ವಾಮೀಜಿ ಸುಮಾರು ಐದು ತಿಂಗಳುಗಳನ್ನು ಕಳೆದರು. ಸ್ವಾಮೀಜಿ ಈ ಸಲದ ಹಿಮಾಲಯದ ಪರ್ಯಟನೆಯಲ್ಲಿ ಹಲವು ಸಾಧುಸಂತರನ್ನು ಕಂಡಿದ್ದರು. ಬಹಳ ಮುಂದುವರಿದ ಆಧ್ಯಾತ್ಮಿಕಜೀವಿಗಳು ಸಾಧಾರಣ ಜನರಿಗೆ ಗೊಚರವಾಗದ ರೀತಿ ಇರುತ್ತಿದ್ದರು. ಒಂದು ಸಲ ಗೊತ್ತಾದರೆ ಸಾಕು, ಹಗಲಿರುಳು ಜನ ಮುತ್ತಿ ಅವರ ಪ್ರಶಾಂತಜೀವನವನ್ನು ಹಾಳುಮಡುವರು. ಜೊತೆಗೆ ಹಲವು ಬಯಕೆಗಳನ್ನು ಇಟ್ಟುಕೊಂಡಿರುವ ಲೌಕಿಕ ವ್ಯಕ್ತಿಗಳು ಇಂತಹ ಸಾಧುಗಳನ್ನು ಮುತ್ತಿ ಇವರಿಂದ ಲೌಕಿಕ ಪ್ರಯೋಜನವನ್ನು ಪಡೆಯಲು ಆಶಿಸುವರು. ಆದಕಾರಣವೆ ಇಂಥವರಿಂದ ಪಾರಾಗಲು ಹಲವು ವಿಚಿತ್ರ ರೀತಿಯಲ್ಲಿ ಇಂತಹ ಸಾಧುಗಳು ವ್ಯವಹರಿಸುವರು. ಕೆಲವು ವೇಳೆ ಹುಚ್ಚರಂತೆ ತೋರುವರು. ಒಂದು ಸಲ ಹುಡುಗರಿಂದ ಕಲ್ಲಿನಿಂದ ಹೊಡೆಸಿಕೊಂಡ ಸಾಧು ಒಬ್ಬನನ್ನು ಸ್ವಾಮೀಜಿ ನೋಡಿದರು. ತಲೆಯಿಂದ ರಕ್ತ ಧಾರಾಕಾರವಾಗಿ ಸುರಿಯುತ್ತಿತ್ತು. ಆದರೂ ಹುಡುಗರು ತನ್ನನ್ನು ಹೊಡೆಯುತ್ತಿದ್ದುದನ್ನು ಒಂದು ತಮಾಷೆ ಎಂದು ಭಾವಿಸಿದ್ದ ಆತ. ಸ್ವಾಮೀಜಿ ಅವನ ಗಾಯವನ್ನು ತೊಳೆದು ಬಟ್ಟೆಯನ್ನು ಸುಟ್ಟು ಆ ಬೂದಿಯನ್ನು ಗಾಯದ ಮೇಲೆ ಇಟ್ಟು ಬ್ಯಾಂಡೇಜನ್ನು ಕಟ್ಟಿದರು. ಆತ ಆಗ “ದೇವರು ಹೀಗೆ ಆಡುವನು” ಎಂದನು. ಇನ್ನೊಬ್ಬ ಸಾಧು ಹೃಷೀಕೇಶದ ಹತ್ತಿರ ಇರುವ ಒಂದು ಗುಹೆಯಲ್ಲಿರುತ್ತಿದ್ದ. ಆತನ ಗುಹೆಯ ಸುತ್ತಲೂ ನರದೇಹದ ಮೂಳೆಗಳು ಬಿದ್ದಿರುತ್ತಿದ್ದವು. ಜನ ಇದನ್ನು ನೋಡಿ ಒಳಗಿರುವ ಸಾಧು ನರಮಾಂಸ ಭಕ್ಷಕನಿರಬೇಕು. ಅದಕ್ಕೇ ಈ ಮೂಳೆಗಳೆಲ್ಲ ಅವನ ಗುಹೆಯ ಹೊರಗೆ ಬಿದ್ದಿವೆ ಎಂದು ಭಾವಿಸಿ ಅವನ ಬಳಿಗೇ ಸುಳಿಯುತ್ತಿರಲಿಲ್ಲ. ಒಂದು ದಿನ ಸ್ವಾಮೀಜಿ ಆ ಗುಹೆಯ ಬಳಿಗೆ ಹೋಗಿ ಇದಕ್ಕೆ ಕಾರಣಗಳನ್ನು ಕೇಳಿದಾಗ, ಆ ಸಾಧು ಕೇವಲ ಕುತೂಹಲದಿಂದ ಪ್ರೇರಿತರಾಗಿ ಬರುವ ಜನರ ಕಾಟದಿಂದ ಪಾರಾಗುವುದಕ್ಕೆ ಈ ಉಪಾಯವನ್ನೆಲ್ಲ ಮಾಡಬೇಕಾಗಿದೆ ಎಂದ. 

ಸ್ವಾಮೀಜಿ ಚೆನ್ನಾಗಿ ಗುಣಮುಖರಾದ ಮೇಲೆ ತಾವು ಇನ್ನು ಮೇಲೆ ಒಬ್ಬರೇ ಹೋಗುವೆನೆಂದೂ, ಯಾರೂ ತಮ್ಮೊಡನೆ ಬರಕೂಡದು ಎಂದೂ ಹೇಳಿದರು. ಅಖಂಡಾನಂದರು ತಾವಾದರೂ ಜೊತೆಗೆ ಇರುತ್ತೇವೆ ಎಂದಾಗ, ಸ್ವಾಮೀಜಿ “ಇದೂ ಬಂಧನವೇ, ನೀನು ಖಾಯಿಲೆ ಬಿದ್ದರೆ ನಾನು ನೋಡಿಕೊಳ್ಳಬೇಕಾಗುವುದು, ನಾನು ಖಾಯಿಲೆ ಬಿದ್ದರೆ ನೀನು ನೋಡಿಕೊಳ್ಳಬೇಕಾಗುವುದು” ಎಂದು ಹೇಳಿ ೧೮೯೧ನೇ ಇಸವಿ ಜನವರಿಯ ಅಂತ್ಯದಲ್ಲಿ ಸ್ವಾಮೀಜಿ ಒಬ್ಬರೇ ದೆಹಲಿಯ ಕಡೆಗೆ ಹೊರಟರು. 

