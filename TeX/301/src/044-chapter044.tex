
\chapter{ಅಮರನಾಥ ಸ್ಮೃತಿಗಳು ಮತ್ತು ಮೃಗಾಲಯ }

 ಸ್ವಾಮೀಜಿ ಬೇಲೂರಮಠದಲ್ಲಿದ್ದಾಗ ಶ್ವಾಸಕೋಶ ವ್ಯಾಧಿಯಿಂದ ನರಳುತ್ತಿದ್ದರು.\break ಕಲ್ಕತ್ತೆಯ ಪ್ರಖ್ಯಾತ ವೈದ್ಯರಾದ ಆರ್. ಎಲ್. ದತ್ತ ಎಂಬುವರು ಸ್ವಾಮೀಜಿ ಅವರನ್ನು ಪರೀಕ್ಷಿಸಿ ವಿಶ್ರಾಂತಿಯನ್ನು ತೆಗೆದುಕೊಳ್ಳಬೇಕೆಂದು ಸೂಚಿಸಿದರು. ಸ್ವಾಮೀಜಿಯವರ ಎಡಗಣ್ಣಿನಲ್ಲಿ ಒಂದು ರಕ್ತದ ಕಲೆಯೂ ಇತ್ತು ಮತ್ತು ಅದು ಕೆಂಪಾಗಿ ಊದಿಕೊಂಡಿತ್ತು. ಬಹುಶಃ ತೀವ್ರ ಧ್ಯಾನದಿಂದ ಹಾಗೆ ಆಗಿರಬಹುದೆಂದು ಭಾವಿಸಿದರು. ಒಂದು ದಿನ ಶರತ್‍ಚಂದ್ರ ಸ್ವಾಮೀಜಿಯವರನ್ನು ನೋಡುವುದಕ್ಕೆ ಬಂದ. ಆಶ್ರಮದವರು ಆತನಿಗೆ ಸ್ವಾಮೀಜಿ ಹತ್ತಿರ ಮಾತನಾಡಿ ಅವರ ಮನಸ್ಸನ್ನು ಕೆಳಗೆ ಇಳಿಸಬೇಕೆಂದು ಹೇಳಿ ಕಳುಹಿಸಿದರು. 

\vskip 1pt

 ಶಿಷ್ಯನು ಮೇಲೆ ಸ್ವಾಮೀಜಿ ಕೊಠಡಿಗೆ ಹೋಗಿ ನೋಡಿದನು. ಸ್ವಾಮೀಜಿ ಮುಕ್ತ ಪದ್ಮಾಸನದಲ್ಲಿ ಪೂರ್ವಮುಖವಾಗಿ ಗಂಭೀರ ಧ್ಯಾನಮಗ್ನರಾಗಿರುವಂತೆ ಕಂಡುಬಂದರು. ಮುಖದಲ್ಲಿ ನಗುವಿಲ್ಲ. ಪ್ರದೀಪ್ತ ನಯನಗಳಲ್ಲಿ ಬಹಿರ್ಮುಖ ದೃಷ್ಟಿಯಿಲ್ಲ. ಒಳಗೆ ಏನೇನೋ ನೋಡುತ್ತಿದ್ದಂತೆ ತೋರುತ್ತಿತ್ತು. ಶಿಷ್ಯನನ್ನು ನೋಡಿದ ಕೂಡಲೇ “ಬಾರಯ್ಯ ಕುಳಿತುಕೊ” ಎಂದರು. ಸ್ವಾಮಿಗಳ ಎಡಗಣ್ಣಿನ ಒಳಭಾಗ ಕೆಂಪೇರಿದುದನ್ನು ನೋಡಿ ಶಿಷ್ಯನು “ತಮ್ಮ ಕಣ್ಣಿನ ಒಳಭಾಗವೇಕೆ ಕೆಂಪೇರಿದೆ” ಎಂದನು. ಸ್ವಾಮೀಜಿ “ಅದೇನೂ ಆಗಿಲ್ಲ” ಎಂದು ಹೇಳಿ ಸ್ಥಿರವಾಗಿ ಕುಳಿತುಕೊಂಡರು. ಬಹಳ ಹೊತ್ತು ಕುಳಿತುಕೊಂಡರೂ ಸ್ವಾಮೀಜಿ ಯಾವ ಮಾತನ್ನೂ ಆಡಲಿಲ್ಲ. ಆಗ ಶಿಷ್ಯನು ಕಳವಳಗೊಂಡು ಸ್ವಾಮೀಜಿ ಪಾದಪದ್ಮಗಳನ್ನು ಮುಟ್ಟಿ “ಅಮರನಾಥದಲ್ಲಿ ಏನೇನು ನೋಡಿದಿರೊ ಅದನ್ನು ಹೇಳುವುದಿಲ್ಲವೆ?” ಎಂದು ಕೇಳಿದನು. ಪಾದಸ್ಪರ್ಶದಿಂದ ಸ್ವಾಮೀಜಿ ಸ್ವಲ್ಪ ಚಕಿತರಾದರು. ಸ್ವಲ್ಪ ಬಹಿರ್ದೃಷ್ಟಿಯಾದಂತೆ ತೋರಿತು. “ಅಮರನಾಥ ದರ್ಶನಾರಭ್ಯ ನನ್ನ ಮನಸ್ಸಿನಲ್ಲಿ ಇಪ್ಪತ್ತನಾಲ್ಕು ಗಂಟೆ ಹೊತ್ತು ಶಿವ ತುಂಬಿಕೊಂಡಂತಿದೆ. ಏನು ಮಾಡಿದರೂ ಮನಸ್ಸು ಕೆಳಗೆ ಇಳಿದು ಬರುವಂತಿಲ್ಲ” ಎಂದರು. 

\vskip 1pt

 ಸ್ವಾಮೀಜಿ: “ಅಮರನಾಥದಲ್ಲಿಯೂ ಆಮೇಲೆ ಕ್ಷೀರಭವಾನಿ ಮಂದಿರದಲ್ಲಿಯೂ ತುಂಬಾ ತಪಸ್ಸು ಮಾಡಿದೆ. ಹೋಗಿ ತಂಬಾಕನ್ನು ಸಿದ್ಧಮಾಡಿಕೊಂಡು ಬಾ.” 

\vskip 1pt

 ಶಿಷ್ಯನು ಪ್ರಪುಲ್ಲವಾದ ಮನಸ್ಸಿನಿಂದ ಸ್ವಾಮೀಜಿ ಆಜ್ಞೆಯನ್ನು ಶಿರಸಾವಹಿಸಿ ತಂಬಾಕನ್ನು ಸಿದ್ಧಪಡಿಸಿ ಕೊಟ್ಟನು. ಸ್ವಾಮೀಜಿ ಮೆಲ್ಲಮೆಲ್ಲನೆ ಧೂಮಪಾನ ಮಾಡುತ್ತ ಹೇಳಿದರು: “ಅಮರನಾಥಕ್ಕೆ ಹೋಗುವಾಗ ಪರ್ವತದ ಒಂದು ಕಡಿದಾದ ತಿಟ್ಟನ್ನು ಹತ್ತಿಹೋದೆ. ಆ ಮಾರ್ಗದಲ್ಲಿ ಯಾತ್ರಿಕರು ಯಾರೂ ಹೋಗುವುದಿಲ್ಲ. ಬೆಟ್ಟದ ಜನ ಮಾತ್ರ ಅಲ್ಲಿ ಹೋಗಿ ಬಂದು ಮಾಡುತ್ತಾರೆ. ನನಗೇನೋ ಆ ದಾರಿಯಲ್ಲೇ ಹೋಗಬೇಕೆಂಬ ಹಟ ಹಿಡಿಯಿತು. ಹಾಗೆಯೇ ಹೊರಟುಹೋದೆ. ಆ ಪ್ರಯತ್ನದಿಂದ ಶರೀರ ಸ್ವಲ್ಪ ಆಯಾಸಗೊಂಡಿದೆ. ಅಲ್ಲಿ ಎಷ್ಟು ಕೊರೆತ ಎಂದರೆ ಮೈಯಲ್ಲಿ ಸೂಜಿ ಚುಚ್ಚಿದಂತಾಗುತ್ತಿತ್ತು.” 

\vskip 1pt

 ಶಿಷ್ಯ: “ಬೆತ್ತಲೆಯಾಗಿ ಹೋಗಿ ಅಮರನಾಥನ ದರ್ಶನವನ್ನು ಮಾಡಬೇಕೆಂದು ಕೇಳಿದ್ದೇನೆ. ಆ ಸಂಗತಿ ನಿಜವೆ?” 

\vskip 1pt

 ಸ್ವಾಮೀಜಿ: “ಹೌದು. ನಾನು ಕೌಪೀನ ಮಾತ್ರ ಹಾಕಿಕೊಂಡು ಬೂದಿ ಬಳಿದುಕೊಂಡು ಹೋದೆ. ಆಗ ಛಳಿ ಸೆಖೆ ಯಾವುದೂ ತೋರಲಿಲ್ಲ. ಆದರೆ ಮಂದಿರದಿಂದ ಹೊರಗೆ ಬಂದಮೇಲೆ ಛಳಿಯಿಂದ ಜಡನಾಗಿ ಹೋದೆ.” 

\newpage

 ಶಿಷ್ಯ: “ಪಾರಿವಾಳ ಹಕ್ಕಿಗಳನ್ನು ನೋಡಿದಿರೇನು? ಅಲ್ಲಿ ಛಳಿಯಿಂದ ಯಾವ ಜೀವ ಜಂತುಗಳೂ ವಾಸಮಾಡುವಂತೆ ಕಾಣಬರುವುದಿಲ್ಲವೆಂದೂ ಕೇವಲ ಬಿಳಿಯ ಪಾರಿವಾಳ ಪಕ್ಷಿಗಳ ಒಂದು ಹಿಂಡು ಎಲ್ಲಿಂದಲೋ ಆಗಿಂದಾಗ್ಗೆ ಬರುತ್ತಿರುತ್ತವೆ ಎಂದೂ ಕೇಳಿದ್ದೇನೆ.” 

 ಸ್ವಾಮೀಜಿ: “ಹೌದು. ಮೂರು ನಾಲ್ಕು ಬಿಳಿಯ ಪಾರಿವಾಳಗಳನ್ನು ನೋಡಿದೆ. ಅವು ಗುಹೆಯಲ್ಲಿರುವುವೋ ಅಥವಾ ಹತ್ತಿರದಲ್ಲಿದ್ದ ಪರ್ವತಗಳಲ್ಲಿರುವುವೋ‌ ತಿಳಿದುಕೊಳ್ಳಲಾಗಲಿಲ್ಲ.” 

 ಶಿಷ್ಯ: “ಮಹಾರಾಜ್, ಗುಹೆಯಿಂದ ಹೊರಕ್ಕೆ ಬಂದು ಬಿಳಿಯ ಪಾರಿವಾಳಗಳನ್ನು ನೋಡಿದರೆ ನಿಜವಾಗಿ ಶಿವನ ದರ್ಶನವಾಯಿತೆಂದು ಅರ್ಥ, ಎಂದು ಜನರು ಹೇಳುತ್ತಾರೆ.” 

 ಸ್ವಾಮೀಜಿ: “ಪಾರಿವಾಳಗಳನ್ನು ನೋಡಿದರೆ ಯಾವ ಉದ್ದೇಶದಿಂದ ಅವನು ಹೋಗಿರುವನೊ ಅದು ಈಡೇರುತ್ತದೆ ಎಂದು ಕೇಳಿದ್ದೇನೆ.” 

 ಶಿಷ್ಯನು ಅನಂತರ ಪ್ರೇತಾತ್ಮಗಳ ಪ್ರಸ್ತಾಪವನ್ನು ಎತ್ತಿ, “ಮಹಾರಾಜ್, ನಾವು ಭೂತ ಪ್ರೇತಗಳ ವಿಷಯವನ್ನು ಕೇಳಿದ್ದೇವಲ್ಲ, ಶಾಸ್ತ್ರದಲ್ಲಿಯೂ ಮೇಲಿಂದ ಮೇಲೆ ಅದರ ಸಮರ್ಥನೆ ಕಂಡು ಬರುತ್ತದೆಯಲ್ಲ, ಅದೆಲ್ಲ ನಿಜವಾಗಿರುವುದೇ?” ಎಂದನು. 

 ಸ್ವಾಮೀಜಿ: “ನಿಜವಲ್ಲದೆ ಮತ್ತೇನು? ನೀನು ಯಾವುದನ್ನು ನೋಡಿಲ್ಲವೋ ಅದು ನಿಜವಲ್ಲವೇನು? ನಿನ್ನ ದೃಷ್ಟಿಯಿಂದಾಚೆಗೆ ಎಷ್ಟು ಬ್ರಹ್ಮಾಂಡಗಳು ಎಷ್ಟು\break ದೂರದಲ್ಲಿ ತಿರುಗುತ್ತಿವೆ. ನೀನು ನೋಡಲಾರದಿದ್ದರಿಂದ ಅವುಗಳೇ ಇಲ್ಲವೇನು? ಆದರೆ ಈ ಭೂತಗಳ ವಿಚಾರಕ್ಕೆ ಮನಸ್ಸು ಕೊಡಬೇಡ. ಭೂತ ಪ್ರೇತಗಳೇನೊ ಇವೆ ಎಂದು ಇಟ್ಟುಕೊ. ನಿನ್ನ ಕೆಲಸ ಏನೆಂದರೆ ಈ ಶರೀರ ಮಧ್ಯದಲ್ಲಿರುವ ಆತ್ಮವನ್ನು ಪ್ರತ್ಯಕ್ಷಮಾಡಿಕೊಳ್ಳುವುದು. ಅದನ್ನು ಪ್ರತ್ಯಕ್ಷಮಾಡಿಕೊಳ್ಳಬಲ್ಲೆಯಾದರೆ ಭೂತ ಪ್ರೇತಗಳು ನಿನ್ನ ದಾಸಾನುದಾಸರಾಗಿ ಬಿಡುವುವು.” 

 ಶಿಷ್ಯ: “ಆದರೆ ಅವುಗಳನ್ನು ನೋಡಿದರೆ ಪುನರ್ಜನ್ಮಾದಿಗಳಲ್ಲಿ ತುಂಬಾ ನಂಬಿಕೆಯುಂಟಾಗಿ ಪರಲೋಕದ ಅಪನಂಬಿಕೆ ತಪ್ಪಿಹೋಗುವುದೆಂದು ತೋರುತ್ತದೆ.” 

 ಸ್ವಾಮೀಜಿ: “ನೀವೋ ಮಹಾಶೂರರು! ನಿಮ್ಮಂಥವರು ಭೂತ ಪ್ರೇತಗಳನ್ನು ನೋಡಿ ಪರಲೋಕದಲ್ಲಿ ದೃಢವಾದ ನಂಬಿಕೆಯನ್ನು ಇಟ್ಟುಕೊಳ್ಳಬೇಕೇನು? ಇಷ್ಟು ಶಾಸ್ತ್ರ ವಿಜ್ಞಾನವನ್ನು ಓದಿದ್ದೀರಿ. ಈ ವಿರಾಟ್ ವಿಶ್ವದ ಎಷ್ಟೋ ಗೂಢತಮ ತತ್ತ್ವಗಳನ್ನು ತಿಳಿದುಕೊಂಡಿದ್ದೀರಿ. ಇದರ ಮೇಲೆ ಆತ್ಮಜ್ಞಾನವನ್ನು ಪಡೆಯಬೇಕಾದರೆ ಭೂತಪ್ರೇತಗಳನ್ನೂ ನೋಡಬೇಕೆ? ಛಿ, ಛಿ.” 

 ಶಿಷ್ಯ: “ಒಳ್ಳೆಯದು, ನೀವು ಯಾವಾಗಲಾದರೂ ಭೂತಪ್ರೇತಗಳನ್ನು ಕಣ್ಣಾರೆ ನೋಡಿದ್ದೀರಾ?” ಸ್ವಾಮೀಜಿ ಹೌದೆಂದು ಹೇಳಿ ತಮಗೆ ಮದ್ರಾಸಿನಲ್ಲಿ ಆದ ಅನುಭವವನ್ನು ವಿವರಿಸಿದರು. 

\newpage

 ಶಿಷ್ಯನು ಈಗ ಶ್ರಾದ್ಧಾದಿಗಳಿಂದ ಪ್ರೇತಾತ್ಮಕ್ಕೆ ತೃಪ್ತಿಯಾಗುತ್ತದೆಯೆ ಇಲ್ಲವೆ ಎಂದು ಕೇಳಲು ಸ್ವಾಮೀಜಿ ಅದು ಸ್ವಲ್ಪವೂ ಅಸಂಭವವಲ್ಲ ಎಂದರು. 

 ಸ್ವಾಮೀಜಿ ತಮ್ಮ ರೋಗದ ಚಿಕಿತ್ಸೆಗಾಗಿ ಬಹಳ ದಿನಗಳು ಕಲ್ಕತ್ತೆಯಲ್ಲಿ ಇರಬೇಕಾಗಿತ್ತು. ಅಲ್ಲಿ ಇರುವವರೆಗೆ ಬೆಳಗಿನಿಂದ ಸಾಯಂಕಾಲದವರೆಗೆ ಸ್ವಾಮೀಜಿಯವರನ್ನು ನೋಡುವ ತಂಡ ಯಾವಾಗಲೂ ಕಾದುಕೊಂಡಿರುತ್ತಿತ್ತು. ಒಬ್ಬರು ಹೋದರೆ ಮತ್ತೊಬ್ಬರು ಬರುತ್ತಿದ್ದರು. ಹೊತ್ತಿಗೆ ಸರಿಯಾಗಿ ಊಟವನ್ನು ಕೂಡ ಮಾಡುವುದಕ್ಕೆ ಆಗುತ್ತಿರಲಿಲ್ಲ. ಅವರ ಸೇವಕರು ಪ್ರೇಕ್ಷಕರಿಗೆ ನೋಡುವುದಕ್ಕೆ ಒಂದು ಕಾಲವನ್ನು ಗೊತ್ತುಮಾಡಿ, ಆ ಸಮಯದಲ್ಲಿ ಮಾತ್ರ ನೋಡಿ ಎಂದರು. ಆದರೆ ಸ್ವಾಮೀಜಿ ‘ಅಷ್ಟು ದೂರದಿಂದ ನೋಡುವುದಕ್ಕೆ ಬರುತ್ತಾರೆ. ಅವರೊಡನೆ ಒಂದು ಮಾತನ್ನೂ ಆಡದೆ, ಇದರಿಂದ ನನ್ನ ಆರೋಗ್ಯಕ್ಕೆ ಭಂಗವಾಗುವುದೆಂದು ಅವರನ್ನು ಹೇಗೆ ಕಳುಹಿಸಿಬಿಡುವುದು?’ ಎಂದರು. 

 ಸ್ವಾಮೀಜಿ ಹೃದಯ ದಯೆಯಿಂದ ಕರಗುತ್ತಿತ್ತು. ಭ್ರಷ್ಟವೃಕ್ತಿಗಳು ಬಂದು ಸ್ವಾಮೀಜಿಯವರನ್ನು ಉಪದೇಶ ಕೇಳಿದರೆ ಅವರು ಕೊಟ್ಟು ಕಳುಹಿಸುತ್ತಿದ್ದರು. ಇದರಿಂದ, ಬಹುಶಃ ಸ್ವಾಮೀಜಿಗೆ ತಮ್ಮ ಬಳಿಗೆ ಬರುವವರ ಜೀವನ ತಿಳಿದಿಲ್ಲ ಎಂದು ಕೆಲವರು ಭಾವಿಸತೊಡಗಿದರು. ಅನೇಕ ಅಯೋಗ್ಯರಿಗೆ ಅನುಮಾನಾಸ್ಪದವಾದ ಶೀಲದವರಿಗೂ ಉಪದೇಶ ದೊರಕಿತು. ಇದನ್ನು ನೋಡಿ ಸ್ವಾಮೀಜಿಯ ಕೆಲವು ಶಿಷ್ಯರು ಹೀಗೆ ಏತಕ್ಕೆ ನೀವು ಕೊಡುತ್ತೀರಿ ಎಂದು ಕೇಳಿದರು. ಅದಕ್ಕೆ ಸ್ವಾಮೀಜಿ ಹೀಗೆ ಹೇಳಿದರು: “ಮಗು, ನನಗೆ ಅವರ ವಿಷಯ ಗೊತ್ತಿಲ್ಲ ಎಂದು ಭಾವಿಸುವೆಯಾ? ನಾನು ಒಬ್ಬನನ್ನು ನೋಡಿದರೆ ಅವನ ಈಗಿನ ಸ್ಥಿತಿ ಮಾತ್ರವಲ್ಲ, ಅವನ ಹಿಂದಿನದೂ ನನಗೆ ಗೊತ್ತಾಗುವುದು. ಅವನ ಈಗಿನ ಮನಸ್ಸಿನ ಅಂತರಾಳದಲ್ಲಿ ಏನಾಗುತ್ತಿದೆಯೆಂಬುದೂ ಗೊತ್ತಾಗುವುದು. ಪಾಪ ಬಡಜೀವಿಗಳು ಸ್ವಲ್ಪ ಶಾಂತಿಗಾಗಿ ಮನೆಮನೆಯನ್ನು ಅಲೆದಿವೆ. ಎಲ್ಲರೂ ಅವರನ್ನು ನಿರಾಕರಿಸಿರುವರು. ಅವರು ಕೊನೆಗೆ ನನ್ನ ಬಳಿಗೆ ಬಂದಿರುವರು. ನಾನೂ ಅವರನ್ನು ನಿರಾಕರಿಸಿದರೆ ಅವರು ಎಲ್ಲಿಗೆ ಹೋಗುವರು? ಆದಕಾರಣವೇ ನಾನು ಜೊಳ್ಳು ಕಾಳು ಇವುಗಳನ್ನು ಪ್ರತ್ಯೇಕಿಸುವುದಿಲ್ಲ. ಅಯ್ಯೋ, ಅವರು ಎಷ್ಟು ದುಃಖಕ್ಕೆ ಸಿಕ್ಕಿರುವರು. ಈ ಪ್ರಪಂಚ ದುಃಖಮಯ!” 

 ಸೋದರಿ ನಿವೇದಿತಾ ಕಲ್ಕತ್ತೆಯಲ್ಲಿ ಬಾಲಕಿಯರಿಗೆ ಒಂದು ಶಾಲೆಯನ್ನು ತೆರೆದಮೇಲೆ ಅಲ್ಲಿಯೇ ಇದ್ದುಕೊಂಡು ಕಷ್ಟವಾದ ಜೀವನವನ್ನು ನಡೆಸುತ್ತಿದ್ದಳು. ಅವರ ಮಠಕ್ಕೆ ಬಂದಾಗಲೆಲ್ಲ ಸ್ವಾಮೀಜಿಯವರಿಗೆ ತಾನೆ ಏನನ್ನಾದರೂ ಮಾಡಿ ತಿನ್ನುವುದಕ್ಕೆ ಕೊಡುತ್ತಿದ್ದಳು. ನಿವೇದಿತಾ ಕೈಯಿಂದಲೇ ಏನನ್ನಾದರೂ ಮಾಡಿಸಿ ಸ್ವಾಮೀಜಿ ತಾವು ತಿಂದು ಆಕೆಗೂ ತಿನ್ನುವಂತೆ ಹೇಳುತ್ತಿದ್ದರು. ಆಗ ಮಾಡಿದುದನ್ನು ಗುರುಭಾಯಿಗಳು ಮತ್ತು ಶಿಷ್ಯರೆಲ್ಲರೂ ತೆಗೆದುಕೊಳ್ಳುವಂತೆ ಮಾಡುತ್ತಿದ್ದರು. ಸೋದರಿ ನಿವೇದಿತೆಯನ್ನು ಮಠದವರೆಲ್ಲ ಒಬ್ಬ ಸೋದರಿಯನ್ನಾಗಿ ತೆಗೆದುಕೊಳ್ಳುವುದಕ್ಕೆ ಎಲ್ಲಾ\break ಅವಕಾಶಗಳನ್ನೂ ಮಾಡಿಕೊಡುತ್ತಿದ್ದರು. ಹೊರಗೆ ಹಿಂದೂ‌ ಸಮಾಜವು ಸೋದರಿ ನಿವೇದಿತಾಳನ್ನು ತಮ್ಮ ಧರ್ಮಕ್ಕೆ ಸಂಸ್ಕೃತಿಗೆ ಸೇರಿದವಳು ಎಂದು ಒಪ್ಪಿಕೊಳ್ಳುವಂತೆ ಮಾಡಲು ಸ್ವಾಮೀಜಿ ಹಲವು ಪ್ರಯತ್ನಗಳನ್ನು ಮಾಡಿದರು. 

 ಒಂದು ದಿನ ಸ್ವಾಮೀಜಿಯವರು ಯೋಗಾನಂದರು, ಸೋದರಿ ನಿವೇದಿತಾ ಮತ್ತು ಶರತ್‍ಚಂದ್ರರನ್ನು ಕರೆದುಕೊಂಡು ಕಲ್ಕತ್ತೆಯ ಮೃಗಾಲಯಕ್ಕೆ ಹೋದರು. ಮೃಗಾಲಯದ ಸೂಪರಿಂಟೆಂಡೆಂಟ್ ಆದ ರಾಮಬ್ರಹ್ಮ ಸನ್ಯಾಲರು ಸ್ವಾಮೀಜಿಯವರನ್ನು ಆದರದಿಂದ ಬರಮಾಡಿಕೊಂಡು ಅವರನ್ನು ಮೃಗಾಲಯವನ್ನು ತೋರಿಸುವುದಕ್ಕೆ ಕರೆದುಕೊಂಡು ಹೋದರು. 

 ರಾಮಬ್ರಹ್ಮ ಬಾಬುಗಳು ಸಸ್ಯ ಮತ್ತು ಪ್ರಾಣಿ ಶಾಸ್ತ್ರದಲ್ಲಿ ಒಳ್ಳೆಯ ಪಂಡಿತರಾಗಿದ್ದರು. ಆದ್ದರಿಂದ ಆ ತೋಟದಲ್ಲಿದ್ದ ನಾನಾ ವೃಕ್ಷಗಳನ್ನು ತೋರಿಸುತ್ತ ಸಸ್ಯ ಶಾಸ್ತ್ರಾನುಸಾರವಾಗಿ ಮರಗಿಡಗಳಲ್ಲಿ ಕಾಲಕ್ರಮದಿಂದ ಎಂಥ ಕ್ರಮಪರಿಣತಿ ಉಂಟಾಗುವುದೆಂಬುದನ್ನು ವಿವರಿಸುತ್ತ ಹೋಗುತ್ತಿದ್ದರು. ನಾನಾ ಜೀವಜಂತುಗಳನ್ನು ನೋಡುತ್ತ ಮಧ್ಯೆಮಧ್ಯೆ ಜೀವನದ ಉತ್ತರೋತ್ತರ ಪರಿಣತಿಯ ಸಂಬಂಧವಾಗಿ ಡಾರ್ವಿನ್ ಮತವನ್ನು ವಿಮರ್ಶಿಸುತ್ತಿದ್ದರು. ಅವರು ಹಾವಿನ ಮನೆಗೆ ಬಂದು ಅಲ್ಲಿ ಚಕ್ರಾಕಾರವಾಗಿ ಸುತ್ತಿಕೊಂಡಿದ್ದ ಒಂದು ಘಟಸರ್ಪವನ್ನು ತೋರಿಸಿ “ಇದರಿಂದಲೇ ಆಮೇಲೆ ಆಮೆ ಬಂದದ್ದು. ಸರ್ಪ ಒಂದೇ ಕಡೆ ಬಹಳ ಕಾಲ ಬಿದ್ದಿದ್ದು ಬಿದ್ದಿದ್ದು ಬೆನ್ನುಳ್ಳದ್ದಾಗಿದೆ” ಎಂದು ಹೇಳಿದರು. ಈ ಮಾತನ್ನು ಕೇಳಿ ಸ್ವಾಮೀಜಿ ಶಿಷ್ಯನನ್ನು ಕುರಿತು “ನೀವು ಆಮೆಯನ್ನು ತಿನ್ನುತ್ತೀರಲ್ಲವೆ? ಡಾರ್ವಿನ್ ಮತಾನುಸಾರವಾಗಿ ಈ ಹಾವೇ ಕಾಲಕ್ರಮೇಣ ಆಮೆಯಾಗಿ ಪರಿಣತವಾಗಿದೆ. ಆದ್ದರಿಂದ ನೀವು ಹಾವನ್ನೂ ತಿಂದ ಹಾಗಾಯಿತು!” ಎಂದು ಹಾಸ್ಯ ಮಾಡಿದರು. 

 ಸಿಂಹ ಹುಲಿಗಳಿಗೆ ಉಣಿಸುವುದನ್ನು ನೋಡಬೇಕೆಂದು ಸ್ವಾಮೀಜಿ ಅಪೇಕ್ಷಿಸಿದರು. ರಾಮಬ್ರಹ್ಮ ಬಾಬುಗಳ ಅಪ್ಪಣೆಯಂತೆ ಕಾವಲಿನವರು ಬೇಕಾದಷ್ಟು ಮಾಂಸವನ್ನು ತಂದು ಸ್ವ್ವಾಮೀಜಿ ಎದುರಿಗೆ ಹುಲಿಸಿಂಹಗಳಿಗೆ ಕೊಡಲು ಮೊದಲು ಮಾಡಿದರು. ಅವು ಸಂತೋಷದಿಂದ ಗರ್ಜಿಸುವುದನ್ನು, ಆತುರದಿಂದ ತಿನ್ನುವುದನ್ನು ನೋಡಿ ಸ್ವಾಮೀಜಿ ಆನಂದಿಸಿದರು. ಸ್ವಲ್ಪ ಹೊತ್ತಿನಮೇಲೆ ರಾಮಬ್ರಹ್ಮ ಬಾಬುಗಳ ಮನೆಗೆ ಎಲ್ಲರೂ ಬಂದರು. ಅಲ್ಲಿ ಉಪಾಹಾರದ ನಂತರ ಮಾತುಕತೆ ಮೊದಲಾಯಿತು: 

 ರಾಮಬ್ರಹ್ಮಬಾಬು: “ಡಾರ್ವಿನ್ನನ ಕ್ರಮವಿಕಾಸ ಮತ್ತು ಅದಕ್ಕೆ ಕಾರಣ ಇವುಗಳ ಸಂಬಂಧವಾಗಿ ನಿಮ್ಮ ಅಭಿಪ್ರಾಯವೇನು?” 

 ಸ್ವಾಮೀಜಿ: “ಡಾರ್ವಿನ್ ಮಾತು ನಿಜವಾದರೂ ಕ್ರಮವಿಕಾಸದ ಕಾರಣ ಸಂಬಂಧವಾಗಿ ಆತನು ಹೇಳಿರುವುದನ್ನು ಚರಮಸಿದ್ಧಾಂತವೆಂದು ನಾವು ಒಪ್ಪುವುದಿಲ್ಲ.” 

 ರಾಮಬ್ರಹ್ಮಬಾಬು: “ಈ ವಿಷಯದಲ್ಲಿ ನಮ್ಮ ದೇಶದ ಪ್ರಾಚೀನ ಪಂಡಿತರು ಏನಾದರೂ ವಿಚಾರ ಮಾಡಿದ್ದಾರೆಯೆ?” 

 ಸ್ವಾಮೀಜಿ: “ಸಾಂಖ್ಯದರ್ಶನದಲ್ಲಿ ಈ ವಿಷಯ ಸೊಗಸಾಗಿ ವಿಚಾರ ಮಾಡಲ್ಪಟ್ಟಿದೆ. ಭರತಖಂಡದ ಪ್ರಾಚೀನ ದಾರ್ಶನಿಕರ ಸಿದ್ಧಾಂತವೇ ಕ್ರಮವಿಕಾಸದ ಕಾರಣ ಸಂಬಂಧವಾಗಿ ಚರಮಸಿದ್ಧಾಂತವೆಂಬುದು ನನ್ನ ಅಭಿಪ್ರಾಯ.” 

 ರಾಮಬ್ರಹ್ಮಬಾಬು: “ಈ ಸಿದ್ಧಾಂತವನ್ನು ಸಂಕ್ಷೇಪವಾಗಿ ತಿಳಿಸಿದರೆ ಕೇಳಬೇಕೆಂದು ನನಗೆ ಆಸೆ.” 

 ಸ್ವಾಮೀಜಿ: “ನಿಮ್ಮ ಜಾತಿಯನ್ನು ಉಚ್ಚ ಜಾತಿಯನ್ನಾಗಿ ಪರಿಣತಗೊಳಿಸುವುದಕ್ಕೆ ಪಾಶ್ಚಾತ್ಯರು ಹೇಳುವ ‘ಜೀವನ ಸಂಗ್ರಾಮ’, ‘ಯೋಗ್ಯತಮವಾದದ್ದು ಮಾತ್ರ ಉಳಿದುಕೊಳ್ಳುವುದು’, ‘ಪ್ರಕೃತಿಯ ಆಯ್ಕೆ’ ಮುಂತಾದ ನಿಯಮಗಳೂ ಕಾರಣಗಳೂ ತಮಗೆ ಗೊತ್ತೇ ಇವೆ. ಪತಂಜಲಿ ದರ್ಶನದಲ್ಲಿ ಮಾತ್ರ ಇದೊಂದು ಕಾರಣವೆಂದು ತೋರಿಸಿಲ್ಲ. ಪತಂಜಲಿ ಮತವೇನೆಂದರೆ ಒಂದು ಜಾತಿಯಿಂದ ಮತ್ತೊಂದು ಜಾತಿಗೆ ಪರಿಣತಿಯು “ಪ್ರಕೃತಿಯ ಆಪೂರಣದಿಂದ ಆಗುವುದು.” ಆವರಣದೊಡನೆ ಹಗಲುರಾತ್ರಿ ಹೋರಾಡಿ ಅದು ಆಗುವುದು ಎಂದಲ್ಲ. ನನ್ನ ಅಭಿಪ್ರಾಯದಂತೆ ಜೀವಕ್ಕೆ ಪೂರ್ಣತೆ ಉಂಟಾಗುವುದರಲ್ಲಿ ಹೋರಾಟ ಮತ್ತು ಸ್ಪರ್ಧೆ ಅನೇಕ ವೇಳೆ ಪ್ರತಿಬಂಧಕವಾಗಿ ನಿಲ್ಲುತ್ತವೆ. ಪಾಶ್ಚಾತ್ಯ ದರ್ಶನಗಳಲ್ಲಿ ಹೇಳಿರುವಂತೆ ಸಾವಿರ ಜೀವಗಳು ಧ್ವಂಸವಾಗಿ ಒಂದು ಜೀವನದ ಕ್ರಮೋನ್ನತಿ ಆಗುವುದಾದರೆ, ಈ ಕ್ರಮ ವಿಕಾಸದಿಂದ ಸಂಸಾರದಲ್ಲಿ ವಿಶೇಷವಾದ ಯಾವುದೊಂದು ಉನ್ನತಿಯೂ ಆಗುವುದಿಲ್ಲವೆಂದು ಹೇಳಬೇಕಾಗುತ್ತದೆ. ಸಾಂಸಾರಿಕ ಉನ್ನತಿಯ ವಿಚಾರದಲ್ಲಿ ಇದನ್ನು ಒಪ್ಪಿಕೊಂಡರೂ ಆಧ್ಯಾತ್ಮಿಕ ವಿಕಾಸಕ್ಕೆ ಅದು ತುಂಬ ಪ್ರತಿಬಂಧಕವೆಂಬುದನ್ನು ಒಪ್ಪಿಕೊಳ್ಳಲೇಬೇಕು. ನಮ್ಮ ದೇಶದ ದಾರ್ಶನಿಕರ ಅಭಿಪ್ರಾಯವೇನೆಂದರೆ, ಜೀವನು ಪರಿಪೂರ್ಣಾತ್ಮ, ಪ್ರತಿಬಂಧಕಗಳಿಂದಾಗಿ ಈ ಪರಿಪೂರ್ಣತೆ ಅಭಿವ್ಯಕ್ತವಾಗುತ್ತಿಲ್ಲ. ಈ ಪ್ರತಿಬಂಧಕಗಳನ್ನು ಪೂರ್ತಿ ತೆಗೆದುಹಾಕಿಬಿಟ್ಟರೆ, ಪೂರ್ಣವಾಗಿ ಆತ್ಮ ಪ್ರಕಾಶವಾಗುವುದು. ಧ್ಯಾನಧಾರಣ, ಮತ್ತು ಮುಖ್ಯವಾಗಿ ತ್ಯಾಗ ಇವುಗಳ ಮೂಲಕ ಪ್ರತಿಬಂಧಕಗಳು ಸರಿದುಹೋಗುವುವು ಅಥವಾ ಆತ್ಮವು ಹೆಚ್ಚು ಪ್ರಕಾಶಿತವಾಗುವುದು. ಆದ್ದರಿಂದ ಪ್ರತಿಬಂಧಕಗಳ ನಿವಾರಣೆ ಆತ್ಮಪ್ರಕಾಶಕ್ಕೆ ನೇರ ಕಾರಣವಲ್ಲ, ಅದು ಸ್ವಯಂಪ್ರಕಾಶ. ಸಾವಿರ ಪಾಪಿಗಳನ್ನು ಸಂಹಾರಮಾಡಿ ಜಗತ್ತಿನಿಂದ ಪಾಪವನ್ನು ಹೋಗಲಾಡಿಸುವ ಪ್ರಯತ್ನದಿಂದ ಜಗತ್ತಿನಲ್ಲಿ ಪಾಪವು ಇನ್ನೂ ಹೆಚ್ಚಾಗುತ್ತದೆ. ಆದರೆ ಉಪದೇಶದ ಮೂಲಕ ಜೀವನವನ್ನು ಪಾಪದಿಂದ ತಪ್ಪಿಸುವುದಾದರೆ ಆಮೇಲೆ ಪಾಪ ನಿಲ್ಲುವುದು. ಈಗ ನೋಡಿ, ಪಾಶ್ಚಾತ್ಯರ ಜೀವಗಳ ಪರಸ್ಪರ ಸಂಗ್ರಾಮ ಮತ್ತು ಪೈಪೋಟಿಯ ಮೂಲಕ ಉತ್ತಮ ಸ್ಥಿತಿಯನ್ನು ಪಡೆಯುವ ಮತವು ಎಷ್ಟು ಭಯಂಕರವಾಗಿ ಪರಿಣಮಿಸಿದೆ?” 

 ಸಂಭಾಷಣೆ ಆದಮೇಲೆ ಮೃಗಾಲಯದ ಅಧಿಕಾರಿಗಳು ಸ್ವಾಮೀಜಿ ಮತ್ತು ಅವರ ವೃಂದವನ್ನು ಬೀಳ್ಕೊಟ್ಟರು. ದಾರಿಯಲ್ಲಿ ಹೋಗುತ್ತಿದ್ದಾಗ ಶಿಷ್ಯ ಸ್ವಾಮೀಜಿ ಅವರನ್ನು ಹೀಗೆ ಪ್ರಶ್ನಿಸಿದ: 

 ಶಿಷ್ಯ: “ತಾವೇ ಮಿಕ್ಕ ಕಡೆ ನನಗೆ ಹೇಳಿದ್ದೀರಿ. ಹೊರಗಿನ ಶಕ್ತಿಗಳೊಡನೆ ಯುದ್ಧ ಮಾಡುವುದಕ್ಕೆ ತಕ್ಕ ಸಾಮರ್ಥ್ಯವೆ ಜೀವನದ ಚಿಹ್ನೆ. ಅದೇ ಉನ್ನತಿಗೆ ಸೋಪಾನ ಎಂದು. ಈ ದಿವಸ ಅದಕ್ಕೆ ವಿರುದ್ಧವಾಗಿ ಹೇಳಿದಂತೆ ಇತ್ತು.” 

 ಸ್ವಾಮೀಜಿ: “ವಿರೋಧವಾಗಿ ಏಕೆ ಹೇಳಲಿ? ನೀನೇ ತಿಳಿದುಕೊಳ್ಳಲಾರದೆಹೋದೆ. ಪ್ರಾಣಿಜಗತ್ತಿನಲ್ಲಿ ನಾವು ನಿಜವಾಗಿಯೂ ಜೀವನ ಸ್ಪರ್ಧೆ, ಯೋಗ್ಯತಮವೇ ಉಳಿಯುವುದು ಮುಂತಾದ ನಿಯಮಗಳನ್ನು ಸ್ಪಷ್ಟವಾಗಿ ನೋಡಬಹುದು. ಅದರಿಂದಲೇ ಡಾರ್ವಿನ್ನನ ಮತ ಕೆಲವು ಮಟ್ಟಿಗೆ ನಿಜವೆಂದು ತೋರುತ್ತದೆ. ಆದರೆ ಬುದ್ಧಿಯ ವಿಕಾಸವುಳ್ಳ ಮನುಷ್ಯ ಜಗತ್ತಿನಲ್ಲಿ ಈ ನಿಯಮದ ವಿರೋಧವೇ ಕಂಡುಬರುತ್ತದೆ. ಇದನ್ನು ಯೋಚಿಸಿ ನೋಡು. ಯಾರು ನಾವು ನಿಜವಾಗಿ ಮಹಾತ್ಮರು ಆದರ್ಶ ಪುರುಷರೆಂದು ಬಲ್ಲೆವೊ ಅವರಲ್ಲಿ ಬಾಹ್ಯ ಹೋರಾಟ ಸುತರಾಂ ಇರುವುದಿಲ್ಲ. ಮನುಷ್ಯೇತರ ಪ್ರಾಣಿ ಜಗತ್ತಿನಲ್ಲಿ ಜ್ಞಾನದ ಪ್ರಾಬಲ್ಯ ಸ್ವಾಭಾವಿಕ. ಮನುಷ್ಯನಾದರೂ ಎಷ್ಟೆಷ್ಟು ದೊಡ್ಡವನಾದರೆ ಅಷ್ಟಷ್ಟು ಅವನಲ್ಲಿ ಬುದ್ಧಿಯ ವಿಕಾಸ. ಆದುದರಿಂದ ಪ್ರಾಣಿಗಳಂತೆ ಮಾನವ ಪ್ರಪಂಚದಲ್ಲಿಯೂ ಧ್ವಂಸಮಾಡಿ ಅದರ ಮೂಲಕ ಉನ್ನತಿ ಆಗುವುದು ಸಾಧ್ಯವಿಲ್ಲ. ಮಾನವನ ಸರ್ವಶ್ರೇಷ್ಠ ಪೂರ್ಣವಿಕಾಸ ತ್ಯಾಗ ಒಂದರಿಂದಲೇ ಸಾಧ್ಯ. ಯಾರು ಅನ್ಯರಿಗೋಸ್ಕರ ಎಷ್ಟೆಷ್ಟು ತ್ಯಾಗ ಮಾಡಬಲ್ಲರೋ ಅವರು ಮನುಷ್ಯರಲ್ಲಿ ಅಷ್ಟಷ್ಟು ಶ್ರೇಷ್ಠರಾಗುವರು. ಕೆಳಗಿನ ಮೆಟ್ಟಲಾದ ಪ್ರಾಣಿಜಗತ್ತಿನಲ್ಲಿ ಯಾವುದು ಎಷ್ಟನ್ನು ನಾಶಪಡಿಸಬಲ್ಲದೊ ಅದು ಅಷ್ಟು ಬಲಶಾಲಿಯಾದ ಮೃಗವಾಗುವುದು. ಆದುದರಿಂದ ಜೀವನ ಸಂಗ್ರಾಮ ಮತವು ಈ ಎರಡು ಪ್ರಪಂಚದಲ್ಲಿಯೂ ಒಂದೇ ಉಪಯೋಗವುಳ್ಳದ್ದು ಆಗಲಾರದು. ಮನುಷ್ಯನ ಸಂಗ್ರಾಮ ಆಗುವುದು ಮನಸ್ಸಿನಲ್ಲಿ. ಮನಸ್ಸನ್ನು ಯಾರು ಎಷ್ಟು ಸಂಯಮ ಮಾಡಿಕೊಳ್ಳುತ್ತಾರೋ ಅವರು ಅಷ್ಟಷ್ಟು ದೊಡ್ಡವರಾಗುತ್ತಾರೆ. ಮನಸ್ಸಿನ ವೃತ್ತಿಗಳೆಲ್ಲವೂ ಪೂರ್ತಿಯಾಗಿ ನಿರುದ್ಧವಾದರೆ ಆತ್ಮದ ವಿಕಾಸ ಉಂಟಾಗುವುದು. ಮಾನವೇತರ ಪ್ರಾಣಿ ಜಗತ್ತಿನಲ್ಲಿ ಸ್ಥೂಲದೇಹವನ್ನು ಕಾಪಾಡಿಕೊಳ್ಳುವುದಕ್ಕಾಗಿ ಸಂಗ್ರಾಮ ಕಂಡುಬರುತ್ತದೆ. ಮಾನವ ಜಗತ್ತಿನಲ್ಲಿ ಮನಸ್ಸಿನ ಮೇಲೆ ಆಧಿಪತ್ಯವನ್ನು ನಡೆಸುವುದಕ್ಕೋಸ್ಕರ ಅಥವಾ ತಾತ್ತ್ವಿಕ ವೃತ್ತಿಯನ್ನು ಪಡೆಯುವುದಕ್ಕೋಸ್ಕರ ಸಂಗ್ರಾಮ ನಡೆಯುವುದು. ಜೀವನದ ವೃಕ್ಷ ಅಥವಾ ಕೊಳದ ನೀರಿನಲ್ಲಿ ಬಿದ್ದ ವೃಕ್ಷಚ್ಛಾಯೆ ಇವುಗಳ ಹಾಗೆ. ಮನುಷ್ಯೇತರ ಪ್ರಾಣಿ ಜಗತ್ತು ಮನುಷ್ಯ ಜಗತ್ತು ಇವುಗಳಲ್ಲಿ ಸಂಗ್ರಾಮ ವ್ಯತ್ಯಾಸವಾಗುತ್ತದೆ.”‌ 

 ಶಿಷ್ಯ: “ಹಾಗಾದರೆ ನೀವು ಶರೀರವನ್ನು ಬಲವಾಗಿ ಮಾಡಬೇಕೆಂದು ಏತಕ್ಕೆ ಹೇಳಿದ್ದು?” 

 ಸ್ವಾಮೀಜಿ: “ನೀವೇನು ಮನುಷ್ಯರೆ? ಎಲ್ಲೋ ಸಲ್ಪ ಬುದ್ಧಿ ಇದೆ ಅಷ್ಟೆ. ದೇಹ ಚೆನ್ನಾಗಿಲ್ಲದೆ ಇದ್ದರೆ ಮನಸ್ಸಿನೊಡನೆ ಸಂಗ್ರಾಮ ಮಾಡುವುದು ಹೇಗೆ? ನೀವೇನು ಜಗತ್ತಿನ ಪೂರ್ಣವಿಕಾಸ ಎನಿಸಿಕೊಳ್ಳುವ ಮನುಷ್ಯರೆಂದು ಭಾವಿಸುವಿರೇನು? ಆಹಾರ, ನಿದ್ರೆ, ಮೈಥುನ ಇವುಗಳನ್ನು ಬಿಟ್ಟು ನಿಮಗೆ ಮತ್ತೇನಿದೆ? ಇನ್ನೂ ನಾಲ್ಕು ಕಾಲಿನಲ್ಲಿ ನಡೆಯುತ್ತಿಲ್ಲವಲ್ಲ ಅದೇ ಪುಣ್ಯ. ಪರಮಹಂಸರು ‘ಯಾರಿಗೆ ಮಾನದ ಮೇಲೆ ಜ್ಞಾನವಿದೆಯೋ ಅವನೇ ಮನುಷ್ಯ’ ಎಂದು ಹೇಳುತ್ತಿದ್ದರು. ನೀವೋ ‘ಹುಟ್ಟುವುದು ಸಾಯುವುದು’ ಎಂಬ ವಾಕ್ಯಗಳಿಗೆ ಸಾಕ್ಷೀಭೂತರಾಗಿ ಸ್ವದೇಶವಾಸಿಗಳಿಗೆ ಹಿಂಸಾಜನಕರೂ, ವಿದೇಶಿಗಳ ಅಗೌರವಕ್ಕೆ ಪಾತ್ರರೂ ಆಗಿದ್ದೀರಿ. ನೀವು ಮಾನವೇತರ ಪ್ರಾಣಿಗಳ ಮಧ್ಯದಲ್ಲಿರುವಿರಿ. ಅದಕ್ಕಾಗಿಯೇ ಸಂಗ್ರಾಮ ಮಾಡಿರೆಂದು, ಸಿದ್ಧಾಂತ ಎಲ್ಲಾ ಕಟ್ಟಿ ಇಡಿ ಎಂದು ಹೇಳುವುದು. ನಿಮ್ಮ ನಿತ್ಯಗಟ್ಟಲೆಯ ಕೆಲಸವನ್ನೂ ನಡವಳಿಕೆಯನ್ನೂ ಸ್ಥಿರಬುದ್ಧಿಯಿಂದ ವಿವೇಚನೆ ಮಾಡಿ ನೋಡಿ. ನೀವು ಮಾನವ ಮತ್ತು ಮಾನವೇತರ ಪ್ರಾಣಿಗಳ ಮಧ್ಯದಲ್ಲಿರುವ ಜೀವ ವಿಶೇಷಗಳು ಹೌದೋ ಅಲ್ಲವೋ ಎಂದು. ದೇಹವನ್ನು ಮೊದಲು ದೃಢಪಡಿಸಿ. ಆಮೇಲೆ ಮನಸ್ಸಿನ ಆಧಿಪತ್ಯ ಬರುತ್ತದೆ. \textbf{‘ನಾಯಮಾತ್ಮಾ ಬಲಹೀನೇನ ಲಭ್ಯಃ’} ತಿಳಿಯಿತೆ?” 

