
\chapter{ಸಿಂಹಳ ದ್ವೀಪದಲ್ಲಿ}

 ಸ್ವಾಮೀಜಿ ಬರುವುದನ್ನೇ ಸಿಂಹಳ ಮತ್ತು ಇಂಡಿಯಾದೇಶದ ಜನರೆಲ್ಲ ಎದುರು\break ನೋಡುತ್ತಿದ್ದರು. ಕಳೆದ ಮೂರು ವರ್ಷಗಳಿಂದ, ಚಿಕಾಗೋ ನಗರದಲ್ಲಿ ಸನಾತನ ಧರ್ಮದ ಧ್ವಜಪತಾಕೆಯನ್ನು ಹರಡಿದ ದಿನಗಳಿಂದ ಭರತಖಂಡದ ಪತ್ರಿಕೆಗಳು ಮತ್ತು ಜನರು ಸ್ವಾಮೀಜಿಯನ್ನು ಕೊಂಡಾಡುತ್ತಿದ್ದರು. ಏಕೆಂದರೆ ಈ ಸಂನ್ಯಾಸಿಯೊಬ್ಬರು ಪ್ರಪಂಚದ ದೃಷ್ಟಿ ಭರತಖಂಡದ ಕಡೆಗೆ ತಿರುಗುವಂತೆ ಮಾಡಿದರು. ಭರತಖಂಡದ ಸನಾತನ ತತ್ತ್ವಗಳಿಗೆ ಪಾಶ್ಚಾತ್ಯ ದೇಶಗಳಿಂದಲೂ ಗೌರವವನ್ನು ತಂದರು. ಭರತಖಂಡದಿಂದ ತತ್ತ್ವ ಮತ್ತು ಆಧ್ಯಾತ್ಮಿಕ ವಿಷಯಗಳನ್ನು ಕಲಿತುಕೊಳ್ಳಬೇಕು, ಅವರಿಗೆ ಅದನ್ನು ಕೊಡುವುದಲ್ಲ ಎಂಬ ಭಾವ ಪಾಶ್ಚಾತ್ಯರಲ್ಲಿ ಮೂಡುವಂತೆ ಮಾಡಿದರು. ಇಂಡಿಯಾ ದೇಶವನ್ನು ಜಗದ ಮರವಿನ ಕಸದ ಬುಟ್ಟಿಯಿಂದ ಕೀರ್ತಿಯ ಶಿಖರಕ್ಕೆ ಒಯ್ದರು. ಇದೆಲ್ಲ ಒಂದು ವ್ಯಕ್ತಿಯ ಏಕಾಂಗ ಸಾಹಸ. ಅವರ ಹಿಂದೆ ನಿಂತು ಪ್ರೋತ್ಸಾಹಿಸುವ ಈಗಿನಂತಹ ಸ್ವತಂತ್ರ ಭರತಖಂಡವಿರಲಿಲ್ಲ ಆಗ. ದಾಸ್ಯ ಮತ್ತು ಬಡತನದಲ್ಲಿ ಬಿದ್ದಿದ್ದ ಭರತಖಂಡದಿಂದ ಹೋಗಿ ಇಂತಹ ಗೌರವವನ್ನು ಸಂಪಾದಿಸಿಕೊಂಡು ಬರಬೇಕಾದರೆ ಆ ವ್ಯಕ್ತಿಯಲ್ಲಿ ಈಶ್ವರನ ಅಂಶವೇ ಇರಬೇಕು. ಸುಪ್ತವಾಗಿದ್ದ ಸನಾತನ ಭರತಖಂಡದ ಸತ್ವವೇ ಸ್ವಾಮೀಜಿ ಮೂಲಕ ವ್ಯಕ್ತವಾಗುತ್ತಿದೆ. ಇಂತಹ ಒಂದು ವ್ಯಕ್ತಿಯನ್ನು ಎಂದು ನೋಡುವೆವು, ಪಾಶ್ಚಾತ್ಯ ಪ್ರಪಂಚವನ್ನು ತಮ್ಮ ವಾಗ್ಧಾರೆಯಿಂದ ಮುಗ್ಧರನ್ನಾಗಿ ಮಾಡಿದ ಸ್ವಾಮೀಜಿಯವರನ್ನು ಎಂದು ಕೇಳುವೆವು ಎಂದು ಜನ ಕಾತರಿಸುತ್ತಿದ್ದರು. ಸ್ವಾಮೀಜಿಯವರು ಆಗ ಕೊಲಂಬೊ ನಗರದಿಂದ ಕಲ್ಕತ್ತೆಗೆ ಸೇರುವವರೆಗೆ ಕೊಟ್ಟ ಉಪನ್ಯಾಸಗಳು ಪ್ರಪಂಚದ ಶ್ರೇಷ್ಠ ಉಪನ್ಯಾಸಗಳ ಗುಂಪಿಗೆ ಸೇರಿವೆ. ಇಂಗ್ಲೀಷ್ ಭಾಷೆ ಅವರಿಂದ ಇನ್ನೂ ವೈಭವಯುಕ್ತವಾಯಿತು ಎಂದು ಬೇಕಾದರೂ ಹೇಳಬಹುದು. ಆ ಭಾಷೆಯ ಒಳಗಿರುವ ಭಾವವಾದರೋ‌ ಬೆಂಕಿಯಂತಹ ಭಾವಗಳು. ತಾಕಿದೆದೆಯನ್ನು ಹತ್ತಿಸುವಂತಹವು. ಸ್ವಾಮೀಜಿ ದಯಾ ದಾಕ್ಷಿಣ್ಯವಿಲ್ಲದೆ ನಮ್ಮ ಲೋಪದೋಷಗಳನ್ನು ಎತ್ತಿ ತೋರಿರುವರು. ಜೊತೆಗೆ ನಮ್ಮಲ್ಲಿರುವ ಒಳ್ಳೆಯದನ್ನು ನಿರ್ವಂಚನೆಯಿಂದ ಹೊಗಳಿರುವರು. ನ್ಯೂನತೆಗೆ ಛೀಮಾರಿ ಮಾಡಿರುವರು. ಭರತಖಂಡ ಅದ್ಭುತವಾದುದನ್ನು ಹಿಂದೆ ಸಾಧಿಸಿತು ಎಂಬ ಗೌರವವನ್ನು ನಮಗೆ ತೋರಿದರು. ಆದರೆ ಹಿಂದಿನದನ್ನೇ ಮೆಲುಕುತ್ತಿರಬೇಡಿ. ಅದನ್ನು ಮೀರಿದ ಭವಿಷ್ಯ ನಿಮಗೆ ಕಾದಿದೆ. ಅದನ್ನು ನಿರ್ಮಿಸಲು ಸೊಂಟಕಟ್ಟಿ\break ಎನ್ನುವರು. 

 ಸ್ವಾಮೀಜಿ ಆಗಿನ ಕಾಲದಲ್ಲಿ ಮಾಡಿದ ಉಪನ್ಯಾಸಗಳನ್ನೊಳಗೊಂಡ ‘ಕೊಲಂಬೊ ಇಂದ ಆಲ್ಮೋರಕೆ’ ಎಂಬ ಪುಸ್ತಕದ ಮುನ್ನುಡಿಯಲ್ಲಿ ಕರ್ನಾಟಕದ ಹೆಸರಾಂತ ಸಾಹಿತಿಗಳಾದ ಕುವೆಂಪು ಅವರು ಹೀಗೆ ಹೇಳುತ್ತಾರೆ: 

 “ದಕ್ಷಿಣೇಶ್ವರ ದೇವಮಾನವನ ಚಿತ್​ತಪಸ್ಸು ಸ್ವಾಮಿ ವಿವೇಕಾನಂದರ ಸಮುದ್ರ ಘೋಷ ಸ್ಪರ್ಧಿಯಾದ ವೀರವಾಣಿಯಲ್ಲಿ ವಾಗ್ ಮಂತ್ರವಾಗಿ ಹೊಮ್ಮಿ, ಭಾರತೀಯರ ಧಮನಿಧಮನಿಗಳಲ್ಲಿ ಧುಮುಧುಮುಕಿ ಹರಿದು, ಕ್ಲೈಬ್ಯವನ್ನು ಕೊಚ್ಚಿತು, ಧೈರ್ಯಧ್ವಜವನ್ನು ಎತ್ತಿತು, ಅಲ್ಪವನ್ನು ಕಿತ್ತು ಭೂಮವನ್ನು ನೆಟ್ಟಿತು, ಸಂಕುಚಿತ ಮನೋಭಾವನೆಯ ಗೋಡೆಗಳನ್ನು ಒಡೆದು ಮನಸ್ಸನ್ನು ಗಗನ ವಿಶಾಲವನ್ನಾಗಿ ಮಾಡಿತು. ಕುರಿಗಳಂತಿದ್ದವರನ್ನು ಸಿಂಹಗಳನ್ನಾಗಿ ಪರಿವರ್ತಿಸಿತು. “ಉತ್ತಿಷ್ಠತ, ಜಾಗ್ರತ, ಪ್ರಾಪ್ಯವರಾನ್ನಿಬೋಧತ” ಎಂಬ ಧೀರಮಂತ್ರ ದಶದಿಕ್ಕುಗಳಿಂದಲೂ ಶಕ್ತಿ ಸಂಚಾರಕವಾಗಿ ಮೊಳಗಿದರೆ ಅಧೈರ್ಯ, ಅಶಕ್ತಿಗಳಿಗೆ ಹುದುಗಿಕೊಳ್ಳುವುದಕ್ಕಾದರೂ ಜಾಗವಿರುತ್ತದೆಯೆ? 

 “ಸ್ವಾತಂತ್ರ್ಯಪೂರ್ವದ ಯಾವ ವಿಧವಾದ ರಾಷ್ಟ್ರೀಯ ಕಾರ್ಯಗಳಲ್ಲಿ ಭಾಗವಹಿಸಿದ ಯಾರನ್ನಾದರೂ ಕೇಳಿ ಗೊತ್ತಾಗುತ್ತದೆ; ಪ್ರತಿಯೊಬ್ಬರೂ ಹಿರಿಯ ಕಿರಿಯ ಪ್ರಸಿದ್ಧ ಅಪ್ರಸಿದ್ಧರೆಂಬ ಭೇದವಿಲ್ಲದೆ ಸ್ವಾಮಿ ವಿವೇಕಾನಂದರಿಗೆ ಋಣಿಗಳಾಗಿರುತ್ತಾರೆ. ಒಬ್ಬೊಬ್ಬರೂ ತಮ್ಮದೇ ಆದ ಒಂದಲ್ಲ ಒಂದು ರೀತಿಯಲ್ಲಿ ‘ಕೊಲಂಬೊದಿಂದ ಆಲ್ಮೋರಕೆ’ ಎಂಬ ಉಪನ್ಯಾಸಗಳಲ್ಲಿ ಕೇಂದ್ರೀಕೃತವಾಗಿರುವ ವಿದ್ಯುನ್ಮಯ ಭಾಷಣದ ಪರಂಪರೆಯ ಪ್ರಭಾವದಿಂದ ತಮ್ಮತಮ್ಮ ವ್ಯಕ್ತಿತ್ವವನ್ನು ರೂಪುಗೊಳಿಸಿಕೊಂಡವರಾಗಿರುತ್ತಾರೆ. ದೇಶಭಕ್ತಿಯ\break ಹೃದಯದಲ್ಲಿ ರಾಜಕೀಯ ವ್ಯಕ್ತಿಗಳ ಸಹವಾಸದ ಪ್ರಾಣಸ್ಥಾನದಲ್ಲಿ, ಸಮಾಜ ಸುಧಾರಕರ ವೀರೋದ್ಯಮದ ಅಂತರಾಳದಲ್ಲಿ, ಆರ್ಥಿಕ ಅಭ್ಯುದಯದ ಆಕಾಂಕ್ಷೆಯ ನಾಡಿನಲ್ಲಿ, ಧರ್ಮೋದ್ಧಾರದ ಪ್ರಯತ್ನದ ಧಮನಿಯಲ್ಲಿ, ಆಧ್ಯಾತ್ಮಿಕ ಅಭೀಪ್ಸೆಯ ಅಂತರತಮ ನಿಗೂಢಗಹ್ವರದಲ್ಲಿ, ಕಡೆಗೆ ಕಲಾವಿಜ್ಞಾನ ಸಾಹಿತ್ಯದ ಕ್ಷೇತ್ರದಲ್ಲಿ, ತಪಸ್ವಿಗಳಾದವರ ಮಹತ್ ಸಾಧನೆಯ ಮೂಲದಲ್ಲಿ ಎಲ್ಲೆಲ್ಲಿಯೂ ಸ್ವಾಮಿ ವಿವೇಕಾನಂದರ ಚಿತ್‍ತಪಸ್ಸು, ‘ಸ್ವಧಾ’ ಶಕ್ತಿಯಾಗಿ, ‘ಪ್ರಯತಿ’ ಶಕ್ತಿಯಾಗಿ ನೂಕುವ ಪ್ರೇರಣೆಯಾಗಿ ಅಧಿಕಾರ ಮಾಡುತ್ತಿರುವುದನ್ನು ಕಾಣುತ್ತೇವೆ. 

 “ಹಿಂದೂಧರ್ಮದ ವೈಶಾಲ್ಯದ ಮತ್ತು ವೇದಾಂತ ತತ್ತ್ವದ ಗಂಭೀರತೆಯ ಪರಿಚಯ ಸರ್ವಸಾಮಾನ್ಯವೂ, ಸರ್ವಸುಲಭಗ್ರಾಹ್ಯವೂ ಆಗಬೇಕಾದರೆ, ಸ್ವಾಮೀಜಿಯವರ\break ಉಪನ್ಯಾಸಗಳನ್ನು ಬಿಟ್ಟರೆ ಬೇರಾವುದೂ ಇಲ್ಲ. ಸಂಕುಚಿತ ಮತಭಾವಗಳನ್ನು ಕತ್ತರಿಸಿ ಕೆಡಹಿ ಮನಸ್ಸಿನಲ್ಲಿ ಉದಾತ್ತ ಸಮನ್ವಯ ದೃಷ್ಟಿಯನ್ನು ಪ್ರಜ್ವಲಿಸುವ ಶಕ್ತಿ ಈ ಉಪನ್ಯಾಸಗಳಲ್ಲಿ ಇರುವಂತೆ ಬೇರೆಲ್ಲಿಯೂ ಇಲ್ಲ. ಪತನ ಸಮಯದಲ್ಲಿ ನಮ್ಮನ್ನು ಕೈಹಿಡಿದೆತ್ತುವ ಔದಾರ್ಯ, ಹೃದಯ ದೌರ್ಬಲ್ಯದ ಸಮಯದಲ್ಲಿ ಕ್ಲೈಬ್ಯವನ್ನು ಕಿತ್ತೊಗೆದು ಕೆಚ್ಚನ್ನು ನೆಡುವ ಸಿಡಿಲಾಳ್ಮೆ ಈ ಭಾಷಣಗಳಲ್ಲಿ ಅನುಭವ ಪ್ರತ್ಯಕ್ಷವಾಗುವಂತೆ ಬೇರೆಲ್ಲಿಯೂ ಆಗುವುದಿಲ್ಲ. ಇಲ್ಲಿ ಬುದ್ಧಿಗೆ ಪುಷ್ಟಿ ಇದೆ, ಹೃದಯಕ್ಕೆ ತುಷ್ಟಿ ಇದೆ. ನಮ್ಮ ವ್ಯಕ್ತಿತ್ವ ಸಮಸ್ತವನ್ನೂ ಸರ್ವಾವಯವ ಸಂಪೂರ್ಣವಾಗಿ ವಿಕಾಸಗೊಳಿಸಿ ಪೂರ್ಣತೆಯ ಕಡೆಗೆ ನಮ್ಮನ್ನು ಕೊಂಡೊಯ್ಯುವ ಪೂರ್ಣದೃಷ್ಟಿಯೂ ಇಲ್ಲಿ ಸಿದ್ಧಿಸುತ್ತಿದೆ. ಇದು ಅಮೃತದ ಮಡು, ಮಿಂದು ಧನ್ಯರಾಗಿ! ಇದು ಜ್ಯೋತಿಯ ಖನಿ! ಹೊಕ್ಕು ಪ್ರಬುದ್ಧರಾಗಿ!” 

 ಸ್ವಾಮೀಜಿ ಇಡೀ ರಾಷ್ಟ್ರವನ್ನೇ ಜಾಗ್ರತಗೊಳಿಸುವುದಕ್ಕೆ, ಬೆಂಕಿಯಂತಹ ಮಾತಿನ ಸುರಿಮಳೆಯನ್ನು ಪ್ರಾರಂಭ ಮಾಡುವುದಕ್ಕೆ ಕೊಲಂಬೊ ನಗರಕ್ಕೆ ಬರುತ್ತಿರುವರು. ಇಲ್ಲಿಂದ ಅಗ್ನಿಶಿಖೆಯಂತಿರುವ ಅವರ ಬಾಯಿಂದ ಮಾತಿನ ಕಿಡಿಗಳು ಉದುರುವುದನ್ನು ನೋಡುವೆವು. 

 ಸ್ವಾಮೀಜಿ ೧೮೯೭ ಜನವರಿ ೧೫ನೇ ತಾರೀಖು ಮಧ್ಯಾಹ್ನದ ಮೇಲೆ ಕೊಲಂಬೊ ನಗರವನ್ನು ಮುಟ್ಟುತ್ತಾರೆ ಎಂಬ ಸುದ್ದಿ ಊರಿನಲ್ಲೆಲ್ಲ ಹರಡಿತು. ಸಹಸ್ರಾರು ಜನ ಸ್ವಾಮೀಜಿಯವರ ದರ್ಶನವನ್ನು ಮಾಡಬೇಕೆಂದು ಬಂದರಿನ ಸಮೀಪದಲ್ಲಿ ನೆರೆದಿದ್ದರು. ಊರಿಗೆ ಊರೇ ಸ್ವಾಮೀಜಿಯವರನ್ನು ಸ್ವಾಗತಿಸಲು ಸಿದ್ಧವಾಗಿತ್ತು. ಮುಖ್ಯ ಬೀದಿಗಳಲ್ಲಿ ತಳಿರು ತೋರಣಗಳನ್ನು ಕಟ್ಟಿದ್ದರು. ಸ್ವಾಮೀಜಿಯವರನ್ನು ಸ್ವಾಗತಿಸಿ ಬಿನ್ನವತ್ತಳೆಯನ್ನು ಅರ್ಪಿಸುವುದಕ್ಕಾಗಿ ಸುಂದರವಾಗಿ ಅಲಂಕೃತವಾದ ಹಸಿರು ಚಪ್ಪರ ಸಿದ್ಧವಾಗಿತ್ತು. ಮನೆ ಮನೆಗಳ ಮುಂದೆಯೂ ಸಾರಿಸಿ ರಂಗೋಲೆ ಇಟ್ಟು ದೀಪಗಳನ್ನು ಹತ್ತಿಸಿ ಪುಷ್ಪವೃಷ್ಟಿಯನ್ನು ಸ್ವಾಮೀಜಿ ಮೇಲೆ ಎರೆಯುವುದಕ್ಕೆ ಆಬಾಲವೃದ್ಧ ಸ್ತ್ರೀ ಪುರುಷರೆಲ್ಲರೂ ಸಿದ್ಧರಾಗಿದ್ದರು. ಇಂತಹ ವೈಭವದ ಸ್ವಾಗತವನ್ನು ಆ ನಗರ ಹಿಂದೆ ಎಂತಹ ಚಕ್ರವರ್ತಿಗೂ ಇತ್ತಿರಲಿಲ್ಲ, ಇಷ್ಟೊಂದು ಜನರ ಹೃದಯಗಳು ಒಂದು ವ್ಯಕ್ತಿಗೆ ಭಕ್ತಿ ಗೌರವಗಳಿಂದ ಕಾದಿರಲಿಲ್ಲ. ಸ್ವಾಮೀಜಿಯವರ ಗುರುಭಾಯಿಗಳಲ್ಲಿ ಆಗಲೇ ಕೆಲವರು ಅವರನ್ನು ಎದುರುಗೊಳ್ಳಲು ಕೊಲಂಬೊ ನಗರಕ್ಕೆ ಬಂದಿದ್ದರು. ಮದ್ರಾಸಿನಿಂದ ಕೆಲವು ಶಿಷ್ಯರು ಬಂದಿದ್ದರು. ಇಡೀ ಭರತಖಂಡದ ಜನರ ಕಣ್ಣು ಇತ್ತ ಹರಿದಿತ್ತು. ಅವರ ಸ್ವಾಗತವನ್ನು ಅಂದಿನ ಕಾಲದ ಒಂದು ಪತ್ರಿಕೆಯಿಂದಲೇ ಉದಾಹರಿಸುತ್ತೇವೆ. ಅದನ್ನು ಓದಿ ನಮ್ಮ ಮನಸ್ಸಿನಲ್ಲಿ ಆ ಚಿತ್ರವನ್ನು ಕಲ್ಪಿಸಿಕೊಳ್ಳಬಹುದು: 

 “ಕೊಲಂಬೊ ನಗರದ ಹಿಂದೂ ಜನರ ಇತಿಹಾಸದಲ್ಲಿ ಜನವರಿ ಹದಿನೈದನೆಯ ದಿನ ಚಿರಸ್ಮರಣೀಯವಾಗಿ ನಿಲ್ಲುವುದು. ಹಿಂದೂಗಳ ಅತ್ಯಂತ ಪವಿತ್ರವಾದ ಸಂನ್ಯಾಸಾಶ್ರಮಕ್ಕೆ ಸೇರಿದ, ಮಹಿಮಾನ್ವಿತರಾದ ಆಚಾರ್ಯ ವಿವೇಕಾನಂದರನ್ನು ಬರಮಾಡಿಕೊಂಡ ದಿನ ಅದು. ಅವರ ಬರವು ಚರಿತ್ರಾರ್ಹವಾದ ಘಟನೆ. ಅಭೂತಪೂರ್ವವಾದ ಆಧ್ಯಾತ್ಮಿಕ ಜಾಗ್ರತಿಯ ಶುಭ ಅರುಣೋದಯವನ್ನು ಅದು ಸಾರುತ್ತದೆ. 

 “ದಿನ ಕಳೆದು ರಾತ್ರಿ ಸಮಯ ಸನ್ನಿಹಿತವಾಗುತ್ತಿರುವಾಗ, ಹಿಂದೂಗಳಿಗೆ ಪವಿತ್ರವಾದ ಆ ಗೋಧೂಳಿಯ ಸಂಧ್ಯಾಕಾಲ, ಮುಂದೆ ಬರುವ ಪವಿತ್ರವಾದ ಘಟನೆಗಳನ್ನು ಸಾರುವಂತೆ ಇತ್ತು. ಆ ಸಮಯದಲ್ಲೆ ಗಂಭೀರಾಕೃತಿಯ ನಕ್ಷತ್ರಗಳಂತೆ ಹೊಳೆಯುತ್ತಿರುವ ಕಣ್ಣುಗಳಿಂದ ಕೂಡಿದ, ಗೈರಿಕವಸನಧಾರಿಗಳಾದ ಸ್ವಾಮಿ ವಿವೇಕಾನಂದರು ನಿರಂಜನಾನಂದರೊಡನೆ ಬಂದರು. ಅವರನ್ನು ಹೊತ್ತು ಬರುತ್ತಿದ್ದ ದೋಣಿ ಪೋರ್ಟಿನ ಕಡೆ ಬರುತ್ತಿರುವಾಗ ದೂರದಿಂದ ನೋಡುತ್ತಿದ್ದ ಸಹಸ್ರಾರು ಜನರ ಹೃದಯದ ಭಾವಗಳನ್ನು, ಅವರಿಗೆ ಸ್ವಾಮೀಜಿ ಮೇಲೆ ಇರುವ ಪ್ರೀತಿಯನ್ನು ಭಾಷೆ ಬಣ್ಣಿಸಲಾರದು. ಸ್ವಾಮೀಜಿ ದರ್ಶನ ಮಾಡಿದ ಜನರ ಜಯ ಜಯಕಾರಗಳು ಅವರು ಹರ್ಷೋತ್ಕರ್ಷದಿಂದ ಮಾಡುತ್ತಿದ್ದ ಕರತಾಡನದ ತುಮುಲ, ಸಮುದ್ರದ ಮೊರೆಯನ್ನು ಕೂಡ ಮುಳುಗಿಸಿಬಿಟ್ಟಿತ್ತು! ಆನರಬಲ್ ಶ‍್ರೀಕುಮಾರಸ್ವಾಮಿಯರು ತಮ್ಮ ತಮ್ಮನೊಡನೆ ಮುಂದೆ ಬಂದು ಸ್ವಾಮೀಜಿ ಕೊರಳಿಗೆ ಒಂದು ಸುಂದರವಾದ ಮಲ್ಲಿಗೆಯ ಹಾರವನ್ನು ಹಾಕಿ ಸ್ವಾಗತಿಸಿದರು. ಅನಂತರ ಸ್ವಾಮೀಜಿ ದರ್ಶನಕ್ಕೆ ಕಾತರವಾದ ಜನರ ನೂಕುನುಗ್ಗಲು… ಎಷ್ಟೇ ಪ್ರಯತ್ನಪಟ್ಟರೂ ಆ ಉಕ್ಕಿ ಬರುವ ಜನಸಂದಣಿಯನ್ನು ತಡೆಯಲು ಆಗಲಿಲ್ಲ… ಬಾರ್‍ನಸ್ ರೋಡಿನ ಪ್ರಾರಂಭದಲ್ಲಿ ಒಂದು ಸುಂದರವಾಗಿ ಅಲಂಕೃತವಾದ ಹಸಿರು ತೋರಣವಿತ್ತು. ಅದರ ಮುಂದೆ ಸ್ವಾಮೀಜಿಯವರನ್ನು ಸ್ವಾಗತಿಸುವ ವಾಕ್ಯಗಳು ಚಿತ್ರಿಸಲ್ಪಟ್ಟಿದ್ದವು. ಸ್ವಾಮೀಜಿ ನೆಲಕ್ಕೆ ಕಾಲಿಟ್ಟ ಒಡನೆಯೆ ಸುಂದರವಾದ ಎರಡು ಕುದುರೆಗಳಿಂದ ಎಳೆಯಲ್ಪಟ್ಟ ಅಲಂಕೃತವಾದ ಕೋಚ್‍ಗಾಡಿ ಸ್ವಾಮೀಜಿಯವರನ್ನು ಬಾರ್‍ನಸ್ ರೋಡಿನಲ್ಲಿರುವ ಚಪ್ಪರಕ್ಕೆ ಕರೆದುಕೊಂಡು ಹೋಯಿತು. ಸ್ವಾಮೀಜಿಯವರನ್ನು ಸ್ವಾಗತಿಸುವುದಕ್ಕೆ ಕಟ್ಟಲ್ಪಟ್ಟ ಸುಂದರವಾದ ವಿಶಾಲ ಚಪ್ಪರದೆಡೆಗೆ ಜನರೆಲ್ಲ ಧಾವಿಸುತ್ತಿದ್ದರು. ಇದ್ದ ಗಾಡಿಗಳನ್ನೆಲ್ಲ ಜನರು ಬಾಡಿಗೆಗೆ ತೆಗೆದುಕೊಂಡುಬಿಟ್ಟಿದರು, ಉಳಿದವರು ನಡೆದುಕೊಂಡು ಹೋದರು. ಅನಂತರ ಸ್ವಾಮೀಜಿ ಗಾಡಿಯಿಂದ ಇಳಿದು ಮೆರವಣಿಗೆಯಲ್ಲಿ ನಡೆಯುತ್ತಾ ಹೋದರು. ಛತ್ರಿ ಬಾವುಟಗಳನ್ನು ಮೆರವಣಿಗೆಯಲ್ಲಿ ಹಿಡಿದರು. ಸ್ವಾಮೀಜಿ ನಡೆಯುವುದಕ್ಕೆ ಮುಂದೆ ಬಟ್ಟೆಯನ್ನು ಹಾಸಿದರು.\break ನಾಗಸ್ವರದ ಮೇಳ ಜೊತೆಯಲ್ಲಿ. ಹಲವು ಜನ ಬಾರ್‍ನಸ್ ರೋಡಿನಲ್ಲಿ ಮೆರವಣಿಗೆಯಲ್ಲಿ ಭಾಗವಹಿಸಿದರು. ಸ್ವಾಮೀಜಿಯವರಿಗೆ ಇಳಿದುಕೊಳ್ಳುವುದಕ್ಕಾಗಿ ಒಂದು ಮನೆಯನ್ನು ಗೊತ್ತುಮಾಡಿದ್ದರು. ಅದರ ಮುಂದೆ ದೊಡ್ಡ ಚಪ್ಪರವನ್ನು ಹಾಕಿ ತಳಿರು\break ತೋರಣಗಳಿಂದ ಅಲಂಕರಿಸಿದ್ದರು. ಅದೇ ಇವರನ್ನು ಸ್ವಾಗತಿಸುವ ಚಪ್ಪರ. ಮೊದಲನೇ ಚಪ್ಪರದಿಂದ ಅವರು ಇಳಿದುಕೊಳ್ಳುವುದಕ್ಕೆ ಗೊತ್ತುಮಾಡಿರುವ ಮನೆಯ ಚಪ್ಪರದವರೆಗೆ ಸುಮಾರು ಕಾಲು ಮೈಲಿ ದಾರಿಯ ಎರಡು ಭಾಗಗಳಲ್ಲಿಯೂ ತೆಂಗು ಮಾವು ಬಾಳೆ ಮುಂತಾದ ಹಸಿರು ತಳಿರುಗಳಿಂದ ಅಲಂಕರಿಸಲಾಗಿತ್ತು. ಸ್ವಾಮೀಜಿಯವರು ಎರಡನೆ ಚಪ್ಪರದ ಒಳಗೆ ಪ್ರವೇಶ ಮಾಡಿದ ತಕ್ಷಣವೆ ಅಲಂಕೃತವಾದ ಒಂದು ಕೃತಕವಾದ ದೊಡ್ಡ ಕಮಲ ಅರಳಿತು. ಒಳಗಿನಿಂದ ಒಂದು ಹಕ್ಕಿ ಹಾರಿಹೋಯಿತು. ಆದರೆ ಜನರ ದೃಷ್ಟಿ ಚಪ್ಪರದ ಸೌಂದರ್ಯದ ಕಡೆ ಇರಲೇ ಇಲ್ಲ. ಎಲ್ಲರೂ ಸ್ವಾಮಿಗಳ ದರ್ಶನಾಕಾಂಕ್ಷಿಗಳಾಗಿದ್ದರು. ಅವರನ್ನು ನೋಡುವ ನೂಕುನುಗ್ಗಲಲ್ಲಿ ಚಪ್ಪರದ ಕೆಲವು ಅಲಂಕಾರಗಳು ನಾಶವಾದವು. ಸ್ವಾಮಿ ವಿವೇಕಾನಂದರು ಮತ್ತು ಅವರ ಶಿಷ್ಯರು ಪೀಠದ ಮೇಲೆ ಕುಳಿತುಕೊಂಡರು. ಜನರೆಲ್ಲ ಪುಷ್ಪವೃಷ್ಟಿಯನ್ನು ಅವರ ಮೇಲೆ ಕರೆದರು. ಗದ್ದಲವೆಲ್ಲ ನಿಂತಮೇಲೆ ಸಂಗೀತಗಾರರೊಬ್ಬರು ತಮ್ಮ ಪಿಟೀಲಿನಲ್ಲಿ ಒಂದು ಹಾಡನ್ನು ನುಡಿಸಿದರು. ಅನಂತರ ಎರಡು ಸಾವಿರ ವರ್ಷಗಳಷ್ಟು ಪುರಾತನವಾದ ತೇವಾರಂ ಹಾಡಿದರು. ಸ್ವಾಮೀಜಿಯವರ ಗೌರವಾರ್ಥವಾಗಿ ರಚಿಸಿದ ಒಂದು ಸಂಸ್ಕೃತ ಸ್ವಾಗತ ಗೀತೆಯನ್ನು ಹಾಡಿದರು. ಆನರಬಲ್ ಶ‍್ರೀ ಪಿ. ಕುಮಾರಸ್ವಾಮಿಯವರು ಎದ್ದು ಬಂದು ಸ್ವಾಮೀಜಿಯವರಿಗೆ ಸಾಷ್ಟಾಂಗ ಪ್ರಣಾಮ ಮಾಡಿ ಹಿಂದೂಗಳ ಪರವಾಗಿ ಅವರಿಗೆ ಒಂದು ಬಿನ್ನವತ್ತಳೆಯನ್ನು ಅರ್ಪಿಸಿದರು. ಅದರಲ್ಲಿ ಯೂರೋಪ್ ಮತ್ತು ಅಮೇರಿಕಾ ದೇಶಗಳಿಗೆ, ಎಲ್ಲಾ ಧರ್ಮಗಳನ್ನು ಸಮನ್ವಯ ದೃಷ್ಟಿಯಿಂದ\break ನೋಡುವ ಉದಾರವಾದ ಸನಾತನಧರ್ಮದ ಸಂದೇಶವನ್ನು ಬೋಧಿಸಿದುದಕ್ಕಾಗಿ\break ಸ್ವಾಮೀಜಿಯವರನ್ನು ಅಭಿನಂದಿಸಿದರು.” 

 ಸ್ವಾಮೀಜಿ ಉಚಿತವಾದ ರೀತಿಯಲ್ಲಿ ಅಭಿನಂದನೆಗೆ ಉತ್ತರವನ್ನು ಇತ್ತು ಹೀಗೆ ಹೇಳಿದರು: “ನೀವು ತೋರಿಸುವ ಗೌರವ ಒಬ್ಬ ದೊಡ್ಡ ರಾಜಕಾರಣಿಗಲ್ಲ, ದೊಡ್ಡ ಯೋಧನಿಗಲ್ಲ, ಕೋಟ್ಯಾಧೀಶನಿಗೂ ಅಲ್ಲ. ಇದು ಒಬ್ಬ ಭಿಕ್ಷೆಯನ್ನು ಬೇಡುವ ಸಂನ್ಯಾಸಿಗೆ. ಹಿಂದೂವಿಗೆ ಧರ್ಮದಲ್ಲಿರುವ ಶ್ರದ್ಧೆಯನ್ನು ಇದೇ ತೋರುವುದು. ಭರತಖಂಡ ಬದುಕಿರಬೇಕಾದರೆ ದರ್ಮವನ್ನೇ ಭರತಖಂಡದ ಬೆನ್ನುಮೂಳೆಯಾಗಿ ಉಳಿಸಿಕೊಂಡು ಬರಬೇಕು... ಜನರು ಕೊಡುವ ಗೌರವ ನನ್ನ ವ್ಯಕ್ತಿತ್ವಕ್ಕಲ್ಲ, ಇದು ನಮ್ಮ ಸನಾತನ ಧರ್ಮಕ್ಕೆ ಎಂದು ನಾನು ಭಾವಿಸುತ್ತೇನೆ” ಎಂದು ಹೇಳಿದರು. 

 ಸ್ವಾಮೀಜಿ ಅನಂತರ ವಿಶ್ರಾಂತಿಗೆ ಮನೆಯ ಒಳಗೆ ಹೋದರು. ಸರಿಯಾಗಿ ಅವರ ದರ್ಶನವನ್ನು ಪಡೆಯಲು ಅನೇಕರು ಅವರ ಮನೆಯನ್ನು ಸುತ್ತುಗಟ್ಟಿರುವುದನ್ನು ನೋಡಿ ಸ್ವಾಮೀಜಿ ಹೊರಗೆ ಬಂದು ದರ್ಶನವನ್ನು ಕೊಟ್ಟು “ನಮೋ ನಾರಾಯಣ” ಎಂದು ಎಲ್ಲರನ್ನೂ ಹರಸಿ ಹೋದರು. ಅನಂತರ ನೆರೆದ ಜನರು ಹಿಂದಿರುಗಿದರು. 

 ಸ್ವಾಮೀಜಿ ತಂಗಿದ್ದ ಮನೆಯನ್ನು \enginline{Vivekananda Lodge} ಎಂದು ಅನಂತರ ಕರೆಯತೊಡಗಿದರು. ಅನೇಕ ಜನ ಸ್ವಾಮೀಜಿಯವರನ್ನು ನೋಡುವುದಕ್ಕೆ ಬರುತ್ತಿದ್ದರು. ಅದೊಂದು ತೀರ್ಥಕ್ಷೇತ್ರವಾಯಿತು. ದೊಡ್ಡ ದೊಡ್ಡ ನೌಕರರಿಂದ ಹಿಡಿದು ಅತಿ ದರಿದ್ರನವರೆಗೆ ಅವರನ್ನು ನೋಡುವುದಕ್ಕೆ ಹೋಗುತ್ತಿದ್ದರು. ಒಬ್ಬ ಬಡ ಹೆಂಗಸು ಕೈನಲ್ಲಿ ಒಂದೆರಡು ಬಾಳೆಹಣ್ಣನ್ನು ಹಿಡಿದುಕೊಂಡು ಅವರನ್ನು ನೋಡಲು ಬಂದಳು. ಅವಳ ಗಂಡ ಸಂನ್ಯಾಸಿಯಾಗಿ ಹೊರಟುಹೋಗಿದ್ದ. ಆಕೆ ತಾನೂ ಕೂಡ ತನ್ನ ಗಂಡನಂತೆಯೇ ಜೀವನ ನಡೆಸಬೇಕೆಂದು ದೇವರ ವಿಷಯವಾಗಿ ತಿಳಿದುಕೊಳ್ಳಬೇಕೆಂದು ಸ್ವಾಮೀಜಿಯವರನ್ನು ಕೇಳಿದಳು. ಸ್ವಾಮೀಜಿ ಆಕೆಗೆ ಭಗವದ್ಗೀತೆಯನ್ನು ಪಾರಾಯಣಮಾಡಿ ತನ್ನ ಪಾಲಿನ ಕರ್ತವ್ಯವನ್ನು ಮಾಡುವಂತೆ ಹೇಳಿದರು. ಆ ಹೆಂಗಸು “ಗೀತೆಯನ್ನೇನೋ ನಾನು ಪಾರಾಯಣ ಮಾಡುತ್ತೇನೆ, ಅದು ಅರ್ಥವಾಗದೆ ಇದ್ದರೆ ಏನು ಪ್ರಯೋಜನ?” ಎಂದಳು. ಸ್ವಾಮೀಜಿ “ನಿನ್ನ ಪಾಲಿನ ಕರ್ತವ್ಯವನ್ನು ಫಲಾಪೇಕ್ಷೆಯಿಲ್ಲದೆ ಮಾಡುತ್ತಿದ್ದರೆ ಗೀತೆಯ ಅರ್ಥವನ್ನು ಕ್ರಮೇಣ ತಿಳಿಯಲು ಸಾಧ್ಯವಾಗುವುದು” ಎಂದು ಹೇಳಿದರು. 

 ಸ್ವಾಮೀಜಿ ಹದಿನಾರನೇ ತಾರೀಖು ಸಾಯಂಕಾಲ ಫ್ಲೋರಲ್‍ನಲ್ಲಿ ತಮ್ಮ ಮೊದಲನೆಯ ಬಹಿರಂಗ ಉಪನ್ಯಾಸವನ್ನು ಮಾಡಿದರು. ಅಲ್ಲಿ ಭರತಖಂಡದ ಆದರ್ಶವನ್ನು ವಿವರಿಸಿದರು. ಭರತಖಂಡದ ಪ್ರಾಣವೆಲ್ಲಿದೆ ಎಂಬುದನ್ನು ಹೇಳಿದರು. ಆ ಯತಿಯ ಹೃದಯದಲ್ಲಿ ಎಂತಹ ದೇಶಪ್ರೇಮ ಹರಿಯುತ್ತಿದೆ ಎಂಬುದಕ್ಕೆ ಅದು ಸಾಕ್ಷಿ. ಆ ಉಪನ್ಯಾಸದ ಒಂದೆರಡು ಪುಟಗಳನ್ನು ಓದುಗರ ಅವಗಾಹನೆಗೆ ಕೊಡುವೆವು. ಇಂಗ್ಲೀಷ್ ಭಾಷೆಯಲ್ಲಿ ಅದೊಂದು ಕಾವ್ಯಧಾರೆಯಂತೆ ಹರಿಯುತ್ತದೆ. ಅದರ ಕನ್ನಡ ಅನುವಾದದಲ್ಲಿ ಮೂಲದ ಶಕ್ತಿ ಸೌಂದರ್ಯಗಳನ್ನು ತರಲು ಕಷ್ಟವಾದರೂ ಅವರ ಭಾವಧಾರೆಯನ್ನು ಗಮನಿಸಬಹುದು. ಮೊದಲ ಉಪನ್ಯಾಸವೆ ಪವಿತ್ರಭೂಮಿ ಭರತಖಂಡವನ್ನು ಕುರಿತದ್ದು: 

 “ನಾನು ಮಾಡಿದ ಅಲ್ಪಕಾರ್ಯ ಕೇವಲ ನನ್ನಲ್ಲಿರುವ ಶಕ್ತಿಯಿಂದ ಮಾಡಿದುದಲ್ಲ. ನನ್ನ ಪರಮಪವಿತ್ರ ಪ್ರಿಯತಮ ಮಾತೃಭೂಮಿಯಿಂದ ಹೊರಟ ಉತ್ತೇಜನ ಶುಭಾಶಯ ಆಶೀರ್ವಾದಗಳೇ ಅದಕ್ಕೆ ಮೂಲ. ಪಾಶ್ಚಾತ್ಯ ದೇಶದಲ್ಲಿ ನಿಸ್ಸಂದೇಹವಾಗಿ ಸ್ವಲ್ಪ ಒಳ್ಳೆಯ ಕಾರ್ಯವೇನೊ ಆಗಿದೆ. ಆದರೆ ಸ್ವತಃ ಅದರಿಂದ ನನಗೇ ಪ್ರಯೋಜನವಾಗಿದೆ. ಯಾವುದು ಹಿಂದೆ ಕೇವಲ ಭಾವಪೂರ್ಣವಾಗಿತ್ತೊ ಅದು ಈಗ ಪ್ರಮಾಣಸಿದ್ಧ ಸತ್ಯದಂತೆ ತೋರುತ್ತಿದೆ. ಅದನ್ನು ಈಗ ಕಾರ‍್ಯ‍ರೂಪಕ್ಕೆ ತರುವುದಕ್ಕೆ ಶಕ್ತಿ ಬಂದಿದೆ. ಹಿಂದೆ ನಾನು ಎಲ್ಲಾ ಹಿಂದೂಗಳು ಭಾವಿಸುವಂತೆ, ನಿಮ್ಮ ಘನ ಅಧ್ಯಕ್ಷರು ಈಗತಾನೆ ಹೇಳಿದಂತೆ, ಭರತಖಂಡವನ್ನು ಪುಣ್ಯಭೂಮಿ ಕರ್ಮಭೂಮಿ ಎಂದು ಭಾವಿಸಿದ್ದೆ. ಇಂದು ನಾನು ನಿಮ್ಮ ಎದುರಿಗೆ ಸತ್ಯವಾಗಿ ಅದನ್ನು ಘಂಟಾಘೋಷದಿಂದ ಸಾರುತ್ತೇನೆ. ಪ್ರಪಂಚದಲ್ಲಿ ಯಾವುದಾದರೂ ಒಂದು ದೇಶ ಪುಣ್ಯಭೂಮಿ ಎಂದು ಕರೆಸಿಕೊಳ್ಳಲು ಅರ್ಹವಾಗಿದ್ದರೆ, ಜೀವಿಗಳು ತಮ್ಮ ಬಾಳಿನ ಕೊನೆಯ ಕರ್ಮವನ್ನು ಸವೆಯಿಸಲು ಬರಬೇಕಾದ ಸ್ಥಳವೊಂದಿದ್ದರೆ, ಭಗವಂತನೆಡೆಗೆ ಸಂಚರಿಸುತ್ತಿರುವ ಪ್ರತಿಯೊಂದು ಜೀವಿಯು ತನ್ನ ಕೊನೆಯ ಯಾತ್ರೆಯನ್ನು ಪೂರೈಸುವುದಕ್ಕೆ ಒಂದು ಕರ್ಮಭೂಮಿಗೆ ಬರಬೇಕಾಗಿದ್ದರೆ, ಯಾವುದಾದರೂ ದೇಶದಲ್ಲಿ ಮಾನವಕೋಟಿ ಮಾಧುರ್ಯ, ಔದಾರ್ಯ, ಪವಿತ್ರತೆ ಮತ್ತು ಶಾಂತಿ ಇವುಗಳಲ್ಲಿ ಮತ್ತು ಎಲ್ಲಕ್ಕಿಂತ ಹೆಚ್ಚಾಗಿ ಧ್ಯಾನದಲ್ಲಿ ಮತ್ತು ಅಂತರ್ಮುಖ ಜೀವನದಲ್ಲಿ ತನ್ನ ಪರಾಕಾಷ್ಠೆಯನ್ನು ಮುಟ್ಟಿದ್ದರೆ, ಅದು ಈ ಭರತಖಂಡವೇ ಆಗಿದೆ. ಅನಾದಿಕಾಲದಿಂದಲೂ ಇಲ್ಲಿಂದ ಧರ್ಮಸಂಸ್ಥಾಪನಾಚಾರ್ಯರು ಪೃಥ್ವಿಯನ್ನೆಲ್ಲ ತಮ್ಮ ಪವಿತ್ರವಾದ ಎಂದೆಂದಿಗೂ ಬತ್ತದ ಆಧ್ಯಾತ್ಮಿಕ ಮಹಾಸತ್ಯದ ಪ್ರವಾಹದಿಂದ ತೋಯಿಸಿರುವರು. ಇಲ್ಲಿಂದ ಎದ್ದ ಆಧ್ಯಾತ್ಮಿಕ ಮಹಾ ಪ್ರವಾಹದ ಅಲೆ, ಪೂರ್ವ ಪಶ್ಚಿಮ ಉತ್ತರ ದಕ್ಷಿಣವೆನ್ನದೆ ಜಗತ್ತನ್ನು ಆವರಿಸುವುದು. ಇಂದು ಜಡ ನಾಗರಿಕ ಪ್ರಪಂಚಕ್ಕೆ ಆಧ್ಯಾತ್ಮವನ್ನು ಧಾರೆಯೆರೆಯುವ ಮಹಾ ಪ್ರವಾಹವೂ ಇಲ್ಲಿಂದ ಉದಯಿಸಬೇಕಾಗಿದೆ. ಇಲ್ಲಿದೆ ಬಾಳಿಗೆ ಹೊಸ ಬೆಳಕನ್ನು ಕೊಡುವ ಅಮೃತ ಪ್ರವಾಹ. ಅನ್ಯದೇಶಗಳಲ್ಲಿ ಕೋಟ್ಯಂತರ ಜೀವಿಗಳ ಎದೆಯನ್ನು ದಹಿಸುತ್ತಿರುವ ಜಡವಾದದ ದಳ್ಳುರಿಯ ಶಮನಕ್ಕೆ ಅಮೃತ ಪ್ರವಾಹ ಇಲ್ಲಿದೆ.

\vskip 3pt

 “ಭರತಖಂಡ ಅನ್ಯ ದೇಶಕ್ಕೆ ಶಾಂತಿಯ ಬೆಳಕನ್ನು ಕೊಟ್ಟಿದ್ದು ಪ್ರೀತಿಯ ಮೂಲಕ. ಅವರು ಎಂದಿಗೂ ಇನ್ನೊಬ್ಬರನ್ನು ಆಳಲು, ಅವರಲ್ಲಿರುವುದನ್ನು ಕಸಿದುಕೊಳ್ಳಲು ಹೋಗಲಿಲ್ಲ. ಹಲವು ಚಕ್ರಾಧಿಪತ್ಯಗಳು ಜನಾಂಗಗಳು ನಾಶವಾದರೂ ಭರತಖಂಡ ಮನುವಿನ ಕಾಲದಲ್ಲಿ ಹೇಗಿತ್ತೊ ಈಗಲೂ ಹಾಗೆಯೇ ಇರುವುದು. ಇತರ ದೇಶಗಳಲ್ಲಿ ಧರ್ಮ ಹಲವು ವಸ್ತುಗಳಲ್ಲಿ ಒಂದು, ಆದರೆ ಇಂಡಿಯಾ ದೇಶದಲ್ಲಿ ಆದರೊ ಅದೇ ಪ್ರಾಣ. ರಾಜಕೀಯ ಮತ್ತು ಬೇರೆ ಬೇರೆ ವಿಷಯಗಳು ಜನ ಸಾಮಾನ್ಯರಿಗೆ ಗೊತ್ತಿಲ್ಲದೇ ಇದ್ದರೂ\break ದಾರಿಹೋಕನಿಗೂ ಕೂಡ ಒಬ್ಬ ಸಂನ್ಯಾಸಿ ಅಮೇರಿಕಾದೇಶಕ್ಕೆ ಹೋಗಿ ಇಂಡಿಯಾ ದೇಶಕ್ಕೆ ಕೀರ್ತಿ ತಂದನೆನ್ನುವುದು ಗೊತ್ತಿದೆ. ರಾಜಕೀಯ ಮಹತ್ವವಾಗಲಿ, ಸೇನಾಶಕ್ತಿಯಾಗಲಿ ನಮ್ಮ ಗುರಿಯಲ್ಲ. ಅಧ್ಯಾತ್ಮ ವಿದ್ಯೆಯನ್ನೇ ಭರತಖಂಡ ಹಿಂದೆ ಇತರರಿಗೆ ಕೊಟ್ಟಿದೆ. ಮುಂದೆಯೂ ಅದನ್ನೇ ಕೊಡಬೇಕಾಗಿದೆ. ಭರತಖಂಡ ತನ್ನನ್ನು ಆಳುತ್ತಿರುವವರ ಮೂಲಕವಾಗಿಯೇ ತನ್ನ ಪ್ರಭಾವವನ್ನು ಇತರರಿಗೆ ಬೀರುತ್ತಿದೆ. ಪಾಶ್ಚಾತ್ಯ ದೇಶಗಳಲ್ಲಿ ವಿಜ್ಞಾನದ ಮುನ್ನಡೆ ಕ್ರೈಸ್ತಧರ್ಮದ ಅನೇಕ ಸಣ್ಣ ಅಭಿಪ್ರಾಯವನ್ನು ನುಚ್ಚುನೂರು ಮಾಡಿದೆ. ವೇದಾಂತದ ತಾತ್ತ್ವಿಕ ಭಾವನೆಗೆ ಮಾತ್ರ ಅದನ್ನು ಎದುರಿಸುವ ಶಕ್ತಿಯಿದೆ. ವೇದಾಂತ ಯಾವುದೋ ಸಣ್ಣ ದೃಷ್ಟಿಯಿಂದ ನೋಡುವುದಿಲ್ಲ, ಅದು ಎಲ್ಲಾ ತತ್ತ್ವಗಳನ್ನೂ ಮತ್ತು ಮತಗಳನ್ನೂ ಒಳಗೊಂಡು, ಮುಂದೆ ಬರುವುದಕ್ಕೂ ಒಂದು ಸ್ಥಾನವನ್ನು ನೀಡಬಲ್ಲದು. ಇರುವುದೊಂದೇ ಸತ್ಯ, ಅದನ್ನು ಜನರು ಹಲವು ಹೆಸರುಗಳಿಂದ ಕರೆಯುವರು ಎಂಬುದೇ ಅದರ ಮುಖ್ಯ ಪಲ್ಲವಿ. ನದಿಗಳೆಲ್ಲ ಎಲ್ಲೆಲ್ಲಿಯೋ ಹರಿದುಕೊಂಡು ಹೋಗಿ ಕೊನೆಗೆ ಸಾಗರವನ್ನು ಸೇರುವಂತೆ, ಜೀವಿಗಳು ಎಲ್ಲಿಯಾದರೂ ಹುಟ್ಟಲಿ, ಯಾವ ಧರ್ಮವನ್ನಾದರೂ ಸ್ವೀಕರಿಸಲಿ, ಕೊನೆಗೆ ಪರಮ ಸತ್ಯದ ಅನಂತ ಸಾಗರಕ್ಕೆ ಸೇರುತ್ತಿವೆ ಎಂಬ ಭಾವವನ್ನು ಅನುಷ್ಠಾನಕ್ಕೆ ತರಬೇಕೆಂದು ಅದು ಬೋಧಿಸುತ್ತದೆ.” 

\vskip 3pt

 ಮಾರನೆ ದಿನ ಸ್ವಾಮೀಜಿಯವರು ತಮ್ಮನ್ನು ಭೇಟಿ ಮಾಡುವುದಕ್ಕೆ ಬಂದವರೊಂದಿಗೆ ಸಂಜೆಯ ತನಕ ಮಾತನಾಡಿದರು. ಸಾಯಂಕಾಲ ಅಲ್ಲಿರುವ ಶಿವನ ದೇವಾಲಯಕ್ಕೆ ಹೋದರು. ದಾರಿಯಲ್ಲಿ ಹೋಗುತ್ತಿರುವಾಗ ಗಾಡಿಯನ್ನು ಹಲವು ಮನೆಗಳ ಮುಂದೆ ನಿಲ್ಲಿಸಬೇಕಾಯಿತು. ಅಲ್ಲಿಯ ಭಕ್ತರು ಕೊಡುತ್ತಿದ್ದ ಹೂವು ಹಣ್ಣನ್ನು ಸ್ವೀಕರಿಸಬೇಕಾಯಿತು. ದೇವಸ್ಥಾನದಲ್ಲಿ ‘ಜಯ ಮಹಾದೇವ’ ಎಂಬ ಘೋಷದೊಡನೆ ಸ್ವಾಮೀಜಿಯವರನ್ನು ಬರಮಾಡಿಕೊಂಡರು. ಅಲ್ಲಿ ನೆರೆದ ಭಕ್ತವೃಂದದೊಡನೆ ಮತ್ತು ಅರ್ಚಕರೊಡನೆ ಸ್ವಲ್ಪ ಹೊತ್ತು ಮಾತನಾಡಿದ ಮೇಲೆ ಅವರು ತಂಗಿದ್ದ ಮನೆಗೆ ಬಂದರು. ಅಲ್ಲಿ ಇವರೊಡನೆ ಮಾತನಾಡಲು ಬಂದಿದ್ದ ಬ್ರಾಹ್ಮಣರೊಡನೆ ಅರ್ಧರಾತ್ರಿಯ ತನಕ ಸಂಭಾಷಣೆಯನ್ನು ಮಾಡಿದರು. 

\vskip 3pt

 ಸೋಮವಾರ ಚಲ್ಲಯ್ಯ ಎಂಬ ಭಕ್ತನ ಮನೆಗೆ ಹೋದರು. ಇವರನ್ನು ಎದುರುಗೊಳ್ಳುವುದಕ್ಕಾಗಿ ಅವರ ಮನೆಯನ್ನು ತಳಿರುತೋರಣಗಳಿಂದ ಅಲಂಕರಿಸಿದ್ದರು. ಸ್ವಾಮೀಜಿ ಇಲ್ಲಿಗೆ ಬರುತ್ತಾರೆ ಎಂಬುದನ್ನು ತಿಳಿದ ಮೇಲೆ ಸಹಸ್ರಾರು ಜನ ಅವರ ದರ್ಶನಕ್ಕಾಗಿ ಇಲ್ಲಿ ನೆರೆದರು. ಭಕ್ತನ ಮನೆಗೆ ಬಂದಮೇಲೆ ಸ್ವಾಮೀಜಿಯವರನ್ನು ಒಂದು ಪೀಠದ ಮೇಲೆ ಕೂರಿಸಿ ಗಂಗಾನದಿಯ ಜಲವನ್ನು ಅವರ ಮೇಲೆ ಪ್ರೋಕ್ಷಣೆ ಮಾಡಿದರು. ಅನಂತರ ಸ್ವಾಮೀಜಿ ಅಲ್ಲಿ ನೆರೆದವರಿಗೆಲ್ಲ ವಿಭೂತಿಯನ್ನು ಕೊಟ್ಟರು. ಎಲ್ಲರೂ ಸಂತೋಷದಿಂದ ಸ್ವೀಕರಿಸಿದರು. ಅಲ್ಲಿ ಒಂದು ಕಡೆ ಶ‍್ರೀರಾಮಕೃಷ್ಣರ ಭಾವಚಿತ್ರವನ್ನು ನೇತುಹಾಕಿದ್ದರು. ಸ್ವಾಮೀಜಿ ತಮ್ಮ ಗುರುದೇವರ ಭಾವಚಿತ್ರವನ್ನು ಕಂಡೊಡನೆಯೇ ಅದಕ್ಕೆ ಭಕ್ತಿಯಿಂದ ನಮಸ್ಕಾರ ಮಾಡಿದರು. ಅವರು ಮನೆಯಲ್ಲಿ ಸ್ವಲ್ಪ ಫಲಾಹಾರವನ್ನು ಸ್ವೀಕರಿಸಿದರು. ದೇವರ ಭಜನೆಯಾದ ನಂತರ ಸ್ವಾಮೀಜಿ ಹಿಂದಿರುಗಿದರು. 

\vskip 3pt

 ಅಂದಿನ ಸಾಯಂಕಾಲ ವೇದಾಂತ ತತ್ತ್ವದ ಮೇಲೆ ಸ್ಫೂರ್ತಿದಾಯಕವಾದ ಬಹಿರಂಗ ಉಪನ್ಯಾಸವನ್ನು ಮಾಡಿದರು. ದ್ವೈತ, ವಿಶಿಷ್ಟಾದ್ವೈತಗಳನ್ನೆಲ್ಲ ಒಳಗೊಳ್ಳುವುದು ಅದ್ವೈತ ವೇದಾಂತ. ಯಾರು ಯಾವ ಮೆಟ್ಟಲಲ್ಲಿ ಇರಲಿ, ಅವರನ್ನು ದೂರದೆ ಹಳಿಯದೆ, ಅವರನ್ನು ಮುಂದೆ ಕೊಡೊಯ್ಯುವಂತಹದು ವೇದಾಂತದ ವಿಶ್ವಧರ್ಮ ಎಂಬುದನ್ನು ಒತ್ತಿ ಹೇಳಿದರು. ತಾವು ಉಪನ್ಯಾಸ ಮಾಡುತ್ತಿದ್ದಾಗ ಸಭಿಕರಲ್ಲಿ ಹಲವರು ಪಾಶ್ಚಾತ್ಯ ಉಡುಗೆಯಲ್ಲಿದ್ದುದನ್ನು ನೋಡಿದರು. ಸ್ವಾಮೀಜಿ ವೇಷ ಭೂಷಣಗಳಲ್ಲಿ ಪಾಶ್ಚಾತ್ಯರ ಅನುಕರಣವನ್ನು ಖಂಡಿಸಿದರು. ಪ್ರತಿ ದೇಶದವರಿಗೆ ತಕ್ಕ ಉಡಿಗೆ ತೊಡಿಗೆ ಇದೆ. ಅದನ್ನು ಬಿಟ್ಟು ಯಾರದನ್ನೋ ಅನುಸರಿಸುತ್ತಿರುವುದನ್ನು ನೋಡಿ ಸ್ವಾಮೀಜಿಗೆ ಸಹನೆ ಮೀರಿತು. 

\vskip 3pt

 ಮುಂಚೆ ಸ್ವಾಮೀಜಿ ಕೊಲಂಬೊಯಿಂದ ನೇರವಾಗಿ ಮದ್ರಾಸಿಗೆ ಹೋಗಬೇಕೆಂದು ಆಶಿಸಿದ್ದರು. ಆದರೆ ಸಿಲೋನಿನ ಅನೇಕ ಕಡೆಗಳಿಂದ ಮತ್ತು ಮದ್ರಾಸಿಗೆ ದಕ್ಷಿಣದಲ್ಲಿರುವ ಅನೇಕ ನಗರಗಳಿಂದ ಅವರು ತಮ್ಮ ಊರಿನ ಮೂಲಕ ಹೋಗಬೇಕೆಂದು ಬೇಕಾದಷ್ಟು ತಂತಿಗಳು ಮತ್ತು ಪತ್ರಗಳು ಬಂದವು. ಆದಕಾರಣ ಸ್ವಾಮೀಜಿ ೧೯ನೇ ತಾರೀಖು ಬೆಳಿಗ್ಗೆ ರೈಲ್ವೆ ಸೆಲೂನ್‍ನಲ್ಲಿ ಕ್ಯಾಂಡಿಗೆ ಹೊರಟರು. ಕ್ಯಾಂಡಿಯ ರೈಲ್ವೆ ನಿಲ್ದಾಣದಲ್ಲಿ ಸ್ವಾಮೀಜಿಯವರನ್ನು ಎದುರುಗೊಳ್ಳಲು ಸಹಸ್ರಾರು ಜನ ನೆರೆದಿದ್ದರು. ಅಲ್ಲಿ ಪೂರ್ಣಕುಂಭ ವಾಲಗ ಮುಂತಾದುವುಗಳೊಡನೆ ಒಂದು ಮೆರವಣಿಗೆ ಮಾಡಿ ಅನಂತರ ವಿಶ್ರಾಂತಿಗಾಗಿ ಒಂದು ದೊಡ್ಡ ಮನೆಗೆ ಕರೆದುಕೊಂಡು ಹೋದರು. ಅಲ್ಲಿ ಒಂದು ಬಿನ್ನವತ್ತಳೆಯನ್ನು ಪುರಜನರು ಸ್ವಾಮೀಜಿಗೆ ಅರ್ಪಿಸಿದರು. ಸ್ವಾಮೀಜಿ ಅದಕ್ಕೆ ಸೂಕ್ತವಾದ ಉತ್ತರವನ್ನು ಕೊಟ್ಟು, ಆ ಊರಿನಲ್ಲಿ ಪ್ರೇಕ್ಷಣೀಯವಾದ ಸ್ಥಳಗಳನ್ನು ನೋಡಿಕೊಂಡು ಅಲ್ಲಿಂದ ಹೊರಟು ಅಂದಿನ ಸಾಯಂಕಾಲವೇ ಮೇಕಳಿ ಎಂಬ ಊರನ್ನು ತಲುಪಿದರು. 

\vskip 3pt

 ಮಾರನೆ ದಿನ ಅಲ್ಲಿಂದ ಜಾಫ್ನಕ್ಕೆ ಸುಮಾರು ಇನ್ನೂರು ಮೈಲಿಗಳನ್ನು ಕೋಚ್ ಗಾಡಿಯಲ್ಲಿ ಪ್ರಯಾಣಮಾಡಿಕೊಂಡು ಹೊರಟರು. ಪ್ರಪಂಚದಲ್ಲೆಲ್ಲ ಅತ್ಯಂತ ರಮ್ಯವಾದ ದೃಶ್ಯಗಳು ಆ ದಾರಿಯಲ್ಲಿ ಸಿಕ್ಕುತ್ತವೆ. ದಾಬೂಲ್ ನಿಂದ ಕೆಲವು ಮೈಲಿಗಳಾಚೆ ಗಾಡಿಗೆ ಒಂದು ಆಕಸ್ಮಿಕವಾಯಿತು. ಒಂದು ಬೆಟ್ಟವನ್ನು ಇಳಿಯುತ್ತಿದ್ದಾಗ ಗಾಡಿಯ ಮುಂದಿನ ಚಕ್ರ ಒಂದಕ್ಕೆ ಪೆಟ್ಟು ತಗಲಿ ಅದು ಊನವಾಯಿತು. ಇದರಿಂದ ರೋಡಿನ ಪಕ್ಕದಲ್ಲಿ ಮೂರುಗಂಟೆಗಳು ನಿಲ್ಲಬೇಕಾಯಿತು. ದೇವರ ದಯೆಯಿಂದ ಚಕ್ರ ಹೊರಗೆ ಬರಲಿಲ್ಲ. ಹಾಗೇನಾದರೂ ಆಗಿದ್ದರೆ ಗಾಡಿ ತಲೆಕೆಳಕಾಗಿ ಉರುಳಿಕೊಳ್ಳುತ್ತಿತ್ತು. ಬಹಳ ಹೊತ್ತು ಕಾದ ಮೇಲೆ ದೂರದ ಒಂದು ಹಳ್ಳಿಯಿಂದ ಒಂದು ಎತ್ತಿನ ಗಾಡಿ ಮಾಡಿಕೊಂಡು ಬಂದರು. ಅದರಲ್ಲಿ ಶ‍್ರೀಮತಿ ಸೇವಿಯರ್ಸ್‍‍ ಮತ್ತು ಸಾಮಾನುಗಳನ್ನು ಕಳುಹಿಸಿ ಉಳಿದವರು\break ನಡೆದುಕೊಂಡು ಹೋದರು. ಅನಂತರ ಬೇರೆ ಗಾಡಿಗಳು ಬಂದಮೇಲೆ ಅದರಲ್ಲಿ ಇತರರು ಕುಳಿತುಕೊಂಡರು. ಅಂದಿನ ರಾತ್ರಿಯೆಲ್ಲ ಎತ್ತಿನ ಗಾಡಿಯಲ್ಲೆ ಕಳೆಯಬೇಕಾಯಿತು. ಕನಹರಿ ಮತ್ತು ತಿನ್‍ಪಾನಿ ಊರುಗಳನ್ನು ಹಾದು ಅನುರಾಧಾಪುರಕ್ಕೆ ಬಂದರು. ಅಲ್ಲಿ ಒಂದು ಅಶ್ವತ್ಥವೃಕ್ಷದ ಕೆಳಗೆ ನೆರೆದ ಎರಡು ಮೂರು ಸಾವಿರ ಜನರಿಗೆ ಸ್ವಾಮೀಜಿಯವರು ‘ಪೂಜೆ’ ಎಂಬ ವಿಷಯದ ಮೇಲೆ ಉಪನ್ಯಾಸ ಮಾಡತೊಡಗಿದರು. ಆ ಸಮಯದಲ್ಲಿ ಬೌದ್ಧಭಿಕ್ಷುಗಳು ಅವರ ಧರ್ಮಕ್ಕೆ ಸೇರಿದ ಗಂಡಸರು ಹೆಂಗಸರು ಮಕ್ಕಳನ್ನು ಕರೆದುಕೊಂಡು ಬಂದು ಟಿನ್ನು, ತಮಟೆ, ಜಾಗಟೆ ಇವುಗಳನ್ನು ಬಡಿದು ಗಲಾಟೆ ಎಬ್ಬಿಸಿದರು, ಅಹಿಂಸೆಯ ಮೂಲಕ ಸ್ವಾಮೀಜಿ ಉಪನ್ಯಾಸವನ್ನು ಹಿಂದೂಗಳು ಕೇಳದಂತೆ ಮಾಡುವುದಕ್ಕೆ‌! ಹಿಂದೂಗಳು ಕೋಪಗೊಂಡು ಬೌದ್ಧರ ಮೇಲೆ ಬೀಳುವುದರಲ್ಲಿ ಇದ್ದರು. ಆ ಸಮಯದಲ್ಲಿ ಸ್ವಾಮೀಜಿ ಅವರಿಗೆ ತಾಳ್ಮೆಯಿಂದಿರಲು ಹೇಳಿದರು. ಒಂದೇ ಸತ್ಯವನ್ನು ಶಿವ, ವಿಷ್ಣು, ಬುದ್ಧನೆಂದು ಕರೆಯುತ್ತಿರುವರೆಂದೂ, ಒಂದು ಧರ್ಮದವರು ಮತ್ತೊಂದು ಧರ್ಮಕ್ಕೆ ಗೌರವವನ್ನು ತೋರಬೇಕೆಂದೂ, ಪ್ರೀತಿಯಿಂದ ಮಾತ್ರ ಮತ್ತೊಬ್ಬರನ್ನು ಒಲಿಸಿಕೊಳ್ಳಲು ಸಾಧ್ಯವೆಂದೂ ನೆರೆದವರಿಗೆ ಹೇಳಿದರು. ಅನುರಾಧಾಪುರದಿಂದ ಜಾಫ್ನಕ್ಕೆ ನೂರ ಇಪ್ಪತ್ತು ಮೈಲಿಗಳು. ಕುದುರೆ ಜೊತೆಗೆ ಗಾಡಿ ಎರಡೂ ಚೆನ್ನಾಗಿ ಇಲ್ಲದೇ ಇದ್ದುದರಿಂದ ಪ್ರಯಾಣ ಬಹಳ ತ್ರಾಸವಾಯಿತು. ಆದರೆ ಸುತ್ತಮುತ್ತಲ ನೈಸರ್ಗಿಕ ಸೌಂದರ್ಯ ಬಹಳ ಆಹ್ಲಾದಕರವಾಗಿ ಇದ್ದುದರಿಂದ ಪಯಣದ ಕಷ್ಟವನ್ನು ಹೇಗೋ ಸಹಿಸಿಕೊಂಡು ಹೋದರು. ವಿನೋನಿಯ ಎಂಬ ಊರಿನಲ್ಲಿ ಸ್ವಾಮೀಜಿಯವರನ್ನು ಸಕಲ ಮರ್ಯಾದೆಗಳೊಡನೆ ಆಹ್ವಾನಿಸಿ ಒಂದು ಬಿನ್ನವತ್ತಳೆಯನ್ನು ಕೊಟ್ಟರು. ಸ್ವಾಮೀಜಿಯವರು ಅದಕ್ಕೆ ಉತ್ತರವನ್ನು ಕೊಟ್ಟಾದಮೇಲೆ ಸಿಲೋನ್ ದೇಶದ ಅರಣ್ಯಗಳ ಮೂಲಕ ಜಾಫ್ನಕ್ಕೆ ಹೊರಟರು. ಮಾರನೆ ದಿನ, ಜಾಫ್ನ ದ್ವೀಪವನ್ನು ಸಿಲೋನಿನೊಂದಿಗೆ ಒಂದುಗೂಡಿಸುವ Elephant Pass ಎಂಬ ದೊಡ್ಡ ಸೇತುವೆಯ ಬಳಿ ಒಂದು ಸಣ್ಣ ಸ್ವಾಗತವನ್ನು ಸ್ವಾಮೀಜಿಯವರಿಗೆ ನೀಡಲಾಯಿತು. ಜಾಫ್ನಕ್ಕೆ ಇನ್ನೂ ಹನ್ನೆರಡು ಮೈಲಿಗಳು ಇರುವಾಗಲೇ ಜಾಫ್ನದ ಪುರಪ್ರಮುಖರು ಸ್ವಾಮೀಜಿಯವರನ್ನು ಎದುರುಗೊಳ್ಳಲು ಬಂದು ಕಾದಿದ್ದರು. ಅಲ್ಲಿಂದ ಹಲವು ಗಾಡಿಗಳಲ್ಲಿ ಎಲ್ಲರೂ ಹೊರಟರು. ಊರಿನ ಬೀದಿಗಳನ್ನೆಲ್ಲ ಸ್ವಾಮೀಜಿಯವರ ಸ್ವಾಗತಕ್ಕಾಗಿ ಅಲಂಕರಿಸಿದ್ದರು. ಸಾಯಂಕಾಲ ಸ್ವಾಮೀಜಿಯವರನ್ನು ದೀಪಗಳೊಂದಿಗೆ ಮೆರವಣಿಗೆಯಲ್ಲಿ ಜಾಫ್ನ ಕಾಲೇಜಿನ ಸಮೀಪದಲ್ಲಿ ಹಾಕಿರುವ ಚಪ್ಪರದ ಕಡೆಗೆ ಕರೆದುಕೊಂಡು ಹೋದರು. ಅಲ್ಲಿ ಸಹಸ್ರಾರು ಜನರು ನೆರೆದಿದ್ದರು. ಸ್ವಾಮೀಜಿಗೆ ಒಂದು ಬಿನ್ನವತ್ತಳೆಯನ್ನು ಅರ್ಪಿಸಿದರು. ಸ್ವಾಮೀಜಿಯವರು ಸ್ವಾಗತಕ್ಕೆ ಉತ್ತರವನ್ನು ಕೊಟ್ಟರು. ಮಾರನೆ ದಿನ ಸಾಯಂಕಾಲ ಜಾಫ್ನ ಹಿಂದೂ ಕಾಲೇಜಿನಲ್ಲಿ ಸ್ವಾಮೀಜಿ ‘ವೇದಾಂತ’ ಎಂಬುದರ ಮೇಲೆ ಹೃದಯಸ್ಪರ್ಶಿಯಾದ ಭಾಷೆಯಲ್ಲಿ ಮಾತನಾಡಿದರು. ಉಪನ್ಯಾಸವಾದ ನಂತರ ಸ್ವಾಮೀಜಿ ಜೊತೆಯಲ್ಲಿ ಬಂದಿದ್ದ ಸೇವಿಯರ‍್ಸ ಅವರನ್ನು ಅವರು ಏತಕ್ಕೆ ವೇದಾಂತವನ್ನು ಸ್ವೀಕರಿಸಿದರು ಎಂದು ಜನ ಕೇಳಿದುದಕ್ಕೆ,\break ಸೇವಿಯರ‍್ಸ ಉತ್ತರವನ್ನು ನೀಡಿ, ತಾವು ಹಿಂದೂ ದೇಶಕ್ಕೆ ಸ್ವಾಮೀಜಿಯ ಕೆಲಸದಲ್ಲಿ ನೆರವಾಗಲು ಬಂದಿರುವೆ ಎಂದು ಹೇಳಿದರು. 

 ಸ್ವಾಮೀಜಿ ಪ್ರಭಾವ ಸಿಂಹಳ ದ್ವೀಪದ ಹಿಂದೂಗಳ ಮೇಲೆ ಅಗಾಧವಾಗಿತ್ತು. ಅನೇಕ ಊರುಗಳಿಂದ ಹಿಂದೂ ಬೋಧಕರನ್ನು ಕಳುಹಿಸಿಕೊಡಬೇಕೆಂದು ಸ್ವಾಮೀಜಿಗೆ ಪತ್ರಗಳು ಬಂದವು. ಇನ್ನೂ ಹಲವಾರು ಊರುಗಳಿಂದ ತಮ್ಮ ಊರಿಗೆ ಬರಬೇಕೆಂದು ಸ್ವಾಮೀಜಿಯವರನ್ನು ಕೋರುವ ತಂತಿಗಳು ಪತ್ರಗಳು ಬೇಕಾದಷ್ಟು ಬಂದವು. ಆದರೆ ಅವುಗಳನ್ನೆಲ್ಲ ಒಪ್ಪಿಕೊಳ್ಳುವುದಕ್ಕೆ ಸಾಧ್ಯವಿರಲಿಲ್ಲ. ಅವರೇನಾದರೂ ಇನ್ನು ಹೆಚ್ಚು ದಿನಗಳಿದ್ದರೆ ಜನರು ತಮ್ಮ ಪ್ರೀತಿಯ ಆಧಿಕ್ಯದಿಂದ ಸ್ವಾಮೀಜಿಯವರನ್ನು ಕೊನೆಗಾಣಿಸುತ್ತಿದ್ದರೆಂದು ಸ್ವಾಮೀಜಿ ಜೊತೆಯಲ್ಲಿದ್ದವರೊಬ್ಬರು ಹೇಳಿದರು. 

