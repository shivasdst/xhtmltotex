
\chapter{ಕಲ್ಕತ್ತೆಗೆ}

 ಕೊಲಂಬೊ ನಗರದಿಂದ ಮದ್ರಾಸಿನವರೆಗೆ ಸ್ವಾಮೀಜಿಯವರಿಗೆ ಸಂದ ಸ್ವಾಗತಗಳನ್ನು ಮತ್ತು ಅವರು ಮಾಡಿದ ಭಾಷಣಗಳನ್ನು ಬಂಗಾಳದ ಜನ ಬಹಳ ಕುತೂಹಲದಿಂದ ನೋಡುತ್ತಿದ್ದರು. ಸ್ವಾಮೀಜಿ ಎಂದು ತಮ್ಮ ಜನ್ಮಸ್ಥಳಕ್ಕೆ ಬರುವರೋ ಎಂದು ಕಾತುರತೆಯಿಂದ ಕಾಯುತ್ತಿದ್ದರು. ಸ್ವಾಮೀಜಿ ಕಲ್ಕತ್ತೆಗೆ ಬರುವುದಕ್ಕಾಗಿ ಹಡಗಿನಲ್ಲಿ ಮದ್ರಾಸನ್ನು ಬಿಟ್ಟರು ಎಂಬ ಸುದ್ದಿಯನ್ನು ಕೇಳಿ ಅವರ ಆನಂದಕ್ಕೆ ಪಾರವಿರಲಿಲ್ಲ. ತಾವು ಸ್ವಾಮೀಜಿಯರಿಗೆ ಸ್ವಾಗತ ನೀಡುವುದಕ್ಕೆ ಯಾರಿಗೂ ಹಿಂದೆ ನಿಲ್ಲಕೂಡದೆಂದು ಕಲ್ಕತ್ತೆಯ ಪುರಜನರು ಸ್ವಾಮೀಜಿಯವರನ್ನು ವೈಭವದಿಂದ ಸ್ವಾಗತಿಸುವುದಕ್ಕೆ ಒಂದು ಸ್ವಾಗತಸಮಿತಿಯನ್ನು ಮಾಡಿದರು. ಅದಕ್ಕೆ ದರ್ಭಾಂಗದ ಮಹಾರಾಜರನ್ನು ಅಧ್ಯಕ್ಷರನ್ನಾಗಿ ಮಾಡಿದರು. 

\newpage

 ಕೊಲಂಬೊ ನಗರಕ್ಕೆ ಬಂದು ಮುಟ್ಟಿದಾಗಿನಿಂದ ಮದ್ರಾಸನ್ನು ಬಿಡುವವರೆಗೆ ಜನರನ್ನು ಕಂಡು ಮಾತನಾಡುವುದು, ಮೆರವಣಿಗೆ, ಬಿನ್ನವತ್ತಳೆಗೆ ಉತ್ತರವನ್ನು ಕೊಡುವುದು,\break ಬಹಿರಂಗ ಸಭೆಗಳಲ್ಲಿ ಭಾಗವಹಿಸುವುದು ಇವುಗಳಿಂದ ಸ್ವಾಮೀಜಿ ಬಹಳ ಆಯಾಸಗೊಂಡಿದ್ದರು. ಸಮುದ್ರದ ಮೇಲೆ ಪ್ರಯಾಣ ಮಾಡಿದರೆ ಪ್ರತಿಯೊಂದು ರೈಲ್ವೆ ನಿಲ್ದಾಣದಲ್ಲಿ ಆಗುವ ಸ್ವಾಗತಗಳನ್ನಾದರೂ ತಪ್ಪಿಸಿಕೊಂಡು ಸ್ವಲ್ಪ ವಿಶ್ರಾಂತಿಯನ್ನು ಪಡೆಯಬಹುದೆಂದು ಅವರು ಆಶಿಸಿದರು. ಸ್ವಾಮೀಜಿಯವರೊಡನೆ ಮೂರು ಜನ ಮದ್ರಾಸಿನ ಭಕ್ತರೂ ಇದ್ದರು. ಅವರೆ ಅಳಸಿಂಗ ಪೆರುಮಾಳ್, ನರಸಿಂಹಾಚಾರಿ ಮತ್ತು ಶಿಂಗಾರುವೇಲು ಮೊದಲಿಯಾರ್. ಜೊತೆಗೆ ಸೇವಿಯರ್ಸ್‍‍ ದಂಪತಿಗಳು ಮತ್ತು ಜೆ. ಜೆ. ಗುಡ್‍ವಿನ್ ಇದ್ದರು. ಮದ್ರಾಸನ್ನು ಬಿಡುವಾಗ ಸ್ವಾಮೀಜಿಯವರಿಗೆ ವೈದ್ಯರು ನೀರಿನ ಬದಲು ಎಳನೀರು ಕುಡಿದರೆ ಒಳ್ಳೆಯದೆಂದು ಹೇಳಿದ್ದರು. ಈ ಮಾತನ್ನು ಕೇಳಿದೊಡನೆಯೆ ಭಕ್ತರು ನೂರಾರು ಎಳನೀರನ್ನು ಸ್ವಾಮೀಜಿಯವರಿಗಾಗಿ ಹಡಗಿನ ಒಳಗೆ ತಂದು ಇಟ್ಟರು. ಇಷ್ಟೊಂದು ಎಳನೀರು ಹಡಗಿಗೆ ಬರುತ್ತಿರುವುದನ್ನು ನೋಡಿ ಸೇವಿಯರ್ಸ್‍‍ ಅವರು ಈ ಹಡಗು ಸಾಮಾನನ್ನು ಹೊರುವ ಕಾರ್ಗೋಬೋಟೆ ಎಂದು ಕೇಳಿದರು. ಅದಕ್ಕೆ ಸ್ವಾಮೀಜಿ, ಅಲ್ಲ, ಇವೆಲ್ಲ ನಾನು ಕುಡಿಯುವುದಕ್ಕೆ ಇರುವ ಎಳನೀರುಗಳು ಎಂದರು. ಹಡಗಿನಲ್ಲಿ ಕ್ಯಾಪ್ಟನ್ ಮತ್ತು ಇತರರಿಗೆಲ್ಲ ಹಂಚಿ ಸ್ವಾಮೀಜಿ ತಾವೂ ಅದನ್ನು ತೆಗೆದುಕೊಳ್ಳುತ್ತಿದ್ದರು. 

\vskip 3pt

 ಸ್ವಾಮೀಜಿ ಕುಳಿತ ಹಡಗು ಕಿಡ್ಡರ್‍ಪೂರ್ ರೇವನ್ನು ಮುಟ್ಟಿದಾಗ ರಾತ್ರಿ ಆಗಿತ್ತು. ಮಾರನೆಯ ದಿನ ಬೆಳಗ್ಗೆ ಅವರನ್ನು ಸಿಯೊಲ್ಡಾ ರೈಲ್ವೆ ಸ್ಟೇಷನ್‍ಗೆ ಒಯ್ಯಲು ಒಂದು ಸ್ಪೆಷಲ್ ಟ್ರೈನ್ ಕಾದಿತ್ತು. ಬೆಳಗ್ಗೆ ಏಳೂವರೆ ಗಂಟೆಗೆ ಸ್ವಾಮೀಜಿ ಮತ್ತು ಅವರ ಪರಿವಾರ ಸ್ಪೆಷಲ್ ಟ್ರೈನ್ ಏರಿದರು. ರೈಲ್ವೆ ನಿಲ್ದಾಣದಲ್ಲಿ ಸ್ವಾಮೀಜಿಯವರನ್ನು ಎದುರುಗೊಳ್ಳಲು ಸಹಸ್ರಾರು ಜನ ನೆರೆದಿದ್ದರು. ಅಮೇರಿಕ ಮತ್ತು ಇಂಗ್ಲೆಂಡ್ ದೇಶೀಯರು ಸ್ವಾಮೀಜಿಯವರಿಗೆ ಬರೆದ ಅಭಿನಂದನ ಪತ್ರಗಳ ಕಾಪಿಯನ್ನು ಪ್ರಿಂಟ್ ಮಾಡಿ ಅದನ್ನು ಜನರಿಗೆ ಹಂಚಿದರು. ರೈಲು ರೈಲ್ವೆ ನಿಲ್ದಾಣವನ್ನು ಮುಟ್ಟಿದೊಡನೆಯೆ ‘ವಿವೇಕಾನಂದರಿಗೆ ಜೈ’, ‘ಶ‍್ರೀರಾಮಕೃಷ್ಣರಿಗೆ ಜೈ’ ಎಂಬುವ ಧ್ವನಿ ಅಣುರಣಿತವಾಗುತ್ತಿತ್ತು. ಸ್ವಾಮೀಜಿಯವರು ಕೈಮುಗಿದು ಜನರ ಸ್ವಾಗತವನ್ನು ಸ್ವೀಕರಿಸಿದರು. ಸುತ್ತಮುತ್ತಲಿದ್ದವರು ಸ್ವಾಮೀಜಿಯ ಚರಣಧೂಳಿಯನ್ನು ತೆಗೆದುಕೊಳ್ಳಲು ಧಾವಿಸಿದರು. ದೊಡ್ಡ ಜನಸಂದಣಿಯು ಅವರ ಸುತ್ತಲೂ ಆವರಿಸಿತು. ರೈಲ್ವೆ ನಿಲ್ದಾಣದಿಂದ ಸಮಿತಿಯ ಪರವಾಗಿ ಬಂದ \enginline{Indian Mirror} ನ ಸಂಪಾದಕರಾದ ಶ‍್ರೀ ನರೇಂದ್ರನಾಥಸೇನ್ ಮತ್ತು ಇತರ ಸದಸ್ಯರು ಸ್ವಾಮೀಜಿಯವರನ್ನು ಹೊರಗೆ ನಿಂತಿರುವ ಗಾಡಿಗೆ ಕರೆದುಕೊಂಡು ಹೋಗುವುದೇ ಕಷ್ಟವಾಯಿತು. ಸ್ವಾಮೀಜಿಯವರ ಅನೇಕ ಗುರುಭಾಯಿಗಳು ರೈಲ್ವೆ ನಿಲ್ದಾಣದಲ್ಲಿ ಹಾಜರಿದ್ದರು. ಸ್ವಾಮೀಜಿಯವರಿಗೆ ಹೊರಲಾರದಷ್ಟು ಹೂವಿನ ಹಾರಗಳನ್ನು ಹಾಕಲಾಯಿತು. 

\vskip 3pt

 ಸ್ವಾಮೀಜಿ ಹೊರಗೆ ಬಂದು ಸೇವಿಯರ್ಸ್‍‍ ದಂಪತಿಗಳೊಡನೆ ಗಾಡಿಯಲ್ಲಿ ಕುಳಿತೊಡನೆ ಯುವಕರು ಉತ್ಸಾಹದಿಂದ ಕುದುರೆಯನ್ನು ಕಳಚಿ ಗಾಡಿಯನ್ನು ತಾವೇ ಎಳೆಯಲು ಮೊದಲು ಮಾಡಿದರು. ರಿಪ್ಪನ್ ಕಾಲೇಜಿನ ಕಡೆಗೆ ಮೆರವಣಿಗೆ ಹೊರಟಿತು. ಮುಂದೆ ಬ್ಯಾಂಡ್, ಅದರ ಹಿಂದೆ ಬೇಕಾದಷ್ಟು ಭಜನೆ ಪಾರ್ಟಿಗಳಿದ್ದುವು. ದಾರಿ ಸ್ವಾಮೀಜಿಯವರ ಬರುವಿಗಾಗಿ ತಳಿರುತೋರಣಗಳಿಂದ ಅಲಂಕೃತವಾಗಿತ್ತು. ಸರ್‍ಕ್ಯುಲರ್ ರೋಡಿನಲ್ಲಿ ಒಂದು ವಿಜಯ ಮಂಟಪವನ್ನು ಕಟ್ಟಿದ್ದರು. ಅದರ ಮುಂದೆ ‘ಜಯ ವಿವೇಕಾನಂದ’ ಎಂದು ಚಿತ್ರಿಸಿದ್ದರು. ಹ್ಯಾರಿಸನ್ ರೋಡಿನ ಮುಂದೆ ಮತ್ತೊಂದು ಚಪ್ಪರ. ಅದರ ಮೇಲೆ ‘ಜಯ ರಾಮಕೃಷ್ಣ’ ಎಂದು ಚಿತ್ರಿಸಿದ್ದರು. ರಿಪ್ಪನ್ ಕಾಲೇಜಿನ ಮುಂದುಗಡೆ ಇನ್ನೊಂದು ದೊಡ್ಡ ಚಪ್ಪರವನ್ನು ಕಟ್ಟಿದ್ದರು. ಅದರ ಮೇಲೆ ‘ಸ್ವಾಗತ’ ಎಂದು ಬರೆದಿದ್ದರು. ಕಾಲೇಜಿನ ಮುಂದೆ ಸಹಸ್ರಾರು ಜನರು ಸ್ವಾಮೀಜಿಯವರನ್ನು ನೋಡಲು ನೆರೆದಿದ್ದರು. ಕಾಲೇಜಿನ ಮುಂದೆ ಒಂದು ಅಧಿಕೃತವಾದ ಸ್ವಾಗತವನ್ನು ಮಾಡಿ ಸ್ವಾಮೀಜಿಯವರನ್ನು ಕೊಂಡಾಡಿದರು. ಸ್ವಾಮೀಜಿ ಅದಕ್ಕೆ ಸಣ್ಣ ಉತ್ತರವನ್ನು ನೀಡಿದರು. ಆ ಸಮಯ ಮಾತನಾಡುವುದಕ್ಕೆ ಬಿನ್ನವತ್ತಳೆ ಅರ್ಪಿಸುವದಕ್ಕಲ್ಲ. ಮೊದಲು ಕಲ್ಕತ್ತೆಯ ಲಕ್ಷಾಂತರ ಜನರು ಸ್ವಾಮೀಜಿಯವರ ದರ್ಶನಾಕಾಂಕ್ಷಿಗಳಾಗಿದ್ದರು. ಅವರನ್ನು ಮೊದಲು ತೃಪ್ತಿಪಡಿಸಬೇಕಾಗಿತ್ತು. ಆದಕಾರಣ ಬಿನ್ನವತ್ತಳೆಯನ್ನು ಅರ್ಪಿಸುವುದನ್ನು ಒಂದು ವಾರ ಮುಂದೆ ಹಾಕಲಾಯಿತು. ಅಷ್ಟು ಹೊತ್ತಿಗೆ ಜನರು ನೋಡುವ ಕುತೂಹಲ ತಗ್ಗಿ ಉಪನ್ಯಾಸವನ್ನು ಕೇಳುವ ಸ್ಥಿತಿಗೆ ಬರುವರು ಎಂದು ಆಶಿಸಿದರು. ಅಂದಿನ ಮಧ್ಯಾಹ್ನ ಸ್ವಾಮೀಜಿಯವರನ್ನು ಬಾಗ್ ಬಜಾರಿನಲ್ಲಿರುವ ರಾಯ್ ಪಶುಪತಿನಾಥಬೋಸ್ ಅವರು ಊಟಕ್ಕೆ ಕರೆದಿದ್ದರು. ಅನಂತರ ಸಾಯಂಕಾಲ ಸ್ವಾಮೀಜಿ ಮತ್ತು ಅವರ ಪರಿವಾರವನ್ನೆಲ್ಲ ಕಾಶೀಪುರದಲ್ಲಿ ಗೋಪಾಲಲಾಲಶೀಲರ ತೋಟದ ಮನೆಯಲ್ಲಿ ಇಳಿಸಲಾಯಿತು. ತಾತ್ಕಾಲಿಕವಾಗಿ ಸ್ವಾಮೀಜಿಯವರಿಗೆ ತಂಗುವುದಕ್ಕೆ ಆ ಬಿಡಾರವನ್ನು ಕೊಟ್ಟರು. 

 ಸ್ವಾಮೀಜಿ ಹಗಲು ಹೊತ್ತಿನಲ್ಲಿ ಶೀಲರ ತೋಟದ ಮನೆಯಲ್ಲಿರುತ್ತಿದ್ದರು. ಕಲ್ಕತ್ತಾ ಪುರಜನರು ಬಿಡುವಿಲ್ಲದೆ ಸ್ವಾಮೀಜಿಯವರ ದರ್ಶನಕ್ಕೆ ಬರುತ್ತಿದ್ದರು. ಅವರ ಹಳೆಯ ಸ್ನೇಹಿತರು ಮತ್ತು ಶ‍್ರೀರಾಮಕೃಷ್ಣರ ಭಕ್ತವೃಂದ ಅಮೇರಿಕಾ ದೇಶದಲ್ಲಿ ಸ್ವಾಮೀಜಿಯವರು ಸಾಧಿಸಿದುದನ್ನು ಅವರ ಬಾಯಿನಿಂದಲೇ ಕೇಳಬೇಕೆಂದು ಅವರೊಡನೆ ಆ ಪ್ರಸ್ತಾಪವನ್ನು ಎತ್ತುತ್ತಿದ್ದರು. ಇತರ ವಿದ್ವಾಂಸರು ಪಂಡಿತರು ಚರ್ಚೆಗೆ ಬರುತ್ತಿದ್ದರು. ಸ್ವಾಮೀಜಿಯವರು ಅದಕ್ಕೆಲ್ಲ ಕಾಲವನ್ನು ಕೊಡಬೇಕಾಯಿತು. ಪ್ರಖ್ಯಾತಿಗೆ ಒಮ್ಮೆ ಏರಿದರೆ ಒಬ್ಬ ಕೊಡುವ ಬೆಲೆ ಇದು. ಸ್ವಾಮೀಜಿ ರಾತ್ರಿಯ ವೇಳೆ ಆಲಂಬಜಾರಿನಲ್ಲಿದ್ದ ಶ‍್ರೀರಾಮಕೃಷ್ಣಮಠಕ್ಕೆ ಹೋಗಿರುತ್ತಿದ್ದರು. ಅಲ್ಲಿ ತಮ್ಮ ಗುರುಭಾಯಿಗಳೊಡನೆ ಅತ್ಯಂತ ಆನಂದದಿಂದ ಕಾಲ ಕಳೆಯುತ್ತಿದ್ದರು. 

 ಫೆಬ್ರವರಿ ೨೮ನೇ ತಾರೀಖು ೧೮೯೭ನೇ ದಿನದಂದು ಸ್ವಾಮೀಜಿಗೆ ಬಿನ್ನವತ್ತಳೆಯನ್ನು ಅರ್ಪಿಸುವುದೆಂದು ಕಲ್ಕತ್ತೆಯ ಪುರಜನರು ನಿಷ್ಕರ್ಷಿಸಿದರು. ಅದಕ್ಕೆ ಸ್ಥಳವನ್ನು ಶೋಭಾ ಬಜಾರಿನಲ್ಲಿರುವ ರಾಜಾ ಸರ್ ರಾಧಾಕೃಷ್ಣದೇವ ಬಹದ್ದೂರ್ ಅವರ ವಿಸ್ತಾರವಾದ ಮನೆಯಲ್ಲಿ ಅಣಿಮಾಡಿದ್ದರು. ಸ್ವಾಮೀಜಿಯವರು ಬಿನ್ನವತ್ತಳೆಯನ್ನು ಸ್ವೀಕರಿಸುವುದಕ್ಕೆ ಬಂದಾಗ ಕಲ್ಕತ್ತೆಯ ಪ್ರಖ್ಯಾತ ಪುರವಾಸಿಗಳೆಲ್ಲ ನೆರೆದಿದ್ದರು. ಇನ್ನು ಯಾವ ಸಾರ್ವಜನಿಕ ಸಮಾರಂಭಕ್ಕೂ ಇಷ್ಟೊಂದು ಗಣ್ಯವ್ಯಕ್ತಿಗಳು ಬಂದಿರಲಿಲ್ಲ. ಅವರೆಲ್ಲ ಸ್ವಾಮೀಜಿಯವರನ್ನು ಸ್ವಾಗತಿಸಿದರು. ಬಿನ್ನವತ್ತಳೆಯನ್ನು ಅರ್ಪಿಸುವ ಕಡೆಯೇ ಸುಮಾರು ಐದು ಸಾವಿರ ಜನ ಕುಳಿತಿದ್ದರು. ಸಹಸ್ರಾರು ಜನ ಹೊರಗೆ ಇದ್ದರು. ಅಂದಿನ ಸಮಾರಂಭಕ್ಕೆ ರಾಜ ವಿನಯಕೃಷ್ಣದೇವ್ ಬಹದ್ದೂರ್ ಅವರು ಅಧ್ಯಕ್ಷರಾಗಿದ್ದರು. ಅವರು ಭರತಖಂಡದಲ್ಲಿ ಸ್ವಾಮೀಜಿ ಅತ್ಯಂತ ಪ್ರಖ್ಯಾತರಾದ ವ್ಯಕ್ತಿಗಳೆಂದು ಸಭಿಕರಿಗೆ ಪರಿಚಯ ಮಾಡಿಸಿದರು. ಅಂದಿನ ಸಮಾರಂಭದಲ್ಲಿ ರಾಜ ಮಹಾರಾಜರುಗಳು, ಸಂನ್ಯಾಸಿಗಳು, ಪ್ರಮುಖ ಐರೋಪ್ಯ ಅಧಿಕಾರಿಗಳು, ವಿದ್ವಾಂಸರು, ವಿದ್ಯಾರ್ಥಿಗಳು ಮತ್ತು ಇತರ ಪ್ರಮುಖ ಸಾರ್ವಜನಿಕ ವ್ಯಕ್ತಿಗಳೆಲ್ಲ ಹಾಜರಿದ್ದರು. ಸ್ವಾಮೀಜಿಯವರೆದುರಿಗೆ ಬಿನ್ನವತ್ತಳೆಯನ್ನು ಓದಿ ಅದನ್ನು ಒಂದು ಬೆಳ್ಳಿಯ ಕರಂಡದಲ್ಲಿಟ್ಟು ಅರ್ಪಿಸಲಾಯಿತು. ಸ್ವಾಮೀಜಿ ಅದಕ್ಕೆ ತಮ್ಮ ಅತ್ಯಂತ ಸ್ಫೂರ್ತಿಯುತವಾದ ಭಾಷಣದ ಮೂಲಕ ಉತ್ತರ ಇತ್ತರು. ಸ್ವಾಮೀಜಿಯವರು ಆಗ ಮಾಡಿದ ಭಾಷಣ, ಉಪನ್ಯಾಸ ಕಲೆಯಲ್ಲಿ ಉತ್ತುಂಗ ಶಿಖರಕ್ಕೆ ಸೇರಿದ ಮಹಾಭಾಷಣ. ಕೆಳಗೆ ಅದರ ಸ್ವಲ್ಪ ಸಾರವನ್ನು ಮಾತ್ರ ಕೊಡುವೆವು: 

 ಮನುಷ್ಯ ಯಾವುದನ್ನು ಮರೆತರೂ ಜನನಿ ಮತ್ತು ಜನ್ಮಭೂಮಿ ಇದನ್ನು ಮರೆಯುವುದಕ್ಕಾಗುವುದಿಲ್ಲ. ತಾವು ಕಲ್ಕತ್ತೆಗೆ ಬಂದುದು ತಮಗೆ ಮಹದಾನಂದವನ್ನು ಕೊಡುತ್ತದೆ ಎಂದು ಹೇಳಿದರು. ಅಮೇರಿಕಾ ದೇಶದ ಜನ ಮತ್ತು ಇಂಗ್ಲೆಂಡಿನವರು ಆದರದಿಂದ ಸ್ವಾಗತಿಸಿ ಎಲ್ಲಾ ಸೌಕರ‍್ಯಗಳನ್ನು ಕೊಟ್ಟು ವೇದಾಂತ ಬೋಧನೆಗೆ ಅವಕಾಶ ಮಾಡಿದ್ದಕ್ಕಾಗಿ ಅವರು ಅಭಿನಂದನೆಗೆ ಅರ್ಹರು ಎಂದರು. ಸ್ವಾಮೀಜಿ ಅವರಿಗೆ ಅರ್ಪಿಸಿದ ಬಿನ್ನವತ್ತಳೆಯಲ್ಲಿ ಅವರ ಗುರುಗಳಾದ ಶ‍್ರೀರಾಮಕೃಷ್ಣರ ಹೆಸರನ್ನು ಎತ್ತಿದ್ದರು. ಅದಕ್ಕೆ ಉತ್ತರವಾಗಿ ಅವರು ಹೀಗೆ ಹೇಳಿದರು: “ಸಹೋದರರೆ, ನನ್ನ ಹೃದಯದ ಮತ್ತೊಂದು ನಾಡಿಯನ್ನು ಮಿಡಿದಿರುವಿರಿ. ಗಂಭೀರತಮವಾದುದು ಅದು. ಅದೇ ನನ್ನ ಗುರುದೇವ, ನನ್ನ ಜೀವನಾದರ್ಶ, ನನ್ನ ಇಷ್ಟ, ನನ್ನ ಪ್ರಾಣ, ನನ್ನ ದೇವರಾದ ಶ‍್ರೀರಾಮಕೃಷ್ಣಪರಮಹಂಸರ ಪವಿತ್ರ ನಾಮೋಚ್ಚಾರಣೆಯನ್ನು ಮಾಡಿರುವಿರಿ. ನಾನು ಮನೋವಾಕ್ಕಾಯವಾಗಿ ಯಾವುದಾದರೂ ಸತ್ಕರ್ಮವನ್ನು ಮಾಡಿದ್ದರೆ, ನನ್ನ ಬಾಯಿಯಿಂದ ಯಾರಿಗಾದರೂ ಸಹಾಯವಾಗುವಂತಹ ನುಡಿಯೊಂದು ಹೊರಟಿದ್ದರೆ, ಅದು ನನ್ನದಲ್ಲ, ಅದೆಲ್ಲ ಅವರದು. ಆದರೆ ನನ್ನ ಬಾಯಿಂದ ನಿಂದೆಯ ನುಡಿ ಹೊರಟಿದ್ದರೆ, ದ್ವೇಷದ ಭಾವನೆ ವ್ಯಕ್ತವಾಗಿದ್ದರೆ ಅದೆಲ್ಲ ನನ್ನದು, ಅವರದಲ್ಲ. ದುರ್ಬಲವಾಗಿರುವುದೆಲ್ಲ ನನ್ನದು. ಯಾವುದು ಜೀವನಪ್ರದವೊ, ಪವಿತ್ರವಾಗಿರುವುದೋ ಅದೆಲ್ಲ ಅವರ ಶಕ್ತಿಯ ಲೀಲೆ, ಅವರ ವಾಣಿ. ಸ್ವಯಂ ಅವರೇ ಆಗಿರುವರು. ಭರತಖಂಡದಲ್ಲಿ ಹಲವಾರು ಶತಮಾನಗಳಿಂದ ವ್ಯಕ್ತವಾಗದ ಒಂದು ಪ್ರಚಂಡ ಶಕ್ತಿ ಇಲ್ಲಿ ಆವಿರ್ಭೂತವಾಗಿದೆ. ಭರತಖಂಡದ ಪುನರುದ್ಧಾರಕ್ಕೆ ಅದರ ಹಿತಕ್ಕೆ ಮತ್ತು ಇಡೀ ವಿಶ್ವದ ಹಿತಕ್ಕೆ, ಈ ಒಂದು ವ್ಯಕ್ತಿ ಏನು ಮಾಡಿತು ಎಂಬುದನ್ನು ತಿಳಿದುಕೊಳ್ಳುವುದು ಅತ್ಯಾವಶ್ಯಕ. ಪ್ರಪಂಚದ ಇತರ ಕಡೆಗಳಲ್ಲಿ ವಿಶ್ವಧರ್ಮ ಮತ್ತು ಅನ್ಯಧರ್ಮಗಳಿಗೆ ಸಹಾನುಭೂತಿ ಎನ್ನುವುದನ್ನು ಆಲೋಚಿಸುವದಕ್ಕೆ ಮುಂಚೆಯೇ, ಇಲ್ಲೆ, ನಮ್ಮ ನಗರದಲ್ಲೆ, ಯಾರ ಜೀವನವೇ ಒಂದು ವಿಶ್ವಧರ್ಮ ಸಮ್ಮೇಳನದಂತೆ ಇತ್ತೋ ಅಂತಹ ಮಹಾಪುರುಷರು ವಾಸಿಸುತ್ತಿದ್ದರು. ಇವರು ನಾವು ನೋಡಿರುವ ವ್ಯಕ್ತಿಗಳೆಲ್ಲಕ್ಕಿಂತಲೂ ಪವಿತ್ರತಮರು. ಇದನ್ನು ಇನ್ನೂ ಸ್ವಲ್ಪ ವಿಶದವಾಗಿ ಹೇಳುತ್ತೇನೆ, ನೀವು ಓದಿರುವ ವ್ಯಕ್ತಿಗಳೆಲ್ಲಕ್ಕಿಂತಲೂ ಪವಿತ್ರೋತ್ತಮರು. ನೀವು ಇದುವರೆಗೆ ಓದಿದ, ನೋಡುವುದಕ್ಕೆ ಕೂಡ ಅಸಾಧ್ಯವಾದ ಆತ್ಮಶಕ್ತಿಯ ಅದ್ಭುತ ಆವಿರ್ಭಾವ ನಮ್ಮ ಕಣ್ಣೆದುರಿಗೆ ಇದೆ.” 

 ಸ್ವಾಮೀಜಿ ತಮ್ಮ ಗುರುಗಳ ವಿಷಯವಾಗಿ ಹೊರಗೆ ಅಷ್ಟು ಮಾತನಾಡುತ್ತಿರಲಿಲ್ಲ. ಆದರೆ ಕಲ್ಕತ್ತ ಪರಮಹಂಸರ ಲೀಲಾಭೂಮಿ, ಪರಮಹಂಸರ ಪ್ರತ್ಯಕ್ಷ ಶಿಷ್ಯರು ಎದುರಿಗಿದ್ದರು. ಅವರೆದುರಿಗೆ ಸ್ವಾಮೀಜಿ ತಮ್ಮ ಹೃದಯದ ಅತ್ಯಂತ ಪವಿತ್ರವಾದ ಭಾವನೆಗಳನ್ನು ವ್ಯಕ್ತಪಡಿಸುವರು. “ನನ್ನ ದೃಷ್ಟಿಯಿಂದ ಪರಮಹಂಸರನ್ನು ಅಳೆಯಬೇಡಿ.\break ನಾನೊಂದು ದುರ್ಬಲ ವ್ಯಕ್ತಿ ಮಾತ್ರ. ಅವರು ತಮ್ಮ ಕೆಲಸಕ್ಕೆ ಸಹಸ್ರಾರು ಜನರನ್ನು ಧೂಳಿನಿಂದ ಬೇಕಾದರೆ ಸೃಷ್ಟಿಸಬಲ್ಲರು.” ಶಿಷ್ಯರಾದ ವಿವೇಕಾನಂದರು ತಮ್ಮ ಪರಮಗುರುಗಳಿಗೆ ಕೊಡುವ ಕಾಣಿಕೆ ಇದು. ವಿಶ್ವವೇ ವಿವೇಕಾನಂದರನ್ನು ಮೆಚ್ಚುತ್ತಿತ್ತು. ಆದರೆ ವಿವೇಕಾನಂದರೇ ತಮ್ಮ ಗುರುಗಳೆದುರಿಗೆ ತಾವು ಧೂಳಿಗೆ ಸಮವೆಂದು ಭಾವಿಸುವರು. 

 ಭಾರತೀಯರ ವಿಕಾಸವಾಗಬೇಕು. ವಿಕಾಸವೇ ಜೀವನ, ಸಂಕೋಚವೇ ಮರಣ. ತಮ್ಮ ಭಾವನೆಗಳನ್ನು ಹೊರಗೆ ವ್ಯಕ್ತಪಡಿಸಬೇಕು. ಹೊರಗೆ ಹೋಗಿ ನಾವು ಅನ್ಯ ರಾಷ್ಟ್ರಗಳಿಂದ ಒಳ್ಳೆಯದನ್ನು ಕಲಿತುಕೊಳ್ಳಬೇಕು. ನಮ್ಮಲ್ಲಿರುವ ಅನರ್ಘ್ಯ ಅಧ್ಯಾತ್ಮ ವಿಷಯಗಳನ್ನು\break ಕೊಡಬೇಕು. ಕೊಟ್ಟು ತೆಗೆದುಕೊಳ್ಳುವ ನಿಯಮದ ಮೇಲೆ ವಿಶ್ವ ನಿಂತಿದೆ. ನಾವು\break ಯಾವಾಗಲೂ ಪರರಿಂದ ಕಲಿಯುವುದಕ್ಕೆ, ಬೇಡುವುದಕ್ಕೆ ಭಿಕ್ಷುಕನಂತೆ ಹೊರಗೆ ಹೋಗಕೂಡದು. ನಮಗೂ ಇತರರಿಗೆ ಕೊಡುವುದಕ್ಕೆ ನಮ್ಮ ಪೂರ್ವಿಕರು ಏನನ್ನೊ, ಬಿಟ್ಟುಹೋಗಿರುವರು. ಅದೇ ಅಧ್ಯಾತ್ಮವಿದ್ಯೆ. ಅದನ್ನು ಉದಾರವಾಗಿ ಕೊಡುವ, ಇತರರಿಂದ ಅವರಲ್ಲಿರುವ ಒಳ್ಳೆಯದನ್ನು ಕಲಿಯುವ. 

 ಯಮನನ್ನು ಸಂಧಿಸಿದ ನಚಿಕೇತನಿಗೆ ಯಾವ ಶ್ರದ್ಧೆ ಇತ್ತೋ ಅದನ್ನು ನಾವು ರೂಢಿಸಿಕೊಳ್ಳಬೇಕಾಗಿದೆ. ಶ್ರದ್ಧೆಯ ಅಭಾವವೇ ನಮ್ಮ ಅಧೋಗತಿಗೆಲ್ಲ ಕಾರಣ. ಅದನ್ನು ಪುನಃ ನಮ್ಮ ಜೀವನದಲ್ಲಿ ಜ್ವಲಂತಗೊಳಿಸಬೇಕು ಎಂದು ಹೇಳಿದರು. 

 ಸ್ವಾಮೀಜಿ ಕಲ್ಕತ್ತೆಗೆ ಬಂದು ಕೆಲವು ದಿನಗಳು ಆದಮೇಲೆ ಶ‍್ರೀರಾಮಕೃಷ್ಣರ ಜನ್ಮದಿನ ಬಂದಿತು. ಅದನ್ನು ಶ‍್ರೀರಾಮಕೃಷ್ಣರ ಲೀಲಾ ನಾಟಕದ ರಂಗಭೂಮಿಯಾದ ದಕ್ಷಿಣೇಶ್ವರದಲ್ಲಿ ಆಚರಿಸಿದರು. ಸ್ವಾಮೀಜಿ ತಾವೇ ಖುದ್ದಾಗಿ ಅದರಲ್ಲಿ ಭಾಗವಹಿಸುತ್ತಾರೆ ಎಂದು ಜನರಿಗೆ ಗೊತ್ತಾದ ಮೇಲೆ ಸಹಸ್ರಾರು ಜನ ದಕ್ಷಿಣೇಶ್ವರಕ್ಕೆ ಬಂದರು. ಸ್ವಾಮೀಜಿಯವರು ಕೆಲವು ಗುರುಭಾಯಿಗಳೊಡನೆ ಸುಮಾರು ಒಂಭತ್ತು ಗಂಟೆ ಹೊತ್ತಿಗೆ ಬಂದರು. ಬರಿಯ ಕಾಲು, ತಲೆಗೆ ಕಾವಿಯ ರುಮಾಲು. ಜನಸಂಘ ಅವರನ್ನು ನೋಡುತ್ತ ಅಲ್ಲಿಂದಿಲ್ಲಿಗೆ ಓಡಾಡುತ್ತಿದೆ. ಅವರ ಆನಂದಿತವಾದ ಮುಖವನ್ನು ದರ್ಶನ ಮಾಡುವುದಕ್ಕೆ ಅವರ ಪಾದಪದ್ಮಗಳನ್ನು ಸ್ಪರ್ಶ ಮಾಡುವುದಕ್ಕೆ, ಅವರ ಬಾಯಿಂದ ಪ್ರಜ್ವಲಿಸುತ್ತಿರುವ ಅಗ್ನಿಯ ಜ್ವಾಲೆಯಂತಿರುವ ಮಾತನ್ನು ಕೇಳುವುದಕ್ಕೆ ಜನ ಕುತೂಹಲಿಗಳಾಗಿರುವರು. 

 ಕಾಳೀ ಮಾತೆಯ ಮಂದಿರದ ಮುಂದೆ ಅಸಂಖ್ಯಾತ ಜನ. ಸ್ವಾಮೀಜಿ ಜಗನ್ಮಾತೆಗೆ ನೆಲದ ಮೇಲೆ ಉದ್ದಕ್ಕೆ ದಂಡಪ್ರಣಾಮ ಮಾಡಿದರು. ಜೊತೆಯಲ್ಲಿಯೇ ಸಾವಿರಾರು ತಲೆಗಳು ಬಗ್ಗಿದವು. ಆಮೇಲೆ ರಾಧಾಕಾಂತ ಸ್ವಾಮಿಗೆ ಪ್ರಣಾಮ ಮಾಡಿ ಅವರು ಪರಮಹಂಸರ ವಾಸಗೃಹಕ್ಕೆ ಬಂದರು. ಆ ತೊಟ್ಟಿಯಲ್ಲಿ ಆಗ ಒಂದು ಎಳ್ಳು ಹಾಕಿದರೆ ಹಿಡಿಸುವಷ್ಟು ಜಾಗವೂ ಇಲ್ಲ. ‘ಜಯ ರಾಮಕೃಷ್ಣ’ ಎಂಬ ಧ್ವನಿಯಿಂದ ಕಾಳಿ ದೇವಸ್ಥಾನದ ದಿಕ್ಕುಗಳೆಲ್ಲವೂ ಮುಖರಿತವಾಗುತ್ತಿದೆ. ಸಾವಿರಾರು ನೋಟಕರನ್ನು ಕೂರಿಸಿಕೊಂಡು ಹೋರ್ ಮಿಲ್ಲರ್ ಕಂಪನಿಯ ಜಹಜು ಬಾರಿ ಬಾರಿಗೂ ಕಲ್ಕತ್ತೆಯಿಂದ ಓಡಿಯಾಡುತ್ತಿದೆ. ನಗಾರಿ ನೌಬತ್ತಿನ ವಾದ್ಯಧ್ವನಿಯಿಂದ ಗಂಗಾನದಿ ನರ್ತನಮಾಡುತ್ತಿದೆ. ಉತ್ಸಾಹ ಆಕಾಂಕ್ಷೆ ಧರ್ಮಪಿಪಾಸೆ ಮತ್ತು ಅನುರಾಗ ಇವು ಮೂರ್ತಿವತ್ತಾಗಿ ಶ‍್ರೀರಾಮಕೃಷ್ಣ ಭಕ್ತಿರೂಪದಲ್ಲಿ ಅಲ್ಲಲ್ಲಿ ಬೆಳಗುತ್ತಿದೆ. ಈ ಸಲದ ಈ ಉತ್ಸವ, ಹೃದಯದಲ್ಲಿ ಅನುಭವ ಮಾಡಿಕೊಳ್ಳಬೇಕಾದ ವಿಷಯ, ಮಾತಿನಲ್ಲಿ ವ್ಯಕ್ತಪಡಿಸಲು ಅಸಾಧ್ಯ. 

 ಸ್ವಾಮೀಜಿಯವರೊಡನೆ ಬಂದಿದ್ದ ಮದ್ರಾಸಿನ ಭಕ್ತರು ಮತ್ತು ಸೇವಿಯರ್ಸ್‍‍ ದಂಪತಿಗಳು ಮತ್ತು ಗುಡ್‍ವಿನ್ ಅಲ್ಲಿ ಇದ್ದರು. ಸ್ವಾಮೀಜಿಯವರು ಶಿಷ್ಯರನ್ನು ಕರೆದುಕೊಂಡುಹೋಗಿ ಪವಿತ್ರವಾದ ಪಂಚವಟಿ ಮತ್ತು ಬಿಲ್ವಮೂಲಗಳನ್ನು ದರ್ಶನ ಮಾಡಿಸಿದರು. ಪಂಚವಟಿಯ ಒಂದು ಕಡೆಯಲ್ಲಿ ಪರಮಹಂಸರ ಗೃಹಸ್ಥಭಕ್ತರು ಇಳಿದುಕೊಂಡಿದ್ದರು. ಗಿರೀಶಬಾಬುಗಳು ಪಂಚವಟಿ ಉತ್ತರದಲ್ಲಿ ಗಂಗೆಯ ಕಡೆ ತಿರುಗಿಕೊಂಡು ಕುಳಿತಿದ್ದರು ಮತ್ತು ಅವರನ್ನು ಸುತ್ತಿಕೊಂಡು ಮಿಕ್ಕ ಭಕ್ತ ಜನರು ಶ‍್ರೀರಾಮಕೃಷ್ಣರ ಗುಣಗಾನದಲ್ಲಿಯೂ ಕಥಾಪ್ರಸಂಗದಲ್ಲಿಯೂ ವಿಸ್ಮೃತರಾಗಿ ಕುಳಿತುಕೊಂಡಿದ್ದರು. ಆ ಸಮಯದಲ್ಲಿ ಸ್ವಾಮೀಜಿ ಬಹುಜನರೊಡನೆ ನಮಸ್ಕಾರ ಮಾಡಿದರು. ಗಿರೀಶಬಾಬುಗಳು ಅವರಿಗೆ ಕೈಜೋಡಿಸಿ ಪ್ರತಿನಮಸ್ಕಾರ ಮಾಡಿದರು. ಸ್ವಾಮೀಜಿ ಗಿರೀಶಬಾಬುಗಳಿಗೆ ಹಿಂದಿನ ವೃತ್ತಾಂತವನ್ನು ಜ್ಞಾಪಿಸಿ “ಘೋಷಜರೆ ಅದೇ ಒಂದು ಕಾಲ” ಎಂದರು. ಗಿರೀಶಬಾಬು ಸ್ವಾಮೀಜಿಗೆ, ಅದೇನೊ ನಿಜ, ಆದರೆ ಈಗಲೂ ಅದನ್ನೇ ಹೆಚ್ಚಾಗಿ ನೋಡಬೇಕೆಂದು ಹಂಬಲಿಸುತ್ತದೆ ಎಂದರು. ಸ್ವಲ್ಪ ಹೊತ್ತು ಮಾತುಕತೆಯಾದ ಮೇಲೆ ಸ್ವಾಮೀಜಿ ಪಂಚವಟಿಯ ಕಡೆ ಹೋದರು. 

\vskip 3pt

 ಗಿರೀಶಘೋಷರು ಅನಂತರ ನೆರೆದಿದ್ದ ಭಕ್ತಮಂಡಲಿಯನ್ನು ಉದ್ದೇಶಿಸಿ ಹೀಗೆ\break ಹೇಳಿದರು: “ಒಂದು ದಿನ ಹರಮೋಹನ ಮಿತ್ರರು, ವರ್ತಮಾನ ಪತ್ರಿಕೆಯನ್ನು ನೋಡಿಕೊಂಡು ಬಂದು ಅಮೇರಿಕಾದಲ್ಲಿ ಸ್ವಾಮಿಗಳ ವಿಷಯವಾಗಿ ಅಪಯಶಸ್ಸು ಹರಡಿದೆ ಎಂದು ಹೇಳಿದರು. ಆಗ ನಾನು ಅವರಿಗೆ ನರೇನನು ಏನಾದರೂ ಕೆಟ್ಟದ್ದು ಮಾಡಿದರೆ ಅದನ್ನು ನಾನು ಕಣ್ಣಿನಿಂದ ನೋಡಿದರೂ ಆಗ ಅದು ನನ್ನ ಕಣ್ಣಿನ ದೋಷವೆಂದು ಹೇಳಿಬಿಡುತ್ತೇನೆ, ಕಣ್ಣಿನ ಮೇಲೆ ಹಾಕಿಬಿಡುತ್ತೇನೆ. ಅವರು ಹೊತ್ತು ಹುಟ್ಟುವುದಕ್ಕೆ ಮುಂಚೆ ತೆಗೆದ ಬೆಣ್ಣೆ. ಅವರು ನೀರಿನಲ್ಲಿ ಕರಗಿಹೋಗುತ್ತಾರೇನು?

\vskip 3pt

 “ಅವರಲ್ಲಿ ಯಾರಾದರೂ ಯಾವುದಾದರೂ ದೋಷವನ್ನು ಎತ್ತಿ ಆಡಿದರೆ ಅಂಥವನಿಗೆ ನರಕ ಪ್ರಾಪ್ತಿಯಾಗುತ್ತದೆ ಎಂದು ಹೇಳಿದೆನು” ಎಂದರು. 

\vskip 3pt

 ಸ್ವಾಮೀಜಿಯವರು ಕೆಲವು ದಿನಗಳಾದ ಮೇಲೆ ವೇದಾಂತ ದರ್ಶನದ ವಿವಿಧ ಮುಖಗಳು ಎಂಬ ವಿಷಯದ ಮೇಲೆ ಮಾತನಾಡಿದರು. ಅದರಲ್ಲಿ ಹಿಂದೂಧರ್ಮದ ತಳಪಾಯವೆ ವೇದ ಮತ್ತು ಉಪನಿಷತ್ತೆಂದೂ, ಎಲ್ಲರೂ ಉಪನಿಷತ್ತನ್ನು ತಿಳಿದುಕೊಳ್ಳಬೇಕೆಂದೂ ಹೇಳಿದರು. ಜನರು ಅದನ್ನು ತಿಳಿದುಕೊಳ್ಳುವುದನ್ನು ಬಿಟ್ಟು ನಮ್ಮನ್ನು ದುರ್ಬಲರು ದುರಾಚಾರಿಗಳನ್ನಾಗಿ ಮಾಡುವ ವಾಮಾಚಾರವನ್ನು ಅನುಸರಿಸುವುದನ್ನು ಖಂಡಿಸಿದರು. 

\vskip 3pt

 ಪರಮಹಂಸರ ಗೃಹಸ್ಥ ಭಕ್ತರಾದ ನವಗೋಪಾಲ ಘೋಷರು ಒಂದು ಹೊಸಮನೆಯನ್ನು ಕಟ್ಟಿ ಅಲ್ಲಿಯೇ ದೇವರ ಮನೆಯಲ್ಲಿ ಶ‍್ರೀರಾಮಕೃಷ್ಣರನ್ನು ಪ್ರತಿಷ್ಠೆ ಮಾಡುವುದಕ್ಕಾಗಿ ಸ್ವಾಮೀಜಿ ಮತ್ತು ಅವರ ಗುರುಭಾಯಿಗಳನ್ನು ಕರೆದರು. ಅಂದಿನ ದಿನ ಮೂರು ದೋಣಿಗಳನ್ನು ಬಾಡಿಗೆಗೆ ಗೊತ್ತುಮಾಡಿಕೊಂಡು ಸ್ವಾಮೀಜಿಗಳೊಡನೆ ಮಠದ ಸಂನ್ಯಾಸಿಗಳು ಮತ್ತು ಬ್ರಹ್ಮಚಾರಿಗಳು ಎಲ್ಲರೂ ರಾಮಕೃಷ್ಣಪುರದ ಘಾಟಿಗೆ ತಲುಪಿದರು. ಸ್ವಾಮಿಗಳ ಪೋಷಾಕೆಲ್ಲ ಕಾವಿಯ ರಂಗಿನ ಪಂಚೆ, ತಲೆಗೆ ಒಂದು ಕುಲಾವಿ ಇಷ್ಟೆ. ಕಾಲಿನಲ್ಲಿ ಏನೂ ಇಲ್ಲ. ರಾಮಕೃಷ್ಣಪುರದ ಘಾಟಿನಿಂದ ಅವರು ಯಾವ ದಾರಿಯಲ್ಲಿ ನವಗೋಪಾಲ ಬಾಬು ಅವರ ಮನೆಗೆ ಹೋಗುವವರಾಗಿದ್ದರೊ ಆ ದಾರಿಯ ಎರಡು ಕಡೆಯಲ್ಲಿಯೂ ಲೆಕ್ಕಿಸಲಾರದಷ್ಟು ಜನರು ಅವರ ದರ್ಶನಾಕಾಂಕ್ಷಿಗಳಾಗಿ ನಿಂತಿದ್ದರು. ಘಾಟಿನಲ್ಲಿ ಇಳಿಯುತ್ತಿದ್ದ ಹಾಗೆಯೇ ಸ್ವಾಮೀಜಿ

\begin{verse}
“ದುಃಖನೀ ಬ್ರಾಹ್ಮಣೀ ಕೋಲೇ ಕೇ ಶುಯೇಛೆ~।\\ಆಲೋ ಕರೇ ಕೇರೆ ಓರೇ ದಿಗಂಬರ ಏಸೆಛೆ ಕುಟೀರ ಘರೆ~॥”
\end{verse}

 ಎಂಬ ಕೀರ್ತನೆಯನ್ನು ಆರಂಭಿಸಿ ತಾವೇ ಮೃದಂಗವನ್ನು ಬಾರಿಸುತ್ತ ಹೊರಟರು. ಇನ್ನಿಬ್ಬರು ಮೂರು ಜನರೂ‌ ಮೃದಂಗವನ್ನು ಜೊತೆಯಲ್ಲಿ ಬಾರಿಸಲು ಮೊದಲು ಮಾಡಿದರು. ಅಲ್ಲಿ ಸೇರಿದ್ದ ಭಕ್ತರೆಲ್ಲರೂ‌ ಸಮಸ್ವರದಲ್ಲಿ ಆ ಕೀರ್ತನೆಯನ್ನು ಹಾಡುತ್ತ ಅವರ ಹಿಂದೆ ಹೊರಟರು. ಉದ್ದಾಮ ನೃತ್ಯದಿಂದಲೂ ಮೃದಂಗ ಧ್ವನಿಯಿಂದಲೂ ರಸ್ತೆಯು ಮುಖರಿತವಾಗುತ್ತಿತ್ತು. ಹೋಗುತ್ತ ಹೋಗುತ್ತ ಈ ಗುಂಪು ಶ‍್ರೀಯುತ ರಾಮಲಾಲ್ ಬಾಬುಗಳ ಮನೆಯ ಹತ್ತಿರ ಸ್ವಲ್ಪ ಹೊತ್ತು ನಿಂತುಕೊಂಡಿತು. ರಾಮಲಾಲ ಬಾಬುಗಳೂ ಬಹುಬೇಗ ಮನೆಯಿಂದ ಹೊರಕ್ಕೆ ಓಡಿ ಬಂದು ಜೊತೆಯಲ್ಲಿ ಹೊರಟರು. ಜನರು ಮನಸ್ಸಿನಲ್ಲಿ ಮಾಡಿಕೊಂಡಿದ್ದೇನೆಂದರೆ ಸ್ವಾಮೀಜಿ ಎಷ್ಟೋ ವೇಷಭೂಷಣಗಳಿಂದ ಅಲಂಕೃತರಾಗಿ ಆಡಂಬರದಿಂದ ಹೋಗುತ್ತಾರೆಂದು. ಆದರೆ ಅವರು ಮಠದ ಇತರ ಸಂನ್ಯಾಸಿಗಳಂತೆ ಸಾಮಾನ್ಯವಾದ ಉಡುಪಿನಲ್ಲಿಯೂ ಬರಿಯ ಕಾಲಿನಲ್ಲಿಯೂ ಮೃದಂಗವನ್ನು ಬಾರಿಸುತ್ತ ಬಾರಿಸುತ್ತ ಬಂದದ್ದನ್ನು ನೋಡಿ ಅನೇಕರಿಗೆ ಮೊದಲು ಅವರ ಗುರುತನ್ನು ಹಿಡಿಯುವುದಕ್ಕೆ ಆಗದೆ ಆಮೇಲೆ ಇತರರನ್ನು ಕೇಳಿ ತಿಳಿದುಕೊಂಡು “ಇವರೇನೆ ಪ್ರಪಂಚವನ್ನು ಗೆದ್ದ ವಿವೇಕಾನಂದ ಸ್ವಾಮಿಗಳು” ಎಂದು ಅಚ್ಚರಿಪಟ್ಟರು. ಸ್ವಾಮೀಜಿಯ ಆ ಅಪೂರ್ವ ನಮ್ರತೆಯನ್ನು ನೋಡಿ, ಎಲ್ಲರೂ ಅವರನ್ನು ಒಟ್ಟಿಗೆ ಹೊಗಳುವುದಕ್ಕೂ “ಜಯ ರಾಮಕೃಷ್ಣ” ಧ್ವನಿಯಿಂದ ಹೋಗುತ್ತಿದ್ದ ದಾರಿಯನ್ನು ನಡುಗಿಸುವುದಕ್ಕೂ ಮೊದಲು ಮಾಡಿದರು. 

\vskip 2pt

 ಗೃಹಸ್ಥರಿಗೆ ಆದರ್ಶ ಸ್ವರೂಪವಾದ ನವಗೋಪಾಲ ಬಾಬುಗಳ ಹೃದಯ ಇಂದು ಆನಂದದಿಂದ ತುಂಬಿಹೋಗಿದೆ. ಅವರು ಪರಮಹಂಸರ ಪೂಜೆಗೋಸ್ಕರ ಯಥೇಷ್ಟವಾಗಿ ಸಾಮಾನುಗಳನ್ನು ಸಿದ್ಧಪಡಿಸಿಕೊಂಡು ನಾಲ್ಕು ಕಡೆಗೂ ಸಡಗರದಿಂದ ಓಡಿಯಾಡುತ್ತ ವಿಚಾರಣೆಯನ್ನು ತೆಗೆದುಕೊಳ್ಳುತ್ತ ಮಧ್ಯೆ ಮಧ್ಯೆ ಉಲ್ಲಾಸದಿಂದ ‘ಜಯರಾಮ್,\break ಜಯರಾಮ್’ ಎಂದು ಉದ್ಘೋಷಿಸುತ್ತ ಇದ್ದರು. 

\vskip 2pt

 ಕ್ರಮವಾಗಿ ಗುಂಪು ನವಗೋಪಾಲ ಬಾಬುಗಳ ಮನೆಯ ಬಾಗಿಲಿಗೆ ಬಂದ ಕೂಡಲೆ ಮನೆಯ ಒಳಗೆ ಶಂಖಧ್ವನಿಯೂ ಆಯಿತು. ಸ್ವಾಮೀಜಿ ಮೃದಂಗವನ್ನು ಕೆಳಗಿಟ್ಟು ಬೈಠಕ್‍ಖಾನೆಯಲ್ಲಿ ಸ್ವಲ್ಪ ಕಾಲ ವಿಶ್ರಮಿಸಿಕೊಂಡು ಅನಂತರ ಪೂಜಾ ಗೃಹವನ್ನು ನೋಡಲು ಮೇಲಕ್ಕೆ ಹೋದರು. ಪೂಜಾಗೃಹವು ಅಮೃತಶಿಲೆಯಿಂದ ರಚಿತವಾಗಿತ್ತು. ಮಧ್ಯದಲ್ಲಿ ಸಿಂಹಾಸನ, ಅದರ ಮೇಲೆ ಪರಮಹಂಸರ ಪೋರ್ಸಿ‍‍ಲೇನಿನ ಮೂರ್ತಿ. ಹಿಂದೂಗಳ ದೇವರ ಪೂಜೆಗೆ ಯಾವ ಯಾವ ಉಪಕರಣಗಳು ಬೇಕೋ ಅವೆಲ್ಲ ಸಿದ್ಧವಾಗಿದ್ದವು. ಯಾವ ಭಾಗದಲ್ಲಿಯೂ ಏನೂ ಕಡಿಮೆಯಾಗಿರಲಿಲ್ಲ. ಸ್ವಾಮೀಜಿ ಇದನ್ನು ನೋಡಿ ಸಂತುಷ್ಟರಾದರು. 

\vskip 2pt

 ನವಗೋಪಾಲ ಬಾಬುಗಳ ಪತ್ನಿ ಇತರ ಹೆಂಗಸರೊಡನೆ ಸ್ವಾಮಿಗಳಿಗೆ ಸಾಷ್ಟಾಂಗ ಪ್ರಣಾಮ ಮಾಡಿ ಬೀಸಣಿಗೆಯನ್ನು ತೆಗೆದುಕೊಂಡು ಅವರಿಗೆ ಗಾಳಿ ಹಾಕುವುದಕ್ಕೆ ಮೊದಲು ಮಾಡಿದರು. ಸ್ವಾಮೀಜಿ ಬಾಯಿಂದ ಎಲ್ಲಾ ವಿಚಾರಗಳ ಪ್ರಶಂಸೆಯನ್ನೂ ಕೇಳಿ ಮನೆಯ ಯಜಮಾನಿಯು ಅವರನ್ನು ಸಂಬೋಧಿಸಿ, “ನಾವು ಪರಮಹಂಸರ ಸೇವೆಯನ್ನು ಮಾಡುವ ಅಧಿಕಾರವನ್ನು ಪಡೆಯುವುದೆಂದರೇನು? ಸಾಮಾನ್ಯವಾದ ಮನೆ, ಸಾಮಾನ್ಯವಾದ\break ಸಂಪತ್ತು. ತಾವು ಈ ದಿವಸ ಸ್ವಂತ ನಿಂತುಕೊಂಡು ಕೃಪೆಮಾಡಿ ಪರಮಹಂಸರ ಪ್ರತಿಷ್ಠೆಯನ್ನು ಮಾಡಿಸಿ ನಮ್ಮನ್ನು ಧನ್ಯರನ್ನಾಗಿ ಮಾಡಬೇಕು” ಎಂದು ಕೋರಿಕೊಂಡರು. 

\vskip 2pt

 ಸ್ವಾಮೀಜಿಯವರು ಅದಕ್ಕೆ ಉತ್ತರವಾಗಿ ಪರಿಹಾಸ್ಯ ಮಾಡುತ್ತ, “ನಮ್ಮ ಪರಮಹಂಸರು ಇಂತಹ ಅಮೃತಶಿಲೆಯ ಮಹಡಿಯಲ್ಲಿ ಹದಿನಾಲ್ಕು ತಲೆಗಳಿಂದ ವಾಸ ಮಾಡಿಕೊಂಡು ಬರಲಿಲ್ಲ. ಹುಟ್ಟಿದ್ದು ಒಂದು ಹಳ್ಳಿಯ ಗುಡಿಸಲಲ್ಲಿ. ಹಾಗೂ ಹೀಗೂ ದಿನಗಳು ಕಳೆದಿದ್ದಾಯಿತು. ಇಲ್ಲಿ ಇಂಥ ಉತ್ತಮವಾದ ಸೇವೆಯಲ್ಲಿ ಇರದಿದ್ದರೆ ಮತ್ತೆಲ್ಲಿ ಇರುವರು” ಎಂದರು. ಎಲ್ಲರೂ ಸ್ವಾಮಿಗಳ ಮಾತನ್ನು ಕೇಳಿ ನಗುವುದಕ್ಕೆ ಮೊದಲು ಮಾಡಿದರು. ಈಗ ವಿಭೂತಿ ಭೂಷಿತಾಂಗರಾದ ಸ್ವಾಮೀಜಿ ಸಾಕ್ಷಾತ್ ಮಹಾದೇವನ ಹಾಗೆ ಪೂಜಕರ ಆಸನದಲ್ಲಿ ಕುಳಿತು ಪರಮಹಂಸರನ್ನು ಆವಾಹನೆ ಮಾಡುವುದಕ್ಕೆ ಹೊರಟರು. 

 ಪ್ರಕಾಶಾನಂದ ಸ್ವಾಮಿಗಳು ಸ್ವಾಮೀಜಿ ಹತ್ತಿರ ಕುಳಿತುಕೊಂಡು ಮಂತ್ರ ಮುಂತಾದವುಗಳನ್ನು ಹೇಳುತ್ತಿದ್ದರು. ಪೂಜೆಯ ನಾನಾ ಅಂಗಗಳು ಕ್ರಮವಾಗಿ ಮುಗಿದವು. ಮಂಗಳಾರತಿ ಶಂಖ ಗಂಟೆಗಳ ಧ್ವನಿ ಮೊದಲಾಯಿತು. ಪ್ರಕಾಶಾನಂದ ಸ್ವಾಮಿಗಳೇ ಅದನ್ನು ಬಾರಿಸಿದರು. ಮಂಗಳಾರತಿ ಆದಮೇಲೆ ಸ್ವಾಮೀಜಿ ಪೂಜಾಗೃಹದಲ್ಲಿ ಕುಳಿತುಕೊಂಡಿದ್ದ ಹಾಗೆ ಶ‍್ರೀರಾಮಕೃಷ್ಣ ದೇವರಿಗೆ ನಮಸ್ಕಾರ ಮಾಡುವ ಮಂತ್ರವನ್ನು ಬಾಯಲ್ಲಿ ರಚಿಸಿ ಹೀಗೆ ಹೇಳಿಬಿಟ್ಟರು:

\begin{verse}
“ಸ್ಥಾಪಕಾಯ ಚ ಧರ್ಮಸ್ಯ ಸರ್ವಧರ್ಮಸ್ವರೂಪಿಣೇ~। \\ಅವತಾರವರಿಷ್ಠಾಯ ರಾಮಕೃಷ್ಣಾಯ ತೇ ನಮಃ”~॥ 
\end{verse}

 ಅನಂತರ ಎಲ್ಲರೂ ಮನೆಗೆ ಹಿಂದಿರುಗಿದರು. 

