
\chapter{ಮಾಯಾವತಿ ಮತ್ತು ಪೂರ್ವ ಬಂಗಾಳಕ್ಕೆ }

 ಮಾಯಾವತಿ ಅದ್ವೈತ ಆಶ್ರಮ ಸ್ಥಾಪಿಸುವುದಕ್ಕೆ ದ್ರವ್ಯರೂಪದಲ್ಲಿ ಸಹಾಯ ಮಾಡಿದ ಜೆ. ಎಸ್. ಸೇವಿಯರ್ಸ್‍‍ ಅವರು ೧೯೦೦, ಅಕ್ಟೋಬರ್ ೨೮ನೇ ತಾರೀಖು ನಿಧನರಾಗಿದ್ದರು. ಸ್ವಾಮೀಜಿ ಬೇಲೂರು ಮಠಕ್ಕೆ ಬಂದೊಡನೆಯೇ ಸೇವಿಯರ್ಸ್‍‍ ಸತಿಗೆ ತಾವು ಮಾಯಾವತಿಗೆ ಬರುತ್ತೇವೆ ಎಂದು ಒಂದು ತಂತಿಯನ್ನು ಕಳುಹಿಸಿದರು. ಅವರಿಗೆ ಸಾಂತ್ವನ ನುಡಿಗಳನ್ನು ಹೇಳುವುದಕ್ಕಾಗಿ ತ್ವರೆಯಿಂದ ಮಾಯಾವತಿ ಕಡೆಗೆ ಹೊರಟರು. ಅವರ ಜೊತೆಯಲ್ಲಿ ಶಿವಾನಂದ ಮತ್ತು ಸದಾನಂದರು ಇದ್ದರು.\break ಸ್ವಾಮೀಜಿ ಕಾಥ್‍ಗೊಡಾಮ್ ಎಂಬಲ್ಲಿಗೆ ಡಿಸೆಂಬರ್ ೨೮ನೇ ತಾರೀಖು ಬಂದರು. ಅಂದು ಸ್ವಾಮೀಜಿಗೆ ಜ್ವರ ಬಂದಿತು. ಅಂದು ವಿಶ್ರಾಂತಿ ತೆಗೆದುಕೊಂಡು ಮಾರನೆಯ ದಿನ ಕುದುರೆಯ ಮೇಲೆ ಅಲ್ಲಿಂದ ಹೊರಟರು. ಹಿಮಾಲಯದ ರಸ್ತೆಯಲ್ಲಿ ಸುಮಾರು ೬೫ಮೈಲಿಗಳು ಹೋಗಬೇಕಾಗಿತ್ತು. ಸ್ವಾಮೀಜಿ ಜನವರಿ ೩ನೇ ತಾರೀಖು ೧೯೦೧ರಲ್ಲಿ ಮಾಯಾವತಿಯನ್ನು ತಲುಪಿದರು. ಆಶ್ರಮವನ್ನು ಸ್ವಾಮೀಜಿ ಜ್ಞಾಪಕಾರ್ಥವಾಗಿ ತಳಿರುತೋರಣಗಳಿಂದ ಅಲಂಕರಿಸಿದರು. ಸ್ವಾಮೀಜಿ ಅಲ್ಲಿದ್ದಾಗ ಬಹು ದಿನಗಳು ಸುತ್ತಮುತ್ತಲು ಬರೀ ಹಿಮರಾಶಿಯಿಂದ ಮುಚ್ಚಿಹೋಗಿತ್ತು. ಹೊರಗೆ ಸ್ವತಂತ್ರವಾಗಿ ಸಂಚರಿಸಲು ಅವರಿಗೆ ಸಾಧ್ಯವಾಗುತ್ತಿರಲಿಲ್ಲ. ಆದರೂ ಮಠದಲ್ಲಿರುವಾಗ ಅಲ್ಲಿ ಇರುವವರಿಗೆ ಸ್ಫೂರ್ತಿಯಿಂದ ಮಾತನಾಡುತ್ತಿದ್ದರು. ಮೇಲೆ ಇರುವವರು ಕೆಳಗಿನವರಿಗೆ ಬರೀ ಅಪ್ಪಣೆ ಮಾಡಿದರೆ ಮಾತ್ರ ಅದು ಫಲಕಾರಿಯಾಗುವುದಿಲ್ಲ; ಅವರ ಮೇಲೆ ಪ್ರೀತಿ ವಿಶ್ವಾಸಗಳಿದ್ದರೆ ಮಾತ್ರ ಅದು ಪರಿಣಾಮಕಾರಿಯಾಗುವುದು ಎಂದು ಮೇಲಿರುವವರಿಗೆ ಹೇಳಿದರು. ಅದರಂತೆಯೇ ಕೆಳಗೆ ಇರುವವರು ಮೇಲಿರುವವರಿಗೆ ಗೌರವ ತೋರಿಸುವುದು ಅತ್ಯಂತ ಮುಖ್ಯವೆಂದು ಒತ್ತಿ ಹೇಳಿದರು. ಒಂದು ಕೆಲಸದ ಮೇಲ್ವಿಚಾರಕನ ಮಾತಿಗೆ, ನಾವು ಮಾಡುತ್ತಿರುವ ಕೆಲಸಕ್ಕೆ ಮತ್ತು ಸಂಸ್ಥೆಗೆ ನಾವು ಗೌರವ ತೋರಬೇಕೆಂಬುದನ್ನು ಒತ್ತಿ ಹೇಳಿದರು. ಸೇವಿಯರ್ಸ್ ಪರದೇಶದಿಂದ ಬಂದರೂ ತಾವು ಸ್ಥಾಪಿಸಿದ ಕೆಲಸದಲ್ಲಿ ಮಗ್ನರಾಗಿ ಹೇಗೆ ಪ್ರಾಣ ಕೊಟ್ಟಿರುವರು, ಇದು ಎಲ್ಲರಿಗೂ ಒಂದು ಮೇಲ್ಪಂಕ್ತಿಯಾಗಬೇಕು ಎಂದರು. 

 ಒಂದು ದಿನ ಆಶ್ರಮದ ಮೇಲ್ವಿಚಾರಕರಾದವರಿಗೆ ಆಶ್ರಮದ ಕಾರ‍್ಯ‍ಗಳನ್ನು ಸರ್ವತೋಮುಖವಾಗಿ ಮುಂದುವರಿಸಬೇಕೆಂದು ಹೇಳಿದರು. ಅದಕ್ಕೆ ಅವರು ತಾವು ತಮ್ಮ ಕೈಲಾದುದನ್ನೆಲ್ಲವನ್ನು ಮಾಡಲು ಸಿದ್ಧವಿರುವುದಾಗಿಯೂ, ಆದರೆ ಸಂಸ್ಥೆಯ ಕೆಲಸ ಸರ್ವತೋಮುಖವಾಗಿ ಬೆಳೆಯಬೇಕಾದರೆ ಉಳಿದವರೆಲ್ಲ ಸಹಕರಿಸಿದರೆ ಮಾತ್ರ ಸಾಧ್ಯವೆಂದೂ, ಪ್ರತಿಯೊಬ್ಬರು ಕಡಿಮೆ ಎಂದರೆ ಮೂರು ವರ್ಷವಾದರೂ ಒಂದು ಆಶ್ರಮದಲ್ಲಿದ್ದರೆ ಮಾತ್ರ ಸಾಧ್ಯವೆಂದೂ ಹೇಳಿದರು. ಸ್ವಾಮೀಜಿ ಅದರಂತೆ ಎಲ್ಲರೆದುರಿಗೆ ಈ ಮಾತನ್ನು ಪ್ರಸ್ತಾಪಿಸಿದಾಗ, ವಿರಜಾನಂದರ ವಿನಃ ಉಳಿದವರೆಲ್ಲ ಅದಕ್ಕೆ ಒಪ್ಪಿಕೊಂಡರು. ವಿರಜಾನಂದರು ಸ್ವಾಮೀಜಿಯವರ ಮಾತನ್ನು ಗೌರವಿಸಿದರೂ ತಾವು ದೀರ್ಘಧ್ಯಾನದಲ್ಲಿ ಕೆಲವು ಕಾಲ ಕಳೆದಲ್ಲದೆ ಕರ್ಮದಲ್ಲಿ ನಿರತರಾಗಲು ಸಾಧ್ಯವಿಲ್ಲವೆಂದು ಹೇಳಿದರು. ಸರಿಯಾಗಿ ಧ್ಯಾನಮಾಡಿದ ಮೇಲೆಯೇ ದೃಷ್ಟಿಯಲ್ಲಿ ಫಲಾಪೇಕ್ಷೆ ಇಲ್ಲದೆ, ನಿಸ್ಸಂಗರಾಗಿ ಕರ್ಮಮಾಡಲು ಸಾಧ್ಯ ಎಂದರು. ಸ್ವಾಮೀಜಿ ಹೊರಗೆ ಅದಕ್ಕೆ ಸಮ್ಮತಿಯನ್ನು ತೋರದೇ ಇದ್ದರೂ ಒಳಗೆ ಅವರನ್ನು ಮೆಚ್ಚಿ, ಅದಕ್ಕೆ ಅವಕಾಶವನ್ನು ಕೊಟ್ಟರು. 

 ಮಾಯಾವತಿಗೆ ಸಮೀಪದಲ್ಲಿ ಧರ್ಮಘರ್ ಎಂಬ ಎತ್ತರ ಬೆಟ್ಟವಿದೆ. ಅಲ್ಲಿಂದ ಹಿಮಾಲಯದ ರಮ್ಯವಾದ ದೃಶ್ಯ ಕಾಣಿಸುವುದು. ಸ್ವಾಮೀಜಿಯವರು ಒಂದು ದಿನ ಅಲ್ಲಿಗೆ ಹೋಗಿ ಅಲ್ಲಿ ಧ್ಯಾನಮಾಡುವುದಕ್ಕೆ ಒಂದು ಕುಟೀರವನ್ನು ಕಟ್ಟಬೇಕೆಂದು ಸಲಹೆ ಮಾಡಿದರು. 

\newpage

 ಅದ್ವೈತ ಆಶ್ರಮವನ್ನು ಪ್ರಾರಂಭಮಾಡಿದಾಗ ಅಲ್ಲಿ ಯಾವ ಬಾಹ್ಯಪೂಜೆಗಳೂ ಇರಕೂಡದೆಂದು ನಿರ್ಧರಿಸಿದ್ದರು. ಆದರೆ ಕೆಲವರು ಹೇಗೋ ಒಂದು ಕೋಣೆಯಲ್ಲಿ ಶ‍್ರೀರಾಮಕೃಷ್ಣರ ಭಾವಚಿತ್ರವನ್ನಿಟ್ಟು ಅದಕ್ಕೆ ಪೂಜಾದಿಗಳನ್ನು ಮಾಡುತ್ತಿದ್ದುದನ್ನು ನೋಡಿದರು. ಸ್ವಾಮೀಜಿಯವರು ಈ ಸ್ಥಳದಲ್ಲಿ ಧ್ಯಾನ, ಜಪ, ಅಧ್ಯಯನ ವಿನಃ ಬಾಹ್ಯಪೂಜೆ ಇರಕೂಡದೆಂದು ಹೇಳಿದರು. ಹಾಗೆ ಬಾಹ್ಯಪೂಜೆ ಮಾಡಲು ಆಸಕ್ತರಾದವರಿಗೆ ಇಲ್ಲಿ ಇರಲು ಸ್ಥಳವಿಲ್ಲ ಎಂದು ಖಂಡಿತವಾಗಿ ಒತ್ತಿಹೇಳಿದರು. ಅನಂತರ ಒಮ್ಮೆ ಸ್ವಾಮೀಜಿ ಹಾಸ್ಯವಾಗಿ, “ಒಂದು ಸ್ಥಳದಲ್ಲಿಯಾದರೂ ಶ‍್ರೀರಾಮಕೃಷ್ಣರ ಬಾಹ್ಯಪೂಜೆ ಇರಕೂಡದೆಂದು ನಾನು ಮಾಡಿದ್ದರೆ, ಆ ವೃದ್ಧರು ಹೇಗೊ ಅಲ್ಲಿಗೂ ಧಾಳಿ ಇಟ್ಟಿರುವರು!” ಎಂದರು. 

 ಸ್ವಾಮೀಜಿ ಅಲ್ಲಿದ್ದಾಗ ಹಲವು ಲೇಖನಗಳನ್ನು ಬರೆದರು, ಪತ್ರಗಳನ್ನು ಬರೆದರು, ಆಶ್ರಮದವರಿಗೆ ಪ್ರವಚನಾದಿಗಳನ್ನು ಕೊಟ್ಟರು. ಅನೇಕ ಜನ ಸುತ್ತಮುತ್ತಲಿಂದ ಸ್ವಾಮೀಜಿಯವರನ್ನು ನೋಡಲು ಬಂದರು. ಅಲ್ಲಿ ಅವರ ಶಿಷ್ಯರು ಗುರುವಿಗೆ ಮಾಡಬೇಕಾದ ಸೇವೆಯನ್ನು ಕುರಿತು ಸೇವಿಯರ್ಸ್ ಅವರಿಗೆ ವಿವರಿಸಿದರು. ಶಿಷ್ಯನ ಸಾಧನೆಯಲ್ಲಿ ಗುರುಶುಶ್ರೂಷೆ ಅತ್ಯಂತ ಮುಖ್ಯ, ಇಂಡಿಯಾದೇಶದಲ್ಲಿ. ಅದರ ಬಲದಿಂದಲೇ ಶಿಷ್ಯ ಎಲ್ಲವನ್ನೂ ಬ್ರಹ್ಮಜ್ಞಾನವನ್ನೂ ಕೂಡ ಪಡೆಯಬಲ್ಲ ಎಂಬುದನ್ನು ವಿವರಿಸಿದರು. 

 ಸ್ವಾಮೀಜಿ ತಮ್ಮ ಒಂದು ಬಯಕೆ ಜೀವನದ ಕೊನೆಗಾಲವನ್ನು ಇಲ್ಲಿ ಬಂದು ಕಳೆಯಬೇಕೆಂದೂ, ಕೇವಲ ಕೆಲವು ಗ್ರಂಥಗಳನ್ನು ಬರೆಯುವುದು, ಸಂಚಾರಮಾಡುವುದು ಇದರಲ್ಲೆ ತನ್ಮಯರಾಗಬೇಕೆಂದೂ ಶ‍್ರೀಮತಿ ಸೇವಿಯರ್ಸ್‍‍ ಅವರಿಗೆ ಹೇಳಿದರು. ಆದರೆ ವಿಧಾತ ಸ್ವಾಮೀಜಿಗೆ ಜೀವನದಲ್ಲಿ ವಿಶ್ರಾಂತಿಯನ್ನು ಕೊಡಲಿಲ್ಲ. ಅವರು ಈ ಲೋಕಕ್ಕೆ ಬಂದದ್ದು ದುಡಿತಕ್ಕೆ. ಉಸಿರಾಡುವವರೆಗೂ ದುಡಿದು ಅನಂತರ ವಿಶ್ರಾಂತಿಗೆ ಬೇರೆ ಲೋಕಕ್ಕೆ ಹೋಗುವುದಕ್ಕೆ. 

 ಮಾಯಾವತಿಯ ಛಳಿ ಮತ್ತು ಹಿಮ ಇವುಗಳನ್ನು ಸಹಿಸಲು ಕಷ್ಟವಾಗಿ ಅಲ್ಲಿಂದ ಕೆಳಗೆ ಇಳಿದು ಬೇಲೂರು ಮಠವನ್ನು ಜನವರಿ ೨೪ನೇ ತಾರೀಖು ತಲುಪಿದರು.\break ಮಠದಲ್ಲಿ ಕೆಲವು ಕಾಲಗಳಿದ್ದು ಅನಂತರ ಸ್ವಾಮೀಜಿ ಪೂರ್ವ ಬಂಗಾಳದ ಕಡೆ ಹೊರಟರು. ತಮ್ಮ ತಾಯಿಯವರನ್ನು ಅಲ್ಲಿರುವ ಕೆಲವು ತೀರ್ಥಸ್ಥಳಗಳನ್ನು ತೋರಿಸುವುದಕ್ಕಾಗಿ ಕರೆದುಕೊಂಡು ಹೋದರು. ಸ್ವಾಮೀಜಿಯವರನ್ನು ಢಾಕಾದಲ್ಲಿ ಜನರು ಬಹಳ ವಿಜೃಂಭಣೆಯಿಂದ ಸ್ವಾಗತಿಸಿದರು. ಜನರ ಕೋರಿಕೆಯಂತೆ ಅಲ್ಲಿ ಎರಡು ಉಪನ್ಯಾಸಗಳನ್ನು ಕೊಟ್ಟರು. ಸ್ವಾಮೀಜಿಯವರು ಅಲ್ಲಿದ್ದ ಕಾಲದಲ್ಲಿಯೇ\break ಬುಧಾಷ್ಟಮಿಯ ಪುಣ್ಯದಿನ ಬಂದಿತು. ಸಹಸ್ರಾರು ಜನರು ಆ ಸಮಯದಲ್ಲಿ ಬ್ರಹ್ಮಪುತ್ರ ನದಿಯ ಸ್ನಾನಕ್ಕಾಗಿ ನೆರೆಯುವರು. ಸ್ವಾಮೀಜಿಯವರ ಗುರುಭಾಯಿ ಮತ್ತು ಶಿಷ್ಯರು ಕೂಡ ಸ್ನಾನಕ್ಕೆ ಹೊರಟರು. ಆ ಸಮಯದಲ್ಲಿ ಅಲ್ಲಿ ಕಾಲರಾ ಇತ್ತು. ಅದಕ್ಕಾಗಿ ಸ್ವಾಮೀಜಿ ತಮ್ಮ ಜೊತೆಯಲ್ಲಿ ಸ್ನಾನ ಮಾಡುವವರಿಗೆ ಯಾರೂ ನೀರನ್ನು\break ಕುಡಿಯಕೂಡದೆಂದು ಹೇಳಿದರು. ಸ್ನಾನವೆಲ್ಲ ಆದಮೇಲೆ ಯಾರಾದರೂ ನೀರನ್ನು ಕುಡಿದಿರುವರೆ ಎಂದು ಪ್ರಶ್ನಿಸಿದಾಗ ಯಾರೂ ಇಲ್ಲ ಎಂದರು. ಆಗ ಸ್ವಾಮೀಜಿ, ತಾವು ಸ್ನಾನಕ್ಕೆ ನೀರಿನಲ್ಲಿಳಿದಾಗ ಸ್ವಲ್ಪ ಮುಂದಕ್ಕೆ ಈಜಿಕೊಂಡು ಹೋಗಿ ಆಳದಲ್ಲಿ ಮುಳುಗಿ ಕೆಳಗಡೆ ಒಂದು ಗುಟುಕು ನೀರು ಕುಡಿದೆನೆಂದರು; ಪುಣ್ಯದಿನದಲ್ಲಿ ತೀರ್ಥವನ್ನು ಹೇಗೆ ಸೇವಿಸದೆ ಇರುವುದು ಎಂದರು. 

 ಸ್ವಾಮೀಜಿ ಢಾಕಾದಲ್ಲಿದ್ದಾಗ ಒಬ್ಬ ವೇಶ್ಯೆ ಆಭರಣಗಳಿಂದ ಅಲಂಕೃತಳಾಗಿ ತನ್ನ ತಾಯಿಯೊಡನೆ ಸ್ವಾಮೀಜಿ ಬಳಿಗೆ ಬಂದು ನಮಸ್ಕರಿಸಿ ತಮಗೆ ಅಸ್ತಮಾ ಇರುವುದಾಗಿಯೂ ದಯವಿಟ್ಟು ಯಾವುದಾದರೂ ಔಷಧಿ ಗೊತ್ತಿದ್ದರೆ ಅದನ್ನು ಕೊಟ್ಟು ಗುಣ ಮಾಡಬೇಕೆಂದೂ‌ ಪ್ರಾರ್ಥಿಸಿಕೊಂಡಳು. ಸ್ವಾಮೀಜಿ ಅವಳ ಕಥೆಯನ್ನು ಸಾವಧಾನದಿಂದ ಕೇಳಿದರು. ಅನಂತರ ತುಂಬ ಕನಿಕರದಿಂದ ತಮಗೇ ಆ ಜಾಡ್ಯ ಇರುವುದಾಗಿಯೂ, ತಮಗೇ ಅದರಿಂದ ಪಾರಾಗಲು ಸಾಧ್ಯವಿಲ್ಲವೆಂದೂ ಹೇಳಿದರು. ಗುಣವಾಗುವ ಔಷಧ ತಮಗೆ ಗೊತ್ತಿದ್ದರೆ ನಿಜವಾಗಿ ಹೇಳಿತ್ತಿದ್ದೆನು ಎಂದರು. 

 ಸ್ವಾಮೀಜಿ ಢಾಕಾದಿಂದ ಚಂದ್ರನಾಥ, ಕಾಮಾಖ್ಯ ಮುಂತಾದ ತೀರ್ಥಸ್ಥಳಗಳನ್ನು ನೋಡಿಕೊಂಡು ಅಸ್ಸಾಮಿನಲ್ಲಿರುವ ಷಿಲಾಂಗ್‍ಗೆ ಹೋದರು. ಅಲ್ಲಿ ಹೆನ್ರಿ ಕಾಟನ್ ಎಂಬುವನು ಚೀಫ್ ಕಮೀಷನರ್ ಆಗಿದ್ದನು. ಆತನಿಗೆ ಹಿಂದೆ ಸ್ವಾಮೀಜಿ ಪರಿಚಯವಿತ್ತು. ಆತ ಇಂಡಿಯಾ ದೇಶವನ್ನು ಪ್ರೀತಿಸುತ್ತಿದ್ದ. ಸಾಧ್ಯವಾದಷ್ಟು ಅದಕ್ಕೆ ಒಳ್ಳೆಯದನ್ನು ಮಾಡಲು ಯತ್ನಿಸುತ್ತಿದ್ದ. ಸ್ವಾಮೀಜಿ ಆರೋಗ್ಯ ಅಲ್ಲಿ ಹದಗೆಡಲು ಬೇಗ ಕಲ್ಕತ್ತೆಗೆ ಹಿಂತಿರುಗಿದರು. 

 ಸ್ವಾಮೀಜಿ ಪೂರ್ವ ಬಂಗಾಳ ಮತ್ತು ಅಸ್ಸಾಂನಿಂದ ಹಿಂತಿರುಗಿ ಕೆಲವು ದಿನಗಳು ಆಗಿತ್ತು. ಅವರಿಗೆ ಖಾಯಿಲೆಯಾಗಿ ಕಾಲುಗಳು ಊದಿದ್ದುವು. ಮಠಕ್ಕೆ ಬಂದ ಶಿಷ್ಯ ಸ್ವಾಮಿಗಳ ಪಾದಕ್ಕೆ ಸಾಷ್ಟಾಂಗ ಪ್ರಣಾಮ ಮಾಡಿದನು. ಖಾಯಿಲೆಯಲ್ಲಿದ್ದರೂ ಸ್ವಾಮೀಜಿ ತಮ್ಮ ಸಹಜವಾದ ನಗುಮುಖ ಕರುಣಾಪೂರಿತ ದೃಷ್ಟಿಯಿಂದಲೇ ಕೂಡಿದ್ದರು. 

 ಶಿಷ್ಯ: “ಸ್ವಾಮೀಜಿ, ಈಗ ಹೇಗಿದ್ದೀರಿ?” 

 ಸ್ವಾಮೀಜಿ: “ನನ್ನ ಆರೋಗ್ಯದ ವಿಷಯ ಏನೆಂದು ಹೇಳಲಿ ಮಗು? ದಿನ ದಿನಕ್ಕೆ ಈ ಶರೀರ ಕೆಲಸ ಮಾಡಲನರ್ಹವಾಗುತ್ತಿದೆ. ವಂಗಭೂಮಿಯಲ್ಲಿ ಹುಟ್ಟಿದ ದೇಹ. ಯಾವುದಾದರೊಂದು ರೋಗ ಯಾವಾಗಲೂ ಅದನ್ನು ಕಾಡುತ್ತಲೇ ಇರುತ್ತದೆ. ಶರೀರ ಪ್ರಕೃತಿ ಕೊಂಚವೂ ದೃಢವಾಗಿಲ್ಲ. ಯಾವುದಾದರೂ ಕಷ್ಟವಾದ ಕೆಲಸ ಮಾಡಬೇಕಾದರೆ ಅದರ ಪ್ರಯಾಸವನ್ನು ಸಹಿಸಲಾಗುವುದಿಲ್ಲ. ಆದರೆ ಈ ದೇಹವಿರುವವರೆಗೂ ನಾನು ಕೆಲಸ ಮಾಡೇ ತೀರುವೆ, ಕರ್ಮ ಪ್ರಪಂಚದಲ್ಲಿ ದುಡಿಯುತ್ತಲೆ ಸಾಯುವೆ.” 

 ಶಿಷ್ಯ: “ನೀವು ಕೊಂಚಕಾಲ ಕೆಲಸಮಾಡುವುದನ್ನು ಬಿಟ್ಟು ವಿಶ್ರಾಂತಿ ತೆಗೆದುಕೊಂಡರೆ ಆರೋಗ್ಯ ಹೊಂದುವಿರಿ. ನಿಮ್ಮ ಬದುಕಿನಿಂದ ಜಗತ್ಕಲ್ಯಾಣ ಆಗುವುದು.” 

 ಸ್ವಾಮೀಜಿ: “ನಾನು ಸುಮ್ಮನೆ ಮೌನವಾಗಿ ಕುಳಿತಿರಲು ಸಾಧ್ಯವೆ ಮಗು? ಶ‍್ರೀರಾಮಕೃಷ್ಣರು ನಿರ್ಯಾಣವಾಗಲು ಎರಡು ಮೂರು ದಿನ ಮೊದಲು ಅವರು ಯಾರನ್ನು ‘ಕಾಳಿ’ ಎನ್ನುತ್ತಿದ್ದರೋ ಆ ದೇವಿ ಈ ದೇಹಪ್ರವೇಶ ಮಾಡಿದಳು. ಆಕೆಯೇ ನನ್ನನ್ನು ಸುಮ್ಮನಿರಿಸದೆ, ನನ್ನ ಸ್ವಂತ ಆರೋಗ್ಯದ ಕಡೆ ಕೂಡ ಗಮನಕೊಡದಂತೆ ಅಲ್ಲಿ ಇಲ್ಲಿ ಎಲ್ಲೆಡೆಗೂ ನನ್ನನ್ನು ಕರೆದೊಯ್ದು ಕೆಲಸ ಮಾಡಿಸುತ್ತ ಇರುವಳು.” 

 ಶಿಷ್ಯ: “ನೀವು ಇದನ್ನು ರೂಪಕ ದೃಷ್ಟಿಯಿಂದ ಹೇಳುತ್ತಿದ್ದೀರೇನು?” 

 ಸ್ವಾಮೀಜಿ: “ಹಾಗಲ್ಲ, ಮಹಾಸಮಾಧಿಗೆ ಮೂರು ದಿನ ಮೊದಲು ಅವರು ನನ್ನನ್ನು ತಮ್ಮ ಪಕ್ಕಕ್ಕೆ ಕರೆದು ತಮ್ಮೆದುರು ಕುಳ್ಳಿರಲು ಹೇಳಿ ಎವೆಯಿಕ್ಕದೆ ನನ್ನನ್ನೇ ನೋಡುತ್ತಾ ಸಮಾಧಿಸ್ಥರಾದರು. ಆಗ ನಿಜವಾಗಿಯೂ ನನ್ನ ದೇಹದಲ್ಲಿ ವಿದ್ಯುತ್ ಪ್ರವಾಹ ಸಂಚರಿಸಿದಂತಾಯ್ತು. ಕೊಂಚ ಹೊತ್ತಿನಲ್ಲಿಯೇ ನನಗೂ ಬಾಹ್ಯಪ್ರಜ್ಞೆ ತಪ್ಪಿದಂತಾಗಿ ಸ್ಥಿರವಾಗಿ ಕುಳಿತುಕೊಂಡೆ. ಹಾಗೇ ಎಷ್ಟು ಹೊತ್ತು ಕುಳಿತಿದ್ದೆನೋ ನನಗೆ ಗೊತ್ತಿಲ್ಲ. ನನಗೆ ಪ್ರಜ್ಞೆ ಬಂದಾಗ ಶ‍್ರೀರಾಮಕೃಷ್ಣರು ಕಂಬನಿಗರೆಯುತ್ತಿದ್ದುದನ್ನು ಕಂಡೆ. ಕಾರಣವೇನೆಂದು ಪ್ರಶ್ನಿಸಿದಾಗ ಅವರು ವಾತ್ಸಲ್ಯದಿಂದ ಹೇಳಿದರು: ‘ಇಂದು ನಿನಗೆ ನನ್ನ ಸರ್ವಸ್ವವನ್ನೂ ದಾನ ಮಾಡಿದ್ದೇನೆ. ನಾನಿಂದು ಒಬ್ಬ ದರಿದ್ರ ಫಕೀರ, ನೀನು ಹಿಂತಿರುಗುವ ಮುಂಚೆ ಈ ಶಕ್ತಿಯಿಂದ ಲೋಕಕ್ಕೆ ಮಹತ್ತಾದ ಉಪಕಾರವನ್ನು ಮಾಡಬೇಕು’. ಆ ಶಕ್ತಿಯೇ ಈಗ ನನ್ನನ್ನು ಈ ಕೆಲಸವನ್ನೆಲ್ಲಾ ಮಾಡಲು ಪ್ರೇರೇಪಿಸುತ್ತಿದೆ ಎಂದು ಅನುಭವವಾಗುತ್ತಿದೆ. ಈ ದೇಹ ಕೇವಲ ಸೋಮಾರಿತನಕ್ಕಾಗಿ ಹುಟ್ಟಿಬರಲಿಲ್ಲ.” 

 ಈ ಮಾತನ್ನು ಕೇಳಿ ವಿಸ್ಮಯಪಟ್ಟ ಶಿಷ್ಯ ಯೋಚಿಸಿದ: ಸಾಧಾರಣ ಜನರು ಈ ವಿಷಯವನ್ನು ಹೇಗೆ ತೆಗೆದುಕೊಳ್ಳುವರೆಂಬುದು ಯಾರಿಗೆ ಗೊತ್ತು? ಆ ಕೂಡಲೇ ಅವನು ವಿಚಾರವನ್ನು ಬದಲಾಯಿಸಿ ಕೇಳಿದ: “ಸ್ವಾಮೀಜಿ, ನೀವು ಪೂರ್ವಬಂಗಾಳವನ್ನು ಹೇಗೆ ಇಷ್ಟಪಟ್ಟಿರಿ?” 

 ಸ್ವಾಮೀಜಿ: “ಒಟ್ಟಿನಲ್ಲಿ ಇಷ್ಟಪಟ್ಟೆ. ನಾನು ಹೋದಾಗ ಹೊಲಗಳಲ್ಲಿ ಬೆಳೆಗಳು ಹಚ್ಚನೆ ಹಸುರಾಗಿದ್ದುವು. ಹವಾ ಕೂಡ ಚೆನ್ನಾಗಿತ್ತು. ಬೆಟ್ಟದ ಮೇಲಿನ ಪ್ರಕೃತಿ ಸೌಂದರ್ಯ ಮನೋಹರವಾಗಿದೆ. ಬ್ರಹ್ಮಪುತ್ರಾನದಿಯ ಕಣಿವೆಯ ಸೌಂದರ‍್ಯ ಅಸದೃಶವಾದುದು. ಪೂರ್ವ ಬಂಗಾಳಿದ ಜನರು ಇಲ್ಲಿಯ ಜನರಿಗಿಂತ ಹೆಚ್ಚು ದೃಢಕಾಯರು, ಉತ್ಸಾಹಿಗಳು. ಬಹುಶಃ ಅವರು ಮೀನು ಮಾಂಸ ತಿನ್ನುವುದರಿಂದ ಹಾಗಿದ್ದಿರಬಹುದು. ಅವರು ಹಿಡಿದ ಕೆಲಸವನ್ನು ಮಾಡೇ ತೀರುವರು. ಅವರು ಆಹಾರದಲ್ಲಿ ಹೆಚ್ಚು ಎಣ್ಣೆ ಮತ್ತು ಕೊಬ್ಬನ್ನು ಸೇವಿಸುವರು. ಅದು ಅಷ್ಟು ಒಳ್ಳೆಯದಲ್ಲ. ಏಕೆ ಎಂದರೆ ಎಣ್ಣೆ ಕೊಬ್ಬನ್ನು ಹೆಚ್ಚಾಗಿ ಸೇವಿಸುವುದರಿಂದ ದೇಹ ಸ್ಥೂಲವಾಗುವುದು.” 

 ಶಿಷ್ಯ: “ಅವರಲ್ಲಿ ಧಾರ್ಮಿಕ ಪ್ರವೃತ್ತಿ ಹೇಗಿದೆ?” 

 ಸ್ವಾಮೀಜಿ: “ಧಾರ್ಮಿಕ ಭಾವನೆಗಳ ವಿಚಾರವಾಗಿ ಹೇಳಬೇಕೆಂದರೆ ಅಲ್ಲಿಯ ಜನರು ತೀವ್ರವಾದ ಪೂರ್ವಾಚಾರಪ್ರಿಯರು. ಧರ್ಮದಲ್ಲಿ ಹೆಚ್ಚು ಉದಾರ\break ಭಾವದಿಂದ ಇರಲು ಹೋಗಿ ಅನೇಕರು ಮತಭ್ರಾಂತರಾಗಿರುವರು. ಒಮ್ಮೆ ಢಾಕಾದಲ್ಲಿ ಮೋಹಿನಿಬಾಬುಗಳ ಮನೆಯಲ್ಲಿದ್ದಾಗ ಒಬ್ಬ ತರುಣನು ಒಂದು ಭಾವಚಿತ್ರವನ್ನು ತೋರಿಸಿ ‘ಸ್ವಾಮಿ, ದಯವಿಟ್ಟು ಹೇಳಿ ಈ ಚಿತ್ರದಲ್ಲಿರುವವನಾರು? ಅವರು ಅವತಾರ ಪುರುಷರೆ?’ ಎಂದ. ನಾನು ನಯವಾಗಿ ನನಗೆ ಅವರ ವಿಷಯ ಏನೂ ಗೊತ್ತಿಲ್ಲವೆಂದು ಹೇಳಿದೆ. ಮೂರು ನಾಲ್ಕು ಬಾರಿ ಹೇಳಿದರೂ ಆ ಮನುಷ್ಯ ಪದೇ ಪದೇ ನನ್ನನ್ನು ಪ್ರಶ್ನಿಸುವುದನ್ನು ಬಿಡಲಿಲ್ಲ. ಕೊನೆಗೊಮ್ಮೆ ವಿಧಿಯಿಲ್ಲದೆ ನಾನು ಹೇಳಿದೆ: 

 ‘ಮಗು, ಇನ್ನು ಮುಂದೆ ಕೊಂಚ ಪುಷ್ಟಿಕರವಾದ ಆಹಾರವನ್ನು ತೆಗೆದುಕೊ. ನಿನ್ನ ಮೆದುಳು ನಂತರ ಬಲಿಷ್ಠವಾಗುವುದು. ಪುಷ್ಟಿಕರವಾದ ಆಹಾರದ ಅಭಾವದಿಂದ ನಿನ್ನ ಮೆದುಳು ಬತ್ತಿಹೋಗಿದೆ.’ ಈ ಮಾತುಗಳಿಂದ ಆ ತರುಣ ಅಸಂತುಷ್ಟನಾಗಿರಬೇಕು. ಆದರೆ ನಾನೇನು ಮಾಡಲಿ? ಆದರೆ ನಾನೀ ರೀತಿ ಯುವಕರಿಗೆ ಹೇಳದಿದ್ದರೆ ಅವರು ಸ್ವಲ್ಪದರಲ್ಲೇ ಹುಚ್ಚರಾಗಿಬಿಡುವರು.” 

 ಶಿಷ್ಯ: “ನಮ್ಮ ಪೂರ್ವಬಂಗಾಳದಲ್ಲಿ ಈಚೀಚೆಗೆ ಅನೇಕ ಅವತಾರಗಳು ತಲೆಯೆತ್ತಿಕೊಂಡಿವೆ.” 

 ಸ್ವಾಮೀಜಿ: “ಜನರು ತಮ್ಮ ಗುರುವನ್ನು ಅವತಾರಪುರುಷರೆಂದು ಕರೆಯಬಹುದು. ಅವರ ವಿಷಯದಲ್ಲಿ ಅವರಿಷ್ಟ ಬಂದ ಭಾವನೆ ಹೊಂದಿರಬಹುದು. ಆದರೆ ಎಲ್ಲೆಂದರಲ್ಲಿ ಹೊತ್ತು ಗೊತ್ತಿಲ್ಲದೆ ಭಗವಂತನ ಅವತಾರ ಆಗುವುದಿಲ್ಲ. ಢಾಕಾದಲ್ಲೇ ಮೂರು ನಾಲ್ಕು ಅವತಾರಗಳಿವೆ ಎಂದು ಕೇಳಿದೆ.” 

 “ಅಲ್ಲಿನ ಸ್ರೀಯರು ಸಾಮಾನ್ಯವಾಗಿ ಎಲ್ಲಾ ಕಡೆಯೂ ಒಂದೇ ಬಗೆಯಾಗಿರುವರು. ಢಾಕಾದಲ್ಲಿ ವೈಷ್ಣವ ಮತದ ಪ್ರಾಧಾನ್ಯ ಕಂಡುಬಂದಿತು. ಹ-ರ ಹೆಂಡತಿ ಬಹಳ ಜಾಣೆಯಂತೆ ಕಂಡಳು. ಬಹು ಎಚ್ಚರಿಕೆಯಿಂದ ಆಕೆ ನನ್ನ ಆಹಾರವನ್ನು ಸಿದ್ಧಪಡಿಸಿ ಕಳುಹಿಸುತ್ತಿದ್ದಳು.” 

 ಶಿಷ್ಯ: “ನೀವು ನಾಗಮಹಾಶಯರ ಸ್ಥಳಕ್ಕೆ ಹೋಗಿದ್ದಿರೆಂದು ಕೇಳಿದೆ.” 

 ಸ್ವಾಮೀಜಿ: “ಹೌದು, ಅಷ್ಟುದೂರ ಹೋದಮೇಲೆ ಅಂತಹ ಮಹಾಪುರುಷರ ಜನ್ಮಸ್ಥಳವನ್ನು ನೋಡದೆ ಬರುವುದುಂಟೆ? ಅವರ ಹೆಂಡತಿ ತಾನೇ ಮಾಡಿದ್ದ ಅನೇಕ ರುಚಿಕರವಾದ ಪದಾರ್ಥಗಳಿಂದ ಊಟಮಾಡಿಸಿದಳು. ಆ ಕುಟೀರ ಶಾಂತಿಮಯವಾದ ಏಕಾಂತಸ್ಥಳವಾಗಿ ಸುಂದರವಾಗಿದೆ. ಅಲ್ಲಿ ನಾನು ಹಳ್ಳಿಯ ಕೆರೆಯೊಂದರಲ್ಲಿ ಈಜಾಡಿದೆ. ಅನಂತರ ನಾನು ಎಂತಹ ಸುಖನಿದ್ರೆಯಲ್ಲಿ ಮುಳುಗಿದೆನೆಂದರೆ ನಾನು ಎದ್ದಾಗ ಮಧ್ಯಾಹ್ನ ಎರಡು ಗಂಟೆಯಾಗಿತ್ತು. ನನ್ನ ಜೀವಮಾನದಲ್ಲಿ ಗಾಢನಿದ್ರೆಯಲ್ಲಿ ಕಳೆದ ದಿನಗಳು ಅತಿ ವಿರಳ. ನಾಗಮಹಾಶಯರ ಮನೆಯಲ್ಲಿ ಮಲಗಿದ್ದ ದಿನವೂ ಆ ದಿನಗಳಲ್ಲೊಂದು. ನಿದ್ರೆಯಿಂದೆಚ್ಚೆತ್ತ ಮೇಲೆ ನನಗೆ ಪುಷ್ಕಳ ಭೋಜನವಾಯಿತು. ನಾಗಮಹಾಶಯರ ಪತ್ನಿ ನನಗೊಂದು ಬಟ್ಟೆಯನ್ನು ಕೊಟ್ಟಳು. ಅದನ್ನು ತಲೆಗೆ ರುಮಾಲಿನಂತೆ ಸುತ್ತಿಕೊಂಡು ನಾನು ಢಾಕಾಕ್ಕೆ ಹೊರಟೆ. ಅಲ್ಲಿ ನಾಗಮಹಾಶಯರ\break ಭಾವಚಿತ್ರವನ್ನು ಪೂಜಿಸುವರು. ಅವರ ಅವಶೇಷವನ್ನಿಟ್ಟಿರುವ ಸ್ಥಳವನ್ನು ಚೆನ್ನಾಗಿಟ್ಟಿರಬೇಕು. ಈಗಲೂ ಕೂಡ ಅದು ಸರಿಯಾಗಿಲ್ಲ.” 

 ಶಿಷ್ಯ: “ಆ ಭಾಗದ ಜನರು ನಾಗಮಹಾಶಯರನ್ನು ಇಷ್ಟೊಂದು ಗೌರವಿಸುವುದಿಲ್ಲ.” 

 ಸ್ವಾಮೀಜಿ: “ಸಾಧಾರಣ ಮನುಷ್ಯರು ಹೇಗೆ ತಾನೆ ಅಂತಹ ಮಹಾಪುರುಷರನ್ನು ಗೌರವಿಸುವರು? ಅವರ ಸಹವಾಸ ಹೊಂದಿರುವವರು ನಿಜವಾಗಿ ಧನ್ಯರು.” 

 ಶಿಷ್ಯ: “ನೀವು ಕಾಮಾಖ್ಯದಲ್ಲೇನು ನೋಡಿದಿರಿ?” 

 ಸ್ವಾಮೀಜಿ: “ಶಿಲ್ಹಾಂಗ್ ಬೆಟ್ಟಗಳು ಬಹು ಸುಂದರವಾಗಿವೆ. ನಾನು ಅಲ್ಲಿ ಅಸ್ಸಾಮಿನ ಮುಖ್ಯ ಕಮೀಷನರ್ ಆದ ಹೆನ್ರಿ ಕಾಟನ್ ಅವರನ್ನು ಭೇಟಿ ಮಾಡಿದೆ. ಅವರು ನನ್ನನ್ನು ಕೇಳಿದರು: ‘ಸ್ವಾಮೀಜಿ, ಯೂರೋಪ್ ಅಮೇರಿಕಾ ದೇಶಗಳ ಪ್ರವಾಸದ ನಂತರ ಈ ದೂರದ ಬೆಟ್ಟಗಳಲ್ಲೇನನ್ನು ನೋಡಲು ಬಂದಿರಿ?’ ಸರ್\break ಹೆನ್ರಿಕಾಟನ್‍ರಂತಹ ಒಳ್ಳೆಯ ಮತ್ತು ದಯಾರ್ದ್ರ ಹೃದಯಿಗಳು ಬಹಳ ಅಪರೂಪ. ನನ್ನ ಖಾಯಿಲೆಯ ವಿಚಾರ ಕೇಳಿ ಅವರು ಸೀನಿಯರ್ ಸರ್ಜನ್ನರನ್ನು ಕಳುಹಿಸಿದರು. ಅವರು ಬೆಳಿಗ್ಗೆ ಸಂಜೆ ಎರಡು ಹೊತ್ತು ನನ್ನ ಆರೋಗ್ಯವನ್ನು ವಿಚಾರಿಸುತ್ತಿದ್ದರು. ನಾನು ಅಲ್ಲಿ ಹೆಚ್ಚು ಭಾಷಣ ಮಾಡಲಾಗಲಿಲ್ಲ. ನನ್ನ ಆರೋಗ್ಯ ಕೆಟ್ಟಿತು. ದಾರಿಯಲ್ಲಿ ನಿತಾಯ್ ನನ್ನನ್ನು ಚೆನ್ನಾಗಿ ನೋಡಿಕೊಂಡರು.” 

 ಶಿಷ್ಯ: “ಆ ಭಾಗದಲ್ಲಿ ಧಾರ್ಮಿಕ ಭಾವನೆ ಹೇಗಿದೆ?” 

 ಸ್ವಾಮೀಜಿ: “ಅದು ತಾಂತ್ರಿಕ ಪ್ರದೇಶ. ‘ಹಂಕರದೇವ’ ನೆಂಬೊಬ್ಬನನ್ನು ಅವತಾರವೆಂದು ಪೂಜಿಸುವರೆಂದು ಕೇಳಿದೆ. ಆತನ ಪಂಥ ಬಹಳ ಹರಡಿದೆ ಎಂದು ಕೇಳಿದೆ. ನನಗೆ ಹಂಕರದೇವನೆಂಬುದು ಶಂಕರಾಚಾರ‍್ಯರ ಮತ್ತೊಂದು ರೂಪವೊ ಏನೋ ಎಂದು ವಿಚಾರಿಸಲಾಗಲಿಲ್ಲ. ಅವರು ಸಂನ್ಯಾಸಿಗಳಿರಬಹುದು - ಎಂದರೆ ತಾಂತ್ರಿಕ ಸಂನ್ಯಾಸಿಗಳು ಅಥವಾ ಶಂಕರನ ಪಂಥದವರಾಗಿರಬಹುದು.” 

 ಶಿಷ್ಯ: “ಪೂರ್ವ ಬಂಗಾಳಿಗಳು ನಾಗಮಹಾಶಯರನ್ನು ಹೇಗೆ ಅರ್ಥಮಾಡಿಕೊಂಡಿಲ್ಲವೊ ಹಾಗೆಯೇ ನಿಮ್ಮನ್ನೂ ಅರಿತಿಲ್ಲ.” 

 ಸ್ವಾಮೀಜಿ: “ಅವರು ನನ್ನನ್ನು ಗೌರವಿಸಲಿ, ಬಿಡಲಿ, ಆ ಭಾಗದ ಜನರು ಇಲ್ಲಿನವರಿಗಿಂತ ಹೆಚ್ಚು ಉತ್ಸಾಹಿಗಳು, ಕಾರ್ಯಪರರು. ಕಾಲಕ್ರಮೇಣ ಹೆಚ್ಚು ಮುಂದುವರಿಯುವರು. ಈ ವರ್ತಮಾನಕಾಲದಲ್ಲಿ ನಾವು ಕರೆಯುವ ‘ನಾಗರಿಕತೆ’ ದೇಶದ ಆ ಭಾಗವನ್ನು ಇನ್ನೂ ಪೂರ್ತಿ ಒಳಹೊಕ್ಕಿಲ್ಲ. ಕ್ರಮೇಣ ಅದು ಆಗುತ್ತದೆ. ಎಲ್ಲಾ ಕಾಲದಲ್ಲಿಯೂ ರಾಜದಾನಿಯಿಂದ ಹಳ್ಳಿಗಳಿಗೆ ಸಮಾಜ ಮರ್ಯಾದೆಗಳು ಶೈಲಿಗಳು ಹರಡುವುವು. ಪೂರ್ವ ಬಂಗಾಳದಲ್ಲಿಯೂ ಇದೇ ರೀತಿಯಾಗುತ್ತಿದೆ. ನಾಗಮಹಾಶಯರಂಥ ಮಹಾತ್ಮರಿಗೆ ಜನ್ಮವಿತ್ತ ದೇಶ ಧನ್ಯ. ಅವರ ಭವಿಷ್ಯವೂ ಆಶಾದಾಯಕವಾಗಿದೆ. ಆ ಮಹಾತ್ಮನ ಸ್ವಯಂ ಪ್ರತಿಭೆಯಿಂದ ಪೂರ್ವಬಂಗಾಳ ರಂಜಿಸುತ್ತಿದೆ.” 

 ಶಿಷ್ಯ: “ಆದರೆ ಸ್ವಾಮೀಜಿ, ಜನಸಾಮಾನ್ಯರಿಗೆ ಅವರು ಅಂತಹ ದೊಡ್ಡ ವ್ಯಕ್ತಿಯೆಂದು ತಿಳಿದೇ ಇರಲಿಲ್ಲ. ಅವರು ತಾವೇ ಜನರಿಂದ ಬಹುಮಟ್ಟಿಗೆ ಅಜ್ಞಾತರಾಗಿದ್ದರು.” 

 ಸ್ವಾಮೀಜಿ: “ಅಲ್ಲಿ ಅವರೆಲ್ಲಾ ನನ್ನ ಆಹಾರದ ವಿಚಾರವಾಗಿ ಎಷ್ಟೊಂದು ಗೊಂದಲವೆಬ್ಬಿಸುತ್ತಿದ್ದರು! ‘ನೀವೇಕೆ ಆ ಆಹಾರ ಸೇವಿಸುವಿರಿ? ಅವರ ಕೈಯಲ್ಲಿ ಏಕೆ ಊಟ ಮಾಡುವಿರಿ?’ ಮುಂತಾಗಿ. ಅದಕ್ಕೆ ನಾನು ಉತ್ತರ ಕೊಡಬೇಕಾಗಿತ್ತು. ‘ನಾನೊಬ್ಬ ಸಂನ್ಯಾಸಿ, ಭಿಕ್ಷು. ಆಹಾರದ ವಿಷಯವಾಗಿ ನಾನೇಕೆ ಅಷ್ಟೊಂದು ಕಟ್ಟುನಿಟ್ಟನ್ನು ಅನುಸರಿಸಬೇಕು?’ ಎಂದೆ. ನಿಮ್ಮ ಧರ್ಮಶಾಸ್ತ್ರಗಳೇ ಹೇಳುವುದಿಲ್ಲವೆ ‘ಮನೆ ಮನೆಗೂ ಹೋಗಿ ಭಿಕ್ಷೆ ಬೇಡಬೇಕು. ಚಂಡಾಲನ ಮನೆಗೂ ಹೋಗಿ ಭಿಕ್ಷೆ ಎತ್ತಬೇಕು’ ಎಂದು. ಆದರೆ ಪ್ರಾರಂಭದಲ್ಲಿ ಧರ್ಮವನ್ನು ಸೂಕ್ಷ್ಮವಾಗಿ ಅರಿಯಬೇಕಾದರೆ ಧರ್ಮಗ್ರಂಥಗಳಲ್ಲಿರುವ ಸತ್ಯವನ್ನು ನಮ್ಮ ಜೀವನದಲ್ಲಿ ತರಲು ಬಾಹ್ಯನಡವಳಿಕೆಗೂ ನಾವು ಹೆಚ್ಚು ಗಮನಕೊಡಬೇಕು. ನೀನು ಶ‍್ರೀರಾಮಕೃಷ್ಣರು ಹೇಳುತ್ತಿದ್ದ ಪಂಚಾಂಗವನ್ನು ಹಿಂಡಿ ನೀರು ಬರಿಸುವುದರ ವಿಚಾರ ಕೇಳಿಲ್ಲವೆ? ಬಾಹ್ಯರೂಪ ನಡವಳಿಕೆಗಳು ಮನುಷ್ಯನ ಆಂತರ್ಯದಲ್ಲಿರುವ ಮಹಾಶಕ್ತಿ ವಿಕಾಸಗೊಳ್ಳಲು ಮಾತ್ರ. ಎಲ್ಲಾ ಧರ್ಮಗ್ರಂಥಗಳ ಉದ್ದೇಶವೂ‌ ಮನುಷ್ಯನ ಆಂತರಿಕ ಶಕ್ತಿಯನ್ನು ಜಾಗ್ರತಗೊಳಿಸುವುದು ಮತ್ತು ತನ್ನ ನೈಜಸ್ವಭಾವವನ್ನು ಸಾಕ್ಷಾತ್ಕರಿಸಿಕೊಳ್ಳುವುದಾಗಿದೆ. ಈ ವಿಧಿ ನಿಷೇಧಗಳೆಲ್ಲಾ ಅದರ ಸಾಧನಾ ಮಾರ್ಗಗಳು. ನೀನು ನಿನ್ನ ಗುರಿಯನ್ನೇ ಮರೆತು ಈ ಉಪಕರಣಗಳಿಗಾಗಿ ಕಾದಾಡಿದರೆ ಅದರಿಂದೇನು ಪ್ರಯೋಜನ? ಶ‍್ರೀರಾಮಕೃಷ್ಣರು ಈ ಸತ್ಯಪರಿಚಯ ಮಾಡಿಕೊಡಲು ಅವತರಿಸಿದರು.’ 

 “ಸತ್ಯ ಸಾಕ್ಷಾತ್ಕಾರವೇ ಎಲ್ಲಕ್ಕಿಂತ ಮುಖ್ಯವಾದುದು. ನೀನು ಗಂಗಾನದಿಯಲ್ಲೇ ಸಾವಿರ ವರ್ಷ ಸ್ನಾನಮಾಡಿದರೂ, ದೀರ್ಘಕಾಲ ಸಸ್ಯಾಹಾರ ಸೇವಿಸಿದರೂ ಅದು ನಿನ್ನ ಆತ್ಮವಿಕಾಸಕ್ಕೆ ಒಯ್ಯುವುದೇ? ಯಾವುದರಿಂದಲೂ ಏನೂ ಪ್ರಯೋಜನವಿಲ್ಲ. ಆದರೆ ಯಾರಾದರೂ ಈ ಬಾಹ್ಯಾಚರಣೆಗಳನ್ನಾಚರಿಸದೆ ಆತ್ಮಸಾಕ್ಷಾತ್ಕಾರ ಮಾಡಿಕೊಂಡಿದ್ದರೆ ಈ ಬಾಹ್ಯಾಚರಣೆಗಳನ್ನನುಸರಿಸದಿರುವ ಹಾದಿಯೇ ಶ್ರೇಷ್ಠ ಮಾರ್ಗ. ಆದರೆ ಬ್ರಹ್ಮಸಾಕ್ಷಾತ್ಕಾರವಾದ ಮೇಲೂ ಇತರರಿಗೆ ಮೇಲ್ಪಂಕ್ತಿಯಾಗಲೋಸುಗ ಕೊಂಚ ಬಾಹ್ಯವಿಧಿಗಳನ್ನನುಸರಿಸಬೇಕು. ಮುಖ್ಯವಾದುದೇನೆಂದರೆ ನೀನು ಮನಸ್ಸನ್ನು ಯಾವುದರ ಮೇಲಾದರೂ ಏಕಾಗ್ರ ಮಾಡಬೇಕು. ಅದು ಒಂದು ವಸ್ತುವಿನಲ್ಲಿ ಏಕಾಗ್ರತೆ ಪಡೆದರೆ ಅದರಿಂದ ಚಿತ್ತೈಕಾಗ್ರತೆ ಬರುವುದು. ಅದರ ಇತರ ಬದಲಾವಣೆಗಳೆಲ್ಲಾ ಮಾಯವಾಗಿ ಮನಸ್ಸು ಏಕಪ್ರಕಾರವಾಗಿ ಒಂದೇ ಕಡೆಗೆ ಹರಿಯುವುದು. ಅನೇಕರು ಸಂಪೂರ್ಣವಾಗಿ ಕೇವಲ ಬಾಹ್ಯರೂಪ, ಸಂಪ್ರದಾಯಗಳಲ್ಲೇ ಮಗ್ನರಾಗುವರು. ಅದರಿಂದ ತಮ್ಮ ಮನಸ್ಸನ್ನು ಆತ್ಮನ ಕಡೆ ತಿರುಗಿಸಲಸಮರ್ಥರಾಗುವರು. ನೀನು ಹಗಲು ರಾತ್ರಿ ಬರೀ ಉಪವಾಸ, ನಿಷೇಧಗಳು ಸಂಕುಚಿತ ವೃದ್ಧಾಚಾರಗಳಲ್ಲೇ ಮುಳುಗಿದ್ದರೆ ಆತ್ಮಪ್ರಕಾಶಕ್ಕೆ ಅವಕಾಶವೆಲ್ಲಿದೆ? ಯಾರು ಆತ್ಮಸಾಕ್ಷಾತ್ಕಾರದಲ್ಲಿ ಹೆಚ್ಚು ಹೆಚ್ಚು ಮುಂದುವರಿದಿರುವರೋ ಅವರು ಈ ಬಾಹ್ಯರೂಪ ಆಚರಣೆಗಳನ್ನು ಆದಷ್ಟು ಕಡಿಮೆ ಅವಲಂಬಿಸುವರು. ಶಂಕರಾಚಾರ‍್ಯರೂ‌ ಹೇಳಿದ್ದಾರೆ, ‘ಯಾರ ಮನಸ್ಸು ಯಾವಾಗಲೂ ಗುಣಾತೀತವಾಗಿದೆಯೋ ಅಂಥವನಿಗೆ ಈ ವಿಧಿ ನಿಷೇಧಗಳೆಲ್ಲಿವೆ’ ಎಂದು. ಆದ್ದರಿಂದ ಆತ್ಮಸಾಕ್ಷಾತ್ಕಾರ ಪ್ರಧಾನವಾದುದು. ಅದೇ ನಿನ್ನ ಗುರಿ ಎಂದು ತಿಳಿ, ಪ್ರತಿಯೊಂದು ಭಿನ್ನ ಪಂಥವೂ ಸತ್ಯದ ಒಂದು ಪಥ. ನೀನು ಎಷ್ಟು ಮಟ್ಟಿಗೆ ತ್ಯಾಗಿಯಾಗಿರುವೆಯೊ ಅದೇ ನೀನು ಮುಂದುವರಿದಿರುವುದಕ್ಕೆ ನಿದರ್ಶನ. ಎಲ್ಲಿ ನೀನು ಕಾಮಿನಿ ಕಾಂಚನಗಳಲ್ಲಿ ಆಸಕ್ತಿ ಸಾಕಷ್ಟು ಕಡಿಮೆಯಾಗಿರುವುದನ್ನು ನೋಡುವೆಯೋ, ಅಲ್ಲಿ ಅವನು ಯಾವ ಪಂಗಡಕ್ಕೆ ಸೇರಿರಲಿ ಅವನ ಆತ್ಮಶಕ್ತಿ ಜಾಗ್ರತವಾಗಿದೆ ಎಂದು ತಿಳಿ. ಅಂಥವನ ಆತ್ಮಸಾಕ್ಷಾತ್ಕಾರದ ಹೆಬ್ಬಾಗಿಲು ತೆರೆದಿದೆ. ಅದಕ್ಕೆ ಬದಲಾಗಿ ಸಾವಿರಾರು ಶಿಷ್ಟಾಚಾರ ಸಂಪ್ರದಾಯವನ್ನು ಅನುಸರಿಸಿಕೊಂಡು ಧರ್ಮಗ್ರಂಥಗಳನ್ನೇ ಬಾಯಲ್ಲಿ ಪಠಿಸುತ್ತಿದ್ದರೂ ತ್ಯಾಗ ನಿನ್ನಲ್ಲಿ ಸುಳಿಯದಿದ್ದಲ್ಲಿ ನಿನ್ನ ಜೀವನ ನಿಷ್ಫಲವೆಂದು ತಿಳಿ. ಆತ್ಮ ಸಾಕ್ಷಾತ್ಕಾರಕ್ಕೆ ಮನಃಪೂರ್ವಕವಾಗಿ ಪ್ರಯತ್ನಿಸು. ನೀನು ಸಾಕಷ್ಟು ಧರ್ಮಗ್ರಂಥಗಳನ್ನು ವ್ಯಾಸಂಗ ಮಾಡಿರುವೆ. ಈಗ ಹೇಳು, ಅವುಗಳಿಂದೆಷ್ಟು ಪ್ರಯೋಜನವಿದೆಯೆಂದು? ಬರೀ ಹಣವನ್ನು ಚಿಂತಿಸುತ್ತಿದ್ದರೆ ಕೋಟ್ಯಾಧೀಶರಾಗುವರೆ? ನೀನು ಧರ್ಮಗ್ರಂಥಗಳ ವಿಷಯವನ್ನೇ ಯೋಚಿಸುತ್ತಾ ಪಂಡಿತನಾಗಿರುವೆ. ಆದರೆ ಎರಡೂ ಬಂಧನಗಳೇ. ವಿದ್ಯೆ ಅವಿದ್ಯೆ, ಜ್ಞಾನ ಅಜ್ಞಾನವನ್ನು ಮೀರಿ ಮುಂದೆ ಹೋಗಿ ಅನಂತಜ್ಞಾನ ಪಡೆ.” 

 ಶಿಷ್ಯ: “ನಿಮ್ಮ ಕೃಪೆಯಿಂದ ನಾನೆಲ್ಲವನ್ನೂ ಅರ್ಥಮಾಡಿಕೊಳ್ಳುವೆ. ಆದರೆ ನನ್ನ ಪೂರ್ವಾರ್ಜಿತ ಕರ್ಮ ಈ ಬೋಧೆಗಳನ್ನು ರಕ್ತಗತಮಾಡಿಕೊಳ್ಳಲು ಬಿಡುವುದಿಲ್ಲ.” 

 ಸ್ವಾಮೀಜಿ: “ನಿನ್ನ ಕರ್ಮ ಮುಂತಾದುವನ್ನು ಆಚೆಗೆ ಬಿಸಾಡು. ನಿನ್ನ\break ಪೂರ್ವಾರ್ಜಿತ ಕರ್ಮದಿಂದ ನೀನು ಈ ಶರೀರವನ್ನು ಹೊಂದಿರುವುದು ನಿಜವಾದರೆ ಹೀನ ಕೆಲಸಗಳ ಪರಿಣಾಮವನ್ನು ಸತ್ಕಾರ್ಯಗಳ ಪುಣ್ಯದಿಂದ ತೊಡೆದುಹಾಕು. ನಿನ್ನ ಈ ದೇಹದಲ್ಲೇ ಜೀವನ್ಮುಕ್ತನಾಗಬಾರದೇಕೆ? ಆತ್ಮಜ್ಞಾನ ನಿನ್ನ ಕೈಯಲ್ಲೇ ಇದೆ ಎಂದು ತಿಳಿ. ಯಥಾರ್ಥ ಜ್ಞಾನದಲ್ಲಿ ಕೆಲಸದ ಸಂಪರ್ಕವೇ ಇರುವುದಿಲ್ಲ. ಜೀವನ್ಮುಕ್ತರಾದ ಮೇಲೂ ಕೆಲಸ ಮಾಡುವವರು ಇತರರ ಕಲ್ಯಾಣಕ್ಕಾಗಿ ಮಾಡುವರು. ಅವರಲ್ಲಿ ಕರ್ಮ ಪ್ರತಿಫಲಾಪೇಕ್ಷೆ ಇರುವುದಿಲ್ಲ. ಆಸೆಯ ಬೀಜಾಂಕುರ ಅವರಲ್ಲಿ ಕೊಂಚವೂ ಇರುವುದಿಲ್ಲ. ಖಂಡಿತವಾದ ಮಾತಿನಲ್ಲಿ ಹೇಳಬೇಕಾದರೆ ಗೃಹಸ್ಥ ಜೀವನದಲ್ಲಿದ್ದು ವಿಶ್ವದ ಕಲ್ಯಾಣಕ್ಕೋಸ್ಕರವಾಗಿ ಕೆಲಸ ಮಾಡುವುದು ಖಂಡಿತ ಅಸಾಧ್ಯ. ಹಿಂದೂ ಧಾರ್ಮಿಕ ಗ್ರಂಥಗಳಲ್ಲೆಲ್ಲಾ ಈ‌ರೀತಿ ಇರುವುದು. ಜನಕರಾಜನ ನಿದರ್ಶನ ಒಂದೇ ಒಂದು. ಈಗಿನ ಕಾಲದಲ್ಲಿ ನೀವೆಲ್ಲಾ ವರ್ಷವರ್ಷಕ್ಕೂ ಮಕ್ಕಳನ್ನು ಹೆರುತ್ತಾ ಜನಕರಾಜನಂತೆ ಸೋಗುಹಾಕಲು ಪ್ರಯತ್ನಿಸುವಿರಿ. ಅವನಿಗಾದರೋ ದೇಹದ ಪ್ರಜ್ಞೆ ಕೂಡ ಇರಲಿಲ್ಲ.” 

 ಶಿಷ್ಯ: “ದಯವಿಟ್ಟು ನಾನೀ ಜನ್ಮದಲ್ಲೇ ಆತ್ಮಸಾಕ್ಷಾತ್ಕಾರ ಹೊಂದುವಂತೆ ಹರಸಿ.” 

 ಸ್ವಾಮೀಜಿ: “ಭಯವೇನು? ನಿನ್ನಲ್ಲಿ ನಿಜವಾಗಿ ಪ್ರಾಮಾಣಿಕತನವಿದ್ದಲ್ಲಿ ಖಂಡಿತವಾಗಿಯೂ ನೀನು ಈ ಜನ್ಮದಲ್ಲೇ ಹೊಂದುವೆ. ಆದರೆ ಮನುಷ್ಯ ಪ್ರಯತ್ನ ಆವಶ್ಯಕ. ಅದೇನೆಂದು ನಿನಗೆ ಗೊತ್ತೆ? ‘ನಾನು ಖಂಡಿತ ಆತ್ಮಸಾಕ್ಷಾತ್ಕಾರ ಹೊಂದುವೆ. ಯಾವ ರೀತಿ ಆತಂಕಗಳೇ ಬರಲಿ ನಾನು ಅವುಗಳನ್ನೆಲ್ಲಾ ಜಯಿಸುವೆ’ - ಇಂತಹ ಅಚಲ ನಿರ್ಧಾರವೇ ಪುರುಷಕಾರ. ‘ನನ್ನ ತಂದೆ, ತಾಯಿ, ಅಣ್ಣ, ತಮ್ಮ, ಗೆಳೆಯ, ಹೆಂಡತಿ, ಮಕ್ಕಳು ಯಾರಿಗೂ ಏನಾದರೂ ಆಗಲಿ. ಅವರು ಬದುಕಲಿ, ಸಾಯಲಿ. ನನಗೆ ಆತ್ಮಸಾಕ್ಷಾತ್ಕಾರವಾಗುವವರೆಗೂ ನಾನು ಎಂದಿಗೂ ಹಿಂತಿರುಗುವುದಿಲ್ಲ’ - ಹೀಗೆ ಎಲ್ಲಾ ನಿಮಿತ್ತಗಳನ್ನೂ ಒತ್ತಟ್ಟಿಗಿಟ್ಟು ಗುರಿ ಸೇರಬೇಕೆಂಬ ಏಕಪ್ರಕಾರವಾದ ಅಚಲ ನಿರ್ಧಾರದಿಂದ ಹೊರಡುವುದೇ ಪುರುಷ ಪ್ರಯತ್ನ. ಇಲ್ಲದಿದ್ದಲ್ಲಿ ಕೇವಲ ಮೃಗಪಕ್ಷಿಗಳೂ ತಮ್ಮ ಹೊಟ್ಟೆಪಾಡಿಗೆ ಕಷ್ಟಪಡುವುವು. ಮನುಷ್ಯನಿಗೆ ಈ ಶರೀರವಿರುವುದು ಆತ್ಮಸಾಕ್ಷಾತ್ಕಾರ ಪಡೆಯಲು. ನೀನು ಪ್ರಪಂಚದ ಸಾಮಾನ್ಯ ಜನರಂತೆ ಜೀವಿಸುತ್ತಾ, ಜನಸಾಮಾನ್ಯರ ಪ್ರವಾಹದಲ್ಲೇ ತೇಲುತ್ತಿದ್ದರೆ, ನಿನಗೆ ಪುರುಷ ಪ್ರಯತ್ನವೆಲ್ಲಿದೆ? ಜನಸಾಮಾನ್ಯರು ಮೃತ್ಯುವಿನ ದವಡೆಗೆ ನುಗ್ಗುತ್ತಿದ್ದಾರೆ. ಆದರೆ ನೀನು ಅದನ್ನು ಜಯಿಸಬೇಕು. ವೀರನಂತೆ ಮುನ್ನುಗ್ಗು, ಯಾವುದರಿಂದಲೂ ಹಿಂತೆಗೆಯಬೇಡ. ಈ ಸುಖದುಃಖದಿಂದ ಕೂಡಿದ ಶರೀರ ಲಭಿಸಿರುವಾಗ ನಿನ್ನ ಆತ್ಮವನ್ನು ಹೊಡೆದೆಬ್ಬಿಸು. ‘ನಾನು ನಿರ್ಭಯಾವಸ್ಥೆಯನ್ನು ಹೊಂದಿರುವೆ, ನಾನು ಆತ್ಮ, ನನ್ನ ಕೀಳುಅಹಂಕಾರ ಶಾಶ್ವತವಾಗಿ ನಾಶ ಹೊಂದಿದೆ’ - ಈ ಭಾವನೆಯಲ್ಲಿ ಪರಿಪೂರ್ಣನಾಗಿ ‘ನಾನು’ ಎಂದು ಹೇಳು. ಅನಂತರ ಎಲ್ಲಿಯವರೆವಿಗೆ ಈ ದೇಹವಿರುವುದೋ ಅಲ್ಲಿಯವರೆಗೂ ಇತರರಿಗೆ ಈ ನಿರ್ಭಯತೆಯ ಸಂದೇಶವನ್ನು ಸಾರು. ‘ನೀನು ಅದೇ ಆಗಿರುವೆ - ಉತ್ತಿಷ್ಠತ ಜಾಗ್ರತ ಪ್ರಾಪ್ಯವರಾನ್ನಿಬೋಧತ…’ ‘ಎದ್ದೇಳು ಎಚ್ಚರಗೊಳ್ಳು, ಗುರಿ ಮುಟ್ಟುವವರೆಗೂ‌ನಿಲ್ಲಬೇಡ.’ ನೀನಿದನ್ನು ಪಡೆಯಬಲ್ಲೆಯಾದರೆ ನೀನು ನಿಜವಾಗಿಯೂ ಸಮರ್ಥ ಪೂರ್ವ ಬಂಗಾಳಿಯೆಂದು ತಿಳಿಯುತ್ತೇನೆ.” 

