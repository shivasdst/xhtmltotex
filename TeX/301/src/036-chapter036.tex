
\chapter{ಹಲವು ಬಗೆಯ ಪ್ರೇಕ್ಷಕರು}

 ಸ್ವಾಮೀಜಿಯವರು ಕಲ್ಕತ್ತೆಯಲ್ಲಿದ್ದಾಗ ಭಿನ್ನ ಭಿನ್ನ ಪ್ರಕೃತಿಯ ಜನರು ಅವರ ಭೇಟಿಗೆ ಬರುತ್ತಿದ್ದರು. ಅವುಗಳಲ್ಲಿ ಕೆಲವು ಚಿತ್ತಾಕರ್ಷಕವಾಗಿದೆ. ಆ ಭೇಟಿಗಳಲ್ಲಿ ಕೆಲವನ್ನು ಇಲ್ಲಿ ಕೊಡುವೆವು. ಒಂದು ದಿನ ಬಂಗಾಳದ ಥಿಯಾಸಫಿ ಸೊಸೈಟಿಯಲ್ಲಿದ್ದ ಒಬ್ಬ ಯುವಕ ಸ್ವಾಮೀಜಿಯವರನ್ನು ನೋಡುವುದಕ್ಕೆ ಬಂದನು. 

 ಯುವಕ: “ಸ್ವಾಮೀಜಿ, ನಾನು ಎಷ್ಟೋ ಸಂಸ್ಥೆಗಳಿಗೆ ಹೋಗಿರುವೆನು. ಆದರೆ ಯಾವುದು ಸತ್ಯ ಎಂಬುದನ್ನು ತಿಳಿಯದೆ ಇರುವೆನು.” 

 ಸ್ವಾಮೀಜಿ: “ವತ್ಸ! ಇದಕ್ಕೆ ಅಂಜಬೇಕಾಗಿಲ್ಲ. ನಾನು ಕೂಡ ಹಿಂದೆ ಒಂದು ಸಲ ಹೀಗೆಯೇ ಇದ್ದೆ. ನಿನಗೆ ಯಾರು ಯಾರು ಏನು ಹೇಳಿಕೊಟ್ಟಿರುವರು ಮತ್ತು ಅವುಗಳಲ್ಲಿ ನೀನು ಎಷ್ಟನ್ನು ಅನುಷ್ಠಾನಕ್ಕೆ ತಂದಿರುವೆ ಹೇಳು?” 

 ಯುವಕ: “ಥಿಯಾಸಫಿಗೆ ಸೇರಿದ ಒಬ್ಬ ಪಂಡಿತರು ವಿಗ್ರಹಾರಾಧನೆಯ ಆವಶ್ಯ ಕತೆಯನ್ನು ಕುರಿತು ಹೇಳಿದಾಗ ಅದರಿಂದ ನನಗೆ ಬಹಳ ತೃಪ್ತಿಯಾಯಿತು. ಅದರಂತೆಯೆ ಬಹಳ ಕಾಲ ಪೂಜೆ ಮತ್ತು ಜಪವನ್ನು ಬಹಳ ಭಕ್ತಿಯಿಂದ ಮಾಡಿರುವೆನು. ಆದರೂ ಶಾಂತಿ ಸಿಕ್ಕಲಿಲ್ಲ. ಅನಂತರ ಯಾರೊ ಧ್ಯಾನ ಮಾಡುವ ಸಮಯದಲ್ಲಿ ಮನಸ್ಸನ್ನು ಖಾಲಿಮಾಡು ಎಂದು ಹೇಳಿದರು. ಹಾಗೆ ಮಾಡುವುದಕ್ಕೆ ತುಂಬಾ ಪ್ರಯತ್ನಪಟ್ಟೆ. ಆದರೂ ಮನಸ್ಸು ಇನ್ನೂ ನಿಗ್ರಹಕ್ಕೆ ಬಂದಿಲ್ಲ ಮತ್ತು ಶಾಂತಿ ಸಿಕ್ಕಿಲ್ಲ. ಆದರೆ ಈಗಲೂ ನನ್ನ ಕೋಣೆಯ ಬಾಗಿಲುಗಳನ್ನು ಹಾಕಿ ಕಣ್ಣನ್ನು ಮುಚ್ಚಿಕೊಂಡು ಧ್ಯಾನ ಮಾಡುವೆನು. ಆದರೂ ಶಾಂತಿ ದೊರಕಿಲ್ಲ. ನನಗೆ ಯಾವುದಾದರೂ ಮಾರ್ಗವನ್ನು ತೋರುವಿರಾ?”‌ 

 ಸ್ವಾಮೀಜಿ (ಬಹಳ ವಿಶ್ವಾಸದಿಂದ): “ವತ್ಸ, ನೀನು ನನ್ನ ಮಾತನ್ನು ಗಮನಿಸುವ ಹಾಗೆ ಇದ್ದರೆ ಮೊದಲು ಬಾಗಿಲನ್ನು ತೆರೆದು ಕಣ್ಣನ್ನು ಬಿಟ್ಟು ಸುತ್ತಲೂ ನೋಡು ಎಂದು ಹೇಳುತ್ತೇನೆ. ನಿನ್ನ ಸುತ್ತಮುತ್ತಲೂ‌ಅನೇಕ ದೀನ ದರಿದ್ರರು ಇರುವರು. ಅವರನ್ನು ನಿನ್ನ ಕೈಲಾದಷ್ಟು ಚೆನ್ನಾಗಿ ಸೇವೆ ಮಾಡು. ರೋಗಿಗೆ ಯಾರೂ ಗತಿಯಿಲ್ಲದೇ ಇದ್ದರೆ ಅವನಿಗೆ ಔಷಧಿ ಪಥ್ಯ ಉಪಚಾರವನ್ನು ಮಾಡು. ತಿನ್ನುವುದಕ್ಕೆ ಏನೂ ಇಲ್ಲದೇ ಇದ್ದರೆ ಅವನಿಗೆ ಏನಾದರೂ ಆಹಾರವನ್ನು ಕೊಡಬೇಕು. ನಿರಕ್ಷರ ಕುಕ್ಷಿಯಾಗಿದ್ದರೆ ನೀನು ವಿದ್ಯಾವಂತನಾದುದರಿಂದ ಅವನಿಗೆ ವಿದ್ಯೆಯನ್ನು ಕೊಡಬೇಕು. ನಿನಗೆ ಶಾಂತಿ ಬೇಕಾದರೆ, ನೀನು ಸಾಧ್ಯವಾದಷ್ಟು ಚೆನ್ನಾಗಿ ಇತರರಿಗೆ ಸೇವೆಯನ್ನು ಮಾಡಬೇಕು.” 

 ಯುವಕ: “ರೋಗಿಗೆ ಸೇವೆ ಮಾಡಲು ಹೋಗಿ ಹೊತ್ತಿಗೆ ಗೊತ್ತಿಗೆ ನಿದ್ರೆ ಆಹಾರ ಇವುಗಳಿಲ್ಲದೆ ನಾನೇ ಖಾಯಿಲ ಬಿದ್ದರೆ?” 

 ಸ್ವಾಮೀಜಿ: “ನಿನ್ನ ಮತುಕತೆಯನ್ನು ಕೇಳಿದರೆ, ನಿನ್ನ ಆರೋಗ್ಯವನ್ನೇ ಬಹು ಮುಖ್ಯವೆಂದು ಗಣಿಸುವ ನೀನು ಅಂತಹ ಅತಿಗೆ ಹೋಗುವವನಲ್ಲ ಎಂದು ಎಲ್ಲರಿಗೂ ಗೊತ್ತಾಗುವುದು.” 

 ಮತ್ತೊಂದು ದಿನ ಶ‍್ರೀರಾಮಕೃಷ್ಣರ ಶಿಷ್ಯನಾದ ಒಬ್ಬ ಪ್ರಾಧ್ಯಾಪಕರು ಸ್ವಾಮೀಜಿ ಅವರನ್ನು ಕೇಳಿದರು: “ನೀವು ಪ್ರಪಂಚಕ್ಕೆ ಸೇವೆ, ಲೋಕಕಲ್ಯಾಣ ಮುಂತಾದುವನ್ನು ಮಾಡಿ ಎಂದು ಹೇಳುತ್ತಿರುವಿರಿ. ಇವುಗಳೆಲ್ಲಾ ಮಾಯಾರಾಜ್ಯದಲ್ಲಿವೆ. ಆದರೆ ವೇದಾಂತದ ಪ್ರಕಾರ ಮಾಯಾಬಂಧನವನ್ನೆಲ್ಲ ಖಂಡಿಸಿ ಮುಕ್ತಿಯನ್ನು ಪಡೆಯುವುದೇ ಜೀವಿಯ ಗುರಿ. ಕೇವಲ ಸಂಸಾರದ ಗೊಂದಲದಲ್ಲಿಯೇ ಸಿಕ್ಕಿಹಾಕಿಕೊಳ್ಳುವಂತಹ ವಿಷಯಗಳನ್ನು ಬೋಧಿಸಿ ಪ್ರಯೋಜನವೇನು?” 

 ಸ್ವಾಮೀಜಿ ತಕ್ಷಣವೇ ಹೇಳಿದರು: “ನಿನ್ನ ಮುಕ್ತಿ ಎಂಬ ಭಾವನೆಯೂ ಮಾಯಾವರಣದ ಒಳಗೆ ತಾನೆ ಇರುವುದು? ವೇದಾಂತ ಆತ್ಮ ನಿತ್ಯಮುಕ್ತವಾದುದು ಎಂದು ಸಾರುವುದಿಲ್ಲವೆ? ನಿತ್ಯಮುಕ್ತನಾದವನು ಮುಕ್ತಿ ಗೆ ಪ್ರಯತ್ನ ಪಡುವುದು ಎಂದರೇನು?” 

 ಶಿಷ್ಯ ನಿರುತ್ತರನಾದ. ಸ್ವಾಮೀಜಿಯವರ ದೃಷ್ಟಿಯಲ್ಲಿ ಎಲ್ಲಿಯವರೆಗೆ ನಾವು ಮಾಯೆಯಲ್ಲಿ ಇರುವೆವು ಎಂದು ಭಾವಿಸುವೆವೋ ಅಲ್ಲಿಯವರೆಗೆ ಇತರರ ಸೇವೆಯಲ್ಲಿ ನಮ್ಮ ಬಾಳನ್ನು ಸವೆಸಿದರೆ, ಫಲಾಪೇಕ್ಷೆಯನ್ನು ಬಿಟ್ಟರೆ, ಅದು ನಮ್ಮ ಚಿತ್ರವನ್ನು ಶುದ್ಧಿಮಾಡುವುದು. ಚಿತ್ತಶುದ್ಧವಾದರೆ ಎಂತಹ ಜ್ಞಾನವಾದರೂ ತಕ್ಷಣವೇ ಹೊಳೆಯುವುದು. ಸ್ವಾರ್ಥದ ರೋಗಕ್ಕೆ ನಿಃಸ್ವಾರ್ಥದ ಮದ್ದು ಕೊಡಬೇಕು. ಅನಂತರ ಸ್ವಾರ್ಥ ಮತ್ತು ನಿಃಸ್ವಾರ್ಥದ ಆಚೆ ಹೋಗಬೇಕು. ಆದಕಾರಣವೆ ಸ್ವಾಮೀಜಿ ಸೇವಾಧರ್ಮವನ್ನು ಒತ್ತಿ ಒತ್ತಿ ಹೇಳುತ್ತಿದ್ದುದು. 

 ಒಂದು ದಿನ ಕಾಮದ ನಿಗ್ರಹದ ವಿಷಯವಾಗಿ ಮಾತನಾಡುತ್ತಿದ್ದಾಗ, “ಕಾಮವೇ ಮನುಷ್ಯನನ್ನು ದೇಹ ಪಂಜರದಲ್ಲಿ ಕೂಡಿಹಾಕುವುದು. ತಾನು ಒಬ್ಬ ಗಂಡಸು ಮತ್ತು ಹೆಂಗಸು ಎಂಬ ಭಾವಕ್ಕೆ ನಮ್ಮನ್ನು ತುತ್ತು ಮಾಡುವುದು. ನಾವು ಅಂತಹ ಭಾವದಿಂದ ಪಾರಾಗುವುದಕ್ಕೆ ಎಂತಹ ಉಗ್ರವಾದ ಮಾರ್ಗವನ್ನು ಬೇಕಾದರೂ ಕೈಗೊಳ್ಳುವುದು ಯೋಗ್ಯವೇ” ಎಂದರು. ತಮ್ಮ ಜೀವನದಿಂದಲೇ ಅದಕ್ಕೆ ಒಂದು ನಿದರ್ಶನವನ್ನು ಕೊಟ್ಟರು. ತಾವು ಯುವಕರಾಗಿದ್ದಾಗ ಒಮ್ಮೆ ಆ ಭಾವನೆ ಮನಸ್ಸಿನಲ್ಲಿ ಬಂದಿತೆಂದೂ, ಅದರಿಂದ ಪಾರಾಗಲು ಎಷ್ಟು ಪ್ರಯತ್ನಪಟ್ಟರೂ ಸಾರ್ಥಕವಾಗಲಿಲ್ಲವೆಂದೂ, ಕೊನೆಗೆ ರೋಸಿ ಒಂದು ಬೆಂಕಿಯ ಅಗ್ಗಿಷ್ಠಿಕೆಯ ಮೇಲೆ ಕುಳಿತುಕೊಂಡರೆಂದೂ ಹೇಳಿದರು. ಬೆಂಕಿಯ ಸುಡತದಿಂದ ಆದ ಗಾಯ ಮಾಗುವುದಕ್ಕೆ ಹಲವು ದಿನಗಳು ಹಿಡಿಯಿತೆಂದೂ ಹೇಳಿದರು. ಅನೇಕ ವೇಳೆ ನಾವು ಕಾಮ ಮುಂತಾದುವು ನಮ್ಮಂತಹ ಅಲ್ಪ ವ್ಯಕ್ತಿಗಳನ್ನು ಮಾತ್ರ ಕಾಡುತ್ತವೆ, ವಿವೇಕಾನಂದರಂತಹ ಮಹತ್ ವ್ಯಕ್ತಿಗಳ ಸಮೀಪಕ್ಕೂ ಅವು ಬರುವುದಿಲ್ಲ ಎಂದು ಭಾವಿಸುವೆವು. ಆದರೆ ಅವು ಬರುತ್ತವೆ. ನಮ್ಮ ಹತ್ತಿರ ಬರುವುದಕ್ಕಿಂತ ಹೆಚ್ಚು ಬಲವಾಗಿಯೇ ಬರುತ್ತವೆ. ಆದರೆ ವ್ಯತ್ಯಾಸವೇನು ಎಂದರೆ ಅವರು ಅಷ್ಟೇ ಬಲವಾಗಿ ಕಟುಕರಂತೆ ನಿಷ್ಟುರತೆಯಿಂದ ಅವನ್ನು ಆಚೆಗೆ ದಬ್ಬುವರು. ನಾವಾದರೋ ಅದಕ್ಕೆ ದಯಾ ದಾಕ್ಷಿಣ್ಯವನ್ನು ತೋರಿ ಅದರ ಬಲೆಗೆ ಬೀಳುವೆವು. 

 ಒಂದು ದಿನ ಸ್ವಾಮೀಜಿಯವರ ರಾಜಯೋಗ ಓದಿದ ಕೆಲವರು ಅದರ ವಿಷಯದ ಮೇಲೆ ಮಾತನಾಡುವುದಕ್ಕೆ ಬಂದರು. ಸ್ವಾಮೀಜಿ ಪ್ರಾಣಾಯಾಮ ವಿಷಯದಲ್ಲಿ ವಿವರವಾಗಿ ಮಾತನಾಡತೊಡಗಿದರು. ಕುಳಿತವರಿಗೆ ಅನ್ನಿಸಿತು, ಸ್ವಾಮೀಜಿಯವರು ಪುಸ್ತಕದಲ್ಲಿ ಬರೆದಿರುವುದು ಅವರಿಗೆ ಗೊತ್ತಿರುವುದರ ಯಾವುದೋ ಅಂಶ ಎಂದು. ಅದೂ ಅಲ್ಲದೆ ತಾವು ಕೇಳುವುದಕ್ಕೆ ಬಂದಿದ್ದು ಪ್ರಣಾಯಾಮದ ವಿಷಯವೆಂಬುದನ್ನು ಸ್ವಾಮೀಜಿಗೆ ತಿಳಿಸಿರಲಿಲ್ಲ. ಆದರೂ ಸ್ವಾಮೀಜಿ ಪ್ರಾಣಾಯಾಮದ ಮೇಲೆಯೇ ಮಾತನಾಡುವುದನ್ನು ನೋಡಿ ಆಶ್ಚರ್ಯವಾಯಿತು. ತಮ್ಮ ಮನಸ್ಸಿನಲ್ಲಿ ಇರುವುದು ಸ್ವಾಮೀಜಿಗೆ ಹೇಗೆ ಗೊತ್ತಾಯಿತು ಎಂದು ಕೇಳಿದರು. ಅದಕ್ಕೆ ಸ್ವಾಮೀಜಿ ಪಾಶ್ಚಾತ್ಯ ದೇಶಗಳಲ್ಲಿಯೂ ಕೆಲವು ವೇಳೆ ಜನ ನನ್ನನ್ನು ಹೀಗೆ ಪ್ರಶ್ನಿಸುತ್ತಿದ್ದರೆಂದು ಹೇಳಿದರು. ಯೋಗಶಕ್ತಿಯಿಂದ ಮತ್ತೊಬ್ಬರ ಮನಸ್ಸಿನಲ್ಲಿರುವ ಭಾವನೆಯನ್ನು ತಿಳಿದುಕೊಳ್ಳುವುದು ಸಾಧ್ಯ ಎಂದರು. ಅದರಂತೆಯೇ ಪೂರ್ವಜನ್ಮಗಳ ಸ್ಮರಣೆ ಕೂಡ ಸಾಧ್ಯವೆಂದರು. ಬಂದವರು, “ಸ್ವಾಮೀಜಿ, ನಿಮಗೆ ಹಿಂದಿನ ಜನ್ಮದ ಸ್ಮೃತಿ ಇದೆಯೆ?” ಎಂದು ಕೇಳಿದರು. ಸ್ವಾಮೀಜಿ ತಕ್ಷಣವೆ ಹೇಳಿದರು: “ಹೌದು ನನಗೆ ಗೊತ್ತಿದೆ” ಎಂದು. ಆಗ ವಿವರವಾಗಿ ಅವರು ಹಿಂದೆ ಯಾರು ಆಗಿದ್ದರು, ಅವುಗಳನ್ನೆಲ್ಲ ಹೇಳಿ ಎಂದು ಕೇಳಿದಾಗ ಸ್ವಾಮೀಜಿ: “ನನಗೆ ಅದು ಸಾಧ್ಯ. ಅದು ಗೊತ್ತಿದೆ. ಆದರೆ ಆ ವಿಷಯವಾಗಿ ನಿಮಗೆ ಏನನ್ನೂ ಹೇಳಬಯಸುವುದಿಲ್ಲ” ಎಂದರು. 

 ಒಂದು ದಿನ ಸ್ವಾಮೀಜಿ ಶೀಲರ ತೋಟದ ಮನೆಯಲ್ಲಿ ಪ್ರೇಮಾನಂದರೊಡನೆ\break ಸಾಯಂಕಾಲ ಮಾತನಾಡುತ್ತಿದ್ದರು. ಇದ್ದಕ್ಕಿದ್ದಂತೆಯೇ ಮಾತನ್ನು ನಿಲ್ಲಿಸಿ ಪ್ರೇಮಾನಂದರನ್ನು “ನಿನಗೆ ಏನಾದರೂ ಕಂಡಿತೆ?” ಎಂದು ಕೇಳಿದರು. ಇಲ್ಲ ಎಂದಾಗ ಸ್ವಾಮೀಜಿ, “ತಲೆಯಿಲ್ಲದ ಪ್ರೇತ ಒಂದು ರುಂಡವನ್ನು ಕೈಯಲ್ಲಿ ಹಿಡಿದುಕೊಂಡು ಬಂದು ತನ್ನನ್ನು ಆ ಸ್ಥಿತಿಯಿಂದ ಪಾರುಮಾಡೆಂದು ಬೇಡಿಕೊಳ್ಳುತ್ತಿತ್ತು” ಎಂದರು. ಸ್ವಾಮೀಜಿ ಮನಸ್ಸಿನಲ್ಲಿಯೇ ಆ ಪ್ರೇತಶರೀರಿಗೆ ಆ ಅವಸ್ಥೆಯಿಂದ ಪಾರಾಗಲಿ ಎಂದು ಪ್ರಾರ್ಥನೆ ಮಾಡಿದರು. ಅನಂತರ ವಿಚಾರಿಸಿದಾಗ ಹಿಂದೆ ಆ ಮನೆಯಲ್ಲಿ ಒಬ್ಬ ಬಡ್ಡಿ ಆಸೆಗೆ ಹಣವನ್ನು ಕೊಡುತ್ತಿದ್ದನೆಂದೂ, ಅನಂತರ ಯಾರೋ ಅವನ ಕತ್ತನ್ನು ಕತ್ತರಿಸಿ ಗಂಗಾನದಿಗೆ ಬಿಸಾಡಿದರೆಂದೂ ತಿಳಿಯಬಂದಿತು. ಸ್ವಾಮೀಜಿಯವರ ಹೃದಯ ಸ್ಥೂಲದೇಹಧಾರಿಗಳಿಗೆ ಮಾತ್ರವಲ್ಲ, ಸೂಕ್ಷ್ಮ ಪ್ರೇತ ಶರೀರಿಗಳಿಗೂ ಅಷ್ಟೇ ಮರುಕ ಪಡುತ್ತಿತ್ತು. 

 ಗೋರಕ್ಷಣೀ ಸಭೆಗಾಗಿ ಕೆಲಸ ಮಾಡುತ್ತಿದ್ದ ಪ್ರಚಾರಕನೊಬ್ಬನು ಸ್ವಾಮೀಜಿ ದರ್ಶನಕ್ಕಾಗಿ ಬಂದನು. ಆತನ ವೇಷಭೂಷಾಣಾದಿಗಳು ಬಹಳ ಮಟ್ಟಿಗೆ ಸಂನ್ಯಾಸಿಯ ಹಾಗೆ. ತಲೆಗೆ ಕಾವಿಯ ಬಣ್ಣದ ಕುಲಾವಿ ಹಾಕಿಕೊಂಡಿದ್ದ. ನೋಡಿದ ಕೂಡಲೆ ಉತ್ತರ ಹಿಂದೂಸ್ಥಾನದವನೆಂದೂ ಹೇಳಿಬಿಡಬಹುದು. ಈ ಗೋರಕ್ಷಾ ಪ್ರಚಾರಕನು ಬಂದಿದ್ದ ವರ್ತಮಾನವನ್ನು ಕೇಳಿ ಸ್ವಾಮೀಜಿ ಹೊರಗಿನ ಮನೆಗೆ ಬಂದರು. ಪ್ರಚಾರಕನು ಸ್ವಾಮೀಜಿಯವರನ್ನು ಅಭಿನಂದಿಸಿ ಗೋಮಾತೆಯ ಒಂದು ಚಿತ್ರವನ್ನು ಅವರಿಗೆ ಸಮರ್ಪಿಸಿದನು. ಸ್ವಾ,ಮೀಜಿ ಅದನ್ನು ತೆಗೆದುಕೊಂಡು ಹತ್ತಿರದಲ್ಲಿದ್ದ ಒಬ್ಬರ ಕೈಯಲ್ಲಿ ಕೊಟ್ಟು ಆತನೊಡನೆ ಮುಂದೆ ಹೇಳುವಂತೆ ಸಂಭಾಷಣೆ ಮಾದಿದರು: 

 ಸ್ವಾಮೀಜಿ: “ನಿಮ್ಮ ಸಭೆಯ ಉದ್ದೇಶವೇನು?” 

 ಪ್ರಚಾರಕ: “ನಮ್ಮ ದೇಶದ ಗೋಮಾತೆಯನ್ನು ಕಟುಕರ ಕೈಯಿಂದ ತಪ್ಪಿಸಿ ಕಾಪಾಡುತ್ತೇವೆ. ಅಲ್ಲಲ್ಲಿ ದೊಡ್ಡಿಗಳನ್ನು ಸ್ಥಾಪಿಸುವೆವು. ಅವುಗಳಲ್ಲಿ ಖಾಯಿಲೆಯ, ಕೈಲಾಗದ ಮತ್ತು ಕಟುಕರಿಂದ ಕೊಂಡು ತಂದ ಗೋಮಾತೆಯನ್ನು ರಕ್ಷಿಸುತ್ತೇವೆ.” 

 ಸ್ವಾಮೀಜಿ: “ಇದು ಬಹಳ ಒಳ್ಳೆಯ ಕೆಲಸ. ನಿಮ್ಮ ಸಂಪಾದನೆಗೆ ಮಾರ್ಗ?” 

 ಪ್ರಚಾರಕ: “ದಯಾವಂತರಾದ ತಮ್ಮಂಥವರು ಏನಾದರೂ ಕೊಡುತ್ತಾರೆಯಲ್ಲ, ಅದರಿಂದಲೇ ಸಭೆಯ ಕಾರ್ಯ ನಡೆಯುವುದು” 

 ಸ್ವಾಮೀಜಿ: “ನಿಮ್ಮ ಹತ್ತಿರ ಮೂಲಧನ ಎಷ್ಟು ರೂಪಾಯಿಗಳು ಇವೆ?” 

 ಪ್ರಚಾರಕ: “ಮಾರ್ವಾಡಿ ವರ್ತಕರು ಈ ನಮ್ಮ ಕಾರ್ಯಕ್ಕೆ ಒಳ್ಳೆಯ ಪೋಷಕರಾಗಿದ್ದಾರೆ. ಅವರು ಈ ಸತ್ಕಾರ‍್ಯಕ್ಕಾಗಿ ಬಹಳ ದ್ರವ್ಯವನ್ನು ಕೊಟ್ಟಿರುವರು.” 

 ಸ್ವಾಮೀಜಿ: “ಮಧ್ಯ ಹಿಂದೂಸ್ಥಾನದಲ್ಲಿ ಎಂತಹ ಭಯಂಕರ ಕ್ಷಾಮ ಬಂದಿದೆ. ಹೊಟ್ಟೆಗೆ ಹಿಟ್ಟು ಇಲ್ಲದೆ ಒಂದು ಲಕ್ಷ ಜನ ಸತ್ತುಹೋದರೆಂದು ಇಂಡಿಯಾ ಸರ್ಕಾರದವರು ಪಟ್ಟಿ ಕೊಟ್ಟಿದ್ದಾರೆ. ನಿಮ್ಮ ಸಭೆ ಈ ಬರಗಾಲದಲ್ಲಿ ಏನಾದರೂ ಸಹಾಯ ಮಾಡುವುದಕ್ಕೆ ಏರ್ಪಾಡು ಮಾಡಿದೆ ಏನು?” 

 ಪ್ರಚಾರಕ: “ನಾವು ಬರಗಾಲ ಮೊದಲಾದವುಗಳಲ್ಲಿ ಸಹಾಯ ಮಾಡುವುದಿಲ್ಲ. ಕೇವಲ ಗೋಮಾತೆಯ ರಕ್ಷಣೆಗೆ ಈ ಸಭೆ ಸ್ಥಾಪಿಸಲ್ಪಟ್ಟಿರುವುದು.” 

 ಸ್ವಾಮೀಜಿ: “ಅಣ್ಣ ತಮ್ಮಂದಿರಾದ ನಮ್ಮ ದೇಶದ ಜನರು ಲಕ್ಷಗಟ್ಟಲೆ ಸಾಯುತ್ತಿರುವಾಗ ಕೈಯ್ಯಲಾಗುತ್ತಿದ್ದರೂ ಇಂತಹ ಭಯಂಕರವಾದ ದುಷ್ಕಾಲದಲ್ಲಿ ಅವರಿಗೆ ಅನ್ನ ಕೊಟ್ಟು ಸಹಾಯ ಮಾಡುವುದು ಯುಕ್ತವೆಂದು ಮನಸ್ಸಿಗೆ ತೋಚುವುದಿಲ್ಲವೋ?” 

 ಪ್ರಚಾರಕ: “ಇಲ್ಲ, ಜನರ ಕರ್ಮಫಲದಿಂದ ಪಾಪದಿಂದ ಈ ಕ್ಷಾಮ ಬಂದಿದೆ. ಕರ್ಮಕ್ಕೆ ತಕ್ಕ ಫಲವಾಗಿದೆ.” 

 ಪ್ರಚಾರಕರ ಮಾತನ್ನು ಕೇಳಿ ಸ್ವಾಮೀಜಿಯ ವಿಶಾಲವಾದ ಕಣ್ಣುಗಳಲ್ಲಿ ಬೆಂಕಿಯ ಕಿಡಿಗಳು ಉದುರುವಂತೆ ಕಂಡಿತು. ಮುಖ ಕೆಂಪಾಯಿತು. ಆದರೆ ಮನಸ್ಸಿನ ಭಾವವನ್ನು ಬಿಗಿಹಿಡಿದುಕೊಂಡು ಹೇಳಿದ್ದೇನೆಂದರೆ “ಯಾವ ಸಭಾ ಸಮಿತಿಗಳು ಮನುಷ್ಯರಲ್ಲಿ ಸಹಾನುಭೂತಿಯನ್ನು ತೋರಿಸದೆ, ತಮ್ಮ ಅಣ್ಣ-ತಮ್ಮಂದಿರು ಹೊಟ್ಟೆಗೆ ಇಲ್ಲದೆ ಸಾಯುತ್ತಿದ್ದಾರೆಂದು ನೋಡಿಯೂ ಅವರ ಜೀವವನ್ನು ಉಳಿಸುವುದಕ್ಕಾಗಿ ಒಂದು ತುತ್ತು ಅನ್ನವನ್ನು ಕೊಡದೆ, ಪಶುಪಕ್ಷಿಗಳ ಸಂರಕ್ಷಣೆಗಾಗಿ ರಾಶಿ ರಾಶಿ ಅನ್ನವನ್ನು ದಾನಮಾಡುತ್ತವೆಯೋ, ಅವುಗಳೊಡನೆ ನನಗೆ ಸ್ವಲ್ಪವಾದರೂ ಸಹಾನುಭೂತಿಯಿಲ್ಲ. ಅವುಗಳಿಂದ ಸಮಾಜಕ್ಕೆ ಹೆಚ್ಚು ಉಪಕಾರವಾಗುತ್ತದೆಯೆಂದು ನಾನು ನಂಬುವುದಿಲ್ಲ. ಕರ್ಮಫಲದಿಂದ ಸಾಯುತ್ತಾರೆ. ಹೀಗೆ ಕರ್ಮದ ನೆವವನ್ನು ಹೇಳುವುದಾದರೆ ಜಗತ್ತಿನಲ್ಲಿ ಯಾವ ವಿಷಯದಲ್ಲಿಯೂ ಮಾಡುವುದು ಒಟ್ಟಿಗೆ ನಿಷ್ಪ್ರಯೋಜನವೆಂದು ನಿಶ್ಚಯಿಸಬಹುದು. ನಿಮ್ಮ ಪಶು ರಕ್ಷಣೆಯ ಕೆಲಸವೂ ಆಮೇಲೆ ನಡೆಯುವುದಿಲ್ಲ. ಈ ಕೆಲಸದ ವಿಚಾರದಲ್ಲಿಯೂ ‘ಗೋಮಾತೆಗಳು ತಮ್ಮ ತಮ್ಮ ಕರ್ಮಫಲದಿಂದಲೇ ಕಟುಕರ ಕೈಗೆ ಹೋಗುತ್ತವೆ. ಆದುದರಿಂದ ಅದಕ್ಕೆ ನಾವು ಏನೂ ಮಾಡಬೇಕಾದ ಆವಶ್ಯಕತೆ ಇಲ್ಲ’ ಎಂದು ಹೇಳಬಹುದು.” 

 ಪ್ರಚಾರಕನು ಸ್ವಲ್ಪ ಅಪ್ರತಿಭನಾಗಿ ಹೇಳಿದ: “ತಾವು ಹೇಳುವುದೇನೋ ನಿಜ. ಆದರೆ ಹಸು ನಮಗೆ ತಾಯಿ ಎಂದು ಶಾಸ್ತ್ರ ಹೇಳುತ್ತದೆ.” 

 ಸ್ವಾಮೀಜಿ (ನಗುನಗುತ್ತ): “ವಿಲಕ್ಷಣಾರ್ಥದಲ್ಲಿ ಅರ್ಥಮಾಡಿಕೊಂಡಿದ್ದೇನೆ. ಇಲ್ಲದೇ ಇದ್ದರೆ ಇಂತಹ ಧನ್ಯರಾದ ಪುತ್ರರನ್ನೆಲ್ಲ ಇನ್ನಾರು ಹೆತ್ತಾರು?” 

 ಆ ಹಿಂದೂಸ್ಥಾನಿ ಪ್ರಚಾರಕನು ಈ ವಿಷಯದಲ್ಲಿ ಮತ್ತೇನನ್ನೂ ಹೇಳದೆ ಸ್ವಾಮೀಜಿಯವರನ್ನು ಕುರಿತು ಆ ಸಭೆಗಾಗಿ ಅವರಿಂದ ತಾನು ಏನನ್ನಾದರೂ ಭಿಕ್ಷೆ ಬೇಡುವೆನು ಎಂದು ತಿಳಿಸಿದನು. 

 ಸ್ವಾಮೀಜಿ: “ನನ್ನನ್ನು ನೋಡಿದರೆ ಸಂನ್ಯಾಸಿ ಭಿಕ್ಷುಕ, ನಿಮಗೆ ಸಹಾಯ ಮಾಡುವುದಕ್ಕೆ ನಾನು ಹಣವನ್ನು ಎಲ್ಲಿಂದ ತರಲಿ? ನನ್ನ ಕೈಯಲ್ಲಿ ಯಾವುದಾದರೂ ಯಾವಾಗಲಾದರೂ ಹಣವಿದ್ದರೆ ಅದನ್ನು ಮೊದಲು ಮನುಷ್ಯ ಸೇವೆಯಲ್ಲಿ ವೆಚ್ಚ ಮಾಡುತ್ತೇನೆ. ಮನುಷ್ಯನನ್ನು ಮೊದಲು ಬದುಕಿಸಿಕೊಳ್ಳಬೇಕು. ಅನ್ನದಾನ, ವಿದ್ಯಾದಾನ, ಧರ್ಮದಾನಗಳನ್ನು ಮಾಡಬೇಕು. ಅದೆಲ್ಲವನ್ನೂ ಮಾಡಿ ಹಣ ಮಿಕ್ಕರೆ ಆಗ ನಿಮ್ಮ ಸಭೆಗೆ ಏನಾದರೂ ಕೊಡುವುದಕ್ಕೆ ಬಂದೇನು.” 

 ಈ ಮಾತನ್ನು ಕೇಳಿ ಪ್ರಚಾರಕ ಮಹಾಶಯನು ಸ್ವಾಮೀಜಿಗೆ ನಮಸ್ಕರಿಸಿ ಹೊರಟುಹೋದನು. ಆಗ ಸ್ವಾಮೀಜಿ ಅಲ್ಲಿದ್ದವರಿಗೆ ಹೇಳಿದರು: “ಆತನು ಹೇಳಿದ್ದೇನು? ಕರ್ಮಫಲದಿಂದ ಮನುಷ್ಯರು ಸತ್ತುಹೋಗುತ್ತಾರೆ. ಅವರಿಗೆ ದಯೆ ತೋರಿದರೆ ಆಗುವುದೇನು? ದೇಶ ತೀರ ಹೀನಸ್ಥಿತಿಗೆ ಬಂದಿದೆ ಎಂಬುದಕ್ಕೆ ಇದು ದೊಡ್ಡ ಸಾಕ್ಷಿಯಾಗಿದೆ. ನಮ್ಮ ಹಿಂದೂಧರ್ಮದ ಕರ್ಮಸಿದ್ಧಾಂತ ಎಲ್ಲಿಗೆ ಬಂದು ನಿಂತುಕೊಂಡಿದೆ ನೋಡಿದೆಯಾ? ಮನುಷ್ಯರ ಕಷ್ಟಕ್ಕಾಗಿ ಅನುಕಂಪದಿಂದ ದುಃಖಿಸದ ಮನುಷ್ಯನಿದ್ದರೆ ಅಂಥವನು ಮನುಷ್ಯನೆ?” ಈ ಮಾತಗಳನ್ನು ಆಡುತ್ತ ಆಡುತ್ತ ಸ್ವಾಮೀಜಿಯ ದೇಹಾದ್ಯಂತವೂ ಚಿತ್ತಕ್ಷೋಭದಿಂದಲೂ ದುಃಖದಿಂದಲೂ ರೋಮಾಂಚವಾದಂತಾಯಿತು. 

 ಕಲ್ಕತ್ತೆಯ ಬಡೋ ಬಜಾರಿನಲ್ಲಿ ಪಂಡಿತರು ವಾಸವಾಗಿದ್ದರು. ಹಣವಂತರಾದ ಮಾರ್ವಾಡಿ ವರ್ತಕರ ಅನ್ನದಿಂದ ಇವರು ಪೋಷಿತರು. ಈ ವೇದಶಾಸ್ತ್ರಜ್ಞರಾದ ಪಂಡಿತರೆಲ್ಲ ಆ ಕಾಲದಲ್ಲಿ ಸ್ವಾಮಿಗಳ ಕೀರ್ತಿಯನ್ನು ಕೇಳಿದ್ದರು. ಅವರಲ್ಲಿ ಕೆಲವು ದೊಡ್ಡ ಪಂಡಿತರು ಸ್ವಾಮೀಜಿಯವರೊಡನೆ ವಾಕ್ಯಾರ್ಥ ಮಾಡುವ ಅಭಿಲಾಷೆಯಿಂದ ಒಂದು ದಿನ ಸ್ವಾಮೀಜಿಯವರು ಇದ್ದಕಡೆ ಬಂದರು. ಬಂದಿದ್ದ ಪಂಡಿತರೆಲ್ಲ ಸಂಸ್ಕೃತ ಭಾಷೆಯಲ್ಲಿ ನಿರರ್ಗಳವಾಗಿ ಮಾತನಾಡುವ ಸಾಮರ್ಥ್ಯವುಳ್ಳವರಾಗಿದ್ದರು. ಅವರು ಬಂದ ಕೂಡಲೆ ಗುಂಪಿನಿಂದ ಸುತ್ತುವರಿಯಲ್ಪಟ್ಟ ಸ್ವಾಮಿಗಳನ್ನು ಮಾತನಾಡಿಸಿ, ಅವರೊಡನೆ ಸಂಸ್ಕೃತದಲ್ಲಿ ಮಾತು ಮೊದಲು ಮಾಡಿದರು. ಸ್ವಾಮೀಜಿ ಸಂಸ್ಕೃತದಲ್ಲಿಯೇ ಅವರಿಗೆ ಉತ್ತರ ಕೊಡುತ್ತಿದ್ದರು. ಸ್ವಾಮೀಜಿ ಪಂಡಿತರೊಡನೆ ವಾದಮಾಡುವಾಗ ಸಿದ್ಧಾಂತ ಪಕ್ಷವನ್ನು ಹಿಡಿದಿದ್ದರು. ಪಂಡಿತರು ಪೂರ್ವಪಕ್ಷವನ್ನು ಹಿಡಿದಿದ್ದರು. ಸ್ವಾಮೀಜಿ ಒಂದು ಕಡೆ ‘ಸ್ವಸ್ತಿ’ ಎಂದು ಹೇಳುವಾಗ ಬಾಯಿತಪ್ಪಿ ‘ಅಸ್ತಿ’ ಎಂದರು. ಪಂಡಿತರು ನಗುವುದಕ್ಕೆ ತೊಡಗಿದರು. ಅದನ್ನು ನೋಡಿ ಸ್ವಾಮೀಜಿ ತಕ್ಷಣವೆ, “ಪಂಡಿತಾನಾಂ ದಾಸೋಽಹಂ ಕ್ಷಂತವ್ಯಮೇತತ್ ಸ್ಖಲನಂ” (ನಾನು ಪಂಡಿತರ ದಾಸ, ನನ್ನ ಈ ವ್ಯಾಕರಣದ ತಪ್ಪನ್ನು ಕ್ಷಮಿಸಬೇಕು) ಎಂದು ಹೇಳಿದರು. ಪಂಡಿತರು ಸ್ವಾಮಿಗಳ ಈ ವಿಧವಾದ ನಮ್ರತೆಯನ್ನು ನೋಡಿ ಬೆರಗಾಗಿ ಹೋದರು. ಬಹಳ ಹೊತ್ತು ವಾದಾನುವಾದಗಳಾದ ಮೇಲೆ ಸಿದ್ಧಾಂತಪಕ್ಷವನ್ನು ಹಿಡಿದು ವಿಚಾರಮಾಡಿದ್ದು ಸಮರ್ಪಕವಾಯಿತೆಂದು ಪಂಡಿತರು ಒಪ್ಪಿಕೊಂಡು ಸಂತೋಷದಿಂದ ಅಲ್ಲಿಂದ ಹೊರಡುವುದಕ್ಕೆ ಸಿದ್ಧರಾದರು. ಅಲ್ಲಿಗೆ ಬಂದಿದ್ದ ನಾಲ್ಕೈದು ಜನ ದೊಡ್ಡ ಮನುಷ್ಯರು ಅವರ ಹಿಂದೆ ಹೋಗಿ, “ಮಹಾಶಯರೆ, ಸ್ವಾಮೀಜಿ ವಿಷಯವಾಗಿ ತಾವೇನು ತಿಳಿದುಕೊಳ್ಳೋಣವಾಯಿತು?” ಎಂದು ಕೇಳಿದರು. ಅದಕ್ಕೆ ಅವರಲ್ಲಿ ಎಲ್ಲರಿಗಿಂತಲೂ ಹಿರಿಯರಾದ ಪಂಡಿತರು “ವ್ಯಾಕರಣದಲ್ಲಿ ಹೆಚ್ಚು ಆಳವಾದ ಪಾಂಡಿತ್ಯ ಇಲ್ಲದೇ ಇದ್ದರೂ ಸ್ವಾಮಿಗಳು ಶಾಸ್ತ್ರದ ಗೂಡಾರ್ಥವನ್ನು ಬಲ್ಲವರು, ವಾಕ್ಯಾರ್ಥ ಮಾಡುವುದರಲ್ಲಿ ಅದ್ವಿತೀಯರು. ತಮ್ಮ ಪ್ರತಿಭಾ ಬಲದಿಂದ ಖಂಡನೆ ಮಾಡುತ್ತ ಅದ್ಭುತ ಪಾಂಡಿತ್ಯವನ್ನು ತೋರಿಸಿದ್ದಾರೆ” ಎಂದು ಉತ್ತರ ಕೊಟ್ಟರು. ಸ್ವಾಮೀಜಿ ಮೇಲೆ ಅವರ ಗುರುಭಾಯಿಗಳಿಗೆ ಎಂತಹ ಅದ್ಭುತವಾದ ಪ್ರೀತಿ ಇತ್ತು! ಪಂಡಿತರೊಡನೆ ಸ್ವಾಮೀಜಿಗೆ ಜಟಿಲವಾದ ಹತ್ತಿಕೊಂಡಾಗ ಅವರು ಕುಳಿತಿದ್ದ ಕೊಠಡಿ ಉತ್ತರದಲ್ಲಿದ್ದ ಕೊಠಡಿಯಲ್ಲಿ ರಾಮಕೃಷ್ಣಾನಂದ ಸ್ವಾಮಿಗಳು ಜಪ ಮಾಡುತ್ತಿದ್ದುದು ಕಂಡಿತು. ಪಂಡಿತರೆಲ್ಲರೂ ಹೊರಟುಹೋದಮೇಲೆ ಇದಕ್ಕೆ ಕಾರಣವನ್ನು ವಿಚಾರಿಸಲಾಗಿ, ಸ್ವಾಮೀಜಿಗೆ ವಿಜಯವಾಗುವುದಕ್ಕೆ ಒಂದೇ ಮನಸ್ಸಿನಿಂದ ಪರಮಹಂಸರ ಪಾದಪದ್ಮಗಳನ್ನು ಅವರು ಪ್ರಾರ್ಥನೆ ಮಾಡುತ್ತಿದ್ದರೆಂದು ತಿಳಿದು ಬಂತು. 

 ಪಂಡಿತರು ಹೊರಟುಹೋದಮೇಲೆ ಪೂರ್ವಪಕ್ಷಪರವಾಗಿ ಮಾತನಾಡುತ್ತಿದ್ದ ಆ ಪಂಡಿತರು ಪೂರ್ವಮೀಮಾಂಸಾ ಶಾಸ್ತ್ರದಲ್ಲಿ ದೊಡ್ಡ ವಿದ್ವಾಂಸರೆಂದು ಶಿಷ್ಯನು ಸ್ವಾಮಿಗಳಿಂದ ತಿಳಿದುಕೊಂಡನು. ಸ್ವಾಮೀಜಿ ಉತ್ತರ ಮೀಮಾಂಸಾ ಪಕ್ಷವನ್ನು ಹಿಡಿದು ಅವರಿಗೆ ಜ್ಞಾಪಕಾಂಡದ ಶ್ರೇಷ್ಠತೆಯನ್ನು ತೋರಿಸಿದರು. ಪಂಡಿತರು ಸ್ವಾಮೀಜಿಯವರ ಸಿದ್ಧಾಂತವನ್ನು ಒಪ್ಪಿಕೊಳ್ಳಲೇಬೇಕಾಯಿತು. 

 ಸ್ವಾಮೀಜಿಯವರು ಅನಂತರ ಹೀಗೇ ಹೇಳಿದರು. “ಪಾಶ್ಚಾತ್ಯ ದೇಶದಲ್ಲಿ ವಾದ ಮಾಡುತ್ತಿರುವಾಗ ಮೂಲ ವಿಷಯವನ್ನು ಬಿಟ್ಟು, ಹೀಗೆ ಭಾಷೆಯ ಸಾಮಾನ್ಯವಾದ ದೋಷವನ್ನು ಹಿಡಿಯುವುದು ಪ್ರತಿಪಕ್ಷದ ಮಹಾ ಅಸೌಜನ್ಯವನ್ನು ತೋರುತ್ತದೆ. ಸಭ್ಯಸಮಾಜ ಅರ್ಥವನ್ನು ಗ್ರಹಿಸುತ್ತದೆಯೆ ಹೊರತು ಭಾಷೆಯನ್ನು ಗಮನಿಸುವುದಿಲ್ಲ.” 

