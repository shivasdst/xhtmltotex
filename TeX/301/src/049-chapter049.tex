
\chapter{ಸ್ವಾಮೀಜಿ, ಸೋದರಿ ನಿವೇದಿತೆಯ ಲೇಖನಿ ಮೂಲಕ}

 ಸ್ವಾಮೀಜಿಯವರ ಜೊತೆಯಲ್ಲಿ ನಿವೇದಿತಾ ಇಂಗ್ಲೆಂಡಿನವರೆಗೆ ಹೊರಟಳು ಎಂಬುದನ್ನು ಆಗಲೇ ಹೇಳಿರುವೆವು. ಸ್ವಾಮೀಜಿ ಪ್ರತಿದಿನವೂ ಆಕೆಯೊಡನೆ ಮಾತನಾಡುತ್ತಿದ್ದಾಗ ಸೋದರಿ ಅದನ್ನೆಲ್ಲ ಬರೆದು ಇಡುತ್ತಿದ್ದಳು. ಅನಂತರ ಅದನ್ನು “ನಾ ಕಂಡಂತೆ ನನ್ನ ಗುರುದೇವ” ಎಂಬ ಗ್ರಂಥದಲ್ಲಿ ಚಿತ್ರಿಸಿರುವರು. ಸ್ವಾಮೀಜಿಯವರ ಬಹುಮುಖದ ಶೀಲವನ್ನು ತಿಳಿದುಕೊಳ್ಳುವುದಕ್ಕೆ ನಮಗೆ ಈ ಗ್ರಂಥ ಅತ್ಯಂತ ಸಹಾಯ ಮಾಡುವುದು. ಇದರಲ್ಲಿ ಸೋದರಿ ತಾನು ಏನನ್ನು ಸ್ವಾಮೀಜಿ ಬಾಯಿಂದ ಕೇಳಿದಳೊ, ಏನನ್ನು ಅವರಲ್ಲಿ ನೋಡಿದಳೋ ಅದನ್ನು ಮಾತ್ರ ವಿವರಿಸುತ್ತಾಳೆ. ನಾವು ಆಕೆಯ ಲೇಖನದ ಮೂಲಕವಾಗಿಯೇ ಅದನ್ನು ಓದುಗರಿಗೆ ಕೊಡುವೆವು.

\section*{ಅರ್ಧ ಪ್ರಪಂಚದ ಸುತ್ತ ಸ್ವಾಮೀಜಿಯವರೊಡನೆ }

 ಆರು ವಾರಗಳ ಪ್ರಯಾಣ ಕಾಲದಲ್ಲಿ ಸ್ವಾಮೀಜಿ ಸಾನ್ನಿಧ್ಯದಲ್ಲಿ ನಾನಿದ್ದುದನ್ನು ನನ್ನ ಜೀವನದಲ್ಲಿ ಅತ್ಯಂತ ಮುಖ್ಯವಾದ ಕಾಲವೆಂದು ನಾನು ಭಾವಿಸುತ್ತೇನೆ. ಸ್ವಾಮೀಜಿ ಸಮೀಪದಲ್ಲಿದ್ದು ಅವರು ಹೇಳುವುದನ್ನು ಕೇಳುವ ಯಾವ ಅವಕಾಶವನ್ನೂ ನಾನು ಕಳೆದುಕೊಳ್ಳಲಿಲ್ಲ. ನನ್ನ ಇತರ ಕಾಲವನ್ನು ಬರೆಯುವುದು ಕಸೂತಿಯ ಕೆಲಸ ಇವುಗಳಲ್ಲಿ ಕಳೆದೆ. ಅವರ ಮನಸ್ಸು ಮತ್ತು ವ್ಯಕ್ತಿತ್ವದ ಪ್ರಭಾವಕ್ಕೆ ಅಷ್ಟು ಕಾಲ ಒಳಗಾದುದಕ್ಕೆ ನಾನು ಎಷ್ಟು ಋಣಿಯಾಗಿದ್ದರೂ ಸಾಲದು. 

 ನಮ್ಮ ಪ್ರಯಾಣದ ಪ್ರಾರಂಭದಿಂದಲೂ ಕೊನೆಯವರೆಗೆ ಅವರ ಆಲೋಚನೆ ಮತ್ತು ಕಥೆಗಳಿಗೆ ಬಿಡುವೇ ಇರಲಿಲ್ಲ. ಯಾವ ಸಮಯದಲ್ಲಿ ಯಾವ ಸ್ಫೂರ್ತಿಯಿಂದ ಪ್ರೇರಿತರಾಗಿ ಯಾವ ಹೊಸ ವಿಷಯವನ್ನು ಹೇಳುತ್ತಿದ್ದರೋ ಗೊತ್ತಿರಲಿಲ್ಲ. ಮೊದಲನೆ ಮಧ್ಯಾಹ್ನದಲ್ಲಿ ಹಡಗು ಇನ್ನೂ ಗಂಗಾ ನದಿಯಲ್ಲಿ ಹೋಗುತ್ತಿದ್ದಾಗ ಅವರು ಇದ್ದಕ್ಕಿದ್ದಂತೆಯೇ ಹೇಳಿದರು: “ದಿನ ಕಳೆದಂತೆ ಎಲ್ಲಾ ಪೌರುಷದಲ್ಲಿರುವಂತೆ ನನಗೆ ತೋರುತ್ತಿದೆ. ಇದೇ ನನ್ನ ಹೊಸ ಸಂದೇಶ. ಪಾಪ ಮಾಡಬೇಕಾದರೂ ಧೀರನಂತೆ ಮಾಡು. ಬೇರೆ ವಿಧಿ ಇಲ್ಲದೇ ಇದ್ದರೆ ಮಹಾದುಷ್ಟನಾಗು!” ನಾನು ಮತ್ತೊಂದು ದಿನ ಕೇಳಿದ ಮತ್ತೊಂದು ಮಾತು ಇದಕ್ಕೆ ಸರಿಸಮವಾಗಿದೆ. ಇಂಡಿಯಾ ದೇಶದಲ್ಲಿ ಕ್ರೌರ್ಯಕೃತ್ಯಗಳು ಬಹಳ ಕಡಿಮೆ ಎಂದೆ. ಅವರು ಅದಕ್ಕಾಗಿ ವ್ಯಥೆ ಪಟ್ಟಂತೆ “ದೇವರ ದಯೆಯಿಂದ ನನ್ನ ದೇಶದಲ್ಲಿ ಅದು ಹೆಚ್ಚಾಗಿದ್ದಿದ್ದರೆ! ಸಾಧು ಸ್ವಭಾವ ಮೃತ್ಯು ಚಿಹ್ನೆ” ಎಂದರು. ಶಿವರಾತ್ರಿಯ ಕಥೆ, ಪೃಥ್ವಿರಾಜನ ಕಥೆ, ವಿಕ್ರಮಾದಿತ್ಯನ ನಿರ್ಣಯ, ಬುದ್ಧ ಮತ್ತು ಯಶೋಧರೆಯರು, ಹೀಗೆ ನೂರಾರು ವಿಷಯಗಳು ಒಂದಾದ ಮೇಲೊಂದು ಬರುತ್ತಿದ್ದವು. ಅದರಲ್ಲಿ ಒಂದು ಗಮನಾರ್ಹವಾದ ವಿಷಯವೇನೆಂದರೆ ಎರಡು ಬಾರಿ ಒಂದೇ ಭಾವನೆಯನ್ನು ಒಮ್ಮೆಯೂ ಕೇಳಲಿಲ್ಲ. ವರ್ಣಗಳ ವಿಷಯವನ್ನು ಪದೇ ಪದೇ ವಿವರಿಸುತ್ತಿದ್ದರು. ಆದರ್ಶಗಳನ್ನು ಪದೇ ಪದೇ ವಿಮರ್ಶಿಸುತ್ತಾ ಒತ್ತಿ ಹೇಳುತ್ತಿದ್ದರು. ಹಿಂದೆ ಈಗ ಮತ್ತು ಮುಂದಿನ ಕೆಲಸ ಇವನ್ನು ಎಂದಿಗೂ ನಿರ್ಲಕ್ಷಿಸಿದವರಲ್ಲ. ದುರ್ಬಲರು ದಿಕ್ಕಿಲ್ಲದವರ ಪರವಾಗಿ ಅವರು ಯಾವಾಗಲೂ ಮಾತನಾಡುತ್ತಿದ್ದರು. ನನ್ನ ಗುರುದೇವ ಬಂದರು, ಹೋದರು. ಆದರೆ ಅವರು ನಮಗಾಗಿ ಬಿಟ್ಟುಹೋದ ಅಸಾಧಾರಣ ಸ್ಮೃತಿ ನಿಧಿಯಲ್ಲಿ ಮಾನವಕೋಟಿಯ ಮೇಲಿನ ಪ್ರೇಮದಷ್ಟು ಮತ್ತಾವುದೂ ಇಲ್ಲ. 

 ನರಭಕ್ಷಣ ಕೆಲವು ಸಮಾಜದಲ್ಲಿ ಸ್ವಾಭಾವಿಕ ಎಂಬ ಕೆಲವು ಐರೋಪ್ಯರ ಅಭಿಪ್ರಾಯವನ್ನು ಕೇಳಿದಾಗ ಸ್ವಾಮೀಜಿಗೆ ಆದ ಕೋಪವನ್ನು ನಾನು ಮರೆಯುವಂತೆ ಇಲ್ಲ. “ಇದು ಸತ್ಯವಲ್ಲ. ಯಾರೂ ಮತ್ತೊಬ್ಬರ ಮಾಂಸಭಕ್ಷಣ ಮಾಡಲಿಲ್ಲ. ಕೆಲವು ವೇಳೆ ದ್ವೇಷದಿಂದಲೋ ಅಥವಾ ಕೆಲವು ಧರ್ಮಾಚಾರ ದೃಷ್ಟಿಯಿಂದಲೋ ಅವರು ತಿಂದಿರಬೇಕು. ನಿನಗೆ ಗೊತ್ತಾಗುವುದಿಲ್ಲವೆ? ಇದು ಸಮುದಾಯಜೀವಿಗಳ ಲಕ್ಷಣವಲ್ಲ. ಈ ನಡತೆ ಆ ಸಮಾಜಕ್ಕೆ ಕುಠಾರಪ್ರಾಯವಾಗುವುದು.” ಕ್ರಿಪೋಟ್‌ಕಿನ್ ಎಂಬುವನು ಬರೆದ ‘ಪರಸ್ಪರ ಸಹಾಯ’ ಎಂಬ ಪುಸ್ತಕ ಆಗಿನ್ನೂ ಬಂದಿರಲಿಲ್ಲ. ಸ್ವಾಮೀಜಿಯಲ್ಲಿದ್ದ ಮಾನವಪ್ರೇಮ ಮತ್ತು ಪ್ರತಿಯೊಬ್ಬರನ್ನೂ ಅವರವರ ದೃಷ್ಟಿಯಿಂದ ಅಳೆಯುವಂತಹ ಶಕ್ತಿಯಿಂದಲೆ ಅವರು ಇಷ್ಟು ಸ್ಪಷ್ಟವಾಗಿ ವಿಷಯವನ್ನು ನೋಡಬಲ್ಲವರಾಗಿದ್ದರು. 

 ಮತ್ತೊಮ್ಮೆ ಧಾರ್ಮಿಕಭಾವನೆ ಕುರಿತು ಮಾತನಾಡುವಾಗ ಹೀಗೆ ಹೇಳಿದರು. “ಕಾಮ ಮತ್ತು ಪ್ರಜೋತ್ಪತ್ತಿಯೇ ಹಲವು ಧರ್ಮಗಳ ಮೂಲದಲ್ಲಿರುವುದು. ಇದನ್ನೇ ಇಂಡಿಯಾ ದೇಶದಲ್ಲಿ ವೈಷ್ಣವರೆಂದು, ಪಾಶ್ಚಾತ್ಯರಲ್ಲಿ ಕ್ರೈಸ್ತರೆಂದು ಕರೆಯುವರು. ಕಾಳಿಯನ್ನು ಎಂದರೆ ಮೃತ್ಯುವನ್ನು ಆರಾಧಿಸಲು ಎಷ್ಟು ಜನಕ್ಕೆ ಧೈರ‍್ಯವಿದೆ? ರೌದ್ರವನ್ನು ಆಲಂಗಿಸುವ, ಅದರ ಉಗ್ರತೆ ಕಡಿಮೆಯಾಗಬೇಕೆಂದು ಕೇಳದೆ ಇರುವ. ದುಃಖವನ್ನು ದುಃಖಕ್ಕಾಗಿ ಸ್ವೀಕರಿಸೋಣ!” 

 ನದಿಯ ನೀರು ಸಮುದ್ರದ ನೀರನ್ನು ಸಂಧಿಸಿದಾಗ ಸಮುದ್ರವನ್ನು ‘ಕಾಳಪಾನಿ’ ಎನ್ನುವರು. ನದಿಯ ನೀರನ್ನೇತಕ್ಕೆ ಶುದ್ಧಜಲ ಎಂದು ಕರೆಯುತ್ತಾರೆಂಬುದು ಗೊತ್ತಾಯಿತು. ಹಿಂದೂಗಳಿಗೆ ಸಾಗರದ ಮೇಲಿರುವ ಗೌರವದಿಂದ ಅದನ್ನು ದಾಟಬಾರದು ಎಂದು ನಿಷೇಧಿಸಿ ಹಾಗೆ ದಾಟಿದವರನ್ನು ಬಹಳ ಶತಮಾನಗಳವರೆಗೆ ಬಹಿಷ್ಕರಿಸಿದರು. ಹಡಗು ಪ್ರಥಮಬಾರಿ ಸಮುದ್ರದ ಜಲಕ್ಕೆ ಪ್ರವೇಶಿಸಿದಾಗ ಸ್ವಾಮೀಜಿ “ನಮಃ ಶಿವಾಯ, ನಮಃ ಶಿವಾಯ, ತ್ಯಾಗಭೂಮಿಯಿಂದ ಭೋಗ ಭೂಮಿಗೆ ಹೋಗುತ್ತಿರುವೆ” ಎಂದರು. 

 ಯಾರು ಮಹಾಪುರುಷರಾಗಲು ಯತ್ನಿಸುವರೋ ಅವರು ವ್ಯಥೆಪಡಲೇ ಬೇಕಾಗುವುದು ಎಂದರು. ಕೆಲವರಿಗಂತೂ ಅವರ ಪ್ರತಿಯೊಂದು ಸುಖವೂ ಉರಿದು ಬೂದಿಯಾಗುವುದು. ಜೀವನವೆಲ್ಲ ಒಂದು ಹಂಸಗೀತೆ. ನೋಡು, ಕೆಳಗೆ ಬರುವ ವಾಕ್ಯಗಳನ್ನು ಎಂದಿಗೂ ಮರೆಯಬೇಡ ಎಂದರು. “ಸಿಂಹವನ್ನು ಕೆಣಕಿದಾಗ ಆರ್ಭಟಿಸುವುದು. ತಲೆಯಮೇಲೆ ಹೊಡೆದಾಗ ನಾಗರಹಾವು ಹೆಡೆ ಎತ್ತುವುದು. ಅದರಂತೆಯೇ ಹೃದಯದ ಅಂತರಾಳವನ್ನೆಲ್ಲ ಬಗೆಯುವಂತೆ ಮಾನವ ವ್ಯಥೆಪಟ್ಟಾಗ ಅವನ ಆತ್ಮ ಪ್ರಜ್ವಲಿಸುವುದು.” 

 ಒಮ್ಮೆ ಅಸಾಧಾರಣ ಸಹನೆಯಿಂದ ಕೇಳಿದ ಪ್ರಶ್ನೆಗೆ ಉತ್ತರವೀಯುವರು.\break ಮತ್ತೊಮ್ಮೆ ಚರಿತ್ರೆ ಮತ್ತು ಸಾಹಿತ್ಯದ ಊಹೆಗಳನ್ನು ಕುರಿತು ಮಾತನಾಡುವರು. ಹಿಂದೂದೇಶದ ಚರಿತ್ರೆಯನ್ನು ತಿಳಿದುಕೊಳ್ಳಬೇಕಾದರೆ ಬೌದ್ಧಕಾಲ ಅತಿ ಮುಖ್ಯವೆಂದು ಪದೇ ಪದೇ ಹೇಳುತ್ತಿದ್ದರು. 

 ಬೌದ್ಧ ಧರ್ಮದ ಮೂರು ಕಾಲವನ್ನು ಅವರು ಹೀಗೆ ವಿವರಿಸುತ್ತಿದ್ದರು: “ಐನೂರು ವರ್ಷಗಳು ಧರ್ಮ, ಐನೂರು ವರ್ಷಗಳು ವಿಗ್ರಹ, ಐನೂರು ವರ್ಷಗಳು ಆಚಾರ. ಭರತಖಂಡದಲ್ಲಿ ಎಂದಿಗೂ ಬೌದ್ಧಧರ್ಮವೆಂಬ ಬೇರೆ ಧರ್ಮವಾಗಲೀ ಅವರಿಗೆ ಸಂಬಂಧಪಟ್ಟ ದೇವಸ್ಥಾನ ಮತ್ತು ಪುರೋಹಿತರು ಇದ್ದರೆಂದು ನೀನು ಭಾವಿಸಬಾರದು.” ಒಂದು ಸಮಯದಲ್ಲಿ ಬುದ್ಧನ ಪ್ರಭಾವ ಬಹಳ ಹೆಚ್ಚಾಗಿತ್ತು. ಆಗ ದೇಶದಲ್ಲಿ ಸಂನ್ಯಾಸಿಗಳು ಹೆಚ್ಚಾದರು. ಕಾಶ್ಮೀರದಲ್ಲಿ ಪೂಜಿಸುತ್ತಿದ್ದ ಸರ್ಪಗಳೇ ಬೌದ್ಧ ಧರ್ಮದಲ್ಲಿ ಮಹಾ ಗುರುಗಳಾಗಿರುವರು ಎಂಬುದನ್ನು ವಿವರಿಸಿದರು. 

 ಸೋಮಬಳ್ಳಿಯ ವಿಷಯವಾಗಿ ಅವರು ಮಾತನಾಡತೊಡಗಿದರು. ಹಿಮಾಲಯದ ಕಾಲವಾದಮೇಲೆ ಒಂದು ಸಾವಿರ ವರ್ಷಗಳವರೆಗೆ ಈ ಬಳ್ಳಿಯನ್ನು ಪ್ರತಿಯೊಂದು ಹಳ್ಳಿಯಲ್ಲಿಯೂ ಒಬ್ಬ ದೊರೆಯನ್ನು ಸ್ವಾಗತಿಸುವಂತೆ ಬರಮಾಡಿಕೊಳ್ಳುತ್ತಿದ್ದರು. ಒಂದು ಗೊತ್ತಾದ ದಿನದಲ್ಲಿ ಅದನ್ನು ಊರಿನವರೆಲ್ಲ ಹೋಗಿ ಸಂಭ್ರಮದಿಂದ ತೆಗೆದುಕೊಂಡು ಬರುತ್ತಿದ್ದರು. ಈಗ ಆ ಬಳ್ಳಿಯನ್ನು ಕಂಡು ಹಿಡಿಯುವುದಕ್ಕೆ ಆಗದಾಗಿದೆ ಎಂದರು. 

 ಮತ್ತೊಂದು ದಿನ ಶೇರ್‌ಷಾ ವಿಚಾರ ಮಾತನಾಡುತ್ತಿದ್ದರು. ಹುಮಾಯೂನನ ಆಳ್ವಿಕೆಯ ಕಾಲದಲ್ಲಿ ಅವನು ಮಧ್ಯೆ ಮೂವತ್ತು ವರ್ಷಗಳು ಬರುತ್ತಾನೆ. ಸ್ವಾಮೀಜಿ ಎಷ್ಟು ಸಂತೋಷದಿಂದ ಈ ವಿಷಯವನ್ನು ಎತ್ತಿದರು ಎಂದು ನನಗೆ ಈಗಲೂ ಜ್ಞಾಪಕವಿದೆ. “ಒಮ್ಮೆ ಅವನು ಹುಡುಗನಾಗಿದ್ದಾಗ ಬಂಗಾಳದ ರಸ್ತೆಗಳಲ್ಲಿ ಅಲೆದಾಡುತ್ತಿದ್ದನು” ಎಂದರು. ಚಿತ್ತಗಾಂಗ್ ನಿಂದ ಪೇಷಾವರ್ ವರೆಗೆ ಇರುವ ಗ್ರಾಂಡ್ ಟ್ರಂಕ್ ರೋಡ್, ಅಂಚೆಯ ಇಲಾಖೆ, ಸರ್ಕಾರದ ಬ್ಯಾಂಕ್ ಇವುಗಳೆಲ್ಲ ಅವನ ಕೆಲಸ. ಅನಂತರ ಸ್ವಲ್ಪ ಹೊತ್ತು ಮೌನವಾಗಿದ್ದರು. ಅನಂತರ ಗುರುಗೀತದಿಂದ ಕೆಲವು ಶ್ಲೋಕಗಳನ್ನು ಉದಾಹರಿಸತೊಡಗಿದರು:

\begin{verse}
ಗುರುರ್ಬ್ರಹ್ಮಾ ಗುರುರ್ವಿಷ್ಣುಃ ಗುರುರ್ದೇವೋ ಮಹೇಶ್ವರಃ~।\\ ಗುರುರೇವ ಪರಂ ಬ್ರಹ್ಮ ತಸ್ಮೈ ಶ‍್ರೀಗುರವೇ ನಮಃ~॥\\ ಗುರುರಾದಿರನಾದಿಶ್ಚ ಗುರುಃಪರಮದೈವತಂ~।\\ ಗುರೋಃ ಪರತರಂ ನಾಸ್ತಿ ತಸ್ಮೈ ಶ‍್ರೀಗುರವೇ ನಮಃ~॥
\end{verse}

 ಅವರ ಮನಸ್ಸಿನಲ್ಲಿ ಮೂಡಿದ ಯಾವುದೋ ಭಾವನೆಗಳನ್ನು ಅವಲೋಕಿಸುತ್ತಿದ್ದರು. ಅದಕ್ಕಾಗಿ ಮೇಲಿನ ಶ್ಲೋಕವೆಂದು ಕಾಣುವುದು. ಒಂದೆರಡು ನಿಮಿಷಗಳಾದ ಮೇಲೆ ಪುನಃ ಹೀಗೆ ಹೇಳಿದರು: “ಹೌದು ಬುದ್ಧ ಹೇಳಿದ್ದು ನಿಜ. ಕರ್ಮದಲ್ಲಿರುವವರೆಗೆ ಕಾರ‍್ಯಕಾರಣಗಳು ಇದ್ದೇ ತೀರಬೇಕು. ನಾನೆಂಬ ಭಾವನೆ ಭ್ರಾಂತಿಯಲ್ಲದೇ ಬೇರಲ್ಲ.” ಮಾರನೇ ದಿನ ಅವರು ಕುರ್ಚಿಯಲ್ಲಿ ಕುಳಿತುಕೊಂಡು ತೂಕಡಿಸುತ್ತಿದ್ದರು ಎಂದು ಕಾಣುವುದು. ಆಗ ಇದ್ದಕ್ಕೆ ಇದ್ದಂತೆಯೇ ಹೀಗೆ ಹೇಳಿದರು: “ಒಂದು ಜನ್ಮದ ನೆನಪೇ ಕೋಟ್ಯಂತರ ವರ್ಷಗಳ ಸೆರೆಮನೆಯ ವಾಸದಂತೆ ಇದೆ. ಇದು ಸಾಲದೆ ಹಲವು ಜನ್ಮಗಳನ್ನು ಜ್ಞಾಪಿಸಿಕೊಳ್ಳಲು ಯತ್ನಿಸುವರು. ಈಗಿರುವ ಹೊರೆಯೇ ಸಾಕಾಗಿದೆ.” 

 ಮತ್ತೊಂದು ದಿನ ಮಧ್ಯಾಹ್ನದ ಊಟಕ್ಕೆ ಮುಂಚೆ ಅವರನ್ನು ಕಂಡಾಗ ಸ್ವಾಮೀಜಿ ಈ ವಿಷಯವನ್ನು ಎತ್ತಿ ಅದರಲ್ಲೇ ನಿರತರಾದರು: “ನಾನೀಗ ತಾನೆ ತುರೀಯಾನಂದರೊಡನೆ ಸಂಪ್ರದಾಯಬದ್ಧರು ಮತ್ತು ಪ್ರಗತಿಪರ ಉದಾರಬುದ್ಧಿಯ ಸ್ವಭಾವದವರು ಇವರನ್ನು ಕುರಿತು ಮಾತನಾಡುತ್ತಿದ್ದೆ” ಎಂದರು. ಮತ್ತೆ ಮುಂದುವರೆಸಿದರು:

 “ಸಂಪ್ರದಾಯಬದ್ಧನ ಗುರಿಯೆಲ್ಲ ಆಶ್ರಿತನಾಗುವುದು. ನಿಮ್ಮದು ಹೋರಾಟ. ಆದಕಾರಣ ನಾವು ಜೀವನವನ್ನು ಅನುಭವಿಸುವವರು, ನೀವಲ್ಲ. ನೀವು ಯಾವಾಗಲೂ ಉತ್ತಮ ಆದರ್ಶಕ್ಕಾಗಿ ಹೋರಾಡುತ್ತಿರುವಿರಿ. ಆದರ್ಶದ ಕೋಟಿಯಲ್ಲಿ ಒಂದು ಪಾಲಿನಷ್ಟು ಕಾರ್ಯಗತವಾಗುವುದರೊಳಗೆ ನಿಮಗೆ ಮೃತ್ಯುಕರೆ ಬರುವುದು.\break ಪಾಶ್ಚಾತ್ಯದ ಆದರ್ಶ ಹೋರಾಟ, ಪೌರಾತ್ಯನ ಆದರ್ಶ ಸಹಿಷ್ಣುತೆ. ಪೂರ್ಣಜೀವನ ಇವೆರಡರ ಸಾಮರಸ್ಯ. ಆದರೆ ಇದು ಎಂದಿಗೂ ಸಾಧ್ಯವಿಲ್ಲ.

 “ನಮ್ಮ ಧರ್ಮದಲ್ಲಿ ಮನುಷ್ಯನಿಗೆ ಏನೇನು ಬೇಕೋ ಅದೆಲ್ಲ ದೊರಕಲಾರದು ಎಂದು ಒಪ್ಪಿಕೊಳ್ಳುತ್ತೇವೆ. ಜೀವನದಲ್ಲಿ ನಾವು ಎಷ್ಟೋ ಆಸೆಗಳನ್ನು ನಿಗ್ರಹಿಸಬೇಕಾಗಿದೆ. ಇದೇನೋ ಭಯಾನಕ. ಆದರೂ ಈ ಆದರ್ಶದ ಒಳ್ಳೆಯ ಮತ್ತು ಕೆಟ್ಟ ವಿಷಯಗಳು ಚೆನ್ನಾಗಿ ವ್ಯಕ್ತವಾಗುತ್ತವೆ. ನಮ್ಮ ಪ್ರಗತಿಪರರು ಇರುವ ಲೋಪದೋಷಗಳನ್ನು ಮಾತ್ರ ನೋಡುತ್ತಾರೆ. ಅದನ್ನು ನಿರ್ಮೂಲ ಮಾಡಲು ಯತ್ನಿಸುವರು. ಆದರೆ ಅವರು ಅದರ ಬದಲು ತರುವುದು ಹಿಂದಿನಷ್ಟೇ ಹೀನವಾಗಿರುವುದು. ಹೊಸ ಭಾವನೆಗಳು ಸಮಾಜದಲ್ಲಿ ಹರಡಿ ಫಲಕಾರಿಯಾಗಬೇಕಾದರೆ ಹಳೆಯದರಷ್ಟೇ ಸಮಯ ತೆಗೆದುಕೊಳ್ಳುವುದು.” ಬದಲಾವಣೆಯಿಂದ ನಮ್ಮ ಇಚ್ಛಾಶಕ್ತಿ ಬದಲಾಗುವುದಿಲ್ಲ. ಅದರಿಂದ ದುರ್ಬಲವಾಗಿ ಅದಕ್ಕೆ ಅಡಿಯಾಳಾಗುವುದು. ಆದರೆ ನಾವು ಯಾವಾಗಲೂ ಹೊಸ ಹೊಸ ಭಾವನೆಗಳನ್ನು ಹೀರುತ್ತಿರಬೇಕು. ನಮ್ಮ ಇಚ್ಛಾಶಕ್ತಿ ಹೀರಿಕೊಳ್ಳುವುದರಿಂದ ಬಲವಾಗುವುದು. ನಮಗೆ ಗೊತ್ತಿರಲಿ ಇಲ್ಲದೆ ಇರಲಿ, ನಾವು ಪ್ರಪಂಚದಲ್ಲಿ ಮೆಚ್ಚುವುದು ಈ ಇಚ್ಛಾಶಕ್ತಿಯನ್ನು. ಪ್ರಪಂಚದ ದೃಷ್ಟಿಯಲ್ಲಿ ಸತಿ ಮಹತ್ವವಾದುದು. ಏಕೆಂದರೆ ಅಲ್ಲಿ ಅದ್ಭುತವಾದ ಇಚ್ಛಾಶಕ್ತಿಯ ಆವಿರ್ಭಾವವಿದೆ. 

 “ನಾನು ನಿರ್ಮೂಲಮಾಡಬೇಕಾಗಿರುವುದು ಸ್ವಾರ್ಥತೆಯನ್ನು. ನಾನು ಎಂದಾದರೂ ತಪ್ಪು ಮಾಡಿದ್ದರೆ ಅದಕ್ಕೆ ಕಾರಣ ಸ್ವಾರ್ಥ. ಎಲ್ಲಿ ಸ್ವಾರ್ಥ ಇಲ್ಲವೋ‌ ಅಲ್ಲಿ ನನ್ನ ತೀರ್ಪು ಯಾವಾಗಲೂ ಸರಿಯಾಗಿತ್ತು.

 “ಈ ಸ್ವಾರ್ಥ ಇಲ್ಲದೇ ಇದ್ದರೆ ಯಾವ ಧರ್ಮವೂ ಇರುತ್ತಿರಲಿಲ್ಲ. ಮನುಷ್ಯನಿಗೆ ಏನೂ ಬೇಡದೆ ಇದ್ದರೆ ಈ ಪೂಜೆ, ಪ್ರಾರ್ಥನೆ ಇವುಗಳನ್ನೆಲ್ಲ ಮಾಡುತ್ತಿದ್ದನೇನು? ಏತಕ್ಕೆ, ಅವನು ದೇವರನ್ನೇ ಚಿಂತಿಸುತ್ತಿರಲಿಲ್ಲ. ಯಾವಾಗಲಾದರೊಮ್ಮೆ ಸುಂದರ ದೃಶ್ಯವನ್ನು ನೋಡಿದಾಗ ಅದರ ಸೃಷ್ಟಿಕರ್ತನನ್ನು ಸ್ವಲ್ಪ ಹೊಗಳುತ್ತಿದ್ದನು. ಇದೊಂದೇ ನಮ್ಮ ಆದರ್ಶವಾಗಿರಬೇಕು. ನಾವು ಸ್ವಾರ್ಥದಿಂದ ಪಾರಾಗಲು ಸಾಧ್ಯವಾದರೆ ಎಲ್ಲವೂ ಸ್ತೋತ್ರವಾಗುವುದು, ಧನ್ಯವಾಗುವುದು.

 “ಹೋರಾಟ ಜೀವನದ ಚಿಹ್ನೆ ಎಂದು ತಿಳಿದುಕೊಂಡರೆ ಅದು ತಪ್ಪು. ಅದು ಹಾಗಲ್ಲ. ಹೀರಿಕೊಳ್ಳುವುದು ಜೀವನದ ಚಿಹ್ನೆ. ಹಿಂದೂಧರ್ಮ ಇತರರಿಂದ ಒಳ್ಳೆಯದನ್ನು ಕಲಿತುಕೊಳ್ಳುವುದರಲ್ಲಿ ಬಹಳ ಪ್ರಖ್ಯಾತವಾಗಿದೆ. ನಮ್ಮ ದೃಷ್ಟಿ ಹೋರಾಟದ ಕಡೆಗೆ ಇಲ್ಲವೇ ಇಲ್ಲ. ಹೌದು, ಕೆಲವು ವೇಳೆ ಸ್ವದೇಶ ರಕ್ಷಣೆಗಾಗಿ ವೈರಿಗಳನ್ನು ಸದೆಬಡಿದಿರುವೆವು. ಇದು ಧರ್ಮ. ಆದರೆ ಬರೀ ಹೋರಾಟಕ್ಕೆ ನಾವು ಎಂದಿಗೂ ಕಾಲು ಕೆರೆಯುತ್ತಿರಲಿಲ್ಲ. ಪ್ರತಿಯೊಬ್ಬರೂ ಇದನ್ನು ಕಲಿಯಬೇಕಾಗಿದೆ. ಈಗತಾನೆ ಹೊಸದಾಗಿ ಬಂದ ಜನಾಂಗಗಳು ಸ್ವಲ್ಪಕಾಲ ಮೆರೆದಾಡಲಿ. ಕೊನೆಗೆ ಇವುಗಳೆಲ್ಲ ಹಿಂದೂ ಧರ್ಮದ ಪಾಲಾಗುವುವು.” 

 ತಮ್ಮ ಧರ್ಮವನ್ನಾಗಲಿ ಮಾತೃಭೂಮಿಯನ್ನಾಗಲಿ ಎಂದಿಗೂ ಅವರು ಆಶ್ರಿತ ಸ್ಥಾನದಲ್ಲಿ ನೋಡುತ್ತಿರಲಿಲ್ಲ. ಕೆಲವುವೇಳೆ ಯಾವುದಾದರೂ ಕೆಲಸವನ್ನು ಕುರಿತು ಮಾತನಾಡುತ್ತಿದ್ದಾಗ ವ್ಯಂಗ್ಯವಾಗಿ “ಐರೋಪ್ಯ ಸ್ತ್ರೀ ಪುರುಷರೂ ಕೂಡ ಭರತಖಂಡದಲ್ಲಿ ಕೆಲಸ ಮಾಡಬೇಕಾದರೆ ಅವರು ಕಪ್ಪು ಮನುಷ್ಯರ ಕೆಳಗೆ ಕೆಲಸಮಾಡಬೇಕು. ಹೌದು, ಹೀಗೆಯೇ ಅವರು ಇರಬೇಕು” ಎನ್ನುತ್ತಿದ್ದರು. 

 ತಮ್ಮ ಜನಾಂಗ ಏನನ್ನು ಸಾಧಿಸಿದೆ ಎಂಬುದನ್ನು ಕುರಿತು ಆಲೋಚಿಸತೊಡಗಿದರು: “ಆಗಲಿ, ಮತ್ತಾವ ಜನಾಂಗವೂ ಸಾಧಿಸದ ಒಂದನ್ನು ನಾವು ಸಾಧಿಸಿರುವೆವು. ಇಡೀ ದೇಶದವರನ್ನು ಒಂದೆರಡು ಭಾವನೆಗಳನ್ನು ಒಪ್ಪಿಕೊಳ್ಳುವಂತೆ ಮಾಡಿರುವೆವು. ಉದಾಹರಣೆಗೆ, ದನದ ಮಾಂಸ ನಿಷೇಧ. ಯಾವ ಹಿಂದೂಗಳೂ ಅದನ್ನು ತಿನ್ನುವುದಿಲ್ಲ. ಇದು ಐರೋಪ್ಯರು ಬೆಕ್ಕಿನ ಮಾಂಸವನ್ನು ತಿನ್ನದೆ ಇರುವಂತೆ ಅಲ್ಲ. ಏಕೆಂದರೆ ಹಿಂದೆ ಒಂದು ಕಾಲದಲ್ಲಿ ದನದ ಮಾಂಸವನ್ನು ಇಡೀ ದೇಶವೇ ತಿನ್ನುತ್ತಿತ್ತು.” 

 ಸ್ವಾಮೀಜಿಗಳ ವಿರೋಧಿಗಳೊಬ್ಬರನ್ನು ಕುರಿತು ಅವರು, ತಮ್ಮ ಪಂಗಡವನ್ನು ದೇಶದ ಹಿತಕ್ಕಿಂತ ಮೇಲೆಣಿಸುವರು ಎಂದೆವು. ಸ್ವಾಮೀಜಿ ಸಂತೋಷದಿಂದ “ಹೌದು, ಏಷ್ಯಾ ಜನಾಂಗದ ರೀತಿಯೇ ಇದು. ತುಂಬಾ ಚೆನ್ನಾಗಿದೆ. ಆದರೆ ಆ ಮನುಷ್ಯನಿಗೆ ತಾಳ್ಮೆಯಾಗಲಿ, ಮುಂದಾಲೋಚನೆಯಾಗಲಿ ಇಲ್ಲ ಅಷ್ಟೇ.” ಎಂದರು. ಅನಂತರ ಕಾಳಿಯನ್ನು ಕುರಿತು ಹೀಗೆ ಹೇಳತೊಡಗಿದರು: “ನಾನು ನಿನ್ನ ಕೊರಳಿಗೆ ರುಂಡಮಾಲೆಯನ್ನು ಹಾಕಿ ಅಂಜಿಕೆಯಿಂದ ಹಿಂದೆ ಸರಿದು ತಾಯಿ ದಯಾಮಯಿ ಎಂದು\break ಗೋಗರಿಯುವವನಲ್ಲ. ಹೃದಯ ಸ್ಮಶಾನಸದೃಶವಾಗಬೇಕು. ಅಹಂಕಾರ ಸ್ವಾರ್ಥ ಆಸೆ ಆಕಾಂಕ್ಷೆಗಳೆಲ್ಲ ಚೂರ್ಣೀಭೂತವಾಗಬೇಕು. ಆಗ ಮಾತ್ರ ತಾಯಿ ಅಲ್ಲಿ ತಾಂಡವವಾಡುವಳು.” 

 “ನಾನು ರೌದ್ರವನ್ನು ರೌದ್ರಕ್ಕಾಗಿ ಪ್ರೀತಿಸುತ್ತೇನೆ” ಎಂದು ಹೇಳತೊಡಗಿದರು. “ನಿರಾಶೆಯನ್ನು ನಿರಾಶೆಗಾಗಿ ದುಃಖವನ್ನು ದುಃಖಕ್ಕಾಗಿ ಪ್ರೀತಿಸುತ್ತೇನೆ. ಹೋರಾಡಿ ಸತತ ಹೋರಾಡಿ, ಸೋಲುತ್ತಿದ್ದರೂ ಹೋರಾಡಿ. ಇದೇ ಆದರ್ಶ, ಇದೇ ಆದರ್ಶ.” 

 “ಎಲ್ಲಾ ಜೀವಿಗಳ ಮೊತ್ತವೇ ಈಶ್ವರ. ಸಮಷ್ಟಿಯ ಇಚ್ಛೆಯನ್ನು ಯಾರೂ ಧಿಕ್ಕರಿಸಲಾರರು. ಇದನ್ನೇ ನಾವು ಋತ ಎನ್ನುವುದು. ಇದನ್ನೇ ನಾವು ಶಿವ, ಶಕ್ತಿ, ಕಾಳಿ ಎನ್ನುವುದು.” 

 ಪ್ರಪಂಚದ ಸುಂದರತರ ದೃಶ್ಯಗಳ ಸಾನ್ನಿಧ್ಯದಲ್ಲಿ ಸ್ವಾಮೀಜಿಯವರ ವಾಣಿಯನ್ನು ಆಲಿಸಿದ ಮೇಲೆ ಅವು ಸುಂದರತಮವಾಗಿವೆ. 

 ನಾವು ಸಿಸಿಲಿಯನ್ನು ಸಮೀಪಿಸಿದಾಗ ಕತ್ತಲಾಗಿತ್ತು. ಪಶ್ಚಿಮ ದಿಗಂತಕ್ಕೆ ಅಭಿಮುಖವಾಗಿದ್ದ ಎಟ್ನ ಜ್ವಾಲಾಮುಖಿ ಸ್ವಲ್ಪ ಕೆರಳಿತ್ತು. ನಾವು ಮೆಸ್ಸೀನ ಜಲಸಂಧಿಯನ್ನು ಪ್ರವೇಶಿಸಿದ ಮೇಲೆ ಚಂದ್ರೋದಯವಾಯಿತು. ನಾನು ಸ್ವಾಮೀಜಿ ಜೊತೆಯಲ್ಲಿ ಡೆಕ್ಕಿನ ಮೇಲೆ ಸಂಚಾರ ಮಾಡುತ್ತಿದ್ದೆ. ಆಗ ಸ್ವಾಮೀಜಿ ಸೌಂದರ‍್ಯವೆಂಬುದು ಹೊರಗಿಲ್ಲ, ಅದು ಆಂತರ್ಯದ ವಸ್ತು ಎಂಬುದನ್ನು ವಿವರಿಸುತ್ತಿದ್ದರು. ಒಂದು ಕಡೆ ಇಟಲಿ ಕರಾವಳಿಯ ತೀರ, ಕೋಪದಿಂದ ಮುನಿಸಿಕೊಂಡಂತೆ ಕಪ್ಪಾಗಿತ್ತು. ಮತ್ತೆ ಒಂದು ಕಡೆ ಇದ್ದ ದ್ವೀಪ ಚಂದ್ರಮನ ಸ್ನಿಗ್ಧಕಾಂತಿಯಲ್ಲಿ ಬೆಳಗುತ್ತಿತ್ತು. “ಮೆಸ್ಸೀನ ನನಗೆ ವಂದಿಸಬೇಕು. ಏಕೆಂದರೆ ನಾನೇ ಇದಕ್ಕೆ ಸೌಂದರ್ಯವನ್ನು ಕೊಡುವವನು” ಎಂದರು ಸ್ವಾಮೀಜಿ. 

 ಅನಂತರ ಸ್ವಾಮೀಜಿ ತಮ್ಮ ಬಾಲ್ಯದಲ್ಲಿದ್ದ ಭಗವಂತನ್ನು ಸಾಕ್ಷಾತ್ಕಾರ ಮಾಡಿಕೊಳ್ಳಬೇಕು ಎಂಬ ಉತ್ಕಟಾಕಾಂಕ್ಷೆಯನ್ನು ಕುರಿತು ಹೇಳತೊಡಗಿದರು. ಸೂರ್ಯೋದಯಕ್ಕೆ ಮುಂಚೆ ಯಾವುದಾದರೂ ಶಾಸ್ತ್ರವನ್ನು ಪಾರಾಯಣ ಮಾಡಲು ಪ್ರಾರಂಭಿಸಿ ದಿನವೆಲ್ಲ ಅದರಲ್ಲಿ ಉದ್ಯುಕ್ತರಾಗಿರುತ್ತಿದ್ದರು. ತಪಸ್ಸು ಎಂದರೆ ಏನು ಎಂದು ಕೇಳಿದಾಗ ಅಂದಿನ ಕಾಲದ ಒಂದು ವ್ರತವನ್ನು ಹೇಳಿದರು. ನಾಲ್ಕು ಕಡೆಯೂ ಬೆಂಕಿ ಹಚ್ಚಿ ಕಾಯುತ್ತಿರುವ ಸೂರ‍್ಯನ ಕೆಳಗೆ ಕುಳಿತು ಧ್ಯಾನಮಾಡುವ ಪಂಚತಪವನ್ನು ವಿವರಿಸಿದರು. “ರೌದ್ರೋಪಾಸಕರಾಗಿ ಮೃತ್ಯುವನ್ನು ಆರಾಧಿಸಿ. ಉಳಿದವೆಲ್ಲ ವ್ಯರ್ಥ. ಉಳಿದ ಹೋರಾಟವೆಲ್ಲ ವ್ಯರ್ಥ; ಇದೇ ಪರಮ ನೀತಿ. ಆದರೂ‌ ಇದು ಹೇಡಿಯ ಮೃತ್ಯು ಪ್ರೇಮವಲ್ಲ, ದುರ್ಬಲನ ಹತ್ಯಾಕಾಂಕ್ಷಿಯ ಮೃತ್ಯುಪ್ರೇಮವಲ್ಲ. ಇದು ಧೀರಾಧಿ ಧೀರನ ಮೃತ್ಯು ಸ್ವಾಗತ. ಪ್ರಪಂಚದಲ್ಲಿ ಪ್ರತಿಯೊಂದು ವಸ್ತುವನ್ನೂ ಆದ್ಯಂತವಾಗಿ ವಿಮರ್ಶಿಸಿ, ಇನ್ನು ಬೇರೆ ವಿಧಿಯೇ ಇಲ್ಲ ಎಂದು ಅರಿತವನ ಆಲಿಂಗನ ಮೃತ್ಯು ಪ್ರೇಮ.”


\section*{ಮಹಾಪುರುಷರ ಸ್ಮೃತಿಚಿತ್ರಗಳು}

 ಸ್ವಾಮೀಜಿಯವರು ತಾವು ಕಂಡ ಮಹಾಪುರುಷರ ವಿಷಯವಾಗಿ ಒಂದು ದಿನ\break ಮಾತನಾಡಿದರು. ವಿಷಯ ಬಹುಶಃ ನಾಗಮಹಾಶಯರಿಂದ ಪ್ರಾರಂಭವಾಗಿರಬಹುದು. ಕೆಲವು ವಾರಗಳ ಹಿಂದೆ ಅವರು ಕಲ್ಕತ್ತೆಯಲ್ಲಿ ಸ್ವಾಮೀಜಿಯವರ ದರ್ಶನಕ್ಕಾಗಿ ಬಂದಿದ್ದರು. ನಾವು ಕಲ್ಕತ್ತೆ ಬಿಡುವುದಕ್ಕೆ ಒಂದೆರಡು ದಿನಗಳು ಮುಂಚೆ ನಾಗಮಹಾಶಯರು ಕಾಲವಾಗಿದ್ದರು. ಹಡಗು ಇನ್ನೂ ನದಿಯ ಮೇಲೆ ಹೋಗುತ್ತಿದ್ದಾಗ ಅವರಿಗೆ ನಾಗಮಹಾಶಯರ ನಿಧನವಾರ್ತೆ ತಿಳಿಯಿತು. ನಾಗಮಹಾಶಯ ಶ‍್ರೀರಾಮಕೃಷ್ಣರ ಒಂದು ಶ್ರೇಷ್ಠತಮಕೃತಿ ಎಂದು ಪದೇ ಪದೇ ಹೇಳುತ್ತಿದ್ದರು. ಭಕ್ತಿಯ ಆವಶ್ಯಕತೆಯನ್ನು ಕುರಿತು (ನಾಗಮಹಾಶಯರು) ಮಾತನಾಡುತ್ತಿದ್ದಾಗ\break ತನ್ನಂತಹ ಭಕ್ತಿಹೀನನಾದ ಅಯೋಗ್ಯನ ದೇಹಕ್ಕೆ ಊಟವನ್ನು ಏತಕ್ಕೆ ಕೊಡಬೇಕೆಂದು ಅದಕ್ಕೆ ಆಹಾರವನ್ನೇ ಕೆಲವು ದಿನಗಳು ಕೊಡಲಿಲ್ಲ. ಒಂದು ದಿನ ಅವರ ಮನೆಗೆ ಅತಿಥಿ ಬಂದಾಗ ಅಡಿಗೆ ಮಾಡಲು ಸೌದೆ ಇಲ್ಲದಿರುವಾಗ ಅವರು ತಮ್ಮ ಮನೆಯ ಛಾವಣಿಯನ್ನೇ ಹೇಗೆ ಮುರಿದು ಒಲೆಗೆ ಹಾಕಿದರು ಎಂಬುದನ್ನು ಕುರಿತು ಹೇಳಿದರು. 

 ಶ‍್ರೀರಾಮಕೃಷ್ಣ ಪರಮಹಂಸರು ಸ್ಪರ್ಶಿಸಿದ ಯುವಕನ ಕಥೆಯನ್ನು ಅನಂತರ ಹೇಳಿದರೆಂದು ಕಾಣುವುದು. ಆ ಹುಡುಗ ಅನಂತರ “ನನ್ನ ಪ್ರಿಯತಮ, ನನ್ನ ಪ್ರಿಯತಮ” ಇದಲ್ಲದೇ ಮತ್ತೇನನ್ನೂ ಮಾತನಾಡಲಿಲ್ಲ. ಆತ ಹತ್ತು ವರುಷ ಅನಂತರ ಬದುಕಿದ್ದ. ಆದರೆ ಮತ್ತೇನನ್ನೂ ಅವನು ಮಾತನಾಡಲಿಲ್ಲ. 

 ಶ‍್ರೀರಾಮಕೃಷ್ಣರ ಕಾಲದಲ್ಲಿ ಬಂದು ಹೋದವರಲ್ಲಿ ಇವರು ಸ್ಪರ್ಶಿಸಿದಾಗ ಸಮಾಧಿ ಪಡೆದ ಅನೇಕರ ಹೆಸರುಗಳು ಶ‍್ರೀರಾಮಕೃಷ್ಣರ ಸಂಘದ ಸಂನ್ಯಾಸಿಗಳಿಗೆ ಗೊತ್ತಿದೆ. ಅನೇಕ ವೇಳೆ ಈ ಘಟನೆಯಲ್ಲದೆ ಅವರ ಪೂರ್ವಾಪರ ವಿಚಾರಗಳಾವುದೂ ಗೊತ್ತಿಲ್ಲ. ಇದು ಒಬ್ಬ ಸ್ತ್ರೀಯ ವಿಷಯದಲ್ಲಿಂದು ಪ್ರಮುಖವಾಗಿದೆ. ಒಂದು ದಿನ ಒಬ್ಬಾಕೆ ಗಾಡಿಯಲ್ಲಿ ದಕ್ಷಿಣೇಶ್ವರಕ್ಕೆ ಬಂದಳು. ಶ‍್ರೀರಾಮಕೃಷ್ಣರು ಇವಳನ್ನು ನೋಡಿದೊಡನೆಯೆ ಇವಳು ದೈವೀಶಕ್ತಿಯ ಅಂಶವೆಂದು ಗ್ರಹಿಸಿದರು. ಈ ಸ್ತ್ರೀಯನ್ನು ಜಗನ್ಮಾತೆಯಂತೆ ನೋಡಿ ಅವಳಿಗೆ ನಮಸ್ಕರಿಸಿ ಪಾದಗಳಿಗೆ ಪುಷ್ಪವನ್ನು ಅರ್ಪಿಸಿ ಧೂಪವನ್ನು ನಿವಾಳಿಸಿದರು. ಆಕೆ ತಕ್ಷಣ ಗಾಢ ಸಮಾಧಿಸ್ಥಳಾದಳು. ಇದರಲ್ಲಿ ಆಶ್ಚರ್ಯವೇ ಇರಲಿಲ್ಲ. ಆಕೆಯನ್ನು ಪ್ರಕೃತಿಸ್ಥಳನ್ನಾಗಿ ಮಾಡಲು ಆಗಲೇ ಇಲ್ಲ. ಆಕೆಗೆ ಪ್ರಪಂಚದ ಮೇಲೆ ಪ್ರಜ್ಞೆ ಬಂದಾಗ ಅವಳ ಮುಖವೆಲ್ಲ ಅಮಲೇರಿದವಳಂತೆ ಭಾವಾವಿಷ್ಟಳಾಗಿ ಹೋಗಿದ್ದಳು. ಆಕೆಗೆ ಬಾಹ್ಯಜ್ಞಾನ ಬಂದಾಗ ಸುತ್ತಲಿದ್ದವರಿಗೆ ಸ್ವಲ್ಪ ಸಮಾಧಾನವಾಯಿತು. ಏಕೆಂದರೆ ಆಕೆಗೆ ಬೇಗ ಪ್ರಜ್ಞೆ ಬರದೇ ಇದ್ದರೆ ಇವರ ಮನೆಯವರಿಗೆ ಸ್ತ್ರೀಯ ವಿಚಾರ ಗೊತ್ತಾಗದೆ ಎಲ್ಲಿ ವ್ಯಸ್ತರಾಗುವರೋ ಎಂದು ಸುತ್ತಲಿದ್ದವರು ಭಾವಿಸುತ್ತಿದ್ದರು. ಆಕೆಗೆ ಪ್ರಜ್ಞೆ ಬಂದುದನ್ನು ನೋಡಿ ಎಲ್ಲರೂ ನಿಶ್ಚಿಂತರಾಗಿ ಆಕೆ ಹಿಂದಿರುಗಿ ಹೋಗುವುದಕ್ಕೆ ಸಹಾಯ ಮಾಡಿದರು. ಯಾರಿಗೂ ಅವಳ ಹೆಸರು, ಆಕೆ ಎಲ್ಲಿಂದ ಬಂದಳು ಎಂಬುದನ್ನು ಕೇಳಲು ಜ್ಞಾಪಕವೇ ಬರಲಿಲ್ಲ. ಈಕೆ ಶ‍್ರೀರಾಮಕೃಷ್ಣರು ಆರಾಧಿಸಿದ ಒಂದು ಸತೀತ್ವ ಮತ್ತು ಮಾತೃತ್ವದ ಆದರ್ಶ. ಈ ಘಟನೆ ಸಂಘದ ಸ್ಮೃತಿಯಲ್ಲಿ ಚಿರಮುದ್ರಿತವಾಗಿದೆ. ಶ‍್ರೀರಾಮಕೃಷ್ಣರೇ ಈಕೆಯನ್ನು ದೈವೀಶಕ್ತಿಯ ಅಂಶ ಎಂದು ಹೇಳಲಿಲ್ಲವೆ! 

 ನನಗೆ ಆಧ್ಯಾತ್ಮಿಕ ವಿಷಯಗಳು ಅಷ್ಟು ಪರಿಚಯವಿಲ್ಲದಾಗ ನನ್ನ ಮನಸ್ಸು ಪರಮಹಂಸರ ಪ್ರಭಾವಕ್ಕೆ ಸಿಕ್ಕಿ ಮೇಲಿನಂತಹ ಅಪರೂಪ ನಿದರ್ಶನಗಳನ್ನು ಯೋಚಿಸತೊಡಗಿತು. ಇವು ದೂರದಲ್ಲಿ ತಮ್ಮದೇ ಪಥದಲ್ಲಿ ಸಂಚರಿಸುತ್ತಿರುವ ಬೆಳಗುವ ತಾರೆಗಳಂತೆ. ಅವು ನಮ್ಮ ವಾತಾವರಣಕ್ಕೆ ಪುನಃ ಬರುವಂತೆಯೇ ಇಲ್ಲ. ಇಂತಹ ಜೀವನಗಳಲ್ಲಿ ಕೂಡ ಹಲವು ವರ್ಷಗಳ ಹಿಂದೆ ಅವರು ಪಡೆದ ಅನುಭವವನ್ನು ಮರೆತು ಬಿಡಲು ಸಾಧ್ಯವೆ? ಈಗ ನಾವು ಅವರ ವಿಷಯವನ್ನು ಕೇಳುವಂತೆ ಅವರ ಜೀವನದಲ್ಲಿ ಒಬ್ಬ ಮಹಾಪುರುಷನ ಸಾನ್ನಿಧ್ಯ ಮತ್ತು ಸ್ಪರ್ಶ ಯಾವುದೋ ಅತಿ ಹಿಂದಿನಕಾಲದ ಸ್ವಪ್ನದ ಒಂದು ಘಟನೆಯಂತೆ ಆಗಿರಬಹುದೆ? ನಾನು ಇಂತಹ ಅನುಭವಗಳಿಂದ ಆಗುವ ಪ್ರಯೋಜನವನ್ನೆಲ್ಲಾ ವಿಮರ್ಶೆ ಮಾಡಬೇಕೆಂದು ಇದ್ದೆ. ಆದರೆ ಇಡೀ ಹಿಂದೂ ಜನಾಂಗ ಇಂತಹ ಅನುಭವಕ್ಕೆ ಅದ್ಭುತವಾದ ಬೆಲೆಯನ್ನು ಇತ್ತಿದೆ ಎಂಬುದು ಆಗ ನನಗೆ ಅರಿವಾಗಲಿಲ್ಲ. ಸಾಮೀಜಿಗೆ ನನ್ನ ಸಮಸ್ಯೆ ಸರಿಯಾಗಿ ಗೊತ್ತಾಗಲಿಲ್ಲ. ಸ್ವಾಮೀಜಿ “ಶ‍್ರೀರಾಮಕೃಷ್ಣರು ಮತ್ತೊಂದು ಜೀವವನ್ನು ಮುಟ್ಟುವುದೇನೂ ಒಂದು ಹಾಸ್ಯ ಎಂದುಕೊಂಡೆಯಾ? ಅವರಿಗೆ ಕ್ಷಣಕಾಲ ಇಂತಹ ಸಂಪರ್ಕ ಸಿಕ್ಕಿದರೂ ಶ‍್ರೀರಾಮಕೃಷ್ಣರು ಅಂತಹವರನ್ನು ಹೊಸಮನುಷ್ಯರನ್ನಾಗಿ ಮಾಡಿಬಿಡುತ್ತಿದ್ದರು” ಎಂದರು. ಅನಂತರ ಹಲವು ಶಿಷ್ಯರ ಕಥೆಗಳನ್ನು ಒಂದಾದ ಮೇಲೊಂದು ಹೇಳುತ್ತಿದ್ದರು. ಒಬ್ಬ ಶಿಷ್ಯ ಪದೇ ಪದೇ ಬರುತ್ತಿದ್ದ. ಶ‍್ರೀರಾಮಕೃಷ್ಣರು ಅವನನ್ನು ನೋಡಿ “ಹೋಗು ಈಗ ಸ್ವಲ್ಪ ದುಡ್ಡು ಮಾಡು. ಅನಂತರ ಬಾ” ಎಂದು ಕಳುಹಿಸಿಬಿಟ್ಟರು. ಆತ ಈಗ ಹಣಗಳಿಸುತ್ತಿರುವನು. ಆದರೂ ಹಿಂದೆ ಶ‍್ರೀರಾಮಕೃಷ್ಣರ ಮೇಲಿದ್ದ ಪ್ರೇಮ ಕ್ರಮೇಣ ಜಾಗೃತವಾಗುತ್ತಿದೆ. ಈ ಶಿಷ್ಯನ ಅಥವಾ ಇತರರ ಲೋಪದೋಷಗಳನ್ನು ಕುರಿತು ಸ್ವಾಮೀಜಿ ಹೇಳಲಿಲ್ಲ. ಸ್ವಾಮೀಜಿ ನುಡಿಯನ್ನು ಆಲಿಸುತ್ತಿದ್ದರೆ ಪ್ರತಿಯೊಂದು ಜೀವ ಹೋರಾಡುತ್ತಿರುವಾಗ ಅವರಲ್ಲಿರುವ ಔದಾರ್ಯ ನಮ್ಮೆಡೆಗೆ ತಾಕುವುದು. ಪ್ರತಿಯೊಬ್ಬರೂ ಏತಕ್ಕೆ ಸಂನ್ಯಾಸಿಯಾಗಬೇಕು? ಅವನ ಕರ್ಮ ಕ್ಷಯವಾಗುವುದಕ್ಕೆ ಮುಂಚೆ ಇದು ಹೇಗೆ ಸಾಧ್ಯ? ಆದರೆ ಕೊನೆಗೆ ಇದರಲ್ಲಿ ಯಾವ ಸಂಶಯವೂ ಇಲ್ಲ; ಕೊನೆಗೆ ಇವರೆಲ್ಲ ಅವರಿಗೇ ಸೇರಿದವರು. 

\vskip 1pt

 ಇದರಂತೆಯೇ ಸ್ವಾಮೀಜಿ ತಮ್ಮ ಮನಸ್ಸಿನಲ್ಲಿ ಮೂಡಿದ ಇತರ ಮಹಾತ್ಮರ ಚಿತ್ರವನ್ನು ಹೃತ್ಪೂರ್ವಕ ಬಣ್ಣಿಸುತ್ತಿದ್ದರು. ಕೇಳಿದವರಿಗೆ ಈತನನ್ನು ಬಿಟ್ಟರೆ ಇನ್ನು ಯಾರೂ ಇಷ್ಟು ದೊಡ್ಡ ಮಹಾತ್ಮರಿಲ್ಲ ಎನ್ನುವಂತಿತ್ತು. ಪವಾಹಾರಿಬಾಬರ ವಿಷಯವಾಗಿ ಅವರು ನಮಗೆ ಎಲ್ಲಾ ವಿಷಯಗಳನ್ನು ಹೇಳಿದರು. ಅವರನ್ನು ಕುರಿತು ಬೇರೆ ಬೇರೆ ಪ್ರಶ್ನೆ \break ಕೇಳುವುದು ಅನುಚಿತ ಎನ್ನುವಂತೆ ಕಾಣುತ್ತಿತ್ತು. ಅವರ ಮರಣಕಾಲದಲ್ಲಿ ಸ್ವಾಮೀಜಿ ಪವಾಹಾರಿ ಬಾಬರನ್ನು ಶ‍್ರೀರಾಮಕೃಷ್ಣರಿಗೆ ಮಾತ್ರ ಎರಡನೆಯವರೆಂದು ಭಾವಿಸಿದ್ದರು. ಅವರು ತಮ್ಮ ಮೇಲೆ ಇಟ್ಟ ಪ್ರೇಮಕ್ಕೆ ತಾವೆಂದೆಂದಿಗೂ ಕೃತಜ್ಞರು ಎಂದು ಭಾವಿಸಿದ್ದರು. 

\vskip 1pt

 ಅನಂತರ ಒಂದು ಗಂಟೆಯವರೆಗೆ ಸ್ವಾಮೀಜಿ ತಾವು ಕಂಡ ಇತರ ಮಹಾತ್ಮರ ವಿಷಯದ ಮೇಲೆ ಹೇಳಲು ಪ್ರಾರಂಭಿಸಿದರು. ತ್ರೈಲಿಂಗಸ್ವಾಮಿಗಳನ್ನು ಅವರು ಬಹಳ ವೃದ್ಧಾಪ್ಯದಲ್ಲಿ ಕಂಡಿದ್ದರು. ಆಗ ಅವರಿಗೆ ನೂರು ವರ್ಷಗಳ ಮೇಲೆ ವಯಸ್ಸು ಆಗಿತ್ತು. ಯಾವಾಗಲೂ ಅವರು ಮೌನವಾಗಿಯೇ ಇರುತ್ತಿದ್ದರು. ಅವರು ಕಾಶಿಯಲ್ಲಿ ಒಂದು ಶಿವದೇವಾಲಯದಲ್ಲಿ ಶಿವಲಿಂಗದ ಮೇಲೆಯೇ ಕಾಲುಚಾಚಿಕೊಂಡು ಮಲಗಿದ್ದರು. ನೋಡುವುದಕ್ಕೆ ಹುಚ್ಚರಂತೆಯೇ ಇರುತ್ತಿದ್ದರು. ಜನರು ಬೇಕಾದರೆ ಏನಾದರೂ ಪ್ರಶ್ನೆಯನ್ನು ಚೀಟಿಯಲ್ಲಿ ಬರೆದುಕೊಡಬಹುದಾಗಿತ್ತು. ಮನಸ್ಸು ಬಂದರೆ ಅವುಗಳಲ್ಲಿ ಯಾವುದಾದರೂ ಒಂದಕ್ಕೆ ಸಂಸ್ಕೃತದಲ್ಲಿ ಉತ್ತರ ಬರೆದು ಕೊಡುತ್ತಿದ್ದರು. ಇತ್ತೀಚೆಗೆ ಅವರು ಕಾಲವಾಗಿದ್ದರು. 

 ರಘುನಾಥದಾಸ್ ಎಂಬ ಮತ್ತೊಬ್ಬ ಸಾಧು; ಸ್ವಾಮೀಜಿ ಅವರ ಆಶ್ರಮಕ್ಕೆ ಹೋಗುವುದಕ್ಕೆ ಎರಡು ತಿಂಗಳ ಮುಂಚೆ ಕಾಲವಾಗಿದ್ದರು. ಅವರು ಮೊದಲು ಇಂಗ್ಲೀಷರ ಕೆಳಗೆ ಸೈನ್ಯದಲ್ಲಿದ್ದರು. ಗಡಿ ರಕ್ಷಣೆಯಲ್ಲಿ ಪಹರೆಯವರಾಗಿದ್ದಾಗ ಒಳ್ಳೆಯ ಕಾರ್ಯತತ್ಪರಾಗಿದ್ದರು. ಮೇಲಿನ ಅಧಿಕಾರಿಗಳಿಗೆಲ್ಲ ಬೇಕಾಗಿದ್ದವರಾಗಿದ್ದರು. ಒಂದು ರಾತ್ರಿ ರಾಮಭಜನೆ ಕೇಳಿದರು. ತಮ್ಮ ಕಾರ್ಯದಲ್ಲಿ ಎಂದಿನಂತೆ ಕಾರ್ಯತತ್ಪರರಾಗಲು ಯತ್ನಿಸಿದರು. ಆದರೆ “ಜಯಬೋಲೋ ರಾಮಚಂದ್ರಕಿ ಜೈ” ಎಂಬ ನಾಮ ಅವರನ್ನು ಹುಚ್ಚರನ್ನಾಗಿ ಮಾಡಿತು. ಅವರು ತಮ್ಮ ಆಯುಧ ಡ್ರೆಸ್ ಎಲ್ಲವನ್ನೂ ಆಚೆಗೆ ಎಸೆದು ಭಜನೆಯಲ್ಲಿ ಸೇರಿದರು. 

 ಹೀಗೆಯೇ ಕೆಲವು ದಿನಗಳು ಆಯಿತು. ಈ ಸಮಾಚಾರ ಮೇಲಿನ ಅಧಿಕಾರಿಗೆ ಗೊತ್ತಾಯಿತು. ಅವರು ರಘುನಾಥರನ್ನು ಕರೆಸಿ “ ಈ ಸಮಾಚಾರ ನಿಜವೆ? ಇದಕ್ಕೆ ಶಿಕ್ಷೆ ಏನು ಗೊತ್ತೆ” ಎಂದು ಕೇಳಿದರು. ಹೌದು ಗೊತ್ತಿದೆ, ಇದಕ್ಕೆ ಗುಂಡಿನಿಂದ ಹೊಡೆಯುವುದೇ ಶಿಕ್ಷೆ, ಎಂದರು. ಅಧಿಕಾರಿ “ ಈ ಸಲ ಹೊರಡು, ನಾನು ಇದನ್ನು ಯಾರಿಗೂ ಹೇಳುವುದಿಲ್ಲ; ಈ ಸಲ ನಿನ್ನನ್ನು ಕ್ಷಮಿಸುತ್ತೇನೆ. ಆದರೆ ನೀನು ಹೀಗೆಯೇ ಪುನಃ ಮಾಡಿದರೆ ಶಿಕ್ಷೆಯನ್ನು ಅನುಭವಿಸಬೇಕಾಗುವುದು” ಎಂದು ಹೇಳಿ ಕಳುಹಿಸಿದರು. 

 ಅಂದಿನ ರಾತ್ರಿಯೂ ಪಹರೆಯವನು ಭಜನೆ ಪಂಗಡದ ರಾಮನಾಮವನ್ನು ಕೇಳಿದನು. ಹೋಗಕೂಡದೆಂದು ಎಷ್ಟೇ ಪ್ರಯತ್ನಪಟ್ಟರೂ ಸಾಧ್ಯವಾಗಲಿಲ್ಲ. ಕೊನೆಗೆ ಎಲ್ಲವನ್ನೂ ಮರೆತು ಭಜನೆಯಲ್ಲಿ ಬೆಳಗಿನ ತನಕ ಸೇರಿದನು. 

 ಮೇಲಿನ ಅಧಿಕಾರಿಗೆ ರಘುನಾಥದಾಸರ ಮೇಲೆ ಬಹಳ ನಂಬಿಕೆ ಇತ್ತು. ಅವರ ಮೇಲೆ ಯಾರು ಏನು ಹೇಳಲಿ ಅವರು ಅದನ್ನು ಕೇಳುತ್ತಿರಲಿಲ್ಲ. ಆದಕಾರಣ ತಾನೇ ಪರೀಕ್ಷಿಸುವುದಕ್ಕಾಗಿ ಅಧಿಕಾರಿ ರಾತ್ರಿ ಗಡಿಯ ಹತ್ತಿರ ಹೋದರು. ಆಗ ರಘುನಾಥದಾಸ್ ಅಲ್ಲಿಯೇ ಇದ್ದರು. ಮೂರು ವೇಳೆ ಅವರೊಡನೆ ಮಾತನಾಡಿದರು. ಅಧಿಕಾರಿ ಇವರು ಇಲ್ಲೇ ಇರುವರೆಂದು ಧೈರ‍್ಯವಾಗಿ ಮಲಗಲು ಹಿಂತಿರುಗಿದರು. 

 ಮಾರನೆ ದಿನ ಬೆಳಿಗ್ಗೆ ರಘುನಾಥದಾಸ್ ಹಾಜರಾಗಿ ತಮ್ಮ ಆಯುಧವನ್ನು ಅಧಿಕಾರಿಗೆ ಒಪ್ಪಿಸಿದರು. ಆದರೆ ಅಧಿಕಾರಿ ಅದಕ್ಕೆ ಒಡಂಬಡಲಿಲ್ಲ. ಏಕೆಂದರೆ ಹಿಂದಿನ ದಿನ ರಾತ್ರಿ ತಾವೆ ಅವನನ್ನು ನೋಡಿದ್ದರು. ರಘುನಾಥದಾಸರಿಗೆ ಆಶ್ಚರ‍್ಯವಾಯಿತು. ಹೇಗೋ‌ ತನ್ನನ್ನು ಕೆಲಸದಿಂದ ವಜಾ ಮಾಡಿ ಎಂದು ಬೇಡಿಕೊಂಡರು. ಶ‍್ರೀರಾಮನು ತನ್ನ ದಾಸನಿಗೆ ಇದನ್ನು ಮಾಡಿದನು. ಆದಕಾರಣ ಇಂದಿನಿಂದ ರಾಮನನ್ನಲ್ಲದೆ ಮತ್ತಾರ ಸೇವೆಯನ್ನೂ ನಾನು ಮಾಡಲಾರೆ ಎಂದರು. 

 ಇವರು ಅನಂತರ ಸರಸ್ವತಿ ನದೀ ತೀರದಲ್ಲಿ ಬೈರಾಗಿಯಾದರು. ಜನ ಇವರನ್ನು ಪೆದ್ದ ಎನ್ನುತ್ತಿದ್ದರು. ಆದರೆ ಅವರ ಶಕ್ತಿ ಗೊತ್ತಿತ್ತು ಎಂದರು ಸಾಮೀಜಿ. ಇವರು ಪ್ರತಿದಿನ ಸಹಸ್ರಾರು ಜನರಿಗೆ ಊಟವಿಡುತ್ತಿದ್ದರು. ಇವರಿಗೆ ದವಸಕೊಟ್ಟ ವರ್ತಕ ಕೆಲವು ದಿನಗಳಾದ ಮೇಲೆ ದುಡ್ಡಿಗೆ ಬರುವನು. ಆಗ ರಘುನಾಥದಾಸ್ “ಏನು ಒಂದು ಸಾವಿರ ರೂಪಾಯಿಯೆ? ನೋಡೋಣ ನನಗೆ ದುಡ್ಡು ಬಂದು ಒಂದು ತಿಂಗಳು ಆಗುತ್ತಾ ಬಂತು. ಬಹುಶಃ ನಾಳೆ ದುಡ್ಡು ಬಂದೀತು” ಎಂದು ಹೇಳಿ ಕಳುಹಿಸುತ್ತಿದ್ದರು. ಮಾರನೆ ದಿನ ದುಡ್ಡು ಯಾವಾಗಲೂ ಬರುತ್ತಿತ್ತು! 

\vskip 1pt

 ಯಾರೊ ರಘುನಾಥದಾಸರನ್ನು ರಾಮನಾಮ ಪಾರ್ಟಿಯ ಕಥೆ ನಿಜವೆ ಎಂದು ಕೇಳಿದರು. ಇಂತಹ ವಿಷಯವನ್ನು ತಿಳಿದುಕೊಂಡು ಏನು ಪ್ರಯೋಜನ ಎಂದರು ಅವರು. ನಾನು ಕೇವಲ ಕುತೂಹಲದಿಂದ ಕೇಳುತ್ತಿಲ್ಲ; ಇಂತಹ ಘಟನೆ ಸಾಧ್ಯವೆ ಎನ್ನುವುದಕ್ಕೆ ನಿಮ್ಮನ್ನು ಕೇಳುತ್ತಿರುವೆನು ಎಂದಾಗ, “ಭಗವಂತನಿಗೆ ಯಾವುದೂ ಅಸಾಧ್ಯವಲ್ಲ” ಎನ್ನುತ್ತಿದ್ದರು ರಘುನಾಥದಾಸ. 

\vskip 1pt

 ನಾನು ಹೃಷೀಕೇಶದಲ್ಲಿ ಹಲವು ಮಹಾತ್ಮರನ್ನು ನೋಡಿರುವೆನು ಎಂದು ಸ್ವಾಮೀಜಿ ಪ್ರಾರಂಭ ಮಾಡಿದರು: “ನನಗೆ ಒಬ್ಬ ಹುಚ್ಚನಂತೆ ಇದ್ದ ಮನುಷ್ಯನ ಪ್ರಸಂಗ ಜ್ಞಾಪಕವಿದೆ. ಆತ ಒಂದು ಬೀದಿಯಲ್ಲಿ ಬೆತ್ತಲೆ ಬರುತ್ತಿದ್ದ. ಹುಡುಗರು ಅವನನ್ನು ಬೆನ್ನಟ್ಟಿಕೊಂಡು ಬರುತ್ತ ಕಲ್ಲಿನಿಂದ ಹೊಡೆಯುತ್ತಿದ್ದರು. ಆತನ ಮುಖ ಮತ್ತು ಕತ್ತಿನಿಂದ ರಕ್ತ ಸುರಿಯುತ್ತಿದ್ದರೂ ಆತ ನಲಿದಾಡುತ್ತಿದ್ದ. ನಾನು ಅವನನ್ನು ಕರೆದುಕೊಂಡು ಹೋಗಿ ಗಾಯವನ್ನು ತೊಳೆದು ಬೂದಿಯನ್ನು ಹಾಕಿ ಕಟ್ಟಿದೆ. ಆಗ ಅವನು, “ಹುಡುಗರು ಮತ್ತು ನಾವು ಕಲ್ಲೆಸೆವ ಆಟವಾಡುತ್ತಿದೆವು” ಎಂದು ನಗುತ್ತ ಹೇಳುತ್ತಿದ್ದನು. “ತಂದೆ ಹೀಗೆ ಆಟವಾಡುವನು” ಎಂದು ಆತ ಹೇಳಿದ. 

\vskip 1pt

 “ಜನರಿಂದ ಪಾರಾಗುವುದಕ್ಕೆ ಅನೇಕ ಜನ ಹೀಗೆ ಗೋಪ್ಯವಾಗಿರುವರು. ಜನರ ಕಾಟವನ್ನು ಅವರು ಸಹಿಸಲಾರರು. ಒಬ್ಬ ತಾನು ವಾಸಿಸುತ್ತಿದ್ದ ಗುಹೆಯ ಹೊರಗೆ ಮನುಷ್ಯನ ಮೂಳೆಯನ್ನು ಎಸೆದಿದ್ದ. ಇದರಿಂದ ಆತ ನರಭಕ್ಷಕ ಎಂದು ಜನ ಅಂಜಿ ಇವನ ಹತ್ತಿರ ಬರದೆ ಇರಲಿ ಎಂದು. ಮತ್ತೊಬ್ಬ ಜನರ ಮೇಲೆ ಕಲ್ಲೆಸೆಯುತ್ತಿದ್ದ: ಹೀಗೆಯೆ. 

\vskip 1pt

 “ಕೆಲವು ವೇಳೆ ಸತ್ಯ ಇದ್ದಕ್ಕೆ ಇದ್ದಂತೆಯೆ ಅವರಿಗೆ ಮಿಂಚಿನಂತೆ ಹೊಳೆಯುವುದು. ಒಬ್ಬ ಹುಡುಗ ಸ್ವಾಮಿ ಅಭೇದಾನಂದರ ಹತ್ತಿರ ಉಪನಿಷತ್ತನ್ನು ಓದಲು ಬರುತ್ತಿದ್ದ. ಒಂದು ದಿನ ಈ ಹುಡುಗ ‘ಸ್ವಾಮೀಜಿ ಇದೆಲ್ಲ (ಉಪನಿಷತ್ತಿನಲ್ಲಿರುವುದೆಲ್ಲ) ನಿಜವೆ?’ ಎಂದು ಕೇಳಿದ. ಅದಕ್ಕೆ ಅಭೇದಾನಂದರು ‘ಇದೆಲ್ಲ ನಿಜ. ಇದನ್ನೆಲ್ಲಾ ಸಾಕ್ಷಾತ್ಕಾರ ಮಾಡಿಕೊಳ್ಳಲು ಕಷ್ಟವಿರಬಹುದು. ಆದರೆ ಇದು ಸತ್ಯ’ ಎಂದರು. ಮಾರನೆ ದಿನ ಆ ಹುಡುಗ ಮೌನ ಸಂನ್ಯಾಸಿಯಾಗಿ ಬೆತ್ತಲೆಯಾಗಿ ಕೇದಾರನಾಥಕ್ಕೆ ಹೊರಟಿದ್ದ. ಅವನಿಗೆ ಏನಾಯಿತು ಎಂದು ನೀವು ಕೇಳಬಹುದು. ಅವನು ಮೌನಿಯಾದ. 

\newpage

 “ಸಂನ್ಯಾಸಿ ಪೂಜೆ ಯಾತ್ರೆ ಯಾಗ ಯಜ್ಞಗಳನ್ನು ಮಾಡಬೇಕಾಗಿಲ್ಲ. ಆದರೂ ಅವನೇಕೆ ತೀರ್ಥದಿಂದ ತೀರ್ಥಕ್ಕೆ, ದೇವಸ್ಥಾನದಿಂದ ದೇವಸ್ಥಾನಕ್ಕೆ ಅಲೆಯುತ್ತಿರುವುದು? ಅವನು ಪುಣ್ಯಗಳಿಸಿ ಅದನ್ನು ಪ್ರಪಂಚಕ್ಕೆ ಕೊಡುವುದಕ್ಕಾಗಿ ಇದನ್ನು ಮಾಡುತ್ತಿರುವನು.” 

 ಅನಂತರ ಶಿಬಿರಾಣನ ಕಥೆ ಬಂತು ಎಂದು ಕಾಣುವುದು. ಕಥೆಯನ್ನು ಹೇಳುತ್ತಿದ್ದ ಸ್ವಾಮೀಜಿ ಹೀಗೆಂದರು: “ಹೌದು ಈ ಕಥೆಗಳೆಲ್ಲಾ ನಮ್ಮ ಜನಾಂಗದ ಅಂತರಾಳದಲ್ಲಿ ಸುಪ್ತವಾಗಿವೆ. ಸಂನ್ಯಾಸಿ ಎರಡು ಪ್ರತಿಜ್ಞೆ ಮಾಡುವನೆಂಬುದನ್ನು ಮರೆಯದಿರು: ಅದೇ ಆತ್ಮನ ಮೋಕ್ಷ ಮತ್ತು ಜಗತ್ತಿನ ಹಿತ. ಸ್ವರ್ಗಾದಿ ಕಾಮನೆಗಳನ್ನು ಎಂದೆಂದಿಗೂ ತ್ಯಜಿಸುವುದು. ಅತ್ಯಂತ ಕಠಿಣತಮವಾದುದು”


\section*{ಭರತಖಂಡದ ಭೂತ ಭವಿಷ್ಯತ್ತು}

 ಪ್ರಪಂಚವನ್ನು ಸುತ್ತುತ್ತಿರುವಾಗ ಒಬ್ಬನು ಗುರುವಿನೊಂದಿಗೆ ಹೋದರೆ ಅದೂ ಕೂಡ ಒಂದು ಯಾತ್ರೆಯಾಗುವುದು. ಒಂದು ದಿನ ಹಡಗು ಕೆಂಪುಸಮುದ್ರದಲ್ಲಿ ಹೋಗುತ್ತಿದ್ದಾಗ ಸಂಜೆಯಾಗಿತ್ತು. ಇತರರಿಗೆ ಸೇವೆ ಮಾಡುವಾಗ ಯಾವುದು ಸರಿಯಾದ ದೃಷ್ಟಿ ಎಂಬ ವಿಷಯವಾಗಿ ನಾನು ಅವರನ್ನು ಕೇಳಿದೆನು. ಇಂತಹ ಪ್ರಶ್ನೆಗೆ ಉತ್ತರ ಕೊಡುತ್ತಿದ್ದಾಗ ಅವರು ಯಾವುದಾದರೊಂದು ಶ್ಲೋಕವನ್ನು ಉದಾಹರಿಸದೆ ಇದ್ದುದು ಬಹಳ ಅಪರೂಪ. ಒಬ್ಬ ಅನಂತರ ಜೀವನದಲ್ಲಿ ಇದಕ್ಕಾಗಿ ಎಷ್ಟು ಕೃತಜ್ಞನಾಗುತ್ತಾನೆ! ಅವರ ಸ್ವಂತ ಅಭಿಪ್ರಾಯ ನನಗೆ ಬೇಕಾಗಿದ್ದುದು. ಆದರೆ ಇದನ್ನು ಒಂದು ಶ್ಲೋಕದ ಮೇಲಿನ ಭಾಷ್ಯದಂತೆ ಕೊಟ್ಟುದರಿಂದ ಅದು ನನ್ನ ಮನಸ್ಸಿನ ಆಳಕ್ಕೆ ಹೋಗಲು ಸಾಧ್ಯವಾಯಿತು. ಅವರು ಬರೀ ತಮ್ಮ ಅಭಿಪ್ರಾಯವೊಂದನ್ನೇ ಕೊಡುವುದಕ್ಕಿಂತ ಮತ್ತೂ ಹೆಚ್ಚು ವಿಷಯಗಳನ್ನು ಉದಾಹರಿಸಬೇಕಾಯಿತು. 

 ಯಾರು ವ್ರತಭ್ರಷ್ಟರಾಗುವರೋ ಅವರ ಗತಿ ಏನಾಗುವುದು ಎಂದು ಪ್ರಶ್ನಿಸಿದಾಗ ಅವರು ಮತ್ತೊಂದು ಸುಂದರವಾದ ಸಂಸ್ಕೃತ ಶ್ಲೋಕವನ್ನು ಹೇಳತೊಡಗಿದರು. ಈಗಲೂ ಕೂಡ ಹಡಗಿನಲ್ಲಿ ಯಾವ ಶ್ಲೋಕವನ್ನು ಕೇಳಿದೆನೊ ಅದನ್ನು ಸ್ವಾಮೀಜಿ ಅತಿ ಮಧುರವಾಗಿ ಉಚ್ಚರಿಸಿದ್ದು ಜ್ಞಾಪಕವಿದೆ:

\begin{verse}
ಅಯತಿಃ ಶ್ರದ್ಧಯೋಪೇತೋ ಯೋಗಾಚ್ಚಲಿತಮಾನಸಃ~।\\ಅಪ್ರಾಪ್ಯ ಯೋಗಸಂಸಿದ್ಧಿಂ ಕಾಂ ಗತಿ ಕೃಷ್ಣ ಗಚ್ಛತಿ~॥\\ಕಚ್ಚಿನ್ನೋಭಯವಿಭ್ರಷ್ಟಃ ಛಿನ್ನಾಭ್ರಮಿವ ನಶ್ಯತಿ~।\\ಅಪ್ರತಿಷ್ಠೋ ಮಹಾಬಾಹೋ ವಿಮೂಢೋ ಬ್ರಹ್ಮಣಃ ಪಥಿ~॥\\
\begin{flushright}
(ಗೀತಾ \general{\enginline{vi}}, ೩೭–೩೮)
\end{flushright}
\end{verse}

 ಶ್ರದ್ಧೆಯಿಂದ ಪ್ರಾರಂಭಮಾಡಿ ಕೊನೆಗೆ ಯೋಗಭ್ರಷ್ಟರಾದರೆ ಇಂತಹವರು ಯಾವ ಸ್ಥಿತಿಗೆ ಬರುತ್ತಾರೆ? ಗ್ರೀಷ್ಮಕಾಲದ ತುಂಡು ಮುಗಿಲಿನಂತೆ ಇಲ್ಲಿಯೂ ಭ್ರಷ್ಟರಾಗಿ ಮುಂದೆಯೂ ಭ್ರಷ್ಟರಾಗುತ್ತಾರೆಯೇ? ಶ‍್ರೀಕೃಷ್ಣ ಅದಕ್ಕೆ ಧೈರ‍್ಯವಾಗಿ ಉತ್ತರಕೊಡುತ್ತಾನೆ:

\begin{verse}
ಪಾರ್ಥ ನೈವೇಹ ನಾಮುತ್ರ ವಿನಾಶಸ್ತಸ್ಯ ವಿದ್ಯತೇ~। \\ನ ಹಿ ಕಲ್ಯಾಣಕೃತ್ ಕಶ್ಚಿತ್ ದುರ್ಗತಿಂ ತಾತ ಗಚ್ಛತಿ~॥ \\
\begin{flushright}
(ಗೀತಾ \general{\enginline{vi}}, ೪೦)
\end{flushright}
\end{verse}

 “ಹೇ ಪಾರ್ಥ, ಇಲ್ಲಿಯಾಗಲಿ ಅನಂತರದಲ್ಲಾಗಲಿ ಇಂತಹವನು ಎಂದಿಗೂ ನಾಶವಾಗುವುದಿಲ್ಲ. ಮಗು, ಒಳ್ಳೆಯದನ್ನು ಮಾಡಿದವನು ಎಂದಿಗೂ ನಾಶವಾಗುವುದಿಲ್ಲ.” 

 ಅನಂತರ ಅವರು ಬೇರೊಂದು ಮಾತನ್ನು ಎತ್ತಿದರು. ಅದನ್ನು ನಾನು ಎಂದೂ ಮರೆಯುವಂತಿಲ್ಲ. ಮೊದಲು ನಾವು ಮನೋವಾಕ್ಕಾಯವಾಗಿ ನಮ್ಮನ್ನು ನಾವು ನಿಗ್ರಹಿಸದೆ ಇದ್ದರೆ ಸ್ವೇಚ್ಛಾಜೀವನ ನಡೆಸಿದಂತೆ. ಕೆಲವುವೇಳೆ ಭ್ರಷ್ಟರಾದವರು ರಾಜರಾಗಿ ಹುಟ್ಟಿ ತಮ್ಮ ಇಚ್ಛೆಯನ್ನು ಪೂರೈಸಿಕೊಳ್ಳುವರು. ಅವರು ಸಿಂಹಾಸನದ ಮೇಲೆ ಇದ್ದರೂ ಹಿಂದಿನ ಜನ್ಮದ ಆ ಸಾಧುವಿನ ಚಿಹ್ನೆ ಅವರಲ್ಲಿ ಇರುವುದು. ಇಂತಹ ಒಂದು ನೆನಪಿನ ಛಾಯೆ ಇರುವುದೇ ಮಹಾಪುರುಷರ ಒಂದು ಲಕ್ಷಣವೆಂದು ಹೇಳುತ್ತಿದ್ದರು. ಅಕ್ಬರನಿಗೆ ಇಂತಹ ಒಂದು ಜ್ಞಾಪಕವಿತ್ತು. ತಾನೊಬ್ಬ ಭ್ರಷ್ಟನಾದ ಸಾಧು ಎಂದು ತಿಳಿದಿದ್ದನು. ಆದರೆ ಮುಂದಿನ ಜನ್ಮದಲ್ಲಿ ಒಳ್ಳೆಯ ಅನುಕೂಲವಾದ ವಾತಾವರಣದಲ್ಲಿ ಹುಟ್ಟಿ ಅಲ್ಲಿ ಜಯಶೀಲನಾಗುವನು. ಅನಂತರ ನನ್ನ ಗುರುದೇವರು ತಮ್ಮ ಬಾಲ್ಯದ ವಿಷಯವನ್ನು ಹೇಳಿದರು. ಅಪರೂಪ ಅವರು ಅದನ್ನು ಪ್ರಸ್ತಾಪ ಮಾಡುತ್ತಿದ್ದುದು. ನೆನಪಿನ ವಿಷಯವಾಗಿ ಮಾತನಾಡುತ್ತಿದ್ದಾಗ ಸ್ವಲ್ಪ ಹೊತ್ತು ತಮ್ಮ ಬಾಲ್ಯದ ವಿಷಯವನ್ನೇ ಹೇಳತೊಡಗಿದರು. ನನ್ನ ಕಡೆ ತಿರುಗಿನೋಡಿ ನನ್ನ ಹೆಸರನ್ನು ಹೇಳುತ್ತ ಹೀಗೆಂದರು: “ನೀನು ಏನಾದರೂ ತಿಳಿದುಕೊ, ನನಗೇನೋ ಅಂತಹ ನೆನಪಿದೆ. ನನಗೆ ಎರಡು ವರುಷವಾದಾಗ ನಾನು ಕುದುರೆಯವನೊಡನೆ ಆಡುತ್ತಿದ್ದಾಗ ಮೈಗೆ ಬೂದಿ ಬಳಿದುಕೊಂಡು ಕೌಪೀನ ಧರಿಸಿ ಭೈರಾಗಿಯಂತೆ ಆಡುತ್ತಿದ್ದೆ. ಯಾರಾದರೂ ಸಾಧುಗಳು ಮನೆಗೆ ಬಂದರೆ ನನ್ನನ್ನು ರೂಮಿನಲ್ಲಿ ಕೂಡಿಹಾಕುತ್ತಿದ್ದರು, ಎಲ್ಲಿ ಅವರಿಗೆ ಹೆಚ್ಚಾಗಿ ಕೊಟ್ಟುಬಿಡುವೆನೊ ಎಂದು. ನಾನು ಕೂಡ ಶಿವನ ಹತ್ತಿರ ಭೈರಾಗಿಯಂತೆ ಇದ್ದೆ, ಏನೋ ತಪ್ಪು ಮಾಡಿದುದಕ್ಕಾಗಿ ನನ್ನನ್ನು ಭೂಲೋಕಕ್ಕೆ ಕಳುಹಿಸಿರಬೇಕೆಂದು ಯೋಚಿಸುತ್ತಿದ್ದೆ. ನನ್ನ ಮನೆಯವರು ಕೂಡ ಈ ಅಭಿಪ್ರಾಯವನ್ನೆ ಪುಷ್ಟಿಗೊಳಿಸಿದರು. ನಾನು ಬಹಳ ಚೇಷ್ಟೆ ಮಾಡುತ್ತಿದ್ದಾಗ “ಅಯ್ಯೋ ರಾಮ, ನಾನು ಎಷ್ಟು ತಪಸ್ಸು ಮಾಡಿದರೂ ಶಿವ ಕೊನೆಗೆ ಈ ದುಷ್ಟನನ್ನು ಕಳುಹಿಸಿದನಲ್ಲ, ಇನ್ನು ಯಾವುದಾದರೂ ಒಳ್ಳೆಯ ಮಗುವನ್ನು ಕಳುಹಿಸುವುದು ಬಿಟ್ಟು” ಎಂದು ಹೇಳುತ್ತಿದ್ದರು ಅಥವಾ ತುಂಬಾ ಅಸಾಧ್ಯವಾದಾಗ ನನ್ನ ತಲೆಯ ಮೇಲೆ ಒಂದು ಕೊಡ ನೀರನ್ನು ಸುರಿದು ‘ಶಿವ ಶಿವ’ ಎನ್ನುತ್ತಿದ್ದರು. ಅನಂತರ ನಾನು ಯಾವಾಗಲೂ ತೆಪ್ಪಗಾಗುತ್ತಿದ್ದೆ. ಈಗಲೂ ಕೂಡ ತುಂಟತನ ಜಾಸ್ತಿಯಾದಾಗ ಆ ಪದ ನನ್ನನ್ನು ಸರಿಮಾಡುವುದು. ಇಲ್ಲ, ಇನ್ನುಮೇಲೆ ಹೀಗೆ ಮಾಡುವುದಿಲ್ಲ ಎಂದುಕೊಳ್ಳುತ್ತೇನೆ.” 

 ಅನಂತರ ಗೀತೆಯ ವಿಷಯವಾಗಿ ಸಧ್ಯದಲ್ಲಿ ಮಾತನಾಡತೊಡಗಿದರು. ಗೀತೆಯಲ್ಲಿ ತಾಮಸಿಕ, ರಾಜಸಿಕ ಮತ್ತು ಸಾತ್ತ್ವಿಕ ಎಂಬ ಮೂರು ಬಗೆಯ ದಾನಗಳಿವೆ ಎನ್ನುತ್ತಾರೆ. ತಾಮಸ ದಾನ ಉದ್ವೇಗದಿಂದ ಮಾಡುವ ದಾನ, ಅದು ಯಾವಾಗಲೂ ತಪ್ಪು ಮಾಡುವುದು. ಕೊಡುವವನು ತಾನು ದಾನಿಯಾಗಬೇಕೆಂದು ಬಯಸುವನೇ ಹೊರತು ಸ್ವೀಕರಿಸುವವನ ಹಿತವನ್ನು ಬಯಸುವುದಿಲ್ಲ. ರಾಜಸಿಕ ದಾನ ಕೀರ್ತಿಗೋಸುಗವಾಗಿ ಮಾಡುವನು. ಸಾತ್ತ್ವಿಕ ದಾನ ಸಕಾಲದಲ್ಲಿ ಸರಿಯಾದ ಪಾತ್ರವನ್ನು ನೋಡಿ ಮಾಡುವ ದಾನ. ನಾನು ಕೇಳಿದ ಪ್ರಸಂಗವನ್ನು ತೆಗೆದುಕೊಂಡು ನಿನ್ನ ದಾನವಾದರೋ ತಾಮಸಿಕ ಎಂದರು. ಸಾತ್ತ್ವಿಕ ದಾನವನ್ನು ಆಲೋಚಿಸಿದರೆ ಒಬ್ಬ ಪ್ರಖ್ಯಾತ ಪಾಶ್ಚಾತ್ಯ ಮಹಿಳೆಯಲ್ಲಿ ನಾನು ಇದನ್ನು ನೋಡಿರುವೆನು. ಆಕೆ ಯಾವಾಗಲೂ ಸರಿಯಾದ ಮನುಷ್ಯನಿಗೆ, ಸರಿಯಾದ ರೀತಿಯಲ್ಲಿ, ಸರಿಯಾದ ಕಾಲದಲ್ಲಿ ಸದ್ದುಗದ್ದಲವಿಲ್ಲದೆ ಕೊಡುತ್ತಿರುವಳು ಎಂದರು. ದಾನ ಕೂಡ ಕೆಲವು ವೇಳೆ ಮಿತಿ ಮೀರಬಹುದೆಂದು ನಾನು ಯೋಚಿಸುವೆನು ಎಂದರು. 

 ಮಾತನಾಡುತ್ತಿದ್ದಂತೆ ಅವರು ಮೌನವಾದರು. ನಾವು ತಾರಾಕಾಂತಿಯ ಕೆಳಗಡೆ ಇದ್ದ ಸಾಗರವನ್ನು ನೋಡುತ್ತಿದ್ದೆವು. ಅನಂತರ ಅವರು ಪುನಃ ಮಾತನಾಡತೊಡಗಿದರು: “ನನಗೆ ವಯಸ್ಸಾದಂತೆ ಸಣ್ಣ ಸಣ್ಣ ವಿಷಯಗಳಲ್ಲಿ ಮಹಾತ್ಮೆಯನ್ನು ನಾನು ನೋಡಲಪೇಕ್ಷಿಸುವೆನು. ಒಬ್ಬ ಮಹಾಪುರುಷ ಏನು ಉಡುತ್ತಾನೆ, ಏನು ತಿನ್ನುತ್ತಾನೆ, ಅವನು ಆಳುಗಳೊಡನೆ ಹೇಗೆ ಮಾತನಾಡುತ್ತಾನೆ ಎಂಬುದನ್ನು ನೋಡಬಯಸುತ್ತೇನೆ. ಸರ್ ಫಿಲಿಪ್ ಸಿಡ್ನಿಯ ಮಹಾತ್ಮೆಯನ್ನು ನೋಡಬಯಸುತ್ತೇನೆ. ತಾವು ಸಾಯುವಾಗಲೂ ಇತರರ ದಾಹವನ್ನು ನೆನಪಿನಲ್ಲಿಡುವಂತಹವರು ಬಹಳ ಅಪರೂಪ.” 

 “ಒಂದು ಕೀಟ ಮೌನವಾಗಿ ಎಡಬಿಡದೆ ತನ್ನ ಕರ್ತವ್ಯವನ್ನು ಪರಿಪಾಲಿಸುವುದರಲ್ಲಿ ನಿಜವಾದ ಮಹಾತ್ಮೆ ಹೆಚ್ಚು ವ್ಯಕ್ತವಾಗುವುದು.” ಭೂಪಟದ ಮೇಲಿರುವ ಹಲವು ಸ್ಥಳಗಳು ನನಗೀಗ ಅಷ್ಟು ಸುಂದರವಾಗಿ ಕಾಣುತ್ತಿವೆ! ಆಯಾ ಸ್ಥಳಗಳಲ್ಲಿ ಯಾವ ವಿಷಯವನ್ನು ಸ್ವಾಮೀಜಿ ಮಾತನಾಡಿದರು ಎಂಬುದನ್ನು ಅದು ನನ್ನ ನೆನಪಿಗೆ ತರುತ್ತದೆ. ನಾವು ಇಟಲಿಯ ತೀರದಲ್ಲಿ ಹೋಗುತ್ತಿದ್ದಾಗ ಚರ್ಚಿನ ವಿಷಯ ಮಾತನಾಡಿದೆವು. ಭೋನಿಪ್ಯಾಟಿಯೊ ಜಲಸಂಧಿಯ ಮೂಲಕ ಹೋಗುತ್ತಿದ್ದಾಗ ಕಾರ‍್ಸಿಕಾ ದ್ವೀಪದ ದಕ್ಷಿಣ ಭಾಗವನ್ನು ನೋಡುತ್ತಾ ಕುಳಿತರು. ಯೋಧ ಸಾಮ್ರಾಟ್ ನೆಪೋಲಿಯನ್ನಿಗೆ ಜನ್ಮವಿತ್ತ ದ್ವೀಪದ ವಿಷಯವಾಗಿ ಮೆಲ್ಲನೆ ಮಾತನಾಡತೊಡಗಿದರು. ರಾಬಸ್‌ಪಿಯರಿನ ಶಕ್ತಿ ಮತ್ತು ವಿಕ್ಟರ್‌ಹ್ಯೂಗೊ ಮೂರನೆ ನೆಪೋಲಿಯನ್‌ನನ್ನು ನಿಕೃಷ್ಟ ದೃಷ್ಟಿಯಿಂದ ನೋಡುತ್ತಿದ್ದುದು ಇವನ್ನೆಲ್ಲಾ ವಿವರಿಸಿದರು. 

 ನಾವು ಜಿಬ್ರಾಲ್ಟರ್ ಜಲಸಂಧಿಯನ್ನು ದಾಟಿದ ಪ್ರಾತಃಕಾಲ ಡೆಕ್ಕಿನಮೇಲೆ ಬಂದಾಗ ಸ್ವಾಮೀಜಿ, “ನೀನು ಅವರನ್ನು ಕಂಡೆಯಾ? ‘ದೀನ್ ಮಹಮದ್’ ಎಂದು ಘೋಷಿಸುತ್ತಾ ದಂಡೆಯಾತ್ರೆ ಎತ್ತಿದವರನ್ನು?” ಎಂದು ಪ್ರಶ್ನಿಸಿದರು. ಸುಮಾರು ಅರ್ಧ ಗಂಟೆಯ ಕಾಲ ಮೂವರು ಸ್ಪೆಯಿನ್ ದೇಶವನ್ನು ಆಕ್ರಮಿಸಿದ ವಿಚಾರವನ್ನು ಕಣ್ಣಿಗೆ ಕಾಣುವಂತೆ ನಾಟಕೀಯವಾಗಿ ಚಿತ್ರಿಸಿದರು. 

 ಪುನಃ ಭಾನುವಾರ ಸಾಯಂಕಾಲ ಬುದ್ಧನ ವಿಷಯವನ್ನು ಮಾತನಾಡತೊಡಗಿದರು. ಎಲ್ಲರಿಗೂ ಪರಿಚಿತವಾದ ಅವನ ಘಟನೆಗಳಲ್ಲಿ ಹೊಸ ಚೈತನ್ಯವನ್ನು ತುಂಬಿ ಎಂತಹ ಅದ್ಭುತವಾದ ತ್ಯಾಗ ಅವನದು ಎಂದು ಅವನ ದೃಷ್ಟಿಯಿಂದ ವಿವರಿಸತೊಡಗಿದರು. 

 ಅವರ ಮಾತು ಕೇವಲ ಮನರಂಜಕ ಅಥವಾ ಶಿಕ್ಷಣಪ್ರದವಾಗಿರಲಿಲ್ಲ. ಮಧ್ಯೆ ಮಧ್ಯೆ ಜೀವನದ ಉದ್ದೇಶವನ್ನು ಕುರಿತು ಉತ್ಸಾಹಪೂರಿತವಾಗಿ ಮಾತನಾಡುತ್ತಿದ್ದರು. ಇವರು ಹಾಗೆ ಮಾತನಾಡಿದಾಗಲೆಲ್ಲ ಅವರು ಹೇಳಿದ ಪ್ರತಿಯೊಂದು ಮಾತು ಅನರ್ಘ್ಯ ರತ್ನದಂತೆ ನನ್ನ ಹೃದಯದಲ್ಲಿ ಸಂರಕ್ಷಿಸತೊಡಗಿದೆ. ಸ್ವಾಮೀಜಿ ಇತ್ತುದನ್ನು ಅವರ ದೇಶದವರಿಗೆ ರವಾನೆ ಮಾಡುವುದಕ್ಕೆ ಇರುವ ವ್ಯಕ್ತಿಯೆಂದೂ, ಅವರಿಗೂ ದೇಶದ ಅಸಂಖ್ಯಾತ ಜನರಿಗೂ ಇರುವ ಸೇತುವೆ ನಾನು ಎಂದೂ ಭಾವಿಸಿದೆ. ಭರತಖಂಡದ ಜನರು ಜಾಗ್ರತರಾಗಿ ಸ್ವಾಮೀಜಿಯವರ ಕನಸುಗಳನ್ನು ಒಂದಲ್ಲ ಒಂದು ದಿನ ಕಾರ‍್ಯಗತ ಮಾಡುವರು. 

 ಇಂತಹ ಒಂದು ಸನ್ನಿವೇಶ ನಾವು ಒಂದು ದಿನ ಸಂಜೆ ಏಡನ್ ನಗರವನ್ನು ಸಮೀಪಿಸಿದಾಗ ಬಂದಿತು. ನಾನು ಅಂದಿನ ಬೆಳಿಗ್ಗೆ ಭರತಖಂಡದ ಹಿತದ ವಿಷಯದಲ್ಲಿ ಸ್ವಾಮೀಜಿ ಯೋಚನೆಗೂ ಇತರರ ದೃಷ್ಟಿಗೂ ಏನು ವ್ಯತ್ಯಾಸ ಎಂಬುದನ್ನು ಕೇಳಿದೆ. ಈ ವಿಷಯದ ಮೇಲೆ ಅವರ ಅಭಿಪ್ರಾಯವನ್ನು ತಿಳಿದುಕೊಳ್ಳಲು ಆಗಲಿಲ್ಲ. ಅದಕ್ಕೆ ಬದಲು ಇತರರು ಮಾಡುತ್ತಿರುವ ಕೆಲಸ ಅವರ ವ್ಯಕ್ತಿತ್ವ ಇವುಗಳನ್ನು ಶ್ಲಾಘಿಸತೊಡಗಿದರು. ಪ್ರಶ್ನೆ ಇಲ್ಲಿಗೇ ಮುಕ್ತಾಯವಾಯಿತೆಂದು ಭಾವಿಸಿದೆ. ಅಂದಿನ ಸಂಜೆ ಇದ್ದಕ್ಕಿದ್ದಂತೆ ಸ್ವಾಮೀಜಿ ಬೆಳಗಿನ ಪ್ರಶ್ನೆಯನ್ನೇ ಪ್ರಸ್ತಾಪಿಸತೊಡಗಿದರು: 

 “ನಮ್ಮ ಜನರಿಗೆ ಮೂಢನಂಬಿಕೆಗಳನ್ನು ಪುನಃ ಕೊಡಲು ಪ್ರಯತ್ನಿಸುತ್ತಿರುವವರನ್ನು ನಾನು ಮೆಚ್ಚುವುದಿಲ್ಲ. ಈಜಿಪ್ಟಿನ ಪ್ರಾಕ್ತನ ವಿಮರ್ಶಕನಿಗೆ (\enginline{Egiptologist}) ಗತಕಾಲದ ಈಜಿಪ್ಟಿನಲ್ಲಿರುವ ಆಸಕ್ತಿಯಂತೆ ಇದೆ, ಭರತಖಂಡದ ಜನರ ಮೇಲೆ ಇವರಿಗೆ ಇರುವ ಆಸಕ್ತಿ, ಅದು ಬರೀ ಸ್ವಾರ್ಥ. ನಾವು ಹಿಂದೆ ಗ್ರಂಥಗಳಲ್ಲಿ ಓದಿದ ಭರತಖಂಡವನ್ನು, ಕನಸಿನಲ್ಲಿ ಕಟ್ಟಿಕೊಂಡ ಭರತಖಂಡವನ್ನು ನೋಡಲು ಇಚ್ಛಿಸಬಹುದು. ಆದರೆ ನನ್ನ ದೃಷ್ಟಿಯಲ್ಲಿ ಹಿಂದಿನ ಕಾಲದ ಒಳ್ಳೆಯದೂ ಇರಬೇಕು, ಜೊತೆಗೆ ಈಗಿನ ಕಾಲದ ಒಳ್ಳೆಯದೂ ಇರಬೇಕು. ಇವೆರಡೂ ಸ್ವಾಭಾವಿಕವಾಗಿ ಸಮ್ಮಿಲನವಾಗಿರಬೇಕು.

 “ಆದಕಾರಣವೇ ನಾನು ಬರೀ ಉಪನಿಷತ್ತನ್ನು ಬೋಧಿಸುವುದು. ನೀನು ಹಾಗೆಯೇ ನೋಡಿದರೆ ನಾನು ಉಪನಿಷತ್ತಲ್ಲದೆ ಬೇರೆ ಏನನ್ನೂ ಉದಾಹರಿಸಿಲ್ಲ. ಉಪನಿಷತ್ತಿನಲ್ಲಿ ಕೂಡ ‘ಬಲ’ ಎಂಬ ಒಂದು ಮಾತನ್ನು ಮಾತ್ರವೆ. ವೇದ ವೇದಾಂತಗಳ ಸಾರವೆಲ್ಲ ಆ ಒಂದು ಮಾತಿನಲ್ಲಿ ಹುದುಗಿದೆ. ಬುದ್ಧ ಅಹಿಂಸೆಯನ್ನು ಬೋಧಿಸಿದ. ಆದರೆ ಬಲ ಎಂಬುದು ಅದೇ ಅಹಿಂಸೆಯನ್ನು ಬೋಧಿಸುವುದಕ್ಕೆ ಉತ್ತಮ ಮಾರ್ಗ ಎಂದು ನಾನು ಭಾವಿಸುತ್ತೇನೆ. ಆ ಅಹಿಂಸೆಯ ಹಿಂದೆ ಅತಿ ಹೀನವಾದ ದೌರ್ಬಲ್ಯವಿತ್ತು. ಎದುರಿಸಬೇಕು ಎಂಬ ಭಾವನೆಗೆ ಮೂಲ ದೌರ್ಬಲ್ಯ. ನಾನೊಂದು ಕೀಟವನ್ನು ಶಿಕ್ಷಿಸಬೇಕು ಎಂದಾಗಲಿ, ಅದರಿಂದ ಪಾರಾಗಬೇಕೆಂದಾಗಲಿ ಯೋಚಿಸುವುದಿಲ್ಲ. ನಾನು ಇದನ್ನು ಗಣನೆಗೇ ತರುವುದಿಲ್ಲ. ಆದರೆ ಸೊಳ್ಳೆಗೆ ಅದೊಂದು ಮಹಾ ಪ್ರಮಾದ. ಎಲ್ಲಾ ಹಿಂಸೆಯನ್ನು ನಾನು ಈ ದೃಷ್ಟಿಯಿಂದ ನೋಡುತ್ತೇನೆ. ಧೈರ್ಯ ಮತ್ತು ಬಲ ಬೇಕಾಗಿರುವುದು. ನನ್ನ ಆದರ್ಶವೇ ಸಿಪಾಯಿದಂಗೆಯಲ್ಲಿ ಖೂನಿಗೆ ತುತ್ತಾದ ಆ ಮಹಾ ಸಾಧುಪುರುಷ. ಆತನನ್ನು ಯಾರೋ ತಿವಿದಾಗ ಕೊಲೆಪಾತಕನನ್ನು ನೋಡಿ “ನೀನು ಕೂಡ ಅವನೆ” ಎಂದನು. 

 “ಇಲ್ಲಿ ಶ‍್ರೀರಾಮಕೃಷ್ಣರ ಸ್ಥಾನ ಯಾವುದು?” ಎಂದು ನೀವು ಕೇಳಬಹುದು. 

 “ಅವರೇ ಮಾರ್ಗ, ಅದ್ಭುತವಾದ ನಮಗೆ ಅರಿವಿಲ್ಲದ ಮಾರ್ಗ. ಅವರು ತಮ್ಮನ್ನೇ ತಿಳಿದಿರಲಿಲ್ಲ. ಅವರಿಗೆ ಇಂಗ್ಲೆಂಡಾಗಲಿ, ಇಂಗ್ಲೀಷರಾಗಲಿ ಯಾರೂ ಪರಿಚಯವಿರಲಿಲ್ಲ. ಎಲ್ಲೋ ಸಮುದ್ರದಾಚೆಯಿಂದ ಬರುವ ವಿಚಿತ್ರ ಮನುಷ್ಯರು ಇವರು ಎಂದು ಭಾವಿಸಿದ್ದರು. ಆದರೆ ಅವರು ಅದ್ಭುತವಾದ ಜೀವನವನ್ನು ಬಾಳಿದರು. ನನಗೆ ಅವರ ಮಹಾತ್ಮೆ ಗೊತ್ತಾಯಿತು. ಅವರು ಯಾರನ್ನೂ ಒಮ್ಮೆಯೂ ಜರಿದವರಲ್ಲ. ನಾನು ಒಂದು ಸಲ ವಾಮಾಚಾರವನ್ನು ಖಂಡಿಸುತ್ತಿದ್ದೆ. ಮೂರು ಘಂಟೆಗಳ ಕಾಲ ಸಿಕ್ಕಾಪಟ್ಟೆ ಮಾತನಾಡಿದೆ. ನಾನು ಪೂರೈಸಿದ ಮೇಲೆ ಸಾವಧಾನದಿಂದ “ಆಗಲಿ ಆಗಲಿ, ಪ್ರತಿಯೊಂದು ಮನೆಗೂ ಒಂದು ಹಿತ್ತಲು ಬಾಗಿಲು ಇರಬಹುದು” ಎಂದರು ಆ ವೃದ್ಧರು ಕೊನೆಗೆ. 

 “ಇದುವರೆಗೆ ನಮ್ಮ ಹಿಂದೂಧರ್ಮದಲ್ಲಿ ಇವೆರಡೇ ಪದಗಳಿರುವುವು - ತ್ಯಾಗ ಮತ್ತು ಮುಕ್ತಿ. ಇದೇ ಮಹಾದೋಷ. ಇಲ್ಲಿ ತ್ಯಾಗಿಗೆ ಮಾತ್ರ ಮುಕ್ತಿ, ಗೃಹಸ್ಥನಿಗೆ ಏನೂ ಇಲ್ಲ. ಇದೇ ಮಹಾ ದೋಷ. 

 “ನಾನು ಸಹಾಯ ಮಾಡಬೇಕೆಂದಿರುವುದು ಗೃಹಸ್ಥರಿಗೆ. ಎಲ್ಲಾ ಆತ್ಮರೂ ಒಂದೇ ಅಲ್ಲವೆ? ಎಲ್ಲರ ಗುರಿಯೂ ಒಂದೇ ಅಲ್ಲವೆ? ಆದಕಾರಣವೆ ಶಿಕ್ಷಣದ ಮೂಲಕ ಶಕ್ತಿ ಬರಬೇಕು.” 

 ನನ್ನ ಗುರುದೇವನ ಈ ಒಂದು ಮಾತನ್ನು ಕೇಳಿದ್ದು ಸಾಕು, ಇದರಿಂದ ನನ್ನ ಇಡೀ ಪ್ರಯಾಣ ಸಾರ್ಥಕವಾಯಿತು ಎಂದು ನಾನಂದು ಭಾವಿಸಿದೆ. ಇಂದೂ ದಿನ ಕಳೆದಂತೆ ಅದನ್ನು ಹೆಚ್ಚು ಹೆಚ್ಚು ಮನನ ಮಾಡುತ್ತಿರುವೆನು. 

\delimiter

 ಸ್ವಾಮೀಜಿ ಸಮುದ್ರದ ಮೇಲೆ ಪಾಶ್ಚಾತ್ಯ ದೇಶಗಳಿಗೆ ತೇಲಿಹೋಗುತ್ತಿದ್ದರು. ಅವರ ಹೃದಯ ಮಾತ್ರ ದೇಹ ಎಲ್ಲಿರಲಿ ಚಿಂತೆಯಿಲ್ಲ, ಭರತಖಂಡದ ಹಿತವನ್ನು ಮಾತ್ರ ಚಿಂತಿಸುತ್ತಿತ್ತು. ಸೋದರಿ ನಿವೇದಿತಾ ಆ ಚಿಂತನೆಯ ಹನಿಗಳನ್ನು, ತನ್ನ ಹೃದಯದಲ್ಲಿ ಸಂಗ್ರಹಿಸಿ ಅನಂತರ ತನ್ನ ಲೇಖನಿಯ ಮೂಲಕ ವ್ಯಕ್ತಪಡಿಸಿರುವಳು. ನಾವು ಅವುಗಳಲ್ಲಿ ಕೆಲವನ್ನು ಮಾತ್ರ ಕೊಟ್ಟಿರುವೆವು.

