
\chapter{ಶ‍್ರೀರಾಮಕೃಷ್ಣರ ಪ್ರೇಮದ ಆರೈಕೆಯಲ್ಲಿ}

ಶ‍್ರೀರಾಮಕೃಷ್ಣರ ಮತ್ತು ನರೇಂದ್ರನಾಥನ ಸಮಾಗಮ ಒಂದು ಅಪೂರ್ವ ಘಟನೆ. ಹೊರಗಿನಿಂದ ನೋಡಿದರೆ ಒಂದಕ್ಕೆ ಮತ್ತೊಂದು ವಿರೋಧವಾಗಿ ಕಾಣುವುದು. ಆದರೆ ಆಂತರ‍್ಯದಲ್ಲಿ ಮಾತ್ರ ಸಾಮ್ಯವಿರುವುದನ್ನು ನೋಡುವೆವು. ಅದು ಎಷ್ಟರಮಟ್ಟಿಗೆ ಸಾಮ್ಯವೆಂದರೆ ಒಂದೇ ನಾಣ್ಯದ ಎರಡು ಕಡೆಗಳಂತೆ ಕಾಣುವಷ್ಟು ಸಾಮ್ಯ. ಶ‍್ರೀರಾಮಕೃಷ್ಣರಿಗೆ ನರೇಂದ್ರನಾಥನನ್ನು ನೋಡಿದೊಡನೆಯೆ ಇವನಾರು ಎನ್ನುವುದು ಗೊತ್ತಾಗುವುದು. ಆದರೆ ನರೇಂದ್ರ ಇದನ್ನು ಹಲವು ವರುಷಗಳು ಸಾಧನೆ ಮಾಡಿ, ಸಂಶಯದಲ್ಲಿ ತೊಳಲಾಡಿ, ತರ್ಕದ ಯಂತ್ರದಲ್ಲಿ ಜಾಲಾಡಿ ತಿಳಿದುಕೊಳ್ಳಬೇಕಾಯಿತು.

ನರೇಂದ್ರನಾಥನನ್ನು ದಕ್ಷಿಣೇಶ್ವರದಲ್ಲಿ ನೋಡಿದಾಗ ಶ‍್ರೀರಾಮಕೃಷ್ಣರಿಗೆ ಸುಮಾರು ನಲವತ್ತೈದು ವರ್ಷಗಳಷ್ಟು ಇದ್ದಿರಬಹುದು. ನರೇಂದ್ರನಿಗಾದರೋ ಹತ್ತೊಂಭತ್ತು ಇಪ್ಪತ್ತು ವರ್ಷಗಳು ಇರಬಹುದು. ತಂದೆಗೂ ಮಗನಿಗೂ ಇರುವ ಸಾಮ್ಯ ವಯಸ್ಸಿನಲ್ಲಿ. ಶ‍್ರೀರಾಮಕೃಷ್ಣರೂ ಕೂಡ ತಂದೆಯ ಪ್ರೀತಿಯನ್ನು ನರೇಂದ್ರನ ಮೇಲೆ ಹರಿಸಿದರು. ನರೇಂದ್ರನಾದರೋ ಮಗುವಿನ ತುಂಟತನವನ್ನೇ ತೋರುವನು. ಪ್ರಚಂಡ ಸಾಧನೆ ಮಾಡಿ ಶ‍್ರೀರಾಮಕೃಷ್ಣರ ದೇಹಾರೋಗ್ಯವಾದರೋ ಬಹಳ ಸೂಕ್ಷ್ಮವಾಗಿತ್ತು. ನರೇಂದ್ರನಾದರೋ ಗರಡಿಯ ಮನೆಯಲ್ಲಿ ಚೆನ್ನಾಗಿ ಅಂಗಸಾಧನೆ ಮಾಡಿ ಗಟ್ಟಿಮುಟ್ಟಾಗಿ ಬೆಳೆದಿದ್ದ.

ಇನ್ನೂ ಒಳಹೊಕ್ಕು ನೋಡಿದರಂತೂ ರಾಮಕೃಷ್ಣರಿಗೂ ನರೇಂದ್ರನಿಗೂ ಉತ್ತರ ದಕ್ಷಿಣ ಧ್ರುವದಷ್ಟು ಅಂತರ. ರಾಮಕೃಷ್ಣರು ಬಾಲ್ಯದಲ್ಲಿ ಅವರ ಹಳ್ಳಿಯ ಶಾಲೆಗೆ ಕೆಲವು ದಿನ ಹೋದವರೆ ಹೊರತು ಅವರಿಗೆ ಯಾವ ಶಾಲೆಯ ವಿದ್ಯೆಯಾಗಲಿ ಅಥವಾ ಮನೆಯಲ್ಲಿ ಗ್ರಂಥಗಳನ್ನು ಓದಿ ಸಂಗ್ರಹಿಸಿದ ವಿದ್ಯೆಯಾಗಲಿ ಇರಲಿಲ್ಲ. ಜೀವನವೇ ಅವರ ಪುಸ್ತಕ. ಕಣ್ಣಿಗೆ ಕಂಡುದನ್ನು ನೋಡಿದರು. ಅದರ ಮೂಲಕ್ಕೆ ನಾಮರೂಪಗಳನ್ನು ಭೇದಿಸಿಕೊಂಡು ಹೋದರು. ತಮಗೆ ಆಧ್ಯಾತ್ಮಿಕ ಜೀವನದಲ್ಲಿ ಆದ ಅನುಭವಗಳಲ್ಲದೆ ಉಪನಿಷತ್ತು ಗೀತೆ ಮತ್ತು ಇತರ ಶಾಸ್ತ್ರಗಳಾವುದನ್ನೂ ಓದಿದವರಲ್ಲ. ಆತ್ಮ ಸಾಕ್ಷಾತ್ಕಾರದ ಸೋಪಾನಕ್ಕೆ ನೇರವಾಗಿ ಹೋದರು. ಆದರೆ ನರೇಂದ್ರನಾದರೋ ಪುಸ್ತಕ ವಿದ್ಯೆಯಲ್ಲಿ ಪಂಡಿತನಾಗಿದ್ದ. ಶಾಲೆಯಲ್ಲಿ ಎಷ್ಟು ಚುರುಕೋ ಹಲವಾರು ಪ್ರೌಢಗ್ರಂಥಗಳನ್ನು ಓದಿ ವಿಷಯ ಸಂಗ್ರಹಿಸುವುದರಲ್ಲೂ ಅಷ್ಟೇ ಚುರುಕು ನರೇಂದ್ರನಿಗೆ ಅಸ್ಪಷ್ಟವಾಗಿ ಬಾಲ್ಯದಲ್ಲಿ ಕೆಲವು ಅನುಭವಗಳಾಗಿದ್ದರೂ, ಶ್ರದ್ಧೆ ಭಕ್ತಿಗಳಿದ್ದರೂ, ಹೇಳಿಕೊಳ್ಳುವಷ್ಟು ಉತ್ತಮವರ್ಗದ, ಸಂಶಯಗಳನ್ನೆಲ್ಲ ನಿರ್ಮೂಲ ಮಾಡಬಲ್ಲಂತಹ ಅನುಭವ ದಕ್ಕಿರಲಿಲ್ಲ. ಶ‍್ರೀರಾಮಕೃಷ್ಣರಿಗಾದರೋ ಆಧ್ಯಾತ್ಮಿಕ ಅನುಭವ ಅಂಗೈ ಮೇಲಿರುವ ನೆಲ್ಲೀಕಾಯಿ. ಅವರ ದಿನ ನಿತ್ಯದ ವ್ಯವಹಾರ ಭೂಮಿಕೆಯೇ ಇದು. ದೇವದೇವಿಯರು, ಸಾಕಾರ ನಿರಾಕಾರ, ದ್ವೈತ ಅದ್ವೈತ ಇವುಗಳ ಸತ್ಯವನ್ನೆಲ್ಲ ಪ್ರತ್ಯಕ್ಷ ಅನುಭವಿಸಿದವರು. ಇದರಲ್ಲಿ ಅವರಿಗೆ ಎಳ್ಳಷ್ಟೂ ಅನುಮಾನವಿರಲಿಲ್ಲ. ಇದರ ವಿಷಯವಾಗಿ ತರ್ಕಿಸುವುದಕ್ಕೆ ಹೋಗುತ್ತಿರಲಿಲ್ಲ. ಆದರೆ ನರೇಂದ್ರನಾದರೋ ಇಂತಹ ಅನುಭವಗಳ ರುಚಿಯನ್ನು ಶ‍್ರೀರಾಮಕೃಷ್ಣರೇ ಅವನಿಗೆ ಕೊಟ್ಟರೂ ಅದನ್ನು ಅನುಮಾನಿಸುವನು. ಬೇರೊಂದು ದೇಶದ ನಾಣ್ಯವನ್ನು ಕೊಟ್ಟರೆ ಹೇಗೆ ಒಬ್ಬ ಕಕ್ಕಾಬಿಕ್ಕಿಯಾಗುತ್ತಾನೋ ಹಾಗೆ ಆಗುತ್ತಿದ್ದನು. ಶ‍್ರೀರಾಮಕೃಷ್ಣರು ಈ ಭೂಮಿಯ ಮೇಲೆ ನಿಂತಿದ್ದರೂ ಅವರ ಮನಸ್ಸಾದರೋ ಲೋಕಾಲೋಕಗಳ ರಹಸ್ಯವನ್ನೆಲ್ಲ ಕ್ಷಣಾರ್ಧದಲ್ಲಿ ಭೇದಿಸಿ ತಿಳಿದುಕೊಳ್ಳಬಲ್ಲದ್ದಾಗಿತ್ತು. ಶ‍್ರೀರಾಮಕೃಷ್ಣರ ಪಾದಗಳು ಮಾತ್ರ ವ್ಯವಹಾರ ಜಗತ್ತಿನಲ್ಲಿದ್ದವು. ಅವರ ಶಿರವಾದರೋ ಯಾವಾಗಲೂ ಸಚ್ಚಿದಾನಂದ ಪರಬ್ರಹ್ಮನ ಸತ್ಯದಲ್ಲಿ ನೆಲೆಸಿತ್ತು. ತುಂಬಾ ಕಷ್ಟಪಟ್ಟು ತಮ್ಮ ಮನಸ್ಸನ್ನು ನಮ್ಮ ವ್ಯವಹಾರ ಭೂಮಿಕೆಗೆ ಇಳಿಸಬೇಕಾಗಿತ್ತು. ನಾವು ಮನಸ್ಸನ್ನು ದೇವರೆಡೆಗೆ ತೆಗೆದುಕೊಂಡು ಹೋಗಬೇಕಾದರೆ ಎಷ್ಟು ಕಷ್ಟಪಡಬೇಕೊ ತಮ್ಮ ಮನಸ್ಸನ್ನು ನಮ್ಮ ಮೆಟ್ಟಲಿಗೆ ಇಳಿಸುವುದಕ್ಕೆ. ಅವರು ಅಷ್ಟು ಕಷ್ಟಪಡುತ್ತಿದ್ದರು. ಒಣಗಿದ ಸೋರೆಬುರುಡೆಯನ್ನು ನೀರಿನಲ್ಲಿ ಮುಳುಗಿಸುವಂತೆ ಇದು. ಅದಕ್ಕೆ ಮೇಲಿನಿಂದ ಬಲಾತ್ಕಾರ ಉಪಯೋಗಿಸಬೇಕು, ಇಲ್ಲವೆ ಅದಕ್ಕೆ ಏನಾದರೂ ತೂಕವಾಗಿರುವುದನ್ನು ಕಟ್ಟಬೇಕು. ಶ‍್ರೀರಾಮಕೃಷ್ಣರನ್ನು ನೋಡಿದಾಗ ನರೇಂದ್ರನ ಬುದ್ಧಿಯಾದರೋ ವಿಚಾರದ ರೈಲ್ವೆ ಕಂಬಿಯ ಮೇಲೆ ಹೋಗುತ್ತಿದ್ದ ಎಂಜಿನ್ನಿನಂತೆ ಇತ್ತು. ಕಾರ‍್ಯ ಕಾರಣ ಸಂಬಂಧಗಳನ್ನು ಮೀರಿ ಹೋಗಲು ಅದಕ್ಕೆ ಸಾಧ್ಯವಿರಲಿಲ್ಲ. ಯಾರಾದರೂ ಅವನ ಮನಸ್ಸನ್ನು ರೈಲ್ವೆ ಕಂಬಿಯನ್ನು ಬಿಟ್ಟಂತೆ ಮಾಡಿದರೂ ಅದನ್ನು ಒಪ್ಪುತ್ತಿರಲಿಲ್ಲ. ಅವನು ಅದಕ್ಕೆ ತಯಾರಾಗಿರಲಿಲ್ಲ. ಶ‍್ರೀರಾಮಕೃಷ್ಣರ ಮನಸ್ಸಾದರೋ ವಿಚಾರದ ರೈಲ್ವೆಯ ಲೈನಿನ ಮಿತಿಯನ್ನು ಕಂಡವರು. ಅದನ್ನು ಮೀರಿದ ಅನುಭವವನ್ನು ಅವರು ಗಳಿಸಿದ್ದರು. ಅವರು ಸಾಧಾರಣ ಮನುಷ್ಯರಂತೆ ವಿಚಾರವನ್ನು ಮಾಡಬಲ್ಲವರಾಗಿದ್ದರು. ಆದರೆ ಎಲ್ಲಿಗೆ ಹೋಗಬೇಕಾದರೂ ರೈಲ್ವೆ ಇರಲಾರದು ಎಂಬುದನ್ನು ಅರಿತಿದ್ದವರು. ನೀರಿನ ಮೇಲೆ ದೋಣಿ ಬೇಕಾಗುವುದು. ಆಕಾಶದಲ್ಲಿ ರೆಕ್ಕೆ ಬೇಕಾಗುವುದು. ಅದರಂತೆಯೆ ಸರ್ವತೋಮುಖವಾದ ಅನುಭವ ಶ‍್ರೀರಾಮಕೃಷ್ಣರ ಪಾಲಿಗೆ ಬಂದಿತ್ತು.

ಶ‍್ರೀರಾಮಕೃಷ್ಣರು ನರೇಂದ್ರನನ್ನು ಕಂಡೊಡನೆಯೆ ತತ್‍ಕ್ಷಣ ಗ್ರಹಿಸಿದರು, ಇವನಾರು, ಏತಕ್ಕೆ ಬಂದಿರುವನು, ಎಷ್ಟುಕಾಲ ಇರುವನು ಎಂಬುದನ್ನು. ಆದರೆ ಇನ್ನೂ ಮಾಯೆಯ ಪಂಕದಲ್ಲಿ ಸಿಕ್ಕಿದ್ದ ನರೇಂದ್ರನಿಗಾದರೋ ತಾನು ಕೇವಲ ವಿಶ್ವನಾಥದತ್ತನ ಮಗ ಎಂದು ಮಾತ್ರ ಗೊತ್ತಿತ್ತು. ತನ್ನ ಒಳಗೆ ಸುಪ್ತವಾದ ಆಧ್ಯಾತ್ಮಿಕ ಸಾಧ್ಯತೆಗಳಾವುವೂ ಅವನಿಗೆ ಗೊತ್ತಿರಲಿಲ್ಲ. ತನ್ನ ಸ್ಥಿತಿಯನ್ನೇ ಚೆನ್ನಾಗಿ ತಿಳಿದುಕೊಳ್ಳಲು ಶಕ್ತಿ ಇರಲಿಲ್ಲ. ಇನ್ನು ಶ‍್ರೀರಾಮಕೃಷ್ಣರನ್ನು ಹೇಗೆ ತಿಳಿದುಕೊಳ್ಳಬಲ್ಲ? ಮೊದಲು ಅವರು ಒಬ್ಬ ಹುಚ್ಚರಿರಬೇಕು ಎಂದು ಭಾವಿಸಿದನು. ಅನಂತರ ಇವರಿಗೆ ಆದ ಆಧ್ಯಾತ್ಮಿಕ ಅನುಭವಗಳೆಲ್ಲ ಚಿತ್ತಸ್ವಾಸ್ಥ್ಯವಿಲ್ಲದ ದುರ್ಬಲ ನರವುಳ್ಳವನಿಗೆ ಆಗುವ ಅನುಭವ ಇರಬೇಕು ಎಂದು ಭಾವಿಸಿದನು. ಇವನ ಮೇಲೆ ಶಕ್ತಿಯನ್ನು ಉಪಯೋಗಿಸಿ ಶ‍್ರೀರಾಮಕೃಷ್ಣರು, ಹೆಮ್ಮೆ ಕೊಚ್ಚಿಕೊಳ್ಳುತ್ತಿದ್ದ ನರೇಂದ್ರನ ವ್ಯಕ್ತಿತ್ವವನ್ನು ಕ್ಷಣಾರ್ಧದಲ್ಲಿ ಪುಡಿ ಪುಡಿ ಮಾಡಿದಾಗಲೂ, ರಾಮಕೃಷ್ಣರಲ್ಲಿ ಯಾವುದೋ ಮಂತ್ರಶಕ್ತಿ ಇರಬೇಕು, ಇದನ್ನು ದೈವೀಶಕ್ತಿಯೆಂದು ನಂಬುವುದು ಹೇಗೆ ಎಂದು ತರ್ಕಿಸುತ್ತಿದ್ದನು. ನರೇಂದ್ರ ತನಗಾದ ಪ್ರತಿಯೊಂದು ಸಣ್ಣ ಅನುಭವವನ್ನು ತಿಳಿದುಕೊಳ್ಳಬೇಕಾದರೆ ಅಷ್ಟೊಂದು ಹೋರಾಡಬೇಕಾಯಿತು.

ಶ‍್ರೀರಾಮಕೃಷ್ಣರ ದೃಷ್ಟಿ ನರೇಂದ್ರನ ಮೇಲೆ ಬಿದ್ದೊಡನೆಯೆ ಗ್ರಹಿಸಿದರು ತಮ್ಮ ಸಾಕ್ಷಾತ್ಕಾರದ ಶಕ್ತಿಯನ್ನು ಹರಿಸುವುದಕ್ಕೆ ಇವನೊಳ್ಳೆ ಪಾತ್ರ ಎಂದು. ಲಕ್ಷಾಂತರ ಕ್ಯಾಂಡಲ್ ಪವರ್ ಬಲ್ಬ್ ಕಾಂತಿಯನ್ನು ಕೊಡಬೇಕಾದರೆ ಅದಕ್ಕೊಂದು ದೊಡ್ಡ ಶಕ್ತಿಯೇ ಬೇಕು. ಆ ಶಕ್ತಿಪಾತವನ್ನು ಆ ಬಲ್ಬು ಮಾತ್ರ ಸಹಿಸಬಲ್ಲದು. ಒಂದು ಐದು ಕ್ಯಾಂಡಲ್ ಬಲ್ಬಿನ ಒಳಗೆ ಅದು ಪ್ರವೇಶಿಸಿದರೆ ಬಲ್ಬು ಚೂರು ಚೂರಾಗುವುದು. ನರೇಂದ್ರ ಅಂತಹ ಒಂದು ಮಧ್ಯವರ್ತಿಯಾಗಬೇಕಾದರೆ ಅವನು ಸಿದ್ಧನಾಗಬೇಕು. ಅವನ ವ್ಯಕ್ತಿತ್ವವನ್ನು ರೂಪಿಸುವ ಕೆಲಸಕ್ಕೆ ಶ‍್ರೀರಾಮಕೃಷ್ಣರು ಕೈ ಹಾಕಿದರು.

ಶ‍್ರೀರಾಮಕೃಷ್ಣರು ಶಿಷ್ಯನನ್ನು ತರಬೇತು ಮಾಡುತ್ತಿದ್ದ ರೀತಿ ಅಮೋಘವಾದುದು. ಎಲ್ಲರನ್ನೂ ಒಂದೇ ಎರಕದಲ್ಲಿ ಅವರು ತಯಾರು ಮಾಡುತ್ತಿರಲಿಲ್ಲ. ಒಬ್ಬೊಬ್ಬ ಶಿಷ್ಯ ಒಂದೊಂದು ಸಂಸ್ಕಾರದಿಂದ ಹುಟ್ಟುವನು. ಒಬ್ಬೊಬ್ಬನದು ಒಂದೊಂದು ಬಗೆಯ ವ್ಯಕ್ತಿತ್ವ. ಶಿಷ್ಯರನ್ನು ತರಬೇತು ಮಾಡುತ್ತಿದ್ದಾಗ ಅವರ ವ್ಯಕ್ತಿತ್ವಕ್ಕೆ ಭಂಗ ಬರದಂತೆ ನೋಡಿಕೊಂಡರು. ಈ ಪ್ರಪಂಚದಲ್ಲಿ ವೈವಿಧ್ಯತೆ ಮತ್ತು ಅದರ ಹಿಂದೆ ಒಂದು ಐಕ್ಯತೆ ಇದೆ. ಒಂದೇ ಬಗೆಯ ಹೂವಿಲ್ಲ ಈ ಪ್ರಪಂಚದಲ್ಲಿ. ಹಲವು ಬಗೆಯ ಹೂವುಗಳಿವೆ - ಸಣ್ಣದು, ದೊಡ್ಡದು, ಹಲವು ಬಣ್ಣಗಳದ್ದು, ಹಲವು ಆಕಾರದ್ದು, ಹಲವು ಪರಿಮಳದ್ದು. ಅದನ್ನೆಲ್ಲ ಒಂದೇ ಮಾಡಿಬಿಟ್ಟರೆ ಸೃಷ್ಟಿ ಅಷ್ಟು ನಷ್ಟವಾಗುವುದು. ಅದರಂತೆಯೇ ಮಾನವನ ವ್ಯಕ್ತಿತ್ವ. ಅಲ್ಲಿರುವ ಕಳೆ, ದೌರ್ಬಲ್ಯ, ದೋಷ ಕೀಳಬೇಕು. ಆದರೆ ಆ ವ್ಯಕ್ತಿಯಲ್ಲಿರುವ ಅವನದೇ ಆದ ಒಳ್ಳೆಯ ಸ್ವಭಾವಕ್ಕೆ ನಷ್ಟ ತರಕೂಡದು. ನರೇಂದ್ರ ಬಡಪೆಟ್ಟಿಗೆ ಯಾವುದನ್ನೂ ಒಪ್ಪಿಕೊಳ್ಳುವ ಸ್ವಭಾವದವನಲ್ಲ. ಅದಕ್ಕೆ ಶ‍್ರೀರಾಮಕೃಷ್ಣರು ಸಹನೆಗೆಡಲಿಲ್ಲ. “ಆಗಲಿ, ವಿಚಾರಮಾಡು, ನಿನ್ನ ಮನಸ್ಸಿಗೆ ಸರಿತೋರಿದರೆ ಒಪ್ಪಿಕೋ, ಇಲ್ಲದೇ ಇದ್ದರೆ ಬೇಡ” ಎನ್ನುತ್ತಿದ್ದರು. “ದೇವರ ಸಗುಣ ಸಾಕಾರವನ್ನು ಒಪ್ಪಿಕೊಳ್ಳದೇ ಇದ್ದರೆ ಚಿಂತೆಯಿಲ್ಲ. ನಿನಗೆ ಯಾವುದು ಬೇಕೊ ಅದನ್ನು ತೆಗೆದುಕೊ. ಆದರೆ ನಿನಗೆ ಒಗ್ಗದೇ ಇರುವುದು ಸುಳ್ಳು ಎಂದು ಹೇಳುವುದಕ್ಕೆ ನಿನಗೆ ಏನು ಅಧಿಕಾರವಿದೆ?” ಎಂದು ನರೇಂದ್ರನಿಗೆ ಅರ್ಥವಾಗುವ ಭಾಷೆಯಲ್ಲೆ ಮಾತನಾಡಿ ದಾರಿಗೆ ತರುತ್ತಿದ್ದರು. ಸತ್ಯದ ದೇವಸ್ಥಾನಕ್ಕೆ ಅನಂತ ಪಥಗಳು. ಪಥಗಳೆಷ್ಟೋ ಮತಗಳಷ್ಟು. ಯಾರೂ ತಮ್ಮದೇ ಸರಿ ಮತ್ತೊಬ್ಬರದು ಸರಿಯಲ್ಲ ಎನ್ನದಿರಲಿ. ನರೇಂದ್ರನ ವಿಚಾರಶಕ್ತಿಯನ್ನು ಕುಂಠಿತ ಮಾಡುವ ಬದಲು ಅವನನ್ನು ಇತರರೊಡನೆ ವಾದಕ್ಕೆ ಬಿಟ್ಟು ಅದನ್ನು ನೋಡಿ ಆನಂದಿಸುತ್ತಿದ್ದರು. ಮೆಟ್ಟಲು ಮೆಟ್ಟಲಾಗಿ ಶ‍್ರೀರಾಮಕೃಷ್ಣರು ನರೇಂದ್ರನನ್ನು ಆಧ್ಯಾತ್ಮಿಕ ಜೀವನದಲ್ಲಿ ನಡೆಸಿಕೊಂಡು ಹೋದರು.

ಶಿಷ್ಯನ ಮೇಲೆ ಗುರು ಕೆಲಸಮಾಡಬೇಕಾದರೆ, ಶಿಷ್ಯ ಅವನ ವಶಕ್ಕೆ ಬರಬೇಕು. ಆಗ ಮಾತ್ರ ಸಾಧ್ಯ. ಶ‍್ರೀರಾಮಕೃಷ್ಣರು ಹಾಗೆ ಶಿಷ್ಯನನ್ನು ವಶಮಾಡಿಕೊಳ್ಳುವುದಕ್ಕೆ ತಮ್ಮ ಆಧ್ಯಾತ್ಮಿಕ ಭೀಮ ಅನುಭವವನ್ನು ಅವನಿಗೆ ತೋರಿ ವಶಮಾಡಿಕೊಳ್ಳುತ್ತಿರಲಿಲ್ಲ. ಆ ಅನುಭವವನ್ನು ಕೊಟ್ಟರೂ ಎಷ್ಟು ಜನ ಅರ್ಥಮಾಡಿಕೊಳ್ಳಬಲ್ಲರು? ನರೇಂದ್ರನಂಥವನಿಗೆ ಅದನ್ನು ಕೊಟ್ಟಾಗಲೇ ಅವನು ಕಕ್ಕಾಬಿಕ್ಕಿಯಾಗಿ ಹೋದ. ಇತರರ ಪಾಡೇನು? ಶ‍್ರೀರಾಮಕೃಷ್ಣರು ಶಿಷ್ಯನನ್ನು ಒಲಿಸಿಕೊಳ್ಳುವುದಕ್ಕೆ ಉಪಯೋಗಿಸುತ್ತಿದ್ದ ಒಂದು ಮಹಾಸ್ತ್ರವೇ ಪ್ರೀತಿ. ಶಿಷ್ಯರು ಇನ್ನು ಯಾವುದಕ್ಕೆ ಬಗ್ಗದೇ ಇದ್ದರೂ, ಇನ್ನು ಯಾವುದನ್ನು ಅರ್ಥಮಾಡಿಕೊಳ್ಳದೇ ಇದ್ದರೂ, ಪ್ರೀತಿಯನ್ನು ಅರ್ಥಮಾಡಿಕೊಳ್ಳುತ್ತಿದ್ದರು. ಶ‍್ರೀರಾಮಕೃಷ್ಣರು ತಮ್ಮ ಶಿಷ್ಯರನ್ನು ಪ್ರೀತಿಸುತ್ತಿದ್ದ ರೀತಿ ಒಂದು ಅಪೂರ್ವ ಸಂಗತಿ ಈ ಪ್ರಪಂಚದಲ್ಲಿ. ನಾವೆಲ್ಲ ಪ್ರೀತಿಸುತ್ತೇವೆ. ಆ ಪ್ರೀತಿ ಎಷ್ಟು ಕೃತಕ, ಮತ್ತು ಎಲ್ಲಿ ಎಷ್ಟು ಅಂಶ ಪ್ರೀತಿ ಇದೆ! ಆದರೆ ಶ‍್ರೀರಾಮಕೃಷ್ಣರಾದರೋ ಅದ್ಭುತವಾಗಿ ಭಕ್ತರನ್ನು ಪ್ರೀತಿಸಬಲ್ಲವರಾಗಿದ್ದರು. ಯಾರಿಗೂ ಅದರಿಂದ ತಪ್ಪಿಸಿಕೊಂಡುಹೋಗಲು ಆಗಲಿಲ್ಲ. ನರೇಂದ್ರ ಶ‍್ರೀರಾಮಕೃಷ್ಣರನ್ನು ಹುಚ್ಚರಿರಬಹುದು, ಒಬ್ಬ ಮಂತ್ರವಾದಿಗಳಿರಬಹುದು ಎಂದು ಮೊದಲು ಭಾವಿಸಿದರೂ ಆ ಪ್ರೀತಿಯಲ್ಲಿ ಅನುಮಾನವಿರಲಿಲ್ಲ. ಅವರನ್ನು ಅರ್ಥಮಾಡಿಕೊಳ್ಳದೇ ಇದ್ದರೂ ಅವರ ಬಳಿಗೆ ಬರುವಂತೆ ಮಾಡುತ್ತಿದ್ದುದು ಆ ಪ್ರೀತಿ. ಶ‍್ರೀರಾಮಕೃಷ್ಣರು ಶಿಷ್ಯರನ್ನೆಲ್ಲ ಪ್ರೀತಿಸಿದರು. ಆದರೆ ನರೇಂದ್ರನಿಗೇ ಒಂದು ಪ್ರತ್ಯೇಕ ಸ್ಥಾನವಿತ್ತು. ಅಲ್ಲಿ ಕುಳಿತುಕೊಳ್ಳುವುದಕ್ಕೆ ಅವನೊಬ್ಬನೇ ಯೋಗ್ಯ. ಶ‍್ರೀರಾಮಕೃಷ್ಣರಿಗೆ ನರೇಂದ್ರನ ಮೇಲೆ ಇದ್ದ ಪ್ರೀತಿಯನ್ನು ವಿವರಿಸಬೇಕಾದರೆ, ತಾಯಿಗೆ ಮಗುವಿನ ಮೇಲೆ ಇರುವ ಪ್ರೀತಿ, ಗಂಡನಿಗೆ ಹೆಂಡತಿಯ ಮೇಲಿರುವ ಪ್ರೀತಿ ಇವುಗಳೆಲ್ಲ ಸಾಲದಾಗುವುದು. ಇದಕ್ಕೆ ನಾವು ಬೇರೊಂದು ಪದವನ್ನೇ ಸೃಷ್ಟಿಸಬೇಕಾಗುವುದು. ನರೇಂದ್ರನೇ ಅನಂತರ ಹೇಳುತ್ತಿದ್ದ, “ನಾನು ಅವರ ಪ್ರೇಮದ ದಾಸ” ಎಂದು. ಇನ್ನಾವುದರಿಂದಲೂ ಅಲ್ಲ, ಪ್ರೇಮದಿಂದ ಅವರು ತಮ್ಮ ಶಿಷ್ಯರನ್ನು ಗುಲಾಮರನ್ನಾಗಿ ಮಾಡಿಕೊಂಡರು. ಗೃಹಸ್ಥರ ಮನೆಯಲ್ಲಿ ಬೆಳೆಯುವ ಎಲ್ಲರಿಗೂ ಪ್ರೀತಿಯ ಪರಿಚಯವಿದೆ. ಆದರೆ ಶ‍್ರೀರಾಮಕೃಷ್ಣರು ವ್ಯಕ್ತಪಡಿಸುತ್ತಿದ್ದ ಪ್ರೀತಿ ಅಲೌಕಿಕವಾದುದು, ಅಸಾಧಾರಣವಾದುದು, ಉಂಡವರೇ ಬಲ್ಲರು ಅದರ ಸವಿಯನ್ನು.

ನರೇಂದ್ರನಾಥ ದಕ್ಷಿಣೇಶ್ವರದಲ್ಲಿ ಶ‍್ರೀರಾಮಕೃಷ್ಣರನ್ನು ಪರಿಚಯ ಮಾಡಿಕೊಂಡ ಮೇಲೆ ಸುಮಾರು ಐದು ವರುಷಗಳ ಕಾಲ ಅವರನ್ನು ನೋಡುವುದಕ್ಕೆ ಹೋಗುತ್ತಿದ್ದ. ವಾರಕ್ಕೆ ಸಾಧಾರಣವಾಗಿ ಎರಡು ವೇಳೆ ಹೋಗುತ್ತಿದ್ದ. ಹೋದವನು ಕೆಲವು ವೇಳೆ ದಕ್ಷಿಣೇಶ್ವರದಲ್ಲಿಯೇ ತಂಗುತ್ತಿದ್ದ. ಏನಾದರೂ ನರೇಂದ್ರ ಶ‍್ರೀರಾಮಕೃಷ್ಣರನ್ನು ನೋಡಲು ಬರುವುದು ತಡವಾಯಿತು ಎಂದರೆ ಶ‍್ರೀರಾಮಕೃಷ್ಣರು ತಳಮಳಗೊಳ್ಳುತ್ತಿದ್ದರು. ಬರುವ ಭಕ್ತರ ಕೈಯಲ್ಲೆಲ್ಲ ಹೇಳಿಕಳುಹಿಸುತ್ತಿದ್ದರು. ಬರದೇ ಇದ್ದರೆ ಕೊನೆಗೆ ತಾವೇ ಕಲ್ಕತ್ತೆಗೆ ಅವನನ್ನು ನೋಡಿಕೊಂಡು ಬರಲು ಹೋಗುತ್ತಿದ್ದರು.

ಒಂದು ಸಲ ನರೇಂದ್ರ ಕೆಲವು ದಿನಗಳವರೆಗೆ ಶ‍್ರೀರಾಮಕೃಷ್ಣರನ್ನು ನೋಡಲು ಬರಲಿಲ್ಲ. ಒಂದು ದಿನ ರಾತ್ರಿ ರಾಮದಯಾಳು ಮತ್ತು ಬಾಬುರಾಮ ಎಂಬ ಇಬ್ಬರು ಭಕ್ತರು ದಕ್ಷಿಣೇಶ್ವರಕ್ಕೆ ಬಂದರು. ಅವರ ಹತ್ತಿರ ಮಾತನಾಡುವಾಗ, “ನೀವು ನರೇಂದ್ರನಿಗೆ ಒಂದು ಸಲ ದಕ್ಷಿಣೇಶ್ವರಕ್ಕೆ ಬಂದು ಹೋಗುವಂತೆ ಹೇಳಿ” ಎಂದರು ಶ‍್ರೀರಾಮಕೃಷ್ಣರು. ಅವರಿಬ್ಬರು ಅಂದಿನ ರಾತ್ರಿಯನ್ನು ದಕ್ಷಿಣೇಶ್ವರದಲ್ಲಿಯೇ ಕಳೆದರು. ಶ‍್ರೀರಾಮಕೃಷ್ಣರು ಸುಮಾರು ಹನ್ನೊಂದು ಗಂಟೆಯ ಹೊತ್ತಿಗೆ ಅವರ ಹತ್ತಿರ ಬಂದು ಇನ್ನೊಂದು ಸಲ, ನರೇಂದ್ರನಿಗೆ ತಪ್ಪದೇ ಹೇಳಿ ಎಂದರು. “ಅವನನ್ನು ನೋಡದೆ ಇದ್ದರೆ ನನ್ನ ಹೃದಯವನ್ನು ಯಾರೋ ಹಿಂಡುತ್ತಿರುವಂತೆ ಆಗುತ್ತದೆ” ಎಂದು ತಮ್ಮ ಹೆಗಲ ಮೇಲೆ ಇದ್ದ ಟವಲನ್ನು ಕೈಯಲ್ಲಿ ಹಿಂಡಿತೋರಿದರು. ಆ ಭಕ್ತರಿಗೆ ಆಶ್ಚರ‍್ಯ! ಶ‍್ರೀರಾಮಕೃಷ್ಣರು ನರೇಂದ್ರಗಾಗಿ ಹೀಗೆ ತಳಮಳಗೊಳ್ಳುತ್ತಿದ್ದರೂ ನರೇಂದ್ರ ಇದನ್ನು ಲೆಕ್ಕಿಸದೆ ಇರುವುದು ಅವರಿಗೆ ಅಚ್ಚರಿಯನ್ನುಂಟುಮಾಡಿತು.

ಮತ್ತೊಂದು ಸಲ ವೈಕುಂಠ ಸಂನ್ಯಾಲ್ ಎಂಬ ಭಕ್ತ ದಕ್ಷಿಣೇಶ್ವರಕ್ಕೆ ಬಂದನು. ಅವನ ಎದುರಿಗೆ ನರೇಂದ್ರನನ್ನು ಬೇಕಾದಷ್ಟು ಕೊಂಡಾಡಿದರು. ಆದರೆ ನರೇಂದ್ರ ಹಲವು ದಿನಗಳಿಂದ ಅವರನ್ನು ನೋಡಲು ಬರದೆ ಇದ್ದುದಕ್ಕಾಗಿ ವಿಹ್ವಲಗೊಂಡಿದ್ದರು. ತಮ್ಮ ಕೋಣೆಯ ಹೊರಗೆ ಹೋಗಿ ವರಾಂಡದಲ್ಲಿ “ತಾಯಿ, ನಾನು ನರೇಂದ್ರನನ್ನು ನೋಡದೆ ಇರಲಾರೆ” ಎಂದು ಕಂಬನಿದುಂಬಿ ಪ್ರಾರ್ಥಿಸಿದರು. ಕೋಣೆಗೆ ಬಂದ ಮೇಲೆ ಭಕ್ತರೆದುರಿಗೆ “ನಾನು ನರೇಂದ್ರನಿಗಾಗಿ ಅಷ್ಟೊಂದು ಅತ್ತಿರುವೆ. ಆದರೂ ನಿರ್ದಯನಾದ ಅವನು ಬರಲೇ ಇಲ್ಲ. ಅವನನ್ನು ನೋಡದೇ ಇದ್ದರೆ ಯಾರೋ ನನ್ನ ಹೃದಯವನ್ನು ಹಿಂಡುತ್ತಿರುವಂತೆ ತೋರುವುದು” ಎಂದರು. ಸ್ವಲ್ಪ ಹೊತ್ತಾದ ಮೇಲೆ ಶ‍್ರೀರಾಮಕೃಷ್ಣರು “ಮುದುಕನಾದ ನಾನು ಆ ಹುಡುಗನಿಗಾಗಿ ಎಷ್ಟು ತಳಮಳಗೊಳ್ಳುತ್ತಿರುವೆ. ಜನ ನನ್ನನ್ನು ಏನೆಂದು ಭಾವಿಸಿಯಾರು? ನೀವೆಲ್ಲ ನನ್ನವರು. ನನ್ನ ಮನಸ್ಸಿನಲ್ಲಿರುವುದನ್ನು ನಿಮಗೆ ಹೇಳಿಕೊಳ್ಳಲು ನನಗೆ ಸಂಕೋಚವಿಲ್ಲ. ಆದರೆ ಇತರರು ಇದನ್ನು ಹೇಗೆ ನೋಡುತ್ತಾರೆ!” ಎಂದು ಹೇಳಿಕೊಂಡರು. ಆಗ ಇವರಿಗೆ ಸಮಾಧಾನ ಮಾಡಬೇಕಾಗಿ ಬಂತು. “ಭಾಗವತದಲ್ಲಿ ಭಗವಂತನ ಸಾಕ್ಷಾತ್ಕಾರವಾದ ಮೇಲೆ ಪ್ರಪಂಚದಲ್ಲಿರಬೇಕಾಗಿ ಬಂದರೆ ಅಂತಹ ಪರಮಹಂಸರು ತಮ್ಮ ಸುತ್ತಲೂ ಕೆಲವು ಪರಿಶುದ್ಧರಾದ ಹುಡುಗರನ್ನು ಇಟ್ಟುಕೊಂಡಿರುವರು. ಏಕೆಂದರೆ ಅಂತಹ ಪರಿಶುದ್ಧಾತ್ಮರು ಇರದೇ ಇದ್ದರೆ ಅವರು ಪ್ರಪಂಚದಲ್ಲಿ ಬಾಳಲು ಸಾಧ್ಯವಿಲ್ಲ. ಅದಕ್ಕಾಗಿ ನೀವು ಅವನನ್ನು ಅಷ್ಟು ಹಚ್ಚಿಕೊಂಡಿರುವುದು” ಎಂದು ವೈಕುಂಠ ಸಂನ್ಯಾಲ್ ಹೇಳಿದ. ಶ‍್ರೀರಾಮಕೃಷ್ಣರದು ಮಗುವಿನಂತಹ ಸರಳ ಸ್ವಭಾವ. “ಹೌದೇನೋ ಇದ್ದರೂ ಇರಬಹುದು” ಎಂದರು.

ಮತ್ತೊಮ್ಮೆ ನರೇಂದ್ರ ಬಹಳ ಕಾಲದವರೆಗೆ ಬರದೇ ಇದ್ದಾಗ ಶ‍್ರೀರಾಮಕೃಷ್ಣರೇ ಕಲ್ಕತ್ತೆಗೆ ಗಾಡಿಮಾಡಿಕೊಂಡು ಹೋದರು. ಅವನು ಸಂಜೆ ಪ್ರಾರ್ಥನೆಗಾಗಿ ಬ್ರಹ್ಮಸಮಾಜದಲ್ಲಿರಬಹುದೆಂದು ನೇರ ಅಲ್ಲಿಗೆ ಹೋದರು. ಎಲ್ಲರೂ ಬ್ರಹ್ಮಸಮಾಜದ ಕಾರ್ಯಕ್ರಮದಲ್ಲಿ ನಿರತರಾಗಿದ್ದರು. ಶ‍್ರೀರಾಮಕೃಷ್ಣರು ನರೇಂದ್ರನನ್ನು ಕಂಡು ವೇದಿಕೆಯ ಮೇಲೆ ಹೋಗಿ ನಿಂತರು. ಅಲ್ಲಿಯೇ ಭಾವಸಮಾಧಿಯಲ್ಲಿ ಮಗ್ನರಾದರು. ಅಲ್ಲಿ ನೆರೆದಿದ್ದವರು ಇದನ್ನು ಸರಿಯಾದ ದೃಷ್ಟಿಯಲ್ಲಿ ತೆಗೆದುಕೊಳ್ಳಲಿಲ್ಲ. ಕೇಶವ ಮತ್ತು ವಿಜಯರು ಶ‍್ರೀರಾಮಕೃಷ್ಣರನ್ನು ಸಂಪರ್ಕಿಸಿದುದರಿಂದಲೇ ಇತ್ತೀಚೆಗೆ ಬ್ರಹ್ಮಸಮಾಜದಲ್ಲಿ ಒಡಕು ಉಂಟಾಗಿದೆ ಎಂಬ ಭಾವನೆ ಅವರಿಗಿತ್ತು. ಅವರು ಇಲ್ಲಿಗೆ ಬಂದು ಇನ್ನೂ ಏನೇನನ್ನು ಮಾಡುತ್ತಾರೊ ಎಂಬ ಅಂಜಿಕೆಯಿಂದ ಶ‍್ರೀರಾಮಕೃಷ್ಣರು ಬಂದಮೇಲೆ ಪ್ರಾರ್ಥನಾ ಮಂದಿರದಲ್ಲಿದ್ದ ದೀಪವನ್ನೆಲ್ಲ ಆರಿಸಿಬಿಟ್ಟರು. ದೊಡ್ಡ ಒಂದು ಆಂದೋಳನವೆದ್ದಿತು ಅಲ್ಲಿ. ಆಗ ನರೇಂದ್ರ ಶ‍್ರೀರಾಮಕೃಷ್ಣರ ಹತ್ತಿರ ಹೋಗಿ ಅವರನ್ನು ಹಿಂದಿನ ಬಾಗಿಲಿನಿಂದ ಹೊರಕ್ಕೆ ಕರೆದುಕೊಂಡು ಬಂದು ಒಂದು ಗಾಡಿಯನ್ನು ಮಾಡಿಕೊಂಡು ದಕ್ಷಿಣೇಶ್ವರದಲ್ಲಿ ಅವರನ್ನು ಬಿಟ್ಟನು. ಆಗ ಹೇಳಿದ: “ನೀವು ನನ್ನನ್ನು ಇಷ್ಟೊಂದು ಮನಸ್ಸಿಗೆ ಹಚ್ಚಿಕೊಂಡರೆ ಭರತನಂತೆ ಆಗುತ್ತೀರಿ. ಅವನು ಒಂದು ಜಿಂಕೆಯ ಮೇಲಿನ ಪ್ರೇಮಕ್ಕೆ ಸಿಕ್ಕಿಹಾಕಿಕೊಂಡು ಅಂತ್ಯದಲ್ಲಿ ಅದನ್ನು ಚಿಂತಿಸುತ್ತ ಮುಂದಿನ ಜನ್ಮದಲ್ಲಿ ಒಂದು ಜಿಂಕೆಯೇ ಆಗಿ ಹುಟ್ಟಿದ.” ಶ‍್ರೀರಾಮಕೃಷ್ಣರು ಇದನ್ನು ಕೇಳಿದೊಡನೆಯೇ ನರೇಂದ್ರ ಹೇಳಿದುದು ಅವರಿಗೆ ನಿಜವಾಗಿ ತೋರಿತು. “ಹೌದು, ನೀನು ಹೇಳುವುದು ನಿಜ, ಹಾಗಾದರೆ ನನ್ನ ಗತಿ ಏನು? ನಿನ್ನನ್ನು ಅಗಲಿ ಇರುವುದಕ್ಕೆ ಆಗುವುದಿಲ್ಲ” ಎಂದು ವ್ಯಾಕುಲಚಿತ್ತದಿಂದ ಗರ್ಭಗುಡಿಗೆ ಹೋಗಿ ದೇವಿಯಲ್ಲಿ ತಮ್ಮ ದುಗುಡವನ್ನು ವ್ಯಕ್ತಪಡಿಸಿದರು. ಅಲ್ಲಿಂದ ಬಂದು ಸಮಾಧಾನಚಿತ್ತರಾಗಿ ಹಸನ್ಮುಖರಾಗಿ ನಗುತ್ತ ಹೇಳಿದರು: “ತುಂಟ, ಇನ್ನು ಮೇಲೆ ನಿನ್ನ ಮಾತನ್ನು ನಾನು ಕೇಳುವುದಿಲ್ಲ. ನನ್ನ ತಾಯಿ ಹೇಳುತ್ತಾಳೆ, ನಾನು ನಿನ್ನನ್ನು ಪ್ರೀತಿಸುವುದಕ್ಕೆ ಕಾರಣ ನಿನ್ನಲ್ಲಿ ಭಗವಂತ ಕಾಣುವುದರಿಂದ. ಎಂದು ನನಗೆ ಅದು ನಿನ್ನಲ್ಲಿ ಕಾಣುವುದಿಲ್ಲವೋ ಅಂದಿನಿಂದ ನಿನ್ನ ನೆಳಲನ್ನು ಸಹಿಸುವುದಕ್ಕೂ ಆಗುವುದಿಲ್ಲ.”

ಮತ್ತೊಮ್ಮೆ ನರೇಂದ್ರನ ಪರೀಕ್ಷೆಯ ಸಮಯ. ಆ ಸಮಯದಲ್ಲಿ ಲಾಟು ಎಂಬ ಸೇವಕ ಭಕ್ತನನ್ನು ಕರೆದುಕೊಂಡು ನರೇಂದ್ರನನ್ನು ನೋಡುವುದಕ್ಕೆ ಕೈಯಲ್ಲಿ ತಿಂಡಿ ತೆಗೆದುಕೊಂಡು ಒಂದು ಗಾಡಿ ಮಾಡಿಕೊಂಡು ಕಲ್ಕತ್ತೆಗೆ ಹೋದರು. ನರೇಂದ್ರ ಒಂದು ಮನೆಯ ಅಟ್ಟದ ಮೇಲೆ ಕುಳಿತುಕೊಂಡು ಓದುತ್ತಿದ್ದ. ಅದಕ್ಕೆ ಹೊರಗಡೆಯ ಏಣಿಯಿಂದ ಹತ್ತಿಹೋಗಬೇಕಾಗಿತ್ತು. ಮೊದಲು ಲಾಟುವನ್ನು ಮೇಲಕ್ಕೆ ಕಳುಹಿಸಿದರು, ನರೇಂದ್ರ ಇರುವನೆ ನೋಡು ಎಂದು. ಅವನು ಮೇಲೆ ಹೋಗಿ ನರೇಂದ್ರ ಇರುವನು ಎಂದು ಹೇಳಿದ ಮೇಲೆ ಶ‍್ರೀರಾಮಕೃಷ್ಣರು ನರೇಂದ್ರನನ್ನು ನೋಡಲು ಏಣಿ ಹತ್ತಿ ಹೋದರು. ನರೇಂದ್ರನನ್ನು ಕಂಡ ಕೂಡಲೇ ಅವನಿಗೆ ತಿಂಡಿಕೊಟ್ಟು, ಅವನಿಂದ ಒಂದು ಹಾಡನ್ನು ಹೇಳಿಸಿದರು. ಕೇಳಿದ ತತ್‍ಕ್ಷಣವೇ ಸಮಾಧಿಗೆ ಹೋದರು. ಅನಂತರ ದಕ್ಷಿಣೇಶ್ವರಕ್ಕೆ ಹಿಂತಿರುಗಿ ಬಂದರು.

ನರೇಂದ್ರನನ್ನು ಶ‍್ರೀರಾಮಕೃಷ್ಣರು ಅಷ್ಟು ಪ್ರೀತಿಸುತ್ತಿದ್ದರು. ಯಾರಾದರೂ ಅವನ ವಿಷಯದಲ್ಲಿ ಚಾಡಿ ಹೇಳಿದರೆ ಅದನ್ನು ಕೇಳುತ್ತಿರಲಿಲ್ಲ. “ಈಶ್ವರನಿಂದೆಯನ್ನು ನೀನು ಮಾಡುತ್ತಿರುವೆ. ತಾಯಿ ನನಗೆ ತೋರಿರುವಳು, ಅವನು ಎಂದಿಗೂ ತಪ್ಪನ್ನು ಮಾಡಲು ಸಾಧ್ಯವಿಲ್ಲ ಎಂದು. ಇನ್ನು ಮೇಲೆ ಅವನ ವಿಷಯದಲ್ಲಿ ಚಾಡಿ ಹೇಳುವಂತೆ ಇದ್ದರೆ ಇಲ್ಲಿಗೆ ಬರಬೇಡ” ಎಂದು ಹೇಳುತ್ತಿದ್ದರು. ಅಂತಹ ನಂಬಿಕೆ ನರೇಂದ್ರನ ಮೇಲೆ! ನರೇಂದ್ರ ಅನಂತರ ಹೇಳುತ್ತಿದ್ದ: “ನನ್ನ ವಿಷಯದಲ್ಲಿ ನೆಚ್ಚಿನ ಸ್ನೇಹಿತರು, ನೆಂಟರಿಷ್ಟರು ಅನುಮಾನ ಪಡುತ್ತಿದ್ದರೂ ಶ‍್ರೀರಾಮಕೃಷ್ಣರು ಮಾತ್ರ ಒಂದು ದಿನವೂ ಅನುಮಾನಿಸಿರಲಿಲ್ಲ. ಈ ಪ್ರಪಂಚದಲ್ಲಿ ಭರವಸೆಯ ಏಕಮಾತ್ರ ದ್ವೀಪದಂತೆ ಇದ್ದರು ಅವರು.”

ಮೊದಲಿನಿಂದಲೂ ಶ‍್ರೀರಾಮಕೃಷ್ಣರು ನರೇಂದ್ರನನ್ನು ಪ್ರತ್ಯೇಕವಾದ ದೃಷ್ಟಿಯಿಂದ ನೋಡುತ್ತಿದ್ದರು. ಅವನನ್ನು ನಿತ್ಯಸಿದ್ಧ, ಧ್ಯಾನಸಿದ್ಧ ಎಂದು ಕರೆಯುತ್ತಿದ್ದರು. ಅವನು ಮಾಡುವ ಸಾಧನೆ ತನಗಾಗಿ ಅಲ್ಲ, ಇತ್ತರರ ಮೇಲ್ಪಂಕ್ತಿಗಾಗಿ ಎಂದು ಹೇಳುತ್ತಿದ್ದರು. ಅವನಲ್ಲಿ ಜ್ಞಾನಾಗ್ನಿ ಕಾಡ್ಗಿಚ್ಚಿನಂತೆ ಉರಿಯುತ್ತಿದೆ, ಅದು ಎಲ್ಲವನ್ನೂ ಕ್ಷಣದಲ್ಲಿ ಭಸ್ಮೀಭೂತ ಮಾಡಿಬಿಡುವುದು ಎನ್ನುತ್ತಿದ್ದರು. ಶ‍್ರೀರಾಮಕೃಷ್ಣರಿಗೆ ಹಲವು ವೃತ್ತಿಗಳಲ್ಲಿದ್ದ ಅನೇಕ ಶ‍್ರೀಮಂತ ಭಕ್ತರು ತಿನ್ನಲು ಹಲವು ತಿಂಡಿಗಳನ್ನು ತರುತ್ತಿದ್ದರು. ಅವರಿಗೆ ಯಾರು ಯಾರೋ ತಂದು ಕೊಟ್ಟಿದ್ದನ್ನು ತಿನ್ನುವುದಕ್ಕೆ ಆಗುತ್ತಿರಲಿಲ್ಲ. ತಮಗೆ ಯಾವುದನ್ನು ಸೇವಿಸುವುದಕ್ಕೆ ಆಗುತ್ತಿರಲಿಲ್ಲವೋ ಅದನ್ನು ನರೇಂದ್ರನಿಗೆ ಕೊಡುತ್ತಿದ್ದರು. ಅವನು ಇಲ್ಲದೇ ಇದ್ದರೆ ಅವನ ಮನೆಗೆ ಕಳುಹಿಸುತ್ತಿದ್ದರು. ಇದರಿಂದ ಅವನ ಮನಸ್ಸಿನ ಮೇಲೆ ಯಾವ ಪರಿಣಾಮವು ಆಗುವುದಿಲ್ಲ, ದೊಡ್ಡ ಕಾಡ್ಗಿಚ್ಚಿಗೆ ಬಾಳೆಗಿಡವನ್ನು ಹಾಕಿದಂತೆ ಇದು ಎನ್ನುತ್ತಿದ್ದರು.

ಆಹಾರದ ವಿಷಯದಲ್ಲಿ ಶ‍್ರೀರಾಮಕೃಷ್ಣರು ಅನುಸರಿಸುತ್ತಿದ್ದ ರೀತಿ ಮೊದಮೊದಲು ನರೇಂದ್ರನಿಗೆ ಅರ್ಥವಾಗುತ್ತಿರಲಿಲ್ಲ. ಅವರು ಎಲ್ಲರ ಕೈಯಿಂದಲೂ ಆಹಾರವನ್ನು ಸ್ವೀಕರಿಸುತ್ತಿರಲಿಲ್ಲ. ನರೇಂದ್ರ ಅನಂತರ ಅವರು ಹಾಗೆ ತೆಗೆದುಕೊಳ್ಳದೇ ಇದ್ದವರ ಜೀವನವನ್ನು ಚೆನ್ನಾಗಿ ಪರಿಶೀಲಿಸಿದಾಗ ಅವರಲ್ಲಿ ಯಾವುದೋ ಒಂದು ಮಹಾ ದೋಷ ಕಾಣುತ್ತಿತ್ತು. ಶ‍್ರೀರಾಮಕೃಷ್ಣರ ಬುದ್ಧಿ ಕಣ್ಣಿಗೆ ಕಾಣುವ ವರ್ತಮಾನ ಘಟನೆಗಳನ್ನು ಮಾತ್ರವಲ್ಲದೆ ಜನರ ಹೃದಯದಲ್ಲಿ ಏನಿದೆ, ಹಿಂದೆ ಅವರು ಏನು ಮಾಡಿದ್ದರು ಎಂಬುದನ್ನೆಲ್ಲ ತಿಳಿದುಕೊಳ್ಳುತ್ತಿತ್ತು. ಕೆಲವು ವೇಳೆ ನರೇಂದ್ರ ಅಭಕ್ಷ್ಯ ಭೋಜನವನ್ನು ಮಾಡಿ ಶ‍್ರೀರಾಮಕೃಷ್ಣರಿಗೆ ಇವತ್ತು ನಾನು ಇಂಥಾದ್ದನ್ನು ತಿಂದಿರುವೆ ಎನ್ನುತ್ತಿದ್ದ. ಅವರು ಅದರಿಂದ ನಿನಗೆ ಏನೂ ಆಗುವುದಿಲ್ಲ ಎಂದು ಹೇಳಿಬಿಡುತ್ತಿದ್ದರು.

ನರೇಂದ್ರನನ್ನು ಶ‍್ರೀರಾಮಕೃಷ್ಣರು ಅವನೆದುರಿಗೇ ಹೊಗಳುತ್ತಿದ್ದರು, ಇತರರೆಲ್ಲರ ಕಿವಿಗೆ ಬೀಳುವಂತೆಯೇ ಹೇಳುತ್ತಿದ್ದರು. ಆದರೆ ನರೇಂದ್ರನಿಗೆ ಆ ಹೊಗಳಿಕೆಯನ್ನು ಕೇಳಿದಾಗ ತಲೆ ತಿರುಗಿ ಹೋಗಲಿಲ್ಲ. ಅದರ ಬದಲು ಶ‍್ರೀರಾಮಕೃಷ್ಣರನ್ನೇ ಟೀಕಿಸಲು ಯತ್ನಿಸಿದ. ಒಂದು ದಿನ ಶ‍್ರೀರಾಮಕೃಷ್ಣರ ಕೋಣೆಯಲ್ಲಿ ಬ್ರಹ್ಮಸಮಾಜದ ಹೆಸರುವಾಸಿಯಾದ ವಿಜಯಕೃಷ್ಣ ಗೋಸ್ವಾಮಿ ಮತ್ತು ಇತರ ಪ್ರಮುಖ ವ್ಯಕ್ತಿಗಳು ಕುಳಿತಿದ್ದರು. ಶ‍್ರೀರಾಮಕೃಷ್ಣರು ಇನ್ನೂ ವಿದ್ಯಾರ್ಥಿಯ ದೆಸೆಯಲ್ಲಿದ್ದ ನರೇಂದ್ರನನ್ನು ನೋಡಿ ಅನಂತರ ಇತರರ ಕಡೆ ತಿರುಗಿ ಹೇಳಿದರು: “ಕೇಶವನಲ್ಲಿ ಅವನನ್ನು ಪ್ರಖ್ಯಾತಗೊಳಿಸಿದ ಒಂದು ಅಂಶವಿದ್ದರೆ ನರೇಂದ್ರನಲ್ಲಿ ಅಂತಹ ಹದಿನೆಂಟು ಅಂಶಗಳಿವೆ. ಕೇಶವ ಮತ್ತು ವಿಜಯರಲ್ಲಿ ಜ್ಞಾನ ಸೊಡರಿನಂತೆ ಉರಿಯುತ್ತಿದ್ದರೆ ನರೇಂದ್ರನಲ್ಲಿ ಜ್ಞಾನಭಾಸ್ಕರನೇ ಇರುವನು. ಉಳಿದವರೆಲ್ಲ, ಎಂಟು ಹತ್ತು ಇಪ್ಪತ್ತು ದಳದ ಕಮಲಗಳಾದರೆ ನರೇಂದ್ರ ಸಹಸ್ರದಳದ ಪದ್ಮ.” ನರೇಂದ್ರನಿಗೆ ಈ ಹೊಗಳಿಕೆಯನ್ನು ಕೇಳಿ ನಾಚಿಕೆಯಾಯಿತು. “ನೀವು ಏತಕ್ಕೆ ನನ್ನ ವಿಷಯದಲ್ಲಿ ಹೀಗೆ ಹೇಳುತ್ತೀರಿ? ಜನ ನಿಮ್ಮನ್ನು ಹುಚ್ಚರು ಎಂದಾರು. ಜಗದ್ವಿಖ್ಯಾತರಾದ ಕೇಶವಚಂದ್ರಸೇನರು, ಸಂತರಾದ ವಿಜಯಕೃಷ್ಣ ಗೋಸ್ವಾಮಿ ಇವರೊಂದಿಗೆ ಕೇವಲ ವಿದ್ಯಾರ್ಥಿ ದೆಸೆಯಲ್ಲಿರುವ ನನ್ನನ್ನು ಏತಕ್ಕೆ ಹೋಲಿಸುತ್ತೀರಿ? ದಯವಿಟ್ಟು ಮತ್ತೊಮ್ಮೆ ಹಾಗೆ ಮಾಡಬೇಡಿ” ಎಂದು ನರೇಂದ್ರ ಕೇಳಿಕೊಂಡ. ಅದಕ್ಕೆ ಶ‍್ರೀರಾಮಕೃಷ್ಣರು ಹೀಗೆ ಹೇಳಿದರು: “ಹೇಳದೆ ವಿಧಿಯಿಲ್ಲ. ಅದು ಕೇವಲ ನನ್ನ ಮಾತುಗಳು ಎಂದು ಭಾವಿಸುವೆ ಏನು? ಜಗನ್ಮಾತೆ ನನಗೆ ಕೆಲವು ವಿಷಯಗಳನ್ನು ತೋರಿದಳು, ಕಂಡದ್ದನ್ನು ಕಂಡಂತೆ ಹೇಳಿದೆ. ಅವಳು ನನಗೆ ಸತ್ಯವನ್ನಲ್ಲದೆ ಬೇರೇನನ್ನೂ ತೋರುವುದಿಲ್ಲ.” ನರೇಂದ್ರ ಅದನ್ನು ಹೀಗೆ ಟೀಕಿಸಿದ: “ಯಾರಿಗೆ ಗೊತ್ತು ಅದನ್ನು ಜಗನ್ಮಾತೆ ನಿಮಗೆ ತೋರಿದ್ದೋ, ಅಥವಾ ನಿಮ್ಮ ಮೆದುಳಿನ ಭ್ರಾಂತಿಯೋ? ನಾನೇನಾದರೂ ನಿಮ್ಮ ಸ್ಥಿತಿಯಲ್ಲಿದ್ದಿದ್ದರೆ ಅವುಗಳೆಲ್ಲ ಮಾನಸಿಕ ಕಲ್ಪನೆ ಎಂದುಬಿಡುತ್ತಿದ್ದೆ. ಪಾಶ್ಚಾತ್ಯ ವಿಜ್ಞಾನ ಮತ್ತು ತತ್ತ್ವಶಾಸ್ತ್ರಗಳು ನಮ್ಮ ಇಂದ್ರಿಯಗಳು ಅನೇಕವೇಳೆ ನಮಗೆ ಮೋಸಮಾಡುತ್ತವೆ ಎಂಬುದನ್ನು ಸಪ್ರಮಾಣವಾಗಿ ತೋರಿರುವುವು. ಯಾವಾಗ ಒಬ್ಬರ ವಿಷಯದಲ್ಲಿ ಅಧಿಕ ಆಸಕ್ತಿಯಿದೆಯೋ ಆಗ ಅವರ ಬಗ್ಗೆ ವಿಪರೀತ ಭಾವನೆಗಳನ್ನು ಮಾಡಿಕೊಳ್ಳುವುದಕ್ಕೆ ಅವಕಾಶವಿದೆ. ನೀವು ನನ್ನನ್ನು ಪ್ರೀತಿಸುವುದರಿಂದ, ನಾನು ದೊಡ್ಡವನಾಗಬೇಕೆಂದು ನೀವು ಆಶಿಸುವುದರಿಂದ, ನಿಮ್ಮ ಮನಸ್ಸಿಗೆ ಈ ಭ್ರಾಂತಿಯ ಭಾವನೆಗಳೆಲ್ಲ ಬರುವುದು ಸ್ವಾಭಾವಿಕ.” ಕೆಲವು ವೇಳೆ ಈ ಟೀಕೆಗಳನ್ನು ಶ‍್ರೀರಾಮಕೃಷ್ಣರು ಗಮನಿಸುತ್ತಿರಲಿಲ್ಲ. ಮತ್ತೆ ಕೆಲವು ವೇಳೆ ಇದನ್ನು ಮನಸ್ಸಿಗೆ ಹಚ್ಚಿಕೊಂಡಾಗ ಜಗನ್ಮಾತೆಯ ಹತ್ತಿರ ಹೋಗಿ ತಮ್ಮ ವ್ಯಾಕುಲವನ್ನು ತೋಡಿಕೊಳ್ಳುತ್ತಿದ್ದರು. ಅವಳು ಯಾವಾಗಲೂ ಇವರಿಗೆ ಭರವಸೆಯನ್ನು ನೀಡುತ್ತಿದ್ದಳು: “ಅವನು ಹೇಳುವುದನ್ನು ನೀನು ಏತಕ್ಕೆ ಗಮನಿಸುವೆ? ಇನ್ನು ಕೆಲವು ದಿನಗಳಲ್ಲಿ ಅವನು ನಿನ್ನ ಪ್ರತಿಯೊಂದು ನುಡಿಯನ್ನೂ ಸತ್ಯವೆಂದು ಒಪ್ಪಿಕೊಳ್ಳುವನು.”

ಶ‍್ರೀರಾಮಕೃಷ್ಣರು ತಮ್ಮ ಶಿಷ್ಯರಿಗೆ, “ನಾನು ಹೇಳುವುದನ್ನೆಲ್ಲ ಚೆನ್ನಾಗಿ ವಿಮರ್ಶೆ ಮಾಡಿ ತೆಗೆದುಕೊಳ್ಳಿ; ವ್ಯಾಪಾರಿ ದುಡ್ಡನ್ನು ಹೇಗೆ ಪರೀಕ್ಷೆ ಮಾಡಿ ತೆಗೆದುಕೊಳ್ಳುವನೋ ಹಾಗೆ ಮಾಡಿ” ಎಂದು ಹೇಳುತ್ತಿದ್ದರು. ಶ‍್ರೀರಾಮಕೃಷ್ಣರಿಗೆ ಹಣವನ್ನು ಕೈಯಿಂದ ಮುಟ್ಟಿದರೆ ಸಾಕು ಕೈಯೆಲ್ಲ ಸೇದುಹೋಗಿಬಿಡುತ್ತಿತ್ತು. ಚೇಳು ಕುಟುಕಿದಂತೆ ಆಗುತ್ತಿತ್ತು. ಇದನ್ನು ಅನೇಕ ಭಕ್ತರು ನೋಡಿದ್ದರು. ಅವರೆದುರಿಗೆ ಶ‍್ರೀರಾಮಕೃಷ್ಣರು ಹೀಗೆ ನಟನೆ ಮಾಡುತ್ತಿದ್ದರೂ ಇರಬಹುದು ಎಂಬ ಅನುಮಾನ ಬರಲು ಅವಕಾಶವಿದೆ. ಒಂದು ಸಲ ನರೇಂದ್ರ ದಕ್ಷಿಣೇಶ್ವರಕ್ಕೆ ಬಂದ. ಕೋಣೆಯೊಳಗೆ ಶ‍್ರೀರಾಮಕೃಷ್ಣರು ಇರಲಿಲ್ಲ, ಹೊರಗೆ ಹೋಗಿದ್ದರು. ನರೇಂದ್ರ ಶ‍್ರೀರಾಮಕೃಷ್ಣರ ಹಾಸಿಗೆಯ ಕೆಳಗೆ ಒಂದು ರೂಪಾಯಿಯನ್ನು ಬಚ್ಚಿಟ್ಟು ಪಂಚವಟಿಯ ಕೆಳಗೆ ಧ್ಯಾನಮಾಡುವುದಕ್ಕೆ ಹೋದ. ಸ್ವಲ್ಪ ಹೊತ್ತಿನಮೇಲೆ ಶ‍್ರೀರಾಮಕೃಷ್ಣರು ಬಂದು ತಮ್ಮ ಹಾಸಿಗೆಯ ಮೇಲೆ ಕುಳಿತರು. ಕುಳಿತ ತತ್‍ಕ್ಷಣ ಒಂದು ಚೇಳು ಕುಟುಕಿದಂತೆ ಆಯಿತು. ಹತ್ತಿರವಿದ್ದ ಭಕ್ತರಿಗೆ, “ಹಾಸಿಗೆಯನ್ನು ಕೊಡವಿ ನೋಡಿ, ಕೆಳಗೆ ಏನಾದರೂ ಇದೆಯೊ ಏನೋ?” ಎಂದು ಹೇಳಿದರು. ಅದರಂತೆ ಹಾಸಿಗೆಯನ್ನು ಕೊಡವಿ ನೋಡಿದಾಗ ಒಂದು ರೂಪಾಯಿ ಕೆಳಗೆ ಬಿತ್ತು. ನರೇಂದ್ರ ಆ ಸಮಯದಲ್ಲಿ ಬಂದು ಅದನ್ನು ನೋಡಿದ. ಶ‍್ರೀರಾಮಕೃಷ್ಣರಿಗೆ ನರೇಂದ್ರನೇ ಇದನ್ನು ಮಾಡಿರಬೇಕು ಎನ್ನಿಸಿತು. ತಮ್ಮನ್ನು ಪರೀಕ್ಷಿಸಿದ್ದಕ್ಕೆ ಶ‍್ರೀರಾಮಕೃಷ್ಣರಿಗೆ ಸಂತೋಷವಾಯಿತು. ಮತ್ತೆ ಯಾರಾದರೂ ಗುರುಗಳಾಗಿದ್ದರೆ ಯಾವ ಶಾಪ ಶಿಷ್ಯನಿಗೆ ಕಾದಿರುತ್ತಿತ್ತೊ!

ಶ‍್ರೀರಾಮಕೃಷ್ಣರಿಗೆ ನರೇಂದ್ರನ ಮೇಲೆ ಅತ್ಯಂತ ಪ್ರೀತಿಯಿದ್ದರೂ ಅದು ಅಂಧ ಪ್ರೀತಿಯಲ್ಲ. ಎಲ್ಲಿ ನರೇಂದ್ರ ನಿಜವಾಗಿ ತಪ್ಪಿರುವನೋ ಅಲ್ಲಿ ಅವನಿಗೆ ಬುದ್ಧಿಹೇಳಿ ತಿದ್ದುತ್ತಿದ್ದರು. ಅವರಲ್ಲಿರುವ ದೋಷದ ಕಳೆಯನ್ನು ಕಿತ್ತುಹಾಕುತ್ತಿದ್ದರು. ನರೇಂದ್ರ ಮತ್ತು ರಾಖಾಲ್ ಇಬ್ಬರೂ ಶ‍್ರೀರಾಮಕೃಷ್ಣರ ಸಮೀಪಕ್ಕೆ ಬರುತ್ತಿದ್ದರು. ಇಬ್ಬರೂ ಮುಂಚೆ ಬ್ರಹ್ಮಸಮಾಜಕ್ಕೆ ಸೇರಿದವರು. ಅಲ್ಲಿ ತಾವು ವಿಗ್ರಹಕ್ಕೆ ಅಡ್ಡಬೀಳುವುದಿಲ್ಲ ಎಂಬ ಪ್ರತಿಜ್ಞೆಯನ್ನು ಮಾಡಬೇಕು. ಒಮ್ಮೆ ನರೇಂದ್ರ ಮತ್ತು ರಾಖಾಲ್ ದಕ್ಷಿಣೇಶ್ವರಕ್ಕೆ ಒಟ್ಟಿಗೆ ಬಂದಿದ್ದರು. ಶ‍್ರೀರಾಮಕೃಷ್ಣರು ಮಣಿಯುತ್ತಿದ್ದ ಕಾಳಿಕಾಮಾತೆಯ ಮೂರ್ತಿಯನ್ನು ನೋಡಿದಾಗ ರಾಖಾಲನಿಗೆ ಸುಮ್ಮನಿರಲು ಆಗಲಿಲ್ಲ. ದೇವಿಗೆ ಬಾಗಿ ನಮಸ್ಕಾರ ಮಾಡಿಬಿಟ್ಟ. ನರೇಂದ್ರ ಇದನ್ನು ನೋಡಿದ. ವಿಗ್ರಹಕ್ಕೆ ಬಾಗುವುದಿಲ್ಲ ಎಂದು ಪ್ರತಿಜ್ಞೆಯನ್ನು ಮಾಡಿ ಇಲ್ಲಿ ಅದನ್ನು ಮುರಿದ ರಾಖಾಲನನ್ನು ನೋಡಿ ನರೇಂದ್ರನಿಗೆ ಕೋಪ ಬಂತು. ಅವನೊಡನೆ ಮಾತನಾಡುವುದನ್ನು ಬಿಟ್ಟುಬಿಟ್ಟನು. ಶ‍್ರೀರಾಮಕೃಷ್ಣರು ಸ್ನೇಹಿತರ ಮಧ್ಯೆ ಬಂದ ವಿರಸವನ್ನು ನೋಡಿ ನರೇಂದ್ರನಿಗೆ, “ನೋಡು, ನೀನು ರಾಖಾಲನಿಗೆ ಹೆದರಿಸಬೇಡ, ಅವನು ನಿನಗೆ ತುಂಬಾ ಅಂಜುವನು. ಅವನಿಗೆ ದೇವರ ಸಾಕಾರದಲ್ಲಿ ನಂಬಿಕೆ ಇದೆ. ನೀನು ಹೇಗೆ ಅದನ್ನು ಬದಲಾವಣೆ ಮಾಡಬಲ್ಲೆ? ಎಲ್ಲರೂ ಪ್ರಾರಂಭದಲ್ಲಿಯೇ ದೇವರ ನಿರಾಕಾರವನ್ನು ಸಾಕ್ಷಾತ್ಕಾರ ಮಾಡಿಕೊಳ್ಳಲಾಗುವುದಿಲ್ಲ” ಎಂದರು. ನರೇಂದ್ರ ಈ ಮಾತನ್ನು ಕೇಳಿದೊಡನೆ ರಾಖಾಲನೊಂದಿಗೆ ಮರಳಿ ಸೌಹಾರ್ದಭಾವವನ್ನು ಬೆಳಸಿದ.

ಆದರೂ ನರೇಂದ್ರ ವಿಗ್ರಹೋಪಾಸನೆಯನ್ನು ಬಡಪೆಟ್ಟಿಗೆ ಒಪ್ಪಿಕೊಳ್ಳಲಿಲ್ಲ. ಶ‍್ರೀರಾಮಕೃಷ್ಣರು ವಿಗ್ರಹಗಳೆಲ್ಲ ಆಧ್ಯಾತ್ಮಿಕ ಚಿಹ್ನೆಗಳೆಂದು ಎಷ್ಟು ಹೇಳಿದರೂ ಅದಕ್ಕೆ ವಿರೋಧವಾಗಿ ವಾದಮಾಡುವುದನ್ನು ಬಿಡಲಿಲ್ಲ. ಆಗ ಶ‍್ರೀರಾಮಕೃಷ್ಣರು “ನನ್ನ ಜಗನ್ಮಾತೆಯನ್ನು ನೀನು ಒಪ್ಪಿಕೊಳ್ಳದೇ ಇದ್ದರೆ ಇಲ್ಲಿಗೆ ಏತಕ್ಕೆ ಬರುತ್ತೀಯೆ?” ಎಂದು ಕೇಳಿದರು. ನರೇಂದ್ರ ಧೈರ್ಯದಿಂದ “ನಾನು ಇಲ್ಲಿಗೆ ಬರುತ್ತೇನೆ ಎಂದು ಅದನ್ನು ಒಪ್ಪಿಕೊಳ್ಳಬೇಕೇನು?” ಎಂದು ಪ್ರಶ್ನಿಸಿದ. ಆಗ ಶ‍್ರೀರಾಮಕೃಷ್ಣರು “ನೋಡು, ಇನ್ನು ಸ್ವಲ್ಪ ಕಾಲದಲ್ಲೆ ಅವಳನ್ನು ಒಪ್ಪಿಕೊಳ್ಳುವುದು ಮಾತ್ರವಲ್ಲ, ಅವಳ ಹೆಸರಿನಲ್ಲಿ ಕಂಬನಿದುಂಬಿ ಅಳುವೆ” ಎಂದರು. ಹತ್ತಿರವಿದ್ದ ಒಬ್ಬ ಶಿಷ್ಯನಿಗೆ ಹೇಳಿದರು: “ನೋಡಿ, ಈ ಹುಡುಗನಿಗೆ ದೇವರ ಸಾಕಾರದಲ್ಲಿ ನಂಬಿಕೆ ಇಲ್ಲ. ನನ್ನ ಅತೀಂದ್ರಿಯ ಅನುಭವಗಳನ್ನು ನನ್ನ ಮೆದುಳಿನ ಭ್ರಾಂತಿ ಎನ್ನುತ್ತಾನೆ. ಆದರೆ ಹುಡುಗ ಪರಿಶುದ್ಧನಾದ ಒಳ್ಳೆಯ ಸ್ವಭಾವದವನು. ಅವನಿಗೆ ಪ್ರತ್ಯಕ್ಷ ಪ್ರಮಾಣ ಸಿಕ್ಕುವುದಕ್ಕೆ ಮುಂಚೆ ಯಾವುದರಲ್ಲಿಯೂ ನಂಬುವುದಿಲ್ಲ. ಅವನು ಬೇಕಾದಷ್ಟು ಓದಿರುವನು. ಅದ್ಭುತವಾದ ವಿಮರ್ಶಕ ಶಕ್ತಿ ಇದೆ.”

ನರೇಂದ್ರ ಮತ್ತೊಂದನ್ನು ಬಹಳ ಕಟುವಾಗಿ ಟೀಕಿಸುತ್ತಿದ್ದ, ಅದೇ ವೈಷ್ಣವರಲ್ಲಿ ಪ್ರಚಾರದಲ್ಲಿರುವ ರಾಧಾಕೃಷ್ಣರ ಆದರ್ಶ. ಮೊದಲು ಶ‍್ರೀಕೃಷ್ಣ ಒಬ್ಬ ಚಾರಿತ್ರಿಕ ವ್ಯಕ್ತಿ ಎಂದು ಅವನು ನಂಬಿರಲಿಲ್ಲ. ಎರಡನೆಯದಾಗಿ ರಾಧೆ ಮತ್ತೊಬ್ಬರ ಹೆಂಡತಿಯಾಗಿ ಶ‍್ರೀಕೃಷ್ಣನನ್ನು ಪ್ರೀತಿಸುವುದನ್ನು ಒಪ್ಪಲಿಲ್ಲ. ಅದಕ್ಕೆ ಶ‍್ರೀರಾಮಕೃಷ್ಣರು “ಒಂದು ವೇಳೆ ಶ‍್ರೀಕೃಷ್ಣ ಚಾರಿತ್ರಿಕ ವ್ಯಕ್ತಿ ಅಲ್ಲದೆ ಇದ್ದರೂ ಅವನನ್ನು ಒಂದು ಆದರ್ಶವನ್ನಾಗಿ ಏತಕ್ಕೆ ತೆಗೆದುಕೊಳ್ಳಬಾರದು? ರಾಧೆಗೆ ಶ‍್ರೀಕೃಷ್ಣನ ಮೇಲಿದ್ದ ಪ್ರೀತಿಯ ಆದರ್ಶವನ್ನು ತೆಗೆದುಕೊಂಡರೆ ಸಾಕು” ಎಂದರು.

ಮತ್ತೊಮ್ಮೆ ತಾಂತ್ರಿಕ ಸಾಧಕರಲ್ಲಿ ರೂಢಿಯಲ್ಲಿದ್ದ ಕೆಲವು ಅನೈತಿಕ ಆಚಾರಗಳನ್ನು ನರೇಂದ್ರ ತುಂಬಾ ಕಟುವಾಗಿ ಟೀಕಿಸುತ್ತಿದ್ದಾಗ ಶ‍್ರೀರಾಮಕೃಷ್ಣರು “ನೀವು ಆ ಮಾರ್ಗವನ್ನು ಅನುಸರಿಸಬೇಕಾಗಿಲ್ಲ. ಆದರೆ ಅದನ್ನು ಏತಕ್ಕೆ ಟೀಕಿಸುತ್ತೀರಿ? ಒಂದು ಮನೆಗೆ ಗೌರವದಿಂದ ಮುಂದಿನ ಬಾಗಿಲಿನಿಂದ ಹೋಗಬಹುದು. ಅದರಂತೆಯೇ ಓಣಿಯ ಮೂಲಕ ಕಸಗುಡಿಸುವವನು ಬರುವ ಬಾಗಿಲಿನ ಮೂಲಕವೂ ಹೋಗಬಹುದು. ನೀವು ಮುಂದಿರುವ ರಾಜಮಾರ್ಗದಲ್ಲಿ ಹೋಗಿ, ಆದರೆ ಹಿಂದಿರುವ ಬಾಗಿಲನ್ನು ಏತಕ್ಕೆ ದೂರುತ್ತೀರಿ?” ಎಂದರು.

ನರೇಂದ್ರನಿಗೆ ಶ‍್ರೀರಾಮಕೃಷ್ಣರ ಒಂದೊಂದು ನುಡಿ, ಒಂದೊಂದು ಉಪಮಾನ ಸಾಕು ದೊಡ್ಡದೊಂದು ಬೆಳಕನ್ನು ಕೊಡುವುದಕ್ಕೆ. ಕ್ರಮೇಣ ಅವನ ಹೃದಯ ವಿಶಾಲವಾಗುತ್ತ ಬಂತು. ನರೇಂದ್ರನನ್ನು ನೋಡಿ ಇವನು ಅದ್ವೈತ ಅನುಭವಕ್ಕೆ ಯೋಗ್ಯ ವ್ಯಕ್ತಿ ಎಂದು ಶ‍್ರೀರಾಮಕೃಷ್ಣರು ಭಾವಿಸಿ ಅದ್ವೈತದ ಒಂದು ಮುಖ್ಯವಾದ ಗ್ರಂಥವನ್ನು ಓದಿಸಿ ಅದರ ಅರ್ಥವನ್ನು ನರೇಂದ್ರನಿಗೆ ವಿವರಿಸುತ್ತಿದ್ದರು. ನರೇಂದ್ರ ಮುಂಚೆ ವಿಗ್ರಹಗಳನ್ನು ಹೇಗೆ ದೂರುತ್ತಿದ್ದನೋ ಹಾಗೆಯೇ ಅದ್ವೈತ ಅನುಭವವನ್ನೂ ಜರಿಯುತ್ತಿದ್ದ. ಮೊದಲು ಅದನ್ನು ಕೇಳಿದಾಗ ಹೀಗೆ ಟೀಕಿಸುತ್ತಿದ್ದ: “ಇಂತಹ ತತ್ತ್ವಕ್ಕೂ ನಾಸ್ತಿಕತೆಗೂ ಏನೂ ವ್ಯತ್ಯಾಸವೇ ಇಲ್ಲ. ಇದೊಂದು ಈಶ್ವರನಿಂದೆ, ನಾನೇ ಸೃಷ್ಟಿಕರ್ತ ಎಂದು ಭಾವಿಸುವುದಕ್ಕಿಂತ ಪರಮ ಪಾತಕ ಯಾವುದೂ ಇಲ್ಲ. ನಾನು ದೇವರು, ನೀನು ದೇವರು, ಸುತ್ತಮುತ್ತಲಿರುವುದೆಲ್ಲ ದೇವರು! ಇದಕ್ಕಿಂತ ಹಾಸ್ಯಾಸ್ಪದವಾಗಿರುವುದು ಇನ್ನೇನು ಇರುವುದು? ಇಂತಹ ವಿಷಯಗಳನ್ನು ಬರೆದ ಋಷಿಗಳ ತಲೆ ಕೆಟ್ಟುಹೋಗಿದ್ದಿರಬೇಕು.” ಶ‍್ರೀರಾಮಕೃಷ್ಣರಿಗೆ ಇಂತಹ ಕಟುನಿಂದೆಯನ್ನು ಕೇಳಿ ನಗು ಬರುತ್ತಿತ್ತು. ಆಗ “ನೀನು ಋಷಿಗಳು ಹೇಳಿದ ಅಭಿಪ್ರಾಯವನ್ನು ತೆಗೆದುಕೊಳ್ಳದೆ ಇರಬಹುದು. ನೀನು ಅವರಲ್ಲಿ ತಪ್ಪು ಕಂಡುಹಿಡಿಯುವುದೇಕೆ? ಭಗವಂತನ ಅನಂತ ಶಕ್ತಿಗೆ ಮಿತಿಯನ್ನು ಕಲ್ಪಿಸಲು ನೀನಾರು?” ಎಂದರು. ಆದರೆ ನರೇಂದ್ರ ಸುಲಭವಾಗಿ ಸೋಲುವವನಲ್ಲ. ತನ್ನ ಯುಕ್ತಿಗೆ ಸಮರ್ಪಕವಾಗಿ ತೋರದೆ ಇರುವುದನ್ನು ಅವನು ಸುಳ್ಳು ಎಂದು ಹಳಿದುಬಿಡುತ್ತಿದ್ದನು. ಅದ್ವೈತವನ್ನು ಹಂಗಿಸುವುದಕ್ಕೆ ಸಮಯ ಸಿಕ್ಕಿದಾಗಲೆಲ್ಲ ಅದನ್ನು ಬಿಡುತ್ತಿರಲಿಲ್ಲ. ಆದರೆ ಶ‍್ರೀರಾಮಕೃಷ್ಣರು ನರೇಂದ್ರನದು ಜ್ಞಾನದ ಹಾದಿ ಎಂಬುದನ್ನು ಮನಗಂಡರು. ಅವನಿಗೆ ಅದ್ವೈತದ ಮೇಲೆ ಬೋಧನೆಯನ್ನು ಮಾಡಲಾರಂಭಿಸಿದರು. ಒಂದು ದಿನ ಜೀವಬ್ರಹ್ಮೈಕ್ಯವನ್ನು ನರೇಂದ್ರನಿಗೆ ವಿವರಿಸಿ ಹೇಳಲು ಪ್ರಯತ್ನಿಸಿದರು. ಆದರೆ ಸಾಧ್ಯವಾಗಲಿಲ್ಲ. ನರೇಂದ್ರ ಪ್ರತಾಪಚಂದ್ರ ಹಾಜರ ಎಂಬುವನ ಬಳಿ ಹೋಗಿ ಹರಟೆ ಹೊಡೆಯತೊಡಗಿದ. “ಇದು ಹೇಗೆ ಸಾಧ್ಯ? ಈ ಹೂಜಿ ದೇವರು, ನಾವುಗಳೂ ದೇವರು! ಇದರಷ್ಟು ಅವಿವೇಕತನ ಮತ್ತಾವುದೂ ಇಲ್ಲ,” ಹೀಗೆಂದು ಹೇಳಿ ನರೇಂದ್ರ ನಗತೊಡಗಿದ. ತಮ್ಮ ಕೋಣೆಯಲ್ಲಿದ್ದ ಶ‍್ರೀರಾಮಕೃಷ್ಣರು ಪಂಚೆಯ ಅಂಚನ್ನು ಕಂಕುಳಿನಲ್ಲಿ ಸಿಕ್ಕಿಸಿಕೊಂಡು ನಗುತ್ತ ‘ಏನು ನೀವು ಮಾತನಾಡುತ್ತಿರುವುದು?’ ಎಂದು ಕೇಳುತ್ತ ನರೇಂದ್ರನನ್ನು ಸ್ಪರ್ಶಮಾಡಿ ತಾವು ಸಮಾಧಿಮಗ್ನರಾದರು. ಆ ಸ್ಪರ್ಶದ ಪರಿಣಾಮವನ್ನು ನರೇಂದ್ರನ ಮಾತಿನಿಂದಲೇ ಕೇಳೋಣ:

“ಶ‍್ರೀರಾಮಕೃಷ್ಣರ ಅದ್ಭುತವಾದ ಸ್ಪರ್ಶ ‍ತತ್‍ಕ್ಷಣವೇ ನನ್ನ ಮನಸ್ಸಿನಲ್ಲಿ ಒಂದು ಬದಲಾವಣೆಯನ್ನು ಉಂಟುಮಾಡಿತು. ಈ ಪ್ರಪಂಚದಲ್ಲಿ ದೇವರಲ್ಲದೆ ಬೇರೆ ಏನೂ ಇರುವಂತೆ ಕಾಣಲಿಲ್ಲ. ಇದನ್ನು ನೋಡಿ ನಾನು ದಿಗ್ಭ್ರಾಂತನಾದೆ. ಇದನ್ನು ನಾನು ಸ್ಪಷ್ಟವಾಗಿ ನೋಡಿದೆ. ಆದರೆ ಈ ಸ್ಥಿತಿ ಮುಂದೆಯೂ ಇರುವುದೇನೋ ನೋಡೋಣ ಎಂದು ನಾನು ಸುಮ್ಮನೆ ಇದ್ದೆ. ಆದರೆ ಅಂದಿನ ದಿನವೆಲ್ಲ ಆ ಅನುಭವ ನನ್ನನ್ನು ಬಿಟ್ಟು ಹೋಗಲಿಲ್ಲ. ನಾನು ಮನೆಗೆ ಹಿಂತಿರುಗಿದೆ. ಅಲ್ಲಿ ನಾನು ನೋಡಿದುದೆಲ್ಲವೂ ಬ್ರಹ್ಮವಾಗಿಯೇ ಕಂಡುಬಂದವು. ನಾನು ಊಟಕ್ಕೆ ಕುಳಿತೆ. ಆಹಾರ, ತಟ್ಟೆ, ಬಡಿಸುತ್ತಿದ್ದವರು ಮತ್ತು ನಾನು ಎಲ್ಲವೂ ಬ್ರಹ್ಮವಲ್ಲದೇ ಬೇರೆ ಆಗಿರಲಿಲ್ಲ. ನಾನು ಒಂದೆರಡು ತುತ್ತನ್ನು ತಿಂದು ಸುಮ್ಮನೆ ಕುಳಿತುಕೊಂಡೆ. ನನ್ನ ತಾಯಿ ‘ಏತಕ್ಕೆ ಸುಮ್ಮನೆ ಕುಳಿತಿರುವೆ? ನಿನ್ನ ಊಟವನ್ನು ಪೂರೈಸು’ ಎಂದು ಹೇಳಿದುದನ್ನು ಕೇಳಿ ಆಶ್ಚರ‍್ಯವಾಯಿತು. ನಾನು ಪುನಃ ಊಟ ಮಾಡಲು ಆರಂಭಿಸಿದೆ. ಊಟ ಮಾಡುತ್ತಿರುವಾಗ ಆಗಲಿ, ಯಾವಾಗಲೂ ಒಂದೇ ಅನುಭವವಿತ್ತು. ನಾನು ಅರೆನಿದ್ರೆಯಲ್ಲಿರುವಂತೆ ಭಾಸವಾಯಿತು. ನಾನು ದಾರಿಯಲ್ಲಿ ನಡೆಯುತ್ತಿದ್ದಾಗ ಗಾಡಿಗಳು ಸಂಚಾರ ಮಾಡುತ್ತಿದ್ದುದನ್ನು ನೋಡಿದೆ. ನಾನು ಅವುಗಳಿಗೆ ದಾರಿಬಿಡಬೇಕೆಂದು ಅನ್ನಿಸಲಿಲ್ಲ. ಆ ಗಾಡಿ ಮತ್ತು ನಾನು ಎರಡೂ ಒಂದೇ ವಸ್ತುವಿನಿಂದ ಆದಂತೆ ಭಾಸವಾಯಿತು. ನನ್ನ ಇಂದ್ರಿಯಗಳಲ್ಲಿ ಯಾವ ವೇದನೆಯೂ ಆಗುತ್ತಿರಲಿಲ್ಲ. ಅವು ಸಂಪೂರ್ಣವಾಗಿ ನಿಶ್ಚೇಷ್ಟಿತವಾಗಿರುವಂತೆ ತೋರುತ್ತಿದ್ದವು. ಊಟಮಾಡುವಾಗ ನನಗೆ ರುಚಿ ತೋರಲಿಲ್ಲ. ಮತ್ತಾರೋ ಊಟಮಾಡುತ್ತಿರುವಂತೆ ತೋರಿತು. ಕೆಲವು ವೇಳೆ ಊಟದ ಮಧ್ಯದಲ್ಲಿಯೇ ನಿದ್ರೆ ಹೋಗುತ್ತಿದ್ದೆ. ಕೆಲವು ನಿಮಿಷಗಳಾದ ಮೇಲೆ ಪುನಃ ಎದ್ದು ಊಟಮಾಡುತ್ತಿದ್ದೆ. ಇದರ ಪರಿಣಾಮವಾಗಿ ಕೆಲವು ದಿನಗಳು ವಿಪರೀತ ಊಟ ಮಾಡಿಬಿಡುತ್ತಿದ್ದೆ. ಆದರೆ ಇದರಿಂದ ನನಗೆ ಯಾವ ಅಪಾಯವೂ ಆಗಲಿಲ್ಲ. ನನ್ನ ತಾಯಿಗೆ ಇದನ್ನು ನೋಡಿ ತುಂಬಾ ಅಂಜಿಕೆಯಾಯಿತು. ನನಗೆ ಏನೋ ಆಗಿಹೋಗಿದೆ ಎಂದು ಭಾವಿಸಿದಳು. ನಾನು ಬಹಳ ಕಾಲ ಬದುಕಲಾರೆ ಎಂದು ಭಾವಿಸಿದಳು. ಮೇಲಿನ ಸ್ಥಿತಿ ಸ್ವಲ್ಪ ಬದಲಾವಣೆ ಆದಾಗ ಪ್ರಪಂಚವೆಲ್ಲ ಒಂದು ಕನಸಿನಂತೆ ತೋರಿತು. ನಾನು ಕಾರನ್‍ವಾಲೀಸ್ ಚೌಕದಲ್ಲಿ ನಡೆದುಕೊಂಡು ಹೋಗುತ್ತಿದ್ದಾಗ, ದಾರಿಯ ಪಕ್ಕದಲ್ಲಿ ಕಬ್ಬಿಣದ ಕಂಬಕ್ಕೆ ನನ್ನ ತಲೆಯನ್ನು ಡಿಕ್ಕಿ ಹೊಡೆದುಕೊಂಡು ನೋಡಿದೆ, ಇದೇನು ಸ್ವಪ್ನವೋ ಜಾಗ್ರದವಸ್ಥೆಯೋ ಎಂದು. ಈ ಸ್ಥಿತಿ ಕೆಲವು ದಿನಗಳವರೆಗೆ ಇತ್ತು. ನಾನು ಸಹಜಸ್ಥಿತಿಗೆ ಬಂದಮೇಲೆ ಅದ್ವೈತಾನುಭವದ ವಿಹಂಗಮ ದೃಷ್ಟಿಯೊಂದು ನನಗೆ ಬಂದಿದ್ದಿರಬಹುದು ಎಂದು ಭಾವಿಸಿದೆ. ಆಗ ಶಾಸ್ತ್ರದಲ್ಲಿ ಹೇಳಿರುವುದು ಸುಳ್ಳಲ್ಲ ಎಂದು ನನಗೆ ಅರಿವಾಯಿತು. ಅನಂತರ ನನಗೆ ಅದ್ವೈತ ಸಿದ್ಧಾಂತದ ನಿರ್ಣಯಗಳನ್ನು ತಪ್ಪು ಎಂದು ಸಾಧಿಸಲು ಆಗಲಿಲ್ಲ.”

ನರೇಂದ್ರ ಹೇಗೆ ಶ‍್ರೀರಾಮಕೃಷ್ಣರನ್ನು ಚೆನ್ನಾಗಿ ವಿಮರ್ಶೆಮಾಡಿ ನೋಡಿದನೊ ಹಾಗೆಯೇ ಕೆಲವು ವೇಳೆ ನರೇಂದ್ರನನ್ನೂ ಶ‍್ರೀರಾಮಕೃಷ್ಣರು ಪರೀಕ್ಷೆಗೆ ಗುರಿಮಾಡಿದರು. ಶ‍್ರೀರಾಮಕೃಷ್ಣರಿಗೆ ನರೇಂದ್ರ ಎಂದರೆ ಪ್ರಾಣ. ಅವನು ಕೆಲವು ದಿನಗಳವರೆಗೆ ದಕ್ಷಿಣೇಶ್ವರಕ್ಕೆ ಬರಲಿಲ್ಲ ಎಂದರೆ ಹುಚ್ಚರಂತೆ ಆಗಿಬಿಡುತ್ತಿದ್ದರು. ಈಗ ಕೆಲವು ಕಾಲದಿಂದ ನರೇಂದ್ರ ಬಂದರೂ ಅವನೊಡನೆ ಯಾವ ಮಾತನ್ನೂ ಆಡುತ್ತಿರಲಿಲ್ಲ. ಬಂದೆಯಾ ಎಂದೂ ಕೇಳುತ್ತಿರಲಿಲ್ಲ. ನರೇಂದ್ರ ಬರುವನು, ಶ‍್ರೀರಾಮಕೃಷ್ಣರಿಗೆ ಮಣಿಯುವನು, ಅವರ ಸಮೀಪದಲ್ಲಿ ಕುಳಿತುಕೊಳ್ಳುವನು. ಅವರು ತನ್ನೊಡನೆ ಮಾತನಾಡದೇ ಇದ್ದಾಗ ಬಹುಶಃ ಶ‍್ರೀರಾಮಕೃಷ್ಣರು ಯಾವ ಆಲೋಚನೆಯಲ್ಲಿಯೋ ಮಗ್ನರಾಗಿರಬೇಕೆಂದು ಹೊರಗೆ ಹೋಗಿ ಯಾರಾದರೊಬ್ಬರೊಡನೆ ಹರಟುತ್ತಿದ್ದನು, ಗುಡುಗುಡಿ ಸೇದುತ್ತಿದ್ದನು. ಆಗ ಶ‍್ರೀರಾಮಕೃಷ್ಣರು ಇತರ ಭಕ್ತರೊಡನೆ ಮಾತನಾಡುತ್ತಿದ್ದುದು ಕೇಳಿಬರುತ್ತಿತ್ತು. ನರೇಂದ್ರ ಪುನಃ ಅವರ ಎದುರಿನಲ್ಲಿ ಬಂದು ಕುಳಿತುಕೊಳ್ಳುತ್ತಿದ್ದ. ಆಗಲೂ ನರೇಂದ್ರನೊಡನೆ ಮಾತನಾಡುತ್ತಿರಲಿಲ್ಲ. ಅವನ ಕಡೆ ನೋಡುತ್ತಲೂ ಇರಲಿಲ್ಲ. ಎಂದಿನಂತೆ ನರೇಂದ್ರ ಶ‍್ರೀರಾಮಕೃಷ್ಣರಿಗೆ ನಮಿಸಿ ಕಲ್ಕತ್ತೆಗೆ ಹೋಗುತ್ತಿದ್ದ.

ಒಂದು ವಾರವಾದ ಮೇಲೆ ಮತ್ತೊಮ್ಮೆ ನರೇಂದ್ರ ಬಂದ. ಆಗಲೂ ಶ‍್ರೀರಾಮಕೃಷ್ಣರು ನರೇಂದ್ರನನ್ನು ಅಲಕ್ಷ್ಯದಿಂದಲೇ ನೋಡಿದರು. ಸಾಯಂಕಾಲದ ತನಕ ಇದ್ದು ಅನಂತರ ನರೇಂದ್ರ ಮನೆಗೆ ಹೋದ. ಹೀಗೆ ನಾಲ್ಕೈದು ಸಲ ನರೇಂದ್ರ ಬಂದಾಗಲೂ ಮಾಡಿದರು. ಆದರೂ ನರೇಂದ್ರ ಬರುವುದನ್ನು ಬಿಡಲಿಲ್ಲ. ಒಂದು ತಿಂಗಳಾದ ಮೇಲೆ ಶ‍್ರೀರಾಮಕೃಷ್ಣರು ನರೇಂದ್ರನಿಗೆ “ನಾನು ನಿನ್ನೊಡನೆ ಒಂದು ಮಾತನ್ನೂ ಆಡದೇ ಇದ್ದರೂ ನೀನು ಏತಕ್ಕೆ ಬರುತ್ತೀಯಾ?” ಎಂದು ಕೇಳಿದರು. ಅದಕ್ಕೆ “ನಾನು ನಿಮ್ಮೊಡನೆ ಬರೀ ಮಾತಾಡಲು ಮಾತ್ರ ಬರುತ್ತೇನೆಯೇ? ನಾನು ನಿಮ್ಮನ್ನು ಪ್ರೀತಿಸುತ್ತೇನೆ. ನಿಮ್ಮನ್ನು ನೋಡಬೇಕೆಂದು ಆಸೆ, ಅದಕ್ಕೆ ಬರುತ್ತೇನೆ” ಎಂದ ನರೇಂದ್ರ. ಶ‍್ರೀರಾಮಕೃಷ್ಣರಿಗೆ ಬಹಳ ಆನಂದವಾಯಿತು. ಆಗ ಅವರು ನರೇಂದ್ರನಿಗೆ “ನಾನು ನಿನ್ನನ್ನು ಪರೀಕ್ಷಿಸುತ್ತಿದ್ದೆ. ನಾನು ನಿನ್ನನ್ನು ಅಲಕ್ಷ್ಯದಿಂದ ಕಂಡರೆ, ಪ್ರೀತಿಯನ್ನು ತೋರಿಸದೆ ಇದ್ದರೆ, ನೀನು ಬರುತ್ತೀಯೋ ಇಲ್ಲವೋ ಎಂಬುದನ್ನು ನೋಡಬೇಕೆಂದು ಇದ್ದೆ. ಇಷ್ಟೊಂದು ಅನಾದರ ಮತ್ತು ಅಸಡ್ಡೆಯನ್ನು ನಿನ್ನಂತಹ ಸತ್ತ್ವಶಾಲಿಗಳು ಮಾತ್ರ ಸಹಿಸಬಲ್ಲರು” ಎಂದರು.

ಒಂದು ದಿನ ಶ‍್ರೀರಾಮಕೃಷ್ಣರು ನರೇಂದ್ರನನ್ನು ಪಂಚವಟಿ ಕಡೆಗೆ ಕರೆದುಕೊಂಡು ಹೋಗಿ ಹೇಳಿದರು:

“ನಾನು ಉಗ್ರ ಸಾಧನೆಯನ್ನು ಮಾಡಿದುದರ ಪರಿಣಾಮವಾಗಿ ಹಲವು ಸಿದ್ಧಿಗಳು ಬಂದಿವೆ. ಆದರೆ ಅದರಿಂದ ನನಗೇನು ಪ್ರಯೋಜನ? ನಾನು ದೇಹದ ಮೇಲೆ ಬಟ್ಟೆಯನ್ನೂ ಕೂಡ ಹಾಕಿಕೊಳ್ಳುವುದಿಲ್ಲ. ದೇವಿಯ ಇಚ್ಛೆಯಿಂದ ಅವುಗಳನ್ನೆಲ್ಲ ನಿನಗೆ ರವಾನಿಸಬೇಕೆಂದಿರುವೆನು. ನೀನು ಬೇಕಾದಷ್ಟು ಅವಳ ಕೆಲಸವನ್ನು ಮಾಡಬೇಕಾಗಿದೆ ಎಂಬುದನ್ನು ಅವಳು ನನಗೆ ತೋರಿರುವಳು. ನಾನು ಇವುಗಳನ್ನೆಲ್ಲಾ ನಿನಗೆ ಕೊಟ್ಟರೆ ನೀನು ಬೇಕಾದಾಗ ಇವುಗಳನ್ನು ಉಪಯೋಗಿಸಬಹುದು. ನೀನು ಏನು ಹೇಳುತ್ತೀಯೆ?”

ನರೇಂದ್ರ: “ಇವು ಭಗವತ್ ಸಾಕ್ಷಾತ್ಕಾರಕ್ಕೆ ನನಗೆ ಸಹಾಯ ಮಾಡುವುವೆ?”

ಶ‍್ರೀರಾಮಕೃಷ್ಣರು: “ಇಲ್ಲ, ಅವು ನಿನಗೆ ಭಗವತ್ ಸಾಕ್ಷಾತ್ಕಾರಕ್ಕೆ ಸಹಾಯ ಮಾಡಲಾರವು. ಆದರೆ ನೀನು ಭಗವತ್ ಸಾಕ್ಷಾತ್ಕಾರವನ್ನು ಪಡೆದುಕೊಂಡ ಮೇಲೆ ಅವಳ ಕೆಲಸವನ್ನು ಮಾಡುತ್ತಿರುವಾಗ ನಿನಗೆ ಸಹಾಯಮಾಡಬಲ್ಲವು.”

ನರೇಂದ್ರ: “ನನಗೆ ಅವು ಬೇಡ, ಮೊದಲು ನಾನು ದೇವರನ್ನು ಸಾಕ್ಷಾತ್ಕಾರ ಮಾಡಿಕೊಂಡ ನಂತರ ಇವು ನನಗೆ ಬೇಕೋ ಬೇಡವೊ ಎಂಬುದು ಗೊತ್ತಾಗುವುದು. ನಾನು ಅವುಗಳನ್ನು ಈಗ ಪಡೆದರೆ ನನ್ನ ಗುರಿಯನ್ನು ಮರೆತು ಸ್ವಾರ್ಥಕ್ಕೆ ಅವುಗಳನ್ನು ಉಪಯೋಗಿಸಬಹುದು. ಅನಂತರ ದುಃಖ ಪಡಬೇಕಾಗುವುದು.”

ಶ‍್ರೀರಾಮಕೃಷ್ಣರಿಗೆ ಇದನ್ನು ಕೇಳಿದಾಗ ಪರಮಾನಂದವಾಯಿತು. ಅಣಿಮಾದಿ ಅಷ್ಟಸಿದ್ಧಿಗಳಲ್ಲಿ ಯಾವುದಾದರೂ ಒಂದು ಬಂದರೆ ಸಾಕು ಎಂದು ಸಾಧಾರಣ ಜೀವಿಗಳು ಕಾತುರರಾಗಿರುವರು. ಆದರೆ ಇಲ್ಲಿ ಎಲ್ಲಾ ಸಿದ್ಧಿಗಳೂ ತಾವಾಗಿ ಬರುವುದರಲ್ಲಿದ್ದವು. ಆದರೂ ನರೇಂದ್ರ ಅವುಗಳನ್ನೆಲ್ಲ ತೃಣಕ್ಕೆ ಕಡೆಯಾಗಿ ಭಾವಿಸುವನು. ಇದರಿಂದ ನರೇಂದ್ರ ಯಾವ ಮಟ್ಟಕ್ಕೆ ಸೇರಿದವನು ಎಂಬುದು ಗೊತ್ತಾಗುವುದು.

