
\chapter{ಗುರುದರ್ಶನ}

ಶ‍್ರೀರಾಮಕೃಷ್ಣರ ಬಳಿಗೆ ನರೇಂದ್ರನಾಥ ಬರುವುದಕ್ಕೆ ಕೆಲವು ವರ್ಷಗಳ ಮುಂಚೆ ಅವನ ವಿಷಯದಲ್ಲಿ ಅವರಿಗೊಂದು ದರ್ಶನ ಕಂಡಿತು. ಅದನ್ನು ಶ‍್ರೀರಾಮಕೃಷ್ಣರೆ ಹೀಗೆ ವಿವರಿಸುವರು:

“ಒಂದು ದಿನ ನಾನು ಸಮಾಧಿಯಲ್ಲಿದ್ದಾಗ ನನ್ನ ಮನಸ್ಸು ಒಂದು ಜ್ಯೋತಿರ್ಮಯ ಪಥದ ಮೂಲಕ ಊರ್ಧ್ವಮುಖವಾಗಿ ಹೋಗುತ್ತಿತ್ತು. ಅದು ಸೂರ‍್ಯ\-ಮಂಡಲವನ್ನು ಅತಿಕ್ರಮಿಸಿ ಭಾವನೆಗಳ ಸೂಕ್ಷ್ಮವಲಯವನ್ನು ಪ್ರವೇಶಿಸಿತು. ಮನಸ್ಸು ಅಲ್ಲಿಂದಲೂ ಮುಂದೆ ಮುಂದೆ ಹೋಗುತ್ತಿದ್ದಾಗ ಎಡಬಲಗಳಲ್ಲಿ ದೇವದೇವಿಯರ ಆಕಾರಗಳು ಕಂಡವು. ಅಲ್ಲಿಂದ ಮುಂದೆ ಹೋದಾಗ ಸಾಕಾರ ಮತ್ತು ನಿರಾಕಾರ ವಲಯವನ್ನು ಬೇರ್ಪಡಿಸುವ ಒಂದು ವಲಯ ಬಂದಿತು. ಅಲ್ಲಿಂದಲೂ ಮುಂದೆ ಮನಸ್ಸು ಹೋಯಿತು. ಅಲ್ಲಿ ಯಾವ ಸ್ಪಷ್ಟವಾದ ಆಕಾರವೂ ಕಂಡು ಬರಲಿಲ್ಲ. ದೇವತೆಗಳೂ ಕೂಡ ಅಲ್ಲಿಗೆ ಪ್ರವೇಶಿಸುವುದಕ್ಕೆ ಸಾಧ್ಯವಿರಲಿಲ್ಲ. ಅವರು ಕೂಡ ಅದರ ಕೆಳಗೆ ಇರಬೇಕಾಗಿತ್ತು. ಅಲ್ಲಿ ಸಪ್ತಋಷಿಗಳು ಸಮಾಧಿಯಲ್ಲಿರುವುದನ್ನು ಕಂಡೆ. ಅವರು ಪವಿತ್ರತೆ, ಜ್ಞಾನ, ಪ್ರೇಮ ಮತ್ತು ತ್ಯಾಗದಲ್ಲಿ ಮನುಷ್ಯರೇ ಅಲ್ಲ, ದೇವತೆಗಳನ್ನೂ ಕೂಡ ಮೀರಿರುವುದನ್ನು ಅರಿತೆ. ಅವರನ್ನು ಆಶ್ಚರ‍್ಯದಿಂದ ನೋಡುತ್ತಿದ್ದೆ. ಆಗ ಅಖಂಡ ತೇಜೋರಾಶಿಯಲ್ಲಿ ಒಂದು ಅಂಶ ಮಗುವಿನ ರೂಪವನ್ನು ಧರಿಸಿತು. ಆ ಮಗು ಸಪ್ತ ಋಷಿಗಳಲ್ಲಿ ಒಬ್ಬನ ಬಳಿಗೆ ಬಂದು, ತನ್ನ ಮುದ್ದು ಕೈಗಳಿಂದ ಋಷಿಗಳಲ್ಲಿ ಒಬ್ಬನನ್ನು ಅಪ್ಪಿಕೊಂಡಿತು. ತನ್ನ ಮುದ್ದು ಮಾತಿನಿಂದ ಅವನನ್ನು ಸಮಾಧಿಯಿಂದ ಎಚ್ಚರಿಸಿತು. ಆ ಅದ್ಭುತವಾದ ಮಗುವನ್ನು ಎವೆಯಿಕ್ಕದೆ ಅರ್ಧನಿಮೀಲಿತ ಲೋಚನನಾಗಿ ಋಷಿ ನೋಡಿದನು. ಆ ಪ್ರೇಮದ ಕಣ್ಣುಗಳು, ಆ ಮಗು ಇವನ ಹೃದಯದ ಐಶ್ವರ‍್ಯವಾಗಿದ್ದನೆಂಬುದನ್ನು ತೋರಿತು. ಮಗು ತುಂಬಾ ಸಂತೋಷದಿಂದ ‘ನಾನು ಕೆಳಗೆ ಹೋಗುತ್ತಿರುವೆನು. ನೀನೂ ಕೂಡ ಬರಬೇಕು’ ಎಂದಿತು. ಋಷಿ ಮೌನವಾಗಿದ್ದನು. ಆದರೆ ಅವನ ಪ್ರೇಮದ ಕುಡಿನೋಟ ಅದಕ್ಕೆ ಒಪ್ಪಿಗೆಯನ್ನು ಕೊಟ್ಟಿತ್ತು. ಮಗುವನ್ನು ನೋಡುತ್ತಿದ್ದ ಹಾಗೆಯೇ ಆ ಋಷಿ ಪುನಃ ಸಮಾಧಿಯಲ್ಲಿ ಮಗ್ನನಾದನು. ಆ ಋಷಿಯ ದೇಹ ಮತ್ತು ಮನಸ್ಸಿನ ಅಂಶ ಮಿಂಚಿನಂತೆ ಧರೆಗೆ ಇಳಿಯಿತು.” ವಾರಣಾಸಿ ಕಡೆಯಿಂದ ಕಲ್ಕತ್ತೆಯ ಕಡೆಗೆ ಮಿಂಚೊಂದು ಗಗನದಲ್ಲಿ ಹೋಗುತ್ತಿರುವುದನ್ನು ಶ‍್ರೀರಾಮಕೃಷ್ಣರು ಕೆಲವು ಕಾಲದ ಮೇಲೆ ಕಂಡರು. “ತಾಯಿ ಶಿಷ್ಯರನ್ನು ಕಳುಹಿಸಬೇಕೆಂಬ ನನ್ನ ಪ್ರಾರ್ಥನೆಯನ್ನು ಕೇಳಿದಳು. ನನ್ನ ಶಿಷ್ಯ ಒಂದಲ್ಲ ಒಂದು ದಿನ ಬಂದೇ ಬರುತ್ತಾನೆ” ಎಂದರು. ಅನಂತರ ಕೆಲವು ವರ್ಷಗಳ ಮೇಲೆ ನರೇಂದ್ರನನ್ನು ನೋಡಿದೊಡನೆಯೇ ಇವನೆ ತನಗೆ ದರ್ಶನದಲ್ಲಿ ಕಂಡವನು ಎಂದು ಗುರುತನ್ನು ಹಿಡಿದರು.

ನರೇಂದ್ರನ ಜೀವನದಲ್ಲಿ ಆಧ್ಯಾತ್ಮಿಕ ದಾಹ ದಾವಾನಲದಂತೆ ಕಾಡತೊಡಗಿತು. ಆಗಿನ ಕಾಲದಲ್ಲಿ ಪ್ರಖ್ಯಾತರಾದವರೊಡನೆಲ್ಲ ಚರ್ಚಿಸಿ ಸಮಾಧಾನ ಸಿಕ್ಕದೆ ಬೇಸರಗೊಂಡಿದ್ದನು. ರಾಮಚಂದ್ರದತ್ತ ಎಂಬ ಪರಿಚಿತನೊಬ್ಬ ನರೇಂದ್ರನಿಗೆ “ನೀನು ದೇವರ ವಿಷಯವನ್ನು ಪ್ರತ್ಯಕ್ಷ ಅನುಭವಿಸಿದವನೊಬ್ಬನಿಂದ ತಿಳಿಯಬೇಕಾದರೆ ಅಲ್ಲಿ ಇಲ್ಲಿ ಅಲೆದಾಡಿದರೆ ಪ್ರಯೋಜನವಿಲ್ಲ. ದಕ್ಷಿಣೇಶ್ವರದಲ್ಲಿರುವ ಶ‍್ರೀರಾಮಕೃಷ್ಣರ ಸಮೀಪಕ್ಕೆ ಹೋಗು. ಆಗಲೇ ನಿನಗೆ ತೃಪ್ತಿ ದೊರಕುವುದು” ಎಂದು ಹೇಳಿದ್ದ. ಒಮ್ಮೆ ಕೆಲವು ಸ್ನೇಹಿತರೊಡನೆ ಶ‍್ರೀರಾಮಕೃಷ್ಣರನ್ನು ನೋಡಲು ದಕ್ಷಿಣೇಶ್ವರಕ್ಕೆ ಹೋದ. ಮೊದಲ ಬಾರಿ ದಕ್ಷಿಣೇಶ್ವರದಲ್ಲಿ ಕಂಡ ನರೇಂದ್ರನ ಚಿತ್ರವನ್ನು ಶ‍್ರೀರಾಮಕೃಷ್ಣರೇ ಹೀಗೆ ವರ್ಣಿಸುತ್ತಾರೆ:

“ನರೇಂದ್ರ ಪಶ್ಚಿಮದ್ವಾರದ ಮೂಲಕ ಈ ಕೋಣೆಗೆ ಬಂದ. ತನ್ನ ದೇಹ ಮತ್ತು ಉಡಿಗೆ ತೊಡಿಗೆಯ ವಿಷಯದಲ್ಲಿ ಗಮನವನ್ನೇ ಕೊಟ್ಟಿಲ್ಲದಂತೆ ಕಾಣುತ್ತಿದ್ದ. ಇತರರು ಬಾಹ್ಯಜಗತ್ತನ್ನು ಗಮನಿಸುವಂತೆ ಇವನು ಇರಲಿಲ್ಲ, ನಿರ್ಲಕ್ಷ್ಯನಾಗಿದ್ದ. ಅವನ ಕಣ್ಣುಗಳು ಅಂತರ್ಮುಖವಾಗಿರುವಂತೆ ಕಂಡವು. ಯಾವುದನ್ನೋ ಗಾಢವಾಗಿ ಕುರಿತು ಆಲೋಚಿಸುತ್ತಿರುವಂತೆ ಕಂಡಿತು. ಕಲ್ಕತ್ತೆಯಂತಹ ಈಶ್ವರ ಭಕ್ತಿ ಲೋಪವಾದ ಕಡೆಯಿಂದ ಇಂತಹ ಆಧ್ಯಾತ್ಮಿಕ ಜೀವಿ ಬರುವುದನ್ನು ಕಂಡು ನನಗೆ ಆಶ್ಚರ್ಯವಾಯಿತು. ನೆಲದ ಮೇಲೆ ಒಂದು ಚಾಪೆಯನ್ನು ಹಾಸಿದ್ದರು. ಈಗ ಗಂಗಾನೀರು ಇರುವ ದೊಡ್ಡ ಜಾಡಿ ಇದೆಯಲ್ಲ, ಅಲ್ಲೇ ಅವನು ಕುಳಿತುಕೊಂಡ. ಅವನೊಡನೆ ಬಂದ ಸ್ನೇಹಿತರಾದರೋ ವಿಷಯ ವಾಸನೆಗಳಲ್ಲಿ ನಿರತರಾದ ಸಾಧಾರಣ ಮನುಷ್ಯರಂತೆ ಕಂಡರು. ನಾನು ಅವನನ್ನು ಕೆಲವು ಹಾಡುಗಳನ್ನು ಹಾಡು ಎಂದು ಕೇಳಿಕೊಂಡುದರಿಂದ ಅವನು ಹಾಡಿದ. ಅದರಲ್ಲಿ ಒಂದು ಬ್ರಹ್ಮಸಮಾಜದವರು ಹಾಡುವ ಸಾಮಾನ್ಯ ಹಾಡು. ‘ಓ ಮನವೇ ನಿನ್ನೂರಿಗೆ ನೀನು ಹೋಗು. ಈ ಪ್ರಪಂಚವೆಂಬ ಪರದೇಶದಲ್ಲಿ ಗೊತ್ತುಗುರಿಯಿಲ್ಲದೆ ಏಕೆ ವೃಥಾ ಅಲೆಯುತ್ತಿರುವೆ?’ ಎಂದು ಭಾವ ಬರುವ ಹಾಡು. ಆದರೆ ಅದನ್ನು ಹಾಡುವಾಗ ನರೇಂದ್ರ ಅದರಲ್ಲೇ ಮಗ್ನನಾಗಿ ಹೋಗಿದ್ದ. ಹೃದಯ ಕರಗಿ ನೀರಾಗುವಂತಹ ಭಾವದಿಂದ ಹಾಡುತ್ತಿದ್ದ. ಹಾಡನ್ನು ಕೇಳಿದ ಮೇಲೆ ಎಂದಿನಂತೆ ವ್ಯವಹಾರ ಭೂಮಿಕೆಯಲ್ಲಿರಲು ಸಾಧ್ಯವಾಗಲಿಲ್ಲ. ನಾನು ಭಾವಸಮಾಧಿಗೆ ಹೋದೆ. ”

ಅನಂತರ ಏನಾಯಿತು ಎಂಬುದನ್ನು ನರೇಂದ್ರನಾಥನ ಬಾಯಿಂದಲೇ ಕೇಳೋಣ:

“ಅವರ ಕೋರಿಕೆಯಂತೆ ನಾನು ಹಾಡಿದೆ. ಸ್ವಲ್ಪ ಹೊತ್ತಾದಮೇಲೆ ಅವರು ಎದ್ದು ನನ್ನ ಕೈಗಳನ್ನು ಹಿಡಿದುಕೊಂಡು ಉತ್ತರ ವರಾಂಡದ ಕಡೆಗೆ ಕರೆದುಕೊಂಡು ಹೋದರು. ಹಿಂದಿನ ಬಾಗಿಲನ್ನು ಮುಚ್ಚಿದರು. ಆ ಕೋಣೆಗೆ ಹೊರಗೆ ಹೋಗುವ ಬಾಗಿಲಿಗೆ ಬೀಗ ಹಾಕಿತ್ತು. ಆದಕಾರಣ ನಾವಿಬ್ಬರೇ ಇದ್ದೆವು. ಅವರು ನನಗೆ ರಹಸ್ಯವಾಗಿ ಏನನ್ನಾದರೂ ಸಲಹೆಗಳನ್ನು ಕೊಡುತ್ತಾರೆಯೋ ಏನೋ ಎಂದು ಭಾವಿಸಿದೆ. ಆದರೆ ಅವರು ನನ್ನ ಕೈಗಳನ್ನು ಹಿಡಿದುಕೊಂಡು ಆನಂದಬಾಷ್ಪಗಳನ್ನು ಸುರಿಸತೊಡಗಿದರು. ನನಗೆ ಆಶ್ಚರ್ಯವಾಯಿತು ಇದನ್ನು ನೋಡಿ. ನನ್ನನ್ನು ಬಹಳ ಪರಿಚಿತನಾದ ಬಂಧುವಂತೆ ಅತ್ಯಂತ ಪ್ರೀತಿಯಿಂದ ಮಾತನಾಡಿಸತೊಡಗಿದರು! ‘ಅಯ್ಯೋ, ನೀನು ಎಷ್ಟು ಹೊತ್ತಾಗಿ ಬಂದೆ? ಇಷ್ಟು ಕಾಲ ನನ್ನನ್ನು ನೀನು ಕಾಯಿಸಿದೆಯಲ್ಲ. ಎಷ್ಟು ನಿರ್ದಯ ನೀನು? ಪ್ರಾಪಂಚಿಕರು ಮಾತನಾಡುವ ವಿಷಯ ವಸ್ತುಗಳ ಮಾತು ಕಥೆಗಳನ್ನು ಕೇಳಿ ಕೇಳಿ ನನ್ನ ಕಿವಿ ಸುಟ್ಟುಹೋಗಿದೆ. ಯಾರು ನನ್ನ ಹೃದಯಾಂತರಾಳ ಭಾವನೆಗಳನ್ನು ಅರ್ಥಮಾಡಿಕೊಳ್ಳಬಲ್ಲವನೋ ಅವನಿಗೆ ಹೇಳಿ ನನ್ನ ಹೃದಯದ ದುಗುಡವನ್ನು ಪರಿಹರಿಸಿಕೊಳ್ಳಬೇಕೆಂದು ಇರುವೆನು.’ ಹೀಗೆ ಅವರು ಕಂಬನಿದುಂಬಿ ಮಾತನಾಡತೊಡಗಿದರು. ಅನಂತರ ಅವರು ತಮ್ಮ ಕೈಗಳನ್ನು ಮುಗಿದು ನಿಂತುಕೊಂಡು ಹೀಗೆ ಹೇಳತೊಡಗಿದರು: ‘ಸ್ವಾಮಿ, ನನಗೆ ಗೊತ್ತಿದೆ ನಾರಾಯಣ ಅವತಾರ ಸ್ವರೂಪನಾದ ನರಮಹರ್ಷಿಯೇ ನೀನು. ಮಾನವ ಕೋಟಿಯ ದುಃಖವನ್ನು ತಗ್ಗಿಸುವುದಕ್ಕಾಗಿಯೇ ನೀನು ಜನ್ಮವೆತ್ತಿರುವುದು’.”

“ನನಗೆ ಇವರ ನಡತೆ ನೋಡಿ ಬಹಳ ಆಶ್ಚರ್ಯವಾಯಿತು. ನಾನು ಯಾವ ಮನುಷ್ಯನನ್ನು ನೋಡುವುದಕ್ಕೆ ಬಂದೆ ಎಂದು ಭಾವಿಸತೊಡಗಿದೆ. ಅವರು ನಿಜವಾಗಿಯೂ ಹುಚ್ಚರಿರಬೇಕು; ನಾನು ಬರೀ ವಿಶ್ವನಾಥದತ್ತನ ಮಗ. ಆದರೂ ಇವರು ನನ್ನನ್ನು ಏನೋ ಸಂಬೋಧಿಸುತ್ತಿರುವರಲ್ಲ? ಆದರೆ ಸುಮ್ಮನೆ ಏನು ಮನಸ್ಸಿಗೆ ತೋರುವುದೊ ಅದನ್ನು ಮಾತನಾಡಿಕೊಂಡು ಹೋಗಲಿ ಎಂದು ನಾನು ಸುಮ್ಮನಾದೆ. ಅವರು ತಮ್ಮ ಕೋಣೆಗೆ ಹಿಂತಿರುಗಿ ಕಲ್ಲುಸಕ್ಕರೆ ಬೆಣ್ಣೆ ಮತ್ತು ಮಿಠಾಯಿಗಳನ್ನು ತಂದು ಅವರೇ ನನಗೆ ತಿನ್ನಿಸತೊಡಗಿದರು. ‘ಅದನ್ನೆಲ್ಲ ಕೊಡಿ, ನಾನು ಹೊರಗೆ ಹೋಗಿ ನನ್ನ ಸ್ನೇಹಿತರಿಗೆ ಕೊಟ್ಟು ಅನಂತರ ತಿನ್ನುವೆ’ ಎಂದು ಎಷ್ಟು ಹೇಳಿದರೂ ಅವರು ಕೇಳಲಿಲ್ಲ. ‘ಅನಂತರ ಅವರಿಗೂ ಸಿಕ್ಕುವುದು’ ಎಂದು ಮಾತ್ರ ಅವರು ಹೇಳಿದರು. ನಾನೆಲ್ಲವನ್ನೂ ತಿನ್ನುವ ತನಕ ಬಿಡಲೇ ಇಲ್ಲ. ‘ನೀನು ಬೇಗ ನನ್ನನ್ನು ನೋಡಲು ಒಬ್ಬನೇ ಬರುತ್ತೇನೆ ಎಂದು ಮಾತುಕೊಡು’ ಎಂದು ನನ್ನನ್ನು ಅವರು ತಮ್ಮ ಕೈಗಳಿಂದ ಹಿಡಿದುಕೊಂಡು ಒತ್ತಾಯಮಾಡಿದರು. ಅವರ ಬಲವಂತಕ್ಕೆ ಆಗಲಿ ಎಂದು ಹೇಳಿ ನನ್ನ ಸ್ನೇಹಿತರ ಬಳಿಗೆ ಬಂದೆ. ಶ‍್ರೀರಾಮಕೃಷ್ಣರು ನೆರೆದ ತಮ್ಮ ಶಿಷ್ಯರೊಂದಿಗೆ ‘ನೋಡಿ ವಿದ್ಯಾಧಿದೇವತೆ ಸರಸ್ವತಿಯ ಕಾಂತಿಯಿಂದ ನರೇಂದ್ರ ಹೇಗೆ ಶೋಭಿಸುತ್ತಿರುವನು!’ ಎಂದರು. ‘ನೀನು ನಿದ್ರೆಗೆ ಹೋಗುವುದಕ್ಕೆ ಮುಂಚೆ ಜ್ಯೋತಿಯನ್ನು ನೋಡುವಿಯೇನು’ ಎಂದು ಕೇಳಿದರು. ಅದಕ್ಕೆ ನಾನು ‘ಹೌದು’ ಎಂದೆ. ಅದಕ್ಕೆ ಅವರು ‘ಹೌದು ಇವನು ಧ್ಯಾನಸಿದ್ಧ’ ಎಂದರು. ”

“ನಾನು ಅಲ್ಲೇ ಕುಳಿತು ಅವರನ್ನು ಗಮನಿಸಿದೆ. ಇವರು ಇತರರೊಡನೆ ವ್ಯವಹರಿಸಿದ ರೀತಿಯಲ್ಲಿ ಯಾವ ಅಸ್ವಾಭಾವಿಕತೆಯೂ ಇರಲಿಲ್ಲ. ಶ‍್ರೀರಾಮಕೃಷ್ಣರ ಮಾತುಕಥೆ ಮತ್ತು ಅವರ ಸಾಮಾಧಿಗಳು ಇವುಗಳಿಂದ ಅವರೊಬ್ಬ ನಿಜವಾದ ತ್ಯಾಗಿಗಳಂತೆ ತೋರಿದರು. ಅವರ ಮಾತು ಮತ್ತು ನಡತೆ ಒಂದೇ ಆಗಿತ್ತು. ಅವರು ಅತ್ಯಂತ ಸರಳವಾದ ಭಾಷೆಯನ್ನು ಉಪಯೋಗಿಸುತ್ತಿದ್ದರು. ನಿಜವಾಗಿ ಇವರೊಬ್ಬ ಮಹಾಗುರುಗಳಿರಬೇಕು ಎಂದು ನಾನು ಯೋಚಿಸತೊಡಗಿದೆ. ನಾನು ಇತರರನ್ನು ಅನೇಕ ವೇಳೆ ಕೇಳುತ್ತಿದ್ದ ಪ್ರಶ್ನೆಯನ್ನೇ ಇವರಿಗೆ ಹಾಕಿದೆ: ‘ಮಹಾಶಯರೆ, ನೀವು ದೇವರನ್ನು ನೋಡಿರುವಿರಾ?’ ಅದಕ್ಕೆ ಅವರು ಹೀಗೆ ಹೇಳಿದರು: ‘ಹೌದು, ನಾನು ನಿನ್ನನ್ನು ಈಗ ಹೇಗೆ ನೋಡುತ್ತೇನೆಯೋ ಹಾಗೆಯೇ ನೋಡುತ್ತೇನೆ. ಆದರೆ ವ್ಯತ್ಯಾಸವಿಷ್ಟೆ - ಅವನನ್ನು ಇದಕ್ಕಿಂತ ಸ್ಪಷ್ಟವಾಗಿ ನೋಡುತ್ತಿರುವೆ. ದೇವರನ್ನು ಸಾಕ್ಷಾತ್ಕಾರ ಮಾಡಿಕೊಳ್ಳಬಹುದು. ನಾನು ನಿನ್ನೊಡನೆ ಮಾತನಾಡುತ್ತಿರುವಂತೆ, ಅವನೊಡನೆ ಮಾತನಾಡಬಹುದು. ಅವನನ್ನು ನೋಡಬಹುದು. ಆದರೆ ಎಷ್ಟು ಜನಕ್ಕೆ ದೇವರು ಬೇಕು ಹೇಳು? ಜನ ಕೊಡದಷ್ಟು ಕಣ್ಣೀರನ್ನು ತಮ್ಮ ಹೆಂಡತಿ ಮಕ್ಕಳು ಐಶ್ವರ್ಯ, ಆಸ್ತಿ ಮುಂತಾದವುಗಳಿಗೆ ಸುರಿಸುತ್ತಾರೆ. ಆದರೆ ದೇವರಿಗಾಗಿ ವ್ಯಾಕುಲತೆಯಿಂದ ಅತ್ತರೆ ಅವನು ನಿಜವಾಗಿ ಮೈದೋರುವನು.’ ಈ ಮಾತು ನನ್ನ ಮೇಲೆ ತತ್‍ಕ್ಷಣವೇ ಅದ್ಭುತವಾದ ಪರಿಣಾಮವನ್ನುಂಟುಮಾಡಿತು. ಜೀವನದಲ್ಲಿ ಪ್ರಥಮ ಬಾರಿ ಒಬ್ಬರ ಬಾಯಿಂದ ‘ನಾನು ದೇವರನ್ನು ಕಂಡಿರುವೆನು’ ಎಂಬುದನ್ನು ಕೇಳಿದೆ. ಧರ್ಮ ನಾವು ಅನುಭವಿಸಬೇಕಾದ ಒಂದು ಅನುಭವ, ನಾವು ಪ್ರಪಂಚವನ್ನು ಪ್ರತ್ಯಕ್ಷ ಅನುಭವಿಸುವಂತೆ, ಅದಕ್ಕಿಂತ ಲಕ್ಷ ಪಾಲು ಜಾಸ್ತಿಯಾಗಿ ದೇವರನ್ನು ಅನುಭವಿಸಬಹುದು ಎಂಬುದನ್ನು ಕೇಳಿದೆ. ಅವರ ಬಾಯಿಂದ ಮೇಲೆ ಹೇಳಿದ ಮಾತುಗಳನ್ನು ಕೇಳುತ್ತಿದ್ದಾಗ, ಅವರು ಸಾಧಾರಣ ಉಪನ್ಯಾಸಕರಂತೆ ಮಾತನಾಡುತ್ತಿರಲಿಲ್ಲ. ತಮ್ಮ ಅನುಭವದ ಮಾತುಗಳನ್ನು ಆಡುತ್ತಿದ್ದರು ಎನ್ನಿಸಿತು. ಆದರೆ ಅವರ ಮಾತು ಮತ್ತು ನನ್ನೊಡನೆ ಅವರು ವಿಚಿತ್ರವಾಗಿ ವರ್ತಿಸಿದ ರೀತಿಯನ್ನು ಹೊಂದಿಸಿಕೊಳ್ಳಲಾರದೆ ಹೊದೆ. ಅವರು ದೇವರನ್ನೇ ಕುರಿತು ಚಿಂತಿಸಿ ಚಿಂತಿಸಿ ಭ್ರಾಂತರಾಗಿ ಹೋಗಿರುವವರು ಇರಬೇಕು ಎನ್ನಿಸಿತು. ಆದರೂ ಅವರಲ್ಲಿದ್ದ ಅದ್ಭುತವಾದ ತ್ಯಾಗವನ್ನು ನಾನು ಒಪ್ಪಿಕೊಳ್ಳಲೇಬೇಕಾಯಿತು. ಅವರು ಭ್ರಾಂತರಾಗಿರಬಹುದು. ಆದರೆ ಇಂತಹ ಅದ್ಭುತವಾದ ತ್ಯಾಗ ಎಲ್ಲೋ ಕೆಲವು ಅಲ್ಪ ಮಂದಿಗಳಲ್ಲಿ ಮಾತ್ರ ಸಾಧ್ಯ ಎಂದುಕೊಂಡೆ. ಅವರು ಹುಚ್ಚರಾಗಿದ್ದರೂ ಇವರು ಮಹಾತ್ಮರುಗಳಲ್ಲಿ ಮಹಾತ್ಮರು, ನಿಜವಾದ ಸಾಧುಗಳು, ಇದಕ್ಕಾಗಿಯಾದರೂ ಮಾನವಕೋಟಿ ಇವರನ್ನು ಗೌರವಿಸಬೇಕು. ಇಂತಹ ಅನುಮಾನದ ಭಾವನೆಗಳಿಂದ ಅವರಿಗೆ ನಮಸ್ಕರಿಸಿ ಕಲ್ಕತ್ತೆಗೆ ಹಿಂತಿರುಗಲು ಅನುಮತಿಯನ್ನು ಕೋರಿದೆ.”

ಅನಂತರ ನರೇಂದ್ರ ತನ್ನ ಮನಸ್ಸಿನಲ್ಲಿಯೇ ಶ‍್ರೀರಾಮಕೃಷ್ಣರ ವಿಷಯವಾಗಿ ತರ್ಕಿಸಲು ಆರಂಭಿಸಿದ. ಅವರು ಒಂದು ವೇಳೆ ಹುಚ್ಚರಾಗಿದ್ದರೆ ತಾನು ಅವರ ಸಾನ್ನಿಧ್ಯದಲ್ಲಿದ್ದಾಗ ತನ್ನ ಮನಸ್ಸು ಉತ್ತಮ ಸ್ಥಿತಿಗೆ ಏರಲು ಹೇಗೆ ಸಾಧ್ಯವಾಯಿತು? ಸುತ್ತಮುತ್ತಲೂ ಸ್ಪಂದಿಸುತ್ತಿದ್ದ ಆ ಪವಿತ್ರತೆಯ ವಾತಾವರಣವೆಲ್ಲ ಎಲ್ಲಿಂದ ಬಂದಿತು? ಅಂತೂ ಯಾವ ನಿರ್ಧಾರಕ್ಕೂ ಬರಲು ಆಗಲಿಲ್ಲ. ನರೇಂದ್ರ ಶ‍್ರೀರಾಮಕೃಷ್ಣರನ್ನು ಬಿಡುವಾಗ ಬೇಗ ಬರುತ್ತೇನೆ ಎಂದು ಮಾತು ಕೊಟ್ಟಿದ್ದರೂ ಯಾವ ಯಾವುದೋ ಕೆಲಸಗಳಲ್ಲಿ ಮೈಮರೆತುಬಿಟ್ಟಿದ್ದ. ಸುಮಾರು ಒಂದು ತಿಂಗಳಾದ ಮೇಲೆ ಕಲ್ಕತ್ತೆಯಿಂದ ದಕ್ಷಿಣೇಶ್ವರಕ್ಕೆ ನಡೆದುಕೊಂಡು ಬಂದನು. ನರೇಂದ್ರ ಎರಡನೇ ವೇಳೆ ಶ‍್ರೀರಾಮಕೃಷ್ಣರನ್ನು ನೋಡಿದುದನ್ನು ಹೀಗೆ ವಿವರಿಸಿದನು:

“ನಾನು ಮೊದಲನೇ ವೇಳೆ ಗಾಡಿಯಲ್ಲಿ ದಕ್ಷಿಣೇಶ್ವರಕ್ಕೆ ಹೋಗಿದ್ದೆ. ಆಗ ಅದು ಅಷ್ಟು ದೂರವಿದೆ ಎಂದು ಗೊತ್ತಿರಲಿಲ್ಲ. ಈಗ ದಾರಿ ಎಂದಿಗೂ ಮುಗಿಯದೆ ಇರುವಂತೆ ಕಂಡಿತು. ಅಂತೂ ದಕ್ಷಿಣೇಶ್ವರ ತೋಟವನ್ನು ತಲುಪಿದೆ. ನೇರವಾಗಿ ಶ‍್ರೀರಾಮಕೃಷ್ಣರ ಕೋಣೆಯನ್ನು ಪ್ರವೇಶಿಸಿದೆ. ಅವರು ಸಣ್ಣ ಮಂಚದ ಮೇಲೆ ಒಬ್ಬರೇ ಕುಳಿತಿದ್ದುದನ್ನು ಕಂಡೆ. ಅವರಿಗೆ ನನ್ನನ್ನು ನೋಡಿ ತುಂಬಾ ಸಂತೋಷವಾಯಿತು. ಹಾಸಿಗೆಯ ಮೇಲೆ ತಮ್ಮ ಮಂಚದ ಮೇಲೆಯೇ ಕುಳಿತುಕೊಳ್ಳುವಂತೆ ಹೇಳಿದರು. ಸ್ವಲ್ಪ ಹೊತ್ತಿನಲ್ಲಿಯೇ ಅವರನ್ನು ಯಾವುದೋ ಭಾವ ಆವರಿಸಿಕೊಂಡುಬಿಟ್ಟಿತು. ಅವರು ತಮ್ಮೊಳಗೇ ಏನನ್ನೋ ಮಾತನಾಡಿಕೊಂಡು, ನನ್ನನ್ನೇ ತಮ್ಮ ಕಣ್ಣುಗಳಿಂದ ನೋಡುತ್ತ ನನ್ನ ಬಳಿಗೆ ಸಮೀಪಿಸಿದರು. ಈ ಸಲ ಮೊದಲನೆ ಸಲ ಮಾಡಿದಂತೆ ಏನಾದರೂ ವಿಚಿತ್ರವಾಗಿ ವರ್ತಿಸಬಹುದು ಎಂದು ನಾನು ಭಾವಿಸಿದೆ. ಆದರೆ ಕ್ಷಣಾರ್ಧದಲ್ಲಿಯೇ ಅವರು ತಮ್ಮ ಬಲಗಾಲನ್ನು ನನ್ನ ಮೇಲೆ ಇಟ್ಟರು. ಈ ಸ್ಪರ್ಶ ನನಗೆ ಒಂದು ಅಪೂರ್ವ ಅನುಭವವನ್ನು ಕೊಟ್ಟಿತು. ನನ್ನ ಕಣ್ಣೆದುರಿಗೆ, ಗೋಡೆ ಮತ್ತು ಕೋಣೆಯಲ್ಲಿದ್ದ ವಸ್ತುಗಳೆಲ್ಲವೂ ಮಾಯವಾಗತೊಡಗಿದವು. ನನ್ನ ವ್ಯಕ್ತಿತ್ವದೊಡನೆ ಪ್ರಪಂಚವೆಲ್ಲ ಮಹಾಶೂನ್ಯದಲ್ಲಿ ಲಯವಾಗುತ್ತಿರುವುದನ್ನು ನೋಡಿದೆ. ನನಗೆ ಬಹಳ ಭೀತಿಯಾಯಿತು. ನಾನೇನು ಸತ್ತುಹೋಗುತ್ತಿರುವೆನೇನೋ ಎಂದು ಭಾವಿಸಿದೆ. ವ್ಯಕ್ತಿತ್ವ ನಾಶವಾಗುವುದೆಂದರೆ ಮೃತ್ಯುವಿಗಿಂತ ಕಡಮೆ ಏನಲ್ಲ ಅದು. ಅದನ್ನು ಸಹಿಸಲು ನನಗೆ ಸಾಧ್ಯವಾಗದೆ, ‘ನೀವು ನನಗೆ ಏನನ್ನು ಮಾಡುತ್ತಿರುವಿರಿ! ನನಗೆ ಮನೆಯಲ್ಲಿ ತಂದೆತಾಯಿಗಳು ಇರುವರು’ ಎಂದು ಅರಚಿಕೊಂಡೆ. ಅವರು ಗಟ್ಟಿಯಾಗಿ ನಕ್ಕು ನನ್ನ ಎದೆಯನ್ನು ಸವರಿ ‘ಆಗಲಿ ಸದ್ಯಕ್ಕೆ ಇಷ್ಟು ಸಾಕು. ಕ್ರಮೇಣ ಎಲ್ಲಾ ಬರುವುದು’ ಎಂದರು. ಏನು ವಿಚಿತ್ರ, ಅವರು ಹೀಗೆ ಹೇಳಿದ್ದೇ ತಡ, ನನ್ನ ವಿಚಿತ್ರವಾದ ಅನುಭವಗಳೆಲ್ಲ ಮಾಯವಾದವು. ನಾನು ಹಿಂದಿನಂತೆಯೇ ಇದ್ದೆ. ಕೋಣೆಯಲ್ಲಿ ಎಲ್ಲಾ ಮುಂಚಿನಂತೆಯೇ ಇತ್ತು.

“ಇವುಗಳೆಲ್ಲ ನಾನು ಹೇಳುವುದಕ್ಕೆ ಎಷ್ಟು ಸಮಯ ತೆಗೆದುಕೊಳ್ಳುವುದೊ ಅದಕ್ಕಿಂತ ಬೇಗ ಆಗಿಹೋದವು. ಇದು ಏನಿರಬೇಕು ಎಂದು ನಾನು ದಿಗ್ಭ್ರಾಂತನಾಗಿ ಯೋಚಿಸತೊಡಗಿದೆ. ಈ ಅದ್ಭುತ ಮನುಷ್ಯನ ಕೇವಲ ಇಚ್ಛಾನುಸಾರ ಈ ಅನುಭವ ಬಂತು ಮತ್ತು ಹೋಯಿತು. ಇದೇನು ಮಾಯಮಂತ್ರ ಶಕ್ತಿಯೇ, ಸಮ್ಮೋಹಿನಿ ವಿದ್ಯೆಯೇ? ಆದರೆ ಇದು ಅದಾಗಿರಲಿಕ್ಕಿಲ್ಲ, ಏಕೆಂದರೆ ಅವು ಕೇವಲ ದುರ್ಬಲವಾದ ಮನಸ್ಸಿನ ಮೇಲೆ ಮಾತ್ರ ಪರಿಣಾಮಕಾರಿಯಾಗುತ್ತವೆ. ಆದರೆ ನಾನಾದರೋ ಅದಕ್ಕೆ ನಿರೋಧ ಎಂದು ಹೆಮ್ಮೆ ಕೊಚ್ಚಿಕೊಳ್ಳುತ್ತಿದ್ದೆ. ಈ ಮನುಷ್ಯರ ಪ್ರಬಲ ಇಚ್ಛೆಗೆ ನಾನಿನ್ನೂ ವಶನಾಗಿರಲಿಲ್ಲ. ಅವರನ್ನು ಒಬ್ಬ ಭ್ರಾಂತರು ಎಂದು ಭಾವಿಸಿದ್ದೆ. ಆದರೆ ಇಂತಹ ವಿಚಿತ್ರ ಅನುಭವ ಆಯಿತಲ್ಲ! ಇದು ಏತಕ್ಕೆ ಇರಬಹುದು? ನಾನು ಯಾವ ನಿರ್ಣಯಕ್ಕೂ ಬರಲು ಸಾಧ್ಯವಾಗಲಿಲ್ಲ. ಇದೊಂದು ಜಟಿಲವಾದ ಸಮಸ್ಯೆ. ಇದನ್ನು ಬಗೆಹರಿಸದೆ ಇರುವುದೇ ಮೇಲು. ಆದರೆ ನಾನು ಜಾಗರೂಕನಾಗಿದ್ದು, ಇನ್ನೊಂದು ಸಲ ಇವರಿಗೆ ನನ್ನ ಮೇಲೆ ತಮ್ಮ ಶಕ್ತಿಯನ್ನು ಚಲಾಯಿಸಲು ಅವಕಾಶ ಕೊಡುವುದಿಲ್ಲ ಎಂದು ತೀರ್ಮಾನಿಸಿದೆ.”

“ಮರುಕ್ಷಣವೇ, ನನ್ನಂತಹ ಸ್ಥಿರವಾದ ಪ್ರಬಲವಾದ ಮನಸ್ಸನ್ನು ಕ್ಷಣಾರ್ಧದಲ್ಲಿ ಚೂರು ಚೂರು ಮಾಡಬಲ್ಲಂತಹ ಶಕ್ತಿ ಇರುವವರನ್ನು ಕೇವಲ ಭ್ರಾಂತರು ಎಂದು ಹೇಗೆ ಹೇಳುವುದು - ಎಂದು ಆಲೋಚಿಸತೊಡಗಿದೆ. ಆದರೂ ನಾನು ಮೊದಲನೇ ವೇಳೆ ಅವರನ್ನು ಕಂಡ ವಿಷಯವಾಗಿ ನನ್ನ ನಿರ್ಣಯ ಅವರು ಹುಚ್ಚರಾಗಿರಬೇಕು, ಇಲ್ಲದೇ ಇದ್ದರೆ ಒಂದು ಅವತಾರ ವ್ಯಕ್ತಿಯಾಗಿರಬೇಕು ಎಂಬುದು. ಆದರೆ ಅವತಾರ ವ್ಯಕ್ತಿಗೂ ಇದಕ್ಕೂ ಬಹಳ ದೂರ. ಮಗುವಿನಂತೆ ಸರಳವಾದ ಮತ್ತು ಪರಿಶುದ್ಧವಾಗಿದ್ದಂತಹ ಇವರ ವಿಷಯದಲ್ಲಿ ಒಂದು ದೊಡ್ಡ ಸಂದೇಹ ತಲೆದೋರಿತು. ನಾನು ತುಂಬಾ ವಿವೇಕಿ ಎಂದು ಭಾವಿಸಿದ್ದೆ. ಈ ಅನುಭವದ ಸತ್ಯಾಂಶವನ್ನು ನಿರ್ಧರಿಸಲಾರದ ವಿವೇಕ ಎಂತಹದು? ಇದು ನನ್ನ ವಿಚಾರಕ್ಕೆ ಒಂದು ಕೊಡಲಿಯ ಪೆಟ್ಟಿನಂತೆ ಇತ್ತು. ಹೇಗಾದರೂ ಈ ರಹಸ್ಯವನ್ನು ಭೇದಿಸಬೇಕೆಂದು ನಾನು ನಿಶ್ಚಯಿಸಿದೆ.”

“ಅಂದಿನ ದಿನವೆಲ್ಲ ಮೇಲಿನ ಭಾವನೆಗಳು ನನ್ನನ್ನು ಆವರಿಸಿದ್ದವು. ಆದರೆ ಆ ಘಟನೆ ಆದಮೇಲೆ ಅವರು ಸಹಜಸ್ಥಿತಿಗೆ ಬಂದರು. ಹಿಂದಿನಂತೆ ನನ್ನನ್ನು ಬಹಳ ಪ್ರೀತಿ ಆದರಗಳಿಂದ ನೋಡಿದರು. ಒಬ್ಬ ಬಹಳ ಕಾಲದ ಮೇಲೆ ತನ್ನ ಬಂಧುವನ್ನೋ, ಸ್ನೇಹಿತನನ್ನೋ ಸಂಧಿಸಿದಂತೆ ಇತ್ತು ಅವರು ನನ್ನನ್ನು ನೋಡುತ್ತಿದ್ದ ರೀತಿ. ನನಗೆ ಏನು ಕೊಟ್ಟರೂ ಅವರಿಗೆ ತೃಪ್ತಿ ಇರಲಿಲ್ಲ. ಅವರ ಈ ಪ್ರೇಮಪೂರ್ಣವಾದ ನಡತೆ ನನ್ನನ್ನು ಮತ್ತೂ ಆಕರ್ಷಿಸಿತು. ಅಂದು ರಾತ್ರಿಯಾಗುತ್ತಿದ್ದುದರಿಂದ ಹಿಂತಿರುಗಿ ಹೋಗಲು ಅನುಮತಿಯನ್ನು ಬೇಡಿದೆನು. ಅವರು ನಾನು ಹೋಗುತ್ತೇನೆಂಬುದನ್ನು ಕೇಳಿಯೇ ಖಿನ್ನರಾದರು. ನಾನು ಪುನಃ ಬೇಗ ಬರುತ್ತೇನೆಂದು ಹೇಳಿದ ಮೇಲೆಯೇ ಅವರು ನನಗೆ ಹಿಂತಿರುಗಲು ಅನುಮತಿಯನ್ನು ಕೊಟ್ಟರು.”

ಕೆಲವು ದಿನಗಳಾದ ಮೇಲೆ ನರೇಂದ್ರ ಮೂರನೇ ಬಾರಿ ಶ‍್ರೀರಾಮಕೃಷ್ಣರನ್ನು ನೋಡುವುದಕ್ಕೆ ದಕ್ಷಿಣೇಶ್ವರಕ್ಕೆ ಬಂದನು. ಈ ಬಾರಿ ಶ‍್ರೀರಾಮಕೃಷ್ಣರ ಪ್ರಭಾವಗಳಿಗೆ ತುತ್ತಾಗುವುದಿಲ್ಲವೆಂದು ಸಂಕಲ್ಪ ಮಾಡಿಕೊಂಡನು. ಶ‍್ರೀರಾಮಕೃಷ್ಣರು ಅಂದು ನರೇಂದ್ರನನ್ನು ಪಕ್ಕದ ಯದುನಾಥನ ತೋಟಕ್ಕೆ ಕರೆದುಕೊಂಡು ಹೋದರು. ಅಲ್ಲಿ ಸ್ವಲ್ಪ ಹೊತ್ತು ಸಂಚಾರ ಮಾಡಿದ ಮೇಲೆ ಒಂದು ಕಡೆ ಕುಳಿತುಕೊಂಡರು. ಶ‍್ರೀರಾಮಕೃಷ್ಣರು ಭಾವಮುಖದಲ್ಲಿ ನರೇಂದ್ರನನ್ನು ಮುಟ್ಟಿದರು. ಎಷ್ಟೇ ಪ್ರಯತ್ನ ಮಾಡಿದರೂ ನರೇಂದ್ರನಿಗೆ ತನ್ನ ವ್ಯಕ್ತಿತ್ತ್ವವನ್ನು ಕಾದಿರಿಸಿಕೊಂಡಿರಲು ಸಾಧ್ಯವಾಗಲಿಲ್ಲ. ಸಂಪೂರ್ಣವಾಗಿ ಅವನು ವಶವಾಗಿ ಹೋದನು. ತನ್ನ ವ್ಯಕ್ತಿತ್ವವೆಲ್ಲ ಮರೆತುಹೋಯಿತು. ಕೆಲವು ಕಾಲದ ಮೇಲೆ ಆತನಿಗೆ ಪ್ರಜ್ಞೆ ಬಂದಾಗ ಶ‍್ರೀರಾಮಕೃಷ್ಣರು ನರೇಂದ್ರನ ಎದೆಯನ್ನು ಸವರುತ್ತಿದ್ದರು.

ಶ‍್ರೀರಾಮಕೃಷ್ಣರು ಮೊದಲನೆ ವೇಳೆ ನರೇಂದ್ರನನ್ನು ಸ್ಪರ್ಶಮಾಡಿ ಇಂದ್ರಿಯಾತೀತ ಅನುಭವಕ್ಕೆ ಕಳುಹಿಸುತ್ತಿದ್ದಾಗ, ಅವನು ನನಗೆ ಮನೆಯಲ್ಲಿ ತಾಯಿ ತಂದೆಯರು ಇರುವರು ಎಂದು ಕಿರುಚಿಕೊಂಡಿದ್ದ. ಆಗ ಶ‍್ರೀರಾಮಕೃಷ್ಣರಲ್ಲಿ ಸಂದೇಹ ತೋರಿತು. ಯಾರನ್ನು ತಾವು ದರ್ಶನದಲ್ಲಿ ಕಂಡಿದ್ದರೋ ಅವನೇ ಇವನು? ಹಾಗಿದ್ದರೇ ಏತಕ್ಕೆ ಅವನು ಕಿರುಚಿಕೊಂಡ ಎಂದು ತರ್ಕಿಸಿದರು. ಆದಕಾರಣವೇ ಮೂರನೆ ಬಾರಿ ನರೇಂದ್ರನನ್ನು ಇಂದ್ರಿಯಾತೀತ ಅವಸ್ಥೆಗೆ ಒಯ್ದು ಅವನಿಗೆ ಹಲವು ಪ್ರಶ್ನೆಗಳನ್ನು ಹಾಕಿದರು. ಅವನು ಹಿಂದೆ ಯಾರು ಆಗಿದ್ದ? ಈಗ ಏತಕ್ಕೆ ಬಂದಿರುವನು? ಎಷ್ಟು ಕಾಲ ಇಲ್ಲಿರುವನು? ಇತ್ಯಾದಿ. ಇವುಗಳಿಗೆಲ್ಲ ಸಮರ್ಪಕವಾಗಿ ಉತ್ತರ ಬಂದಮೇಲೆ ಶ‍್ರೀರಾಮಕೃಷ್ಣರಿಗೆ ಅವನ ಮೇಲೆ ಇದ್ದ ಸಂಶಯ ಬಗೆ ಹರಿಯಿತು. ಪುನಃ ನರೇಂದ್ರನನ್ನು ಪ್ರಕೃತಿಸ್ಥನಾಗಿ ಮಾಡಿದರು. ಆದರೆ ನರೇಂದ್ರ ತನ್ನ ಅತೀಂದ್ರಿಯ ಸ್ಥಿತಿಯಲ್ಲಿ ಶ‍್ರೀರಾಮಕೃಷ್ಣರಿಗೆ ಏನು ಹೇಳಿದನೊ ಅದಾವುದೂ ಅವನಿಗೇ ನೆನಪಿರಲಿಲ್ಲ. ಶ‍್ರೀರಾಮಕೃಷ್ಣರು ಅವನಿಗೆ ಇದಾವುದನ್ನೂ ಹೇಳಲಿಲ್ಲ. ಏಕೆಂದರೆ ನರೇಂದ್ರನಿಗೆ ತನ್ನ ಹಿಂದಿನ ಸ್ಥಿತಿ ಅರಿವಾಯಿತೆಂದರೆ ಅವನು ಈಗ ಮಾಡಬೇಕಾಗಿರುವ ಕೆಲಸವನ್ನು ಮಾಡುವುದಿಲ್ಲವೆಂದು ಭಾವಿಸಿ, ತಾನು ಯಾವ ಕೆಲಸಕ್ಕೆ ಬಂದಿರುವನೋ ಆ ಕೆಲಸ ತೀರುವುದಕ್ಕೆ ಮುಂಚೆ ಅವನಿಗೆ ಆ ಸ್ಥಿತಿ ತಿಳಿಯದಂತೆ ಮಾಡಿದ್ದರು.

ನರೇಂದ್ರ ಯಾವಾಗಲೂ ವಿಚಾರದ ಒರೆಗಲ್ಲನ್ನು ತನ್ನ ಹತ್ತಿರ ಇಟ್ಟುಕೊಂಡಿದ್ದ. ಸಮರ್ಪಕವಾಗಿ ವಿವರಣೆ ಕೊಡುವುದಕ್ಕೆ ಸಾಧ್ಯವಾದಲ್ಲದೇ ಯಾವುದನ್ನೂ ಅವನು ಸ್ವೀಕರಿಸುತ್ತಿರಲಿಲ್ಲ. ದೇವರ ಅನುಭವವಾದರೂ, ಯಾರಿಗೆ ಗೊತ್ತು, ಇದು ಮನೋವಿಕಾರವಿರಬಹುದು ಎಂದು ಅನುಮಾನಿಸುವನು. ದೇವ ದೇವಿಯರು ವಿಗ್ರಹಗಳು ಇವುಗಳಾವುದರಲ್ಲಿಯೂ ನಂಬಿಕೆ ಇಲ್ಲದವನು. ಶ‍್ರೀರಾಮಕೃಷ್ಣರಿಗೆ ನರೇಂದ್ರನ ಸ್ವಭಾವವನ್ನು ಕಂಡು ಅಸಹನೆಯ ಬದಲು ಪ್ರೀತಿ ಹೆಚ್ಚಿತು. ಶಿಷ್ಯನಲ್ಲಿ ಸತ್ತ್ವವಿದೆ ಎಂದು ಭಾವಿಸಿದರು. ಕ್ರಮೇಣ ನರೇಂದ್ರನ ವ್ಯಕ್ತಿತ್ವವನ್ನು ಶ‍್ರೀರಾಮಕೃಷ್ಣರು ದೊಡ್ಡಶಿಲ್ಪಿ ಅಮೃತ ಶಿಲೆಯಲ್ಲಿ ಹೇಗೆ ಒಂದು ಪ್ರತಿಮೆಯನ್ನು ಕೆತ್ತುತ್ತಾನೋ ಹಾಗೆ ಬಿಡಿಸಲು ಪ್ರಯತ್ನಿಸಿದರು. ಅದರಲ್ಲಿ ಪೂರ್ಣ ಜಯಶಾಲಿಗಳಾದರು. ಶ‍್ರೀರಾಮಕೃಷ್ಣರು ಪ್ರಚಂಡ ಸಾಧನೆಯಿಂದ ಗಳಿಸಿದ ಅಮೃತ ನರೇಂದ್ರನೆಂಬ ನಾಲೆಯ ಮೂಲಕ ಬಹು ಜನರ ಹಿತಕ್ಕೆ ಬಹು ಜನರ ಸುಖಕ್ಕೆ ಹರಿಯತೊಡಗಿತು.

