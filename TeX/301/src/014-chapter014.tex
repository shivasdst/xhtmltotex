
\chapter{ರಾಜಾಸ್ಥಾನದಲ್ಲಿ}

ಸ್ವಾಮೀಜಿ ಗುರುಭಾಯಿಗಳಿಂದ ಬೀಳ್ಕೊಂಡು ದೆಹಲಿಯನ್ನು ಸೇರಿದರು, ವಿವಿದಿಶಾನಂದ ಎಂಬ ಹೊಸ ಹೆಸರಿನಿಂದ ತಮ್ಮ ಹಳೆಯ ಹೆಸರನ್ನು ಮರೆಮಾಚಿಕೊಂಡು, ತಮ್ಮ ಗುರುಭಾಯಿಗಳಿಗೆ ಇವರ ಸುಳಿವು ತಿಳಿಯದಿರಲಿ ಎಂದು. ದೆಹಲಿಯಲ್ಲಿ ಶಾಮಲದಾಸ ಎಂಬುವರ ಮನೆಯಲ್ಲಿ ಇದ್ದರು. ಅಲ್ಲಿ ಭರತಖಂಡದಲ್ಲಿ ಇದ್ದ ಚಕ್ರಾಧಿಪತ್ಯಗಳೆಲ್ಲ ತಮ್ಮ ಅವಶೇಷವನ್ನು ಬಿಟ್ಟು ಹೋಗಿವೆ. ಚಕ್ರಾಧಿಪತ್ಯಗಳು ಎದ್ದು, ಕೆಲವು ಕಾಲ ಕೊರೈಸಿ ಕೊನೆಗೆ ನಿರ್ನಾಮವಾಗಿವೆ. ಚಕ್ರಾಧಿಪತ್ಯಗಳು ಕಾಲವೆಂಬ ನೀರಿನಮೇಲೆ ಕೆಲವು ಕಾಲ ಹರಿದುಕೊಂಡು ಹೋಗುವ ನೀರಿನ ಗುಳ್ಳೆಯಂತೆ. ಚಕ್ರಾಧಿಪತ್ಯಗಳೆಲ್ಲ ತಮ್ಮ ಪ್ರಭಾವವನ್ನು ಬೀರಿದ್ದರೂ ಹಿಂದೂ ಸಂಸ್ಕೃತಿ ಮಾತ್ರ ಅದರಲ್ಲಿರುವ ಒಳ್ಳೆಯದನ್ನೆಲ್ಲ ಹೀರಿಕೊಂಡು ತನ್ನದನ್ನು ಕಳೆದುಕೊಳ್ಳದೆ ಉಳಿದಿರುವುದನ್ನು ಸ್ವಾಮೀಜಿ ಮನಗಂಡರು. 

ಮೀರತ್ತಿನಿಂದ ಉಳಿದ ಗುರುಭಾಯಿಗಳು ದೆಹಲಿಯನ್ನು ನೋಡುವುದಕ್ಕೆ ಬಂದರು. ಅಲ್ಲಿ ಯಾರೋ ಇಂಗೀಷನ್ನು ಮಾತನಾಡುವ ಬಂಗಾಳಿ ಸ್ವಾಮಿಗಳು ಇರುವರೆಂದು ಕೇಳಿ ಅವರನ್ನು ನೋಡಲು ಬಂದರು. ಬಂದು ನೋಡಿದಾಗ ಅವರು ತಮ್ಮ ನರೇಂದ್ರನೇ ಆಗಿದ್ದುದನ್ನು ನೋಡಿ ವಿಸ್ಮಿತರಾದರು. ಸ್ವಾಮೀಜಿ ತಮ್ಮನ್ನು ಅನುಸರಿಸದುದಕ್ಕಾಗಿ ಅಸಂತೋಷವನ್ನು ವ್ಯಕ್ತಪಡಿಸಿದರು. ಆಗ ವಿವೇಕಾನಂದರಿಗೆ, ತಾವು ಇವರನ್ನು ನೋಡುವ ಉದ್ದೇಶದಿಂದ ಬರಲಿಲ್ಲವೆಂದೂ, ಯಾರೋ ಬೇರೆ ಸ್ವಾಮಿಗಳು ಇರಬಹುದೆಂದು ಭಾವಿಸಿ ಬಂದೆವೆಂದೂ ಅವರು ಹೇಳಿದರು. ಇವರೆಲ್ಲಾ ಒಂದೇ ಕಡೆ ಊಟ ಮಾಡಿದರೂ ಒಬ್ಬೊಬ್ಬರು ಒಂದೊಂದು ಕಡೆ ವಾಸಿಸುತ್ತಿದ್ದರು. ಒಂದು ದಿನ ಸ್ವಾಮೀಜಿ ತಮ್ಮ ಗಂಟಲನ್ನು ತೋರಿಸುವುದಕ್ಕೆ ಡೆಲ್ಲಿಯ ಪ್ರಖ್ಯಾತ ವೈದ್ಯನಾದ ಹೇಮಚಂದ್ರಸೇನನೆಡೆಗೆ ಹೋಗಿದ್ದರು. ಆತ ಮೊದಮೊದಲು ಸ್ವಾಮೀಜಿಯವರನ್ನು ಅಷ್ಟು ಚೆನ್ನಾಗಿ ಕಾಣಲಿಲ್ಲ. ಅವರ ಧೈರ‍್ಯ, ಋಜುತ್ವ ಮುಂತಾದುವನ್ನು ನೋಡಿ, ಇದೊಂದು ಉದ್ಧಟತನವೆಂದು ಭಾವಿಸಿ ಅವರನ್ನು ಹಳಿದಿದ್ದ. ಆದರೆ ಈಗ ಅವರೊಂದಿಗೆ ಮಾತುಕತೆಯಾಡಬೇಕೆಂದು ಬಯಸಿದನು. ಒಂದು ದಿನ ಡಾಕ್ಟರ್ ತನ್ನ ಮನೆಗೆ ಕಾಲೇಜಿನಿಂದ ಹಲವು ವಿದ್ಯಾವಂತರಾದ ಪ್ರಾಧ್ಯಾಪಕರನ್ನು ಕರೆಸಿ ಆ ಸಮಯದಲ್ಲಿ ವಿವೇಕಾನಂದರನ್ನು ಕರೆದನು. ಆ ಸಂದರ್ಭದಲ್ಲಿ ಹಲವರು ಧಾರ್ಮಿಕ ಮತ್ತು ತಾತ್ತ್ವಿಕ ಪ್ರಶ್ನೆಗಳನ್ನು ಸ್ವಾಮಿಗಳಿಗೆ ಹಾಕಿದರು. ಸ್ವಾಮಿಗಳು ಸನ್ನಿವೇಶಕ್ಕೆ ಸರಿಯಾಗಿ ಮೇಲೆದ್ದರು. ತಮ್ಮಲ್ಲಿದ್ದ ಪೌರಾತ್ಯ ಮತ್ತು ಪಾಶ್ಚಾತ್ಯ ಜ್ಞಾನಭಂಡಾರದಿಂದ ಅವರನ್ನೆಲ್ಲಾ ಎದುರಿಸಿ ಸಮರ್ಪಕವಾಗಿ ತಮ್ಮ ಅಭಿಪ್ರಾಯಗಳನ್ನು ಸಮರ್ಥಿಸಿದರು. ಅನಂತರ ಆ ವೈದ್ಯನಿಗೆ ಸ್ವಾಮೀಜಿಯವರ ಪ್ರತಿಭೆಯನ್ನು ಕಂಡು ಸಂತೋಷವಾಯಿತು. 

 ೧೮೯೧ರ ಫೆಬ್ರವರಿ ಪ್ರಾರಂಭದಲ್ಲಿ ಸ್ವಾಮೀಜಿ ದೆಹಲಿಯನ್ನು ಬಿಟ್ಟು ಆಳ್ವಾರಿಗೆ ಬಂದರು. ರೈಲ್ವೆ ನಿಲ್ದಾಣದಿಂದ ಆಳ್ವಾರಿನ ದಾರಿಯಲ್ಲಿ ಸ್ವಲ್ಪದೂರ ನಡೆದುಕೊಂಡು ಹೋದರು. ಒಂದು ಆಸ್ಪತ್ರೆಯ ಮುಂದೆ ಗುರುಚರಣ ಲಾಸ್ಕರ್ ಎಂಬ ಬಂಗಾಳಿ ವೈದ್ಯನಿದ್ದನು. ಸ್ವಾಮೀಜಿ “ಸಾಧುಗಳು ತಂಗುವುದಕ್ಕೆ ಇಲ್ಲಿ ಎಲ್ಲಿಯಾದರೂ ಸ್ಥಳವಿದೆಯೆ” ಎಂದು ಕೇಳಿದರು. ಆ ವೈದ್ಯನು ಸ್ವಾಮೀಜಿಯವರ ತೇಜಸ್ಸಿನ ಮುಖವನ್ನು ನೋಡಿ ಆಕರ್ಷಿತನಾದ. ಅವರಿಗೆ ನಮಸ್ಕಾರ ಮಾಡಿ ಬಜಾರಿನಲ್ಲಿ ಒಂದು ಅಂಗಡಿಯ ಮೇಲೆ ಅವರನ್ನು ಬಿಟ್ಟು, ತಾನು ಇನ್ನು ಸ್ವಲ್ಪ ಹೊತ್ತಿನಲ್ಲಿ ಬಂದು ತಮ್ಮ ದರ್ಶನವನ್ನು ತೆಗೆದುಕೊಳ್ಳುತ್ತೇನೆ ಎಂದು ಹೇಳಿ ಹೋದನು. ಆತನಿಗೆ ಒಬ್ಬ ಮೌಲ್ವಿ ಸಾಹೇಬ್ ಎಂಬ ಮಹಮ್ಮದೀಯ ಸ್ನೇಹಿತನಿದ್ದನು. ಆತ ಆ ಊರಿನ ಹೈಸ್ಕೂಲಿನಲ್ಲಿ ಉರ್ದು ಪಾರ್ಸಿಭಾಷೆಯ ಪಂಡಿತನಾಗಿದ್ದ. ಅವನ ಬಳಿಗೆ ವೈದ್ಯ ಹೋಗಿ, ಒಬ್ಬ ಮಹಾ ವಿದ್ವಾಂಸನಂತೆ ತೋರುವ, ತೇಜಸ್ವಿಯಾದ ಬಂಗಾಳಿ ಸಾಧುಗಳೊಬ್ಬರು ಬಂದಿರುವರು, ನೀವು ಅವರನ್ನು ನೋಡಬೇಕೆಂದು ಹೇಳಿದ. ಇಬ್ಬರೂ ಕೂಡಿ ಸ್ವಾಮೀಜಿಯವರನ್ನು ನೋಡಲು ಬಂದರು. ತಮ್ಮ ಪಾದರಕ್ಷೆಗಳನ್ನೆಲ್ಲ ಹೊರಗೆ ಬಿಟ್ಟು ಸ್ವಾಮೀಜಿ ಇದ್ದ ಕೋಣೆಯೊಳಗೆ ಹೋಗಿ ಸ್ವಾಮೀಜಿಗೆ ಗೌರವಪೂರ್ವಕವಾಗಿ ನಮಸ್ಕರಿಸಿದರು. ಆಗ ಸ್ವಾಮೀಜಿ ಹತ್ತಿರ ಇದ್ದುದು, ಒಂದೆರಡು ಪುಸ್ತಕಗಳು, ಹೊದೆಯುವುದಕ್ಕೆ ಕಂಬಳಿ, ಮೈಮೇಲಿದ್ದ ಬಟ್ಟೆ, ಕಮಂಡಲು ಮತ್ತು ದಂಡ ಇಷ್ಟೇ. ಸ್ವಾಮೀಜಿ ಆ ಮುಸ್ಲಿಂ ಸ್ನೇಹಿತನನ್ನು ಹತ್ತಿರ ಕರೆದು ಪ್ರೀತಿಯಿಂದ ಧರ್ಮಸಂಬಂಧವಾಗಿ ಕೊರಾನ್ ಸಂಬಂಧವಾಗಿ ಮಾತನಾಡಿದರು. ವೈದ್ಯನು ಅನಂತರ ತನ್ನ ಆಸ್ಪತ್ರೆಗೆ ಹೋಗಿ ಅಲ್ಲಿಗೆ ಬಂದವರಿಗೆಲ್ಲ ಸ್ವಾಮೀಜಿಯವರ ವಿಷಯವನ್ನು ಹೇಳಿದನು. ಅನೇಕರು ಕುತೂಹಲಿಗಳಾಗಿ ಇವರ ಬಳಿಗೆ ಬಂದರು. ಸ್ವಾಮೀಜಿಯವರ ದರ್ಶನ ಪಡೆಯುವುದಕ್ಕೆ ಅವರ ಮಾತುಕತೆಗಳನ್ನು ಕೇಳುವುದಕ್ಕೆ ಅವರಿದ್ದ ಕೋಣೆ ಸಾಲದಾಯಿತು. ಜನರು ಹೊರಗೆಲ್ಲ ಕುಳಿತುಕೊಂಡು ಸ್ವಾಮೀಜಿಯವರ ಪ್ರವಚನವನ್ನು ಕೇಳುತ್ತಿದ್ದರು. ಪ್ರವಚನದ ಮಧ್ಯೆ ಮಧ್ಯೆ ಉರ್ದು, ಹಿಂದೀ, ಬಂಗಾಳಿ ಹಾಡುಗಳನ್ನು ಹಾಡುತ್ತಿದ್ದರು. ಆಗ ಅವರ ಬಳಿಗೆ ಮಹಮ್ಮದೀಯರಲ್ಲಿ ಶಿಯಾ ಸುನ್ನಿಗಳು, ಹಿಂದೂಗಳಲ್ಲಿ ಶಾಕ್ತರು, ಶೈವರು, ವೈಷ್ಣವರು, ಪಂಡಿತರು, ಪಾಮರರು ದೊಡ್ಡ ಹುದ್ದೆಯಲ್ಲಿದ್ದವರು, ಬಡವರು ಎಲ್ಲಾ ಬಗೆಯ ಜನಗಳೂ ಬಂದು ಹೋಗುತ್ತಿದ್ದರು. ಸ್ವಾಮೀಜಿ, ಎಲ್ಲಾ ಧರ್ಮದ ಮಹಾತ್ಮರ ಜೀವನ ಮತ್ತು ಸಂದೇಶಗಳಿಂದ ಉದಾಹರಿಸುತ್ತಿದ್ದರು. 

 ಬಂದು ಹೋಗುವ ಭಕ್ತಾದಿಗಳ ಸಂಖ್ಯೆ ಹೆಚ್ಚಿದಾಗ ಸ್ವಾಮೀಜಿಯವರನ್ನು ಯಾವುದಾದರೂ ದೊಡ್ಡ ಸ್ಥಳದಲ್ಲಿ ಇಳಿಸುವುದಕ್ಕೆ ವೈದ್ಯರು ಮತ್ತು ಇತರರು ಆಲೋಚಿಸಿದರು. ಆಳ್ವಾರಿನ ನಿವೃತ್ತ ಇಂಜಿನಿಯರ್ ಆದ ಪಂಡಿತ ಶಂಭುನಾಥಜಿ ಅವರ ಮನೆಯನ್ನು ಅವರ ಬಿಡಾರವಾಗಿ ಗೊತ್ತುಮಾಡಿದರು. ಸ್ವಾಮೀಜಿ ಅಲ್ಲಿಗೆ ಹೋದಮೇಲೆ ಬೆಳಿಗ್ಗೆ ಒಂಭತ್ತು ಗಂಟೆಯವರೆಗೆ ತಮ್ಮ ಕೋಣೆಯಲ್ಲಿ ಧ್ಯಾನ ಪಾರಾಯಣ ಮುಂತಾದವುಗಳನ್ನು ಮುಗಿಸಿಕೊಂಡು ಹೊರಗೆ ಬರುತ್ತಿದ್ದರು. ಆಗ ಊರಿನ ಹಲವು ಜನರು ಬಂದು ಇವರೊಡನೆ ಮಾತುಕತೆಗಳನ್ನಾಡಲು ಕಾದುಕೊಂಡಿರುತ್ತಿದ್ದರು. ಬಂದವರು ಹಲವು ಬಗೆಯ ಪ್ರಶ್ನೆಗಳನ್ನು ಹಾಕುತ್ತಿದ್ದರು. ಒಂದು ಸಲ ಸ್ವಾಮೀಜಿಯವರನ್ನು ಯಾವ ಜಾತಿಗೆ ಸೇರಿದವರೆಂದು ಯಾರೊ ಕೇಳಿದರು. ಅದಕ್ಕೆ ಸ್ವಾಮೀಜಿ “ಈ ದೇಹ ಕಾಯಸ್ಥ ಜಾತಿಗೆ ಸೇರಿದುದು” ಎಂದು ಹೇಳಿದರು. ಮತ್ತೊಬ್ಬರು “ನೀವು ಏತಕ್ಕೆ ಕಾವಿಯನ್ನು ಧರಿಸುತ್ತೀರಿ?” ಎಂದು ಕೇಳಿದರು. ಅದಕ್ಕೆ “ಕಾವಿ ಭಿಕ್ಷುಕನ ವಸನ. ಇತರ ಭಿಕ್ಷುಕರು ತಮ್ಮನ್ನು ಭಿಕ್ಷೆ ಕೇಳಲು ಬರದೆ ಇರಲಿ ಎಂದು ನಾನದನ್ನು ಧರಿಸುತ್ತೇನೆ” ಎಂದರು. 

 ಸ್ವಾಮೀಜಿಯವರನ್ನು ಆಳ್ವಾರಿನಲ್ಲಿ ಪ್ರಥಮ ಪರಿಚಯ ಮಾಡಿಕೊಂಡ ಮೌಲ್ವಿ ಸಾಹೇಬನು, ಸ್ವಾಮೀಜಿಯವರನ್ನು ತನ್ನ ಮನೆಗೆ ಒಂದು ದಿನ ಊಟಕ್ಕೆ ಕರೆಯಬೇಕೆಂದು ಇಚ್ಛಿಸಿದನು. ಹಿಂದೂ ಸಂನ್ಯಾಸಿಗಳಾದ ಸ್ವಾಮೀಜಿ ಮಹಮ್ಮದೀಯನ ಮನೆಯಲ್ಲಿ ಊಟ ಮಾಡಬಹುದೆ ಎಂದು ಅನುಮಾನಿಸಿದನು. ಆದರೆ ಸ್ವಾಮೀಜಿ ಜಾತಿಮತಗಳನ್ನು ತಿರಸ್ಕರಿಸಿ ಮೇಲೆದ್ದ ವ್ಯಕ್ತಿಗಳು, ಅವರಿಗೆ ಅಭ್ಯಂತರವಿಲ್ಲದೇ ಇರಬಹುದು. ಆದರೆ ಯಾರ ಮನೆಯಲ್ಲಿ ಅವರು ಇರುವರೋ ಅವರು ಒಪ್ಪದೇ ಇರಬಹುದು, ಎಂದು ಮುಂಚೆ ಶಂಭುನಾಥಜಿ ಹತ್ತಿರ ಅವನು ತನ್ನ ಅಭಿಪ್ರಾಯವನ್ನು ವ್ಯಕ್ತಪಡಿಸಿದನು. “ಬ್ರಾಹ್ಮಣರ ಮನೆಯಲ್ಲಿಯೇ ಅಡಿಗೆ ಮಾಡಿಸುತೇನೆ. ನಾನು ಕೇವಲ ನಮ್ಮ ಮನೆಯಲ್ಲಿ ಸ್ವಾಮೀಜಿ ಊಟ ಮಾಡುವುದನ್ನು ನೋಡುತ್ತೇನೆ. ಅಷ್ಟಕ್ಕೆ ನಾನು ಧನ್ಯನಾದೆ ಎಂದು ಭಾವಿಸುತ್ತೇನೆ” ಎಂದು ಬೇಡಿಕೊಂಡನು. ಶಂಭುನಾಥನಿಗೆ ಇವನ ಕೋರಿಕೆಯನ್ನು ಕೇಳಿ ಮನಕರಗಿತು. ಆತ, “ನೀವು ಹೇಳುವ ರೀತಿಯಲ್ಲಿ ಮಾಡಿದರೆ ನನ್ನಂತಹ ಆಚಾರಶೀಲನಾದ ಬ್ರಾಹ್ಮಣ ಕೂಡನಿಮ್ಮ ಮನೆಯಲ್ಲಿ ಊಟಮಾಡಲು ಯಾವ ಅಭ್ಯಂತರವೂ ಇಲ್ಲ. ಹಾಗಿರುವಾಗ ಎಲ್ಲವನ್ನೂ ತ್ಯಜಿಸಿದ ಮುಕ್ತಪುರುಷರಾದ ಸ್ವಾಮೀಜಿಗೆ ಯಾವ ಅಭ್ಯಂತರ ಇದ್ದೀತು?” ಎಂದು ಹೇಳೀದನು. ಸ್ವಾಮೀಜಿ ಸಂತೋಷದಿಂದ ಮಹಮ್ಮದೀಯ ಭಕ್ತನ ಮನೆಗೆ ಊಟಕ್ಕೆ ಹೋದರು. ಅನಂತರ ಹಲವು ಮಹಮ್ಮದೀಯ ಭಕ್ತರು ಅವರನ್ನು ತಮ್ಮ ತಮ್ಮ ಮನೆಗಳಿಗೆ ಕರೆದುಕೊಂಡು ಹೋದರು. 

 ಸ್ವಾಮೀಜಿಯವರ ಕೀರ್ತಿ ಆಳ್ವಾರಿನಲ್ಲೆಲ್ಲಾ ಹರಡಿತು. ಆಳ್ವಾರಿನ ಮಹಾರಾಜರ ದಿವಾನರಾದ ಮೇಜರ್ ರಾಮಚಂದ್ರಜಿ ಅವರಿಗೆ ಸ್ವಾಮೀಜಿ ವಿಷಯ ಗೊತ್ತಾಯಿತು. ಅವರನ್ನು ಮನೆಗೆ ಒಂದು ದಿನ ಬರಮಾಡಿಕೊಂಡು ಮಾತುಕತೆಯಾಡಿದರು. ಹೇಗಾದರೂ ಸ್ವಾಮೀಜಿಯವರನ್ನು ತಮ್ಮ ದೊರೆಗಳಾದ ಮಂಗಳ ಸಿಂಗರಿಗೆ ಪರಿಚಯ ಮಾಡಿಸಿದರೆ, ಅಡ್ಡಹಾದಿಗೆ ಹೋದ ಅವರು ಸರಿಯಾಗಬಹುದೆಂದು ಭಾವಿಸಿದರು. ಮಂಗಳಸಿಂಗರು ಆಗ ಮೂರು ಮೈಲಿ ದೂರದ ಒಂದು ಅರಮನೆಯಲ್ಲಿ ವಾಸಿಸುತ್ತಿದ್ದರು. ಅದ್ವಿತೀಯ ವಿದ್ವಾಂಸರಾದ ಮತ್ತು ಇಂಗ್ಲಿಷ್ ಭಾಷೆಯಲ್ಲಿ ಅಪಾರ ಪಾಂಡಿತ್ಯವಿರುವ ಒಬ್ಬ ಸಾಧು ಬಂದಿರುವರೆಂದೂ, ಅವರನ್ನು ತಾವು ಬಂದು ನೋಡಬೇಕೆಂದೂ ರಾಜರಿಗೆ ಪತ್ರ ಬರೆದರು. ರಾಜರು ದಿವಾನ್‍ಜಿ ಮನೆಗೆ ಸ್ವಾಮೀಜಿಯವರನ್ನು ನೋಡಲು ಬಂದರು. ಸ್ವಾಮೀಜಿಯವರನ್ನು ಕಂಡೊಡನೆಯೆ ರಾಜರು ಅವರಿಗೆ ನಮಸ್ಕಾರ ಮಾಡಿ ನಿಂತುಕೊಂಡಿದ್ದರು. ಸ್ವಾಮೀಜಿ ಅವರನ್ನು ಕುಳಿತುಕೊಳ್ಳುವಂತೆ ಹೇಳಿದರು. ಅವರಿಬ್ಬರಿಗೆ ಆದ ಸಂಭಾಷಣೆ ಆಕರ್ಷಣೀಯವಾಗಿದೆ: 

 ಮಹಾರಾಜ ಮಂಗಳಸಿಂಗ್: “ಸ್ವಾಮೀಜಿ ಮಹಾರಾಜರೆ, ನೀವು ದೊಡ್ಡ ವಿದ್ವಾಂಸರೆಂದು ನಾನು ಕೇಳಿರುವೆನು. ನೀವು ಯಾವುದಾದರೂ ಹುದ್ದೆಗೆ ಕೈಹಾಕಿ ಸುಲಭವಾಗಿ ನಿಮ್ಮ ಜೀವನೋಪಾಯವನ್ನು ಸಂಪಾದಿಸಬಹುದು. ಆದರೂ ನೀವು ಏತಕ್ಕೆ ಭಿಕ್ಷುಕನಂತೆ ಅಲೆಯುತ್ತಿರುವಿರಿ?” 

 ಸ್ವಾಮೀಜಿ: “ಮಹಾರಾಜರೆ, ನೀವು ಏತಕ್ಕೆ ನಿಮ್ಮ ರಾಜರ ಕರ್ತವ್ಯಗಳನ್ನೆಲ್ಲ ಬಿಟ್ಟುಬಿಟ್ಟು ಐರೋಪ್ಯರೊಡನೆ ಸೇರಿಕೊಂಡು ಶಿಕಾರಿ ಮುಂತಾದ ಆಮೋದ ಪ್ರಮೋದಗಳಲ್ಲಿ ನಿರತರಾಗಿರುವುದು?” 

 ಮಹಾರಾಜರು, ಸಂನ್ಯಾಸಿ ಇಂತಹ ನೇರವಾದ ಪ್ರಶ್ನೆಯನ್ನು ತನಗೆ ಹಾಕುತ್ತಾನೆ ಎಂದು ಕನಸಿನಲ್ಲಿಯೂ ಕಂಡಿರಲಿಲ್ಲ; ಅದೂ ಅಲ್ಲದೆ ಈ ಸಂನ್ಯಾಸಿಗೆ ತನ್ನ ಜೀವನದ ಗುಟ್ಟೆಲ್ಲ ಗೊತ್ತಾಗಿ ಹೋಗಿದೆ ಎಂದು ಭಾವಿಸಿದರು. ತಕ್ಷಣವೇ ಉತ್ತರ ಕೊಡಲು ಆಗಲಿಲ್ಲ. ಸ್ವಲ್ಪ ಹೊತ್ತಿನ ಮೇಲೆ “ಏನೋ ನನಗೆ ಅದರಲ್ಲಿ ಅಭಿರುಚಿ” ಎಂದರು. ಸ್ವಾಮೀಜಿ ಅದಕ್ಕೆ ತಕ್ಕ ಉತ್ತರವನ್ನೇ ಕೊಟ್ಟರು. “ಅದೇ ಕಾರಣಕ್ಕಾಗಿಯೇ ನಾನು ಭಿಕ್ಷುಕನಂತೆ ಅಲೆಯುತ್ತಿರುವೆನು. ನನಗೂ ಅದರಲ್ಲಿ ಆಸೆ.” 

 ಅನಂತರ ಮಹಾರಾಜರು ಮತ್ತೊಂದು ಪ್ರಶ್ನೆಯನ್ನು ಕೇಳಿದರು: “ಸ್ವಾಮೀಜಿ, ನನಗೆ ವಿಗ್ರಹಾರಾಧನೆಯಲ್ಲಿ ನಂಬಿಕೆಯಿಲ್ಲ. ನನಗೆ ಏನಾಗುವುದು ಇದರಿಂದ?” ಹೀಗೆ ಮಾತನಾಡುವಾಗ ರಾಜ ಸ್ವಲ್ಪ ಹಾಸ್ಯದ ನಗುವನ್ನು ಬೀರಿದನು. 

 ಸ್ವಾಮೀಜಿ “ನೀವು ಸುಮ್ಮನೆ ಹಾಸ್ಯ ಮಾಡುತ್ತಿರಬಹುದು” ಎಂದರು. 

 ಮಂಗಳಸಿಂಗ್: “ಇಲ್ಲ, ನಾನು ನಿಜವಾಗಿಯೂ ಹಾಸ್ಯಮಾಡುತ್ತಿಲ್ಲ. ಇತರರಂತೆ ಕಲ್ಲು ಮಣ್ಣು ಮರ ಲೋಹಗಳನ್ನು ದೇವರಂತೆ ನಾನು ಪೂಜಿಸಲಾರೆ. ಹಾಗಾದರೆ ನನ್ನ ಮರಣಾನಂತರ ಯಾವ ಶಿಕ್ಷೆ ನನಗೆ ಕಾದುಕೊಂಡಿದೆಯೊ?” 

 ಸ್ವಾಮೀಜಿ: “ಪ್ರತಿಯೊಬ್ಬನೂ ತನ್ನ ಶ್ರದ್ಧೆಗೆ ತಕ್ಕಂತೆ ಧರ್ಮವನ್ನು ಅನುಸರಿಸಬೇಕಾಗಿದೆ.” 

 ಸ್ವಾಮೀಜಿ ಮಾತನಾಡಿದ ಮೇಲೆ ರಾಜರ ಫೋಟೋ ತಗಲಿ ಹಾಕಿದ್ದುದನ್ನು ನೋಡಿದರು. ದಿವಾನರಿಗೆ ಆ ಫೋಟೋವನ್ನು ತೆಗೆಯುವಂತೆ ಹೇಳಿದರು. “ಇದು ಯಾರ ಫೋಟೋ?” ಎಂದು ಕೇಳಿದರು. “ಇದು ನಮ್ಮ ರಾಜರ ಫೋಟೋ” ಎಂದರು ದಿವಾನರು. “ಇದರ ಮೇಲೆ ಉಗುಳಿ” ಎಂದರು ಸ್ವಾಮೀಜಿ ದಿವಾನರಿಗೆ. ದಿವಾನರಿಗೆ ದಿಗ್‍ಭ್ರಮೆ ಹಿಡಿಯಿತು. ಈ ಸ್ವಾಮಿಗಳು ಹುಚ್ಚರಾಗಿರುವರೇನೊ ಎಂದು ಭಾವಿಸಿ ಅವರು ಸುಮ್ಮನೆ ಇದ್ದರು. 

 ಸ್ವಾಮೀಜಿ: “ಏತಕ್ಕೆ ಅನುಮಾನಿಸುತ್ತಿರುವಿರಿ, ಅದರ ಮೇಲೆ ಉಗುಳಿ. ಅಲ್ಲಿ ನಿಮ್ಮ ರಾಜರಿಲ್ಲ. ಅದು ಬರೀ ನೆರಳು ಬೆಳಕು. ರಾಜರಿಗೆ ಸಂಬಂಧಪಟ್ಟ ರಕ್ತವಾಗಲಿ ಮಾಂಸವಾಗಲೀ ಇಲ್ಲಿ ಇಲ್ಲ” ಎಂದರು. ದಿವಾನ್‍ಜಿ “ರಾಜರಿಲ್ಲದೇ ಇದ್ದರೂ ಅವರ ನೆನಪನ್ನು ಅದು ತರುತ್ತದೆ” ಎಂದರು. ಆಗ ಸ್ವಾಮೀಜಿ ರಾಜರಿಗೆ ಹೇಳಿದರು: “ಹೇಗೆ ಅದು ರಾಜನ ನೆನಪನ್ನು ತರುವುದೋ ಅದರಂತೆಯೇ ಭಕ್ತನಿಗೆ ವಿಗ್ರಹ ದೇವರ ನೆನಪನ್ನು ತರುತ್ತದೆ. ಯಾವ ಭಕ್ತನು ಎಲೈ ಕಲ್ಲೆ, ಮಣ್ಣೆ, ಲೋಹವೆ, ಮರವೆ ಎಂದು ಕರೆಯುವುದಿಲ್ಲ. ಆ ಚಿಹ್ನೆಯ ಮೂಲಕ ದೇವರನ್ನು ನೋಡುತ್ತಾನೆ.” ಈ ವಿಷಯವನ್ನು ರಾಜರಿಗೆ ವಿವರಿಸಿದರು. ಆಗ ರಾಜರು ವಿಗ್ರಹಾರಾಧನೆಯ ರಹಸ್ಯ ತಮಗೆ ಇದುವರೆಗೂ ಗೊತ್ತಿರಲಿಲ್ಲವೆಂದೂ, ಸ್ವಾಮೀಜಿಯವರಿಂದ ಹೊಸ ದೃಷ್ಟಿಯನ್ನು ಪಡೆದೆನೆಂದೂ ಹೇಳಿದರು. “ನನ್ನ ಗತಿ ಏನು? ನೀವು ದಯವಿಟ್ಟು ನನ್ನ ಮೇಲೆ ಕೃಪೆಯನ್ನು ಬೀರಿ” ಎಂದು ಬೇಡಿಕೊಂಡ. ಆಗ ಸ್ವಾಮೀಜಿ: ದೇವರೊಬ್ಬನು ಮಾತ್ರ ಎಲ್ಲರಿಗೂ ಕೃಪೆಯನ್ನು ತೋರಬಲ್ಲ. ಅವನು ಯಾವಾಗಲೂ ದಯಾಮಯ. ಆತನನ್ನು ನೀನು ಪ್ರಾರ್ಥಿಸು. ನಿನಗೆ ಕೃಪೆಯನ್ನು ತೋರುವನು” ಎಂದರು. 

 ಸ್ವಾಮೀಜಿ ಹೊರಟುಹೋದ ಮೇಲೆ ಸ್ವಲ್ಪ ಕಾಲ ಮಂಗಳಸಿಂಗರೂ ಚಿಂತಾಕ್ರಾಂತರಾಗಿದ್ದರು. ಅನಂತರ ದಿವಾನರಿಗೆ “ಇಂತಹ ಸಾಧುವನ್ನು ನಾನು ಇದುವರೆಗೆ ಎಂದಿಗೂ ನೋಡಿರಲಿಲ್ಲ. ನಿಮ್ಮೊಡನೆ ಅವರು ಇನ್ನೂ ಕೊಂಚ ದಿನಗಳು ಇರುವಂತೆ ಮಾಡಿ” ಎಂದರು. ಅದಕ್ಕೆ ದಿವಾನರು “ನಾನು ಪ್ರಯತ್ನ ಮಾಡುತ್ತೇನೆ. ಆದರೆ ಅದು ಎಷ್ಟರ ಮಟ್ಟಿಗೆ ಯಶಸ್ವಿಯಾಗುವುದೊ ಗೊತ್ತಿಲ್ಲ. ಸ್ವಾಮಿಗಳು ಸ್ವತಂತ್ರ ಪ್ರವೃತ್ತಿಯವರು, ಬೆಂಕಿಯ ಮುದ್ದೆಯಂತೆ ಇರುವರು” ಎಂದರು. ದಿವಾನರು ಸ್ವಾಮಿಗಳನ್ನು ತುಂಬ ಬೇಡಿಕೊಂಡಮೇಲೆ ಸ್ವಾಮೀಜಿ ಒಂದು ಷರತ್ತಿನ ಮೇಲೆ ಒಪ್ಪಿಕೊಂಡರು. ಸ್ವಾಮೀಜಿ ಹೊರಗಡೆ ಇದ್ದಾಗ ಯಾರು ಅವರನ್ನು ನೋಡಲು ಬರುತ್ತಿದ್ದರೋ ಅವರಿಗೆ ಇಲ್ಲಿಯೂ ಕೂಡ ಬಂದು ನೋಡುವುದಕ್ಕೆ ಅವಕಾಶವಿರಬೇಕು ಎಂದರು. ಅನೇಕ ಜನ ಸ್ವಾಮೀಜಿಯವರನ್ನು ನೋಡುವುದಕ್ಕೆ ಬರುತ್ತಿದ್ದರು. ಹಲವರು ಸ್ವಾಮೀಜಿಯವರ ಬೋಧನೆಯಿಂದ ತಮ್ಮ ಜೀವನವನ್ನೇ ತಿದ್ದಿಕೊಂಡರು. ಅವರೆಡೆಗೆ ಒಬ್ಬ ವೃದ್ಧ ಬರುತ್ತಿದ್ದ. ಪ್ರತಿದಿನವೂ ದೊಡ್ಡ ದೊಡ್ಡ ಪ್ರಶ್ನೆಗಳನ್ನೆ ಹಾಕುತ್ತಿದ್ದ. ಆದರೆ ಏನನ್ನೂ ಕಾರ‍್ಯತಃ ಮಾಡುತ್ತಿರಲಿಲ್ಲ. ಒಂದುಸಲ ಆತ ಬಂದಾಗ ಸ್ವಾಮೀಜಿ ಸುಮ್ಮನೆ ಇದ್ದರು, ಆತನೊಡನೆ ಮಾತನ್ನೇ ಆಡಲಿಲ್ಲ. ಸುಮಾರು ಒಂದು ಗಂಟೆ ಆದಮೇಲೆ ಆತ ಅಸಮಾಧಾನದಿಂದ ಎದ್ದು ಹೋದ. ಹತ್ತಿರ ಕುಳಿತಿದ್ದವರು “ಏತಕ್ಕೆ ಇಷ್ಟು ಕಟುವಾಗಿ ಅವನೊಡನೆ ವ್ಯವಹರಿಸಿದಿರಿ?” ಎಂದು ಪ್ರಶ್ನಿಸಿದರು. ಅದಕ್ಕೆ ಸ್ವಾಮೀಜಿ ಹೀಗೆ ಹೇಳಿದರು: “ನೋಡಿ ಮಕ್ಕಳೆ, ನಾನು ನಿಮಗೆ ನನ್ನ ಪ್ರಾಣವನ್ನು ಬೇಕಾದರೆ ಕೊಡಬಲ್ಲೆ. ಏಕೆಂದರೆ ನಾನು ಹೇಳಿದಂತೆ ಮಾಡುವುದಕ್ಕೆ ನಿಮಗೆ ಮನಸ್ಸಿದೆ ಮತ್ತು ಶಕ್ತಿ ಇದೆ. ಆದರೆ ಆ ವೃದ್ಧನಾದರೋ ತನ್ನ ಜೀವನದ ಹತ್ತನೇ ಒಂಭತ್ತು ಪಾಲನ್ನು ಪ್ರಾಪಂಚಿಕ ವಸ್ತುಗಳನ್ನು ಅರಸುವುದರಲ್ಲಿ ವ್ಯಯಮಾಡಿದನು. ಈಗ ಅವನ ಕೈಯಲ್ಲಿ ಲೌಕಿಕವೂ ಸಾಧ್ಯವಿಲ್ಲ, ಪರಮಾರ್ಥವೂ ಸಾಧ್ಯವಿಲ್ಲ. ಸುಮ್ಮನೆ ಕೇಳಿದರೆ ಸಾಕು ದೇವರ ಕೃಪೆ ತನಗೆ ದೊರಕುವುದು ಎಂದು ಭಾವಿಸುವನು. ಆಧ್ಯಾತ್ಮಿಕ ಜೀವನದಲ್ಲಿ ಗುರಿಯನ್ನು ಮುಟ್ಟಬೇಕಾದರೆ ಸ್ವಪ್ರಯತ್ನ, ಪುರುಷಕಾರ ಇರಬೇಕು. ಇದನ್ನು ಮಾಡುವುದಕ್ಕೆ ಸಾಧ್ಯವಿಲ್ಲವೊ ಅಂತಹವನಿಗೆ ದೇವರು ಹೇಗೆ ಕೃಪೆ ತೋರಬಲ್ಲ? ಯಾರಲ್ಲಿ ಪೌರುಷ ಇಲ್ಲವೋ ಅವನು ತಮಸ್ಸಿನಲ್ಲಿ ಮುಳುಗಿ ಹೋಗಿರುವನು. ಯೋಧಾಗ್ರಣಿ ಅರ್ಜುನ ತನ್ನ ಪೌರುಷವನ್ನು ಕಳೆದುಕೊಳ್ಳುವುದರಲ್ಲಿ ಇರುವುದನ್ನು ನೋಡಿ ಕೃಷ್ಣ ಅವನಿಗೆ ತನ್ನ ಸ್ವಧರ್ಮವನ್ನು ಆಚರಿಸು, ಕರ್ಮಫಲವನ್ನು ತ್ಯಜಿಸು, ಇದರಿಂದ ಸತ್ತ್ವಗುಣ ಪ್ರಾಪ್ತವಾಗುವುದು, ಹೃದಯ ಶುದ್ಧವಾಗುವುದು, ಅನಂತರ ಕರ್ಮ್ಯತ್ಯಾಗ ಮತ್ತು ಶರಣಾಗತಿ ಎಲ್ಲ ಬರುವುದು ಎಂದು ಬೋಧಿಸಿದನು. ಬಲಿಷ್ಠರಾಗಿ! ಪುರುಷಸಿಂಹರಾಗಿ! ಎಲ್ಲಿಯವರೆಗೆ ಒಬ್ಬ ದುರಾತ್ಮನಾದರೂ ಚಿಂತೆಯಿಲ್ಲ, ಅವನು ಧಿರನಾಗಿದ್ದರೆ, ಬಲಿಷ್ಠನಾಗಿದ್ದರೆ, ಅವನನ್ನು ನಾನು ಗೌರವಿಸುತ್ತೇನೆ. ಅವನಲ್ಲಿರುವ ಶಕ್ತಿಯೇ ಅವನಲ್ಲಿರುವ ದುಷ್ಟ ಸ್ವಭಾವವನ್ನೆಲ್ಲ ಬಿಡುವಂತೆ ಮಾಡುವುದು, ಸ್ವಾರ್ಥ ಕರ್ಮಗಳನ್ನೆಲ್ಲ ತ್ಯಜಿಸುವಂತೆ ಮಾಡಿ ಅವನನ್ನು ಸತ್ಯದ ಸಮೀಪಕ್ಕೆ ಒಯ್ಯುವುದು.” 

 ಸ್ವಾಮೀಜಿಯವರು ತಮ್ಮ ಬಳಿಗೆ ಬರುತ್ತಿದ್ದ ಯುವಕರಿಗೆ, ಸಂಸ್ಕೃತ ಕಲಿಯಿರಿ ಮತ್ತು ಇಂಗ್ಲೀಷ್ ಭಾಷೆ ಕಲಿಯಿರಿ ಎಂದು ಹೇಳುತ್ತಿದ್ದರು. ಇಂಗ್ಲೀಷ್ ಭಾಷೆಯಲ್ಲಿರುವ ವಿಜ್ಞಾನಮಾರ್ಗವನ್ನು ನಾವು ಕಲಿಯಬೇಕು ಎಂದು ಒತ್ತಿ ಹೇಳುತ್ತಿದ್ದರು. ಈಗ ನಾವು ಹೊರಗಿನವರು ಬರೆದ ನಮ್ಮ ಚರಿತ್ರೆಯನ್ನು ಓದುತ್ತಿರುವೆವು. ಆ ಚರಿತ್ರೆಯಲ್ಲಿ ಅವರಿಗೆ ಗಮನಾರ್ಹವಾದುವು ಯಾವುದೂ ಕಾಣುವುದಿಲ್ಲ. ಭಾರತೀಯನು ಮಾತ್ರ ಭರತಖಂಡದ ಚರಿತ್ರೆಯನ್ನು ಬರೆಯಬೇಕು. ಹಾಗೆ ಬರೆಯುವಾಗ ಪಾಶ್ಚಾತ್ಯರ ಮಾರ್ಗವನ್ನು ಮಾತ್ರ ನಾವು ಅನುಸರಿಸಬೇಕೆ ವಿನಃ ಅವರ ನಿರ್ಣಯಗಳನ್ನು ನಾವು ತೆಗೆದುಕೊಳ್ಳಬೇಕಾಗಿಲ್ಲ. ಪಾಶ್ಚಾತ್ಯರಿಗೆ ಭಾರತೀಯ ಭಾವನೆಗಳ ಪರಿಚಯವಿರಲಿಲ್ಲ. ಅವರು ಬರೆದ ನಿರ್ಣಯಗಳು ಲೋಪದೋಷಗಳಿಂದ ಕೂಡಿದ್ದವು ಎಂದು ಸ್ವಾಮೀಜಿ ಹೇಳಿದರು. ಭಾರತೀಯನ ಹೃದಯದಲ್ಲಿ ನಮ್ಮ ಗತಕಾಲದ ಮಹಿಮೆ ಮೂಡುವಂತೆ ಮಾಡಬೇಕು, ಅದಕ್ಕಾಗಿ ನಾವು ಹೆಮ್ಮೆ ತಾಳಬೇಕು, ಹಿಂದಿನದನ್ನು ಗೌರವಿಸಬೇಕು, ಅದಕ್ಕೆ ಸರಿಯಾಗಿ, ಅದಕ್ಕಿಂತ ಮಿಗಿಲಾಗಿ ನಾವು ಮುಂದುವರಿಯಬೇಕೆಂದು ಸ್ವಾಮೀಜಿಯವರ ಪಲ್ಲವಿಯಾಗಿತ್ತು. 

 ಸ್ವಾಮೀಜಿ ಬಳಿಗೆ ಒಬ್ಬ ಬ್ರಾಹ್ಮಣ ಬಡಹುಡುಗ ಬರುತ್ತಿದ್ದ. ಶಿಷ್ಯ ಗುರುವನ್ನು ಗೌರವಿಸುವಂತೆ ಆ ಹುಡುಗ ಸ್ವಾಮೀಜಿಯವರನ್ನು ಗೌರವಿಸುತ್ತಿದ್ದ. ಉಪನಯನಕ್ಕೆ ಪ್ರಾಪ್ತ ವಯಸ್ಸಾಗಿದ್ದರೂ ಹಣಾಭಾವದಿಂದ ಅವನ ತಂದೆ ಉಪನಯನವನ್ನು ಅವನಿಗೆ ಮಾಡಿರಲಿಲ್ಲ. ಸ್ವಾಮೀಜಿ ತಮ್ಮ ಬಳಿಗೆ ಬಂದವರಿಗೆ ಹೇಳಿ, ಸ್ವಲ್ಪ ಹಣವನ್ನು ಚಂದಾ ಎತ್ತಿ ಆ ಹುಡುಗನಿಗೆ ಉಪನಯನವನ್ನು ಮಾಡಿಸಿದರು. 

 ಸ್ವಾಮೀಜಿಯವರು ಸುಮಾರು ಏಳು ವಾರಗಳು ಆಳ್ವಾರಿನಲ್ಲಿ ಇದ್ದಾದ ಮೇಲೆ ಅಲ್ಲಿಂದ ಮುಂದೆ ಹೊರಡಲು ಮನಸ್ಸು ಮಾಡಿದರು. ಜನಗಳೆಲ್ಲ ಅವರನ್ನು ಕೋರಿಕೊಂಡ ಮೇರೆಗೆ ಅವರು ಒಂದು ಎತ್ತಿನ ಗಾಡಿಯಲ್ಲಿ ಪಾಡುಪಲ್ ಎಂಬ ಊರಿನ ವರೆಗೆ ಹೋಗಲು ಒಪ್ಪಿದರು. ಅನೇಕ ಭಕ್ತಾದಿಗಳು ಸ್ವಾಮಿಗಳನ್ನು ಕೆಲವು ಮೈಲಿಗಳಾದರೂ ಅನುಸರಿಸುತ್ತೇವೆ ಎಂದು ಬೇಡಿಕೊಂಡರು. ಮೊದಲು ಪಾಂಡುಪಲ್ ಎಂಬ ಊರಿಗೆ ಹೋದರು. ಅಲ್ಲಿ ಒಂದು ಹನುಮಂತರಾಯನ ಗುಡಿ ಇತ್ತು. ಸ್ವಾಮಿಗಳು ಅಲ್ಲಿ ತಂಗಿದರು. ಮಾರನೆಯ ದಿನ ಸ್ವಾಮೀಜಿ ಗಾಡಿಯನ್ನು ಕಳುಹಿಸಿಕೊಟ್ಟು ಕಾಲುನಡಿಗೆಯಲ್ಲಿ ಸುಮಾರು ಹದಿನಾರು ಮೈಲಿಗಳನ್ನು ತಮ್ಮ ಭಕ್ತರೊಡನೆ ನಡೆದುಕೊಂಡು ಹೋದರು. ಕಾಡು ಮತ್ತು ಬೆಟ್ಟಗಳ ಮೂಲಕ ಹೋಗಬೇಕಾಗಿತ್ತು. ಅಲ್ಲಿ ಬೇಕಾದಷ್ಟು ವನ್ಯ ಮೃಗಗಳು ಇದ್ದವು. ಆದರೆ ಸ್ವಾಮೀಜಿ ಜೊತೆಯಲ್ಲಿದ್ದುದರಿಂದ ನಡೆದುಕೊಂಡು ಹೋಗುತ್ತಿದ್ದ ಭಕ್ತರಿಗೆ ಆಯಾಸವಾಗಲಿ, ಅಂಜಿಕೆಯಾಗಲಿ ಆಗಲಿಲ್ಲ. ಸ್ವಾಮೀಜಿ ಯಾವಾಗಲೂ ಯಾವುದಾದರೂ ಆಧ್ಯಾತ್ಮಿಕ ವಿಷಯಗಳನ್ನು ಸಾಧುಸಂತರ ಜೀವನವನ್ನು ಕುರಿತು ಮಾತನಾಡುತ್ತಿದ್ದರು. ಸಂಜೆ ಹೊತ್ತಿಗೆ ತಹ್ಲ ಎಂಬ ಹಳ್ಳಿಯನ್ನು ಸೇರಿದರು. ಆ ಊರಿನಲ್ಲಿ ನೀಲಕಂಠಮಹಾದೇವ ಎಂಬ ದೇವಸ್ಥಾನವಿತ್ತು. ಎಲ್ಲರೂ ಆ ದೇವಸ್ಥಾನದ ಮುಂದೆ ಮಲಗಿದರು. 

 ಮಾರನೆ ದಿನ ಸುಮಾರು ಹದಿನೆಂಟು ಮೈಲಿಗಳನ್ನು ನಡೆದುಕೊಂಡು ಹೋಗಿ ನಾರಾಯಣಿ ಎಂಬ ಗ್ರಾಮವನ್ನು ಸೇರಿದರು. ಅಲ್ಲಿ ಅದೇ ಹೆಸರಿನ ದೇವಿಯ ಗುಡಿ ಇದೆ. ವರ್ಷಕ್ಕೆ ಒಂದು ಸಲ ಇಲ್ಲಿ ದೊಡ್ಡ ಜಾತ್ರೆ ದೇವಿಯ ಹೆಸರಿನಲ್ಲಿ ಆಗುವುದು. ರಾಜಾಸ್ಥಾನದ ಹಲವು ಕಡೆಗಳಿಂದ ಅಲ್ಲಿಗೆ ಅಸಂಖ್ಯಾತ ಭಕ್ತರು ಬರುವರು. ಅಲ್ಲಿ ಸ್ವಾಮಿಗಳು ಆಳ್ವಾರಿನಿಂದ ತಮ್ಮನ್ನು ಅನುಸರಿಸುತ್ತಿದ್ದ ಭಕ್ತರಿಗೆ ದಯವಿಟ್ಟು ಹಿಂದಿರುಗಿ ಹೊಗಬೇಕೆಂದು ಹೇಳಿ ತಾವು ಕಾಲುನಡಿಗೆಯಲ್ಲಿ ನಡೆದುಕೊಂಡು ಸುಮಾರು ಹದಿನಾರು ಮೈಲಿಗಳು ದೂರದಲ್ಲಿರುವ ಬಾಸ್ಟ ಎಂಬ ಊರನ್ನು ಸೇರಿ ಅಲ್ಲಿಂದ ಜಯಪುರಕ್ಕೆ ರೈಲನ್ನು ಹತ್ತಿದರು. ಇಲ್ಲಿ ಆಳ್ವಾರಿನಲ್ಲಿ ಸ್ವಾಮೀಜಿ ಅವರನ್ನು ಕಂಡ ಭಕ್ತನೊಬ್ಬ ತಮ್ಮ ಮನೆಗೆ ಬರಬೇಕೆಂದು ಕೋರಿಕೊಂಡಿದ್ದ. ಜಯಪುರದಲ್ಲಿ ಮನೆಯ ಭಕ್ತ ಸ್ವಾಮೀಜಿಯವರ ಒಂದು ಫೋಟೊವನ್ನು ತೆಗೆದುಕೊಂಡ. ಸ್ವಾಮೀಜಿಯವರು ಪರಿವ್ರಾಜಕರಾಗಿ ಅಲೆಯುತ್ತಿದ್ದಾಗ ತೆಗೆದ ಮೊದಲನೆ ಭಾವಚಿತ್ರ ಇದು. 

 ಸ್ವಾಮೀಜಿ ಅವರು ಜಯಪುರದಲ್ಲಿ ಎರಡು ವಾರಗಳು ಇದ್ದರು. ಅಲ್ಲಿ ಒಬ್ಬ ಪ್ರಖ್ಯಾತ ವ್ಯಾಕರಣ ಪಂಡಿತನಿದ್ದ. ಆತನಿಂದ ಸಂಸ್ಕೃತ ವ್ಯಾಕರಣವನ್ನು ಕಲಿಯಲು ಯತ್ನಿಸಿದರು. ಆ ಪಂಡಿತ ವಿದ್ಯಾವಂತನಾಗಿದ್ದರೂ ವಿಷಯವನ್ನು ತಿಳಿಸುವ ಕಲೆ ಗೊತ್ತಿರಲಿಲ್ಲ. ಮೂರು ದಿನಗಳವರೆಗೆ ಪಾಣಿನಿಯ ಮೊದಲನೆ ವ್ಯಾಕರಣ ಸೂತ್ರವನ್ನು ವಿವರಿಸಲು ಯತ್ನಿಸಿದ. ಆದರೆ ಸ್ವಾಮೀಜಿಗೆ ಪಂಡಿತ ಹೇಳುತ್ತಿದ್ದುದು ಸರಿಯಾಗಿ ಅರ್ಥವಾಗುತ್ತಿರಲಿಲ್ಲ. ನಾಲ್ಕನೆದಿನ ಪಂಡಿತ ಸ್ವಾಮಿಗಳಿಗೆ, ತಾನು ಮೂರು ದಿನ ಪ್ರಯತ್ನ ಪಟ್ಟರೂ ಸ್ವಾಮಿಗಳಿಗೆ ಒಂದು ಸೂತ್ರವನ್ನೂ ಹೇಳಿಕೊಡಲು ಸಾಧ್ಯವಾಗಲಿಲ್ಲ; ಆದಕಾರಣ ತನಗೆ ಸಾಧ್ಯವಿಲ್ಲ ಎಂದು ಹೇಳಿದ. ಸ್ವಾಮೀಜಿಯವರು ಸಂಕಲ್ಪಮಾಡಿ ತಾವೊಬ್ಬರೆ ಕುಳಿತುಕೊಂಡು ಮೂರು ಗಂಟೆಗಳಲ್ಲಿ ಅದನ್ನೆಲ್ಲ ಚೆನ್ನಾಗಿ ಓದಿ ತಿಳಿದುಕೊಂಡು ಪಂಡಿತನ ಹತ್ತಿರ ಹೋಗಿ ಪಾಠ ಒಪ್ಪಿಸಿದರು. ಪಂಡಿತನಿಗೆ ಆಶ್ಚರ್ಯವಾಯಿತು. ತಾನು ಮೂರು ದಿನಗಳವರೆಗೆ ಪ್ರಯತ್ನ ಮಾಡಿದರೂ ಕಲಿಸುವುದಕ್ಕೆ ಆಗದೆ ಇದ್ದುದು, ಈಗ ಸ್ವಾಮೀಜಿ ಒಬ್ಬರೇ ಅಧ್ಯಯನ ಮಾಡಿ ಮೂರು ಗಂಟೆಗಳಲ್ಲಿ ಚೆನ್ನಾಗಿ ತಿಳಿದುಕೊಂಡುಬಿಟ್ಟಿದ್ದರು. ಅನಂತರ ಲೀಲಾಜಾಲವಾಗಿ ಆ ಗ್ರಂಥವನ್ನೆಲ್ಲ ಪೂರೈಸಿದರು. “ಮನಸ್ಸಿಗೆ ತುಂಬಾ ಆಸಕ್ತಿ ಇದ್ದರೆ ಏನನ್ನು ಬೇಕಾದರೂ ಸಾಧಿಸಬಲ್ಲದು. ಮಹಾ ಪರ್ವತವನ್ನೇ ಧೂಳೀಕಣಗಳನ್ನಾಗಿ ಮಾಡಿಬಿಡಬಹುದು” ಎಂದು ಸ್ವಾಮೀಜಿ ಅನಂತರ ಹೇಳುತ್ತಿದ್ದರು. 

 ಸ್ವಾಮೀಜಿ ಜಯಪುರದಲ್ಲಿ ಇದ್ದಾಗ ಅಲ್ಲಿಯ ಕಮಾಂಡರ್ ಇನ್‍ಚೀಫ್ ಆದ ಸರದಾರ್ ಹರಿಸಿಂಗ್ ಎಂಬುವನೊಡನೆ ಸ್ನೇಹ ಬೆಳಸಿದರು. ಆತ ವೇದಾಂತಿ, ಬಾಹ್ಯಾಚಾರ, ವಿಗ್ರಹಾರಾಧನೆ ಮುಂತಾದವುಗಳನ್ನು ನಂಬುತ್ತಿರಲಿಲ್ಲ. ಸ್ವಾಮೀಜಿಯವರೊಡನೆ ಹಲವು ದಿನಗಳು ಆತ ವೇದಾಂತ ವಿಷಯವಾಗಿ ಚರ್ಚೆಯನ್ನು ನಡೆಸಿದ. ಒಂದು ದಿನ ಜಯಪುರದ ಊರಿನ ರಸ್ತೆಯಲ್ಲಿ ಸ್ವಾಮೀಜಿ ಮತ್ತು ಸರದಾರ್ ಇಬ್ಬರೂ ಹೋಗುತ್ತಿದ್ದರು. ಆಗ ಶ‍್ರೀಕೃಷ್ಣನ ಮೆರವಣಿಗೆಯೊಂದು ಬಂದಿತು. ಹಲವು ಭಕ್ತರು ದೇವರನ್ನು ತಮ್ಮ ಹೆಗಲಮೇಲೆ ಹೊತ್ತುಕೊಂಡು ಹೋಗುತ್ತಿದ್ದರು. ಸ್ವಾಮೀಜಿ ತಮ್ಮ ಕೈಯಿಂದ ಸರದಾರ್ ಅವರನ್ನು ಮುಟ್ಟಿ ದೇವರ ಕಡೆ ತೋರಿ “ನೋಡು, ಅಲ್ಲಿ ಸಾಕ್ಷಾತ್ ಭಗವಂತ ಹೋಗುತ್ತಿರುವುದನ್ನು” ಎಂದು ಹೇಳಿದರು. ತಕ್ಷಣವೇ ಸರದಾರ್ ಆ ವಿಗ್ರಹವನ್ನು ನೋಡಲುಪಕ್ರಮಿಸಿದನು. ಅದನ್ನು ನೋಡುತ್ತ ಇದ್ದಂತೆ ಭಕ್ತಿಯಿಂದ ಪರವಶನಾಗಿ ಅವನ ಕಣ್ಣುಗಳಿಂದ ಅಶ್ರು ಹರಿಯತೊಡಗಿತು. ಅನಂತರ ಸ್ವಾಮೀಜಿಯವರಿಗೆ, ಅದು ತನಗಾದ ಒಂದು ಅಪೂರ್ವ ಅನುಭವವೆಂದೂ, ಗಂಟೆ ಗಂಟೆ ಚರ್ಚಿಸಿ ಯಾವುದನ್ನು ತಾನು ತಿಳಿದುಕೊಳ್ಳಲು ಸಾಧ್ಯವಾಗಲಿಲ್ಲವೋ ಅದನ್ನು ಸ್ವಾಮೀಜಿಯವರ ಸ್ಪರ್ಶಮಾತ್ರದಿಂದ ಅರಿಯಲು ಸಾಧ್ಯವಾಯಿತೆಂದೂ, ನಿಜವಾಗಿ ದೇವರನ್ನು ವಿಗ್ರಹದಲ್ಲಿ ನೋಡಿದೆನೆಂದೂ ಹೇಳಿದನು. 

 ಒಂದು ದಿನ ಪ್ರಖ್ಯಾತ ವೇದಾಂತಿಯಾದ ಸೂರಜ್ ನಾರಾಯಣ ಎಂಬ ಪಂಡಿತ ಸ್ವಾಮಿಗಳನ್ನು ನೋಡಲು ಬಂದ. ಆತ ದೊಡ್ದ ವೇದಾಂತಿ. ಅವತಾರವಾದಿ ತತ್ತ್ವಗಳನ್ನು ನಂಬುವವನಲ್ಲ. “ವೇದಾಂತದ ಪ್ರಕಾರ ಜೀವರೆಲ್ಲ ಬ್ರಹ್ಮರೇ, ಆದಕಾರಣ ಎಲ್ಲವೂ ಅವತಾರವೇ” ಎಂದ. ಆಗ ಸ್ವಾಮೀಜಿ ಹಾಸ್ಯವಾಗಿ “ಹಿಂದೂಗಳಲ್ಲಿ ದೇವರು ಮತ್ಸ್ಯ, ಕೂರ್ಮ, ವರಾಹ, ನರಸಿಂಹ ಅವತಾರಗಳನ್ನು ಮಾಡಿದ ಎಂದು ನಂಬುವರು. ನೀವು ಈ ಗುಂಪಿನಲ್ಲಿ ಯಾವ ಶ್ರೇಣಿಗೆ ಸೇರಿರುವಿರಿ?” ಎಂದು ಕೇಳಿದರು. ಪಂಡಿತ ಅವಾಕ್ಕಾಗಿ ಕುಳಿತ. ಸಭಿಕರೆಲ್ಲ ಗೊಳ್ಳೆಂದು ನಕ್ಕರು. 

 ಸ್ವಾಮೀಜಿ ಅನಂತರ ಅಜ್ಮೀರಕ್ಕೆ ಹೋದರು. ಅಲ್ಲಿ ಕೆಲವು ದಿನಗಳು ಇದ್ದಾದ ಮೇಲೆ ಬೇಸಗೆಯ ಹೊತ್ತಿಗೆ ಅಬು ಬೆಟ್ಟಕ್ಕೆ ಹೋದರು. ಬೇಸಗೆಯ ಕಾಲದಲ್ಲಿ ಅನೇಕ ಜನ ಆ ಬೆಟ್ಟದಮೇಲೆ ಹೋಗಿ ವಾಸಿಸುವರು. ಅಲ್ಲಿ ಹಲವು ಜನ ಮನೆಗಳನ್ನು ಕಟ್ಟಿಕೊಂಡಿರುವರು. ಜೈನರ ಅತ್ಯಂತ ಪ್ರಖ್ಯಾತವಾದ ದಿಲ್‍ವಾರಾ ದೇವಸ್ಥಾನಗಳು ಅಲ್ಲಿವೆ. ಸ್ವಾಮೀಜಿ ಅಬು ಸರೋವರದ ತೀರದಲ್ಲಿ ಒಂದು ಗುಹೆಯಲ್ಲಿ ವಾಸಿಸುತ್ತಿದ್ದರು. ಅಲ್ಲಿ ಧ್ಯಾನ ಅಧ್ಯಯನದಲ್ಲಿ ನಿರತರಾಗಿದ್ದರು. ಸಂಜೆ ಹೊತ್ತು ಸರೋವರದ ತೀರದಲ್ಲಿ ಗಾಳಿಯ ಸಂಚಾರಕ್ಕೆ ಹೋಗುತ್ತಿದ್ದರು. ಒಂದು ದಿನ ಹೋಗುತ್ತಿದ್ದಾಗ ಒಬ್ಬ ಮುಸಲ್ಮಾನಿ ವಕೀಲ ಅವರನ್ನು ಕಂಡ. ಅವರೊಡನೆ ಮಾತುಕತೆ ಆಡಿದಮೇಲೆ ಬಹಳ ಆಕರ್ಷಿತನಾದ. ಸ್ವಾಮೀಜಿಯವರ ಗುಹೆಗೆ ವಿರಾಮವಾದ ಕಾಲದಲ್ಲೆಲ್ಲ ಹೋಗಿ ಮಾತನಾಡಿಕೊಂಡು ಬರುತ್ತಿದ್ದ. ಒಂದು ದಿನ ಆತ ಸ್ವಾಮೀಜಿಯವರಿಗೆ, ಏನಾದರು ಸಹಾಯ ಬೇಕಾದರೆ ತಾನು ಅದನ್ನು ಮಾಡಲು ಸಿದ್ಧನಿರುವೆ ಎಂದನು. ಸ್ವಾಮೀಜಿ ಮುಸಲ್ಮಾನ ವಕೀಲನಿಗೆ, “ಮಳೆಗಾಲ ಬರುತ್ತಿದೆ, ಈ ಗುಹೆಗೆ ಬಾಗಲಿಲ್ಲ. ಅದರ ಮೂಲಕ ನೀರು ಬರುವುದು. ಸಾಧ್ಯವಾದರೆ ಅದಕ್ಕೆ ಬಾಗಿಲನ್ನು ಮಾಡಿಸಿಕೊಡು” ಎಂದು ಕೇಳಿದರು. ಅದಕ್ಕೆ ಆತ ಈ ಹಳೆಯ ಗುಹೆಯಲ್ಲಿ ವಾಸಿಸುವುದಕ್ಕಿಂತ ತನ್ನ ಬಂಗಲೆಯಲ್ಲಿ ಸ್ವಾಮೀಜಿ ವಾಸ ಮಾಡಬಹುದು. ಅಲ್ಲಿ ತಾನು ಒಬ್ಬನೇ ಇರುತ್ತೇನೆ ಎಂದನು. ಬೇಕಾದರೆ ಹಿಂದೂಗಳ ಕೈಯಿಂದ ಸ್ವಾಮೀಜಿಗೆ ಅಡಿಗೆ ಮಾಡಿಸಿ ಬಡಿಸುವಂತೆ ಮಾಡುವುದಾಗಿಯೂ ಹೇಳಿದನು. ಸ್ವಾಮೀಜಿ ಗುಹೆಯನ್ನು ಬಿಟ್ಟು ಆತನ ಮನೆಗೆ ಹೋದರು. ವಕೀಲನ ಮೂಲಕ ಸ್ವಾಮೀಜಿಯವರಿಗೆ ಬೇಸಿಗೆಯಲ್ಲಿ ಕಾಲ ಕಳೆಯುವುದಕ್ಕೆ ಬಂದಿದ್ದ ಅನೇಕರ ಪರಿಚಯವಾಯಿತು. 

 ಒಂದು ದಿನ ಮಧ್ಯಾಹ್ನ ಊಟವಾದ ಮೇಲೆ ಸ್ವಾಮಿಗಳು ವಿಶ್ರಾಂತಿ ಪಡೆಯುತ್ತಿದ್ದರು. ಅವರ ಮೈಮೇಲೆ ಒಂದು ಕೌಪೀನ ಮತ್ತು ಒಂದು ಹೊದಿಕೆ ಮಾತ್ರ ಇತ್ತು. ಮುಸಲ್ಮಾನ ವಕೀಲ ಖೇತ್ರಿ ಮಹಾರಾಜರ ಆಪ್ತ ಕಾರ‍್ಯದರ್ಶಿಯಾದ ಮುನ್ಷಿ ಜಗಮೋಹನ್‍ಲಾಲ್ ಎಂಬುವರನ್ನು ತನ್ನ ಮನೆಗೆ ಸ್ವಾಮೀಜಿಯವರನ್ನು ಪರಿಚಯ ಮಾಡಿಸುವುದಕ್ಕಾಗಿ ಕರೆದಿದ್ದ. ಅವರು ಸ್ವಾಮಿಗಳನ್ನು ನೋಡಿ, ಇವರೊಬ್ಬ ಅಲೆದಾಡುವ ಭಿಕಾರಿಗಳ ಗುಂಪಿಗೆ ಸೇರಿದವರು ಎಂದು ಆಲೋಚಿಸಿದರು. ಸ್ವಾಮೀಜಿ ಎಚ್ಚೆತ್ತರು. ಜಗಮೋಹನ್‍ಲಾಲ್ ಕುರಿತು, “ಸ್ವಾಮೀಜಿ, ನೀವು ಹಿಂದೂಗಳು, ಹೇಗೆ ಮಹಮ್ಮದೀಯರ ಮನೆಯಲ್ಲಿರುವಿರಿ? ನಿಮ್ಮ ಆಹಾರವನ್ನು ಅವರು ಮುಟ್ಟಬಹುದೆ?” ಎಂದು ಕೇಳಿದರು. ಸ್ವಾಮೀಜಿ ಈ ಪ್ರಶ್ನೆಯನ್ನು ಕೇಳಿ ಕೆಂಡದಂತಾದರು. ಆಗ ಅವರು ಹೀಗೆ ಹೇಳಿದರು: “ಮಹಾಶಯರೆ, ಏನು ನೀವು ಹೇಳುತ್ತಿರುವುದು? ನಾನು ಸಂನ್ಯಾಸಿ, ನಾನು ನಿಮ್ಮ ಸಾಮಾಜಿಕ ಆಚಾರಗಳಿಗೆ ಅತೀತ. ನಾನು ಒಬ್ಬ ಪರೆಯನೊಂದಿಗೆ ಬೇಕಾದರೂ ಊಟಮಾಡಬಲ್ಲೆ. ದೇವರು ಇದಕ್ಕೆ ಅನುಮತಿ ಕೊಟ್ಟಿರುವುದರಿಂದ ನಾನು ಅವನಿಗೆ ಅಂಜಬೇಕಾಗಿಲ್ಲ. ನಾನು ನಿಮಗೆ, ನಿಮ್ಮ ಸಮಾಜಕ್ಕೆ ಅಂಜಬೇಕೆ? ನಿಮಗೆ ದೇವರ ವಿಷಯವೂ ಗೊತ್ತಿಲ್ಲ, ಶಾಸ್ತ್ರವೂ ಗೊತ್ತಿಲ್ಲ. ನಾನು ಎಲ್ಲದರಲ್ಲಿಯೂ ಅತ್ಯಂತ ಕೀಳು ಮನುಷ್ಯನಲ್ಲಿಯೂ ಬ್ರಹ್ಮನನ್ನು ನೋಡುತ್ತೇನೆ, ನನಗೆ ಮೇಲುಕೀಳೆಂಬ ಭಾವನೆ ಇಲ್ಲ. ಶಿವ, ಶಿವ,!” ಈ ಮಾತನ್ನು ಕೇಳಿದಾಗ ಮುನ್ಶಿ ಜಗಮೋಹನ್‍ಲಾಲ್ ಮಾತನಾಡದೆ ಹೋದರು. ಒಂದು ಅಪೂರ್ವ ತೇಜೋರಾಶಿ ಇವರನ್ನು ಅಪ್ಪಳಿಸಿದಂತೆ ಇತ್ತು. ಇಂತಹ ವ್ಯಕ್ತಿಯ ಪರಿಚಯವನ್ನು ತಮ್ಮ ರಾಜರು ಮಾಡಿಕೊಂಡರೆ ಬಹಳ ಒಳ್ಳೆಯದಾಗುವುದೆಂದು ಭಾವಿಸಿ, “ಸ್ವಾಮೀಜಿ, ನಮ್ಮ ಮಹಾರಾಜರು ತಮ್ಮನ್ನು ನೋಡಬೇಕೆಂದು ಇಚ್ಛಿಸುವರು, ತಾವು ದಯವಿಟ್ಟು ಬರಬೇಕು” ಎಂದು ಕೇಳಿಕೊಂಡರು. ಅದಕ್ಕೆ ಸ್ವಾಮೀಜಿ “ಹಾಗಾದರೆ ನಾಡಿದ್ದು ಅಲ್ಲಿಗೆ ಬರುತ್ತೇನೆ” ಎಂದು ಹೇಳಿದರು. ಮುನ್ಷಿ ಜಗಮೋಹನ್ ಖೇತ್ರಿ ಮಹಾರಾಜರಿಗೆ ನಡೆದುದನ್ನೆಲ್ಲ ವಿವರಿಸಿದರು. ಇದನ್ನು ಕೇಳಿ ಆದಮೇಲೆ ಮಹಾರಾಜರ ಕುತೂಹಲ ಕೆರಳಿತು. ಅವರು ತಾವೇ ಹೋಗಿ ಈಗಲೇ ಸ್ವಾಮೀಜಿ ದರ್ಶನವನ್ನು ಪಡೇಯುತ್ತೇನೆ ಎಂದು ಹೇಳಿದರು. ಸ್ವಾಮೀಜಿಗೆ ಇದು ಗೊತ್ತಾದಾಗ ಅವರೇ ರಾಜರನ್ನು ನೋಡಲು ಹೋದರು. ಅಲ್ಲಿ ಮಹಾರಾಜರು ಗೌರವದಿಂದ ಸ್ವಾಮಿಗಳನ್ನು ಸ್ವಾಗತಿಸಿದರು. ಅವರು ಸ್ವಾಮಿಗಳನ್ನು ಜೀವನ ಎಂದರೆ ಏನು ಎಂದು ಕೇಳಿದರು. ಸ್ವಾಮೀಜಿಯವರು “ಜೀವನ ಎಂದರೆ ಒಂದು ಚೇತನ ತನ್ನನ್ನು ನುಂಗಿ ನೊಣೆಯಲು ಪ್ರಯತ್ನಿಸುತ್ತಿರುವ ವಾತಾವರಣದೊಂದಿಗೆ ಹೋರಾಡಿ, ತನ್ನಲ್ಲಿ ಸುಪ್ತವಾಗಿರುವ ಸ್ವಭಾವವನ್ನು ವ್ಯಕ್ತಪಡಿಸುವುದು” ಎಂದರು. ವಿದ್ಯಾಭ್ಯಾಸ ಎಂದರೆ ಏನು ಎಂದು ಅನಂತರ ಕೇಳಿದರು. ಅದಕ್ಕೆ ಸ್ವಾಮೀಜಿ “ಭಾವಗಳಿಗೂ ಸ್ನಾಯುಜಾಲಕ್ಕೂ ಒಂದು ಸಂಬಂಧವನ್ನುಂಟು ಮಾಡುವುದೇ ವಿದ್ಯಾಭ್ಯಾಸ” ಎಂದರು. ಅವರು ಇದನ್ನು ವಿವರಿಸಿದರು. ಮೊದಲು ನಾವು ಒಂದು ವಿಷಯವನ್ನು ಬೌದ್ಧಿಕವಾಗಿ ಪ್ರಯತ್ನಪೂರ್ವಕವಾಗಿ ಕಲಿಯುತ್ತೇವೆ. ಆಗ ಅವು ನಮ್ಮ ಮನಸ್ಸಿನ ಮೇಲೆ ಮಾತ್ರ ಇರುತ್ತವೆ. ಅವು ನಮ್ಮ ಸ್ವಭಾವ ಆಗಿರುವುದಿಲ್ಲ. ಅದು ಯಾವಾಗ ಹುಟ್ಟುಗುಣ ಆಗುವುದೊ ನಮ್ಮ ಪ್ರಜ್ಞೆಯ ಆಳಕ್ಕೆ ಹೋಗುವುದೋ, ಆಗ ರಕ್ತಗತವಾಗುವುದು, ಓತಪ್ರೋತವಾಗುವುದು. ಆಗಲೇ ನಾವು ಒಂದು ಭಾವನೆಯನ್ನು ಜೀರ್ಣಿಸಿಕೊಳ್ಳುವುದು. ಎಲ್ಲೋ ಅದನ್ನು ಅಲಂಕಾರಕ್ಕೆ ಬಾಹ್ಯದಲ್ಲಿ ಧರಿಸುವುದಲ್ಲ, ಅದು ನಮ್ಮ ಉಸಿರಲ್ಲಿ ಉಸಿರಾಗಬೇಕು, ರಕ್ತದಲ್ಲಿ ರಕ್ತವಾಗಬೇಕು. 

 ಒಂದು ದಿನ ಮಹಾರಾಜರು ಸ್ವಾಮೀಜಿ ಅವರನ್ನು ನಿಯಮ (\enginline{Law}) ಎಂದರೆ ಏನು ಎಂದು ಕೇಳಿದರು. ಅದಕ್ಕೆ ಸ್ವಾಮೀಜಿ ನಿಯಮ ಅಂತರಿಕವಾದುದು ಎಂದರು. ಇದು ಹೊರಗೆ ಬಿದ್ದಿಲ್ಲ. ಇದು ಅನುಭವ ಜನ್ಯ ಮತ್ತು ಬುದ್ಧಿಯಿಂದ ಹುಟ್ಟಿರುವುದು. ಬಾಹ್ಯದಿಂದ ಬರುವ ಘಟನೆಗಳನ್ನು ಜೋಡಿಸಿ, ಅದನ್ನು ಒಂದು ನಿಯಮಕ್ಕೆ ತರುವುದು ಮನಸ್ಸು. ಅನುಭವ ಯಾವಾಗಲೂ ಆಂತರಿಕವಾದುದು. ಪಂಚೇಂದ್ರಿಯಗಳ ದ್ವಾರ ನಮಗೆ ಒಂದು ಅನುಭವ ಬರುವುದು, ಮತ್ತು ಇದರ ಮೇಲೆ ನಿಯಮಾನುಸಾರವಾದ, ಕ್ರಮಬದ್ಧವಾದ ಬೌದ್ಧಿಕ ಪ್ರತಿಕ್ರಿಯೆಯುಂಟಾಗುವುದು. ಇದನ್ನೇ ನಿಯಮವೆಂದು ಕರೆಯುವುದು. ನಿಯಮ ಎಂಬುದು ಬೌದ್ಧಿಕವಾದುದು. ಅದು, ಸಾಪೇಕ್ಷವಾದ ಬುದ್ಧಿಯಿಂದ ಜನಿಸಿದೆ. ಸ್ವಾಮೀಜಿ, ಮಹಾರಾಜರಿಗೆ ಪಾಶ್ಚಾತ್ಯ ವಿಜ್ಞಾನ ಮತ್ತು ವೈಜ್ಞಾನಿಕ ಮಾರ್ಗ ಮುಂತಾದುವನ್ನು ಕಲಿಯಬೇಕೆಂದೂ, ಜನರಿಗೆ ಅದನ್ನು ಬೋಧಿಸಬೇಕೆಂದೂ ಹೇಳಿದರು. ತಾವೇ ಮಹಾರಾಜರಿಗೆ ಕೆಲವು ವೈಜ್ಞಾನಿಕ ಪುಸ್ತಕಗಳನ್ನು ತರಿಸಿಕೊಟ್ಟರು. 

 ಖೇತ್ರಿಯಲ್ಲಿ ಪ್ರಖ್ಯಾತ ವ್ಯಾಕರಣ ವಿದ್ಯಾಂಸರೊಬ್ಬರು ಇದ್ದರು. ಪಾಣಿನಿಯ ಸೂತ್ರದ ಮೇಲೆ ಪತಂಜಲಿ ಮಹಾಭಾಷ್ಯವನ್ನು ಬರೆದಿರುವನು. ಸ್ವಾಮೀಜಿ ಪಂಡಿತನ ಸಹಾಯದಿಂದ ಅದನ್ನು ಓದಲು ಪ್ರಾರಂಭಿಸಿದರು. ಪಂಡಿತರಿಗೆ ಇಂತಹ ಶಿಷ್ಯ ದೊರಕಿದಾಗ ಆನಂದವಾಯಿತು. ಸ್ವಾಮೀಜಿ ಪಂಡಿತರು ಹೇಳಿದ್ದನ್ನು ಕ್ಷಣಾರ್ಧದಲ್ಲಿ ಗ್ರಹಿಸಿಬಿಡುತ್ತಿದ್ದರು. ಒಂದು ಹಿಂದಿನ ದಿನ ಪಂಡಿತರು ಹಲವು ಗಂಟೆಗಳ ಕಾಲ ಹೇಳಿದ ಪಾಠವನ್ನೆಲ್ಲ ಒಪ್ಪಿಸಿ, ಅದರ ಮೇಲೆ ತಮ್ಮ ಸ್ವಂತ ಅಭಿಪ್ರಾಯವನ್ನು ಕೂಡ ಕೊಟ್ಟರು. ಅನಂತರ ಕೆಲವು ದಿನಗಳಾದ ಮೇಲೆ ಪಂಡಿತರು ಸ್ವಾಮೀಜಿಗೆ, ತಮಗೆ ಗೊತ್ತಿರುವುದನ್ನೆಲ್ಲ ತಾವು ಹೇಳಿಬಿಟ್ಟಿರುವುದಾಗಿಯೂ ಸ್ವಾಮೀಜಿ ಅದನ್ನೆಲ್ಲ ಚೆನ್ನಾಗಿ ಗ್ರಹಿಸಿರುವುದಾಗಿಯೂ, ತಮಗೆ ಇನ್ನು ಮೇಲೆ ಹೇಳಿಕೊಡುವುದಕ್ಕೆ ಏನೂ ಇಲ್ಲವೆಂದೂ ತಿಳಿಸಿದ. ಸ್ವಾಮೀಜಿ ಅವನಿಗೆ ಕೃತಜ್ಞತೆಯನ್ನು ಅರ್ಪಿಸಿದರು. 

 ಖೇತ್ರಿ ಮಹಾರಾಜರಿಗೆ ತಮ್ಮ ಸಿಂಹಾಸನಕ್ಕೆ ಉತ್ತರಾಧಿಕಾರಿಗಳಿರಲಿಲ್ಲ. ಈ ಕೊರತೆಯನ್ನು ಸ್ವಾಮೀಜಿಗೆ ವ್ಯಕ್ತಪಡಿಸಿದಾಗ, ಸ್ವಾಮೀಜಿ, “ಭಗವದಿಚ್ಛೆ ಇದ್ದರೆ ನಿನ್ನ ಅಭೀಷ್ಟ ಕೈಗೂಡಲಿ” ಎಂದು ಆಶೀರ್ವದಿಸಿದಿದರು. ಸ್ವಾಮೀಜಿ ಅರಮನೆಯಲ್ಲಿದ್ದರೂ ಅವರ ಬಳಿಗೆ ಪಂಡಿತ ಪಾಮರರು ಬರುತ್ತಿದ್ದರು. ಖೇತ್ರಿ ಮಹಾರಾಜರು ಸ್ವಾಮೀಜಿಯ ಪರಮ ಭಕ್ತರಾದರು. ಸ್ವಾಮೀಜಿಗೆ ಕೈಕಾಲುಗಳನ್ನು ಕೂಡ ಒತ್ತಲು ಅವರು ಕಾತರರಾಗಿದ್ದರು ಆದರೆ ಇತರರೆದುರಿಗೆ ಸ್ವಾಮೀಜಿ ಇದಕ್ಕೆ ಅವಕಾಶ ಕೊಡುತ್ತಿರಲಿಲ್ಲ. ಇದರಿಂದ ರಾಜರ ಗೌರವಕ್ಕೆ ಕುಂದು ಬಂದೀತು ಎಂದು ಹೇಳುತ್ತಿದ್ದರು. ಸ್ವಾಮೀಜಿ ಖೇತ್ರಿ ಮಹಾರಾಜರಿಗೆ ಅವರ ರಾಜ್ಯದಲ್ಲಿ ನಡೆದ ಒಂದು ಘಟನೆಯನ್ನು ಹೇಳಿದರು. ಒಂದು ಸಲ ಸ್ವಾಮೀಜಿ ಖೇತ್ರಿಗೆ ಸಮೀಪದಲ್ಲಿರುವ ಒಂದು ಊರಿನಲ್ಲಿ ತಂಗಿದ್ದರು. ಸ್ವಾಮೀಜಿಯವರ ಮಾತುಕತೆಗಳನ್ನು ಕೇಳುವುದಕ್ಕೆ ಮೂರು ದಿನಗಳವರೆಗೆ ಜನರು ಬಂದು ಹೋಗುತ್ತಿದ್ದರು. ಸ್ವಾಮೀಜಿಗೆ ಬಿಡುವೇ ಇರಲಿಲ್ಲ. ಅವರ ಊಟ ಉಪಚಾರವನ್ನು ಯಾರೂ ವಿಚಾರಿಸಲಿಲ್ಲ. ಎಲ್ಲಾ ಸ್ವಾಮಿಗಳಿಂದ ತೆಗೆದುಕೊಂಡು ಹೋಗುವುದರಲ್ಲೆ ಸಿದ್ಧರಾಗಿದ್ದರೇ ಹೊರತು ಯಾರೂ ಕೊಡುವುದರ ಕಡೆಗೆ ಗಮನವನ್ನೇ ಕೊಡಲಿಲ್ಲ. ಮೂರು ದಿನಗಳಾದ ಮೇಲೆ ಅಂತ್ಯಜನೊಬ್ಬನು ಸ್ವಾಮೀಜಿ ಬಳಿಗೆ ಬಂದು, “ಸ್ವಾಮೀಜಿ, ಮೂರು ದಿನಗಳಿಂದ ನೀವು ಏನನ್ನೂ ತೆಗೆದುಕೊಂಡಿಲ್ಲ. ಒಂದು ಲೋಟ ನೀರನ್ನು ಕೂಡ ಕುಡಿದಂತೆ ಕಾಣೆ” ಎಂದನು. ಅದಕ್ಕೆ ಸ್ವಾಮೀಜಿ, ಆತನನ್ನು ತನಗೆ ಏನಾದರೂ ಕೊಡಲು ಸಾಧ್ಯವೆ ಎಂದು ಕೇಳಿದರು. ಆತ ತಾನು ಅಂತ್ಯಜನೆಂದೂ, ಸ್ವಾಮೀಜಿಗೆ ಊಟ ಕೊಡುವ ಉತ್ತಮ ಜಾತಿಯವನಲ್ಲವೆಂದೂ ಹೇಳಿ, “ಅಡಿಗೆ ಮಾಡಿಕೊಳ್ಳುವುದಕ್ಕೆ ಸಾಮಾನನ್ನು ತಂದು ಕೊಡುವೆ, ಬೇಕಾದರೆ ನೀವೇ ಅಡಿಗೆ ಮಾಡಿಕೊಳ್ಳಿ” ಎಂದನು. ಸ್ವಾಮೀಜಿ ಅಂತ್ಯಜನ ಕೈಯಿಂದಲೂ ತಾವು ಊಟ ಮಾಡಲು ಸಿದ್ಧವಾಗಿರುವೆ ಎಂದರು. ಆದರೆ ಆತ “ನಮ್ಮ ಮಹಾರಾಜರಿಗೆ ಗೊತ್ತಾದರೆ ಅವರು ನನ್ನನ್ನು ತಮ್ಮ ರಾಜ್ಯದಿಂದಲೇ ಓಡಿಸಿಬಿಡಬಹುದು” ಎಂದನು. ಸ್ವಾಮೀಜಿ, ಅದು ರಾಜರಿಗೆ ತಿಳಿಯುವಂತೆ ಇಲ್ಲ ಎಂದಮೇಲೆ, ಆತ ತಾನು ಮಾಡಿದ ಅಡಿಗೆಯನ್ನು ತಂದು ಕೊಟ್ಟ. ಇಂದ್ರ ಬಡಿಸಿದ ಅಮೃತಕ್ಕಿಂತ ಅದು ಚೆನ್ನಾಗಿತ್ತು ಎಂದು ಅನಂತರ ಸ್ವಾಮಿಗಳು ಹೇಳುತ್ತಿದ್ದರು. ಖೇತ್ರಿ ಮಹಾರಾಜರಿಗೆ ಅವನ ವಿಷಯವನ್ನು ಹೇಳಿದ ಮೇಲೆ, ಖೇತ್ರಿ ಮಹಾರಾಜರು ಅವನನ್ನು ಹುಡುಕಿಸಿ ಅರಮನೆಗೆ ಕರೆಸಿದರು. ಆತನಿಗೆ ಬೇಕಾದಷ್ಟು ಹಣವನ್ನು ಬಹುಮಾನವಾಗಿ ಕೊಟ್ಟು ಕಳುಹಿಸಿದರು. 

 ಸ್ವಾಮೀಜಿ ರಾಜಾಸ್ಥಾನ ಅಥವಾ ಉತ್ತರಪ್ರದೇಶಗಳಲ್ಲಿ ಸಂಚರಿಸುತ್ತಿದ್ದಾಗ ಆದ ಕೆಲವು ಇತರ ಘಟನೆಗಳು ಚಿತ್ತಾಕರ್ಷಕವಾಗಿವೆ. ಅದರಲ್ಲಿ ಕೆಲವನ್ನು ಇಲ್ಲಿ ನಿರೂಪಿಸಿರುವೆವು. 

 ಒಂದು ಸಲ ಸ್ವಾಮೀಜಿ ರಾಜಾಸ್ಥಾನದಲ್ಲಿ ರೈಲಿನಲ್ಲಿ ಹೋಗುತ್ತಿದ್ದಾಗ ತಮ್ಮ ಗಾಡಿಯಲ್ಲೆ ಇಬ್ಬರು ಇಂಗ್ಲೀಷಿನವರು ಹೋಗುತ್ತಿದ್ದರು. ಸ್ವಾಮೀಜಿ ಎರಡನೇ ತರಗತಿಯ ಒಂದು ಸೀಟಿನಲ್ಲಿ ಮಲಗಿಕೊಂಡು ಹೋಗುತ್ತಿದ್ದರು. ಇಬ್ಬರು ಇಂಗ್ಲೀಷಿನವರು ಆಂಗ್ಲಭಾಷೆಯಲ್ಲಿ ಸಂನ್ಯಾಸಿಗಳನ್ನು ಕುರಿತು ಹೀನವಾಗಿ ನಿಂದಿಸುತ್ತಿದ್ದರು. ಇದನ್ನೆಲ್ಲ ಕೇಳಿಯೂ ಸುಮ್ಮನೆ ಇದ್ದರು. ಆ ಇಂಗ್ಲೀಷಿನವರಾದರೊ ಈ ಸ್ವಾಮಿಗಳು ಒಬ್ಬ ಭಿಕ್ಷುಕರು, ಇವರಿಗೆ ಗೊತ್ತಾಗುವುದಿಲ್ಲ ಎಂದು ಭಾವಿಸಿದ್ದರು. ಮುಂದಿನ ಸ್ಟೇಷನ್ ಬಂದಾಗ ಸ್ವಾಮೀಜಿ ಮಾಸ್ಟರರನ್ನು ಇಂಗ್ಲೀಷಿನಲ್ಲಿ ತನಗೆ ಒಂದು ಲೋಟ ನೀರು ಬೇಕು ಎಂದು ಕೇಳಿದರು. ಅದೇ ಗಾಡಿಯಲ್ಲಿ ಸ್ವಾಮೀಜಿ ಅವರನ್ನು ನಿಂದಿಸ್ದುತ್ತಿದ್ದ ಇಂಗ್ಲೀಷಿನವರಿಗೆ ಆಶ್ಚರ್ಯವಾಗಿ, ಸ್ವಾಮೀಜಿಯವರನ್ನು ತಮಗೆ ಇಂಗ್ಲೀಷ್ ಗೊತ್ತಿದ್ದರೂ ಏತಕ್ಕೆ ಏನನ್ನೂ ಹೇಳಲಿಲ್ಲವೆಂದರು. ಅದಕ್ಕೆ ಸ್ವಾಮೀಜಿ, “ಸಹೊದರರೇ, ಇದೇ ಪ್ರಥಮ ಬಾರಿಯಲ್ಲ ನಾನು ಮೂರ್ಖರನ್ನು ನೋಡುವುದು!” ಎಂದರು. ಅವರು ಸ್ವಾಮಿಗಳೊಡನೆ ಜಗಳಕಾಯಬೇಕೆಂದಿದ್ದರು. ಆದರೆ ಸ್ವಾಮೀಜಿ ಅವರ ಮೈಕಟ್ಟು ಮತ್ತು ಅವರ ಪಾಂಡಿತ್ಯಕ್ಕೆ ಅಂಜಿ ಕ್ಷಮಾಪಣೆಯನ್ನು ಕೋರಿ ತೆಪ್ಪಗಾದರು. 

 ಒಂದು ಸಲ ಸ್ವಾಮೀಜಿ ಸಿಂಧೂನದಿಯ ತೀರದಲ್ಲಿ ಸಂಚಾರ ಮಾಡುತ್ತಿದ್ದಾಗ ಒಬ್ಬ ವೃದ್ಧ ನದೀ ತೀರದಲ್ಲಿ ನಿಂತುಕೊಂಡು ವೇದದ ಒಂದು ಮಂತ್ರವನ್ನು ಬಹಳ ಸುಂದರವಾಗಿ ಉಚ್ಚರಿಸುತ್ತಿದ್ದುದನ್ನು ನೋಡಿದರು. ಇದರಿಂದ ಪುರಾತನ ಆರ್ಯರು ಈ ರೀತಿ ಉಚ್ಚರಿಸುತ್ತಿದ್ದಿರಬಹುದು ಎಂಬುದನ್ನು ಭಾವಿಸತೊಡಗಿದರು. 

 ಒಂದು ಸಲ ಸ್ವಾಮೀಜಿ ಮೂರನೆ ತರಗತಿಯ ಬಂಡಿಯಲ್ಲಿ ರೈಲಿನಲ್ಲಿ ಹೋಗುತ್ತಿದ್ದರು. ಭಕ್ತರು ಯಾರೋ ಟಿಕೀಟನ್ನು ತೆಗೆದುಕೊಟ್ಟಿದ್ದರು. ಸ್ವಾಮೀಜಿ ಆಗ ದುಡ್ಡನ್ನು ಮುಟ್ಟುತ್ತಿರಲಿಲ್ಲ. ಊಟವನ್ನು ಯಾರಾದರೂ ಕೊಟ್ಟಲ್ಲದೆ ಕೇಳುತ್ತಿರಲಿಲ್ಲ. ಒಂದು ದಿನದಿಂದ ಊಟವಿರಲಿಲ್ಲ. ಬಾಯಾರಿಕೆ, ಬಿಸಿಲ ಝಳ. ಆಗ ನೀರನ್ನು ಕುಡಿಯಬೇಕಾದರೆ ದುಡ್ಡನ್ನು ಕೊಟ್ಟು ಕೊಂಡುಕೊಳ್ಳಬೇಕಾಗಿತ್ತು. ಅದೇ ಗಾಡಿಯಲ್ಲಿ ಬನಿಯ ವರ್ತಕನೊಬ್ಬ ಹೋಗುತ್ತಿದ್ದ. ಆತ ನೀರನ್ನು ಕೊಂಡು ತಾನು ಕುಡಿಯುತ್ತ, ಸೋಮಾರಿ ಸಂನ್ಯಾಸಿ ಜೀವನವನ್ನು ಕೈಕೊಂಡ ಸ್ವಾಮೀಜಿಯವರನ್ನು ಜರೆಯುತ್ತಿದ್ದ. ಹಣ ಸಂಪಾದನೆ ಮಾಡಿ ಸುಖ ಅನುಭವಿಸುವುದು ಅವನ ತತ್ತ್ವವಾಗಿತ್ತು. ಸ್ವಾಮೀಜಿಯವರ ಜೀವನಾದರ್ಶವೋ ಬೇರೆಯಾಗಿತ್ತು. ಅವರಿಬ್ಬರೂ ಟೆಹರಿ ಸ್ಟೇಷನ್ನಿನಲ್ಲಿ ಇಳಿದರು. ಆತ ಒಂದು ಮರದ ಕೆಳಗೆ ಕುಳಿತು ಹತ್ತಿರ ಇದ್ದ ಅಂಗಡಿಯಿಂದ ಬಿಸಿ ಬಿಸಿ ಪೂರಿ ಖಚೋರಿಯನ್ನು ತರಿಸಿ, ಸ್ವಾಮಿಗಳೆದುರಿಗೆ ಅವರನ್ನು ಹಾಸ್ಯಮಾಡಿ ತಾನು ತಿನ್ನುತ್ತಿದ್ದ. ಸ್ವಾಮೀಜಿಗೆ ಹೊಟ್ಟೆ ಹಸಿಯುತ್ತಿತ್ತು. ಆದರೆ ಯಾರಾದರೂ ತಾವೇ ಆಹಾರವನ್ನು ಕೊಟ್ಟರೆ ಮಾತ್ರ ಸ್ವೀಕರಿಸುವ ವ್ರತವನ್ನು ಆಗ ಕೈಗೊಂಡಿದ್ದುದರಿಂದ ಸುಮ್ಮನೆ ಒಂದು ಕಡೆ ನಿಂತಿದ್ದರು. ಆ ಸಮಯಕ್ಕೆ ಸರಿಯಾಗಿ ಒಬ್ಬ ಮಿಠಾಯಿ ಅಂಗಡಿಯವನು ಒಂದು ತಟ್ಟೆಯಲ್ಲಿ ಬಗೆಬಗೆಯ ತಿಂಡಿಗಳನ್ನೆಲ್ಲ ಇಟ್ಟು, ಒಂದು ಚಂಬಿನಲ್ಲಿ ನೀರು ಮತ್ತು ಕುಳಿತುಕೊಳ್ಳುವುದಕ್ಕೆ ಒಂದು ಕುರ್ಚಿಯನ್ನು ತಂದು, ಸ್ವಾಮೀಜಿಗೆ ದಯವಿಟ್ಟು ಕುಳಿತುಕೊಂಡು ಸ್ವೀಕರಿಸಿ ಎಂದು ಕೇಳಿಕೊಂಡ. ಆಗ ಸ್ವಾಮೀಜಿ ಆತನಿಗೆ “ನೀನು ಯಾರನ್ನೊ ತಪ್ಪು ಭಾವಿಸಿರಬೇಕು. ನಾನು ಯಾರಿಗೂ ತಿಂಡಿ ತರಲು ಹೇಳಿರಲಿಲ್ಲ. ಅದೂ ಅಲ್ಲದೆ ನಿನ್ನ ಪರಿಚಯವೇ ನನಗೆ ಇಲ್ಲ” ಎಂದು ಹೇಳಿದರು. ಆತ, “ಇಲ್ಲ ನಿಮ್ಮನ್ನೆ ನಾನು ಇವತ್ತು ಮಧ್ಯಾಹ್ನ ಕನಸಿನಲ್ಲಿ ನೋಡಿದ್ದು” ಎಂದು ತನಗೆ ಆದ ಕನಸನ್ನು ವಿವರಿಸತೊಡಗಿದ. ಅಂದು ಮಧ್ಯಾಹ್ನ ಆತ ವ್ಯಾಪಾರವನ್ನು ಪೂರೈಸಿ ಅಂಗಡಿಯಲ್ಲಿ ಮಲಗಿದ್ದಾಗ ರಾಮ ಅವನಿಗೆ ಕಾಣಿಸಿಕೊಂಡು, ಸ್ವಾಮೀಜಿಯವರನ್ನು ತೋರಿ, ಅವರು ನನ್ನ ಭಕ್ತರು, ಊಟವಿಲ್ಲದೆ ಇರುವರು, ಅವರಿಗೆ ಊಟ ತೆಗೆದುಕೊಂಡು ಹೋಗಿ ಕೊಡು ಎಂದು ಆಜ್ಞಾಪಿಸಿದನು. ಅವನು ಇದು ಬರೀ ಕನಸು ಎಂದು ಹಾಗೆಯೇ ನಿದ್ರಿಸುತ್ತಿದ್ದ. ಮತ್ತೊಮ್ಮೆ ರಾಮ ಕನಸಿನಲ್ಲಿ ಕಾಣಿಸಿಕೊಂಡು “ನಿನಗೆ ಬೇಗ ತೆಗೆದುಕೊಂಡು ಹೋಗಿ ಕೊಡುವಂತೆ ಹೇಳಿದೆ” ಎಂದನು. ಇದನ್ನು ಕೇಳಿದಮೇಲೆ ಸ್ವಾಮೀಜಿ, ಭಗವಂತ ತನ್ನನ್ನು ನಂಬಿದವರನ್ನು ಕೈಬಿಡುವುದಿಲ್ಲ ಎಂಬುದನ್ನು ಹೇಗೆ ತೋರಿಸಿರುವನು ಎಂಬುದನ್ನು ಅರಿತು ಕಂಬನಿದುಂಬಿ ಆಹಾರಕೊಟ್ಟವನಿಗೆ ಕೃತಜ್ಞತೆ ಅರ್ಪಿಸಿದರು. ಅದಕ್ಕೆ ಆತ ಅದೆಲ್ಲ ರಾಮನ ಇಚ್ಛೆ, ತಾನು ಯಾವ ಗೌರವಕ್ಕೂ ಪಾತ್ರನಲ್ಲ ಎಂದನು. ಇದನ್ನೆಲ್ಲ ನೋಡುತ್ತಿದ್ದ ಬನಿಯನಿಗೆ ನಾಚಿಕೆಯಾಯಿತು. ಇಂತಹ ಮಹಾತ್ಮರಿಗೆ ಕೊಡುವ ಅವಕಾಶ ಒದಗಿದ್ದರೂ ಅದನ್ನು ನಿರಾಕರಿಸಿ ಅವರನ್ನು ಜರೆದು ಪಾಪ ಕಟ್ಟಿಕೊಂಡೆ. ಸ್ವಾಮೀಜಿಯವರನ್ನು ನೋಡದ ಕೇಳದ ಯಾರೋ ಬಂದು ಅವರಿಗೆ ಆಹಾರ ಕೊಟ್ಟು ಪುಣ್ಯ ಸಂಪಾದಿಸಿದನು. ಬನಿಯನಿಗೆ ಪಶ್ಚಾತ್ತಾಪವಾಗಿ ಸ್ವಾಮೀಜಿ ಪಾದಗಳಿಗೆ ಬಿದ್ದು ಕ್ಷಮಾಪಣೆ ಬೇಡಿದನು. 

 ಮೊತ್ತೊಮ್ಮೆ ಸ್ವಾಮೀಜಿ ರೈಲಿನಲ್ಲಿ ಹೋಗುತ್ತಿದ್ದಾಗ ಒಬ್ಬ ವಿದ್ಯಾವಂತನನ್ನು ಕಂಡರು. ಆತ ವಿದ್ಯಾವಂತನಾದರೂ ಅದೃಶ್ಯವಾಗಿ ವಾಸಿಸುತ್ತಿರುವ ಮಹಾತ್ಮರುಗಳು, ಅವರು ಏನೇನೊ ಅದ್ಭುತಗಳನ್ನು ಮಾಡುತ್ತಾರೆ; ಅವರು ಸಾವಿರಾರು ವರುಷಗಳಿಂದಲೂ ಹಿಮಾಲಯದಲ್ಲಿ ಜೀವಿಸಿಕೊಂಡಿರುವರು, ಅವರು ವಿಚಿತ್ರ ರೀತಿಯಲ್ಲಿ ಜೀವಿಸುತ್ತಿರುವರು ಎಂಬ ನಂಬಿಕೆ ಅವನನ್ನು ಮೆಟ್ಟಿಕೊಂಡಿತ್ತು. ಸ್ವಾಮೀಜಿ ಹಿಮಾಲಯದಲ್ಲೆಲ್ಲ ಸಂಚಾರ ಮಾಡಿಕೊಂಡು ಬಂದಿರುವರು ಎಂಬುದನ್ನು ಕೇಳಿ ಅಂತಹ ಮಹಾತ್ಮರ ದರ್ಶನ ಸ್ವಾಮೀಜಿಗೆ ಆಗಿದೆಯೆ ಎಂದು ಕೇಳಿದ. ಅದಕ್ಕೆ ಸ್ವಾಮೀಜಿ ಅಂತಹವರನ್ನು ತಾವು ಕಂಡು ಮಾತನಾಡಿರುವುದಾಗಿಯೂ, ಪ್ರಪಂಚವನ್ನು ಉದ್ಧಾರಮಾಡಲು ಅವರು ಏನೇನು ಮಾಡಬೇಕೆಂದಿರುವರು ಎಂಬುದನ್ನು ತಮಗೆ ಹೇಳಿರುವುದಾಗಿಯೂ ತಿಳಿಸಿದರು. ಕುಳಿತು ಕೇಳುತ್ತಿದ್ದವನ ಉತ್ಸಾಹ ಇನ್ನೂ ಹೆಚ್ಚು ಕೆರಳಿತು. ಆತ ತಾನು ತಂದ ತಿಂಡಿತೀರ್ಥಗಳನ್ನು ಸ್ವಾಮೀಜಿಗೆ ಕೊಟ್ಟ. ಸ್ವಾಮೀಜಿ ಅದನ್ನು ತಿಂದಾದ ಮೇಲೆ ಆ ಮನುಷ್ಯನಿಗೆ ಬುದ್ಧಿ ಹೇಳಿದರು. ಅದ್ಭುತಗಳನ್ನು ಮಾಡುವುದಕ್ಕೂ ಆಧ್ಯಾತ್ಮಿಕ ಜೀವನಕ್ಕೂ ಏನೂ ಸಂಬಂಧವಿಲ್ಲವೆಂದೂ, ಹಾಗೆ ಮಾಡುತ್ತಿರುವವರೆಲ್ಲ ಭ್ರಷ್ಟರೆಂದೂ, ಆಧ್ಯಾತ್ಮಿಕ ಜ್ಝೀವನದ ತಳಹದಿಯೇ ಶುದ್ಧಚಾರಿತ್ರವೆಂದೂ ಹೇಳಿದರು. ಯಾವಾಗ ಮಾಯ ಮಂತ್ರಗಳ ಕಡೆ ಮನಸ್ಸು ಹೋಗುವುದೋ ಆಗ ದೇವರನ್ನೂ ಮರೆತು ಅವುಗಳನ್ನು ಜನರಿಗೆ ತೋರಿಸಿ ಅದರಿಂದ ಹಣ ಕೀರ್ತಿ ಇವುಗಳನ್ನು ಪಡೆಯುವ ಕಡೆ ಗಮನ ಕೊಡುವರೆಂದೂ ಅಂತಹವರನ್ನು ಕಂಡರೆ ಜೋಪಾನವಾಗಿರಬೇಕೆಂದೂ ಹೇಳಿದರು. 

 ಒಂದು ಸಲ ಸ್ವಾಮೀಜಿ ಪರಿವ್ರಾಜಕರಾಗಿ ಅಲೆಯುತ್ತಿದ್ದಾಗ ತಮ್ಮ ಜೀವನವನ್ನೇ ಕುರಿತು ಯೋಚಿಸುತ್ತ, “ತಾನು ಏನನ್ನೂ ಸಾಧಿಸಿಲ್ಲ, ಬರೀ ಕಾಗೆಗಳಂತೆ ಮತ್ತೊಬ್ಬರು ಕೊಡುವ ಊಟವನ್ನು ತಿಂದು ಹೊಟ್ಟೆ ಹೊರೆಯುತ್ತಿರುವೆ. ಇಂತಹ ಜೀವನದಿಂದ ಪ್ರಯೋಜನವೇನು. ಕಾಡಿಗೆ ಹೋಗಿ ಪ್ರಾಯೋಪವೇಶ ಮಾಡಿ ಸಾಯುವೆ” ಎಂದು ಕಾಡಿನ ಒಳಗೆ ನಡೆದುಕೊಂಡು ಹೋಗುತ್ತಿದ್ದರು. ಆಗ ದೂರದಲ್ಲಿ ಇವರ ಬಳಿಗೆ ಒಂದು ಹುಲಿ ಬರುತ್ತಿದ್ದುದು ಕಂಡಿತು. ಆಗ ಸ್ವಾಮೀಜಿಗೂ ಹಸಿವಿತ್ತು, ಹುಲಿಯೂ ಹಸಿದಿತ್ತು. ಆ ಹುಲಿಯಾದರೂ ಈ ದೇಹವನ್ನು ತಿಂದು ಹಸಿವನ್ನು ಹಿಂಗಿಸಿಕೊಳ್ಳಲಿ ಎಂದು ಅಲ್ಲಿಯೆ ಮರದ ಕೆಳಗೆ ಧ್ಯಾನಮಗ್ನರಾಗಿ ಕುಳಿತರು. ಆದರೆ ಆ ಹುಲಿ ಸ್ವಲ್ಪ ಹತ್ತಿರ ಬಂದು ಅನಂತರ ಎಲ್ಲಿಯೋ ಓಡಿಹೋಯಿತು. ಎಷ್ಟು ಕಾಲವಾದರೂ ಬರಲಿಲ್ಲ. ರಾತ್ರಿಯೆಲ್ಲ ಆ ಮರದ ಕೆಳಗೆ ಕಳೆದರು. ಬೆಳಗಾಗುವ ಹೊತ್ತಿಗೆ ಕತ್ತಲೆಯ ಜೊತೆಗೆ ಇರುವ ಮನಸ್ಸಿನ ದೌರ್ಬಲ್ಯವೂ ಹೊರಟು ಹೋಗಿತ್ತು. ಒಂದು ಹೊಸ ಶಕ್ತಿ, ಹೊಸ ದೃಷ್ಟಿ, ಇವರ ಜೀವನವನ್ನು ಪ್ರವೇಶಿಸಿತು. ತಮ್ಮ ಬಾಳು ಭಗವದರ್ಪಣೆಗೆ ನಿವೇದಿಸಿದ ಬಾಳು, ಅವನು ಅವನ ಕೆಲಸವನ್ನು ಮಾಡುವುದಕ್ಕೆ ಇದನ್ನು ನಿಮಿತ್ತ ಮಾಡಿಕೊಂಡಿರುವನು ಎಂಬುದನ್ನು ಅರಿತರು. 

 ಮತ್ತೊಮ್ಮೆ ರಾಜಾಸ್ಥನದಲ್ಲಿ ಮರಳು ಕಾಡಿನಲ್ಲಿ ನಡೆದುಕೊಂಡು ಹೋಗುತ್ತಿದ್ದಾಗ\break ಬಾಯಾರಿತು. ದೂರದಲ್ಲಿ ಒಂದು ಸರೋವರ ಇರುವಂತೆ ಕಂಡಿತು. ಆ ಸರೋವರದ ಸುತ್ತಲೂ ಹುಲುಸಾಗಿ ಬೆಳೆದಿದ್ದ ಗಿಡ ಮರಗಳು ಎಲ್ಲವೂ ಕಂಡವು. ಸ್ವಾಮೀಜಿ ಎಷ್ಟು ನಡೆದುಕೊಂಡು ಹೋದರೂ ಅದು ಸಿಕ್ಕಲಿಲ್ಲ. ಇನ್ನೂ ಮುಂದೆ ಮುಂದೆ ಹೋಗುತ್ತಿತ್ತು. ಅನಂತರ ಅದು ಮರೀಚಿಕೆ ಎಂದು ಗೊತ್ತಾಯಿತು. ಮಾಯೆ ಎಂದರೆ ಮರೀಚಿಕೆಯಂತೆ ಎಂಬುದನ್ನು ನಾವು ಓದಿರುವೆವು. ಸ್ವಾಮೀಜಿ ಆಗ ಅದನ್ನು ನೋಡಿದರು. ಅನಂತರ ಅದನ್ನು ಹಲವು ಉಪನ್ಯಾಸಗಳಲ್ಲಿ ಉದಾಹರಿಸುತ್ತಿದ್ದರು. ಅಲ್ಲಿ ಏನೂ ಇಲ್ಲ, ಆದರೆ ಇಲ್ಲಿ ಏನೋ ಇರುವಂತೆ ಭಾವಿಸಿ ಅದನ್ನು ಅನುಸರಿಸಿ ಹೋಗುವೆವು. ಕೊನೆಗೆ ನಮ್ಮ ಅಭೀಷ್ಟ ನೆರವೇರುವುದಿಲ್ಲ, ಬಾಳೆ ಅದಕ್ಕೆ ಬಲಿಯಾಗುವುದು. 

 ಸ್ವಾಮೀಜಿ ಖೇತ್ರಿಯಿಂದ ಅಹಮ್ಮದಾಬಾದಿನ ಕಡೆ ಹೊರಟರು. 

