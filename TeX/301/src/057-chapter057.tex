
\chapter{ಮಹಾಸಮಾಧಿ }

\vskip 2pt

 ಶ‍್ರೀರಾಮಕೃಷ್ಣ ಪರಮಹಂಸರು ದಕ್ಷಿಣೇಶ್ವರದಲ್ಲಿ ಧ್ಯಾನಮಗ್ನರಾಗಿದ್ದಾಗ ಒಂದು ಅಖಂಡ-ತೇಜೋರಾಶಿ ಘನೀಭೂತವಾಗಿ ಧರೆಗಿಳಿದು ಕಲ್ಕತ್ತೆಯ ಒಂದು ಮನೆಯನ್ನು ಪ್ರವೇಶಿಸಿದುದನ್ನು ಕಂಡರು. ಕೆಲವು ವರ್ಷಗಳಾದ ಮೇಲೆ ನರೇಂದ್ರನೆಂಬ ವ್ಯಕ್ತಿ ಸಂಶಯದಲ್ಲಿ ಒದ್ದಾಡುತ್ತ ಅತೃಪ್ತಿಯಲ್ಲಿ ಬೇಯುತ್ತ ದೇವರಿರುವನೆ ಎಂದು ಕೇಳುತ್ತ ಬಂದುದನ್ನು ನೋಡಿದರು. ಹಿಂದೆ ನೋಡಿದ ದಿವ್ಯ ಜ್ಯೋತಿಯೇ ಈ ಮಾಯಾವರಣಕ್ಕೆ ಇಳಿದ ಮೇಲೆ ಸ್ವಲ್ಪ ಅದರ ಕೆಸರು ಅಂಟಿಕೊಂಡು ಹಿಂದಿನದನ್ನು ಮರೆಸಿತ್ತು. 

\vskip 2pt

 ಶ‍್ರೀರಾಮಕೃಷ್ಣರು ತಮ್ಮ ಸ್ಪರ್ಶ ಮಾತ್ರದಿಂದ ಆ ಕೆಸರನ್ನು ಕೊಡವಿದೊಡನೆಯೇ ತಮ್ಮ ಹಿಂದಿನ ಸ್ಥಿತಿಗೆ ಹೋದುದನ್ನು ನೋಡಿದರು. ತಮ್ಮ ಸಂದೇಶವನ್ನು ಜಗತ್ತಿಗೆ ಸಾರುವುದಕ್ಕೆ ಇವರೇ ಯೋಗ್ಯ ವ್ಯಕ್ತಿ ಎಂದು ನಿಶ್ಚಯಿಸಿ ಪುನಃ ಪ್ರಜ್ಞೆಯ ಪ್ರಪಂಚಕ್ಕೆ ತಂದು ತಮ್ಮ ಬೆಳಕನ್ನು ಜಗತ್ತಿಗೆ ಹರಡುವುದಕ್ಕೆ ಅವರನ್ನು ಯೋಗ್ಯವಾದ ಮಧ್ಯವರ್ತಿಯನ್ನಾಗಿ ಮಾಡಿದರು. ಆದರೂ ನರೇಂದ್ರ ನಿರ್ವಿಕಲ್ಪ ಸಮಾಧಿಯರುಚಿ ನೋಡಬೇಕೆಂದು ಪರಮಹಂಸರನ್ನು ಕಾಡಿ ಒಂದು ದಿನ ಅದನ್ನು ಅನುಭವಿಸಿದನು. ಅನಂತರ ಶ‍್ರೀರಾಮಕೃಷ್ಣರು ಬಂದು, “ ಈ ಅನುಭವವನ್ನು ಒಂದು ಮಾವಿನ ಹಣ್ಣಿನಂತೆ ಒಂದು ಪೆಟ್ಟಿಗೆಯಲ್ಲಿಟ್ಟು ಬೀಗ ಹಾಕಿರುವೆನು. ನೀನು ಯಾವುದಕ್ಕೆ ಬಂದೆಯೋ ಆ ಕೆಲಸವನ್ನು ಮಾಡಿದ ಮೇಲೆಯೇ ಆ ರುಚಿಯನ್ನು ನೋಡಬೇಕಾದರೆ” ಎಂದಿದ್ದರು. 

\vskip 2pt

 ನರೇಂದ್ರ ತಮ್ಮ ಗುರುಗಳಾಣತಿಯನ್ನು ಪಾಲಿಸುವುದಕ್ಕೆ ವಿವೇಕಾನಂದರಾದರು. ಭರತಖಂಡದಲ್ಲಿ ತಲೆಯಿಂದ ಕಾಲಿನವರೆಗೆ ಅಲೆದರು. ಜನರ ದೌರ್ಬಲ್ಯವೇನು, ಅವರಲ್ಲಿ ಎಂತಹ ಆದರ್ಶ ಹಿಂದೆ ಇತ್ತು, ಈಗ ಎಂತಹ ಅಧೋಗತಿಗೆ ಇಳಿದಿರುವರು, ಅವರನ್ನು ಪುನಃ ಮೇಲೆತ್ತಬೇಕಾದರೆ ಏನು ಮಾಡಬೇಕು ಎಂಬುದನ್ನೆಲ್ಲ ಕನ್ಯಾಕುಮಾರಿಯ ಕೊನೆಯ ಬಂಡೆಯ ಮೇಲೆ ಕುಳಿತುಕೊಂಡು ವಿಹಂಗಮ ದೃಷ್ಟಿಯಲ್ಲಿ ಆಲೋಚಿಸತೊಡಗಿದರು. ತಾವು ಏನು ಮಾಡಬೇಕು ಎಂಬುದನ್ನು ನಿರ್ಧರಿಸಿದರು. ಅಲ್ಲಿಂದ ಅಮೇರಿಕಾ ದೇಶಕ್ಕೆ ಹೋದರು. ವಿಶ್ವಧರ್ಮ ಸಮ್ಮೇಳನದಲ್ಲಿ ಮಾತನಾಡಿದರು. ಸನಾತನ ಧರ್ಮದ ಧ್ವಜವನ್ನು ಅಲ್ಲಿ ಎತ್ತಿ ಹಿಡಿದರು. ಜಗದ ನಿಕೃಷ್ಟ ದೃಷ್ಟಿಗೆ ಪಾತ್ರವಾದ ಭರತಖಂಡಕ್ಕೆ ಗೌರವ ತಂದರು. ಒಂದು ಕಡೆ ಕ್ರೈಸ್ತ ಧರ್ಮದ ಸಂಕುಚಿತ ದೃಷ್ಟಿ, ಮತ್ತೊಂದು ಕಡೆ ವಿಜ್ಞಾನದ ಪ್ರಗತಿಯ ಪರಿಣಾಮವಾಗಿ ಅದರ ಛಾಯೆಯಂತೆ ಬಂದ ಜಡವಾದ ನಿರೀಶ್ವರವಾದ ಮುಂತಾದ ಭಾವನೆಗೆ ತುತ್ತಾದ ಜನರಿಗೆ ಒಂದು ಹೊಸ ಭರವಸೆಯಂತೆ ಬಂತು ಸ್ವಾಮೀಜಿಯವರ ವಾಣಿ. ಇದರಲ್ಲಿ ಸಂಕೋಚ ದೃಷ್ಟಿಯಿಲ್ಲ. ಆಕಾಶದಷ್ಟು ವಿಶಾಲವಾಗಿದೆ, ಸಾಗರದಷ್ಟು ಆಳವಾಗಿದೆ. ವಿಜ್ಞಾನವನ್ನು ಧಿಕ್ಕರಿಸುವುದಿಲ್ಲ. ವಿಚಾರವನ್ನು ಅಲ್ಲಗಳೆಯುವುದಿಲ್ಲ. ಅವನ್ನು ತೆಗೆದುಕೊಂಡು, ಅವು ಹೋಗುವವರೆಗೂ ಹೋಗಿ, ಅವು ಎಲ್ಲಿ ಬಸವಳಿದು ನಿಲ್ಲುವವೋ ಅಲ್ಲಿಂದ ಮುಂದಕ್ಕೆ ವೇದಾಂತ ಪ್ರಯಾಣ ಮಾಡುವುದೆಂದು ತೋರಿದರು. 

 ಭರತಖಂಡದಲ್ಲಿ ಸ್ವಾಮೀಜಿಯವರ ಕೀರ್ತಿ ಹೆಚ್ಚು ಪ್ರಭಾವಶಾಲಿಯಾಯಿತು. ಇಲ್ಲಿಯ ಜನರಿಗೆ ಒಂದು ಗೌರವ ತಂದಿತು. ತಾವು ಬಾಳುವುದಕ್ಕೆ, ಮತ್ತೊಬ್ಬರಿಗೆ ಕಲಿಸುವುದಕ್ಕೆ, ಒಂದು ಮಹೋನ್ನತ ಆದರ್ಶವಿದೆ; ಅದಕ್ಕಾಗಿ ನಾವು ಮತ್ತೊಬ್ಬರ ಮನೆಯ ಬಾಗಿಲಿಗೆ ಹೋಗಬೇಕಾಗಿಲ್ಲ. ಆಧ್ಯಾತ್ಮಿಕ ಜೀವನದಲ್ಲಿ ನಾವು ಭಿಕ್ಷುಕರಾಗಬೇಕಾಗಿಲ್ಲ ಎಂಬ ಆತ್ಮಶ್ರದ್ಧೆಯನ್ನು ತಂದಿತು. ಸ್ವಾಮೀಜಿ ಭರತಖಂಡಕ್ಕೆ ಬಂದು ಕೊಲಂಬೊಯಿಂದ ಹಿಮಾಲಯದವರೆಗೆ ಜ್ವಾಲಾಮಯವಾದ ಭಾಷೆಯಲ್ಲಿ ತಮ್ಮ ಭಾವವನ್ನು ವ್ಯಕ್ತಪಡಿಸಿ ಸುಪ್ತವಾಗಿದ್ದ ರಾಷ್ಟ್ರದ ಕುಂಡಲಿನಿಯನ್ನೇ ಎಬ್ಬಿಸಿದರು. ತಮ್ಮ ಭಾವನೆಗಳನ್ನು ತಮ್ಮ ನಂತರ ಅನುಷ್ಠಾನಕ್ಕೆ ತರಲು ಒಂದು ಮಹಾ ಸಂಸ್ಥೆಯನ್ನು ಸೃಷ್ಟಿಮಾಡಿದರು. ಅದು ಅವರ ಕಣ್ಣು ಮುಂದೆಯೇ ಚೆನ್ನಾಗಿ ಬಹು ಜನರ ಹಿತಕ್ಕೆ, ಬಹು ಜನರ ಸುಖಕ್ಕೆ, ಒಬ್ಬನ ಮೋಕ್ಷಕ್ಕೆ ಮತ್ತು ಲೋಕಕಲ್ಯಾಣಕ್ಕೆ ಕೆಲಸ ಮಾಡುವುದನ್ನು ನೋಡಿದರು. ತಾವು ಯಾವುದಕ್ಕೆ ಬಂದಿದ್ದರೋ ಆ ಕೆಲಸವನ್ನು ತಾವು ಮಾಡಿ ಆಯಿತು ಎಂಬ ಭಾವ ಅವರನ್ನು ಆವರಿಸಿತು. ಯಾವ ಅಖಂಡ ಲೋಕದಿಂದ ಬಂದು ನಾಮರೂಪಗಳ ಬೇಲಿಯ ಒಳಗೆ ಬಿದ್ದರೋ, ಆ ಅಖಂಡದ ಕಡೆಗೆ ಹೋಗಲು ಇವರ ಪ್ರಾಣಪಕ್ಷಿ ಕಾತರವಾಗಿರುವುದನ್ನು ಇನ್ನು ಮೇಲೆ ನೋಡುವೆವು. ತಾವೇ ಕಟ್ಟಿಹಾಕಿಕೊಂಡಿದ್ದ ಬಂಧನಗಳನ್ನು ಬಿಡಿಸಿಕೊಂಡು ಬಂದೆಡೆಗೆ ಹೋಗುವ ಸನ್ನಾಹದಲ್ಲಿ ನಿರತರಾಗಿರುವುದನ್ನು ನೋಡುವೆವು. ಅಂತಹ ಮಹಾಪುರುಷರು ಪ್ರಪಂಚಕ್ಕೆ ಬರುವುದರಲ್ಲಿ ಒಂದು ಆಕರ್ಷಣೆ ಇದೆ. ಇವರು ಬಾಳುವುದರಲ್ಲಿ ಒಂದು ಆಕರ್ಷಣೆಯಿದೆ, ಈ ಪ್ರಪಂಚವನ್ನು ಬಿಟ್ಟು ಹೋಗುವುದರಲ್ಲಿಯೂ ಒಂದು ಆಕರ್ಷಣೆ ಇದೆ. ಬದ್ಧ ಜೀವರು ಕೊನೆಗಾಲದಲ್ಲಿ ಅಯ್ಯೋ ನೆಂಟರಿಷ್ಟರನ್ನು ಬಿಟ್ಟುಹೋಗಬೇಕಲ್ಲ ಎಂದು ಅಳುತ್ತಾರೆ. ಆದರೆ ಮುಕ್ತರಾದಂತಹ ಸ್ವಾಮೀಜಿ ತಮ್ಮ ಪಾಲಿನ ಕರ್ತವ್ಯವನ್ನು ಮಾಡಿ, ದೇಹ ಬಂಧನದಿಂದ ಪಾರಾದರೆ ಸಾಕು ಎಂದು ಮೃತ್ಯುವನ್ನು ಬಿಡುಗಡೆಯ ಸುದಿನದಂತೆ ಎದುರುಗೊಳ್ಳುವರು. ಒಂದು ಸಲ ಕಾಶ್ಮೀರದಲ್ಲಿದ್ದಾಗ ಸ್ವಾಮೀಜಿ ಹೀಗೆ ಹೇಳಿದ್ದರು: “ಮೃತ್ಯು ನನ್ನನ್ನು ಸಮೀಪಿಸಿದೆ ಎಂದು ಅರಿತಾಗ ನನ್ನ ದೌರ್ಬಲ್ಯವೆಲ್ಲ ಮಾಯವಾಗುವುದು. ನಾನು ಬಾಹ್ಯ ಪ್ರಪಂಚಕ್ಕೆ ಅಂಜುವವನಲ್ಲ. ಅದರ ಮೇಲೆ ನನಗೆ ಆಸಕ್ತಿಯೂ ಇಲ್ಲ, ಅನುಮಾನವೂ ಇಲ್ಲ. ಸುಮ್ಮನೆ ಸಾಯಲು ನಾನು ಅಣಿಯಾಗುವೆ. ನಾನು ಇದರಷ್ಟು ಗಟ್ಟಿಯಾಗಿರುವೆ” ಎಂದು ತಮ್ಮ ಕೈಗಳಲ್ಲಿದ್ದ ಕಲ್ಲನ್ನು ಒಂದಕ್ಕೊಂದು ತಾಕಿಸಿದರು. “ಏಕೆಂದರೆ ನಾನು ಭಗವಂತನ ಪಾದಪದ್ಮಗಳನ್ನು ಮುಟ್ಟಿರುವೆನು.” 

 ಸ್ವಾಮೀಜಿಗೆ ಈಗ ಮೂವತ್ತೊಂಭತ್ತು ವರ್ಷಗಳು ಮಾತ್ರ. ಸಾಧಾರಣ ಮನುಷ್ಯನಿಗೆ ಈ ವಯಸ್ಸಿನಲ್ಲಿ ಬುದ್ಧಿಯೂ ಬಂದಿರುವುದಿಲ್ಲ. ಆದರೆ ಸ್ವಾಮೀಜಿ ಆ ಸಣ್ಣ ವಯಸ್ಸಿನಲ್ಲಿ ಏನೇನು ಸಾಧಿಸಿದರು ಅದನ್ನು ಮನನ ಮಾಡಿದರೇ ಸಾಕು ನಮಗೆ ರೋಮಾಂಚನವಾಗುತ್ತದೆ. ಇಷ್ಟೊಂದು ವಿರಾಡ್ರೂಪದ ಕ್ರಿಯೆಗಳನ್ನು ಪಾಂಚಭೌತಿಕ ದೇಹದ ಮೂಲಕ ಮಾಡಿದರು. ಆ ದೇಹ ವಿರಾಟ್‌ಚೈತನ್ಯದ ಮಹಾಲೀಲೆಗೆ ಸಿಕ್ಕಿ ಜರ್ಝರಿತವಾಗಿತ್ತು. ಅವರ ಆತ್ಮ ದೇಹದಿಂದ ಬೇರೆಯಾಗಲು ಹವಣಿಸುತ್ತ ಇತ್ತು. 

 ಸ್ವಾಮೀಜಿ ಕಾಶಿಯಿಂದ ಮಠಕ್ಕೆ ಬಂದ ಮೇಲೆ ತಮ್ಮ ಗುರುಭಾಯಿಗಳಿಗೆಲ್ಲ ತಮ್ಮನ್ನು ಸಾಧ್ಯವಾದರೆ ಬಂದು ನೋಡಿಕೊಂಡು ಹೋಗಿ ಎಂದು ಕಾಗದ ಬರೆದರು. ಅನೇಕರು ಹತ್ತಿರ ಮತ್ತು ದೂರದಿಂದ ಬಂದು ಸ್ವಾಮೀಜಿಯವರನ್ನು ನೋಡಿಕೊಂಡು ಹೋದರು. ಜೂನ್ ತಿಂಗಳು ಮುಗಿಯುವ ಹೊತ್ತಿಗೆ ಇವರ ಕೊನೆಗಾಲವು ಸನ್ನಿಹಿತವಾಯಿತೆಂದು ಅವರು ಅರಿತರು. ಮಹಾತಪಸ್ಸು ಮತ್ತು ಧ್ಯಾನ ಪ್ರವೃತ್ತಿ ಇವರನ್ನು ವಶಮಾಡಿಕೊಂಡಿತು. ಅಧಿಕಾರವನ್ನು ನಿರ್ಣಯಿಸುವ ಸ್ವಾತಂತ್ರ್ಯವನ್ನು ಇತರರಿಗೆ ಬಿಟ್ಟು ತಾವು ನಿರ್ಲಿಪ್ತರಾಗತೊಡಗಿದರು. “ಗುರು ಯಾವಾಗಲೂ ಶಿಷ್ಯನ ಹತ್ತಿರವೇ ಇದ್ದು ಎಷ್ಟೋ ವೇಳೆ ಶಿಷ್ಯನ ಅಧಃಪತನಕ್ಕೆ ಕಾರಣವಾಗುತ್ತಾನೆ. ಒಂದು ಸಲ ಗುರು ಶಿಷ್ಯನನ್ನು ತರಬೇತು ಮಾಡಿದ ಮೇಲೆ, ಗುರು ಅವನನ್ನು ಬಿಟ್ಟು ಹೋಗುವುದು ಅವಶ್ಯಕ. ಶಿಷ್ಯರು ಬೆಳೆಯಬೇಕಾದರೆ ಗುರುವಿನಿಂದ ದೂರ ಇರಬೇಕು” ಎನ್ನುತ್ತಿದ್ದರು. ಶ‍್ರೀರಾಮಕೃಷ್ಣರು, “ನರೇಂದ್ರ ತನ್ನ ಕೆಲಸವನ್ನು ಮಾಡಿದಮೇಲೆ ತಾನಾರು, ಏತಕ್ಕೆ ಬಂದಿರುವೆನು ಎಂಬುದು ಗೊತ್ತಾಗುತ್ತದೆ. ಹೀಗೆ ಗೊತ್ತಾದ ಮೇಲೆ ಅವನು ಹೆಚ್ಚು ಕಾಲ ದೇಹದಲ್ಲಿ ಇರುವುದಿಲ್ಲ” ಎಂದು ತಮ್ಮ ಶಿಷ್ಯರಿಗೆ ಹೇಳಿದ್ದರು. ಸ್ವಾಮೀಜಿಯವರ ಗುರುಭಾಯಿಗಳು “ಸ್ವಾಮೀಜಿ, ನಿಮಗೆ ಈಗ ನೀವಾರೆಂದು ಅರಿವಾಗಿದೆಯೆ?” ಎಂದರು. ಸ್ವಾಮೀಜಿ “ಹೌದು, ನನಗೀಗ ಗೊತ್ತಿದೆ” ಎಂದರು. ಕೇಳಿದ ಗುರುಭಾಯಿಗಳು ಸ್ತಂಭೀಭೂತರಾಗಿ ಹೋದರು. ಸ್ವಾಮೀಜಿ ಬೇಗ ದೇಹವನ್ನು ಕಿತ್ತೊಗೆಯಲು ಹವಣಿಸುತ್ತಿರುವರು ಎಂದು ಭಾವಿಸಿದರು. 

 ಸ್ವಾಮೀಜಿ ಮಹಾ ನಿರ್ವಾಣಕ್ಕೆ ಒಂದು ವಾರ ಮುಂಚಿತವಾಗಿ ತಮ್ಮ ಶಿಷ್ಯ ಶುದ್ಧಾನಂದರಿಗೆ ಬಂಗಾಳಿಯ ಪಂಚಾಂಗವನ್ನು ತರುವಂತೆ ಹೇಳಿದರು. ಅದರಲ್ಲಿ ತಿಥಿ ವಾರ ನಕ್ಷತ್ರಗಳನ್ನು ತಿರುವಿ ಹಾಕುತ್ತಿದ್ದರು. ಅನಂತರವೂ ಕೆಲವು ವೇಳೆ ಅದನ್ನು ನೋಡುತ್ತಿದ್ದರು. ಆಗ ಇತರರಿಗೆ ಸ್ವಾಮೀಜಿ ಏತಕ್ಕೆ ಹೀಗೆ ಮಾಡುತ್ತಾರೆ ಎಂದು ಊಹಿಸಲೂ ಅಸಾಧ್ಯವಾಗಿತ್ತು. ಅವರು ಮಹಾಸಮಾಧಿ ಹೊಂದಿದ ಮೇಲೆಯೆ ಸ್ವಾಮೀಜಿ ಏತಕ್ಕೆ ಪಂಚಾಂಗವನ್ನು ತರಿಸಿದರು ಎಂಬುದು ಗೊತ್ತಾಯಿತು. ಈ ಪ್ರಪಂಚವನ್ನು ಬಿಟ್ಟುಹೋಗುವುದಕ್ಕೆ ಒಂದು ದಿನವನ್ನು ನೋಡುತ್ತಿದ್ದರು! ಅದೇ ಜುಲೈ ನಾಲ್ಕನೇ ತಾರೀಖು ೧೯೦೨. 

 ಸ್ವಾಮೀಜಿ ಮಹಾಸಮಾಧಿಯ ಮೂರು ದಿನಕ್ಕೆ ಮುಂಚೆ ಪ್ರೇಮಾನಂದರೊಡನೆ ಆಶ್ರಮದ ಬಯಲಿನಲ್ಲಿ ಸಂಚರಿಸುತ್ತಿದ್ದಾಗ ಗಂಗಾನದಿಯ ತೀರದ ಒಂದು ಸ್ಥಳವನ್ನು ತೋರಿ, “ನಾನು ದೇಹವನ್ನು ವಿಸರ್ಜಿಸಿದ ಮೇಲೆ ಇದಕ್ಕೆ ಅಲ್ಲಿ ಅಂತ್ಯಕ್ರಿಯೆ ಮಾಡಿ” ಎಂದು ಹೇಳಿದರು. ಇಂದು ಅಲ್ಲಿಯೇ ಅವರ ಸ್ಮಾರಕ ದೇವಸ್ಥಾನ ಇರುವುದು. 

 ಬುಧವಾರ ಏಕಾದಶಿ ದಿನ. ಪರಿನಿರ‍್ಯಾಣಕ್ಕೆ ಮಹಾಸಮಾಧಿಗೆ ಎರಡು ದಿನಗಳು ಉಳಿದಿವೆ. ಸ್ವಾಮೀಜಿ ಉಪವಾಸ ಮಾಡುತ್ತಿದ್ದರು. ಆದರೂ ಶಿಷ್ಯಳಿಗೆ (ಸೋದರಿ ನಿವೇದಿತೆಗೆ) ತಾವೇ ಬಡಿಸಬೇಕೆಂದಿದ್ದರು. ಅವಳಿಗೆ ಬೇಯಿಸಿದ ಹಲಸಿನ ಬೀಜ, ಆಲೂಗಡ್ಡೆ, ಅನ್ನ, ತಂಪಾದ ಹಾಲನ್ನು ಕೊಟ್ಟರು. ಊಟದ ಕಾಲದಲ್ಲಿ ಸಂತೋಷವಾಗಿ ಮಾತನಾಡುತ್ತಿದ್ದರು. ಸ್ವಾಮೀಜಿಯೇ ಕೈ ತೊಳೆಯಲು ನೀರುಕೊಟ್ಟು ಟವಲಿನಿಂದ ತಾವೇ ಶಿಷ್ಯಳ ಕೈಯನ್ನು ಒರಸಿದರು. ಆಗ ಶಿಷ್ಯಳು ಸ್ವಾಭಾವಿಕವಾಗಿ, “ಸ್ವಾಮೀಜಿ, ನಾನು ಇದನ್ನು ನಿಮಗೆ ಮಾಡಬೇಕು, ನೀವು ನನಗೆ ಮಾಡಬಾರದು” ಎಂದಳು. ಅದಕ್ಕೆ ಸ್ವಾಮೀಜಿ ಗಾಂಭೀರ‍್ಯದಿಂದ ಇತ್ತ ಉತ್ತರ ಅನಿರೀಕ್ಷಿತವಾಗಿತ್ತು. “ಜೀಸಸ್ ತನ್ನ ಶಿಷ್ಯರ ಕಾಲನ್ನು ತೊಳೆದನು” ಎಂದರು. 

 ಮುಂದೆ ಮಾತಾಡಲು ಶಿಷ್ಯಳಿಗೆ ಬಾಯಿ ಬರಲಿಲ್ಲ. ಅವನು ಹಾಗೆ ಮಾಡಿದ್ದು ಕೊನೆಗಾಲದಲ್ಲಿ ಎಂದು ಹೇಳಬೇಕೆಂದು ಇದ್ದವಳು ಅದನ್ನು ಉಚ್ಚರಿಸಲಾರದೆ ಹೋದಳು. ಇಲ್ಲಿಯೂ ಕೂಡ ಕೊನೆಯ ಸಮಯವೇ ಬಂದಿತು. 

\newpage

 ಕೊನೆಯ ದಿನ ಬಳಿ ಸಾರಿತು. ಸ್ವಾಮೀಜಿ ಮೊಗದಲ್ಲಿ ಯಾವ ವ್ಯಸನವಾಗಲೀ ಉದ್ವೇಗವಾಗಲಿ ಇರಲಿಲ್ಲ. ಅವರಿಗೆ ಎಲ್ಲಿ ಆಯಾಸವಾಗುವುದೋ ಎಂದು ಎಲ್ಲರೂ ಜೋಪಾನವಾಗಿ ಇವರನ್ನು ನೋಡಿಕೊಳ್ಳುತ್ತಿದ್ದರು. ಬೇಕೆಂತಲೇ ಲಘುವಾದ ವಿಷಯಗಳನ್ನು ಸ್ವಾಮೀಜಿ ಹತ್ತಿರ ಮಾತನಾಡುತ್ತಿದ್ದರು. ಸುತ್ತಲಿರುವ ಮೃಗಗಳು, ತೋಟ ಇವರು ಮಾಡುತ್ತಿದ್ದ ಪ್ರಯೋಗ, ಓದುತ್ತಿದ್ದ ಪುಸ್ತಕ, ದೂರದ ಸ್ನೇಹಿತರು ಈ ವಿಷಯವಾಗಿಯೇ ಮಾತನಾಡಿಸುತ್ತಿದ್ದರು. ಆದರೆ ಯಾವುದೋ ಒಂದು ಅನಿರ್ವಚನೀಯ ಜ್ಯೋತಿ ಇವರನ್ನು ಆವರಿಸಿದಂತೆ ತೋರಿತು. ಆ ಜ್ಯೋತಿಯ ನೆರಳಿನಲ್ಲಿತ್ತು ಇವರ ದೇಹ, ಅದಕ್ಕೆ ಚಿಹ್ನೆಯಂತೆ ಇತ್ತು. ಈಗಿನಷ್ಟು ಹಿಂದೆ ಎಂದಿಗೂ ಇವರ ಮುಂದೆ ನಿಂತಾಗ ಒಂದು ಅನಂತ ಜ್ಯೋತಿಯ ಎದುರಿಗೆ ಇರುವರು ಎಂದು ಭಾಸವಾಗುತ್ತಿರಲಿಲ್ಲ. ಆದರೆ ಯಾರೂ ಅವರಿಷ್ಟು ಬೇಗ ಮಹಾಸಮಾಧಿಯನ್ನು ಪಡೆಯುವರೆಂದು ಎಣಿಸಿಯೇ ಇರಲಿಲ್ಲ. ಅದರಲ್ಲಿಯೂ ೧೯೦೨ನೇ ಜುಲೈ ನಾಲ್ಕನೇ ತೇದಿ ಎಂದಿಗಿಂತ ಅವರು ಬಲವಾಗಿರುವಂತೆ ಕಂಡರು, ಆರೋಗ್ಯವಾಗಿರುವಂತೆ ಕಂಡರು. 

 ಸ್ವಾಮೀಜಿ ಜುಲೈ ನಾಲ್ಕನೇ ತಾರೀಖು ಶುಕ್ರವಾರ ಬೆಳಿಗ್ಗೆ ಎಂಟು ಗಂಟೆಯಿಂದ ಹನ್ನೊಂದು ಗಂಟೆಯವರೆಗೆ ಗಾಢ ಧ್ಯಾನದಲ್ಲಿದ್ದರು. ಧ್ಯಾನವಾದ ಮೇಲೆ ಕಾಳಿಕಾ ಮಾತೆಯ ಮೇಲೆ ಅಪೂರ್ವ ಭಕ್ತಿಯಿಂದ ಒಂದು ಹಾಡನ್ನು ಹಾಡಿದರು. ಶ್ರೇಷ್ಠ ಧ್ಯಾನ ಮತ್ತು ಭಕ್ತಿಭಾವಗಳಿಂದ ಕೂಡಿತ್ತು ಅವರು ಹಾಡಿದ ಹಾಡು. ಅಲ್ಲಿಂದ ಕೆಳಕ್ಕೆ ಇಳಿಯುವಾಗ ತಮಗೆ ತಾವೇ ಮತ್ತೊಬ್ಬರಿಗೆ ಸ್ವಲ್ಪ ಕೇಳುವಂತೆ ಹೀಗೆ ಹೇಳಿಕೊಂಡರು: “ಮತ್ತೊಬ್ಬ ವಿವೇಕಾನಂದ ಇದ್ದಿದ್ದರೆ ಈ ವಿವೇಕಾನಂದ ಏನು ಮಾಡಿದ ಎಂಬುದು ಗೊತ್ತಾಗುತ್ತಿತ್ತು. ಆದರೂ ಕಾಲಗರ್ಭದಲ್ಲಿ ಇಂತಹ ವಿವೇಕಾನಂದರು ಎಷ್ಟು ಹುದುಗಿರುವರೊ!” 

 ಸ್ವಾಮೀಜಿ ತಾವು ಯಾವಾಗಲೂ ಬೇರೆ ಊಟ ಮಾಡುತ್ತಿದ್ದವರು ಇಂದು ಎಲ್ಲರೊಡನೆ ಸಹಪಂಕ್ತಿಯಲ್ಲಿ ಭೋಜನ ಮಾಡಿದರು. ಎಂದಿಗಿಂತ ಚೆನ್ನಾಗಿ ಆಹಾರ ರುಚಿಸಿತು. ಶುದ್ಧಾನಂದರಿಗೆ ಶುಕ್ಲ ಯಜುರ್ವೇದವನ್ನು ತರುವಂತೆ ಹೇಳಿದರು. ಅವರು ತಂದಮೇಲೆ ‘ಸುಷುಮ್ನಃ ಸೂರ್ಯರಶ್ಮಿಶ್ಚ’ ಎಂಬ ಮಂತ್ರವನ್ನು ಓದುವಂತೆ ಹೇಳಿದರು. ಅದರ ಭಾಷ್ಯ ಸ್ವಾಮೀಜಿಯವರಿಗೆ ಅಷ್ಟು ತೃಪ್ತಿಕರವಾಗಿ ಕಾಣಲಿಲ್ಲ. ತಮ್ಮ ಶಿಷ್ಯರು ಧ್ಯಾನ ಮತ್ತು ಮನನದಿಂದ ಅದಕ್ಕೆ ಸರಿಯಾದ ಅರ್ಥವನ್ನು ಕಂಡುಹಿಡಿಯಬೇಕೆಂದು ಹೇಳಿದರು. 

 ಅಂದಿನ ಮಧ್ಯಾಹ್ನ ಬ್ರಹ್ಮಚಾರಿಗಳಿಗೆ ಲಘುಕೌಮುದಿಯ ಮೇಲೆ ಸುಮಾರು ಮೂರು ಗಂಟೆಗಳ ಕಾಲ ಪಾಠವನ್ನು ತೆಗೆದುಕೊಂಡರು. ಸಾಯಂಕಾಲ ಪ್ರೇಮಾನಂದರೊಡನೆ ಹೊರಗೆ ಸಂಚಾರಕ್ಕೆ ಹೋಗಿ ಬಂದರು. ತಮ್ಮ ಕೋಣೆಯ ಒಳಗೆ ಪ್ರವೇಶಿಸುವಾಗ ಶಿಷ್ಯನಿಗೆ ಒಳಗೆ ತಾವು ಕರೆದಲ್ಲದೆ ಯಾರೂ ಬರಕೂಡದೆಂದು ಹೇಳಿ ಗಂಗಾನದಿಯ ಕಡೆ ಮುಖಮಾಡಿ ದೀರ್ಘ ಧ್ಯಾನದಲ್ಲಿ ತನ್ಮಯರಾದರು. ಒಂದು ಗಂಟೆ ಆದಮೇಲೆ ಒಬ್ಬ ಶಿಷ್ಯನನ್ನು ಕರೆದು ತಮ್ಮ ತಲೆಗೆ ಗಾಳಿ ಬೀಸುವಂತೆ ಹೇಳಿದರು. ತಲೆಯನ್ನು ಹಾಸಿಗೆಯ ಮೇಲಿಟ್ಟು ಕಣ್ಣನ್ನು ಮುಚ್ಚಿದರು. ಶಿಷ್ಯ ಸ್ವಾಮೀಜಿ ನಿದ್ರಿಸುತ್ತಿರಬಹುದು ಅಥವಾ ಧ್ಯಾನಿಸುತ್ತಿರಬಹುದು ಎಂದು ಭಾವಿಸಿದನು. ಒಂದು ಗಂಟೆಯಾದ ಮೇಲೆ ಒಂದು ಸಲ ಸ್ವಲ್ಪ ಗಟ್ಟಿಯಾಗಿ ಉಸಿರುಬಿಟ್ಟರು. ಅನಂತರ ಇನ್ನು ಕೆಲವು ನಿಮಿಷಗಳ ಮೇಲೆ ಮತ್ತೊಮ್ಮೆ ಗಟ್ಟಿಯಾಗಿ ಉಸಿರುಬಿಟ್ಟರು. ಕಣ್ಣು ಭ್ರೂಮಧ್ಯದಲ್ಲಿ ನೆಲಸಿತು. ಒಂದು ಅವರ್ಣನೀಯ ಕಾಂತಿ ಅವರ ಮುಖವನ್ನು ಆವರಿಸಿತು. ಅವರ ಆತ್ಮ ಧ್ಯಾನ ವಿಹಂಗಮದ ಹೆಗಲನ್ನೇರಿ ಪುನಃ ಬರಲಾರದ ಲೋಕಕ್ಕೆ ಹಾರಿಯೇ ಹೋಯಿತು. ಭೂಮಿಯ ಮೇಲೆ ಕಳಚಿಬಿಟ್ಟ ಅಂಗಿಯಂತೇ ಇತ್ತು ಅವರ ದೇಹ. ಇದಾದಾಗ ಜುಲೈ ನಾಲ್ಕನೆ ತಾರೀಖು ಶುಕ್ರವಾರ ರಾತ್ರಿ ಒಂಭತ್ತು ಗಂಟೆಯಾಗಿ ಹತ್ತು ನಿಮಿಷಗಳಾಗಿತ್ತು. 

 ಸ್ವಲ್ಪ ಕಾಲದ ಮೇಲೆ ಮಠದಲ್ಲಿ ಎಲ್ಲರಿಗೂ ಸಿಡಿಲಿನಂತೆ ಸುದ್ದಿ ಹರಡಿತು. ವೈದ್ಯರನ್ನು ಕರೆಸಿ ಪರೀಕ್ಷಿಸಿದಾಗ ಸ್ವಾಮೀಜಿ ದೇಹವನ್ನು ತ್ಯಜಿಸಿರುವರು ಎಂದು ಹೇಳಿದರು. ಕಲ್ಕತ್ತೆಯಲ್ಲಿರುವ ಭಕ್ತರಿಗೆಲ್ಲ ಸುದ್ದಿ ಹರಡಿತು. ಮಾರನೆ ದಿನ ಸಹಸ್ರಾರು ಜನ ಸ್ವಾಮೀಜಿ ಅವರ ಕಳೇಬರವನ್ನು ನೋಡಲು ಬಂದರು. ಮಧ್ಯಾಹ್ನದ ಹೊತ್ತಿಗೆ ಅವರು ಮಲಗಿದ್ದ ಮಂಚದ ಮೇಲೆಯೇ ಅವರ ದೇಹವನ್ನು ಇಟ್ಟು ಕೆಳಗೆ ತಂದರು. ಗಂಗಾಸ್ನಾನ ಮಾಡಿಸಿದರು. ಚಂದನಾದಿಗಳನ್ನು ಲೇಪಿಸಿದರು. ಹೊಸ ಕಾವಿಯ ಬಟ್ಟೆಯನ್ನು ಉಡಿಸಿ, ಭಜನೆ ಸಂಕೀರ್ತನೆಗಳೊಡನೆ ಗಂಗಾನದಿಯ ತೀರಕ್ಕೆ, ಸ್ವಾಮೀಜಿ ಕೆಲವು ದಿನಗಳ ಹಿಂದೆ ತೋರಿದ ಸ್ಥಳಕ್ಕೆ ದೇಹವನ್ನು ತಂದರು. ಚಂದನಾದಿ ಸುಗಂಧ ದ್ರವ್ಯಗಳಿಂದ ಅಲಂಕೃತವಾದ ಚಿತೆಯಮೇಲೆ ದೇಹವನ್ನು ಇಟ್ಟರು. ನೆರೆದ ಸಹಸ್ರಾರು ಜನ ತಮ್ಮ ಅಂತ್ಯಕಾಣಿಕೆಯನ್ನು ಅರ್ಪಿಸಿದರು. ‘ಜಯ ಶ‍್ರೀ ಗುರುಮಹಾರಾಜರಿಗೆ!’ ‘ಜಯ ಸ್ವಾಮೀಜಿ ಮಹಾರಾಜರಿಗೆ’ ಎಂಬ ಧ್ವನಿ ದಿಕ್‌ತಟಗಳನ್ನು ಭೇದಿಸುತ್ತಿದ್ದ ಹಾಗೆಯೆ, ಚಿತಾಗ್ನಿಯ ಜ್ವಾಲೆ ಮೇಲೆದ್ದಿತು, ಸ್ವಾಮಿ ವಿವೇಕಾನಂದರು ಉಪಯೋಗಿಸಿದ ದೇಹ ವಸನವನ್ನು ಪಂಚಭೂತಗಳಿಗೆ ವರ್ಗಾಯಿಸಿತು. ಸೂರ್ಯ ಮುಳುಗುವ ಹೊತ್ತಿಗೆ ಚಿತಾಗ್ನಿಯೂ ಮೌನವಾಗಿ ಬೂದಿಯಾಗಿತ್ತು. ಅಖಂಡದಿಂದ ಬಂದ ಜ್ಯೋತಿ ಅಖಂಡಕ್ಕೆ ತೆರಳಿತು. ನೀರು ಗುಳ್ಳೆ ಒಡೆದು ವಿಶ್ವದಲ್ಲಿ ಐಕ್ಯವಾಯಿತು. ಬಿಂದು ಸಿಂಧುವಿನಲ್ಲಿ ಲಯವಾಯಿತು. 

 ಆದರೆ ಸ್ವಾಮಿ ವಿವೇಕಾನಂದರಂತಹ ಮಹಾಪುರುಷರಿಗೆ ಮೃತ್ಯು ಕೊನೆಯಲ್ಲ. ಅವರೇ “ನಾನೊಂದು ಜೀರ್ಣ ವಸನದಂತೆ ದೇಹವನ್ನು ಎಸೆಯಬಹುದು, ಆದರೆ ಜಗತ್ತಿನ ಕಲ್ಯಾಣಮಾಡುವುದನ್ನು ಬಿಡುವುದಿಲ್ಲ. ವಿಶ್ವಕೋಟಿ ತಾವು ದೇವರಲ್ಲಿ ಒಂದು ಎಂದು ಅರಿಯುವವರೆಗೆ ಚಿರಂತರ ಸ್ಫೂರ್ತಿಯಾಗಿ ನಾನು ಕೆಲಸ ಮಾಡುವೆನು” ಎಂದು ಹೇಳುತ್ತಿದ್ದರು. 

 ಸ್ವಾಮೀಜಿಯವರ ಜೀವನ ಮತ್ತು ಸಂದೇಶ ಭರತಖಂಡದಲ್ಲಿ ಮಾತ್ರವಲ್ಲ ವಿಶ್ವದಲ್ಲೆಲ್ಲಾ ಎಲ್ಲಾ ದೇಶಭಾಷೆಗಳ ಮೂಲಕವಾಗಿ ಈಗ ಹಬ್ಬಿ ಹರಡುತ್ತಿದೆ. ಪತನದಿಂದ ಅಭ್ಯುದಯದ ಕಡೆಗೆ ಪ್ರಯಾಣಮಾಡುತ್ತಿರುವ ಭರತಖಂಡದ ರಥವನ್ನು ಎಳೆಯುತ್ತಿರುವವರಿಗೆ ಶಕ್ತಿ ಸ್ಫೂರ್ತಿ ಭರವಸೆಗಳನ್ನು ಒದಗಿಸುತ್ತಿದೆ. 

\delimiter

