
\chapter{ಮರಳಿ ಅಮೇರಿಕಾಕ್ಕೆ  }

 ಸ್ವಾಮೀಜಿ ಅಮೇರಿಕಾದಲ್ಲಿದ್ದಾಗ ಅವರ ಶಿಷ್ಯರಾದ ಅಭಯಾನಂದ ಕೃಪಾನಂದ ಮತ್ತು ಮಿಸ್ ವಾಲ್ಡೊ ಅವರು ವೇದಾಂತ ವಿಷಯಗಳನ್ನು ಜನರಿಗೆ ಹೇಳುತ್ತಿದ್ದರು. ಬಫೆಲೋ ಮತ್ತು ಡೆಟ್ರಾಯಿಟ್ ಗಳಲ್ಲಿ ಅವರು ಎರಡು ಕೇಂದ್ರಗಳನ್ನು ತೆರೆದರು. ಸ್ವಾಮೀಜಿ ಡಿಸೆಂಬರ್ ಆರನೇ ತಾರೀಖಿನ ಹೊತ್ತಿಗೆ ನ್ಯೂಯಾರ್ಕಿಗೆ ಬಂದರು. ಎರಡು ಪ್ರಶಸ್ತವಾದ ಕೋಣೆಗಳನ್ನು ಬಾಡಿಗೆಗೆ ತೆಗೆದುಕೊಂಡು ಅಲ್ಲಿ ಪ್ರವಚನಗಳನ್ನು ಕೊಡಲು ಪ್ರಾರಂಭಿಸಿದರು. ಆ ಸಮಯದಲ್ಲೆ ಸ್ವಾಮೀಜಿ ಕರ್ಮಯೋಗದ ಉಪನ್ಯಾಸಗಳನ್ನು ಕೊಟ್ಟರು. ಅದನ್ನು ಸ್ವಾಮೀಜಿಯವರ ಶ್ರೇಷ್ಠವಾದ ಕೃತಿ ಎಂದು ಭಾವಿಸುವರು. ಹಲವು ಪ್ರಮುಖವಾದ ಉಪನ್ಯಾಸಗಳನ್ನು ಅವರು ಆ ಕಾಲದಲ್ಲಿ ಕೊಟ್ಟರು. ಅವೇ ಧರ್ಮದ ಹಕ್ಕುದಾರಿಕೆ, ಅದರ ಸತ್ಯ ಮತ್ತು ಪ್ರಯೋಜನ, ವಿಶ್ವಧರ್ಮದ ಭಾವನೆ, ಬ್ರಹ್ಮಾಂಡ, ಪಿಂಡಾಂಡ ಮುಂತಾದವು. 

 ಸ್ವಾಮೀಜಿಯವರ ಉಪನ್ಯಾಸಗಳನ್ನು ಶ‍್ರೀಘ್ರಲಿಪಿಯಲ್ಲಿ ತೆಗೆದುಕೊಂಡು ಅದನ್ನು ಮುದ್ರಿಸಬೇಕೆಂದು ಅವರ ಶಿಷ್ಯರು ಆಶಿಸಿ ಒಬ್ಬ ಶೀಘ್ರಲಿಪಿಕಾರನನ್ನು ಗೊತ್ತುಮಾಡಿದರು. ಆದರೆ ಆತನಿಗೆ ಆಧ್ಯಾತ್ಮ ಮತ್ತು ತಾತ್ತ್ವಿಕ ವಿಷಯಗಳನ್ನು ತೆಗೆದುಕೊಂಡು ಅಭ್ಯಾಸವಿರಲಿಲ್ಲ. ಅವನು ಅಷ್ಟು ಯಶಸ್ವಿಯಾಗಿ ಕೆಲಸಮಾಡಲು ಸಾಧ್ಯವಾಗಲಿಲ್ಲ. ಅನಂತರ ಇನ್ನು ಕೆಲವರನ್ನು ಪ್ರಯತ್ನಿಸಿ ನೋಡಿ ಕೊನೆಗೆ ಇಂಗ್ಲೆಂಡಿನಿಂದ ಆಗತಾನೆ ಅಮೇರಿಕಾ ದೇಶಕ್ಕೆ ಬಂದಿದ್ದ ಜೆ. ಜೆ. ಗುಡ್‍ವಿನ್ ಎಂಬುವನನ್ನು ನೇಮಕ ಮಾಡಿದರು. ಆತ ಉಪನ್ಯಾಸವನ್ನೆಲ್ಲಾ ತೆಗೆದುಕೊಂಡು ಮಾರನೆಯ ದಿನ ಅದನ್ನು ಪೇಪರಿಗೆ ಕೊಡುವುದಕ್ಕೆ ದೀರ್ಘಲಿಪಿಯಲ್ಲಿಯೂ ತಯಾರುಮಾಡುತ್ತಿದ್ದನು. ಆತನ ಶೀಘ್ರಲಿಪಿ ತೃಪ್ತಿಕರವಾಗಿತ್ತು. ಆತ ಕೆಲಸಕ್ಕಾಗಿ ಸ್ವಾಮೀಜಿಯವರ ಹತ್ತಿರ ಬಂದರೂ ಸ್ವಾಮೀಜಿಯವರನ್ನು ನೋಡಿ ಮಾತುಕತೆಗಳನ್ನು ಕೇಳಿ, ಅನಂತರ ತಾನು ಸಂಬಳವನ್ನೇ ತೆಗೆದುಕೊಳ್ಳುವುದನ್ನು ಬಿಟ್ಟು ಉಚಿತವಾಗಿ ಕೆಲಸಮಾಡತೊಡಗಿದನು. ಇದು ಮಾತ್ರವಲ್ಲದೆ ಅವರ ಸೇವಾಕಾರ‍್ಯಗಳನ್ನೂ ಮಾಡತೊಡಗಿದರು. ಸ್ವಾಮೀಜಿಯವರ ‘ರಾಜಯೋಗ’ ಗ್ರಂಥ ವಿನಃ ಅಮೇರಿಕಾ ದೇಶದಲ್ಲಿ ಯಾವ ಉಪನ್ಯಾಸಗಳನ್ನೂ ಪೂರ್ವಭಾವಿಯಾಗಿ ಸಿದ್ಧಮಾಡಿಕೊಂಡು ಮಾತನಾಡಲಿಲ್ಲ. ಗುಡ್‍ವಿನ್ ಇಲ್ಲದೇ ಇದ್ದರೆ ಇಂದು ನಮಗೆ ಸಿಕ್ಕಿರುವ ಸ್ವಾಮೀಜಿಯವರ ಅನರ್ಘ್ಯ ಉಪನ್ಯಾಸಗಳೆಲ್ಲ ಬೆಳಕಿಗೆ ಬರುತ್ತಿರಲಿಲ್ಲ. ದೇವರೇ ಉಪದೇಶ ಪ್ರಪಂಚದಲ್ಲಿ ಆಚಂದ್ರಾರ್ಕವಾಗಿರಲಿ ಎಂದು ಈ ವ್ಯಕ್ತಿಯನ್ನು ಕಳುಹಿಸಿದಂತೆ ಇತ್ತು. ಭರತಖಂಡದಲ್ಲಿ ಪ್ರತಿಯೊಂದು ದೇಶಭಾಷೆಯಲ್ಲಿಯೂ ಸ್ವಾಮೀಜಿ ಅವರ ಕೃತಿಶ್ರೇಣಿಯನ್ನು ಪ್ರಕಟಿಸಿರುವೆವು. ಅದರಲ್ಲಿ ಬಹುಪಾಲು ಗುಡ್‍ವಿನ್ ಮುಖೇನ ಬಂದುದು. ಪೌರಪಾಶ್ಚಾತ್ಯರಿಬ್ಬರೂ ಆ ಒಂದು ವ್ಯಕ್ತಿಗೆ ಇದಕ್ಕಾಗಿ ಚಿರಋಣಿಗಳಾಗಿರಬೇಕು. 

 ಡಿಸೆಂಬರ್ ಅಂತ್ಯದಲ್ಲಿ ಸ್ವಾಮೀಜಿ ಶ‍್ರೀಮತಿ ಓಲ್‍ಬುಲ್ ಅವರ ಮನೆಯಲ್ಲಿ ಅತಿಥಿಗಳಾಗಿ ಹೋದರು. ಅಲ್ಲಿಂದ ಬಂದ ಮೇಲೆ ೧೮೯೩ನೇ ಜನವರಿ ತಿಂಗಳಲ್ಲಿ ಪ್ರತಿ ಭಾನುವಾರ ಹಾರ್ಡ್‍ಮೆನ್ ಹಾಲಿನಲ್ಲಿ ಉಚಿತವಾಗಿ ಭಾಷಣವನ್ನು ಕೊಟ್ಟರು. ಬ್ರುಕ್‍ಲಿನ್ ಮೆಟಫಿಸಿಕಲ್ ಸೊಸೈಟಿ, ನ್ಯೂಯಾರ್ಕಿನ ಪೀಪಲ್ಸ್ ಚರ್ಚಿನಲ್ಲಿ ಸ್ವಾಮೀಜಿಯವರ ಉಪನ್ಯಾಸ ಜಯಪ್ರದವಾಗಿ ನಡೆದವು. ಹಲವು ಜನ ಈ ಉಪನ್ಯಾಸಗಳಿಗೆ ಬರುತ್ತಿದ್ದರು. ಈ ಬಹಿರಂಗ ಉಪನ್ಯಾಸಗಳಲ್ಲದೆ ಸ್ವಾಮೀಜಿ ತಾವಿದ್ದ ಕಡೆಯೇ ಬೆಳಿಗ್ಗೆ ಮತ್ತು ಸಾಯಂಕಾಲ ಪ್ರತಿದಿನವೂ ಪ್ರವಚನಗಳನ್ನು ನಡೆಸಿದರು. ಇಲ್ಲಿಯೂ ಸ್ವಾಮೀಜಿ ನಿರೀಕ್ಷಿಸಿದುದಕ್ಕಿಂತ ಹೆಚ್ಚಾಗಿ ಜನರು ಬಂದರು. ಯಾರು ಸ್ವಾಮೀಜಿಯವರ ಬಹಿರಂಗ ಉಪನ್ಯಾಸಕ್ಕೆ ಬರುತ್ತಿದ್ದರೋ ಅವರು ಸ್ವಾಮೀಜಿ ಇದ್ದ ಕಡೆ ಕೊಡುತ್ತಿದ್ದ ಪ್ರವಚನಕ್ಕೂ ಬರುತ್ತಿದ್ದರು. ಹಾರ್ಡ್‍ಮನ್ ಹಾಲಿನ ಬಹಿರಂಗ ಉಪನ್ಯಾಸದ ಸ್ಥಳದಲ್ಲಿ ಜನರಿಗೆ ಸಾಕಷ್ಟು ಸ್ಥಳವಿರಲಿಲ್ಲ. ಅನೇಕರು ಉಪನ್ಯಾಸದ ಕೊನೆಯವರೆಗೂ ನಿಂತುಕೊಂಡೇ ಕೇಳುತ್ತಿದ್ದರು. ಸ್ವಾಮೀಜಿಯವರನ್ನು “ವಿದ್ಯುತ್ ವಾಗ್ಮಿ” ಎಂದು ಜನ ಕರೆಯತೊಡಗಿದರು. ಫೆಬ್ರವರಿ ತಿಂಗಳಿಂದ ಮ್ಯಾಡಿಸನ್ ಸ್ಕ್ವೇರ್ ಗಾರ್ಡನ್‍ನಲ್ಲಿ ಒಂದು ದೊಡ್ಡ ಪ್ರಾಂಗಣವನ್ನು ಬಾಡಿಗೆಗೆ ತೆಗೆದುಕೊಂಡರು. ಅದು ಸಾವಿರದ ಐನೂರು ಜನರು ಕುಳಿತುಕೊಳ್ಳುವಷ್ಟು ವಿಶಾಲವಾಗಿತ್ತು. ಅಲ್ಲಿ ‘ಭಕ್ತಿಯೋಗ’, ‘ನಿಜವಾದ ಮತ್ತು ತೋರಿಕೆಯ ಮಾನವ’, ‘ನನ್ನ ಗುರುದೇವ’, ಮುಂತಾದ ಪ್ರಮುಖ ಉಪನ್ಯಾಸಗಳನ್ನು ಕೊಟ್ಟರು. ಫೆಬ್ರವರಿಯಲ್ಲಿ ಹಾರ್ಡ್‍ಫರ್ಡ್ ಮೆಟಫಿಸಿಕಲ್ ಸೊಸೈಟಿಯ ಆಶ್ರಯದಲ್ಲಿ ಉಪನ್ಯಾಸವನ್ನು ಕೊಡುವಂತೆ ಸ್ವಾಮೀಜಿಯವರನ್ನು ನಿಮಂತ್ರಣ ಮಾಡಿದರು. ಅಲ್ಲಿ ಸ್ವಾಮೀಜಿ ಜೀವ ಮತ್ತು ಈಶ್ವರ ಎಂಬ ವಿಷಯದ ಮೇಲೆ ಮಾತನಾಡಿದರು. ಈ ಉಪನ್ಯಾಸದ ವಿಷಯವಾಗಿ ಹಾರ್ಡ್‍ಫರ್ಡ್ ಡೈಲಿ ಟೈಮ್ಸ್ ಎಂಬ ಪತ್ರಿಕೆ, ಅನೇಕ ಕ್ರೈಸ್ತರು ಮಾತನಾಡುವುದಕ್ಕಿಂತ ಹೆಚ್ಚಾಗಿ ಸ್ವಾಮೀಜಿಯವರ ಉಪನ್ಯಾಸ ಕ್ರಿಸ್ತನ ವಾಣಿಯನ್ನು ಹೋಲುತ್ತಿತ್ತು ಎಂದು ವಿವರಿಸಿರುವುದು. 

 ಫೆಬ್ರವರಿ ತಿಂಗಳಲ್ಲಿ ಬ್ರೂಕ್‍ಲಿನ್ ಎಥಿಕಲ್ ಸೊಸೈಟಿ ಆಶ್ರಯದಲ್ಲಿ ಸ್ವಾಮೀಜಿ ಮಾತನಾಡಿದರು. ಅನೇಕ ಪತ್ರಿಕೆಗಳು ಅವರನ್ನು ಮುಕ್ತಕಂಠದಿಂದ ಹೊಗಳಿದವು. ಈ ಸಮಯದಲ್ಲಿ ಸ್ವಾಮೀಜಿ ಭಕ್ತಿಯೋಗ, ಜ್ಞಾನಯೋಗ, ಸಾಂಖ್ಯ ಮತ್ತು ವೇದಾಂತ ಎಂಬ ವಿಷಯಗಳ ಮೇಲೆ ಪ್ರವಚನಗಳನ್ನು ಕೊಡುತ್ತಿದ್ದರು. ಸ್ವಾಮೀಜಿ ಮ್ಯಾಡಿಸನ್ ಗಾರ್ಡನ್ ಸ್ಕ್ವೇರಿನಲ್ಲಿ ಫೆಬ್ರವರಿ ೨೪ನೇ ತಾರೀಖು ‘ನನ್ನ ಗುರುದೇವ’ ಎಂಬ ಉಪನ್ಯಾಸವನ್ನು ಕೊಟ್ಟು ಅಲ್ಲಿಯದನ್ನು ಪೂರೈಸಿದರು. ‘ನನ್ನ ಗುರುದೇವ’ ಎಂಬುದು ಸ್ವಾಮೀಜಿಯವರ ಒಂದು ಶ್ರೇಷ್ಠವಾದ ಉಪನ್ಯಾಸವಾಗಿದೆ. ಉಪನ್ಯಾಸದ ಕಲೆಯಲ್ಲಿ ಉತ್ತುಂಗ ಶಿಖರಕ್ಕೆ ಇಲ್ಲಿ ಮುಟ್ಟುವರು. ಉಪನ್ಯಾಸದ ಕೊನೆಯಲ್ಲಿ ಸ್ವಾಮೀಜಿ ಹೀಗೆ ಸಾರುವರು: “ಆಧುನಿಕ ಪ್ರಪಂಚಕ್ಕೆ ಶ‍್ರೀರಾಮಕೃಷ್ಣರ ಸಂದೇಶ ಇದು: ಸಿದ್ಧಾಂತಗಳನ್ನು ಲೆಕ್ಕಿಸಬೇಡಿ. ಮೂಢನಂಬಿಕೆಗಳನ್ನು ಲೆಕ್ಕಿಸಬೇಡಿ. ಕೋಮು ಚರ್ಚ್ ದೇವಸ್ಥಾನವನ್ನು ಲೆಕ್ಕಿಸಬೇಡಿ. ಪ್ರತಿಯೊಬ್ಬ ಮಾನವನ ಜೀವನದ ಸಾರವಾದ ಅಧ್ಯಾತ್ಮದೊಂದಿಗೆ ಹೋಲಿಸಿದರೆ ಅದಕ್ಕೆ ಬೆಲೆಯಿಲ್ಲ. ಮನುಷ್ಯನಲ್ಲಿ ಇದು ಹೆಚ್ಚು ವಿಶದವಾಗಿ ಸ್ವಷ್ಟವಾಗಿ ತೋರಿದಷ್ಟು ಸತ್ಕಾರ‍್ಯಸಾಧನೆಗೆ ಮಹಾಶಕ್ತಿಯಾಗುವನು. ಮೊದಲು ಅದನ್ನು ಸಂಪಾದಿಸಿ ಸಾಧಿಸಿ. ಯಾರನ್ನೂ ದೂರಬೇಡಿ. ಎಲ್ಲಾ ಕೋಮಿನಲ್ಲಿಯೂ ಮತದಲ್ಲಿಯೂ ಸತ್ಯಾಂಶಗಳಿವೆ. ನಿಮ್ಮ ಜೀವನದಲ್ಲಿ ಧರ್ಮವೆಂದರೆ ಮಾತಲ್ಲ, ಹೆಸರಲ್ಲ, ಪಂಗಡವಲ್ಲ, ಅದು ಆಧ್ಯಾತ್ಮಿಕ ಸಾಕ್ಷಾತ್ಕಾರ ಎಂಬುದನ್ನು ತೋರಿ. ಯಾರು ಇದನ್ನು ಅನುಭವಿಸುವರೊ ಅವರು ಮಾತ್ರ ಇದನ್ನು ತಿಳಿದುಕೊಳ್ಳಬಲ್ಲರು. ಯಾರಲ್ಲಿ ಅಧ್ಯಾತ್ಮಿಕತೆ ಇದೆಯೊ ಅವರು ಮಾತ್ರ ಮತ್ತೊಬ್ಬರಿಗೆ ಇದನ್ನು ಕೊಡಬಲ್ಲರು, ಮಾನವಕೋಟಿಯ ಮಹಾಗುರುಗಳಾಗಬಲ್ಲರು. ಅವರೇ ಜ್ಞಾನಶಕ್ತಿಯ ಗಣಿ.” 

 “ಎಲ್ಲಾ ಧರ್ಮದ ಅಂತರಾಳದಲ್ಲಿರುವ ಅತಿ ಮುಖ್ಯವಾದ ಐಕ್ಯತೆಯನ್ನು ವಿವರಿಸುವುದು ಮತ್ತು ಅದನ್ನು ಜಗತ್ತಿಗೆ ಸಾರುವುದು ನನ್ನ ಗುರುದೇವನ ಜೀವನ ಧ್ಯೇಯವಾಗಿತ್ತು. ಉಳಿದ ಗುರುಗಳು ತಮ್ಮ ಹೆಸರಿನಲ್ಲಿ ಬೇರೆ ಬೇರೆ ಧರ್ಮಗಳನ್ನು ಬೋಧಿಸಿದರು. ಆದರೆ ಇಪ್ಪತ್ತನೇ ಶತಮಾನದ ಮಹಾಪುರುಷ ಯಾವುದನ್ನೂ ತನ್ನದೆಂದು ಸಮರ್ಥಿಸಲಿಲ್ಲ, ಯಾರ ಧರ್ಮದ ಗೋಜಿಗೂ ಹೋಗಲಿಲ್ಲ. ಏಕೆಂದರೆ ಧರ್ಮಗಳೆಲ್ಲಾ ಪಾರಮಾರ್ಥಿಕ ದೃಷ್ಟಿಯಿಂದ ಒಂದೆ ಸನಾತನತತ್ವದ ಹಲವು ದೃಶ್ಯಗಳು.” 

 ಇದಕ್ಕೆ ಮುಂಚೆ ಕೆಲವರಿಗೆ ಮಂತ್ರೋಪದೇಶವನ್ನು ಕೊಟ್ಟಿದ್ದರು. ಡಾಕ್ಟರ್ ಸ್ಲೇಟರ್ ಎನ್ನುವವರಿಗೆ ಸಂನ್ಯಾಸ ದೀಕ್ಷೆಯನ್ನು ಕೊಟ್ಟು ಅವರಿಗೆ ಯೋಗಾನಂದ ಎಂಬ ಹೆಸರನ್ನು ಕೊಡುವರು. ಇವರೇ ಸ್ವಾಮೀಜಿಯವರ ಮೂರನೆ ಪಾಶ್ಚಾತ್ಯ ಸಂನ್ಯಾಸಿಗಳು. 

 ಸ್ವಾಮೀಜಿಯವರ ನ್ಯೂಯಾರ್ಕ್ ಉಪನ್ಯಾಸ ಬಹಳ ಯಶಸ್ವಿಯಾಗಿ ಸಾಗಿತು. ಅನೇಕ ವಿಜ್ಞಾನಿಗಳು ತತ್ತ್ವಜ್ಞರು ಸ್ವಾಮೀಜಿಯವರ ಉಪನ್ಯಾಸವನ್ನು ಕೇಳಿ ಭರತ ಖಂಡದ ತತ್ತ್ವಕ್ಕೆ ಆಕರ್ಷಿತರಾದರು. ಆ ಸಮಯದಲ್ಲಿ ಸ್ವಾಮೀಜಿಯವರು ಇಂಡಿಯಾ ದೇಶಕ್ಕೆ ಬರೆದ ಒಂದು ಕಾಗದದಲ್ಲಿ ಹೀಗೆ ವಿವರಿಸುವರು: 

 “ನಾನು ಈಗ ಅಮೇರಿಕಾ ನಾಗರಿಕತೆಯ ಕೇಂದ್ರವೆನಿಸಿರುವ ನ್ಯೂಯಾರ್ಕ್‍ನ್ನು ಜಾಗ್ರತ ಮಾಡುವುದರಲ್ಲಿ ಯಶಸ್ವಿಯಾಗಿರುವೆನು. ಹಿಂದೂಗಳ ಭಾವವನ್ನು ಆಂಗ್ಲ ಭಾಷೆಯಲ್ಲಿಡುವುದು, ಅನಂತರ ಒಣತತ್ತ್ವ, ಜಟಿಲವಾದ ಪುರಾಣಗಳು, ವಿಚಿತ್ರವಾಗಿ ಆಶ್ಚರ‍್ಯಕರವೆನಿಸುವ ಮಾನಸಿಕ ಶಾಸ್ತ್ರ ಇವುಗಳಿಂದ ಸುಲಭವೂ ಮತ್ತು ಮೇಧಾವಿಗಳು ಕೂಡ ತಲೆದೂಗುವಂತಹ ಒಂದು ಧರ್ಮಶಾಸ್ತ್ರವನ್ನು ಮಾಡುವುದೆಂದರೆ ಬಹಳ ಕಷ್ಟ. ಯಾರು ಇದಕ್ಕೆ ಪ್ರಯತ್ನಪಟ್ಟಿರುವರೋ ಅವರಿಗೆ ಮಾತ್ರ ಇದು ಅನುಭವ ವೇದ್ಯ. ಅದ್ವೈತ ತತ್ತ್ವ ಅನುಷ್ಠಾನದಲ್ಲಿ ವ್ಯಯವಾಗಬೇಕು. ಬಿಡಿಸುವುದಕ್ಕೆ ಆಗದಷ್ಟು ಗೋಜಾಗಿ ಹೋಗಿರುವ ಪುರಾಣಗಳಿಂದ ಸ್ಪಷ್ಟವಾಗಿ ಕಾಣುವ ನೀತಿ ನಿಯಮಗಳು ಬರಬೇಕು. ಅತಿ ವಿಸ್ಮಯಕರವಾದ ಯೋಗದಿಂದ ವೈಜ್ಞಾನಿಕವೂ ಅನುಷ್ಠಾನ ಯೋಗ್ಯವೂ ಆದ ಮಾನಸಿಕ ಶಾಸ್ತ್ರ ಬರಬೇಕು. ಇವುಗಳೆಲ್ಲವನ್ನೂ ಒಂದು ಸರಳ ಶೈಲಿಯಲ್ಲಿಡಬೇಕು. ಮಕ್ಕಳಿಗೆ ಕೂಡ ಅರ್ಥವಾಗುವಂತೆ ಇರಬೇಕು. ಇದೇ ನನ್ನ ಜೀವನದ ಗುರಿ. ನಾನು ಇದರಲ್ಲಿ ಎಷ್ಟುಮಟ್ಟಿಗೆ ಜಯಶೀಲನಾಗಿರುವೆನೊ ಅದು ದೇವರಿಗೇ ಗೊತ್ತು.” 

 ಸ್ವಾಮೀಜಿ ಇಷ್ಟು ಹೊತ್ತಿಗೆ ರಾಜಯೋಗ, ಭಕ್ತಿಯೋಗದ ಮೇಲೆ ಪ್ರವಚನಗಳನ್ನು ಕೊಟ್ಟಿದ್ದರು. ಜೆ. ಜೆ. ಗುಡ್‍ವಿನ್ ಅವರ ಶ್ರಮದಿಂದ ಅವರು ಇತ್ತ ಭಾನುವಾರದ ಬಹಿರಂಗ ಉಪನ್ಯಾಸಗಳು ಕಿರುಹೊತ್ತಿಗೆಯ ರೂಪದಲ್ಲಿ ಆಗಲೇ ಪ್ರಕಟವಾಗಿದ್ದವು. ಕ್ರಮೇಣ ಅವರ ಬಾಯಿಂದ ಬಂದ ವಾಣಿ ಮಾಯವಾಗದೆ ಪುಸ್ತಕ ರೂಪವನ್ನು ಧರಿಸಿ ಜನರಲ್ಲಿ ವ್ಯಾಪಿಸತೊಡಗಿತು. 

 ಇಲ್ಲಿ ಕೆಲಸವನ್ನು ಪೂರೈಸಿದ ನಂತರ ಸ್ವಾಮೀಜಿ ಅವರು ಡೆಟ್ರಾಯಿಟ್‍ಗೆ ಗುಡ್‍ವಿನ್ ಒಡನೆ ಎರಡುವಾರ ಉಪನ್ಯಾಸಗಳನ್ನು ಕೊಡಲು ಹೋದರು. ಅಲ್ಲಿ ರಿಚಿಲಿ ಹೋಟಲ್ ಎಂಬುದರಲ್ಲಿ ಇಬ್ಬರೂ ಇಳಿದುಕೊಂಡಿದ್ದರು. ಪ್ರವಚನಾದಿಗಳಿಗೆ ಆ ಹೋಟೆಲಿನಲ್ಲಿರುವ ಪ್ರಾಂಗಣವನ್ನು ಉಪಯೋಗಿಸುತ್ತಿದ್ದರು. ಬರುವ ಜನರನ್ನು ಹಿಡಿಸುವಷ್ಟು ಪ್ರಾಂಗಣ ದೊಡ್ಡದಾಗಿರಲಿಲ್ಲ. ಸುತ್ತಮುತ್ತಲೂ ಇದ್ದ ಕೋಣೆಗಳ ಒಳಗೆ ಹೊರಗೆಲ್ಲ ಜನ ಕುಳಿತುಕೊಳ್ಳುತ್ತಿದ್ದರು. ಆ ಸಮಯದಲ್ಲಿ ಸ್ವಾಮೀಜಿ ಭಕ್ತಿಯೋಗದ ಮೇಲೆ ಮಾತನಾಡುತ್ತಿದ್ದರು. ಕೊನೆಯ ದಿನ ಆ ಊರಿನ ಯಹೂದ್ಯರ ಒಂದು ದೇವಸ್ಥಾನದಲ್ಲಿ ಮಾತನಾಡಿದರು. ಅಲ್ಲಿಯ ರಬ್ಬಿಯಾದ ಲೂಯಿಸ್ ಕ್ರಾಸ್ ಮೆನ್ ಎಂಬುವನು ಸ್ವಾಮೀಜಿಯವರ ದೊಡ್ಡ ಮೆಚ್ಚುಗಾರನಾಗಿದ್ದ. ಅಂದು ನೂರಾರು ಜನ ಉಪನ್ಯಾಸಕ್ಕೆ ಸೇರಿದ್ದರು. ಅಲ್ಲಿ ಸ್ವಾಮೀಜಿ “ವಿಶ್ವ ಧರ್ಮದ ಆದರ್ಶ” ಎಂಬುದರ ಮೇಲೆ ಉಪನ್ಯಾಸ ಮಾಡಿದರು. 

 ಹಾರ್‍ವರ್ಡ್ ಯೂನಿವರ್ಸಿಟಿಯ ಮಿಸ್ಟರ್ ಫಾಕ್ಸ್ ಎಂಬುವರು ಸ್ವಾಮೀಜಿಯವರನ್ನು ಅದೇ ವಿಶ್ವವಿದ್ಯಾನಿಲಯದ ತತ್ತ್ವಭಾಗದ ಆಶ್ರಮದಲ್ಲಿ ಮಾತನಾಡಬೇಕೆಂದು ಕೇಳಿಕೊಂಡಿದ್ದರು. ಸ್ವಾಮೀಜಿ ಮಾರ್ಚಿ ೨೫ನೇ ತಾರೀಖು “ವೇದಾಂತದ ತತ್ತ್ವ” ಎಂಬ ವಿಷಯದ ಮೇಲೆ ಅಲ್ಲಿ ನೆರೆದ ವಿದ್ವಜ್ಜನರು ಮೆಚ್ಚಿ ತಲೆದೂಗುವಂತೆ ಮಾತನಾಡಿದರು. ಆ ಉಪನ್ಯಾಸವನ್ನು ಕೇಳಿ ಆದಮೇಲೆ ಸ್ವಾಮೀಜಿಯವರನ್ನು ಈ ವಿಶ್ವವಿದ್ಯಾನಿಲಯದಲ್ಲಿ ಪೌರ ದಾರ್ಶನಿಕರ ಭಾಗಕ್ಕೆ ಮುಖ್ಯಸ್ಥರನ್ನಾಗಿ ಮಾಡಲು ಇಚ್ಛಿಸಿದರು. ಆದರೆ ಸ್ವಾಮೀಜಿ ಅದಕ್ಕೆ ಒಪ್ಪಲಿಲ್ಲ. ಉಪನ್ಯಾಸವಾದ ಮೇಲೆ ಅಲ್ಲಿ ನೆರೆದವರು ಹಲವು ಪ್ರಶ್ನೆಗಳನ್ನು ಕೇಳಿದರು. ಸ್ವಾಮೀಜಿ ಆ ಪ್ರಶ್ನೆಗಳನ್ನು ಲೀಲಾಜಾಲವಾಗಿ ಬಗೆಹರಿಸಿದರು. ಅವರ ಅನುಮಾನಗಳನ್ನು ಪರಿಹರಿಸಿದರು. ಕೊನೆಗೆ “ವೇದಾಂತದ ರೀತಿ ನಾಗರಿಕತೆ ಯಾವ ಪವಿತ್ರತೆ ಮಾನವನಲ್ಲಿ ಅಂತರ್ಗತವಾಗಿದೆಯೋ ಅದನ್ನು ವ್ಯಕ್ತವಾಗುವಂತೆ ಮಾಡಬೇಕು. ಯಾವ ದೇಶದಲ್ಲಿ ಶ್ರೇಷ್ಠತಮ ಭಾವನೆಗಳು ಕಾರ್ಯರೂಪಕ್ಕೆ ಬರುವುವೋ ಆ ದೇಶವೇ ನಾಗರಿಕತೆಯಲ್ಲಿ ಎಲ್ಲಕ್ಕಿಂತ ಮುಂದುವರಿದ ದೇಶ” ಎಂದು ಹೇಳಿದರು. 

 ಸ್ವಾಮೀಜಿ ಇಷ್ಟು ಹೊತ್ತಿಗೆ ನ್ಯೂಯಾರ್ಕಿನ ವೇದಾಂತ ಸೊಸೈಟಿಯನ್ನು ಸ್ಥಾಪಿಸಿದರು. ಯಾವ ಧರ್ಮದವರು ಬೇಕಾದರೂ ತಮ್ಮ ಧರ್ಮವನ್ನು ಬದಲಾಯಿಸದೆ ಇದಕ್ಕೆ ಸದಸ್ಯರಾಗಬಹುದಿತ್ತು. ಅನ್ಯ ಧರ್ಮ ಸಹಿಷ್ಣುತೆ ಮತ್ತು ಎಲ್ಲಾ ಧರ್ಮಗಳೂ ಒಂದೇ ಸತ್ಯದೆಡೆಗೆ ಒಯ್ಯುವ ಪಥಗಳು ಎಂಬುದನ್ನು ಒಪ್ಪಿಕೊಳ್ಳಬೇಕಾಗಿತ್ತು. ವೇದಾಂತ ತತ್ತ್ವ ಎಲ್ಲಾ ಧರ್ಮಗಳಿಗೂ ಅನ್ವಯಿಸುವುದು. ಅದು ಯಾವುದೋ ಒಂದು ಧರ್ಮದ ಮೀಸಲಲ್ಲ. ವೇದಾಂತ ತತ್ತ್ವವನ್ನು ಸರಿಯಾಗಿ ತಿಳಿದುಕೊಂಡರೆ ಕ್ರೈಸ್ತನು ಉತ್ತಮ ಕ್ರೈಸ್ತನಾಗುವನು, ಹಿಂದು ಉತ್ತಮ ಹಿಂದೂ ಆಗುವನು, ಅದರಂತೆ ಉಳಿದ ಧರ್ಮದ ಅನುಯಾಯಿಗಳೆಲ್ಲಾ. ಆಯಾ ಧರ್ಮದವರು ತಮ್ಮ ಧರ್ಮವನ್ನು ಅನುಸರಿಸುವುದು, ಅದನ್ನು ವೇದಾಂತ ತತ್ತ್ವದ ಹಿನ್ನೆಲೆಯಿಂದ ತಿಳಿದುಕೊಳ್ಳುವುದು, ಇತರ ಧರ್ಮಗಳಿಗೆ ಸೌಹಾರ್ದ ಭಾವನೆಯನ್ನು ತೋರುವುದು ಇವೇ ವೇದಾಂತ ಸಂಘದ ಮುಖ್ಯ ಉದ್ದೇಶವಾಗಿದ್ದವು. 

 ಇಷ್ಟು ಹೊತ್ತಿಗೆ ಸ್ವಾಮೀಜಿಯವರ ರಾಜಯೋಗ ಭಕ್ತಿಯೋಗ ಗ್ರಂಥಗಳು ಪ್ರಕಾಶವಾಗಿದ್ದವು. ಹಲವು ಪತ್ರಿಕೆಗಳು ಮತ್ತು ಶ್ರೇಷ್ಠ ವಿದ್ವಾಂಸರು ತಮ್ಮ ಮೆಚ್ಚುಗೆಯನ್ನು ಅದಕ್ಕೆ ತೋರಿದ್ದರು. ಸ್ವಾಮೀಜಿ ನ್ಯೂಯಾರ್ಕ್ ಕೇಂದ್ರದಲ್ಲಿ ಹಲವು ಅಮೇರಿಕಾ ದೇಶದ ಬ್ರಹ್ಮಚಾರಿ ಸನ್ಯಾಸಿಗಳಿಗೆ ವೇದಾಂತ ತತ್ತ್ವವನ್ನು ಹೇಳಿ ಅವರ ಮೂಲಕ ತಮ್ಮ ನಂತರ ವೇದಾಂತ ಭಾವನೆ ಅಮೇರಿಕಾ ದೇಶದವರಿಗೆ ಪರಿಚಯವಾಗಬೇಕೆಂದು ಬಯಸಿದರು. ಮಿಸ್ ವಾಲ್ಡೊ ಎಂಬ ಮಹಿಳೆಗೆ ಹರಿದಾಸಿ ಎಂಬ ಬ್ರಹ್ಮಚಾರಿಣಿಯ ದೀಕ್ಷೆಯನ್ನು ಕೊಟ್ಟು, ಆಕೆಗೆ ರಾಜಯೋಗವನ್ನು ಚೆನ್ನಾಗಿ ಹೇಳಿಕೊಟ್ಟು, ಆಕೆ ಇತರರಿಗೆ ಅದನ್ನು ಕೊಡಲು ಸಮರ್ಥಳಾಗಿರುವಳು ಎಂದು ಹೇಳಿದರು. ಅದರಂತೆ ಕೃಪಾನಂದ ಅಭಯಾನಂದ ಯೋಗಾನಂದ ಎಂಬ ಅವರ ಮೂವರು ಸಂನ್ಯಾಸಿ ಶಿಷ್ಯರನ್ನು ವೇದಾಂತಕ್ಕೆ ಸಂಬಂಧಪಟ್ಟ ವಿಷಯಗಳನ್ನು ಜನರಿಗೆ ಹೇಳಿಕೊಡುವುದಕ್ಕೆ ಅರ್ಹರಾದ ಶಿಕ್ಷಕರನ್ನಾಗಿ ಮಾಡಿದರು. 

 ನ್ಯೂಯಾರ್ಕಿನ ವೇದಾಂತ ಸಂಘಕ್ಕೆ ಫ್ರಾನ್ಸಿಸ್ ಲೆಗೆಟ್ ಎಂಬುವರು ಅಧ್ಯಕ್ಷರಾದರು. ಸ್ವಾಮೀಜಿಯಿಂದ ಉಪದೇಶವನ್ನು ತೆಗೆದುಕೊಂಡ ಶಿಷ್ಯರು ಇದಕ್ಕೆ ಸದಸ್ಯರಾಗಿದ್ದರು. ಹಲವು ಜನ ಹೊರಗಿನವರೂ ಇದಕ್ಕೆ ಸೇರಿದ್ದರು. ಅವರಲ್ಲಿ ಪ್ರಮುಖರಾದವರೆ, ನ್ಯೂಯಾರ್ಕ್ ನಗರದ ಸಾಮಾಜಿಕ ಮತ್ತು ಬೌದ್ಧಿಕ ಕಾರ್ಯಕ್ಷೇತ್ರದಲ್ಲಿ ಪ್ರಖ್ಯಾತರಾದ ಮಿಸ್ ಮೇರಿ ಫಿಲಿಪ್ಸ್, ಶ‍್ರೀಮತಿ ಸ್ಮಿತ್, ಶ‍್ರೀಮಾನ್ ಮತ್ತು ಶ‍್ರೀಮತಿ ವಾಲ್ಟರ್ ಗುಡ್‍ಇಯರ್, ಪ್ರಖ್ಯಾತ ಗಾಯಿಕೆಯಾದ ಯೆಮ್ಮ ಥರ‍್ಸ ಬಿ ಎಂಬುವರು. ಸ್ವಾಮೀಜಿ ಇಲ್ಲದಾಗ ನ್ಯೂಯಾರ್ಕಿನಲ್ಲಿ ಪ್ರವಚನಾದಿಗಳನ್ನು ಮಾಡಲು ಇನ್ನೊಬ್ಬರು ಸ್ವಾಮಿಗಳನ್ನು ಭಾರತದಿಂದ ಕರೆಸಬೇಕೆಂದು ಶಿಷ್ಯರು ಕೇಳಿಕೊಂಡರು. ಅದರಂತೆಯೇ ಸ್ವಾಮೀಜಿ ತಮ್ಮ ಗುರುಭಾಯಿಗಳೊಬ್ಬರಿಗೆ ಭರತಖಂಡದಿಂದ ಇಂಗ್ಲೆಂಡಿಗೆ ಬರುವಂತೆ ಕಾಗದ ಬರೆದು, ತಾವು ಕೂಡ ಏಪ್ರಿಲ್ ೧೫ನೇ ತಾರೀಖು ಇಂಗ್ಲೆಂಡಿಗೆ ಹೊರಟರು. 

