
\chapter{ಕನ್ಯಾಕುಮಾರಿಗೆ}

 ಸ್ವಾಮೀಜಿ ಅವರು ಕೊಚ್ಚಿನ್ ನೋಡಿಕೊಂಡು ೧೮೯೨ ನೇ ಡಿಸೆಂಬರ್ ಅಂತ್ಯದ ಸಮಯದಲ್ಲಿ ತಿರುವನಂತಪುರಕ್ಕೆ ಬಂದರು. ಅಲ್ಲಿಯ ರಾಜಕುಮಾರರಾದ ಮಾರ್ತಾಂಡವರ್ಮರಿಗೆ ಹಾಗೂ ಉಪಾಧ್ಯಾಯರಾದ ಪ್ರೊಫೆಸರ್ ಕೆ. ಸುಂದರರಾಮ ಅಯ್ಯರ್ ಎಂಬುವರಿಗೆ ಒಂದು ಪತ್ರವನ್ನು ತಂದಿದ್ದರು. ಮನೆಯ ಮುಂದೆ ಅವರ ಹುಡುಗ ನಿಂತಿದ್ದನು. ಸ್ವಾಮೀಜಿ ಆ ಹುಡುಗನಿಗೆ ಇಂಗ್ಲೀಷಿನಲ್ಲಿ “ಪ್ರೊಫೆಸರ್ ಸುಂದರರಾಮ ಅಯ್ಯರ್ ಇರುವರೆ? ಅವರಿಗೆ ನಾನು ಒಂದು ಪತ್ರವನ್ನು ತಂದಿದ್ದೇನೆ” ಎಂದರು. ಆ ಹುಡುಗ ಸ್ವಾಮಿಗಳ ವೇಷಭೂಷಣಗಳನ್ನು ನೋಡಿದನು, ತಲೆಗೆ ಒಂದು ಕಾವಿಯ ರುಮಾಲು ಇತ್ತು, ಕಾವಿಯ ಕೋಟು ಮತ್ತು ಪಂಚೆ, ಕೈಯಲ್ಲಿ ಒಂದು ಕೋಲು, ಅವರ ಜೊತೆಯಲ್ಲಿ ಒಬ್ಬ ಆಳು ಇದ್ದನು. ಆ ಹುಡುಗ ಇದನ್ನೆಲ್ಲ ನೋಡಿ, ಇವರು ಯಾರೋ ಒಬ್ಬ ಮಹಾರಾಜರು ಇರಬೇಕೆಂದು ಭಾವಿಸಿ ಒಳಗೆ ಓಡಿಹೋಗಿ “ಒಬ್ಬ ಮಹಾರಾಜರು ಬಂದಿದ್ದಾರೆ, ಅವರು ನಿನಗೆ ಒಂದು ಪತ್ರವನ್ನು ತಂದಿದ್ದಾರೆ” ಎಂದು ಹೇಳಿದನು. ತಂದೆ ಮಗನಿಗೆ “ನಮ್ಮಂಥವರ ಮನೆಯ ಬಾಗಿಲಿಗೆ ಮಹಾರಾಜರು ಬರುತ್ತಾರೆಯೆ? ನೀನು ಇನ್ನೂ ಏನೂ ತಿಳಿಯದ ಹುಡುಗ” ಎಂದು ಹೊರಗೆ ಹೋಗಿ ನೋಡಿದರು. ಸ್ವಾಮಿಗಳನ್ನು ಮೇಲಕ್ಕೆ ಕರೆದುಕೊಂಡುಹೋದರು. ಸ್ವಲ್ಪಹೊತ್ತು ಅವರೊಡನೆ ಮಾತನಾಡಿದ ಮೇಲೆ ಅವರ ಮಗನಿಗೆ ಹೇಳಿದರು, “ಹೌದು, ನೀನು ಹೇಳಿದ್ದು ನಿಜ, ಅವರು ಮಹಾರಾಜರೆ, ಆದರೆ ಅವರು ರಾಜ್ಯ ಹೊರಗಡೆ ಇರುವುದಲ್ಲ. ಅವರು ಮನುಷ್ಯರ ಹೃದಯವನ್ನಾಳುವ ರಾಜರು, ಮೇರೆಯಿಲ್ಲದ ರಾಜ್ಯ ಅವರದು.” 

 ಪ್ರೊಫೆಸರ್ ಸ್ವಲ್ಪ ಹೊತ್ತು ಮಾತನಾಡಿದ ಮೇಲೆ ಅವರು ಮಹಾರಾಜರ ಮಗನಿಗೆ ಪಾಠ ಹೇಳಿಕೊಡಲು ಅರಮನೆಗೆ ಹೋಗಬೇಕಾಗಿತ್ತು. ಆದರೆ ಅವತ್ತು ರಜ ತೆಗೆದುಕೊಂಡುಬಿಟ್ಟರು. ಅವರು ಸ್ವಾಮೀಜಿಯವರೊಡನೆ ಮಾತನಾಡತೊಡಗಿದರು. ಸ್ವಾಮೀಜಿ ಬಂಗಾಳಿಗಳು ಎಂಬುದನ್ನು ತಿಳಿದ ಪ್ರೊಫೆಸರ್, “ಬಂಗಾಳದೇಶವು ರಾಜರಾಮ ಮೋಹನರಾಯ್, ಕೇಶವಚಂದ್ರಸೇನ ಮುಂತಾದ ಮಹಾವ್ಯಕ್ತಿಗಳಿಗೆ ಜನ್ಮ ನೀಡಿದೆ” ಎಂದರು. ಅದಕ್ಕೆ ಸ್ವಾಮೀಜಿ “ಅವರೆಲ್ಲರಿಗಿಂತ ಮೀರಿದವರೊಬ್ಬರು ಇರುವರು. ಅವರನ್ನು ಪ್ರಪಂಚ ಇನ್ನೂ ಅರಿಯಬೇಕಾಗಿದೆ” ಎಂದು ತಮ್ಮ ಗುರುಗಳಾದ ಶ‍್ರೀರಾಮಕೃಷ್ಣರ ವಿಷಯವನ್ನು ಪ್ರಸ್ತಾಪಿಸಿ, ಕೇಶವಚಂದ್ರಸೇನ ಮುಂತಾದ ಮಹಾವಾಗ್ಮಿಗಳು ಮತ್ತು ವಿದ್ವಾಂಸರು ಮಕ್ಕಳಂತೆ ಶ‍್ರೀರಾಮಕೃಷ್ಣರ ಬಳಿ ಬಂದು ಅವರು ಹೇಳುತ್ತಿದ್ದ ಬೋಧನೆಯನ್ನು ಕೇಳಿ ಹೋಗುತ್ತಿದ್ದರು, ಶ‍್ರೀರಾಮಕೃಷ್ಣ ರೊಂದಿಗೆ ಹೋಲಿಸಿದರೆ ಅವರೆಲ್ಲ ಸೂರ‍್ಯನೆದುರಿಗೆ ಇರುವ ಕುಡಿ ದೀಪಗಳು ಎಂದರು. ಅನಂತರ ತಮ್ಮ ಗುರುಗಳ ಆಧ್ಯಾತ್ಮಿಕ ಜೀವನವನ್ನು ಕುರಿತು ಹೇಳಿದರು. ಸುಂದರರಾಮ ಅಯ್ಯರ್ ಅವರಿಗೆ ಸ್ವಾಮೀಜಿಯವರ ಸಂಗದಿಂದಾದ ಮಹಾನಂದವನ್ನು ಮತ್ತೊಬ್ಬರೊಡನೆ ಹಂಚಿಕೊಳ್ಳಬೇಕೆಂದು ಕುತೂಹಲ. ಸ್ವಾಮೀಜಿಯವರ ಊಟ ಉಪಚಾರಗಳಾದ ಮೇಲೆ ಅಲ್ಲಿಯ ಕಾಲೇಜಿನಲ್ಲಿ ಕೆಮಿಸ್ಟ್ರಿ ಪ್ರೊಫೆಸರ್ ರಂಗಾಚಾರ್ ಅವರ ಪರಿಚಯ ಮಾಡಿಸಲು ಅವರ ಮನೆಗೆ ಕರೆದುಕೊಂಡು ಹೋದರು. ಅವರು ವೃತ್ತಿಯಲ್ಲಿ ಕೆಮಿಸ್ಟ್ರಿ ಪ್ರೊಫೆಸರ್ ಆದರೂ, ದೊಡ್ಡ ಸಂಸ್ಕೃತ ವಿದ್ವಾಂಸರಾಗಿದ್ದರು. ಹಿಂದೂಶಾಸ್ತ್ರದಲ್ಲಿ ಅವರಿಗೆ ವಿಶಾಲವಾದ ಪಾಂಡಿತ್ಯವಿತ್ತು. ಮಹಾವಿದ್ವಾಂಸರು ಮತ್ತು ಸಾತ್ತ್ವಿಕ ವ್ಯಕ್ತಿಗಳೆಂದು ಊರಿನಲ್ಲೆಲ್ಲ ಪ್ರಖ್ಯಾತರಾಗಿದ್ದರು. ಅವರು ಮನೆಯಲ್ಲಿ ಇರಲಿಲ್ಲ. ಅನಂತರ ಊರಿನ ಮುಖ್ಯಸ್ಥರೆಲ್ಲ ಸೇರುವ ಟ್ರಾವಂಕೂರ್ ಕ್ಲಬ್ಬಿಗೆ ಹೋದರು. ಅಲ್ಲಿ ರಂಗಾಚಾರ್ ಅಲ್ಲದೆ ಈ ಊರಿನ ದಿವಾನರು ಮುಂತಾದವರೆಲ್ಲರೂ ನೆರೆದಿದ್ದರು. ಸ್ವಾಮೀಜಿಯವರನ್ನು ಅವರಿಗೆಲ್ಲ ಪರಿಚಯ ಮಾಡಿಸಿದರು. ಸ್ವಾಮೀಜಿ ಅನಂತರ ಅಲ್ಲಿ ಕುಳಿತುಕೊಂಡು ಜನರ ಆಚಾರ ವ್ಯವಹಾರ ಮಾತುಕತೆಗಳನ್ನು ಕೇಳುತ್ತಿದ್ದರು. ಬ್ರಾಹ್ಮಣ ದಿವಾನರಿಗೆ ಅಲ್ಲಿಯ ನಾರಾಯಣ ಮೆನನ್ ಎಂಬುವರು ತಾವು ಹೋಗುತ್ತೇವೆಂದು ನಮಸ್ಕರಿಸಿ ಹೇಳಿದಾಗ, ಅವರು ತಮ್ಮ ಎಡಗೈಯನ್ನು ಸ್ವಲ್ಪ ಮೇಲಕ್ಕೆ ಮಾತ್ರ ಎತ್ತಿದರು. ಇದು ಮೇಲಿನ ಕುಲದವರು ಕೆಳಗಿನವರಿಗೆ ತಮ್ಮ ಒಪ್ಪಿಗೆಯನ್ನು ತೋರುವ ರೀತಿ. ಸ್ವಾಮೀಜಿ ಇದನ್ನು ನೋಡಿದರು. ಇದೊಂದು ವಿಚಿತ್ರ ಅನುಭವವಾಗಿತ್ತು ಅವರಿಗೆ. ಅನಂತರ ದಿವಾನರು ಸ್ವಾಮೀಜಿಗೆ ಕೈಮುಗಿದು, ತಾವು ಮನೆಗೆ ಹೋಗಲು ಅಪ್ಪಣೆ ಕೊಡಿ ಎಂದು ಕೇಳಿದಾಗ ಸ್ವಾಮೀಜಿ ಬರೀ “ನಾರಾಯಣ” ಎಂದರು. ಆ ದಿವಾನರು ತಾವು ಸ್ವಾಮೀಜಿಗೆ ಕೈ ಮುಗಿದರೆ ಇವರು ನನಗೆ ಪ್ರತಿ ಕೈಮುಗಿಯದೆ ಇದ್ದುದು ಒಂದು ಅಗೌರವ ಎಂದು ಭಾವಿಸಿ ಸ್ವಾಮೀಜಿಯವರ ನಡವಳಿಕೆಯ ಬಗ್ಗೆ ಅವರು ಅತೃಪ್ತಿಯನ್ನು ಸೂಚಿಸಿದರು. ಅದಕ್ಕೆ ಸ್ವಾಮೀಜಿ ದಿವಾನರಿಗೆ, “ಸ್ವಲ್ಪ ಹೊತ್ತಿನ ಹಿಂದೆ ಆ ನಾರಾಯಣ ಮೆನನ್ ಅವರು ನಿಮಗೆ ಕೈ ಮುಗಿದರೆ, ನೀವು ಎಡಗೈಯನ್ನು ಮಾತ್ರ ಸ್ವಲ್ಪ ಎತ್ತಿದಿರಿ. ಅದು ನಿಮ್ಮ ರೂಢಿ. ಅದರಂತೆ ಸಂನ್ಯಾಸಿಗಳಿಗೂ ಒಂದು ರೂಢಿಯಿದೆ. ಅವರು ಯಾರಿಗೂ ಕೈ ಮುಗಿಯುವುದಿಲ್ಲ. ಬರೀ ನಾರಾಯಣ ಎಂದು ಹೇಳುತ್ತಾರೆ” ಎಂದರು. ದಿವಾನರು ಮನೆಗೆ ಹೋದಮೇಲೆ ಈ ವಿಷಯವನ್ನು ಮನನಮಾಡಿ ತಮ್ಮ ತಮ್ಮನ ಕೈಯಲ್ಲಿ ಕ್ಷಮಾಪಣೆಯ ಪತ್ರವನ್ನು ಕಳುಹಿಸಿದರು. 

 ಸುಂದರರಾಮ ಅಯ್ಯರ್ ಸ್ವಾಮೀಜಿಯವರನ್ನು ರಾಜಕುಮಾರರ ಬಳಿಗೆ ಕರೆದುಕೊಂಡುಹೋದರು. ಸ್ವಾಮೀಜಿ ಉತ್ತರ ಇಂಡಿಯಾದ ಹಲವು ರಾಜರುಗಳನ್ನು ತಾವು ನೋಡಿರುವುದಾಗಿ ತಿಳಿಸಿದರು. ಅವರಲ್ಲಿ ಗಾಯಕವಾಡದ ಮಹಾರಾಜರು ಮತ್ತು ಖೇತ್ರಿ ಮಹಾರಾಜರು ತಮ್ಮ ಮೆಚ್ಚಿಗೆಗೆ ಪಾತ್ರರಾಗಿರುವರೆಂದು ಹೇಳಿದರು. ತಮ್ಮ ಪ್ರಜೆಗಳ ಹಿತಕ್ಕಾಗಿ ಸುಧಾರಣೆಗಳನ್ನು ಮಾಡುವುದರಲ್ಲಿ ಅವರು ನಿರತರಾಗಿರುವರೆಂದೂ ಹೇಳಿದರು. ಮಾರ್ತಾಂಡವರ್ಮರು ತಾವು ಕೂಡ ದೊಡ್ಡವರಾದ ಮೇಲೆ ಸಿಂಹಾಸನಕ್ಕೆ ಏರಿದಾಗ ಪ್ರಜಾಹಿತಕ್ಕೆ ತಮ್ಮ ಜೀವನವನ್ನೆಲ್ಲ ವಿನಿಯೋಗಿಸುವೆನು ಎಂದು ಹೇಳಿದರು. ಅವರು ಆಗತಾನೆ ಫೋಟೋ ತೆಗೆಯುವುದನ್ನು ಕಲಿಯುತ್ತಿದ್ದರು. ಸ್ವಾಮೀಜಿ ಅವರ ಒಂದು ಫೋಟೋವನ್ನು ಅವರು ತೆಗೆದುಕೊಂಡರು. 

 ಸುಂದರರಾಮ ಅಯ್ಯರ್ ಒಂದಿಗೆ ಸ್ವಾಮೀಜಿ ಮಾತನಾಡುತ್ತಿದ್ದಾಗ ವಿಜ್ಞಾನ ನಮಗೆ ಇಂದು ಆವಶ್ಯಕವಾಗಿ ಬೇಕೆಂದೂ, ಆದರೆ ಅದೇ ಸರ್ವಸ್ವವಲ್ಲ, ಅದಕ್ಕೆ ಮಿತಿಗಳಿವೆ, ತಮ್ಮ ವೇದಾಂತ ಆ ಮಿತಿಯನ್ನು ಹೋಗಲಾಡಿಸುವುದೆಂದೂ ಹೇಳಿದರು. ವಿಜ್ಞಾನ ತನ್ನ ಹೊರಗಿರುವ ಪ್ರಪಂಚವನ್ನು ವಿಭಜನೆ ಮಾಡಿ ನಿಯಮಕ್ಕೆ ತರುವುದು. ಆದರೆ ನಮ್ಮ ಮನಸ್ಸು ಕೂಡ ಪ್ರಕೃತಿಯ ಒಂದು ಭಾಗ. ಪಾಶ್ಚಾತ್ಯ ವಿಜ್ಞಾನ ಇದನ್ನು ಇನ್ನೂ ಚೆನ್ನಾಗಿ ಅಧ್ಯಯನ ಮಾಡಿಲ್ಲ. ನಮ್ಮ ವೇದಾಂತಿಗಳು, ಪತಂಜಲಿ ಮುಂತಾದವರು, ಆ ಮನಸ್ಸು ಎಂದರೇನು, ಅದನ್ನು ನಿಗ್ರಹಿಸುವುದು ಹೇಗೆ, ಅದಕ್ಕೆ ಎಂತಹ ಶಕ್ತಿಗಳಿವೆ, ಅದನ್ನು ಅತಿಕ್ರಮಿಸಿ ಹೋಗುವುದು ಹೇಗೆ ಎಂಬುದನ್ನೆಲ್ಲ ಚೆನ್ನಾಗಿ ವಿವರಿಸಿರುವರು. ಆ ಮಹಾ ಋಷಿಗಳೊಡನೆ ಹೋಲಿಸಿ ನೋಡಿದರೆ, ಈಗಿನ ಐರೋಪ್ಯ ಮಾನಸಿಕ ವಿಜ್ಞಾನಿಗಳು ಇನ್ನೂ ಮಕ್ಕಳು. ಏಕೆಂದರೆ ಮಾನಸಿಕ ರಹಸ್ಯವನ್ನು ಅರಿಯಬೇಕಾದರೆ ಒಬ್ಬನು ಮೊದಲು ಜಿತೇಂದ್ರಿಯನಾಗಿರಬೇಕು, ಮನಸ್ಸು ಪರಿಶುದ್ಧವಾಗಿರಬೇಕು. ಆಗ ಮಾತ್ರ ಅವನು ಶಾಸ್ತ್ರವನ್ನು ಅಧ್ಯಯನ ಮಾಡಲು ಸಾಧ್ಯ. 

 ಸ್ವಾಮೀಜಿ ವರ್ಣಾಶ್ರಮದ ವಿಚಾರ ಮಾತನಾಡಿದರು. ಬ್ರಾಹ್ಮಣ ಹಿಂದೆ ಬೇಕಾದಷ್ಟು ಸಮಾಜಕ್ಕೆ ಒಳ್ಳೆಯದನ್ನು ಮಾಡಿರುವನು ಮತ್ತು ಮುಂದೆಯೂ ಅವನು ಮಾಡುವನು. ಅವನ ದೃಷ್ಟಿ ಎಲ್ಲಿಯವರೆಗೆ ತಾನು ಕಲಿತುದುದನ್ನು ಇತರರಿಗೆ ಕೊಡುವುದರ ಮೇಲಿರುವುದೋ ಅಲ್ಲಿಯವರೆಗೆ ಒಂದು ಸಮಾಜಕ್ಕೆ ಅಂತಹ ವ್ಯಕ್ತಿಗಳು ಅನಿವಾರ‍್ಯ ಎಂದರು. ಆದರೆ ತಾನು ಕಲಿತಿರುವುದು ತನ್ನ ಸುಖ ಮತ್ತು ಭೋಗಕ್ಕೆ ಎಂಬ ದೃಷ್ಟಿ ಬಂದರೆ ಅದು ವಿನಾಶಹೇತು. ಅದರಂತೆಯೇ ಸ್ತ್ರೀಯರ ಸಮಸ್ಯೆಗೆ ಸ್ವಾಮಿಗಳೇ ಒಂದು ಪರಿಹಾರೋಪಾಯವಿತ್ತರು. ಅವರ ಬಾಲ್ಯ ವಿವಾಹ ಹೋಗಬೇಕು. ಪುರುಷರಿಗೆ ಸರಿಸಮವಾಗಿ ಅವರಿಗೆ ವಿದ್ಯೆ ಸಿಕ್ಕಬೇಕು. ನಮ್ಮ ಪುರಾತನ ಋಷಿಗಳು ಮಾಡಿದ ಮಹದಾಲೋಚನೆಗಳನ್ನು ಅರಿಯಬೇಕು. ಹಿಂದಿನದನ್ನು ಅರಿತು, ಈಗಿರುವುದರಲ್ಲಿ ಯಾವುದು ಒಳ್ಳೆಯದೋ ಅದನ್ನೆಲ್ಲ ಹೀರಿಕೊಳ್ಳಬೇಕು. ಸರಿಯಾದ ವಿದ್ಯಾಭ್ಯಾಸ ಅವರಿಗೆ ದೊರೆತರೆ ಅವರು ಇತರರ ಸಹಾಯವಿಲ್ಲದೆ ತಮ್ಮ ಸಮಸ್ಯೆಗಳನ್ನು ತಾವೇ ಬಗೆಹರಿಸಿಕೊಳ್ಳಬಲ್ಲರು. ನಾವು ಅವರಿಗೆ ಯೋಗ್ಯವಾದ ವಿದ್ಯಾಭ್ಯಾಸವನ್ನು ಕೊಡುವುದೇ ನಮ್ಮ ಆದ್ಯ ಕರ್ತವ್ಯವೆಂದು ಹೇಳಿದರು. 

 ಒಂದು ದಿನ ತಿರುವನಂತಪುರ ಸಂಸ್ಥಾನದ ದೇಶಭಾಷೆಯ ಡೈರೆಕ್ಟರ್ ಎಸ್. ರಾಮರಾವ್, ಸ್ವಾಮೀಜಿಯವರನ್ನು ಇಂದ್ರಿಯ ಜಯವನ್ನು ಕುರಿತು ಪ್ರಶ್ನೆ ಮಾಡಿದರು. ಸ್ವಾಮೀಜಿ ಅದನ್ನು ಉದಾಹರಿಸುವ ಸಲುವಾಗಿ ಬಿಲ್ವಮಂಗಳನ ಕಥೆಯನ್ನು ಹೇಳಿದರು. ಬಿಲ್ವಮಂಗಳ ಹಲವು ಕಾಲ ಸಾಧು ಜೀವನ ನಡೆಸಿ ಒಂದು ದಿನ ವರ್ತಕನ ಹೆಂಡತಿಯನ್ನು ನೋಡಿದಾಗ ಲೌಕಿಕ ವಾಸನೆ ಅವನಲ್ಲಿ ಕೆರಳುವುದು. ಆಗ ಆತ ವರ್ತಕನ ಮನೆಗೆ ಹೋಗಿ ಅವನ ಹೆಂಡತಿಯ ತಲೆಯಲ್ಲಿರುವ ಸೂಜಿಯನ್ನು ಕೇಳಿ, ಅದರಿಂದ ತನ್ನ ಎರಡು ಕಣ್ಣುಗಳನ್ನೂ ತಿವಿದುಕೊಂಡು ಕುರುಡನಾಗುವನು. ಯಾವ ಕಣ್ಣು ಪ್ರಲೋಭನೆಯ ಕಡೆ ಎಳೆಯುವುವೋ ಅವು ನಾಶವಾಗಬೇಕು. ಅವು ಪರಮ ವೈರಿ ಎನ್ನುವನು. ಕೊನೆಗೆ ಬೃಂದಾವನಕ್ಕೆ ಹೋಗಿ ಶ‍್ರೀಕೃಷ್ಣನ ಚಿಂತನೆಯಲ್ಲೇ ಕಾಲ ಕಳೆಯುವನು. ಕೆಲವು ವೇಳೆ ಇಂದ್ರಿಯಗಳನ್ನು ಗೆಲ್ಲಬೇಕಾದರೆ ಇಂತಹ ಉಗ್ರ ಕ್ರಮವನ್ನು ಕೈಕೊಳ್ಳಬೇಕಾಗುವುದು ಎಂದರು ಸ್ವಾಮೀಜಿ. 

 ಸ್ವಾಮೀಜಿ ಸುಂದರರಾಮ ಅಯ್ಯರ್ ಅವರ ಮನೆಯಲ್ಲಿ ಮೂರು ನಾಲ್ಕು ದಿನ ಕಳೆದಾದಮೇಲೆ ಒಂದು ದಿನ ಆ ಊರಿನಲ್ಲಿರುವ ಅಸಿಸ್ಟೆಂಟ್ ಅಕೌಂಟೆಂಟ್ ಜನರಲ್ ಮನ್ಮಥನಾಥ ಭಟ್ಟಾಚಾರ್ಯ ಅವರ ಮನೆಯನ್ನು ಕಂಡು ಹಿಡಿದರು. ಅವರು ಸ್ವಾಮೀಜಿಯವರ ಕಾಲೇಜಿನ ದಿನಗಳ ಸ್ನೇಹಿತರಾಗಿದ್ದರು. ಸ್ವಾಮೀಜಿ ಅವರ ಮನೆಗೆ ಪ್ರತಿದಿನವೂ ಬೆಳಗ್ಗೆ ಹೊತ್ತು ಹೋಗಿ ಬರುತ್ತಿದ್ದರು. ಒಂದು ದಿನ ಸುಂದರರಾಮ ಅಯ್ಯರ್ ಅವರು ಸ್ವಾಮೀಜಿ ತಮ್ಮ ಕಾಲವನ್ನೆಲ್ಲಾ ಭಟ್ಟಾಚಾರ‍್ಯರ ಮನೆಯಲ್ಲಿಯೇ ಕಳೆಯುತ್ತಿರುವರು ಎಂದು ಆಕ್ಷೇಪಣೆ ಮಾಡಿದರು. ಅದಕ್ಕೆ ಸ್ವಾಮೀಜಿ, “ಬಂಗಾಳಿಗಳಿಗೆ ಬಂಗಾಳಿಗಳನ್ನು ಕಂಡರೆ ಪ್ರೀತಿ. ಅದೂ ಅಲ್ಲದೆ ಆತ ನನ್ನ ಕಾಲೇಜಿನ ಸ್ನೇಹಿತ ಮತ್ತು ಪ್ರಖ್ಯಾತ ವಿದ್ವಾಂಸರಾದ ಕಲ್ಕತ್ತೆಯ ಸಂಸ್ಕೃತ ಕಾಲೇಜಿನಲ್ಲಿ ಪ್ರಿನ್ಸಿಪಾಲರಾದ ಪಂಡಿತ ಮಹೇಶಚಂದ್ರ ನ್ಯಾಯರತ್ನ ಅವರ ಮಗ” ಎಂದರು. “ಅದೂ ಅಲ್ಲದೆ ದಕ್ಷಿಣ ಇಂಡಿಯಾದಲ್ಲಿ ಬ್ರಾಹ್ಮಣ ಮನೆಗಳಲ್ಲಿ ಊಟ ಮಾಡಿ ಬಾಯಿ ಸಪ್ಪೆಯಾಗಿ ಹೋಗಿದೆ. ಬಂಗಾಳಿಗಳಿಗೆ ಬೇಕಾದ ಮೀನು, ಮಾಂಸ ಸಿಕ್ಕುತ್ತಿರಲಿಲ್ಲ. ಆದಕಾರಣ ನನ್ನ ಸ್ನೇಹಿತನ ಮನೆಗೆ ಹೋಗಿ ಅದನ್ನು ತಿಂದು ಬರುವೆ” ಎಂದರು. ಇದನ್ನು ಕೇಳಿದ ತಕ್ಷಣವೇ ಬ್ರಾಹ್ಮಣರಾದ ಸುಂದರ ರಾಮ ಅಯ್ಯರ್ “ಸಂನ್ಯಾಸಿಗಳು ಇವುಗಳನ್ನು ತಿನ್ನಬಹುದೆ?” ಎಂದು ಪ್ರಶ್ನಿಸಿದರು. ಅದಕ್ಕೆ ಸ್ವಾಮಿಗಳು, ಹಿಂದಿನ ಕಾಲದಲ್ಲಿ ಬ್ರಾಹ್ಮಣರು ಇವುಗಳನ್ನು ತಿನ್ನುತ್ತಿದ್ದರೆಂದೂ, ದನದ ಮಾಂಸವನ್ನು ಕೂಡ ತಿನ್ನುತ್ತಿದ್ದರೆಂದೂ ಹೇಳಿದರು. ಅತಿಥಿಗಳಿಗೆ ಕೊಡುವ ಮಧುಪರ್ಕದಲ್ಲಿ ದನದ ಮಾಂಸ ಜೇನುತುಪ್ಪ ಮುಂತಾದುವುಗಳು ಬೆರೆತಿದ್ದುವು ಎಂದು ಸ್ವಾಮೀಜಿ ವಾದಿಸಿದರು. ಬೌದ್ಧರು ಅಹಿಂಸಾ ಪರಮೋಧರ್ಮ ಎಂದು ಸಾರಿದರೂ ಮಾಂಸವನ್ನು ತೆಗೆದುಕೊಳ್ಳುತ್ತಿದ್ದರು. ಜೈನಧರ್ಮದ ಪ್ರಾಬಲ್ಯದಿಂದ ಇದು ಕಡಿಮೆಯಾಗುತ್ತ ಬಂದು, ಅನಂತರ ಬಂದ ಬ್ರಾಹ್ಮಣರು ಅದನ್ನು ನಿಷೇಧಿಸತೊಡಗಿದರು. ಮಾಂಸವನ್ನು ತಿನ್ನಬಹುದು ಮುಂತಾದ ವಾಕ್ಯಗಳಿಗೆ ಬೇರೆ ಅರ್ಥವನ್ನು ಕೊಟ್ಟು ಅವೆಲ್ಲ ಹಿಂದಿನ ಯುಗಕ್ಕೆ ಅನ್ವಯಿಸುವುದೆಂದೂ, ಕಲಿಯುಗದಲ್ಲಿ ಅವನ್ನು ವರ್ಜಿಸಬೇಕು ಎಂದೂ ಹೇಳಿದರು. ಸ್ವಾಮೀಜಿಯವರ ದೃಷ್ಟಿಯಲ್ಲಿ ಜನಸಾಧಾರಣರು ಬಲಿಷ್ಠರಾಗಬೇಕಾದರೆ ಪ್ರಪಂಚದ ಇತರ ರಾಷ್ಟ್ರಗಳೊಡನೆ ಸ್ಪರ್ಧಿಸಬೇಕಾದರೆ ಮಾಂಸಾಹಾರವನ್ನು ಸ್ವೀಕರಿಸಲೇಬೇಕು. 

 ಒಂದು ದಿನ, ಅಸಿಸ್ಟೆಂಟ್ ದಿವಾನರಾದ ಪಿರಾವಿ ಪೆರುಮಾಳ್ ಪಿಳ್ಳೈ ಸ್ವಾಮೀಜಿಯವರೊಡನೆ ವೇದಾಂತದ ವಿಷಯದಲ್ಲಿ ಚರ್ಚಿಸತೊಡಗಿದರು. ಅದ್ವೈತ ವೇದಾಂತಕ್ಕೆ ವಿರುದ್ಧವಾದ ತಮ್ಮ ಅಭಿಪ್ರಾಯಗಳನ್ನೆಲ್ಲ ವ್ಯಕ್ತಪಡಿಸಿದರು. ಸ್ವಾಮೀಜಿ ಅವುಗಳನ್ನೆಲ್ಲ ತೃಪ್ತಿಕರವಾಗಿ ಬಗೆಹರಿಸಿದರು. ಒಂದು ದಿನ ಸ್ವಾಮೀಜಿಯವರನ್ನು ಒಂದು ಬಹಿರಂಗ ಉಪನ್ಯಾಸ ಕೊಡಿ ಎಂದು ಕೇಳಿದಾಗ, ಬಹಿರಂಗದಲ್ಲಿ ಉಪನ್ಯಾಸಮಾಡಿ ತಮಗೆ ಅಭ್ಯಾಸವಿಲ್ಲವೆಂದರು. ಆಗ ಸುಂದರರಾಮ ಅಯ್ಯರ್ “ನಿಮಗೆ ಇಲ್ಲಿ ಮಾತನಾಡುವುದಕ್ಕೆ ಅಭ್ಯಾಸ ಇಲ್ಲ ಎಂದರೆ ಅಮೇರಿಕಾ ದೇಶಕ್ಕೆ ಹೋಗಿ ಅಲ್ಲಿ ಆ ದೊಡ್ಡ ಸಭೆಯಲ್ಲಿ ಹೇಗೆ ಮಾತನಾಡುತ್ತೀರಿ?” ಎಂದರು. ಇದಕ್ಕೆ ಸ್ವಾಮೀಜಿ “ಆ ಸಮಯದಲ್ಲಿ ದೇವರು ಬೇಕಾದರೆ ನನ್ನನ್ನು ತನ್ನ ನಿಮಿತ್ತ ಮಾಡಿಕೊಂಡು ಬೇಕಾದ ಸ್ಫೂರ್ತಿಯನ್ನು ಅವನು ಒದಗಿಸಬಹುದು.” ಎಂದರು. ಅದಕ್ಕೆ ಸುಂದರರಾಮ ಅಯ್ಯರ್ ಈಗ ಇಲ್ಲದೇ ಇರುವುದನ್ನು ಆಗ ನಿಮಗೆ ಅನುಗ್ರಹಿಸುತ್ತಾನೆ ಎನ್ನುವುದರಲ್ಲಿ ತಮಗೆ ಸಂದೇಹ ಇದೆ ಎಂಬುದನ್ನು ವ್ಯಕ್ತಪಡಿಸಿದಾಗ, ಸ್ವಾಮೀಜಿ ಹೀಗೆ ಹೇಳಿದರು: “ನೀನು ನೋಡುವುದಕ್ಕೆ ಆಚಾರಶೀಲನಾದ ಬ್ರಾಹ್ಮಣನಂತೆ ಇರುವೆ. ಆದರೆ ದೇವರು ಹೇಳುವುದರಲ್ಲೆ ನಿನಗೆ ನಂಬಿಕೆ ಇಲ್ಲವಲ್ಲ? ದೇವರು ಮೂಕನನ್ನು ವಾಚಾಳಿಯನ್ನಾಗಿ ಮಾಡುತ್ತಾನೆ. ಹೆಳವನ ಕೈಯಿಂದ ಬೆಟ್ಟವನ್ನು ದಾಟಿಸುತ್ತಾನೆ. ಎಂದು ಪ್ರತಿದಿನ ಪಾರಾಯಣ ಮಾಡುತ್ತಿದ್ದರೂ, ಇವುಗಳಲ್ಲಿ ನಂಬಿಕೆಯೇ ಇಲ್ಲ.” ಅನೇಕರಿಗೆ ದೇವರ ವಾಣಿ ಉದಾಹರಿಸುವುದಕ್ಕಾಗಿ ಇದೆ, ಜೀವನದ ಊರುಗೋಲು ಆಗಿಲ್ಲ. 

 ಮತ್ತೊಂದು ದಿನ ಸುಂದರರಾಮ ಅಯ್ಯರ್ ಬ್ರಾಹ್ಮಣ ವರ್ಣ ಪರಿಶುದ್ಧವಾದದ್ದು, ಅದು ಮತ್ತಾವ ಜಾತಿಯೊಂದಿಗೂ ಬೆರೆತಿಲ್ಲವೆಂದು ವಾದಿಸುತ್ತಿದ್ದರು. ಅದಕ್ಕೆ ಸ್ವಾಮೀಜಿ “ಬ್ರಾಹ್ಮಣರಲ್ಲಿಯೂ ಮಿಶ್ರಣವಾಗಿದೆ” ಎಂದರು. ಹಿಂದಿನ ಕಾಲದಿಂದಲೂ ಬ್ರಾಹ್ಮಣ ಉಳಿದ ನಾಲ್ಕು ವರ್ಣಗಳಲ್ಲಿ ಯಾರನ್ನು ಬೇಕಾದರೂ ಮದುವೆ ಮಾಡಿಕೊಳ್ಳಬಹುದೆಂದು ಶಾಸ್ತ್ರದಲ್ಲಿಯೇ ಇದೆ. ಹಾಗೆಯೇ ಅನೇಕ ವೇಳೆ ಆಗಿಯೂ ಇದ್ದುವು. ಪರಿಶುದ್ಧವಾದ ಪ್ರತ್ಯೇಕವಾದ ಒಂದು ಜಾತಿಯಿದೆ ಎಂಬುದು ಕೆಲವರ ಬುಡವಿಲ್ಲದ ಭಾವನೆ ಎಂಬುದು ಸ್ವಾಮೀಜಿ ಅವರ ಮತ. 

 ಸ್ವಾಮೀಜಿ ಹದಿನಾಲ್ಕು ವಯಸ್ಸಿನ ಸುಂದರರಾಮ ಅಯ್ಯರ್ ಅವರ ಮಗನೊಡನೆ ಕೆಲವು ವೇಳೆ ಮಾತನಾಡುತ್ತಿದ್ದರು. ಒಂದು ಸಲ ಆ ಹುಡುಗ ಕಾಳಿದಾಸನ ‘ಕುಮಾರ ಸಂಭವ’ವನ್ನು ಓದುತ್ತಿದ್ದ. ಸ್ವಾಮೀಜಿ ಆ ಹುಡುಗನಿಗೆ “ಅದರಲ್ಲಿ ಬರುವ ಹಿಮಾಲಯದ ವರ್ಣನೆಯನ್ನು ಓದು” ಎಂದರು. ಹುಡುಗ ಅದನ್ನು ಓದಿದ:

\begin{verse}
ಅಸ್ತ್ಯುತ್ತರಸ್ಯಾಂ ದಿಶಿ ದೇವತಾತ್ಮಾ\\ ಹಿಮಾಲಯೋ ನಾಮ ನಗಾಧಿರಾಜಃ~।\\ ಪೂರ್ವಾಪರೌ ತೋಯನಿಧೀವಗಾಹ್ಯಃ\\ ಸ್ಥಿತಃ ಪೃಥಿವ್ಯಾಇವ ಮಾನದಂಡಃ~॥ 
\end{verse}

 “ಇದಕ್ಕೆ ಅರ್ಥ ಗೊತ್ತಿದೆಯೆ?” ಎಂದು ಹುಡುಗನನ್ನು ಸ್ವಾಮೀಜಿ ಕೇಳಿದರು. ಹುಡುಗ ಕೇವಲ ಅದರ ಬಾಹ್ಯಾರ್ಥವನ್ನು ವಿವರಿಸಿದ. ಸ್ವಾಮೀಜಿ “ನೀನು ಕೊಟ್ಟ ಅರ್ಥವೇನೋ ಸರಿಯಾಗಿದೆ, ಆದರೆ ಅದು ಅಷ್ಟೇ ಅಲ್ಲ. ಕವಿಯ ಮನಸ್ಸಿನಲ್ಲಿ ಅಗಾಧ ಭಾವಗಳಿವೆ” ಎಂಬುದನ್ನು ವಿವರಿಸತೊಡಗಿದರು. ಈ ಶ್ಲೋಕದಲ್ಲಿ ‘ದೇವತಾತ್ಮ’ ಮತ್ತು ‘ಮಾನದಂಡ’ ಎಂಬ ಪದಗಳು ಧ್ವನಿಪೂರ್ಣವಾದುವು. ಹಿಮಾಲಯ ಬರೀ ಬೆಟ್ಟವಲ್ಲ. ಅದು ದೇವತಾತ್ಮಾ, ಭರತಖಂಡವನ್ನು ಶತ್ರುಗಳಿಂದ ರಕ್ಷಿಸುವುದಕ್ಕೆ ಮತ್ತು ಅಲ್ಲಿಯ ಜನರಿಗೆ ಜೀವನದಿಗಳನ್ನು ಕೊಡುವುದಕ್ಕಾಗಿ ಇದೆ. ಇದರಿಂದ ಪೋಷಿತವಾದ ಜನರ ನಾಗರಿಕತೆ ಪ್ರಪಂಚದ ಇತರ ನಾಗರಿಕತೆಗಳನ್ನು ಅಳೆಯುವುದಕ್ಕೆ ಮಾನದಂಡ ಎಂದರು. ಹಿಮಾಲಯ ಪವಿತ್ರತೆಗೆ ಚಿಹ್ನೆ. ಅದು ಬರೀ ಬೆಟ್ಟವಲ್ಲ. ಅದರಿಂದ ಪೋಷಿತವಾದ ನಾಗರಿಕತೆ ಜಗದ ನಾಗರಿಕತೆಗಳೆಲ್ಲಕ್ಕಿಂತ ಪುರಾತನವಾದುದು, ಬೃಹತ್ತಾದುದು, ಸನಾತನವಾದುದು ಎಂಬ ಭಾವನೆಗಳೆಲ್ಲಾ ಇದರಲ್ಲಿ ಅಂತರ್ಗತವಾಗಿದೆ ಎಂದು ವಿವರಿಸಿದರು. ಕಾಳಿದಾಸ ಭಾರತದ ಆತ್ಮವನ್ನೇ ಆ ಎರಡು ಪದಗಳಲ್ಲಿ ಹುದುಗಿಸಿ ಇಟ್ಟಿರುವಂತೆ ಸ್ವಾಮೀಜಿಯವರಿಗೆ ಭಾಸವಾಯಿತು. 

 ತಿರುವನಂತಪುರದಿಂದ ಸ್ವಾಮೀಜಿ ಮಧುರೆಗೆ ಹೋದರು. ಅಲ್ಲಿ ಮೀನಾಕ್ಷಿಯ ಭವ್ಯವಾದ ದೇವಾಲಯವನ್ನು ನೋಡಿದರು. ರಾಮನಾಡಿನ ರಾಜರಾದ ಭಾಸ್ಕರ ಸೇತುಪತಿಯನ್ನು ಕಂಡು ಮಾತನಾಡಿಸಿದರು. ಅವರ ಮನೆಯಲ್ಲಿ ಕೆಲವು ದಿನ ಅತಿಥಿಗಳಾಗಿದ್ದರು. ಅಲ್ಲಿಯೇ ಸ್ವಾಮೀಜಿ ಭರತಖಂಡದ ಸಮಸ್ಯೆಗಳ ವಿಷಯವಾಗಿ ರಾಜರೊಡನೆ ಮಾತನಾಡಿದರು. ಜನರಲ್ಲಿ ವಿದ್ಯಾಭ್ಯಾಸ ಹರಡಬೇಕು, ಅವರ ವ್ಯವಸಾಯದ ಉತ್ಪಾದನೆ ಜಾಸ್ತಿಯಾಗುವಂತೆ ಮಾಡಬೇಕು ಎಂದು ಹೇಳಿದರು. ಭರತ ಖಂಡದಲ್ಲಿ ಅನರ್ಘ್ಯವಾದ ಆಧ್ಯಾತ್ಮಿಕ ಶಕ್ತಿ ಇದೆ, ಅದನ್ನು ವ್ಯಕ್ತವಾಗುವಂತೆ ಮಾಡಬೇಕು. ಆಗಲೆ ಭರತಖಂಡ ಉದ್ಧಾರವಾಗಬೇಕಾದರೆ ಎಂದರು. ಆಗ ರಾಜರು ಸ್ವಾಮೀಜಿಯವರನ್ನು ವಿಶ್ವಧರ್ಮ ಸಮ್ಮೇಳನಕ್ಕೆ ಹೋಗಿ ಅಲ್ಲಿ ಹಿಂದೂಧರ್ಮದ ಪರವಾಗಿ ಮಾತನಾಡಿ ಎಂದು ಕೋರಿಕೊಂಡರು. ಆಗ ಜಗದ ಕಣ್ಣು ಇದರ ಕಡೆ ತಿರುಗುತ್ತದೆ. ಭಾರತೀಯರಿಗೂ ಕೂಡ ಎಂತಹ ಶ್ರೇಷ್ಠವಾದ ವಸ್ತು ತಮ್ಮಲ್ಲಿದೆ ಎಂಬುದು ಗೊತ್ತಾಗಿ ಅವರು ಜಾಗ್ರತವಾಗುವುದಕ್ಕೆ ಸಹಕಾರಿಯಾಗುವುದು ಎಂದರು. ಸ್ವಾಮೀಜಿ ಹೋಗುವುದಾದರೆ ತಾವು ದ್ರವ್ಯಸಹಾಯ ಮಾಡುವೆನು ಎಂದರು. ಆದರೆ ಸ್ವಾಮೀಜಿ ಮೊದಲು ರಾಮೇಶ್ವರ ಯಾತ್ರೆಯನ್ನು ಪೂರೈಸಬೇಕೆಂದು ಸಂಕಲ್ಪ ಮಾಡಿಕೊಂಡಿದ್ದರು. ಆದಕಾರಣ ಈಗ ಅದರ ವಿಷಯವಾಗಿ ತಾವು ಏನನ್ನೂ ಆಲೋಚಿಸುತ್ತಿಲ್ಲವಾದರೂ ಅನಂತರ ಹೋಗಬೇಕಾಗಿ ಬಂದಾಗ ಅವರಿಗೆ ತಿಳಿಸುವುದಾಗಿ ಹೇಳಿದರು. 

 ಅನಂತರ ಸ್ವಾಮೀಜಿ ರಾಮೇಶ್ವರಕ್ಕೆ ಹೋದರು. ಹಿಂದೂಗಳಿಗೆಲ್ಲ ಅತ್ಯಂತ ಪವಿತ್ರವಾಗಿರುವ ದಕ್ಷಿಣದಲ್ಲಿರುವ ಈ ಶಿವ ದೇವಾಲಯ ಪುರಾಣ ಪ್ರಸಿದ್ಧವಾದುದು. ಶ‍್ರೀರಾಮಚಂದ್ರನೇ ರಾವಣಾಸುರನನ್ನು ಕೊಂದಾದ ಮೇಲೆ ಭರತಖಂಡಕ್ಕೆ ಕಾಲಿಟ್ಟಾಗ ತನ್ನ ಕೈಗಳಿಂದಲೇ ಇಲ್ಲಿ ಶಿವಲಿಂಗವನ್ನು ಪ್ರತಿಷ್ಠೆ ಮಾಡಿದನು ಎಂದು ಪುರಾಣಗಳು ಹೇಳುವುವು. ಅನಂತರ ಸ್ವಾಮೀಜಿ ಕನ್ಯಾಕುಮಾರಿಯ ಕಡೆಗೆ ಹೊರಟರು. ಕನ್ಯಾಕುಮಾರಿ ಭರತಖಂಡದ ದಕ್ಷಿಣದ ತುತ್ತತುದಿ. ಭಾರತಮಾತೆಯ ಪಾದಗಳನ್ನು ಹಿಂದೂಸಾಗರ, ಬಂಗಾಳ ಸಮುದ್ರ ಮತ್ತು ಅರಬ್ಬಿ ಸಮುದ್ರಗಳು ಅನುಗಾಲವೂ ಇಲ್ಲಿ ತೊಳೆಯುತ್ತಿವೆ. ಸ್ವಾಮೀಜಿ ಉತ್ತರ ತೀರ್ಥ ಕ್ಷೇತ್ರಗಳು, ಇತಿಹಾಸದ ಪ್ರಸಿದ್ಧವಾದ ನಗರಗಳು, ಹಿಂದೆ ಚಕ್ರಾಧಿಪತ್ಯಗಳ ರಾಜಧಾನಿಯಾಗಿ ಈಗ ಪಾಳು ಬಿದ್ದುಹೋಗಿರುವ ಪುರಾತನ ಪಟ್ಟಣಗಳು, ಮಹಾನದಿಗಳು, ಹಿಂದೂ, ಜೈನ, ಬೌದ್ಧ, ಮುಸಲ್ಮಾನ ಕಾಲಕ್ಕೆ ಸೇರಿದ ವಾಸ್ತುಶಿಲ್ಪಗಳು ಎಲ್ಲವನ್ನೂ ನೋಡಿಕೊಂಡು ಈಗ ಕೊನೆಯ ತೀರ್ಥಸ್ಥಳಕ್ಕೆ ಹೋಗುತ್ತಿರುವರು. ಇಷ್ಟು ಹೊತ್ತಿಗೆ ಸ್ವಾಮೀಜಿಯವರಿಗೆ ಭರತಖಂಡದ ಆಮೂಲಾಗ್ರವಾದ ಅನುಭವ ಆಗಿತ್ತು. ಅವರು ರಾಜರ ಅರಮನೆಗಳಲ್ಲಿದ್ದರು. ಅವರ ಆತಿಥ್ಯವನ್ನು ಸ್ವೀಕರಿಸಿದ್ದರು. ಅವರಿಗೆ ಬೋಧನೆ ಮಾಡಿದ್ದರು. ಬಡವರ ಗುಡಿಸಲುಗಳಲ್ಲಿದ್ದರು. ಅವರು ಕೊಟ್ಟ ಕಾರ್ಪಣ್ಯದ ಊಟವನ್ನು ತಿಂದರು, ಇಂದ್ರನು ತಾನೇ ಕೊಡುವ ಅಮೃತಕ್ಕಿಂತ ಅದು ರುಚಿಯಾಗಿತ್ತೆಂದು ಹೇಳಿದರು. ಗುಡಿಸಲು ಬಡತನದಲ್ಲಿದ್ದರೂ, ಅವರ ಉದಾರ ಹೃದಯವನ್ನು ಮನಗಂಡಿದ್ದರು. ಅವರ ಸರಳಜೀವನದಲ್ಲಿದ್ದ ಪ್ರೀತಿ ವಿಶ್ವಾಸ ಸಹಾನುಭೂತಿಗಳು ಇವರಿಗೆ ಪ್ರತ್ಯಕ್ಷ ಅರಿವಾಗಿದ್ದವು. ಮರಳುಕಾಡಿನಲ್ಲಿ, ಬೆಟ್ಟಗುಡ್ಡಗಳಲ್ಲಿ, ಮಹಾರಣ್ಯಗಳಲ್ಲಿ ಪಾದಚಾರಿಯಾಗಿ ಅನೇಕ ವೇಳೆ ಸಂಚರಿಸಿದ್ದರು. ರಾಜರುಗಳಿಂದ ಪಾದಸೇವೆ ಮಾಡಿಸಿಕೊಂಡರು. ನಾಯಿಗೆ ಮತ್ತು ಸಾಧುವಿಗೆ ಇಲ್ಲಿ ಸ್ಥಳವಿಲ್ಲ ಎಂಬ ಕಟು ನಿಂದೆಯನ್ನು ಕೇಳಿದ್ದರು. ಶ‍್ರೀರಾಮಕೃಷ್ಣರ ಪದತಲದಲ್ಲಿ ಕುಳಿತು ಆಧ್ಯಾತ್ಮಿಕ ಜೀವನದ ಸವಿ ಅನುಭವಿಸಿದ್ದರು. ನಮ್ಮ ಪುಣ್ಯಗ್ರಂಥಗಳಲ್ಲಿ ಹುದುಗಿರುವ ವಿಷಯವನ್ನರಿತಿದ್ದರು. ದೇಶವನ್ನು ಅಲೆದಾಡುತ್ತಿದ್ದಾಗ ಧರ್ಮದ ಹೆಸರಿನಲ್ಲಿರುವ ಕೆಲಸಕ್ಕೆ ಬಾರದ ಮೂಢ ನಂಬಿಕೆಗಳು, ಆಚಾರಗಳ ಕಳೆಗಳೇ ತುಂಬಿಕೊಂಡು ಮೂಲವಾದ ಮರ ಯಾವುದೆಂಬುದನ್ನು ಮರೆಸುವ ರೀತಿಯಲ್ಲಿದ್ದುದನ್ನು ನೋಡಿದರು. ವೈವಿಧ್ಯಪೂರ್ಣವಾದ ಅನುಭವವನ್ನು ಪಡೆದುಕೊಂಡು, ನಮ್ಮ ಗತಕಾಲದ ವೈಭವ, ಈಗಿನ ನಮ್ಮ ಅಧೋಗತಿ, ಮುಂದೆ ವೈಭವಯುಕ್ತವಾದ ರಾಷ್ಟ್ರ ನಿರ್ಮಾಣಕ್ಕೆ ನಾವು ಮಾಡಬೇಕಾದುದೇನೆಂಬುದನ್ನೆಲ್ಲ ಮನನಮಾಡುತ್ತ ಕನ್ಯಾಕುಮಾರಿಯ ದೇವಾಲಯದ ಕಡೆ ಸಾಗಿದರು ಸ್ವಾಮಿಗಳು. 

 ಡಿಸೆಂಬರ್ ತಿಂಗಳಲ್ಲಿ ಒಂದು ದಿನ ಬೆಳಿಗ್ಗೆ ಕನ್ಯಾಕುಮಾರಿಯ ದೇವಸ್ಥಾನಕ್ಕೆ ಹೋದರು. ಮಾತೆಯ ಮುಂದೆ ಸಾಷ್ಟಾಂಗ ಪ್ರಣಾಮ ಮಾಡಿದರು. ಭಕ್ತಿಯಿಂದ ಪುಳಕಿತರಾದರು. ಸ್ವಲ್ಪ ಕಾಲ ಧ್ಯಾನದಲ್ಲಿ ತನ್ಮಯರಾಗಿದ್ದರು. ಅನಂತರ ಅಲ್ಲಿಂದ ದೇವಸ್ಥಾನದ ಹೊರಗೆ ಬಂದರು. ಒಂದು ನಿರ್ಜನ ಪ್ರದೇಶದಲ್ಲಿ ಕೆಲವು ಕಾಲ ಧ್ಯಾನದಲ್ಲಿ ಕಳೆಯಬೇಕೆಂದು ಸ್ಥಳವನ್ನು ಹುಡುಕಿ ನೋಡಿದರು. ದೇವಸ್ಥಾನದ ಮುಂದೆ ಸಮುದ್ರದ ಮಧ್ಯದಲ್ಲಿ ಭೀಮಾಕಾರದ ಒಂದು ಬಂಡೆ ಎದ್ದು ನಿಂತಿದ್ದುದು ಸ್ವಾಮೀಜಿಯವರಿಗೆ ಕಂಡು ಬಂದಿತು. ಜನರಿಂದ ದೂರವಿರುವ ಆ ನಿರ್ಜನ ಪ್ರದೇಶ ಧ್ಯಾನಕ್ಕೆ ಯೋಗ್ಯವಾದ ಸ್ಥಾನವೆಂದು ನಿರ್ಧರಿಸಿದರು. ಅಲ್ಲಿಗೆ ಹೋಗಬೇಕಾದರೆ ಸಣ್ಣ ಮರದ ದೋಣಿಯಲ್ಲಿ ಹೋಗಬೇಕು. ಒಬ್ಬ ದೋಣಿಯವನನ್ನು ಅಲ್ಲಿಗೆ ತನ್ನನ್ನು ಕರೆದುಕೊಂಡು ಹೋಗು ಎಂದು ಕೇಳಿದರು. ಆತ ಅದಕ್ಕೆ ಕೂಲಿ ಕೇಳಿದ. ಸ್ವಾಮೀಜಿಯವರ ಹತ್ತಿರ ಒಂದು ಕುರುಡು ಕಾಸೂ ಇರಲಿಲ್ಲ. ಅವರು ಸಮುದ್ರದಲ್ಲಿ ಈಜಿಕೊಂಡು ಅಲೆಗಳನ್ನು ದಾಟಿಕೊಂಡು ಆ ಭೀಮ ಬಂಡೆಯ ನೆತ್ತಿಯನ್ನು ಮುಟ್ಟಿದರು. ಅದೋ ವಿಶಾಲವಾದ ಸಣ್ಣ ದ್ವೀಪದಂತೆಯೇ ಇರುವ ಬಂಡೆ. ಅಲ್ಲಿ ಒಂದು ಸಿಹಿ ನೀರಿನ ಚಿಲುಮೆಯೂ ಇದೆ. ಸುತ್ತಲೂ ಸಮುದ್ರ, ಅದರ ಮುಂದೆ ಭರತಖಂಡದ ಪಾದಕಮಲಗಳಂತಿರುವ ಕನ್ಯಾಕುಮಾರಿಯ ದೇವಸ್ಥಾನ. ಅಲ್ಲಿ ದೀರ್ಘ ಧ್ಯಾನದಲ್ಲಿ ತನ್ಮಯರಾದರು. ಮುಂದೇನು ಮಾಡಬೇಕೆಂಬುದನ್ನು ನಿಶ್ಚಯಿಸಿದರು. 

 ಕನ್ಯಾಕುಮಾರಿಯ ಬಂಡೆಯ ಮೇಲೆ ಕುಳಿತುಕೊಂಡು ಭರತಖಂಡದ ವಿಷಯವಾಗಿ ಚಿಂತಿಸತೊಡಗಿದರು. ಭರತಖಂಡದ ಅಧಃಪತನಕ್ಕೆ ಮೂಲಕಾರಣವನ್ನು ಯೋಚಿಸತೊಡಗಿದರು. ಭರತಖಂಡ ಅತ್ಯುನ್ನತ ಕೀರ್ತಿಯ ಶಿಖರದಿಂದ ವರ್ತಮಾನದ ಅವಹೇಳನೀಯವಾದ ಸ್ಥಿತಿಗೆ ಬರಲು ಕಾರಣಗಳನ್ನು ಕಂಡುಹಿಡಿದರು. ಯತಿಯನ್ನು ಮಹಾಸುಧಾರಕರಾಗುವಂತೆ, ಮಹಾಸಂಸ್ಥೆಯನ್ನು ಕಟ್ಟುವ ಚಾಲಕ ಶಕ್ತಿಯಾಗುವಂತೆ, ಭರತಖಂಡದ ರಾಷ್ಟ್ರನಿರ್ಮಾಪಕರಂತೆ ಮಾಡಿದ ಸ್ಥಳ ಇದು. ಆಮೂಲಾಗ್ರವಾಗಿ ಅಖಂಡ ಭರತಖಂಡವನ್ನು ಚಿಂತಿಸತೊಡಗಿದರು. ಎಲ್ಲಾ ಭಾಷೆಗಳ ಪ್ರಾಂತ್ಯಗಳ ಜನರು, ಹಿಂದಿನದು ಮತ್ತು ಈಗಿನದು ಇವುಗಳನ್ನೆಲ್ಲ ಒಂದುಮಾಡಿದ ಹಿಂದೂ ಸಂಸ್ಕೃತಿಯ ಸೂತ್ರವನ್ನು ಕಂಡುಹಿಡಿದರು. ಅದರಲ್ಲಿರುವ ಶಕ್ತಿ ಮತ್ತು ಸಾಧ್ಯತೆ ಅವರಿಗೆ ಅರಿವಾಗತೊಡಗಿತು. ಧರ್ಮವೇ ಭಾರತೀಯರ ಪ್ರಾಣದುಸಿರು, ಅವರ ಧಮನಿ ಧಮನಿಯಲ್ಲಿ ಪ್ರತಿಬಿಂಬಿಸುತ್ತಿರುವುದು. ಕೋಟಿ ಕೋಟಿ ಭಾರತೀಯರ ಜೀವನದ ಮೂಲವೆ ಇದು ಎಂಬುದನ್ನು ಕಂಡುಹಿಡಿದರು. ಧರ್ಮ ಜಾಗ್ರತಿಯ ಮೂಲಕ ಭರತಖಂಡ ಜಾಗ್ರತವಾಗಬೇಕು. ಧರ್ಮಜಾಗ್ರತವಾದರೆ, ಉಳಿದವುಗಳೆಲ್ಲ - ರಾಜಕೀಯ ಆರ್ಥಿಕ ಮುಂತಾದವುಗಳೆಲ್ಲ ಅದರ ಹಿಮ್ಮೇಳದಂತೆ ಬರುವುವು. ಅದನ್ನು ಸರಿಪಡಿಸಿದರೆ, ಉಳಿದವುಗಳೆಲ್ಲ ತಮಗೆ ತಾವೇ ಸರಿಯಾಗುವುವು. ಬೇರಿಗೆ ನೀರನ್ನು ಹಾಕಿದರೆ ಶಾಖೋಪಶಾಖೆಗಳಿಗೆಲ್ಲ ಅದು ತಾಕುವುದು. ಅವರಿಗೆ ಭಾರತೀಯರ ಶಕ್ತಿ ಅರಿವಾಯಿತು, ರಾಷ್ಟ್ರ ಈಗ ತನ್ನ ವ್ಯಕ್ತಿತ್ವವನ್ನು ಕಳೆದುಕೊಂಡ ದೋಷವೂ ಅರಿವಾಯಿತು. 

 ಭರತಖಂಡದಲ್ಲಿರುವ ದಟ್ಟ ದಾರಿದ್ರ್ಯವನ್ನು ಕುರಿತು ಚಿಂತಿಸತೊಡಗಿದರು. ಧರ್ಮ ಲಕ್ಷೋಪಲಕ್ಷ ಸಾಧಾರಣ ಜೀವಿಗಳ ಸಹಾಯಕ್ಕೆ ಬರದೆ ಇದ್ದರೆ ಅದರಿಂದ ಏನು ಪ್ರಯೋಜನವಾಗುವುದು? ಪುರೋಹಿತರು ಮತ್ತು ಮೇಲಿನ ವರ್ಗದವರ ದೌರಾತ್ಮ್ಯ ಕೆಳಗಿನವರ ಪ್ರಾಣವನ್ನೆಲ್ಲ ಹೀರಿತು. ಇವರೇ ಭರತಖಂಡದ ಉನ್ನತಿಗೆ ಕಂಟಕಪ್ರಾಯರಂತೆ ಕಂಡರು. ಯಾರು ಧರ್ಮರಕ್ಷಕರೆಂದು ಹೆಮ್ಮೆ ಕೊಚ್ಚಿಕೊಳ್ಳುತ್ತಿರುವರೋ, ಅವರು ಸಾಧಾರಣ ಜನರಿಗೆ ಧರ್ಮದ ತಿರುಳನ್ನು ಕೊಡಲಿಲ್ಲ. ಯಾವಾಗಲೂ ಅವರನ್ನು ಬಂಧನದಲ್ಲಿ ಕಟ್ಟಿಹಾಕಿದರು. \enginline{1894} ಮಾರ್ಚ್ \enginline{19}ರಂದು ಚಿಕಾಗೋದಿಂದ ಸ್ವಾಮಿ ರಾಮಕೃಷ್ಣಾನಂದರಿಗೆ ಬರೆದ ಪತ್ರದಲ್ಲಿ ಈ ಸಂದರ್ಭದ ಅವರ ಮನೋಭಾವ ವ್ಯಕ್ತವಾಗಿರುತ್ತದೆ: “ಸುಮಾರು ಹತ್ತು ಇಪ್ಪತ್ತು ಲಕ್ಷ ಸಾಧುಸಂತರು ಮತ್ತು ಅನೇಕ ಕೋಟಿ ಬ್ರಾಹ್ಮಣರು ಬಡವರ ರಕ್ತವನ್ನು ಹೀರುತ್ತಿರುವರು. ಅವರನ್ನು ಮೇಲೆತ್ತುವುದಕ್ಕೆ ಸ್ವಲ್ಪವೂ ಶ್ರಮ ಪಡುವುದಿಲ್ಲ. ಇದು ಒಂದು ದೇಶವೋ ಅಥವಾ ನರಕದ ಬೀಡೋ, ಇಲ್ಲಿರುವುದು ಧರ್ಮವೋ ಅಥವಾ ಪಿಶಾಚಿಯ ನೃತ್ಯವೋ? ನನ್ನ ಸೋದರನೆ, ನೀನು ಇದನ್ನು ಕೂಲಂಕುಷವಾಗಿ ವಿಚಾರ ಮಾಡಬೇಕಾಗಿದೆ. ನಾನು ಇವುಗಳನ್ನು ನೋಡಿ, ಅದರಲ್ಲಿಯೂ ಬಡತನ ಮತ್ತು ಮೌಢ್ಯತೆಗಳನ್ನು ಕಂಡು ನನಗೆ ನಿದ್ರೆ ಬರಲಿಲ್ಲ. ಕನ್ಯಾಕುಮಾರಿಯಲ್ಲಿ ಭಾರತವರ್ಷದ ಕೊನೆಯ ಕಲ್ಲುಬಂಡೆಯ ಮೇಲೆ ಕುಳಿತಾಗ ನನಗೆ ಒಂದು ಆಲೋಚನೆ ಹೊಳೆಯಿತು. ನಮ್ಮಂತಹ ಹಲವು ಸಂನ್ಯಾಸಿಗಳು ಸಂಚಾರ ಮಾಡುತ್ತ ತತ್ತ್ವಬೋಧನೆಯನ್ನು ಜನರಿಗೆ ಮಾಡುತ್ತೇವೆ ಅಲ್ಲವೆ? ಇದೆಲ್ಲ ಹುಚ್ಚುತನ. ‘ಹಸಿದವನಿಗೆ ಧರ್ಮ ಬೋಧಿಸುವುದು ತರವಲ್ಲ’ ಎಂದು ನಮ್ಮ ಗುರುಗಳು ಹೇಳಲಿಲ್ಲವೆ? ಬಡವರು ಪಶುಗಳಂತೆ ಜೀವಿಸುತ್ತಿರುವರು. ಇದಕ್ಕೆ ಅಜ್ಞಾನವೇ ಕಾರಣ. ಅನೇಕ ಶತಮಾನದಿಂದ ನಾವು ಅವರ ರಕ್ತವನ್ನು ಹೀರಿರುವೆವು. ಅವರನ್ನು ಕಾಲಿನಿಂದ ತುಳಿಯುತ್ತಿರುವೆವು.

 “ಪರಹಿತವನ್ನು ಬಯಸುವ ಸ್ವಾರ್ಥಶೂನ್ಯರಾದ ಕೆಲವು ಸಂನ್ಯಾಸಿಗಳು ಗ್ರಾಮ ಸಂಚಾರ ಹೊರಟು ವಿದ್ಯಾಪ್ರಚಾರ ಮಾಡುತ್ತ, ಭೂಪಟ, ಮಾಯಾದೀಪ, ಕೃತಕ ಗೋಳ ಇವೇ ಮುಂತಾದ ಉಪಕರಣಗಳ ಮೂಲಕ, ಉತ್ತಮರಿಂದ ಹಿಡಿದು ಚಂಡಾಲನವರೆಗೆ ತಿಳುವಳಿಕೆಯನ್ನು ಕೊಡುತ್ತ, ಸಕಲರ ಮೇಲ್ಮೆಗಾಗಿ ಪ್ರಯತ್ನಪಟ್ಟಲ್ಲಿ ಅದರಿಂದ ಕಾಲಾನುಕಾಲಕ್ಕೆ ಕಲ್ಯಾಣವಾಗದೆ ಉಂಟೆ? ನನ್ನ ಮನಸ್ಸಿನಲ್ಲಿರುವ ಹಂಚಿಕೆಯನ್ನೆಲ್ಲ ಈ ಒಂದು ಪತ್ರದಲ್ಲಿ ಬರೆಯಲಾರೆ. ಒಟ್ಟೇನೆಂದರೆ ಮಹಮ್ಮದನ ಬಳಿ ಬೆಟ್ಟ ಬಾರದೆ ಇದ್ದರೆ, ಮಹಮ್ಮದನು ಬೆಟ್ಟದ ಬಳಿಗೆ ಹೋಗಬೇಕು. ಪಾಠಶಾಲೆಗಳಿಗೆ ಬಡವರು ಬರುವುದು ಸಾಧ್ಯವಿಲ್ಲ. ಕಾವ್ಯ ಮುಂತಾದವುಗಳನ್ನು ಓದುವುದರಿಂದ ಅವರಿಗೆ ಅಷ್ಟೇನೂ ಲಾಭವಾಗುವುದಿಲ್ಲ. ನಾವು ಜನಾಂಗದ ವ್ಯಕ್ತಿತ್ವವನ್ನು ಕಳೆದುಕೊಂಡಿರುವೆವು. ಅದೇ ಭರತಖಂಡದಲ್ಲಿ ಎಲ್ಲಾ ವಿಪತ್ತಿಗೂ ಕಾರಣ. ಕಳೆದುಕೊಂಡ ವ್ಯಕ್ತಿತ್ವವನ್ನು ದೇಶಕ್ಕೆ ಪುನಃ ಕೊಡಬೇಕು. ಹಿಂದೆ ಬಿದ್ದವರನ್ನು ಉದ್ಧರಿಸಬೇಕು. ಹಿಂದೂಗಳು, ಮಹಮ್ಮದೀಯರು, ಕ್ರೈಸ್ತರು ಎಲ್ಲರೂ ಅವರನ್ನು ತಿಳಿದಿರುವರು. ಅವರನ್ನು ಉದ್ಧಾರ ಮಾಡುವ ಶಕ್ತಿ ಒಳಗಿನಿಂದಲೇ ಅಂದರೆ ಆಚಾರಶೀಲ ಹಿಂದೂಗಳಿಂದ, ಬರಬೇಕು. ಪ್ರತಿಯೊಂದು ದೇಶದಲ್ಲಿಯೂ ದೋಷಗಳಿವೆ. ಅವು ಧರ್ಮದಿಂದ ಆದವುಗಳಲ್ಲ, ಅವು ಧರ್ಮಪರಿಪಾಲನೆಗೆ ವಿರುದ್ಧವಾಗಿವೆ.

 “ಇದರ ಪರಿಹಾರಕ್ಕೆ ಮೊದಲು ನಮಗೆ ಬೇಕಾಗಿರುವುದು ಜನ, ಅನಂತರ ಹಣ. ಪ್ರತಿಯೊಂದು ಊರಿನಿಂದಲೂ ಹತ್ತು ಹದಿನೈದು ಮಂದಿ ಯುವಕರು ಗುರುದೇವನ ಕೃಪೆಯಿಂದ ಸಿಕ್ಕುವರು ಎಂದು ಭಾವಿಸಿರುವೆನು. ಅನಂತರ ಹಣದ ಪ್ರಶ್ನೆ. ಭರತಖಂಡದ ಜನರು ದುಡ್ಡನ್ನು ಬಿಚ್ಚುತ್ತಾರೆ ಎಂದು ತಿಳಿದೆಯೇನು? ಸ್ವಾರ್ಥ ಮೂರ್ತಿಗಳು ಅವರು!” 

 ಆದಕಾರಣವೆ ಪಾಶ್ಚಾತ್ಯ ದೇಶಗಳಿಗೆ ಹೋಗಿ ಧರ್ಮವನ್ನು ಅವರಿಗೆ ಬೋಧಿಸಿ ಅಲ್ಲಿಂದ ಹಣವನ್ನು ತಂದು, ಇಲ್ಲಿ ಭರತಖಂಡದ ಪುನರುದ್ಧಾರಕ್ಕೆ ಪ್ರಯತ್ನ ಮಾಡುವೆನು ಎಂದು ಶಪಥ ತೊಟ್ಟರು. ಭರತಖಂಡದಲ್ಲಿ ಧರ್ಮ ಜಾಗೃತಿಯನ್ನು ಉಂಟುಮಾಡಬೇಕು, ಧರ್ಮ ಜೀವನದ ಎಲ್ಲಾ ಕಾರ‍್ಯಕ್ಷೇತ್ರದಲ್ಲಿಯೂ ವ್ಯಕ್ತವಾಗಬೇಕು, ಅದಕ್ಕಾಗಿ ತಮ್ಮ ಬಾಳನ್ನೇ ಅರ್ಪಣ ಮಾಡಬೇಕು, ತಮ್ಮ ನಿರ್ವಿಕಲ್ಪ ಸಮಾಧಿಯನ್ನೂ ಒತ್ತಟ್ಟಿಗಿಡಬೇಕು ಎಂದು ದೀಕ್ಷಾಬದ್ಧರಾದರು. ಸ್ವಾಮೀಜಿ ಮುಂದೆ ಏನು ಮಾಡಬೇಕು ಎಂಬ ವಿಷಯದಲ್ಲಿ ಕೃತನಿಶ್ಚಯರಾಗಿದ್ದರು. ಆ ವಿಷಯದಲ್ಲಿ ಯಾವ ಅನುಮಾನವೂ ಇರಲಿಲ್ಲ. ‘ಧರ್ಮಜಾಗೃತಿಯ ಮೂಲಕ ಭರತಖಂಡದ ಉದ್ಧಾರ’ ಎಂಬ ಧ್ರುವತಾರೆಯೊಂದು ಅವರ ಜೀವನದಲ್ಲಿ ಸದಾ ಮಿನುಗುತ್ತಿರುವುದನ್ನು ನೋಡುವೆವು. 

