
\chapter*{೧೨. ಉತ್ತರಖಂಡದ ತೀರ್ಥಸ್ಥಳಗಳು}

ನರೇಂದ್ರನಾಥರು ಇತರ ಗುರುಭಾಯಿಗಳೊಡನೆ ಸಂನ್ಯಾಸ ದೀಕ್ಷೆಯನ್ನು ತೆಗೆದುಕೊಂಡು ಮೊದಲು ಸ್ವಾಮಿ ವಿವಿದಿಶಾನಂದ ಎಂಬ ಹೆಸರನ್ನು ಧಾರಣೆ ಮಾಡಿದರು. ಮುಂದೆ ಭಾರತ ಪರಿಕ್ರಮದ ಸಮಯದಲ್ಲಿ ತಮ್ಮ ಹೆಸರನ್ನು, ಗುರುಭಾಯಿಗಳು ಗುರುತು ಹಿಡಿಯದಿರಲೆಂದು, ಆಗಾಗ ಬದಲಾಯಿಸಿಕೊಳ್ಳುತ್ತಿದ್ದರು. ಕೊನೆಗೆ ವಿದೇಶಕ್ಕೆ ಹೊರಡುವ ಮುನ್ನ ವಿವೇಕಾನಂದ ಎಂಬ ಹೆಸರನ್ನು ಇಟ್ಟುಕೊಂಡರು. ಇನ್ನು ಮೇಲೆ ನಾವು ಅವರನ್ನು ಈ ಹೆಸರಿನಿಂದಲೇ ಕರೆಯುತ್ತೇವೆ. ಶ‍್ರೀರಾಮಕೃಷ್ಣರ ನಿರ‍್ಯಾಣಾನಂತರ ಅವರ ಸಂನ್ಯಾಸಿ ಶಿಷ್ಯರು ಕಾಶೀಪುರದಿಂದ ಬಂದು ಬಾರಾನಗರದ ಮಠದಲ್ಲಿ ವಾಸಿಸುತ್ತಿದ್ದರು. ಅವರಲ್ಲಿ ಅನೇಕರಿಗೆ ತೀರ್ಥಯಾತ್ರೆ ಮಾಡಬೇಕೆಂಬ ಆಸೆ ಒಬ್ಬರಾದ ಮೇಲೆ ಒಬ್ಬರಲ್ಲಿ ಮೂಡುತ್ತಿತ್ತು. ಹಲವರು ಸುತ್ತಮುತ್ತ ಇರುವ ಸ್ಥಳಗಳು ಮತ್ತು ದೂರದೂರದ ಯಾತ್ರಾ ಸ್ಥಳಗಳಿಗೆ ಹೋಗುತ್ತಿದ್ದರು. ಗುರುಭಾಯಿಗಳಲ್ಲೆಲ್ಲಾ ಸ್ವಾಮಿ ರಾಮಕೃಷ್ಣಾನಂದರು ಮಾತ್ರ ಮಠವನ್ನು ಬಿಡದೆ ಆ ಸ್ಥಳ ಒಂದರಲ್ಲಿಯೇ ಸಾಧನೆ ಭಜನೆ ಮತ್ತು ಪೂಜೆ ಇವುಗಳಲ್ಲಿ ತಮ್ಮ ಕಾಲವನ್ನೆಲ್ಲ ಕಳೆಯುತ್ತಿದ್ದರು. ಅವರಿಗೆ ಶ‍್ರೀರಾಮಕೃಷ್ಣರೇ ಸಕಲ ತೀರ್ಥಗಳೂ ಆಗಿದ್ದರು. ಸಕಲ ದೇವ ದೇವಿ ಸ್ವರೂಪರೂ ಆಗಿದ್ದರು.

ವಿವೇಕಾನಂದರ ನಾಯಕತ್ವದಲ್ಲಿ ಬಾರಾನಗರ ಮಠ ಸ್ಥಾಪಿತವಾಯಿತು. ಅವರೇ ಅದಕ್ಕೆ ಸ್ಫೂರ್ತಿಯನ್ನು ತುಂಬಿದವರು. ಆದರೂ ಮನೆಯನ್ನು ಬಿಟ್ಟು ಬಂದಾದ ಮೇಲೆ ಮಠವೂ ಒಂದು ಬಂಧನವೆಂದು ಕೆಲವು ವೇಳೆ ಭಾವಿಸುತ್ತಿದ್ದರು. ಒಂದು ಕಬ್ಬಿಣದ ಸರಪಳಿ, ಮತ್ತೊಂದು ಚಿನ್ನದ ಸರಪಳಿ, ಅಷ್ಟೇ ವ್ಯತ್ಯಾಸ. ತಾವು ಸಂಪೂರ್ಣವಾಗಿ ಕೆಲವು ಕಾಲ ನಿರ್ಲಿಪ್ತವಾಗಬೇಕು ಎಂದು ವಿವೇಕಾನಂದರಿಗೆ ಅನ್ನಿಸಿತು. ಕೆಲವು ಕಾಲ ಸಂಪೂರ್ಣವಾಗಿ ದೇವರನ್ನು ನೆಚ್ಚಿ ಪರ್ಯಟನೆ ಮಾಡಬೇಕು. ಅವನು ಹೇಗೆ ತನ್ನನ್ನು ನೆಚ್ಚಿದವನನ್ನು ನೋಡಿಕೊಳ್ಳುತ್ತಾನೆ ಎಂಬುದನ್ನು ಪರೀಕ್ಷಿಸಬೇಕು ಎಂಬ ಒಂದು ಕುತೂಹಲ ಬೇರೆ ಅವರನ್ನು ಬಾಧಿಸಿತು. ಕೊನೆಗೆ ಎಲ್ಲಕ್ಕಿಂತ ಮುಖ್ಯವಾದ ಕಾರಣವೆಂದರೆ, ಅವರು ಬಂದದ್ದು ತಮ್ಮ ಮುಕ್ತಿಗಾಗಿ ಅಲ್ಲ; ಶ‍್ರೀರಾಮಕೃಷ್ಣರು ತಮ್ಮ ಧ್ಯಾನದಲ್ಲಿ ನೋಡಿದ್ದಂತೆ, ಅವರು ಬರುವುದಕ್ಕೆ ಮುಂಚೆ ಮುಕ್ತ ಪುರುಷರಾಗಿದ್ದರು; ಅವರು ಬಂದದ್ದೇ ಭವಜೀವಿಗಳ ಉದ್ಧಾರಕ್ಕೆ. ಸಂಸಾರದಲ್ಲಿ ತನ್ನ ನೈಜತ್ವವನ್ನು ಮರೆತ ಬದ್ಧಜೀವಿ ಯಾವ ಅನುಮಾನಗಳ ಮೂಲಕ ಸಾಗಿಹೋಗುತ್ತಾನೋ, ಕಷ್ಟಕಾರ್ಪಣ್ಯಗಳಲ್ಲಿ ನವೆಯುತ್ತಾನೋ ಅವುಗಳ ಅಗ್ನಿ ಪರೀಕ್ಷೆಯ ಮೂಲಕವೂ ವಿವೇಕಾನಂದರು ಬಂದಿದ್ದರು. ಕೇವಲ ತಮಗೆ ಮಾತ್ರವಲ್ಲ, ಅನಂತ ಜೀವಿಗಳಿಗೆ ಸಹಾಯವಾಗಬೇಕಾದರೆ, ಅವರ ಅನುಮಾನ ಸಂದೇಹಗಳು ಕಷ್ಟಕಾರ್ಪಣ್ಯಗಳು ಇವುಗಳ ಪ್ರತ್ಯಕ್ಷ ಪರಿಚಯವಿರಬೇಕು. ಅದಕ್ಕಾಗಿ ವಿಧಾತನು ವಿವೇಕಾನಂದರನ್ನು ಆ ಅನುಭವದ ಗಾಣದಲ್ಲಿ ಸಾಗಿಬರುವಂತೆ ಮಾಡಿದನು. ಅವರು ಬಂದದ್ದು ಇಡೀ ರಾಷ್ಟ್ರ ಕುಂಡಲಿಯನ್ನೇ ಜಾಗ್ರತಗೊಳಿಸಲೋಸುಗ. ಹಾಗೆ ಮಾಡಬೇಕಾದರೆ ನಮ್ಮ ಜನರ ನಿಕಟ ಪರಿಚಯ ಅವರಿಗೆ ಅತ್ಯಂತ ಅವಶ್ಯಕವಾಗಿ ಆಗಬೇಕಾಗಿತ್ತು. ನಮ್ಮ ಜನರ ಆಚಾರ, ವ್ಯವಹಾರ, ಧರ್ಮ, ನಮ್ಮಲ್ಲಿರುವ ಲೋಪದೋಷಗಳು, ನಾವು ಎಂತಹ ಸ್ಥಿತಿಯಲ್ಲಿದ್ದೆವು, ನಾವು ಎಂತಹ ಸ್ಥಿತಿಗೆ ಬಂದಿರುವೆವು, ಅಖಂಡ ಹಿಂದೂಸ್ಥಾನವನ್ನು ಏಕಸೂತ್ರದಲ್ಲಿ ಬಂಧಿಸಿರುವುದು ಯಾವುದು ಇವುಗಳನ್ನೆಲ್ಲ ಕಂಡುಹಿಡಿಯಲು ಬಯಸಿದರು. ಪ್ರತ್ಯಕ್ಷ ಅನುಭವ ಇದಕ್ಕೆ ಅತ್ಯಂತ ಅವಶ್ಯಕ. ವಿವೇಕಾನಂದರು ಇದಕ್ಕೆ ಸಿದ್ಧರಾದರು.

 ಸ್ವಾಮಿ ವಿವೇಕಾನಂದರು ಪ್ರೇಮಾನಂದ ಮತ್ತು ಫಕೀರಬಾಬು ಎಂಬ ಗೃಹಸ್ಥ ಶಿಷ್ಯರೊಡನೆ ಕಾಶಿಗೆ ಹೋದರು. ಅಲ್ಲಿ ದ್ವಾರಕದಾಸರ ಆಶ್ರಮದಲ್ಲಿ ತಂಗಿದ್ದರು. ವಿವೇಕಾನಂದರು ಯಾವ ಸ್ಥಳದಲ್ಲಿ ಹೋಗಲಿ ಹಿಂದೂ ಸಂಸ್ಕೃತಿ ಮತ್ತು ಧರ್ಮದ ದೃಷ್ಟಿಯಿಂದ ಆಯಾಯಾ ಸ್ಥಳಗಳ ಅಧ್ಯಯನ ನಡೆಸುತ್ತಿದ್ದರು. ಕಾಶಿ ಹಿಂದಿನ ಕಾಲದಿಂದಲೂ ಪ್ರಖ್ಯಾತ ತೀರ್ಥಸ್ಥಳ. ಯಾವ ಮಹಾ ಆಚಾರ‍್ಯನಾಗಲಿ, ಕಾಶಿಗೆ ಬಂದು ತನ್ನ ಬೋಧನೆಯನ್ನು ಎಲ್ಲ ವಿದ್ವಜ್ಜನರೆದುರಿಗಿಟ್ಟು ಇಲ್ಲಿ ಮಾನ್ಯತೆಯನ್ನು ಪಡೆದು ಹಿಂತಿರುಗಿದರೆ ಮಾತ್ರ ಅವನು ಪ್ರಚಾರ ಮಾಡುವ ಬೊಧನೆಗೆ ಇತರ ಕಡೆ ಪುರಸ್ಕಾರ. ಶ‍್ರೀಶಂಕರಾಚಾರ್ಯರು ಇಲ್ಲಿಗೆ ಬಂದರು, ಇಲ್ಲಿ ಕಲಿತರು, ಇಲ್ಲಿ ತಮ್ಮ ಭಾಷ್ಯವನ್ನು ಬರೆದರು. ಬುದ್ಧ ಹಿಂದೆ ತಾನು ಬೆಳಕನ್ನು ಕಂಡಮೇಲೆ ಕಾಶಿಗೆ ಸಮೀಪದಲ್ಲಿರುವ ಸಾರಾನಾಥದಲ್ಲಿ ಮೊದಲನೆಯ ಬೋಧನೆಯನ್ನು ಮಾಡಿದನು. ಕಾಶಿಯ ನಗರದಲ್ಲಿ ಎಲ್ಲಾ ಹಿಂದೂಗಳಿಗೂ ಪರಮ ಪವಿತ್ರವಾದ ಗಂಗಾನದಿ ಹರಿಯುತ್ತಿದೆ. ಭರತಖಂಡದ ಮೂಲೆ ಮೂಲೆಯಿಂದ ವರ್ಷಾದ್ಯಂತವೂ ಯಾತ್ರಿಕರು ಇಲ್ಲಿಗೆ ಬರುತ್ತಿರುವರು. ಭರತಖಂಡದ ಜನರನ್ನು ಈ ಒಂದು ತೀರ್ಥಕ್ಷೇತ್ರ ಒಂದುಗೂಡಿಸುವಂತೆ ಮತ್ತಾವ ತೀರ್ಥಕ್ಷೇತ್ರವೂ ಮಾಡಿಲ್ಲ ಎಂದು ಹೇಳಬಹುದು. ವಿವೇಕಾನಂದರ ಜೀವನ ದೃಷ್ಟಿಯಿಂದಲೂ ಈ ಸ್ಥಳ ಪವಿತ್ರವಾಗಿದೆ. ಇವರ ತಾಯಿ ಕಾಶಿಯಲ್ಲಿರುವ ವೀರೇಶ್ವರನಿಗೆ ಹರಕೆ ಹೊತ್ತ ಫಲವಾಗಿ ಅವನ ಶಕ್ತಿಯ ಅಂಶದಿಂದ ನರೇಂದ್ರ ಜನಿಸಿದ. ಬೀದಿಬೀದಿಗಳಲ್ಲಿ ಅಗಣಿತ ದೇವಾಲಯಗಳಿಂದ ತುಂಬಿತುಳುಕಾಡುತ್ತಿದೆ ಕಾಶಿ. ಕಾಶಿಯಲ್ಲಿ ವಿಶ್ವೇಶ್ವರದೇವಸ್ಥಾನ ಅತಿಮುಖ್ಯವಾದರೂ ಆ ನಗರಿ ಅಸಂಖ್ಯಾತ ದೇವಾಲಯಗಳಿಂದ ತುಂಬಿಹೋಗಿದೆ. ಆಧ್ಯಾತ್ಮಿಕ ವಾತವರಣದಿಂದ ತುಂಬಿ ತುಳುಕಾಡುತ್ತಿದೆ ಕಾಶಿ. ವಿವೇಕಾನಂದರು ಆ ವಾತಾವರಣದಲ್ಲಿ ಬಾಳಿದರು. 

 ದ್ವಾರಕದಾಸರು ವಿವೇಕಾನಂದರನ್ನು ಭೂದೇವಚಂದ್ರ ಮುಖ್ಯೋಪಾಧ್ಯಾಯ ಎಂಬ ವಂಗಸಾಹಿತಿ ಮತ್ತು ವಿದ್ಯಾಂಸರಿಗೆ ಪರಿಚಯಮಾಡಿಸಿದರು. ಅವರು ಇವರೊಡನೆ ಮಾತನಾಡಿದಮೇಲೆ “ಈ ಸಣ್ಣ ವಯಸ್ಸಿನಲ್ಲಿ ಎಂತಹ ಅನುಭವ ಮತ್ತು ದಿವ್ಯಶಕ್ತಿ ಇದೆ, ಇದು ಅದ್ಭುತ. ಈತ ಮಹಾಪುರುಷನಾಗುವುದರಲ್ಲಿ ಸಂದೇಹವಿಲ್ಲ” ಎಂದರು. ಆಗ ಕಾಶಿಯಲ್ಲಿ ತ್ರೈಲಿಂಗಸ್ವಾಮಿಗಳು ಎಂಬ ಮಹಾಜ್ಞಾನಿಗಳು ಪ್ರಖ್ಯಾತರಾಗಿದ್ದರು. ಅವರು ಯಾವಾಗಲೂ ಬ್ರಹ್ಮಚಿಂತನೆಯಲ್ಲಿಯೇ ಮಗ್ನರಾಗಿದ್ದರು. ಕಾಲದೇಶನಿಮಿತ್ತದ ಪ್ರಪಂಚ ಅವರ ಪಾಲಿಗೆ ರದ್ದಾಗಿತ್ತು. ಶ‍್ರೀರಾಮಕೃಷ್ಣರು ಕೂಡ ಅವರನ್ನು ತಾವು ಕಾಶಿಗೆ ಹೋಗಿದ್ದಾಗ ನೋಡಿದ್ದರು. ಸ್ವಾಮಿ ವಿವೇಕಾನಂದರು ತ್ರೈಲಿಂಗಸ್ವಾಮಿಯಲ್ಲಿ ಉರಿಯುತ್ತಿದ್ದ ಉಜ್ವಲ ಆಧ್ಯಾತ್ಮಿಕ ವ್ಯಕ್ತಿತ್ವವನ್ನು ನೋಡಿದರು. ತ್ರೈಲಿಂಗಸ್ವಾಮಿಗಳು ಮಾತನಾಡಿ ಜನರನ್ನು ಜಾಗ್ರತರನ್ನಾಗಿ ಮಾಡುವ ಗುಂಪಿಗೆ ಸೇರಿದವರಲ್ಲ. ಅವರ ವ್ಯಕ್ತಿತ್ವವೇ ಒಂದು ಮೌನ ಉಪದೇಶ. ಒಬ್ಬ ಮೌನವಾಗಿ ಭಗವತ್ ಧ್ಯಾನದಲ್ಲಿ ತಲ್ಲೀನನಾಗಿದ್ದರೂ ಅವನ ಭಾವನೆ ಸುತ್ತಮುತ್ತಲಿರುವ ಜೀವಿಗಳ ಮೇಲೆಲ್ಲ ತನ್ನ ಪರಿಣಾಮವನ್ನು ಬೀರಬಲ್ಲದು ಎಂದು ವಿವೇಕಾನಂದರು ಹೇಳುತ್ತಿದ್ದರು.

 ಅನಂತರ ಭಾಸ್ಕರಾನಂದ ಎಂಬ ಬಹಳ ವಿದ್ವಾಂಸರಾದ ಸಂನ್ಯಾಸಿಗಳನ್ನು ನೋಡಿದರು. ಅವರು ವಿವೇಕಾನಂದರೊಡನೆ ಮಾತನಾಡುತ್ತಿದ್ದಾಗ ಕಾಮಕಾಂಚನ ತ್ಯಾಗದ ವಿಷಯ ಬಂದಿತು. ಭಾಸ್ಕರಾನಂದರು ಕಾಮಕಾಂಚನವನ್ನು ಪ್ರಪಂಚಂದಲ್ಲಿ ಯಾರೂ ಸಂಪೂರ್ಣ ತ್ಯಜಿಸಲಾರರು ಎಂದು ಹೇಳಿದರು. ವಿವೇಕಾನಂದರು, “ಹಾಗೆ ಸಂಪೂರ್ಣ ತ್ಯಜಿಸಿದವರು ಹಲವರು ಇರುವರು. ನಾನೇ ಇಂತಹ ಒಬ್ಬರನ್ನು ಕಂಡಿರುವೆನು, ಅವರೇ ನಮ್ಮ ಗುರುಗಳಾದ ಶ‍್ರೀರಾಮಕೃಷ್ಣರು” ಎಂದು ಹೇಳಿದರು. ಆದರೆ ಭಾಸ್ಕರಾನಂದರು ವಿವೇಕಾನಂದರನ್ನು ನಂಬಲಿಲ್ಲ. ಈ ಹುಡುಗ ಸಂನ್ಯಾಸಿಗೆ ಇನ್ನೂ ಅನುಭವ ಸಾಲದು ಎಂದು ವ್ಯಂಗ್ಯವಾದ ನಗೆಯನ್ನು ಬೀರಿದರು. ಶ‍್ರೀರಾಮಕೃಷ್ಣರಂತಹ ಒಂದು ವ್ಯಕ್ತಿಯನ್ನು ನೋಡಿದಲ್ಲದೇ ನಂಬಲು ಸಾಧ್ಯವಿಲ್ಲ. ಏಕೆಂದರೆ ಅಂತಹ ವ್ಯಕ್ತಿ ನಮ್ಮ ಕಲ್ಪನೆಗೂ ನಿಲುಕದವರು. ಇವರ ವಿಷಯವನ್ನು ಆ ವಿದ್ವಾಂಸರು ಹೇಗೆ ಗ್ರಹಿಸಲು ಸಾಧ್ಯ! ವಿವೇಕಾನಂದರು ಇಂತಹವರ ಹತ್ತಿರ ವಾದಮಾಡಿ ಪ್ರಯೋಜನವಿಲ್ಲ ಎಂದು ಅಲ್ಲಿಂದ ಹೊರಟರು. 

 ವಿವೇಕಾನಂದರು ಒಂದು ದಿನ ಕಾಶಿಯ ಬೀದಿಯಲ್ಲಿ ಹೋಗುತ್ತಿದ್ದರು. ಕೆಲವು ಕಪಿಗಳು ಅವರನ್ನು ಅಟ್ಟಿಸಿಕೊಂಡು ಬಂದವು. ಮೊದಲು ಅವುಗಳಿಂದ ತಪ್ಪಿಸಿಕೊಳ್ಳಲು ಓಡತೊಡಗಿದರು. ಯಾರೋ ಒಬ್ಬರು ಒಂದು ಮಹಡಿಯ ಮೇಲೆ ನಿಂತು ಇದನ್ನು ನೋಡುತ್ತಿದ್ದವರು, “ಸ್ವಾಮಿಗಳೇ ಓಡಿಹೋಗಬೇಡಿ, ಕಪಿಗಳನ್ನು ಎದುರಿಸಿ” ಎಂದು ಹೇಳಿದರು. ವಿವೇಕಾನಂದರು ಆತನ ಬುದ್ಧಿವಾದದಂತೆ ಹಿಂತಿರುಗಿ ಕಪಿಗಳನ್ನು ಧೈರ‍್ಯವಾಗಿ ಎದುರಿಸಿ ನಿಂತರೋ ಇಲ್ಲವೋ ಕಪಿಗಳೆಲ್ಲ ಪಲಾಯನ ಮಾಡಿದವು. ವಿವೇಕಾನಂದರು ಆ ಒಂದು ಘಟನೆಯ ಉದಾಹರಣೆಯನ್ನು ಅನೇಕ ವೇಳೆ ತಮ್ಮ ಉಪನ್ಯಾಸಗಳಲ್ಲಿ ಕೊಡುತ್ತಿದ್ದರು. ಪ್ರಕೃತಿಯನ್ನು ಎದುರಿಸಬೇಕು, ಅಜ್ಞಾನವನ್ನು ಎದುರಿಸಬೇಕು, ಭ್ರಾಂತಿಯನ್ನು ಎದುರಿಸಬೇಕು. ಆಗಲೇ ಅವು ತಮ್ಮ ಹಿಂದಿರುವ ಸತ್ಯವನ್ನು ನಮಗೆ ತೋರಿಸುತ್ತವೆ. ಅಂಜುಕುಳಿಗೆ ಇಹವೂ ಇಲ್ಲ, ಪರವೂ ಇಲ್ಲ. ಧೀರರಿಗೆ ಮಾತ್ರ ಪ್ರಪಂಚ ತನ್ನ ರಹಸ್ಯವನ್ನು ಹೊರಪಡಿಸುತ್ತದೆ. ಸ್ವಾಮಿ ವಿವೇಕಾನಂದರಿಗೆ ಕೇವಲ ಶಾಸ್ತ್ರಗಳು ಮಾತ್ರ ಮಾತನಾಡುತ್ತಿರಲಿಲ್ಲ; ಅವರಿಗೆ ಪ್ರತಿಯೊಂದು ವಸ್ತು, ಪ್ರತಿಯೊಂದು ಘಟನೆಯೂ, ಜೀವನದ ಆಳದಲ್ಲಿ ಹುದುಗಿರುವ ರಹಸ್ಯವನ್ನು ಸಾರುತ್ತಿದ್ದವು. 

 ವಿವೇಕಾನಂದರು ಅನಂತರ ತಮ್ಮ ಗುರುಭಾಯಿಗಳೊಡನೆ ಬಾರಾನಗರ ಮಠಕ್ಕೆ ಬಂದರು. ಧ್ಯಾನ, ಅಧ್ಯಯನ ಪ್ರವಚನಗಳಲ್ಲಿ ಕಾಲವನ್ನು ಕಳೆಯತೊಡಗಿದರು. ವೇದಾಂತದ ಸತ್ಯವನ್ನು ಎಲ್ಲರಿಗೂ ಬೋಧಿಸಬೇಕೆಂಬುದು ಅವರಿಗೆ ಅಂದಿನಿಂದಲೂ ಇತ್ತು. ಅನೇಕ ವೇಳೆ ತಮ್ಮ ಗುರುಭಾಯಿಗಳಿಗೆ ಇತರರಿಗೆ ಆ ವಿಷಯವನ್ನು ಬೋಧಿಸಬೇಕು ಎನ್ನುತ್ತಿದ್ದರು. ಇದರಿಂದ ನಾವು ತಿಳಿದುಕೊಂಡಿರುವುದು ಸ್ಪಷ್ಟವಾಗುವುದು. ಈ ಪ್ರಪಂಚದಲ್ಲಿ ಸರ್ವವೂ ಏನನ್ನೋ ಸಾರುತ್ತಿವೆ. ಕೆಲವು ಮೌನವಾಗಿ ಮತ್ತೆ ಕೆಲವು ಶಬ್ಧದ ಮೂಲಕ. ಗಿಡಮರಗಳು, ಗಿರಿತೊರೆಗಳು, ಪಶುಪಕ್ಷಿಗಳು ಎಲ್ಲವೂ ಭಾವನೆಯನ್ನು ವ್ಯಕ್ತಗೊಳಿಸುತ್ತಿವೆ. ಅವು ಯಾವುದನ್ನು ತಮಗೆ ತಿಳಿಯದೆ ಮಾಡುತ್ತಿವೆಯೋ ನಾವು ಅದನ್ನು ತಿಳಿದು ಮಾಡಬೇಕು ಎಂದು ತಮ್ಮ ಗುರುಭಾಯಿಗಳಿಗೆ ಹೇಳುತ್ತಿದ್ದರು. ಆದರೆ ನರೇಂದ್ರನ ಮಾತಿಗೆ ಅವರು ಸುಲಭವಾಗಿ ವಶರಾಗುತ್ತಿರಲಿಲ್ಲ. ಅವರು ಶ‍್ರೀರಾಮಕೃಷ್ಣರನ್ನು ಉದಾಹರಿಸುತ್ತಿದ್ದರು. ಶ‍್ರೀರಾಮಕೃಷ್ಣರು ಮುಂಚೆ ದೇವರ ಸಾಕ್ಷಾತ್ಕಾರವನ್ನು ನೀವು ಪಡೆಯಿರಿ, ಅನಂತರ ಅದನ್ನು ಇತರರಿಗೆ ಬೋಧಿಸಲು ಹೋಗಿ ಎನ್ನುತ್ತಿದ್ದರು. ಆದರೆ ವಿವೇಕಾನಂದರು ಮತ್ತೊಂದು ದೃಷ್ಟಿಯಿಂದ ಇದನ್ನು ನೋಡುತ್ತಿದ್ದರು. ನಾವು ಎಷ್ಟು ಹೊತ್ತು ಧ್ಯಾನ ಜಪಗಳನ್ನು ಮಾಡಲು ಸಾಧ? ಮಿಕ್ಕ ಕಾಲವನ್ನು ಶಾಸ್ತ್ರಚಿಂತನೆ, ಚರ್ಚೆ ಬೋಧನೆಗಳಿಗೆ ಉಪಯೋಗಿಸಿದರೆ ಅವು ನಮ್ಮ ಸಾಕ್ಷಾತ್ಕಾರಕ್ಕೆ ಸಹಾಯ ಮಾಡುವುವು, ಏನೂ ತಿಳಿಯದವನಿಗೆ ನಾವು ಸ್ವಲ್ಪ ವಿಷಯಗಳನ್ನು ಹೇಳಿದಂತೆ ಆಗುವುದು ಮತ್ತು ಹಾಗೆ ಮಾಡುವ ಪ್ರಯತ್ನದಿಂದ ನಮ್ಮ ತಿಳುವಳಿಕೆಯೂ ಸುಸ್ಪಷ್ಟವಾಗುವುದು. ಉಪನಿಷತ್ತಿನ ಕಾಲದಿಂದಲೇ ಅಧ್ಯಯನ ಪ್ರವಚನಗಳನ್ನು ಎಂದಿಗೂ ಬಿಡಬೇಡಿ ಎಂದು ಹೇಳುವುದು ರೂಢಿಯಲ್ಲಿತ್ತು. 

 ವಿವೇಕಾನಂದರು ಕೆಲವು ಕಾಲ ಬಾರಾನಗರ ಮಠದಲ್ಲಿದ್ದು ಅಲ್ಲಿಂದ ಪುನಃ ಒಬ್ಬರೇ ಕಾಶಿಗೆ ಯಾತ್ರೆ ಹೊರಟರು. ಅಲ್ಲಿ ಪ್ರಮದದಾಸಮಿತ್ರ ಎಂಬ ಪ್ರಖ್ಯಾತರಾದ ಸಂಸ್ಕೃತ ವಿದ್ವಾಂಸರ ಪರಿಚಯವಾಯಿತು. ಅನಂತರ ಅವರಿಬ್ಬರೂ ದೊಡ್ಡ ಸ್ನೇಹಿತರಾದರು. ವಿವೇಕಾನಂದರು ಹಲವು ವೇಳೆ ಶಾಸ್ತ್ರದ ಜಟಿಲ ಸಮಸ್ಯೆಗಳಿಗೆ ಅವರಿಂದ ಪರಿಹಾರವನ್ನು ಪಡೆಯಲು ಕಾಗದಗಳನ್ನು ಬರೆಯುತ್ತಿದ್ದರು. 

 ಅನಂತರ ಅಯೋಧ್ಯೆಗೆ ಹೋದರು. ಮೊದಲಿನಿಂದಲೂ ಸ್ವಾಮೀಜಿಯವರಿಗೆ ರಾಮಾಯಣವು ಪ್ರಿಯವಾದ ಗ್ರಂಥವಾಗಿತ್ತು. ಶ‍್ರೀರಾಮಕೃಷ್ಣರೇ ಅವರಿಗೆ ರಾಮ ಮಂತ್ರವನ್ನು ಕೊಟ್ಟಿದ್ದರು. ಶ‍್ರೀರಾಮ ಜನಿಸಿದ ಊರು, ಅವನು ಆಳಿದ ಊರು, ಅವನು ಪ್ರಪಂಚದಿಂದ ಕಣ್ಮರೆಯಾದ ಊರು ಅಯೋಧ್ಯೆ. ಕೆಲವು ಕಾಲ ಅಲ್ಲಿದ್ದು ಅಲ್ಲಿಂದ ಲಕ್ನೋ ನಗರಕ್ಕೆ ಹೋದರು. ಆ ಊರಿನಲ್ಲಿ ಹಿಂದಿನ ಅರಸರುಗಳು ಬಿಟ್ಟುಹೋದ ಸುಂದರವಾದ ಅರಮನೆಗಳು, ಉದ್ಯಾನವನಗಳು, ಮಸೀದಿಗಳು ಎಲ್ಲವನ್ನೂ ನೋಡಿದರು. ಅನಂತರ ವಿಶ್ವದಲ್ಲೆಲ್ಲ ಪ್ರಸಿದ್ಧವಾದ ತಾಜ್‍ಮಹಲ್ ನೋಡಲು ಆಗ್ರ ನಗರಕ್ಕೆ ಬಂದರು. ತಾಜ್‍ಮಹಲ್ಲಿನ ಸೌಂದರ‍್ಯಕ್ಕೆ ಪರವಶರಾದರು. ಅದನ್ನು ಬೇರೆ ಬೇರೆ ದೃಷ್ಟಿಕೋನಗಳಿಂದ ನೋಡಿದರು. ಅನಂತರ, ‘ತಾಜ್‍ಮಹಲ್ಲಿನ ಒಂದು ಚದುರ ಅಂಗುಲವನ್ನೂ ಚೆನ್ನಾಗಿ ನೋಡಬೇಕಾದರೆ ಇಡೀ ದಿನವೆಲ್ಲ ಹಿಡಿಯುವುದು, ಇಡೀ ತಾಜ್‍ಮಹಲ್ಲನ್ನು ಚೆನ್ನಾಗಿ ತಿಳಿದುಕೊಳ್ಳಬೇಕಾದರೆ ಆರು ತಿಂಗಳುಗಳು ಹಿಡಿಯುವುದು’ ಎಂದು ಹೇಳುತ್ತಿದ್ದರು. ವಿವೇಕಾನಂದರು ಗೊಡ್ಡು ಸಂನ್ಯಾಸಿಗಳಲ್ಲ. ಅವರ ಸಂನ್ಯಾಸದ ಹಿಂದೆ ಒಬ್ಬ ದೊಡ್ಡ ಕಲೋಪಾಸಕರನ್ನು ನೋಡಬಹುದು. ಸುಂದರವಾಗಿರುವ ದೃಶ್ಯವಾಗಲಿ, ಕಟ್ಟಡವಾಗಲಿ, ಭಾವವಾಗಲಿ, ನಾದವಾಗಲಿ ಎಲ್ಲಾ ಪರಬ್ರಹ್ಮನೆಡೆಗೆ ಒಯ್ಯುವ ಸೋಪಾನ ಪಂಕ್ತಿಗಳಾಗಿದ್ದವು ಅವರಿಗೆ. ಅವರು ಮಹಮ್ಮದೀಯರ ಆಳ್ವಿಕೆಯನ್ನೆಲ್ಲ ದೂರಲಿಲ್ಲ. ಚಾರಿತ್ರಿಕ ದೃಷ್ಟಿಯಿಂದ ಅವರಿಗೆ ಹಿಂದೂದೇಶದಲ್ಲಿ ಒಂದು ಸ್ಥಾನವಿದೆ. ಹಿಂದೂಗಳ ವಾಸ್ತುಶಿಲ್ಪ, ಸಂಗೀತ, ಸಾಹಿತ್ಯದ ಮೇಲೆ ಮುಸ್ಲಿಮರು ತಮ್ಮ ಪ್ರಭಾವವನ್ನು ಬೀರಿರುವರು. ಅವರ ಆಳ್ವಿಕೆ ನಾಮಾವಶೇಷವಾಗಿ ಹೋದರೂ ಹಲವು ಸುಂದರವಾದ ಕಟ್ಟಡಗಳು, ಮಸೀದಿಗಳು, ಭರತಖಂಡದಲ್ಲಿ ಈಗಲೂ ಶೋಭಿಸುತ್ತಿವೆ. ವಿವೇಕಾನಂದರು ಪ್ರತಿಯೊಂದನ್ನೂ ಉದಾರವಾದ ದೀರ್ಘದೃಷ್ಟಿಯಿಂದ ನೋಡುತ್ತಿದ್ದರು. 

 ಆಗ್ರಾದಿಂದ ಬೃಂದಾವನಕ್ಕೆ ಹೊರಟರು. ಕೊನೆಯ ಮೂವತ್ತು ಮೈಲಿ ಪಾದಚಾರಿಗಳಾಗಿಯೇ ಹೋದರು. ಇನ್ನೇನು ಬೃಂದಾವನಕ್ಕೆ ಎರಡು ಮೈಲಿಗಳಿವೆ ಎನ್ನುವಾಗ ದಾರಿಯ ಪಕ್ಕದಲ್ಲಿ ಒಂದು ಮರದ ಕೆಳಗೆ ಒಬ್ಬ ಗುಡಿಗುಡಿಯನ್ನು ಸೇದುತ್ತಿದ್ದುದನ್ನು ನೋಡಿದರು. ಸ್ವಾಮಿಗಳು ಅವನ ಹತ್ತಿರ ಹೋಗಿ ತಮಗೆ ಸ್ವಲ್ಪ ಸೇದುವುದಕ್ಕೆ ಕೊಡು ಎಂದು ಕೇಳಿದರು. ಆತ ಕೇಳುತ್ತಿರುವ ಸ್ವಾಮಿಗಳನ್ನು ನೋಡಿ “ನಾನು ಭಂಗಿ, ಕಸಗುಡಿಸುವ ಜಾತಿಗೆ ಸೇರಿದವನು” ಎಂದನು. ಆಗ ಸ್ವಾಮಿಗಳು ಅವನಿಂದ ತೆಗೆದುಕೊಳ್ಳದೆ ಹಾಗೆಯೇ ಮುಂದೆ ನಡೆದುಕೊಂಡು ಹೋಗುತ್ತಿರುವಾಗ, ತಾವು ಮನೆ ಮಠಗಳನ್ನೆಲ್ಲ ಬಿಟ್ಟು, ಜಾತಿಕುಲಗಳನ್ನು ತ್ಯಜಿಸಿ ಸಂನ್ಯಾಸಿಯಾಗಿದ್ದರೂ, ಆತ ತಾನು ಭಂಗಿ ಕುಲಕ್ಕೆ ಸೇರಿದವನು ಎಂಬುದನ್ನು ಕೇಳಿದ ಕೂಡಲೇ ಅವನಿಂದ ಸ್ವೀಕರಿಸದೆ ಹೋದುದು ಸರಿಯೆ ಎಂದು ತಮ್ಮ ಮನಸ್ಸಿನಲ್ಲಿಯೇ ತರ್ಕಿಸತೊಡಗಿದರು. ಹಿಂದಿನಿಂದ ಬಂದ ಆಚಾರ ಎಷ್ಟು ಆಳಕ್ಕೆ ಹೋಗಿರುವುದು! ಅರಿವಿಲ್ಲದೆ ಅದು ನಮ್ಮ ಮೇಲೆ ತನ್ನ ಪ್ರಭಾವವನ್ನು ಬೀರುತ್ತದೆ. ಎಲ್ಲಬಿಟ್ಟ ಸಂನ್ಯಾಸಿಗಳಲ್ಲಿಯೇ ಇದು ಹೀಗಿದ್ದರೆ ಮಿಕ್ಕವರಲ್ಲಿ ಅದು ಇನ್ನೂ ಎಷ್ಟು ಆಳಕ್ಕೆ ಹೋಗಿರುವುದು ಎಂದು ಚಿಂತಿಸತೊಡಗಿದರು. ಅವರು ಪುನಃ ಹಿಂದಕ್ಕೆ ನಡೆದು ಬಂದು ಭಂಗಿಯವನಿಗೆ ಗುಡುಗುಡಿಯನ್ನು ಮಾಡಿಕೊಡೆಂದು ಕೇಳಿದರು. ಆತ ಅದಕ್ಕೆ ಎಷ್ಟೂ ಒಪ್ಪದೇ ಇದ್ದರೂ ಬಲಾತ್ಕಾರದಿಂದ ಅವನಿಂದ ಮಾಡಿಸಿ ಅದನ್ನು ಸೇವಿಸಿ ಮುಂದೆಹೋದರು. ಆನಂತರ ಸ್ವಾಮೀಜಿ ಹೇಳುತ್ತಿದ್ದರು, ಸಂನ್ಯಾಸಿ ಯಾವಾಗಲೂ ತನ್ನ ನಡತೆಯನ್ನು ಒರೆಗಲ್ಲಿಗೆ ತಿಕ್ಕಿ ನೋಡುತ್ತಿರಬೇಕು ಎಂದು. ಸ್ವಾಮಿ ವಿವೇಕಾನಂದರು ೧೮೮೮ ಆಗಸ್ಟ್ ಪ್ರಾರಂಭದಲ್ಲಿ ಬೃಂದಾವನಕ್ಕೆ ಬಂದರು. ಅಲ್ಲಿ ಶ‍್ರೀರಾಮಕೃಷ್ಣರ ಗೃಹಸ್ಥಭಕ್ತನಾದ ಬಲರಾಮ ಬಾಬುಗಳ ಪೂರ್ವಿಕರಿಗೆ ಸೇರಿದ ಕಾಳಿಬಾಬುಕುಂಜ ಎಂಬಲ್ಲಿ ತಂಗಿದರು. ಶ‍್ರೀಕೃಷ್ಣ ಮತ್ತು ರಾಧೆಯ ಜೀವನದಲ್ಲಿ ಸ್ಮೃತಿಗಳಿಂದ ಆಚ್ಛಾಧಿತವಾದ ಬೃಂದಾವನ, ಸ್ವಾಮೀಜಿಯವರ ಮನಸ್ಸಿಗೆ ಮಹಾನಂದವನ್ನು ಕೊಟ್ಟಿತು. ಭಾಗವತವನ್ನು ಬಾಲ್ಯದಿಂದಲೂ ಓದಿದ್ದ ಅವರಿಗೆ ಶ‍್ರೀಕೃಷ್ಣ ಬೃಂದಾವನದಲ್ಲಿ ಮಾಡಿದ ಪ್ರತಿಯೊಂದು ಲೀಲೆಯ ಸ್ಥಳವು ಕೂಡ ಪವಿತ್ರ. ಅವನ ಎಲ್ಲ ಲೀಲಾಸ್ಥಾನಗಳನ್ನೂ ಒಂದೊಂದಾಗಿ ನೋಡುತ್ತ ಹೋದರು. ಶ‍್ರೀಕೃಷ್ಣ ಎತ್ತಿದ ಗೋವರ್ಧನ ಬೆಟ್ಟ ಇರುವುದು ಇಲ್ಲಿಯೇ. ಬೃಂದಾವನಕ್ಕೆ ಬರುವ ಯಾತ್ರಿಕರು ಗೋವರ್ಧನ ಬೆಟ್ಟವನ್ನು ಪ್ರದಕ್ಷಿಣೆ ಮಾಡುವುದು ರೂಢಿ. ಸ್ವಾಮೀಜಿ ಅವರು ಪ್ರದಕ್ಟಿಣೆ ಮಾಡುತ್ತಿದ್ದಾಗ ಒಂದು ಪ್ರತಿಜ್ಞೆಯನ್ನು ತೊಟ್ಟರು. ಶ‍್ರೀಕೃಷ್ಣ ಗೀತೆಯಲ್ಲಿ ಯಾರು ಅನನ್ಯಚಿತ್ತರಾಗಿ ತನ್ನನ್ನೇ ಧ್ಯಾನಿಸುವರೋ ಅಂತಹವರ ಯೋಗಕ್ಷೇಮವನ್ನು ತಾನು ನೋಡಿಕೊಳ್ಳುತ್ತೇನೆಂದು ಸಾರಿರುವನು. ಅದು ಎಷ್ಟು ನಿಜವೆನ್ನುವುದನ್ನು ಪರೀಕ್ಷಿಸಬೇಕೆನಿಸಿತು ಅವರಿಗೆ. ತಾವು ಪ್ರದಕ್ಷಿಣೆ ಮಾಡಿಕೊಂಡು ಬರುವಾಗ ತಾವಾಗೇ ಯಾರನ್ನೂ ಆಹಾರ ಕೇಳುವುದಿಲ್ಲ, ಯಾರಾದರೂ ಇವರನ್ನು ಕರೆದು ಕೊಟ್ಟರೆ ಮಾತ್ರ ತೆಗೆದುಕೊಳ್ಳುತ್ತೇನೆಂದು ಶಪಥ ತೊಟ್ಟರು. ಬೆಳಗ್ಗೆ ನಡೆಯಲು ಮೊದಲು ಮಾಡಿದರು. ಮಧ್ಯಾಹ್ನದವರೆಗೂ ನಡೆದರು. ಸಾಕಾಯಿತು. ಜೊತೆಗೆ ಮಳೆ ಬೇರೆ ಬಂತು. ಅದರಲ್ಲಿ ನೆಂದದ್ದೂ ಆಯಿತು. ಹೊಟ್ಟೆ ಹಸಿವು ಜೋರಾಯಿತು. ಆದರೂ ನಿಲ್ಲದೆ ನಡೆದುಕೊಂಡು ಹೋಗುತ್ತಿರುವಾಗ ಯಾರೋ ಒಬ್ಬ ಇವರನ್ನು ಕರೆದನು. ತಿರುಗಿ ನೋಡಿದಾಗ ಅವನು ಕೈಯಲ್ಲಿ ಏನನ್ನೋ ತರುತ್ತಿದ್ದನು. ಸ್ವಾಮಿಗಳು ಬಹುಶಃ ಆತ ಯಾರೋ ತನಗೆ ಪರಿಚಿತನೆಂದು ತನ್ನನ್ನು ತಪ್ಪು ತಿಳಿದುಕೊಂಡಿರಬೇಕೆಂದು ನಿಲ್ಲದೇ ಹೋಗುತ್ತಿದ್ದರು. ಆದರೆ ಆ ಮನುಷ್ಯ ವೇಗವಾಗಿ ಇವರ ಕಡೆ ಚಲಿಸುತ್ತಿರುವುದು ಕಂಡಿತು. ಸ್ವಾಮಿಗಳು ಪರೀಕ್ಷಿಸಬೇಕೆಂದು ಮತ್ತೂ ವೇಗವಾಗಿ ನಡೆಯಲಾರಂಭಿಸಿದರು. ಆದರೆ ಹಿಂದೆ ಬರುತ್ತಿದ್ದವನು ಬೇಗ ಇವರನ್ನು ಸಂಧಿಸಿ ಇವರಿಗೆ ತಂದಿದ್ದ ಆಹಾರವನ್ನು ಬಲಾತ್ಕಾರದಿಂದ ಕೊಟ್ಟು ಅನಂತರ ಯಾವ ಮಾತನ್ನೂ ಆಡದೆ ಹೊರಟುಹೋದನು. ಶ‍್ರೀಕೃಷ್ಣ ಗೀತೆಯಲ್ಲಿ ಹೇಳಿರುವುದು ಎಷ್ಟು ನಿಜ ಎನ್ನುವುದು ಆಗ ಅವರ ಎದೆಗೆ ತಾಕಿದಂತೆ ಆಯಿತು. 

 ಗೋವರ್ಧನದಿಂದ ರಾಧಾಕುಂಡದ ಕಡೆಗೆ ಹೋದರು. ಶ‍್ರೀರಾಧೆಯ ಜೀವನದ ಸ್ಮೃತಿಯಿಂದ ಕೂಡಿದ ಆ ಸ್ಥಳವನ್ನು ವೈಷ್ಣವರೆಲ್ಲರೂ ಪರಮ ಪವಿತ್ರ ಎಂದು ಭಾವಿಸುವರು. ಆಗ ಸ್ವಾಮೀಜಿಯವರ ದೇಹದ ಮೇಲೆ ಒಂದೇ ಒಂದು ವಸ್ತ್ರ ಮಾತ್ರ ಇತ್ತು. ಅದನ್ನು ಯಮುನೆಯಲ್ಲಿ ಒಗೆದು ತೀರದಲ್ಲಿ ಒಣಗಲು ಹಾಕಿ ನದಿಗೆ ಇಳಿದು ಸ್ನಾನ ಮಾಡುತ್ತಿದ್ದರು. ಆಗ ಸ್ವಾಮಿಗಳಿಗೆ ಕಾಣದಂತೆ ಕಪಿಯೊಂದು ಬಂದು ಅದನ್ನು ತೆಗೆದುಕೊಂಡು ಸ್ವಲ್ಪ ದೂರದಲ್ಲಿರುವ ಮರದ ಮೇಲೆ ಕುಳಿತುಕೊಂಡಿತು. ಸ್ವಾಮಿಗಳು ಸ್ನಾನ ಮಾಡಿದ ಮೇಲೆ ಬಂದು ನೋಡಿದಾಗ, ಅವರು ಒಣಗಲು ಹಾಕಿದ್ದ ಬಟ್ಟೆ ಅಲ್ಲಿರಲಿಲ್ಲ. ಸ್ವಲ್ಪ ದೂರದಲ್ಲಿ ಒಂದು ಮರದ ಮೇಲೆ ಕಪಿ ಅದನ್ನು ತೆಗೆದುಕೊಂಡು ಕುಳಿತಿತ್ತು. ಸ್ವಾಮೀಜಿಯವರು ಆ ಬಟ್ಟೆಯನ್ನು ಕಪಿಯ ಕೈಯಿಂದ ಕೆಳಗೆ ಬೀಳಿಸಲು ಏನೇನೋ ಪ್ರಯತ್ನ ಮಾಡಿದರು. ಆದರೆ ಎಲ್ಲಾ ವಿಫಲವಾಯಿತು. ತಮ್ಮ ಹತ್ತಿರ ಇದ್ದ ಒಂದು ತುಂಡು ಬಟ್ಟೆಯೂ ಹೋಯಿತು. ಬಟ್ಟೆ ತನಗೆ ಸಿಕ್ಕದೆ ಇದ್ದರೆ ಹತ್ತಿರ ಇರುವ ಕಾಡಿನ ಒಳಗೆ ಹೋಗಿ ಪ್ರಾಯೋಪವೇಶ ಮಾಡುವೆನೆಂದು ಪ್ರತಿಜ್ಞೆ ಮಾಡಿದರು. ಶ‍್ರೀರಾಧಿಕೆಯ ಮೇಲೆ ಅಸಮಾಧಾನಪಟ್ಟರು. ಶ‍್ರೀಕೃಷ್ಣನ ಮಹಾಭಕ್ತಳವಳು ಎಂದು ಆ ಸ್ಥಳಕ್ಕೆ ಬಂದರೆ, ತಮಗೆ ಇದ್ದ ಏಕಮಾತ್ರ ಆಸ್ತಿಯಾದ ವಸ್ತ್ರವೂ ಹೋಗಬೇಕೆ! ಇಲ್ಲದವರ ಹತ್ತಿರ ದೇವರು ಕಿತ್ತುಕೊಳ್ಳುತ್ತಾನೆಂಬ ನುಡಿ ಸತ್ಯವಾಯಿತು. ಆ ಸಮಯದಲ್ಲಿ ಒಬ್ಬ ಒಂದು ಜೊತೆ ಹೊಸ ಕಾವಿಯ ಬಟ್ಟೆ ಮತ್ತು ಊಟವನ್ನು ಸ್ವಾಮಿಗಳಿಗೆ ತಂದುಕೊಟ್ಟು ಅದನ್ನು ಸ್ವೀಕರಿಸಬೇಕೆಂದು ಬೇಡಿಕೊಂಡ. ಸ್ವಾಮೀಜಿಯವರಿಗೆ “ದೇವರನ್ನು ನೆಚ್ಚಿದವನು ಎಂದಿಗೂ ನಾಶವಾಗುವುದಿಲ್ಲ” ಎಂಬ ಶ‍್ರೀಕೃಷ್ಣನ ಮತ್ತೊಂದು ನುಡಿ ಜ್ಞಾಪಕಕ್ಕೆ ಬಂದಿರಬಹುದು. ಸ್ವಲ್ಪ ಹೊತ್ತಿನ ಮೇಲೆ ಹೋಗಿ ನೋಡಲಾಗಿ ಒಣಗಲು ಹಾಕಿದ್ದ ಸ್ಥಳದಲ್ಲಿ ಕಪಿ ಬಟ್ಟೆಯನ್ನು ಬಿಟ್ಟು ಹೋಗಿತ್ತು. 

 ಅನಂತರ ಸ್ವಾಮೀಜಿ ಹತ್ರಾಸ್ ರೈಲ್ವೆ ನಿಲ್ದಾಣದಲ್ಲಿ ಒಬ್ಬರೇ ಕುಳಿತುಕೊಂಡಿರುವುದನ್ನು ನೋಡುವೆವು. ಸ್ಟೇಷನ್ ಮಾಸ್ಟರ್ ಆದ ಶರತ್‍ಚಂದ್ರಗುಪ್ತ ಎಂಬ ಬಂಗಾಳಿ ಇವರನ್ನು ನೋಡಿದ. ಆತ ಜಾನ್‍ಪುರದ ಮಹಮ್ಮದೀಯರ ಮಧ್ಯದಲ್ಲಿ ಬೆಳೆದವನು. ಹಿಂದಿ ಮತ್ತು ಉರ್ದು ಭಾಷೆಯಲ್ಲಿ ಚೆನ್ನಾಗಿ ಮಾತನಾಡುತ್ತಿದ್ದ. ಆತ ಸ್ವಾಮೀಜಿಯವರ ವರ್ಚಸ್ಸಿಗೆ ಮಾರುಹೋಗಿ, ಅವರ ಬಳಿಗೆ ಹೋಗಿ, ಸ್ವಲ್ಪ ಮಾತುಕತೆ ಆಡಿದ. ಅನಂತರ ಸ್ವಾಮಿಗಳಿಗೆ “ಹಸಿವು ಆಗಿದೆಯೆ” ಎಂದು ಕೇಳಿದ. ಅದಕ್ಕೆ ಸ್ವಾಮೀಜಿ “ಹೌದು” ಎಂದರು. “ಹಾಗಾದರೆ ನಮ್ಮ ಮನೆಗೆ ಬನ್ನಿ” ಎಂದು ಆತ ಪ್ರಾರ್ಥಿಸಿಕೊಂಡನು. ಆಗ ಸ್ವಾಮೀಜಿ ಹುಡುಗನಂತೆ “ನಿನ್ನ ಮನೆಗೆ ಬಂದರೆ ತಿನ್ನುವುದಕ್ಕೆ ಏನನ್ನು ಕೊಡುವೆ?” ಎಂದು ಕೇಳಿದರು. ಅದಕ್ಕೆ ಆತ ಪಾರ್ಸಿ ಭಾಷೆಯಲ್ಲಿ ಒಂದು ಸೂಫಿಗಳ ಕವನವನ್ನು ಉದಹರಿಸಿದ: “ಪ್ರಿಯತಮೆ, ನೀನು ನನ್ನ ಮನೆಗೆ ಬಂದಿರುವೆ. ನನ್ನ ಹೃದಯವನ್ನೇ ಬೇಯಿಸಿ ನಿನಗೆ ಅತ್ಯಂತ ರುಚಿಕರವಾದ ಅಡಿಗೆಯನ್ನು ಮಾಡಿ ಬಡಿಸುವೆ.” ಸ್ವಾಮಿಗಳು ಬಂಗಾಳಿಯಿಂದ ಒಂದು ಹಾಡನ್ನು ಹಾಡಿದರು. ಅದರ ಭಾವ ಹೀಗಿದೆ: “ನನ್ನ ಪ್ರಿಯತಮೆ, ನಿನ್ನ ಚಂದ್ರವದನವನ್ನು ಬೂದಿಯಿಂದ ಲೇಪಿಸಿಕೊಂಡು ಬರಬೇಕು.” ರೈಲ್ವೆ ಮಾಸ್ಟರ್ ಸ್ವಲ್ಪವೂ ಅನುಮಾನಿಸಲಿಲ್ಲ. ತನ್ನ ಬಟ್ಟೆಯನ್ನು ಬಿಚ್ಚಿ ಮುಖಕ್ಕೆ ಬೂದಿಯನ್ನು ಬಳಿದುಕೊಂಡು ಸ್ವಾಮೀಜಿಯ ಎದುರಿಗೆ ಬಂದು ನಿಂತ. 

 ಸ್ವಾಮೀಜಿಯವರು ಆ ಊರಿನಲ್ಲಿ ತಮ್ಮ ಹಳೆಯ ಸ್ನೇಹಿತರಾದ ಬ್ರಜೇನ್‍ಬಾಬು ಇರುವನು ಎಂಬುದನ್ನು ಕೇಳಿದರು. ಊಟವಾದಮೇಲೆ ಅವನ ಮನೆಯನ್ನು ಶರತ್‍ಚಂದ್ರನೊಡನೆ ಹುಡುಕಿಕೊಂಡು ಹೋದರು. ಬ್ರಜೇನ್‍ಬಾಬು ಸ್ವಾಮಿಗಳನ್ನು ನೋಡಿ ಅತ್ಯಾನಂದದಿಂದ ತನ್ನ ಮನೆಯಲ್ಲಿ ಬಂದು ಇರಬೇಕೆಂದು ಕೇಳಿಕೊಂಡನು. ಅಲ್ಲಿ ಕೆಲವು ದಿನಗಳು ಇದ್ದರು. ಆ ಊರಿನ ಪ್ರಮುಖರು ಮತ್ತು ಹಿರಿಯರೆಲ್ಲ ಸ್ವಾಮಿಗಳನ್ನು ನೋಡುವುದಕ್ಕೆ ಬರುತ್ತಿದ್ದರು. ಆಧ್ಯಾತ್ಮಿಕ ಜೀವನದ ಬಗ್ಗೆ ಹಲವು ಪ್ರಶ್ನೆಗಳನ್ನು ಕೇಳುತ್ತಿದ್ದರು. ಅದಕ್ಕೆ ಸಮರ್ಪಕವಾಗಿ ಸ್ವಾಮಿಗಳು ಉತ್ತರವನ್ನು ಕೊಡುತ್ತಿದ್ದರು. ಸಂಜೆಯ ಹೊತ್ತು ಸ್ವಾಮಿಗಳು ತಮ್ಮ ಸುಂದರವಾದ ಕಂಠದಿಂದ ಭಕ್ತಿಯ ಹಾಡುಗಳನ್ನು ಹಾಡುತ್ತಿದ್ದರು. ಅದನ್ನು ಕೇಳುತ್ತಿದ್ದವರಿಗೆ ಇನ್ನೂ ಕೇಳಬೇಕು ಎಂಬ ಕುತೂಹಲ ಹೆಚ್ಚುತ್ತಿತ್ತು. ಸ್ವಾಮೀಜಿ ಬ್ರಜೇನ್‍ಬಾಬುವಿನ ಮನೆಯಲ್ಲಿರುವವರೆಗೆ, ಶರತ್ ಮತ್ತು ನಟಕೃಷ್ಣ ಎಂಬ ಅವನ ಸ್ನೇಹಿತ ಬಂದು ಹೋಗುತ್ತಿದ್ದರು. ಸ್ವಾಮೀಜಿ ಪರಿಚಯವಾದಂತೆ, ಅವರ ಮಾತು ಮತ್ತು ಅವರ ಗಾನವನ್ನು ಕೇಳಿದಂತೆಲ್ಲ ಶರತ್‍ಚಂದ್ರಗುಪ್ತನಿಗೆ ಅವರ ಮೇಲೆ ಭಕ್ತಿ ಹೆಚ್ಚುತ್ತಾ ಹೋಯಿತು. 

 ಒಂದು ದಿನ ಸ್ವಾಮೀಜಿ ಚಿಂತಾಮಗ್ನರಾಗಿದ್ದರು. ಶರತ್ ಅದಕ್ಕೆ ಕಾರಣವನ್ನು ಕೇಳಿದ. ಸ್ವಾಮೀಜಿ ಹೀಗೆಂದರು: “ವತ್ಸ, ನಾನೊಂದು ಮಹಾಕಾರ್ಯವನ್ನು ಮಾಡಬೇಕಾಗಿದೆ. ಆದರೆ ಅದಕ್ಕೆ ನನಗೆ ಇರುವ ಯೋಗ್ಯತೆಯನ್ನು ಚಿಂತಿಸಿದರೆ ನನಗೆ ನಿರಾಶೆಯಾಗುವುದು. ನನ್ನ ಗುರು ಈ ಕಾರ್ಯವನ್ನು ಮಾಡು ಎಂದು ಆಜ್ಞಾಪಿಸಿದರು: ಅದೇ ನಮ್ಮ ಮಾತೃಭೂಮಿಯನ್ನು ಆಮೂಲಾಗ್ರವಾಗಿ ಜಾಗ್ರತೆಗೊಳಿಸುವುದು.. ನಮ್ಮ ದೇಶದಲ್ಲಿ ಆಧ್ಯಾತ್ಮಿಕತೆ ಬಹಳ ಹೀನ ಮಟ್ಟಕ್ಕೆ ಬಂದಿದೆ. ದೇಶದಲ್ಲೆಲ್ಲ ದಾರಿದ್ಯ್ರ ತುಂಬಿ ತುಳುಕಾಡುತ್ತಿದೆ. ಭರತಖಂಡ ಕಾರ‍್ಯೋನ್ಮುಖವಾಗಬೇಕು. ಜಗತ್ತನ್ನು ತನ್ನ ಆಧ್ಯಾತ್ಮದಿಂದ ಗೆಲ್ಲಬೇಕು.” ಶರತ್ ಇದನ್ನು ಕೇಳಿ ವಿಸ್ಮಿತನಾದ, “ಸ್ವಾಮೀಜಿ, ಇಲ್ಲಿರುವವನು ನಿಮ್ಮ ದಾಸ. ಏನು ಕಾರ‍್ಯವನ್ನು ಮಾಡಲಿ ಹೇಳಿ?” ಎಂದ. ಅದಕ್ಕೆ ಸ್ವಾಮೀಜಿ, “ನೀನು ಭಿಕ್ಷಾಪಾತ್ರೆಯನ್ನು ಹಿಡಿದು ಕಮಂಡಲುಧಾರಿಯಾಗಿ ಈ ಕೆಲಸಕ್ಕೆ ನೆರವಾಗಬಲ್ಲೆಯಾ?” ಎಂದು ಕೇಳಿದರು. ಆತ ಧೈರ್ಯದಿಂದ “ಆಗಲಿ” ಎಂದ. ತತ್‍ಕ್ಷಣವೇ ಒಂದು ಭಿಕ್ಷಾಪಾತ್ರೆಯನ್ನು ತೆಗೆದುಕೊಂಡು ರೈಲ್ವೆಯ ಕೂಲಿಗಳ ಮನೆಗಳಿಂದ ಭಿಕ್ಷೆ ಎತ್ತಲು ಹೋದ. 

 ಸ್ವಾಮಿಗಳು ಒಂದು ದಿನ ಹತ್ರಾಸನ್ನು ಬಿಡಲು ಮನಸ್ಸು ಮಾಡಿದರು. ಶರತ್‍ಚಂದ್ರನಿಗೆ “ಸಂನ್ಯಾಸಿ ಒಂದೇ ಸ್ಥಳದಲ್ಲಿ ಬಹಳ ದಿನಗಳು ಇರಕೂಡದು. ತನ್ನ ಸುತ್ತಮುತ್ತಲ ಜನರ ವಿಶ್ವಾಸಗಳಿಂದ ಅವನು ಬದ್ಧನಾಗುತ್ತಾನೆ; ಆದಕಾರಣ ನೀವೂ ಇನ್ನು ನನ್ನನ್ನು ಇಲ್ಲಿಯೇ ಇರಲು ಬಲವಂತ ಮಾಡಬೇಡಿ” ಎಂದರು. ಶರಚ್ಚಂದ್ರನಿಗೆ ಇವರಿಂದ ಅಗಲಬೇಕಲ್ಲ ಎಂದು ಮನಸ್ಸು ವ್ಯಾಕುಲವಾಯಿತು. ಆಗ ತನ್ನನ್ನು ಶಿಷ್ಯನನ್ನಾಗಿ ಸ್ವೀಕರಿಸಿ ಉಪದೇಶವನ್ನು ಕೊಡಬೇಕು ಎಂದು ಬೇಡಿದನು. ಸ್ವಾಮೀಜಿಯವರು “ಬರೀ ಶಿಷ್ಯನಾದರೆ ಬಂದ ಪ್ರಯೋಜನವೇನು? ದೇವರು ಸರ‍್ವಾಂತರ‍್ಯಾಮಿ. ಎಲ್ಲರಲ್ಲಿಯೂ ಇರುವನು. ನಿನ್ನ ಪಾಲಿನ ಕರ್ತವ್ಯವನ್ನು ಅನಾಸಕ್ತಿಯಿಂದ ಮಾಡುತ್ತ ಬಂದರೆ ಅವನ ಸಾಕ್ಷಾತ್ಕಾರ ಪಡೆಯುವೆ” ಎಂದರು. ಆದರೂ ಶರತ್‍ಚಂದ್ರ ಬಿಡಲಿಲ್ಲ. ಸ್ವಾಮಿಗಳು ಅವನಿಗೆ ಉಪದೇಶವನ್ನು ಕೊಟ್ಟು ಅವನನ್ನು ಶಿಷ್ಯನನ್ನಾಗಿ ಸ್ವೀಕರಿಸಿದರು. ಶರತ್ ಮತ್ತಾರಿಗೋ ಕೆಲಸದ ಜವಾಬ್ದಾರಿಯುನ್ನು ವಹಿಸಿ ವಿವೇಕಾನಂದರೊಡನೆ ಹೃಷೀಕೇಶಕ್ಕೆ ಹೊರಟ. ಆದರೆ ಆ ಮನುಷ್ಯನಿಗೆ ಕಷ್ಟದ ಜೀವನ ಹೊಸದು. ಅದು ಹಿಮಾಲಯ ತಪ್ಪಲ ಪ್ರದೇಶ. ಒಂದು ಸಲ ನಡೆದು ಸಾಕಾಗಿ ಶರತ್ ಹಸಿವು ಮತ್ತು ಬಾಯಾರಿಕೆಯನ್ನು ತಡೆಯಲಾರದೆ ಮೂರ್ಛೆಗೊಂಡ. ಸ್ವಾಮೀಜಿ ಅವನನ್ನು ನೋಡಿಕೊಳ್ಳಬೇಕಾಗಿ ಬಂತು. ಶಿಷ್ಯ ಗುರುವಿಗೆ ಸೇವೆ ಮಾಡುವ ಬದಲು ಗುರುವೆ ಶಿಷ್ಯನ ಸೇವೆ ಮಾಡುವ ಸ್ಥಿತಿಗೆ ಬಂದಿತು. ಅನೇಕ ವೇಳೆ ಶರತ್‍ಚಂದ್ರನಿಗಾಗಿ ಸ್ವಾಮಿಗಳು ತುಂಬಾ ಕಷ್ಟಕ್ಕೆ ತುತ್ತಾಗಬೇಕಾಗುತ್ತಿತ್ತು. ಅವನಿಗೆ ನಡೆಯಲು ನಿತ್ರಾಣವಾದಾಗ ಅವನ ಸಾಮಾನನ್ನೆಲ್ಲ ಸ್ವಾಮಿಗಳೇ ಹೊತ್ತರು. ತನ್ನ ಪಾದರಕ್ಷೆಗಳನ್ನು ಕೂಡ ಅವರು ಹೊತ್ತರು ಎಂದು ಶರತ್ ಅನಂತರ ಹೇಳುತ್ತಿದ್ದ. ಶರತ್‍ಗೆ ಅನೇಕ ವೇಳೆ ತಾನು ಸ್ವಾಮಿಗಳಿಗೆ ಒಂದು ಭಾರವಾಗಿ ಪರಿಣಮಿಸಿರುವೆ ಎಂಬ ಭಾವ ಬಾಧಿಸುತ್ತಿತ್ತು. ಆಗ ಆತ ಖಿನ್ನನಾಗಿ “ಏನು ನನ್ನನ್ನು ಕೈಬಿಡುವಿರಾ?” ಎಂದು ಕೇಳಿದಾಗ, ಸ್ವಾಮೀಜಿಯವರು ನಗುತ್ತ ಹೀಗೆ ಹೇಳುತ್ತಿದ್ದರು: “ಮೂರ್ಖ, ನಿನ್ನ ಪಾದರಕ್ಷೆಗಳನ್ನು ಕೂಡ ನಾನು ಹೊತ್ತಿರುವೆನೆಂಬುದು ನಿನಗೆ ಗೊತ್ತಿಲ್ಲವೆ?” 

 ಒಂದು ಸಲ ಗುರು-ಶಿಷ್ಯರಿಬ್ಬರೂ ಕಾಡಿನ ದಾರಿಯಲ್ಲಿ ಹೋಗುತ್ತಿದ್ದಾಗ, ಮನುಷ್ಯನ ಮೂಳೆ ಮತ್ತು ಕಾವಿಯ ತುಂಡುಗಳನ್ನು ನೋಡಿದರು. ಸ್ವಾಮೀಜಿ ಶರತ್‍ಗೆ ಅದನ್ನು ತೋರಿ “ನೋಡು, ಹುಲಿಯೊಂದು ಸಾಧುವನ್ನು ತಿಂದು ಹಾಕಿದೆ. ನಿನಗೆ ಅಂಜಿಕೆಯೆ?” ಎಂದು ಕೇಳಿದರು. ಶಿಷ್ಯ ತತ್‍ಕ್ಷಣವೇ “ನೀವು ಜೊತೆಯಲ್ಲಿದ್ದರೆ, ಅಂಜಿಕೆಯಿಲ್ಲ” ಎಂದ. ಸ್ವಾಮೀಜಿ ಸಮೀಪದಲ್ಲಿದ್ದರೆ ಮೃತ್ಯುಭಯ ವ್ಯಾಘ್ರಭಯಗಳೆಲ್ಲ ಪಲಾಯನ ಮಡುತ್ತಿದ್ದುವು. 

 ಸ್ವಾಮೀಜಿ ಶರತ್ ಒಂದಿಗೆ ಹೃಷೀಕೇಶದಲ್ಲಿದ್ದರು. ಗಂಗಾನದಿ ತೀರ, ಹಿಮಾಲಯದ ಪವಿತ್ರ ವಾತಾವರಣ, ಅನೇಕ ಸಾಧು ಸಂತರ ಸಾನ್ನಿಧ್ಯ ಸ್ವಾಮೀಜಿಗೆ ಬಹಳ ಪ್ರಿಯವಾಗಿದ್ದವು. ಸಾಧನ, ಭಜನೆಯಲ್ಲಿ ಅಲ್ಲಿ ಕಾಲವನ್ನು ಕಳೆದರು. ಅನಂತರ ತಾವು ಕೇದಾರನಾಥ ಮತ್ತು ಬದರಿಗೆ ಹೋಗಬೇಕೆಂದು ಇದ್ದರು. ಆದರೆ ಶರತ್ ಖಾಯಿಲೆ ಬಿದ್ದನು. ಅವನನ್ನು ನೋಡಿಕೊಳ್ಳುವವರು ಯಾರೂ ಇಲ್ಲದೆ ಪುನಃ ಅವನನ್ನು ಹತ್ರಾಸಿಗೆ ಕರೆದುಕೊಂಡುಹೋಗಿ ಬಿಟ್ಟರು. ಅಲ್ಲಿ ಶರತ್ ಗುಣಮುಖವಾಗುವುದರೊಳಗಾಗಿ ಸ್ವಾಮೀಜಿಯೂ ಖಾಯಿಲೆ ಬಿದ್ದರು. ಬಾರಾನಗರ ಮಠದಿಂದ ವಿವೇಕಾನಂದರಿಗೆ ಪದೇ ಪದೇ ಕಲ್ಕತ್ತೆಗೆ ಬರಬೇಕೆಂದು ಕಾಗದಗಳನ್ನು ಬರೆಯುತ್ತಿದ್ದರು. ಸ್ವಾಮೀಜಿ ೧೮೮೯ರ ಬೇಸಿಗೆಯ ಹೊತ್ತಿಗೆ ಬಾರಾನಗರ ಮಠವನ್ನು ಸೇರಿದರು. 

 ಸ್ವಾಮೀಜಿ ಬಾರಾನಗರ ಮಠಕ್ಕೆ ಈ ಸಲ ಬಂದವರು ಸುಮಾರು ಒಂದು ವರ್ಷ ಕಾಲದವರೆಗೆ ಇಲ್ಲೆಯೇ ಇದ್ದರು. ಇಲ್ಲಿ ಧ್ಯಾನ ಅಧ್ಯಯನಗಳನ್ನು ಮುಂದುವರಿಸಿ ತಮ್ಮ ಗುರುಭಾಯಿಗಳಿಗೆ ಹಿಂದೂಶಾಸ್ತ್ರಗಳನ್ನು ಓದುವಂತೆ ಹೇಳುತ್ತಿದ್ದುದಲ್ಲದೆ ಅವರೇ ಕೆಲವು ಪ್ರವಚನಗಳನ್ನು ತೆಗೆದುಕೊಳ್ಳುತ್ತಿದ್ದರು. ಪಾಣಿನಿಯ ವ್ಯಾಕರಣ ಮುಂತಾದ ಗ್ರಂಥಗಳನ್ನು ಕಾಶಿಯ ಪಂಡಿತರಾದ ಪ್ರಮದದಾಸಬಾಬು ಅವರಿಂದ ತರಿಸಿ ತಮ್ಮ ಗುರುಭಾಯಿಗಳಿಗೆ ಅದನ್ನು ಓದುವಂತೆ ಉತ್ಸಾಹವನ್ನು ತುಂಬುತ್ತಿದ್ದರು. ಇದನ್ನು ತಿಳಿದಕೊಂಡರೇನೆ ವೇದಗಳ ಸಂಹಿತಾ ಭಾಗವನ್ನು ತಿಳಿದುಕೊಳ್ಳಲು ಸಾಧ್ಯ. ಇದರಂತೆಯೇ ಹಿಂದೂಧರ್ಮವನ್ನು ಪೂರ್ಣವಾಗಿ ತಾತ್ತ್ವಿಕವಾಗಿ ನೋಡುವುದನ್ನು ಗುರುಭಾಯಿಗಳಿಗೆ ಪರಿಚಯ ಮಾಡಿಸಿದರು. ಶ‍್ರೀರಾಮಕೃಷ್ಣರ ಜೀವನದಲ್ಲಿ ಹಿಂದೂಧರ್ಮದ ಪೂರ್ಣ ಪ್ರತಿನಿಧಿಯನ್ನು ಅವರೆಲ್ಲ ತಮ್ಮ ಕಣ್ಣು ಮುಂದೆ ನೋಡಿದ್ದರು. ಆದರೆ ಎಲ್ಲರಿಗೂ ತಿಳಿಯುವಂತೆ ಹೇಳಬೇಕಾದರೆ ಅದನ್ನು ಬೌದ್ಧಿಕವಾಗಿ ಅರ್ಥಮಾಡಿಕೊಂಡಿದ್ದರೆ ಮಾತ್ರ ಸಾಧ್ಯ. 

 ಹಿಂದೂಧರ್ಮವನ್ನು ಹೊರಗಿನಿಂದ ನೋಡಿದರೆ ಅಲ್ಲಿ ಬೇಕಾದಷ್ಟು ವಿರೋಧಾಬಾಸಗಳು ಕಾಣುವುವು. ಸತ್ಯ ಒಂದೇ. ಅದನ್ನು ಒಬ್ಬೊಬ್ಬರು ಒಂದೊಂದು ರೀತಿ ಹೇಳುವುದರ ಜೊತೆಗೆ ಪ್ರತಿಯೊಬ್ಬ ಸಿದ್ಧಾಂತಕಾರನೂ ತನ್ನದೇ ಸತ್ಯವೆಂದು ಒತ್ತಿ ಹೇಳುವುದನ್ನು ನೋಡುತ್ತೇವೆ. ಬ್ರಹ್ಮ ಎಲ್ಲರಲ್ಲಿಯೂ ಇರುವನು ಎಂಬುವರು, ಆದರೆ ಶೂದ್ರನಿಗೆ ವೇದಾಧ್ಯಯನಕ್ಕೆ ಅಧಿಕಾರವಿಲ್ಲವೆನ್ನುವರು. ಯಾವ ಜಾತಿ ಬಹಳ ಹಿಂದೆ ಕೇವಲ ವೃತ್ತಿಯನ್ನು ಅನುಸರಿಸಿತ್ತೋ ಅದು ಕೇವಲ ಜನ್ಮದ ಮೇಲೆಯೇ ಈಗ ನಿಂತಿರುವಂತೆ ಕಾಣುತ್ತಿದೆ. ಈ ವಿಷಯಗಳ ಮೇಲೆಲ್ಲ ಪ್ರಮದಬಾಬುಗಳಿಗೆ ದೀರ್ಘವಾದ ಪತ್ರಗಳನ್ನು ಬರೆದು ತಿಳಿದುಕೊಳ್ಳುವುದಕ್ಕೆ ಸ್ವಾಮೀಜಿ ಪ್ರಯತ್ನಿಸಿದರು. ಜನರಿಗೆ ಧರ‍್ಮ ಎಂದರೆ ಉಪನಿಷತ್ತು ಮತ್ತು ಗೀತಾತತ್ತ್ವಗಳು ಎಂಬ ಅರಿವೇ ಬರುವುದಿಲ್ಲ. ಯಾವುದೋ ಕಂದಾಚಾರ, ಮೂಢನಂಬಿಕೆ ಇವುಗಳೇ ಹಿಂದೂ ಧರ್ಮದ ತತ್ತ್ವದ ವೇಷದಲ್ಲಿ ಮೆರೆಯುತ್ತಿದ್ದವು. ಭರತಖಂಡದ ಉದ್ಧಾರ ನಿಂತಿರುವುದು, ಉಪನಿಷತ್ತು ಗೀತೆ ಬಹ್ಮಸೂತ್ರ ಮುಂತಾದ ಗ್ರಂಥಗಳನ್ನು ಯಾವ ಜಾತಿ ಮತಗಳ ಭೇದಭಾವವಿಲ್ಲದೆ ಎಲ್ಲರಿಗೂ ತಿಳಿಯುವ ರೀತಿ ಕೊಡುವುದರ ಮೇಲೆ ಎಂಬುದನ್ನು ಅವರು ಚೆನ್ನಾಗಿ ಮನಗಂಡರು. ಪರಸ್ಪರ ವಿರೋಧದಿಂದ ಕೂಡಿದ ಭಾವನೆಗಳನ್ನು ಮೀರಿಹೋಗಿ, ಅದರ ಹಿಂದಿರುವ ಏಕ ಅಖಂಡಸೂತ್ರದ ಪರಿಚಯ ಮಾಡಿಕೊಳ್ಳುವುದಕ್ಕೆ ಸ್ವಾಮೀಜಿ ಮನಸ್ಸು ಕಾತರಿಸುತ್ತಿರುವುದನ್ನು ನೋಡುವೆವು. 

 ಕಲ್ಕತ್ತೆಯಲ್ಲಿದ್ದಾಗ ಸ್ವಾಮೀಜಿ ಮನಸ್ಸು ಕೆಲವು ವೇಳೆ ಕ್ಷೋಭೆಗೆ ಒಳಗಾಗುತ್ತಿತ್ತು. ಅವರ ತಾಯಿ ಮತ್ತು ಸಹೋದರ ಸಹೋದರಿಯರು ಜೀವನೋಪಾಯಕ್ಕಾಗಿ ಕಷ್ಟಪಡಬೇಕಾಗಿತ್ತು. ಕೋರ್ಟಿನಲ್ಲಿ ಮನೆಯ ಮೇಲಿನ ಕೇಸು ಬೇರೆ ಇನ್ನೂ ಮುಗಿದಿರಲಿಲ್ಲ. ಅವರ ಜವಾಬ್ದಾರಿಯೆಲ್ಲ ಮನೆಯ ಹಿರಿಯ ಮಗನಾದುದರಿಂದ ಇವರ ಮೇಲೆ ಬಿದ್ದಿತು. ಅದು ಇವರ ಪರವಾಗಿ ಕೊನೆಗೆ ಇತ್ಯರ್ಥವಾಯಿತು. ಅನಂತರ ಕಲ್ಕತ್ತೆಯಲ್ಲಿರುವುದಕ್ಕಿಂತ ತೀರ್ಥಯಾತ್ರೆಗೆ ಹೊಗಿ ಯಾವುದಾದರೂ ಸ್ಥಳದಲ್ಲಿ ಧ್ಯಾನದಲ್ಲಿ ತತ್ಪರನಾಗಬೇಕೆಂದು ಆಶಿಸಿದರು. ಕಾಶಿಗೆ ಹೋಗಿ ಪ್ರಮದಬಾಬುಗಳ ಹತ್ತಿರ ಶಾಸ್ತ್ರಗಳಿಗೆ ಸಂಬಂಧಪಟ್ಟ ಚರ್ಚೆ ಮತ್ತು ಧ್ಯಾನದಲ್ಲಿ ಕಾಲ ಕಳೆಯಬೇಕೆಂದು ಯೋಚಿಸಿದರು. ಅವರ ಗುರುಭಾಯಿಗಳಾದ ಅಖಂಡಾನಂದರು ಹಿಮಾಲಯಗಳ ಮೂಲಕ ಟಿಬೆಟ್ಟಿಗೆ ಹೋಗಿ ಅಲ್ಲಿಯ ಜನರ ವಿಚಿತ್ರ ಆಚಾರ ವ್ಯವಹಾರಗಳ ವಿಷಯವನ್ನು ವಿವರಿಸಿ ವಿವೇಕಾನಂದರಿಗೆ ಕಾಗದ ಬರೆದಿದ್ದರು. ಇನ್ನು ನಾಲ್ಕು ಜನ ಗುರುಭಾಯಿಗಳು ಹಿಮಾಲಯದಲ್ಲಿ ಸಂಚರಿಸುತ್ತಿದ್ದರು. ಸ್ವಾಮೀಜಿ ತಾವು ತೀರ್ಥಯಾತ್ರೆಗೆ ಹೋಗಬೇಕೆಂದು ಮನಸ್ಸು ಮಾಡಿದರು. ಅದ್ಕ್ಕಾಗಿ ೧೮೮೯ನೇ ಡಿಸೆಂಬರ್ ಕೊನೆಯ ವಾರದಲ್ಲಿ ವೈದ್ಯನಾಥಕ್ಕೆ ತೆರಳಿದರು. ಆಗ ಪ್ರಮದ ದಾಸರಿಗೆ ಒಂದು ಪತ್ರವನ್ನು ಬರೆದು ಕೊನೆಯಲ್ಲಿ ಹೀಗೆ ಹೇಳುವರು: “ಅಲ್ಲಿ ಕೊಂಚ ದಿನಗಳಿದ್ದು ವಿಶ್ವನಾಥ ಮತ್ತು ಅನ್ನಪೂರ್ಣ ನನ್ನ ಅದೃಷ್ಟವನ್ನು ಯಾವ ರೀತಿ ರೂಪಿಸುವರು ಎಂಬುದನ್ನು ನಿರೀಕ್ಷಿಸಬೇಕೆಂದು ಇಚ್ಛೆ ಇದೆ. ನನ್ನ ಪ್ರಾಣವನ್ನು ಅರ್ಪಿಸಬೇಕು, ಇಲ್ಲವೆ ನನ್ನ ಇಷ್ಟ ದೈವವನ್ನು ಸಾಕ್ಷಾತ್ಕಾರಗೊಳಿಸಿಕೊಳ್ಳಬೇಕು ಎಂಬುದೇ ನನ್ನ ನಿರ್ಧಾರ. ಹೇ ಕಾಶಿನಾಥ, ನನಗೆ ನೆರವು ನೀಡು.” 

 ಸ್ವಾಮೀಜಿ ವೈದ್ಯನಾಥದಲ್ಲಿದ್ದಾಗ ಯೋಗಾನಂದ ಎಂಬ ಇವರ ಗುರುಭಾಯಿಗಳು ಅಲಹಾಬಾದಿನಲ್ಲಿ ದಡಾರದಿಂದ ನರಳುತ್ತಿರುವರೆಂಬುದನ್ನು ಕೇಳಿದರು. ವೈದ್ಯನಾಥದಿಂದ ಅಲ್ಲಿಗೆ ಹೋಗಿ ಅವರಿಗೆ ಶುಶ್ರೂಷೆ ಮಾಡಿದರು. ಅಲ್ಲಿದ್ದಾಗ ಅನೇಕರು ಸ್ವಾಮೀಜಿ ಅವರ ಮಾತುಕತೆಗಳನ್ನು ಕೇಳುವುದಕ್ಕೆ ಬರುತ್ತಿದ್ದರು. ಅವರಿಗೆಲ್ಲ ಸ್ವಾಮೀಜಿಯವರ ಪ್ರತಿಭೆ ಮತ್ತು ವಿದ್ವತ್ತನ್ನು ನೋಡಿ ಆಶ್ಚರ್ಯವಾಯಿತು. ಈ ಊರಿನಲ್ಲಿದ್ದಾಗಲೇ ಒಬ್ಬ ಮುಸಲ್ಮಾನ್ ಫಕೀರನನ್ನು ಕಂಡರು. ಇವನೊಬ್ಬ ಪರಮಹಂಸ ಎಂಬುದನ್ನು ಅವನ ಮುಖದ ಪ್ರತಿಯೊಂದು ಗೆರೆಯೂ ಸಾರುತ್ತಿತ್ತು ಅನ್ನುವರು. ಸ್ವಾಮೀಜಿ ಇಲ್ಲಿದ್ದಾಗಲೆ ಗಾಜೀಪುರದಲ್ಲಿ ಒಬ್ಬ ಪವಾಹಾರಿಬಾಬಾರೆಂಬ ಮಹಾಯೋಗಿಗಳು ಇರುವರು ಎಂಬುದನ್ನು ಕೇಳಿ ಅವರನ್ನು ನೋಡಲು ಅಲ್ಲಿಗೆ ಹೋದರು. ಸ್ವಾಮೀಜಿ ಪವಾಹಾರಿಬಾಬಾರಿಂದ ತುಂಬಾ ಆಕರ್ಷಿತರಾದರು. ಅವರ ವಿಷಯವಾಗಿ ತಾವೇ ಅನಂತರ ಒಂದು ಲೇಖನವನ್ನು ಬರೆದು ಅದರ ಕೊನೆಯಲ್ಲಿ “ಎಷ್ಟೇ ಅನರ್ಹನಾದರೂ ನಾನು ಪ್ರೀತಿಸಿದ, ಸೇವಿಸಿದ ಪ್ರಪಂಚದ ಪ್ರಖ್ಯಾತ ಮಹಾತ್ಮರಲ್ಲೊಬ್ಬರ ಜ್ಞಾಪಕಾರ್ಥವಾಗಿ ಈ ಲೇಖನವನ್ನು ಸಮರ್ಪಿಸುವೆನು” ಎಂದು ಹೇಳುವರು. ಅವರ ಲೇಖನದಿಂದಲೇ ಪವಾಹಾರಿಬಾಬಾರ ಸಂಕ್ಷೇಪ ಜೀವನವನ್ನು ಕೊಡುತ್ತೇವೆ: 

 “ಕಾಶಿಯ ಸಮೀಪದಲ್ಲಿ ಪ್ರೇಮಪುರವೆಂಬ ಒಂದು ಗ್ರಾಮ. ಅಲ್ಲಿ ಬ್ರಾಹ್ಮಣ ಕುಟುಂಬ ಒಂದರಲ್ಲಿ ಒಂದು ಶಿಶು ಜನಿಸಿತು. ಅನಂತರ ಅದರ ಹೆಸರೆ ಪವಾಹಾರಿಬಾಬಾ ಎಂದಾಯಿತು. ಚಿಕ್ಕ ಹುಡುಗರಾಗಿದ್ದ ಕಾಲದಿಂದಲೂ ಅವರು ಗಾಜೀಪುರದಲ್ಲಿ ತಮ್ಮ ಮಾವನ ಮನೆಯಲ್ಲಿಯೇ ವ್ಯಾಸಂಗಮಾಡುತ್ತಿದ್ದರು.” 

 “ಪವಾಹಾರಿಬಾಬಾರ ಮಾವ ರಾಮಾನುಜ ಪಂಥಕ್ಕೆ ಸೇರಿದ ಒಬ್ಬ ನೈಷ್ಠಿಕ ಬ್ರಹ್ಮಚಾರಿಗಳು. ಗಾಜೀಪುರದ ಉತ್ತರಕ್ಕೆ ಸುಮಾರು ಎರಡು ಮೈಲಿಗಳ ದೂರದಲ್ಲಿ ಗಂಗಾತೀರದಲ್ಲಿ ಅವರಿಗೆ ಒಂದು ಜಮೀನು ಇತ್ತು. ಅಲ್ಲೆ ಅವರು ನೆಲಸಿದ್ದರು. ಅವರಿಗೆ ಅನೇಕ ಮಂದಿ ಸೋದರ ಅಳಿಯಂದಿರಿದ್ದರು. ಅವರಲ್ಲಿ ಪವಾಹಾರಿಬಾಬಾ ಅವರನ್ನು ಮಾವ ತಮ್ಮ ಮನೆಗೆ ಕರೆದೊಯ್ದು ತಮ್ಮ ಕಾಲಾನಂತರ ಅವರೇ ತಮ್ಮ ಆಸ್ತಿಗೆಲ್ಲ ಉತ್ತರಾಧಿಕಾರಿಯಾಗಬೇಕೆಂಬ ಇಚ್ಛೆಯಿಂದ ದತ್ತು ತೆಗೆದುಕೊಂಡರು.” 

 “ಪವಾಹಾರಿಬಾಬಾ ಧರ್ಮಶಾಸ್ತ್ರ, ವ್ಯಾಕರಣ, ನ್ಯಾಯಶಾಸ್ತ್ರ ಇವುಗಳ ಅಧ್ಯಯನದಲ್ಲಿ ಆಸಕ್ತಿ ತೋರುವ ವಿದ್ಯಾರ್ಥಿಗಳಾಗಿದ್ದರು. ಅವರು ವ್ಯಾಸಂಗಕ್ಕೆ ಹೆಚ್ಚು ಗಮನ ಕೊಡುತ್ತಿದ್ದರು, ಮತ್ತು ಭಾಷಾಶಾಸ್ತ್ರದಲ್ಲಿ ಅವರಿಗೆ ಅಸಾಧಾರಣ ಪಾಂಡಿತ್ಯವಿತ್ತು. ಮೊದಲನೆಯ ಬಾರಿ ಆ ತರುಣ ವಿದ್ವಾಂಸರಿಗೆ ಜೀವನದ ಮಹತ್ವವನ್ನು ಅರ್ಥಮಾಡಿಕೊಳ್ಳುವ ಪ್ರಸಂಗ ಬಂದೊದಗಿತು. ಇಲ್ಲಿಯವರೆಗೂ ಪುಸ್ತಕದ ಮೇಲೆ ನೆಟ್ಟಿದ್ದ ಕಣ್ಣುಗಳನ್ನು ತೆರೆದು ತಮ್ಮ ಮಾನಸಿಕ ದಿಗಂತವನ್ನು ವಿಮರ್ಶಾತ್ಮಕ ದೃಷ್ಟಿಯಿಂದ ಪರಿಶೀಲನೆಮಾಡಿ, ಧರ್ಮದಲ್ಲಿ ಪುಸ್ತಕಪಾಂಡಿತ್ಯ ಕೈ ನಿಲುಕದ ವಾಸ್ತವಿಕಾಂಶ ಯಾವುದಿರುವುದೋ ಅದಕ್ಕೆ ಹಾತೊರೆಯುವಂತೆ ಮಾಡಿತು. ಆ ಪ್ರಸಂಗವೆ ಅವರ ಮಾವ ಸ್ವರ್ಗಸ್ಥರಾದುದು. ಆ ಎಳೆ ಹೃದಯದ ಪ್ರೇಮವೆಲ್ಲ ಯಾವ ವ್ಯಕ್ತಿಯ ಮೇಲೆ ಸಂಪೂರ್ಣ ನೆಲಸಿತ್ತೋ ಆ ವ್ಯಕ್ತಿ ಕಣ್ಮರೆಯಾದ. ಅವರು ದುಃಖತಾಡಿತ ಹೃದಯಿಯಾದರು. ತಮ್ಮ ಹೃದಯ ಶೂನ್ಯತೆಯನ್ನು ಎಂದೆಂದಿಗೂ ನಾಶವಾಗದ, ಅತೀಂದ್ರಿಯ ದರ್ಶನದಿಂದ ಪೂರ್ಣಮಾಡಿಕೊಳ್ಳಲು ಬದ್ಧಕಂಕಣರಾದರು.” 

 “...ಅವರ ಜೀವನಗತಿಯ ಮುಂದಿನ ಸಂಧಿಕಾಲ ಕಾಶಿಯ ಸಮೀಪದ ಗಂಗಾ ತೀರದಲ್ಲಿದೆ. ಅಲ್ಲಿ ಯೋಗಾಭ್ಯಾಸ ಮಾಡುತ್ತ ನದಿಯ ಕಡಿದಾದ ದಡದ ಗುಹೆಯೊಂದರಲ್ಲಿ ವಾಸಿಸುತ್ತಿದ್ದ ಸಂನ್ಯಾಸಿಯೊಬ್ಬರ ಶಿಷ್ಯರಾಗಿ ಕೆಲವು ಕಾಲ ಕಳೆದರೆಂದು ತಿಳಿದು ಬರುತ್ತದೆ. ಅನಂತರ ಗಾಜೀಪುರದ ಬಳಿ ಗಂಗಾತೀರದ ನೆಲದಲ್ಲಿ ಒಂದು ದೊಡ್ಡ ಗುಂಡಿಯನ್ನು ತೋಡಿ ಪವಾಹಾರಿಬಾಬಾರವರು ವಾಸಮಾಡಿಕೊಂಡಿದ್ದುದು ಅದರ ಪರಿಣಾಮವಾಗಿ ಇರಬೇಕು.” 

 “ಅವರು ನೆಲದಲ್ಲಿ ಒಂದು ಗುಹೆಯನ್ನು ತೋಡಿ ಕಾಶಿಯಲ್ಲಿದ್ದ ತಮ್ಮ ಗುರುವಿನಂತೆಯೇ ಇದ್ದರು. ಇಡೀ ದಿನ ಅವರು ತಮ್ಮ ಆಶ್ರಮದ ಕೆಲಸದಲ್ಲಿ ನಿರತರಾಗಿರುತ್ತಿದ್ದರು. ಇಷ್ಟದೇವನಾದ ರಾಮಚಂದ್ರನ ಪೂಜೆಯನ್ನು ಮಾಡುತ್ತಿದ್ದರು. ಒಳ್ಳೊಳ್ಳೆಯ ಭಕ್ಷ್ಯಗಳನ್ನು ನೈವೇದ್ಯಕ್ಕೆ ತಯಾರುಮಾಡುತ್ತಿದ್ದರು. ಪಾಕಶಾಸ್ತ್ರದಲ್ಲಿ ಅವರು ಅತಿ ನಿಪುಣರಂತೆ. ಶ‍್ರೀರಾಮಚಂದ್ರನಿಗೆ ಅರ್ಪಿಸಿದ ಪ್ರಸಾದವನ್ನೆಲ್ಲ ಸ್ನೇಹಿತರಿಗೆ ಮತ್ತು ಬಡಬಗ್ಗರಿಗೆ ಹಂಚಿಬಿಡುತ್ತಿದ್ದರು. ರಾತ್ರಿಯವರೆಗೆ ಅಲ್ಲಿದ್ದು ಅವರ ಸೌಕರ‍್ಯಗಳನ್ನೆಲ್ಲ ನೋಡಿಕೊಂಡು ಅವರೆಲ್ಲ ಹಾಸಿಗೆಯಲ್ಲಿ ಮಲಗಿದಮೇಲೆ, ಆ ತರುಣರು ಗೋಪ್ಯವಾಗಿ ಎದ್ದು, ಈಜಿಕೊಂಡು ಗಂಗಾನದಿಯನ್ನು ದಾಟಿ ಅತ್ತ ಕಡೆ ಹೋಗುತ್ತಿದ್ದರು. ಅಲ್ಲಿ ಇಡೀ ರಾತ್ರಿ ಸಾಧನೆ ಭಜನೆಯಲ್ಲಿ ಕಳೆದು ಬೆಳಿಗ್ಗೆ ಹೊತ್ತಿಗೆ ಮುಂಚೆ ಹಿಂತಿರುಗಿ ಬಂದು ತಮ್ಮ ಸ್ನೇಹಿತರನ್ನೆಲ್ಲ ಎಬ್ಬಿಸಿ ಪುನಃ ಯಥಾಪ್ರಕಾರ ನಿತ್ಯವಿಧಿಯಲ್ಲಿ ಎಂದರೆ ಇತರರ ಸೇವೆಯಲ್ಲಿ ನಿರತರಾಗುತ್ತಿದ್ದರು.” 

 “ಆ ಸಮಯದಲ್ಲಿ ಅವರು ತಮ್ಮ ಆಹಾರವನ್ನು ದಿನದಿನಕ್ಕೂ ಕಡಿಮೆ ಮಾಡುತ್ತ ಬಂದರು. ಕಡೆಗೆ ದಿನಕ್ಕೆ ಒಂದು ಹಿಡಿ ಬೇವಿನ ಎಲೆ ಅಥವಾ ಮೆಣಸಿನ ಕಾಳು ಇವುಗಳನ್ನು ತಿನ್ನುವುದರಲ್ಲೇಅವರ ಊಟವೆಲ್ಲ ಕೊನೆಗೊಳ್ಳುತ್ತಿತ್ತು. ಅನಂತರ ಅವರು ಪ್ರತಿ ರಾತ್ರಿಯೂ ನದಿಯ ಆಚೆ ದಡದ ಕಾಡಿಗೆ ಹೋಗುವುದನ್ನು ನಿಲ್ಲಿಸಿಬಿಟ್ಟು ಹೆಚ್ಚಾಗಿ ತಮ್ಮ ಗುಹೆಯಲ್ಲಿಯೇ ಧ್ಯಾನದಲ್ಲಿ ತಲ್ಲೀನರಾಗಿದ್ದು ಅನಂತರ ಹೊರಗೆ ಬರುತ್ತಿದ್ದರಂತೆ. ಅಷ್ಟು ಕಾಲ ಅವರು ಗುಹೆಯೊಳಗೆ ಏನನ್ನು ತಿಂದು ಜೀವಿಸುತ್ತಿದ್ದರೋ ಜನರಿಗೆ ಅದು ಗೊತ್ತಿರಲಿಲ್ಲ. ಆದಕಾರಣವೇ ಅವರನ್ನು ಗಾಳಿಯನ್ನೇ ಆಹಾರವಾಗಿ ಸೇವಿಸುತ್ತಿದ್ದ ಬಾಬಾ, ಪವಾಹಾರಿಬಾಬಾ ಎಂದು ಕರೆಯತೊಡಗಿದರು.” 

 ಸ್ವಾಮೀಜಿ ಪವಾಹಾರಿಬಾಬಾರನ್ನು ನೋಡುವುದಕ್ಕೆ ಗಾಜೀಪುರಕ್ಕೆ ಬಂದವರು ಬಾಬು ಸತೀಶಚಂದ್ರಮುಖರ್ಜಿ ಮತ್ತು ಗಗನ್‍ಚಂದ್ರ ರಾಯ್‍ಬಹದ್ದೂರ್ ಅವರುಗಳ ಮನೆಯಲ್ಲಿ ಇದ್ದರು. ಇವರಲ್ಲಿ ಮೊದಲನೆಯವರು ಸ್ವಾಮೀಜಿ ಅವರ ಪೂರ್ವಾಶ್ರಮದ ಸ್ನೇಹಿತರಾಗಿದ್ದರು. ಸ್ವಾಮೀಜಿ ದರ್ಶನ ಮತ್ತು ಅವರ ಮಾತುಕತೆಗಳನ್ನು ಕೇಳುವುದಕ್ಕೆ ಅನೇಕ ಜನ ಬರುತ್ತಿದ್ದರು. “ಅವರು ಒಳ್ಳೆ ಸ್ವಭಾವದವರು. ಆದರೆ ಹೆಚ್ಚು ಪಾಶ್ಚಾತ್ಯರನ್ನು ಅನುಕರಿಸುವರು. ಇದೊಂದು ದುಃಖದ ಸಂಗತಿ, ಎಲ್ಲ ಪಾಶ್ಯಾತ್ಯರ ಅನುಕರಣೆಯನ್ನೂ ನಾನು ವಿರೋಧಿಸುತ್ತೇನೆ. ಆದರೆ ನನ್ನ ಸ್ನೇಹಿತ ಇನ್ನೂ ಅದಕ್ಕೆ ಗಮನ ಕೊಟ್ಟಿಲ್ಲ. ಎಂತಹ ಥಳಕಿನ ನಾಗರೀಕತೆಯನ್ನು ಹೊರಗಿನವರು ತಂದಿರುವರು! ಎಂತಹ ಒಂದು ಚಾರ್ವಾಕ ಭ್ರಾಂತಿಯನ್ನು ಅವರು ಸೃಷ್ಟಿಸಿರುವರು! ವಿಶ್ವನಾಥ ಇಂಥ ದುರ್ಬಲರನ್ನು ರಕ್ಷಿಸಲಿ. ಬಾಬಾಜಿಯವರನ್ನು ನೋಡಿದ ಮೇಲೆ ನಿಮಗೆ ವಿವರವಾದ ವರದಿಯನ್ನು ಕಳುಹಿಸುತ್ತೇನೆ. ಅಯ್ಯೋ! ನಮ್ಮ ದುರದೃಷ್ಟ, ಭಗವಾನ್ ಶುಕ ಜನ್ಮವೆತ್ತಿದ ನಮ್ಮ ದೇಶದಲ್ಲಿ ತ್ಯಾಗವನ್ನು ಒಂದು ಹುಚ್ಚುತನ ಮತ್ತು ಪಾಪ ಎಂದು ಪರಿಗಣಿಸುವರು.” ಹಿಗೆಂದು ಸ್ವಾಮೀಜಿ ಪ್ರಮದದಾಸ ಮಿತ್ರರಿಗೆ ಕಾಗದ ಬರೆದಿದ್ದರು. 

 ಬಾಬಾಜಿ ದರ್ಶನ ಸಿಕ್ಕುವುದು ಬಹು ಕಷ್ಟವಾಯಿತು. ಅವರು ಗುಹೆಯಿಂದ ಹೊರಗೆ ಬರುತ್ತಿರಲಿಲ್ಲ. ಮಾತನಾಡಲಿಚ್ಛೆಯಿದ್ದಲ್ಲಿ ಬಾಗಿಲಿನ ಬಳಿ ಬಂದು ಒಳಗಿನಿಂದಲೇ ಮಾತನಾಡುವರು. ಸುತ್ತಲೂ ಆವೃತವಾದ ಎತ್ತರದ ಗೋಡೆಗಳು, ಹೊಗೆಯ ಕೊಳವೆಯ ತುದಿಗಳು ಇವುಗಳಿಂದ ಕೂಡಿರುವ ತೋಟದ ಮನೆಯ ದರ್ಶನ ಮಾತ್ರ ಸ್ವಾಮೀಜಿಗೆ ಮೊದಲು ಆದದ್ದು. ಒಳಗಡೆ ಗುಹೆಯಂತಹ ಕೊಠಡಿಗಳು ಇವೆಯೆಂದೂ, ಅಲ್ಲೇ ಬಾಬಾಜಿ ವಾಸಿಸುವರೆಂದೂ, ಮತ್ತಾರೂ ಅಲ್ಲಿ ಇಣಕಿ ನೋಡಲೂ ಸಾಧ್ಯವಿಲ್ಲವೆಂದೂ ಅನಂತರ ತಿಳಿಯಿತು. ಅವರೇನು ಅಲ್ಲಿ ಮಾಡುವರೆಂಬುದು ಬಾಬಾಜಿಗೆ ಮಾತ್ರ ಗೊತ್ತಿತ್ತು. ಒಂದು ದಿನ ಸ್ವಾಮೀಜಿ ಕಾದು ಬಳಲಿ ಬೆಂಡಾಗಿ ಹೊರಟು ಬಂದರು. ಅಂತೂ ಕೊನೆಗೊಂದು ದಿನ ಅವರ ದರ್ಶನವಾಯಿತು. 

 ಸ್ವಾಮೀಜಿ ಬಾಬಾಜಿ ಅವರನ್ನು ನೋಡಿದ ಮೇಲೆ ಹೀಗೆ ಬರೆಯುವರು: 

 “ನನ್ನ ಮಹತ್‍ಭಾಗ್ಯದಿಂದ ನಾನು ಬಾಬಾಜಿ ಅವರನ್ನು ಸಂದರ್ಶಿಸಿದೆ. ನಿಜವಾಗಿ ಅವರೊಬ್ಬ ಮಹರ್ಷಿಗಳು! ಅದೆಷ್ಟು ಅದ್ಭುತ! ಈ ನಾಸ್ತಿಕ ಯುಗದಲ್ಲಿ ಭಕ್ತಿ ಮತ್ತು ಯೋಗದಿಂದ ಜನಿಸಿದ ಅತ್ಯದ್ಭುತ ಶಕ್ತಿಯ ಭವ್ಯ ಸಂಕೇತ! ನಾನು ಇಲ್ಲಿಯೇ ಕೆಲವು ದಿನಗಳು ಇರಬೇಕೆಂದು ಬಾಬಾಜಿಯವರ ಇಚ್ಛೆ. ಅವರು ನನಗೆ ಸ್ವಲ್ಪ ಒಳ್ಳೆಯದನ್ನು ಮಾಡುತ್ತಾರಂತೆ. ಆ ಸಾಧುವಿನ ಅಪ್ಪಣೆಯಂತೆ ನಾನು ಸ್ವಲ್ಪ ಕಾಲ ಇಲ್ಲಿಯೇ ಇರುವೆ.” 

 ಸ್ವಾಮೀಜಿಯವರು ಗಗನ ಬಾಬುವಿನ ತೋಟದ ಮನೆಗೆ ಹೋಗಿ ವಾಸಿಸತೊಡಗಿದರು. ಭಿಕ್ಷೆಯಿಂದ ಜೀವಿಸುತ್ತಿದ್ದರು. ಪ್ರತಿದಿನ ಬಾಬಾಜಿ ಗುಹೆಯ ಬಳಿಗೆ ಹೋಗುತ್ತಿದ್ದರು. ಮಾತಿನಲ್ಲಿ ಅಷ್ಟೊಂದು ಮಾಧುರ‍್ಯವನ್ನು ಸ್ವಾಮೀಜಿ ಮತ್ತೆಲ್ಲೂ ಕೇಳಿರಲಿಲ್ಲ. ಅವರು ಪ್ರಶ್ನೆಗಳಿಗೆ ನೇರವಾಗಿ ಉತ್ತರವನ್ನು ಕೊಡುವುದಿಲ್ಲ, ಈ ದಾಸನೇನು ಬಲ್ಲ ಎನ್ನುವರು. ಆದರೆ ಮಾತು ಮುಂದುವರಿದಂತೆಲ್ಲ ತೇಜಸ್ಸು ವ್ಯಕ್ತವಾಗುವುದು. ಬಾಬಾಜಿಯವರನ್ನು ಬಹಳ ಒತ್ತಾಯಪಡಿಸಿದಾಗ, ಕೆಲವು ದಿನ ಇಲ್ಲಿದ್ದು ದಾಸನಿಗೆ ಅನುಗ್ರಹ ಮಾಡಿ ಎನ್ನುತ್ತಿದ್ದರು. 

 ಸ್ವಾಮೀಜಿ ಸೊಂಟದ ನೋವಿನಿಂದ ನರಳುತ್ತಿದ್ದಾಗ ಬಾಬಾಜಿಯವರನ್ನು ನೋಡುವುದಕ್ಕೆ ಹೋಗಲು ಆಗಲಿಲ್ಲ. ಆಗ ಬಾಬಾಜಿಯವರೆ ಯಾರನ್ನೊ ಕಳುಹಿಸಿ ಸ್ವಾಮೀಜಿಯ ಆರೋಗ್ಯವನ್ನು ವಿಚಾರಿಸಿಕೊಳ್ಳುತ್ತಿದ್ದರು. ಗಗನ ಬಾಬುವಿನ ಮೂಲಕ ಓಪಿಯಂ ಇಲಾಖೆಯಲ್ಲಿದ್ದ ಮಿಸ್ಟರ್ ರಾಸ್ ಎಂಬುವರನ್ನು ಪರಿಚಯ ಮಾಡಿಕೊಂಡರು. ಅವರು ಹಿಂದೂಗಳ ಹಬ್ಬದ ವಿಷಯದಲ್ಲಿ ಸ್ವಾಮೀಜಿಯವರನ್ನು ಪ್ರಶ್ನಿಸಿದರು. ಹೋಳಿ ಹಬ್ಬದ ಮೇಲೆ ಒಂದು ಲೇಖನವನ್ನು ಬರೆಯಿರಿ ಎಂದು ಸ್ವಾಮೀಜಿ ಅವರನ್ನು ಕೋರಿಕೊಂಡು ಅದನ್ನು ಅವರ ಕೈಯಿಂದ ಬರೆಸಿದರು. ಮಿಸ್ಟರ್ ರಾಸ್ ಅವರು ಗಾಜೀಪುರದ ಜಿಲ್ಲೆಯ ನ್ಯಾಯಾಧಿಪತಿಗಳಾದ ಪೆನ್ನಿಂಗ್‍ಟನ್ ಅವರಿಗೆ ಪರಿಚಯ ಮಾಡಿಸಿದರು. ಆತ ಸ್ವಾಮೀಜಿಯವರು ಹಿಂದೂಧರ್ಮ ಮತ್ತು ಅದರ ಆಚಾರ ವ್ಯವಹಾರಗಳು ಇವುಗಳ ವಿಷಯದಲ್ಲಿ ವಿವರಣೆಯನ್ನು ಕೇಳಿ ಮುಗ್ಧನಾಗಿ ಹೋದ. ಆತ ಸ್ವಾಮೀಜಿ ಅವರನ್ನು ಇಂಗ್ಲೆಂಡಿಗೆ ಹೋಗಿ ಇವುಗಳನ್ನು ಬೋಧಿಸಿ ಎಂದು ಹೇಳಿದ. ಅನಂತರ ಅಲ್ಲಿದ್ದ ಕರ‍್ನಲ್ ರಿವೆಟ್ ಕಾರ‍್ನ್ಯಾಕ್ ಅವರೊಡನೆ ವೇದಾಂತ ವಿಷಯದ ಮೇಲೆ ಸೂಕ್ಷ್ಮ ಚರ್ಚೆಯನ್ನು ನಡೆಸಿದರು. 

 ಸ್ವಾಮೀಜಿ ಟಿಬೆಟ್ಟಿನ ವಿಷಯವಾಗಿ ಕಾಗದ ಬರೆಯುತ್ತಿದ್ದ ಅಖಂಡಾನಂದರಿಗೆ ಗಾಜೀಪುರಕ್ಕೆ ಬರಲು ಕಾಗದ ಬರೆದರು. ಅವರು ಬಂದರೆ ಅನಂತರ ಇಬ್ಬರೂ ನೇಪಾಳಕ್ಕೆ ಹೋಗಬಹುದೆಂದೂ, ಅಲ್ಲಿ ನೇಪಾಳ ಮಹಾರಾಜರಿಗೆ ತಮ್ಮ ಹಳೆಯ ಸ್ನೇಹಿತ ಉಪಾಧ್ಯಾಯನಾಗಿರುವುದರಿಂದ ಅವನ ಮೂಲಕ ಟಿಬೆಟ್ಟಿಗೆ ಹೋಗಲು ಸಾಧ್ಯವಾಗಬಹುದೆಂದೂ ತಿಳಿಸಿದ್ದರು. ನೇಪಾಳ ಸರ್ಕಾರದ ಕೆಲವು ಅಧಿಕಾರಿಗಳು ಪ್ರತಿವರ್ಷವೂ ಸೇನೆಯ ರಕ್ಷಣೆಯೊಂದಿಗೆ ಟಿಬೆಟ್ಟಿನ ರಾಜಧಾನಿಯಾದ ಲಾಸಾಕ್ಕೆ ಹೋಗುತ್ತಿದ್ದರು. ಸ್ವಾಮೀಜಿ ಅವರ ಜೊತೆಯಲ್ಲಿ ಅಲ್ಲಿಗೆ ಹೋಗಬಹುದೆಂದು ಭಾವಿಸಿದರು. ಆದರೆ ಸ್ವಾಮೀಜಿಗೆ ಸೊಂಟ ನೋವು ದಾರುಣವಾಗಿತ್ತು. ಇಂತಹ ಸ್ಥಿತಿಯಲ್ಲಿ ಹಿಮಾಲಯದಲ್ಲಿ ಹತ್ತಿಕೊಂಡು ಹೋಗುವುದನ್ನು ಬಿಟ್ಟುಬಿಟ್ಟರು. ಆ ಸಮಯದಲ್ಲಿ ಹೃಷೀಕೇಶದಲ್ಲಿ ಅಭೇದಾನಂದರು ಅತಿಸಾರದಿಂದ ನರಳುತ್ತಿರುವರು ಎಂಬ ವಾರ್ತೆ ಬಂತು. ತನ್ನ ಸಹಾಯ ಬೇಕಾದರೆ ತಕ್ಷಣ ಬರುತ್ತೇನೆ ಎಂದು ಒಂದು ತಂತಿಯನ್ನು ಅವರಿಗೆ ಕಳುಹಿಸಿ ಉತ್ತರಕ್ಕಾಗಿ ಕಾಯುತ್ತಿದ್ದರು. ಸ್ವಾಮೀಜಿ ಬಾಬಾ ಅವರಿಂದ ಏನನ್ನು ಆಶಿಸಿದ್ದರೋ ಅದು ಲಭಿಸಲಿಲ್ಲ. ಸ್ವಾಮೀಜಿ ತಾವು ಅವರ ಶಿಷ್ಯರಾದರೆ ಮಾತ್ರ ಅವರು ಅನುಗ್ರಹಿಸುವುದಾದರೆ, ತಾವು ಅದಕ್ಕೂ ಸಿದ್ಧನಾಗಿರುವೆ ಎಂದು ನಿರ್ಧರಿಸಿದರು. ಹೀಗೆ ನಿರ್ಧಾರವನ್ನು ಮಾಡಿದೊಡನೆಯೇ ಶ‍್ರೀರಾಮಕೃಷ್ಣರು ಇವರಿಗೆ ಕಾಣಿಸಿಕೊಂಡು ಅವರ ಮುಖವನ್ನೇ ದಿಟ್ಟಿಸಿ ನೋಡತೊಡಗಿದರು. ಅವರನ್ನು ನೋಡಿದೊಡನೆಯೇ ವಿವೇಕಾನಂದರ ಕಣ್ಣಿನಲ್ಲಿ ನೀರು ಬಂದಿತು. ಯಾರ ಸ್ಪರ್ಶಮಾತ್ರದಿಂದಲೇ ಅತೀಂದ್ರಿಯ ಅನುಭವ ಪಡೆದಿದ್ದರೋ ಯಾರು ತಮ್ಮ ತಪೋಶಕ್ತಿಯನ್ನೆಲ್ಲ ಇವರಿಗಾಗಿ ಧಾರೆ ಎರೆದಿದ್ದರೋ, ಯಾರು ಪ್ರೇಮಾಮೃತವನ್ನೇ ತಮ್ಮ ಮೇಲೆ ವರ್ಷಿಸಿದ್ದರೋ ಅವರ ಮೂರ್ತಿ ಮುಂದೆ ಬಂದು ನಿಂತಾಗ, ಅವರನ್ನು ಬಿಟ್ಟು ಇನ್ನೊಬ್ಬರನ್ನು ಸ್ವೀಕರಿಸುವುದಕ್ಕೆ ನಾಚಿಕೆಯಾಯಿತು. ಈ ಅನುಭವ ಆದಮೇಲೂ ವಿವೇಕಾನಂದರು ಪವಾಹಾರಿ ಬಾಬಾರ ಶಿಷ್ಯರಾಗಬೇಕೆಂದು ಬಯಸಿದರು. ಆದರೆ ಪ್ರತಿಸಲವೂ ಸ್ವಾಮೀಜಿಯವರಿಗೆ ಶ‍್ರೀರಾಮಕೃಷ್ಣರು ಕಾಣಿಸಿಕೊಂಡು ಅವರ ಪ್ರತಿಜ್ಞೆ ಜಾರುವಂತೆ ಮಾಡುತ್ತಿದ್ದರು. ಅನಂತರ ಪವಾಹಾರಿ ಬಾಬಾರ ಮೇಲೆ ಗೌರವವನ್ನು ಇಟ್ಟಿದ್ದರೂ ಅವರನ್ನು ಗುರುವಿನ ಸ್ಥಾನಕ್ಕೆ ಏರಿಸಲಿಲ್ಲ. ಆಗ ವಿವೇಕಾನಂದರು ಹೀಗೆ ಬರೆಯುವರು: 

 “ಈಗ ನನ್ನ ದೃಢ ನಿಶ್ಚಯವೇನೆಂದರೆ ಶ‍್ರೀರಾಮಕೃಷ್ಣರಿಗೆ ಸರಿಸಮಾನರಿಲ್ಲ. ನನ್ನ ಕೋಟ್ಯಂತರ ಅಪರಾಧಗಳನ್ನೆಲ್ಲ ಅವರು ಕ್ಷಮಿಸಿರುವರು. ನನ್ನ ತಂದೆ ತಾಯಿಗಳು ಕೂಡ ನನ್ನನ್ನು ಅಷ್ಟು ಪ್ರೀತಿಸುತ್ತಿರಲಿಲ್ಲ, ಇದು ಅಲಂಕಾರವಲ್ಲ, ಅತಿಶಯೋಕ್ತಿಯಲ್ಲ; ಇದು ಶುದ್ಧ ಸತ್ಯ. ಅವರ ಶಿಷ್ಯರಿಗೆಲ್ಲ ಇದು ಗೊತ್ತು. ಬಹಳ ಅಪಾಯದ ಸಮಯದಲ್ಲಿ, ಪ್ರಲೋಭನದ ಸಮಯದಲ್ಲಿ, ದಾರುಣ ದುಃಖದಿಂದ ‘ಹೇ ದೇವ, ನನ್ನನ್ನು ರಕ್ಷಿಸು’ ಎಂದು ಕಂಬನಿದುಂಬಿ ಬೇಡಿರುವೆನು. ಯಾರಿಂದಲೂ ಇದಕ್ಕೆ ಉತ್ತರ ದೊರೆಯಲಿಲ್ಲ. ಆದರೆ ಆ ಮಹಾಪುರುಷನಾದರೋ, ಅವತಾರವೊ ಅಥವಾ ಇನ್ನು ಯಾರೊ, ಮಾನವನ ಹೃದಯವನ್ನು ತಿಳಿದುಕೊಳ್ಳಬಲ್ಲ ತನ್ನ ಆಂತರಿಕ ದೃಷ್ಟಿಯಿಂದ ನನ್ನ ದುಃಖವನ್ನೆಲ್ಲ ಅರಿತು ತನ್ನೆಡೆಗೆ ಬರಮಾಡಿಕೊಂಡು, ನಾನು ಇಚ್ಛಿಸದೇ ಇದ್ದರೂ ಅವೆಲ್ಲದರಿಂದ ಪಾರುಮಾಡಿರುವರು.” 

 ಸ್ವಾಮೀಜಿ ಅವರು ಪವಾಹಾರಿಬಾಬಾರಿಂದ ರಾಜಯೋಗವನ್ನು ಕಲಿಯಬೇಕೆಂದು ಇಚ್ಛಿಸಿದರು. ಅವರಿಗೆ ಭಕ್ತಿ ಜ್ಞಾನಯೋಗಗಳು ಪರಿಚಯವಾಗಿತ್ತು. ಬಂಗಾಳ ದೇಶದಲ್ಲಿ ಯೋಗಶಾಸ್ತ್ರವನ್ನು ಯಾರೂ ಅಭ್ಯಾಸ ಮಾಡಿರಲಿಲ್ಲ. ಅದಕ್ಕಾಗಿ ಅದನ್ನು ಕಲಿಯಬೇಕೆಂದು ಅವರ ಹತ್ತಿರ ಹೋಗಿದ್ದರು. ಆದರೆ ಸ್ವಾಮೀಜಿಯವರ ಬಯುಕೆ ಬಾಬಾ ಅವರಿಂದ ಈಡೇರಲಿಲ್ಲ. 

 ಅಭೇದಾನಂದರಿಗೆ ಅನಾರೋಗ್ಯವಾದ ಪ್ರಯುಕ್ತ ಅವರನ್ನು ಕಾಶಿಗೆ ಬರಹೇಳಿ ಸ್ವಾಮೀಜಿ ಕಾಶಿಗೆ ಬಂದರು. ಅಲ್ಲಿ ಅಭೇದಾನಂದರ ಚಿಕಿತ್ಸೆಗೆ ಏರ್ಪಾಡು ಮಾಡಿ, ಪ್ರಮದದಾಸ ಮಿತ್ರರ ಉದ್ಯಾನಮನೆಯಲ್ಲಿ ಸ್ವಾಮೀಜಿ ತಪಸ್ಸಿನಲ್ಲಿ ನಿರತರಾದರು. ಆ ಸಮಯದಲ್ಲಿಯೇ ಶ‍್ರೀರಾಮಕೃಷ್ಣರ ಗೃಹೀಭಕ್ತರಾದ ಬಲರಾಮಬೋಸರು ನಿಧನರಾದರೆಂಬ ವಾರ್ತೆಯನ್ನು ಸ್ವಾಮೀಜಿ ಕೇಳಿದರು. ಅದನ್ನು ಕೇಳಿದಾಗ ಸ್ವಾಮೀಜಿಯ ದುಃಖಕ್ಕೆ ಪಾರವಿರಲಿಲ್ಲ. ಶ‍್ರೀರಾಮಕೃಷ್ಣರು ಅವರ ಮನೆಗೆ ಪದೇ ಪದೇ ಹೊಗುತ್ತಿದ್ದರು. ಅಲ್ಲಿಯೇ ನರೇಂದ್ರಾದಿಗಳನ್ನು ಕರೆಸಿ ಅವರು ಮಾತನಾಡುತ್ತಿದ್ದುದು. ಶ‍್ರೀರಾಮಕೃಷ್ಣರ ಗೃಹಸ್ಥ ಭಕ್ತರಲ್ಲೆಲ್ಲ ಅತಿ ಮುಖ್ಯರಾದವರಲ್ಲಿ ಒಬ್ಬರು ಬಲರಾಮ ಬೋಸರು. ಸ್ವಾಮೀಜಿ ಅವರಿಗೆ ಸೋದರ ಸಹಜವಾದ ಪ್ರೇಮವಿತ್ತು ಅವರ ಮೇಲೆ. ಪ್ರಮದಬಾಬುಗಳು, “ವೇದಾಂತಿ, ಜೊತೆಗೆ ಸಂನ್ಯಾಸಿಯಾಗಿರುವ ಸ್ವಾಮೀಜಿ ಯಾರೋ ತೀರಿಹೋದದ್ದಕ್ಕೆ ಇಷ್ಟು ವ್ಯಸನಪಡುವುದು ತರವಲ್ಲ” ಎಂದು ಹೇಳಿದರು. ವಿವೇಕಾನಂದರು ಆಗ ಹೀಗೆ ಹೇಳಿದರು: “ದಯವಿಟ್ಟು ಹಾಗೆ ಮಾತನಾಡಬೇಡಿ. ನಾವು ಶುಷ್ಕ ಸಂನ್ಯಾಸಿಗಳಲ್ಲ. ಒಬ್ಬ ಸಂನ್ಯಾಸಿಯಾದರೆ ಅವನಿಗೆ ಹೃದಯ ಇಲ್ಲವೆ?” ಬಲರಾಮಬೋಸರ ಸಂಸಾರಕ್ಕೆ ಸಾಂತ್ವನವನ್ನು ಕೊಡುವುದಕ್ಕೆ ಸ್ವಾಮೀಜಿ ಕಲ್ಕತ್ತೆಗೆ ಹಿಂತಿರುಗಿದರು. 

 ಸ್ವಾಮೀಜಿ ತಾವು ಬಾರಾನಗರ ಮಠಕ್ಕೆ ಬಂದು ಬಲರಾಮಬೋಸರ ಮನೆಯವರಿಗೆ ಸಮಾಧಾನ ವಚನಗಳನ್ನು ಹೇಳಿ ಮಠದಲ್ಲಿ ಆಧ್ಯಾತ್ಮಿಕ ಸಾಧನೆಯಲ್ಲಿ ನಿರತರಾದರು. ತಾವು ಏನನ್ನು ಕಂಡರೋ, ಓದಿದರೋ, ಯಾವುದರ ಮೇಲೆ ಆಲೋಚಿಸಿದ್ದರೊ ಅವನ್ನೆಲ್ಲ ಗುರುಭಾಯಿಗಳಿಗೆ ಹೇಳುತ್ತಿದ್ದರು. ಸ್ವಾಮೀಜಿ ಜ್ಞಾನದ ವಿಶ್ವಕೋಶದಂತೆ ಇದ್ದರು. 

 ಶ‍್ರೀರಾಮಕೃಷ್ಣರ ಸ್ಮಾರಕವಾಗಿ ಏನನ್ನಾದರೂ ಮಾಡಬೇಕೆಂದು ಅದಕ್ಕಾಗಿ ಹಣವನ್ನು ಚಂದಾ ಎತ್ತುವ ಸಲುವಾಗಿ ಪ್ರಮದದಾಸಮಿತ್ರರಿಗೆ ಕೆಳಗಿನ ಕಾಗದವನ್ನು ಬರೆದರು: 

 “ಪ್ರಿಯ ಮಹಾಶಯರೆ, 

 ಅನೇಕ ಅನಿವಾರ್ಯ ಪರಿಸ್ಥಿತಿಗಳು ಮತ್ತು ಮನದ ಕಳವಳ ಇವುಗಳ ಗೊಂದಲದಲ್ಲಿ ಸಿಕ್ಕಿರುವಾಗ ನಿಮಗೆ ಪತ್ರವನ್ನು ಬರೆಯುತ್ತಿರುವೆನು. ವಿಶ್ವನಾಥನಿಗೆ ನಮಿಸಿ ನಾನು ಮುಂದೆ ಬರೆಯುವುದು ಯೋಗ್ಯವಾಗಿ ಕಂಡರೆ, ಸಾಧ್ಯವಾದರೆ ದಯವಿಟ್ಟು ನನಗೆ ಉತ್ತರ ಕೊಟ್ಟು ಉಪಕಾರ ಮಾಡಿ. 

 ೧. ಮೊದಲೇ ಹೇಳಿರುವಂತೆ ನಾನು ಶ‍್ರೀರಾಮಕೃಷ್ಣರ ದಾಸ. ನನ್ನ ದೇಹವನ್ನು ಎಳ್ಳು ತುಳಸಿಗಳೊಡನೆ ಅವರ ಪಾದಾರವಿಂದದಲ್ಲಿ ಧಾರೆ ಎರೆದಿರುವೆನು. ಅವರ ಆಜ್ಞೆಯನ್ನು ನಾನು ನಿರಾಕರಿಸಲಾರೆ. ಜ್ಞಾನ ಪ್ರೀತಿ ಶಕ್ತಿ ಶಿಖರವನ್ನೇರಿದ, ನಲ್ವತ್ತು ವರ್ಷಗಳು ಉಗ್ರ ತ್ಯಾಗ ವೈರಾಗ್ಯ ಪವಿತ್ರತೆ ಕಠೊರತಮ ಸಾಧನೆಗಳನ್ನು ಅಭ್ಯಾಸಮಾಡಿದ, ಆ ಮಹಾತಪಸ್ವಿಗಳ ಜೀವನ ನಿಷ್ಪ್ರಯೋಜನವಾದರೆ, ಈ ಜಗತ್ತಿನಲ್ಲಿ ನೆಚ್ಚಿಗೆ ಇಡುವ ಆಸರೆ ಮತ್ತಾವುದಿರುವುದು? ಅದಕ್ಕೋಸುಗವೇ ಅವರ ವಾಣಿಯನ್ನು ಸತ್ಯೈಕ್ಯರಾದವರ ವಾಣಿ ಎಂದು ನಂಬಲೇಬೇಕಾಗುವುದು. 

 ೨. ಅವರಿಂದಸ್ಥಾಪಿತವಾದ ಸಂಘದಲ್ಲಿರುವ, ಎಲ್ಲವನ್ನೂ ತ್ಯಜಿಸಿದ ಭಕ್ತರ ಸೇವೆಗೆ ಗಮನಕೊಡಬೇಕೆಂದು ಅವರು ನನಗೆ ಮಾಡಿದ ಆಜ್ಞೆ. ನಾನು ಇದರಲ್ಲಿ ಎಷ್ಟು ವೇಳೆ ಸೋತರೂ ಮರಳಿ ಯತ್ನವನ್ನು ಮಾಡಲೇಬೇಕು. ಇದರಿಂದ ಸ್ವರ್ಗ ಬರಲಿ, ನರಕ ಬರಲಿ, ಮೋಕ್ಷ ಸಿದ್ಧಿಸಲಿ ಅಥವಾ ಇನ್ನೇನೆ ಪ್ರಾಪ್ತವಾಗಲಿ ಎಲ್ಲವನ್ನೂ ಅನುಭವಿಸಲು ಸಿದ್ಧನಾಗಿರಬೇಕು. 

 ೩. ಸರ್ವಸ್ವವನ್ನೂ ತ್ಯಾಗಮಾಡಿದ ಭಕ್ತರೆಲ್ಲರೂ ಒಂದೆಡೆ ಒಟ್ಟಿಗೆ ಸೇರಬೇಕು ಅನ್ನುವುದು ಅವರ ಆಜ್ಞೆ. ಈ ಕಾರ್ಯವನ್ನು ಅವರು ನನಗೆ ವಹಿಸಿರುವರು. ಯಾವುದಾದರೂ ಸ್ಥಳವನ್ನು ನೋಡುವುದಕ್ಕೆ ನಮ್ಮಲ್ಲಿ ಯಾರಾದರೂ ಹೋಗುವುದಕ್ಕೆ ಅಭ್ಯಂತರವೇನೂ ಇಲ್ಲ. ಆದರೆ ಇದು ತತ್ಕಾಲಕ್ಕೆ ಮಾತ್ರ. ಅವರ ಅಭಿಪ್ರಾಯವೇನೆಂದರೆ, ಮನೆ ಮಠಗಳಿಲ್ಲದೆ ಸರ್ವದಾ ಪರಿವ್ರಾಜಕನಾಗಿ ಅಲೆಯುವುದು ಪೂರ್ಣತೆಯ ಪರಾಕಾಷ್ಠತೆಯನ್ನು ಮುಟ್ಟಿದವನಿಗೆ ಮಾತ್ರ ಸರಿಹೋಗುವುದು. ಆದರೆ ಅದಕ್ಕೆ ಮುಂಚೆ ಎಲ್ಲಿಯಾದರೂ ನಿಂತು ಸಾಧನೆಯಲ್ಲಿ ನಿರತನಾಗಿರುವುದು ಒಳ್ಳೆಯದು. ದೇಹ ಮತ್ತು ಇತರ ಉಪಾಧಿಗಳೆಲ್ಲ ತಮಗೆ ತಾವೆ ಕಳಚಿ ಬಿದ್ದಮೇಲೆ ತನಗೆ ಯಾವ ಭಾವ ಬರುವುದೋ ಅದರಂತೆ ನಡೆಯಬಹುದು. 

 ೪. ಅವರ ಆಜ್ಞೆಯ ಪ್ರಕಾರ ಸಂನ್ಯಾಸಿಗಳ ಸಂಸ್ಥೆಯೊಂದು ವರಾಹನಗರದಲ್ಲಿರುವ ಒಂದು ಪಾಳುಮನೆಯಲ್ಲಿ ಸ್ಥಾಪಿತವಾಗಿದೆ. ಬಾಬು ಸುರೇಶಚಂದ್ರಮಿತ್ರ ಮತ್ತು ಬಾಬು ಬಲರಾಮಬೋಸ್ ಇಬ್ಬರು ಗೃಹಸ್ಥಭಕ್ತರು ಇದುವರೆವಿಗೂ ಅವರಿಗೆ ಮನೆ ಬಾಡಿಗೆ ಮತ್ತು ಆಹಾರವನ್ನು ಒದಗಿಸುತ್ತಿರುವರು. 

 ೫. ಅನೇಕ ಕಾರಣಗಳಿಂದ ಭಗವಾನ್ ಶ‍್ರೀರಾಮಕೃಷ್ಣರ ಮೃತದೇಹವನ್ನು ದಹಿಸಬೇಕಾಯಿತು. ಇದು ಬಹಳ ನಿಂದಾಸ್ಪದವಾದುದೆಂಬುದರಲ್ಲಿ ಸಂದೇಹವಿಲ್ಲ. ಅವರ ಬೂದಿಯನ್ನು ರಕ್ಷಿಸಿರುವೆವು. ಎಲ್ಲಿಯಾದರೂ ಗಂಗಾನದಿಯ ತೀರದಲ್ಲಿ ಅದನ್ನು ಸರಿಯಾಗಿ ಒಂದು ಮಠದಲ್ಲಿ ಸ್ಥಾಪನೆ ಮಾಡಿದರೆ ನಮ್ಮ ಪಾಪದಿಂದ ಸ್ವಲ್ಪವಾದರೂ ಪಾರಾಗಬಹುದೆಂದು ಯೋಚಿಸುತ್ತೇನೆ. ಅವರ ಭಸ್ಮಾವಶೇಷ ಅವರ ಆಸನ, ಅವರ ಭಾವಚಿತ್ರ ಇವುಗಳನ್ನು ನಮ್ಮ ಮಠದಲ್ಲಿ ಯಥೋಚಿತವಾಗಿ ಪೂಜಿಸುತ್ತಿರುವೆವು. ಬ್ರಾಹ್ಮಣ ಮತಸ್ಥರಾದ ಗುರುಭಾಯಿಯೊಬ್ಬರು ಈ ಕಾರ‍್ಯದಲ್ಲಿ ಹಗಲು ರಾತ್ರಿ ನಿರತರಾಗಿರುವರೆಂಬುದು ನಿಮಗೆ ಗೊತ್ತಿದೆ. 

 ೬. ಈ ವಂಗದೇಶದಲ್ಲಿ ತಮ್ಮ ಸಾಧನೆಯ ಕಾಲವನ್ನು ಕಳೆದ ಸ್ಥಳದಲ್ಲಿ ಅವರಿಗೆ ಒಂದು ಸ್ಮಾರಕ ಮಂದಿರವನ್ನು ಕಟ್ಟದೇ ಇರುವುದಕ್ಕಿಂತ ಹೆಚ್ಚು ಪಶ್ಚಾತ್ತಾಪಕರವಾದುದು ಯಾವುದಿರುವುದು? ಅವರು ಇಲ್ಲಿ ಜನ್ಮವೆತ್ತಿದುದರಿಂದ ವಂಗದೇಶವೇ ಪಾವನವಾಗಿದೆ, ಈ ನಾಡು ಪುಣ್ಯಭೂಮಿಯಾಗಿದೆ. ಪಾಶ್ಚಾತ್ಯ ನಾಗರಿಕತೆಯ ಸಮ್ಮೋಹನಾಸ್ತ್ರದಿಂದ ಭಾರತೀಯರನ್ನು ಎಚ್ಚರಿಸುವುದಕ್ಕೆ ಅವರು ಬಂದರು. ಅದಕ್ಕೋಸುಗವೇ ತಮ್ಮ ಸರ್ವಸಂಗ ಪರಿತ್ಯಾಗ ಮಾಡಿದ ಶಿಷ್ಯರನ್ನು ಪ್ರೌಢ ವಿದ್ಯಾಲಯಗಳಿಂದ ಆರಿಸಿದರು. 

 ೭. ಮೇಲೆ ಹೇಳಿದ ಗೃಹಸ್ಥ ಶಿಷ್ಯರೊಬ್ಬರಿಗೆ ಗಂಗಾನದಿಯ ತೀರದಲ್ಲಿ ಸ್ವಲ್ಪ ಜಾಗವನ್ನು ಕೊಳ್ಳಬೇಕು, ಅಲ್ಲಿ ಅವರ ಅಸ್ಥಿಯನ್ನು ಸ್ಥಾಪನೆಮಾಡಿ ಒಂದು ಸ್ಮಾರಕ ಮಂದಿರವನ್ನು ಕಟ್ಟಿ ಅವರ ಶಿಷ್ಯರೆಲ್ಲರೂ ಅಲ್ಲಿ ವಾಸಿಸಬೇಕು ಎಂಬ ಆಸೆ ಇತ್ತು. ಸುರೇಶಬಾಬು ಆಗಲೇ ಒಂದು ಸಾವಿರ ರೂಪಾಯಿಗಳನ್ನು ಕೊಟ್ಟಿದ್ದರು. ಇನ್ನೂ ಹೆಚ್ಚನ್ನು ಕೊಡುತ್ತೇನೆಂದೂ ಭರವಸೆ ಇತ್ತಿದ್ದರು. ಅಯ್ಯೋ! ಅವರನ್ನು ದೇವರು ಕೆಲವು ಕಾರಣಾಂತರಗಳಿಂದ ನೆನ್ನೆ ಜಗದಿಂದ ಒಯ್ದನು. ಬಲರಾಮ ಬಾಬುವಿನ ಮರಣದ ವಿಷಯ ನಿಮಗೆ ಆಗಲೇ ಗೊತ್ತಿದೆ. 

 ೮. ಈಗ ಅವರ ಶಿಷ್ಯರು, ಅವರ ಪವಿತ್ರ ಭಸ್ಮಾವಶೇಷ ಮತ್ತು ಪೀಠದೊಂದಿಗೆ ಎಲ್ಲಿ ನಿಲ್ಲುತ್ತಾರೆ ಎಂಬುದು ಗೊತ್ತಿಲ್ಲ. (ವಂಗದೇಶದಲ್ಲಿ ಜನರು ಮಾತಿನ ಮಲ್ಲರು. ಕಾರ್ಯತಃ ಎಳ್ಳಷ್ಟೂ ಪ್ರಯೋಜನವಿಲ್ಲವೆಂಬುದು ನಿಮಗೂ ಗೊತ್ತಿದೆ.) ಅವರ ಸಂನ್ಯಾಸೀ ಶಿಷ್ಯರು ತ್ಯಾಗದ ದಾರಿ ಎಲ್ಲಿಗೆ ಒಯ್ಯುವುದೊ ಅಲ್ಲಿಗೆ ಹೋಗಲು ಸಿದ್ಧರಾಗಿರುವರು. ಆದರೆ ನಾನು ಅವರ ಸೇವಕ. ಈಗ ಬಹಳ ವ್ಯಥೆಗೆ ಈಡಾಗಿರುವೆನು. ಭಗವಾನ್ ಶ‍್ರೀರಾಮಕೃಷ್ಣರ ಭಸ್ಮಾವಶೇಷವನ್ನು ಇಡಲು ಒಂದು ಸಣ್ಣ ಸ್ಥಳ ಸಿಕ್ಕಲಿಲ್ಲವಲ್ಲ ಎಂದು ನನ್ನ ಹೃದಯ ದುಃಖದಿಂದ ಕೊರಗುತ್ತಿದೆ. 

 ೯. ಕೇವಲ ಒಂದು ಸಾವಿರ ರೂಪಾಯಿನಿಂದ ಸ್ಥಳವನ್ನು ಕೊಂಡು ಒಂದು ಮಂದಿರವನ್ನು ಕಲ್ಕತ್ತೆಯ ಸಮೀಪದಲ್ಲಿ ಕಟ್ಟುವುದು ಸಾಧ್ಯವಿಲ್ಲ. ಅಂತಹ ಸ್ಥಳಕ್ಕೆ ಐದು ಸಾವಿರದವರೆಗಾದರೂ ಬೇಕಾಗುವುದು. 

 ೧೦. ನೀವು ಒಬ್ಬರೆ ಈಗ ಶ‍್ರೀರಾಮಕೃಷ್ಣರ ಶಿಷ್ಯರಿಗೆ ಇರುವ ಸ್ನೇಹಿತರು ಮತ್ತು ಉಪಕಾರಿಗಳು. ವಾಯುವ್ಯ ಪ್ರಾಂತ್ಯದಲ್ಲಿ ನಿಮ್ಮ ಕೀರ್ತಿ ಮತು ಗೌರವ ಪರಿಚಿತರಾದವರು ಬಹಳ ಮಂದಿ ಇರುವರು. ನಿಮ್ಮ ಪ್ರಾಂತ್ಯದಲ್ಲಿ ಪರಿಚಿತರಾದ ಕೆಲವು ಶ‍್ರೀಮಂತರಾದ ಧರ್ಮಸ್ಥರಿಂದ ಚಂದಾ ಎತ್ತುವಂತೆ ನಿಮಗೆ ಎನಿಸಿದರೆ, ಅದು ಸೂಕ್ತವಾಗಿ ತೋರಿದರೆ, ಅದನ್ನು ದಯವಿಟ್ಟು ವಿಚಾರಮಾಡಿ ನೋಡಿ. ವಂಗದೇಶದಲ್ಲಿ, ಗಂಗಾನದಿಯ ತೀರದಲ್ಲಿ, ಭಗವಾನ್ ಶ‍್ರೀರಾಮಕೃಷ್ಣರಿಗೆ ಮತ್ತು ಅವರ ಶಿಷ್ಯರು ವಾಸಿಸುವುದಕ್ಕೆ ಒಂದು ಸ್ಮಾರಕ ಮಂದಿರವನ್ನು ಕಟ್ಟುವುದು ಸೂ ಕ್ತವೆಂದು ತೋರಿಬಂದರೆ, ನೀವು ನನಗೆ ಅನುಮತಿ ಕೊಟ್ಟರೆ, ನಾನು ಅಲ್ಲಿಗೆ ಬರುವುದಕ್ಕೆ ಸಿದ್ಧನಾಗಿರುವೆನು. ನನ್ನ ಗುರುದೇವನ ಮತ್ತು ಶಿಷ್ಯರ ಮಹಾಪವಿತ್ರವಾದ ಆ ಕಾರ‍್ಯಕ್ಕೆ ಮನೆಯಿಂದ ಮನೆಗೆ ಭಿಕ್ಷವೆತ್ತಲು ಹೋಗುವುದಕ್ಕೆ ನನಗೆ ಸ್ವಲ್ಪವಾದರೂ ಅನುಮಾನವಿಲ್ಲ. ವಿಶ್ವನಾಥನನ್ನು ಧ್ಯಾನಿಸಿ ಈ ಸಲಹೆಗಳ ಮೇಲೆ ದೀರ್ಘಾಲೋಚನೆ ಮಾಡಿ ನೋಡಿ. ನಿಷ್ಕಪಟಿಗಳಾದ ವಿದ್ಯಾವಂತರು, ಯುವಕರು, ಒಳ್ಳೆಯ ಮನೆತನಗಳಿಂದ ಬಂದ ಸಂನ್ಯಾಸಿಗಳು ಕೇವಲ ಇಳಿದುಕೊಳ್ಳುವುದಕ್ಕೆ ಒಂದು ಮನೆ ಮತ್ತು ಇತರ ಸಹಾಯಗಳಿಲ್ಲದೆ ಶ‍್ರೀರಾಮಕೃಷ್ಣರ ಆದರ್ಶಗಳಿಗೆ ಸರಿಯಾಗಿ ಬಾಳಲಾಗಲಿಲ್ಲವೆಂದರೆ ಅದು ನಮ್ಮ ದೇಶದ ದುರ್ದೈವವಷ್ಟೆ. 

 ೧೧. ‘ನೀವು ಸಂನ್ಯಾಸಿಗಳು. ಈ ಆಸೆಗಳನ್ನೆಲ್ಲ ಮನಸ್ಸಿಗೆ ಏತಕ್ಕೆ ಹಚ್ಚಿಕೊಳ್ಳುತ್ತೀರಿ?’ ಎಂದು ನೀವು ನನ್ನನ್ನು ಕೇಳಿದರೆ ನಾನು ಹೀಗೆ ಹೇಳುವೆನು: ನಾನು ಶ‍್ರೀರಾಮಕೃಷ್ಣರ ದಾಸ; ಅವರು ಜನ್ಮವೆತ್ತಿ ಸಾಧನೆಮಾಡಿದ ದೇಶದಲ್ಲಿ ಅವರ ಹೆಸರನ್ನು ಸ್ಥಾಪಿಸುವುದಕ್ಕೆ, ಅವರ ಆದರ್ಶಗಳನ್ನು ಅನುಷ್ಠಾನಕ್ಕೆ ತರಲು ಪ್ರಯತ್ನಿಸುವ ಶಿಷ್ಯರಿಗೆ ಸ್ವಲ್ಪ ಸಹಾಯ ಮಾಡುವುದಕ್ಕೆ, ನಾನು ಕಳ್ಳತನ, ದರೋಡೆಯನ್ನು ಮಾಡಲೂ ಕೂಡ ಸಿದ್ಧನಾಗಿರುವೆನು. ಅದಕ್ಕೆ ನನ್ನ ಹೃದಯದಲ್ಲಿರುವ ವಿಷಯಗಳನ್ನು ನಿಮ್ಮ ಮುಂದೆ ಇಡುವೆನು. ಇದಕ್ಕೋಸ್ಕರವಾಗಿಯೇ ನಾನು ಕಲ್ಕತ್ತೆಗೆ ಪುನಃ ಬಂದಿರುವೆನು. ನಾನು ನಿಮ್ಮಲ್ಲಿಂದ ಹೊರಡುವುದಕ್ಕೆ ಮುಂಚೆ ನಿಮಗೆ ಎಲ್ಲವನ್ನೂ ಹೇಳಿರುತ್ತೇನೆ. ಈಗ ನಿಮಗೆ ಯಾವುದು ಸೂಕ್ತವೆಂದು ತೋರುವುದೋ ಅದನ್ನು ಮಾಡಲು ಬಿಟ್ಟಿರುವೆನು. 

 ೧೨. ಕಾಶಿ ಮುಂತಾದ ಸ್ಥಳಗಳಲ್ಲಿ ನೀವು ಅದನ್ನು ಸ್ಥಾಪಿಸುವುದು ಉತ್ತಮವೆಂದು ವಾದಿಸಿದರೆ ಅವರು ಜನ್ಮವೆತ್ತಿದ, ಸಾಧನೆಮಾಡಿದ ಸ್ಥಳದಲ್ಲಿ ಸ್ಮಾರಕ ಮಂದಿರವನ್ನು ಕಟ್ಟದೆ ಇದ್ದರೆ ಬಹಳ ಶೋಚನೀಯವಾದ ಸಂಗತಿ ಎಂದು ನಾನು ಹೇಳಬೇಕಾಗುವುದು! ವಂಗದೇಶದಲ್ಲಿ ಸ್ಥಿತಿ ಅತಿಶೋಚನೀಯವಾಗಿದೆ. ತ್ಯಾಗವೆಂದರೆ ಅದರ ನಿಜವಾದ ಅರ್ಥವೇನೆಂಬುದನ್ನು ನಮ್ಮ ಜೀವನದ ಮೂಲಕ ತೋರಿಸಬೇಕಾಗಿದೆ. ಭಗವಂತನು ತ್ಯಾಗವನ್ನೂ ನಿಷ್ಪ್ರಾಪಂಚಿಕತೆಯನ್ನೂ ಇಲ್ಲಿಗೆ ದಯಪಾಲಿಸಲಿ. ಇಲ್ಲಿಯ ಜನಗಳಿಗೆ ಹೊಗಳಿಕೊಳ್ಳುವುದಕ್ಕೆ ಏನೂ ಇಲ್ಲ. ಇಂತಹ ಧರ್ಮಕಾರ್ಯಗಳಲ್ಲಿ ವಾಯುವ್ಯ ಪ್ರಾಂತ್ಯದ ಜನರು ಬಹಳ ಉತ್ಸಾಹಿಗಳು. ದಯವಿಟ್ಟು ನಿಮಗೆ ಯಾವುದು ಸೂಕ್ತವೆಂದು ತೋರುವುದೋ ಆ ಉತ್ತರವನ್ನು ಕಳುಹಿಸಿ. ಗ-ಇನ್ನೂ ಬಂದಿಲ್ಲ. ಬಹುಶಃ ನಾಳೆ ಬರಬಹುದು. ಅವನನ್ನು ಪುನಃ ನೋಡಲು ತುಂಬಾ ಕಾತರನಾಗಿದ್ದೇನೆ. ದಯವಿಟ್ಟು ಮೇಲಿನ ವಿಳಾಸಕ್ಕೆ ಪತ್ರವನ್ನು ಬರೆಯಿರಿ.” 

 ಈ ಪತ್ರವನ್ನು ನೋಡಿದರೆ, ಸ್ವಾಮೀಜಿಯವರ ಮನಸ್ಸು ಶ‍್ರೀರಾಮಕೃಷ್ಣರ ಆದರ್ಶಕ್ಕೆ ತಕ್ಕಂತೆ ಒಂದು ಸುಬದ್ಧ ಸಂಸ್ಥೆಯನ್ನು ಕುರಿತು ಆಲೋಚಿಸುವುದು ಮತ್ತು ಅದನ್ನು ಕಾರ‍್ಯಗತ ಮಾಡಲು ತವಕಪಡುತ್ತಿರುವುದು ನಮಗೆ ಕಾಣುವುದು. 

 ವಿವೇಕಾನಂದರು ಸುಮಾರು ಒಂದು ತಿಂಗಳು ಬಾರಾನಗರ ಮಠದಲ್ಲಿದ್ದರು. ಅಖಂಡಾನಂದರು, ಟಿಬೆಟ್ಟಿನಲ್ಲಿರುವ ಲಾಮಾಗಳು, ಅವರ ಆಚಾರ ವ್ಯವಹಾರ, ಕಾಶ್ಮೀರದ ಸೌಂದರ‍್ಯ, ಭವ್ಯ ಕೇದಾರ ಮತ್ತು ಬದರಿಯ ತೀರ್ಥಸ್ಥಳಗಳನ್ನೆಲ್ಲ ನೋಡಿಕೊಂಡು ಬಂದು ಅವುಗಳ ಬಗ್ಗೆ ಸ್ವಾಮೀಜಿ ಮನಸ್ಸನ್ನು ಆಕರ್ಷಿಸುವಂತೆ ಹೇಳತೊಡಗಿದರು. ಅಖಂಡಾನಂದರೊಡನೆ ಹಿಮಾಲಯವನ್ನು ನೋಡಲು ಮತ್ತು ಅಲ್ಲಿ ಸಾಧನೆ ಮಾಡಲು ನಿಶ್ಚಯಿಸಿದರು. ೬ನೇ ಜುಲೈ ೧೮೯೦ರಲ್ಲಿ ಬರೆದ ಕಾಗದದಲ್ಲಿ ಹೀಗೆ ವಿವರಿಸುವರು: “ನನ್ನ ದಾರಿಯ ಖರ್ಚಿಗೆ ಸ್ವಲ್ಪ ದುಡ್ಡು ಸಿಕ್ಕಿದೊಡನೆಯೇ ಆಲ್ಮೋರದವರೆಗೂ ಹೊಗಿ ಅಲ್ಲಿಂದ ಗರ‍್ವಾಲ್ ಪ್ರಾಂತ್ಯದಲ್ಲಿ ಗಂಗಾತೀರದಲ್ಲಿ ಎಲ್ಲಿಯಾದರೂ ದೀರ್ಘ ಧ್ಯಾನದಲ್ಲಿ ತಲ್ಲೀನನಾಗಬೇಕೆಂದಿರುವೆನು. ಗಂಗಾಧರ (ಅಖಂಡಾನಂದ) ನನ್ನೊಡನೆ ಬರುವನು. ಅದಕ್ಕಾಗಿಯೇ ನಾನು ಕಾಶ್ಮಿರದಿಂದ ಅವನನ್ನು ಕರೆಸಿದ್ದು. ಹಿಮಾಲಯಕ್ಕೆ ಹಾರಿಹೋಗಬೇಕೆಂದು ನಾನು ತವಕಪಡುತ್ತಿರುವೆನು.” ಅವರು ತಮ್ಮ ಗುರುಭಾಯಿಗಳಿಗೆ: “ನನ್ನ ಸ್ಪರ್ಶಮಾತ್ರವೆ ಮನುಷ್ಯನನ್ನು ಬದಲಾಯಿಸುವಷ್ಟು ತಪೋಶಕ್ತಿಯನ್ನು ಪಡೆಯುವವರೆಗೆ ನಾನು ಬರುವುದಿಲ್ಲ” ಎಂದು ಕಾಗದದಲ್ಲಿ ಬರೆದರು. ಹೋಗುವುದಕ್ಕೆ ಮುಂಚೆ ಗಂಗಾನದಿಯ ತೀರದ ಗುಸೂರಿ ಎಂಬ ಗ್ರಾಮದಲ್ಲಿ ಇದ್ದ ಶ‍್ರೀಶಾರದಾದೇವಿಯವರನ್ನು ಕಂಡು ಅವರಿಂದ ಆಶೀರ್ವಾದ ಪಡೆಯಲು ಹೋದರು. “ತಾಯಿ, ಶ್ರೇಷ್ಠ ಜ್ಞಾನ ದೊರಕುವವರೆಗೆ ನಾನು ಹಿಂತಿರುಗಿ ಬರುವುದಿಲ್ಲ” ಎಂದರು. ಶ‍್ರೀಶಾರದಾದೇವಿಯವರು ಸ್ವಾಮೀಜಿ ಹೊರಡುವುದಕ್ಕೆ ಮುಂಚೆ “ನಿನ್ನ ತಾಯಿಯನ್ನು ನೋಡುವುದಿಲ್ಲವೆ?” ಎಂದು ಅವರನ್ನು ಕೇಳಿದರು. “ನೀವು ಒಬ್ಬರೇ ನನ್ನ ತಾಯಿ” ಎಂದರು ಸ್ವಾಮೀಜಿ. ಶ‍್ರೀಶಾರದಾದೇವಿಯವರು ಶ‍್ರೀರಾಮಕೃಷ್ಣರ ಶಿಷ್ಯಾಗ್ರಣಿ ನರೇಂದ್ರನನ್ನು ಹೃತ್ಪೂರ್ವಕವಾಗಿ ಹರಸಿ ಕಳುಹಿಸಿದರು.

