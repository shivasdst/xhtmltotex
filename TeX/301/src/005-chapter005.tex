
\chapter{ಶ‍್ರೀರಾಮಕೃಷ್ಣ}

ಭರತಖಂಡದ ಇತಿಹಾಸದಲ್ಲಿ ಹತ್ತೊಂಭತ್ತನೆಯ ಶತಮಾನ ಅತ್ಯಂತ ಮುಖ್ಯವಾದುದು. ಕನ್ಯಾಕುಮಾರಿಯಿಂದ ಹಿಮಾಲಯದವರೆಗೆ ಇರುವ ಇಡೀ ಭರತ ಖಂಡ ಆಂಗ್ಲ ಸಾಮ್ರಾಜ್ಯದ ಶ್ವೇತಚ್ಛತ್ರದ ಅಡಿಗೆ ಬಂದಿತು. ಅವರ ಸರ್ಕಾರದೊಂದಿಗೆ ಅವರ ಧರ್ಮ ಅವರ ಸಂಸ್ಕೃತಿ ಅವರ ಸಾಹಿತ್ಯ ಅವರ ಆಚಾರ ವ್ಯವಹಾರಗಳೆಲ್ಲ ಭರತಖಂಡಕ್ಕೆ ಬಂದು ಕ್ರಮೇಣ ಇಲ್ಲಿ ಬೇರುಬಿಡಲು ಆರಂಭವಾಯಿತು. ಭರತಖಂಡದ ಸಂಸ್ಕೃತಿಗೂ ಐರೋಪ್ಯ ಸಂಸ್ಕೃತಿಗೂ ಒಂದು ಘರ್ಷಣೆ ಪ್ರಾರಂಭವಾಯಿತು. ಆ ಘರ್ಷಣೆಯಲ್ಲಿ ಹಲವು ಒಳ್ಳೆಯವೂ ಕೆಟ್ಟವೂ ಇದ್ದವು. ವೈಜ್ಞಾನಿಕ ದೃಷ್ಟಿ, ಯಂತ್ರ ಸಾಮಗ್ರಿಗಳ ಉತ್ಪಾದನೆ, ರೈಲು ಅಂಚೆ ತಂತಿ – ಇವೇ ಮುಂತಾದ ಸೌಲಭ್ಯಗಳು ನಮಗೆ ದೊರಕಿದವು. ಸಾಹಿತ್ಯ ಕಲೆ ಮುಂತಾದ ಕ್ಷೇತ್ರಗಳಲ್ಲಿ ಒಂದು ಸ್ತೋತ್ರ ಆರಂಭವಾಯಿತು. ಇವುಗಳೆಲ್ಲ ನಮಗೆ ಆವಶ್ಯಕ ಎಂದು ಬೇಕಾದರೆ ಹೇಳಬಹುದು. ಆದರೆ ಇದರೊಡನೆ ಒಂದು ಮಹಾ ಅಪಾಯವೂ ಬಂದೊದಗಿತು. ಹಿಂದೂಗಳಲ್ಲಿ ತಮ್ಮ ಆಚಾರ ವ್ಯವಹಾರ ಧರ್ಮ ಇವುಗಳ ಮೇಲಿದ್ದ ಗೌರವವೇ ಹೋಗಿಬಿಡುವುದರಲ್ಲಿತ್ತು. ನಮ್ಮದನ್ನೆಲ್ಲ ಕಳೆದುಕೊಂಡು ಅನ್ಯರ ಅನುಕರಣೆಯಲ್ಲಿ ಕಾಲಯಾಪನೆ ಮಾಡುವ ಪರಿಸ್ಥಿತಿ ಬರುವುದರಲ್ಲಿತ್ತು. ಆಗ ಎರಡು ಪಂಥದವರು ಇದ್ದರು. ಒಂದು ಆಂಗ್ಲೇಯ ವಿದ್ಯಾಭ್ಯಾಸದ ಪರಿಣಾಮದಿಂದ ಅವರ ದೃಷ್ಟಿಯನ್ನು ತೆಗೆದುಕೊಂಡು ಪ್ರತಿಭಾವಂತರಾಗಿ ಕಾರ‍್ಯಕ್ಷೇತ್ರದಲ್ಲೆಲ್ಲ ಬೆಳಗುತ್ತಿದ್ದವರು. ಅಧಿಕಾರ, ಐಶ್ವರ‍್ಯ, ಅಂತಸ್ತು ಎಲ್ಲಾ ಅವರ ಪಾಲಾಗಿತ್ತು. ಮತ್ತೊಂದು, ಅದು ಬಹಳ ದುರ್ಬಲವಾದ ವರ್ಗ ಎಂದು ಹೇಳಬಹುದು, ಪೂರ್ವಾಚಾರ ಪರಾಯಣರ ಗುಂಪು. ಹೊಸದನ್ನೆಲ್ಲ ಅನುಮಾನದಿಂದ ನೋಡುವುದು ಪವಿತ್ರವಾಗಿರುವುದೆಲ್ಲ ಹಿಂದೆ ಇತ್ತು ನಾವು ಅದನ್ನೇ ಅನುಸರಿಸಬೇಕು; ಏನಾದರೂ ಅದನ್ನು ತ್ಯಜಿಸಕೂಡದು ಎಂದು ಹಿಂದಿನದನ್ನು ಕಪಿಮುಷ್ಟಿಯಲ್ಲಿ ಹಿಡಿದುಕೊಂಡಿದ್ದವರು. ಧಾರ್ಮಿಕ ಜೀವನದಲ್ಲಿ ಇವೆರಡರ ಮಧ್ಯೆ ಒಂದು ಸಾಮರಸ್ಯವನ್ನು ತರಬೇಕಾಗಿತ್ತು. ಹೊಸತು ಹಳೆಯದರಲ್ಲಿ ಒಂದು ಸಾಮರಸ್ಯವನ್ನು ತರಬೇಕಾಗಿತ್ತು. ಮತ್ತು ಭರತಖಂಡ ಹತ್ತೊಂಭತ್ತನೆಯ ಶತಮಾನದ ಹೊತ್ತಿಗೆ ಯಾವುದೋ ಒಂದು ಧರ್ಮದ ಜನರಿಗೆ ಮಾತ್ರ ಮೀಸಲಾಗಿರಲಿಲ್ಲ. ಇಲ್ಲಿ ಭಿನ್ನ ಧರ್ಮದವರು ಇದ್ದರು. ಅವರಲ್ಲೇ ಒಂದು ಸಾಮರಸ್ಯವನ್ನು ತರುವಂತಹ ಒಂದು ವ್ಯಕ್ತಿ ಬೇಕಾಗಿತ್ತು. ದೇಹದಲ್ಲಿ ಶಕ್ತಿ ಇದ್ದರೆ ಹೇಗೆ ಹೊರಗಿನಿಂದ ವಿಷಕ್ರಿಮಿಗಳು ಬಂದರೆ ಅದನ್ನು ಹೊರದೂಡಲು ಹೊಸ ಜೀವಾಣುಗಳನ್ನು ಸೃಜಿಸುವುದೋ ಹಾಗೆಯೇ ಒಂದು ಧರ್ಮದಲ್ಲಿ ಶಕ್ತಿ ಸಾಮರ್ಥ್ಯವಿದ್ದರೆ ಬರುವ ಸಮಸ್ಯೆಯನ್ನು ಬಗೆಹರಿಸಬಲ್ಲಂತಹ ವ್ಯಕ್ತಿಗಳನ್ನು ಸೃಜಿಸುವುದು. ಆ ಶಕ್ತಿ ಇಲ್ಲದಿದ್ದರೆ ಅಂಧ ಅನುಕರಣೆಗೆ ಕೈಹಾಕಿ ತನ್ನದನ್ನೆಲ್ಲ ಕಳೆದುಕೊಳ್ಳುವುದು. ಹಿಂದೂಸಂಸ್ಕೃತಿ ಒಂದು ದೃಷ್ಟಿಯಿಂದ ನೋಡಿದರೆ ಬಹಳ ಪುರಾತನ ಸಂಸ್ಕೃತಿ. ಆದರೆ ಶಕ್ತಿ ಸಾಮರ್ಥ್ಯವನ್ನು ಕಳೆದುಕೊಂಡು ವೃದ್ಧಾಪ್ಯಕ್ಕೆ ತುತ್ತಾದ ಸಂಸ್ಕೃತಿಯಲ್ಲ. ಭರತ ಖಂಡದ ಧರ್ಮದ ನಾಡಿನಲ್ಲಿ ಬಿಸಿರಕ್ತ ಹರಿಯುತ್ತಿತ್ತು. ಭಿನ್ನತೆಯ ಹಿಂದೆ ಮುಳುಗಿ ಅದರಲ್ಲಿ ಇರುವ ಐಕ್ಯತೆಯನ್ನು ಕಂಡು ಹಿಡಿದು, ಯಾವ ಧರ್ಮವೂ ಮತ್ತೊಂದು ಆಗಬೇಕಾಗಿಲ್ಲ; ಎಲ್ಲವೂ ಒಂದೇ ನಿತ್ಯಧರ್ಮದ ಅನಂತ ಸಾಗರದ ಅಲೆಗಳು ಎಂಬುದನ್ನು ಸಾರಿ ತೋರುವಂತಹ ಒಂದು ವ್ಯಕ್ತಿಯನ್ನು ಸೃಜಿಸಿತು. ಅದೇ ಶ‍್ರೀರಾಮಕೃಷ್ಣ ಪರಮಹಂಸರು. ಒಂದು ವಿಚಿತ್ರವೇನೆಂದರೆ ಹತ್ತೊಂಭತ್ತನೇ ಶತಮಾನದ ಮೇರುಸದೃಶ ವ್ಯಕ್ತಿಯಂತಿದ್ದ ಇವರು ಯಾವ ಸ್ಕೂಲು ಕಾಲೇಜುಗಳಿಗೂ ಹೋದವರಲ್ಲ. ಆದರೆ ಅದರಲ್ಲಿ ನುರಿತ ಮಹಾ ನಿಪುಣರು ಇವರ ಪದತಲದಲ್ಲಿ ಕುಳಿತು, ಇವರ ವಾಣಿಯನ್ನು ಕೇಳಿ, ತಮ್ಮ ಬಾಳನ್ನು ಪವಿತ್ರ ಮಾಡಿಕೊಂಡರು.

ಶ‍್ರೀರಾಮಕೃಷ್ಣರು ಆಧುನಿಕ ನಾಗರೀಕತೆಗೆ ದೂರವಾಗಿದ್ದ ಕಾಮಾರಪುಕುರವೆಂಬ ಕಲ್ಕತ್ತೆಗೆ ಸುಮಾರು ಐವತ್ತು ಮೈಲಿ ದೂರದಲ್ಲಿದ್ದ ಒಂದು ಗ್ರಾಮದಲ್ಲಿ ೧೮ನೇ ಮಾರ್ಚಿ, ೧೮೩೬ರಲ್ಲಿ ಜನಿಸಿದರು. ಅವರ ತಂದೆ ತಾಯಿಗಳು ಬ್ರಾಹ್ಮಣ ಕುಟುಂಬಕ್ಕೆ ಸೇರಿದ ಸಾತ್ತ್ವಿಕ ದಂಪತಿಗಳು. ಖುದೀರಾಮ ಚಟ್ಟೊಪಾಧ್ಯಾಯ ಮತ್ತು ಚಂದ್ರಮಣಿದೇವಿಯೇ ಆ ದೈವದಂತಿಗಳು. ಶ‍್ರೀರಾಮಕೃಷ್ಣರ ಬಾಲ್ಯಜೀವನ ವಿಚಿತ್ರವಾಗಿತ್ತು. ಅವರಿಗೆ ಸ್ಕೂಲು ಎಂಬುದು ಬಾಲ್ಯದಿಂದಲೂ ಹಿಡಿಸುತ್ತಿರಲಿಲ್ಲ. ಅದಕ್ಕಿಂತ ಹೆಚ್ಚಾಗಿ ಭಜನೆ ಪ್ರಾರ್ಥನೆ, ಪ್ರಕೃತಿಯ ರಮ್ಯದೃಶ್ಯಗಳನ್ನು ನೋಡಿದಾಗ ಇಂದ್ರಿಯಾತೀತವಾದ ಭಾವಾವಸ್ಥೆಗೆ ಹೋಗುವುದು ಇವುಗಳಲ್ಲೆ ಆಸಕ್ತಿ. ಸತ್ಯನಿಷ್ಠರು, ಎಂದಿಗೂ ಮಾತಿಗೆ ತಪ್ಪುವವರಲ್ಲ ಅವರು. ಶ‍್ರೀರಾಮಕೃಷ್ಣರ ಉಪನಯನದ ಸಮಯದಲ್ಲಿ ಧನಿ ಎಂಬ ಅಕ್ಕಸಾಲಿಗ ಹೆಂಗಸಿನಿಂದ ಪ್ರಥಮಭಿಕ್ಷೆಯನ್ನು ಸ್ವಿಕರಿಸಿದರು, ಯಾರು ಬೇಡವೆಂದರೂ ಕೇಳಲಿಲ್ಲ. ಒಮ್ಮೆ ಅವಳಿಗೆ ಮಾತು ಕೊಟ್ಟಿರುವೆ, ಅದರಂತೆ ನಾನು ನಡೆಯುತ್ತೇನೆ ಎಂದು ಹಟಹಿಡಿದರು. ಬಾಲ್ಯದಿಂದಲೂ ಆಧ್ಯಾತ್ಮಿಕ ಪ್ರಕೃತಿ ಅವರದು. ಎಲ್ಲರಿಗೂ ಅಂಜಿಕೆಯಾಗುವಂತಹ ಹತ್ತಿರಿದ್ದ ಸ್ಮಶಾನಕ್ಕೆ ಹೋಗಿ ಅಲ್ಲಿ ದೇವರನ್ನು ಕುರಿತು ತಪಸ್ಸು ಮಾಡುತ್ತಿದ್ದರು. ದೇವರಿಗೆ ಸಂಬಂಧಪಟ್ಟ ಹಾಡುಗಳನ್ನು ಒಂದು ಸಲ ಕೇಳಿದರೆ ಸಾಕು ಲೀಲಾಜಾಲವಾಗಿ ಹಾಡಿಬಿಡುತ್ತಿದ್ದರು. ಅವರ ಕಂಠವೋ ಅಸಾಧಾರಣವಾಗಿತ್ತು. ಪುರಾಣ ಪುಣ್ಯಕಥೆಗಳಿಗೆ ಸಂಬಂಧಿಸಿದ ಬಯಲು ನಾಟಕವನ್ನು ಒಮ್ಮೆ ನೋಡಿದರೆ ಸಾಕು ಅದರಂತೆ ಅಭಿನಯಿಸುತ್ತಿದ್ದರು. ಒಂದು ಸಲ ಊರಿನ ಜನರೆಲ್ಲ ಶಿವರಾತ್ರಿ ಜಾಗರಣೆಗೆಂದು ಒಂದು ಶಿವನಿಗೆ ಸಂಬಂಧಪಟ್ಟ ನಾಟಕ ಅಭಿನಯಿಸಿದರು. ಶ‍್ರೀರಾಮಕೃಷ್ಣರಿಗೆ ಶಿವನ ಪಾತ್ರ ಕೊಟ್ಟರು. ಶಿವನಂತೆ ಭಸ್ಮ ಲೇಪಿಸಿಕೊಂಡು ದಂಡಕಮಂಡಲುಧಾರಿಯಾಗಿ ರಂಗಭೂಮಿಯ ಮೇಲೆ ಶ‍್ರೀರಾಮಕೃಷ್ಣರು ನಿಂತೊಡನೆಯೆ ಶಿವನ ಭಾವದಲ್ಲಿ ತನ್ಮಯರಾಗಿ ಹೋದರು. ನಾಟಕ ಮುಂದುವರಿಯಲಾಗಲಿಲ್ಲ. ಹಲವು ದೇವದೇವಿಯವರ ಸುಂದರವಾದ ವಿಗ್ರಹಗಳನ್ನು ಮಾಡಿ ಪೂಜಿಸುತ್ತಿದ್ದರು, ಸುಂದರವಾದ ಚಿತ್ರಗಳನ್ನು ರಚಿಸುತ್ತಿದ್ದರು. ಆದರೆ ಶಾಲೆಯ ವಿದ್ಯೆ ಮಾತ್ರ ಇವರಿಗೆ ಹಿಡಿಸುತ್ತಿರಲಿಲ್ಲ.

ಶ‍್ರೀರಾಮಕೃಷ್ಣರ ತಂದೆ ಕಾಲವಾದಮೇಲೆ ಅವರ ಹಿರಿಯ ಮಗ ಕಲ್ಕತ್ತೆಗೆ ಹೊಗಿ ಅಲ್ಲಿ ಒಂದು ಸಂಸ್ಕೃತ ಪಾಠಶಾಲೆಯನ್ನು ತೆರೆದ. ಹಳ್ಳಿಯಲ್ಲೇ ಇದ್ದ ಶ‍್ರೀರಾಮಕೃಷ್ಣರನ್ನು ಕಲ್ಕತ್ತೆಗೆ ಕರೆಸಿ ಅಲ್ಲಿ ವಿದ್ಯಾಭ್ಯಾಸವನ್ನು ಮುಂದುವರಿಸಬೇಕೆಂದು ಅಣ್ಣ ಪ್ರಯತ್ನಿಸಿದ. ಅಣ್ಣ ಎಷ್ಟು ಪ್ರಯತ್ನ ಮಾಡಿದರು ತಮ್ಮ ಅದಕ್ಕೆ ಒಪ್ಪಲಿಲ್ಲ. “ಈ ವಿದ್ಯೆಯಿಂದ ದೇವರು ಸಿಕ್ಕುತ್ತಾನೇನು” ಎಂದು ಶ‍್ರೀರಾಮಕೃಷ್ಣರು ಅಣ್ಣನನ್ನು ಕೇಳಿದರು. ಆತ ಹೇಳಿದ: “ದೇವರಲ್ಲ ಮುಖ್ಯ ದೇಹಕ್ಕೆ, ಅನ್ನ ಬಟ್ಟೆ; ಅದು ಈ ವಿದ್ಯೆಯಿಂದ ದೊರಕುವುದು”. ಆದರೆ ಅಂತಹ ವಿದ್ಯೆ ತನಗೆ ಪ್ರಯೊಜನವಿಲ್ಲ, ಯಾವ ವಿದ್ಯೆ ಜನನ ಮರಣಗಳ ಕೋಟೆಯಿಂದ ತನ್ನನ್ನು ಪಾರುಮಾಡುವುದೋ ಅದು ತನಗೆ ಬೇಕೆಂದು ಹಟಹಿಡಿದ ತಮ್ಮ. ತಮ್ಮನ ಈ ಬಯಕೆಯನ್ನು ಅಣ್ಣ ಈಡೇರಿಸುವ ಸ್ಥಿತಿಯಲ್ಲಿರಲಿಲ್ಲ. “ಕೆಲವು ಮನೆಗೆ ಹೊಗಿ ದೇವರ ಪುಜೆಯನ್ನಾದರೂ ಮಾಡು” ಎಂದು ಹೇಳಿದನು. ಶ‍್ರೀರಾಮಕೃಷ್ಣರು ಅದನ್ನು ಸಂತೋಷದಿಂದ ಒಪ್ಪಿಕೊಂಡರು. ಆ ವೃತ್ತಿಯಲ್ಲಿ ನಿರತರಾಗಿದ್ದರು. ಆದರೆ ಅದರಿಂದ ಬರುವ ದಾನ ದಕ್ಷಿಣೆಗಾಗಿ ಅಲ್ಲ, ಮನೆಯಲ್ಲಿ ಭಗವಂತನ ಪುಜೆಯನ್ನು ಮಾಡುವ ಅವಕಾಶ ಸಿಕ್ಕಿದುದಕ್ಕಾಗಿ.

ಈ ಸಮಯದಲ್ಲಿ ಕಲ್ಕತ್ತೆಯ ಪ್ರಖ್ಯಾತ ಶ‍್ರೀಮಂತಳಾದ ರಾಣಿ ರಾಸ್ಮಣಿ ಎಂಬ ಬೆಸ್ತರ ಹೆಂಗಸು ಲಕ್ಷಾಂತರ ರೂಪಾಯಿ ಖರ್ಚಿನಿಂದ ಕಾಳಿ ದೇವಾಲಯವನ್ನು ನಿರ್ಮಿಸಿ ಯೋಗ್ಯನಾದ ಪೂಜಾರಿಗಾಗಿ ಕಾಯುತ್ತಿದ್ದಳು. ಶ‍್ರೀರಾಮಕೃಷ್ಣರ ಅಣ್ಣ ಮೊದಲು ಅಲ್ಲಿ ಪೂಜೆಗೆ ಒಪ್ಪಿಕೊಂಡು ಕೆಲವು ಕಾಲದ ಮೇಲೆ ಶ‍್ರೀರಾಮಕೃಷ್ಣರಿಗೆ ಆ ಕೆಲಸಮಾಡುವಂತೆ ಹೇಳಿ ಹಳ್ಳಿಗೆ ಹೋದವನು ಹಿಂತಿರುಗಲಿಲ್ಲ.

ಶ‍್ರೀರಾಮಕೃಷ್ಣರು ಕಾಳಿಕಾದೇವಾಲಯದಲ್ಲಿ ಪೂಜಾರಿಯಾಗಿ ಕಾಲಿಟ್ಟರು. ಆದರೆ ಒಬ್ಬ ಅವತಾರ ವ್ಯಕ್ತಿಯಾಗಿ ಅಲ್ಲಿ ಬದಲಾವಣೆಯಾದರು. ಪೂಜಾರಿಗಳ ಗುಂಪಿನಿಂದ ಒಬ್ಬ ಅವತಾರ ಪುರುಷ ಈ ಪ್ರಪಂಚಕ್ಕೆ ಬರುವುದು ಅಪರೂಪ. ಪೂಜೆ ಮಾಡುವುದಕ್ಕೆ ಗರ್ಭಗುಡಿಗೆ ಹೋದ ದಿನದಿಂದಲೇ ಶ‍್ರೀರಾಮಕೃಷ್ಣರು ತಾವು ಯಾವ ದೇವಿಯನ್ನು ಪೂಜೆ ಮಾಡುತ್ತಿರುವರೋ ಅದು ಕೇವಲ ಶಿಲ್ಪಿಗಳು ಕೊರೆದ ಒಂದು ಕಾಲ್ಪನಿಕ ಪ್ರತಿಮೆಯೋ, ಅಥವಾ ಅದರ ಹಿಂದೆ ಒಂದು ಚೈತನ್ಯವಿದೆಯೋ, ಇದ್ದರೆ ಅದನ್ನು ನೋಡಲು ಸಾಧ್ಯವೋ, ಅದಕ್ಕೆ ಏನು ಮಾಡಬೇಕು ಎಂದು ಯೋಚಿಸತೊಡಗಿದರು. ಪೂಜೆ ಮಾಡಿದಮೇಲೆ ಹತ್ತಿರದಲ್ಲಿರುವ ದಟ್ಟವಾದ ದಕ್ಷಿಣೇಶ್ವರದ ಕಾಡಿನೊಳಗೆ ಹೋಗಿ ದೇಹವನ್ನು ಕೂಡ ಮರೆತು ದೇವಿಯನ್ನು ಚಿಂತಿಸತೊಡಗಿದರು. ಜಾತಿ ಕುಲ ದೇಹಾಭಿಮಾನವನ್ನೆಲ್ಲ ತ್ಯಜಿಸಿದರು. “ತಾಯಿ ನಿನ್ನ ದರ್ಶನವನ್ನು ನೀಡು, ನನ್ನನ್ನು ಉದ್ಧಾರಮಾಡೆಂದು” ಬೇಡಿದರು. ಸಂಜೆಯ ಹೊತ್ತು ಗಂಗಾನದಿಯ ತೀರದ ಸಮೀಪದಲ್ಲಿ ಸೂರ‍್ಯ ಮುಳುಗುತ್ತಿರುವುದನ್ನು ನೋಡಿ, “ಅಯ್ಯೋ ನನ್ನ ಜೀವನದಲ್ಲಿ ಆಗಲೇ ಒಂದು ದಿನ ಕಳೆದುಹೋಯಿತು, ತಾಯಿ ಇನ್ನೂ ನನಗೆ ದರ್ಶನವನ್ನು ನೀಡಲಿಲ್ಲ” ಎಂದು ನೆಲದ ಮೇಲೆ ಬಿದ್ದು ‘ತಾಯಿ ತಾಯಿ’ ಎಂದು ಕಂಬನಿ ಸುರಿಸಿ ಅಳುತ್ತಿದ್ದರು. ನದಿಗೆ ಬರುವ ಹೋಗುವ ಜನರು ಈ ಹುಡುಗನ ಪ್ರಲಾಪವನ್ನು ನೋಡಿ “ಅಯ್ಯೋ ಪಾಪ, ಹೆತ್ತತಾಯಿ ತೀರಿಕೊಂಡಿರಬಹುದು, ಅದಕ್ಕಾಗಿ ಇಷ್ಟು ಉನ್ಮತ್ತನಾಗಿ ಅಳುತ್ತಿುರುವನು” ಎಂದು ಹೇಳಿ ಹೋಗುತ್ತಿದ್ದರು.

ಅತ್ತರು, ಪ್ರಾರ್ಥಿಸಿದರು, ತಪಸ್ಸು ಮಾಡಿದರು. ಆದರೂ ದೇವಿ ದರ್ಶನ ಕೊಡಲಿಲ್ಲ. ಯಾವ ಜೀವನದಲ್ಲಿ ದೇವರನ್ನು ನೋಡಲು ಸಾಧ್ಯವಿಲ್ಲವೋ ಆ ಬಾಳಿನಿಂದ ಪ್ರಯೋಜನವೇನು?– ಎಂದು ಗರ್ಭಗುಡಿಯಲ್ಲಿ ಪ್ರಾಣಿಗಳನ್ನು ಬಲಿಕೊಡುವುದಕ್ಕೆ ಗೋಡೆಗೆ ತಗಲು ಹಾಕಿದ್ದ ದೊಡ್ಡ ಕತ್ತಿಯನ್ನೇ ತೆಗೆದುಕೊಂಡು ತಮ್ಮ ಶಿರವನ್ನೇ ದೇವಿಯ ಅಡಿಗೆ ಉರುಳಿಸುವುದರಲ್ಲಿದ್ದರು. ಇನ್ನೇನು ಶಿರಕ್ಕೆ ಪೆಟ್ಟು ಬೀಳಬೇಕು ಖಡ್ಗದಂಚಿನಿಂದ, ಅಷ್ಟು ಹೊತ್ತಿಗೆ ಅವರ್ಣನೀಯ ತೇಜೋರಾಶಿ ದಿಕ್ಕು ದಿಕ್ಕುಗಳಿಂದ ಹರಿದುಬಂದು ಇವರ ವ್ಯಕ್ತಿತ್ವವನ್ನೆಲ್ಲ ಮುಳುಗಿಸಿಬಿಟ್ಟಿತು. ಅನಂತ ತೇಜೋರಾಶಿಯಿಂದ ಜಗನ್ಮಾತೆ ಮಂದಹಾಸವನ್ನು ಬೀರುತ್ತ ಇವರ ಮುಂದೆ ನಿಂತಳು. ಹಲವು ದಿನಗಳವರೆಗೆ ‘ತಾಯಿ, ತಾಯಿ’ ಎಂಬ ಮಾತಲ್ಲದೆ ಬೇರೆ ಯಾವ ನುಡಿಯೂ ಇವರ ಬಾಯಿನಿಂದ ಬರುತ್ತಿರಲಿಲ್ಲ. ಪ್ರಪಂಚವನ್ನೆಲ್ಲ ವ್ಯಾಪಿಸಿಕೊಂಡು ಹಲವಾರು ಹೆಸರಿನಿಂದ ಕರೆಸಿಕೊಳ್ಳುತ್ತಿರುವ ಆ ಜಗದಾದಿ ಚೈತನ್ಯವನ್ನು ಪರಮಹಂಸರು ಕಂಡು ಧನ್ಯರಾದರು.

ಶ‍್ರೀರಾಮಕೃಷ್ಣರು ಪ್ರಪಂಚಕ್ಕೆ ಬಂದದ್ದು ಯಾವುದೋ ಒಂದು ದಾರಿಯಲ್ಲಿ ನಡೆದುಕೊಂಡು ಹೋಗಿ ಗುರಿಯನ್ನು ಕಾಣುವುದಕ್ಕಲ್ಲ. ಅವರು ಬಂದದ್ದು ಸತ್ಯಕ್ಕೆ ಅನಂತ ದ್ವಾರಗಳಿವೆ, ಯಾವ ದ್ವಾರದ ಮೂಲಕ ಹೋದರೂ ಒಂದೇ ಗುರಿ ಎಡೆಗೆ ಬರುತ್ತೇವೆ ಎಂಬುದನ್ನು ಸಾರುವುದಕ್ಕೆ. ಹಾಗೆ ಸಾರಿದ ಮಾತಿನಲ್ಲಿ ಒಂದು ಶಕ್ತಿ ಇರಬೇಕಾದರೆ, ಮೊದಲು ಅದನ್ನು ಒಬ್ಬ ಮನಗಂಡಿರಬೇಕು. ಶ‍್ರೀರಾಮಕೃಷ್ಣರು ಬೇರೆ ಬೇರೆ ಮಾರ್ಗದ ಮೂಲಕ ದೇವರನ್ನು ನೋಡಲು ಉಗ್ರಸಾಧನೆಯನ್ನು ಕೈಕೊಂಡರು. ಅವರ ಸಾಧನೆಯೇ ಒಂದು ಅದ್ಭುತ. ಯಾವ ಒಂದು ದಾರಿಯಲ್ಲಿ ನಡೆಯಬೇಕಾದರೆ ಸಾಧಾರಣ ಜೀವಿಗೆ ಹಲವು ವರುಷಗಳು ಬೇಕೊ, ಅದನ್ನು ಶ‍್ರೀರಾಮಕೃಷ್ಣರು ಕೆಲವು ದಿನಗಳಲ್ಲಿ ಪೂರೈಸಿ ಗುರಿಯನ್ನು ಮುಟ್ಟಿಬಿಡುತ್ತಿದ್ದರು. ದೇವರನ್ನು ಬಾಲಗೊಪಾಲನಂತೆ ಕಂಡರು, ತಾಯಿಯಂತೆ ಕಂಡರು, ಮಧುರ ಭಾವದಲ್ಲಿದ್ದರು, ಹನುಮಂತ ಭಾವವನ್ನು ಆರೋಪಮಾಡಿಕೊಂಡು ಮರದ ಮೇಲೆಯೇ ವಾಸಮಾಡುತ್ತ ಗೆಡ್ಡೆಗೆಣಸುಗಳನ್ನು ತಿಂದು ಶ‍್ರೀರಾಮನನ್ನು ಚಿಂತಿಸತೊಡಗಿದರು. ಹಿಂದೂಧರ್ಮದಲ್ಲಿ ಎಷ್ಟು ಸಾಧನೆಗಳಿವೆಯೋ ಅವನ್ನೆಲ್ಲ ಮಾಡಿದರು. ಅನಂತರ ಮಹಮ್ಮದೀಯರೊಬ್ಬರಿಂದ ಉಪದೇಶವನ್ನು ಪಡೆದು ಅವರ ಸಾಧನೆಯನ್ನೂ ಮಾಡಿ ಆ ಧರ್ಮದ ಸತ್ಯವನ್ನು ಕಂಡರು. ಕ್ರೈಸ್ತನೊಬ್ಬನಿಂದ ಉಪದೇಶವನ್ನು ಪಡೆದು ಆ ಪಥದಲ್ಲಿಯೂ ಗುರಿಯನ್ನು ಮುಟ್ಟಿದರು.

ಶ‍್ರೀರಾಮಕೃಷ್ಣರು ಆಧ್ಯಾತ್ಮಿಕ ಜೀವನದಲ್ಲಿ ಪರಿಣತರಾದಮೇಲೆ ಅವರು ಪೂಜೆ ಮಾಡುತ್ತಿದ್ದ ರೀತಿಯೇ ಬೇರೆ ಆಯಿತು. ಅವರ ಪಾಲಿಗೆ ದೇವರು ಸಾಕಾರ ಮತ್ತು ನಿರಾಕಾರ ಎರಡೂ ಆಗಿದ್ದನು. ಯಾವಾಗ ದೇವರನ್ನು ಸಾಕಾರ ಎಂದು ಭಾವಿಸುತ್ತಿದ್ದರೋ ಅವನನ್ನು ಯಾವ ರೂಪಿನಲ್ಲಿ ಬೇಕಾದರೂ ನೋಡಬಲ್ಲವರಾಗಿದ್ದರು. ಪೂಜಾರಿಯ ಪಾತ್ರದಲ್ಲಿ ಗರ್ಭಗುಡಿಯ ಒಳಗೆ ಇರುವಾಗ ನಿರಾಕಾರ ಪರಬ್ರಹ್ಮನ ಸಾಗರದಿಂದ ಎದ್ದ ಒಂದು ಭೀಮ ಅಲೆಯಂತೆ ಇದ್ದಳು ಕಾಳಿಕಾ ಮೂರ್ತಿ. ಅವಳೆ ಸಗುಣ ಬ್ರಹ್ಮ, ಸೃಷ್ಟಿ ಸ್ಥಿತಿ ಪ್ರಳಯಗಳನ್ನು ಮಾಡುವವಳು. ಅವಳು ಹೇಗೆ ಚಿನ್ಮಯ ಮೂರ್ತಿಯೋ ಹಾಗೆಯೇ ಭಕ್ತರ ಸಹಾಯಕ್ಕಾಗಿ ವಿಗ್ರಹದ ಮೂಲಕವೂ ಮೈದೋರುವಳು. ಶ‍್ರೀರಾಮಕೃಷ್ಣರು ಕಾಳಿಕಾ ಮೂರ್ತಿಯನ್ನು ಪೂಜಿಸುತ್ತಿದ್ದಾಗ ಅವರ ಪಾಲಿಗೆ ಅವಳೊಂದು ವಿಗ್ರಹವಾಗಿರಲಿಲ್ಲ. ಸಚೇತನ ಮೂರ್ತಿಯಾಗಿ ಕಂಡುಬಂದಳು. ಶ‍್ರೀರಾಮಕೃಷ್ಣರು ತಾವು ಪಂಚೆಯಂಚನ್ನು ಮಡಿಸಿ ಗರ್ಭಗುಡಿ ಒಳಗೆ ಇರುವ ದೇವರ ಮೂಗಿನ ಹೊಳ್ಳೆಗಳ ಹತ್ತಿರ ಇಟ್ಟಾಗ ಅದು ದೇವಿ ಸೆಳೆದುಕೊಳ್ಳುವ ಮತ್ತು ಬಿಡುವ ಉಸಿರಿನಿಂದ ಹಿಂದಕ್ಕೂ ಮುಂದಕ್ಕೂ ಚಲಿಸಿತು ಎಂದು ಶ‍್ರೀರಾಮಕೃಷ್ಣರು ಹೇಳುತ್ತಿದ್ದರು. ಬೆಳಿಗ್ಗೆ ಹೊತ್ತು ಪೂಜೆಗೆ ಹೂವನ್ನು ಕೊಯ್ಯಲು ಹೋದರೆ ದೇವಿ ಒಂದು ಚಿಕ್ಕ ಮಗುವಿನಂತೆ ಶ‍್ರೀರಾಮಕೃಷ್ಣರ ಜೊತೆಯಲ್ಲಿ ಬರುತ್ತಿದ್ದಳು. ಅವಳು ತನ್ನ ಪುಟ್ಟ ಕೈಗಳಿಂದ ಹೂವಿನ ಕೊಂಬೆಗಳನ್ನು ಕೆಳಗೆ ಬಗ್ಗಿಸುತ್ತಿದ್ದಳು, ಶ‍್ರೀರಾಮಕೃಷ್ಣರು ಹೂವನ್ನು ಬಿಡಿಸುತ್ತಿದ್ದರು. ನೈವೇದ್ಯ ಮಾಡುವ ಸಮಯ ಪಕ್ಕದಲ್ಲಿ ದೇವಿಗೆ ಆಹಾರವನ್ನು ಕೊಡುವಾಗ ಕೆಲವು ವೇಳೆ ತಾಯಿಗೆ ತಿನ್ನಿಸಿ ತಾವೂ ತಿನ್ನುತ್ತಿದ್ದರು. ದೇವಿಯನ್ನು ಮಲಗಿಸುವಾಗ ಅವರೂ ದೇವಿಯ ಪಕ್ಕದಲ್ಲಿ ಮಲಗುತ್ತಿದ್ದರು. ಮಂಗಳಾರತಿ ಮಾಡುವಾಗ ಕೆಲವು ವೇಳೆ ಬೇಗ ಪೂರೈಸುತ್ತಿದ್ದರು, ಮತ್ತೆ ಕೆಲವು ವೇಳೆ ನಿಧಾನವಾಗಿ ಮಾಡುತ್ತಿದ್ದರು. ಒಂದು ದಿನ ಆ ದೇವಸ್ಥಾನವನ್ನು ಕಟ್ಟಿಸಿದ ರಾಣಿ ರಾಸ್ಮಣಿ ಪೂಜೆ ಸಮಯಕ್ಕೆ ಬಂದು ಕುಳಿತುಕೊಂಡು ದೇವಿಯ ಮೇಲೆ ಒಂದೆರಡು ಹಾಡನ್ನು ಹೇಳಿ ಎಂದು ಶ‍್ರೀರಾಮಕೃಷ್ಣರನ್ನು ಕೇಳಿಕೊಂಡಳು. ಶ‍್ರೀರಾಮಕೃಷ್ಣರು ಭಕ್ತಿಯಿಂದ ಹಾಡುತ್ತಿದ್ದಾಗ ಆಕೆಯ ಮನಸ್ಸು ಕೋರ್ಟಿನಲ್ಲಿದ್ದ ಒಂದು ಕೇಸಿನ ಮೇಲೆ ಹೋಗಿತ್ತು. ಶ‍್ರೀರಾಮಕೃಷ್ಣರು ಆಕೆಯ ಸಮೀಪಕ್ಕೆ ಬಂದು ‘ಇಲ್ಲಿಯೂ ಅದರ ಯೋಚನೆಯೇ?’ ಎಂದು ಅವಳ ಕಪಾಳಕ್ಕೆ ಹೊಡೆದರು. ದೇವಸ್ಥಾನವನ್ನು ಕಟ್ಟಿಸಿದ ರಾಣಿಯನ್ನೇ ಹೊಡೆದ ಆ ಪೂಜಾರಿ ಹುಚ್ಚನಾಗಿರಬೇಕೆಂದು ಸುತ್ತಮುತ್ತಲಿರುವವರೆಲ್ಲ ತಿರ್ಮಾನಕ್ಕೆ ಬಂದರು.

ಕಾಮಾರಪುಕುರದಲ್ಲಿದ್ದ ಶ‍್ರೀರಾಮಕೃಷ್ಣರ ತಾಯಿ ಚಂದ್ರಮಣೀದೇವಿಗೆ ಇದು ಗೊತ್ತಾಗಿ ಮಗನನ್ನು ಹಳ್ಳಿಗೆ ಕರೆಸಿಕೊಂಡರು. ಔಷಧೋಪಚಾರ ಮಾಡಿದರು. ಮಂತ್ರವಾದಿಯನ್ನು ಕರೆಸಿ ಮಂತ್ರಗಳನ್ನು ಹಾಕಿಸಿದರು. ಆಗ ಹಾಸ್ಯವಾಗಿ ಶ‍್ರೀರಾಮಕೃಷ್ಣರು ವೈದ್ಯನಿಗೆ, “ನೀನು ಬೇಕಾದರೆ ನನ್ನ ದೇಹದಲ್ಲಿ ಖಾಯಿಲೆ ಇದ್ದರೆ ಗುಣಮಾಡು, ನನ್ನ ಭಕ್ತಿಯನ್ನು ಗುಣಮಾಡಬೇಡ!” ಎಂದರು. ಎಲ್ಲರೂ ಆಲೋಚಿಸಿದರು, ಶ‍್ರೀರಾಮಕೃಷ್ಣರಿಗೆ ಮದುವೆ ಮಾಡಿಬಿಟ್ಟರೆ ಅವರ ದೇವರ ಹುಚ್ಚೆಲ್ಲ ಬಿಟ್ಟು ಹೋಗಿಬಿಡುವುದು ಎಂದು. ಶ‍್ರೀರಾಮಕೃಷ್ಣರಿಗೆ ಹತ್ತಿರದ ಒಂದು ಹಳ್ಳಿಯಲ್ಲಿದ್ದ ಐದು ವರ್ಷದ ಶಾರದಾದೇವಿ ಎಂಬ ಹುಡುಗಿಯೊಂದಿಗೆ ಮದುವೆ ಮಾಡಿದರು. ಆಗ ರಾಮಕೃಷ್ಣರಿಗೆ ೨೩ ವರುಷಗಳು. ಮದುವೆಯಾದ ಕೆಲವು ದಿನಗಳ ಮೇಲೆ ಶ‍್ರೀರಾಮಕೃಷ್ಣರು ದಕ್ಷಿಣೇಶ್ವರಕ್ಕೆ ಮರಳಿದರು. ಪುನಃ ಆಧ್ಯಾತ್ಮಿಕ ಜೀವನದ ಮಹಾಸಾಹಸದಲ್ಲಿ ತನ್ಮಯರಾಗಿಹೋದರು. ಅವರ ದೇಹದ ಪರಿವೆಯೇ ಇಲ್ಲದಿರುವಾಗ ಕೈಹಿಡಿದ ಹೆಂಡತಿಯ ನೆನಪಾದರೂ ಬರುವುದು ಹೇಗೆ?

ಇದೇ ಸಮಯದಲ್ಲಿ ತೋತಾಪುರಿ ಎಂಬ ಅದ್ವೈತ ಸಂನ್ಯಾಸಿ ದಕ್ಷಿಣೇಶ್ವರಕ್ಕೆ ಬಂದನು. ಆತ ಶ‍್ರೀರಾಮಕೃಷ್ಣರನ್ನು ನೋಡಿದನು. ಅದ್ವೈತ ಸಾಧನೆಗೆ ಇವರು ತಕ್ಕ ಶಿಷ್ಯರಂತೆ ಕಂಡರು. “ನಾನು ಅದ್ವೈತ ಸಾಧನೆಯನ್ನು ಬೋಧಿಸುತ್ತೇನೆ, ನೀನು ಸ್ವೀಕರಿಸುವೆಯೊ” ಎಂದು ತೋತಾಪುರಿ ಕೇಳಿದನು. ಶ‍್ರೀರಾಮಕೃಷ್ಣರು ತಮಗೆ ಏನೂ ಗೊತ್ತಿಲ್ಲವೆಂದೂ ತಾವು ತಮ್ಮ ಕಾಳಿಯನ್ನು ಕೇಳಿ ಬರುತ್ತೇನೆಂದೂ ಗರ್ಭಗುಡಿಗೆ ಹೋಗಿ ಜಗನ್ಮಾತೆಯನ್ನು ಕೇಳಿದರು. ದೇವಿ, “ನೀನು ಅದ್ವೈತ ಸಾಧನೆಯನ್ನು ಕೈಗೊಳ್ಳುವುದಕ್ಕಾಗಿಯೇ ನಾನು ಅವನನ್ನು ಇಲ್ಲಿಗೆ ಕರೆತಂದಿರುವುದು” ಎಂದು ಹೇಳಿದಂತೆ ಆಯಿತು.

ಅದ್ವೈತ ಸಾಧನೆಗೆ ಶ‍್ರೀರಾಮಕೃಷ್ಣರು ಒಪ್ಪಿಕೊಂಡರು. ವೇದಾಂತ ಸಂನ್ಯಾಸಿಯಾದ ತೋತಾಪುರಿ ಪ್ರಪಂಚವನ್ನು ಮಿಥ್ಯೆ ಎಂದು ನೋಡುತ್ತಿದ್ದ. ನಲವತ್ತು ವರ್ಷಗಳ ಕಾಲ ಅದ್ವೈತ ಸಾಧನೆಯಲ್ಲಿ ತಲ್ಲೀನನಾಗಿ ಸಿದ್ಧಿಯನ್ನು ಪಡೆದಿದ್ದ. ಅವನು ಶ‍್ರೀರಾಮಕೃಷ್ಣರಿಗೆ ಅದ್ವೈತ ಸಾಧನೆಯನ್ನು ಬೋಧಿಸಿದನು. ಶ‍್ರೀರಾಮಕೃಷ್ಣರು ಬಹಳ ಅಲ್ಪ ಕಾಲದಲ್ಲೆ ನಿರ್ವಿಕಲ್ಪ ಸಮಾಧಿಯಲ್ಲಿ ಸ್ಥಿತರಾದರು. ದೇಹ ನಿಶ್ಚಲ ಶಿಲಾವಿಗ್ರಹದಂತೆ ಆಯಿತು. ತೋತಾಪುರಿಗೆ ಆಶ್ಚರ್ಯ – ಯಾವುದಕ್ಕೆ ತನಗೆ ಹಲವಾರು ವರ್ಷಗಳು ಹಿಡಿಯಿತೋ ಅದನ್ನು ಶಿಷ್ಯ ಕೆಲವೇ ಗಂಟೆಗಳಲ್ಲಿ ಸಾಧಿಸಿದನು. ಕೆಲವು ದಿನಗಳು ಶ‍್ರೀರಾಮಕೃಷ್ಣರು ಆ ಸ್ಥಿತಿಯಲ್ಲಿದ್ದಾದ ಮೇಲೆ ತೋತಾಪುರಿ ಶ‍್ರೀರಾಮಕೃಷ್ಣರ ಮನಸ್ಸನ್ನು ಪ್ರಪಂಚದ ಕಡೆ ತಂದನು. ಆತ ಎಲ್ಲಿಯೂ ಮೂರು ದಿನಗಳಿಗಿಂತ ಹೆಚ್ಚು ಇದ್ದವನಲ್ಲ. ದಕ್ಷಿಣೇಶ್ವರದಲ್ಲಿ ತನ್ನ ಶಿಷ್ಯನ ಬಳಿ ಹನ್ನೊಂದು ತಿಂಗಳುಗಳು ಕಾಲ ಕಳೆದು ವೇದಾಂತ ಅದ್ವೈತ ತತ್ತ್ವಗಳನ್ನು ಶಿಷ್ಯನಿಗೆ ಬೋಧಿಸಿದನು.

ಕೆಲವು ಕಾಲವಾದ ಮೇಲೆ ಅವರು ಕೈಹಿಡಿದ ಧರ್ಮಪತ್ನಿ ಶ‍್ರೀಶಾರದಾದೇವಿ\-ಯವರು ದಕ್ಷಿಣೇಶ್ವರಕ್ಕೆ ಪತಿಯನ್ನು ನೋಡಲು ಬಂದರು. ಆಗ ಅವರಿಗೆ ಹದಿನಾರು ವರ್ಷಗಳಾಗಿತ್ತು. ಶ‍್ರೀರಾಮಕೃಷ್ಣರು ಅವರನ್ನು ಆದರದಿಂದ ಬರಮಾಡಿಕೊಂಡು, ತಾವು ಎಲ್ಲರಲ್ಲಿಯೂ ಜಗನ್ಮಾತೆಯನ್ನು ನೋಡುತ್ತಿರುವುದರಿಂದ ತಮ್ಮ ಸತಿಯೊಡನೆ ದೈಹಿಕ ಸಂಬಂಧವನ್ನು ಇಟ್ಟುಕೊಳ್ಳಲು ಸಾಧ್ಯವಿಲ್ಲವೆಂದರು. ಆದರೆ ಕೈಹಿಡಿದ ಸತಿಗೆ ಗಂಡನ ಮೇಲೆ ಒಂದು ಅಧಿಕಾರವಿದೆ. “ಆಧ್ಯಾತ್ಮಿಕ ಜೀವನವನ್ನು ನೀನು ಇಚ್ಛಿಸಿದರೆ ಅದಕ್ಕೆ ಸಹಾಯಕನಾಗುವೆ, ನಿನಗೆ ಲೌಕಿಕದ ಕಡೆ ಮನಸ್ಸಿದ್ದರೆ ಅದರ ಕಡೆಗೂ ವಾಲುವುದಕ್ಕೆ ಸಿದ್ಧನಿರುವೆ. ನಿನಗೆ ಯಾವುದು ಬೇಕು ಕೇಳು” ಎಂದರು. ಶ‍್ರೀರಾಮಕೃಷ್ಣರನ್ನು ಅವತಾರ ವ್ಯಕ್ತಿಗಳ ಗುಂಪಿಗೆ ಸೇರಿಸುವುದು ಬಿಡುವುದು ಶಾರದಾದೇವಿಯವರ ಕೈಯಲ್ಲಿತ್ತು. ಅವರು ಮರುಮಾತಿಲ್ಲದೆ “ನಾನೇಕೆ ನಿಮ್ಮ ಧರ್ಮ ಜೀವನಕ್ಕೆ ಅಡ್ಡಬರಲಿ? ನಿಮ್ಮಂತೆ ನನ್ನನ್ನೂ ದೇವಿಯ ದಾಸಿಯನ್ನಾಗಿ ಮಾಡಿ” ಎಂದು ಕೇಳಿಕೊಂಡರು. ಶ‍್ರೀಶಾರದಾದೇವಿ ಕೈಹಿಡಿದ ಹೆಂಡತಿಯಾಗಿ ಬಂದರೂ ಶಿಷ್ಯರಲ್ಲಿ ಪ್ರಥಮರಾದರು. ಶ‍್ರೀರಾಮಕೃಷ್ಣರು ತಾವೂ ಕೈಹಿಡಿದ ಸತಿಯನ್ನು ಮೆಟ್ಟಲು ಮೆಟ್ಟಲಾಗಿ ಚರಮ ಗುರಿಯವರೆಗೆ ಒಯ್ದರು. ಒಂದು ದಿನ ದೇವಿಯ ಪೀಠದ ಮೇಲೆ ಶ‍್ರೀಶಾರದಾದೇವಿಯನ್ನೇ ಕೂಡಿಸಿ ಪೂಜೆ ಮಾಡಿ ಪುಷ್ಪಾಂಜಲಿಯನ್ನು ಅರ್ಪಿಸಿ ತಮ್ಮ ತಪಃಫಲವನ್ನೆಲ್ಲ ಅರ್ಪಿಸಿದರು. ಇದಾದ ಮೇಲೆ ಅವರು ಇನ್ನು ಯಾವ ಸಾಧನೆಯನ್ನೂ ಮಾಡಲಿಲ್ಲ.

ಶ‍್ರೀರಾಮಕೃಷ್ಣರ ಜೀವನ ಆಧ್ಯಾತ್ಮಿಕ ಘಟನಾವಳಿಗಳ ಸುಂಟರಗಾಳಿ. ಒಂದು ಸಾಧನೆಯಾದ ಮೇಲೆ ಮತ್ತೊಂದು ಸಾಧನೆ, ಒಂದು ಭಾವದ ಮೂಲಕ ದೇವರನ್ನು ಕಂಡೊಡನೆಯೇ ಬೇರೊಂದು ಭಾವದ ಮೂಲಕ ನೋಡುವುದು, ಒಂದು ತತ್ತ್ವದ ದೃಷ್ಟಿಕೋಣದಿಂದ ನೋಡಿದ ಮೇಲೆ ಮತ್ತೊಂದು ತತ್ತ್ವದ ದೃಷ್ಟಿಯಿಂದ ನೋಡುವುದು, ಹಿಂದೂ ಧರ್ಮದ ಸಾಧನೆಗಳಾದ ಮೇಲೆ ಇಸ್ಲಾಂ ಮತ್ತು ಕ್ರೈಸ್ತ ಸಾಧನೆಗಳನ್ನು ಕೈಗೊಳ್ಳುವುದು – ಹೀಗೆ ಅವರು ಮುಂದುವರಿದರು. ಅವರಿಗೆ ಸವಿಕಲ್ಪ ನಿರ್ವಿಕಲ್ಪ ಸಮಾಧಿಗಳು ಮೇಲಿನ ಮೆಟ್ಟಲಿಂದ ಕೆಳಗಿನ ಮೆಟ್ಟಲಿಗೆ ಇಳಿದು ಬಂದು ಹೋಗುವಷ್ಟು ಸರಾಗವಾಗಿತ್ತು. ದೇವರನ್ನು ಸಾಕಾರವಾಗಿ ಆಗಲಿ ನಿರಾಕಾರವಾಗಿ ಆಗಲಿ; ನಿರ್ಗುಣವಾಗಿ ಆಗಲಿ, ಸಗುಣವಾಗಿ ಆಗಲಿ, ಯಾವ ದೃಷ್ಟಿಯಿಂದ ಬೇಕಾದರೆ ಅನುಭವಿಸಬಲ್ಲವರಾಗಿದ್ದರು. ದೇಶಕಾಲನಿಮಿತ್ತಾತೀತವಾದ ಸ್ಥಿತಿಗೂ ಕಣ್ಣು ಮಿಟುಕಿಸುವುದರಲ್ಲಿ ಹಾರಿ ಹೊಗಬಲ್ಲವರಾಗಿದ್ದರು.

ಶ‍್ರೀರಾಮಕೃಷ್ಣರ ಜೀವನ ಹಲವು ಸಾಧನೆಗಳೆಂಬ ಮಹಾನದಿಗಳಿಂದ ಪೋಷಿತವಾದ ಅಸೀಮ ಅನಂತ ಸಾಗರದಂತೆ ಇತ್ತು. ಸಾಕ್ಷಾತ್ಕಾರದ ಅಮೃತವನ್ನು ತಮ್ಮ ಹೃದಯದಲ್ಲಿ ಶೇಖರಿಸಿಟ್ಟುಕೊಂಡಿದ್ದರು. ಆದರೂ ತಾವೊಬ್ಬರೇ ಅದನ್ನು ಆಸ್ವಾದಮಾಡಿ ಸಂತೋಷ ಪಡುವುದಕ್ಕೆ ಬಂದವರಲ್ಲ. ಯೋಗ್ಯರಾದ ಭವಜೀವಿಗಳಿಗೆಲ್ಲ ಇದನ್ನು ಧಾರೆಯೆರೆದು ಅವರನ್ನು ಅಮೃತತ್ವಕ್ಕೆ ಹಕ್ಕುದಾರರನ್ನಾಗಿ ಮಾಡಬೇಕೆಂದು ಬಯಸಿದರು. ಯೋಗ್ಯರಾದ ಶಿಷ್ಯರನ್ನು ಕಳಿಸೆಂದು ದೇವಿಗೆ ಪ್ರಾರ್ಥನೆ ಮಾಡಿದರು. ದೇವಿಯಾದರೊ ಕ್ರಮೇಣ ನಿನ್ನ ಶಿಷ್ಯರು ಬರುತ್ತಾರೆ ಎಂದು ಭರವಸೆ ಕೊಟ್ಟಿದ್ದಳು. ಶ‍್ರೀರಾಮಕೃಷ್ಣರು ಕಾದರು. ಇನ್ನೂ ಬರಲಿಲ್ಲ. ಎದೆಯಲ್ಲಿ ಹಾಲು ಕಟ್ಟಿಕೊಂಡ ತಾಯಿ ತನ್ನ ಎಳೆಯ ಮಗುವಿಗೆ ಕಾತರಳಾಗುವಂತೆ ಇದ್ದರು ಶ‍್ರೀರಾಮಕೃಷ್ಣರು. ಅವರ ಸಾಧನೆಯ ಪ್ರಾರಂಭದಲ್ಲಿ ಮುಳುಗುವ ಸೂರ್ಯನನ್ನು ನೋಡಿ ‘ನನ್ನ ಜೀವನದಲ್ಲಿ ಆಗಲೇ ಒಂದು ದಿನ ವ್ಯರ್ಥವಾಗಿ ಹೋಯಿತು; ತಾಯಿ ಎಂದು ಬರುವಳೊ?’ ಎಂದು ಅತ್ತುದನ್ನು ನೋಡಿರುವೆವು. ಈಗ ಎಲ್ಲಾ ಸಾಧನೆಯ ಚರಮ ಸೀಮೆಯನ್ನು ಮುಟ್ಟಿದ ಮೇಲೆ, ಆಧ್ಯಾತ್ಮಿಕ ಜೀವನದಲ್ಲಿ ಗಳಿಸಬೇಕಾಗಿರುವುದನ್ನೆಲ್ಲ ಗಳಿಸಿ ಆದಮೇಲೆ, ಮತ್ತೊಂದು ದಾರುಣಯಾತನೆ. ಅದು ಮೊದಲಿನ ಯಾತನೆಗಿಂತ ಕಡಿಮೆಯೇನೂ ಆಗಿರಲಿಲ್ಲ. ಪ್ರತಿದಿನ ಸಂಜೆ ಮಂಗಳಾರತಿ ಆದಮೇಲೆ ಕತ್ತಲಲ್ಲಿ ಮುಳುಗಿಹೋಗಿದ್ದ ಕಲ್ಕತ್ತೆಯ ಕಡೆ ತಿರುಗಿ ನೋಡಿ ‘ಎಲ್ಲಿ ನನ್ನ ಮಕ್ಕಳು ಇನ್ನೂ ಬರಲಿಲ್ಲವಲ್ಲ. ಅವರು ಯಾವಾಗ ಬರುವರು?’ ಎಂದು ಆಧ್ಯಾತ್ಮಿಕ ಪಿಪಾಸುಗಳಿಗಾಗಿ ಅತ್ತರು. ಯಾವ ತಾಯಿಯೂ ತನ್ನ ಮಗುವಿಗೆ ಇಷ್ಟು ಕಾತರವನ್ನು ವ್ಯಕ್ತಪಡಿಸಿರಲಾರಳು.

ಶ‍್ರೀರಾಮಕೃಷ್ಣರ ಜೀವನ ಸಹಸ್ರದಳದ ಪದ್ಮದಂತೆ ದಕ್ಷಿಣೇಶ್ವರದಲ್ಲಿ ವಿಕಾಸವಾಯಿತು. ಅದರ ಪರಾಗ ದಿಕ್ಕುದಿಕ್ಕಿಗೂ ಹರಡಿತು. ಆಧ್ಯಾತ್ಮಿಕ ಪಿಪಾಸುಗಳು ಹಲವು ಕಡೆಗಳಿಂದ ಬಂದರು. ದುಂಬಿಗಳು ಮಕರಂದವನ್ನು ಹೀರುವುದಕ್ಕೆ ಹೇಗೆ ಬರುವುವೋ ಹಾಗೆ ಹಲವು ಕಡೆಗಳಿಂದ ಪಿಪಾಸುಗಳು ಬಂದರು. ಬಂದವರಲ್ಲಿ ಹಲವು ಬಗೆಯ ಜನರಿದ್ದರು. ಕಲ್ಕತ್ತೆಯ ಮಹಾ ವಿದ್ವಾಂಸ ವರ್ಗಕ್ಕೆ ಸೇರಿದವರಿದ್ದರು, ಏನೂ ಅರಿಯದ ಪಾಮರರೂ ಇದ್ದರು. ಸಹಸ್ರಾರು ಸಭಿಕರನ್ನು ಮುಗ್ಧರನ್ನಾಗಿ ಮಾಡಬಲ್ಲಂತಹ ಮಹಾ ವಾಗ್ಮಿಗಳಿದ್ದರು. ಕೇಳಿದವರೆದೆಯನ್ನು ಹಿಂಡಿ ಕಣ್ಣಿನಲ್ಲಿ ನೀರನ್ನು ತರುವಂತೆ ಹಾಡಬಲ್ಲ ಗಾಯಕರಿದ್ದರು. ವಂಗ ಸಾಹಿತ್ಯ ಪ್ರಪಂಚದಲ್ಲಿ ಅಮರ ಕೀರ್ತಿ ಪಡೆದವರಿದ್ದರು. ಪುಣ್ಯಾತ್ಮರಿದ್ದರು. ನಾನು ಮಾಡದ ಪಾಪವೇ ಇಲ್ಲ, ಕುಡಿದ ಸಾರಾಯಿ ಬುಡ್ಡಿಯನ್ನು ಒಂದರಮೇಲೆ ಒಂದು ಇಟ್ಟರೆ ಅದು ಗೌರೀಶಂಕರ ಶಿಖರವನ್ನು ಮೀರುವುದು ಎಂಬುವಂತಹವರಿದ್ದರು. ದೇವರ ಸಾಕಾರದಲ್ಲಿ ನಂಬುವವರಿದ್ದರು. ನಿರಾಕಾರದಲ್ಲಿ ನಂಬುವವರಿದ್ದರು. ದೇವರನ್ನೇ ನಂಬದವರೂ ಇದ್ದರು. ಹಿಂದುಗಳಿದ್ದರು, ಕ್ರೈಸ್ತರಿದ್ದರು, ಮಹಮ್ಮದೀಯರಿದ್ದರು, ಪುರುಷರಿದ್ದರು ಸ್ತ್ರೀಯರಿದ್ದರು, ಬಾಲರಿದ್ದರು.

ಶ‍್ರೀರಾಮಕೃಷ್ಣರೆಡೆಗೆ ಬಂದ ಮಹಾ ಯಾತ್ರಿಕರ ತಂಡದಲ್ಲಿ ನರೇಂದ್ರನೊಬ್ಬ. ಪಾಶ್ಚಾತ್ಯ ವಿಜ್ಞಾನ, ತತ್ತ್ವ, ಸಾಹಿತ್ಯ ಮುಂತಾದುವುಗಳನ್ನು ಹೀರಿ ತನ್ನ ವ್ಯಕ್ತಿತ್ವವನ್ನು ಬೆಳಸಿಕೊಂಡ ವ್ಯಕ್ತಿ ನರೇಂದ್ರ. ಯಾವುದನ್ನೂ ಬಡಪೆಟ್ಟಿಗೆ ನಂಬದವನು, ದೇವರ ಅನುಭವವನ್ನು ಕೊಟ್ಟರೂ, ಇದೇನು ಮನೋವಿಕಾರವೋ ಎಂದು ಅನುಮಾನಿಸುವವನು. ಪ್ರಚಂಡ ವಿದ್ಯಾಬುದ್ಧಿ ಯೌವನಗಳಿಂದ ಕೂಡಿದ ನರೇಂದ್ರ ಬಾಯಾರಿ ಶ‍್ರೀರಾಮಕೃಷ್ಣರೆಂಬ ಅಮೃತ ಸರೋವರಕ್ಕೆ ನೀರನ್ನು ಕುಡಿಯಲು ಬರುವುದನ್ನು ನೋಡುತ್ತೇವೆ.

