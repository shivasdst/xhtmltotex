
\chapter{ಕರ್ಣಾಟಕದಲ್ಲಿ}

 ಸ್ವಾಮೀಜಿ ಕೊಲ್ಲಾಪುರದಿಂದ ಬೆಳಗಾವಿಗೆ ಬಂದರು. ಅಲ್ಲಿ ಭಾಟಿ ಎಂಬುವರ ಮನೆಯಲ್ಲಿ ಇಳಿದುಕೊಂಡರು.ಅವರನ್ನು ನೋಡಿದೊಡನೆಯೆ ಅವರು ಸಾಧಾರಣ ವರ್ಗಕ್ಕೆ ಸೇರಿದ ಮನುಷ್ಯರಲ್ಲ ಎಂಬುದು ಮನೆಯವರಿಗೆ ಗೊತ್ತಾಯಿತು. ಮನೆಯವರು ಪೂರ್ವಾಚಾರ ಸಂಪ್ರದಾಯದವರು. ಸಂಪ್ರದಾಯಸ್ಥ ಮಠಾಧಿಪತಿಗಳನ್ನು ನೋಡಿದ್ದ ಅವರು ಸಂನ್ಯಾಸಿಗಳ ಬಗ್ಗೆ ತಮ್ಮದೇ ಆದ ಕಲ್ಪನೆಗಳನ್ನಿಟ್ಟುಕೊಂಡಿದ್ದರು. ಆದರೆ ಸ್ವಾಮೀಜಿ ಈ ಸಾಂಪ್ರದಾಯಿಕ ಸಂನ್ಯಾಸಿಗಳಂತೆ ಇರಲಿಲ್ಲ. ಹೆಜ್ಜೆ ಹೆಜ್ಜೆಗೆ ಸಂನ್ಯಾಸಿ ಎಂದರೆ ಯಾವ ಭಾವನೆಯನ್ನು ತಮ್ಮ ಮನಸ್ಸಿನಲ್ಲಿ ಮನೆಯವರು ಮುಂಚೆ ಕಲ್ಪಿಸಿಕೊಂಡಿದ್ದರೋ ಅವನ್ನೆಲ್ಲಾ ವ್ಯತ್ಯಾಸಮಾಡುತ್ತ ಬರಬೇಕಾಯಿತು. 

\vskip 2pt

 ಮೊದಲನೆಯದೇ ಸ್ವಾಮೀಜಿ ಅವರ ಉಡಿಗೆ ತೊಡಿಗೆ. ಸಾಧಾರಣ ಸಂನ್ಯಾಸಿಗಳ ಬಣ್ಣದ ಬಟ್ಟೆಯನ್ನೇ ಸ್ವಾಮೀಜಿ ಹಾಕಿದ್ದರೂ ಅವರಂತೆ ಇರಲಿಲ್ಲ ಸ್ವಾಮೀಜಿ ಪೋಷಾಕು. ಇವರು ಒಂದು ಬನಿಯನ್ ಹಾಕಿಕೊಂಡಿದ್ದರು. ಇತರ ಸಂನ್ಯಾಸಿಗಳು ಒಂದು ಉತ್ತರೀಯವನ್ನು ಸಾಧಾರಣವಾಗಿ ಹೊದ್ದುಕೊಂಡಿರುತ್ತಾರೆ. ಅವರು ಹೊಲಿದ ಬಟ್ಟೆಯನ್ನು ಹಾಕಿಕೊಳ್ಳುವುದಿಲ್ಲ. ಸಾಧಾರಣ ಮಠಾಧಿಪತಿಗಳ ದಂಡದಂತಿರಲಿಲ್ಲ, ಸ್ವಾಮೀಜಿ ಕೈಯಲ್ಲಿದ್ದ ಕೋಲು. ಅದೊಂದು ನಡೆಯುವಾಗ ಪ್ರಯೋಜನಕ್ಕೆ ಬರುವ ಒರಟು ಬೆತ್ತವಾಗಿತ್ತು. ಅಲಂಕಾರ ಅಥವಾ ಜಾಹಿರಾತಿನ ಪಟ್ಟದ ದಂಡವಾಗಿರಲಿಲ್ಲ. ಈ ಸ್ವಾಮಿಗಳಾದರೋ ಲೀಲಾಜಾಲವಾಗಿ ಇಂಗ್ಲೀಷನ್ನು ತಮ್ಮ ವ್ಯವಹಾರದಲ್ಲೆಲ್ಲ ಉಪಯೋಗಿಸುತ್ತಿದ್ದರು. ಮಾತನಾಡುವಾಗ ಬರಿ ಶಾಸ್ತ್ರಪುರಾಣಗಳೆ ಅಲ್ಲ; ಪೌರ್ವಾತ್ಯ ಮತ್ತು ಪಾಶ್ಚಾತ್ಯ ವಿಜ್ಞಾನ, ಚರಿತ್ರೆ, ತತ್ತ್ವ, ಸಾಹಿತ್ಯ ಮುಂತಾದವುಗಳಿಂದ ಉದಹರಿಸುತ್ತಿದ್ದರು. ಇವೆಲ್ಲ ಪೂರ್ವಾಚಾರ ಸಂಪ್ರದಾಯಸ್ಥರ ಬಾಯಿಯಿಂದ ಬರುವ ಮಾತುಗಳಾಗಿರಲಿಲ್ಲ. 

\vskip 2pt

 ಮೊದಲನೆ ದಿನವೆ ಊಟವಾದ ಮೇಲೆ ಹಾಕಿಕೊಳ್ಳುವುದಕ್ಕೆ ಸ್ವಾಮೀಜಿ ಎಲೆ ಅಡಿಕೆಯನ್ನು ಕೇಳಿದರು. ಇದನ್ನು ಕೇಳಿ ಮನೆಯವರಿಗೆ ಆಕಾಶವೆ ತಲೆಕೆಳಗಾಗಿ ಬಿದ್ದಂತೆ ಆಯಿತು! ಸಂನ್ಯಾಸಿ ಎಲೆ ಅಡಿಕೆ ಹಾಕಿಕೊಳ್ಳಕೂಡದು ಎಂಬ ವಾಡಿಕೆ ಬಂದುಹೋಗಿದೆ. ಸಂನ್ಯಾಸದ ಸಾರವೆಲ್ಲ ಅದರಲ್ಲೆ ಇದೆ ಎಂದು ಅನೇಕರು ಭಾವಿಸುವರು. ಇನ್ನೊಂದು ದಿನ ಸ್ವಲ್ಪ ಹೊಗೆಸೊಪ್ಪನ್ನು ಕೇಳಿದರು. ಇದನ್ನು ಕೇಳಿದಾಗಲಂತೂ ಅವರ ಪೂರ್ವ ಅಭಿಪ್ರಾಯಕ್ಕೆ ಒಂದು ಕೊಡಲಿ ಪೆಟ್ಟು ಬಿದ್ದಂತೆ ಆಯಿತು. ಮನಸ್ಸಿನಲ್ಲಿ ಅವರಿಗಾಗುತ್ತಿರುವ ಆಂದೋಳನವನ್ನು ಗ್ರಹಿಸಿ ಸ್ವಾಮೀಜಿ ತಮ್ಮ ನಡತೆಗೆ ತಾವೇ ಒಂದು ವಿವರಣೆಯನ್ನು ಕೊಟ್ಟರು. ತಾವು ಹಿಂದೆ ಹುಡುಗಾಟದಲ್ಲಿ ಮಗ್ನರಾದ ಹುಡುಗರಾಗಿದ್ದರೆಂದೂ, ತಾವು ಬ್ರಾಹ್ಮಣರಲ್ಲವೆಂದೂ, ಕಾಯಸ್ಥ ಕುಲದಿಂದ ಬಂದವರೆಂದೂ, ತಮ್ಮ ಗುರುಗಳಾದ ಶ‍್ರೀರಾಮಕೃಷ್ಣರ ಬಳಿಗೆ ಹೋದಾಗ ಅವರ ಇಡೀ ಜೀವನವೇ ಬದಲಾಯಿಸಿತೆಂದೂ ಹೇಳಿದರು. ಗುರುಗಳು ತಿನ್ನುವುದು ಮುಂತಾದುವುಗಳ ಮೇಲೆ ಅವರಿಗೆ ಅಷ್ಟು ಗಮನ ಕೊಡು ಎಂದು ಹೇಳಿರಲಿಲ್ಲ. ತಾಂಬೂಲಸೇವನೆ ಹಿಂದಿನಿಂದ ಬಂದ ಅಭ್ಯಾಸವಾಗಿತ್ತು. ಅದು ಆಧ್ಯಾತ್ಮಿಕ ಜೀವನಕ್ಕೆ ಆತಂಕವಿಲ್ಲದ್ದೇ ಇರುವುದರಿಂದ ಅದನ್ನು ಹಾಗೆಯೇ ಬಿಟ್ಟಿರುವೆ ಎಂದರು. ಮನೆಯವರು ಸ್ವಾಮಿಯವರನ್ನು, ಅವರು ಶಾಖಾಹಾರಿಗಳೇ, ಎಂದು ಕೇಳಿದಾಗ ಸ್ವಾಮೀಜಿಯವರು “ನಾನು ‘ಪರಮಹಂಸ’ ಎಂಬ ಸಂನ್ಯಾಸಿಗಳ ಪಂಥಕ್ಕೆ ಸೇರಿದವನು. ಭಿಕ್ಷೆಗೆ ಹೋದರೆ ಏನು ಸಿಕ್ಕುವುದೋ ಅದನ್ನು ತೆಗೆದುಕೊಳ್ಳುವೆನು. ಸಿಕ್ಕದೇ ಇದ್ದರೆ ಉಪವಾಸವಿರುವೆನು” ಎಂದರು. ಸ್ವಾಮೀಜಿಯವರನ್ನು ಹಿಂದೂಗಳಲ್ಲದವರಿಂದ ಊಟವನ್ನು ಸ್ವೀಕರಿಸುವರೆ ಎಂದು ಪ್ರಶ್ನಿಸಿದುದಕ್ಕೆ, ಅವರು ತಾವು ಅನೇಕ ವೇಳೆ ಮಹಮ್ಮದೀಯರ ಮನೆಗಳಲ್ಲಿಯೂ ಊಟಮಾಡಿರುವುದಾಗಿ ಹೇಳಿದರು. 

 ಮನೆಯ ಯಜಮಾನನ ಮಗ ಪಾಣಿನಿಯ ಅಷ್ಟಾಧ್ಯಾಯಿಯನ್ನು ಓದುತ್ತಿದ್ದ. ಆದರೆ ಅದು ಸುಲಭವಾಗಿ ತಲೆಗೆ ಹತ್ತುತ್ತಿರಲಿಲ್ಲ. ಮನೆಯ ಯಜಮಾನ ತನ್ನ ಹುಡುಗನನ್ನು ಕರೆದು ಸ್ವಾಮೀಜಿ ಎದುರಿಗೆ ವ್ಯಾಕರಣದಲ್ಲಿ ಏನು ಕಲಿತಿರುವೆಯೋ ಅದನ್ನು ಹೇಳು ಎಂದರು. ಆ ಹುಡುಗ ಮಧ್ಯೆ ಮಧ್ಯೆ ತಪ್ಪಿದಾಗ ಸ್ವಾಮೀಜಿಯವರೇ ಅದನ್ನು ತಿದ್ದಿದರು. 

 ಮನೆಯವರು ಸ್ವಾಮೀಜಿಯವರೊಡನೆ ಚರ್ಚೆಮಾಡಲು ಬೆಳಗಾವಿಯಲ್ಲಿರುವ ಅನೇಕ ಬುದ್ಧಿವಂತರನ್ನು ಕರೆಯಿಸುತ್ತಿದ್ದರು. ಸ್ವಾಮೀಜಿಯವರಾದರೋ ಲೀಲಾಜಾಲವಾಗಿ ಅವರನ್ನು ವಾದದಲ್ಲಿ ಸೋಲಿಸಿಬಿಡುತ್ತಿದ್ದರು. ಆದರೆ ಅವರನ್ನು ಸೋಲಿಸಿದೆ ಎಂಬ ಅಹಂಕಾರವಿರಲಿಲ್ಲ ಅವರಲ್ಲಿ. ಅವರ ಟೀಕೆಯ ಹಿಂದೆ ವಿಷವಿರಲಿಲ್ಲ. ವಾದಮಾಡುವಾಗ ಎಂದಿಗೂ ಉದ್ವಿಗ್ನರಾಗುತ್ತಿರಲಿಲ್ಲ. ಶಾಂತಚಿತ್ತರಾಗಿ, ಕೆಲವು ವೇಳೆ ತಮಾಷೆಯಾಗಿ ಉತ್ತರವನ್ನು ಹೇಳುತ್ತಿದ್ದರು. ಇತರರು ವಾದಮಾಡುವಾಗ ಸೋತು ಕೋಪಗೊಂಡರೂ ಇವರು ಅದಕ್ಕೆ ಅವಕಾಶ ಕೊಡುತ್ತಿರಲಿಲ್ಲ. ಸ್ವಾಮೀಜಿಯವರೊಡನೆ ವಾದದಲ್ಲಿ ಕತ್ತಿಮಸೆಯಬೇಕಾದರೆ, ಅನುಭವದಲ್ಲಿ, ವಿದ್ವತ್ತಿನಲ್ಲಿ, ಜಾಣತನದಲ್ಲಿ ಸರಿಸಮಾನರಾಗಿರಬೇಕು ಇಲ್ಲವೆ, ಮೇಲಾಗಿರಬೇಕಾಗಿತ್ತು. ಅಂತಹ ವ್ಯಕ್ತಿಗಳು ಅಪರೂಪ ಅಥವಾ ಯಾರೂ ಇಲ್ಲವೆಂದೇ ಹೇಳಬಹುದು. ಮಹಾ ಮಹಾ ವಿದ್ವಾಂಸರ ವಾದಗಳೇ ಸ್ವಾಮೀಜಿಗೆ ಮಕ್ಕಳ ಆಟದಂತೆ ಇರುತ್ತಿದ್ದವು. 

 ಬೆಳಗಾವಿಯಲ್ಲಿ ಅರಣ್ಯ ಇಲಾಖೆಯಲ್ಲಿ ಕೆಲಸ ಮಾಡುತ್ತಿದ್ದ ಬಾಬು ಹರಿಪದ ಮಿತ್ರ ಎಂಬ ಬಂಗಾಳಿಯವರ ಮನೆಗೆ ಸ್ವಾಮೀಜಿ ವಕೀಲರೊಡನೆ ಹೋದರು. ಹರಿಪದ ಮಿತ್ರರಿಗೆ ಮೊದಲಿನಿಂದಲೂ ಸಂನ್ಯಾಸಿಗಳನ್ನು ಕಂಡರೆ ಅಸಡ್ಡೆ. ಅವರೆಲ್ಲ ಆಷಾಢಭೂತಿಗಳೆಂದೂ, ಅವರಿಗೆ ಯಾವ ಅನುಭವವೂ ಇರುವುದಿಲ್ಲ ಎಂದೂ ಭಾವಿಸಿದ್ದರು. ಸ್ವಾಮೀಜಿಯವರನ್ನು ನೋಡಿ, ಎಲ್ಲೊ ತಮ್ಮ ಮನೆಯಲ್ಲಿ ಇರಲು ಇವರು ಕೇಳುವುದಕ್ಕೆ ಬಂದಿರಬಹುದು ಎಂದು ಭಾವಿಸಿದರು. ಆದರೆ ಸ್ವಾಮೀಜಿಯವರೊಡನೆ ಮಾತನಾಡಿದ ಮೇಲೆ ತನಗಿಂತ ಇವರು ಸಾವಿರಪಾಲು ಎಲ್ಲಾ ವಿಧದಲ್ಲಿಯೂ ಮೇಲೆ ಎನ್ನಿಸಿತು. ಅನಂತರ ತಮ್ಮ ಮನೆಯಲ್ಲಿಯೇ ಇರಬೇಕೆಂದು ಅವರೇ ಸ್ವಾಮೀಜಿಯವರನ್ನು ಕೇಳಿಕೊಂಡರು. ಆದರೆ ಸ್ವಾಮೀಜಿ ಒಪ್ಪಲಿಲ್ಲ. “ಮಹಾರಾಷ್ಟ್ರ ಬ್ರಾಹ್ಮಣ ವಕೀಲನ ಮನೆಯಲ್ಲಿ ನನ್ನನ್ನು ಚೆನ್ನಾಗಿ ನೋಡಿಕೊಳ್ಳುತ್ತಿರುವರು. ಒಬ್ಬ ಬಂಗಾಳಿ ಸಿಕ್ಕಿಬಿಟ್ಟನೆಂದು ಅವರನ್ನು ತ್ಯಜಿಸಿ ಬರುವುದು ಯೋಗ್ಯವಲ್ಲ” ಎಂದರು. ಮಾರನೆ ದಿನ ತಮ್ಮ ಮನೆಗೆ ತಿಂಡಿಗಾದರೂ ಬನ್ನಿ ಎಂದಾಗ ಸ್ವಾಮೀಜಿ ಅದಕ್ಕೆ ಒಪ್ಪಿದರು. 

\newpage

 ಮಾರನೆ ದಿನ ಬೆಳಗ್ಗೆ ಹರಿಪದ ಬಾಬು ಸ್ವಾಮೀಜಿಯವರಿಗೆ ಕಾದರು. ಆದರೆ ಎಷ್ಟು ಹೊತ್ತಾದರೂ ಬರಲಿಲ್ಲ. ಹರಿಪದ ಬಾಬುವೇ ಸ್ವಾಮೀಜಿ ಇದ್ದ ಮನೆಗೆ ಹೋದರು. ಅಲ್ಲಿ ನೋಡಲಾಗಿ, ಅವರ ಮನೆಯಲ್ಲಿ ಹಲವು ವಿದ್ಯಾವಂತರು, ವಕೀಲರು, ಪಂಡಿತರು ಸ್ವಾಮಿಗಳೊಡನೆ ಮಾತನಾಡುತ್ತಿದ್ದರು. ಹಲವು ಪ್ರಶ್ನೆಗಳನ್ನು ಅವರಿಗೆ ಹಾಕುತ್ತಿದ್ದರು. ಸ್ವಾಮೀಜಿ ಸ್ವಲ್ಪವೂ ಅನುಮಾನಿಸಿದೆ ತಕ್ಷಣವೆ ಇಂಗ್ಲೀಷ್, ಹಿಂದಿ, ಸಂಸ್ಕೃತ ಮುಂತಾದ ಭಾಷೆಗಳಲ್ಲಿ ಉತ್ತರವನ್ನು ಕೊಡುತ್ತಿದ್ದರು. ಹರಿಪದ ಬಾಬು ಸ್ವಾಮೀಜಿಗೆ ನಮಸ್ಕಾರ ಮಾಡಿ ಅವುಗಳೆಲ್ಲ ಮುಗಿಯುವವರೆಗೆ ಕಾದರು. ಮಾತನಾಡುವುದಕ್ಕೆ ಬಂದವರೆಲ್ಲ ಹೊರಟು ಹೋದಮೇಲೆ ಸ್ವಾಮೀಜಿ ಹರಿಪದ ಬಾಬುವನ್ನು ನೋಡಿ, “ಇವರೆಲ್ಲ ಬಂದಿದ್ದರು ಇವರನ್ನು ಬಿಟ್ಟು ಬರಲು ನನಗೆ ಮನಸ್ಸಾಗಲಿಲ್ಲ” ಎಂದರು. ಹರಿಪದ ಬಾಬು ಭಾಟೆಯವರನ್ನು ಒಪ್ಪಿಸಿ ಸ್ವಾಮೀಜಿಯವರನ್ನು ತಮ್ಮ ಮನೆಯಲ್ಲಿ ಇಟ್ಟುಕೊಳ್ಳಲು ಕರೆದುಕೊಂಡು ಹೋದರು. 

 ಹರಿಪದ ಬಾಬುಗಳ ಮನೆಯಲ್ಲಿ ಸ್ವಾಮೀಜಿ ಮೂರು ದಿನಗಳನ್ನು ಕಳೆದರು. ಹರಿಪದ ಬಾಬುವಿನ ಮನಸ್ಸಿನಲ್ಲಿ ಇದುವರೆಗು ಇದ್ದ ಸಂದೇಹಗಳೆಲ್ಲ ಪರಿಹಾರವಾದವು. ಆ ಊರಿನ ಅನೇಕ ಜನರು ಸ್ವಾಮೀಜಿ ಬಳಿಗೆ ಬಂದು ಹೋಗುತ್ತಿದ್ದರು. ಅವರು ಕೂಡ ಹರಿಪದ ಬಾಬುಗಳಂತೆ ಶಾಂತಿಯನ್ನು ಪಡೆದರು. ಅನಂತರ ಸ್ವಾಮೀಜಿ ತಾವು ಇನ್ನು ಮುಂದಕ್ಕೆ ಹೋಗುತ್ತೇನೆ ಎಂದು ಹೇಳಿದಾಗ ಹರಿಪದ ಬಾಬು ಇನ್ನು ಕೆಲವು ದಿನಗಳು ಇರಿ ಎಂದು ಬಲಾತ್ಕಾರ ಪಡಿಸಿದುದರಿಂದ ಸ್ವಾಮೀಜಿಯವರು ಮತ್ತೆ ಕೆಲವು ದಿನಗಳು ಇದ್ದರು. ಸ್ವಾಮೀಜಿಯವರು ಹರಿಪದ ಬಾಬುವಿನ ಮನೆಯಲ್ಲಿದ್ದಾಗ ತಮ್ಮ ಪರಿವ್ರಾಜಕ ಜೀವನದ ಹಲವು ಘಟನೆಗಳನ್ನು ಹೇಳಿದರು. ಒಮ್ಮೆ ಅವರಿಗೆ ಕೊಟ್ಟ ಆಹಾರ ತುಂಬ ಖಾರವಾಗಿತ್ತು. ಅದನ್ನು ತಿಂದಾದ ಮೇಲೆ ಬಾಯಿ ಮತ್ತು ಹೊಟ್ಟೆ ಬಹಳ ಕಾಲದದವರೆಗೆ ಉರಿಯುತ್ತಿತ್ತು. ಅದರ ಉಪಶಮನಕ್ಕೆ ಮತ್ತೊಬ್ಬನಿಂದ ಹುಣಸೆಹಣ್ಣನ್ನು ತೆಗೆದುಕೊಂಡು ಅದನ್ನು ನೀರಿನಲ್ಲಿ ನೆನಸಿ ಆ ನೀರನ್ನು ಕುಡಿದೆನೆಂದು ಹೇಳಿದರು. ಮತ್ತೊಬ್ಬರ ಮನೆಗೆ ಹೋದಾಗ ಆ ಮನೆಯವರು, ಇಲ್ಲಿ ಸಾಧುಗಳಿಗೆ ಮತ್ತು ನಾಯಿಗಳಿಗೆ ಸ್ಥಳವಿಲ್ಲ ಎಂದು ಹೇಳಿ ಕಳುಹಿಸಿದರು. ಒಮ್ಮೊಮ್ಮೆ ಸಿ.ಐ.ಡಿ. ಇಲಾಖೆಯವರು ಸ್ವಾಮೀಜಿಯವರ ಮೇಲೆ ಹಲವು ದಿನಗಳವರೆಗೆ ಕಣ್ಣಿಟ್ಟಿದ್ದರು. ಸ್ವಾಮೀಜಿ ಇದನ್ನೆಲ್ಲ ತಮಾಷೆಯಂತೆ ಹೇಳುತ್ತಿದ್ದರು. ಇದೆಲ್ಲ ಜಗನ್ಮಯಿಯ ಆಟ ಎಂದರು. 

 ಸ್ವಾಮೀಜಿಯವರ ಜ್ಞಾಪಕಶಕ್ತಿ ಅದ್ಭುತವಾದುದು. ಒಮ್ಮೆ ಅವರು ಮಾತನಾಡುತ್ತಿದ್ದಾಗ ಚಾರ್ಲ್ಸ್ ಡಿಕನ್ಸನ್ ಬರೆದಿರುವ ‘ಪಿಕ್‍ವಿಕ್ ಪೇಪರ್ಸ್’ನಿಂದ ಒಂದೊಂದು ಪುಟವನ್ನೇ ಉದಾಹರಿಸುತ್ತಿದ್ದರು. ಹರಿಪದ ಬಾಬು ಅದನ್ನು ಕೇಳಿ “ಸ್ವಾಮೀಜಿ ಅದನ್ನು ಎಷ್ಟು ಬಾರಿ ಓದಿರುವಿರಿ?” ಎಂದು ಪ್ರಶ್ನೆ ಹಾಕಿದಾಗ, ಸ್ವಾಮೀಜಿ “ಕೇವಲ ಎರಡು ಬಾರಿ” ಎಂದು ಉತ್ತರ ಕೊಟ್ಟರು. ಹರಿಪದ ಬಾಬುಗಳು ಹಲವು ಔಷಧಿಗಳನ್ನು ತೆಗೆದುಕೊಳ್ಳುತ್ತಿದ್ದರು. ಇದೊಂದು ಅವರ ಚಾಳಿಯಾಗಿ ಹೋಗಿತ್ತು. ಆಗ ಸ್ವಾಮೀಜಿ ಅವರಿಗೆ, “ಮುಕ್ಕಾಲು ಪಾಲು ಖಾಯಿಲೆಗಳಿಗೆ ಮನಸ್ಸೇ ಮೂಲ. ಮನಸ್ಸನ್ನು ಸರಿಮಾಡಿದರೆ ಅದು ಸರಿಯಾಗುವುದು” ಎಂದರು. “ಯಾವಾಗಲೂ ರೋಗವನ್ನೇ ಕುರಿತು ಚಿಂತಿಸುತ್ತಿದ್ದರೆ ಪ್ರಯೋಜನವೇನು? ಸಂತೋಷಚಿತ್ತನಾಗಿರು, ಒಳ್ಳೆಯ ಜೀವನವನ್ನು ನಡೆಸು. ಒಳ್ಳೆಯ ಆಲೋಚನೆಗಳನ್ನು ಮಾಡು. ಅನಂತರ ಪಶ್ಚಾತ್ತಾಪ ಪಡುವಂತಹ, ದೇಹವನ್ನು ಕ್ಷೀಣಿಸುವ ಯಾವ ಸುಖಕ್ಕೂ ಕೈಹಾಕಬೇಡ. ಆಗಲೆ ಎಲ್ಲವೂ ಸರಿಯಾಗುವುದು. ಒಂದು ವೇಳೆ ನನ್ನಂತಹ ಮತ್ತು ನಿನ್ನಂತಹ ಕೆಲವು ವ್ಯಕ್ತಿಗಳು ಸತ್ತರೆ ಪ್ರಪಂಚಕ್ಕೆ ಏನು ನಷ್ಟ? ಇದರಿಂದ ಪ್ರಪಂಚವೇನೂ ಬಿದ್ದು ಹೋಗಿಬಿಡುವುದಿಲ್ಲ. ನಾವಿಲ್ಲದೇ ಪ್ರಪಂಚ ನಡೆಯುವುದಿಲ್ಲ ಎಂದು ನೀನು ಭಾವಿಸಬಾರದು” ಎಂದರು. ಅನಂತರ ಹರಿಪದ ಬಾಬು ತಮ್ಮ ಅಭ್ಯಾಸವನ್ನು ತ್ಯಜಿಸಿದರು. 

 ಹರಿಪದ ಬಾಬು ಸರ್ಕಾರದಲ್ಲಿ ದೊಡ್ಡ ಹುದ್ದೆಯಲ್ಲೇ ಇದ್ದರು. ಚೆನ್ನಾಗಿ ಸಂಬಳವೂ ಬರುತ್ತಿತ್ತು. ಆದರೂ ಮೇಲಿನವರ ಕೈಯಿಂದ ಕೆಲವು ವೇಳೆ ಮಾತುಗಳನ್ನು ಕೇಳಿದಾಗ ತಾವು ಎಂತಹ ಬಂಧನದಲ್ಲಿ ಸಿಕ್ಕಿ ನರಳುತ್ತಿರುವೆ ಎಂದು ಗೊಣಗುತ್ತಿದ್ದರು. ಅದಕ್ಕೆ ಸ್ವಾಮೀಜಿ ಹೀಗೆ ಹೇಳಿದರು: “ನೀನೇ ಹಣಕ್ಕಾಗಿ ಈ ಕೆಲಸವನ್ನು ತೆಗೆದುಕೊಂಡಿರುವೆ. ಅದಕ್ಕೆ ಸರಿಯಾಗಿ ಅವರು ನಿನಗೆ ಸಂಬಳ ಕೊಡುತ್ತಿರುವರು. ಅಯ್ಯೋ, ನಾನು ಎಂತಹ ಬಂಧನದಲ್ಲಿರುವೆ ಎಂದು ಚಿಂತಿಸುತ್ತ ಏತಕ್ಕೆ ಕೊರಗುವೆ? ನಿನ್ನನ್ನು ಯಾರೂ ಬಂಧನದಲ್ಲಿ ಇಟ್ಟಿಲ್ಲ. ನೀನು ಇಚ್ಛೆಪಟ್ಟರೆ ಕೆಲಸಕ್ಕೆ ರಾಜೀನಾಮೆ ಕೊಡಬಹುದು. ನಿನ್ನ ಮೇಲಿರುವವರಲ್ಲಿ ಏತಕ್ಕೆ ಯಾವಾಗಲೂ ತಪ್ಪನ್ನು ಕಂಡುಹಿಡಿಯುತ್ತಿರುವೆ? ನಿನ್ನ ಈಗಿನ ಸ್ಥಿತಿಯಿಂದ ನಿನಗೆ ತೊಂದರೆ ಆಗಿದ್ದರೆ ಅದಕ್ಕೆ ನೀನೇ ಕಾರಣ. ನೀನು ಕೆಲಸಕ್ಕೆ ರಾಜೀನಾಮೆ ಕೊಟ್ಟರೆ ಅವರು ಲೆಕ್ಕಿಸುವರು ಎಂದು ಭಾವಿಸುವೆಯಾ? ನಿನ್ನ ಸ್ಥಳಕ್ಕೆ ಬರುವುದಕ್ಕೆ ನೂರಾರು ಜನರು ಇರುವರು. ನೀನು ನಿನ್ನ ಕೆಲಸ ಮತ್ತು ಅದರ ಜವಾಬ್ದಾರಿಯನ್ನು ಮಾತ್ರ ಗಮನಿಸಬೇಕು. ನೀನು ಒಳ್ಳೆಯವನಾಗು, ಆಗ ಪ್ರಪಂಚವೇ ಒಳ್ಳೆಯದಾಗುವುದು. ಆಗ ನೀನು ಇತರರಲ್ಲಿ ಒಳ್ಳೆಯದನ್ನೇ ನೋಡುವೆ. ನಾವು ಒಳಗೆ ಇರುವುದನ್ನೇ ಹೊರಗೆ ನೋಡುವೆವು. ಇತರರಲ್ಲಿ ತಪ್ಪು ಕಂಡುಹಿಡಿಯುವುದನ್ನು ಬಿಟ್ಟುಬಿಡು. ಅನಂತರ ನಿನ್ನನ್ನು ಕಂಡರೆ ಆಗದವರು ಕ್ರಮೇಣ ನಿನ್ನ ಕಡೆಗೆ ಹೇಗೆ ಬದಲಾಯಿಸುವರು ಎಂಬುದನ್ನು ನೋಡು.” ಹರಿಪದ ಬಾಬು ಇದನ್ನು ಕೇಳಿ ತನ್ನ ಜೀವನದ ದೃಷ್ಟಿಯನ್ನೆ ಬದಲಾಯಿಸಿಕೊಂಡರು. 

 ಹರಿಪದ ಬಾಬು ಭಗವದ್ಗೀತೆಯನ್ನು ಓದುತ್ತಿದ್ದರು. ಆದರೆ ಅದರ ಸೂಕ್ಷ್ಮವಾದ ಸಂದೇಶವನ್ನು ಗ್ರಹಿಸಲು ಸಾಧ್ಯವಾಗದೆ ಬಿಟ್ಟುಬಿಟ್ಟಿದ್ದರು. ಸ್ವಾಮೀಜಿ ಗೀತೆಯ ಕೆಲವು ಶ್ಲೋಕಗಳನ್ನು ವಿವರಿಸಿದಾಗ ಎಂತಹ ಜ್ಞಾನಭಂಡಾರ ಅಲ್ಲಿದೆ ಎಂಬುದು ಅರ್ಥವಾಯಿತು. ಅನಂತರ ಆ ದೃಷ್ಟಿಯಿಂದ ಪುನಃ ಗೀತೆಯನ್ನು ಓದತೊಡಗಿದರು. ಅದರಂತೆಯೇ ಥಾಮಸ್ ಕಾರ‍್ಲೈಲ್‍ನ ಕೃತಿಗಳು, ಜೂಲ್ಸ್ ‍ವರ್​ನ ಕಾದಂಬರಿಗಳನ್ನು ಓದಿ ಆನಂದಿಸುವುದನ್ನು ಸ್ವಾಮೀಜಿಯಿಂದ ಕಲಿತರು. ಸ್ವಾಮೀಜಿಯವರಿಗೆ ಮತಭ್ರಾಂತರನ್ನು ಕಂಡರೆ ಆಗದು. ಅದನ್ನು ವಿವರಿಸುವುದಕ್ಕೆ ಒಂದು ಹಾಸ್ಯದ ಕಥೆಯನ್ನು ಹೇಳಿದರು: “ಒಂದು ಊರಿನಲ್ಲಿ ಒಬ್ಬ ರಾಜನಿದ್ದ. ಪಕ್ಕದ ರಾಜರು\break ಸೈನ್ಯದೊಡನೆ ಅವನ ಮೇಲೆ ಧಾಳಿ ನಡೆಸಲು ಬರುತ್ತಿರುವುದನ್ನು ಕೇಳಿ ಒಂದು ಸಭೆಯನ್ನು ನಡೆಸಿದ. ಶತ್ರುಗಳನ್ನು ತಡೆಯಬೇಕಾದರೆ ಏನು ಮಾಡಬೇಕು ಎಂದು ಎಲ್ಲರ ಸಲಹೆ ಕೇಳಿದನು ಇಂಜಿನಿಯರ್ ಎದ್ದು, ಒಂದು ಕೋಟೆಯನ್ನು ಕಟ್ಟಿ, ಅದರ ಹತ್ತಿರ ಒಂದು ಕಂದಕವನ್ನು ತೆಗೆಯಬೇಕು, ಅದರೊಳಗೆ ನೀರನ್ನು ಬಿಡಬೇಕು ಎಂದನು. ಬಡಗಿಯವನು ಮರದಿಂದ ಒಂದು ಗೋಡೆಯನ್ನು ಕಟ್ಟಬೇಕು ಎಂದನು. ಚರ್ಮದೊಂದಿಗೆ ಕೆಲಸ ಮಾಡುವವನು ಚರ್ಮದಿಂದ ಕೋಟೆಯನ್ನು ಹಾಕಬೇಕು ಎಂದನು. ಏಕೆಂದರೆ ಚರ್ಮದಷ್ಟು ಮತ್ತಾವುದೂ ಶ್ರೇಷ್ಠವಲ್ಲ ಎಂದನು. ಕಮ್ಮಾರ ಎದ್ದು ಇವರದೆಲ್ಲ ತಪ್ಪು, ಕಬ್ಬಿಣದ ತಗಡಿನಿಂದಲೇ ಕೋಟೆಯನ್ನು ಮಾಡಬೇಕು ಎಂದನು. ವಕೀಲ ಎದ್ದು ಶತ್ರುಗಳಿಗೆ, ಮತ್ತೊಬ್ಬರ ರಾಷ್ಟ್ರವನ್ನು ಅಪಹರಿಸುವುದು ಅನ್ಯಾಯ ಎಂಬುದನ್ನು ವಿವರಿಸೋಣ ಎಂದನು. ಕೊನೆಗೆ ಪುರೋಹಿತನು ಬಂದನು. ಉಳಿದವರಿಗೆಲ್ಲ, ನೀವೆಲ್ಲ ಹುಚ್ಚರಂತೆ ಮಾತನಾಡುತ್ತಿರುವಿರಿ, ಮುಂಚೆ ಯಾಗಯಜ್ಞಗಳನ್ನು ಮಾಡಿ ದೇವತೆಗಳನ್ನು ತೃಪ್ತಿಪಡಿಸಬೇಕು, ಅನಂತರ ನಮ್ಮನ್ನು ಯಾರೂ ಗೆಲ್ಲುವುದಕ್ಕೆ ಆಗುವುದಿಲ್ಲ ಎಂದನು. ದೇಶವನ್ನು ರಕ್ಷಿಸುವ ಬದಲು ಅವರೆಲ್ಲ ತಮ್ಮೊಳಗೆ ತಾವೇ ಕಾದಾಡುತ್ತಿದ್ದರು. ಶತ್ರುಗಳು ಬಂದು ಊರನ್ನು ಕೊಳ್ಳೆ ಹೊಡೆದುಕೊಂಡು ಹೋದರು. ಇದರಂತೆಯೆ ಮತಭ್ರಾಂತರು.” 

 ಸರ್ವಸಂಗ ಪರಿತ್ಯಾಗ ಮಾಡಿದ ಸಂನ್ಯಾಸಿಯಾದರೆ ಅಹಿಂಸೆಯನ್ನು ಅನುಷ್ಠಾನ ಮಾಡಬಹುದು; ಆದರೆ ಹಲವು ಹೊರೆ ಹೊಣೆಗಳುಳ್ಳ ಗೃಹಸ್ಥರಾದರೊ ಭುಸುಗುಟ್ಟಬೇಕು, ಆದರೆ ಕಚ್ಚಕೂಡದು. ಸ್ವಾಮೀಜಿ ಈ ನೀತಿಯನ್ನು ಗೃಹಸ್ಥರು ಅನುಸರಿಸಬೇಕೆಂದು ಹೇಳುತ್ತಿದ್ದರು. ಒಂದುದಿನ ಹರಿಪದ ಒಬ್ಬರೇ ಇದ್ದಾಗ ಸ್ವಾಮೀಜಿ ತಾವು ಅಮೇರಿಕಾ ದೇಶಕ್ಕೆ ಹೋಗಬೇಕೆಂದಿರುವುದನ್ನು ಹೇಳಿದರು. ಅವರದಕ್ಕೆ ಸಂತೋಷಪಟ್ಟು ಸ್ವಾಮೀಜಿಯವರ ಖರ್ಚಿಗೆ ಚಂದಾ ಎತ್ತುವುದರಲ್ಲಿದ್ದರು. ಆದರೆ ಸ್ವಾಮೀಜಿ, ಸದ್ಯಕ್ಕೆ ಚಂದಾವನ್ನು ಎತ್ತಬೇಡ ಎಂದರು. ಹರಿಪದ ಬಾಬುಗಳ ಹೆಂಡತಿ ಮಂತ್ರೋಪದೇಶವನ್ನು ತೆಗೆದುಕೊಳ್ಳಬೇಕೆಂದು ಬಯಸಿದ್ದಳು. ಅದಕ್ಕೆ ತಕ್ಕ ಯೋಗ್ಯ ಗುರು ಸಿಕ್ಕಿದರೆ ಬೇಕಾದರೆ ತೆಗೆದುಕೊಳ್ಳಬಹುದು, ಇಲ್ಲದೇ ಇದ್ದರೆ ಜೀವಾವಧಿ ಪಶ್ಚಾತ್ತಾಪ ಪಡಬೇಕಾಗುವುದೆಂದಿದ್ದರು, ಸ್ವಾಮೀಜಿ ಇವರ ಮನೆಗೆ ಬರುವುದಕ್ಕೆ ಮುಂಚೆ. ಈಗ ಸ್ವಾಮೀಜಿಯವರನ್ನು ಕಂಡಾದ ಮೇಲೆ ಸ್ವಾಮೀಜಿ ಮಂತ್ರೋಪದೇಶ ಕೊಡುವುದಕ್ಕೆ ಒಪ್ಪಿದರೆ, ತನ್ನ ಸತಿ ಅವರಿಂದ ಉಪದೇಶವನ್ನು ಪಡೆಯಬಹುದೆಂದು ಹೇಳಿದರು. ಮೊದಲು ಸ್ವಾಮೀಜಿ ಅದಕ್ಕೆ ಒಪ್ಪಲಿಲ್ಲ. ಅನಂತರ ಅವರ ಬಲಾತ್ಕಾರಕ್ಕೆ ಕಟ್ಟುಬಿದ್ದು ದಂಪತಿಗಳಿಬ್ಬರಿಗೂ ಉಪದೇಶವನ್ನು ಕೊಟ್ಟು ಆ ಊರನ್ನು ಬಿಟ್ಟರು. 

 ಸ್ವಾಮೀಜಿ ಬೆಳಗಾಂನಲ್ಲಿದ್ದಾಗ ಕೆಲವು ದಿನಗಳ ಮಟ್ಟಿಗೆ ಗೋವಾಕ್ಕೆ ಹೋಗಿಬಂದಿದ್ದರು. ಅನಂತರ ಅವರು ಮೈಸೂರಿಗೆ ಬಂದರು. ಮೈಸೂರಿನಲ್ಲಿ ಕೆಲವು ಕಾಲ ಅನಾಮಧೇಯರಾಗಿ ಸಂಚರಿಸುತ್ತಿದ್ದವರು ಕೊನೆಗೆ ಒಂದು ದಿನ ದಿವಾನ್ ಸರ್ ಕೆ. ಶೇಷಾದ್ರಿ ಅಯ್ಯರ್ ಅವರ ಪರಿಚಯ ಮಾಡಿಕೊಂಡರು. ದಿವಾನರು ಸ್ವಾಮೀಜಿಯವರೊಂದಿಗೆ ಸ್ವಲ್ಪ ಹೊತ್ತು ಮಾತನಾಡುವುದರೊಳಗಾಗಿಯೇ ಅವರ ಪ್ರತಿಭೆಯ ಪರಿಚಯ ಇವರಿಗೆ ಆಯಿತು. ಸ್ವಾಮೀಜಿಯವರಲ್ಲಿ ಒಂದು ಅದ್ಭುತವಾದ ಆಕರ್ಷಣೆ ಇತ್ತು; ಈ ದೈವೀಶಕ್ತಿ ಭರತಖಂಡದ ಇತಿಹಾಸದಲ್ಲಿ ತನ್ನ ಪ್ರಭಾವವನ್ನು ಬೀರುವುದರಲ್ಲಿ ಸಂದೇಹವಿಲ್ಲ ಎಂಬುದು ದಿವಾನರಿಗೆ ಹೊಳೆಯಿತು. ಸ್ವಾಮೀಜಿ ಮೂರು ನಾಲ್ಕು ವಾರಗಳವರೆಗೆ ಶೇಷಾದ್ರಿ ಅವರ ಮನೆಯಲ್ಲಿದ್ದರು. ಅವರನ್ನು ನೋಡಲು ಕೇವಲ ಹಿಂದೂಗಳು ಮಾತ್ರವೇ ಬರುತ್ತಿರಲಿಲ್ಲ, ಅನ್ಯಮತೀಯರೂ ಬರುತ್ತಿದ್ದರು. ಶೇಷಾದ್ರಿ ಅಯ್ಯರ್ ಸ್ವಾಮೀಜಿಯವರನ್ನು ಬಹಳ ಮೆಚ್ಚಿದರು. ಒಂದು ದಿನ ಅವರು ತಮ್ಮ ಆಪ್ತರಿಗೆ ಹೀಗೆ ಹೇಳಿದರು: “ನಮ್ಮಲ್ಲಿ ಅನೇಕರು ಧಾರ್ಮಿಕ ವಿಷಯವಾಗಿ ಬೇಕಾದಷ್ಟು ಓದಿರುವರು. ಆದರೂ ಅದರಿಂದ ಏನು ಪ್ರಯೋಜನವಾಯಿತು? ಈ ಯುವಕ ಸಂನ್ಯಾಸಿಯ ಅಂತರ್‍ದೃಷ್ಟಿ ಮತ್ತು ಜ್ಞಾನ ನಾನು ನೋಡಿರುವವರೆಲ್ಲರನ್ನೂ ಮೀರಿಸುವುದು. ಇದೊಂದು ಅದ್ಭುತ. ಇವರು ಹುಟ್ಟುವಾಗಲೇ ಬ್ರಹ್ಮಜ್ಞಾನಿಗಳಾಗಿ ಹುಟ್ಟಿರಬೇಕು. ಇಲ್ಲದೇ ಇದ್ದರೆ ಇಷ್ಟು ಸಣ್ಣ ವಯಸ್ಸಿನಲ್ಲಿ ಇಷ್ಟೊಂದು ಜ್ಞಾನವನ್ನು ಗಳಿಸುವುದೆಂದರೇನು?” ಸ್ವಾಮೀಜಿಯವರನ್ನು ದಿವಾನರು ಚಾಮರಾಜೇಂದ್ರ ಒಡೆಯರವರಿಗೆ ಪರಿಚಯ ಮಾಡಿಸಿದರು. ಅನಂತರ ಸ್ವಾಮೀಜಿ ರಾಜರ ಅತಿಥಿಗಳಾದರು. ರಾಜರು ಸ್ವಾಮೀಜಿಯವರ ಬಳಿ ಹಲವು ಮುಖ್ಯವಾದ ವಿಷಯಗಳ ಮೇಲೆ ಸಲಹೆಗಳನ್ನು ತೆಗೆದುಕೊಳ್ಳುತ್ತಿದ್ದರು. 

 ಮಹಾರಾಜರು ಒಂದು ದಿನ ಆಸ್ಥಾನದಲ್ಲಿದ್ದಾಗ ಸ್ವಾಮೀಜಿಯವರನ್ನು ತಮ್ಮ ಆಸ್ಥಾನದ ಜನರು ಹೇಗೆ ಎಂದು ಪ್ರಶ್ನೆ ಮಾಡಿದರು. ಸ್ವಾಮೀಜಿ, “ನೀವು ಬಹಳ ಒಳ್ಳೆಯವರು, ಆದರೆ ನೀವು ನಿಮ್ಮ ಆಸ್ಥಾನಿಕರಿಂದ ಆವೃತರಾಗಿರುವಿರಿ. ಆಸ್ಥಾನದಲ್ಲಿರುವವರು ಎಂದಿಗೂ ಸ್ತುತಿಪಾಠಕರೇ” ಎಂದರು. ಅದಕ್ಕೆ ಮಹಾರಾಜರು “ನಮ್ಮ ದಿವಾನರಾದರೊ ಹಾಗಲ್ಲ. ಅವರು ಬಹಳ ಬುದ್ಧಿವಂತರು ಮತ್ತು ಪ್ರಾಮಾಣಿಕರು” ಎಂದು ಹೇಳಿದರು. ಸ್ವಾಮೀಜಿ, “ಮಹಾರಾಜರೆ, ದಿವಾನರ ಕೆಲಸ ರಾಜನಿಂದ ವಸೂಲಿಮಾಡಿ ಪೊಲಿಟಿಕಲ್ ಏಜಂಟಿಗೆ (ರೆಸಿಡೆಂಟ್) ಕೊಡುವುದು” ಎಂದು ನಿರ್ಭಯವಾಗಿ ಹೇಳಿಬಿಟ್ಟರು. ರಾಜರು ಸಂಭಾಷಣೆಯನ್ನು ಬದಲಾಯಿಸಿದರು. ಆಸ್ಥಾನದವರೆಲ್ಲ ಹೊರಟುಹೋದಮೇಲೆ ಮಹಾರಾಜರು ಸ್ವಾಮೀಜಿ ಅವರನ್ನು ತಮ್ಮ ಪ್ರತ್ಯೇಕ ಕೋಣೆಗೆ ಕರೆದು “ಸ್ವಾಮೀಜಿ, ಅತಿಯಾಗಿ ಮುಚ್ಚು ಮರೆಯಿಲ್ಲದೆ ವರ್ತಿಸುವುದು ಯಾವಾಗಲೂ ಹಿತಕಾರಿಯಲ್ಲ. ನೀವು ನನ್ನ ಎದುರಿಗೇ ನನ್ನ ಆಸ್ಥಾನದವರ ವಿಷಯವಾಗಿ ಕಟುವಾಗಿ ಮಾತನಾಡಿದರೆ ಬಹುಶಃ ಅವರು ನಿಮಗೆ ವಿಷವನ್ನು ಕೊಡಬಹುದು” ಎಂದರು. ಅದಕ್ಕೆ ಸ್ವಾಮೀಜಿ, “ನಿಜವಾದ ಸಂನ್ಯಾಸಿ ತನ್ನ ಪ್ರಾಣಕ್ಕೆ ಸಂಚಕಾರ ಬಂದೀತೆಂದು ಸುಳ್ಳನ್ನು ಹೇಳುವನು ಎಂದು ಭಾವಿಸಿದಿರಾ? ನಾಳೆ ನಿಮ್ಮ ಮಗನೇ ಬಂದು ‘ಸ್ವಾಮೀಜಿ ನನ್ನ ತಂದೆಯವರು ಹೇಗೆ?’ ಎಂದು ಕೇಳಿದರೆ ನಿಮ್ಮಲ್ಲಿ ಇಲ್ಲದ ಸಲ್ಲದ ಒಳ್ಳೆಯ\break ಗುಣಗಳನ್ನೆಲ್ಲ ಆರೋಪಮಾಡಿ ಹೇಳಲೆ? ನಾನು ಸುಳ್ಳನ್ನು ಹೇಳಲೆ? ಇಲ್ಲ, ಎಂದಿಗೂ ಇಲ್ಲ” ಎಂದರು. 

 ಸ್ವಾಮೀಜಿ ಅವರು ಆಸ್ಥಾನದಲ್ಲಿದ್ದಾಗ ಆಸ್ಟ್ರಿಯಾದ ಒಬ್ಬ ಸಂಗೀತಗಾರನನ್ನು ಕಂಡರು. ಆತನೊಡನೆ ಪಾಶ್ಚಾತ್ಯ ಸಂಗೀತದ ವಿಷಯವನ್ನು ಮಾತನಾಡಿದರು. ಆಗ ತಾನೆ ಅರಮನೆಗೆ ವಿದ್ಯುಚ್ಛಕ್ತಿಯನ್ನು ಹಾಕುತ್ತಿದ್ದರು. ಅದನ್ನು ಮಾಡುತ್ತಿದ್ದ ವಿದ್ಯುತ್‍ಶಕ್ತಿಯ ಇಂಜಿನಿಯರನನ್ನು ಕಂಡು ಅವನೊಡನೆ ವಿದ್ಯುತ್‍ಶಕ್ತಿಯ ವಿಷಯವಾಗಿ ಮಾತನಾಡಿದರು. ಪಾಶ್ಚಾತ್ಯ ಸಂಗೀತವಾಗಲಿ, ವಿದ್ಯುತ್‍ಶಕ್ತಿಯಾಗಲಿ ಯಾವುದೂ ಸ್ವಾಮೀಜಿಗೆ ಹೊಸ ವಿಷಯಗಳಾಗಿರಲಿಲ್ಲ. ಅವುಗಳನ್ನೆಲ್ಲ ಅವರು ಅಷ್ಟು ಚೆನ್ನಾಗಿ ತಿಳಿದುಕೊಂಡಿದ್ದರು. 

 ಒಂದು ದಿನ ದಿವಾನರು ಆಸ್ಥಾನದ ವಿದ್ವಾಂಸರನ್ನೆಲ್ಲ ಕರೆದು ಅವರಿಗೆ ವೇದಾಂತದ ಮೇಲೆ ಸಂಸ್ಕೃತದಲ್ಲಿ ಮಾತನಾಡುವಂತೆ ಹೇಳಿದರು. ದಿವಾನರು ಅಧ್ಯಕ್ಷರಾಗಿದ್ದರು. ಅವರ ಪಕ್ಕದಲ್ಲಿ ಸ್ವಾಮೀಜಿ ಕುಳಿತು ಕೇಳುತ್ತಿದ್ದರು. ಪ್ರತಿಯೊಬ್ಬ ವಿದ್ವಾಂಸರೂ ತಮ್ಮ ತಮ್ಮ ದೃಷ್ಟಿಕೋನದಿಂದ ವೇದಾಂತದ ಸಿದ್ಧಾಂತವನ್ನು ವಿವರಿಸಿದರು. ಕೊನೆಗೆ ದಿವಾನರು ಸ್ವಾಮೀಜಿಯವರಿಗೆ ಉಪನ್ಯಾಸ ಮಾಡಲು ಕೇಳಿಕೊಂಡರು. ಸ್ವಾಮೀಜಿ, ಭಿನ್ನ ತತ್ತ್ವಗಳನ್ನೆಲ್ಲ ತೆಗೆದುಕೊಂಡು ಯಾವುದನ್ನೂ ಅಲ್ಲಗಳೆಯದೆ ಅದರಲ್ಲಿ ಒಂದು ಸಮನ್ವಯ ಬರುವಂತೆ ಉದಾರ ದೃಷ್ಟಿಯಿಂದ ವಿವರಿಸಿದಾಗ ಅಲ್ಲಿ ಕುಳಿತವರೆಲ್ಲ ಮೆಚ್ಚಿ ಕರತಾಡನ ಮಾಡಿದರು. 

 ದಿವಾನರು ಸ್ವಾಮೀಜಿಯವರಿಗೆ ಏನನ್ನಾದರೂ ಬಹುಮಾನವನ್ನು ಕೊಡಬೇಕೆಂದು ಇಚ್ಛಿಸಿದರು. ತಮ್ಮ ಆಪ್ತ ಕಾರ್ಯದರ್ಶಿಯ ಕೈಗೆ ಚೆಕ್ ಬುಕ್ ಕೊಟ್ಟು, ಸ್ವಾಮೀಜಿಯವರನ್ನು ದೊಡ್ಡ ಅಂಗಡಿಗೆ ಕರೆದುಕೊಂಡುಹೋಗಿ ಅವರು ಏನು ಬೇಕೊ ಅದನ್ನು ತೆಗೆದುಕೊಳ್ಳಲಿ, ದುಡ್ಡನ್ನು ನೀನು ಕೊಡು ಎಂದು ಹೇಳಿ ಕಳುಹಿಸಿದರು. ಆತನೊಡನೆ ಸ್ವಾಮೀಜಿ ಅಂಗಡಿಗೆ ಹೋಗಿ ಹಲವು ವಸ್ತುಗಳನ್ನು ನೋಡಿದರು, ಮೆಚ್ಚಿದರು. ಆದರೆ ಏನನ್ನೂ ತೆಗೆದುಕೊಳ್ಳಲಿಲ್ಲ. ದಿವಾನರಿಗೆ ತಾವು ಏನನ್ನಾದರೂ ತೆಗೆದುಕೊಳ್ಳದೇ ಇದ್ದರೆ ವ್ಯಥೆಯಾಗುವುದೆಂದು ಭಾವಿಸಿ ಕೆಲವು ಆಣೆಗಳ ಬೆಲೆಬಾಳುವ ಒಂದು ಚುಟ್ಟವನ್ನು ತೆಗೆದುಕೊಂಡರು. ಅಂಗಡಿಯಿಂದ ಹೊರಗೆ ಬಂದವರೇ ಅದನ್ನು ಹತ್ತಿಸಿ ಸೇದಿ ಬಿಸುಟರು. ಅಷ್ಟೇ ಅವರ ವ್ಯಾಪಾರ! 

 ಒಂದು ದಿನ ಮಹಾರಾಜರು ಸ್ವಾಮೀಜಿ ಅವರನ್ನು ಕರೆದು “ತಾವು ಯಾವರೀತಿ ಸ್ವಾಮೀಜಿಯವರಿಗೆ ಸೇವೆ ಮಾಡಲು ಸಾಧ್ಯ?” ಎಂದು ಕೇಳಿಕೊಂಡರು. ಸ್ವಾಮೀಜಿ ನೇರವಾಗಿ ಉತ್ತರ ಹೇಳದೆ ತಮ್ಮ ಉದ್ದೇಶವನ್ನು ಕುರಿತು ವಿವರಿಸಿದರು. ಇಂಡಿಯಾ ದೇಶದಲಿ ತತ್ತ್ವ ಮತ್ತು ಧರ್ಮ ಇವೆ. ಆದರೆ ಅದು ಪಾಶ್ಚಾತ್ಯ ದೇಶಗಳಿಂದ ವಿಜ್ಞಾನ ಮತ್ತು ಸಂಯೋಜನಾ ಕ್ರಮವನ್ನು ಕಲಿಯಬೇಕು. ಇಂಡಿಯಾ ತನ್ನಲ್ಲಿರುವುದನ್ನು ಪಾಶ್ಚಾತ್ಯರಿಗೆ ಕೊಡಬೇಕು. ಅಲ್ಲಿರುವುದನ್ನು ನಾವು ಕಲಿಯಬೇಕು. ಅದರಲ್ಲೆ ನಮ್ಮ ಉದ್ಧಾರ ನಿಂತಿರುವುದು ಎಂದರು. ಅದಕ್ಕಾಗಿಯೇ ನಮ್ಮ ವೇದಾಂತದ ತತ್ತ್ವಗಳನ್ನು ಸಾರಲು ತಾವು ಅಮೇರಿಕಾ ದೇಶಕ್ಕೆ ಹೋಗಲು ಮನಸ್ಸು ಮಾಡಿರುವೆ ಎಂದು ಹೇಳಿದರು. ಇದನ್ನು ಕೇಳಿದ ಮಹಾರಾಜರು, “ಅಲ್ಲಿಗೆ ಹೋಗಬೇಕಾದರೆ ತಗಲುವ ವೆಚ್ಚವನ್ನೆಲ್ಲ ನಾನು ಈಗಲೇ ಕೊಡುತ್ತೇನೆ” ಎಂದು ಹೇಳಿದರು. ಆದರೆ ಸ್ವಾಮೀಜಿ ತತ್‍ಕ್ಷಣವೇ ಅದನ್ನು ಸ್ವೀಕರಿಸದೆ, ಅದಕ್ಕೆ ಅವರಿಗೆ ಕೃತಜ್ಞತೆಗಳನ್ನು ಅರ್ಪಿಸಿದರು. 

 ಸ್ವಾಮೀಜಿ ಮೈಸೂರಿನಿಂದ ಹೋರಡುತ್ತೇನೆ ಎಂದು ಹೇಳಿದಾಗ ಮಹಾರಾಜರು ಖಿನ್ನರಾದರು. “ಇನ್ನೂ ಕೆಲವು ದಿನಗಳು ಇದ್ದು ಹೋಗಿ” ಎಂದರು. “ನಿಮ್ಮ ಜ್ಞಾಪಕಾರ್ಥವಾಗಿ ಒಂದು ಫೋನೋಗ್ರಾಫಿನ ರೆಕಾರ್ಡ್ ಕೊಡಿ” ಎಂದು ಕೇಳಿಕೊಂಡರು. ಅದಕ್ಕೆ ಸ್ವಾಮೀಜಿ ಒಪ್ಪಿಕೊಂಡು ಕೊಟ್ಟರು. ಈಗ ಅದು ಹಾಳಾಗಿ ಹೋಗಿದೆ. ಮಹಾರಾಜರು ಸ್ವಾಮಿಗಳ ಪಾದಪೂಜೆಯನ್ನು ಮಾಡಬೇಕೆಂದು ಬಯಸಿದರು. ಆದರೆ ಅದಕ್ಕೆ ಸ್ವಾಮೀಜಿ ಅವಕಾಶ ಕೊಡಲಿಲ್ಲ. ಕೆಲವು ದಿನಗಳಾದ ಮೇಲೆ ತಾವು ಹೋಗಲೇಬೇಕೆಂದು ಸ್ವಾಮೀಜಿ ವ್ಯಕ್ತಪಡಿಸಿದಾಗ ಮಹಾರಾಜರು ಅವರಿಗೆ ಬೇಕಾದಷ್ಟು ಬಹುಮಾನವನ್ನು ಹೊರಿಸಲು ಯತ್ನಿಸಿದರು. ಸ್ವಾಮೀಜಿ ಎಲ್ಲವನ್ನೂ ನಿರಾಕರಿಸಿದರು. ತಮಗೆ ಒಂದು ಮಣ್ಣಿನಿಂದ ಮಾಡಿದ ಹೊಗೆ ಕೊಳವೆ ಕೊಟ್ಟರೆ ಸಾಕು ಎಂದರು. ಅದರಂತೆಯೇ ಮಹಾರಾಜರು ಒಂದು ಹೊಗೆಸೊಪ್ಪನ್ನು ಸೇದುವ ಕೊಳವೆಯನ್ನು ಕೊಟ್ಟರು. ದಿವಾನರು ಒಂದು ಕಂತೆ ನೋಟನ್ನು ಸ್ವಾಮೀಜಿ ಕಿಸೆಗೆ ಇಡಲು ಯತ್ನಿಸಿದರು. ಆಗ ಸ್ವಾಮೀಜಿ ತಮಗೆ ಹಣ ಬೇಡವೆಂದೂ, ದಿವಾನರು ಬೇಕಾದರೆ ಕೊಚಿನ್ನಿಗೆ ಒಂದು ರೈಲ್ವೆ ಟಿಕೇಟನ್ನು ಮತ್ತು ಅಲ್ಲಿಯ ದಿವಾನರಿಗೆ ಒಂದು ಪರಿಚಯ ಪತ್ರವನ್ನು ಕೊಟ್ಟರೆ ಸಾಕು ಎಂದರು. ಅದರಂತೆಯೇ ದಿವಾನರು ಸ್ವಾಮೀಜಿಗೆ ಕೊಚಿನ್‍ಗೆ ಎರಡನೇ ತರಗತಿಯ ಟಿಕೇಟನ್ನು ಮತ್ತು ಪರಿಚಯ ಪತ್ರವನ್ನು ಕೊಟ್ಟು ಬೀಳ್ಕೊಟ್ಟರು. 

 ಸ್ವಾಮೀಜಿ ಎಲ್ಲೋ ಕೆಲವು ದಿನಗಳು ಮಾತ್ರ ಮೈಸೂರಿನಲ್ಲಿದ್ದರೂ ಮುಂದೆ ಅದು ತಮ್ಮ ಸಂಸ್ಥೆಯ ಒಂದು ಮುಖ್ಯ ಕೇಂದ್ರವಾಗುವುದೆಂದು ಅನಂತರ ಭಕ್ತರೊಬ್ಬರಿಗೆ ಬರೆದ ಕಾಗದದಲ್ಲಿ ಹೇಳಿರುವರು. ಅನೇಕ ವೇಳೆ ಸ್ವಾಮೀಜಿ ರಾಜ, ಮಹಾರಾಜ, ದಿವಾನರು ಮುಂತಾದವರ ವಲಯದೊಳಗೆ ಸಂಚಾರ ಮಾಡುವುದನ್ನು ನೋಡಿದಾಗ, ಕೆಳಗಿನವರನ್ನು ಬಿಟ್ಟು ದೊಡ್ಡ ದೊಡ್ಡ ಸ್ಥಳಗಳಲ್ಲಿ ಇವರು ವ್ಯವಹರಿಸುತ್ತಿರುವರು ಎಂಬ ಭಾವ ಬರುವುದು. ಇದೇ ಪ್ರಶ್ನೆಯನ್ನು ಸ್ವಾಮೀಜಿಗೆ ಹಾಕಿದಾಗ ಅವರೆಂದರು: ದಿವಾನರು ಮಹಾರಾಜರು ಮುಂತಾದವರಿಗೆ ಮಾತ್ರ ತಮ್ಮ ದೇಶದಲ್ಲಿ ಸುಧಾರಣೆಯನ್ನುಂಟುಮಾಡಲು, ವಿದ್ಯಾಭ್ಯಾಸ, ವಿಜ್ಞಾನ, ಯಾಂತ್ರಿಕ ನಾಗರಿಕತೆಯನ್ನು ಪ್ರಚಾರಮಾಡಲು ಮತ್ತು ಕಾರ್ಯಗತ ಮಾಡಲು ಅಧಿಕಾರವಿದೆ ಮತ್ತು ಶಕ್ತಿ ಇದೆ. ಅಂತಹ ಒಬ್ಬನಿಂದ ಸಾವಿರಾರು ಜನರಿಗೆ ಉಪಯೋಗವಾಗುತ್ತದೆ. ಆದಕಾರಣವೇ ತಾವು ಅಂತಹವರ ಪರಿಚಯವನ್ನು ಆಶಿಸುವುದು ಎಂದರು. ಆದರೆ ಅವರು ಅಲ್ಲೇ ಇರುತ್ತಿರಲಿಲ್ಲ. ಬಡವರ ಗುಡಿಸಲು, ಅವರು ಕೊಡುವ ಸರಳವಾದ ಆಹಾರ, ರಾಜರ ಅರಮನೆ ಮತ್ತು ಮೃಷ್ಟಾನ್ನ ಭೋಜನದಷ್ಟೇ ಅವರಿಗೆ ಪ್ರಿಯವಾಗಿದ್ದುವು. ಸ್ವಾಮೀಜಿ ಕೇವಲ ತಮ್ಮ ವ್ಯಕ್ತಿ ದೃಷ್ಟಿಯಿಂದ ನೋಡುತ್ತಿರಲಿಲ್ಲ. ತಾವು ದೇಶಕ್ಕೆ ಏನನ್ನಾದರೂ ಸೇವೆ ಮಾಡಬೇಕು ಎಂಬುದೇ ಅವರ ಧ್ಯೇಯವಾಗಿತ್ತು. ಹಾಗೆ ಮಾಡಬೇಕಾದರೆ, ಮುಖ್ಯವಾದ ಗಣ್ಯ ವ್ಯಕ್ತಿಗಳ ಮೂಲಕ ಮಾತ್ರ ಸಾಧ್ಯ. ಸ್ವಾಮೀಜಿ ಅನಂತರ ಚಾಮರಾಜ ಒಡೆಯರಿಗೆ ನಮ್ಮ ದೇಶದ ಲೋಪದೋಷಗಳು, ಅದನ್ನು ನಿರ್ಮೂಲ ಮಾಡುವ ಬಗೆ ಹೇಗೆ, ಮತ್ತು ಪಾಶ್ಚಾತ್ಯ ದೇಶದಿಂದ ನಾವು ಏನನ್ನು ಕಲಿತುಕೊಳ್ಳಬೇಕು - ಎಂಬುದನ್ನು ವಿವರಿಸಿ ಅಮೆರಿಕದಿಂದ ಒಂದು ಪತ್ರವನ್ನು ಬರೆದಿರುವರು. ಭಗವಂತನ ಮಕ್ಕಳಾದ ಪ್ರಜೆಗಳಿಗಾಗಿ ಒಬ್ಬ ಬಾಳಬೇಕು, ಅವರಿಗಾಗಿ ಮರುಗಬೇಕು, ಅವರಿಗಾಗಿ ದುಡಿಯಬೇಕೆಂದು ಪ್ರೋತ್ಸಾಹಿಸುವ ಪತ್ರವನ್ನು ಕೆಳಗೆ ಉಲ್ಲೇಖಿಸುವೆವು: 

 “ಮಹಾರಾಜರೆ, 

 ಶ‍್ರೀಮನ್ನಾರಾಯಣನು ತಮಗೂ ತಮ್ಮ ಕುಟುಂಬಕ್ಕೂ ಕಲ್ಯಾಣವನ್ನುಂಟು\break ಮಾಡಲಿ. ನೀವು ಪ್ರೀತಿಯಿಂದ ಮಾಡಿದ ಸಹಾಯದಿಂದ ನಾನು ಈ ದೂರದೇಶಕ್ಕೆ ಬರಲು ಸಾಧ್ಯವಾಯಿತು. ಅಂದಿನಿಂದ ಇಲ್ಲಿ ಪ್ರಖ್ಯಾತನಾಗಿರುವೆನು. ಅತಿಥಿ ಸತ್ಕಾರಪರರು ಇಲ್ಲಿಯ ಜನರು. ನನ್ನ ಕೊರತೆಗಳೆಲ್ಲವನ್ನೂ ಬಗೆಹರಿಸಿರುವರು. ಇದು ಹಲವು ವಿಧದಲ್ಲಿ ಒಂದು ಆಶ್ಚರ‍್ಯಕರವಾದ ದೇಶ. ಇಲ್ಲ್ಲಿಯ ಜನರೊಂದು ಅದ್ಭುತ ಜನಾಂಗ. ಇಲ್ಲಿಯ ಜನರು ನಿತ್ಯ ಜೀವನದಲ್ಲಿ ಉಪಯೋಗಿಸುವಷ್ಟು ಯಂತ್ರವನ್ನು ಜಗತ್ತಿನಲ್ಲಿ ಇನ್ನಾವ ಜನಾಂಗವೂ ಉಪಯೋಗಿಸುವುದಿಲ್ಲ. ಎಲ್ಲಾ ಬಗೆಯ ಯಂತ್ರಗಳಿವೆ. ಇವರು ಜನಸಂಖ್ಯೆಯಲ್ಲಿ ಜಗತ್ತಿನಲ್ಲಿ ಎಪ್ಪತ್ತನೇ ಒಂದು ಪಾಲು. ಆದರೆ ಪ್ರಪಂಚದ ಐಶ್ವರ‍್ಯದಲ್ಲಿ ಆರನೇ ಒಂದು ಭಾಗ ಇವರಲ್ಲಿದೆ. ಇಲ್ಲಿ ಐಶ್ವರ‍್ಯಕ್ಕೆ, ಭೋಗವಿಲಾಸಗಳಿಗೆ ಒಂದು ಮೇರೆಯೇ ಇಲ್ಲ. ಆದರೂ ಇಲ್ಲಿ ಎಲ್ಲಕ್ಕೂ ಬಹಳ ಬೆಲೆ. ಇಲ್ಲಿ ಕೂಲಿಗಳ ದರ ಜಗತ್ತಿನ ಎಲ್ಲಾ ಕಡೆಗಿಂತಲೂ ಅಧಿಕ. ಆದರೂ ಬಂಡವಾಳಗಾರರಿಗೂ ಕೂಲಿಗಳಿಗೂ ನಿತ್ಯ ವ್ಯಾಜ್ಯ ತಪ್ಪದು. ಅಮೇರಿಕಾದ ಸ್ತ್ರೀಯರಿಗಿರುವಷ್ಟು ಅಧಿಕಾರ ಜಗತ್ತಿನ ಮತ್ತಾವ ಸ್ತ್ರೀಯರಿಗೂ ಇಲ್ಲ. ಅವರು ಪ್ರತಿಯೊಂದನ್ನೂ ಮೆಲ್ಲಗೆ ತಮ್ಮ ವಶಮಾಡಿಕೊಳ್ಳುತ್ತಿರುವರು. ಸುಸಂಸ್ಕೃತರಾದ ಹೆಂಗಸರು ಗಂಡಸರಿಗಿಂತ ಹೆಚ್ಚು ಎಂದು ಹೇಳಲು ಆಶ್ಚರ್ಯವಾಗುತ್ತದೆ. ಆದರೆ ಉಚ್ಚ ಪ್ರತಿಭಾಶಾಲಿಗಳು ಬಹುಮಟ್ಟಿಗೆ ಪುರುಷರೆಂಬುದೇನೋ ನಿಜ. ಪಾಶ್ಚಾತ್ಯರು ನಮ್ಮನ್ನು ಜಾತಿಯ ವಿಷಯವಾಗಿ ಎಷ್ಟು ದೂರಿದರೂ, ಅವರಲ್ಲಿ ಇದಕ್ಕಿಂತ ನೀಚವಾದ ಜಾತಿಭೇದವಿದೆ. ಅದೇ ಹಣದ ಜಾತಿ. ಅಮೇರಿಕಾದವರು ಹೇಳುವಂತೆ ಸರ್ವಶಕ್ತಿ ಸ್ವರೂಪವಾದ ಡಾಲರ್ ಮಾಡದ ಕಾರ‍್ಯವಿಲ್ಲ.

 “ಈ ದೇಶದಲ್ಲಿರುವಷ್ಟು ಕಾನೂನುಗಳು ಮತ್ತೆಲ್ಲಿಯೂ ಇಲ್ಲ. ಆದರೆ ಅದಕ್ಕೆ ಇಲ್ಲಿರುವಷ್ಟು ಅಸಡ್ಡೆ ಮತ್ತೆಲ್ಲಿಯೂ ಇಲ್ಲ. ಒಟ್ಟಿನಲ್ಲಿ ನಮ್ಮ ದೀನ ಹಿಂದೂಗಳು ಪಾಶ್ಚಾತ್ಯರಿಗಿಂತ ಎಷ್ಟೋ ಪಾಲು ಹೆಚ್ಚು ನೀತಿವಂತರು. ಧರ್ಮದಲ್ಲಿ ಮಿಥ್ಯಾಚಾರ ಇಲ್ಲವೆ ಅಂಧ ಅಭಿಮಾನ ಇವೆರಡನ್ನೇ ಇವರು ಅಭ್ಯಾಸ ಮಾಡುವುದು. ಇಲ್ಲಿನ ಆಲೋಚನಾಪರರಿಗೆ ಮೂಢನಂಬಿಕೆಗಳಿಂದ ಕೂಡಿದ ಅವರ ಧರ್ಮದಲ್ಲಿ ಬೇಜಾರು ಹುಟ್ಟಿ ಹೊಸ ಬೆಳಕು ಭಾರತದಿಂದ ಬರುವುದೆಂದು ಕಾಯುತ್ತಿರುವರು. ಆಧುನಿಕ ವೈಜ್ಞಾನಿಕ ಶಾಸ್ತ್ರಗಳನ್ನು ಮತ್ತು ಸಿದ್ಧಾಂತಗಳನ್ನು ಪವಿತ್ರ ವೇದದಲ್ಲಿರುವ ಮಹದಾಲೋಚನೆಗಳು ಸಮರ್ಥವಾಗಿ ಎದುರಿಸುತ್ತವೆ, ಅವುಗಳ ಆಘಾತದಿಂದ ನಾಶವಾಗದೆ ಇವೆ. ಇಂತಹ ಎಷ್ಟು ಸ್ವಲ್ಪ ವಿಷಯಗಳನ್ನಾದರೂ ಕೂಡ ಅಮೇರಿಕಾ ದೇಶ ಉತ್ಸಾಹದಿಂದ ಸ್ವೀಕರಿಸುವುದು. ನೀವು ಇದನ್ನು ನೋಡಿದಲ್ಲದೆ ನಂಬಲಾಗುವುದಿಲ್ಲ. ಶೂನ್ಯದಿಂದ ಸೃಷ್ಟಿ, ಹೊಸದಾಗಿ ಸೃಷ್ಟಿಸಿದ ಜೀವ, ಸ್ವರ್ಗವೆನ್ನುವಲ್ಲಿ ಸಿಂಹಾಸನದ ಮೇಲೆ ಕುಳಿತು ಆಳುತ್ತಿರುವ ಕ್ರೂರನೂ ಅತ್ಯಾಚಾರಿಯೂ ಆದ ಈಶ್ವರ, ಕೆಳಗೆ ನಿತ್ಯ ನರಕಯಾತನೆ, ಇವು ವಿದ್ಯಾವಂತರನ್ನೆಲ್ಲ ಬೇಸರಿಸಿವೆ. ಸೃಷ್ಟಿಯ ಅನಾದಿತ್ವ ಅದರಂತೆ ಆತ್ಮದ ನಿತ್ಯತ್ವ, ಮತ್ತು ನಮ್ಮ ಆತ್ಮದಲ್ಲೇ ಅವಸ್ಥಿತವಾದ ಪರಮಾತ್ಮ ಇವುಗಳನ್ನು ಒಂದಲ್ಲ ಒಂದು ವಿಧದಲ್ಲಿ ಕಲಿಯುತ್ತಿರುವರು. ಇನ್ನು ಐವತ್ತು ವರ್ಷಗಳಲ್ಲಿ ಪ್ರಪಂಚದ ವಿದ್ಯಾವಂತರುಗಳೆಲ್ಲರೂ ವೇದದಲ್ಲಿ ಹೇಳಿರುವ ಸೃಷ್ಟಿ ಮತ್ತು ಆತ್ಮದ ಅನಾದಿತ್ವವನ್ನು, ದೇವತ್ವವೇ ಮಾನವನ ಚರಮ ಗುರಿ ಮತ್ತು ಪೂರ್ಣಾವಸ್ಥೆ ಎಂಬ ವಿಷಯವನ್ನು ನಂಬುವರು. ಈಗಲೂ ಕೂಡ ವಿದ್ಯಾವಂತರಾದ ಪಾದ್ರಿಗಳು ಬೈಬಲ್ಲನ್ನು ಆ ದೃಷ್ಟಿಯಿಂದ ವಿವರಿಸುತ್ತಿರುವರು. ಅವರಿಗೆ ಹೆಚ್ಚು ಆಧ್ಯಾತ್ಮಿಕ ವಿದ್ಯೆ ಬೇಕು. ನಮಗೆ ಹೆಚ್ಚು ಐಹಿಕ ಉನ್ನತಿಯ ಶಿಕ್ಷಣ ಬೇಕು ಎಂಬುದು ನನ್ನ ನಿರ್ಧಾರ.

 “ಭಾರತವರ್ಷದಲ್ಲಿರುವ ದೌರ್ಭಾಗ್ಯಕ್ಕೆಲ್ಲ ಮೂಲ ಅಲ್ಲಿರುವ ಬಡಜನರ ಸ್ಥಿತಿ. ಪಾಶ್ಚಾತ್ಯ ದೇಶಗಳಲ್ಲಿ ಬಡವರು ಪಿಶಾಚಿಗಳಂತೆ. ಅವರೊಂದಿಗೆ ಹೋಲಿಸಿದರೆ ನಮ್ಮ ಬಡವರು ದೇವತೆಗಳು. ಆದುದರಿಂದಲೇ ಅವರನ್ನು ಉದ್ಧಾರ ಮಾಡುವುದು ಸುಲಭ. ಹೀನಸ್ಥಿತಿಯಲ್ಲಿರುವ ನಮ್ಮ ಬಡವರಿಗೆ ನಾವು ಮಾಡಬೇಕಾದ ಮುಖ್ಯ ಕರ್ತವ್ಯವೆಂದರೆ ಅವರಿಗೆ ವಿದ್ಯೆಯನ್ನು ಕೊಟ್ಟು ಅವರನ್ನು ಪುನಃ ಮಾನವರನ್ನಾಗಿ ಮಾಡುವುದು. ನಮ್ಮ ಹಿಂದೂ ರಾಜರಿಗೂ, ಅವರ ಪ್ರಜೆಗಳಿಗೂ ಇರುವ ಪರಸ್ಪರ ಕೆಲಸವೇ ಇದು. ಆದರೆ ಇದುವರೆವಿಗೂ ಆ ದಾರಿಯಲ್ಲಿ ಯಾವ ಪ್ರಯತ್ನವು ನಡೆದಿಲ್ಲ. ಪುರೋಹಿತರ ದರ್ಪ, ಪರದೇಶದ ಜನರ ಕೈಯಲ್ಲಿ ಪಡೆದ ಸೋಲು, ಅನೇಕ ಶತಮಾನದಿಂದ ಅವರನ್ನು ನೆಲಕ್ಕೆ ತುಳಿದಿದೆ. ಕೊನೆಗೆ ಭರತಖಂಡದ ಬಡವರು ನಾವು ಮಾನವರೆಂಬುದನ್ನು ಮರೆತಿರುವರು. ಅವರನ್ನು ಆಲೋಚನಾಪರರನ್ನಾಗಿ ಮಾಡಬೇಕು. ಪ್ರಪಂಚದಲ್ಲಿ ಅವರ ಸುತ್ತಲೂ ಏನು ಆಗುತ್ತಿದೆ ಎಂಬುದನ್ನು ನೋಡಲು ಕಣ್ಣು ತೆರೆಯುವಂತೆ ಮಾಡಬೇಕು. ಅನಂತರ ಅವರ ಉದ್ಧಾರವನ್ನು ಅವರೇ ಮಾಡಿಕೊಳ್ಳುವರು. ಪ್ರತಿಯೊಂದು ಜನಾಂಗವೂ ಪ್ರತಿಯೊಬ್ಬ ಪುರುಷನೂ, ಪ್ರತಿಯೊಬ್ಬ ಸ್ತ್ರೀಯೂ, ತಮ್ಮ ವಿಮೋಚನೆಗೆ ತಾವೇ ಪ್ರಯತ್ನಪಡಬೇಕು. ಆಲೋಚನಾ ಶಕ್ತಿಯನ್ನು ಅವರಿಗೆ ಕೊಡಿ. ಅವರಿಗೆ ಬೇಕಾಗಿರುವ ಸಹಾಯ ಅದೊಂದೇ. ಅನಂತರ ಇವುಗಳ ಪರಿಣಾಮವಾಗಿ ಉಳಿದುದೆಲ್ಲ ಬಂದೆ ಬರುವುದು. ಸಾಮಗ್ರಿಗಳನ್ನು ಒದಗಿಸಿಕೊಡುವುದು ನಮ್ಮ ಕೆಲಸ.\break ಉಳಿದುದು ಪ್ರಕೃತಿ ನಿಯಮದ ಕೆಲಸ. ಅವರ ತಲೆಯಲ್ಲಿ ಆಲೋಚನೆಯನ್ನು ತುಂಬುವುದು ನಮ್ಮ ಕೆಲಸ. ಉಳಿದುದು ಪ್ರಕೃತಿ ನಿಯಮದ ಕೆಲಸ. ಮಿಕ್ಕಕೆಲಸವನ್ನು ಅವರೆ ಮಾಡುವರು. ಭರತಖಂಡದಲ್ಲಿ ಮಾಡಬೇಕಾದ ಕೆಲಸವಿದು. ಈ ಒಂದು ಆಲೋಚನೆಯೇ ನನ್ನ ಮನಸ್ಸಿನಲ್ಲಿ ಅನೇಕ ಕಾಲದಿಂದ ಇರುವುದು. ಭರತಖಂಡದಲ್ಲಿ ನಾನು ಇದನ್ನು ನೆರವೇರಿಸಲು ಆಗಲಿಲ್ಲ. ಅದಕ್ಕೋಸುಗವೇ ನಾನು ಇಲ್ಲಿಗೆ ಬಂದುದು. ಬಡವರಿಗೆ ಕಲಿಸುವ ವಿದ್ಯೆಯಲ್ಲಿ ಒಂದು ತೊಂದರೆ ಇದೆ. ನೀವು ಪ್ರತಿಯೊಂದು ಹಳ್ಳಿಯಲ್ಲಿಯೂ ಒಂದು ಪಾಠಶಾಲೆಯನ್ನು ತೆರೆಯಬಹುದು, ಆದರೆ ಇದರಿಂದ ಅಷ್ಟು ಅನುಕೂಲವಾಗುವುದಿಲ್ಲ. ಇಂಡಿಯಾದಲ್ಲಿ ಬಡತನ ಬಹಳ ಹೆಚ್ಚು. ಬಡಹುಡುಗರು ತಮ್ಮ ತಂದೆಗೆ ಸಹಾಯಕ್ಕಾಗಿ ಹೊಲಕ್ಕೆ ಹೋಗುವರು ಅಥವಾ ಜೀವನೋಪಾಯಕ್ಕೆ ಮತ್ತಾವುದಾದರೂ ಕಸುಬನ್ನು ಕಲಿಯುವರೇ ಹೊರತು ಶಾಲೆಗೆ ಬರುವುದಕ್ಕೆ ಆಗುವುದಿಲ್ಲ. ಬೆಟ್ಟ ಮಹಮ್ಮದನ ಸಮೀಪಕ್ಕೆ ಹೋಗದೆ ಇದ್ದರೆ ಮಹಮ್ಮದನು ಬೆಟ್ಟದ ಸಮೀಪಕ್ಕೆ ಬರಬೇಕು. ಬಡಹುಡುಗನು ವಿದ್ಯಾವ್ಯಾಸಂಗಕ್ಕೆ ಬರದೆ ಇದ್ದರೆ ವಿದ್ಯೆ ಅವನ ಬಳಿಗೆ ಬರಬೇಕು. ನಮ್ಮ ದೇಶದಲ್ಲಿ ಸ್ವಾರ್ಥತ್ಯಾಗ ಮಾಡಿದ ಹಲವು ಸಂನ್ಯಾಸಿಗಳು ಹಳ್ಳಿಯಿಂದ ಹಳ್ಳಿಗೆ ಧರ್ಮವನ್ನು ಬೋಧನೆ ಮಾಡುತ್ತ ಹೋಗುತ್ತಿರುವರು. ಅವರಲ್ಲಿ ಕೆಲವರನ್ನು ಸೇರಿಸಿ ಧರ್ಮಬೋಧನೆಯೊಡನೆ ಲೌಕಿಕ ವಿಚಾರಗಳನ್ನು ಜನಗಳಿಗೆ ತಿಳಿಸುವಂತೆ ಮಾಡಿದರೆ, ಆ ಸಂನ್ಯಾಸಿಗಳು ಪ್ರತಿಯೊಂದು ಮನೆಗೂ ಹೋಗಿ ವಿದ್ಯೆಯನ್ನು ಕಲಿಸಬಹುದು. ಇದರಲ್ಲಿ ಒಬ್ಬರು ಸಂಜೆ ಒಂದು ಹಳ್ಳಿಗೆ ಮಾಯಾದೀಪ, ಭೂಪಟ, ಕೃತಕಗೋಳ ಇವುಗಳೊಂದಿಗೆ ಹೋದರೆ ಗ್ರಾಮಸ್ಥರಿಗೆ ಖಗೋಳಶಾಸ್ತ್ರಗಳನ್ನು ಕಲಿಸಬಹುದು. ಬೇರೆ ಬೇರೆ ದೇಶಗಳ ಕಥೆಯನ್ನು ಅವರಿಗೆ ಕಲಿಸುವುದರಿಂದ ಬಡವರಿಗೆ ತಮ್ಮ ಜೀವಮಾನದಲ್ಲೆಲ್ಲ ಓದಿ ಕಲಿಯುವುದಕ್ಕಿಂತ ನೂರು ಪಾಲು ಹೆಚ್ಚು ಕೇಳುವುದರಿಂದ ಒದಗಿಸಬಹುದು. ಇದಕ್ಕೆ ಒಂದು ಸಂಸ್ಥೆ ಬೇಕು. ದುಡ್ಡಿಲ್ಲದೆ ಒಂದು ಸಂಸ್ಥೆಯಾಗದು. ಒಂದು ಗಾಲಿಯನ್ನು ಚಲಿಸುವಂತೆ ಮಾಡುವುದು ಬಹಳ ಕಷ್ಟ. ಆದರೆ ಎಂದು ಅದನ್ನು ಚಲಿಸುವಂತೆ ಮಾಡುವೆವೋ ಅಂದಿನಿಂದ ಬರಬರುತ್ತ ಹೆಚ್ಚು ವೇಗದಲ್ಲಿ ಚಲಿಸುವುದು. ನಮ್ಮ ದೇಶದಲ್ಲಿ ಇದಕ್ಕೆ ಸಹಾಯವನ್ನು ಹುಡುಕಿ ಬೇಸತ್ತು ಅಲ್ಲಿಯ ಯಾವ ಶ‍್ರೀಮಂತರಿಂದಲೂ ಸಹಾನುಭೂತಿ ಸಿಕ್ಕದೆ ಮಹಾರಾಜರ ಸಹಾಯದಿಂದ ಈ ದೇಶಕ್ಕೆ ಬಂದೆನು. ಅಮೇರಿಕಾ ದೇಶೀಯರಿಗೆ ಇಂಡಿಯಾ ದೇಶದ ಬಡಜನರು ಸತ್ತರೇನು? ಬದುಕಿದರೆ ಏನು? ನಮ್ಮ ಜನರೇ ತಮ್ಮ ಸ್ವಾರ್ಥವನ್ನಲ್ಲದೆ ಬೇರೆ ಏನನ್ನು ಆಲೋಚಿಸದೆ ಇರುವಾಗ ಅನ್ಯರಿಗೆ ಇವರ ಪಾಡೇಕೆ?

 “ಎಲೈ, ಮಹಾನುಭಾವನಾದ ದೊರೆಯೆ, ಕೆಲವು ದಿನ ಬಾಳುವುದು ಈ ಜನ್ಮ. ಜಗತ್ತಿನ ಸುಖಭೋಗಗಳೆಲ್ಲ ಕ್ಷಣಿಕ. ಯಾರು ಪರರ ಹಿತಕ್ಕಾಗಿ ಬದುಕುವರೋ ಅವರೇ ಜೀವಂತರು. ಉಳಿದವರು ಜೀವಚ್ಛವಗಳು. ತಮ್ಮಂತಹ ಉದಾರ ಮನಸ್ಸಿನ ಭರತಖಂಡದ ರಾಜರೊಬ್ಬರು, ಅಧೋಗತಿಗಿಳಿದ ಭರತಖಂಡವನ್ನು ಪುನಃ ಮೇಲೆತ್ತುವುದಕ್ಕೆ ಎಷ್ಟೋ ಸಹಾಯ ಮಾಡಬಹುದು. ಮುಂದಿನ ಜನಾಂಗ ದೀರ್ಘಕಾಲ ತಮ್ಮನ್ನು ಸ್ಮರಿಸಿ ಕೊಂಡಾಡುವಂತಹ ಕೀರ್ತಿಯನ್ನು ಗಳಿಸಬಹುದು. ಅಜ್ಞಾನಾಂಧಕಾರದಲ್ಲಿ ಮುಳುಗಿ ನರಳುತ್ತಿರುವ ಕೋಟ್ಯನುಕೋಟಿ ಭಾರತೀಯರಿಗಾಗಿ ಭಗವಂತನು ನಿಮ್ಮ ಹೃದಯ ಮರುಗುವಂತೆ ಮಾಡಲಿ ಎಂಬುದೇ ನನ್ನ ಪ್ರಾರ್ಥನೆ.” 

