
\chapter{ಬೊಂಬಾಯಿ ಪ್ರಾಂತ್ಯದಲ್ಲಿ }

ಅಹಮದಾಬಾದಿನಲ್ಲಿ ಸ್ವಾಮೀಜಿ ಕೆಲವು ದಿನ ಛತ್ರ ಮುಂತಾದುವುಗಳಲ್ಲಿ ತಂಗಿ ಭಿಕ್ಷೆಯಿಂದ ಜೀವಿಸುತ್ತಿದ್ದರು. ಕೊನೆಗೆ ಅವರಿಗೆ ಲಾಲ್‍ಶಂಕರ್ ಉಮಾಶಂಕರ್ ಎಂಬ ಅಹಮದಾಬಾದಿನ ಸಬ್‍ಜಡ್ಜರ ಪರಿಚಯವಾಯಿತು. ಅವರ ಮನೆಯಲ್ಲಿ ಸ್ವಾಮೀಜಿ ಅನಂತರ ವಾಸಿಸತೊಡಗಿದರು. ಅಹಮದಾಬಾದ್ ಗುಜರಾತಿನ ಸುಲ್ತಾನರ ರಾಜಧಾನಿಯಾಗಿತ್ತು. ಅದಕ್ಕೆ ಮುಂಚೆ ಹಿಂದೂ ಅರಸರ ರಾಜಧಾನಿ ಆಗಿತ್ತು. ಊರಿನಲ್ಲಿ ಹಲವು ಸುಂದರವಾದ ಹಳೆಯ ಅರಮನೆಗಳು, ಮಸೀದಿ ಮತ್ತು ಗೋರಿಗಳು ಇವೆ. ಸ್ವಾಮೀಜಿ ಇದನ್ನೆಲ್ಲ ನೋಡಿದರು. ಜೈನರ ಮುಖ್ಯಕೇಂದ್ರ ಈ ಊರು. ಹಲವು ಜೈನ ಪಂಡಿತರ ಪರಿಚಯ ಮಾಡಿಕೊಂಡು ಜೈನಧರ್ಮ ಮತ್ತು ತತ್ತ್ವಗಳ ಆಳಕ್ಕೆ ಹೋದರು.

ಸ್ವಾಮೀಜಿ ಅನಂತರ ವಾದ್‍ವಾನ್‍ಗೆ ಹೋದರು. ಅಲ್ಲಿಂದ ಲಿಂಬ್ಡಿಗೆ ಹೋದರು. ಲಿಂಬ್ಡಿ ನಗರಿಯಲ್ಲಿ ತಂಗುವುದಕ್ಕೆ ಸ್ಥಳವನ್ನು ಹುಡುಕುತ್ತಿದ್ದಾಗ ಊರಿನ ಹೊರಗೆ ಸಾಧುಗಳು ತಂಗುವ ಮಠವಿತ್ತು. ಹೋಗಿ ವಿಚಾರಿಸಲಾಗಿ ಆ ಸಾಧುಗಳು ಸಂತೋಷದಿಂದ ಸ್ವಾಮೀಜಿಗೆ ಇರಲು ಅವಕಾಶ ಮಾಡಿಕೊಟ್ಟರು. ಆಗ ಸ್ವಾಮೀಜಿಗೆ ನಡೆದು ನಡೆದು ಸಾಕಾಗಿತ್ತು. ಉಪವಾಸ ಬೇರೆ, ತಂಗುವುದಕ್ಕೆ ಸ್ಥಳ ದೊರಕಿದ ತಕ್ಷಣ ಹಾಯಾಗಿ ಸ್ವಲ್ಪ ವಿಶ್ರಮಿಸಿಕೊಂಡರು. ಕೆಲವು ದಿನಗಳು ಅಲ್ಲಿ ತಂಗಿದ ಮೇಲೆ ಅಲ್ಲಿ ವಾಸಿಸುತ್ತಿದ್ದ ಸಾಧುಗಳಲ್ಲಿ ಹೆಂಗಸರೂ ಇದ್ದುದು ಗೊತ್ತಾಯಿತು. ಅವರು ಅಧೋಗತಿಗೆ ಇಳಿದ ಒಂದು ಬಗೆಯ ವಾಮಾಚಾರಕ್ಕೆ ಸೇರಿದವರು. ಇದನ್ನು ತಿಳಿದೊಡನೆಯೆ ಸ್ವಾಮೀಜಿ ಅಲ್ಲಿಂದ ಹೊರಟುಹೋಗಲು ಯತ್ನಿಸಿದರು. ಆದರೆ ಸ್ವಾಮೀಜಿ ಇದ್ದ ಕೋಣೆಯ ಬಾಗಿಲನ್ನು ಹಾಕಿದ್ದರು. ಜೊತೆಗೆ ಅವರು ತಪ್ಪಿಸಿಕೊಂಡು ಹೋಗದಂತೆ ಒಬ್ಬ ಕಾವಲುಗಾರನನ್ನು ಬೇರೆ ಇಟ್ಟಿದ್ದರು. ಆ ಸಾಧುಗಳ ಮುಖ್ಯಸ್ಥ ಸ್ವಾಮೀಜಿಯವರನ್ನು ಕರೆಸಿ, ಸ್ವಾಮೀಜಿಗಳಿಗೆ “ನಿಮ್ಮನ್ನು ನೋಡಿದರೆ ನೀವು ಮಹಾತೇಜಸ್ವಿಗಳಾಗಿ ಕಾಣುವಿರಿ. ನೀವು ನೈಷ್ಠಿಕ ಬ್ರಹ್ಮಚಾರಿಗಳಾಗಿರಬೇಕು. ನಾವು ಒಂದು ವ್ರತವನ್ನು ಮಾಡಿ ನಿಮ್ಮ ಬ್ರಹ್ಮಚರ್ಯವನ್ನು ಆ ಸಮಯದಲ್ಲಿ ಆಹುತಿ ಕೊಡುತ್ತೇವೆ. ಅದರಿಂದ ನಮಗೆ ಕೆಲವು ಸಿದ್ಧಿಗಳು ಬರುತ್ತವೆ” ಎಂದು ಹೇಳಿದರು. ಸ್ವಾಮೀಜಿಯವರು ಕಳವಳವನ್ನು ವ್ಯಕ್ತಪಡಿಸಲಿಲ್ಲ. ಸುಮ್ಮನೆ ಇದ್ದರು. ಹೇಗೆ ಇವರ ಬಲೆಯಿಂದ ತಪ್ಪಿಸಿಕೊಂಡು ಹೋಗುವುದು ಎಂಬುದನ್ನು ಕುರಿತು ಯೋಚಿಸುತ್ತಿದ್ದರು. ಸ್ವಾಮೀಜಿ ಬಳಿಗೆ ಒಬ್ಬ ಹುಡುಗ ಬರುತ್ತಿದ್ದ. ಆತ ಸ್ವಾಮೀಜಿಯವರನ್ನು ಬಹಳ ಪ್ರೀತಿಸುತ್ತಿದ್ದ. ಸ್ವಾಮೀಜಿಯವರು ತಮ್ಮ ಪರಿಸ್ಥಿತಿಯನ್ನು ಕುರಿತು ಒಂದು ಪತ್ರವನ್ನು ಬರೆದು ಅದನ್ನು ಲಿಂಬ್ಡಿ ಮಹಾರಾಜರಿಗೆ ತಲುಪಿಸುವಂತೆ ಹುಡುಗನಿಗೆ ಹೇಳಿದರು. ಆ ಹುಡುಗ ಉಪಾಯದಿಂದ ಸ್ವಾಮೀಜಿಯವರ ಪತ್ರವನ್ನು ಮಹಾರಾಜರಿಗೆ ತಲುಪಿಸಿದ. ಮಹಾರಾಜರು ತಮ್ಮ ಸಿಪಾಯಿಗಳನ್ನು ಕಳುಹಿಸಿ ಸ್ವಾಮೀಜಿಯವರನ್ನು ಆ ಮಠದ ಸೆರೆಮನೆಯಿಂದ ಬಿಡಿಸಿ ಅರಮನೆಗೆ ಕರೆದುಕೊಂಡು ಹೋದರು. ಲಿಂಬ್ಡಿ ಮಹಾರಾಜರ ಕೋರಿಕೆಯಂತೆ ಅವರು ಅರಮನೆಯಲ್ಲಿಯೇ ತಂಗಿದರು. ಅಲ್ಲಿ ಹಲವು ಪಂಡಿತರೊಡನೆ ಸಂಸ್ಕೃತದಲ್ಲಿ ಚರ್ಚೆ ಮಾಡಿದರು. ಆ ಸಮಯದಲ್ಲಿ ಪೂರಿಗೋವರ್ಧನ ಮಠದ ಶಂಕಾರಾಚಾರ್ಯರು ಇದ್ದರು. ಅವರು ಸ್ವಾಮೀಜಿಯವರ ವಿದ್ವತ್ ಮತ್ತು ಔದಾರ‍್ಯತೆಯನ್ನು ಕಂಡು ವಿಸ್ಮಿತರಾದರು. 

 ಲಿಂಬ್ಡಿಯನ್ನು ಬಿಟ್ಟಾದಮೇಲೆ ಸ್ವಾಮೀಜಿ ಭಾವನಗರ ಮತ್ತು ಸೀಹೋರಿಗಳನ್ನು ನೋಡಿಕೊಂಡು ಜುನಗಡಕ್ಕೆ ಬಂದರು. ಅಲ್ಲಿ ದಿವಾನರಾದ ಬಾಬು ಹರಿದಾಸ್ ವಿಹಾರಿ ದಾಸರ ಅತಿಥಿಗಳಾಗಿದ್ದರು. ದಿವಾನರು ಪ್ರತಿದಿನ ಸಂಜೆ ತಮ್ಮ ಅಧಿಕಾರಿಗಳೊಡನೆ ಬಂದು ಸ್ವಾಮೀಜಿಯವರ ಪ್ರವಚನವನ್ನು ಕೇಳಿ ಸಂತೋಷಪಡುತ್ತಿದ್ದರು. ಜುನಗಡದ ಸಮೀಪದಲ್ಲೆಯೇ ಗಿರ‍್ನಾರ್ ಮುಂತಾದ ಯಾತ್ರಾಸ್ಥಳಗಳಿವೆ. ಬೌದ್ಧ, ಜೈನ ಮತ್ತು ಹಿಂದೂಧರ್ಮಕ್ಕೆ ಸಂಬಂಧಪಟ್ಟ ದೇವಸ್ಥಾನಗಳು ಮತ್ತು ಅವಶೇಷಗಳು ಎಲ್ಲವೂ ಅಲ್ಲಿವೆ. ಸ್ವಾಮೀಜಿಯವರು ಇವುಗಳನ್ನೆಲ್ಲ ನೋಡಿಕೊಂಡು ಗಿರ‍್ನಾರ್‍ಗೆ ಹೋದರು. ಅಲ್ಲಿಯ ಒಂದು ಗುಹೆಯಲ್ಲಿ ಕೆಲವು ದಿನಗಳು ತಪಸ್ಸು ಮಾಡಿ ಜುನಗಡಕ್ಕೆ ಹಿಂತಿರುಗಿದರು. ಅನಂತರ ಅಲ್ಲಿಂದ ಭೂಜ್‍ಗೆ ದಿವಾನರಿಂದ ಕೆಲವು ಪರಿಚಯ ಪತ್ರಗಳನ್ನು ತೆಗೆದುಕೊಂಡು ಹೋದರು. ಅಲ್ಲಿ ದಿವಾನರ ಅತಿಥಿಗಳಾಗಿದ್ದರು. ಅಲ್ಲಿಯ ದಿವಾನರು ಅವರ ಮಹಾರಾಜರಿಗೆ ಸ್ವಾಮೀಜಿಯವರನ್ನು ಪರಿಚಯ ಮಾಡಿಸಿದರು. ಅನಂತರ ಪುನಃ ಜುನಗಡಕ್ಕೆ ಬಂದರು. ಅಲ್ಲಿ ಕೆಲವು ದಿನಗಳು ವಿಶ್ರಾಂತಿಯನ್ನು ಪಡೆದುಕೊಂಡು ವೀರವೆಲ್, ಸೋಮನಾಥ ಮತ್ತು ಪ್ರಭಾಸಗಳಿಗೆ ಹೋದರು. ವೀರವೆಲ್ ಬಹಳ ಪುರಾತನ ನಗರಿ. ಸೋಮನಾಥದಲ್ಲಿ ಪ್ರಖ್ಯಾತವಾದ ದೇವಸ್ಥಾನವಿರುವುದು. ಸಮುದ್ರ ತೀರದಲ್ಲಿರುವ ಬೃಹದಾಕಾರದ ಸುಂದರವಾದ ಶಿವ ದೇವಾಲಯವನ್ನು ಹಿಂದೆ ಮೂರು ವೇಳೆ ನೆಲಸಮ ಮಾಡಿದ್ದರು. ಮೂರು ವೇಳೆಯೂ ಅದನ್ನು ಹಿಂದಿಗಿಂತ ಸುಂದರವಾಗಿ ಕಟ್ಟಿದ್ದರು. ಹಿಂದಿನ ಕಾಲದಲ್ಲಿ ಆ ದೇವಸ್ಥಾನದ ಸೇವೆಗೆಂದು ಹತ್ತುಸಾವಿರ ಹಳ್ಳಿಯ ಹುಟ್ಟುವಳಿಯನ್ನು ಬಿಟ್ಟಿದ್ದರಂತೆ. ಮುನ್ನೂರು ಜನ ಸಂಗೀತ ವಿದ್ವಾಂಸರು ದೇವಸ್ಥಾನದಲ್ಲಿ ಸೇವೆ ಸಲ್ಲಿಸುತ್ತಿದ್ದರು. ಅಲ್ಲಿಂದ ಸ್ವಾಮೀಜಿ ಪ್ರಭಾಸಕ್ಕೆ ಹೋದರು. ಇಲ್ಲೇ ಯಾದವರು ತಮ್ಮ ತಮ್ಮಲ್ಲಿ ಕಾದಾಡಿ ಎಲ್ಲರೂ ನಿರ್ನಾಮವಾಗಿದ್ದು. ಶ‍್ರೀಕೃಷ್ಣನೂ ಕೂಡ ತನಗೆ ಪ್ರಪಂಚವನ್ನು ಬಿಡಲು ಸಮಯ ಬಂದಿತೆಂದು ಒಂದು ಮರದ ಮೇಲೆ ಕುಳಿತುಕೊಂಡು ಕಾಲನ್ನು ಕೆಳಗೆ ಇಳಿಬಿಟ್ಟುಕೊಂಡು ಯೋಗಮಗ್ನ ನಾಗಿದ್ದಾಗ, ಕಾಡು ಮನುಷ್ಯನು ಅವನನ್ನು ಒಂದು ಜಿಂಕೆ ಎಂದು ಭಾವಿಸಿ ಒಂದು ಬಾಣವನ್ನು ಹೆದೆಗೆ ಏರಿಸಿ ಬಿಟ್ಟನು. ಶ‍್ರೀಕೃಷ್ಣ ಈ ಪ್ರಪಂಚವನ್ನು ಬಿಡುವುದಕ್ಕೆ ಇದು ಒಂದು ನಿಮಿತ್ತವಾಯಿತು. ಸ್ವಾಮೀಜಿ ಈ ಸ್ಥಳವನ್ನೆಲ್ಲ ನೋಡಿ ಹಿಂದಿನದನ್ನೆಲ್ಲ ಮನನ ಮಾಡತೊಡಗಿದರು. ಕುಚ್ ಬಿಹಾರಿನ ಮಹಾರಾಜರು ಇಲ್ಲಿಗೆ ಬಂದಿದ್ದರು. ಅವರು ಸ್ವಾಮೀಜಿಯವರೊಡನೆ ಮಾತನಾಡಿದ ಮೇಲೆ ಅವರ ವಿದ್ವತ್ತಿಗೆ ಮಾರು ಹೋಗಿ ಅವರು ಹೀಗೆ ಹೇಳಿದರು: “ನಾವು ಹಲವು ಗ್ರಂಥಗಳನ್ನು ಓದಿದ ಮೇಲೆ ತಲೆ ತಿರುಗುವುದಕ್ಕೆ ಪ್ರಾರಂಭವಾಗುವುದು. ಹಾಗೆಯೇ ನಿಮ್ಮ ಪ್ರವಚನವನ್ನು ಕೇಳಿ ಆದ ಮೇಲೆ ನೀವು ಇಷ್ಟೊಂದು ವಿದ್ಯೆಯನ್ನು ಹೇಗೆ ಉಪಯೋಗಿಸಿಕೊಳ್ಳುತ್ತೀರಿ ಎಂದು ಆಶ್ಚರ‍್ಯವಾಗುವುದು. ನೀವು ಏನನ್ನಾದರೂ ಮಹತ್ ಕೆಲಸವನ್ನು ಮಾಡದೆ ಸುಮ್ಮನಾಗುವುದಿಲ್ಲ.”

ಕೆಲವು ದಿನಗಳಾದ ಮೇಲೆ ಸ್ವಾಮೀಜಿ ಪುನಃ ಜುನಗಡಕ್ಕೆ ಬಂದರು. ಜುನಗಡದಿಂದ ಅಲ್ಲಿಯ ದಿವಾನರಿಂದ ಒಂದು ಪತ್ರವನ್ನು ಪೋರ್‍ಬಂದರಿನ ದಿವಾನರಿಗೆ ತೆಗೆದುಕೊಂಡು ಹೋದರು. ಪೋರ್‍ಬಂದರ್ ನಗರವನ್ನು ಸುಧಾಮಪುರಿ ಎಂದು ಕರೆಯುತ್ತಾರೆ. ಶ‍್ರೀಕೃಷ್ಣನ ದರಿದ್ರ ಸಖನಾದ ಸುಧಾಮನ ಹೆಸರಿನಲ್ಲಿರುವ ನಗರ. ಇಲ್ಲಿಯೇ ಸುಧಾಮನ ಒಂದು ದೇವಸ್ಥಾನವಿದೆ. ಸ್ವಾಮೀಜಿ ಅಲ್ಲಿಗೆ ಹೋಗಿಬಂದರು. ಪೋರ್‍ಬಂದರಿನ ದಿವಾನರಾದ ಪಂಡಿತ ಶಂಕರ್ ಪಾಂಡುರಂಗ ಅವರನ್ನು ಕಂಡರು. ರಾಜನಿಗೆ ಪ್ರಾಪ್ತವಯಸ್ಸಾಗಿಲ್ಲದೆ ಇದ್ದುದರಿಂದ ದಿವಾನರೇ ರಾಜರ ಪರವಾಗಿ ಅಧಿಕಾರವನ್ನು ವಹಿಸಿಕೊಂಡಿದ್ದರು. ದಿವಾನರು ದೊಡ್ಡ ವೈದಿಕ ವಿದ್ವಾಂಸರು. ಅವರು ಆಗ ವೇದಗಳನ್ನು ಭಾಷಾಂತರಮಾಡುತ್ತಿದ್ದರು. ಅವರು ಕೆಲವು ವೇದಗಳ ಭಾಗಕ್ಕೆ ಸ್ವಾಮೀಜಿಯ ವಿವರಣೆಯನ್ನು ಕೇಳಿದರು. ಸ್ವಾಮೀಜಿ ಕೊಟ್ಟ ವಿವರಣೆ ವಿಚಾರ ಪೂರಿತವಾಗಿಯೂ ಸರಳವಾಗಿಯೂ ಇತ್ತು. ಅವರು ತಮ್ಮ ಕೆಲಸಕ್ಕೆ ಸ್ವಾಮೀಜಿಯವರ ಸಹಾಯವನ್ನು ಕೋರಿಕೊಂಡರು. ಸ್ವಾಮೀಜಿಯವರು ಹನ್ನೊಂದು ತಿಂಗಳುಗಳು ಇಲ್ಲಿದ್ದರು. ವೇದಗಳಲ್ಲಿ ಬರುವ ಗಹನವಾದ ವಿಷಯಗಳನ್ನು ತಿಳಿದುಕೊಳ್ಳುವುದಕ್ಕೆ ಇದು ಸಹಾಯ ಮಾಡಿತು. ಆಗಲೆ ಪಾಣಿನಿಯ ವ್ಯಾಕರಣದ ಮೇಲೆ ಪತಂಜಲಿಯ ಮಹಾಭಾಷ್ಯವನ್ನು ಪೂರ್ಣವಾಗಿ ಓದಿದರು. ದಿವಾನರು ಸ್ವಾಮೀಜಿಗೆ ಫ್ರೆಂಚ್ ಭಾಷೆಯನ್ನು ಕಲಿಯಲು ಒತ್ತಾಯ ಮಾಡಿದರು. ಮುಂದೆ ಸ್ವಾಮೀಜಿಗೆ ಇದರಿಂದ ಪ್ರಯೋಜನವಾಗಬಹುದೆಂದು ಊಹಿಸಿದರು. ಸ್ವಾಮೀಜಿಯವರ ವಿದ್ವತ್ ಮತ್ತು ಅವರ ಉದಾರವಾದ ನಿರ್ಣಯಗಳು ಇವನ್ನು ನೋಡಿದ ದಿವಾನರು ಸ್ವಾಮೀಜಿಗೆ “ಸ್ವಾಮೀಜಿ, ಈ ದೇಶದಲ್ಲಿ ನೀವು ಹೆಚ್ಚು ಕೆಲಸಮಾಡಲು ಸಾಧ್ಯವಿಲ್ಲ. ನಿಮ್ಮನ್ನು ಮೆಚ್ಚುವಂತಹವರು ಈ ದೇಶದಲ್ಲಿ ಬಹಳ ಅಲ್ಪ. ನೀವು ಪಾಶ್ಚಾತ್ಯ ದೇಶಗಳಿಗೆ ಹೋದರೆ ಅವರಿಗೆ ನಿಮ್ಮ ಯೋಗ್ಯತೆ ಗೋತ್ತಾಗುತ್ತದೆ. ನೀವು ಅಲ್ಲಿಗೆ ಹೋಗಿ ಸನಾತನ ಧರ್ಮವನ್ನು ಬೋಧಿಸಿದರೆ ಅವರ ಸಂಸ್ಕೃತಿಯ ಮೇಲೆ ಬೆಳಕನ್ನು ಬೀರಿದಂತೆ ಆಗುವುದು” ಎಂದರು. ಸ್ವಾಮೀಜಿಯವರ ಮನಸ್ಸಿನಲ್ಲಿಯೂ ಇದೇ ಭಾವನೆ ಇತ್ತು. ಜುನಗಡದಲ್ಲಿ ಸಿ.ಎಚ್. ಪಾಂಡ್ಯರಿಗೆ ಸ್ವಾಮೀಜಿ ಪರದೇಶಗಳಿಗೆ ಹೋಗುವ ತಮ್ಮ ಅಭಿಪ್ರಾಯವನ್ನು ವ್ಯಕ್ತಪಡಿಸಿದ್ದರು. ಈ ಹೊತ್ತಿಗೆ ಚಿಕಾಗೊ ನಗರದಲ್ಲಿ ನಡೆಯುವ ವಿಶ್ವಧರ್ಮ ಸಮ್ಮೇಳನ ಸ್ವಾಮಿಗಳಿಗೆ ಯಾರ ಮೂಲಕವೊ ಗೊತ್ತಾಗಿ, ಅವರ ಮನಸ್ಸು ಕೆಲವು ವೇಳೆ ಅದನ್ನು ಕುರಿತು ಚಿಂತಿಸುತಿತ್ತು. ಆದರೆ ಯಾವ ಒಂದು ನಿರ್ದಿಷ್ಟ ನಿರ್ಣಯಕ್ಕೂ ಬಂದಿರಲಿಲ್ಲ.

ಸ್ವಾಮೀಜಿ ಪೋರ್‍ಬಂದರಿನಲ್ಲಿರುವಾಗ ಒಂದು ಘಟನೆ ನಡೆಯಿತು. ಸ್ವಾಮಿ ತ್ರಿಗುಣಾತೀತರು ಗುಜರಾತಿನಿಂದ ಪೋರ್‍ಬಂದರಿಗೆ ಬಂದು ಕೆಲವು ಸಾಧುಗಳೊಡನೆ ಇದ್ದರು. ಆ ಸಾಧುಗಳ ಜೊತೆ ಸಿಂಧ್ ದೇಶದಲ್ಲಿರುವ ಹಿಂಗಲಾಜ್ ಎಂಬ ತೀರ್ಥಸ್ಥಳಕ್ಕೆ ಹೋಗಬೇಕೆಂದು ಇದ್ದರು. ಅವರಿಗೆಲ್ಲ ನಡೆದು ನಡೆದು ಸಾಕಾಗಿತ್ತು. ಆದಕಾರಣ ಪೋರ್‍ಬಂದರಿನಿಂದ ಕರಾಚಿಗೆ ಹಡಗಿನಲ್ಲಿ ಹೋಗಿ ಅಲ್ಲಿಂದ ಒಂಟೆಯ ಮೇಲೆ ಹಿಂಗಲಾಜಿಗೆ ಹೋಗಬಯಸಿದರು. ಹಾಗೆ ಹೋಗಬೇಕಾದರೆ ದುಡ್ಡು ಬೇಕಾಗಿತ್ತು. ಇವರ ಕೈಯಲ್ಲಿ ಕಾಸು ಇರಲಿಲ್ಲ. ಯಾರೊ ಒಬ್ಬರು ಸಾಧುಗಳು, ಪೋರ್‍ಬಂದರಿನ ದಿವಾನರೊಡನೆ ಒಬ್ಬ ಇಂಗ್ಲೀಷ್ ಮಾತನಾಡುವ ಸಂನ್ಯಾಸಿಗಳು ಇರುವರು; ತ್ರಿಗುಣಾತೀತರು ಅವರನ್ನು ಕಂಡು ದಿವಾನರ ಮೂಲಕ ತಮ್ಮ ಯಾತ್ರೆಗೆ ಹಣ ಕೊಡಿಸುವಂತೆ ಮಾಡಿದರೆ ಅನುಕೂಲವಾಗುವುದು ಎಂದು ಹೇಳಿದರು. ತ್ರಿಗುಣಾತೀತರು ದಿವಾನರ ಮನೆಗೆ ಹೊರಟರು. ದಿವಾನರ ಮನೆಯಲ್ಲಿ ಇಂಗ್ಲೀಷ್ ಮಾತನಾಡುವ ಸ್ವಾಮಿಯನ್ನು ಕಂಡಾಗ, ಅವರು ತಮ್ಮ ಗುರುಭಾಯಿ ನರೇಂದ್ರನೇ ಆಗಿದ್ದನು. ಸ್ವಾಮೀಜಿ ತ್ರಿಗುಣಾತೀತರ ಪರವಾಗಿ ದಿವಾನರ ಹತ್ತಿರ ಮಾತನಾಡಿ ಅವರಿಗೆ ಯಾತ್ರೆಗೆ ಬೇಕಾಗುವ ದುಡ್ಡನ್ನು ಕೊಡಿಸಿ, ತ್ರಿಗುಣಾತೀತರಿಗೆ ತಮ್ಮನ್ನು ಹಿಂಬಾಲಿಸಬೇಡವೆಂದು ಹೇಳಿದರು. ಸ್ವಾಮೀಜಿ ಕೆಲವು ದಿನಗಳ ಮೇಲೆ ಪೋರ್‍ಬಂದರನ್ನು ಬಿಟ್ಟು ದ್ವಾರಕೆಗೆ ಹೋದರು. ದ್ವಾರಕೆ ಶ‍್ರೀಕೃಷ್ಣನ ಜೀವನದಿಂದ ಓತಪ್ರೋತವಾದ ಸ್ಥಳ. ಆದರೆ ಶ‍್ರೀಕೃಷ್ಣನ ಕಾಲದ ದ್ವಾರಕೆ ಈಗ ಸಮುದ್ರದ ಪಾಲಾಗಿದೆ. ಸಮುದ್ರದ ಅಲೆಗಳು ಅದರ ಮೇಲೆ ತಾಂಡವವಾಡುತ್ತಿವೆ. ಸ್ವಾಮೀಜಿ ಸಮುದ್ರತೀರದಲ್ಲಿ ಕುಳಿತು ಹಿಂದಿನದನ್ನೆಲ್ಲ ತಮ್ಮ ಬಗೆಗಣ್ಣಿಗೆ ತಂದು ಮನನ ಮಾಡಿದರು. ಅನಂತರ ಈಗಿನ ದ್ವಾರಕೆಗೆ ನಡೆದುಕೊಂಡು ಹೋದರು. ದ್ವಾರಕೆಯಲ್ಲಿ ಶಂಕರಾಚಾರ‍್ಯರ ಒಂದು ಶಾರದಾ ಮಠವಿದೆ. ಸ್ವಾಮೀಜಿ ಅಲ್ಲಿಗೆ ಹೋಗಿ ಅಲ್ಲಿಯ ಸ್ವಾಮಿಗಳನ್ನು ಕಂಡಾಗ ಅವರು ಸ್ವಾಮಿಗಳಿಗೆ ತಂಗಲು ಒಂದು ಕೋಣೆಯನ್ನು ಕೊಟ್ಟರು.

 ದ್ವಾರಕೆಯಿಂದ ಸ್ವಾಮೀಜಿ ಮಾಂಡ್ವಿಗೆ ಬಂದರು. ಅಲ್ಲಿ ತಮ್ಮನ್ನು ಡೆಲ್ಲಿಯಿಂದ ಹುಡುಕಿಕೊಂಡು ಬರುತ್ತಿದ್ದ ಅಖಂಡಾನಂದರನ್ನು ಕಂಡರು. ಇಬ್ಬರು ಗುರುಭಾಯಿಗಳೂ ಸುಮಾರು ಎರಡು ವಾರಗಳು ಒಟ್ಟಿಗೆ ಇದ್ದರು. ಅನಂತರ ತಮ್ಮ ಜೊತೆ ಬಿಟ್ಟುಹೋಗುವಂತೆ ಸ್ವಾಮೀಜಿ ಅಖಂಡಾನಂದರಿಗೆ ಹೇಳಿದ ಮೇಲೆ ಅವರು ತೀರ್ಥಯಾತ್ರೆಗೆ ಬೇರೆ ಕಡೆ ಹೊರಟರು. ಸ್ವಾಮೀಜಿ ಅಲ್ಲಿಂದ ನಾರಾಯಣ ಸರೋವರ ಮತ್ತು ಆಶಾಪುರಿಗಳನ್ನು ನೋಡಿಕೊಂಡು ಪುನಃ ಮಾಂಡ್ವಿಗೆ ಬಂದರು. ಮಹಾರಾಜರ ಕೋರಿಕೆಯಂತೆ ಅವರು ಪುನಃ ಭೂಜ್‍ಗೆ ಬಂದರು. ಅಲ್ಲಿಂದ ಪಾಲಿತಾನ ಎಂಬ ಊರಿಗೆ ಹೋದರು. ಸಮೀಪದಲ್ಲೇ ಜೈನರ ಯಾತ್ರಾಸ್ಥಳವಾದ ಶತೃಂಜಯ ಎಂಬ ಬೆಟ್ಟವಿದೆ. ಅಲ್ಲೆ ಹನುಮಂತನ ಒಂದು ಗುಡಿಯೂ ಹೆಗಾರ್ ಎಂಬ ಮಹಮ್ಮದೀಯ ಸಾಧುವಿನ ಹೆಸರಿನಲ್ಲಿ ಒಂದು ಗೋರಿಯೂ ಇದೆ. ಆ ಬೆಟ್ಟದ ಮೇಲಿನಿಂದ ರಮ್ಯವಾದ ದೃಶ್ಯ ಕಾಣುವುದು. ಸ್ವಾಮೀಜಿ ಇದನ್ನು ನೋಡುವುದಕ್ಕಾಗಿ ಮೇಲಕ್ಕೆ ಹತ್ತಿಹೋದರು. 

 ಸ್ವಾಮೀಜಿ ಅನಂತರ ಬರೋಡಕ್ಕೆ ಬಂದರು. ಅಲ್ಲಿಯ ದಿವಾನರಾದ ದಿವಾನ್ ಬಹದ್ದೂರ್ ಮನಿಭಾಯಿ ಎಂಬುವರ ಮನೆಯಲ್ಲಿ ಕೆಲವು ದಿನಗಳಿದ್ದು ಅಲ್ಲಿಂದ ಮಧ್ಯಪ್ರಾಂತ್ಯದಲ್ಲಿರುವ ಖಾಂಡ್ವ ಎಂಬಲ್ಲಿಗೆ ಹೋದರು. ಆ ಊರಿನಲ್ಲಿ ಅಲೆಯುತ್ತಿದ್ದಾಗ ಆ ಊರಿನ ವಕೀಲರಾದ ಬಾಬು ಹರಿದಾಸ ಚಟರ್ಜಿಯವರ ಮನೆಯ ಮುಂದೆ ನಿಂತಿದ್ದರು. ವಕೀಲರು ಕೋರ್ಟಿನಿಂದ ಹಿಂದಿರುಗುತ್ತಿದ್ದಾಗ ತಮ್ಮ ಮನೆಯ ಸಮೀಪದಲ್ಲಿ ಸ್ವಾಮೀಜಿ ನಿಂತಿರುವುದನ್ನು ನೋಡಿದರು. ಮುಂಚೆ ಅವರನ್ನು ಸಾಧಾರಣ ಸಂನ್ಯಾಸಿಗಳೆಂದು ತಿಳಿದುಕೊಂಡರು. ಅನಂತರ ಅವರೊಡನೆ ಮಾತಾಡುವಾಗ, ತಾವು ಇಂತಹ ವಿದ್ವಾಂಸರನ್ನು ಇದುವರೆಗೆ ಕಂಡಿಲ್ಲ ಎಂದು ಅರ್ಥವಾಯಿತು. ಅನಂತರ ಅವರನ್ನು ತಮ್ಮ ಮನೆಯಲ್ಲಿರಿಸಿಕೊಂಡು ಅವರನ್ನು ಮನೆಯ ಒಬ್ಬರಂತೆ ನೋಡಿಕೊಂಡರು. ಆ ಊರಿನಲ್ಲಿದ್ದ ಬಂಗಾಳಿಗಳು ಮತ್ತು ಇತರರು ಸ್ವಾಮೀಜಿ ಅವರನ್ನು ನೋಡಲು ಪ್ರತಿದಿನ ಬರುತ್ತಿದ್ದರು. ಅವರ ಪ್ರವಚನಾದಿಗಳನ್ನು ಕೇಳಿ ಪ್ರಯೋಜನ ಪಡೆದುಕೊಂಡು ಹೋಗುತ್ತಿದ್ದರು. ಸ್ವಾಮೀಜಿ ಇಲ್ಲಿದ್ದಾಗ ಅವರ ಜ್ಞಾಪಕಾರ್ಥವಾಗಿ ಅಲ್ಲಿರುವ ಬಂಗಾಳಿಗಳಿಗೆಲ್ಲ ಒಂದು ಔತಣವನ್ನು ಕೊಟ್ಟರು. ಊಟಕ್ಕೆ ಹಲವು ವಿದ್ವಾಂಸರು ವಕೀಲರು ಎಲ್ಲರೂ ಬಂದಿದ್ದರು. ಸ್ವಾಮೀಜಿ ಆ ಸಮಯದಲ್ಲಿ ವೇದಾಂತದಿಂದ ಕೆಲವು ಶ್ಲೋಕಗಳನ್ನು ತೆಗೆದುಕೊಂಡು ಎಲ್ಲರಿಗೂ ಅರ್ಥವಾಗುವಂತೆ ವಿವರಿಸಿದರು. ಮೊದಲು ಅವರೊಡನೆ ವಾದ ಮಾಡಬೇಕೆಂದು ಬಂದಿದ್ದವರು ಸ್ವಾಮೀಜಿಯವರ ವಿವರಣೆಯನ್ನು ಕೇಳಿ ಆದಮೇಲೆ ಅವರೊಡನೆ ಚರ್ಚೆಮಾಡುವುದು ವ್ಯರ್ಥ ಎಂದು ಭಾವಿಸಿದರು. ಸ್ವಾಮೀಜಿಯವರ ವ್ಯಕ್ತಿತ್ವದಲ್ಲೆ ಒಂದು ಮಹತ್ವ ಇದ್ದಿತು. ಇದನ್ನು ಸ್ವ್ಬಾಮೀಜಿಗೆ ಹರಿದಾಸ್‍ಬಾಬು ಹೇಳಿದಾಗ ಸ್ವಾಮೀಜಿ, “ಅದು ನನಗೆ ಗೊತ್ತಿಲ್ಲ. ಆದರೆ ನನ್ನ ಗುರುಗಳೂ ಕೂಡ ನನ್ನ ವಿಷಯದಲ್ಲಿ ಅದನ್ನೇ ಹೇಳುತ್ತಿದ್ದರು; ಮತ್ತೂ ಹೆಚ್ಚು ಉಜ್ವಲವಾಗಿ ಹೇಳುತ್ತಿದ್ದರು” ಎಂದರು. ಇಷ್ಟು ಹೊತ್ತಿಗೆ ಸ್ವಾಮೀಜಿ ಮನಸ್ಸಿನಲ್ಲಿ ತಾವು ಚಿಕಾಗೊ ವಿಶ್ವಧರ್ಮ ಸಮ್ಮೇಳನಕ್ಕೆ ಹೋಗಬೇಕು ಎಂದು ದೃಢಮಾಡಿಕೊಂದಿದ್ದರು. ಹರಿದಾಸ ಬಾಬುವಿಗೆ “ಯಾರಾದರೂ ನನಗೆ ದಾರಿಯ ಖರ್ಚನ್ನು ಕೊಟ್ಟರೆ ನಾನು ಹೋಗುತ್ತೇನೆ” ಎಂದಿದ್ದರು. 

ಸ್ವಾಮೀಜಿ ಖಾಂಡ್ವಾ ಬಿಟ್ಟು ಹೋಗುವಾಗ ಹರಿದಾಸಬಾಬುಗಳ ಸಹೋದರರು ಬೊಂಬಾಯಿನಲ್ಲಿ ಪ್ರಖ್ಯಾತ ವಕೀಲರಾದ ಸೇಟ್ ರಾಮದಾಸ್ ಚಬೀಲ್‍ದಾಸ್ ಅವರಿಗೆ ಒಂದು ಪರಿಚಯ ಪತ್ರವನ್ನು ಕೊಟ್ಟರು. ಖಾಂಡ್ವದಲ್ಲಿ ಹಲವು ಸ್ನೇಹಿತರು ಮತ್ತು ಭಕ್ತರಿಂದ ಬೀಳ್ಕೊಂಡು ಸ್ವಾಮೀಜಿ ೧೮೯೨ನೇ ಜುಲೈ ಕೊನೆಯ ವಾರದ ಹೊತ್ತಿಗೆ ಬೊಂಬಾಯಿಗೆ ಬಂದರು. ಚಬೀಲ್‍ದಾಸರು ಇವರನ್ನು ಸ್ವಾಗತಿಸಿ ತಮ್ಮ ಮನೆಗೆ ಕರೆದುಕೊಂಡು ಹೋದರು. ಒಂದು ದಿನ ಸ್ವಾಮೀಜಿ ಬೊಂಬಾಯಿನಲ್ಲಿ ಒಬ್ಬ ಪ್ರಸಿದ್ಧ ರಾಜಕೀಯ ವ್ಯಕ್ತಿಯನ್ನು ನೋಡಿದಾಗ ಆತ ಸ್ವಾಮೀಜಿಯವರಿಗೆ ಒಂದು ಕಲ್ಕತ್ತೆಯ ವೃತ್ತಪತ್ರಿಕೆಯನ್ನು ಕೊಟ್ಟ. ಸ್ತ್ರೀಯರ ಬಾಲ್ಯವಿವಾಹವನ್ನು ರದ್ದುಗೊಳಿಸುವ ಮಸೂದೆಯ ವಿರುದ್ಧ ಕಲ್ಕತ್ತೆಯ ವಿದ್ಯಾವಂತರು ಚಳುವಳಿ ಹೂಡಿರುವುದನ್ನು ಅದರಲ್ಲಿ ಓದಿ ಸ್ವಾಮೀಜಿ ಬಹಳ ವ್ಯಥೆಪಟ್ಟರು. ಬಾಲ್ಯವಿವಾಹ ಪದ್ಧತಿಯನ್ನು ಅತ್ಯಂತ ಕಟುವಾಗಿ ಖಂಡಿಸಿದರು. ಸ್ವಾಮೀಜಿ ಬೊಂಬಾಯಿನಲ್ಲಿ ಹಲವು ವಾರಗಳಿದ್ದರು.

ಸ್ವಾಮೀಜಿ ಬೊಂಬಾಯಿ ರೈಲ್ವೆ ಸ್ಟೇಷನ್ನಿನಲ್ಲಿ ಪುಣೆಗೆ ಹೋಗುವ ಗಾಡಿಯಲ್ಲಿ ಕುಳಿತಿದ್ದರು. ಅನೇಕ ಗುಜರಾತಿ ಭಕ್ತರು ಅವರೊಡನೆ ಬಂದಿದ್ದರು. ಅದೇ ಗಾಡಿಯಲ್ಲಿ ಬಾಲಗಂಗಾಧರತಿಲಕರು ಪ್ರಯಾಣಮಾಡುತ್ತಿದ್ದರು. ಕೆಲವು ಸ್ವಾಮೀಜಿಯ ಭಕ್ತರು ತಿಲಕರಿಗೆ ಸ್ವಾಮೀಜಿಯವರನ್ನು ತಮ್ಮ ಮನೆಯಲ್ಲಿ ಅತಿಥಿಗಳಾಗಿ ಇಟ್ಟುಕೊಳ್ಳಬೇಕೆಂದು ಕೇಳಿಕೊಂಡಾಗ ತಿಲಕರು ಸಂತೋಷದಿಂದ ಒಪ್ಪಿಕೊಂಡರು. ತಿಲಕರು ಸ್ವಾಮೀಜಿಯವರೊಡನೆ ಹಲವು ವಿಷಯಗಳ ಮೇಲೆ ಮಾತನಾಡಿದರು. ಪೂನಾದಲ್ಲಿ ಅವರ ಮನೆಯಲ್ಲಿ ಸ್ವಾಮೀಜಿ ಒಂದು ವಾರ ಇದ್ದರು. ಸಮೀಪದಲ್ಲೆ ಇದ್ದ ಮಹಾಬಲೇಶ್ವರ ಬೆಟ್ಟದಲ್ಲಿ ಲಿಂಬ್ಡಿ ಮಹಾರಾಜರು ಇರುವರು ಎಂಬುದನ್ನು ಕೇಳಿ ಸ್ವಾಮೀಜಿ ಅಲ್ಲಿಗೆ ಹೋದರು. ಮಹಾರಾಜರಿಗೆ ಸಂತೋಷವಾಗಿ ಇನ್ನುಮೇಲೆ ಎಂದೆಂದಿಗೂ ಬಂದು ತಮ್ಮ ಅರಮನೆಯಲ್ಲಿ ಇದ್ದುಬಿಡಿ ಎಂದು ಕೇಳಿಕೊಂಡರು. ಆದರೆ ಸ್ವಾಮೀಜಿ ತಾವು ಈಗ ಯಾವುದೋ ಒಂದು ಕೆಲಸವನ್ನು ಮಾಡಬೇಕಾಗಿದೆ ಎಂದೂ, ಅದಾದಮೇಲೆ ತಾವೇನಾದರೂ ನಿವೃತ್ತಿಜೀವನ ನಡೆಸುವ ಹಾಗಿದ್ದರೆ ಅಲ್ಲಿಗೆ ಬರುತ್ತೇನೆ ಎಂದೂ ಹೇಳಿದರು. ಆದರೆ ವಿರಾಮ, ವಿಶ್ರಾಂತಿ ಸ್ವಾಮೀಜಿಯ ಹಣೆಯಲ್ಲಿ ಬರೆದಿರಲಿಲ್ಲ. ಅವರು ಬಂದದ್ದೇ ಮಹಾಕಾರ್ಯಸಾಧನೆಗೆ, ಅದನ್ನು ಮಾಡುತ್ತಿರುವಾಗಲೇ ಅವರು ಪ್ರಪಂಚವನ್ನು ತ್ಯಜಿಸಿದರು. 

 ಸ್ವಾಮೀಜಿ ಅನಂತರ ಕೊಲ್ಲಾಪುರಕ್ಕೆ ಹೋದರು. ಅಲ್ಲಿಯ ಮಹಾರಾಜರೊಡನೆ ಮಾತುಕತೆಯಾಡಿದರು. ಕೊಲ್ಲಾಪುರದ ಮಹಾರಾಣಿ ಸ್ವಾಮೀಜಿಯವರ ಭಕ್ತೆಯಾದಳು. ಆಕೆ ಸ್ವಾಮೀಜಿಗೆ ಒಂದು ಹೊಸ ಕಾವಿಯಬಟ್ಟೆಯನ್ನು ಕೊಟ್ಟಳು. ಕೊಲ್ಲಾಪುರದ ದೊಡ್ಡ ಸರ್ಕಾರಿ ಅಧಿಕಾರಿ ಗೋಲ್‍ವಾಲ್‍ಕರ್ ಎಂಬುವರು ಬೆಳಗಾಂನಲ್ಲಿರುವ ಒಬ್ಬ ಮಹಾರಾಷ್ಟ್ರ ಮಹನೀಯರಿಗೆ ಒಂದು ಪರಿಚಯ ಪತ್ರವನ್ನು ಕಳುಹಿಸಿಕೊಟ್ಟರು. 

