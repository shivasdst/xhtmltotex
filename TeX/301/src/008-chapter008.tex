
\chapter{ಕಷ್ಟದ ಕುಲುಮೆಯಲ್ಲಿ}

ನರೇಂದ್ರ ದಕ್ಷಿಣೇಶ್ವರಕ್ಕೆ ಮಧ್ಯೆ ಮಧ್ಯೆ ಹೋಗಿ ಬರುತ್ತಿದ್ದನು. ಶ‍್ರೀರಾಮಕೃಷ್ಣರ ಸಂಬಂಧ ಅವನ ಆಧ್ಯಾತ್ಮಿಕ ಸ್ವಭಾವದ ಬೀಜಗಳಿಗೆ ನೀರೆರೆದಂತೆ ಇತ್ತು. ಕಾಲೇಜಿನಲ್ಲಿ\break ತಾತ್ತ್ವಿಕ ವಿಷಯಗಳ ಮೇಲೆ ಹಲವು ಗ್ರಂಥಗಳನ್ನು ಓದುತ್ತಿದ್ದನು. ಮನೆಯಲ್ಲಿ ದೀರ್ಘ\-ಧ್ಯಾನದಲ್ಲಿಯೂ ತಲ್ಲೀನನಾಗಿರುತ್ತಿದ್ದ. ಅವನಿಗೇ ಒಂದು ಪ್ರತ್ಯೇಕವಾದ ಕೋಣೆ ಇತ್ತು. ವಿಶ್ವನಾಥದತ್ತನಿಗೆ ತನ್ನ ಮಗ ನರೇಂದ್ರನನ್ನೂ ಒಬ್ಬ ಪ್ರಮುಖ ವಕೀಲನನ್ನಾಗಿ ಮಾಡಬೇಕೆಂಬ ಆಸೆ. ಅದಕ್ಕಾಗಿ ಕಲ್ಕತ್ತೆಯಲ್ಲಿ ನಿಮಯಚರಣ ಬೋಸ್ ಎಂಬ ಪ್ರಖ್ಯಾತ ವಕೀಲನ ಹತ್ತಿರ ಅವನನ್ನು ಸೇರಿಸಿದನು. ಅನೇಕರು ನರೇಂದ್ರನಿಗೆ ಹೆಣ್ಣನ್ನು ಕೊಡಲು ಬರುತ್ತಿದ್ದರು. ಆದರೆ ಅವನಿಗೆ ಸಂಸಾರದ ಮೇಲೆ ಅಭಿರುಚಿಯಿರಲಿಲ್ಲ. ಒಂದು ಸಲ ಮದುವೆ ಆಯಿತು ಎಂದರೆ ಆಧ್ಯಾತ್ಮಿಕ ಅಭೀಪ್ಸೆಗೆಲ್ಲ ತಿಲತರ್ಪಣವನ್ನು ಕೊಟ್ಟಂತೆ ಎಂದು ಭಾವಿಸಿದನು. ಶ‍್ರೀರಾಮಕೃಷ್ಣರಂತೂ ನರೇಂದ್ರನ ಮನೆಗೆ ಹೆಣ್ಣಿನವರು ಬಂದಿರುವರು ಎಂಬುದನ್ನು ಕೇಳಿದ ತತ್‍ಕ್ಷಣವೇ ವ್ಯಥೆಪಡುತ್ತಿದ್ದರು. ಕಾಳಿಕಾ ಮಂದಿರಕ್ಕೆ ಹೋಗಿ ನರೇಂದ್ರನ ಮದುವೆ ಆಗದಂತೆ ಮಾಡು ಎಂದು ಬೇಡುತ್ತಿದ್ದರು. ನರೇಂದ್ರ ಬಂದಿರುವುದು ಮದುವೆಯಾಗಿ ಯಾವುದೋ ಒಂದು ಹೆಣ್ಣಿಗೆ ಮತ್ತು ಕೆಲವು ಮಕ್ಕಳಿಗೆ ಆನಂದವನ್ನು ಕೊಡುವುದಕ್ಕೆ ಅಲ್ಲ. ಅವನು ಹಲವು ಜೀವಿಗಳ ಉದ್ಧಾರ ಮಾಡುವುದಕ್ಕೆ ಭವಸಾಗರದಲ್ಲಿ ಒಂದು ನಾವೆಯಂತೆ ಇರಬೇಕಾಗಿದೆ ಎಂಬುದು ಅವರ ಇಚ್ಛೆ. ಅಂತೂ ಒಂದು ಸಲ ವಿಶ್ವನಾಥದತ್ತ ಶ‍್ರೀಮಂತರೊಬ್ಬರ ಮನೆಯಿಂದ ಹೆಣ್ಣನ್ನು ತರಲು ನಿಶ್ಚಯಿಸಿದ. ಅವರು ಬೇಕಾದಷ್ಟು ವರದಕ್ಷಿಣೆಯನ್ನು ನರೇಂದ್ರನಿಗೆ ಕೊಡುವುದಕ್ಕೆ ಸಿದ್ಧರಾಗಿದ್ದರು. ಮದುವೆ ಆದಮೇಲೆ ಐ.ಸಿ.ಎಸ್. ಪರೀಕ್ಷೆಗೆ ನರೇಂದ್ರನನ್ನು ಇಂಗ್ಲೆಂಡಿಗೆ ಕಳುಹಿಸುವುದಕ್ಕೂ ಸಿದ್ಧರಾಗಿದ್ದರು. ಅದಕ್ಕೆ ಆಗುವ ವೆಚ್ಚವನ್ನೆಲ್ಲ ಅವರು ವಹಿಸಲು ಸಿದ್ಧವಾಗಿದ್ದರು. ಆದರೆ ಭಗವದಿಚ್ಛೆಯಿಂದ ಅದು ನೆರವೇರಲಿಲ್ಲ. ಯಾವುದೋ ಒಂದು ಕಾರಣದಿಂದ ನಿಂತುಹೋಯಿತು.

ನರೇಂದ್ರ ತ್ಯಾಗಜೀವನದ ಆನಂದವನ್ನು ತನ್ನ ಸ್ನೇಹಿತರಿಗೆ ವಿವರಿಸಿದನು. ಆದರೆ ಅವರು ಇದನ್ನು ತಿಳಿದುಕೊಳ್ಳುವ ಸ್ಥಿತಿಯಲ್ಲಿರಲಿಲ್ಲ. ಅವರು ಲೌಕಿಕ ರೀತಿಯಲ್ಲಿ ಅವನೆದುರಿಗೆ ವಾದಮಾಡತೊಡಗಿದರು: “ನರೇಂದ್ರ! ಯಾವುದಾದರೂ ಒಂದು ಉದ್ದೇಶವನ್ನು ಇಟ್ಟುಕೊಂಡು ಅದನ್ನು ಸಾಧಿಸಲು ಏತಕ್ಕೆ ಪಯತ್ನಿಸಬಾರದು? ನಿನಗೆ ಒಳ್ಳೆಯ ಭವಿಷ್ಯವಿದೆ. ನೀನು ಅವುಗಳ ಕಡೆ ಗಮನ ಕೊಟ್ಟರೆ ಸಾಕು. ಅವುಗಳನ್ನು ಪಡೆಯುವ ಶಕ್ತಿ ಇದೆ.” ಅದಕ್ಕೆ ನರೇಂದ್ರ, “ನಾನೂ ಅನೇಕ ವೇಳೆ ಐಶ್ವರ್ಯ ಕೀರ್ತಿ ಅಧಿಕಾರ ಪಡೆಯಬೇಕೆಂದು ಯೋಚಿಸುವೆ. ಆದರೆ ಇವುಗಳನ್ನೆಲ್ಲ ಮೃತ್ಯು ಬಂದು ಅಪಹರಿಸಿಕೊಂಡು ಹೋಗಿಬಿಡುವುದು. ಯಾವ ಪ್ರಾಪಂಚಿಕ ದೊಡ್ಡಸ್ತಿಕೆಯನ್ನು ಮೃತ್ಯು ಸರ್ವನಾಶ ಮಾಡಿಬಿಡುವುದೋ ಅದನ್ನು ಪಡೆಯಲು ಏತಕ್ಕೆ ಪ್ರಯತ್ನಿಸಬೇಕು? ತ್ಯಾಗಜೀವನ ಒಂದೇ ಶ್ರೇಷ್ಠ. ತ್ಯಾಗಿಯೇ ಮೃತ್ಯುವಿಗೂ ಅತೀತನಾಗಿ ಹೋಗುವನು. ಅವನು ಎಂದೆಂದಿಗೂ ನಾಶವಾಗದ ಸತ್ಯವನ್ನು ಪಡೆಯುವನು. ಆದರೆ ಪ್ರಾಪಂಚಿಕ ಸುಖ ಐಶ್ವರ್ಯಗಳಾದರೊ ಪ್ರಪಂಚದ ವಿಕಾರಕ್ಕೆ ಸಿಕ್ಕಿ ನೀರುಗುಳ್ಳೆಯಂತೆ ಸರ್ವನಾಶವಾಗುವುವು” ಎಂದನು. ಸ್ನೇಹಿತರು, ನರೇಂದ್ರ ದಕ್ಷಿಣೇಶ್ವರಕ್ಕೆ ಹೋಗಿ ಶ‍್ರೀರಾಮಕೃಷ್ಣರ ಪ್ರಭಾವಕ್ಕೆ ತುತ್ತಾಗಿರುವುದೇ ಇದಕ್ಕೆಲ್ಲ ಕಾರಣ; ಅವರು ಯಾವಾಗಲೂ ದೇವರಲ್ಲದೆ ಬೇರೆ ಏನನ್ನೂ ಚಿಂತಿಸುವುದಿಲ್ಲ, ಅವರು ನರೇಂದ್ರನ ತಲೆಯನ್ನು ಕೆಡಿಸಿಬಿಟ್ಟಿರುವರು ಎಂದು ಭಾವಿಸಿದರು. ನರೇಂದ್ರನಿಗೆ, “ನಿನಗೆ ಭವಿಷ್ಯದಲ್ಲಿ ಏನಾದರೂ ಅಭಿಲಾಷೆಯಿದ್ದರೆ ದಕ್ಷಿಣೇಶ್ವರಕ್ಕೆ ಹೊಗುವುದನ್ನು ಬಿಟ್ಟುಬಿಡು” ಎಂದು ಚಿತಾವಣೆ ಮಾಡಿದರು. ಅದಕ್ಕೆ ನರೇಂದ್ರ “ನೋಡಿ, ಅದು ನಿಮಗೆ ಗೊತ್ತಾಗುವುದಿಲ್ಲ. ನನಗೂ ಅದು ಗೊತ್ತಾಗುವುದಿಲ್ಲ. ಆ ರಾಮಕೃಷ್ಣರೆಂಬ ಸಾಧುಪುರುಷರನ್ನು ನಾನು ಪ್ರೀತಿಸುತ್ತೇನೆ. ಅದು ಏತಕ್ಕೆ ಎಂಬುದು ನನಗೂ ಗೊತ್ತಿಲ್ಲ” ಎಂದನು.

ನರೇಂದ್ರ ದಕ್ಷಿಣೇಶ್ವರಕ್ಕೆ ಹೋಗದೆ ಇದ್ದರೆ ಶ‍್ರೀರಾಮಕೃಷ್ಣರೇ ನರೇಂದ್ರನ ಕೋಣೆಗೆ ಬಂದು ಧ್ಯಾನ ಮುಂತಾದ ವಿಷಯಗಳ ಮೇಲೆ ಸಲಹೆಗಳನ್ನು ಕೊಡುತ್ತಿದ್ದರು, ಬ್ರಹ್ಮಚಾರಿಯಾಗಿರುವಂತೆ ಅವನಿಗೆ ಹೇಳುತ್ತಿದ್ದರು. “ಹನ್ನೆರಡು ವರ್ಷಗಳು ಒಬ್ಬ ಬ್ರಹ್ಮಚರ್ಯ ವ್ರತವನ್ನು ಪರಿಪಾಲನೆ ಮಾಡಿದರೆ ಸೂಕ್ಷ್ಮವಾದ ಶಕ್ತಿಯನ್ನು ಪಡೆಯುವನು. ಆಗ ಎಂತಹ ಸೂಕ್ಷ್ಮವಾದ ವಿಷಯಗಳನ್ನಾದರೂ ಅರ್ಥಮಾಡಿಕೊಳ್ಳಬಲ್ಲ. ಬ್ರಹ್ಮಚರ್ಯವಿಲ್ಲದೇ ಇದ್ದರೆ ಸಾಧ್ಯವಿಲ್ಲ. ಸಾಧಕ ಇದನ್ನು ಅರಿತರೆ ಇದರಿಂದ ಪ್ರತ್ಯಕ್ಷವಾಗಿ ದೇವರನ್ನು ಸಾಕ್ಷಾತ್ಕಾರ ಮಾಡಿಕೊಳ್ಳಬಹುದು” ಎಂದು ಬೋಧಿಸುತ್ತಿದ್ದರು.

ಮನೆಯಲ್ಲಿದ್ದ ಹೆಂಗಸರಾದರೋ ನರೇಂದ್ರ ದಕ್ಷಿಣೇಶ್ವರಕ್ಕೆ ಹೋಗುತ್ತಿರುವುದರಿಂದಲೇ ಮದುವೆಯನ್ನು ಮಾಡಿಕೊಳ್ಳುವುದಕ್ಕೆ ಒಪ್ಪುತ್ತಿಲ್ಲ ಎಂದು ನಿರ್ಧಾರಕ್ಕೆ ಬಂದರು. ಒಂದು ದಿನ ಶ‍್ರೀರಾಮಕೃಷ್ಣರು ನರೇಂದ್ರನ ಕೋಣೆಯಲ್ಲಿ ಬ್ರಹ್ಮಚರ್ಯದ ವಿಷಯವನ್ನು ಮಾತನಾಡುತ್ತಿದ್ದುದನ್ನು ನರೇಂದ್ರನ ಅಜ್ಜಿ ಕೇಳಿದಳು. ಆಕೆ ಅದನ್ನು ನರೇಂದ್ರನ ತಂದೆ ತಾಯಿಗಳಿಗೆ ಹೇಳಿದಳು. ಅವರಿಗೆ ನರೇಂದ್ರ ಎಲ್ಲಿ ಮನೆಬಿಟ್ಟು ಸಂನ್ಯಾಸಿಯಾಗಿಬಿಡುವನೊ ಎಂಬ ಅಂಜಿಕೆ ಮೂಡಿತು. ಆಗಲಂತೂ ಸಾಧ್ಯವಾದಷ್ಟು ಬೇಗ ನರೇಂದ್ರನಿಗೆ ಮದುವೆ ಮಾಡಲು ಇನ್ನೂ ಹೆಚ್ಚು ಕಾತರವನ್ನು ತೋರಿದರು. ಆದರೆ ಶ‍್ರೀರಾಮಕೃಷ್ಣರ ಇಚ್ಛಾಮಾತ್ರದಿಂದ ಅವರ ಯೋಜನೆಗಳೆಲ್ಲ ಮುರಿದುಬಿತ್ತು.

೧೮೮೪ ರಲ್ಲಿ ನರೇಂದ್ರ ಬಿ.ಎ. ಪರೀಕ್ಷೆಗೆ ಕುಳಿತಿದ್ದ. ಇನ್ನೂ ಫಲಿತಾಂಶ ಹೊರಬಿದ್ದಿರಲಿಲ್ಲ. ಒಂದು ದಿನ ಸಂಜೆ ಕಲ್ಕತ್ತೆಯಿಂದ ಎರಡು ಮೈಲಿ ದೂರದಲ್ಲಿರುವ ವರಾಹನಗರದಲ್ಲಿ ನರೇಂದ್ರ ತನ್ನ ಸ್ನೇಹಿತನೊಬ್ಬನನ್ನು ನೋಡಲು ಹೋಗಿದ್ದ. ಊಟ ತಿಂಡಿ ಆದ ಮೇಲೆ ಸಂಗೀತವಾಗುತ್ತಿದ್ದಾಗ ನರೇಂದ್ರನಾಥನ ಮನೆಯಿಂದ ಒಬ್ಬರು ಬಂದು ವಿಶ್ವನಾಥದತ್ತ ಹೃದಯಸ್ತಂಭನದಿಂದ ಹಟಾತ್ತಾಗಿ ಮೃತ್ಯುವಶನಾದನೆಂಬ ಸಿಡಿಲಿನಂತಹ ಸುದ್ದಿಯನ್ನು ಕೊಟ್ಟರು. ಸುದ್ದಿಯನ್ನು ಕೇಳಿದ ಕೂಡಲೆ ನರೇಂದ್ರ ಸ್ತಂಭಿತನಾದ, ಕಲ್ಕತ್ತೆಗೆ ವೇಗವಾಗಿ ಪ್ರಯಾಣ ಮಾಡಿದ. ತಂದೆಯ ಕಳೇಬರದ ಸುತ್ತಲೂ ತಾಯಿ, ಇಬ್ಬರು ಸೋದರಿಯರು ಮತ್ತು ಇಬ್ಬರು ತಮ್ಮಂದಿರು ಕುಳಿತು ಅಳುತ್ತಿದ್ದರು. ರೂಢಿಯಂತೆ ಅಂತ್ಯಕ್ರಿಯೆಯನ್ನು ಮಾಡಿ ಆಯಿತು.

ವಿಶ್ವನಾಥದತ್ತನ ಆಕಸ್ಮಿಕ ಮರಣದಿಂದ ಮನೆಗೆ ದರಿದ್ರಲಕ್ಷ್ಮಿ ಕಾಲಿಟ್ಟಳು. ಬದುಕಿರುವ ತನಕ ವಿಶ್ವನಾಥದತ್ತ ಚೆನ್ನಾಗಿ ಸಂಪಾದನೆಯನ್ನು ಮಾಡುತ್ತಿದ್ದರೂ ಅದಕ್ಕಿಂತ ಹೆಚ್ಚಾಗಿ ಖರ್ಚುಮಾಡುತ್ತಿದ್ದನು. ನರೇಂದ್ರ ಒಂದು ಸಲ ತಂದೆಯನ್ನು ನನಗೆ ಏನು ಆಸ್ತಿಯನ್ನು ಬಿಟ್ಟುಹೋಗುವೆ ಎಂದು ಕೇಳಿದಾಗ “ಕನ್ನಡಿಯನ್ನು ನೋಡಿಕೋ, ಆಗ ಗೊತ್ತಾಗುವುದು!” ಎಂದಿದ್ದ. ಮಕ್ಕಳಿಗೆ ದುಡ್ಡು ಮುಂತಾದುವನ್ನು ಬಿಟ್ಟು ಹೋಗಿರಲಿಲ್ಲ. ಜೊತೆಗೆ ಅವನು ಸಾಲವನ್ನು ಬೇರೆ ಮಾಡಿದ್ದ. ವಿಶ್ವನಾಥದತ್ತ ತೀರಿದೊಡನೆಯೆ ಸಾಲಗಾರರು ಬಂದು ಮುತ್ತಿದರು. ಏಕೆಂದರೆ ವಿಶ್ವನಾಥದತ್ತ ಏನಾದರೂ ದುಡ್ಡು ಅಥವಾ ಆಸ್ತಿಯನ್ನು ಇಟ್ಟಿದ್ದರೆ ಅದು ಖರ್ಚಾಗಿ ಹೋಗುವುದಕ್ಕೆ ಮುಂಚೆಯೇ ತಾವು ಕೊಟ್ಟ ಸಾಲವನ್ನು ವಸೂಲಿಮಾಡಬೇಕೆಂದು ಅವರು ಇಚ್ಛಿಸಿದರು. ಇದೇ ಸಮಯವೆಂದು ಕೆಲವರು ನರೇಂದ್ರ ಇದ್ದ ಮನೆಯ ಮೇಲೆ ತಮಗೆ ಬರಬೇಕಾಗಿದ್ದ ಹಣಕ್ಕಾಗಿ ದಾವೆ ಹೂಡಿದರು. ಕಷ್ಟಗಳು ಯಾವಾಗಲೂ ಒಂಟಿಯಾಗಿ ಬರುವುದಿಲ್ಲ; ಅವು ಒಂದು ದೊಡ್ಡ ಬಳಗವನ್ನೇ ಕರೆದುಕೊಂಡುಬರುತ್ತವೆ. ನರೇಂದ್ರ ಇದುವರೆಗೆ ಮನೆಯ ಜವಾಬ್ದಾರಿಯನ್ನು ಹೊತ್ತಿರಲಿಲ್ಲ, ಸ್ವಚ್ಛಂದ ಹಕ್ಕಿಯಂತೆ ಬೆಳೆದವನು. ಈಗ ಇದ್ದಕ್ಕಿದ್ದಂತೆ ಪ್ರಪಂಚದ ಕ್ರೂರ ಪರಿಸ್ಥಿತಿಯನ್ನು ಎದುರಿಸಬೇಕಾಯಿತು. ಆದರೆ ನರೇಂದ್ರ ಕಷ್ಟದ ಭಾರಕ್ಕೆ ಕುಗ್ಗಿ ಹೋಗುವವನಲ್ಲ. ಪರಿಸ್ಥಿತಿಯನ್ನು ಎದುರಿಸಿದನು, ಸೋಲನ್ನು ಒಪ್ಪಿಕೊಳ್ಳಲಿಲ್ಲ. ಮನೆಯಲ್ಲಿ ಸುಮಾರು ಎಂಟು ಜನಗಳಿಗೆ ಊಟವನ್ನು ಒದಗಿಸಬೇಕಾಗಿತ್ತು. ಆಫೀಸಿನಲ್ಲಿ ಸಿಕ್ಕಿದ ಸಣ್ಣ ಪುಟ್ಟ ಕೆಲಸಗಳನ್ನು ಮಾಡಿ ಮನೆಯವರಿಗೆ ಒಂದು ತುತ್ತು ಅನ್ನವನ್ನು ಹೇಗೋ ಒದಗಿಸಿದ. ಬಿ.ಎ. ಫಲಿತಾಂಶ ಹೊರಬಿತ್ತು. ಅವನು ಪಾಸಾಗಿದ್ದ. ಲಾ ಕಾಲೇಜಿಗೆ ಸೇರಿದ. ಆಗ ದರಿದ್ರರಲ್ಲಿ ದರಿದ್ರ ವಿದ್ಯಾರ್ಥಿಯಾಗಿದ್ದ. ಮೆಟ್ಟಿಕೊಳ್ಳುವುದಕ್ಕೆ ಜೋಡಿರಲಿಲ್ಲ. ಮೈಮೇಲಿದ್ದ ಬಟ್ಟೆಯೆಲ್ಲ ಹರಿದುಹೋಗಿತ್ತು. ಮನೆಯಲ್ಲಿ ಸಾಕಷ್ಟು ಊಟ ಮಾಡಲು ಆಹಾರವಿರಲಿಲ್ಲ. ಇತರರು ಸ್ವಲ್ಪ ಜಾಸ್ತಿ ತಿನ್ನಲಿ ಎಂದು ನರೇಂದ್ರ ಸ್ನೇಹಿತರ ಮನೆಗೆ ತನ್ನನ್ನು ಕರೆದಿರುವರು ಎಂದು ಹೇಳಿ, ಪಾರ್ಕಿನಲ್ಲಿರುವ ನಲ್ಲಿಯ ನೀರನ್ನು ಕುಡಿದು ಕ್ಲಾಸಿಗೆ ಹೋಗುತ್ತಿದ್ದ. ಕ್ಲಾಸಿನಲ್ಲಿ ಉಪವಾಸದ ಶ್ರಮದಿಂದ ಕೆಲವು ವೇಳೆ ಪ್ರಜ್ಞೆಯನ್ನು ಕಳೆದುಕೊಳ್ಳುತ್ತಿದ್ದ. ಸ್ನೇಹಿತರ ಮನೆಗಳಿಗೆ ಹೋದಾಗ ಅವನನ್ನು ಊಟಕ್ಕೆ ಏಳಿಸುತ್ತಿದ್ದರು. ಆಗ ಮನೆಯಲ್ಲಿ ಸಂಕಟಪಡುತ್ತಿರುವ ತಾಯಿ ತಮ್ಮ ಮತ್ತು ಅಕ್ಕಂದಿರ ನೆನಪು ಬಂದು ಊಟವನ್ನು ಮಾಡಲಾರದೆ ತನಗೆ ಯಾವುದೋ ತುರ್ತು ಕೆಲಸವಿದೆ ಎಂದು ಹೊರಟುಹೋಗುತ್ತಿದ್ದನು. ಇಷ್ಟೊಂದು ಕಾರ್ಪಣ್ಯದಲ್ಲಿ ನಡೆಯುತ್ತಿದ್ದರೂ ಹೊರಗಿನಿಂದ ನೋಡಿದರೆ ಹಿಂದಿನಂತೆಯೇ ಸಂತೋಷಚಿತ್ತದವನಂತೆ ತೋರಿಸಿಕೊಳ್ಳುತ್ತಿದ್ದ. ತನ್ನ ಕಷ್ಟವನ್ನು ಇತರರಿಗೆ ಹೇಳಿಕೊಂಡು ಅವರ ಕನಿಕರವನ್ನು ಸಂಪಾದಿಸಲು ಯತ್ನಿಸಲಿಲ್ಲ.

ಇದೇ ಸಮಯದಲ್ಲಿ ಅವರ ಬಂಧುಗಳೊಬ್ಬರು ಮನೆಯ ಬಹುಪಾಲು ತಮಗೆ ಬರಬೇಕೆಂದು ಕೋರ್ಟಿನಲ್ಲಿ ಮೊಕದ್ದಮೆಯನ್ನು ಹೂಡಿದರು. ಎಷ್ಟೇ ಬಡತನ ಬಂದರೂ, ಕಷ್ಟ ಬಂದರೂ, ಇದುವರೆಗೆ ತಮ್ಮ ಮನೆ ಎಂಬುದೊಂದು ಇತ್ತು, ಅಲ್ಲಿರುತ್ತಿದ್ದರು. ಈಗ ಮನೆಯೂ ಹೋದರೆ ಜೀವನದ ಏಕಮಾತ್ರ ವಸ್ತುವನ್ನು ಕಳೆದುಕೊಂಡಂತೆ ಆಗುವುದು. ತಾಯಿ ಇದನ್ನು ಕುರಿತು ಆಲೋಚಿಸಿಯೇ ಹತಾಶಳಾದಳು. ಕೋರ್ಟಿನಲ್ಲಿ ಕೇಸು ಬಹಳ ದಿನಗಳವರೆಗೆ ಸಾಗಿತು. ನರೇಂದ್ರ ತನ್ನ ಪರವಾಗಿ ಕೋರ್ಟಿನಲ್ಲಿ ಯಾವ ಲಾಯರನ್ನೂ ಇಡದೆ ತಾನೇ ವಾದಿಸುತ್ತಿದ್ದ. ಲಾಯರಿಗೆ ಕೊಡುವುದಕ್ಕೆ ದುಡ್ಡು ಎಲ್ಲಿ ಬರಬೇಕು? ನರೇಂದ್ರನ ವಾದಸರಣಿಯನ್ನು ಕೇಳಿದ ನ್ಯಾಯಾಧಿಪತಿಗಳು, “ನೀನು ಏತಕ್ಕೆ ವಕೀಲನ ಕೆಲಸಕ್ಕೆ ಕೈ ಹಾಕಬಾರದು?” ಎಂದರು. ನರೇಂದ್ರನೂ ಅದಕ್ಕಾಗಿ ಓದುತ್ತಿದ್ದ. ಆದರೆ ವಿಧಿ ಅವನನ್ನು ಬೇರೆ ಕಡೆ ವಕೀಲನಾಗುವಂತೆ ಮಾಡಿತು. ಅದೇ ಜೀವಿಗಳ ಪರವಾಗಿ ದೇವರೆಡೆಯಲ್ಲಿ ವಾದಿಸುವುದು. ಅಂತೂ ಕೊನೆಗೆ ಕೇಸು ನರೇಂದ್ರನ ಪರವಾಗಿ ಆಯಿತು. ಅವರು ಮನೆಯನ್ನು ಬಿಡಬೇಕಾಗಿ ಬರಲಿಲ್ಲ.

ಹೊಟ್ಟೆಗೆ ಬಟ್ಟೆಗೆ ಸಂಪಾದಿಸುವುದಕ್ಕೆ ನರೇಂದ್ರ ತನ್ನ ಕೈಲಾದ ಸಾಹಸವನ್ನೆಲ್ಲ ಮಾಡಿದ. ಅವನೊಬ್ಬ ಪ್ರಿಮೇಸನ್ ಆದ. ಏಕೆಂದರೆ ಅಲ್ಲಿಗೆ ಬರುವ ಅನೇಕರ ಪರಿಚಯವಾಗಿ ಒಂದು ಚೆನ್ನಾದ ಕೆಲಸ ಸಿಕ್ಕೀತೇನೊ ಎಂದು. ವಿದ್ಯಾಸಾಗರನ ಒಂದು ಪಾಠಶಾಲೆಯಲ್ಲಿ ಉಪಾಧ್ಯಾಯನಾಗಿದ್ದು, ಕೆಲವು ಕಾಲದ ಮೇಲೆ ಬಿಟ್ಟುಬಿಟ್ಟ. ಅಂತೂ ಅಹನ್ಯಹನಿ ಕಾಲಕ್ಷೇಪದಿಂದ, ಕೆಲವು ವೇಳೆ ನರೇಂದ್ರನಿಗೆ ತುಂಬಾ ನಿರಾಶೆ ಆಗುತ್ತಿತ್ತು. ಆದರೆ ಎಂದಿಗೂ ಅದನ್ನು ಹೊರಗೆ ವ್ಯಕ್ತಪಡಿಸುತ್ತಿರಲಿಲ್ಲ. ನರೇಂದ್ರ ಬಡಪೆಟ್ಟಿಗೆ ಸೋಲನ್ನು ಜೀವನದಲ್ಲಿ ಒಪ್ಪಿಕೊಳ್ಳುವ ಚೇತನವಲ್ಲ. ಅಂತೂ ಜೀವನದ ಹೋರಾಟ ಮುಂದೆ ಸಾಗಿತು. ಆ ಕಾಲದ ನರೇಂದ್ರನ ಸ್ಥಿತಿಯನ್ನು ಅವನ ಬಾಯಿಯಿಂದಲೇ ಕೇಳೋಣ:

“ನನಗೆ ತಂದೆ ಸತ್ತ ದುಃಖವನ್ನು ಮರೆಯುವುದಕ್ಕೆ ಮುಂಚೆಯೇ ಒಂದು ಕೆಲಸಕ್ಕೆ ಅಲೆದಾಡಬೇಕಾಯಿತು. ಹೊಟ್ಟೆಗೆ ಹಿಟ್ಟು ಇಲ್ಲದೆ, ಕಾಲಿಗೆ ಹಾಕಿಕೊಳ್ಳುವುದಕ್ಕೆ ಮೆಟ್ಟು ಇಲ್ಲದೆ, ಉರಿಬಿಸಿಲಿನಲ್ಲಿ ಕಂಕುಳಲ್ಲಿ ಒಂದು ಕೆಲಸಕ್ಕೆ ಅರ್ಜಿಯನ್ನು ತೆಗೆದುಕೊಂಡು ಆಫೀಸಿನಿಂದ ಆಫೀಸಿಗೆ ಅಲೆಯಬೇಕಾಯಿತು. ನನ್ನ ದುಃಖದಲ್ಲಿ ಸಹಾನುಭೂತಿಯನ್ನು ತೋರುತ್ತಿದ್ದ ಪ್ರಿಯ ಸ್ನೇಹಿತರು ಒಂದಿಬ್ಬರು ಯಾವಾಗಲೂ ನನ್ನೊಡನೆ ಸಹಾಯಕ್ಕೆ ಬರುತ್ತಿದ್ದರು. ಆದರೆ ಎಲ್ಲಿ ಹೋದರೂ ಕೆಲಸ ಸಿಕ್ಕಲಿಲ್ಲ. ಪ್ರಪಂಚದ ನಿಷ್ಠುರವಾದ ಸತ್ಯವನ್ನು ನಾನು ಪ್ರಥಮ ಬಾರಿ ಸಂದರ್ಶಿಸಿದಾಗ ಪ್ರಪಂಚದಲ್ಲಿ ನಿಃಸ್ವಾರ್ಥವಾದ ಕರುಣೆ ಎಂಬುದು ಬಹಳ ಅಪರೂಪ ಎಂಬುದು ವೇದ್ಯವಾಯಿತು. ಇಲ್ಲಿ ಬಡವರಿಗೆ ದರಿದ್ರರಿಗೆ ಅನಾಥರಿಗೆ ಸ್ಥಳವಿಲ್ಲ ಎನ್ನಿಸಿತು. ಕೆಲವು ದಿನಗಳಿಗೆ ಮುಂಚೆ, ತಂದೆ ಸಾಯುವುದಕ್ಕೆ ಮುಂಚೆ ನನಗೆ ಯಾವ ರೀತಿಯಲ್ಲಿ ಬೇಕಾದರೂ ಸಹಾಯ ಮಾಡುವುದಕ್ಕೆ ಸಿದ್ಧರಾಗಿದ್ದವರು, ತಾವು ಅನುಭವಿಸಿ ಇತರರಿಗೆ ನೀಡುವುದಕ್ಕೆ ಬೇಕಾದಷ್ಟು ಇದ್ದರೂ, ಈಗ ನನ್ನ ಕಡೆ ಕಣ್ಣೆತ್ತಿ ಕೂಡ ನೋಡಲಿಲ್ಲ. ಇದನ್ನೆಲ್ಲ ನೋಡಿದಾಗ ಪ್ರಪಂಚ ಸೈತಾನನಿಂದ ನಿರ್ಮಾಣವಾದ ನರಕದಂತೆ ತೋರಿತು. ಒಂದು ದಿನ ನಡೆದು ನಡೆದು ಸಾಕಾಗಿ ಅಕ್ಬರ್ ಲೋನಿ ‍ ಮೈದಾನದಲ್ಲಿ ಕುಳಿತಿದ್ದೆ. ನನ್ನ ಕೆಲವು ಸ್ನೇಹಿತರು ಅಲ್ಲಿದ್ದರು. ಅವರಲ್ಲಿ ಒಬ್ಬ, ಬಹುಶಃ ನನ್ನನ್ನು ಸಮಾಧಾನ ಮಾಡುವುದಕ್ಕೆ ಇರಬೇಕು ಎಂದು ತೋರುವುದು, ಭಗವಂತನ ಅನಂತಕೃಪೆಯನ್ನು ವಿವರಿಸುವ ಹಾಡೊಂದನ್ನು ಹಾಡಿದನು. ಇದು ನನ್ನ ತಲೆಗೆ ಒಂದು ಪ್ರಹಾರದಂತೆ ಬಿತ್ತು. ಮನೆಯಲ್ಲಿ ನಿಸ್ಸಹಾಯಕರಾಗಿದ್ದ ತಾಯಿ ಸಹೋದರ ಸಹೋದರಿಯರ ನೆನಪು ಬಂತು. ಅಸದಳವಾದ ನಿರಾಸೆಯಿಂದ ರೋಸಿಹೋಗಿ ನಾನು ಹೀಗೆಂದೆ: ‘ನೀನು ದಯವಿಟ್ಟು ಆ ಹಾಡನ್ನು ಸಾಕುಮಾಡು. ಯಾರು ಆಗರ್ಭ ಶ‍್ರೀಮಂತರಾಗಿ ಹುಟ್ಟಿರುವರೋ, ಯಾರಿಗೆ ಮನೆಯಲ್ಲಿ ಉಪವಾಸದಿಂದ\break ನರಳುತ್ತಿರುವ ಬಂಧುಬಳಗಗಳಿಲ್ಲವೋ, ಅಂತಹವರಿಗೆ ಅದು ರುಚಿಸಬಹುದು. ಹೌದು, ನಾನು ಕೂಡ ನಿಮ್ಮಂತೆ ಯೋಚಿಸುತ್ತಿದ್ದ ಒಂದು ಕಾಲವಿತ್ತು. ಆದರೆ ಇಂತಹ ಪ್ರಪಂಚದ ನಗ್ನಸತ್ಯವನ್ನು ಅರಿತಿರುವಾಗ ಅವುಗಳೆಲ್ಲ ಹಾಸ್ಯಾಸ್ಪದವಾಗಿ ತೋರುವುದು.”

“ನನ್ನ ಸ್ನೇಹಿತರಿಗೆ ಇದರಿಂದ ತುಂಬಾ ವ್ಯಥೆಯಾಗಿರಬೇಕು. ಇಂತಹ ಮಾತನ್ನು ಆಡುವಂತೆ ಪ್ರೇರೇಪಿಸಿದ ನನ್ನ ದುಃಖ ಕಷ್ಟಗಳು ಅವನಿಗೆ ಹೇಗೆ ಅರ್ಥವಾಗಬೇಕು? ಕೆಲವು ವೇಳೆ ಮನೆಯಲ್ಲಿ ಸಾಕಾದಷ್ಟು ಊಟಕ್ಕೆ ಇಲ್ಲದೆ ಇರುವಾಗ, ನನ್ನ ಜೇಬಿನಲ್ಲಿ ಕಾಸಿಲ್ಲದೆ ಇದ್ದಾಗ, ನಾನು ನನ್ನ ತಾಯಿಗೆ ಹೊರಗೆ ಯಾರೊ ನನ್ನನ್ನು ಊಟಕ್ಕೆ ಕರೆದಿರುವರು ಎಂದು ಹೇಳಿ ಊಟವಿಲ್ಲದೆ ಇರುತ್ತಿದ್ದೆ. ಆತ್ಮಗೌರವದಿಂದ ನಾನು ಇದನ್ನು ಇತರರಿಗೆ ಹೇಳಿಕೊಳ್ಳುತ್ತಿರಲಿಲ್ಲ. ನನ್ನ ಶ‍್ರೀಮಂತ ಸ್ನೇಹಿತರು ಕೆಲವು ವೇಳೆ ನನ್ನನ್ನು ಹಾಡುವುದಕ್ಕೆ ಅವರ ಮನೆಗೆ ಕರೆಯುತ್ತಿದ್ದರು. ತಪ್ಪಿಸಿಕೊಳ್ಳುವುದಕ್ಕೆ ಸಾಧ್ಯವಿಲ್ಲದೆ ನಾನು ಒಪ್ಪಿಕೊಳ್ಳಬೇಕಾಗುತ್ತಿತ್ತು. ನನ್ನ ಗೋಳನ್ನು ಅವರಿಗೆ ಹೇಳಿಕೊಳ್ಳುವುದಕ್ಕೆ ನನಗೆ ಇಚ್ಛೆಯಾಗಲಿಲ್ಲ. ಅವರು ತಾವೇ ಸ್ವಪ್ರಯತ್ನದಿಂದ ನಮ್ಮ ಮನೆಯ ದುಃಸ್ಥಿತಿಯನ್ನು ಅರಿಯಲು ಯತ್ನಿಸಲೂ ಇಲ್ಲ. ಎಲ್ಲೋ ಕೆಲವರು ‘ನೀನು ಏತಕ್ಕೆ ಇವತ್ತು ಇಷ್ಟು ದುರ್ಬಲನಾಗಿರುವೆ, ನಿಸ್ತೇಜನಾಗಿರುವೆ’, ಎಂದು ಕೇಳುತ್ತಿದ್ದರು. ಅದರಲ್ಲಿ ಒಬ್ಬ ನಾನು ಹೇಳದೇ ಹೋದರೂ ಹೇಗೊ ನನ್ನ ಕಷ್ಟವನ್ನು ತಿಳಿದು, ಅನಾಮಧೇಯನಂತೆ ಮಧ್ಯೆ ಮಧ್ಯೆ ನನ್ನ ತಾಯಿಯ ಹೆಸರಿಗೆ ಸ್ವಲ್ಪ ದುಡ್ಡನ್ನು ಕಳುಹಿಸುತ್ತಿದ್ದ. ಅವನ ದಯೆಯಿಂದ ನನ್ನನ್ನು ಅವನಿಗೆ ಚಿರಕೃತಜ್ಞನಾಗಿರುವಂತೆ ಮಾಡಿರುವನು.

“ನನ್ನ ಕೆಲವು ಸ್ನೇಹಿತರು ಅಯೋಗ್ಯವಾದ ರೀತಿಯಲ್ಲಿ ಜೀವನೋಪಾಯವನ್ನು ಮಾಡುತ್ತಿದ್ದರು, ನನ್ನನ್ನು ಕೂಡ ಹಾಗೆಯೇ ಮಾಡು ಎಂದರು. ಕೆಲವರು ನನ್ನಂತೆ ದುರದೃಷ್ಟಕ್ಕೆ ವಶರಾಗಿ ವಿಧಿಯಿಲ್ಲದೆ ಇಂತಹ ಜೀವನಕ್ಕೆ ಕೈಹಾಕಬೇಕಾದವರು ನಿಜವಾಗಿ ನನಗೆ ಸಹಾನುಭೂತಿಯನ್ನು ತೋರಿದರು. ಇನ್ನೂ ಇತರ ತೊಂದರೆಗಳು ಕೂಡ ಬಂದವು, ಹಲವಾರು ಪ್ರಲೋಭನೆಗಳು ನನ್ನ ಮುಂದೆ ಬಂದವು. ಒಬ್ಬ ಶ‍್ರೀಮಂತ ಹೆಂಗಸು ಅಯೋಗ್ಯವಾದ ಸಲಹೆಯನ್ನು ನೀಡಿದಳು. ಅದರಂತೆ ನಡೆಯುವ ಹಾಗಿದ್ದರೆ ನನ್ನ ಬಡತನವನ್ನು ಕೊನೆಗಾಣಿಸುವೆ ಎಂದು ಹೇಳಿದಳು. ಅದನ್ನು ತುಚ್ಛೀಕರಿಸಿ ತಿರಸ್ಕರಿಸಿದೆ. ಮತ್ತೊಬ್ಬ ಹೆಂಗಸು ಕೂಡ ಇಂತಹ ಸಲಹೆಯನ್ನೇ ಕೊಟ್ಟಳು. ಅದಕ್ಕೆ ನಾನು ಅವಳಿಗೆ ಹೀಗೆ ಹೇಳಿದೆ: ‘ನೀನೂ ದೇಹ ಸೌಖ್ಯವನ್ನು ಅನುಸರಿಸಿ ನಿನ್ನ ಜೀವನವನ್ನು ವ್ಯರ್ಥಮಾಡಿಕೊಂಡಿರುವೆ. ಮೃತ್ಯುವಿನ ಕಾರ್ಮೋಡ ನಿನ್ನ ಮುಂದೆ ಇದೆ. ಅದನ್ನು ಎದುರಿಸುವುದಕ್ಕೆ ನೀನು ಏನನ್ನಾದರೂ ಮಾಡಿರುವೆಯಾ? ಇಂತಹ ಅವಹೇಳನಾಸ್ಪದವಾದ ಆಸೆಯನ್ನು ತ್ಯಜಿಸಿ ಭಗವಂತನನ್ನು ಚಿಂತಿಸು.’

“ಇಷ್ಟೊಂದು ಕಷ್ಟಗಳಿಗೆ ಈಡಾದರೂ, ಈಶ್ವರನಲ್ಲಿ ನಂಬಿಕೆ ಮತ್ತು ಅವನ ದಯೆಯಲ್ಲಿ ನನ್ನ ನಂಬಿಕೆ ಜಾರಲಿಲ್ಲ. ಪ್ರತಿದಿನವೂ ಅವನ ಹೆಸರನ್ನು ಸ್ಮರಿಸುತ್ತ ಏಳುತ್ತಿದ್ದೆ. ಅನಂತರ ಚಾಕರಿ ಶಿಕಾರಿಗೆ ಹೋಗುತ್ತಿದ್ದೆ. ಒಂದು ದಿನ ದೇವರನ್ನು ಪ್ರಾರ್ಥಿಸುತ್ತಿದ್ದುದು ನನ್ನ ತಾಯಿಯ ಕಿವಿಗೆ ಬಿತ್ತು. ಅವಳು ‘ಮಗು, ಸಾಕು ನಿಲ್ಲಿಸು. ನೀನು ಬಾಲ್ಯಾರಭ್ಯ ಗಂಟಲು ಕಿತ್ತು ಹೋಗುವವರೆಗೆ ದೇವರನ್ನು ಪ್ರಾರ್ಥಿಸಿರುವೆ. ನೋಡು, ಅವನು ನಿನಗೆ ಏನು ಮಾಡಿರುವನು?’ ಎಂದು ಹೇಳಿದಳು. ನನಗೆ ಮರ್ಮಾಂತಕ ಯಾತನೆ ಆಯಿತು. ಮನಸ್ಸಿನಲ್ಲಿ ಸಂದೇಹ ಮೂಡಿತು. ದೇವರು ನಿಜವಾಗಿ ಇರುವನೆ? ಇದ್ದರೆ ಮನುಷ್ಯ ವ್ಯಾಕುಲತೆಯಿಂದ ಮಾಡುವ ಪ್ರಾರ್ಥನೆಯನ್ನು ಕೇಳುವನೆ! ಕೇಳಿದರೆ ನನ್ನ ಹೃತ್ಪೂರ್ವಕ ಪ್ರಾರ್ಥನೆಗೆ ಏಕೆ ಅವನು ನಿರ್ದಯನಾಗಿರುವನು? ಅವನ ಕೃಪಾಶ್ರಯದಲ್ಲಿ ಇರುವ ಈ ಪ್ರಪಂಚದಲ್ಲಿ ಇಷ್ಟೊಂದು ದುಃಖ ಏತಕ್ಕೆ ಇದೆ? ದಯಾಮಯನಾದ ತಂದೆಯ ಬದಲು ಕ್ರೂರನಾದ ಸೈತಾನ್ ಏತಕ್ಕೆ ಆಳುತ್ತಿರುವನು? ‘ದೇವರು ಒಳ್ಳೆಯವನು ಮತ್ತು ಅನಂತ ದಯಾಸಿಂಧುವೂ ಆಗಿದ್ದರೆ, ಬರಗಾಲಗಳಲ್ಲಿ ಒಂದು ತುತ್ತು ಕೂಳಿಲ್ಲದೆ ಲಕ್ಷಾಂತರ ಜನರು ಏತಕ್ಕೆ ಸಾಯುತ್ತಾರೆ?’ ಎಂಬ ಈಶ್ವರಚಂದ್ರ ವಿದ್ಯಾಸಾಗರನ ನುಡಿ ನನಗೆ ಜ್ಞಾಪಕಕ್ಕೆ ಬಂದು ನನ್ನ ಹೃದಯಗಹ್ವರದಲ್ಲಿ ಅದು ನಿಷ್ಠುರವಾಗಿ ಪ್ರತಿಧ್ವನಿತವಾಗುತ್ತಿತ್ತು.

“ನಾನು ಯಾವುದನ್ನೂ ಗೋಪ್ಯವಾಗಿ ಇಡಲಾರೆ. ಅದು ನನ್ನ ಸ್ವಭಾವಕ್ಕೆ ವಿರುದ್ಧ. ಬಾಲ್ಯದಿಂದಲೂ ನನ್ನ ಮನಸ್ಸಿನಲ್ಲಿರುವ ಭಾವನೆಗಳನ್ನು ಇತರರಿಗೆ ಅಂಜಿ ಹೇಳದೆ ಸುಮ್ಮನಿರುತ್ತಿರಲಿಲ್ಲ. ಈಗ ದೇವರು ಒಂದು ಭ್ರಾಂತಿ, ಒಂದು ವೇಳೆ ಅವನಿದ್ದರೂ ಅವನನ್ನು ಪ್ರಾರ್ಥನೆ ಮಾಡಿದರೆ ಪ್ರಯೋಜನ ಇಲ್ಲವೆನ್ನುವುದನ್ನು ಸಪ್ರಮಾಣವಾಗಿ ಪ್ರಪಂಚಕ್ಕೆ ತೋರಲು ಯತ್ನಿಸಿದೆ. ನಾನೊಬ್ಬ ನಾಸ್ತಿಕನಾಗಿ ಹೋಗಿರುವೆನೆಂದೂ, ಸುರಾಪಾನಾದಿಗಳನ್ನು ಮಾಡುತ್ತಿರುವೆನೆಂದೂ, ವೇಶ್ಯಾ ಸ್ತ್ರೀಯರ ಮನೆಗೆ ಹೋಗುತ್ತಿರುವೆನೆಂದೂ ಸುಳ್ಳು ಸುದ್ದಿ ಹರಡಿತು. ಬುಡವಿಲ್ಲದ ಅಪಪ್ರಚಾರ ನನ್ನ ಮನಸ್ಸನ್ನು ಮತ್ತಷ್ಟು ಗಟ್ಟಿ ಮಾಡಿತು. ನಾನು ಬಹಿರಂಗವಾಗಿಯೇ ಈ ದುಃಖ ಪ್ರಪಂಚದಲ್ಲಿ ಏನಾದರೂ ಒಂದು ಸ್ವಲ್ಪ ಸುಖ ಸಿಕ್ಕಿದರೆ ಅದನ್ನು ಅನುಭವಿಸುವುದರಲ್ಲಿ ಯಾವ ಪಾಪವೂ ಇಲ್ಲವೆಂದು ಸಾರಿಬಿಟ್ಟೆ. ಇದು ಮಾತ್ರ ಸರಿಯಾದ ಮಾರ್ಗ ಎಂದು ನನಗೆ ನಿಸ್ಸಂದೇಹವಾಗಿ ತೋರಿದ್ದರೆ, ಯಾರಾದರೂ ಏನಾದರೂ ಹೇಳಿಕೊಳ್ಳಲಿ ಅಂಜಿಕೆಯಿಂದ ಆ ಮಾರ್ಗದಲ್ಲಿ ಹೋಗುವುದನ್ನು\break ಬಿಡುತ್ತಿರಲಿಲ್ಲ.

“ನನ್ನ ವಿಷಯದಲ್ಲಿ ಇಲ್ಲ ಸಲ್ಲದ ಸುದ್ದಿಗಳೆಲ್ಲ ದಕ್ಷಿಣೇಶ್ವರದಲ್ಲಿ ಶ‍್ರೀರಾಮಕೃಷ್ಣರಿಗೂ, ಕಲ್ಕತ್ತೆಯಲ್ಲಿರುವ ಅವರ ಭಕ್ತರಿಗೂ ತಲುಪಿತು. ಕೆಲವರು ಸಮಾಚಾರವೇನು ಎಂಬುದನ್ನು ತಿಳಿದುಕೊಳ್ಳುವುದಕ್ಕೆ ನನ್ನ ಹತ್ತಿರ ಬಂದರು. ಕೆಲವರು ಅಪಪ್ರಚಾರಗಳಲ್ಲಿ ತಮಗೆ ನಂಬಿಕೆ ಇರುವುದಾಗಿಯೂ ಹೇಳಿದರು. ಈ ಸಮಾಚಾರವನ್ನು ಕೇಳಿದಾಗ, ನಾನು ಇಷ್ಟು ಅಧೋಗತಿಗೆ ಇಳಿದುಹೋಗಿರುವೆನು ಎಂದು ಇವರು ನಂಬುತ್ತಾರಲ್ಲ ಎಂದು ನನಗೆ ಬಹಳ ವ್ಯಥೆಯಾಯಿತು. ಸುಮ್ಮನೆ ನರಕದ ಭೀತಿಯಿಂದ ದೇವರನ್ನು ನಂಬುವುದು ಹೇಡಿತನವೆಂದು ವಾದಿಸಿ ಅದನ್ನು ಸಮರ್ಥಿಸುವುದಕ್ಕೆ ಹಲವು ಪಾಶ್ಚಾತ್ಯ ದಾರ್ಶನಿಕ ಭಾವನೆಗಳನ್ನು ಉದಾಹರಿಸಿದೆ. ಇದನ್ನು ಕೇಳಿದಾಗ, ಇವನು ತುಂಬಾ ಹಾಳಾಗಿ ಹೋಗಿರುವನು ಎಂದು ನಿರ್ಧಾರಕ್ಕೆ ಬಂದು ಹೊರಟುಹೋದರು. ಅವರು ಹೋದದ್ದು ನನಗೆ ಸಂತೊಷವೇ ಆಯಿತು. ಇದು ಶ‍್ರೀರಾಮಕೃಷ್ಣರಿಗೂ ಗೊತ್ತಾಗಬಹುದೆಂದು ಭಾವಿಸಿದೆ. ಇದನ್ನು ಯೋಚಿಸಿದಾಗ ‘ಚಿಂತೆಯಿಲ್ಲ, ಒಬ್ಬ ನನ್ನ ವಿಷಯದಲ್ಲಿ ಇಟ್ಟುಕೊಳ್ಳುವ ಒಳ್ಳೆಯ ಅಥವಾ ಕೆಟ್ಟ ಅಭಿಪ್ರಾಯಗಳು, ಇಂತಹ ಬುಡವಿಲ್ಲದ ಅಪಪ್ರಚಾರದ ಮೇಲೆ ನಿಂತರೆ ನಾನೇನು ಮಾಡುವುದು?’ ಎಂದುಕೊಂಡೆ. ಆದರೆ ಶ‍್ರೀರಾಮಕೃಷ್ಣರು ಇದನ್ನು ಹೇಗೆ ಸ್ವೀಕರಿಸಿದರು ಎಂಬುದನ್ನು ಕೇಳಿ ನನಗೆ ಆಶ್ಚರ್ಯವಾಯಿತು. ಮೊದಲು ಅವರು ಇದರ ಕಡೆಗೆ ಗಮನವನ್ನೇ ಕೊಡಲಿಲ್ಲ, ಪರವಾಗಿ ಅಥವಾ ವಿರೋಧವಾಗಿ ಯಾವ ಅಭಿಪ್ರಾಯವನ್ನೂ ಕೊಡಲಿಲ್ಲ. ಭವನಾಥನೆಂಬ ಅವರ ಪ್ರಿಯ ಶಿಷ್ಯನೊಬ್ಬ ಕಂಬನಿ ಸುರಿಸುತ್ತ ‘ನರೇಂದ್ರ ಇಷ್ಟು ಕೆಳಗೆ ಇಳಿಯುವನು ಎಂದು ನಾನು ಕನಸಿನಲ್ಲಿಯೂ ಊಹಿಸಲಾರದವನಾಗಿದ್ದೆ.’ ಎಂದ. ಶ‍್ರೀರಾಮಕೃಷ್ಣರಿಗೆ ಇದನ್ನು ಕೇಳಿ ಕೋಪ ಬಂತು. ‘ಮೂರ್ಖ, ಸುಮ್ಮನಿರು! ತಾಯಿ ನನಗೆ ತೋರಿರುವಳು, ಇದು ಸತ್ಯವಲ್ಲ ಎಂದು. ನೀನೇನಾದರೂ ಮತ್ತೊಮ್ಮೆ ಹೀಗೆ ಮಾತನಾಡಿದರೆ ನಿನ್ನನ್ನು ಕತ್ತೆತ್ತಿಯೂ ನೋಡಲಾರೆ’ ಎಂದರು ಶ‍್ರೀರಾಮಕೃಷ್ಣರು

“ಬಲಾತ್ಕಾರವಾಗಿ ನಾಸ್ತಿಕ ಭಾವಗಳನ್ನು ನಾನು ಎಷ್ಟು ಸ್ವಾಗತಿಸಿದರೂ, ಬಾಲ್ಯಾರಭ್ಯ ನನಗೆ ಆದ ಆಧ್ಯಾತ್ಮಿಕ ಅನುಭವಗಳು, ಅದಕ್ಕಿಂತ ಹೆಚ್ಚಾಗಿ ಶ‍್ರೀರಾಮಕೃಷ್ಣರ ಪರಿಚಯವಾದಮೇಲೆ ನನ್ನ ಮನಸ್ಸಿನ ಮೇಲೆ ಆದ ಪರಿಣಾಮಗಳು, ದೇವರು ಇರಲೇಬೇಕು, ಅವನನ್ನು ಸಾಕ್ಷಾತ್ಕಾರಮಾಡಿಕೊಳ್ಳಲು ಯಾವುದಾದರೂ ಒಂದು ಮಾರ್ಗವಿರಬೇಕೆಂಬ ನಂಬಿಕೆಯನ್ನು ದೃಢೀಕರಿಸಿದ್ದವು. ಇಲ್ಲದೆ ಇದ್ದರೆ ಜೀವನ ನಿಸ್ಸಾರವಾಗುವುದು. ಈ ಕಷ್ಟಕಾರ್ಪಣ್ಯಗಳ ಮಧ್ಯದಲ್ಲಿ ನಾನು ಭಗವತ್ ಸಾಕ್ಷಾತ್ಕಾರಕ್ಕೆ ಒಂದು ಮಾರ್ಗವನ್ನು ಕಂಡು ಹಿಡಿಯಲೇಬೇಕು ಎಂದೆನಿಸಿತು. ಹಲವು ದಿನಗಳು ಹೀಗೆ ಕಳೆದವು. ಮನಸ್ಸು ದೇವರು ಇರುವನು, ಇಲ್ಲ ಎಂಬುದರ ಮಧ್ಯದಲ್ಲಿ ತೂಗುಯ್ಯಾಲೆಯಂತೆ ತೂಗಾಡುತ್ತಿತ್ತು. ನನ್ನ ಕಾರ್ಪಣ್ಯವಾದರೊ ಹಾಗೆ ಇತ್ತು.

\vskip  3pt

“ಬಿಸಿಲ ಕಾಲ ಆಯಿತು, ಮಳೆಗಾಲ ಕಾಲಿಟ್ಟಿತು. ಕೆಲಸವನ್ನು ಹುಡುಕುವುದು ಹಾಗೆಯೇ ಮುಂದುವರಿಯಿತು. ಒಂದು ದಿನ ಸಾಯಂಕಾಲ ದಿನವೆಲ್ಲಾ ಉಪವಾಸ ಮಾಡಿ ಮಳೆಯಲ್ಲಿ ನೆನೆದು, ಸೋತ ಕಾಲುಗಳು ಮತ್ತು ದುರ್ಬಲವಾದ ಮನಸ್ಸಿನಿಂದ ಮನೆಗೆ ಹಿಂತಿರುಗುತ್ತಿದ್ದೆ. ನನಗೆ ಸಾಕಾಗಿ ಮುಂದೆ ಒಂದು ಹೆಜ್ಜೆಯನ್ನೂ ಇಡಲು ಸಾಧ್ಯವಿಲ್ಲದೆ, ರಸ್ತೆಯ ಪಕ್ಕದಲ್ಲಿ ಒಂದು ಮನೆಯ ಮುಂದೆ ಬಿದ್ದುಬಿಟ್ಟೆ. ತತ್ಕಾಲದಲ್ಲಿ ನನಗೆ ಪ್ರಜ್ಞೆ ಇತ್ತೊ ಇಲ್ಲವೊ ಗೊತ್ತಿರಲಿಲ್ಲ. ಮನಸ್ಸಿನಲ್ಲಿ ಹಲವು ಯೋಚನೆಗಳು ಹೊಳೆದವು. ಅವುಗಳನ್ನು ಹೊರದೂಡಿ ಯಾವುದೋ ಒಂದರ ಮೇಲೆ ಕೇಂದ್ರ ಮಾಡಲು ಪ್ರಯತ್ನಿಸಿದೆ. ಮನಸ್ಸು ತುಂಬಾ ನಿತ್ರಾಣವಾಗಿತ್ತು. ಇದ್ದಕ್ಕಿದ್ದಂತೆಯೇ ಯಾವುದೋ ಒಂದು ದಿವ್ಯಶಕ್ತಿ, ನನ್ನ ಮನಸ್ಸನ್ನು ಆವರಿಸಿದ್ದ ತೆರೆಯನ್ನು ಒಂದಾದ ಮೇಲೆ ಒಂದರಂತೆ ತೆಗೆದುಬಿಟ್ಟಿತು. ಭಗವಂತನ ನ್ಯಾಯ ಮತ್ತು ಅವನ ಕೃಪೆಯೊಡನೆ ಈ ಜಗತ್ತಿನಲ್ಲಿ ಹೇಗೆ ಇಷ್ಟೊಂದು ದುಃಖವಿರಬಲ್ಲದು ಎಂಬ ಸಂಶಯ ತಾನಾಗಿಯೇ ಬಗೆಹರಿಯಿತು. ಗಾಢ ಆಲೋಚನಾಮಗ್ನನಾಗಿ ಅವುಗಳ ಅರ್ಥವನ್ನೆಲ್ಲ ಗ್ರಹಿಸಿ ತೃಪ್ತನಾದೆ. ನಾನು ಅನಂತರ ಮನೆಯ ಕಡೆಗೆ ನಡೆದುಕೊಂಡು ಹೋದಾಗ ದೇಹದಲ್ಲಿ ಯಾವ ದೌರ್ಬಲ್ಯವೂ ಇದ್ದಂತೆ ಕಾಣಲಿಲ್ಲ, ಮನಸ್ಸಾದರೋ ಅದ್ಭುತವಾದ ಶಾಂತಿ ಮತ್ತು ಶಕ್ತಿಯಿಂದ ದೃಢವಾಗಿತ್ತು. ಅಂದಿನ ಆ ರಾತ್ರಿ ಕಳೆಯಿತು.

\vskip  3pt

“ಅಂದಿನಿಂದ ಜನರ ನಿಂದೆ ಮತ್ತು ಸ್ತುತಿಯ ಕಡೆ ಗಮನವನ್ನು ಕೊಡುತ್ತಿರಲಿಲ್ಲ. ಸಾಧಾರಣ ಮನುಷ್ಯರಂತೆ ಹಣ ಸಂಪಾದನೆ ಮಾಡಿ ಸಂಸಾರವನ್ನು ನಿರ್ವಹಿಸಿಕೊಂಡು ಹೋಗುವುದಕ್ಕಾಗಲಿ, ವಿಷಯ ಸುಖವನ್ನು ಪಡೆಯುವುದಕ್ಕಾಗಲಿ ನಾನು ಹುಟ್ಟಿಲ್ಲ ಎಂದು ತಿಳಿದೆ. ನಾನು ನನ್ನ ಅಜ್ಜನಂತೆ ಸಂಸಾರವನ್ನು ತ್ಯಜಿಸುವುದಕ್ಕೆ ರಹಸ್ಯವಾಗಿ ಅಣಿಯಾಗುತ್ತಿದ್ದೆ. ಅದಕ್ಕಾಗಿ ನಾನು ದಿನವನ್ನು ಗೊತ್ತು ಮಾಡಿದೆ. ಅಂದೇ ಶ‍್ರೀರಾಮಕೃಷ್ಣರು ಕಲ್ಕತ್ತೆಗೆ ಬರುವರು ಎಂಬುದನ್ನು ಕೇಳಿ ಸಂತೋಷವಾಯಿತು. ಇದು ನನ್ನ ಅದೃಷ್ಟ, ನನ್ನ ಗುರುವಿನ ಆಶೀರ್ವಾದ ಪಡೆದು ನಾನು ಸಂಸಾರವನ್ನು ತ್ಯಜಿಸುತ್ತೇನೆ ಎಂದುಕೊಂಡೆ. ನಾನು ಶ‍್ರೀಗುರುದೇವನನ್ನು ಕಂಡೊಡನೆಯೆ ಅವರು ನನ್ನನ್ನು ತಮ್ಮೊಡನೆ ದಕ್ಷಿಣೇಶ್ವರಕ್ಕೆ ಬಂದು ಅಂದಿನ ರಾತ್ರಿಯನ್ನು ಕಳೆಯಬೇಕೆಂದು ಒತ್ತಾಯಮಾಡಿದರು. ನಾನು ಬರುವುದಕ್ಕೆ ಆಗುವುದಿಲ್ಲವೆಂದು ಏನೇನೊ ಕಾರಣಗಳನ್ನು ಕೊಟ್ಟೆ. ಆದರೆ ಅದರಿಂದ ಏನೂ ಪ್ರಯೋಜನವಾಗಲಿಲ್ಲ. ನಾನು ಅವರೊಡನೆ ಹೋಗಲೇಬೇಕಾಗಿ ಬಂತು. ನಾನು ದಕ್ಷಿಣೇಶ್ವರವನ್ನು ಸೇರಿದಮೇಲೆ ಇತರರೊಡನೆ ಅವರ ಕೋಣೆಯಲ್ಲಿ ಕುಳಿತಿದ್ದೆ. ಆಗ ಶ‍್ರೀರಾಮಕೃಷ್ಣರು ಸಮಾಧಿಸ್ಥರಾದರು. ಅವರು ಸ್ವಲ್ಪ ಹೊತ್ತಿನ ಮೇಲೆ ನನ್ನ ಸಮೀಪಕ್ಕೆ ಬಂದು ನನ್ನನ್ನು ಬಹಳ ಪ್ರೀತಿಯಿಂದ ಮೈದಡವಿದರು. ಕಣ್ಣಿನಲ್ಲಿ ನೀರನ್ನು ಸುರಿಸುತ್ತ ಅವರು ಒಂದು ಹಾಡನ್ನು ಹೇಳಿದರು. ಇದುವರೆಗೆ ನಾನು ನನ್ನ ಮನಸ್ಸಿನ ಭಾವವನ್ನು ನಿಗ್ರಹಿಸಿದ್ದೆ. ಆದರೆ ಇನ್ನುಮೇಲೆ ಸಾಧ್ಯವಾಗಲಿಲ್ಲ. ಧಾರಾಕಾರವಾಗಿ ನನ್ನ ಕಣ್ಣಿನಲ್ಲಿ ನೀರು ಸುರಿಯಿತು. ಆ ಹಾಡಿನ ಅರ್ಥ ಸರಳವಾಗಿಯೇ ಇತ್ತು. ಅವರು ನನ್ನ ಮನಸ್ಸಿನಲ್ಲಿದ್ದ ಭಾವನೆಯನ್ನು ಗ್ರಹಿಸಿದರು. ಕುಳಿತವರು ನಮ್ಮಿಬ್ಬರ ಮಧ್ಯೆ ಆಗುತ್ತಿದ್ದ ಮಾನಸಿಕ ಸಂಭಾಷಣೆಯನ್ನು ಕುರಿತು ವಿಸ್ಮಿತರಾದರು. ಶ‍್ರೀರಾಮಕೃಷ್ಣರು ಪೂರ್ಣವಾಗಿ ಪ್ರಕೃತಿಸ್ಥರಾದಮೇಲೆ, ಕೆಲವರು ಶ‍್ರೀರಾಮಕೃಷ್ಣರ ಆಶ್ಚರ‍್ಯಕರವಾದ ನಡತೆಗೆ ಕಾರಣವನ್ನು ಕೇಳಿದರು. ಅದಕ್ಕೆ ಶ‍್ರೀರಾಮಕೃಷ್ಣರು ನಗುತ್ತ ‘ಓ ಅದು ನನಗೂ ಅವನಿಗೂ ಸಂಬಂಧಪಟ್ಟದ್ದು,’ ಎಂದು ಹೇಳಿ ಸುಮ್ಮನಾಗಿಬಿಟ್ಟರು. ಅಂದಿನ ರಾತ್ರಿ ಅವರು ಇತರರನ್ನೆಲ್ಲ ತಮ್ಮ ಕೋಣೆಯಿಂದ ಹೊರಗೆ ಕಳಿಸಿ, ನನ್ನನ್ನು ಒಳಗೆ ಕರೆದರು. ಕಣ್ಣಿನಲ್ಲಿ ನೀರು ಸುರಿಸುತ್ತ ಅವರು ಹೀಗೆ ಹೇಳಿದರು: ‘ನೀನು ಜಗನ್ಮಾತೆಯ ಕೆಲಸಕ್ಕಾಗಿ ಬಂದಿರುವೆ ಎನ್ನುವುದು ನನಗೆ ಗೊತ್ತು. ನೀನು ಸಂಸಾರದಲ್ಲಿ ಇರಲಾರೆ, ಆದರೆ ನನಗೋಸುಗ, ನಾನು ಇರುವ ಪರಿಯಂತ ನೀನು ಸಂಸಾರದಲ್ಲಿರು.’ ಮಾರನೆ ದಿನ ಅವರ ಅಪ್ಪಣೆಯನ್ನು ಪಡೆದು ಮನೆಗೆ\break ಹಿಂದಿರುಗಿದೆ. ಮನೆಯವರನ್ನು ನೋಡಿಕೊಳ್ಳುವುದು ಹೇಗೆ ಎಂದು ನೂರಾರು ಯೋಚನೆಗಳು ನನ್ನನ್ನು ಬಾಧಿಸಿದವು. ಜೀವನೋಪಾಯಕ್ಕೆ ನಾನೊಂದು ಕೆಲಸವನ್ನು ಹುಡುಕಿದೆ. ಒಬ್ಬ ಲಾಯರನ ಆಫೀಸಿನಲ್ಲಿ ಕೆಲಸ ಮಾಡುವುದು, ಕೆಲವು ಪುಸ್ತಕಗಳನ್ನು ಭಾಷಾಂತರಿಸುವುದು, ಇವುಗಳಿಂದ ಹೇಗೊ ಉಪವಾಸದಿಂದ ಪಾರಾಗಲು ಸ್ವಲ್ಪ ಹಣವನ್ನು ಪಡೆದೆ. ಆದರೆ ಇದು ಸ್ಥಿರವಾದ ಕೆಲಸವಾಗಿರಲಿಲ್ಲ. ನನ್ನ ತಾಯಿ ಸಹೋದರ ಸಹೋದರಿಯನ್ನು ನೋಡಿಕೊಳ್ಳಲು ಒಂದು ನಿಗದಿಯಾದ ವರಮಾನವಿರಲಿಲ್ಲ.

\vskip  3pt

“ಒಂದು ದಿನ ನನಗೆ ಈ ಭಾವನೆ ಹೊಳೆಯಿತು. ದೇವರು ಶ‍್ರೀರಾಮಕೃಷ್ಣರ ಪ್ರಾರ್ಥನೆಯನ್ನು ಕೇಳುವನು. ನನ್ನ ಬಡತನವನ್ನು ಹೋಗಲಾಡಿಸಿ ಎಂದು\break ಶ‍್ರೀರಾಮಕೃಷ್ಣರನ್ನು ನನ್ನ ಪರವಾಗಿ ದೇವರಿಗೆ ಪ್ರಾರ್ಥನೆ ಮಾಡಿ ಎಂದು ಏತಕ್ಕೆ ಕೇಳಬಾರದು? ನನ್ನ ಈ ಕೋರಿಕೆಯನ್ನು ಅವರು ಎಂದೂ ನಿರಾಕರಿಸುವವರಲ್ಲ. ನಾನು ದಕ್ಷಿಣೇಶ್ವರಕ್ಕೆ ಹೋಗಿ ನನ್ನ ಪರವಾಗಿ ದೇವಿಯನ್ನು ಪ್ರಾರ್ಥನೆ ಮಾಡಿ ಎಂದು ಅವರನ್ನು ಬೇಡಿಕೊಂಡೆ. ಅದಕ್ಕೆ ಅವರು ‘ಮಗು, ಅಂತಹ ಪ್ರಾರ್ಥನೆಯನ್ನು ಮಾಡಲು ನನ್ನಿಂದ ಸಾಧ್ಯವಿಲ್ಲ. ಆದರೆ ನೀನೇ ಹೋಗಿ ತಾಯಿಯನ್ನು ಏತಕ್ಕೆ ಕೇಳಬಾರದು? ನೀನು ಅವಳನ್ನು ಲೆಕ್ಕಿಸದೇ ಇರುವುದೇ ನಿನ್ನ ದುಃಖಕ್ಕೆ ಕಾರಣ’ ಎಂದರು. ನಾನು, ‘ನನಗೆ ತಾಯಿ ಗೊತ್ತಿಲ್ಲ. ನೀವು ನನ್ನ ಪರವಾಗಿ ಕೇಳಿಕೊಳ್ಳಿ. ನೀವು ದಯವಿಟ್ಟು ಹಾಗೆ ಮಾಡಬೇಕು’ ಎಂದೆ. ಅವರು ಪ್ರೀತಿಯಿಂದ ಹೀಗೆಂದರು: ‘ನನ್ನ ಪ್ರಿಯ ವತ್ಸ, ನಾನು ಹಾಗೆ ಪದೇ ಪದೇ ಮಾಡಿರುವೆನು. ಆದರೆ ನೀನು ಅವಳನ್ನು ಒಪ್ಪಿಕೊಳ್ಳುತ್ತಿಲ್ಲ. ಅದಕ್ಕಾಗಿ ಅವಳು ನನ್ನ ಪ್ರಾರ್ಥನೆಯನ್ನು ಈಡೇರಿಸುತ್ತಿಲ್ಲ. ಆಗಲಿ, ಇವತ್ತು ಮಂಗಳವಾರ. ಇಂದಿನ ರಾತ್ರಿ ಕಾಳಿಕಾ ಮಾತೆಯ ದೇವಸ್ಥಾನಕ್ಕೆ ಹೋಗು. ದೇವಿಗೆ ನಮಸ್ಕಾರ ಮಾಡಿ, ನಿನಗೆ ಏನು ಬೇಕೋ ಅದನ್ನು ಕೇಳಿಕೊ. ಅವಳು ಅದನ್ನು ಅನುಗ್ರಹಿಸುವಳು. ಅವಳೇ ಅಖಂಡ ಜ್ಞಾನ, ಬ್ರಹ್ಮನ ಮಾಯಾಶಕ್ತಿ. ಅವಳು ಕೇವಲ ತನ್ನ ಇಚ್ಛಾಮಾತ್ರದಿಂದ ಈ ಪ್ರಪಂಚವನ್ನು ಸೃಷ್ಟಿಸಿರುವಳು. ಯಾವುದನ್ನು ಬೇಕಾದರೂ ಅನುಗ್ರಹಿಸುವುದು ಅವಳ ಕೈಯಲ್ಲಿದೆ.’ ಅವರು ಹೇಳಿದ ಪ್ರತಿಯೊಂದು ಮಾತನ್ನೂ ನಂಬಿದೆ. ರಾತ್ರಿ ಆಗುವವರೆಗೂ ಕಾದೆ. ಒಂಭತ್ತು ಗಂಟೆಯ ಸಮಯದಲ್ಲಿ ದೇವಿಯ ಸನ್ನಿಧಿಗೆ ಹೋಗುವಂತೆ ಶ‍್ರೀಗುರುಗಳು ನನಗೆ ಆಜ್ಞಾಪಿಸಿದರು. ನಾನು ಹೋದಾಗ ನನ್ನ ಮನಸ್ಸು ಭಕ್ತಿಯಿಂದ ಉನ್ಮತ್ತವಾಗಿತ್ತು. ನಾನು ನನ್ನ ಕಾಲಮೇಲೆ ಸರಿಯಾಗಿ ನಡೆಯಲು ಆಗಲಿಲ್ಲ. ಚಿನ್ಮಯದೇವಿಯನ್ನು ನೋಡುವುದಕ್ಕೆ, ಅವಳ ಮಾತುಗಳನ್ನು ಕೇಳುವುದಕ್ಕೆ, ನನ್ನ ಹೃದಯ ಕುದಿಯುತ್ತಿತ್ತು. ನಾನು ಈ ಭಾವನೆಯಿಂದಲೇ ತುಂಬಿಹೋಗಿದ್ದೆ. ಗರ್ಭಗುಡಿಯನ್ನು ಪ್ರವೇಶಿಸಿ ವಿಗ್ರಹದ ಕಡೆ ನೋಡಿದಾಗ ಅವಳು ಸಚೇತನಳಾಗಿರುವುದು ಕಂಡಿತು. ಅವಳೇ ಪರಮ ಪವಿತ್ರವಾದ ಪ್ರೇಮ ಮತ್ತು ಸೌಂದರ್ಯದ ಮೂಲಸ್ಥಾನವಾಗಿ ಕಂಡುಬಂದಳು. ಭಕ್ತಿ ಮತ್ತು ಪ್ರೀತಿಯ ಭಾವದಿಂದ ಓತಪ್ರೋತನಾದೆ, ಸಂತೋಷಾಧಿಕ್ಯದಿಂದ ದೇವಿಗೆ ಪುನಃ ಪುನಃ ದಂಡಪ್ರಣಾಮಗಳನ್ನು\break ಮಾಡಿದೆ. ‘ತಾಯಿ ನನಗೆ ತ್ಯಾಗ ಕೊಡು, ವೈರಾಗ್ಯ ಕೊಡು, ಎಂದೆಂದಿಗೂ ನಾನು ನಿನ್ನನ್ನು ನೋಡುತ್ತಿರುವಂತೆ ಅನುಗ್ರಹಿಸು’ ಎಂದು ಬೇಡಿದೆ. ಪರಮಶಾಂತಿ ನನ್ನ ಹೃದಯವನ್ನೆಲ್ಲ ಆವರಿಸಿತು. ಪ್ರಪಂಚ ಮರೆತೇ ಹೋಯಿತು. ದೇವಿಯೊಬ್ಬಳೇ ನನ್ನ ಹೃದಯದಲ್ಲಿ ರಾರಾಜಿಸುತ್ತಿದ್ದಳು.

“ನಾನು ಹಿಂತಿರುಗಿ ಬಂದೊಡನೆಯೇ ಶ‍್ರೀಗುರುಗಳು ‘ಬಡತನದಿಂದ ಪಾರು ಮಾಡಬೇಕೆಂದು ದೇವಿಯನ್ನು ಕೇಳಿಕೊಂಡೆಯಾ?’ ಎಂದು ಕೇಳಿದರು. ‘ಇಲ್ಲ ಗುರುಗಳೇ. ನಾನು ಅದನ್ನೆಲ್ಲ ಮರೆತೇ ಬಿಟ್ಟೆ. ಈಗ ಅದಕ್ಕೆ ಏನಾದರೂ ಮಾರ್ಗವಿದೆಯೆ?’ ಎಂದು ಕೇಳಿದೆ. ‘ನೀನು ಪುನಃ ಹೋಗಿ ನಿನ್ನ ಕೋರಿಕೆಯನ್ನು ಮುಂದಿಡು’ ಎಂದರು. ನಾನು ಪುನಃ ಗರ್ಭಗುಡಿಯ ಮುಂದೆ ಹೋದೆ. ದೇವಿಯನ್ನು ನೋಡಿದೊಡನೆಯೇ ಎಲ್ಲಾ ಮರೆತುಹೋಯಿತು. ಅವಳಿಗೆ ಪದೇ ಪದೇ ನಮಸ್ಕರಿಸಿ, ಜ್ಞಾನ, ಭಕ್ತಿ, ವೈರಾಗ್ಯವನ್ನು ಮಾತ್ರ ಪ್ರಾರ್ಥಿಸಿದೆ. ಶ‍್ರೀಗುರುಗಳು ಎರಡನೇ ವೇಳೆ ‘ಏನು ನೀನು ಕೇಳಿದೆಯಾ?’ ಎಂದರು. ‘ಈ ಸಲವೂ ನಾನು ಮರೆತುಬಿಟ್ಟೆ’ ಎಂದೆ. ಅದಕ್ಕೆ ಅವರು ‘ಎಂತಹ ಮರೆವು ನಿನ್ನದು. ಆ ಕೆಲವು ಪದಗಳನ್ನು ಜ್ಞಾಪಕದಲ್ಲಿ ಇಟ್ಟುಕೊಳ್ಳುವುದು ನಿನಗೆ ಸಾಧ್ಯವಿಲ್ಲವೆ? ಆಗಲಿ, ಇನ್ನೊಂದು ಸಲ ಹೋಗಿ ಅವಳನ್ನು ಕೇಳು. ಬೇಗ ಹೋಗು,’ ಎಂದರು. ನಾನು ಮೂರನೇ ವೇಳೆ ಹೋದೆ. ಆದರೆ ದೇವಿಯ ಮುಂದೆ ನಿಂತಾಗ ನನಗೆ ಸಹಿಸಲಸದಳವಾದ ಲಜ್ಜೆ ಉಂಟಾಯಿತು. ನಾನು ಆಲೋಚಿಸಿದೆ: ಅಯ್ಯೋ ಎಂತಹ ಕ್ಷುದ್ರ ವಸ್ತುವನ್ನು ಬೇಡುವುದಕ್ಕೆ ನಾನು ಬಂದಿರುವೆ? ಇದು ಉದಾರ ಹೃದಯದ ಒಬ್ಬ ಮಹಾರಾಜನನ್ನು ತನಗೆ ಸ್ವಲ್ಪ ತರಕಾರಿ ಕೊಡು ಎಂದು ಬೇಡಿದಂತೆ! ನಾನು ಎಂತಹ ಮೂರ್ಖ ಎಂದು ಭಾವಿಸಿದೆ. ನಾನು ಪಶ್ಚಾತ್ತಾಪ ಮತ್ತು ನಾಚಿಕೆಯಿಂದ ಕುಗ್ಗಿ ಹೋಗಿ ದೇವಿಗೆ ಭಕ್ತಿಯಿಂದ ಮಣಿದು, ‘ತಾಯಿ ನನಗೆ ಜ್ಞಾನ ಭಕ್ತಿ ವೈರಾಗ್ಯಗಳಲ್ಲದೇ ಬೇರೇನೂ ಬೇಡ’ ಎಂದೆ. ನಾನು ದೇವಸ್ಥಾನದಿಂದ ಹೊರಗೆ ಬಂದ ಮೇಲೆ, ಇದೆಲ್ಲ ಶ‍್ರೀಗುರುವಿನ ಇಚ್ಛೆಯಿಂದಾದುದು ಎಂದು ಗ್ರಹಿಸಿದೆ. ಇಲ್ಲದೆ ಇದ್ದರೆ ಮೂರು ಬಾರಿಯೂ ಇದನ್ನು ಕೇಳಲು ಹೇಗೆ ಮರೆಯುತ್ತಿದ್ದೆ? ನಾನು ಹಿಂತಿರುಗಿ ಬಂದ ಮೇಲೆ, ‘ಪೂಜ್ಯರೆ, ನೀವೇ ನನ್ನ ಮೇಲೆ ಒಂದು ಸಮ್ಮೋಹನಾಸ್ತ್ರವನ್ನು ಬಿಟ್ಟು ಮರೆಯುವಂತೆ ಮಾಡಿದಿರಿ. ಈಗ ನನ್ನ ಮನೆಯವರು ಬಡತನದಿಂದ ನರಳದಂತೆ ದಯವಿಟ್ಟು ವರವನ್ನು ಅನುಗ್ರಹಿಸಿ’ ಎಂದು ಕೇಳಿಕೊಂಡೆ. ಅದಕ್ಕೆ ಅವರು ‘ಅಂತಹ ಪ್ರಾರ್ಥನೆ ಎಂದಿಗೂ ನನ್ನಿಂದ ಸಾಧ್ಯವಿಲ್ಲ. ನೀನೇ ಬೇಡು ಎಂದು ನಾನು ನಿನಗೆ ಹೇಳಿದೆ. ಆದರೆ ನೀನೇ ಅದನ್ನು ಮಾಡದೇ ಹೋದೆ. ನೀನು ಲೌಕಿಕ ಸೌಖ್ಯಗಳನ್ನು ಅನುಭವಿಸುವುದು ನಿನ್ನ ಅದೃಷ್ಟದಲ್ಲಿ ಇಲ್ಲ ಎಂದು ತೋರುವುದು. ಆಗಲಿ ನಾನೇನು ಮಾಡಲು ಸಾಧ್ಯ?’ ಎಂದರು. ಆದರೆ ನಾನು ಅವರನ್ನು ಬಿಡಲಿಲ್ಲ. ನನ್ನ ಪ್ರಾರ್ಥನೆಯನ್ನು ಈಡೇರಿಸಬೇಕೆಂದು ಬಲಾತ್ಕಾರ ಮಾಡಿದೆ. ಕೊನೆಗೆ ಅವರು ಹೇಳಿದರು: ‘ಆಗಲಿ, ನಿನ್ನ ಮನೆಯವರಿಗೆ ಹೊಟ್ಟೆಗೆ ಬಟ್ಟೆಗೆ ಯಾವ ಕೊರತೆ ಇಲ್ಲದಿರಲಿ’ ಎಂದರು.”

\newpage

ನರೇಂದ್ರನ ಜೀವನದಲ್ಲಿ ಇದೊಂದು ಹೊಸ ಅನುಭವ. ಇಂದಿನಿಂದ ಹೊಸ ಅಧ್ಯಾಯ ಪ್ರಾರಂಭವಾಗಿದೆ ಎನ್ನಬಹುದು. ಇದುವರೆಗೆ ವಿಗ್ರಹಗಳನ್ನು ಅವನು ಒಪ್ಪಿಕೊಳ್ಳುತ್ತಿರಲಿಲ್ಲ. ಈಗ ಭಕ್ತಿಯಿಂದ ಬೇಡಿದರೆ ದೇವರು ವಿಗ್ರಹದ ಮೂಲಕವೂ ವ್ಯಕ್ತವಾಗುತ್ತಾನೆ ಎಂಬುದನ್ನು ಅನುಭವದ ಮೂಲಕ ತಿಳಿದುಕೊಂಡನು. ಶ‍್ರೀರಾಮಕೃಷ್ಣರಿಗೆ ನರೇಂದ್ರನ ಈ ಬದಲಾವಣೆಯನ್ನು ನೋಡಿ ಪರಮ ಸಂತೋಷವಾಯಿತು. ಮಾರನೆ ದಿನ ಬೆಳಗ್ಗೆ ವೈಕುಂಠನಾಥ ಸನ್ಯಾಲ ದಕ್ಷಿಣೇಶ್ವರಕ್ಕೆ ಬಂದಿದ್ದ. ಅವನು ಹೀಗೆ ವಿವರಿಸುವನು:

“ನಾನು ಮಧ್ಯಾಹ್ನ ದಕ್ಷಿಣೇಶ್ವರಕ್ಕೆ ಬಂದಾಗ ಶ‍್ರೀರಾಮಕೃಷ್ಣರು ಮಾತ್ರ ಕೋಣೆಯಲ್ಲಿ ಇದ್ದರು. ನರೇಂದ್ರ ಹೊರಗೆ ಮಲಗಿದ್ದನು. ಶ‍್ರೀರಾಮಕೃಷ್ಣರು ಅಂದು ಸಂತೋಷದಲ್ಲಿದ್ದರು. ನಾನು ಅವರಿಗೆ ನಮಸ್ಕಾರ ಮಾಡಿದೊಡನೆ ನರೇಂದ್ರನ ಕಡೆ ತೋರಿ ಹೇಳಿದರು: ‘ನೋಡು ಇಲ್ಲಿ. ಆ ಹುಡುಗ ತುಂಬಾ ಒಳ್ಳೆಯವನು. ಅವನ ಹೆಸರು ನರೇಂದ್ರ. ಅವನು ಮುಂಚೆ ಜಗನ್ಮಯಿಯನ್ನು ಒಪ್ಪಿಕೊಳ್ಳುತ್ತಿರಲಿಲ್ಲ. ಆದರೆ ನಿನ್ನೆ ಒಪ್ಪಿಕೊಂಡ. ಈ ನಡುವೆ ಅವನು ತುಂಬಾ ಕಷ್ಟದಲ್ಲಿರುವನು. ಅದಕ್ಕೆ ನಾನು ಅವನಿಗೆ ಐಶ್ವರ‍್ಯವನ್ನು ಕೊಡೆಂದು ದೇವರಿಗೆ ಪ್ರಾರ್ಥಿಸು ಎಂದು ಹೇಳಿದೆ. ಅವನಿಗೆ ಅದನ್ನು ಬೇಡಲು ಸಾಧ್ಯವಾಗಲಿಲ್ಲ. ಅವನಿಗೆ ತುಂಬಾ ನಾಚಿಕೆ ಆಯಿತು. ಗುಡಿಯಿಂದ ಬಂದಮೇಲೆ ದೇವಿಗೆ ಸಂಬಂಧಪಟ್ಟ ಒಂದು ಹಾಡನ್ನು ತನಗೆ ಕಲಿಸಿಕೊಡಬೇಕೆಂದು ಕೇಳಿಕೊಂಡ. ನಾನು ಒಂದು ಹಾಡನ್ನು ಅವನಿಗೆ ಕಲಿಸಿದೆ. ನಿನ್ನೆ ರಾತ್ರಿಯೆಲ್ಲ ಅವನು ಅದನ್ನು ಹಾಡಿದ. ಆದಕಾರಣವೇ ಅವನು ಈಗ ಮಲಗಿರುವುದು. ನರೇಂದ್ರ ಜಗನ್ಮಯಿಯನ್ನು ಒಪ್ಪಿಕೊಂಡಿರುವುದು ಒಂದು ಅದ್ಭುತವಲ್ಲವೇ?’ ಎಂದು ನನ್ನನ್ನು ಕೇಳಿದರು. ಅದಕ್ಕೆ ನಾನು ‘ಹೌದು’ ಎಂದೆ. ಅವರು ಸ್ವಲ್ಪ ಹೊತ್ತಾದ ಮೇಲೆ ಪುನಃ ಈ ಪ್ರಶ್ನೆಯನ್ನು ಕೇಳಿದರು. ಹೀಗೆಯೇ ಸ್ವಲ್ಪ ಕಾಲ ಕಳೆಯಿತು.

“ಸುಮಾರು ನಾಲ್ಕು ಗಂಟೆ ಹೊತ್ತಿಗೆ ನರೇಂದ್ರ ಕಲ್ಕತ್ತೆಗೆ ಹೋಗುವುದಕ್ಕೆ ಮುಂಚೆ ಶ‍್ರೀರಾಮಕೃಷ್ಣರ ಕೋಣೆಗೆ ಬಂದ. ಆದರೆ ಶ‍್ರೀರಾಮಕೃಷ್ಣರು ಅವನನ್ನು ನೋಡುವುದೇ ತಡ ಅವನ ಹತ್ತಿರ ಹತ್ತಿರಕ್ಕೆ ಹೋಗಿ ಅವನ ತೊಡೆಯ ಮೇಲೆಯೇ ಕುಳಿತುಕೊಳ್ಳುವಷ್ಟು ಸಮೀಪಕ್ಕೆ ಹೋದರು. ಮೊದಲು ತಮ್ಮನ್ನು ತೋರಿಸಿಕೊಂಡು ಅನಂತರ ನರೇಂದ್ರನನ್ನು ತೋರಿಸುತ್ತಾ ಹೀಗೆ ಹೇಳಿದರು: ‘ನೋಡು ನಾನು ಇದು ಮತ್ತು ಅದು ಎಂದು ಭಾಸವಾಗುವುದು. ನಿಜವಾಗಿಯೂ ನನಗೆ ಯಾವ ವ್ಯತ್ಯಾಸವೂ ಕಾಣುವುದಿಲ್ಲ. ಗಂಗಾನದಿಯ ಮೇಲೆ ತೇಲಿಕೊಂಡು ಹೋಗುತ್ತಿರುವ ಕೋಲು ಗಂಗೆಯನ್ನು ಇಬ್ಭಾಗವಾಗಿ ಮಾಡಿದಂತೆ ಇದೆ. ಆದರೆ ನೀರು ನಿಜವಾಗಿ ಒಂದೇ. ನಾನು ಹೇಳುವುದು ನಿನಗೆ ಅರ್ಥವಾಯಿತೇ? ತಾಯಿಯಲ್ಲದೆ ಈ ಪ್ರಪಂಚದಲ್ಲಿ ಮತ್ತೇನಿದೆ! ನೀನು ಏನು ಹೇಳುವೆ?’ ಎಂದು ಕೇಳಿದರು. ಸ್ವಲ್ಪ ಕಾಲ ಹೀಗೆ ಮಾತನಾಡಿದ ಮೇಲೆ ಅವರು ಗುಡುಗುಡಿಯನ್ನು ಸೇದಲು ಇಚ್ಛೆಯನ್ನು ವ್ಯಕ್ತಪಡಿಸಿದರು. ನಾನು ಅದನ್ನು ಅಣಿಮಾಡಿ ಅವರಿಗೆ ಕೊಟ್ಟೆ. ಅವರು ತಾವು ಒಂದೆರಡು ಸಲ ಎಳೆದಮೇಲೆ ತಮ್ಮ ಕೈಗಳಲ್ಲಿ ಹಿಡಿದುಕೊಂಡೇ ನರೇಂದ್ರನಿಗೆ, ಇದರಿಂದ ಸೇದು, ಎಂದರು. ನರೇಂದ್ರ ಅನುಮಾನಿಸಿದ. ತನ್ನ ತುಟಿಗಳಿಂದ ಅದನ್ನು ಸೇದಿ ಗುರುಗಳ ಕೈಗಳನ್ನು ಹೇಗೆ ಅವನು ಮೈಲಿಗೆ ಮಾಡುತ್ತಾನೆ? ಅದಕ್ಕೆ ಶ‍್ರೀರಾಮಕೃಷ್ಣರು ‘ನಿನ್ನದು ಎಂತಹ ಮೌಢ್ಯ ಭಾವನೆ? ನಾನು ನಿನ್ನಿಂದ ಬೇರೆಯೆ? ಇದೂ ನಾನೇ, ಅದೂ ನಾನೇ’ ಎಂದು ಹೇಳಿ ಪುನಃ ತಮ್ಮ ಕೈಗಳನ್ನು ನರೇಂದ್ರನ ಬಾಯಿಯ ಹತ್ತಿರ ತಂದರು. ನರೇಂದ್ರನಿಗೆ ಆಗ ಸೇದದೆ ಬೇರೆ ದಾರಿಯೇ ಇರಲಿಲ್ಲ. ಆತ ಎರಡು ಮೂರು ಸಲ ಎಳೆದ. ಅನಂತರ ಶ‍್ರೀರಾಮಕೃಷ್ಣರು ತಾವು ಸೇದುವುದರಲ್ಲಿದ್ದರು. ನರೇಂದ್ರ ತತ್‍ಕ್ಷಣವೇ ‘ದಯವಿಟ್ಟು ನಿಮ್ಮ ಕೈಗಳನ್ನು ತೊಳೆದುಕೊಳ್ಳಿ’ ಎಂದ. ಆದರೆ ಶ‍್ರೀರಾಮಕೃಷ್ಣರು ‘ಎಂತಹ ಮೂಢವಾದ ಭೇದಭಾವನೆ ನಿನ್ನದು’ ಎಂದು ಹೇಳಿ ತಮ್ಮ ಕೈಗಳನ್ನು ತೊಳೆದುಕೊಳ್ಳದೆ ಸೇದತೊಡಗಿದರು. ಶ‍್ರೀರಾಮಕೃಷ್ಣರು ಎಂದಿಗೂ ಇನ್ನೊಬ್ಬರಿಗೆ ಕೊಟ್ಟ ಆಹಾರವನ್ನು ತೆಗೆದುಕೊಳ್ಳುತ್ತಿರಲಿಲ್ಲ. ನರೇಂದ್ರನ ವಿಷಯದಲ್ಲಿ ಈ ವಿನಾಯಿತಿ ಮಾಡಿದುದನ್ನು ನೋಡಿ ನನಗೆ ಆಶ್ಚರ್ಯವಾಯಿತು. ಇದರಿಂದ ಶ‍್ರೀರಾಮಕೃಷ್ಣರಿಗೆ ನರೇಂದ್ರನ ಮೇಲೆ ಎಂತಹ ವಿಶ್ವಾಸವಿದೆ ಮತ್ತು ಅವರ ಸಂಬಂಧ ಎಂತಹುದು ಎಂಬುದು ಗೊತ್ತಾಯಿತು. ಸುಮಾರು ರಾತ್ರಿ ಎಂಟು ಗಂಟೆಯ ಹೊತ್ತಿಗೆ ಶ‍್ರೀರಾಮಕೃಷ್ಣರು ತಮ್ಮ ಸಹಜಸ್ಥಿತಿಯಲ್ಲಿದ್ದರು. ನಾನು ಮತ್ತು ನರೇಂದ್ರ ಇಬ್ಬರೂ ಅವರಿಗೆ ನಮಸ್ಕಾರ ಮಾಡಿ ಕಲ್ಕತ್ತೆಗೆ ಹಿಂತಿರುಗಿದೆವು.”

ಅನಂತರ ನರೇಂದ್ರ ಶ‍್ರೀರಾಮಕೃಷ್ಣರ ವಿಷಯದಲ್ಲಿ ಹೀಗೆ ಹೇಳುತ್ತಿದ್ದ:\break “ಶ‍್ರೀರಾಮಕೃಷ್ಣರೊಬ್ಬರೇ ನನ್ನನ್ನು ಪ್ರಥಮ ಬಾರಿ ಸಂಧಿಸಿದ ದಿನದಿಂದ ಒಂದೇ ಸಮನಾಗಿ ನಂಬುತ್ತಿದ್ದುದು. ನನ್ನ ತಾಯಿ ಮತ್ತು ಸಹೋದರರೂ ಕೂಡ ಹಾಗೆ ಮಾಡಲಿಲ್ಲ. ನನ್ನ ಮೇಲೆ ಅವರಿಗಿದ್ದ ಎಂದಿಗೂ ಬತ್ತದ ನಂಬಿಕೆ ಮತ್ತು ಪ್ರೇಮವೇ ನನ್ನನ್ನು ಎಂದೆಂದಿಗೂ ಅವರೊಡನೆ ಬಂಧಿಸಿತು. ಮತ್ತೊಬ್ಬರನ್ನು ಪ್ರೀತಿಸುವುದು ಅವರೊಬ್ಬರಿಗೇ ಗೊತ್ತಿತ್ತು. ಪ್ರಾಪಂಚಿಕರು ತಮ್ಮ ಸ್ವಾರ್ಥಸಿದ್ಧಿಗೆ ತಾವು ಪ್ರೀತಿಸುತ್ತಿರುವೆವು ಎಂದು ನಟಿಸುವರು ಅಷ್ಟೆ.”

