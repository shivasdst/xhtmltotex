
\chapter{ಎರಡನೆಯ ಸಲ ಇಂಗ್ಲೆಂಡಿಗೆ}

 ೧೮೯೬ನೇ ಏಪ್ರಿಲ್ ೧೫ನೇ ತಾರೀಖು ಸ್ವಾಮೀಜಿ ನ್ಯೂಯಾರ್ಕ್ ಬಿಟ್ಟು ಸಮುದ್ರಯಾನ ಮಾಡಿಕೊಂಡು ಆ ತಿಂಗಳ ಕೊನೆಯ ಹೊತ್ತಿಗೆ ಲಂಡನ್ ಮುಟ್ಟಿದರು. ಅಲ್ಲಿಗೆ ಇಂಡಿಯಾ ದೇಶದಿಂದ ಆಗಲೇ ಶಾರದಾನಂದರು ಬಂದು ಸ್ಟರ‍್ಡಿ ಮನೆಯಲ್ಲಿ ಸ್ವಾಮೀಜಿಯವರನ್ನು ಕಾಯುತ್ತಿದ್ದರು. ಸ್ವಾಮೀಜಿಯವರಿಗೆ ಹಲವು ವರ್ಷಗಳ ಮೇಲೆ ಅವರ ಒಬ್ಬ ಗುರುಭಾಯಿಯ ದರ್ಶನ ಸಿಕ್ಕಿತು. ಅವರು ಕಲ್ಕತ್ತೆಯಲ್ಲಿರುವ ಮಠ ಇತರ ಗುರುಭಾಯಿಗಳು ಇವರ ವಿಷಯವನ್ನೆಲ್ಲಾ ಕೇಳಿದರು. ಲಂಡನ್ನಿಗೆ ಬಂದ ಒಂದೆರಡು ದಿನಗಳಲ್ಲಿಯೇ ಸ್ವಾಮೀಜಿ ಬೋಧನೆಯ ಕೆಲಸಕ್ಕೆ ಪ್ರಾರಂಭ ಮಾಡಿದರು. ಲಂಡನ್ನಿನ ಜನರಿಗೆ ಸ್ವಾಮೀಜಿ ಆಗಲೆ ಪರಿಚಯವಾಗಿದ್ದರು. ಸ್ವಾಮೀಜಿ ಮತ್ತು ಶಾರದಾನಂದರು ಸೇಂಟ್ ಜಾರ್ಜ್ ರೋಡಿನಲ್ಲಿ ಮಿಸ್ ಮುಲ್ಲರ್ ಮತ್ತು ಸ್ಟರ‍್ಡಿ ಅವರ ಅತಿಥಿಗಳಾಗಿದ್ದರು. 

 ಮೇ ಪ್ರಾರಂಭದಲ್ಲಿ ಸ್ವಾಮೀಜಿ ಜ್ಞಾನಯೋಗದ ಮೇಲೆ ಉಪನ್ಯಾಸಗಳನ್ನು ಪ್ರಾರಂಭಮಾಡಿದರು. ಮೇ ತಿಂಗಳ ಕೊನೆಯ ಹೊತ್ತಿಗೆ ಪಿಕಾಡಿಲ್ಲಿಯ \enginline{Royal Institute of Painters in water-colour} ಎಂಬಲ್ಲಿ ಭಾನುವಾರ ಉಪನ್ಯಾಸಗಳನ್ನು ಪ್ರಾರಂಭಿಸಿದರು. ಅಲ್ಲಿ ‘ಧರ್ಮದ ಅವಶ್ಯಕತೆ’, ‘ವಿಶ್ವಧರ್ಮ’, ‘ಮಾನವ ಅವನ ತೋರಿಕೆಯ ಮತ್ತು ನೈಜ ಸ್ವಭಾವ’ ಎಂಬ ವಿಷಯಗಳ ಮೇಲೆ ಮಾತನಾಡಿದರು. ಅನಂತರ ಜೂನ್ ಕೊನೆಯಿಂದ ಜುಲೈ ಮಧ್ಯದವರೆಗೆ ಪ್ರಿನ್‍ಸೆಸ್ ಹಾಲಿನಲ್ಲಿ ‘ಭಕ್ತಿಯೋಗ’, ‘ತ್ಯಾಗ’, ‘ಆತ್ಮ ಸಾಕ್ಷಾತ್ಕಾರ’ ಮುಂತಾದವುಗಳ ಮೇಲೆ ಮಾತನಾಡಿದರು. ಇದಲ್ಲದೆ ಪ್ರತಿವಾರವೂ ಐದು ಪ್ರವಚನಗಳನ್ನು ಕೊಡುತ್ತಿದ್ದರು, ಮತ್ತು ಶುಕ್ರವಾರ ಸಾಯಂಕಾಲ ಒಂದು ಪ್ರಶ್ನೋತ್ತರದ ತರಗತಿಯನ್ನು ಪ್ರಾರಂಭ ಮಾಡಿದರು. ಜೊತೆಗೆ ಹಲವು ಕ್ಲಬ್ಬುಗಳಲ್ಲಿ ಮನೆಗಳಲ್ಲಿ ಬೇರೆ ಪ್ರತ್ಯೇಕ ಕೋರಿಕೆಯಿಂದ ಉಪನ್ಯಾಸಗಳನ್ನು ಮಾಡಿದರು. ಶ‍್ರೀಮತಿ ಅನಿಬೆಸೆಂಟರ ಕೋರಿಕೆ ಮೇಲೆ ಅವೆನ್ಯೂ ರೋಡ್‍ನಲ್ಲಿ ಸೇಂಟ್ ಜಾನ್ಸ್ ವುಡ್ ಎಂಬುವಳ ಮನೆಯಲ್ಲಿ ಸ್ವಾಮೀಜಿ ಅವರು ಭಕ್ತಿಯ ಮೇಲೆ ಉಪನ್ಯಾಸವನ್ನು ಮಾಡಿದರು. ನಾಟಿಂಗ್‍ಹಾಮ್ ಗೇಟಿನಲ್ಲಿರುವ ಶ‍್ರೀಮತಿ ಹಿಲ್ ಅವರ ಮನೆ ಮತ್ತು ವಿಂಬಲ್‍ಡನ್ ಎಂಬಲ್ಲಿಯೂ ಉಪನ್ಯಾಸ ಕೊಟ್ಟರು. ಅಲ್ಲಿ ವಿದ್ಯಾಭ್ಯಾಸ ಎಂಬುದರ ಮೇಲೆ ಮಾತನಾಡಿ ಉಪನಿಷತ್ತಿನ ಕಾಲದ ವಿದ್ಯಾಭ್ಯಾಸದ ಆದರ್ಶವನ್ನು ಒತ್ತಿ ಹೇಳಿದರು. ಅದು ಪುರುಷಸಿಂಹರನ್ನು ಮಾಡುವ ವಿದ್ಯೆ ಆಗಿತ್ತೇ ಹೊರತು ಸುಮ್ಮನೆ ವಿಷಯಗಳನ್ನು ಕಂಠಪಾಠಮಾಡಿ ಅರಗಿಳಿಯಂತೆ ಆಗುವುದಲ್ಲ ಎಂದು ವಿವರಿಸಿದರು. 

 ಶ‍್ರೀ ಕ್ಯಾನನ್ ವಿಲ್‌ಬರ್‌ಫೋರ್ಸ್‍ ಎಂಬುವರ ಮನೆಯಲ್ಲಿ ಸ್ವಾಮೀಜಿ ಅವರಿಗೆ ಒಂದು ಸತ್ಕಾರ ಕೂಟವನ್ನು ಏರ್ಪಡಿಸಿದ್ದರು. ಎರಿಕ್ ಹ್ಯಾಮಂಡ್ ಎಂಬುವರು ಸ್ವಾಮೀಜಿಯವರ ಒಂದು ಉಪನ್ಯಾಸವನ್ನು ಕೇಳಿ ಆದಮೇಲೆ ತಮ್ಮ ಅಭಿಪ್ರಾಯವನ್ನು ಒಂದು ಪತ್ರದ ಮೂಲಕ ವಿವರಿಸಿರುವರು: “ಒಂದು ದಿನ ಉಪನ್ಯಾಸವನ್ನು ಪೂರೈಸಿದ ಮೇಲೆ ಬಿಳಿಯ ಕೂದಲಿನ ಲಂಡನ್ ಜನರಿಗೆ ಚಿರಪರಿಚಿತನಾದ ತತ್ತ್ವಜ್ಞಾನಿಯೊಬ್ಬನೆದ್ದು ಸ್ವಾಮಿಗಳಿಗೆ ‘ನೀವು ಬಹಳ ಚೆನ್ನಾಗಿ ಮಾತನಾಡಿದಿರಿ ಸ್ವಾಮೀಜಿ. ಅದಕ್ಕೆ ನನ್ನ ಹೃತ್ಪೂರ್ವಕ ಅಭಿನಂದನೆಗಳು. ಆದರೆ ನೀವು ಯಾವ ಹೊಸ ವಿಷಯವನ್ನೂ ಹೇಳಲಿಲ್ಲ’ ಎಂದ. ಅದಕ್ಕೆ ಸ್ವಾಮೀಜಿ ತಮ್ಮ ಗಂಭೀರವಾದ ಮಧುರವಾದ ವಾಣಿಯಲ್ಲಿ ಹೀಗೆ ಹೇಳಿದರು: ‘ಸರ್, ನಾನು ನಿಮಗೆ ಸತ್ಯವನ್ನು ಹೇಳಿರುವೆ. ಆ ಸತ್ಯವಾದರೋ ಈ ಪರ್ವತದಷ್ಟೇ ಹಳೆಯದು, ಅದು ಮಾನವ ಕೋಟಿಯಷ್ಟೇ ಹಳೆಯದು, ಸೃಷ್ಟಿಯಷ್ಟೇ ಹಳೆಯದು, ಮಹದೇಶ್ವರನಷ್ಟೇ ಹಳೆಯದು. ನಿಮ್ಮನ್ನು ಆಲೋಚಿಸುವಂತೆ ಮಾಡಿ, ಅದಕ್ಕೆ ಸರಿಯಾಗಿ ಬಾಳುವಂತೆ ಪ್ರಚೋದಿಸಿದ್ದರೆ, ಅದನ್ನು ಹೇಳಿದ್ದು ಸಾರ್ಥಕವಾಗಲಿಲ್ಲವೆ?’ ಇದನ್ನು ಕೇಳಿ ಪ್ರೇಕ್ಷಕರೆಲ್ಲರೂ ‘\enginline{Hear Hear}’ ಎಂದು ಕರತಾಡನ ಮಾಡಿದರು.” 

 ಒಬ್ಬ ಸ್ತ್ರೀ ಆ ಉಪನ್ಯಾಸ ಮತ್ತು ಸ್ವಾಮೀಜಿಯವರ ಇತರ ಉಪನ್ಯಾಸಗಳನ್ನು ಕೇಳುತ್ತಿದ್ದಾಗ ಆಕೆ ತನ್ನ ಅಭಿಪ್ರಾಯವನ್ನು ಹೀಗೆ ಕೊಡುತ್ತಾಳೆ: “ನಾನು ಚರ್ಚುಗಳ ಭಾಷಣವನ್ನು ಜೀವನ ಪರ್ಯಂತರವೂ ಕೇಳಿರುವೆನು. ನಿಸ್ತೇಜವಾದ ಸಪ್ಪೆಯಾದ ಆ ಮಾತುಗಳನ್ನು ಕೇಳಿ ಕೇಳಿ ನನಗೆ ಅತೃಪ್ತಿಯಾಗಿತ್ತು. ಎಲ್ಲರೂ ಹೊಗಳುತ್ತಾರೆ ಎಂದು ನಾನೂ ಹೋಗುತ್ತಿದ್ದೆ. ನಾನೊಬ್ಬಳೆ ವಿಚಿತ್ರವಾಗಿರಲು ಬಯಸಲಿಲ್ಲ. ನಾನು ಸ್ವಾಮೀಜಿಯವರನ್ನು ಕೇಳಿದಮೇಲೆ ಧರ್ಮಜ್ಯೋತಿ ಪೂರ್ಣವಾಯಿತು. ಧರ್ಮ ಸತ್ಯವಾಗಿ ಕಂಡಿತು, ಅದು ಜೀವಂತವಾಯಿತು. ಹೊಸ ಸಂತೋಷದಾಯಕವಾದ ಸಂದೇಶ ಅದರಲ್ಲಿ ತೋರಿತು. ಅದು ನನ್ನನ್ನು ಸಂಪೂರ್ಣ ಬದಲಾಯಿಸಿದೆ.” 

 ಸ್ವಾಮೀಜಿಯವರು ತಮ್ಮ ಒಂದು ಉಪನ್ಯಾಸದಲ್ಲಿ ಹೀಗೆ ಹೇಳುವರು: 

 “ಸತ್ಯ ನನಗೆ ದೊರಕಿದೆ. ಏಕೆಂದರೆ ಅದಾಗಲೆ ನನ್ನಲ್ಲಿತ್ತು. ನೀವು ಮೋಸ ಹೋಗಬೇಡಿ. ಈ ಮತದಲ್ಲೋ ಆ ಮತದಲ್ಲೋ ನಿಮಗೆ ದೊರಕುವುದು ಎಂದು ನೀವು ಭ್ರಾಂತರಾಗಬೇಡಿ. ಅದು ನಿಮ್ಮಲ್ಲಿಯೇ ಇರುವುದು. ನಿಮ್ಮ ಮತ ನಿಮಗೆ ಅದನ್ನು ಕೊಡಲಾರದು. ನೀವು ನಿಮ್ಮ ಮತಕ್ಕೆ ಸತ್ಯವನ್ನು ಕೊಡಬೇಕಾಗಿದೆ. ಪಾದ್ರಿಗಳು ಮತ್ತು ಮನುಷ್ಯರು ಅದಕ್ಕೆ ಬೇರೆ ಬೇರೆ ಅರ್ಥಗಳನ್ನು ಕೊಡುವರು. ಅವರು ನಿಮಗೆ ಯಾವು ಯಾವುದನ್ನೋ ನಂಬಿ ಎನ್ನುವರು. ಅನರ್ಘ್ಯವಾದ ಸತ್ಯ ನಿಮ್ಮಲ್ಲಿಯೇ ಇದೆ ಎಂಬುದನ್ನು ನಂಬಿ. ಯಾವುದು ಸತ್ಯವೋ ಅದು ಒಂದು, ನೀನೇ ಅದು.” ಅವರು ತಮ್ಮ ಭಾಷಣದಲ್ಲಿ ಮೊದಲಿನಿಂದ ಕೊನೆಯವರೆಗೆ ಶ‍್ರೀರಾಮಕೃಷ್ಣರ ಸಂದೇಶದ\break ವಿಷಯವಾಗಿ ಹೇಳಿದರು. ಅವರು ತಮ್ಮಲ್ಲಿ ಶ‍್ರೀರಾಮಕೃಷ್ಣರಲ್ಲದೆ ಬೇರೆ ಯಾವ ಒಂದು ಪದವಾಗಲೀ, ಯಾವ ಒಂದು ಅತ್ಯಂತ ಸಣ್ಣ ಭಾವನೆಯಾಗಲಿ ಇಲ್ಲ ಎಂದರು. ಅವರು ಹೇಳಿದ ಪ್ರತಿಯೊಂದೂ, ಅವರು ಏನಾಗಿರುವರೊ, ಅದೆಲ್ಲ ಆ ಒಂದು ಮೂಲದಿಂದ ಆ ಪರಿಶುದ್ಧಾತ್ಮನಿಂದ, ಅಸೀಮವಾದ ಅನಂತ ಸ್ಫೂರ್ತಿಯಿಂದ ಬಂದಿದೆ. ಆ ನನ್ನ ಪರಮಪ್ರಿಯವಾದ ಭರತಖಂಡದಲ್ಲಿ ಕುಳಿತುಕೊಂಡು ಈ ರಹಸ್ಯವನ್ನು ಅವರು ಭೇದಿಸಿದರು. ಅವರು ತಾವು ಕಂಡ ಬೆಳಕನ್ನು ಸ್ವಲ್ಪವೂ ಸಂಕೋಚವಿಲ್ಲದೆ ಮುಕ್ತ ಹಸ್ತದಿಂದ ಇಡೀ ಪ್ರಪಂಚಕ್ಕೆ ದಾನ ಮಾಡಿದರು. 

 “ಶ‍್ರೀರಾಮಕೃಷ್ಣರ ಜೀವನದಲ್ಲಿ ನಾನು ಎಂಬುದು ಸಂಪೂರ್ಣವಾಗಿ ಮಾಯವಾಗಿತ್ತು. ಅದು ಲವಲೇಶವೂ ಇರಲಿಲ್ಲ. ನಾನು ಏನಾಗಿರುವೆನೊ ಅದೆಲ್ಲಾ ಅವರ ದಯೆಯಿಂದ. ನನ್ನ ಜೀವನದಲ್ಲಿ ಮತ್ತು ಮಾತಿನಲ್ಲಿ ಒಳ್ಳೆಯದಾಗಿರುವುದು ಸತ್ಯವಾಗಿರುವುದು, ಸನಾತನವಾಗಿರುವುದು ಯಾವುದು ಇದೆಯೊ ಅದೆಲ್ಲಾ ಅವರ ಬಾಯಿಂದ, ಅವರ ಹೃದಯದಿಂದ, ಅವರಾತ್ಮದಿಂದ ಬಂದಿತು. ಅವರೇ ಈ ವಿಶ್ವದ ಆಧ್ಯಾತ್ಮಿಕ ಜೀವನದ ಮಹಾ ಉಗಮಸ್ಥಾನ. ಅದು ವಿಶ್ವದಲ್ಲೆಲ್ಲ ಹರಡುತ್ತಿರುವುದು. ಇದಕ್ಕೆಲ್ಲ ಶ‍್ರೀರಾಮಕೃಷ್ಣರೇ ಮೂಲ. ನಾನು ಪ್ರಪಂಚಕ್ಕೆ ನನ್ನ ಗುರುದೇವರ ಒಂದು ಕಿಂಚಿತ್ ಪ್ರಭೆಯನ್ನು ಕೊಟ್ಟರೆ ಸಾಕು. ನನ್ನ ಜೀವನ ಸಾರ್ಥಕವಾಯಿತೆಂದು ಭಾವಿಸುವೆನು.” 

 ಇಡೀ ಪ್ರಪಂಚವೇ ವಿವೇಕಾನಂದರನ್ನು ಕೊಂಡಾಡುತ್ತಿದ್ದಾಗ ಸ್ವಾಮೀಜಿ ಅದನ್ನು ಹೇಗೆ ಸ್ವೀಕರಿಸಿದರು ಎಂಬುದನ್ನು ನೋಡುವೆವು. ತಮ್ಮ ವ್ಯಕ್ತಿತ್ವವೋ ಆ ಗುರುಗಳ ಅನುಗ್ರಹದಿಂದ ಮೂಡಿದುದು, ತಾವು ಶ‍್ರೀರಾಮಕೃಷ್ಣರ ಸಂದೇಶವನ್ನು ಹರಡುವುದಕ್ಕೆ ಇರುವ ಒಂದು ನಿಮಿತ್ತ ಮಾತ್ರವೆಂದು ಭಾವಿಸಿ, ತಮಗೆ ಬಂದ ಹೆಸರು ಕೀರ್ತಿ ಮುಂತಾದುವುಗಳನ್ನೆಲ್ಲ ಅವರ ಗುರುಗಳ ಪದತಳದಲ್ಲಿ ಸಮರ್ಪಿಸಿ ಧನ್ಯರಾಗುತ್ತಿದ್ದರು. 

 ಇಂಗ್ಲೆಂಡ್ ಮತ್ತು ಐರ್ಲೆಂಡಿನಲ್ಲಿದ್ದ ಭರತಖಂಡದ ವಿದ್ಯಾರ್ಥಿವರ್ಗದವರಿಗೆಲ್ಲ ಸ್ವಾಮೀಜಿ ಬೇಕಾದವರಾದರು. ಜುಲೈ ೧೮ನೇ ತಾರೀಖು ಹಿಂದೂ ಅಸೋಸಿಯೇಷನ್ ಎಂಬ ಹೆಸರಿನಲ್ಲಿ ಆ ವಿದ್ಯಾರ್ಥಿಗಳು ಒಂದು ಸಭೆಯನ್ನು ಸೇರಿಸಿದರು. ಆ ಸಮಯದಲ್ಲಿ ಸ್ವಾಮೀಜಿಯವರನ್ನು ಅಧ್ಯಕ್ಷತೆ ವಹಿಸಬೇಕೆಂದು ಕೋರಿಕೊಂಡರು. ಸ್ವಾಮೀಜಿ ಅದಕ್ಕೆ ಒಪ್ಪಿಕೊಂಡು “ಹಿಂದೂಗಳು ಮತ್ತು ಅವರ ಅವಶ್ಯಕತೆ” ಎಂಬ ವಿಷಯವಾಗಿ ಮಾತನಾಡಿದರು. ಅಷ್ಟೊಂದು ಉಪನ್ಯಾಸ ಪ್ರವಚನಾದಿಗಳ ಮಧ್ಯದಲ್ಲಿದ್ದರೂ ಸ್ಟರ‍್ಡಿ ಅವರಿಗೆ ನಾರದಭಕ್ತಿ ಸೂತ್ರವನ್ನು ಇಂಗ್ಲೀಷಿಗೆ ಭಾಷಾಂತರ ಮಾಡಲು ಸಹಾಯ ಮಾಡುತ್ತಿದ್ದರು. 

 ಸ್ವಾಮೀಜಿಯವರು ಈ ಸಮಯದಲ್ಲಿ ಪ್ರಖ್ಯಾತ ಪೌರಶಾಸ್ತ್ರ ವಿದ್ವಾಂಸರಾದ\break ಪ್ರೊಫೆಸರ್ ಮ್ಯಾಕ್ಸ್‌ಮುಲ್ಲರ್ ಅವರನ್ನು ಆಕ್ಸ್​‍ಫರ್ಡ್‍ನಲ್ಲಿ ಕಂಡರು. ಅವರು ಅಷ್ಟು ಹೊತ್ತಿಗೆ ‘ಹತ್ತೊಂಬತ್ತನೇ ಶತಮಾನ’ ಎಂಬ ಮಾಸಪತ್ರಿಕೆಯಲ್ಲಿ ಶ‍್ರೀರಾಮಕೃಷ್ಣರ ಮೇಲೆ ‘ನಿಜವಾದ ಮಹಾತ್ಮ’ ಎಂಬ ಲೇಖನವನ್ನು ಬರೆದಿದ್ದರು. ಸ್ವಾಮೀಜಿಯವರ\break ಪರಿಚಯವಾದ ಮೇಲೆ ಸ್ವಾಮೀಜಿ ಬಾಯಿಂದಲೇ ಶ‍್ರೀರಾಮಕೃಷ್ಣರ ಹಲವು ವಿಷಯಗಳನ್ನು ಕೇಳಿ ಶ‍್ರೀರಾಮಕೃಷ್ಣರ ಜೀವನ ಮತ್ತು ಸಂದೇಶ ಎಂಬ ಒಂದು ಪುಸ್ತಕವನ್ನು ಅನಂತರ ಬರೆದರು. ಇಂಗ್ಲೀಷ್ ಭಾಷೆಯಲ್ಲಿ ಶ‍್ರೀರಾಮಕೃಷ್ಣರ ಮೇಲೆ ಬರೆದ ಮೊದಲನೇ ಪುಸ್ತಕವಾಗಿದೆ ಇದು. ಇದನ್ನು ಬರೆದದ್ದು ಪಾಶ್ಚಾತ್ಯ ವಿದ್ವಾಂಸ. ಸ್ವಾಮೀಜಿ ಮ್ಯಾಕ್ಸ್‌ಮುಲ್ಲರ್ ಅವರ ಭೇಟಿಯನ್ನು ಕುರಿತು ಅನಂತರ ‘ಉದ್ಬೋಧನ’ ಎಂಬ ಮಾಸಪತ್ರಿಕೆಗೆ ಒಂದು ಲೇಖನವನ್ನು ಬರೆದು ಕಳುಹಿಸಿರುವರು. ಅದನ್ನು ಕೆಳಗೆ ಕೊಡುವೆವು: 

 “ಶ‍್ರೀರಾಮಕೃಷ್ಣರ ಮೇಲೆ ಮ್ಯಾಕ್ಸ್‌ಮುಲ್ಲರ್ ‘ಹತ್ತೊಂಬತ್ತನೆ ಶತಮಾನ’ ಎಂಬುದರಲ್ಲಿ ಒಂದು ಲೇಖನವನ್ನು ಬರೆದಿರುವರು. ಅದು ನಿಮಗೆ ಗೊತ್ತೆ? ಸಾಕಷ್ಟು ಸಾಮಗ್ರಿ ಸಿಕ್ಕಿದರೆ ಮತ್ತೊಂದು ವಿಶದವಾದ ಅವರ ಜೀವನ ಮತ್ತು ಸಂದೇಶವನ್ನು ಅವರು ಬರೆಯುವುದಕ್ಕೆ ಸಿದ್ಧರಾಗಿರುವರು. ಎಂತಹ ಅಪೂರ್ವ ವ್ಯಕ್ತಿ! ಕೆಲವು ದಿನಗಳ ಹಿಂದೆ ನಾನು ಅವರ ಭೇಟಿಗೆ ಹೋಗಿದ್ದೆ. ನಾನು ಅವರಿಗೆ ಗೌರವವನ್ನು ಅರ್ಪಿಸಲು ಹೋಗಿದ್ದೆ ಎಂದು ಹೇಳಬಹುದು. ಯಾರು ಶ‍್ರೀರಾಮಕೃಷ್ಣರನ್ನು ಪ್ರೀತಿಸುವರೊ ಅವರು ಯಾವ ಕುಲಗೋತ್ರ ಜನಾಂಗಕ್ಕೆ ಸೇರಿರಲಿ, ಅವರನ್ನು ನೋಡಲು ಹೋಗುವುದು ಒಂದು ಯಾತ್ರೆಗೆ ಹೋದಂತೆ ಎಂದು ಭಾವಿಸುವೆನು. ‘ಯಾರು ನನ್ನ ಭಕ್ತರನ್ನು ಭಕ್ತಿಯಿಂದ ಕಾಣುವರೋ ಅವರೇ ನನ್ನ ಶ್ರೇಷ್ಠ ಭಕ್ತರು’- ಇದು ನಿಜವಲ್ಲವೆ? 

 “ಬ್ರಹ್ಮ ಸಮಾಜದ ಪ್ರಮುಖ ನಾಯಕನಾದ ಕೇಶವಚಂದ್ರಸೇನನಲ್ಲಿ ಅಕಸ್ಮಾತ್ತಾಗಿ ಆದ ಅದ್ಭುತ ಬದಲಾವಣೆಯ ಹಿಂದೆ ಯಾವ ಶಕ್ತಿ ಇತ್ತು ಎಂಬುದನ್ನು ಪ್ರೊಫೆಸರ್ ಮೊದಲು ವಿಚಾರಮಾಡತೊಡಗಿದರು. ಅಂದಿನಿಂದಲೂ ಶ‍್ರೀರಾಮಕೃಷ್ಣರ ಜೀವನ ಮತ್ತು ಸಂದೇಶವನ್ನು ಮೆಚ್ಚುವ ಶ್ರದ್ಧಾವಂತ ಶಿಷ್ಯರಾಗಿರುವರು. ‘ಇಂದು ಸಾವಿರಾರು ಜನ ಶ‍್ರೀರಾಮಕೃಷ್ಣರನ್ನು ಪೂಜಿಸುತ್ತಿರುವರು, ಪ್ರೊಫೆಸರ್’ ಎಂದಾಗ, ‘ಇಂಥವರಿಗೆ ಪೂಜೆಮಾಡದೆ ಮತ್ತಾರಿಗೆ ಪೂಜೆಮಾಡಬೇಕು?’ ಎಂದರು. ಪ್ರೊಫೆಸರ್ ದಯೆಯೇ ಮೂರ್ತಿವೆತ್ತಂತೆ ಇದ್ದರು. ಮಿಸ್ಟರ್ ಸ್ಟರ‍್ಡಿ ಮತ್ತು ನನ್ನನ್ನು ಊಟಕ್ಕೆ ಆಹ್ವಾನಿಸಿದರು. ಆಕ್ಸ್​‍ಫರ್ಡ್‍ನಲ್ಲಿ ಹಲವು ಕಾಲೇಜುಗಳು ಬಾಡ್ಲಿಯನ್ ಲೈಬ್ರರಿ ಮುಂತಾದವುಗಳನ್ನು ತೋರಿಸಿದರು. ರೈಲ್ವೆ ಸ್ಟೇಷನ್ನಿಗೂ ನಮ್ಮ ಒಡನೆ ಬಂದಿದ್ದರು. ‘ನೀವು ಏತಕ್ಕೆ ಇಷ್ಟು ತೊಂದರೆ ತೆಗೆದುಕೊಳ್ಳುತ್ತೀರಿ?’ ಎಂದು ಕೇಳಿದಾಗ, ‘ನಾನು ನಿತ್ಯವೂ ಶ‍್ರೀರಾಮಕೃಷ್ಣ ಪರಮಹಂಸರ ಶಿಷ್ಯರನ್ನು ಕಾಣುವುದಿಲ್ಲ’ ಎಂದರು. 

 “ಈ ಭೇಟಿ ನಿಜವಾಗಿಯೂ ಒಂದು ಅಪೂರ್ವ ಅನುಭವ. ಅವರ ಸುಂದರವಾದ ಸಣ್ಣಮನೆ ರಮ್ಯವಾದ ಉದ್ಯಾನವನದಿಂದ ಆವೃತವಾಗಿತ್ತು. ಬಿಳಿಯ ಕೂದಲುಗಳಿಂದ ಕೂಡಿತ್ತು ಆ ಋಷಿಯ ತಲೆ. ಕರುಣೆ ಮತ್ತು ಶಾಂತಿ ಉಕ್ಕುತ್ತಿದ್ದ ಮೊಗ. ಎಪ್ಪತ್ತು ಛಳಿಗಾಲಗಳ ಪೆಟ್ಟು ತಾಕಿದ್ದರೂ ಅವರ ಹಣೆ ಮಗುವಿನ ಹಣೆಯಂತೆ ಮೃದುವಾಗಿತ್ತು. ಆ ಮುಖದಲ್ಲಿ ಚಿತ್ರವಾಗಿದ್ದ ಪ್ರತಿಯೊಂದು ರೇಖಾ ವಿನ್ಯಾಸವೂ ಅದರ ಹಿಂದೆ ಇದ್ದ ಅಗಾಧವಾದ ಆಧ್ಯಾತ್ಮಿಕ ಗಣಿಯನ್ನು ಸೂಚಿಸುತ್ತಿತ್ತು. ಅವರ ಸಾಧ್ವೀಮಣಿ\break ಜೀವನದ ಸಂಗಾತಿಯಾದ ಸತಿ, ದೀರ್ಘಕಾಲದ ಸ್ವಾರಸ್ಯಪೂರ್ಣವಾದ ವಿದ್ವತ್ ಜೀವನದಲ್ಲಿ - ಖಂಡನೆ ಮತ್ತು ತಾತ್ಸಾರಗಳನ್ನು ಮನಸ್ಸಿಗೆ ತಾರದೆ ಕೊನೆಗೆ ಪುರಾತನ ಆರ್ಯಮಹರ್ಷಿಗಳ ಭಾವನೆಗೆ ಗೌರವವನ್ನು ಸಂಪಾದಿಸಿ ಕೊಟ್ಟರು. ಅಲ್ಲಿದ್ದ ತರುಲತೆಗಳು ವೃಕ್ಷಗಳು, ಪ್ರಶಾಂತವಾತಾವರಣ, ಸ್ವಚ್ಛವಾದ ನೀಲಿ ಆಗಸ, ಇವುಗಳೆಲ್ಲ ಕಲ್ಪನೆಯ ಮೂಲಕ ನನ್ನನ್ನು ಭಾರತದ ಬ್ರಹ್ಮಋಷಿಗಳು, ರಾಜಋಷಿಗಳು, ಪ್ರಖ್ಯಾತ ವಾನಪ್ರಸ್ಥರಾದ ವಸಿಷ್ಠ ಅರುಂಧತಿಯರ ಕಾಲಕ್ಕೆ ಒಯ್ದಿತು.” 

 “ನಾನು ನೋಡಿದ ವ್ಯಕ್ತಿ ದಾರ್ಶನಿಕನೂ ಅಲ್ಲ, ವಿದ್ವಾಂಸನೂ ಅಲ್ಲ. ಪ್ರತಿದಿನ ಬ್ರಹ್ಮನಲ್ಲಿ ತಾನು ಒಂದು ಎಂಬುದನ್ನು ಅರಿಯುತ್ತಿರುವ ವಿಶ್ವದೊಂದಿಗೆ ಒಂದಾಗಲು ಪ್ರತಿಕ್ಷಣವೂ ವಿಕಾಸವಾಗುತ್ತಿರುವ ಹೃದಯವನ್ನು ನಾನು ನೋಡಿದ್ದು. ಎಲ್ಲಿ ಇತರರು ನೀರಸವಾದ ಮರಳು ಕಾಡಿನಲ್ಲಿ ದಾರಿ ತಪ್ಪುವರೊ ಅಲ್ಲಿ ಇವರು ಚೈತನ್ಯದ ಚಿಲುಮೆಯನ್ನು ಮುಟ್ಟಿದ್ದರು. ನಿಜವಾಗಿ ‘ಆತ್ಮನನ್ನು ಮಾತ್ರ ಅರಿ. ಉಳಿದ ಮಾತುಗಳನ್ನೆಲ್ಲ ತ್ಯಜಿಸು’, ಎಂಬ ಉಪನಿಷತ್ತಿನ ಸಂದೇಶದೊಂದಿಗೆ ಅವರ ಹೃದಯ ಸ್ಪಂದಿಸುತ್ತಿದೆ.” 

 “ಪ್ರಪಂಚದಲ್ಲೆಲ್ಲಾ ಪ್ರಖ್ಯಾತರಾದ ವಿದ್ವಾಂಸರಾಗಿದ್ದರೂ, ದಾರ್ಶನಿಕರಾಗಿದ್ದರೂ, ವಿದ್ವತ್ ಮತ್ತು ದರ್ಶನ ಅವರನ್ನು ಆತ್ಮಸಾಕ್ಷಾತ್ಕಾರದ ಶಿಖರದೆಡೆಗೆ ಹೆಚ್ಚು ಹೆಚ್ಚು ಒಯ್ದಿದೆ, ಅವರ ಅಪರಾ ವಿದ್ಯೆ ಪರಾ ವಿದ್ಯೆಯನ್ನು ಪಡೆಯಲು ಸಹಾಯಮಾಡಿದೆ.” 

 “ಭರತಖಂಡದ ಮೇಲೆ ಅವರಿಗಿರುವ ಪ್ರೀತಿ ಎಷ್ಟು! ಅವರ ಪ್ರೀತಿಯ ನೂರನೇ ಒಂದು ಭಾಗವಾದರೂ ನನ್ನಲ್ಲಿ ನನ್ನ ತಾಯ್ನಾಡಿಗೆ ಇದ್ದಿದ್ದರೆ ಚೆನ್ನಾಗಿರುತ್ತಿತ್ತು ಎಂದು ಭಾವಿಸುವೆನು. ಅದ್ಭುತವಾದ ಮತ್ತು ತೀವ್ರವಾದ ಕಾರ‍್ಯೋನ್ಮುಖವಾದ ಮನಸ್ಸು ಅವರಿಗಿದೆ. ತೀವ್ರವಾದ ಮಾನಸಿಕ ಚಟುವಟಿಕೆಯಲ್ಲಿ, ಭಾರತೀಯ ಭಾವನಾ ವಾತಾವರಣದಲ್ಲಿ ಸುಮಾರು ಐವತ್ತು ವರುಷಗಳಿಗಿಂತ ಹೆಚ್ಚಾಗಿ ಬಾಳಿರುವರು. ಸಂಸ್ಕೃತ ಸಾಹಿತ್ಯದ ಅಗಾಧವಾದ ನೆರಳು ಬೆಳಕುಗಳನ್ನು ಉತ್ಸಾಹದಿಂದ ಮತ್ತು ಪ್ರೀತಿಯಿಂದ ಗಮನಿಸಿರುವರು. ಈ ಭಾವನೆಗಳೆಲ್ಲಾ ಅವರ ಜೀವನದಲ್ಲಿ ಓತಪ್ರೋತವಾಗಿ ಅವರ ಜೀವನವನ್ನೇ ರೂಪಿಸಿದೆ. 

 “ಮ್ಯಾಕ್ಸ್‌ಮುಲ್ಲರ್ ವೇದಾಂತಿಗಳಲ್ಲಿ ವೇದಾಂತಿಗಳು, ವೇದಾಂತದಲ್ಲಿರುವ ಸಾಮ\-ರಸ್ಯ ಭಾವನೆಗಳನ್ನು ಮತ್ತು ವಿರೋಧವಾದ ಭಾವನೆಗಳ ಹಿಂದೆ ಇರುವ ವೇದಾಂತದ ಮೂಲ ಸಾರವನ್ನೇ ಇವರು ಪಡೆದಿರುವರು. ಅದೇ ಯಾವ ಒಂದು ತತ್ತ್ವವನ್ನು ಪ್ರಪಂಚದ ಇತರ ಧರ್ಮಗಳೆಲ್ಲಾ ಬೇರೆ ಬೇರೆ ರೀತಿಯಲ್ಲಿ ಹೇಳುತ್ತಿದೆಯೋ ಅದನ್ನು ಇವರು ಪಡೆದಿರುವರು. ಶ‍್ರೀರಾಮಕೃಷ್ಣ ಪರಮಹಂಸರು ಯಾರು? ಅವರು ಆ ಸನಾತನ ತತ್ತ್ವವನ್ನು ಈ ಜೀವನದಲ್ಲಿ ಅನುಷ್ಠಾನಮಾಡಿ ತೋರಿದವರು. ಸನಾತನ ಭರತಖಂಡವೇ ರೂಪವೆತ್ತಂತೆ, ಪ್ರಪಂಚದ ಜನಾಂಗಗಳಿಗೆ ಜ್ಯೋತಿಯನ್ನು ನೀಡುವ ಭವಿಷ್ಯ ಭಾರತದ ಅರುಣೋದಯದಂತೆ ಇದೆ ಅವರ ಜೀವನ. ವಜ್ರ ವ್ಯಾಪಾರಿ ಮಾತ್ರ ವಜ್ರದ ಬೆಲೆ ತಿಳಿಯಬಲ್ಲನು ಎಂಬುದು ಒಂದು ಹಳೆಯ ನಾಣ್ಣುಡಿ. ಈ ಯೂರೋಪಿನ ಋಷಿ, ಭರತಖಂಡದ ಭಾವನೆಯ ಆಕಾಶದಲ್ಲಿ ಮೂಡಿದ, ಪ್ರತಿಯೊಂದು ಹೊಸ ತಾರೆಯನ್ನು ಭಾರತೀಯರು ಅದರ ಮಹಿಮೆ ಅರಿಯುವುದಕ್ಕೆ ಮುಂಚೆ, ಅದನ್ನು ಅಧ್ಯಯನಮಾಡಿ ತಿಳಿದುಕೊಳ್ಳಲು ಪ್ರಯತ್ನಿಸಿರುವರು.” 

 ಸ್ವಾಮೀಜಿ ಹೃದಯ ಯಾವ ರೀತಿ ಪ್ರತಿಕ್ರಿಯೆಯನ್ನು ತೋರಿತು ಪ್ರೊಫೆಸರ್ ಮ್ಯಾಕ್ಸ್‌ಮುಲ್ಲರ್ ಅವರನ್ನು ನೋಡಿದಾಗ ಎಂಬುದಕ್ಕೆ ಅವರ ಮೇಲಿನ ಲೇಖನವೇ ಸಾಕ್ಷಿ. ಇಂತಹ ಮಹಾವ್ಯಕ್ತಿಗಳನ್ನು ನೋಡುವುದೇ ಸಂಸಾರದ ಮರಳುಕಾಡಿನಲ್ಲಿ ಸಂಚರಿಸುತ್ತಿರುವಾಗ ಸಿಕ್ಕುವ ಮರುವನಗಳು. 

 ಈ ಸಮಯದಲ್ಲಿ ಸ್ವಾಮೀಜಿ ತಮ್ಮ ಇಂಗ್ಲೆಂಡಿನ ಕೆಲಸದ ವಿಷಯವಾಗಿ\break ನ್ಯೂಯಾರ್ಕಿನ ಫ್ರಾನ್ಸಿಸ್ ಲೆಗಟ್ ಅವರಿಗೆ ಒಂದು ಪತ್ರ ಬರೆಯುವರು. ಆ ಸಮಯದಲ್ಲಿ ಅವರ ಭಾವನೆ ಮತ್ತು ಅವರ ಹೃದಯ ಎಂತಹ ಉತ್ತಮ ಮಟ್ಟದಲ್ಲಿತ್ತು ಎಂಬುದನ್ನು ನೋಡುವೆವು: 

 “ಕೆಲಸ ಇಂಗ್ಲೆಂಡಿನಲ್ಲಿ ಮೌನವಾದರೂ ನಿಜವಾಗಿಯೂ ಹಬ್ಬುತ್ತಿದೆ. ಇಷ್ಟೊಂದು ನ್ಯೂನತೆಗಳು ಇದ್ದರೂ ಬ್ರಿಟಿಷ್ ಚಕ್ರಾಧಿಪತ್ಯ ಆಲೋಚನೆಗಳನ್ನು ಹರಡುವುದಕ್ಕೆ ಹಿಂದೆ ಎಂದಿಗೂ ಇಲ್ಲದ ಒಂದು ಮಹಾದ್ಭುತ ಯಂತ್ರವಾಗಿದೆ. ಈ ಯಂತ್ರದ ಕೇಂದ್ರದಲ್ಲಿ ನಾನು ಆಲೋಚನೆಯನ್ನು ಹಾಕಬೇಕೆಂದು ಮನಸ್ಸು ಮಾಡಿಕೊಂಡಿರುವೆನು. ಅವು ಕ್ರಮೇಣ ಪ್ರಪಂಚದಲ್ಲೆಲ್ಲ ಹರಡುವುವು. ಎಲ್ಲಾ ಮಹಾಕಾರ್ಯಗಳೂ ಸಾಗುವುದು ಬಹಳ ನಿಧಾನ. ಅದಕ್ಕೆ ಇರುವ ಆತಂಕಗಳು ಬಹಳವೆನ್ನುವುದೇನೋ ನಿಜ. ಅದರಲ್ಲಿಯೂ ನಾವು ಹಿಂದೂಗಳು, ಪರಾಜಿತರು. ಆದರೂ ಅದು ಫಲಕಾರಿಯಾಗುವುದಕ್ಕೆ ಇದೇ ಕಾರಣ. ಆಧ್ಯಾತ್ಮಿಕ ಅನುಭವಗಳು ಯಾವಾಗಲೂ ಪದದಳಿತರಿಂದ ಬಂದವು. ಯಹೂದ್ಯರು ರೋಮನ್ ಚಕ್ರಾಧಿಪತ್ಯದ ಮೇಲೆ ತಮ್ಮ ಪ್ರಭಾವವನ್ನು ಬೀರಿದರು. ನಾನೂ ಕೂಡ ಪ್ರತಿದಿನ ತಾಳ್ಮೆಯಲ್ಲಿ, ಎಲ್ಲಕ್ಕಿಂತ ಹೆಚ್ಚಾಗಿ ಸಹಾನುಭೂತಿಯಲ್ಲಿ, ಪಾಠಗಳನ್ನು ಕಲಿಯುತ್ತಿರುವೆನೆಂಬುದನ್ನು ತಿಳಿದು ನಿನಗೆ ಸಂತೋಷವಾಗಿರಬಹುದು. ಪ್ರೌಢ ಪ್ರಚಂಡ ಆಂಗ್ಲೇಯರಲ್ಲಿಯೂ ಕೂಡ ನನಗೆ ದೇವತ್ವ ಕಾಣುವುದಕ್ಕೆ ಮೊದಲಾಗಿದೆ ಎಂದು ನಾನು ತಿಳಿಯುತ್ತೇನೆ. ನಾನು ಕ್ರಮೇಣ ಏನಾದರೂ ಒಂದು ಸೈತಾನ್ ಇದ್ದರೆ ಅವನನ್ನು ಕೂಡ ಪ್ರೀತಿಸುವ ಸ್ಥಿತಿಗೆ ಬರುತ್ತಿದ್ದೆ.” 

 “ನನಗೆ ಇಪ್ಪತ್ತು ವಯಸ್ಸಾದಾಗ ನಾನು ಅತ್ಯಂತ ಸಹಾನುಭೂತಿ ಹೀನನಾಗಿದ್ದೆ. ಎಂದಿಗೂ ರಾಜಿಯಾಗದ ಧರ್ಮಾಂಧನಾಗಿದ್ದೆ. ಕಲ್ಕತ್ತಾ ನಗರದಲ್ಲಿ ನಾಟಕಗಳಿರುವ ಹಾದಿಯ ಪಕ್ಕದಲ್ಲಿ ಕೂಡ ನಡೆಯುತ್ತಿರಲಿಲ್ಲ. ನನ್ನ ಮೂವತ್ತು ಮೂರನೇ ವಯಸ್ಸಿನಲ್ಲಿ ವೇಶ್ಯಾಂಗನೆಯರು ಇರುವ ಮನೆಯಲ್ಲೇ ವಾಸಮಾಡಬಲ್ಲೆ. ಅವರನ್ನು ಒಂದಾದರೂ ನಿಂದಾಸ್ಪದವಾದ ಮಾತುಗಳಿಂದ ನೋಯಿಸುವುದನ್ನು ಸಹಿಸಲಾರೆ. ಇದೇನು ಹೀನಸ್ಥಿತಿಯೊ! ಅಥವಾ ನಾನು ದೇವತಾಸ್ವರೂಪವಾದ ವಿಶ್ವಪ್ರೇಮಕ್ಕೆ ವಿಕಾಸವಾಗುತ್ತಿರುವೆನೋ! ತನ್ನ ಸುತ್ತಲೂ ಪಾಪವನ್ನು ನೋಡದೆ ಇದ್ದರೆ ಪುಣ್ಯಕೆಲಸಗಳನ್ನು ಮಾಡಲಾಗುವುದಿಲ್ಲವೆಂದೂ, ಹಣೆಯಲ್ಲಿ ಬರೆದದ್ದು ಆಗುವುದು ಎಂಬ ಸ್ವಭಾವಕ್ಕೆ ಬರುವನು ಎಂದೂ ನಾನು ಕೇಳಿರುವೆನು. ನಾನು ಪಾಪವನ್ನು ನೋಡುವುದಿಲ್ಲ. ಆದರೂ ನನ್ನ ಕೆಲಸಮಾಡುವ ಶಕ್ತಿ ಅಧಿಕವಾಗುತ್ತಿದೆ. ಅಧಿಕವಾಗಿ ಫಲಕಾರಿಯಾಗುತ್ತಿದೆ. ಕೆಲವು ದಿನ ನಾನು ಪರವಶನಾಗುವೆನು. ನಾನು ಎಲ್ಲರನ್ನೂ ಪ್ರತಿಯೊಂದು ವಸ್ತುವನ್ನೂ ಹರಸಬೇಕೆನ್ನಿಸುವುದು. ಎಲ್ಲವನ್ನೂ ಪ್ರೀತಿಸಿ ಆಲಿಂಗನ ಮಾಡಿಕೊಳ್ಳಬೇಕು ಎನ್ನಿಸುವುದು. ಪಾಪವೆಂಬುದು ಕೇವಲ ಭ್ರಮೆ ಎಂಬುದು ಈಗ ಕಾಣುತ್ತಿದೆ. ಈಗ ನಾನು ಅಂಥ ಭಾವಾವಸ್ಥೆಯಲ್ಲಿರುವೆನು. ನೀನು ಮತ್ತು ನಿನ್ನ ಶ‍್ರೀಮತಿ ತೋರಿದ ಪ್ರೀತಿ ಮತ್ತು ದಯೆಗಾಗಿ ಆನಂದಾಶ್ರುಗಳನ್ನು ಸುರಿಸುತ್ತಿರುವೆನು. ಧನ್ಯವಾಯಿತು ನಾನು ಹುಟ್ಟಿದ ದಿನ. ಅಷ್ಟೊಂದು ಪ್ರೀತಿ ಮತ್ತು ದಯೆ ನನಗೆ ಇಲ್ಲಿ ಸಿಕ್ಕಿದೆ. ನನ್ನ ಸೃಷ್ಟಿಗೆ ಕಾರಣವಾದ ಅನಂತ ಪ್ರೇಮವೇ, ನನ್ನ ಪ್ರತಿಯೊಂದು ಒಳ್ಳೆಯ ಮತ್ತು ಕೆಟ್ಟ ನಡತೆಯನ್ನು ಕಾದಿರುವುದು. ನಾನು ಈಗ ಇರುವುದು, ಹಿಂದೆ ಇದ್ದುದು ಅವನ ಕೈಯಲ್ಲಿ ಒಂದು ಯಂತ್ರವಲ್ಲದೆ ಮತ್ತೇನು? ನನ್ನ ಆಪ್ತ! ಆತನ ಸೇವೆಗಾಗಿ ನನ್ನ ಜೀವನವನ್ನು, ಸಂತೋಷವನ್ನು, ನನ್ನ ಸರ್ವಸ್ವವನ್ನೂ ಆನಂದದಿಂದ ಅರ್ಪಿಸುವೆನು. ಅವನು ನನ್ನೊಡನೆ ಆಟವಾಡುವ ಪ್ರಿಯತಮ, ನಾನು ಅವನ ಆಟಗಾರ. ಈ ಪ್ರಪಂಚದಲ್ಲಿ ಪ್ರಾಸವೂ ಇಲ್ಲ, ನಿಯಮವೂ ಇಲ್ಲ, ಯಾವ ನಿಯಮ ಅವನನ್ನು ಕಟ್ಟಬಲ್ಲದು? ಆಟಗಾರನವನು. ನಾಟಕದಲ್ಲಿ ಅಳುನಗುಗಳೊಡನೆ ಆಡುತ್ತಿರುವ. ಜ…ಹೇಳುವಂತೆ ಇದೊಂದು ದೊಡ್ಡ ವಿನೋದ! ದೊಡ್ಡ ವಿನೋದ! 

 “ಇದು ಒಂದು ವಿನೋದ ಪ್ರಪಂಚ. ಆ ಪ್ರಿಯತಮನೇ, ಆದಿ ಅಂತ್ಯರಹಿತನೆ, ನೀನು ನೋಡಿರುವುದರಲ್ಲೆಲ್ಲ ಅತ್ಯಂತ ವಿಚಿತ್ರವಾದ ವ್ಯಕ್ತಿ! ಇದು ಒಂದು ವಿನೋದ ಅಲ್ಲವೆ? ಸಹೋದರರೋ ಅಥವಾ ಆಟಗಾರರ ತಂಡವೋ ಕುಣಿದಾಡುವ ಪಾಠಶಾಲೆ ಹುಡುಗರನ್ನು ಜಗತ್ತೆಂಬ ಆಟದ ಮೈದಾನದಲ್ಲಿ ಬಿಟ್ಟಂತಿದೆ! ಯಾರನ್ನು ನಿಂದಿಸುವುದು, ಯಾರನ್ನು ಕೊಂಡಾಡುವುದು! ಎಲ್ಲಾ ಅವನ ಲೀಲೆ; ಅವರಿಗೆ ವಿವರಣೆಗಳು ಬೇಕಂತೆ! ಆದರೆ ಅವನನ್ನು ನೀನು ಹೇಗೆ ವಿವರಿಸುವುದು? ಅವನಿಗೆ ಮೆದುಳಿದೆಯೇ? ಅದು ಇಲ್ಲ. ವಿಚಾರವಾದರೂ ಇದೆಯೆ? ಅದೂ ಇಲ್ಲ. ಅವನು ನಮಗೆ ಸ್ವಲ್ಪ ಮೆದುಳು ಮತ್ತು ವಿಚಾರವನ್ನು ಕೊಟ್ಟು ಹುಚ್ಚರನ್ನಾಗಿ ಮಾಡುತ್ತಿರುವನು. ಆದರೆ ಈ ಸಲ ನಾನು ಅವನ ಬಲೆಗೆ ಬೀಳುವುದಿಲ್ಲ.” 

 “ನಾನು ಒಂದೆರಡು ವಿಷಯಗಳನ್ನು ಕಲಿತಿರುವೆನು. ಆಲೋಚನೆ ಪಾಂಡಿತ್ಯ ಮಾತುಗಾರಿಕೆ ಇವುಗಳಿಂದ ದೂರ, ಬಹಳ ದೂರದಲ್ಲಿ ‘ಪ್ರೇಮ’ ‘ಪ್ರಿಯತಮ’ ಎಂಬ ಭಾವ ಇರುವುದು. ಹೇ ಸಖಿ! ಬಟ್ಟಲನ್ನು ತುಂಬು, ಉನ್ಮತ್ತರಾಗೋಣ.” 

\newpage

 ಸ್ವಾಮೀಜಿಯವರು ಈ ಸಮಯದಲ್ಲಿ ಭಕ್ತಿಯೋಗವನ್ನು ಬೋಧಿಸುತ್ತಿದ್ದರು. ಅವರ ಹೃದಯ ಎಷ್ಟರಮಟ್ಟಿಗೆ ಅದರಿಂದ ಓತಪ್ರೋತವಾಗಿತ್ತು ಎಂಬುದನ್ನು ಈ ಪತ್ರದಲ್ಲಿ ಕಾಣುವೆವು. 

 ಸ್ವಾಮೀಜಿ ಇಂಗ್ಲೆಂಡಿನಲ್ಲಿ ವ್ಯಕ್ತಪಡಿಸಿದ ಶ್ರೇಷ್ಠವಾದ ಭಾವನೆಗಳು ವಿಕಸಿತ\break ಹೃದಯಕ್ಕೆ ಚೆನ್ನಾಗಿ ತಾಗಿದವು. ಅಂತಹ ವ್ಯಕ್ತಿಗಳು ಜೀವನದ ಎಲ್ಲಾ ಕಾರ‍್ಯಕ್ಷೇತ್ರ\-ದಲ್ಲಿಯೂ ಇರುವರು. ಆದರೆ ಅಂತಹವರ ಸಂಖ್ಯೆ ಯಾವಾಗಲೂ ವಿರಳ. ಅವರೇ ಒಂದು ಜನಾಂಗದ ಸಾರ. ಮೊದಲನೆ ವೇಳೆ ಸ್ವಾಮೀಜಿ ಇಂಗ್ಲೆಂಡಿಗೆ ಬಂದಿದ್ದಾಗ ಶ‍್ರೀಸ್ಟರ‍್ಡಿ, ಮಿಸ್ ಮುಲ್ಲರ್, ಮಾರ್ಗರೇಟ್ ನೋಬಲ್ ಮುಂತಾದವರ ಪರಿಚಯವಾಗಿತ್ತು. ಎರಡನೆಯ ವೇಳೆ ಸ್ವಾಮೀಜಿ ಬಂದಮೇಲೆ ಅವರೊಡನೆ ನಿಕಟವಾಗಿ ಬೆರೆತಮೇಲೆ ಅವರು ಸ್ವಾಮೀಜಿಯ ಜೀವನದ ಔನ್ನತ್ಯಕ್ಕೆ, ಸೌಂದರ್ಯಕ್ಕೆ ಬಾಗಿದರು. ಅವರನ್ನು ಗುರುವಿನಂತೆ ಸ್ವೀಕರಿಸಿದರು. ಅವರ ಕೆಲಸಕ್ಕಾಗಿ ತಮ್ಮ ಸರ್ವಸ್ವವನ್ನೂ ಧಾರೆಯೆರೆಯಲು ಸಿದ್ಧರಾದರು. ಎರಡನೆಯ ವೇಳೆ ಸ್ವಾಮೀಜಿ ಲಂಡನ್ನಿನಲ್ಲಿ ಮತ್ತೊಬ್ಬ ಇಂಗ್ಲೀಷ್ ದಂಪತಿಗಳ ಪರಿಚಯವನ್ನು ಮಾಡಿಕೊಂಡರು. ಅವರೇ ಸೇವಿಯರ್ಸ್‍‍ ದಂಪತಿಗಳು. ಈ ದಂಪತಿಗಳು ಹಲವು ವರ್ಷಗಳಿಂದ ಕ್ರೈಸ್ತಧರ್ಮದ ಸಂಕುಚಿತ ದೃಷ್ಟಿಗೆ ಬೇಜಾರಾಗಿ ಬೆಳಕಿಗಾಗಿ ಅಲೆಯುತ್ತಿದ್ದರು. ಅಧ್ಯಾತ್ಮ ಮತ್ತು ಧಾರ್ಮಿಕ ಉಪನ್ಯಾಸಗಳನ್ನು ಹಲವರಿಂದ ಕೇಳಿದ್ದರು. ಆದರೆ ಅವರಿಗೆ ಎಲ್ಲಿಯೂ ತೃಪ್ತಿ ಸಿಕ್ಕಿರಲಿಲ್ಲ. ಒಂದು ದಿನ ಸ್ವಾಮೀಜಿಯವರ ಉಪನ್ಯಾಸವನ್ನು ದಂಪತಿಗಳಿಬ್ಬರೂ ಕೇಳಿದರು. ಆ ಉಪನ್ಯಾಸವಾದ ಮೇಲೆ “ಇಂತಹ ವ್ಯಕ್ತಿಯನ್ನೇ, ಇಂತಹ ತತ್ತ್ವವನ್ನೇ ಇದುವರೆವಿಗೂ ನಾವು ಹುಡುಕುತ್ತಿದ್ದುದು. ಆದರೆ ಅದು ಫಲಿಸಿರಲಿಲ್ಲ. ಇಂದು ಫಲಿಸಿತು” ಎಂದು ಆನಂದಿಸಿದರು. ಸ್ವಾಮೀಜಿಯವರ ಅದ್ವೈತ ದೃಷ್ಟಿ ಈ ದಂಪತಿಗಳಿಗೆ ತುಂಬಾ ಹಿಡಿಸಿತು. ಅವರಿಗೆ ಮಕ್ಕಳಿರಲಿಲ್ಲ. ಹಲವು ವೇಳೆ ದೇವರಿಗೆ ಅದಕ್ಕಾಗಿ ಪ್ರಾರ್ಥಿಸಿದ್ದರು. ಆದರೆ ಬೇರೆ ವಿಧದಲ್ಲಿ ಅವರಿಗೆ ದೇವರು ಒಂದು ಪುತ್ರರತ್ನವನ್ನು ಕೊಟ್ಟನು. ಸ್ವಾಮೀಜಿಯವರಿಗೆ ಶ‍್ರೀಮತಿ ಸೇವಿಯರ್ಸ್‍‍ ಅವರ ಪರಿಚಯವಾದ ಮೇಲೆ ಅವರನ್ನು ತಾಯಿ ಎಂದು ಕರೆಯಲು ಪ್ರಾರಂಭಿಸಿದರು. ಆ ದಂಪತಿಗಳು ಕೂಡ ವಿವೇಕಾನಂದರನ್ನು ತಮ್ಮ ಮಗನಂತೆ ನೋಡಿಕೊಂಡರು. ಆಧಾತ್ಮಿಕ ಜೀವನದಲ್ಲಿ ಈ ವೃದ್ಧ ದಂಪತಿಗಳು ಸ್ವಾಮಿ ವಿವೇಕಾನಂದರನ್ನು ಗುರುವಾಗಿ ಸ್ವೀಕರಿಸಿದರೂ ಮಾನವ ಸಂಬಂಧದ ದೃಷ್ಟಿಯಿಂದ ಮಗನಂತೆ ನೋಡಿಕೊಂಡರು. ಸ್ವಾಮಿ ವಿವೇಕಾನಂದರು ಹಿಮಾಲಯದಲ್ಲಿ ಅದ್ವೈತ ಆಶ್ರಮವನ್ನು ಸ್ಥಾಪಿಸಿದುದಕ್ಕೆ ಈ ದಂಪತಿಗಳೇ ಹಣವನ್ನೆಲ್ಲಾ ಕೊಟ್ಟರು. ತಮ್ಮ ತಾಯ್ನಾಡಾದ ಇಂಗ್ಲೆಂಡನ್ನು ಬಿಟ್ಟು ಹಿಮಾಲಯದಲ್ಲಿ ನೆಲಸತೊಡಗಿದರು. ಅಲ್ಲಿಯೇ ಶ‍್ರೀಮಾನ್ ಸೇವಿಯರ್ ಅನಂತರ ಶ‍್ರೀಮತಿ ಸೇವಿಯರ್ ಕಾಲವಾದರು. ಈ ದಂಪತಿಗಳು ವಿವೇಕಾನಂದರ ಕಾರ್ಯದಲ್ಲಿ ಒಂದು ಮುಖ್ಯವಾದ ಅಧ್ಯಾಯದಂತೆ ಇರುವರು. ಅದರಂತೆಯೇ ಮಾರ್ಗರೇಟ್ ನೋಬಲ್ ಎಂಬಾಕೆ. ಅನಂತರ ಇಂಡಿಯಾದೇಶಕ್ಕೆ ಬಂದು ಸ್ವಾಮೀಜಿಯವರಿಂದ ಸೋದರಿ ನಿವೇದಿತಾ ಎಂದು ನಾಮಧೇಯವನ್ನು ಸ್ವೀಕರಿಸಿ, ಭರತಖಂಡದ ಪುನರುದ್ಧಾರದ ಕೆಲಸದಲ್ಲಿ ಒಂದು ಅಗ್ರಗಣ್ಯ ವ್ಯಕ್ತಿಯಾಗಿ ತನ್ನ ಬಾಳನ್ನು ಅರ್ಪಿಸಿರುವಳು. ಭರತಖಂಡದ ಹಲವು ಕಲಾವಿದರು, ವಿಜ್ಞಾನಿಗಳು, ಸಾಹಿತಿಗಳು ಭರತಖಂಡದ ಮಹಿಮೆಗೆ ಕಣ್ಣು ತೆರೆಯುವಂತೆ ಆಕೆ ಮಾಡಿದಳು. ಆಕೆ ನಮ್ಮನ್ನಾಳುವವರ ದೇಶದಿಂದ ಬಂದರೂ ಭರತಖಂಡವನ್ನು ತನ್ನ ಸಾಕುತಾಯಿಯಂತೆ ಸ್ವೀಕರಿಸಿದಳು. ಈ ಸಾಕುಮಗಳು ನಮ್ಮ ದೇಶವನ್ನು ಪ್ರೀತಿಸಿದಂತೆ, ಸ್ವಂತ ಮಗಳೂ ಪ್ರೀತಿಸಿರಲಾರಳು. ಸ್ವಾಮಿ ವಿವೇಕಾನಂದರು ಇಂಗ್ಲೆಂಡಿನಲ್ಲಿ ಇನ್ನು ಏನನ್ನು ಮಾಡದೆ ಇದ್ದರೂ ಮಾರ್ಗರೆಟ್ ನೋಬಲ್ ಎಂಬಾಕೆಯನ್ನು ಭರತಖಂಡಕ್ಕೆ ತಂದು ಆಕೆಯನ್ನು ಭರತಖಂಡದ ಸೇವೆಗೆ ಬದ್ಧಕಂಕಣಳಾಗುವಂತೆ ಮಾಡಿದ್ದು ಒಂದು ಅದ್ಭುತವಾದ ಕೆಲಸ. 

 ಸ್ವಾಮೀಜಿಯವರಿಗೆ ಇಂಗ್ಲೆಂಡಿನಲ್ಲಿ ಅವರ ಶಿಷ್ಯರಲ್ಲಿ ಅತ್ಯಂತ ಪ್ರಮುಖರಾದವರು ಕೆಲವರು ದೊರೆತರು. ಅವರೇ ಮಿಸ್ ಮುಲ್ಲರ್, ಇ.ಟಿ. ಸ್ಟರ‍್ಡಿ, ಜೆ.ಜೆ.ಗುಡ್‍ವಿನ್, ಸೇವಿಯರ್ಸ್‍‍ ದಂಪತಿಗಳು ಮತ್ತು ಮಾರ್ಗರೇಟ್ ನೋಬಲ್ ಎಂಬುವರು. 

 ಸ್ವಾಮೀಜಿಯವರಿಗೆ ಇಂಗ್ಲೆಂಡಿನಲ್ಲಿ ಅವಿಶ್ರಾಂತರಾಗಿ ವೇದಾಂತದ ಭಾವನೆಗಳನ್ನು ಹರಡಲು ದುಡಿದು ಸಾಕಾಯಿತು. ಅವರಿಗೆ ಕೆಲವು ಕಾಲ ವಿರಾಮ ಅತ್ಯಂತ ಆವಶ್ಯಕ ಎಂದು ಭಾವಿಸಿ ಸೇವಿಯರ್ಸ್‍‍ ದಂಪತಿಗಳು ಮತ್ತು ಮಿಸ್ ಮುಲ್ಲರ್ ಸ್ವಾಮೀಜಿಯವರನ್ನು ಸ್ವಿಟ್ಜರ್‌ಲೆಂಡಿಗೆ ಕೆಲವು ಕಾಲ ಕರೆದುಕೊಂಡು ಹೋಗುವುದಾಗಿ ಹೇಳಿದರು. ಆಗ ಸ್ವಾಮೀಜಿಯವರೂ ಹಿಮದಿಂದ ಆವೃತವಾದ ಶಿಖರವನ್ನು ನೋಡಬೇಕು, ಅವುಗಳ ಮಧ್ಯದಲ್ಲಿ ನಿರ್ಜನಪ್ರದೇಶದಲ್ಲಿ ಸಂಚಾರಮಾಡಬೇಕು ಎಂದು ಆಶಿಸಿದರು. ಇಂಗ್ಲೆಂಡಿನಿಂದ ಜುಲೈ ತಿಂಗಳ ಕೊನೆಯಲ್ಲಿ ಸ್ವಾಮೀಜಿ ತಮ್ಮ ಮೂರು ಜನ ಶಿಷ್ಯರೊಡನೆ ಜಿನೀವಾಕ್ಕೆ ಹೊರಟರು. ಮಧ್ಯೆ ಪ್ಯಾರಿಸಿನಲ್ಲಿ ಒಂದು ದಿನ ತಂಗಿ ಅನಂತರ ಜಿನೀವ ಸೇರಿದರು. ಅಲ್ಲಿ ಸರೋವರದ ತೀರದಲ್ಲಿ ಒಂದು ಹೋಟೆಲಿನಲ್ಲಿದ್ದರು. ಆ ಪ್ರಶಾಂತ ವಾತಾವರಣ, ಸರೋವರದ ಆ ನೀಲಿಯ ಬಣ್ಣ, ಬೀಸು ತಂಗಾಳಿ, ಸುತ್ತಲೂ ಇರುವ ಮರಗಿಡಗಳು ಸ್ವಾಮೀಜಿ ಮನಸ್ಸಿಗೆ ಆಹ್ಲಾದಕರವಾಗಿದ್ದವು. 

 ಜಿನೀವ ಸರೋವರ ಈಜುವುದಕ್ಕೆ ಪ್ರಸಿದ್ಧವಾದ ಸ್ಥಳ. ಸ್ವಾಮೀಜಿ ಅಲ್ಲಿ ಎರಡು ಸಲ ಈಜಿದರು. ಹತ್ತಿರ ಇದ್ದ ಪುರಾತನವಾದ ಒಂದು ಕೋಟೆಯನ್ನು ನೋಡಿಕೊಂಡು ಬಂದರು. ಮೂರು ದಿನಗಳಾದ ಮೇಲೆ ಅಲ್ಲಿಂದ ನಲವತ್ತು ಮೈಲಿ ದೂರದಲ್ಲಿ ಆಲ್ಫ್ಸ್ ಪರ್ವತದ ತಪ್ಪಲಲ್ಲಿರುವ ಚಮೋನಿಕ್ಸ್ ಎಂಬ ಊರಿಗೆ ಹೋದರು. ಅಲ್ಲಿಗೆ ಹೊದಾಗ ಆಲ್ಫ್ಸ್ ಪರ್ವತದ ಮೌಂಟ್‍ಬ್ಲಾಕ್ ಎಂಬ ಶಿಖರ ಆಕಾಶವನ್ನು ಭೇದಿಸಿಕೊಂಡು ನಿಂತಿರುವುದನ್ನು ಕಂಡರು. ಸ್ವಾಮೀಜಿ ಆ ಸೌಂದರ್ಯಕ್ಕೆ ಮನಸೋತರು. ಹಿಮಾಲಯದಲ್ಲಿಯೂ ಇಂತಹ ದೃಶ್ಯವಿಲ್ಲ ಎಂದು ಹೇಳಿದರು. ಸ್ವಾಮೀಜಿ ಮೌಂಟ್‍ಬ್ಲಾಕ್ ಶಿಖರವನ್ನು ಹತ್ತಬೇಕೆಂದು ಬಯಸಿದರು. ಆದರೆ ಅವರಿಗೆ ದಾರಿತೋರುವವನು, ಅದು ಬಹಳ ಕಷ್ಟವೆಂದೂ ಪರಿಣತರಾದ ಪರ್ವತಾರೋಹಿಗಳಿಗೆ ಮಾತ್ರ ಸಾಧ್ಯವೆಂದೂ\break ಹೇಳಿದಾಗ ಅದನ್ನು ಕೈಬಿಟ್ಟರು. ಆದರೆ ಆಲ್ಫ್ಸ್ ಪರ್ವತಕ್ಕೆ ಬಂದು ಒಂದು (ನೀರ್ಗಲ್ಲ ಪ್ರವಾಹ) ಕೂಡ ದಾಟದೆ ಹೋದರೆ ಬಂದದ್ದು ಸಾರ್ಥಕವಿಲ್ಲ ಎಂದು ಹೇಳಿದಾಗ, ಹತ್ತಿರ ಇದ್ದ ಮರ್‌ಡಿಗ್ಲೇಸ್ ಎಂಬಲ್ಲಿಗೆ ಕಷ್ಟಪಟ್ಟುಕೊಂಡು ಹೋಗಿಬಂದರು. ಇಂತಹ ಒಂದು ಸ್ಥಳದಲ್ಲಿ ಒಂದು ಮಠವಿದ್ದಿದ್ದರೆ ಎಷ್ಟು ಚೆನ್ನಾಗಿರುವುದು ಎಂದು ಸೇವಿಯರ್ಸ್‍‍ ದಂಪತಿಗಳಿಗೆ ಹೇಳಿದರು. ಇದರ ಪರಿಣಾಮವಾಗಿಯೇ ಸೇವಿಯರ್ಸ್‍‍ ಅವರು ಹಿಮಾಲಯದಲ್ಲಿ ಅದ್ವೈತ ಆಶ್ರಮವನ್ನು ಕಟ್ಟಿಸಿದರು. 

 ಚಮೋನಿಕ್ಸ್‌ನಿಂದ ಸ್ವಾಮೀಜಿ ಪಂಗಡದವರು ಸೈಂಟ್ ಬರ್‌ನಾರ್ಡ್ ಎಂಬಲ್ಲಿಗೆ ಹೋದರು. ಅಲ್ಲಿ ಸ್ವಲ್ಪ ಮೇಲೆಯೇ ಪ್ರಖ್ಯಾತವಾದ ಸೈಂಟ್ ಬರ್‌ನಾರ್ಡ್ ಕಣಿವೆ ಇರುವುದು. ಅದರ ಮೇಲೆಯೇ ಸಾಧು ಅಗಸ್ಟೈನನ ಅನುಯಾಯಿಗಳ ಒಂದು ಮಠ ಇರುವುದು. ಹತ್ತಿರದ ಒಂದು ಹಳ್ಳಿಯಲ್ಲಿ ಹದಿನೈದು ದಿನಗಳು ಇದ್ದರು. ಸುತ್ತಲೂ ಹಿಮಾವೃತ ಆಲ್ಫ್ಸ್ ಪರ್ವತ, ಸರಳವಾದ ಹಳ್ಳಿಯ ಜನರು, ಮತ್ತಾವ ದೊಡ್ಡ ನಗರದ ಗಡಿಬಿಡಿಯೂ ಇರಲಿಲ್ಲ. ಅನೇಕ ವೇಳೆ ಸ್ವಾಮೀಜಿ ಅವರೊಬ್ಬರೆ ಆ ಪರ್ವತ ಪ್ರಾಂತ್ಯದ ರಸ್ತೆಯಲ್ಲಿ ಆತ್ಮಾರಾಮರಾಗಿ ಸಂಚರಿಸುತ್ತಿದ್ದರು. ಸ್ವಾಮೀಜಿ ಆ ವಾತಾವರಣದಲ್ಲಿ ಯಾವಾಗಲೂ ಧ್ಯಾನ ಮತ್ತು ಮನನದಲ್ಲಿ ಕಳೆಯುತ್ತಿದ್ದರು. ಕೇವಲ ಸ್ಪರ್ಶದಿಂದ, ನೋಟ ಮಾತ್ರದಿಂದ ಆಧ್ಯಾತ್ಮಿಕತೆಯನ್ನು ಒಂದು ಜೀವಿಗೆ ನೀಡುವ ಸ್ಥಿತಿ ಸ್ವಾಮೀಜಿಗೆ ಅಲ್ಲಿದ್ದಾಗ ಪ್ರಾಪ್ತವಾಗಿತ್ತೆಂದು ಅವರ ಹತ್ತಿರ ಇದ್ದವರು ಹೇಳುತ್ತಿದ್ದರು. 

 ಸ್ವಾಮೀಜಿ ಆಲ್ಫ್ಸ್ ಪರ್ವತಗಳ ಪ್ರಾಂತ್ಯಗಳಲ್ಲಿ ತಮ್ಮ ಜೀವನದ ಕೆಲವು ಶ್ರೇಷ್ಠ ದಿನಗಳನ್ನು ಕಳೆದರು. ಒಂದು ದಿನ ಹಿಮದ ಮೇಲೆ ನಡೆದುಕೊಂಡು ಹೋಗುತ್ತಿದ್ದಾಗ ಸ್ವಾಮೀಜಿ ಉಪನಿಷತ್ತಿನಿಂದ ಕೆಲವು ಶ್ಲೋಕಗಳನ್ನು ಶಿಷ್ಯರಿಗೆ ಹೇಳಿ ವಿವರಿಸುತ್ತಿದ್ದರು. ಯಾವುದೋ ಕೆಲಸಕ್ಕಾಗಿ ಅವರು ಸ್ವಲ್ಪ ಹಿಂದೆ ಬಿದ್ದರು. ಸ್ವಲ್ಪ ಕಾಲದ ಮೇಲೆ\break ಸೇವಿಯರ್ಸ್‍‍ ದಂಪತಿಗಳು ಮತ್ತು ಮಿಸ್ ಮುಲ್ಲರ್ ಅವರ ಸಮೀಪಕ್ಕೆ ಓಡಿಬರುತ್ತಿದ್ದರು. ಅನಂತರ ಅವರು ಹಿಮದ ಒಂದು ಬಿರುಕಿನಲ್ಲಿ ತಾವು ಬೀಳುವುದರಲ್ಲಿ ಇದ್ದೆವೆಂದೂ ದೈವವಶಾತ್ ತಪ್ಪಿಸಿಕೊಂಡೆವೆಂದೂ ಹೇಳಿದರು. ಅಂದಿನಿಂದ ಸ್ವಾಮೀಜಿಯವರ ಜೊತೆಯಲ್ಲಿ ಅವರು ಸಂಚಾರ ಮಾಡುತ್ತಿದ್ದಾಗ ಯಾವಾಗಲೂ ಯಾರಾದರೂ ಜೊತೆಯಲ್ಲಿ ಇರುತ್ತಿದ್ದರು. ಅಲ್ಲಿಂದ ಅವರಿದ್ದ ಮನೆಯ ಕಡೆ ಬರುತ್ತಿದ್ದಾಗ ಒಂದು ಸಣ್ಣ ಗುಡ್ಡದ ಮೇಲೆ ಒಂದು ಚರ್ಚ್ ಕಂಡಿತು. ಸ್ವಾಮೀಜಿ ಅಂತಹ ಸುಂದರವಾದ ಸ್ಥಳದಲ್ಲಿದ್ದ ಚರ್ಚನ್ನು ನೋಡಿ ಸಂತೋಷಪಟ್ಟರು. ಹತ್ತಿರ ಇದ್ದ ಕೆಲವು ಪುಷ್ಪಗಳನ್ನು ಕಿತ್ತು\break ಸೇವಿಯರ್ಸ್‍‍ ಕೈಗೆ ಕೊಟ್ಟು ನನ್ನ ಪರವಾಗಿ ಇದನ್ನು ಮೇರಿಗೆ ಅರ್ಪಿಸು ಎಂದರು. “ಇದು ನನ್ನ ಪ್ರೀತಿಯ ಮತ್ತು ಭಕ್ತಿಯ ಕಾಣಿಕೆ. ಅವಳೇ ತಾಯಿಯೂ ಆಗಿರುವಳು. ಕ್ರೈಸ್ತನಲ್ಲದೆ ಇರುವುದರಿಂದ ಆ ಪುಷ್ಪವನ್ನು ನಾನು ಅರ್ಪಿಸಿದರೆ ಚರ್ಚಿನಲ್ಲಿರುವವರು ಆಕ್ಷೇಪಣೆ ತೆಗೆದಾರು” ಎಂದು ಸ್ವಾಮೀಜಿ ಇನ್ನೊಬ್ಬರ ಕೈಯಲ್ಲಿ ಕೊಟ್ಟರು. 

\newpage

 ಸ್ವಾಮೀಜಿ ಈ ಆಲ್ಫ್ಸ್ ಪರ್ವತದ ಹಳ್ಳಿಯಲ್ಲಿದ್ದಾಗ ಪಾಲ್ ಡಾಯ್‍ಸನ್ ಎಂಬ ಕೈಲ್ ವಿಶ್ವವಿದ್ಯಾನಿಲಯದ ತತ್ತ್ವಶಾಸ್ತ್ರ ಪ್ರಾಧ್ಯಾಪಕರು ಅವರಿಗೆ ಕೈಲಿಗೆ ಬರಬೇಕೆಂದು ನಿಮಂತ್ರಣವನ್ನು ಕಳುಹಿಸಿದ್ದರು. ಅವರು ವೇದಾಂತಕ್ಕೆ ಆಕರ್ಷಿತರಾದವರು, ಕೆಲವು ಗ್ರಂಥಗಳನ್ನೂ ಅದರ ಮೇಲೆ ಬರೆದಿದ್ದರು. ಸ್ವಾಮೀಜಿಯವರೊಡನೆ ಹಲವು\break ವಿಷಯಗಳ ಮೇಲೆ ಚರ್ಚೆ ಮಾಡಬೇಕೆಂದು ಇದ್ದರು. ಈ ಪತ್ರವನ್ನು ಸ್ವೀಕರಿಸಿ ಆದಮೇಲೆ ಮತ್ತೆ ಕೆಲವು ಸ್ಥಳಗಳನ್ನು ನೋಡಿಕೊಂಡು ಕೈಲಿಗೆ ಹೋಗಬೇಕೆಂದು ಸ್ವಾಮೀಜಿ ನಿರ್ಧರಿಸಿದರು. 

 ಸ್ವಾಮೀಜಿ ಅವರು ಲಸ್ಸರ‍್ನಿಗೆ ಬಂದರು. ಅಲ್ಲಿ ರಿಗಿ ಎಂಬ ಬೆಟ್ಟದ ಮೇಲಕ್ಕೆ ಬೆಟ್ಟದ ರೈಲಿನಲ್ಲಿ ಹೋದರು. ಅಲ್ಲಿಂದ ಆಲ್ಫ್ಸ್ ಬೆಟ್ಟದ ಅತ್ಯಂತ ಸುಂದರವಾದ ದೃಶ್ಯಗಳು ಕಾಣುತ್ತವೆ. ಅಲ್ಲಿಂದ ಮಿಸ್ ಮುಲ್ಲರ್ ಕಾರ‍್ಯ ನಿಮಿತ್ತ ಇಂಗ್ಲೆಂಡಿಗೆ ಹೋಗಬೇಕಾಗಿ ಬಂತು. ಸ್ವಾಮೀಜಿಯಿಂದ ಆಕೆ ಬೀಳ್ಕೊಂಡಳು. ಸೇವಿಯರ್ಸ್‍‍ ದಂಪತಿಗಳೊಡನೆ ಸ್ವಾಮೀಜಿ ಸ್ವಿಟ್ಜರ್‌ಲೆಂಡಿನಲ್ಲಿ ಅತ್ಯಂತ ರಮಣೀಯ ಸ್ಥಳಗಳಲ್ಲಿ ಒಂದಾದ ಜೆರಮಾಟ್ ಎಂಬಲ್ಲಿಗೆ ಹೋದರು. ಸ್ವಾಮೀಜಿ ಅಲ್ಲಿ ಕೋರ್‌ನೀರ್‌ಗಾಟ್ ಎಂಬ ಪರ್ವತವನ್ನು ಹತ್ತಿ ಅಲ್ಲಿಂದ ಮೆಟರ್‌ಹಾರನ್ ಶಿಖರವನ್ನು ನೋಡಬೇಕೆಂದು ಬಯಸಿದರು. ಅವರಲ್ಲಿ ಸೇವಿಯರ್ಸ್‍‍ ಮಾತ್ರ ಮೇಲಿನ ತನಕ ಹೋಗಲು ಸಾಧ್ಯವಾಯಿತು. ಉಳಿದ ಇಬ್ಬರಿಗೆ ಹತ್ತಿ ಹೋಗಲು ಸಾಧ್ಯವಾಗಲಿಲ್ಲ. ಅನಂತರ ರೈನ್ ನದಿಯು ಬೀಳುವ ಸುಂದರವಾದ ಟೇಪ್‍ಹೌಸಿನ ಜಲಪಾತವನ್ನು ನೋಡಿದರು. 

 ಅನಂತರ ಜರ್ಮನಿಯ ಮುಖ್ಯ ವಿಶ್ವವಿದ್ಯಾನಿಲಯದ ಕೇಂದ್ರವಾದ ಹೈಡಲ್‍ಬರ್ಗಿಗೆ ಹೋದರು. ಅಲ್ಲಿ ಎರಡು ದಿನಗಳು ಇದ್ದರು. ಅಲ್ಲಿರುವ ಕೋಟೆ ಮತ್ತು ವಿಶ್ವವಿದ್ಯಾನಿಲಯಗಳನ್ನು ನೋಡಿಕೊಂಡು ಕೋಬ್ಲೆಂಜ್‍ಗೆ ಹೋಗಿ ಅಲ್ಲಿ ರಾತ್ರಿ ತಂಗಿದರು. ಮಾರನೆಯ ದಿನ ಬೆಳಗ್ಗೆ ರೈನ್ ನದಿಯ ಮೇಲೆ ಸ್ಟೀಮರ್‌ನಲ್ಲಿ ಕೋಲೋಗ್ನಿಗೆ ಹೋದರು. ಸ್ವಾಮೀಜಿ ಪಾರ್ಟಿ ಅಲ್ಲಿ ಕೆಲವು ದಿನಗಳಿದ್ದರು. ಅಲ್ಲಿಯ ಕೋಟೆ ಕೊತ್ತಲಗಳನ್ನು ನೋಡಿದರು. ಒಂದು ದಿನ ಅಲ್ಲಿಯ ಚರ್ಚಿನಲ್ಲಿ ಆಗುವ ಪೂಜೆಯನ್ನು ನೋಡಿದರು. ಸೇವಿಯರ್ಸ್‍‍ ದಂಪತಿಗಳು ಸ್ವಾಮೀಜಿಯವರನ್ನು ಕೋಲೋಗ್ನಿಯಿಂದ ಕೈಲಿಗೆ ಕರೆದುಕೊಂಡು ಹೊಗಬೇಕೆಂದು ಇದ್ದರು. ಆದರೆ ಸ್ವಾಮೀಜಿ ಇತಿಹಾಸ ಪ್ರಖ್ಯಾತವಾದ ಬರ್ಲಿನ್ ನಗರವನ್ನು ನೋಡಬೇಕೆಂದು ಇಚ್ಛಿಸಿದುದರಿಂದ ಅವರನ್ನು ಬರ್ಲಿನ್‍ಗೆ ಕರೆದುಕೊಂಡು ಹೋದರು. ಸ್ವಾಮೀಜಿ ಬರ್ಲಿನ್ ನಗರದ ವಿಶಾಲವಾದ ರಸ್ತೆಗಳು, ಉದ್ಯಾನವನಗಳು, ಪ್ರಖ್ಯಾತವಾದ ಸ್ಮಾರಕಗಳು, ಇವುಗಳನ್ನು ತುಂಬಾ ಕೊಂಡಾಡಿ ಇದು ಪ್ಯಾರಿಸ್‍ನಷ್ಟೆ ಸುಂದರವಾಗಿದೆ ಎಂದು ಹೇಳಿದರು. ಅಲ್ಲಿಂದ ಕೈಲ್‍ನಗರಕ್ಕೆ ಹೋದರು. ಡಾಯಸನ್ ಅವರೊಡನೆ ಮಾತುಕತೆಯಾಡಿದರು. ಆತ ಅದ್ವೈತ ವೇದಾಂತ ಪ್ರೇಮಿ. ಉಪನಿಷತ್ತು, ಬ್ರಹ್ಮಸೂತ್ರ ಮತ್ತು ಶಂಕರಾಚಾರ್ಯರ ಭಾಷ್ಯಗಳ ಮೇಲೆ ಪ್ರತಿಷ್ಠಿತವಾಗಿರುವ ವೇದಾಂತದರ್ಶನ ತಾತ್ತ್ವಿಕ ಪ್ರಪಂಚದಲ್ಲಿ ಒಂದು ಅದ್ಭುತವಾದ ಸೌಧ ಎಂದು ಅಭಿಪ್ರಾಯಪಟ್ಟರು. 

 ಸ್ವಾಮೀಜಿ ಡಾಯ್‍ಸನ್‍ರ ಮನೆಯಲ್ಲಿದ್ದಾಗ ಒಂದು ಪದ್ಯದ ಪುಸ್ತಕವನ್ನು\break ಓದುತ್ತಿದ್ದರು. ಆಗ ಡಾಯ್‍ಸನ್‍ ಅವರೊಡನೆ ಮಾತನಾಡಿದಾಗ ಸ್ವಾಮೀಜಿ ತಾವು ಯಾವುದನ್ನೋ ಓದುವುದರಲ್ಲಿ ತಲ್ಲೀನರಾಗಿದ್ದುದರಿಂದ ಅದು ಕೇಳಲಿಲ್ಲ ಎಂದರು. ಡಾಯ್‍ಸನ್‍ಗೆ ಇದು ತೃಪ್ತಿಕರವಾದ ವಿವರಣೆಯಂತೆ ಕಾಣಲಿಲ್ಲ. ಸ್ವಲ್ಪ ಹೊತ್ತಿನ ಮೇಲೆ ಮಾತನಾಡುತ್ತಿದ್ದಾಗ ಸ್ವಾಮೀಜಿ ಆಗ ತಾನೆ ಓದುತ್ತಿದ್ದ ಪುಸ್ತಕದಿಂದ ಉದಾಹರಿಸತೊಡಗಿದರು. ಅದನ್ನು ಕೇಳಿ ಡಾಯ್‍ಸನ್‍ಗೆ ಆಶ್ಚರ್ಯವಾಯಿತು. ಸ್ವಾಮೀಜಿ ಸ್ವಲ್ಪ ಹೊತ್ತಿನ ಹಿಂದೆ ಓದಿದುದನ್ನು ಪುಸ್ತಕದಲ್ಲಿ ಇದ್ದಂತೆಯೇ ವಿವರಿಸುತ್ತಿದ್ದರು.\break ಡಾಯ್‍ಸನ್‍ ಇದು ಹೇಗೆ ಸಾಧ್ಯ ಎಂದಾಗ, ಸ್ವಾಮೀಜಿ ತಾವು ಓದುವಾಗ ಅದರಲ್ಲಿ ತನ್ಮಯರಾಗುವುದಾಗಿಯೂ, ಆ ಸಮಯದಲ್ಲಿ ಒಂದು ಉರಿಯುತ್ತಿರುವ ಕೆಂಡವನ್ನು ತಮ್ಮ ಮೇಲೆ ಇಟ್ಟರೂ ಗೊತ್ತಾಗುವುದಿಲ್ಲವೆಂದೂ ಹೇಳಿದರು. ಹಾಗೆ ತನ್ಮಯರಾಗಿ ಓದುವುದರಿಂದಲೇ ಅದರಲ್ಲಿರುವ ವಿಷಯಗಳನ್ನೆಲ್ಲ ಅವರು ಅಷ್ಟು ಬೇಗ ಗ್ರಹಿಸಲು ಸಾಧ್ಯ ಎಂದು ಹೇಳಿದರು. 

 ಈ ಸಮಯದಲ್ಲಿ ಕೈಲ್ ನಗರದಲ್ಲಿ ಜರ್ಮನಿಯ ದೇಶದ ಒಂದು ವಸ್ತು ಪ್ರದರ್ಶನ ನಡೆಯುತ್ತಿತ್ತು. ಡಾಯ್‍ಸನ್‍ ಸ್ವಾಮೀಜಿಯವರನ್ನು ಅಲ್ಲಿಗೆ ಕರೆದುಕೊಂಡು ಹೋಗಿ ತೋರಿಸಿದರು. ಮಾರನೆ ದಿನ ಆಗತಾನೆ ಕೈಸರ್ ತೆರೆದ ಕೈಲ್ ಬಂದರನ್ನು ನೋಡಿ ಬಂದರು. ಅಷ್ಟು ಹೊತ್ತಿಗೆ ಪರ್ಯಟನದಲ್ಲಿ ಸ್ವಾಮೀಜಿ ಆರು ವಾರಗಳನ್ನು ಕಳೆದಿದ್ದರು. ಅವರ ಆರೋಗ್ಯ ಸ್ಥಿತಿ ಉತ್ತಮಗೊಂಡಿತ್ತು. ಲಂಡನ್ನಿಗೆ ಹೋಗಿ ಪ್ರವಚನಗಳ ಕೆಲಸವನ್ನು ಮತ್ತೆ ಆರಂಭಮಾಡುವುದಕ್ಕೆ ಕಾತರರಾಗಿದ್ದರು. ಪಾಲ್ ಡಾಯ್‍ಸನ್‍ ಸ್ವಾಮೀಜಿ ಜತೆಯಲ್ಲಿ ಇನ್ನು ಕೆಲವು ತಾತ್ತ್ವಿಕ ವಿಷಯಗಳನ್ನು ಚರ್ಚಿಸಬೇಕೆಂದು ಇದ್ದರು. ಆದರೆ ಸ್ವಾಮೀಜಿ ಅವರು ಲಂಡನ್ನಿಗೆ ಹೋಗಲು ನಿಶ್ಚಯಮಾಡಿರುವಾಗ ಆತ ತಾನು ಲಂಡನ್ನಿಗೆ ಬಂದು ಸ್ವಾಮಿಗಳೊಡನೆ ಮಾತುಕತೆಯಾಡುತ್ತೇನೆ ಎಂದು ಹ್ಯಾಮ್‍ಬರ್ಗಿನಲ್ಲಿ ಸ್ವಾಮೀಜಿ ಮತ್ತು ಸೇವಿಯರ್ಸ್‍‍ ದಂಪತಿಗಳನ್ನು ಸೇರಿಕೊಂಡರು. ಅನಂತರ ಎಲ್ಲರೂ ಲಂಡನ್ನಿಗೆ ಹೋದರು. 

 ಶಾರದಾನಂದರಿಗೆ ಪಾಶ್ಚಾತ್ಯ ದೇಶಗಳಲ್ಲಿ ಹೇಗೆ ಉಪನ್ಯಾಸಾದಿಗಳನ್ನು ಮಾಡಬೇಕು, ಅಲ್ಲಿಯ ಜನರೊಂದಿಗೆ ಹೇಗೆ ವ್ಯವಹರಿಸಬೇಕು ಎಂಬುದನ್ನೆಲ್ಲಾ ಸ್ವಾಮೀಜಿ ಹೇಳಿಕೊಟ್ಟರು. ಅಷ್ಟು ಹೊತ್ತಿಗೆ ಅಮೇರಿಕಾ ದೇಶದಿಂದ ಒಬ್ಬ ಸ್ವಾಮಿಗಳನ್ನು ಕಳುಹಿಸಬೇಕೆಂಬ ಕರೆ ಬಂದುದರಿಂದ ಗುಡ್‍ವಿನ್ ಜೊತೆಯಲ್ಲಿ ಶಾರದಾನಂದರನ್ನು ಅಮೇರಿಕಾ ದೇಶಕ್ಕೆ ಕಳುಹಿಸಿದರು. ಅವರು ಅಲ್ಲಿ ಉಪನ್ಯಾಸಾದಿಗಳನ್ನು ಕೊಡುವುದರಲ್ಲಿ ಯಶಸ್ವಿಯಾಗಿರುವರು ಎಂಬುದನ್ನು ಕೇಳಿದಾಗ ಸ್ವಾಮೀಜಿಯವರಿಗೆ ಆನಂದವಾಯಿತು. 

 ಸೇವಿಯರ್ಸ್‍‍ ದಂಪತಿಗಳೊಡನೆ ಕೆಲವು ಕಾಲ ಇದ್ದಾದ ಮೇಲೆ ವಿಂಬಲ್‍ಡನ್‍ನಲ್ಲಿ\-ರುವ ಮಿಸ್ ಮುಲ್ಲರ್ ಮನೆಯಲ್ಲಿ ಎರಡು ಉಪನ್ಯಾಸಗಳನ್ನು ಕೊಟ್ಟರು. ಮೊದಲನೆ ದಿನ “ನಮ್ಮ ನಾಗರಿಕತೆಯಲ್ಲಿ ವೇದಾಂತ ಪ್ರಭಾವ” ಎಂಬ ವಿಷಯದ ಮೇಲೆ ಮಾತನಾಡಿದರು. ಅನಂತರ ಸ್ವಾಮೀಜಿ ಪ್ರತಿದಿನವೂ ಆಸಕ್ತರಾದ ಕೆಲವು ವಿದ್ಯಾರ್ಥಿಗಳಿಗೆ ರಾಜಯೋಗ ಮತ್ತು ಧ್ಯಾನ ಮುಂತಾದುವುಗಳ ಮೇಲೆ ಪ್ರವಚನಗಳನ್ನು ಕೊಡಲು ಪ್ರಾರಂಭಮಾಡಿದರು. 

 ಲಂಡನ್ನಿನಲ್ಲಿ ಸ್ವಾಮೀಜಿ ಮಾಡಿದ ಬಹಿರಂಗ ಉಪನ್ಯಾಸಗಳಲ್ಲಿ ಬಹುಪಾಲು ಜ್ಞಾನಯೋಗಕ್ಕೆ ಸೇರಿವೆ. ಸ್ವಾಮೀಜಿಯವರ ಪ್ರವಚನಗಳಿಗೆ ಹೆಚ್ಚು ಜನ ಬಂದು ಕೇಳಲು ಅನುಕೂಲವಾಗುವಂತೆ ಸ್ಟರ‍್ಡಿ ದೊಡ್ಡ ಒಂದು ಕೋಣೆಯನ್ನು ವಿಕ್ಟೋರಿಯ\break ಬೀದಿಯಲ್ಲಿ ಬಾಡಿಗೆಗೆ ತೆಗೆದುಕೊಂಡರು. ಅಷ್ಟು ಹೊತ್ತಿಗೆ ಇಂಡಿಯಾ ದೇಶದಿಂದ ಅವರ ಗುರುಭಾಯಿಗಳಾದ ಅಭೇದಾನಂದರು ಅವರಿಗೆ ಸಹಾಯಕರಾಗಿ ಇಂಗ್ಲೆಂಡಿಗೆ ಬಂದರು. ಇಬ್ಬರು ಸ್ವಾಮಿಗಳು ವಾಸಿಸುವುದಕ್ಕೆ ಸೇವಿಯರ್ಸ್‍‍ ದಂಪತಿಗಳು ವೆಸ್ಟಮಿನಿಸ್ಟರ್‌ನಲ್ಲಿ ಒಂದು ಮನೆಯನ್ನು ಬಾಡಿಗೆಗೆ ತೆಗೆದುಕೊಂಡರು. ಸ್ವಾಮೀಜಿ ವರುಷದ ಕೊನೆಯಲ್ಲಿ ಇಂಡಿಯಾ ದೇಶಕ್ಕೆ ಹೋಗಬೇಕೆಂದು ಮನಸ್ಸು ಮಾಡಿದ್ದರು. ಆದಕಾರಣ ಅಭೇದಾನಂದರನ್ನು ಉಪನ್ಯಾಸಗಳನ್ನು ಕೊಡುವುದು, ಆಧ್ಯಾತ್ಮಿಕ ವಿಷಯಗಳನ್ನು ಕುರಿತು ಮಾತನಾಡುವುದು ಮುಂತಾದ ಕೆಲಸದಲ್ಲಿ ತರಬೇತು ಮಾಡಿದರು, ತಮ್ಮ ನಂತರ ಅಭೇದಾನಂದರು ಆ ಕಾರ‍್ಯಗಳನ್ನು ಮಾಡಲಿ ಎಂದು. ಆ ಸಮಯದಲ್ಲೇ ಇಂಡಿಯಾದೇಶದ ತಮ್ಮ ಭಕ್ತರಿಗೆ ಹುರಿದುಂಬಿಸಿ ಕಾಗದ ಬರೆಯುತ್ತಿದ್ದರು. ತಾವು ಇಂಡಿಯಾ ದೇಶಕ್ಕೆ ಇನ್ನು ಬಾರದೆ ಇದ್ದರೂ, ತಾವು ಹೊರಗೆ ಮಾಡಿದ ಕೆಲಸ ಭರತಖಂಡದಲ್ಲಿ ಹೆಚ್ಚು ಪರಿಣಾಮಕಾರಿಯಾಗುವುದು ಎಂದು ಹೇಳಿದರು. 

 ಪ್ರೊಫೆಸರ್ ಡಾಯ್‍ಸನ್‍ ಅವರು ಎರಡು ವಾರಗಳು ಸ್ವಾಮೀಜಿಯವರನ್ನು ನೋಡಲು ಬಂದರು. ವೇದಾಂತ ವಿಷಯಗಳ ಮೇಲೆ ಬಹಳ ಮಾತುಕತೆ ಆಡಿದರು. ಸ್ವಾಮೀಜಿ ಪಾಶ್ಚಾತ್ಯನು ತನ್ನ ಪೂರ್ವಕಲ್ಪಿತ ಅಭಿಪ್ರಾಯಗಳನ್ನೆಲ್ಲ ಸಂಪೂರ್ಣವಾಗಿ ಬಿಟ್ಟಲ್ಲದೆ ಭಾರತೀಯ ದರ್ಶನಗಳನ್ನು ಚೆನ್ನಾಗಿ ತಿಳಿದುಕೊಳ್ಳಲಾರನೆಂಬುದನ್ನು ಅವರು ವಿವರಿಸಿದರು. ಮ್ಯಾಕ್ಸ್‌ಮುಲ್ಲರ್ ಕೂಡ ಆ ಸಮಯದಲ್ಲಿ ಸ್ವಾಮೀಜಿಯವರೊಂದಿಗೆ ಸಂಬಂಧವನ್ನು ಇಟ್ಟುಕೊಂಡಿದ್ದರು. 

 ಸ್ವಾಮೀಜಿ ಅಲ್ಲಿ ಮಾಯಾಸಿದ್ಧಾಂತದ ಮೇಲೆ ಉಪನ್ಯಾಸಗಳನ್ನು ಕೊಟ್ಟರು.\break ಅದು ಪಾಶ್ಚಾತ್ಯರಿಗೆ ಸುಲಭವಾಗಿ ಅರ್ಥವಾಗುವಂತಹುದಲ್ಲ. ಅಷ್ಟೇ ಏಕೆ ಭರತಖಂಡದಲ್ಲಿ ಕೂಡ ಅದನ್ನು ಸರಿಯಾಗಿ ತಿಳಿದುಕೊಂಡವರ ಸಂಖ್ಯೆ ಬಹಳ ಕಡಿಮೆ ಎಂತಲೇ ಹೇಳಬಹುದು. ‘ಮಾಯೆ ಮತ್ತು ಈಶ್ವರನ ಭಾವನೆಯ ವಿಕಸನ’, ‘ಮಾಯೆ ಮತ್ತು ಮುಕ್ತಿ’, ‘ನಿರಪೇಕ್ಷ ಸಾಪೇಕ್ಷ’ ಎಂಬ ವಿಷಯಗಳ ಮೇಲೆ ಮಾತನಾಡಿದರು. ಇದೇ ಕಾಲದಲ್ಲಿಯೇ ‘ಸರ್ವರಲ್ಲಿಯೂ ಈಶ್ವರನ ಭಾವನೆ’, ‘ಆತ್ಮ ಸಾಕ್ಷಾತ್ಕಾರ’,\break ‘ವೈವಿಧ್ಯತೆಯಲ್ಲಿ ಏಕತೆ’ ಎಂಬ ವಿಷಯಗಳ ಮೇಲೆ ಮಾತನಾಡಿದರು. ಉಪನ್ಯಾಸಗಳೆಲ್ಲ ಅವರ ‘ಜ್ಞಾನಯೋಗ’ ಎಂದು ಅನಂತರ ಪ್ರಕಟಮಾಡಿದ ಪುಸ್ತಕದಲ್ಲಿ ಬರುತ್ತವೆ. ಕೊನೆಗೆ ಅನುಷ್ಠಾನ ವೇದಾಂತದ ಮೇಲೆ ನಾಲ್ಕು ಉಪನ್ಯಾಸಗಳನ್ನು ಕೊಟ್ಟರು. ಸ್ವಾಮೀಜಿ ಅದ್ವೈತ ವೇದಾಂತ ಉಪನ್ಯಾಸಗಳನ್ನು ಕೊಡುವಾಗ ಅವರ ಮನಸ್ಸು ಅಂತಹ ಉತ್ತಮ ಸ್ಥಿತಿಯಲ್ಲಿತ್ತು. ಅದನ್ನು ಕೇಳಿದವರು ಕಾಲ ದೇಶ ನಿಮಿತ್ತವೆಂಬ ಮಾಯಾಜಾಲವನ್ನು ಹರಿದುಕೊಂಡು ಹೋದಂತೆ ಭಾವಿಸುತ್ತಿದ್ದರು. ಸ್ವಾಮೀಜಿ ತಮ್ಮ ಉಪನ್ಯಾಸದಲ್ಲಿ\break ಕೇಳಿದವರಿಗೆ ತಮ್ಮ ಅನುಭವದ ಶಕ್ತಿಯನ್ನು ನೀಡುತ್ತಿದ್ದರು. ಅವು ಕೇವಲ ಬೌದ್ಧಿಕ ಸಾಹಸಗಳು ಮಾತ್ರ ಆಗಿರಲಿಲ್ಲ. ಅನುಭವದ ರಸದೂಟಗಳು ಕೂಡ ಆಗಿದ್ದವು. ಮಾಯೆ ಎಂಬ ಭಾವನೆಯೇ ನೀರಸವಾಗಿ ಕಾಣುವುದು. ಆದರೆ ಸ್ವಾಮೀಜಿ ಅದನ್ನು ವಿವರಿಸುವಾಗ ಕಾವ್ಯಮಯವಾಗಿ ಮಾಡಿರುವರು. ಮೇಲಿನ ಕೆಲವು ಅವರ ಉಪನ್ಯಾಸಗಳು ಅಪೂರ್ವ ಸಾಹಿತ್ಯರಾಶಿಗೆ ಸೇರಿವೆ. ‘ಮಾಯೆ ಮತ್ತು ಮುಕ್ತಿ’ ಎಂಬುದರಲ್ಲಿ ಸ್ವಲ್ಪಭಾಗವನ್ನು ಮಾತ್ರ ಭಾವನೆಯ ಸೌಂದರ‍್ಯತೆಗೆ ಒಂದು ಉದಾಹರಣೆಯಂತೆ ಕೊಡುವೆವು: 

 “ದಿವ್ಯ ಪ್ರಭೆಯ ಮೇಘಮಾಲೆಯಂತೆ ನಾವು ಬರುತ್ತೇವೆ ಎನ್ನುವನು ಕವಿ. ನಾವೆಲ್ಲ ದಿವ್ಯಪ್ರಭೆಯ ಮುಗಿಲಿನಂತೆ ಬರುವುದಿಲ್ಲ. ನಮ್ಮಲ್ಲಿ ಕೆಲವರು ಕರ‍್ಪೊಗೆಯಂತೆ ಬರುವರು. ಇದರಲ್ಲಿ ಸಂದೇಹವಿಲ್ಲ. ಪ್ರತಿಯೊಬ್ಬರೂ ರಣರಂಗದಲ್ಲಿ ಹೋರಾಡುವುದಕ್ಕೆ ಬರುವಂತೆ ಬರುವರು. ಅಳುತ್ತಾ ಒಂದು ಮಾರ್ಗಕ್ಕಾಗಿ ಹೋರಾಡಲು ಸಾಧ್ಯವಾದಷ್ಟು ಪ್ರಯತ್ನಿಸುವೆವು. ಅನಂತ ಭವಸಾಗರದಲ್ಲಿ ನಮಗೊಂದು ದಾರಿಗಾಗಿ ಯತ್ನಿಸುತ್ತೇವೆ. ಮುಂದೆ ಮುಂದೆ ಹೋಗುತ್ತೇವೆ. ನಮ್ಮ ಹಿಂದೆ ಅನಂತ ಕಾಲವಿದೆ. ನಮ್ಮ ಮುಂದೆ ಅನಂತ ಅಸೀಮ ದೇಶವಿದೆ. ಮೃತ್ಯು ಬರುವವರೆಗೆ ಹೀಗೆ ಸಾಗುತ್ತಿರುತ್ತೇವೆ. ಸೋತೆವೋ, ಗೆದ್ದೆವೋ, ಯಾವುದೂ ಗೊತ್ತಿಲ್ಲ. ಅಂತೂ ಸಾವು ನಮ್ಮನ್ನು ಕಾರ‍್ಯಕ್ಷೇತ್ರದಿಂದ ಒಯ್ಯುವುದು ಇದೇ ಮಾಯೆ. 

 “ಬಾಲ್ಯದಲ್ಲಿ ಭರವಸೆ ಬಲವಾಗಿರುವುದು, ಕಣ್ತೆರೆಯುವ ಮಗುವಿಗೆ ಪ್ರಪಂಚವೆಲ್ಲ ಒಂದು ಸ್ವರ್ಣಸ್ವಪ್ನದಂತೆ ಕಾಣುವುದು. ತನ್ನ ಇಚ್ಛೆಯನ್ನು ಮೀರುವುದಾವುದೂ ಇಲ್ಲ ಎಂದು ಭಾವಿಸುವುದು. ಮನುಷ್ಯ ಮುಂದುವರಿದಂತೆ ಹೆಜ್ಜೆ ಹೆಜ್ಜೆಗೂ ಅವನ ಭವಿಷ್ಯದ ಏಳಿಗೆಗೆ ಆತಂಕವಾಗಿ ಅಭೇದ್ಯಕೋಟೆಯಂತೆ ಪ್ರಕೃತಿ ನಿಲ್ಲುವುದು. ಅವನು ಭೇದಿಸುವುದಕ್ಕೆ ಪದೇ ಪದೇ ಅದರ ಮೇಲೆ ಬೀಳಬಹುದು. ಅವನು ಮುಂದೆ ಸರಿದಂತೆ ಆದರ್ಶವೂ ಮೃತ್ಯು ಸಮೀಪಿಸಿರುವ ಪರ್ಯಂತರ ಮುಂದೆ ಮುಂದೆ ಸರಿಯುವುದು. ಬಹುಶಃ ಮೃತ್ಯುವಿನಲ್ಲಿ ಬಿಡುಗಡೆ ಇರಬಹುದು! ಇದೇ ಮಾಯೆ. 

 “ಜ್ಞಾನಾಕಾಂಕ್ಷಿಯಾಗಿರುವ ವಿಜ್ಞಾನಿಯೊಬ್ಬ ಏಳುವನು. ಅವನಿಗೆ ಇದಕ್ಕಾಗಿ ಯಾವ ತ್ಯಾಗವೂ ದೊಡ್ಡದಲ್ಲ. ಯಾವ ಹೋರಾಟವೂ ನಿರಾಶಾಜನಕವಲ್ಲ. ಪ್ರಕೃತಿಯ ರಹಸ್ಯಗಳನ್ನು ಅವನು ಒಂದಾದ ಮೇಲೊಂದು ಕಂಡುಹಿಡಿಯುತ್ತ ಹೋಗುವನು. ಪ್ರಕೃತಿಯ ಹೃದಯಾಂತರಾಳದಿಂದ ರಹಸ್ಯಗಳನ್ನು ಭೇದಿಸಿ ತೆಗೆಯುವನು. ಇದು ಏತಕ್ಕೆ? ಇದೆಲ್ಲಾ ಏತಕ್ಕೆ? ಅವನಿಗೇತಕ್ಕೆ ಕೀರ್ತಿಕೊಡಬೇಕು? ಅವನೇತಕ್ಕೆ ಕೀರ್ತಿ ಸಂಪಾದಿಸಬೇಕು? ಪ್ರಕೃತಿ ಜಡ ಅಚೇತನ. ಅಚೇತನ ಪ್ರಕೃತಿಯನ್ನು ಜನ ಅನುಸರಿಸುವುದಕ್ಕೆ ಏತಕ್ಕೆ ಕೀರ್ತಿ? ಪ್ರಕೃತಿ ಎಷ್ಟು ದೂರಕ್ಕೆ ಬೇಕಾದರೂ, ಯಾವ ಪ್ರಮಾಣದ ಮಿಂಚನ್ನು ಬೇಕಾದರೂ ಎಸೆಯಬಲ್ಲದು. ಅದರಲ್ಲಿ ಎಲ್ಲೋ ಒಂದು ಅಂಶ ಮಾತ್ರ ಮಾಡಿದ ಮನುಷ್ಯನನ್ನು ಹೊಗಳಿ ಆಕಾಶಕ್ಕೆ ಏರಿಸುವೆವು. ಏಕೆ? ಪ್ರಕೃತಿಯ ಅನುಕರಣಕ್ಕೆ, ಮೃತ್ಯುವಿನ ಅನುಕರಣಕ್ಕೆ, ಜಡ ಅಚೇತನದ ಅನುಕರಣಕ್ಕೆ, ಅವನನ್ನು ಏಕೆ ಹೊಗಳಬೇಕು? ಆಕರ್ಷಣ ಶಕ್ತಿ ಪ್ರಪಂಚದಲ್ಲಿರುವ ವಸ್ತುಗಳಲ್ಲೆಲ್ಲಾ ಬೃಹದಾಕಾರವುಳ್ಳದ್ದನ್ನು ಕೂಡ ಸೆಳೆದು ಪುಡಿಮಾಡಬಲ್ಲದು. ಆದರೂ ಇದು ಜಡ. ಜಡವನ್ನು ಅನುಕರಿಸುವುದು ಒಂದು ಮಹಿಮೆ ಏನು? ಆದರೆ ನಾವೆಲ್ಲ ಇದಕ್ಕಾಗಿ ಹೋರಾಡುತ್ತಿರುವೆವು. ಇದೇ ಮಾಯೆ. 

\vskip 1pt

 “ಇಂದ್ರಿಯಗಳು ಮನುಷ್ಯನನ್ನು ಬಹಿರ್ಮುಖಿಯನ್ನಾಗಿ ಮಾಡುವುವು. ಮನುಷ್ಯ ಆನಂದವನ್ನು, ಅದು ಎಲ್ಲಿ ದೊರಕದೊ ಅಲ್ಲಿ, ಹುಡುಕುತ್ತಿರುವನು, ಇದೆಲ್ಲ ವ್ಯರ್ಥ, ಪ್ರಯೋಜನವೆಂಬುದಿಲ್ಲವೆಂದು ಇಲ್ಲಿ ಸುಖವೆಂಬುದಿಲ್ಲವೆಂದು ಅನಾದಿಕಾಲದಿಂದಲೂ ನಮಗೆ ಬೋಧಿಸಿರುವರು. ಆದರೂ ನಾವು ಕಲಿತುಕೊಳ್ಳುವುದಿಲ್ಲ. ನಮ್ಮ ಅನುಭವದ ಮೂಲಕವಲ್ಲದೆ ಬೇರೆ ಹಾಗೆ ಕಲಿಯಲು ಸಾಧ್ಯವೇ ಇಲ್ಲ. ನಾವು ಪ್ರಯತ್ನಿಸುವೆವು. ಪೆಟ್ಟೊಂದು ಬೀಳುವುದು. ಆಗಲಾದರೂ ಕಲಿಯುವೆವೆ? ಆಗಲೂ ಇಲ್ಲ. ಪತಂಗ ದೀಪಕ್ಕೆ ಧಾವಿಸಿದಂತೆ, ತೃಪ್ತಿ ಸಿಕ್ಕುವುದೆಂದು ಇಂದ್ರಿಯ ಸುಖದೆಡೆಗೆ ನಾವು ಪುನಃ ಪುನಃ ಧಾವಿಸುವೆವು. ಮತ್ತೆ ಮತ್ತೆ ಹೊಸ ಉತ್ಸಾಹದಿಂದ ಬರುವೆವು. ಮೋಸಗೊಂಡು ಶಕ್ತಿಗುಂದುವವರೆಗೂ ಹೀಗೆ ಹೋಗಿ ಒಂದು ದಿನ ಸಾಯುವೆವು.” 

\vskip 1pt

 “ಇದರಂತೆಯೇ ನಮ್ಮ ಯುಕ್ತಿ ಕೂಡ. ವಿಶ್ವದ ರಹಸ್ಯವನ್ನು ಭೇದಿಸಬೇಕೆಂಬ ಇಚ್ಛೆಯಿಂದ ನಾವು ಪ್ರಯತ್ನಿಸುವುದನ್ನು ನಿಲ್ಲಿಸುವುದೇ ಇಲ್ಲ. ತಿಳಿಯಬೇಕೆಂದು ಆಸೆ. ಯಾವ ಜ್ಞಾನವೂ ಸಾಧ್ಯವಿಲ್ಲ ಎಂಬುದನ್ನು ನಂಬುವುದೇ ಇಲ್ಲ. ಕೆಲವು ಹೆಜ್ಜೆಗಳಷ್ಟು ಹೋಗುವುದರೊಳಗೆ ನಾವು ಭೇದಿಸಲಾಗದ ಆದಿ ಅಂತ್ಯಗಳಿಲ್ಲದ ಕಾಲವೆಂಬ ಗೋಡೆ, ಕೆಲವು ಹೆಜ್ಜೆಗಳಷ್ಟು ಹೋಗುವುದರೊಳಗೆ ನಾವು ಪಾರಾಗಲು ಅಸಾಧ್ಯವಾದ ಅಸೀಮ ದೇಶದ (\enginline{Space}) ಕೋಟೆ ಏಳುವುದು. ವಿಶ್ವವೆಲ್ಲ ಕಾರ್ಯಕಾರಣಗಳ ಬಿಡಿಸಲು ಅಸದಳವಾದ ಕೋಟೆಯಿಂದ ಸುತ್ತುವರಿಯಲ್ಪಟ್ಟಿದೆ. ಇದನ್ನು ಮೀರಿ ನಾವು ಹೋಗಲಾರೆವು. ಆದರೂ ಹೋರಾಡುತ್ತೇವೆ, ಹೋರಾಡಲೇ ಬೇಕು. ಇದೇ ಮಾಯೆ.” 

\vskip 1pt

 “ನಮ್ಮ ಉಚ್ಛ್ವಾಸ ನಿಶ್ವಾಸಗಳೊಂದಿಗೆ, ನಮ್ಮ ಹೃದಯದ ಪ್ರತಿಯೊಂದು ಸ್ಪಂದನದೊಂದಿಗೆ, ನಮ್ಮ ಪ್ರತಿಯೊಂದು ಚಲನವಲನಗಳಲ್ಲಿ, ನಾವು ಸ್ವತಂತ್ರರು ಎಂದು ಭಾವಿಸುತ್ತೇವೆ. ಅದೇ ಕ್ಷಣ ನಾವು ಅಸ್ವತಂತ್ರರೆಂಬುದನ್ನು ಅದು ತೋರುವುದು. ನಾವೆಲ್ಲ ಬದ್ಧರು ಗುಲಾಮರು. ದೇಹ ಮನಸ್ಸು, ಆಲೋಚನೆ, ಭಾವನೆ, ಪ್ರತಿಯೊಂದರಲ್ಲಿಯೂ ನಾವು ಪ್ರಕೃತಿಯ ಬದ್ಧಸೇವಕರು, ಇದೇ ಮಾಯೆ.” 

\vskip 1pt

 “ತನ್ನ ಮಗ ಪ್ರಪಂಚದ ಪ್ರತಿಭಾವಂತ, ಇದುವರೆಗೆ ಪ್ರಪಂಚದಲ್ಲಿ ಹುಟ್ಟಿದ ಅಸಾಧಾರಣ ಶಿಶು ಎಂದು ಭಾವಿಸದ ತಾಯಿಯೇ ಇಲ್ಲ. ಆ ಮಗುವನ್ನು ಮುದ್ದಿಸುವಳು. ಆಕೆಯ ಪ್ರಾಣವೆಲ್ಲ ಮಗುವಿನಲ್ಲಿರುವುದು ಮಗು ಬೆಳೆಯುವುದು, ಬಹುಶಃ ಒಬ್ಬ ಕುಡುಕನೊ ಕ್ರೂರನೊ ಆಗಿ ಅದು ತಾಯಿಯನ್ನು ಹಿಂಸಿಸುವುದು. ಅದು\break ಹಿಂಸಿಸಿದಷ್ಟೂ ತಾಯಿಯ ಪ್ರೀತಿ ಹೆಚ್ಚುವುದು. ತಾಯಿಯ ನಿಸ್ವಾರ್ಥಪ್ರೇಮವೆಂದು ಇದನ್ನು ಪ್ರಪಂಚ ಹೊಗಳುವುದು. ಅದನ್ನು ಪ್ರೀತಿಸದೆ ವಿಧಿಯೇ ಇಲ್ಲ ಎನ್ನುವುದು ಪ್ರಪಂಚಕ್ಕೆ ತಿಳಿಯದು. ಸಾವಿರ ಸಲ ಈ ಹೊರೆಯಿಂದ ಪಾರಾಗಬೇಕೆಂದು ಅವಳು ಆಲೋಚಿಸಬಹುದು. ಆದರೂ ಸಾಧ್ಯವಿಲ್ಲ. ಅದನ್ನು ಹೂವಿನ ರಾಶಿಯಿಂದ ಮುಚ್ಚಿ ಅದ್ಭುತ ಪ್ರೇಮ ಎಂದು ಕರೆಯುವಳು. ಇದೇ ಮಾಯೆ.” 

 ಅಕ್ಟೋಬರ್ ಮತ್ತು ನವೆಂಬರ್ ತಿಂಗಳಲ್ಲಿ, ಲಂಡನ್ ಮತ್ತು ಆಕ್ಸ್‌ಫರ್ಡ್‍ ನಲ್ಲಿ ಹಲವು ಮನೆಗಳಲ್ಲಿ ಮತ್ತು ಕ್ಲಬ್ಬುಗಳಲ್ಲಿ ಮಾತನಾಡಲು ಕರೆಗಳು ಬಂದವು. ವಿಲಿಯಂ ವಿಲ್‌ಬರ್‌ಫೋರ್ಸ್‍ ಎಂಬ ಪ್ರಖ್ಯಾತ ಸಾಹಿತಿ ಸ್ವಾಮೀಜಿಯವರನ್ನು ತಮ್ಮ ಮನೆಗೆ ಆಹ್ವಾನಿಸಿ ಗೌರವ ತೋರಿದನು. ಆತ ವೇದಾಂತದ ವಿದ್ಯಾರ್ಥಿಯಾದ. ಸೀಸೆಮ್ ಕ್ಲಬ್ಬಿನಲ್ಲಿ ಸ್ವಾಮೀಜಿ ಅನೇಕವೇಳೆ ಮಾತನಾಡಿದರು. ಅಲ್ಲಿನ ಸದಸ್ಯರನೇಕರು ಅವರಿಗೆ ಸ್ನೇಹಿತರಾದರು. ಫ್ರೆಡರಿಕ್ ಹೆಚ್ ಮೈಯರ್ಸ್‍ ಎಂಬ ಪ್ರಖ್ಯಾತ ಮಾನಸಿಕ ಶಾಸ್ತ್ರಜ್ಞನು, (ಈತ ಹಲವು ಗ್ರಂಥಗಳನ್ನು ಬರೆದಿರುವನು) ಸ್ವಾಮೀಜಿಯವರಿಗೆ ಪರಿಚಿತನಾದ.\break ನಾನ್‌ಕನ್‌ಫಾರ್‌ಮಿಸ್ಟ್ ಪಾದ್ರಿಗಳಾದ ರೆವರೆಂಡ್ ಜಾನ್ ಪೇಜ್ ಹಾಪ್ಸ್,\break ಮಾನ್‌ಕ್ಯೂರ್‌ಡಿಕಾನ್ಯೆ ಎಂಬ ಪಾಸಿಟಿವಿಸ್ಟ್ ಸಿದ್ಧಾಂತವಾದಿ (ದೃಷ್ಟ ಪ್ರಮಾಣವಾದಿ), ಎಡ್‌ವರ್ಡ್ ಕಾರ್‌ಪೆಂಟರ್ ಮುಂತಾದ ವಿದ್ವಾಂಸರ ಪರಿಚಯವೂ ಆಯಿತು.\break ಆಂಗ್ಲಿಕನ್ ಚರ್ಚಿಗೆ ಸೇರಿದ ಅನೇಕ ಪಾದ್ರಿಗಳು ಕೂಡ ವೇದಾಂತದ ವಿಷಯದಲ್ಲಿ ಕುತೂಹಲವನ್ನು ತೋರಿದರು. 

 ಒಂದು ಸಲ ಸ್ವಾಮೀಜಿ ತಮ್ಮ ಬದಲು ಅಭೇದಾನಂದರನ್ನು ಮಾತನಾಡಲು ಹೇಳಿದರು. ಅವರು ಸುಂದರವಾಗಿ ಮಾತನಾಡಿದರು. ಸ್ವಾಮೀಜಿಯವರಿಗೆ ಬಹಳ ಸಂತೋಷವಾಯಿತು. ತಾವು ಇಲ್ಲದೆ ಇದ್ದರೂ ತಮ್ಮ ಗುರುಭಾಯಿಗಳು ಕೆಲಸವನ್ನು ಸುಸೂತ್ರವಾಗಿ ನಡೆಸಿಕೊಂಡು ಹೋಗಬಲ್ಲರು ಎಂಬ ಧೈರ್ಯ ಬಂದಿತು. ಇತ್ತ ಅಮೇರಿಕಾದೇಶಕ್ಕೆ ಹೋದ ಶಾರದಾನಂದರು ಯಶಸ್ವಿಯಾಗಿ ಪ್ರವಚನಾದಿಗಳನ್ನು ನಡೆಸಿಕೊಂಡು ಹೋಗುತ್ತಿರುವರು ಎಂಬ ಸುದ್ದಿಗಳು ಬಂದವು. ಸ್ವಾಮೀಜಿಯವರ ಶಿಷ್ಯೆಯಾದ ಮಿಸ್ ವಾಲ್ಡೊ ಕೂಡ ಯಶಸ್ವಿಯಾಗಿ ಪ್ರವಚನಾದಿಗಳನ್ನು ನಡೆಸುತ್ತಿದ್ದಳು. ಸ್ವಾಮೀಜಿ ತಾವು ಬಿತ್ತಿದ ವೇದಾಂತ ಬೀಜ ಕ್ರಮೇಣ ಮೊಳೆಯುತ್ತಿರುವುದನ್ನು ಕಂಡರು. 

 ಭರತಖಂಡದಿಂದ ಭಕ್ತಾದಿಗಳು ಸ್ವಾಮೀಜಿಯವರನ್ನು ಎಂದು ಇಂಡಿಯಾ ದೇಶಕ್ಕೆ ಬರುತ್ತೀರಿ ಎಂದು ಕೇಳತೊಡಗಿದರು. ಅದೂ ಅಲ್ಲದೆ ಒಂದು ಸುವ್ಯವಸ್ಥಿತವಾದ ಸಂಘವನ್ನು ಇಂಡಿಯಾದೇಶದಲ್ಲಿ ಸ್ಥಾಪಿಸಬೇಕೆಂದು ಸ್ವಾಮೀಜಿ ಹಂಬಲಿಸಿದರು. ಅಮೇರಿಕಾದಲ್ಲಿ ಅವರ ಶಿಷ್ಯೆ ಶ‍್ರೀಮತಿ ಓಲ್‌ಬುಲ್ ಅವರು ಇಂಡಿಯಾ ದೇಶದ ಅವರ ಕೆಲಸಕ್ಕೆ ಎಂದರೆ ಸಂಘಸ್ಥಾಪನೆಯ ಕೆಲಸಕ್ಕೆ ದ್ರವ್ಯದ ಮೂಲಕ ಸಹಾಯ ಮಾಡುವುದಾಗಿ ತಿಳಿಸಿದರು. ಸ್ವಾಮೀಜಿ ತಕ್ಷಣವೇ ಅವರಿಂದ ದುಡ್ಡನ್ನು ತೆಗೆದುಕೊಳ್ಳಲಿಲ್ಲ. ಇಂಡಿಯಾ ದೇಶಕ್ಕೆ ಹೋಗಿ ಅಲ್ಲಿಯ ಸಾಧ್ಯಾಸಾಧ್ಯತೆಗಳನ್ನೆಲ್ಲ ಪರಿಶೀಲಿಸಿ ಪುನಃ ಅವರಿಗೆ ಕಾಗದ ಬರೆಯುವುದಾಗಿ ಉತ್ತರ ಕೊಟ್ಟರು. ಸ್ವಾಮೀಜಿ ನೇಪಲ್ಸ್‌ನಿಂದ ಡಿಸೆಂಬರ್ ಹದಿನಾರನೇ ತಾರೀಖು ಕೊಲಂಬೋಗೆ \enginline{Prince Regent Lutipold}ನಲ್ಲಿ ಪ್ರಯಾಣ ಮಾಡಲು ನಿಶ್ಚಯಿಸಿದರು. ಅದರಂತೆಯೆ ಟಿಕೀಟುಗಳನ್ನು ಸೇವಿಯರ್ಸ್‍‍ ದಂಪತಿಗಳು ಕೊಂಡರು. ನಾಲ್ಕು ಜನ ಹೋಗುವುದಕ್ಕೆ ನಿರ್ಧಾರವಾಯಿತು - ಸ್ವಾಮೀಜಿ, ಸೇವಿಯರ್ಸ್‍‍ ದಂಪತಿಗಳು ಮತ್ತು ಶೀಘ್ರಲಿಪಿಕಾರ ಗುಡ್‌ವಿನ್ ಎಂಬುವರು.

 ಸ್ವಾಮೀಜಿ ಮೂರು ಕೇಂದ್ರಗಳನ್ನು ಮೊದಲು ಇಂಡಿಯಾದೇಶದಲ್ಲಿ ಸ್ಥಾಪಿಸಬೇಕೆಂದು ಬಯಸಿದರು. ದಕ್ಷಿಣದಲ್ಲಿ ಮದ್ರಾಸು, ಉತ್ತರದಲ್ಲಿ ಕಲ್ಕತ್ತ ಮತ್ತು ಪೌರಾತ್ಯ ದೇಶದಿಂದ ಬರುವ ಭಕ್ತರಿಗಾಗಿ ಹಿಮಾಲಯದ ಕೇಂದ್ರ. ಕೊನೆಯದಕ್ಕೆ ಸೇವಿಯರ್ಸ್‍‍ ದಂಪತಿಗಳು ತಮ್ಮ ಸರ್ವಸ್ವವನ್ನು ಧಾರೆ ಎರೆಯುತ್ತೇವೆ ಎಂದು ಹೇಳಿದ್ದರು. ತಮ್ಮ ಹತ್ತಿರ ಇದ್ದ ಹಣ ಜೊತೆಗೆ ಅವರಿಗೆ ಸೇರಿದ ಮನೆ, ಸಾಮಾನು ಅವರ ಮೈಮೇಲಿದ್ದ ಒಡವೆಗಳು, ಮತ್ತು ಪುಸ್ತಕ ಇವುಗಳನ್ನೆಲ್ಲ ಮಾರಿ ಅದರಿಂದ ಬಂದ ಹಣವನ್ನು ಸ್ವಾಮೀಜಿಯವರ ಕೈಯಲ್ಲಿ ಕೊಟ್ಟರು. ತಮ್ಮ ನಂತರದ ಜೀವಮಾನವನ್ನು ತಾವು ಇಂಡಿಯಾ ದೇಶದಲ್ಲಿ ಕಳೆಯುತ್ತೇವೆ ಎಂದು ಮನಸ್ಸುಮಾಡಿದರು. ಸ್ವಾಮೀಜಿ ಭರತಖಂಡದ ಸ್ತ್ರೀಯರ ಸಮಸ್ಯೆಯನ್ನು ಮರೆತಿರಲಿಲ್ಲ. ಅವರ ದೃಷ್ಟಿಯಲ್ಲಿ ಒಂದು ಹಕ್ಕಿಗೆ ಇರುವ ಎರಡು ರೆಕ್ಕೆಗಳಂತೆ, ಸಮಾಜದಲ್ಲಿ ಪುರುಷರು ಮತ್ತು ಸ್ತ್ರೀಯರು. ಸ್ತ್ರೀಯರಿಗೆ ತಿಳುವಳಿಕೆ ಕೊಡಬೇಕು, ಅವರು ಆದರ್ಶ ಸತಿಯರು ತಾಯಂದಿರಾಗಬೇಕು ಮತ್ತು ಸ್ತ್ರೀಯರ ಮೇಲ್ಮೆಗೆ ದುಡಿಯಲು ಬದ್ಧಕಂಕಣರಾಗಿರುವಂತಹ ಕೆಲವು ಸಂನ್ಯಾಸಿನಿಯರನ್ನೂ ತಯಾರು ಮಾಡಬೇಕೆಂದು ಅವರ ಇಚ್ಛೆಯಾಗಿತ್ತು. ಅದಕ್ಕಾಗಿ ಅವರ ಶಿಷ್ಯೆ ಮಿಸ್ ಮುಲ್ಲರ್ ಅವರು ಸ್ವಾಮೀಜಿಗೆ ಸಹಾಯ ಮಾಡಲು ಒಪ್ಪಿದರು. ಮಾರ್ಗರೇಟ್ ನೋಬಲ್‌ಳನ್ನು ಅನಂತರ ಇಂಡಿಯಾ ದೇಶಕ್ಕೆ ಕರೆಸಿ ಅವಳ ಕೈಗೆ ಸ್ತ್ರೀಯರ ಕೆಲಸವನ್ನೆಲ್ಲಾ ಕೊಡಬೇಕೆಂದು ಮನಸ್ಸು ಮಾಡಿದ್ದರು. 

 ಸ್ವಾಮೀಜಿ ಇನ್ನೇನು ಲಂಡನ್ ನಗರವನ್ನು ಬಿಟ್ಟು ಭರತಖಂಡಕ್ಕೆ ಮರಳಿ ಹೋಗುತ್ತಾರೆ ಎಂಬುದನ್ನು ಕೇಳಿ ಅವರ ಶಿಷ್ಯವೃಂದಕ್ಕೆ ಬಹಳ ವ್ಯಥೆಯಾಯಿತು. ಸ್ವಾಮೀಜಿಯವರನ್ನು ಬೀಳ್ಕೊಡುವುದಕ್ಕೆ ೧೩ನೇ ಭಾನುವಾರ ಒಂದು ಸಮಾರಂಭವನ್ನು ನಡೆಸಿದರು. ಅನೇಕ ಜನ ಸ್ವಾಮೀಜಿಯವರನ್ನು ಬೀಳ್ಕೊಡಲು ನೆರೆದರು, ಬೀಳ್ಕೊಡುವ ಭಾಷಣಗಳನ್ನು ಮಾಡಿದರು. ಸ್ವಾಮೀಜಿಯರು “ನಾವು ಪುನಃ ಒಬ್ಬರನ್ನೊಬ್ಬರು ನೋಡುತ್ತೇವೆ” ಎಂದು ಹೇಳಿದರು. ಸ್ವಾಮೀಜಿ ಹ್ಯಾಮಂಡ್ ಎಂಬುವರೊಬ್ಬರಿಗೆ “ನಾನು ದೇಹವನ್ನು ಜೀರ್ಣವಾದ ವಸ್ತ್ರದಂತೆ ಆಚೆಗೆ ಎಸೆಯಬಹುದು. ಆದರೆ ಮನುಷ್ಯರೆಲ್ಲ ಪರಮ ಸತ್ಯದ ಕಡೆಗೆ ಬರುವುದಕ್ಕೆ ಬೋಧಿಸುವುದನ್ನು, ಅವರಿಗೆ ಸಹಾಯ ಮಾಡುವುದನ್ನು ಬಿಡುವುದಿಲ್ಲ.” ಎಂದರು ಇಂದು ಸ್ವಾಮೀಜಿ ಇಲ್ಲ. ಆದರೆ ಎಷ್ಟೋ ಜೀವಿಗಳಿಗೆ ಅವರು ದಾರಿ ಬೆಳಕಾಗಿರುವರು. ಒಂದು ಹೊಸ ಸ್ಫೂರ್ತಿ ಅವರ ಜೀವನದಲ್ಲಿ ಜಿನುಗುವಂತೆ ಮಾಡಿರುವರು. ಇದೆಲ್ಲ ಅವರ ಸಾಹಿತ್ಯ ಅಧ್ಯಯನದಿಂದ. ಇಂದು ಸ್ವಾಮೀಜಿಯವರ ಸಾಹಿತ್ಯ ರಾಶಿಯ ದೇವಾಲಯ ಸಚೇತನವಾಗಿರುವುದು. ಇದೇ ಅವರ ದೇವಾಲಯ. ಇಲ್ಲಿಗೆ ಬಂದವರು ನಿರಾಶರಾಗಿ ಎಂದಿಗೂ ಹೋಗಲಾರರು. ಸ್ವಾಮೀಜಿ ಈ ದೇವಾಲಯಕ್ಕೆ ಬಂದವರಿಗೆ ಶಕ್ತಿ ಸ್ಫೂರ್ತಿ ಭರವಸೆಗಳನ್ನು ಕೊಟ್ಟು ಕಳುಹಿಸುವರು. 

\vskip 2pt

 ಸ್ವಾಮಿ ವಿವೇಕಾನಂದರ ಉಪನ್ಯಾಸಗಳು ಇಂಗ್ಲೆಂಡಿನ ವಿದ್ವನ್ಮಣಿಗಳ ಮೇಲೆ ಅಮೋಘವಾದ ಪರಿಣಾಮವನ್ನು ಉಂಟುಮಾಡಿದವು. ಸುಮ್ಮನೆ ಉದ್ವೇಗದಿಂದ ಅವರು ಯಾವುದನ್ನೂ ಒಪ್ಪಿಕೊಳ್ಳುವಂತಹವರಲ್ಲ. ವಿಚಾರಮಾಡಿ ತಮಗೆ ಸಮರ್ಪಕವಾಗಿ ಕಂಡರೆ ಮಾತ್ರ ಒಪ್ಪಿಕೊಳ್ಳುವಂತಹವರು. ಭಾರತೀಯ ದಾರ್ಶನಿಕ ಭಾವನೆಗಳು ಚೆನ್ನಾಗಿ ಇಂಗ್ಲೆಂಡಿನಲ್ಲಿ ಬೇರೂರುವುದಕ್ಕೆ ಇದರಿಂದ ಸಾಧ್ಯವಾಯಿತು. ಲಂಡನ್ನಿನಲ್ಲಿ ಸ್ವಾಮೀಜಿಯವರ ಪ್ರಭಾವ ಎಷ್ಟು ಪರಿಣಾಮಕಾರಿಯಾಗಿತ್ತು ಎಂಬುದಕ್ಕೆ ಇಂಡಿಯಾದೇಶದಿಂದ ಲಂಡನ್ನಿಗೆ ಹೋದ ಬಿಪಿನ್‌ಚಂದ್ರಪಾಲ್ ಅನಂತರ \enginline{The Indian Mirror} ಎಂಬ ಪತ್ರಿಕೆಯಲ್ಲಿ ಹೀಗೆ ಬರೆದಿರುವರು: 

\vskip 2pt

 “ಭರತಖಂಡದಲ್ಲಿ ಕೆಲವು ಭಾರತೀಯರು, ಸ್ವಾಮಿ ವಿವೇಕಾನಂದರು ಇಂಗ್ಲೆಂಡಿನಲ್ಲಿ ಉಪನ್ಯಾಸ ಮಾಡಿದ್ದು ಅಷ್ಟೇನು ಫಲಕಾರಿಯಾಗಿಲ್ಲ, ಕೇವಲ ಅವರ ಅನುಯಾಯಿಗಳು ಅದನ್ನು ಉತ್ಪ್ರೇಕ್ಷೆ ಮಾಡುತ್ತಿರುವರು ಎಂದು ಭಾವಿಸುವರು. ಆದರೆ ನಾನು ಇಲ್ಲಿಗೆ ಬಂದಮೇಲೆ ಸ್ವಾಮೀಜಿ ಎಲ್ಲಾ ಕಡೆಯಲ್ಲಿಯೂ ತಮ್ಮ ಪ್ರಭಾವವನ್ನು ಚೆನ್ನಾಗಿ ಬೀರಿರುವರು ಎಂಬುದು ಗೊತ್ತಾಯಿತು. ನಾನು ಇಂಗ್ಲೆಂಡಿನಲ್ಲಿ ಹಲವು ಕಡೆ ಸ್ವಾಮಿ ವಿವೇಕಾನಂದರನ್ನು ಪೂಜ್ಯ ದೃಷ್ಟಿಯಿಂದ ಗೌರವಿಸುವುದನ್ನು ನೋಡಿರುವೆನು. ನಾನು ಅವರ ಪಂಥಕ್ಕೆ ಸೇರದೆ ಇದ್ದರೂ, ನನಗೂ ಅವರಿಗೂ ಬೇಕಾದಷ್ಟು ಭಿನ್ನಾಭಿಪ್ರಾಯಗಳಿರುವುದು ನಿಜವಾದರೂ, ಇಲ್ಲಿ ಅವರು ಹಲವರಲ್ಲಿ ಜ್ಞಾನೋದಯವಾಗುವಂತೆ ಮಾಡಿರುವರು, ಹೃದಯವನ್ನು ವಿಕಾಸ ಮಾಡಿರುವರು. ಅವರ ಬೋಧನೆಯಿಂದ ಪುರಾತನ ಹಿಂದೂ ಶಾಸ್ತ್ರಗಳಲ್ಲಿ ಅದ್ಭುತವಾದ ಜ್ಞಾನದ ಗಣಿ ಹುದುಗಿದೆ ಎಂದು ಇಲ್ಲಿ ಅನೇಕರು ದೃಢವಾಗಿ ನಂಬುತ್ತಾರೆ. ಅವರು ಇಂಗ್ಲೆಂಡಿನ ಜನರಲ್ಲಿ ಇಂತಹ ಭಾವನೆ ಹುಟ್ಟುವಂತೆ ಮಾಡಿರುವುದು ಮಾತ್ರವಲ್ಲ, ಇಂಡಿಯಾ ಮತ್ತು ಇಂಗ್ಲೆಂಡುಗಳ ನಡುವೆ ಮಧುರ ಬಾಂಧವ್ಯ ಬೆಳೆಯುವಂತೆ ಮಾಡಿರುವರು... ವಿವೇಕಾನಂದರ ಸಂದೇಶದ ಪ್ರಚಾರದಿಂದ ಅನೇಕರು ಕ್ರೈಸ್ತ ಧರ್ಮವನ್ನು ತ್ಯಜಿಸುತ್ತಿರುವರು. 

 “ನೆನ್ನೆ ಸಾಯಂಕಾಲ ದಕ್ಷಿಣ ಲಂಡನ್ನಿನಲ್ಲಿ ನಾನು ಒಬ್ಬ ಸ್ನೇಹಿತನನ್ನು ನೋಡುವುದಕ್ಕೆ ಹೋಗುತ್ತಿದ್ದೆ. ನನಗೆ ದಾರಿ ತಪ್ಪಿತು. ನಾನು ಒಂದು ರಸ್ತೆಯ ಅಂಚಿನಲ್ಲಿ ನಿಂತುಕೊಂಡು ಎತ್ತ ಹೋಗಬೇಕು ಎಂದು ಆಲೋಚಿಸುತ್ತಿದ್ದೆ. ಆಗ ಒಬ್ಬ ಹುಡುಗನನ್ನು ಕರೆದುಕೊಂಡು ಹೋಗುತ್ತಿದ್ದ ಹೆಂಗಸೊಬ್ಬಳು ನನಗೆ ದಾರಿ ತೋರಬೇಕೆಂದು ಬಂದಳು. ಆಕೆ ‘ನಿಮಗೆ ದಾರಿ ತಪ್ಪಿದಂತೆ ಕಾಣುವುದು. ನಾನು ನಿಮಗೆ ಸಹಾಯ ಮಾಡಲೆ?’ ಎಂದು ಕೇಳಿದಳು. ಆಕೆ ನನಗೆ ಸರಿಯಾದ ದಾರಿಯನ್ನು ತೋರಿದಳು. ಅನಂತರ ‘ಕೆಲವು ವೃತ್ತಪತ್ರಿಕೆಗಳಿಂದ ನೀವು ಲಂಡನ್ನಿಗೆ ಬರುವುದಾಗಿ ತಿಳಿಯಿತು. ನಿಮ್ಮನ್ನು ದೂರದಲ್ಲಿ ನೋಡಿದ ತಕ್ಷಣವೇ ನನ್ನ ಮಗನಿಗೆ ‘ನೋಡು ಸ್ವಾಮಿ ವಿವೇಕಾನಂದ’ ಎಂದು ತೋರಿಸಿದೆ’ ಎಂದು ಆಕೆ ಹೇಳಿದಳು. ಆದರೆ ಆಗ ನಾನು ಬೇಗ ಹೋಗಿ ರೈಲನ್ನು ಹಿಡಿಯಬೇಕಾಗಿದ್ದುದರಿಂದ ನಾನು ವಿವೇಕಾನಂದ ಅಲ್ಲ ಎಂದು ಅವರಿಗೆ ಹೇಳಲು ಆಗಲಿಲ್ಲ. ಆ ಹೆಂಗಸು ವಿವೇಕಾನಂದರನ್ನು ಇನ್ನೂ ಕಂಡಿಲ್ಲದೇ ಇರುವಾಗಲೂ ಅಂತಹ ಭಕ್ತಿಯನ್ನು ಅವರ ಮೇಲೆ ಇಟ್ಟುಕೊಡಿದ್ದಳು. ನನಗಂತೂ ಆ ಘಟನೆ ಸೋಜಿಗವಾದರೂ ಆನಂದದಾಯಕವಾಯಿತು. ನಾನು ಧರಿಸಿದ್ದ ಕಾವಿಯ ಬಣ್ಣದ ರುಮಾಲಿಗೆ ಧನ್ಯವಾದಗಳು! ಸ್ವಾಮೀಜಿ ಉಪನ್ಯಾಸದ ಪರಿಣಾಮವಾಗಿ ಅನೇಕ ವಿದ್ವಾಂಸರಾದ ಆಂಗ್ಲೇಯರು ಭರತಖಂಡಕ್ಕೆ ಸಂಬಂಧಪಟ್ಟ ಧರ್ಮ ಮತ್ತು ತತ್ತ್ವ ವಿಷಯಗಳನ್ನು ಕೇಳಿ ಭರತಖಂಡವನ್ನು ಗೌರವಿಸುವುದನ್ನು ನಾನು ಕಣ್ಣಾರೆ ಕಂಡಿರುವೆನು.” 

 ಸ್ವಾಮೀಜಿಯವರಿಗೆ ಲಂಡನ್ನಿನ ಪ್ರಚಾರ ಕೆಲಸ ಬಹಳ ತೃಪ್ತಿಕರವಾಗಿ ಕಂಡಿತು. ಆಂಗ್ಲೇಯರು ಒಂದು ವಿಷಯವನ್ನು ಹಿಡಿಯುವುದು ನಿಧಾನ. ಆದರೆ ಒಮ್ಮೆ ಹಿಡಿದರೆಂದರೆ ಕೊನೆಯ ತನಕ ಅವರು ಬಿಡುವುದಿಲ್ಲ. ಆ ಸಮಯದಲ್ಲಿ ಸ್ವಾಮೀಜಿ ಅಮೇರಿಕಾದ ಶಿಷ್ಯರೊಬ್ಬರಿಗೆ ಬರೆದ ಕಾಗದದಲ್ಲಿ ಆಂಗ್ಲೇಯರ ಚಿತ್ರವನ್ನು ಹೀಗೆ ವಿವರಿಸುವರು: “ಆಂಗ್ಲೇಯರ ವಿಷಯದಲ್ಲಿ ನನಗಿದ್ದ ಭಾವನೆಗಳು ಸಂಪೂರ್ಣವಾಗಿ ಬದಲಾವಣೆ ಆದವು. ಇತರ ಎಲ್ಲಾ ಜನಾಂಗಗಳಿಗಿಂತ ಹೆಚ್ಚಾಗಿ ದೇವರು ಅವರನ್ನೇ ಏಕೆ ಅಷ್ಟೊಂದು ಅದೃಷ್ಟಶಾಲಿಗಳನ್ನಾಗಿ ಮಾಡಿರುವನು ಎಂಬುದು ನನಗೆ ಈಗ ಅರ್ಥವಾಗುತ್ತಿದೆ. ಅವರು ನಿಶ್ಚಲಚಿತ್ತರು, ಸಂಪೂರ್ಣವಾಗಿ ಪ್ರಾಮಾಣಿಕರು, ಮಹಾ ಗಾಢವಾದ ಮನೋಭಾವವುಳ್ಳವರು. ಕೇವಲ ನೋಟಕ್ಕೆ ವಿರಾಗಿಗಳಂತೆ ಕಂಡರೂ ಅದು ಜಾರಿಹೋದಾಗ ನಮಗೆ ಬೇಕಾದ ಮನುಷ್ಯ ಸಿಕ್ಕುವನು.” 

 ಅನಂತರ ಸ್ವಾಮೀಜಿ ಕಲ್ಕತ್ತೆಯಲ್ಲಿ ಇಂಗ್ಲೀಷಿನವರ ವಿಚಾರದಲ್ಲಿ ಮಾತನಾಡುತ್ತ ಹೀಗೆ ಹೇಳುವರು: “ನಾನು ಅಮೇರಿಕಾ ದೇಶದಲ್ಲಿ ಮಾಡಿದ ಕೆಲಸಕ್ಕಿಂತ ಇಂಗ್ಲೆಂಡಿನಲ್ಲಿ ಮಾಡಿದುದು ಹೆಚ್ಚು ತೃಪ್ತಿದಾಯಕವಾಗಿದೆ. ಧೀರ ಅಚಲ ಸ್ಥಿರ ಸ್ವಭಾವದ ಆಂಗ್ಲೇಯನಿಗೆ ವಿಷಯವನ್ನು ತಿಳಿದುಕೊಳ್ಳುವುದು ಇತರರಿಗಿಂತ ತಡವಾಗುವುದು. ಅವನಿಗೆ ಯಾವುದಾದರೂ ಹೊಸ ಭಾವನೆ ಕೊಟ್ಟರೆ ಅದು ಎಂದಿಗೂ ವ್ಯರ್ಥವಾಗುವುದಿಲ್ಲ. ಆ ಜನಾಂಗದಲ್ಲಿ ಶಕ್ತಿಬಾಹುಳ್ಯ ಮತ್ತು ಅದ್ಭುತವಾದ ವ್ಯವಹಾರ ಜ್ಞಾನ ಇರುವುದರಿಂದಲೇ ಭಾವನೆ ಬೇರೂರಿ ಬೇಗ ಫಲಕ್ಕೆ ಬರುವುದು, ಬೇರೆ ದೇಶದಲ್ಲಿ ಹಾಗಲ್ಲ. ಆ ಜನಾಂಗದಲ್ಲಿರುವ ಅದ್ಭುತ ಶಕ್ತಿ ಮತ್ತು ಕಾರ‍್ಯ ತತ್ಪರತೆ ಅನ್ಯ ಜನಾಂಗದಲ್ಲಿ ಇಲ್ಲ. ಅಲ್ಲಿ ಕಲ್ಪನೆ ಕಡಿಮೆ ಕಾರ‍್ಯ ಹೆಚ್ಚು. ಆಂಗ್ಲೇಯರ ಹೃದಯ ಮೂಲವನ್ನು ಯಾರು ಬಲ್ಲರು? ಅಲ್ಲಿ ಎಷ್ಟು ಕಲ್ಪನೆ ಭಾವನೆ ಇದೆ ಎಂಬುದು ಯಾರಿಗೆ ಗೊತ್ತು? ಅವರು ನಿಜವಾದ ಕ್ಷತ್ರಿಯರು, ಧೀರ ಜನಾಂಗ ಅವರದು. ತಮ್ಮ ಭಾವನೆಯನ್ನು ವ್ಯಕ್ತಪಡಿಸುವುದಿಲ್ಲ, ಯಾವಾಗಲೂ ಗೋಪ್ಯವಾಗಿಡುವರು. ಅವರ ಶಿಕ್ಷಣ ಹಾಗೆ ಇರುವುದು. ಆಂಗ್ಲೇಯರಲ್ಲಿ ತಮ್ಮ ಭಾವನೆಯನ್ನು ವ್ಯಕ್ತಪಡಿಸುವವರು ಬಹಳ ಅಪರೂಪ... ಇಷ್ಟೊಂದು ಧೀರ ಸ್ವಭಾವದ ಹಿಂದೆ, ಯೋಧನ ಸ್ವಭಾವದ ಹಿಂದೆ, ಆಂಗ್ಲೇಯನ ಹೃದಯದಲ್ಲಿ ಭಾವನಾ ಝರಿ ಇದೆ. ಅದರ ರಹಸ್ಯ ನಿಮಗೆ ಗೊತ್ತಾದರೆ ನೀವು ಅಲ್ಲಿಗೆ ಹೋದರೆ, ಅವರ ಪರಿಚಯವನ್ನು ಮಾಡಿಕೊಂಡು ನಿಕಟವಾಗಿ ಬೆರೆತರೆ, ಅವರು ತಮ್ಮ ಹೃದಯವನ್ನು ಬಿಚ್ಚುವರು. ಅವರು ಎಂದೆಂದಿಗೂ ನಿಮ್ಮ ಸ್ನೇಹಿತರು. ನೀವು ಹೇಳಿದಂತೆ ಕೇಳುವರು. ಆದಕಾರಣವೆ ನನ್ನ ದೃಷ್ಟಿಯಲ್ಲಿ ಬೇರೆ ಕಡೆಗಿಂತ ಇಂಗ್ಲೆಂಡಿನಲ್ಲಿ ಕೆಲಸ ಹೆಚ್ಚು ತೃಪ್ತಿಕರವಾಗಿದೆ. ನಾಳೆ ನಾನೇ ಕಾಲವಾದರೂ, ಇಂಗ್ಲೆಂಡಿನಲ್ಲಿ ಕೆಲಸ ಚ್ಯುತಿ ಬರುವುದಿಲ್ಲ. ಅದು ವೃದ್ಧಿಯಾಗುತ್ತ ಹೋಗುವುದೆಂದು ನಾನು ದೃಢವಾಗಿ ನಂಬುತ್ತೇನೆ.” 

 ಇಂಗ್ಲೆಂಡಿನಲ್ಲಿ ನಡೆದ ಸ್ವಾಮೀಜಿ ಜೀವನದ ಒಂದು ಘಟನೆಯನ್ನು ಇಲ್ಲಿ ಕೊಡುವೆವು. ಒಂದು ಸಲ ಸ್ವಾಮೀಜಿ ಮಿಸ್ ಮುಲ್ಲರ್ ಮತ್ತು ಮತ್ತೊಬ್ಬ ಇಂಗ್ಲೀಷ್ ಸ್ನೇಹಿತನೊಂದಿಗೆ ಒಂದು ಹೊಲದಲ್ಲಿ ಹೋಗುತ್ತಿದ್ದರು. ಆ ಸಮಯದಲ್ಲಿ ಹೋರಿಯೊಂದು ಕೋಪದಿಂದ ಇವರ ಕಡೆ ಧಾವಿಸಿ ಬಂದಿತು. ಆಗ ನಡೆದ ಪ್ರಸಂಗವನ್ನು ಮಾರ್ಗರೇಟ್ ನೋಬಲ್ ಹೀಗೆ ವಿವರಿಸುವಳು: 

 “ಆಂಗ್ಲೇಯನು ಓಡಿಹೋಗಿ ದಿಬ್ಬದ ಅತ್ತಕಡೆ ಸುರಕ್ಷಿತವಾಗಿ ಸೇರಿದನು. ಹೆಂಗಸಾದರೋ ತನ್ನ ಕೈಲಾದಮಟ್ಟಿಗೆ ಓಡಿ ಅನಂತರ ನೆಲದ ಮೇಲೆ ಬಿದ್ದಳು. ಇದನ್ನು ನೋಡಿದ ಸ್ವಾಮಿಗಳಿಗೆ ಆಕೆಗೆ ಬೇರೆ ವಿಧವಾದ ಸಹಾಯ ಮಾಡಲು ಆಗಲಿಲ್ಲ. ಅವರು ಅವಳನ್ನು ತಮ್ಮ ಹಿಂದೆ ಇಟ್ಟುಕೊಂಡು ಓಡಿಬರುತ್ತಿದ್ದ ಹೋರಿಯ ಕಡೆ ತಮ್ಮ ಕೈಗಳನ್ನು ಚಾಚಿ ನಿಂತುಕೊಂಡರು. ಅಂತೂ ಜೀವ ಹೀಗೆ ಕೊನೆಗಾಣಬೇಕಾಗುತ್ತದೆ ಎಂದು ಭಾವಿಸತೊಡಗಿದರು. ಅನಂತರ ಸ್ವಾಮೀಜಿ ಈ ಘಟನೆಯನ್ನು ವಿವರಿಸುತ್ತ ಆ ಹೋರಿ ತಮ್ಮನ್ನು ಎಷ್ಟು ದೂರಕ್ಕೆ ಒಗೆಯಬಹುದು ಎಂಬುದನ್ನು ಲೆಕ್ಕಾಚಾರ ಮಾಡುತ್ತಿದ್ದೆ ಎಂದು ಹೇಳುವರು. ಆದರೆ ಆ ಹೋರಿ ಕೆಲವು ಗಜಗಳ ದೂರದವರೆಗೆ ಬಂದು ಅನಂತರ ಬೇರೆ ಕಡೆಗೆ ಓಡಿಹೋಯಿತು.” 

 ಸಾಧಾರಣವಾಗಿ ಅಪಾಯ ಬಂದಾಗ ಮನುಷ್ಯ ತನ್ನನ್ನು ಸಂರಕ್ಷಿಸಿಕೊಳ್ಳಲು ಯತ್ನಿಸುವನು. ಆ ಕೆಲಸ ನಮ್ಮ ಇಚ್ಛೆಯಿಲ್ಲದೇ ಆಗುವುದು. ಆದರೆ ಸ್ವಾಮೀಜಿಯವರ ಜೀವನದಲ್ಲಿ ಅಂತಹ ಸಮಯದಲ್ಲಿಯೂ ಪರರನ್ನು ಉಳಿಸಬೇಕು, ಎಂಬುದರ ಕಡೆ ಮಾತ್ರ ಅವರ ಮನಸ್ಸು ಹೋಗುತ್ತಿತ್ತು. 

 ಡಿಸೆಂಬರ್ ೧೬ನೇ ತಾರೀಖು ಸೇವಿಯರ್ಸ್‍‍ ದಂಪತಿಗಳು ಮತ್ತು ಸ್ವಾಮೀಜಿ ಲಂಡನ್ ನಗರವನ್ನು ಬಿಟ್ಟು ನೇಪಲ್ಸ್ ಕಡೆಗೆ ಹೊರಟರು. ಗುಡ್‌ವಿನ್ ಅವರು ಸೌತ್ ಹ್ಯಾಮ್‌ಟನ್‌ನಿಂದ ನೇಪಲ್ಸ್‌ನಲ್ಲಿ ಸ್ವಾಮೀಜಿಯವರನ್ನು ಸಂಧಿಸುವುದಕ್ಕೆ ಹಡಗಿನಲ್ಲಿ ಹೊರಟರು. ಸ್ವಾಮೀಜಿಯವರನ್ನು ಬೀಳ್ಕೊಡುವ ಸಂದರ್ಭವನ್ನು ಸ್ಟರ್ಡಿ ಅವರು ತಮ್ಮ ಸ್ನೇಹಿತನಿಗೆ ಒಂದು ಪತ್ರದಲ್ಲಿ ಹೀಗೆ ವಿವರಿಸಿರುವರು:

 “ಇವತ್ತು ಸ್ವಾಮಿ ವಿವೇಕಾನಂದರು ನಮ್ಮನ್ನು ಅಗಲಿದರು. ಅವರಿಗೆ \enginline{Royal Institute of Painters in Water Colours} ಎಂಬ ಭವನದಲ್ಲಿ ಅದ್ಭುತವಾದ ಬೀಳ್ಕೊಡುಗೆ ಸಮಾರಂಭ ಜರುಗಿತು. ಸುಮಾರು ಐನೂರು ಜನಗಳು ನೆರೆದಿದ್ದರು. ಅನೇಕ ಜನ ಸ್ನೇಹಿತರು ಆ ಸಮಯದಲ್ಲಿ ಊರಿನಲ್ಲಿ ಇರಲಿಲ್ಲ. ಅವರ ಸ್ವಭಾವ ಅನೇಕರ ಹೃದಯದಲ್ಲಿ ಆಳವಾಗಿ ಪ್ರವೇಶಿಸಿದೆ. ಅವರ ಕೆಲಸವನ್ನು ನಾವು ಮುಂದೆ ಸಾಗಿಸಿಕೊಂಡು ಹೋಗುತ್ತಿರುವೆವು. ಅವರ ಗುರುಭಾಯಿಯೊಬ್ಬರು, ಒಳ್ಳೆಯ ಆಕರ್ಷಣೀಯವಾದ ತ್ಯಾಗೀ ಮನೋಭಾವದ ಯುವಕ ಸ್ವಾಮಿಗಳು ನನಗೆ ಸಹಾಯ ಮಾಡುವರು.” 

 “ನೀವು ಊಹಿಸಿದ್ದು ಸರಿ. ಈ ಜನ್ಮದಲ್ಲಿ ಸಂದರ್ಶಿಸಿದ ಶ್ರೇಷ್ಠತಮ ಸ್ನೇಹಿತ ಮತ್ತು ಪರಿಶುದ್ಧವಾದ ಗುರು ನನ್ನನ್ನು ಅಗಲಿ ಹೋಗುವುದರಿಂದ ನಾನು ವ್ಯಥೆ ಪಡುತ್ತಿರುವೆನು. ಹಿಂದಿನ ಜನ್ಮದ ಪುಣ್ಯವಿಶೇಷವಿರಬೇಕು, ಇಂತಹವರಿಂದ ಈ ಜನ್ಮದಲ್ಲಿ ಆಶೀರ್ವಾದವನ್ನು ನಾನು ಪಡೆದುದು. ನಾನು ಇಡೀ ಜೀವನದಲ್ಲಿ ಯಾವುದಕ್ಕಾಗಿ ಹಂಬಲಿಸುತ್ತಿದ್ದೆನೋ ಅದು ನನಗೆ ದೊರಕಿತು.”

