
\chapter{ಪೂರ್ವಿಕರು}

ಸ್ವಾಮಿ ವಿವೇಕಾನಂದರ ಜೀವನವನ್ನು ಓದುವುದೆಂದರೆ ಪೂರ್ಣತೆಯನ್ನು ಮುಟ್ಟಿದ ಒಂದು ಬೃಹತ್ ಜೀವಿಯ ವಿರಸನವನ್ನೆಲ್ಲ ಪರಿಚಯ ಮಾಡಿಕೊಂಡಂತೆ. ಜೀವನದ ಇತರ ಕಾರ‍್ಯಕ್ಷೇತ್ರಗಳಲ್ಲಿ ಪ್ರಖ್ಯಾತಿಯನ್ನು ಗಳಿಸಿದ ವ್ಯಕ್ತಿಗಳಾದರೋ ಮಧ್ಯದಲ್ಲಿ ಯಾವುದೋ ಒಂದು ಮೆಟ್ಟಲಿನಲ್ಲಿ ಹೋರಾಡುತ್ತಿರುವುದನ್ನು ನೋಡುವೆವು. ಪೂರ್ಣತೆಯ ಕಡೆಗೆ ಅವರ ಗಮನವಿರಬಹುದು. ಆದರೆ ಅದನ್ನು ಮುಟ್ಟಿದವರು ಅಪರೂಪ. ಸ್ವಾಮಿ ವಿವೇಕಾನಂದರಾದರೊ ಅಂಗೈ ಮೇಲಿನ ನೆಲ್ಲಿಕಾಯಿಯಂತೆ ಮುಕ್ತಿಯನ್ನು ವಶಪಡಿಸಿಕೊಂಡವರು. ಅವರ ಜೀವನ ಪಂಜರದಲ್ಲಿ ಸಿಕ್ಕಿದ ಕೇಸರಿಯ ಮರಿಯಂತೆ ಆರಂಭವಾಗುವುದು. ಪ್ರಚಂಡ ಸಾಧನೆಯನ್ನು ಮಾಡಿ, ತನ್ನ ಕ್ರೂರ ನಖಗಳಿಂದ ಬಂಧಿಸಿದ್ದ ಸಲಾಕೆಗಳನ್ನು ಕಿತ್ತು ಹೊರಬಂದು ಮುಕ್ತಿಯ ಸ್ವಾತಂತ್ರ್ಯದಿಂದ ಗರ್ಜಿಸುತ್ತಿರುವುದನ್ನು ಕೇಳುವೆವು. ತಾವು ಮಾತ್ರ ಮುಕ್ತಿಯನ್ನು ಗಳಿಸಿಕೊಂಡು ಧನ್ಯರಾಗಿ ಹೋದವರಲ್ಲ ಇವರು, ಭವ ಜೀವಿಗಳಿಗೆಲ್ಲ ಮುಕ್ತಿಯ ದಾರಿಯನ್ನು ತೋರಿದರು. ಇವರು ಯಾವುದೋ ಒಂದು ಸಣ್ಣ ಜಾತಿಯ ದೃಷ್ಟಿಯಿಂದ, ಮತದ ದೃಷ್ಟಿಯಿಂದ ಒಂದು ಮಾರ್ಗವನ್ನು ಮಾಡಿದವರಲ್ಲ. ಬದ್ಧಜೀವಿ ಯಾವ ಕಾಲದಲ್ಲಿರಲಿ, ಯಾವ ದೇಶದಲ್ಲಿ ಇರಲಿ, ಯಾವ ಕುಲಗೋತ್ರಗಳಿಗೆ ಸೇರಿರಲಿ, ಯಾವ ಸಂಸ್ಕಾರದ ಬಲೆಗೆ ಸಿಕ್ಕಿ ನರಳುತ್ತಿರಲಿ ಎಲ್ಲರ ನೆರವಿಗೆ ಒದಗಿಬರುವ ಸಂದೇಶ ಇವರದು.

ಸ್ವಾಮಿ ವಿವೇಕಾನಂದರು ಆಡಿದ ಮಾತಿಗಿಂತ, ಬರೆದ ಬರಹಗಳಿಗಿಂತ ಶತಪಾಲು ದೊಡ್ಡದು ಅವರ ಜೀವನ. ಅವರ ಬದುಕಿನ ಕಥೆ ಜೀವನದ ಅಂಧಕಾರವನ್ನು ಪರಿಹರಿಸುವುದು, ಬೇಸತ್ತ ಜೀವಿಗೆ ಒಂದು ನವಸ್ಫೂರ್ತಿಯನ್ನು ಕೊಡುವುದು, ಚೇತನವನ್ನು ತುಂಬುವುದು, ಹೊಸ ಶಕ್ತಿಯ ಝರಿಯೊಂದು ಆತನ ಜೀವನದಲ್ಲಿ ಜಿನುಗುವಂತೆ ಮಾಡುವುದು. ನಾವು ಅವರ ಬದುಕನ್ನು ಒಂದು ಕಥೆ ಎಂದು ಕರೆದೆವು. ಅನೇಕ ವೇಳೆ ಕಥೆ ಎಂದರೆ ಕೇವಲ ಕಲ್ಪನೆಯ ಸೃಷ್ಟಿ, ಇಲ್ಲವೆ ಘಟನೆಗಳ ಒಂದು ಜೋಡಣೆ ಎಂಬ ಅರ್ಥ ಬರುವುದು. ಇಲ್ಲಿ ಕಥೆ ಎಂಬುದನ್ನು ಈ ದೃಷ್ಟಿಯಿಂದ ಉಪಯೋಗಿಸಿಲ್ಲ. ಕಥೆ ಎಂಬ ಪದದಲ್ಲಿ ಒಂದು ಸಮ್ಮೋಹನ ಶಕ್ತಿ ಇದೆ, ಆಕರ್ಷಕ ಶಕ್ತಿ ಇದೆ, ಮುಂದೇನಾಗುವುದು ಎಂಬ ಕುತೂಹಲವನ್ನು ಕೆರಳಿಸುವ ಶಕ್ತಿ ಇದೆ. ನಾವು ಆ ದೃಷ್ಟಿಯಿಂದ ಈ ಪದವನ್ನು ಉಪಯೋಗಿಸಿರುವೆವು. ಸತ್ಯ ಮಿಥ್ಯಕ್ಕಿಂತ ಆಶ್ಚರ್ಯಭರಿತವಾಗಿದೆ, ಎಂಬ ವಾಕ್ಯ ಇಂಗ್ಲೀಷಿನಲ್ಲಿ ಇದೆ. ಸ್ವಾಮಿ ವಿವೇಕಾನಂದರ ಜೀವನ ಅದಕ್ಕೆ ಒಂದು ಉದಾಹರಣೆ. ಸ್ವಾಮಿ ವಿವೇಕಾನಂದರ ಜೀವನದಲ್ಲಿ ನಾವು ಮುಂದೆ ಏನನ್ನು ಚಿತ್ರಿಸುವೆವೊ ಅಲ್ಲಿ ಉತ್ಪ್ರೇಕ್ಷೆಯಿಲ್ಲ, ಅನೃತವಿಲ್ಲ, ಯಾವ ಪವಾಡಗಳೂ ಇಲ್ಲ. ಆದರೂ ನವರಸಭರಿತವಾಗಿದೆ, ರೋಮಾಂಚಕಾರಿಯಾಗಿದೆ, ಚೇತೊಹಾರಿಯಾಗಿದೆ. ಸ್ವಾಮಿ ವಿವೇಕಾನಂದರು ಎಂಬ ಭೀಮ ವ್ಯಕ್ತಿ ಭೋರ್ಗರೆದು ಹರಿಯುತ್ತಿರುವ ಸಂಸಾರವೆಂಬ ಮಹಾ ಪ್ರವಾಹದ ಮೇಲೆ ಅದಕ್ಕೆ ವಿರೋಧವಾಗಿ ಈಜುತ್ತಿರುವುದನ್ನು ನೋಡುವೆವು. ನೂರಾರು ಜೀವಿಗಳನ್ನು ನುಂಗಿ ನೊಣೆದಿರುವ ಈ ಮಹಾ\break ಪ್ರವಾಹದಲ್ಲಿರುವ ಸುಳಿಗಳಿಂದ ಹೇಗೆ ತಪ್ಪಿಸಿಕೊಳ್ಳುವರು, ಮೇಲೆದ್ದು ಇವರನ್ನು ಅಪ್ಪಳಿಸಲು ಬರುತ್ತಿರುವ ನೊರೆತಗಳಿಂದ ಹೇಗೆ ಪಾರಾಗುವರು, ಭವಸಾಗರದ ಅತ್ತ ಇರುವ ಅಭಯದ ಕರೆಯನ್ನು ಹೇಗೆ ಸೇರುವರು, ಸೇರಿದ ಮೇಲೆ ಅತ್ತ ಕಡೆಗೆ ಇತರರನ್ನು ಹೇಗೆ ಕೊಂಡೊಯ್ಯುವರು, ಹಾಗೆ ಕೊಂಡೊಯ್ಯುವಾಗ ತಾವು ನಾಶವಾದರೂ ಚಿಂತೆಯಿಲ್ಲ ಇತರರು ಉದ್ಧಾರವಾಗಲಿ ಎಂಬ ಅವರ ವಿಶ್ವಾನುಕಂಪ – ಇವುಗಳ ಪರಿಚಯ ಮಾಡಿಕೊಳ್ಳುವುದೇ ಒಂದು ರಸದೂಟ, ಇವುಗಳನ್ನು ಮನನ ಮಾಡುವುದೇ ಒಂದು ಸಾಧನೆ, ತಪಸ್ಸು, ಚಿತ್ತಶುದ್ಧಿಗೆ ಉಪಾಯ.

ಸ್ವಾಮಿ ವಿವೇಕಾನಂದರು ಈ ಧರೆಗೆ ಇಳಿದು ಬರುವಾಗಲೇ ಒಂದು ಮಹತ್ತಾದ ತಪಸ್​ಶಕ್ತಿಯನ್ನು ತಂದಿದ್ದರು. ಬಡಪಾಯಿಯಂತೆ ಅವರು ಇಲ್ಲಿಗೆ ಬರಲಿಲ್ಲ. ಆದರೆ ಅವರಲ್ಲಿದ್ದ ಆ ಶಕ್ತಿ ಇಲ್ಲಿ ಬೇಗ ವಿಕಾಸವಾಗಬೇಕಾದರೆ ಅದಕ್ಕೆ ಯೋಗ್ಯ ವಾತಾವರಣ ಬೇಕಾಗಿತ್ತು. ಬೇರೆ ಎಲ್ಲಿಯೋ ಹುಟ್ಟದೆ ಯಾವುದೋ ಒಂದು ದಂಪತಿಗಳ ಏಣಿಯ ಮೂಲಕ ಇಳಿದು ಬರಬೇಕಾದರೆ ಆ ವಂಶದಲ್ಲಿ ಒಂದು ಪವಿತ್ರತೆ ಇರಬೇಕು. ಮಹಾತ್ಮರು “ಹುಟ್ಟಿದ ಕುಲ ಪವಿತ್ರ, ಹೆತ್ತ ತಾಯಿ ಧನ್ಯಳು, ಹೊತ್ತ ದೇಶ ಪುಣ್ಯವತಿಯಾಗುವಳು” ಎಂಬ ಹಿಂದಿನಿಂದ ಬಂದ ನಾಣ್ನುಡಿಯೊಂದು ಇರುವುದು. ಸ್ವಾಮಿ ವಿವೇಕಾನಂದರು ಜನ್ಮತಾಳಿದ್ದು ಕಾಯಸ್ಥ ಅಥವಾ ಕ್ಷತ್ರಿಯ ಕುಲದಲ್ಲಿ. ಅವರು ದತ್ತ ವಂಶಕ್ಕೆ ಸೇರಿದವರು. ಅವರು ಜನ್ಮತಾಳಿದ ವಂಶದಲ್ಲಿ ಎರಡು ಗುಣಗಳು ಮೇಲೆದ್ದು ಕಾಣುವುವು. ಅವೇ ವಕೀಲಿಯ ವೃತ್ತಿ ಮತ್ತು ವೈರಾಗ್ಯ ಪ್ರವೃತ್ತಿ. ಸ್ವಾಮಿ ವಿವೇಕಾನಂದರ ತಂದೆ ವಕೀಲರು, ಅಜ್ಜ ವಕೀಲರು, ಮುತ್ತಜ್ಜ ವಕೀಲರು. ವಿವೇಕಾನಂದರೂ ಕೂಡ ಲಾ ಕಾಲೇಜಿನಲ್ಲಿ ಓದಿ ಲಾಯರ್ ವೃತ್ತಿಗೆ ಕೈ ಹಾಕಬೇಕೆಂದು ಇದ್ದವರು. ವಿಧಿ ಬೇರೊಂದು ಬಗೆಯಿತು. ಅವರು ಆ ವೃತ್ತಿಗೆ ಕೈ ಹಾಕಿ ದ್ರವ್ಯಾರ್ಜನೆ ಮಾಡಲಾಗಲಿಲ್ಲ. ಅವರು ಸಭಿಕರೆದುರಿಗೆ ವಿಷಯಗಳನ್ನು ಪ್ರತಿಪಾದನೆ ಮಾಡುವ ರೀತಿ ಅಮೋಘವಾಗಿತ್ತು. ಅವರು ತಮ್ಮ ಉಪನ್ಯಾಸಗಳಲ್ಲಿ ವಿಷಯಗಳನ್ನು ಜೋಡಿಸುವ ರೀತಿಯಲ್ಲಿ, ಅದನ್ನು ಪರಿಣಾಮಕಾರಿಯಾಗಿ ಹೇಳುವ ರೀತಿಯಲ್ಲಿ, ವಕೀಲನ ವೃತ್ತಿಯ ನೈಪುಣ್ಯತೆಯನ್ನೆಲ್ಲ ಉಪಯೋಗಿಸುವರು. ಅವರ ಉಪನ್ಯಾಸದಲ್ಲಿರುವ ವಿಷಯ ಜೋಡಣೆಗಳನ್ನು ನೋಡಿದವರು “ಭಾವ ಜೋಡಣೆಯಲ್ಲಿ ಇವರೊಬ್ಬ ಮಂತ್ರವಾದಿಯಂತೆ ಇರುವರು” ಎಂದು ಹೇಳುವರು. ಇದು ವಾಗ್ ವೈಖರಿ ಮಾತ್ರವಲ್ಲ, ಅದು ಕೇವಲ ಮೇಲಿರುವ ಸೌಂದರ‍್ಯ. ಅದರ ಒಳಗೆ ಇರುವ ಭಾವಗಳ ಜೋಡಣೆಯ\break ಸೌಂದರ‍್ಯ ಅಷ್ಟೇ ಚಮತ್ಕಾರವಾಗಿತ್ತು.

ಸ್ವಾಮಿ ವಿವೇಕಾನಂದರ ವಂಶದ ರಕ್ತದಲ್ಲಿ ಹರಿಯುತ್ತಿದ್ದ ಮತ್ತೊಂದು ಪ್ರವೃತ್ತಿಯೇ ವೈರಾಗ್ಯ. ಇವರ ಅಜ್ಜ ತನ್ನ ಇಪ್ಪತ್ತೈದನೆಯ ವಯಸ್ಸಿನಲ್ಲಿಯೇ ಉತ್ತಮ ವಕೀಲನಾಗಿ ಬೇಕಾದಷ್ಟು ಧನವನ್ನು ಸಂಪಾದಿಸುತ್ತ ಹೆಂಡತಿ ಮಕ್ಕಳೊಂದಿಗೆ ಸಂತೋಷದಿಂದ ಇದ್ದ ಕಾಲದಲ್ಲಿಯೇ ವೈರಾಗ್ಯದಿಂದ ಎಲ್ಲವನ್ನೂ ತ್ಯಜಿಸಿ ಸಂನ್ಯಾಸಿಯಾಗಿ ಹೋಗುವರು. ಸ್ವಾಮಿ ವಿವೇಕಾನಂದರ ತಂದೆ ಮನೆ ಬಿಟ್ಟು ಹೋಗಲಿಲ್ಲ. ಆದರೆ ಆ ಮನುಷ್ಯನಲ್ಲಿದ್ದ ಅನುಕಂಪ ದಯೆ ಸಹಾನುಭೂತಿಗಳು ಯಾವ ಸಂನ್ಯಾಸಿಗಾದರೂ ಭೂಷಣಪ್ರಾಯವಾಗಿದ್ದವು. ಸ್ವಾಮಿ ವಿವೇಕಾನಂದರ ಜೀವನದಲ್ಲಿ ಈ ಗುಣಗಳೆರಡೂ ಹಾಸುಹೊಕ್ಕಾಗಿವೆ.

ಸ್ವಾಮಿ ವಿವೇಕಾನಂದರ ಪೂರ್ವಾಶ್ರಮದ ಹೆಸರು ನರೇಂದ್ರನಾಥದತ್ತ ಎಂದು. ಈ ದತ್ತವಂಶದ ಮೂರು ನಾಲ್ಕು ತಲೆಮಾರಿನವರೆಲ್ಲ ಒಂದೇ ವೃತ್ತಿಯಲ್ಲಿದ್ದುದನ್ನು ನೋಡಿದೆವು. ನರೇಂದ್ರನಾಥನ ಮುತ್ತಜ್ಜ ರಾಮಮೋಹನದತ್ತ ಎಂಬುವನು. ಆತ ಒಂದು ಇಂಗ್ಲೀಷ್ ಫರ್ಮಿನಲ್ಲಿ ಕೆಲಸ ಮಾಡುತ್ತಿದ್ದುದಲ್ಲದೆ ಅದಕ್ಕೆ ಸೇರಿದ ವ್ಯವಹಾರವನ್ನೆಲ್ಲ ನೋಡಿಕೊಳ್ಳುತ್ತಿದ್ದ ವಕೀಲಿ ವೃತ್ತಿಯ ಫರ್ಮಿನಲ್ಲಿ ಕೆಲಸಕ್ಕೆ ಇದ್ದ. ಅವನು ಕಲ್ಕತ್ತೆಯಲ್ಲಿ ಗೌರಮೋಹನ ಮುಖರ್ಜಿ ರಸ್ತೆಯಲ್ಲಿರುವ ದೊಡ್ಡ ಮನೆಯಲ್ಲಿದ್ದನು. ಬೇಕಾದಷ್ಟು ನೆಂಟರಿಷ್ಟರು, ಆಳುಕಾಳುಗಳು ಇದ್ದರು. ದೊಡ್ಡ ಸಂಸಾರ, ಅಷ್ಟೇ ದೊಡ್ಡ ಹೃದಯ ಈ ವಂಶದವರದು. ಈತನಿಗೆ ಇಬ್ಬರು ಗಂಡುಮಕ್ಕಳಿದ್ದರು. ಅವರೆ ದುರ್ಗಾಚರಣದತ್ತ ಮತ್ತು ಕಾಲೀಪ್ರಸಾದದತ್ತ. ದುರ್ಗಾಚರಣದತ್ತ ಪಾರ್ಸಿ ಮತ್ತು ಇಂಗ್ಲೀಷ್ ಭಾಷೆಯಲ್ಲಿ ಪ್ರವೀಣನಾಗಿ ವಕೀಲಿ ವೃತ್ತಿಗೆ ತಾನೂ ತಂದೆಯೊಡನೆ ಇಳಿದವನು. ಬೇಕಾದಷ್ಟು ಹಣವನ್ನು ಸಂಪಾದಿಸುತ್ತಿದ್ದ. ಪ್ರಖ್ಯಾತ ವಕೀಲನೂ ಆದ. ಪ್ರಾಪ್ತ ವಯಸ್ಸು ಬಂದಾಗ ಅವನಿಗೆ ಒಳ್ಳೆಯ ಕಡೆಯಿಂದ ಹೆಣ್ಣನ್ನು ತಂದು ಮದುವೆಯನ್ನೂ ಮಾಡಿದರು. ಪ್ರಾಪಂಚಿಕ ದೃಷ್ಟಿಯಿಂದ ನೋಡಿದರೆ ಎಲ್ಲವೂ ಸರಿಯಾಗಿತ್ತು. ಬೇಕಾದಷ್ಟು ಹಣವನ್ನು ಕೊಡುತ್ತಿದ್ದ ವಕೀಲಿ ವೃತ್ತಿ ಇನ್ನೂ ಬೆಳೆಯುತ್ತಿತ್ತು. ಸುಂದರವಾದ ಯುವತಿ ಪತ್ನಿ ಇದ್ದಳು. ಜೊತೆಗೆ ಆಗತಾನೆ ಒಂದು ಗಂಡುಮಗು ಬೇರೆ ಆಗಿತ್ತು. ಅದಕ್ಕೆ ವಿಶ್ವನಾಥದತ್ತ ಎಂದು ನಾಮಕರಣ ಮಾಡಿದ್ದರು. ಆದರೆ ದುರ್ಗಾಚರಣದತ್ತನಿಗೆ ಎಲ್ಲಾ ಇದ್ದರೂ ಒಂದು ಕೊರತೆ ಇವನ ಬಾಳನ್ನೆಲ್ಲಾ ದಹಿಸುತ್ತಿತ್ತು. ಈ ಲೌಕಿಕ ಜೀವನದಲ್ಲಿ ಏನು ಬೇಕೋ ಅದೆಲ್ಲಾ ಇದ್ದರೂ ಅವನಿಗೆ ಅದಾವುದೂ ರುಚಿಸಲಿಲ್ಲ. ಎಷ್ಟೇ ಪ್ರಯತ್ನ ಮಾಡಿದರೂ ಸಂಸಾರಕ್ಕೆ ಹೊಂದಿಕೊಂಡು ಇರಲು ಸಾಧ್ಯವಾಗಲಿಲ್ಲ. ಒಂದು ದಿನ ಇದ್ದಕ್ಕಿದ್ದಂತೆಯೇ ದುರ್ಗಾಚರಣದತ್ತ ಮನೆಯನ್ನು ತ್ಯಜಿಸಿ ಹೊರಟು ಹೋದ. ಇಂದು ಬರುವನು, ನಾಳೆ ಬರುವನು ಎಂದು ಮನೆಯವರೆಲ್ಲ ಕಾದರು. ಆದರೆ ಆತನಿಂದ ಸುದ್ದಿ ಸಮಾಚಾರವೇ ಇಲ್ಲ. ಕಾಶಿಯಲ್ಲಿ ಆತ ಸಂನ್ಯಾಸಿ ಆಗಿ ಹೋಗಿರುವನು ಎಂದು ಯಾರೋ ಸುದ್ದಿಯನ್ನು ಹರಡಿದರು. ಅವನ ಕೈ ಹಿಡಿದ ಸತಿ ಪರಿಸ್ಥಿತಿಯನ್ನು ಎದುರಿಸಬೇಕಾಗಿ ಬಂತು. ಹಿಂದೂ ಸಮಾಜದಲ್ಲಿ ಗಂಡ ಸಂನ್ಯಾಸಿಯಾದರೆ ಹೆಂಡತಿಯ ಪಾಲಿಗೆ ಅವನು ಇಲ್ಲದಂತೆಯೇ. ಆದರೆ ದುರ್ಗಾಚರಣನ ಸತಿ ಧೀರಳು. ತನ್ನ ಪಾಲಿಗೆ ಬಂದ ದುಃಖವನ್ನು ಮೆಲಕುತ್ತ ಕೊರಗಿ ವಿಧಿಯನ್ನು ಜರೆಯುವ ಗುಂಪಿಗೆ ಸೇರಿದವಳಲ್ಲ. ಪರಿಸ್ಥಿತಿಯನ್ನು ಧೈರ್ಯವಾಗಿ ಎದುರಿಸಿದಳು. ಮಗುವಿನ ಲಾಲನೆ ಪಾಲನೆಯಲ್ಲಿ ನಿರತಳಾದಳು. ತಂದೆಯ ಪಾಲಿನ ಪ್ರೇಮವನ್ನೂ ತಾನೇ ಕೊಟ್ಟು\break ಸಲಹಿದಳು.

ವಿಶ್ವನಾಥದತ್ತನಿಗೆ ಮೂರು ವರ್ಷ ವಯಸ್ಸಾಯಿತು. ದುರ್ಗಾಚರಣದತ್ತನ ಸತಿಗೆ ಒಂದು ಸಲ ಮಗುವಿನೊಡನೆ ಕಾಶಿಗೆ ಹೋಗಬೇಕು ಎನ್ನಿಸಿತು. ಒಂದು ವೇಳೆ ಪತಿ ಅಲ್ಲಿದ್ದರೆ ಆತನ ದರ್ಶನ ಸಿಕ್ಕೀತೇನೋ ಎಂಬ ದೂರದ ಆಸೆಯೊಂದು ಅವಳ ಹೃದಯದಲ್ಲಿತ್ತು. ಆಕೆ ತನ್ನ ಬಯಕೆಯನ್ನು ವ್ಯಕ್ತಪಡಿಸಿದ ಮೇಲೆ ಮನೆಯಲ್ಲಿದ್ದ ಕೆಲವು ಹಿರಿಯರು ಕಾಶಿಗೆ ಅವಳೊಡನೆ ಹೋಗಲು ಒಪ್ಪಿದರು. ಆಗಿನ ಕಾಲದಲ್ಲಿ ಕಾಶಿಯಿಂದ ಕಲ್ಕತ್ತೆಗೆ ಇನ್ನೂ ರೈಲನ್ನು ಹಾಕಿರಲಿಲ್ಲ. ಹೋಗಬೇಕಾದರೆ ಗಂಗಾ ನದಿಯ ಮೇಲೆ ದೋಣಿಯಲ್ಲಿ ಪ್ರಯಾಣ ಮಾಡಬೇಕಾಗಿತ್ತು. ಅದಕ್ಕಾಗಿ ಒಂದು ದೋಣಿಯನ್ನು ಗೊತ್ತುಮಾಡಿದರು. ಅದರಲ್ಲಿ ಸುಮಾರು ಮೂರು ವಾರಗಳ ಪ್ರಯಾಣ ಮಾಡಬೇಕು. ಅಡಿಗೆ ಊಟ ನಿದ್ರೆ ಎಲ್ಲಾ ದೋಣಿಯಲ್ಲೇ ಆಗಬೇಕಾಗಿತ್ತು. ಅದಕ್ಕೆ ಬೇಕಾಗುವ ಸಾಮಾನುಗಳನ್ನೆಲ್ಲ ಅಣಿಗೊಳಿಸಿಕೊಂಡು ಕಾಶಿಯಾತ್ರೆಗೆ ಹೊರಟರು.

ಕಲ್ಕತ್ತೆಯಿಂದ ಕಾಶಿಗೆ ಹೋಗುವಾಗ ತೀರದಲ್ಲಿ ಸಿಕ್ಕುವ ಊರುಗಳನ್ನೆಲ್ಲ ನೋಡಿಕೊಂಡು ಹೊರಟರು. ಪವಿತ್ರ ಗಂಗಾನದಿಯ ತೀರದಲ್ಲಿ ಹಲವು ದೇವಸ್ಥಾನಗಳು ಸಿಕ್ಕುವುವು. ದೋಣಿಯಲ್ಲಿ ಪ್ರತಿದಿನ ಪ್ರಾರ್ಥನೆ ಪಾರಾಯಣ ಮುಂತಾದುವು ನಡೆದೇ ಇದ್ದುವು. ದೋಣಿ ಇನ್ನೇನು ಪವಿತ್ರ ಕಾಶಿಯನ್ನು ಸಮೀಪಿಸುತ್ತಿದೆ. ದೂರದಲ್ಲಿ ಕಾಶಿಯಲ್ಲಿರುವ ಅಸಂಖ್ಯ ಗುಡಿಗೋಪುರಗಳು ಕಾಣುತ್ತಿವೆ. ದುರ್ಗಾಚರಣನ ಸತಿಯ ಮನಸ್ಸಿನಲ್ಲಿದ್ದ ಕಾಶಿಯ ವಿಶ್ವನಾಥ ದರ್ಶನ, ಜೊತೆಗೆ ಆ ವಿಶ್ವೇಶ್ವರ ಅನುಗ್ರಹಿಸಿದರೆ, ತನ್ನ ಪತಿಯ ಮುಖದರ್ಶನ ಸಿಕ್ಕುವ ಸಮಯ ಸಮೀಪಿಸುತ್ತಿದೆ. ಅವಳ ಮನಸ್ಸಿನಲ್ಲಿ ಒಂದು ಅವರ್ಣನೀಯ ಆನಂದ ವ್ಯಕ್ತವಾಗುತ್ತಿತ್ತು.

ಆ ಸಮಯದಲ್ಲೇ ಒಂದು ಘಟನೆ ನಡೆಯಿತು. ಈ ಜೀವನದಲ್ಲಿ ಇನ್ನೇನು ನಾವು ನೆನೆಸಿದುದು ಸಿಕ್ಕಿತು ಎಂದು ಭಾವಿಸುವುದರಲ್ಲಿಯೇ ನಾವು ಕನಸಿನಲ್ಲಿಯೂ ಊಹಿಸದ ಪ್ರಸಂಗಗಳು ಜರುಗಿಹೋಗುವುವು. ವಿಶ್ವನಾಥ ಮೂರು ವರುಷದ ಮಗು ದೋಣಿಯ ಅಂಚಿನಲ್ಲಿ ಆಡುತ್ತಿದ್ದಾಗ ಕಾಲುಜಾರಿ ಅದು ದೋಣಿಯಿಂದ ನದಿಗೆ ಮಗುಚಿಕೊಂಡಿತು. ಅದಕ್ಕೆ ಈಜು ಬರದು ಆ ವಯಸ್ಸಿನಲ್ಲಿ. ಆ ಮಗು ಬಿದ್ದುದನ್ನು ನೋಡಿದೊಡನೆಯೆ ತಾಯಿಯೂ ಕೂಡ ತನಗೆ ಈಜು ಬರದೇ ಇದ್ದರೂ ಆ ಮಗುವಿನ ಹಿಂದೆಯೇ ನೀರಿಗೆ ಬಿದ್ದಳು. ನೀರಿನಲ್ಲಿ ಆ ಮಗುವನ್ನು ತನ್ನ ಕೈಗಳಿಂದ ಬಲವಾಗಿ ಹಿಡಿದುಕೊಂಡಳು. ದೋಣಿಯಲ್ಲಿ ಈ ಘಟನೆಯನ್ನು ಕಂಡ ಈಜುಬಲ್ಲ ಕೆಲವರು ತಕ್ಷಣವೇ ನೀರಿಗೆ ಬಿದ್ದು ಆ ಮಗು ಮತ್ತು ತಾಯಿಯನ್ನು ದೋಣಿಯ ಮೇಲಕ್ಕೆ ತಂದರು. ತಾಯಿಗೆ ಪ್ರಜ್ಞೆ ತಪ್ಪಿಹೋಗಿತ್ತು. ಆದರೂ ಮಗುವನ್ನು ಹಿಡಿದ ಮುಷ್ಟಿ ಮಾತ್ರ ಸಡಿಲವಾಗಿರಲಿಲ್ಲ, ಆ ತಾಯಿ ಮಗುವನ್ನು ಅಷ್ಟು ಬಿಗಿಯಾಗಿ ಹಿಡಿದಿದ್ದಳು. ಆ ಮಗುವಿಗೆ ಕೆಲವು ವರುಷಗಳಾದ ಮೇಲು ಆ ಮಚ್ಚೆ ಹೋಗಲಿಲ್ಲ. ವಿಶ್ವನಾಥದತ್ತ ಮುಂದೆ ತಾಯಿಯಿಂದಾದ ತನ್ನ ಕೈ ಮೇಲಿರುವ ಮಚ್ಚೆಯನ್ನು ಇತರರಿಗೆ ತೋರುತ್ತಿದ್ದ. ಮೃತ್ಯುವಿನ ದವಡೆಯೊಳಗಿಂದ ಮಗುವನ್ನು ಎಳೆದು ತಂದ ಮಾತೃಪ್ರೇಮದ ಕುರುಹಾಗಿತ್ತು ಅದು.

ಕಾಶಿಯ ಊರನ್ನು ಸೇರಿದ ಮೇಲೆ ಅವರ ಮನೆಯ ದೇವರಾದ ವೀರೇಶ್ವರನ ದೇವಸ್ಥಾನಕ್ಕೆ ಹೊಗಿ ಪ್ರಾರ್ಥನೆಯನ್ನು ಮಾಡಿದರು. ಅನಂತರ ಅಲ್ಲಿರುವ ಇತರ\break ದೇವಾಲಯಗಳನ್ನೂ ಪ್ರತಿದಿನ ಹೋಗಿ ನೋಡುತ್ತಿದ್ದರು. ದುರ್ಗಾಚರಣನ ಹೆಂಡತಿ ಒಂದು ದಿನ ಗಂಗಾಸ್ನಾನ ಮಾಡಿ ವಿಶ್ವನಾಥನ ದರ್ಶನಕ್ಕೆ ರಸ್ತೆಯಲ್ಲಿ ನಡೆದುಕೊಂಡು ಹೊಗುತ್ತಿದ್ದಳು. ಆಗ ಅಕಸ್ಮಾತ್ತಾಗಿ ಎಡವಿ ಬಿದ್ದಳು. ನೆಲದ ಮೇಲೆ ಜೋರಾಗಿ ಬಿದ್ದುದರಿಂದ ಪ್ರಜ್ಞೆ ತಪ್ಪಿತು. ಆ ಸ್ಥಿತಿಯಲ್ಲಿ ಹೆಂಗಸು ಹಾಗೆ ಬಿದ್ದಿದ್ದಾಗ ರಸ್ತೆಯಲ್ಲಿ ಹೊಗುತ್ತಿದ್ದ ಸಂನ್ಯಾಸಿಯೊಬ್ಬ ಅವಳನ್ನು ಎತ್ತಿ ಪಕ್ಕದಲ್ಲಿದ್ದ ಮನೆಯ ಜಗುಲಿಯ ಮೇಲೆ ಮಲಗಿಸಿದ. ಆ ಸಮಯಕ್ಕೆ ಸರಿಯಾಗಿ ಹೆಂಗಸು ಕಣ್ಣನ್ನು ತೆರೆದಳು. ಆಶ್ಚರ್ಯ! ಯಾವ ಪತಿಯ ದರ್ಶನ ಸಿಕ್ಕಬಹುದೇನೊ ಎಂದು ಆಶಿಸಿ ಕಾಶಿಗೆ ಬಂದಿದ್ದಳೊ ಆತನೇ ಈಕೆಯನ್ನು ಎತ್ತಿ ಒಂದು ಕಡೆ ಮಲಗಿಸುತ್ತಿರುವನು. ಆ ಸಂನ್ಯಾಸಿಗೆ ಈ ಹೆಂಗಸಿನ ಪರಿಚಯವೂ ಆಯಿತು. ಈಕೆಯೇ ತಾನು ಹಿಂದೆ ಕೈ ಹಿಡಿದ ಸತಿ ಎಂದು ಹೊಳೆಯಿತು. ತತ್‍ಕ್ಷಣವೇ ಆ ಸಂನ್ಯಾಸಿ ಆಕೆಯನ್ನು ಅಲ್ಲಿ ಹಾಗೆಯೇ ಬಿಟ್ಟು ಮಾಯೆಯ ಪಾಶಕ್ಕೆ ಒಳಗಾಗಕೂಡದು ಎಂದು ಹೊರಟೇ ಹೋದನು. ಆ ಸತಿಯ ಪ್ರಾರ್ಥನೆಯನ್ನೇನೋ ಶಿವ ತನ್ನದೇ ವಿಚಿತ್ರ ರೀತಿಯಲ್ಲಿ ಈಡೇರಿಸಿದ. ಅವಳಿಗೆ ಪತಿಯ ದರ್ಶನ ಕೊಟ್ಟ, ಆದರೆ ಪತಿಯನ್ನು ಕೊಡಲಿಲ್ಲ.

ಕಾಶಿಯ ಯಾತ್ರೆಯನ್ನು ಪೂರೈಸಿಕೊಂಡು ಎಲ್ಲರೂ ಕಲ್ಕತ್ತೆಗೆ ಹಿಂದಿರುಗಿದರು. ಕೆಲವು ಕಾಲದ ಮೇಲೆ ಸಂನ್ಯಾಸಿಯಾದ ದುರ್ಗಾಚರಣನು ಕಲ್ಕತ್ತೆಗೆ ಬಂದನು. ಬಂದವನು ತನ್ನ ಪುರ್ವಾಶ್ರಮದ ಮನೆಗೆ ಹೊಗಲಿಲ್ಲ. ತನ್ನ ಸ್ನೇಹಿತನ ಮನೆಯೊಂದರಲ್ಲಿ ಇಳಿದುಕೊಂಡಿದ್ದ. ಆತ ಗೋಪ್ಯವಾಗಿರಬೇಕೆಂದು ಇಚ್ಛಿಸಿದರೂ ಕೆಲವು ಸ್ನೇಹಿತರು ಮನೆಗೆ ಈ ಸುದ್ದಿಯನ್ನು ಕೊಟ್ಟರು. ಪೂರ್ವಾಶ್ರಮದ ಮನೆಯ ದೊಡ್ಡವರೆಲ್ಲ ಈತನಿದ್ದ ಮನೆಗೆ ಬಂದು ಬಲಾತ್ಕಾರವಾಗಿ ತಮ್ಮ ಮನೆಗೆ ಕರೆದುಕೊಂಡು ಹೊರಟುಹೋದರು. ಆ ಮನೆಯಲ್ಲಿ ಈತ ತಪ್ಪಿಸಿಕೊಂಡು ಹೋಗದಂತೆ ಒಂದು ಕೋಣೆಯಲ್ಲಿ ಕೂಡಿದರು. ಕೆಲವು ದಿನಗಳಾದ ಮೇಲೆ ಅವನು ತನ್ನ ಸಂನ್ಯಾಸ ವ್ರತವನ್ನು ತ್ಯಜಿಸುವಂತೆ ಮಾಡಿ ಸಾಧಾರಣ ಗೃಹಸ್ಥನನ್ನಾಗಿ ಮಾಡಬೇಕೆಂದು ಅವರು ಬಯಸಿದರು. ಆದರೆ ದುರ್ಗಾಚರಣನಾದರೊ ಮೂರು ದಿನಗಳ ವರೆಗೆ ಆಹಾರವನ್ನು ಮುಟ್ಟಲಿಲ್ಲ. ಆತ ಕೋಣೆಯೊಳಗೆ ಪ್ರಾಯೋಪವೇಶ ಮಾಡಿ ಬೇಕಾದರೆ ಸಾಯಲು ಸಿದ್ಧನಾಗಿದ್ದನೇ ಹೊರತು ತನ್ನ ಹಿಂದಿನ ಜೀವನಕ್ಕೆ ಬರಲು ಒಪ್ಪಿಕೊಳ್ಳಲಿಲ್ಲ. ಹೀಗೆ ಉಪವಾಸ ಮಾಡುತ್ತಿರುವುದನ್ನು ನೋಡಿ ಈತ ಸತ್ತರೆ ತಮಗೆಲ್ಲ ಆ ಪಾತಕ ಬರುವುದೆಂದು ಕನಿಕರಗೊಂಡು ಕೋಣೆಯ ಬಾಗಿಲನ್ನು ತೆರೆದರು. ತತ್‍ಕ್ಷಣವೇ ಆತ ಅಲ್ಲಿಂದ ಕಣ್ಮರೆಯಾದ. ಅನಂತರ ಆತನ ಸಮಾಚಾರವೇ ಮನೆಯವರಿಗೆ ತಿಳಿಯಲಿಲ್ಲ.

ಮನೆಯವರ ಭವಿಷ್ಯವೆಲ್ಲ ಬೆಳೆಯುತ್ತಿರುವ ಮಗು ವಿಶ್ವನಾಥದತ್ತನ ಮೇಲೆ ನಿಂತಿತು. ಮಗು ಬೆಳೆದಂತೆ ಬುದ್ಧಿವಂತನಾಗುತ್ತ ಬಂದನು. ಮುಂದೆ ವಿದ್ಯಾವಂತನಾಗಿ ತನ್ನ ಕುಲದ ಕಸುಬಾದ ವಕೀಲಿ ವೃತ್ತಿಗೆ ಕೈಹಾಕಿದನು. ಕಲ್ಕತ್ತೆಯ ಕೋರ್ಟಿನಲ್ಲಿ ಅಭ್ಯಾಸ ಮಾಡತೊಡಗಿದನು. ಕ್ರಮೇಣ ಆ ವೃತ್ತಿಯಲ್ಲಿ ಪ್ರಖ್ಯಾತನಾಗುತ್ತ ಬಂದನು. ಆತನದು ಬಹುಮುಖವಾದ ವ್ಯಕ್ತಿತ್ವ. ಬರೀ ಕೋರ್ಟು ಕಛೇರಿ, ಕಕ್ಷಿಗಳು, ಕೇಸುಗಳು ಇವುಗಳಲ್ಲೇ ಮುಳುಗಿ ಹೋದ ವಕೀಲನಲ್ಲ ಅವನು. ಅವನ ಅಭಿರುಚಿ ವಿಶಾಲವಾಗಿತ್ತು,\break ಸುಸಂಸ್ಕೃತವಾಗಿತ್ತು. ಆತ ಇಂಗ್ಲೀಷನ್ನು ಅಭ್ಯಾಸ ಮಾಡಿದವನು. ಅದರಲ್ಲಿರುವ ಸಾಹಿತ್ಯದ ಸವಿಯನ್ನು ಹೀರಿದನು. ಅದರಂತೆಯೇ ಪಾರ್ಸಿ ಭಾಷೆಯನ್ನು ಚೆನ್ನಾಗಿ ಬಲ್ಲವನಾಗಿದ್ದ. ಅದರಲ್ಲಿರುವ ಕಾವ್ಯಗಳು ಮತ್ತು ಸೂಫಿ ಪಂಥಕ್ಕೆ ಸೇರಿದ ವಿಷಯಗಳು ಇವುಗಳನ್ನೆಲ್ಲ ಚೆನ್ನಾಗಿ ತಿಳಿದುಕೊಂಡ. ಸಂಗೀತದಲ್ಲಿಯೂ ಅವನಿಗೆ ಅಭಿರುಚಿ. ಶಾಸ್ತ್ರೀಯ ಸಂಗೀತವನ್ನು ಕೇಳಿ ಆನಂದಿಸಬಲ್ಲವನಾಗಿದ್ದನು.

ವಿಶ್ವನಾಥದತ್ತನ ಮನೆಯಲ್ಲಿ ಬಹುಜನರಿದ್ದರು. ಈತನಿಗೆ ವಕೀಲಿ ವೃತ್ತಿಯಲ್ಲಿ ವರಮಾನ ಹೆಚ್ಚುತ್ತಾ ಹೋದಂತೆಲ್ಲಾ ಖರ್ಚು ಹೆಚ್ಚುತ್ತಾ ಹೋಯಿತು. ದೂರ ದೂರದ ನಿರ್ಗತಿಕರಾದ ನೆಂಟರಿಷ್ಟರೆಲ್ಲ ಇವನ ಮನೆಗೆ ಬರುತ್ತಿದ್ದರು. ಯಾರೇ ಬರಲಿ ಅವರಿಗೆ ಆಶ್ರಯ ಕೊಡುತ್ತಿದ್ದನು. ಜೊತೆಗೆ ದಾನಿ ಎಂದು ಬೇರೆ ಪ್ರಖ್ಯಾತನಾಗಿದ್ದನು. ಭಿಕ್ಷುಕರು ಯಾರು ಬರಲಿ ಅವರನ್ನು ಹಾಗೆಯೇ ಕಳಿಸುತ್ತಿರಲಿಲ್ಲ. ಅವರಿಗೆ ಊಟವನ್ನು ಕೊಟ್ಟು ಅನಂತರ ಬೀಡಿ ಸಿಗರೇಟನ್ನು ಸೇದುವುದಕ್ಕೆ ಕೆಲವು ಪುಡಿಕಾಸುಗಳನ್ನು ಕೂಡ ಕೊಡುತ್ತಿದ್ದನು. ಅನಂತರ ಇವನಿಗೆ ಆದ ಮಗ (ನರೇಂದ್ರನಾಥ), ತಂದೆ ಈ ಶುದ್ಧ ಸೋಮಾರಿಗಳಿಗೆ ಬಿಟ್ಟಿ ಊಟವನ್ನು ಬೇರೆ ಕೊಡುವುದು; ಜೊತೆಗೆ ಅವರ ದುರಭ್ಯಾಸ ಮುಂದುವರಿಯುವುದಕ್ಕೆ ದಕ್ಷಿಣೆಯನ್ನೂ ಕೊಡುತ್ತಿರುವುದನ್ನು ನೋಡಿ ಸಹಿಸಲಾರದೆ ತಂದೆಯನ್ನು ಏತಕ್ಕೆ ಹೀಗೆ ಮಾಡುತ್ತಿರುವೆ ಎಂದು ಕೇಳಿದ. ಅದಕ್ಕೆ ತಂದೆ ಮಗುವಿಗೆ ಕೊಟ್ಟ ಉತ್ತರ ಅವನ ಅನುಕಂಪಕ್ಕೆ ಸಾಕ್ಷಿಯಾಗಿದೆ: “ಮಗು, ಯಾರು ನನ್ನ ಬಳಿಗೆ ಬರುವರೋ ಅವರು ಜೀವನದಲ್ಲಿ ನೆಚ್ಚುಗೆಟ್ಟವರು, ಬೆಂದವರು, ನೊಂದವರು. ಅವರು ಬೀಡಿ ಸಿಗರೇಟನ್ನು ಸೇದುವಾಗ ಈ ಪ್ರಪಂಚದ ದುಃಖವನ್ನು ಮರೆತರೆ ನಾನು ಏತಕ್ಕೆ ಅಡ್ಡ ಬರಲಿ?” ಮತ್ತೊಮ್ಮೆ ಅದೇ ಮಗ “ನನಗೆ ಏನು ಆಸ್ತಿ ಬಿಟ್ಟಿರುವೆ?” ಎಂದು ಕೇಳಿದಾಗ ಹೋಗಿ ಕನ್ನಡಿಯಲ್ಲಿ ನೋಡಿಕೋ ಎಂದಿದ್ದನು. ಒಂದು ಸಲ ನರೇಂದ್ರ ವಿಶ್ವನಾಥದತ್ತನನ್ನು “ಒಳ್ಳೆಯ ನಡತೆಯ ಸಾರ ಏನು?” ಎಂದು ಕೇಳಿದನು. ಅದಕ್ಕೆ ವಿಶ್ವನಾಥದತ್ತ “ಎಂದಿಗೂ ವಿಸ್ಮಯವನ್ನು ವ್ಯಕ್ತಪಡಿಸಬೇಡ. ಮನಸ್ಸಿನಲ್ಲಿ ಅದನ್ನು ಅನುಭವಿಸಬಹುದು. ಆದರೆ ಅದನ್ನು ವ್ಯಕ್ತಪಡಿಸುವುದನ್ನು ಬಿಗಿ ಹಿಡಿಯಬೇಕು. ಇಲ್ಲದೇ ಇದ್ದರೆ ಬಾಹ್ಯ ಘಟನೆಯ ಪ್ರವಾಹಕ್ಕೆ ಸಿಕ್ಕಿ ಕೊಚ್ಚಿ ಹೋಗುವೆ” ಎಂದನು.

ವಿಶ್ವನಾಥದತ್ತನಿಗೆ ಪ್ರಾಪ್ತ ವಯಸ್ಸಿನಲ್ಲಿ ಭುವನೇಶ್ವರೀದೇವಿ ಎಂಬ ಹುಡುಗಿಯೊಡನೆ ಮದುವೆ ಮಾಡಿದರು. ಎಲ್ಲಾ ವಿಧದಲ್ಲಿಯೂ ಕೈಹಿಡಿದ ಗಂಡನಿಗೆ ತಕ್ಕ ಸತಿ ಆಕೆ. ಅವಳಲ್ಲಿ ಎಲ್ಲಕ್ಕಿಂತ ಮೇಲೆದ್ದು ಕಾಣುತ್ತಿದ್ದ ಗುಣವೇ ದೈವಭಕ್ತಿ. ದೇವರು ಎಂಬ ಕಂಬವನ್ನು ಹಿಡಿದುಕೊಂಡು ಜೀವನದಲ್ಲಿ ತನ್ನ ಪಾಲಿಗೆ ಬಂದ ಕರ್ತವ್ಯವನ್ನು ಮಾಡುತ್ತಿದ್ದಳು. ಜೀವನದಲ್ಲಿ ಏನಾದರೂ ಆಗಲಿ ಪ್ರಾರ್ಥನೆಯನ್ನು ಒಂದು ದಿನವೂ ಬಿಟ್ಟವಳಲ್ಲ. ಆಕೆ ಅಷ್ಟೊಂದು ದೊಡ್ಡ ಪಂಡಿತೆಯರ ಗುಂಪಿಗೆ ಸೇರಿದ ಸ್ತ್ರೀ ಅಲ್ಲದೇ ಇದ್ದರೂ ರಾಮಾಯಣ, ಮಹಾಭಾರತ, ಚೈತನ್ಯ ಚರಿತಾಮೃತ ಮುಂತಾದುವನ್ನು ಓದಿ ಅರ್ಥ ಮಾಡಿಕೊಳ್ಳಬಲ್ಲವಳಾಗಿದ್ದಳು. ಆಕೆಗೆ ಅದ್ಭುತವಾದ ಜ್ಞಾಪಕಶಕ್ತಿ ಇತ್ತು. ಒಂದು ವೇಳೆ ಓದಿದುದನ್ನು ಮತ್ತು ಕೇಳಿದುದನ್ನು ಎಂದಿಗೂ ಮರೆಯುತ್ತಿರಲಿಲ್ಲ. ರಾಮಾಯಣ ಮಹಾಭಾರತದ ಕಥೆಗಳನ್ನು ಹೃದಯಂಗಮವಾಗಿ ಹೇಳುತ್ತಿದ್ದಳು. ಅವಳ ಮಗ ಸ್ವಾಮಿ ವಿವೇಕಾನಂದರು ಅಮೇರಿಕಾ ದೇಶದಲ್ಲಿ ಹೇಳುತ್ತಿದ್ದ ರಾಮಾಯಣ ಮಹಾಭಾರತಗಳ ಕಥೆಗಳನ್ನು ಕೇಳಿ ಜನ ಮುಗ್ಧರಾಗಿ ಹೋಗುತ್ತಿದ್ದರು. ಅದನ್ನು ಕೇಳಿಯಾದ ಮೇಲೆ ಅವರು “ಸ್ವಾಮೀಜೀ, ನೀವು ಅದ್ಭುತ ಕತೆಗಾರರು. ಈ ಕಲೆಯನ್ನು ನೀವು ಯಾರಿಂದ ಕಲಿತಿರಿ?” ಎಂದು ಕೇಳಿದರು. ಅದಕ್ಕೆ ಸ್ವಾಮೀಜಿ “ನನ್ನ ತಾಯಿಯಿಂದ” ಎಂದು ಹೇಳುತ್ತಿದ್ದರು. ಭುವನೇಶ್ವರೀದೇವಿ ತನ್ನ ಮಕ್ಕಳ ಬೆಳವಣಿಗೆಗೆ ಎದೆಹಾಲನ್ನು ಮಾತ್ರ ಕೊಟ್ಟವಳಲ್ಲ. ಅವರ ಮಾನಸಿಕ ಬೆಳವಣಿಗೆಗೆ ಬೇಕಾಗುವ ಭಾವನಾಮೃತವನ್ನೂ ಧಾರೆ ಎರೆದಳು. ತನಗೆ ವಿರಾಮವಾದಾಗಲೆಲ್ಲ ಮಕ್ಕಳನ್ನು ತನ್ನ ಸುತ್ತಲೂ ಕುಳ್ಳಿರಿಸಿಕೊಂಡು ರಾಮಾಯಣ ಮಹಾಭಾರತದಲ್ಲಿ ಬರುವ ಮಹಾ ವ್ಯಕ್ತಿಗಳ ಕುರಿತು ಮಾತನಾಡುತ್ತಿದ್ದಳು. ಕೆಲವು ವೇಳೆ ಕರ್ಣನ ತ್ಯಾಗವಾಗಿರಬಹುದು, ಅರ್ಜುನನ ಪರಾಕ್ರಮವಾಗಿರಬಹುದು, ಕುಂತಿಯ ತಾಯೊಲುಮೆ ಆಗಬಹುದು. ರಾಮಾಯಣದಲ್ಲಿ ಬರುವ ಸೀತಾ ರಾಮ ಹನುಮಂತರ ಶೀಲವನ್ನು ಅಷ್ಟು ಮನಮುಟ್ಟುವಂತೆ ಮಕ್ಕಳಿಗೆ ಹೇಳುತ್ತಿದ್ದಳು. ಸ್ವಾಮಿ ವಿವೇಕಾನಂದರಿಗೆ ನಮ್ಮ ಸಂಸ್ಕೃತಿಯ ಪ್ರಥಮ ಪರಿಚಯವಾದದ್ದು ತಾಯಿಯ ಮೂಲಕ. ಇಂತಹ ವಾತಾವರಣದಲ್ಲಿ ಸ್ವಾಮಿ ವಿವೇಕಾನಂದರ ವ್ಯಕ್ತಿತ್ವ ವಿಕಾಸವಾಯಿತು. ಪ್ರಖ್ಯಾತ ವಕೀಲನಾದ ವಿಶ್ವನಾಥದತ್ತ ತನ್ನ ಮಗುವಿನ ಬೆಳವಣಿಗೆಗೆ ಏನನ್ನು ಕೊಟ್ಟನೋ, ಅದಕ್ಕಿಂತ ತಾಯಿ ಕೊಟ್ಟದ್ದು ಕಡಿಮೆಯಾಗಿರಲಿಲ್ಲ. ವಿವೇಕಾನಂದರು ತಮ್ಮ ತಾಯಿಯನ್ನು ತಮ್ಮ ಜೀವಿತದ ಅಂತ್ಯದವರೆವಿಗೂ ಪೂಜ್ಯದೃಷ್ಟಿಯಿಂದ ನೋಡುತ್ತಿದ್ದರು. ಒಂದು ಸಲ ಅಮೇರಿಕಾದಲ್ಲಿ ಅವರು ಭರತಖಂಡದ ಮಹಿಳೆಯರು ಎಂಬ ವಿಷಯದ ಮೇಲೆ ಮಾತನಾಡಿದರು. ಅದನ್ನು ಕೇಳಿದ ಅನೇಕ ಮಹಿಳೆಯರು ಭರತಖಂಡದಲ್ಲಿದ್ದ ಅವರ ತಾಯಿಗೆ ಒಂದು ಪತ್ರವನ್ನು ಬರೆದು, ಅಂತಹ ಪುತ್ರನನ್ನು ಜಗತ್ತಿಗೆ ನೀಡುವ ಭಾಗ್ಯ ನಿಮ್ಮದಾಯಿತಲ್ಲ ಎಂದು ಆಕೆಯನ್ನು ಶ್ಲಾಘಿಸಿದರು. ಇಡೀ ಸ್ತ್ರೀಕುಲವನ್ನೇ ಪೂಜ್ಯದೃಷ್ಟಿಯಿಂದ ನೋಡುವುದನ್ನು ಸ್ವಾಮಿ ವಿವೇಕಾನಂದರು ಕಲಿತರು. ಇದರ ಹಿಂದೆ ಆ ಭಾವನೆಗೆ ಪೀಠಿಕೆ ಹಾಕಿದವಳು ಅವರ ತಾಯಿ. ಸ್ವಾಮಿ ವಿವೇಕಾನಂದರು ಅನಂತರ ‘ತನ್ನ ತಾಯಿಯನ್ನು ಗೌರವಿಸದ ಯಾರೂ ಮಹಾತ್ಮನಾಗಿಲ್ಲ’ ಎಂದು ಹೇಳಿದರು.

ಭುವನೇಶ್ವರೀದೇವಿ ಮನೆಗೆ ಸಂಬಂಧಪಟ್ಟ ಕೆಲಸಗಳನ್ನೆಲ್ಲ ಮಾಡುತ್ತಿದ್ದಳು. ಆ ದೊಡ್ಡ ಸಂಸಾರದ ಜವಾಬ್ದಾರಿಯನ್ನು ಯಶಸ್ವಿಯಾಗಿ ನೆರವೇರಿಸುವುದೊಂದು ಸಣ್ಣ ಕೆಲಸವಲ್ಲ. ಆದರೆ ಅವಳ ಜೀವನ ಪರಿಧಿ ತನ್ನ ಮನೆಯಲ್ಲಿಯೇ ಪೂರೈಸಲಿಲ್ಲ. ಸುತ್ತಮುತ್ತಲಿರುವ ಹಲವು ಸಂಸಾರಗಳ ಸುಖದುಃಖದಲ್ಲಿ ಭಾಗಿಯಾಗುತ್ತಿದ್ದಳು. ಕಷ್ಟದಲ್ಲಿ ಬಿದ್ದವರಿಗೆ ಸಹಾಯ ಮಾಡಲು ಸದಾ ಸಿದ್ಧಳಾಗಿದ್ದಳು. ಅವಳ ನಡೆನುಡಿಯಲ್ಲಿ ಗಾಂಭೀರ್ಯ ವ್ಯಕ್ತವಾಗುತ್ತಿತ್ತು. ಸಣ್ಣತನವಿರಲಿಲ್ಲ, ಅವಳ ಮಾತಿನಲ್ಲಾಗಲೀ ಕೆಲಸದಲ್ಲಾಗಲಿ. ಸಿಂಹಿಣಿಯಂತಿದ್ದ ಅವಳ ನಡಿಗೆಯೆ ಅವಳ ವ್ಯಕ್ತಿತ್ವಕ್ಕೆ ಉದಾಹರಣೆಯಾಗಿತ್ತು. ಸ್ವಾಮಿ ವಿವೇಕಾನಂದರು ಮತ್ತು ಅವರ ತಾಯಿಯನ್ನು ನೋಡಿದವರು, ಸ್ವಾಮೀಜಿ ಅವರು ತಮ್ಮ ತಾಯಿಯಿಂದ ನಡಿಗೆಯ\break ಗಾಂಭೀರ‍್ಯವನ್ನು ಪಡೆದರು ಎಂದು ಹೇಳುವರು. ನಮ್ಮ ಪುರಾಣಗಳು ಕಾವ್ಯಗಳು\break ಸ್ತ್ರೀರೂಪವನ್ನು ವರ್ಣಿಸುವಾಗ ಮತ್ತು ಅವರ ನಡಿಗೆಯ ಗಾಂಭೀರ‍್ಯವನ್ನು ವಿವರಿಸುವಾಗ ಮಂದಗಮನೆ, ಗಜಗಮನೆ, ಹಂಸಗತಿ ಮುಂತಾದ ಗುಣವಾಚಕಗಳನ್ನು ಉಪಯೋಗಿಸುವರು. ಇದೊಂದು ಸೌಂದರ‍್ಯ. ನಮ್ಮ ವ್ಯಕ್ತಿತ್ವವೆಲ್ಲ ಮೌನವಾಗಿ ಮಾತನಾಡುವ ಸೌಂದರ‍್ಯ.

ವಿಶ್ವನಾಥದತ್ತ ಮತ್ತು ಭುವನೇಶ್ವರೀದೇವಿ ಎಂಬ ದಂಪತಿಗಳು ಬುದ್ಧಿ ಮತ್ತು ಹೃದಯಕ್ಕೆ ಸೇರಿದ ಗುಣಗಳಲ್ಲಿ ಸಾಧಾರಣ ಮಾನವರಿಗಿಂತ ಮೇಲೆದ್ದು ನಿಂತಿರುವುದನ್ನು ನೋಡುವೆವು. ಆಕಾಶದಲ್ಲಿ ಸಂಚರಿಸುತ್ತಿರುವ ಮಿಂಚಿನ ಗೊಂಚಲು ಕೆಳಗೆ ಇಳಿಯಬೇಕಾದರೆ ತನಗೆ ಸಮೀಪದಲ್ಲಿರುವ ಉನ್ನತವಾದ ವಸ್ತುವನ್ನು ಆರಿಸುತ್ತದೆ. ಅದನ್ನು ಆಶ್ರಯಿಸಿ ಧರೆಗೆ ಇಳಿದುಬರುವುದು.

