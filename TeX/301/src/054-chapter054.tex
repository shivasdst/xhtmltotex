
\chapter{ಸ್ವಾಮೀಜಿ ಒಬ್ಬ ಕಲಾವಿದರೊಡನೆ}

 ಇಂದು ಶಿಷ್ಯ ಜ್ಯೂಬಿಲಿ ಕಲಾಶಾಲೆಯ ಸ್ಥಾಪಕ ಮತ್ತು ಅಧ್ಯಾಪಕನಾದ ರನದದಾಸಗುಪ್ತರೊಡನೆ ಮಠಕ್ಕೆ ಬಂದಿದ್ದ. ರನದಬಾಬುಗಳು ಪ್ರವೀಣ ಕಲಾವಿದರು, ಘನ ವಿದ್ವಾಂಸರು, ಸ್ವಾಮಿಗಳನ್ನು ಮೆಚ್ಚಿದವರು. ಕುಶಲ ಪ್ರಶ್ನೆಯಾದನಂತರ ಸ್ವಾಮಿಗಳು ರನದಬಾಬುಗಳೊಡನೆ ಕಲೆಯ ಮೇಲೆ ಅನೇಕ ಸಂಗತಿಗಳನ್ನು ಮಾತನಾಡಲಾರಂಭಿಸಿದರು. 

 ಸ್ವಾಮೀಜಿ: “ಪ್ರಪಂಚದಲ್ಲಿರುವ ಎಲ್ಲಾ ನಾಗರಿಕ ದೇಶಗಳ ಕಲಾ ಸೌಂದರ‍್ಯಗಳನ್ನು ನೋಡುವ ಸುಯೋಗ ನನಗೆ ದೊರಕಿತು. ಆದರೆ ನಮ್ಮ ದೇಶದಲ್ಲಿ ಬೌದ್ಧರ ಕಾಲದಲ್ಲಿ ಕಲೆಯ ವಿಕಾಸವಾದಷ್ಟು ಮತ್ತೆಲ್ಲಿಯೂ ಕಾಣಬರಲಿಲ್ಲ. ಮೊಗಲರ ಕಾಲದಲ್ಲಿಯೂ ಕಲೆಯ ಬೆಳವಣಿಗೆ ತಕ್ಕಷ್ಟು ಇತ್ತು. ತಾಜ್, ಜುಮ್ಮಾ ಮಸೀದಿ ಮುಂತಾದವು ಆ ಸಂಸ್ಕೃತಿಯ ಚಿರಸ್ಮರಣೀಯ ಸ್ಮಾರಕಗಳು.” 

 “ಮಾನವನು ಯಾವುದನ್ನೇ ನಿರ್ಮಿಸಲಿ, ಆ ಕಲೆಯ ಮೂಲಕ ಯಾವುದೋ ಒಂದು ಭಾವವನ್ನು ಸ್ಪಷ್ಟಪಡಿಸುವುದಾಗಿದೆ. ಎಲ್ಲಿ ಆ ಉದ್ದೇಶದ ಅಭಾವವಿದೆಯೋ ಅಲ್ಲಿ ಎಷ್ಟೇ ಬಣ್ಣ ಮುಂತಾದವುಗಳ ಪ್ರದರ್ಶನವಿರಲಿ ಅದನ್ನು ಸತ್ಯವಾದ ಕಲೆ ಎನ್ನಲಾಗುವುದಿಲ್ಲ. ನಮ್ಮ ನಿತ್ಯಜೀವನದಲ್ಲಿ ಉಪಯೋಗಿಸುವ ನೀರಿನ ಚೊಂಬು ತಟ್ಟೆ, ಬಟ್ಟಲುಗಳೂ ಕೂಡ ಒಂದು ಭಾವದ ಪ್ರತಿಬಿಂಬವಾಗಿರಬೇಕು. ಪ್ಯಾರಿಸ್ಸಿನ ವಸ್ತು ಸಂಗ್ರಹಶಾಲೆಯಲ್ಲಿ ಅಮೃತಶಿಲೆಯಲ್ಲಿ ಕೆತ್ತಿದ ಒಂದು ಸುಂದರ ವಿಗ್ರಹವನ್ನು ನೋಡಿದೆ. ಆ ವಿಗ್ರಹದ ಕೆಳಗಡೆ ಈ ಮಾತುಗಳು ಕೆತ್ತಲ್ಪಟ್ಟಿದ್ದುವು: ‘ಕಲೆ ಪ್ರಕೃತಿಯನ್ನು ಅನಾವರಣ ಮಾಡುವುದು’. ಅಂದರೆ ಕಲೆ ಪ್ರಕೃತಿಯ ಆಂತರಿಕ ಸೌಂದರ‍್ಯದ ಮೇಲೆ ಮುಸುಕಿರುವ ತೆರೆಯನ್ನು ದೂರ ಸರಿಸಿ ತೋರುವುದು ಎಂದು. ಆ ಕೆಲಸ ಯಾವ ಭಾವನೆಯನ್ನು ವ್ಯಕ್ತಪಡಿಸುತ್ತಿತ್ತು ಎಂದರೆ ಪ್ರಕೃತಿಯ ಸೌಂದರ‍್ಯದ ಮೇಲೆ ಮುಸುಕಿರುವ ತೆರೆಯನ್ನು ಕಲಾವಿದ ತನ್ನ ಕೈಗಳಿಂದ ದೂರ ಸರಿಸಿ ತೋರುವನು ಎಂದು. ಈ ಮನೋಹರ ಭಾವನೆಯನ್ನು ವ್ಯಕ್ತಪಡಿಸಲು ಹೊರಟಿದ್ದ ಆ ಶಿಲ್ಪಿಯನ್ನು ಯಾರೂ ಹೊಗಳದಿರಲಾರರು. ನೀನು ಹಾಗೆಯೇ ಏನಾದರೂ ಕಲ್ಪನೆಯಿಂದ ಮಾಡಬೇಕು.” 

 ರನದಬಾಬು: “ನನಗೂ ಹಾಗೆಯೇ ಸ್ವತಂತ್ರವಾಗಿ ವಿರಾಮವಾಗಿರುವಾಗ ರಚಿಸಬೇಕೆಂದು ಆಸೆಯಿದೆ. ಆದರೆ ಅದಕ್ಕೆ ಈ ದೇಶದಲ್ಲಿ ಸ್ವಲ್ಪವೂ ಪ್ರೋತ್ಸಾಹವಿಲ್ಲ. ಇದು ಬಡದೇಶ- ಪ್ರಶಂಸೆಯೇ ವಿರಳ.” 

 ಸ್ವಾಮೀಜಿ: “ನೀನು ಹೃದಯವನ್ನೆಲ್ಲಾ ಧಾರೆ ಎರೆದು ಒಂದು ಸ್ವತಂತ್ರ ಕೃತಿಯನ್ನು ರಚಿಸಬಲ್ಲೆಯಾದರೆ, ಕಲೆಯ ಒಂದು ಭಾಗಕ್ಕೆ ಪೂರ್ತಿ ಕಳೆಕೊಡಬಲ್ಲೆಯಾದರೆ, ಒಂದಲ್ಲ ಒಂದು ದಿನ ಅದು ಜನರ ಮೆಚ್ಚಿಗೆ ಪಡೆಯುವುದು. ಈ ಜಗತ್ತಿನಲ್ಲಿ ಸತ್ಯವಾದ ಯಾವುದೇ ಆಗಲಿ ಪ್ರಕಾಶಕ್ಕೆ ಬಂದೇ ತೀರುವುದು. ಎಷ್ಟೋ ಮಂದಿ ಕಲಾವಿದರ ಕೃತಿಗಳು ಅವರು ಸತ್ತ ಸಾವಿರಾರು ವರ್ಷಗಳಾದಮೇಲೆ ಬೆಳಕಿಗೆ ಬಂದುವೆಂದು ಕೇಳಿದ್ದೇವೆ.” 

 ರನದಬಾಬು: “ಅದು ನಿಜ. ಆದರೆ ನಾವು ನಿಷ್ಪ್ರಯೋಜಕರಾಗಿದ್ದೇವೆ. ನಿಷ್ಪ್ರಯೋಜನವಾಗಿ ಅಷ್ಟೊಂದು ಶಕ್ತಿ ವ್ಯಯಮಾಡಲು ಧೈರ‍್ಯವಿಲ್ಲದವರಾಗಿದ್ದೇವೆ. ಐದುವರ್ಷದ ಪ್ರಯತ್ನದಿಂದ ನಾನು ಸ್ವಲ್ಪ ಜಯಶೀಲನಾಗಿರುವೆನು. ನನ್ನ ಪರಿಶ್ರಮ ಸಾರ್ಥಕವಾಗುವಂತೇ ಹರಸಿ.” 

 ಸ್ವಾಮೀಜಿ: “ನೀನು ಉತ್ಸಾಹಪೂರಿತವಾಗಿ ಕೆಲಸಮಾಡಿದರೆ ಖಂಡಿತ ಜಯಹೊಂದುವೆ. ಯಾರು ತ್ರಿಕರಣಪೂರ್ವಕ ಕೆಲಸಮಾಡುವರೋ ಅವರು ಜಯವೊಂದನ್ನೇ ಅಲ್ಲ, ಅದರಲ್ಲಿರುವ ಏಕಾಗ್ರತೆಯಿಂದ ಪರಮಾತ್ಮನನ್ನು ಕೂಡ ಸಾಕ್ಷಾತ್ಕರಿಸಿಕೊಳ್ಳಬಲ್ಲರು. ಯಾರು ಹೃತ್ಪೂರ್ವಕವಾಗಿ ತಮ್ಮ ಕೆಲಸ,ಮಾಡುವರೋ ಅವರಿಗೆ ದೇವರು ಸಹಾಯಮಾಡುವರು.” 

 ಶಿಷ್ಯ: “ಹಿಂದೂದೇಶದ ಮತ್ತು ಪಾಶ್ಚಾತ್ಯದೇಶಗಳ ಕಲೆಗಳಲ್ಲಿ ನೀವು ಯಾವ ವ್ಯತ್ಯಾಸ ನೋಡಿದಿರಿ?” 

 ಸ್ವಾಮೀಜಿ: “ಎಲ್ಲಾ ಕಡೆಯೂ ಅದು ಸುಮಾರಾಗಿ ಒಂದೇ ತರಹ ಇದೆ. ಸ್ವತಂತ್ರ ಕೃತಿಗಳು ಬಹು ವಿರಳ. ಆ ದೇಶದಲ್ಲಿ ಬಗೆಬಗೆಯ ವಸ್ತುಗಳ ಛಾಯಾಚಿತ್ರಗಳ ಪ್ರಕೃತಿಯನ್ನು ಮುಂದಿಟ್ಟುಕೊಂಡು ಚಿತ್ರಗಳನ್ನು ಬರೆಯುವರು. ಯಾವಾಗ ಯಂತ್ರದ ಸಹಾಯವನ್ನು ತೆಗೆದುಕೊಳ್ಳುವರೋ ಆಗ ಕಲ್ಪನಾಶಕ್ತಿಯೇ ಮಾಯವಾಗುತ್ತದೆ. ನಮ್ಮ ಭಾವ ವ್ಯಕ್ತವಾಗಲು ಅವಕಾಶವೇ ಇರುವುದಿಲ್ಲ. ಪ್ರಾಚೀನಕಾಲದ ಕಲಾವಿದರು ಸ್ವತಂತ್ರ ಭಾವನೆಗಳನ್ನು ಚಿತ್ರದ ಮೂಲಕ ವ್ಯಕ್ತಪಡಿಸುತ್ತಿದ್ದರು. ಈಗ ವರ್ಣಚಿತ್ರ ಎಲ್ಲಾ ಕಡಿಮೆಯಾಗುತ್ತಿದೆ. ಆದರೆ ಪ್ರತಿಯೊಂದು ಜನಾಂಗವೂ ಕೇವಲ ತನ್ನದೇ ಆದ ವೈಶಿಷ್ಟ್ಯವನ್ನು ಹೊಂದಿದೆ. ಅದರ ನಡೆ, ನುಡಿ ಜೀವನದ ರೀತಿ ಈ ತೈಲಚಿತ್ರ ಮತ್ತು ಶಿಲ್ಪಕಲೆಗಳಲ್ಲಿ ತನ್ನದೇ ಆದೊಂದು ವೈಶಿಷ್ಟ್ಯದೊಂದಿಗೆ ವ್ಯಕ್ತಪಡುತ್ತದೆ. ಉದಾಹರಣೆಗೆ ಪಾಶ್ಚಾತ್ಯದೇಶದಲ್ಲಿ ಸಂಗೀತ ಮತ್ತು ನಾಟ್ಯಕಲೆಯ ಭಾವಗಳು ಬಹುತೀಕ್ಷ್ಣವಾಗಿರುತ್ತವೆ. ನಾಟ್ಯದಲ್ಲಿ ಅವರ ಅಂಗಾಂಗಗಳು ಮುರಿಯುವಂತೆ ಇರುತ್ತವೆ. ವಾದ್ಯ ಸಂಗೀತದಲ್ಲಿ ಅವರ ಧ್ವನಿ ಭರ್ಜಿಯಂತೆ ನಮ್ಮ ಕಿವಿಯನ್ನಿರಿಯುತ್ತದೆ; ಹಾಗೇ ಹಾಡುಗಾರಿಕೆಯೂ ಕೂಡ. ಅದಕ್ಕೆ ವಿರುದ್ಧವಾಗಿ ನಮ್ಮ ದೇಶದಲ್ಲಿ ನಾಟ್ಯಕಲೆ ಅಲೆಅಲೆಯಾಗಿ ತೇಲುತ್ತ ಹೊರಳಿ ಬಂದಂತೆ ಇರುವುದು. ಹಾಗೆಯೇ ಸಂಗೀತದ ವಿವಿಧ ಸ್ವರಗಳಲ್ಲಿ. ಕಲೆಯ ವಿಚಾರದಲ್ಲಿಯೂ ವಿವಿಧ ಜನರಲ್ಲಿ ವಿವಿಧ ಭಾವಗಳು ಸ್ಫುಟಗೊಳ್ಳುವುವು. ಯಾರು ತೀವ್ರ ದೇಹಾತ್ಮವಾದಿಗಳೋ ಅವರು ಪ್ರಕೃತಿಯನ್ನು ಗುರಿಯಾಗಿಟ್ಟುಕೊಂಡು ಅದಕ್ಕೆ ಸಂಬಂಧಪಟ್ಟಂತೆ ಕಲೆಯ ಭಾವವನ್ನು ವ್ಯಕ್ತಪಡಿಸುವರು. ಯಾರು ಪ್ರಕೃತಿಯನ್ನು ಮೀರಿದ ಅತೀಂದ್ರಿಯ ಸತ್ಯವನ್ನು ತಮ್ಮ ಗುರಿಯಾಗಿಟ್ಟುಕೊಂಡಿರುವರೋ ಅವರು ಪ್ರಕೃತಿ ಶಕ್ತಿಯ ಮೂಲಕ ಕಲೆಯಲ್ಲಿ ತಮ್ಮ ಭಾವವನ್ನು ವ್ಯಕ್ತಪಡಿಸುವರು. ಮೊದಲಿನ ವರ್ಗದ ಜನರಿಗೆ ಪ್ರಕೃತಿಯೇ ಕಲೆಯ ಮೂಲಾಧಾರ. ಎರಡನೆ ವರ್ಗದವರಿಗೆ ಭಾವನೆಯೇ ಕಲೆಯ ಬೆಳವಣಿಗೆಗೆ ಮುಖ್ಯ ಉದ್ದೇಶ. ಆದ್ದರಿಂದ ಕಲೆಯಲ್ಲಿ ಎರಡೂ ತಮ್ಮದೇ ಆದ ಹಾದಿಯಲ್ಲಿ ಮುಂದುವರಿದಿವೆ. ಪಾಶ್ಚಾತ್ಯರ ಕೆಲವು ತೈಲಚಿತ್ರಗಳನ್ನು ನೋಡಿದರೆ ಅವು ಸಂಪೂರ್ಣ ಪ್ರಕೃತಿಯ ವಸ್ತುಗಳೆಂದು ಭ್ರಾಂತಿ ಪಡುವಂತಾಗುವುದು. ಈ ದೇಶದಲ್ಲಿಯೂ ನಮ್ಮ ಪೂರ್ವಿಕರು ಶಿಲ್ಪಕಲೆಯಲ್ಲಿ ಉಚ್ಚ ಶಿಖರವನ್ನೇರಿದ್ದರು. ಆಗಿನ ಕಾಲದ ಯಾವುದಾದರೊಂದು ವಿಗ್ರಹವನ್ನು ನೋಡಿದರೆ ಅದು ನಮ್ಮನ್ನು ಈ ಭೌತಿಕ ಪ್ರಪಂಚವನ್ನು ಮರೆಯುವಂತೆ ಮಾಡಿ ಒಂದು ಹೊಸ ಆದರ್ಶ ಪ್ರಪಂಚಕ್ಕೆ ಒಯ್ಯುವುದು. ಈಗ ಪಶ್ಚಿಮ ದೇಶಗಳಲ್ಲಿ ಹಿಂದಿನ ಕಾಲದ ತೈಲಚಿತ್ರಗಳನ್ನು ಚಿತ್ರಿಸಲಾಗುತ್ತಿಲ್ಲ. ಅದರಂತೆ ಈ ದೇಶದಲ್ಲಿಯೂ ತಮ್ಮದೇ ಆದ ಸ್ವತಂತ್ರ ಭಾವಗಳನ್ನು ವ್ಯಕ್ತಪಡಿಸುವುದೂ ಕಂಡು ಬರುವುದಿಲ್ಲ. ಉದಾಹರಣೆಗೆ ನಿಮ್ಮ ಕಲಾಶಾಲೆಯ ತೈಲಚಿತ್ರಗಳು ಭಾವಶೂನ್ಯ. ಅದಕ್ಕೆ ಬದಲು ನೀವು ಹಿಂದೂಗಳು ಉಪಯೋಗಿಸುವ ನಿತ್ಯಜೀವನದ ವಸ್ತುಗಳನ್ನು ಅದಕ್ಕೆ ಪೂರ್ವಿಕರ ಆದರ್ಶವನ್ನೆರೆದು ಚಿತ್ರಿಸಲು ಪ್ರಯತ್ನಪಡಿ.” 

 ರನದಬಾಬು: “ತಮ್ಮ ಮಾತುಗಳಿಂದ ನನಗೆ ಬಹಳ ಉತ್ತೇಜನ ದೊರಕಿದಂತಾಗಿದೆ. ನಿಮ್ಮ ಸಲಹೆಯಂತೆ ನಡೆಯಲು ಪ್ರಯತ್ನಪಡುತ್ತೇನೆ.” 

 ಸ್ವಾಮೀಜಿ: “ಉದಾಹರಣೆಗೆ ಕಾಳಿಕಾಮಾತೆಯ ಮೂರ್ತಿಯನ್ನೇ ತೆಗೆದುಕೊಳ್ಳಿ. ಅದರಲ್ಲಿ ಆನಂದ ಹಾಗೂ ರೌದ್ರಭಾವಗಳ ಮಿಲನವಿದೆ. ಆದರೆ ಯಾವ ತೈಲಚಿತ್ರದಲ್ಲೂ ಈ ಭಾವಗಳ ಸ್ಪಷ್ಟ ಪ್ರದರ್ಶನವಿಲ್ಲ. ಅದೊಂದೇ ಅಲ್ಲ, ಯಾವ ಒಂದು ಭಾವವನ್ನೂ ಸರಿಯಾಗಿ ನಿರೂಪಿಸುವುದಿಲ್ಲ. ನಾನು ನನ್ನ “ಕಾಳಿಕಾಮಾತೆ” ಎಂಬ ಆಂಗ್ಲಪದ್ಯದಲ್ಲಿ ಕಾಳಿಕಾಮಾತೆಯ ಕೆಲವು ಉಗ್ರಭಾವವನ್ನು ಕೊಡಲು ಯತ್ನಿಸಿರುವೆ. ನೀನು ಈ ಭಾವನೆಗಳನ್ನು ಒಂದು ಚಿತ್ರದಲ್ಲಿ ವ್ಯಕ್ತಪಡಿಸಬಲ್ಲೆಯಾ?” 

 ರನದಬಾಬು: “ದಯವಿಟ್ಟು ನನಗೆ ಅದನ್ನು ತಿಳಿಸಿ.” 

 ಸ್ವಾಮೀಜಿ: ಪುಸ್ತಕ ಸಂಗ್ರಹಾಲಯದಿಂದ ಆ ಪದ್ಯವನ್ನು ತಂದು ರನದಬಾಬುಗಳ ಮನಸ್ಸಿಗೆ ನಾಟುವಂತೆ ಓದತೊಡಗಿದರು. ರನದಬಾಬು ಮೌನವಾಗಿ ಅದನ್ನು ಕೇಳಿದರು. ಸ್ವಲ್ಪ ಕಾಲದ ನಂತರ ಆ ರೂಪವನ್ನು ತಮ್ಮ ಮಾನಸಚಕ್ಷುಗಳಿಂದ ಪ್ರತ್ಯೇಕಿಸಿಕೊಂಡರೊ ಎಂಬಂತೇ ಅವರು ಸ್ವಾಮಿಗಳ ಕಡೆ ಭಯಚಕಿತ ಕಣ್ಣುಗಳಿಂದ ನೋಡಿದರು. 

 ಸ್ವಾಮೀಜಿ: “ನೀನೀ ಭಾವವನ್ನು ಚಿತ್ರದಲ್ಲಿ ವ್ಯಕ್ತಪಡಿಸಲು ಸಾಧ್ಯವೆ?” 

 ರನದಬಾಬು: “ಆಗಲಿ ನಾನು ಪ್ರಯತ್ನಿಸುತ್ತೇನೆ. ಆ ಭಾವ ಕೇವಲ ಕಲ್ಪಿಸಿಕೊಂಡ ಮಾತ್ರದಿಂದಲೇ ತಲೆ ತಿರುಗುವಂತೆ ಮಾಡುತ್ತಿದೆ.” 

 ಸ್ವಾಮೀಜಿ: “ಚಿತ್ರ ಪೂರೈಸಿದ ಮೇಲೆ ದಯವಿಟ್ಟು ನನಗೆ ತೋರಿಸು. ಅದನ್ನು ಪೂರ್ತಿಗೊಳಿಸಲು ಬೇಕಾಗುವ ಹಲವು ಸಲಹೆಗಳನ್ನು ನಾನು ಕೊಡುವೆ.” 

 ಅನಂತರ ಸ್ವಾಮೀಜಿ, ತಾವು ಗುರುತು ಹಾಕಿಟ್ಟಿದ್ದ ಶ‍್ರೀರಾಮಕೃಷ್ಣ ಸಂಸ್ಥೆಯ ಚಿಹ್ನೆಯ ನಕಾಶೆಯನ್ನು ತರಿಸಿ ರನದಬಾಬುಗಳಿಗೆ ತೋರಿಸಿ ಅವರ ಅಭಿಪ್ರಾಯವನ್ನು ಕೇಳಿದರು. ಅದರಲ್ಲಿ ಅರಳಿರುವ ತಾವರೆಯೂ ಸರೋವರವೂ, ಒಂದು ಹಂಸವೂ, ಅದರ ಸುತ್ತ ಬಳಸಿ ಹೊರಗೆ ಒಂದು ಸರ್ಪವೂ ಚಿತ್ರಿತವಾಗಿದ್ದುವು. ರನದಬಾಬುವಿಗೆ ಮೊದಲು ಅದರ ಭಾವನೆ ಅರ್ಥವಾಗಲಿಲ್ಲ. ಸ್ವಾಮಿಗಳನ್ನು ವಿವರಿಸಲು ಕೇಳಿದರು. ಸ್ವಾಮಿಗಳು ಹೇಳಿದರು: “ಸರೋವರದ ತರಂಗಮಾಲೆಗಳು ಕರ್ಮವನ್ನೂ, ತಾವರೆ ಭಕ್ತಿಯನ್ನೂ, ಮೂಡುತ್ತಿರುವ ಸೂರ್ಯ ಜ್ಞಾನವನ್ನೂ ಸೂಚಿಸುತ್ತವೆ. ಸುತ್ತಲೂ ಬಳಸಿರುವ ಸರ್ಪ ಜಾಗ್ರತವಾದ ಕುಂಡಲಿನಿ ಶಕ್ತಿಯನ್ನೂ, ಹಂಸ ಪರಮಾತ್ಮನನ್ನೂ ಸೂಚಿಸುವುದು. ಯಾವಾಗ ಯೋಗ ಜ್ಞಾನ ಭಕ್ತಿ ಕರ್ಮಗಳೊಡನೆ ಐಕ್ಯವಾಗುವುದೋ ಆಗ ಪರಮಾತ್ಮ ಸಾಕ್ಷಾತ್ಕಾರವಾಗುವುದೆಂಬ ಭಾವನೆಯನ್ನು ಈ ಚಿತ್ರ ಸೂಚಿಸುವುದು.” 

 ರನದಬಾಬು ಚಿತ್ರದ ವ್ಯಾಖ್ಯಾನವನ್ನು ಕೇಳಿ ಕೃತಜ್ಞತೆಯಿಂದ ಕೊಂಚ ಹೊತ್ತು ಮೌನವಾಗಿದ್ದರು. ಅನಂತರ ಅವರು “ನಿಮ್ಮಿಂದ ಕಲೆಯ ವಿಚಾರ ಕಲಿಯಬೇಕೆಂದು ಆಸೆಯಾಗುವುದು” ಎಂದರು. 

 ಅನಂತರ ಸ್ವಾಮೀಜಿ ರನದಬಾಬುಗಳಿಗೆ ತಾವು ಹಾಕಿಟ್ಟಿದ್ದ ಮುಂದಿನ ಶ‍್ರೀರಾಮ ಕೃಷ್ಣ ದೇವಾಲಯ ಮತ್ತು ಮಠದ ಹಂಚಿಕೆಯ ನಕಾಶೆಯನ್ನು ತೋರಿಸಿದರು. ಅನಂತರ ಅವರು ಹೀಗೆ ಹೇಳತೊಡಗಿದರು: “ಈ ಭಾವೀ ದೇವಾಲಯ ಮತ್ತು ಮಠದ ಕಟ್ಟಡಗಳಲ್ಲಿ ನಾನು ಪೂರ್ವ ಮತ್ತು ಪಾಶ್ಚಾತ್ಯ ಕಲೆಗಳಲ್ಲಿ ಅತ್ಯುತ್ತಮವಾದುದನ್ನೇ ತೆಗೆದುಕೊಂಡು ಬರಬೇಕೆಂದಿದ್ದೇನೆ. ನಾನು ಇಡೀ ಜಗತ್ತನ್ನು ಸುತ್ತಿದುದರ ಪರಿಣಾಮವಾಗಿ ನನಗೆ ಗೊತ್ತಾಗಿರುವ ಶಿಲ್ಪ ಕಲೆಯ ಎಲ್ಲಾ ಭಾವನೆಗಳನ್ನು ಅದರ ರಚನೆಯಲ್ಲಿ ರೂಪಿಸಬೇಕೆಂದಿದ್ದೇನೆ. ಅಸಂಖ್ಯಾತ ಕಂಬಗಳ ಸಮುದಾಯದ ಆಧಾರದ ಮೇಲೆ ನಿಂತಿರುವ ಒಂದು ದೊಡ್ಡ ಪ್ರಾರ್ಥನಾ ಮಂದಿರ ಕಟ್ಟಲ್ಪಡುವುದು. ಅದರ ಗೋಡೆಗಳ ಮೇಲೆ ನೂರಾರು ಅರಳಿರುವ ತಾವರೆಗಳು ಕೊರೆಯಲ್ಪಡುವುವು. ಆ ಮಂದಿರ ಸಾವಿರ ಮಂದಿ ಕುಳಿತು ಧ್ಯಾನಮಾಡಲು ಸಾಧ್ಯವಾಗುವಂತಿರಬೇಕು. ಶ‍್ರೀರಾಮಕೃಷ್ಣ ದೇವಾಲಯ ಮತ್ತು ಪ್ರಾರ್ಥನಾ ಮಂದಿರ ಯಾವ ರೀತಿ ಕಟ್ಟಲ್ಪಡಬೇಕೆಂದರೆ ದೂರದಿಂದ ಅದನ್ನು ನೋಡಿದರೆ ಅದು ‘ಓಂ’ ಎಂಬ ಚಿಹ್ನೆಯ ಪ್ರತಿಬಿಂಬವಾಗಿರುವಂತೆ ಕಾಣಬೇಕು. ಶ‍್ರೀರಾಮಕೃಷ್ಣರ ಪ್ರತಿಮೆ ಇರಬೇಕು. ಬಾಗಿಲಿನ ಇಕ್ಕೆಲದಲ್ಲೂ ಒಂದು ಸಿಂಹ ಮತ್ತು ಕುರಿಮರಿ ಪ್ರೇಮದಿಂದ ಒಂದನ್ನೊಂದು ನೆಕ್ಕುತ್ತಿರುವಂತೆ ಚಿತ್ರಿಸಬೇಕು - ಮಹಾಶಕ್ತಿ ಮತ್ತು ಸಾಧು ಸ್ವಭಾವ ಪ್ರೇಮದಲ್ಲಿ ಮಿಳಿತವಾಗಿರುವ ಭಾವ ಮೂಡಿರಬೇಕು. ನನ್ನ ಮನಸ್ಸಿನಲ್ಲಿ ಈ ಉದ್ದೇಶಗಳಿವೆ. ನಾನು ಸಾಕಷ್ಟು ದಿನ ಜೀವಿಸಿದ್ದರೆ ಇವುಗಳನ್ನು ಕಾರ್ಯರೂಪಕ್ಕೆ ತಂದೇ ತೀರುವೆ. ಇಲ್ಲದಿದ್ದಲ್ಲಿ ಭಾವೀ ಸಂತತಿಯವರು ಕ್ರಮೇಣ ಇವುಗಳನ್ನು ಕಾರ‍್ಯರೂಪಕ್ಕೆ ತರಲು ಪ್ರಯತ್ನಿಸುವರು. ನನ್ನ ಅಭಿಪ್ರಾಯವೇನೆಂದರೆ, ನಮ್ಮ ದೇಶದ ಕಲೆ ಮತ್ತು ಸಂಸ್ಕೃತಿಯ ಎಲ್ಲಾ ಶಾಖೆಗಳನ್ನೂ ಪುನರುಜ್ಜೀವನಗೊಳಿಸಲೆಂದು ಶ‍್ರೀರಾಮಕೃಷ್ಣರು ಅವತರಿಸಿದ್ದು. ಆದ್ದರಿಂದ ಧರ್ಮ, ಕರ್ಮ, ಶಿಕ್ಷಣ, ಜ್ಞಾನ, ಭಕ್ತಿ ಇವೆಲ್ಲಾ ಈ ಕೆಂದ್ರದಿಂದ ಜಗತ್ತಿನಲ್ಲೆಲ್ಲಾ ಹರಡುವಂತಹ ರೀತಿಯಲ್ಲಿ ಈ ಮಠ ಕಟ್ಟಲ್ಪಡಬೇಕು. ನೀವೆಲ್ಲಾ ಈ ಕೆಲಸಕ್ಕೆ ನನ್ನ ಸಹಾಯಕರಾಗಿ.” 

 ರನದಬಾಬುಗಳು, ಅಲ್ಲಿ ಸೇರಿದ್ದ ಸಂನ್ಯಾಸಿಗಳು ಮತ್ತು ಬ್ರಹ್ಮಚಾರಿಗಳು ಎಲ್ಲರೂ ಸ್ವಾಮಿಗಳ ಮಾತುಗಳನ್ನು ಮೌನಮುಗ್ಧರಾಗಿ ಕೇಳುತ್ತಿದ್ದರು. ಸ್ವಲ್ಪ ಕಾಲಾನಂತರ ಸ್ವಾಮೀಜಿ ಮುಂದುವರಿಸಿದರು: “ನೀನು ಈ ಹಾದಿಯಲ್ಲಿ ಪ್ರವೀಣನಾದುದರಿಂದ ನಿನ್ನೊಡನೆ ಈ ವಿಷಯವನ್ನು ಅಷ್ಟು ದೀರ್ಘವಾಗಿ ಚರ್ಚಿಸುತ್ತಿದ್ದೇನೆ. ದಯವಿಟ್ಟು ಈಗ ನನಗೆ ಹೇಳು, ನೀನು ಇಷ್ಟು ದೀರ್ಘಕಾಲ ಕಲೆಯನ್ನು ಅಭ್ಯಸಿಸುತ್ತಿರುವೆಯಲ್ಲ, ಕಲೆಯ ಅತ್ಯುಚ್ಚ ಧ್ಯೇಯಗಳಾವುವೆಂಬುದರ ಬಗ್ಗೆ ಏನು ಕಲಿತಿರುವಿ?” 

 ರನದಬಾಬು: “ನಿಮಗೆ ನಾನು ಯಾವ ಹೊಸ ವಿಷಯ ಹೇಳಲಿ? ಅದಕ್ಕೆ ಬದಲಾಗಿ ಈ ವಿಷಯದಲ್ಲಿ ನೀವೇ ನನ್ನ ಕಣ್ಣನ್ನು ತೆರೆದಿರುವಿರಿ. ಕಲೆಯ ವಿಚಾರವಾಗಿ ಇಷ್ಟೊಂದು ಶಿಕ್ಷಣರೂಪವಾದ ಮಾತುಗಳನ್ನು ನನ್ನ ಜೀವಮಾನದಲ್ಲೇ ಎಂದೂ ಕೇಳಿರಲಿಲ್ಲ. ಸ್ವಾಮೀಜಿ, ದಯವಿಟ್ಟು, ನಿಮ್ಮಿಂದ ಪಡೆದ ಭಾವನೆಗಳನ್ನು ಕಾರ್ಯರೂಪಕ್ಕೆ ತರುವಂತೆ ನನ್ನನ್ನು ಹರಸಿ.” 

 ಅನಂತರ ಸ್ವಾಮೀಜಿಗಳು ತಮ್ಮ ಪೀಠದಿಂದೆದ್ದು ಅಂಗಳದಲ್ಲಿ ಸುತ್ತುತ್ತಾ ಶಿಷ್ಯನಿಗೆ ಹೇಳಿದರು.: “ಆ ಯುವಕ ಬಹಳ ಉತ್ಸಾಹಿ.” 

 ಶಿಷ್ಯ: “ಅವನು ನಿಮ್ಮ ಮಾತುಗಳನ್ನು ಕೇಳಿ ಸ್ತಂಭೀಭೂತನಾದ.” 

 ಸ್ವಾಮೀಜಿ ಶಿಷ್ಯನಿಗೆ ಉತ್ತರ ಕೊಡದೆ ಶ‍್ರೀರಾಮಕೃಷ್ಣರು ಹಾಡುತ್ತಿದ್ದ ಕೆಲವು ಹಾಡುಗಳನ್ನು ಮೆಲುದನಿಯಲ್ಲಿ ಹಾಡತೊಡಗಿದರು: “ಅಂಕೆಯಲ್ಲಿರುವ ಮನಸ್ಸು ಸ್ಪರ್ಶಮಣಿಯಂತೆ, ನೀನಾವುದನ್ನು ಇಚ್ಛಿಸುವೆಯೋ ಅದನ್ನೆಲ್ಲಾ ಕೊಡುವುದು.” 

 ಕೊಂಚದೂರ ನಡೆದಮೇಲೆ ಸ್ವಾಮೀಜಿ ಮುಖ ತೊಳೆದು ಶಿಷ್ಯನೊಡನೆ ತಮ್ಮ ಕೊಠಡಿಯನ್ನು ಪ್ರವೇಶಿಸಿದರು. “ಎನ್‍ಸೈಕ್ಲೋಪಿಡಿಯಾ ಬ್ರಿಟಾನಿಕ” ಎಂಬ ಪುಸ್ತಕದಲ್ಲಿದ್ದ ಕಲೆಯ ಮೇಲಿನ ಒಂದು ಲೇಖನವನ್ನು ಓದಿದರು. ಅದನ್ನು ಮುಗಿಸಿದ ಮೇಲೆ ಶಿಷ್ಯನನ್ನು ಪೂರ್ವ ಬಂಗಾಳದ ಪದಗಳ ಉಚ್ಚಾರಗಳನ್ನು ಗೇಲಿ ಮಾಡತೊಡಗಿದರು. 

