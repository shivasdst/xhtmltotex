
\chapter{ಸಂಘಸ್ಥಾಪನೆ}

 ಸ್ವಾಮೀಜಿಯವರು ಪಾಶ್ಚಾತ್ಯ ದೇಶಗಳಲ್ಲಿ ಸಂಚಾರ ಮಾಡುತ್ತಿದ್ದಾಗ ಅಲ್ಲಿಯ ಸಮಾಜದ ಸಂಸ್ಥಾಬದ್ಧ ಜೀವನವನ್ನು ನೋಡಿದರು. ತಮ್ಮ ದೇಶದಲ್ಲಿಯೂ ಶ‍್ರೀರಾಮಕೃಷ್ಣರ ತಳಹದಿಯ ಮೇಲೆ ನಿಂತ ಅಂತಹ ಒಂದು ಸಂಸ್ಥೆಯನ್ನು ಕಟ್ಟಿದರೆ ಮಾತ್ರ\break ಶ‍್ರೀರಾಮಕೃಷ್ಣರು ಯಾವ ಉದ್ದೇಶಕ್ಕೆ ಬಂದರೋ ಅದನ್ನು ಬಹುಕಾಲ ಇತರರಿಗೆ ಸಾರಲು ಸಾಧ್ಯವಾಗಬಹುದು ಎಂದು ಆಲೋಚಿಸಿದರು. ವ್ಯಕ್ತಿ ಎಷ್ಟೇ ದೊಡ್ಡವನಾಗಿದ್ದರೂ ಅವನ ಪ್ರಭಾವವೆಲ್ಲ ಬದುಕಿರುವ ತನಕ. ಒಂದು ಸಂಸ್ಥೆ ಮಾತ್ರ ದೀರ್ಘ ಕಾಲ ಬದುಕಬಲ್ಲದು. ಅದೇ ಭಾರತೀಯರಿಗೂ‌ ಪಾಶ್ಚಾತ್ಯರಿಗೂ ಇರುವ ವ್ಯತ್ಯಾಸ. ನಮ್ಮಲ್ಲಿ ದೊಡ್ಡ ವ್ಯಕ್ತಿಗಳಿರುವರು, ಆದರೆ ಅವರೆಲ್ಲ ಒಂದು ಸಾಮಾನ್ಯ ನಿಯಮಕ್ಕೆ ಬಾಗಿ ನಡೆಯುವುದು ಅಪರೂಪ. ಪ್ರತಿಯೊಬ್ಬನೂ ತನ್ನದೇ ಒಂದು ಮಾರ್ಗದಲ್ಲಿ ಹೋಗಬಯಸುವನು. ಇದರಿಂದ ಆತ್ಮನ ಮೋಕ್ಷವನ್ನು ಸಾಧಿಸಬಹುದು. ಆದರೆ ಜಗತ್ತಿಗೆ ಶ್ರೇಷ್ಠ ರೀತಿಯಲ್ಲಿ ಒಳ್ಳೆಯದಾಗಬೇಕಾದರೆ ಇಂತಹ ವ್ಯಕ್ತಿಗಳೆಲ್ಲ ಒಟ್ಟು ಕಲೆತು ಒಂದು ಸಾಮಾನ್ಯ ನಿಯಮಕ್ಕೆ ಬಾಗಿ ದುಡಿದರೆ ಮಾತ್ರ ಸಾಧ್ಯ. 

 ಸ್ವಾಮೀಜಿ ಆತ್ಮನ ಮೋಕ್ಷ, ಜೊತೆಗೆ ಜಗತ್ತಿನ ಹಿತ ಎರಡನ್ನೂ ಸೇರಿಸಿ ಒಂದು ಆದರ್ಶ ಮಾಡಲೆತ್ನಿಸಿದರು. ಆತ್ಮನ ಮೋಕ್ಷದ ಜೊತೆಗೆ ಜಗತ್ತಿನ ಹಿತವೂ ನಮ್ಮ ದೃಷ್ಟಿಯಲ್ಲಿರಬೇಕು. ಜಗದ ಹಿತವನ್ನು ಫಲಾಪೇಕ್ಷೆಯಿಲ್ಲದೆ ಭಗವತ್ ಸೇವೆಯ ದೃಷ್ಟಿಯಿಂದ ಮಾಡಿದರೆ ಅದು ಅವನು ಇಚ್ಛಿಸದೆ ಇದ್ದರೂ ಆತ್ಮನ ಮೋಕ್ಷವನ್ನು ಕೊಡುವುದು. ಇಂತಹ ಒಂದು ಸಂಸ್ಥೆಗೆ ಒಂದು ತಳಹದಿ ಬೇಕಾಗಿದೆ. ಈ ಪ್ರಪಂಚದಲ್ಲಿ ಯಾವುದಾದರೂ ದೊಡ್ಡ ಕಟ್ಟಡವನ್ನು ನೋಡಿದರೆ ಅದಕ್ಕೆ ಒಂದು ಆಳವಾದ ತಳಪಾಯ ಇದೆ ಎಂದು ಊಹಿಸುವೆವು. ಅದರಂತೆಯೇ ಒಂದು ದೊಡ್ಡ ಸಂಸ್ಥೆಗೆ ಒಂದು ಅದ್ಭುತವಾದ ಜೀವನದ ತಳಹದಿ ಇರಬೇಕು. ಆಗಲೇ ಅದು ಹಲವು ಶತಮಾನಗಳು ಇರುವುದಕ್ಕೆ ಸಾಧ್ಯ. ಶ‍್ರೀರಾಮಕೃಷ್ಣರ ಜೀವನ ಮತ್ತು ಉಪದೇಶದ ಮೇಲೆ ಅಂತಹ ಸಂಸ್ಥೆಯನ್ನು ಸ್ವಾಮೀಜಿ ಕಟ್ಟತೊಡಗಿದರು. ಶ‍್ರೀರಾಮಕೃಷ್ಣರನ್ನು ವಿವೇಕಾನಂದರು ಅರ್ಥಮಾಡಿಕೊಂಡಷ್ಟು ಇತರರು ಅರ್ಥಮಾಡಿಕೊಂಡಿರಲಿಲ್ಲ. ಇತರರು ಭಕ್ತಿಯಿಂದ ಅವರು ಹೇಳಿದುದನ್ನು ಕೇಳಿ ಶ್ರದ್ಧೆಯಿಂದ ಅನುಸರಿಸುತ್ತಿದ್ದರು. ಆದರೆ ವಿವೇಕಾನಂದರಾದರೋ ಶ‍್ರೀರಾಮಕೃಷ್ಣರನ್ನು ಪರೀಕ್ಷೆ ಮಾಡುತ್ತಿದ್ದರು. ಅವರನ್ನು ವಿಚಾರದ ಒರೆಗಲ್ಲಿಗೆ ತಿಕ್ಕಿ ಅನಂತರ ತೆಗೆದುಕೊಂಡರು. ಶ‍್ರೀರಾಮಕೃಷ್ಣರ ಜೀವನ ಆಧ್ಯಾತ್ಮಿಕ ಜೀವನದ ಒಂದು ಪರಾಕಾಷ್ಠೆ ಎಂಬುದನ್ನು ಮನಗಂಡರು. ಅವರು ಮುಕ್ತಪುರುಷರು. ಆದರೆ ಯಾವುದೋ ಒಂದು ದಾರಿಯಲ್ಲಿ ಹೋಗಿ ಮುಕ್ತರಾದವರಲ್ಲ. ಧರ್ಮದ ಎಲ್ಲಾ ಮಾರ್ಗಗಳಲ್ಲಿ ಹೋಗಿ ಒಂದೇ ಸತ್ಯವನ್ನು ಕಂಡು ಹಿಡಿದವರು. ಸತ್ಯವೊಂದೇ ಅದನ್ನು ಬೇರೆ ಬೇರೆ ಹೆಸರುಗಳಿಂದ ಕರೆಯುವರು ಎಂಬುದನ್ನು ನಾವು ವೇದದಲ್ಲಿ ನೋಡುತ್ತೇವೆ. ಆದರೆ ಶ‍್ರೀರಾಮಕೃಷ್ಣರ ಜೀವನದಲ್ಲಿ ಅದರ ಅನುಷ್ಠಾನವನ್ನು ನೋಡುವೆವು. ಅವರ ಜೀವನವೇ ಒಂದು ವಿಶ್ವಧರ್ಮ ಸಮ್ಮೇಳನವಾಗಿತ್ತು. ತಾವು ಏನನ್ನು ಕಂಡರೋ, ಅದನ್ನು ಪಡೆಯುವುದಕ್ಕೆ ಎಷ್ಟು ಕಾತರಪಟ್ಟರೋ ಅಷ್ಟೇ ಕಾತರಪಟ್ಟರು ತಾವು ಪಡೆದುದನ್ನು ಇತರರಿಗೆ ಹೇಳುವುದಕ್ಕೆ. ಯೋಗ್ಯರಾದ ಶಿಷ್ಯರನ್ನು ಕಳುಹಿಸಿ ಎಂದು ಜಗನ್ಮಯಿಯನ್ನು ಪ್ರಾರ್ಥಿಸಿದರು. ತಾವು ಕಂಡ ಬೆಳಕನ್ನು, ಅನುಭವಿಸಿದ ಶಾಂತಿಯನ್ನು, ಪ್ರಪಂಚದಲ್ಲಿ ಎಲ್ಲರಿಗೂ ಸಾರಬೇಕು, ಇದು ಎಲ್ಲ ಜೀವಿಗಳಿಗೂ ಆವಶ್ಯಕ ಎಂಬುದನ್ನು ಮನಗಂಡರು. ತಮ್ಮ ಆರೋಗ್ಯ, ಅನುಕೂಲ ಇವುಗಳನ್ನು ಒತ್ತಟ್ಟಿಗಿಟ್ಟು ತಾವು ಬದುಕಿರುವವರೆಗೆ ಅದನ್ನು ಬೋಧಿಸಿದರು. ಒಂದು ಜೀವಿ ದೇವರ ಕಡೆ ಹೋಗುವುದಕ್ಕೆ ಸಹಾಯಮಾಡಲು ತಾವು ಎಷ್ಟು ಯಾತನೆಯನ್ನಾದರೂ ಸಹಿಸುವುದಕ್ಕೆ ಸಿದ್ಧವಾಗಿರುವೆ ಎಂದಿದ್ದರು. ಬಡತನ ಕಷ್ಟ ದುಃಖಗಳಿಗೆ ಈಡಾದ ಜೀವಿಗಳು ಬಂದರೆ ಅವರು ಅಷ್ಟೇ ಅನುಕಂಪ ತೋರುತ್ತಿದ್ದರು. ಅವರಲ್ಲಿ ಈಶ್ವರನಿರುವನು, ಅವರಿಗೆ ಮಾಡುವ ಸೇವೆಯೇ ಈಶ್ವರನ ಸೇವೆ ಎಂದು ಅವರು ಬೋಧಿಸಿದ್ದರು. ಅವರಲ್ಲಿ ಎಲ್ಲಾ ಯೋಗಗಳ ತತ್ತ್ವಗಳ ಭಾವಗಳ ಸಮನ್ವಯವನ್ನು ನೋಡುವೆವು. ಇದಕ್ಕಿಂತ ಭೂಮವಾದ ಒಂದು ಆಧಾರ ವಿಶ್ವದಲ್ಲಿ ಸಿಕ್ಕಲಾರದೆಂದು ಅವರ ಹೆಸರಿನಲ್ಲಿ ಒಂದು ಸುವ್ಯವಸ್ಥಿತವಾದ ಸಂಸ್ಥೆಯನ್ನು ಕಟ್ಟಬೇಕೆಂದು ಸ್ವಾಮೀಜಿ ಸಂಕಲ್ಪ ಮಾಡಿಕೊಂಡರು. ಇಂತಹ ಸಂಸ್ಥೆ ಆತ್ಮನ ಮೋಕ್ಷ, ಜಗತ್ತಿನ ಹಿತ ಎರಡನ್ನೂ ಸಾಧಿಸಬೇಕು, ಅದು ಬಹು ಜನರ ಸುಖಕ್ಕೆ ಆಗಿರಬೇಕು. 

 ಸಂಸ್ಥೆಯನ್ನು ನಡೆಸಬೇಕಾದರೆ ಹಲವು ಶ್ರೇಷ್ಠ ವ್ಯಕ್ತಿಗಳ ಸಹಾಯ ಆವಶ್ಯಕ. ವಿವೇಕಾನಂದರು ತಮ್ಮ ಗುರುಭಾಯಿಗಳ ಸಹಾಯವನ್ನು ಇದಕ್ಕಾಗಿ ಕೋರಿದರು. ಆದರೆ ಅವರನ್ನು ಶ‍್ರೀರಾಮಕೃಷ್ಣರ ಹೆಸರಿನಲ್ಲಿ ಒಂದು ಸಂಸ್ಥೆಗೆ ಬದ್ಧರಾಗಿ ಎಂದು ಒಪ್ಪಿಸುವುದು ಬಹಳ ದೊಡ್ಡ ಕೆಲಸವಾಯಿತು. ಅವರಲ್ಲಿ ಅನೇಕರು ಕೆಲವು ಪುಣ್ಯ ಸ್ಥಳಗಳಲ್ಲಿ ತಪಸ್ಸಿನಲ್ಲಿ ನಿರತರಾಗಿದ್ದರು. ಕಲ್ಕತ್ತೆಯ ಮಠದಲ್ಲಿ ಕೂಡ ಕೆಲವರು ಸಾಧನೆ ಭಜನೆಯಲ್ಲಿ ನಿರತರಾಗಿದ್ದರು. ಮೊದಮೊದಲು ಲೋಕಕ್ಕೆ ಒಳ್ಳೆಯದನ್ನು ಮಾಡಬೇಕು, ಸಾಧನ ಭಜನೆಯನ್ನು ಅದರೊಡನೆ ಸೇರಿಸಬೇಕೆಂದಾಗ ಕೆಲವರಿಗೆ ಈ ಆದರ್ಶ ಕತ್ತಲೆ ಬೆಳಕಿನ ಮಿಶ್ರದಂತೆ ಕಂಡಿತು. ಈ ಪ್ರಪಂಚವೇ ಒಂದು ಮಿಥ್ಯೆ, ಇದೊಂದು ನಾಯಿಯ ಡೊಂಕು ಬಾಲದಂತೆ. ಈ ಪ್ರಪಂಚವನ್ನು ಯಾರಾದರೂ ನೇರ ಮಾಡಲಾದೀತೆ, ಮಾಡಿದರೆ ಎಷ್ಟು ಕಾಲ ಇರುವುದು? ಇದರ ಕಡೆ ಗಮನವನ್ನೇಕೆ ಕೊಡಬೇಕು? ಅದೂ ಅಲ್ಲದೆ, ಅನೇಕರಿಗೆ ಶ‍್ರೀರಾಮಕೃಷ್ಣರ ಜೀವನದಲ್ಲಿ ಭಕ್ತಿ ಜ್ಞಾನವೈರಾಗ್ಯಗಳು ಕಾಣುತ್ತಿದ್ದವೆ ಹೊರತು ಇನ್ನೊಬ್ಬರಿಗೆ ಒಳ್ಳೆಯದನ್ನು ಮಾಡುವುದು ಮುಂತಾದ ಬೋಧನೆಗಳಿಗೆ ಶ‍್ರೀರಾಮಕೃಷ್ಣರು ಪ್ರಾಧಾನ್ಯತೆಯನ್ನು ಕೊಟ್ಟಂತೆ ಕಂಡುಬರುವುದಿಲ್ಲ. ಅವರು ಬಡಪೆಟ್ಟಿಗೆ ವಿವೇಕಾನಂದರ ಮಾತಿಗೆ ಬಗ್ಗಲಿಲ್ಲ. ಆದರೆ ವಿವೇಕಾನಂದರು ಸುಮ್ಮನೆ ಬಿಡುವವರ ಗುಂಪಿಗೆ ಸೇರಿದವರಲ್ಲ. ಮೊದಲು ವಾದ ಮಾಡಿದರು, ಅನಂತರ ಪ್ರೀತಿ ವಿಶ್ವಾಸದಿಂದ ಅವರನ್ನು ಒಲಿಸಿಕೊಂಡರು. ಪ್ರಾರ್ಥನೆ ಭಜನೆ ಎಂಬುದು ಕೂಡ ಮಿಥ್ಯಾಪ್ರಪಂಚಕ್ಕೆ ಸೇರಿದ್ದು ತಾನೆ ಎಂದರು. ವೇದಾಂತದ ಪ್ರಕಾರ ಆತ್ಮ ನಿತ್ಯಮುಕ್ತವಾದುದು. ಸಾಧನೆ ಭಜನೆಯನ್ನೆ ಒಬ್ಬ ಎಷ್ಟು ಕಾಲ ಮಾಡಲು ಸಾಧ್ಯ? ಉಳಿದ ಕಾಲವನ್ನು ಕಳೆಯುವುದು ಹೇಗೆ? ಪರನಿಂದೆ ಪರಚರ್ಚೆ ಮತ್ತು ಸ್ವಾರ್ಥದಲ್ಲಿ ಕಳೆಯುವುದಕ್ಕಿಂತ ಮತ್ತೊಬ್ಬರಿಗೆ ನಿಃಸ್ವಾರ್ಥವಾಗಿ ಸೇವೆ ಮಾಡಿದರೆ ಅದರಿಂದ ಚಿತ್ತಶುದ್ಧಿಯಾಗುವುದು. ಚಿತ್ತ ಶುದ್ಧವಾದರೆ ಚೆನ್ನಾಗಿ ಧ್ಯಾನ ಜಪಾದಿಗಳನ್ನು ಮಾಡಬಹುದು. ನಮ್ಮ ಸಾಧನೆಯ ಜೀವನಕ್ಕೆ ಕರ್ಮ ಸಹಾಯಮಾಡುವುದು. ಶ‍್ರೀರಾಮಕೃಷ್ಣರು ತಮಗಾಗಿ ಬರಲಿಲ್ಲ, ಪ್ರಪಂಚದಲ್ಲಿ ಎಲ್ಲರಿಗಾಗಿ ಬಂದರು. ಹಾಗೆ ಅವರ ಜೀವನವನ್ನು ಎಲ್ಲರಿಗೂ ಹಲವು ಶತಮಾನಗಳು ಸಾರಬೇಕಾದರೆ ಅದಕ್ಕಾಗಿ ಬಾಳನ್ನೇ ತೆತ್ತ ಯತಿಗಳಿಂದ ಮಾತ್ರ ಸಾಧ್ಯ. ಅಂತಹ ಯತಿಯಾಗುವಂತಹ ವ್ಯಕ್ತಿಗಳನ್ನು ಆಕರ್ಷಿಸಿ ಅವರಿಗೆ ತರಬೇತು ನೀಡಲು ಒಂದು ಸಂಸ್ಥೆ ಆವಶ್ಯಕ. ಹೀಗೆ ವಾದದ ಮೂಲಕ ಸ್ವಾಮೀಜಿ ತಮ್ಮ ಗುರುಭಾಯಿಗಳನ್ನು ಗೆಲ್ಲಲು ಪ್ರಯತ್ನಿಸಿದರು. ವಾದಮಾಡಿ ಅವರನ್ನು ಸೋಲಿಸಿದರೂ ಅವರ ಹೃದಯ ಮಾತ್ರ ಇದಕ್ಕೆ ಬಡಪೆಟ್ಟಿಗೆ ಒಪ್ಪಲಿಲ್ಲ. ಆಗ ಸ್ವಾಮೀಜಿ ತಮ್ಮ ಅದ್ಭುತವಾದ ಪ್ರೀತಿಯಿಂದ ಅವರನ್ನು ಒಲಿಸಿಕೊಂಡರು. ಜಗತ್ತು ಅವರ ವಾಕ್ಚಾತುರ್ಯವನ್ನು ನೋಡಿದೆ, ಅವರ ಪಾಂಡಿತ್ಯವನ್ನು ನೋಡಿದೆ, ಆದರೆ ಅವರ ಹೃದಯದಲ್ಲಿದ್ದ ಪ್ರೀತಿ ಇವನ್ನೆಲ್ಲ ಮೀರುತ್ತಿತ್ತು. ಅದಕ್ಕೆ ಬಾಗದವರೇ ಇರಲಿಲ್ಲ. ಆ ಪ್ರೀತಿಯ ಬ್ರಹ್ಮಾಸ್ರವನ್ನು ಉಪಯೋಗಿಸಿದಾಗ ಮಾತ್ರ ಎಲ್ಲರೂ ಇವರಿಗೆ ಶರಣಾದರು. ಇವರ ಇಚ್ಛೆಯಂತೆ ಮಾಡತೊಡಗಿದರು. ಶ‍್ರೀರಾಮಕೃಷ್ಣರ ಬೋಧನೆಗೂ ಸ್ವಾಮೀಜಿ ಇದಕ್ಕೆ ಕೊಡುವ ವಿವರಣೆಗೂ ಯಾವ ವ್ಯತ್ಯಾಸವೂ ಇಲ್ಲವೆಂದು ಅನಂತರ ಭಾವಿಸತೊಡಗಿದರು. ವಿವೇಕಾನಂದರ ವಿವರಣೆಗಳನ್ನು ಮೊದಲು ಎಷ್ಟೇ ವಿರೋಧಿಸಿದರೂ ಅವರ ಮೇಲೆ ಇದ್ದ ಪ್ರೀತಿ ಮಾತ್ರ ಯಾರಿಗೂ ಕಡಿಮೆ ಆಗಿರಲಿಲ್ಲ. ಅವರ ಆದರ್ಶವನ್ನು ಕ್ರಮೇಣ ಮೆಚ್ಚತೊಡಗಿದರು. ಸ್ವಾಮೀಜಿ ಅಮೇರಿಕಾ ದೇಶದಲ್ಲಿ ಇದ್ದಾಗಲೇ ತಮ್ಮ ಭಾವನೆಗಳನ್ನು ಪತ್ರದ ಮೂಲಕ ಗುರುಭಾಯಿಗಳಿಗೆ ತಿಳಿಸುತ್ತಿದ್ದರು. ಅವರು ಅಲ್ಲಿದ್ದಾಗಲೆ ಇಂಡಿಯಾ ದೇಶದಿಂದ ಅಭೇದಾನಂದ ಮತ್ತು ಶಾರದಾನಂದ ಅವರ ಸಹಾಯಕ್ಕೆ ಪಾಶ್ಚಾತ್ಯದೇಶಗಳಿಗೆ ಹೋದರು. ರಾಮಕೃಷ್ಣಾನಂದ ಸ್ವಾಮಿಗಳು ಹನ್ನೆರಡು ವರ್ಷಗಳು ಕಲ್ಕತ್ತೆಯ ಮಠದಲ್ಲಿ ತಮ್ಮ ಶ‍್ರೀಗುರುದೇವರ ಪೂಜೆ ಮತ್ತು ಸೇವಾಕಾರ್ಯದಲ್ಲಿ ನಿರತರಾಗಿದ್ದವರು ಸಂಘದ ಕೆಲಸಕ್ಕಾಗಿ ಮದ್ರಾಸಿಗೆ ಹೊರಟುಬಂದರು. ಅಖಂಡಾನಂದರು ಮುರ್ಷಿದಾಬಾದಿನಲ್ಲಿ ಒಂದು ಸೇವಾಶ್ರಮವನ್ನು ತೆರೆದರು. 

 ಸ್ವಾಮೀಜಿ ಕಲ್ಕತ್ತೆಯಲ್ಲಿದ್ದಾಗ ಹಿಮಾಲಯದಲ್ಲಿ ಒಂದು ಆಶ್ರಮ ತೆರೆಯಬೇಕೆಂದೂ ಕಲ್ಕತ್ತೆಯಲ್ಲಿ ಸ್ವಂತ ಸ್ಥಳದಲ್ಲಿ ಒಂದು ಮಠವನ್ನು ಸ್ಥಾಪಿಸಬೇಕೆಂದೂ ಶ‍್ರೀರಾಮಕೃಷ್ಣ ಮಿಷನ್ ಎಂಬ ಸಂಸ್ಥೆಯನ್ನು ಕಟ್ಟಬೇಕೆಂದೂ ಆಲೋಚಿಸುತ್ತಿದ್ದರು. ಬಿಡುವಿಲ್ಲದ ದುಡಿತದಿಂದ ಅವರ ಆರೋಗ್ಯ ಹದಗೆಟ್ಟಿತು. ಅದಕ್ಕಾಗಿ ವೈದ್ಯರು ಹವಾ ಬದಲಾವಣೆಗೆ ಹೋಗುವಂತೆ ಸ್ವಾಮೀಜಿಯವರಿಗೆ ಹೇಳಿದರು. ಅದರಂತೆ ಸ್ವಾಮೀಜಿ ಡಾರ್ಜಿಲಿಂಗಿಗೆ ಹೋದರು. ಅವರಿಗೆ ಮುಂಚೆ ಆಗಲೆ ಸೇವಿಯರ‍್ಸ ದಂಪತಿಗಳು ಅಲ್ಲಿಗೆ ಹೋಗಿದ್ದರು. ಸ್ವಾಮೀಜಿಯವರೊಡನೆ ಅವರ ಕೆಲವು ಗುರುಭಾಯಿಗಳು ಮತ್ತು ಭಕ್ತರು ಹೋದರು. ಬರ್ದವಾನ್ ಮಹಾರಾಜರು ಸ್ವಾಮೀಜಿಯವರನ್ನು ಬಹಳ ಗೌರವಿಸುತ್ತಿದ್ದರು. ಅವರು ಸ್ವಾಮೀಜಿಯವರ ವೃಂದಕ್ಕೆ Rose Bank ಎಂಬ ತಮ್ಮ ಮನೆಯ ಒಂದು ಭಾಗವನ್ನು ಕೊಟ್ಟರು. ಸ್ವಾಮೀಜಿಯವರೊಡನೆ ಬ್ರಹ್ಮಾನಂದ ತ್ರಿಗುಣಾತೀತ ಮತ್ತೆ ಇನ್ನು ಕೆಲವು ಪಾಶ್ಚಾತ್ಯ ಭಕ್ತರೂ ಮದ್ರಾಸಿನಿಂದ ಹೋದ ಮೂರು ಜನರೂ ಇದ್ದರು. ಈಗ ಸ್ವಾಮೀಜಿ ಸಂಪೂರ್ಣವಾಗಿ ವಿಶ್ರಾಂತಿಯನ್ನು ಅನುಭವಿಸತೊಡಗಿದರು. ಹಿಮಾಲಯದ ದಾರಿಗಳಲ್ಲಿ ನಡೆಯುವುದು, ಹತ್ತಿರ ಇದ್ದ ಒಂದು ಬುದ್ಧ ವಿಹಾರಕ್ಕೆ ಹೋಗುವುದು, ತಮ್ಮ ಗುರುಭಾಯಿ ಮತ್ತು ಶಿಷ್ಯರೊಡನೆ ಮಾತುಕತೆ ಧ್ಯಾನ ಇವುಗಳಲ್ಲಿ ಕಳೆಯತೊಡಗಿದರು. 

 ಒಮ್ಮೆ ಮಾತ್ರ ಸ್ವಾಮೀಜಿ ಡಾರ್ಜಿಲಿಂಗ್‍ನಲ್ಲಿ ಇದ್ದಾಗ ಕಲ್ಕತ್ತೆಗೆ ಬಂದು ಹೋದರು. ಖೇತ್ರಿ ಮಹಾರಾಜರು ಸ್ವಾಮೀಜಿಯವರನ್ನು ನೋಡಬೇಕೆಂದು ಕಲ್ಕತ್ತೆಗೆ ಬಂದರು. ಈ ಸಮಾಚಾರವನ್ನು ಕೇಳಿ ಸ್ವಾಮೀಜಿ ಕಲ್ಕತ್ತೆಗೆ ಬಂದರು. ಅವರಿಗೆ ಮಠದಲ್ಲಿ ಆದರಾತಿಥ್ಯಗಳನ್ನು ಮಾಡಿದರು. ತಮ್ಮ ಪಾಶ್ಚಾತ್ಯ ಅನುಭವಗಳನ್ನೆಲ್ಲ ಹೇಳಿದರು. ಖೇತ್ರಿ ಮಹಾರಾಜರು ಇತರ ಕೆಲವು ರಾಜರೊಡನೆ ಇಂಗ್ಲೆಂಡಿಗೆ ಹೋಗುವುದರಲ್ಲಿದ್ದರು. ಸ್ವಾಮೀಜಿಯವರನ್ನೂ ಜೊತೆಗೆ ಬರಬೇಕೆಂದು ಕೋರಿಕೊಂಡರು. ಆದರೆ ಸ್ವಾಮೀಜಿ ವೈದ್ಯರ ಸಲಹೆಯ ಮೇರೆಗೆ ಹೋಗಲು ಆಗಲಿಲ್ಲ. ಸ್ವಾಮೀಜಿ ಅನಂತರ ಡಾರ್ಜಿಲಿಂಗಿಗೆ ಹೋಗಿ ಪುನಃ ಎರಡು ವಾರಗಳ ಮಟ್ಟಿಗೆ ಕಲ್ಕತ್ತೆಗೆ ಬಂದರು. ಕಲ್ಕತ್ತೆಯ ಮಠದಲ್ಲಿ ತಮ್ಮ ಶಿಷ್ಯರಿಗೆ ಪಾಠ ಪ್ರವಚನಗಳನ್ನು ತೆಗೆದುಕೊಳ್ಳುತ್ತಿದ್ದರು. ಆ ಸಮಯದಲ್ಲಿಯೇ ಕೆಲವು ಬ್ರಹ್ಮಚಾರಿಗಳಿಗೆ ಸಂನ್ಯಾಸ ದೀಕ್ಷೆ ಕೊಟ್ಟರು. ಆಗ ಇವರಿಂದ ಸಂನ್ಯಾಸ ದೀಕ್ಷೆಯನ್ನು ಪಡೆದವರೇ ವಿರಜಾನಂದ, ನಿರ್ಭಯಾನಂದ, ಪ್ರಕಾಶಾನಂದ, ನಿತ್ಯಾನಂದ. ಸಂನ್ಯಾಸ ಕೊಟ್ಟಾದಮೇಲೆ ಸ್ವಾಮೀಜಿ ಹೀಗೆ ಅವರಿಗೆ ಉಪದೇಶವಿತ್ತರು: 

 “ಯಾರು ಸಂನ್ಯಾಸಿಗಳೊ ಅವರು ಪರರಿಗೆ ಹಿತವನ್ನು ಮಾಡುವುದಕ್ಕೆ ಪ್ರಯತ್ನಿಸಬೇಕು. ಸಂನ್ಯಾಸವೆಂದರೆ ಮೃತ್ಯು ಪ್ರೇಮ. ಪ್ರಾಪಂಚಿಕರು ಸಂಸಾರವನ್ನು ಪ್ರೀತಿಸುತ್ತಾರೆ. ಸಂನ್ಯಾಸಿ ಮೃತ್ಯುವನ್ನು ಪ್ರೀತಿಸಬೇಕು. ಹಾಗಾದರೆ ನಾವು ಆತ್ಮಹತ್ಯೆಯನ್ನು ಮಾಡಿಕೊಳ್ಳಬೇಕೆ? ಅದಲ್ಲ. ಆತ್ಮಹತ್ಯೆಯನ್ನು ಮಾಡಿಕೊಳ್ಳುವವರು ಮೃತ್ಯು ಪ್ರೇಮಿಗಳಲ್ಲ. ಹಾಗಾದರೆ ಮೃತ್ಯು ಪ್ರೇಮವೆಂದರೇನು? ನಾವು ಸಾಯಬೇಕು ಅದು ನಿಶ್ಚಯ. ಹಾಗಾದರೆ ಒಂದು ಒಳ್ಳೆಯ ಆದರ್ಶಕ್ಕೆ ಸಾಯುವ. ನಮ್ಮ ಕ್ರಿಯೆಗಳೆಲ್ಲ ಊಟಮಾಡುವುದು, ಕುಡಿಯುವುದು ಪ್ರತಿಯೊಂದೂ ಆತ್ಮತ್ಯಾಗಕ್ಕೆ ಸಹಾಯ ಮಾಡಬೇಕು. ಆಹಾರದಿಂದ ನಿಮ್ಮ ದೇಹವನ್ನು ರಕ್ಷಿಸುವಿರಿ. ಇತರರ ಸೇವೆಗೆ ಉಪಯೋಗಿಸದೆ ಇದ್ದರೆ ಅದನ್ನು ರಕ್ಷಿಸಿ ಪ್ರಯೋಜನವೇನು? ಗ್ರಂಥವನ್ನು ಓದಿ ಮನಸ್ಸನ್ನು ವಿಕಾಸ ಮಾಡಿಕೊಳ್ಳುವಿರಿ. ಇದನ್ನು ಪ್ರಪಂಚಕ್ಕೆ ಕೊಡದೆ ಇದ್ದರೆ ಇದರಿಂದ ಪ್ರಯೋಜನವೇನು? ನಿಮ್ಮ ಕ್ಷುದ್ರ ಅಹಂಕಾರವನ್ನು ಬೆಳೆಸಿಕೊಳ್ಳುವುದಕ್ಕಿಂತ ನಿಮ್ಮ ಲಕ್ಷಾಂತರ ಸಹೋದರರನ್ನು ಸೇವಿಸುವುದು ಮೇಲು. ನೀವು ಹಾಗೆ ಕ್ರಮೇಣ ಮೃತ್ಯು ಮುಖರಾಗಬೇಕು. ಇಂತಹ ಮರಣದಲ್ಲಿ ಸ್ವರ್ಗವಿದೆ. ಸರ್ವ ಶುಭದ ಆಗರ ಇದು. ಇದಕ್ಕೆ ವಿರೋಧವಾಗಿರುವುದೇ ನರಕ, ಪಾಪ.” 

 “ಅನಂತರ ಈ ದರ್ಶನವನ್ನು ಅನುಷ್ಠಾನಕ್ಕೆ ತರುವ ಮಾರ್ಗ. ಅನುಷ್ಠಾನ ಮಾಡಲು ಆದರ್ಶ ಅಸಾಧ್ಯವಾಗಿರಬಾರದು. ಇದನ್ನು ತಿಳಿದುಕೊಳ್ಳಬೇಕು. ನಮ್ಮ ಸಾಮರ್ಥ್ಯಕ್ಕೆ ನಿಲುಕದ ಆದರ್ಶವಿದ್ದರೆ ಸ್ಫೂರ್ತಿ ದುರ್ಬಲವಾಗಿ ಅಧೋಗತಿಗೆ ಇಳಿಯುವುದು. ನಾವು ಆದರ್ಶವನ್ನು ಬಹಳ ಕೆಳಗೂ ತರಕೂಡದು. ಈ ಎರಡು ಅತಿಗಳಿಂದ ಪಾರಾಗಬೇಕು. ನಮ್ಮ ದೇಶದಲ್ಲಿ ಹಿಂದಿನಕಾಲದ ಆದರ್ಶ ಒಂದು ಗುಹೆಯಲ್ಲಿ ಕುಳಿತು ಧ್ಯಾನ ಮಾಡಿ ಸಾಯುವುದು.” 

 “ಮುಕ್ತಿಗಾಗಿಯೂ ಇತರರಿಗಿಂತ ಮುಂಚೆಯೆ ಹೋಗುವುದು ತಪ್ಪು. ನಮ್ಮ ಸಹೋದರರ ಮುಕ್ತಿಗೆ ನಾವು ಪ್ರಯತ್ನಿಸದೆ ಇದ್ದರೆ ನಮಗೂ ಮುಕ್ತಿ ಇಲ್ಲ ಎಂಬುದನ್ನು ಇಂದೋ ನಾಳೆಯೋ‌ ಕಲಿಯಬೇಕಾಗಿದೆ. ಅದ್ಭುತ ಆದರ್ಶ ಮತ್ತು ಅದ್ಭುತ ಅನುಷ್ಠಾನ ಅವೆರಡನ್ನೂ ನಿಮ್ಮ ಜೀವನದಲ್ಲಿ ಸೇರಿಸುವುದನ್ನು ಕಲಿಯಬೇಕು. ಈ ಕ್ಷಣ ಧ್ಯಾನಪರವಶರಾಗಲು ಸಿದ್ಧರಾಗಬೇಕು. ಮರುಕ್ಷಣ ಎದುರಿಗಿರುವ ಹೊಲವನ್ನು ಉಳಲು ಸಿದ್ಧರಾಗಿರಬೇಕು. ನಂತರ ಶಾಸ್ತ್ರದ ಜಟಿಲ ಸಮಸ್ಯೆಯನ್ನು ಬಗೆಹರಿಸಲು ಸಿದ್ಧರಾಗಿರಬೇಕು.” 

 “ಅನಂತರ ನಾವು ಜ್ಞಾಪಕದಲ್ಲಿಡಬೇಕಾದುದೇ ಈ ಸಂಘದ ಉದ್ದೇಶ ಪುರುಷ ನಿರ್ಮಾಣ ಎಂಬುದು. ಋಷಿಗಳ ಬೋಧನೆಯನ್ನು ನೀವು ಕಲಿತುಕೊಳ್ಳುವುದು ಮಾತ್ರವಲ್ಲ. ಆ ಋಷಿಗಳೆಲ್ಲ ಹೊರಟುಹೋದರು. ಅವರ ಭಾವನೆಯೂ ಅವರೊಂದಿಗೆ ಹೋಯಿತು. ನೀವು ಋಷಿಗಳಾಗಬೇಕು. ಪ್ರಪಂಚದಲ್ಲಿ ಜನಿಸಿದ ಶ್ರೇಷ್ಠ ತಮ ವ್ಯಕ್ತಿಗಳಂತೆ, ಅವತಾರಗಳಂತೆ ನೀವು ಕೂಡ ಮನುಷ್ಯರೇ. ಕೇವಲ ಪುಸ್ತಕ ಪಾಂಡಿತ್ಯದಿಂದ ಪ್ರಯೋಜನವೇನು? ಧ್ಯಾನದಿಂದ ತಾನೆ ಏನು ಪ್ರಯೋಜನ? ಮಂತ್ರತಂತ್ರಗಳಿಂದ ಏನು ಪ್ರಯೋಜನ? ನೀವು ನಿಮ್ಮ ಕಾಲುಗಳ ಮೇಲೆ ನಿಲ್ಲಬೇಕು. ಈ ಒಂದು ಹೊಸ ಮಾರ್ಗ ನಿಮಗೆ ಬೇಕು. ಪುರುಷ ನಿರ‍್ಮಾಣ ಮಾರ್ಗ. ನಿಜವಾದ ವೀರಪುರುಷ ಶಕ್ತಿಯೇ ರೂಪು ತಾಳಿದಂತೆ ಇರುವನು. ಆದರೂ ಅವನಲ್ಲಿ ಅಬಲೆಯ ಕೋಮಲ ಹೃದಯವಿರುವುದು. ನಮ್ಮ ಸುತ್ತಲಿರುವ ಮಾನವಕೋಟಿಯ ಮೇಲೆ ಅನುಕಂಪ ತಾಳಬೇಕು. ಆದರೂ ಬಲಾಢ್ಯನಾಗಿರಬೇಕು, ಅದಮ್ಯನಾಗಿರಬೇಕು. ಆದರೂ ವಿಧೇಯತೆ ಇರಬೇಕು. ಇದು ವಿರೋಧಾಭಾಸದಂತೆ ತೋರಬಹುದು. ಆದರೂ ತೋರಿಕೆಗೆ ವಿರೋಧವಾಗಿರುವ ಈ ಗುಣಗಳು ಇರಬೇಕು. ನಮ್ಮ ಹಿರಿಯರು ನದಿಗೆ ಧುಮುಕಿ ಮೊಸಳೆ ಹಿಡಿಯಿರಿ ಎಂದರೆ ಮೊದಲು ಅದನ್ನು ಪಾಲಿಸಿ ಅನಂತರ ಚರ್ಚಿಸಬೇಕು. ಅಪ್ಪಣೆ ತಪ್ಪಾಗಿದ್ದರೂ ಮೊದಲು ಅದನ್ನು ಪಾಲಿಸಿ ಅನಂತರ ಅದನ್ನು ವಿರೋಧಿಸಬೇಕು. ಸಂಘದ ಮೇಲೆ ಶ್ರದ್ಧಾ ವಿಶ್ವಾಸಗಳಿರಬೇಕು. ಆಜ್ಞೋಲ್ಲಂಘನೆಗೆ ಇಲ್ಲಿ ಅವಕಾಶವಿಲ್ಲ. ನಿರ್ದಾಕ್ಷಿಣ್ಯವಾಗಿ ಅದನ್ನು ನಿರ್ಮೂಲ ಮಾಡಿ. ಈ ಸಂಘದಲ್ಲಿ ಆಜ್ಞೆಯನ್ನು ಪಾಲಿಸದವರಿಗೆ ಎಡೆಯಿಲ್ಲ. ಅವರನ್ನು ಆಚೆಗೆ ಕಳುಹಿಸಬೇಕು. ಸಂಘದಲ್ಲಿ ದ್ರೋಹಿಗಳಾರೂ ಇರಕೂಡದು. ವಾಯುವಿನಂತೆ ಸ್ವಚ್ಛಂದವಾಗಿರಬೇಕು. ಈ ಗಿಡ ಅಥವಾ ನಾಯಿಯಂತೆ ನಮ್ರನಾಗಿರಬೇಕು.” 

 ಸ್ವಾಮೀಜಿ ಕೆಲವು ದಿನಗಳಾದ ಮೇಲೆ ಒಬ್ಬ ಬ್ರಹ್ಮಚಾರಿಗೆ ಮತ್ತು ಶರತ್ ಚಂದ್ರ ಚಕ್ರವರ್ತಿ ಎಂಬ ಗೃಹೀಭಕ್ತನಿಗೆ ಉಪದೇಶವನ್ನು ಕೊಟ್ಟರು. ಅವರಿಗೆ ಉಪದೇಶವನ್ನು ಕೊಡುವ ಸಮಯದಲ್ಲಿ ಹೀಗೆ ಹೇಳಿದರು: “ಈಗಿನ ಕಾಲದಲ್ಲಿ ದೇಶಕ್ಕೆ ಎಂತಹ ದುರವಸ್ಥೆ ಪ್ರಾಪ್ತವಾಗಿದೆ! ಶಾಸ್ತ್ರೋಕ್ತ ಮಾರ್ಗವನ್ನು ಬಿಟ್ಟು ಕೇವಲ ಕೆಲವು ದೇಶಾಚಾರ, ಲೋಕಾಚಾರ ಮತ್ತು ಹೆಂಗಸಿನ ಆಚಾರದಲ್ಲಿ ದೇಶವೆಲ್ಲ ಮುಳುಗಿಹೋಗಿದೆ. ಆದ್ದರಿಂದಲೇ ನೀವು ಪ್ರಾಚೀನ ಕಾಲದಂತೆ ಶಾಸ್ತ್ರಮಾರ್ಗವನ್ನು ಅನುಸರಿಸಿ ಹೋಗಿರೆಂದು ನಿಮಗೆ ಹೇಳುತ್ತೇನೆ. ನೀವು ಶ್ರದ್ಧಾವಂತರಾಗಿ ದೇಶದಲ್ಲಿಯೂ ಶ್ರದ್ಧೆಯನ್ನು ಉಂಟುಮಾಡಿ. ನಚಿಕೇತನಗಿದ್ದಂತೆ ಶ್ರದ್ಧೆಯನ್ನು ಹೃದಯದಲ್ಲಿ ತಂದುಕೊಳ್ಳಿ. ನಚಿಕೇತನ ಹಾಗೆ ಯಮಲೋಕಕ್ಕೆ ಬೇಕಾದರೂ ಹೋಗಿ. ಆತ್ಮತತ್ತ್ವವನ್ನು ತಿಳಿದುಕೊಳ್ಳುವುದಕ್ಕೋಸ್ಕರ, ಆತ್ಮೋದ್ಧಾರಕ್ಕೋಸುಗ ಈ ಜನನ ಮರಣ ಸಮಸ್ಯೆಯ ವಿಚಾರಕ್ಕೋಸ್ಕರ ನಿರ್ಭಯ ಹೃದಯಕ್ಕಾಗಿ ಯಮನ ಬಾಯಿಗೆ ಹೋಗಿ ಬೀಳಬೇಕು. ಭಯವೇ ಮೃತ್ಯು. ಭಯದ ಆಚೆಯ ದಡಕ್ಕೆ ಹೋಗಬೇಕು. ಇಂದಿನಿಂದ ಭಯರಹಿತನಾಗು. ನಡೆ, ಹೊರಡು, ನಿನ್ನ ಮೋಕ್ಷಕ್ಕೆ ಜಗದ ದೇಹವನ್ನು ಕೊಡು. ಕೇವಲ ರಕ್ತಮಾಂಸ ಮೂಳೆಗಳ ಮುದ್ದೆಯನ್ನು ಹೊತ್ತುಕೊಂಡು ಏನು ಪ್ರಯೋಜನ? ಈಶ್ವರಾರ್ಥವಾಗಿ ಸರ್ವಸ್ವ ತ್ಯಾಗರೂಪವಾದ ಮಂತ್ರದೀಕ್ಷೆಯನ್ನು ಪಡೆದು ದಧೀಚಿ ಮುನಿಯ ಹಾಗೆ ಪರರಿಗೋಸ್ಕರ ಮೂಳೆಮಾಂಸವನ್ನು ದಾನಮಾಡು.” 

 ಸ್ವಾಮೀಜಿ ಕೆಲವು ದಿನಗಳಿಂದ ಬಾಗ್‍ಬಜಾರಿನ ಬಲರಾಮ ಬಾಬುಗಳ ಮನೆಯಲ್ಲಿದ್ದರು. ೧೮೯೭, ಮೇ ೧ ರಂದು ಅವರು ಪರಮಹಂಸರ ಗೃಹಸ್ಥ ಭಕ್ತರಿಗೆ ಒಂದು ಕಡೆ ಸೇರುವಂತೆ ಹೇಳಿ ಕಳುಹಿಸಿದ್ದುದರಿಂದ ಮೂರು ಗಂಟೆಯ ಮೇಲೆ ಸಾಯಂಕಾಲದ ಹೊತ್ತಿಗೆ ಪರಮಹಂಸರ ಬಹು ಭಕ್ತರು ಆ ಮನೆಯಲ್ಲಿ ಸೇರಿದ್ದಾರೆ. ಯೋಗಾನಂದ ಸ್ವಾಮಿಗಳೂ ಅಲ್ಲಿಗೆ ಬಂದಿದ್ದಾರೆ. ಎಲ್ಲರೂ ಕುಳಿತುಕೊಂಡಮೇಲೆ ಸ್ವಾಮೀಜಿ ಹೀಗೆಂದರು: 

 “ನಾನಾ ದೇಶಗಳನ್ನು ಸುತ್ತಿ ನನಗೆ ದೃಢವಾದ ನಂಬಿಕೆ ಬಂದುಹೋಗಿದೆ, ಸಂಘವಿಲ್ಲದೆ ಯಾವ ದೊಡ್ಡ ಕಾರ್ಯವೂ ನಡೆಯಲಾರದು ಎಂದು. ಆದರೆ ನಮ್ಮದರಂತಹ ದೇಶಗಳಲ್ಲಿ ಮೊದಲಿನಿಂದಲೂ ಸಾಧಾರಣ ರೀತಿಯಲ್ಲಿ ಸಂಘವನ್ನು ಏರ್ಪಡಿಸುವುದು ಅಥವಾ ಸಾಧಾರಣ ಜನರ ಸಮ್ಮತಿಯನ್ನು ತೆಗೆದುಕೊಂಡು ಕೆಲಸ ಮಾಡುವುದು ಅಷ್ಟು ಸುಲಭವೆಂದು ತೋರುವುದಿಲ್ಲ. ಪಾಶ್ಚಾತ್ಯ ನರನಾರಿಯರು ಸುಶಿಕ್ಷಿತರಾಗಿದ್ದಾರೆ, ನಮ್ಮ ಹಾಗೆ ದ್ವೇಷಪರಾಯಣರಲ್ಲ. ಅವರು ಗುಣಕ್ಕೆ ಗೌರವಕೊಡುವುದನ್ನು ಕಲಿತುಕೊಂಡಿದ್ದಾರೆ. ಇದನ್ನು ನೋಡುವುದಿಲ್ಲವೇನು? ನಾನು ಒಬ್ಬ ಲೆಕ್ಕಕ್ಕೆ ಇಲ್ಲದ ಮನುಷ್ಯ. ನನ್ನನ್ನು ಆ ದೇಶದಲ್ಲಿ ಎಷ್ಟು ಆದರಿಸುತ್ತಾರೆ! ಈ ದೇಶದಲ್ಲಿ ಶಿಕ್ಷಣ ಪ್ರಸರಿಸಿ ಅದರಿಂದ ಯಾವಾಗ ಸಾಮಾನ್ಯ ಜನರು ಹೆಚ್ಚು ಸಹೃದಯರಾಗುತ್ತಾರೊ ಯಾವಾಗ ಜಾತಿ ಮತಗಳ ಹೊರಗೆ ಬರುವುದನ್ನು ಕಲಿಯುತ್ತಾರೊ, ಆಗ ಜನಾಭಿಪ್ರಾಯವನ್ನು ಅನುಸರಿಸಿ ಕೆಲಸ ನಡೆಸಬಹುದು. ಅದಕ್ಕೋಸ್ಕರವೇ ಈ ಸಂಘಕ್ಕೆ ಒಬ್ಬ ಪ್ರಧಾನ ಪ್ರಚಾರಕನು ಬೇಕು. ಎಲ್ಲರೂ ಆತನ ಆಜ್ಞೆಗೆ ಗೌರವ ಕೊಟ್ಟುಕೊಂಡು ಹೋಗಬೇಕು. ಆಮೇಲೆ ಕ್ರಮೇಣ ಎಲ್ಲರ ಮತವನ್ನು ತಿಳಿದುಕೊಂಡು ಕಾರ‍್ಯವನ್ನು ಮಾಡಬಹುದು. ನಾನು ಯಾರ ಹೆಸರಿನಿಂದ ಸಂನ್ಯಾಸಿಯಾಗಿದ್ದೇನೆಯೋ, ನೀವು ಯಾರನ್ನು ಜೀವನದಲ್ಲಿ ಆದರ್ಶವಾಗಿಟ್ಟುಕೊಂಡು ಸಂಸಾರಾಶ್ರಮವನ್ನು ಸ್ವೀಕರಿಸಿ ಕಾರ‍್ಯಕ್ಷೇತ್ರದಲ್ಲಿರುವಿರೊ, ಯಾರ ಪುಣ್ಯನಾಮವೂ ಅದ್ಭುತ ಜೀವನವೂ‌ ದೇಹಾವಸಾನವಾದ ಇಪ್ಪತ್ತು\break ವರ್ಷಗಳ ಒಳಗೆ ಪ್ರಾಚ್ಯ ಮತ್ತು ಪಾಶ್ಚಾತ್ಯ ಜಗತ್ತಿನಲ್ಲಿ ಆಶ್ಚರ್ಯಕರವಾದ ರೀತಿಯಲ್ಲಿ ಪ್ರಸಾರಿತವಾಯಿತೊ, ಆತನ ಹೆಸರಿನಲ್ಲಿಯೇ ಈ ಸಂಘ ಪ್ರತಿಷ್ಠಿತವಾಗಿದೆ. ನಾನು ಪ್ರಭುವಿನ ದಾಸ. ನೀವು ಈ ಕಾರ್ಯದಲ್ಲಿ ಸಹಾಯ ಮಾಡಬೇಕು.” 

 ಶ‍್ರೀಯುತ ಗಿರೀಶಚಂದ್ರ ಘೋಷರೇ ಮೊದಲಾದ ಅಲ್ಲಿ ಬಂದ ಗೃಹಸ್ಥರು ಈ ಪ್ರಸ್ತಾಪವನ್ನು ಅನುಮೋದಿಸಿದರು. ಗೃಹೀ ಮತ್ತು ಸಂನ್ಯಾಸೀ ಭಕ್ತರನ್ನೊಳಗೊಂಡ ಒಂದು ಆಡಳಿತ ಸಮಿತಿಯ ರಚನೆಯೊಂದಿಗೆ ರಾಮಕೃಷ್ಣ ಸಂಘವು ಅಸ್ತಿತ್ವಕ್ಕೆ ಬಂತು. ಸಂಘಕ್ಕೆ ರಾಮಕೃಷ್ಣರ ಪ್ರಚಾರ ಅಥವಾ ರಾಮಕೃಷ್ಣ ಮಿಷನ್ ಎಂದು ಹೆಸರು ಕೊಟ್ಟು ಆಯಿತು. ಅದರ ಉದ್ದೇಶ ಮುಂತಾದುವನ್ನು ಕೆಳಗೆ ಕೊಟ್ಟಿದ್ದೇವೆ. 

 ಉದ್ದೇಶ: ಮಾನವ ಹಿತಾರ್ಥವಾಗಿ ಶ‍್ರೀರಾಮಕೃಷ್ಣ ಪರಮಹಂಸರು ಯಾವ ತತ್ತ್ವಗಳನ್ನು ವಿವರಿಸಿದರೊ ಮತ್ತು ತಮ್ಮ ಜೀವನದ ಅನುಷ್ಠಾನದಲ್ಲಿ ತೋರಿದರೊ ಅವುಗಳ ಪ್ರಚಾರ ಮತ್ತು ಈ ತತ್ತ್ವಗಳನ್ನು ಅನುಭವಕ್ಕೆ ತಂದುಕೊಳ್ಳಲು ಸಹಕಾರಿಗಳಾದ ಮನುಷ್ಯನ ದೈಹಿಕ ಮಾನಸಿಕ ಮತ್ತು ಪಾರಮಾರ್ಥಿಕ ಉನ್ನತಿಗಳಲ್ಲಿ ಸಹಾಯ ಮಾಡುವುದು ಈ ಮಿಷನ್ನಿನ ಉದ್ದೇಶ. 

 ವ್ರತ: ಜಗತ್ತಿನ ಎಲ್ಲಾ ಧರ್ಮಗಳೂ ಒಂದು ಅಕ್ಷಯ ಸನಾತನ ಧರ್ಮದ ರೂಪಾಂತರ ಮಾತ್ರ ಎಂಬ ಜ್ಞಾನದಿಂದ ಸಕಲ ಧರ್ಮಾನುಯಾಯಿಗಳಲ್ಲಿಯೂ ಆತ್ಮೀಯತೆಯನ್ನು ಬೆಳೆಸುವುದಕ್ಕೋಸ್ಕರ ಶ‍್ರೀರಾಮಕೃಷ್ಣ ಪರಮಹಂಸರು ಯಾವ ಕಾರ್ಯವನ್ನು ಆರಂಭಿಸಿದರೋ ಅದರ ಪರಿಚಾಲನವೇ ಈ ಮಿಷನ್ನಿನ ವ್ರತ. 

 ಕಾರ್ಯಕ್ರಮ: ಮನುಷ್ಯರ ಸಾಂಸಾರಿಕ ಮತ್ತು ಆಧ್ಯಾತ್ಮಿಕ ಉನ್ನತಿಗೋಸುಗ ವಿದ್ಯೆಯನ್ನು ದಾನಮಾಡುವುದಕ್ಕೆ ಉಪಯೋಗವಾಗುವಂತೆ ಜನರಿಗೆ ಶಿಕ್ಷಣ ಕೊಟ್ಟು ತಯಾರು ಮಾಡುವುದು. ಕುಶಲಕಲೆಗಳಲ್ಲಿಯೂ ಕಷ್ಟಪಟ್ಟು ಕೆಲಸ ಮಾಡಬೇಕಾದ ಕಸಬುಗಳಲ್ಲಿಯೂ ಉತ್ಸಾಹವನ್ನು ಹೆಚ್ಚಿಸುವುದು ಮತ್ತು ವೇದಾಂತ ಮತ್ತು ಇತರ ಧರ್ಮಭಾವಗಳು ಶ‍್ರೀರಾಮಕೃಷ್ಣರ ಜೀವನದಲ್ಲಿ ಹೇಗೆ ವಿವರಿಸಲ್ಪಟ್ಟವೋ‌ ಹಾಗೆ ಅವುಗಳನ್ನು ಜನರಲ್ಲಿ ಪ್ರವರ್ತಿಸುವಂತೆ ಮಾಡುವುದು. 

 ಭರತಖಂಡದಲ್ಲಿ ಕಾರ್ಯಕ್ರಮ: ಭರತಖಂಡದ ಪ್ರತಿ ನಗರಗಳಲ್ಲಿಯೂ ಬೋಧಕರಾಗಲು ಇಚ್ಛಿಸುವ ಗೃಹಸ್ಥ ಅಥವಾ ಸಂನ್ಯಾಸಿಗಳಿಗೆ ಶಿಕ್ಷಣ ಕೊಡುವುದಕ್ಕೋಸ್ಕರ ಆಶ್ರಮಗಳನ್ನು ಸ್ಥಾಪಿಸುವುದು ಮತ್ತು ಅವರು ಯಾವ ರೀತಿಯಲ್ಲಿ ದೇಶ ದೇಶಾಂತರಗಳಿಗೆ ಹೋಗಿ ಜನರನ್ನು ಶಿಕ್ಷಿತರನ್ನಾಗಿ ಮಾಡಲು ಸಾಧ್ಯವೋ ಅದನ್ನು ಅನುಸರಿಸುವುದು. 

 ವಿದೇಶೀಯ ಕಾರ್ಯವಿಭಾಗ: ಭರತಖಂಡದ ಹೊರಗೆ ಇರುವ ಪ್ರದೇಶಗಳಲ್ಲಿ ವ್ರತಧಾರಿಗಳಾಗುವಂತೆ ಪ್ರೇರೇಪಿಸುವುದು ಮತ್ತು ಆಯಾ ದೇಶಗಳಲ್ಲಿ ಸ್ಥಾಪಿಸಲ್ಪಟ್ಟ ಆಶ್ರಮಗಳೊಡನೆ ಭಾರತೀಯ ಆಶ್ರಮಗಳಿಗೆ ನಿಕಟ ಸಂಬಂಧವೂ ಸಹಾನುಭೂತಿಯೂ ಬೆಳೆಯುವಂತೆ ಮಾಡುವುದು ಮತ್ತು ಹೊಸ ಹೊಸ ಆಶ್ರಮಗಳನ್ನು ಸ್ಥಾಪಿಸುವುದು. 

 ಸ್ವಾಮೀಜಿ ಮೇಲೆ ಹೇಳಿದ ಸಮಿತಿಯ ಸಾಧಾರಣ ಸಭಾಪತಿಗಳಾದರು. ಬ್ರಹ್ಮಾನಂದ ಸ್ವಾಮಿಗಳು ಕಲ್ಕತ್ತೆಯ ಕೇಂದ್ರಕ್ಕೆ ಸಭಾಪತಿಗಳೂ, ಯೋಗಾನಂದರು ಅವರ ಸಹಕಾರಿಗಳೂ ಆದರು. ಅಟಾರ‍್ನಿ ಬಾಬು ನರೇಂದ್ರನಾಥ ಮಿತ್ರರು ಇದರ ಕಾರ್ಯದರ್ಶಿಗಳಾಗಿಯೂ ಡಾಕ್ಟರ್ ಶಶಿಭೂಷಣಘೋಷರು ಮತ್ತು ಶರತ್‍ಚಂದ್ರಸರ್ಕಾರರು ಉಪಕಾರ್ಯದರ್ಶಿಗಳಾಗಿಯೂ ಚುನಾಯಿಸಲ್ಪಟ್ಟರು. ಇದರ ಜೊತೆಯಲ್ಲಿ ಪ್ರತಿ ಭಾನುವಾರವೂ ನಾಲ್ಕು ಗಂಟೆಯ ಮೇಲೆ ಬಲರಾಮಬಾಬುಗಳ ಮನೆಯಲ್ಲಿ ಸಮಿತಿ ಸೇರಬೇಕೆಂಬ ಮತ್ತೊಂದು ನಿಯಮವೂ ಮಾಡಲ್ಪಟ್ಟಿತು. ಹಿಂದೆ ಹೇಳಿದ ಸಭೆಯು ಏರ್ಪಾಡಾದ ನಂತರ ಮೂರು ವರುಷಗಳ ಕಾಲ ಬಲರಾಮ ಬಸು ಅವರ ಮನೆಯಲ್ಲಿ ಸಮಿತಿಯ ಅಧಿವೇಶನ ನಡೆಯಿತು. ಸ್ವಾಮೀಜಿ ಮತ್ತೆ ವಿಲಾಯತಿಗೆ ಹೋಗುವ ತನಕ ಅನುಕೂಲವಾದಾಗ ಕೂಟಕ್ಕೆ ಬರುತ್ತ, ಮಧ್ಯೆ ಮಧ್ಯೆ ಉಪದೇಶಮಾಡುತ್ತಲೂ ತಮ್ಮ ಕಿನ್ನರ ಕಂಠದಲ್ಲಿ ಗಾನಮಾಡುತ್ತಲೂ ಇದ್ದರೆಂದು ಹೇಳಬೇಕಾಗಿಲ್ಲ. ಅಂದಿನ ಸಭೆ ಮುಗಿದ ಕೂಡಲೇ ಸಭಿಕರೆಲ್ಲರೂ ಹೊರಟು ಹೋಗಲು ಸ್ವಾಮೀಜಿ, ಯೋಗಾನಂದರನ್ನು ಕುರಿತು “ಕಾರ್ಯವೇನೋ ಹೀಗೆ ಆರಂಭವಾಯಿತು. ಈಗ ನೋಡು, ಪರಮಹಂಸರ ದಯೆಯಿಂದ ಕಾರ್ಯವು ಎಷ್ಟು ದೂರ ಮುಂದುವರಿಯುವುದು” ಎಂದು ಹೇಳಿದರು. 

 ಸ್ವಾಮಿ ಯೋಗಾನಂದ: “ನಿಮ್ಮ ಈ ಏರ್ಪಾಡೆಲ್ಲ ವಿದೇಶೀಯ ರೀತಿಯಲ್ಲಿ ಮಾಡಲ್ಪಟ್ಟಿದೆ. ಪರಮಹಂಸರ ಉಪದೇಶವೇನು ಈ ರೂಪವಾಗಿತ್ತೆ?” 

 ಸ್ವಾಮೀಜಿ: “ಇದೆಲ್ಲ ಪರಮಹಂಸರ ಅಭಿಪ್ರಾಯವಾಗಿರಲಿಲ್ಲವೆಂದು ನೀನು ಹೇಗೆ ತಿಳಿದುಕೊಂಡೆ? ಅನಂತ ಭಾವಮಯರಾದ ಪರಮಹಂಸರನ್ನು ನೀವು ನಿಮ್ಮ ಎಲ್ಲೆಗಳಲ್ಲಿ ಕಟ್ಟಿಹಾಕಲಿಚ್ಛಿಸುವಿರೆಂದು ತೋರುತ್ತದೆ. ನಾನು ಈ ಎಲ್ಲೆಯನ್ನು ಒಡೆದುಹಾಕಿ ಅವರ ಭಾವ ಜಗತ್ತಿನಲ್ಲೆಲ್ಲ ವ್ಯಾಪಿಸುವಂತೆ ಮಾಡುತ್ತೇನೆ. ಪರಮಹಂಸರು ತಮ್ಮ ಪೂಜೆಯನ್ನು ಮತ್ತು ಪಾಠಪ್ರವಚನಗಳನ್ನು ಮಾಡಬೇಕೆಂದು ನನಗೆ ಯಾವಾಗಲೂ ಉಪದೇಶ ಕೊಡಲಿಲ್ಲ. ಸಾಧನ, ಭಜನ, ಧ್ಯಾನಧಾರಣ ಮತ್ತು ಉಚ್ಚಧರ್ಮ ಭಾವಗಳಿಗೆ ಸಂಬಂಧಪಟ್ಟ ಯಾವ ಯಾವ ಉಪದೇಶಗಳನ್ನು ಅವರು ಕೊಟ್ಟುಹೋದರೋ ಅವುಗಳನ್ನೇ ಪಡೆದುಕೊಂಡು ಜೀವರಿಗೆ ಶಿಕ್ಷಣ ಕೊಡಬೇಕು. ಅನಂತ ಮತ, ಅನಂತ ಪಥ. ಸಂಪ್ರದಾಯಗಳಿಂದ ತುಂಬಿಹೋಗಿರುವ ಈ ಜಗತ್ತಿನಲ್ಲಿ ಮತ್ತೊಂದು ನೂತನ ಸಂಪ್ರದಾಯವನ್ನು ಕಲ್ಪಿಸುವುದಕ್ಕೆ ಹೋಗುವುದು ನನ್ನ ಜನ್ಮದಲ್ಲಿಯೇ ಆಗಲಾರದು. ಪ್ರಭುದೇವನ ಪದತಲದಲ್ಲಿ ಆಶ್ರಯ ಪಡೆದು ನಾನು ಧನ್ಯನಾಗಿದ್ದೇನೆ. ಮೂರು ಜಗತ್ತಿನ ಜನರಿಗೂ ಅವರ ಭಾವಗಳನ್ನು ಕೊಡುವುದಕ್ಕೇ ನಾನು ಹುಟ್ಟಿರುವುದು.” 

 ಯೋಗಾನಂದ ಸ್ವಾಮಿಗಳು ಈ ಮಾತಿಗೆ ಬದಲು ಹೇಳದೆ ಇದ್ದದ್ದರಿಂದ ಸ್ವಾಮೀಜಿ ಮತ್ತೂ ಹೇಳತೊಡಗಿದರು: “ಪ್ರಭುವಿನ ದಯೆಗೆ ನಿದರ್ಶನವನ್ನು ಮೇಲಿಂದ ಮೇಲೆ ಈ ಜೀವನದಲ್ಲಿ ನೋಡಿದ್ದೇನೆ. ಆತನು ಹಿಂದೆ ನಿಂತುಕೊಂಡೇ ಈ ಕೆಲಸವನ್ನೆಲ್ಲ\break ಮಾಡಿಸುತ್ತಿದ್ದಾನೆ. ಯಾವಾಗ ಹಸಿವಿನಿಂದ ಬಳಲಿ ಬೇಸತ್ತು ಮರದ ಕೆಳಗೆ ಬಿದ್ದಿರುತ್ತಿದ್ದೆನೊ ಆವಾಗಲೂ ಪರಮಹಂಸರ ದಯೆಯಿಂದ ಎಲ್ಲಾ ಸಹಾಯವನ್ನೂ ಪಡೆದಿದ್ದೆ. ಮತ್ತು ಯಾವಾಗ ಇದೇ ವಿವೇಕಾನಂದರನ್ನು ದರ್ಶನ ಮಾಡಬೇಕೆಂದು ಚಿಕಾಗೊ ನಗರದ ರಸ್ತೆಯಲ್ಲಿ ನೂಕುನುಗ್ಗಲುಗಳಾಗುತ್ತಿದ್ದವೋ, ಯಾವ ಸನ್ಮಾನದ ನೂರರಲ್ಲಿ ಒಂದು ಅಂಶವನ್ನು ಪಡೆದರೂ ಸಾಧಾರಣ ಮನುಷ್ಯ ಉನ್ಮತ್ತನಾಗಿಬಿಡುವನೊ, ಅಂಥ ಸನ್ಮಾನವನ್ನು ಪರಮಹಂಸರ ದಯೆಯಿಂದ ಸುಲಭವಾಗಿ ಅರಗಿಸಿಕೊಂಡಿದ್ದೇನೆ. ಪ್ರಭುವಿನ ಇಚ್ಛೆಯಿಂದ ಎಲ್ಲೆಲ್ಲಿಯೂ ವಿಜಯ. ಈಗ ಈ ದೇಶದಲ್ಲಿ ಸ್ವಲ್ಪ ಕೆಲಸವನ್ನು ಮಾಡುವುದಕ್ಕೆ ಹೊರಟಿದ್ದೇನೆ. ನೀವು ಸಂಶಯವನ್ನು ಬಿಟ್ಟು ಕೆಲಸದಲ್ಲಿ ಸಹಾಯಮಾಡಿ. ಅವರ ಇಚ್ಛೆಯಂತೆ ಎಲ್ಲಾ ಕೈಗೂಡುತ್ತದೆ ಎಂಬುದನ್ನು ನೀವೇ ನೋಡುವಿರಿ.” 

 ಯೋಗಾನಂದ: “ನೀವು ಯಾವುದನ್ನು ಮನಸ್ಸಿನಲ್ಲಿ ತರುತ್ತೀರೊ ಅದೇ ಆಗುತ್ತದೆ. ನಾವಾದರೂ ಚಿರಕಾಲ ನಿಮ್ಮ ಆಜ್ಞಾವರ್ತಿಗಳು. ಪರಮಹಂಸರು ನಿಮ್ಮ ಮೂಲಕ ಅದೆಲ್ಲವನ್ನು ಮಾಡಿಸುತ್ತಾರೆ ಎಂಬುದನ್ನು ನಾನು ಚೆನ್ನಾಗಿ ಕಂಡುಕೊಂಡಿದ್ದೇನೆ. ಆದರೆ ಪರಮಹಂಸರ ಕೆಲಸ ಕಾರ್ಯಗಳು ಬೇರೆ ವಿಧವಾಗಿದ್ದುದನ್ನು ನೋಡಿಲ್ಲವೆ ಎಂಬುದಾಗಿ ಮಧ್ಯೆ ಮಧ್ಯೆ ಇಂತಹ ಸಂಶಯಗಳು ಬರುತ್ತವೆ. ಅದಕ್ಕೋಸ್ಕರವೇ ನಿಮಗೆ ಪ್ರತಿವೇಳೆ ಎಚ್ಚರಿಕೆ ಕೊಟ್ಟೆ.” 

 ಸ್ವಾಮೀಜಿ: “ನೀನು ಏನು ತಿಳಿದುಕೊಂಡಿದ್ದೀಯ? ಸಾಧಾರಣ ಜನರು ಪರಮಹಂಸರನ್ನು ಎಷ್ಟು ತಿಳಿದುಕೊಂಡಿದ್ದಾರೆಯೋ ಅವರು ನಿಜವಾಗಿ ಇರುವುದು ಅಷ್ಟೇ ಅಲ್ಲ. ಅವರು ಅನಂತ ಭಾವಮಯರಾದವರು. ಬ್ರಹ್ಮಜ್ಞಾನಕ್ಕಾದರೂ ಎಲ್ಲೆಯುಂಟು. ಪ್ರಭುವಿನ ದುರ್ಜ್ಞೇಯವಾದ ಭಾವಕ್ಕೆ ಎಲ್ಲೆಯಿಲ್ಲ. ಅವರ ಕೃಪಾಕಟಾಕ್ಷದಿಂದ ಲಕ್ಷ ವಿವೇಕಾನಂದರು ಈಗ ತಯಾರಾಗಬಲ್ಲರು. ಆದರೆ ಅವರೇ ಹಾಗೆ ಮಾಡದೆ, ನನ್ನನ್ನು ಒಂದು ಯಂತ್ರರೂಪವಾಗಿ ಬೇಕೆಂದೇ ಮಾಡಿಕೊಂಡು ನನ್ನ ಮೂಲಕ ಹೀಗೆ ಮಾಡಿಸುತ್ತಿದ್ದಾರೆ. ಅದಕ್ಕೆ ನಾನೇನು ಮಾಡಲಿ ಹೇಳು?” 

 ಹೀಗೆಂದು ಹೇಳಿ ಸ್ವಾಮೀಜಿ ಬೇರೆ ಕೆಲಸಕ್ಕಾಗಿ ಮತ್ತೆಲ್ಲಿಗೊ ಹೋದರು. ಯೋಗಾನಂದಸ್ವಾಮಿಗಳು ಹತ್ತಿರ ಇದ್ದ ಶಿಷ್ಯರನ್ನು ಕುರಿತು ಹೇಳಿದರು: “ವಿವೇಕಾನಂದ ಸ್ವಾಮಿಗಳ ನಂಬಿಕೆಯನ್ನು ಕೇಳಿದೆಯೋ? ಪರಮಹಂಸರ ಕೃಪಾಕಟಾಕ್ಷದಿಂದ ಲಕ್ಷ ವಿವೇಕಾನಂದರು ತಯಾರಾಗಬಲ್ಲರೆಂದು ಹೇಳಿದರಲ್ಲವೆ? ಏನು ಗುರುಭಕ್ತಿ? ನಮಗೆ ಅದರಲ್ಲಿ ನೂರರಲ್ಲಿ ಒಂದು ಭಾಗ ಇದ್ದರೂ ಧನ್ಯರಾಗಿಬಿಡುತ್ತಿದ್ದೆವು.” 

 ಶಿಷ್ಯ: “ಮಹಾರಾಜ್, ಸ್ವಾಮಿಗಳ ವಿಷಯವಾಗಿ ಪರಮಹಂಸರು ಏನು ಹೇಳುತ್ತಿದ್ದರು?” 

 ಯೋಗಾನಂದ: “ಇಂಥ ಆಧಾರ ಈ ಯುಗದಲ್ಲಿ ಜಗತ್ತಿನಲ್ಲಿ ಮತ್ತಾವಾಗಲೂ ಬಂದಿರಲಿಲ್ಲ ಎಂದು ಹೇಳುತ್ತಿದ್ದರು. ಕೆಲವು ವೇಳೆ ನರೇನನು ಪುರುಷ ತಾವು ಪ್ರಕೃತಿ ಎಂದೂ, ನರೇನನು ತಮ್ಮ ಮಾವನ ಮನೆಯೆಂದೂ ಹೇಳುತ್ತಿದ್ದರು. ಮತ್ತೆ ಕೆಲವು ವೇಳೆ, ಅಖಂಡ ಆಲಯದಲ್ಲಿ ಎಲ್ಲಿ ದೇವ ದೇವಿಯರಲ್ಲೆ ಬ್ರಹ್ಮನಿಂದ ಬೇರೆಯಾಗಿ ತಮ್ಮ ಅಸ್ತಿತ್ವನ್ನು ಇಟ್ಟುಕೊಳ್ಳುವುದಕ್ಕೆ ಆಗುತ್ತಿರಲಿಲ್ಲವೋ, ಅದರಲ್ಲಿಯೇ ಲೀನವಾಗಿ ಹೋಗಿಬಿಡುತ್ತಿದ್ದರೋ ಅಲ್ಲಿ ಏಳು ಜನ ಋಷಿಗಳು ತಮ್ಮ ಅಸ್ತಿತ್ವವನ್ನು ಬೇರೆ ಇಟ್ಟುಕೊಂಡು ಧ್ಯಾನದಲ್ಲಿ ನಿಮಗ್ನರಾಗಿದ್ದುದನ್ನು ನೋಡಿದೆ, ನರೇನನು ಅವರಲ್ಲಿ ಒಬ್ಬರ ಅಂಶಾವತಾರ ಎಂದು ಹೇಳುತ್ತಿದ್ದರು. ಮತ್ತೆ ಕೆಲವು ವೇಳೆ, ಜಗತ್ಪಾಲಕನಾದ ನಾರಾಯಣನು ನರ ಮತು ನಾರಾಯಣ ಎಂಬ ಹೆಸರಿನಿಂದ ಯಾವ ಎರಡು ಋಷಿ ರೂಪಗಳನ್ನು ಧಾರಣೆಮಾಡಿ ಲೋಕ ಕಲ್ಯಾಣಕ್ಕೋಸುಗ ತಪಸ್ಸು ಮಾಡಿದನೊ, ನರೇಂದ್ರ ಅದೇ ನರ ಋಷಿಯ ಅವತಾರ ಎಂದು ಹೇಳುತಿದ್ದರು. ಮತ್ತೆ ಕೆಲವು ವೇಳೆ ಶುಕದೇವನ ಹಾಗೆ ಮಾಯೆಯೂ ಅವನನ್ನು ಮುಟ್ಟದೆ ಹೋಯಿತು ಎಂದು ಹೇಳುತ್ತಿದ್ದರು. ಅವರು ನರೇಂದ್ರನನ್ನು ಇವರೆಲ್ಲರ ಸಮಷ್ಟಿ ಪ್ರಕಾಶವೆಂದು ಹೇಳುತ್ತಿದ್ದರು. ನರೇನನಲ್ಲಿ ಋಷಿಯ ವೇದಜ್ಞಾನ, ಶಂಕರನ ತ್ಯಾಗ, ಬುದ್ಧನ ಹೃದಯ, ಶುಕದೇವನ ಮಾಯಾರಾಹಿತ್ಯ ಮತ್ತು ಬ್ರಹ್ಮಜ್ಞಾನದ ಪೂರ್ಣ ವಿಕಾಸ ಇವೆಲ್ಲವೂ ಒಟ್ಟಿಗೆ ಇರುವುದನ್ನು ನೋಡಲಾರದೆ ಹೋದೆಯೇನು?” 

 ಸ್ವಾಮೀಜಿಯವರು ೧೮೯೯ರಲ್ಲಿ ಬೇಲೂರು ಮಠವನ್ನು ಸ್ಥಾಪಿಸಿ ಒಂದು ಟ್ರಸ್ಟ್ ಡೀಡಿನ ಮೂಲಕ ಅದನ್ನು ನೋಡಿಕೊಳ್ಳುವುದಕ್ಕೆ ಕೆಲವು ಟ್ರಸ್ಟಿಗಳನ್ನು ನೇಮಕ ಮಾಡಿ ಅಧಿಕಾರವನ್ನೆಲ್ಲಾ ಅವರಿಗೆ ವಹಿಸಿದರು. ಶ‍್ರೀರಾಮಕೃಷ್ಣ ಮಿಷನ್ ಮಾಡುತ್ತಿದ್ದ ಕೆಲಸವನ್ನೂ ಅವರೇ ನೋಡಿಕೊಳ್ಳಲು ಪ್ರಾರಂಭಿಸಿದ ಮೇಲೆ ಹಿಂದೆ ಸ್ಥಾಪಿತವಾದುದು ಇದರಲ್ಲಿ ಐಕ್ಯವಾಯಿತು. ಕೆಲವು ವರ್ಷಗಳಾದ ಮೇಲೆ ಸಾರ್ವಜನಿಕ ಕೆಲಸಗಳು ಮತ್ತು ಸಂನ್ಯಾಸಿ ವೃಂದದ ತರಬೇತು ಹಾಗೂ ಪ್ರಚಾರಶಾಖೆಯನ್ನು ಬೇರೆ ಮಾಡಿದರೆ ಇನ್ನೂ ಹೆಚ್ಚು ಚೆನ್ನಾಗಿ ಕೆಲಸ ನಡೆಯುವುದೆಂದು, ೧೯೦೯ರಲ್ಲಿ ೧೮೯೦ನೇ ಇಸವಿ ೨೧ನೇ ಆಕ್ಟಿನ ಪ್ರಕಾರ ಶ‍್ರೀರಾಮಕೃಷ್ಣ ಮಿಷನ್ನನ್ನು ರಿಜಿಸ್ಟರ್ ಮಾಡಿದರು. ಬೇಲೂರು ಮಠದ ಟ್ರಸ್ಟಿಗಳು ಮಿಷನ್ನಿನ ಗೌರ‍್ನಿಂಗ್ ಬಾಡಿಯಲ್ಲಿದ್ದರು. ಅವರೇ ಹೊರಗಿನ ಸಹಾಯದಿಂದ ಮಿಷನ್ನಿನ ಕಾರ್ಯಕ್ರಮಗಳನ್ನು ನಡೆಸಿಕೊಂಡು ಬಂದರು. ಈ ಬೇಲೂರ ಮಠವೇ ಕಾಲಕ್ರಮೇಣ ಇಂಡಿಯಾ ದೇಶದಲ್ಲೆಲ್ಲ ತನ್ನ ಶಾಖೆಗಳನ್ನು ಸ್ಥಾಪಿಸಿತು. ರಾಮಕೃಷ್ಣ ಮಿಷನ್ ಮತ್ತು ರಾಮಕೃಷ್ಣ ಮಠ ಇವೆರಡೂ ಒಂದೇ ಸಂಸ್ಥೆಯ ಕೆಳಗೆ ಬರುವುವು. ಶ‍್ರೀರಾಮಕೃಷ್ಣರ ಮಠ ಎನ್ನುವುದು ಆಧ್ಯಾತ್ಮಿಕ ಜೀವನ ಧರ್ಮ ಪ್ರಚಾರ ಮುಂತಾದವುಗಳ ಕಡೆಗೆ ಹೆಚ್ಚು ಗಮನ ಕೊಡುತ್ತದೆ. ಶ‍್ರೀರಾಮಕೃಷ್ಣ ಮಿಷನ್, ಸ್ಕೂಲು ಕಾಲೇಜು ಆಸ್ಪತ್ರೆ ಮುಂತಾದ ಸಾರ್ವಜನಿಕ ಕೆಲಸಗಳಿಗೆ ಹೆಚ್ಚು ಪ್ರಾಮುಖ್ಯತೆಯನ್ನು ಕೊಡುತ್ತದೆ. ಇವೆರಡರ ಕೇಂದ್ರವೂ ಈಗ ಬೇಲೂರು ಮಠದಲ್ಲಿದೆ. 

 ಸ್ವಾಮೀಜಿ ಇಚ್ಛೆ ನೇರವೇರಿತು. ಈಗ ಅವರು ಸ್ಥಾಪಿಸಿದ ಸಂಸ್ಥೆ ಭರತಖಂಡದಲ್ಲೆಲ್ಲ ವ್ಯಾಪಿಸಿದೆ. ಪಾಶ್ಚಾತ್ಯ ದೇಶಗಳಲ್ಲಿಯೂ ಹಬ್ಬಿದೆ. ಭರತಖಂಡದಲ್ಲಿ ಹಲವಾರು ಸಾರ್ವಜನಿಕ ಉಪಕಾರಾರ್ಥ ಸಂಸ್ಥೆಗಳನ್ನು ಯಶಸ್ವಿಯಾಗಿ ನಡೆಸಿಕೊಂಡು ಬರುತ್ತಿದೆ. ಹಿಂದೂಗಳ ಮೂಲಶಾಸ್ತ್ರಗಳನ್ನು ಸುಲಭವಾಗಿ ದೇಶೀಯ ಭಾಷೆಗಳಲ್ಲಿ ದೇಶದಲ್ಲೆಲ್ಲ ಪ್ರಚಾರ ಮಾಡುತ್ತಿದೆ. ಹಿಂದೆ ಯಾವ ವಿದ್ಯೆ ಎಲ್ಲೊ ಕೆಲವು ಪಂಡಿತರಿಗೆ ಮೀಸಲಾಗಿತ್ತೊ, ಅದನ್ನು ಜನಸಾಧಾರಣರೆಲ್ಲ ತಿಳಿದುಕೊಳ್ಳುವುದಕ್ಕೆ ಸಹಾಯ ಮಾಡಿದೆ. ಪರದೇಶಗಳಲ್ಲಿ ಉದಾರವಾದ ವೇದಾಂತಭಾವನೆಯ ಪ್ರಚಾರವನ್ನು ಮಾಡುತ್ತಿದೆ. ಇದರ ಉದ್ದೇಶ ಮತಾಂತರಗೊಳಿಸುವುದಲ್ಲ. ಆಯಾ ಮತಾನುಯಾಯಿಗಳು ಆಯಾ ಮತದಲ್ಲಿಯೇ ಇದ್ದುಕೊಂಡು ಇತರರನ್ನು ಗೌರವಿಸಬೇಕು, ತಮ್ಮಂತೆಯೇ ಒಂದೇ ಸತ್ಯದ ಕಡೆಗೆ ಎಲ್ಲರೂ ಪ್ರಯಾಣ ಮಾಡುತ್ತಿರುವರು ಎಂಬ ಉದಾರ ದೃಷ್ಟಿಯನ್ನು ಬೆಳಸಿಕೊಳ್ಳಬೇಕು - ಎಂಬುದು ಅದರ ಉದ್ದೇಶ. 

