
\chapter{ಸಮುದ್ರದ ಮೇಲೆ }

 ಸ್ವಾಮೀಜಿ ಇದುವರೆಗೆ ದಂಡಕಮಂಡಲುಧಾರಿಗಳಾಗಿ ಕೆಲವು ವೇಳೆ ನಡೆಯುತ್ತ ಕೆಲವು ವೇಳೆ ಭಕ್ತಾದಿಗಳು ಟಿಕೆಟ್ಟು ತೆಗೆದುಕೊಟ್ಟರೆ ವಾಹನಾದಿಗಳಲ್ಲಿ ಸಂಚರಿಸುತ್ತ ಇದ್ದರು. ಈಗ ಹಡಗಿನ ಮೊದಲನೆಯ ತರಗತಿಯಲ್ಲಿರುವರು. ಬಟ್ಟೆ ಬರೆ, ಇತರ ಸಾಮಾನುಗಳು ಈಗ ಬೇಕಾದಷ್ಟು ಇವೆ ಸ್ವಾಮಿಗಳ ಹತ್ತಿರ. ಅವುಗಳನ್ನು ನೋಡಿಕೊಳ್ಳುವುದೇ ಅವರಿಗೊಂದು ಭಾರವಾಯಿತು ಮೊದಲಲ್ಲಿ. ಅನಂತರ ಅದಕ್ಕೆ ಒಗ್ಗಿಕೊಂಡು ಹೋದರು. ಹಡಗಿನ ಹಲವರು ಸ್ವಾಮೀಜಿಯವರೊಡನೆ ಪರಿಚಯ ಮಾಡಿಕೊಂಡರು. ಹಡಗಿನ ಕ್ಯಾಪ್ಟನ್ ಸ್ವಾಮೀಜಿಯವರ ಸ್ನೇಹ ಮಾಡಿಕೊಂಡನು. ಆತ ಸ್ವಾಮೀಜಿಯವರಿಗೆ ಹಡಗಿನ ಭಾಗಗಳನ್ನೆಲ್ಲ ತೋರಿ ಮಧ್ಯ ಮಧ್ಯ ದಾರಿಯಲ್ಲಿ ಸಿಕ್ಕುವ ಪಟ್ಟಣಗಳು ದೃಶ್ಯಗಳು ಇವನ್ನು ವಿವರಿಸುತ್ತಿದ್ದನು. ಸ್ವಾಮೀಜಿ ಸುತ್ತಮುತ್ತಲ ದೃಶ್ಯಗಳನ್ನೆಲ್ಲ ನೋಡುತ್ತ ಅದನ್ನು ಮನದಲ್ಲಿಯೇ ವಿಚಾರ ಮಾಡುತ್ತ ಹೋದರು. ಸ್ವಾಮೀಜಿಯವರದು ತುಂಬಾ ಸುಸಂಸ್ಕೃತವಾದ ಜೀವ. ಅವರ ಮೇಲೆ ಘಟನೆಗಳು ವಸ್ತುಗಳು ಬಿದ್ದಾಗ ಅದರ ಪ್ರತಿಕ್ರಿಯೆ ಕಾವ್ಯದಂತೆ ಹರಿಯುತ್ತಿತ್ತು. ಅವರು ಅಮೇರಿಕಾ ಸೇರುವುದಕ್ಕೆ ಮುಂಚೆ ತಾವು ಸಮುದ್ರ ಪ್ರಯಾಣ ಮಾಡುತ್ತಿದ್ದಾಗ ಅದನ್ನು ವಿವರಿಸಿ ಬರೆದ ಪತ್ರದಿಂದಲೇ ನಾವು ಅದನ್ನು ವಿವರಿಸುವುದು ಮೇಲು: 

 “ಬೊಂಬಾಯಿನಿಂದ ಕೊಲಂಬೊ ಮುಟ್ಟಿದೆವು. ನಮ್ಮ ಜಹಜು ಸುಮಾರು ಹಗಲೆಲ್ಲ ಬಂದರಿನಲ್ಲಿತ್ತು. ಈ ಸಮಯದಲ್ಲಿ ಪಟ್ಟಣವನ್ನು ನೋಡುವುದಕ್ಕೆ ಹೊರಟೆವು. ಗಾಡಿ ಮಾಡಿಕೊಂಡು ರಸ್ತೆಯಲ್ಲಿ ಹೊರಟೆವು. ಒಂದು ಮಂದಿರ. ಅದರಲ್ಲಿ ದೊಡ್ಡ ಬುದ್ಧನ ವಿಗ್ರಹ. ಅದು ಒರಗಿಕೊಂಡು ನಿರ್ವಾಣ ಸ್ಥಿತಿಯಲ್ಲಿದೆ. ಇದೊಂದೇ ನನಗೆ ಜ್ಞಾಪಕ ಇರುವುದು.” 

 “ಮುಂದಿನ ನಿಲ್ದಾಣವೇ ಪಿನಾಂಗ್. ಇದು ಮಲಯ ಪರ್ಯಾಯದ್ವೀಪಕ್ಕೆ ಸೇರಿ, ಸಮುದ್ರ ಕರೆಗೆ ಅಂಟಿಕೊಂಡಿರುವ ಭೂಪ್ರದೇಶ. ಮಲಯ ವಾಸಿಗಳೆಲ್ಲರೂ ಮಹಮ್ಮದೀಯರು. ಹಿಂದಿನ ಕಾಲದಲ್ಲಿ ಅವರೆಲ್ಲ ಸಮುದ್ರದ ಕಳ್ಳರಾಗಿದ್ದರು. ವ್ಯಾಪಾರದ ಹಡುಗುಗಳಿಗೆ ಒಂದು ಗಂಡಾಂತರವಾಗಿದ್ದರು. ಆದರೆ ಈಗಿನ ಕಾಲದ, ಅನೇಕ ಅಂತಸ್ತುಗಳುಳ್ಳ ದೊಡ್ಡ ಯುದ್ಧದ ಹಡಗುಗಳ ಭಾರೀ ಫಿರಂಗಿಗಳು, ಮಲಯರನ್ನು ಶಾಂತಜೀವನೋಪಾಯವನ್ನು ಹುಡುಕುವಂತೆ ಮಾಡಿವೆ. ಪಿನಾಂಗಿನಿಂದ ಸಿಂಗಪುರಕ್ಕೆ ಹೋಗುವ ದಾರಿಯಲ್ಲಿ ಎತ್ತರವಾದ ಬೆಟ್ಟಗಳಿಂದ ಕೂಡಿದ ಸುಮಾತ್ರ ದ್ವೀಪ ಕೊಂಚ ಕೊಂಚ ಕಾಣುತ್ತಿತ್ತು. ಜಹಜಿನ ನಾಯಕನು ಅನೇಕ ಸ್ಥಳಗಳನ್ನು ತೋರಿಸಿ, ಅವುಗಳು ಹಿಂದೆ ಸಮುದ್ರಕಳ್ಳರ ವಾಸಸ್ಥಾನವಾಗಿತ್ತೆಂದು ಹೇಳಿದನು. ಸ್ಟ್ರೈಟ್ಸ್ ಸೆಟ್ಲಿಮೆಂಟಿಗೆ ಸಿಂಗಪುರ ರಾಜಧಾನಿ. ಇಲ್ಲಿ ತಾಳೇ ಜಾತಿಗೆ ಸೇರಿದ, ಅನೇಕ ವಿಧದ ಸುಂದರ ಮರಗಳನ್ನೊಳಗೊಂಡ ವನಸ್ಪತಿಯ ಉದ್ಯಾನವಿದೆ. ‘ಯಾತ್ರಿಕರ ತಾಳೆ’ ಎಂದು ಕರೆಯುವ ಸುಂದರವಾದ ಬೀಸಣಿಗೆಯಂತಿರುವ ಮರ ಇಲ್ಲಿ ಯಥೇಚ್ಛವಾಗಿ ಬೆಳೆಯುವುದು. ಎಲ್ಲೆಲ್ಲಿಯೂ ಬ್ರೆಡ್ ಫ್ರೂಟ್ ಮರ ಬೆಳೆಯುವುದು. ಪ್ರಖ್ಯಾತವಾದ ಮ್ಯಾಂಗೋಸ್ಪೀನ್ ಎಂಬುದು ಇಲ್ಲಿ ಮದ್ರಾಸಿನಲ್ಲಿ ಬೆಳೆಯುವ ಮಾವಿನ ಗಿಡದಂತೆ ಅಧಿಕವಾಗಿದೆ. ಆದರೆ ಮಾವಿಗೆ ಸಮನಲ್ಲ ಇದು. ಇಲ್ಲಿಯ ಜನರು ವಿಷುವದ್ರೇಖೆಗೆ ಸಮೀಪದಲ್ಲಿದ್ದರೂ ಮದ್ರಾಸಿನ ಜನರ ಅರ್ಧದಷ್ಟೂ ಕೂಡ ಕಪ್ಪಾಗಿಲ್ಲ. ಸಿಂಗಪುರದಲ್ಲಿ ಒಂದು ಸೊಗಸಾದ ವಸ್ತುಪ್ರದರ್ಶನ ಶಾಲೆಯೂ ಇದೆ.” 

 “ಮುಂದೆ ಹಾಂಕಾಂಗ್ ಮನಸ್ಸಿಗೆ ಚೈನಾವನ್ನು ಮುಟ್ಟಿದಂತೆಯೇ ಕಾಣುವುದು. ಇಲ್ಲಿ ಚೀಣರು ತುಂಬಾ ಪ್ರಾಬಲ್ಯದಲ್ಲಿರುವರು. ಉದ್ಯೋಗ ವ್ಯಾಪಾರಗಳೆರಡೂ ಇವರ ಕೈಯಲ್ಲಿರುವಂತೆ ಕಾಣುವುದು. ಹಾಂಕಾಂಗ್ ನಿಜವಾಗಿ ಚೀಣ. ಹಡಗು ಲಂಗರು ಹಾಕಿದೊಡನೆಯೇ ನೂರಾರು ಚೀಣಿ ದೋಣಿಯವರು ಕರೆಗೆ ಒಯ್ಯಲು ಮುತ್ತುವರು. ಎರಡು ಚುಕ್ಕಾಣಿಗಳುಳ್ಳ ದೋಣಿಗಳು ನೋಡಲು ವಿಚಿತ್ರವಾಗಿವೆ. ದೋಣಿಯವರು ಸಂಸಾರ ಸಮೇತ ದೋಣಿಯಲ್ಲೇ ವಾಸಮಾಡುವರು. ಸಾಮಾನ್ಯವಾಗಿ ಹೆಂಡತಿ ಯಾವಾಗಲೂ ಚುಕ್ಕಾಣಿಯ ಹತ್ತಿರ ಕೆಲಸಮಾಡುವಳು. ಒಂದನ್ನು ಕೈಯಿಂದಲೂ ಮತ್ತೊಂದನ್ನು ಕಾಲಿನಿಂದಲೂ ನಡೆಸುವಳು. ಶೇಕಡ ತೊಂಬತ್ತು ಮಂದಿ ಹೆಂಗಸರು ಬೆನ್ನಿಗೆ ಕೈಕಾಲುಗಳನ್ನು ಇಳಿಬಿಟ್ಟಿರುವ ಮಗುವನ್ನು ಕಟ್ಟಿಕೊಂಡಿರುವರು. ತಾಯಿ ಭಾರವಾದ ಮೂಟೆಯನ್ನು ತನ್ನ ಬಲವನ್ನೆಲ್ಲ ಬಿಟ್ಟು ನೂಕುವಾಗಲೂ, ದೋಣಿಯಿಂದ ದೋಣಿಗೆ ಆಶ್ಚರ್ಯಕರವಾಗಿ ನೆಗೆಯುವಾಗಲೂ, ಆಕೆಯ ಬೆನ್ನಿಗೆ ಪುಟ್ಟ ಚೀಣೀಯನು ಅಂಟಿಕೊಂಡಿರುವುದು ಒಂದು ವಿಚಿತ್ರ. ಬರುತ್ತಲೂ ಹೋಗುತ್ತಲೂ ಇರುವ ದೋಣಿಗಳ ಮತ್ತು ಜಹಜುಗಳ ನುಗ್ಗಾಟವೆಷ್ಟು! ಪುಟ್ಟ ಚೀಣೀಯನ ಎರಡೂ ಪಕ್ಕದಲ್ಲಿಯೂ ಅವನ ತಲೆಯನ್ನು ಜಡೆಸಹಿತ ನುಚ್ಚುನೂರಾಗಿಸುವ ಅಪಾಯ ಯಾವಾಗಲೂ ಇರುತ್ತದೆ. ಆದರೂ ಅದನ್ನು ಅವನು ಲಕ್ಷಿಸುವುದೇ ಇಲ್ಲ. ಈ ಗಡಿಬಿಡಿಯ ಜೀವನಕ್ಕೆ ಅವನು ಮನಸ್ಸನ್ನು ಕೊಡುವಂತೆಯೇ ಇಲ್ಲ. ಹುಚ್ಚು ಹಿಡಿದಂತೆ ಕೆಲಸ ಮಾಡುವ ತಾಯಿ ಮಗುವಿಗೆ ಆಗಿಂದಾಗ್ಗೆ ಕೊಡುವ ರೊಟ್ಟಿಯ ಚೂರಿನ ರುಚಿಯನ್ನು ನೋಡುವುದರಲ್ಲಿಯೇ ಅವನಿಗೆ ತೃಪ್ತಿ. ಚೀಣೀಯ ಮಗ ಹುಟ್ಟು ತತ್ತ್ವಜ್ಞಾನಿ. ಇಂಡಿಯಾದೇಶದಲ್ಲಿ ಮಗು ಅಂಬೆಗಾಲಿಡುವ ವಯಸ್ಸಿನಲ್ಲೇ ಅದು ಕೆಲಸಕ್ಕೆ ಹೋಗುವುದು. ದಾರಿದ್ರ್ಯದ ತತ್ತ್ವವನ್ನು ಚೆನ್ನಾಗಿ ಅರಿತ ಕೂಸು ಅದು. ಚೀಣಾ ದೇಶೀಯರು ಮತ್ತು ಇಂಡಿಯಾ ದೇಶಿಯರು ಇನ್ನೂ ಅರ್ಧ ನಾಗರಿಕ ಸ್ಥಿತಿಯಲ್ಲಿ ಉಳಿದಿರುವುದಕ್ಕೆ ಅವರ ಬಡತನವೇ ಕಾರಣ. ಸಾಧಾರಣ ಹಿಂದೂ ಮತ್ತು ಚೀಣೀ ದೇಶೀಯನಿಗೆ ಪ್ರತಿದಿನದ ಜೀವನದ ಅಗತ್ಯ, ಉಳಿದ ಎಲ್ಲಾ ವಿಷಯಗಳನ್ನೂ ಹಿಂದಕ್ಕೆ ತಳ್ಳುವಷ್ಟು ಘೋರವಾಗಿದೆ.” 

”ಹಾಂಕಾಂಗ್ ಬಹಳ ಸುಂದರ ನಗರ. ಬೆಟ್ಟದ ಇಳಿಜಾರಿನ ಮೇಲೆ, ತಂಪಾಗಿ ಇರುವ ಶಿಖರದ ಮೇಲೆ ಪಟ್ಟಣ ಇರುವುದು. ತಂತಿಯ ಹಗ್ಗದಿಂದಲೂ, ಉಗಿಯ ಯಂತ್ರದಿಂದಲೂ ಬೆಟ್ಟದ ತಳದಿಂದ ಹಿಡಿದು ಶಿಖರದವರೆಗೆ ನೇರವಾಗಿ ಹೋಗುವ ಟ್ರಾಂ ಗಾಡಿಗಳೂ ಇವೆ.” 

 “ಹಾಂಕಾಂಗಿನಲ್ಲಿ ಮೂರು ದಿನಗಳು ಇದ್ದೆವು. ಅನಂತರ ನದಿಯ ಮೇಲೆ ಎಂಭತ್ತು ಮೈಲಿಗಳು ದೂರದಲ್ಲಿರುವ ಕ್ಯಾಂಟನ್ ನಗರವನ್ನು ನೋಡಲು ಹೋದೆವು. ಎಷ್ಟು ಗದ್ದಲ ಮತ್ತು ಜನಸಂದಣಿಯ ದೃಶ್ಯ! ನದಿಯ ನೀರನ್ನೆಲ್ಲ ಮುಚ್ಚಿಕೊಂಡಂತೆ ಕಾಣುವ ಅಸಂಖ್ಯಾತ ದೋಣಿಗಳೆಷ್ಟು! ವ್ಯಾಪಾರಕ್ಕೆ ಸಂಬಂಧಪಟ್ಟ ದೋಣಿಗಳು ಮಾತ್ರವಲ್ಲ. ನೂರಾರು ವಾಸಮಾಡುವುದಕ್ಕೆ ತಕ್ಕ ಮನೆಗಳಂತಿವೆ ದೋಣಿಗಳು. ಅನೇಕ ದೋಣಿಗಳು ಸುಂದರವಾಗಿ ವಿಶಾಲವಾಗಿವೆ. ಎರಡು ಮೂರು ಅಂತಸ್ತುಗಳುಳ್ಳ ದೊಡ್ಡ ಮನೆಗಳು ಆ ದೋಣಿಗಳು. ಸುತ್ತಲೂ ವರಾಂಡ, ಅದರ ಮಧ್ಯೆ ರಸ್ತೆ; ಎಲ್ಲಾ ತೆಪ್ಪೋತ್ಸವ!” 

 “ಚೀಣಾ ಸರ್ಕಾರವನ್ನು ಪರದೇಶೀಯರ ವಾಸಕ್ಕೆ ಕೊಟ್ಟ ಸ್ಥಳದ ಮೇಲೆ ಕಾಲಿಟ್ಟೆವು. ನಮ್ಮ ಸುತ್ತಲೂ ನದಿಯ ಎರಡು ದಡಗಳ ಮೇಲೆ, ಅನೇಕ ಮೈಲಿಗಳಷ್ಟು ವಿಸ್ತಾರವಾಗಿ ಹಬ್ಬಿರುವುದು ಈ ಕ್ಯಾಂಟನ್ ನಗರ. ಅಬ್ಬ ಅದು ಏನು ನೂಕಾಟ! ಏನು ಹೋರಾಟ! ಏನು ಎಳೆದಾಟ! ಇವುಗಳಿಂದ ಅಲ್ಲೋಲಕಲ್ಲೋಲವಾದ ದೊಡ್ಡ ಪಟ್ಟಣ ಅದು. ಅಷ್ಟು ಜನಸಂದಣಿ ಇದ್ದರೂ, ಅಷ್ಟು ಉದ್ಯೋಗವಿದ್ದರೂ ನಾನು ನೋಡಿದ ಪಟ್ಟಣಗಳಲ್ಲೆಲ್ಲ ಹೊಲಸು ಪಟ್ಟಣ ಅದು. ಇಂಡಿಯಾ ದೇಶದಲ್ಲಿ ಗಲೀಜು ಎಂದು ಹೇಳುವ ದೃಷ್ಟಿಯಿಂದ ಅಲ್ಲ. ಆ ದೃಷ್ಟಿಯಿಂದ ನೋಡಿದರೆ ಅವರು ಒಂದು ಚೂರು ಕಸವನ್ನೂ ಪೋಲು ಮಾಡುವಂತೆ ಇಲ್ಲ. ಆದರೆ ಚೀಣಾ ದೇಶೀಯರು ತಾವು ಸ್ನಾನ ಮಾಡುವುದಿಲ್ಲ ಎಂದು ಶಪಥವನ್ನು ಮಾಡಿದಂತೆ ತೋರುವುದು. ಪ್ರತಿಯೊಂದು ಮನೆಯೂ ಒಂದು ಅಂಗಡಿ. ಮಹಡಿಯ ಮೇಲೆ ಮಾತ್ರ ಮನೆಯವರು ವಾಸಮಾಡುವರು. ರಸ್ತೆಗಳು ಬಲು ಇಕ್ಕಟ್ಟು. ಆದುದರಿಂದ ನಾವು ಹೋಗುವಾಗ ಎದುರು ಬದುರಿಗಿರುವ ಅಂಗಡಿಯನ್ನು ಮುಟ್ಟಿಕೊಂಡು ಹೋದಂತೆ ಕಾಣುವುದು. ಪ್ರತಿ ಹತ್ತು ಹೆಜ್ಜೆಗೂ ಒಂದೊಂದು ಮಾಂಸದ ಅಂಗಡಿ, ಬೆಕ್ಕು ನಾಯಿಗಳ ಮಾಂಸವನ್ನು ಕೂಡ ಮಾರುವ ಅಂಗಡಿಗಳಿವೆ. ಆದರೆ ಅತಿ ಬಡವರು ಮಾತ್ರ ಅದನ್ನು ತಿನ್ನುತ್ತಾರೆ. ಉತ್ತಮ ದರ್ಜೆಯ ಚೀಣೀ ಹೆಂಗಸರನ್ನು ಯಾವ ಸಮಯದಲ್ಲಿಯೂ ನೋಡಲಾಗುವುದಿಲ್ಲ. ಉತ್ತರ ಹಿಂದೂಸ್ತಾನದಷ್ಟೇ ಘೋಷಾ ಪದ್ಧತಿ ಇಲ್ಲಿ ಕಟ್ಟು ನಿಟ್ಟು. ಕೂಲಿಯ ಹೆಂಗಸರನ್ನು ಮಾತ್ರ ನೋಡಬಹುದು. ಅವರಲ್ಲಿಯೂ ಕೆಲವು ವೇಳೆ ಸಣ್ಣ ಮಗುವಿಗಿಂತಲೂ ಚಿಕ್ಕದಾದ ಪಾದಗಳುಳ್ಳ ಹೆಂಗಸರನ್ನು ನೋಡಬಹುದು. ಅವರನ್ನು ನಡೆಯುತ್ತಾರೆ ಎನ್ನಲಾಗುವುದಿಲ್ಲ, ಕುಪ್ಪಳಿಸುತ್ತಾರೆ ಎನ್ನಬಹುದು.” 

”ಅನೇಕ ಚೀಣಾದ ದೇವಾಲಯವನ್ನು ನಾವು ನೋಡಲು ಹೋದೆವು. ಅವುಗಳಲ್ಲಿ ಅತಿ ದೊಡ್ಡದಾಗಿದ್ದುದು, ಮೊದಲನೆಯ ಬೌದ್ಧ ಚಕ್ರವರ್ತಿ ಮತು ಬೌದ್ಧ ಮತಗಳನ್ನು ಅವಲಂಬಿಸಿದ, ಮೊದಲ ಐದುನೂರು ಶಿಷ್ಯರ ಜ್ಞಾಪಕಾರ್ಥವಾಗಿ ಕಟ್ಟಿರುವುದು. ಅದರಲ್ಲಿ ಮುಖ್ಯವಾದ ವಿಗ್ರಹವೇನೋ ಬುದ್ಧನದು. ಅವನ ಕೆಳಗೆ ಚಕ್ರವರ್ತಿಯದು, ಎರಡು ಪಕ್ಕದಲ್ಲಿಯೂ ಶಿಷ್ಯರದು. ಎಲ್ಲವೂ ಮರದಿಂದ ಚೆನ್ನಾಗಿ ಕೆತ್ತಿದ ವಿಗ್ರಹಗಳು!” 

 ಅಲ್ಲಿ ಬಂಗಾಳಿ ಲಿಪಿಯಲ್ಲಿ ಬರೆದ ಹಲವು ಸಂಸ್ಕೃತ ಶ್ಲೋಕಗಳಿದ್ದವು. ಬುದ್ಧನ ಶಿಷ್ಯರೂ ಕೂಡಾ ಬಂಗಾಳ ದೇಶಿಯರಂತೆಯೇ ಇದ್ದರು. ಇದರಿಂದ ಬಂಗಾಳ ಮತ್ತು ಚೈನಾ ದೇಶಗಳಿಗೆ ಅತಿ ಹಿಂದಿನ ಕಾಲದಲ್ಲಿ ಒಂದು ನಿಕಟ ಸಂಬಂಧ ನೋಡಬಯಸಿದರು. ಆದರೆ ಪರದೇಶಿಯರಾರನ್ನೂ ಒಳಗೆ ಬಿಡುತ್ತಿರಲಿಲ್ಲ. ಸ್ವಾಮೀಜಿ ಮಾರ್ಗದರ್ಶಕನನ್ನು ತಮ್ಮನ್ನು ಮತ್ತು ಇನ್ನು ಕೆಲವರನ್ನು ಒಳಗೆ ಕರೆದುಕೊಂಡು ಹೋಗು, ನಾವು ಏನಾದರೂ ಸಹಿಸುತ್ತೇವೆ ಎಂದರು. ಆತ ವಿಧಿಯಿಲ್ಲದೆ ಒಳಗೆ ಅವರನ್ನು ಕರೆದುಕೊಂಡು ಹೋದನೋ ಇಲ್ಲವೊ ಅಷ್ಟು ಹೊತ್ತಿಗೆ ಕೋಪಾನಲವನ್ನು ಕಾರುತ್ತ ಒಬ್ಬ ಭಿಕ್ಷು ಇವರ ಕಡೆಗೆ ಓಡಿ ಬರುತ್ತಿದ್ದನು. ಈ ದೃಶ್ಯವನ್ನು ಕಂಡೊಡನೆಯೇ ಎಲ್ಲರೂ ಓಡಿದರು. ಮಾರ್ಗದರ್ಶಕನೂ ಓಡಿಹೋಗುವುದರಲ್ಲಿದ್ದ. ಆದರೆ ಸ್ವಾಮೀಜಿ ಅವನ ಕೈಯನ್ನು ಬಲವಾಗಿ ಹಿಡಿದು ನಿಲ್ಲಿಸಿ ಕೋಪದಿಂದ ಬರುತ್ತಿದ್ದ ಭಿಕ್ಷುವಿಗೆ, ತಾನು ಇಂಡಿಯಾ ದೇಶದಿಂದ ಬಂದ ಸಂನ್ಯಾಸಿ ಎಂದು ಹೇಳು ಎಂದರು. ಆತ ಹಾಗೆ ಹೇಳಿದೊಡನೆಯೇ ಭಿಕ್ಷು ಇವರಿಗೆ ಗೌರವದಿಂದ ಬಾಗಿ ‘ಕವಚ’ ಕೊಡಿ ಎಂದ. ಎಂದರೆ ‘ಯಂತ್ರ’, ಭೂತಪ್ರೇತಗಳಿಂದ ಪಾರಾಗುವುದಕ್ಕೆ. ದೇವರನ್ನೇ ನಂಬದ ಬೌದ್ಧರಿಗೆ ದೆವ್ವ ಪೀಡೆಗಳ ನಂಬಿಕೆ ಬಲವಾಗಿದೆ. ಸ್ವಾಮೀಜಿ ತಮ್ಮ ಜೇಬಿನಿಂದ ಒಂದು ಚೂರು ಕಾಗದವನ್ನು ತೆಗೆದು ಅದನ್ನು ಹರಿದು ಅದರ ಮೇಲೆ ಸಂಸ್ಕೃತದಲ್ಲಿ ‘ಓಂ’ ಎಂದು ಬರೆದುಕೊಟ್ಟರು. ಆತ ಅದನ್ನು ಕಿಸೆಯಲ್ಲಿಟ್ಟುಕೊಂಡು ಸ್ವಾಮೀಜಿಗೆ ಮಠವನ್ನೆಲ್ಲ ತೋರಿ ಕಳುಹಿಸಿದನು. 

 ಸ್ವಾಮೀಜಿ ಮುಂದುವರೆಸುತ್ತಾರೆ:

 “ಕಾಂಟನ್‍ನಿಂದ ಪುನಃ ಹಾಂಕಾಂಗಿಗೆ ಬಂದೆವು. ಅಲ್ಲಿಂದ ಜಪಾನಿಗೆ ಹೋದೆವು. ನಾವು ಸೇರಿದ ಮೊದಲನೇ ರೇವುಪಟ್ಟಣವೇ ನಾಗಸಾಕಿ. ಕೆಲವು ಗಂಟೆಗಳ ಕಾಲ ಅಲ್ಲಿದ್ದು ಪಟ್ಟಣವನ್ನು ನೋಡಲು ಗಾಡಿಯಲ್ಲಿ ಹೊರಟೆವು. ಏನು ವ್ಯತ್ಯಾಸ! ಜಪಾನೀಯರು ಜಗತ್ತಿನಲ್ಲಿ ಅತ್ಯಂತ ಶುಚಿಯಾದ ಜನಾಂಗದಲ್ಲಿ ಒಬ್ಬರು. ಇಲ್ಲಿ ಪ್ರತಿಯೊಂದೂ ಶುಚಿಯಾಗಿದೆ, ಚೊಕ್ಕಟವಾಗಿದೆ. ಇಲ್ಲಿರುವ ರಸ್ತೆಗಳು ವಿಶಾಲವಾಗಿವೆ. ಮೇಲೆ ಕಲ್ಲು ಹಾಸಿದೆ. ಅವರ ಸಣ್ಣ ಮನೆಗಳು ಹಕ್ಕಿಯ ಪಂಜರದಂತಿವೆ. ಯಾವಾಗಲೂ ಹಸುರಾಗಿರುವ ದೇವದಾರು ಮರಗಳು, ನಿಬಿಡವಾದ ಬೆಟ್ಟಗಳು, ಸಾಮಾನ್ಯವಾಗಿ ಪ್ರತಿಯೊಂದು ಹಳ್ಳಿಯ ಮತ್ತು ಊರಿನ ಹಿನ್ನೆಲೆಯಾಗಿವೆ. ಜಪಾನರು ಕುಳ್ಳರು. ಅವರ ದೇಹದ ಬಣ್ಣ ಸುಂದರವಾಗಿದೆ. ಉಟ್ಟ ಬಟ್ಟೆ ವಿಚಿತ್ರವಾಗಿದೆ. ಜಪಾನ್ ದೇಶವೇ ಸುಂದರ ವಿಚಿತ್ರ ದೃಶ್ಯಗಳ ತೌರೂರು. ಪ್ರತಿಯೊಂದು ಮನೆಯ ಹಿಂದೆಯೂ ಒಂದು ಸುಂದರ ತೋಟವಿದೆ. ಜಪಾನೀಯರ ರೀತಿಯಂತೆ ಅಲ್ಲಿ ಸಣ್ಣ ಗಿಡಗಳು ಹುಲ್ಲುಮಡಿಗಳು ಕೃತಕ ಜಲಾಶಯಗಳು, ಅದರ ಮೇಲೆ ಸಣ್ಣ ಕಲ್ಲಿನ ಸೇತುವೆಗಳು ಇವುಗಳಿಂದ ಅಲಂಕರಿಸಿರುವರು.” 

 “ನಾಗಸಾಕಿಯಿಂದ ಕೋಬಿಗೆ ಹೋದೆವು. ಅಲ್ಲಿ ನಾನು ಜಹಜನ್ನು ಬಿಟ್ಟು ಜಪಾನ್ ದೇಶದ ಒಳ ಭಾಗವನ್ನು ನೋಡಬೇಕೆಂದು ಯಾಕೋಹಾಮಕ್ಕೆ ಭೂಮಾರ್ಗವನ್ನು ಹಿಡಿದೆನು. ಒಳ ಸೀಮೆಯಲ್ಲಿ ಮೂರು ದೊಡ್ಡ ಪಟ್ಟಣಗಳನ್ನು ನೋಡಿದೆನು. ಒಸಾಕ ದೊಡ್ಡ ಕೈಗಾರಿಕೆಯ ಸ್ಥಳ. ಕಿಯೋಟೋ ಹಿಂದೆ ರಾಜಧಾನಿಯಾಗಿತ್ತು. ಟೋಕಿಯೋ ಇಂದಿನ ರಾಜಧಾನಿ. ಕಲ್ಕತ್ತೆಗೆ ಎರಡರಷ್ಟಿದೆ. ರಹದಾರಿಯಿಲ್ಲದೆ ಯಾವ ಪರದೇಶಿಯರಿಗೂ ಒಳಗೆ ಸಂಚಾರ ಮಾಡಲು ಅವಕಾಶವಿಲ್ಲ. ಜಪಾನೀಯರು ಆಧುನಿಕ ಕಾಲದ ಅವಶ್ಯಕತೆಗೆ ಸಂಪೂರ್ಣ ಜಾಗ್ರತರಾಗಿರುವರು. ಈಗ ಅವರಿಗೆ ಒಂದು ಸುಶಿಕ್ಷಿತ ಸೈನ್ಯವಿದೆ. ಅವರ ಒಬ್ಬ ಸೈನ್ಯಾಧಿಕಾರಿಯೇ ಕಂಡುಹಿಡಿದ ಫಿರಂಗಿಗಳಿಂದ ಸುಸಜ್ಜಿತರಾಗಿರುವರು. ಇದು ಯಾವ ಪರದೇಶೀಯ ನಿರ್ಮಾಣಕ್ಕೂ ಕೀಳಲ್ಲ. ಅವರು ತಮ್ಮ ನೌಕಾಪಡೆಯನ್ನು ಕ್ರಮೇಣ ಹೆಚ್ಚಿಸುತ್ತಲೆ ಇರುವರು. ಒಬ್ಬ ಜಪಾನ್ ಶಿಲ್ಪಜ್ಞನೇ ಕೊರೆದ ಒಂದು ಮೈಲಿ ದೂರದ ಸುರಂಗವನ್ನು ನಾನು ನೋಡಿದೆನು.” 

 “ಇಲ್ಲಿನ ಬೆಂಕಿಯ ಕಡ್ಡಿಯ ಕಾರ್ಖಾನೆಗಳು ನೋಟಕ್ಕೆ ಒಂದು ಹಬ್ಬ. ಅವರು ತಮಗೆ ಬೇಕಾಗುವ ಸಾಮಾನುಗಳನ್ನೆಲ್ಲ ತಮ್ಮ ದೇಶದಲ್ಲೇ ತಯಾರಿಸುತ್ತೇವೆ ಎಂದು ಹಟತೊಟ್ಟಿರುವರು. ಚೀಣ ಮತ್ತು ಜಪಾನುಗಳ ನಡುವೆ ಆಗಲೆ ಜಪಾನರ ಹಡಗುಗಳ ಸಾಲು ಓಡುತ್ತಿದೆ. ಶೀಘ್ರದಲ್ಲಿಯೇ ಯೋಕೋಹಾಮ ಮತ್ತು ಬೊಂಬಾಯಿಗೆ ವಿಸ್ತರಿಸುವ ಯೋಜನೆಯೂ ಇದೆ.” 

 “ಇಲ್ಲಿ ಅನೇಕ ಗುಡಿಗಳನ್ನು ನಾನು ನೋಡಿದೆನು. ಪ್ರತಿಯೊಂದು ಗುಡಿಯಲ್ಲಿಯೂ ಹಳೆಯ ಬೆಂಗಾಳಿ ಲಿಪಿಯಿಂದ ಬರೆದ ಸಂಸ್ಕೃತ ಮಂತ್ರಗಳಿವೆ. ಎಲ್ಲೊ ಕೆಲವು ಪೂಜಾರಿಗಳಿಗೆ ಮಾತ್ರ ಇಲ್ಲಿ ಸಂಸ್ಕೃತ ಗೊತ್ತಿದೆ. ಆದರೂ ಅವರು ಬುದ್ಧಿ ಶಾಲಿಗಳು. ಮುಂದುವರಿಯಬೇಕೆಂಬ ಉತ್ಸಾಹ ಭಿಕ್ಷುಗಳಿಗೂ ಹಿಡಿದಿರುವುದು! ಜಪಾನೀಯರ ವಿಷಯದಲ್ಲಿ ನನ್ನ ಅಭಿಪ್ರಾಯವನ್ನೆಲ್ಲ ಒಂದು ಸಣ್ಣ ಪತ್ರದಲ್ಲಿ ಬರೆಯಲಾಗುವುದಿಲ್ಲ. ನಮ್ಮ ದೇಶದ ಹಲವು ಯುವಕರು ಜಪಾನಿಗೆ ಭೇಟಿ ಕೊಡಬೇಕು ಎಂಬುದೇ ನನ್ನಾಸೆ. ಉತ್ತಮವೂ ಭವ್ಯವೂ ಆದ ಆದರ್ಶಗಳಿಗೆ ಭಾರತ ತೌರೂರು ಎಂದು ಜಪಾನೀಯರು ತಿಳಿದಿರುವರು. ಆದರೆ ನೀವು ಏನಾಗಿದ್ದೀರಿ? ಒಣ ಹರಟೆಯಲ್ಲಿ ಜೀವಮಾನವನ್ನೆಲ್ಲ ಕಳೆಯುವಿರಿ. ಬರಿಯ ಮಾತಾಳಿಗಳು ನೀವು. ಬನ್ನಿ, ಈ ಜನರನ್ನು ನೋಡಿ ನಿಮ್ಮ ದೇಶಕ್ಕೆ ಹಿಂತಿರುಗಿ ನಾಚಿಕೆಯಿಂದ ಮುಖವನ್ನು ಮುಚ್ಚಿಕೊಳ್ಳಿ. ನೀವೆಲ್ಲರೂ ಅರಳು ಮರಳು ಮುದುಕರು. ಹೊರಗೆ ಬಂದರೆ ಜಾತಿ ಕೆಡುವುದು! ನೂರಾರು ವರ್ಷಗಳಿಂದ ಎಡೆಬಿಡದೆ ಹೆಚ್ಚು ಬೆಳೆದು ಕಲ್ಲಾದ ಮೂಢ ನಂಬಿಕೆಯ ಹೊರೆಯನ್ನು ತಲೆಯಮೇಲೆ ಹೊತ್ತಿರುವರು. ನೂರಾರು ವರ್ಷಗಳಿಂದ ಈ ಆಹಾರ ಮುಟ್ಟಲು ಯೋಗ್ಯವೋ, ಆ ಆಹಾರ ಮುಟ್ಟಲು ಯೋಗ್ಯವೋ ಎಂಬ ಘನ ವಿಚಾರವನ್ನು ವಿಮರ್ಶಿಸುವುದರಲ್ಲೆ ನಿಮ್ಮ ಬುದ್ಧಿವಂತಿಕೆಯನ್ನು ವ್ಯಯ ಮಾಡಿರುವಿರಿ. ಅನೇಕ ಶತಮಾನಗಳಿಂದ ಸಮಾಜವನ್ನು ಸತತ ಪೀಡಿಸುವುದರಲ್ಲಿ ನಿಮ್ಮ ಪುರುಷತ್ವವೆಲ್ಲ ವ್ಯಯವಾಗಿದೆ. ನೀವು ಈಗ ಏನಾಗಿದ್ದೀರಿ? ಯಾವ ಮಹಾ ಕಾರ್ಯವನ್ನು ಮಾಡುತ್ತಿದ್ದೀರಿ? ಪುಸ್ತಕವನ್ನು ಕೈಯಲ್ಲಿ ಹಿಡಿದುಕೊಂಡು ಸಮುದ್ರದ ದಡದ ಮೇಲೆ ಓಡುತ್ತಿರುವುದು- ಐರೋಪ್ಯರು ಬರೆದ ಕೆಲವು ವಿಷಯಗಳನ್ನು ಓದುವುದು. ಅದನ್ನು ಅರಗಿಸಿಕೊಳ್ಳದೆ ಅರಗಿಳಿಯಂತೆ ಮಾತಾಡುವಿರಿ. ಚಾಕರಿಗೆ ಷಿಕಾರಿ ಮಾಡುವುದು, ಹೆಚ್ಚೆಂದರೆ ಒಬ್ಬ ವಕೀಲನಾಗುವುದು - ಇವುಗಳಲ್ಲೆ ನಿಮ್ಮ ಬಾಳು ಮುಗಿಯುತ್ತದೆ. ಇದೇ ಆಧುನಿಕ ಭಾರತೀಯನ ಪರಮಾವಧಿ. ಹೊಟ್ಟೆಗಿಲ್ಲದೆ ಅಳುತ್ತ, ಹಿಟ್ಟಿಗೆ ಕಾಡುವ ಮಕ್ಕಳ ಪಡೆ, ಪ್ರತಿಯೊಬ್ಬ ವಿದ್ಯಾರ್ಥಿಯ ಹಿಂದೆ! ನಿಮ್ಮನ್ನೂ, ನಿಮ್ಮ ಪುಸ್ತಕಗಳನ್ನೂ ಗೌರವದ ನಿಲುವಂಗಿಗಳನ್ನೂ ನಿಮ್ಮ ಯೋಗ್ಯತಾ ಪತ್ರಗಳನ್ನೂ ಮುಳುಗಿಸುವಷ್ಟು ನೀರಿಲ್ಲವೆ ಬಂಗಾಳಕೊಲ್ಲಿಯಲ್ಲಿ!” 

 “ಮುಂದೆ ಬನ್ನಿ, ಪುರುಷಸಿಂಹರಾಗಿ! ಪುರೋಭಿವೃದ್ಧಿಗೆ ಯಾವಾಗಲೂ ಕಂಟಕ ಪ್ರಾಯರಾದ ಪುರೋಹಿತರನ್ನು ಹೊಡೆದಟ್ಟಿ. ಅವರೆಂದಿಗೂ ತಿದ್ದಿಕೊಳ್ಳಲಾರರು. ಅವರ ಹೃದಯ ವಿಶಾಲವಾಗಲಾರದು. ಅನೇಕ ಶತಮಾನಗಳಿಂದ ಬಂದ ಮೂಢ ನಂಬಿಕೆ ಮತ್ತು ಕಿರುಕುಳದ ಮಕ್ಕಳವರು. ಮೊದಲು ಅವರನ್ನು ಆಚೆಗೆ ಅಟ್ಟಿ. ಬನ್ನಿ ಮುಂದೆ, ಧೀರರಾಗಿ ಸಣ್ಣ ಬಿಲಗಳಿಂದ ಹೊರಗೆ ಬನ್ನಿ, ವಿಶಾಲವಾದ ಜಗತ್ತನ್ನು ಸುತ್ತಲೂ ನೋಡಿ. ಇತರ ಜನಾಂಗಗಳು ಹೇಗೆ ಮುಂದುವರಿಯುತ್ತಿವೆ ಎಂಬುದನ್ನು ನೋಡಿ. ನೀವು ಮಾನವರನ್ನು ಪ್ರೀತಿಸುವಿರೇನು? ಹಾಗಾದರೆ ಬನ್ನಿ, ಉತ್ತಮವಾದ ಭವ್ಯಜೀವನ ಆದರ್ಶಗಳಿಗೆ ಹೋರಾಡುವ. ನಿಮ್ಮ ಪ್ರಿಯ ಸ್ನೇಹಿತರು ಹತ್ತಿರದ ಬಂಧುಗಳು ಕಂಬನಿಗರೆದರೂ ತಿರುಗಿ ನೋಡದಿರಿ, ಎಂದಿಗೂ ಹಿಂತಿರುಗದಿರಿ. ಯಾವಾಗಲೂ ಮುಂದೆ, ಮುಂದೆ.” 

 “ಭಾರತ ಮಾತೆಗೆ ಕನಿಷ್ಟಪಕ್ಷ ಒಂದು ಸಾವಿರ ಯುವಕರ ಬಲಿಯಾದರೂ ಬೇಕು. ಪುರುಷಸಿಂಹರು, ಮೂರ್ಖರಲ್ಲ, ಮನಸ್ಸಿನಲ್ಲಿಡಿ. ಪುರೋಗಮನಕ್ಕೆ ವಿರುದ್ಧವಾದ ನಿಮ್ಮ ಸಂಸ್ಕೃತಿಯನ್ನು ಒಡೆಯಲು, ಒಂದು ಆಯುಧವಾದ ಆಂಗ್ಲಸರ್ಕಾರವನ್ನು ದೇವರೇ ಭರತಖಂಡಕ್ಕೆ ತಂದಿರುವನು. ಆಂಗ್ಲೇಯರು ಮೊಟ್ಟಮೊದಲು ಬೇರೂರಲು ಸಹಾಯಮಾಡಿದ ಜನರನ್ನು ಮದ್ರಾಸು ಒದಗಿಸಿರುವುದು. ಬಡವರಲ್ಲಿ ಅನುತಾಪವನ್ನು, ಹಸಿದ ಹೊಟ್ಟೆಗೆ ಹಿಟ್ಟನ್ನು, ಜನಸಾಮಾನ್ಯರಿಗೆ ತಿಳುವಳಿಕೆಯನ್ನು ಕೊಡುವಂತಹ ನೂತನ ಸಂಸ್ಥೆಯನ್ನು ಸ್ಥಾಪಿಸುವುದಕ್ಕಾಗಿ, ನಿಮ್ಮ ಪೂರ್ವಿಕರ ಕಿರುಕುಳದಿಂದ ಪಶುಸಮಾನಕ್ಕೆ ಇಳಿದವರನ್ನು ಮಾನವರನ್ನಾಗಿ ಮಾಡುವುದಕ್ಕಾಗಿ, ಪ್ರಾಣ ಹೋಗುವವರೆಗೆ ಹಿಂಜರಿಯದೆ ಹೋರಾಡುವ ಸ್ವಾರ್ಥತ್ಯಾಗಿಗಳಾದ ಎಷ್ಟು ಮಂದಿ ಯುವಕರನ್ನು ಮದ್ರಾಸು ಈಗ ಕೊಡುವುದು?” 

 ಸ್ವಾಮೀಜಿ ತಮ್ಮ ಪ್ರಯಾಣವನ್ನು ವಿವರಿಸಿ ಬರೆದ ಪತ್ರವನ್ನೇ ನಾವು ಮೇಲೆ ಕೊಟ್ಟಿರುವೆವು. ಅವರು ಎಷ್ಟು ಸುಂದರವಾಗಿ ಪದಗಳ ಮೂಲಕ ಚಿತ್ರಿಸಬಲ್ಲರು! ಅದನ್ನು ಓದುತ್ತಿದ್ದರೆ ಅವರೊಡನೆ ನಾವೂ ಹೋಗುತ್ತಿರುವಂತೆ ಭಾಸವಾಗುವುದು. ಅವರು ನೋಡಿದ ಚಿತ್ರ ನಮ್ಮ ಕಣ್ಣು ಮುಂದೆ ಕಟ್ಟಿದಂತಾಗುವುದು, ಅದು ಕೇವಲ ಕುತೂಹಲದಿಂದ ಪ್ರೇರೇಪಿತರಾದ ಪ್ರಯಾಣಿಕರ ಚಿತ್ರದಂತೆ ಅಳಿಸಿಹೋಗುವುದಿಲ್ಲ. ಎಲ್ಲೆಲ್ಲಿ ಚೆನ್ನಾಗಿರುವುದು ಇದೆ. ಅದನ್ನು ಭರತಖಂಡ ಕಲಿಯಬೇಕು ಎಂದು ಆಶಿಸುವರು. ತಮ್ಮ ದೇಶವನ್ನು ಇತರ ದೇಶಗಳೊಡನೆ ಸದಾ ತುಲನೆ ಮಾಡಿ ನೋಡುತ್ತಿರುವರು. ಹಾಗೆ ಮಾಡುವಾಗ ಕಾದ ಹೃದಯದಿಂದ ಕಿಡಿಗಳು ಹೇಗೆ ಏಳುತ್ತಿವೆ ಎಂಬುದನ್ನು ಪತ್ರದ ಕೊನೆಯಲ್ಲಿ ನೋಡಿದೆವು. ಇದು ಮದ್ರಾಸಿನ ಶಿಷ್ಯನೊಬ್ಬನಿಗೆ ಬರೆದ ಪತ್ರ. ಆದರೆ ಎಲ್ಲಾ ಭಾರತೀಯರಿಗೂ ಅನ್ವಯಿಸುವುದು. 

 ಸ್ವಾಮೀಜಿಯವರು ಯೋಕೋಹಾಮದಿಂದ ಬ್ರಿಟಿಷ್ ಕೊಲಂಬಿಯಾದಲ್ಲಿರುವ ವಾಂಕೂವರ್ ಅನ್ನು ಜುಲೈ ಮಧ್ಯಭಾಗದಲ್ಲಿ ತಲುಪಿದರು. ಅಲ್ಲಿಂದ ರೈಲಿನಲ್ಲಿ ಮೂರು ದಿನಗಳು ಸಂಚರಿಸಿ ಚಿಕಾಗೊ ನಗರವನ್ನು ಸೇರಿಸರು. ತಮ್ಮ ಕನಸಿನ ನಗರ ಚಿಕಾಗೊ, ಭರತಖಂಡದ ಕಡೆ ನೋಡುವಂತೆ ಮಾಡಿದ ನಗರ. 

