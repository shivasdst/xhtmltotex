
\chapter{ಬೇಲೂರು ಮಠದ ಪ್ರತಿಷ್ಠಾಪನೆ}

 ಸ್ವಾಮೀಜಿಯವರು ಕಾಶ್ಮೀರದಿಂದ ೧೮೯೮ನೇ ಅಕ್ಟೋಬರ್ ೧೮ರಲ್ಲಿ ಹಿಂತಿರುಗಿದ ಮೇಲೆ ಬೇಲೂರು ಮಠದಲ್ಲಿ ಹೊಸದಾಗಿ ಸೇರಿದ ಬ್ರಹ್ಮಚಾರಿಗಳಿಗೆ ತರಬೇತು ಕೊಡುತ್ತಿದ್ದರು. ಅವರನ್ನು ಸಂಘ ಕಾರ‍್ಯಗಳಲ್ಲಿ ನಿಪುಣರನ್ನಾಗಿ ಮಾಡಲು ಯತ್ನಿಸಿದರು. ನವೆಂಬರ್ ತಿಂಗಳಿಂದ ಮಠದಲ್ಲಿ ಕೆಲವು ಕಾಲ, ಬಲರಾಮ ಬಾಬುಗಳ ಮನೆಯಲ್ಲಿ ಕೆಲವು ಕಾಲ ಕಳೆಯುತ್ತಿದ್ದರು. ಇದೇ ತಿಂಗಳು ೫ನೇ ತಾರೀಖು ಸ್ವಾಮೀಜಿ ಕಾಶ್ಮೀರದಿಂದ ಬಂದ ಮುಖ್ಯ ನ್ಯಾಯಾಧಿಪತಿ ಋಷಿಬೀರ್ ಮುಖರ್ಜಿ ಮತ್ತು ಕಾಶ್ಮೀರದ ಪ್ರಧಾನ ಮಂತ್ರಿಗಳಾದ ನೀಲಾಂಬರ ಮುಖರ್ಜಿ ಎಂಬುವರನ್ನು ಮಠದಲ್ಲಿ ಬರಮಾಡಿಕೊಂಡರು. ಮಾರನೆ ದಿನ ಕಾಶ್ಮೀರದಲ್ಲಿ ತಮ್ಮೊಡನಿದ್ದ ಪಾಶ್ಚಾತ್ಯ ಶಿಷ್ಯರು ಇತರ ಉತ್ತರ ಇಂಡಿಯಾದ ಸ್ಥಳಗಳನ್ನು ನೋಡಿಕೊಂಡು ಮಠಕ್ಕೆ ಬಂದರು. 

 ನವೆಂಬರ್ ೧೨ನೇ ತಾರೀಖು ಬೇಲೂರು ಮಠದ ಹೊಸ ನಿವೇಶನಕ್ಕೆ\break ಶ‍್ರೀಶಾರದಾದೇವಿಯವರು ಹಲವು ಸ್ತ್ರೀ ಭಕ್ತರೊಡನೆ ಬಂದರು. ಮಠದಲ್ಲಿ ಸಂನ್ಯಾಸಿಗಳೆಲ್ಲ ಬಂದಿದ್ದರು. ಶ‍್ರೀರಾಮಕೃಷ್ಣರ ಪೂಜಾನಂತರ ಅಂದು ಮಧ್ಯಾಹ್ನ ಸ್ವಾಮೀಜಿ, ಬ್ರಹ್ಮಾನಂದ ಮತ್ತು ಶಾರದಾನಂದರೊಡನೆ ಕಲ್ಕತ್ತೆಗೆ ಹೊರಟರು. ಮಾರನೆ ದಿನ ಬೆಳಿಗ್ಗೆ ಬಾಗಬಜಾರಿನಲ್ಲಿ ಸೋದರಿ ನಿವೇದಿತಾ ಬಾಲಿಕಾ ಪಾಠಶಾಲೆಯನ್ನು ಶ‍್ರೀಶಾರದಾದೇವಿ\-ಯವರು ಉದ್ಘಾಟಿಸಿದರು. ಶ‍್ರೀಮಾತೆಯವರು ಪೂಜಾನಂತರ ಜಗಜ್ಜನನಿಯ ಆಶೀರ್ವಾದ ಈ ಪಾಠಶಾಲೆಯ ಬಾಲಕಿಯರ ಮೇಲೆಲ್ಲ ಇರಲಿ, ಅವರೆಲ್ಲ ಆದರ್ಶ ನಾರಿಯರಾಗಲಿ ಎಂದು ಪ್ರಾರ್ಥಿಸಿದರು. 

 ಬೇಲೂರು ಗ್ರಾಮದಲ್ಲಿ ಶ‍್ರೀರಾಮಕೃಷ್ಣ ಮಠದ ಉದ್ಘಾಟನಾ ಮಹೋತ್ಸವ ಡಿಸೆಂಬರ್ ೯ನೇ ತಾರೀಖು ಆಯಿತು. ಬೆಳಿಗ್ಗೆ ಗಂಗಾಸ್ನಾನ ಮಾಡಿ ಸ್ವಾಮೀಜಿ ದೇವರ ಮನೆಗೆ ಹೋದರು. ಅನಂತರ ಪೂಜಾಸನದಲ್ಲಿ ಕುಳಿತುಕೊಂಡು ಹೂವಿನ ಬುಟ್ಟಿಯಲ್ಲಿ ಎಷ್ಟು ಹೂವು ಮತ್ತು ಪತ್ರೆಗಳಿದ್ದವೋ ಎಲ್ಲವನ್ನೂ ಒಟ್ಟಿಗೆ ಎರಡೂ ಕೈಗಳಿಂದಲೂ ತೆಗೆದುಕೊಂಡು ಶ‍್ರೀರಾಮಕೃಷ್ಣರ ಪಾದುಕೆಯ ಮೇಲೆ ಅರ್ಪಿಸಿ ಧ್ಯಾನಸ್ಥರಾದರು. ಅದು ಅಪೂರ್ವ ದರ್ಶನ. ಅವರ ಧರ್ಮ–ಪ್ರಭಾ–ವಿಭಾಷಿತವಾದ ಸ್ನಿಗ್ಧೋಜ್ವಲ ತೇಜಸ್ಸಿನಿಂದ ದೇವರ ಮನೆಯು ಒಂದು ಅದ್ಭುತವಾದ ಪ್ರಕಾಶದಿಂದ ತುಂಬಿತು! ಪ್ರೇಮಾನಂದರು ಮತ್ತು ಇತರ ಸ್ವಾಮಿಗಳು ಪೂಜಾಗೃಹದ ಬಾಗಿಲಿನಲ್ಲಿ ನಿಂತುಕೊಂಡಿದ್ದರು. 

 ಧ್ಯಾನ ಪೂಜೆಗಳಾದ ಮೇಲೆ ಮಠದ ನಿವೇಶನಕ್ಕೆ ಹೊರಡಲು ಎಲ್ಲವೂ ಸಿದ್ಧವಾಯಿತು. ಶ‍್ರೀರಾಮಕೃಷ್ಣರ ಭಸ್ಮಾಸ್ಥಿಗಳನ್ನು ಇಟ್ಟಿದ್ದ ತಾಮ್ರ ಸಂಪುಟವನ್ನು ಸ್ವಾಮೀಜಿ ತಾವೇ ಬಲಭುಜದ ಮೇಲೆ ಇಟ್ಟುಕೊಂಡು ಹೊರಟರು. ಇತರ ಸಂನ್ಯಾಸಿಗಳೂ ಹಿಂದೆ ಹಿಂದೆ ಹೋದರು. ಶಂಖ ಘಂಟೆ ಇವುಗಳ ಧ್ವನಿಯಿಂದ ತಟ ಪ್ರದೇಶವು ಅದುರುತಿತ್ತು; ಗಂಗಾನದಿ ಹಾವಭಾವನೆಗಳನ್ನು ತೋರಿಸುತ್ತ ನರ್ತನ ಮಾಡುವಂತೆ ಇತ್ತು. ಹೋಗುತ್ತ ಹೋಗುತ್ತ ಸ್ವಾಮೀಜಿ ತಮ್ಮ ಶಿಷ್ಯರಿಗೆ ಹೀಗೆಂದರು; “ಪರಮ ಹಂಸರು ‘ನೀನು ನನ್ನನ್ನು ಭಜದಮೇಲೆ ಇಟ್ಟುಕೊಂಡು ಎಲ್ಲಿಗೆ ಕರೆದುಕೊಂಡು ಹೋದರೆ ಅಲ್ಲಿಗೆ ನಾನು ಬರುತ್ತೇನೆ, ಅಲ್ಲಿಯೇ ನಾನು ನೆಲೆಸಿಬಿಡುತ್ತೇನೆ, ಅದು ಮರದ ಕೆಳಗೇ ಆಗಲಿ ಒಂದು ಗುಡಿಸಲೇ ಆಗಲಿ ಚಿಂತೆಯಿಲ್ಲ’ ಎಂದು ನನಗೆ ಹೇಳಿದ್ದಾರೆ. ಅದಕ್ಕೋಸ್ಕರವೇ ಇವತ್ತು ನಾನೇ ಅವರನ್ನು (ಅವರ ಭಸ್ಮಾಸ್ಥಿಗಳನ್ನು ) ಹೆಗಲಿನ ಮೇಲೆ ಕೂರಿಸಿಕೊಂಡು ಹೊಸ ಮಠ ಪ್ರದೇಶಕ್ಕೆ ಹೋಗುತ್ತಿದ್ದೇನೆ. ಇದನ್ನು ಖಂಡಿತವಾಗಿ ತಿಳಿದುಕೊಳ್ಳಿ, ಬಹು ಜನರ ಹಿತಕ್ಕೆ ಬಹು ಜನರ ಸುಖಕ್ಕೆ ಪರಮಹಂಸರು ಈ ಸ್ಥಾನದಲ್ಲಿ ನಿಶ್ಚಿಂತೆಯಿಂದ ನಿಲ್ಲುವರು.” 

 “ನಮ್ಮ ಈ ಮಠವಿದೆಯಲ್ಲ, ಇದರಿಂದ ಸಕಲ ಮತದ ಮತ್ತು ಭಾವಗಳ ಸಾಮಂಜಸ್ಯವು ನಿಂತಿರುತ್ತದೆ. ಇದು ಪರಮಹಂಸರ ಮತವು ಎಷ್ಟು ಉದಾರವಾಗಿತ್ತೋ, ಅಷ್ಟೇ ಉದಾರವಾದ ಅದೇ ಭಾವಕ್ಕೆ ಕೇಂದ್ರಸ್ಥಾನವಾಗುತ್ತದೆ. ಈ ಸ್ಥಳದಿಂದ ಹೊರಟ ಉಕ್ಕಿ ಹರಿಯುವ ಸಮನ್ವಯದ ಪ್ರವಾಹದಿಂದ ಪ್ರಪಂಚವೆಲ್ಲ ತುಂಬಿಹೋಗುತ್ತದೆ.” 

 ಹೀಗೆ ಮಾತುಗಳು ನಡೆಯುತ್ತ ಎಲ್ಲರೂ ಮಠದ ನಿವೇಶನಕ್ಕೆ ಬಂದರು. ಸ್ವಾಮೀಜಿ ಭುಜದ ಮೇಲಿದ್ದ ಸಂಪುಟವನ್ನು ಅಲ್ಲಿ ಹಾಕಿದ್ದ ಆಸನದ ಮೇಲೆ ಇಳಿಸಿ ನೆಲಮುಟ್ಟಿ ನಮಸ್ಕಾರ ಮಾಡಿದರು. ಮಿಕ್ಕವರೂ ನಮಸ್ಕಾರ ಮಾಡಿದರು. ಆಮೇಲೆ ಸ್ವಾಮೀಜಿ ಪುನಃ ಪೂಜೆಗೆ ಕುಳಿತರು. ಪೂಜೆ ಮುಗಿದ ಮೇಲೆ ಹೋಮಾಗ್ನಿಯನ್ನು ಪ್ಪ್ರಜ್ವಲನ ಮಾಡಿ ಅದರಲ್ಲಿ ಹೋಮ ಮಾಡಿದರು. ಸ್ವಾಮೀಜಿ ಸಂನ್ಯಾಸಿ ಭ್ರಾತೃಗಳ ಸಹಾಯದಿಂದ ತಮ್ಮ ಕೈಯಿಂದಲೇ ಪರಮಾನ್ನವನ್ನು ಮಾಡಿ ನೈವೇದ್ಯ ಮಾಡಿದರು. ಆ ಹೊತ್ತು ಕೆಲವು ಜನ ಗೃಹಸ್ಥರಿಗೆ ದೀಕ್ಷೆ ಕೊಟ್ಟರು. ಪೂಜೆಯನ್ನು ಮುಗಿಸಿಕೊಂಡು ಸ್ವಾಮೀಜಿ ಅಲ್ಲಿದ್ದವರನ್ನೆಲ್ಲ ಆದರದಿಂದ ಬರಮಾಡಿಕೊಂಡು ಅವರನ್ನು ಸಂಬೋಧಿಸಿ, “ಮಹಾಯುಗಾವತಾರ ಶ‍್ರೀರಾಮಕೃಷ್ಣ ಪರಮಹಂಸರು ಇಂದಿನಿಂದ ಬಹು ಕಾಲದವರೆಗೆ ಬಹು ಜನರ ಹಿತಕ್ಕೆ ಬಹು ಜನರ ಸುಖಕ್ಕೆ ಈ ಪುಣ್ಯಕ್ಷೇತ್ರದಲ್ಲಿ ನಿಂತು, ಇದನ್ನು ಸರ್ವಧರ್ಮದ ಅಪೂರ್ವ ಸಮನ್ವಯಕ್ಕೆ ಕೇಂದ್ರವನ್ನಾಗಿ ಮಾಡಬೇಕೆಂದು ತಾವೆಲ್ಲರೂ ಇಂದು ಮನೋವಕ್ಕಾಯಗಳಿಂದ ಪರಮಹಂಸರ ಪಾದಪದ್ಮಗಳಲ್ಲಿ ಪ್ರಾರ್ಥನೆ ಮಾಡಬೇಕು” ಎಂದರು. ಎಲ್ಲರೂ ಕೈಮುಗಿದುಕೊಂಡು ಹಾಗೆಯೇ ಪ್ರಾರ್ಥನೆ ಮಾಡಿದರು. ಅನಂತರ ಸ್ವಾಮೀಜಿ “ಪರಮಹಂಸರ ಇಚ್ಛೆಯಿಂದ ಇಂದು ಅವರ ಧರ್ಮಕ್ಷೇತ್ರ ಪ್ರತಿಷ್ಠಿತವಾಯಿತು. ಹನ್ನೆರಡು ವರ್ಷಗಳ ಚಿಂತಾಭಾರ ಕೆಳಗಿಳಿದಂತಾಯಿತು” ಎಂದರು. 

 ಮತ್ತೊಂದು ದಿನ ಸ್ವಾಮೀಜಿ ಶಿಷ್ಯನೊಡನೆ ಮಠದ ನಿವೇಶನದಲ್ಲಿ ಸಂಚಾರ ಮಾಡುತ್ತ ಹೀಗೆ ಹೇಳಿದರು: “ಇಲ್ಲಿ ಸಾಧುಗಳು ಇರುವುದಕ್ಕೆ ಸ್ಥಳವಾಗುವುದು. ಸಾಧನೆ ಭಜನೆ ಜ್ಞಾನ ವಿಚಾರಗಳಿಗೆ ಈ ಮಠ ಪ್ರಧಾನ ಕೇಂದ್ರವಾಗುವುದು– ಇದೇ ನನ್ನ ಅಭಿಪ್ರಾಯ. ಇಲ್ಲಿ ಉದಯಿಸುವ ಶಕ್ತಿಯಿಂದ ಜಗತ್ತು ತಲ್ಲಣಿಸಿ ಹೋಗುತ್ತದೆ. ಮನುಷ್ಯನ ಜೀವನಗತಿ ತಿರುಗಿಸಲ್ಪಡುತ್ತದೆ. ಇಲ್ಲಿ ಉಗಮಿಸುವ ಜ್ಞಾನ, ಭಕ್ತಿ, ಯೋಗ, ಕರ್ಮ ಇವುಗಳ ಸಮನ್ವಯ ಭಾವನೆ ಎಲ್ಲೆಲ್ಲಿಯೂ ಪಸರಿಸುವುದು. ಈ ಮಠಕ್ಕೆ ಸೇರಿದ ಸಂನ್ಯಾಸಿಗಳ ಇಚ್ಛೆಯಿಂದ ಕಾಲಕ್ರಮದಲ್ಲಿ ಜನರ ಸುಪ್ತಪ್ರಾಣವು ಜಾಗ್ರತಗೊಳ್ಳುವುದು. ಮನಸ್ಸಿಗೆ ಇಂತಹ ಯೋಚನೆಗಳು ಎಷ್ಟೋ ಬರುವುವು. 

 “ಮಠದ ದಕ್ಷಿಣ ಭಾಗದಲ್ಲಿ ಸ್ಥಳವಿದೆಯಲ್ಲ ಅಲ್ಲಿಯೇ ವಿದ್ಯೆಯ ಕೇಂದ್ರ ಸ್ಥಾನವಾಗುವುದು. ವ್ಯಾಕರಣ ದರ್ಶನ ವಿಜ್ಞಾನ ಕಾವ್ಯ ಅಲಂಕಾರ ಸ್ಮೃತಿ ಭಕ್ತಿಶಾಸ್ತ್ರ ಮತ್ತು ರಾಜಕೀಯ ಭಾವನೆ ಇವುಗಳು ಇಲ್ಲಿ ಬೋಧಿಸಲ್ಪಡುವುವು. ಬಾಲಬ್ರಹ್ಮಚಾರಿಗಳು ಅಲ್ಲಿ ವಾಸಮಾಡಿಕೊಂಡು ಶಾಸ್ತ್ರವನ್ನು ಓದುವರು. ಅವರ ಅನ್ನ ಬಟ್ಟೆಗಳೆಲ್ಲವೂ ಮಠದಿಂದ ದೊರೆಯುವುವು. ಈ ಬ್ರಹ್ಮಚಾರಿಗಳೆಲ್ಲಾ ಐದು ವರ್ಷ ಶಿಕ್ಷಣ ಪಡೆದಮೇಲೆ ಇಷ್ಟವಿದ್ದರೆ ಮನೆಗೆ ಹೋಗಿ ಸಂಸಾರಿಗಳಾಗಬಹುದು. ಹಾಗಿಲ್ಲದೆ ಸಂನ್ಯಾಸದ ಮೇಲೆ ಇಷ್ಟವಿದ್ದು, ಮಠಾಧಿಪತಿಗಳು ಒಪ್ಪಿದರೆ ಸಂನ್ಯಾಸವನ್ನು ತೆಗೆದುಕೊಳ್ಳಬಹುದು. ಈ ಬ್ರಹ್ಮಚಾರಿಗಳಲ್ಲಿ ಯಾರಲ್ಲಿಯಾದರೂ ಉಚ್ಛೃಂಖಲತೆಯಾಗಲಿ ದುಶ್ಚರಿತ್ರವಾಗಲಿ ಕಂಡುಬಂದರೆ ಅಂಥವರನ್ನು ಮಠದ ಸ್ವಾಮಿಗಳು ಆಗಲೇ‌ ಹೊರಡಿಸುವರು. ಇಲ್ಲಿ ಜಾತಿ ವರ್ಣಗಳ ಭೇದವಿಲ್ಲದೆ ವ್ಯಾಸಂಗ ನಡೆಯುವುದು. ಇದಕ್ಕೆ ಯಾರ ಆಕ್ಷೇಪಣೆ ಇದೆಯೋ ಅವರನ್ನು ತೆಗೆದುಕೊಳ್ಳುವುದಿಲ್ಲ. ಆದರೆ ಯಾರು ಜಾತಿ ವರ್ಣಾಶ್ರಮಗಳನ್ನು ನಡೆಸಿಕೊಂಡುಹೋಗಲು ಇಷ್ಟಪಡುವರೊ ಅವರು ತಮ್ಮ ಊಟ ಉಪಚಾರಗಳ ಏರ್ಪಾಡನ್ನು ತಾವೇ ವಹಿಸಿಕೊಳ್ಳಬೇಕು. ಅವರು ವ್ಯಾಸಂಗವನ್ನು ಮಾತ್ರ ಇಲ್ಲಿ ಮಾಡಬೇಕು. ಅವರ ನಡತೆಯ ವಿಚಾರಗಳಲ್ಲಿ ಸ್ವಾಮಿಗಳು ತೀಕ್ಷ್ಣ ದೃಷ್ಟಿಯನ್ನು ಇಟ್ಟೇ ಇರುವರು. ಇಲ್ಲಿ ಶಿಕ್ಷಿತರಾಗದೆ ಹೋದರೆ ಯಾರೂ ಸಂನ್ಯಾಸಕ್ಕೆ ಯೋಗ್ಯರಾಗುವುದಿಲ್ಲ. 

 “ಮಠದ ದಕ್ಷಿಣಕ್ಕೆ ಇರುವ ಈ ಸ್ಥಳವನ್ನು ಕಾಲಕ್ರಮೇಣ ಕೊಂಡುಕೊಳ್ಳಬೇಕು. ಇಲ್ಲಿ ಮಠದ ಅನ್ನಸತ್ರವಾಗುವುದು. ಇಲ್ಲಿ ನಿಜವಾದ ಬಡಬಗ್ಗರಿಗೆ ನಾರಾಯಣ ಬುದ್ಧಿಯಿಂದ ಸೇವೆ ಮಾಡುವುದಕ್ಕೆ ಏರ್ಪಾಡು ಇರುವುದು. ಈ ಅನ್ನಸತ್ರವು ಪರಮಹಂಸರ ಹೆಸರಿನಲ್ಲಿ ಸ್ಥಾಪಿಸಲ್ಪಡುವುದು. ಮೊದಲು ಇಬ್ಬರು ಮೂವರನ್ನು ಇಟ್ಟುಕೊಂಡು ಪ್ರಾರಂಭವಾದರೆ ಸಾಕು. ಉತ್ಸಾಹಿಗಳಾದ ಬ್ರಹ್ಮಚಾರಿಗಳಿಗೆ ಈ ಅನ್ನಸತ್ರ ನಡೆಸುವುದರಲ್ಲಿ ಶಿಕ್ಷಣ ಕೊಡಬೇಕು. ಅವರು ಅಲ್ಲಿ ಇಲ್ಲಿ ಶೇಖರಿಸುವುದರಿಂದ ಇಲ್ಲವೆ ಭಿಕ್ಷೆ ಮಾಡುವುದರಿಂದ ಈ ಅನ್ನ ಸತ್ರ ನಡೆಯುವುದು. ಮಠ ಇದಕ್ಕೆ ಯಾವ ವಿಧವಾದ ದ್ರವ್ಯಸಹಾಯವನ್ನೂ ಮಾಡುವುದಕ್ಕೆ ಆಗುವುದಿಲ್ಲ. ಬ್ರಹ್ಮಚಾರಿಗಳೇ ಅದಕ್ಕಾಗಿ ದುಡ್ಡನ್ನು ಸೇರಿಸಿಕೊಂಡು ಬರಬೇಕು. ಸೇವಾಸತ್ರದಲ್ಲಿ ಹೀಗೆ ಐದು ವರ್ಷವೂ ವಿದ್ಯಾಶ್ರಮದಲ್ಲಿ ಐದು ವರ್ಷವೂ ಒಟ್ಟಿಗೆ ಹತ್ತು ವರ್ಷ ಶಿಕ್ಷಣವಾದಮೇಲೆ ಮಠದ ಸ್ವಾಮಿಗಳಿಂದ ದೀಕ್ಷೆ ಪಡೆದು ಸಂನ್ಯಾಸಾಶ್ರಮವನ್ನು ಪಡೆಯುವುದಕ್ಕೆ ಯೋಗ್ಯರಾಗುವರು. ಸಂನ್ಯಾಸಿಗಳಾಗುವುದಕ್ಕೆ ಅವರಿಗೆ ಇಷ್ಟವಿದ್ದು, ಮಠಾಧ್ಯಕ್ಷರೂ ಸಂನ್ಯಾಸ ಕೊಡುವುದಕ್ಕೆ ಇಷ್ಟಪಟ್ಟರೆ ಅವರು ಸಂನ್ಯಾಸಿಗಳಾಗಬಹುದು. ಆದರೆ ಮಠಾಧ್ಯಕ್ಷರು ಕೆಲವು ವಿಶೇಷ ಸದ್ಗುಣಸಂಪನ್ನರಾದ ಬ್ರಹ್ಮಚಾರಿಗಳಿಗೆ ತಮಗೆ ಇಷ್ಟ ಬಂದಾಗ ಸಂನ್ಯಾಸವನ್ನು ಕೊಡಬಹುದು. ಸಾಧಾರಣ ಬ್ರಹ್ಮಚಾರಿಗಳಿಗೆ ಮಾತ್ರ ಹಿಂದೆ ಹೇಳಿದ ರೀತಿಯಲ್ಲಿ ಕ್ರಮಕ್ರಮವಾಗಿ ಸಂನ್ಯಾಸಾಶ್ರಮದಲ್ಲಿ ಪ್ರವೇಶಮಾಡಿಸಬೇಕು. 

 “ಅನ್ನದಾನ, ವಿದ್ಯಾದಾನ ಎಲ್ಲಕ್ಕೂ ಮೇಲೆ ಆಧ್ಯಾತ್ಮ ಜ್ಞಾನದಾನ ಈ ಮೂರು ಭಾವಗಳ ಸಮನ್ವಯವನ್ನು ಈ ಮಠದಿಂದ ಮಾಡಬೇಕಾಗಿದೆ. ಅನ್ನದಾನ ಮಾಡುವುದಕ್ಕೆ ಪ್ರಯತ್ನ ಮಾಡುತ್ತ ಮಾಡುತ್ತ ಬ್ರಹ್ಮಚಾರಿಗಳ ಮನಸ್ಸಿನಲ್ಲಿ ಪರೋದ್ದೇಶದಿಂದ ಕರ್ಮಮಾಡುವುದು, ಈಶ್ವರ ಬುದ್ಧಿಯಿಂದ ಜೀವಸೇವೆ ಮಾಡುವುದು ದೃಢವಾಗುತ್ತದೆ. ಇದರಿಂದ ಅವರ ಚಿತ್ತ ಕ್ರಮೇಣ ನಿರ್ಮಲವಾಗಿ ಅದರಿಂದ ಸತ್ತ್ವಭಾವ ಅಭಿವೃದ್ಧಿ ಹೊಂದುತ್ತದೆ. ಹಾಗಾದರೆ ಬ್ರಹ್ಮಚಾರಿಗಳು ಸಕಾಲದಲ್ಲಿ ಬ್ರಹ್ಮವಿದ್ಯೆಯನ್ನು ಪಡೆಯುವುದಕ್ಕೆ ಯೋಗ್ಯತೆಯನ್ನೂ ಸಂನ್ಯಾಸಾಶ್ರಮದಲ್ಲಿ ಪ್ರವೇಶ ಮಾಡುವುದಕ್ಕೆ ಅಧಿಕಾರವನ್ನೂ ಪಡೆಯುತ್ತಾರೆ. 

 “ಈಶ್ವರ ಸಂಕಲ್ಪವಿದ್ದರೆ ಈ ಮಠವನ್ನು ಮಹಾಸಮನ್ವಯ ಕ್ಷೇತ್ರವನ್ನಾಗಿ ಮಾಡಬಹುದು. ಪರಮಹಂಸರು ನಮ್ಮ ಸರ್ವಭಾವಗಳ ಸಮನ್ವಯಮೂರ್ತಿ. ಈ ಸಮನ್ವಯದ ಭಾವವನ್ನು ಇಲ್ಲಿ ಜಾಗ್ರತಗೊಳಿಸಿದರೆ ಪರಮಹಂಸರು ಜಗತ್ತಿನಲ್ಲಿ ಪ್ರತಿಷ್ಠಿತರಾಗಿ ನಿಲ್ಲುವರು. ಎಲ್ಲ ಮತಪಂಥದ ಜನರೂ, ಬ್ರಾಹ್ಮಣರಿಂದ ಚಂಡಾಲರವರೆಗೆ ಎಲ್ಲರೂ ಇಲ್ಲಿಗೆ ಬಂದು ತಮ್ಮ ತಮ್ಮ ಆದರ್ಶಗಳನ್ನು ಹೇಗೆ ಕಾಣಬಹುದೊ ಹಾಗೆ ಮಾಡಬಹುದು. ಅಂದು ಮಠದ ನಿವೇಶನದಲ್ಲಿ ಪರಮಹಂಸರನ್ನು ಪ್ರತಿಷ್ಠೆ ಮಾಡಿದಾಗ ಅಲ್ಲಿಂದ ಅವರ ಭಾವ ವಿಕಾಸಗೊಂಡು ಚರಾಚರ ಪ್ರಪಂಚವನ್ನೆಲ್ಲಾ ಆವರಿಸಿಬಿಟ್ಟಂತೆ ನನ್ನ ಮನಸ್ಸಿಗೆ ಭಾವನೆ ಉಂಟಾಯಿತು. ನಾನೇನೊ ಕೈಲಾದಷ್ಟು ಮಟ್ಟಿಗೆ ಮಾಡುತ್ತಿದ್ದೇನೆ, ಮುಂದೆಯೂ ಮಾಡುತ್ತೇನೆ. ನೀವು ಪರಮಹಂಸರ ಉದಾರಭಾವವನ್ನು ಜನರಿಗೆ ತಿಳಿಸಿಕೊಡಿ. ಸುಮ್ಮನೆ ವೇದಾಂತವನ್ನು ಓದಿ ಆಗುವುದೇನು? ಜೀವನದಲ್ಲಿ ಶುದ್ಧ ಅದ್ವೈತವಾದದ ಸತ್ಯವನ್ನು ಸಪ್ರಮಾಣವಾಗಿ ತೋರಿಸಿ ಕೊಡಬೇಕು. ಶಂಕರಾಚಾರ‍್ಯರು ಈ ಅದ್ವೈತವಾದವನ್ನು ಕಾಡುಮೇಡುಗಳಲ್ಲಿ ಬೆಟ್ಟ ಗುಡ್ಡಗಳಲ್ಲಿ ಇಟ್ಟುಹೋಗಿದ್ದಾರೆ. ಈಗ ನಾನು ಅದನ್ನು ಇಲ್ಲಿಂದ ಸಂಸಾರದ ಮತ್ತು ಸಮಾಜದ ಎಲ್ಲಾ ಕಡೆಗಳಲ್ಲಿಯೂ ಇಟ್ಟು ಹೋಗೋಣವೆಂದು ಬಂದಿದ್ದೇನೆ. ಮನೆಮಠ ಮೈದಾನ ಬೆಟ್ಟ ಗುಡ್ಡ ಕಾಡುಮೇಡು ಎಲ್ಲಾ ಕಡೆಗಳಲ್ಲಿಯೂ ಈ ಅದ್ವೈತನಾದದ ದುಂದುಭಿಯನ್ನು ಎಬ್ಬಿಸಬೇಕು. ನೀವು ನನಗೆ ಸಹಾಯಕರಾಗಿರಿ.” 

 ಶಿಷ್ಯ: “ಮಹಾರಾಜ್, ಧ್ಯಾನದ ಸಹಾಯದಿಂದ ಅದ್ವೈತ ಭಾವವನ್ನು ಅನುಭವ ಮಾಡಿಕೊಂಡರೆ ನಮಗೆ ಒಳೆಯದೆಂದು ತೋರುತ್ತದೆ. ಸುಮ್ಮನೆ ಕರ್ಮ ಮಾಡುವುದರಲ್ಲಿ ಇಷ್ಟವಿಲ್ಲ.” 

 ಸ್ವಾಮೀಜಿ: ಜಡಸಮಾಧಿಯಲ್ಲಿ ನಿಶ್ಚೇಷ್ಟಿತನಾಗಿ ಇದ್ದುಬಿಡುವುದರಿಂದ ಏನು ಪ್ರಯೋಜನ? ಅದ್ವೈತದ ಪ್ರೇರಣೆಯಿಂದ ಕೆಲವು ವೇಳೆ ತಾಂಡವನೃತ್ಯ ಮಾಡುವೆ, ಮತ್ತೆ ಕೆಲವು ವೇಳೆ ಬಾಹ್ಯಜ್ಞಾನವಿಲ್ಲದೆ ಇರುವೆ. ಒಳ್ಳೆಯ ಪದಾರ್ಥ ಸಿಕ್ಕಿದರೆ ಏನು ಒಬ್ಬನೇ ತಿಂದರೆ ಆನಂದವಾಗುವುದೆ? ಹತ್ತು ಜನಕ್ಕೆ ಕೊಡಬೇಕು, ತಾನೂ ತಿನ್ನಬೇಕು. ಆತ್ಮಾನುಭವ ಮಾಡಿಕೊಂಡು ನೀನು ಮುಕ್ತನಾಗಿಬಿಡುವೆ. ಆದರೆ ಜಗತ್ತಿಗೆ ಅದರಿಂದ ಆದದ್ದೇನು? ಮೂರು ಲೋಕಗಳಿಗೂ ಮುಕ್ತಿಯನ್ನು ಕೊಟ್ಟು ಕರೆದುಕೊಂಡುಹೋಗಬೇಕು. ಮಹಾ ಮಾಯೆಯ ರಾಜ್ಯಕ್ಕೆ ಬೆಂಕಿ ಹಾಕಿ ಹತ್ತಿಸಿಬಿಡಬೇಕು. ಆಗತಾನೆ ನಿತ್ಯವಾಗಿ ಸತ್ಯದಲ್ಲಿ ಪ್ರತಿಷ್ಠಿತನಾಗುವೆ. ಆ ಆನಂದಕ್ಕೆ ಸಮ ಉಂಟೆ? ಆಕಾಶಸಮಾನವಾದ ಮಹಾನಂದದಲ್ಲಿ ಪ್ರತಿಷ್ಠಿತನಾಗುವೆ. ಜೀವಜಗತ್ತಿನ ಎಲ್ಲಾ ಕಡೆಗಳಲ್ಲಿಯೂ ನಿನ್ನ ನಿಜಸತ್ತೆಯನ್ನು ನೋಡಿ ಬೆರಗಾಗಿ ನಿಲ್ಲುವೆ. ಸ್ಥಾವರ ಜಂಗಮಗಳೆಲ್ಲವೂ ನಿನಗೆ ನಿನ್ನ ಸತ್ತಾಬಲದಲ್ಲಿ ಬೋಧೆಯಾಗುತ್ತವೆ. ಆಗ ಮಿಕ್ಕವರನ್ನು ನಿನ್ನ ಹಾಗೆ ಯತ್ನ ಮಾಡಗೊಡಿಸದೆ ಇರುವುದಕ್ಕೆ ಆಗುವುದಿಲ್ಲ. ಈ ಸ್ಥಿತಿಯೇ ಅನುಷ್ಠಾನ ವೇದಾಂತ. ತಿಳಿಯಿತೆ?” 

 ಈ ಅಧ್ಯಾಯದ ಪ್ರಾರಂಭದಲ್ಲಿ ನಡೆದದ್ದು ಕೇವಲ ಉದ್ಘಾಟನಾ ಸಮಾರಂಭ. ಮಠ ಇನ್ನು ಇಳಿದುಕೊಳ್ಳುವುದಕ್ಕೆ ಪೂರ್ತಿ ಆಗಿರಲಿಲ್ಲ. ಎಲ್ಲೋ ಕೆಲವು ಸ್ವಾಮಿಗಳು ಮತ್ತು ಬ್ರಹ್ಮಚಾರಿಗಳು ಮಾತ್ರ ಉದ್ಘಾಟನೆ ಆದಮೇಲೆ ಇಲ್ಲಿಗೆ ಬಂದರು. ಕೊನೆಗೆ ಎಲ್ಲರೂ ಬಂದದ್ದು ಜನವರಿ ೨ನೇ ತಾರೀಖು ೧೮೯೯ರಲ್ಲಿ ಮಾತ್ರ. 

