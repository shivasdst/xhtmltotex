
\chapter{ನಾಗಮಹಾಶಯರ ಭೇಟಿ }

\vskip 2pt

 ಸ್ವಾಮೀಜಿ ಡಿಸೆಂಬರ್ ೧೯ನೇ ತಾರೀಖು ಸ್ವಲ್ಪ ಬದಲಾವಣೆಗಾಗಿ ದೇವಘರಕ್ಕೆ ಹೋದರು. ಸ್ವಾಮೀಜಿಯವರು ಇಲ್ಲಿದ್ದಾಗಲೂ ಅವರ ಆರೋಗ್ಯ ಅಷ್ಟು ತೃಪ್ತಿಕರವಾಗಿರಲಿಲ್ಲ. ಶ್ವಾಸಕೋಶ ಇವರನ್ನು ಪೀಡಿಸತೊಡಗಿತು. ಸ್ವಾಮೀಜಿ ಕೆಲವು ಕಾಲ ಅಲ್ಲಿದ್ದು ಪುನಃ ಮಠಕ್ಕೆ ಬಂದಮೇಲೆ ಯತಿವೃಂದವನ್ನು ತರಬೇತು ಮಾಡಲು ತಮ್ಮ ಮನಸ್ಸನ್ನೆಲ್ಲ ಕೊಟ್ಟರು. ಅವರಿಗೆ ಹಿಂದೂ ಶಾಸ್ತ್ರಗಳ ಮೇಲೆ ತಾವೇ ಪ್ರವಚನಗಳನ್ನು ಕೊಡುತ್ತಿದ್ದರು. ಸಂನ್ಯಾಸ ಜೀವನದ ಆದರ್ಶವನ್ನು ಜಾಜ್ವಲ್ಯಮಾನವಾಗಿ ಅವರ ಮುಂದೆ ಇಡುತ್ತಿದ್ದರು. ಧ್ಯಾನ ಅಧ್ಯಯನ ಸೇವೆ ಪ್ರತಿಯೊಬ್ಬರ ಜೀವನದಲ್ಲಿ ಹಾಸುಹೊಕ್ಕಾಗಿರಬೇಕೆಂದು ಹೇಳಿದರು. ತಮ್ಮ ಇಬ್ಬರು ಶಿಷ್ಯರನ್ನು - ತುರೀಯಾನಂದ ಮತ್ತು ಪ್ರಕಾಶಾನಂದ ಪೂರ್ವಬಂಗಾಳಕ್ಕೆ ಪ್ರಚಾರ ಕಾರ‍್ಯಕ್ಕೆ ಕಳುಹಿಸಿದರು. ಅಲ್ಲಿ ಪ್ರಚಾರ ಕಾರ‍್ಯ ಯಶಸ್ವಿಯಾಗಿ ನಡೆದು ಢಾಕಾದಲ್ಲಿ ಒಂದು ಕೇಂದ್ರವನ್ನು ಸ್ಥಾಪಿಸಿದರು. ಪಶ್ಚಿಮ ಭರತಖಂಡಕ್ಕೆ ಶಾರದಾನಂದ ಮತ್ತು ತುರೀಯಾನಂದರನ್ನು ಪ್ರಚಾರ ಕಾರ‍್ಯಕ್ಕೆ ಕಳುಹಿಸಿದರು. ಸಿಲೋನಿಗೆ ಸ್ವಾಮಿ ಶಿವಾನಂದರನ್ನು ಕಳುಹಿಸಿದರು. ಎಲ್ಲಾ ಕಡೆಗಳಲ್ಲಿಯೂ ಪ್ರಚಾರಕಾರ‍್ಯ ಯಶಸ್ವಿಯಾಗಿ ನಡೆಯಲು ಪ್ರಾರಂಭವಾಯಿತು. ಮದ್ರಾಸಿನಲ್ಲಿ ಸ್ವಾಮೀಜಿ ಕಳುಹಿಸಿದ ರಾಮಕೃಷ್ಣಾನಂದ ಸ್ವಾಮಿಗಳು ಯಶಸ್ವಿಯಾಗಿ ತಮ್ಮ ಪ್ರಭಾವವನ್ನು ಬೀರತೊಡಗಿದರು. 

\vskip 2pt

 ಸಂಘದ ಆದರ್ಶವನ್ನು ಹರಡುವುದಕ್ಕೆ ಮೂರು ಮಾಸಪತ್ರಿಕೆಗಳು ಪ್ರಾರಂಭವಾದವು. ಮದ್ರಾಸಿನಲ್ಲಿ ಬ್ರಹ್ಮವಾದಿನಿ, ಬಂಗಾಳದಲ್ಲಿ ಉದ್ಬೋಧನ, ಮಾಯಾವತಿಯಲ್ಲಿ ಪ್ರಬುದ್ಧ ಭಾರತ ಮಾಸಪತ್ರಿಕೆಗಳು ಯಶಸ್ವಿಯಾಗಿ ನಡೆಯಲು ಆರಂಭವಾದವು. ಕೆಲವು\break ಅನಾಥಾಲಯಗಳು ಸ್ಥಾಪಿಸಲ್ಪಟ್ಟವು. ಪ್ಲೇಗು, ಬರಗಾಲ ನಿವಾರಣ ಕೆಲಸಗಳು ಮುಂದುವರಿದವು. ಕಲ್ಕತ್ತೆಯಲ್ಲಿ ಬಾಲಕಿಯರಿಗಾಗಿ ನಿವೇದಿತಾ ಶಾಲೆ ಪ್ರಾರಂಭವಾಯಿತು. ಈ ಸಮಯದಲ್ಲಿ ಗಮನಾರ್ಹವಾದ ಒಂದು ಸಂಗತಿಯೇ ಹಿಮಾಲಯ ಬೆಟ್ಟಗಳ ಮಧ್ಯದಲ್ಲಿ ಮಾಯಾವತಿ ಎಂಬ ಕಡೆ ಒಂದು ಆಶ್ರಮವನ್ನು ಪ್ರಾರಂಭಿಸಿದ್ದು. ಆರು ಸಾವಿರ ಅಡಿ ಎತ್ತರದಲ್ಲಿದ್ದ ಹಲವು ಎಕರೆಗಳಷ್ಟು ವಿಸ್ತಾರವಾದ ಒಂದು ದೊಡ್ಡ ಬೆಟ್ಟವನ್ನೇ ಸೇವಿಯರ್ ಅವರು ತೆಗೆದುಕೊಂಡರು. ಅದು ತುಂಬಾ ಹಿತಕರವಾದ ತಂಪಾದ ಸ್ಥಳವಾಗಿತ್ತು. ಸುತ್ತಲೂ ದೂರದಲ್ಲಿ ಹಿಮಾವೃತ ಪರ್ವತಗಳ ರಾಶಿ ನೋಡುವುದಕ್ಕೆ ಕಣ್ಣಿಗೆ ಒಂದು ಹಬ್ಬದಂತೆ ಇತ್ತು. ಅಂತರ‍್ಮುಖ ಜೀವನಕ್ಕೆ ಧ್ಯಾನಕ್ಕೆ ಅತ್ಯಂತ ಪವಿತ್ರವಾದ ಸ್ಥಳ. ಆ ಆಶ್ರಮವನ್ನು ಸ್ಥಾಪನೆ ಮಾಡುವಾಗ ಸ್ವಾಮೀಜಿ ಹೀಗೆ ಹೇಳುವರು: “ಇಲ್ಲಿ ಅದ್ವೈತವು ಎಲ್ಲಾ ಮೂಢನಂಬಿಕೆಗಳಿಂದಲೂ ಮತ್ತು ದುರ್ಬಲಗೊಳಿಸುವ ಮಿಶ್ರಣಗಳಿಂದಲೂ ರಕ್ಷಿಸಲ್ಪಡುವುದು. ಪರಿಶುದ್ಧವಾದ ಸರಳವಾದ ಅದ್ವೈತವನ್ನಲ್ಲದೆ ಇಲ್ಲಿ ಬೇರಾವುದನ್ನೂ ಬೋಧಿಸುವುದಿಲ್ಲ, ಅಭ್ಯಾಸ ಮಾಡುವುದಿಲ್ಲ. ನಾವು ಉಳಿದ ಎಲ್ಲಾ ಸಿದ್ಧಾಂತಗಳೊಂದಿಗೆ ಸಹಾನುಭೂತಿಯನ್ನು ತೋರಿದರೂ ಈ ಆಶ್ರಮವನ್ನು ಅದ್ವೈತಕ್ಕೆ ಮಾತ್ರ ಮೀಸಲಾಗಿಡುವೆವು.” 

\vskip 2pt

 ಈ ಸಮಯದಲ್ಲಿ ಗಮನಾರ್ಹ ಒಂದು ಸಂಗತಿಯೆ ಶ‍್ರೀರಾಮಕೃಷ್ಣರ ಪ್ರಖ್ಯಾತ ಗೃಹಸ್ಥ ಭಕ್ತರಾದ ನಾಗಮಹಾಶಯರು ಬಂದದ್ದು. ಶ‍್ರೀರಾಮಕೃಷ್ಣರು ನರೇಂದ್ರ ಮತ್ತು ನಾಗಮಹಾಶಯರನ್ನು ಹೀಗೆ ಹೋಲಿಸುತ್ತಿದ್ದರು: ಮಾಯೆ ನರೇಂದ್ರನನ್ನು ಹಿಡಿಯಬೇಕೆಂದು ಬಲೆ ಬೀಸಿದಳು. ಅವನು ಬಲೆಗಿಂತ ದೊಡ್ಡದಾಗಿ ಬೆಳೆದ. ಮಾಯೆ ಎಷ್ಟೆಷ್ಟು ದೊಡ್ಡ ಬಲೆಯನ್ನು ಬೀಸಿದರೆ ನರೇಂದ್ರ ಅಷ್ಟಷ್ಟು ದೊಡ್ಡದಾಗಿ ಬೆಳೆಯುತ್ತಾ ಹೋದ. ಇವನನ್ನು ಹಿಡಿಯುವುದಕ್ಕೆ ಆಗುವುದಿಲ್ಲ ಎಂದು ಬಿಟ್ಟು ಬಿಟ್ಟಳು. ನಾಗಮಹಾಶಯ ಸಣ್ಣ ಮೀನಿನಂತೆ. ಅವರನ್ನು ಹಿಡಿಯಲು ಸಣ್ಣ ಕಂಡಿಯ ಬಲೆಯನ್ನು ಬೀಸಿದಳು. ನಾಗಮಹಾಶಯ ಆ ಕಂಡಿಗಿಂತ ಸಣ್ಣದಾಗಿ ತಪ್ಪಿಸಿಕೊಂಡು ಹೋದರು. ಮಾಯೆ ಇನ್ನೂ ಸಣ್ಣದಾಗಿರುವ ಕಂಡಿಯ ಬಲೆಯನ್ನು ಬೀಸಿದಳು. ಕಂಡಿ ಎಷ್ಟು ಸಣ್ಣದಾಗಿದ್ದರೂ ಅದಕ್ಕೂ ಸಣ್ಣದಾಗಿ ನಾಗಮಹಾಶಯರು ತಪ್ಪಿಸಿಕೊಂಡು ಹೊರಟುಹೋಗುತ್ತಿದ್ದರು. ಮಾಯೆ ಅವರನ್ನು ಬಿಟ್ಟುಬಿಟ್ಟಳು. ಇಂತಹ ಎರಡು ವ್ಯಕ್ತಿಗಳು ಪರಸ್ಪರ ಸಂಧಿಸುವುದನ್ನು ನಾವು ಬೇಲೂರು ಮಠದಲ್ಲಿ ನೋಡುತ್ತೇವೆ. ಸ್ವಾಮೀಜಿ ಅಂದು ಮಧ್ಯಾಹ್ನದ ಮೇಲೆ ಮಠದಲ್ಲಿರುವವರಿಗೆ ಪ್ರವಚನ ತೆಗೆದುಕೊಳ್ಳುತ್ತಿದ್ದರು. ನಾಗಮಹಾಶಯರು ಬಂದದ್ದನ್ನು ನೋಡಿ ಪ್ರವಚನವನ್ನು ನಿಲ್ಲಿಸಿ, “ಇವತ್ತು ರಜಾ, ನಾಗಮಹಾಶಯರು ಬಂದಿರುವರು. ನೀವೆಲ್ಲ ಬಂದು ಅವರ ದರ್ಶನಮಾಡಿ ಅವರು ಆಡುವ ಮಾತನ್ನು ಕೇಳಿ” ಎಂದು ಕರೆದರು. ನಾಗಮಹಾಶಯರಿಗೆ ಸ್ವಾಮೀಜಿ ಕೈಮುಗಿದು “ಕ್ಷೇಮವೆಂದು ಭಾವಿಸಿದ್ದೇನೆ” ಎಂದರು. 

\newpage

 ನಾಗಮಹಾಶಯ: “ನಾನಿಂದು ನಿಮ್ಮ ದರ್ಶನಕ್ಕೆ ಬಂದಿರುವೆ. ಜಯಶಂಕರ!\break ಜಯಶಂಕರ! ಇಂದು ಶಿವದರ್ಶನದಿಂದ ಪುನೀತನಾದೆ.” ಎಂದು ಹೇಳುತ್ತ\break ನಾಗಮಹಾಶಯರು ಭಕ್ತಿಯಿಂದ ಕೈಜೋಡಿಸಿ ನಿಂತರು. 

 ಸ್ವಾಮೀಜಿ: “ನಿಮ್ಮ ಆರೋಗ್ಯ ಹೇಗಿದೆ?” 

 ನಾಗಮಹಾಶಯ: “ಈ ಕ್ಷುದ್ರ ದೇಹದ ವಿಷಯವನ್ನೇಕೆ ಎತ್ತುವಿರಿ? ಈ ಮಾಂಸಮೂಳೆಯಿಂದ ಕೂಡಿದ ಪಂಜರ! ಸತ್ಯವಾಗಿ ಇಂದು ನಿಮ್ಮ ದರ್ಶನದಿಂದ ಧನ್ಯನಾದೆ” ಎಂದು ಹೇಳಿ ಅಡ್ಡಬಿದ್ದರು. 

 ಸ್ವಾಮೀಜಿ: (ಅವರನ್ನು ಎಬ್ಬಿಸುತ್ತ) “ನೀವೇಕೆ ನನಗೆ ಹೀಗೆ ಮಾಡುವಿರಿ?” 

 ನಾಗಮಹಾಶಯ: “ನನ್ನ ಅಂತರ್ ದೃಷ್ಟಿಯಿಂದ ಸಾಕ್ಷಾತ್ ಶಿವದರ್ಶನವನ್ನು\break ಮಾಡುತ್ತಿದ್ದೇನೆ. ನಾನು ಧನ್ಯ, ಜಯ ರಾಮಕೃಷ್ಣ.” 

 ಸ್ವಾಮೀಜಿ: (ಶಿಷ್ಯನನ್ನು ಉದ್ದೇಶಿಸಿ) “ನೋಡುತ್ತಿರುವೆಯೇನು? ನಿಜವಾದ ಭಕ್ತಿ ಮಾನವ ಸ್ವಭಾವವನ್ನು ಎಷ್ಟುಮಟ್ಟಿಗೆ ಬದಲಾಯಿಸುವುದು ಎಂಬುದನ್ನು.\break ನಾಗಮಹಾಶಯರು ದೈವೀಭಾವದಲ್ಲಿ ತನ್ಮಯರಾಗಿದ್ದಾರೆ. ಅವರ ದೇಹ ಭಾವನೆ\break ಸಂಪೂರ್ಣವಾಗಿ ಅಳಿಸಿಹೋಗಿದೆ. (ಸ್ವಾಮಿ ಪ್ರೇಮಾನಂದರಿಗೆ) ನಾಗಮಹಾಶಯರಿಗೆ ಕೊಂಚ ಪ್ರಸಾದವನ್ನು ತಾ.” 

 ನಾಗಮಹಾಶಯ: “ಪ್ರಸಾದ! ನಿಮ್ಮನ್ನು ನೋಡಿ ನನ್ನ ಪ್ರಾಪಂಚಿಕ ಕ್ಷುಧೆ ಹಿಂಗಿ ಹೋಗಿದೆ!” 

 ಸ್ವಾಮೀಜಿ: (ಸುತ್ತ ಕುಳಿತಿರುವ ತಮ್ಮ ಶಿಷ್ಯರಿಗೆ) “ನಾಗಮಹಾಶಯರನ್ನು ನೋಡಿ. ಅವರು ಒಬ್ಬ ಗೃಹಸ್ಥರು. ಆದರೂ ಸಂಸಾರದ ಜ್ಞಾನ ಅವರಿಗೆ ಕೊಂಚವೂ ಇಲ್ಲ. ಯಾವಾಗಲೂ ದೈವೀಭಾವದಲ್ಲಿ ಮಗ್ನರಾಗಿರುವರು. (ನಾಗಮಹಾಶಯರಿಗೆ) ದಯವಿಟ್ಟು ನಮಗೆಲ್ಲ ಶ‍್ರೀರಾಮಕೃಷ್ಣರ ವಿಷಯವನ್ನು ಏನಾದರೂ ಹೇಳಿ. 

 ನಾಗಮಹಾಶಯ: “ನಾನೇನು ಹೇಳಲಿ? ನಾನು ನಿಮ್ಮನ್ನು ನೋಡಲು, ಈ ವೀರರನ್ನು, ಶ‍್ರೀರಾಮಕೃಷ್ಣರ ಲೀಲೆಗೆ ಸಹಾಯಕರಾಗಿ ಬಂದ ನಿಮ್ಮನ್ನು, ನೋಡಲು ಬಂದೆ. ಈಗ ಅವರ ಸಂದೇಶ ಬೋಧನೆಗಳನ್ನು ಜನರೆಲ್ಲ ಮೆಚ್ಚುವರು ಜಯ ರಾಮಕೃಷ್ಣ!” 

 ಸ್ವಾಮೀಜಿ: “ಶ‍್ರೀರಾಮಕೃಷ್ಣರನ್ನು ಸರಿಯಾಗಿ ಅರ್ಥಮಾಡಿಕೊಂಡು ಮೆಚ್ಚಿದವರೆಂದರೆ ನೀವೊಬ್ಬರೇ. ನಾವೆಲ್ಲ ಕೇವಲ ನಿರರ್ಥಕವಾದ ಅಲೆದಾಟದಲ್ಲಿ ಕಾಲವನ್ನು ಕಳೆದವರು.” 

 ನಾಗಮಹಾಶಯ: “ಏನು ಮಾತನಾಡುತ್ತೀರಿ? ನೀವು ಶ‍್ರೀರಾಮಕೃಷ್ಣರ ಪ್ರತಿಬಿಂಬ. ನಾಣ್ಯದ ಎರಡು ಮುಖಗಳಂತೆ. ಯಾರಿಗೆ ಕಣ್ಣಿದೆಯೋ ಅವರು ನೋಡಲಿ.” 

 ಸ್ವಾಮೀಜಿ: “ಈ ಮಠ ಮತ್ತು ಆಶ್ರಮಗಳ ಸ್ಥಾಪನೆ ಮುಂತಾದವು ಸರಿಯಾದ ಹಾದಿಯಲ್ಲಿ ಇಟ್ಟಿರುವ ಹೆಜ್ಜೆಯೆ?” 

\newpage

 ನಾಗಮಹಾಶಯ: “ನಾನೊಬ್ಬ ಕ್ಷುದ್ರ ಮನುಷ್ಯ, ನನಗೇನು ಅರ್ಥವಾಗುತ್ತದೆ? ನೀವೇನು ಮಾಡಿದರೂ ಅದರಿಂದ ಜಗತ್ ಕಲ್ಯಾಣವಾಗುವುದು. ಖಂಡಿತವಾಗಿ ಪ್ರಪಂಚಕ್ಕೆ ಒಳ್ಳೆಯದಾಗುವುದು.” 

 ಅನೇಕರು ಪೂಜ್ಯಭಾವದಿಂದ ನಾಗಮಹಾಶಯರ ಪಾದಧೂಳಿಯನ್ನು\break ತೆಗೆದುಕೊಳ್ಳಲು ಹತ್ತಿರ ಬಂದರು. ಇದರಿಂದ ಅವರು ಮತ್ತಷ್ಟು ಕಳವಳಗೊಂಡರು. ಸ್ವಾಮೀಜಿ ಅವರನ್ನು ಉದ್ದೇಶಿಸಿ “ನಾಗಮಹಾಶಯರಿಗೆ ನೋವುಂಟುಮಾಡಬೇಡಿ. ಅವರು ವ್ಯಥಿತರಾಗುವಂತೆ ಮಾಡಬೇಡಿ” ಎಂದರು. ಇದನ್ನು ಕೇಳಿದ ಮೇಲೆ ಎಲ್ಲರೂ ಮೌನವಾದರು. 

 ಸ್ವಾಮೀಜಿ: “ದಯವಿಟ್ಟು ಮಠಕ್ಕೆ ಆಗಾಗ್ಗೆ ಬಂದು ಹೋಗುತ್ತಿರಿ. ಇಲ್ಲಿನ ಹುಡುಗರಿಗೆಲ್ಲ ಮೇಲ್ಪಂಕ್ತಿಯಾಗಿರಿ.” 

 ನಾಗಮಹಾಶಯ: “ನಾನು ಒಮ್ಮೆ ಶ‍್ರೀರಾಮಕೃಷ್ಣರನ್ನು ಈ ವಿಷಯವಾಗಿ ಕೇಳಿದೆ. ಅವರು ನೀನು ಈಗ ಇರುವಂತೆಯೇ ಗೃಹಸ್ಥನಾಗಿರು ಎಂದರು. ಅದರಂತಯೇ ನಾನು ಅವರ ಉಪದೇಶವನ್ನೇ ಅನುಸರಿಸುತ್ತೇನೆ.” 

 ಸ್ವಾಮೀಜಿ: “ಒಮ್ಮೆ ನಾನು ನಿಮ್ಮ ಸ್ಥಳಕ್ಕೆ ಬರಬೇಕೆಂದು ಇರುವೆ.” 

 ನಾಗಮಹಾಶಯ: (ಆನಂದದಿಂದ ಉನ್ಮತ್ತರಾಗಿ) “ಅಂತಹ ಸುದಿನ ನಿಜವಾಗಿ ಬರುವುದೆ? ನನ್ನ ಊರು ಕಾಶಿಯಂತೆ ನಿಮ್ಮ ಪಾದಧೂಳಿಯಿಂದ ಪವಿತ್ರವಾಗುವುದು. ನಾನು ಅಷ್ಟೊಂದು ಅದೃಷ್ಟಶಾಲಿಯೇ?” 

 ಸ್ವಾಮಿಜಿ: “ನನಗೂ ಆಸೆಯಿದೆ. ಇನ್ನು ನಾನಲ್ಲಿಗೆ ಬರುವುದು ಮಾತೆಯ ಇಚ್ಛೆ.” 

 ನಾಗಮಹಾಶಯ: “ಯಾರಿಗೆ ನಿಮ್ಮನ್ನು ತಿಳಿಯಲು ಸಾಧ್ಯ? ಅಂತರ್‍ದೃಷ್ಟಿ ತೆರೆದ ಹೊರತು ಯಾರಿಗೂ ನಿಮ್ಮನ್ನು ಅರ್ಥಮಾಡಿಕೊಳ್ಳಲು ಸಾಧ್ಯವಿಲ್ಲ. ಕೇವಲ\break ಶ‍್ರೀರಾಮಕೃಷ್ಣರು ಮಾತ್ರ ನಿಮ್ಮನ್ನು ಅರಿತಿದ್ದರು. ಉಳಿದವರೆಲ್ಲ ಅವರ ಮಾತುಗಳಲ್ಲಿ ಶ್ರದ್ಧೆ ಇಟ್ಟಿದ್ದರು. ಆದರೆ ಯಾರೂ ನಿಮ್ಮನ್ನು ಸರಿಯಾಗಿ ಅರ್ಥಮಾಡಿಕೊಂಡಿಲ್ಲ.” 

 ಸ್ವಾಮೀಜಿ: “ಈಗ ನನಗಿರುವ ಒಂದು ಆಸೆಯೆಂದರೆ, ಇಡೀ ದೇಶವನ್ನು ನಿದ್ರಿಸುತ್ತಿರುವ ಈ ರಾಕ್ಷಸಾಕಾರನನ್ನು, ತನ್ನ ಶಕ್ತಿಯಲ್ಲಿರುವ ನಂಬಿಕೆಯನ್ನೆಲ್ಲ ಕಳೆದುಕೊಂಡು ಪ್ರತಿಕ್ರಿಯೆಯೆ ಇಲ್ಲದೆ ಬಿದ್ದಿರುವ ಇದನ್ನು ಎಬ್ಬಿಸಬೇಕು ಎಂಬುದು. ನಾನು ಇದನ್ನು ಸನಾತನ ಧರ್ಮದ ಆದರ್ಶದ ದೃಷ್ಟಿಯಿಂದ ಜಾಗ್ರತಗೊಳಿಸಬಲ್ಲೆನಾದರೆ ನನ್ನ ಜನ್ಮ ಮತ್ತು ಶ‍್ರೀರಾಮಕೃಷ್ಣರ ಅವತಾರವೂ ಸಾರ್ಥಕವೆಂದು ಭಾವಿಸುತ್ತೇನೆ. ಇದೊಂದೇ ನನ್ನ ಹೃದಯದ ತೀವ್ರ ಆಕಾಂಕ್ಷೆ. ಮುಕ್ತಿ ಮುಂತಾದವುಗಳಿಂದ ಯಾವ ಪ್ರಯೋಜನವೂ ನನಗೆ ಕಾಣುವುದಿಲ್ಲ. ದಯವಿಟ್ಟು ನಾನು ಇದರಲ್ಲಿ ಜಯಶಾಲಿಯಾಗುವಂತೆ ಹರಸಿ.” 

 ನಾಗಮಹಾಶಯ: “ಶ‍್ರೀ ರಾಮಕೃಷ್ಣರು ಆಶೀರ್ವದಿಸುವರು. ನಿಮ್ಮ ಇಚ್ಛೆಯನ್ನು ತಿರುಗಿಸಬಲ್ಲವರಾರು? ನೀವು ಇಚ್ಛೆ ಪಟ್ಟಿದ್ದೆಲ್ಲ ಕಾರ‍್ಯರೂಪಕ್ಕೆ ಬಂದೇ‌ ಬರುವುದು.” 

 ಸ್ವಾಮೀಜಿ: “ಆತನ ಇಚ್ಛೆ ಇಲ್ಲದೇ ಇದ್ದರೆ ಯಾವುದೂ ಸಂಭವಿಸುವುದಿಲ್ಲ.” 

 ನಾಗಮಹಾಶಯ:‌“ಆತನ ಇಚ್ಛೆ ಮತ್ತು ನಿಮ್ಮ ಇಚ್ಛೆ ಎರಡೂ ಒಂದೇ ಆಗಿದೆ. ನಿಮ್ಮ ಇಚ್ಛೆ ಯಾವುದೋ ಅದೇ ಆತನ ಇಚ್ಛೆ. ಜಯ ಶ‍್ರೀರಾಮಕೃಷ್ಣ!” 

 ಸ್ವಾಮೀಜಿ: “ಕೆಲಸ ಮಾಡಲು ದೃಢಕಾಯವಾದ ದೇಹ ಆವಶ್ಯಕ. ಈ ದೇಶಕ್ಕೆ ಬಂದಾಗಿನಿಂದಲೂ ನನಗೆ ಆರೋಗ್ಯ ಸರಿಯಿಲ್ಲ. ಪಶ್ಚಿಮ ದೇಶದಲ್ಲಿ ನನ್ನ ಆರೋಗ್ಯ ಚೆನ್ನಾಗಿತ್ತು.” 

 ನಾಗಮಹಾಶಯ: “ಈ ಶರೀರಧಾರಣೆ ಮಾಡಿದವರೆಲ್ಲರೂ ಅದರ ಬಾಡಿಗೆಯ ಸುಂಕವನ್ನು ತೆತ್ತೇ ತೀರಬೇಕು. ರೋಗ ಮತ್ತು ದುಃಖವೇ ಅದರ ಸುಂಕ. ಆದರೆ ನಿಮ್ಮ ದೇಹ ಚಿನ್ನದ ಮೊಹರಿರುವ ಪೆಟ್ಟಿಗೆ. ಅದರ ವಿಷಯದಲ್ಲಿ ಬಹಳ ಜೋಪಾನವಾಗಿರಬೇಕು. ಹಾಗೆ ನೋಡಿಕೊಳ್ಳುವವರು ಯಾರು? ಯಾರಿಗೆ ಅರ್ಥಮಾಡಿಕೊಳ್ಳಲು ಸಾಧ್ಯ? ಶ‍್ರೀರಾಮಕೃಷ್ಣರು ಮಾತ್ರ ಅದನ್ನು ಅರ್ಥಮಾಡಿಕೊಳ್ಳುತ್ತಿದ್ದರು. ಜೈ ಶ‍್ರೀರಾಮಕೃಷ್ಣ!” 

 ಸ್ವಾಮೀಜಿ: “ಮಠದಲ್ಲಿರುವವರೆಲ್ಲ ನನ್ನ ವಿಷಯದಲ್ಲಿ ಬಹಳ ಎಚ್ಚರಿಕೆ ತೆಗೆದುಕೊಳ್ಳುತ್ತಾರೆ.” 

 ನಾಗಮಹಾಶಯ: “ಅವರು ಅದನ್ನು ಮಾಡಿದರೆ, ಅವರು ತಿಳಿಯಲಿ ಇಲ್ಲದೆ ಇರಲಿ ಅದೆಲ್ಲ ಅವರ ಶ್ರೇಯಸ್ಸಿಗೆ. ನಿಮ್ಮ ದೇಹಕ್ಕೆ ತಕ್ಕ ಗಮನ ಕೊಡದಿದ್ದಲ್ಲಿ ದೇಹ ಬಲುಬೇಗ ಬಿದ್ದುಹೋಗುವುದು.” 

 ಸ್ವಾಮೀಜಿ: “ನಾಗಮಹಾಶಯರೆ! ನಾನು ಮಾಡುತ್ತಿರುವುದು ಸರಿಯೋ ತಪ್ಪೊ ಎಂದು ನನಗೆ ಸಂಪೂರ್ಣ ಗೊತ್ತಾಗಿಲ್ಲ. ಕೆಲವು ನಿರ್ದಿಷ್ಟ ಸಮಯಗಳಲ್ಲಿ ನನಗೆ ಯಾವುದೊ ಒಂದು ಮಾರ್ಗದಲ್ಲಿ ಹೋಗುವ ತೀವ್ರ ಇಚ್ಛೆಯುಂಟಾಗುತ್ತದೆ. ಅದರಂತೆಯೇ ನಾನು ಮಾಡುತ್ತೇನೆ. ಅದು ಒಳ್ಳೆಯದಕ್ಕೊ ಕೆಟ್ಟದಕ್ಕೊ ನನಗೆ ಅರ್ಥವಾಗುವುದಿಲ್ಲ.” 

 ನಾಗಮಹಾಶಯ: “ ಶ‍್ರೀರಾಮಕೃಷ್ಣರು, ‘ನಿಧಿಗೆ ಈಗ ಬೀಗಮುದ್ರೆ ಹಾಕಲ್ಪಟ್ಟಿದೆ’ ಎಂದು ಹೇಳುತ್ತಿದ್ದರು. ಅದಕ್ಕೇ ಅವರು ನೀವು ಪೂರ್ತಿ ತಿಳಿದುಕೊಳ್ಳಲು ಬಿಡುವುದಿಲ್ಲ. ಯಾವ ಘಳಿಗೆ ನೀವು ಅದನ್ನು ತಿಳಿಯುವಿರೊ ಆಗ ನಿಮ್ಮ ಮಾನವ ಜನ್ಮದ ಲೀಲೆ ಕೊನೆಗೊಳ್ಳುವುದು.” 

 ಸ್ವಾಮೀಜಿ ಏನನ್ನೋ ಎವೆಯಿಕ್ಕದೆ ಯೋಚಿಸುತ್ತಿದ್ದರು. ಪ್ರೇಮಾನಂದರು ಸ್ವಲ್ಪ ಪ್ರಸಾದವನ್ನು ಪ್ರೇಮೋನ್ಮಾದದಲ್ಲಿ ಮುಳುಗಿದ್ದ ನಾಗಮಹಾಶಯರಿಗೆ ನೀಡಿದರು. ಸ್ವಲ್ಪ ಕಾಲಾನಂತರ ನಾಗಮಹಾಶಯರು, ಸ್ವಾಮೀಜಿ ಕೊಳದ ಹತ್ತಿರ ನೆಲವನ್ನು ಗುದ್ದಲಿಯಿಂದ\break ನಿಧಾನವಾಗಿ ಅಗೆಯುತ್ತಿದ್ದುದನ್ನು ನೋಡಿ, ಅವರ ಕೈಹಿಡಿದು “ನಾವಿರುವಾಗ ನೀವೇಕೆ ಇದನ್ನು ಮಾಡಬೇಕು” ಎಂದು ತಡೆದರು. ಸ್ವಾಮೀಜಿ ಗುದ್ದಲಿಯನ್ನು ಅಲ್ಲೇ ಬಿಟ್ಟು ತೋಟದಲ್ಲಿ ಸ್ವಲ್ಪ ಹೊತ್ತು ಸುತ್ತಾಡುತ್ತ ಶಿಷ್ಯನಿಗೆ ಹೇಳಿದರು: “ಶ‍್ರೀರಾಮಕೃಷ್ಣರ ನಿರ‍್ಯಾಣಾನಂತರ ನಾಗಮಹಾಶಯರು ತಮ್ಮ ಬಡಕುಟೀರದಲ್ಲಿ ಉಪವಾಸ ಮಲಗಿರುವರೆಂದು ಒಂದು ದಿನ ಕೇಳಿದೆವು. ನಾನು ತುರೀಯಾನಂದ ಮತ್ತು ಇನ್ನು ಕೆಲವರು ಒಟ್ಟಿಗೆ ಆ‌ ಕುಟೀರಕ್ಕೆ ಹೋದೆವು. ನಮ್ಮನ್ನು ನೋಡಿದೊಡನೆ ಅವರು ಹಾಸಿಗೆಯಿಂದ ಎದ್ದರು. ನಾವಿಂದು ಭಿಕ್ಷೆಯನ್ನು ಇಲ್ಲೇ ತೆಗೆದುಕೊಳ್ಳುತ್ತೇವೆ ಎಂದೆವು. ತಕ್ಷಣ ನಾಗಮಹಾಶಯರು ಅಕ್ಕಿ ತಪ್ಪಲೆ ಸೌದೆ ಮುಂತಾದವನ್ನೆಲ್ಲ ಅಂಗಡಿಯಿಂದ ಕೊಂಡು ತಂದು ಅಡಿಗೆ ಮಾಡಲಾರಂಭಿಸಿದರು. ನಾವು ಯೋಚಿಸಿದ್ದು ನಾವು ಊಟ ಮಾಡೋಣ, ನಾಗಮಹಾಶಯರಿಗೂ ಮಾಡಿಸೋಣ ಎಂದು. ಅಡಿಗೆ ಮಾಡಿ ನಮಗೆ ಬಡಿಸಿದರು. ನಾವು ಸ್ವಲ್ಪ ಭಾಗವನ್ನು ಬೇರೆ ತೆಗೆದಿರಿಸಿ ಊಟಮಾಡಲು ಕುಳಿತೆವು. ಅನಂತರ ಅವರನ್ನು ಊಟಮಾಡುವಂತೆ ಕೇಳಿದೆವು. ಅವರು ತಕ್ಷಣ ಮಡಿಕೆಯನ್ನು ಒಡೆದು ತಲೆಯನ್ನು ಚಚ್ಚಿಕೊಳ್ಳುತ್ತ “ದೈವಸಾಕ್ಷಾತ್ಕಾರವಾಗದ ಶರೀರಕ್ಕೆ ಆಹಾರವನ್ನು ಕೊಡಲೆ?” ಎಂದರು. ಇದನ್ನು ನೋಡಿ ನಾವು ಸ್ತಂಭೀಭೂತರಾದೆವು. ಸ್ವಲ್ಪ ಕಾಲಾನಂತರ ತುಂಬ ಬಲವಂತ ಮಾಡಿ ಅವರಿಗೆ ಸ್ವಲ್ಪ ಊಟ ಮಾಡಿಸಿ ಹಿಂತಿರುಗಿದೆವು. 

 ಸ್ವಾಮೀಜಿ: “ನಾಗಮಹಾಶಯರು ಈ ರಾತ್ರಿ ಮಠದಲ್ಲೇ ಉಳಿಯುವರೆ?” 

 ಶಿಷ್ಯ: “ಇಲ್ಲ, ಅವರಿಗೆ ಸ್ವಲ್ಪ ಕೆಲಸವಿದೆ. ಅವರು ಇಂದೇ ಹಿಂತಿರುಗಬೇಕು.” 

 ಸ್ವಾಮೀಜಿ: “ಹಾಗಾದರೆ ದೋಣಿಯನ್ನು ನೋಡು. ಆಗಲೇ ಕತ್ತಲಾಗುತ್ತಿದೆ.”\break ದೋಣಿಯು ಬಂದೊಡನೆಯೆ ಶಿಷ್ಯನೂ ನಾಗಮಹಾಶಯರೂ ಸ್ವಾಮಿಗಳಿಗೆ ನಮಸ್ಕಾರ ಮಾಡಿ ಕಲ್ಕತ್ತೆಗೆ ಹೊರಟರು. 

