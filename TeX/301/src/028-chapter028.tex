
\chapter{ಇಂಗ್ಲೆಂಡಿಗೆ}

 ಸ್ವಾಮೀಜಿ ಅವರು ಸಹಸ್ರದ್ವೀಪೋದ್ಯಾನದಿಂದ ಬಂದು ನ್ಯೂಯಾರ್ಕಿನಲ್ಲಿದ್ದರು.\break ಬ್ರಿಟಿಷ್ ಚಕ್ರಾಧಿಪತ್ಯದ ರಾಜಧಾನಿಯಾದ ಇಂಗ್ಲೆಂಡಿಗೆ ಹೋಗಿ ಅಲ್ಲಿ ವೇದಾಂತ ಸಂದೇಶದ ಬೀಜಗಳನ್ನು ಬಿತ್ತಬೇಕೆಂದು ಅವರಿಗೆ ಆಸೆ ಇತ್ತು. ಏಕೆಂದರೆ ಆ ಚಕ್ರಾಧಿಪತ್ಯ ವಿಶ್ವವ್ಯಾಪಿಯಾಗಿದೆ. ಯಾವ ಭಾವನೆ ಅಲ್ಲಿ ಬೇರೂರುವುದೊ ಅದು ಕಾಲ ಕ್ರಮೇಣ ಆ ಚಕ್ರಾಧಿಪತ್ಯದಲ್ಲೆಲ್ಲಾ ಹರಡುವುದು. ಭಾವನೆಯನ್ನು ಪ್ರಸಾರ ಮಾಡುವುದಕ್ಕೆ ಬಿಟಿಷ್ ಚಕ್ರಾಧಿಪತ್ಯ ಒಂದು ಪ್ರಚಂಡ ಯಂತ್ರದಂತೆ ಇದೆ. ಈ ಉದ್ದೇಶದ ಜೊತೆಗೆ ಲಂಡನ್ನಿನಲ್ಲಿರುವ ಮಿಸ್ ಹೆನ್‍ರಿಟಾ ಮುಲ್ಲರ್ ಎಂಬಾಕೆ ಸ್ವಾಮೀಜಿಯವರಿಗೆ ಒಂದು ಪತ್ರ ಬರೆದು ಲಂಡನ್ನಿಗೆ ಬಂದು ಅಲ್ಲಿ ತಮ್ಮ ಅತಿಥಿಗಳಾಗಿರಬೇಕು ಎಂದು ಕೋರಿಕೊಂಡರು. ಹಿಂದೂಗಳ ವಿಷಯವಾಗಿ ಬಹಳ ಉದಾರ ಭಾವನೆಯನ್ನೊಳಗೊಂಡ ಮತ್ತು ಇಂಡಿಯಾ ದೇಶಕ್ಕೆ ಹೋಗಿ ಕೆಲವು ಕಾಲ ಸಾಧನೆ ಮಾಡಿ ಬಂದಿದ್ದ ಶ‍್ರೀ ಇ.ಟಿ. ಸ್ಟರ‍್ಡಿ ಎಂಬುವರು ಕೂಡ ಲಂಡನ್ನಿನಲ್ಲಿ ತಮ್ಮ ಮನೆಯಲ್ಲಿ ಇರಬೇಕೆಂದು ಕೋರಿಕೊಂಡು ಒಂದು ಕಾಗದವನ್ನು ಬರೆದರು. ಇಷ್ಟು ಹೊತ್ತಿಗೆ ಸ್ವಾಮೀಜಿ ಅವರ ನ್ಯೂಯಾರ್ಕಿನ ಭಕ್ತರೊಬ್ಬರು ಪ್ಯಾರಿಸ್ಸಿಗೆ ಹೊರಡುವುದರಲ್ಲಿದ್ದರು. ಅವರು ಸ್ವಾಮೀಜಿಯವರನ್ನು ಪ್ಯಾರಿಸ್ಸಿಗೆ ಕರೆದುಕೊಂಡು ಹೋಗಿ ಅಲ್ಲಿಂದ ಅವರನ್ನು ಲಂಡನ್ನಿಗೆ ಬಿಡುವೆನು ಎಂದರು. ಅವರೊಡನೆ ಯೂರೋಪಿಗೆ ಹೋಗುವುದಕ್ಕೆ ಅಣಿಯಾಗುತ್ತಿದ್ದರು. 

 ಇತ್ತ ಅಮೇರಿಕಾ ದೇಶದಲ್ಲಿ ಸಂಪ್ರದಾಯಬದ್ಧ ಕ್ರೈಸ್ತರು ಸ್ವಾಮಿಗಳ ಮೇಲೆ ಬಹಳ ಕೋಪಗೊಂಡರು. ಸ್ವಾಮೀಜಿ ಹಿಂದೂಧರ್ಮದ ಪುನರ್ಜನ್ಮ ಸಿದ್ಧಾಂತ, ಕರ್ಮ ಸಿದ್ಧಾಂತ ಮತ್ತು ಆತ್ಮದ ಅಮರತ್ವವನ್ನು ಬೋಧಿಸುತ್ತಾ ಹೋದಂತೆ ಕ್ರೈಸ್ತ ಧರ್ಮದ ತಾತ್ವಿಕ ಹುಳುಕುಗಳು ಎದ್ದು ಕಾಣತೊಡಗಿದವು. ವಿಚಾರದ ಬಿರುಗಾಳಿಗೆ ನಿಲ್ಲಬಲ್ಲ ಶಕ್ತಿ ಆ ಗೊಡ್ಡು ನಂಬಿಕೆಗಳಿಗೆ ಇರಲಿಲ್ಲ. ತಮ್ಮ ಧರ್ಮವನ್ನೇ ಸ್ವಾಮಿಗಳು ನಿರ್ಮೂಲ ಮಾಡತೊಡಗಿದರು ಎಂದು ಪಾದ್ರಿಗಳು ಭಾವಿಸಿ ಅಮೇರಿಕಾ ದೇಶದಲ್ಲಿ ಹಲವು ಕಡೆ ಸ್ವಾಮೀಜಿ ಶೀಲದ ಮೇಲೆ ಅಪಪ್ರಚಾರ ಮಾಡತೊಡಗಿದರು. ಆದರೆ ಅವರ ಅಪಪ್ರಚಾರ ಬಹಳ ಯಶಸ್ವಿಯಾಗಲಿಲ್ಲ. ಕೆಲವರು ಹಣದಾಸೆಗೆ ಪಾದ್ರಿಗಳು ಹೇಳಿದಂತೆ ಅಪಪ್ರಚಾರ ಮಾಡತೊಡಗಿದರು. ಅದಕ್ಕೆ ಹತ್ತರಷ್ಟು ಹೆಚ್ಚುಮಂದಿ ಸ್ವಾಮೀಜಿ ಪರವಾಗಿ ನಿಂತು ಅವರ ಪವಿತ್ರ ಜೀವನಕ್ಕೆ ಸಾಕ್ಷಿಯಾಗಿದ್ದರು. ಕ್ರೈಸ್ತ ಪಾದ್ರಿಗಳು ಅಮೇರಿಕಾ ದೇಶದಲ್ಲಿ ಅಪಪ್ರಚಾರ ಮಾಡಿದ್ದು ಅಲ್ಲದೆ ಇಂಡಿಯಾ ದೇಶದಲ್ಲಿಯೂ ತಮ್ಮ ಮತೀಯ ಪತ್ರಿಕೆಗಳಲ್ಲಿ ಸ್ವಾಮೀಜಿಯವರ ಮೇಲೆ ಅಪಪ್ರಚಾರ ಮಾಡತೊಡಗಿದರು. ಇಂಡಿಯಾ ದೇಶದಲ್ಲಿ ಅದನ್ನು ಓದಿದ ಕೆಲವು ಭಕ್ತರು ಅಂಜಿ ಸ್ವಾಮೀಜಿಗಳಿಗೆ ಕಾಗದ ಬರೆದರು. ಆ ಸಮಯದಲ್ಲಿ ಆ ಭಕ್ತನಿಗೆ ಬರೆದ ಕಾಗದದಲ್ಲಿ ಸ್ವಾಮೀಜಿಯವರ ದುರ್ದಮ್ಯ ಚೈತನ್ಯವನ್ನು ನೋಡುತ್ತೇವೆ. ಅವರು ತಮ್ಮನ್ನು ಸಹಸ್ರಾರು ಜನ ಹೊಗಳುತ್ತಿದ್ದಾಗ ಆ\break ಕೀರ್ತಿಯನ್ನು ಹೇಗೆ ಅರಗಿಸಿಕೊಂಡರೊ, ಹಾಗೆಯೇ ಹಲವು ಜನ ತಮ್ಮ ವಿರೋಧವಾಗಿ ಅಪಪ್ರಚಾರ ಮಾಡುತ್ತಿದ್ದಾಗಲೂ ಅದರ ಭಾರಕ್ಕೆ ಕುಗ್ಗಿಹೋಗಲಿಲ್ಲ. ಅದನ್ನು ಧೈರ್ಯವಾಗಿ ಎದುರಿಸಿದರು. ಸತ್ಯ ಮತ್ತು ದೇವರನ್ನು ಹೃತ್ಪೂರ್ವಕ ನಂಬಿದವರಿಗೆ ಮಾತ್ರ ಇಂತಹ ಶಕ್ತಿ ಸಾಧ್ಯ. ಆ ಪಾದ್ರಿಗಳ ಜೊತೆಗೆ ಪೂರ್ವಾಚಾರ ಪರಾಯಣರಾದ ಹಿಂದೂಗಳೂ ಸೇರಿ ಸ್ವಾಮೀಜಿ ನಿಷಿದ್ಧ ಆಹಾರಗಳನ್ನೆಲ್ಲ ತೆಗೆದುಕೊಳ್ಳುತ್ತಿರುವರೆಂದೂ, ಮಡಿ ಆಚಾರ ಯಾವುದನ್ನೂ ಪರಿಪಾಲಿಸುತ್ತಿಲ್ಲವೆಂದೂ ಬೊಬ್ಬೆ ಹೊಡೆಯಲು ಆರಂಭಿಸಿದರು. ಭರತಖಂಡದಿಂದ ಒಬ್ಬ ಹೋಗಿ ಹಿಂದೂಧರ್ಮಕ್ಕೆ ಅವನು ಗೌರವ ತಂದಿರುವನು, ಆತ ಏಕಾಂಗಿಯಾಗಿ ಹೋರಾಡುತ್ತಿರುವನು, ಅವನ ನೆರವಿಗೆ ನಾವು ಹೋಗಬೇಕೆಂಬುದನ್ನೂ ಈ ಜನ ಮರೆತು, ಸ್ವಾಮೀಜಿಯವರನ್ನು ಅಲ್ಲಗಳೆಯುತ್ತಿದ್ದ ಜನರೊಡನೆ ಬಾಂಧವ್ಯ ಬೆಳಸಿ ಟೀಕಿಸಲು ಮೊದಲು ಮಾಡಿದರು. ಸ್ವಾಮೀಜಿಯವರು ಅನ್ಯಧರ್ಮದವರ ಮೇಲೆ ಮಾತ್ರ ಹೋರಾಡುವುದಲ್ಲ, ತಮ್ಮ ಧರ್ಮದ ಪೂರ್ವಾಚಾರ ಪರಾಯಣರ ಮೇಲೆಯೂ ಹೋರಾಡಬೇಕಾಗುವುದು. ಸ್ವಾಮೀಜಿ ಸ್ವಭಾವ ಸೋಲನ್ನು ಒಪ್ಪಿಕೊಳ್ಳುವುದಿಲ್ಲ. ಇಡೀ ಜಗತ್ತಿಗೆ ವಿರೋಧವಾಗಿ ನಿಂತು ಏಕಾಂಗಿಯಾಗಿ ತಾವು ಹೋರಾಡಬಲ್ಲರು. ಏಕೆಂದರೆ ಅವರ ಹಿಂದೆ ಸತ್ಯವಿದೆ, ದೇವರು ಅವರನ್ನು ಹಿಡಿದುಕೊಂಡಿರುವನು ಎಂಬ ಭರವಸೆ ಧೃವತಾರೆಯಂತೆ ಯಾವಾಗಲೂ ಮಿರುಗುತ್ತಿತ್ತು. ಆ ಸಮಯದಲ್ಲಿ ಬರೆದ ಒಂದು ಪತ್ರವನ್ನು ನೋಡಿದರೆ ಅವರ ಸತ್ತ್ವದ ಕೆಚ್ಚು ಏನು ಎಂಬುದು ಅನುಭವವಾಗುವುದು: 

 “ಪಾದ್ರಿಗಳು ಹೇಳುವ ಕೆಲಸಕ್ಕೆ ಬಾರದ ವಿಷಯಗಳಿಗೆಲ್ಲ ನೀನು ಅಷ್ಟು ಮಹತ್ವವನ್ನು ಕೊಡುವುದನ್ನು ನೋಡಿ ನನಗೆ ಆಶ್ಚರ‍್ಯವಾಗಿದೆ, ಇಂಡಿಯಾದೇಶದ ಜನರು ಹಿಂದೂ ರೀತಿಯ ಭೋಜನವನ್ನು ನನಗೆ ತಪ್ಪದೇ ಅನುಸರಿಸಬೇಕೆಂದು ಹೇಳಿದರೆ, ದಯವಿಟ್ಟು ಒಬ್ಬ ಅಡಿಗೆಯವನನ್ನು ಮತ್ತು ಅವನನ್ನು ನೋಡಿಕೊಳ್ಳಲು ಸಾಕಷ್ಟು ಹಣವನ್ನು ಕಳುಹಿಸುವಂತೆ ಹೇಳು. ಎಳ್ಳಷ್ಟೂ ಸಹಾಯ ಮಾಡದೆ ಬೇಕಾದಷ್ಟು ಬುದ್ಧಿವಾದವನ್ನು ನೋಡಿ ನನಗೆ ನಗು ಬರುತ್ತದೆ. ಅದಲ್ಲದೆ ಪಾದ್ರಿಗಳು ಸನ್ಯಾಸಧರ್ಮದ ಎರಡು ನಿಯಮಗಳಾದ ಬಡತನ ಮತ್ತು ಬ್ರಹ್ಮಚರ‍್ಯ ಇವುಗಳಿಂದ ನಾನು ಚ್ಯುತನಾಗಿರುವೆನೆಂದು ಹೇಳಿದರೆ, ಅವುಗಳೆಲ್ಲ ಅಪ್ಪಟ ಸುಳ್ಳು ಎಂದು ಹೇಳು. ದಯವಿಟ್ಟು ಆ ಮಿಶಿನರಿಗೆ ನನ್ನ ನಡತೆಯಲ್ಲಿ ಯಾವ ವಿಧವಾದ ಅನುಮಾನವನ್ನು ಆತನು ನೋಡಿದನು ಅಥವಾ ಈ ವಿಚಾರವಾಗಿ ಆತನಿಗೆ ಸುದ್ದಿ ಕೊಟ್ಟವರು ಯಾರು ಈ ವಿಷಯವನ್ನು ತಿಳಿಸುವಂತೆ ಹೇಳು. ಈ ಸಮಾಚಾರವನ್ನು ಪ್ರತ್ಯೇಕವಾಗಿ ನೋಡಿದವನು ಹೇಳಿದುದೇ ಎಂಬುದನ್ನು ಖಂಡಿತವಾಗಿ ಪ್ರಶ್ನೆಮಾಡಿ ಕಾಗದ ಬರೆ. ಅದು ಪ್ರಶ್ನೆಯನ್ನು ಬಗೆಹರಿಸುವುದು, ಗುಟ್ಟೆಲ್ಲವನ್ನೂ ರಟ್ಟುಮಾಡುವುದು.” 

 “ನಾನಾದರೋ‌ ಯಾರ ಅಣತಿಯನ್ನೂ ಪಾಲಿಸುವವನಲ್ಲ ಎಂಬುದನ್ನು ಜ್ಞಾಪಕದಲ್ಲಿಡು. ನನ್ನ ಜೀವನದ ಉದ್ದೇಶವೇನೆಂಬುದು ನನಗೆ ಗೊತ್ತಿದೆ. ನನ್ನ ವಿಷಯದಲ್ಲಿ ಯಾವ ಕಪಟತೆಯೂ ಇಲ್ಲ. ನಾನು ಎಷ್ಟು ಮಟ್ಟಿಗೆ ಇಂಡಿಯಾ ದೇಶಕ್ಕೆ ಸೇರಿದವನೋ ಅಷ್ಟೇ ಮಟ್ಟಿಗೆ ವಿಶ್ವಕ್ಕೆ ಸೇರಿದವನು. ಈ ವಿಚಾರವಾಗಿ ಏನೂ ಮೋಸವಿಲ್ಲ. ಸಾಧ್ಯವಾದ ಮಟ್ಟಿಗೆ ನಿಮಗೆಲ್ಲ ನಾನು ಸಹಾಯ ಮಾಡಿರುವೆನು. ಈಗ ನಿಮ್ಮನ್ನು ನೀವೇ ಕಾಪಾಡಿಕೊಳ್ಳಬೇಕು. ನನ್ನ ಮೇಲೆ ಯಾವ ದೇಶಕ್ಕೆ ತಾನೆ ಅಧಿಕಾರವಿದೆ? ನಾನು ಯಾವುದಾದರೊಂದು ಜನಾಂಗದ ಗುಲಾಮನೆ? ಶ್ರದ್ಧಾಹೀನರಾದ ನಾಸ್ತಿಕರೆ! ಕೆಲಸಕ್ಕೆ ಬರದ ಮಾತನ್ನು ಹೆಚ್ಚು ಆಡಬೇಡಿ. ನಾನು ಕಷ್ಟಪಟ್ಟು ಕೆಲಸಮಾಡಿ ಸಿಕ್ಕಿದ ಹಣವನ್ನೆಲ್ಲ ಮದ್ರಾಸು ಮತ್ತು ಕಲ್ಕತ್ತೆಗೆ ಕಳುಹಿಸಿರುವೆನು. ಇಷ್ಟನ್ನೆಲ್ಲ ಮಾಡಿದಮೇಲೆ ಈ ಮೂರ್ಖರ ಮಾತನ್ನು ಕೇಳಬೇಕೇನು? ನಿಮಗೆ ನಾಚಿಕೆ ಆಗುವುದಿಲ್ಲವೆ? ನಾನು ಅವರಿಗೆ ಏನನ್ನು ಕೊಡಬೇಕಾಗಿದೆ? ಏನು ಅವರ ಕೀರ್ತಿಗೆ ನಾನು ಆಸೆ ಪಡುತ್ತೇನೆಯೆ? ಅವರ ನಿಂದೆಗೆ ಅಂಜುವೆನೆ? ನನ್ನ ಮಗು, ನಾನು ಏಕಾಂಗಿ; ನೀನು ಕೂಡ ನನ್ನನ್ನು ತಿಳಿದುಕೊಂಡಿಲ್ಲ. ನಿನ್ನ ಕೆಲಸವನ್ನು ಮಾಡು. ಸಾಧ್ಯವಿಲ್ಲದೇ ಇದ್ದರೆ ನಿಲ್ಲಿಸು. ನಿನ್ನ ಕೆಲಸಕ್ಕೆ ಬಾರದ ಕಾಡುಹರಟೆಯಿಂದ ನನ್ನನ್ನು ಆಳಲು ಯತ್ನಿಸಬೇಡ. ಮಾನವನಿಗಿಂತ, ದೇವತೆಗಿಂತ, ಭೂತಗಳಿಗಿಂತ, ಯಾವುದೋ ಬಲವಾದ ಶಕ್ತಿ ನನ್ನ ಹಿಂದೆ ಇರುವುದು. ನನಗೆ ಯಾರ ಸಹಾಯವೂ ಬೇಕಾಗಿಲ್ಲ. ಇದುವರೆಗೆ ನಾನು ಇನ್ನೊಬ್ಬರಿಗೆ ಸಹಾಯವನ್ನು ಮಾಡುತ್ತಿರುವೆನು. ಯಾರು ಎಳ್ಳಷ್ಟೂ ಸಹಾಯ ಮಾಡುವುದಿಲ್ಲವೋ ಅವರಿಗೆ ನಾನು ಸಾಧ್ಯವಾದಷ್ಟು ಮಾಡಿರುವೆನು. ಅಂತಹವನ ಮೇಲೆ ತಮ್ಮ ಅಧಿಕಾರವನ್ನು ಚಲಾಯಿಸಲು ಬರುವವರು ಎಷ್ಟು ಕೃತಘ್ನರು!” 

 “ಕೃತವಿದ್ಯರಾದ ಹಿಂದೂಗಳಲ್ಲಿ ಇಂದು ಕಾಣುವ ಜಾತಿಬಂಧನದಲ್ಲಿ ಸಿಕ್ಕಿ ನಿರ್ದಯರಾದ, ಕಪಟಿಗಳಾದ, ನಾಸ್ತಿಕರಾದ ಹೇಡಿಗಳಂತೆ ನಾನು ಹುಟ್ಟಿ ಸಾಯುವೆನೆಂದು ತಿಳಿದಿರುವಿರಾ? ಹೇಡಿತನವನ್ನು ನಾನು ದ್ವೇಷಿಸುವೆನು. ಹೇಡಿಗಳೊಂದಿಗೆ ಮತ್ತು ಕೆಲಸಕ್ಕೆ ಬಾರದ ರಾಜಕೀಯ ವ್ಯವಹಾರಗಳೊಂದಿಗೆ ನಮಗೆ ಯಾವ ಸಂಬಂಧವೂ ಇಲ್ಲ. ನನಗೆ ಯಾವ ರಾಜತಂತ್ರದಲ್ಲಿಯೂ ನಂಬಿಕೆ ಇಲ್ಲ. ಜಗತ್ತಿನಲ್ಲಿ ದೇವರು ಮತ್ತು ಸತ್ಯ ಇವೆರಡೇ ರಾಜತಂತ್ರಗಳು, ಉಳಿದವುಗಳೆಲ್ಲ ಕೆಲಸಕ್ಕೆ ಬಾರದವುಗಳು.” 

 ಸ್ವಾಮೀಜಿ ೧೮೯೫ನೇ ಸೆಪ್ಟೆಂಬರ್ ಮಧ್ಯಭಾಗದಲ್ಲಿ ಪ್ಯಾರಿಸ್ಸನ್ನು ತಲುಪಿದರು. ಅವರು ಅಲ್ಲಿ ಪ್ರಖ್ಯಾತವಾದ ಸ್ಥಳಗಳನ್ನೆಲ್ಲ ನೋಡಿದರು. ಜನರ ಕಲಾದೃಷ್ಟಿಯನ್ನು ನೋಡಿ ಮೆಚ್ಚಿದರು. ಅನಂತರ ಲಂಡನ್ನಿಗೆ ಬಂದು ಮಿಸ್ ಹೆನ್ರಿಯಟಾ ಮುಲ್ಲರ್ ಅವರ ಮನೆಯಲ್ಲಿ ಅತಿಥಿಗಳಾಗಿದ್ದರು. ಸ್ಟರ‍್ಡಿಯವರ ಪರಿಚಯವೂ ಆಯಿತು. ಮೊದಮೊದಲು ಮನೆಗಳಲ್ಲಿ ಪ್ರವಚನಗಳು ಆರಂಭವಾದವು. ಅನೇಕ ವಿದ್ವಾಂಸರು ಅವರ ಪ್ರವಚನಾದಿಗಳನ್ನು ಕೇಳಲು ಬರುತ್ತಿದ್ದರು. ಅನೇಕ ವೇಳೆ ಸರಿಯಾಗಿ ಕುಳಿತುಕೊಳ್ಳುವುದಕ್ಕೆ ಸ್ಥಳವಿಲ್ಲದೇ ಇದ್ದರೂ ನೆಲದ ಮೇಲೆ ಗಂಟೆಗಳ ಕಾಲ ಕುಳಿತುಕೊಂಡು ಕೇಳುತ್ತಿದ್ದರು. ಆ ಸಮಯದಲ್ಲಿಯೇ ಮುಂದೆ ಇವರ ಅಗ್ರಗಣ್ಯ ಶಿಷ್ಯರಲ್ಲೊಬ್ಬರಾದ ಮಾರ್ಗರೇಟ್ ಇ. ನೋಬಲ್ ಎಂಬಾಕೆಯ ಪರಿಚಯವಾಯಿತು. ಆಕೆ ಬಹಳ ಮೇಧಾವಿ ಹೆಂಗಸು; ತಾನೇ ಸ್ಥಾಪಿಸಿದ ಒಂದು ಶಾಲೆಯ ಪ್ರಿನ್ಸಿಪಾಲಳಾಗಿದ್ದಳು. ಆಕೆ ಸ್ವಾಮೀಜಿಯವರನ್ನು ನೋಡಲು ಬಂದು ಹಲವು ಪ್ರವಚನಗಳನ್ನು ಕೇಳಿದಳು. ಸ್ವಾಮೀಜಿ ಹೇಳುವುದನ್ನೆಲ್ಲ ಚೆನ್ನಾಗಿ ವಿಚಾರ ಮಾಡಿ ಅನಂತರ ಸ್ವೀಕರಿಸುತ್ತಿದ್ದಳು. ಆಕೆ ‘ನಾ ಕಂಡಂತೆ ನನ್ನ ಗುರುದೇವ’ ಎಂಬ ಪುಸ್ತಕದಲ್ಲಿ ಸ್ವಾಮೀಜಿಯವರನ್ನು ಪ್ರಥಮಬಾರಿ ಕಂಡ ಚಿತ್ರವನ್ನು ಹೀಗೆ ಬಣ್ಣಿಸುವಳು: “ಅವರನ್ನು ಪ್ರಥಮ ಬಾರಿ ಸಂದರ್ಶಿಸಿದ ಕಾಲ ನವೆಂಬರ್ ತಿಂಗಳ ಭಾನುವಾರ ಮಧ್ಯಾಹ್ನ. ಸ್ಥಳ ವೆಸ್ಟ್ ಎಂಡಿನ ಬೈಠಕ್ ಖಾನೆ. ಅವರ ಮುಂದೆ ಸುತ್ತಲೂ ಸ್ನೇಹಿತರು ಕುಳಿತುಕೊಂಡಿದ್ದರು. ಹಿಂದೆ ಒಲೆಯಲ್ಲಿ ಬಿಂಕಿ ಉರಿಯುತ್ತಿತ್ತು. ಅವರು ಪ್ರಶ್ನೆಗಳಾದ ಮೇಲೆ ಪ್ರಶ್ನೆಗಳಿಗೆ ಉತ್ತರವೀಯುತ್ತಿದ್ದರು. ಕೆಲವು ವೇಳೆ ತಮ್ಮ ಅಭಿಪ್ರಾಯವನ್ನು ಸಮರ್ಥಿಸುವುದಕ್ಕಾಗಿ, ಸಂಸ್ಕೃತ ಶ್ಲೋಕಗಳನ್ನು ಉದಾಹರಿಸುತ್ತಿದ್ದರು. ಇಂಗ್ಲೆಂಡಿನಲ್ಲಿ ಅನಂತರ ಸ್ವಾಮೀಜಿ ಅಷ್ಟು ಸರಳವಾಗಿ ಕುಳಿತಿದ್ದುದನ್ನು ಕಾಣಲಿಲ್ಲ. ಅವರು ನಮ್ಮ ಮಧ್ಯದಲ್ಲಿ ಕಾಷಾಯ ವಸ್ತ್ರಧಾರಿಗಳಾಗಿ ಯಾವುದೋ ಅತಿ ದೂರದ ಸಂದೇಶವನ್ನು ತಂದ ಋಷಿಯಂತೆ ಕುಳಿತಿದ್ದರು. ಮಾತಿನ ಮಧ್ಯೆ ‘ಶಿವ, ಶಿವ’ ಎನ್ನುವ ವಿಚಿತ್ರ ಸ್ವಭಾವ. ಅವರ ದೃಷ್ಟಿಯಲ್ಲಿ ಮಾಧುರ್ಯ ಮತ್ತು ಭವ್ಯತೆಗಳೆರಡೂ ಸಮ್ಮಿಳಿತವಾಗಿದ್ದವು. ಧ್ಯಾನಾವಸ್ಥೆಯಲ್ಲಿ ಬಹಳ ಕಾಲ ಕಳೆದವರ ವಿಶ್ರಾಂತಿ ಅವರ ಮುಖದ ಮೇಲಿತ್ತು. ರ‍್ಯಾಫಿಲ್ (\enginline{Raphael}) ಚಿತ್ರಿಸಿದ ಮೇರಿಯ ಮಗು ಕ್ರಿಸ್ತನ ಹುಬ್ಬಿನಮೇಲೆ ಪ್ರಶಾಂತಿಯಂತೆ ಇತ್ತು ಅದು.” 

 ಸ್ವಾಮೀಜಿ ಪಿಕಾಡಿಲಿ ಹಾಲಿನಲ್ಲಿ ಆತ್ಮಜ್ಞಾನ ಎನ್ನುವುದರ ಮೇಲೆ ಒಂದು ಬಹಿರಂಗ ಉಪನ್ಯಾಸವನ್ನು ಕೊಟ್ಟರು. ಲಂಡನ್ನಿನ ವಿದ್ವನ್ಮಣಿಗಳೆಲ್ಲ ಈ ಉಪನ್ಯಾಸಕ್ಕೆ ಬಂದಿದ್ದರು. ವೈಜ್ಞಾನಿಕವಾಗಿ, ವಿಚಾರಪೂರಿತವಾಗಿ ವೇದಾಂತ ಭಾವನೆಗಳನ್ನು ಸಮರ್ಥಿಸುವುದನ್ನು ಎಲ್ಲರೂ ಕೇಳಿ ಮೆಚ್ಚಿದರು. ಅನೇಕ ಪತ್ರಿಕಾ ಪ್ರತಿನಿಧಿಗಳು ಸ್ವಾಮೀಜಿಯವರನ್ನು ಭೇಟಿಮಾಡಿಕೊಂಡುಹೋಗಿ ತಮ್ಮ ವೃತ್ತಿಪತ್ರಿಕೆಗಳಲ್ಲಿ ಅವುಗಳನ್ನು ಪ್ರಕಟಿಸಿದರು. ಅದರಲ್ಲಿ ಒಂದನ್ನು ಕೆಳಗೆ ಕೊಡುವೆವು: 

 \enginline{West Minister Gazett, 21st October, 1895:} “ಇತ್ತೀಚೆಗೆ ಇಂಡಿಯಾ ದೇಶದ ತತ್ವಶಾಸ್ತ್ರ ಅನೇಕರ ಮೇಲೆ ಬಹಳ ಪರಿಣಾಮಕಾರಿಯಾದ ಪ್ರಭಾವವನ್ನು ಬೀರುತ್ತಿದೆ. ಇದುವರೆಗೆ ಭಾರತೀಯ ತತ್ತ್ವಶಾಸ್ತ್ರವನ್ನು ಇಲ್ಲಿ ಸಾರಿದವರೆಲ್ಲಾ ತಮ್ಮ ಭಾವನೆಯಲ್ಲಿ ಮತ್ತು ಅನುಷ್ಠಾನದಲ್ಲಿ ಪಾಶ್ಚಾತ್ಯರೇ ಆಗಿದ್ದರು. ಆದಕಾರಣವೇ ವೇದಾಂತ ತತ್ತ್ವದ ಗಹನ ಸತ್ಯಗಳಲ್ಲಿ ಎಲ್ಲೋ ಸ್ವಲ್ಪ ಭಾಗ ಮಾತ್ರ ಇಲ್ಲಿಯ ಜನಗಳಿಗೆ ಗೊತ್ತಿತ್ತು. ಆ ಸ್ವಲ್ಪವೂ ಎಲ್ಲೋ ಕೆಲವು ಮಂದಿಗಳಿಗೆ ಮಾತ್ರ. ಕೇವಲ ಭಾಷಾಶಾಸ್ತ್ರಗಳ ದೃಷ್ಟಿಯಿಂದ ಮಾಡಿರುವ ದೊಡ್ಡ ದೊಡ್ಡ ಭಾಷಾಂತರಗಳನ್ನು ತಿಳಿದುಕೊಳ್ಳಲು ಧೈರ‍್ಯವಾಗಲಿ ಉತ್ಸಾಹವಾಗಲಿ ಅನೇಕರಿಗೆ ಇಲ್ಲ. ಪ್ರಾಚ್ಯಸಂಸ್ಕೃತಿಯ ಪರಂಪರೆಯಲ್ಲಿ ಬೆಳೆದ ಯೋಗ್ಯವಾದ ಜ್ಞಾನಿಗೆ ಮಾತ್ರ ವೇದಾಂತದಲ್ಲಿರುವ ಪರಮ ಸತ್ಯಗಳು ಕಾಣಿಸುವುವು.” 

 “ಆಸಕ್ತಿಯಿಂದಲೂ ಮತ್ತು ಸ್ವಲ್ಪ ಕುತೂಹಲದಿಂದಲೂ ಪ್ರೇರಿತನಾಗಿ, ಪಾಶ್ಚಾತ್ಯರಿಗೆ ಅಪರಿಚಿತನಾದ ಒಬ್ಬ ಜ್ಞಾನಿಯನ್ನು ನೋಡಲು ಹೋದೆ” ಎಂದು ಒಬ್ಬ ಪತ್ರಿಕೆಯ ಬಾತ್ಮೀದಾರರು ಬರೆಯುತ್ತಾರೆ: “ಆ ಜ್ಞಾನಿಗಳೇ ಇಂಡಿಯಾ ದೇಶದ ನಿಜವಾದ ಯೋಗಿಗಳಾದ ಸ್ವಾಮಿ ವಿವೇಕಾನಂದರು. ಪುರಾತನ ಕಾಲದ ಮಹರ್ಷಿಗಳಿಂದ ವಂಶಾನುಗತವಾಗಿ ಬಂದ ಈ ಜ್ಞಾನವನ್ನು ಪಾಶ್ಚಾತ್ಯರಿಗೆ ಕೊಡುವುದಕ್ಕಾಗಿ ಧೈರ್ಯದಿಂದ ಪಾಶ್ಚಾತ್ಯ ದೇಶಗಳಿಗೆ ಅವರು ಬಂದಿರುವರು. ಈ ಉದ್ದೇಶದ ಸಾಧನೆಗಾಗಿಯೇ ಕಳೆದ ರಾತ್ರಿ ಪ್ರಿನ್‍ಸೆಸ್ ಹಾಲಿನಲ್ಲಿ ಅವರು ಒಂದು ಉಪನ್ಯಾಸವನ್ನು ಕೊಟ್ಟರು. 

 “ಸ್ವಾಮಿ ವಿವೇಕಾನಂದರು ಆಕರ್ಷಣೀಯ ಗಂಭೀರ ವ್ಯಕ್ತಿ. ತಲೆಗೆ ಒಂದು ರುಮಾಲನ್ನು ಸುತ್ತಿರುವರು, ಸುಪ್ರಸನ್ನರು, ದಯಾರ್ದ್ರಹೃದಯರು.” 

 “ನಿಮ್ಮ ಹೆಸರಿಗೆ ಏನಾದರೂ ಅರ್ಥವಿದೆಯೆ?” ಎಂದು ಕೇಳಿದ್ದಕ್ಕೆ ಸ್ವಾಮೀಜಿ ಹೇಳಿದರು: “ಈಗ ನನ್ನನ್ನು ಜನರು ಕರೆಯುವುದು ಸ್ವಾಮಿ ವಿವೇಕಾನಂದ ಎಂದು. ಇದರಲ್ಲಿ ಸ್ವಾಮಿ ಎಂದರೆ ಪ್ರಪಂಚವನ್ನು ತ್ಯಜಿಸಿದ ತ್ಯಾಗಿ ಎಂದು ಅರ್ಥ. ಎರಡನೆಯದೆ ನಾನಾಗೆ ಇಟ್ಟುಕೊಂಡ ಹೆಸರು. ಮನೆಯನ್ನು ಬಿಟ್ಟಮೇಲೆ ಸಂನ್ಯಾಸಿಗಳು ಬೇರೊಂದು ಹೆಸರನ್ನು ಇಟ್ಟುಕೊಳ್ಳುವರು. ಇದರ ಅರ್ಥ ವಿವೇಕದ ಆನಂದ.” 

 “ಸ್ವಾಮಿ, ನೀವು ಈ ದಾರಿಯನ್ನು ಹಿಡಿಯುವುದಕ್ಕೆ ಕಾರಣವೇನಿರಬಹುದು?” ಎಂದು ನಾನು ಕೇಳಿದೆ. ಸ್ವಾಮೀಜಿ: “ಬಾಲ್ಯದಿಂದಲೂ ನನಗೆ ಧರ್ಮ ಮತ್ತು ತತ್ವ ಎಂದರೆ ಬಹಳ ಆಸೆ. ಶಾಸ್ತ್ರಗಳು ಅನುಸರಿಸುವಂತೆ ನನ್ನನ್ನು ಪ್ರೇರೇಪಿಸಿವುದಕ್ಕೆ ಒಂದು ಕಿಡಿ ಮಾತ್ರ ಬೇಕಾಗಿತ್ತು. ಅದು ಶ‍್ರೀರಾಮಕೃಷ್ಣ ಪರಮಹಂಸರಿಂದ ಬಂದಿತು. ಅವರೂ ಕೂಡ ತ್ಯಾಗಿಗಳಾಗಿದ್ದರು. ಅವರಲ್ಲಿ ನನ್ನ ಜೀವನದ ಪರಮ ಧ್ಯೇಯ ರೂಪುಗೊಂಡಿರುವುದನ್ನು ನೋಡಿದೆ.” 

 ಬಾತ್ಮೀದಾರ: “ಹಾಗಾದರೆ ನೀವು ಈಗ ಯಾವುದರ ಪ್ರತಿನಿಧಿಯಾಗಿ ಬಂದಿರುವಿರೋ ಆ ಪಂಗಡವನ್ನು ಅವರು ಸ್ಥಾಪಿಸಿದರೆ?” 

 ಸ್ವಾಮೀಜಿ: “ಇಲ್ಲ. ಅವರ ಇಡಿಯ ಬಾಳು, ಮತಗಳಿಗೆ ಮತ್ತು ಕೋಮುಗಳಿಗೆ ಹಾಕಿರುವ ಬೇಲಿಯನ್ನು ತೆಗೆದುಹಾಕುವುದಕ್ಕೆ ಮೀಸಲಾಗಿತ್ತು. ಅವರು ಯಾವ ಪಂಗಡವನ್ನೂ ಸ್ಥಾಪನೆ ಮಾಡಲಿಲ್ಲ. ಅದಕ್ಕೆ ಬದಲಾಗಿ ಎಲ್ಲರಿಗೂ ಭಾವನಾ ಸ್ವಾತಂತ್ರ್ಯವನ್ನು ಕೊಡಲು ಹೋರಾಡಿದರು. ಅವರು ಮಹಾ ಯೋಗಿಗಳು.” 

 ಬಾತ್ಮೀದಾರ: “ಹಾಗಾದರೆ ನೀವು ಈ ದೇಶದ ಥಿಯಾಸಫಿಕಲ್ ಸೊಸೈಟಿ, ಕ್ರಿಶ್ಚಿಯನ್ ಸೈನ್ಸ್ ಮುಂತಾದ ಯಾವುದಕ್ಕೂ ಸೇರಿಲ್ಲವೆ?” 

 ಸ್ವಾಮೀಜಿ: “ಇಲ್ಲ (ಮಗುವಿನಂತೆ ಅವರ ಮೊಗ ಬೆಳಗಿತು. ಸರಳವಾಗಿ ನೇರವಾಗಿ ಸತ್ಯವಾಗಿತ್ತು ಅವರಿತ್ತ ಉತ್ತರ). ನಾನು ಹೇಳುವುದು ನಮ್ಮ ಶಾಸ್ತ್ರಗಳಿಗೆ ಕೊಡುವ ನನ್ನ ಸ್ವಂತ ವಿವರಣೆಯಷ್ಟೆ. ನನ್ನ ಗುರುದೇವರು ತೋರಿದ ಬೆಳಕಿನಲ್ಲಿ ಅದನ್ನೇ ವಿವರಿಸುತ್ತಿರುವೆನು. ನನಗೆ ಯಾವ ವಿಶೇಷ ಅಧಿಕಾರವೂ ಇದೆ ಎಂದು ಹೇಳಿಕೊಳ್ಳುವುದಿಲ್ಲ. ಮೇಧಾವಿಗಳು ಅದನ್ನು ಒಪ್ಪಿಕೊಂಡರೆ, ತಮ್ಮ ಜೀವನದಲ್ಲಿ ಅದನ್ನು ಬಳಸಿಕೊಂಡರೆ ಸಾಕು, ನಾನು ಪಟ್ಟ ಪ್ರಯತ್ನ ಸಾರ್ಥಕವಾಯಿತೆಂದು ಭಾವಿಸುವೆನು. ಎಲ್ಲಾ ಧರ್ಮಗಳಲ್ಲಿಯೂ ಜ್ಞಾನ ಭಕ್ತಿ ಯೋಗಗಳು ಇದ್ದೇ ಇವೆ. ವೇದಾಂತ ತತ್ತ್ವ ಗಹನವಾದ ವಿಜ್ಞಾನ ಶಾಸ್ತ್ರದಂತೆ. ಅದು ಮೇಲಿನ ಮಾರ್ಗಗಳನ್ನೆಲ್ಲಾ ಒಳಗೊಂಡಿರುವುದು. ನಾನು ಬೋಧಿಸುವುದೂ ಇದನ್ನೇ. ಪ್ರತಿಯೊಬ್ಬನೂ ತನ್ನ ಧರ್ಮಕ್ಕೆ ಇದನ್ನು ಉಪಯೋಗಿಸಿಕೊಳ್ಳಬಹುದು. ನಾನು ಪ್ರತಿಯೊಬ್ಬನ ಅನುಭವಕ್ಕೂ ನಿಲುಕುವಂತಹ ವಿಷಯಗಳನ್ನು ಹೇಳುತ್ತೇನೆ. ಆ ಪುಸ್ತಕಗಳಲ್ಲಿ ನೀವು ನೋಡಿದರೆ ಆ ವಿಷಯಗಳು ನಿಮಗೆ ದೊರಕುತ್ತವೆ. ಯಾರೋ ಕಣ್ಣಿಗೆ ಕಾಣಿಸದ ಮಹಾತ್ಮರು ನನ್ನ ಮೂಲಕ ಏನನ್ನೊ ಹೇಳುತ್ತಾರೆಂದಾಗಲಿ, ಯಾರ ಕಣ್ಣಿಗೂ ಬೀಳದ ಶಾಸ್ತ್ರಗಳಿಂದ ನಾನು ಹೇಳುತ್ತೇನೆ ಎಂದಾಗಲಿ ಭಾವಿಸಬಾರದು. ನಾನು ಯಾವ ಗುಪ್ತ ಸಿದ್ಧಾಂತವನ್ನೂ ಜನರಿಗೆ ಬೋಧಿಸುವುದಿಲ್ಲ. ಸತ್ಯ ತನ್ನ ಸ್ವಂತ ಆಧಾರದ ಮೇಲೆ ನಿಂತಿರುವುದು. ಸತ್ಯ ಯಾವ ಟೀಕೆಯನ್ನು ಬೇಕಾದರೂ ಇದಿರಿಸಬಲ್ಲದು.” 

 ಬಾತ್ಮೀದಾರ: “ನೀವು ಯಾವುದಾದರೊಂದು ಸಂಘವನ್ನು ಸ್ಥಾಪನೆ ಮಾಡಬೇಕೆಂದು ಇರುವಿರಾ?” 

 ಸ್ವಾಮೀಜಿ: “ಇಲ್ಲ, ಯಾವ ಸಮಾಜವನ್ನೂ ನಾನು ಸ್ಥಾಪಿಸಬೇಕೆಂದು ಇಲ್ಲ. ಎಲ್ಲದರಲ್ಲಿಯೂ ಇರುವ, ಎಲ್ಲರಿಗೂ ಅನ್ವಯಿಸುವ ಆತ್ಮವೊಂದನ್ನೇ ನಾನು ಬೋಧಿಸುವುದು. ಆತ್ಮವನ್ನು ಅರಿತ, ಅದರ ಬೆಳಕಿನಲ್ಲಿ ಜೀವಿಸುತ್ತಿರುವ ಕೆಲವು ಜನರು ಈಗಲೂ ಕೂಡ ಪ್ರಪಂಚವನ್ನೆ ಜಾಗೃತಗೊಳಿಸಬಹುದು. ಹಿಂದೆ ಹೇಗೆ ಇದು ಅಂತಹ ಮಹಾ ವ್ಯಕ್ತಿಗಳಿಗೆ ಸಾಧ್ಯವಾಗಿತ್ತೋ ಹಾಗೆಯೇ ಇಂದಿಗೂ ಸಾಧ್ಯ.” 

 ಬಾತ್ಮೀದಾರ: “ನೀವು ಈಗತಾನೆ ಭರತಖಂಡದಿಂದ ಬಂದಿರಾ?” ( ಸ್ವಾಮೀಜಿ ಅವರು ಭರತಖಂಡದ ಬಿಸಿಲಿನಲ್ಲಿ ಬಾಡಿದ್ದಂತೆ ಕಂಡರು.) 

 ಸ್ವಾಮೀಜಿ: “ಇಲ್ಲ. ನಾನು ಚಿಕಾಗೊ ನಗರದಲ್ಲಿ ೧೮೯೩ರಲ್ಲಿ ನಡೆದ ವಿಶ್ವಧರ್ಮ ಸಮ್ಮೇಳನದಲ್ಲಿ ಹಿಂದೂ ಧರ್ಮದ ಪ್ರತಿನಿಧಿಯಾಗಿ ಹೋಗಿದ್ದೆ. ಅಂದಿನಿಂದ ನಾನು ಅಮೇರಿಕಾ ದೇಶದಲ್ಲಿ ಸಂಚಾರ ಮಾಡುತ್ತಾ ಉಪನ್ಯಾಸ ಕೊಡುತ್ತಿರುವೆ. ಅಮೇರಿಕಾ ದೇಶದ ಜನರು ನನ್ನ ಉಪನ್ಯಾಸಗಳನ್ನು ಬಹಳ ಕುತೂಹಲದಿಂದ ಕೇಳುವರು. ನನ್ನ ಹಲವು ದಯಾವಂತರಾದ ಸ್ನೇಹಿತರು ಅವರಲ್ಲಿ ಇರುವರು. ನಾನು ಅಲ್ಲಿ ಮಾಡಿರುವ ಕೆಲಸ ಚೆನ್ನಾಗಿ ಬೇರುಬಿಟ್ಟಿದೆ. ಬೇಗ ಅಲ್ಲಿಗೆ ಹೋಗಬೇಕಾಗಿದೆ.” 

 ಬಾತ್ಮೀದಾರ: “ಪಾಶ್ಚಾತ್ಯ ಧರ‍್ಮವನ್ನು ನೀವು ಯಾವ ದೃಷ್ಟಿಯಿಂದ ನೋಡುತ್ತೀರಿ?” 

 ಸ್ವಾಮೀಜಿ: “ನಾನು ಬೋಧಿಸುವ ತತ್ತ್ವ ಪ್ರಪಂಚದ ಧರ್ಮಗಳಿಗೆಲ್ಲಾ ತಳಹದಿಯಾಗಬಲ್ಲದು. ನಾನು ಎಲ್ಲಾ ಧರ್ಮಗಳನ್ನೂ ಪೂಜ್ಯಭಾವದಿಂದ ನೋಡುತ್ತೇನೆ. ನಾನು ಯಾವ ಧರ್ಮವನ್ನೂ ವಿರೋಧಿಸುವುದಿಲ್ಲ. ನಾನು ವ್ಯಕ್ತಿಗೆ ಪ್ರತ್ಯೇಕವಾಗಿ ಗಮನ ಕೊಡುತ್ತೇನೆ. ಅವನನ್ನು ಬಲಾಢ್ಯನನ್ನಾಗಿ ಮಾಡಲು ಯತ್ನಿಸುತ್ತೇನೆ. ಅವನು ಪವಿತ್ರಾತ್ಮ ಎಂದು ಬೋಧಿಸುತ್ತೇನೆ. ಎಲ್ಲರೂ ತಮ್ಮಲ್ಲಿರುವ ಪವಿತ್ರತೆಯನ್ನು ವ್ಯಕ್ತಗೊಳಿಸಬೇಕೆಂಬುದೇ ನನ್ನ ಆಶಯ. ಇದೇ ಪ್ರತಿಯೊಂದು ಧರ್ಮದ (ಪ್ರತ್ಯಕ್ಷ ಅಥವಾ ಪರೋಕ್ಷವಾದ) ಆದರ್ಶವಾಗಿದೆ.” 

 ಬಾತ್ಮೀದಾರ: “ಈ ದೇಶದಲ್ಲಿ ನಿಮ್ಮ ಕಾರ‍್ಯ ಯಾವ ರೀತಿ ಆಗುವುದು?” 

 ಸ್ವಾಮೀಜಿ: “ನಾನು ಮೇಲೆ ಹೇಳಿದ ಭಾವನೆಯ ಮೇಲೆ ಕೆಲವು ವ್ಯಕ್ತಿಗಳಲ್ಲಿ ಕುತೂಹಲ ಹುಟ್ಟುವಂತೆ ಮಾಡಬೇಕೆಂದಿರುವೆನು. ಅನಂತರ ಅವರು ಈ ಭಾವನೆಯನ್ನು ತಮ್ಮ ರೀತಿಯಲ್ಲೆ ಇತರರಿಗೂ ಹೇಳಬಹುದು. ನಾನು ಅದನ್ನು ಒಂದು ಮತದಂತೆ ಹೇಳುವುದಿಲ್ಲ. ಕೊನೆಗೆ ಸತ್ಯವೇ ಜಯಿಸಬೇಕು.” 

 “ನಾನು ಹೇಗೆ ಕೆಲಸ ಮಾಡಬೇಕೋ ಅದು ನನ್ನ ಕೆಲವು ಸ್ನೇಹಿತರ ಕೈಯಲ್ಲಿದೆ. ಅಕ್ಟೋಬರ್ ೨೨ನೆಯ ತಾರೀಖು ಇಂಗ್ಲಿಷ್ ಸಭಿಕರಿಗೆ ಪಿಕಾಡಿಲ್ಲಿಯಲ್ಲಿರುವ ಪ್ರಿನ್ಸೆ ಸ್ ಹಾಲಿನಲ್ಲಿ ೫–೩೦ ಗಂಟೆಗೆ ಉಪನ್ಯಾಸ ಮಾಡುವೆನು. ಈ ಕಾರ್ಯಕ್ರಮವನ್ನು ಪೇಪರಿನಲ್ಲಿ ಜಾಹೀರಾತು ಮಾಡಿರುವರು. ವಿಷಯ ನನ್ನ ತತ್ತ್ವದ ತಿರುಳಾದ ಆತ್ಮಜ್ಞಾನ ಎನ್ನುವುದು. ಅನಂತರ ಕೆಲಸ ಯಾವ ರೀತಿ ತೋರುವುದೋ ಅದನ್ನೇ ಅನುಸರಿಸುವೆನು. ಜನರನ್ನು ಭೇಟಿಮಾಡುವುದು, ಪತ್ರಗಳಿಗೆ ಉತ್ತರ ಕೊಡುವುದು ಅಥವಾ ಪ್ರತ್ಯಕ್ಷವಾಗಿ ಅವರೊಡನೆ ಚರ್ಚಿಸುವುದು ಇತ್ಯಾದಿ. ಈ ಹಣದ ಯುಗದಲ್ಲಿ ನಾನು ಮಾಡುವುದು ಯಾವುದೂ ಹಣವನ್ನು ಗಳಿಸುವುದಕ್ಕಾಗಿ ಅಲ್ಲ.” 

 “ಜೀವನದಲ್ಲೇ ಅಪೂರ್ವ ವ್ಯಕ್ತಿಯೊಂದನ್ನು ನೋಡುವ ಸುಯೋಗ ನನಗೆ ಒದಗಿತ್ತು. ಅನಂತರ ಅವರಿಂದ ನಾನು ಬೀಳ್ಕೊಂಡೆ” ಎನ್ನುವನು ಬಾತ್ಮಿದಾರ. 

 ಸ್ವಾಮೀಜಿ ಅವರ ಪ್ರವಚನಾದಿಗಳು ಮತ್ತು ಉಪನ್ಯಾಸಗಳು ಲಂಡನ್ನಿನಲ್ಲಿ ಬಹಳ ಫಲಕಾರಿಯಾದುವು. ಅವರ ಭಾವನೆ ಬೇರೂರುವುದಕ್ಕೆ ಯೋಗ್ಯ ವಾತಾವರಣ ಲಂಡನ್ನಿನಲ್ಲಿ ಇತ್ತು. ಭಾರತೀಯರ ಭಾವನೆ ಅವರಿಗೇನೂ ಹೊಸದಾಗಿರಲಿಲ್ಲ. ಆದರೆ ಕೆಲವರಿಗೆ ಇದು ಪರಿಚಯವಾಗಿತ್ತು. ಈಗ ಸ್ವಾಮೀಜಿಯವರ ಮುಖೇನ ಅದನ್ನು ಮತ್ತಷ್ಟು ವಿಶದವಾಗಿ ಆಳವಾಗಿ ತಿಳಿದುಕೊಳ್ಳಲು ಸಾಧ್ಯವಾಯಿತು. ಜನರು ಇನ್ನೂ ಅವರ ಮಾತುಕತೆಗಳನ್ನು ಕೇಳಬೇಕು ಎಂದು ಕುತೂಹಲಿಗಳಾಗಿರುವಾಗಲೇ ಅಮೇರಿಕಾ ದೇಶದಿಂದ ಹಲವು ಶಿಷ್ಯರು ಸ್ವಾಮೀಜಿಯವರಿಗೆ ಕಾಗದ ಬರೆಯತೊಡಗಿದರು. ಅಮೇರಿಕಾ ದೇಶದಲ್ಲಿ ವೇದಾಂತ ಕೇಂದ್ರಗಳನ್ನು ಸ್ಥಾಪಿಸುವುದಕ್ಕೆ ಹಲವು ಜನ ಕುತೂಹಲಿಗಳಾಗಿರುವರೆಂದು ಕಾಗದ ಬರೆದುದರಿಂದ ಅದನ್ನು ಒಂದು ವ್ಯವಸ್ಥಿತ ಸ್ಥಿತಿಗೆ ತರುವುದಕ್ಕೆ ಅವರು ಪುನಃ ಅಮೇರಿಕಾ ದೇಶಕ್ಕೆ ಡಿಸೆಂಬರ್ ತಿಂಗಳಲ್ಲಿ ಹಿಂತಿರುಗಿ ಹೋದರು. 

