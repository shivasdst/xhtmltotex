
\chapter{ಆಲ್ಮೋರದಲ್ಲಿ}

 ಚಿಕಾಗೋ ವಿಶ್ವಧರ್ಮ ಸಮ್ಮೇಳನದಲ್ಲಿ ಸ್ವಾಮೀಜಿ ಪರಿಚಯ ಮಾಡಿಕೊಂಡಿದ್ದ\break ಸಿಲೋನಿನ ಬೌದ್ಧ ಭಿಕ್ಷುಗಳಾದ ಅನಾಗರಿಕ ಧರ್ಮಪಾಲರು ಒಂದು ದಿನ ಸ್ವಾಮೀಜಿಯವರನ್ನು ನೋಡಲು ಬಂದರು. ಅವರು ಶ‍್ರೀಮತಿ ಓಲ್‍ಬುಲ್ ಅವರನ್ನು ನೋಡಬೇಕೆಂದು ಬಯಸಿದರು. ಅಂದು ಬಹಳ ಮಳೆ ಸುರಿಯುತ್ತಿತ್ತು. ಎಷ್ಟು ಹೊತ್ತು ಕಾದರೂ ಮಳೆ ನಿಲ್ಲಲಿಲ್ಲ. ದಾರಿಯೆಲ್ಲ ಕೆಸರಿನ ಮಯವಾಗಿತ್ತು. ಅಂತೂ ಸ್ವಾಮೀಜಿಯವರು ಧರ್ಮಪಾಲರನ್ನು ಕರೆದುಕೊಂಡು ಶ‍್ರೀಮತಿ ಓಲ್‍ಬುಲ್ ಅವರನ್ನು ನೋಡುವುದಕ್ಕೆ ಹೋದರು. ಒಂದು ಕಡೆ ಕೆಸರಿನಲ್ಲಿ ಧರ್ಮಪಾಲರ ಕಾಲು ಹೂತುಕೊಂಡಿತು. ಸ್ವಾಮೀಜಿಯವರು ಅವರನ್ನು ಮೇಲಕ್ಕೆ ಎತ್ತಿ ಓಲ್‍ಬುಲ್ ಮನೆಗೆ ಕರೆದುಕೊಂಡು ಹೋದರು. ಅನಂತರ ಎಲ್ಲರೂ ತಮ್ಮ ಕೆಸರಿನ ಕಾಲನ್ನು ತೊಳೆದುಕೊಳ್ಳುವುದಕ್ಕೆ ಹೋದರು. ಧರ್ಮಪಾಲರು ನೀರಿನ ಪಾತ್ರೆಯನ್ನು ಕಾಲು ತೊಳೆಯುವುದಕ್ಕೆ ತೆಗೆದುಕೊಂಡಾಗ ಸ್ವಾಮೀಜಿಯವರು ಅದನ್ನು ತಾವು ತೆಗೆದುಕೊಂಡು “ನೀವು ನಮ್ಮ ಮನೆಗೆ ಅತಿಥಿಗಳಾಗಿ ಬಂದಿರುವಿರಿ. ನಿಮ್ಮ ಕಾಲನ್ನು ತೊಳೆಯುವುದು ನನ್ನ ಕರ್ತವ್ಯ” ಎಂದು ಹೇಳಿದರು. 

 ಸ್ವಾಮೀಜಿಯವರು ಈ ಕಾಲದಲ್ಲೆ ಮಿಸ್ ಮಾರ್ಗರೆಟ್ ನೋಬಲ್ ಎಂಬ ಇಂಗ್ಲಿಷ್ ಮಹಿಳೆಗೆ ಬ್ರಹ್ಮಚರ್ಯ ದೀಕ್ಷೆಯನ್ನು ಇತ್ತು ಅವಳಿಗೆ ‘ಸೋದರಿ ನಿವೇದಿತಾ’ ಎಂಬ ಹೆಸರನ್ನು ಕೊಟ್ಟರು. ಇವಳು ಸ್ವಾಮೀಜಿಯವರ ಮಾನಸಿಕ ಪುತ್ರಿಯಂತೆ ಇದ್ದಳು. ಹಿಂದೂಧರ್ಮ ಮತ್ತು ಸಂಸ್ಕೃತಿಯನ್ನು ಅಷ್ಟು ಪ್ರೀತಿಸುತ್ತಿದ್ದಳು. ಕಾಲಾನಂತರ ಅನೇಕ ಭಾರತೀಯರಿಗೆ ಸ್ಫೂರ್ತಿಯನ್ನು ಕೊಟ್ಟಳು. ಅನಂತರ ಮಾರ್ಚ್ ೨೯ನೇ ತಾರೀಖು ಸ್ವರೂಪಾನಂದರು ಮತ್ತು ಸುರೇಶ್ವರಾನಂದರಿಗೆ ಸ್ವಾಮೀಜಿ ಸಂನ್ಯಾಸವನ್ನು ಕೊಟ್ಟರು. ಸ್ವರೂಪಾನಂದರನ್ನು ಸ್ವಲ್ಪಕಾಲದಲ್ಲಿಯೇ ‘ಪ್ರಬುದ್ಧ ಭಾರತ’ದ ಸಂಪಾದಕರನ್ನಾಗಿ ಮಾಡಿದರು. ಮಾಯಾವತಿ ಆಶ್ರಮ ಸ್ಥಾಪನೆ ಮಾಡಿದ ಮೇಲೆ ಇವರನ್ನೇ ಅಧ್ಯಕ್ಷರನ್ನಾಗಿಯೂ ಮಾಡಿದರು. 

 ಈ ಸಮಯದಲ್ಲಿ ಸ್ವಾಮೀಜಿಯವರು ಬಹಿರಂಗದಲ್ಲಿ ಅಷ್ಟು ಉಪನ್ಯಾಸ ಮಾಡುತ್ತಿರಲಿಲ್ಲ. ಮಠದಲ್ಲಿ ವ್ಯಕ್ತಿಗಳನ್ನು ರೂಪಿಸುವುದರ ಕಡೆಗೇ ಗಮನವನ್ನು ಕೊಡುತ್ತಿದ್ದರು. ಒಮ್ಮೆ ಮಾತ್ರ ಸೋದರಿ ನಿವೇದಿತೆಯನ್ನು ಕಲ್ಕತ್ತದ ಪುರಜನರಿಗೆ ಪರಿಚಯ ಮಾಡಿಕೊಡುವ ಸಮಯದಲ್ಲಿ ಸ್ಟಾರ್ ಥಿಯೇಟರಿನಲ್ಲಿ ಅಧ್ಯಕ್ಷರಾಗಿ ಸೋದರಿ ನಿವೇದಿತಾಳನ್ನು ಇಂಡಿಯಾದೇಶಕ್ಕೆ ಇಂಗ್ಲೆಂಡಿನ ಕಾಣಿಕೆ ಎಂದು ಕರೆದರು. ಆ ಸಮಯದಲ್ಲಿ ಸೋದರಿ ನಿವೇದಿತಾ ಇಂಗ್ಲೆಂಡಿನಲ್ಲಿ ಭಾರತೀಯ ಆಧ್ಯಾತ್ಮಿಕ ಪ್ರಭಾವ ಎಂಬ ವಿಷಯದ ಮೇಲೆ ಮಾತನಾಡಿದಳು. ಅದೇ ಸಮಯದಲ್ಲಿ ಮಿಸ್ ಮುಲ್ಲರ್ ಮತ್ತು ಓಲ್‍ಬುಲ್ ಅವರಿಗೂ ಮಾತನಾಡುವಂತೆ ಕೋರಿಕೊಂಡರು. 

 ಬೇಲೂರು ಮಠದ ಆವರಣದಲ್ಲಿ ಪ್ರತ್ಯೇಕವಾಗಿರುವ ಮನೆಯಲ್ಲಿ ಅಮೇರಿಕಾ ದೇಶದಿಂದ ಬಂದ ಮಿಸ್ ಮೇಕ್ಲಿಯಾಡ್, ಶ‍್ರೀಮತಿ ಓಲ್‍ಬುಲ್ ಇಂಗ್ಲೆಂಡಿನಿಂದ ಬಂದಿದ್ದ ಮಿಸ್ ಮುಲ್ಲರ್ ಮತ್ತು ಸೋದರಿ ನಿವೇದಿತಾ ಇವರೆಲ್ಲರೂ ವಾಸಮಾಡು\-ತ್ತಿದ್ದರು. ಸ್ವಾಮೀಜಿ ಪ್ರತಿದಿನವೂ ಬೆಳಿಗ್ಗೆ ಸಮಯದಲ್ಲಿ ಬಂದು ಹಿಂದೂ ಸಂಸ್ಕೃತಿ ಆಚಾರ ವ್ಯವಹಾರ ಇವುಗಳನ್ನು ತಿಳಿದುಕೊಳ್ಳುವುದು ಹೇಗೆ ಎಂಬುದನ್ನು ಪಾಶ್ಚಾತ್ಯ ಶಿಷ್ಯರಿಗೆ ವಿವರಿಸುತ್ತಿದ್ದರು. ಪಾಶ್ಚಾತ್ಯ ಶಿಷ್ಯರ ಹಿನ್ನೆಲೆಯೇ ಬೇರೆ. ಅವರ ಧರ್ಮ ಆಚಾರ ವ್ಯವಹಾರ ಎಲ್ಲವೂ ಭಾರತೀಯರಿಗಿಂತ ಬೇರೆ ಆದುದು. ಅವರು ಭಾರತೀಯರ ದೃಷ್ಟಿಯನ್ನು ಬೆಳೆಸಿಕೊಳ್ಳಬೇಕಾದರೆ ಅದಕ್ಕೆ ಬಹಳಕಾಲ ಹಿಡಿದು ಮನಸ್ಸಿನಲ್ಲಿ ಹಲವು ಘರ್ಷಣೆಗಳು ಆಗುವುವು. ಸ್ವಾಮೀಜಿ ಅದನ್ನೆಲ್ಲ ಸಹನೆಯಿಂದ ನೋಡುತ್ತ ಸಹಾನುಭೂತಿಯನ್ನು ಬೀರುತ್ತಿದ್ದರು. ಸೋದರಿ ನಿವೇದಿತಾ ಭರತಖಂಡವನ್ನು ಸೇವಿಸುವುದಕ್ಕೆ ಬಂದಿದ್ದಳು. ಸ್ವಾಮೀಜಿ ಆಕೆಗೆ ಭರತಖಂಡವನ್ನು ಸೇವಿಸಬೇಕಾದರೆ ಒಬ್ಬ ಮನೋವಾಕ್ಕಾಯವಾಗಿ ಭಾರತೀಯನಾಗಬೇಕು. ಆಹಾರ ಮತ್ತು ಭೂಷಣಗಳಲ್ಲಿಯೂ ಹಾಗೆ ಆದಲ್ಲದೆ ಅವರ ಹೃದಯವನ್ನು ಪ್ರವೇಶಿಸಲಾಗುವುದಿಲ್ಲ ಎಂಬುದನ್ನು ಅವಳಿಗೆ ವ್ಯಕ್ತಪಡಿಸಿದರು. ನಿವೇದಿತಾ ನಿಜವಾಗಿಯೂ ಭಾರತೀಯರಲ್ಲಿ ಭಾರತೀಯಳಾದಳು. ಭರತಖಂಡದಲ್ಲಿ ಹುಟ್ಟಿದವನು ಕೂಡ ಆಕೆ ಇಂಡಿಯಾ ದೇಶವನ್ನು ಪ್ರೀತಿಸುತ್ತಿದ್ದಂತೆ ಪ್ರೀತಿಸಲು ಸಾಧ್ಯವಿಲ್ಲ. ಅಷ್ಟು ಆಳವಾಗಿತ್ತು, ಅಷ್ಟು ನಿಕಟವಾಗಿತ್ತು ಆಕೆ ಸ್ವೀಕರಿಸಿದ ದೇಶದ ಮೇಲಿನ ಭಕ್ತಿ. 

 ಮಾರ್ಚ್ ೩೦ನೇ ತಾರೀಖು ಸ್ವಾಮೀಜಿ ಡಾರ್ಜಿಲಿಂಗಿಗೆ ಹೋದರು. ಅಲ್ಲಿ ಸಂಪೂರ್ಣವಾಗಿ ವಿಶ್ರಾಂತಿಯನ್ನು ಅನುಭವಿಸಿದರು. ಯಾವ ಕೆಲಸವನ್ನೂ ಹಚ್ಚಿಕೊಳ್ಳಲಿಲ್ಲ. ಕೆಲವು ಕಾಲವಾಗುವುದರೊಳಗೆ ಕಲ್ಕತ್ತೆಯಲ್ಲಿ ಪ್ಲೇಗಿನ ಉಪದ್ರವ ಪ್ರಾರಂಭವಾಗಿದೆ, ಎಂಬುದನ್ನು ಕೇಳಿದರು. ಮೇ ತಿಂಗಳ ಮೊದಲನೇ ವಾರದಲ್ಲಿ ಕಲ್ಕತ್ತೆಗೆ ಬಂದು ಪ್ಲೇಗಿನ ನಿವಾರಣೆ ಕೆಲಸಕ್ಕೆ ತಮ್ಮ ಶಿಷ್ಯರನ್ನು ಬಿಟ್ಟರು. ಅದಕ್ಕೆ ಅಷ್ಟೊಂದು ಹಣ ಎಲ್ಲಿ ಬರುತ್ತದೆ ಎಂದು ಕೇಳಿದಾಗ “ಈಗತಾನೇ ತೆಗೆದುಕೊಂಡಿರುವ ಬೇಲೂರು ಮಠದ ನಿವೇಶನವನ್ನು ಬೇಕಾದರೆ ಮಾರಿ ಆ ಕೆಲಸವನ್ನು ಮಾಡೋಣ. ಜನರೆಲ್ಲಾ ಇಷ್ಟೊಂದು ಸಂಕಟಕ್ಕೆ ತುತ್ತಾಗಿರುವಾಗ ಇದು ಏತಕ್ಕೆ ನಮಗೆ ಬೇಕು” ಎಂದರು. ಆದರೆ ಮಠವನ್ನು ಮಾರುವ ತನಕ ಹೋಗಬೇಕಾಗಲಿಲ್ಲ. ಹತ್ತಾರು ಕಡೆಗಳಿಂದ ದುಡ್ಡು ಬಂತು. ಹಲವು ಜನರೂ ಇವರ ಸಹಾಯಕ್ಕೆ ಬಂದರು. ಸ್ವಾಮೀಜಿ ಜಯಪ್ರದವಾಗಿ ಪ್ಲೇಗಿನ ನಿವಾರಣೆಯನ್ನು ಮಾಡಿದರು. 

 ಸೇವಿಯರ್ಸ್‍‍ ದಂಪತಿಗಳು ಹಿಮಾಲಯದಲ್ಲಿ ಆಲ್ಮೋರದಲ್ಲಿದ್ದರು. ಅವರು ಪದೇ ಪದೇ ಸ್ವಾಮೀಜಿಯವರಿಗೆ ಅಲ್ಲಿಗೆ ಬರಬೇಕೆಂದು ಕಾಗದ ಬರೆಯುತ್ತಿದ್ದರು. ಅದರಂತೆಯೇ ಸ್ವಾಮೀಜಿ ಮೇ ೧೧ನೇ ತಾರೀಖು ಅಲ್ಲಿಗೆ ಹೋಗುವುದಕ್ಕೆ ಕಲ್ಕತ್ತೆಯನ್ನು ಬಿಟ್ಟರು. ಅವರ ಜೊತೆಯಲ್ಲಿ ಸಂನ್ಯಾಸಿ ಭ್ರಾತೃವರ್ಗದವರಾದ ತುರೀಯಾನಂದ, ನಿರಂಜನಾನಂದ, ಸದಾನಂದ, ಸ್ವರೂಪಾನಂದರು, ಪಾಶ್ಚಾತ್ಯ ಶಿಷ್ಯರಾದ ಶ‍್ರೀಮತಿ ಓಲ್‍ಬುಲ್, ಸೋದರಿ ನಿವೇದಿತಾ, ಜೋಸಿಪೈನ್ ಮೇಕ್ಲಿಯಾಡ್ ಮತ್ತು ಶ‍್ರೀಮತಿ ಪ್ಯಾಟರ್‍ಸನ್ ಅನ್ನುವರು ಕೂಡ ಇದ್ದರು. ಶ‍್ರೀಮತಿ ಪ್ಯಾಟರ್‍ಸನ್ ಎಂಬಾಕೆ ಕಲ್ಕತ್ತೆಯಲ್ಲಿರುವ ಅಮೆರಿಕನ್ ಕಂಸೂಲ್ ಜನರಲ್ ಅವರ ಪತ್ನಿ. ಆಕೆ ಅಮೇರಿಕಾ ದೇಶದಲ್ಲಿದ್ದಾಗ ಒಮ್ಮೆ ಸ್ವಾಮೀಜಿಯವರಿಗೆ ತಂಗಲು ಹೋಟೆಲಿನಲ್ಲಿ ಇವರ ಬಣ್ಣದ ದೆಸೆಯಿಂದ ರೂಮು ಸಿಕ್ಕದಿದ್ದಾಗ ತನ್ನ ಮನೆಯಲ್ಲಿ ಅವರನ್ನು ಇಟ್ಟುಕೊಂಡಿದ್ದಳು. ಅನಂತರ ಸ್ವಾಮೀಜಿಯವರನ್ನು ಬಹಳ ಗೌರವಿಸುತ್ತಿದ್ದಳು. ಸ್ವಾಮೀಜಿಯವರ ಈಗಿನ ಪರ‍್ಯಟನದಲ್ಲಿ ಕೆಲವು ಪಾಶ್ಚಾತ್ಯ ಮಹಿಳೆಯರು ಕೂಡ ಇರುವರೆಂಬುದನ್ನು ಕೇಳಿ ಆಕೆಯೂ ಇದರಲ್ಲಿ ಸೇರಿಕೊಂಡಳು. 

 ಸ್ವಾಮೀಜಿ ಕಲ್ಕತ್ತೆಯಿಂದ ನೈನಿತಾಲಿಗೆ ಹೋಗುವಾಗ ದಾರಿಯಲ್ಲಿ ಕಂಡ\break ಆಕರ್ಷಣೀಯ ವಿಷಯಗಳನ್ನೆಲ್ಲ ಶಿಷ್ಯರಿಗೆ ವಿವರಿಸತೊಡಗಿದರು. ಇವರ ಜೊತೆಯಲ್ಲಿ ಪ್ರಯಾಣ ಮಾಡಿದ ಸೋದರಿ ನಿವೇದಿತಾ ಹೀಗೆ ಹೇಳುವಳು: “ಹಿಂದಿನ ಕಾಲದ ಪಾಟಲೀಪುತ್ರವೆಂಬ ಪ್ರಖ್ಯಾತವಾದ ಪಾಟ್ನಾ ನಗರದಿಂದಲೇ ನಮ್ಮ ಪಾಠ ಪ್ರಾರಂಭವಾಯಿತೆಂದು ಹೇಳಬಹುದು. ಪೂರ್ವ ದಿಕ್ಕಿನಿಂದ ರೈಲಿನಲ್ಲಿ ನದೀ ಮೂಲಕ ಕಾಶಿಯನ್ನು ಪ್ರವೇಶಿಸಿದಾಗ ಪ್ರಪಂಚದಲ್ಲೆಲ್ಲ ಒಂದು ಅದ್ಭುತ ದೃಶ್ಯ ನಮಗೆ ಕಾಣುವುದು. ನಮ್ಮ ನಾಯಕರು ಉತ್ಸಾಹದಿಂದ ಅದನ್ನು ಹೊಗಳತೊಡಗಿದರು. ಲಕ್ನೋ ನಗರದ ಕೈಗಾರಿಕೆ ಮತ್ತು ಭೋಗ್ಯ ವಸ್ತುಗಳನ್ನು ವಿವರಿಸತೊಡಗಿದರು. ಸ್ವಾಮೀಜಿ ಚರಿತ್ರಾರ್ಹವಾದ ಸುಂದರವಾದ ದೊಡ್ಡ ದೊಡ್ಡ ನಗರಗಳನ್ನು ಮಾತ್ರ ಕುತೂಹಲದಿಂದ ಕೊಂಡಾಡಲಿಲ್ಲ. ದೂರದ ಬೈಲು ಪ್ರದೇಶಗಳಲ್ಲಿ ಹೊಲ ಮನೆ ಹಳ್ಳಿಗಳ ಮೂಲಕ ಹಾದುಹೋಗುವಾಗ, ಅದನ್ನು ನೋಡಿದಾಗ ಸ್ವಾಮೀಜಿಯವರ ಹೃದಯದಲ್ಲಿ ಚಿಮ್ಮಿದಷ್ಟು ಪ್ರೇಮ ಮತ್ತು ಅದರಲ್ಲೆ ತನ್ಮಯರಾಗುವುದನ್ನು ಇನ್ನೆಲ್ಲಿಯೂ ಕಾಣಲಿಲ್ಲ ಎನ್ನಬಹುದು. ಇಂತಹ ಸಮಯಗಳಲ್ಲಿ ಸ್ವಾಮೀಜಿ ಇಡೀ ಭರತಖಂಡವನ್ನೇ ಮನನ ಮಾಡತೊಡಗಿದರು. ಸಮುದಾಯದ ಒಕ್ಕಲುತನ, ರೈತನ ಮನೆಯಲ್ಲಿರುವ ಗೃಹಿಣಿಯ ನಿತ್ಯಜೀವನ ಇವುಗಳನ್ನು ಕುರಿತು ವಿವರಿಸುವುದರಲ್ಲಿ ಗಂಟೆಗಟ್ಟಲೆ ಕಳೆಯುತ್ತಿದ್ದರು. ಹೇಳುವಾಗ ರಾತ್ರಿ ಒಲೆಯಮೇಲೆ ಬೆಳಿಗ್ಗೆ ತಿಂಡಿಗಾಗಿ ಬೇಯಿಸುವುದಕ್ಕೆ ಇಟ್ಟಿದ್ದ ಕಾಳನ್ನೂ ಮರೆಯುತ್ತಿರಲಿಲ್ಲ. ಪರಿವ್ರಾಜಕರಾಗಿ ಅಲೆಯುತ್ತಿದ್ದಾಗ ತಮಗೆ ಆದ ಅನುಭವಗಳೇ ಇದಕ್ಕೆ ಕಾರಣ ಎಂದು ಗೊತ್ತಾಗುವುದು. ಅವರು ನಮ್ಮೊಡನೆ ಮಾತನಾಡುವಾಗ ಉತ್ಸಾಹ ಕಣ್ಣಿನಲ್ಲಿ ಹೊಳೆಯುತ್ತಿತ್ತು, ಮಾತಿನಲ್ಲಿ ಅನುರಣಿತವಾಗುತ್ತಿತ್ತು. ದೀನ ರೈತನ ಮನೆಯವರಷ್ಟು ಅತಿಥಿ ಸತ್ಕಾರಪರರು ಭರತಖಂಡದಲ್ಲಿ ಬೇರೆ ಇಲ್ಲ ಎಂದು ಸಾಧುಗಳು ಹೇಳಿರುವುದನ್ನು ನಾನು ಕೇಳಿರುವೆನು. ನಿಜ, ಗೃಹಿಣಿಗೆ ಅತಿಥಿಗೆ ಮಲಗಲು ಕೊಡುವುದಕ್ಕೆ ಚಾಪೆಯಲ್ಲದೆ ಬೇರಿಲ್ಲ, ಆದರೆ ರಾತ್ರಿ ಮಲಗಲು ಹೋಗುವುದಕ್ಕೆ ಮುಂಚೆ ಅತಿಥಿಯನ್ನು ನೋಡಲು ಅವನಿಗೆ ಕಾಣದಂತೆ ಬರುವಳು. ಬೆಳಗ್ಗೆ ಎದ್ದ ಮೇಲೆ ಹಲ್ಲನ್ನು ಉಜ್ಜಿಕೊಳ್ಳಲು ಒಂದು ಕಡ್ಡಿಯನ್ನು ಇಟ್ಟು ಕುಡಿಯಲು ಒಂದು ಬಟ್ಟಲು ಹಾಲನ್ನು ಅವನ ಸಮೀಪದಲ್ಲಿಟ್ಟು ಹೋಗುವಳು. ತನ್ನ ಮನೆಯಲ್ಲಿ ಅತಿಥಿ ಸುಖವಾಗಿ ಸುಧಾರಿಸಿಕೊಂಡು ಹೋಗಲಿ ಎಂದು. 

 “ಸ್ವಾಮೀಜಿ ಪ್ರವಾಸದಲ್ಲಿದ್ದಾಗ ಭರತಖಂಡದ ಭೂತಕಾಲದಲ್ಲೆ ಬಾಳಿ ಸಂಚರಿಸುತ್ತ ಇದ್ದಂತೆ ಇತ್ತು. ಕೆಲವು ವೇಳೆ ಅದೇ ಅವರ ಜೀವನದ ಉಸಿರಿನಂತಿತ್ತು ಎಂದು ಅನುಭವಿಸಬಹುದಾಗಿತ್ತು. ಅವರಲ್ಲಿದ್ದ ಚಾರಿತ್ರಿಕ ದೃಷ್ಟಿ ಅದ್ಭುತವಾಗಿತ್ತು. ನಾವು ಟೆರೈ ಪ್ರಾಂತ್ಯದಲ್ಲಿ ಮಳೆಗಾಲ ಪ್ರಾರಂಭವಾಗುವುದಕ್ಕೆ ಮುಂಚೆ ಗ್ರೀಷ್ಮ ಋತುವಿನ ಒಂದು ಮಧ್ಯಾಹ್ನ ಪ್ರಯಾಣ ಮಾಡುತ್ತಿದ್ದೆವು. ಅದೇ ದಾರಿಯಲ್ಲಿ ಯುವಕ ಸಿದ್ಧಾರ್ಥ ಸಂಚರಿಸಿದ್ದು. ಆ ತ್ಯಾಗಿ ಬುದ್ಧನನ್ನು ಸ್ಪರ್ಶಿಸಿದ ಭೂಮಿ ಇದೇ ಎಂದು ನಮಗೆ ಅರಿವಾಗುವಂತೆ ಮಾಡಿದರು. ಸ್ವೇಚ್ಛೆಯಾಗಿ ಸಂಚರಿಸುತ್ತಿದ್ದ ನವಿಲು ತಂಡ ರಜಪುಟಾಣ ಮತ್ತು ಅವರ ಲಾವಣಿಗಳನ್ನು ಮನಸ್ಸಿಗೆ ತರುತ್ತಿತ್ತು. ಮಧ್ಯೆ ಯಾವಾಗಲಾದರೊಮ್ಮೆ ಒಂದು ಆನೆಯನ್ನು ನೋಡಿದರೆ ಸ್ವಾಮೀಜಿ ಹಿಂದಿನ ಯುದ್ಧ ಹೇಗೆ ನಡೆಯುತ್ತಿತ್ತು ಎಂಬುದನ್ನು ಹೇಳಲುಪಕ್ರಮಿಸುವರು. ಎಲ್ಲಿಯವರೆಗೆ ವೈರಿಗಳನ್ನು ಹಿಮ್ಮೆಟ್ಟಿಸಲು ಸಜೀವ ಫಿರಂಗಿಗಳಂತೆ ಇರುವ ಆನೆಗಳು ಇರುವುವೋ ಅಲ್ಲಿಯವರೆಗೆ ಯಾರೂ ಅವರನ್ನು ಸೋಲಿಸುವಂತೆ ಇರಲಿಲ್ಲ. 

 “ಸ್ವಾಮೀಜಿ ಹುಟ್ಟು ಪ್ರೇಮಿಗಳು. ಅವರ ಪ್ರೇಮದ ಹೃದಯೇಶ್ವರಿಯೇ ಭರತಖಂಡ. ಅವರ ಹೃದಯ ಅತಿ ಸೂಕ್ಷ್ಮವಾದ ತಂತುವಿನಿಂದ ನೇತುಹಾಕಿರುವ ಒಂದು ಗಂಟೆಯಂತೆ ಇತ್ತು. ಹೇಗೆ ಅದನ್ನು ಸ್ಪರ್ಶ ಮಾಡುವ ಯಾವ ಅತಿ ಸೂಕ್ಷ್ಮವಾದ ಶಬ್ದ ಸ್ಪಂದನವೂ ಅದರ ಮೇಲೆ ಪ್ರಭಾವವನ್ನು ಬೀರಿ ಮರುದನಿಯ ಕಿರು ಅಲೆಯನ್ನು ಎಬ್ಬಿಸಬಲ್ಲದೊ, ಹಾಗೆಯೇ ಭಾರತಖಂಡಕ್ಕೆ ಸಂಬಂಧಪಟ್ಟ ಪ್ರಶ್ನೆಗಳಿಗೆಲ್ಲ ಅವರ ಹೃದಯ ಪ್ರತಿಕ್ರಿಯೆಯಿಂದ ಅನುರಣಿತವಾಗುತ್ತಿತ್ತು. ಭರತಖಂಡದಲ್ಲಿ ಯಾರ ಸಂಕಟವಾದರೂ ಆಗಲಿ ಅವರ ಹೃದಯದಲ್ಲಿ ಅದು ಮರುದನಿಗೈಯುತ್ತಿತ್ತು. ಇವರು ಅರಿಯಲಾರದ, ಅನುಭವಿಸಲಾರದ ಯಾವ ಅಂಜಿಕೆಯು ಇರಲಿಲ್ಲ, ದೌರ್ಬಲ್ಯವಿರಲಿಲ್ಲ, ಅನುಮಾನವಿರಲಿಲ್ಲ. ತಾಯ್ನಾಡಿನ ಲೋಪದೋಷಗಳನ್ನು ದಯಾದಾಕ್ಷಿಣ್ಯವಿಲ್ಲದೆ ಖಂಡಿಸುತ್ತಿದ್ದರು. ಇಲ್ಲಿರುವ ವ್ಯಾವಹಾರಿಕ ಜ್ಞಾನದ ಅಭಾವವನ್ನು ಸಹಿಸುತ್ತಿರಲಿಲ್ಲ. ಆದರೆ ಈ ದೌರ್ಬಲ್ಯಗಳೆಲ್ಲಾ ಯಾರದೋ ಹೊರಗಿನದು ಎಂದು ಭಾವಿಸಲಿಲ್ಲ. ಇವೆಲ್ಲ ತಮ್ಮ ದೌರ್ಬಲ್ಯವೆಂತಲೇ ಪರಿಗಣಿಸುತ್ತಿದ್ದರು. ಆದರೆ ಭರತಖಂಡದ ಭವಿಷ್ಯ ಎಂದಿಗಿಂತಲೂ ಮಹಿಮಾನ್ವಿತವಾಗಿರುವುದು ಎಂಬ ಹಿರಿಯಾಸೆ ಅವರಲ್ಲಿದ್ದಂತೆ ಮತ್ತೆ ಯಾರಲ್ಲಿಯೂ ಇರಲಿಲ್ಲ. ಅವರ ದೃಷ್ಟಿಯಲ್ಲಿ ಭರತಖಂಡ ಆಂಗ್ಲೇಯರಿಗೆ ಸಂಸ್ಕೃತಿ ಕೊಟ್ಟಿತು… ಭರತಖಂಡದ ಚರಿತ್ರೆ, ಭೂಗೋಳ ಮಾನವಕುಲಶಾಸ್ತ್ರ ಎಲ್ಲವೂ ಬತ್ತದಿರುವ ಒಂದು ಮಹಾಪ್ರವಾಹದಂತೆ ಅವರ ಬಾಯಿಂದ ಹೊರಬರುತ್ತಿತ್ತು.” 

 ಮೇ ೧೭ನೇ ತಾರೀಖು ಸ್ವಾಮೀಜಿಯವರ ವೃಂದದವರು ನೈನಿತಾಲನ್ನು ತಲುಪಿದರು. ಅಲ್ಲಿ ಸ್ವಾಮೀಜಿಯವರ ಶಿಷ್ಯರಾದ ಖೇತ್ರಿ ಮಹಾರಾಜರು ವಾಸವಾಗಿದ್ದರು. ಅವರನ್ನು ಕಂಡು ಸ್ವಾಮೀಜಿ ಸಂತೋಷದಿಂದ ತಮ್ಮ ಪಾಶ್ಚಾತ್ಯ ಶಿಷ್ಯೆಯರನ್ನು ಅವರಿಗೆ ಪರಿಚಯ ಮಾಡಿಸಿದರು. ಸ್ವಾಮೀಜಿಯವರು ಇಲ್ಲಿ ಹಲವರೊಡನೆ ಮಾತನಾಡು\-ತ್ತಿದ್ದರು. ಆ ಸಮಯದಲ್ಲಿ ಸ್ವಾಮೀಜಿ ಒಂದು ದಿನ ರಾಜಾರಾಮಮೋಹನರಾಯ್ ಅವರ ವಿಷಯವನ್ನು ಕುರಿತು ಮಾತನಾಡುತ್ತಿದ್ದಾಗ, ಅವರು ವೇದಾಂತವನ್ನು ಒಪ್ಪಿಕೊಂಡಿದ್ದು, ಅವರಿಗೆ ಇದ್ದ ದೇಶಭಕ್ತಿ ಮತ್ತು ಹಿಂದೂ ಮುಸ್ಲಿಂ ಬಾಂಧವ್ಯ ಇವುಗಳನ್ನು ವಿವರಿಸಿದರು. ಸ್ವಾಮೀಜಿಯವರು ತಮ್ಮ ಪಾಶ್ಚಾತ್ಯ ಶಿಷ್ಯರಿಗೆ ಭರತಖಂಡದಲ್ಲಿ ಬಡತನ ಒಂದು ಪಾಪವಲ್ಲ, ಇಲ್ಲಿ ಸುಸಂಸ್ಕೃತರಾದವರೂ ಇರುವರು, ಧರ್ಮ ಮತ್ತು ಸಂಸ್ಕೃತಿ ಇಲ್ಲಿ ಅವರಿಗೂ ತಾಕಿದೆ ಎಂದು ಹೇಳಿದರು. ಆದರೆ ಪಾಶ್ಚಾತ್ಯದೇಶದಲ್ಲಿ ಕೆಳಮಟ್ಟದ ಜನರಿಗೆ ಇದು ಇನ್ನೂ ಅಷ್ಟು ತಾಕಿಲ್ಲವೆಂದು ವಿವರಿಸಿ ತಮ್ಮ ಜೀವನದ ಒಂದು ಅನುಭವವನ್ನು ಕೊಟ್ಟರು. ಒಂದು ದಿನ ಸ್ವಾಮೀಜಿ ತಮ್ಮ ದೇಶದ ಪೋಷಾಕಿನಲ್ಲಿ ಲಂಡನ್ ನಗರದ ಮಧ್ಯದಲ್ಲಿ ಹೋಗುತ್ತಿದ್ದಾಗ ಕಲ್ಲಿದ್ದಲು ಗಾಡಿಯನ್ನು ಹೊಡೆಯುವವನು ಅವರನ್ನು ನೋಡಿ ಅವರ ಕಡೆಗೆ ದೊಡ್ಡ ಒಂದು ಕಲ್ಲಿದ್ದಲನ್ನು ಎಸೆದನು. ಸ್ವಾಮೀಜಿ ಅವನ ಪಾಲಿಗೆ ಒಂದು ವಿಚಿತ್ರ ಮೃಗವಾಗಿ ಕಂಡಿರಬೇಕು! 

 ಸ್ವಾಮೀಜಿ ನೈನಿತಾಲಿನಲ್ಲಿ ತಮ್ಮ ಹಳೆಯ ಸ್ನೇಹಿತರಾದ ಜೋಗೇಶ್ ಚಂದ್ರದತ್ತ ಎಂಬುವರನ್ನು ಕಂಡರು. ಆತ ಸ್ವಾಮೀಜಿಯವರೊಡನೆ ಮಾತನಾಡುತ್ತಿದ್ದಾಗ ಸ್ವಲ್ಪ ಹಣವನ್ನು ಚಂದಾ ಎತ್ತಿ ಕೆಲವು ಯುವಕರನ್ನು ಐ.ಸಿ.ಎಸ್. ಪರೀಕ್ಷೆಗೆ ಇಂಗ್ಲೆಂಡಿಗೆ ಕಳುಹಿಸಬೇಕು, ಅವರು ಬಂದಮೇಲೆ ಇಂಡಿಯಾದೇಶಕ್ಕೆ ಬಹಳ ಸಹಾಯ ಮಾಡುವರು - ಎಂದರು. ಅದಕ್ಕೆ ಸ್ವಾಮೀಜಿಯವರು ಅಲ್ಲಿಂದ ಐರೋಪ್ಯರನ್ನು ಅನುಕರಿಸಿಕೊಂಡು ಬರುವರೆಂದೂ, ತಮ್ಮ ಸ್ವಾರ್ಥಕ್ಕಾಗಿ ಬಾಳುವರೆಂದೂ, ಅಂತಹವರಿಂದ ಏನೂ ಪ್ರಯೋಜನವಾಗುವುದಿಲ್ಲವೆಂದೂ ಖಂಡಿಸಿದರು. 

 ನೈನಿತಾಲಿನಿಂದ ಸ್ವಾಮೀಜಿ ಆಲ್ಮೋರಕ್ಕೆ ಹೋದರು. ಅಲ್ಲಿ ಸ್ವಾಮೀಜಿ ಗುರುಭಾಯಿಗಳೆಲ್ಲ ಸೇವಿಯರ್ಸ್‍‍ ದಂಪತಿಗಳ ಅತಿಥಿಗಳಾದರು. ಸ್ವಾಮೀಜಿಯವರ ಪಾಶ್ಚಾತ್ಯ ಮಹಿಳೆಯರು ಬೇರೆ ಒಂದು ಮನೆಯನ್ನು ಬಾಡಿಗೆಗೆ ಮಾಡಿಕೊಂಡರು. ಸ್ವಾಮೀಜಿ ಪ್ರತಿದಿನ ಬೆಳಿಗ್ಗೆ ಪಾಶ್ಚಾತ್ಯ ಶಿಷ್ಯವರ್ಗದವರಿದ್ದ ಮನೆಗೆ ಬಂದು ಬೆಳಿಗಿನ ಉಪಾಹಾರವನ್ನು ಸ್ವೀಕರಿಸಿದ ಮೇಲೆ ಹಲವು ವಿಷಯಗಳ ಮೇಲೆ ಮಾತನಾಡುತ್ತಿದ್ದರು. ಇಲ್ಲಿಯೇ ತಮ್ಮ ಪಾಶ್ಚಾತ್ಯ ಶಿಷ್ಯರಿಗೆ ಹಿಂದೂ ಸಂಸ್ಕೃತಿಯ ಪ್ರಾಣವೆಲ್ಲಿದೆ ಎಂಬುದನ್ನು ಹೃದಯಂಗಮವಾಗಿ ವಿವರಿಸಿದರು. ಅವರಿಗೆ ಹಿಂದೂ ಸಂಸ್ಕೃತಿಯ ದೀಕ್ಷೆ ಕೊಟ್ಟರು, ಎಂದು ಬೇಕಾದರೆ ಹೇಳಬಹುದು. 

 ಸ್ವಾಮೀಜಿಯವರು ಆಲ್ಮೋರದಲ್ಲಿದ್ದಾಗ ಹಲವು ಜನರನ್ನು ಕಂಡರು. ಶ‍್ರೀಮತಿ ಅನಿಬೆಸೆಂಟರು ಸ್ವಾಮೀಜಿಯವರ ಸ್ನೇಹಿತರಾದ ಕೆ.ಎನ್.ಚಕ್ರವರ್ತಿ ಎಂಬುವರ\break ಮನೆಯಲ್ಲಿದ್ದರು. ಚಕ್ರವರ್ತಿ ಸ್ವಾಮೀಜಿಯವರನ್ನು ಒಂದು ದಿನ ಟೀಗೆ ತಮ್ಮ ಮನೆಗೆ ಕರೆದರು. ಅನಂತರ ಶ‍್ರೀಮತಿ ಬುಲ್ ಅವರು ತಮ್ಮ ಮನೆಗೆ ಅನಿಬೆಸೆಂಟ್ ಅವರನ್ನು ಕರೆದರು. ಬೆಸೆಂಟ್ ಮತ್ತು ಸ್ವಾಮೀಜಿ ಇಬ್ಬರೂ ಹಲವು ವಿಷಯಗಳ ಮೇಲೆ ಮಾತನಾಡಿದರು. 

 ಸ್ವಾಮೀಜಿ ಇಲ್ಲಿದ್ದಾಗ ಕೆಲವು ದಿನಗಳ ಮಟ್ಟಿಗೆ ಸ್ವಲ್ಪ ದೂರದಲ್ಲಿರುವ ಶಿಯಾದೇವಿ ಎಂಬ ಹಳ್ಳಿಗೆ ತಾವೊಬ್ಬರೇ ಹೋಗಿ ಮೂರು ನಾಲ್ಕು ದಿನಗಳು ಅಲ್ಲಿ ದೀರ್ಘಧ್ಯಾನದಲ್ಲಿ ಕಳೆದು ಬಂದರು. ಅನಂತರ ಮೇ ೩೦ನೇ ತಾರೀಖು ಸುಮಾರು ಒಂದು ವಾರದವರಿಗೆ ಸುತ್ತಮುತ್ತಲಿರುವ ಸ್ಥಳಗಳನ್ನು ನೋಡಿ ಬರುತ್ತಿದ್ದರು. ಹಿಮಾಲಯದಲ್ಲಿ ಆಶ್ರಮವನ್ನು ಸ್ಥಾಪಿಸುವುದಕ್ಕೆ ಒಂದು ಸೂಕ್ತವಾದ ಸ್ಥಳ ಸಿಕ್ಕಲಿಲ್ಲ. ಹಿಂತಿರುಗಿ ಬಂದಮೇಲೆ ಸ್ವಾಮೀಜಿಯವರಿಗೆ ಎರಡು ದುಃಖದ ವಾರ್ತೆಗಳು ಕಾದಿದ್ದವು. ಅದೇ ಪವಾಹಾರಿಬಾಬರ ಮರಣ. ಸ್ವಾಮೀಜಿ, ಅವರನ್ನು ಪರಮಹಂಸರಿಗೆ ಎರಡನೆಯವರು ಎಂದು ಗೌರವಿಸುತ್ತಿದ್ದರು. ಮತ್ತೊಂದು ಇವರ ಶೀಘ್ರಲಿಪಿಕಾರ ಗುಡ್‍ವಿನ್​ಮರಣ. ಆತ ಸ್ವಾಮೀಜಿಯವರಿಂದ ಬೀಳ್ಕೊಂಡಮೇಲೆ ಮದ್ರಾಸಿನಲ್ಲಿ \enginline{Madras Mail} ಆಫೀಸಿನಲ್ಲಿ ಕೆಲಸಕ್ಕೆ ಇದ್ದ. ಆತ ಉದಕಮಂಡಲಕ್ಕೆ ವಿಶ್ರಾಂತಿಗೆ ಹೋಗಿದ್ದವನು ಟೈಫಾಯಿಡ್ ಖಾಯಿಲೆಯಿಂದ ಜೂನ್ ೨ನೇ ತಾರೀಖು ನಿಧನನಾದ. ಸ್ವಾಮೀಜಿ ಈ ವಾರ್ತೆಯನ್ನು ಕೇಳಿದಾಗ ತಮ್ಮ ಬಲಗೈನಂತೆ ಇದ್ದವನು ಹೋದನೆಂದು ವ್ಯಥೆಪಟ್ಟರು. ಆತನ ಹೆಸರಿನಲ್ಲಿ ಒಂದು ಕವನವನ್ನು ಇಂಗ್ಲೀಷಿನಲ್ಲಿ ಬರೆದು ಅವನ ವಿಧವೆಯಾದ ತಾಯಿಗೆ ಕಳುಹಿಸಿದರು. 

 ಸ್ವಾಮೀಜಿ ಇಲ್ಲಿ ಇದ್ದಾಗ ಮದ್ರಾಸಿನಲ್ಲಿ ಪ್ರಚಾರವಾಗುತ್ತಿದ್ದ ‘ಪ್ರಬುದ್ಧ ಭಾರತ’ ಎಂಬ ಮಾಸಪತ್ರಿಕೆ ಸದ್ಯಕ್ಕೆ ನಿಂತುಹೋಗಿತ್ತು. ಅದನ್ನು ನಡೆಸುತ್ತಿದ್ದ ರಾಜಮ್ ಅಯ್ಯರ್ ಎಂಬುವರು ಅಲ್ಪ ವಯಸ್ಸಿನಲ್ಲಿಯೇ ತೀರಿಹೋದರು. ಅದನ್ನು ಆಲ್ಮೋರಕ್ಕೆ ಬದಲಾಯಿಸಬೇಕೆಂದು ಸ್ವಾಮೀಜಿ ಮನಸ್ಸು ಮಾಡಿದರು. ಸ್ವರೂಪಾನಂದರನ್ನು ಅದಕ್ಕೆ ಸಂಪಾದಕರನ್ನಾಗಿ ಮಾಡಿ, ಸೇವಿಯರ್ಸ್‍‍ ಅವರನ್ನು ಮ್ಯಾನೇಜರ್ ಆಗಿ ಮಾಡಿದರು. ಸೇವಿಯರ್ಸ್‍‍ ಅವರು ಅದಕ್ಕಾಗಿ ಒಂದು ಸಣ್ಣ ಕೈಯಿಂದ ನಡೆಸುವ ಪ್ರೆಸ್ಸನ್ನು ಕೊಳ್ಳಲು ಸಿದ್ಧವಾದರು. ಸ್ವಾಮೀಜಿಯವರು ಹೊಸದಾಗಿ ಬರುವ ‘ಪ್ರಬುದ್ಧ ಭಾರತ’ಕ್ಕೆ ಇಂಗ್ಲೀಷಿನಲ್ಲಿ ಒಂದು ಪದ್ಯವನ್ನು ಬರೆದು ಹರಸಿಕಳುಹಿಸಿದರು. 

 ಒಂದು ದಿನ ಸ್ವಾಮೀಜಿಯ ಬಾಲ್ಯ ಸ್ನೇಹಿತರೂ ಮತ್ತು ಪರಮಹಂಸರ ಶಿಷ್ಯರೂ ಆದ ಅಶ್ವಿನೀಕುಮಾರದತ್ತರು ಸ್ವಾಮೀಜಿಯವರನ್ನು ನೋಡುವುದಕ್ಕೆ ಆಲ್ಮೋರಕ್ಕೆ\break ಬಂದರು. ಅವರಿದ್ದ ಮನೆಯಲ್ಲಿ “ಇಲ್ಲಿ ನರೇಂದ್ರದತ್ತರು ಇರುವರೆ” ಎಂದು ವಿಚಾರಿಸಿದರು. ಅದಕ್ಕೆ ಆ ಮನೆಯಲ್ಲಿದ್ದ ಒಬ್ಬ ಯುವಕ ಸ್ವಾಮಿಗಳು ಜುಗುಪ್ಸೆಯಿಂದ “ನರೇಂದ್ರನಾಥದತ್ತ ಯಾವತ್ತೋ ಕಾಲವಾದ. ಈಗ ಸ್ವಾಮಿ ವಿವೇಕಾನಂದರಿರುವರು” ಎಂದು ಹೇಳಿದರು. ಅದಕ್ಕೆ ಅಶ್ವಿನೀಕುಮಾರ ದತ್ತರು “ನನಗೆ ಸ್ವಾಮಿ ವಿವೇಕಾನಂದ ಬೇಕಾಗಿಲ್ಲ. ಪರಮಹಂಸರ ನರೇಂದ್ರ ಬೇಕು” ಎಂದರು. ಸ್ವಾಮೀಜಿಗೆ ವಿಷಯ ಗೊತ್ತಾಗಿ ಅವರು ಅಶ್ವಿನೀಬಾಬುಗಳನ್ನು ಒಳಕ್ಕೆ ಕರೆದರು. ಅಶ್ವಿನೀಬಾಬುಗಳನ್ನು ನೋಡಿದಾಗ ಸ್ವಾಮೀಜಿ ಎದ್ದು ನಿಂತು ಅವರನ್ನು ಸ್ವಾಗತಿಸಿದರು. ಅಶ್ವಿನೀಬಾಬು “ಸ್ವಾಮೀಜಿ, ಶ‍್ರೀರಾಮಕೃಷ್ಣರು ನನಗೆ ನರೇಂದ್ರನೊಂದಿಗೆ ಮಾತನಾಡೆಂದು ಹೇಳಿದ್ದರು. ಆದರೆ ಆಗ ಮಾತನಾಡಲು ಆಗಲಿಲ್ಲ. ಈಗ ಹದಿನಾಲ್ಕು ವರ್ಷಗಳು ಆಗಿಹೋದವು. ಈಗ ನಿಮ್ಮನ್ನು ನೋಡುತ್ತಿರುವೆನು. ಶ‍್ರೀ ಗುರುದೇವರ ಮಾತು ಎಂದಿಗೂ ವ್ಯರ್ಥವಾಗಕೂಡದು” ಎಂದರು. ಅಶ್ವಿನೀಬಾಬು ಸ್ವಾಮೀಜಿಯವರನ್ನು ‘ಸ್ವಾಮೀಜಿ’ ಎಂದು ಕರೆದಾಗ, “ನಾನು ಎಂದು ನಿನಗೆ ಸ್ವಾಮಿಯಾದೆ? ನಾನು ಈಗಲೂ ಅದೇ ನರೇಂದ್ರ. ನನ್ನ ಗುರುದೇವರು ನನ್ನನ್ನು ಯಾವ ಹೆಸರಿನಿಂದ ಕರೆಯುತ್ತಿದ್ದರೊ ಅದೇ ಅತ್ಯಂತ ಪ್ರಿಯವಾದುದು. ನನ್ನನ್ನು ಅದೇ ಹೆಸರಿನಿಂದ ಕರೆ” ಎಂದರು. 

 ಅಶ್ವಿನೀಬಾಬು: “ನೀವು ಪ್ರಪಂಚದಲ್ಲೆಲ್ಲಾ ಸಂಚಾರ ಮಾಡಿ ಸಹಸ್ರಾರು ಜೀವಿಗಳಿಗೆ ಆಧ್ಯಾತ್ಮಿಕ ಜೀವನದಲ್ಲಿ ಸ್ಫೂರ್ತಿಯನ್ನು ಕೊಟ್ಟಿರುವಿರಿ. ಭರತಖಂಡದ ಉದ್ಧಾರ ಹೇಗೆ ಆಗಬಹುದು ಎಂಬುದನ್ನು ಹೇಳಬಲ್ಲಿರಾ?” 

 ಸ್ವಾಮೀಜಿ: “ನೀವು ಶ‍್ರೀಗುರುದೇವರ ಮುಖದಿಂದ ಏನನ್ನು ಕೇಳಿದಿರೋ ಅದಕ್ಕಿಂತ ಹೆಚ್ಚು ನಾನು ಏನನ್ನೂ ಹೇಳಲಾರೆ. ಧರ್ಮವೇ ನಮ್ಮ ಜೀವಾಳ. ಸಾಧಾರಣ ಜನರು ಸ್ವೀಕರಿಸಬಲ್ಲಂತಹ ಸುಧಾರಣೆಗಳೆಲ್ಲ ಇದರ ಮೂಲಕ ಬರಬೇಕು. ಹಾಗಲ್ಲದೆ ಬೇರೆ ದಾರಿಯನ್ನು ಹಿಡಿದರೆ ಗಂಗಾನದಿಯನ್ನು ಹಿಮಾಲಯದ ಮೂಲಕ್ಕೆ ತಳ್ಳಿ ಬೇರೊಂದು ಪಾತ್ರದಲ್ಲಿ ಹರಿಯುವಂತೆ ಮಾಡಿದಂತೆ.” 

 ಪ್ರಶ್ನೆ: “ಕಾಂಗ್ರೆಸ್ ಏನು ಮಾಡುತ್ತಿದೆ? ನಿಮಗೆ ಅದರಲ್ಲಿ ನಂಬಿಕೆಯಿಲ್ಲವೆ?” 

 ಸ್ವಾಮೀಜಿ: “ಇಲ್ಲ. ಆದರೆ ಏನೂ ಮಾಡದೆ ಇರುವುದಕ್ಕಿಂತ ಏನನ್ನಾದರೂ‌ ಮಾಡುವುದು ಒಳ್ಳೆಯದು. ನಿದ್ರಿಸುತ್ತಿರುವ ದೇಶವನ್ನು ಎಲ್ಲಾ ಕಡೆಯಿಂದಲೂ ಎಚ್ಚರಿಸಬೇಕಾಗಿದೆ. ಕಾಂಗ್ರೆಸ್ ಜನಸಾಧಾರಣರಿಗೆ ಏನನ್ನು ಮಾಡುತ್ತಿದೆ ನನಗೆ ಹೇಳಬಲ್ಲೆಯಾ? ಕೆಲವು ಠರಾವುಗಳನ್ನು ಪಾಸುಮಾಡಿದರೆ ಸ್ವಾತಂತ್ರ್ಯ ಸಿಕ್ಕುವುದೆ? ನನಗೆ ಇದರಲ್ಲಿ ನಂಬಿಕೆ ಇಲ್ಲ. ಮೊದಲು ಜನಸಾಧಾರಣರನ್ನು ಜಾಗ್ರತಗೊಳಿಸಬೇಕು. ಅವರಿಗೆ ಹೊಟ್ಟೆ ತುಂಬ ಊಟ ಸಿಕ್ಕಲಿ. ಅನಂತರ ಅವರು ತಮ್ಮ ಉದ್ಧಾರವನ್ನು ತಾವೇ ಮಾಡಿಕೊಳ್ಳುವರು. ಕಾಂಗ್ರೆಸ್ ಅವರಿಗೇನಾದರೂ ಮಾಡಿದರೆ ಅವರಿಗೆ ನನ್ನ ಸಹಾನುಭೂತಿಯನ್ನು ತೋರುವೆ. ಇಂಗ್ಲೀಷಿನವರಲ್ಲಿನ ಒಳ್ಳೆಯ ಗುಣಗಳನ್ನು ಕೂಡ ನಾವು ಅನುಷ್ಠಾನಕ್ಕೆ ತರಬೇಕು.” 

 ಪ್ರಶ್ನೆ: “ನಿಮ್ಮ ಪ್ರಕಾರ ‘ಧರ್ಮ’ ಎಂದರೆ ಯಾವುದಾದರೂ ಮತವೆ?” 

 ಸ್ವಾಮೀಜಿ: “ಶ‍್ರೀ ಗುರುದೇವರು ಯಾವುದಾದರೂ ಮತಗಳನ್ನು ಬೋಧಿಸಿದರೆ? ಅವರು ವೇದಾಂತವನ್ನು ಬೋಧಿಸಿದರು. ಅದು ಎಲ್ಲವನ್ನೂ ಒಳಗೊಳ್ಳುವ ಸಮನ್ವಯ ದೃಷ್ಟಿ. ನಾನು ಕೂಡ ಅದನ್ನೇ ಬೋಧಿಸುವೆ. ಆದರೆ ನನ್ನ ಧರ್ಮದ ಸಾರವೇ ಶಕ್ತಿ. ಯಾವ ಧರ್ಮ ಹೃದಯಕ್ಕೆ ಶಕ್ತಿಯನ್ನು ಕೊಡಲಾರದೊ ಅದು ನನ್ನ ದೃಷ್ಟಿಯಿಂದ ಧರ್ಮವೇ ಅಲ್ಲ. ಅದು ಉಪನಿಷತ್ತು, ಗೀತಾ, ಭಾಗವತ ಯಾವುದು ಬೇಕಾದರೂ ಆಗಬಹುದು. ಶಕ್ತಿಯೇ ಧರ್ಮ, ಶಕ್ತಿಗಿಂತ ಮಿಗಿಲಾಗಿರುವುದು ಯಾವುದೂ ಇಲ್ಲ. 

 ಪ್ರಶ್ನೆ: “ನಾವು ಏನು ಮಾಡಬೇಕು. ದಯವಿಟ್ಟು ಹೇಳಿ.” 

 ಸ್ವಾಮೀಜಿ: “ನೀವು ಯಾವುದೋ ವಿದ್ಯಾಪ್ರಚಾರದ ಕೆಲಸದಲ್ಲಿ ಇರುವಿರೆಂದು ಕೇಳಿದೆ. ಅದು ನಿಜವಾದ ಕೆಲಸ. ನಿಮ್ಮ ಮೂಲಕ ಒಂದು ದೊಡ್ಡ ಶಕ್ತಿ ಕೆಲಸ ಮಾಡುತ್ತಿದೆ. ವಿದ್ಯಾದಾನ ಶ್ರೇಷ್ಠವಾದ ದಾನ. ಜನಸಾಧಾರಣದಲ್ಲಿ ಪುರುಷಸಿಂಹರಾಗುವಂತಹ ವಿದ್ಯೆಯನ್ನು ಪ್ರಚಾರ ಮಾಡಿ. ನಿಮ್ಮ ವಿದ್ಯಾರ್ಥಿಗಳ ಶೀಲವನ್ನು ವಜ್ರಾಯುಧದಂತೆ ಅಭೇದ್ಯವನ್ನಾಗಿ ಮಾಡಿ. ಬಂಗಾಳಿ ಯುವಕರ ಮೂಳೆ ವಜ್ರಾಯುಧದಂತೆ ಇರಬೇಕು. ಗುಲಾಮಗಿರಿಯ ಶೃಂಖಲೆಯನ್ನು ಇಂದು ಕಿತ್ತೊಗೆಯಬೇಕು. ನೀವು ನನಗೆ ಯೋಗ್ಯರಾದ ಕೆಲವು ವಿದ್ಯಾರ್ಥಿಗಳನ್ನು ಕೊಡಬಲ್ಲಿರಾ? ಆಗ ಪ್ರಪಂಚವನ್ನೇ ನಾನು ಅಲ್ಲಾಡಿಸಿ ಬಿಡುತ್ತೇನೆ. 

 “ರಾಧಾಕೃಷ್ಣರ ಸಂಗೀತ ಎಲ್ಲಿ ಆಗುತ್ತಿದೆಯೋ ಅಲ್ಲಿ ಚಾವಟಿಯಿಂದ ಹೊಡೆದು ಅದನ್ನು ನಿಲ್ಲಿಸಿ. ಅದರಿಂದ ಇಡೀ ದೇಶ ಸರ್ವನಾಶವಾಗುತ್ತಿದೆ. ಇಂದ್ರಿಯ ನಿಗ್ರಹವನ್ನು ಮಾಡದ ಜನರು ಅಂತಹ ಹಾಡುಗಳಲ್ಲಿ ಉನ್ಮತ್ತರಾಗಿರುವರು. ಜೀವನದಲ್ಲಿ ಉತ್ತಮ ಚಾರಿತ್ರ್ಯವಿಲ್ಲದೇ ಅಧ್ಯಾತ್ಮ ಶಕ್ತಿ ವ್ಯಕ್ತವಾಗಲಾರದು. ಇದು ಒಂದು ಹುಡುಗಾಟವೆ? ನಾವು ಬೇಕಾದಷ್ಟು ಹಾಡಿ ಕುಣಿದಿರುವೆವು. ಸದ್ಯಕ್ಕೆ ಸ್ವಲ್ಪ ಕಾಲ ಅದನ್ನು ನಿಲ್ಲಿಸಿದರೆ ಅದರಿಂದ ಯಾವ ತೊಂದರೆಯೂ ಇಲ್ಲ. ಸದ್ಯದಲ್ಲಿ ದೇಶ ಬಲಿಷ್ಠವಾಗಲಿ.” 

 “ಹರಿಜನರು, ಚಮ್ಮಾರರು, ಜಾಡಮಾಲಿಗಳು ಮುಂತಾದವರ ಬಳಿಗೆ ಹೋಗಿ ಹೇಳಿ: “ನೀವೇ ಜನಾಂಗದ ಜೀವಾಳ. ಪ್ರಪಂಚವನ್ನೇ ಜಾಗ್ರತಮಾಡುವ ಶಕ್ತಿ ನಿಮ್ಮಲ್ಲಿದೆ. ಎದ್ದುನಿಂತು ನಿಮ್ಮ ಬಂಧನಗಳನ್ನೆಲ್ಲ ಕಿತ್ತೊಗೆಯಿರಿ. ಪ್ರಪಂಚವೇ ನಿಮ್ಮನ್ನು ನೋಡಿ ಆಶ್ಚರ‍್ಯ ಪಡುತ್ತದೆ.” ಅಂತಹವರ ಮಧ್ಯದಲ್ಲಿ ಶಾಲೆಗಳನ್ನು ತೆಗೆಯಿರಿ. ಅವರಿಗೆ ಯಜ್ಞೋಪವೀತವನ್ನು ಕೊಡಿ.” 

 ಸ್ವಾಮೀಜಿಯವರು ಊಟದ ಸಮಯವಾದುದರಿಂದ ಊಟಕ್ಕೆ ಎದ್ದರು.\break ಅಶ್ವಿನೀಬಾಬು ಕೂಡ ಹೋಗುವುದಕ್ಕೆ ಎದ್ದರು. ಆಗ ಅಶ್ವಿನೀಬಾಬು ಸ್ವಾಮೀಜಿ ಅವರನ್ನು “ಮದ್ರಾಸಿನ ಬ್ರಾಹ್ಮಣರು ನಿಮ್ಮನ್ನು, ‘ನೀವು ಶೂದ್ರರು, ನಿಮಗೆ ವೇದವನ್ನು ಬೋಧಿಸಲು ಅಧಿಕಾರವಿಲ್ಲ’ ಎಂದು ಬೋಧಿಸಿದಾಗ, ನೀವು ‘ನಾನು ಶೂದ್ರನಾದರೆ ನೀವು ಪರೆಯರಲ್ಲಿ ಪರೆಯರು’ ಎಂದು ಹೇಳಿದ್ದು ನಿಜವೆ?” ಎಂದು ಕೇಳಿದರು. 

 ಅದಕ್ಕೆ ಸ್ವಾಮೀಜಿ “ಹೌದು” ಎಂದರು. “ನಿಮ್ಮಂತಹ ಗುರುಗಳು ಮತ್ತು ಇಂದ್ರಿಯ ಜಯಿಗಳು ಹಾಗೆ ಕೋಪಮಾಡಿಕೊಂಡಿದ್ದು ಕೆಟ್ಟದ್ದಲ್ಲವೆ?” ಎಂದು ಅಶ್ವಿನೀಬಾಬು ಕೇಳಿದರು. 

 ಸ್ವಾಮೀಜಿ: “ನಾನು ಸರಿ ಎಂದು ಯಾವಾಗಲೂ ಹೇಳಲಿಲ್ಲ. ಆ ಜನರ ಮೌಢ್ಯದಿಂದ ನನಗೆ ಕೋಪಬಂದು ಏನೇನೋ ಅಂದುಬಿಟ್ಟೆ. ನಾನೇನು ಮಾಡಲಿ. ಆದರೆ ನಾನು ಅದನ್ನು ಸರಿ ಎಂದು ಸಾಧಿಸುವುದಿಲ್ಲ.” 

 ಆಗ ಅಶ್ವಿನೀಬಾಬು ಸ್ವಾಮೀಜಿಯವರನ್ನು ಆಲಿಂಗಿಸಿಕೊಂಡರು. “ಈಗ ನೀವು ನನ್ನ ದೃಷ್ಟಿಯಲ್ಲಿ ಎಂದಿಗಿಂತಲೂ ಮಹಿಮಾಮಯರಾಗಿರುವಿರಿ. ನೀವು ಹೇಗೆ ವಿಶ್ವ ದಿಗ್ವಿಜಯಗಳಾದಿರಿ, ಮತ್ತು ಶ‍್ರೀ ಗುರುದೇವರು ನಿಮ್ಮನ್ನು ಏತಕ್ಕೆ ಅಷ್ಟು ಪ್ರೀತಿಸುತ್ತಿದ್ದರು ಎಂಬುದು ಗೊತ್ತಾಯಿತು” ಎಂದರು. 

 ಸ್ವಾಮೀಜಿ ಜೂನ್ ೧೧ನೇ ತಾರೀಖು ಪಾಶ್ಚಾತ್ಯ ಶಿಷ್ಯರೊಡನೆ ಕಾಶ್ಮೀರದ ಕಡೆಗೆ ಶ‍್ರೀಮತಿ ಓಲ್‍ಬುಲ್ ಅವರ ಅತಿಥಿಗಳಾಗಿ ಹೊರಟರು. 

