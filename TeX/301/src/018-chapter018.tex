
\chapter{ಮದ್ರಾಸಿನಲ್ಲಿ}

ಸ್ವಾಮೀಜಿಯವರು ಸಮುದ್ರದ ಬಂಡೆಯಿಂದ ಪುನಃ ಈಜಿಕೊಂಡು ಬಂದು ದೇವಿಗೆ ನಮಿಸಿ ಪಾದಚಾರಿಗಳಾಗಿ ರಾಮನಾಡಿನ ಕಡೆ ಹೊರಟರು. ಅಲ್ಲಿಂದ ಪಾಂಡಿಚೇರಿಗೆ ಹೊರಟರು. ಅಲ್ಲಿ ಸ್ವಾಮೀಜಿ ಒಬ್ಬ ಪಂಡಿತರ ಹತ್ತಿರ ಸಮುದ್ರಯಾನದ ವಿಚಾರವಾಗಿ ಮಾತನಾಡುತ್ತಿದ್ದಾಗ ಆ ಪಂಡಿತರು ಎಂದಿಗೂ ಸಮುದ್ರ ಯಾನ ಮಾಡಕೂಡದು ಎಂದು ವಾದಿಸಿದರು. ಸ್ವಾಮೀಜಿ ಎಷ್ಟು ವಾದಿಸಿದರೂ ಆ ಪಂಡಿತರು ತಮ್ಮ ಅಭಿಪ್ರಾಯವನ್ನು ಬದಲಾಯಿಸುವಂತೆ ಇರಲಿಲ್ಲ. ಹಿಂದೆ ಭರತಖಂಡದಿಂದ ಜನ ಹೊರಗೆ ಹೋಗುತ್ತಿದ್ದರು. ಇಲ್ಲಿಯ ಸಂಸ್ಕೃತಿ, ಸಾಹಿತ್ಯ, ಧರ್ಮ ಇವುಗಳೆಲ್ಲ ಹೊರಗೆ ಹೋಗಿವೆ. ಜಾವ ಸುಮಾತ್ರ, ಸಿಲಬೆಸ್ ಬೋರ್ನಿಯೋ ಮುಂತಾದ ಕಡೆ ರಾಮಾಯಣ ಮಹಾಭಾರತ ಗ್ರಂಥಗಳು ಪ್ರಚಾರದಲ್ಲಿವೆ, ಹಿಂದೂ ದೇವದೇವಿಯರ ದೇವಸ್ಥಾನಗಳು ಅಲ್ಲಿವೆ, ಇಂಡಿಯಾ ದೇಶದ ನಗರಗಳ ಹೆಸರುಗಳಿವೆ. ಅವೆಲ್ಲ ಅಲ್ಲಿಗೆ ಹೇಗೆ ಹೋದವು? ವಿಕಾಸವೇ ಜೀವನ, ಸಂಕೋಚವೇ ಮರಣ. ಯಾವ ಒಂದು ಸಂಸ್ಕೃತಿಯಾಗಲಿ ತನ್ನಲ್ಲಿರುವುದನ್ನು ಹೊರಗೆ ಕೊಡಬೇಕು. ಹೊರಗೆ ಇರುವ ಒಳ್ಳೆಯದನ್ನು ಹೀರಿಕೊಂಡು ನಮ್ಮದನ್ನಾಗಿ ಮಾಡಿಕೊಳ್ಳಬೇಕು. ಕೊಟ್ಟರೆ ವಿಕಾಸವಾಗುವುದು, ಬೆಳೆಯುವುದು. ಯಾವಾಗ ಹೊರಗಿನಿಂದ ನಮಗೆ ಕಲಿಯುವುದಕ್ಕೆ ಏನೂ ಇಲ್ಲ ಎಂದು ಭಾವಿಸುವೆವೂ ಆಗ ನಾವು ಬಾವಿಯ ಕಪ್ಪೆಗಳಾಗುವೆವು. ಭರತಖಂಡದ ಅವನತಿಗೆ ಇದೊಂದು ಮುಖ್ಯ ಕಾರಣವೆಂದು ಸ್ವಾಮೀಜಿ ಹೇಳುತಿದ್ದರು. ಭರತಖಂಡದಿಂದ ಹೋಗಿ ಬರುವ ಪ್ರತಿಯೊಬ್ಬರೂ ಭರತಖಂಡಕ್ಕೆ ಸಹಾಯ ಮಾಡುತ್ತಿರುವರು ಎನ್ನುತ್ತಿದ್ದರು. ಹೊರಗೆ ಹೋದಾಗ ಇತರ ಜನಾಂಗಗಳು ಹೇಗೆ ಕೆಲಸ ಮಾಡುತ್ತಿವೆ ಎಂಬುದನ್ನು ನೋಡಬಹುದು. ಆಗ ನಮ್ಮ ದೃಷ್ಟಿ ವಿಕಾಸವಾಗುವುದು. 

 ತಿರುವನಂತಪುರದಲ್ಲಿ ಪರಿಚಯವಿದ್ದ ಮನ್ಮಥನಾಥ ಭಟ್ಟಾಚಾರ‍್ಯರು ಸ್ವಾಮೀಜಿ ಅವರನ್ನು ಇಲ್ಲಿ ಕಂಡು ಅವರನ್ನು ಮದ್ರಾಸಿಗೆ ಕರೆದುಕೊಂಡು ಹೋಗಿ ತಮ್ಮ ಮನೆಯಲ್ಲಿ ಅತಿಥಿಗಳಾಗಿ ಇಟ್ಟುಕೊಂಡರು. ಮದ್ರಾಸಿನಲ್ಲೆ ಸ್ವಾಮೀಜಿಯವರಿಗೆ, ಅವರ ಅಭಿಪ್ರಾಯಗಳನ್ನೆಲ್ಲ ಚೆನ್ನಾಗಿ ಅರ್ಥಮಾಡಿಕೊಂಡು ಅದನ್ನು ಕಾರ್ಯಗತ ಮಾಡಬಲ್ಲಂತಹ ಶಿಷ್ಯರು ಮತ್ತು ಸ್ನೇಹಿತರು ಸಿಕ್ಕಿದರು. ವಿವೇಕಾನಂದರನ್ನು ಮದ್ರಾಸಿನ ಜನರೇ ಅಮೇರಿಕಾ ದೇಶಕ್ಕೆ ಕಳುಹಿಸಿದರು ಎಂದು ಬೇಕಾದರೆ ಹೇಳಬಹುದು. ಇಲ್ಲಿಯೇ ಸ್ವಾಮೀಜಿಯವರ ಆಧ್ಯಾತ್ಮಿಕ ಮಾಸಪತ್ರಿಕೆಗಳು ಪ್ರಾರಂಭವಾದವು. ಸ್ವಾಮೀಜಿಯವರೊಡನೆ ಮಾತನಾಡಲು ಹಲವು ಬಗೆಯ ಜನರು ಬರುತ್ತಿದ್ದರು, ಹಲವು ವಿಷಯಗಳನ್ನು ಅವರ ಹತ್ತಿರ ಮಾತನಾಡುತ್ತಿದ್ದರು. 

 ಒಂದು ದಿನ ಒಬ್ಬರು, “ಹಿಂದೂಗಳಲ್ಲಿ ಅಂತಹ ವೇದಾಂತದ ತತ್ತ್ವವಿದ್ದರೂ ಏತಕ್ಕೆ ಅವರು ವಿಗ್ರಹಾರಾಧಕರಾಗಿರುವರು?” ಎಂದು ಪ್ರಶ್ನಿಸಿದರು. ಅದಕ್ಕೆ ಸ್ವಾಮೀಜಿ, “ಏತಕ್ಕೆಂದರೆ, ನಮಗೆ ಹಿಮಾಲಯ ಪರ್ವತವಿರುವುದರಿಂದ” ಎಂದು ಉತ್ತರಕೊಟ್ಟರು. ಅಂದರೆ ಅಂತಹ ಸೌಂದರ‍್ಯ ಸುತ್ತಮುತ್ತಲೂ ಇರುವಾಗ ಆ ಸೌಂದರ‍್ಯವನ್ನು ನಿರಾಕರಿಸದೆ ಆ ಸೌಂದರ್ಯ ಭಗವಂತನೆಡೆಗೆ ಒಯ್ಯುವ ಮಾರ್ಗವೆಂದು ಸ್ವೀಕರಿಸಬೇಕು ಎಂಬುದು ಅದರ ಅಂತರಾರ್ಥ. 

 ಒಂದು ದಿನ ಒಬ್ಬ ವಿದ್ಯಾವಂತರು “ಸಮಯವಿಲ್ಲದೇ ಇದ್ದರೆ ಸಂಧ್ಯಾವಂದನೆಯನ್ನು ಬಿಡುವುದರಲ್ಲಿ ಏನೂ ಅಪಾಯವಿಲ್ಲವೆ?” ಎಂದು ಕೇಳಿದರು. ಸ್ವಾಮೀಜಿ ಅವರು “ಹಿಂದಿನ ಕಾಲದ ಮಹಾಋಷಿಗಳಿಗೆ ಇರುವ ಹಲವಾರು ಕೆಲಸಗಳೊಂದಿಗೆ ಸಂಧ್ಯಾವಂದನಾದಿ ಕರ್ಮಗಳನ್ನು ಮಾಡಲು ಸಮಯ ಇರುತ್ತಿತ್ತು. ಅವರ ಜೀವನದೊಂದಿಗೆ ಹೋಲಿಸಿದರೆ ನಿಮ್ಮದು ವಿರಾಮದ ಜೀವನ. ನಿಮಗೆ ಸ್ವಲ್ಪ ಸಂಧ್ಯಾವಂದನೆ ಮಾಡುವುದಕ್ಕೂ ಸಮಯವಿಲ್ಲವೆ” ಎಂದು ಮೂದಲಿಸಿದರು. 

 ಒಬ್ಬರು ವೇದಗಳನ್ನು ಅಗೌರವದಿಂದ ಕಂಡರು, “ಕೆಲಸಕ್ಕೆ ಬಾರದ ಬೋಧನೆಗಳು ಇಲ್ಲಿ ಬೇಕಾದಷ್ಟು ಇವೆ” ಎಂದರು. ಅದಕ್ಕೆ ಸ್ವಾಮೀಜಿ, “ನಿಮ್ಮ ಪೂರ್ವಿಕರನ್ನು ಟೀಕಿಸಲು ನಿಮಗೆ ಎದೆಗಾರಿಕೆ ಎಷ್ಟು? ಒಂದು ಸ್ವಲ್ಪ ವಿದ್ಯೆ ನಿಮ್ಮ ತಲೆಯನ್ನು ತಿರುಗಿಸಿಬಿಟ್ಟಿದೆ. ಋಷಿಗಳು ಹೇಳಿರುವುದನ್ನು ನೀವು ಪರೀಕ್ಷೆ ಮಾಡಿ ನೋಡಿರುವಿರಾ? ನೀವು ಅವುಗಳನ್ನು ಸ್ವಲ್ಪ ಓದಿಯಾದರೂ ಇರುವಿರಾ? ಆ ಋಷಿಗಳು ನಿಮಗೆ ಒಂದು ಸವಾಲನ್ನು ಹಾಕಿರುವರು. ಧೈರ್ಯವಿದ್ದರೆ ಅದನ್ನು ಸ್ವೀಕರಿಸಿ” ಎಂದರು. 

 ಮದ್ರಾಸಿನಲ್ಲಿರುವಾಗ ಸಂಜೆ ಹೊತ್ತು ಸಮುದ್ರ ತೀರದ ಕಡೆ ಗಾಳಿ ಸಂಚಾರಕ್ಕೆ ಹೋಗುತ್ತಿದ್ದರು. ಅಲ್ಲಿ ತೀರದಲ್ಲಿ ಬಡ ಬೆಸ್ತರು, ತಮ್ಮ ವೃದ್ಧ ತಾಯಂದಿರೊಂದಿಗೆ ಮೀನು ಹಿಡಿಯುವುದನ್ನು ನೋಡಿ “ದೇವರೇ ಇಷ್ಟೊಂದು ವ್ಯಥೆಪಡುತ್ತಿರುವ ಜೀವಿಗಳನ್ನು ಏತಕ್ಕೆ ಸೃಷ್ಟಿಸಿದೆ?” ಎಂದು ಕಂಬನಿ ಸುರಿಸಿದರು. 

 ಒಂದು ದಿನ ಮನ್ಮಥನಾಥ ಭಟ್ಟಾಚಾರ‍್ಯರು ಸ್ವಾಮೀಜಿ ಜ್ಞಾಪಕಾರ್ಥವಾಗಿ ತಮ್ಮ ಮನೆಯಲ್ಲಿ ಒಂದು ಔತಣವನ್ನು ಕೊಟ್ಟರು. ಮದ್ರಾಸಿನ ಪ್ರಮುಖರೆಲ್ಲ ಅಲ್ಲಿಗೆ ಬಂದಿದ್ದರು. ಅವರೆದುರಿಗೆ ಸ್ವಾಮೀಜಿ ಧೈರ್ಯದಿಂದ ತಾವು ಅದ್ವೈತಿಗಳು ಎಂದು ಸಾರಿದರು. ಮತ್ತೆ ಕೆಲವರು ಅವರೊಂದಿಗೆ ವಾದ ಮಾಡುವುದಕ್ಕಾಗಿ “ನೀವು ಬ್ರಹ್ಮನೊಂದಿಗೆ ಒಂದು ಎಂದು ಹೇಳುತ್ತೀರಿ. ಹಾಗಾದರೆ ನಿಮ್ಮ ಜವಾಬ್ದಾರಿ ಏನೂ ಇರುವುದಿಲ್ಲ. ನೀವು ತಪ್ಪು ಮಾಡಿದರೆ ಅದನ್ನು ತಡೆಯುವವರಾರು?” ಎಂದು ಕೇಳಿದರು. ಅದಕ್ಕೆ ಸ್ವಾಮೀಜಿ “ನಾನು ಸತ್ಯವಾಗಿ ದೇವರಲ್ಲಿ ಒಂದಾಗಿರುವೆನು ಎಂಬುದನ್ನು ನಂಬಿದರೆ, ಆಗ ನಾನು ಪಾಪಕ್ಕೆ ಆಸ್ಪದವನ್ನೇ ಕೊಡುವುದಿಲ್ಲ. ಆಗ ಕೆಟ್ಟ ಮಾರ್ಗಕ್ಕೆ ಹೋಗದಂತೆ ನನ್ನನ್ನು ತಡೆಯುವುದಕ್ಕೆ ಯಾರು ಬೇಕಾಗಿರುವರು?” ಎಂದರು. ಹಿಂದೆ ಒಂದು ಸಲ ಅವರು ರಾಮನಾಡಿನ ಅರಮನೆಯಲ್ಲಿದ್ದಾಗ ಅವ್ಯಕ್ತನಾದ ಬ್ರಹ್ಮನನ್ನು ನೋಡಲು ಯಾರಿಗಾದರೂ ಸಾಧ್ಯವಾಗುವುದೆ ಎಂದು ಯಾರೋ ಅಣಕಿಸಿದ್ದರು. ಅದಕ್ಕೆ ಸ್ವಾಮೀಜಿ, “ನಾನು ಅವ್ಯಕ್ತವನ್ನು ಕಂಡಿರುವೆನು” ಎಂದು ಧೈರ್ಯವಾಗಿ ಉತ್ತರ ಕೊಟ್ಟಿದ್ದರು. 

 ಮದ್ರಾಸಿನಲ್ಲಿದ್ದ ಸಮಾಜ ಸುಧಾರಕರು ಸ್ವಾಮೀಜಿ ಹತ್ತಿರ ಬಂದು ವಾದಿಸುತ್ತಿದ್ದರು. ಅವರಿಗೆ ಸ್ವಾಮೀಜಿ, ತಾವೂ ಕೂಡ ಒಬ್ಬ ಸುಧಾರಕರೆ, ಆದರೆ ನಮ್ಮ ಹಿಂದಿನ ಒಳ್ಳೆಯದನ್ನೆಲ್ಲ ಬಿಟ್ಟು ಈಗ ಹೊಸತಾಗಿ ಏನೇನೋ ಸೃಷ್ಟಿ ಮಾಡುತ್ತೇವೆ ಅನ್ನುವರ ಗುಂಪಿಗೆ ಸೇರಿಲ್ಲ ಎಂದರು. ಸುಧಾರಣೆ ಒಳಗಿನಿಂದ ಬರಬೇಕು. ಮೇಲಿನಿಂದ ಬಲಾತ್ಕಾರವಾಗಿ ಹೇರುವುದಲ್ಲ. 

 ಒಂದು ಸಲ ಕ್ರಿಶ್ಚಿಯನ್ ಕಾಲೇಜಿನ ವಿಜ್ಞಾನ ಶಾಸ್ತ್ರದ ಪ್ರಾಧ್ಯಾಪಕರಾದ ಶಿಂಗಾರವೇಲು ಮೊದಲಿಯಾರ್ ಎಂಬುವರು ಸ್ವಾಮೀಜಿಯವರನ್ನು ನೋಡಲು ಬಂದನು. ಆತ ನಾಸ್ತಿಕನಾಗಿದ್ದ. ಸ್ವಾಮೀಜಿಯವರೊಡನೆ ಮಾತನಾಡಿದಮೇಲೆ ಅವನ ನಾಸ್ತಿಕತೆ ಹಾರಿಹೋಯಿತು. ಸ್ವಾಮೀಜಿ ಅವನನ್ನು ಕಿಡ್ಡಿ ಎಂದು ಹಾಸ್ಯದಿಂದ ಕರೆಯುತ್ತಿದ್ದರು. ಶಿಂಗಾರವೇಲು ಸ್ವಾಮೀಜಿಯವರ ಅವಿಚ್ಛಿನ್ನ ಭಕ್ತನಾದನು. ಒಂದು ಸಲ ಸ್ವಾಮೀಜಿ ಹಾಸ್ಯವಾಗಿ ಹೇಳಿದರು: “ಸೀಸರ್ ಬಂದ, ನೋಡಿದ, ಗೆದ್ದ ಎಂಬ ಗಾದೆ ಇದೆ. ಅದರಂತೆಯೇ ಶಿಂಗಾರವೇಲು ಬಂದ, ನೋಡಿದ, ಆದರೆ ಸೋತ!” 

 ವಿ. ಸುಬ್ರಹ್ಮಣ್ಯ ಅಯ್ಯರ್ ಅವರು ತಮ್ಮ ಸಹಪಾಠಿಗಳೊಡನೆ ಸ್ವಾಮೀಜಿ ಇದ್ದ ಸ್ಥಳಕ್ಕೆ ಬಂದು ಅವರ ಹತ್ತಿರ ಚರ್ಚೆ ಮಾಡಬೇಕೆಂದು ಬಯಸಿದರು. ಆಗ ಸ್ವಾಮೀಜಿ ಕುರ್ಚಿಯ ಮೇಲೆ ಕುಳಿತುಕೊಂಡು ಗುಡಿಗುಡಿಯನ್ನು ಸೇದುತ್ತಿದ್ದರು. ಅವರು ಆಗ ಯಾವುದೋ ಕನಸಿನ ಲೋಕದಲ್ಲಿದ್ದಂತೆ ಭಾಸವಾಗುತ್ತಿತ್ತು. ಬಂದ ಹುಡುಗರು ಸ್ವಾಮೀಜಿಗೆ “ದೇವರು ಎಂದರೆ ಏನು?” ಎಂದು ಪ್ರಶ್ನೆ ಹಾಕಿದರು. ಸ್ವಾಮೀಜಿ ಹುಡುಗರಿಗೆ “\enginline{ENERGY} ಎಂದರೆ ಏನು?” ಎಂದು ಮರುಪ್ರಶ್ನೆ ಹಾಕಿದರು. ಆ ಹುಡುಗರು ಸ್ವಾಮೀಜಿಗೆ ಪ್ರಶ್ನೆ ಹಾಕುವುದಕ್ಕೆ ಬಂದಿದ್ದರೇ ಹೊರತು ಅವರು ತಮಗೆ ಪ್ರಶ್ನೆ ಹಾಕುವರೆಂದು ಕನಸ್ಸಿನಲ್ಲಿಯೂ ಭಾವಿಸಿರಲಿಲ್ಲ. ಅವರು ಕಕ್ಕಾಬಿಕ್ಕಿಯಾದರು. ಆಗ ಸ್ವಾಮೀಜಿ “ಯಾವ \enginline{Energy}ಯನ್ನು ನೀವು ಬೆಳಗಿನಿಂದ ಸಾಯಂಕಾಲದವರೆಗೆ ಉಪಯೋಗಿಸುತ್ತಿರುವಿರೋ ಅದೇ ನಿಮಗೆ ಗೊತ್ತಿಲ್ಲ. ನನ್ನನ್ನು ದೇವರು ಎಂದರೆ ಏನು ಎಂದು ಪ್ರಶ್ನೆ ಮಾಡುವಿರಲ್ಲ” ಎಂದರು. ಇತರ ಪ್ರಶ್ನೆಗಳನ್ನು ಕೇಳಿಯಾದ ಮೇಲೆ ಸ್ವಾಮೀಜಿ ಅದಕ್ಕೆ ತಕ್ಕ ಉತ್ತರವನ್ನು ಕೊಟ್ಟರು. ಸುಬ್ರಹ್ಮಣ್ಯ ಅಯ್ಯರ್ ಸ್ವಾಮೀಜಿಯವರೊಡನೆ ತುಂಬಾ ಸ್ನೇಹ ಬೆಳಸಿದರು. ಅವರೊಡನೆ ಸಮುದ್ರ ತೀರಕ್ಕೆ ತಿರುಗಾಡಲು ಹೋದರು. ಆಗ ಸ್ವಾಮೀಜಿ ಅಯ್ಯರ್ ಅವರನ್ನು “ನಿನಗೆ ಕುಸ್ತಿ ಬರುವುದೆ?” ಎಂದು ಕೇಳಿದರು. ಅವರು ಬರುತ್ತದೆ ಎಂದಾಗ “ನನ್ನೊಡನೆ ಮಾಡು” ಎಂದರು. ಕುಸ್ತಿಯಾಡಿ ಸ್ವಾಮೀಜಿ ಅಯ್ಯರ್ ಅವರನ್ನು ಮರಳ ಮೇಲೆ ಉರುಳಿಸಿಬಿಟ್ಟರು. ಅವರು ಧೂಳನ್ನು ಕೊಡವಿಕೊಂಡು ಮೇಲೆದ್ದು, “ಸ್ವಾಮಿಗಳು ಹೊರಗೆ ಕಾಷಾಯ ಹಾಕಿಕೊಂಡಿದ್ದರೂ ಒಳಗೆ ಜಟ್ಟಿಯ ಮೈಕಟ್ಟಿದೆ” ಎಂದರು. ಅವರನ್ನು “ಪೈಲ್ವಾನ್ ಸ್ವಾಮಿಗಳು” ಎಂದು ಅನಂತರ ಕರೆಯುತ್ತಿದ್ದರು.

 ಒಂದು ದಿನ ಸ್ವಾಮೀಜಿ ಹತ್ತಿರ ಇದ್ದ ಮೈಸೂರು ಮಹಾರಾಜರು ಕೊಟ್ಟ ಹೊಗೆಯ ಕೊಳವೆಯ ಸೌಂದರ್ಯವನ್ನು ಮನ್ಮಥನಾಥ ಭಟ್ಟಾಚಾರ್ಯನ ಅಡಿಗೆಯವನು ನೋಡಿ ಆನಂದಿಸುತ್ತಿದ್ದ. ಸ್ವಾಮೀಜಿ ಅದನ್ನು ಗ್ರಹಿಸಿದರು. “ಅದು ನಿನಗೆ ಬೇಕೆ?” ಎಂದು ಕೇಳಿದರು. ಅವನಿಗೆ ಯಾವ ಉತ್ತರವನ್ನೂ ಕೊಡುವುದಕ್ಕೆ ತೋಚಲಿಲ್ಲ. ಸ್ವಾಮೀಜಿಯವರು ಅವನಿಗೆ “ತೆಗೆದುಕೋ” ಎಂದು ಕೊಟ್ಟುಬಿಟ್ಟರು. 

 ಸ್ವಾಮೀಜಿಯವರು ಮದ್ರಾಸಿನಲ್ಲಿದ್ದಾಗ ಅವರಿಗೆ ಒಂದು ವಿಚಿತ್ರ ಅನುಭವವಾಯಿತು. ಕಾಲವಾದ ಅವರ ಹತ್ತಿರ ಬಂಧುಗಳ ಪ್ರೇತ ಅವರಿಗೆ ಕಾಣಿಸಿಕೊಂಡಂತೆ ಆಗಿ ಹಲವು ಸುದ್ದಿಗಳನ್ನು ಕೊಡುತ್ತಿತ್ತು. ಒಂದು ದಿನ ಸ್ವಾಮೀಜಿಯವರ ತಾಯಿ ಕಲ್ಕತ್ತೆಯಲ್ಲಿ ತೀರಿಹೋದಳು ಎಂದು ಹೇಳಿತು. ಸ್ವಾಮೀಜಿಗೆ ಮೊದಲಿನಿಂದಲೂ ತಾಯಿ ಎಂದರೆ ಪ್ರಾಣ. ತಕ್ಷಣವೇ ಕಲ್ಕತ್ತೆಗೆ ಒಂದು ತಂತಿಯನ್ನು ಕಳುಹಿಸಿದರು. ತಾಯಿ\break ಆರೋಗ್ಯದಿಂದ ಇರುವಳು ಎಂಬ ಸುದ್ದಿ ಬಂತು. ಆ ಪ್ರೇತ ಮತ್ತೊಮ್ಮೆ ಇವರ ಮುಂದೆ ಬಂದಾಗ ಸ್ವಾಮೀಜಿ, “ನೀನು ಏತಕ್ಕೆ ಸುಳ್ಳನ್ನು ಹೇಳಿದೆ?” ಎಂದು ಪ್ರಶ್ನಿಸಿದರು. ಅದಕ್ಕೆ ಆ ಪ್ರೇತ ಸ್ವರೂಪಿ ತಾನು ಆ ಸ್ಥಿತಿಯಲ್ಲಿ ಮತ್ತೇನನ್ನೂ ಮಾಡಲು ಸಾಧ್ಯವಿಲ್ಲವೆಂದೂ, ತನ್ನನ್ನು ಪ್ರೇತಸ್ವರೂಪದಿಂದ ಪಾರು ಮಾಡಬೇಕೆಂದೂ ಬೇಡಿಕೊಂಡಿತು. ಸ್ವಾಮೀಜಿ ಮಾರನೆ ದಿನ ಸಮುದ್ರತೀರಕ್ಕೆ ಹೋಗಿದ್ದಾಗ ಕೈಯಲ್ಲಿ ಮರಳನ್ನು ಹಿಡಿದುಕೊಂಡು ಸಮುದ್ರದಲ್ಲಿ ನಿಂತು ಆ ಪ್ರೇತ ಸ್ವರೂಪಿಗಾಗಿ ಪ್ರಾರ್ಥಿಸಿದ ಮೇಲೆ ಅದು ಆ ಸ್ಥಿತಿಯಿಂದ ಪಾರಾಯಿತು. 

 ಸ್ವಾಮೀಜಿಯವರು ಮದ್ರಾಸಿನಲ್ಲಿದಾಗ ಅವರ ಭಕ್ತಾದಿಗಳೆಲ್ಲ ಸ್ವಾಮೀಜಿಯವರನ್ನು ಅಮೆರಿಕಾ ದೇಶಕ್ಕೆ ಹೋಗಬೇಕೆಂದು ಒತ್ತಾಯ ಮಾಡುತ್ತಿದ್ದರು. ಸ್ವಾಮೀಜಿ ಮನಸ್ಸಿನಲ್ಲಿಯೂ ಆ ಭಾವನೆ ಇತ್ತು. ಆದರೆ ಅದು ಕೇವಲ ತಮ್ಮ ಇಚ್ಛೆಯಾದ ಮಾತ್ರಕ್ಕೆ ಅವರು ಹೋಗಲು ಇಚ್ಛಿಸಲಿಲ್ಲ. ಅದು ಭಗವಂತನ ಪ್ರೇರಣೆ ಆದರೆ ಮಾತ್ರ ಹೋಗುತ್ತೇನೆ ಎಂದರು. ಸ್ವಾಮೀಜಿಯವರ ದಾರಿ ಖರ್ಚಿಗೆಂದು ಅಳಸಿಂಗ ಪೆರುಮಾಳ್ ಎಂಬ ಸ್ವಾಮೀಜಿ ಭಕ್ತರು ಮತ್ತು ಇನ್ನು ಕೆಲವರು ಹಣವನ್ನು ಚಂದಾ ಎತ್ತುತ್ತಿದ್ದರು. ಅವರು ಸುಮಾರು ಐನೂರು ರೂಪಾಯಿಗಳನ್ನು ಸಂಗ್ರಹಿಸಿದಾಗ, ಸ್ವಾಮೀಜಿ ಅವರಿಗೆ “ಆ ದುಡ್ಡನ್ನೆಲ್ಲ ಬಡಬಗ್ಗರಿಗೆ ದಾನ ಮಾಡಿಬಿಡಿ. ನಾನು ಅಮೆರಿಕಾ ದೇಶಕ್ಕೆ ಹೋಗಬೇಕೆನ್ನುವುದು ಈಶ್ವರ ಪ್ರೇರಣೆ ಆಗಿದ್ದರೆ ಅವನೇ ಹೇಗೊ ಸಹಾಯವನ್ನು ಒದಗಿಸುವನು” ಎಂದು ಹೇಳಿಬಿಟ್ಟರು. 

 ಸ್ವಾಮೀಜಿಯವರು ಮದ್ರಾಸಿನ ಜನರ ಮೇಲೆ ಮಾಡುತ್ತಿರುವ ಪ್ರಭಾವ\break ಹೈದರಾಬಾದಿನಲ್ಲಿ ಹಲವರಿಗೆ ಗೊತ್ತಾಯಿತು. ಅಲ್ಲಿಂದ ಹಲವರು ಸ್ವಾಮೀಜಿಯವರನ್ನು ಹೈದರಾಬಾದಿಗೆ ಬರುವಂತೆ ಕೇಳಿಕೊಂಡರು. ಮನೆಯ ಯಜಮಾನರು ಹೈದರಾಬಾದಿನಲ್ಲಿರುವ ಬಾಬು ಮಧುಸೂಧನ ಚಟರ್ಜಿ ಎಂಬ ಸೂಪರಿಂಟೆಂಡಿಂಗ್ ಇಂಜಿನೀಯರ್ ಅವರಿಗೆ ತಂತಿ ಕಳುಹಿಸಿ ಸ್ವಾಮೀಜಿಯವರನ್ನು ರೈಲಿನಲ್ಲಿ\break ಹೈದರಾಬಾದಿಗೆ ಹಳುಹಿಸಿದರು. 

