
\chapter{ಅಮೇರಿಕಾ ದೇಶಕ್ಕೆ ಹೋಗುವ ಮುನ್ನ}

 ಮದ್ರಾಸಿಗೆ ಬಂದಮೇಲೆ ಸ್ವಾಮೀಜಿಯವರು ಎಂದಿನಂತೆ ಭಕ್ತಾದಿಗಳಿಗೆ ಪ್ರವಚನ\break ಮಾಡುತ್ತಿದ್ದರು. ಅವರ ಶಿಷ್ಯರು, ಅಳಸಿಂಗ ಪೆರುಮಾಳ್ ಎಂಬುವರ ನೇತೃತ್ವದಲ್ಲಿ ಸ್ವಾಮೀಜಿಯವರು ಅಮೇರಿಕಾ ಪ್ರವಾಸಕ್ಕೆ ಹಣವನ್ನು ಸಂಗ್ರಹಿಸಲು ಪುನಃ ಪ್ರಯತ್ನಿಸತೊಡಗಿದರು. ಹಿಂದೆ ಸ್ವಾಮೀಜಿಯವರಿಗೆ ಸಹಾಯ ಮಾಡುತ್ತೇವೆ ಎಂಬುವರನ್ನೆಲ್ಲಾ ಅಳಸಿಂಗ ಪೆರುಮಾಳ್ ನೋಡಲು ಹೋದರು. ಮದ್ರಾಸಿನಲ್ಲೆ ಮಧ್ಯಮವರ್ಗದವರಿಂದ ಸಾಧ್ಯವಾದಷ್ಟು ಹಣವನ್ನು ಚಂದಾ ಎತ್ತಲು ನಿರತರಾದರು. ಸ್ವಾಮೀಜಿ ಶಿಷ್ಯರ ಉತ್ಸಾಹವನ್ನು ನೋಡಿ “ಬಹುಶಃ ಇದು ನಾನು ಅಮೇರಿಕಾಕ್ಕೆ ಹೋಗಬೇಕು ಎಂಬುದಕ್ಕೆ ಪ್ರಥಮ ಸೂಚನೆ ಇರಬಹುದು” ಎಂದು ಭಾವಿಸತೊಡಗಿದರು. 

 ಸ್ವಾಮೀಜಿ ತಮ್ಮ ಗುರುದೇವರನ್ನು ಪ್ರಾರ್ಥಿಸತೊಡಗಿದರು. ತಾವು ಹೋಗುವುದು ಸರಿಯೇ, ಅಲ್ಲವೆ, ಎಂಬುದಕ್ಕೆ ಸಪ್ರಮಾಣವಾಗಿ ತಮಗೆ ಅನುಭವ ಆಗಬೇಕೆಂದು ಕೋರಿಕೊಂಡರು. ಸ್ವಾಮೀಜಿಯವರು ಒಂದು ದಿನ ಅರೆನಿದ್ರೆಯಲ್ಲಿರುವಾಗ ಅವರಿಗೊಂದು ಕನಸಾಯಿತು. ಆ ಕನಸಿನಲ್ಲಿ ಶ‍್ರೀರಾಮಕೃಷ್ಣರು ಸಮುದ್ರದ ಮೇಲೆ ನಡೆದುಕೊಂಡು ಹೋಗುತ್ತಿದ್ದರು, ಹಿಂದಿನಿಂದ ಸ್ವಾಮೀಜಿಗೆ ತನ್ನನ್ನನುಸರಿಸುವಂತೆ ಸೂಚಿಸುತ್ತಿದ್ದರು.\break ಇದನ್ನು ಕಂಡಾದಮೇಲೆ, ಸ್ವಾಮೀಜಿಗೆ ಇದು ಗುರುದೇವನ ಆಜ್ಞೆಯನ್ನು ಸೂಚಿಸುವ ಕನಸೆಂದು ತೃಪ್ತರಾದರು. ಹೋಗುವುದಕ್ಕೆ ಮುಂಚೆ ಕಲ್ಕತ್ತೆಯ ಬಳಿ ಇದ್ದ ತಮ್ಮ ಗುರುಪತ್ನಿಗಳಾದ ಶ‍್ರೀಶಾರದಾದೇವಿಯವರಿಗೆ ತಾವು ಅಮೇರಿಕಾದೇಶಕ್ಕೆ ಹೋಗೆಂದು ಜನ ಹೇಳುತ್ತಿರುವರೆಂದೂ, ಅದಕ್ಕೆ ತಮ್ಮ ಒಪ್ಪಿಗೆ ಇದ್ದರೆ ಆಶೀರ್ವದಿಸಿ ಒಂದು ಪತ್ರವನ್ನು ಬರೆಯಬೇಕೆಂದೂ ಪ್ರಾರ್ಥಿಸಿಕೊಂಡು ಒಂದು ಪತ್ರವನ್ನು ಬರೆದರು. ಆ ಪತ್ರದಲ್ಲಿಯೇ ತಾವು ಅಮೇರಿಕಾದೇಶಕ್ಕೆ ಹೋಗುವ ವಿಚಾರವನ್ನು ಯಾರಿಗೂ ಹೇಳಕೂಡದೆಂದು ವಿಜ್ಞಾಪಿಸಿಕೊಂಡಿದ್ದರು. ಶ‍್ರೀಶಾರದಾದೇವಿಯವರು ಸ್ವಾಮೀಜಿಯವರಿಗೆ ತಮ್ಮ ಆಶೀರ್ವಚನವನ್ನು ಕಳುಹಿಸಿ, ಶ‍್ರೀಗುರುದೇವರು ಅವರನ್ನು ಕಾಪಾಡಲೆಂದು ಪ್ರಾರ್ಥಿಸಿ ಒಂದು ಕಾಗದವನ್ನು ಬರೆದರು. ಸ್ವಾಮೀಜಿಯವರಿಗೆ ಆ ಕಾಗದ ಬಂದ ಮೇಲೆ ಮನಸ್ಸಿನಲ್ಲಿ ಯಾವ ಅನುಮಾನವೂ ಇರಲಿಲ್ಲ. ಅಮೇರಿಕಾದೇಶಕ್ಕೆ ಹೋಗುವುದಕ್ಕೆ ಅಣಿಯಾದರು. 

\newpage

 ಆ ಸಮಯದಲ್ಲಿ ಮತ್ತೊಂದು ಪ್ರಸಂಗ ಜರುಗಿತು. ಸುಮಾರು ಎರಡು ವರ್ಷಗಳ ಹಿಂದೆ ಸ್ವಾಮೀಜಿಯವರು ಖೇತ್ರಿ ಅರಮನೆಯಲ್ಲಿದ್ದಾಗ ರಾಜರು ತಮಗೆ ಒಂದು ಗಂಡುಮಗುವಾಗುವಂತೆ ಅನುಗ್ರಹಿಸಬೇಕೆಂದು ಪ್ರಾರ್ಥಿಸಿಕೊಂಡಿದ್ದರು. ಸ್ವಾಮೀಜಿ, ದೇವರ ಇಚ್ಛೆಯಿದ್ದರೆ ನಿಮ್ಮ ಅಭೀಷ್ಟ ನೆರವೇರಲೆಂದು ಹರಸಿದ್ದರು. ಅನಂತರ ಅವರಿಗೆ ಒಂದು ಗಂಡುಮಗುವಾಯಿತು. ಖೇತ್ರಿ ಮಹಾರಾಜರು ಸ್ವಾಮೀಜಿಯವರ ಅನುಗ್ರಹದಿಂದಲೇ ಇದು ಆಯಿತೆಂದು ಆ ಮಗುವಿನ ನಾಮಕರಣದ ಸಮಯಕ್ಕೆ ಸ್ವಾಮೀಜಿಯವರನ್ನು ಒಂದು ದಿನದ ಮಟ್ಟಿಗಾದರೂ ಕರೆದುಕೊಂಡು ಬರುವಂತೆ ತಮ್ಮ ಆಪ್ತಕಾರ್ಯದರ್ಶಿಗಳಾದ ಜಗಮೋಹನಲಾಲ್ ಎಂಬುವರನ್ನು ಕಳುಹಿಸಿದರು. ಅವರು ಅಲ್ಲಿ ಇಲ್ಲಿ ವಿಚಾರಿಸಿಕೊಂಡು ಕೊನೆಗೆ ಮದ್ರಾಸಿಗೆ ಬಂದರು. ಅಲ್ಲಿ ಭಟ್ಟಾಚಾರ‍್ಯರ ಮನೆಯಲ್ಲಿ ಸ್ವಾಮೀಜಿ ಇದ್ದುದನ್ನು ನೋಡಿದರು. ಸ್ವಾಮೀಜಿಗೆ ಖೇತ್ರಿಯ ಮನುಷ್ಯ ಮದ್ರಾಸಿನಲ್ಲಿರುವುದನ್ನು ನೋಡಿ ಪರಮಾಶ್ಚರ್ಯವಾಯಿತು. ಬಂದ ಕಾರಣವನ್ನು ಕೇಳಲಾಗಿ, ಜಗಮೋಹನ್‍ಲಾಲ್ ಖೇತ್ರಿ ಮಹಾರಾಜರು ಸ್ವಾಮೀಜಿಯವರನ್ನು ಒಂದು ದಿನದ ಮಟ್ಟಿಗಾದರೂ ಕರೆದುಕೊಂಡು ಬರುವಂತೆ ಹೇಳಿರುವರೆಂದು ಭಿನ್ನವಿಸಿದರು. ಅದಕ್ಕೆ ಸ್ವಾಮೀಜಿ, ತಾವು ಮೇ ತಿಂಗಳು ೩೧ನೇ ತಾರೀಖು ಹಡಗಿನಲ್ಲಿ ಅಮೇರಿಕಾ ದೇಶಕ್ಕೆ ಹೊರಡಲು ಅಣಿಯಾಗಿರುವುದಾಗಿಯೂ, ಕೇವಲ ಇನ್ನೊಂದು ತಿಂಗಳು ಮಾತ್ರ ಇರುವುದಾಗಿಯೂ, ಈಗ ಕಾಲವಿಲ್ಲವೆಂದೂ ಹೇಳಿದರು. ಜಗಮೋಹನ್‍ಲಾಲ್ ಅವರು, ಖೇತ್ರಿ ಮಹಾರಾಜರೇ ಸ್ವಾಮೀಜಿಗೆ ಬೊಂಬಾಯಿನಿಂದ ಹಡಗಿನಲ್ಲಿ ಹೋಗುವುದಕ್ಕೆ ಎಲ್ಲವನ್ನೂ ಅಣಿಮಾಡುತ್ತಾರೆಂದೂ, ತಾವು ಒಂದು ದಿನದ ಮಟ್ಟಿಗಾದರೂ ಬರಬೇಕೆಂದೂ ಒತ್ತಾಯ ಮಾಡಿದುದರಿಂದ ಸ್ವಾಮೀಜಿ ಮದ್ರಾಸಿನ ಭಕ್ತರಿಂದ ಬೀಳ್ಕೊಂಡು ಖೇತ್ರಿ ಕಡೆಗೆ ಹೊರಟರು. 

 ಸ್ವಾಮೀಜಿ ವಪೀಂಗಣ, ಬೊಂಬಾಯಿ ಮತ್ತು ಜಯಪುರದ ಮಾರ್ಗವಾಗಿ ಖೇತ್ರಿ ತಲುಪಿದರು. ಊರೆಲ್ಲ ಸಂಭ್ರಮದಿಂದ ಅಲಂಕೃತವಾಗಿತ್ತು. ಹಲವು ಕಡೆಗಳಲ್ಲಿ ನೃತ್ಯಗಳಾಗುತ್ತಿದ್ದವು. ಅರಮನೆ ದೀಪದಿಂದ ಸುಂದರವಾಗಿ ಅಲಂಕರಿಸಲ್ಪಟ್ಟಿತ್ತು. ಆ ಸಮಯದಲ್ಲಿ ಖೇತ್ರಿ ಮಹಾರಾಜರು ದೊಡ್ಡ ಅಲಂಕೃತವಾದ ನೌಕೆಯಲ್ಲಿ ತಮ್ಮ ಇತರ ಸ್ನೇಹಿತರೊಡನೆ ಸರೋವರದಲ್ಲಿ ತೇಲುತ್ತಿದ್ದರು. ಸ್ವಾಮೀಜಿಯವರನ್ನು ಒಳಗೆ ಬರಮಾಡಿಕೊಂಡ ತಕ್ಷಣವೇ ಖೇತ್ರಿ ಮಹಾರಾಜರು ಅವರ ಪಾದಗಳಿಗೆ ಪ್ರಣಾಮ ಮಾಡಿದರು. ಇತರರೆಲ್ಲರೂ ಸ್ವಾಮೀಜಿಯವರಿಗೆ ತಮ್ಮ ಭಕ್ತಿಯ ಕಾಣಿಕೆಯನ್ನು ಅರ್ಪಿಸಿದ ಮೇಲೆ ಎಲ್ಲರೂ ನೌಕಾಸಭಾಂಗಣದಲ್ಲಿ ಕುಳಿತರು. ತಮಗೆ ಆದ ಮಗುವನ್ನು ಸ್ವಾಮೀಜಿಯವರಿಗೆ ತಂದು ತೋರಿಸಿ ಅದನ್ನು ಅವರ ಪದತಲದಲ್ಲಿ ಇಟ್ಟರು. ಸ್ವಾಮೀಜಿಯವರು ಅದಕ್ಕೆ\break ಆಶೀರ್ವದಿಸಿದರು. 

 ಕೆಲವು ದಿನಗಳು ಖೇತ್ರಿಯಲ್ಲಿದ್ದ ಮೇಲೆ ಸ್ವಾಮೀಜಿ ತಾವು ಅಮೇರಿಕಾದೇಶಕ್ಕೆ ಹೋಗಬೇಕಾಗಿರುವುದರಿಂದ ಬೊಂಬಾಯಿಗೆ ತೆರಳುತ್ತೇನೆ ಎಂದರು. ರಾಜರು ಮತ್ತು ಜಗಮೋಹನ್‍ಲಾಲರು ಜಯಪುರದವರೆಗೆ ಸ್ವಾಮೀಜಿ ಜೊತೆಯಲ್ಲಿ ಬಂದರು.\break ಜಯಪುರದ ಹೊರವಲಯದಲ್ಲಿ ಕೆಲವು ಡೇರೆಗಳನ್ನು ಹಾಕಿ ಅಲ್ಲಿ ಬಿಡಾರಮಾಡಿದ್ದರು. ಒಂದು ದಿನ ಖೇತ್ರಿ ಮಹಾರಾಜರ ಮುಂದೆ ಆಸ್ಥಾನದ ಗಾಯಕಿಯೊಬ್ಬಳನ್ನು ಹಾಡುವುದಕ್ಕೆ ಗೊತ್ತುಮಾಡಿದ್ದರು. ಆಕೆ ತಾವು ಸ್ವಾಮೀಜಿ ಸಮ್ಮುಖದಲ್ಲಿ ಕೆಲವು ಹಾಡುಗಳನ್ನು ಹಾಡಬೇಕೆಂಬ ಇಚ್ಛೆಯನ್ನು ವ್ಯಕ್ತಪಡಿಸಿದ್ದಳು. ರಾಜರು ಸ್ವಾಮೀಜಿಗೆ ಹೇಳಿಕಳುಹಿಸಿದರು. ಮೊದಲು ಸ್ವಾಮೀಜಿಯವರು ಒಬ್ಬ ನರ್ತಕಿ ಹಾಡುವ ಹಾಡನ್ನು ಕೇಳುವುದಕ್ಕೆ ತಾವು ಬರುವುದಿಲ್ಲವೆಂದು ಉತ್ತರ ಕಳುಹಿಸಿದರು. ನರ್ತಕಿ ತುಂಬಾ ವ್ಯಥೆಯಿಂದ ಭಕ್ತ ಸೂರದಾಸನ ಒಂದು ಹಾಡನ್ನು ಭಕ್ತಿಯಿಂದ ಕೇಳಿದವರೆದೆ ಕರಗುವಂತೆ ಹಾಡಿದಳು. ಆಕೆ ಹಾಡುತ್ತಿದ್ದ ಡೇರೆಗೆ ಸ್ವಲ್ಪ ದೂರದಲ್ಲಿ ಸ್ವಾಮೀಜಿ ಡೇರೆ ಇತ್ತು. ಅವರು ಬರದೇ ಇದ್ದರೂ ಆಕೆಯ ಹಾಡಿನ ಧ್ವನಿಸ್ಪಂದನ ಅವರ ಕಿವಿಗೆ ಕೇಳತೊಡಗಿತು. ಆ ಹಾಡಿನ ಭಾಷಾಂತರವೇ ಕೆಳಗಿನದು:

\begin{verse}
ನೋಡದಿರೆನ್ನಯ ದೋಷವ ದೊರೆಯೆ\\ಸಮದರ್ಶಿಯು ನೀನಲ್ಲವೆ ಹರಿಯೆ~। \\ಗುಡಿಯೊಳು ಮೂರ್ತಿಯ ಕಬ್ಬಿಣವೊಂದು \\ಕಟುಕನ ಕತ್ತಿಯು ಮತ್ತೊಂದು;\\ಪರುಸವೇದಿ ತಾ ಸೋಕಿದರೆರಡೂ \\ತಳತಳಿಸದೆ ಚೆಂಬೊನ್ನಾಗಿ~।। ೧~।।
\end{verse}

\begin{verse}
ಯಮುನೆಯಲಿರುವುದು ನೀರ್ಪನಿಯೊಂದು \\ಕೊಚ್ಚಿಯೊಳಿರುವುದು ಮತ್ತೊಂದು\\ಆ ಹನಿ ಎರಡೂ ಮಂಗಳವಾಗವೆ\\ಗಂಗೆಯ ಸೇರಲು ತೀರ್ಥದಲಿ~।। ೨~।। 
\end{verse}

 ಸ್ವಾಮೀಜಿಯವರಿಗೆ ಈ ಹಾಡನ್ನು ಕೇಳಿದೊಡನೆಯೆ ಮನಸ್ಸು ಕರಗಿತು. ದೇವರ ಕೃಪೆಗೆ ಯಾರೂ ಬಾಹಿರರಲ್ಲ. ಒಂದು ದೃಷ್ಟಿಯಿಂದ ನೋಡಿದರೆ ಸಮಾಜದ ಕಸದ ಬುಟ್ಟಿಯಲ್ಲಿ ಬಿದ್ದಿರುವ ವ್ಯಕ್ತಿ ಈಕೆ. ಆದರೆ ಭಗವತ್ ಕೃಪೆಯನ್ನು ಪಡೆಯಲು ಎಲ್ಲರಿಗೂ ಒಂದೇ ಹಕ್ಕಿದೆ. ಜೊತೆಗೆ ಸ್ವಾಮೀಜಿ ಅದ್ವೈತ ವೇದಾಂತಿ ಬೇರೆ, ಎಲ್ಲ ಬ್ರಹ್ಮಮಯ ಎಂದು ಸಾರುವವರ ಗುಂಪಿಗೆ ಸೇರಿದವರು. ಆದರೂ ವ್ಯವಹಾರದಲ್ಲಿ ಎಂತಹ ತಾರತಮ್ಯಭಾವವನ್ನು ತೋರಿರುವೆ ಎಂದು ತಮ್ಮ ನಡತೆಗೆ ತಾವೇ ನಾಚಿ ಆಕೆ ಹೇಳುವ ಹಾಡನ್ನು ಕೇಳಲು ಎದ್ದು ಬಂದರು. 

 ಅನಂತರ ಸ್ವಾಮೀಜಿ ಖೇತ್ರಿರಾಜರಿಂದ ಬೀಳ್ಕೊಂಡು ಅವರ ಆಪ್ತ ಕಾರ್ಯದರ್ಶಿಗಳೊಡನೆ ಬೊಂಬಾಯಿಗೆ ಹೊರಟರು. ಅವರಿಗೆ ಸ್ವಾಮೀಜಿ ಅಮೆರಿಕಾದೇಶಕ್ಕೆ ಹೋಗುವ ಖರ್ಚನ್ನೆಲ್ಲ ತಮ್ಮ ಪರವಾಗಿ ವಹಿಸಿಕೊಳ್ಳಬೇಕೆಂದು ಹೇಳಿದರು. ಜಯಪುರದಿಂದ ಅಬು ರೈಲ್ವೆ ಸ್ಟೇಷನ್ನಿಗೆ ಬಂದರು. ಅಲ್ಲಿ ರೈಲ್ವೆ ಅಧಿಕಾರಿಯೊಬ್ಬನ ಮನೆಯಲ್ಲಿ ರಾತ್ರಿ ಕಳೆದರು. ಆತ ಹಿಂದೆ ಸ್ವಾಮೀಜಿ ಪರಿವ್ರಾಜಕರಾಗಿ ಅಲೆಯುತ್ತಿದ್ದಾಗ ಕೆಲವು ದಿನ ಅವರಿಗೆ ತನ್ನ ಮನೆಯಲ್ಲಿ ಆತಿಥ್ಯವನ್ನು ನೀಡಿದ್ದ. 

ಅಬು ರೈಲ್ವೆ ನಿಲ್ದಾಣದಲ್ಲಿ ಸ್ವಾಮೀಜಿ ತಮ್ಮ ಗುರುಭಾಯಿಗಳಾದ ಸ್ವಾಮಿ ಬ್ರಹ್ಮಾನಂದ ಮತ್ತು ತುರೀಯಾನಂದರನ್ನು ಕಂಡರು. ಆಗ ಸ್ವಾಮೀಜಿ ತುರೀಯಾನಂದರಿಗೆ ಹೀಗೆ ಹೇಳಿದರು: “ಭಾಯ್, ನಿನ್ನ ಧರ್ಮವನ್ನು ನಾನು ಇನ್ನೂ ಚೆನ್ನಾಗಿ ತಿಳಿದುಕೊಂಡಿಲ್ಲ. ಆದರೆ ನನ್ನ ಹೃದಯ ವಿಶಾಲವಾಗಿದೆ. ಈಗ ನಾನು ಎಲ್ಲರಿಗೂ ಪರಿತಪಿಸುತ್ತೇನೆ. ಅತ್ಯಂತ ಉತ್ಕಟವಾಗಿ ಪರಿತಪಿಸುತ್ತೇನೆ.” ಇದು ಸ್ವಾಮೀಜಿ ಹೃದಯದಲ್ಲಿದ್ದ ವಿಶ್ವಕಾರುಣ್ಯವನ್ನು ವ್ಯಕ್ತಪಡಿಸುವುದು. ನಿಜವಾದ ಧರ್ಮ ಜಾತಿ ಮತಗಳ ಮೇರೆಯನ್ನು ಕಿತ್ತುಹಾಕಿ, ವಿಶ್ವವೆಲ್ಲ ಒಂದು ಕುಟುಂಬ, ನಾವು ಅದಕ್ಕೆ ಸೇರಿದವರು ಎಂಬ ಭಾವ ಬರುವಂತೆ ಮಾಡುವುದು. 

 ಸ್ವಾಮೀಜಿ ಅಬು ರೈಲ್ವೆ ಸ್ಟೇಶನ್ನಿನಲ್ಲಿ ಕುಳಿತಿದ್ದಾಗ ಒಂದು ಘಟನೆ ಜರುಗಿತು. ಸ್ವಾಮೀಜಿಯವರನ್ನು ಬೀಳ್ಕೊಡಲು ರೈಲ್ವೆ ಇಲಾಖೆಯಲ್ಲಿ ಕೆಲಸದಲ್ಲಿ ಇರುವವನೇ ಒಬ್ಬ ಸ್ವಾಮೀಜಿ ಪಕ್ಕದಲ್ಲಿ ಕುಳಿತಿದ್ದ. ಟಿಕೀಟನ್ನು ಚೆಕ್ ಮಾಡುವ ಒಬ್ಬ ಆಂಗ್ಲೇಯನು, ಸ್ವಾಮೀಜಿ ಪಕ್ಕದಲ್ಲಿ ಕುಳಿತಿದ್ದವರನ್ನು ರೈಲಿನಿಂದ ಇಳಿಯಬೇಕೆಂದು ಬಹಳ ಹೀನಾಯವಾದ ಭಾಷೆಯನ್ನು ಉಪಯೋಗಿಸಿ ಹೇಳಿದ. ಆತ ತನ್ನನ್ನು ಹಾಗೆ ಇಳಿಸಲು ಅವನಿಗೆ ಯಾವ ಕಾನೂನಿನ ಅಧಿಕಾರವೂ ಇಲ್ಲ ಎಂದ. ಇಬ್ಬರಿಗೂ ಬಿಸಿ ಬಿಸಿ ಮಾತುಗಳು ಆಗುತ್ತಿದ್ದಾಗ, ಸ್ವಾಮೀಜಿ ಸಮಾಧಾನ ಮಾಡಲು ಯತ್ನಿಸಿದರು. ಆಗ ಟಿಕೆಟ್ ಕಲೆಕ್ಟರ್ ಕೋಪ ಸ್ವಾಮೀಜಿ ಮೇಲೆ ತಿರುಗಿತು. ಹಿಂದಿಯಲ್ಲಿ ಸ್ವಾಮೀಜಿಗೆ “ತುಮ್ ಕಹೆ ಬಾತ್ ಕರ್‍ತೇ‌ ಹೋ” ಎಂದರೆ ನೀನು ಏತಕ್ಕೆ ಇಲ್ಲಿ ಮಾತನಾಡುತ್ತೀಯಾ ಎಂದು. ಹಿಂದಿಯಲ್ಲಿ ತುಮ್ ಎಂಬುದನ್ನು ತಮಗಿಂತ ಕೆಳಗೆ ಇರುವವರಿಗೆ ಉಪಯೋಗಿಸುತ್ತಾರೆ. ಗೌರವವಾಗಿ ಹೇಳಬೇಕಾದರೆ “ಆಪ್” ಎಂದು ಹೇಳಬೇಕು. ಸ್ವಾಮೀಜಿ ತಕ್ಷಣವೇ ಕುಪಿತರಾಗಿ “ತುಮ್ ಎಂದರೆ ಏನು? ತಮಗೆ ಯೋಗ್ಯವಾದ ರೀತಿಯಲ್ಲಿ ವ್ಯವಹರಿಸುವುದಕ್ಕೆ ಗೊತ್ತಿಲ್ಲವೆ? ಮೊದಲನೆ ಎರಡನೆ ದರ್ಜೆಯ ಪ್ರಯಾಣಿಕರೊಡನೆ ನೀವು ಮಾತನಾಡುತ್ತಿರುವಿರಿ. ನಿಮಗೆ ಸರಿಯಾದ ನಡತೆಯೇ ಗೊತ್ತಿಲ್ಲ. ನೀವು ಆಪ್ ಎಂದು ಏತಕ್ಕೆ ಉಪಯೋಗಿಸಲಿಲ್ಲ?” ಎಂದರು. ಟಿಕೆಟ್ ಕಲೆಕ್ಟರ್ ತನ್ನ ತಪ್ಪನ್ನು ಒಪ್ಪಿಕೊಂಡು ಇಂಗ್ಲೀಷಿನಲ್ಲಿ “ದಯವಿಟ್ಟು ಕ್ಷಮಿಸಿ. ನನಗೆ ಭಾಷೆ ಗೊತ್ತಿಲ್ಲ. ನಾನು ಬರೀ ಈ ಮನುಷ್ಯನನ್ನು (\enginline{man}) ಕೆಳಗೆ ಇಳಿ ಎಂದು ಹೇಳಿದೆ” ಎಂದನು. ಸ್ವಾಮೀಜಿ ತಕ್ಷಣವೇ‌ “ನಿಮಗೆ ಹಿಂದಿ ಗೊತ್ತಿಲ್ಲವೆಂದು ಹೇಳಿದಿರಿ. ಆದರೆ ನಿಮ್ಮ ಮಾತೃಭಾಷೆಯಾದ ಇಂಗ್ಲೀಷ್ ಕೂಡ ಗೊತ್ತಿಲ್ಲ. ನೀವು ಯಾರನ್ನು ಮನುಷ್ಯ (\enginline{man}) ಎಂದು ಕರೆದಿರೋ, ಅವರು ಸಭ್ಯ ಮನುಷ್ಯರು (\enginline{Gentleman})” ಎಂದರು. ಅನಂತರ ಟಿಕೆಟ್ ಕಲೆಕ್ಟರ್ ಮಾತನಾಡದೆ ಹೊರಟು ಹೋದ. ಅನಂತರ ಸ್ವಾಮೀಜಿ ಜಗಮೋಹನ್‍ಲಾಲ್‍ರೊಂದಿಗೆ ನಾವು ಐರೋಪ್ಯರೊಂದಿಗೆ ವ್ಯವಹರಿಸುವಾಗ ಆತ್ಮಗೌರವವನ್ನು ಬಿಡಕೂಡದೆಂದು ಹೇಳಿದರು. 

 ಸ್ವಾಮೀಜಿ ಬೊಂಬಾಯಿ ತಲುಪಿದರು. ಮದ್ರಾಸಿನಿಂದ ಸ್ವಾಮೀಜಿಯವರನ್ನು\break ಬೀಳ್ಕೊಡುವುದಕ್ಕೆ ಅಳಸಿಂಗ್ ಪೆರುಮಾಳ್ ಬಂದಿದ್ದರು. ಹಡಗು ಹೊರಡುವುದಕ್ಕೆ ಕೆಲವು ದಿನಗಳು ಮಾತ್ರ ಇದ್ದುವು. ತಮ್ಮ ಹಳೆಯ ಸ್ನೇಹಿತರನ್ನು ಕಂಡು ಮಾತುಕತೆಯಾಡಿದರು. ಅಮೇರಿಕಾದೇಶದಲ್ಲಿ ಅವರು ಹಾಕಿಕೊಳ್ಳುವುದಕ್ಕೆ ಸರಿಯಾದ ಪೋಷಾಕನ್ನು ಹೊಲಿಸಿದರು. ಕೆಲವು ಸ್ನೇಹಿತರು ಸ್ವಾಮೀಜಿಗೆ ಕಾವಿಯ ಬಣ್ಣದ ರೇಷ್ಮೆಯ ಶರಾಯಿ ಮತ್ತು ಅದೇ ಬಣ್ಣದ ನಿಲುವಂಗಿಯನ್ನು ಹೊಲಿಸಿದರು. ಸ್ವಾಮೀಜಿಯವರನ್ನು ಆ ವೇಷದಲ್ಲಿ ನೋಡಿದರೆ ಒಬ್ಬ ರಾಜರಂತೆ ಕಾಣುತ್ತಿದ್ದರೇ ಹೊರತು ಸಂನ್ಯಾಸಿಯಂತೆ ಕಾಣುತ್ತಿರಲಿಲ್ಲ. ಅವರು ಎರಡೂ ಹೌದು. ಘನ ಗಾಂಭೀರ‍್ಯದಲ್ಲಿ ರಾಜರಿಗೆ ರಾಜರವರು; ಜೀವನದಲ್ಲಿ ಸಂನ್ಯಾಸಿಗೆ ಸಂನ್ಯಾಸಿಗಳು ಅವರು. 

 ಸ್ವಾಮೀಜಿಯವರಿಗೆ \enginline{P.O.C} ಗೆ ಸೇರಿದ ಪೆನಿನ್​ಸುಲಾರ್ ಎಂಬ ಹಡಗಿನಲ್ಲಿ ಮೊದಲನೆ ದರ್ಜೆಯ ಟಿಕೆಟ್ ಮತ್ತು ದಾರಿಯ ಖರ್ಚಿಗೆ ದುಡ್ಡು, ಬೇಕಾಗುವ ಬಟ್ಟೆಬರೆಗಳು ಎಲ್ಲವನ್ನೂ ಖೇತ್ರಿ ರಾಜರೇ ಜಗಮೋಹನಲಾಲರ ಮೂಲಕ ಕೊಟ್ಟಿದ್ದರು. ಮೇ ೩೧ನೇ ತಾರೀಖು ೧೮೯೩ರಲ್ಲಿ ಹಡಗು ಬೊಂಬಾಯಿ ರೇವುಪಟ್ಟಣವನ್ನು ಬಿಟ್ಟಿತು. ಹಡಗು ರೇವನ್ನು ಬಿಡುವವರೆಗೆ ಮದ್ರಾಸಿನಿಂದ ಬಂದ ಅಳಸಿಂಗ್ ಪೆರುಮಾಳ್ ಮತ್ತು\break ಜಗಮೋಹನಲಾಲ್ ಅವರು ಅಲ್ಲಿಯೇ ಇದ್ದರು. ಅನಂತರ ಅವರು ಕಂಬನಿದುಂಬಿ ಹಿಂತಿರುಗಿದರು. 

 ಹಡಗು ನಿಧಾನವಾಗಿ ತೇಲಿಕೊಂಡು ದಿಗಂತದಲ್ಲಿ ಕಣ್ಮರೆಯಾಗತೊಡಗಿತು. ಸ್ವಾಮೀಜಿ ಹಡಗಿನ ಮೇಲೆ ನಿಂತುಕೊಂಡು ಭರತಖಂಡವನ್ನು ಕಣ್ಣಿಗೆ ಕಾಣುವವರೆಗೆ ನೋಡಿದರು. ಕೆಲವು ವರುಷಗಳಿಂದ ಆಪಾದಮಸ್ತಕದವರೆಗೆ ಸಂಚರಿಸಿದ ಭರತಖಂಡ ಕಣ್ಮರೆಯಾಗತೊಡಗಿತು. ಅವರ ಮನಸ್ಸಿನಲ್ಲಿ ಬಾರಾನಗರ ಮಠದ ಗುರುಭಾಯಿಗಳು,\break ಶ‍್ರೀರಾಮಕೃಷ್ಣರು, ಶ‍್ರೀಶಾರದಾದೇವಿ – ಈ ಎಲ್ಲ ಚಿತ್ರಗಳು ಒಂದೊಂದಾಗಿ ಬಂದುಹೋದುವು. ಸ್ವಾಮೀಜಿ ತ್ಯಾಗಭೂಮಿಯಿಂದ ಈಗ ಭೋಗಭೂಮಿಯ ಕಡೆ ಹೋಗುತ್ತಿರುವೆನು ಎಂದು ನಿಡುಸುಯ್ದರು. 

 ಸ್ವಾಮೀಜಿ ಭರತಖಂಡದ ಶ್ರೇಷ್ಠವಾದ ಅಂಶಗಳನ್ನೆಲ್ಲ ಹೀರಿಕೊಂಡಿದ್ದರು. ಭರತಖಂಡ ಹಿಂದೆ ಎಂದೂ ನಮಗೆ ತಿಳಿದಿರುವ ಮಟ್ಟಿಗೆ ಇಂತಹ ಪ್ರತಿನಿಧಿಯನ್ನು ಹೊರಗೆ ಕಳುಹಿಸಿರಲಾರದು. ನಮ್ಮ ಪುರಾಣ ಪುಣ್ಯಕಥೆಗಳ ಸಾರ, ವೇದೋಪನಿಷತ್ತು ದರ್ಶನಗಳ ತತ್ತ್ವ, ಅಖಂಡ ಹಿಂದೂದೇಶದಲ್ಲಿದ್ದ ಪ್ರಮುಖ ಸಾಧುಸಂತರ ಜೀವನ ಮತ್ತು ಬೋಧನೆ ಮತ್ತು ಭರತಖಂಡದ ಸಂಸ್ಕೃತಿಯಲ್ಲಿರುವ ಸಾರವಾದ ವಸ್ತುವನ್ನೆಲ್ಲ ಸ್ವಾಮೀಜಿ ಗ್ರಹಿಸಿದ್ದರು. ವಿಶಾಲವಾದ ಪಾಂಡಿತ್ಯ, ಅದಕ್ಕಿಂತ ಹೆಚ್ಚಾದ ಅನುಭೂತಿ ಸರ್ವಮತಗಳೂ ಪರಮ ಸತ್ಯದೆಡೆಗೆ ಒಯ್ಯುವ ಒಂದೊಂದು ಪಥ ಎಂಬ ಉದಾರವಾದ ದೃಷ್ಟಿ, ಉಜ್ವಲವಾದ ರಾಷ್ಟ್ರಪ್ರೇಮ, ಇವುಗಳೆಲ್ಲಾ ಒಂದು ಅನುಪಮ ಪ್ರಮಾಣದಲ್ಲಿ ಸ್ವಾಮಿ ವಿವೇಕಾನಂದರಲ್ಲಿ ಸಂಧಿಸಿದ್ದುವು. ಸ್ವಾಮಿ ವಿವೇಕಾನಂದರೇ ಸಂಕ್ಷಿಪ್ತ ಭರತಖಂಡ. ಅವರನ್ನು ತಿಳಿದರೆ ಭರತಖಂಡವನ್ನು ತಿಳಿದಂತೆ. ಸಾವಿರಾರು ವರ್ಷಗಳು ಒಮ್ಮೆ ಮೇಲೆ, ಮತ್ತೊಮ್ಮೆ ಕೆಳಗೆ ಸಾಗಿ ಬಂದು, ಹೊರಗಡೆಯಿಂದ ನೋಡಿದರೆ ಬಡತನ ಅಜ್ಞಾನದಿಂದ ಆವೃತವಾದ ಭರತಖಂಡ, ತನ್ನ ಶ್ರೇಷ್ಠತಮ ಪುತ್ರನನ್ನು, ವಿಜ್ಞಾನ ಪ್ರಪಂಚದಲ್ಲಿ, ಯಂತ್ರ ನಾಗರೀಕತೆಯಲ್ಲಿ, ಸುಖಭೋಗಗಳಲ್ಲಿ ಉಚ್ಛ್ರಾಯದ ಪರಮ ಶಿಖರವನ್ನು ಮುಟ್ಟಿದ ಅಮೇರಿಕಾ ದೇಶಕ್ಕೆ ಕಳುಹಿಸುತ್ತಿದೆ. ಭಿಕ್ಷೆಯನ್ನು ಬೇಡುವುದಕ್ಕೆ ಅಲ್ಲ; ಅವರ ಕಣ್ಣಿಗೆ ಬೆಳಕನ್ನು ಕೊಡುವುದಕ್ಕೆ; ಅವರ ಸಂಕುಚಿತ ಹೃದಯವನ್ನು ವಿಶಾಲ ಮಾಡುವುದಕ್ಕೆ. 

