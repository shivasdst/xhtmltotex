
\chapter{ಯೂರೋಪಿನಲ್ಲಿ}

 ಸ್ವಾಮೀಜಿ ಪ್ಯಾರಿಸ್ಸಿನಲ್ಲಿದ್ದಾಗ ಅವರಿಗೆ ಹಲವು ಜನರ ಪರಿಚಯವಾಯಿತು. ಅವರಲ್ಲಿ ಮಾನ್ಯಿಯರ್ ಜೂಲಿಯಸ್ ಬಾಯ್ಸ್ ಎಂಬುವರು ಒಬ್ಬರು. ಈತ ಫ್ರಾನ್ಸಿನಲ್ಲಿ ಹೆಸರಾಂತ ತತ್ತ್ವಜ್ಞಾನಿ ಮತ್ತು ಸಾಹಿತಿ. ಹಲವು ಧರ್ಮ ಮತ್ತು ಮೂಢನಂಬಿಕೆಗಳಲ್ಲಿ ಚಾರಿತ್ರಿಕ ಭಾಗವನ್ನು ಕಂಡುಹಿಡಿಯುವುದರಲ್ಲಿ ಈತ ನಿಸ್ಸೀಮ. ಮಧ್ಯಕಾಲದ ಯೂರೋಪಿನಲ್ಲಿ ರೂಢಿಯಲ್ಲಿದ್ದ ಭೂತಾರಾಧನೆ, ಮಾಯ ಮಾಟ, ಮಂತ್ರ ಮುಂತಾದುವುಗಳನ್ನು ಮತ್ತು ಈಗಲೂ ಇರುವ ಅವಶೇಷಗಳನ್ನು ಚಾರಿತ್ರಿಕ ದೃಷ್ಟಿಯಿಂದ ವಿವರಿಸಿ ಪ್ರಖ್ಯಾತ ಪುಸ್ತಕವೊಂದನ್ನು ಬರೆದಿರುವನು. ಎಡಿನ್‍ಬರೋ ಯೂನಿವರ್ಸಿಟಿಯ ಪೆಟ್ರಿಕ್ ಗಿಡ್ಸಿಸ್ ಎಂಬುವರು ಮತ್ತೊಬ್ಬರು. ಮಿಶಿನ್‍ಗನ್ನನ್ನು ಕಂಡುಹಿಡಿದ ಹಿರಾಮ್ ಮಾಕ್ಸಿಮ್ ಎಂಬುವನೂ ಸ್ವಾಮೀಜಿಗೆ ಒಳ್ಳೆಯ ಸ್ನೇಹಿತನಾದ. ಮಾಕ್ಸಿಮ್ ಅಮೇರಿಕಾ ದೇಶದವನು. ಆದರೆ ಅವನ ಬಂದೂಕದ ಕಾರ್ಖಾನೆ ಇಂಗ್ಲೆಂಡಿನಲ್ಲಿದೆ. ಮಾಕ್ಸಿಮ್ ಎದುರಿಗೆ ಯಾರಾದರೂ ಅವನ ಬಂದೂಕವನ್ನು ಕುರಿತು ಹೊಗಳಿದರೆ, ಆತ, ಈ ಹತ್ಯಾಕಾರಿಯಾದ ಯಂತ್ರವನ್ನಲ್ಲದೆ ನಾನು ಮತ್ತೆ ಏನನ್ನೂ ಮಾಡಿಲ್ಲವೆ? ಎಂದು ಅಸಮಾಧಾನಗೊಳ್ಳುವನು. ಮಾಕ್ಸಿಮ್ ಇಂಡಿಯಾ ಮತ್ತು ಚೈನಾ ದೇಶವನ್ನು ಪ್ರೀತಿಸುವನು. ಆತ ಸ್ವಾಮೀಜಿ ಪುಸ್ತಕವನ್ನು ಓದಿದ್ದುದರಿಂದ ಅವರನ್ನು ಬಹಳ ಗೌರವಿಸುತ್ತಿದ್ದ. ಅದೇ ಸಮಯದಲ್ಲಿ ಡಾಕ್ಟರ್ ಬೋಸ್ ಅವರು ಅಲ್ಲಿಗೆ ಹೋಗಿದ್ದರು. ಅವರು ಆಗ ತಾನೆ ವಿದ್ಯುತ್ ಶಕ್ತಿಯ ಮೇಲೆ ಮಾಡುತ್ತಿದ್ದ ತಮ್ಮ ಪ್ರಯೋಗವನ್ನು ಪ್ಯಾರಿಸ್ಸಿನಲ್ಲಿ ತೋರಿಸಿ ಎಲ್ಲರ ಗೌರವಕ್ಕೆ ಪಾತ್ರರಾದರು. ಒಬ್ಬ ಪ್ರಖ್ಯಾತ ವಿಜ್ಞಾನಿಯ ವಿದ್ಯಾರ್ಥಿನಿಯೊಬ್ಬಳು ಸ್ವಾಮೀಜಿಗೆ ಅವಳ ಪ್ರಾಧ್ಯಾಪಕರು' ಒಂದು ಬಾಡಿಹೋದ ‘ಲಿಲ್ಲಿ’ಯ ಮೇಲೆ ಪ್ರಯೋಗ ಮಾಡುತ್ತಿರುವರು ಎಂದು ಹೇಳಿದಾಗ, ಸ್ವಾಮೀಜಿ “ಅದೇನು ಮಹಾ, ಆ ಲಿಲ್ಲಿ ಬೆಳೆಯುವ ಕುಂಡಕ್ಕೂ ಜೀವ ತುಂಬಬಲ್ಲನು ನಮ್ಮ ಬೋಸ್!” ಎಂದು ಹೇಳಿದರು.

 ಸ್ವಾಮೀಜಿ ಕೆಲವು ಕಾಲವಾದ ಮೇಲೆ ಶ‍್ರೀಮತಿ ಓಲ್‍ಬುಲ್ ಅವರ ನಿಮಂತ್ರಣವನ್ನು ಒಪ್ಪಿಕೊಂಡು ಬ್ರಿಟನ್ನಿಗೆ ಹೋದರು. ಅಲ್ಲಿಗೆ ಇಂಗ್ಲೆಂಡಿನಿಂದ ಸೋದರಿ\break ನಿವೇದಿತೆಯೂ ಸ್ವಾಮೀಜಿಯವರನ್ನು ನೋಡಲು ಬಂದಳು. ಆಕೆಯೂ ಅವರ ಮನೆಯಲ್ಲಿಯೇ ಇದ್ದಳು. ಆಕೆಯನ್ನು ಬೀಳ್ಕೊಡುವಾಗ ಅವರು ಹೀಗೆಂದರು: “ಮಹಮ್ಮದೀಯರಲ್ಲಿ ಒಂದು ವಿಚಿತ್ರ ಪಂಗಡವಿದೆ. ಅವರು ಹೊಸದಾಗಿ ತಮಗಾದ ಮಕ್ಕಳನ್ನು ಹೊರಗೆ ಇಟ್ಟು ‘ದೇವರು ನಿನ್ನನ್ನು ಮಾಡಿದ್ದರೆ ನಾಶವಾಗು, ಆಲಿ ನಿನ್ನನ್ನು ಮಾಡಿದ್ದರೆ ಜೀವಿಸು’ ಎಂದು ಹೇಳುತ್ತಿದ್ದರು. ಯಾವುದನ್ನು ಅವರು ಮಗುವಿಗೆ ಹೇಳುತ್ತಿದ್ದರೋ ಅದನ್ನೇ ಬೇರೆ ಅರ್ಥದಲ್ಲಿ ನಿನಗೆ ಈ ರೀತಿ ಹೇಳುತ್ತೇನೆ: ‘ಪ್ರಪಂಚಕ್ಕೆ ಹೋಗು. ನಾನು ನಿನಗೆ ಕಾರಣವಾಗಿದ್ದರೆ ನಾಶವಾಗು. ಜಗನ್ಮಾತೆ ಅದಕ್ಕೆ ಕಾರಣವಾಗಿದ್ದರೆ ಬದುಕು’.” 

 ಬ್ರಿಟಾನಿಯಿಂದ ಹಿಂತಿರುಗಿ ಬಂದಮೇಲೆ ಜೂಲಿಯಸ್ ಬಾಯ್ಸ್ ಎಂಬುವನ ಅತಿಥಿಯಾಗಿ ಕೆಲವು ಕಾಲ ಇದ್ದರು. ಅನಂತರ ಯೂರೋಪನ್ನು ನೋಡಿಕೊಂಡು ಇಂಡಿಯಾದೇಶದ ಕಡೆ ತಿರುಗಿದರು. ಅವರು ಹಿಂದಿರುಗುವಾಗ ಸ್ವಾಮೀಜಿಯವರ ವೃಂದದಲ್ಲಿ ಮ್ಯಾಡಮ್ ಕಾಲ್ವಿ, ಪೆರಿ ಹಯಾಸಿಂತ್, ಅವನ ಹೆಂಡತಿ ಮತ್ತು ಮ್ಯಾಕ್ಲಿಯಾಡ್ ಇದ್ದರು. ಮೇಡಂ ಕಾಲ್ವಿ ಪರಿಚಯ ಸ್ವಾಮೀಜಿಯವರಿಗೆ ಹಿಂದೆಯೇ ಅಮೇರಿಕಾದಲ್ಲಿ ಆಗಿತ್ತು. ಈಗ ಸ್ವಾಮೀಜಿ ಜೊತೆ ಆಕೆಯೂ ಹೊರಟಳು ಮತ್ತು ಅವರ ಖರ್ಚನ್ನೆಲ್ಲ ಈಕೆಯೇ ವಹಿಸಿಕೊಂಡಳು. ಈಕೆ ಪ್ರಖ್ಯಾತ ಸಂಗೀತಗಾರಳು. ಕೇವಲ ಸಂಗೀತದಿಂದಲೇ ವರುಷಕ್ಕೆ ಮೂರು ನಾಲ್ಕು ಲಕ್ಷ ರೂಪಾಯಿಗಳನ್ನು ಸಂಪಾದನೆ ಮಾಡುತ್ತಿದ್ದಳು. ಪೆರಿ ಹಯಾಸಿಂತ್ ಎಂಬುವನು ರೋಮನ್ ಸಂನ್ಯಾಸಿ ಪಂಗಡದ ಉಗ್ರ ಕ್ಯಾಥೋಲಿಕ್ ಪ್ರಪಂಚದಲ್ಲಿ ದೊಡ್ಡ ಹೆಸರನ್ನು ಗಳಿಸಿದ. ಆತ ತನ್ನ ನಲವತ್ತನೇ ವಯಸ್ಸಿನಲ್ಲಿ ಒಬ್ಬ ಅಮೇರಿಕನ್ ಸ್ತ್ರೀಯಲ್ಲಿ ಅನುರಕ್ತನಾಗಿ ಅವಳನ್ನು ಮದುವೆಯಾದ. ಅದರಿಂದ ದೊಡ್ಡ ಆಂದೋಲನವೆದ್ದಿತು. ಅವನು ತನ್ನ ಪಾದ್ರಿ ಕೆಲಸಕ್ಕೆ ರಾಜಿನಾಮೆ ಕೊಟ್ಟು ಗೃಹಸ್ಥನಾಗಿ ಲಾಯ್​ಸನ್ ಆದ. ಆತನ ಹೆಂಡತಿ ಎಲ್ಲ ಟೀಕೆಗೂ ಪಾತ್ರಳಾದಳು. ಫ್ರಾನ್ಸ್ ದೇಶದ ಸ್ತ್ರೀಯರು ಪಾದ್ರಿಗಳು ಮದುವೆ ಆಗುವುದನ್ನು ಒಪ್ಪುವುದಿಲ್ಲ. ಒಂದು ದಿನ ಲಾಯ್​ಸನ್ ಹೆಂಡತಿ, ಮದುವೆಯಾಗದೆ ಮತ್ತೊಬ್ಬನೊಂದಿಗೆ ವಾಸಿಸುತ್ತಿದ್ದ ಯಾರೋ ಒಬ್ಬಳು ನಟಿಗೆ ಹಾಗೆ ಇರುವುದು ಮಹಾಪಾಪ ಎಂದಳು. ಅದಕ್ಕೆ ಆ ನಟಿ ಇವಳಿಗೆ ಬುದ್ಧಿ ಹೇಳಿದಳು: “ನಾನು ನಿನಗಿಂತ ಸಾವಿರ ಪಾಲು ಮೇಲು. ನಾನು ಸಾಧಾರಣ ಪುರುಷನೊಂದಿಗೆ ವಾಸಿಸುತ್ತೇನೆ. ನಾನು ಅವನನ್ನು ಮದುವೆ ಆಗದೆ ಇರಬಹುದು. ಆದರೆ ನೀನು ಮಹಾಪಾಪಿ. ಅಂತಹ ಯತಿಗೆ ಭ್ರಷ್ಟತನ ತಂದಿರುವೆ, ನಿನಗೆ ಅವನ ಮೇಲೆ ಅಷ್ಟು ಪ್ರೀತಿಯಿದ್ದರೆ ಅವನ ಪರಿಚಾರಕಗಳಾಗಿರಬಹುದಾಗಿತ್ತು. ಅವನನ್ನು ಮದುವೆಯಾಗಿ ಭ್ರಷ್ಟನನ್ನಾಗಿ ಮಾಡಿ ಅವನನ್ನು ಒಬ್ಬ ಗೃಹಸ್ಥನನ್ನಾಗಿ ಮಾಡಿ ಏತಕ್ಕೆ ಹಾಳುಮಾಡಿದೆ?” 

 ಸ್ವಾಮೀಜಿ ಪೆರಿಯನ್ನು ಮೆಚ್ಚುತ್ತಿದ್ದರು. ಆತ ಭಕ್ತಿಸ್ವಭಾವದವನು, ನಡತೆಯಲ್ಲಿ ಮೃದುಮಧುರ, ಜೊತೆಯಲ್ಲಿದ್ದ ಮತ್ತೊಬ್ಬರು ಮ್ಯಾಕ್ಸಿಮಾಕ್ ಎಂಬ ಹೆಸರಿನವರು. ಅವರಿಗೆ ಅಮೇರಿಕಾ ದೇಶದಲ್ಲಿ ಸ್ವಾಮೀಜಿ ಪರಿಚಯವಾಗಿತ್ತು. ಎಲ್ಲರೂ ೨೪ನೇ ತಾರೀಖು ಓರಿಯಂಟಲ್ ಎಕ್ಸ್ ಪ್ರೆಸ್‍ನಲ್ಲಿ ಹೊರಟರು. ೨೫ನೇ ತಾರೀಖು ವಿಯನ್ನಾ ತಲುಪಿದರು. ಇಲ್ಲಿ ಮೂರು ದಿನಗಳು ತಂಗಿದ್ದರು. ಇಲ್ಲಿ ಶಾನ್‍ಬ್ರಸ್ ಅರಮನೆ ನೋಡಿದರು. ಈ ಅರಮನೆಯಲ್ಲಿಯೇ ನೆಪೋಲಿಯನ್ ಮಗನನ್ನು ಸೆರೆ ಇಟ್ಟಿದ್ದರು. ಆತ ಅಲ್ಲೇ ಕಾಲವಾಗಿ ಹೋಗಿದ್ದನು. ಇದನ್ನು Young Eagle ಎಂಬ ನಾಟಕ ಮಾಡಿ ಸಾರಾ ಬಾರ್‍ನಾರ್ಡ್ ಎಂಬ ಪ್ರಸಿದ್ಧ ನಟಿ ಪ್ರದರ್ಶಿಸಿದ್ದಳು. ಸ್ವಾಮೀಜಿ ಆ ನಾಟಕವನ್ನು ಇತ್ತೀಚೆಗೆ ನೋಡಿದ್ದರು. ಅವನನ್ನು ಸೆರೆಯಿಟ್ಟ ಅರಮನೆಯ ಸೊಬಗನ್ನು ನೋಡಿದರು. 

 ೨೮ನೇ ತಾರೀಖು ವಿಯನ್ನದಿಂದ ಕಾನ್‍ಸ್ಟೇಂಟಿನೋಪಲ್‍ಗೆ ಹೊರಟರು. ಹಂಗೇರಿ ಸರ್ಬಿಯ ರುಮೇನಿಯ ಬಲ್ಗೇರಿಯಾ ಮೂಲಕವಾಗಿ ಕಾನ್‍ಸ್ಟೇಂಟಿನೋಪಲ್‍ ತಲುಪಿದರು. ಮಾರನೆ ದಿನ ಸ್ವಾಮೀಜಿ ಮ್ಯಾಕ್ಲಿಯಾಡ್ ಒಡನೆ ಸ್ಕುಟಾರಿ ಎಂಬಲ್ಲಿಗೆ ಹೋದರು. ಅಲ್ಲಿ ಪೆರಿ ಹಯಾಸಿಂತ್ ಅವರನ್ನು ಬೀಳ್ಕೊಟ್ಟ ಮೇಲೆ ಪುನಃ ಕಾನ್‍ಸ್ಟೇಂಟಿನೋಪಲ್‍ನಲ್ಲಿ ಹಲವು ಸ್ಥಳಗಳನ್ನು ನೋಡಿಕೊಂಡು ಅಥೆನ್ಸ್ ಗೆ ಹೋದರು. ಅಲ್ಲಿಯ ಒಂದು ದ್ವೀಪದಲ್ಲಿ ಪಚ್ಚೈಯಪ್ಪ ಕಾಲೇಜಿನಲ್ಲಿ ಪ್ರಾಧ್ಯಾಪಕರಾಗಿದ್ದ ಲೆಪ್ಟರ್ ಎನ್ನುವವರನ್ನು ಕಂಡರು. ನಾಲ್ಕು ದಿನಗಳು ಅಥೆನ್ಸ್ ನೋಡಿಯಾದ ಮೇಲೆ ಜಾಲ್ ಎಂಬ ರಷ್ಯಾದ ಹಡಗಿನಲ್ಲಿ ಈಜಿಪ್ಟಿಗೆ ಬಂದರು. ಕೈರೋದಲ್ಲಿ ಮ್ಯೂಸಿಯಂ ನೋಡಿದರು. ಆ ಭೂಮಾಕಾರದ ಪಿರಮಿಡ್ಡುಗಳು, ಅಲ್ಲಿಯ ಸ್ಪಿನಿಸ್ ವಿಗ್ರಹಗಳು ಎಲ್ಲಾ ಮಾನವ ಜೀವನದ ನಶ್ವರತೆಯ ಚಿಹ್ನೆಯಂತೆ ಕಂಡವು. ಸ್ವಾಮೀಜಿ ಅಂತರ್ಮುಖರಾಗಿ ಧ್ಯಾನಮಗ್ನರಾಗತೊಡಗಿದರು. ಸ್ವಾಮೀಜಿ ಕೈರೋ ನಗರದಲ್ಲಿ ಹೋಗುತ್ತಿದ್ದಾಗ ಒಂದು ದಿನ ದಾರಿ ತಪ್ಪಿ ಜೊತೆಯವರಿಂದ ಬೇರೆಯಾಗಿ ವೇಶ್ಯೆಯರಿರುವ ಒಂದು ಗಲ್ಲಿಗೆ ಹೋದರು. ಅಲ್ಲಿ ಒಂದು ಮನೆಯ ಮುಂದೆ ಬೆಂಚಿನ ಮೇಲೆ ಅರ್ಧನಗ್ನರಾದ ಕೆಲವು ಸ್ತ್ರೀಯರು ಸ್ವಾಮೀಜಿಯವರನ್ನು ಕರೆಯತೊಡಗಿದರು. ಸ್ವಾಮೀಜಿ ಸ್ತ್ರೀಯರು ಕುಳಿತಿದ್ದ ಸ್ಥಳಕ್ಕೆ ಹೋದರು. ಅವರನ್ನು ಸ್ವಾಮೀಜಿ ನೋಡಿದೊಡನೆಯೇ ಅವರ ದುಃಸ್ಥಿತಿಗೆ ಇವರ ಹೃದಯ ಕರಗಿ ಕಣ್ಣಿನಲ್ಲಿ ನೀರು ಬಂದಿತು. ಕೆಲವು ಹೆಂಗಸರು\break ಸ್ವಾಮೀಜಿಯವರ ಬಟ್ಟೆಯನ್ನು ಕಣ್ಣಿಗೆ ಒತ್ತಿಕೊಂಡರು. ಸ್ಪೇನಿಷ್ ಭಾಷೆಯಲ್ಲಿ, ಇವರು ದೇವರಿಗೆ ಮೀಸಲಾದ ವ್ಯಕ್ತಿ ಎಂದು ಕರೆದರು. ಒಬ್ಬ ಹೆಂಗಸು ತನ್ನ ಮುಖವನ್ನು ತನ್ನ ಕೈಗಳಿಂದ ಮುಚ್ಚಿಕೊಂಡಳು, ಸ್ವಾಮೀಜಿಯವರ ಕಣ್ಣು ಇವಳ ಹೃದಯವನ್ನು ತೂರದೆ ಹೋಗಲಿ ಎಂದು. ಸ್ವಾಮೀಜಿ ಅವರಿಗೆ ಅನುಕಂಪವನ್ನು ತೋರಿ ಹಿಂತಿರುಗಿ ಬಂದಾದ ಮೇಲೆ “ಅಯ್ಯೋ, ಪಾಪ, ದೇಹವೇ ತಾವೆಂಬ ಭ್ರಮೆಯಲ್ಲಿ ಸಿಕ್ಕಿ ನರಳುತ್ತಿರುವವರು ಇವರು” ಎಂದು ಹೇಳಿದರು. 

 ಇಷ್ಟು ಹೊತ್ತಿಗೆ ಮಾಯಾವತಿ ಅದ್ವೈತ ಆಶ್ರಮಕ್ಕೆ ಕಾರಣಕರ್ತರಾದ ಸೇವಿಯರ್ಸ್‍‍ ಅವರು ಇಂಡಿಯಾ ದೇಶದಲ್ಲಿ ತೀರಿಹೋಗಿದ್ದರು. ಸ್ವಾಮೀಜಿ ಸಾಧ್ಯವಾದಷ್ಟು ಬೇಗ ಭರತಖಂಡಕ್ಕೆ ಬರಬೇಕೆಂದು ಮನಸ್ಸು ಮಾಡಿದರು. ಅದರಂತೆಯೇ ಸಿಕ್ಕಿದ ಮೊದಲ ಹಡಗನ್ನು ಹತ್ತಿ ಬೊಂಬಾಯಿಗೆ ಬಂದು ಅಲ್ಲಿಂದ ನೇರವಾಗಿ ಕಲ್ಕತ್ತೆಗೆ ಹೊರಟರು. ಡಿಸೆಂಬರ್ ೯ನೇ ತಾರೀಖು ರಾತ್ರಿ ಕಲ್ಕತ್ತೆಯನ್ನು ಸೇರಿದರು. ಅಲ್ಲಿಂದ ಬೇಲೂರು ಮಠಕ್ಕೆ ಹೊರಟರು. ಆ ಸಮಯದಲ್ಲಿ ಸಾಧುಗಳೆಲ್ಲ ಊಟಕ್ಕೆ ಕುಳಿತಿದ್ದರು. ಬಾಗಿಲು ಕಾಯುವವನು ಓಡಿಬಂದು ಒಬ್ಬ ಸಾಹೇಬರು ಬಂದು ಗೇಟಿನ ಬಳಿ ನಿಂತಿದ್ದಾರೆ ಎಂದು ಊಟ ಮಾಡುತ್ತಿದ್ದವರಿಗೆ ಹೇಳಿದನು. ಸಾಹೇಬರು ಯಾರು ಇರಬಹುದೆಂದು ನೋಡುವುದಕ್ಕೆ ಹೋದಾಗ ಅಷ್ಟು ಹೊತ್ತಿಗೆ ಆಗಲೇ ಸಾಹೇಬರು ಗೇಟನ್ನು ಹತ್ತಿ ಇಳಿದು ಮಠದ ಕಡೆ ಧಾವಿಸುತ್ತಿದ್ದರು. ಮಠದ ಸಾಧುಗಳು ನೋಡಿದರೆ ಆ ಸಾಹೇಬರು ಬೇರೆ ಯಾರೂ ಅಲ್ಲ; ತಮ್ಮ ಗುರುಭಾಯಿ ವಿವೇಕಾನಂದರೇ ಆಗಿದ್ದರು! ಸ್ವಾಮೀಜಿ “ಊಟದ ಗಂಟೆ ಕೇಳಿಸಿತು. ಆಮೇಲೆ ಬಂದರೆ ಊಟಕ್ಕೆ ಎಲ್ಲಿ ಸೊನ್ನೆ ಬೀಳುವುದೋ ಎಂದು ಗೇಟನ್ನ ಹಾರಿ ಬಂದೆ” ಎಂದು ನಗುತ್ತ ಹೇಳಿದರು. ಮಠದಲ್ಲೆಲ್ಲ ಸ್ವಾಮೀಜಿ ಬಂದರು ಎಂಬ ಸಂತೋಷ ವಾರ್ತೆ ಹಬ್ಬಿತು. ಎಲ್ಲರಿಗೂ ಆನಂದ. ಯಾರ ಬಾಯಲ್ಲಿ ಕೇಳಿದರೂ ಅವರ ಸುದ್ದಿಯೆ. ರಾತ್ರಿ ಊಟಕ್ಕೆ ಕಿಚಡಿ ಮಾಡಿದ್ದರು. ಸ್ವಾಮೀಜಿ ಅದನ್ನು ತಿನ್ನದೇ ಒಂದು ವರುಷದ ಮೇಲೆಯೇ ಆಗಿಹೋಗಿತ್ತು. ಮನದಣಿಯೆ ಅದನ್ನು ತಿಂದು ಅನಂತರ ರಾತ್ರಿ ಬಹಳ ಹೊತ್ತಿನ ತನಕ ಮಾತುಕತೆ ಆಡುತ್ತಿದ್ದರು. ಸ್ವಾಮೀಜಿ ಯೂರೋಪಿನಲ್ಲಿ ತಾವು ನೋಡಿದುದನ್ನೆಲ್ಲ ಹೇಳಿದರು. 

