
\chapter{ಹೈದರಾಬಾದಿನಲ್ಲಿ}

 ಸ್ವಾಮೀಜಿ ಹೈದರಾಬಾದಿಗೆ ಬರುವುದಕ್ಕೆ ಒಂದು ದಿನ ಮುಂಚಿತವಾಗಿ ಅಲ್ಲಿಯ ಪುರಜನರೆಲ್ಲ ಸೇರಿ ಸ್ವಾಮೀಜಿಗೆ ಸ್ವಾಗತವನ್ನು ಬಯಸುವ ಒಂದು ಕಮಿಟಿಯನ್ನು ಮಾಡಿದರು. ಮಾರನೆಯ ದಿನ ರೈಲ್ವೆ ನಿಲ್ದಾಣದಲ್ಲಿ ಸುಮಾರು ಐನೂರು ಜನ ನೆರೆದಿದ್ದರು. ಹೈದರಾಬಾದಿನ ನವಾಬರ ಪರವಾಗಿ ಸರ್ಕಾರದ ಮುಖ್ಯ ಅಧಿಕಾರಿಗಳು ಮತ್ತು ಸಾರ್ವಜನಿಕರಲ್ಲಿ ಪ್ರಮುಖಸ್ಥರು ಎಲ್ಲರೂ ಸ್ವಾಗತಿಸುವುದಕ್ಕೆ ಬಂದಿದ್ದರು. ಸ್ವಾಮೀಜಿಗೆ ಇದೊಂದು ಹೊಸ ಅನುಭವ. ಇದುವರೆಗೆ ಇಷ್ಟೊಂದು ಜನ ಇವರನ್ನು ಎದುರುಗೊಳ್ಳುವುದಕ್ಕೆ ಬಂದಿರಲಿಲ್ಲ. ಇದು ಮುಂದೆ ಏನಾಗಬಹುದು ಎಂಬುದರ ಮುನ್ಸೂಚನೆ. ರೈಲ್ವೆ ನಿಲ್ದಾಣದಲ್ಲಿ ಬಂದವರು ಸ್ವಾಮೀಜಿಗೆ ಹಾರ ತುರಾಯಿಗಳನ್ನರ್ಪಿಸಿದರು. ಅವರಿಗೆ ಅರ್ಪಿಸಿದ ಹೂವಿನ ಹಾರಗಳೇ ಒಂದು ರಾಶಿ ಆಯಿತು ಎಂದು ಅದನ್ನು ನೋಡಿದವರು ಹೇಳುವರು. 

 ಫೆಬ್ರವರಿ ಹನ್ನೊಂದನೇ ತಾರೀಖು ಸಿಕಂದರಾಬಾದಿನಿಂದ ಸುಮಾರು ನೂರು ಜನ ಹಿಂದೂಗಳು ಸ್ವಾಮೀಜಿಯವರನ್ನು ನೋಡುವುದಕ್ಕೆ ಬಂದರು. ಸ್ವಾಮೀಜಿಯವರನ್ನು ಮೆಹಬೂಬ್ ಕಾಲೇಜಿನಲ್ಲಿ ಒಂದು ಬಹಿರಂಗ ಉಪನ್ಯಾಸವನ್ನು ಕೊಡಲು ಕೇಳಿಕೊಂಡರು. ಸ್ವಾಮೀಜಿ ಹದಿಮೂರನೆ ತಾರೀಖು ಮಾತನಾಡಲು ಒಪ್ಪಿಕೊಂಡರು. ಅನಂತರ ಕಾಳಿಚರಣ ಬಾಬುಗಳೊಂದಿಗೆ ಇತಿಹಾಸ ಪ್ರಖ್ಯಾತವಾದ ಗೋಲ್ಕಂಡ ಕೋಟೆಯನ್ನು ನೋಡಲು ಹೋದರು. ಅಲ್ಲಿಂದ ಬರುವ ವೇಳೆಗೆ ಹೈದರಾಬಾದಿನ ನವಾಬರ ಆಪ್ತ ಕಾರ್ಯದರ್ಶಿಗಳಿಂದ ಸ್ವಾಮೀಜಿಯವರನ್ನು ಅರಮನೆಗೆ ಬರುವಂತೆ ಕೋರಿ ಒಂದು ಪತ್ರ ಬಂದಿತ್ತು. ಆಪ್ತ ಕಾರ್ಯದರ್ಶಿಯೇ ಸರ್ ಖುರ್‍ಶಿದ್ ಜಾ ಆಮೀರ್ ಇಕಬೀರ್ \enginline{K.C.S.I.}, ಅವರು ಹೈದರಾಬಾದ್ ನೈಜಾಮರ ಭಾವಂದಿರೂ ಆಗಿದ್ದರು. ಮಹಮ್ಮದೀಯರಾಗಿದ್ದರೂ ಉದಾರ ಹೃದಯದವರು, ಇಂಡಿಯಾ ದೇಶದ ಪ್ರಮುಖ ಹಿಂದೂ ದೇವಸ್ಥಾನಗಳನ್ನೆಲ್ಲ ಇವರು ನೋಡಿಕೊಂಡು ಬಂದಿದ್ದರು. ಸ್ವಾಮೀಜಿಯವರೊಡನೆ ಹಿಂದೂಧರ್ಮ, ಇಸ್ಲಾಂ ಮತ್ತು ಕ್ರೈಸ್ತ ಧರ್ಮಗಳನ್ನು ಕುರಿತು ಮಾತನಾಡಿದರು. ನವಾಬರು ಮಹಮ್ಮದೀಯರಾದುದರಿಂದ ವಿಗ್ರಹಾರಾಧನೆಯನ್ನು ಅಷ್ಟು ಒಪ್ಪಲಿಲ್ಲ. ಅದಕ್ಕೆ ಸ್ವಾಮೀಜಿ, ದೇವರನ್ನು ಚಿಂತನೆ ಮಾಡುವಾಗ ಮನಸ್ಸಿಗೆ ಯಾವುದಾದರೂ ಒಂದು ಆಕಾರದ ಸಹಾಯ ಮುಕ್ಕಾಲು ಪಾಲು ಜನರಿಗೆ ಅವಶ್ಯಕ ಎಂದರು. ಎಲ್ಲರೂ ಯಾವುದಾದರೂ ಒಂದು ಆಕಾರಕ್ಕೆ ಬದ್ಧರು. ಹೊರಗೆ ವಿಗ್ರಹವನ್ನು ದೂರಿದರೆ, ಅದರ ಬದಲು ಯಾವುದಾದರೂ ಕಟ್ಟಡವನ್ನೊ, ಕೈ, ಸೂರ‍್ಯ, ಶಿಲುಬೆ ಮುಂತಾದ ಸಂಕೇತವನ್ನೊ ತೆಗೆದುಕೊಳ್ಳುವರು. ಅವುಗಳನ್ನು ಒಪ್ಪಿಕೊಂಡು ತಾವು ಆಕಾರದಿಂದ ಪಾರಾದೆವು ಎಂದು ಭಾವಿಸುವರು. ತಾವು ಆಕಾರವನ್ನು ಕೇವಲ ಬದಲಾಯಿಸಿರುವೆವು, ಅದನ್ನು ಬಿಟ್ಟಿಲ್ಲ ಎಂದು ಅಂತಹವರಿಗೆ ಹೊಳೆಯುವುದೇ ಇಲ್ಲ. ಒಂದು ವೇಳೆ\break ಬಾಹ್ಯಾಕಾರವನ್ನು ಬಿಟ್ಟಿದ್ದರೂ, ಮನಸ್ಸಿನೊಳಗೆ ಅವರು ಮತ್ತಾವುದಾದರೂ ಆಕಾರವನ್ನು ತಾತ್ಕಾಲಿಕವಾಗಿ ಕೊಟ್ಟೇ ಚಿಂತನೆ ಮಾಡಬೇಕಾಗಿದೆ. ಆಕಾರವಿಲ್ಲದೆ ನಾವು ಚಿಂತನೆಯನ್ನು ಮಾಡುವುದಕ್ಕೇ ಆಗುವುದಿಲ್ಲ ಎಂದರು ಸ್ವಾಮೀಜಿ

 ಹಿಂದೂಧರ್ಮ ನಿಂತಿರುವುದು ವಿಗ್ರಹದ ಮೇಲಲ್ಲ, ಮಹಾತ್ಮರ ಮೇಲೂ ಅಲ್ಲ, ಆಧ್ಯಾತ್ಮಿಕ ನಿಯಮಗಳ ಮೇಲೆ. ಆ ನಿಯಮಗಳು ಎಲ್ಲಾ ಕಾಲದಲ್ಲಿಯೂ ಎಲ್ಲಾ ದೇಶದಲ್ಲಿಯೂ ಇವೆ. ಆ ನಿಯಮಗಳನ್ನು ಕಂಡುಹಿಡಿದವರನ್ನು ಮಂತ್ರ ದ್ರಷ್ಟರು ಎನ್ನುತ್ತಾರೆ ಹಿಂದೂಗಳು. ಆ ನಿಯಮ ಅವರು ಕಂಡುಹಿಡಿಯುವುದಕ್ಕೆ ಮುಂಚೆ ಇತ್ತು, ನಾಳೆ ಎಲ್ಲಾ ನಿರ್ನಾಮವಾಗಿ ಹೋದರೂ ಇರುವುದು. ಹಿಂದೂ ಧರ್ಮ ನಿಂತಿರುವುದು ವ್ಯಕ್ತಿಯ ಮೇಲಲ್ಲ, ತತ್ತ್ವದ ಮೇಲೆ. ಈ ತತ್ತ್ವ ಅಪೌರುಷೇಯವಾದುದು ಎಂದು ವಿವರಿಸಿದರು. ಎಲ್ಲಾ ಧರ್ಮಗಳೂ ಸತ್ಯವೇ, ಎಲ್ಲಾ ಸತ್ಯದೆಡೆಗೆ ಹೋಗಲು ಇರುವ ಭಿನ್ನ ಭಿನ್ನ ಮಾರ್ಗಗಳು. ಕೊನೆಗೆ ಎಲ್ಲಾ ಒಂದು ಕೇಂದ್ರದಲ್ಲಿಯೇ ಐಕ್ಯವಾಗುವುದು. ವೇದಾಂತದ ಪ್ರಕಾರ ಮಾನವನೇ ಈ ಸೃಷ್ಟಿಯಲ್ಲಿ ಸರ್ವೋತ್ಕೃಷ್ಟವಾದ ವಸ್ತು. ಏಕೆಂದರೆ, ಯಾವ ಯಾವ ಧರ್ಮ ತತ್ತ್ವ ವಿಜ್ಞಾನವನ್ನು ನೋಡುವೆವೊ ಅವೆಲ್ಲ ಮನುಷ್ಯನಿಂದ ಬಂದಿವೆ. ಮಾನವನ ಮನಸ್ಸು ಮಾತ್ರ ದೇಶಕಾಲ ನಿಮಿತ್ತಾತೀತವಾಗಿ ಪರಬ್ರಹ್ಮನಲ್ಲಿ ಒಂದಾಗಬಲ್ಲದು. ಮಾನವ ಜನ್ಮ ಬಹಳ ಪವಿತ್ರವಾದುದು, ಏಕೆಂದರೆ ಅವನು ಇಲ್ಲಿ ದೇವರಲ್ಲಿ ಒಂದಾಗಬಲ್ಲ. ಅವನು ಒಂದು ಪಾಪದ ಕಂತೆಯಲ್ಲ. ಭಗವಂತನ ಸೃಷ್ಟಿಯಲ್ಲಿರುವ ಶ್ರೇಷ್ಠವಸ್ತುವೇ ಮಾನವ. ಅನಂತರ ಸ್ವಾಮೀಜಿ, ಇಂತಹ ಭಾವನೆಗಳನ್ನೊಳಗೊಂಡ ಸನಾತನವಾದ ತತ್ತ್ವ ವನ್ನು ಬೋಧಿಸುವುದಕ್ಕೆ ಪಾಶ್ಚಾತ್ಯ ದೇಶಗಳಿಗೆ ಹೋಗುವ ಉದ್ದೇಶವಿದೆ ಎಂದು ಹೇಳಿದಾಗ ನವಾಬರು ಅದಕ್ಕಾಗಿ ತಕ್ಷಣವೇ ಒಂದು ಸಾವಿರ ರೂಪಾಯಿಗಳನ್ನು ಕೊಡಲು ಸಿದ್ಧವಾದರು. ಆದರೆ ಸ್ವಾಮೀಜಿ ಸದ್ಯಕ್ಕೆ ಬೇಡವೆಂದೂ, ಬೇಕಾದಾಗ ಕೆಳುತ್ತೇನೆಂದೂ ಹೇಳಿದರು.

 ಮಾರನೆ ದಿನ ಬೆಳಗ್ಗೆ ಸ್ವಾಮಿಯವರು ಹೈದರಾಬಾದಿನ ಪ್ರಧಾನ ಮಂತ್ರಿಗಳಾದ ಸರ್ ಅಸಮನ್ ಜಾ \enginline{K.C.S.I.}, ಸಂಸ್ಥಾನದ ಪೇಶ್ಕಾರರಾದ ಮಹಾರಾಜ ನರೇಂದ್ರ ಕೃಷ್ಣ ಬಹದ್ದೂರ್ ಮತ್ತು ಪ್ರಮುಖ ವ್ಯಕ್ತಿಗಳಾದ ಮಹಾರಾಜ ಶಿವರಾಜ ಬಹದ್ದೂರ್ ಇವರುಗಳೊಡನೆ ಮಾತುಕತೆಯಾಡಿದರು. ಎಲ್ಲರೂ ಸ್ವಾಮೀಜಿಯವರ ಪಾಶ್ಚಾತ್ಯ ಪ್ರವಾಸಕ್ಕೆ ಸಹಾಯಮಾಡಲು ಒಪ್ಪಿಕೊಂಡರು. ಅಂದಿನ ಸಾಯಂಕಾಲವೆ ಸ್ವಾಮೀಜಿಯವರು \enginline{“My Mission to the West”} ಎನ್ನುವುದರ ಮೇಲೆ ಉಪನ್ಯಾಸ ಮಾಡಿದರು. ಇದೇ ಸ್ವಾಮೀಜಿಯವರು ಇಂಡಿಯಾ ದೇಶದಲ್ಲಿ ಮಾಡಿದ ಪ್ರಥಮ ಬಹಿರಂಗ ಭಾಷಣ ಎಂದು ಬೇಕಾದರೂ ಹೇಳಬಹುದು. ಸುಮಾರು ಒಂದು ಸಾವಿರ ಜನ ನೆರೆದಿದ್ದರು. ಅನೇಕ ಐರೋಪ್ಯರೂ ಉಪನ್ಯಾಸಕ್ಕೆ ಬಂದಿದ್ದರು. ಸ್ವಾಮೀಜಿಯವರ ವಿದ್ವತ್ತು, ಅವರ ವಾಗ್‍ಧೋರಣೆ ಮತ್ತು ಅವರಿಗೆ ಇಂಗ್ಲೀಷ್ ಭಾಷೆಯ ಮೇಲಿರುವ ಸ್ವಾಧೀನ ಇವುಗಳನ್ನು ನೋಡಿ ಎಲ್ಲರೂ ಮಾರುಹೋದರು. ಸ್ವಾಮೀಜಿಯವರಿಗೆ ಉಪನ್ಯಾಸವಾದಮೇಲೆ ತಮಗೆ ಇಂಗ್ಲೀಷ್ ಭಾಷೆಯ ಮೇಲಿರುವ ಸ್ವಾಮಿತ್ವ ಮತ್ತು ಅದು ಜನರ ಮೇಲೆ ಯಾವ ಪರಿಣಾಮವನ್ನು ಉಂಟುಮಾಡಬಲ್ಲುದು ಎಂಬುದು ಅರಿವಾಯಿತು. ಬಹಿರಂಗ ಸಭೆಯಲ್ಲಿ ಅವರು ಇನ್ನೂ ಇದುವರೆಗೆ ಪ್ರಯೋಗಮಾಡಿ ನೋಡಿರಲಿಲ್ಲ. ಈಗ ಅವರಿಗೆ ಧೈರ್ಯ ಬಂತು.

 ಮಾರನೆ ದಿನ ಬೇಗಂ ಬಜಾರಿನ ಹಲವು ಜನ ಬ್ಯಾಂಕರ್‍ಗಳು, ಸೇಟ್ ಮೋತಿ‍ಲಾಲ್ ಅವರ ನೇತೃತ್ವದಲ್ಲಿ ಸ್ವಾಮೀಜಿಯವರನ್ನು ಕಂಡು ಅವರಿಗೆ ಪಾಶ್ಚಾತ್ಯ ದೇಶಗಳಿಗೆ ಹೋಗಲು ತಾವೆಲ್ಲ ಸಹಾಯ ಮಾಡುವುದಾಗಿ ಹೇಳಿದರು. 

 ಸ್ವಾಮೀಜಿಯವರು ಹೈದರಾಬಾದಿನಲ್ಲಿದ್ದಾಗ ಒಬ್ಬ ಪ್ರಮುಖ ಮಂತ್ರವಾದಿಯನ್ನು ನೋಡಲು ಹೋದರು. ಅನಂತರ ಸ್ವಾಮೀಜಿಯವರೇ ಅಮೇರಿಕಾ ದೇಶದಲ್ಲಿ ಮಾಡಿದ ಒಂದು ಉಪನ್ಯಾಸದಲ್ಲಿ ಅದನ್ನು ಹೀಗೆ ವಿವರಿಸುವರು: 

 “ನಾನೊಂದು ಸಲ ಹೈದರಾಬಾದಿನಲ್ಲಿದ್ದೆ. ಅಲ್ಲಿ ಒಬ್ಬ ಬ್ರಾಹ್ಮಣ ಏನನ್ನು ಬೇಕಾದರೂ ತೋರಿಸಬಲ್ಲ” ಎಂದು ಹೇಳಿದರು. ಆತ ಯಾವುದೋ ಒಂದು ವೃತ್ತಿಯಲ್ಲಿದ್ದ. ಅವನೊಬ್ಬ ಗೌರವಸ್ಥ. ಅವನಿಗೆ ಜ್ವರ ಬಂದಿತ್ತು. ಇಂಡಿಯಾ ದೇಶದಲ್ಲಿ ಒಬ್ಬ ಮಹಾತ್ಮ ರೋಗಿಯ ಮೇಲೆ ಕೈ ಇಟ್ಟರೆ ಅವನು ಜ್ವರದಿಂದ ಪಾರಾಗುವನು ಎಂಬ ನಂಬಿಕೆ ಇದೆ. ಆತ ನನ್ನನ್ನು ಅವನ ತಲೆಯ ಮೇಲೆ ಕೈ ಇಡಿ ಎಂದು ಕೇಳಿಕೊಂಡ. ನಾನು ಕೈ ಇಟ್ಟೆ. ಅನಂತರ ಅವನು ಮಾಯಾಮಂತ್ರವನ್ನು ತೋರಲು ಬಂದ. ಅವನ ಸೊಂಟದಮೇಲೆ ಒಂದು ಪಂಚೆ ಮಾತ್ರ ಇತ್ತು. ಉಳಿದವುಗಳನ್ನೆಲ್ಲ ನಾವು ತೆಗೆದುಹಾಕಿದೆವು. ಆಗ ಚಳಿಗಾಲವಾಗಿದ್ದುದರಿಂದ ನಾನು ಹೊದ್ದುಕೊಂಡಿದ್ದ ಶಾಲನ್ನೇ ಅವನಿಗೆ ಕೊಟ್ಟೆ. ಇಪ್ಪತ್ತೈದು ಜನರ ಕಣ್ಣುಗಳು ಅವನ ಕಡೆ ನೋಡುತ್ತಿದ್ದವು. ಆಗ ಆತ, ಈಗ ನಿಮಗೆ ಏನು ಬೇಕೋ ಅದನ್ನು ಬರೆಯಿರಿ ಎಂದ. ಆ ದೇಶದಲ್ಲಿ ಬೆಳೆಯದ ದ್ರಾಕ್ಷಿ, ಕಿತ್ತಲೆ ಮುಂತಾದ ಹಣ್ಣಿನ ಹೆಸರನ್ನು ಬರೆದು ಆ ಚೀಟಿಯನ್ನು ಅವನಿಗೆ ಕೊಟ್ಟೆವು. ಆಗ ಆತನ ಕಂಬಳಿಯೊಳಗಿನಿಂದ ದ್ರಾಕ್ಷಿಯ ಜೊಂಪೆಗಳು, ಕಿತ್ತಲೆಹಣ್ಣುಗಳು ಬರತೊಡಗಿದವು. ಅವು ಎಷ್ಟಿತ್ತೆಂದರೆ, ಅವುಗಳನ್ನೆಲ್ಲ ತೂಕಮಾಡಿದ್ದರೆ ಆ ಮನುಷ್ಯನಿಗೆ ಎರಡರಷ್ಟಾಗುವಂತಿತ್ತು. ಆತ ಆ ಹಣ್ಣನ್ನು ನಮಗೆ ತಿನ್ನುವಂತೆ ಹೇಳಿದ. ಆದರೆ ಅದು ಕೇವಲ ಕಣ್ಣುಕಟ್ಟಿನ ವಿದ್ಯೆಯ ಹಣ್ಣೆಂದು ನಾವು ತಿನ್ನಲು ಒಪ್ಪಲಿಲ್ಲ. ಆಗ ಆತನೇ ಹಣ್ಣನ್ನು ತಿನ್ನಲು ಮೊದಲು ಮಾಡಿದ. ಅನಂತರ ನಾವೂ ತಿಂದೆವು. ಅವು ಚೆನ್ನಾಗಿಯೇ ಇದ್ದುವು. 

 “ಅನಂತರ ಅವನು ಗುಲಾಬಿ ಹೂವಿನ ರಾಶಿಯನ್ನೇ ಉತ್ಪತ್ತಿಮಾಡಿದ. ಪ್ರತಿಯೊಂದು ಹೂವೂ ಹೊಸದಾಗಿತ್ತು. ಅದರ ಮೇಲಿನ್ನೂ ಹಿಮಮಣಿಗಳು ಇದ್ದವು. ಒಂದೂ ಸುಕ್ಕಿಹೋಗಿರಲಿಲ್ಲ, ಬಾಡಿ ಹೋಗಿರಲಿಲ್ಲ.” 

 ಇವುಗಳೆಲ್ಲ ಮಾನಸಿಕವಾದುವು. ಮನೋಶಕ್ತಿಯನ್ನು ರೂಢಿಸಿದರೆ ಅದ್ಭುತವಾಗಿರುವುದನ್ನು ಸಾಧಿಸಬಹುದೆಂಬುದನ್ನು ಇದು ತೋರುವುದು.

 ಫೆಬ್ರವರಿ ಹದಿನೇಳನೇ ತಾರೀಖು ಸ್ವಾಮೀಜಿ ಹೈದರಾಬಾದನ್ನು ಬಿಟ್ಟು ಪುನಃ ಮದ್ರಾಸಿಗೆ ಹೊರಟರು. ಸುಮಾರು ಒಂದು ಸಾವಿರ ಜನ ಅವರನ್ನು ಬೀಳ್ಕೊಡಲು ರೈಲ್ವೆ ನಿಲ್ದಾಣಕ್ಕೆ ಬಂದಿದ್ದರು. ಅವರ ಸಾಧುಜೀವನ, ಸರಳತೆ, ಪಾಂಡಿತ್ಯ ಎಲ್ಲರ ಮೇಲೂ ತಮ್ಮ ಪ್ರಭಾವವನ್ನು ಬೀರಿದ್ದವು. 

