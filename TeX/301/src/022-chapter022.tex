
\chapter{ವಿಶ್ವಧರ್ಮ ಸಮ್ಮೇಳನಕ್ಕೆ ಮುಂಚೆ}

 ಮೂರು ದಿನಗಳು ರೈಲಿನಲ್ಲಿ ಪ್ರಯಾಣ ಮಾಡಿ ಸಾಕಾಗಿ ಕೊನೆಗೆ ಚಿಕಾಗೊ ನಗರವನ್ನು ಸೇರಿದರು. ಅಲ್ಲಿ ಒಂದು ಹೋಟಲಿಗೆ ಹೋದರು. ಇದುವರೆಗೆ ಸ್ವಾಮೀಜಿಯವರ ಜೇಬಿನಲ್ಲಿದ್ದ ದುಡ್ಡು ನೀರಿನಂತೆ ವೆಚ್ಚವಾಗುತ್ತಿತ್ತು. ಕಂಡ ಕಂಡವರೆಲ್ಲ ಅವರಿಂದ ಹೆಚ್ಚು ದುಡ್ಡು ಕೇಳುತ್ತಿದ್ದರು. ಅಮೇರಿಕಾ ದೇಶೀಯರಿಗೆ ಇಂಡಿಯಾ ಎಂದರೆ ಇಲ್ಲಿಂದ ಹೋದ ರಾಜರ ಪರಿಚಯ ಮಾತ್ರ ಇತ್ತು. ಅವರ ಜೇಬಿನಲ್ಲಿ ಬೇಕಾದಷ್ಟು ದುಡ್ಡು ಇರುತ್ತದೆ. ಅವರು ಯಥೇಚ್ಛವಾಗಿ ಅದನ್ನು ಖರ್ಚು ಮಾಡುತ್ತಿರುವರು. ವಿವೇಕಾನಂದರ ಉಡಿಗೆತೊಡಿಗೆ ಒಬ್ಬ ಉತ್ತರ ಇಂಡಿಯಾದ ರಾಜನ ಉಡಿಗೆತೊಡಿಗೆಯಂತೆ ಇತ್ತು. ಕೆಲವರು ಅವರನ್ನು ರಾಜಾ ವಿವೇಕಾನಂದ ಎಂತಲೂ ಕರೆದರು. ಆದರೆ ರಾಜನ ದುಡ್ಡು ಇವರಲ್ಲಿ ಇರಲಿಲ್ಲ. ಸ್ವಾಮೀಜಿಯವರ ಹತ್ತಿರ ಪರಿಚಯ ಪತ್ರಗಳು ಯಾವುವೂ ಇರಲಿಲ್ಲ. ಯಾವ ಸ್ನೇಹಿತರೂ ಇರಲಿಲ್ಲ. 

 ಆ ಸಮಯದಲ್ಲಿ ಚಿಕಾಗೊ ನಗರದಲ್ಲಿ ವಿಶ್ವಮೇಳ (\enginline{World fair}) ನಡೆಯುತ್ತಿತ್ತು. ಅಮೇರಿಕಾ ದೇಶದವರು ತಾವು ಸ್ವಾತಂತ್ರ್ಯವನ್ನು ಪಡೆದಂದಿನಿಂದ\break ಏನೇನನ್ನು ಸಾಧಿಸಿರುವರು ಎಂಬುದನ್ನು ಎಲ್ಲರಿಗೂ ತೋರಿಸುವುದಕ್ಕಾಗಿ ದೊಡ್ಡ ವಸ್ತುಪ್ರದರ್ಶನವನ್ನು ನಡೆಸುತ್ತಿದ್ದರು. ಅದನ್ನು ನೋಡಬೇಕಾದರೆ ಎಂಟು ಹತ್ತು ದಿನಗಳು ಹಿಡಿಯುತ್ತಿತ್ತು. ಸ್ವಾಮೀಜಿ ಮೊದಲು ಅದನ್ನು ನೋಡುವುದಕ್ಕೆ ಹೋದರು. ಕೆಲವು ದಿನಗಳು ಹೋಗಿ ನೋಡುತ್ತಿದ್ದಾಗ ಒಂದು ದಿನ ಇಂಡಿಯಾ ದೇಶದ ರಾಜರೊಬ್ಬರು ಸ್ವಾಮೀಜಿ ಕಣ್ಣಿಗೆ ಅಲ್ಲಿ ಕಂಡರು. ಇಷ್ಟು ದೂರ ದೇಶದಲ್ಲಿ ಒಬ್ಬ ಭಾರತೀಯನು ಸಿಕ್ಕಿದನಲ್ಲ ಎಂದು ಮಾತನಾಡುವುದಕ್ಕೆ ಆತನ ಹತ್ತಿರ ಹೋದರು. ಆತ ಸ್ವಾಮೀಜಿಗಳು ಯಾರೋ ಬೈರಾಗಿ ಸಂನ್ಯಾಸಿ ಇರಬಹುದೆಂದು ಅವರೊಡನೆ ಮಾತನಾಡಲು ಇಚ್ಛೆಪಡಲಿಲ್ಲ. ತನ್ನ ಗೌರವ ಎಲ್ಲಿ ಕಡಿಮೆಯಾಗುವುದೋ ಎಂದು ಆತನ ಅಂಜಿಕೆ. ಪಂಚೆ ಉಟ್ಟುಕೊಂಡು ಖಾಯಾಲಿಯಂತೆ ತೋರುವ ಒಬ್ಬ ಮಹಾರಾಷ್ಟ್ರದ ಬ್ರಾಹ್ಮಣನು ಉಗುರಿನಿಂದ ಬರೆದ ಚಿತ್ರವನ್ನು ಅಲ್ಲಿ ಮಾರುತ್ತಿದ್ದ. ಆ ವ್ಯಕ್ತಿಯು ಪತ್ರಿಕೆಯ ಸುದ್ದಿಕಾರರೊಡನೆ ರಾಜನ ವಿಷಯದಲ್ಲಿ ಅವನಿಗೆ ವಿರೋಧವಾಗಿ ಏನೇನೋ ಹೇಳಿದನು. ಇಂತಹ ರಾಜರು ನಿಜವಾಗಿ ರಾಜರಲ್ಲ, ಅವರೆಲ್ಲ ಸಾಮಾನ್ಯವಾಗಿ ವ್ಯಭಿಚಾರಿಗಳು ಎಂದು ಮುಂತಾಗಿ. ಅಮೇರಿಕಾ ದೇಶದ ಪತ್ರಿಕೆಯ ಸಂಪಾದಕರು ಅದಕ್ಕೆ ಬೆಲೆಕೊಟ್ಟು ಮರುದಿನ ತಮ್ಮ ಪತ್ರಿಕೆಯಲ್ಲಿ, ಇಂಡಿಯಾ ದೇಶದಿಂದ ಬಂದ ಜ್ಞಾನಿಯೋರ್ವನನ್ನು ಎಂದರೆ ಸ್ವಾಮಿಗಳನ್ನು ಅತಿಯಾಗಿ ಹೊಗಳಿ, ರಾಜನಿಗೆ ವಿರೋಧವಾಗಿ ಸ್ವಾಮೀಜಿ ಕನಸಿನಲ್ಲಿಯೂ ಕೂಡ ಹೇಳದ ಮಾತುಗಳನ್ನು ವೃತ್ತಪತ್ರಿಕೆಯಲ್ಲಿ ಪ್ರಕಟಿಸಿದರು. ಇದರಿಂದ ರಾಜನ ಮುಖಕ್ಕೆ ಮಸಿ ಬಳಿದಂತೆ ಆಗಿ ಬೇಗ ರಾಜನನ್ನು ಚಿಕಾಗೊ ಸಮಾಜ ತಿರಸ್ಕರಿಸಿತು. 

 ಕೆಲವು ದಿನಗಳಾದ ಮೇಲೆ ಸ್ವಾಮೀಜಿ ವಿಶ್ವಧರ್ಮ ಸಮ್ಮೇಳನದ ಕಛೇರಿಗೆ ಹೋಗಿ, ಅದು ಯಾವಾಗ ನಡೆಯುತ್ತದೆ, ಅದರಲ್ಲಿ ಮಾತನಾಡಬಯಸುವವರಿಗೆ ಹೇಗೆ ಅವಕಾಶ ಕೊಡುತ್ತಾರೆ ಎಂಬುದನ್ನು ವಿಚಾರಿಸಿದರು. ಅದು ಸೆಪ್ಟೆಂಬರ್ ಮೊದಲನೆಯ ಅಥವಾ ಎರಡನೇ ವಾರದಲ್ಲಿ ನಡೆಯುವುದೆಂದು ಹೇಳಿದರು. ಆಗ ಇನ್ನೂ ಜುಲೈ ತಿಂಗಳ ಮಧ್ಯಭಾಗ. ಸುಮಾರು ಎರಡು ತಿಂಗಳು ಹಣ ನೀರಿನಂತೆ ವೆಚ್ಚವಾಗುವ ಚಿಕಾಗೊ ನಗರದಲ್ಲಿ ಇರಬೇಕಾಯಿತು. ಅವರ ಹತ್ತಿರ ಇದ್ದ ದುಡ್ಡು ಆಗಲೇ ಕರಗುತ್ತ ಬಂದಿತ್ತು. ಸಂನ್ಯಾಸಿ ಇಂಡಿಯಾದೇಶದಲ್ಲಿ ಭಿಕ್ಷೆ ಬೇಡಬಹುದು. ಆದರೆ ಅಮೇರಿಕಾ ದೇಶದಲ್ಲಿ ಅದು ಕಾನೂನು ವಿರೋಧ. ಅನಂತರ ವಿಶ್ವಧರ್ಮ ಸಮ್ಮೇಳನಕ್ಕೆ ಸದಸ್ಯರಾಗಬೇಕಾದರೆ ಅವರು ಯಾವ ದೇಶದಿಂದ ಬರುವರೋ ಅಲ್ಲಿಂದ ಯಾವುದಾದರೂ ಒಂದು ಸಂಸ್ಥೆಯ ಮೂಲಕ ಅರ್ಹತಾಪತ್ರದೊಡನೆ ಬಂದಿರಬೇಕು. ಸ್ವಾಮೀಜಿಯವರ ಹತ್ತಿರ ಅಂತಹ ಪತ್ರವಾವುದೂ ಇರಲಿಲ್ಲ. ಅವರು ಮದ್ರಾಸಿನಲ್ಲಿದ್ದಾಗ ಥಿಯಾಸಫಿ ಸಂಸ್ಥೆಗೆ ಹೋಗಿ ಅವರನ್ನು ಪರಿಚಯ ಪತ್ರವೊಂದನ್ನು ಕೊಡಲು ಸಾಧ್ಯವೆ ಎಂದು ಕೇಳಿದಾಗ, ಅವರು ನಮ್ಮ ಸಮಾಜಕ್ಕೆ ಸೇರಿದರೆ ಕೊಡುತ್ತೇವೆ ಎಂದರು. ಸ್ವಾಮೀಜಿ “ನಿಮ್ಮ ಸಮಾಜದ ಹಲವು ನಂಬಿಕೆಗಳನ್ನು ನಾನು ಒಪ್ಪುವುದಿಲ್ಲ. ಆದಕಾರಣ ನಾನು ಹೇಗೆ ಸೇರಲಿ?” ಎಂದರು. ಹಾಗಾದರೆ ನಾವು ಪರಿಚಯ ಪತ್ರವನ್ನು ಕೊಡುವುದಕ್ಕೆ ಆಗುವುದಿಲ್ಲ ಎಂದು ಹೇಳಿದರು. ಯಾವ ಕನಸನ್ನು ಕಟ್ಟಿಕೊಂಡು ಭರತಖಂಡದಿಂದ ಇಲ್ಲಿಗೆ ಬಂದಿದ್ದರೊ, ಅದೊಂದು ನಿಜವಾದ ಕನಸೇ ಆಗಿ ತಮ್ಮ ಕಣ್ಣೆದುರಿಗೇ ಅದು ಮಾಯವಾಯಿತು. ಇಂಡಿಯಾ ದೇಶವನ್ನು ಬಿಡುವಾಗ ಶಿಷ್ಯರು ಇದನ್ನು ಯೋಚಿಸಿಯೇ ಇರಲಿಲ್ಲ. ಸ್ವಾಮೀಜಿಯವ ವ್ಯಕ್ತಿತ್ವವೇ ಅವರಿಗೊಂದು ಪರಿಚಯ ಪತ್ರ. ಅದಕ್ಕಿಂತ ಹೆಚ್ಚು ಏನು ಬೇಕು! ಅವರು ಮಾತನಾಡಲು ಯಾರು ಅವಕಾಶ ಕೊಡುವುದಿಲ್ಲ? – ಎಂದು ಭಾವಿಸಿದ್ದರು. ಸ್ವಾಮೀಜಿಯವರು ಕೂಡ ವ್ಯವಹಾರಕ್ಕೆ ಅಪರಿಚಿತರು. ಅವರು ಇದನ್ನು ಮುಂಚೆ ಆಲೋಚಿಸಿಯೇ ಇರಲಿಲ್ಲ. ಶೀಘ್ರದಲ್ಲಿ ಅವರು ಮತ್ತೊಮ್ಮೆ ದಾರಿದ್ರ್ಯದ ಸ್ಥಿತಿಯನ್ನು ಎದುರಿಸಬೇಕಾಯಿತು. ಆ ಸಮಯದಲ್ಲಿ ಮದ್ರಾಸಿನ ಶಿಷ್ಯನಿಗೆ ಹೀಗೆ ಕಾಗದ ಬರೆಯುತ್ತಾರೆ: 

 “ನಾನು ಇಂಡಿಯಾದೇಶದಿಂದ ಹೊರಡುವುದಕ್ಕೆ ಮುಂಚೆ ಕಟ್ಟಿದ ಕನಸುಗಳೆಲ್ಲ ಈಗ ಮಾಯವಾಗಿವೆ. ನನ್ನನ್ನು ಎದುರಿಸುವ ಯಥಾರ್ಥ ಸ್ಥಿತಿಯೊಂದಿಗೆ ಈಗ ನಾನು ಹೋರಾಡಬೇಕಾಗಿದೆ. ಈ ದೇಶವನ್ನು ಬಿಟ್ಟು ಇಂಡಿಯಾದೇಶಕ್ಕೆ ಹಿಂತಿರುಗಬೇಕೆಂದು ಮೂರು ಸಲ ಮನಸ್ಸು ಮಾಡಿದೆನು. ಆದರೆ ಈಗ ಹೊಗುವುದಿಲ್ಲವೆಂದು ಶಪಥ ಮಾಡಿರುವೆನು. ದೇವರೇ ನನಗೆ ಅಪ್ಪಣೆ ಮಾಡಿರುವನು, ನನಗೆ ದಾರಿಕಾಣದೆ ಇರಬಹುದು, ಆದರೆ ಆತನ ಕಣ್ಣಿಗೆ ಎಲ್ಲವೂ ಕಾಣುತ್ತದೆ. ಸಾಯಲಿ ಬದುಕಲಿ ಹಿಡಿದುದನ್ನು ಬಿಡುವುದಿಲ್ಲ.” ಎಂತಹ ಕೆಚ್ಚದೆ ಸಾಧಾರಣ ಜೀವಿಯನ್ನು ನುಚ್ಚುನೂರು ಮಾಡುವಂತಹ ಪರಿಸ್ಥಿತಿಯಲ್ಲಿಯೂ! 

 ಸ್ವಾಮೀಜಿ ಇಂತಹ ಪರಿಸ್ಥಿತಿಯಲ್ಲಿದ್ದಾಗ ಬಹಳ ಕಡಿಮೆ ಖರ್ಚಿನಿಂದ ವಿಶ್ವ ಧರ್ಮಸಮ್ಮೇಳನ ನಡೆಯುವವರೆಗೆ ಇರಬೇಕಾದರೆ ಯಾವ ಊರು ಅನುಕೂಲ ಎಂದು ವಿಚಾರ ಮಾಡುತ್ತಿದ್ದಾಗ, ಯಾರೋ ಬಾಸ್ಟನ್ ಅದಕ್ಕೆ ಸರಿಯಾದ ಊರು ಎಂದು ಹೇಳಿದರು. ಸ್ವಾಮೀಜಿ ಬಾಸ್ಟನ್ನಿಗೆ ರೈಲಿನಲ್ಲಿ ಹೊರಟರು. ಅವರೊಂದಿಗೆ ಅದೇ ಗಾಡಿಯಲ್ಲಿ ಮಿಸ್ ಕಾಟೀಸ್ಯಾನ್ ಬಾರನ್ ಎಂಬ ಮಹಿಳೆ ಬಾಸ್ಟನ್ನಿಗೆ ಹೋಗುತ್ತಿದ್ದಳು. ಆಕೆ ಸ್ವಾಮೀಜಿಯವರನ್ನು ಹಾಗೆಯೇ ದಿಟ್ಟಿಸಿ ನೋಡಿದಳು. ಸ್ವಾಮೀಜಿಯವರ ವಿಚಿತ್ರವಾದ ಪೋಷಾಕು, ಅವರ ಹೊಳೆಯುತ್ತಿರುವ ಕಣ್ಣುಗಳು, ಅಂತರ್ಮುಖತೆ ಇವುಗಳೆಲ್ಲ ಆಕೆಯನ್ನು ಆಕರ್ಷಿಸಿತು. ಆಕೆ ಸ್ವಾಮೀಜಿ ಹತ್ತಿರ ಹೋಗಿ ಮಾತನಾಡಿದಳು. ಸ್ವಾಮೀಜಿ ತಾವು ಇಂಡಿಯಾದೇಶದಿಂದ ವಿಶ್ವಧರ್ಮ ಸಮ್ಮೇಳನದಲ್ಲಿ ಮಾತನಾಡಲು ಬಂದ ಸಂನ್ಯಾಸಿಗಳು, ಆದರೆ ಅಲ್ಲಿ ಮಾತನಾಡಲು ಅವಕಾಶ ಸಿಕ್ಕಲಿಲ್ಲ ಎಂದು ಹೇಳಿದರು. ಆಗ ಆಕೆ ಸ್ವಾಮೀಜಿಯವರಿಗೆ ತಮ್ಮ ಮನೆಯಲ್ಲಿ ಬಂದು ಇರಿ, ಬಹುಶಃ ನಿಮಗೆ ಅಲ್ಲಿ ಹೇಗಾದರೂ ಸಹಾಯ ಸಿಗಬಹುದು ಎಂದು ಹೇಳಿದಳು. ಸ್ವಾಮೀಜಿ ತಕ್ಷಣವೇ ಅಲ್ಲಿ ಹೋಗಿ ಇರುವುದಕ್ಕೆ ಒಪ್ಪಿಕೊಂಡರು. ಇದರಿಂದ ಒಂದು ದಿನಕ್ಕೆ ಒಂದು ಪೌಂಡು ಉಳಿತಾಯವಾಗುವುದು. ಆಕೆಗೂ ಒಂದು ವಿಧವಾದ ಲಾಭ. ಅದೇನೆಂದರೆ ತನ್ನ ಸ್ನೇಹಿತರಿಗೆಲ್ಲ ಇಂಡಿಯಾದೇಶದಿಂದ ಬಂದಿರುವ ವಿಚಿತ್ರ ವ್ಯಕ್ತಿಯನ್ನು ಪ್ರದರ್ಶಿಸುವುದು. ಸ್ವಾಮೀಜಿ ಬೀದಿಯಲ್ಲಿ ತಮ್ಮ ಪೋಷಾಕಿನಲ್ಲಿ ಹೋಗುತ್ತಿದ್ದರೆ ಹುಡುಗರು ಇದೊಂದು ವೇಷವೆಂದು ಭಾವಿಸಿದರು. ಇದಾವುದೋ ಒಂದು ವಿಚಿತ್ರ ಮೃಗ ಎಂದು ಭಾವಿಸಿ, ಅವರ ಬಟ್ಟೆಯನ್ನು ಹಿಡಿದು ಎಳೆಯುತ್ತಿದ್ದರು. ಸ್ವಾಮೀಜಿ ಆಗ ಇಂಗ್ಲೀಷಿನಲ್ಲಿ “ಏತಕ್ಕೆ ಹೀಗೆ ಮಾತನಾಡುತ್ತೀರಿ?” ಎಂದು ಕೇಳಿದಾಗ ಆ ಹುಡುಗರು, ಈ ಪ್ರಾಣಿ ಚೆನ್ನಾಗಿರುವ ಇಂಗ್ಲೀಷಿನಲ್ಲಿಯೂ ಮಾತನಾಡುತ್ತದೆ ಎಂದು ಆಶ್ಚರ್ಯಪಟ್ಟರು. ಮನೆಯಾಕೆ ತನ್ನ ಸ್ನೇಹಿತರಿಗೆಲ್ಲ ಸ್ವಾಮೀಜಿಯ ಪರಿಚಯ ಮಾಡಿಸಿದಳು. ಸ್ವಾಮೀಜಿಗೆ ಕ್ರಮೇಣ ಕೆಲವು ವ್ಯಕ್ತಿಗಳು ಪರಿಚಯವಾಗುತ್ತ ಬಂದರು. 

ಒಂದು ದಿನ ಸ್ವಾಮೀಜಿಯವರನ್ನು ಅಲ್ಲಿರುವ (\enginline{Reformatory}) ದೋಷ ನಿವೃತ್ತಿ ಆಲಯಕ್ಕೆ ಕರೆದುಕೊಂಡು ಹೋದರು. ಆ ವಿಷಯವನ್ನು ಕುರಿತು ಸ್ವಾಮೀಜಿ ಹೀಗೆ ಹೇಳುತ್ತಾರೆ: “ಇಲ್ಲಿ ಅದನ್ನು ಸೆರೆಮನೆ ಎಂದು ಕರೆಯುವುದಿಲ್ಲ, ದೋಷ ನಿವೃತ್ತಿ ಆಲಯ ಎನ್ನುತ್ತಾರೆ. ನಾನು ಅಮೇರಿಕಾ ದೇಶದಲ್ಲಿ ನೋಡಿದುದರಲ್ಲೆಲ್ಲ ಅತಿ ಮುಖ್ಯವಾದುದು ಇದು. ಪ್ರೀತಿಯಿಂದ ಕೈದಿಗಳನ್ನು ನೋಡಿಕೊಳ್ಳುತ್ತಾರೆ. ಅವರ ಶೀಲವನ್ನು ತಿದ್ದುತ್ತಾರೆ. ಅವರನ್ನು ಸಮಾಜಕ್ಕೆ ಉಪಯುಕ್ತರಾದ ಜೀವಿಗಳನ್ನಾಗಿ ಮಾಡಿ ಪುನಃ ಕಳುಹಿಸಿಕೊಡುತ್ತಾರೆ. ಇದು ಎಷ್ಟು ಚಮತ್ಕಾರವಾಗಿದೆ! ಎಷ್ಟು ಸುಂದರವಾಗಿದೆ! ನೋಡಿದಲ್ಲದೆ ಇದನ್ನು ನಂಬಲು ಆಗುವುದಿಲ್ಲ. ಭರತಖಂಡದಲ್ಲಿ ದರಿದ್ರರಿಗೆ ಪಾಪಿಗಳಿಗೆ ಸ್ನೇಹಿತರಿಲ್ಲ. ಅವರಿಗೆ ಸಹಾಯಕರಿಲ್ಲ. ತಾವು ಎಷ್ಟು ಪ್ರಯತ್ನಪಟ್ಟರೂ ಮೇಲೇರಲಾರರು. ಪ್ರತಿದಿನವೂ ಅವರು ಅಧಃಪತನಕ್ಕೆ ಹೋಗುತ್ತಿರುವರು. ಕ್ರೂರ ಸಮಾಜದಿಂದ ಬೀಳುವ ಪೆಟ್ಟಿನ ಸುರಿಮಳೆಯನ್ನು ಅವರು ಅನುಭವಿಸುವರು. ಯಾವ ಕಡೆಯಿಂದ ಪೆಟ್ಟು ಬೀಳುತ್ತದೆ ಎಂಬುದು ಅವರಿಗೆ ಗೊತ್ತಿಲ್ಲ. ತಾವು ಮಾನವ ವರ್ಗಕ್ಕೆ ಸೇರಿದವರು ಎಂಬುದನ್ನು ಅವರು ಮರೆತಿರುವರು. ಇದರ ಪರಿಣಾಮವೇ ಗುಲಾಮಗಿರಿ. ಕೆಲವು ವರ್ಷಗಳಿಂದ ಆಲೋಚನಾಪರರು ಇದನ್ನು ನೋಡಿರುವರು. ಆದರೆ ದುರದೃಷ್ಟವಶದಿಂದ ತಪ್ಪೆಲ್ಲವನ್ನೂ ಹಿಂದೂಧರ್ಮದ ಮೇಲೆ ಹೊರಿಸಿರುವರು. ಅವರಿಗೆ ಸ್ಥಿತಿಯನ್ನು ಉತ್ತಮಪಡಿಸಬೇಕಾದರೆ ತೋರುವುದು ಒಂದೇ ಮಾರ್ಗ. ಅದೇ ಜಗತ್ತಿನಲ್ಲಿರುವ ಶ್ರೇಷ್ಠವಾದ ಧರ್ಮವನ್ನು ನಾಶಮಾಡುವುದರಿಂದ. ನನ್ನ ಪ್ರಿಯ ಸ್ನೇಹಿತನೇ ಕೇಳು, ದೇವರ ದಯೆಯಿಂದ ನನಗೆ ಇದರ ಗುಟ್ಟು ಗೊತ್ತಾಗಿದೆ. ತಪ್ಪು ಧರ್ಮದಲ್ಲಿಲ್ಲ. ನೀನೋರ್ವನೇ ಅನೇಕವಾಗಿ ಸರ್ವಭೂತಗಳಲ್ಲಿಯೂ ನೀನಿರುವೆ ಎಂದು ನಿಮ್ಮ ಧರ್ಮ ಸಾರುವುದು. ಆದರೆ ಇದನ್ನು ಅನುಷ್ಠಾನಕ್ಕೆ ತರಲಿಲ್ಲ. ಜನರಿಗೆ ಸಹಾನುಭೂತಿಯನ್ನು ತೋರಲಿಲ್ಲ, ದಯೆಯನ್ನು ಬೀರಲಿಲ್ಲ. ಇದು ನಮ್ಮ ತಪ್ಪು. ಮಾನವರ ಉತ್ಕೃಷ್ಟತೆಯನ್ನು ಮಹಿಮೆಯನ್ನು ಹಿಂದೂಧರ್ಮದಷ್ಟು ಉಚ್ಚರೀತಿಯಲ್ಲಿ ಮತ್ತಾವ ಧರ್ಮವೂ ಹೇಳುವುದಿಲ್ಲ. ಆದರೆ ಹಿಂದೂಧರ್ಮದಲ್ಲಿ ದರಿದ್ರರನ್ನು ದೀನರನ್ನು ತುಳಿಯುವಂತೆ ಮತ್ತಾವ ಧರ್ಮದಲ್ಲಿಯೂ ಇಲ್ಲ. ಇದು ಧರ್ಮದ ತಪ್ಪಲ್ಲವೆಂದು ನನಗೆ ದೇವರು ತೋರಿರುವನು. ಹಿಂದೂಧರ್ಮದಲ್ಲಿ ವ್ಯಾವಹಾರಿಕ, ಪಾರಮಾರ್ಥಿಕ ಎಂಬ ನಾನಾ ಹಿಂಸೆಗಳನ್ನು ಕೊಡುವ ಸಿದ್ಧಾಂತಗಳನ್ನು ಪ್ರಚಾರಕ್ಕೆ ತಂದ ಆಷಾಢಭೂತಿಗಳೇ ಅಪರಾಧಿಗಳು.” 

 ಸ್ವಾಮೀಜಿ ರಮಾಬಾಯಿಗೆ ಸಹಾಯ ಮಾಡುವ ಒಂದು ಸ್ತ್ರೀ ಸಮಾಜದಲ್ಲಿ ಮಾತನಾಡಲು ಹೋದರು. ಪಂಡಿತ ರಮಾಬಾಯಿ ಮಹಾರಾಷ್ಟ್ರ ಮಹಿಳೆ. ಮೊದಲು ಹಿಂದೂ ವಿಧವೆಯಾಗಿದ್ದು ಅನಂತರ ಕ್ರೈಸ್ತ ಧರ್ಮಕ್ಕೆ ಸೇರಿ, ಅಮೇರಿಕಾ ದೇಶಕ್ಕೆ ಹೊರಟಳು. ಅಲ್ಲಿ ಭಾರತೀಯರ ವಿಧವೆಯರ ಕೆಲಸಕ್ಕೆ ಎಂದು ಹಣವನ್ನು ಸಂಗ್ರಹಿಸು\-ತ್ತಿದ್ದಳು. ಆಕೆ ಅಮೇರಿಕಾ ದೇಶದ ಜನರ ಮನಸ್ಸನ್ನು ಕರಗಿಸುವುದಕ್ಕಾಗಿ ಹಿಂದೂಗಳು ಸ್ತ್ರೀಯರಿಗೆ ಕೊಡುವ ಸಂಕಟಗಳನ್ನು ಇಲ್ಲದ ಸಲ್ಲದ ರೀತಿಯಲ್ಲಿ ಅಪಪ್ರಚಾರ ಮಾಡುತ್ತಿದ್ದಳು. ಅಲ್ಲಿಯ ಪಾದ್ರಿಗಳಿಗೆ ಆಕೆ ತುಂಬಾ ಸಹಾಯಕ್ಕೆ ಬಂದಳು. ಆಕೆಯ ಮೂಲಕ ಹಿಂದೂಧರ್ಮ ಮತ್ತು ಹಿಂದೂ ಜನರನ್ನು ಅವಹೇಳನ ಮಾಡಲು ಅವಕಾಶ ಸಿಕ್ಕಿತು. ಆಗಿನ ಕಾಲದಲ್ಲಿ ಅಮೇರಿಕಾ ದೇಶದಲ್ಲಿ ಪ್ರಚಲಿತವಾದ ಹಲವು ತಪ್ಪು ಅಭಿಪ್ರಾಯಗಳಿಗೆ ಆಕೆ ಕಾರಣವಾದಳು. ಸ್ವಾಮೀಜಿ ಆಕೆ ಭರತಖಂಡಕ್ಕೆ ಸೇರಿದವಳಾದುದರಿಂದ ಆ ಸಮಾಜದಲ್ಲಿ ಹೋಗಿ ಮಾತನಾಡಿದರು. ನಿಜವಾದ ಪರಿಸ್ಥಿತಿಯನ್ನು ಜನರ ಮುಂದೆ ಇಟ್ಟರು. ಇಲ್ಲಿ ಸ್ವಾಮಿಗಳಿಗೆ ಕೆಲವರ ಪರಿಚಯವಾಯಿತು. ಸುತ್ತಮುತ್ತಲಿದ್ದ ಜನರು ಸ್ವಾಮೀಜಿಯವರನ್ನು ಸಣ್ಣಪುಟ್ಟ ಭಾಷಣಮಾಡಲು ಕರೆಯತೊಡಗಿದರು. ಇಂಡಿಯಾ ದೇಶದ ಧರ್ಮ ಆಚಾರ ವ್ಯವಹಾರದ ವಿಷಯವಾಗಿ ತಿಳಿದುಕೊಳ್ಳಲು ಅಮೇರಿಕಾ ದೇಶದ ಜನರ ಕುತೂಹಲ ಕೆರಳತೊಡಗಿತು. 

 ಸ್ವಾಮೀಜಿಯವರು ಬಾಸ್ಟನ್ನಿನ ಸುತ್ತಮುತ್ತ ಕ್ರಮೇಣ ಪರಿಚಿತರಾಗುತ್ತ ಬಂದರು. ಅವರು ಸಣ್ಣ ಪುಟ್ಟ ಕ್ಲಬ್ಬುಗಳಲ್ಲಿ ಭರತಖಂಡದ ಮಕ್ಕಳ ಆಟಗಳು, ಹಿಂದೂಧರ್ಮ ಮುಂತಾದ ವಿಷಯಗಳ ಮೇಲೆ ಮಾತನಾಡತೊಡಗಿದರು. ಜನರು ಉತ್ಸಾಹದಿಂದ ಅವರ ಉಪನ್ಯಾಸವನ್ನು ಕೇಳಲು ತೊಡಗಿದರು. ಆ ಸಮಯದಲ್ಲೆ ಹಾರ್‍ವರ್ಡ್ ವಿಶ್ವವಿದ್ಯಾನಿಲಯದಲ್ಲಿ ಗ್ರೀಕ್ ಭಾಷೆಯ ಪ್ರಾಧ್ಯಾಪಕರಾದ ಪ್ರೊಫೆಸರ್ ಜಾನ್ ಹೆನ್ರಿ ರೈಟ್ ಎಂಬುವರು ತಮ್ಮ ಮನೆಗೆ ಅತಿಥಿಗಳಾಗಿ ಬರಬೇಕೆಂದು ಸ್ವಾಮೀಜಿಯವರನ್ನು ಕೋರಿಕೊಂಡರು. ಅಲ್ಲಿ ಪ್ರೊಫೆಸರ್ ಮತ್ತು ಸ್ವಾಮಿಗಳು ಗಂಟೆಗಳ ಕಾಲ ಮಾತನಾಡಿದರು. ಸ್ವಾಮೀಜಿ ವಿದ್ವತ್ತಿಗೆ ಪ್ರೊಫೆಸರ್ ಮಾರುಹೋದರು. ಅವರು ಆ ಊರಿನಲ್ಲಿ (ಆನಿಸ್ಕ್ವಾಂಮ್) ಒಂದು ಚರ್ಚಿನಲ್ಲಿ ಸ್ವಾಮೀಜಿಯವರನ್ನು ಮಾತನಾಡುವಂತೆ ಕೋರಿಕೊಂಡರು. ಉಪನ್ಯಾಸವಾದಮೇಲೆ ಹಲವು ಪತ್ರಿಕಾ ಪ್ರತಿನಿಧಿಗಳು ಸ್ವಾಮೀಜಿಯವರನ್ನು ನೋಡಲು ಪ್ರೊಫೆಸರ್ ಮನೆಗೆ ಬಂದರು. ಅವರು ತಾವು ಕಂಡು ಕೇಳಿದುದನ್ನು ಅಂದಿನ ವಾರ್ತಾ ಪತ್ರಿಕೆಯಲ್ಲಿ ಬರೆಯುತ್ತಿದ್ದರು. ಅದನ್ನು ಕೆಳಗೆ ಉದಾಹರಿಸುವೆವು: 

 “೧೮೯೩ನೇ ಆಗಸ್ಟ್ ಕೊನೆಯಲ್ಲಿ ಸ್ವಾಮಿ ವಿವೇಕಾನಂದರು ಆನಿಸ್ಕ್ವಾಂಮ್‍ನಲ್ಲಿ ಪ್ರೊಫೆಸರ್ ರೈಟ್ ಅವರ ಮನೆಯಲ್ಲಿದ್ದರು. ಪ್ರಶಾಂತವಾದ ನ್ಯೂ ಇಂಗ್ಲೆಂಡ್ ಸರಹದ್ದಿನ ಈ ಹಳ್ಳಿಯಲ್ಲಿ, ಜನರ ಮನಸ್ಸಿನಲ್ಲಿ ಈ ಭವ್ಯ ವ್ಯಕ್ತಿ ಯಾರಿರಬಹುದು, ಅವರು ಎಲ್ಲಿಂದ ಬಂದಿರಬಹುದು, ಎಂಬ ಕುತೂಹಲವಿತ್ತು. ವೃತ್ತಪತ್ರಿಕಾ ಪ್ರತಿನಿಧಿಗಳು ಸ್ವಾಮೀಜಿಯವರು ಇದ್ದ ಮನೆಗೆ ಅವರು ಮಾತನಾಡುವುದನ್ನು ಕೇಳುವುದಕ್ಕೆ ಹೋದರು.” 

 “ಸ್ವಾಮೀಜಿಯವರು ತಮ್ಮ ಸುಮಧುರವಾದ ಧ್ವನಿಯಲ್ಲಿ, ‘ಈಗೆಲ್ಲೋ ಸ್ವಲ್ಪ ಕಾಲದ ಹಿಂದೆ, ಸ್ವಲ್ಪ ದಿನಗಳ ಹಿಂದೆ, ನಾನ್ನೂರು ವರುಷಗಳು ಕೂಡ ಆಗಿಲ್ಲ’ ಎಂದು ಮಾತನಾಡತೊಡಗಿದರು, ‘ಸಹಿಷ್ಣುಗಳಾದ, ಯಾವಾಗಲೂ ಸಂಕಟವನ್ನು ಅನುಭವಿಸುತ್ತಿರುವ ಜನಾಂಗದ ಮೇಲೆ ದಬ್ಬಾಳಿಕೆ ಕ್ರೌರ್ಯ ಪ್ರಾರಂಭವಾಯಿತು. ಕೊನೆಗೆ ದೇವರು ತನ್ನ ತೀರ್ಪನ್ನು ಕೊಡುವನು’ ಎಂದರು. ‘ಓ ಇಂಗ್ಲೀಷಿನವರು, ಅವರೆಲ್ಲೊ ಕೆಲವು ವರುಷಗಳ ಹಿಂದೆ ಕಾಡುಮನುಷ್ಯರಾಗಿದ್ದರು. ಹುಳಹುಪ್ಪಟೆಗಳು ಅವರ ಹೆಂಗಸರ ದೇಹದ ಮೇಲೆ ಹರಿದಾಡುತ್ತಿದ್ದವು. ಅವರು ತಮ್ಮ ದೇಹದ ದುರ್ಗಂಧವನ್ನು ಮರೆಸಲು ಸುಗಂಧವನ್ನು ಲೇಪಿಸಿಕೊಳ್ಳುತ್ತಿದ್ದರು. ಇದು ಜುಗುಪ್ಸಾಕಾರಕ! ಈಗಲೂ ಕೂಡ ಅವರು ಆ ಅವಸ್ಥೆಯಿಂದ ಪಾರಾಗಲು ಯತ್ನಿಸುತ್ತಿರುವರು’ ಎಂದರು.” 

 “ಅವರ ಮಾತನ್ನು ಕೇಳುತ್ತಿದ್ದವರೊಬ್ಬರಿಗೆ ರೇಗಿ ‘ನಿಮ್ಮ ಮಾತಿಗೆ ಅರ್ಥವಿಲ್ಲ. ಆ ಸ್ಥಿತಿ ಆಗಿ ಐನೂರು ವರುಷಗಳಾದರೂ ಆಗಿರಬೇಕು’ ಎಂದರು. ಅದಕ್ಕೆ ಸ್ವಾಮೀಜಿ, ‘ನಾನು ಆಗಲೇ ಹೇಳಲಿಲ್ಲವೆ, ಸ್ವಲ್ಪ ದಿನಗಳ ಹಿಂದೆ ಎಂದು ಮಾನವನ ಆತ್ಮದ ಕಾಲದೊಂದಿಗೆ ಹೋಲಿಸಿದರೆ ಕೆಲವು ನೂರು ವರುಷಗಳು ಎಂದರೆ ಏನು?’ ಎಂದರು. ಅನಂತರ ಅವರ ಧ್ವನಿ ಬದಲಾಯಿಸಿತು, ಮಧುರವಾಯಿತು ವಿಚಾರಪೂರಿತವಾಯಿತು. ‘ಅವರೆಲ್ಲ ನಿಜವಾಗಿ ಕಾಡುಜನರು. ಸಹಿಸಲಸಾಧ್ಯವಾದ ಛಳಿ ಮತ್ತು ಆ ಸನ್ನಿವೇಶದಲ್ಲಿ ಒಬ್ಬರೇ ಇರುವುದಕ್ಕೆ ಅವಕಾಶವೇ ಇಲ್ಲ’ ಎಂದು ಹೇಳಿ, ಅನಂತರ ಸ್ವಲ್ಪ ಬಿರುಸಿನಿಂದ ವೇಗವಾಗಿ ಹೀಗೆಂದರು: ‘ಆ ವಾತಾವರಣ ಅವರನ್ನು ಅನಾಗರಿಕರನ್ನಾಗಿ ಮಾಡಿರುವುದು. ಅವರು ಸುಮ್ಮನೇ ಕೊಲ್ಲುವುದನ್ನೇ ಯೋಚಿಸುತ್ತಾರೆ. ಅವರ ಧರ್ಮ ಎಲ್ಲಿ ಹೋಯಿತು? ಆ ಪವಿತ್ರಾತ್ಮನ (ಕ್ರಿಸ್ತನ) ಹೆಸರನ್ನು ತೆಗೆದುಕೊಳ್ಳುವರು, ತಾವು ಮಾನವಕೋಟಿಯನ್ನು ಪ್ರೀತಿಸುತ್ತೇವೆ ಎಂದು ಹೇಳಿಕೊಳ್ಳುವರು. ಕ್ರೈಸ್ತಧರ್ಮದಿಂದ ಇತರರನ್ನು ನಾಗರಿಕರನ್ನಾಗಿ ಮಾಡುತ್ತೇವೆ ಎನ್ನುವರು! ಇಲ್ಲ, ಅವರನ್ನು ನಾಗರಿಕರನ್ನಾಗಿ ಮಾಡಿರುವುದು ಹಸಿವು. ಮಾನವ ಪ್ರೀತಿ ಎಂದರೆ, ಅವರ ಮಾತಿನಲ್ಲಿ ಹಿಂಸೆ ಮತ್ತು ಪಾಪವಲ್ಲದೆ ಬೇರಲ್ಲ. ಸಹೋದರನೆ, ನಾನು ನಿನ್ನನ್ನು ಪ್ರೀತಿಸುತ್ತೇನೆ, ಎಂದು ಹೇಳುತ್ತಿರುವಾಗಲೇ ಅವನ ಕೊರಳನ್ನು ಕತ್ತರಿಸುವರು. ಅವರ ಕೈಗಳು ರಕ್ತದಿಂದ ನೆನೆದು ಹೋಗಿವೆ.’ ಅನಂತರ ಅವರ ಮಾತು ನಿಧಾನವಾಗುತ್ತ ಬಂತು. ಅವರ ಅತಿಮಧುರವಾದ ಧ್ವನಿ ಒಂದು ಗಂಟೆಯ ಧ್ವನಿಯಂತೆ ಸ್ಪಷ್ಟವಾಯಿತು. ‘ಆದರೆ ದೇವರ ತೀರ್ಪು ಅವರನ್ನು ಕಾದಿದೆ. ಸೇಡು ನನ್ನದು, ಅದನ್ನು ನಾನು ತೀರಿಸಿಕೊಳ್ಳುತ್ತೇನೆ ಎನ್ನುವನು ದೇವರು. ಅವರ ನಾಶ ಸಮೀಪಿಸುತ್ತಿದೆ. ನಿಮ್ಮ ಕ್ರೈಸ್ತ ಜನಾಂಗವೇನು ಮಹಾ? ಅವರು ಪ್ರಪಂಚದಲ್ಲಿ ಮೂರನೆ ಒಂದು ಭಾಗ ಕೂಡ ಇಲ್ಲ. ಕೋಟ್ಯಂತರ\break ಜನರಿರುವ ಚೈನೀಯರನ್ನು ನೋಡಿ. ಅವರೇ ಭಗವಂತ ತನ್ನ ಸೇಡನ್ನು ತೀರಿಸಿಕೊಳ್ಳುವುದಕ್ಕೆ ನಿಮ್ಮ ಮೇಲೆ ಬೀಳುವರು; ಮತ್ತೊಂದು ಹೂಣರ ದಂಡಯಾತ್ರೆ ಆಗುವುದು. ಅವರು ಯೂರೋಪಿನ ಮೇಲೆ ಧಾಳಿ ನಡೆಸುವರು. ಅವರು ಯೂರೋಪಿನಲ್ಲಿ ಒಂದು ಕಟ್ಟಡವನ್ನೂ ಬಿಡುವುದಿಲ್ಲ. ಎಲ್ಲವನ್ನೂ ಧ್ವಂಸ ಮಾಡುವರು. ಗಂಡಸರು ಹೆಂಗಸರು ಮಕ್ಕಳು ಯಾರನ್ನೂ ಬಿಡುವುದಿಲ್ಲ. ಎಲ್ಲಾ ನಾಶವಾಗುವುದು. ಪುನಃ ಅನಾಗರಿಕ ಯುಗ ಪ್ರಾರಂಭವಾಗುವುದು– ದೇವರ ಸೇಡು ಬರುತ್ತಿದೆ’.” 

 ‘ಏನು ಬೇಗ ಬರುವುದೆ?’ ಎಂದು ಕುಳಿತವರೆಲ್ಲ ಕೇಳಿದರು. ಸ್ವಾಮೀಜಿ ‘ಅದಾಗುವುದಕ್ಕೆ ಒಂದು ಸಾವಿರ ವರುಷಗಳು ಹಿಡಿಯಬಹುದು’ ಎಂದರು. ಹತ್ತಿರ ಕುಳಿತವರು, ಸಧ್ಯಕ್ಕೆ ಅದು ಈಗ ಆಗುವುದಿಲ್ಲವಲ್ಲ ಎಂದು ಸಮಾಧಾನಪಟ್ಟರು.” 

 “ಸ್ವಾಮೀಜಿ ಮುಂದುವರಿಸಿದರು: ‘ದೇವರು ಸೇಡನ್ನು ತೀರಿಸಿಕೊಳ್ಳುವನು. ನೀವು ಅದನ್ನು ಧರ್ಮದಲ್ಲಿ ಕಾಣದೆ ಇರಬಹುದು, ಅದನ್ನು ರಾಜಕೀಯದಲ್ಲಿ ಕಾಣದೆ ಇರಬಹುದು, ಆದರೆ ಇತಿಹಾಸದಲ್ಲಿ ನೀವು ಅದನ್ನು ನೋಡುವಿರಿ. ಎಂದಿನಂತೆ ಇದೂ ಆಗುವುದು. ನೀವು ಜನರನ್ನು ಹಿಂಸಿಸಿದರೆ ನೀವು ಕೂಡ ಅದೇ ಹಿಂಸೆಗೆ ಒಳಗಾಗಬೇಕಾಗುವುದು. ಭರತಖಂಡದಲ್ಲಿರುವ ನಾವು ದೇವರ ಕೋಪಕ್ಕೆ ತುತ್ತಾಗಿರುವೆವು. ಅವರು ತಮ್ಮ ಐಶ್ವರ‍್ಯವನ್ನು ವೃದ್ಧಿಮಾಡಿಕೊಳ್ಳುವುದಕ್ಕಾಗಿ ಬಡಜನರ ಕೈಯಿಂದ ಚೆನ್ನಾಗಿ ದುಡಿಸಿಕೊಂಡರು. ಅವರ ಗೋಳಿಗೆ ಕಿವಿಕೊಡಲಿಲ್ಲ. ಬಡವರು ಹೊಟ್ಟೆಗೆ ಹಿಟ್ಟು ಇಲ್ಲದೆ ಅಳುತ್ತಿದ್ದಾಗ ಅವರು ಚಿನ್ನ ಬೆಳ್ಳಿಯ ತಟ್ಟೆಯಲ್ಲಿ ಊಟ ಮಾಡುತ್ತಿದ್ದರು. ಮಹಮ್ಮದೀಯರು ಅವರ ಮೇಲೆ ಧಾಳಿ ಇಟ್ಟು ಅವರನ್ನು ಕೊಂದರು. ಎಲ್ಲರನ್ನೂ ಗೆದ್ದರು. ಕಟ್ಟ ಕೊನೆಗೆ ಭಾರತೀಯರಿಗೆ ಬಂದ ಪೀಡೆಯೇ ಆಂಗ್ಲೇಯರ ವಶವಾಗಿದ್ದು.... ಹಿಂದೂಗಳು ಏನನ್ನು ಬಿಟ್ಟಿರುವರು? ದೇಶದಲ್ಲೆಲ್ಲ ಅತಿ ಅದ್ಭುತವಾದ ದೇವಾಲಯಗಳನ್ನು ಬಿಟ್ಟಿರುವರು. ಮಹಮ್ಮದೀಯರು ಏನನ್ನು ಬಿಟ್ಟಿರುವರು? ಸುಂದರವಾದ ಅರಮನೆಗಳನ್ನು ಮಸೀದಿಗಳನ್ನು. ಇಂಗ್ಲೀಷಿನವರು ಏನನ್ನು ಬಿಟ್ಟರು? ಒಡೆದ ಬ್ರಾಂಡಿ ಬುಡ್ಡಿಗಳ ರಾಶಿಯನ್ನಲ್ಲದೆ ಮತ್ತೇನನ್ನೂ ಅಲ್ಲ. ನಮ್ಮ ಜನರ ಮೇಲೆ ದೇವರಿಗೆ ಕರುಣೆ ಇರಲಿಲ್ಲ. ಏಕೆಂದರೆ, ನಮ್ಮ ಜನರಲ್ಲಿಯೇ ಕರುಣೆ ಇರಲಿಲ್ಲ. ತಮ್ಮ ಕ್ರೂರತನದಿಂದ ಜನಸಾಮಾನ್ಯರನ್ನು ಅಧೋಗತಿಗೆ ತಂದರು. ಅವರಿಗೇ ಸಹಾಯ ಬೇಕಾದಾಗ, ಜನ ಸಾಧಾರಣರು ಅವರಿಗೆ ಸಹಾಯ ಮಾಡಲು ಆಗಲಿಲ್ಲ. ಮನುಷ್ಯನಿಗೆ ದೇವರ ಪ್ರತೀಕಾರದಲ್ಲಿ ನಂಬುಗೆ ಇಲ್ಲದೇ ಇದ್ದರೆ ಇತಿಹಾಸದ ಪ್ರತೀಕಾರದಲ್ಲಾದರೂ ನಂಬದೇ ಇರಲಾರ. ಇದು ಇಂಗ್ಲೀಷಿನವರ ಮೇಲೆ ಬರುವುದು. ಅವರು ನಮ್ಮ ಕತ್ತಿನಮೇಲೆ ತಮ್ಮ ಕಾಲನ್ನು ಇಟ್ಟಿರುವರು. ತಮ್ಮ ಸುಖಕ್ಕಾಗಿ ನಮ್ಮ ರಕ್ತದ ಕೊನೆಯ ಬಿಂದುವನ್ನೂ ಹೀರಿರುವರು. ನಮ್ಮ ಹಳ್ಳಿಗಳು ಪ್ರಾಂತ್ಯಗಳು ಉಪವಾಸಕ್ಕೆ ತುತ್ತಾದಾಗ ನಮ್ಮ ದೇಶದ ಐಶ್ವರ್ಯಗಳನ್ನು ಅವರ ದೇಶಕ್ಕೆ ಸಾಗಿಸಿರುವರು. ಈಗ ಚೈನೀ ದೇಶೀಯರು ಅವರ ಮೇಲೆ ಪ್ರತೀಕಾರವನ್ನು ತೀರಿಸಿಕೊಳ್ಳುವುದಕ್ಕೆ ಬೀಳುವರು.\break ಇಂದು ಚೈನೀಯರೆಲ್ಲ ಜಾಗ್ರತರಾಗಿ ಇಂಗ್ಲೀಷಿನವರನ್ನು ಸಮುದ್ರದ ಪಾಲು ಮಾಡಿದರೆ ಅದು ಅವರಿಗೆ ಯೋಗ್ಯವಾದ ಶಿಕ್ಷೆಯೇ ಆಗುವುದು....’.” 

 “ಇದನ್ನು ಹೇಳಿ ಆದ ಮೇಲೆ ಸ್ವಾಮೀಜಿ ಮೌನವಾಗಿದ್ದರು. ಅವರ ವಿಷಯವಾಗಿ ಕೆಲವರು ಮೆಲ್ಲಗೆ ಮಾತನಾಡಿಕೊಳ್ಳುತ್ತಿದ್ದರು. ಸ್ವಾಮೀಜಿ ಅದನ್ನು ಬೇಕು ಬೇಡದಂತೆ ಕೇಳುತ್ತಿದ್ದರು. ಮಧ್ಯೆ ಮಧ್ಯೆ ಅವರ ಮೇಲೆ ತಮ್ಮ ದೃಷ್ಟಿಯನ್ನು ಬೇರುತ್ತ ‘ಶಿವ, ಶಿವ’ ಎಂದು ಹೇಳುತ್ತಿದ್ದರು. ಇವರ ಸುತ್ತಲೂ ಇದ್ದ ಆ ಸಣ್ಣ ತಂಡದವರು ಮಾತ್ರ ಈ ವಿಚಿತ್ರ ವ್ಯಕ್ತಿಯ ಅಂತರಾಳದಿಂದ, ಜ್ವಾಲಾಮುಖದಿಂದ ಉಕ್ಕಿ ಹರಿಯುವ ಶಿಲಾಪ್ರವಾಹದಂತೆ ಇರುವ ಇವರ ಪ್ರಚಂಡ ಭಾವನೆ, ಮತ್ತು ಮತ್ತೊಬ್ಬರನ್ನು ಸದೆಬಡೆಯುವಂತಹ ವಿಚಾರಸರಣಿಗೆ ಅಸ್ತವ್ಯಸ್ಥರಾದರು.” 

 ಪ್ರೊಫೆಸರ್ ರೈಟ್ ಅವರು ಸ್ವಾಮೀಜಿಯವರ ಸಂಭಾಷಣೆಗಳನ್ನು ಕೇಳಿದಮೇಲೆ ವಿಶ್ವಧರ್ಮಸಮ್ಮೇಳನದಲ್ಲಿ ಸ್ವಾಮೀಜಿ ಮಾತನಾಡಿದರೆ ಇಡೀ ದೇಶಕ್ಕೆ ಪರಿಚಿತರಾಗುವರೆಂದು ಹೇಳಿದರು. ಸ್ವಾಮೀಜಿಯವರು ತಾವು ಆ ಉದ್ದೇಶದಿಂದಲೇ ಅಮೇರಿಕಾ ದೇಶಕ್ಕೆ ಬಂದಿರುವುದೆಂದೂ ಆದರೆ ಇಲ್ಲಿ ಭರತಖಂಡದ ಯಾವುದಾದರೂ ಸಂಸ್ಥೆಯ ಅರ್ಹತಾಪತ್ರ ಬೇಕೆಂದು ಕೇಳುತ್ತಿರುವರೆಂದೂ ಹೇಳಿದರು. ಆಗ ಪ್ರೊಫೆಸರ್ “ಸ್ವಾಮೀಜಿ, ನಿಮ್ಮಿಂದ ಅರ್ಹತಾಪತ್ರ ಕೇಳುವುದು, ಸೂರ್ಯನಿಗೆ ನಿನ್ನನ್ನು ಯಾರು ಬೆಳಗುವಂತೆ ಹೇಳಿದರು ಎಂದಂತೆ” ಎಂದು ಹೇಳಿದರು. ವಿಶ್ವಧರ್ಮ ಸಮ್ಮೇಳನದ ಕಮಿಟಿಯಲ್ಲಿ ತಮಗೆ ಪರಿಚಯಸ್ಥರು ಇರುವರೆಂದೂ, ತಾವು ಅವರಿಗೆ ಸ್ವಾಮೀಜಿಯವರನ್ನು ಒಬ್ಬ ಸದಸ್ಯನನ್ನಾಗಿ ಮಾಡುವ ಭಾರ ತಮ್ಮದು ಎಂದೂ ಹೇಳಿದರು. ಅವರ ಕೈಯಲ್ಲಿಯೂ ಒಂದು ಪರಿಚಯ ಪತ್ರವನ್ನು ಕೊಟ್ಟು ಅದರಲ್ಲಿ “ಸ್ವಾಮೀಜಿ, ನಮ್ಮಂತಹ ಪ್ರೊಫೆಸರ್‍ಗಳೆಲ್ಲರನ್ನು ಒಟ್ಟಿಗೆ ಸೇರಿಸಿದರೂ ಮೀರಿ ತೂಗುವ ವಿದ್ವಾಂಸರು” ಎಂದು ಬರೆದರು. ಸ್ವಾಮೀಜಿ ಹತ್ತಿರ ದುಡ್ಡು ಇಲ್ಲದೇ ಇರುವುದನ್ನು ಕಂಡು ಅವರೆ ಚಿಕಾಗೊ ನಗರಕ್ಕೆ ಒಂದು ರೈಲ್ವೆ ಟಿಕೆಟ್ ತೆಗೆದುಕೊಟ್ಟು, ಸ್ನೇಹಿತರಿಗೆ ಕೆಲವು ಪರಿಚಯ ಪತ್ರಗಳನ್ನು ಕೊಟ್ಟರು. 

 ಭಗವಂತನ ಕೃಪಾಹಸ್ತ ಎಲ್ಲಿಂದಲೋ ಬಂದು ಸ್ವಾಮೀಜಿಯವರನ್ನು ಮೇಲೆತ್ತಿತು. ದೇವರಿಗೆ ಮನಸ್ಸಿನಲ್ಲಿಯೇ ಧನ್ಯವಾದವನ್ನು ಅರ್ಪಿಸುತ್ತ ಪ್ರೊಫೆಸರಿಗೆ ಕೃತಜ್ಞತೆಯನ್ನು ವ್ಯಕ್ತಪಡಿಸಿ ಚಿಕಾಗೋ ನಗರಕ್ಕೆ ರೈಲು ಹತ್ತಿದರು. ತಮ್ಮ ಗಾಡಿಯಲ್ಲಿಯೇ ಪ್ರಯಾಣ ಮಾಡುತ್ತಿದ್ದ ಒಬ್ಬ ವರ್ತಕ ಸ್ವಾಮೀಜಿ ಅವರಿಗೆ ಚಿಕಾಗೊ ನಗರದ ಯಾವ ಸಬ್ ಸ್ಟೇಶನಿನ್ನಲ್ಲಿ ಇಳಿಯಬೇಕು, ಧರ್ಮಸಮ್ಮೇಳನದ ಕಾರ್ಯಾಲಯಕ್ಕೆ ಹೋಗಬೇಕಾದರೆ ಎಂಬುದನ್ನು ಸ್ವಾಮೀಜಿಗೆ ತಿಳಿಸುತ್ತೇನೆ ಎಂದು ಹೇಳಿದ್ದ. ಆದರೆ ಆತ ಇಳಿಯುವ ಸ್ಟೇಶನ್ ಬಂದಾಗ ಎಲ್ಲವನ್ನೂ ಮರೆತು ಇಳಿದು ಬಿಟ್ಟ. ಸ್ವಾಮೀಜಿ ಬಳಿ ಇದ್ದ ವಿಶ್ವಧರ್ಮ ಸಮ್ಮೇಳನದ ವಿಳಾಸವನ್ನು ಸೂಚಿಸುವ ಪತ್ರ ಅವಸರದಲ್ಲಿ ಎಲ್ಲಿಯೋ ಕಳೆದು ಹೋಯಿತು. ಎಲ್ಲಿ ಇಳಿಯಬೇಕೆಂದು ಗೊತ್ತಿಲ್ಲದೆ ಚಿಕಾಗೋ ಮುಖ್ಯ ರೈಲ್ವೆ ನಿಲ್ದಾಣಕ್ಕೆ ಒಂದು ಸ್ಟೇಶನ್ ಮುಂಚೆಯೇ ಇಳಿದುಬಿಟ್ಟರು. ಆಗ ರಾತ್ರಿಯಾಗಿತ್ತು. ರೈಲ್ವೆ\break ನಿಲ್ದಾಣದಿಂದ ಸ್ವಲ್ಪದೂರ ನಡೆದುಕೊಂಡು ಹೋದರು. ಅಲ್ಲಿ ವಿಶ್ವಧರ್ಮ ಸಮ್ಮೇಳನ ನಡೆಯುವ ಸ್ಥಳ ಯಾವುದೆಂದು ಯಾರನ್ನು ಕೇಳಿದರೂ ಅವರಿಗೆ ಗೊತ್ತಿರಲಿಲ್ಲ. ಅವರಿಗೆ ಸಿಕ್ಕಿದ ಬಹುಪಾಲು ಜನ ಜರ್ಮನ್ ಕೂಲಿಕಾರರು, ಅವರಿಗೆ ಇಂಗ್ಲೀಷಿನ ಗಂಧವೇ ಇಲ್ಲ. ಒಂದು ಹೋಟೆಲಿನಲ್ಲಿಯಾದರೂ ಇಳಿದುಕೊಳ್ಳೋಣ ಅದು ಎಲ್ಲಿದೆ ಎಂದು ವಿಚಾರಿಸುವುದಕ್ಕೂ ಅವರಿಗೆ ಗೊತ್ತಾಗುವಂತೆ ಮಾಡುವುದು ಕಷ್ಟವಾಯಿತು. ದಿನವೆಲ್ಲ ಪ್ರಯಾಣಮಾಡಿ ಸಾಕಾಗಿತ್ತು. ರೈಲನ್ನು ಏರುವುದಕ್ಕೆ ಮುಂಚೆ ತುಂಬಾ ಹರ್ಷದಿಂದ ಇದ್ದರು. ಈಗ ಎಲ್ಲಿ ತಂಗಬೇಕೋ ಅದೇ ಗೊತ್ತಿರಲಿಲ್ಲ. ಪುನಃ ರೈಲ್ವೆ ನಿಲ್ದಾಣಕ್ಕೆ ಬಂದರು. ಆಗಲೇ ವೇಳೆ ಆಗಿ ಹೋಗಿತ್ತು. ಹತ್ತಿರ ಒಂದು ದೊಡ್ಡ ಖಾಲಿ ಪೆಟ್ಟಿಗೆ ಇತ್ತು. ಸ್ವಾಮೀಜಿ ಅದರಲ್ಲೆ ಮಲಗಿಕೊಂಡರು. ಬೆಳಗ್ಗೆ ಆದ ಮೇಲೆ ಯಾವುದನ್ನೂ ವಿಚಾರಿಸುವ ಎಂದು ನಿದ್ರಿಸತೊಡಗಿದರು. 

 ಬೆಳಗಾಯಿತು. ತಮ್ಮ ಅದೃಷ್ಟವನ್ನು ಪರೀಕ್ಷಿಸಬೇಕೆಂದು ಹೊರಟರು. ಹಿಂದಿನ ರಾತ್ರಿ ಉಪವಾಸ. ಅಲೆದಲೆದು ಸಾಕಾಗಿತ್ತು. ಹೊಟ್ಟೆ ಹಸಿವು. ಹಾಗೇ ನಡೆದುಕೊಂಡು ಹೋದರು. ಅದು ಶ‍್ರೀಮಂತರು ಇರುವ ಭಾಗವಾಗಿತ್ತು. ದೊಡ್ಡ ದೊಡ್ಡ ಮನೆಗಳು ಉದ್ಯಾನವನಗಳು ಇದ್ದುವು. ರಸ್ತೆ ಬಹಳ ಅಗಲವಾಗಿ ಚೊಕ್ಕಟವಾಗಿದ್ದುವು. ಇಂಡಿಯಾ ದೇಶದಲ್ಲಿ ಸಂನ್ಯಾಸಿ ಊಟಕ್ಕೆ ಭಿಕ್ಷೆ ಬೇಡುವನು. ಅದರಂತೆಯೇ ಸ್ವಾಮೀಜಿ ಭಿಕ್ಷೆ ಬೇಡಲು ಹೋದರು. ಅವರ ಮಾಸಿದ ಬಟ್ಟೆ, ಮುಂತಾದುವನ್ನು ನೋಡಿ ಮನೆಯ ಮುಂದೆ ಇದ್ದ ಆಳುಗಳು ಸ್ವಾಮೀಜಿಯವರನ್ನು ಒಳಗೆ ಬಿಡಲೇ ಇಲ್ಲ. ಕೆಲವು ವೇಳೆ ಒಳಗೆ ಹೋದಾಗ ಮನೆಯವರು ಅಗೌರವದಿಂದ ಕಂಡು ಅವರನ್ನು ಕಳುಹಿಸಿಬಿಟ್ಟರು. ನಡೆದು ಸಾಕಾಗಿ ದಾರಿಯ ಪಕ್ಕದಲ್ಲಿ ಒಂದು ಹುಲ್ಲು ಮೈದಾನದಲ್ಲಿ ಕುಳಿತರು. ಅವರ ಎದುರಿಗೇ ಒಂದು ದೊಡ್ಡ ಮನುಷ್ಯರ ಮನೆ ಇತ್ತು. ಒಬ್ಬ ಭದ್ರಮಹಿಳೆ ಬಾಗಿಲು ತೆರೆದು ಸ್ವಾಮೀಜಿ ಹತ್ತಿರ ಬಂದು ವಿಚಾರಿಸಿದಳು; “ನೀವು ವಿಶ್ವಧರ್ಮ ಸಮ್ಮೇಳನಕ್ಕೆ ಹೋಗುವ ಸದಸ್ಯರೇ?” ಎಂದು. ಸ್ವಾಮೀಜಿ ಹೌದು ಎಂದು ಹೇಳಿ ತಮಗೆ ಆ ಸ್ಥಳ ಗೊತ್ತಿಲ್ಲವೆಂದೂ ಹೇಳಿದರು. ಆಕೆ ಸ್ವಾಮೀಜಿಯವರನ್ನು ಒಳಗೆ ಕರೆದು “ಊಟ ಸ್ನಾನಾದಿಗಳನ್ನು ಮಾಡಿ. ಮಧ್ಯಾಹ್ನದ ಮೇಲೆ ವಿಶ್ವಧರ್ಮದ ಆಫೀಸಿಗೆ ನಾನು ನಿಮ್ಮನ್ನು ಕರೆದುಕೊಂಡು ಹೋಗಿ ಬಿಡುತ್ತೇನೆ” ಎಂದಳು. ಈಕೆಯೇ ಶ‍್ರೀಮತಿ ಜಾರ್ಜ್ ಡಬ್ಲ್ಯು ಹೇಲ್ಸ್ ಎಂಬುವರು. ಇವರು ಇವರ ಗಂಡ ಮತ್ತು ಹೆಣ್ಣುಮಕ್ಕಳು ಅನಂತರ ಸ್ವಾಮೀಜಿಯವರ ಆಜೀವ ಸ್ನೇಹಿತರಾಗುವರು. ಸ್ವಾಮೀಜಿ ಆ ಸ್ತ್ರೀಯನ್ನು ಮದರ್ ಚರ್ಚ್ ಎಂದೂ ಅವರ ಯಜಮಾನರನ್ನು ಫಾದರ್ ಪೋಪ್ ಎಂದೂ ಅವರ ನಾಲ್ಕು ಜನ ಹೆಣ್ಣುಮಕ್ಕಳನ್ನು ಸಹೋದರಿಯರೆಂದೂ ಕರೆಯುತ್ತಿದ್ದರು. ಹಲವಾರು ಪತ್ರಗಳನ್ನು ಹೇಲ್ ಸಹೋದರಿಯರಿಗೆ ಸ್ವಾಮೀಜಿ ಬರೆದಿರುವರು. ಯಾವ ಅಣ್ಣನೂ ತಂಗಿಯನ್ನು ಇಷ್ಟು ಪ್ರೀತಿಸಿರಲಾರ. ಆ ಪತ್ರಗಳಲ್ಲೆಲ್ಲ ಪ್ರಪಂಚದ ಇತ್ತಕಡೆ ಇರುವ ಸಹೋದರಿಯರಿಗೆ ತಮ್ಮ ಪ್ರೀತಿ ವಿಶ್ವಾಸಗಳನ್ನು ಅಪೂರ್ವವಾದ ರೀತಿಯಲ್ಲಿ ವ್ಯಕ್ತಪಡಿಸಿರುವರು. 

 ಮಧ್ಯಾಹ್ನ ಊಟವಾದ ಮೇಲೆ ಶ‍್ರೀಮತಿ ಹೇಲ್ ಸ್ವಾಮೀಜಿಯವರನ್ನು ವಿಶ್ವಧರ್ಮದ ಆಫೀಸಿಗೆ ಕರೆದುಕೊಂಡು ಹೋದರು. ಅಲ್ಲಿ ಸ್ವಾಮೀಜಿ ತಮ್ಮ ಪರಿಚಯ ಪತ್ರವನ್ನು ಅವರಿಗೆ ತೋರಿದರು. ಅವರನ್ನು ಒಬ್ಬ ಸದಸ್ಯನನ್ನಾಗಿ ಮಾಡಿ, ಸದಸ್ಯರಿಗೆ ಎಂದು ಅಣಿಮಾಡಿದ ಒಂದು ಮನೆಗೆ ಇವರನ್ನು ಕಳುಹಿಸಿದರು. 

 ಧರ್ಮಸಮ್ಮೇಳನ ಪ್ರಾರಂಭವಾಗುವುದಕ್ಕೆ ಮುಂಚೆ, ಪೌರಾತ್ಯದೇಶಗಳಿಂದ ಬರುವ ಸದಸ್ಯರನ್ನು ಅನುಕೂಲವಿದ್ದವರು ತಮ್ಮ ಮನೆಯಲ್ಲಿ ಅತಿಥಿಗಳಾಗಿ ತೆಗೆದುಕೊಳ್ಳಬೇಕೆಂದೂ, ಹಾಗೆ ಇಚ್ಛೆ ಇರುವವರು, ಆಫೀಸಿಗೆ ಕಾಗದ ಬರೆಯಬೇಕೆಂದೂ ಪತ್ರಿಕೆಗಳ ಮೂಲಕ ಕೋರಿಕೊಂಡಿದ್ದರು. ಶ‍್ರೀ ಜಾನ್ ಬಿ. ಲಿಯನ್ ಎನ್ನುವರು ಉದಾರ ಹೃದಯದ ಯಾರಾದರೂ ಸದಸ್ಯರನ್ನು ತಮ್ಮ ಮನೆಗೆ ಕಳುಹಿಸಿಕೊಡಬಹುದೆಂದು ಕೋರಿಕೊಂಡಿದ್ದರು. ಆತನಿಗೆ ಮತಭ್ರಾಂತರನ್ನು ಕಂಡರೆ ಆಗುತ್ತಿರಲಿಲ್ಲ. ಯಾರು ಅವರ ಮನೆಗೆ ಬರುತ್ತಾರೆಯೊ ಅದನ್ನು ವಿಶ್ವಧರ್ಮಸಮ್ಮೇಳನದ ಆಫೀಸಿನಿಂದ ಇನ್ನೂ ತಿಳಿಸಿರಲಿಲ್ಲ. ಆದರೆ ಮನೆಯಲ್ಲಿ ವಿಶ್ವಮೇಳವನ್ನು ನೋಡುವುದಕ್ಕೆ ದಕ್ಷಿಣದ ಕಡೆಯಿಂದ ಅನೇಕ ನೆಂಟರು ಬಂದಿದ್ದರು. ಅಮೇರಿಕಾ ದೇಶದ ದಕ್ಷಿಣಾತ್ಯ ಜನರು ಇನ್ನೂ ಆಗಿನ ಕಾಲದಲ್ಲಿ ನೀಗ್ರೋಗಳಿಗೆ ವಿರೋಧವಾಗಿದ್ದರು; ಮತ್ತು ಅವರನ್ನು ಬಹಳ ನಿಕೃಷ್ಟ ದೃಷ್ಟಿಯಿಂದ ನೋಡುತ್ತಿದ್ದರು. ಒಂದು ದಿನ ರಾತ್ರಿ ಲಿಯನ್ಸ್ ಮನೆಗೆ ಪೌರಾತ್ಯ ದೇಶದ ಒಬ್ಬ ಸದಸ್ಯರನ್ನು ಅತಿಥಿಗಳಾಗಿ ಕರೆದುಕೊಂಡು ಬರುತ್ತಾರೆ ಎಂದು ಸುದ್ದಿಯನ್ನು ಕಳುಹಿಸಿದರು. ಅವರು ಅರ್ಧರಾತ್ರಿ ಹೊತ್ತಿಗೆ ಬರುವವರಿದ್ದರು. ಮನೆಯವರೆಲ್ಲ ಮಲಗಿ ನಿದ್ರಿಸುತ್ತಿದ್ದರು. ಶ‍್ರೀಮತಿ ಲಿಯನ್ಸ್ ಅತಿಥಿಯನ್ನು ಬರಮಾಡಿಕೊಳ್ಳಲು ಎದ್ದರು. ಅರ್ಧರಾತ್ರಿ ಸಮಯದಲ್ಲಿ ಸ್ವಾಮೀಜಿ ಬಂದರು. ಮನೆಯಾಕೆ ಅವರನ್ನು ಸ್ವಾಗತಿಸಿ ಅವರಿಗೆ ಅಣಿಮಾಡಿದ ಕೋಣೆಯಲ್ಲಿ ಬಿಟ್ಟರು. ಆಕೆ ವಿವೇಕಾನಂದರನ್ನು ಕಂಡು ಸ್ವಲ್ಪ ಚಿಂತಿಸತೊಡಗಿದರು. ತಮ್ಮ ಮನೆಗೆ ಬಂದ ಇತರ ಅತಿಥಿಗಳು ಮತ್ತು ಸ್ನೇಹಿತರು ನೀಗ್ರೋದ್ವೇಷಿಗಳಾದ ದಕ್ಷಿಣಾತ್ಯರು. ಅವರು ಸ್ವಾಮೀಜಿಯವರನ್ನು ಹೇಗೆ ನೋಡುವರೋ, ಅವರಿಗೆ ಎಲ್ಲಿ ತೊಂದರೆಯಾದೀತೊ ಎಂದು ಚಿಂತಿಸತೊಡಗಿದರು. ಬೆಳಿಗ್ಗೆ ಹೊತ್ತಿಗೆ ಮುಂಚೆ ಎದ್ದು ತಮ್ಮ ಗಂಡನೊಡನೆ ಈ ವಿಚಾರವನ್ನು ಪರ‍್ಯಾಲೋಚಿಸಿದರು. ಇತರರು ಆಕ್ಷೇಪಣೆ ತೆಗೆದರೆ ಸ್ವಾಮಿಗಳಿಗೆ ಹೋಟಲಿನಲ್ಲಿ ಒಂದು ರೂಮನ್ನು ಮಾಡಿ ಅಲ್ಲಿ ಇರಿಸಬಹುದು ಎಂದು ಆಲೋಚಿಸಿದರು. ಬೆಳಿಗ್ಗೆ ಸ್ವಾಮಿಗಳೊಡನೆ ಸ್ವಲ್ಪ ಹೊತ್ತು ಮಾತುಕತೆಗಳಾಡಿದ ಮೇಲೆ ಲಿಯನ್ಸ್ ಅವರು ತಮ್ಮ ಸತಿಗೆ ತಾವು ಸ್ವಾಮೀಜಿಯವರನ್ನು ತಮ್ಮ ಮನೆಯಲ್ಲಿಯೇ ಇಟ್ಟುಕೊಳ್ಳುವುದಾಗಿಯೂ, ಇತರರು ಬೇಕಾದರೆ ಬೇರೆ ಕಡೆ ಹೋಗಬಹುದೆಂದೂ ಹೇಳಿದರು. ಶ‍್ರೀಮಾನ್ ಲಿಯನ್ಸ್ ಮತ್ತು ಅವರ ಸತಿ ಸ್ವಾಮಿಗಳನ್ನು ಬಹಳ ಗೌರವದಿಂದ ಕಾಣತೊಡಗಿದರು. ಅವರ ಮನೆಯಲ್ಲಿ ಅವರ ಮಗಳು ಒಂದು ಸಣ್ಣ ಹುಡುಗಿ ಇತ್ತು. ಅದಕ್ಕೆ ಸುಮಾರು ಆರು ವರ್ಷಗಳಿರಬೇಕು. ಸ್ವಾಮೀಜಿಗೂ ಆ ಮಗುವಿಗೂ ದೊಡ್ಡ ಸ್ನೇಹ ಬೆಳೆಯಿತು. ಸ್ವಾಮೀಜಿ ಆ ಮಗುವಿಗೆ\break ಇಂಡಿಯಾ ದೇಶಕ್ಕೆ ಸಂಬಂಧಪಟ್ಟ ಕಥೆಗಳನ್ನು ಹೇಳುತ್ತಿದ್ದರು. ಇಂಡಿಯಾದೇಶದ ನದಿ, ಬೆಟ್ಟ, ಕಾಡು, ಹಕ್ಕಿ, ಮೃಗಗಳು, ಅಲ್ಲಿಯ ಜನರು ಇವುಗಳ ಕಥೆಗಳನ್ನು ಕೇಳಿದಾಗ ಆ ಮಗುವಿಗೆ ಒಂದು ಸ್ವಪ್ನಲೋಕವೇ ತೆರೆದಂತಾಯಿತು. ಸ್ವಾಮೀಜಿ ಹೊರಗೆ ಹೋಗಿ ಸಂಜೆ ಮನೆಗೆ ಬಂದೊಡನೆಯೇ ಆ ಮಗು ಸ್ವಾಮೀಜಿ ತೊಡೆಗಳ ಮೇಲೆ ಕುಳಿತು “ಸ್ವಾಮಿ, ನನಗೆ ಇನ್ನೊಂದು ಕಥೆ ಹೇಳು” ಎಂದು ಕೇಳುತ್ತಿತ್ತು. ಸ್ವಾಮೀಜಿಯವರ ತಲೆಗೆ ಕಟ್ಟಿದ್ದ ಪೇಟವಂತೂ ಆ ಮಗುವಿಗೆ ಒಂದು ವಿಚಿತ್ರವಾಗಿ ಕಂಡಿತು. “ನೀವು ಆ ಪೇಟವನ್ನು ಹೇಗೆ ಕಟ್ಟಿಕೊಳ್ಳುತ್ತೀರಿ ತೋರಿಸಿ?” ಎಂದು ಸ್ವಾಮಿಗಳನ್ನು ಕೇಳುತ್ತಿತ್ತು. ಸ್ವಾಮಿಗಳು ಆ ಮಗುವನ್ನು ತೃಪ್ತಿಪಡಿಸುವುದಕ್ಕಾಗಿ ಅದರ ಎದುರಿಗೆ ಪೇಟವನ್ನು ಸುತ್ತಿ ತೋರಿಸುತ್ತಿದ್ದರು. ತಮ್ಮ ಮನೆಗೆ ಹಲವು ಜನರನ್ನು ಕರೆದು ಲಿಯನ್ಸ್ ಮತ್ತು ಅವರ ಹೆಂಡತಿ ಸ್ವಾಮಿಗಳನ್ನು ಇತರರಿಗೆ ಪರಿಚಯ ಮಾಡಿಸುತ್ತಿದ್ದರು. ಸ್ವಾಮೀಜಿ ಅನಂತರ ಉಪನ್ಯಾಸ ಕೊಟ್ಟಾದ ಮೇಲೆ ಜನ ಕೊಡುತ್ತಾ ಇದ್ದ ಹಣವನ್ನು ಕರವಸ್ತ್ರದಲ್ಲಿ ಕಟ್ಟಿಕೊಂಡು ಬಂದು ಮನೆಯ ಯಜಮಾನಿ ಕೈಗೆ ಕೊಟ್ಟುಬಿಡುತ್ತಿದ್ದರು. ಆಕೆ ಅವುಗಳನ್ನೆಲ್ಲ ಎಣಿಸಿ, ಸ್ವಾಮೀಜಿಗೆ ಎಷ್ಟು ಇದೆ ಎಂದು ಹೇಳಿ, ಅವರ ಹೆಸರಿನಲ್ಲಿ ಬ್ಯಾಂಕಿನಲ್ಲಿಡುತ್ತಿದ್ದರು. ಸ್ವಾಮೀಜಿ ಅನಂತರ ಪ್ರಖ್ಯಾತರಾದ ಮೇಲೆ ಅವರ ಉಪನ್ಯಾಸ ಕೇಳಿ ಪ್ರಭಾವಿತರಾಗಿ ಅನೇಕ ಜನ ಸ್ತ್ರೀಯರು ಅವರನ್ನು ಮುತ್ತುತ್ತಿದ್ದುದನ್ನು ಕಂಡು ಎಲ್ಲಿ ಸ್ವಾಮಿಗಳು ಸ್ತ್ರೀಯರ ಬಲೆಗೆ ಬಿದ್ದುಬಿಡುವರೊ ಎಂಬ ಭಯದಿಂದ ಶ‍್ರೀಮತಿ ಲಿಯನ್ಸ್ ಅವರಿಗೆ ಎಚ್ಚರಿಕೆಯನ್ನು ಕೊಟ್ಟರು. ಆಗ ಸ್ವಾಮೀಜಿ ನಗುತ್ತ ಅಂತಹ ಅಂಜಿಕೆ ತನ್ನ ವಿಷಯದಲ್ಲಿ ಇಟ್ಟುಕೊಳ್ಳಬೇಕಾಗಿಲ್ಲ ಎಂದರು. ತಾವು ಮರದ ಕೆಳಗೆ ಅಥವಾ ಬಡವರ ಗುಡಿಸಿಲಿನಲ್ಲಿ ಹೇಗೆ ಮಲಗುತ್ತಿದ್ದರೊ, ಹಾಗೆಯೇ ರಾಜರ ಅರಮನೆಗಳಲ್ಲಿಯೂ ಮಲಗಿ ಅಭ್ಯಾಸವಿದೆ, ಅಲ್ಲಿ ರಾತ್ರಿಯೆಲ್ಲ ಗಾಳಿ ಬೀಸುವುದಕ್ಕೆ ರೂಪವತಿಯರಾದ ದಾಸಿಯರನ್ನು ನೇಮಿಸುತ್ತಿದ್ದರು ಎಂದರು. 

 ಒಮ್ಮೆ ಸ್ವಾಮೀಜಿ ಅಮೇರಿಕಾ ಜನಾಂಗದಲ್ಲಿರುವ ಸಂಘಟನಾಶಕ್ತಿಯನ್ನು ನೋಡಿ ತಮ್ಮ ದೇಶದಲ್ಲಿ ಅದು ಇದ್ದರೆ ಎಷ್ಟೊಂದು ಶಾಶ್ವತವಾದ ಒಳ್ಳೆಯ ಕೆಲಸವನ್ನು ಮಾಡಬಹುದು ಎನ್ನಿಸಿತು. ಅದಕ್ಕಾಗಿ ಒಂದು ದಿನ ಮನೆಗೆ ಬಂದಾದಮೇಲೆ ಶ‍್ರೀಮತಿ ಲಿಯನ್ಸ್ ಅವರೆದುರಿಗೆ “ನಾನು ಯಾರಿಗೋ ಮಾರುಹೋಗಿರುವೆ” ಎಂದರು. ಮನೆಯ ಯಜಮಾನಿ ಯಾವಳೋ ಯುವತಿ ಸ್ವಾಮಿಗಳನ್ನು ಮರಳು ಮಾಡಿರಬೇಕೆಂದು “ಧನ್ಯಾತ್ಮಳಾವಳೊ ಆಕೆ!” ಎಂದು ಕೇಳಿದರು. ಸ್ವಾಮೀಜಿ, ಅದಕ್ಕೆ ಜೋರಾಗಿ ನಕ್ಕು, ನಿಮ್ಮ ದೇಶದ \enginline{Power of organisation} (ಸಂಸ್ಥಾಬದ್ಧವಾಗಿರುವ ಶಕ್ತಿಗೆ)” ಎಂದರು. 

