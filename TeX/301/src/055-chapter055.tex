
\chapter{ಮಠದ ವಾತಾವರಣದಲ್ಲಿ}

 ಸ್ವಾಮೀಜಿ ಮಠಕ್ಕೆ ಬಂದಮೇಲೆ ಉಪನ್ಯಾಸಾದಿಗಳ ಕೆಲಸವನ್ನೆಲ್ಲಾ ಬಿಟ್ಟುಬಿಟ್ಟರು. ಮಠದಲ್ಲಿರುವವರಿಗೆಲ್ಲ ಸರಿಯಾದ ತರಬೇತನ್ನು ಕೊಡುವುದರ ಕಡೆಗೆ ಗಮನ ಕೊಟ್ಟರು. ಅವರು ಸರಿಯಾಗಿ ಧ್ಯಾನಮಾಡುವುದಕ್ಕಾಗಿ ಬೆಳಿಗ್ಗೆ ಐದುಗಂಟೆಗೆ ಘಂಟೆ ಹೊಡೆಸಿ ಎಲ್ಲರೂ ಕೈಕಾಲು ಮುಖ ತೊಳೆದುಕೊಂಡು ದೇವಸ್ಥಾನಕ್ಕೆ ಧ್ಯಾನಕ್ಕೆ ಬರುವಂತೆ ಹೇಳಿದರು. ತಾವೇ ಎಲ್ಲರಿಗಿಂತಲೂ ಮುಂಚೆ ಧ್ಯಾನಕ್ಕೆ ಹೋಗಿ ಕುಳಿತುಕೊಳ್ಳುತ್ತಿದ್ದರು. 

 ಒಂದು ದಿನ ಧ್ಯಾನದ ಹೊತ್ತಿನಲ್ಲಿ ಕೆಲವು ಹಿರಿಯ ಸ್ವಾಮಿಗಳು ದೇವರಮನೆಗೆ ಧ್ಯಾನಮಾಡಲು ಬರಲಿಲ್ಲ. ಅನಂತರ ಮಠದಲ್ಲಿ ಯಾರು ಧ್ಯಾನಮಾಡುವುದಕ್ಕೆ ಬರಲಿಲ್ಲವೋ ಅವರಿಗೆ ಊಟ ಕೊಡುವುದಿಲ್ಲವೆಂದೂ ಅವರು ಹೊರಗೆ ಹೋಗಿ ಭಿಕ್ಷೆ ಬೇಡಿಕೊಂಡು ಬರಬೇಕೆಂದೂ ಹೇಳಿದರು. ಕಲ್ಕತ್ತೆಯ ಯಾವ ಭಕ್ತರ ಮನೆಗೂ ಅವರು ಊಟಕ್ಕೆ ಹೋಗಕೂಡದೆಂದು ಕಟ್ಟಪ್ಪಣೆ ಮಾಡಿದರು. ದೊಡ್ಡವರೇ ಸಣ್ಣವರಿಗೆ ಮೇಲ್ಪಂಕ್ತಿ ಹಾಕದೆ ಇದ್ದರೆ ಸಂಸ್ಥೆ ಹೇಗೆ ನಡೆಯಬೇಕು ಎಂದು ಅವರನ್ನು ಭರ್ತ್ಸಿಸಿದರು. ಇದನ್ನೆಲ್ಲಾ ಹೇಳಿ ಆದಮೇಲೆ ತಮಗೆ ಕಲ್ಕತ್ತೆಯಲ್ಲಿ ಕೆಲಸ ಇದ್ದುದರಿಂದ ಹೊರಟುಹೋದರು. ಸಂಜೆ ಬಂದ ಮೇಲೆ ಭಿಕ್ಷೆಗೆ ಹೋಗಿರುವವರನ್ನೆಲ್ಲ ಕರೆಸಿ ಅವರಿಗಾದ ವಿಚಿತ್ರ ಅನುಭವಗಳನ್ನು ಕೇಳಿ ತಾವೂ ಅದರಲ್ಲಿ ಭಾಗಿಯಾದರು. 

 ಗುರುಭಾಯಿಗಳೊಡನೆ ಇತರ ಕಾಲದಲ್ಲಿ ಹಾಸ್ಯ ಪರಿಹಾಸ್ಯಗಳು ಮತ್ತು ವೇದಾಂತ ವಿಷಯದಲ್ಲಿ ಚರ್ಚೆ ಮುಂತಾದುವುಗಳಲ್ಲಿ ಭಾಗವಹಿಸುತ್ತಿದ್ದರು. ತಾವೇ ಕೆಲವು ವೇಳೆ ಅಡಿಗೆಯ ಮನೆಗೆ ಹೋಗಿ ಏನಾದರೂ ಅಡಿಗೆಯನ್ನು ಮಾಡಿ ಎಲ್ಲರಿಗೂ ಕೊಡುತ್ತಿದ್ದರು. ಮಠದ ಹಸುಗಳ ಮೇಲೆ ಸ್ವಾಮೀಜಿಗೆ ತುಂಬಾ ಪ್ರೀತಿ. ಅವನ್ನು ನೋಡಿಕೊಳ್ಳುವುದು, ಕರುಗಳೊಂದಿಗೆ ಆಟವಾಡುವುದು ಇವುಗಳಲ್ಲಿ ನಿರತರಾಗಿರುತ್ತಿದ್ದರು. ಕೆಲವು ವೇಳೆ ಬ್ರಹ್ಮಾನಂದರಿಗೂ ಸ್ವಾಮೀಜಿಯವರಿಗೂ ತೋರಿಕೆಯ ಕಲಹ ಆಗುತ್ತಿತ್ತು. ಬ್ರಹ್ಮಾನಂದರಿಗೆ ತರಕಾರಿ ಮತ್ತು ಹೂವಿನ ಗಿಡ ಎಂದರೆ ಪ್ರಾಣ. ಮಠದಲ್ಲಿ ಅವನ್ನು ಬೆಳೆಸುತ್ತಿದ್ದರು. ಕೆಲವು ವೇಳೆ ಸ್ವಾಮೀಜಿಯ ಹಸುಕರುಗಳು ಅದನ್ನು ತಿಂದುಬಿಡುತ್ತಿದ್ದುದಾಗಿ ಬ್ರಹ್ಮಾನಂದರು ಕುಪಿತರಾಗುತ್ತಿದ್ದರು. ಈ ಗುರುಭಾಯಿಗಳ ಆಟದ ಕಲಹ ನೋಡುವುದಕ್ಕೆ ತುಂಬಾ ಆನಂದವಾಗಿರುತ್ತಿತ್ತು. ಮಧ್ಯಾಹ್ನದ ಟೀ ಸಮಯದಲ್ಲಿ ಸ್ವಾಮೀಜಿ ಒಂದು ಪಾತ್ರೆಯನ್ನು ತೆಗೆದುಕೊಂಡು ಹಸುವಿನ ಹತ್ತಿರ ಹೋಗಿ ಟೀಗೆ ಸ್ವಲ್ಪ ಹಾಲುಕೊಡು ಎಂದು ಕೇಳಿ ಕರೆದುಕೊಳ್ಳುತ್ತಿದ್ದರು. ಸ್ವಾಮೀಜಿಗೆ ಬಾಲ್ಯದಿಂದಲೂ ಪ್ರಾಣಿಗಳನ್ನು ಕಂಡರೆ ಇಷ್ಟ. ಈಗ ಪುನಃ ಹಲವು ಪ್ರಾಣಿಗಳನ್ನು ತಮ್ಮ ಸುತ್ತಲೂ ಸಾಕತೊಡಗಿದರು. ಮಠದ ನಾಯಿಗೆ ‘ಭಾಗ’ ಎಂದು ಹೆಸರು ಕೊಟ್ಟರು. ಆ ನಾಯಿ ಸ್ವಾಮೀಜಿಯವರ ಪರಮ ವಿಶ್ವಾಸಕ್ಕೆ ಪಾತ್ರವಾಗಿತ್ತು, ಮಠವನ್ನು ತನ್ನದೇ ಎಂದು ಭಾವಿಸುತ್ತಿತ್ತು. ಒಂದು ಕುರಿಮರಿಯನ್ನು ಸಾಕಿ ಅದನ್ನು ‘ಹಂಸಿ’ ಎಂದು ಕರೆಯುತ್ತಿದ್ದರು. ಕೊಕ್ಕರೆ ಹಂಸ ಬಾತುಗಳು ಬೇರೆ ಇದ್ದವು. ಒಂದು ಜಿಂಕೆಮರಿಗೆ ‘ಮಾತೃ’ ಎಂದು ಹೆಸರಿಟ್ಟು ಅದರ ಕೊರಳಿಗೆ ಒಂದು ಪುಟ್ಟ ಗಂಟೆಯನ್ನು ಕಟ್ಟಿದರು. ಅದು ಮಠದ ಆವರಣದಲ್ಲಿ ಸ್ವಾಮೀಜಿ ಎಲ್ಲಿ ಹೋದರೂ ಅವರ ಹಿಂದೆ ಝಣಝಣ ಘಂಟೆ ಶಬ್ದಮಾಡಿಕೊಂಡು ಹೋಗುತ್ತಿತ್ತು. ಅದು ಸ್ವಾಮೀಜಿ ಕೋಣೆಯಲ್ಲೆ ಮಲಗುತ್ತಿತ್ತು. ಸ್ವಾಮೀಜಿ ಅದನ್ನು ತನ್ನ ಪೂರ್ವ ಜನ್ಮದ ಒಬ್ಬ ನಂಟ ಎಂದು ಭಾವಿಸುತ್ತಿದ್ದರು. ಜಗದ್ವಿಖ್ಯಾತ ವಿವೇಕಾನಂದರು ಈ ಸಾಧು ಪ್ರಾಣಿಗಳ ಮಧ್ಯದಲ್ಲಿ ಎಲ್ಲವನ್ನೂ ಮರೆತು ಆನಂದದಿಂದ ತಲ್ಲೀನರಾಗಿರುತ್ತಿದ್ದರು. ಆ ಸಮಯದಲ್ಲಿ ಸ್ವಾಮೀಜಿ ಬರೆದ ಪತ್ರ ವಿನೋದವಾಗಿದೆ: 

 “ಈಗ ಮಳೆ ಸಮಾಚಾರ- ಚೆನ್ನಾಗಿ ಮಳೆ ಆಗಿದೆ. ಹಗಲೂ ರಾತ್ರಿ ಒಂದು ಪ್ರಳಯ, ಮಳೆ, ಮಳೆ, ಮಳೆ. ನದಿ ದಡಮೀರಿ ಹರಿಯುತ್ತಿದೆ. ಕೊಳ ಕಟ್ಟೆ ತುಂಬಿ ತುಳುಕಾಡುತ್ತಿದೆ. ಈಗ ತಾನೆ ಮಠದ ಆವರಣದೊಳಗೆ ನಿಂತ ನೀರನ್ನು ಹೊರಗೆ ಬಿಡಲು ಒಂದು ಚರಂಡಿ ಮಾಡುವುದಕ್ಕೆ ನಾನೂ ಸಹಾಯ ಮಾಡಿ ಬಂದಿರುವೆನು. ಮಳೆಯ ನೀರು ಹಲವು ಕಡೆ ಕೆಲವು ಅಡಿಗಳಷ್ಟು ನಿಂತಿರುವುದು. ನನ್ನ ದೊಡ್ಡ ಕೊಕ್ಕರೆಗಂತೂ ಪರಮಾನಂದ. ಅದರಂತೆಯೇ ಬಾತಿಗೂ ಕಾಡ. ನನ್ನ ಜಿಂಕೆಯ ಮರಿ ಮಠದಿಂದ ತಪ್ಪಿಸಿಕೊಂಡು ಹೋಗಿ ಅದನ್ನು ಹುಡುಕುವುದಕ್ಕೆ ತುಂಬಾ ಕಾತರನಾಗಬೇಕಾಯಿತು. ದುರದೃಷ್ಟವಶಾತ್ ನಿನ್ನೆ ನನ್ನ ಬಾತೊಂದು ಸತ್ತಿತು. ಒಂದು ವಾರಕ್ಕಿಂತ ಹೆಚ್ಚಾಗಿ ಅದು ಉಸಿರಿಗೆ ಒದ್ದಾಡುತ್ತಿತ್ತು. ನನ್ನ ವಿನೋದಪ್ರಿಯ ಸ್ವಾಮಿಗಳೊಬ್ಬರು ‘ಸ್ವಾಮೀಜಿ, ಬಾತಿಗೆ ಚಳಿ ಹಿಡಿಯುವುದು. ಮಳೆಯಿಂದ ನೆಗಡಿ ಬರುವುದು, ಕಪ್ಪೆ ಸೀನುವುದು ಇವೆಲ್ಲ ಆದರೆ ಈ ಕಲಿಯುಗದಲ್ಲಿ ಬಾಳಿ ಪ್ರಯೋಜನವಿಲ್ಲ!’ ಎನ್ನುವರು.” 

 “ನನ್ನ ಒಂದು ಬಾತಿಗೆ ಪುಕ್ಕಗಳು ಉದುರಿ ಹೋಗುತ್ತಿದ್ದುವು. ಮತ್ತೆ ಯಾವದಾರಿಯೂ ಕಾಣದೆ ಒಂದು ಟಬ್ಬು ನೀರಿನಲ್ಲಿ ಕಾರ್ಬಾಲಿಕ್ ಆಸಿಡ್ ಹಾಕಿ, ಗುಣವಾಗಲಿ, ಇಲ್ಲದೇ ಇದ್ದರೆ ಸಾಯಲಿ ಎಂದು ಅದನ್ನು ಅದರಲ್ಲಿ ಬಿಟ್ಟೆ. ಈಗ ಅದು ಗುಣಮುಖವಾಗಿದೆ.” 

 ಒಂದು ದಿನ ಮಠದ ನಾಯಿ ಭಾಗ ಏನೋ ತಪ್ಪು ಮಾಡಿದುದಕ್ಕಾಗಿ ದೋಣಿಯಲ್ಲಿ ಅದನ್ನು ಗಂಗಾನದಿಯ ಆಚೆ ಬಿಟ್ಟು ಬಂದರು. ಆದರೆ ಆ ನಾಯಿ ಅಲ್ಲಿಂದ ಬರುವ ಒಂದು ದೋಣಿಯೊಳಗೆ ಹೋಗಿ ಕುಳಿತುಕೊಂಡಿತು. ಯಾರು ಅದನ್ನು ಆಚೆಗೆ ಅಟ್ಟುವುದಕ್ಕೆ ಹೋದರೂ ಹತ್ತಿರ ಬರಲು ಬಿಡದೆ ಬೊಗಳುತ್ತಿತ್ತು. ದೋಣಿಯವರು ಅದರ ತಂಟೆಗೇ ಹೋಗಲಿಲ್ಲ. ಆ ದೋಣಿ ಬೇಲೂರು ಮಠವನ್ನು ಸೇರಿದಾಗ ಅದು ಅಲ್ಲಿಂದ ಪುನಃ ತನ್ನ ಸ್ಥಾನಕ್ಕೆ ರಾತ್ರಿ ಹೋಗಿ ಕುಳಿತುಕೊಂಡಿತ್ತು. ಸ್ವಾಮೀಜಿ ಮಾರನೆ ದಿನ ಬೆಳಿಗ್ಗೆ ಅಷ್ಟು ಹೊತ್ತಿಗೆ ಎದ್ದು ಬಂದಾಗ ಭಾಗ ತನ್ನ ಹಿಂದಿನ ಸ್ಥಾನದಲ್ಲಿ ಹಾಜರ್ ಆಗಿತ್ತು. ಸ್ವಾಮೀಜಿ ಅದನ್ನು ಪ್ರೀತಿಯಿಂದ ಮೈದಡವುತ್ತ “ಇನ್ನು ಮೇಲೆ ನಿನ್ನನ್ನು ಹೊರಗೆ ಕಳುಹಿಸುವುದಿಲ್ಲ” ಎಂದು ಭರವಸೆ ಕೊಟ್ಟರು. 

 ಸ್ವಾಮೀಜಿ ದೇಹ ಕ್ರಮೇಣ ದುರ್ಬಲವಾಗುತ್ತಿದ್ದರೂ ಅವರ ಮನಸ್ಸು ಮಾತ್ರ ಯಾವಾಗಲೂ ಪ್ರಪುಲ್ಲವಾಗಿತ್ತು. ತಮ್ಮ ಸಹಜ ಸ್ಥಿತಿಯಲ್ಲಿ ಆನಂದದಲ್ಲಿದ್ದರು. ಯಾವ ಸಮಯದಲ್ಲಿ ಬೇಕಾದರೂ ಅವರು ಪ್ರಪಂಚದಿಂದ ಹೋಗುವುದಕ್ಕೆ ಸಿದ್ಧವಾಗಿದ್ದರು. “ದೇವರಿಗೆ ಧನ್ಯವಾದ. ಸದ್ಯಕ್ಕೆ ಈ ಜೀವನ ಶಾಶ್ವತವಲ್ಲ!” ಎನ್ನುತ್ತಿದ್ದರು. 

 ಸ್ವಾಮೀಜಿಯವರು ಸ್ಥಾಪಿಸಿದ ಬೇಲೂರು ಮಠದ ಸಂನ್ಯಾಸಿಗಳು ಮತ್ತು ಅವರು ಮಾಡುವ ಕೆಲಸ ಅನೇಕ ಪೂರ್ವಾಚಾರಪರಾಯಣರಿಗೆ ಹಿಡಿಸುತ್ತಿರಲಿಲ್ಲ. ಸಂನ್ಯಾಸವೆಂದರೆ ಕೆಲವು ಅಭಿಪ್ರಾಯಗಳು ಬಂದುಹೋಗಿವೆ. ಪಾದಪೂಜೆ, ಪಲ್ಲಕ್ಕಿ, ಮೆರವಣಿಗೆ, ಸಹಸ್ರಾರು ಜನ ಮಾಡುವ ನಮಸ್ಕಾರಗಳನ್ನು ಸ್ವೀಕರಿಸುವುದು ಇವುಗಳೆಲ್ಲ ಮಾಮೂಲು. ಒಬ್ಬ ಸಂನ್ಯಾಸಿ ಆಸ್ಪತ್ರೆಯಲ್ಲಿ ಕೆಲಸಮಾಡುತ್ತಿದ್ದರೆ, ಸ್ಕೂಲು ಅಥವಾ ಅನಾಥಾಲಯ ನಡೆಸುತ್ತಿದ್ದರೆ, ಪ್ಲೇಗು ಬರಗಾಲ ಮತ್ತು ಪ್ರವಾಹದ ಹಾವಳಿಯಲ್ಲಿ ಸಿಕ್ಕಿದವರ ಉಪಶಮನಕ್ಕೆ ಕೆಲಸ ಮಾಡುತ್ತಿದ್ದರೆ, ಅವರ ದೃಷ್ಟಿಯಲ್ಲಿ ಅವನು ಪತಿತನು! ಆದರೆ ಸ್ವಾಮಿ ವಿವೇಕಾನಂದರು ಜಾರಿಗೆ ತಂದ ಸಂನ್ಯಾಸ ಹೊಸ ಬಗೆಯದು. ಆತ ಧ್ಯಾನ ಅಧ್ಯಯನದಲ್ಲಿ ಹೇಗೆ ನಿಪುಣನೋ ಅದರಂತೆಯೇ ಕರ್ಮದಲ್ಲಿಯೂ ನಿಪುಣ. ಕರ್ಮಮಾಡುವಾಗ ಒಂದು ಕೆಲಸ ಮೇಲಲ್ಲ, ಮತ್ತೊಂದು ಕೆಲಸ ಕೀಳಲ್ಲ. ಕರ್ಮ ಮಾಡುವುದರ ಹಿಂದೆ ಇರುವ ದೃಷ್ಟಿ ಮುಖ್ಯ. ಆದರೆ ಸಂಪ್ರದಾಯಸ್ಥರು ಆ ಭಾವವನ್ನು ತಕ್ಷಣ ಮೆಚ್ಚುವುದಿಲ್ಲ. ಅನೇಕ ಜನ ಸಾಧುಗಳನ್ನು ಟೀಕಿಸಲು ಮೊದಲುಮಾಡಿದರು. ಸ್ವಾಮೀಜಿಯವರಿಗೆ ಇದು ಗೊತ್ತಾಯಿತು. ಆದರೆ ಅವರು ಇದಕ್ಕೆ ಅಂಜಿ ಕುಗ್ಗಿಹೋಗುವವರಲ್ಲ. ಆನೆಯೊಂದು ಬೀದಿಯಲ್ಲಿ ಹೋಗುತ್ತಿದ್ದರೆ ದಾರಿಯಲ್ಲಿ ಎಷ್ಟೋ ನಾಯಿಗಳು ಬೊಗಳುತ್ತವೆ. ಅದರಂತೆಯೇ ಟೀಕಿಸುವವರು. ನಾವು ನಿಧಾನವಾಗಿ ಪರಿಶುದ್ಧ ಹೃದಯದಿಂದ ಕೆಲಸವನ್ನು ಮಾಡಿಕೊಂಡು ಹೋದರೆ ಈಗ ಟೀಕಿಸುವವರೇ ನಾಳೆ ಕೊಂಡಾಡುವರು. ಶ‍್ರೀಕೃಷ್ಣನೇ ಗೀತೆಯಲ್ಲಿ ನನ್ನ ಭಕ್ತ ಎಂದಿಗೂ ನಾಶವಾಗುವುದಿಲ್ಲ ಎಂದು ಭರವಸೆಯನ್ನು ಕೊಡುವನು. ಟೀಕೆ ಇಲ್ಲದ ಕಾಲವೇ ಇಲ್ಲ. ಆದರೆ ಕರ್ಮಯೋಗಿ ಅದನ್ನು ಗಣನೆಗೆ ತರುವುದಿಲ್ಲ. ನಿಂದಾಸ್ತುತಿಗಳನ್ನು ಒಂದೇ ಸಮನಾಗಿ ನೋಡುತ್ತ ತನ್ನ ಪಾಲಿಗೆ ಬಂದ ಕರ್ತವ್ಯವನ್ನು ಮಾಡಿಕೊಂಡು ಹೋಗುವುದೇ ಅವನ ಗುರಿ. 

 ಸ್ವಾಮೀಜಿಯವರು ಬೇಲೂರು ಮಠದಲ್ಲಿ ಶಾಸ್ತ್ರೀಯವಾಗಿ ದುರ್ಗಾಪೂಜೆಯನ್ನು ಮಾಡಿದರು. ಅದಕ್ಕೆ ಶ‍್ರೀರಾಮಕೃಷ್ಣಾನಂದ ಸ್ವಾಮೀಜಿಗಳ ತಂದೆಯಾದ ಈಶ್ವರಚಂದ್ರ ಭಟ್ಟಾಚಾರ‍್ಯ ಅವರನ್ನೇ ಪುರೋಹಿತರನ್ನಾಗಿ ಕರೆಸಿದರು. ಈ ಪೂಜೆಯಲ್ಲಿ ಭಾಗಿಗಳಾಗುವುದಕ್ಕೆ ಸಹಸ್ರಾರು ಜನ ಬಂದರು. ಅದರಂತೆಯೇ ಲಕ್ಷ್ಮಿ ಮತ್ತು ಸರಸ್ವತೀ ಪೂಜೆಗಳನ್ನು ಕೂಡ ಶಾಸ್ತ್ರೀಯವಾಗಿ ಮಾಡಿದರು. ಹಾಗೆ ಪ್ರತಿ ವರುಷವೂ ಮಾಡುವ ಅಭ್ಯಾಸವನ್ನು ಕೂಡ ಆಚರಣೆ ತಂದರು. 

 ಸ್ವಾಮೀಜಿಯವರ ತಾಯಿ ಹಲವು ವರುಷಗಳ ಹಿಂದೆ ಕಲ್ಕತ್ತೆಯಲ್ಲಿ ಕಾಳೀಘಾಟಿನಲ್ಲಿರುವ ಕಾಳಿಕಾಮಾತೆಗೆ ಒಮ್ಮೆ ಸ್ವಾಮೀಜಿ ಪರವಾಗಿ ಹರಸಿಕೊಂಡಿದ್ದರು. ಅದರಲ್ಲಿ ಸ್ವಾಮೀಜಿ ಗಂಗಾನದಿಯಲ್ಲಿ ಮಿಂದು ದೇವಸ್ಥಾನದ ಎದುರಿಗೆ ಮೂರು ಬಾರಿ ನೆಲದ ಮೇಲೆ ಹೊರಳುತ್ತಾನೆ ಎಂದು ಪ್ರಾರ್ಥಿಸಿಕೊಂಡಿದ್ದರು. ತಾಯಿ ಸ್ವಾಮೀಜಿಗೆ ಆ ಹರಕೆಯನ್ನು ಜ್ಞಾಪಕಕ್ಕೆ ತಂದಾಗ ಸ್ವಾಮೀಜಿ ಗಂಗೆಯಲ್ಲಿ ಮಿಂದು ಒದ್ದೆ ಬಟ್ಟೆಯಲ್ಲಿ ಬಂದು ಕಾಳಿಕಾಮೂರ್ತಿಯ ಮುಂದೆ ಮೂರು ಬಾರಿ ಹೊರಳಾಡಿದರು. ದೇವಸ್ಥಾನದ ಸುತ್ತಲೂ ಏಳು ಬಾರಿ ಪ್ರದಕ್ಷಿಣೆ ಬಂದರು. ಗುಡಿಯವರು ಸ್ವಾಮೀಜಿಯವರನ್ನು ಗರ್ಭಗುಡಿ ಒಳಗೆ ಬಿಟ್ಟು ತಮಗೆ ಬಂದ ರೀತಿಯಲ್ಲಿ ದೇವಿಯನ್ನು ಪೂಜಿಸುವುದಕ್ಕೆ ಅವಕಾಶ ಕೊಟ್ಟರು. ಅನಂತರ ನಾಟ್ಯಮಂದಿರದಲ್ಲಿ ಒಂದು ಹೋಮವನ್ನು ಕೂಡ ಮಾಡಿದರು. ಸ್ವಾಮಿ ವಿವೇಕಾನಂದರು ಎಷ್ಟು ದೊಡ್ಡ ಜ್ಞಾನ ಯೋಗಿಗಳೋ ಅಷ್ಟೇ ದೊಡ್ಡ ಶಕ್ತಿ ಉಪಾಸಕರಾಗಿದ್ದರು. ವಿಗ್ರಹಗಳನ್ನು ದೂರಲಿಲ್ಲ. ಅದು ಅನೇಕರಿಗೆ ಸಹಕಾರಿ. ತಾವು ಬಾಲ್ಯದಲ್ಲಿ ಒಮ್ಮೆ ಹಾಗೆ ಮಾಡಿದುದಕ್ಕಾಗಿ ಯಾರು ವಿಗ್ರಹಾರಾಧನೆಯಿಂದಲೇ ಆಧ್ಯಾತ್ಮಿಕ ಜೀವನದ ಸರ್ವಸ್ವವನ್ನು ಪಡೆದರೊ ಅವರ ಸೇವಕನಾಗಬೇಕಾಗಿ ಬಂತು ಎಂದು ಹೇಳಿದರು. 

 ೧೯೦೧ರಲ್ಲಿ ಸ್ವಾಮೀಜಿಯವರ ಆರೋಗ್ಯ ಸ್ಥಿತಿ ತುಂಬಾ ಹದಗೆಟ್ಟಿತು. ಕಲ್ಕತ್ತೆಯ ಪ್ರಖ್ಯಾತ ವೈದ್ಯರಾದ ಡಾಕ್ಟರ್ ಸಾಂಡರ‍್ಸ ಎಂಬುವರು ಸ್ವಾಮೀಜಿಯವರು ಯಾವ ವಿಧವಾದ ಭೌತಿಕ ಕ್ರಿಯೆಯನ್ನೂ ಮಾಡಕೂಡದೆಂದು ಕಟ್ಟಪ್ಪಣೆ ಮಾಡಿದರು. ಭಕ್ತಾದಿಗಳು ಅವರೊಡನೆ ಮಾತನಾಡುವಾಗಲೂ ಬಹಳ ಗಂಭೀರವಾದ ವಿಷಯಗಳನ್ನು ಮಾತನಾಡಕೂಡದೆಂದು ಕಟ್ಟಪ್ಪಣೆ ಮಾಡಿದರು. ಸ್ವಾಮೀಜಿ ಸ್ವಲ್ಪ ಗುಣಮುಖರಾದಾಗ ನೆಲ ಅಗೆಯುವುದು ಗಿಡ ನೆಡುವುದು ಮುಂತಾದ ಕೆಲಸವನ್ನು ಮಾಡುತ್ತಿದ್ದರು. 

 ಶ‍್ರೀರಾಮಕೃಷ್ಣರ ಅಸ್ಥಿಯನ್ನು ಶ‍್ರೀರಾಮಕೃಷ್ಣರ ವ್ಯಕ್ತಿತ್ವವೇ ಅದರಲ್ಲಿದೆ ಎಂದು ಪೂಜ್ಯದೃಷ್ಟಿಯಿಂದ ನೋಡುತ್ತಿದ್ದರು. ಒಂದು ಸಲ ಸ್ವಾಮೀಜಿ ಮನಸ್ಸಿನಲ್ಲಿ ಅನುಮಾನ ಬಂದಿತು. ‘ನಿಜವಾಗಿಯೂ ಈ ಅವಶೇಷದಲ್ಲಿ ಶ‍್ರೀರಾಮಕೃಷ್ಣರು ಇದ್ದರೆ ಕಲ್ಕತ್ತೆಗೆ ಬಂದಿರುವ ಗ್ವಾಲಿಯರ್ ಮಹಾರಾಜರು ಇಲ್ಲಿಗೆ ಬರಲಿ’ ಎಂದು ಪ್ರಾರ್ಥಿಸಿಕೊಂಡರು. ಮಾರನೆ ದಿನ ಸ್ವಾಮೀಜಿ ಕಲ್ಕತ್ತೆಗೆ ಹೋಗಿ ಬೇಲೂರು ಮಠಕ್ಕೆ ಹಿಂತಿರುಗಿದಾಗ ಗ್ವಾಲಿಯರ್ ಮಹಾರಾಜರು ತಮ್ಮ ತಮ್ಮನನ್ನು ಕಳುಹಿಸಿ ಸ್ವಾಮೀಜಿಯವರನ್ನು ಯಾವಾಗ ನೋಡಲು ಸಾಧ್ಯ ಎಂದು ವಿಚಾರಿಸಿಕೊಂಡು ಬಾ ಎಂದು ಹೇಳಿರುವರು ಎಂದು ಗೊತ್ತಾಯಿತು. ತಕ್ಷಣವೇ ಸ್ವಾಮೀಜಿ ದೇವಸ್ಥಾನಕ್ಕೆ ಮಗುವಿನಂತೆ ಸಂತೋಷದಿಂದ ಓಡಿಹೋಗಿ ನಮಸ್ಕಾರ ಮಾಡಿದರು. ಅಲ್ಲಿ ಆಗ ಪ್ರೇಮಾನಂದರು ಸ್ವಾಮೀಜಿ ಹಾಗೆ ನಮಸ್ಕಾರ ಮಾಡುತ್ತಿರುವುದನ್ನು ನೋಡಿ ಆಶ್ಚರ್ಯಪಟ್ಟರು. ಸ್ವಾಮೀಜಿ ಅದಕ್ಕೆ ಕಾರಣವನ್ನು ಕೊಟ್ಟರು. 

 ಆ ವರ್ಷದ ಕೊನೆಯಲ್ಲಿ ಜಪಾನಿನಿಂದ ಭಿಕ್ಕು ಓಡಾ ಮತ್ತು ಓಕಾಕುರ ಎಂಬುವರು ಬಂದು ಜಪಾನಿನಲ್ಲಿ ಆಗುವ ಧರ್ಮಸಮ್ಮೇಳನಕ್ಕೆ ಬರಬೇಕೆಂದು ಸ್ವಾಮೀಜಿಯವರನ್ನು ಕೇಳಿಕೊಂಡರು. ಸ್ವಾಮೀಜಿಯವರಂತಹ ವ್ಯಕ್ತಿ ಬಂದರೆ ಮಾತ್ರ ಧಾರ್ಮಿಕ ಜಾಗ್ರತಿ ಬೇಗ ಆಗುವುದು ಸಾಧ್ಯವೆಂದೂ ಬೇಡಿಕೊಂಡರು. ಆದರೆ ಸ್ವಾಮೀಜಿ ಮಾತುಕೊಡುವ ಸ್ಥಿತಿಯಲ್ಲಿ ಇರಲಿಲ್ಲ. ಓಕಾಕುರ ಅವರು ಬುದ್ಧ ತಪಸ್ಸು ಮಾಡಿದ ಸ್ಥಳವಾದ ಗಯೆಗೆ ಸ್ವಾಮೀಜಿ ತಮ್ಮೊಂದಿಗೆ ಬರಬೇಕೆಂದು ಕೋರಿಕೊಂಡರು. ಅದರಂತೆಯೆ ಸ್ವಾಮೀಜಿ ಗಯೆಗೆ ಹೋದರು. ಅಲ್ಲಿ ಹಿಂದೂ ಮಠದ ಮಹಂತರು ಸ್ವಾಮೀಜಿಯವರನ್ನು ಬಹಳ ಆದರಿಸಿದರು. ಅಲ್ಲಿಂದ ಸ್ವಾಮೀಜಿ ಕಾಶಿಗೆ ಹೋದರು. ಓಕಾಕುರ ಅವರು ಇನ್ನು ಬೇರೆ ಬೇರೆ ಸ್ಥಳಗಳನ್ನು ನೋಡಲು ಹೋದರು. 

\newpage

 ಕಾಶಿಯಲ್ಲಿ ಸ್ವಾಮೀಜಿಯವರು ಇದ್ದಾಗ ಅನೇಕ ಸಾಧುಸಂತರು ಮತ್ತು ವಿದ್ವಾಂಸರು ಅವರನ್ನು ನೋಡಲು ಬಂದರು. ಭಿಂಗ ಮಹಾರಾಜರು ಕಾಶಿಯಲ್ಲಿ ಒಂದು ಆಶ್ರಮವನ್ನು ಸ್ಥಾಪಿಸಬೇಕೆಂದೂ ಅದಕ್ಕೆ ತಮ್ಮ ಕೈಲಾದ ಸಹಾಯವನ್ನು ಮಾಡುತ್ತೇವೆಂದೂ ಹೇಳಿದರು. ಸ್ವಾಮೀಜಿಯವರು ಕಲ್ಕತ್ತೆಗೆ ಹೋಗಿ ಸ್ವಾಮಿ ಶಿವಾನಂದರನ್ನು ಅಲ್ಲಿ ಒಂದು ಆಶ್ರಮವನ್ನು ತೆರೆಯುವುದಕ್ಕಾಗಿ ಕಳುಹಿಸಿದರು. ಸ್ವಾಮೀಜಿಯವರ ಸೇವಾಭಾವನೆಯಿಂದ ಸ್ಫೂರ್ತಿಗೊಂಡ ಕೆಲವು ಬಂಗಾಳಿ ಹುಡುಗರು ಒಂದು ಸಣ್ಣ ಮನೆಯನ್ನು ಬಾಡಿಗೆಗೆ ತೆಗೆದುಕೊಂಡು ಅಲ್ಲಿ ಒಂದು ಸೇವಾಶ್ರಮವನ್ನು ಸ್ಥಾಪಿಸಿದರು. ಕಾಶಿಗೆ ಬರುವ ರೋಗಿ ಮತ್ತು ಅನಾಥರ ಸೇವೆಗಾಗಿ ಇದು ಮುಡುಪಾಯಿತು. ಸ್ವಾಮೀಜಿ ಈ ಸೇವಾಶ್ರಮದ ವಾರ್ಷಿಕವರದಿಯಲ್ಲಿ ಸಾರ್ವಜನಿಕರಿಗೆ ಒಂದು ಮನವಿಮಾಡಿಕೊಂಡು ಹೀಗೆ ಹೇಳಿರುವರು: 

 “ಬೇರೆ ತೀರ್ಥಸ್ಥಳಗಳಿಗೆ ಜನ ತಮ್ಮ ಪಾಪವನ್ನು ಕಳೆದುಕೊಳ್ಳಲು ಬರುವರು. ಅಂತಹ ಸ್ಥಳಗಳಲ್ಲಿ ಅವರು ಎಲ್ಲೋ ಕೆಲವು ದಿನಗಳು ಮಾತ್ರ ಇರುವರು. ಬಹಳ ಪುರಾತನವಾದ, ಭಾರತೀಯರ ಧಾರ್ಮಿಕ ಚಳುವಳಿಯ ಸಜೀವಕೇಂದ್ರವಾದ ಈ ನಗರಿಗಾದರೋ ಜನರು ತಮ್ಮ ಕೊನೆಗಾಲದಲ್ಲಿ ಬರುತ್ತಾರೆ. ವಿಶ್ವನಾಥ ದೇವಾಲಯದ ಛಾಯೆಯ ಆಶ್ರಮದಲ್ಲಿ ಮೃತ್ಯುವಿನ ಮೂಲಕ ಮುಕ್ತಿಯನ್ನು ಕಾಣುವವರೆಗೆ ಇಲ್ಲಿರುವರು. 

 “ಇಲ್ಲಿ ಸಾಯುವುದಕ್ಕೆ ಬರುವ ಬಡಜನರಿಗೆ ಅವರು ತಾವು ತಮ್ಮ ಊರಿನಲ್ಲಿ ಇದ್ದಿದ್ದರೆ ಯಾವ ಸಹಾಯ ಒದಗುತ್ತಿತ್ತೋ ಅದು ಸಿಕ್ಕುವಂತೆ ಇಲ್ಲ. ಇಂತಹ ಜನರು ರೋಗಕ್ಕೆ ತುತ್ತಾಗುತ್ತಾರೆ. ಹಿಂದೂಗಳಾದ ನೀವೇ ಅದನ್ನು ಊಹಿಸಿ ಅನುಭವಿಸಿ ಶಮನಕ್ಕೆ ಪ್ರಯತ್ನ ಮಾಡಬೇಕಾಗುವುದು.” 

 “ಸಹೋದರರೆ, ನಮ್ಮ ಅಂತ್ಯಕಾಲಕ್ಕೆ ಅಣಿಯಾಗುವ ಈ ಪವಿತ್ರ ಸ್ಥಳ ಎಷ್ಟೊಂದು ಜನರನ್ನು ಹೇಗೆ ಆಕರ್ಷಿಸುತ್ತಿದೆ ಎಂಬ ವಿಷಯವನ್ನು ನೀವು ಸ್ವಲ್ಪ ವಿಚಾರ ಮಾಡಬೇಡವೆ? ಅನಾದಿಕಾಲದಿಂದ ಅವಿಚ್ಛಿನ್ನವಾಗಿ ಮೃತ್ಯುವಿನ ಮೂಲಕ ಮುಕ್ತಿಯನ್ನು ಪಡೆಯುವುದಕ್ಕೆ ಈ ನಗರಿಗೆ ಬರುವ ಜನರನ್ನು ನೋಡಿದರೆ ನಿಮ್ಮಲ್ಲಿ ಭಯ ಭಕ್ತಿಭಾವ ಮೂಡುವುದಿಲ್ಲವೆ? ಹಾಗೆ ನಿಮಗೆ ಅನ್ನಿಸಿದರೆ ಬಂದು ನಮಗೆ ಸಹಾಯ ನೀಡಿ. ನೀವು ನೀಡುವುದು ಅತ್ಯಲ್ಪವಾದರೂ ಚಿಂತೆಯಿಲ್ಲ. ಮಾಡುವ ಸಹಾಯ ಎಷ್ಟೇ ಅಲ್ಪವಾದರೂ ಚಿಂತೆಯಿಲ್ಲ. ಹಲವು ಹುಲ್ಲುಗಳಿಂದ ಒಂದು ಹಗ್ಗವನ್ನು ಮಾಡಿದರೆ ಒಂದು ಮದಿಸಿದ ಆನೆಯನ್ನೆ ಕಟ್ಟಿಹಾಕಬಹುದು ಎಂಬ ಗಾದೆ ಇದೆ.” 

