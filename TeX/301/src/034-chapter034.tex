
\chapter{ಮದ್ರಾಸಿನಲ್ಲಿ }

 ಹಲವು ದಿನಗಳಿಂದ ಮದ್ರಾಸಿನ ನಗರದಲ್ಲಿ ಸ್ವಾಮೀಜಿಯವರ ಸ್ವಾಗತಕ್ಕೆ ಸಿದ್ಧತೆಗಳಾಗುತ್ತಿತ್ತು. ಮದ್ರಾಸಿನ ಜನರು ಮೊದಲು ಸ್ವಾಮಿ ವಿವೇಕಾನಂದರನ್ನು ಕಂಡು ಹಿಡಿದರು ಎಂದು ಬೇಕಾದರೂ ಹೇಳಬಹುದು. ಅವರಲ್ಲಿ ಎಂತಹ ವಿದ್ವತ್ ಇದೆ, ತಪಸ್ ಇದೆ, ಸಾಹಸಗಳಿವೆ ಎಂಬುದನ್ನು ಮೊದಲು ಮದ್ರಾಸಿನ ಶಿಷ್ಯರು ಕಂಡರು. ಅವರನ್ನು ಅಮೇರಿಕಾ ದೇಶಕ್ಕೆ ಕಳುಹಿಸುವುದಕ್ಕೆ ಮನೆಮನೆಗೆ ಭಿಕ್ಷೆ ಬೇಡಿ ಹಣವನ್ನು ವಸೂಲಿ ಮಾಡಿದವರು ಮದ್ರಾಸಿನವರು. ಅನಂತರ ಸ್ವಾಮಿ ವಿವೇಕಾನಂದರು ಪಾಶ್ಚಾತ್ಯ ದೇಶಗಳಲ್ಲಿ ಕೀರ್ತಿಯನ್ನು ಗಳಿಸಿದ ಮೇಲೆ ಪತ್ರಗಳ ಮೂಲಕ ಬೆಂಕಿಯ ಭಾಷೆಯಲ್ಲಿ ಮದ್ರಾಸಿನ ಶಿಷ್ಯರನ್ನು ಹುರಿದುಂಬಿಸಲು ಯತ್ನಿಸುತ್ತಿದ್ದರು. ಈಗ ಅಂತಹ ಸ್ವಾಮಿಗಳೇ ಬರುತ್ತಿರುವಾಗ ಜನರ ಆನಂದಕ್ಕೆ ಮೇರೆಯಿರಲಿಲ್ಲ. ಮದ್ರಾಸಿನ ಜನರು ದೊಡ್ಡ ಒಂದು ಸ್ವಾಗತಸಮಿತಿಯನ್ನು ಮಾಡಿಕೊಂಡರು. ಅದರಲ್ಲಿ ಮದ್ರಾಸಿನ ಪ್ರಮುಖರೆಲ್ಲರೂ ಇದ್ದರು. ರೈಲ್ವೆ ನಿಲ್ದಾಣದಿಂದ ಸ್ವಾಮೀಜಿಯವರನ್ನು ಇಳಿಸುವುದಕ್ಕಾಗಿ ಗೊತ್ತುಮಾಡಿದ ಸಮುದ್ರತೀರದಲ್ಲಿರುವ \enginline{Castle Kernan House} ವರೆಗೆ ೧೭ ಕಮಾನುಗಳನ್ನು ಅಲಂಕರಿಸಿದರು. ಪ್ರತಿಯೊಂದು ಕಮಾನಿನ ಮೇಲೂ \enginline{“Long live the Venerable Vivekananda” “Hail Servant of God:” “Hail Servant of the Great Sage of the East” “Tha Awakened India’s Hearty Greetings to Vivekananda” “Hail Harbinger of Peace” “Hail Sri RamaKrishna’s worthy son” “Wel–com Prince of man”} ಎಂಬ ಲಿಖಿತಗಳು ರಾರಾಜಿಸುತ್ತಿದ್ದವು. ಎಲ್ಲಿ ನೋಡಿದರೂ ಬಾವುಟಗಳು ಮಂಟಪಗಳು ತಳಿರುತೋರಣಗಳು. ಇಡೀ ಊರಿಗೆ ಊರೇ ಸ್ವಾಮೀಜಿಯವರನ್ನು ಸ್ವಾಗತಿಸುವುದಕ್ಕೆ ಎದ್ದಂತೆ ಇತ್ತು. ಯಾರ ಬಾಯಲ್ಲಿ ನೋಡಿದರೂ ಅವರ ಮಾತೆ. ಶಾಲಾ ಕಾಲೇಜುಗಳಲ್ಲಿ ಕೋರ್ಟು ಕಛೇರಿಗಳಲ್ಲಿ ಬೀದಿಗಳಲ್ಲಿ ಸಮುದ್ರ ತೀರದಲ್ಲಿ ಎಲ್ಲಿ ಯಾರನ್ನು ಸಂಧಿಸಿದರೂ ಸ್ವಾಮಿ ವಿವೇಕಾನಂದರು ಎಂದು ಬರುತ್ತಾರೆ ಎಂಬುದೇ. ಪ್ರೌಢ ಪರೀಕ್ಷೆಗಳಿಗಾಗಿ ದೂರದೇಶದಿಂದ ಮದ್ರಾಸಿಗೆ ಬಂದ ವಿದ್ಯಾರ್ಥಿಗಳು ಪರೀಕ್ಷೆ ಮುಗಿದು ಬಹಳ ದಿನಗಳಾದರೂ ಸ್ವಾಮಿಗಳ ಆಗಮನವನ್ನು ನಿರೀಕ್ಷಿಸುತ್ತ ಅಲ್ಲಿಯೇ ಇದ್ದರು. 

\newpage

 ಫೆಬ್ರವರಿ ಆರನೇ ತಾರೀಖಿನ ದಿನ ಬಂತು. ಅಂದು ಬೆಳಿಗ್ಗೆ ಸ್ವಾಮೀಜಿ ಎಗ್‍ಮೋರ್ ಸ್ಟೇಷನ್ನಿನಲ್ಲಿ ಇಳಿಯುವರು. ರೈಲು ಬರುವುದಕ್ಕೆ ಕೆಲವು ಗಂಟೆಗಳ ಮುಂಚೆಯೇ ರೈಲ್ವೆ ನಿಲ್ದಾಣ ಸಹಸ್ರಾರು ಜನರಿಂದ ಕಿಕ್ಕಿರಿದುಹೋಗಿದ್ದಿತು. ರೈಲ್ವೆ ಸ್ಟೇಷನ್ನಿನ ಹೊರಗಡೆ ಸಹಸ್ರಾರು ಜನ ಸ್ವಾಮೀಜಿ ದರ್ಶನಕ್ಕೆ ಹಾತೊರೆಯುತ್ತಿದ್ದರು. ಸ್ವಾಮೀಜಿಯವರನ್ನು ಹೊತ್ತ ರೈಲು ಪ್ಲಾಟ್‍ಫಾರಂ ಪ್ರವೇಶಿಸಿತು. ಅವರನ್ನು ಸ್ವಾಗತ ಸಮಿತಿಯವರು ಕಂಡಕೂಡಲೆ ಅವರಿಗೆ ಹೂವಿನ ಹಾರವನ್ನು ಹಾಕಿದರು. ಸಹಸ್ರಾರು ಜನರು ಕರತಾಡನ ಮಾಡಿದರು. ಜಯಕಾರ ಮಾಡಿದರು, ಕಿವಿ ಕಿವುಡಾಗುವಂತೆ ಇತ್ತು ಧ್ವನಿ. ಇಂತಹ ಸಂಭ್ರಮದ ಸ್ವಾಗತವನ್ನು ಮದ್ರಾಸು ಇದುವರೆಗೂ ಯಾರಿಗೂ ಕೊಟ್ಟಿರಲಿಲ್ಲ. ಸ್ವಾಮೀಜಿಯವರನ್ನು ಸುಂದರವಾಗಿ ಅಲಂಕರಿಸಿದ ಒಂದು ಕುದುರೆಯ ಗಾಡಿಯಲ್ಲಿ ಮೆರವಣಿಗೆಯಲ್ಲಿ ಕರೆದುಕೊಂಡು ಹೋದರು. ಸ್ವಾಮೀಜಿಯವರ ಗುರುಭಾಯಿಗಳಾದ ನಿರಂಜನಾನಂದ ಮತ್ತು ಶಿವಾನಂದರು ಅವರ ಪಕ್ಕದಲ್ಲಿ ಕುಳಿತರು. ಸ್ವಲ್ಪ ದೂರ ಗಾಡಿ ಹೋದಮೇಲೆ ಕುದುರೆಯನ್ನು ಬಿಡಿಸಿ ಜನರೇ ಅದನ್ನು ಎಳೆದರು. ಹಾಗೆ ಆಗ ಎಳೆದವರಲ್ಲಿ ನಿವೃತ್ತ ಪ್ರಥಮ ಗೌರ್ನರ್ ಜನರಲ್ ಶ‍್ರೀ ಸಿ. ರಾಜಗೋಪಾಲಾಚಾರಿಯವರೂ ಒಬ್ಬರು. ಒಂದು ಸಲ ಮದ್ರಾಸಿನಲ್ಲಿ ಮಾತನಾಡುತ್ತ ಇದ್ದಾಗ ತಾವು ಆ ರಥವನ್ನು ಎಳೆದವರಲ್ಲಿ ಒಬ್ಬರು ಎಂದು ಸಂತೋಷದಿಂದ ಹೇಳಿಕೊಂಡರು. ಸ್ವಾಮೀಜಿ ಗಾಡಿಯಲ್ಲಿ ಹೋಗುತ್ತಿದ್ದಾಗ ಜನರು ಅವರಿಗೆ ತೋರುವ ಗೌರವ, ಅರ್ಪಿಸುವ ಹೂವಿನ ಹಾರಗಳು ಅವುಗಳನ್ನೆಲ್ಲಾ ನಮ್ರತೆಯಿಂದ ಸ್ವೀಕರಿಸಿದರು. ಗಾಡಿ ಕ್ಯಾಸಲ್ ಕರ್ನನ್ ಕಡೆ ಹೋಯಿತು. ದಾರಿಯ ಇಕ್ಕೆಲದಲ್ಲಿ ಹಸಹ್ರಾರು ಮಂದಿ ಅವರನ್ನು ನೋಡುತ್ತಿದ್ದಾಗ ಒಬ್ಬ ಸಂನ್ಯಾಸಿ ಸ್ವಾಮೀಜಿ ಕಣ್ಣಿಗೆ ಬಿದ್ದರು. ಅವರೇ ಸದಾನಂದರು. ಹಿಂದೆ ಸ್ವಾಮೀಜಿಯವರಿಗೆ ಹತ್ರಾಸ್ ರೈಲ್ವೆ ನಿಲ್ದಾಣದಲ್ಲಿ ಸ್ವಾಗತವನ್ನು ಬಯಸಿದವರು, ಅನಂತರ ಅವರ ಪ್ರಥಮ ಶಿಷ್ಯರಾದರು. ಅವರನ್ನು ಕಂಡೊಡನೆಯೇ ಸ್ವಾಮೀಜಿ ‘ಸದಾನಂದ’ ಎಂದು ಕರೆದು ಅವರನ್ನು ತಮ್ಮ ಪಕ್ಕದಲ್ಲಿ ಕುಳ್ಳಿರಿಸಿಕೊಂಡರು. ಅನಂತರ ಮೆರವಣಿಗೆ ಸ್ವಾಮೀಜಿ ಬಿಡಾರವನ್ನು ಸೇರಿತು. ಅವರ ಸ್ವಾಗತವನ್ನು ಮದ್ರಾಸಿನ ಪತ್ರಿಕೆಯೊಂದು ಹೀಗೆ ಬಣ್ಣಿಸುವುದು: 

 “ಸ್ವಾಮೀಜಿ ಇಂದು ಬೆಳಿಗ್ಗೆ ಎಗ್‍ಮೋರ್ ಸ್ಟೇಷನ್ನಿನಲ್ಲಿ ಬಂದು ಇಳಿಯುತ್ತಾರೆ ಎಂಬ ಸುದ್ದಿ ಎಲ್ಲರಿಗೂ ಗೊತ್ತಾಗಿದ್ದುದರಿಂದ ಎಲ್ಲಾ ದರ್ಜೆಗೆ ಸೇರಿದ ಹಿಂದೂಗಳು, ಪ್ರೈಮರಿ ಸ್ಕೂಲು ಹುಡುಗರು, ಕಾಲೇಜಿನ ದೊಡ್ಡ ಯುವಕರು, ವಕೀಲರು, ಜಡ್ಡಿಗಳು ಮತ್ತು ಎಲ್ಲಾ ಕಾರ್ಯಕ್ಷೇತ್ರಕ್ಕೆ ಸೇರಿದ ಜನರು, ಹಲವರು ಸ್ತ್ರೀಯರು ಕೂಡ, ಜಯಪ್ರದವಾಗಿ ಪಾಶ್ಚಾತ್ಯದೇಶಗಳಿಂದ ಬರುತ್ತಿರುವ ಸ್ವಾಮೀಜಿಗಳನ್ನು ಎದುರುಗೊಳ್ಳುವುದಕ್ಕೆ ರೈಲ್ವೆ ನಿಲ್ದಾಣದಲ್ಲಿ ನೆರೆದರು. ಎಗ್‍ಮೋರ್ ಸ್ಟೇಷನ್ನೇ ಮದ್ರಾಸಿನ ಮುಖ್ಯ ರೈಲ್ವೆ ನಿಲ್ದಾಣವಾಗಿದ್ದುದರಿಂದ ಸ್ವಾಗತ ಸಮಿತಿಯವರು ಅದನ್ನು ಸುಂದರವಾಗಿ ಅಲಂಕರಿಸಿದ್ದರು. ಸ್ವಾಮೀಜಿ ಗೌರವಾರ್ಥವಾಗಿ ಅದ್ಭುತವಾದ ಸ್ವಾಗತವನ್ನು ಅವರು\break ಅಣಿಮಾಡಿರುವರು. ಸ್ಟೇಷನ್ನಿನ ಪ್ಲಾಟ್‍ಫಾರಂ ಒಳಗೆ ನೆರೆದ ಜನರಿಗೆ ಅವಕಾಶವಿಲ್ಲದೇ ಇದ್ದುದರಿಂದ ಟಿಕೇಟಿನ ಮೂಲಕ ಅದು ಎಷ್ಟು ಜನರನ್ನು ಹಿಡಿಸಬಹುದೋ ಅಷ್ಟು ಜನರನ್ನು ಒಳಗೆ ಬಿಟ್ಟರು. ಸ್ಟೇಷನ್ನಿನ ಒಳಗೆ ನೆರೆದ ಜನರಲ್ಲಿ ಮದ್ರಾಸಿನ ಪುರಪ್ರಮುಖರನ್ನೆಲ್ಲ ನೋಡಬಹುದಾಗಿತ್ತು. ರೈಲು ಪ್ಲಾಟ್‍ಫಾರಂಗೆ ಬೆಳಿಗ್ಗೆ ೭–೩೦ಕ್ಕೆ ಬಂದಿತು. ಅದು ದಕ್ಷಿಣ ಪ್ಲಾಟ್‍ಫಾರಂ ನಲ್ಲಿ ನಿಂತೊಡನೆಯೆ ಜನರು ಚಪ್ಪಾಳೆ ತಟ್ಟಿ ಜಯಕಾರ ಮಾಡಿದರು. ಮಂಗಳ ವಾದ್ಯಗಳು ಘೋಷಿಸಿದುವು. ಸ್ವಾಮೀಜಿ ಗಾಡಿಯಿಂದ ಇಳಿದಮೇಲೆ ಸ್ವಾಗತಸಮಿತಿಯ ಸದಸ್ಯರು ಅವರನ್ನು ಬರಮಾಡಿಕೊಂಡರು. ಸ್ವಾಮೀಜಿಯವರೊಡನೆ ನಿರಂಜನಾನಂದ, ಶಿವಾನಂದ ಮತ್ತು ಅವರ ಶೀಘ್ರಲಿಪಿಕಾರನಾದ ಜೆ. ಜೆ. ಗುಡ್‍ವಿನ್ ಇದ್ದರು. ಸ್ವಾಮೀಜಿಯವರನ್ನು ವೇದಿಕೆಯ ಮೇಲೆ ಕರೆದುಕೊಂಡು ಹೋದಮೇಲೆ ಅಲ್ಲಿ ಒಂದು ದಿನ ಹಿಂದೆ ಬಂದಿದ್ದ ಸೇವಿಯರ್ಸ್‍‍ ದಂಪತಿಗಳು ಮತ್ತು ಕೊಲಂಬೊ ಇಂದ ಬಂದ ಸ್ವಾಮೀಜಿಯವರ ಮೆಚ್ಚುಗೆಯ ಅನುಯಾಯಿಯಾದ ಹ್ಯಾರಿಸನ್ ದಂಪತಿಗಳು ಇದ್ದರು. ಅನಂತರ ಸ್ವಾಮೀಜಿಯವರ ಪರಿವಾರದವರು ರೈಲ್ವೆ ನಿಲ್ದಾಣದ ಬಾಗಿಲಿಗೆ ಬಂದರು. ಆಗ ಜನರ ಹರ್ಷೋದ್ಗಾರದ ಜೊತೆಗೆ ಬ್ಯಾಂಡುವಾದನ ನಡೆಯುತ್ತಿತ್ತು. ಪೋರ್ಟಿಕೊ ಮುಂದೆ ಸ್ವಾಮೀಜಿಯವರನ್ನು ಪರಿಚಯ ಮಾಡಿಸಿದರು. ಅಲ್ಲಿ ನೆರೆದವರೊಡನೆ ಸ್ವಲ್ಪ ಹೊತ್ತು ಮಾತನಾಡಿದಮೇಲೆ ಹಾನರಬಲ್ ಜಸ್ಟಿಸ್ ಸುಬ್ರಮಣ್ಯ ಅಯ್ಯರ್ ಸ್ವಾಮೀಜಿ ಮತ್ತು ಅವರ ಗುರುಭಾಯಿಗಳೊಡನೆ ವಕೀಲರಾದ ಬಿಳಿಗಿರಿ ಅಯ್ಯಂಗಾರ್ ಅವರ ಮನೆಯಾದ ಕ್ಯಾಸಲ್ ಕರ್ನನ್‍ಗೆ ಹೊರಟರು. ಅಲ್ಲಿ ಸ್ವಾಮೀಜಿ ತಂಗುವರು. ಎಗ್‍ಮೋರ್ ಸ್ಟೇಷನ್ ಬಾವುಟ ತಳಿರು ತೋರಣಗಳಿಂದ ಅಲಂಕೃತವಾಗಿತ್ತು. ಹೊರಗೆ ಹೋಗುವ ದಾರಿಯಲ್ಲಿ ಒಂದು ಕಮಾನನ್ನು ಅಲಂಕರಿಸಿದ್ದರು. ಅದರಮೇಲೆ \enginline{“Welcome to Vivekananda”} ಎಂದು ಬರೆದಿತ್ತು. ಕಾಂಪೌಂಡಿನಿಂದ ಹೊರಗೆ ಬಂದ ಮೇಲೆ ಜನಸಂದಣಿ ಮತ್ತೂ ಹೆಚ್ಚಾಗುತ್ತ ಬಂದಿತು. ಜನರು ಸ್ವಾಮೀಜಿಯವರಿಗೆ ಕಾಣಿಕೆಯನ್ನು ಅರ್ಪಿಸುವುದಕ್ಕಾಗಿ ಗಾಡಿಯನ್ನು ಪದೇ ಪದೇ ನಿಲ್ಲಿಸಬೇಕಾಯಿತು. ಹಿಂದೂಗಳು ದೇವಸ್ಥಾನದಲ್ಲಿ ದೇವರಿಗೆ ಫಲಪುಷ್ಪಾದಿಗಳನ್ನು ಅರ್ಪಿಸುವುವಂತೆ ಸ್ವಾಮೀಜಿಯವರಿಗೆ ಅರ್ಪಿಸಿದರು. ದಾರಿಯ ಉದ್ದಕ್ಕೂ ಸ್ವಾಮೀಜಿ ಮೇಲೆ ಜನರು ಪುಷ್ಪವೃಷ್ಟಿಯನ್ನು ಕರೆದರು. ಸ್ಟೇಷನ್ನಿನಿಂದ ಸ್ವಾಮೀಜಿಯವರು ತಂಗುವ \enginline{Ice House} (ಕ್ಯಾಸಲ್ ಕರ್ನನ್)‌ನವರೆಗೆ ದಾರಿಯ ಉದ್ದಕ್ಕೂ ಬೇಕಾದಷ್ಟು ಸ್ವಾಗತದ ಕಮಾನುಗಳನ್ನು ಹಾಕಿದ್ದರು. ಮೆರವಣಿಗೆ ಚಿಂತಾದ್ರಿಪೇಟೆಯ ಮೂಲಕ ನೇಪಿಯರ್ ರೋಡಿನಲ್ಲಿ ಹೋಗಿ ಅನಂತರ ಗೌರ್ನಮೆಂಟ್ ಹೌಸ್ ಎದುರಿಗೆ ಇರುವ ಮೌಂಟ್‍ರೋಡ್ ಮೂಲಕ, ವಾಲಾಜ ರೋಡ್, ಚೇಪಾಕ್ ಕೊನೆಗೆ ಪೈಕ್ರಾಪ್ಟ್ ರೋಡಿನ ಮೂಲಕ ದಕ್ಷಿಣದ ಸಮುದ್ರ ಕಡೆಗೆ ಹೋಯಿತು. ಮೇಲಿನ ದಾರಿಯ ಮೆರವಣಿಗೆಯಲ್ಲಿ ಮಧ್ಯೆ ಮಧ್ಯೆ ಸ್ವಾಮೀಜಿಗೆ ನೀಡಿದ ಸ್ವಾಗತ ರಾಜರಿಗೆ ನೀಡುವ ವೈಭವದ ಸ್ವಾಗತಕ್ಕಿಂತ ಕಡಿಮೆ ಇರಲಿಲ್ಲ. ಸ್ವಾಮೀಜಿಯವರ\break ಗೌರವಾರ್ಥವಾಗಿ ಕಟ್ಟಿದ ಚಪ್ಪರಗಳು, ಅದರ ಮೇಲೆ ಬರೆದ ವಾಕ್ಯಗಳು ಸ್ಥಳೀಯ ಹಿಂದೂ ಸಮಾಜ ಸ್ವಾಮೀಜಿಯವರಿಗೆ ತೋರಿದ ಅಪೂರ್ವಗೌರವಕ್ಕೆ ಸಾಕ್ಷಿಯಾಗಿದ್ದವು. ಸ್ವಾಮೀಜಿ ಹಿಂದೂ ಧರ್ಮಕ್ಕೆ ಮಾಡಿರುವ ಸೇವೆಯನ್ನು ಎಲ್ಲರೂ ಕೊಂಡಾಡುತ್ತಿದ್ದರು. ನಗರದ ಲಾಯದ ಎದುರಿಗೆ ಸ್ವಾಮೀಜಿ ನಿಂತರು. ಅಲ್ಲಿ ಒಂದು ದೊಡ್ಡ ಚಪ್ಪರದಲ್ಲಿ ಹೂವು ಹಾರ, ಬಿನ್ನವತ್ತಳೆಯನ್ನು ಸ್ವೀಕರಿಸಿದರು. 

 “ಸ್ವಾಗತದಲ್ಲಿ ಜನ ತೋರಿದ ಅಭೂತಪೂರ್ವ ಉತ್ಸಾಹದಲ್ಲಿ ನಾವು ನಡೆದ ಒಂದು ಘಟನೆಯನ್ನು ಮರೆಯಬಾರದು. ವೃದ್ಧಳಾದ ಗೌರವ ಸ್ತ್ರೀಯೊಬ್ಬಳು ಸ್ವಾಮೀಜಿಯವರಿಗೆ ಅರ್ಪಿಸಲು ತನ್ನ ಬಡತನದ ಕಾಣಿಕೆಯನ್ನು ತಂದಿದ್ದಳು. ಕಿಕ್ಕಿರಿದ ಜನರನ್ನು ಆಕೆ ನೂಕಿಕೊಂಡು ಬಂದು ಸ್ವಾಮೀಜಿಯವರ ದರ್ಶನದಿಂದ ತಾನು ತನ್ನ ಪಾಪಗಳಿಂದ ಪಾರಾಗಬೇಕೆಂದು ಆಕೆ ಯತ್ನಿಸಿದಳು. ಅವಳ ಪಾಲಿಗೆ ಸ್ವಾಮೀಜಿ ಸಂಬಂಧಮೂರ್ತಿಯ ಅವತಾರವಾಗಿದ್ದರು. ನಾವು ಇದನ್ನು ಏತಕ್ಕೆ ಹೇಳುತ್ತೇವೆ ಎಂದರೆ, ಜನ ಎಂತಹ ಭಕ್ತಿಭಾವದಿಂದ ಸ್ವಾಮೀಜಿಯವರನ್ನು ಸ್ವಾಗತಿಸಿದರು, ಎಂಬುದನ್ನು ಇದು ತೋರುತ್ತದೆ. ಚಿಂತಾದ್ರಿಪೇಟೆ ಮತ್ತು ಬೇರೆ ಕಡೆ ಸ್ವಾಮೀಜಿಯವರಿಗೆ ಕರ್ಪೂರದಾರತಿಯನ್ನು ಬೆಳಗಿದರು. ಅವರು ತಂಗಿದ ಮನೆಯಲ್ಲಿ ಅಲ್ಲಿಯ ಸ್ತ್ರೀಯರು ಅವರನ್ನು ದೇವರಿಗೆ ಪೂಜೆ ಮಾಡುವಂತೆ ಧೂಪದೀಪಾದಿಗಳ ಮೂಲಕ ಸ್ವಾಗತಿಸಿದರು. ಜನರು ಕೊಡುವ ಪುಷ್ಪಾದಿಗಳನ್ನು ಸ್ವೀಕರಿಸಿಕೊಂಡು ಹೋದುದರಿಂದ ಮೆರವಣಿಗೆ ಬಹಳ ನಿಧಾನವಾಗಿಯೇ ಹೋಯಿತು. ಆದಕಾರಣ ಸ್ವಾಮೀಜಿ ಕ್ಯಾಸಲ್ ಕರ್ನನ್‍ಗೆ ಒಂಭತ್ತೂವರೆ ಗಂಟೆಗೆ ಮುಂಚೆ ಬರಲು ಆಗಲಿಲ್ಲ. ಮಧ್ಯದಲ್ಲಿ ವಿದ್ಯಾರ್ಥಿಗಳು ಗಾಡಿಯ ಕುದುರೆಗಳನ್ನು ಕಳಚಿ ಬಹಳ ಉತ್ಸಾಹದಿಂದ ತಾವೇ ಎಳೆದರು. ಸ್ವಾಮೀಜಿ ಬಿಡಾರಕ್ಕೆ ಬಂದಮೇಲೆ ಹೈಕೋರ್ಟಿನ ವಕೀಲರಾದ ಶ‍್ರೀಕೃಷ್ಣಮಾಚಾರಿ ಅವರು, ಮದ್ರಾಸಿನ ವಿದ್ವನ್ ಮನೋರಂಜನಿಯ ಸಭೆಯ ಪರವಾಗಿ ಸಂಸ್ಕೃತದಲ್ಲಿ ಒಂದು ಬಿನ್ನವತ್ತಳೆಯನ್ನು ಓದಿದರು. ಅನಂತರ ಜಸ್ಟಿಸ್ ಸುಬ್ರಮಣ್ಯ ಅಯ್ಯರ್, ನೆರೆದ ಜನರಿಗೆ ದಯವಿಟ್ಟು ಹೋಗಬೇಕೆಂದು ಕೋರಿಕೊಂಡರು. ಸ್ವಾಮೀಜಿ ಪ್ರಯಾಣ ಮತ್ತು ಮೆರವಣಿಗೆಯಿಂದ ಬಹಳ ಆಯಾಸಗೊಂಡಿರುವುದರಿಂದ ಅವರಿಗೆ ಸ್ವಲ್ಪ ವಿರಾಮ ಅತ್ಯಂತ ಆವಶ್ಯಕ ಎಂದು ಹೇಳಿದರು. ಕ್ಯಾಸಲ್ ಕರ್ನನ್ ಮಹಡಿಯ ಮೇಲೆ ಒಂದು ಭವ್ಯವಾಗಿ ಸುಸಜ್ಜಿತವಾದ ಕೋಣೆಯಲ್ಲಿ ಸ್ವಾಮೀಜಿಯವರನ್ನು ಇಳಿಸಿರುವರು. 

 “ಮದ್ರಾಸು ಬಹಳ ಪುರಾತನಕಾಲದಿಂದಲೂ ಇಂಡಿಯಾದೇಶದವರಿಗಾಗಲಿ ಪರದೇಶೀಯರಿಗಾಗಲಿ ಇಂತಹ ವೈಭವದ ಸ್ವಾಗತವನ್ನು ನೀಡಿರಲಿಲ್ಲ. ಮದ್ರಾಸಿನಲ್ಲಿ ನಡೆದ ಸರ್ಕಾರದ ಸ್ವಾಗತಗಳಾವುವೂ ಸ್ವಾಮೀಜಿಯವರಿಗೆ ನೀಡಿದ ಸ್ವಾಗತಕ್ಕೆ ಸರಿ ಸಮನಾಗಲಾರದು. ಮದ್ರಾಸಿನಲ್ಲಿ ವಾಸವಾಗಿರುವ ಅತ್ಯಂತ ವೃದ್ಧರ ಸ್ಮೃತಿಯಲ್ಲಿಯೂ ಕೂಡ ಹಿಂದೆ ಇಂತಹ ವೈಭವದ ಸ್ವಾಗತದ ನೆನಪು ಇಲ್ಲ. ಇಂದಿನ ದೃಶ್ಯ ಈ ತಲೆಮಾರಿನ ಜನರ ಸ್ಮೃತಿಪಥದಲ್ಲಿ ಚಿರಸ್ಥಾಯಿಯಾಗಿ ನಿಲ್ಲುವುದೆಂದು ನಾವು ಧೈರ‍್ಯವಾಗಿ ಹೇಳುತ್ತೇವೆ.” 

 ಸ್ವಾಮೀಜಿಯವರಿಗೆ ಕೊಡುವ ಬಿನ್ನವತ್ತಳೆಗಳು ಮತ್ತು ಅದಕ್ಕೆ ಸ್ವಾಮೀಜಿ ಕೊಡುವ ಉತ್ತರ ಮತ್ತು ಅನಂತರ ಅವರು ಮಾಡುವ ಭಾಷಣಗಳು ಎಲ್ಲವನ್ನೂ ಕಮಿಟಿಯವರು ವ್ಯವಸ್ಥೆಗೊಳಿಸಿ ಸ್ವಾಮೀಜಿಯವರ ಒಪ್ಪಿಗೆಯನ್ನು ಪಡೆದರು. ಸ್ವಾಮೀಜಿಯವರಿಗೆ ಮದ್ರಾಸಿನಲ್ಲಿ ಒಟ್ಟು ಇಪ್ಪತ್ತನಾಲ್ಕು ಬಿನ್ನವತ್ತಳೆಗಳನ್ನು ಇಂಗ್ಲೀಷ್, ಸಂಸ್ಕೃತ, ತಮಿಳು ಮತ್ತು ತೆಲುಗಿನಲ್ಲಿ ಅರ್ಪಿಸಿದರು. ಚೆನ್ನಪುರಿ ಅನ್ನದಾನ ಸಮಾಜದ ಸಮಾರಂಭದಲ್ಲಿ ಅಧ್ಯಕ್ಷತೆವಹಿಸಿ ಮಾತನಾಡಿದರು. ಹಲವಾರು ಜನ ಭಕ್ತರು ಅವರನ್ನು ಕಂಡು ಮಾತನಾಡಿಹೋಗುತ್ತಿದ್ದರು. ಆ ಸಮಯದಲ್ಲೆ ಸ್ವಾಮೀಜಿ ಒಂದು ಹಾಡನ್ನು ಹೇಳಿ ಎಂದು ಕೋರಿಕೊಂಡಾಗ ಜಯದೇವ ಅಷ್ಟಪದಿಯನ್ನು ಹಾಡಿದರು. ಕುಳಿತು ಕೇಳಿದವರು ಅವರ ಹಾಡಿನ ಮಾಧುರ‍್ಯ ಮತ್ತು ಭಾವಕ್ಕೆ ಮನಸೋತರು. ವೃತ್ತಪತ್ರಿಕೆಯ ಪ್ರತಿನಿಧಿಯೊಬ್ಬರು ಬಂದು ಸ್ವಾಮೀಜಿಯವರಿಂದ ಕೆಲವು ಪ್ರಶ್ನೆಗಳಿಗೆ ಉತ್ತರವನ್ನು ತೆಗೆದುಕೊಂಡು ಹೋದರು. ಅದು ಆಗಿನ ಕಾಲದಲ್ಲಿ ಹೇಗೆ ಮುಖ್ಯವಾಗಿತ್ತೊ ಇಂದಿಗೂ ಹಾಗೆಯೇ ಮುಖ್ಯವಾಗಿದೆ. ಅದಕ್ಕಾಗಿ ಅದನ್ನು ಕೆಳಗೆ ಕೊಡುವೆವು: 

 ಪ್ರಶ್ನೆ: “ಇಂಡಿಯಾದೇಶದ ಪುನರುದ್ಧಾರಕ್ಕೆ ನೀವು ಏನು ಮಾಡಬೇಕೆಂದು ಇರುವಿರಿ?” 

 ಸ್ವಾಮೀಜಿ: “ನಮ್ಮ ಜನಾಂಗದ ಒಂದು ಮಹಾಪಾತಕವೆ ಜನಸಾಮಾನ್ಯರನ್ನು ನಿರ್ಲಕ್ಷಿಸಿದ್ದ. ಇದೇ ನಮ್ಮ ಅವನತಿಗೆ ಮುಖ್ಯ ಕಾರಣ. ಸಾಮಾನ್ಯರು ವಿದ್ಯಾವಂತರಾಗಿ ಅವರಿಗೆ ಊಟಕ್ಕೆ ಬಟ್ಟೆಗೆ ಸಾಕಷ್ಟು ಸಿಕ್ಕುವವರೆಗೆ ಯಾವ ರಾಜಕೀಯದಿಂದಲೂ ಏನೂ ಪ್ರಯೋಜನವಿಲ್ಲ. ಅವರು ನಮ್ಮ ವಿದ್ಯಾಭ್ಯಾಸಕ್ಕೆ ಹಣವನ್ನು ಕೊಡುವರು. ನಮ್ಮ ದೇವಸ್ಥಾನವನ್ನು ಕಟ್ಟುವರು. ಅದಕ್ಕೆ ಬದಲು ನಿಮ್ಮಿಂದ ಅವರಿಗೆ ದೊರಕುವುದು ನಿಂದೆ. ಅವರು ನಿಜವಾಗಿ ನಮ್ಮ ಗುಲಾಮರಾಗಿರುವರು. ನಾವು ಭರತಖಂಡವನ್ನು ಉದ್ಧಾರಮಾಡಬೇಕಾದರೆ ಅವರಿಗಾಗಿ ಕೆಲಸ ಮಾಡಬೇಕು.” 

 “ನನ್ನ ಭರವಸೆಯೆಲ್ಲ ಈಗಿನ ಯುವಕರ ಮೇಲೆ ನಿಂತಿದೆ. ಇವರಿಂದ ಕೆಲಸ ಮಾಡುವ ಯುವಕರು ನಮಗೆ ಬರಬೇಕಾಗಿದೆ. ಅನಂತರ ಅವರು ಕೆಚ್ಚೆದೆಯ ಸಿಂಹದಂತೆ ಕೆಲಸ ಮಾಡುವರು. ನಾನು ಹೇಗೆ ಮಾಡಬೇಕೆಂದಿರುವುದನ್ನು ಹೇಳಿರುವೆನು. ನಾನು ಅದಕ್ಕೆ ನನ್ನ ಜನ್ಮವನ್ನು ಕೊಟ್ಟಿರುವೆನು. ನಾನು ಜಯಶೀಲನಾಗದೆ ಹೋದರೆ ಅದನ್ನು ಮಾಡಲು ಮತ್ತೊಬ್ಬರು ಬರುವರು. ನಾನು ಹೋರಾಡುವುದಕ್ಕೆ ಸಿದ್ಧನಾಗಿರುವೆನು. ಜನಸಾಮಾನ್ಯರಿಗೆ ಅವರ ಹಕ್ಕನ್ನು ಕೊಡಬೇಕು. ಪ್ರಪಂಚದಲ್ಲೆಲ್ಲ ಶ್ರೇಷ್ಥವಾದ ಧರ್ಮ ನಿಮ್ಮಲ್ಲಿದೆ. ಆದರೆ ನೀವು ಜನರಿಗೆ ಕೆಲಸಕ್ಕೆ ಬಾರದ ಮೂಢನಂಬಿಕೆಗಳನ್ನು ಕೊಡುವಿರಿ. ಅಮೃತ ಪ್ರವಾಹ ನಮ್ಮ ಹತಿರವೇ ಹರಿಯುತ್ತಿದೆ. ನೀವು ಅವರಿಗೆ ಚರಂಡಿಯ ನೀರನ್ನು ಕೊಡುವಿರಿ. ನಿಮ್ಮ ಮದ್ರಾಸಿನ ಪದವೀಧರರು ಪರೆಯನನ್ನು ಮುಟ್ಟಲಾರರು. ಆದರೆ ಅವರ ವಿದ್ಯಾಭ್ಯಾಸಕ್ಕೆ ಅವನಿಂದ ಹಣವನ್ನು ಸ್ವೀಕರಿಸಲು ಸಿದ್ಧರಾಗಿರುವರು.” 

 ಸಮಾಜ ಸುಧಾರಣೆಯ ವಿಷಯವಾಗಿ ಮಾತನಾಡುವಾಗ “ಒಂದು ದೇಶದ ಅದೃಷ್ಟ ಅಲ್ಲಿರುವ ವಿಧವೆಯರಿಗೆ ಸಿಕ್ಕುವ ಗಂಡಂದಿರ ಸಂಖ್ಯೆಯ ಮೇಲಿಲ್ಲ” ಎಂದರು. 

 ಸ್ವಾಮೀಜಿಯವರು ಬಂದ ಮೂರು ದಿನಗಳಾದಮೇಲೆ ಪುರಜನರು ಕೊಡುವ ಬಿನ್ನವತ್ತಳೆಯನ್ನು ಸ್ವೀಕರಿಸಲು ನಾಲ್ಕು ಗಂಟೆಗೆ ಹೋದರು. ದಾರಿಯ ಇಕ್ಕೆಲದಲ್ಲಿಯೂ ಅವರನ್ನು ನೋಡುವುದಕ್ಕೆ ಸಹಸ್ರಾರು ಜನರು ನೆರೆದಿದ್ದರು. ವಿಕ್ಟೋರಿಯ ಹಾಲಿನ ಮುಂದೆ ಹಾಕಿರುವ ಚಪ್ಪರದಲ್ಲಿ ಬಿನ್ನವತ್ತಳೆಯನ್ನು ಅರ್ಪಿಸಲು ಅಣಿಯಾಗಿ ಇತ್ತು. ಸರ್. ವಿ. ಭಾಷ್ಯಂ ಅಯ್ಯಂಗಾರ್ ಅವರು ಅಧ್ಯಕ್ಷರಾಗಿದ್ದರು. ಶ‍್ರೀ ಎಂ. ಬಿ. ಪಾರ್ಥಸಾರಿಥಿ ಅಯ್ಯಂಗಾರ್ ಅವರು ಬಿನ್ನವತ್ತಳೆಯನ್ನು ಓದಿದರು. ಅದಕ್ಕೆ ಉತ್ತರವನ್ನು ಕೊಡುವುದಕ್ಕೆ ಸ್ವಾಮೀಜಿ ಎದ್ದು ನಿಂತರು. ಚಪ್ಪರದ ಒಳಗೆ ಇರುವುದಕ್ಕಿಂತ ಅಧಿಕ ಸಂಖ್ಯೆಯಲ್ಲಿ ಅದರ ಹೊರಗಡೆ ಜನ ಸೇರಿದ್ದರು. ಅವರೆಲ್ಲ ಸ್ವಾಮೀಜಿಯವರನ್ನು ಬಹಿರಂಗದಲ್ಲಿ ಬಂದು ಉಪನ್ಯಾಸಮಾಡಿ ಎಂದು ಕೋರಿಕೊಂಡರು. ಸ್ವಾಮೀಜಿಯವರಿಗೆ ಅಷ್ಟೊಂದು ಜನರ ಕೋರಿಕೆಯನ್ನು ಅಲಕ್ಷ್ಯಮಾಡುವ ಮನಸ್ಸಿರಲಿಲ್ಲ. ಹೊರಗೆ ಬಂದರು. ಹೊರಗಿದ್ದ ಜನರಿಗೆಲ್ಲ ಪರಮ ಸಂತೋಷವಾಯಿತು. ಹತ್ತಿರವಿದ್ದ ಒಂದು ಕೋಚ್ ಹಾಡಿಯನ್ನು ಹತ್ತಿ ಅದರ ಮೇಲೆ ನಿಂತು ಮಾತನಾಡಲು ಉಪಕ್ರಮಿಸಿದರು. ಶ‍್ರೀಕೃಷ್ಣನಂತೆ ತಾವು ರಥದ ಮೇಲೆ ನಿಂತುಕೊಂಡು ಮಾತನಾಡುತ್ತಿರುವೆ ಎಂದು ಹಾಸ್ಯವಾಗಿ ಹೇಳಿದರು. ಸುತ್ತಲೂ ಅಸಂಖ್ಯಾತ ಜನ. ಸ್ವಾಮೀಜಿ ಧ್ವನಿ ಎಲ್ಲರಿಗೂ ಕೇಳಲು ಸಾಧ್ಯವಾಗಲಿಲ್ಲ. ಗುಜುಗುಜು ಎದ್ದಿತು. ಆಗ ಸ್ವಾಮೀಜಿ ಜನರ ಉತ್ಸಾಹವನ್ನು ಕೊಂಡಾಡಿದರು. ಈ ಉತ್ಸಾಹವನ್ನು ತಗ್ಗದಂತೆ ನೋಡಿಕೊಳ್ಳಿ ಎಂದರು. ಸ್ವಾಮೀಜಿ ಎಂದಿಗೂ ಇದು ತಮ್ಮ ವ್ಯಕ್ತಿಗೆ ತೋರಿಸುತ್ತಿರುವ ಆದರ ಎಂದು ಭಾವಿಸಲಿಲ್ಲ. ತಾವು ಅದನ್ನು ಸನಾತನಧರ್ಮ ಮತ್ತು ತತ್ತ್ವದ ಪರವಾಗಿ ಸ್ವೀಕರಿಸಿದರು. ಪ್ರಶಾಂತವಾದ ಸಮಯದಲ್ಲಿ ಮತ್ತೊಮ್ಮೆ ಮಾತನಾಡುತ್ತೇನೆಂದು ಅಂದಿನ ಭಾಷಣವನ್ನು ಪೂರೈಸಿದರು. 

 ಸ್ವಾಮೀಜಿಯವರು ಮದ್ರಾಸಿನಲ್ಲಿ ನಾಲ್ಕು ಅತ್ಯಂತ ಶ್ರೇಷ್ಠ ಉಪನ್ಯಾಸಗಳನ್ನು ಮಾಡಿದರು. ಅದರಲ್ಲಿ ಅವರು ೯ನೇ ತಾರೀಖು ಮಾಡಿದ ಉಪನ್ಯಾಸವೇ “ನನ್ನ ಸಮರ ನೀತಿ” ಎಂಬುದು. ಅದನ್ನು ವಿಕ್ಟೋರಿಯಾ ಹಾಲಿನಲ್ಲಿ ಮಾಡಿದರು. ಮುಂಚೆ ಮದ್ರಾಸಿನಲ್ಲಿ ಸ್ವಾಮೀಜಿ ಅವರ ವಿಷಯದಲ್ಲಿ ಹರಡಿದ ತಪ್ಪು ಅಭಿಪ್ರಾಯಗಳನ್ನು ತಿದ್ದಿದರು. ಥಿಯಾಸಫಿಯವರು ತಮಗೆ ಅಮೇರಿಕಾ ದೇಶದಲ್ಲಿ ಸಹಾಯ ಮಾಡಿದರೆಂಬುದು ಸುಳ್ಳೆಂದೂ ಅವರು ತಮಗೆ ವಿರೋಧವಾಗಿ ಅಮೇರಿಕಾ ದೇಶದಲ್ಲಿ ವರ್ತಿಸಿದರೆಂದೂ ಹೇಳಿದರು. ತಾವು ಪ್ರಖ್ಯಾತರಾಗುವುದಕ್ಕೆ ಮುಂಚೆ ಕಾಸಿಲ್ಲದೇ ಇರುವಾಗ ಮದ್ರಾಸಿನ ಕೆಲವು ಜನರಿಗೆ ತಂತಿ ಕಳುಹಿಸಿದರೆಂಬುದು ಥಿಯಸಫಿಸ್ಟರಿಗೆ ಗೊತ್ತಾದಾಗ ಅವರು, “ಈಗ ಈ ಪಿಶಾಚಿ ಸಾಯುತ್ತಿದೆ. ದೇವರು ನಮ್ಮನ್ನು ಈ ಪೀಡೆಯಿಂದ ತಪ್ಪಿಸಿದ” ಎಂದರು. ಎರಡನೆಯವರೇ ಬ್ರಹ್ಮ ಸಮಾಜದವರು. ಅವರು ಸಹಾಯ\break ಮಾಡುವುದಿರಲಿ, ಕ್ರೈಸ್ತ ಪಾದ್ರಿಗಳೊಂದಿಗೆ ಸೇರಿ ಇಲ್ಲಸಲ್ಲದ ಅಪವಾದ ಹರಡುವುದರಲ್ಲಿ ನಿರತರಾಗಿದ್ದರು. ಭರತಖಂಡದ ಮತ್ತೊಬ್ಬ ಖ್ಯಾತಿ ಪಡೆಯುವುದನ್ನು ಅವರು ಸಹಿಸಲಾರದೆ ಹೋದರು. ಕೆಲವು ಆಚಾರಶೀಲರು ಸ್ವಾಮೀಜಿಯವರಿಗೆ ಸಂನ್ಯಾಸಿಯಾಗಲು ಹಕ್ಕಿಲ್ಲ ಎಂದಿರುವರು. ಇದು ಅವರ ಶಾಸ್ತ್ರವೇ ಅವರಿಗೆ ಗೊತ್ತಿಲ್ಲವೆಂಬುದನ್ನು ತೋರುವುದು. ದ್ವಿಜರಿಗೆಲ್ಲ ಸಂನ್ಯಾಸಿಯಾಗಲು ಹಕ್ಕಿದೆ. ಇಲ್ಲಿ ದ್ವಿಜರು ಎಂದರೆ ಬರೀ ಬ್ರಾಹ್ಮಣರು ಮಾತ್ರವಲ್ಲ. ಉಪನಯನ ಸಂಸ್ಕಾರವುಳ್ಳವರೆಲ್ಲ ದ್ವಿಜರು. ಬ್ರಾಹ್ಮಣ ಕ್ಷತ್ರಿಯ ವೈಶ್ಯರಿಗೆಲ್ಲ ಈ ಹಕ್ಕು ಉಂಟು. ಪ್ರತಿದಿನ ಪೂಜೆ ಮಾಡುವ ಶ‍್ರೀಕೃಷ್ಣ ಮತ್ತು ರಾಮ ತಮ್ಮ ಕುಲಕ್ಕೆ ಸೇರಿದವರು ಎಂದರು. ಸಮಾಜಸುಧಾರಕರ ತಂಡ ಸ್ವಾಮೀಜಿ ತಮ್ಮ ಸಮಾಜಕ್ಕೆ ಸೇರದೇ ಇದ್ದರೆ ಅವರಿಗೆ ಅಪಾಯ ಕಾದಿದೆ ಎಂದು ಹೆದರಿಸಿರುವರು. ಸ್ವಾಮೀಜಿ ತಾವು ಹೆದರಿಕೆಗೆ ಬಗ್ಗುವವರಲ್ಲವೆಂದೂ, ಅಪಾಯದ ಕಣಿವೆಯೊಳಗೆ ತಾವು ಪ್ರಯಾಣಮಾಡಿರುವುದಾಗಿಯೂ ತಿಳಿಸಿದರು. ಸಮಾಜ ಸುಧಾರಣೆ ಒಳ್ಳೆಯದು. ಆದರೆ ಅದು ಎಲ್ಲರಿಗೂ ಅನ್ವಯಿಸುವ ಸುಧಾರಣೆಯಾಗಬೇಕು. ಜನರನ್ನು ಇದಕ್ಕಾಗಿ ಜಾಗ್ರತಗೊಳಿಸಬೇಕು. ಅವಸರದಲ್ಲಿ ಮಾಡಿದ ಸುಧಾರಣೆ ತನ್ನ ಗುರಿಯನ್ನು ಕಳೆದುಕೊಳ್ಳುತ್ತದೆ. ನಮ್ಮ ದೇಶದಲ್ಲಿ ಬಂದ ಮಹಾ ಆಚಾರ್ಯರೆಲ್ಲ ಸುಧಾರಕರೆ, ಆದರೆ ಅವರು ಜನರನ್ನು ನಿಂದಿಸಿ ಮೇಲೆ ಎತ್ತಲಿಲ್ಲ. ಸಹಾನುಭೂತಿಯಿಂದ ಪ್ರೀತಿಯಿಂದ ಉದಾತ್ತ ಭಾವನೆಗಳನ್ನು ಅವರ ಮುಂದಿಟ್ಟು ಅವನ್ನು ಹೀರಿಕೊಂಡು ಸಮಾಜ ವಿಕಾಸದ ಏಣಿಯಲ್ಲಿ ಮೇಲಕ್ಕೆ ಏರುವಂತೆ ಮಾಡಿದರು. ಅದೇ ಉಪನ್ಯಾಸದ ಕೊನೆಯಲ್ಲಿ ದೇಶಭಕ್ತಿಯ ವಿಷಯವಾಗಿ ಸ್ವಾಮಿಜಿ ಹೀಗೆ ಹೇಳುವರು: 

 “ನನಗೂ ದೇಶಭಕ್ತಿಯಲ್ಲಿ ವಿಶ್ವಾಸ ಇದೆ. ನನ್ನದೇ ಒಂದು ದೇಶಭಕ್ತಿಯ ಆದರ್ಶವಿದೆ. ಮಹಾತ್ಕಾರ್ಯಸಾಧನೆಗೆ ಮೂರು ವಿಷಯಗಳು ಅತ್ಯಾವಶ್ಯಕ. ಹೃತ್‍ಪೂರ್ವಕವಾಗಿ ದೇಶಭಕ್ತಿಯನ್ನು ಅನುಭವಿಸಿ. ಬುದ್ಧಿಯಲ್ಲಿ ಮತ್ತು ವಿಚಾರದಲ್ಲಿ ಏನಿದೆ? ಕೆಲವು ಹೆಜ್ಜೆಗಳು ಹೋಗಿ ಅಲ್ಲಿ ನಿಲ್ಲುವುವು. ಸ್ಫೂರ್ತಿ ಬರುವುದು ಹೃದಯದಿಂದ. ಪ್ರೇಮ ಅಸಾಧ್ಯವಾದ ಬಾಗಿಲನ್ನು ತೆರೆಯುವುದು. ಪ್ರಪಂಚದ ರಹಸ್ಯಕ್ಕೆಲ್ಲ ಪ್ರೇಮವೆ ಬಾಗಿಲು. ಭಾವೀ ದೇಶಭಕ್ತರೆ, ಸಮಾಜ ಸುಧಾರಕರೆ, ಮೊದಲು ದೇಶಭಕ್ತಿಯನ್ನು ಹೃದಯದಲ್ಲಿ ಅನುಭವಿಸಿ. ಕೋಟ್ಯಂತರ ದೇವಸಂತಾನರು ಮತ್ತು ಋಷಿ ಸಂತಾನರು ಪಶು ಸಂತಾನರಾಗಿರುವರೆಂದು ನಿಮಗೆ ಹೃದಯದಲ್ಲಿ ವ್ಯಥೆ ಇದೆಯೆ?‌ ಕೋಟ್ಯಂತರ ಜನರು ಈಗ ಉಪವಾಸದಿಂದ ನರಳುತ್ತಿರುವರು. ಹಿಂದಿನಿಂದಲೂ ಕೋಟ್ಯಂತರ ಜನರು ನರಳುತ್ತಿದ್ದರು ಎಂಬುದು ಗೊತ್ತೆ? ಇದರಿಂದ ನೀವು ಚಂಚಲರಾಗಿರುವಿರಾ? ಇದರಿಂದ ನಿಮ್ಮ ನಿದ್ರೆಗೆ ಭಂಗ ಬಂದಿದೆಯೆ? ಇದು ನಿಮ್ಮ ರಕ್ತದಲ್ಲಿ ವ್ಯಾಪಿಸಿದೆಯೆ? ನಾಡಿನಲ್ಲಿ ಸಂಚರಿಸಿ ಹೃದಯ ಚಲನದೊಂದಿಗೆ ಸ್ಪಂದಿಸುತ್ತಿದೆಯೆ? ಇದು ನಿಮ್ಮನ್ನು ಉನ್ಮತ್ತರನ್ನಾಗಿ ಮಾಡಿದೆಯೆ? ಈ ಸರ್ವನಾಶದ ದುಃಖ ನಿಮ್ಮನ್ನು ವ್ಯಾಪಿಸಿ, ನಿಮ್ಮ ಕೀರ್ತಿ ಯಶಸ್ಸು ಹೆಂಡಿರು ಮಕ್ಕಳು ಆಸ್ತಿ ಮತ್ತು ದೇಹವನ್ನೇ ನೀವು ಮರೆತಿರುವಿರಾ? ನೀವು ಇದನ್ನು ಮಾಡಿರುವಿರಾ? ದೇಶಭಕ್ತರಾಗುವುದಕ್ಕೆ ಇದೇ ಮೊದಲ ಹೆಜ್ಜೆ. ನಿಮ್ಮಲ್ಲಿ ಅನೇಕರು ತಿಳಿದುಕೊಂಡಿರುವಂತೆ ನಾನು ವಿಶ್ವಧರ್ಮಸಮ್ಮೇಳನಕ್ಕಾಗಿ ಅಮೇರಿಕಾ ದೇಶಕ್ಕೆ ಹೋಗಲಿಲ್ಲ. ಈ ಉದ್ವೇಗದೈತ್ಯ ನನ್ನ ಹೃದಯವನ್ನು ಮೆಟ್ಟಿಕೊಂಡು ಪ್ರಚೋದಿಸುತ್ತಿತ್ತು. ನನ್ನ ದೇಶೀಯರಿಗೆ ಸೇವೆಮಾಡುವ ದಾರಿ ತಿಳಿಯದೆ ಹನ್ನೆರಡು ವರುಷಗಳವರೆಗೆ ದೇಶದಲ್ಲಿ ಸಂಚರಿಸಿದೆ. ಆದಕಾರಣವೆ ನಾನು ಅಮೇರಿಕಾದೇಶಕ್ಕೆ ಹೋದೆ. ಇದು ನನ್ನ ಪರಿಚಯವಿದ್ದ ನಿಮಗೆಲ್ಲಾ ಗೊತ್ತಿದೆ. ವಿಶ್ವಧರ್ಮಸಮ್ಮೇಳನವನ್ನು ಯಾರು ಲಕ್ಷಿಸಿದರು? ನನ್ನ ಸ್ವಂತ ದೇಶೀಯರು ಪ್ರತಿದಿನ ಅಧಃಪಾತಾಳಕ್ಕೆ ಇಳಿಯುತ್ತಿದ್ದರು. ಅವರನ್ನು ಯಾರು ಲೆಕ್ಕಿಸಿದರು? ಇದೇ ನನ್ನ ಪ್ರಥಮ ಹೆಜ್ಜೆ. 

 “ಈ ಉದ್ವೇಗವನ್ನು ಬರಿಯ ಬಾಯಿಮಾತಿನಲ್ಲಿ ವ್ಯಕ್ತಗೊಳಿಸದೆ ಈ ಸಮಸ್ಯೆಯಿಂದ ಪಾರಾಗುವುದಕ್ಕೆ ಯಾವುದಾದರೂ ಒಂದು ಮಾರ್ಗವನ್ನು ಕಂಡು ಹಿಡಿದಿರುವಿರಾ?… ಇದು ಮಾತ್ರವಲ್ಲ. ಪರ್ವತೋಪಮ ಆತಂಕಗಳನ್ನು ದಾಟಬಲ್ಲ ಇಚ್ಛಾಶಕ್ತಿ ನಿಮ್ಮಲ್ಲಿದೆಯೆ? ಇಡಿ ಪ್ರಪಂಚವೇ ಹಿರಿದ ಕತ್ತಿಯಿಂದ ನಿಮ್ಮನ್ನು ಎದುರಿಸಿದರೂ ಸರಿ ಎಂದು ತೋರಿದುದನ್ನು ಮಾಡುವ ಶಕ್ತಿ ನಿಮ್ಮಲ್ಲಿದೆಯೆ?” 

 ಸ್ವಾಮೀಜಿ ಎರಡನೆಯ ಉಪನ್ಯಾಸ ‘ಭಾರತೀಯರ ಜೀವನದಲ್ಲಿ ವೇದಾಂತದ ಪ್ರಭಾವ’ ಎಂಬುದು. ನಾವು ನಿಜವಾಗಿ ಹಿಂದೂಗಳಲ್ಲ, ಆರ್ಯರು. ಅವರ ಗ್ರಂಥವೇ ವೇದ. ಅಲ್ಲಿ ಕರ್ಮಕಾಂಡ ಮತ್ತು ಜ್ಞಾನಕಾಂಡಗಳಿವೆ. ಉಪನಿಷತ್ತು ಜ್ಞಾನಕಾಂಡದಲ್ಲಿ ಬರುವುದು. ಇದನ್ನೇ ವೇದಾಂತ ಎನ್ನುವರು. ಈ ವೇದಾಂತದಲ್ಲಿ ದ್ವೈತ, ವಿಶಿಷ್ಟಾದ್ವೈತ ಮತ್ತು ಅದ್ವೈತ ಭಾವನೆಗಳೆಲ್ಲ ಇವೆ. ಈ ಭಾವನೆಗಳು ಒಂದಕ್ಕೆ ಮತ್ತೊಂದು ಪೂರಕವಾಗಿವೆಯೇ ಹೊರತು, ಒಂದು ಮತ್ತೊಂದನ್ನು ವಿರೋಧಿಸುವುದಿಲ್ಲ. ಇವುಗಳಲ್ಲಿ ಸ್ವಾರಸ್ಯವಿದೆ. ಉಪನಿಷತ್ತುಗಳು ಕೇವಲ ತತ್ತ್ವ ದೃಷ್ಟಿಯಿಂದ ಮಾತ್ರ ಪ್ರಪಂಚದ ಅತ್ಯುತ್ತಮ ಗ್ರಂಥವಲ್ಲ. ಕಾವ್ಯದ ದೃಷ್ಟಿಯಿಂದಲೂ ಮಹಿಮೋನ್ನತವಾದ ಭಾವನೆಗಳನ್ನು ಹೊಂದಿವೆ. ಉಪನಿಷತ್ತಿನಲ್ಲೆಲ್ಲ ಶಕ್ತಿ ಧ್ವನಿತವಾಗುತ್ತಿದೆ. 

 ನಮ್ಮ ಅವನತಿಗೆಲ್ಲ ನಮ್ಮ ದೌರ್ಬಲ್ಯವೇ ಕಾರಣವೆಂದು ಸ್ವಾಮೀಜಿ ಒತ್ತಿ ಹೇಳಿದರು: “ನಿಮ್ಮ ದೇಹ ದುರ್ಬಲ, ಮನಸ್ಸು ದುರ್ಬಲ, ನಿಮ್ಮಲ್ಲಿ ಆತ್ಮಶ್ರದ್ಧೆ ಇಲ್ಲ. ನನ್ನ ಸಹೋದರರೆ, ಶತಮಾನ ಶತಮಾನಗಳಿಂದಲೂ ಕಳೆದ ಸಾವಿರ ವರ್ಷಗಳಿಂದಲೂ ಜಾತಿಭಾವನೆ, ರಾಜರು, ಪರದೇಶೀಯರು ಮತ್ತು ಸ್ವಜನರ ದಬ್ಬಾಳಿಕೆ ಇವು ನಿಮ್ಮ ಸತ್ತ್ವವನ್ನೆಲ್ಲ ನಾಶಮಾಡಿರುವುವು. ನಿಮಗೆ ಬೆನ್ನೆಲುಬೇ ಇಲ್ಲ. ನೀವು ಹೊರಳಾಡುತ್ತಿರುವ ಕ್ರಿಮಿಕೀಟಗಳಂತೆ. ಯಾರು ನಿಮಗೆ ಶಕ್ತಿಯನ್ನು ನೀಡಬಲ್ಲರು? ಶಕ್ತಿ, ಶಕ್ತಿ ನಿಮಗೆ ಬೇಕಾಗಿರುವುದು. ಇದನ್ನೇ ನಾನು ಒತ್ತಿ ಹೇಳುತ್ತೇನೆ. ಉಪನಿಷತ್ತಿನ ಮೇಲೆ ನಿಂತು ನಾನು ಆತ್ಮ ಎಂಬುದನ್ನು ನಂಬಬೇಕು.” 

 ಉಪನಿಷತ್ತಿನ ಜ್ಞಾನ ಕೇವಲ ಸಂನ್ಯಾಸಿಗಳು ಅಥವಾ ಪಂಡಿತರಿಗೆ ಮೀಸಲಲ್ಲ. ಅದನ್ನು ಎಲ್ಲರಿಗೂ ಅವರು ತಿಳಿದುಕೊಳ್ಳುವಂತಹ ಭಾಷೆಯಲ್ಲಿ ಹೇಳಬೇಕು.\break ಉಪನಿಷತ್ತಿನ ಅಮರಸಂದೇಶ ಒಬ್ಬನು ಜೀವನದ ಯಾವ ಕಾರ್ಯಕ್ಷೇತ್ರದಲ್ಲಿರಲಿ ಅದು ಅವನ ಸಹಾಯಕ್ಕೆ ಬರಬೇಕು. 

 ಸ್ವಾಮೀಜಿಯವರ ಮೂರನೇ ಉಪನ್ಯಾಸವೇ ‘ಭಾರತದ ಮಹಾಪುರುಷರು’ ಎಂಬುದು. ವೇದಾಂತ ದರ್ಶನ ಸನಾತನವಾದ ತತ್ತ್ವದ ಮೇಲೆ ನಿಂತಿದೆ. ಅದನ್ನು ವಿವರಿಸುವುದಕ್ಕೆ ತಮ್ಮ ಜೀವನದಲಿ ಅನುಷ್ಠಾನ ಮಾಡಿಕೊಂಡು ಇತರರಿಗೂ ಇದು ಸಾಧ್ಯ ಎಂಬುದನ್ನು ತೋರುವ ಮಹಿಮೋನ್ನತ ವ್ಯಕ್ತಿಗಳನ್ನು ಭರತಖಂಡ ಸೃಷ್ಟಿಸಿದೆ. ಅವರಲ್ಲಿ ಸ್ತ್ರೀಯರಿರುವರು, ಪುರುಷರಿರುವರು. ವೇದಗಳ ಕಾಲದಲ್ಲೆ ಮಹಾ ವಿದ್ವಾಂಸರಾದ ಸ್ತ್ರೀಯರಿದ್ದರು. ಶ‍್ರೀರಾಮ, ಕೃಷ್ಣ, ಬುದ್ಧ, ಶಂಕರ, ರಾಮಾನುಜ, ಮಧ್ವ, ಚೈತನ್ಯ, ಸಿಕ್ಕರ ಗುರುಗಳು ಮುಂತಾದವರ ವಿಷಯವನ್ನೆಲ್ಲ ವಿಹಂಗಮ ದೃಷ್ಟಿಯಲ್ಲಿ ಚಿತ್ರಿಸಿ ಆದಮೇಲೆ ತಮ್ಮ ಗುರುಗಳಾದ ಶ‍್ರೀರಾಮಕೃಷ್ಣರ ವಿಷಯವನ್ನು ಹೀಗೆ ಪ್ರಸ್ತಾಪಿಸುವರು: 

 “ಅವರೇ ಯುಗಾವತಾರ ಶ‍್ರೀರಾಮಕೃಷ್ಣ ಪರಮಹಂಸರು. ಒಬ್ಬ ಅಲೌಕಿಕ ಮಹಾಪುರುಷರು ಇದೊಂದು ಬೃಹತ್ಕಥೆ. ಅದನ್ನು ಹೇಳಲು ನನಗಿಂದು ಸಮಯವಿಲ್ಲ. ಭಾರತೀಯ ಮಹಾಪುರುಷರ ಪೂರ್ಣಪ್ರಕಾಶಸ್ವರೂಪರು. ಅವರ ಉದ್ದೇಶ ಇಂದಿಗೆ ವಿಶೇಷ ಕಲ್ಯಾಣಕಾರಿ. ಅವರಲ್ಲಿ ಈಶ್ವರೀ ಶಕ್ತಿ ಕೆಲಸ ಮಾಡುತ್ತಿರುವುದನ್ನು ಇಂದು ಗಮನಿಸಿ… ನಾನೇನಾದರೂ ಒಂದು ಸತ್ಯ ವಾಕ್ಯವನ್ನು ನುಡಿದಿದ್ದರೆ ಅದೆಲ್ಲ ಅವರ ವಾಕ್ಯ. ಯಾವುದು ಅಸತ್ಯವಾಗಿರುವುದೋ, ಭ್ರಮೆಯಿಂದ ಕೂಡಿರುವುದೋ ಅಥವಾ ಮಾನವ ಕೋಟಿಗೆ ಹಿತಕಾರಿಯಲ್ಲವೋ ಅದೆಲ್ಲಾ ನನ್ನದು. ಅದರ ಜವಾಬ್ದಾರಿ ನನ್ನದು.” 

 ‘ಭವಿಷ್ಯ ಭಾರತ’ ಎಂಬ ಮತ್ತೊಂದು ಉಪನ್ಯಾಸದಲ್ಲಿ ಸ್ವಾಮೀಜಿ ಹೀಗೆ ಹೇಳುವರು: “ಬರುವ ಐವತ್ತು ವರ್ಷಗಳವರೆಗೆ ಈ ನಮ್ಮ ಮಾತೃಭೂಮಿಯೆ ಆರಾಧನೆಯ ಇಷ್ಟದೈವವಾಗಬೇಕು. ಇನ್ನುಳಿದ ದೇವತೆಗಳೆಲ್ಲ ಕೆಲವು ಕಾಲ ನಮ್ಮಿಂದ ಕಣ್ಮರೆಯಾಗಲಿ. ಜಾಗ್ರತನಾಗಿರುವ ದೇವರು ಇದೊಂದೇ. ಇದೇ ನಮ್ಮ ಜನಾಂಗ. ಎಲ್ಲೆಲ್ಲೂ ಅವನ ಕಾಲುಗಳೇ, ಎಲ್ಲೆಲ್ಲಿಯೂ ಅವನ ಕೈಗಳೇ, ಎಲ್ಲೆಲ್ಲಿಯೂ ಅವನ ಕಿವಿಗಳೇ. ಅವನೇ ಸರ್ವವ್ಯಾಪಿಯಾಗಿರುವನು. ಇತರ ದೇವತೆಗಳೆಲ್ಲಾ ನಿದ್ರಿಸುತ್ತಿರುವರು. ನಿಮ್ಮ ಸುತ್ತಲೂ ಇರುವ ವಿರಾಟ್ ಮಹೇಶ್ವರನ ಆರಾಧನೆಯನ್ನು ತೊರೆದು ಕೆಲಸಕ್ಕೆ ಬಾರದ ಇತರ ದೇವರುಗಳನ್ನು ಏಕೆ ಅರಸಿಕೊಂಡು ಹೋಗುವುದು?” 

 ಸ್ವಾಮೀಜಿ ಮದ್ರಾಸಿನಲ್ಲಿದ್ದಾಗ ಕೆಲವು ಘಟನೆಗಳು ಜರುಗಿದವು. ತಿರುಪತ್ತೂರಿನಿಂದ ಕೆಲವು ಶೈವ ಸಿದ್ಧಾಂತಿಗಳು ಸ್ವಾಮೀಜಿಯವರನ್ನು ಅದ್ವೈತ ಸಿದ್ಧಾಂತದ ಮೇಲೆ ಕೆಲವು ಪ್ರಶ್ನೆಗಳನ್ನು ಕೇಳಲು ಬಂದರು. ಅವರ ಮೊದಲನೆ ಪ್ರಶ್ನೆಯೇ ನಿರಪೇಕ್ಷ ಹೇಗೆ ಸಾಪೇಕ್ಷವಾಯಿತು ಎಂಬುದು. ಸ್ವಾಮೀಜಿ ಪ್ರಶ್ನೆಗೆ ಉತ್ತರವಾಗಿ, ಏಕೆ ಹೇಗೆ ಎಂಬುದನ್ನು ಸಾಪೇಕ್ಷವಸ್ತುವಿಗೆ ಬೇಕಾದರೆ ಹಾಕಬಹುದೇ ಹೊರತು ನಿರಪೇಕ್ಷೆ ವಸ್ತುವಿಗೆ ಹಾಕುವುದಕ್ಕೆ ಆಗುವುದಿಲ್ಲ ಎಂದರು. ಏಕೆ, ಹೇಗೆ ಎಂಬುದು ದೇಶಕಾಲ ನಿಮಿತ್ತದ ಒಳಗಡೆ ಇರುವ ವಸ್ತುವಿಗೆ ಅನ್ವಯಿಸುವುದು, ಅದರಾಚೆ ಇಂತಹ ಪ್ರಶ್ನೆಗಳೇ ಏಳಲಾರವು. ಕನಸಿನಲ್ಲಿ ಆದ ಒಂದು ಘಟನೆಗೆ ಎದ್ದಾದಮೇಲೆ ಕಾರಣವನ್ನು ಹುಡುಕಿದಂತೆ. ಸ್ವಾಮೀಜಿಯವರು ಅನಂತರ ಬಂದವರಿಗೆ, ತರ್ಕದಿಂದ ದೇವರು ಒಲಿಯಲಾರನೆಂದೂ ಅವನನ್ನು ಪಡೆಯಬೇಕಾದರೆ ಅವನ ಮಕ್ಕಳ ಸೇವೆಯನ್ನು ಮಾಡಬೇಕೆಂದೂ ಹೇಳಿದರು. 

 ಒಂದು ದಿನ ಮಧ್ಯಾಹ್ನ ಒಬ್ಬ ಐರೋಪ್ಯ ಮಹಿಳೆ ಸ್ವಾಮೀಜಿಯವರೊಡನೆ ಹಲವು ವೇದಾಂತ ವಿಷಯಗಳ ಮೇಲೆ ಚರ್ಚಿಸಿದಳು. ಸ್ವಾಮೀಜಿಯವರು ಸೂಕ್ತವಾದ ಉತ್ತರವನ್ನು ಕೊಟ್ಟರು. ಅನಂತರ ಸಾಯಂಕಾಲವಾದ ಮೇಲೆ ಆಕೆ ತನ್ನ ತಂದೆಯನ್ನು ಕರೆದುಕೊಂಡು ಬಂದಳು. ಆತ ಒಬ್ಬ ಪಾದ್ರಿ. ಆತನು ಸ್ವಾಮೀಜಿಯವರಿಗೆ ಹಲವು ವಿಷಯಗಳ ಮೇಲೆ ಪ್ರಶ್ನೆ ಹಾಕಿ ಸಮರ್ಪಕವಾದ ಉತ್ತರವನ್ನು ಪಡೆದನು. ಅಂತೂ ಸ್ವಾಮೀಜಿ ಬಹಿರಂಗ ಉಪನ್ಯಾಸದ ವೇದಿಕೆಯ ಮೇಲೆ ಇಲ್ಲದೇ ಇದ್ದರೂ ಇದ್ದ ಕಡೆ ಹಲವು ಜನ ಮಾತನಾಡುವುದಕ್ಕೆ ಬರುತ್ತಿದ್ದರು. ಸ್ವಾಮೀಜಿಯವರು ಮದ್ರಾಸಿನಲ್ಲಿರುವ ತನಕ ಅವ್ಯಾಹತವಾದ ಚಟುವಟಿಕೆಯ ಸುಳಿಯಲ್ಲಿದ್ದರು. 

 ಒಂದು ದಿನ ಒಬ್ಬ ಪಂಡಿತರು ಬಂದು ಸ್ವಾಮೀಜಿಯವರನ್ನು “ನೀವು ದ್ವೈತಿಗಳೆ, ವಿಶಿಷ್ಟಾದ್ವೈತಿಗಳೆ ಅಥವಾ ಅದ್ವೈತಿಗಳೇ?” ಎಂದು ಪ್ರಶ್ನೆ ಹಾಕಿದರು. ಅದಕ್ಕೆ ಸ್ವಾಮೀಜಿ ಎಲ್ಲಿಯವರೆಗೆ ದೇಹಭಾವನೆ ಇರುವುದೊ ಅಲ್ಲಿಯವರೆಗೆ ತಾವು ದ್ವೈತಿಗಳು ಎಂದೂ, ಅನಂತರ ಅಲ್ಲವೆಂದೂ ಹೇಳಿದರು. ಆಗ ಪಂಡಿತರು ಹಾಗಾದರೆ ನೀವು ಅದ್ವೈತಿಗಳು ಎಂದರು. ಸ್ವಾಮೀಜಿ ನೀವು ಬೇಕಾದ ಹಾಗೆ ಭಾವಿಸಬಹುದು ಎಂದರು. 

 ಸ್ವಾಮೀಜಿಯವರು ಮೂರು ತತ್ತ್ವಗಳಲ್ಲಿ ಒಂದು ನಿಜ ಮತ್ತೊಂದು ಸುಳ್ಳು ಎನ್ನುವ ಗೋಜಿಗೆ ಹೋಗದೆ ಒಂದು ಮತ್ತೊಂದರ ಪೂರೈಕೆ ಎಂದು ಹೇಳುತ್ತಿದ್ದರು. ಇದನ್ನು ಒಬ್ಬ ಪಂಡಿತರು ಕೇಳಿ, “ಇದು ಹಾಗೆ ಇದ್ದರೆ ಹಿಂದಿನ ಆಚಾರ‍್ಯರು ಏತಕ್ಕೆ ಇದನ್ನು ಹೇಳಲಿಲ್ಲ?” ಎಂದು ಕೇಳಿದರು. ಅದಕ್ಕೆ ಸ್ವಾಮಿಗಳು, “ಅದನ್ನು ನನಗೆ ಹೇಳುವುದಕ್ಕೆ ಬಿಟ್ಟುಹೋಗಿರುವರು?” ಎಂದರು. 

 ಸ್ವಾಮೀಜಿಯವರು ಮದ್ರಾಸಿನಲ್ಲಿದ್ದಾಗ ಅಮೇರಿಕಾ ಮತ್ತು ಇಂಗ್ಲೆಂಡುಗಳಿಂದ ಸ್ವಾಮೀಜಿಯವರು ಆ ದೇಶಗಳಲ್ಲಿ ವೇದಾಂತ ವಿಷಯದಲ್ಲಿ ಅಲ್ಲಿಯ ಜನರ ಮನಸ್ಸು ಜಾಗ್ರತವಾಗುವಂತೆ ಮಾಡಿದುದಕ್ಕಾಗಿ ಅವರನ್ನು ಅಭಿನಂದಿಸಿ ಪತ್ರಗಳನ್ನು ಬರೆದರು. 

 ಸ್ವಾಮೀಜಿಯವರು ಫೆಬ್ರವರಿ ಹದಿನೈದನೆಯ ದಿನ ಹಡಗಿನಲ್ಲಿ ಕಲ್ಕತ್ತೆಗೆ ತೆರಳಿದರು. ಮದ್ರಾಸಿನ ಜನರು ಸ್ವಾಮೀಜಿಯವರನ್ನು ಮದ್ರಾಸಿನಲ್ಲಿಯೇ ಇದ್ದು ತಮ್ಮ ಸಂಘದ ಕಾರ್ಯವನ್ನು ಮುಂದುವರಿಸುವಂತೆ ಕೇಳಿಕೊಂಡರು. ಸ್ವಾಮೀಜಿ ಅದಕ್ಕೆ ತಾವು ಮದ್ರಾಸಿನ ಕೆಲಸಕ್ಕೆ ತಮ್ಮ ಗುರುಭಾಯಿಯನ್ನು ಕಳುಹಿಸುತ್ತೇನೆ ಎಂದು ಹೇಳಿದರು. ಪೂನಾದಿಂದ ಬಾಲಗಂಗಾಧರತಿಲಕರು ಸ್ವಾಮೀಜಿಯವರಿಗೆ ಪುಣೆಗೆ ಬರಬೇಕು ಎಂದು ಕೋರಿ ಒಂದು ಪತ್ರವನ್ನು ಬರೆದರು. ಆದರೆ ಸ್ವಾಮೀಜಿ ಕೊಲಂಬೊ ಮುಟ್ಟಿದಂದಿನಿಂದ ಇದುವರೆಗೆ ಬಿಡುವಿಲ್ಲದೆ ಚಟುವಟಿಕೆಯ ಸುಂಟರ ಗಾಳಿಯಲ್ಲಿ ಸಿಕ್ಕಿ ತಮಗೆ ಈಗ ವಿಶ್ರಾಂತಿ ಸ್ವಲ್ಪ ಆವಶ್ಯಕವಿರುವುದರಿಂದ ಸಧ್ಯಕ್ಕೆ ಬರಲು ಸಾಧ್ಯವಿಲ್ಲವೆಂದು ತಿಳಿಸಿದರು. ಮದ್ರಾಸಿನ ರೇವುಪಟ್ಟಣದಲ್ಲಿ ಆರ್ಯವೈಶ್ಯ ಜನಾಂಗದವರು ಸ್ವಾಮೀಜಿ ಹಿಂದೂಧರ್ಮಕ್ಕೆ ಮಾಡಿದ ಸೇವೆಯನ್ನು ಪ್ರಶಂಸಿಸಿ ಒಂದು ಭಿನ್ನವತ್ತಳೆಯನ್ನು ಅರ್ಪಿಸಿದರು. ರಾಜಮಹೇಂದ್ರಿಯ ಹಾನರಬಲ್ ಶ‍್ರೀ ಸುಬ್ಬರಾವ್ ಅವರು ಒಂದು ಬಿನ್ನವತ್ತಳೆಯನ್ನು ಆ ಊರಿನ ಜನರ ಪರವಾಗಿ ಅರ್ಪಿಸಿದರು. ಇವೆರಡಕ್ಕೂ ಸೂಕ್ತವಾದ ಉತ್ತರವನ್ನು ಕೊಟ್ಟರು. ಶ‍್ರೀ ಸುಂದರಂ ಅಯ್ಯರ್ ಅವರು ಪಾಶ್ಚಾತ್ಯ ದೇಶಗಳಲ್ಲಿ ವೇದಾಂತ ಬೋಧನೆಯನ್ನು ಮಾಡಿ ಏನಾದರೂ ಒಳ್ಳೆಯದಾಗಿದೆಯೆ? ಎಂದು ಪ್ರಶ್ನೆ ಹಾಕಿದರು. ಅದಕ್ಕೆ ಸ್ವಾಮೀಜಿ, ಅಲ್ಲೊಂದು ಇಲ್ಲೊಂದು ಈ ಮಹಾಭಾವನೆಯ ಬೀಜವನ್ನು ಬಿತ್ತಿರುವೆನು. ಅದು ಕಾಲಕ್ರಮದಲ್ಲಿ ಫಲವನ್ನು ಕೊಟ್ಟು ಕೆಲವು ಜೀವಿಗಳಿಗೆ ಸಹಾಯ ಮಾಡಬಹುದು ಎಂದರು. ಎರಡನೆಯದಾಗಿ ನಾವು ನಿಮ್ಮನ್ನು ಪುನಃ ಇಲ್ಲಿ ನೋಡುವ ಸೌಭಾಗ್ಯವಿದೆಯೆ, ನೀವು ದಕ್ಷಿಣ ಭರತಖಂಡದಲ್ಲಿ ನಿಮ್ಮ ಸಂಘದ ಕೆಲಸವನ್ನು ಮುಂದುವರಿಸುವಿರಾ? ಎಂದು ಕೇಳಿದ್ದಕ್ಕೆ “ಈ ವಿಷಯದಲ್ಲಿ ನಿಮಗೆ ಸಂಶಯ ಇಲ್ಲದಿರಲಿ, ಕೆಲವು ಕಾಲ ಹಿಮಾಲಯ ಪ್ರಾಂತ್ಯದಲ್ಲಿ ವಿಶ್ರಾಂತಿ ತೆಗೆದುಕೊಂಡಮೇಲೆ ದೇಶದ ಮೇಲೆಲ್ಲ ಮಹಾಪರ್ವತದಂತೆ ಬೀಳುತ್ತೇನೆ” ಎಂದರು. ಅನಂತರ ಜಹಜು ಕಲ್ಕತ್ತೆಯ ಕಡೆ ಹೊರಟಿತು. ಸ್ವಾಮೀಜಿ ಮದ್ರಾಸಿನಲ್ಲಿ ಒಂಬತ್ತು ದಿನಗಳನ್ನು ಕಳೆದರು. ಅಲ್ಲಿಯ ಜನರಿಗೆ ಅವರ ಮಾತನ್ನು ಕೇಳುವ, ಅವರ ದರ್ಶನ ಪಡೆಯುವ ಒಂದು ಅಪೂರ್ವ ಸುಯೋಗ ಲಭಿಸಿತ್ತು. 

