
\chapter{ಪತ್ರಗಳ ಆಗ್ನೇಯಾಸ್ತ್ರಗಳು}

 ಸ್ವಾಮಿ ವಿವೇಕಾನಂದರು ಪಾಶ್ಚಾತ್ಯದೇಶದಲ್ಲಿ ಕೀರ್ತಿಯನ್ನು ಗಳಿಸಿದ ಮೇಲೆ ಭರತಖಂಡದ ಧರ್ಮದ ವಿಷಯವಾಗಿ ಆ ದೇಶದಲ್ಲಿ ಉಪನ್ಯಾಸ ಮಾಡತೊಡಗಿದರು. ಆ ದೇಶಕ್ಕೆ ತಮ್ಮ ವಾಗ್ಧಾರೆಯಿಂದ ಅಮೃತವನ್ನು ನೀಡಿದರು. ಆದರೆ ಆ ಸಮಯದಲ್ಲಿಯೂ ಭರತಖಂಡವನ್ನು ಮರೆಯಲಿಲ್ಲ. ಅಮೇರಿಕಾ ದೇಶದಲ್ಲಿರುವ ಒಳ್ಳೆಯ ವಿಷಯವಾಗಿ ಭರತಖಂಡದ ತಮ್ಮ ಭಕ್ತಾದಿಗಳಿಗೆ ಕಾಗದ ಬರೆದರು. ಭರತಖಂಡದ ಲೋಪ ದೋಷಗಳನ್ನು ನಮಗೆ ಕಾಣುವಂತೆ ವ್ಯಕ್ತಪಡಿಸಿದರು. ಬೆಂಕಿಯಂತಹ ಭಾಷೆಯಲ್ಲಿ ಬರೆದ ಪತ್ರಗಳ ಮೂಲಕ ಜನರನ್ನು ಎಚ್ಚರಗೊಳಿಸಲೆತ್ನಿಸಿದರು. ಈಗಲೂ ಅದನ್ನು ಓದಿದವರು ಸ್ವಾಮೀಜಿ ಹೃದಯದ ಉದ್ವೇಗವನ್ನು ಅನುಭವಿಸುವರು. ಹಿಂದಿನ ಭರತಖಂಡದ ಅಧ್ಯಕ್ಷರಾದ ಶ‍್ರೀ‌ರಾಧಾಕೃಷ್ಣನ್ ಅವರು ಆ ಸಮಯದಲ್ಲಿ ಮದ್ರಾಸಿನ ಕಾಲೇಜಿನಲ್ಲಿ ಓದುತ್ತಿದ್ದರು. ಅವರು ಒಮ್ಮೆ ವಿವೇಕಾನಂದರ ವಿಷಯವಾಗಿ ಮಾತನಾಡುತ್ತಿದ್ದಾಗ ಸ್ವಾಮೀಜಿ ಅಮೇರಿಕಾ ದೇಶದಿಂದ ಬರೆಯುತ್ತಿದ್ದ ಆ ಪತ್ರಗಳನ್ನು ತಾವು ಓದುತ್ತಿದ್ದೆವೆಂದೂ, ಅವು ಅದ್ಭುತವಾದ ಪರಿಣಾಮವನ್ನು ತಮ್ಮ ಮೇಲೆ ಉಂಟುಮಾಡಿತೆಂದೂ ಹೇಳಿರುವರು. 

 ಸ್ವಾಮಿ ವಿವೇಕಾನಂದ ಸಾಹಿತ್ಯನಿಧಿಯಲ್ಲಿ ಅಮೇರಿಕಾ ದೇಶದಿಂದ ಆ ಸಮಯದಲ್ಲಿ ಭರತಖಂಡದ ತಮ್ಮ ಭಕ್ತಾದಿಗಳಿಗೆ ಬರೆದ ಪತ್ರಗಳಿಗೆ ಒಂದು ಪ್ರತ್ಯೇಕ ಸ್ಥಾನವೇ ಇದೆ. ಹೃದಯವನ್ನು ಸೂರೆಗೊಳ್ಳುವ ಅವರ ಉಪನ್ಯಾಸವನ್ನು ಕೇಳುವುದಕ್ಕೆ ಸಹಸ್ರಾರು ಮಂದಿ ಪ್ರಾಚ್ಯ ಮತ್ತು ಪಾಶ್ಚಾತ್ಯ ದೇಶದಲ್ಲಿ ನೆರೆಯುತ್ತಿದ್ದರು. ಅವರು ಬರೆದ ಪುಸ್ತಕಗಳನ್ನು ಅಷ್ಟೇ ಜನ ಕುತೂಹಲದಿಂದ ಓದುತ್ತಿದ್ದರು. ಆದರೆ ಪತ್ರಗಳನ್ನಾದರೋ ‌ಅವರು ಬರೆದಿದ್ದು ಆರಿಸಿದ ಕೆಲವು ಸ್ನೇಹಿತರಿಗೆ, ನೆಚ್ಚಿನ ಗುರುಭಾಯಿಗಳಿಗೆ ಮತ್ತು ಭಕ್ತರಿಗೆ. ಸಹಸ್ರಾರು ಜನರ ಸಭೆಯ ಮುಂದೆ ನಿಂತುಕೊಂಡು ಕೊಡುವ ಬಹಿರಂಗ ಉಪನ್ಯಾಸದಂತಿಲ್ಲ ಇಲ್ಲಿಯ ವಾತಾವರಣ. ನಿರ್ಜನ ಪ್ರದೇಶದಲ್ಲಿ ಗುರು ತನ್ನ ಶಿಷ್ಯನಿಗೆ ಕೊಡುವ ಮಂತ್ರೋಪದೇಶದಂತೆ ಇದೆ ಈ ಪತ್ರಗಳು. 

 ಸ್ವಾಮಿ ವಿವೇಕಾನಂದರು ತಮ್ಮ ಪತ್ರಗಳಲ್ಲಿ ಭಾರತೀಯರ ಹಿತಕ್ಕಾಗಿ ವಿಚಾರಿಸದ ಸಮಸ್ಯೆಯೇ ಇಲ್ಲವೆನ್ನಬಹುದು. ಪಾಶ್ಚಾತ್ಯ ದೇಶಗಳು ಇನ್ನೂ ಅರೆನಾಗರಿಕರಾಗಿ, ಕಾಡುಮೇಡುಗಳಲ್ಲಿ ಗುಹೆ ಪೊದೆಗಳಲ್ಲಿ ವಾಸಮಾಡುತ್ತಿದ್ದ ಕಾಲದಲ್ಲೆ ಭಾರತವರ್ಷದಲ್ಲಿ ಇಪ್ಪತ್ತನೆಯ ಶತಮಾನದ ನಾಗರಿಕತೆಯನ್ನು ಕೂಡ ಅಣಕಿಸುವಂತಹ ನಾಗರಿಕತೆ ಇತ್ತು. ಇಲ್ಲಿಯ ಧರ್ಮ ಇಲ್ಲಿಯ ತತ್ತ್ವಶಾಸ್ತ್ರ ಇಲ್ಲಿಯ ಋಷಿಗಳು ರಚಿಸಿದ ಉಪನಿಷತ್ತು ಮತ್ತು ಗೀತೆ ಇವುಗಳನ್ನು ಮೀರಬಲ್ಲದು ಯಾವುದಿತ್ತು ಅಂದು? ಆದರೆ ಇಂದು ಅವೆಲ್ಲ ಗತವೈಭವವಾಗಿದೆ. ಕಳೆದ ಸವಿದಿನಗಳನ್ನು ಮೆಲುಕು ಹಾಕುವುದೇ ನಮಗೆ ಈಗ ಉಳಿದಿರುವುದು. ಭಾರತವರ್ಷ ಇಂದು ಜಗದ ಕಸದ ಬುಟ್ಟಿಯೊಳಗೆ ಇರುವುದು. ಇಂತಹ ಅಧೋಗತಿಗೆ ಕಾರಣಗಳೇನು ಎಂಬುದನ್ನು ಸ್ವಾಮೀಜಿಯವರು ಈ ಪತ್ರಗಳಲ್ಲಿ ಕೂಲಂಕುಷವಾಗಿ ವಿಚಾರಿಸಿರುವರು. ಮತ್ತೊಮ್ಮೆ ಭಾರತವರ್ಷ ಉಳಿದ ಜನಾಂಗಗಳೆದುರಿಗೆ ತಲೆ ಎತ್ತಿ ನಿಲ್ಲಬೇಕಾದರೆ, ನಮ್ಮ ಕುಂದುಕೊರತೆಗಳನ್ನು ಹೇಗೆ ಸರಿಪಡಿಸಿಕೊಳ್ಳಬೇಕು ಎಂಬುದನ್ನು ತಿಳಿಯಪಡಿಸಿರುವರು. ಇವರ ಭಾಷೆ ನೇರವಾಗಿ ಹೃದಯದಿಂದ ಬಂದ ಭಾಷೆ, ನೇರವಾಗಿ ಹೃದಯಕ್ಕೆ ತೂರುವುದು. ಪ್ರಜ್ವಲಿಸುವ ಭಾವಾಗ್ನಿಕುಂಡದಿಂದ ಸಿಡಿದುಬಂದ ಕಿಡಿಗಳಂತಿವೆ ಆ ಪತ್ರಗಳು. ಅವು ಒಂದು ಶಕ್ತಿ ಸಂಜೀವಿನಿ. ತಮ್ಮ ಶಿಷ್ಯರೆದೆಯಲ್ಲಿ ಬಿತ್ತಿದ ಉಜ್ವಲ ದೇಶಪ್ರೇಮದ ಬೀಜಗಳು ಅವು. ದೀನರ ಪದದಳಿತರ ದುಃಖಿಗಳ ನಿರ್ಭಾಗ್ಯರ ಸ್ಥಿತಿಗೆ ಮನಕರಗುವಂತೆ ಮಾಡಿದ ನಡೆನುಡಿಗಳು ಅವು. ತಮ್ಮ ಸ್ವಂತ ಸುಖಕ್ಕೆ ತಿಲತರ್ಪಣ ಕೊಟ್ಟು, ದೀನರ ಸೇವೆಗಾಗಿ ಸೊಂಟಕಟ್ಟಿ ನಿಲ್ಲುವಂತೆ ಪ್ರೇರೇಪಿಸಿದ ಮಹಾವಾಕ್ಯಗಳು. ಅವುಗಳಲ್ಲಿ ಕೆಲವನ್ನು ಕೆಳಗೆ ಉಲ್ಲೇಖಿಸುವೆವು. ಅದರ ಭಾವವನ್ನೇ ಕೆಲವು ಕಾಲ ಸವಿಯೋಣ. 

 ಸ್ವಾಮೀಜಿಯವರು ಅಮೇರಿಕಾ ತಲುಪಿದರು. ಆದರೆ ವಿಶ್ವಧರ್ಮ ಸಮ್ಮೇಳನದಲ್ಲಿ ಮೊದಲು ಮಾತನಾಡಲು ಅವಕಾಶ ಸಿಕ್ಕಲಿಲ್ಲ. ಜೇಬಿನಲ್ಲಿದ್ದ ಪುಡಿಕಾಸು ಬೇಗ ಮಾಯವಾಗುತ್ತಿತ್ತು. ಯಾವ ಸ್ನೇಹಿತರೂ ಇನ್ನೂ ಅವರಿಗೆ ಅಲ್ಲಿ ಸಿಕ್ಕಿರಲಿಲ್ಲ. ಆ ಸಮಯದಲ್ಲಿ ಮದ್ರಾಸಿನ ಆಳಸಿಂಗ ಪೆರುಮಾಳ್ ಅವರಿಗೆ ಬರೆದ ಪತ್ರದಲ್ಲಿ ಅವರಲ್ಲಿ ಎಂತಹ ದುರ್ದಮ್ಯ ಶಕ್ತಿ ಸಾಹಸಗಳಿವೆ ಎಂಬುದನ್ನು ನೋಡುತ್ತೇವೆ: 

 “ನಿರಾಶನಾಗದಿರು, ‘ಕೆಲಸ ಮಾಡುವುದಕ್ಕೆ ನಿನಗೆ ಅಧಿಕಾರವಿದೆ. ಅದರಿಂದ ಬರುವ ಫಲಗಳಿಗಲ್ಲ’ ಎಂದು ಶ‍್ರೀಕೃಷ್ಣ ಗೀತೆಯಲ್ಲಿ ಸಾರುತ್ತಾನೆ. ವತ್ಸ, ಸೊಂಟಕಟ್ಟಿ ನಿಲ್ಲು, ಸಿದ್ಧನಾಗು. ಈ ಮಹಾಕಾರ‍್ಯಕ್ಕೇ ದೇವರು ನನ್ನನ್ನು ಕರೆದಿರುವನು. ಕಷ್ಟ ನಷ್ಟಗಳ ಕಣಿವೆಯಲ್ಲಿ ನಾನು ಇಡೀ ಜೀವನವನ್ನು ನಡೆಸಿರುವೆನು. ಬಹಳ ಹತ್ತಿರದ ಬಂಧುಗಳು ಪ್ರಿಯತಮರು ಉಪವಾಸದಿಂದ ನರಳಿ ಸಾಯುವುದನ್ನು ನೋಡಿರುವೆನು. ಯಾರು ನನ್ನನ್ನು ದೂರತ್ತಾರೊ ಅವರ ಮೇಲೆ ನನಗಿರುವ ಅನುತಾಪಕ್ಕಾಗಿ ಅಣಕಿಸಿರುವೆನು. ನನ್ನ ಮೇಲೆ ನಂಬಿಕೆ ಕಳೆದುಕೊಂಡು ಇರುವರು. ನನ್ನ ಅನುತಾಪಕ್ಕೆ ನಾನೇ ಕಷ್ಟಪಟ್ಟಿರುವೆನು. ಎಲೈ ವತ್ಸ, ಇದು ಒಂದು ದುಃಖದ ಶಿಕ್ಷಾಲಯ. ಹೃದಯದಲ್ಲಿ ದಯೆ ಇರಲಿ, ಸಹನೆ ಇರಲಿ, ನಮ್ಮ ಕಾಲಿನ ಕೆಳಗೆ ಇಡೀ ಪೃಥ್ವಿ ಪುಡಿ ಪುಡಿ ಆದರೂ ಅಂಜದಿರುವ ವಜ್ರೋಪಮ ದೃಢಸಂಕಲ್ಪ ಇರಬೇಕು. ಇಂತಹ ಶೀಲಗಳನ್ನು ಪೋಷಿಸುವ ಮಹಾವ್ಯಕ್ತಿಗಳ ಶಿಕ್ಷಾಲಯ ಈ ಜಗತ್ತು. ನನ್ನನ್ನು ದೂರುವವರಿಗೆ ದಯೆ ತೋರುತ್ತೇನೆ. ಇದು ಅವರ ತಪ್ಪಲ್ಲ. ಅವರು ಮಕ್ಕಳು, ಇನ್ನೂ ಎಳೆ ಹಸುಳೆಗಳು. ಸಮಾಜದಲ್ಲಿ ಉನ್ನತ ಪದವಿಯಲ್ಲಿದ್ದರೇನು? ತಮ್ಮ ಮುಂದಿರುವ ಕೆಲವು ಅಡಿಗಳಲ್ಲದೆ ಅವರಿಗೆ ಮತ್ತೇನೂ ಕಾಣದು. ಊಟ ಮಾಡುವುದು, ಕುಣಿಯುವುದು, ದ್ರವ್ಯಾರ್ಜನೆ ಮಾಡುವುದು, ಮಕ್ಕಳನ್ನು ಪಡೆಯುವುದು ಇದೇ ಒಂದಾದ ಮೇಲೊಂದು ಲೆಕ್ಕಾಚಾರದಂತೆ ಬರುವುದು. ಇದರಲ್ಲೇ ಅವರ ದಿನಚರಿ ಕೊನೆಗಾಣುವುದು. ಇದಕ್ಕಿಂತ ಮೇಲಿನದನ್ನು ಅವರು ಕಾಣರು. ಪಾಪ! ಸುಖಾಭಿಲಾಷಿತರಾದ ಅಲ್ಪ ಜೀವಿಗಳು! ಅವರ ನಿದ್ರೆಗೆ ಎಂದೂ ಭಂಗವಿಲ್ಲ. ಅನೇಕ ಶತಮಾನದ ಕಿರುಕುಳದ ಪರಿಣಾಮವಾಗಿ ದುಃಖಾತ್ಮರ ಮತ್ತು ದೀನರ ಗೋಳು ಭರತಖಂಡವನ್ನೆಲ್ಲ ಆವರಿಸಿದ್ದರೂ ಈ ಸುಖೀ ಜೀವಿಗಳ ಸ್ವರ್ಣಸ್ವಪ್ನ ಭಂಗವಾಗಿಲ್ಲ. ಅನೇಕ ಶತಮಾನಗಳಿಂದಲೂ ಕೊಟ್ಟ ದೈಹಿಕ ಮಾನಸಿಕ ನೈತಿಕ ಕಿರುಕುಳಗಳಿಂದ ದೇವರ ಪ್ರತಿಬಿಂಬದಂತೆ ಇದ್ದ ಮಾನವನನ್ನು ಹೊರೆಹೊರುವ ಮೃಗಕ್ಕೆ ಸರಿಸಮಾನ ಮಾಡಿರುವರು. ಜಗನ್ಮಯಿಯ ಸಾದೃಶ್ಯರಾದ ಸ್ತ್ರೀಯರನ್ನು ಮಕ್ಕಳನ್ನು ಹೆರುವ ದಾಸಿಯರನ್ನಾಗಿ ಮಾಡಿರುವರು. ಮಾನವನ ಬಾಳನ್ನು ಒಂದು ಶಾಪದಂತೆ ಮಾಡಿರುವರು. ಇದು ಅವರಿಗೆ ಸ್ವಲ್ಪವೂ ಹೊಳೆಯುವುದಿಲ್ಲ. 

 “ಚಳಿಯಿಂದ ಹಸಿವಿನ ಬಾಧೆಯಿಂದ ನಾನು ಇಲ್ಲಿ ನಾಶವಾಗಬಹುದು. ತರುಣರೇ, ದೀನರನ್ನು ಮೂಢರನ್ನು ಪದದಳಿತರನ್ನು ಮೇಲಕ್ಕೆ ತರುವ ಎದೆಕರಗುವ ಮರುಕ, ಇದೇ ನಾನು ನಿಮಗೆ ಬಿಡುವ ಆಸ್ತಿ. ಈ ಕ್ಷಣವೇ ನೀನು ಪಾರ್ಥಸಾರಥಿ ಗುಡಿಗೆ ಹೋಗು. ದೀನರೂ ಆರ್ತರೂ ಆದ ಗೋಪಾಲರ ಬಂಧು ಆತ. ರಾಮಾವತಾರದಲ್ಲಿ ಗುಹನನ್ನು ಕೂಡ ಆಲಿಂಗನ ಮಾಡಿಕೊಳ್ಳಲು ಹಿಂಜರಿಯಲಿಲ್ಲ. ಅನೇಕ ರಾಜಪುತ್ರರ ಆಹ್ವಾನವನ್ನು ಕೂಡ ನಿರಾಕರಿಸಿ, ವೇಶ್ಯಾಂಗನೆಯ ಸತ್ಕಾರವನ್ನು ಸ್ವೀಕರಿಸಿ ಬುದ್ಧಾವತಾರದಲ್ಲಿ ಅವರನ್ನು ಉದ್ಧಾರಮಾಡಿದನು. ಆತನ ಮುಂದೆ ಸಾಷ್ಟಾಂಗ ಪ್ರಣಾಮಮಾಡು. ಮಹಾತ್ಯಾಗಮಾಡು. ಯಾರ ಉದ್ಧಾರಕ್ಕೆ ಅನೇಕ ವೇಳೆ ಜನ್ಮವೆತ್ತಿಬಂದನೋ, ಯಾರು ಎಲ್ಲರಿಗಿಂತ ಹೆಚ್ಚಾಗಿ ದೀನರನ್ನು ನಮ್ರರನ್ನು ದಲಿತರನ್ನು ಪ್ರೀತಿಸಿದನೊ, ಅವರ ಉದ್ಧಾರಕ್ಕೆ ನಿನ್ನ ಜೀವನವನ್ನು ನಿವೇದಿಸು. ಪ್ರತಿದಿನವೂ ಅಧಃಪಾತಾಳಕ್ಕೆ ಹೋಗುತ್ತಿರುವ ಮೂವತ್ತು ಕೋಟಿ ಭಾರತೀಯರ ಉದ್ಧಾರಕ್ಕೆ ಆಯುಷ್ಯವನ್ನೆಲ್ಲ ವಿನಿಯೋಗಿಸುತ್ತೇನೆ ಎಂದು ಶಪಥಮಾಡು, ಬದ್ಧ ಕಂಕಣನಾಗು.

 “ಇದು ಒಂದು ದಿನದಲ್ಲಿ ಆಗುವ ಕಾರ‍್ಯವಲ್ಲ. ದಾರಿಯಲ್ಲಿ ಅತ್ಯಂತ ಭಯಂಕರವಾದ ಕಂಟಕಗಳಿವೆ. ಪಾರ್ಥಸಾರಥಿಯೂ ನಮ್ಮ ಸಾರಥಿ ಆಗುವುದಕ್ಕೆ ಸಿದ್ಧನಾಗಿರುವನು. ಅದು ನಮಗೆ ಗೊತ್ತಿದೆ. ಆತನ ಹೆಸರನ್ನು ಹಿಡಿದು ಆತನಲ್ಲಿ ಅನಂತ ವಿಶ್ವಾಸವಿಟ್ಟು ಅನೇಕ ಶತಮಾನಗಳಿಂದ ಮೇಲೆ ಮೇಲೆ ಮುತ್ತಿಕೊಂಡಿರುವ ಪಾಪಕ್ಕೆ ಕಿಚ್ಚನ್ನು ಇಡೋಣ. ಇದು ಭಸ್ಮೀಭೂತವಾಗಿ ಬೀಳುವುದು. ಹಾಗಾದರೆ ಬನ್ನಿ ಪ್ರತ್ಯಕ್ಷ ನೋಡಿ. ಇದು ಪವಿತ್ರ ಕಾರ್ಯ. ನಾವು ಹೀನಸ್ಥಿತಿಯಲ್ಲಿ ಇರುವೆವು. ಆದರೇನು? ಜ್ಞಾನದ ಸಂತಾನರು ನಾವು. ದೇವರ ಮಕ್ಕಳು ನಾವು. ಭಗವಂತನಿಗೆ ಜಯವಾಗಲಿ, ನಾವು ಜಯಶೀಲರಾಗಿಯೇ ಆಗುವೆವು. ನೂರಾರು ಜನ ಸಾಯುವರು ಈ ಹೋರಾಟದಲ್ಲಿ. ಮತ್ತೆ ನೂರಾರು ಜನ ಅವರ ಸ್ಥಳವನ್ನು ತೆಗೆದುಕೊಳ್ಳಲು ಸಿದ್ಧರಾಗಿರುವರು. ನಾನು ಜಯಶೀಲನಾಗದೆ ಸಾಯಬಹುದು, ಮತ್ತೊಬ್ಬನು ಆ ಕಾರ್ಯಕ್ಕೆ ಕೈಹಾಕುವನು. ನಿನಗೆ ರೋಗದ ಸ್ಥಿತಿ ಗೊತ್ತಿದೆ. ದೇವರಲ್ಲಿ ನಂಬಿಕೆಯನ್ನು ಮಾತ್ರ ಇಡು. ಹಣವಂತರಿಗೆ ದೊಡ್ಡವರಿಗೆ ಕಾದು ಕುಳಿತುಕೊಳ್ಳಬೇಡ. ವಿಶ್ವಾಸ, ಸಹಾನುಭೂತಿ, ಅಗ್ನಿಮಯ ವಿಶ್ವಾಸ ಅಗ್ನಿಮಯ ಸಹಾನುಭೂತಿ! ಈ ಬಾಳೇನು! ಸಾವೇನು! ಹಸಿವೇನು! ಚಳಿಯೇನು! ಜಯ ಭಗವಾನ್! ಮುಂದಿಡಿ ಹೆಜ್ಜೆಯನ್ನು. ದೇವರೇ ನಮ್ಮ ಸೇನಾನಾಯಕ. ಯಾರು ಬಿದ್ದರೆಂದು ನೋಡಲು ಹಿಂತಿರುಗಬೇಡಿ. ಒಬ್ಬನು ಬೀಳುವನು, ಮತ್ತೊಬ್ಬನು ಆ ಸ್ಥಳಕ್ಕೆ ಬರುವನು.” 

 ಮುಂದೆ ಬರುವುದೆಲ್ಲ ತಮ್ಮ ಹಲವು ಶಿಷ್ಯರಿಗೆ ಮತ್ತು ಗುರುಭಾಯಿಗಳಿಗೆ ಭರತಖಂಡದ ಹಲವು ಸಮಸ್ಯೆಗಳನ್ನು ಕುರಿತು ಚಿಂತಿಸಿ ಬರೆದ ಪತ್ರಗಳಿಂದ ಆಯ್ದ ಭಾಗಗಳು: 

 ಜಾತಿ ಮತ್ತು ಇತರ ವಿಷಯಗಳಲ್ಲಿ ಸ್ವಾಮೀಜಿ ಹೀಗೆ ಬರೆಯುವರು: “ಅತಿ ದೀನರೂ ಮತ್ತು ದರಿದ್ರರೂ ಆದವರ ಮನೆಯ ಬಾಗಿಲಿಗೆ, ಭರತಖಂಡದಲ್ಲಿ ಮತ್ತು ಹೊರಗೆ ಮಹಾವ್ಯಕ್ತಿಗಳು ಆಲೋಚಿಸಿದ ಮಹಾಭಾವನೆಗಳನ್ನು ಕೊಡಬೇಕೆಂಬುದೇ ನನ್ನ ಬಯಕೆ. ಅನಂತರ ಅವರೇ ಆಲೋಚಿಸಲಿ. ಜಾತಿ ಇರಬೇಕೆ ಬೇಡವೆ, ಸ್ತ್ರೀಯರು ಸಂಪೂರ್ಣ ಸ್ವತಂತ್ರರಾಗಬೇಕೆ ಬೇಡವೆ ಎಂಬುದಕ್ಕೂ ನನಗೂ ಸಂಬಂಧವಿಲ್ಲ. ಜೀವನ, ಅದರ ಬೆಳವಣಿಗೆ ಮತ್ತು ಆರೋಗ್ಯದಿಂದ ಇರುವುದಕ್ಕೆ ಸ್ವತಂತ್ರ ಆಲೋಚನೆ ಮತ್ತು ಕಾರ್ಯ ಆವಶ್ಯಕ. ಇದು ಎಲ್ಲಿ ಇಲ್ಲವೋ ಅಲ್ಲಿಯ ಮನುಷ್ಯ ಅಲ್ಲಿಯ ಜನಾಂಗ, ಅಲ್ಲಿಯ ದೇಶ ಅವನತಿಗೆ ಇಳಿಯಬೇಕು.

 “ಎಲ್ಲಿಯವರೆಗೆ ಸ್ವತಂತ್ರ ಆಲೋಚನೆ ಮತ್ತು ಕ್ರಿಯೆ ನೆರೆಯವರಿಗೆ ತೊಂದರೆ ಕೊಡುವುದಿಲ್ಲವೋ‌, ಅಲ್ಲಿಯವರೆಗೆ, ಯಾವ ಪಂಗಡ ಜಾತಿ ದೇಶ ಸಂಸ್ಥೆ ಆ ಸ್ವಾತಂತ್ರ್ಯಕ್ಕೆ ಅಡ್ಡಿ ತರುವುದೋ ಅದು ಪೈಶಾಚಿಕವಾದುದು. ಅದು ನಿರ್ನಾಮವಾಗಬೇಕು.

 “ಒಂದು ದೇಶದ ಭವಿಷ್ಯ ನಿಂತಿರುವುದು ಅಲ್ಲಿಯ ವಿಧವೆಯರಿಗೆ ಸಿಕ್ಕುವ ಗಂಡಂದಿರ ಸಂಖ್ಯೆಯ ಮೇಲಲ್ಲ. ಜನಸಾಮಾನ್ಯರ ಸ್ಥಿತಿಯ ಮೇಲೆ ಇರುವುದು. ಅವರನ್ನು ಮೇಲೆತ್ತಬಲ್ಲಿರಾ? ಅವರೊಂದಿಗೆ ಬಂದಿರುವ ಆಧ್ಯಾತ್ಮಿಕ ಪ್ರವೃತ್ತಿಯನ್ನು ನಾಶಮಾಡದೆ ಅವರು ಕಳೆದುಕೊಂಡ ವ್ಯಕ್ತಿತ್ವವನ್ನು ಅವರಿಗೆ ಪುನಃ ಕೊಡಬಲ್ಲಿರಾ? ಸ್ವಾತಂತ್ರ್ಯದಲ್ಲಿ ಕ್ರಿಯೆಯಲ್ಲಿ ಶಕ್ತಿಯಲ್ಲಿ ಪಾಶ್ಚಾತ್ಯರನ್ನು ಕೂಡ ಮೀರಿಸಿ, ಅದೇ ಸಮಯದಲ್ಲಿ ಧಾರ್ಮಿಕ ಸಂಸ್ಕೃತಿಯ ಸ್ವಭಾವಗಳಲ್ಲಿ ಪೂರ್ಣ ಭಾರತೀಯರಾಗಬಲ್ಲಿರಾ? ಇದನ್ನು ನೆರವೇರಿಸಬೇಕಾಗಿದೆ. ಇದನ್ನು ನಾವು ಮಾಡಿಯೇ ಮಾಡುವೆವು. ಇದನ್ನೆಲ್ಲ ಸಾಧಿಸುವುದಕ್ಕೆ ನೀವು ಜನ್ಮ ತಾಳಿರುವುದು. ಮೊದಲು ನಿಮ್ಮಲ್ಲಿ ನಂಬಿಕೆ ಇರಲಿ, ಮಹತ್ವಪೂರ್ಣವಾದ ದೃಢನಂಬಿಕೆ ಮಹಾಕಾರ್ಯಕ್ಕೆ ಮೂಲ. ಮುಂದೆ, ಯಾವಾಗಲೂ ಮುಂದೆ ನಡೆಯಿರಿ. ಪ್ರಾಣಹೋದರೂ ಸರಿ, ದೀನರಿಗೆ ದಲಿತರಿಗೆ ದಯೆ ತೋರುವುದು ನಮ್ಮ ಗುರಿ. ನನ್ನ ಕೆಚ್ಚೆದೆಯ ಮಕ್ಕಳೆ, ಮುಂದೆ.

 “ಪ್ರತಿ ಜೀವವೂ ಪಾವನವಾದುದು. ಅದು ಬ್ರಹ್ಮಸ್ವರೂಪವೆಂದು ನಾವು ನಂಬುತ್ತೇವೆ. ಪ್ರತಿಯೊಂದು ಜೀವವೂ ಕೂಡ ಅಜ್ಞಾನ ಮೋಡದಿಂದ ಕವಿದುಕೊಂಡಿರುವ ಸೂರ್ಯನಂತೆ. ಒಂದು ಜೀವಿಗೂ ಮತ್ತೊಂದು ಜೀವಿಗೂ ಇರುವ ವ್ಯತ್ಯಾಸವೇ ಇದನ್ನು ಮುಚ್ಚಿಕೊಂಡಿರುವ ಮೋಡದಲ್ಲಿದೆ. ಕೆಲವೆಡೆ ಮೋಡದ ಆವರಣ ಮಂದವಾಗಿರುವುದು, ಕೆಲವೆಡೆ ಮೋಡದ ಆವರಣ ತೆಳುವಾಗಿರುವುದು. ತಿಳಿದೋ ತಿಳಿಯದೆಯೋ ಇದು ಸರ್ವಧರ್ಮಗಳ ತಳಹದಿ ಎಂದು ನಾವು ನಂಬುತ್ತೇವೆ. ಭೌತಿಕ ಮಾನಸಿಕ ಮತ್ತು ಆಧ್ಯಾತ್ಮಿಕ ಪ್ರಪಂಚದಲ್ಲಿ ಮಾನವನ ಉನ್ನತಿಯ ಸಮಸ್ತ ಚರಿತ್ರೆಗೂ ಇದೇ ವ್ಯಾಖ್ಯಾನ, ಇದೇ ಸಮಾಧಾನ. ಒಂದೇ ಆತ್ಮ ಉಪಾಧಿ ಭೇದಗಳ ಮೂಲಕ ಪ್ರಕಾಶಿಸುತ್ತಿದೆ.

 “ಸಹೋದರನೆ, ನನಗೆ ಬುದ್ಧಿ ಬಂದಿದೆ. ಸುಮ್ಮಸುಮ್ಮನೆ ಇತರರು ಮುಂದೆ ಬರುವುದಕ್ಕೆ ಅಡ್ಡಿ ಮಾಡುವವರು ಎಂತಹವರೋ ನಾವು ಅರಿಯೆವು. ನಾವು ಎಲ್ಲದರಿಂದಲೂ ಪಾರಾಗಬಹುದು, ಆದರೆ ಈ ಹೊಟ್ಟೆಕಿಚ್ಚನ್ನು ಬಿಡಲು ಸಾಧ್ಯವಿಲ್ಲ. ಇನ್ನೊಬ್ಬರನ್ನು ನಿಂದಿಸುವುದು, ಮತ್ತೊಬ್ಬರನ್ನು ನೋಡಿ ಕರುಬುವುದು, ನನ್ನದೇ ದೊಡ್ಡಸ್ತಿಕೆ, ಇತರರು ದೊಡ್ಡವರಾಗಕೂಡದು! ಇವೆಲ್ಲ ನಮ್ಮ ದೇಶದ ಹುಟ್ಟುಗುಣ.

 “ನಮ್ಮಂತಹ ಕೂಪಮಂಡೂಕಗಳನ್ನು ನಾನೆಲ್ಲಿಯೂ ನೋಡಿಲ್ಲ. ಪರದೇಶದಿಂದ ಏನು ಹೊಸ ವಿಷಯ ಬರಲಿ ಅದನ್ನು ಅಮೇರಿಕಾ ಮೊದಲು ಸ್ವೀಕರಿಸುವುದು. ಆದರೆ ನಾವೊ! ನಮ್ಮ ಹಾಗೆ ಪ್ರಪಂಚದಲ್ಲಿ ಯಾರೂ ಇಲ್ಲ! ಆರ್ಯ ಸಂತಾನರಲ್ಲವೆ ನಾವು!‌ಆ ಹುಟ್ಟುಗುಣ ಎಲ್ಲಿ ಕಾಣಿಸುತ್ತಿದೆಯೋ ನನಗೆ ತಿಳಿಯದು. ಆದರೂ ನಾವೂ ಆರ್ಯ ಸಂತಾನರು!

 “ಸುಶಿಕ್ಷಿತರಾದ ಯುವಕರ ಮೇಲೆ ನಿಮ್ಮ ಪ್ರಭಾವವನ್ನು ಬೀರಿ. ಅವರೆಲ್ಲರನ್ನೂ ಒಟ್ಟಿಗೆ ಕರೆದುಕೊಂಡು ಬಂದು ಅವರಿಗೆ ಕೆಲಸದ ಅಭ್ಯಾಸವನ್ನು ಕೊಡಿ. ಮಹಾಕಾರ‍್ಯಗಳು ಮಹಾತ್ಯಾಗದಿಂದ ಮಾತ್ರ ಸಿದ್ಧಿಸುವುವು. ಸ್ವಾರ್ಥ ಬೇಡ, ಹೆಸರು ಬೇಡ, ಕೀರ್ತಿ ಬೇಡ, ನಿಮಗೂ ಬೇಡ, ನಮಗೂ ಬೇಡ, ನಮ್ಮ ಗುರುದೇವನಿಗೂ ಬೇಡ. ಕೆಚ್ಚೆದೆಯುಳ್ಳವರು, ಉದಾರಮನಸ್ಸಿನವರು ಆದ ಯುವಕರಾಗಿ! ನಿಮ್ಮ ಆದರ್ಶವನ್ನು ಕಾರ್ಯರೂಪಕ್ಕೆ ತನ್ನಿ. ಸೊಂಟಕಟ್ಟಿ ಸಿದ್ಧರಾಗಿ, ಹೆಗಲು ಕೊಡಿ ಕಾರ್ಯಕ್ಕೆ. ಕೀರ್ತಿ ಗೌರವಗಳೆಂಬ ಹೀನ ವಿಷಯಗಳನ್ನು ಹಿಂತಿರುಗಿ ನೋಡುವುದಕ್ಕೂ ನಿಲ್ಲಬೇಡಿ. ಸ್ವಾರ್ಥತೆಯನ್ನು ಸಂಪೂರ್ಣ ತ್ಯಾಗ ಮಾಡಿ ಕೆಲಸಕ್ಕೆ ಕೈಹಾಕಿ. ‘ಹುಲ್ಲಿನ ಎಸಳನ್ನು ಸೇರಿಸಿ ಹಗ್ಗಮಾಡಿದರೆ ಮದಿಸಿದ ಆನೆಯನ್ನಾದರೂ ಕಟ್ಟಿಹಾಕಬಹುದು’ ಎನ್ನುವುದನ್ನು ನೆನಪಿನಲ್ಲಿಡಿ. ದೇವರು ನಿಮ್ಮೆಲ್ಲರಿಗೂ ಆಶೀರ್ವದಿಸಲಿ. ನಾನಾಗಲೇ ತಿಳಿದುಕೊಂಡಿರುವಂತೆ ಅವನ ಶಕ್ತಿ ನಿಮ್ಮೆಲ್ಲರಲ್ಲಿಯೂ ಸಂಚರಿಸಲಿ. ‘ಜಾಗ್ರತರಾಗಿ, ಗುರಿಸೇರುವವರೆಗೂ ನಿಲ್ಲಬೇಡಿ’ ಎಂದು ಸಾರುವುದು ವೇದಗಳು. ಏಳಿ, ಏಳಿ, ದೀರ್ಘನಿದ್ರೆ ಕಳೆಯುತ್ತಿದೆ. ಉದಯ ಸನ್ನಿಹಿತವಾಗುತ್ತಿದೆ. ಅಲೆ ಮೇಲೆದ್ದಿದೆ. ಅದರ ರಭಸವನ್ನು ತಡೆಯಲು ಯಾರಿಗೂ ಶಕ್ತಿ ಸಾಲದು. ಉತ್ಸಾಹ, ನನ್ನ ಹುಡುಗರಿರಾ! ಉತ್ಸಾಹ! ಪ್ರೇಮ! ಶ್ರದ್ಧಾಭಕ್ತಿಗಳು. ಯಾವುದಕ್ಕೂ ಅಂಜಬೇಡಿ. ಅಂಜಿಕೆಯೇ ಮಹಾಪಾಪ.

 “ಪೌರ್ವಾತ್ಯ ಪಾಶ್ಚಾತ್ಯ ದೇಶಗಳಿಗೆ ಇರುವ ಮುಖ್ಯ ವ್ಯತ್ಯಾಸ ಇದು. ಅವರು ಒಂದು ಜನಾಂಗ, ನಾವು ಒಂದು ಜನಾಂಗವಾಗಿಲ್ಲ. ವಿದ್ಯೆ ಮತ್ತು ಸಂಸ್ಕೃತಿ ಇಲ್ಲಿ ಸಾಧಾರಣವಾಗಿದೆ. ಸಮಾಜದ ಕೆಳಗೆ ಇರುವವರಿಗೂ ಕೂಡ ತಲಪುತ್ತದೆ. ನಮ್ಮಲ್ಲಿ ಯಾರಾದರೂ ಪ್ರಖ್ಯಾತ ಪುರುಷರು ಕಾಲವಾದರೆ ಮತ್ತೊಂದು ಅಂತಹ ವ್ಯಕ್ತಿ ಬರುವುದಕ್ಕೆ ಅನೇಕ ಶತಮಾನಗಳು ಕಾದು ಕುಳಿತುಕೊಳ್ಳಬೇಕು. ಅಲ್ಲಾದರೊ ಸತ್ತ ಮರುಕ್ಷಣವೇ ದೇಶ ಅಂತಹ ಪ್ರಖ್ಯಾತ ಪುರುಷರನ್ನು ಮುಂದಕ್ಕೆ ತರುವುದು. ಅವರಿಗೆ ಮುಂದಾಳುಗಳನ್ನು ಆರಿಸುವುದಕ್ಕೆ ಕ್ಷೇತ್ರ ವಿಶಾಲವಾಗಿದೆ. ನಮ್ಮಲ್ಲಿ ಅದು ಬಹಳ ಕಿರಿದು. ಮೂವತ್ತು ಕೋಟಿ ಜನಸಂಖ್ಯೆಯುಳ್ಳ ಭರತಖಂಡದಲ್ಲಿ ಮುಂದಾಳುಗಳನ್ನು ಹುಡುಕಬೇಕಾದರೆ ಎಲ್ಲೋ ಅಲ್ಪ ಮಂದಿಗಳ ಪೈಕಿ. ಅಲ್ಲಿ ಮೂರು ನಾಲ್ಕು ಅಥವಾ ಐದು ಕೋಟಿ ಜನಸಂಖ್ಯೆ ಇದ್ದರೂ ವಿದ್ಯಾವಂತರು ಮತ್ತು ಸುಸಂಸ್ಕೃತರು ಹೆಚ್ಚಾಗಿರುವುದರಿಂದ ಮುಂದಾಳುಗಳನ್ನು ಚುನಾಯಿಸುವುದಕ್ಕೆ ನಮಗಿಂತ ಹೆಚ್ಚು ವಿಶಾಲವಾದ ಕ್ಷೇತ್ರವಿದೆ.

 “ಸತ್ಯ ಪ್ರೇಮ ನಿಷ್ಕಾಪಟ್ಯದ ಎದುರಿಗೆ ಯಾವುದೂ ನಿಲ್ಲಲಾರದು. ನೀನು ನಿಷ್ಕಪಟಿಯೇ? ಸಾಯುವವರೆಗೆ ಸ್ವಾರ್ಥರಹಿತವಾಗಿರುವೆಯಾ? ಹಾಗಿದ್ದರೆ ಭಯಪಡಬೇಡ, ಮೃತ್ಯುವಿಗೂ ಅಂಜಬೇಡ. ಮುಂದೆ ನನ್ನ ಹುಡುಗರೆ, ಮುಂದೆ! ಇಡಿ ಜಗತ್ತಿಗೇ ಜ್ಞಾನ ಜ್ಯೋತಿ ಬೇಕಾಗಿದೆ, ಅದು ನಿರೀಕ್ಷಿಸುತ್ತಿದೆ. ಕೇವಲ ಭರತಖಂಡ ಮಾತ್ರ ಅದನ್ನು ಕೊಡಬಲ್ಲದು. ಇಂದ್ರಜಾಲ, ಮಂತ್ರತಂತ್ರ, ಬೂಟಾಟಿಕೆಗಳಿಂದ ಅಲ್ಲ. ನಿಜವಾದ ಧಾರ್ಮಿಕ ಚೈತನ್ಯ ಮಹಾತ್ಮೆಯನ್ನು, ಅತ್ಯುನ್ನತವಾದ ಆಧ್ಯಾತ್ಮಿಕ ಸತ್ಯಗಳನ್ನು ಕೊಡಬಲ್ಲದು. ಇದಕ್ಕೇ ದೇವರು ಎಲ್ಲಾ ಸಂದಿಗ್ಧ ಪರಿಸ್ಥಿತಿಯಿಂದಲೂ ಈ ಜನಾಂಗವನ್ನು ಇಂದಿನವರೆಗೆ ಕಾಪಾಡಿಕೊಂಡು ಬಂದಿರುವನು. ಈಗ ಸಕಾಲವಾಗಿದೆ. ನನ್ನ ಕೆಚ್ಚೆದೆಯ ಹುಡುಗರೇ, ನೀವೆಲ್ಲ ಮಹತ್ಕಾರ್ಯಗಳನ್ನು ಮಾಡುವುದಕ್ಕೆ ಜನ್ಮವೆತ್ತಿರುವಿರಿ ಎಂಬುದರಲ್ಲಿ ವಿಶ್ವಾಸವನ್ನು ಇಡಿ. ನಾಯಿಗಳ ಬೊಗಳುವಿಕೆ ನಿಮ್ಮನ್ನು ಬೆದರಿಸದಿರಲಿ. ಅಷ್ಟೇ ಏಕೆ? ಸ್ವರ್ಗದ ಸಿಡಿಲುಗಳಿಗೂ ಅಂಜಬೇಡಿ. ಎದ್ದೇಳಿ, ಕಾರ‍್ಯೋನ್ಮುಖರಾಗಿ.” 

 “ನಾನು ನಿಜವಾಗಿ ಇಲ್ಲಿಯ ಹೆಂಗಸರನ್ನು ನೋಡಿ ಆಶ್ಚರ್ಯಚಕಿತನಾಗಿರುವೆನು. ಜಗನ್ಮಯಿಯು ಅವರ ಮೇಲೆ ಎಷ್ಟೊಂದು ಕರುಣೆಯನ್ನು ಬೀರಿದ್ದಾಳೆ! ಅತ್ಯಾಶ್ಚರ್ಯಕರವಾದ ಸ್ತ್ರೀಯರಿವರು, ಪರಸ್ಪರ ಪೈಪೋಟಿಯಲ್ಲಿ ಸೋತ ಗಂಡಸನ್ನು ಮೂಲೆಗೊತ್ತುವುದರಲ್ಲಿ ಇರುವರು. ಜಗನ್ಮಯಿ, ಇದೆಲ್ಲ ನಿನ್ನ ಅನುಗ್ರಹದಿಂದ. ಈ ಲಿಂಗಭೇದಗಳಿಂದ ಬಂದಿರುವ ಮೇಲು ಕೀಳೆಂಬ ವಿವಾದವನ್ನು ಸಂಪೂರ್ಣ ನಾಶಮಾಡುವವರೆಗೆ ನಾನು ಶಾಂತನಾಗಲಾರೆ. ಆತ್ಮನಲ್ಲಿ ಏನಾದರೂ‌ ಲಿಂಗಭೇದವಿದೆಯೇನು? ಗಂಡಸಿಗೂ ಹೆಂಗಸಿಗೂ ಇರುವ ಭೇದಭಾವ ಹೋಗಬೇಕು. ಎಲ್ಲವೂ ಆತ್ಮಮಯ. ದೇಹ ನಾನೆಂಬ ಭ್ರಮೆಯನ್ನು ಬಿಡಿ. ಅಸ್ತಿ, ಅಸ್ತಿ. ಎಲ್ಲವೂ ಇದೆ ಎಂದು ಧೈರ್ಯವಾಗಿ ಹೇಳಿ. ಸ್ಪಷ್ಟವಾದ ಉತ್ತಮ ಭಾವಗಳನ್ನು ಮನಸ್ಸಿನಲ್ಲಿಡಿ. ನಾಸ್ತಿ ನಾಸ್ತಿ, ಇಲ್ಲ ಇಲ್ಲ ಎಂಬ ನಿಷೇದಾರ್ಥವನ್ನು ಕೊಡುವ ಪದಗಳನ್ನು ಪಠಿಸಿ ಪಠಿಸಿ ದೇಶವೆಲ್ಲ ಅಧೋಗತಿಗೆ ಇಳಿಯುತ್ತಿದೆ. ಸೋಽಹಂ, ಸೋಽಹಂ, ಶಿವೋಽಹಂ ಎಂದು ಹೇಳಿ. ಪ್ರತಿಯೊಂದು ಆತ್ಮನಲ್ಲಿಯೂ ಅನಂತ ಶಕ್ತಿ ನೆಲಸಿದೆ. ಯಾವಾಗಲೂ ಇಲ್ಲ ಇಲ್ಲ ಎಂಬ ಭಾವವನ್ನು ಮನಸ್ಸಿನಲ್ಲಿಟ್ಟುಕೊಂಡು ಬೆಕ್ಕು ನಾಯಿಗಳಂತೆ ಆಗಬೇಕೇನು ನೀವು? ಯಾರಿಗೆ ನಿಷೇಧವನ್ನು ಬೋಧಿಸುವುದಕ್ಕೆ ಧೈರ್ಯವಿದೆ? ಯಾರನ್ನು ನೀವು ದುರ್ಬಲರು ಶಕ್ತಿಹೀನರು ಎಂದು ಕರೆಯುವುದು? ಶಿವನೇ ನಾನು. ಎಂದು ಜನರು ನಾಸ್ತಿ ಎಂಬ ಭಾವವನ್ನು ಮೆಲ್ಲುತ್ತಾರೋ ಆಗ ನನ್ನ ತಲೆಗೆ ಒಂದು ಸಿಡಿಲು ಬಡಿದಂತೆ ಆಗುವುದು. ಆತ್ಮನಿಂದೆ ಮಾಡಿಕೊಳ್ಳುವುದು ಮತ್ತೊಂದು ಜಾಡ್ಯ. ಇದನ್ನು ನಮ್ರತೆ ಎಂದು ಕರೆಯುವಿರೇನು? ಇದು ಮರೆಮಾಡಿದ ಅಹಂಕಾರ. ‘ನ ಲಿಂಗಂ ಧರ‍್ಮಕಾರಣಂ ಸಮತಾ ಸರ್ವಭೂತೇಷು, ಏತನ್ಮುಕ್ತಸ್ಯ ಲಕ್ಷಣಂ.’ ‘ಅಸ್ತಿ, ಅಸ್ತಿ, ಸೋಽಹಂ ಚಿದಾನಂದರೂಪಃ ಶಿವೋಽಹಂ ಶಿವೋಽಹಂ.’ ಇದೆ, ಇದೆ ನಾನೇ ಅವನು, ಚಿದಾನಂದರೂಪನಾದ ಶಿವನು. ‘ನಿರ್ಗಚ್ಛತಿ ಜಗಜ್ಜಾಲಾತ್ ಪಿಂಜರಾದಿವ ಕೇಸರೀ' - ಸಿಂಹ ತನ್ನ ನಖದಿಂದ ಪಂಜರವನ್ನು ಸೀಳಿ ಬರುವಂತೆ ಮಾಯಾ ಪ್ರಪಂಚದ ಬಂಧನಗಳಿಂದ ಅವನು ಸೀಳಿಬರುವನು. ‘ನಾಯಮಾತ್ಮಾ ಬಲಹೀನೇನ ಲಭ್ಯಃ’ - ಬಲಹೀನರಿಗೆ ಆತ್ಮ ಲಭಿಸುವುದಿಲ್ಲ. ಮಹಾ ಹಿಮರಾಶಿಯು ಪರ್ವತದ ಮೇಲೆ ಬೀಳುವಂತೆ ಜಗದಮೇಲೆ ಬೀಳಿ. ನಿಮ್ಮ ಭಾರಕ್ಕೆ ಭೂಮಿ ಕುಸಿದು ಚೂರಾಗಲಿ. ಹರಹರ ಮಹಾದೇವ! ಉದ್ಧರೇದಾತ್ಮನಾತ್ಮಾನಂ-ತನ್ನನ್ನು ತನ್ನಿಂದಲೇ ಉದ್ಧರಿಸಿಕೊಳ್ಳಬೇಕು. ಪುರುಷಕಾರದ ಬಲದಿಂದ ಸ್ವಪ್ರಯತ್ನದ ಸಾಹಸದಿಂದ ನಾವಿರುವ ಬಂಧನದಿಂದ ಪಾರಾಗಬೇಕು.” 

 “ಪರರ ಹಿತದಲ್ಲಿ ನಮ್ಮ ಜೀವನ ನಶಿಸಿ ಹೋಗುವಂತಹ ಸುದಿನ ಎಂದು ಬರುವುದು? ಪ್ರಪಂಚವೆಂಬುದು ಮಕ್ಕಳಾಟವಲ್ಲ. ಹಿಂದೆ ಬರುವವರಿಗೆ ಗುರಿಯೆಡೆಗೆ ಪ್ರಯಾಣ ಮಾಡುವುದಕ್ಕೆ ರಾಜಮಾರ್ಗವನ್ನು ತಮ್ಮ ದೇಹದ ರಕ್ತವನ್ನಾದರೂ ಧಾರೆಯೆರೆದು ಯಾರು ಮಾಡುತ್ತಾರೆಯೋ ಅವರು ಮಹಾಪುರುಷರು. ಜಗದಾದಿಯಿಂದಲೂ ಹೀಗೆಯೇ ನಡೆಯುತ್ತಿದೆ. ತನ್ನ ದೇಹವನ್ನು ಅಡ್ಡಲಾಗಿರಿಸಿ ಸೇತುವೆಯನ್ನು ಒಬ್ಬ ಕಟ್ಟುವನು. ಇದರ ಸಹಾಯದಿಂದ ಸಾವಿರಾರು ಜನರು ಸಂಸಾರವೆಂಬ ನದಿಯನ್ನು ದಾಟುವರು.

 “ನನ್ನ ಸಖನೆ’ ಯಾವುದು ನಿನ್ನನ್ನು ಆಳುವಂತೆ ಮಾಡುವುದು? ಎಲ್ಲಾ ಶಕ್ತಿಯೂ ನಿನ್ನಲ್ಲಿದೆ. ನಿನ್ನ ಅನಂತ ಶಕ್ತಿಯ ಪ್ರಭಾವವನ್ನು ಪ್ರಕಟಗೊಳಿಸು. ಎಲೈ ಮಹಾನುಭಾವನೆ! ಈ ಪ್ರಪಂಚವೆಲ್ಲ ನಿನ್ನ ಪದತಳಕ್ಕೆ ಬರುವುದು. ಆತ್ಮ ಒಂದೇ ಮುಖ್ಯ, ಜಡವಸ್ತುವಲ್ಲ.

 “ನಿರ್ಭೀತಿಯಿಂದ ಕಾರ್ಯೋನ್ಮುಖರಾಗಿ. ಏನು ಅಂಜಿಕೆ! ನಿನಗೆ ಅಡ್ಡಿ ತರುವುದಕ್ಕೆ ಯಾರಿಗೆ ಶಕ್ತಿ ಇದೆ? ನಕ್ಷತ್ರಗಳನ್ನು ಕೂಡ ಧೂಳೀಪಟ ಮಾಡುವೆವು. ಜಗತ್ತಿನ ಕಟ್ಟನ್ನು ಕೂಡ ಸಡಿಲಿಸುವೆವು. ನಾವು ಯಾರೆಂಬುದು ನಿಮಗೆ ಗೊತ್ತಿಲ್ಲವೆ? ಶ‍್ರೀರಾಮಕೃಷ್ಣರ ದಾಸರು ನಾವು. ಅಂಜಿಕೆ, ಯಾರಿಗೆ ಅಂಜಿಕೆ! ಖಂಡಿತ ಯಾರಿಗೂ ಇಲ್ಲ!” 

 “ಈ ಪರದೇಶೀಯರು ನಮ್ಮ ಬೋಧನೆಯನ್ನು ಅರ್ಥಮಾಡಿಕೊಳ್ಳುವರು. ಸ್ವದೇಶೀಯರಾದ ನೀವು ‘ಇಲ್ಲ’ ಎಂಬ ಆಲೋಚನೆಯಲ್ಲಿ ಮಗ್ನರಾಗಿ ಕಷ್ಟದಲ್ಲಿ ನರಳುತ್ತಿರುವಿರಿ. ನಿನಗೆ ರೋಗವೆಂದು ಯಾರು ಹೇಳುವರು? ರೋಗಕ್ಕೂ ನಿನಗೂ ಸಂಬಂಧವೇನು? ಇಂತಹ ರೋಗದ ಆೇಲೋಚನೆಗಳನ್ನು ನೂಕು. ‘ವೀರ‍್ಯಮಸಿ ವೀರ್ಯಂ, ಬಲಮಸಿ ಬಲಂ, ಓಜೋ ಅಸಿ ಓಜೋ ಮಯಿ ದೇಹಿ, ಸಹೋ ಅಸಿ ಸಹೋ ಮಯಿ ದೇಹಿ.’ ಹೇ ದೇವ! ನೀನು ಶಕ್ತಿ, ಶಕ್ತಿಯನ್ನು ನೀಡೆನಗೆ; ನೀನು ಬಲ, ಬಲವನ್ನು ನೀಡೆನಗೆ; ನೀನು ಓಜಸ್ಸು, ಓಜಸ್ಸನ್ನು ನೀಡೆನಗೆ. ನೀನು ಧೈರ್ಯ, ಧೈರ್ಯವನ್ನು ನೀಡೆನಗೆ. ದೇವರ ಪೂಜೆಗೆ ಕುಳಿತುಕೊಳ್ಳುವಾಗ ಪ್ರತಿದಿನವೂ ಮಾಡುವ ಆಸನ ಶುದ್ಧಿಯ ಮಂತ್ರದಲ್ಲಿ ಬರುವ ‘ಆತ್ಮಾನಮಚ್ಛಿದ್ರಂ ಭಾವಯೇತ್’ - ನಾನು ಆತ್ಮ, ನನ್ನನ್ನು ಯಾರೂ ಜಯಿಸಲಾರರು ಎಂದು ಭಾವಿಸಬೇಕು. ಹೀಗೆಂದರೆ ಏನು ಅರ್ಥ? ಎಂದರೆ ಎಲ್ಲವೂ ನನ್ನಲ್ಲಿದೆ. ನನ್ನ ಇಚ್ಛೆಯಿಂದ ಅದನ್ನು ವ್ಯಕ್ತಪಡಿಸಬಲ್ಲೆ. ಯಾರು ರೋಗದಿಂದ ನರಳುತ್ತಿರುವರೋ ಅವರೆಲ್ಲರೂ ಆತ್ಮಸ್ವರೂಪರು. ಅನಂತಾತ್ಮರು ಅವರು. ಅವರಿಗೆ ರೋಗ ಹೇಗೆ ಬರುವುದೆಂದು ಆಲೋಚಿಸು. ಕೆಲವು ದಿನಗಳು, ಇದನ್ನು ಕೆಲವು ಗಂಟೆಗಳ ಹೊತ್ತು ಆಲೋಚಿಸು! ರೋಗಗಳೆಲ್ಲ ಮಾಯವಾಗುವುವು.” 

 “ಕೊಟ್ಟು ತೆಗೆದುಕೊಳ್ಳುವುದೇ ಜೀವನದ ನಿಯಮ. ಭಾರತವರ್ಷ ಮತ್ತೊಮ್ಮೆ ಉನ್ನತಿಯನ್ನು ಪಡೆಯಬೇಕೆಂಬ ಆಸೆ ಇದ್ದ ಪಕ್ಷದಲ್ಲಿ ಭರತಮಾತೆಯು ತನ್ನ ಭಂಡಾರವನ್ನು ತೆಗೆದು ಅನರ್ಘ್ಯ ರತ್ನಗಳನ್ನು ಜಗದ ಇತರ ಜನಾಂಗಗಳಿಗೆ ದಾನಮಾಡಬೇಕು. ಇದಕ್ಕೆ ಪ್ರತಿಯಾಗಿ ಇತರರು ತಮಗೆ ಯಾವುದನ್ನು ಕೊಡುವರೋ ಅದನ್ನು ಸ್ವೀಕರಿಸುವುದು ಅತ್ಯಾವಶ್ಯಕ. ವಿಕಾಸವೇ ಬಾಳು. ಸಂಕೋಚವೇ ಸಾವು. ಪ್ರೇಮವೇ ಪ್ರಾಣ, ದ್ವೇಷವೇ ಮರಣ. ಎಂದು ನಾವು ಇತರ ಜನಾಂಗಗಳನ್ನು ದ್ವೇಷಿಸುವುದಕ್ಕೆ ಮೊದಲು ಮಾಡಿದೆವೋ ಅಂದಿನಿಂದಲೇ ನಮ್ಮ ಅವನತಿ ಪ್ರಾರಂಭವಾಯಿತು. ನಮ್ಮ ಹೃದಯ ವಿಕಾಸವಾಗಬೇಕು. ಇದು ಚೇತನದ ಚಿಹ್ನೆ. ಹಾಗಲ್ಲದೇ ಇದ್ದರೆ ಯಾರೂ ನಮ್ಮ ನಾಶವನ್ನು ತಡೆಯಲಾರರು.” 

 “ನನ್ನ ಮಗು, ಯಾರು ಇನ್ನೊಬ್ಬರನ್ನು ಪ್ರೀತಿಸುವನೊ ಅವನು ಮಾತ್ರ ಬದುಕಿರುವನು. ಅವನೇ ಜೀವಂತನು. ನನ್ನ ಮಕ್ಕಳಿರಾ!‌ ಬಡವರಿಗೆ ಪದದಳಿತರಿಗೆ, ಮೂರ್ಖರಿಗೆ ಮನಮರುಗಿ. ನಿಮ್ಮ ಎದೆಯ ಚಲನೆ ನಿಲ್ಲುವವರೆಗೆ, ತಲೆತಿರುಗಿ ಎಲ್ಲಿ ಹುಚ್ಚರಾಗಿಹೋಗುತ್ತೇವೋ ಎನ್ನುವವರೆಗೆ, ಅವರಿಗೆ ಅನುಕಂಪವನ್ನು ತೋರಿ. ತರುವಾಯ ನಿಮ್ಮ ಎದೆಯ ಭಾರವನ್ನೆಲ್ಲ ಜಗದೀಶನ ಅಡಿದಾವರೆಯಲ್ಲಿಡಿ. ಬಲ ಸಹಾಯ ಎಂದಿಗೂ ಕುಗ್ಗದ ಶಕ್ತಿ ಆಗ ಬರುವುದು. ಹೋರಾಟ, ಹೋರಾಟ ಎಂಬುದೇ ಕಳೆದ ಹತ್ತು ವರ್ಷಗಳಿಂದಲೂ ನನ್ನ ಆದರ್ಶವಾಗಿತ್ತು. ನಾನು ಈಗಲೂ ಹೇಳುವುದೇ ಹೋರಾಡಿ ಎಂದು. ಸುತ್ತಲೂ ಗಾಢಾಂಧಕಾರ ಕವಿದಿರುವಾಗ ಹೋರಾಡಿ ಎನ್ನುತ್ತಿದ್ದೆ. ಇನ್ನೇನು ಬೆಳಕು ಬರುತ್ತಿದೆ ಅನ್ನುವಾಗಲೂ ಹೋರಾಡಿ ಅನ್ನುವೆನು. ನನ್ನ ಮಕ್ಕಳಿರಾ, ಎಂದಿಗೂ ಭಯಪಡಬೇಡಿ. ಅನಂತ ತಾರಾಖಚಿತನಭ ಎಲ್ಲಿ ನಮ್ಮ ಮೇಲೆ ಬೀಳುವುದೋ ಎಂದು ಅಂಜಿಕೆಯಿಂದ ಓಡಬೇಡಿ. ಸ್ವಲ್ಪ ತಡೆಯಿರಿ, ಇನ್ನು ಕೊಂಚ ಕಾಲದಲ್ಲಿ ಅವುಗಳೆಲ್ಲ ಬಂದು ನಿಮ್ಮ ಪಾದದಡಿ ಬೀಳುವುವು. ಸ್ವಲ್ಪ ತಾಳಿ. ಐಶ್ವರ್ಯದಿಂದ ಪ್ರಯೋಜನವಿಲ್ಲ ವಿದ್ಯೆಯಿಂದ ಪ್ರಯೋಜನವಿಲ್ಲ. ಪ್ರೀತಿ ಮಾತ್ರ ಫಲಕಾರಿಯಾಗುವುದು. ಕಡುಕಷ್ಟವೆಂಬ ಗೋಡೆಯನ್ನು ಭೇದಿಸಿಕೊಂಡು ಮುಂದೆ ಹೋಗುವುದು ಚಾರಿತ್ರ್ಯ ಶುದ್ಧಿ, ನಿರ್ಮಲಶೀಲ.” 

 “ಉತ್ಸಾಹಪೂರಿತರಾಗಿ, ಉತ್ಸಾಹದ ಕಿಡಿಯನ್ನು ಎಲ್ಲೆಲ್ಲಿಯೂ ಹರಡಿ. ದುಡಿಯಿರಿ, ದುಡಿಯಿರಿ, ಮುಂದಾಳಾಗಿ ನಿಂತರೂ ಸೇವಕನಂತಿರಬೇಕು, ನಿಸ್ವಾರ್ಥಪರರಾಗಿರಿ,” 

 “ಠಕ್ಕಾಬಿಕ್ಕಿ, ರಹಸ್ಯ, ನೀಚತನ, ಗೂಢತೆ, ಮೋಸ ಇವುಗಳು ಇರಕೂಡದು… ನನ್ನ ಕೆಚ್ಚೆದೆಯ ಯುವಕರಿರಾ! ಮುಂದೆ ನಡೆಯಿರಿ-ಹಣವಿರಲಿ ಇಲ್ಲದೆ ಇರಲಿ, ಜನರಿರಲಿ ಇಲ್ಲದೇ ಇರಲಿ, ಪ್ರೀತಿ ಇದೆಯೇನು ನಿಮ್ಮಲ್ಲಿ? ದೇವರಲ್ಲಿ ನಂಬಿಕೆ ಇದೆಯೇನು ನಿಮ್ಮಲ್ಲಿ? ನುಗ್ಗಿ ನಡೆಯಿರಿ ಮುಂದೆ, ನಿಮ್ಮನ್ನು ತಡೆಯುವವರಾರು?” 

 “ಎಚ್ಚರಿಕೆ. ಸುಳ್ಳು ಯಾವ ವೇಷದಲ್ಲಿ ಬರಲಿ ಜೋಕೆ!” 

 “ನೀವು ನಿಜವಾಗಿ ನನ್ನ ಮಕ್ಕಳೇ ಆಗಿದ್ದರೆ, ನೀವು ಯಾವುದಕ್ಕೂ ಅಂಜುವುದಿಲ್ಲ. ಏನೇ ಆದರೂ ಕೆಲಸ ನಿಲ್ಲಿಸುವುದಿಲ್ಲ. ನೀವು ಸಿಂಹಸದೃಶರಾಗುವಿರಿ. ನಾವು ಭರತಖಂಡವನ್ನು ಹುರಿದುಂಬಿಸಿ ಪ್ರಚೋದಿಸಬೇಕು. ಹೇಡಿತನ ಬೇಡ. ನಾನು ನಕಾರವನ್ನು ಒಪ್ಪುವುದಿಲ್ಲ. ನಿಮಗೆ ಅರ್ಥವಾಯಿತೆ? ಸಾಯುವವರೆಗೆ ಸತ್ಯವಂತರಾಗಿರಿ ಗುರುಭಕ್ತಿಯೇ ಇದರ ರಹಸ್ಯ!” 

 “ನಾನು ಬದುಕಿದ್ದರೇನು ಸತ್ತರೇನು? ನಾನು ಭರತಖಂಡಕ್ಕೆ ಹೋದರೇನು, ಹೋಗದೆ ಇದ್ದರೇನು? ನಿಮ್ಮ ಪ್ರೇಮವನ್ನು ಹರಡಿ. ಸಾವು ಎಂದಿಗೂ ನಿಶ್ಚಯವಾಗಿರುವಾಗ ಒಂದು ಕಾರ್ಯಕ್ಕೆ ದುಡಿಯುವುದು ಮೇಲು.” 

 “ನಿನಗೆ ಮುಕ್ತಿ ದೊರಕದೆ ಇದ್ದರೇನಂತೆ? ಎಂತಹ ಮಕ್ಕಳಾಟ ಇದು! ಹೇ ಜಗದೀಶ! ಕೃಷ್ಣ ಸರ್ಪದ ವಿಷವನ್ನೂ ಕೂಡ ಕಚ್ಚಲಿಲ್ಲವೆಂದು ಸಾಧಿಸಿದರೆ ಇಲ್ಲದಂತೆ ಮಾಡಬಹುದು ಎಂದು ಹೇಳುತ್ತಾರೆ. ‘ನನಗೇನೂ ಗೊತ್ತಿಲ್ಲ. ನಾನು ಯಾವ ಕೆಲಸಕ್ಕೂ ಬಾರದ ವಸ್ತು’ ಎಂದು ಹೇಳಿಕೊಳ್ಳುವುದು ಎಂತಹ ದೈನ್ಯತೆ! ಇದು ಕಪಟ ತ್ಯಾಗ, ಹಾಸ್ಯಾಸ್ಪದವಾದ ವಿನಯ ಎಂದು ಹೇಳುತ್ತೇನೆ. ಇಂತಹ ಆತ್ಮ ಅವಹೇಳನದ ಸ್ಥಿತಿಯಿಂದ ಪಾರಾಗು.” 

 “ಬೈರಾಗಿಗಳು ಸಂನ್ಯಾಸಿಗಳು, ಒಂದು ಪಂಗಡದ ಬ್ರಾಹ್ಮಣರು ಈ ದೇಶವನ್ನು ಹಾಳುಮಾಡಿರುವರು. ಕಳ್ಳತನ ಮೋಸ ಇವುಗಳನ್ನೆಲ್ಲ ಆಲೋಚಿಸುತ್ತ ಈ ಜನರು ಧರ್ಮಬೋಧಕರಂತೆ ನಟಿಸುವರು. ಇತರ ಜನರಿಂದ ದಾನವನ್ನು ಸ್ವೀಕರಿಸುವರು. ತಕ್ಷಣವೇ ನನ್ನನ್ನು ಮುಟ್ಟಬೇಡ ಎಂದು ಕಿರುಚುವರು. ಎಂತಹ ಮಹತ್ಕಾರ‍್ಯವನ್ನು ಅವರು ಮಾಡುತ್ತಿರುವರು! ಏನಾದರೂ ಒಂದು ಆಲೂಗೆಡ್ಡೆಯನ್ನು ಬದನೆಕಾಯಿಯನ್ನು ಮುಟ್ಟಿದರೆ ಇನ್ನೇನು ಪ್ರಳಯ ಸನ್ನಿಹಿತವಾಗಿದೆ! ಪ್ರಪಂಚ ಇನ್ನೆಷ್ಟು ದಿನ ಬಾಳುತ್ತದೆಯೊ! ತಮ್ಮ ಕೈಯನ್ನು ತೊಳೆದುಕೊಳ್ಳುವುದಕ್ಕೆ ಮಣ್ಣನ್ನು ಹನ್ನೆರಡು ಸಲ ಉಪಯೋಗಿಸದೇ ಇದ್ದರೆ ಹದಿನಾಲ್ಕು ತಲೆತಲಾಂತರದವರೆಗೂ ನರಕಕ್ಕೆ ಹೋಗಿ ಬಿಡುವರು! ಕಳೆದ ಎರಡು ಸಾವಿರ ವರ್ಷಗಳಿಂದ ನಾಲ್ಕನೆ ಒಂದು ಪಾಲು ಜನ ಸಾಯುತ್ತಿರುವಾಗ, ಇಂತಹ ಗಹನ ಸಮಸ್ಯೆಗಳಿಗೆ ಶಾಸ್ತ್ರೀಯವಾದ ವಿವರಣೆಯನ್ನು ಕಂಡುಹಿಡಿಯುತ್ತಿರುವರು! ಎಂಟು ವರ್ಷದ ಹುಡುಗಿಯನ್ನು ಮೂವತ್ತು ವರ್ಷದ ಮುದುಕನಿಗೆ ಕೊಟ್ಟು ಮದುವೆಮಾಡಿ ತಂದೆ ತಾಯಿಗಳು ಹಿಗ್ಗುವರು. ಇದನ್ನು ಯಾರಾದರೂ ವಿರೋಧಿಸಿದರೆ, ‘ನಮ್ಮ ಧರ್ಮವನ್ನು ತಲೆಕೆಳಗೆ ಮಾಡುತ್ತಿರುವರು’ ಎನ್ನುವರು. ಇವುಗಳನ್ನೆಲ್ಲಾ ನಾನು ಏತಕ್ಕೆ ಹೇಳಿದೆ ಎಂದರೆ ಪೂರ್ವಾಚಾರದಲ್ಲಿ ಎಷ್ಟೋ ಒಳ್ಳೆಯದಿವೆ. ಅದರಂತೆಯೇ ಎಷ್ಟೋ ಕೆಲಸಕ್ಕೆ ಬಾರದವುಗಳೂ ಇವೆ. ಒಳ್ಳೆಯದನ್ನು ನಾವು ಇಟ್ಟುಕೊಳ್ಳಬೇಕು. ಮುಂದೆ ಬರುವ ಭಾರತ ಪುರಾತನ ಭಾರತಕ್ಕಿಂತ ಅತಿಶಯವಾಗಿರಬೇಕು.” 

 “ಪ್ರತಿಯೊಂದು ಜೀವಕ್ಕೆ ಅದರ ಅಂತರಾಳದಲ್ಲಿರುವ ದೈವತ್ವವನ್ನು ವ್ಯಕ್ತಪಡಿಸುವಂತೆ ಮಾಡುವ ಕಾರ‍್ಯವೆಲ್ಲ ಒಳ್ಳೆಯದೆ. ಆ ದೈವತ್ವವನ್ನು ಯಾವುದು ಮುಚ್ಚುವುದೋ ಅದೆಲ್ಲ ಕೆಟ್ಟದ್ದು.” 

 “ನಮ್ಮಲ್ಲಿ ದೈವತ್ವದ ಪ್ರಕಾಶಕ್ಕೆ ಮಾರ್ಗವೊಂದೇ ಇರುವುದು. ಅದೇ ಮತ್ತೊಬ್ಬರ ದೈವತ್ವದ ಪ್ರಕಾಶಕ್ಕೆ ಮಾರ್ಗವನ್ನು ಕಲ್ಪಿಸುವುದು.” 

 “ವರ್ತಮಾನ ಕಾಲದ ಹಿಂದೂಧರ್ಮ ವೇದದಲ್ಲಿ ಇಲ್ಲ, ಪುರಾಣದಲ್ಲಿ ಇಲ್ಲ, ಭಕ್ತಿಯಲ್ಲಿಲ್ಲ, ಮುಕ್ತಿಯಲ್ಲಿಲ್ಲ. ಆ ಧರ್ಮವೆಲ್ಲ ಅಡಿಗೆ ಮಾಡುವ ಪಾತ್ರೆಯೊಳಗೆ ಪ್ರವೇಶಿಸಿದೆ. ಇಂದಿನ ಹಿಂದೂಧರ್ಮ ಜ್ಞಾನಮಾರ್ಗವೂ ಅಲ್ಲ ಭಕ್ತಿಮಾರ್ಗವೂ ಅಲ್ಲ. ಅವರ ಧರ್ಮವೇ ಅಸ್ಪೃಶ್ಯತೆ, ನನ್ನನ್ನು ಮುಟ್ಟಬೇಡಿ ಎಂಬುದರಲ್ಲೇ ಅವರ ವರ್ಣನೆ ಮುಗಿಯುತ್ತದೆ. ಅಸ್ಪೃಶ್ಯತೆ ಎಂಬ ಅಂಧ ಅಧರ್ಮದಲ್ಲಿ ನಿಮ್ಮ ಜೀವನ ನಾಶವಾಗದಿರಲಿ…” 

 “ಹಸಿದವನಿಗೆ ಒಂದು ತುತ್ತು ಅನ್ನವನ್ನು ಕೊಡದವನು ಮೋಕ್ಷ ಹೇಗೆ ಕೊಟ್ಟಾನು? ಮತ್ತೊಬ್ಬರ ಉಸಿರಿನ ಸೋಂಕಿನಿಂದಲೇ ಇವನು ಪಾಪಿಯಾಗುವ ಹಾಗಿದ್ದರೆ ಇನ್ನೊಬ್ಬರನ್ನು ಹೇಗೆ ಶುದ್ಧಿಮಾಡುವನು? ಅಸ್ಪೃಶ್ಯತೆ ಎಂಬುದು ಒಂದು ತರದ ಮನೋರೋಗ. ಎಚ್ಚರಿಕೆ! ವಿಕಾಸವೇ ಜೀವನ, ಸಂಕೋಚವೇ ಮರಣ.” 

 “ನಾವೆಲ್ಲ ಸೊಂಟಕಟ್ಟಿ ನಿಲ್ಲದೇ ಇದ್ದರೆ ಏನಾದರೂ ಕೆಲಸ ಸಾಧ್ಯವೆ? ಎದ್ದು ಕಾರ‍್ಯೋನ್ಮುಖರಾಗಿ. ಯಾರು ಧೀರರೊ ಸಾಹಸಿಗಳೊ ಅಂತಹವರನ್ನು ಅದೃಷ್ಟ ದೇವತೆ ಒಲಿಯುವುದು.” 

 “ದುಃಖವೆಲ್ಲ ಅಜ್ಞಾನದಿಂದ ಜನಿಸುವುದು. ಜಗತ್ತಿಗೆ ಜ್ಞಾನವನ್ನು ಯಾರು ಕೊಡುವರು? ಹಿಂದೆ ಆತ್ಮತ್ಯಾಗವೇ ನಿಯಮವಾಗಿತ್ತು. ಅಯ್ಯೋ, ಮುಂದೆ ಬರುವ ಅನೇಕ ಯುಗಗಳಲ್ಲಿಯೂ ಇದೇ ನಿಯಮವಾಗುವುದು. ಜಗತ್ತಿನ ಅತಿ ಧೀರರು ಮತ್ತು ಪುಣ್ಯಾತ್ಮರು, ಹಲವರ ಹಿತಕ್ಕೋಸುಗವಾಗಿ ಹಲವರ ಮೇಲ್ಮೆಗೋಸುಗವಾಗಿ ತಮ್ಮ ಆತ್ಮವನ್ನು ಅರ್ಪಣೆ ಮಾಡಬೇಕು. ಅನಂತ ಪ್ರೇಮ ಮತ್ತು ಅನಂತ ದಯೆಯಿಂದ ಕೂಡಿದ ನೂರಾರು ಬುದ್ಧರು ಜಗತ್ತಿಗೆ ಬೇಕಾಗಿರುವರು.” 

 “ಜಗದ ಧರ್ಮಗಳೆಲ್ಲ ನಿಸ್ತೇಜವಾಗಿ ನಗೆಗೀಡಾಗಿವೆ. ಜಗತ್ತಿಗೆ ಬೇಕಾಗಿರುವುದು ಶುದ್ಧ ಚಾರಿತ್ರ್ಯ. ಯಾರ ಜೀವನದಲ್ಲಿ ನಿರಂತರವೂ ಪ್ರೇಮ ಸ್ವಾರ್ಥ ತ್ಯಾಗಗಳು ನಂದಾದೀವಿಗೆಯಂತೆ ಉರಿಯುತ್ತಿವೆಯೋ ಅಂತಹ ಪುಣ್ಯಾತ್ಮರು ಜಗತ್ತಿಗೆ ಬೇಕಾಗಿದ್ದಾರೆ. ಆ ಪ್ರೇಮದಿಂದ ಕೂಡಿದ ಪ್ರತಿಯೊಂದು ಮಾತು ಸಿಡಿಲಿನಂತೆ ಪರಿಣಾಮಕಾರಿಯಾಗುವುದು.” 

 “ಮಹಾವ್ಯಕ್ತಿಗಳೇ ಏಳಿ! ಜಾಗ್ರತರಾಗಿ. ಪ್ರಪಂಚ ದುಃಖದಲ್ಲಿ ಬೇಯುತ್ತಿದೆ. ನಿದ್ದೆ ಮಾಡುವಿರೇನು? ನಮಗಿರುವ ದೇವರು ಜಾಗ್ರತನಾಗುವವರೆಗೆ, ಅಂತರ್ಯಾಮಿಯು ನಮ್ಮ ಬೇಡಿಕೆಯನ್ನು ಕೇಳುವವರೆಗೆ, ನಾವು ಅವನನ್ನು ಎಡಬಿಡದೆ ಕರೆಯೋಣ. ಜಗತ್ತಿನಲ್ಲಿ ಇದಕ್ಕಿಂತ ಹೆಚ್ಚು ಪುಣ್ಯ ಕೆಲಸ ಏನಿರುವುದು? ಇದನ್ನು ಮೀರಿದ ಕೆಲಸ ಯಾವುದಿರುವುದು?” 

 ಸ್ವಾಮೀಜಿ ಅಮೇರಿಕಾ ದೇಶದಲ್ಲಿದ್ದಾಗ ಬರೆದ ನುಡಿಗಳಿವು. ಅವರಿಗೆ ಸಿಕ್ಕಿದ ಕೀರ್ತಿ ವೈಭವಗಳಲ್ಲಿ ಭರತಖಂಡದ ದುಃಸ್ಥಿತಿಯನ್ನು ಮರೆಯಲಿಲ್ಲ. ಅವರ ಹೃದಯ ಜ್ವಾಲಾಮುಖಿಯಂತೆ ಕುದಿಯುತ್ತಿರುವುದನ್ನು ನೋಡುವೆವು. ಅದರಿಂದ ಉಕ್ಕಿ ಬಂದ ಶಿಲಾಪ್ರವಾಹದಂತೆ ಇದೆ ಅವರ ಪತ್ರಗಳು. ಭರತಖಂಡವನ್ನು ಅವರು ಗೌರವಿಸಿದರು, ಪ್ರೀತಿಸಿದರು. ಯಾವ ಪ್ರೇಮಿಯೂ ತನ್ನ ಪ್ರಿಯೆಯನ್ನು ಹೀಗೆ ಪ್ರೀತಿಸಲಾರ. ಆದರೆ ಧರ್ಮದ ಹೆಸರಿನಲ್ಲಿ ಹೆಚ್ಚಿರುವ ಕೆಲಸಕ್ಕೆ ಬಾರದ ಅನಾಚಾರಗಳು ನಿಜವಾದ ಧರ್ಮವನ್ನು ಮರೆಸಿಬಿಟ್ಟಿದ್ದವು. ಯಾವ ಉಪನಿಷತ್ತು ಗೀತಾ ಜೀವಿಯಲ್ಲಿ ಪೌರುಷವನ್ನು ತುಂಬಬೇಕೋ, ಜೀವನವನ್ನು ಧೈರ್ಯದಿಂದ ಎದುರಿಸುವ ಪರಾಕ್ರಮವನ್ನು ಕೊಡಬೇಕೋ, ಅದನ್ನೆಲ್ಲ ಮರೆತು ನಾವು ಕೆಲಸಕ್ಕೆ ಬಾರದವರು, ಅಯೋಗ್ಯರು, ಪಾಪಿಗಳು ಎಂದು ಮೆಲ್ಲುತ್ತ ಅಧೋಗತಿಗೆ ಇಳಿದುಹೋಗುತ್ತಿರುವುದನ್ನು ಸ್ವಾಮೀಜಿಗೆ ಸಹಿಸಲು ಅಸಾಧ್ಯವಾಯಿತು. ಅದನ್ನು ನೇರಮಾಡಲು ಪತ್ರಗಳಲ್ಲಿ ತಮ್ಮ ಆ ವೇಗವನ್ನೆಲ್ಲ ಅಭಿಮಂತ್ರಿಸಿ ಮಹಾಸ್ತ್ರದಂತೆ ಭರತಖಂಡದ ಕಡೆ ಬಿಟ್ಟರು. ಅದು ಬರೆದದ್ದು ತಮ್ಮ ನೆಚ್ಚಿನ ಶಿಷ್ಯರಿಗೆ ಇರಬಹುದು. ಆದರೆ ಅವರು ಕೇವಲ ನಿಮಿತ್ತಮಾತ್ರ. ಶ‍್ರೀಕೃಷ್ಣ ಪ್ರಪಂಚಕ್ಕೆ ಗೀತೆಯನ್ನು ಕೊಡಲು ಅರ್ಜುನ ಹೇಗೆ ನಿಮಿತ್ತನಾದನೋ ಹಾಗೆ. ಅದು ಭರತಖಂಡದಲ್ಲಿರುವ ಪ್ರತಿಯೊಬ್ಬರಿಗೂ ಅನ್ವಯಿಸುವುದು. ಇಂದಿಗೂ ಅವರ ಪತ್ರಗಳಲ್ಲಿ ಆ ಕಾವು, ಕಾಂತಿ ಮ್ಲಾನವಾಗದೆ ಧಗಧಗಿಸುತ್ತಿದೆ. 

