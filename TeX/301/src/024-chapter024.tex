
\chapter{ಕೆಲವು ವ್ಯಕ್ತಿಗಳು ಮತ್ತು ಘಟನೆಗಳು }

 ಸ್ವಾಮೀಜಿ ಚಿಕಾಗೊ ನಗರದಲ್ಲಿದ್ದಾಗ ಮೇಡಮ್ ಕಾಲ್ವಿ ಎಂಬ ಪ್ರಸಿದ್ಧ ಗಾಯಕಿ ಅಲ್ಲಿಯ ಅಪೆರಾ ಹೌಸಿನಲ್ಲಿ ಹಾಡುತ್ತಿದ್ದಳು. ಒಂದು ದಿನ ಆಕೆ ಸಂಜೆ ಕಾರ್‍ಮನ್ ಎಂಬ ಗೀತಾ ನಾಟಕದಲ್ಲಿ ಹಾಡುತ್ತಿದ್ದಳು. ಅದಕ್ಕೆ ಮೂರು ಅಂಕಗಳು. ಮೊದಲನೆಯ ಅಂಕವನ್ನು ಪೂರೈಸಿ ತೆರೆಯ ಹಿಂದೆ ಬಂದಳು. ಏನೋ ದೊಡ್ಡ ಜುಗುಪ್ಸೆ ಮನಸ್ಸನ್ನೆಲ್ಲ ವ್ಯಾಪಿಸಿತು; ಹಾಡುವುದಕ್ಕೇ ಮನಸ್ಸಿರಲಿಲ್ಲ. ಆದರೆ ಪ್ರೇಕ್ಷಕರು ಕಿಕ್ಕಿರಿದು ಕುಳಿತಿರುವರು. ಅಂದಿನ ದೃಶ್ಯದಲ್ಲಿ ಕೇವಲ ಮೂರನೆ ಒಂದು ಭಾಗ ಮಾತ್ರ ಆಗಿದೆ. ಮನಸ್ಸಿಲ್ಲದ ಮನಸ್ಸಿನಿಂದ ಪುನಃ ರಂಗಭೂಮಿಯ ಮೇಲೆ ಬಂದು ನಿಂತುಕೊಂಡು ಬಹಳ ಶ್ರಮಪಟ್ಟು ಎರಡನೆಯದನ್ನು ಪೂರೈಸಿ ತೆರೆಯ ಹಿಂದಿನ ಕೋಣೆಗೆ ಹೋಗಿ ಮ್ಯಾನೇಜರ್ ಅನ್ನು ಕರೆದು ‘ನನಗೆ ತುಂಬಾ ನಿತ್ರಾಣವಾಗಿದೆ, ಮುಂದೆ ಹಾಡಲಾರೆ, ಸಭಿಕರಿಗೆ ನನಗೆ ಅನಾರೋಗ್ಯವಾಗಿದೆ ಎಂದು ಹೇಳಿಬಿಡಿ’ ಎಂದು ಹೇಳಿದಳು. ಆದರೆ ಮ್ಯಾನೇಜರ್ ಹಾಗೆ ಮಾಡುವುದಕ್ಕಾಗುವುದಿಲ್ಲವೆಂದೂ, ಹೇಗಾದರೂ ಮಾಡಿ ತಾವು ಪೂರೈಸಿಬಿಡಬೇಕೆಂದೂ ಕೇಳಿಕೊಂಡರು. ಮೇಡಮ್ ಕಾಲ್ವಿ ಕಷ್ಟಪಟ್ಟು ರಂಗಭೂಮಿಯ ಮೇಲೆ ಬಂದು ಉಳಿದ ಭಾಗವನ್ನು ಪೂರೈಸಿದಳು. ಆದರೆ ಅಂದು ಹಾಡಿದಷ್ಟು ಉತ್ತಮ ರೀತಿಯಲ್ಲಿ ಆಕೆ ಎಂದೂ ಹಾಡಿರಲಿಲ್ಲವೆಂದು ಪ್ರೇಕ್ಷಕರೆಲ್ಲ ಹೊಗಳಿದರು. ಆಕೆ ಪೂರೈಸಿದ ತಕ್ಷಣವೇ ತನ್ನ ಕೋಣೆಗೆ ಓಡಿಹೋದಳು. ಅಲ್ಲಿ ಮ್ಯಾನೇಜರ್ ಮತ್ತು ಅವಳ ಮನೆಯಿಂದ ಬಂದವರು ಮತ್ತೆ ಕೆಲವರು ಇದ್ದರು. ಯಾರೂ ಇವಳೊಡನೆ ಮಾತನಾಡಲಿಲ್ಲ. ಸುಮ್ಮನೆ ಇವಳನ್ನು ನೋಡುತ್ತ ಖಿನ್ನರಾಗಿದ್ದರು. ಆಕೆ ಇವರ ಮುಖವನ್ನು ನೋಡಿ ಏನೋ ತನಗೆ ಅಪ್ರಿಯವಾಗಿರುವುದು ಕಾದುಕೊಂಡಿರಬೇಕೆಂದು ಅಂಜತೊಡಗಿದಳು. ಅವರು ಅನಂತರ ಅವಳ ಏಕಮಾತ್ರ ಪುತ್ರಿ ಸ್ವಲ್ಪ ಹೊತ್ತಿನ ಹಿಂದೆ ಬಟ್ಟೆ ಸುಟ್ಟುಕೊಂಡು ಸತ್ತುಹೋದಳು ಎಂಬ ವರದಿಯನ್ನು ಕೊಟ್ಟರು. ಆಕೆಗೆ ಅದನ್ನು ಕೇಳಿ ಸಹಿಸಲಾಗಲಿಲ್ಲ. ಅವಳು ತನ್ನ ಒಬ್ಬಳೇ ಮಗಳನ್ನು ಅಷ್ಟು ಪ್ರೀತಿಸುತ್ತಿದ್ದಳು. ಅವಳಿಲ್ಲದ ಜೀವನ ಶೂನ್ಯವಾಯಿತು. ಅನಂತರ ಹಾಡುವುದನ್ನು ಬಿಟ್ಟುಬಿಟ್ಟಳು. ಯಾವಾಗಲೂ ತನ್ನ ಮಗಳನ್ನೇ ಚಿಂತಿಸುತ್ತ ಹುಚ್ಚಿಯಂತಾದಳು. ಆಕೆಯ ಸ್ನೇಹಿತೆ ಒಬ್ಬಳು ಸ್ವಾಮೀಜಿ ಉಪನ್ಯಾಸಗಳಿಗೆ ಬರುತ್ತಿದ್ದಳು. ಆಕೆ ಸ್ವಾಮೀಜಿಯನ್ನು ಕಂಡು ಮಾತನಾಡಿದರೆ ಮನಸ್ಸಿಗೆ ಸಮಾಧಾನ ಸಿಕ್ಕಬಹುದೆಂದು ಹೇಳಿದಳು. ಆದರೆ ಮೇಡಮ್ ಕಾಲ್ವಿಗೆ ಬೇಕಾಗಿರಲಿಲ್ಲ. ಮಗಳಿಲ್ಲದ ಆ ಬಾಳಿನಿಂದ ಪಾರಾಗಲೆತ್ನಿಸಿದಳು. ಅಲ್ಲಿರುವ ಸರೋವರಕ್ಕೆ ಬಿದ್ದು ಸಾಯಬೇಕೆಂದು ಮೂರು ಸಲ ನಡೆದುಕೊಂಡು ಹೋದಳು. ಪ್ರತಿಸಲವೂ ದಾರಿ ತಪ್ಪಿ ಸ್ವಾಮೀಜಿಯವರಿದ್ದ ತನ್ನ ಸ್ನೇಹಿತೆಯ ಮನೆಯ ಬಾಗಿಲಿಗೆ ಬಂದಳು. ಒಳಗೆ ಹೋಗಿ ಸ್ವಾಮೀಜಿಯವರನ್ನು ನೋಡಲು ಮನಸ್ಸು ಬರಲಿಲ್ಲ. ಕೊನೆಗೆ ಒಂದು ಸಲ ಹಾಗೇ ಅವರ ಮನೆಯ ಮುಂದೆ ಬಂದು ನಿಂತಳು. ಆ ಮನೆಯ ಬಟ್ಲರ್ ಬಾಗಿಲನ್ನು ತೆರೆದ. ಸ್ವಾಮೀಜಿ ಇದ್ದ ಕೋಣೆಗೆ ಆಕೆಯನ್ನು ಕರೆದುಕೊಂಡು ಹೋದ. ಸ್ವಾಮೀಜಿ ಸಮ್ಮುಖದಲ್ಲಿ ಆಕೆ ಕುಳಿತುಕೊಂಡಳು. ಆಕೆ ತನ್ನ ಆತ್ಮಕಥೆಯಲ್ಲಿ ತಾನು ಸ್ವಾಮೀಜಿಯವರನ್ನು ಸಂಧಿಸಿದುದನ್ನು ಹೀಗೆ ವಿವರಿಸುವಳು: 

 “ನಾನು ಅವರ ಕೋಣೆಯನ್ನು ಪ್ರವೇಶಿಸಿದ ಮೇಲೆ ಒಂದು ನಿಮಿಷ ಮೌನವಾಗಿದ್ದೆ. ಅವರು ಧ್ಯಾನಸ್ಥರಾಗಿ ಕುಳಿತಿದ್ದರು. ಅವರ ಮೈಮೇಲಿನ ಗೈರಿಕವಸನ ನೆಲಕ್ಕೆ ತಾಕುತ್ತಿತ್ತು. ರುಮಾಲನ್ನು ಸುತ್ತಿದ ತಲೆ ಸ್ವಲ್ಪ ಮುಂದೆ ವಾಲಿತ್ತು. ಕಣ್ಣುಗಳು ನೆಲದ ಮೇಲೆ ಇದ್ದುವು. ಕ್ಷಣಗಳಾದ ಮೇಲೆ ಅವರು ‘ನನ್ನ ಮಗಳೆ, ನೀನು ಎಂತಹ ದುಗುಡದ ಮನಸ್ಸಿನಿಂದ ಬಂದಿರುವೆ! ಶಾಂತಳಾಗು! ಇದು ನಿನಗೆ ಅತ್ಯಾವಶ್ಯಕ'’ ಎಂದರು.” 

 “ಅನಂತರ ಅನಾಸಕ್ತರಾಗಿ ಯಾವ ಉದ್ವಿಗ್ನತೆಗೂ ಸಿಕ್ಕದೆ ನನ್ನ ಹೆಸರನ್ನು ತಿಳಿದಿರದ ಅವರು ನನ್ನ ಜೀವನದ ಅತ್ಯಂತ ಮಾರ್ಮಿಕವಾದ ಸಂಗತಿಯನ್ನು ಕುರಿತು ಮಾತನಾಡತೊಡಗಿದರು. ನನ್ನ ನಿಕಟ ಸ್ನೇಹಿತರಿಗೂ ಇದು ಗೊತ್ತಿರಲಿಲ್ಲ. ಅದೊಂದು ಅಲೌಕಿಕವಾದ ಅನುಭವವಾಗಿತ್ತು; ಅದ್ಭುತವಾಗಿತ್ತು!” 

 ‘ನಿಮಗೆ ಇದೆಲ್ಲ ಹೇಗೆ ಗೊತ್ತಾಯಿತು? ಯಾರು ಹೇಳಿದರು?’ ಎಂದು ನಾನು ಕೇಳಿದೆ. 

 “ಅವರು ಸುಮ್ಮನೆ ನನ್ನನ್ನು ಮಂದಹಾಸದಿಂದ ನೋಡಿದರು. ಎಲ್ಲೋ ಒಂದು ಮಗು ಕೆಲಸಕ್ಕೆ ಬಾರದ ಪ್ರಶ್ನೆಯನ್ನು ಕೇಳುತ್ತಿರುವಂತೆ ಭಾವಿಸಿದರು.” 

 ‘ನನಗೆ ನಿನ್ನ ವಿಷಯವನ್ನು ಯಾರೂ ಹೇಳಬೇಕಾಗಿಲ್ಲ. ಹಾಗೆ ಮತ್ತೊಬ್ಬರು ನನಗೆ ಹೇಳಬೇಕೇನು? ತೆರೆದ ಪುಸ್ತಕದಂತೆ ನಿನ್ನ ಮನಸ್ಸನ್ನು ಓದಿದೆ’ ಎಂದು ಮೃದುವಾಗಿ ಹೇಳಿದರು. ಅನಂತರ ನಾನು ಎದ್ದು ಹೋಗುವ ಸಮಯದಲ್ಲಿ ಹೀಗೆ ಹೇಳಿದರು: ‘ನೀನು ಅದನ್ನು ಮರೆಯಬೇಕು. ಪುನಃ ಸಂತೋಷಚಿತ್ತದವಳಾಗಬೇಕು, ಆನಂದಭರಿತಳಾಗಬೇಕು. ನಿನ್ನ ಆರೋಗ್ಯವನ್ನು ಉತ್ತಮಪಡಿಸಿಕೊಳ್ಳಬೇಕು. ನಿನ್ನ ಜೀವನದ ದುಃಖವನ್ನೇ ಇನ್ನುಮೇಲೆ ಕುರಿತು ಮೆಲುಕುತ್ತಿರಬೇಡ. ಆ ವ್ಯಾಕುಲವನ್ನು ಮಾರ್ಪಡಿಸು. ಇದು ನಿನ್ನ ಕಲೆಯಲ್ಲಿ ವ್ಯಕ್ತವಾಗಲಿ. ನಿನ್ನ ಆಧ್ಯಾತ್ಮಿಕ ಜೀವನಕ್ಕೆ ಇದು ಆವಶ್ಯಕ. ನಿನ್ನ ಕಲೆಗೆ ಇದು ಆವಶ್ಯಕ.’ ನಾನು ಅವರಿಂದ ಬೀಳ್ಕೊಳ್ಳುವಾಗ ಅವರ ಮಾತು ಮತ್ತು ವ್ಯಕ್ತಿತ್ವ ನನ್ನ ಮನಸ್ಸಿನ ಮೇಲೆ ತುಂಬಾ ಪರಿಣಾಮಕಾರಿಯಾಯಿತು. ನನ್ನ ಮೆದುಳಿನಲ್ಲಿ ಚೆಲ್ಲಾಪಿಲ್ಲಿಯಾಗಿ ಓಡಾಡುತ್ತ ಮನಸ್ಸಿನ ಪ್ರಕ್ಷುಬ್ಧ ಸ್ಥಿತಿಗೆ ಕಾರಣವಾದ ಭಾವಗಳನ್ನೆಲ್ಲ ಓಡಿಸಿ ಸ್ವಾಮೀಜಿಯವರು ತಮ್ಮ ಸ್ವಷ್ಟವಾದ ಶಾಂತಿದಾಯಕವಾದ ಭಾವನೆಯನ್ನು ಇಟ್ಟಂತೆ ತೋರಿತು. 

 “ನಾನು ಅನಂತರ ಹಿಂದಿನಂತೆ ಉತ್ಸಾಹದಿಂದ ತುಂಬಿ ಸಂತೋಷಭರಿತಳಾದೆ. ಸ್ವಾಮೀಜಿಯವರ ಪ್ರಚಂಡ ಇಚ್ಛಾಶಕ್ತಿ ನನ್ನ ಮೇಲೆ ಆ ಪ್ರಭಾವವನ್ನು ತಂದಿತ್ತು. ಅದಕ್ಕೆ ಧನ್ಯವಾದಗಳು. ಅವರು ಯಾವ ಸಮ್ಮೋಹಿನೀ ವಿದ್ಯೆಯನ್ನೂ ನನ್ನ ಮೇಲೆ ಉಪಯೋಗಿಸಿರಲಿಲ್ಲ. ಅವರ ಪುಣ್ಯಚಾರಿತ್ರ್ಯದ ಶಕ್ತಿ ಅವರ ಪವಿತ್ರ ಉದ್ದೇಶಗಳೇ ನನ್ನ ಮನಸ್ಸಿಗೆ ತೃಪ್ತಿಯನ್ನು ಕೊಟ್ಟಿದ್ದುವು. ಅವರ ಪರಿಚಯ ನನಗೆ ಹೆಚ್ಚು ಆದಮೇಲೆ ಅವರು ಯಾರೊಂದಿಗೆ ಮಾತನಾಡಬೇಕೆಂದು ಬಯಸುವರೊ ಅವರ ಚಿತ್ತಕ್ಕೆ ಸ್ವಾ ಸ್ಥ್ಯವನ್ನು ತಂದು ಅನಂತರ ಅವರು ಮಾತನಾಡುತ್ತಿದ್ದರು, ಎಂಬುದು ಗೊತ್ತಾಯಿತು. ಇಲ್ಲದೆ ಇದ್ದರೆ ಪ್ರಕ್ಷುಬ್ಧ ಸ್ಥಿತಿಯಲ್ಲಿದ್ದ ನನ್ನ ಮನಸ್ಸು ಅವರನ್ನು ಅರ್ಥಮಾಡಿಕೊಳ್ಳುವ ಸ್ಥಿತಿಯಲ್ಲಿರಲಿಲ್ಲ.” 

 “ಅವರು ಅನೇಕ ವೇಳೆ ನಮಗೆ ಉತ್ತರವನ್ನು ಕೊಡುವಾಗ ದೃಷ್ಟಾಂತ ಕಥೆಗಳನ್ನು ಹೇಳುತ್ತಿದ್ದರು. ಕಾವ್ಯಮಯವಾದ ಉಪಮಾನಗಳನ್ನು ಉಪಯೋಗಿಸುತ್ತಿದ್ದರು. ಒಂದು ದಿನ ನಾವು ಆತ್ಮನ ಅಮರತ್ವ ಮತ್ತು ಮರಣೋತ್ತರ ಜೀವನ ಇವುಗಳನ್ನು ಕುರಿತು ಮಾತನಾಡುತ್ತಿದ್ದೆವು. ಸ್ವಾಮೀಜಿಯವರು ಪುನರ್ಜನ್ಮದ ಸಿದ್ಧಾಂತವನ್ನು ವಿವರಿಸಿದರು. ಇದು ಅವರ ಬೋಧನೆಯ ಅತ್ಯಂತ ಮುಖ್ಯವಾದ ಭಾಗವಾಗಿತ್ತು.” 

 ‘ನಾನು ನನ್ನ ವ್ಯಕ್ತಿತ್ವವನ್ನು ಕಳೆದುಕೊಳ್ಳುವುದಕ್ಕೆ ಇಚ್ಛೆಯಿಲ್ಲ, ಅದನ್ನು ಚಿಂತಿಸಲೂ ಒಲ್ಲೆ. ಅದು ಎಷ್ಟೇ ಗೌಣವಾಗಿದ್ದರೂ ಅದನ್ನು ಬಿಡಲು ಇಚ್ಛೆಯಿಲ್ಲ. ಅಖಂಡ ಸಚ್ಚಿದಾನಂದದಲ್ಲಿ ನನ್ನನ್ನು ಕರಗಿಸಿಕೊಳ್ಳುವ ಇಚ್ಛೆಯಿಲ್ಲ. ಆ ಭಾವನೆಯೇ ನನಗೆ ಕಳವಳವನ್ನುಂಟುಮಾಡುವುದು’ ಎಂದೆ. 

 ಅದಕ್ಕೆ ಸ್ವಾಮೀಜಿ ಒಂದು ಉಪಮಾನದ ಮೂಲಕ ವಿವರಿಸಿದರು; “ಒಂದು ದಿನ ಒಂದು ನೀರಿನ ಹನಿ ಸಾಗರಕ್ಕೆ ಬಿತ್ತು. ಅದು ಸಾಗರಕ್ಕೆ ಬಿದ್ದು ತನ್ನ ವ್ಯಕ್ತಿತ್ವವನ್ನು ಕಳೆದುಕೊಳ್ಳುತ್ತಿರುವಾಗ ಈಗ ನಿನ್ನಂತೆ ಅದು ಆಳಲು ಮೊದಲು ಮಾಡಿತು. ಮಹಾ ಸಮುದ್ರ ಈ ನೀರಿನ ಬಿಂದುವಿಗೆ ಹೇಳಿತು: ‘ನೀನು ನನ್ನಲ್ಲಿ ಸೇರಿದರೆ ನಿನ್ನ ಬಂಧುಬಳಗದವರನ್ನೆಲ್ಲ ಸೇರುವೆ. ಅವರೆಲ್ಲ ನನ್ನಿಂದಲೇ‌ಆದವರು. ನೀನೂ ಒಂದು ಮಹಾಸಾಗರವೇ ಆಗುವೆ. ನೀನು ನನ್ನನ್ನು ಅಗಲಬೇಕೆಂದು ಇಚ್ಛೆಯಿದ್ದರೆ ಸೂರ್ಯನ ಕಿರಣದ ಮೂಲಕ ಆಕಾಶಕ್ಕೆ ಏಳಬಹುದು. ಅನಂತರ ಭೂಮಿಯ ಮೇಲೆ ಒಂದು ಹನಿಯಂತೆ ಬೀಳಬಹುದು. ಆಗ ನೀನು ಬಾಯಾರಿದ ಭೂಮಿಗೆ ಒಂದು ವರದಂತೆ ಬರುವೆ’.” 

 ಸ್ವಾಮೀಜಿಯವರು ಅಮೇರಿಕಾ ದೇಶವನ್ನು ಬಿಟ್ಟು ಹೋಗುವ ಸಮಯದಲ್ಲಿ ಮೇಡಮ್ ಕಾಲ್ವಿ ಸ್ವಾಮೀಜಿ ಜೊತೆಯಲ್ಲಿ ಯೂರೋಪಿನಲ್ಲೆಲ್ಲ ಸಂಚರಿಸಿ ಅಲೆಗ್ಸಾಂಡ್ರಿಯಾವರೆಗೂ ಬಂದು ಸ್ವಾಮಿಗಳನ್ನು ಬೀಳ್ಕೊಟ್ಟಳು. ಸ್ವಾಮೀಜಿ ನಿರ್ಯಾಣಾನಂತರ ಹಲವು ವರ್ಷಗಳಮೇಲೆ ಅವಳು ಇಂಡಿಯಾ ದೇಶಕ್ಕೆ ಬಂದಿದ್ದಾಗ ಸ್ವಾಮೀಜಿ ಜ್ಞಾಪಕಾರ್ಥವಾಗಿ ಅವರ ಶಿಷ್ಯರು ಕಟ್ಟಿಸಿದ್ದ ಅವರ ಸ್ಮಾರಕವನ್ನು ನೋಡಲು ಬಂದು ಕೆಲವು ಗಂಟೆಗಳು ಇದ್ದು ಹೋದಳು. ಆಕೆ ತನ್ನ ಆತ್ಮ ಕಥೆಯಲ್ಲಿ ಸ್ವಾಮೀಜಿ ವಿಷಯವನ್ನು ಬರೆಯುತ್ತ ಹೀಗೆ ಹೇಳುತ್ತಾಳೆ: “ನಿಜವಾಗಿಯೂ ಭಗವಂತನೊಂದಿಗೆ ನಡೆಯುತ್ತಿದ್ದ ವ್ಯಕ್ತಿಯ ಪರಿಚಯವನ್ನು ಪಡೆದುದು ನನ್ನ ಜೀವನದ ಒಂದು ಮಹಾ ಭಾಗ್ಯ, ನನ್ನ ಪಾಲಿಗೆ ಬಂದ ಒಂದು ಅಪೂರ್ವ ಸಂತೋಷ. ಅವರೊಬ್ಬ ಮಹಾ ಗೌರವಸ್ಥರು, ಸಾಧುಗಳು, ತತ್ತ್ವಜ್ಞಾನಿಗಳು ಮತ್ತು ನಿಜವಾದ ಸ್ನೇಹಿತರು. ಅವರು ನನಗೆ ಜೀವನದಲ್ಲಿ ಹೊಸ ಅನುಭವವನ್ನು ಕೊಟ್ಟರು. ನನ್ನ ಆಧ್ಯಾತ್ಮಿಕ ಭಾವನೆಗಳನ್ನು ವಿಸ್ತಾರಗೊಳಿಸಿದರು, ಅವನ್ನು ಏಕೀಕರಿಸಿದರು. ಸತ್ಯವನ್ನು ಉದಾರವಾಗಿ ವಿಶಾಲವಾಗಿ ತಿಳಿಯುವಂತೆ ಮಾಡಿದರು. ನನ್ನ ಆತ್ಮ ಎಂದೆಂದಿಗೂ ಅವರಿಗೆ ಋಣಿಯಾಗಿರುವುದು.” 

 ಈಗ ಭರತಖಂಡದಲ್ಲಿರುವ ಹಲವರಿಗೆ ರಾಕ್‍ಫೆಲ್ಲರ್ ಸ್ಮಾರಕ ನಿಧಿಯ ಪರಿಚಯವಿದೆ. ಪ್ರತಿ ವರ್ಷವೂ ವಿಶ್ವದಲ್ಲಿರುವ ಸಾರ್ವಜನಿಕ ಸಂಸ್ಥೆಗಳಿಗೆಲ್ಲ, ಅವು ಯಾವ ದೇಶದಲ್ಲಿರಲಿ, ಯಾವ ಒಳ್ಳೆಯ ಕೆಲಸಗಳನ್ನು ಮಾಡುತ್ತಿರಲಿ, ಜಾತಿಮತಗಳ ಭೇದಭಾವವಿಲ್ಲದೆ ಆ ನಿಧಿಯಿಂದ ಹಣ ಹಂಚುವುದನ್ನು ನೋಡುವೆವು. ಆತ ಆ ನಿಧಿಯನ್ನು ಮಾಡಿದ್ದು ಸ್ವಾಮೀಜಿಯವರನ್ನು ಕಂಡಮೇಲೆ, ಅವರ ಬೋಧನೆಯನ್ನು ಕೆಳಿದ ಮೇಲೆ ಎಂದರೆ ನಿಮಗೆ ಆಶ್ಚರ್ಯವಾಗಬಹುದು. ಆತನ ಹತ್ತಿರಕ್ಕೆ ಬರುತ್ತಿದ್ದ ದ್ರವ್ಯದ ಪ್ರವಾಹವನ್ನು ಸಾರ್ವಜನಿಕ ಉಪಕಾರ ಕ್ಷೇತ್ರಕ್ಕೆ ಹರಿದುಹೋಗುವಂತೆ ಮಾಡಿದವರು ಸ್ವಾಮೀಜಿ. 

 ಚಿಕಾಗೊ ನಗರದಲ್ಲಿ ಸ್ವಾಮೀಜಿ ಯಾರ ಮನೆಯಲ್ಲಿದ್ದರೋ ಆ ಮನೆಯ ಯಜಮಾನ ಮತ್ತು ರಾಕ್‍ಫೆಲ್ಲರ್ ಸ್ನೇಹಿತರಾಗಿದ್ದರು. ಆ ಸ್ನೇಹಿತ ತನ್ನ ಮನೆಯಲ್ಲಿರುವ ಸ್ವಾಮೀಜಿ ಗುಣಕಥನವನ್ನು ರಾಕ್‍ಫೆಲ್ಲರ್ ಮುಂದೆ ಮಾಡುತ್ತಿದ್ದ. ಅವರನ್ನು ಬಂದು ನೋಡುವಂತೆ ಹೇಳುತ್ತಿದ್ದ. ಆದರೆ ಆತ ಯಾವುದೋ ಕಾರಣಗಳಿಂದ ಬರಲು ಆಗಲಿಲ್ಲ. ರಾಕ್‍ಫೆಲ್ಲರ್ ಆಗ ಶ‍್ರೀಮಂತಿಕೆಯ ಶಿಖರವನ್ನು ಏರುತ್ತಿದ್ದ. ಅವನು ಇನ್ನೊಬ್ಬರ ಮಾತನ್ನು ಕೇಳುವವನಲ್ಲ. ತನ್ನದೇ ದಾರಿಯಲ್ಲಿ ಹೋಗುವವನು. ಪ್ರಚಂಡ ಇಚ್ಛಾಶಕ್ತಿಯ ಮನುಷ್ಯ. 

 ಒಂದು ದಿನ ಸ್ವಾಮೀಜಿಯವರನ್ನು ನೋಡುವುದಕ್ಕೆ ಇಚ್ಛೆ ಇಲ್ಲದೇ ಇದ್ದರೂ, ಯಾವುದೋ ಭಾವನೆಯಿಂದ ಪ್ರೇರಿತನಾಗಿ ಅವರಿದ್ದ ಮನೆಗೆ ಬಂದ. ಬಾಗಿಲನ್ನು ತೆರೆದು ಬಟ್ಲರ್‍ಗೆ ತಾನು ಸ್ವಾಮಿಗಳನ್ನು ನೋಡಬೇಕು ಎಂದ. ಆತ ಸ್ವಾಮೀಜಿಯವರ ಕೋಣೆಯನ್ನು ತೆಗೆದು ಹಿಂದೂ ಸಂನ್ಯಾಸಿಗಳನ್ನು ರಾಕ್‍ಫೆಲ್ಲರ್ ನೋಡಬೇಕೆಂದು ಬಂದಿರುವರು ಎಂದು ಹೇಳಿದನು. ಸ್ವಾಮೀಜಿ ಒಳಗೆ ಬಾ ಎಂದು ಕೂಡ ಹೇಳಲಿಲ್ಲ. ರಾಕ್‍ಫೆಲ್ಲರ್ ತಾನೇ ಮುಂದೆ ಬಂದು ಸ್ವಾಮೀಜಿ ಹತ್ತಿರ ನಿಂತುಕೊಂಡ. ಸ್ವಾಮೀಜಿಯವರು ಏನನ್ನೋ ಓದುತ್ತಿದ್ದವರು ಇವನ ಕಡೆ ಕತ್ತೆತ್ತಿ ಕೂಡ ನೋಡಲಿಲ್ಲ. 

 ಅನಂತರ ಸ್ವಾಮೀಜಿ ರಾಕ್‍ಫೆಲ್ಲರ್ ಜೀವನದ ಕೆಲವು ವಿಷಯಗಳನ್ನು ಹೇಳಿದರು. ರಾಕ್‍ಫೆಲ್ಲರ್ ಆಪ್ತ ಸ್ನೇಹಿತರಿಗೂ ಅದು ಗೊತ್ತಿರಲಿಲ್ಲ. ಅನಂತರ “ನೀನು ಸಂಪಾದನೆ ಮಾಡುತ್ತಿರುವ ಹಣ ನಿಜವಾಗಿ ನಿನಗೆ ಸೇರಿದ್ದಲ್ಲ. ನೀನು ಕೇವಲ ಒಂದು ಮಧ್ಯವರ್ತಿ. ಅದರಿಂದ ಪ್ರಪಂಚಕ್ಕೆ ಒಳ್ಳೆಯದನ್ನು ಮಾಡಬೇಕು. ಜನರಿಗೆ ಒಳ್ಳೆಯದನ್ನು ಮಾಡುವುದಕ್ಕೆ ನಿನಗೆ ಒಂದು ಅವಕಾಶ ಸಿಕ್ಕಲಿ ಎಂದು ದೇವರು ಐಶ್ವರ‍್ಯವನ್ನು ಕೊಟ್ಟಿರುವುದು” ಎಂದು ನುಡಿದರು ಸ್ವಾಮೀಜಿ. 

 ರಾಕ್ ಫೆಲ್ಲರಿಗೆ ಈ ಮಾತನ್ನು ಕೇಳಿ ಸ್ವಲ್ಪ ಕೋಪವೇ ಬಂತು. ಜೀವನದಲ್ಲಿ ಅವನು ಯಾರಿಂದಲೂ ಹೀಗೆ ಮಾಡು ಎಂದು ಹೇಳಿಸಿಕೊಂಡವನಲ್ಲ. ಸ್ವಾಮೀಜಿಗೆ ಹೋಗಿ ಬರುತ್ತೇನೆ ಎಂದು ಕೂಡ ಹೇಳದೆ ರೂಮಿನಿಂದ ಹೊರಗೆ ಬಂದ. ಆದರೆ ಒಂದು ವಾರವಾದಮೇಲೆ ಸ್ವಾಮೀಜಿಗೆ ತಿಳಿಸದೆ ಅವರ ರೂಮಿನ ಒಳಗೆ ಹೋದ. ಸ್ವಾಮೀಜಿ ಟೇಬಲ್ ಮುಂದುಗಡೆ ದೊಡ್ಡ ಸಾರ್ವಜನಿಕ ಸಂಸ್ಥೆಗೆ ಮೊಟ್ಟಮೊದಲು ಮಾಡಿದ ದಾನಪತ್ರವನ್ನು ಇಟ್ಟನು. ಆತ ಸ್ವಾಮೀಜಿಗೆ “ಈಗ ನೋಡಿ ನಿಮಗೆ ತೃಪ್ತಿ ಯಾಗಿರಬಹುದು. ಈಗ ನೀವು ನನಗೆ ಅದಕ್ಕೆ ಧನ್ಯವಾದವನ್ನು ಅರ್ಪಿಸಬಹುದು” ಎಂದ. 

 ಸ್ವಾಮೀಜಿ ಕಣ್ಣನ್ನೂ ಎತ್ತಲಿಲ್ಲ. ಚಲಿಸಲೂ ಇಲ್ಲ. ಆಗ ಇಟ್ಟಪತ್ರವನ್ನು ಓದಿದ ಮೇಲೆ, “ನೀನು ನನಗೆ ಧನ್ಯವಾದವನ್ನು ಅರ್ಪಿಸಬೇಕಾಗಿದೆ” ಎಂದರು. ಕೊಡುವುದಕ್ಕೆ ಒಂದು ಅವಕಾಶ ಮಾಡಿದವನು ದೇವರು, ಅದನ್ನು ಪ್ರಚೋದಿಸಿದವನು ದೇವರ ಭಕ್ತ. ಇಂದು ಜಗತ್ತಿಗೆಲ್ಲ ಪರಿಚಯವಾಗಿರುವ ರಾಕ್‍ಫೆಲ್ಲರ್ ನಿಧಿಯ ಹಿಂದೆ, ಆತನನ್ನು ಹಾಗೆ ಮಾಡುವಂತೆ ಪ್ರೇರೇಪಿಸಿದವರು ಸ್ವಾಮೀಜಿ. 

 ಸ್ವಾಮೀಜಿಯವರು ಅಮೇರಿಕಾ ದೇಶದಲ್ಲಿ ದೊಡ್ಡ ವಾಗ್ಮಿ ಮತ್ತು ವಕೀಲನಾದ ರಾಬರ್ಟ್ ಇಂಗರ್‍ಸಾಲ್ ಎಂಬುವನ ಪರಿಚಯವನ್ನು ಮಾಡಿಕೊಂಡರು. ಆತ ನಿರೀಶ್ವರವಾದಿ. ಆದರೆ ಈ ಜೀವನದಲ್ಲಿ ಚೆನ್ನಾಗಿ ಯೋಗ್ಯವಾಗಿ ಬಾಳಬೇಕೆಂಬ ಗುಂಪಿಗೆ ಸೇರಿದವನು. ಆತನು ಷೇಕ್ಸ್ ಪಿಯರ್ ಮುಂತಾದ ಕವಿಗಳ ಮೇಲೆ ಅತಿ ಸುಂದರವಾಗಿ ಮಾತನಾಡುತ್ತಿದ್ದ. ಉಪನ್ಯಾಸದ ಮೂಲಕ ಸಂಪಾದನೆ ಮಾಡುವುದು ಸಾಧ್ಯ ಎಂಬುದನ್ನು ಆತ ತೋರಿದನು. ಆತ ಕೆಲವು ಉಪನ್ಯಾಸಗಳಿಗೆ ಒಂದೊಂದಕ್ಕೆ ಸುಮಾರು ಎರಡು ಸಾವಿರ ಡಾಲರುಗಳವರೆಗೆ ಸಂಪಾದನೆ ಮಾಡುತ್ತಿದ್ದನು. ವಕೀಲ ವೃತ್ತಿಗಿಂತ ಹೆಚ್ಚು ಲಾಭದಾಯಕವಾಗಿತ್ತು ಉಪನ್ಯಾಸಕ ವೃತ್ತಿ. ಒಂದು ಸಲ ಸ್ವಾಮೀಜಿಯವರೊಂದಿಗೆ ಮಾತನಾಡುತ್ತಿದ್ದಾಗ “ನೀವೇನಾದರೂ ಐವತ್ತು ವರ್ಷಗಳ ಹಿಂದೆ ಇಲ್ಲಿಗೆ ಬಂದು ಈಗ ಮಾತನಾಡುತ್ತಿರುವಂತೆ ಮಾತನಾಡಿದ್ದರೆ ಜನ ನಿಮ್ಮನ್ನು ಜೀವಸಹಿತ ನೇಣುಹಾಕಿಬಿಡುತ್ತಿದ್ದರು, ಇಲ್ಲವೆ ಸುಡುತ್ತಿದ್ದರು. ನಿಮ್ಮ ಅಭಿಪ್ರಾಯಗಳನ್ನು ವ್ಯಕ್ತಪಡಿಸುವುದರಲ್ಲಿ ಜೋಪಾನವಾಗಿರಿ?” ಎಂದನು. ಸ್ವಾಮೀಜಿ ಅಮೇರಿಕಾ ದೇಶದಲ್ಲಿರುವ ಜನ ಇಷ್ಟು ಮತಭ್ರಾಂತರು ಎಂದು ಭಾವಿಸಿರಲಿಲ್ಲ. ಸ್ವಾಮೀಜಿ ಅಮೇರಿಕಾ ದೇಶಕ್ಕೆ ಹೋದ ವರುಷದಲ್ಲಿಯೇ ಇಲ್ಲಿಯ ಕೆಲವು ವಿಶ್ವವಿದ್ಯಾನಿಲಯಗಳಲ್ಲಿ ಡಾರ್ವಿನ್ನನ ವಿಕಾಸವಾದವನ್ನು ಬೋಧಿಸಕೂಡದು ಎಂದು ಕಾನೂನು ಇತ್ತು. ಏಕೆಂದರೆ ಡಾರ್ವಿನ್ನನ ಸಿದ್ಧಾಂತ ಮಂಗನಿಂದ ಮಾನವ ಬಂದನೆಂದು ಸಾರುವುದು ಮತ್ತು ಅದರಲ್ಲಿ ಜೀವ ಕ್ರಮೇಣ ಕೆಳಗಿನ ಮೆಟ್ಟಲನ್ನು ಹತ್ತಿ ಕೊನೆಗೆ ಮಾನವ ಜನ್ಮಕ್ಕೆ ಬಂದಿದೆ ಎಂದು ಒಪ್ಪಿಕೊಳ್ಳುವುದು. ಅದು ಬೈಬಲ್ಲಿಗೆ ವಿರೋಧವಾಗಿ ಹೋಗುತ್ತದೆ. ಬೈಬಲ್ಲಿನಲ್ಲಿ ದೇವರು ಸೃಷ್ಟಿಯನ್ನೆಲ್ಲ ಒಂದೇ ಸಲ ಮಾಡುತ್ತಾನೆ. ಒಂದಾದಮೇಲೆ ಮತ್ತೊಂದನ್ನು ಮಾಡುವುದಿಲ್ಲ. ಮನುಷ್ಯನನ್ನು ದೇವರು ತನ್ನ ಸ್ವಂತ ಕೈಗಳಿಂದಲೆ ಸೃಷ್ಟಿಮಾಡುತ್ತಾನೆ ಎಂದಿದೆ. ಆದಕಾರಣ ಪೂರ್ವಾಚಾರ ಪರಾಯಣರ ದೃಷ್ಟಿಯಲ್ಲಿ ಡಾರ್ವಿನ್ನನ ಸಿದ್ಧಾಂತವನ್ನು ಬೋಧಿಸುವುದು ನಿರೀಶ್ವರ ಸಿದ್ಧಾಂತದಂತೆ ಎಂಬ ನಂಬಿಕೆ ಬಂದು ಹೋಗಿತ್ತು. ಅಂತಹ ಜನರಲ್ಲಿ ಬೈಬಲ್ಲು, ಆಧ್ಯಾ ತ್ಮಿಕ ವಿಷಯಗಳಿಗೆ ಮಾತ್ರವಲ್ಲ ಪ್ರಮಾಣ, ಸಕಲ ಲೌಕಿಕ ಶಾಸ್ತ್ರಗಳಿಗೂ ಪ್ರಮಾಣ ಎಂಬ ನಂಬಿಕೆ ಬಂದುಹೋಗಿದೆ. ರಸಾಯನಶಾಸ್ತ್ರ, ಭೌತಶಾಸ್ತ್ರ, ಖಗೋಳಶಾಸ್ತ್ರ ಯಾವ ಶಾಸ್ತ್ರವಾದರೂ ಆಗಲಿ ಅದೆಲ್ಲ ಬೈಬಲ್ಲಿಗೆ ಬಾಗಿ ಅದರ ಅಭಿಪ್ರಾಯವೇ ಕಟ್ಟ ಕೊನೆಯದು ಎಂದು ಒಪ್ಪಿಕೊಳ್ಳಬೇಕು, ಅದಕ್ಕೆ ಎಳ್ಳಷ್ಟೂ ಅವಿಧೇಯವಾಗಿರಕೂಡದು. ಗೆಲಿಲಿಯೇ ತನ್ನ ದೂರದರ್ಶಕ ಯಂತ್ರದ ಮೂಲಕ ಭೂಮಿ ಸೂರ‍್ಯನ ಸುತ್ತಲೂ ಸುತ್ತುತ್ತಿದೆ, ಸೂರ‍್ಯ ಭೂಮಿಯ ಸುತ್ತಲೂ ಸುತ್ತುತ್ತಿಲ್ಲ ಎಂದು ಸಾರಿದಾಗ ಅವನನ್ನು ಜೈಲಿನಲ್ಲಿಟ್ಟರು. ಏಕೆಂದರೆ ಅದು ಬೈಬಲ್ಲಿಗೆ ವಿರೋಧವಾಗಿ ಹೋಗುತ್ತದೆ. ಅಂತೂ ಅವನು ತಾನು ಹೇಳಿದ್ದು ತಪ್ಪು ಎಂದು ಒಪ್ಪಿಕೊಂಡಾಗಲೇ ಅವನನ್ನು ಜೈಲಿನಿಂದ ಬಿಟ್ಟರು. ಅವನು ಹೊರಗೆ ಬಂದ ಮೇಲೆ “ನಾನು ಹೇಳಿದರೇನು, ಭೂಮಿ ಸೂರ‍್ಯನ ಸುತ್ತಲೂ ಸುತ್ತುವುದನ್ನು ಬಿಡುತ್ತದೆಯೆ?” ಎಂದು ಹಾಸ್ಯವಾಗಿ ಹೇಳಿದ. 

 ಆದರೆ ಭರತಖಂಡದಲ್ಲಿ ಅದಕ್ಕೆ ವಿರೋಧ. ಎಂದಿಗೂ ವೇದವನ್ನು ರಸಾಯನ ಶಾಸ್ತ್ರ ಬೌತಶಾಸ್ತ್ರ ಮತ್ತು ಖಗೋಳಾದಿ ಶಾಸ್ತ್ರಗಳಿಗೆ ಪ್ರಮಾಣ ಎಂದು ಅನುಸರಿಸುವುದಿಲ್ಲ. ಆಧ್ಯಾತ್ಮ ವಿದ್ಯೆಗೆ ಮಾತ್ರ ಧರ್ಮಗ್ರಂಥ ಪ್ರಮಾಣವೇ ಹೊರತು ಲೌಕಿಕ ವಿದ್ಯೆಗಳಿಗಲ್ಲ. ಪಾಶ್ಚಾತ್ಯ ದೇಶದಲ್ಲಾದರೋ ವಿಜ್ಞಾನಿಗಳು ಚರ್ಚಿನವರೆದುರಿಗೆ ಹೆಜ್ಜೆ ಹೆಜ್ಜೆಗೆ ಮುಂದುವರಿಯಬೇಕಾದರೆ ಬೇಕಾದಷ್ಟು ಹೋರಾಡಬೇಕಾಗಿ ಬಂತು. ಸ್ವತಂತ್ರ ವಿಚಾರಪರರು ಮತ್ತು ತತ್ತ್ವಜ್ಞಾನಿಗಳೆಲ್ಲ ಚರ್ಚಿನ ಬೇಲಿಯಿಂದ ಹೊರಗೆ ಹೋದವರು. 

 ಇಂಗರ್‍ಸಾಲ್ ಸ್ವಾಮೀಜಿಯವರೊಂದಿಗೆ ಮಾತನಾಡುತ್ತಿದ್ದಾಗ “ಪ್ರಪಂಚದಲ್ಲಿ ನಾನು ಸಾಕಷ್ಟು ಸುಖವಾಗಿರಬೇಕೆಂದು ಬಯಸುವೆನು. ಕಿತ್ತಲೆಹಣ್ಣಿನಲ್ಲಿರುವ ರಸವನ್ನೆಲ್ಲಾ ಹೀರಬೇಕೆಂದಿರುವೆನು. ಏಕೆಂದರೆ ಅನಂತರ ಏನಾಗುವುದೋ ನನಗೆ ಗೊತ್ತಿಲ್ಲ” ಎಂದ. ಅದಕ್ಕೆ ಸ್ವಾಮೀಜಿ ಹೀಗೆಂದರು: “ಈ ಪ್ರಪಂಚವೆಂಬ ಕಿತ್ತಲೆಯ ಹಣ್ಣಿನ ರಸವನ್ನು ತೆಗೆಯಲು ನಿನಗಿಂತ ಉತ್ತಮವಾದ ಮಾರ್ಗ ನನಗೆ ಗೊತ್ತಿದೆ. ಅದರಿಂದ ನನಗೆ ಹೆಚ್ಚು ರಸ ಸಿಕ್ಕುವುದು. ನನಗೆ ಸಾವೇ ಅಂತ್ಯವಲ್ಲ ಎಂದು ಗೊತ್ತಿದೆ. ಆದಕಾರಣವೇ ನಾನು ಅವಸರಪಡುವುದಿಲ್ಲ. ನನಗೆ ಅಂಜಿಕೆಯಿಲ್ಲ. ಆದಕಾರಣ ನಾನು ನಿಧಾನವಾಗಿ ಕುಳಿತುಕೊಂಡು ಹಿಂಡುವೆನು. ನನಗೆ ಯಾವ ಕರ್ತವ್ಯದ ಬಂಧನವಾಗಲೀ ಹೆಂಡತಿ ಮಕ್ಕಳು ಆಸ್ತಿ ಬಂಧನವಾಗಲೀ ಇಲ್ಲ. ಆದ ಕಾರಣ ನಾನು ಎಲ್ಲಾ ಸ್ತ್ರೀ ಪುರುಷರನ್ನೂ ಪ್ರೀತಿಸುತ್ತೇನೆ. ನನಗೆ ಪ್ರತಿಯೊಬ್ಬರೂ ದೇವರೇ. ಮನುಷ್ಯನನ್ನು ದೇವರೆಂದು ಪ್ರೀತಿಸುವುದರಿಂದ ಬರುವ ಆನಂದವನ್ನು ಆಲೋಚಿಸಿ ನೋಡು. ನೀನು ಕಿತ್ತಲೆಯ ಹಣ್ಣನ್ನು ಹೀಗೆ ಹಿಂಡು. ಈಗ ಬರುವುದಕ್ಕಿಂತ ಹತ್ತು ಸಾವಿರ ಪಾಲು ಹೆಚ್ಚು ರಸ ಬರುವುದು. ಅದರಲ್ಲಿರುವ ಪ್ರತಿಯೊಂದು ಬಿಂದುವೂ ಬರುವುದು.” 

 ಒಂದು ಸಲ ಸ್ವಾಮೀಜಿ ಚಿಕಾಗೊ ನಗರಕ್ಕೆ ಪಶ್ಚಿಮದಲ್ಲಿರುವ ಒಂದು ಊರಿನಲ್ಲಿ ಉಪನ್ಯಾಸ ಮಾಡುವುದಕ್ಕೆ ಹೋದರು. ಅಲ್ಲಿದ್ದವರೆಲ್ಲ ದನಕಾಯುವವರು. ಸ್ವಾಮೀಜಿ ನಿಂತುಕೊಂಡು ಮಾತನಾಡುವುದಕ್ಕೆ ಒಂದು ಪಿಪಾಯಿಯನ್ನೇ ತಂದು ಇಟ್ಟರು. ಅವರು ಅದರ ಮೇಲೆ ನಿಂತುಕೊಂಡು ಮಾತನಾಡಲುಪಕ್ರಮಿಸಿದರು. ಅಲ್ಲಿದ್ದ ಕೆಲವರು ಸ್ವಾಮೀಜಿಯವರನ್ನು ಅಂಜಿಸಬೇಕೆಂದು ದೂರದಿಂದ ಬಂದೂಕಿನಿಂದ ಹಲವು ಬುಲೆಟ್‍ಗಳನ್ನು ಅವರ ಪಕ್ಕದಲ್ಲಿ ಹಾರಿಹೋಗುವಂತೆ ಬಿಟ್ಟರು. ಆದರೆ ಸ್ವಾಮೀಜಿ ಅದನ್ನು ಗಮನಿಸಲೇ ಇಲ್ಲ. ಅವರು ತಾವು ಮಾತನಾಡುವುದರಲ್ಲಿ ಮೈಮರೆತುಹೋಗಿದ್ದರು. ಅನಂತರ ಸ್ವಾಮೀಜಿಯವರ ಏಕಾಗ್ರತೆಯನ್ನು ಅವರು ಕೊಂಡಾಡಿದರು. 

 ಸ್ವಾಮೀಜಿಯವರ ಕಂದುಬಣ್ಣ ಕೆಲವುವೇಳೆ ತಮಾಷೆಯಾದ ಮತ್ತೆ ಕೆಲವು ವೇಳೆ ದುಃಖಕರವಾದ ಪ್ರಸಂಗಗಳಿಗೆ ಈಡುಮಾಡುತ್ತಿತ್ತು. ಅವರ ಬಣ್ಣದಿಂದ ಅನೇಕರು ನೀಗ್ರೋ ಕುಲಕ್ಕೆ ಸೇರಿದವರೆಂದು ತಿಳಿದುಕೊಂಡು ನೀಗ್ರೋಗಳನ್ನು ನೋಡುವಂತೆ ನಿರ್ಲಕ್ಷ್ಯದಿಂದ ನೋಡುತ್ತಿದ್ದರು. ಒಂದು ಸಲ ಅಮೇರಿಕಾದ ಒಂದು ಊರಿನಲ್ಲಿ ಸ್ವಾಮೀಜಿಯವರನ್ನು ಬೀಳ್ಕೊಡುವುದಕ್ಕೆ ಅನೇಕ ಜನ ಬಿಳಿಯ ಜನರು ಬಂದಿದ್ದರು. ಅವರ ಸಾಮಾನುಗಳನ್ನು ನೀಗ್ರೋ ಕೂಲಿಯವನು ಎತ್ತಿಟ್ಟನು. ಸ್ವಾಮಿಗಳನ್ನು ಗೌರವದಿಂದ ಅಲ್ಲಿಯ ಜನ ನೋಡುತ್ತ ಇದ್ದುದರಿಂದ, ತಮ್ಮ ನೀಗ್ರೋ ಕುಲಕ್ಕೆ ಅದು ಹೆಮ್ಮೆಯನ್ನು ತರುವ ವಿಷಯವೆಂದು, ಆತ ಸ್ವಾಮೀಜಿಗೆ ಧನ್ಯವಾದವನ್ನು ಅರ್ಪಿಸಿ ಅವರಿಗೆ ಹಸ್ತಲಾಘವವನ್ನು ಕೊಟ್ಟನು. “ಸಹೋದರನೆ, ಧನ್ಯವಾದ” ಎಂದು ಅವನನ್ನು ವಂದಿಸಿದರು. ತಾವು ನೀಗ್ರೋ ಅಲ್ಲವೆಂದು ಹೇಳಿ ಅವನಿಗೆ ವ್ಯಥೆಯನ್ನುಂಟುಮಾಡಲು ಇಚ್ಛಿಸಲಿಲ್ಲ. ಕೆಲವು ಕಡೆ ಅವರನ್ನು ನೀಗ್ರೋ ಎಂದು ಭಾವಿಸಿ ಹೋಟೆಲಿನಲ್ಲಿ ಇರಲು ಅವರಿಗೆ ಅವಕಾಶವನ್ನು ಕೊಡುತ್ತಿರಲಿಲ್ಲ. ಕ್ಷೌರದ ಸಲೂನಿಗೆ ಹೋದರೆ ಅಲ್ಲಿ ನೀಗ್ರೋ‌ ಜನರಿಗೆ ಕ್ಷೌರ ಮಾಡುವುದಿಲ್ಲ ಎಂದು ಸ್ವಾಮೀಜಿಗಳನ್ನು ಆಚೆಗೆ ಕಳಿಸಿದರು. ತಾವು ನೀಗ್ರೋಗಳಲ್ಲ ಎಂದು ಸ್ವಾಮೀಜಿ ಹೇಳಲಿಲ್ಲ. ಆಗ ಮತ್ತಾರೋ ಭಕ್ತರು “ನೀವು ಇಂಡಿಯಾ ದೇಶದಿಂದ ಬಂದ ಆರ್ಯರು ಎಂದು ಏತಕ್ಕೆ ಹೇಳಬಾರದಿತ್ತು, ಸ್ವಾಮೀಜಿ” ಎಂದು ಕೇಳಿದರು. ಅದಕ್ಕೆ ಸ್ವಾಮೀಜಿ,‌‌ “ಏನು ಇನ್ನೊಬ್ಬರನ್ನು ತುಳಿದು ಮೇಲಕ್ಕೆ ಏಳಬೇಕೆ? ಅದಕ್ಕಾಗಿ ನಾನು ಪ್ರಪಂಚಕ್ಕೆ ಬರಲಿಲ್ಲ‌” ಎಂದರು. ತಾವು ಆರ್ಯರಿಗೆ ಎಷ್ಟು ಋಣಿಗಳೋ, ಅಷ್ಟೇ ನೀಗ್ರೋಕುಲದವರಿಗೆ ಎಂದರು. ಬರೀ ದೇಹದ ಬಣ್ಣದಿಂದಲೇ ಒಬ್ಬ ಮೇಲೇರಲಾರನು. 

 ಸ್ವಾಮೀಜಿಯವರು ಉಪನ್ಯಾಸಕ್ಕೆ ಒಪ್ಪಿಕೊಂಡಾಗ ಕೆಲವು ವೇಳೆ ವಾರಕ್ಕೆ ಹನ್ನೆರಡು ಹದಿನಾಲ್ಕು ಉಪನ್ಯಾಸಗಳನ್ನು ಮಾಡಬೇಕಾಗಿತ್ತು. ಗೊತ್ತಿರುವ ವಿಷಯಗಳನ್ನೆಲ್ಲ ಹೇಳಿ ಮುಗಿಸಿಬಿಟ್ಟಿದ್ದರು. ನಾಳೆ ಏನು ಮಾಡುವುದು ಎಂದು ಯೋಚಿಸುತ್ತಿದ್ದಾಗ ನಿದ್ರೆ ಬರುವುದಕ್ಕೆ ಪ್ರಾರಂಭವಾದ ಕೂಡಲೆ, ಯಾರೋ‌ ದೂರದಿಂದ ಇವರಿಗೆ ಸ್ಪಷ್ಟವಾಗಿ ನಾಳೆ ಏನನ್ನು ಮಾತನಾಡಬೇಕಾಗುವುದೋ ಅದನ್ನೆಲ್ಲ ಹೇಳಿದಂತೆ ಆಗುತ್ತಿತ್ತು. ಕೆಲವು ವೇಳೆ ಇಬ್ಬರು ಬಂದು ತಮ್ಮಲ್ಲಿಯೇ ವಾದ ವಿವಾದಗಳಲ್ಲಿ ನಿರತರಾಗುತ್ತಿದ್ದುದನ್ನು ನೋಡುತ್ತಿದ್ದರು. ಅವರು ಮಾರನೆ ದಿನ ಉತ್ತರಕೊಡಬೇಕಾದ ಪ್ರಶ್ನೆ ಜೊತೆಗೆ ಉತ್ತರಗಳೂ ಎಲ್ಲವೂ ಇರುತ್ತಿತ್ತು. ಇದನ್ನು ಅವರು ಕೇವಲ ಕಲ್ಪನೆ ಎಂದು ಬೇಕಾದರೆ ಹೇಳಿಬಿಡಬಹುದಿತ್ತು. ಆದರೆ ಅವರ ಕೋಣೆಯ ಹೊರಗೆ ಇದ್ದವರು ಸ್ವಾಮೀಜಿ ಕೋಣೆಯಲ್ಲಿ ರಾತ್ರಿ ಯಾರೋ ಮಾತನಾಡುತ್ತಿದ್ದುದನ್ನು ಕೇಳುತ್ತಿದ್ದರು. ಮಾರನೆ ದಿನ ಬೆಳಿಗ್ಗೆ ಎದ್ದು ಬಂದು ಸ್ವಾಮೀಜಿಯವರನ್ನು “ನೀವು ಯಾರೊಡನೆ ನಿನ್ನೆ ರಾತ್ರಿ ಮಾತನಾಡುತ್ತಿದ್ದಿರಿ?” ಎಂದು ಕೇಳುತ್ತಿದ್ದರು. ಸ್ವಾಮೀಜಿಯವರು ಅದನ್ನು ಸುಮ್ಮನೆ ತೇಲಿಬಿಡುತ್ತಿದ್ದರು. ಇದೆಲ್ಲ ಆಧ್ಯಾತ್ಮಿಕ ಜೀವನದಲ್ಲಿ ಆಗುವ ಸಾಧಾರಣ ಅನುಭವಗಳು ಅವರಿಗೆ. ಜನರೆದುರಿಗೆ ಇದನ್ನೆಲ್ಲ ಹೇಳಿಕೊಂಡು ಮೆರೆಯುವುದೇನೂ ಇಲ್ಲ ಇದರಲ್ಲಿ. 

 ಒಂದು ಸಲ ಚಿಕಾಗೊ ನಗರದ ಶ‍್ರೀಮಂತನೊಬ್ಬ ಬಂದು ಸ್ವಾಮೀಜಿ ಹತ್ತಿರ ಮಾತನಾಡುತ್ತಿದ್ದಾಗ ಸ್ವಾಮೀಜಿ ಮನೋನಿಗ್ರಹದಿಂದ ಅದ್ಭುತ ಶಕ್ತಿಗಳು ಬರುವುದು ಎಂಬ ವಿಷಯಗಳನ್ನು ವಿವರಿಸುತ್ತಿದ್ದರು. ಆಗ ಆತ “ಹಾಗಾದರೆ ನನ್ನ ಮನಸ್ಸಿನಲ್ಲಿ ಏನಿರುವುದೋ ಅದನ್ನು ಹೇಳಿ” ಎಂದ. ಸ್ವಾಮೀಜಿ ಸುಮ್ಮನೆ ಅವನ ಕಣುಗಳನ್ನು ದುರುಗುಟ್ಟಿಕೊಂಡು ನೋಡಿದರು. ಆತ ಅಸ್ತವ್ಯಸ್ತನಾಗಿ, “ಅಯ್ಯೋ ನೀವು ನನಗೆ ಏನು ಮಾಡುತ್ತಿರುವಿರಿ? ನನ್ನ ಜೀವನವನ್ನೆಲ್ಲ ಬಗೆದು ರಹಸ್ಯಗಳನ್ನೆಲ್ಲ ಹೊರಗೆ ಎಳೆದು ಹಾಕುತ್ತಿರುವಿರಿ” ಎಂದು ಅರಚಿಕೊಂಡ. ಸ್ವಾಮೀಜಿ ಅಲ್ಲಿಗೇ ನಿಲ್ಲಿಸಿದರು. ಅವರಿಗೆ ಮತ್ತೊಬ್ಬನ ಮನಸ್ಸಿನಲ್ಲಿರುವುದು ಗ್ಲಾಸಿನ ಹಿಂದೆ ಇರುವ ವಸ್ತುಗಳಂತೆ ಕಾಣುತ್ತಿದ್ದವು. ಆದರೆ ನೋಡಲು ಬಯಸುತ್ತಿರಲಿಲ್ಲ. ಹಾಗೆ ನೋಡಿದರೆ ಆ ಮನುಷ್ಯನ ಮೇಲೆ ಇರುವ ಅನುಕಂಪವೆಲ್ಲ ಹೊರಟು ಹೋಗುತ್ತದೆ ಎಂದು ಹೇಳುತ್ತಿದ್ದರು. ಸಾಧಾರಣ ಮನುಷ್ಯನ ಒಳಗೆ ಮುಕ್ಕಾಲುಪಾಲು ಇರುವುದೆಲ್ಲ ಚರಂಡಿಯಂತೆ ರೊಚ್ಚು. ಇದನ್ನು ತಿಳಿದುಕೊಂಡ ಮೇಲೆ ಎಲ್ಲೋ ತೇಲುತ್ತಿರುವ ಸ್ವಲ್ಪ ಒಳ್ಳೆಯದನ್ನು ಮರೆಯುವ ಸ್ವಭಾವವೇ ಹೆಚ್ಚು. ಕೆಲವು ವೇಳೆ ಈ ಶಕ್ತಿ ಸ್ವಾಮೀಜಿ ಜೀವನದಲ್ಲಿ ಹೆಚ್ಚು ವ್ಯಕ್ತವಾದಾಗ ಇದರಿಂದ ಪಾರುಮಾಡೆಂದು ಶ‍್ರೀಗುರುದೇವರನ್ನು ಅವರು ಪ್ರಾರ್ಥಿಸುತ್ತಿದ್ದರು. 

 ಇತ್ತ ಭರತಖಂಡಕ್ಕೆ ಸ್ವಾಮೀಜಿ ಅಮೇರಿಕಾ ದೇಶದಲ್ಲಿ ಗಳಿಸಿದ ಕೀರ್ತಿ ಮುಟ್ಟಿತು. ಇಲ್ಲಿ ಅವರ ಪ್ರಭಾವ ಹತ್ತುಪಾಲು ಜಾಸ್ತಿಯಾಯಿತು. ಒಬ್ಬ ಹಿಂದೂ ಸಂನ್ಯಾಸಿ ಅಮೇರಿಕಾ ದೇಶದಲ್ಲಿ ಹೆಸರನ್ನು ಗಳಿಸಿದನು ಎಂಬುದು ಮನೆಮಾತಾಯಿತು. ಇಲ್ಲಿರುವ ಜನರಿಗೆ ಒಂದು ಆತ್ಮಗೌರವವನ್ನು ತಂದುಕೊಟ್ಟಿತು. ನಾವು ಕಳಪೆ ಜನವಲ್ಲ, ನಾವು ಯಾವಾಗಲೂ ಬೇಡುತ್ತಲೇ ಇರಬೇಕಾಗಿಲ್ಲ, ನಮ್ಮಲ್ಲಿಯೂ ಯೋಗ್ಯವಾಗಿರುವುದನ್ನು ದೇವರು ಇಟ್ಟಿರುವನು ಎಂಬ ಧೈರ್ಯವನ್ನು ಇಲ್ಲಿನ ಜನರ ಹೃದಯದಲ್ಲಿ ಬಿತ್ತಿತು. ತಮ್ಮ ಹಿಂದಿನ ಶ್ರೇಷ್ಠವಾದ ವಸ್ತುವನ್ನು ಗೌರವಿಸಲು ಕಲಿತರು. ಅತ್ತ ಪಾಳುಮನೆಯ ಬಾರಾನಗರ ಮಠದಲ್ಲಿದ್ದ ಶಿಷ್ಯರಿಗೆ, ವಿವೇಕಾನಂದರು ಗಳಿಸಿರುವ ಕೀರ್ತಿಯನ್ನು ಕೇಳಿ, ಶ‍್ರೀರಾಮಕೃಷ್ಣರು ನರೇಂದ್ರನ ವಿಷಯದಲ್ಲಿ ಹೇಳುತ್ತಿದ್ದ ನುಡಿಗಳು ಸಫಲವಾದವು; ಅವನು ಪ್ರಪಂಚವನ್ನೇ ಅಲ್ಲಾಡಿಸುವನು ಎಂಬ ಅವರ ಮಾತು ಸತ್ಯವಾಯಿತು ಎಂದು ಆನಂದಿಸಿದರು. 

