
\chapter{ವೃತ್ತಪತ್ರಿಕೆಯ ಕಿಟಕಿಗಳ ಮೂಲಕ}

ಸ್ವಾಮಿ ವಿವೇಕಾನಂದರು ಅಮೇರಿಕಾ ದೇಶದ ಊರುಊರುಗಳಲ್ಲಿ ಉಪನ್ಯಾಸ\break ಮಾಡುತ್ತಿದ್ದಾಗ ಅಲ್ಲಿಯ ವೃತ್ತಪತ್ರಿಕೆಗಳು ಇವರನ್ನು ಹೇಗೆ ನೋಡುತ್ತಿದ್ದವು ಮತ್ತು ಇವರು ಹೇಳುವ ವಿಷಯಗಳು ಯಾವ ರೀತಿ ಅಲ್ಲಿಯ ಜನರ ಮೇಲೆ ತಮ್ಮ ಪ್ರಭಾವವನ್ನು ಬೀರಿದವು ಎಂಬುದನ್ನು ಪತ್ರಿಕಾ ವರದಿಗಳ ಮೂಲಕವಾಗಿ ನೋಡುವುದೇ ಮೇಲು. ಅವು ವಿವೇಕಾನಂದರನ್ನು ಕೆಲವು ವೇಳೆ ‘ಕಾನಂದ’ ಎಂತಲೂ ‘ಬ್ರಾಹ್ಮಣ ವಿವೇಕಾನಂದ’ ಎಂತಲೂ, ‘ರಾಜ ವಿವೇಕಾನಂದ’ ಎಂತಲೂ, ಕರೆಯುತ್ತಿದ್ದವು. ಕೆಲವು ಪತ್ರಿಕೆಗಳು ಕೊಡುವ ಸ್ವಾಮೀಜಿಯವರ ಉಪನ್ಯಾಸದ ವರದಿಗಳು ಅಷ್ಟು ಸಮರ್ಪಕವಾಗಿ ಇಲ್ಲದೇ ಇದ್ದರೂ ಅವರ ವ್ಯಕ್ತಿತ್ವದ ವಿವರಣೆಯನ್ನು ಬಹಳ ಸುಂದರವಾಗಿ ಕೊಡುತ್ತಿದ್ದವು. ಇವುಗಳನ್ನು ಆ ಪತ್ರಿಕೆಗಳ ಮೂಲಕವಾಗಿಯೇ ವಿವರಿಸುವೆವು.

~\hfill{\fontsize{11pt}{13.75pt}\selectfont ಕ್ರಿಟಿಕ್, ೭ನೇ ಅಕ್ಟೋಬರ್, ೧೮೯೩}

 ವಿಶ್ವಧರ್ಮ ಸಮ್ಮೇಳನದ ಉಪನ್ಯಾಸಗಳು ಆದಮೇಲೆ ನಮ್ಮ ಕಣ್ಣುಗಳು ತೆರೆದವು. ಪುರಾತನ ಧರ್ಮಗಳಲ್ಲಿ ಆಧುನಿಕರಿಗೆ ಅತಿ ಸುಂದರವಾದ ಭಾವನೆಗಳು ಅಡಗಿವೆ ಎಂಬುದು ನಮಗೆ ಗೊತ್ತಾಯಿತು. ನಾವು ಇದನ್ನು ಸ್ಪಷ್ಟವಾಗಿ ಅರಿತಮೇಲೆ ಅವುಗಳನ್ನು ಮೀರಿಸುವವರ ವಿಷಯದಲ್ಲಿ ನಮ್ಮ ಆಸಕ್ತಿ ಕೆರಳಿತು. ಸಮ್ಮೇಳನ ಮುಕ್ತಾಯಗೊಂಡ ಮೇಲೆ ಅದನ್ನು ತಿಳಿದುಕೊಳ್ಳುವುದಕ್ಕೆ ಸದವಕಾಶ ದೊರೆತುದು ಸ್ವಾಮಿ ವಿವೇಕಾನಂದರ ಉಪನ್ಯಾಸಗಳ ಮೂಲಕ. ಚಿಕಾಗೋ ನಗರದಲ್ಲಿ ಇವರು ಇರುವರು. ಇವರು ಅಮೇರಿಕಾ ದೇಶಕ್ಕೆ ಬಂದಾಗ ಭರತಖಂಡದಲ್ಲಿ ಕೈಗಾರಿಕೆಗಳನ್ನು ಸ್ಥಾಪಿಸುವ ಸಲುವಾಗಿ ಅಮೇರಿಕಾ ದೇಶೀಯರ ನೆರವು ತೆಗೆದುಕೊಳ್ಳುವುದಾಗಿತ್ತು. ಆದರೆ ಸದ್ಯಕ್ಕೆ ಅವರು ಅದನ್ನು ಬದಿಗಿರಿಸಿರುವರು. 

 ವಿವೇಕಾನಂದರು ಬ್ರಾಹ್ಮಣರಲ್ಲಿ ಒಬ್ಬರು. ಸಂನ್ಯಾಸಿಗಳ ಸಂಘವನ್ನು ಸೇರುವುದಕ್ಕೆ ಅವರು ತಮ್ಮ ಜಾತಿಯನ್ನು ಬಿಟ್ಟರು. ಸಂನ್ಯಾಸಧರ್ಮದಲ್ಲಿ ಎಲ್ಲಾ ಜಾತಿ ಗೌರವಗಳನ್ನೂ ತ್ಯಜಿಸಬೇಕು. ಆದರೂ ವಿವೇಕಾನಂದರ ವ್ಯಕ್ತಿತ್ವದಲ್ಲಿ ಆ ಜನಾಂಗದ ಚಿಹ್ನೆ ಇದೆ. ಅವರ ಸಂಸ್ಕೃತಿ, ವಾಕ್ಚಾತುರ್ಯ, ಮೋಹಕರವಾದ ವ್ಯಕ್ತಿತ್ವ ಇವು ಹಿಂದೂ ಸಂಸ್ಕೃತಿಯ ವಿಷಯದಲ್ಲಿ ನಮಗೆ ಹೊಸ ಭಾವನೆಯನ್ನು ಕೊಟ್ಟಿವೆ. ಅವರು ಸ್ವಾರಸ್ಯವಾದ ವ್ಯಕ್ತಿಗಳು. ಗೈರಿಕವಸನದ ಹಿನ್ನೆಲೆಯಲ್ಲಿರುವ ಸುಂದರವಾದ ತೇಜೋಮಯವಾದ ಮುಖಮಂಡಲ, ಆಳವಾದ ಸುಶ್ರಾವ್ಯವಾದ ಧ್ವನಿ, ಇವುಗಳೆಲ್ಲ ಜನರನ್ನು ಒಲಿಯುವಂತೆ ಮಾಡುವುವು. ಸಾಹಿತ್ಯಮಂಡಳಿಗಳು ಅವರನ್ನು ಕರೆದುದರಲ್ಲಿ ಆಶ್ಚರ್ಯವೇನೂ ಇಲ್ಲ. ಅವರು ಯಾವ ಟಿಪ್ಪಣಿಯೂ ಇಲ್ಲದೆ ಮಾತನಾಡುವರು. ಅವರು ತಮ್ಮ ವಿಷಯಗಳನ್ನು ಮತ್ತು ನಿರ್ಣಯಗಳನ್ನು ಅತ್ಯಂತ ಕಲಾಮಯವಾಗಿ ನಮಗೆಲ್ಲ ತೃಪ್ತಿದಾಯಕವಾಗುವಂತೆ ವಿವರಿಸಬಲ್ಲರು. ಕೆಲವು ವೇಳೆ ಅತಿ ಸ್ಫೂರ್ತಿಯಿಂದ ಕೂಡಿದ ವಾಗ್ವೈಖರಿಯ ಶಿಖರಕ್ಕೆ ಏರುವುದನ್ನು ನಾವು ನೋಡುತ್ತೇವೆ. ತುಂಬಾ ವಿದ್ಯಾವಂತರಾದ ಜೆಸುಇಸ್ಟ್ ಮಿಶಿನೆರಿಯಷ್ಟೇ ಇವರು ವಿದ್ಯಾವಂತರು ಮತ್ತು ವಿನಯಸಂಪನ್ನರು. ಅವರ ವ್ಯಕ್ತಿತ್ವದಲ್ಲಿ ಕೂಡ ಜೀಸಸ್ಸನ ಜೀವನದ ಗುಣಗಳನ್ನು ಕಾಣುತ್ತೇವೆ. ಅವರು ಮಾತಿನ ಮಧ್ಯದಲ್ಲಿ ಉಪಯೋಗಿಸುವ ವ್ಯಂಗ್ಯೋಕ್ತಿ ಕತ್ತಿಯಂತೆ ಹರಿತವಾಗಿದ್ದರೂ, ಅವು ಅಷ್ಟು ಸೂಕ್ಷ್ಮವಾಗಿರುವುದರಿಂದ ಹಲವು ಸಭಿಕರಿಗೆ ಇದು ಗೋಚರಕ್ಕೆ ಬರುವುದಿಲ್ಲ. ಆದರೆ ಅವರು ಎಂದಿಗೂ ಅವಿನಯವನ್ನು\break ವ್ಯಕ್ತಪಡಿಸುವುದಿಲ್ಲ. ಎಂದಿಗೂ ಅವರು ನಮ್ಮನ್ನು ನೇರವಾಗಿ ಕಟುವಾಗಿ ಟೀಕಿಸುವುದಿಲ್ಲ. ಸದ್ಯಕ್ಕೆ ವಿವೇಕಾನಂದರು ಅವರ ಧರ್ಮ ಮತ್ತು ತತ್ತ್ವಗಳನ್ನು ಮಾತ್ರ ಹೇಳುವುದರಲ್ಲಿ ತೃಪ್ತರಾಗಿರುವರು. ನಾವೆಲ್ಲ ವಿಗ್ರಹವನ್ನು ಮೀರಿಹೋಗುವ ಕಾಲವನ್ನು, ಬಾಹ್ಯಪೂಜೆಗೂ ಅತೀತವಾಗಿ ಹೋಗುವ ಕಾಲವನ್ನು, ಪ್ರಕೃತಿಯಲ್ಲಿ ದೇವರಿರುವನು ಎಂಬ ಭಾವನೆಗೂ ಅತೀತರಾಗಿ ಮಾನವನಲ್ಲಿಯೇ ಪವಿತ್ರತೆಯನ್ನು ನೋಡುವ ಮತ್ತು ಎಲ್ಲಕ್ಕೂ ಅವನನ್ನೇ ಜವಾಬ್ದಾರನನ್ನಾಗಿ ಮಾಡುವ ಒಂದು ಕಾಲ ಬರುವುದನ್ನು ಅವರು ಆಶಿಸುವರು. ಪ್ರಪಂಚವನ್ನು ತ್ಯಜಿಸಿದ ಬುದ್ಧ ಹೇಳಿದಂತೆ, “ನಿಮ್ಮ ಉದ್ಧಾರವನ್ನು ನೀವೇ ಮಾಡಿಕೊಳ್ಳಬೇಕಾಗಿದೆ. ನಾನು ನಿಮಗೆ ಸಹಾಯ ಮಾಡಲಾರೆ ಅಥವಾ ಮತ್ತಾರೂ ನಿಮಗೆ ಸಹಾಯ ಮಾಡಲಾರರು. ನಿಮ್ಮನ್ನು ನೀವೇ ಉದ್ಧಾರ ಮಾಡಿಕೊಳ್ಳಬೇಕಾಗಿದೆ.” ಎಂದು ಬೋಧಿಸುವರು.

\section*{ಪುನರ್ಜನ್ಮ }

~\hfill{\fontsize{11pt}{13.75pt}\selectfont ಇವಾನ್‍ಸ್ಟನ್ ಇನ್‍ಡೆಕ್ಸ್, ಅಕ್ಟೋಬರ್ ೭, ೧೮೯೩}

ಕಳೆದ ವಾರ ಕಾಂಗ್ರಿಗೇಶನಲ್ ಚರ್ಚಿನಲ್ಲಿ ಕೆಲವು ಉಪನ್ಯಾಸಗಳನ್ನು ಕೊಟ್ಟರು. ಅಲ್ಲಿ ಉಪನ್ಯಾಸಗಳು ಈಗ ತಾನೆ ಮುಕ್ತಾಯಗೊಂಡ ಧರ್ಮ ಸಮ್ಮೇಳನದ ಉಪನ್ಯಾಸಗಳಂತೆಯೇ ಇದ್ದುವು. ಉಪನ್ಯಾಸಕರಲ್ಲಿ ಸ್ವಾಮಿ ವಿವೇಕಾನಂದರು ಒಬ್ಬರು. ಅವರು ವಿಶ್ವಧರ್ಮ ಸಮ್ಮೇಳನಕ್ಕೆ ಬಂದ ಭರತಖಂಡದ ಪ್ರತಿನಿಧಿಗಳು. ಅವರು ತಮ್ಮ ವಿಚಿತ್ರವಾದ ಪೋಷಾಕು, ತೇಜಸ್ಸಿನಿಂದ ಕೂಡಿದ ವ್ಯಕ್ತಿತ್ವ, ಅಪೂರ್ವವಾದ ವಾಗ್ಮಿತೆ, ಹಿಂದೂತತ್ತ್ವಗಳನ್ನು ಅತಿ ಸುಂದರವಾಗಿ ವಿವರಿಸುವುದು, ಇವುಗಳ ಮೂಲಕ ಜನರ ಮನಸ್ಸನ್ನು ಸೂರೆಗೊಂಡಿರುವರು. ಚಿಕಾಗೊ ನಗರದಲ್ಲಿ ಜನರು ಅವರನ್ನು ಒಂದೇ ಸಮನಾಗಿ ಸ್ವಾಗತಿಸುತ್ತಿರುವರು. ಮೂರು ದಿನ ಸಂಜೆ ಉಪನ್ಯಾಸ ಕೊಡುವುದಕ್ಕೆ ಏರ್ಪಾಡು ಮಾಡಿದ್ದರು. 

 ಗುರುವಾರ ಸಾಯಂಕಾಲ ಅಕ್ಟೋಬರ್ ೫ನೇ ತಾರೀಖು ಹಿಂದೂ ಸಂನ್ಯಾಸಿಗಳು ಪುನರ್ಜನ್ಮ ಎಂಬ ವಿಷಯದ ಮೇಲೆ ಮಾತನಾಡಿದರು. ಈ ಉಪನ್ಯಾಸ ಬಹಳ ಸ್ವಾರಸ್ಯವಾಗಿತ್ತು. ಇಲ್ಲಿರುವ ಜನರಿಗೆ ಅದು ಗೊತ್ತಿಲ್ಲದ ವಿಷಯವಾಗಿತ್ತು. ಈ ಸಿದ್ಧಾಂತ ಈ ದೇಶಕ್ಕೆ ಇತ್ತೀಚೆಗೆ ಬಂದದು. ಅದನ್ನು ಈ ದೇಶದಲ್ಲಿ ತಿಳಿದುಕೊಂಡಿರುವವರು ಅತಿ ವಿರಳ. ಆದರೆ ಪೌರ್ವಾತ್ಯದೇಶದಲ್ಲಿ ಆದರೋ ಇದು ಎಲ್ಲರಿಗೂ ಗೊತ್ತಾದ ಭಾವನೆಯಾಗಿದೆ. ಅಲ್ಲಿಯ ಧರ್ಮದ ಬಹು ಮುಖ್ಯವಾದ ತಳಹದಿಯೇ ಇದು ಎಂದು ಬೇಕಾದರೂ ಹೇಳಬಹುದು. ಈ ಸಿದ್ಧಾಂತದ ವಿಷಯದಲ್ಲಿ ನಾವು ನಿಶ್ಚಯವಾಗಿ ಎದುರಿಸಬೇಕಾಗಿರುವುದೇ ನಮಗೆ ಏನಾದರೂ ಹಿಂದಿನ ಜನ್ಮವಿತ್ತೆ ಎಂಬುದು. ಈಗ ನಮಗೊಂದು ಜನ್ಮವಿದೆ ಎಂಬುದು ಗೊತ್ತಿದೆ. ಮುಂದಿನ ನಮ್ಮ ಸ್ಥಿತಿಯಲ್ಲಿ ಯಾವ ಸಂದೇಹವೂ ಇಲ್ಲ. ಆದರೂ ಹಿಂದೆ ಇಲ್ಲದೆ ಈಗ ಇರುವುದು ಹೇಗೆ ಸಾಧ್ಯ? ಆಧುನಿಕ ವಿಜ್ಞಾನ ಈಗಿರುವ ವಸ್ತು ಹಿಂದೆ ಇತ್ತು ಮತ್ತು ಮುಂದೆಯೂ ಇರುವುದು ಎಂದು ಹೇಳುವುದು. ಸೃಷ್ಟಿ ಎಂದರೆ ಆಕಾರದಲ್ಲಿ ಒಂದು ಬದಲಾವಣೆ ಅಷ್ಟೇ? ನಾವು ಯಾವುದೋ ಶೂನ್ಯದಿಂದ ಬರಲಿಲ್ಲ. ಕೆಲವರು ದೇವರೇ ಎಲ್ಲಕ್ಕೂ ಕಾರಣ ಎಂದು, ಈಗಿರುವ ವಸ್ತುಗಳಿಗೆಲ್ಲ ಇದೇ ಸಾಕಷ್ಟು ವಿವರಣೆಯನ್ನು ಕೊಡುತ್ತದೆ ಎಂದೂ ನಿಸ್ಸಂದೇಹರಾಗಿರುವರು. ಆದರೆ ನಾವು ಎಲ್ಲಾ ಕಡೆಯಲ್ಲಿಯೂ ಕಾರ‍್ಯಕಾರಣ ಸಂಬಂಧವನ್ನು ಪರ‍್ಯಾಲೋಚಿಸಬೇಕಾಗಿದೆ, ಎಲ್ಲಿಂದ ಯಾವಾಗ ವಸ್ತು ಬಂದಿತು ಎಂಬುದನ್ನು ತಿಳಿಯಬೇಕಾಗಿದೆ. ಜೀವಿಗೆ ಭವಿಷ್ಯವಿದೆ ಎಂಬ ವಾದಗಳೇ, ಅದಕ್ಕೆ ಭೂತಕಾಲ ಒಂದಿತ್ತು ಎಂಬುದನ್ನು ಪ್ರತಿಪಾದಿಸುವುದು. ಭಗವದಿಚ್ಛೆ ಅಲ್ಲದೆ ಬೇರೊಂದು ಕಾರಣ ಇರಬೇಕಾಗುವುದು. ಆನುವಂಶಿಕತೆ ಸಾಕಷ್ಟು ಪ್ರಮಾಣವನ್ನು ಕೊಡಲಾರದು. ಕೆಲವರು ತಮಗೆ ತಮ್ಮ ಹಿಂದಿನ ನೆನಪು ಗೊತ್ತಿಲ್ಲ ಎನ್ನುವರು. ಆದರೆ ಕೆಲವರಿಗೆ ತಮ್ಮ ಹಿಂದಿನ ಜೀವನದ ಸ್ಮೃತಿ ಇರುವ ಪ್ರಸಂಗಗಳಿವೆ. ಪುರಾತನಕಾಲದ ಹಿಂದೂ ಋಷಿಯೊಬ್ಬ ಯಾವುದು ನಮ್ಮನ್ನು ಮೇಲಕ್ಕೆ ಎತ್ತುವುದೋ ಅದೇ ಧರ್ಮ ಎನ್ನುವನು ಮೃಗೀಯತೆಯನ್ನು ಆಚೆಗೆ ದಬ್ಬಬೇಕು, ಅನಂತರ ಬರುವ ಮಾನವತ್ವ ದೇವತ್ವಕ್ಕೆ ಆಸ್ಪದವಾಗುವುದು. ಪುನರ್ಜನ್ಮ ಸಿದ್ಧಾಂತ ಈ ಗ್ರಹದಲ್ಲಿಯೇ ಒಬ್ಬ ಮುಂದೆಯೂ ಜನ್ಮವೆತ್ತಬೇಕು ಎನ್ನುವುದಿಲ್ಲ. ಅವನ ಆತ್ಮ ಇನ್ನೂ ಉತ್ತಮ ಜೀವಿಗಳು ಇರುವ ಲೋಕಕ್ಕೆ ಹೋಗಬಹುದು. ಅಲ್ಲಿ ಪಂಚೇಂದ್ರಿಯಗಳ ಬದಲು ಎರಡು ಇಂದ್ರಿಯಗಳನ್ನು ಪಡೆಯಬಹುದು. ಹೀಗೆ ಮುಂದುವರಿಯುತ್ತ ಹೋಗಿ ಪೂರ್ಣತೆ ಮತ್ತು ಪವಿತ್ರತೆಯನ್ನು ಪಡೆಯುವನು. ಮುಕ್ತಾತ್ಮರುಗಳು ಇರುವ ಲೋಕದ ಆನಂದವನ್ನು ಅನುಭವಿಸಲು ಹಕ್ಕುದಾರನಾಗಿ ಈ ಲೋಕವನ್ನು ಮರೆಯುವನು.


\section*{ಹಿಂದೂಧರ್ಮ}

~\hfill{\fontsize{11pt}{13.75pt}\selectfont ಮಿನಿಯಪೋಲೀಸ್ ಸ್ಟಾರ್, ನವಂಬರ್ ೨೫, ೧೮೯೩.}

 ಸ್ವಾಮಿ ವಿವೇಕಾನಂದರು ಕಳೆದ ರಾತ್ರಿ ಮಿನಿಯಪೋಲೀಸ್‍ನಲ್ಲಿರುವ ಮೊದಲನೇ ಯೂನಿಟೆರಿಯನ್ ಚರ್ಚಿನಲ್ಲಿ ಹಿಂದೂಧರ್ಮದ ಮೇಲೆ ಮಾತನಾಡಿದರು. ಬ್ರಾಹ್ಮಣಧರ್ಮದ ಅತ್ಯಂತ ಸೂಕ್ಷ್ಮ ತತ್ತ್ವಗಳೆಲ್ಲ ಅಲ್ಲಿದ್ದುವು. ಅವು ಸನಾತನವಾದ ಸತ್ಯವಾದ ವಿಷಯಗಳನ್ನು ಒಳಗೊಂಡಿದ್ದವು. ಉಪನ್ಯಾಸಕ್ಕೆ ನೆರೆದವರಲ್ಲಿ ಮೇಧಾವಿಗಳಾದ\break ಸ್ತ್ರೀಪುರುಷರು ಇದ್ದರು. ಏಕೆಂದರೆ ಪೆರಿಪಾಟಿ ಟಿಕ್ ಸಂಘದವರು ಸ್ವಾಮೀಜಿಯವರನ್ನು ಕರೆಸಿದ್ದರು. ಅವರೊಡನೆ ಈ ಹಕ್ಕಿಗೆ ಭಾಗಿಗಳಾದ ಸ್ನೇಹಿತರು, ಹಲವು ಚರ್ಚಿಗೆ ಸೇರಿದ ಪಾದ್ರಿಗಳು, ವಿದ್ಯಾರ್ಥಿಗಳು ಮತ್ತು ವಿದ್ವಾಂಸರೂ ಇದ್ದರು. ವಿವೇಕಾನಂದರು ಬ್ರಾಹ್ಮಣ ಸಂನ್ಯಾಸಿಗಳು. ಅವರು ತಮ್ಮ ದೇಶೀಯ ವೇಷದಲ್ಲಿ ಉಪನ್ಯಾಸಮಾಡಿದರು. ತಲೆಗೆ ಅವರು ಒಂದು ರುಮಾಲನ್ನು ಕಟ್ಟಿಕೊಂಡಿದ್ದರು. ದೇಹದಮೇಲೆ ಗೈರಿಕ ನಿಲುವಂಗಿ ಇತ್ತು. ಅದನ್ನು ಸೊಂಟದ ಮೇಲೆ ಕೆಂಪುವಸ್ತ್ರದಿಂದ ಕಟ್ಟಿಕೊಂಡಿದ್ದರು. ಇದರೊಳಗೆ ಕೆಂಪುಬಣ್ಣದ ಶರಾಯಿಯನ್ನು ಹಾಕಿಕೊಂಡಿದ್ದರು. 

 ಅವರು ತಮ್ಮ ಧರ್ಮದ ವಿಷಯವನ್ನು ಬಹಳ ಶ್ರದ್ಧೆಯಿಂದ ವಿವರಿಸಿದರು. ಅವರು ನಿಧಾನವಾಗಿ ಸ್ಪಷ್ಟವಾಗಿ ಮಾತನಾಡುತ್ತಿದ್ದರು. ವೇಗವಾದ ಚಲನವಲನಗಳಿಗೆ ಅವಕಾಶಕೊಡದೆ ಮಾಧುರ್ಯದಿಂದ ತಮ್ಮ ಪ್ರಭಾವವನ್ನು ಬೀರಿದರು. ಅವರು ತಾವು ಆಡಿದ ಮಾತುಗಳನ್ನು ಬಹಳ ಜೋಪಾನವಾಗಿ ಉಪಯೋಗಿಸುತ್ತಿದ್ದರು. ಆಡಿದ ಪ್ರತಿಯೊಂದು ಮಾತೂ ಎಲ್ಲರ ಹೃದಯಕ್ಕೆ ತಾಕುವಂತೆ ಇತ್ತು. ಹಿಂದೂ ಧರ್ಮದ ಸರಳವಾದ ತತ್ತ್ವಗಳನ್ನು ಅವರು ವಿವರಿಸಿದರು. ಕ್ರೈಸ್ತಧರ್ಮದ ವಿರುದ್ಧವಾಗಿ ಏನನ್ನೂ ಹೇಳದೇ ಇದ್ದರೂ, ಎಲ್ಲರ ಮುಂದೆ ಬ್ರಾಹ್ಮಣಧರ್ಮ ಸ್ಪಷ್ಟವಾಗಿ ನಿಲ್ಲುವುದಕ್ಕೆ ಎಷ್ಟನ್ನು ಕ್ರೈಸ್ತಧರ್ಮದಿಂದ ಉದಾಹರಿಸಬೇಕೋ ಅಷ್ಟನ್ನು ಮಾತ್ರ ತೆಗೆದುಕೊಂಡರು. ಹಿಂದುಧರ್ಮದ ಸರ್ವವ್ಯಾಪಿಯಾದ ಮತ್ತು ಮುಖ್ಯವಾದ ತತ್ತ್ವವೇ ಆತ್ಮನ ಆಜನ್ಮ ಪವಿತ್ರತೆ. ಆತ್ಮ ಸ್ವಭಾವತಃ ಪರಿಪೂರ್ಣವಾದುದು. ಧರ್ಮ ಎಂದರೆ ಆಗಲೇ ಮನುಷ್ಯನಲ್ಲಿರುವ ಪವಿತ್ರತೆಯನ್ನು ವ್ಯಕ್ತಗೊಳಿಸುವುದಾಗಿದೆ. ಈಗಿನ ಸ್ಥಿತಿ ಮನುಷ್ಯನ ಹಿಂದಿನ ಸ್ಥಿತಿಯನ್ನು ತೋರುವ ಒಂದು ಎಲ್ಲೆಯಂತೆ ಇದೆ. ಅವನಲ್ಲಿರುವ ಎರಡು ಸ್ವಭಾವಗಳಲ್ಲಿ ಪವಿತ್ರತೆ ಹೆಚ್ಚಾಗಿದ್ದರೆ, ಅವನು ಉತ್ತಮಸ್ಥಿತಿಗೆ ಹೋಗುವನು, ಕೆಟ್ಟದ್ದು ಹೆಚ್ಚಾಗಿದ್ದರೆ, ಅಧೋಗತಿಗೆ ಹೋಗುವನು. ಎರಡು ಸ್ವಭಾವಗಳೂ ಯಾವಾಗಲೂ ಮನುಷ್ಯನಲ್ಲಿ ಕೆಲಸ ಮಾಡುತ್ತಿವೆ. ಯಾವುದು ಅವನನ್ನು ಮೇಲಕ್ಕೆತ್ತುವುದೋ ಅದೇ ಪುಣ್ಯ. ಯಾವುದು ಅವನನ್ನು ಅಧೋಗತಿಗೆ ಒಯ್ಯುವುದೋ ಅದೇ ಪಾಪ.


\section*{ಹಿಂದೂ ಸಂನ್ಯಾಸಿ}

~\hfill{\fontsize{11pt}{13.75pt}\selectfont ಅಪೀಲ್ ಅವಲಾಂಚ್, ೧೬ನೇ ಜನವರಿ, ೧೮೯೪}

 ಮೆಂಪಿಸ್‍ನ ಆಡಿಟೋರಿಯಂನಲ್ಲಿ ಇಂದಿನ ರಾತ್ರಿ ಉಪನ್ಯಾಸ ಮಾಡಲಿರುವ ಹಿಂದೂ ಸಂನ್ಯಾಸಿಗಳಾದ ವಿವೇಕಾನಂದರು, ದೇಶದಲ್ಲಿ ಧಾರ್ಮಿಕ ವೇದಿಕೆಯ ಮೇಲಾಗಲೀ, ಉಪನ್ಯಾಸ ವೇದಿಕೆಯ ಮೇಲಾಗಲೀ ಒಂದು ಅಪೂರ್ವ ವ್ಯಕ್ತಿ. ಅವರ ಅನುಪಮವಾದ ವಾಗ್‍ವೈಖರಿ, ರಹಸ್ಯವಾದ ವಿಷಯಗಳ ಮೇಲೆ ಅವರಿಗೆ ಇರುವ ಅಪಾರ ಅನುಭವ, ವಾದ ಮಾಡುವುದರಲ್ಲಿ ಅವರಿಗೆ ಇರುವ ಕೌಶಲ್ಯ ಮತ್ತು ಅವರಲ್ಲಿರುವ ಶ್ರದ್ಧೆ ಉತ್ಸಾಹಗಳು ವಿಶ್ವಧರ್ಮ ಸಮ್ಮೇಳನಕ್ಕೆ ಬಂದ ವಿದ್ವಾಂಸರನ್ನೆಲ್ಲ ಆಕರ್ಷಿಸಿತು. ಅಂದಿನಿಂದ ಅವರು ಅಮೇರಿಕಾ ದೇಶದಲ್ಲಿ ಕೊಟ್ಟ ಉಪನ್ಯಾಸಗಳನ್ನು ಸಹಸ್ರಾರು ಜನ ಮೆಚ್ಚಿರುವರು. 

 ಮಾತನಾಡುವಾಗ ಅವರು ಬಹಳ ಹರ್ಷಪ್ರದ ವ್ಯಕ್ತಿಗಳು. ಅವರು ಉಪಯೋಗಿಸುವ ಮಾತುಗಳಾದರೋ ಆಂಗ್ಲಭಾಷೆಯ ಅನರ್ಘ್ಯ ರತ್ನಗಳಂತೆ ಇವೆ. ಅವರು ಆಕೃತಿಯಲ್ಲಿ, ಆಚಾರ ವ್ಯವಹಾರದಲ್ಲಿ ಬಹಳ ಸುಸಂಸ್ಕೃತರಾದ ಪಾಶ್ಚಾತ್ಯ ವ್ಯಕ್ತಿಗಳಿಗೆ ಸರಿಸಮಾನರಾಗಿರುವರು. ಅವರು ಒಬ್ಬ ಅಪೂರ್ವ ಮೋಹಕ ವ್ಯಕ್ತಿ. ಸಂಭಾಷಣೆಯಲ್ಲಂತೂ ಪಾಶ್ಚಾತ್ಯ ಪ್ರಪಂಚದಲ್ಲಿ ಯಾವ ನಗರದ ದಿವಾನಖಾನೆಯಲ್ಲಿಯೂ ಮಾತಿನಲ್ಲಿ ಅವರನ್ನು ಮೀರಿಸುವವರು ಸಿಕ್ಕುವುದಿಲ್ಲ. ಅವರು ಇಂಗ್ಲೀಷ್ ಭಾಷೆಯನ್ನು ಸ್ಪಷ್ಟವಾಗಿ ಮಾತ್ರ ಮಾತನಾಡುವುದಲ್ಲ, ಅವರ ಭಾವನೆಗಳಾದರೊ ಹೊಳೆಯುತ್ತಿರುವ ನವರತ್ನಗಳಂತೆ ಇವೆ. ಅವ್ಯಾಹತವಾಗಿ ಅವರ ನಾಲಗೆಯ ಮೂಲಕ ಸುಂದರವಾದ ಭಾಷೆ ನಮಗೇ ಆಶ್ಚರ್ಯಕರವಾಗುವಂತೆ\break ಹರಿಯುತ್ತಿರುತ್ತದೆ. ವಿಶ್ವಧರ್ಮ ಸಮ್ಮೇಳನದಲ್ಲಿ ಅವರು ಮಾಡಿದ ಮೊದಲನೆ ಉಪನ್ಯಾಸದಿಂದಲೇ ಆ ಧಾರ್ಮಿಕರ ಸಂಘದಲ್ಲಿ ನಾಯಕತ್ವಕ್ಕೆ ಏರಿರುವರು. ಸಮ್ಮೇಳನದ ಕಾಲದಲ್ಲಿ ಅವರ ಧರ್ಮದ ಪರವಾಗಿ ಅನೇಕ ವೇಳೆ ಸಭಿಕರು ಅವರ ಉಪನ್ಯಾಸವನ್ನು ಕೇಳಿರುವರು. ಅವರು ಮಾನವ ಮಾನವರಿಗೆ, ಮಾನವ ಮತ್ತು ದೇವರಿಗೆ ಇರುವ ಕರ್ತವ್ಯಗಳನ್ನು ವಿವರಿಸುವಾಗ, ಇಂಗ್ಲೀಷ್ ಭಾಷೆಯಲ್ಲಿ ಶೋಭಾಯಮಾನವಾಗಿರುವಂತಹ ಅತ್ಯಂತ ಪ್ರಖ್ಯಾತವಾದ ಭಾವರತ್ನಗಳು ಅವರಿಂದ ವ್ಯಕ್ತವಾಗಿವೆ. ಅವರು ತಮ್ಮ ಭಾವನೆಗಳನ್ನು ವ್ಯಕ್ತಪಡಿಸುವುದರಲ್ಲಿ ಒಬ್ಬ ಕಲೆಗಾರರು, ತಮ್ಮ ನಂಬಿಕೆಗಳಲ್ಲಿ ಒಬ್ಬ ಆದರ್ಶವಾದಿಗಳು, ವೇದಿಕೆ ಮೇಲೆ ನಟ ಸಾರ್ವಭೌಮರು. 

 ಮೆಂಪಿಸ್‍ಗೆ ಅವರು ಬಂದಾಗಿನಿಂದಲೂ ಶ‍್ರೀ ಎಚ್.ಎಲ್. ಬ್ರನ್‍ಕ್ಲಿ ಅವರ ಅತಿಥಿಗಳಾಗಿರುವರು. ಅಲ್ಲಿಗೆ ತಮ್ಮ ಗೌರವವನ್ನು ತೋರುವುದಕ್ಕೆ ಹಗಲು ರಾತ್ರಿ ಬಂದ ಜನರನ್ನು ಸ್ವಾಗತಿಸಿರುವರು. ಟೆನಿಸ್ ಕ್ಲಬ್ಬಿನಲ್ಲಿ ಅನೌಪಚಾರಿಕವಾದ ಅತಿಥಿಗಳಾಗಿದ್ದರು. ಶನಿವಾರ ಸಾಯಂಕಾಲ ಶ‍್ರೀಮತಿ ಹೆಚ್.ಆರ್. ಶಪರ‍್ಡ ಅವರು ಕೊಟ್ಟ ಸ್ವಾಗತದಲ್ಲಿ ಸ್ವಾಮೀಜಿ ಅತಿಥಿಗಳಾಗಿದ್ದರು. ಕರ್ನಲ್ ಆರ್.ಬಿ.ಸ್ನೋಡನ್ ಅವರು ಪ್ರಖ್ಯಾತರಾದ ಸ್ವಾಮೀಜಿ ಅವರ ಗೌರವಾರ್ಥವಾಗಿ ಒಂದು ಭೋಜನ ಕೂಟವನ್ನು ಅನ್ಸ‍ಡೇಲಿನಲ್ಲಿರುವ ತಮ್ಮ ಮನೆಯಲ್ಲಿ ಭಾನುವಾರ ಏರ್ಪಡಿಸಿದ್ದರು. ಅಲ್ಲಿ ಅಸಿಸ್ಟೆಂಟ್ ಬಿಷಪ್ ಆಗಿರುವ ಥಾಮಸ್ ರೆವರೆಂಡ್ ಡಾಕ್ಟರ್ ಜಾರ್ಜ್ ಪ್ಯಾಟರ್‍ಸನ್ ಮತ್ತು ಇತರ ಪಾದ್ರಿಗಳು ಸ್ವಾಮೀಜಿಯನ್ನು\break ಸಂದರ್ಶಿಸಿದರು.


\section*{ಔದಾರ‍್ಯತೆಗಾಗಿ ಒಂದು ಮನವಿ}

~\hfill{\fontsize{11pt}{13.75pt}\selectfont ಅಮೆಂಪಿಸ್ ಕಮರ್ಷಿಯಲ್, ೧೭ನೇ ಜನವರಿ, ೧೮೯೪ }

 ನಿನ್ನೆ ರಾತ್ರಿ ಸ್ವಾಮಿ ವಿವೇಕಾನಂದರು ಹಿಂದೂಧರ್ಮದ ಮೇಲೆ ಆಡಿಟೋರಿಯಂನಲ್ಲಿ ಮಾಡಿದ ಉಪನ್ಯಾಸವನ್ನು ಕೇಳಲು ಹಲವು ಮಂದಿ ನೆರೆದಿದ್ದರು. ಜಡ್ಜ್ ಆರ್. ಜೆ. ಮೋರ್ಗನ್ ಅವರು ಸಂಕ್ಷೇಪವಾಗಿ ಉಪನ್ಯಾಸಕರನ್ನು ಪರಿಚಯ ಮಾಡಿಕೊಟ್ಟರು. ಆ ಸಮಯದಲ್ಲಿ ಪ್ರಖ್ಯಾತ ಆರ‍್ಯ ಜನಾಂಗದಿಂದಲೇ ಐರೋಪ್ಯರು ಮತ್ತು ಹಿಂದೂಗಳು ಬಂದಿರುವರು ಎಂಬುದನ್ನು ಹೇಳಿ ಉಪನ್ಯಾಸಕರಿಗೂ ಅಮೇರಿಕಾ ದೇಶೀಯರಿಗೂ ಒಂದು ಸಂಬಂಧವನ್ನು ಕಲ್ಪಿಸಿದರು. 

 ಪ್ರಖ್ಯಾತರಾದ ಪೌರ್ವಾತ್ಯ ವ್ಯಕ್ತಿಯನ್ನು ಧಾರಾಳವಾಗಿ ಕರತಾಡನಗಳಿಂದ ಸ್ವಾಗತಿಸಲಾಯಿತು. ಬಹಳ ಆಸಕ್ತಿಯಿಂದ ಅವರ ಉಪನ್ಯಾಸವನ್ನೆಲ್ಲ ಕೇಳಿದರು. ಅವರು ದೃಢಕಾಯರು, ಕಂದುಬಣ್ಣ ಮತ್ತು ನೋಡಲು ಲಕ್ಷಣವಾಗಿರುವರು. ಅವರು ಹಳದಿಯ ರೇಷ್ಮೆಯ ನಿಲುವಂಗಿಯನ್ನು ತೊಟ್ಟಿದ್ದರು. ಒಂದು ಕಪ್ಪು ವಸ್ತ್ರವನ್ನು ನಡುವಿನ ಮೇಲೆ ಕಟ್ಟಿಕೊಂಡಿದ್ದರು. ಕಪ್ಪು ಶರಾಯಿಯನ್ನು ಹಾಕಿಕೊಂಡಿದ್ದರು. ತಲೆಯ ಮೇಲೆ ಹಳದಿಯ ಬಣ್ಣದ ರೇಶ್ಮೆಯ ರುಮಾಲನ್ನು ಸುತ್ತಿಕೊಂಡು ಕುಚ್ಚನ್ನು ಕೆಳಗೆ ಇಳಿಬಿಟ್ಟಿದ್ದರು. ಅವರ ಉಚ್ಚಾರಣೆ ಬಹಳ ಚೆನ್ನಾಗಿದೆ. ಸುಂದರವಾದ ಪದಗಳನ್ನು ಬಳಸುವರು. ಭಾಷೆ ವ್ಯಾಕರಣಬದ್ಧವಾಗಿದೆ. ಪ್ರೇಕ್ಷಕರು ಲಕ್ಷ್ಯವಿಟ್ಟು ಅವರ ಮಾತುಗಳನ್ನು ಕೇಳುತ್ತಿದ್ದರು. ಅವರು ಆಸಕ್ತಿ ಇಟ್ಟು ಕೇಳಿದುದಕ್ಕೆ ತಕ್ಕ ಪ್ರತಿಫಲವೇ ದೊರೆಯಿತು ಎಂದು ಹೇಳಬಹುದು. ಅವರ ಉಪನ್ಯಾಸದಲ್ಲಿ ಸ್ವತಂತ್ರವಾದ ಭಾವನೆಗಳು, ಬೇಕಾದಷ್ಟು ವಿಷಯ ಸಂಗ್ರಹ ಮತ್ತು ಅನುಭವಗಳಿದ್ದುವು. ಉಪನ್ಯಾಸವನ್ನು ಔದಾರ‍್ಯ ಮನೋಭಾವ ಬೆಳಸಿಕೊಳ್ಳುವುದಕ್ಕೆ ಮಾಡಿದ ಕರೆ ಎಂತಲೇ ಹೇಳಬಹುದು. ಇದನ್ನು ಭರತಖಂಡದ ಧರ್ಮದ ಮೂಲಕ ವಿವರಿಸಿದರು. ಔದಾರ್ಯ ಮತ್ತು ಪ್ರೀತಿಯ ಮನೊಭಾವ ಯೋಗ್ಯವಾದ ಎಲ್ಲಾ ಧರ್ಮಗಳ ಮೂಲಸ್ಫೂರ್ತಿ. ಯಾವ ಬಗೆಯ ಧರ್ಮವಾಗಲೀ ಇದನ್ನು ಸಾಧಿಸಬೇಕಾಗಿದೆ. 

 ಕ್ರೈಸ್ತರು ವರ್ತಮಾನ ಭವಿಷ್ಯಕಾಲವನ್ನು ನಂಬುವಂತೆಯೇ ಹಿಂದೂಗಳು ಜೀವಿಯ ಹಿಂದಿನದನ್ನೂ ನಂಬುವರು. ಅವರ ಧರ್ಮ ಮಾನವನ ಆದಿಪಾಪವನ್ನು ನಂಬುವುದಿಲ್ಲ ಎಂಬುವುದನ್ನು ಅವರು ವ್ಯಕ್ತಪಡಿಸಿದರು. ಮಾನವ ಪೂರ್ಣನಾಗಬಲ್ಲ. ಅದಕ್ಕಾಗಿ ಅವನು ಮಾಡಿದ ಪ್ರಯತ್ನ ನಿಷ್ಫಲವಾಗುವುದಿಲ್ಲ... ಮಾನವನ ಪ್ರಗತಿ ಎಂದರೆ ಅವನು ಹಿಂದೆ ಇದ್ದ ಪೂರ್ಣತೆಗೆ ಹಿಂತಿರುಗುವುದು ಎಂದು ಅರ್ಥ. ಭಕ್ತಿಯ ಸಾಧನೆಯಿಂದ ಪರಿಪೂರ್ಣತೆ ಮತ್ತು ಪವಿತ್ರತೆ ಸಿದ್ಧಿಸುವುದು… ಭರತಖಂಡ ದಬ್ಬಾಳಿಕೆಗೆ ತುತ್ತಾದ ಹಲವು ರಾಷ್ಟ್ರಗಳಿಗೆ ಆಶ್ರಯವನ್ನು ಕೊಟ್ಟಿತು. ಟೈಟಸ್ ಜೆರೂಸಲಂ ನಗರವನ್ನು ಧ್ವಂಸಮಾಡಿ, ಅಲ್ಲಿಯ ದೇವಾಲಯಗಳನ್ನು ಹಾಳು ಮಾಡಿದಾಗ ಅಲ್ಲಿಂದ ಓಡಿಬಂದ ಯಹೂದ್ಯರಿಗೆ ಭರತಖಂಡ ಆಶ್ರಯವನ್ನು ಕೊಟ್ಟ ಸಂಗತಿಯನ್ನು ಅವರು ವಿವರಿಸಿದರು. 

 ಹಿಂದೂಗಳು ಬಾಹ್ಯ ಆಕಾರದ ಮೇಲೆ ಅಷ್ಟು ಪ್ರಾಮುಖ್ಯತೆಯನ್ನು ಕೊಡುವುದಿಲ್ಲ ಎಂಬುದನ್ನು ವಿವರಿಸಿದರು. ಕೆಲವು ವೇಳೆ ಒಂದೇ ಮನೆಯಲ್ಲಿರುವವರ ಇಷ್ಟದೇವತೆಗಳು ಬೇರೆ ಬೇರೆಯಾಗಿರಬಹುದು. ಆದರೆ ಎಲ್ಲರೂ ಒಂದೇ ದೇವರನ್ನು ಪೂಜಿಸುತ್ತಿರುವರು. ಹಿಂದೂಗಳು ಎಲ್ಲಾ ಧರ್ಮಗಳಲ್ಲಿಯೂ ಒಳ್ಳೆಯ ಅಂಶಗಳಿವೆ ಎಂಬುದನ್ನು ನಂಬುವರು. ಧರ್ಮಗಳೆಲ್ಲ ಮನುಷ್ಯನು ಪವಿತ್ರನಾಗುವುದಕ್ಕೆ ಮಾಡಿದ ಸನ್ನಾಹಗಳು. ಆದಕಾರಣ ಎಲ್ಲಾ ಧರ್ಮಗಳನ್ನೂ ಗೌರವಿಸಬೇಕು ಎಂದರು. ಇದನ್ನು ವೇದೋಕ್ತಿಯಿಂದ ಉದಾಹರಿಸಿದರು. 

 ಧರ್ಮಗಳೆಲ್ಲ ಒಂದು ಚಿಲುಮೆಯಿಂದ ನೀರನ್ನು ತರುವುದಕ್ಕೆ ಉಪಯೋಗಿಸುವ ಹಲವು ಆಕಾರಗಳುಳ್ಳ ಪಾತ್ರೆಗಳು ಎಂದರು. ಪಾತ್ರೆಗಳ ಆಕಾರ ವಿಧವಿಧವಾಗಿ ಇರಬಹುದು. ಆದರೆ ಅದರೊಳಗೆ ಎಲ್ಲರೂ ತುಂಬಿಸುವುದಕ್ಕೆ ಬಂದಿರುವುದು ಸತ್ಯವೆಂಬ ನೀರನ್ನೇ. ದೇವರಿಗೆ ಎಲ್ಲಾ ಧರ್ಮಗಳೂ ಗೊತ್ತಿವೆ. ಯಾವ ರೀತಿ ಅವನನ್ನು ಪ್ರಾರ್ಥಿಸಿದರೂ ಅದೆಲ್ಲ ತನಗಾಗಿ ಮಾಡಿದ ಪ್ರಾರ್ಥನೆ ಎಂಬುದು ಅವನಿಗೆ ಗೊತ್ತಿದೆ. ಕ್ರೈಸ್ತರು ಯಾವ ದೇವರನ್ನು ಪೂಜಿಸುವರೋ ಅದೇ ದೇವರನ್ನು ಹಿಂದೂಗಳು ಪೂಜಿಸುವರು. ಹಿಂದೂಗಳಲ್ಲಿರುವ ಬ್ರಹ್ಮ ವಿಷ್ಣು ಮಹೇಶ್ವರರು ಕೇವಲ ಭಗವಂತನ ಸೃಷ್ಟಿ ಸ್ಥಿತಿ ಪ್ರಳಯ ಕೆಲಸಗಳನ್ನು ಸೂಚಿಸುವ ವಿವಿಧ ಅಂಶಗಳು. ಮೂರು ದೇವರುಗಳನ್ನು ಬೇರೆ ಬೇರೆ ಎಂದು ಭಾವಿಸುವುದು ತಪ್ಪು. ಸಾಧಾರಣ ಮನುಷ್ಯನಿಗೆ ವಿಷಯವನ್ನು ಒತ್ತಿ ಹೇಳಲು ಮಾಡಿದ ಪ್ರಯತ್ನ ಇದು. ಅದರಂತೆಯೇ ಸ್ಥೂಲವಾಗಿರುವ ಹಿಂದೂಗಳ ವಿಗ್ರಹ ಕೂಡ ಭಗವಂತನ ಗುಣಗಳ ಚಿಹ್ನೆ ಅಷ್ಟೇ. 

 ಸ್ವಾಮೀಜಿಯವರ ಉಪನ್ಯಾಸವೆಲ್ಲ ಇಲ್ಲಿ ಉಲ್ಲೇಖಿಸಲು ಸಾಧ್ಯವಿಲ್ಲ. ಅದು ಮಾನವ ಸಹೋದರತ್ವಕ್ಕಾಗಿ ಮಾಡಿದ ಒಂದು ಅದ್ಭುತವಾದ ಬಿನ್ನಹ. ಅತಿ ಸುಂದರವಾದ ಧಾರ್ಮಿಕ ಭಾವನೆಯನ್ನು ಅತಿ ಚಮತ್ಕಾರವಾದ ವಾಗ್ವೈಖರಿಯ ಮೂಲಕ ವ್ಯಕ್ತಪಡಿಸಿರುವರು. ಅವರ ಭಾಷಣದ ಮುಕ್ತಾಯ ಬಹಳ ಮನೋಹರವಾಗಿತ್ತು. ತಾವು ಕ್ರಿಸ್ತನನ್ನು ಸ್ವೀಕರಿಸುವುದಕ್ಕೆ ಸಿದ್ಧನಾಗಿರುವೆ ಎಂದ ಅವರು ಜೊತೆಗೆ ಶ‍್ರೀಕೃಷ್ಣ ಮತ್ತು ಬುದ್ಧರಿಗೂ ನಮಿಸಬೇಕು ಎಂದರು. ನಮ್ಮ ನಾಗರಿಕತೆಯಲ್ಲಿರುವ ಕ್ರೌರ‍್ಯಗಳನ್ನು ನೋಡಿ, ಕ್ರಿಸ್ತ ಇದಕ್ಕೆ ಕಾರಣಕರ್ತನಲ್ಲವೆಂದರು.


\section*{ಮಾನವನ ಪವಿತ್ರತೆ }

~\hfill{\fontsize{11pt}{13.75pt}\selectfont ಡೆಟ್ರಾಯಿಟ್ ಫ್ರೀ ಪ್ರೆಸ್, ೧೮ನೇ ಫೆಬ್ರವರಿ ೧೮೯೪ }

 ಹಿಂದೂ ತತ್ತ್ವಜ್ಞಾನಿ ಮತ್ತು ಸಂನ್ಯಾಸಿ ಸ್ವಾಮಿ ವಿವೇಕಾನಂದರು ಕಳೆದ ರಾತ್ರಿ ಯೂನಿಟೇರಿಯನ್ ಚರ್ಚಿನಲ್ಲಿ ತಮ್ಮ ಬೋಧನೆಯನ್ನು ಮುಕ್ತಾಯಗೊಳಿಸಿದರು. ಅವರ ಕೊನೆಯ ಉಪನ್ಯಾಸ ಮಾನವನ ಪವಿತ್ರತೆ ಎಂಬುದು. ಹವಾಗುಣ ಅಷ್ಟು ಚೆನ್ನಾಗಿ ಇಲ್ಲದೇ ಇದ್ದರೂ ಪೌರ್ವಾತ್ಯ ಸಹೋದರನು (ಅವರಿಗೆ ಪ್ರಿಯವಾದ ಹೆಸರು) ಬರುವುದಕ್ಕೆ ಅರ್ಧಗಂಟೆ ಮುಂಚೆಯೇ ಬಾಗಿಲಿನವರೆಗೆ ಜನರು ಕಿಕ್ಕಿರಿದುಹೋಗಿದ್ದರು. ಉತ್ಸಾಹದಿಂದ ಉಪನ್ಯಾಸವನ್ನು ಕೇಳುತ್ತಿದ್ದ ಜನರಲ್ಲಿ ಜೀವನದ ಎಲ್ಲಾ ಕಾರ‍್ಯಕ್ಷೇತ್ರಕ್ಕೆ ಸೇರಿದವರೂ ಇದ್ದರು. ಲಾಯರುಗಳು, ನ್ಯಾಯಾಧಿಪತಿಗಳು, ಬೈಬಲ್ಲನ್ನು ಬೋಧಿಸುವ ಪಾದ್ರಿಗಳು, ವರ್ತಕರು ಮತ್ತು ಒಬ್ಬ ರಬ್ಬಿಯೂ (ಯಹೂದ್ಯ ಪುರೋಹಿತ) ಇದ್ದರು. ಪ್ರತಿದಿನವೂ ಅವರ ಉಪನ್ಯಾಸ ಕೇಳುವುದಕ್ಕೆ ಬರುತ್ತಿದ್ದ ಮಹಿಳೆಯರ ವಿಷಯವಾಗಿ ಹೇಳಲೇಬೇಕಾಗಿಲ್ಲ. ಅವರು ಆ ಕಂದುಬಣ್ಣ ಆಗಂತುಕ ವ್ಯಕ್ತಿಯಮೇಲೆ ತಮ್ಮ ಮೆಚ್ಚುಗೆಯ ಪುಷ್ಪವೃಷ್ಟಿಯನ್ನೇ ಕರೆದಿರುವರು. ಅವರು ವೇದಿಕೆಯಮೇಲೆ ಎಷ್ಟು ಚೆನ್ನಾಗಿ ಉಪನ್ಯಾಸ ಮಾಡಬಲ್ಲರೊ ಹಾಗೆಯೇ ದಿವಾನ್ ಖಾನೆಯ ಸಂಭಾಷಣೆಯಲ್ಲಿಯೂ ಅಗ್ರಗಣ್ಯರು. 

 ನಿನ್ನೆ ಮಾಡಿದ ಉಪನ್ಯಾಸದಲ್ಲಿ ಹಿಂದಿನ ಉಪನ್ಯಾಸಗಳ ವಿವರಣೆಗಳಿರಲಿಲ್ಲ.\break ಸುಮಾರು ಎರಡು ಗಂಟೆಗಳ ಕಾಲ ವಿವೇಕಾನಂದರು ಮಾನವ ಮತ್ತು ದೇವರಿಗೆ ಸಂಬಂಧಪಟ್ಟ ತಾತ್ತ್ವಿಕ ವಿಷಯಗಳ ರಮ್ಯವಾದ ಕಲೆಯ ಬಲೆಯನ್ನು ನೇಯ್ದರು. ಅವರ ಉಪನ್ಯಾಸ ಎಷ್ಟು ತರ್ಕಬದ್ಧವಾಗಿತ್ತು ಎಂದರೆ, ಅಲ್ಲಿ ವಿಜ್ಞಾನ ವ್ಯವಹಾರ ಜ್ಞಾನದಂತೆ ಕಂಡುಬಂದಿತು. ಅವರು ಮಾಡಿದ ಉಪನ್ಯಾಸ ತರ್ಕಬದ್ಧವಾದ ಸುಂದರವಾದ ಕೆತ್ತನೆಯ ಕೆಲಸದಂತೆ ಇತ್ತು. ಅವರ ದೇಶದಲ್ಲಿ ಕೈಮಗ್ಗದಿಂದ ನೇಯ್ದ ಅತಿ ಸುಂದರವಾದ ಆಕಾರಗಳನ್ನೊಳಗೊಂಡ, ಪಾಶ್ಚಾತ್ಯದೇಶದ ಸುಗಂಧ ದ್ರವ್ಯಗಳಿಂದ ಲೇಪಿತವಾದ ವಸ್ತ್ರದಂತೆ ಅದು ಆಕರ್ಷಣೀಯವಾಗಿತ್ತು. ಮನನ ಮಾಡಲು ಯೋಗ್ಯವಾಗಿತ್ತು. ಈ ಕಂದುಬಣ್ಣದ ಉಪನ್ಯಾಸಕರು, ಚಿತ್ರಕಾರ ಬಣ್ಣವನ್ನು ಉಪಯೋಗಿಸುವಂತೆ ಕಾವ್ಯಮಯವಾದ ಚಿತ್ರಗಳನ್ನು ಉಪಯೋಗಿಸುವರು. ಎಲ್ಲಿ ಬಣ್ಣವನ್ನು ಬಳಿಯಬೇಕೊ ಅಲ್ಲಿ ಅದನ್ನು ಇಡುವರು. ಪರಿಣಾಮ ಹೆಚ್ಚು ಅಲಂಕಾರಮಯವಾಗುವುದು. ಆದರೆ ಅದರಲ್ಲಿ ಒಂದು ವಿಚಿತ್ರವಾದ ಆಕರ್ಷಣೆ ಇತ್ತು. ವೇಗವಾಗಿ ಬರುತ್ತಿದ್ದ ತರ್ಕಬದ್ಧವಾದ ನಿರ್ಣಯಗಳು ಬಣ್ಣ ಬಣ್ಣದ ಗಾಜಿನ ಮೂಲಕ ಹೊಳೆಯುವಂತೆ ಇದ್ದವು. ಇದನ್ನು ಬಳಸುವುದರಲ್ಲಿ ಕುಶಲಿಗಳಾದ ಪ್ರೇಕ್ಷಕರಿಂದ ಬರುತ್ತಿದ್ದುವು. 

 ಉಪನ್ಯಾಸಕ್ಕೆ ಮುಂಚೆ, ಸಭಿಕರು ಅನೇಕ ಪ್ರಶ್ನೆಗಳನ್ನು ಹಾಕಿರುವರು ಎಂದು ಪೀಠಿಕೆಯನ್ನು ಕೊಟ್ಟರು. ಇವುಗಳಲ್ಲಿ ಕೆಲವನ್ನು ಏಕಾಂತವಾಗಿ ಉತ್ತರಿಸುವೆನು ಎಂದರು. ಅವರಿಗೆ ಹಾಕಿದ ಪ್ರಶ್ನೆಗಳಲ್ಲಿ ಬಹಿರಂಗವಾಗಿ ವೇದಿಕೆಯ ಮೇಲಿನಿಂದ ಉತ್ತರಿಸುವುದಕ್ಕೆ ಮೂರು ಪ್ರಶ್ನೆಗಳನ್ನು ತೆಗೆದುಕೊಂಡರು. ಆ ಪ್ರಶ್ನೆಗಳೇ ಇವು: 

 “ಭರತಖಂಡದ ಜನರು ಮಕ್ಕಳನ್ನು ಮೊಸಳೆಯ ಬಾಯಿಗೆ ಕೊಡುವರೆ?” 

 “ಜಗನ್ನಾಥ ದೇವರ ರಥದ ಮುಂದೆ ಬಿದ್ದು ಆತ್ಮಹತ್ಯೆಯನ್ನು ಮಾಡಿಕೊಳ್ಳುವರೆ?” 

 “ವಿಧವೆಯರನ್ನು ಅವರ ಗಂಡನ ಶವದೊಡನೆ ಸುಡುವರೆ?” 

 ಮೊದಲನೆ ಪ್ರಶ್ನೆಯನ್ನು ತೆಗೆದುಕೊಂಡು, ಅಮೇರಿಕಾ ದೇಶೀಯನು ಪರದೇಶಕ್ಕೆ ಹೋದರೆ ಅಲ್ಲಿ ನ್ಯೂಯಾರ್ಕ್ ನಗರದ ಬೀದಿಯಲ್ಲಿ ರೆಡ್ ಇಂಡಿಯನ್ನರನ್ನು ಓಡಿಸಿಕೊಂಡು ಹೋಗುತ್ತಿರುವರೆ? ಎಂಬ ಪ್ರಶ್ನೆಯನ್ನು ಪರಿಗಣಿಸುವಂತೆಯೇ ಇದನ್ನು\break ಪರಿಗಣಿಸುವರು. ಪ್ರಶ್ನೆ ತುಂಬಾ ಹಾಸ್ಯಾಸ್ಪದವಾಗಿತ್ತು. ಅದಕ್ಕೆ ಒಂದು ಉತ್ತರವನ್ನು ಕೊಡುವಷ್ಟು ಯೋಗ್ಯವಲ್ಲ ಎಂದರು. 

 ಕೆಲವರು “ಹೆಣ್ಣುಮಕ್ಕಳನ್ನು ಏತಕ್ಕೆ ಇಂಡಿಯ ದೇಶದಲ್ಲಿ ಮೊಸಳೆಯ ಬಾಯಿಗೆ ಎಸೆಯುತ್ತಾರೆ?” ಎಂದು ಪ್ರಶ್ನೆ ಕೇಳಿದರು. ಸ್ವಾಮೀಜಿ ನಗುತ್ತ “ಬಹುಶಃ ಹೆಣ್ಣುಮಕ್ಕಳು ಎಳೆಯದಾಗಿ ಮೃದುವಾಗಿದ್ದವು ಎಂತಲೇನೋ! ಆ ಪುಣ್ಯನದಿಯಲ್ಲಿ ವಾಸಿಸುವ ಮೊಸಳೆಗಳು ಚೆನ್ನಾಗಿ ಅಗಿದು ಜೀರ್ಣಿಸಿಕೊಳ್ಳುವುದಕ್ಕೆ ಸುಲಭವಾಗಿದ್ದುದರಿಂದಲೇನೋ!” ಆತ್ಮಹತ್ಯೆ ಮಾಡಿಕೊಳ್ಳುವ ಪ್ರಸಂಗವನ್ನು ತೆಗೆದುಕೊಂಡರು. “ಬಹುಶಃ ಕೆಲವರು ಉತ್ಸಾಹದಿಂದ ಜಗನ್ನಾಥನ ರಥವನ್ನು ಎಳೆಯಬೇಕೆಂದು ಹಗ್ಗವನ್ನು ಹಿಡಿದುಕೊಳ್ಳುವುದಕ್ಕೆ ಹೋದಾಗ ಜಾರಿಬಿದ್ದು ಗಾಲಿಗೆ ಸಿಕ್ಕಿ ಸತ್ತಿರಬಹುದು. ಇಂತಹ ಕೆಲವು ಪ್ರಸಂಗಗಳನ್ನು ಉತ್ಪ್ರೇಕ್ಷೆಮಾಡಿ ಸತ್ಯ ದೂರವಾಗಿ ಮಾಡಿರಬೇಕು. ನಿಮ್ಮ ದೇಶದ ಜನ ಇಂತಹ ಘಟನೆಯನ್ನು ಕೇಳಿ ಕರಗಿಹೋಗಿರಬೇಕು ಎಂದರು. 

 ವಿವೇಕಾನಂದರು ವಿಧವೆಯರನ್ನು ಸುಡುವುದು ಸುಳ್ಳು ಎಂದು ಹೇಳಿದರು.\break ವಿಧವೆಯರು ಕೆಲವು ವೇಳೆ ತಾವೇ ಬಿಂಕಿಗೆ ಬಿದ್ದದ್ದು ನಿಜವಿರಬಹುದು. ಹೀಗೆ ಆದ ಕೆಲವು ಘಟನೆಗಳಿಗೆ ಪೂರ್ವದಲ್ಲಿ ಪುರೋಹಿತರು ಮತ್ತು ಗುರುಹಿರಿಯರು ಹೀಗೆ ಮಾಡುವುದು ಒಳ್ಳೆಯದಲ್ಲ ಎಂದು ಅವರಿಗೆ ಎಷ್ಟೋ ಹೇಳುತ್ತಿದ್ದರು. ಅವರು ಯಾವಾಗಲೂ ಆತ್ಮಹತ್ಯೆಗೆ ವಿರೋಧವಾಗಿದ್ದರು. ಆದರೆ ಅಲ್ಲಿ ಪತಿವ್ರತೆಯರಾದ ವಿಧವೆಯರು, ತಾವು ಗಂಡನನ್ನು\break ಅನುಸರಿಸಿ ಹೋಗುತ್ತೇವೆ ಎಂದು ಹಟ ಹಿಡಿದರೋ ಆಗ ಅವರನ್ನು ಒಂದು ಅಗ್ನಿಪರೀಕ್ಷೆಗೆ ಗುರಿಮಾಡುತ್ತಿದ್ದರು. ಅದೇ ಉರಿಗೆ ಕೈಯನ್ನು ಒಡ್ಡುವುದು. ಅದು ಸುಡುತ್ತಿರುವಾಗ ಅವರು ಸ್ವಲ್ಪವೂ ಹಿಂದೆಗೆಯದೇ ಇದ್ದರೆ, ಅನಂತರ ಚಿತೆಯಲ್ಲಿ ಬೀಳುವುದಕ್ಕೆ ಅವಕಾಶ ಕೊಡುತ್ತಿದ್ದರು. ಆದರೆ ಪ್ರೀತಿಸುತ್ತಿದ್ದ ಪತಿಯನ್ನು ಮೃತ್ಯುವಿನ ಅನಂತರವೂ ಅನುಸರಿಸಿದ ಸತಿಯರು ಇಂಡಿಯಾ ದೇಶ ಒಂದರಲ್ಲೆ ಅಲ್ಲ, ಬೇರೆ ಬೇರೆ ದೇಶಗಳಲ್ಲಿಯೂ ಇರುವರು. ಇಂತಹ ಆತ್ಮಹತ್ಯೆಗಳು ಪ್ರತಿಯೊಂದು ದೇಶದಲ್ಲೂ ಆಗಿವೆ. ಇದು ಎಲ್ಲಾ ದೇಶದಲ್ಲಿಯೂ ಇರುವ ಒಂದು ಅಸ್ವಾಭಾವಿಕವಾದ ಆಚಾರ. ಇಲ್ಲ, ಇಂಡಿಯಾ ದೇಶದಲ್ಲಿ ಜನರು ಎಂದಿಗೂ ಸ್ತ್ರೀಯರನ್ನು ಸುಡುವುದಿಲ್ಲ; ಅಥವಾ ಅವರು ಮಾಟಗಾತಿಯರನ್ನು (\enginline{Witches}) ಎಂದಿಗೂ ಸುಟ್ಟಿಲ್ಲ. 

 ಅನಂತರ ಸ್ವಾಮೀಜಿ ಮುಖ್ಯ ಉಪನ್ಯಾಸವನ್ನು ಪ್ರಾರಂಭ ಮಾಡಿದರು.\break ವಿವೇಕಾನಂದರು ಜೀವಿಯ, ದೇಹ ಮನಸ್ಸು ಆತ್ಮಗಳನ್ನು ವಿವರಿಸತೊಡಗಿದರು. ದೇಹ ಕೇವಲ ಒಂದು ಗೂಡಿನಂತೆ. ಮನಸ್ಸು ಅದರ ಮೂಲಕ ಕೆಲಸ ಮಾಡುವುದು. ಆತ್ಮನಿಗಾದರೋ ವೈಶಿಷ್ಟ್ಯವಿದೆ. ಆತ್ಮದ ಅನಂತತೆಯನ್ನು ಅರಿಯುವುದೇ ಮುಕ್ತಿ. ನಮ್ಮ ಮನಸ್ಸಿಗೆ ಒಪ್ಪುವ ರೀತಿಯಲ್ಲಿ ಅವರು ವಾದಿಸುತ್ತ ಪ್ರತಿಯೊಂದು ಆತ್ಮವೂ ಸ್ವತಂತ್ರ ಎಂದು ಹೇಳಿದರು. ಅದೇನಾದರೂ ಬದ್ಧವಾಗಿದ್ದರೆ ಅನಂತರ ಅದು ಅಮರತ್ವವನ್ನು ಪಡೆಯುತ್ತಿರಲಿಲ್ಲ. ಇದರ ಸಾಕ್ಷಾತ್ಕಾರ ಒಬ್ಬನಿಗೆ ಹೇಗೆ ಬರುತ್ತದೆ ಎಂಬುದನ್ನು ವಿವರಿಸುವುದಕ್ಕೆ ಅವರ ದೇಶದಲ್ಲಿ ರೂಢಿಯಲ್ಲಿರುವ ಒಂದು ಕಥೆಯನ್ನು ಹೇಳಿದರು. ಒಂದು ಗರ್ಭಿಣಿ ಸಿಂಹ ಕುರಿಯ ಮಂದೆಯ ಮೇಲೆ ನೆಗೆದಾಗ ಒಂದು ಮರಿಯನ್ನು ಹೆತ್ತು ಸತ್ತುಹೊಯಿತು. ಕುರಿಗಳು ಸಿಂಹದ ಮರಿಗೆ ಹಾಲನ್ನು ಕೊಟ್ಟು ಸಂರಕ್ಷಿಸಿದವು. ಈ ಸಿಂಹದ ಮರಿ ಕುರಿಯ ಮಂದೆಯಲ್ಲಿಯೇ ಬೆಳೆಯುತ್ತ ಹೋಯಿತು. ಆ ಕುರಿಗಳಂತೆಯೇ ವ್ಯವಹರಿಸತೊಡಗಿತು. ಆದರೆ ಒಂದು ದಿನ ಮತ್ತೊಂದು ಸಿಂಹ ಬಂತು. ಅದು ಕುರಿಯ ಮಂದೆಯಲ್ಲಿದ್ದ ಮತ್ತೊಂದು ಸಿಂಹವನ್ನು ಒಂದು ಬಾವಿಯ ಸಮೀಪಕ್ಕೆ ಕರೆದುಕೊಂಡು ಹೋಗಿ, ತನ್ನ ನೆರಳು ಮತ್ತು ಅದರ ನೆರಳನ್ನು ತೋರಿಸಿತು. ಆಗ ಕುರಿಯ ಮಂದೆಯಲ್ಲಿದ್ದ ಸಿಂಹ ತಾನು ಕೂಡ ಸಿಂಹವೇ ಎಂಬುದನ್ನು ಅರಿತುಕೊಂಡಿತು. ಅನೇಕ ಜನ ಕುರಿಯಂತೆ ಇರುವ ಸಿಂಹಗಳು. ಒಂದು ಮೂಲೆಗೆ ಹೋಗಿ ತಾವು ಪಾಪಿಗಳೆಂದೂ ಯಾವ ಕೆಲಸಕ್ಕೂ ಬಾರದವರೆಂದೂ ಭಾವಿಸುವರು. ತಮ್ಮಲ್ಲೆ ಹುದುಗಿರುವ ಪೂರ್ಣತೆಯನ್ನಾಗಲಿ ಪವಿತ್ರತೆಯನ್ನಾಗಲೀ ಇನ್ನೂ ತಿಳಿದುಕೊಂಡಿಲ್ಲ. ಸ್ತ್ರೀಪುರುಷರ ಅಹಂಕಾರವೇ ಆತ್ಮ. ಅದು ಸ್ವತಂತ್ರವಾಗಿದ್ದರೆ ಅದನ್ನು ಪೂರ್ಣವಾದ ಅನಂತದಿಂದ ಹೇಗೆ ಪ್ರತ್ಯೇಕಿಸಲು ಸಾಧ್ಯ? ಸೂರ್ಯ ದೊಡ್ಡದೊಂದು ಸರೋವರದ ಮೇಲೆ ಬೆಳಗುತ್ತಿರುವನು. ಆಗ ಅನೇಕ ಪ್ರತಿಬಿಂಬಗಳು ಕಾಣುವುವು. ಪ್ರತಿಬಿಂಬಗಳಿಗೆಲ್ಲ ಸೂರ್ಯನೇ ಮೂಲ ಎಂದು ಗೊತ್ತಾಗಿದ್ದರೂ ಪ್ರತಿಯೊಂದು ಪ್ರತಿಬಿಂಬವನ್ನೂ ಬೇರೆ ಬೇರೆ ಎಂದು ಭಾವಿಸುವಂತೆ ಇದೆ ಇದು. ಜೀವನಿಗೆ ಯಾವ ಲಿಂಗವೂ ಇಲ್ಲ; ನಿರಪೇಕ್ಷವಾದ ಮುಕ್ತಿಯನ್ನು ಅದು ಪಡೆದ ಮೇಲೆ ದೈಹಿಕವಾಗಿ ಲಿಂಗವನ್ನು ಕಟ್ಟಿಕೊಂಡು\break ಅದೇನು ಮಾಡುವುದು? ಈ ವಿಷಯದಲ್ಲಿ ಉಪನ್ಯಾಸಕರು ಸ್ವೀಡನ್‍ಬರ್ಗನ ಧರ್ಮತತ್ತ್ವದ ವಿಷಯವನ್ನು ಕೂಲಂಕುಷವಾಗಿ ವಿಚಾರಿಸಿದರು. ಹಿಂದೂಗಳ ಭಾವನೆಗೂ ಇತ್ತೀಚೆಗೆ ಬಂದ ಒಬ್ಬ ಮಹಾತ್ಮನ ಭಾವನೆಗೂ ಇರುವ ಸಾಮಾನ್ಯ ವಿಷಯಗಳು ಸ್ಪಷ್ಟವಾಗಿ ಗೊತ್ತಾದವು… ಪ್ರತಿಯೊಂದು ವ್ಯಕ್ತಿಯ ಅಂತರಾಳದಲ್ಲಿಯೂ ಪರಿಪೂರ್ಣತೆ ಇದೆ. ಅದು ಅವನ ದೇಹದ ಹಿಂದೆ ರಹಸ್ಯವಾಗಿರುವ ಗುಹೆಯಲ್ಲಿ ಅವಿತಿರುವುದು. ದೇವರು ತನ್ನಲ್ಲಿರುವ ಪೂರ್ಣತೆಯನ್ನು ಮನುಷ್ಯನಿಗೆ ಸ್ವಲ್ಪ ದಾನ ಮಾಡಿದುದರಿಂದ ಮನುಷ್ಯ ಪವಿತ್ರನಾದ ಎಂದು ಭಾವಿಸಿದರೆ, ಇದನ್ನು ದಾನ ಮಾಡಿದಮೇಲೆ ದೇವರ ಪೂರ್ಣತೆಯಲ್ಲಿ ಅಷ್ಟು ಕಡಿಮೆಯಾಗುವುದು. ವೈಜ್ಞಾನಿಕ ನಿಯಮವು ಮಾರ್ಪಡಿಸಲಾಗದ ಪೂರ್ಣತೆ ಆಗಲೇ ಆತ್ಮನಲ್ಲಿ ಸುಪ್ತವಾಗಿದೆ ಎಂದು ಸಾರುವುದು. ಅದನ್ನು ಪಡೆಯುವುದೇ ಮೋಕ್ಷ ಅಥವಾ ಮುಕ್ತಿ. 

 ಧರ್ಮಗಳೆಲ್ಲ ಒಳ್ಳೆಯವೆ. ಒಂದು ಗ್ಲಾಸಿನ ಬುಡ್ಡಿಯ ಕೆಳಗೆ ಇರುವ ಗಾಳಿಯ ಗುಳ್ಳೆ ಮೇಲಿರುವ ಅನಂತ ಗಾಳಿಯಲ್ಲಿ ಒಂದಾಗಬಯಸುವುದು. ಎಣ್ಣೆ, ವೆನಿಗರ್ ಮತ್ತು ಇತರ ದ್ರವ್ಯಗಳಲ್ಲಿ ಅದರ ಘನದ ತರತಮಕ್ಕೆ ತಕ್ಕಂತೆ ಗುಳ್ಳೆ ಮೇಲೇಳುವುದಕ್ಕೆ ಬೇಕಾಗುವ ಕಾಲ ವ್ಯತ್ಯಾಸವಾಗುವುದು. ಇದರಂತೆಯೆ ಜೀವ ಹಲವು ವಾತಾವರಣಗಳ ಮೂಲಕ ತನ್ನ ಅನಂತತೆಯನ್ನು ಪಡೆಯಲು ಯತ್ನಿಸುವುದು. ಒಂದು ಧರ್ಮ ಒಬ್ಬರಿಗೆ ಸರಿಯಾಗಿರುವುದು. ಏಕೆಂದರೆ ಅವರ ಜೀವನದ ಪರಸ್ಪರ ಅಭ್ಯಾಸ, ಇರುವ ವಾತಾವರಣ, ಆನುವಂಶಿಕವಾಗಿ ಬಂದ ಸ್ವಭಾವಗಳು, ಮತ್ತು ಅಲ್ಲಿಯ ಹವಾಗುಣದ ಪರಿಣಾಮ ಇವುಗಳೆಲ್ಲದರ ಪರಿಣಾಮ ಅವರ ಮೇಲೆ ಬೀಳುವುದು. ಇದೇ ಕಾರಣಕ್ಕಾಗಿಯೇ ಬೇರೊಂದು ಧರ್ಮ ಮತ್ತೊಬ್ಬನಿಗೆ ಅನ್ವಯಿಸುವುದು. ಈಗ ಇರುವುದೆಲ್ಲ ಒಳ್ಳೆಯದೆ ಎಂದು ಹೇಳಿದಂತೆ ಇತ್ತು ಉಪನ್ಯಾಸಕರು ಭಾಷಣವನ್ನು ಮುಕ್ತಾಯಮಾಡಿದಾಗ. ಒಂದು ದೇಶದ ಧರ್ಮವನ್ನು ಇದ್ದಕ್ಕಿದ್ದಂತೆ ಬದಲಾವಣೆ ಮಾಡುವುದು ಆಲ್ಫ್ಸ್ ಪರ್ವತದಿಂದ ಹರಿದುಬಂದ ನದಿಯನ್ನು ನೋಡಿ, ಅದು ಹರಿದುಕೊಂಡು ಬಂದ ಮಾರ್ಗದಲ್ಲಿ ತಪ್ಪನ್ನು ಕಂಡುಹಿಡಿದಂತೆ. ಮತ್ತೊಬ್ಬ ಒಂದು ಮಹಾಪ್ರವಾಹ ಹಿಮಾಲಯದಿಂದ ಹರಿದುಕೊಂಡು ಬರುವುದನ್ನು ನೋಡುವನು. ಅದು ಹಲವಾರು ತಲೆಮಾರುಗಳಿಂದ ಸಹಸ್ರಾರು ವರ್ಷಗಳಿಂದಲೂ ಹರಿದು ಬರುತ್ತಿದೆ. ಅದನ್ನು ನೋಡಿ ಶೀಘ್ರವಾದ ಸರಿಯಾದ ಮಾರ್ಗವನ್ನು ಈ ನದಿ ಆರಿಸಿಕೊಂಡಿಲ್ಲ ಎಂದು ತಪ್ಪು ಕಂಡುಹಿಡಿಯುವನು. ಕ್ರೈಸ್ತರಿಗೆ ದೇವರು ಎಂದರೆ ಎಲ್ಲೋ ನಮ್ಮ ಮೇಲೆ ಕುಳಿತಿರುವನು, ಎಂದು ತೋರುತ್ತದೆ. ಕ್ರೈಸ್ತನು ಸ್ವರ್ಗದಲ್ಲಿದ್ದರೂ ಅಲ್ಲಿಯ ಚಿನ್ನದ ಬೀದಿಯ ಕೊನೆಯಲ್ಲಿ ನಿಂತು ಪ್ರಪಂಚಕ್ಕೂ ಅದಕ್ಕೂ ಇರುವ ವ್ಯತ್ಯಾಸವನ್ನು ನೋಡದೆ ಇದ್ದರೆ ಅವನು ತೃಪ್ತನಾಗಲಾರ. ನಿನ್ನನ್ನು ಹೇಗೆ ಕಾಣುವರೋ ಹಾಗೆ ನೀನು ಅವರನ್ನು ಕಾಣು ಎಂಬ ಸಂದೇಶಕ್ಕಿಂತ ಹೆಚ್ಚಾಗಿ ಹಿಂದೂಗಳು ಎಲ್ಲಾ ನಿಸ್ವಾರ್ಥತೆಯು ಪುಣ್ಯ, ಸ್ವಾರ್ಥ ಪಾಪ ಎಂದು ನಂಬುವರು. ಈ ನಂಬಿಕೆಯ ಮೂಲಕ ಜೀವಿ ಸರಿಯಾದ ಕಾಲದಲ್ಲಿ ಪೂರ್ಣತೆ ಮತ್ತು ಅನಂತತೆಯನ್ನು ಪಡೆಯಲು ಸಾಧ್ಯವಾಗುವುದು. 

\newpage

 ತಾವು ಕ್ರೈಸ್ತರನ್ನು ಮತಾಂತರಗೊಳಿಸುವುದಕ್ಕೆ ಇಚ್ಛಿಸುವುದಿಲ್ಲ ಎಂದು ಹೇಳಿದರು. ಅವರು ಕ್ರೈಸ್ತರು, ಅದು ಒಳ್ಳೆಯದೆ. ತಾವು ಹಿಂದೂಗಳು; ಅದೂ ಒಳ್ಳೆಯದೆ. ತಮ್ಮ ದೇಶದಲ್ಲಿ ಬುದ್ಧಿ ಬೆಳವಣಿಗೆಗೆ ತಕ್ಕಂತೆ ಧರ್ಮವನ್ನು ಕೊಡುವರು. ಇವುಗಳು ಆಧ್ಯಾತ್ಮಿಕ ವಿಕಾಸಕ್ಕೆ ಸಹಾಯಮಾಡುವುವು. ಹಿಂದೂ ಧರ್ಮ ಅಹಂಕಾರದ ಮೇಲೆ ನಿಂತಿಲ್ಲ. ಎಂದಿಗೂ ಅದು ತನ್ನ ಅಹಂಕಾರವನ್ನು ಮೆರೆಸಲಿಲ್ಲ. ಎಂದಿಗೂ ಬಹುಮಾನ ಮತ್ತು ಶಿಕ್ಷೆಯನ್ನು ಅದು ಜನರಿಗೆ ತೋರಿಸಲಿಲ್ಲ. ನಿಸ್ವಾರ್ಥದಿಂದ ಒಬ್ಬನಿಗೆ ಮುಕ್ತಿ ದೊರಕುವುದು ಎಂಬುದನ್ನು ಅದು ತೋರುವುದು. ಕ್ರೈಸ್ತ ಧರ್ಮ ಸಾಕ್ಷಾತ್ ದೇವರಿಂದ ಬಂದಿತೆಂಬುದು, ಪ್ರಪಂಚದಲ್ಲಿ ದೇವರು ಕ್ರಿಸ್ತನೊಬ್ಬನಿಗೆ ಮಾತ್ರ ಕಾಣಿಸಿಕೊಂಡನೆಂಬುದು, ಕ್ರೈಸ್ತರಾದವರಿಗೆ ಲಂಚಕೊಡುವುದು ಮುಂತಾದುವೆಲ್ಲ ಬಹಳ ಹೀನಾಯವಾದ ಭಾವನೆಗಳು. ಇವು ಮನುಷ್ಯನನ್ನು ಬಹಳ ಅಧೋಗತಿಗೆ ಒಯ್ಯುವುವು.

~\hfill{\fontsize{11pt}{13.75pt}\selectfont ಡೆಟ್ರಾಯಿಟ್ ಜರ‍್ನಲ್, ಫೆಬ್ರವರಿ ೧೨, ೧೮೯೪}

 ಈ ಊರಿನಲ್ಲಿ ಉಪನ್ಯಾಸ ಮಾಡುತ್ತಿರುವ ಬ್ರಾಹ್ಮಣ ಸ್ವಾಮೀಜಿ ವಿವೇಕಾನಂದರನ್ನು ಮತ್ತೊಂದು ವಾರ ಇರುವಂತೆ ಮಾಡಿದರೆ ಡೆಟ್ರಾಯಿಟ್‍ನಲ್ಲಿ ಇರುವ ಅತಿ ದೊಡ್ಡ ಹಾಲಿಗೂ ಸ್ವಾಮೀಜಿಯವರ ಉಪನ್ಯಾಸಗಳನ್ನು ಕೇಳಲು ಬರುವವರನ್ನು ಹಿಡಿಸುವುದು ಕಷ್ಟವಾಗುವುದು. ಜನರಿಗೆ ಅವರೊಂದು ಆಕರ್ಷಣೆ ಆಗಿರುವರು. ಕಳೆದ ಸಾಯಂಕಾಲದ ಉಪನ್ಯಾಸಕ್ಕೆ ಹಾಲಿನ ಪ್ರತಿಯೊಂದು ಸ್ಥಳವೂ ಭರ್ತಿಯಾಗಿತ್ತು. ಅನೇಕರು ಉಪನ್ಯಾಸ ಮುಗಿಯುವವರೆಗೆ ನಿಂತುಕೊಂಡೇ ಕೇಳಿದರು. 

 ಉಪನ್ಯಾಸಕರು ಭಗವತ್ ಪ್ರೇಮ ಎಂಬ ವಿಷಯದ ಮೇಲೆ ಮಾತನಾಡಿದರು. ಅವರು ಭಕ್ತಿ ಎಂದರೆ ಇತ್ತ ವಿವರಣೆ ಇದು: “ಅಲ್ಲಿ ಸ್ವಾರ್ಥ ಎಳ್ಳಷ್ಟೂ ಇರುವುದಿಲ್ಲ. ನಮ್ಮ ಪ್ರೀತಿ ವಿಶ್ವಾಸಗಳನ್ನೆಲ್ಲ ಅರ್ಪಿಸುವ ಭಗವಂತನನ್ನು ಸ್ತುತಿಸುವುದು ವಿನಃ ಅಲ್ಲಿ ಮತ್ತೆ ಯಾವುದೂ ಇಲ್ಲ. ಪ್ರೀತಿ ಭಗವಂತನಿಗೆ ಬಾಗಿ ನಮಸ್ಕರಿಸುವುದು. ಅವನಿಂದ ಮತ್ತೇನು ಪ್ರತಿಫಲವನ್ನೂ ಕೇಳುವುದಿಲ್ಲ.” ಭಗವತ್ ಪ್ರೀತಿ ಎಂದರೆ ಬೇರೆ. ನಮಗೆ ದೇವರು ಅತ್ಯಂತ ಆವಶ್ಯಕ ಎಂದು ಅವನನ್ನು ನಾವು ಸ್ವೀಕರಿಸುವುದಿಲ್ಲ. ಕೇವಲ ಸ್ವಾರ್ಥ ದೃಷ್ಟಿಯಿಂದ ಅವನ ಬಳಿಗೆ ಹೋಗುತ್ತೇವೆ. ಅವರ ಉಪನ್ಯಾಸದಲ್ಲಿ ಬೇಕಾದಷ್ಟು ಕಥೆಗಳು ಮತ್ತು ಉದಾಹರಣೆಗಳು ಇದ್ದವು. ಅವುಗಳೆಲ್ಲ ಭಗವತ್ ಪ್ರೇಮದಲ್ಲಿರುವ ಸ್ವಾರ್ಥವನ್ನು ತೋರುವುದಕ್ಕೆ ಆಗಿತ್ತು. ಕ್ರೈಸ್ತರ ಬೈಬಲ್ಲಿನಲ್ಲಿ ಸಾಲಮನ್ನನ ಪ್ರಾರ್ಥನೆಗಳು ಅತ್ಯಂತ ಸುಂದರವಾದ ಭಾಗ; ಅದನ್ನು ಬೈಬಲ್ಲಿನಿಂದ ತೆಗೆದು ಹಾಕುವರು ಎಂಬುದನ್ನು ಕೇಳಿ ತಮಗೆ ವಿಷಾದವಾಗಿದೆ ಎಂದು ಹೇಳಿದರು. ಕೊನೆಗೆ ಕಟುವಾಗಿ “ದೇವರ ಮೇಲೆ ನಮಗಿರುವ ಪ್ರೀತಿ ಅವನಿಂದ ನಾವು ಏನನ್ನು ಪಡೆಯೋಣ” ಎಂಬುದರ ಮೇಲೆ ನಿಂತಿದೆ ಎಂದರು. ಕ್ರೈಸ್ತರ ಪ್ರೇಮ ಸ್ವಾರ್ಥದಿಂದ ಕೂಡಿದೆ. ಅವರು ಯಾವಾಗಲೂ ದೇವರನ್ನು ಏನನ್ನಾದರೂ ಬೇಡುತ್ತಲೇ ಇರುವರು. ಆದಕಾರಣ ಆಧುನಿಕ ಧರ್ಮ ಒಂದು ಶೋಕಿ ಆಗಿದೆ. ಕುರಿಯ ಮಂದೆಯಂತೆ ಜನ ಚರ್ಚಿಗೆ ಹೋಗುವರು ಎಂದರು.

