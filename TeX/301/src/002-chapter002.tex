
\chapter{ಜನನ ಮತ್ತು ಬಾಲ್ಯ}

ಭುವನೇಶ್ವರೀದೇವಿಗೆ ಎರಡು ಹೆಣ್ಣು ಮಕ್ಕಳುಗಳಾಗಿದ್ದರೂ ಗಂಡು ಮಕ್ಕಳಾಗಿರಲಿಲ್ಲ. ವಂಶವನ್ನು ಮುಂದುವರಿಸುವುದಕ್ಕೆ ಒಬ್ಬ ಪುತ್ರನಿಗಾಗಿ ಹಾತೊರೆಯುವರು ಹಿಂದೂ ದಂಪತಿಗಳು. ಹಿಂದಿನ ಕಾಲದಿಂದಲೂ ಅದಕ್ಕೆ ಪ್ರಾಮುಖ್ಯತೆ. ಇದಕ್ಕಾಗಿ ಯಾತ್ರೆ ಇತ್ಯಾದಿಗಳನ್ನು ಮಾಡುತ್ತಿದ್ದರು. ಏಕೆಂದರೆ ಪುತ್ರ ವಂಶದ ಧಾರ್ಮಿಕ ಪರಂಪರೆಯನ್ನು ಮುಂದುವರಿಸಿಕೊಂಡು ಹೋಗತಕ್ಕವನು. ಭುವನೇಶ್ವರೀದೇವಿಯೂ ಒಬ್ಬ ಪುತ್ರನನ್ನು ಕರುಣಿಸೆಂದು ಭಗವಂತನಿಗೆ ಪ್ರಾರ್ಥನೆ ಮಾಡಿದಳು. ಅದಕ್ಕಾಗಿ ವ್ರತ ಪೂಜೆಗಳನ್ನು ಮಾಡಿದಳು. ಕಾಶಿಯಲ್ಲಿದ್ದ ಅವರ ನೆಂಟರೊಬ್ಬರಿಗೆ ತಮ್ಮ ಮನೆದೇವರಾದ ವೀರೇಶ್ವರನಿಗೆ ತನ್ನ ಪರವಾಗಿ ಒಂದು ಹರಕೆಯನ್ನು ಸಲ್ಲಿಸು ಎಂದು ಕಾಗದ ಬರೆದಳು. ಅದೆಲ್ಲವನ್ನು ಮಾಡಿ ಕಾಯುತ್ತಿದ್ದಾಗ ಒಂದು ದಿನ ರಾತ್ರಿ ಅವಳಿಗೊಂದು ಕನಸಾಯಿತು. ಆ ಕನಸಿನಲ್ಲಿ ತೇಜಃಪುಂಜ ಮೂರ್ತಿಯಾದ ಶಿವ ಧ್ಯಾನಮಗ್ನನಾಗಿದ್ದುದನ್ನು ನೋಡಿದಳು. ನೋಡುತ್ತ ಇದ್ದಹಾಗೆಯೇ ಅವನ ಕಾಂತಿ ಒಂದು ಮುದ್ದು ಬಾಲಕನ ರೂಪವನ್ನು ಧರಿಸಿ ಇವಳ ಸಮೀಪಕ್ಕೆ ಬಂದಿತು. ಇಷ್ಟು ಹೊತ್ತಿಗೆ ಕನಸು ಜಾರಿತು. ಆದರೂ ಆ ಕನಸಿ\-ನಲ್ಲಿ ಆದ ಆನಂದ ಇವಳ ಹೃದಯವನ್ನೆಲ್ಲ ತುಂಬಿ ತುಳುಕಾಡುತ್ತಿತ್ತು. ಮುಂದೆ ತನ್ನಲ್ಲಿ ಜನಿಸುವವನು ಶಿವಾಂಶಸಂಭೂತನಾಗುವನು ಎಂದು ಭಾವಿಸಿದಳು. ಅಂತಹ ಮಹದ್\-‍ವ್ಯಕ್ತಿಯನ್ನು ಜಗತ್ತಿಗೆ ಧಾರೆ ಎರೆಯಬೇಕಾದರೆ ತಪಸ್ಸಿನಿಂದ ಪೂತವಾಗಿರಬೇಕು ಆ ವ್ಯಕ್ತಿಯ ಮನಸ್ಸು. ಮೊದಲೇ ಆಧ್ಯಾತ್ಮಿಕ ಪ್ರವೃತ್ತಿಯ ಜೀವನ ಭುವನೇಶ್ವರೀದೇವಿಯದು. ಮೇಲಿನ ಕನಸಾದ ಮೇಲಂತೂ ಹೆಚ್ಚು ಅಂತರ್ಮುಖವಾದಳು. ಮಗು ಧರೆಗೆ ಇಳಿದುಬರುವ ದಿನ ಸಮೀಪಿಸಿತು.

ಕ್ರಿ.ಶ. ೧೮೬೩ನೇ ಇಸವಿ ಜನವರಿ ೧೨ನೇ ತಾರೀಖು ಸೋಮವಾರ ಸೂರ‍್ಯೋ\-ದಯಕ್ಕೆ ಆರು ನಿಮಿಷಗಳಿವೆ. ಆ ಸಮಯದಲ್ಲಿ ಭುವನೇಶ್ವರೀದೇವಿಗೆ ಗಂಡುಮಗು ಜನಿಸಿತು. ಆಗ ಚಂದ್ರ ಕನ್ಯಾರಾಶಿಯಲ್ಲಿದ್ದ. ಗುರು ಏಕಾದಶ ಸ್ಥಾನದಲ್ಲಿದ್ದ. ಶನಿ ದಶಮಸ್ಥಾನದಲ್ಲಿದ್ದ. ಅದು ಪುಷ್ಯಮಾಸದ ಸಪ್ತಮಿ ದಿನ. ಅವತ್ತು ಮಕರ ಸಂಕ್ರಾಂತಿಯ ಪವಿತ್ರ ದಿನ ಬೇರೆ. ಸಹಸ್ರಾರು ಜನ ಗಂಗಾನದಿಯ ಸ್ನಾನಕ್ಕೆ ಹೋಗಿಬರುತ್ತಿದ್ದರು, ಪ್ರಾರ್ಥನೆ ಸಲ್ಲಿಸಿ ತರ್ಪಣಗಳನ್ನು ಕೊಡುತ್ತಿದ್ದರು. ವಿಶ್ವನಾಥದತ್ತನ ಮನೆ ಮೊದಲನೆ ಗಂಡುಶಿಶುವಿನ ಜನನದ ಶುಭವಾರ್ತೆಯಿಂದ ತುಂಬಿ ತುಳುಕಾಡಿತು. ನಾಮಕರಣದ ಸಮಯದಲ್ಲಿ ಮಗುವಿಗೆ ವಿಶ್ವನಾಥದತ್ತ, ತನ್ನ ತಂದೆಯಾದ ದುರ್ಗಾಚರಣನ ಹೆಸರನ್ನು ಕೊಡಬೇಕೆಂದು ಇದ್ದ. ಆದರೆ ತನ್ನ ಮನೆದೇವರಾದ ವೀರೇಶ್ವರನ ಅನುಗ್ರಹದಿಂದ ಜನಿಸಿದುದರಿಂದ ಮಗುವಿಗೆ ವೀರೇಶ್ವರನೆಂದು ರಾಶಿನಕ್ಷತ್ರದ ಹೆಸರಿಟ್ಟರು. ಮಗುವನ್ನು ತಾಯಿ ವೀರೇಶ್ವರ ಎಂದು ಕರೆಯದೆ ಬಿಲೆ ಎಂದು ಮುದ್ದಾಗಿ ಕರೆಯುತ್ತ ಇದ್ದಳು. ಮಗು ಬೆಳೆಯುತ್ತ ತನ್ನ ನೆರೆಹೊರೆಯವರಲ್ಲಿ ನರೇಂದ್ರ ಎಂದು ಪ್ರಖ್ಯಾತನಾಗುತ್ತ ಬಂದನು.

\vskip 2pt

ಮಗುವಾದಾಗಿಲಿನಿಂದಲೂ ನರೇಂದ್ರನ ಒಂದು ಸ್ವಭಾವ ಯಾರು ಕರೆದರೆ ಅವರ ಹತ್ತಿರ ನಾಚಿಕೆ ಇಲ್ಲದೇ ಹೋಗುತ್ತಿದ್ದುದು. ಸ್ವಲ್ಪ ದೊಡ್ಡ ವಯಸ್ಸಾದ ಹುಡುಗನಾದ ಮೇಲೂ ನೆರೆಹೊರೆಯ ಗಂಡು ಹೆಣ್ಣು ಹುಡುಗರನ್ನೆಲ್ಲ ಪರಿಚಯ ಮಾಡಿಕೊಂಡು ಅವರ ಹೆಸರಿನ ಜೊತೆಗೆ ಅಣ್ಣ ತಮ್ಮ ಚಿಕ್ಕಪ್ಪ ಮಾವ ಎಂಬುದನ್ನು ಸೇರಿಸುತ್ತಿದ್ದನು. ನೆರೆಹೊರೆಯ ಹುಡುಗಿಯರನ್ನು ಅವರ ವಯಸ್ಸಿಗೆ ತಕ್ಕಂತೆ, ಅಕ್ಕ, ತಂಗಿ, ಚಿಕ್ಕಮ್ಮ ಎಂದು ಕರೆಯುತ್ತಿದ್ದನು. ಈ ಹೆಸರನ್ನು ಕೇಳಿದಾಗ ಕೇಳಿದವರಿಗೆ ಆನಂದವಾಗುತ್ತಿತ್ತು. ಇದೊಂದು ಸಣ್ಣ ಅಭ್ಯಾಸ ಇರಬಹುದು. ಇದರಲ್ಲಿ ಆ ಮಗು ಹೇಗೆ ಬೆಳೆಯುತ್ತಿತ್ತು ಎಂಬುದನ್ನು ನೋಡುವೆವು. ಮನೆಯಲ್ಲಿ ಅವರ ಬೇರು ಬಿಟ್ಟಿದ್ದರೂ ನೆರೆಹೊರೆಯ ಮನೆಯ ಮೇಲೆಲ್ಲ ಆ ಮಗುವಿನ ವಿಶ್ವಾಸದ ರೆಂಬೆ ಕೊಂಬೆಗಳು ಹರಡಿದ್ದವು. ನರೇಂದ್ರನಂತಹ ವ್ಯಕ್ತಿಗಳು ಹುಟ್ಟುವುದು ಹೆತ್ತ ತಂದೆ ತಾಯಿಗಳಿಗೆ ಸಂತೋಷವನ್ನು ಕೊಡುವುದಕ್ಕೆ ಮಾತ್ರವಲ್ಲ, ಸುತುಮುತ್ತ ಇರುವವರಿಗೆಲ್ಲ ಸಂತೋಷವನ್ನು ಕೊಡುವುದಕ್ಕೆ. ಅವರು ಎಲ್ಲರ ವಿಶ್ವಾಸಕ್ಕೂ ಪಾತ್ರರಾಗುವಂತೆ ಬೆಳೆಯುತ್ತಾರೆ.

ನೆರೆಹೊರೆಯ ಹುಡುಗರು ಹೇಗೆ ನರೇಂದ್ರನ ಜೀವನದಲ್ಲಿ ಹಾಸುಹೊಕ್ಕಾಗಿ ಇದ್ದರೊ ಹಾಗೆಯೇ ಪಶುಪಕ್ಷಿ ಪ್ರಾಣಿಗಳೂ ಅವನ ಜೀವನದ ಸಂಗಾತಿಗಳು. ಮನೆಯ ಹಸು ಮತ್ತು ಕರುಗಳನ್ನು ಕಂಡರೆ ಅವನಿಗೆ ಪರಮಪ್ರೇಮ. ಜೊತೆಗೆ ಆಡುವುದಕ್ಕೆ ಒಂದು ಮಂಗ, ಕುರಿಯ ಮರಿಗಳು ಬೇರೆ ಇದ್ದವು. ಒಂದು ಮೊಲವನ್ನು ಕೂಡ ಸಾಕಿದ್ದ. ನವಿಲು ಪಾರಿವಾಳಗಳನ್ನು ಬೇರೆ ಕೂಡಿಕೊಂಡು ಆಡುತ್ತಿದ್ದ.

ನರೇಂದ್ರನಿಗೆ ಎಲ್ಲರಿಗಿಂತ ಹೆಚ್ಚಾಗಿ ತನ್ನ ಅಕ್ಕಂದಿರನ್ನು ಕಾಡುವುದೆಂದರೆ ಪರಮ\-ಪ್ರೀತಿ. ಅವರಿಗೆ ಗುದ್ದು ಕೊಟ್ಟು ಓಡಿಹೋಗುವನು. ಅವರು ಹಿಂದಿನಿಂದ ಅಟ್ಟಿಸಿಕೊಂಡು ಬಂದರೆ ಚರಂಡಿಯ ಒಳಗಿಳಿದು ಅದರ ಕೊಚ್ಚೆಯನ್ನು ಕೈಯಲ್ಲಿ ಹಿಡಿದು “ಬನ್ನಿ ಈಗ ನನ್ನನ್ನು ಮುಟ್ಟಿ” ಎಂದು ಹೇಳುತ್ತಿದ್ದ. ಪಾಪ ಹುಡುಗಿಯರು ಮೈಲಿಗೆಯಾದ ಹುಡುಗನನ್ನು ಹೇಗೆ ಮುಟ್ಟುವರು! ಅವನನ್ನು ಬೈದು ಹಿಂದಿರುಗುತ್ತಿದ್ದರು.

ಮೊದಲಿನಿಂದಲೂ ನರೇಂದ್ರನದು ತಂಟೆಯ ಸ್ವಭಾವ. ಅಂಜಿಕೆ ಬೆದರಿಕೆಗಳಿಗೆ ತಗ್ಗುವವನೇ ಅಲ್ಲ. ಕೆಲವು ವೇಳೆ ಅವನ ತುಂಟತನ ಮೀರಿದಾಗ ತಾಯಿ ಅವನ ತಲೆಯ ಮೇಲೆ ನೀರನ್ನು ಸುರಿದು ‘ಶಿವ, ಶಿವ’ಎನ್ನುತ್ತಿದ್ದಳು. “ನೋಡು ಬಿಲೆ, ನೀನು ಹೀಗೆಲ್ಲ ತಂಟೆಯನ್ನು ಮಾಡಿದರೆ ಶಿವ ನಿನ್ನನ್ನು ಆಮೇಲೆ ಕೈಲಾಸಕ್ಕೆ ಸೇರಿಸುವುದೇ ಇಲ್ಲ” ಎಂದು ಕೋಪಿಸಿಕೊಂಡು ಹೇಳುತ್ತಿದ್ದಳು. ನರೇಂದ್ರ ಶಿವನಾಮ ಮತ್ತು ಕೈಲಾಸದ ಹೆಸರನ್ನು ಕೇಳಿದೊಡನೆ ಪುಂಗಿನಾದವನ್ನು ಕೇಳಿದ ಹಾವಿನಂತೆ ಸಮ್ಮೋಹನಾಸ್ತ್ರಕ್ಕೆ ತುತ್ತಾಗಿ ತೆಪ್ಪಗಾಗುತ್ತಿದ್ದ. ತಾಯಿ ಇವನ ತುಂಟತನವನ್ನು ಸಹಿಸಲಾರದೆ ಇರುವಾಗ, “ಶಿವನನ್ನು ಒಂದು ಮಗುವನ್ನು ಕರುಣಿಸು ಎಂದು ಕೇಳಿಕೊಂಡರೆ ಅವನು ತಾನೆ ಬರುವುದರ ಬದಲು ತನ್ನ ಭುತಗಳಲ್ಲಿ ಒಂದನ್ನು ಕಳುಹಿಸಿರುವನು” ಎನ್ನುತ್ತಿದ್ದಳು.

ನರೇಂದ್ರನ ಮನೆಯ ಕೋಚುಗಾಡಿ ಹೊಡೆಯುವವನು ನರೇಂದ್ರನ ಬಾಲ್ಯ ಸ್ನೇಹಿತರಲ್ಲಿ ಒಬ್ಬ. ನರೇಂದ್ರ ತನ್ನ ಮನಸ್ಸಿನಲ್ಲಿದ್ದುದನ್ನೆಲ್ಲ ಅವನಿಗೆ ಹೇಳಿಕೊಳ್ಳುತ್ತಿದ್ದ. ಅವನು ಠೀವಿಯಿಂದ ಪೋಷಾಕನ್ನು ತೊಟ್ಟು ಉದ್ದವಾದ ಚಾವಟಿ ಹಿಡಿದುಕೊಂಡು ಕುದುರೆಯನ್ನು ಹೊಡೆಯುವುದನ್ನು ನೋಡಿದಾಗಲಂತೂ ತಾನು ಒಬ್ಬ ಕೋಚ್‍ಮ್ಯಾನ್ ಆಗಬೇಕು ಎಂದು ಬಯಸಿದ್ದ.

ನರೇಂದ್ರನು ಬಾಲ್ಯದಿಂದಲೂ ತಾಯಿಯ ಬಾಯಿಯಿಂದ ಸೀತಾರಾಮರ ಜೀವನ ಚರಿತ್ರೆಯನ್ನೆಲ್ಲ ಕೇಳಿ ರಾಮನ ವ್ಯಕ್ತಿತ್ವದಿಂದ ಮುಗ್ಧನಾಗಿ ಹೋಗಿದ್ದ. ಒಂದು ದಿನ ಮನೆಯವರು ತಿಂಡಿ ತಿನ್ನುವುದಕ್ಕೆ ಕಾಸು ಕೊಟ್ಟಿದ್ದರೆ, ಅದರಿಂದ ಒಂದು ಸುಂದರವಾದ ಮಣ್ಣಿನಿಂದ ಮಾಡಿದ ಸೀತಾರಾಮರ ವಿಗ್ರಹವನ್ನು ತಂದು ಪೂಜಿಸಿದ. ತನ್ನ ಕೋಚ್‍ಮ್ಯಾನ್ ಸ್ನೇಹಿತನಿಗೆ ಈ ಸಂತೋಷದ ಸುದ್ದಿಯನ್ನು ಹೇಳಿದ. ಅವನು ನರೇಂದ್ರನನ್ನು ಸ್ವಲ್ಪ ಹಾಸ್ಯ ಮಾಡಬೇಕು ಎಂದು “ನೋಡು ನರೇನ್, ಮನುಷ್ಯನೇನೋ ಮದುವೆಯಾಗುತ್ತಾನೆ. ಆದರೆ ರಾಮನೂ ಸೀತೆಯನ್ನು ಮದುವೆಯಾಗಿ ತಾಪತ್ರಯ ಪಡಬೇಕಾಯಿತಲ್ಲ. ಮನುಷ್ಯನಂತೇ ರಾಮನೂ ಆದ. ಅವನೆಂತಹ ದೇವರು” ಎಂದು ಹೇಳಿದ. ಪಾಪ ನರೇಂದ್ರನಿಗೆ ಇದು ಒಂದು ಸಿಡಿಲಿನಂತೆ ತಾಕಿತು. ಅವನು ಆಲೋಚನೆ ಮಾಡಿದ. “ಮದುವೆ ಕೆಟ್ಟದ್ದಾದರೆ ರಾಮನೇತಕ್ಕೆ ಮದುವೆಯಾಗಬೇಕಾಗಿತ್ತು? ನನಗೆ ಅಂತಹ ದೇವರು ಬೇಡ” ಎಂದು ಭಯಭಕ್ತಿಯಿಂದ ಪೂಜಿಸಿದ ಶ‍್ರೀ ಸೀತಾರಾಮರ ವಿಗ್ರಹವನ್ನು ಮನೆಯ ಮಹಡಿಯ ಮೇಲಿನಿಂದ ರೊಯ್ಯನೆ ಆಚೆಗೆ ಬಿಸಾಡಿದ. ಅದು ಚೂರು ಚೂರಾಗಿ ಕೆಳಗೆ ಬಿತ್ತು. ಆದರೆ ದೇವರಿಗೆ ಮೀಸಲಾಗಿದ್ದ ನರೇಂದ್ರನ ಹೃದಯ ಬಹಳ ಕಾಲ ಖಾಲಿಯಾಗಿರಲಿಲ್ಲ. ಸ್ವಲ್ಪ ಹೊತ್ತಿನಲ್ಲಿಯೇ ಪೇಟೆಗೆ ಹೋಗಿ ಧ್ಯಾನದಲ್ಲಿ ಕುಳಿತ ಶಿವನ ವಿಗ್ರಹವನ್ನು ಕೊಂಡುಕೊಂಡು ಬಂದ. ಅದನ್ನು ಪೂಜೆ ಮಾಡಿ ಅದರ ಮುಂದೆ ಒಂದು ದಿನ ಕೋಣೆಯೊಂದರಲ್ಲಿ ಬಾಗಿಲನ್ನು ಹಾಕಿಕೊಂಡು ಧ್ಯಾನಕ್ಕೆ ಕುಳಿತ. ಬಾಹ್ಯ ಪ್ರಪಂಚದ ಪ್ರಜ್ಞೆಯೇ ಮರೆತುಹೋಯಿತು. ಮನೆಯಲ್ಲಿರುವವರು ಊಟದ ಹೊತ್ತಿನವರೆಗೆ ಕಾದರೂ ಎಲ್ಲಿಯೂ ನರೇಂದ್ರ ಪತ್ತೆಯಿಲ್ಲ್ಲ. ಮನೆಯ ಕೋಣೆಯನ್ನೆಲ್ಲ ಒಂದೊಂದಾಗಿ ನೋಡಿಕೊಂಡು ಕೊನೆಗೆ ನರೇಂದ್ರ ಇದ್ದ ಕೊಠಡಿಗೆ ಬಂದರು. ಒಳಗಿನಿಂದ ಬಾಗಿಲು ಹಾಕಿಕೊಂಡಿತ್ತು. ಎಷ್ಟು ಕಿರುಚಿದರೂ ಬಾಗಿಲು ತೆಗೆಯಲಿಲ್ಲ. ಒಳಗೆ ನರೇಂದ್ರನಿಗೆ ಏನೋ ಆಗಿಹೋಗೆದೆಯೊ ಎಂದು ಬಲಾತ್ಕಾರವಾಗಿ ಬಾಗಿಲನ್ನು ತೆರೆದು ನೋಡುತ್ತಾರೆ, ಧ್ಯಾನಮಗ್ನ ಶಿವನ ವಿಗ್ರಹದ ಮುಂದೆ ನರೇಂದ್ರನೂ ಧ್ಯಾನದಲ್ಲಿ ತನ್ಮಯನಾಗಿರುವನು. ಇವರೆಲ್ಲ ಹತ್ತಿರ ಬಂದು ನಿಂತು ಗಲಾಟೆ ಮಾಡಿದ ಮೇಲೆಯೇ ಅವನಿಗೆ ಎಚ್ಚರವಾಗಿದ್ದು.

ಮತ್ತೊಂದು ದಿನ ಅವನು ಜೊತೆಯ ಹುಡುಗರನ್ನೆಲ್ಲ ಕೂಡಿಕೊಂಡು ಆಟಕ್ಕಾಗಿ ದೇವರ ಪೂಜೆ ಭಜನೆ ಮಾಡಿ ಧ್ಯಾನಕ್ಕೆ ಕುಳಿತನು. ಕುಳಿತ ಹಲವು ಹುಡುಗರು ಕಣ್ಣನ್ನು ಮುಚ್ಚಿದರೂ ಅವರ ಪ್ರಜ್ಞೆಯೆಲ್ಲ ಸುತ್ತಮುತ್ತಲ ವಸ್ತುವಿನ ಕಡೆ. ನರೇಂದ್ರನು ಆಟಕ್ಕಾಗಿ ಧ್ಯಾನಕ್ಕೆ ಕುಳಿತುಕೊಂಡರೂ ಅದು ನಿಜವಾದ ಧ್ಯಾನವಾಗಿ ಸುತ್ತಮುತ್ತಲಿರುವುದನ್ನೆಲ್ಲಾ ಸಂಪೂರ್ಣ ಮರೆತ. ಆ ಸಮಯದಲ್ಲಿ ಒಂದು ಸರ್ಪ ಮೂಲೆಯೊಂದರಿಂದ ಹೆಡೆ ಎತ್ತಿಕೊಂಡು ಬರುತ್ತಿತ್ತು. ಕಣ್ಣು ಬಿಟ್ಟುಕೊಂಡಿದ್ದ ಇತರ ಹುಡುಗರು ‘ಹಾವು’ ಎಂದು ಕಿರುಚಿಕೊಂಡೊಡನೆಯೇ ಎಲ್ಲರೂ ಪರಾರಿಯಾದರು. ಅವರಲ್ಲಿ ಒಬ್ಬ ಧೈರ್ಯಶಾಲಿ ಹುಡುಗ ಬಂದು, ನರೇಂದ್ರನನ್ನು ಅಲ್ಲಾಡಿಸಿ, “ಹಾವೊಂದು ಬರುತ್ತಿದೆ ಎದ್ದು ಓಡು” ಎಂದು ಹೇಳಿ ತಾನೂ ಪರಾರಿಯಾದನು. ಆದರೆ ನರೇಂದ್ರನಾದರೋ ಸ್ವಲ್ಪವೂ ಚಲಿಸಲಿಲ್ಲ. ಹೆಡೆ ಎತ್ತಿದ ಹಾವು ಬಂತು. ಇವನ ಸುತ್ತ ಸಂಚರಿಸಿ ಹುಡುಗನಿಗೆ ಯಾವ ಅಪಾಯವನ್ನೂ ಮಾಡದೆ ಹೊರಟುಹೋಯಿತು. ಅದು ಹೋದ ಮೇಲೆ ಹುಡುಗರು ಬಂದು ನರೇಂದ್ರನಿಗೆ, “ಏತಕ್ಕೆ ನೀನು ಎದ್ದು ಓಡಲಿಲ್ಲ ಹಾವು ಬಂದಿದೆ ಎಂದರೂ?” ಎಂದು ಕೇಳಿದಾಗ, ನರೇಂದ್ರ “ಆ ಸಮಯದಲ್ಲಿ ಮನಸ್ಸು ಒಂದು ಅವರ್ಣನೀಯ ಆನಂದದಲ್ಲಿತ್ತು” ಎಂದನು. ಈ ಹುಡುಗನನ್ನು ಅನಂತರ ಇವನ ಗುರುಗಳಾದ ಶ‍್ರೀರಾಮಕೃಷ್ಣ ಪರಮಹಂಸರು ಕಂಡ ತತ್‍ಕ್ಷಣವೇ ಇವನ ಕಣ್ಣುಗಳನ್ನು ನೋಡಿ ಹೇಳಿದರು “ಇವನು ಧ್ಯಾನಸಿದ್ಧ” ಎಂದು. ನರೇಂದ್ರ ಅದ್ಭುತವಾದ ಮಾನಸಿಕ ಏಕಾಗ್ರತೆಯನ್ನು ಪಡೆದು ಬಂದಿದ್ದನು. ಸಂಕಲ್ಪಿಸಿದೊಡನೆಯೇ ಮನಸ್ಸು ಎಲ್ಲಿ ಅಂದರೆ ಅಲ್ಲಿ ಹೋಗಿ ಕುಳಿತುಕೊಳ್ಳುತ್ತಿತ್ತು. “ಮನಸ್ಸು ನನ್ನ ಕೈಯಲ್ಲಿ ಒಂದು ಜೇಡಿಮಣ್ಣಿನ ಮುದ್ದೆಯಂತೆ. ಅದಕ್ಕೆ ಏನು ಆಕಾರ ಕೊಟ್ಟರೆ ಅದನ್ನು ಅದು ತೆಗೆದುಕೊಳ್ಳಬೇಕು” ಎಂದು ನರೇಂದ್ರ ಹೇಳುತ್ತಿದ್ದ.

ನರೇಂದ್ರ ತಾಯಿಯೊಡನೆ ಪುರಾಣವನ್ನು ಕೇಳುವುದಕ್ಕೆ ಹೋಗುತ್ತಿದ್ದ. ಆಗ ಪಂಡಿತರು ರಾಮಾಯಣವನ್ನು ಹೇಳುತ್ತಿದ್ದರು. ಒಂದು ದಿನ ನರೇಂದ್ರ ಕೇಳುವುದಕ್ಕೆ ಹೋಗಿದ್ದಾಗ ಹನುಮಂತನ ಶೌರ‍್ಯ ಸ್ವಾಮಿಭಕ್ತಿ ಮುಂತಾದುವನ್ನೆಲ್ಲ ವರ್ಣಿಸುತ್ತಿದ್ದರು. ಅವನನ್ನು ಹಿಂದೂ ಪುರಾಣಗಳಲ್ಲಿ ಒಬ್ಬ ಚಿರಂಜೀವಿಯನ್ನಾಗಿ ಮಾಡಿರುವರು. ಹನುಮಂತನ ವರ್ಣನೆಯನ್ನು ಕೇಳಿ ನರೇಂದ್ರ ಮುಗ್ಧನಾಗಿ ಹೋದ. ಅಂತಹ ಸ್ವಾಮಿ ಭಕ್ತ ಚಿರಂಜೀವಿಯಾಗಿದ್ದರೆ ನೋಡಲು ಸಿಕ್ಕಬೇಕು. ಅವನು ಎಲ್ಲಿ ಸಿಕ್ಕುತ್ತಾನೆ ಎಂದು ಪುರಾಣ ಹೇಳುವವರನ್ನು ನರೇಂದ್ರ ಕೇಳಿದ. ಆ ಪುರಾಣ ಹೇಳುವವರು ಹುಡುಗನ ಹತ್ತಿರ ಸ್ವಲ್ಪ ತಮಾಷೆ ಮಾಡಬೇಕೆಂದು ಮನೆ ಹಿಂದಿನ ಬಾಳೆ ತೋಟದಲ್ಲಿ ಅರ್ಧ ರಾತ್ರಿ ಹೊತ್ತು ಕಾದರೆ ಅಲ್ಲಿಗೆ ಅವನು ಬರುತ್ತಾನೆ ಎಂದು ಹೇಳಿದರು. ನರೇಂದ್ರ ಮನೆಗೆ ಹೋದಮೇಲೆ ಅರ್ಧರಾತ್ರಿ ಸಮಯದಲ್ಲಿ ಬಾಳೆಯ ತೋಟದಲ್ಲಿ ಕಾದ, ಹನುಮಂತನ ದರ್ಶನಕ್ಕೆ. ಹನುಮಂತನೂ ಇಲ್ಲ ಯಾರೂ ಇಲ್ಲ. ಎಷ್ಟು ಕಾದರೂ ಏನೂ ಪ್ರಯೋಜನವಾಗಲಿಲ್ಲ. ಬೆಳಕು ಹರಿದ ಮೇಲೆ ನರೇಂದ್ರ ತಾಯಿಯನ್ನು, “ಪುರಾಣ ಹೇಳಿದವರು ಬಾಳೆ ತೋಟದಲ್ಲಿ ಅರ್ಧರಾತ್ರಿ ಹೊತ್ತಿನಲ್ಲಿ ಹನುಮಂತ ಬರುತ್ತಾನೆ ಎಂದು ಹೇಳಿದ್ದರು, ಬರಲೇ ಇಲ್ಲವಲ್ಲ” ಎಂದು ಕೇಳಿದ. ಅದಕ್ಕೆ ತಾಯಿ ತಾನೇ ಏತಕ್ಕೆ ಇದು ಸುಳ್ಳು ಎಂದು ಹೇಳಬೇಕು, ಆ ಮಗುವೇ ಆ ನಿರ್ಣಯಕ್ಕೆ ಬರಲಿ ಎಂದು, “ಹನುಮಂತನಿಗೆ ಒಂದೇ ದಿವಸ ಕಾದರೆ ಸಾಕೆ, ಇನ್ನೂ ಕೆಲವು ದಿನ ಕಾದು ನೋಡು” ಎಂದಳು. ನರೇಂದ್ರನು ಹಾಗೆಯೇ ಮಾಡಿ ಕೊನೆಗೆ ಇದೆಲ್ಲ ಸುಳ್ಳು ಎಂಬ ನಿರ್ಣಯಕ್ಕೆ ಬಂದನು. ಇಲ್ಲಿ ನರೇಂದ್ರನ ತಾಯಿ ಮಗನನ್ನು ಹೇಗೆ ಬೆಳೆಸುತ್ತಿದ್ದಳು ಎಂಬುದನ್ನು ಗಮನಿಸಬೇಕು. ಒಂದು ನಿರ್ಣಯವನ್ನು ನಾವೇ ಹೇಳಿಬಿಡುವುದಕ್ಕಿಂತ ಅದನ್ನು ವ್ಯಕ್ತಿ ತಾನೇ ಪಡೆಯಬೇಕು. ಪಡೆಯಬೇಕಾದರೆ ಪ್ರಯೋಗಮಾಡಿ ನೋಡಬೇಕು. ಅನಂತರ ಬರುವ ಜ್ಞಾನವೇ ಭದ್ರವಾಗಿ ನಿಲ್ಲುತ್ತದೆ.

ನರೇಂದ್ರನದು ಮೊದಲಿನಿಂದಲೂ ಯಾರು ಏನು ಹೇಳಿದರೆ ಅದನ್ನು ನಂಬಿಬಿಡುವ ಹುಡುಗನ ಸ್ವಭಾವವಲ್ಲ. ತಾಯಿ ಆ ಸ್ವಭಾವವನ್ನು ನಾಶಮಾಡುವ ಬದಲು ಅದು ಚೆನ್ನಾಗಿ ವಿಕಾಸವಾಗುವುದಕ್ಕೆ ಅವಕಾಶವನ್ನು ಒದಗಿಸಿಕೊಟ್ಟಳು. ನರೇಂದ್ರನ ತಂದೆ ವಿಶ್ವನಾಥದತ್ತ ಕಲ್ಕತ್ತೆಯಲ್ಲಿ ಪ್ರಮುಖನಾದ ವಕೀಲನಾಗಿದ್ದ. ಅವನ ಮನೆಗೆ ಹಲವು ಜಾತಿಯ ಕಕ್ಷಿಗಳು ಬರುತ್ತಿದ್ದರು. ಒಂದೊಂದು ಜಾತಿಯವರಿಗೂ\break ಸೇದುವುದಕ್ಕೆ ಒಂದೊಂದು ಗುಡಿಗುಡಿಯನ್ನು ಇಟ್ಟಿದ್ದರು. ಒಬ್ಬರು ಮತ್ತೊಬ್ಬರದರಲ್ಲಿ ಕುಡಿಯುತ್ತಿರಲಿಲ್ಲ. ಈ ವಿಚಿತ್ರ ದೃಶ್ಯವನ್ನು ನರೇಂದ್ರ ನೋಡಿದ. ಜನ ಏತಕ್ಕೆ ಹೀಗೆ ಮಾಡುತ್ತಾರೆ ಎಂದು ಯೋಚಿಸತೊಡಗಿದ. ಮನೆಯಲ್ಲಿದ್ದ ಹಿರಿಯರನ್ನು ಕೇಳಿದರೆ, ಅವರು ಒಬ್ಬ ಮತ್ತೊಬ್ಬನ ಹುಕ್ಕದಿಂದ ಸೇದಿದರೆ ಪಾಪ ಬರುತ್ತದೆ, ಏನೋ ಕೆಟ್ಟದ್ದು ಆಗುವುದು ಎಂದು ಅಂಜಿಕೆಯನ್ನು ಹುಟ್ಟಿಸಿದರು. ನರೇಂದ್ರ ಒಂದು ದಿನ ಕಕ್ಷಿಗಳು ಕುಳಿತುಕೊಳ್ಳುವ ಕೋಣೆಗೆ ಹೋದ. ಅವತ್ತು ಇನ್ನೂ ಯಾರೂ ಬಂದಿರಲಿಲ್ಲ. ಅವನು ಒಂದೊಂದು ಹುಕ್ಕದ ಹತ್ತಿರವೂ ಹೋಗಿ ಸ್ವಲ್ಪ ಹೊತ್ತು ಹುಕ್ಕವನ್ನು ಸೇದಿದ. ಸುತ್ತಲೂ ನೋಡಿದ. ಏನಾದರೂ ಅಪಾಯ ಸಂಭವಿಸಬಹುದೆ ಎಂದು, ಏನೂ ಆಗಲಿಲ್ಲ. ಯಾವ ಭೂತಪ್ರೇತವೂ ಬರಲಿಲ್ಲ. ಚಾವಣಿಯೂ ಕುಸಿದು ಬೀಳಲಿಲ್ಲ. ಇದೊಂದು ಹಿಂದಿನಿಂದ ಬಂದ ಆಚಾರ ಅಷ್ಟೆ. ಇದರಲ್ಲಿ ಇನ್ನು ಯಾವ ಪಾಪವೂ ಇಲ್ಲ, ಭೀತಿಯೂ ಇಲ್ಲ ಎಂದು ತಾನೆ ನಿರ್ಧಾರಕ್ಕೆ ಬಂದ.

ನರೇಂದ್ರನ ಓರಗೆ ಹುಡುಗರೆಲ್ಲ ಸೇರಿಕೊಂಡು ಪಕ್ಕದ ಮನೆಯ ಹಿಂದೆ ಇದ್ದ ಒಂದು ದೊಡ್ಡ ಮರದಲ್ಲಿ ಮರಕೋತಿ ಆಟ ಆಡುತ್ತಿದ್ದರು. ಒಳ್ಳೆ ಮಧ್ಯಾಹ್ನದ ಹೊತ್ತು. ಮನೆಯಲ್ಲಿದ್ದ ಮುದುಕನೊಬ್ಬನು ನಿದ್ರಿಸುತ್ತಿದ್ದ ಸಮಯ. ಈ ಹುಡುಗರ ಗಲಾಟೆಯಿಂದ ಅವನಿಗೆ ನಿದ್ರಾಭಂಗವಾಗುತ್ತಿತ್ತು. ಇವರನ್ನು ಆಚೆ ಕಳಿಸುವುದಕ್ಕೆ ಒಂದು ಉಪಾಯವನ್ನು ಕುರಿತು ಯೋಚಿಸಿ ಒಳಗಿನಿಂದ ಬಂದು ಹುಡುಗರಿಗೆ “ಈ ಮರದಲ್ಲಿ ಒಂದು ಬ್ರಹ್ಮರಾಕ್ಷಸ ಇದೆ. ಮಧ್ಯಾಹ್ನದ ಹೊತ್ತಿನಲ್ಲಿ ಯಾರಾದರೂ ಸಿಕ್ಕಿದರೆ ಅವರ ಕತ್ತನ್ನು ಅದು ಹಿಸುಕಿಬಿಡುತ್ತದೆ, ಜೋಕೆ!” ಎಂದು ಅಂಜಿಸಿದನು. ಬ್ರಹ್ಮರಾಕ್ಷಸ ಎಂಬ ಹೆಸರನ್ನು ಕೇಳಿದೊಡನೆಯೇ ಹುಡುಗರೆಲ್ಲ ಪರಾರಿಯಾದರು. ಮನೆಯ ವೃದ್ಧ ತನ್ನ ಉಪಾಯ ಚೆನ್ನಾಗಿ ಫಲಕಾರಿಯಾಯಿತು ಎಂದು ಸಂತೋಷದಿಂದ ಹೋಗಿ ಹಾಯಾಗಿ ಮಲಗಿಕೊಂಡ. ಸ್ವಲ್ಪ ಹೊತ್ತಿನ ಮೇಲೆ ಕಣ್ಣು ಬಿಟ್ಟು ಕಿಟಕಿಯ ಮೂಲಕ ಮರದ ಕಡೆಗೆ ನೋಡುತ್ತಾನೆ. ಒಬ್ಬ ಹುಡುಗ ಅಲ್ಲಿ ಜೋತಾಡುತ್ತಿರುವುದು ಕಂಡು ಬಂತು. ಮಲಗಿದ್ದವನು ಎದ್ದು “ಬ್ರಹ್ಮರಾಕ್ಷಸ ಇದೆ ಅಂತ ನಾನು ಹೇಳಿದರೂ ಪುನಃ ಬಂದೆಯಲ್ಲ” ಎಂದು ಅವನನ್ನು ಅಂಜಿಸಿದ. ಅದಕ್ಕೆ ಜೋಲುತ್ತಿದ್ದ ಹುಡುಗ ಹೇಳಿದ: “ಬ್ರಹ್ಮರಾಕ್ಷಸನೂ ಇಲ್ಲ ಯಾರೂ ಇಲ್ಲ. ನೋಡೋಣ, ಇದ್ದರೆ ಬಂದು ನನ್ನ ಕತ್ತನ್ನು ಹಿಸುಕಲಿ!” ಹೀಗೆ ಹೇಳಿದವನೇ ನರೇಂದ್ರ.

ನರೇಂದ್ರನಿಗೆ ಅವನ ತಂದೆಯ ಬಳಿಗೆ ಬರುತ್ತಿದ್ದವರಲ್ಲಿ ಹಲವು ಸ್ನೇಹಿತರಿದ್ದರು. ಅವರಲ್ಲಿ ಒಬ್ಬ ಮಹಮ್ಮದೀಯ ಕಕ್ಷಿಯವನನ್ನು ಮಾತ್ರ ನರೇಂದ್ರನು ಎಂದೂ ಮರೆತಿರಲಿಲ್ಲ. ಆ ಕಕ್ಷಿ ಬರುವಾಗಲೆಲ್ಲ ಏನಾದರೂ ನರೇಂದ್ರನಿಗೆ ತಿಂಡಿ ತಂದುಕೊಡುತ್ತಿದ್ದ. ಅನಂತರ ದೂರದ ತಮ್ಮ ಅರಬ್ಬಿದೇಶ, ಅಲ್ಲಿರುವ ಮೆಕ್ಕ ಮದೀನ, ಅಲ್ಲಿರುವ ಮರಳು ಕಾಡುಗಳು, ಅಲ್ಲಿಗೆ ಹೋಗಬೇಕಾದರೆ ಹೇಗೆ ತಿಂಗಳಗಟ್ಲೆ ಒಂಟೆ ಮೇಲೆ ಸವಾರಿಮಾಡಿಕೊಂಡು ಹೋಗಬೇಕು ಎಂಬುದನ್ನೆಲ್ಲ ವಿವರಿಸುತ್ತಿದ್ದ. ನರೇಂದ್ರ ಈ ಪ್ರಯಾಣದ\break ಕಥೆಗಳನ್ನು ಉತ್ಸಾಹದಿಂದ ಕೇಳುತ್ತಿದ್ದ. “ಇನ್ನೊಂದು ಸಲ ನೀನು ಹೋಗುವಾಗ ನನ್ನನ್ನು ಕರೆದುಕೊಂಡು ಹೋಗಬೇಕು” ಎಂದು ಹೇಳುತ್ತಿದ್ದ.

ನರೇಂದ್ರನಿಗೆ ಮನೆಯಲ್ಲಿ ಅವನ ತಾಯಿಯೇ ಮೊದಲು ಅಕ್ಷರಗಳನ್ನು ಹೇಳಿಕೊಟ್ಟಳು. ಇಂಗ್ಲೀಷಿನ ಅಕ್ಷರಗಳನ್ನೆಲ್ಲ ಆಕೆಯೇ ಕಲಿಸಿದಳು; ಪ್ರಥಮ ಪಾಠಗಳನ್ನು ಮಗುವಿಗೆ ಆಕೆಯೇ ಮಾಡಿದಳು. ಒಂದು ದೃಷ್ಟಿಯಿಂದ ನರೇಂದ್ರನಾಥನ ವಿದ್ಯಾಭ್ಯಾಸಕ್ಕೆ ತಳಪಾಯವನ್ನು ಹಾಕಿದವಳು ನರೇಂದ್ರನ ತಾಯಿ ಎಂತಲೇ ಹೇಳಬಹುದು. ಕೆಲವು ಕಾಲದ ಮೇಲೆ ಅವನನ್ನು ಹತ್ತಿರ ಇದ್ದ ಪಾಠಶಾಲೆಗೆ ಕಳಿಸಿದರು. ಅಲ್ಲಿಗೆ ಬರುತ್ತಿದ್ದ ಹುಡುಗರೊಂದಿಗೆಲ್ಲ ನರೇಂದ್ರ ಬೆರೆಯತೊಡಗಿದ. ಶಾಲೆಯಲ್ಲಿ ಕಲಿಯುತ್ತಿದ್ದ ಪಾಠಕ್ಕಿಂತ ಹೆಚ್ಚಾಗಿ ಆ ಹುಡುಗರು ಉಪಯೋಗಿಸುತ್ತಿದ್ದ ಅಯೋಗ್ಯ ಪದಗಳನ್ನೆಲ್ಲ ಹುಡುಗ ಕಲಿತ. ಮನೆಯವರಿಗೆ ಜುಗುಪ್ಸೆ ಬಂತು. ಆಗ ಅವನನ್ನು ಆ ಶಾಲೆಯಿಂದ ಬಿಡಿಸಿದರು. ಇವರ ಮನೆಯಲ್ಲಿಯೇ ಒಬ್ಬ ಉಪಾಧ್ಯಾಯನನ್ನು ಗೊತ್ತುಮಾಡಿ ಪಾಠ ಹೇಳಿಸಲು ಪ್ರಯತ್ನ ಮಾಡಿದರು. ನರೇಂದ್ರನ ಬುದ್ಧಿ ಚುರುಕು. ಒಂದು ಸಲ ಕೇಳಿದರೆ ಸಾಕು ಎಲ್ಲವನ್ನೂ ಗ್ರಹಿಸಿಬಿಡುತ್ತಿದ್ದ. ಆದರೆ ಅವನು ಏಕಾಗ್ರವಾಗಿ ಉಪಾಧ್ಯಾಯರು ಹೇಳುತ್ತಿದ್ದುದನ್ನು ಕೇಳುತ್ತಿದ್ದಾಗ ಕಣ್ಣನ್ನು ಮುಚ್ಚಿ ಕುಳಿತಿರುತ್ತಿದ್ದ. ಆ ಉಪಾಧ್ಯಾಯನಿಗೆ ಮೊದಲು ನರೇಂದ್ರನ ಸ್ವಭಾವ ಗೊತ್ತಾಗಲಿಲ್ಲ. ಈ ಹುಡುಗ ನಿದ್ದೆ ಮಾಡುತ್ತಿರಬಹುದೆಂದು ಭಾವಿಸಿದ. ಒಂದು ದಿನ ಪಾಠ ಹೇಳಿದಮೇಲೆ ನಾನು ಏನು ಹೇಳಿದೆ ಅದನ್ನು ಹೇಳು ಎಂದು ಕೇಳಿದ. ನರೇಂದ್ರ ಉಪಾಧ್ಯಾಯ ಹೇಳಿದ ಎಲ್ಲವನ್ನೂ ಚಾರುಚೂರು ತಪ್ಪದೆ ಹೇಳಿಬಿಟ್ಟ. ಅನಂತರ ಆ ಉಪಾಧ್ಯಾಯನಿಗೆ ಗೊತ್ತಾಯಿತು ನರೇಂದ್ರ ಎಂತಹವನು ಎಂಬುದು. ಅಂದಿನಿಂದ ಈ ಹುಡುಗನ ಮೇಲೆ ಒಂದು ವಿಶ್ವಾಸ ಮತ್ತು ಗೌರವ ಮೂಡಿತು. ನರೇಂದ್ರನು ತನಗೆ ಏಳು ವರ್ಷ ವಯಸ್ಸಾಗುವ ಹೊತ್ತಿಗೆ ಮುಗ್ಧಬೋಧವೆಂಬ ಸಂಸ್ಕೃತ ವ್ಯಾಕರಣ ಪುಸ್ತಕವನ್ನೆಲ್ಲ ಕಂಠಪಾಠ ಮಾಡಿದ್ದ. ರಾಮಾಯಣ ಮಹಾಭಾರತಗಳಿಗೆ ಸಂಬಂಧಿಸಿದ ಹಲವು ಹಾಡುಗಳು ಹುಡುಗನಿಗೆ ಬರುತ್ತಿದ್ದವು.

ಮನೆಗೆ ಭಿಕ್ಷೆ ಬೇಡುವುದಕ್ಕೆಂದು ಬಂದ ಭಿಕ್ಷುಕರು ರಾಮಾಯಣದ ಹಾಡುಗಳನ್ನು ಹಾಡುತ್ತಿದ್ದರು. ಮಧ್ಯೆ ಅವರು ತಪ್ಪಿದಾಗ ಈ ಹುಡುಗ ಅದನ್ನು ತಿದ್ದುತ್ತಿದ್ದನು. ಮೊದಲಿನಿಂದಲೂ ನರೇಂದ್ರನಿಗೆ ಸಾಧು ಸಂನ್ಯಾಸಿಗಳನ್ನು ಕಂಡರೆ ಪ್ರೀತಿ. ತನ್ನ ಹತ್ತಿರ ಏನು ಇದ್ದರೂ ಅವರು ಕೇಳಿದರೆ ಅದನ್ನು ಕೊಟ್ಟುಬಿಡುತ್ತಿದ್ದನು. ಒಂದು ದಿನ ಮೈಮೇಲೆ ಹೊಸದೊಂದು ಉತ್ತರೀಯವನ್ನು ಹೊದ್ದುಕೊಂಡಿದ್ದನು. ಭಿಕ್ಷುಕನೊಬ್ಬ ಬಂದು ತನಗೆ ಬಟ್ಟೆ ಇಲ್ಲವೆಂದು ಕೇಳಿಕೊಂಡನು. ಹುಡುಗ ಮರುಮಾತನಾಡದೆ ಮೈಮೇಲೆ ಹೊದ್ದ ಉತ್ತರೀಯವನ್ನು ಅವನಿಗೆ ಕೊಟ್ಟುಬಿಟ್ಟನು. ಆ ಭಿಕ್ಷುಕನಾದರೊ ಅದನ್ನು ತಲೆಗೆ ಒಂದು ರುಮಾಲಿನಂತೆ ಸುತ್ತಿಕೊಂಡು ಹೊರಟುಹೋದ. ಮನೆಯವರಿಗೆ ಇದು ಗೊತ್ತಾದಮೇಲೆ ಯಾರಾದರೂ ಭಿಕ್ಷುಕರು ಬಂದರೆ ಹುಡುಗನನ್ನು ಒಳಗೆ ಒಂದು ರೂಮಿನಲ್ಲಿ ಕೂಡಿಹಾಕುತ್ತಿದ್ದರು. ಅಲ್ಲಿಯೂ ನರೇಂದ್ರ ಏನು ಸುಮ್ಮನಿರುವವನಲ್ಲ. ಕಿಟಕಿಯ ಮೂಲಕ ಕೋಣೆಯಲ್ಲಿ ಏನು ಇತ್ತೊ ಅದನ್ನು ಎಸೆದುಬಿಡುತ್ತಿದ್ದ. ನರೇಂದ್ರನಿಗೆ ಯಾವಾಗಲೂ ಸಾಧು ಜೀವನದ ಮೇಲೆ ಪ್ರೀತಿ. ತನ್ನ ಸಂಗಡಿಗರಿಗೆ ಅವನು ತನ್ನ ಕೈಯಲ್ಲಿರುವ ಒಂದು ರೇಖೆಯನ್ನು ತೋರಿ ಇದು ನಾನು ಸಂನ್ಯಾಸಿಯಾಗುತ್ತೇನೆ ಎಂದು ತೋರುತ್ತದೆ ಎಂದು ಹೇಳಿ ಸಂತೋಷಪಡುತ್ತಿದ್ದ. ಆ ಸಣ್ಣ ವಯಸ್ಸಿನಲ್ಲೆ ತ್ಯಾಗ ಜೀವನದ ಮೇಲೆ ಅಂತಹ ಆಕರ್ಷಣೆ!

ನರೇಂದ್ರ ತನ್ನ ಓರಗೆಯವರೊಂದಿಗೆ ಹಲವು ಆಟಗಳನ್ನು ಆಡುತ್ತಿದ್ದ. ಅವುಗಳಲ್ಲಿ, ರಾಜ ಮತ್ತು ಅವನ ಓಲಗದವರ ಆಟ ಅವನಿಗೆ ತುಂಬ ಪ್ರಿಯ. ಮನೆಯ ಮೆಟ್ಟಲಿನಮೇಲೆ ಕುಳಿತುಕೊಂಡು ತಾನು ರಾಜ ನರೇಂದ್ರ ಎಂದು ಹೇಳಿಕೊಳ್ಳುತ್ತಿದ್ದ. ಅದಕ್ಕಿಂತ ಒಂದು ಮೆಟ್ಟಲು ಕೆಳಗಡೆ ಮಂತ್ರಿ, ಅದರ ಕೆಳಗೆ ಸೇನಾನಿ ಮುಂತಾದವರೆಲ್ಲ ಇರುತ್ತಿದ್ದರು. ತಪ್ಪಿತಸ್ಥರನ್ನು ಕರೆದುಕೊಂಡು ಬಂದಾಗ, ನರೇಂದ್ರ ಅವರಿಗೆ ಶಿಕ್ಷೆ ವಿಧಿಸುತ್ತಿದ್ದ. ಆಟದಲ್ಲಿ ಯಾರಾದರೂ ಗಲಾಟೆ ಮಾಡಿದರೆ ತತ್‍ಕ್ಷಣವೇ ನರೇಂದ್ರನ ಜ್ವಲಿಸುವ ಕಣ್ಣುಗಳು ಅವರ ಕಡೆ ದುರುಗುಟ್ಟಿಕೊಂಡು ನೋಡುವುವು. ಆಗ ಅವರೆಲ್ಲ ತೆಪ್ಪಗಾಗುತ್ತಿದ್ದರು.

ನರೇಂದ್ರ ನಿದ್ರೆಗೆ ಹೋಗುವಾಗ ಒಂದು ಬೆಳಕಿನ ಚುಕ್ಕೆಯನ್ನು ನೋಡುತ್ತಿದ್ದನು. ಅದು ಅವನ ಹತ್ತಿರ ಹತ್ತಿರ ಬಂದು ಬಂದಂತೆಲ್ಲ ದೊಡ್ಡದಾಗಿ ಕ್ರಮೇಣ ಭ್ರೂಮಧ್ಯ ಪ್ರವೇಶಿಸಿ ಇವನನ್ನು ಕಾಂತಿಯಿಂದ ಮುಚ್ಚಿಬಿಡುವಂತೇ ಭಾಸವಾಗುತ್ತಿತ್ತು. ಅನಂತರ ತತ್‍ಕ್ಷಣವೇ ನಿದ್ರೆ ಬರುತ್ತಿತ್ತು. ನರೇಂದ್ರ ಇದು ಬಹುಶಃ ಪ್ರತಿಯೊಬ್ಬ ಹುಡುಗರಿಗೂ ಆಗುವ ಅನುಭವ ಇರಬೇಕು ಎಂದು ಭಾವಿಸಿದ್ದ. ಒಂದು ದಿನ ತನ್ನ ಸ್ನೇಹಿತನೊಬ್ಬನನ್ನು “ನೀನು ನಿದ್ರೆಗೆ ಹೋಗುವಾಗ ಒಂದು ಜ್ಯೋತಿಯ ಗೋಳವನ್ನು ನೋಡುತ್ತೀಯಾ?”ಎಂದು ಕೇಳಿದ. ಅದಕ್ಕೆ ಆ ಹುಡುಗ “ಅದೇನು ನನಗೆ ಕಾಣುವುದಿಲ್ಲ” ಎಂದು ಹೇಳಿದ. ಇಂತಹ ಅನುಭವ ಎಲ್ಲರ ಪಾಲಿಗೂ ಬರುವುದಲ್ಲ. ಎಲ್ಲೋ ಕೆಲವು ನಿತ್ಯಸಿದ್ಧರಿಗೆ ಮಾತ್ರ ಆಗುವ ಅನುಭವಗಳೆಂದು ಆಗ ನರೇಂದ್ರನಿಗೆ ಗೊತ್ತಾಗಿರಲಿಲ್ಲ. ಈ ನರೇಂದ್ರನೇ ಮುಂದೆ ಶ‍್ರೀರಾಮಕೃಷ್ಣ ಪರಮಹಂಸರ ಹತ್ತಿರ ಹೋಗಿ ನಿಂತುಕೊಂಡಾಗ, ಅವರು ಇವನ ಕಣ್ಣುಗಳನ್ನು ನೋಡಿ “ನೀನು ನಿದ್ದೆಗೆ ಹೊಗುವಾಗ ಒಂದು ಜ್ಯೋತಿಯ ಗೋಳವನ್ನು ನೋಡುತ್ತೀಯ?” ಎಂದು ಕೇಳಿದರು. ವಜ್ರಪಡಿ ವ್ಯಾಪಾರಿಗೆ ಮಾತ್ರ ವಜ್ರದ ಚಿಹ್ನೆ ಗೊತ್ತಾಗುವುದು. ಯೋಗ ಜೀವನದಲ್ಲಿ ಬಹಳ ಮುಂದುವರಿದು ಹೋದವನಿಗೆ ಇಂತಹ ಅನುಭವ ಆಗುತ್ತದೆ ಎಂದು ಶ‍್ರೀರಾಮಕೃಷ್ಣರು ಹೇಳಿದರು.

