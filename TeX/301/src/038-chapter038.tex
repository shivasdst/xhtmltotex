
\chapter{ಉತ್ತರ ಹಿಂದೂಸ್ಥಾನದಲ್ಲಿ }

 ಕಲ್ಕತ್ತೆಯಲ್ಲಿ ಸ್ವಾಮೀಜಿಯ ಆರೋಗ್ಯ ಪರಿಸ್ಥಿತಿ ಕೆಡುತ್ತ ಬಂದಿತು. ವೈದ್ಯರು ಯಾವುದಾದರೂ ಪರ್ವತ ಪ್ರಾಂತ್ಯಗಳಿಗೆ ಹೋಗಬೇಕೆಂದು ಸೂಚನೆ ಕೊಟ್ಟರು. ಅದರಂತೆಯೇ ಹಿಮಾಲಯ ಪ್ರಾಂತ್ಯದಲ್ಲಿರುವ ಆಲ್ಮೋರಕ್ಕೆ ಮೇ ೧೧ನೇ ತಾರೀಖು ಹೊರಟರು. ಅವರ ಜೊತೆಯಲ್ಲಿ ಕೆಲವು ಗುರುಭಾಯಿಗಳೂ ಹೊರಟರು. ಇಂಗ್ಲೆಂಡಿನಿಂದ ಸ್ವಾಮೀಜಿಯವರಿಗೆ ಸ್ತ್ರೀಯರ ಕೆಲಸಕ್ಕೆ ಸಹಾಯ ಮಾಡಬೇಕೆಂದು ಮಿಸ್ ಮುಲ್ಲರ್ ಎಂಬಾಕೆ ಬಂದಿದ್ದಳು. ಆಕೆಯೂ ಸ್ವಾಮೀಜಿಗೆ ಮುಂಚೆಯೆ ಆಲ್ಮೋರಕ್ಕೆ ಹೋಗಿದ್ದಳು. ಆಲ್ಮೋರಕ್ಕೆ ಹೋಗುವ ದಾರಿಯಲ್ಲಿ ಸಿಕ್ಕುವ ಲಕ್ನೋ ನಗರದಲ್ಲಿ ಒಂದು ರಾತ್ರಿ ತಂಗಿದ್ದರು. ಅಲ್ಲಿಯ ಪುರಜನರು ಹಾರ್ದಿಕ ಸ್ವಾಗತವನ್ನು ಬಯಸಿದರು. ಅಲ್ಲಿಂದ ಕಾತ್‍ಗೋಡಾಂ ಎಂಬ ಊರಿಗೆ ಹೋದರು. ಅಲ್ಲಿ ಆಲ್ಮೋರದಿಂದ ಅನೇಕ ಜನ ಸ್ವಾಮೀಜಿಯವರನ್ನು ಸ್ವಾಗತಿಸುವುದಕ್ಕೆ ಬಂದಿದ್ದರು. ಆಲ್ಮೋರದಲ್ಲಿಯೇ ಸಧ್ಯದಲ್ಲಿ ವಾಸಿಸುತ್ತಿದ್ದ ಜೆ. ಜೆ. ಗುಡ್‍ವಿನ್ ಕೂಡ ಸ್ವಾಮೀಜಿಯವರನ್ನು ನೋಡುವುದಕ್ಕೆ ಬಂದಿದ್ದರು. ಆಲ್ಮೋರಕ್ಕೆ ಕೆಲವು ಮೈಲಿಗಳ ದೂರದ ಒಂದು ಹಳ್ಳಿಯಾದ ಲೋಡಿಯಾದಲ್ಲಿ ಸ್ವಾಮೀಜಿಯವರನ್ನು ಒಂದು ಅಲಂಕೃತವಾದ ಕುದುರೆಯ ಮೇಲೆ ಕೂಡಿಸಿ ಮೆರವಣಿಗೆಯಲ್ಲಿ ಆಲ್ಮೋರಕ್ಕೆ ಕರೆದುಕೊಂಡು ಹೋದರು. ಆಲ್ಮೋರದಲ್ಲಿ ಸಹಸ್ರಾರು ಜನ ನರನಾರಿಯರು ಸ್ವಾಮೀಜಿಯವರನ್ನು ಸ್ವಾಗತಿಸಿದರು. ಊರಿನ ಮಧ್ಯದಲ್ಲಿ ಒಂದು ಪೇಟೆಯ ಬೀದಿಗೆ ಚಪ್ಪರ ಹಾಕಿ ಸ್ವಾಮೀಜಿಯವರಿಗೆ ಬಿನ್ನವತ್ತಳೆಯನ್ನು ಕೊಡಲು ಅಣಿ ಮಾಡಿದ್ದರು. ಸ್ವಾಗತ ಸಮಿತಿಯ ಪರವಾಗಿ ಪಂಡಿತ್ ಜ್ವಾಲಾದತ್ ಜೋಷಿ ಹಿಂದಿಯಲ್ಲಿ ಬಿನ್ನವತ್ತಳೆಯನ್ನು ಓದಿದರು. ಅನಂತರ ಪಂಡಿತ ಹರಿರಾಮ್ ಪಾಂಡೆ ಅವರು ಮತ್ತೊಂದು ಬಿನ್ನವತ್ತಳೆಯನ್ನು ಅರ್ಪಿಸಿದರು. ಅನಂತರ ಇನ್ನೊಬ್ಬರು ಪಂಡಿತರು ಸಂಸ್ಕೃತದಲ್ಲಿ ಒಂದು ಬಿನ್ನವತ್ತಳೆಯನ್ನು ಓದಿದರು. 

 ಸ್ವಾಮೀಜಿ ಸೂಕ್ತವಾಗಿ ಉತ್ತರವೀಯುತ್ತ ಹಿಮಾಲಯದ ಮೇಲೆ ತಮಗೆ ಇರುವ ಗೌರವ ಮತ್ತು ಅಭಿಮಾನವನ್ನು ಕಾವ್ಯಮಯವಾದ ಭಾಷೆಯಲ್ಲಿ ಹೀಗೆ ವಿವರಿಸಿದರು: “ನಮ್ಮ ಪೂರ್ವಜರ ಸ್ವಪ್ರದೇಶವಿದು. ಭಾರತ ಜನನಿ ಪಾರ್ವತಿ ಉದಯಿಸಿದ ಸ್ಥಳ. ಭಾರತವರ್ಷದ ಸತ್ಯಪಿಪಾಸುಗಳೆಲ್ಲ ತಮ್ಮ ಜೀವನದ ಕೊನೆಯ ದಿನವನ್ನು ಕಳೆಯಬೇಕೆಂದು ಬಯಸುವ ಸ್ಥಳ. ಈ ಪವಿತ್ರ ಗಿರಿಶಿಖರದ ಮೇಲೆ, ಸುತ್ತಮುತ್ತಲಿರುವ ಗಹ್ವರಗಳಲ್ಲಿ, ರಭಸದಿಂದ ಸಾಗುತ್ತಿರುವ ನದೀ ತೀರದಲ್ಲಿ ಅತಿ ಗಹನವಾದ ವಿಷಯಗಳನ್ನು ನಮ್ಮ ಪೂರ್ವಿಕರು ಪ್ರಾಚೀನ ಕಾಲದಲ್ಲಿ ಅಲೋಚಿಸಿದರು. ಅದರಲ್ಲಿ ಒಂದು ಅಂಶ ಮಾತ್ರ ಪ್ರಕಟವಾಗಿದೆ. ಅದೇ ಪಾಶ್ಚಾತ್ಯರ ಆದರಕ್ಕೆ ಅಷ್ಟೊಂದು ಪಾತ್ರವಾಗಿದೆ. ಮಹಾ ಮಹಾ ವಿದ್ಯಾವಂತರೆ ಅದಕ್ಕೆ ವಿಸ್ಮಯಪಟ್ಟಿರುವರು. ನಾನು ಬಾಲ್ಯದಿಂದಲೂ ವಾಸಮಾಡಬೇಕೆಂದು ಕನಸು ಕಾಣುತ್ತಿದ್ದ ಸ್ಥಳವೇ ಇದು. ನಾನು ಪುನಃ ಇಲ್ಲಿಗೆ ಬಂದು ವಾಸಿಸುವುದಕ್ಕೆ ಪ್ರಯತ್ನ ಪಡುತ್ತಿರುವೆನು ಎಂಬುದು ನಿಮಗೆಲ್ಲಾ ಗೊತ್ತು. ಅದಕ್ಕೆ ಸರಿಯಾದ ಸಮಯ ಒದಗಿ ಬರಲಿಲ್ಲ. ಮಾಡುವುದಕ್ಕೆ ಕೆಲಸವಿತ್ತು. ಈ ಪವಿತ್ರ ಸ್ಥಳದಿಂದ ಹೊರಗೆ ಹೋಗಬೇಕಾಗಿ ಬಂತು. ಆದರೂ ಋಷಿಗಳ ನೆಲೆಬೀಡಾಗಿದ್ದ ನಮ್ಮ ತತ್ತ್ವಶಾಸ್ತ್ರಗಳ ಜನ್ಮಸ್ಥಾನವಾದ ಗಿರಿರಾಜನ ಆಶ್ರಯದಲ್ಲಿ ನನ್ನ ಜೀವನವನ್ನು ಕಳೆಯಬೇಕೆಂದು ಇಚ್ಛೆ. ಬಹುಶಃ ನಾನು ಹಿಂದೆ ಆಲೋಚಿಸಿದಂತೆ ಮಾಡುವುದಕ್ಕೆ ಆಗದೇ ಇರಬಹುದು. ಆ ಮೌನ ಅಜ್ಞಾತವಾಸವನ್ನು ದಯಪಾಲಿಸೆಂದು ನಾನು ಎಷ್ಟು ಪ್ರಾರ್ಥಿಸುತ್ತೇನೆ! ಪ್ರಪಂಚದಲ್ಲೆಲ್ಲ ಈ ಸ್ಥಳದಲ್ಲಿ ನನ್ನ ಕೊನೆಗಾಲವನ್ನು ಕಳೆಯಬೇಕೆಂದು ಇಚ್ಛಿಸುವೆನು.” 

 “ನನ್ನ ಕಣ್ಣಿಗೆ ಈ ಗಿರಿರಾಜನ ಶಿಖರಗಳು ಒಂದಾದ ಮೇಲೊಂದು ಕಂಡಾಗ ಕೆಲಸಮಾಡುವ ಕುತೂಹಲವೆಲ್ಲ, ಹಲವು ವರುಷಗಳಿಂದ ಕುದಿಯುತ್ತಿದ್ದ ಈ ಇಚ್ಛೆಯೆಲ್ಲ ತಣ್ಣಗಾಯಿತು. ಏನು ಹಿಂದೆ ಮಾಡಿದೆ, ಮುಂದೆ ಮಾಡಬೇಕೆಂದಿರುವೆ ಎಂಬುದನ್ನು ಮಾತನಾಡುವ ಬದಲು, ಹಿಮಾಲಯ ನಮಗೆ ಸನಾತನವಾಗಿ ಯಾವ ಸಂದೇಶವನ್ನು ಬೋಧಿಸುತ್ತಿದೆಯೊ, ಯಾವ ಸಂದೇಶ ಈ ವಾತಾವರಣದಲ್ಲಿ ಅನುರಣಿತವಾಗುತ್ತಿದೆಯೋ, ಈ ಗಿರಿಝರಿಗಳು ಹಾಡುವ ಯಾವ ಪಲ್ಲವಿ ನನ್ನ ಕಿವಿಗೆ ಬೀಳುತ್ತಿದೆಯೋ ಆ ‘ಸರ್ವಂ ವಸ್ತುಭಾಯಾನ್ವಿತಂ ಭುವಿ ನೃಣಾಂ ವೈರಾಗ್ಯಮೇವಾಭಯಂ’ - ಈ ಪ್ರಪಂಚದಲ್ಲಿ ಎಲ್ಲಾ ಭಯದಿಂದ ಕೂಡಿದೆ, ವೈರಾಗ್ಯವೊಂದೇ ಒಬ್ಬನನ್ನು ನಿರ್ಭೀತನನ್ನಾಗಿ ಮಾಡುವುದು - ಎಂಬ ಭಾವನೆಯನ್ನು ಮನಸ್ಸು ಚಿಂತಿಸತೊಡಗಿದೆ.” 

 ಸ್ವಾಮೀಜಿ ಆಲ್ಮೋರದಲ್ಲಿದ್ದಾಗ ಇಡಿ ದಿನ ಬಂದುಹೋಗುವ ಭಕ್ತರೊಡನೆ ಮಾತುಕತೆ ಆಡುವುದು ಚರ್ಚೆಮಾಡುವುದು ಇದರಲ್ಲಿ ನಿರತರಾಗಿದ್ದರು. ಅವರ ಆರೋಗ್ಯ ಕ್ರಮೇಣ ಆ ಪ್ರಶಾಂತ ಪರ್ವತದ ವಾತಾವರಣದಲ್ಲಿ ಉತ್ತಮಗೊಂಡಿತು. ಸ್ವಾಮೀಜಿಯವರೊಂದಿಗೆ ಹಲವು ಸಂನ್ಯಾಸಿಗಳಿದ್ದರು. ಯೋಗಾನಂದ, ನಿರಂಜನಾನಂದ, ಅದ್ಭುತಾನಂದ, ಅಚ್ಯುತಾನಂದ, ವಿಜ್ಞಾನಾನಂದ, ಸದಾನಂದ, ವೃದ್ಧಸಚ್ಚಿದಾನಂದ, ಶುದ್ಧಾನಂದ, ಬ್ರಹ್ಮಚಾರಿ ಕೃಷ್ಣಲಾಲ ಮತ್ತು ಗುಡ್‍ವಿನ್ ಇವರುಗಳೊಡನೆ ಸಂತೋಷದಿಂದ ಕಾಲವನ್ನು ಕಳೆಯುತ್ತಿದ್ದರು. ಬೋಧಪ್ರದವಾದುದನ್ನು ವಿರಾಮವಿದ್ದಾಗಲೆಲ್ಲ ಇವರಿಗೆ ಕಲಿಸುತ್ತಿದ್ದರು. 

 ಸ್ವಾಮಿ ವಿವೇಕಾನಂದರ ಗುರುಭಾಯಿಗಳಾದ ಅಖಂಡಾನಂದರು ಮುರ್ಷಿದಾಬಾದ್ ಜಿಲ್ಲೆಯಲ್ಲಿ ಸಂಚಾರ ಮಾಡುತ್ತಿದ್ದಾಗ ಬರಗಾಲದಲ್ಲಿ ಅಲ್ಲಿಯ ಜನ ನರಳುತ್ತಿರುವುದನ್ನು ಕಂಡರು. ತಮ್ಮ ಕೈಲಾದ ಸಹಾಯವನ್ನು ಮಾಡಬೇಕೆಂದು ಪರಿಹಾರದ ಕಾರ್ಯಕ್ರಮವನ್ನು ಕೈಗೊಂಡರು. ಸ್ವಾಮೀಜಿಯವರಿಗೆ ಇದು ಗೊತ್ತಾದ ಒಡನೆಯೆ, ಅವರ ಸಹಾಯಕ್ಕೆ ಸ್ವಾಮಿ ನಿತ್ಯಾನಂದ ಮತ್ತು ಸುರೇಶ್ವರಾನಂದ ಅವರನ್ನು ಕಳುಹಿಸಿದರು. ಕಲ್ಕತ್ತೆಯ ಪೇಪರುಗಳ ಮೂಲಕ ಪರಿಹಾರನಿಧಿಗೆ ಮನವಿ ಮಾಡಿಕೊಂಡರು. ಕಲ್ಕತ್ತಾ, ಕಾಶಿ, ಮದ್ರಾಸ್ ಮಹಾಬೋಧಿ ಸೊಸೈಟಿಗಳಿಂದ ಬೇಕಾದಷ್ಟು ಹಣ ಒದಗಿತು. ಅದನ್ನೆಲ್ಲ ವಿತರಣೆಯಿಂದ ಯೋಗ್ಯರಾದವರಿಗೆ ಹಂಚಿದರು. ಇತ್ತ ಕಲ್ಕತ್ತೆಯಲ್ಲಿ ಸ್ಥಾಪಿಸತವಾದ ವೇದಾಂತ ಸಂಘ ಯಶಸ್ವಿಯಾಗಿ ಕೆಲಸ ಮಾಡುತ್ತಿರುವುದನ್ನು ಕೇಳಿ ತೃಪ್ತರಾದರು. ಮದ್ರಾಸಿಗೆ ಹೋಗಿ ರಾಮಕೃಷ್ಣಾನಂದ ಸ್ವಾಮಿಗಳು ತಮ್ಮ ಅದ್ಭುತವಾದ ಆಧ್ಯಾತ್ಮಿಕ ಜೀವನದಿಂದ ಇತರರನ್ನು ಆಕರ್ಷಿಸಿದರು. ಗೀತಾ ಉಪನಿಷತ್ತಿನ ಮೇಲೆ ಹಲವು ಪ್ರವಚನಾದಿಗಳನ್ನು ಕೊಟ್ಟರು. ಜನರು ಇದನ್ನು ಮೆಚ್ಚುತ್ತ ಬಂದರು. ಈ ಸಮಾಚಾರ ಸ್ವಾಮೀಜಿಗೆ ಮುಟ್ಟಿದಾಗ ಅವರು ಬಹಳ ಆನಂದಪಟ್ಟರು. ತಾವು ನೆಟ್ಟ ಸಸಿಗಳು ಈ ಆಶ್ರಮ. ಅವು ಕ್ರಮೇಣ ಬೇರು ತಾಕಿ ಚೆನ್ನಾಗಿ ಬೆಳೆಯುತ್ತಿರುವುದನ್ನು ನೋಡಿದಾಗ ಅವರಿಗೆ ಸಹಜವಾಗಿಯೇ ಸಂತೋಷವಾಯಿತು. 

 ಸ್ವಾಮೀಜಿ ಇನ್ನೇನು ಆಲ್ಮೋರವನ್ನು ಬಿಡುವ ಹೊತ್ತಿಗೆ ಅಲ್ಲಿಯ ಜನ ಅವರ ಉಪನ್ಯಾಸವನ್ನು ಕೇಳಬೇಕೆಂದು ಕೋರಿದರು. ಅದಕ್ಕಾಗಿ ಅಲ್ಲಿಯ ಜಿಲ್ಲೆಯ ಸ್ಕೂಲಿನಲ್ಲಿ ಒಂದು ಉಪನ್ಯಾಸವನ್ನು, ಇಂಗ್ಲಿಷ್ ಕ್ಲಬ್ಬಿನಲ್ಲಿ ಒಂದು ಉಪನ್ಯಾಸವನ್ನು ಮಾಡಿದರು. ಅದರಲ್ಲಿ ಒಂದು ಉಪನ್ಯಾಸವನ್ನು ಸ್ವಾಮೀಜಿ ಹಿಂದೀಭಾಷೆಯಲ್ಲಿಯೇ ಮಾಡಿದರು. ಇದೇ ಹಿಂದೀ ಭಾಷೆಯಲ್ಲಿ ಮಾಡಿದ ಮೊಟ್ಟಮೊದಲನೆಯ ಉಪನ್ಯಾಸವಾದರೂ ಅದರಲ್ಲಿ ಚೆನ್ನಾಗಿಯೇ ಅವರು ಯಶಸ್ವಿಯಾದರು. ಇಂಗ್ಲೀಷ್ ಕ್ಲಬ್ಬಿನಲ್ಲಿ ಮಾಡಿದ ಉಪನ್ಯಾಸ ಇಂಗ್ಲೀಷಿನಲ್ಲಿಯೇ ಇತ್ತು. ಅಲ್ಲಿಗೆ ಆಲ್ಮೋರದಲ್ಲಿರುವ ಅನೇಕ ಜನ ಇಂಗ್ಲೀಷಿನವರು ಬಂದಿದ್ದರು. ಗೂರ್ಖಾ ರೆಜಿಮೆಂಟಿನ ಕರ್ನಲ್ ಪುಲ್ಲಿ ಎಂಬಾತ ಅಧ್ಯಕ್ಷನಾಗಿದ್ದ. ಸ್ವಾಮೀಜಿ ಅತಿ ಸೂಕ್ಷ್ಮವಾದ ವೇದಾಂತದ ಗಹನ ಭಾವನೆಗಳನ್ನು ಎಲ್ಲರಿಗೂ ಮನಮುಟ್ಟುವಂತೆ ಹೇಳಿ ಉಪನ್ಯಾಸವನ್ನು ಕೇಳಿದವರಿಗೆ ಅವರ ಭಾವ ಅನುಭವ ವೇದ್ಯವಾಗುವಂತೆ ಮಾಡಿದರು. 

 ಸ್ವಾಮೀಜಿಯವರು ಸುಮಾರು ಎರಡೂವರೆ ತಿಂಗಳನ್ನು ಆಲ್ಮೋರದಲ್ಲಿ ಕಳೆದಾದ ಮೇಲೆ ಪಂಜಾಬು ಮತ್ತು ಕಾಶ್ಮೀರಗಳಿಂದ ಅವರಿಗೆ ಕರೆ ಬರುತ್ತಿದ್ದವು. ಅದನ್ನು ಒಪ್ಪಿಕೊಳ್ಳುವುದಕ್ಕಾಗಿ ಬಯಲು ಸೀಮೆಗೆ ಇಳಿದರು. ಸ್ವಾಮೀಜಿ ಮೊದಲು ಆಗಸ್ಟ್ ೯ನೇ ತಾರೀಖು ಬರೇಲಿಗೆ ಬಂದರು. ಅಲ್ಲಿಯವರು ಸ್ವಾಮೀಜಿಯವರನ್ನು ಸ್ವಾಗತಿಸಿ ಅವರನ್ನು ಕ್ಲಬ್ ಹೌಸಿಗೆ ವಾಸಕ್ಕಾಗಿ ಕರೆದುಕೊಂಡು ಹೋದರು. ಸ್ವಾಮೀಜಿಯವರು ಬರೇಲಿಗೆ ಬಂದೊಡನೆ ಜ್ವರ ಬಂದಿತು. ಸುಮಾರು ನಾಲ್ಕು ದಿನಗಳು ಸ್ವಾಮೀಜಿ ಅಲ್ಲಿ ಕಳೆದರು. ತಮ್ಮ ಆರೋಗ್ಯ ಚೆನ್ನಾಗಿಲ್ಲದೇ ಇದ್ದರೂ ಬಂದವರೊಡನೆ ಮಾತುಕತೆಯಾಡುವುದು ಅವರಿಗೆ ಧಾರ್ಮಿಕ ವಿಷಯಗಳನ್ನು ಹೇಳುವುದು ಇದರಲ್ಲಿ ನಿರತರಾಗಿದ್ದರು. ಮಾರನೆ ದಿನ ಅಲ್ಲಿಯ ಆರ‍್ಯ ಸಮಾಜದ ಅನಾಥಾಲಯಕ್ಕೆ ಭೇಟಿಕೊಟ್ಟರು. ಆ ಊರಿನ ವಿದ್ಯಾರ್ಥಿಗಳನ್ನು ಉದ್ದೇಶಿಸಿ ಮಾತನಾಡಿ ಅವರಲ್ಲೆ ಒಂದು ಸಂಘವನ್ನು ಮಾಡಿಕೊಂಡು ಜನಸೇವೆಯನ್ನು ಮಾಡಬೇಕು, ವೇದಾಂತ ಭಾವನೆಯನ್ನು ಜನರಲ್ಲಿ ಸಾರಬೇಕು ಎಂದು ಹೇಳಿದುದರಿಂದ ಅಂದೇ ಒಂದು ಸಂಘವನ್ನು ಸ್ಥಾಪಿಸಿದರು. ಅಂದಿನ ದಿನ ಸಾಯಂಕಾಲ ಸ್ವಾಮೀಜಿ ತಮ್ಮ ಶಿಷ್ಯರಾದ ಅಚ್ಯುತಾನಂದರೊಡನೆ ಮಾತನಾಡುತ್ತಿದ್ದಾಗ ತಮ್ಮ ಕ್ಷೀಣಿಸುತ್ತಿರುವ ಆರೋಗ್ಯವನ್ನು ನೋಡಿ ತಾವು ನಾಲ್ಕು ವರ್ಷಕ್ಕಿಂತ ಹೆಚ್ಚು ಬದುಕುವುದಿಲ್ಲ ಎಂಬ ಭಾವನೆಯನ್ನು ವ್ಯಕ್ತ ಪಡಿಸಿದರು. ಆದರೆ ಆಗ ಯಾರೂ ಅದಕ್ಕೆ ಗಮನಕೊಡಲಿಲ್ಲ. ಏನೋ ಅಂದಾಜಿನ ಮೇಲೆ ಇವುಗಳನ್ನೆಲ್ಲಾ ಹೇಳುತ್ತಿರುವರು ಎಂದು ಭಾವಿಸಿದರು. ಆದರೆ ಅವರು ಅನಂತರದ ಜೀವನವನ್ನು ನೋಡಿದರೆ ಇದು ಸತ್ಯವಾಗಿ ಕಾಣುವುದು. 

 ಹನ್ನೆರಡನೆಯ ತಾರೀಖು ರಾತ್ರಿ ಸ್ವಾಮೀಜಿ ಅಂಬಾಲಕ್ಕೆ ಹೊರಟರು. ಅಲ್ಲಿ ಅವರು ಒಂದು ವಾರ ಇದ್ದರು. ಅನೇಕ ಜನ ಅಲ್ಲಿ ಅವರನ್ನು ನೋಡುವುದಕ್ಕೆ ಬಂದಿದ್ದರು. ಸಿಮ್ಲಾದಿಂದ ಸೇವಿಯರ‍್ಸ ದಂಪತಿಗಳು ಸ್ವಾಮೀಜಿಯವರನ್ನು ಕಾಣಲು ಬಂದಿದ್ದರು. ಸ್ವಾಮೀಜಿಯವರು ಅಲ್ಲಿ ಇರುವ ತನಕ ಶಾಸ್ತ್ರ ವಿಷಯದ ಮೇಲೆ ಪ್ರವಚನವನ್ನು ಕೊಡುತ್ತಿದ್ದರು. ಅದನ್ನು ಕೇಳುವುದಕ್ಕೆ ಹಿಂದೂಗಳು, ಮುಸ್ಲಿಮರು, ಆರ‍್ಯಸಮಾಜ ಮತ್ತು ಬ್ರಹ್ಮಸಮಾಜದವರು ಬರುತ್ತಿದ್ದರು. ಹದಿನಾರನೆ ತಾರೀಖು ಸಾಯಂಕಾಲ ಲಾಹೋರಿನ ಕಾಲೇಜಿನ ಒಬ್ಬ ಪ್ರಾಧ್ಯಾಪಕರು ಸ್ವಾಮೀಜಿಯವರ ಒಂದು ಸಣ್ಣ ಉಪನ್ಯಾಸವನ್ನು ಫೋನೋಗ್ರಾಮಿನಲ್ಲಿ ರಿಕಾರ್ಡ್ ಮಾಡಬೇಕು ಎಂದು ಕೋರಿಕೊಂಡ ಪ್ರಯುಕ್ತ ಒಂದು ಸಣ್ಣ ಉಪನ್ಯಾಸವನ್ನು ಮಾಡಿದರು. ಇವಾವುವೂ ನಮಗೆ ಸಿಕ್ಕಿಲ್ಲ. ಅವೇನಾದರೂ ಸಿಕ್ಕಿದರೆ ನಾವು ಅವರ ಧ್ವನಿಯನ್ನು ಕೇಳಬಹುದಾಗಿತ್ತು! ಮಾರನೆಯ ದಿನ ಅವರ ಆರೋಗ್ಯ ಅಷ್ಟು ಚೆನ್ನಾಗಿಲ್ಲದೆ ಇದ್ದರೂ ಸುಮಾರು ಒಂದೂವರೆ ಗಂಟೆ ಮಾತನಾಡಿದರು. ಅಲ್ಲಿ ಹಿಂದೂ ಮುಸ್ಲಿಮ್ ಸ್ಕೂಲ್ ಒಂದನ್ನು ಅವರು ನೋಡಿದರು. 

 ೧೦ನೇ ತಾರೀಖು ಸ್ವಾಮೀಜಿ ಮತ್ತು ಅವರ ಪರಿವಾರದವರು ಅಮೃತಸರಕ್ಕೆ ಹೋದರು. ಅವರ ಜೊತೆಯಲ್ಲಿ ಸೇವಿಯರ್ಸ್‍‍ ದಂಪತಿಗಳು ಕೂಡ ಇದ್ದರು. ಅಲ್ಲಿಯೂ ಕೂಡ ಊರಿನ ಅನೇಕ ಜನ ಸ್ವಾಮೀಜಿಯವರನ್ನು ಸ್ವಾಗತಿಸಲು ರೈಲ್ವೆ ನಿಲ್ದಾಣಕ್ಕೆ ಬಂದಿದ್ದರು. ಆದರೆ ಸ್ವಾಮೀಜಿಯವರಿಗೆ ದೇಹಾರೋಗ್ಯ ಚೆನ್ನಾಗಿಲ್ಲದೆ ಇದ್ದುದರಿಂದ ಎಲ್ಲೋ ಕೆಲವು ಗಂಟೆಗಳನ್ನು ಮಾತ್ರ ಅಲ್ಲಿಯ ವಕೀಲರಾದ ತೋದರಮಲ್ಲ ಅವರ ಮನೆಯಲ್ಲಿ ಕಳೆದು ಅಲ್ಲಿಂದ ಧರ್ಮಶಾಲಾ ಎಂಬ ಸುಂದರವಾದ ಪರ್ವತನಗರಿಗೆ ಸೇವಿಯರ‍್ಸ ದಂಪತಿಗಳೊಡನೆ ಹೊರಟರು. ಅಲ್ಲಿ ಇವರನ್ನು ನೋಡುವುದಕ್ಕೆ ಬರುತ್ತಿರುವರ ಸಂಖ್ಯೆ ಬಹಳ ಕಡಿಮೆಯಾಗಿತ್ತು. ಅಲ್ಲಿ ಅವರು ಮೂವತ್ತೊಂದನೇ ತಾರೀಖಿನವರೆಗೆ ಇದ್ದರು. ಅನಂತರ ಬಯಲುಸೀಮೆಗೆ ಹೋಗಿ ತಮ್ಮ ಸಂದೇಶವನ್ನು ಸಾರಬೇಕೆಂದು ಬಯಸಿದರು. ಧರ್ಮಶಾಲೆಯಿಂದ ಅಮೃತಸರಕ್ಕೆ ಬಂದು ಅಲ್ಲಿ ಎರಡು ದಿನಗಳು ಇದ್ದರು. ಅಲ್ಲಿ ಪ್ರಖ್ಯಾತ ಆರ‍್ಯಸಮಾಜದ ರಾಯ್ ಮುಲ್‍ರಾಜ್ ಮತ್ತು ಇತರರೊಡನೆ ಮಾತುಕತೆ ಆಡಿದರು. ಅಲ್ಲಿಂದ ಅವರು ರಾವಲ್ಪಿಂಡಿಗೆ ಹೋದರು. ಅಲ್ಲಿ ಸ್ವಾಮೀಜಿಯವರಿಗೆ ತಂಗುವುದಕ್ಕೆ ಅಣಿಮಾಡಿದ್ದರು. ಆ ಸ್ಥಳ ಅಷ್ಟು ಚೆನ್ನಾಗಿಲ್ಲದೇ ಇದ್ದುದರಿಂದ ಅಲ್ಲಿಂದ ಮುರ‍್ರಿಗೆ ಹೋದರು. ಸ್ವಾಮೀಜಿ ಮುರ‍್ರಿಯಲ್ಲಿದ್ದಾಗ ಅಲ್ಲಿಯ ಪ್ರಖ್ಯಾತ ವಕೀಲರಾದ ಹನ್ಸ್‌ರಾಜ್ ಎಂಬುವರ ಅತಿಥಿಗಳಾಗಿದ್ದರು. ಅನೇಕ ವೇಳೆ ಜನ ಇವರನ್ನು ಉಪನ್ಯಾಸಕ್ಕೆ ಕರೆಯುತ್ತಿದ್ದರೂ, ಸ್ವಾಮೀಜಿ ಆರೋಗ್ಯ ಅಷ್ಟು ಚೆನ್ನಾಗಿಲ್ಲದೇ ಇದ್ದುದರಿಂದ ಬಹಿರಂಗ ಉಪನ್ಯಾಸವನ್ನು ಒಪ್ಪಿಕೊಳ್ಳಲಿಲ್ಲ. ಆದರೆ ಮನೆಯಲ್ಲಿಯೇ ಬೇಕಾದಷ್ಟು ಪ್ರವಚನಾದಿಗಳು ಆದುವು. ಅಲ್ಲಿಯೇ ತಮ್ಮ ಭಾವಗಳನ್ನು ವ್ಯಕ್ತಪಡಿಸಿದರು. 

 ಸ್ವಾಮೀಜಿ ಮುರ‍್ರಿಯಲ್ಲಿ ಎಲ್ಲೊ ಕೆಲವು ದಿನಗಳು ಮಾತ್ರ ಇದ್ದರು. ಸೆಪ್ಟೆಂಬರ್ ಆರನೇ ತಾರೀಖು ಕಾಶ್ಮೀರಕ್ಕೆ ಹೋಗಲು ನಿಶ್ಚಯಿಸಿದರು. ಕಾಶ್ಮೀರಕ್ಕೆ ಸ್ವಾಮೀಜಿ ಜೊತೆಯಲ್ಲಿ ಹೋಗಬೇಕೆಂದಿದ್ದ ಸೇವಿಯರ್ಸ್‍‍ ಅವರಿಗೆ ಅನಾರೋಗ್ಯವಾಯಿತು. ಅವರು ತಾವಿದ್ದ ಸ್ಥಳದಿಂದ ಸ್ವಾಮೀಜಿಯವರಿಗೆ ಒಂದು ಪತ್ರವನ್ನು ಬರೆದು ಅದರಲ್ಲಿ ತಮಗೆ ಸ್ವಾಮೀಜಿಯವರೊಂದಿಗೆ ಬರುವುದಕ್ಕಾಗುವುದಿಲ್ಲವೆಂದು ತಿಳಿಸಿ ಅವರ ಖರ್ಚಿಗೆ ಎಂಟು ನೂರು ರೂಪಾಯಿಗಳನ್ನು ಕಳುಹಿಸಿದರು. ಸ್ವಾಮೀಜಿಗೆ ಈ ಕಾಗದ ಮತ್ತು ಹಣ ಬಂದಾಗ ಸಂಜೆಯಾಗಿತ್ತು. ಹತ್ತಿರವಿದ್ದ ಜೋಗೇಶ್‍ಗೆ “ನಾನು ಇಷ್ಟೊಂದು ಹಣವನ್ನು ಏನು ಮಾಡಲಿ? ನಾವು ಫಕೀರರು. ಈ ಹಣ ನಮ್ಮಲ್ಲಿದ್ದರೆ ಇದನ್ನೆಲ್ಲ ಖರ್ಚು ಮಾಡಿಬಿಡುವೆವು. ಅದರಲ್ಲಿ ಅರ್ಧ ಮಾತ್ರ ತೆಗೆದುಕೊಳ್ಳುತ್ತೇನೆ. ಅದು ನನಗೆ ಮತ್ತು ನನ್ನ ಜೊತೆಯಲ್ಲಿರುವ ಗುರುಭಾಯಿಗಳಿಗೆ ಸಾಲುತ್ತದೆ” ಎಂದರು. ತಕ್ಷಣವೇ ಸೇವಿಯರ್ಸ್‍‍ ಇದ್ದ ಹೋಟಲಿಗೆ ಹೋಗಿ ಅವರನ್ನು ಕಂಡು ನಾನೂರು ರೂಪಾಯಿಗಳನ್ನು ಅವರಿಗೆ ವಾಪಸ್ ತೆಗೆದುಕೊಳ್ಳುವಂತೆ ಹೇಳಿದರು. 

 ಸ್ವಾಮೀಜಿ ಪರಿವಾರದೊಡನೆ ಮುರ‍್ರಿಯಿಂದ ಎಂಟನೇ ತಾರೀಖು ಬಾರಾಮುಲ್ಲಕ್ಕೆ ಕುದುರೆಯ ಗಾಡಿಯಲ್ಲಿ ಹೋದರು. ಅಲ್ಲಿಂದ ದೋಣಿಯಲ್ಲಿ ಶ‍್ರೀನಗರಕ್ಕೆ ಹೊರಟರು. ಹತ್ತನೆ ತಾರೀಖು ಶ‍್ರೀನಗರವನ್ನು ತಲುಪಿದರು. ಅಲ್ಲಿ ಜಸ್ಟಿಸ್ ಋಷಿಬೀರ್ ಮುಖ್ಯೋಪಾಧ್ಯಾಯ ಅವರ ಮನೆಯಲ್ಲಿ ಅತಿಥಿಗಳಾಗಿದ್ದರು. ಇಲ್ಲಿ ಸ್ವಾಮೀಜಿಯವರನ್ನು ನೋಡಲು ಅನೇಕ ಜನ ಬಂದರು. ಮಾರನೆ ದಿನ ಸ್ವಾಮೀಜಿ ಅರಮನೆಯನ್ನು ನೋಡುವುದಕ್ಕೆ ಹೋದರು. ಅರಮನೆಯ ಉನ್ನತ ಉದ್ಯೋಗಸ್ಥರು ಸ್ವಾಮೀಜಿಯವರೊಡನೆ ಮಾತುಕತೆ ಆಡಿದರು. ಅಲ್ಲಿ ಮಿತ್ರ ಎಂಬ ಅಧಿಕಾರಿಯೊಬ್ಬ, ಮಹಾರಾಜರು ಈಗ ಜಮ್ಮುನಲ್ಲಿರುವರೆಂದೂ ಅವರ ತಮ್ಮ ರಾಜಾರಾಮಸಿಂಗ್ ಎಂಬುವರು ಸ್ವಾಮೀಜಿಯವರನ್ನು ಮಾರನೆಯ ದಿನ ಸಂದರ್ಶಿಸಲು ಆಕಾಂಕ್ಷಿತರಾಗಿರುವರೆಂದೂ ಹೇಳಿದರು. ಮಾರನೆಯ ದಿನ ರಾಮಸಿಂಗರೂ ಸ್ವಾಮೀಜಿಯವರನ್ನು ಗೌರವದಿಂದ ಬರಮಾಡಿಕೊಂಡರು. ಸ್ವಾಮೀಜಿಯವರನ್ನು ಕುರ್ಚಿಯ ಮೇಲೆ ಕುಳ್ಳಿರಿಸಿ, ತಾವೆಲ್ಲ ನೆಲದ ಮೇಲೆ ಕುಳಿತುಕೊಂಡು ಸ್ವಾಮೀಜಿಯವರಿಂದ ಧರ್ಮ ಮತ್ತು ಜನ ಸಾಮಾನ್ಯರನ್ನು ಮೇಲೆತ್ತುವುದು ಮುಂತಾದ ವಿಷಯಗಳ ಮೇಲೆ ಮಾತುಕತೆಯನ್ನು ಕೇಳಿ ತುಂಬಾ ಉತ್ಸಾಹದಿಂದ ಸೂಚಿಸಿದರು. ತಾವು ಸ್ವಾಮೀಜಿಯವರ ಉದ್ದೇಶ ಪ್ರಚಾರಕ್ಕೆ ತಮ್ಮ ಕೈಲಾದ ಸೇವೆಯನ್ನು ಮಾಡಲು ಸಿದ್ಧವಿರುವುದಾಗಿ ಹೇಳಿದರು. 

 ಸ್ವಾಮೀಜಿಯವರು ಕಾಶ್ಮೀರದಲ್ಲಿರುವವರೆಗೆ ವಿದ್ಯಾರ್ಥಿಗಳು, ಪಂಡಿತರು, ಅಧಿಕಾರಿಗಳು ಮತ್ತು ಜನಸಾಧಾರಣರು ಎಲ್ಲರನ್ನೂ ಕಂಡು ಅವರಿಗೆ ಧಾರ್ಮಿಕ ವಿಷಯಗಳನ್ನು ಹೇಳಿದರು. ಕಾಶ್ಮೀರದ ಮಂತ್ರಿಗಳು ಸ್ವಾಮೀಜಿಯವರ ವಶಕ್ಕೆ ಒಂದು ಮನೆಯಂತಿರುವ ದೋಣಿಯನ್ನು ಕೊಟ್ಟರು. ಶ‍್ರೀನಗರದ ಸುತ್ತಮುತ್ತಲಿರುವ ಸರೋವರದ ಮೇಲೆ ಸ್ವಾಮೀಜಿ ತಮ್ಮ ಗುರುಭಾಯಿಗಳೊಡನೆ ಸಂಚರಿಸುತ್ತ ಮಧ್ಯೆ ಮಧ್ಯೆ ರಮ್ಯವಾದ ಸ್ಥಳಗಳನ್ನು ನೋಡುತ್ತಿದ್ದರು. ಸೆಪ್ಟೆಂಬರ್ ಇಪ್ಪತ್ತನೆ ತಾರೀಖು ದೋಣಿಯಲ್ಲಿ ಪಾಂಪುರ ಮತ್ತು ಅನಂತನಾಗಕ್ಕೆ ಹೋದರು. ಅಲ್ಲಿ ಇತಿಹಾಸ ಪ್ರಸಿದ್ಧವಾದ ಎಡಾಬೀರ್ ಎಂಬ ದೇವಸ್ಥಾನವನ್ನು ನೋಡಿದರು. ಅಲ್ಲಿಂದ ಕಾಲು ನಡಿಗೆಯಲ್ಲಿ ಮಾರ್ತಾಂಡ ಎಂಬ ತೀರ್ಥಸ್ಥಳಕ್ಕೆ ಹೋಗಿ ಅಲ್ಲಿ ಛತ್ರದಲ್ಲಿ ತಂಗಿದ್ದರು. ಅಲ್ಲಿ ನೆರೆದ ಪಂಡಿತರಿಗೆ ಶಾಸ್ತ್ರದ ವಿಷಯದ ಮೇಲೆ ವಿದ್ವತ್ ಪೂರ್ಣವಾದ ಒಂದು ಪ್ರವಚನವನ್ನು ಕೊಟ್ಟರು. ಅಲ್ಲಿಂದ ಅಚ್‍ವಾಲ್ ಎಂಬಲ್ಲಿಗೆ ಹೋದರು. ಹೋಗುವಾಗ ದಾರಿಯಲ್ಲಿ ಒಂದು ಪುರಾತನವಾದ ದೇವಸ್ಥಾನವನ್ನು ನೋಡಿದರು. ಅದು ಪಾಂಡವರ ಕಾಲಕ್ಕೆ ಸೇರಿದ್ದೆಂದು ಸ್ಥಳಪುರಾಣ ಹೇಳುವುದು. ಸ್ವಾಮೀಜಿ ಅಲ್ಲಿನ ಕೆತ್ತನೆ ಕೆಲಸವನ್ನು ನೋಡಿ ಸಂತೋಷಪಟ್ಟರು. ಅವರು ಅನಂತರ ಉಲಾರ್ ಸರೋವರದ ಮೂಲಕ ಬಾರಾಮುಲ್ಲಕ್ಕೆ ಬಂದರು. ಅನಂತರ ಮುರ‍್ರಿಗೆ ಬಂದರು. 

 ಸ್ವಾಮೀಜಿ ಮುರ‍್ರಿಯಲ್ಲಿ ಸೇವಿಯರ್ಸ್‍‍ ದಂಪತಿಗಳನ್ನು, ಪಂಜಾಬ್ ಮತ್ತು ಬಂಗಾಳಿ ಭಕ್ತರನ್ನು ಕಂಡರು. ಸ್ವಾಮೀಜಿ ಇಲ್ಲಿ ನಿಬಾರನ್‍ಬಾಬು ಎಂಬುವರ ಮನೆಯಲ್ಲಿ ಅತಿಥಿಗಳಾಗಿದ್ದರು. ಅಲ್ಲಿಗೆ ಅವರನ್ನು ಭೇಟಿ ಮಾಡಲು ಅನೇಕ ಜನ ಹೋಗುತ್ತಿದ್ದರು. ಅಕ್ಟೋಬರ್ ಹದಿನಾಲ್ಕನೇ ತಾರೀಖು ಸಾಯಂಕಾಲ ಮುರ‍್ರಿಯ ಪಂಜಾಬಿ ಮತ್ತು ಬಂಗಾಳಿ ಪುರಜನರು ಸ್ವಾಮೀಜಿಗೆ ಒಂದು ಬಿನ್ನವತ್ತಳೆಯನ್ನು ಅರ್ಪಿಸಿದರು. ಸ್ವಾಮೀಜಿ ಇದಕ್ಕೆ ಸೂಕ್ತವಾದ ಉತ್ತರವನ್ನು ಕೊಟ್ಟರು. ಮಾರನೆ ದಿನ ಸ್ವಾಮೀಜಿ ರಾವಲ್‍ಪಿಂಡಿಗೆ ಹೋದರು. ಅಲ್ಲಿ ಹನ್ಸ್‌ರಾಜ್ ಎಂಬುವರು ಸ್ವಾಮೀಜಿಯವರನ್ನು ಆದರದಿಂದ ಸತ್ಕರಿಸಿದರು. ಸ್ವಾಮೀಜಿ ಸುಜನ್‍ಸಿಂಗ್ ಎಂಬುವರ ಉದ್ಯಾನದಲ್ಲಿ ಒಂದು ಉಪನ್ಯಾಸವನ್ನು ಇತ್ತರು. ಅಕ್ಟೋಬರ್ ೨೦ನೇ ತಾರೀಖು ರಾತ್ರಿ ಕಾಶ್ಮೀರದ ಮಹಾರಾಜರು ಪ್ರಾರ್ಥಿಸಿಕೊಂಡಿದ್ದರಿಂದ ಜಮ್ಮುವಿಗೆ ಹೋದರು. ಸ್ವಾಮೀಜಿಯವರನ್ನು ರೈಲ್ವೆ ನಿಲ್ದಾಣದಲ್ಲಿ ಎದುರುಗೊಂಡು ಅವರನ್ನು ಸರ್ಕಾರದ ಅತಿಥಿಗಳನ್ನಾಗಿ ಮಾಡಿದರು. ಸ್ವಾಮೀಜಿ ಮಾರನೆಯ ದಿನ ಅಲ್ಲಿಯ ಪುಸ್ತಕಾಲಯವನ್ನು ನೋಡಿದರು. ಕಾಶ್ಮೀರದ ಅಧಿಕಾರಿಗಳಾದ ಬಾಬು ಮಹೇಶ ಚಂದ್ರಭಟ್ಟಾಚಾರ‍್ಯ ಎಂಬುವರೊಡನೆ ಕಾಶ್ಮೀರದಲ್ಲಿ ಸಂಘದ ಒಂದು ಕೇಂದ್ರವನ್ನು ತೆರೆಯುವ ವಿಷಯದಲ್ಲಿ ಮಾತುಕತೆ ಆಡಿದರು. ೨೩ನೇ ತಾರೀಖು ಮಹಾರಾಜರೊಡನೆ ಒಂದು ದೀರ್ಘ ಭೇಟಿ ಆಯಿತು. ಆ ಸಮಯದಲ್ಲಿ ಸ್ವಾಮೀಜಿಯವರ ಇಬ್ಬರು ಸಹೋದರರು ಮತ್ತು ಸಂಸ್ಥಾನದ ಉನ್ನತ ಅಧಿಕಾರಿಗಳೂ ಇದ್ದರು. ಅವರೊಡನೆ ಮಾತನಾಡುತ್ತಿದ್ದಾಗ ಹಿಂದೂಧರ್ಮ ಕೇವಲ ಬಾಹ್ಯಾಚಾರದಲ್ಲಿ ಕೊನೆಗೊಂಡಿದೆ ಎಂದೂ ಅದರ ಒಳಗಡೆ ಇರುವ ತಿರುಳನ್ನು ಜನ ಮರೆತುಬಿಟ್ಟಿರುವರೆಂದೂ ಅದನ್ನು ಎಲ್ಲರಿಗೂ ತಿಳಿಯುವಂತೆ ಮಾಡುವುದು ತಮ್ಮ ಕರ್ತವ್ಯವೆಂದೂ ಹೆಳಿದರು. ಅನಂತರ ಸ್ವಾಮೀಜಿ ಯುವರಾಜರನ್ನು ನೋಡುವುದಕ್ಕೆ ಹೋದರು. ಅವರೂ ಕೂಡ ಗೌರವದಿಂದ ಸ್ವಾಮೀಜಿಯವರನ್ನು ಸ್ವಾಗತಿಸಿದರು. ಮಾರನೆ ದಿನ ಒಂದು ಬಹಿರಂಗ ಉಪನ್ಯಾಸವನ್ನು ಕೊಟ್ಟರು. ಅದಕ್ಕೆ ಮಹಾರಾಜರು ಮುಂತಾದ ಪ್ರಮುಖ ವ್ಯಕ್ತಿಗಳೂ ಬಂದಿದ್ದರು. ಸ್ವಾಮೀಜಿಯವರನ್ನು ಮಾರನೆ ದಿನವೂ ಮತ್ತೊಂದು ಉಪನ್ಯಾಸವನ್ನು ಕೊಡುವಂತೆ ಕೋರಿಕೊಂಡರು. ಅದೂ ಅಲ್ಲದೆ ಹತ್ತು ಹನ್ನೆರಡು ದಿನಗಳಿದ್ದು ಪ್ರತಿದಿನವೂ ಉಪನ್ಯಾಸವನ್ನು ಕೊಡುವಂತೆ ಕೋರಿಕೊಂಡರು. ಆದರೆ ಅಷ್ಟೊಂದು ದಿನಗಳು ಸ್ವಾಮೀಜಿಯವರಿಗೆ ಇರಲು ಸಾಧ್ಯವಾಗಲಿಲ್ಲ. ಸ್ವಾಮೀಜಿಯವರು ೨೫ನೇ ತಾರೀಖಿನ ದಿನ ಮುನಿಸಿಪಲ್ ಪವರ್ ಹೌಸನ್ನು ನೋಡಿ ಅಲ್ಲಿಯವರೊಡನೆ ಸಂಭಾಷಣೆ ಮಾಡಿದರು. ಅಂದೇ ಆರ‍್ಯ ಸಮಾಜದ ಸ್ವಾಮಿಗಳಾದ ಅಚ್ಯುತಾನಂದರೊಡನೆ ಮಾತನಾಡುತ್ತ ಆರ್ಯಸಮಾಜದಲ್ಲಿರುವ ಒಳ್ಳೆಯ ವಿಷಯಗಳನ್ನು ತಾವು ಮೆಚ್ಚುವರೆಂದೂ ಆದರೆ ಅದರಲ್ಲಿರುವ ಕೆಲವು ಲೋಪದೋಷಗಳನ್ನು ತಿದ್ದಿಕೊಳ್ಳುವುದು ಮೇಲೆಂದೂ‌ ಅವರಿಗೆ ನಿಸ್ಸಂಕೋಚವಾಗಿ ವ್ಯಕ್ತ ಪಡಿಸಿದರು. ಅಂದಿನ ಸಾಯಂಕಾಲ ಮಹಾರಾಜರ ಕೋರಿಕೆಯಂತೆ ಹಿಂದೂಧರ್ಮದ ಮೇಲೆ ಮಾತನಾಡಿದರು. ಅಂದು ಪುಸ್ತಕಭಂಡಾರಕ್ಕೆ ಹೋಗಿ ಅಪೂರ್ವ ಓಲೆಗರಿಯ ಪ್ರತಿಗಳು ಮುಂತಾದುವನ್ನು ನೋಡಿದರು. ಸಂಜೆ ದೀಪಾವಳಿಯ ಪ್ರಯುಕ್ತ ಮಾಡಿದ ದೀಪಾಲಂಕಾರವನ್ನು ನೋಡಿದರು. ಅನಂತರ ಮೂರು ದಿನಗಳು ಹಲವರೊಡನೆ ಹಿಂದೂಧರ್ಮದ ವಿಷಯವಾಗಿ ಚರ್ಚೆ ಮಾಡಿದರು. ಸ್ವಾಮೀಜಿಯವರು ಈ ಸಮಯದಲ್ಲಿ ಹಿಂದಿಯಲ್ಲೇ ಮಾತುಕತೆ ಭಾಷಣ ಮುಂತಾದುವನ್ನು ಮಾಡುತ್ತಿದ್ದರು. ಮಾತನಾಡುತ್ತಿದ್ದ ವೀರ‍್ಯವತ್ತಾದ ಹಿಂದಿಭಾಷೆ ಮಹಾರಾಜರಿಗೆ ತುಂಬಾ ಮೆಚ್ಚಿಗೆಯಾಗಿ ಹಿಂದಿಯಲ್ಲಿ ಕೆಲವು ಲೇಖನಗಳನ್ನು ಬರೆದುಕೊಡುವಂತೆ ಸ್ವಾಮೀಜಿಯವರನ್ನು ಕೋರಿಕೊಂಡರು. ಸ್ವಾಮೀಜಿ ಅವರ ಕೋರಿಕೆಯಂತೆ ಒಂದೆರಡು ಲೇಖನಗಳನ್ನು ಬರೆದು ಕೊಟ್ಟರು. ಅದು ಎಲ್ಲರ ಮೆಚ್ಚುಗೆಯನ್ನೂ ಪಡೆಯಿತು. ಸ್ವಾಮೀಜಿಯವರು ೨೩ನೇ ತಾರೀಖು ಮಹಾರಾಜರನ್ನು ನೋಡಿ ತಾವು ಸಿಯಾಲ್‍ಕೋಟೆಗೆ ಹೋಗುವುದಾಗಿ ತಿಳಿಸಿದರು. ಮಹಾರಾಜರು ಮನಸ್ಸಿಲ್ಲದ ಮನಸ್ಸಿನಿಂದ ಸ್ವಾಮೀಜಿಯವರನ್ನು ಬೀಳ್ಕೊಟ್ಟರು. ತಾವು ಯಾವಾಗ ಬಂದರೂ ಕಾಶ್ಮೀರ ಅವರನ್ನು ಆದರದಿಂದ ಸ್ವಾಗತಿಸುವುದೆಂದು ಹೇಳಿದರು. 

 ಸ್ವಾಮೀಜಿಯವರು ಜಮ್ಮುವಿನಿಂದ ಸಿಯಾಲ್‍ಕೋಟೆಗೆ ಹೋದರು. ಅಲ್ಲಿ ಮೂಲ್‍ಚಂದ್ ಅವರ ಅತಿಥಿಗಳಾಗಿದ್ದರು. ಆ ಊರಿನಲ್ಲಿ ಇಂಗ್ಲೀಷ್‍ನಲ್ಲಿ ಒಂದು ಉಪನ್ಯಾಸವನ್ನೂ ಹಿಂದಿಯಲ್ಲಿ ಒಂದು ಉಪನ್ಯಾಸವನ್ನೂ ಮಾಡಿದರು. 

 ಅನಂತರ ಸ್ವಾಮೀಜಿಯವರು ಲಾಹೋರಿಗೆ ಹೋದರು. ಸನಾತನ ಧರ್ಮ ಸಮಾಜದವರು ಸ್ವಾಮೀಜಿಯವರನ್ನು ಸ್ವಾಗತಿಸುವುದಕ್ಕೆ ಒಂದು ಸಮಿತಿಯನ್ನು ರಚಿಸಿಕೊಂಡಿದ್ದರು. ಅವರು ರೈಲ್ವೆ ನಿಲ್ದಾಣಕ್ಕೆ ಬಂದು ಸ್ವಾಮೀಜಿಯವರನ್ನು ಸಾಗತಿಸಿ, ಅಲ್ಲಿಂದ ರಾಜ ಧ್ಯಾನಸಿಂಗ್ ಅವರ ಅರಮನೆಯವರೆಗೆ ಮೆರವಣಿಗೆಯಲ್ಲಿ ಕರೆದುಕೊಂಡು ಹೋದರು. ಅನಂತರ ಲಾಹೋರಿನ ಟ್ರಿಬ್ಯೂನ್ ಪತ್ರಿಕೆಯ ಸಂಪಾದಕ ಎಲ್. ಎನ್. ಗುಪ್ತ ಎಂಬುವರ ಮನೆಯಲ್ಲಿ ಅತಿಥಿಗಳಾಗಿದ್ದರು. ಅಲ್ಲಿ ಹಳೆಯ ಅರಮನೆಯೊಂದರಲ್ಲಿ ಸ್ವಾಮೀಜಿಯವರು ಹಿಂದೂಧರ್ಮದ ಮೇಲೆ ಉಪನ್ಯಾಸವನ್ನು ಮಾಡಿದರು. ಸ್ವಾಮೀಜಿಯವರು ಎಲ್ಲಾ ಹಿಂದುಗಳಿಗೆ ಪವಿತ್ರವಾಗಿರುವ ಶಾಸ್ತ್ರಗಳನ್ನು ಹೇಳಿ, ಅನಂತರ ಜೀವ, ಜಗತ್, ಈಶ್ವರನಿಗೆ ಸಂಬಂಧಪಟ್ಟ ವಿಷಯಗಳನ್ನು ಹೇಳಿದರು. ಆ ಸಮಯದ ಉಪನ್ಯಾಸದಲ್ಲಿ ಅವರು ಹೀಗೆ ಹೇಳುವರು: 

 “ಯಾವನು ಅಹೋರಾತ್ರಿ ತಾನು ಕೆಲಸಕ್ಕೆ ಬಾರದವನೆಂದು ಆಲೋಚಿಸುವನೊ ಅವನಿಂದ ಪ್ರಪಂಚಕ್ಕೆ ಯಾವ ಪ್ರಯೋಜನವೂ ಆಗುವಂತಿಲ್ಲ. ಹಗಲೂ ರಾತ್ರಿ ಒಬ್ಬನು ತಾನು ದುಃಖಿ, ನೀಚ, ಶೂನ್ಯನೆಂದು ಎಣಿಸಿದ್ದೇ ಆದರೆ ಅವನು ಶೂನ್ಯನಾಗುವುದರಲ್ಲಿ ಸಂದೇಹವಿಲ್ಲ. ಒಬ್ಬನು ತಾನು ಹಾಗೆ ತಾನು ಹೀಗೆ ಎಂದು ಮನಃಪೂರ್ವಕವಾಗಿ ಅಂದುಕೊಂಡರೆ ಅವನು ಹಾಗೆಯೇ ಆಗುವನು: ಈ ವಿಷಯವನ್ನು ನೀವು ನೆನಪಿನಲ್ಲಿಡಬೇಕು. ನಾವು ಸರ್ವಶಕ್ತನ ಪುತ್ರರು ಅನಂತವಾದ ದಿವ್ಯವಾದ ಜ್ವಾಲೆಯ ಕಿಡಿಗಳೂ ಆಗಿರುವಾಗ ನಾವು ಕೆಲಸಕ್ಕೆ ಬಾರದವರು ಆಗುವುದು ಹೇಗೆ/? ನಾವು ಎಲ್ಲವೂ ಆಗಿದ್ದೇವೆ, ಎಲ್ಲವನ್ನೂ ಮಾಡಲು ಸಿದ್ಧವಾಗಿದ್ದೇವೆ, ಎಲ್ಲವನ್ನೂ ಮಾಡಬಲ್ಲೆವು. ನನ್ನ ಅಭಿಪ್ರಾಯದಲ್ಲಿ ಮಾನವ ಎಲ್ಲವನ್ನೂ ಸಾಧಿಸಲೇಬೇಕು. ನಮ್ಮ ಪೂರ್ವಿಕರಲ್ಲಿ ಅಂತಹ ಆತ್ಮಶ್ರದ್ಧೆ ಇತ್ತು. ಈ ಆತ್ಮಶ್ರದ್ಧೆಯೆ ಅವರಿಗೆ ಶಕ್ತಿಯನ್ನಿತ್ತು ಅವರನ್ನು ನಾಗರಿಕರನ್ನಾಗಿ ಮಾಡಿತು. ನಾವು ಅಧೋಗತಿಗೆ ಇಳಿಯುವುದು, ನಮ್ಮಲ್ಲಿ ಕುಂದುಕೊರತೆಗಳು ಇರುವುದು ನಿಮಗೆ ನಿಜವಾಗಿ ತೋರಿದರೆ, ಇದು ಪ್ರಾರಂಭವಾದುದು ನಮ್ಮ ಆತ್ಮಶ್ರದ್ಧೆ ಯನ್ನು ಕಳೆದುಕೊಂಡ ದಿನದಿಂದ ಎಂದು ತಿಳಿಯಿರಿ. ಆತ್ಮಶ್ರದ್ಧೆ ತಪ್ಪಿತೆಂದರೆ ಈಶ್ವರನ ಶ್ರದ್ಧೆಯೂ ತಪ್ಪಿದಂತೆ. ಅನಂತನೂ ಮಂಗಳದಾಯಕನೂ ಆದ ಈಶ್ವರನು ನಿಮ್ಮಲ್ಲಿ ನಿಂತು ನಿಮ್ಮ ಮೂಲಕ ಕೆಲಸ ಮಾಡುತ್ತಿರುವನೆಂದೂ ಅವನು ಸರ್ವವ್ಯಾಪಿಯೆಂದೂ, ಅಂತರ್ಯಾಮಿಯೆಂದೂ, ಪ್ರತಿಯೊಂದು ಅಣುರೇಣುವಿನಲ್ಲೂ ನಮ್ಮ ದೇಹ ಮನಸ್ಸು ಜೀವ ಇವುಗಳೆಲ್ಲದರಲ್ಲಿಯೂ ಓತಪ್ರೋತನಾಗಿರುವನೆಂದೂ ನಾವು ನಂಬಿರುವಾಗ ನಾವು ಅಧೈರ‍್ಯ ಪಡುವುದು ತಾನೆ ಹೇಗೆ? ಅಗಾಧವಾದ ಸಮುದ್ರದಲ್ಲಿ ನಾವೊಂದು ಸಣ್ಣ ಗುಳ್ಳೆ ಯಾಗಿರಬಹುದು, ನೀವು ದೊಡ್ಡದೊಂದು ಅಲೆಯಾಗಿರಬಹುದು. ಆದರೇನು? ನಮ್ಮ ಮತ್ತು ನಿಮ್ಮ ಇಬ್ಬರ ಹಿಂದೆ ಇರುವುದು ಒಂದು ವಿಸ್ತಾರವಾದ ಸಾಗರ. ನಮ್ಮ ಈರ್ವರ ಹಿಂದೆಯೂ ಆತ್ಮಚೈತನ್ಯವೆಂಬ ಅನಂತಸಾಗರ ಇರುವುದು. ಭ್ರಾತೃವರ‍್ಯರೇ, ನೀವು ಎಲ್ಲರಿಗೂ ಉದ್ಧರಿಸುವ, ಉತ್ತಮಗೊಳಿಸುವ ಭವ್ಯವಾದ ಸಿದ್ಧಾಂತಗಳನ್ನು ನಿಮ್ಮ ಪುತ್ರ ಪೌತ್ರರಿಗೆ ಹುಟ್ಟಿದಂದಿನಿಂದ ಬೋಧಿಸಿ.” 

 ಅನಂತರ ಲಾಹೋರಿನ ಕಾಲೇಜಿನಲ್ಲಿ ಮಾಡಿದ ವೇದಾಂತ ಉಪನ್ಯಾಸದಲ್ಲಿ ಸ್ವಾಮೀಜಿಯವರು ಮಹಿಮೋನ್ನತವಾದ ಭಾವಗಳನ್ನು ಮುಂದಿಟ್ಟು ಕೊನೆಯಲ್ಲಿ ಹುರಿದುಂಬಿಸುವ ಜ್ವಾಲಾಮಯವಾದ ಭಾಷೆಯಲ್ಲಿ ಹೀಗೆ ಹೇಳುವರು: 

 “ಎದ್ದೇಳಿ! ಎಚ್ಚರಗೊಳ್ಳಿ! ಗುರಿ ದೊರಕುವವರೆಗೂ ಮುಂದೆ ನುಗ್ಗಿ, ನಿಲ್ಲದಿರಿ! ‘ಉತ್ತಿಷ್ಠತ, ಜಾಗ್ರತ, ಪ್ರಾಪ್ಯವರಾನ್ನಿಬೋಧತ!’ ಎದ್ದೇಳಿ, ತ್ಯಾಗವಿಲ್ಲದೆ ಯಾವುದೂ ಸಾಧ್ಯವಿಲ್ಲ. ಇತರರಿಗೆ ನೀವು ಸಹಾಯ ಮಾಡಬೇಕೆಂದು ಇದ್ದರೆ, ಮೊದಲು ನಿಮ್ಮ ಆತ್ಮ ವಿಸರ್ಜನೆ ಆಗಬೇಕು. ಹೌದು, ಕ್ರಿಸ್ತನು ಹೇಳುವಂತೆ ದೇವರನ್ನು ಮತ್ತು ಸೈತಾನನನ್ನು ಒಟ್ಟಿಗೆ ಸೇವಿಸಲಾರಿರಿ. ನಮಗೀಗ ಬೇಕಾಗಿರುವುದು ವೈರಾಗ್ಯ ಮತ್ತು ತ್ಯಾಗ. ನಿಮ್ಮ ಪೂರ್ವಿಕರು ಮಹಾಕಾರ‍್ಯಗಳನ್ನು ಮಾಡಲೋಸುಗ ಐಹಿಕಸುಖ ತ್ಯಾಗ ಮಾಡಿದರು. ಆದರೆ ಈಗ ಜನರು ತಮ್ಮ ಸ್ವಂತ ಸುಖಕ್ಕಾಗಿ ಐಹಿಕಸುಖವನ್ನು ಬಿಡುತ್ತಾರೆ. ಎಲ್ಲವನ್ನೂ ತ್ಯಜಿಸಿಬಿಡಿ. ನಿಮ್ಮ ಸ್ವಂತ ಮೋಕ್ಷವನ್ನೂ ಬಿಸಾಡಿ. ಹೋಗಿ ಇತರರ ಸೇವೆಮಾಡಿ. ನೀವು ಯಾವಾಗಲೂ ಧೀರ ವಚನಗಳನ್ನು ಆಡುತ್ತಿರುತ್ತೀರಿ. ಈಗ ನಿಮ್ಮ ಮುಂದಿದೆ ಕಾರ‍್ಯಕಾರಿಯಾದ ವೇದಾಂತ. ಇತರರ ಕ್ಷೇಮಕ್ಕಾಗಿ ನಿಮ್ಮ ಕ್ಷುದ್ರ ಜೀವಗಳನ್ನು ವಿಸರ್ಜಿಸಿ. ದೇಶದ ಹಿತಕ್ಕಾಗಿ ನಿಮ್ಮ ಸಣ್ಣ ಜೀವನವನ್ನು ಬಲಿದಾನಮಾಡಿ. ಹೊಟ್ಟೆಗಿಲ್ಲದೆ ಸತ್ತರೇನಂತೆ, ನಾವು ನೀವೂ ನಿಮ್ಮಂತೆ ಇನ್ನೂ ಸಹಸ್ರ ಜನರು ಮಡಿದರೇನಂತೆ? ಈ ಜನಾಂಗ ಉಳಿದರೆ ಸಾಕಲ್ಲವೆ? ದೇಶ ಮುಳುಗುತ್ತಿದೆ. ಅಸಂಖ್ಯ ಲಕ್ಷೋಪಲಕ್ಷ ಜೀವಿಗಳ ಅಭಿಶಾಪವು ನಮ್ಮ ತಲೆಯ ಬಿದ್ದಿದೆ. ಯಾರು ಬಾಯಾರಿಕೆಯಿಂದ ಸಾಯುತ್ತಿದ್ದಾಗ ಪಕ್ಕದಲ್ಲಿ ಅಮೃತವಾಹಿನಿ ಹರಿಯುತ್ತಿದ್ದರೂ ಬಚ್ಚಲ ನೀರನ್ನು ಕುಡಿಯಲು ಕೊಟ್ಟೆವೊ, ಯಾರನ್ನು ಸುಭಿಕ್ಷವನ್ನು ನೋಡುತ್ತ ನೋಡುತ್ತ ಕ್ಷಾಮಪೀಡಿತರಾಗಿ ಸಾಯುವಂತೆ ಮಾಡಿದೆವೊ, ಅದ್ವೈತವನ್ನು ಕುರಿತು ಮಾತನಾಡುತ್ತ ಹೃತ್ಪೂರ್ವಕವಾಗಿ ಯಾರನ್ನು ದ್ವೇಷಿಸಿದೆವೊ, ಯಾರಿಗೆ ಬಾಯಿಮಾತಿನಲ್ಲಿ ಮಾತ್ರ ಎಲ್ಲರೂ ಒಂದೇ, ಎಲ್ಲರಲ್ಲಿಯೂ ಈಶ್ವರನಿದ್ದಾನೆ ಎಂದು ಹೇಳುತ್ತ ಕಾರ‍್ಯತಃ ಅದನ್ನು ಒಂದಿನಿತೂ ಅಭ್ಯಾಸಮಾಡಲಿಲ್ಲವೋ ಅಂತಹ ಅಸಂಖ್ಯ ಲಕ್ಷೋಪಲಕ್ಷ ಜನಗಳ ಅಭಿಶಾಪವು ಸಿಡಿಲಿನಂತೆ ನಮ್ಮ ತಲೆಯಮೇಲೆ ಬೀಳುತ್ತಿದೆ. ಈ ಭೇದಭಾವದಿಂದ ಉತ್ಪನ್ನವಾದ ಪಾಪವನ್ನು ತೊರೆದುಹಾಕಿ. ಆಹಾರದಲ್ಲಿಯೂ ಸಮತ್ವವನ್ನು ಅನುಷ್ಠಾನ ಮಾಡಿ. ಈ ಸಣ್ಣ ಜೀವ ಹೋದರೇನಂತೆ? ಎಲ್ಲರೂ ಸಾಯಲೇ ಬೇಕು. ಮಹರ್ಷಿಯಾಗಲಿ, ಪಾಪಿಯಾಗಲಿ, ಶ‍್ರೀಮಂತನಾಗಲಿ ಯಾರ ದೇಹವೂ ಶಾಶ್ವತವಾದದ್ದಲ್ಲ. ಎದ್ದೇಳಿ ಎಚ್ಚರಗೊಳ್ಳಿ! ಹೃತ್ಪೂರ್ವಕವಾಗಿ ಪ್ರಯತ್ನ ಮಾಡಿ. ಈ ದೇಶದಲ್ಲಿ ನಮ್ಮ ವಕ್ರತೆಯೂ ಭೀಕರವಾಗಿದೆ. ನಮಗೀಗ ಬೇಕಾಗಿರುವುದು ಶೀಲ. ಪ್ರಾಣಹೋದರೂ ಹಿಂಜರಿಯದೆ ಕಾರ್ಯವನ್ನು ಸಾಧಿಸುವ ದೃಢಚಿತ್ತ ಬೇಕಾಗಿದೆ. ‘ತಾರ್ಕಿಕರು ನಿಂದಿಸಲಿ ಅಥವಾ ಪ್ರಶಂಸಿಸಲಿ, ಲಕ್ಷ್ಮಿ ಕೈಸೇರಲಿ ಬಿಡಲಿ! ಸಾವು ಈಗಲೇ ಬರಲಿ ಅಥವಾ ಇನ್ನೊಂದು ನೂರು ವರ್ಷಗಳ ಮೇಲೆ ಬರಲಿ, ದೃಢ ಮನಸ್ಕನು ಸನ್ಮಾರ್ಗದಿಂದ ಒಂದು ಹೆಜ್ಜೆಯನ್ನು ತಪ್ಪಿ ಇಡುವುದಿಲ್ಲ’. ಎದ್ದೇಳಿ, ಎಚ್ಚರಗೊಳ್ಳಿ! ಸಮಯ ಕಳೆಯುತ್ತಿದೆ. ನಮ್ಮ ಶಕ್ತಿಯೆಲ್ಲ ಬರೀ ಕಾಡುಹರಟೆಯಲ್ಲಿ ವ್ಯಕ್ತವಾಗುತ್ತಿದೆ. ಸಣ್ಣ ಪುಟ್ಟ ಜಗಳಗಳೂ ವಾದಗಳೂ ಕೊನೆಗಾಣಲಿ. ಏಕೆಂದರೆ ನಮ್ಮ ಮುಂದೆ ಒಂದು ಮಹಾಕಾರ್ಯ ನಿಂತಿದೆ. ಮುಳುಗುತ್ತಿರುವ ಲಕ್ಷೋಪಲಕ್ಷ ಜನರನ್ನು ಮೇಲೆತ್ತಿ ಉದ್ಧಾರ ಮಾಡುವ ಮಹಾ ಸೇವಾಕಾರ್ಯ… ” 

 “ಎಲ್ಲಿಯವರೆಗೆ ನಿಮ್ಮ ಹೃದಯದಲ್ಲಿ ಕನಿಕರ ಇರುವುದಿಲ್ಲವೋ, ಎಲ್ಲಿಯವರೆಗೆ ವೇದಗಳು ಬೋಧಿಸುವಂತೆ ಎಲ್ಲರೂ ನಿಮ್ಮ ಆತ್ಮದ ಅಂಶಗಳೆಂದು, ಶ‍್ರೀಮಂತರು ದರಿದ್ರದು ಮಹರ್ಷಿಗಳು ಪಾಪಿಗಳು ಎಲ್ಲರೂ ಅನಂತ ಬ್ರಹ್ಮದ ಅಂಶಗಳೆಂದು ತಿಳಿಯುವುದಿಲ್ಲವೊ ಅಲ್ಲಿಯವರೆಗೆ ನಿಮ್ಮ ಸಿದ್ಧಾಂತ ಬೋಧೆಯೂ ಮತಪ್ರಚಾರವೂ ಕೆಲಸಕ್ಕೆ ಬಾರವು.” 

 ಲಾಹೋರಿನಲ್ಲಿ ಹಿಂದೂಗಳಲ್ಲೆ ಎರಡು ವಿರೋಧ ಪಕ್ಷಗಳಿದ್ದವು. ಅವರೇ ಆರ್ಯ ಸಮಾಜದವರು ಮತ್ತು ಸನಾತನ ಹಿಂದೂಗಳು. ಸ್ವಾಮೀಜಿಯವರು ಕೆಲವು ವಿಷಯಗಳಲ್ಲಿ ಆರ್ಯ ಸಮಾಜದವರನ್ನು ಒಪ್ಪದೇ ಇದ್ದರೂ ಅವರಿಗೆ ವಿರೋಧವಾಗಿ ಬಹಿರಂಗದಲ್ಲಿ ಎಲ್ಲಿಯೂ ಮಾತನಾಡಲಿಲ್ಲ. ಸನಾತನ ಹಿಂದೂಗಳು ಸ್ವಾಮೀಜಿಯವರನ್ನು ‘ಶ್ರಾದ್ಧ’ ಆಚಾರದ ಮೇಲೆ ಮಾತನಾಡುವಂತೆ ಕೇಳಿಕೊಂಡರು. ಆರ್ಯಸಮಾಜದವರು ಇದನ್ನು ನಂಬುವುದಿಲ್ಲ. ಆದರೆ ಸ್ವಾಮೀಜಿ ಶ್ರಾದ್ಧಾದಿ ಆಚಾರವನ್ನು ಕುರಿತು ಮಾತನಾಡುತ್ತಿರುವಾಗಲೂ ಪಿತೃಪೂಜೆ ಮುಂಚೆ ಬಂದಿತೆಂತಲೂ ಅನಂತರ ದೇವರುಗಳ ಪೂಜೆ ಬಂದಿತೆಂತಲೂ ಹೇಳಿ, ಹಿಂದೂಧರ್ಮದಲ್ಲಿ ಇದಕ್ಕೆ ಇರುವ ಸ್ಥಾನವನ್ನು ವಿವರಿಸಿದರೆ ಹೊರತು ಅದನ್ನು ನಂಬದೆ ಇರುವವರನ್ನು ಖಂಡಿಸುವುದಕ್ಕೆ ಹೋಗಲಿಲ್ಲ. ಸ್ವಾಮೀಜಿಯವರು ಎರಡೂ ಗುಂಪಿನವರಿಗೂ ರಾಜಿಮಾಡಲು ಯತ್ನಿಸಿದರು. ಅವರ ಉಪನ್ಯಾಸಕ್ಕೆ ಆರ್ಯರು ಮತ್ತು ಹಿಂದೂಗಳು ಇಬ್ಬರೂ ಬರುತ್ತಿದ್ದರು. ಆರ್ಯಸಮಾಜದಲ್ಲಿ ಕೆಲವರು ಸ್ವಾಮೀಜಿಯವರನ್ನು ಅವರ ಸಮಾಜಕ್ಕೆ ಮುಖ್ಯಸ್ಥರನ್ನಾಗಿ ಮಾಡಿಕೊಳ್ಳಬೇಕೆಂತಲೂ ಇದ್ದರು. ಆದರೆ ಅದು ಫಲಕಾರಿಯಾಗಲಿಲ್ಲ. ಪಂಜಾಬಿನ ಜನರಲ್ಲಿ ಭಾವ ಮತ್ತು ಭಕ್ತಿಯ ಪ್ರದರ್ಶನ ಕಡಿಮೆ. ಭಜನ ಪ್ರಾರ್ಥನೆ ಮುಂತಾದುವನ್ನು ಅಲ್ಲಿಯ ಜನ ಸ್ವಲ್ಪ ಅಭ್ಯಾಸ ಮಾಡಬೇಕೆಂಬ ತಮ್ಮ ಇಚ್ಛೆಯನ್ನು ವ್ಯಕ್ತಪಡಿಸಿದರು. 

 ಸಾಮೀಜಿಯವರು ಲಾಹೋರಿನಲ್ಲಿದ್ದಾಗಲೇ ತೀರ್ಥರಾಮ ಗೋಸ್ವಾಮಿಯವರನ್ನು ಕಂಡರು. ಅವರು ಆಗ ಲಾಹೋರಿನ ಒಂದು ಕಾಲೇಜಿನಲ್ಲಿ ಗಣಿತಶಾಸ್ತ್ರದ ಪ್ರಾಧ್ಯಾಪಕರಾಗಿದ್ದರು. ಕೆಲವು ಕಾಲದ ಮೇಲೆ ಅವರು ಸಂನ್ಯಾಸವನ್ನು ತೆಗೆದುಕೊಂಡು ಸ್ವಾಮಿ ರಾಮತೀರ್ಥರಾದರು. ಅನಂತರ ಇವರು ವೇದಾಂತವನ್ನು ಇಂಡಿಯಾ ಮತ್ತು ಅಮೇರಿಕಾ ದೇಶಗಳಲ್ಲಿ ಪ್ರಚಾರಮಾಡಿ ಪ್ರಖ್ಯಾತರಾದರು. ಅವರ ನೇತೃತ್ವದಲ್ಲಿಯೇ ಲಾಹೋರಿನ ವಿದ್ಯಾರ್ಥಿಗಳು ಸ್ವಾಮೀಜಿಯವರ ಉಪನ್ಯಾಸವನ್ನು ಕಾಲೇಜಿನಲ್ಲಿ ಅಣಿಮಾಡಿದರು. ಅವರು ಸ್ವಾಮೀಜಿಯವರನ್ನು ಬಹಳ ಗೌರವಿಸಿದರು. ಅವರು ಮತ್ತು ಅವರ ಗುರುಭಾಯಿ ಮತ್ತು ಗುಡ್‍ವಿನ್ ಅವರನ್ನು ತಮ್ಮ ಮನೆಗೆ ಊಟಕ್ಕೆ ಕರೆದರು. ಊಟವಾದ ಮೇಲೆ ಸ್ವಾಮೀಜಿ “ರಾಮನಿರುವ ಕಡೆ ಕಾಮವಿಲ್ಲ, ಕಾಮವಿರುವ ಕಡೆ ರಾಮನಿಲ್ಲ” ಎಂಬ ಹಾಡನ್ನು ಹಾಡಿದರು. ಸ್ವಾಮೀಜಿಯವರ ಅದ್ಭುತವಾದ ಶಾರೀರದ ಮೂಲಕ ಹಾಡಿನ ಭಾವ ಕುಳಿತು ಕೇಳಿದವರಿಗೆಲ್ಲ ಹೃದಯ ಸ್ಪರ್ಶಿಯಾಗುವಂತೆ ಇತ್ತು. ಅವರ ಲೈಬ್ರರಿಯನ್ನೆಲ್ಲ ಸ್ವಾಮೀಜಿಯವರ ವಶಕ್ಕೆ ಬಿಟ್ಟು ತಾವು ಯಾವ ಪುಸ್ತಕವನ್ನು ಬೇಕಾದರೂ ಅದರಿಂದ ತೆಗೆದುಕೊಳ್ಳಬಹುದೆಂದು ಹೇಳಿದರು. ಸ್ವಾಮೀಜಿ ಆ ಗ್ರಂಥಗಳಲ್ಲಿ ವಾಲ್ಟ್ ವಿಟಮನ್ ಅವರ \enginline{Leaves of Grass} ಎಂಬ ಒಂದು ಪುಸ್ತಕವನ್ನು ತೆಗೆದುಕೊಂಡರು. ಈತ ಅಮೇರಿಕಾ ಸಂನ್ಯಾಸಿ ಎಂದು ಹೇಳಿದರು. 

 ತೀರ್ಥರಾಮರಿಗೂ ಸ್ವಾಮೀಜಿಯವರಿಗೂ ಪರಿಚಯ ತುಂಬಾ ನಿಕಟವಾಯಿತು. ತೀರ್ಥರಾಮರು ಸ್ವಾಮೀಜಿಯವರಿಗೆ ಒಂದು ಚಿನ್ನದ ಗಡಿಯಾರವನ್ನು ಕೊಟ್ಟರು. ಸ್ವಾಮೀಜಿ ಅದನ್ನು ತೆಗೆದುಕೊಂಡು ಪುನಃ ತೀರ್ಥರಾಮರ ಜೇಬಿನೊಳಗಿಟ್ಟು ತಾನು ಅದನ್ನು ಅವರ ಮೂಲಕ ಅನುಭವಿಸುತ್ತೇನೆ ಎಂದರು. 

 ಲಾಹೋರಿನಲ್ಲಿ ಸ್ವಾಮೀಜಿ ಹೋದ ಸಮಯದಲ್ಲಿ ಪ್ರೊಫೆಸರ್ ಬೋಸ್ ಅವರ ಸರ್ಕಸ್ ಪಾರ್ಟಿ ಬಂದಿತ್ತು. ಆ ಸರ್ಕಸ್ಸಿನ ಮ್ಯಾನೇಜರ್ ಆದ ಮೋತಿಲಾಲ್ ಬೋಸ್ ಎಂಬುವರು ಸ್ವಾಮೀಜಿಯವರ ಬಾಲ್ಯ ಸ್ನೇಹಿತರು. ಅವರು ಜನ ಸಾಮೀಜಿಯವರಿಗೆ ತೋರುತ್ತಿದ್ದ ಗೌರವವನ್ನು ನೋಡಿ ಆಶ್ಚರ್ಯಚಕಿತನಾದರು. ಆತ ಸ್ವಾಮೀಜಿಯವರನ್ನು ನೋಡುವುದಕ್ಕೆ ಬಂದಾಗ ಅವರನ್ನು “ನಿಮ್ಮನ್ನು ಈಗ ನರೇನ್ ಎಂದು ಕರೆಯಲೆ ಅಥವಾ ಸ್ವಾಮೀಜಿ ಎಂದು ಕರೆಯಲೆ?” ಎಂದು ಕೇಳಿದರು. 

 ಅದಕ್ಕೆ ಸ್ವಾಮೀಜಿ “ಮೋತಿ, ನಿನಗೆ ಏನು ಹುಚ್ಚು ಹಿಡಿದಿರುವುದೇನು? ನಾನು ಅದೇ ನರೇನ್ ನೀನು ಅದೇ ಮೋತಿ ಎಂಬುದು ಗೊತ್ತಿಲ್ಲವೆ?” ಎಂದರು. ಸ್ವಾಮೀಜಿ ತಮ್ಮ ಬಾಲ್ಯ ಸ್ನೇಹಿತರೊಂದಿಗೆ ತಮಗೆ ಎಷ್ಟೇ ಕೀರ್ತಿ ಗೌರವಗಳು ಬಂದಿದ್ದರೂ ಅದೇ ಹಳೆಯ ಸ್ನೇಹಿತನಂತೆಯೇ ವರ್ತಿಸುತ್ತಿದ್ದರು. ಇದರಂತೆಯೇ ಸ್ವಾಮೀಜಿ ಕಲ್ಕತ್ತೆಯಲ್ಲಿದ್ದಾಗ, ‘ನೋಡು ಮುಂದೆ ನಾನು ಏನಾಗುತ್ತೇನೆ’ ಎಂದು ಉಪೇನ್ ಬಾಬು ಎಂಬ ಸ್ನೇಹಿತರಿಗೆ ಹೇಳಿದ್ದರು. ಸ್ವಾಮೀಜಿ ಕಲ್ಕತ್ತೆಯಲ್ಲಿ ರೂಮಿನಲ್ಲಿದ್ದಾಗ ಬಂದೊಡನೆಯೆ ಸ್ವಾಮೀಜಿ ಎದ್ದು ಬಂದು ಅವರನ್ನು ಅಪ್ಪಿಕೊಂಂಡರು. 

 ಸ್ವಾಮೀಜಿ ಲಾಹೋರಿನಲ್ಲಿ ಹತ್ತು ದಿನಗಳಿದ್ದಾದ ಮೇಲೆ ಅವರ ಆರೋಗ್ಯ ಹದಗೆಟ್ಟಿತು. ಆದಕಾರಣ ಅವರು ಅಲ್ಲಿಂದ ಡೆಹರಾಡೂನಿಗೆ ಹೋದರು. ಅಲ್ಲಿ ಸುಮಾರು ಹತ್ತು ದಿನಗಳಿದ್ದರು. ಆದರೂ ಸುಮ್ಮನೆ ಏನೂ ಇರಲಿಲ್ಲ. ಸ್ವಾಮೀಜಿ ಅಲ್ಲಿ ತಮ್ಮ ಗುರುಭಾಯಿಗಳಿಗೆ ರಾಮಾನುಜರ ಬ್ರಹ್ಮಸೂತ್ರ ಭಾಷ್ಯದ ಮೇಲೆ ಪ್ರವಚನವನ್ನು ನಡೆಸುತ್ತಿದ್ದರು. ಅಲ್ಲಿಂದ ಮುಂದಕ್ಕೆ ಬೇರೆ ಊರಿಗೆ ಹೋಗುವಾಗಲೂ ಬಿಡುವಾದಾಗಲೆಲ್ಲ ಈ ಪ್ರವಚನವನ್ನು ಬಿಡದೆ ನಡೆಸುತ್ತಿದ್ದರು. ಸಾಂಖ್ಯತತ್ತ್ವದ ಮೇಲೂ ಪ್ರವಚನವನ್ನು ಪ್ರಾರಂಭಿಸಿ ತಮ್ಮ ಶಿಷ್ಯರಾದ ಅಚ್ಯುತಾನಂದರಿಗೆ ಅದನ್ನು ಬೋಧಿಸುವಂತೆ ಹೇಳಿದರು. ಅಚ್ಯುತಾನಂದರು ದೊಡ್ಡ ವಿದ್ವಾಂಸರಾಗಿದ್ದರೂ ಕೆಲವು ಸೂತ್ರಗಳಿಗೆ ಸರಿಯಾದ ಅರ್ಥವನ್ನು ಹೇಳುವುದಕ್ಕೆ ಆಗುತ್ತಿರಲಿಲ್ಲ. ಆಗ ಸ್ವಾಮೀಜಿ ಅದನ್ನು ವಿವರಿಸುತ್ತಿದ್ದರು. 

 ಖೇತ್ರಿ ಮಹಾರಾಜರು ಸ್ವಾಮೀಜಿಯವರಿಗೆ ತಮ್ಮ ರಾಜ್ಯಕ್ಕೆ ಬರಬೇಕೆಂದು ಪದೇ ಪದೇ ನಿಮಂತ್ರಣವನ್ನು ಕಳುಹಿಸುತ್ತಿದ್ದರು. ಸ್ವಾಮೀಜಿಯವರ ಬೋಧನೆಯನ್ನು ತಮ್ಮ ಪ್ರಜೆಗಳೆಲ್ಲ ಕೇಳಲಿ ಎಂಬುದು ಅವರ ಆಸೆ. ಅದೂ ಅಲ್ಲದೆ ತಾವು ಕೂಡ ಖುದ್ದಾಗಿ ಸ್ವಾಮೀಜಿಯವರನ್ನು ತಮ್ಮ ಊರಿನಲ್ಲಿ ಗೌರವಿಸಬೆಕೆಂದು ತುಂಬಾ ಆಸೆ ಇತ್ತು. ಸ್ವಾಮೀಜಿ ಡೆಹರಾಡೂನಿನಿಂದ ರಾಜಪುತ್ರ ಸ್ಥಾನದ ಕಡೆ ಹೊರಟರು. ದಾರಿಯಲ್ಲಿ ದೆಹಲಿಗೆ ಹೋದರು. ಅಲ್ಲಿ ನಟಕೃಷ್ಣ ಎಂಬ ಒಬ್ಬ ಗೃಹಸ್ಥ ಭಕ್ತನ ಮನೆಯಲ್ಲಿದ್ದರು. ತಮ್ಮ ಪರಿವ್ರಾಜಕ ದಿನಗಳಲ್ಲಿ ಹತ್ರಾಸಿನಲ್ಲಿ ಅವರನ್ನು ಸ್ವಾಮೀಜಿ ಕಂಡಿದ್ದರು. ದೆಹಲಿಯ ಅನೇಕ ಶ‍್ರೀಮಂತರು ತಮ್ಮ ಮನೆಯಲ್ಲಿ ಸ್ವಾಮೀಜಿಯವರು ಅತಿಥಿಗಳಾಗಿ ಬಂದಿರಬೇಕೆಂದು ಕೋರಿದರೂ ಅವರು ಒಪ್ಪಲಿಲ್ಲ. ಡೆಲ್ಲಿಯಲ್ಲಿ ಅನೇಕ ಪ್ರಸಿದ್ಧ ವಿದ್ವಾಂಸರೊಡನೆ ಸಣ್ಣಪುಟ್ಟ ಮಾತುಕತೆಗಳೂ ನಡೆದವು. ಸ್ವಾಮೀಜಿ ತಮ್ಮ ಗುರುಭಾಯಿಗಳು ಮತ್ತು ಸೇವಿಯರ್ಸ್‍‍ ದಂಪತಿಗಳನ್ನು ದೆಹಲಿಯಲ್ಲಿರುವ ಪ್ರಸಿದ್ಧವಾದ ಸ್ಥಳಗಳಿಗೆಲ್ಲ ಕರೆದುಕೊಂಡು ಹೋಗಿ ತೋರಿಸಿ ಅದನ್ನು ವಿವರಿಸಿದರು. 

 ಡೆಲ್ಲಿಯಿಂದ ಸ್ವಾಮೀಜಿ ಆಳ್ವಾರಿಗೆ ಹೋದರು. ಆಳ್ವಾರಿನಲ್ಲಿ ಮಹಾರಾಜರಿಗೆ ಸೇರಿದ ಒಂದು ಮನೆಯಲ್ಲಿ ಸ್ವಾಮೀಜಿ ಮತ್ತು ಅವರ ಪರಿವಾರದವರೆಲ್ಲರೂ ಇದ್ದರು. ಆ ಸಮಯದಲ್ಲಿ ಮಹಾರಾಜರು ರಾಜಧಾನಿಯಲ್ಲಿ ಇರಲಿಲ್ಲ. ಆಸ್ಥಾನದ ಅನೇಕ ಉನ್ನತ ಅಧಿಕಾರಿಗಳೊಡನೆ ಸಂಭಾಷಣೆ ಮಾಡಿದರು. ಆ ಊರಿನಲ್ಲಿ ಸ್ವಾಮೀಜಿ ತಾವು ಹಿಂದೆ ಪರಿವ್ರಾಜಕರಾಗಿ ಬಂದಿದ್ದಾಗ ಯಾರ ಮನೆಯಲ್ಲಿದ್ದರೋ ಅವರನ್ನು ಕಂಡುಹಿಡಿದರು. ಸ್ವಾಮೀಜಿ ಅವರನ್ನು ಸ್ವಾಗತಿಸಲು ಅನೇಕ ದೊಡ್ಡ ಮನುಷ್ಯರು ರೈಲ್ವೆ ನಿಲ್ದಾಣದಲ್ಲಿ ಸತ್ತುವರಿದಿದ್ದಾಗ ದೂರದಲ್ಲಿ ಇವರ ಹತ್ತಿರ ಬರಲು ಒಬ್ಬ ಅಂಜಿ ನಿಂತಿದ್ದನು. ಸ್ವಾಮೀಜಿ ತಕ್ಷಣವೇ ಅವನ ಗುರುತನ್ನು ಹಿಡಿದು ‘ರಾಮಸ್ನೇಹಿ’ ಎಂದು ಕೂಗಿ ಅವನನ್ನು ತಮ್ಮ ಮುಂದಕ್ಕೆ ಕರೆದುಕೊಂಡು ಬರುವಂತೆ ಹೇಳಿ ಅವನ ಕುಶಲ ಪ್ರಶ್ನೆಗಳನ್ನು ಮಾಡಿದರು. ಆ ಊರಿನಲ್ಲಿ ಒಂದು ದಿನ ಒಬ್ಬ ಮುದುಕಿಯ ಮನೆಯಲ್ಲಿ ಊಟವನ್ನು ಮಾಡಿದ್ದರು ತಾವು ಹಿಂದೆ ಪರಿವ್ರಾಜಕರಾಗಿ ಬಂದಿದ್ದಾಗ. ಈಗ ಸ್ವಾಮೀಜಿ ಆಕೆಯ ಮನೆಗೆ ತಾವೇ ಊಟಕ್ಕೆ ಬರುತ್ತೇವೆಂದೂ ಅವಳು ಮಾಡಿದ ದಪ್ಪನಾದ ಚಪಾತಿ ರುಚಿಯಾಗಿದ್ದಿತೆಂದೂ ಹೇಳಿ ಕಳುಹಿಸಿದರು. ಆ ಹೆಂಗಸಾದರೋ ಹಿಗ್ಗಿ ಹೋದಳು, ಇಂತಹ ಜಗದ್ವಿಖ್ಯಾತರಾದ ಸ್ವಾಮೀಜಿ ತಮ್ಮ ಮನೆಗೆ ಊಟಕ್ಕೆ ಬರುತ್ತಾರಲ್ಲ ಎಂದು. ತಾನು ಅಡಿಗೆಯನ್ನು ಮಾಡಿ ಬಡಿಸುವಾಗ ಆನಂದಬಾಷ್ಪವನ್ನು ಸುರಿಸುತ್ತ ಸ್ವಾಮೀಜಿಗೆ “ಮಗು, ನನ್ನಂತಹ ದರಿದ್ರ ಹೆಂಗಸು ನಿನಗೆ ರುಚಿಕರವಾದ ತಿಂಡಿಯನ್ನು ಹೇಗೆ ತಾನು ಮಾಡಿ ಹಾಕುವುದು” ಎಂದು ಹೇಳಿದಳು. ಸ್ವಾಮೀಜಿ ಊಟವನ್ನು ಅಮೃತದಂತೆ ಇತ್ತು ಎಂದು ಮೆಚ್ಚಿದರು. ಆಕೆಯು ಮಾಡಿದ ಚಪಾತಿ ಎಂತಹ ಸಾತ್ತ್ವಿಕ ಆಹಾರ ಎಂದು ಅನಂತರ ತಮ್ಮ ಗುರುಭಾಯಿಗಳಿಗೆ ಹೇಳುತ್ತಿದ್ದರು. ಊಟವಾದ ಮೇಲೆ ಸ್ವಾಮೀಜಿ ಆ ಹೆಂಗಸಿನ ನೆಂಟರೊಬ್ಬರ ಕೈಯಲ್ಲಿ ಅವಳ ಖರ್ಚಿಗೆ ಎಂದು ಸ್ವಲ್ಪ ಹಣವನ್ನು ಕೊಟ್ಟರು. ಅವರು ಬೇಡವೆಂದು ಹೇಳಿದರೂ ಕೇಳದೆ ಸ್ವಾಮೀಜಿ ಅದನ್ನು ಬಲಾತ್ಕಾರಿಸಿ ಸ್ವೀಕರಿಸುವಂತೆ ಮಾಡಿದರು. ಸ್ವಾಮೀಜಿ ಅನಂತರ ಖೇತ್ರಿಗೆ ಹೋದರು. 

 ಖೇತ್ರಿ ಮಹಾರಾಜರು ತಾವೇ ಗಾಡಿ ಮುಂತಾದುವನ್ನು ಮಾಡಿಕೊಂಡು ಜಯಪುರದಲ್ಲಿ ಸ್ವಾಮೀಜಿಯವರನ್ನು ಸ್ವಾಗತಿಸಲು ಬಂದರು. ಅಲ್ಲಿಂದ ಖೇತ್ರಿಗೆ ವೈಭವದಿಂದ ಕರೆದುಕೊಂಡು ಹೋದರು. ಖೇತ್ರಿಯ ಜನರೆಲ್ಲ ಆನಂದಭರಿತರಾಗಿದ್ದರು. ಖೇತ್ರಿ ಮಹಾರಾಜರು ಆಗ ತಾನೆ ಇಂಗ್ಲೆಂಡಿಗೆ ಹೋಗಿ ಬಂದಿದ್ದರು. ಸ್ವಾಮಿಜಿಯವರು ಕೂಡ ಪಾಶ್ಚಾತ್ಯ ದೇಶಗಳಿಂದ ಬಂದಿದ್ದರು. ಅವರಿಬ್ಬರನ್ನೂ ಸ್ವಾಗತಿಸುವುದಕ್ಕೆ ಊರೆಲ್ಲ ಅಲಂಕೃತವಾಗಿತ್ತು. ಬಾಣಬಿರುಸುಗಳನ್ನು ಬಿಟ್ಟರು. ಇಬ್ಬರಿಗೂ ಪುರಜನರು ಬಿನ್ನವತ್ತಳೆಗಳನ್ನು ಅರ್ಪಿಸಿದರು. ಡಿಸೆಂಬರ್ ೧೧ನೇ ತಾರೀಖು ಪುರಜನರು ರಾಜರಿಗೆ ಮತ್ತು ಸ್ವಾಮೀಜಿಗಳಿಗೆ ಒಂದು ಸನ್ಮಾನಪತ್ರವನ್ನು ಒಪ್ಪಿಸಿದರು. ಆಗ ದೊಡ್ಡ ಒಂದು ದರ್ಬಾರು ನಡೆಯಿತು. ಆ ದರ್ಬಾರಿನಲ್ಲಿ ಪುರಜನರು ಐದು ತಟ್ಟೆಗಳ ತುಂಬ ಚಿನ್ನದ ಮೊಹರುಗಳನ್ನು ರಾಜರಿಗೆ ಕೊಟ್ಟರು. ರಾಜರು ಅದನ್ನು ತಮ್ಮ ದೇಶದ ವಿದ್ಯಾಸಂಸ್ಥೆಗೆ ಹಂಚಿದರು. ಅನಂತರ ಅಧಿಕಾರಿಗಳೂ ಪ್ರಜೆಗಳೂ ಎಲ್ಲರೂ ಬಂದು ಸ್ವಾಮೀಜಿಯವರಿಗೆ ನಮಸ್ಕರಿಸಿ ಪ್ರತಿಯೊಬ್ಬರೂ ಎರಡೆರಡು ರೂಪಾಯಿಯಂತೆ ಕಾಣಿಕೆಯನ್ನು ಕೊಟ್ಟರು. ಸ್ವಾಮೀಜಿ ಖೇತ್ರಿಯನ್ನು ಬಿಟ್ಟು ಹೋಗುವ ಸಮಯದಲ್ಲಿ ರಾಜರು ಅವರಿಗೆ ಮೂರು ಸಾವಿರ ರೂಪಾಯಿಗಳನ್ನು ಕೊಟ್ಟರು. ಹಣವನ್ನೆಲ್ಲ ಬೆಲೂರು ಮಠಕ್ಕೆ ಕಳುಹಿಸಿದರು. 

 ಡಿಸೆಂಬರ್ ೨೦ನೇ ತಾರೀಖು ಸ್ವಾಮೀಜಿ ತಾವು ಇಳಿದುಕೊಂಡಿದ್ದ ಬಂಗಲೆಯಲ್ಲಿ ವೇದಾಂತದ ಮೇಲೆ ಒಂದು ಉಪನ್ಯಾಸವನ್ನು ಮಾಡಿದರು. ಊರಿನ ಗಣ್ಯರು ಮತ್ತು ಐರೋಪ್ಯರು ಬಂದಿದ್ದರು. ಸ್ವಾಮೀಜಿಯವರು ಇಲ್ಲಿ ತಮ್ಮ ಗುರುಭಾಯಿಗಳು ಮತ್ತು ರಾಜರೊಂದಿಗೆ ಹತ್ತಿರವಿದ್ದ ಹಲವು ಇತಿಹಾಸ ಪ್ರಸಿದ್ಧವಾದ ಸ್ಥಳಗಳನ್ನು ನೋಡಿಕೊಂಡು ಬರುತ್ತಿದ್ದರು. ಒಂದು ದಿನ ರಾಜರು ಮತ್ತು ಸ್ವಾಮೀಜಿ ಕುದುರೆಯ ಮೇಲೆ ಒಂದು ಸಣ್ಣ ದಾರಿಯಲ್ಲಿ ಹೋಗುತ್ತಿದ್ದರು. ದಾರಿಯ ಪಕ್ಕದಲ್ಲಿ ಇದ್ದ ಕುರುಚಲ ಗಿಡದ ಕೊಂಬೆ ಒಂದು ಅಡ್ಡಲಾಗಿತ್ತು. ರಾಜರು ಕುದುರೆಯಿಂದ ಇಳಿದು ಸ್ವಾಮೀಜಿಯವರ ಕುದುರೆ ಹೋಗುವ ತನಕ ಅಡ್ಡಲಾಗಿದ್ದ ಕೊಂಬೆಯನ್ನು ಹಿಡಿದುನಿಂತರು. ಸ್ವಾಮೀಜಿ ರಾಜರಿಗೆ, ತಾವು ಏತಕ್ಕೆ ಆ ಕೆಲಸವನ್ನು ಮಾಡಬೇಕು ಎಂದು ಕೇಳಿದರು. ಅದಕ್ಕೆ ರಾಜರು “ಧರ್ಮರಕ್ಷಣೆ ಕ್ಷತ್ರಿಯರ ಕರ್ತವ್ಯವಾಗಿದೆ” ಎಂದರು. 

 ಅನಂತರ ಸ್ವಾಮೀಜಿ ಕೃಷ್ಣಗರ್, ಅಜ್ಮೀರ್, ಜೋದ್‍ಪುರ, ಇಂದೋರ್ ಮೂಲಕ ಖಂಡ್ವಾಕ್ಕೆ ಹೋದರು. ಸ್ವಾಮೀಜಿ ಜೋದ್‍ಪುರದಲ್ಲಿದ್ದಾಗ ರಾಜ ಸರ್ ಪ್ರತಾಪಸಿಂಗ ಎಂಬ ಪ್ರಧಾನ ಮಂತ್ರಿಯ ಮನೆಯಲ್ಲಿ ಹತ್ತು ದಿನಗಳು ಇದ್ದರು. ಖಂಡ್ವಾದಲ್ಲಿ ಹರಿದಾಸ ಜಟರ್ಜಿ ಎಂಬುವರ ಮನೆಯಲ್ಲಿ ಇದ್ದರು. ಅಲ್ಲಿ ಒಂದು ದಿನ ಜ್ವರ ಬಂದು ಅನಂತರ ಗುಣಮುಖರಾದರು. ಒಂದು ವಾರ ಖಂಡ್ವಾದಲ್ಲಿದ್ದು ಸ್ವಾಮೀಜಿ ಕಲ್ಕತ್ತೆಗೆ ಬಂದಮೇಲೆ ಸಂಘಕ್ಕೆ ಸಂಬಂಧಪಟ್ಟ ಹಲವು ಕಾರ್ಯಕಲಾಪಗಳಲ್ಲಿ ಮಗ್ನರಾದರು. 

