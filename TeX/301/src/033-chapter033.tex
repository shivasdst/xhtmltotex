
\chapter{ದಕ್ಷಿಣದ ಊರುಗಳು}

 ಜನವರಿ ೨೬ನೇ ಮಂಗಳವಾರ ಸ್ವಾಮೀಜಿ ಮತ್ತು ಅವರ ವೃಂದದವರು ಸ್ಟೀಮರಿನಲ್ಲಿ ಇಂಡಿಯಾದೇಶಕ್ಕೆ ಹೊರಟರು. ಅಲ್ಲಿಂದ ಸುಮಾರು ಐವತ್ತು ಮೈಲಿಗಳು ಸಮುದ್ರದ ಮೇಲೆ ಪಾಂಬನ್‌ಗೆ ಪ್ರಯಾಣ ಚೆನ್ನಾಗಿತ್ತು. ಸುಮಾರು ಮಧ್ಯಾಹ್ನ ಮೂರುಗಂಟೆ ಹೊತ್ತಿಗೆ ಸ್ಟೀಮರ್ ಪಾಂಬನ್ ರೋಡ್ ಎಂಬಲ್ಲಿಗೆ ತಲುಪಿತು. ಸ್ವಾಮೀಜಿ ರಾಮನಾಡಿನ ರಾಜರ ಕೋರಿಕೆಯಂತೆ ರಾಮೇಶ್ವರಕ್ಕೆ ಹೋಗುವುದರಲ್ಲಿದ್ದರು. ಆಗ ರಾಮನಾಡಿನ ರಾಜರೆ ಸ್ವಾಮೀಜಿಯವರನ್ನು ಸಂಧಿಸಲು ಪಾಂಬನ್‌ಗೆ ಬಂದಿರುವರೆಂದು ಗೊತ್ತಾಯಿತು. ಅನಂತರ ಸ್ಟೀಮರನ್ನು ಬಿಟ್ಟು ರಾಮನಾಡಿನ ರಾಜರ ದೋಣಿಗೆ ಹೋದರು. ಸ್ವಾಮೀಜಿಯವರು ದೋಣಿಯಲ್ಲಿ ಪ್ರವೇಶಿಸಿದ ಒಡನೆ ಎಲ್ಲರೂ ಅವರಿಗೆ ಸಾಷ್ಟಾಂಗ ಪ್ರಣಾಮಗಳನ್ನು ಮಾಡಿದರು. ರಾಮನಾಡಿನ ರಾಜರೆ ಹಿಂದೆ ಸ್ವಾಮೀಜಿಯವರನ್ನು ಅಮೇರಿಕಾದೇಶಕ್ಕೆ ಹೋಗುವಂತೆ ಹೇಳಿ ಅದಕ್ಕೆ ದ್ರವ್ಯ ಸಹಾಯ ಮಾಡಿದ್ದರು. ಈಗ ಇಂಡಿಯಾದೇಶವನ್ನು ಮುಟ್ಟುತ್ತಲೇ ಆ ರಾಜರೇ ಪ್ರಥಮದಲ್ಲಿ ನೋಡುವುದು ಸೂಕ್ತವಾಗಿದೆ ಎಂದರು ಸ್ವಾಮೀಜಿ. ರಾಜರು ಮತ್ತು ಸ್ವಾಮೀಜಿ ಕುಳಿತಿದ್ದ ದೋಣಿ ತೀರದಲ್ಲಿ ಸಂಚರಿಸಿಕೊಂಡು ಪಾಂಬನ್ ಊರನ್ನು ಸೇರಿದಾಗ ಅಲ್ಲಿಯ ಜನರೆಲ್ಲ ಅಭೂತಪೂರ್ವ ಸ್ವಾಗತವನ್ನು ನೀಡಿದರು. ಅಲ್ಲಿ ಅಲಂಕೃತವಾದ ಚಪ್ಪರದೊಳಗೆ ಸ್ವಾಮೀಜಿಯವರಿಗೆ ಬಿನ್ನವತ್ತಳೆಯನ್ನು ಅರ್ಪಿಸಿದರು. 

 ಸ್ವಾಮೀಜಿ ತಮ್ಮ ಉತ್ತರದಲ್ಲಿ, ಭರತಖಂಡದ ಆದರ್ಶ ರಾಜನೀತಿಯಲ್ಲ, ಸೈನ್ಯ ಶಕ್ತಿಯಲ್ಲ, ಆರ್ಥಿಕ ಅಥವಾ ಯಾಂತ್ರಿಕ ಪರಮಾಧಿಕಾರವೂ ಅಲ್ಲ. ಧರ್ಮ ಒಂದೇ ಅಧ್ಯಾತ್ಮವೊಂದೇ ಭಾರತೀಯರ ವಿಧಿಯಾಗಿತ್ತು ಎಂದರು. ವಿಜ್ಞಾನ ಮತ್ತು ಯಂತ್ರಗಳ ಮೂಲಕ ಅದ್ಭುತ ಕಾರ್ಯಗಳಾಗಿವೆ. ಆದರೆ ಮಾನವನ ಶೀಲವನ್ನು ತಿದ್ದುವುದರಲ್ಲಿ\break ಅಧ್ಯಾತ್ಮ ಸಹಾಯ ಮಾಡುವಂತೆ ಮತ್ತಾವುದೂ ಮಾಡಲಾರದು. ಜನ ಹಿಂದೂಗಳನ್ನು ಸೋಮಾರಿಗಳು, ಅಂಜುಕುಳಿಗಳು ಎಂದು ಹಾಸ್ಯಮಾಡುವರು. ಆದರೆ ತಾವು ಅದನ್ನು ಒಪ್ಪುವುದಿಲ್ಲವೆಂದೂ ಹಿಂದೂಗಳು ಯಾವಾಗಲೂ ಧಾರ್ಮಿಕ ಜಗತ್ತಿನಲ್ಲಿ ಕಾರ್ಯೋನ್ಮುಖರಾಗಿರುವರೆಂದೂ ಧರ್ಮಕ್ಕಾಗಿ ಎಂತಹ ಧೈರ್ಯವನ್ನು ಬೇಕಾದರೂ ತೋರಿರುವರೆಂದೂ ಹೇಳಿದರು. ಇತರ ಕಡೆ ತಮ್ಮ ಧರ್ಮವೇ ಶ್ರೇಷ್ಠ ಧರ್ಮ, ತಮ್ಮದೇ ಜಗತ್ತನ್ನೆಲ್ಲ ಉದ್ಧರಿಸಬಲ್ಲದು ಎಂಬ ಸಂಕುಚಿತ ದೃಷ್ಟಿಯಲ್ಲಿ ಬದ್ಧರಾಗಿರುವರು. ಆದರೆ ಭರತಖಂಡವಾದರೊ ಸತ್ಯ ಯಾವ ಧರ್ಮದ ಮೀಸಲೂ ಅಲ್ಲ, ಅದು ಎಲ್ಲದರಲ್ಲಿಯೂ ಇದೆ ಎಂದು ಹೇಳಿರುವುದು. ಆ ಉದಾರಭಾವನೆ ಭರತಖಂಡಕ್ಕೆ ಮಾತ್ರ ಸೇರಿದುದು. ಇಂದು ಪಾಶ್ಚಾತ್ಯ ದೇಶಗಳು ಆಧ್ಯಾತ್ಮಿಕ ಸಮಸ್ಯೆಗಳನ್ನು ಪರಿಹರಿಸಿಕೊಳ್ಳಲು ನಮ್ಮೆಡೆಗೆ ಬರುವರು. ಅವುಗಳನ್ನು ನಿರ್ವಹಿಸಲು ಸಿದ್ಧರಾಗಬೇಕು. ಇತರ ದೇಶಗಳಲ್ಲಿ ತಾವು ಪಾಳೆಗಾರನೋ ದರೋಡೆಗಾರನೋ ಆದ ಕುಲಕ್ಕೆ ಸೇರಿದವರೆಂದು ಹೆಮ್ಮೆ ತಾಳುವರು. ಆದರೆ ಇಂಡಿಯಾ ದೇಶದಲ್ಲಿ ಕಾಡುಗಳಲ್ಲಿ ಗೆಡ್ಡೆ ಗೆಣಸುಗಳನ್ನು ತಿಂದು ತಪಸ್ಸು ಮಾಡುತ್ತಿದ್ದ ಋಷಿಗಳ ಕುಲಕ್ಕೆ ತಾವು ಸೇರಿದವರು ಎಂದು ಹೇಳಿಕೊಳ್ಳಲು ಸಂತೋಷಪಡುವರು. ಕೊನೆಗೆ ಸ್ವಾಮೀಜಿ ರಾಮನಾಡಿನ ರಾಜರ ಸಂಬಂಧವಾಗಿ, “ನನ್ನಿಂದ ಮತ್ತು ನನ್ನ ಮೂಲಕ ಏನಾದರೂ ಒಳ್ಳೆಯದಾಗಿದ್ದರೆ ಭರತಖಂಡ ಈ ವ್ಯಕ್ತಿಗೆ ಹೆಚ್ಚು ಋಣಿ” ಎಂದು ಅವರನ್ನು ಕೊಂಡಾಡಿದರು. 

 ಬಿನ್ನವತ್ತಳೆಯನ್ನು ಅರ್ಪಿಸಿ ಆದಮೇಲೆ ಸ್ವಾಮೀಜಿಯವರನ್ನು ಸುಂದರವಾದ ಅರಮನೆಯ ಕುದುರೆಯ ಗಾಡಿಯಲ್ಲಿ ಕೂರಿಸಿ ಮೆರವಣಿಗೆಯಲ್ಲಿ ಕರೆದುಕೊಂಡು ಹೋದರು. ಸ್ವಲ್ಪ ದೂರ ಹೋದಮೇಲೆ ಜನರ ಕೋರಿಕೆಯಂತೆ ಗಾಡಿಯಿಂದ ಕುದುರೆಯನ್ನು ತೆಗೆದುಹಾಕಿದರು. ಜನರೇ ಸ್ವಾಮೀಜಿ ಕುಳಿತಿದ್ದ ಗಾಡಿಯನ್ನು ಎಳೆದುಕೊಂಡು ಹೋದರು. ರಾಮನಾಡಿನ ರಾಜರು ಕೂಡ ಸ್ವಾಮೀಜಿ ಕುಳಿತಿದ್ದ ಗಾಡಿಯನ್ನು ಎಳೆದವರಲ್ಲಿ ಒಬ್ಬರಾಗಿದ್ದರು. ಸ್ವಾಮೀಜಿ ಮೂರು ದಿನಗಳು ಪಾಂಬನ್ ನಗರದಲ್ಲಿ ಇದ್ದರು. ರಾಜರಿಗೆ ಮತ್ತು ಪ್ರಜೆಗಳಿಗೆ ಜಗದ್‌ವಿಜಯಿ ವಿವೇಕಾನಂದರು ತಮ್ಮ ಮಧ್ಯದಲ್ಲಿರುವುದು ಅತ್ಯಂತ ಸಂತೋಷದಾಯಕವಾಗಿತ್ತು. 

 ಪಾಂಬನ್ನಿಗೆ ಬಂದ ಮಾರನೆ ದಿನವೆ ಸ್ವಾಮೀಜಿ ರಾಮೇಶ್ವರ ದೇವಸ್ಥಾನಕ್ಕೆ ಹೊರಟರು. ಸ್ವಾಮೀಜಿ ಅಲ್ಲಿಗೆ ಹೋಗುತ್ತಿದ್ದಾಗ ಸುಮಾರು ಐದು ವರ್ಷಗಳ ಹಿಂದೆ ಅನಾಮಧೇಯರಾಗಿ ಪಾದಚಾರಿಗಳಾಗಿ ಆ ದೇವಸ್ಥಾನಕ್ಕೆ ಬಂದ ನೆನಪು ಅವರಿಗೆ ಬಂದಿತು. ಈಗ ಅವರು ರಾಜಮರ‍್ಯಾದೆಯೊಡನೆ ಗಾಡಿಯಲ್ಲಿ ಬರುತ್ತಿರುವರು. ಇನ್ನೇನು ದೇವಸ್ಥಾನವನ್ನು ಸಮೀಪಿಸುತ್ತಿರುವಾಗ ದೇವಸ್ಥಾನದ ಮರ‍್ಯಾದೆಯೊಡನೆ ಸ್ವಾಮೀಜಿ ಅವರನ್ನು ಸ್ವಾಗತಿಸಲಾಯಿತು. ಕುದುರೆ, ಆನೆ, ಒಂಟೆ, ಸಂಗೀತ, ನಾದಸ್ವರಗಳೆಲ್ಲವೂ ಇದ್ದುವು. ಸ್ವಾಮೀಜಿಯವರನ್ನು ಬರಮಾಡಿಕೊಳ್ಳುವುದಕ್ಕೆ ದೇವರ ನಗ ನಾಣ್ಯಗಳನ್ನು ತೋರಿಸಿದರು. ರಾಮೇಶ್ವರ ದೇವಸ್ಥಾನದ ಸಾವಿರ ಕಂಬಗಳುಳ್ಳ\break ವಿಶಾಲವಾದ ಪ್ರಾಂಗಣ ಮತ್ತು ಗರ್ಭಗುಡಿಯಲ್ಲಿ ನೆಲಸಿರುವ ರಾಮೇಶ್ವರ ಶಿವನ ದರ್ಶನ ಮಾಡಿ ಆದಮೇಲೆ ಅಲ್ಲಿ ನೆರೆದ ಅನೇಕ ಜನ ಸ್ವಾಮೀಜಿಯವರನ್ನು ದೇವಸ್ಥಾನದ ಒಳಗೆ ಏನಾದರೂ ಮಾತನಾಡಬೇಕೆಂದು ಕೇಳಿದರು. ಸ್ವಾಮೀಜಿ ನಿಜವಾದ ಪೂಜೆ ಎಂಬುದರ ಮೇಲೆ ಒಂದು ಸಣ್ಣ ಉಪನ್ಯಾಸ ಕೊಟ್ಟರು. ಅದನ್ನು ಸಾಧಾರಣ ಜನರ ತಿಳುವಳಿಕೆಗೆಂದು ತಮಿಳು ಭಾಷೆಗೆ ಇತರರು ಭಾಷಾಂತರಮಾಡಿ ಹೇಳಿದರು. ಸ್ವಾಮೀಜಿ ಈ ಉಪನ್ಯಾಸದಲ್ಲಿ ನಿಜವಾದ ಪೂಜೆ ಎಂದರೆ ಏನು ಎಂಬುದನ್ನು ವಿವರಿಸಿದರು. ಧರ್ಮ ಇರುವುದು ಪ್ರೀತಿಯಲ್ಲಿ, ಬಾಹ್ಯಾಚಾರದಲ್ಲಿ ಅಲ್ಲ. ಪರಿಶುದ್ಧ ಹೃದಯದಿಂದ ಪ್ರಾರ್ಥಿಸಿದರೆ ಶಿವ ಒಲಿಯುವನು. ಬಾಹ್ಯಾಡಂಬರಕ್ಕೆ ಅವನು ಒಲಿಯುವವನಲ್ಲ ಎಂದರು. ತೀರ್ಥಸ್ಥಳದಲ್ಲಿ ವಾಸಮಾಡುವುದು ಕಷ್ಟ ಎಂದರು. ಏಕೆಂದರೆ ಮಿಕ್ಕ ಸ್ಥಳದಲ್ಲಿ ಪಾಪ ಮಾಡಿದರೆ ಅದಕ್ಕೆ ಪ್ರಾಯಶ್ಚಿತ್ತವಿದೆ. ತೀರ್ಥಕ್ಷೇತ್ರದಲ್ಲಿ ಪಾಪ ಮಾಡುವುದು ಮಹಾಪಾತಕವಾಗಿ ಪರಿಣಮಿಸುವುದು. ಆದಕಾರಣ ಅಲ್ಲಿರುವವರು ಜೋಪಾನವಾಗಿರಬೇಕು ಎಂದರು. ನಾವು ಶುದ್ಧರಾಗಿರುವುದು, ಇತರರಿಗೆ ಒಳ್ಳೆಯದನ್ನು ಮಾಡುವುದು ಇದೇ ಎಲ್ಲಾ ಪೂಜೆಯ ಸಾರ. ಯಾರು ದೀನರಲ್ಲಿ ದುರ್ಬಲರಲ್ಲಿ ರೋಗಿಗಳಲ್ಲಿ ಶಿವನನ್ನು ನೋಡುವರೊ ಅವರೇ ನಿಜವಾಗಿ ಶಿವನನ್ನು ಪೂಜಿಸುವವರು. ನಿಸ್ವಾರ್ಥತೆಯೆ ಧರ್ಮದ ಪರೀಕ್ಷೆ. ಯಾರಲ್ಲಿ ಈ ನಿಸ್ವಾರ್ಥತೆ ಹೆಚ್ಚು ಇದೆಯೊ ಅವನು ಹೆಚ್ಚು ಧಾರ್ಮಿಕ. ಅವನು ದೇವರ ಸಮೀಪದಲ್ಲಿರುವನು. ಪಂಡಿತನಾಗಲಿ ಪಾಮರನಾಗಲಿ, ಅವನಿಗೆ ಗೊತ್ತಿರಲಿ ಇಲ್ಲದೆ ಇರಲಿ ಅವನು ಇತರರಿಗಿಂತ ಹೆಚ್ಚು ದೇವರ ಸಮೀಪದಲ್ಲಿರುವನು. ಸ್ವಾರ್ಥಿಯಾಗಿದ್ದರೆ ಅವನು ಎಲ್ಲಾ ದೇವಸ್ಥಾನಗಳನ್ನು ನೋಡಿದ್ದರೂ ಶಿವನಿಗೆ ಬಹಳ ದೂರದಲ್ಲಿರುವನು ಎಂದರು. 

 ರಾಮನಾಡಿನ ರಾಜರು ಸ್ವಾಮೀಜಿಯವರಿಗೆ ಪರಮಪ್ರಿಯವಾದ ದರಿದ್ರ ನಾರಾಯಣ ಸೇವೆಯನ್ನು ಮಾರನೆ ದಿನ ಮಾಡಿದರು. ಸಹಸ್ರಾರು ಜನರಿಗೆ ಭೋಜನ ಮತ್ತು ಬಟ್ಟೆಯನ್ನು ಕೊಟ್ಟರು. ಅನಂತರ ಪಾಂಬನ್ನಿಗೆ ಹೊರಟರು. ಪಾಂಬನ್ನಿನಲ್ಲಿ ಸ್ವಾಮೀಜಿ ಮೊದಲು ಇಂಡಿಯಾ ದೇಶಕ್ಕೆ ಕಾಲಿಟ್ಟ ಸ್ಥಳದಲ್ಲಿ ರಾಮನಾಡಿನ ರಾಜರು ನಲವತ್ತು ಅಡಿ ಎತ್ತರ ಉಳ್ಳ ಒಂದು ಸ್ಮಾರಕಸ್ತಂಭವನ್ನು ಕಟ್ಟಿ ಅದರಲ್ಲಿ ಹೀಗೆ ಬರೆಸಿದರು: 

 “ ‘ಸತ್ಯಮೇವ ಜಯತೇ’. ಈ ಸ್ಥಳದಲ್ಲಿ ಪರಮ ಪೂಜ್ಯ ಸ್ವಾಮೀಜಿ ವಿವೇಕಾನಂದರು, ವೇದಾಂತ ಸಂದೇಶವನ್ನು ಪಾಶ್ಚಾತ್ಯ ಜಗತ್ತಿಗೆ ಬೋಧಿಸಿ ಅದ್ವಿತೀಯ ದಿಗ್ವಿಜಯಾನಂತರ ತಮ್ಮ ಆಂಗ್ಲೇಯ ಶಿಷ್ಯರೊಡನೆ ಬಂದು ಭರತಖಂಡದಲ್ಲಿ ಅವರ ಪವಿತ್ರ ಪಾದಗಳು ಮೊಟ್ಟಮೊದಲು ಸ್ಪರ್ಶಿಸಿದ ಸ್ಥಳ. ಇದನ್ನು ಚಿರಸ್ಮರಣೀಯವನ್ನಾಗಿ ಮಾಡುವ ಉದ್ದೇಶದಿಂದ ಈ ಸ್ಮೃತಿಸ್ತಂಭವನ್ನು ರಾಮನಾಡಿನ ರಾಜ ಭಾಸ್ಕರವರ್ಮ ಸೇತುಪತಿಗಳು ಕ್ರಿ.ಶ. ೧೮೯೭ನೇ ಜನವರಿ ೨೬ನೆಯ ದಿನ ನಿರ್ಮಾಣ ಮಾಡಿದರು.” 

 ಮೂರು ದಿನಗಳ ಅನಂತರ ಸ್ವಾಮೀಜಿ ಪಾಂಬನ್ನಿನಿಂದ ರಾಮನಾಡಿಗೆ ಗಾಡಿಯಲ್ಲಿ ಹೊರಟರು. ದಾರಿಯಲ್ಲಿ ರಾಜರು ಕಟ್ಟಿದ್ದ ಒಂದು ಛತ್ರದಲ್ಲಿ ಊಟ ಮಾಡಿ ಅಲ್ಲಿಂದ ಮುಂದುವರಿದು ತಿರುಪ್ಪುಲಾನಿ ಎಂಬ ಊರನ್ನು ಸೇರಿದರು. ಅಲ್ಲಿಯ ಜನರು ಸ್ವಾಮೀಜಿಯವರನ್ನು ಬರಮಾಡಿಕೊಂಡರು. ಅನಂತರ ಮುಂದುವರಿದು ಸಂಜೆಯ ಸಮಯಕ್ಕೆ ರಾಮನಾಡಿನ ಹತ್ತಿರ ಬಂದರು. ಅಲ್ಲಿ ಗಾಡಿಯಿಂದಿಳಿದು ಅವರಿಗಾಗಿ ಕಾಯುತ್ತಿದ್ದ ಅರಮನೆಯ ಅಲಕೃಂತ ದೋಣಿಯನ್ನು ಹತ್ತಿದರು. ಆ ದೋಣಿ ಹತ್ತಿರದ ಸರೋವರದ ಮೂಲಕ ತೇಲಿಹೋಗಿ ರಾಮನಾಡನ್ನು ತಲುಪಿತು. 

 ರಾಮನಾಡಿನ ರಾಜರು ಸ್ವಾಮೀಜಿಯವರನ್ನು ಅಲ್ಲಿ ಸ್ವಾಗತಿಸಿದರು. ಅನಂತರ ದೇಶದ ಪ್ರಮುಖರ ಪರಿಚಯವನ್ನು ಸ್ವಾಮೀಜಿಗೆ ಮಾಡಿಸಿದರು. ಅಂದಿನ ರಾತ್ರಿ ಬೇಕಾದಷ್ಟು ಬಾಣ ಬಿರುಸು ತಾರಾಮಂಡಲಗಳನ್ನು ಜನ ಸ್ವಾಮೀಜಿ ಜ್ಞಾಪಕಾರ್ಥವಾಗಿ ಹಾರಿಸಿದರು. ಸ್ವಾಮೀಜಿ ರಾಜಮರ‍್ಯಾದೆಯೊಡನೆ ಅವರ ಅಂಗರಕ್ಷಕರನ್ನೊಡಗೂಡಿ, ರಾಜರ ತಮ್ಮನ ನೇತೃತ್ವದಲ್ಲಿ ಮೆರವಣಿಗೆ ಹೊರಟರು. ರಾಜರು ತಾವು ಪಾದಚಾರಿಗಳಾಗಿಯೇ ಮೆರವಣಿಗೆಯ ಮೇಲ್ವಿಚಾರಣೆಯನ್ನು ನೋಡಿಕೊಳ್ಳುತ್ತ ಹೋಗುತ್ತಿದ್ದರು. ಮೆರವಣಿಗೆಯ ಎಡ ಬಲಗಳಲ್ಲಿ ಬೇಕಾದಷ್ಟು ದೀಪಗಳ ಸಾಲು, ಕರ್ಣಾಟಕ ಸಂಗೀತದ ತಾಳಮೇಳಗಳು ಮುಂದೆ ಸಾಗಿತು. ಇಂಗ್ಲೀಷ್‌ಬ್ಯಾಂಡ್ \enginline{“See the Conquering Hero Comes”} ಎಂಬ ಹಾಡನ್ನು ಹಾಡಿದರು. ಅರ್ಧದೂರ ಗಾಡಿಯಲ್ಲಿ ಮೆರವಣಿಗೆ ಆದಮೇಲೆ ರಾಜರು ಸ್ವಾಮೀಜಿಯವರನ್ನು ಸಾಲಂಕೃತ ಪಲ್ಲಕಿಯೊಳಗೆ ಕುಳಿತುಕೊಳ್ಳಬೇಕೆಂದು ಕೋರಿಕೊಂಡರು. ಅದರಲ್ಲಿ ವಿಜೃಂಭಣೆಯಿಂದ ಮುಂದುವರಿದು “ಶಂಕರವಿಲ್ಲ” ಎಂಬ ರಾಜರ ಅರಮನೆಯನ್ನು ಸೇರಿದರು. 

 ಕೆಲವು ಕಾಲ ಅಲ್ಲಿ ವಿಶ್ರಾಂತಿಯನ್ನು ಪಡೆದು ಅರಮನೆಯ ದೊಡ್ಡ ದರ್ಬಾರ್ ಹಾಲಿಗೆ ಹೋದರು. ಸ್ವಾಮೀಜಿಯವರಿಗೆ ಕೊಡುವ ಬಿನ್ನವತ್ತಳೆಗೆ ಉತ್ತರವಾಗಿ ಅವರು ಭಾಷಣವನ್ನು ಮಾಡುತ್ತಾರೆ ಎಂದು ಸಹಸ್ರಾರು ಮಂದಿ ಅಲ್ಲಿ ನೆರೆದಿದ್ದರು. ಸ್ವಾಮೀಜಿ ದರ್ಬಾರು ಹಾಲಿಗೆ ಕಾಲಿಟ್ಟೊಡನೆಯೆ ನೆರೆದ ಸಹಸ್ರಾರು ಜನ ಎದ್ದುನಿಂತು ಹರ್ಷದಿಂದ ಕರತಾಡನಗಳನ್ನು ಮಾಡಿದರು. ಸ್ವಾಮೀಜಿ ಕುಳಿತು ಕೊಂಡಮೇಲೆ ಪ್ರಾರ್ಥನಾದಿಗಳು ಆದ ನಂತರ ರಾಮನಾಡಿನ ರಾಜರು ಸ್ವಾಮೀಜಿಯವರು ಮಾಡಿರುವ ಕಾರ‍್ಯವನ್ನು ಶ್ಲಾಘಿಸಿದರು. ಅನಂತರ ಅವರ ತಮ್ಮ ರಾಜ ದಿನಕರ ಸೇತುಪತಿ ಬಿನ್ನವತ್ತಳೆಯನ್ನು ಓದಿ ಅದನ್ನು ಒಂದು ಸುಂದರವಾದ ಕೆತ್ತನೆಯ ಕೆಲಸ ಮಾಡಿದ ಚಿನ್ನದ ಕರಂಡದಲ್ಲಿಟ್ಟು ಸ್ವಾಮೀಜಿಯವರಿಗೆ ಅರ್ಪಿಸಿದರು. ಸ್ವಾಮೀಜಿ ಅನಂತರ ಅದಕ್ಕೆ ಹೀಗೆ ಉತ್ತರ ಕೊಟ್ಟರು: 

\vskip 2pt

 “ಸುದೀರ್ಘ ರಾತ್ರಿ ಕಡೆಗಿಂದು ಕೊನೆಗಾಣುತ್ತಿದೆ. ಬಹುಕಾಲದ ಶೋಕ ತಾಪಗಳು ಕಡೆಗಿಂದು ಮಾಯವಾಗುತ್ತಿವೆ. ಇದುವರೆಗೆ ಶವದಂತೆ ಇದ್ದ ಶರೀರವಿಂದು ಸಚೇತನವಾಗುತ್ತಿದೆ. ಅದೋ ಕಿವಿಗೊಡಿ! ವಾಣಿಯೊಂದು ಕೇಳಿ ಬರುತ್ತಿದೆ. ಬಹು ಪುರಾತನ ಕಾಲದ ಗರ್ಭದಿಂದ ಹೊಮ್ಮಿ, ಪರ್ವತಶಿಖರಗಳಿಂದ ಮರುದನಿಯಾಗಿ ಚಿಮ್ಮಿ, ಅರಣ್ಯಾರಣ್ಯಗಳ ಕಂದರ ಕಂದರಗಳಲ್ಲಿ ಸಂಚರಿಸಿ, ಬರುಬರುತ್ತಾ ಪ್ರಬಲವಾಗಿ, ಬಂದಂತೆಲ್ಲ ಅಪ್ರತಿಹತವಾಗಿ, ನಮ್ಮೀ ಪುಣ್ಯಭೂಮಿಯನ್ನು ನಿದ್ದೆಯಿಂದ ಒದೆದು ಎಬ್ಬಿಸಿ, ಜ್ಞಾನ ಭಕ್ತಿ ವೈರಾಗ್ಯದ ಸೇವಾತತ್ತ್ವಗಳನ್ನು ಉಚ್ಚಕಂಠದಿಂದ ಸಾರುವ ತೂರ‍್ಯವಾಣಿಯೊಂದು ಕೇಳಿಬರುತ್ತಿದೆ. ಹಿಮಾಲಯದಿಂದ ಬೀಸಿ ಬರುವ ಪುಣ್ಯ ಸಮೀರಣದಂತೆ ನಿರ್ಜೀವದಂತಿದ್ದ ಅಸ್ಥಿ ಮಾಂಸಗಳಿಗೆ ಜೀವದಾನ ಮಾಡುತ್ತಿದೆ. ಜಡನಿದ್ರೆಯನ್ನು ಪರಿಹರಿಸುತ್ತಿದೆ. ಕಾರ್ಯೋತ್ಸಾಹ ಸ್ಥೈರ‍್ಯ ಧೈರ‍್ಯಗಳನ್ನು ಉದ್ರೇಕಿಸುತ್ತಿದೆ. ಕುರುಡರಿಗೆ ಕಾಣದು, ಮೂರ್ಖರಿಗೆ ತಿಳಿಯದು. ನಮ್ಮೀ ಭಾರತ ಭೂಮಿ ಯುಗಯುಗಗಳ ನಿದ್ರೆಯಿಂದ ಮೇಲೇಳುತ್ತಿರುವಳು. ಆಕೆಯನ್ನು ಇನ್ನು ಯಾರೂ ತಡೆಯಬಲ್ಲವರಿಲ್ಲ. ಇನ್ನು ಆಕೆ ನಿದ್ರೆ ಮಾಡುವುದಿಲ್ಲ. ಯಾವ ಶಕ್ತಿಯೂ ಆಕೆಯನ್ನು ಬಗ್ಗಿಸಲಾರದು. ಏಕೆಂದರೆ ಅದೋ ನೋಡಿ! ಮಹಾಕಾಳಿ ಮತ್ತೊಮ್ಮೆ ಎಚ್ಚೆತ್ತು ಮೈಕೊಡಹಿ ಉಸಿರೆಳೆದು ನಿಲ್ಲುತ್ತಿರುವಳು.” 

\vskip 2pt

 ಭಾರತದ ಪ್ರಾಣವೇ ಧರ್ಮ; ಅದು ಹೋದರೆ ಸರ್ವನಾಶವಾದಂತೆ, ಅದಿಲ್ಲದೆ ರಾಜನೀತಿ ಸಾಮಾಜಿಕ ಪ್ರಗತಿ ಮತ್ತು ಪ್ರತಿಯೊಬ್ಬ ಭಾರತೀಯನಿಗೂ ಕುಬೇರನ ಐಶ್ವರ‍್ಯ ಇವುಗಳಿದ್ದರೂ ಪ್ರಯೋಜನವಿಲ್ಲ ಎಂಬ ಭಾವನೆ ಇಲ್ಲಿಯ ಜನರಲ್ಲಿ ಮೂಡಿದ್ದರೆ, ಹಾಗೆಯೇ ಮೂಡುವಂತೆ ಹೊರಗಿನ ದೇಶದಲ್ಲಿ ತಾವು ಪ್ರಚಾರ ಮಾಡಿದ್ದರೆ, ಅದಕ್ಕೆಲ್ಲ ರಾಮನಾಡಿನ ದೊರೆಗಳೇ ಕಾರಣಕರ್ತರು ಎಂದು ತುಂಬಿದ ಸಭೆಯಲ್ಲಿ ಸ್ವಾಮೀಜಿ ಸಾರಿದರು. ವಿಶ್ವದ ರಾಷ್ಟ್ರಗಳ ಸಾಮರಸ್ಯದಲ್ಲಿ ಪ್ರತಿಯೊಂದು ದೇಶವೂ ಒಂದು ರಾಗದಂತೆ ಇದೆ. ಇದೇ ಅದರ ಜೀವಾಳ, ಸಾರ, ಇದೇ ಅದರ ಬೆನ್ನೆಲುಬು, ಅದರ ರಾಷ್ಟ್ರ ಜೀವನದ ತಳಹದಿ. ಈ ಪವಿತ್ರ ಭೂಮಿಯಲ್ಲಿ ಈ ಬೆನ್ನೆಲುಬು, ಈ ತಳಹದಿ, ಜೀವನದ ಕೇಂದ್ರವೇ ಧರ್ಮ. ಭರತಖಂಡ ವಿಶ್ವಕ್ಕೆ ಇದನ್ನು ಧಾರೆಯೆರೆಯಬೇಕಾಗಿದೆ, ಅದಕ್ಕಾಗಿಯೇ ಇದು ಇನ್ನೂ ಬದುಕಿರುವುದು ಎಂದರು. 

 ‘ವಿಷಯಾನ್ ವಿಷವತ್ ತ್ಯಜ’, ಎಂಬುದೇ ಭಾರತೀಯ ಶಾಸ್ತ್ರದ ಸಾರ. ಆದರೆ ನಾವು ಪಾಶ್ಚಾತ್ಯರಿಂದಲೂ ವಿಜ್ಞಾನಶಾಸ್ತ್ರ, ಸಂಘಟನಾ ರೀತಿನೀತಿಗಳು, ಅಧಿಕಾರವನ್ನು ಉಪಯೋಗಿಸುವ ರೀತಿ, ಅತ್ಯಲ್ಪ ಉಪಾದಾನಗಳಿಂದ ಮಹಾಫಲವನ್ನು ಪಡೆಯುವ ಕೌಶಲ್ಯ ಇವುಗಳನ್ನು ಕಲಿಯಬೇಕು. ಆದರೆ ಇದಕ್ಕಾಗಿ ನಮ್ಮ ಜೀವನದ ಮುಖ್ಯ ಆದರ್ಶವನ್ನು ಬಲಿಕೊಡಕೂಡದು. ದೇಶದಲ್ಲಿ ಅನೇಕರಿಗೆ ಈ ಪ್ರಪಂಚವನ್ನು ಅನುಭವಿಸಬೇಕೆಂಬ ಹಂಬಲವಿದೆ. ಅವರು ಅದನ್ನು ಧಾರ್ಮಿಕವಾಗಿ ಅನುಭವಿಸಲು ಅವಕಾಶ ಕೊಡಬೇಕು. ಬಲಾತ್ಕಾರವಾಗಿ ಅವರನ್ನು ತ್ಯಾಗಿಗಳಾಗಿ ಮಾಡಕೂಡದು. ಭೋಗ ಸಲಕರಣೆಗಳು ಆಹಾರವನ್ನು ಉತ್ಪಾದನೆ ಮಾಡುವುದು ಇವುಗಳನ್ನು ನಾವು ಪಾಶ್ಚಾತ್ಯರಿಂದ ಕಲಿಯಬೇಕು. ಭೋಗದ ಮೂಲಕ ಜೀವಿ ಸಾಗಿ ಹೋಗಬೇಕೆ ಹೊರತು ಅದೇ ಅವನ ಗುರಿ ಎಂದು ಎಂದಿಗೂ ಭಾರತ ಭಾವಿಸಕೂಡದು ಎಂದು ಸ್ವಾಮೀಜಿ ಹೇಳಿದರು. 

 ಭರತಖಂಡದ ಶ್ರೇಯಸ್ಸಿನ ದಾರಿಯಲ್ಲಿ ಎರಡು ಆತಂಕಗಳಿವೆ. ಒಂದು ಪೂರ್ವಾಚಾರ ಪರಾಯಣತೆ; ಮತ್ತೊಂದು ಆಧುನಿಕ ಪಾಶ್ಚಾತ್ಯರ ಅಂಧ ಅನುಕರಣೆ. ಇವುಗಳೆರಡರ ಮಧ್ಯದಲ್ಲಿ ನಾವು ಹೋಗಬೇಕಾಗಿದೆ. ಹಾಗೆ ಹೋಗುವಾಗ ನಮ್ಮ ಶಕ್ತಿಯ ಮೇಲೆ ನಾವು ನಿಂತುಕೊಳ್ಳಬೇಕು. ಈ ಪ್ರಪಂಚದಲ್ಲಿ ಏನಾದರೂ ಪಾತಕವಿದ್ದರೆ ಅದೇ ದೌರ್ಬಲ್ಯ. ಎಲ್ಲಾ ಬಗೆಯ ದೌರ್ಬಲ್ಯದಿಂದ ಪಾರಾಗಬೇಕು. ದೌರ್ಬಲ್ಯವೇ ಮಹಾ ಪಾತಕ, ದೌರ್ಬಲ್ಯವೇ ಮರಣ. 

 ಒಳ್ಳೆಯದು ಎಲ್ಲಿರಲಿ, ಅದನ್ನು ನಾವು ಸ್ವೀಕರಿಸಲು ಸಿದ್ಧರಾಗಬೇಕು. ನಮ್ಮ ಶಾಸ್ತ್ರ ಹೀಗೆ ಹೇಳುತ್ತದೆ: “ಅತಿ ನೀಚನಿಂದಲೂ ಶ್ರದ್ಧೆಯಿಂದ ಒಳ್ಳೆಯ ವಿಷಯವನ್ನು ಕಲಿತುಕೊಳ್ಳಿ, ಮುಕ್ತಿಯ ಮಾರ್ಗ ಒಬ್ಬ ನೀಚನ ಬಾಯಿ ಮೂಲಕ ಬಂದರೂ ಶ್ರದ್ಧಾ ಭಕ್ತಿಗಳಿಂದ ಕಲಿತುಕೊಳ್ಳಿ. ಸ್ತ್ರೀರತ್ನ ಅತಿ ನೀಚ ಕುಲದಿಂದ ಬಂದಿದ್ದರೂ ಅವಳನ್ನು ಸ್ತ್ರೀಯಾಗಿ ಸ್ವೀಕರಿಸಿ.” ಯೋಗ್ಯವಾದುದು ಎಲ್ಲಿಂದಲಾದರೂ ಬರಲಿ ಅದನ್ನು ಸ್ವೀಕರಿಸಬೇಕು. ಅದನ್ನು ಜೀರ್ಣಿಸಿಕೊಂಡು ನಮ್ಮ ಆದರ್ಶಕ್ಕೆ ಸರಿಯಾಗಿ ಅದನ್ನು ಹೊಂದಿಸಿಕೊಳ್ಳಬೇಕು, ಅದನ್ನು ಸುಮ್ಮನೆ ಅಂಧರಾಗಿ ಅನುಕರಿಸುವುದಲ್ಲ. 

 ನಮ್ಮ ಮಾತೃಭೂಮಿಯಲ್ಲಿ ಈಶ್ವರನ ಭಾವನೆ ಸ್ಪಷ್ಟವಾಗಿರುವಷ್ಟು ಅನ್ಯದೇಶಗಳಲ್ಲಿ ಇಲ್ಲ. ಇತರ ಕಡೆ ಇರುವ ದೇವರ ಭಾವನೆ; ಅವನು ಯಾವುದೋ ಒಂದು ಕೋಮಿಗೆ, ದೇಶಕ್ಕೆ, ಮತಕ್ಕೆ ಸೇರಿದವನು, ಎಂದು. ಆದರೆ ಭಾರತೀಯನ ಈಶ್ವರನ ಭಾವನೆ ಭೂಮವಾದುದು. “ಯಾರು ಶೈವನ ಶಿವನೋ, ವೈಷ್ಣವರ ವಿಷ್ಣುವೊ ಮೀಮಾಂಸಕರ ಕರ್ಮವೋ, ಬೌದ್ಧರ ಬುದ್ಧನೋ, ಮಹಮ್ಮದೀಯರ ಅಲ್ಲನೊ, ಯಾರು ಪ್ರತಿಯೊಂದು ಧರ್ಮದ ದೇವರಾಗಿರುವನೋ, ವೇದಾಂತಿಗಳ ಬ್ರಹ್ಮನಾಗಿರುವನೊ, ಅವನು ಸರ್ವವ್ಯಾಪಿಯಾಗಿರುವನು. ಅವನ ಮಹಾತ್ಮೆಯನ್ನು ಈ ದೇಶಮಾತ್ರ ತಿಳಿದಿರುವುದು. ಈ ಆದರ್ಶವನ್ನು ಅನುಷ್ಠಾನದಲ್ಲಿ ತರುವುದಕ್ಕೆ ಅವನು ನಮ್ಮನ್ನು ಆಶೀರ್ವದಿಸಲಿ, ಸಹಾಯ ಮಾಡಲಿ, ಶಕ್ತಿ ನೀಡಲಿ ಎಂದು ಬಿನ್ನವತ್ತಳೆಗೆ ಉತ್ತರವನ್ನು ನೀಡಿದರು. 

 ಸ್ವಾಮೀಜಿ ರಾಮನಾಡಿಗೆ ಬಂದ ಜ್ಞಾಪಕಾರ್ಥವಾಗಿ ಮದ್ರಾಸಿನ ಕ್ಷಾಮ ನಿವಾರಣಾ ಸಮಿತಿಗೆ ಪುರಜನರು ಸಹಾಯವನ್ನು ಮಾಡಬೇಕೆಂದು ರಾಜರು ಕೋರಿದರು. ಸ್ವಾಮೀಜಿ ರಾಮನಾಡಿನಲ್ಲಿದ್ದಾಗ ಹಲವರೊಡನೆ ಮಾತುಕತೆಯಾಡಿದರು. ಒಂದು ಕ್ರೈಸ್ತ ಮಿಷನರಿಯ ಸ್ಕೂಲಿನ ಹಾಲಿನಲ್ಲಿ ಬಹಿರಂಗ ಉಪನ್ಯಾಸ ಮಾಡಿದರು. ಅರಮನೆಯವರು ಸ್ವಾಮೀಜಿ ಜ್ಞಾಪಕಾರ್ಥವಾಗಿ ನೆರವೇರಿಸಿದ ದರ್ಬಾರಿನಲ್ಲಿ ಇದ್ದರು. ಅಂದಿನ ದರ್ಬಾರಿನಲ್ಲಿ ಪುರಜನರು ತಮಿಳು ಮತ್ತು ಸಂಸ್ಕೃತದಲ್ಲಿ ಸ್ವಾಮೀಜಿಯವರಿಗೆ ಮತ್ತೊಂದು ಬಿನ್ನವತ್ತಳೆಯನ್ನು ಅರ್ಪಿಸಿದರು. ಸ್ವಾಮೀಜಿ ಅದಕ್ಕೆ ಸೂಕ್ತವಾದ\break ಉತ್ತರವನ್ನು ಕೊಟ್ಟರು. ರಾಮನಾಡಿನ ರಾಜರಿಗೆ ‘ರಾಜರ್ಷಿ’ ಎಂಬ ಹೆಸರನ್ನು ಆ ತುಂಬಿದ ಸಭೆಯಲ್ಲಿ ಸ್ವಾಮೀಜಿ ದಯಪಾಲಿಸಿದರು. ರಾಜರ ಕೋರಿಕೆಯಂತೆ ಸ್ವಾಮೀಜಿ ಶಕ್ತಿಪೂಜೆಯ ಮೇಲೆ ಕೊಟ್ಟ ಭಾಷಣವನ್ನು ಫೋನೋಗ್ರಾಮಿನಲ್ಲಿ ರಿಕಾರ್ಡ್ ಮಾಡಲಾಯಿತು. 

 ರಾಮನಾಡಿನಿಂದ ಸ್ವಾಮೀಜಿ ಪರಮಕುಡಿಗೆ ಹೊರಟರು. ಅಲ್ಲಿಯೂ ಜನರು ಸ್ವಾಮಿಜಿಗೆ ಒಂದು ಬಿನ್ನವತ್ತಳೆಯನ್ನು ಅರ್ಪಿಸಿದರು. ಸ್ವಾಮೀಜಿ ಅದಕ್ಕೆ ಸೂಕ್ತವಾಗಿ ಕೆಳಕಂಡಂತೆ ಉತ್ತರವಿತ್ತರು: 

 ಸ್ವಾಮೀಜಿ ಭರತಖಂಡದಿಂದ ಪಾಶ್ಚಾತ್ಯ ದೇಶಗಳಿಗೆ ಸಂದೇಶವನ್ನು ಇತ್ತ ಮೊದಲಿಗರು. ತಮ್ಮ ಅನಂತರ ಹಲವು ಜನ ಪ್ರಪಂಚದ ಮೂಲೆ ಮೂಲೆಗೆ ಹೋಗಿ ಸನಾತನ ಭಾರತ ಸಂದೇಶವನ್ನು ಪ್ರಚಾರಮಾಡುವ ಸುದಿನ ಕಾದಿದೆ ಎಂದು ಆಶಿಸಿದರು. ಪ್ರಪಂಚದಲ್ಲಿ ಯಾವಾಗಲೂ ಎರಡು ಆದರ್ಶಗಳು ಎದ್ದು ನಿಂತಿವೆ. ಇದೇ ಅಧ್ಯಾತ್ಮ ಮತ್ತು ಇಹಲೋಕದ ಉನ್ನತಿ. ಕೆಲವು ವೇಳೆ ಒಂದು ಮುಂದಿರುವುದು, ಮತ್ತೆ ಕೆಲವು ವೇಳೆ ಇನ್ನೊಂದು ಮುಂದೆ ಇರುವುದು. ಭರತಖಂಡದಲ್ಲೇ ಇಂತಹ ಪರಸ್ಪರ ವಿರೋಧವಾದ ಎರಡು ಆದರ್ಶಗಳು ಒಂದಾದಮೇಲೆ ಒಂದರಂತೆ ಬಂದಿವೆ. ಜಡವಾದ ಎಲ್ಲವೂ ನಿರರ್ಥಕವಲ್ಲ. ಒಂದು ಪ್ರಮಾಣದಲ್ಲಿ ಅದರಿಂದಲೂ ಪ್ರಯೋಜನವಾಗಿದೆ ಎಂದರು ಸ್ವಾಮೀಜಿ. ಇದು ವರ್ಣಗಳ ಪ್ರತ್ಯೇಕ ಹಕ್ಕು ಬಾಧ್ಯತೆಗಳನ್ನು ನಾಶಮಾಡಿದೆ. ಇಂಡಿಯಾ ದೇಶದಲ್ಲಿ ಉತ್ತಮ ವರ್ಗದಲ್ಲಿರುವವರು ಮುಖ್ಯವಾದ ತತ್ತ್ವಗಳನ್ನು ಇತರರಿಗೆ ಕೊಡಲಿಲ್ಲ. ತಾವು ಕೂಡ ಅದರಂತೆ ಬಾಳಲಿಲ್ಲ. ಇದು ಧರ್ಮದ ಹೆಸರಿನಲ್ಲಿ ಆಗುತ್ತಿರುವ ಅತ್ಯಾಚಾರ. ಅವುಗಳನ್ನು ಎಲ್ಲರಿಗೂ ದಾನ ಮಾಡಲು ಮುಂದೆ ಬರಬೇಕು. ಉಪನಿಷತ್ತಿನ ಸಿದ್ಧಾಂತವನ್ನು ಸರ್ವಜನರೂ ಗ್ರಹಿಸುವಂತೆ ಮಾಡಬೇಕು. ಅದಾವುದೋ ಕೆಲವು ಕೋಮಿಗೆ ಮೀಸಲಾದುದಲ್ಲ. ಇಡೀ ಮಾನವಕೋಟಿಗೆ ಕಲ್ಯಾಣಪ್ರದವಾಗಿರುವ ತತ್ತ್ವ ಅದರಲ್ಲಿದೆ. ಅವರು ಒಂದು ಆತ್ಮನನ್ನು ನಂಬುವರು. ಜೀವಿಯಲ್ಲಿ ಆಗಲೇ ಜ್ಞಾನ, ಪವಿತ್ರತೆ ಎಲ್ಲಾ ಸುಪ್ತಸ್ಥಿತಿಯಲ್ಲಿವೆ. ಅದನ್ನು ನಾವು ವ್ಯಕ್ತಪಡಿಸಬೇಕು. ಇದನ್ನು ನಾವು ಮರೆತಿರುವೆವು. ಇದೇ ನಮ್ಮ ಅಜ್ಞಾನಕ್ಕೆ ಕಾರಣ. ಅತಿ ಪವಿತ್ರನಾದ ಮಾನವನಿಗೂ ಪಾದದಡಿಯಲ್ಲಿರುವ ಕೀಟಕ್ಕೂ ವ್ಯತ್ಯಾಸವೆಲ್ಲಿರುವುದು? ಆ ಅಜ್ಞಾನದ ಸಾಂದ್ರತೆಯ ತರತಮದಲ್ಲಿದೆ. ಪ್ರತಿಯೊಂದು ಜೀವಿಯಲ್ಲಿಯೂ ಅನಂತ ಶಕ್ತಿ, ಜ್ಞಾನ, ಪವಿತ್ರತೆ ಸುಪ್ತವಾಗಿವೆ, ಅವು ವ್ಯಕ್ತವಾಗಬೇಕು. ಈ ಮಹಾಸತ್ಯವನ್ನು ಭರತಖಂಡ ಪ್ರಪಂಚಕ್ಕೆ ಬೋಧಿಸಬೇಕಾಗಿದೆ. ನಾವು ಶಕ್ತಿಯ ಸಂದೇಶವನ್ನು ರಾಷ್ಟ್ರದಲ್ಲೆಲ್ಲ ಹರಡಬೇಕು. ಶಕ್ತಿಯೇ ಪುಣ್ಯ, ಅಶಕ್ತಿಯೇ ಪಾಪ. ಯಾವುದಾದರೊಂದು ಪದ ಉಪನಿಷತ್ತಿನ ಮಹಾಗಣಿಯಿಂದ, ಸಿಡಿಮದ್ದಿನಂತೆ ಅಜ್ಞಾನಿಗಳ ಮೇಲೆ ಸಿಡಿಯುತ್ತಿದ್ದರೆ ಅದೇ “ಅಭೀಃ” ನಿರ್ಭಯನಾಗು ಎಂಬ ಪದ. ಅಂಜಿಕೆಯೇ ದುಃಖಕ್ಕೆ ಮೂಲ, ಸಾವಿಗೆ ಮೂಲ, ಅನಂತ ಪಾಪಗಳಿಗೆ ಮೂಲ. ಈ ಅಂಜಿಕೆಗೆ ಕಾರಣ ನಮ್ಮ ಸ್ವಭಾವದ ಅಜ್ಞಾನ. 

\vskip 2pt

 ಅಜ್ಞಾನವನ್ನು ಪರಿಹರಿಸಿಕೊಳ್ಳುವುದು ನಮ್ಮ ಕೈಯಲ್ಲಿದೆ ಎನ್ನುವುದು ನಮ್ಮ ಧರ್ಮ. ನಮ್ಮ ಅದೃಷ್ಟಕ್ಕೆ ನಾವೇ ಹೊಣೆ, ಮತ್ತಾರನ್ನೋ ಅದಕ್ಕಾಗಿ ದೂರಿ ಪ್ರಯೋಜನವಿಲ್ಲ ಎಂದು ಹೇಳಿದರು. 

\vskip 2pt

 ಸ್ವಾಮೀಜಿ ಅನಂತರ ಮನಮಧುರೆಗೆ ಹೋದರು. ಅಲ್ಲಿ ಸ್ವಾಮೀಜಿಯವರನ್ನು ವೈಭವಯುತವಾದ ಮೆರವಣಿಗೆಯಲ್ಲಿ ಕರೆದುಕೊಂಡುಹೋಗಿ ಒಂದು ಚಪ್ಪರದಲ್ಲಿ ಮನಮಧುರೆಯ ಪುರಜನರು ಬಿನ್ನವತ್ತಳೆ ಕೊಟ್ಟರು. ಸ್ವಾಮೀಜಿ ಅದಕ್ಕೆ ಉತ್ತರವನ್ನು ಕೊಟ್ಟರು. ತಮಗೆ ಶರೀರಾಲಸ್ಯವಾಗಿರುವುದರಿಂದ ದೀರ್ಘ ಭಾಷಣವನ್ನು ಮಾಡುವ ಸ್ಥಿತಿಯಲ್ಲಿ ತಾವು ಇಲ್ಲ ಎಂದರು. ಭರತಖಂಡದ ಪ್ರಾಣಪಕ್ಷಿ ಧರ್ಮದಲ್ಲಿದೆ ಎಂದರು. ಆದರೆ ನಾವು ಧರ್ಮವನ್ನು ಅನುಷ್ಠಾನದಲ್ಲಿ ಚೆನ್ನಾಗಿ ತಂದಿಲ್ಲ. ನಾವು ನಮ್ಮ ಧರ್ಮಾಮೃತವನ್ನು ಇತರರಿಗೆ ಸರಿಯಾಗಿ ಹಂಚದೆ ಇರುವುದರಿಂದ ಕ್ರೈಸ್ತರು, ಮಹಮ್ಮದೀಯರು ಇವರ ಸಂಖ್ಯೆ ಹೆಚ್ಚಾಗಲು ಕಾರಣವಾಯಿತು. ಮುಖ್ಯ ವಿಷಯಗಳನ್ನು ಬಿಟ್ಟು ಗೌಣ ವಿಷಯಗಳ ಮೇಲೆ ಪ್ರಾಮುಖ್ಯತೆಯನ್ನು ಕೊಡುತ್ತಾ ಬಂದೆವು. ಆ ಸಮಯದಲ್ಲಿ ಸ್ವಾಮೀಜಿ ಹೀಗೆ ಹೇಳಿದರು: “ನೂರಾರು ಮಂದಿ ವಿದ್ಯಾವಂತರು ನಾವು ನೀರನ್ನು ಎಡಗೈಯಿಂದ ಕುಡಿಯುವುದೇ ಬಲಗೈಯಿಂದ ಕುಡಿಯುವುದೇ, ಕೈಯನ್ನು ಮೂರು ಸಲ ತೊಳೆಯುವುದೇ ಅಥವಾ ನಾಲ್ಕು ಸಲ ತೊಳೆಯುವುದೇ, ಬಾಯನ್ನು ಐದು ಸಲ ಮುಕ್ಕಳಿಸುವುದೇ ಅಥವಾ ಆರು ಸಲ ಮುಕ್ಕಳಿಸುವುದೆ ಮುಂತಾದುವನ್ನು ಚರ್ಚಿಸುವುದನ್ನು ನೋಡಿ! ಇಂತಹ ಪ್ರಮುಖ ವಿಷಯವನ್ನು ಚರ್ಚಿಸುತ್ತ, ಇವುಗಳ ಮೇಲೆ ಬೇಕಾದಷ್ಟು ಉದ್‌ಗ್ರಂಥಗಳನ್ನೇ ಬರೆಯುವವರಿಂದ ನೀವು ಮತ್ತೇನನ್ನು ನಿರೀಕ್ಷಿಸಬಲ್ಲಿರಿ? ನಮ್ಮ ಧರ್ಮ ಅಡಿಗೆ ಮನೆಗೆ ಹೋಗುವ ಅಪಾಯವಿದೆ. ಮುಕ್ಕಾಲು ಪಾಲು ಜನ ಈಗ ವೇದಾಂತಿಗಳೂ ಅಲ್ಲ, ಪೌರಾಣಿಕರೂ ಅಲ್ಲ, ಶಾಕ್ತರೂ ಅಲ್ಲ. ನಾವು ‘ಮುಟ್ಟಬೇಡಿ’ ಎನ್ನುವವರು ಆಗಿಹೋಗಿರುವೆವು. ನಮ್ಮ ಧರ್ಮ ಅಡಿಗೆಮನೆಯಲ್ಲಿದೆ. ಅಡಿಗೆ ಮಾಡುವ ಪಾತ್ರೆಯೇ ನಮ್ಮ ದೇವರು. ‘ನಾನು ಮಡಿ, ನನ್ನನ್ನು ಮುಟ್ಟಬೇಡಿ.’ ಇದೇ ನಮ್ಮ ಧರ್ಮವಾಗಿದೆ. ಹೀಗೆ ಮತ್ತೊಂದು ಶತಮಾನ ಕಳೆದರೆ, ನಾವೆಲ್ಲ ಹುಚ್ಚರ ಆಸ್ಪತ್ರೆಯಲ್ಲಿರಬೇಕಾಗುವುದು!” 

 ನಾವು ಬೇಕಾದಷ್ಟು ಅನ್ನದಾನ ವಸ್ತ್ರದಾನಾದಿಗಳನ್ನು ಮಾಡಿರುವೆವು. ಇಂದು ಉಪನಿಷತ್ತಿನ ಅಧ್ಯಾತ್ಮ ವಿದ್ಯೆಯನ್ನು ಎಲ್ಲರಿಗೂ ದಾನಮಾಡಬೇಕಾಗಿದೆ. ಇದು ನಮ್ಮ ಕರ್ತವ್ಯ ಎಂದು ಹೇಳಿದರು. 

 ಸ್ವಾಮೀಜಿಯವರು ಮೀನಾಕ್ಷಿಯ ದೇವಸ್ಥಾನಕ್ಕೆ ಪ್ರಖ್ಯಾತವಾದ ಮಧುರೆಗೆ ಹೋದರು. ಅಲ್ಲಿ ರಾಮನಾಡಿನ ರಾಜರ ಬಂಗಲೆಯಲ್ಲಿ ಇಳಿದುಕೊಂಡರು. ಅಂದಿನ ದಿನ ಸಾಯಂಕಾಲ ಸ್ವಾಮೀಜಿಯವರಿಗೆ ಒಂದು ಬಿನ್ನವತ್ತಳೆಯನ್ನು ಮಕಮಲ್ಲಿನ ಚೀಲದಲ್ಲಿಟ್ಟು ಅರ್ಪಿಸಿದರು. ಜನರು ತೋರಿದ ಸ್ವಾಗತಕ್ಕೆ ಸ್ವಾಮೀಜಿ ಕೃತಜ್ಞತೆಯನ್ನು ಅರ್ಪಿಸಿದರು. ಸ್ವಾಮೀಜಿ ತಮ್ಮ ಉಪನ್ಯಾಸದಲ್ಲಿ ಪ್ರಪಂಚವೆಲ್ಲ ಕೊಟ್ಟು ತೆಗೆದುಕೊಳ್ಳುವುದರ ಮೇಲೆ ನಿಂತಿದೆ ಎಂದರು. ಭರತಖಂಡ ಅಧ್ಯಾತ್ಮ ಮತ್ತು ಧರ್ಮ ಮುಂತಾದುವನ್ನು ಪಾಶ್ಚಾತ್ಯದೇಶಕ್ಕೆ ಕೊಡಬೇಕು, ಅಲ್ಲಿಂದ ವಿಜ್ಞಾನ, ಯಂತ್ರ, ಸಂಘಟನಾ ವ್ಯವಸ್ಥೆ ಮುಂತಾದುವನ್ನು ಕಲಿತುಕೊಳ್ಳಬೇಕು. ಇಬ್ಬರಿಗೂ ಮತ್ತೊಬ್ಬರದು ಸ್ವಲ್ಪ ಹಿತಕಾರಿ. ನಾವು ಪೂರ್ವಾಚಾರ ಪರಾಯಣತೆ ಮತ್ತು ಪಾಶ್ಚಾತ್ಯ ಅಂಧ ಅನುಕರಣೆ ಇವುಗಳ ಮಧ್ಯದಲ್ಲಿ ಹೋಗಬೇಕಾಗಿದೆ. ನಾವು ಪಾಶ್ಚಾತ್ಯರನ್ನು ಅನುಕರಿಸಲಾರೆವು. ಹಾಗೆ ಮಾಡಿದರೆ ನಾವು ನಿರ್ನಾಮವಾಗುವೆವು. ಅದರಂತೆಯೇ ಕೆಲಸಕ್ಕೆ ಬಾರದ ಹಲವು ಬಾಹ್ಯಾಚಾರಗಳೇ ಧರ್ಮ ಎಂಬ ಒಂದು ಮೂಢನಂಬಿಕೆ ಬಂದು ಹೋಗಿದೆ. ಅದು ಧರ್ಮದ ಬಾಹ್ಯವೇಷವೇ ಹೊರತು ತಿರುಳಲ್ಲ. ಬಾಹ್ಯವೇಷ ಕಾಲಕಾಲಕ್ಕೆ ಬದಲಾವಣೆಯಾಗುತ್ತ ಬಂದಿದೆ. ಅದರಂತೆಯೇ ನಮ್ಮ ಸಮಾಜದ ಬಾಹ್ಯ ನಡವಳಿಕೆಯಲ್ಲಿ ಈಗಿನ ಕಾಲಕ್ಕೆ ತಕ್ಕಂತೆ ಬದಲಾವಣೆಗಳು ಆಗಬೇಕು. ಹಾಗೆ ಆದರೆ ಧರ್ಮವೇನು ನಿರ್ನಾಮವಾಗಿಬಿಡುವುದಿಲ್ಲ. ನಮ್ಮ ನಿಜವಾದ ತತ್ತ್ವ ಇರುವುದು ಉಪನಿಷತ್ತಿನಲ್ಲಿ. ಅದನ್ನು ಕಂಡವರು ಮಂತ್ರದ್ರಷ್ಟರಾದ ಋಷಿಗಳು. ಅವರು ಜೀವಿಯ ನಿಜವಾದ ಸ್ಥಿತಿಯೇ ಪರಬ್ರಹ್ಮ ಸ್ವಭಾವ ಎಂಬುದನ್ನು ಅನುಭವಿಸಿದವರು. ಅವರು ಸಾರಿದ ಆತ್ಮನಿಗೆ ಸಂಬಂಧಪಟ್ಟ ನಿಯಮಗಳು ದೇಶ ಕಾಲಾತೀತವಾಗಿ ಎಂದೆಂದಿಗೂ ಸತ್ಯವಾಗಿ ರಾರಾಜಿಸುತ್ತಿರುವುದು. ನಾವೆಲ್ಲರೂ ಅಂತಹ ಋಷಿಗಳಾಗಬೇಕು. ನಮ್ಮಲ್ಲಿ ಅನಂತ ಶಕ್ತಿ ಸುಪ್ತವಾಗಿದೆ. ನಮಗೆ ಇಂದು ಆತ್ಮಶ್ರದ್ಧೆ ಬೇಕು. ನಾವು ಸುಪ್ತವಾಗಿರುವ ಚೈತನ್ಯವನ್ನು ಬಾಹ್ಯದಲ್ಲಿ ಪ್ರಕಟಗೊಳಿಸಬೇಕು. ಆಗ ನಮ್ಮ ಬಾಯಿಂದ ಹೊರಬಿದ್ದ ಪ್ರತಿಯೊಂದು ನುಡಿಯೂ ಫಲಕಾರಿಯಾಗುವುದು. ನಮ್ಮ ಮತ್ತು ಇತರರ ಮುಕ್ತಿಗಾಗಿ ನಾವು ಋಷಿಗಳಾಗುವಂತೆ ಭಗವಂತ ನಮ್ಮನ್ನೆಲ್ಲ ಆಶೀರ್ವದಿಸಲಿ ಎಂದು ಉಪನ್ಯಾಸವನ್ನು ಮುಕ್ತಾಯ ಮಾಡಿದರು. 

 ಮಧುರೆಯಲ್ಲಿದ್ದಾಗ ಸ್ವಾಮೀಜಿ ದೇವಸ್ಥಾನಕ್ಕೆ ಹೋದರು. ದೇವಸ್ಥಾನದವರು ಯೋಗ್ಯ ಮರ‍್ಯಾದೆಗಳೊಡನೆ ಅವರನ್ನು ಎದುರುಗೊಂಡು ಎಲ್ಲವನ್ನೂ ತೋರಿಸಿದರು. ಕಲಾಮಯವಾದ ಭವ್ಯವಾದ ದೇವಸ್ಥಾನವನ್ನು ನೋಡಿ ಸ್ವಾಮೀಜಿ ಆನಂದಿಸಿದರು. ಅಲ್ಲಿಯ ಪೂಜಾರಿಗಳೊಡನೆ ಮಾತುಕತೆಗಳಾಡಿದರು. ಮಧುರೆಯಲ್ಲಿ ಇರುವಾಗ ಮದ್ರಾಸಿನ ಹಿಂದೂ ಪತ್ರಿಕೆಯ ಬಾತ್ಮೀದಾರರೊಬ್ಬರು ಸ್ವಾಮೀಜಿಯವರನ್ನು ಭೇಟಿಮಾಡಿದರು. ಆ ಸಂದರ್ಶನವನ್ನು ಕೆಳಗೆ ಕೊಟ್ಟಿರುವೆವು: 

 ಪ್ರಶ್ನೆ: “ಪ್ರಪಂಚ ಮಿಥ್ಯೆ ಎನ್ನುವುದನ್ನು ಕೆಳಗಿನ ಕೆಲವು ದೃಷ್ಟಿಗಳಿಂದ ಹೇಳಿರುವರು: ೧. ಅನಂತತೆಯೊಡನೆ ಹೋಲಿಸಿ ನೋಡಿದರೆ ನಾಮರೂಪಗಳ ಜಗತ್ತು ಬಹಳ ಕ್ಷಣಿಕ. ೨. ಎರಡು ಪ್ರಳಯಗಳಿಗೂ ಮಧ್ಯೆ ಇರುವ ಅಂತರ ಅನಂತತೆಯೊಡನೆ ಹೋಲಿಸಿದರೆ ಬಹಳ ಕ್ಷಣಿಕ. ೩. ಕಪ್ಪೆಯ ಚಿಪ್ಪಿನಲ್ಲಿ ಬೆಳ್ಳಿ ಮತ್ತು ಹಗ್ಗದಲ್ಲಿ ಹಾವು ಕಾಣಿಸುವಂತೆ ಈ ಪ್ರಪಂಚ ಈಗ ಸತ್ಯದಂತೆ ಕಾಣುತ್ತಿದೆ. ಇದಕ್ಕೆ ಕಾರಣ ಮನಸ್ಸಿನ ಈಗಿನ ಸ್ಥಿತಿ. ಆದರೆ ಪ್ರಪಂಚ ಎಂದೆಂದಿಗೂ ಸತ್ಯವಲ್ಲ. ೪. ಮೊಲದ ಕೊಂಬಿನಂತೆ ಅಥವಾ ಬಂಜೆಯ ಮಗನಂತೆ ಪ್ರಪಂಚ ಒಂದು ಭ್ರಾಂತಿ. ಮೇಲಿನ ದೃಷ್ಟಿಗಳಲ್ಲಿ ಯಾವುದರ ದೃಷ್ಟಿಯಿಂದ ಪ್ರಪಂಚವನ್ನು ಮಿಥ್ಯೆ ಎನ್ನುವರು?” 

 ಉತ್ತರ: “ಹಲವು ಬಗೆಯ ಅದ್ವೈತ ಸಿದ್ಧಾಂತಗಳಿವೆ. ಒಬ್ಬೊಬ್ಬರು ಒಂದೊಂದು ದೃಷ್ಟಿಯಿಂದ ಮಿಥ್ಯೆ ಎನ್ನುವರು. ಈಗಿರುವ ಮನಸ್ಸಿನ ದೃಷ್ಟಿಯಿಂದ ಈ ಪ್ರಪಂಚ ಎಲ್ಲರಿಗೂ ಎಲ್ಲಾ ವ್ಯವಹಾರಗಳಿಗೂ ಸತ್ಯವಾಗಿ ಕಾಣಿಸುತ್ತಿದೆ. ಆದರೆ ಮನಸ್ಸು ಇನ್ನೂ ಮೇಲಿನ ಮಟ್ಟಕ್ಕೆ ಹೋದರೆ ಪ್ರಪಂಚ ಮಾಯವಾಗುವುದು. ಎದುರಿಗಿರುವ ಒಂದು ಮರದ ತುಂಡನ್ನು ನೋಡಿ ದೆವ್ವ ಎಂದು ಭ್ರಾಂತಿ ಪಡುವಿರಿ. ತತ್ಕಾಲಕ್ಕೆ ದೆವ್ವದ ಭಾವನೆ ಸತ್ಯ. ಅದು ನಿಜವಾದ ದೆವ್ವ ನಿಮ್ಮ ಮನಸ್ಸಿನ ಮೇಲೆ ಯಾವ ಪ್ರತಿಕ್ರಿಯೆಯನ್ನು ಉಂಟುಮಾಡುವುದೋ, ಅದನ್ನೆಲ್ಲಾ ಮಾಡುವುದು. ಆದರೆ ಅದೊಂದು ಮರದ ದಿಮ್ಮಿಯೆಂದು ಅರಿತೊಡನೆ ದೆವ್ವದ ಭಾವನೆ ಮಾಯವಾಗುವುದು. ಮರದ ದಿಮ್ಮಿ ಮತ್ತು ದೆವ್ವ ಎರಡೂ ಒಟ್ಟಿಗೆ ಇರಲಾರವು. ಒಂದು ಇದ್ದರೆ ಮತ್ತೊಂದು ಇರುವುದಿಲ್ಲ.” 

 ಪ್ರಶ್ನೆ: “ನಾಲ್ಕನೆಯ ದೃಷ್ಟಿಯನ್ನು ಕೂಡ ಕೆಲವು ಕಡೆಗಳಲ್ಲಿ ಶಂಕರಾಚಾರ್ಯರು ಉಪಯೋಗಿಸಲಿಲ್ಲವೆ?” 

 ಉತ್ತರ: “ಇಲ್ಲ, ಇತರರು ಯಾರೊ ಶಂಕರಾಚಾರ್ಯರನ್ನು ಚೆನ್ನಾಗಿ ತಿಳಿದುಕೊಳ್ಳದವರು ಬರೆಯುವಾಗ ನಾಲ್ಕನೆಯ ದೃಷ್ಟಿಯನ್ನು ಉಪಯೋಗಿಸಿರುವರು. ಮೊದಲನೆಯ ಮತ್ತು ಎರಡನೆಯ ದೃಷ್ಟಿಗಳನ್ನು ಬೇರೆ ಬೇರೆ ಅದ್ವೈತ ಸಿದ್ಧಾಂತಿಗಳು ಉಪಯೋಗಿಸಿರುವರೇ ಹೊರತು ಶಂಕರಾಚಾರ್ಯರು ಅದಕ್ಕೆ ಒಪ್ಪಿಗೆಯನ್ನು ಕೊಟ್ಟಿಲ್ಲ.” 

 ಪ್ರಶ್ನೆ: “ತೋರಿಕೆ ಸತ್ಯಕ್ಕೆ ಕಾರಣವೇನು?” 

 ಉತ್ತರ: “ನೀವು ಮರದ ದಿಮ್ಮಿಯನ್ನು ದೆವ್ವವೆಂದು ಭ್ರಮಿಸುವುದಕ್ಕೆ ಕಾರಣವೇನು? ಪ್ರಪಂಚ ಯಾವಾಗಲೂ ಒಂದೇ ಸಮನಾಗಿರುವುದು. ಆದರೆ ನಿಮ್ಮ ಮನಸ್ಸೆ ಈ ವ್ಯತ್ಯಾಸಗಳಿಗೆಲ್ಲ ಕಾರಣ.” 

 ಪ್ರಶ್ನೆ: “ವೇದ ಅನಾದಿ, ನಿತ್ಯವಾದುದು ಎನ್ನುವುದಕ್ಕೆ ನಿಜವಾದ ಅರ್ಥವೇನು? ಇದು ವೇದಗಳಲ್ಲಿರುವ ಸತ್ಯಕ್ಕೆ ಅನ್ವಯಿಸುತ್ತದೆಯೆ ಅಥವಾ ವೇದೋಕ್ತಿಗೆ ಅನ್ವಯಿಸುತ್ತದೆಯೆ? ಇದು ವೇದಗಳಲ್ಲಿ ಬರುವ ಸತ್ಯಕ್ಕೆ ಅನ್ವಯಿಸುವ ಹಾಗಿದ್ದರೆ ತರ್ಕ ರೇಖಾಗಣಿತ ರಸಾಯನಶಾಸ್ತ್ರ ಮುಂತಾದವುಗಳು ಕೂಡ ಆದಿ ಅಂತ್ಯಗಳಿಲ್ಲದವೇ ಆಯಿತಲ್ಲ?” 

 ಉತ್ತರ: “ವೇದಗಳಲ್ಲಿರುವ ಸತ್ಯ, ಸ್ಥಿರವಾದುದು ಮತ್ತು ಅವಿಕಾರಿಯಾದುದು ಎಂಬ ದೃಷ್ಟಿಯಿಂದ ವೇದವನ್ನು ಅನಾದಿ ಅನಂತ ಎಂದು ನೋಡುತ್ತಿದ್ದ ಕಾಲವಿತ್ತು. ಅನಂತರ ವೇದಮಂತ್ರಗಳು ಮುಖ್ಯವಾದುವು. ಈ ಮಂತ್ರಗಳು ಭಗವಂತನಿಂದ ಬಂದವು ಎಂದು ಭಾವಿಸಿದರು. ಇನ್ನೂ ಕೆಲವು ಕಾಲದಮೇಲೆ, ಇವುಗಳ ಅರ್ಥದ ಮೂಲಕ ನೋಡಿದರೆ, ಎಲ್ಲಾ ಭಗವಂತನ ಮೂಲದಿಂದ ಬಂದವುಗಳಲ್ಲ ಎಂಬ ಭಾವನೆ ಬಂತು. ಏಕೆಂದರೆ ಅವುಗಳಲ್ಲಿ ಎಷ್ಟೋ ಪಾಪಕರವಾದ ಕೆಲಸಗಳನ್ನು ಮನುಷ್ಯ ಮಾಡಬೇಕೆಂದು ಹೇಳಿದೆ. ಉದಾಹರಣೆಗೆ ಪ್ರಾಣಿ ಹಿಂಸೆ. ವೇದಗಳಲ್ಲಿ ಎಷ್ಟೋ ಕೆಲಸಕ್ಕೆ ಬಾರದ ಕಥೆಗಳೂ ಇರುತ್ತವೆ. ವೇದ ಅನಾದಿ ಅನಂತ ಎಂಬುದಕ್ಕೆ ನಿಜವಾದ ಅರ್ಥ ಅದರಲ್ಲಿರುವ ಸತ್ಯ ನಿಜವಾದುದು, ಎಂದಿಗೂ ಬದಲಾಗುವುದಿಲ್ಲ ಎಂಬುದು. ತರ್ಕ ರೇಖಾಗಣಿತ ರಸಾಯನಶಾಸ್ತ್ರ ಇವುಗಳೂ ಕೂಡ ನಿಜವಾಗಿರುವ ಮತ್ತು ಬದಲಾಗದೆ ಇರುವ ಸತ್ಯವನ್ನು ವಿವರಿಸುತ್ತವೆ. ಆ ದೃಷ್ಟಿಯಲ್ಲಿ ಇವು ಕೂಡ ಅನಾದಿ ಮತ್ತು ಅನಂತವೆ. ಆದರೆ ವೇದದಲ್ಲಿ ಇಲ್ಲದೆ ಇರುವ ಸತ್ಯವೇ ಇಲ್ಲ. ವೇದದಲ್ಲಿ ಇಲ್ಲದೆ ಇರುವ ಯಾವುದಾದರೂ ಸತ್ಯವನ್ನು ಹೇಳಿ ಎಂದು ಯಾರನ್ನಾದರೂ ಪ್ರಶ್ನಿಸುತ್ತೇನೆ.” 

\vskip 1pt

 ಪ್ರಶ್ನೆ: “ಅದ್ವೈತ ಸಿದ್ಧಾಂತದ ಪ್ರಕಾರ ಮುಕ್ತಿ ಎಂದರೇನು? ಈ ಸ್ಥಿತಿಯಲ್ಲಿ ಪ್ರಜ್ಞೆ ಇರುವುದೆ? ಅದ್ವೈತದ ಮುಕ್ತಿಗೂ ಬೌದ್ಧರ ನಿರ್ವಾಣಕ್ಕೂ ಏನಾದರೂ ವ್ಯತ್ಯಾಸವಿದೆಯೆ?” 

\vskip 1pt

 ಉತ್ತರ: “ಮುಕ್ತಿಯಲ್ಲಿ ಪ್ರಜ್ಞೆಯಿದೆ. ನಾವು ಅದನ್ನು ಅತಿಪ್ರಜ್ಞೆ ಎನ್ನುವೆವು. ಅದು ನಿಮ್ಮ ಈಗಿನ ಪ್ರಜ್ಞೆಗಿಂತ ಬೇರೆಯಾಗಿರುವುದು. ಮುಕ್ತಿಯಲ್ಲಿ ಪ್ರಜ್ಞೆಯಿಲ್ಲ ಎನ್ನುವುದು ತರ್ಕಬದ್ಧವಲ್ಲ. ಪ್ರಜ್ಞೆಯಲ್ಲಿ ಬೆಳಕಿನಂತೆ ಮೂರು ವಿಧಗಳಿವೆ: ಮಂದ, ಮಧ್ಯಮ ಮತ್ತು ಉತ್ತಮ. ಬೆಳಕು ತುಂಬಾ ಕೋರೈಸುತ್ತಿರುವಾಗ ಕಾಣಿಸುವುದಿಲ್ಲ. ಹೇಗೆ ಒಂದು ಮಂದವಾದ ಕಾಂತಿಯಲ್ಲಿ ಕಣ್ಣಿಗೆ ಏನೂ ಕಾಣಿಸುವುದಿಲ್ಲವೋ ಹಾಗೇ ಅತಿ ತೀವ್ರ ಬೆಳಕಿನಲ್ಲಿಯೂ ಕಣ್ಣಿಗೆ ಕಾಣಿಸುವುದಿಲ್ಲ. ಬೌದ್ಧರು ಏನಾದರೂ ಹೇಳಲಿ, ಅವರ ನಿರ್ವಾಣದಲ್ಲಿ ಪ್ರಜ್ಞೆ ಇರಲೇಬೇಕು. ನಾವು ಕೊಡುವ ಮುಕ್ತಿಯ ವಿವರಣೆ ಅನ್ವಯಾತ್ಮಕವಾದುದು (\enginline{affirmative}); ಬೌದ್ಧರು ಕೊಡುವ ವಿವರಣೆ ನಿಷೇಧಾತ್ಮಕವಾದುದು.” 

\vskip 1pt

 ಪ್ರಶ್ನೆ: “ಅವ್ಯಕ್ತನಾದ ಬ್ರಹ್ಮ ವಿಶ್ವವನ್ನು ನಿರ್ಮಿಸುವುದಕ್ಕಾಗಿ ಏತಕ್ಕೆ ವ್ಯಕ್ತ ಸ್ವರೂಪವನ್ನು ತಾಳಬೇಕು?” 

\vskip 1pt

 ಉತ್ತರ: “ನಿಮ್ಮ ಪ್ರಶ್ನೆಯೇ ತಾರ್ಕಿಕವಾಗಿಲ್ಲ. ಬ್ರಹ್ಮ ವಾಕ್ ಮತ್ತು ಮನಸ್ಸಿಗೆ ಅಗೋಚರ. ದೇಶ ಕಾಲ ನಿಮಿತ್ತದ ಆಚೆ ಇರುವುದನ್ನು ಮನಸ್ಸು ಗ್ರಹಿಸಲಾರದು. ವಿಚಾರ ತರ್ಕ ಇವೆಲ್ಲ ಇರುವುದು ದೇಶಕಾಲ ನಿಮಿತ್ತಗಳ ಒಳಗೆ. ಸ್ಥಿತಿ ಹೀಗಿರುವಾಗ ಮನಸ್ಸಿಗೆ ಮೀರಿದುದನ್ನು ಕುರಿತು ಪ್ರಶ್ನಿಸುವುದು ವ್ಯರ್ಥ.” 

 ಪ್ರಶ್ನೆ: “ಪುರಾಣಗಳಲ್ಲಿ ಕೆಲವು ಕಡೆ ಔಪಮಾನಿಕವಾಗಿ ಸತ್ಯವನ್ನು ವಿವರಿಸುವ ಪ್ರಯತ್ನಗಳು ಮಾಡಲ್ಪಟ್ಟಿವೆ. ಪುರಾಣದಲ್ಲಿ ಯಾವ ಚಾರಿತ್ರಿಕ ಸತ್ಯವೂ ಇಲ್ಲದೆ ಇರಬಹುದು, ಆದರೆ ಒಂದು ಆಧ್ಯಾತ್ಮಿಕ ಸತ್ಯವನ್ನು ವಿವರಿಸಲು ಕಾಲ್ಪನಿಕ ಘಟನೆಗಳನ್ನು ಮತ್ತು ವ್ಯಕ್ತಿಗಳನ್ನು ತೆಗೆದುಕೊಳ್ಳುವರು ಎಂದು ಕೆಲವರು ಹೇಳುವರು. ವಿಷ್ಣು ಪುರಾಣ, ರಾಮಾಯಣ, ಭಾರತ ಇವನ್ನು ತೆಗೆದುಕೊಳ್ಳಿ. ಅಲ್ಲಿ ಚಾರಿತ್ರಿಕ ಘಟನೆಗಳಿವೆಯೋ? ಇಲ್ಲವೆ ಇವುಗಳೆಲ್ಲ ಆಧ್ಯಾತ್ಮಿಕ ತತ್ತ್ವಗಳನ್ನು ವಿವರಿಸುವ ಕಾಲ್ಪನಿಕ ಘಟನೆಗಳೋ? ಅಥವಾ ಮಾನವನು ಹೇಗಿರಬೇಕೆಂದು ಅವನ ಮುಂದಿರುವ ಪರಮ ಆದರ್ಶಗಳೊ? ಅಥವಾ ಹೋಮರನ ಕಾವ್ಯದಂತೆ ಇವುಗಳೆಲ್ಲ ಬರಿಯ ಕಾವ್ಯಗಳೊ?” 

 ಉತ್ತರ: “ಎಲ್ಲಾ ಪುರಾಣಗಳ ಹಿಂದೆಯೂ ಕೆಲವು ಚಾರಿತ್ರಿಕ ಘಟನೆಗಳೇ ತಳಹದಿಯಾಗಿವೆ. ಪುರಾಣದ ಗುರಿ ಪರಮಸತ್ಯವನ್ನು ಜನರಿಗೆ ಬೇರೆ ವಿಧದಲ್ಲಿ ಹೇಳುವುದಾಗಿದೆ. ಅವುಗಳಲ್ಲಿ ಚಾರಿತ್ರಿಕ ಸತ್ಯವಿಲ್ಲದೆ ಇದ್ದರೂ ಪಾರಮಾರ್ಥಿಕ ದೃಷ್ಟಿಯಿಂದ ಅವು ಒಂದು ಅಧಿಕಾರವಾಣಿಯಿಂದ ಮಾತನಾಡುತ್ತವೆ. ರಾಮಾಯಣವನ್ನು ತೆಗೆದುಕೊಳ್ಳಿ. ಆದರ್ಶಶೀಲದ ದೃಷ್ಟಿಯಿಂದ ನೋಡುವುದಾದರೆ ರಾಮನಂತಹ ಒಂದು ವ್ಯಕ್ತಿ ಇರಬೇಕೆಂಬ ಅವಶ್ಯಕವೇನೂ ಇಲ್ಲ. ರಾಮಾಯಣ ಮಹಾಭಾರತಗಳಲ್ಲಿ ಬರುವ ಧರ್ಮದ ಮಹಿಮೆ ರಾಮನ ಅಥವಾ ಕೃಷ್ಣರ ಇತಿಹಾಸದ ಮೇಲೆ ನಿಂತಿಲ್ಲ. ಅವರಿರಲಿಲ್ಲವೆಂದು ನಂಬಿದರೂ ಯಾವ ಪರಮ ಆದರ್ಶಗಳನ್ನು ಅವು ಮಾನವನ ಮುಂದೆ ಇಡುವುದೋ ಅದಕ್ಕೆ ದೊಡ್ಡ ಪ್ರಮಾಣದಂತೆ ಇವೆ ಈ ಗ್ರಂಥಗಳು. ನಿಮ್ಮ ದಾರ್ಶನಿಕರಿಗೆ ಸತ್ಯವನ್ನು ಬೋಧಿಸಲು ಯಾವ ವ್ಯಕ್ತಿಗಳೂ ಬೇಕಿಲ್ಲ. ಕೃಷ್ಣ ತನ್ನದೇ ಆದ ಯಾವ ಹೊಸ ಸಿದ್ಧಾಂತವನ್ನೂ ಪ್ರಪಂಚಕ್ಕೆ ಹೇಳಲಿಲ್ಲ. ಆಗಲೆ ಹಿಂದಿನ ಶಾಸ್ತ್ರದಲ್ಲಿ ಇಲ್ಲದೇ ಇರುವುದನ್ನು ರಾಮಾಯಣ ಏನೂ ಹೇಳುವುದಿಲ್ಲ. ಕ್ರಿಸ್ತನಿಲ್ಲದೆ ಕ್ರೈಸ್ತಧರ್ಮ, ಮಹಮ್ಮದನಿಲ್ಲದೆ ಮಹಮ್ಮದೀಯ ಧರ್ಮ, ಬುದ್ಧನಿಲ್ಲದೆ ಬೌದ್ಧಧರ್ಮ ನಿಲ್ಲಲಾರದು ಎಂದು ಬೇಕಾದರೆ ಹೇಳಬಹುದು. ಆದರೆ ಹಿಂದೂಧರ್ಮ ಯಾವ ವ್ಯಕ್ತಿಯ ಮೇಲೂ ನಿಂತಿಲ್ಲ. ಅಲ್ಲಿ ಬರುವ ವ್ಯಕ್ತಿಗಳು ಚಾರಿತ್ರಿಕವೇ ಅಥವಾ ಕಾಲ್ಪನಿಕವೇ ಎಂಬುದನ್ನು ಗಮನಿಸಬೇಕಾಗಿಲ್ಲ. ಪುರಾಣದ ಉದ್ದೇಶ ಜನರನ್ನು ಶಿಕ್ಷಿತರನ್ನಾಗಿ ಮಾಡುವುದು. ಅದನ್ನು ಬರೆದ ಋಷಿಗಳು ಆಗಿನ ಕಾಲದಲ್ಲಿ ಜ್ಞಾಪಕದಲ್ಲಿದ್ದ ಕೆಲವು ಚಾರಿತ್ರಿಕ ವ್ಯಕ್ತಿಗಳನ್ನು ತೆಗೆದುಕೊಂಡು, ಅವರಲ್ಲಿ ಒಳ್ಳೆಯದನ್ನೋ ಕೆಟ್ಟದ್ದನ್ನೋ ಆರೋಪಮಾಡಿ ಮಾನವನ ನಡವಳಿಕೆಗೆ ನೀತಿ ನಿಯಮಗಳನ್ನು ಮಾಡಿದರು. ರಾಮಾಯಣದಲ್ಲಿ ಬರುವ ದಶಕಂಠ ನಿಜವಾಗಿದ್ದಿರಬಹುದೆ? ನಾವು ತಿಳಿದುಕೊಳ್ಳಬೇಕಾದ ಒಂದು ಆದರ್ಶದ ಪ್ರತಿನಿಧಿ ನಿಜವಾಗಿ ಇದ್ದನೇ ಇಲ್ಲವೆ ಎಂಬುದಲ್ಲ ಮುಖ್ಯ. ನೀವು ಬೇಕಾದರೆ ಕೃಷ್ಣನನ್ನು ಇನ್ನೂ ಆಕರ್ಷಕ ರೀತಿಯಲ್ಲಿ ಚಿತ್ರಿಸಬಹುದು. ನಿಮ್ಮ ಭಾವನೆಯ ಭವ್ಯತೆಗೆ ತಕ್ಕಂತೆ ವಿವರಣೆಗಳೂ ವ್ಯತ್ಯಾಸವಾಗುವುವು. ಆದರೆ ಅದರ ಹಿಂದೆ ಪುರಾಣದಲ್ಲಿರುವ ಗಹನ ಸತ್ಯವಿದೆ.” 

 ಪ್ರಶ್ನೆ: “ಒಬ್ಬ ಸಿದ್ಧನಾದರೆ ತನ್ನ ಪೂರ್ವಜನ್ಮಗಳನ್ನು ಜ್ಞಾಪಿಸಿಕೊಳ್ಳುವುದಕ್ಕೆ ಸಾಧ್ಯವೆ? ಹಿಂದಿನ ಜನ್ಮದ ಮೆದುಳಿನಲ್ಲಿ ಯಾವ ಅನುಭವಗಳನ್ನು ಶೇಖರಿಸಿದ್ದನೋ ಆ ಮೆದುಳು ಈಗಿರುವುದಿಲ್ಲ. ಈ ಜನ್ಮದಲ್ಲಿ ಅವನಿಗೆ ಬೇರೊಂದು ಮೆದುಳಿದೆ. ಹೀಗಿದ್ದರೆ ಈ ಮೆದುಳಿಗೆ ಹಿಂದಿನ ಜನ್ಮದ ಮತ್ತೊಂದು ಮೆದುಳಿನ ಅನುಭವಗಳನ್ನು ಜ್ಞಾಪಿಸಿಕೊಳ್ಳುವುದಕ್ಕೆ ಹೇಗೆ ಸಾಧ್ಯ?” 

 ಸ್ವಾಮೀಜಿ: “ಸಿದ್ಧ ಎಂದರೇನು?” 

 ಪ್ರಶ್ನೆ: “ತನ್ನ ಸ್ವಭಾವದ ಸುಪ್ತಶಕ್ತಿಗಳನ್ನು ಜಾಗೃತಗೊಳಿಸಿಕೊಂಡಿರುವನು.” 

 ಉತ್ತರ: “ಸುಪ್ತವಾಗಿರುವುದು ಹೇಗೆ ಜಾಗೃತವಾಗುವುದೆಂದು ನನಗೆ ಗೊತ್ತಿಲ್ಲ. ನೀವು ಏನನ್ನು ಉದ್ದೇಶಿಸುವಿರೊ ಅದು ನನಗೆ ಗೊತ್ತಿದೆ, ಆದರೆ ಉಪಯೋಗಿಸುವ ಪದ ನಿರ್ದಿಷ್ಟವಾಗಿರಬೇಕು, ಸ್ಪಷ್ಟವಾಗಿರಬೇಕು. ಯಾವ ಶಕ್ತಿಯ ಮೇಲೆ ಆವರಣವಿತ್ತೊ ಅದು ಆಚೆಗೆ ಸರಿಯಿತು ಎನ್ನಬಹುದು. ಯಾರು ತಮ್ಮ ನಿಜವಾದ ಶಕ್ತಿಯನ್ನು ಅರಿಯುವರೋ ಅವರಿಗೆ ತಮ್ಮ ಹಿಂದಿನ ಜನ್ಮದ ಸ್ಮರಣೆ ಸಾಧ್ಯ. ಏಕೆಂದರೆ ಈಗಿರುವ ಅವರ ಮೆದುಳು ಹಿಂದಿನ ಸೂಕ್ಷ್ಮ ಶರೀರದ ಬೀಜವಾಗಿದೆ.” 

 ಪ್ರಶ್ನೆ: “ಹೊರಗಿನವರು ಹಿಂದೂಗಳಾಗುವುದಕ್ಕೆ ಹಿಂದೂಧರ್ಮ ಅವಕಾಶವನ್ನು ಕೊಡುವುದೆ? ಬ್ರಾಹ್ಮಣ ಚಂಡಾಲನ ಉಪದೇಶವನ್ನು ಕೇಳಬಹುದೆ?” 

 ಉತ್ತರ: “ಹಿಂದೂಗಳ ವರ್ಣವ್ಯವಸ್ಥೆ ಬದಲಾವಣೆಯನ್ನು ಒಪ್ಪಿಕೊಳ್ಳುವುದು. ಯಾರಾದರೂ ಆಗಿರಲಿ, ಶೂದ್ರನಾಗಲಿ, ಚಂಡಾಲನಾಗಿರಲಿ, ಅವನು ತತ್ತ್ವವನ್ನು ಬ್ರಾಹ್ಮಣನಿಗೂ ಹೇಳಬಹುದು. ಸತ್ಯವನ್ನು ಪಾಮರನಿಂದಲೂ ಕಲಿತುಕೊಳ್ಳಬಹುದು, ಅವನು ಯಾವ ಜಾತಿಗೆ ಅಥವಾ ಕುಲಕ್ಕೆ ಸೇರಿದ್ದರೂ ಚಿಂತೆಯಿಲ್ಲ.” 

 ಇಲ್ಲಿ ಸ್ವಾಮಿಗಳು ಇದನ್ನು ವಿವರಿಸುವುದಕ್ಕೆ ಶಾಸ್ತ್ರದಿಂದ ಒಂದು ಶ್ಲೋಕವನ್ನು ಉದಾಹರಿಸಿದರು. 

 ಸಂದರ್ಶನ ಮುಗಿಯಿತು. ಅವರ ಕಾರ್ಯಕ್ರಮದ ಪ್ರಕಾರ ದೇವಸ್ಥಾನಕ್ಕೆ ಹೋಗುವ ಸಮಯವಾಯಿತು. ಅವರು ಬಾತ್ಮೀದಾರರನ್ನು ಬೀಳ್ಕೊಂಡು ದೇವಸ್ಥಾನಕ್ಕೆ ಹೋದರು. 

 ಸ್ವಾಮೀಜಿ ಅಲ್ಲಿಂದ ಕುಂಭಕೋಣದ ಕಡೆ ಸಂಜೆ ರೈಲಿನಲ್ಲಿ ಹೋದರು. ದಾರಿಯಲ್ಲಿ ಸಿಕ್ಕುವ ಸ್ಟೇಶನ್ನಿನಲ್ಲೆ ಜನರು ನೆರೆದಿದ್ದರು. ಹೂ ಹಣ್ಣುಗಳನ್ನು ಅರ್ಪಿಸಿದರು. ಕೆಲವು ಕಡೆ ಬಿನ್ನವತ್ತಳೆಗಳನ್ನು ಕೊಟ್ಟರು. ಆದರೆ ಸ್ವಾಮೀಜಿಗೆ ಅದಕ್ಕೆಲ್ಲ ಉತ್ತರ ಹೇಳುವುದಕ್ಕೆ ಅವಕಾಶವಿರಲಿಲ್ಲ. ಎಲ್ಲರಿಗೂ ಶುಭಾಶಯವನ್ನು ಕೋರಿದರು. ತಿರುಚಿನಾಪಲ್ಲಿಗೆ ಬೆಳಗಿನ ಜಾವ ರೈಲು ಬಂದಿತು. ಆ ಸಮಯದಲ್ಲಿ ಸ್ವಾಮೀಜಿಯವರನ್ನು ನೋಡಲು ಸಹಸ್ರಾರು ಜನ. ಸ್ಟೇಶನ್ನಿನಲ್ಲಿ ನೆರೆದಿದ್ದರು. ಅಲ್ಲಿಯ ರೈಲ್ವೆ ಪ್ಲಾಟ್‌ಫಾರಂ ಮೇಲೆಯೆ ಪುರಜನರು ಒಂದು ಬಿನ್ನವತ್ತಳೆಯನ್ನು ಅರ್ಪಿಸಿದರು. ಅಲ್ಲಿನ ನ್ಯಾಷನಲ್ ಹೈಸ್ಕೂಲಿನ ಕೌನ್ಸಿಲಿನವರು, ಆ ಊರಿನ ವಿದ್ಯಾರ್ಥಿವೃಂದದ ಪರವಾಗಿ ಒಂದು ಬಿನ್ನವತ್ತಳೆಯನ್ನು ಅರ್ಪಿಸಿದರು. ಸ್ವಾಮೀಜಿಯವರು ಸಂಕ್ಷೇಪವಾಗಿ ಇವುಗಳಿಗೆಲ್ಲ ಉತ್ತರವನ್ನು ಕೊಟ್ಟರು. ಕೆಲವು ಗಂಟೆಗಳಾದ ಮೇಲೆ ತಂಜಾವೂರಿನಲ್ಲಿ ಮತ್ತೊಂದು ದೊಡ್ಡ ಪ್ರೇಕ್ಷಕವೃಂದ ಅವರನ್ನು ನೋಡುವುದಕ್ಕೆ ಕಾಯುತ್ತಿದ್ದಿತು. ಅನಂತರ ಸ್ವಾಮೀಜಿ ಕುಂಭಕೋಣವನ್ನು ಸೇರಿದರು. ಅಲ್ಲಿ ಮೂರು ದಿನಗಳಿದ್ದರು. ಏಕೆಂದರೆ ಮದ್ರಾಸಿನಲ್ಲಿ ತಮಗೆ ಬಹಳ ಕಾಲ ಕಾರ‍್ಯಕ್ರಮಗಳಿರುವುದರಿಂದ ಅದಕ್ಕಿಂತ ಮುಂಚೆ ಸ್ವಲ್ಪ ವಿಶ್ರಾಂತಿಯನ್ನು ಪಡೆಯಬೇಕೆಂದು ನಿಶ್ಚಯಿಸಿದರು. ಅಲ್ಲಿ ಸ್ವಾಮೀಜಿಯವರಿಗೆ ಹಿಂದೂ ಪುರಜನರ ಪರವಾಗಿ ಒಂದು ವಿದ್ಯಾರ್ಥಿ ವೃಂದದ ಪರವಾಗಿ ಒಂದು ಹೀಗೆ ಎರಡು ಬಿನ್ನವತ್ತಳೆಗಳನ್ನು ಅರ್ಪಿಸಿದರು. ಅದಕ್ಕೆ ಉತ್ತರವನ್ನು ಕೊಡುವಾಗಲೆ ಸ್ವಾಮೀಜಿ ವೇದಾಂತದ ಸಂದೇಶ ಎಂಬ ವಿಷಯದ ಮೇಲೆ ಅದ್ಭುತವಾದ ಭಾಷಣವನ್ನು ಮಾಡಿದರು. ಆ ಉಪನ್ಯಾಸದ ಸಾರಾಂಶವನ್ನು ಕೆಳಗೆ ಕೊಡುವೆವು: 

 ಭರತಖಂಡದ ಮೂಲ ಧರ್ಮದಲ್ಲಿದೆ. ನಾವು ಅದನ್ನು ಬಿಟ್ಟು ಬೇರೆಯದನ್ನು ಒಪ್ಪಿಕೊಳ್ಳಿ ಎನ್ನುವುದಕ್ಕೆ ಆಗುವುದಿಲ್ಲ. ಇತರ ದೇಶದವರಿಗೆ ಧರ್ಮ ಹಲವು ವಸ್ತುಗಳಲ್ಲಿ ಒಂದು. ಆದರೆ ಭಾರತೀಯನಿಗಾದರೋ ಅದಕ್ಕಾಗಿ ಉಳಿದವೆಲ್ಲ. ಈ ಸಂಸಾರ ಅನಿತ್ಯ. ಇದರ ಮೂಲಕ ನಾವು ಮುಕ್ತಿಯ ಕಡೆ ಹೋಗಬೇಕಾಗಿದೆ. ಇದೊಂದು ದಾರಿಯೇ ಹೊರತು ಗುರಿಯಲ್ಲ. ಪಾಶ್ಚಾತ್ಯರಿಗೆ ಬೇಕಾದಷ್ಟು ಬಾಹ್ಯಸಂಪತ್ತು ಇರಬಹುದು, ಆದರೆ ಅವರ ಸಮಸ್ಯೆಯನ್ನು ಅಧ್ಯಾತ್ಮ ಮಾತ್ರ ಬಗೆಹರಿಸಬಲ್ಲದು, ಅವರು ಭರತಖಂಡದ ಕಡೆ ತಿರುಗಿ ನೋಡುತ್ತಿರುವರು. ನಾವದನ್ನು ಅವರಿಗೆ ಕೊಡುವ ಯೋಗ್ಯತೆಯನ್ನು ಸಂಪಾದಿಸಬೇಕಾಗಿದೆ. 

 ಇತರ ಧರ್ಮಗಳೆಲ್ಲ ಆಯಾ ಮತಸ್ಥಾಪಕರ ಜೀವನ ಮತ್ತು ಉಪದೇಶದ ಮೇಲೆ ನಿಂತಿವೆ. ವ್ಯಕ್ತಿಯ ಮೇಲೆ ನಿಂತ ಧರ್ಮ ವಿಶ್ವವ್ಯಾಪಿಯಾಗಲಾರದು. ಆದರೆ ಹಿಂದೂಧರ್ಮ ತತ್ತ್ವದ ಮೇಲೆ ನಿಂತಿದೆ. ಇಲ್ಲಿ ಬರುವ ವ್ಯಕ್ತಿಗಳು ತತ್ತ್ವವನ್ನು ಉದಾಹರಿಸುವುದಕ್ಕೆ ಬರುವರೇ ಹೊರತು ಹೊಸ ತತ್ತ್ವವನ್ನು ಸಾರುವುದಕ್ಕೆ ಬರುವುದಿಲ್ಲ. ಹಿಂದೂಗಳಲ್ಲಿ ವ್ಯಕ್ತಿಗಳು ಗೌಣ; ತತ್ತ್ವ ಮುಖ್ಯ. ವ್ಯಕ್ತಿಗಳೆಲ್ಲ ಕಲ್ಪನೆ ಎಂದು ನಿರ್ಣಯವಾದರೂ ಈ ಧರ್ಮ ಬಿದ್ದು ಹೋಗುವುದಿಲ್ಲ. ಏಕೆಂದರೆ ಇದರ ತಳಹದಿ ವಜ್ರೋಪಮವಾಗಿರುವ ತತ್ತ್ವ. ವೇದಾಂತ ಪಾಶ್ಚಾತ್ಯರಿಗೆ ಪ್ರಿಯವಾಗಿರುವುದಕ್ಕೆ ಮತ್ತೊಂದು ಕಾರಣವೆ ಇದು. ವೈಜ್ಞಾನಿಕ ಅನ್ವೇಷಣೆಗಳೆಲ್ಲವನ್ನೂ ತೆಗೆದುಕೊಂಡು ತಮ್ಮ ತತ್ತ್ವವನ್ನು ವಿವರಿಸುವರು. ವಿಜ್ಞಾನಕ್ಕೆ ವೇದಾಂತ ಅಂಜುವುದಿಲ್ಲ ಅಥವಾ ಅದರೊಂದಿಗೆ ವ್ಯಾಜ್ಯವನ್ನು ಮಾಡುವುದಿಲ್ಲ. ಅದಕ್ಕೆ ಒಂದು ಸ್ಥಾನವನ್ನು ಕೊಡುವುದು. ಅದರ ಸಿದ್ಧಾಂತಗಳನ್ನು ಸ್ವೀಕರಿಸಿ ಕೆಲವು ವೇಳೆ ಅದಕ್ಕೂ ಒಂದು ಹೆಜ್ಜೆ ಮುಂದೆ ಹೋಗುವುದು. ಇದು ಪಾಶ್ಚಾತ್ಯರಲ್ಲಿ ಪ್ರಿಯವಾಗುವುದಕ್ಕೆ ಇರುವ ಮತ್ತೊಂದು ಕಾರಣವೇ ಇದು ಯುಕ್ತಿಯುಕ್ತವಾಗಿರುವುದು. 

 ಇತರ ಧರ್ಮದಲ್ಲಿರುವ ದೇವರು ಆಯಾ ಧರ್ಮಕ್ಕೆ ಮತ್ತು ದೇಶಕ್ಕೆ ಮಾತ್ರ ಅನ್ವಯಿಸುವನು. ಅವರ ದೇವರಿಗೆ ಯಾರೋ ಒಬ್ಬರ ಮೇಲೆ ಪಕ್ಷಪಾತ. ಅಂತಹ ಜನಾಂಗಕ್ಕೆ ದೇವದೂತನನ್ನೂ ಮಗನನ್ನೂ ಕಳುಹಿಸಿ ಅವರನ್ನು ರಕ್ಷಿಸುವನು. ಇತರರು ಉದ್ಧಾರವಾಗಬೇಕಾದರೆ ಅವರೂ ಅವರನ್ನು ಅನುಸರಿಸಿದರೆ ಮಾತ್ರ ಸಾಧ್ಯ. ಆದರೆ ಹಿಂದೂ ದೇಶದಲ್ಲಿ ಮಾತ್ರ ಸತ್ಯವೊಂದೇ; ಅದನ್ನು ಹಲವು ಹೆಸರುಗಳಿಂದ ಕರೆಯುವರು, ಎಲ್ಲರೂ ತಮ್ಮ ತಮ್ಮ ಮತದ ತಿರುಳನ್ನು ಅನುಷ್ಠಾನ ಮಾಡಿದರೆ ಒಂದೇ ಗುರಿಯೆಡೆಗೆ ಹೋಗುವರು ಎಂಬುದನ್ನು ನಂಬುವರು. 

\newpage

 ಸ್ವಾಮೀಜಿ ಅನಂತರ ನಮಗೆ ಶಕ್ತಿ ಬೇಕು, ಶ್ರದ್ಧೆ ಬೇಕು, ಬಲ ಬೇಕು ಎಂದರು. ನಮ್ಮ ದೇಶಕ್ಕೆ ಇಂದು ಬೇಕಾಗಿರುವ ಕಬ್ಬಿಣದಂತಹ ಮಾಂಸಖಂಡಗಳು, ಉಕ್ಕಿನಂತಹ ನರಗಳು, ಯಾವುದನ್ನೂ ಲೆಕ್ಕಿಸದೆ ವಿಶ್ವದ ರಹಸ್ಯಮತ ಸತ್ಯಗಳನ್ನು ಭೇದಿಸಿ ಸಾಧ್ಯವಾದರೆ, ಕಡಲಿನ ಅಂತರಾಳಕ್ಕೂ ಹೋಗಿ, ಮೃತ್ಯುವಿನೊಂದಿಗೆ ಹೋರಾಡಿ ತಮ್ಮ ಗುರಿಯನ್ನು ಸಾಧಿಸಬಲ್ಲ ಪ್ರಚಂಡ ಇಚ್ಛಾಶಕ್ತಿ. 

 ನಮಗೆ ಇಂದು ಶ್ರದ್ಧೆ ಬೇಕು, ಆತ್ಮಶ್ರದ್ಧೆ, ಈಶ್ವರನಲ್ಲಿ ಶ್ರದ್ಧೆಬೇಕು. ಇದೇ ಮಹಾತ್ಮೆಯ ಮೂಲ. ನೀವು ನಿಮ್ಮ ಮೂವತ್ತು ಮೂರು ಕೋಟಿ ದೇವರುಗಳನ್ನು ನಂಬಿ ಜೊತೆಗೆ ಪಾಶ್ಚಾತ್ಯರು ಆಮದು ಮಾಡಿರುವ ದೇವರನ್ನು ನಂಬಿದರೂ ನಿಮ್ಮಲ್ಲಿ ಆತ್ಮಶ್ರದ್ಧೆ ಇಲ್ಲದೆ ಇದ್ದರೆ ನಿಮಗೆ ಮೋಕ್ಷವಿಲ್ಲ, ನಮ್ಮ ದೇಶದ ಜನ ಇಂದು ಅದನ್ನು ಕಳೆದುಕೊಂಡಿರುವರು. ನಾವು ಅವರಿಗೆ ಅದನ್ನು ಉಪನಿಷತ್ತಿನ ಮೂಲಕವಾಗಿಯೇ ಕೊಡಬೇಕಾಗಿದೆ. 

 ಸಮಾಜ ಸುಧಾರಣೆ ಜಾತಿ ನಿರ್ಮೂಲ ಮುಂತಾದವು ಗೌಣ. ಜೀವಿಯ ವಿಕಾಸಕ್ಕೆ ವಾತಾವರಣವನ್ನು ಕಲ್ಪಿಸಿದರೆ ಜೀವಿ ವಿಕಾಸವಾದಂತೆ ತಾನೇ ಅಯೋಗ್ಯವಾದುದನ್ನು ತ್ಯಜಿಸಿ ಮೇಲೇಳುವನು. ನಮ್ಮ ಸಮಾಜದ ಆದರ್ಶವೇ ಬ್ರಾಹ್ಮಣ, ಬ್ರಹ್ಮೀಭೂತನಾಗಿರುವುದು, ಎಲ್ಲಾ ಕಡೆಯಲ್ಲಿಯೂ ಆ ಪರಬ್ರಹ್ಮನನ್ನೇ ನೋಡುವಂತಹ ವ್ಯಕ್ತಿಗಳನ್ನು ಸೃಷ್ಟಿಸುವುದಾಗಿದೆ. ಜಾತಿಯಲ್ಲಿ ಬ್ರಾಹ್ಮಣನಲ್ಲ; ಕರ್ಮದಲ್ಲಿ ಭಾವದಲ್ಲಿ ಹೃದಯದ ಔನ್ನತ್ಯದಲ್ಲಿ ಅವನು ಬ್ರಾಹ್ಮಣನಾಗಿರಬೇಕು. ಇದು ಜಾತಿಕುಲಗಳಿಗೆ ಅತೀತವಾದ ಅವಸ್ಥೆ. ಕೊನೆಯದಾಗಿ ನಮ್ಮ ಜನಾಂಗವನ್ನು ಪ್ರೀತಿಸಬೇಕು. “ಮಾತೃಭೂಮಿಯೆಂಬ ಹಡಗು ಸಹಸ್ರಾರು ವರ್ಷಗಳಿಂದಲೂ ಸಂಚರಿಸುತ್ತಿದೆ. ಇಂದು ಅದು ಎಲ್ಲೋ ಅಲ್ಪಸ್ವಲ್ಪ ಸೋರುತ್ತಿರಬಹುದು. ಅದರ ಯಾವುದೋ ಒಂದು ಭಾಗ ಸ್ವಲ್ಪ ಸವೆದುಹೋಗಿರಬಹುದು. ಇಂತಹ ಸಮಯದಲ್ಲಿ ಸಾಧ್ಯವಾದ ಮಟ್ಟಿಗೆ ಆ ರಂಧ್ರವನ್ನು ಮುಚ್ಚುವುದು ನಿಮ್ಮ ಮತ್ತು ನನ್ನ ಕರ್ತವ್ಯ” ಎಂದು ಹೇಳಿದರು. 

 ಮೂರು ದಿನಗಳು ಸ್ವಾಮೀಜಿಯವರು ಕುಂಭಕೋಣದಲ್ಲಿ ಇದ್ದಾದ ಮೇಲೆ ಮದ್ರಾಸಿನ ಕಡೆ ಹೊರಟರು. ದಾರಿಯ ರೈಲ್ವೆ ನಿಲ್ದಾಣಗಳಲ್ಲೆ ತಂಡೋಪತಂಡವಾಗಿ ಜನರು ಉತ್ಸಾಹದಿಂದ ಬಂದು ಸ್ವಾಮೀಜಿಯವರ ದರ್ಶನ ಮಾಡಲು ಕಾಯುತ್ತಿದ್ದರು. ಮಾಯಾವರದ ರೈಲ್ವೆ ನಿಲ್ದಾಣದಲ್ಲಿ ಶ‍್ರೀ ಡಿ. ನಟೇಶ ಅಯ್ಯರ್ ಅವರ ನೇತೃತ್ವದಲ್ಲಿ ಅಲ್ಲಿಯ ಪುರಜನರು ಒಂದು ಬಿನ್ನವತ್ತಳೆಯನ್ನು ಅರ್ಪಿಸಿದರು. ಸ್ವಾಮೀಜಿ ಅದಕ್ಕೆ ಸೂಕ್ತವಾದ ಉತ್ತರ ಕೊಟ್ಟರು. ನಂತರ ದಾರಿಯ ಒಂದು ರೈಲ್ವೆ ನಿಲ್ದಾಣದಲ್ಲಿ ಸ್ವಾಮೀಜಿಯವರನ್ನು ನೋಡಲು ನೂರಾರು ಜನ ಕಾತರತೆಯಿಂದ ಕಾದುಕೊಂಡಿದ್ದರು. ಆದರೆ ಆ ರೈಲು ಮೇಲ್‌ಟ್ರೈನ್ ಆದುದರಿಂದ ಆ ಸ್ಟೇಷನ್ನಿನಲ್ಲಿ ನಿಲ್ಲುತ್ತಿರಲಿಲ್ಲ. ಜನರು ಸ್ಟೇಷನ್‌ಮಾಸ್ಟರಿಗೆ ಹೇಗಾದರೂ ರೈಲನ್ನು ಕೆಲವು ನಿಮಿಷಗಳಾದರೂ ನಿಲ್ಲಿಸಬೇಕೆಂದು ಕೋರಿಕೊಂಡರು. ಆದರೆ ರೈಲ್ವೆಯ ಅಧಿಕಾರಿ ಒಪ್ಪಲಿಲ್ಲ. ಮದ್ರಾಸಿನ ಕಡೆ ಹೋಗುತ್ತಿರುವ ರೈಲು ದೂರದಲ್ಲಿ ಬಂತು. ನೂರಾರು ಜನ ರೈಲ್ವೆ ಕಂಬಿಯ ಮೇಲೆ ಮಲಗಿಬಿಟ್ಟರು. ಸ್ಟೇಷನ್‌ಮಾಸ್ಟರ್ ಇದನ್ನು ನೋಡಿ ವಿಧಿಯಿಲ್ಲದೆ ರೈಲನ್ನು ನಿಲ್ಲಿಸಬೇಕಾಗಿ ಬಂದಿತು. ರೈಲು ನಿಂತಮೇಲೆ ಸ್ವಾಮೀಜಿ ದರ್ಶನ ಮಾಡಿದರು. ಅನಂತರ ರೈಲು ಮುಂದೆ ಹೋಗಲು ಸಾಧ್ಯವಾಯಿತು. ರೈಲು ಚಂಗಲ್‌ಪೇಟೆಯ ಸ್ಟೇಷನ್ನಿಗೆ ಬಂದಿತು. ಆಗ ಹಿಂದೂ ಪತ್ರಿಕೆಯ ಬಾತ್ಮೀದಾರರೊಬ್ಬರು ಸ್ವಾಮೀಜಿಯವರನ್ನು ಸಂದರ್ಶಿಸಿ ಕೆಲವು ಪ್ರಶ್ನೆಗಳನ್ನು ಹಾಕಿ ಅದಕ್ಕೆ ಉತ್ತರವನ್ನು ಪಡೆದುಕೊಂಡು ಅನಂತರ ತಮ್ಮ ಪತ್ರಿಕೆಯಲ್ಲಿ ಪ್ರಕಟಿಸಿದರು. ಅದನ್ನು ಕೆಳಗೆ ಕೊಡುವೆವು: 

 ಪ್ರಶ್ನೆ: “ಸ್ವಾಮೀಜಿ, ನಿಮ್ಮನ್ನು ಯಾವುದು ಅಮೇರಿಕಾದೇಶಕ್ಕೆ ಹೋಗುವಂತೆ ಮಾಡಿತು?” 

 ಸ್ವಾಮಿ: “ಇದಕ್ಕೆ ಸಂಕ್ಷೇಪವಾಗಿ ಉತ್ತರ ಕೊಡುವುದು ಬಹಳ ಕಷ್ಟ. ಅದಕ್ಕೆ ಈಗ ಸ್ವಲ್ಪ ಮಾತ್ರ ಉತ್ತರ ಕೊಡಬಲ್ಲೆ. ನಾನು ಇಂಡಿಯಾ ದೇಶವನ್ನೆಲ್ಲಾ ಸಂಚಾರ ಮಾಡಿದಮೇಲೆ ಬೇರೆ ದೇಶಗಳನ್ನು ನೋಡಬೇಕೆನಿಸಿತು. ಆದಕಾರಣವೇ ಜಪಾನಿನ ಮೂಲಕ ಅಮೇರಿಕಾ ದೇಶಕ್ಕೆ ಹೋದೆ,” 

 ಪ್ರಶ್ನೆ: “ನೀವು ಜಪಾನಿನಲ್ಲಿ ಏನನ್ನು ನೋಡಿದಿರಿ? ಭರತಖಂಡವೂ ಜಪಾನಿನಂತೆ ಪ್ರಗತಿಪರವಾದ ಹಾದಿಯಲ್ಲಿ ಹೋಗುವ ಸಂಭವ ಉಂಟೆ?” 

 ಸ್ವಾಮೀಜಿ: “ಇಂಡಿಯಾ ದೇಶದ ಮೂವತ್ತು ಕೋಟಿ ಜನರೆಲ್ಲ ಒಟ್ಟಿಗೆ ಕಲೆತರೆ ಮಾತ್ರ ಸಾಧ್ಯ, ಇಲ್ಲದೇ ಇದ್ದರೆ ಇಲ್ಲ. ಅವರಲ್ಲಿರುವ ವೈಶಿಷ್ಟ್ಯ ಇದು. ಯೂರೋಪಿನಲ್ಲಿ ಮತ್ತು ಇತರ ದೇಶಗಳಲ್ಲಿ ಕಲಾಭವನ ಮತ್ತು ಕೊಳೆ ಒಟ್ಟಿಗೆ ಹೋಗುವುದು. ಆದರೆ ಜಪಾನಿನಲ್ಲಿ ಕಲೆ ಮತ್ತು ಶುಭ್ರತೆ ಒಟ್ಟಿಗೆ ಇರುವುವು. ನಮ್ಮ ಜನರು ತಮ್ಮ ಜೀವನ ಕಾಲದಲ್ಲಿ ಒಮ್ಮೆಯಾದರೂ ಜಪಾನನ್ನು ನೋಡಿ ಬರಲಿ ಎಂದು ನಾನು ಆಶೀಸುತ್ತೆನೆ. ಅಲ್ಲಿಗೆ ಹೋಗುವುದು ಬಹಳ ಸುಲಭ. ಜಪಾನೀಯರು ಹಿಂದೂಗಳಿಗೆ ಸಂಬಂಧಪಟ್ಟದ್ದನ್ನೆಲ್ಲ ಮಹತ್ತರವಾದುದು ಎಂದು ಭಾವಿಸುವರು. ಭರತಖಂಡವನ್ನು ಪುಣ್ಯಭೂಮಿ ಎಂದು ಭಾವಿಸುವರು. ಜಪಾನಿನಲ್ಲಿರುವ ಬೌದ್ಧಧರ್ಮ ಸಿಲೋನಿನ ಬೌದ್ಧಧರ್ಮದಂತೆ ಅಲ್ಲ. ಅದು ವೇದಾಂತದಂತೆಯೆ ಇದೆ. ಇದು ಈಶ್ವರನನ್ನು ಒಪ್ಪಿಕೊಳ್ಳುವ ಸ್ಪಷ್ಟವಾದ ಸಿದ್ಧಾಂತ; ಸಿಲೋನಿನಲ್ಲಿರುವ ನಿಷೇಧಾತ್ಮಕವಾದ ನಿರೀಶ್ವರವಾದವಲ್ಲ.” 

 ಪ್ರಶ್ನೆ: “ಜಪಾನಿನ ಹಠಾತ್ ಅಭಿವೃದ್ಧಿಗೆ ಕಾರಣವೇನು? “ 

 ಸ್ವಾಮೀಜಿ: “ಜಪಾನೀಯರಲ್ಲಿ ಇರುವ ಆತ್ಮಶ್ರದ್ಧೆ ಮತ್ತು ಅವರ ದೇಶ ಪ್ರೇಮ. ಭರತಖಂಡದಲ್ಲಿ ಸಂಪೂರ್ಣ ನಿಸ್ವಾರ್ಥಪರರಾಗಿ, ದೇಶಕ್ಕಾಗಿ ತಮ್ಮ ಸರ್ವಸ್ವವನ್ನೂ ಧಾರೆಯೆರೆಯಬಲ್ಲ ಮಹಾವ್ಯಕ್ತಿಗಳು ಜನಿಸಿದರೆ, ಆಗ ಭರತಖಂಡ ಪ್ರತಿಯೊಂದು ಕಾರ್ಯಕ್ಷೇತ್ರದಲ್ಲಿಯೂ ಪ್ರಖ್ಯಾತವಾಗುವುದು. ಬರಿಯ ದೇಶದಲ್ಲಿ ಏನಿದೆ? ಸಮಾಜದಲ್ಲಿ ಮತು ರಾಜಕೀಯದಲ್ಲಿ ಅವರ ನೀತಿಯನ್ನು ನೀವು ಅನುಸರಿಸಿದರೆ ನೀವು ಅವರಷ್ಟೇ ಪ್ರಖ್ಯಾತರಾಗುವಿರಿ. ಆದರೆ ಹಾಗಿಲ್ಲ, ನೀವು ಹಾಗೆ ಅಗಲಾರಿರಿ. ನೀವು ನಿಮ್ಮ ಸಂಸಾರಕ್ಕೆ ಮತ್ತು ಆಸ್ತಿಗೆ ಮಾತ್ರ ಸರ್ವಸ್ವವನ್ನೂ ಧಾರೆಯೆರೆಯಬಲ್ಲಿರಿ.” 

 ಪ್ರಶ್ನೆ: “ಭರತಖಂಡ ಜಪಾನಿನಂತೆ ಆಗಬೇಕೆಂದು ನಿಮ್ಮ ಇಚ್ಛೆಯೇ?” 

 ಸ್ವಾಮೀಜಿ: “ಎಂದಿಗೂ ಇಲ್ಲ. ಭರತಖಂಡ ಹಿಂದೆ ಇದ್ದಂತೆ ಇರಬೇಕು. ಇಂಡಿಯಾ ದೇಶ ಜಪಾನಿನಂತೆ ಹೇಗೆ ಆಗಬಲ್ಲದು? ಅಥವಾ ಒಂದು ದೇಶ ಮತ್ತೊಂದು ದೇಶದಂತೆ ಹೇಗೆ ಆಗಬಲ್ಲದು? ಪ್ರತಿಯೊಂದು ದೇಶದಲ್ಲಿಯೂ ಸಂಗೀತದಲ್ಲಿರುವಂತೆ ಒಂದು ಮುಖ್ಯ ಸ್ವರವಿದೆ. ಪ್ರತಿಯೊಂದೂ ಅದಕ್ಕೆ ಅಧೀನ. ಭರತಖಂಡದ ಆದರ್ಶ ಧರ್ಮ, ಸಾರ್ವಜನಿಕ ಮತ್ತು ಇತರ ಸುಧಾರಣೆಗಳೆಲ್ಲ ಗೌಣ. ಆದಕಾರಣ ಇಂಡಿಯಾ ಜಪಾನಿನಂತೆ ಆಗಲಾರದು. ಹೃದಯ ಕರಗಿದಾಗ ಭಾವ ಹೊರಹೊಮ್ಮುವುದು ಎಂದು ಹೇಳುವರು. ಭರತಖಂಡದ ಹೃದಯ ಕರಗಬೇಕು, ಆಧ್ಯಾತ್ಮಿಕ ಪ್ರವಾಹ ಅಲ್ಲಿಂದ ಹರಿಯುವುದು. ಭರತಖಂಡ ಭರತಖಂಡವೆ. ನಾವು ಜಪಾನೀಯರಲ್ಲ, ಹಿಂದೂಗಳು. ಭರತಖಂಡದ ವಾತಾವರಣ ಬೆಂದ ಜೀವಕ್ಕೆ ಭರವಸೆಯನ್ನು ನೀಡುವುದು. ನಾನು ಇಲ್ಲಿ ಬಿಡುವಿಲ್ಲದ ಕೆಲಸದಲ್ಲಿ ನಿರತನಾಗಿರುವೆನು. ಈ ಕೆಲಸದ ಮಧ್ಯೆ ನನಗೆ ವಿರಾಮ ದೊರಕುವುದು. ನೀವು ಪ್ರಾಪಂಚಿಕ ಕೆಲಸದಲ್ಲಿ ನಿರತರಾದರೆ ಮಧು ಮೇಹದಿಂದ (\enginline{diabetes}) ಸಾಯುವಿರಿ.” 

 ಪ್ರಶ್ನೆ: “ಇಷ್ಟು ಜಪಾನಿನ ಕಥೆಯಾಯಿತು. ಅಮೇರಿಕಾ ದೇಶದಲ್ಲಿ ನಿಮ್ಮ ಪ್ರಥಮ ಅನುಭವ ಹೇಗಿತ್ತು, ಸ್ವಾಮೀಜಿ?” 

 ಸ್ವಾಮೀಜಿ: “ಮೊದಲಿನಿಂದ ಕೊನೆಯವರೆಗೂ ಅದು ತುಂಬಾ ಚೆನ್ನಾಗಿತ್ತು. ಮಿಷನರಿಗಳು ಮತ್ತು ‘ಕೆಲವು ಚರ್ಚಿನ ಹೆಂಗಸರು’ ಇವರನ್ನು ಬಿಟ್ಟರೆ ಉಳಿದವರೆಲ್ಲ ಅತಿಥಿಸತ್ಕಾರಪರರು, ದಯಾಳುಗಳು ಮತ್ತು ಸರಳ ಸ್ವಭಾವದವರು.” 

 ಪ್ರಶ್ನೆ: ನೀವು ಹೇಳಿದ ‘ಚರ್ಚಿನ ಹೆಂಗಸರು’ ಯಾರು, ಸ್ವಾಮೀಜಿ?” 

 ಸ್ವಾಮೀಜಿ: “ಹೆಂಗಸು ಒಬ್ಬ ಗಂಡನನ್ನು ಹುಡುಕುವುದಕ್ಕಾಗಿ ಬೇಕಾದಷ್ಟು ಪ್ರಯತ್ನ ಮಾಡುವಳು. ಅವಳು ಸಮುದ್ರತೀರದಲ್ಲಿರುವ ಎಲ್ಲಾ ಫ್ಯಾಷನಬಲ್ ಹೋಟೇಲುಗಳಿಗೆ ಹೋಗಿ ತನ್ನ ಬುದ್ಧಿವಂತಿಕೆಯನ್ನು ಮತ್ತು ಉಪಾಯವನ್ನೆಲ್ಲ ಒಬ್ಬ ಗಂಡನನ್ನು ಪಡೆಯುವುದಕ್ಕೆ ಉಪಯೋಗಿಸುವಳು. ಕೊನೆಗೆ ಅವಳು ತನ್ನ ಪ್ರಯತ್ನದಲ್ಲಿ ವಿಫಲಳಾದಗ \enginline{old maid} ಎಂದರೆ ಪ್ರಾಯವಿಳಿದ ಹೆಂಗಸಾಗಿ ಚರ್ಚಿಗೆ ಸೇರುವಳು, ಅವರಲ್ಲಿ ಕೆಲವರು ‘ಚರ್ಚಿ’ಗಳಾಗುವರು. ಈ ‘ಚರ್ಚಿನ ಹೆಂಗಸರ’ ಮತಭ್ರಾಂತಿ ಹೇಳತೀರದು. ಅವರು ಅಲ್ಲಿ ಪಾದ್ರಿಯ ಕೈಕೆಳಗೆ ಇರುವರು. ಅವರು ಮತ್ತು ಪಾದ್ರಿಗಳು ಒಟ್ಟಿಗೆ ಸೇರಿ ಪ್ರಪಂಚವನ್ನು ನರಕಸದೃಶವನ್ನಾಗಿ ಮಾಡುವರು, ಧರ್ಮವನ್ನೆಲ್ಲಾ ಹಾಳುಮಾಡುವರು. ಅವರನ್ನು ಬಿಟ್ಟರೆ ಅಮೇರಿಕಾದವರು ಬಹಳ ಒಳ್ಳೆಯವರು. ಅವರು ನನ್ನನ್ನು ಪ್ರೀತಿಸುತ್ತಿದ್ದರು. ನಾನು ಅವರನ್ನು ವಿಶೇಷವಾಗಿ ಪ್ರೀತಿಸುವೆನು, ನಾನೂ ಅವರಲ್ಲಿ ಒಬ್ಬನೆಂದು ಭಾವಿಸಿದ್ದೆ.” 

 ಪ್ರಶ್ನೆ: “ವಿಶ್ವಧರ್ಮ ಸಮ್ಮೇಳನದ ಪರಿಣಾಮವಾಗಿ ನಿಮ್ಮ ಅಭಿಪ್ರಾಯವೇನು?” 

 ಸ್ವಾಮೀಜಿ: “ನನಗೆ ಕಂಡಂತೆ ಅದು ಪ್ರಪಂಚಕ್ಕೆ ಕ್ರೈಸ್ತೇತರ ಧರ್ಮಗಳ ಲೋಪದೋಷವನ್ನು ತೋರುವುದಕ್ಕೆ ಕಟ್ಟಿದ್ದ ನಾಟಕ. ಆದರೆ ಕ್ರೈಸ್ತೇತರ ಧರ್ಮಗಳೇ ಪ್ರಬಲವಾಗಿ ಅದೊಂದು ಕ್ರೈಸ್ತೇತರ ನಾಟಕವೇ ಆಯಿತು. ಕ್ರೈಸ್ತರ ದೃಷ್ಟಿಯಿಂದ ವಿಶ್ವಧರ್ಮ ಸಮ್ಮೇಳನ ನಿಷ್ಪ್ರಯೋಜಕವಾಯಿತು. ಮತ್ತೊಂದು ವಿಶ್ವಧರ್ಮವನ್ನು ಪ್ಯಾರಿಸ್ಸಿನಲ್ಲಿ ನಡೆಸಬೇಕೆಂದಾಗ ರೋಮನ್ ಕ್ಯಾಥೋಲಿಕರು ಅದನ್ನು ವಿರೋಧಿಸುತ್ತಿರುವರು. ಆದರೆ ಚಿಕಾಗೋ ವಿಶ್ವಧರ್ಮ ಸಮ್ಮೇಳನ, ಭರತಖಂಡದ ಮತ್ತು ಭಾರತೀಯ ಭಾವನೆಯ ದೃಷ್ಟಿಯಿಂದ ಯಶಸ್ವಿಯಾಯಿತು. ಅದು ವೇದಾಂತದ ಭಾವನೆ ಪ್ರಪಂಚಕ್ಕೆಲ್ಲ ಹರಡುವಂತೆ ಮಾಡಿತು. ಕ್ರೈಸ್ತ ಪಾದ್ರಿಗಳು ಮತ್ತು ಚರ್ಚಿನ ಹೆಂಗಸು ಬಿಟ್ಟರೆ ಉಳಿದ ಅಮೇರಿಕಾದವರಿಗೆ ವಿಶ್ವಧರ್ಮ ಸಮ್ಮೇಳನದ ಪರಿಣಾಮದಿಂದ ಸಂತೋಷವೇ ಆಗಿದೆ.” 

 ಪ್ರಶ್ನೆ: “ಇಂಗ್ಲೆಂಡಿನಲ್ಲಿ ನಿಮ್ಮ ಭಾವನೆ ಪ್ರಚಾರವಾಗುವ ಸಂಭವ ಹೇಗಿದೆ ಸ್ವಾಮೀಜಿ?” 

 ಸ್ವಾಮೀಜಿ: “ಬೇಕಾದಷ್ಟು ಅವಕಾಶವಿದೆ. ಕೆಲವು ವರುಷಗಳಲ್ಲಿ ಅನೇಕ ಜನ ಇಂಗ್ಲೀಷರು ವೇದಾಂತಿಗಳಾಗುವರು. ಅಮೇರಿಕಾ ದೇಶಕ್ಕಿಂತ ಇಲ್ಲಿ ಹೆಚ್ಚು ಅವಕಾಶವಿದೆ. ಅಮೇರಿಕಾ ದೇಶದಲ್ಲಿ ಬೇಕಾದಷ್ಟು ಡಂಭವಿದೆ. ಅದು ಇಂಗ್ಲೀಷಿನವರಲ್ಲಿ ಇಲ್ಲ. ಕ್ರೈಸ್ತರೂ ಕೂಡ ವೇದಾಂತವನ್ನು ಓದದೆ ನ್ಯೂಟೆಸ್ಟಮೆಂಟನ್ನು ಅರ್ಥಮಾಡಿಕೊಳ್ಳಲಾರರು. ವೇದಾಂತವೇ ಎಲ್ಲಾ ಧರ್ಮಗಳ ತತ್ತ್ವ. ವೇದಾಂತ ತತ್ತ್ವವಿಲ್ಲದೇ ಇದ್ದರೆ ಧರ್ಮಗಳೆಲ್ಲ ಮೂಢನಂಬಿಕೆ, ವೇದಾಂತವಿದ್ದರೆ ಅವುಗಳೆಲ್ಲ ಧರ್ಮವಾಗುವುವು.” 

 ಪ್ರಶ್ನೆ: “ಆಂಗ್ಲೇಯರ ಶೀಲದಲ್ಲಿ ನೀವು ಯಾವ ವಿಶೇಷವನ್ನು ಕಂಡಿರಿ?” 

 ಸ್ವಾಮೀಜಿ: “ಆಂಗ್ಲೇಯನು ಏನನ್ನಾದರೂ ನಂಬುವುದೇ ತಡ ಅದನ್ನು ಕಾರ್ಯಗತಮಾಡಲು ಯತ್ನಿಸುವನು. ಕೆಲಸ ಮಾಡುವುದಕ್ಕೆ ಅವನಲ್ಲಿ ಪ್ರಚಂಡ ಶಕ್ತಿಯಿದೆ. ಆಂಗ್ಲೇಯ ಸ್ತ್ರೀಯರನ್ನು ಅಥವಾ ಪುರುಷರನ್ನು ಪ್ರಪಂಚದಲ್ಲಿ ಯಾರೂ ಮೀರಲಾರರು. ಆದಕಾರಣವೆ ನಾನು ಅವರಲ್ಲಿ ಶ್ರದ್ಧೆಯನ್ನು ಇಡುವುದು. ಜಾನ್‌ಬುಲ್ (ಇಂಗ್ಲೀಷಿನವ) ವಿಷಯವನ್ನು ಗ್ರಹಿಸುವುದು ಸ್ವಲ್ಪ ನಿಧಾನ. ಅವನಿಗೆ ಪದೇ ಪದೇ ಒಂದು ವಿಷಯವನ್ನು ಒತ್ತಿ ಹೇಳುತ್ತಿರಬೇಕು. ಆದರೆ ಒಮ್ಮೆ ಅದನ್ನು ತಿಳಿದುಕೊಂಡರೆ ಅವನು ಸುಲಭವಾಗಿ ಅದನ್ನು ಬಿಡುವುದಿಲ್ಲ. ಇಂಗ್ಲೆಂಡಿನಲ್ಲಿ ಯಾವ ಮಿಷನರಿಯಾಗಲಿ, ಇತರರಾಗಲಿ ನನಗೆ ವಿರೋಧವಾಗಿ ಏನನ್ನೂ ಹೇಳಲಿಲ್ಲ. ಆಶ್ಚರ‍್ಯವೇನೆಂದರೆ, ನನ್ನ ಅನೇಕ ಸ್ನೇಹಿತರು ಇಂಗ್ಲೆಂಡಿನ ಚರ್ಚಿಗೆ ಸೇರಿದವರು. ಈ ಮಿಷನರಿಗಳು ಇಂಗ್ಲೆಂಡಿನ ಮೇಲಿನ ವರ್ಗದವರಿಂದ ಬಂದವರು ಎಂದು ಕೇಳಿದೆ. ಜಾತಿ ಇಲ್ಲಿರುವಷ್ಟೇ ಅಲ್ಲಿಯೂ ಬಹಳ ಕಟ್ಟುನಿಟ್ಟು. ಆಂಗ್ಲೇಯ ಪಾದ್ರಿಗಳು ಭದ್ರ ಮನುಷ್ಯರ ಗುಂಪಿಗೆ ಸೇರಿದವರು. ನಿಮ್ಮಂತೆ ಅವರ ಅಭಿಪ್ರಾಯ ಇಲ್ಲದೆ ಇರಬಹುದು. ಆದರೆ ಇದರಿಂದ ನಮ್ಮೊಡನೆ ಸ್ನೇಹಿತರಾಗುವುದಕ್ಕೆ ಯಾವ\break ಅಭ್ಯಂತರವೂ ಇರುವುದಿಲ್ಲ. ಆದಕಾರಣ ನಮ್ಮ ದೇಶದವರಿಗೆ ನಾನು ನೀಡುವ ಹಿತವಚನವಿದು: ನಿಂದಿಸುತ್ತಿರುವ ಪಾದ್ರಿಗಳನ್ನು ಗಮನಿಸಬೇಡಿ. ಅವರೇನು ಎಂಬುವುದು ನನಗೆ ಗೊತ್ತಿದೆ. ಅಮೇರಿಕಾದವರು ಹೇಳುವಂತೆ ಅವರ ಸ್ಥಾನವನ್ನು ಅವರಿಗೆ ಕೊಟ್ಟಾಗಿದೆ. ಅವರನ್ನು ಲಕ್ಷ್ಯಕ್ಕೆ ತರದಿರುವ ಸ್ವಭಾವ ನಮ್ಮದಾಗಬೇಕು.” 

 ಪ್ರಶ್ನೆ: “ ಅಮೇರಿಕ ಮತ್ತು ಇಂಗ್ಲೆಂಡಿನ ಸಮಾಜ ಸುಧಾರಣೆ ವಿಷಯದಲ್ಲಿ ನಿಮಗೆ ತಿಳಿದಿರುವುದನ್ನು ಹೇಳುತ್ತೀರಾ, ಸ್ವಾಮೀಜಿ?” 

 ಸ್ವಾಮೀಜಿ: “ಆಗಲಿ, ಅಲ್ಲಿ ಈಗ ಆಗುತ್ತಿರುವ ಸಮಾಜದ ಕ್ರಾಂತಿಯು ಅದರಲ್ಲೂ ಅದರ ಮುಂದಾಳುಗಳು, ಅವರ ತಮ್ಮ ಸಮತಾವಾದ ಸಿದ್ಧಾಂತಕ್ಕೆ ಒಂದು ಆಧ್ಯಾತ್ಮಿಕ ತಳಹದಿ ಬೇಕೆಂದು ಮನಗಂಡಿರುವರು. ಆ ತಳಹದಿ ವೇದಾಂತದಲ್ಲಿ ಮಾತ್ರ ದೊರಕುವುದು. ನನ್ನ ಉಪನ್ಯಾಸವನ್ನು ಕೇಳುವುದಕ್ಕೆ ಬರುತ್ತಿದ್ದ ಅನೇಕ ಮುಂದಾಳುಗಳು ಹೊಸ ಜಾಗೃತಿಗೆ ತಳಹದಿಯಾಗಿ ವೇದಾಂತ ಬೇಕಾಗಿದೆ ಎಂದು ಹೇಳುತ್ತಿದ್ದರು.” 

 ಪ್ರಶ್ನೆ: “ಭರತಖಂಡದ ಜನಸಾಮಾನ್ಯರ ವಿಷಯವಾಗಿ ನಿಮ್ಮ ಅಭಿಪ್ರಾಯವೇನು?” 

 ಸ್ವಾಮೀಜಿ: “ಓ, ನಮ್ಮ ದಾರಿದ್ರ್ಯವನ್ನು ಹೇಳತೀರದು. ನಮ್ಮ ಜನರಿಗೆ ಲೌಕಿಕ ವಿಷಯವೇ ತಿಳಿಯದು. ನಮ್ಮ ಜನಸಾಮಾನ್ಯರು ಒಳ್ಳೆಯ ಸ್ವಭಾವದವರು. ಏಕೆಂದರೆ ಬಡತನ ಎಂಬುದು ಇಲ್ಲಿ ಒಂದು ಅಪರಾಧವಲ್ಲ. ನಮ್ಮ ಜನರು ಹಿಂಸಾ ಪ್ರವೃತ್ತಿಯವರಲ್ಲ. ಅನೇಕ ವೇಳೆ ಅಮೇರಿಕಾದಲ್ಲಿ ಮತ್ತು ಇಂಗ್ಲೆಂಡಿನಲ್ಲಿ ಜನರು ನನ್ನ ಉಡಿಗೆ ತೊಡಿಗೆಗಾಗಿ ನನ್ನನ್ನು ಸುತ್ತು ಗುಟ್ಟುತ್ತಿದ್ದರು. ಬೇರೆ ವಿಷಯಗಳಲ್ಲಿ ಸಾಮಾನ್ಯ ಜನರು ಹೆಚ್ಚು ಸುಸಂಸ್ಕೃತರು.” 

 ಪ್ರಶ್ನೆ: “ನಮ್ಮ ಜನಸಾಮಾನ್ಯರನ್ನು ಮೇಲೆತ್ತಲು ನೀವು ಏನು ಸಲಹೆಯನ್ನು ಕೊಡುತ್ತೀರಿ?” 

 ಸ್ವಾಮೀಜಿ: “ಅವರಿಗೆ ಲೌಕಿಕ ಶಿಕ್ಷಣ ಕೊಡಬೇಕು. ನಮ್ಮ ಪೂರ್ವಿಕರ ಮಾರ್ಗವನ್ನು ಅನುಸರಿಸಬೇಕು. ಆದರೆ ಆದರ್ಶವನ್ನು ಕ್ರಮೇಣ ಎಲ್ಲರಿಗೂ ತಿಳಿಯುವಂತೆ ಮಾಡಬೇಕು. ಲೌಕಿಕ ವಿದ್ಯೆಯನ್ನು ಕೂಡ ಧರ್ಮದ ಮೂಲಕ ಬೋಧಿಸಿ.” 

 ಪ್ರಶ್ನೆ: “ಸ್ವಾಮೀಜಿ, ಇದೇನು ಸುಲಭವಾಗಿ ಸಾಧಿಸುವ ಕೆಲಸ ಎಂದು ಭಾವಿಸಿದ್ದೀರಾ?” 

 ಸ್ವಾಮೀಜಿ: “ಇದನ್ನು ನಾವು ಕ್ರಮೇಣ ಮಾಡಬೇಕಾಗಿದೆ. ನನ್ನೊಡನೆ ಕೆಲಸ ಮಾಡಬಲ್ಲ, ಸ್ವಾರ್ಥತ್ಯಾಗ ಮಾಡಬಲ್ಲ ತರುಣರಿದ್ದರೆ ನಾಳೆಯೇ ಇದನ್ನು ಮಾಡಬಹುದು. ಕೆಲಸಕ್ಕೆ ನಾವು ತರುವ ಉತ್ಸಾಹ ಮತ್ತು ತ್ಯಾಗದ ಮೇಲೆ ಇದು ನಿಂತಿದೆ.”‌ 

 ಪ್ರಶ್ನೆ: “ಆದರೆ ಈಗಿನ ಸ್ಥಿತಿ ಅವರ ಹಿಂದಿನ ಕರ್ಮದ ಫಲವಾದರೆ, ಅವರು ಅಷ್ಟು ಸುಲಭವಾಗಿ ಹೇಗೆ ಇದರಿಂದ ಪಾರಾದಾರು? ನೀವು ಅವರಿಗೆ ಹೇಗೆ ಸಹಾಯ ಮಾಡಬಲ್ಲಿರಿ?” 

 ಸ್ವಾಮೀಜಿ: (ತಕ್ಷಣಉತ್ತರ ಕೊಟ್ಟರು) “ಕರ್ಮ ಮಾನವನ ಸ್ವಾತಂತ್ರ್ಯವನ್ನು ನಿತ್ಯವೂ ತೋರುತ್ತದೆ. ನಾವು ನಮ್ಮ ಕರ್ಮದಿಂದ ಅಧೋಗತಿಗೆ ಬಂದಿದ್ದರೆ, ಪುನಃ ಕರ್ಮದಿಂದಲೇ ಉತ್ತಮಸ್ಥಿತಿಗೂ ಬರುವುದರಲ್ಲಿ ಯಾವ ಸಂದೇಹವೂ ಇಲ್ಲ.” 

 “ಜನಸಾಮಾನ್ಯರೇ ತಮ್ಮ ಕರ್ಮದಿಂದ ಈ ಸ್ಥಿತಿಗೆ ಬರಲಿಲ್ಲ. ಆದಕಾರಣ ಅವರು ಕೆಲಸ ಮಾಡುವುದಕ್ಕೆ ಒಳ್ಳೆಯ ವಾತಾವರಣವನ್ನು ಕಲ್ಪಿಸಿಕೊಡಬೇಕು. ನಾವು ಈ ಮಾರ್ಗವನ್ನೇ ಅನುಸರಿಸಬೇಕಾಗಿದೆ. ನಿಜವಾಗಿ ಜಾತಿ ಎಂದರೆ ಏನೆಂದು ಹತ್ತು ಲಕ್ಷದಲ್ಲಿ ಒಬ್ಬರಿಗೂ ತಿಳಿಯದು. ಪ್ರಪಂಚದಲ್ಲಿ ಜಾತಿ ಇಲ್ಲದ ದೇಶವೇ ಇಲ್ಲ. ಇಂಡಿಯಾ ದೇಶದಲ್ಲಿ ಜಾತಿಯಿಂದ ಜಾತ್ಯತೀತ ಸ್ಥಿತಿಗೆ ನಾವು ಹೋಗುತ್ತೇವೆ. ಜಾತಿ ಎಲ್ಲ ಕಡೆಗಳಲ್ಲೂ ಈ ಸಿದ್ಧಾಂತದ ಮೇಲೆ ನಿಂತಿರುವುದು. ಇಂಡಿಯಾ ದೇಶದಲ್ಲಿ ಪ್ರತಿಯೊಬ್ಬರನ್ನೂ ಬ್ರಾಹ್ಮಣರನ್ನಾಗಿ ಮಾಡಬೇಕು. ಇದೇ ನಮ್ಮ ಗುರಿ. ಬ್ರಾಹ್ಮಣನೇ ಮಾನವನ ಆದರ್ಶ. ನೀವು ಇಂಡಿಯಾದೇಶದ ಚರಿತ್ರೆಯನ್ನು ಓದಿದರೆ ಅನೇಕ ವೇಳೆ ಜನಸಾಮಾನ್ಯರನ್ನು ಮೇಲಕ್ಕೆ ಎತ್ತಿರುವ ಪ್ರಯತ್ನ ಮಾಡಿರುವುದು ಕಾಣಿಸುವುದು. ಎಷ್ಟೋ ಜಾತಿಗಳನ್ನು ಮೇಲಕ್ಕೆ ಎತ್ತಿರುವರು. ಎಲ್ಲರೂ ಬ್ರಾಹ್ಮಣರಾಗುವವರೆಗೆ ಇನ್ನೂ ಹಲವು ಪ್ರಯತ್ನಗಳು ನಡೆಯುವುವು. ಇದೇ ನಮ್ಮ ಯೋಜನೆ. ನಾವು ಯಾರನ್ನೂ ಕೆಳಗೆ ಎಳೆಯದೆ ಎಲ್ಲರನ್ನೂ ಮೇಲಕ್ಕೆ ಎತ್ತಬೇಕಾಗಿದೆ. ಹೆಚ್ಚಿನ ಮಟ್ಟಿಗೆ ಇದನ್ನು ಬ್ರಾಹ್ಮಣರು ಮಾಡಬೇಕಾಗಿದೆ. ಅವರು ಎಷ್ಟು ಬೇಗ ಅದನ್ನು ಮಾಡಿಕೊಂಡರೆ ಅಷ್ಟು ಎಲ್ಲರಿಗೂ ಅದು ಒಳ್ಳೆಯದು. ಕಾಲವನ್ನು ವಿಳಂಬ ಮಾಡಕೂಡದು. ಯೂರೋಪ್ ಮತ್ತು ಅಮೇರಿಕಾ ದೇಶದಲ್ಲಿರುವ ಜಾತಿಗಿಂತ ಇಂಡಿಯಾದೇಶದ ಜಾತಿ ಮೇಲು. ಇದು ಎಲ್ಲಾ ಒಳ್ಳೆಯದು ಎಂದು ನಾನು ಹೇಳುವುದಿಲ್ಲ. ಜಾತಿ ಇಲ್ಲದೆ ಇದ್ದರೆ ನೀವು ಎಲ್ಲಿ ಇರುತ್ತಿದ್ದಿರಿ? ಜಾತಿ ಇಲ್ಲದೇ ಇದ್ದರೆ ನಿಮ್ಮ ವಿದ್ಯೆ ಮುಂತಾದುವೆಲ್ಲ ಎಲ್ಲಿ ಇರುತ್ತಿದ್ದುವು? ಜಾತಿ ಇಲ್ಲದೆ ಇದ್ದರೆ ಐರೋಪ್ಯರು ತಿಳಿದುಕೊಳ್ಳುವುದಕ್ಕೆ ಇಂಡಿಯಾದೇಶದಲ್ಲಿ ಏನೂ ಇರುತ್ತಿರಲಿಲ್ಲ. ಮಹಮ್ಮದೀಯರು ಎಲ್ಲವನ್ನೂ ಧ್ವಂಸಮಾಡಿಬಿಡುತ್ತಿದ್ದರು. ಹಿಂದೂ ಸಮಾಜ ಎಂದು ಬದಲಾಗದೆ ಇತ್ತು? ಅದು ಯಾವಾಗಲೂ ಬದಲಾಗುತ್ತಿದೆ. ಕೆಲವು ವೇಳೆ ಪರದೇಶದವರು ಧಾಳಿಯನ್ನು ಇಟ್ಟಾಗ ಚಲನೆ ಬಹಳ ಮಂದವಾಗಿತ್ತು. ಇತರ ಕಾಲಗಳಲ್ಲಿ ಅದು ಚುರುಕಾಗಿತ್ತು. ನಮ್ಮ ದೇಶದವರಿಗೆ ಇದನ್ನೇ ನಾನು ಹೇಳುವುದು. ನಾನು ಜಾತಿಯನ್ನು ದ್ವೇಷಿಸುವುದಿಲ್ಲ. ಅವು ಹಿಂದೆ ಏನನ್ನು ಸಾಧಿಸಿವೆ ಎಂಬುದನ್ನು ನೋಡಿರುವೆನು. ಆ ಸ್ಥಿತಿಯಲ್ಲಿ ಮತ್ತಾವ ಜನಾಂಗವು ಅದಕ್ಕಿಂತ ಚೆನ್ನಾಗಿ ಏನನ್ನೂ ಮಾಡಿರಲಾರದು. ಅವರು ಚೆನ್ನಾಗಿಯೇ ಮಾಡಿರುವರು ಎಂದು ನಾನು ಹೇಳುತ್ತೇನೆ. ಮತ್ತೂ ಚೆನ್ನಾಗಿ ಮಾಡಿ ಎಂದು ಮಾತ್ರ ಹೇಳುತ್ತೇನೆ.” 

 ಪ್ರಶ್ನೆ: “ವರ್ಣಕ್ಕೂ ಆಚಾರಕ್ಕೂ ಇರುವ ಸಂಬಂಧದಲ್ಲಿ ನಿಮ್ಮ ಅಭಿಪ್ರಾಯವೇನು ಸ್ವಾಮೀಜಿ?” 

 ಸ್ವಾಮೀಜಿ: “ವರ್ಣಗಳು ಯಾವಾಗಲೂ ಬದಲಾಗುತ್ತಿವೆ. ಅದರಂತೆಯೇ ಆಚಾರವೂ ಬದಲಾಗುತ್ತಿವೆ. ಅದರಂತೆಯೇ ಅವುಗಳನ್ನು ಮಾಡುವ ರೀತಿ ಕೂಡ. ಅದರ ಹಿಂದಿರುವ ತತ್ತ್ವ ಮತ್ತು ಸತ್ಯ ಮಾತ್ರ ಬದಲಾಗುವುದಿಲ್ಲ. ನಾವು ನಮ್ಮ ಧರ್ಮವನ್ನು ವೇದದಲ್ಲಿ ತಿಳಿದುಕೊಳ್ಳಬೇಕಾಗಿದೆ. ವೇದದ ವಿನಃ ಉಳಿದೆಲ್ಲವೂ ಬದಲಾಗಬೇಕು. ವೇದ ಎಲ್ಲ ಕಾಲಕ್ಕೂ ಪ್ರಮಾಣ. ಇತರ ಶಾಸ್ತ್ರಗಳು ಪ್ರಮಾಣವಾದರೂ ಆಯಾ ಕಾಲಕ್ಕೆ ಮಾತ್ರ. ಉದಾಹರಣೆಗೆ ಒಂದು ಸ್ಮೃತಿ ಒಂದು ಕಾಲದಲ್ಲಿ ಹೆಚ್ಚು ರೂಢಿಯಲ್ಲಿರುವುದು, ಮತ್ತೊಂದು ಕಾಲದಲ್ಲಿ ಮತ್ತೊಂದು ಸ್ಮೃತಿ ಬಳಕೆಯಲ್ಲಿರುವುದು. ಮಹಾತ್ಮರು ಅನೇಕ ವೇಳೆ ಬಂದು ಯಾವ ಮಾರ್ಗದಲ್ಲಿ ನಾವು ಹೋಗಬೇಕೆಂಬುದನ್ನು ತೋರುತ್ತಿರುವರು. ಕೆಲವು ಮಹಾತ್ಮರು ಅಂತ್ಯಜರಿಗೆ ಅನುಕೂಲವನ್ನು ಮಾಡಿಕೊಟ್ಟರು. ಮಾಧ್ವರಂತಹ ಕೆಲವರು ಸ್ತ್ರೀಯರಿಗೂ ವೇದವನ್ನು ಓದಲು ಅಧಿಕಾರವಿದೆ ಎಂದು ಹೇಳಿದರು. ವರ್ಣ ಹೋಗಬೇಕಿಲ್ಲ, ಆದರೆ ಅದು ಕಾಲಕ್ಕೆ ತಕ್ಕಂತೆ ಹೊಂದಿಕೊಂಡು ಹೋಗಬೇಕು ಅಷ್ಟೆ. ಆ ಹಳೆಯದರೊಳಗೆ ಬೇಕಾದರೆ ಎರಡು ಲಕ್ಷ ಜಾತಿಗಳನ್ನು ಮಾಡುವಷ್ಟು ಸಾಮಗ್ರಿ ಇದೆ. ವರ್ಣವನ್ನು ಧ್ವಂಸ ಮಾಡುವುದು ತಿಳಿಗೇಡಿತನ. ಹಳೆಯದು ರೂಪಾಂತರ ಹೊಂದಬೇಕು. ಅದೇ ಹೊಸ ಮಾರ್ಗ.” 

 ಪ್ರಶ್ನೆ: “ಹಿಂದೂಗಳಿಗೆ ಸಮಾಜ ಸುಧಾರಣೆ ಬೇಡವೆ?” 

 ಸ್ವಾಮೀಜಿ: “ಸಮಾಜ ಸುಧಾರಣೆ ಆವಶ್ಯಕವೇನೋ ನಿಜ. ಹಿಂದೆ ಮಹಾತ್ಮರು ಯಾವ ರೀತಿ ಸಮಾಜ ಮುಂದುವರಿಯಬೇಕೆಂಬುದನ್ನು ಹೇಳುತ್ತಿದ್ದರು. ರಾಜರು ಅದನ್ನು ಒಂದು ಕಾನೂನು ಮಾಡುತ್ತಿದ್ದರು. ಇಂದು ಸಮಾಜ ಸುಧಾರಣೆಯನ್ನು ಜಾರಿಗೆ ತರಬೇಕಾದರೆ ಅದನ್ನು ಮಾಡಬಲ್ಲ ಅಧಿಕಾರಿಗಳನ್ನು ಸೃಷ್ಟಿಸಬೇಕಾಗಿದೆ. ಆದಕಾರಣ ಜನರೆಲ್ಲ ವಿದ್ಯಾವಂತರಾಗುವವರೆಗೆ, ತಮ್ಮ ಅವಶ್ಯಕತೆಗಳನ್ನು ತಾವು ಮನಗಂಡು ಅವುಗಳನ್ನು ಪರಿಹರಿಸುವ ಸ್ಥಿತಿಗೆ ಅವರು ಬರುವ ತನಕ ನಾವು ಕಾಯಬೇಕು. ಅಲ್ಪ ಸಂಖ್ಯಾತರ ಬಲಾತ್ಕಾರವು, ಪ್ರಪಂಚದಲ್ಲೆಲ್ಲ ನಡೆಯುವ ದೊಡ್ಡ ದೌರ್ಜನ್ಯವಾಗಿದೆ. ಆದಕಾರಣ ಆದರ್ಶ, ಸುಧಾರಣೆಯ ವಿಷಯದಲ್ಲಿ ನಮ್ಮ ಶಕ್ತಿಯನ್ನು ವ್ಯಯ ಮಾಡುವುದಕ್ಕಿಂತ, (ಅದೆಂದಿಗೂ ಕಾರ್ಯಗತವಾಗುವುದಿಲ್ಲ), ಸಮಸ್ಯೆಯ ಮೂಲಕ್ಕೆ ಹೋಗೋಣ. ಶಾಸನ ಮಾಡತಕ್ಕ ವ್ಯಕ್ತಿಗಳನ್ನು ತರಬೇತು ಮಾಡೋಣ. ಅಂದರೆ ತಮ್ಮ ಸಮಸ್ಯೆಯನ್ನು ತಾವೇ ಪರಿಹರಿಸಿಕೊಳ್ಳುವುದಕ್ಕೆ ಜನರನ್ನು ವಿದ್ಯಾವಂತರನ್ನಾಗಿ ಮಾಡೋಣ. ಇದಾಗುವವರೆಗೆ ಆದರ್ಶ, ಸುಧಾರಣೆಗಳೆಲ್ಲ ಆದರ್ಶಗಳಾಗಿಯೇ ಉಳಿಯುವುವು. ಜನರು ತಮ್ಮ ವಿಮೋಚನೆಯನ್ನು ತಾವೇ ಮಾಡಿಕೊಳ್ಳಬೇಕಾಗಿದೆ. ಇದೇ ನವವಿಧಾನ. ಹಿಂದೆ ಯಾವಾಗಲೂ ಅರಸರು ಜನರನ್ನು ಆಳುತ್ತಿದ್ದುದರಿಂದ ಇದನ್ನು ಭರತಖಂಡದಲ್ಲಿ ಜಾರಿಗೆ ತರಬೇಕಾದರೆ ಕಾಲ ಹಿಡಿಯುವುದು.” 

 ಪ್ರಶ್ನೆ: “ಹಿಂದೂ ಸಮಾಜ ಐರೋಪ್ಯ ಸಮಾಜದ ರೀತಿನೀತಿಗಳನ್ನು ಯಶಸ್ವಿ ಯಾಗಿ ಅನುಸರಿಸುವುದಕ್ಕೆ ಸಾಧ್ಯವೆಂದು ನೀವು ಭಾವಿಸುವಿರಾ?” 

 ಸ್ವಾಮೀಜಿ: “ಇಲ್ಲ, ನಾವು ಅದನ್ನು ಪೂರ್ತಿ ಅನುಸರಿಸಬೇಕಾಗಿಲ್ಲ. ನಾನು ಸೂಚಿಸುವುದು ಗ್ರೀಕರ ಮನಸ್ಸು, ಅಂದರೆ ಈಗಿನ ಯೂರೋಪಿಯನ್ನರ ಶಕ್ತಿ, ಹಿಂದೂಗಳ ಅಧ್ಯಾತ್ಮದೊಂದಿಗೆ ಸಂಗಮವಾದರೆ ಭರತಖಂಡದಲ್ಲಿ ಅದೊಂದು ಆದರ್ಶ ಸಮಾಜವಾಗುವುದು. ಉದಾಹರಣೆಗೆ, ನಮ್ಮ ಶಕ್ತಿಯನ್ನು ಕೆಲಸಕ್ಕೆ ಬಾರದ ಆದರ್ಶಗಳನ್ನು ಕುರಿತು ಚರ್ಚಿಸಿ ವ್ಯರ್ಥ ಮಾಡಿಕೊಳ್ಳುವುದಕ್ಕಿಂತ ಇಂಗ್ಲೀಷಿನವರಿಂದ ಅವರು ನಾಯಕನಿಗೆ ತೋರುವ ಅಚಲವಾದ ವಿಧೇಯತೆ, ಅಸೂಯೆ ಇಲ್ಲದೆ ಇರುವುದು, ಹಿಡಿದ ಕೆಲಸವನ್ನು ಬಿಡದೆ ಮಾಡುವ ಛಲ ಮತ್ತು ಅಚಲವಾದ ಆತ್ಮಶ್ರದ್ಧೆ ಇವುಗಳನ್ನು ಕಲಿಯುವುದು ಮೇಲು. ಅವರು ಯಾವುದಾದರೂ ಒಂದು ಕೆಲಸಕ್ಕೆ ಒಬ್ಬ ಮುಂದಾಳುವನ್ನು ಗೊತ್ತುಮಾಡಿದರೆ ಜಯಾಪಜಯಗಳಲ್ಲಿ ಯಾವಾಗಲೂ ಅವನನ್ನು ಅನುಸರಿಸುವರು. ಆದರೆ ಇಂಡಿಯಾದೇಶದಲ್ಲಿ ಪ್ರತಿಯೊಬ್ಬನೂ ನಾಯಕನಾಗಲು ಬಯಸುವನು. ಅಪ್ಪಣೆಯನ್ನು ಪಾಲಿಸುವುದಕ್ಕೆ ಯಾರೂ ಇಲ್ಲ. ಪ್ರತಿಯೊಬ್ಬನೂ ಅಪ್ಪಣೆ ಕೊಡುವುದಕ್ಕೆ ಮುಂಚೆ ಆಣತಿಯನ್ನು ಪಾಲಿಸುವುದನ್ನು ಕಲಿಯಬೇಕು. ನಮ್ಮ ಅಸೂಯೆಗೆ ಕೊನೆ ಮೊದಲಿಲ್ಲ. ಹಿಂದೂ ಪ್ರಮುಖನಾದಷ್ಟೂ ಅಸೂಯೆ ಹೆಚ್ಚು. ಅಸೂಯೆ ಇಲ್ಲದಿರುವುದು, ನಾಯಕನ ಅಪ್ಪಣೆಯನ್ನು ಪಾಲಿಸುವುದು ಇವೆರಡನ್ನೂ ಹಿಂದೂಗಳಾದ ನಾವು ಕಲಿಯುವವರೆಗೆ ನಮ್ಮಲ್ಲಿ ಯಾವ ಸಂಘಟನಾ ಶಕ್ತಿಯೂ ಇರುವಹಾಗಿಲ್ಲ. ಈಗಿರುವಂತೆ ಯಾವಾಗಲೂ ಅನೈಕಮತ್ಯದಿಂದ ಚೆಲ್ಲಾಪಿಲ್ಲಿಯಾಗಿರುವ ದೊಂಬಿಯ ಜನರಂತೆ ಇರಬೇಕಾಗುವುದು. ಯಾವಾಗಲೂ ಕನಸು ಕಾಣುವುದು, ಕಾರ್ಯತಃ ಮಾತ್ರ ಏನೂ ಇಲ್ಲ. ಭಾರತೀಯರು ಐರೋಪ್ಯರಿಂದ ಬಾಹ್ಯ ಪ್ರಕೃತಿಯನ್ನು ಹೇಗೆ ನಿಗ್ರಹಿಸಬೇಕು ಎಂಬುದನ್ನು ಕಲಿತುಕೊಳ್ಳಬೇಕು. ಐರೋಪ್ಯರು ಭಾರತೀಯರಿಂದ ಆಂತರಿಕ ಪ್ರಕೃತಿಯನ್ನು ಹೇಗೆ ನಿಗ್ರಹಿಸಬೇಕು ಎಂಬುದನ್ನು ಕಲಿತುಕೊಳ್ಳಬೇಕು. ಆಗ ಹಿಂದೂವು ಇರುವುದಿಲ್ಲ, ಐರೋಪ್ಯರೂ ಇರುವುದಿಲ್ಲ. ಬಾಹ್ಯ ಮತ್ತು ಆಂತರಿಕ ಪ್ರಕೃತಿಯನ್ನು ಗೆದ್ದ ಆದರ್ಶ ಜನಾಂಗ ಒಂದು ಇರುವುದು. ಅವರು ಮಾನವತೆಯ ಒಂದು ಭಾಗದಲ್ಲಿ ಮುಂದುವರಿದಿರುವರು. ಇವೆರಡರ ಸಂಗಮ ನಮಗೆ ಬೇಕಾಗಿರುವುದು. ನಮ್ಮ ಧರ್ಮದ ಪಲ್ಲವಿಯಾದ ಸ್ವಾತಂತ್ರ್ಯ ಎಂಬುದು ದೈಹಿಕ, ಮಾನಸಿಕ ಮತ್ತು ಆಧ್ಯಾತ್ಮಿಕ ಸ್ವಾತಂತ್ರ್ಯ.” 

 ಪ್ರಶ್ನೆ: “ಧರ್ಮಕ್ಕೂ ಆಚಾರಕ್ಕೂ ಏನು ಸಂಬಂಧ?” 

 ಸ್ವಾಮೀಜಿ: “ಆಚಾರಗಳು ಮಕ್ಕಳಿಗೆ ಶಿಶುವಿಹಾರಗಳಿದ್ದಂತೆ. ಈಗಿನ ಸ್ಥಿತಿಯಲ್ಲಿರುವ ಪ್ರಪಂಚಕ್ಕೆ ಇದು ಅತ್ಯಾವಶ್ಯಕ. ನಾವು ಜನರಿಗೆ ಹೊಸ ಹೊಸ ಆಚಾರಗಳನ್ನು ಕೊಡಬೇಕಾಗಿದೆ. ಕೆಲವು ಮೇಧಾವಿಗಳು ಇದನ್ನು ಕುರಿತು ಆಲೋಚಿಸಬೇಕು. ಹಳೆಯ ಆಚಾರಗಳನ್ನು ಬಿಟ್ಟು ಹೊಸ ಆಚಾರಗಳನ್ನು ತೆಗೆದುಕೊಳ್ಳಬೇಕಾಗಿದೆ.” 

 ಪ್ರಶ್ನೆ: “ಹಾಗಾದರೆ ಆಚಾರವನ್ನು ಬಿಡಬೇಕು ಎಂಬುದನ್ನು ನೀವು ಅನುಮೋದಿಸುತ್ತೀರಾ?” 

 ಸ್ವಾಮೀಜಿ: “ನನ್ನ ಸಂದೇಶ ಧ್ವಂಸವಲ್ಲ, ನಿರ್ಮಾಣ. ಈಗಿರುವ ಆಚಾರಗಳಿಂದ ಹೊಸ ಆಚಾರವನ್ನು ಮಾಡಬೇಕು. ಪ್ರತಿಯೊಬ್ಬರಲ್ಲಿಯೂ ವಿಕಾಸವಾಗುವುದಕ್ಕೆ ಅನಂತ ಶಕ್ತಿಯಿದೆ. ಇದೇ ನನ್ನ ನಂಬಿಕೆ. ಒಂದು ಅಣುವಿನ ಅಂತರಾಳದಲ್ಲಿ ಇಡಿಯ ವಿಶ್ವಶಕ್ತಿ ನಿಂತಿದೆ. ಹಿಂದೂ ಜನಾಂಗದ ಇತಿಹಾಸದಲ್ಲಿ ಹಿಂದಿನಿಂದಲೂ ಎಂದಿಗೂ ಧ್ವಂಸಕಾರ‍್ಯವಿರಲಿಲ್ಲ; ಹಿಂದೂಗಳು ಯಾವಾಗಲೂ ಹೊಸದನ್ನು ನಿರ್ಮಿಸುತ್ತಿದ್ದರು. ಒಂದು ಪಂಗಡ ಮಾತ್ರ ಧ್ವಂಸಕ್ಕೆ ಯತ್ನಿಸಿತು. ಭರತಖಂಡ ಅದನ್ನು ಹೊರದೂಡಿತು. ಅದೇ ಬೌದ್ಧಧರ್ಮ. ಶಂಕರ, ರಾಮಾನುಜ ಚೈತನ್ಯರೆಂಬ ಹಲವು ಸುಧಾರಕರು ಬಂದರು. ಇವರೆಲ್ಲ ದೊಡ್ಡ ದೊಡ್ಡ ಸುಧಾರಕರು; ಇವರು ಯಾವಾಗಲೂ ಸೃಷ್ಟಿ ಮಾರ್ಗವನ್ನು ಅನುಸರಿಸಿದರು, ಇವರು ಕಾಲಕ್ಕೆ ತಕ್ಕಂತೆ ಸಮಾಜವನ್ನು ರಚಿಸಿದರು. ನಮ್ಮ ಕೆಲಸದ ವಿಶೇಷ ರೀತಿಯೇ ಇದು. ಆಧುನಿಕ ಸುಧಾರಕರೆಲ್ಲ ಧ್ವಂಸಕಾರಕ ಐರೋಪ್ಯ ಕ್ರಾಂತಿಯನ್ನು ಅನುಸರಿಸಿರುವರು. ಅದು ಹಿಂದೆ ಯಾರಿಗೂ ಒಳ್ಳೆಯದನ್ನು ಮಾಡಿಲ್ಲ ಮತ್ತು ಮುಂದೆ ಮಾಡುವಂತೆಯೂ ಇಲ್ಲ. ಆಧುನಿಕರಲ್ಲಿ ಹೆಚ್ಚು ನಿರ್ಮಾಣ ಪಂಥದ ಸುಧಾರಕರೆಂದರೆ ರಾಜಾರಾಮ ಮೋಹನರಾಯ್ ಒಬ್ಬರೇ. ಹಿಂದೂ ಜನಾಂಗ ವೇದಾಂತ ಆದರ್ಶದ ಕಡೆಗೆ ಮುಂದುವರಿಯುತ್ತಿದೆ. ಭಾರತೀಯರ ಇತಿಹಾಸವೆಲ್ಲ ಸುಖ ಮತ್ತು ದುಃಖದ ಮಾರ್ಗದ ಮೂಲಕ ವೇದಾಂತ ಆದರ್ಶವನ್ನು ಸಾಕ್ಷಾತ್ಕಾರ ಮಾಡಿಕೊಳ್ಳುವುದಾಗಿದೆ. ಯಾವ ಸುಧಾರಕ ಪಂಥ ವೇದಾಂತದ ಆದರ್ಶವನ್ನು ಕಿತ್ತೊಗೆಯಿತೊ ಅದು ಭರತಖಂಡದಿಂದ ನಿರ್ಮಾಣನಿರ್ನಾಮವಾಗಿ ಹೋಗುವಂತೆ ಮಾಡಲ್ಪಟ್ಟಿತು.” 

 ಪ್ರಶ್ನೆ: “ನೀವು ಇಲ್ಲಿ ಯಾವ ಕೆಲಸವನ್ನು ಮಾಡುತ್ತೀರಿ?” 

 ಸ್ವಾಮೀಜಿ: “ನನ್ನ ಯೋಜನೆಯನ್ನು ಕಾರ್ಯಗತ ಮಾಡಲು ಒಂದು ಕೇಂದ್ರವನ್ನು ಮದ್ರಾಸಿನಲ್ಲಿ, ಮತ್ತೊಂದು ಕೇಂದ್ರವನ್ನು ಕಲ್ಕತ್ತೆಯಲ್ಲಿ ತೆರೆಯಲು ಯತ್ನಿಸುವೆನು. ವೇದಾಂತ ಆದರ್ಶವನ್ನು ಪಾಪಿ ಮತ್ತು ಪುಣ್ಯವಂತ ಪಂಡಿತ ಮತ್ತು ಪಾಮರ, ಬ್ರಾಹ್ಮಣ ಅಥವಾ ಚಂಡಾಲ ಎಲ್ಲರ ಜೀವನದಲ್ಲಿಯೂ ಕಾರ್ಯಕಾರಿಯಾಗುವಂತೆ ಮಾಡುವುದೇ ನನ್ನ ಉದ್ದೇಶ.” 

 ನಮ್ಮ ಪ್ರತಿನಿಧಿಗಳು ಇಲ್ಲಿ ಇಂಡಿಯಾದ ರಾಜಕೀಯಕ್ಕೆ ಸಂಬಂಧಿಸಿದ ಕೆಲವು ಪ್ರಶ್ನೆಗಳನ್ನು ಹಾಕಲು ಯತ್ನಿಸಿದರು. ಸ್ವಾಮೀಜಿಯವರು ಉತ್ತರ ಕೊಡುವುದರೊಳಗೆ ರೈಲು ಎಗ್‌ಮೋರ್ ನಿಲ್ದಾಣಕ್ಕೆ ತಲುಪಿತು. ಅವಸರದಲ್ಲಿ ಸ್ವಾಮೀಜಿ ಇಂಡಿಯಾ ಮತ್ತು ಯೂರೋಪಿನ ರಾಜಕೀಯ ಚಟುವಟಿಕೆಗಳನ್ನು ಕಂಡರೇನೇ ನನಗಾಗುವುದಿಲ್ಲ ಎಂದು ಹೇಳಿದರು. ಭೇಟಿ ಕೊನೆಗೊಂಡಿತು. 

