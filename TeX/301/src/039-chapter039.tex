
\chapter{ಗಾಯತ್ರಿ ಉಪದೇಶ ಮತ್ತು ಕರ್ಮಯೋಗ}

 ಸ್ವಾಮೀಜಿ ೧೮೯೮ ಜನವರಿಯಿಂದ ಅಕ್ಟೋಬರ್ ವರೆಗೆ ಕಲ್ಕತ್ತೆಯ ಮಠದಲ್ಲಿದ್ದರು. ಮಠವನ್ನು ಆಲಂಬಜಾರಿನಿಂದ ಗಂಗೆಯ ದಕ್ಷಿಣಭಾಗದ ಗಂಗಾತೀರದಲ್ಲಿರುವ ನೀಲಾಂಬರ ಮುಖರ್ಜಿ ಉದ್ಯಾನಕ್ಕೆ ವರ್ಗಾಯಿಸಿದ್ದರು. ಅದು ಹೌರಾ ಪಕ್ಕದಲ್ಲಿರುವ ಬೆಲೂರು ಎಂಬ ಹಳ್ಳಿಯಲ್ಲಿದೆ. ಸ್ವಾಮೀಜಿ ಈ ಕಾಲದಲ್ಲಿ ಕಲ್ಕತ್ತೆಯ ಹಲವು ಭಕ್ತರ ಮನೆಗೆ ಹೋಗುತ್ತಿದ್ದರು. ಮಠದಲ್ಲಿದ್ದಾಗ ಹಲವು ಜನರೊಡನೆ ಮಾತುಕತೆಯಾಡುತ್ತಿದ್ದರು. ಯಾವುದಾದರೂ ಲೇಖನಗಳೊ, ಪತ್ರವ್ಯವಹಾರಗಳೊ ಇವುಗಳಲ್ಲಿ ನಿರತರಾಗಿದ್ದರು. ಮಠದಲ್ಲಿ ಸಂನ್ಯಾಸಿಗಳು ಮತ್ತು ಬ್ರಹ್ಮಚಾರಿಗಳೊಡನೆ ಬಹಳ ಕಾಲವನ್ನು ಕಳೆಯುತ್ತಿದ್ದರು. ಅವರಿಗೆ ಪಾಠಪ್ರವಚನಗಳನ್ನು ತೆಗೆದುಕೊಳ್ಳುವುದು ಹಾಡುವುದು ಮತ್ತು ತಮ್ಮ ಅನುಭವಗಳನ್ನು ಹೇಳುವುದು ಇವುಗಳಲ್ಲಿ ನಿರತರಾಗಿದ್ದರು. ಗೀತಾ ಉಪನಿಷತ್ತುಗಳಲ್ಲದೆ, ವಿಜ್ಞಾನ ಸಾಹಿತ್ಯ ಮುಂತಾದ ಹಲವು ಲೌಕಿಕ ವಿಷಯಗಳ ಮೇಲೆಯೂ ಕೇಳಿದ ಹಲವು ಪ್ರಶ್ನೆಗಳಿಗೆ ಉತ್ತರವನ್ನು ಕೊಡುತ್ತಿದ್ದರು. 

 ಸ್ವಾಮೀಜಿ ೧೮೯೮ ಆದಿಯಲ್ಲೆ ಬೇಲೂರು ಗ್ರಾಮದಲ್ಲಿ ಗಂಗಾತೀರದಲ್ಲಿ ಸುಮಾರು ಏಳು ಎಕರೆಗಳಷ್ಟು ವಿಸ್ತ್ತಾರವಾದ ಸ್ಥಳವನ್ನು ಕೊಂಡರು. ಆ ಸ್ಥಳದಲ್ಲಿ ಕೆಲವು ಮನೆಗಳೂ ಕೂಡ ಇದ್ದವು. ಅದಕ್ಕೆ ಬೇಕಾದ ಹಣವನ್ನೆಲ್ಲ ಸ್ವಾಮೀಜಿ ಅವರ ಆಂಗ್ಲ ಶಿಷ್ಯಳಾದ ಮಿಸ್ ಹೆನ್‍ರಿಟಾ ಮುಲ್ಲರ್ ಎಂಬುವವರು ಕೊಟ್ಟರು. ಇದು ಗಂಗಾತೀರದಲ್ಲಿ ಬಾರಾನಗರದ ಸ್ನಾನಘಟ್ಟಕ್ಕೆ ಎದುರಾಗಿ ಇರುವುದು. ಮಿಸ್ ಮುಲ್ಲರ್ ಶ‍್ರೀಮಂತ ಆಂಗ್ಲೇಯ ಸಾಧ್ವಿ. ಧಾರ್ಮಿಕ ಸ್ವಭಾವದವಳು. 

 ಸ್ವಾಮೀಜಿ ಇಂಗ್ಲೆಂಡಿನಲ್ಲಿದ್ದಾಗ ಅವರ ಖರ್ಚನ್ನೆಲ್ಲ ವಹಿಸಿಕೊಳ್ಳುತ್ತಿದ್ದಳು. ಈಗ ಆಕೆಯ ದುಡ್ಡಿನಿಂದಲೇ‌ ಬೇಲೂರು ಮಠದ ನಿವೇಶನವೂ ಕೂಡ ಕೊಂಡಾಯಿತು. ಆಕೆ ಒಂದು ಸಮಯದಲ್ಲಿ ಸಂಸಾರವನ್ನು ತ್ಯಜಿಸಬೇಕೆಂದು ಸಂಕಲ್ಪ ಮಾಡಿದ್ದಳು. ಆದರೆ ಸ್ವಾಮೀಜಿ ಅವಳಿಗೆ ಪ್ರಪಂಚದಲ್ಲಿ ಸೇವೆಮಾಡು ಎಂದು ಹೇಳಿದ್ದರು. ಅವಳು ಸ್ವಾಮೀಜಿಯವರ ಭರತಖಂಡದ ಕೆಲಸಕ್ಕೆ ಸಹಾಯಕಳಾಗಬೇಕೆಂದು ಇಂಡಿಯಾದೇಶಕ್ಕೆ ಬಂದಳು. 

 ಬೇಲೂರ ಮಠದ ನಿವೇಶನ ಮತ್ತು ಮನೆಗಳನ್ನು ೧೮೯೮ನೇ ಪ್ರಾರಂಭದಲ್ಲಿ ತೆಗೆದುಕೊಂಡರೂ ಅದನ್ನು ವಾಸಯೋಗ್ಯ ಮಾಡಬೇಕಾದರೆ ಒಂದು ವರುಷವೇ ಹಿಡಿಯಿತು. ತಗ್ಗು ದಿಣ್ಣೆಗಳಿದ್ದ ನೆಲವನ್ನು ಒಂದೇ ಸಮನಾಗಿ ಮಾಡಬೇಕಾಗಿತ್ತು. ಆಗಲೇ ಇದ್ದ ಮನೆಗಳಿಗೆ ರಿಪೇರಿಯಾಗಬೇಕಾಗಿತ್ತು. ಸ್ವಲ್ಪ ಭಾಗವನ್ನು ಹೊಸದಾಗಿ ಕಟ್ಟಬೇಕಾಗಿತ್ತು. ಕೆಲವು ಕಾಲವಾದ ಮೇಲೆ ಶ‍್ರೀಮತಿ ಓಲ್‍ಬುಲ್ ಎಂಬ ಸ್ವಾಮೀಜಿಯವರ ಶಿಷ್ಯೆ ಕೆಲವು ಸಹಸ್ರ ರೂಪಾಯಿಗಳನ್ನು ಸ್ವಾಮೀಜಿಯವರ ಕೆಲಸಕ್ಕೆ ಕೊಟ್ಟಳು. ಇದರಿಂದ ಬೇಲೂರು ಮಠಕ್ಕೆ ಮೂಲಧನವನ್ನು ಮಾಡುವುದಕ್ಕೂ ಮತ್ತು ಶ‍್ರೀರಾಮಕೃಷ್ಣರ ಹೆಸರಿನಲ್ಲಿ ಒಂದು ದೇವಸ್ಥಾನವನ್ನು ಕಟ್ಟುವುದಕ್ಕೂ ಸಾಧ್ಯವಾಯಿತು. ಮಠ ಮೂಲಧನ ಎಲ್ಲ ಸೇರಿಕೊಂಡು ಈಗ ಸುಮಾರು ಒಂದು ಲಕ್ಷ ಬೆಲೆ ಬಾಳುವಷ್ಟು ಆಯಿತು. ಇದರಲ್ಲಿ ಬಹುಭಾಗ ಸ್ವಾಮೀಜಿಯವರ ಪಾಶ್ಚಾತ್ಯ ಶಿಷ್ಯರಿಂದಲೇ ಬಂದಿತು. 

 ನೀಲಾಂಬರ ಮುಖರ್ಜಿಯ ಉದ್ಯಾನಮನೆಯ ಮಠ ಈಗ ಹಲವು ಸ್ವಾಮಿಗಳಿಂದ ಕಿಕ್ಕಿರಿದಿದೆ. ಶಿವರಾತ್ರಿಯ ಸಮಯ, ಶ‍್ರೀರಾಮಕೃಷ್ಣರ ಜನ್ಮದಿನಕ್ಕೆ ಇನ್ನು ಮೂರು ನಾಲ್ಕು ದಿನ ಮಾತ್ರ ಇದೆ. ಸ್ವಾಮಿ ಶಿವಾನಂದರು ಸಿಂಹಳ ದ್ವೀಪದಲ್ಲಿ ಪ್ರಚಾರಕಾರ‍್ಯವನ್ನು ಮಾಡಿಕೊಂಡು ಬಂದಿದ್ದರು. ಸ್ವಾಮಿ ತ್ರಿಗುಣಾತೀತರು ದಿನಾಜ್ ಪುರದಲ್ಲಿ ಕ್ಷಾಮನಿವಾರಣಾ ಕೆಲಸವನ್ನು ಪೂರೈಸಿಕೊಂಡು ಬಂದಿದ್ದರು. ಸ್ವಾಮೀಜಿಯವರಿಗೆ ತಮ್ಮ ಗುರುಭಾಯಿಗಳು ಮಾಡಿದ ಕೆಲಸವನ್ನೆಲ್ಲ ಕಂಡು ತುಂಬಾ ಆನಂದವಾಗಿದೆ. ಬ್ರಹ್ಮಾನಂದರು ತಮ್ಮ ನೇತೃತ್ವದಲ್ಲಿ ರಾಮಕೃಷ್ಣ ಮಿಶನ್ ಕೆಲಸವನ್ನು ಯಶಸ್ವಿಯಾಗಿ ನೆರವೇರಿಸುತ್ತಿದ್ದರು. ತುರಿಯಾನಂದರು ಮಠದಲ್ಲಿ ಹೊಸದಾಗಿ ಸೇರಿದ್ದ ಸಾಧು ಬ್ರಹ್ಮಚಾರಿಗಳನ್ನು ತರಬೇತು ಮಾಡುತ್ತಿದ್ದರು. ಶಿವರಾತ್ರಿಯ ದಿನ ಕೆಲಸಮಾಡಿಕೊಂಡು ಬಂದ ಸ್ವಾಮೀಜಿಗಳಿಗೆಲ್ಲಾ ಧನ್ಯವಾದವನ್ನು ಅರ್ಪಿಸುವ ಒಂದು ಸಣ್ಣ ಸಮಾರಂಭವನ್ನು ಮಠದಲ್ಲಿಯೇ ಅಣಿಮಾಡಲಾಯಿತು. ಸ್ವಾಮೀಜಿಯವರು ಅಧ್ಯಕ್ಷರಾಗಿದ್ದರು. ಆಯಾ ಕೆಲಸವನ್ನು ಮಾಡಿದವರಿಗೆ ಧನ್ಯವಾದವನ್ನು ಅರ್ಪಿಸಿದ ಮೇಲೆ ಸೂಕ್ತವಾದ ಉತ್ತರವನ್ನು ಕೊಡುವಂತೆ ಸ್ವಾಮೀಜಿಯವರು ತಮ್ಮ ಗುರುಭಾಯಿಗಳಿಗೆ ಹೇಳಿದರು. ಆ ಸಮಯದಲ್ಲಿ ಮಾತನಾಡಿದ ಸ್ವಾಮಿಗಳನ್ನು ಸ್ವಾಮೀಜಿ ಮೆಚ್ಚಿದರು. ಅದರಲ್ಲಿಯೂ ಸ್ವಾಮಿ ತುರಿಯಾನಂದರನ್ನು ಕುರಿತು, ಆತನಿಗೆ ದೊಡ್ಡ ವಾಗ್ಮಿಯ ಧ್ವನಿಯಿದೆ ಎಂದು ಹೇಳಿದರು. ದೊಡ್ಡ ಸಭೆಯಲ್ಲಿ ಮಾತನಾಡುವಾಗ ನಮ್ಮನ್ನು ಮರೆಯುವುದು ಸುಲಭ. ಆದರೆ ಸಣ್ಣ ಸಣ್ಣ ಸಭೆಗಳಲ್ಲಿ ಮಾತನಾಡುವುದು ಕಷ್ಟ ಎಂದು ಸ್ವಾಮಿಗಳು ಹೇಳಿದರು. ಸ್ವಾಮೀಜಿಯವರೇ ಅನಂತರ ಕುಳಿತವರಿಗೆಲ್ಲ ಕರ್ಮ ಮಾಡುವಾಗ ವ್ಯಕ್ತಿಯ ದೃಷ್ಟಿ ಮತ್ತು ಹೊರಗಿನ ಸಾಮಾಜಿಕ ದೃಷ್ಟಿ ಇವುಗಳ ಮೇಲೆ ಚೆನ್ನಾಗಿ ಮಾತನಾಡಿದರು. 

 ಅನಂತರ ಶ‍್ರೀರಾಮಕೃಷ್ಣರ ಜನ್ಮದಿನ ಬಂದಿತು. ಈ ಸಲ ಅದು ಸ್ವಾಮೀಜಿಯವರ ನೈತೃತ್ವದಲ್ಲಿಯೇ ಆಗುವುದು. ಜನ್ಮತಿಥಿಯ ಬೆಳಿಗ್ಗೆ ಎಲ್ಲರಿಗೂ ಆನಂದ. ಪರಮಹಂಸರ ಮಾತನ್ನು ಬಿಟ್ಟು ಭಕ್ತರ ಬಾಯಲ್ಲಿ ಮತ್ತೊಂದು ಮಾತಿಲ್ಲ. ಈಗ ಸ್ವಾಮೀಜಿ ಪೂಜಾಮಂದಿರದ ಮುಂದೆ ಬಂದು ಪೂಜೆಗೆ ಸಿದ್ಧವಾಗುತ್ತಿದ್ದುದನ್ನು ನೋಡುತ್ತಾ ನಿಂತುಕೊಂಡರು. ಪೂಜೆಗೆ ಆಗಬೇಕಾದ ಏರ್ಪಾಡನ್ನು ನೋಡಿ ಸ್ವಾಮೀಜಿ ಶಿಷ್ಯ ಶರತ್‍ಚಂದ್ರ ಚಕ್ರವರ್ತಿಯನ್ನು ಕುರಿತು ಯಜ್ಞೋಪವೀತಗಳನ್ನು ತಂದಿದ್ದೀಯಷ್ಟೆ ಎಂದು ಕೇಳಿದರು. ಶಿಷ್ಯ: “ತಮ್ಮ ಅಪ್ಪಣೆಯಂತೆ ಎಲ್ಲವೂ ಸಿದ್ಧವಾಗಿದೆ. ಆದರೆ ಇಷ್ಟೊಂದು ಯಜ್ಞೋಪವೀತಗಳನ್ನು ಸಿದ್ಧಪಡಿಸಿಕೊಂಡದ್ದು ಏಕೆ ನನಗೆ ಗೊತ್ತಾಗಲಿಲ್ಲ.” 

 ಸ್ವಾಮೀಜಿ: “ದ್ವಿಜರಿಗೆ ಮಾತ್ರ ಉಪನಯನ ಸಂಸ್ಕಾರಕ್ಕೆ ಅಧಿಕಾರ. ಇದಕ್ಕೆ ವೇದವೇ ಪ್ರಮಾಣ. ಇಂದು ಪರಮಹಂಸರ ಜನ್ಮೋತ್ಸವಕ್ಕೆ ಯಾರು ಬರುತ್ತಾರೆಯೋ ಅವರೆಲ್ಲರಿಗೂ ಜನಿವಾರವನ್ನು ಹಾಕಿಸುತ್ತೇನೆ. ಇವರೆಲ್ಲ ಪತಿತರಾಗಿಬಿಟ್ಟಿರುವರು. ಪ್ರಾಯಶ್ಚಿತ್ತ ಮಾಡಿಕೊಂಡ ಮೇಲೆ ಉಪನಯನಕ್ಕೆ ಅರ್ಹನಾಗುವರೆಂದು ಶಾಸ್ತ್ರದಲ್ಲಿ ಹೇಳಿದೆ. ಇದು ಪರಮಹಂಸರ ಶುಭ ಜನ್ಮೋತ್ಸವ. ಎಲ್ಲರೂ ಅವರ ನಾಮೋಚ್ಚಾರಣೆ ಮಾಡಿ ಪರಿಶುದ್ಧರಾಗುವರು. ಅದಕ್ಕೋಸ್ಕರವೇ ಇಂದು ಬರುವ ಭಕ್ತರಿಗೆಲ್ಲ ಜನಿವಾರವನ್ನು ಹಾಕಿಸಬೇಕು ತಿಳಿಯಿತೆ?” 

 ಶಿಷ್ಯ: “ನಾನು ಅಪ್ಪಣೆ ಮೇರೆ ಬೇಕಾದಷ್ಟು ಜನಿವಾರವನ್ನು ತಂದಿರುವೆನು. ಪೂಜೆಯಾದ ಮೇಲೆ ತಮ್ಮ ಅನುಮತಿಯನ್ನು ಪಡೆದು ಭಕ್ತರಿಗೆಲ್ಲ ಅದನ್ನು ಹಾಕಿಸುತ್ತೇನೆ.” 

 ಸ್ವಾಮೀಜಿ: “ಬ್ರಾಹ್ಮಣೇತರ ಭಕ್ತರಿಗೆ ಗಾಯತ್ರೀ ಮಂತ್ರವನ್ನು ಹೇಳಿಕೊಡುತ್ತೇನೆ. ಕಾಲಕ್ರಮದಲ್ಲಿ ದೇಶವನ್ನೆಲ್ಲ ಬ್ರಾಹ್ಮಣ ಪದವಿಗೆ ಏರಿಸಬೇಕಾಗಿದೆ. ಪರಮಹಂಸರ ಭಕ್ತರ ಮಾತಂತೂ ಹೇಳಲೇಬೇಕಾಗಿಲ್ಲ. ಹಿಂದೂಗಳೆಲ್ಲರೂ ಪರಸ್ಪರ ಅಣ್ಣ ತಮ್ಮಂದಿರು. ಮುಟ್ಟಬೇಡ ಎಂದು ಹೇಳಿ, ನಾವೇ ಅವರನ್ನು ಕೀಳು ಮಾಡಿ ಕೂರಿಸಿದ್ದೇವೆ. ಇದೇ ದೇಶದ ಹೀನತೆ, ಭೀರುತೆ, ಮೂರ್ಖತೆ, ಕಾಪುರುಷೆತೆ ಇವುಗಳ ಪರಾಕಾಷ್ಠೆಯಲ್ಲಿ ಪರಿಣಮಿಸಿದೆ. ಇವರನ್ನು ಮೇಲಕ್ಕೆ ಎತ್ತಬೇಕು, ಅಭಯದ ಮಾತನ್ನು ಹೇಳಬೇಕು. ‘ನೀವೂ ನಮ್ಮ ಹಾಗೆಯೇ ಮನುಷ್ಯರು, ನಿಮಗೂ ನಮ್ಮ ಹಾಗೆಯೇ ಎಲ್ಲ ಅಧಿಕಾರವೂ ಇದೆ’ ಎಂದು ಹೇಳಬೇಕು. ತಿಳಿಯಿತೆ?” 

 ಶಿಷ್ಯ: “ತಿಳಿಯಿತು.” 

 ಸ್ವಾಮೀಜಿ: “ಈಗ ಯಾರಿಗೆ ಜನಿವಾರ ಬೇಕೊ ಅವರಿಗೆಲ್ಲಾ ಗಂಗಾಸ್ನಾನ ಮಾಡಿಕೊಂಡು ಬರುವಂತೆ ಹೇಳು. ಆಮೇಲೆ ಪರಮಹಂಸರಿಗೆ ನಮಸ್ಕಾರ ಮಾಡಿ ಎಲ್ಲರೂ ಜನಿವಾರವನ್ನು ಹಾಕಿಕೊಳ್ಳಿ.” 

 ಸ್ವಾಮೀಜಿ ಅಪ್ಪಣೆಮೇರೆ ಬಂದಿದ್ದ ಸುಮಾರು ನಲವತ್ತು ಐವತ್ತು ಜನ ಭಕ್ತರು ಕ್ರಮವಾಗಿ ಗಂಗಾಸ್ನಾನ ಮಾಡಿಬಂದು ಶಿಷ್ಯನಿಂದ ಗಾಯತ್ರಿ ಉಪದೇಶವನ್ನು ಹೊಂದಿ ಜನಿವಾರವನ್ನು ಹಾಕಿಕೊಳ್ಳುವುದಕ್ಕೆ ಮೊದಲುಮಾಡಿದರು. ಮಠದಲ್ಲೆಲ್ಲ ಮಂಗಳರವ ಉಂಟಾಯಿತು. ಜನಿವಾರ ಹಾಕಿಕೊಂಡು ಭಕ್ತರು ಪುನಃ ಪರಮಹಂಸರಿಗೆ ಪ್ರಣಾಮಮಾಡಿದರು. ಸ್ವಾಮೀಜಿ ಪಾದಪದ್ಮದಲ್ಲಿಯೂ ಅಡ್ಡ ಬಿದ್ದರು. ಅದನ್ನು ನೋಡಿ ಸ್ವಾಮೀಜಿ ಮುಖಾರವಿಂದ ಅರಳಿತು. ಇದಾದ ಸ್ವಲ್ಪಹೊತ್ತಿಗೆ ಗಿರೀಶ್ ಚಂದ್ರ ಘೋಷರು ಮಠಕ್ಕೆ ಬಂದರು. 

 ಈಗ ಸ್ವಾಮೀಜಿ ಅಪ್ಪಣೆಯಂತೆ ಸಂಗೀತಕ್ಕೆ ಆರಂಭವಾಯಿತು. ಮಠದ ಸಂನ್ಯಾಸಿಗಳು ಸ್ವಾಮೀಜಿಯವರಿಗೆ ತಮ್ಮ ಮನದಣಿಯೆ ವೇಷಭೂಷಣಗಳನ್ನು ಹಾಕತೊಡಗಿದರು. ಅವರ ಕಿವಿಯಲ್ಲಿ ಶಂಖದ ಕುಂಡಲ, ಮೈಯಲ್ಲಿ ವಿಭೂತಿ, ತಲೆಯಲ್ಲಿ ಕಾಲುತನಕ ಜೋಲಾಡುವ ಜಟೆಗಳು, ಎಡಗೈಯಲ್ಲಿ ತ್ರಿಶೂಲ, ಎರಡು ತೋಳುಗಳಲ್ಲಿಯೂ ರುದ್ರಾಕ್ಷಿಯ ವಲಯ, ಕತ್ತಿನಲ್ಲಿ ಮಂಡಿಯವರೆಗೆ ಬಂದಿದ್ದ ಮೂರು ಸಾಲು ರುದ್ರಾಕ್ಷಿಯ ಹಾರ ಮುಂತಾದವುಗಳೆಲ್ಲಾ ಆದವು. ಇವೆಲ್ಲವನ್ನು ಧರಿಸಿಕೊಳ್ಳಲು, ಸ್ವಾಮಿಗಳ ರೂಪದಲ್ಲಿ ಉಂಟಾದ ರಮಣೀಯತೆಯನ್ನು ಬಾಯಲ್ಲಿ ಹೇಳತೀರದು. ಆ ದಿನ ಯಾರು ಯಾರು ಆ ಮೂರ್ತಿಯನ್ನು ನೋಡಿದರೊ ಅವರೆಲ್ಲರೂ ಸಾಕ್ಷಾತ್ ಭೈರವನೇ‌ ಸ್ವಾಮಿಗಳ ಶರೀರದಿಂದ ಭೂಮಿಯಲ್ಲಿ ಅವತರಿಸಿದ್ದನೆಂದು ಹೇಳಿದರು. ಸ್ವಾಮಿಗಳೂ ಇತರರಿಗೆ ವಿಭೂತಿಯನ್ನು ಬಳಿದರು. ಇತರರು ಸ್ವಾಮಿಗಳ ನಾಲ್ಕು ಕಡೆಗೂ ಭೈರವಗಣಗಳ ಹಾಗೆ ಇದ್ದುಕೊಂಡು, ಮಠಪ್ರದೇಶದಲ್ಲಿ ಕೈಲಾಸದ ಆನಂದವನ್ನು ಬೀರುತ್ತಿದ್ದರು. 

 ಈಗ ಸ್ವಾಮೀಜಿ ಪಶ್ಚಿಮದಿಕ್ಕಿಗೆ ತಿರುಗಿಕೊಂಡು ಮುಕ್ತಪದ್ಮಾಸನದಲ್ಲಿ ಕುಳಿತುಕೊಂಡು, “ಕೂಜಂತ ರಾಮರಾಮೇತಿ” ಎಂದು ಪ್ರಾರ್ಥನಾಶ್ಲೋಕವನ್ನು ಮಧುರಮಧುರವಾಗಿ ಹೇಳುತ್ತ ಅದು ಮುಗಿದ ಮೇಲೆ “ರಾಮ, ರಾಮ, ಶ‍್ರೀರಾಮ” ಎಂಬುದನ್ನೇ ಪುನಃ ಪುನಃ ಹೇಳತೊಡಗಿದರು. ಅಕ್ಷರ ಅಕ್ಷರದಲ್ಲಿಯೂ ಅಮೃತ ತೊಟ್ಟಿಕ್ಕುವಂತೆ ಇತ್ತು. ಸ್ವಾಮೀಜಿಯ ಕಣ್ಣು ಅರ್ಧ ಮುಚ್ಚಿದೆ. ಕೈಯಲ್ಲಿ ತಂಬೂರಿಯ ಸ್ವರ ಶ್ರುತಿಗೂಡುತ್ತಿದೆ. “ರಾಮ, ರಾಮ, ಶ‍್ರೀರಾಮ” ಎಂಬ ಧ್ವನಿಯನ್ನು ಬಿಟ್ಟು ಸ್ವಲ್ಪ ಹೊತ್ತಿನತನಕ ಮತ್ತಾವುದೂ ಕೇಳಬರುತ್ತಿಲ್ಲ. ಹೀಗೆ ಸುಮಾರು ಅರ್ಧಗಂಟೆಗಿಂತ ಹೆಚ್ಚಾಗಿ ಕಳೆಯಿತು; ಆಗಲೂ ಯಾರ ಬಾಯಲ್ಲೂ ಬೇರೆ ಯಾವ ಮಾತೂ ಇಲ್ಲ. ಸ್ವಾಮೀಜಿ ಬಾಯಿನಿಂದ ಬರುತ್ತಿದ್ದ ರಾಮನಾಮಾಮೃತವನ್ನು ಪಾನಮಾಡಿ ಎಲ್ಲರೂ ಇಂದು ಮತ್ತರಾಗಿಬಿಟ್ಟಿದ್ದರು. ಇದೇನು ನಿಜವಾಗಿಯೂ ಇವತ್ತು ಶಿವಭಾವದಲ್ಲಿ ಮುಕ್ತರಾಗಿ ರಾಮನಾಮವನ್ನು ಹೇಳುವುದಕ್ಕೆ ತೊಡಗಿದರೊ ಎಂದು ಭಾವಿಸತೊಡಗಿದರು. ಸ್ವಾಮೀಜಿ ಮುಖದ ಸ್ವಾಭಾವಿಕ ಗಾಂಭೀರ್ಯ ಇಂದು ನೂರ್ಮಡಿಯಾದ ಗಂಭೀರತೆಯನ್ನು ಪಡೆದಂತೆ ಇತ್ತು. ಅರೆ ಮುಚ್ಚಿದ ಕಣ್ಕಡೆಯಲ್ಲಿ ಪ್ರಭಾತ ಸೂರ್ಯನ ಕಾಂತಿ ಹೊರಟು ಹೊರಗೆ ಬರುವಂತೆ ಇತ್ತು. ಅತ್ಯಂತವಾದ ಆನಂದದ ಮತ್ತತೆಯಲ್ಲಿ ಆ ವಿಪುಲವಾದ ದೇಹ ತೂರಾಡುವಂತೆ ಇತ್ತು. ಈ ರೂಪವನ್ನು ವರ್ಣಿಸಲು ಸಾಧ್ಯವಿಲ್ಲ, ಅನುಭವ ವೇದ್ಯ. 

 ರಾಮನಾಮ ಸಂಕೀರ್ತನ ಮುಗಿದ ಮೇಲೆ ಸ್ವಾಮೀಜಿ ಮೊದಲಿನಂತೆ ಆನಂದ ಪ್ರವೃತ್ತಿಯಲ್ಲಿಯೇ “ಸೀತಾಪತಿ ರಾಮಚಂದ್ರ ರಘುರಾಈ” ಎಂಬುದನ್ನು ಗಾನಮಾಡುವುದಕ್ಕೆ ಮೊದಲು ಮಾಡಿದರು. ಮೃದಂಗವಾದನ ಸರಿಯಾಗಿಲ್ಲದೇ ಇದ್ದುದರಿಂದ ಸ್ವಾಮಿಗಳಿಗೆ ರಸಭಂಗವಾದಂತೆ ತೋರಿತು. ಆಗ ಅವರು ಶಾರದಾನಂದ ಸ್ವಾಮಿಗಳಿಗೆ ಹಾಡುವಂತೆ ಹೇಳಿ ತಾವೇ ಮೃದಂಗವನ್ನು ತೆಗೆದುಕೊಂಡರು. ಶಾರದಾನಂದ ಸ್ವಾಮಿಗಳು ಮೊದಲು “ಏಕರೂಪ ಅರೂಪ ನಾಮವರಣ” ಎಂಬ ಕೀರ್ತನೆಯನ್ನು ಹಾಡಿದರು. ಮೃದಂಗದ ಸ್ನಿಗ್ಧ ಗಂಭೀರನಿನಾದ ಗಂಗೆಯಂತೆ ಉಕ್ಕಿ ಉಕ್ಕಿ ಬರುವಂತೆ ಇತ್ತು. ಶಾರದಾನಂದಸ್ವಾಮಿಗಳ ಕಂಠವೂ ಮಧುರ ಆಲಾಪನೆಯೂ ಸುತ್ತಲೂ ಆವರಿಸಿಕೊಂಡಿತು. ಆಮೇಲೆ ಶ‍್ರೀರಾಮಕೃಷ್ಣರು ಹಾಡುತ್ತಿದ್ದ ಹಾಡುಗಳನ್ನೆಲ್ಲ ಒಂದೊಂದಾಗಿ ಹಾಡತೊಡಗಿದರು. 

 ಈಗ ಸ್ವಾಮೀಜಿ ಇದ್ದಕ್ಕೆ ಇದ್ದಹಾಗೆ ತಮ್ಮ ವೇಷಭೂಷಣಗಳನ್ನೆಲ್ಲ ತೆಗೆದು ಹಾಕಿ ಗಿರೀಶಬಾಬುಗಳಿಗೆ ಅದನ್ನು ತೊಡಿಸುವುದಕ್ಕೆ ಹೊರಟರು. ತಮ್ಮ ಕೈಯಿಂದಲೇ ಗಿರೀಶಬಾಬುಗಳ ವಿಶಾಲವಾದ ದೇಹಕ್ಕೆ ವಿಭೂತಿಯನ್ನು ಬಳಿದು ಕಿವಿಯಲ್ಲಿ ಕುಂಡಲವನ್ನೂ ತಲೆಯಲ್ಲಿ ಜಡೆಯನ್ನೂ ಕತ್ತಿನಲ್ಲಿ ರುದ್ರಾಕ್ಷಿಗಳನ್ನೂ ತೋಳುಗಳಲ್ಲಿ ರುದ್ರಾಕ್ಷಿ ವಲಯಗಳನ್ನೂ ಇಟ್ಟರು. ಗಿರೀಶಬಾಬುಗಳು ಇದರಿಂದ ಬೇರೊಬ್ಬ ಮೂರ್ತಿಯಂತೆ ಕಂಡರು. ಇದನ್ನು ನೋಡಿದ ಭಕ್ತರಿಗೆ ಬೆರಗಾಯಿತು. ಆಮೇಲೆ ಸ್ವಾಮೀಜಿ, “ಈತ ಭೈರವನ ಅವತಾರ, ನನಗೂ ಈತನಿಗೂ ಏನೂ ವ್ಯತ್ಯಾಸವಿಲ್ಲ ಎಂದು ಪರಮಹಂಸರು ಹೇಳುತ್ತಿದ್ದರು.” ಎಂದರು. ಗಿರೀಶಬಾಬುಗಳು ವಿಸ್ಮಿತರಾಗಿ ಕುಳಿತುಕೊಂಡಿದ್ದರು. ಅವರ ಸಂನ್ಯಾಸಿ ಗುರುಭಾಯಿಗಳು ಅವರನ್ನು ಇಂದು ಹೇಗೆ ಅಲಂಕರಿಸಬೇಕೆಂದು ಇಷ್ಟಪಟ್ಟರೂ ಆ ವೇಷದಲ್ಲಿಯೇ ಅವರು ಒಪ್ಪುತ್ತಿದ್ದರು. ಕೊನೆಗೆ ಸ್ವಾಮೀಜಿ ಅಪ್ಪಣೆಯಂತೆ ಒಂದು ಕಾವಿಯ ಬಟ್ಟೆಯನ್ನು ತಂದು ಗಿರೀಶ ಬಾಬುಗಳಿಗೆ ಉಡಿಸಿದ್ದಾಯಿತು. ಗಿರೀಶಬಾಬುಗಳು ಯಾವ ಅಡ್ಡಿಯನ್ನೂ ಮಾಡಲಿಲ್ಲ. ಗುರುಭಾಯಿಗಳ ಇಚ್ಛೆಯಂತೆ ಅವರು ಇಂದು ಕೈಕಾಲುಗಳನ್ನು ಧಾರಾಳವಾಗಿ ನೀಡಿದರು. ಆಗ ಸ್ವಾಮೀಜಿ ಗಿರೀಶಘೋಷರಿಗೆ “ನೀವು ಈಗ ಪರಮಹಂಸರ ವಿಚಾರವನ್ನು ಹೇಳಬೇಕು” ಎಂದರು. ಗಿರೀಶಬಾಬುಗಳಿಗೆ ಬಾಯಲ್ಲಿ ಮಾತು ಹೊರಡಲಿಲ್ಲ. ಯಾವ ಜನ್ಮೋತ್ಸವಕ್ಕಾಗಿ ಇಂದು ಎಲ್ಲರೂ ಸೇರಿದ್ದರೊ ಅವರ ಲೀಲಾದರ್ಶನ, ಮತ್ತು ಸಾಕ್ಷಾತ್ ಶಿಷ್ಯರ ಆನಂದ ದರ್ಶನ ಇವುಗಳಿಂದ ಉಂಟಾದ ಸುಖದಲ್ಲಿ ಜಡರಂತಾಗಿಬಿಟ್ಟಿದ್ದರು. ಕೊನೆಗೆ ಗಿರೀಶ ಬಾಬುಗಳು ಹೀಗೆಂದು ಹೇಳಿದರು. “ದಯಾಮಯರಾದ ಪರಮಹಂಸರ ವೃತ್ತಾಂತವನ್ನು ನಾನು ಮತ್ತೇನು ಹೇಳಲಿ. ಕಾಮಕಾಂಚನ ತ್ಯಾಗಿಗಳಾದಂಥ ಬಾಲ ಸಂನ್ಯಾಸಿಗಳ ಜೊತೆಯಲ್ಲಿ, ಅವರು ಈ ಅಧಮನಿಗೆ ಏಕಾಸನದಲ್ಲಿ ಕುಳಿತುಕೊಳ್ಳುವುದಕ್ಕೆ ಅಧಿಕಾರವನ್ನು ಕೊಟ್ಟರಲ್ಲ, ಇದರಿಂದಲೇ ಅವರ ಕರುಣೆಯನ್ನು ಅರ್ಥಮಾಡಿಕೊಂಡಿದ್ದೇನೆ.” ಈ ಮಾತನ್ನು ಹೇಳುತ್ತ ಹೇಳುತ್ತ ಗಿರೀಶ ಬಾಬುಗಳಿಗೆ ಗದ್ಗದ ಕಂಠದಿಂದ ಮಾತು ಹೊರಡದಂತಾಯಿತು. ಅವರು ಆ ದಿವಸ ಮತ್ತೇನನ್ನೂ ಹೇಳದಂತಾಯಿತು. 

 ಜನಿವಾರವನ್ನು ಹಾಕಿಕೊಂಡು ಕುಳಿತುಕೊಂಡಿದ್ದ ಒಬ್ಬ ಗೃಹಸ್ಥನನ್ನು ಸಂಬೋಧಿಸಿ\break ಸ್ವಾಮೀಜಿ ಹೀಗೆಂದರು: “ನೀವು ದ್ವಿಜರು. ಆದರೆ ಬಹುಕಾಲದ ಹಿಂದಿನಿಂದ ವ್ರಾತ್ಯರಾಗಿದ್ದಿರಿ. ಇಂದಿನಿಂದ ಮತ್ತೆ ದ್ವಿಜರಾದಿರಿ. ಪ್ರತಿನಿತ್ಯವೂ ನೂರು ಗಾಯತ್ರಿ ಜಪವನ್ನು ಮಾಡಿ ತಿಳಿಯಿತೆ?” ಗೃಹಸ್ಥನು “ಅಪ್ಪಣೆ” ಎಂದು ಸ್ವಾಮೀಜಿ ಆಜ್ಞೆಯನ್ನು ಶಿರಸಾವಹಿಸಿದನು. 

 ಈ ಮಧ್ಯೆ ಮಹೇಂದ್ರನಾಥ ಗುಪ್ತರು ಬಂದರು. ಸ್ವಾಮೀಜಿ ಅವರನ್ನು ನೋಡಿ: “ಇಂದು ಪರಮಹಂಸರ ಜನ್ಮದಿನ. ಅವರ ವೃತ್ತಾಂತವನ್ನು ನಮಗೆ ಏನಾದರೂ ತಿಳಿಸಬೇಕು” ಎಂದರು. ಮಹೇಂದ್ರನಾಥರು ಮಾತನಾಡಲಿಲ್ಲ. ತಲೆಯನ್ನು ತಗ್ಗಿಸಿಕೊಂಡು ಇದ್ದರು. ಇಷ್ಟರಲ್ಲಿ ಅಖಂಡಾನಂದಸ್ವಾಮಿಗಳು ಮುರ್ಷಿದಾಬಾದಿನಿಂದ ಸುಮಾರು ಒಂದೂವರೆ ಮಣ ತೂಕವುಳ್ಳ ಎರಡು ಪಂತ್ವಗಳನ್ನು (ಮಿಠಾಯಿ) ತೆಗೆದುಕೊಂಡು ಬಂದಿದ್ದರು. ಈ ಅದ್ಭುತವಾದ ಮಿಠಾಯಿಯನ್ನು ನೋಡುವುದಕ್ಕೆ ಎಲ್ಲರೂ ಎದ್ದರು. ಸ್ವಾಮೀಜಿ ಅದನ್ನು ನೋಡಿದ ಮೇಲೆ ದೇವರ ಮನೆಗೆ ಅದನ್ನು ನೈವೇದ್ಯಕ್ಕಾಗಿ ಕೊಡು ಎಂದರು. ಅಖಂಡಾನಂದರ ಕಡೆ ನೋಡುತ್ತ ಸ್ವಾಮೀಜಿ ಶಿಷ್ಯನಿಗೆ ಹೀಗೆ ಹೇಳಿದರು: “ನೋಡಿದೆಯೋ ಎಂಥ ಕರ್ಮವೀರ! ಭಯ ಮೃತ್ಯು ಇದೊಂದರ ಜ್ಞಾನವೂ ಇಲ್ಲ. ಒಂದೇ ಸಮನಾಗಿ ಕರ್ಮ ಮಾಡಿಕೊಂಡು ಹೋಗುತ್ತಿದ್ದಾನೆ.” 

 ಶಿಷ್ಯ: “ಮಹಾರಾಜ್, ಎಷ್ಟೋ ಜನ್ಮಗಳ ತಪಸ್ಸಿನ ಪ್ರಭಾವದಿಂದ ಅವರಲ್ಲಿ ಈ ಶಕ್ತಿ ಬಂದಿದೆ.” 

 ಸ್ವಾಮೀಜಿ: “ತಪಸ್ಸಿನ ಫಲದಿಂದ ಶಕ್ತಿ ಬರುತ್ತದೆ. ಅದರಂತೆ ಸೇವಾಭಾವದಿಂದ ಕರ್ಮ ಮಾಡಿದರೂ ಅದು ತಪಸ್ಸಾಗುತ್ತದೆ. ಕರ್ಮಯೋಗಿ ಕರ್ಮವನ್ನು ತಪಸ್ಸಿನ ಅಂಗವೆಂದೇ ಹೇಳುತ್ತಾನೆ, ತಪಸ್ಸನ್ನು ಮಾಡುತ್ತಿರುವಾಗ ಪರಹಿತೇಚ್ಛೆಯು ಬಲವತ್ತರವಾಗುತ್ತ ಸಾಧಕನ ಕೈಯಿಂದ ಕರ್ಮವನ್ನು ಮಾಡಿಸುತ್ತ ಹೋಗುತ್ತದೆ. ಪರಾರ್ಥವಾಗಿ ಕರ್ಮ ಮಾಡುತ್ತಿದ್ದರೆ ಚಿತ್ತಶುದ್ಧಿಯೂ ಪರಮಾತ್ಮನ ದರ್ಶನವೂ ಲಭಿಸುತ್ತದೆ.” 

 ಶಿಷ್ಯ: “ಮಹಾರಾಜ್, ಮೊದಲಿನಿಂದಲೂ ಅನ್ಯರಿಗೆ ಪ್ರಾಣವನ್ನಾದರೂ ಕೊಟ್ಟು ಎಷ್ಟು ಜನ ಕೆಲಸ ಮಾಡುತ್ತಾರೆ? ಜೀವನು ಆತ್ಮ ಸುಖೇಚ್ಛೆಯನ್ನು ಬಲಿಕೊಟ್ಟು ಪರಾರ್ಥವಾಗಿ ಪ್ರಾಣ ಕೊಡುವಂತಾಗುವಷ್ಟು ಉದಾರತೆ ಹೇಗೆ ಮನಸ್ಸಿಗೆ ಉಂಟಾದೀತು?” 

 ಸ್ವಾಮೀಜಿ: “ತಪಸ್ಸಿನ ಮೇಲೆ ತಾನೆ ಎಷ್ಟು ಜನಕ್ಕೆ ಮನಸ್ಸು ಹೋಗುತ್ತದೆ? ಕಾಮಕಾಂಚನದ ಆಕರ್ಷಣದಲ್ಲಿ ಎಷ್ಟು ಜನರು ತಾನೆ ಭಗವಂತನನ್ನು ಅಪೇಕ್ಷಿಸುವರು? ತಪಸ್ಸು ಎಷ್ಟು ಕಠಿಣವೋ ಅಷ್ಟೇ ಕಠಿಣ ಕರ್ಮ. ಆದುದರಿಂದ ಯಾರು ಪರ ಹಿತಾರ್ಥವಾಗಿ ಕರ್ಮವನ್ನು ಮಾಡುವುದಕ್ಕೆ ಹೊರಡುತ್ತಾರೆಯೋ ಅವರಿಗೆ ವಿರೋಧವಾಗಿ ನೀನು ಏನೂ ಹೇಳುವುದಕ್ಕೂ ಆಗುವುದಿಲ್ಲ. ನಿನಗೆ ತಪಸ್ಸು ಒಗ್ಗಿದರೆ ಹೋಗು ಅದನ್ನು ಮಾಡು. ಮತ್ತೊಬ್ಬನಿಗೆ ಕರ್ಮ ಒಗ್ಗುತ್ತದೆ. ಅವನಿಗೆ ಬೇಡವೆಂದು ಹೇಳುವುದಕ್ಕೆ ಏನು ಅಧಿಕಾರ? ಕರ್ಮ ತಪಸ್ಸಲ್ಲವೆಂಬುದು ನಿನ್ನ ದೃಢವಾದ ಅಭಿಪ್ರಾಯ, ನಾನು ಬಲ್ಲೆ.” 

 ಶಿಷ್ಯ: “ಆದರೆ ಹಿಂದೆ ತಪಸ್ಸು ಎಂದರೆ ನಾನು ಬೇರೊಂದು ಬಗೆಯ ಅರ್ಥವನ್ನು ಮಾಡಿಕೊಂಡಿದ್ದೆ.” 

 ಸ್ವಾಮೀಜಿ: “ಸಾಧನೆ ಭಜನೆಗಳನ್ನು ಮಾಡುತ್ತ ಮಾಡುತ್ತ ಅದರಲ್ಲಿ ಹೇಗೆ ಪ್ರಬಲವಾದ ಪ್ರವೃತ್ತಿ ಹುಟ್ಟುತ್ತದೆಯೋ, ಹಾಗೆಯೇ ಇಚ್ಛೆಯಿಲ್ಲದೆ ಹೋದರೂ ಕರ್ಮವನ್ನು ಮಾಡುತ್ತ ಮಾಡುತ್ತ ಹೋದರೆ ಮನುಷ್ಯ ಕ್ರಮೇಣ ಅದರಲ್ಲಿಯೇ ಮುಳುಗಿ ಹೋಗುವನು. ಕ್ರಮವಾಗಿ ಪರಾರ್ಥ ಕರ್ಮದಲ್ಲಿ ಪ್ರವೃತ್ತಿ ಉಂಟಾಗುತ್ತದೆ. ತಿಳಿಯಿತೆ? ಇಷ್ಟವಿಲ್ಲದೇ ಇದ್ದರೂ ಪರರಿಗೆ ಒಂದು ಸಾರಿ ಸೇವೆ ಮಾಡಿ ನೋಡಿ. ಆಗ ಅದರಿಂದ ಫಲ ಬರುತ್ತದೆಯೋ ಇಲ್ಲವೋ ಎಂಬುದನ್ನು ನೋಡು. ಪರಾರ್ಥವಾದ ಕರ್ಮದಿಂದ, ಮನಸ್ಸಿನ ತೊಡಕುಗಳೆಲ್ಲ ನೇರವಾಗಿ ಮನುಷ್ಯನು ಅಕಪಟದಿಂದ ಪರಹಿತಕ್ಕಾಗಿ ಜೀವವನ್ನು ತೆಯ್ಯುವುದಕ್ಕೆ ಹೊರಡುವನು,” 

 ಶಿಷ್ಯ: “ಆದರೆ ಪರಹಿತವೇತಕ್ಕೆ ಆಗಬೇಕು?” 

 ಸ್ವಾಮೀಜಿ: “ತನ್ನ ಹಿತಕ್ಕೋಸ್ಕರ. ಈ ದೇಹದಲ್ಲಿ ಏತರಿಂದ ನಾನು ಎಂಬ ಅಭಿಮಾನವನ್ನು ಇಟ್ಟುಕೊಂಡು ಕುಳಿತಿದ್ದೀಯೋ ಆ ದೇಹವನ್ನು ಪರರಿಗೋಸ್ಕರ ತೆಗೆದಿಟ್ಟಿರುವೆನು ಎಂಬ ವಿಷಯವನ್ನು ಭಾವಿಸಿಕೊಳ್ಳುತ್ತ ಹೋದರೆ ಈ ಅಹಂಕಾರವನ್ನು ಮರೆಯಬೇಕಾಗುತ್ತದೆ. ಕಡೆಯಲ್ಲಿ ವಿದೇಹ ಬುದ್ಧಿ ಬರುತ್ತದೆ. ನೀನು ಎಷ್ಟೆಷ್ಟು ಏಕಾಗ್ರತೆಯಿಂದ ಪರರ ಹಿತವನ್ನು ಭಾವಿಸುತ್ತೀಯೊ ಅಷ್ಟಷ್ಟು ನಿನ್ನನ್ನು ಮರೆಯುತ್ತ ಹೋಗುತ್ತೀಯೆ. ಹೀಗೆ ಕರ್ಮದಿಂದ ಚಿತ್ತಶುದ್ಧಿಯಾದರೆ, ನಿನ್ನಲ್ಲಿ ಇರುವ ಆತ್ಮವೇ ಸಮಸ್ತರ ಜೀವನದಲ್ಲಿಯೂ, ಸಮಸ್ತ ವಸ್ತುಗಳಲ್ಲಿಯೂ ವಿರಾಜಿಸುತ್ತಿದೆ ಎಂಬ ತತ್ತ್ವವನ್ನು ಅರಿಯಲು ಸಾಧ್ಯವಾಗುವುದು. ಆದಕಾರಣವೆ ತನ್ನಾತ್ಮ ವಿಕಾಸಕ್ಕೆ ಪರಹಿತ ಸಾಧನೆ ಒಂದು ಮಾರ್ಗ. ಇದರ ಉದ್ದೇಶವೂ ಆತ್ಮವಿಕಾಸವೇ. ಜ್ಞಾನ, ಭಕ್ತಿ ಮುಂತಾದ ಸಾಧನೆಯಿಂದ ಹೇಗೆ ಆತ್ಮವಿಕಾಸವಾಗುತ್ತದೆಯೋ ಪರಾರ್ಥಕರ್ಮ ಮಾಡುವುದರಿಂದಲೂ ಹಾಗೆಯೇ ಆಗುತ್ತದೆ.” 

 ಶಿಷ್ಯ: “ಆದರೆ ನಾನು ಹಗಲು ರಾತ್ರಿ ಪರರ ಚಿಂತನೆಯನ್ನೇ ಹಚ್ಚಿಕೊಂಡರೆ ಆತ್ಮವಿಚಾರವನ್ನು ಮಾಡುವುದು ಯಾವಾಗ? ಒಂದು ವಿಶೇಷ ಭಾವವನ್ನು ಹಿಡಿದುಕೊಂಡು ಕುಳಿತರೆ ಅಭಾವರೂಪಿಯಾದ ಆತ್ಮವನ್ನು ಸಾಕ್ಷಾತ್ಕಾರ ಮಾಡಿಕೊಳ್ಳುವುದು ಹೇಗೆ?” 

 ಸ್ವಾಮೀಜಿ: “ಆತ್ಮ ಜ್ಞಾನಲಾಭವೇ ಸಕಲ ಸಾಧನೆಯ, ಸಕಲ ಮಾರ್ಗದ ಮುಖ್ಯ ಉದ್ದೇಶ. ನೀನು ಸೇವಾಪರನಾಗಿ, ಈ ಕರ್ಮದಿಂದ ಚಿತ್ತಶುದ್ಧಿಯನ್ನು ಪಡೆದು, ಸರ್ವಜೀವಿಯನ್ನೂ ಆತ್ಮದ ಹಾಗೆ ನೋಡಬಲ್ಲೆ. ಆದರೆ ಆತ್ಮದರ್ಶನದಲ್ಲಿ ಇನ್ನು ಉಳಿದದ್ದು ಏನು? ಆತ್ಮದರ್ಶನವೆಂದರೇನು ಜಡವಸ್ತುವಿನ ಹಾಗೆ ಈ ಗೋಡೆಯ ಹಾಗೆ ಅಥವಾ ಮರದ ತುಂಡಿನ ಹಾಗೆ ಬಿದ್ದಿರುವುದೆಂದು ತಿಳಿದುಕೊಂಡೆಯಾ?” 

 ಶಿಷ್ಯ: “ಹಾಗಿಲ್ಲದೇ ಇದ್ದರೂ ಸರ್ವವೃತ್ತಿ ಮತ್ತು ಕರ್ಮದ ನಿರೋಧನವನ್ನೇ ತಾನೆ ಶಾಸ್ತ್ರ ಆತ್ಮದ ಸ್ವಸ್ವರೂಪ ಅವಸ್ಥಾನವೆಂದು ಹೇಳುವುದು?” 

 ಸ್ವಾಮೀಜಿ: “ಶಾಸ್ತ್ರದಲ್ಲಿ ಯಾವುದನ್ನು ಸಮಾಧಿ ಎಂದು ಹೇಳಿದೆಯೋ ಆ ಸ್ಥಿತಿ ಸುಲಭವಾಗಿ ದೊರೆಯುವುದಿಲ್ಲ. ಯಾವಾಗಲಾದರೂ, ಯಾರಿಗಾದರೂ, ದೊರೆತರೂ ಹೆಚ್ಚು ಕಾಲ ಸ್ಥಾಯಿಯಾಗಿ ಇರುವುದಿಲ್ಲ. ಆಗ ಅವನು ಯಾವುದರ ಮೇಲೆ ಇರಬೇಕು ಹೇಳು? ಆದಕಾರಣ ಶಾಸ್ತ್ರೋಕ್ತವಾದ ಸ್ಥಿತಿ ಬಂದನಂತರ ಸಾಧಕನು ಪ್ರತಿ ಪ್ರಾಣಿಯಲ್ಲಿಯೂ ಆತ್ಮನನ್ನು ಕಾಣುತ್ತಾನೆ ಮತ್ತು ಅಭೇದಸ್ಥಾನದಿಂದ ಸೇವೆಮಾಡುತ್ತ ಪ್ರಾರಬ್ಧ ಕರ್ಮವನ್ನು ಸವೆಸುತ್ತಾನೆ. ಈ ಅವಸ್ಥೆಗೆ ಶಾಸ್ತ್ರಕಾರರು ಜೀವನ್ಮುಕ್ತಿ ಅವಸ್ಥೆ ಎಂದು ಹೇಳಿದ್ದಾರೆ.” 

 ಶಿಷ್ಯ: “ಹಾಗಾದರೆ ಜೀವನ್ಮುಕ್ತಿ ಅವಸ್ಥೆಯನ್ನು ಪಡೆಯದೇ ಇದ್ದರೆ ಸರಿಯಾಗಿ ಪರಾರ್ಥ ಕರ್ಮವನ್ನು ಮಾಡುವುದಕ್ಕೆ ಆಗುವುದಿಲ್ಲವೆಂದು ಹೇಳಿದ ಹಾಗಾಯಿತು.” 

 ಸ್ವಾಮೀಜಿ: “ಶಾಸ್ತ್ರದಲ್ಲಿ ಈ ಮಾತು ಹೇಳಿದೆ. ಇನ್ನೂ ಹೇಳಿರುವುದೇನೆಂದರೆ, ಪರಾರ್ಥ ಸೇವೆ ಮಾಡುತ್ತ ಮಾಡುತ್ತ ಸಾಧಕನಿಗೆ ಜೀವನ್ಮುಕ್ತಿಯ ಅವಸ್ಥೆಯು ಬರುತ್ತದೆ. ಹಾಗಲ್ಲದೇ ಇದ್ದರೆ, ಕರ್ಮಯೋಗವೆಂದು ಬೇರೊಂದು ಮಾರ್ಗವನ್ನು ಶಾಸ್ತ್ರ ಉಪದೇಶ ಮಾಡಬೇಕಾಗಿರಲಿಲ್ಲ.” 

