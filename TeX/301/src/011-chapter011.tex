
\chapter{ಬಾರಾನಗರ ಮಠ }

ಶ‍್ರೀರಾಮಕೃಷ್ಣರ ನಿರ‍್ಯಾಣಾನಂತರ ಅವರ ಶಿಷ್ಯವರ್ಗ ನರೇಂದ್ರನ ನೇತೃತ್ವದಲ್ಲಿ ಕಾಶೀಪುರದ ತೋಟದ ಮನೆಯಲ್ಲಿಯೇ ಕೆಲವು ಕಾಲ ಇದ್ದರು. ಶ‍್ರೀರಾಮಕೃಷ್ಣರ ಅವಶೇಷವನ್ನು ಅವರು ಜೀವಿಸುತ್ತಿದ್ದ ಕೋಣೆಯೊಳಗೆ ಇಟ್ಟು ಪ್ರತಿದಿನವೂ ಪೂಜೆಯನ್ನು ಮಾಡುತ್ತಿದ್ದರು. ಗೃಹಸ್ಥ ಶಿಷ್ಯರಲ್ಲಿ ಪ್ರಮುಖನಾದ ರಾಮಚಂದ್ರದತ್ತ ಎಂಬುವನು ಕಾಂಕೂರಗಾಚಿ ಎಂಬಲ್ಲಿ ಯೋಗೋದ್ಯಾನವೆಂಬ ಆಶ್ರಮವನ್ನು ಕಟ್ಟಿದ್ದನು. ಅಲ್ಲಿಗೆ ಶ‍್ರೀರಾಮಕೃಷ್ಣರ ಅಸ್ಥಿಯನ್ನು ತೆಗೆದುಕೊಂಡು ಹೋಗಬೇಕೆಂದು ಶ‍್ರೀರಾಮಕೃಷ್ಣರ ಸಂನ್ಯಾಸೀ ಶಿಷ್ಯರನ್ನು ಕೇಳಿಕೊಂಡನು. ಅವರು ಕೊಡಲು ಒಪ್ಪಲಿಲ್ಲ. ಅವರ ಪಾಲಿಗೆ ಶ‍್ರೀರಾಮಕೃಷ್ಣರ ನೆನಪನ್ನು ಕೊಡತಕ್ಕದ್ದು ಈಗ ಅವರ ಅವಶೇಷ ಒಂದೇ ಆಗಿತ್ತು. ಉಳಿದವರು ಯಾರು ಒಪ್ಪಿದರೂ ಶಶಿ ಮತ್ತು ನಿರಂಜನ ಅದನ್ನು ಒಪ್ಪುತ್ತಿರಲಿಲ್ಲ. ಅವರು ನರೇಂದ್ರನ ಹತ್ತಿರ ಬಂದು ಕೇಳಿಕೊಂಡರು. ಆಗ ನರೇಂದ್ರ ತನ್ನ ಇತರ ಗುರುಭಾಯಿಗಳಿಗೆ ಹೀಗೆ ಹೇಳಿ ಅವರನ್ನು ಒಪ್ಪಿಸಿದನು: “ ಶ‍್ರೀರಾಮಕೃಷ್ಣರ ಶಿಷ್ಯರು ಅವರ ಅಸ್ಥಿಗಾಗಿ ಜಗಳ ಕಾದರು ಎಂದು ಜನ ಹೇಳದೇ ಇರಲಿ. ಅವರು ಅಸ್ಥಿಯನ್ನು ತೆಗೆದುಕೊಂಡು ಹೋಗಲಿ. ನಾವು ಶ‍್ರೀಗುರುದೇವನ ಬೋಧನೆಯ ಮೇರೆಗೆ ಜೀವನವನ್ನು ರೂಪಿಸಿಕೊಳ್ಳೋಣ. ನಾವು ಅವರ ಆದರ್ಶಕ್ಕೆ ತಕ್ಕಂತೆ ನಡೆದುಕೊಂಡರೆ, ಕೇವಲ ಅವರ ಅವಶೇಷಗಳನ್ನು ಪೂಜಿಸುವುದಕ್ಕಿಂತ ಹೆಚ್ಚನ್ನು ಸಾಧಿಸಿದಂತೆ ಆಗುವುದು.” ಅಸ್ಥಿಯನ್ನು ತೆಗೆದುಕೊಂಡು ಹೋಗುವ ಹಿಂದಿನ ದಿನ ರಾತ್ರಿ ಶಿಷ್ಯರು ಪುನಃ ಯೋಚಿಸತೊಡಗಿದರು. ಎಲ್ಲವನ್ನೂ ಅವರಿಗೇ ಏತಕ್ಕೆ ಕೊಡಬೇಕು? ಅವರಿಗೆ ಸ್ವಲ್ಪ ಭಾಗವನ್ನು ಕೊಟ್ಟು ನಾವು ಉಳಿದುದನ್ನು ಇಟ್ಟುಕೊಳ್ಳೋಣ ಎಂದು ತೀರ್ಮಾನಮಾಡಿದರು. ನರೇಂದ್ರನ ಅನುಮತಿಯೊಂದಿಗೆ ಅಸ್ಥಿಯ ಬಹುಭಾಗವನ್ನು ಬೇರೊಂದು ಪಾತ್ರೆಗೆ ವರ್ಗಾಯಿಸಿ ಶಿಷ್ಯರು ಅದನ್ನು ಬಲರಾಮನ ಮನೆಗೆ ಕಳುಹಿಸಿದರು. ಮಾರನೆಯ ದಿನ ರಾಮಚಂದ್ರದತ್ತನ ಕಡೆಯವರು ಬಂದಾಗ ಮೊದಲನೆಯ ಅಸ್ಥಿಯ ಪಾತ್ರೆಯನ್ನು ಅವರಿಗೆ ಕೊಟ್ಟು ಕಳುಹಿಸಿದರು. ಅವರು ಅದನ್ನು ಯೋಗೋದ್ಯಾನದಲ್ಲಿ ಪ್ರತಿಷ್ಠೆ ಮಾಡಿ ಪ್ರತಿವರುಷವೂ ಉತ್ಸವವನ್ನು ಆಚರಿಸತೊಡಗಿದರು.

\vskip  3pt

ಶ‍್ರೀಶಾರದಾದೇವಿಯವರು ಬೃಂದಾವನಕ್ಕೆ ಲಾಟು ಮತ್ತು ಯೋಗಾನಂದರೊಡನೆ ತೀರ್ಥಯಾತ್ರೆಗೆ ಹೋದಾಗ ಶ‍್ರೀರಾಮಕೃಷ್ಣರ ಅವಶೇಷದ ಒಂದು ಭಾಗವನ್ನು ತಮ್ಮ ಜೊತೆಯಲ್ಲಿಯೇ ತೆಗೆದುಕೊಂಡು ಹೋಗಿ ಪ್ರತಿದಿನವೂ ಅದಕ್ಕೆ ಪೂಜೆಯನ್ನ ಮಾಡುತ್ತಿದ್ದರು. ಅದರ ಸ್ವಲ್ಪ ಭಾಗವನ್ನು ಹರಿದ್ವಾರ ಕಾಶಿ ಮುಂತಾದ ಪವಿತ್ರ ಸ್ಥಳದಲ್ಲಿ ವಿಸರ್ಜಿಸುವುದಕ್ಕೆ ಕಳುಹಿಸಿ ಉಳಿದುದನ್ನು ಹಿಂತಿರುಗಿ ಬಂದಮೇಲೆ ನರೇಂದ್ರನಾಥನ ವಶಕ್ಕೆ ಕೊಟ್ಟರು.

\vskip  3pt

ಕಾಶೀಪುರದ ತೋಟದ ಮನೆಯನ್ನು ಬಿಡಬೇಕಾಗಿ ಬಂದಿತು. ಶ‍್ರೀರಾಮಕೃಷ್ಣರು ಜೀವಿಸುತ್ತಿದ್ದಾಗಲೇ ಕೆಲವು ಶಿಷ್ಯರು ತಮ್ಮ ಮನೆಯನ್ನು ತ್ಯಜಿಸಿ ಅವರ ಸೇವೆಯಲ್ಲಿ ನಿರತರಾಗಿದ್ದರು. ತಾರಕ, ಲಾಟು, ಗೋಪಾಲ ಇವರುಗಳೇ ಅವರು. ಅವರಿಗೆ ಪುನಃ ಮನೆಗೆ ಹಿಂತಿರುಗಿ ಹೋಗಲು ಇಚ್ಛೆಯಾಗಲಿಲ್ಲ. ಶ‍್ರೀಶಾರದಾದೇವಿಯೊಡನೆ ಲಾಟು ಮತ್ತು ಯೋಗಿನ್ ಬೃಂದಾವನಕ್ಕೆ ಹೋಗಿದ್ದರು. ತಾರಕ ಕೆಲವು ದಿನಗಳಲ್ಲಿಯೇ ಅವರನ್ನು ಅನುಸರಿಸಿದರು. ಯುವಕ ಶಿಷ್ಯರು ಸಂನ್ಯಾಸಿಗಳಾಗಿ ಮನೆಯನ್ನು ತ್ಯಜಿಸಬೇಕೆಂದು ನರೇಂದ್ರ ಇಚ್ಛಿಸಿದ. ಆದರೆ ಆ ಯುವಕರ ತಂದೆ ತಾಯಿಗಳಾದರೋ ಆ ಹುಡುಗರು ಮನೆ ಬಿಟ್ಟುಹೋಗುವುದನ್ನು ಒಪ್ಪಲಿಲ್ಲ. ಶ‍್ರೀರಾಮಕೃಷ್ಣರು ಬದುಕಿದ್ದ ಕಾಲದಲ್ಲಿಯೇನೋ ಅವರ ಸೇವೆಗಾಗಿ ಸ್ಕೂಲು ಕಾಲೇಜನ್ನು ಮರೆಯಬಹುದಾಗಿತ್ತು. ಈಗ ಅವರು ತೀರಿಹೋದಮೇಲೆ ಹುಡುಗರು ಮುಂದೆ ಯಾವುದಾದರೂ ಕೆಲಸಕ್ಕೆ ಕೈ ಹಾಕಿ ಅದರಲ್ಲಿ ನಿರತರಾಗಬೇಕೆಂದು ಆಶಿಸಿದರು. ತಂದೆತಾಯಿಗಳಿಗೆ ಹುಡುಗರು ಎಲ್ಲಾ ಬಿಟ್ಟು ಸಂನ್ಯಾಸಿಗಳಾಗುವುದು ಇಚ್ಛೆಯಿರಲಿಲ್ಲ. ಕೆಲವು ತಂದೆ ತಾಯಿಗಳು ತಮ್ಮ ಮಕ್ಕಳನ್ನು ಮನೆಗೆ ಬರುವುದಕ್ಕೆ ಬಲಾತ್ಕಾರ ಮಾಡಿದರು. ಅವರು ವಿಧಿಯಿಲ್ಲದೆ ಮನೆಗಳಿಗೆ ಹಿಂತಿರುಗಿ ತಮ್ಮ ಅಧ್ಯಯನವನ್ನು ಮುಂದುವರಿಸತೊಡಗಿದರು. ಆದರೆ ಯಾರು ಮನೆಗೆ ಹಿಂತಿರುಗಿ ಹೋಗುವುದೇ ಇಲ್ಲ ಎಂದು ಶಪಥ ತೊಟ್ಟಿದ್ದರೊ ಅವರು ವಾಸಿಸುವುದಕ್ಕೆ ಒಂದು ಸ್ಥಳವಿರಲಿಲ್ಲ. ಶ‍್ರೀರಾಮಕೃಷ್ಣರ ಗೃಹಸ್ಥ ಶಿಷ್ಯನಾದ ಸುರೇಂದ್ರನಾಥಮಿತ್ರ ಎಂಬುವನು ಶ‍್ರೀರಾಮಕೃಷ್ಣರು ಬದುಕಿರುವ ಪರ್ಯಂತರವೂ ಅವರ ಖರ್ಚಿಗೆ ತಗಲುವ ಬಹುಪಾಲು ವೆಚ್ಚವನ್ನು ವಹಿಸುತ್ತಿದ್ದನು. ಈಗ ಆ ವೆಚ್ಚವೆಲ್ಲ ಉಳಿಯಿತು. ಒಂದು ದಿನ ಸುರೇಂದ್ರನಿಗೆ ಒಂದು ಅನುಭವ ಆಯಿತು. ಅವನಿಗೆ ಶ‍್ರೀರಾಮಕೃಷ್ಣರು ಕಾಣಿಸಿಕೊಂಡು, “ನೀನೇನು ಮಾಡುತ್ತಿರುವೆ? ನನ್ನ ಮಕ್ಕಳು ನಿರ್ಗತಿಕರಾಗಿದ್ದಾರೆ. ಅವರಿಗಾಗಿ ಏನಾದರು ವ್ಯವಸ್ಥೆ ಮಾಡು,” ಎಂದು ಆದೇಶವಿತ್ತರು. ಮಾರನೇ ದಿನವೇ ಸುರೇಂದ್ರ ನರೇಂದ್ರನಾಥನನ್ನು ಕಂಡು “ಶ‍್ರೀರಾಮಕೃಷ್ಣರ ತರುವಾಯ ಗೃಹಸ್ಥರಿಗೆ ಪ್ರಾಪಂಚಿಕ ವ್ಯಾಕುಲದಿಂದ ಪಾರಾಗುವ ಒಂದು ಸ್ಥಳವಿಲ್ಲ. ನಿಮ್ಮಂತಹ ತ್ಯಾಗಿಗಳೆಲ್ಲ ಒಂದು ಕಡೆ ನಿಂತು ಆಧ್ಯಾತ್ಮಿಕ ಸಾಧನೆಯನ್ನು ಮಾಡಿದರೆ ಗೃಹಸ್ಥರು ಮಧ್ಯೆ ಮಧ್ಯೆ ಆ ಸ್ಥಳಕ್ಕೆ ಬಂದು ಶಾಂತಿಯನ್ನು ಪಡೆದು ಹೋಗಬಹುದು. ಅದಕ್ಕಾಗಿ ಶ‍್ರೀರಾಮಕೃಷ್ಣರು ಬದುಕಿದ್ದಾಗ ಎಷ್ಟು ಹಣವನ್ನು ಅವರ ಸೇವೆಗೆ ಕೊಡುತ್ತಿದ್ದೆನೋ ಅದನ್ನೇ ಶ‍್ರೀರಾಮಕೃಷ್ಣರ ಯುವಕ ಶಿಷ್ಯವರ್ಗಕ್ಕೆ ಕೊಡಲು ಸಿದ್ಧನಾಗಿರುವೆ. ಮನೆಯ ಬಾಡಿಗೆ ಮತ್ತು ಕೆಲವು ಜನ ಸರಳ ಜೀವನ ಮಾಡಲು ಎಷ್ಟು ವೆಚ್ಚ ತಗಲುವುದೋ ಅದನ್ನೆಲ್ಲಾ ನಾನೇ ವಹಿಸಿಕೊಳ್ಳುತ್ತೇನೆ” ಎಂದ. ನರೇಂದ್ರನಿಗೆ ಅದನ್ನು ಕೇಳಿ ತುಂಬಾ ಸಂತೋಷವಾಯಿತು.

\vskip  3pt

ನರೇಂದ್ರ ಬಾರಾನಗರದಲ್ಲಿ ಒಂದು ಮನೆಯನ್ನು ಗೊತ್ತುಮಾಡಿದನು. ಅದಕ್ಕೆ ಬಾಡಿಗೆ ತಿಂಗಳಿಗೆ ಹತ್ತು ರೂಪಾಯಿ. ಆ ಮನೆಯನ್ನು ದೆವ್ವದ ಮನೆ ಎಂದು ಜನ ಹೇಳುತ್ತಿದ್ದುದರಿಂದ ಅದರಲ್ಲಿ ಇರಲು ಯಾರೂ ಬರುತ್ತಿರಲಿಲ್ಲ. ಜೊತೆಗೆ ತುಂಬಾ ಹಳೆಯದು ಬೇರೆ, ಹಲವು ಭಾಗಗಳು ಕುಸಿದು ಬಿದ್ದು ಹೋಗುತ್ತಿತ್ತು. ಆ ಮನೆಗೆ ಎರಡು ಅಂತಸ್ತು ಇತ್ತು. ಕೆಳಗಿನ ಅಂತಸ್ತಿನಲ್ಲಿ ಹಲ್ಲಿ ಹಾವುಗಳು ಮನೆ ಮಾಡಿಕೊಂಡಿದ್ದುವು. ಮನೆಗೆ ಬರುವ ಹೊರಗಿನ ಪೌಳಿಯ ಬಾಗಿಲು ಕುಸಿದುಹೋಗಿತ್ತು. ಕೆಳಗಿನ ವರಾಂಡವಾದರೋ ಬಿರುಕು ಬಿಟ್ಟು ಇಂದೊ ನಾಳೆಯೋ ಬೀಳುವ ಸ್ಥಿತಿಯಲ್ಲಿತ್ತು. ಸಾಧುಗಳೆಲ್ಲ ಇದ್ದ ಮುಖ್ಯ ಕೋಣೆಯೂ ಅದೇ ದುರವಸ್ಥೆಯಲ್ಲಿದ್ದಿತು. ಮನೆಯ ಪೂರ್ವದಲ್ಲಿದ್ದ ಒಂದು ಕೋಣೆಯನ್ನೇ ಪೂಜಾಮಂದಿರವಾಗಿ ಮಾಡಿಕೊಂಡರು. ಮನೆಯ ಪಶ್ಚಿಮಕ್ಕೆ ಒಂದು ಸೊಪ್ಪುಸದೆಗಳಿಂದ ಬೆಳೆದುಕೊಂಡು ಹೋದ ತೋಟವಿತ್ತು. ಮನೆಯ ಹಿಂದೆ ಒಂದು ಕೊಳವಿತ್ತು. ಆ ನೀರಿನ ಮೇಲೆ ಪಾಚಿ ಬೆಳೆದುಕೊಂಡು ಹೋಗಿ ಸೊಳ್ಳೆಯ ತವರುಮನೆಯಾಗಿತ್ತು. ಮನೆಯನ್ನು ನೋಡಿದರೆ ಯಾವುದೋ ಭೂತದ ಮನೆಯ ಜ್ಞಾಪಕ ಬರುತ್ತಿತ್ತು. ಆದರೂ ಅದನ್ನೇ ತೆಗೆದುಕೊಂಡರು. ಏಕೆಂದರೆ ಬಾಡಿಗೆ ಬಹಳ ಕಡಿಮೆ, ಗಂಗಾನದಿಯ ಸಮೀಪದಲ್ಲಿತ್ತು ಮತ್ತು ಶ‍್ರೀರಾಮಕೃಷ್ಣರ ಸಮಾಧಿಸ್ಥಳಕ್ಕೆ ಹತ್ತಿರವಿತ್ತು. ಸಂನ್ಯಾಸಿಗಳಿಗಾದರೋ\break ಇರುವುದಕ್ಕೆ ಒಂದು ಸ್ಥಳ ಸಿಕ್ಕಿತು. ಮತ್ತು ಈ ಸ್ಥಳ ಗಲಾಟೆಯ ಕಲ್ಕತ್ತೆಯಿಂದ\break ದೂರದಲ್ಲಿದ್ದುದರಿಂದ ಪ್ರಶಾಂತ ವಾತಾವರಣವಿತ್ತು. ಮೊದಲು ಬೃಂದಾವನದಿಂದ\break ಹಿಂತಿರುಗಿ ಬಂದ ತಾರಕ ಮತ್ತು ಗೋಪಾಲದಾದ ಇವರು ಮಾತ್ರ ಇಲ್ಲಿ ವಾಸಿಸತೊಡಗಿದರು. ಉಳಿದವರು ತಮ್ಮ ತಮ್ಮ ಮನೆಗಳಿಗೆ ಹೋಗಿ ಇಲ್ಲಿಗೆ ಬಂದು ಹೋಗುತ್ತಿದ್ದರು. ಮತ್ತೆ ಕೆಲವರಾದರೊ ಯಾತ್ರೆಗೆ ಹೋಗಿದ್ದರು. ರಾಖಾಲ ಮಾಂಘೀರದಲ್ಲಿದ್ದನು. ಕಾಳಿ, ಯೋಗಿನ್, ಲಾಟು ಇವರು ಬೃಂದಾವನದಲ್ಲಿ ಶ‍್ರೀಶಾರದಾದೇವಿಯವರೊಡನೆ ಇದ್ದರು. ನರೇಂದ್ರನಾದರೊ ಮಧ್ಯೆ ಮಧ್ಯೆ ಮನೆಗೆ ಹೋಗಿ ಬರಬೇಕಾಗಿತ್ತು. ಏಕೆಂದರೆ ಅವನ ಮನೆಗೆ ಸಂಬಂಧಪಟ್ಟ ಕೇಸು ಕೋರ್ಟಿನಲ್ಲಿ ನಡೆಯುತ್ತಿತ್ತು. ಅವನೇ ಮನೆಗೆ ಹಿರಿಯನಾದುದರಿಂದ ಅದರ ಇತ್ಯರ್ಥವನ್ನು ಮಾಡಬೇಕಾಗಿತ್ತು.

\vskip  3pt

ನರೇಂದ್ರನ ಮನೆಯ ವ್ಯವಹಾರ ಪೂರೈಸಿದ ಮೇಲೆ ಅವರ ಜೀವನೋಪಾಯಕ್ಕೆ ಅಣಿಮಾಡಿ ಒಂದೇ ಸಲ ಬಾರಾನಗರ ಮಠಕ್ಕೆ ಬಂದುಬಿಟ್ಟ. ಅವನು ಬಂದು ನೆಲಸಿದ ಮೇಲೆ ಮನೆಗೆ ಹಿಂತಿರುಗಿ ಹೋಗಿದ್ದ ಹುಡುಗರೊಡನೆ, ಲೌಕಿಕ ವಿದ್ಯೆಯಿಂದ ಆಧ್ಯಾತ್ಮಿಕ ಜೀವನಕ್ಕೆ ಏನೂ ಪ್ರಯೋಜನವಾಗುವುದಿಲ್ಲವೆಂದೂ, ಶ‍್ರೀರಾಮಕೃಷ್ಣರು ತಮಗಾಗಿ ಎಷ್ಟೊಂದು ತೊಂದರೆ ತೆಗೆದುಕೊಂಡಿರುವರೆಂದೂ, ನಾವೆಲ್ಲ ಮುಂದೆ ಒಂದು ಸಂನ್ಯಾಸಿಗಳ ಸೋದರ ಕೂಟವಾಗಬೇಕೆಂಬುದೇ ಅವರ ಇಚ್ಛೆ ಆಗಿದ್ದಿತೆಂದೂ, ಅದಕ್ಕಾಗಿ ಅವರು ತಮ್ಮ ಜೀವನದ ಕೊನೆಯ ಕಾಲದಲ್ಲಿ ಕಾವಿಬಟ್ಟೆಯನ್ನು ಕೊಟ್ಟಿದ್ದರೆಂದೂ ಹೇಳಿದನು. ನರೇಂದ್ರನ ಮಾತಿನ ಸಂಮ್ಮೋಹನಾಸ್ತ್ರದ ಮುಂದೆ ಅವರಿಗೆ ಮಾತನಾಡಲು ಶಕ್ತಿ ಇರಲಿಲ್ಲ. ಬಾರಾನಗರ ಮಠಕ್ಕೆ ಒಮ್ಮೆ ಬಂದಮೇಲೆ, ಆ ಸಹೋದರ ಪ್ರೀತಿಯನ್ನು ಅನುಭವಿಸಿದ ಮೇಲೆ, ಮನೆಯ ಗೀಳು ಬಿಟ್ಟುಹೋಗಿತ್ತು. ನರೇಂದ್ರ ಶ‍್ರೀರಾಮಕೃಷ್ಣರ ಸಂನ್ಯಾಸಿ ಶಿಷ್ಯರನ್ನೆಲ್ಲ ಆಕರ್ಷಿಸಿ ಒಂದುಗೂಡಿಸಿದ. ಅಲ್ಲಿ ಅವರೆಲ್ಲ ಸಾಧನೆ ಭಜನೆಯಲ್ಲಿ ನಿರತರಾದರು. ಶ‍್ರೀರಾಮಕೃಷ್ಣರ ಜೀವನದ ಘಟನೆಗಳು, ಅವರು ಮಾಡಿದ ಉಪದೇಶಗಳು, ಇವುಗಳನ್ನು ನರೇಂದ್ರ ಇತರರಿಗೆ ಮನಮುಟ್ಟುವಂತೆ ಹೇಳುತ್ತಿದ್ದ. ಹಿರಿಯ ಅಣ್ಣನಂತೆ ಅವರನ್ನೆಲ್ಲ ಅಮೋಘವಾದ ಪ್ರೇಮದಿಂದ ಗುರುಭಾಯಿಗಳನ್ನೆಲ್ಲ ಒಂದುಗೂಡಿಸಿದ.

ಕೆಲವು ದಿನಗಳಾದ ಮೇಲೆ ಬಾಬುರಾಮನ ತಾಯಿ ನರೇಂದ್ರನ ಗುರುಭಾಯಿಗಳಿಗೆಲ್ಲ ಕಲ್ಕತ್ತೆಗೆ ಸುಮಾರು ಮೂವತ್ತು ಮೈಲಿಗಳ ದೂರದಲ್ಲಿರುವ ಆಂಟ್​ಪುರಕ್ಕೆ ಬರಬೇಕೆಂದು ನಿಮಂತ್ರಣವನ್ನು ಕಳುಹಿಸಿದಳು. ನರೇಂದ್ರ ಸಂತೋಷದಿಂದ ಒಪ್ಪಿಕೊಂಡು ಬಾಬುರಾಮ, ಶರತ್, ಶಶಿ, ತಾರಕ, ಕಾಳಿ, ನಿರಂಜನ, ಗಂಗಾಧರ, ಶಾರದ ಇವರುಗಳೊಡನೆ ಅಲ್ಲಿಗೆ ಹೋದನು. ಅಲ್ಲಿ ತೀವ್ರವಾದ ಧ್ಯಾನದಲ್ಲಿ ಕಾಲವನ್ನು ಕಳೆದರು. ನರೇಂದ್ರ “ಪುರುಷಸಿಂಹರಾಗುವುದೇ ನಮ್ಮ ಜೀವನದ ಗುರಿಯಾಗಬೇಕು. ಇದೊಂದೇ ನಮ್ಮ ಸಾಧನವಾಗಬೇಕು. ಬರೀ ಪಾಂಡಿತ್ಯದಿಂದ ಪ್ರಯೋಜನವಿಲ್ಲ. ಈ ಪ್ರಪಂಚದ ಪ್ರಲೋಭನಗಳು ನಮ್ಮ ಮನಸ್ಸನ್ನು ಕ್ಷಣಕಾಲವೂ ತಮ್ಮ ಕಡೆಗೆ ಸೆಳೆಯದಿರಲಿ. ಭಗವಂತನ ಸಾಕ್ಷಾತ್ಕಾರವೇ ನಮ್ಮ ಏಕಮಾತ್ರ ಗುರಿಯಾಗಬೇಕು. ಶ‍್ರೀರಾಮಕೃಷ್ಣರ ಜೀವನವೇ ಅದಕ್ಕೆ ಜ್ವಲಂತ\break ಉದಾಹರಣೆ. ನಾವು ಭಗವಂತನನ್ನು ಸಾಕ್ಷಾತ್ಕಾರ ಮಾಡಿಕೊಳ್ಳಬೇಕು” ಎಂದು ಇತರರನ್ನು ಹುರಿದುಂಬಿಸಿದನು. ಶ‍್ರೀರಾಮಕೃಷ್ಣರ ವಿಷಯವಾಗಿ ಯಾವಾಗಲೂ ಮಾತನಾಡುತ್ತಿದ್ದನು. ಎಲ್ಲರ ಬಾಯಿಯಲ್ಲೂ ಅದೇ, ಎಲ್ಲರ ಮನಸ್ಸಿನಲ್ಲಿಯೂ ಅದೇ. ತೀವ್ರ ವೈರಾಗ್ಯದ ಭಾವ ಎಲ್ಲರ ಜೀವನವನ್ನು ಆವರಿಸಿತು.

ಒಂದು ದಿನ ರಾತ್ರಿ ದೊಡ್ಡ ಧುನಿಯನ್ನು ಹಚ್ಚಿ ಸುತ್ತಲೂ ಧ್ಯಾನ ಮಾಡುವುದಕ್ಕೆ ಕುಳಿತರು. ಮೇಲೆ ಅನಂತ ತಾರಾಖಚಿತ ಗಡನ(ಗುಂಪು) ಆಗಸ. ಸುತ್ತಲೂ ನೀರವ ಮೌನ. ಎಲ್ಲರೂ ದೀರ್ಘಕಾಲ ಧ್ಯಾನವನ್ನು ಮಾಡಿದರು. ಅನಂತರ ನರೇಂದ್ರ ಜೀಸಸ್ ಕ್ರೈಸ್ತನ ಜೀವನವಿಷಯವನ್ನು ಗುರುಭಾಯಿಗಳಿಗೆ ಹೇಳತೊಡಗಿದನು. ಅವನ ಜೀವನ, ಜನರ ಅಜ್ಞಾನವನ್ನು ಬಿಡಿಸುವುದಕ್ಕಾಗಿ ಅವನು ಮಾಡಿದ ಪ್ರಯತ್ನ, ಕೊನೆಗೆ ಅವನು ಶಿಲುಬೆಯ ಮೇಲೆ ತನ್ನ ಪ್ರಾಣವನ್ನು ಬಿಟ್ಟದ್ದು ಎಲ್ಲವನ್ನೂ ಮನಮುಟ್ಟುವಂತೆ ಹೇಳಿದನು. ನರೇಂದ್ರ ತನ್ನ ಗುರುಭಾಯಿಗಳಿಗೆ, ತಾವುಗಳೆಲ್ಲಾ ಜಗದ ಕಲ್ಯಾಣಕ್ಕಾಗಿ ಕ್ರೈಸ್ತನಂತೆ ಆಗಬೇಕು ಎಂದು ಹೇಳಿದನು. ಅಂದಿನ ರಾತ್ರಿ ಉರಿಯುತ್ತಿರುವ ಧುನಿ ಮುಂದೆ, ಭಗವಂತನ ಸಾಕ್ಷಿಯಾಗಿ ಎಲ್ಲರೆದುರಿಗೆ ಸಂನ್ಯಾಸ ವ್ರತವನ್ನು ಸ್ವೀಕರಿಸಿದರು. ಬಹುಜನರ ಹಿತಕ್ಕೆ ಬಹುಜನರ ಸುಖಕ್ಕೆ ತಮ್ಮ ಬಾಳನ್ನು ಅರ್ಪಿಸುತ್ತೇವೆ ಎಂದು ಶಪಥತೊಟ್ಟರು. ಅನಂತರ ಅಂದು ಪವಿತ್ರ ದಿನ, ಕ್ರೈಸ್ತ ಹುಟ್ಟಿದ ರಾತ್ರಿ ಎಂದು ಗೊತ್ತಾಯಿತು. ಶ‍್ರೀರಾಮಕೃಷ್ಣ ಸಂಸ್ಥೆ ಜನ್ಮ ತಾಳಿದ್ದು ಅಂದು ಎಂದು ಬೇಕಾದರೆ ಹೇಳಬಹುದು. ಅನಂತರ ಬಾರಾನಗರ ಮಠಕ್ಕೆ ಬಂದಮೇಲೆ ೧೮೮೭ರ ಜನವರಿ ಮೂರನೆಯ ವಾರದಲ್ಲಿ ವಿರಜಾ ಹೋಮಾದಿಗಳನ್ನು ಮಾಡಿ ಶ‍್ರೀರಾಮಕೃಷ್ಣರ ಪಾದುಕೆಗಳ ಮುಂದೆ ವಿಧಿವತ್ತಾಗಿ ಸಂನ್ಯಾಸವನ್ನು ಸ್ವೀಕರಿಸಿ ಬೇರೆ ಬೇರೆ ಹೆಸರುಗಳನ್ನು ತೆಗೆದುಕೊಂಡರು. ಅವರ ಸಂನ್ಯಾಸದ ಹೆಸರುಗಳು ಹೀಗಿದ್ದುವು:

\begin{longtable}{ll}
ರಾಖಾಲ್–ಬ್ರಹ್ಮಾನಂದ & ಹರಿ–ತುರೀಯಾನಂದ \\
ಯೋಗಿನ್–ಯೋಗಾನಂದ & ಶಾರದ–ತ್ರಿಗುಣಾತೀತಾನಂದ \\
ಬಾಬುರಾಮ–ಪ್ರೇಮಾನಂದ & ಗೋಪಾಲ–ದಾದಅದ್ವೈತಾನಂದ \\
ನಿರಂಜನ–ನಿರಂಜನಾನಂದ & ತಾರಕ–ಶಿವಾನಂದ \\
ಶಶಿ–ರಾಮಕೃಷ್ಣಾನಂದ & ಕಾಳಿ–ಅಭೇದಾನಂದ \\
ಶರತ್–ಶಾರದಾನಂದ & ಗಂಗಾಧರ–ಅಖಂಡಾನಂದ \\
ಲಾಟು–ಅದ್ಭುತಾನಂದ & ಸುಬೋಧ್–ಸುಬೋಧಾನಂದ \\
 & ಹರಿಪ್ರಸನ್ನ–ವಿಜ್ಞಾನಾನಂದ \\
\end{longtable}

\vskip  3pt

ದೆವ್ವದ ಮನೆ ಎಂದು ಖ್ಯಾತಿಗೊಂಡ ಆ ಪಾಳು ಮನೆಯಲ್ಲಿ ಮಠ ಪ್ರಾರಂಭವಾದರೂ ಮುಂದಿನ ಶ‍್ರೀರಾಮಕೃಷ್ಣ ಸಂಸ್ಥೆಗೆ ತಪಸ್ಸಿನ ತಳಪಾಯವನ್ನು ಹಾಕಿದ್ದು ಅಲ್ಲಿ. ಆಗ ಅವರಿಗೆ ತಿನ್ನಲು ಸಾಕಷ್ಟು ಇರಲಿಲ್ಲ. ಉಡಲು ಸರಿಯಾದ ಬಟ್ಟೆಗಳು ಇರಲಿಲ್ಲ. ಆದರೂ ಅದರ ಕಡೆ ಗಮನ ಕೊಡುತ್ತಿರಲಿಲ್ಲ. ಧ್ಯಾನ ಅಧ್ಯಯನ ಪ್ರವಚನಗಳಲ್ಲಿ ತನ್ಮಯರಾಗಿ ಹೋಗುತ್ತಿದ್ದರು. ಕಾಲ ಹೇಗೆ ಹೋಗುತ್ತಿದೆ ಎಂಬುದು ಗೊತ್ತಾಗುತ್ತಿರಲಿಲ್ಲ. ಶಶಿ ಮಾತ್ರ ಶ‍್ರೀಗುರುದೇವರ ಭಾವಚಿತ್ರ ಮತ್ತು ಅವರ ಅವಶೇಷಗಳಿಗೆ ಪೂಜೆ ಮಾಡಿ ಉಗ್ರಾಣದಲ್ಲಿ ಏನಿತ್ತೋ ಅದನ್ನು ಅಡಿಗೆ ಮಾಡಿ, ನೈವೇದ್ಯ ಮಾಡಿ, ಸಾಧನೆಯಲ್ಲಿ ನಿರತರಾದ ಗುರುಭಾಯಿಗಳನ್ನು ಊಟಕ್ಕೆ ಎಬ್ಬಿಸಿಕೊಂಡು ಬರುತ್ತಿದ್ದ. ಒಳಗೆ ವಾಸಮಾಡುವಾಗ ಬಹಳ ಸರಳವಾಗಿದ್ದರು. ಬರೀ ಚಾಪೆ ಮಲಗುವುದಕ್ಕೆ. ಒಳಗೆ ಇರುವಾಗ ಕೌಪೀನವನ್ನು ಧರಿಸಿಯೇ ಬಹಳ ಕಾಲ ಇರುತ್ತಿದ್ದರು. ಹೊರಗೆ ಹೋಗಬೇಕಾದರೆ ಮಾತ್ರಪಂಚೆ ಮತ್ತು ಉತ್ತರೀಯವನ್ನು ಧರಿಸಿಕೊಂಡು ಹೋಗುತ್ತಿದ್ದರು. ಅದನ್ನು ಬಾಗಿಲಿನ ಮುಂದೆಯೇ ಇಟ್ಟಿದ್ದರು. ಅದು ಎಲ್ಲರಿಗೂ ಸೇರಿದ ವಸ್ತುವಾಗಿತ್ತು. ಉಪನಿಷತ್ತು ಗೀತಾ ಮುಂತಾದ ಸುಮಾರು ನೂರು ಪುಸ್ತಕಗಳು ಮಾತ್ರ ಇದ್ದವು. ಆದರೆ ಹೊರಗಿನಿಂದ ಬೇಕಾದಷ್ಟು ಪುಸ್ತಕಗಳನ್ನು ತಂದು ಓದುತ್ತಿದ್ದರು. ನರೇಂದ್ರ ಪೌರ ಮತ್ತು ಪಾಶ್ಚಾತ್ಯ ದಾರ್ಶನಿಕರ ಭಾವನೆಗಳನ್ನೆಲ್ಲ ಚೆನ್ನಾಗಿ ತಿಳಿದುಕೊಂಡಿದ್ದನು. ಅದರ ಮೇಲೆ ಇತರರಿಗೆ ಪ್ರವಚನ ಕೊಡುತ್ತಿದ್ದನು. ಸಂಜೆಯ ಹೊತ್ತು ಮಂಗಳಾರತಿ ಭಜನೆ ಧ್ಯಾನ ಇವುಗಳಲ್ಲಿ ಕಳೆದುಹೋಗುತ್ತಿತ್ತು. ನರೇಂದ್ರ ಎಲ್ಲರಿಗಿಂತ ಮುಂಚೆಯೇ ಎದ್ದು ‘ಅಮರ ಪುತ್ರರೇ ಕೇಳಿ’ ಎಂಬ ಉಪನಿಷತ್ತಿನ ವಚನವನ್ನು ಹಾಡುತ್ತ ಇತರರನ್ನು ಎಬ್ಬಿಸಿ ಧ್ಯಾನದಲ್ಲಿ ನಿರತರಾಗುವಂತೆ ಮಾಡುತ್ತಿದ್ದನು. ಇತರ ಗುರುಭಾಯಿಗಳ ಮನಸ್ಸಿನಲ್ಲಿ ಬರುವ ಸಂದೇಹಾದಿಗಳನ್ನು ಪರಿಹರಿಸುತ್ತಿದ್ದನು. ಯಾವಾಗಲೂ ಅವರಿಗೆ ಸ್ಫೂರ್ತಿಯನ್ನು ಕೊಟ್ಟು ಆಧ್ಯಾತ್ಮಿಕ ಜೀವನದಲ್ಲಿ ಮುನ್ನಡೆಯುವಂತೆ ಪ್ರೇರೇಪಿಸುತ್ತಿದ್ದನು.

\vskip  3pt

ನರೇಂದ್ರ ಜ್ಞಾನ ಮತ್ತು ಭಕ್ತಿಗಳ ಅಸದೃಶ ಸಮ್ಮೇಳನ. ಅವನು ಹೊರಗಿನಿಂದ ನೋಡಿದರೆ ಕೇವಲ ಜ್ಞಾನಿಯಂತೆ ಕಂಡರೂ ಒಳಗೆಲ್ಲ ಭಕ್ತಿ ಅಂತರಗಂಗೆಯಂತೆ ಹರಿಯುತ್ತಿತ್ತು. ಭಕ್ತಿಯ ದೃಷ್ಟಿಯಿಂದ ಕೆಲವರನ್ನು ಹುರಿದುಂಬಿಸುತ್ತಿದ್ದನು. ಜ್ಞಾನದ ದೃಷ್ಟಿಯಿಂದ ಕೆಲವರನ್ನು ಹುರಿದುಂಬಿಸುತ್ತಿದ್ದನು. ಅವರವರ ಭಾವಕ್ಕೆ ತಕ್ಕಂತೆ ಅವರಿಗೆ ಸ್ಫೂರ್ತಿಯನ್ನು ಕೊಡುತ್ತಿದ್ದನು. ಒಂದು ಸಲ ಭಗವತ್ಸಾಕ್ಷಾತ್ಕಾರ ಸಾಧ್ಯವಾಗಲಿಲ್ಲವಲ್ಲ ಎಂದು ಅತ್ಯಂತ ನಿರಾಶೆಗೊಂಡ ಗುರುಭಾಯಿಯೊಬ್ಬನಿಗೆ ಭಕ್ತಿಯ ಹಿರಿಮೆಯನ್ನು ಕುರಿತು ನರೇಂದ್ರ ಹೀಗೆಂದನು:

\vskip  3pt

ನರೇಂದ್ರ: “ನೀನು ಗೀತೆಯನ್ನು ಓದಿಲ್ಲವೆ? ದೇವರು ಎಲ್ಲರ ಹೃದಯದಲ್ಲಿಯೂ ನೆಲಸಿರುವನು. ನಾವಿರುವ ಸಂಸಾರಚಕ್ರವನ್ನು ನಡೆಸುತ್ತಿರುವವನು ಅವನೇ. ನೀನೊಂದು ತೆವಳುತ್ತಿರುವ ಕೀಟಕ್ಕಿಂತ ಕಡೆ. ನೀನು ನಿಜವಾದ ದೇವರು ಹೇಗಿರುವನೋ ಹಾಗೆ ತಿಳಿದುಕೊಳ್ಳಬಲ್ಲೆಯಾ? ಮನುಷ್ಯನ ಸ್ಥಿತಿಯನ್ನು ಕುರಿತು ಒಂದು ಕ್ಷಣ ವಿಚಾರಮಾಡಿ ನೋಡು. ನಮಗೆ ಕಾಣುವ ನಕ್ಷತ್ರಗಳಲ್ಲಿ ಎಲ್ಲವೂ ಒಂದೊಂದು ವ್ಯೂಹ. ನಾವಿರುವುದು ಅಂತಹ ಯಾವುದೋ ಒಂದು ಸೌರವ್ಯೂಹದಲ್ಲಿ. ಅದರಲ್ಲಿಯೂ ನಮಗೆ ಗೊತ್ತಿರುವುದು ಬಹಳ ಅಲ್ಪ. ನಮ್ಮ ಭೂಮಿಯನ್ನು ಸೂರ‍್ಯನೊಡನೆ ಹೋಲಿಸಿದರೆ ಒಂದು ಪುಟ್ಟ ಚೆಂಡಿನಂತೆ ಇದೆ. ಮನುಷ್ಯನಾದರೊ ಒಂದು ಕನಿಷ್ಟ ಪಕ್ಷದ ಕೀಟದಂತೆ ಅದರ ಮೇಲೆ\break ಚಲಿಸುತ್ತಿರುವನು.

“ಅವನಲ್ಲಿ ಶರಣಾಗು. ಅವನ ಪಾದಕಮಲಗಳಲ್ಲಿ ಮೊರೆಹೋಗು. ನಿನಗೆ\break ಶ‍್ರೀರಾಮಕೃಷ್ಣರು ಹೇಳುತ್ತಿದ್ದುದು ಜ್ಞಾಪಕವಿಲ್ಲವೆ? ದೇವರು ಒಂದು ಸಕ್ಕರೆಯ ಬೆಟ್ಟ ಇದ್ದಂತೆ, ನೀನೊಂದು ಇರುವೆಯಂತೆ. ನಿನಗೆ ಒಂದು ಸಕ್ಕರೆಯ ಹರಳು ಸಾಕು. ನೀನು ಸಕ್ಕರೆಯ ಬೆಟ್ಟವನ್ನೇ ಮನೆಗೆ ಹೊತ್ತುಕೊಂಡು ಹೋಗಬೇಕೆಂದು ಬಯಸುವೆ, ಶುಕದೇವ ಎಲ್ಲೋ ಒಂದು ಇರುವೆ ಆಗಿದ್ದ. ಅದಕ್ಕೇ ನಾನು ಕಾಳಿ ಪ್ರಸನ್ನನಿಗೆ, ‘ನೀನು ನಿನ್ನ\break ಗಜಕಡ್ಡಿಯಿಂದ ದೇವರನ್ನು ಅಳೆಯಲೆತ್ನಿಸುವೆಯಾ?’ ಎಂದು ಹೇಳುತ್ತಿದ್ದೆ. ಭಗವಂತ ಅನಂತ ಕೃಪಾಸಿಂಧು. ಅವನು ನಿನ್ನ ಮೇಲೆ ತನ್ನ ಕೃಪೆಯನ್ನು ವರ್ಷಿಸುವನು. ‘ದೇವರೇ, ಕೃಪೆದೋರಿ ನಮ್ಮನ್ನು ಉದ್ಧಾರಮಾಡು’ ಎಂದು ಅವನನ್ನು ಬೇಡು. ‘ದೇವರೇ, ನನ್ನನ್ನು ಅಸತ್ಯದಿಂದ ಸತ್ಯದೆಡೆಗೆ, ತಮಸ್ಸಿನಿಂದ ಜ್ಯೋತಿಯ ಕಡೆಗೆ, ಮೃತ್ಯುವಿನಿಂದ ಅಮೃತತ್ವದ ಕಡೆಗೆ ಒಯ್ಯಿ’ ಎಂದು ಅವನನ್ನು ಪ್ರಾರ್ಥಿಸು.”

ಪ್ರಶ್ನೆ: “ಅವನನ್ನು ಹೇಗೆ ಪ್ರಾರ್ಥಿಸಬೇಕು?”

ಪ್ರಶ್ನೆ: “ನೀನು ಈ ಸಮಯದಲ್ಲಿ ದೇವರು ಇರುವನು ಎನ್ನುವೆ. ಮತ್ತೊಂದು ಸಲ ಚಾರ್ವಾಕ ಮತ್ತು ಇತರ ದರ್ಶನಗಳ ಪ್ರಕಾರ, ಈ ಪ್ರಪಂಚವನ್ನು ಯಾರೋ ಅದರ ಹೊರಗಡೆ ಇರುವವರು ಸೃಷ್ಟಿಸಿಲ್ಲ, ಅದು ತನಗೆ ತಾನೇ ಆಯಿತು ಎಂದು ಹೇಳುವೆ.”

ನರೇಂದ್ರ: “ನೀನು ರಸಾಯನ ಶಾಸ್ತ್ರವನ್ನು ಓದಿಲ್ಲವೆ? ಹೈಡ್ರೋಜನ್ ಮತ್ತು ಆಕ್ಸಿಜನ್ ಅನಿಲಗಳು ನೀರು ಮತ್ತು ಇತರ ವಸ್ತುಗಳಾಗುವುದಕ್ಕೆ ಒಂದುಗೂಡುತ್ತವೆ. ಹಾಗೆ ಒಂದುಗೂಡಬೇಕಾದರೆ ಮನುಷ್ಯನೋ ಅಥವಾ ಮತ್ತಾವುದಾದರೂ ಚೇತನವು ಅವರ ಹಿಂದೆ ಇರಬೇಕು. ಈ ಪರಸ್ಪರ ಸಂಯೋಗದ ಮೇಲ್ವಿಚಾರಣೆ ನೋಡಿಕೊಳ್ಳುವುದಕ್ಕೆ ಯಾರೋ ಒಬ್ಬ ಸರ್ವಜ್ಞ ಅದರ ಹಿಂದೆ ಇರಬೇಕು ಎಂದು ಊಹಿಸಲೇಬೇಕಾಗುವುದು.”

ಪ್ರಶ್ನೆ: “ಅವನು ದಯಾಮಯ ಎಂದು ನಮಗೆ ಹೇಗೆ ಗೊತ್ತು?”

ನರೇಂದ್ರ: “ವೇದಗಳು ‘ನಿನ್ನ ಕರುಣಾಪೂರಿತ ಮುಖ’ ಎಂದು ಹೇಳುವುವು. ಜಾನ್‍ಸ್ಟುಯರ‍್ಟ ಮಿಲ್ಲನು ಕೂಡ ಇದನ್ನೇ ಸಮರ್ಥಿಸುವನು. ಮಾನವ ಹೃದಯದಲ್ಲಿ ಒಂದು ಅನುಕಂಪೆಯ ಬಿಂದುವನ್ನು ಇಟ್ಟ ದೇವರು. ಅನುಕಂಪೆಯ ಮಹೋದಧಿಯಾಗಿಯೇ ಇರಬೇಕು. ಶ‍್ರೀರಾಮಕೃಷ್ಣರೂ ಕೂಡ ‘ದೇವರು ನಮಗೆ ಅತ್ಯಂತ ಸಮೀಪದಲ್ಲಿರುವನು. ಅವನನ್ನು ಪಡೆಯಬೇಕಾದರೆ ಶ್ರದ್ಧೆ ಬೇಕು’ ಎಂದು ಹೇಳುತ್ತಿದ್ದರು.”

ಆಗ ಒಬ್ಬ ಗುರುಭಾಯಿ ಸ್ವಲ್ಪ ಹಾಸ್ಯವಾಗಿಯೇ ನರೇಂದ್ರನಿಗೆ ಹೀಗೆ ಪ್ರಶ್ನೆಯನ್ನು ಹಾಕುವನು:

“ಕೆಲವು ವೇಳೆ ನೀನು ದೇವರೇ ಇಲ್ಲ ಎನ್ನುವೆ. ಈಗ ಅವನು ಇರುವನು ಎಂದು ಹೇಳುತ್ತಿರುವೆ. ನೀನು ಒಂದೊಂದು ಸಲ ಒಂದೊಂದನ್ನು ಹೇಳುವೆ. ಅದರಲ್ಲಿ ಯಾವುದನ್ನು ನಿಜ ಎಂದು ಭಾವಿಸುವುದು?”

“ನಾನು ಇನ್ನೆಂದೂ ಈ ಮಾತನ್ನು ಬದಲಾಯಿಸುವುದಿಲ್ಲ: ನಮ್ಮಲ್ಲಿ ಅಹಂಕಾರ ಮತ್ತು ಆಸೆಗಳಿರುವವರೆಗೂ ದೇವರ ಮೇಲೆ ನಂಬಿಕೆಯುಂಟಾಗುವುದಿಲ್ಲ. ಯಾವುದಾದರೊಂದು ಆಸೆ ಉಳಿದುಕೊಂಡಿರುತ್ತದೆ.” ಹೀಗೆಂದ ನರೇಂದ್ರ ಭಕ್ತಿಯಿಂದ ತುಂಬಿ ತುಳುಕಾಡುವ ಹಲವು ಹಾಡುಗಳನ್ನು ಹಾಡಿದ.

ನರೇಂದ್ರ ಬೇರೆ ಬೇರೆ ಮನೆಗಳಿಂದ ಬಂದ ತರುಣ ಸೋದರ ತಂಡವನ್ನು ಪ್ರೀತಿಯಿಂದ ಒಂದುಗೂಡಿಸಿದ. ಅವರ ಮುಂದೆ ಹಗಲು ರಾತ್ರಿ ಒಂದು ಆದರ್ಶವನ್ನು ಇಟ್ಟ. ಆತ್ಮನ ಮೋಕ್ಷ ಮತ್ತು ಜಗದ ಹಿತ ಎಂಬುದೇ ಅದು. ಅದಕ್ಕಾಗಿ ಎಲ್ಲರೂ ಬಾಳಬೇಕು, ಎಲ್ಲರೂ ದುಡಿಯಬೇಕು, ಎಲ್ಲರೂ ಮಡಿಯಬೇಕು ಎಂಬ ಭಾವನೆ ಅವರಲ್ಲಿ ರಕ್ತಗತವಾಗುವಂತೆ ಮಾಡಿದನು.

