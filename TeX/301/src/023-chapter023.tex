
\chapter{ವಿಶ್ವಧರ್ಮ ಸಮ್ಮೇಳನ}

೧೮೯೩ನೇ ಇಸವಿ ಸೆಪ್ಟೆಂಬರ್ ೧೧ನೇ ಸೋಮವಾರ ಬೆಳಿಗ್ಗೆ ಹತ್ತುಗಂಟೆಗೆ ವಿಶ್ವಧರ್ಮ ಸಮ್ಮೇಳನದ ಮಹಾಸಭೆ ಆಕ್ಟ್ ಪ್ಯಾಲೇಸ್ ಎಂಬ ಮಹಾಭವನದಲ್ಲಿ ಆರಂಭವಾಯಿತು. ಅಲ್ಲಿನ ಸಭೆಗೋಸುಗವಾಗಿ ಒಂದು ವಿಶಾಲವಾದ ಪ್ರಾಂಗಣ ಅಣಿಯಾಗಿತ್ತು. ಅಲ್ಲಿ ಹಲವು ಕಿರಿಯ ತಾತ್ಕಾಲಿಕ ಕೋಣೆಗಳು ಇದ್ದುವು. ಎಲ್ಲಾ ದೇಶದ ಜನಾಂಗದವರೂ ಅಲ್ಲಿದ್ದರು. ಭರತಖಂಡದಿಂದ ಬ್ರಹ್ಮಸಮಾಜದ ಮಜುಂದಾರರು, ಬೊಂಬಾಯಿಯಿಂದ\break ಬಂದ ನಗರ್‍ಕರ್, ಜೈನರ ಪ್ರತಿನಿಧಿಯಾದ ಗಾಂಧಿ, ಥಿಯಾಸಫಿ ಪ್ರಧಾನಿಯಾಗಿ ಶ‍್ರೀಮತಿ ಅನಿಬೆಸೆಂಟ್ ಮತ್ತು ಅವರೊಂದಿಗೆ ಬಂದ ಚಕ್ರವರ್ತಿ ಇದ್ದರು. ಇವರಲ್ಲಿ ಮಜುಂದಾರರು ಮತ್ತು ಸ್ವಾಮಿಗಳು ಹಳೆಯ ಸ್ನೇಹಿತರಾಗಿದ್ದರು. ಸದಸ್ಯರುಗಳನ್ನೆಲ್ಲ ದೊಡ್ಡ ಮೆರವಣಿಗೆಯಲ್ಲಿ ಕರೆದುಕೊಂಡು ಹೋಗಿ ಗದ್ದುಗೆಯ ಮೇಲೆ ಕೂಡಿಸಿದರು. ಕೆಳಗೆ ಒಂದು ದೊಡ್ಡ ಪ್ರಾಂಗಣ. ಮೇಲೆ ದೊಡ್ಡ ಗ್ಯಾಲರಿ. ಆರು ಏಳು ಸಾವಿರ ಮಂದಿ ಸುಶಿಕ್ಷಿತರಾದ ಗಂಡಸರು ಮತ್ತು ಹೆಂಗಸರು ಕಿಕ್ಕಿರಿದು ನೆರೆದಿರುವರು. ಗದ್ದುಗೆಯ ಮೇಲೆ ಜಗತ್ತಿನಲ್ಲಿರುವ ಎಲ್ಲಾ ಜನಾಂಗದ ವಿದ್ಯಾವಂತರೂ ಇದ್ದರು. ಸಂಗೀತ ಮತ್ತು ಪ್ರಾರ್ಥನೆಗಳಿಂದ ಮಹಾಸಭೆ ಪ್ರಾರಂಭವಾಯಿತು. ಅನಂತರ ಸದಸ್ಯರನ್ನು ಒಬ್ಬೊಬ್ಬರನ್ನಾಗಿ ಸಭೆಗೆ ಪರಿಚಯ ಮಾಡಿಕೊಟ್ಟರು. ಅವರು ಮುಂದೆ ಬಂದು ಭಾಷಣ ಮಾಡಿದರು. ಸ್ವಾಮೀಜಿ ಎದೆ ಕಂಪಿಸುತ್ತಿತ್ತು. ನಾಲಿಗೆ ಒಣಗಿತು. ಅವರಿಗೇನೋ ಅಂಜಿಕೆಯಾಯಿತು. ಬೆಳಿಗ್ಗೆ ಮಾತನಾಡಲು ಧೈರ್ಯ ಬರಲಿಲ್ಲ. ಮಜುಂದಾರರು ಒಂದು ಸುಂದರವಾದ ಭಾಷಣವನ್ನು ಮಾಡಿದರು. ಚಕವರ್ತಿಯವರದೂ ಸುಂದರವಾಗಿತ್ತು. ಪ್ರೇಕ್ಷಕರು ಅವರನ್ನು ಕೊಂಡಾಡಿದರು. ಅವರೆಲ್ಲರೂ ಸಿದ್ಧರಾಗಿ ಆಗಲೇ ಅಣಿಮಾಡಿಕೊಂಡ ಭಾಷಣದೊಂದಿಗೆ ಬಂದಿದ್ದರು. ಸ್ವಾಮೀಜಿ ಯಾವುದನ್ನೂ ಅಣಿಮಾಡಿಕೊಂಡು ಬಂದಿರಲಿಲ್ಲ. ಶಾರದೆಗೆ ನಮಿಸಿ ಪ್ರೇಕ್ಷಕರ ಮುಂದೆ ಬಂದರು. ಡಾಕ್ಟರ್ ಬರೋಸ್ ಅವರು ಸ್ವಾಮೀಜಿಯವರನ್ನು ಸಭೆಗೆ ಪರಿಚಯ ಮಾಡಿಕೊಟ್ಟರು. ಸ್ವಾಮೀಜಿ ಪುಟ್ಟ ಭಾಷಣ ಮಾಡುವುದಕ್ಕೆ ಮುಂಚೆ “ಅಮೇರಿಕದ ಸಹೋದರ ಸಹೊದರಿಯರೆ!” ಎಂದು ಸಭಿಕರನ್ನು ಸಂಬೋಧಿಸಿದರು. ಕಿವಿ ಕಿವುಡಾಗುವಂತೆ ಎರಡು ನಿಮಿಷಗಳ ಕರತಾಡನವಾಯಿತು. ಇದುವರೆಗೆ ಯಾವ ಉಪನ್ಯಾಸಕರೂ ಪ್ರೇಕ್ಷಕರನ್ನು ಸಹೋದರ ಸಹೋದರಿಯರೆ ಎಂದು ಕರೆದಿರಲಿಲ್ಲ. ಈ ಸ್ವಾಮಿಗಳು ಭೂಗೋಳದ ಅತ್ತಕಡೆಯಿಂದ ಬಂದವರು. ಅಮೇರಿಕಾದೇಶದ ಜನರ ಧರ್ಮಕ್ಕೆ ಕೂಡ ಸೇರಿದವರಲ್ಲ. ಆದರೂ ಅವರನ್ನೆಲ್ಲ ಸಹೋದರ ಸಹೋದರಿಯರೆ ಎಂದು ಕರೆಯಬೇಕಾದರೆ ಅವರ ಹೃದಯ ಎಷ್ಟು ವಿಶಾಲವಾಗಿರಬೇಕು, ಅಲ್ಲಿ ಎಷ್ಟು ಪ್ರೀತಿ ಇರಬೇಕು ಎನ್ನುವುದು ಅವರಿಗೆ ಅರಿವಾಯಿತು. ಆ ಎರಡು ಪದಗಳೇ ಸ್ವಾಮೀಜಿಯವರ ಹೃದಯದ ಅನುಭವವನ್ನು ಹೊತ್ತುಕೊಂಡು ಅನಂತರ ಶರಗಳಂತೆ ಕುಳಿತವರ ಹೃದಯವನ್ನೆಲ್ಲ ತೂರಿಕೊಂಡು ಪ್ರವೇಶಿಸಿದವು. ರೋಮ ರೋಲಾ ಅವರು ಅದನ್ನು Tongue of Flame, ಜ್ವಾಲೆಯ ನಾಲಿಗೆ ಅನ್ನುವರು. ಹಿಂದೂ ಧರ್ಮದ ಪರವಾಗಿ ಸ್ವಾಮೀಜಿ ಎಲ್ಲರನ್ನೂ ಅಭಿನಂದಿಸಿದರು; ಯಾವ ಭಾಷಣದಿಂದ ವಿವೇಕಾನಂದರು ಜಗದ್ವಿಖ್ಯಾತರಾದರೋ ಆ ಪುಟ್ಟ ಭಾಷಣವನ್ನು ಕೆಳಗೆ ಕೊಡುತ್ತೇವೆ. ಸ್ವಾಮೀಜಿಯವರು ಮಾತಿನ ಐಂದ್ರಜಾಲಿಕರು ಎಂಬುದನ್ನು ಇದು ತೋರುವುದು: “ನೀವು ನಮಗೆ ಪ್ರೀತಿಪೂರ್ವಕವಾಗಿ ಕೊಟ್ಟ ಸ್ವಾಗತಕ್ಕೆ ಉತ್ತರವನ್ನು ಕೇಳಬೇಕೆಂದು ಬಯಸಿದಾಗ, ನನ್ನ ಹೃದಯ ಒಂದು ಅವರ್ಣನೀಯ ಆನಂದದಿಂದ ತುಂಬಿ ತುಳುಕಾಡುತ್ತದೆ. ಜಗತ್ತಿನಲ್ಲಿ ಅತ್ಯಂತ ಪ್ರಾಚೀನತಮ ಸಂನ್ಯಾಸಿಗಳ ಪಂಥದ ಪರವಾಗಿ ನಾನು ನಿಮಗೆ ಕೃತಜ್ಞತೆಯನ್ನು ಅರ್ಪಿಸುತ್ತೇನೆ. ಎಲ್ಲಾ ಪಂಗಡ ಎಲ್ಲಾ ಮತಕ್ಕೆ ಸೇರಿದ ಕೋಟ್ಯಾನುಕೋಟಿ ಭಾರತೀಯರ ಪರವಾಗಿ ನನ್ನ ಕೃತಜ್ಞತೆಯನ್ನು ಅರ್ಪಿಸುತ್ತೇನೆ.

“ಇದೇ ವೇದಿಕೆಯ ಮೇಲೆ ನಿಂತು, ಪೌರ್ವಾತ್ಯ ದೇಶಗಳಿಂದ ಬಂದ ಪ್ರತಿನಿಧಿಗಳನ್ನು ಕುರಿತು ‘ಬಹಳ ದೂರದಿಂದ ಬಂದ ಈ ಮಹನೀಯರು ಸಹಾನುಭೂತಿಯ ಭಾವನೆಗಳನ್ನು ತಮ್ಮ ದೇಶಗಳಿಗೆ ಒಯ್ಯುವ ಗೌರವಕ್ಕೆ ಭಾಗಿಗಳಾಗಿರುವರು’ ಎಂದು ಹೇಳಿದವರಿಗೂ ನನ್ನ ಕೃತಜ್ಞತೆಗಳು. ಜಗತ್ತಿಗೆಲ್ಲ ಧಾರ್ಮಿಕ ಸಹಾನುಭೂತಿ ಮತ್ತು ಸಾರ್ವತ್ರಿಕ ಅಂಗೀಕಾರವನ್ನು ಬೋಧಿಸಿದ ಧರ್ಮಕ್ಕೆ ನಾನು ಸೇರಿರುವೆನು ಎಂಬುದು ಒಂದು ಹೆಮ್ಮೆ. ನಾವು ವಿಶ್ವ ಸಹಾನುಭೂತಿಯನ್ನು ಮಾತ್ರ ಒಪ್ಪಿಕೊಳ್ಳುತ್ತೇವೆ. ಎಲ್ಲಾ ಜನಾಂಗದಲ್ಲಿಯೂ ಎಲ್ಲಾ ಧಾರ್ಮಿಕ ಪಂಥಗಳಲ್ಲಿಯೂ ಕೋಟಲೆಗೆ ಸಿಕ್ಕಿ ತುತ್ತಾಗಿ ನಿರಾಶ‍್ರೀತರಾದವರಿಗೆ ಒಂದು ಆಶ್ರಯವನ್ನು ಕಲ್ಪಿಸಿಕೊಟ್ಟ ದೇಶಕ್ಕೆ ನಾನು ಸೇರಿದವನೆಂಬುದು ಒಂದು ಹೆಮ್ಮೆ. ಯಾವ ವರ್ಷ ರೋಮನ್ ಚಕ್ರಾಧಿಪತ್ಯದ ದೌರ್ಜನ್ಯದಿಂದ ಯಹೂದ್ಯರ ಪವಿತ್ರ ದೇವಾಲಯ ನುಚ್ಚುನೂರಾಗಿ ಹೋಯಿತೋ, ಅದೇ ವರ್ಷ ಅವರು ದಕ್ಷಿಣ ಭರತಖಂಡಕ್ಕೆ ಬಂದು ನಮ್ಮ ಆಶ್ರಯವನ್ನು ಪಡೆದರು. ಅವರನ್ನು ಕೂಡ ನಮ್ಮ ಹೃದಯದಲ್ಲಿ ಅಳವಡಿಸಿಕೊಂಡಿರುವೆವು ಎಂದು ಹೇಳುವುದಕ್ಕೆ ಒಂದು ಹೆಮ್ಮೆ. ಹಿಂದೆ ಪ್ರಖ್ಯಾತರಾದ ಜ಼ರತುಷ್ಟ, ಜನಾಂಗದ ಅವಶೇಷಕ್ಕೂ (ಪಾರ್ಸಿಮತ) ಆಶ್ರಯವಿತ್ತು, ಇಂದಿಗೂ ಅದನ್ನು ಪೋಷಿಸುತ್ತಿರುವ ಧರ್ಮಕ್ಕೆ ನಾನು ಸೇರಿದವನೆಂದು ಹೇಳಿಕೊಳ್ಳುವುದಕ್ಕೆ ಒಂದು ಹೆಮ್ಮೆ. ಸಹೋದರರೇ, ಬಾಲ್ಯಾರಭ್ಯ ನಾವು ಪಠಿಸುವ ಶ್ಲೋಕ ಒಂದನ್ನು ನಿಮಗೆ ಉದಾಹರಿಸುತ್ತೇನೆ:

\begin{verse}
ತ್ರಯೀ ಸಾಂಖ್ಯಂ ಯೋಗಃ ಪಶುಪತಿಮತಂ ವೈಷ್ಣವಮಿತಿ\\ಪ್ರಭಿನ್ನೇ ಪ್ರಸ್ಥಾನೇ ಪರಮಿದಮದಃ ಪಥ್ಯಮಿತಿ ಚ~।\\ರುಚೀನಾಂ ವೈಚಿತ್ರ್ಯಾದೃಜುಕುಟಿಲನಾನಾಪಥಜುಷಾಂ~।\\ನೃಣಾಮೇಕೋ ಗಮ್ಯಸ್ತ್ವಮಸಿ ಪಯಸಾಮರ್ಣವ ಇವ~।
\end{verse}

“ಬೇರೆ ಬೇರೆ ಸ್ಥಳಗಳಲ್ಲಿ ಹುಟ್ಟುವ ಹಲವು ನದಿಗಳು ಕೊನೆಗೆ ಎಲ್ಲಾ ಸಾಗರಕ್ಕೆ ಸೇರುವಂತೆ, ಹೇ ಭಗವಂತನೇ, ಮಾನವರು ಅವರವರ ಸಂಸ್ಕಾರಗಳಿಗೆ ತಕ್ಕಂತೆ ಅನುಸರಿಸುವ, ನೇರವಾಗಿಯೋ ವಕ್ರವಾಗಿಯೋ ಇರುವ ಬೇರೆ ಬೇರೆ ದಾರಿಗಳೆಲ್ಲ ನಿನ್ನೆಡೆಗೆ ಕರೆದೊಯ್ಯುವುವು.

 “ಜಗತ್ತಿನ ಇತಿಹಾಸದಲ್ಲಿ ಮಹಾದ್ಭುತ ಸಭೆಗಳಲ್ಲಿ ಇದು ಒಂದು. ಗೀತೆಯಲ್ಲಿ ಹೇಳಿರುವ:

\begin{verse}
ಯೇ ಯಥಾ ಮಾಂ ಪ್ರಪದ್ಯನ್ತೇ ತಾಂಸ್ತಥೈವ ಭಜಾಮ್ಯಹಂ~।\\ಮಮ ವರ್ತ್ಮಾನು ವರ್ತಂತೇ ಮನುಷ್ಯಾಃ ಪಾರ್ಥ ಸರ್ವಶಃ~॥
\end{verse}

“ಯಾರಾದರೂ ಆಗಲಿ, ಯಾವ ಆಕಾರವನ್ನಾದರೂ ಆಗಲಿ ಪೂಜಿಸಿ ನಮ್ಮಲ್ಲಿಗೆ ಬಂದರೆ ನಾನು ಅವರನ್ನು ಹಾಗೆಯೇ ಅನುಗ್ರಹಿಸುವೆನು; ಕೊನೆಗೆ ನನ್ನನ್ನೇ ಮುಟ್ಟುವ ಹಲವು ದಾರಿಗಳಲ್ಲಿ ಮಾನವರೆಲ್ಲ ಪ್ರಯತ್ನಪಡುತ್ತಿರುವರು – ಎಂಬ ಸಂದೇಶವನ್ನು ಜಗತ್ತಿಗೆ ಸಾರುವುದಕ್ಕೆ ಈ ಸಭೆಯೊಂದೇ ಸಾಕು. ಸ್ವಮತಾಭಿಮಾನ, ಅನ್ಯಮತದ್ವೇಷ ಮತ್ತು ಇವುಗಳಿಂದ ಉತ್ಪನ್ನವಾದ ಘೋರಧರ್ಮ, ದುರಭಿಮಾನ, ಈ ಸುಂದರ ಜಗತ್ತನ್ನು ಬಹುಕಾಲದಿಂದಲೂ ಆವರಿಸಿಕೊಂಡಿರುವುವು. ಇವು ಜಗತ್ತನ್ನೆಲ್ಲ ಹಿಂಸೆಯಿಂದ ತುಂಬಿರುವುವು. ಹಲವು ವೇಳೆ ನರರಕ್ತದಿಂದ ತೋಯಿಸಿರುವುವು, ಸಂಸ್ಕೃತಿಗಳನ್ನು ನಾಶಮಾಡಿರುವುವು, ಹಲವು ದೇಶಗಳನ್ನು ನಿರಾಶೆಯ ಕೂಪಕ್ಕೆ ತಳ್ಳಿರುವುವು. ಇಂತಹ ಉಗ್ರದೈತ್ಯನಿಲ್ಲದೇ ಇದ್ದಿದ್ದರೆ, ಮಾನವ ಜನಾಂಗ ಇಂದಿಗಿಂತಲೂ ಎಷ್ಟೋ ಮುಂದುವರಿದು ಹೋಗಬಹುದಾಗಿತ್ತು. ಆದರೆ ಅವುಗಳ ಅಂತ್ಯಕಾಲ ಸಮೀಪಿಸಿದೆ. ಈ ಸಮಾರಂಭವನ್ನು ಸೂಚಿಸಿದ ಘಂಟಾನಾದವು, ಖಡ್ಗದ ಮೂಲಕವಾಗಿ ಆಗಲಿ, ಲೇಖನಿಯ ಮೂಲಕವಾಗಿ ಆಗಲಿ, ಎಲ್ಲಾ ಮತಾಂಧತೆಯ, ಅನ್ಯಮತ ಹಿಂಸೆಯ, ಮತ್ತು ಒಂದೇ ಗುರಿ ಎಡೆಗೆ ನಡೆಯುತ್ತಿರುವ ದಾರಿಗಳಲ್ಲಿ ಆದ ಮನಸ್ತಾಪಗಳೆಂಬ ಭೂತಗಳ ಅಂತ್ಯಕ್ರಿಯೆಯನ್ನು ಜಗತ್ತಿಗೆ ಸೂಚಿಸುವ ಧ್ವನಿಯಾಗಲೆಂಬುದೆ ನನ್ನ ಆಶಯ.”

ಪ್ರೇಕ್ಷಕರಿಗೆ ಸ್ವಾಮೀಜಿ ಮಾತನಾಡುವುದಕ್ಕೆ ಮುಂಚೆ, ಆಯಾ ಧರ್ಮಬೋಧಕರು, ತಮ್ಮ ತಮ್ಮ ಮತಗಳೇ ಹೆಚ್ಚು ಎಂದು ಹೇಳುವ ಪಲ್ಲವಿಯನ್ನು ಕೇಳಿ ಸಾಕಾಗಿ ಹೋಗಿತ್ತು. ಸ್ವಾಮೀಜಿಯವರ ಉದಾರವಾಣಿಯನ್ನು ಕೇಳಿದಾಗ ಕತ್ತಲ ಕೋಣೆಗೆ ಬೆಳಕು ಬಂದಂತೆ ಆಯಿತು. ಅಂದು ಸ್ವಾಮೀಜಿ ಹದಿನೈದು ನಿಮಿಷಗಳು ಮಾತ್ರ ಮಾತನಾಡಿದರು. ಅದರಿಂದ ಜಗದ್ವಿಖ್ಯಾತರಾದರು. ಅವರು ಉಪನ್ಯಾಸ ಮಾಡುವುದಕ್ಕೆ ಮುಂಚೆ ಅನಾಮಧೇಯರಾಗಿದ್ದರು. ಉಪನ್ಯಾಸ ಮಾಡಿ ಆದಮೇಲೆ ವೃತ್ತಪತ್ರಿಕೆಗಳಲ್ಲೆಲ್ಲ ಅವರ ಭಾವಚಿತ್ರವೆ, ಎಲ್ಲರ ಬಾಯಿಯಲ್ಲಿಯೂ ಅವರ ಹೆಸರು. ಉಪನ್ಯಾಸವಾದಮೇಲೆ ಸಹಸ್ರಾರು ಮಂದಿ ನಾರಿಯರು ಸ್ವಾಮೀಜಿಯವರ ನಿಲುವಂಗಿ ತುದಿಯನ್ನು ಮುಟ್ಟುವುದಕ್ಕೆ ಧಾವಿಸುತ್ತಿದ್ದುದನ್ನು ನೋಡಿದ ಒಬ್ಬ ಹಿರಿಯ ಹೆಂಗಸು ಹೀಗೆ ಹೇಳುತ್ತಾಳೆ: “ಮಗು, ನೀನು ಇದನ್ನು ಜಯಿಸಿದರೆ ಸಾಕ್ಷಾತ್ ದೇವರ ಮಗನೆ” ಎಂದು. ಕೋಟ್ಯಾಧೀಶರು ಸ್ವಾಮೀಜಿಯವರನ್ನು ತಮ್ಮ ಮನೆಗೆ ಕರೆಯಲು ಸ್ಪರ್ಧಿಸುತ್ತಿದ್ದರು. ಅರಮನೆಯಂತಹ ಮನೆಗಳು ಮೃಷ್ಟಾನ್ನ ಭೋಜನ ಹಂಸತೂಲಿಕಾತಲ್ಪಗಳು ಇವರಿಗೆ ಕಾದಿದ್ದವು. ಆದರೆ ಸ್ವಾಮೀಜಿಯವರ ತಲೆ ಇದರಿಂದ ತಿರುಗಲಿಲ್ಲ. ಜನ ಇವರನ್ನು ನಿಕೃಷ್ಟ ದೃಷ್ಟಿಯಿಂದ ನೋಡಿದಾಗ ಹೇಗೆ ನಿರಾಶರಾಗಲಿಲ್ಲವೋ ಹಾಗೆಯೇ ಅವರನ್ನು ಕೊಂಡಾಡುವಾಗಲೂ ಅದರ ಅಮಲಿನಲ್ಲಿ ತಮ್ಮನ್ನು ಮರೆಯಲಿಲ್ಲ. ಸ್ತುತಿನಿಂದೆಗಳಲ್ಲಿ ಸಮನಾಗಿದ್ದರು. 

 ಅವರನ್ನು ಕಣ್ಣಾರೆ ಕಂಡ ಅನಿಬೆಸಂಟರು ಹೀಗೆ ಬರೆಯುತ್ತಾರೆ (ಕುವೆಂಪು ಅನುವಾದ): 

 “ಗೈರಿಕವಸನ ಭೂಷಿತ ಮಹಿಮಾಮಯ ಮೂರ್ತಿ. ಚಿಕಾಗೋ ನಗರದ ಧೂಮ ಮಲಿನ ಧೂಸರ ವಕ್ಷಸ್ಥಳದಲ್ಲಿ ಭಾರತೀಯ ಸೂರ್ಯನು ಮೂಡಿದಂತಿತ್ತು. ಉನ್ನತ ಶಿರ, ಮರ್ಮಭೇದಿಯಾದ ದೃಷ್ಟಿ, ಪೂರ್ಣ ವಿಶಾಲ ನೇತ್ರದ್ವಯ, ಚಟುಲ ಚಂಚಲವಾದ ಓಷ್ಠಾಧರ, ಮನೋಹರವಾದ ಕಾಯಭಂಗಿ, ಇವುಗಳಿಂದ ಕೂಡಿದ ಸ್ವಾಮಿ\break ವಿವೇಕಾನಂದರು ಸರ್ವಮತ ಸಮ್ಮೇಳನದ ಪ್ರತಿನಿಧಿಗಳಿಗೆ ಏರ್ಪಡಿಸಿದ್ದ ವೇದಿಕೆಯ ಮೇಲೆ ಮೊತ್ತಮೊದಲು ನನ್ನ ದೃಷ್ಟಿಗೆ ಬಿದ್ದರು. ಅವರು ಸಂನ್ಯಾಸಿಗಳೆಂಬ ಖ್ಯಾತಿಯನ್ನು ಹಿಂದೆಯೇ ಕೇಳಿದ್ದೆ. ಆದರೆ ನೋಡಿದ ಕೂಡಲೆ ಅವರನ್ನು ಸಂನ್ಯಾಸಿ ಎನ್ನುವುದಕ್ಕಿಂತ ಯೋಧ ಎನ್ನುವುದು ಯೋಗ್ಯ ಎಂದುಕೊಂಡೆ. ಅಂತೂ ಕಡೆಗೆ ಯೋಧ ಸಂನ್ಯಾಸಿಯಾದರು! ಅವರ ವ್ಯಕ್ತಿತ್ವದಲ್ಲಿ ಸ್ವಜನ ಸ್ವದೇಶಾಭಿಮಾನಗಳು ತುಂಬಿ ತುಳುಕಾಡುತ್ತಿದ್ದವು. ಮತಗಳಲ್ಲಿ ಪ್ರಾಚೀನತಮವಾದ ಮತದ ಪ್ರತಿನಿಧಿ ವಯಸ್ಸಿನಲ್ಲಿ ಎಲ್ಲ ಪ್ರತಿನಿಧಿಗಳಿಗಿಂತಲು ಕನಿಷ್ಠತಮನಾಗಿದ್ದರೂ ಪ್ರತಿಭೆಯಲ್ಲಿ ಯಾರಿಗೇನು ಕಡಿಮೆಯಾಗಿರಲಿಲ್ಲ. ಗರ್ವಿಷ್ಠ ಪಾಶ್ಚಾತ್ಯ ದೇಶಗಳಿಗೆ ಭಾರತಮಾತೆ ತನ್ನ ಯೋಗ್ಯತಮ ಪುತ್ರನನ್ನೇ ದೂತನನ್ನಾಗಿ ಕಳುಹಿ ಗೌರವಾನ್ವಿತೆಯಾದಳು. ದೂತನು ತನ್ನ ಪುಣ್ಯ ಜನ್ಮಭೂಮಿಯ ಸನಾತನ ಕೀರ್ತಿ ಗೌರವಗಳನ್ನು ಮರೆಯದೆ ಆಕೆಯ ಸಂದೇಶವನ್ನು ಸಾರಿದನು. ಆಶಿಷ್ಠನೂ ದ್ರಢಿಷ್ಠನೂ ಬಲಿಷ್ಠನೂ ಮೇಧಾವಿಯೂ ಆದ ಆತನು ಎಲ್ಲರನ್ನೂ ಮೀರಿ ಗಂಡರ ಗಂಡನಾಗಿ ಮುಂದೆ ನಿಂತು ತನ್ನ ಕೆಲಸವನ್ನು ಸಾಧಿಸಿದನು.

“ವೇದಿಕೆಯ ಮೇಲೆ ಮತ್ತೊಂದು ದೃಶ್ಯ. ಸ್ವಾಮೀಜಿ ಮಾತನಾಡಲು ಎದ್ದು ನಿಂತರು. ಅದೇ ಗಾಂಭೀರ್ಯ, ಅದೇ ಶಕ್ತಿ ಎಲ್ಲವೂ ಇದ್ದವು. ಆದರೆ ಆತನು ತಂದ ಆ ಧರ್ಮಸಂದೇಶದ ಸಮ್ಮುಖದಲ್ಲಿ, ಆ ಅಪ್ರತಿದ್ವಂದ್ವಿಯಾದ ಅತುಲನೀಯವಾದ ಪ್ರಾಚ್ಯ ಋಷಿಗಳ ಆಧ್ಯಾತ್ಮಿಕ ಸಂದೇಶದ ಮಹಿಮೆಯ ಸಮ್ಮುಖದಲ್ಲಿ ಅವುಗಳೆಲ್ಲ ವಿನಮ್ರವಾಗಿದ್ದವು. ಅಲ್ಲಿ ನೆರೆದಿದ್ದ ಜನವಾರಿಧಿ ಮಂತ್ರಮುಗ್ಧವಾದಂತೆ ಆತನ ಕೊರಳ ಧ್ವನಿಯನ್ನು ಕೇಳಲು ನೀರವವಾಗಿತ್ತು, ಒಂದು ಮಾತನ್ನೂ ಹಾಳಾಗಗೊಡಲಿಲ್ಲ; ಒಂದು ಸ್ವರವೂ ವ್ಯರ್ಥವಾಗಲಿಲ್ಲ.” 

 ಅಲ್ಲಿಯ ಕೆಲವು ಪತ್ರಿಕೆಗಳು ಸ್ವಾಮೀಜಿ ವಿಷಯವಾಗಿ ಉಲ್ಲೇಖಿಸಿರುವುದನ್ನು ಕೆಳಗೆ ಕೊಡುವೆವು: 

“ಸುಂದರವಾದ ಚಿತ್ತಾಕರ್ಷಕ ವ್ಯಕ್ತಿ, ಅಮೋಘವಾದ ವಾಕ್ ಚಾತುರ‍್ಯವುಳ್ಳ ಈ ವ್ಯಕ್ತಿಯೇ ಮಹಾಸಭೆಯಲ್ಲಿ ಅಗ್ರಗಣ್ಯ.” 

 “ಸರ್ವಮತ ಸಮ್ಮೇಳನದಲ್ಲಿ ವಿವೇಕಾನಂದರೇ ಶ್ರೇಷ್ಠ ವ್ಯಕ್ತಿ. ಅವರ ಭಾಷಣವನ್ನು ಕೇಳಿದಮೇಲೆ ಇಂತಹ ಸುಸಂಸ್ಕೃತ ಜನಾಂಗಕ್ಕೆ ನಾವು ಧರ್ಮ ಪ್ರಚಾರಕರನ್ನು ಕಳುಹಿಸುವುದು ಶುದ್ಧ ಮೂರ್ಖತನವೆಂದು ಭಾವಿಸುತ್ತೇವೆ.” 

 ಮತ್ತೊಂದು ಪತ್ರಿಕೆ ಹೀಗೆ ಬರೆಯಿತು: 

 “ವಿವೇಕಾನಂದರು ಧರ್ಮ ಮಹಾಸಭೆಯ ನೆಚ್ಚಿನ ಕಣ್ಣು. ಅವರ ಭದ್ರಾಕಾರವೂ, ಅವರಾಡುವ ತತ್ತ್ವಗಳ ವೈಭವವೂ ಸರ್ವರನ್ನೂ ಸೂರೆಗೊಂಡಿವೆ. ಅವರು ವೇದಿಕೆಯ ಮೇಲೆ ಒಂದು ಕಡೆಯಿಂದ ಇನ್ನೊಂದು ಕಡೆಗೆ ಹೋದರೆ ಸಾಕು ಕೈಚಪ್ಪಾಳೆಗಳಿಂದ ಮಂದಿರವು ಕಂಪಿಸುವುದು. ಆದರೆ ಅವರು ಮಾತ್ರ ಅಷ್ಟೊಂದು ಜನರ ಹೊಗಳಿಕೆಗಳನ್ನು ಬಾಲಕರಂತೆ ಸ್ವೀಕರಿಸುತ್ತಾರೆ. ಅಹಂಕಾರದ ಸುಳಿವೂ ಕೂಡ ಅವರಲ್ಲಿ ಇಲ್ಲ. ಸಭೆ ಮುಕ್ತಾಯವಾಗುವವರೆಗೆ ಜನರನ್ನು ಹಿಡಿದು ನಿಲ್ಲಿಸುವುದಕ್ಕೆ! ಧರ್ಮಸಭೆಯಲ್ಲಿ ವಿವೇಕಾನಂದರನ್ನು ಕಟ್ಟಕಡೆಯ ಉಪನ್ಯಾಸಕರನ್ನಾಗಿ ಮಾಡುತ್ತಾರೆ. ಬಿಸಿಲ ಬೇಗೆ ಹೆಚ್ಚಾದ ದಿನಗಳಲ್ಲಿ ಯಾರಾದರೂ ನೀರಸವಾಗಿ ಬಹಳ ಹೊತ್ತು ಮಾತನಾಡಿದರೆ ಜನರು ತಂಡ ತಂಡವಾಗಿ ಮನೆಗೆ ಹೋಗಲು ಉಪಕ್ರಮಿಸುತ್ತಾರೆ. ಅದನ್ನು ತಿಳಿದ ಕೂಡಲೆ ಅಧ್ಯಕ್ಷರು ಎದ್ದು ನಿಂತು ವಿವೇಕಾನಂದರು ಕಡೆಯಲ್ಲಿ ಒಂದು ಭಾಷಣವನ್ನು ಮಾಡುತ್ತಾರೆ ಎಂದು ಹೇಳುವರು. ಕೂಡಲೆ ಜನರು ಶಾಂತರಾಗಿ ಕುಳಿತುಕೊಳ್ಳುವರು. ಸಾವಿರಾರು ಜನರು ಬೀಸಣಿಗೆಯಿಂದ ಬೀಸಿಕೊಳ್ಳುತ್ತ ವಿವೇಕಾನಂದರ ಹದಿನೈದೇ ನಿಮಿಷದ ಭಾಷಣವನ್ನು ಕೇಳಲೆಂದು ಎರಡು ಮೂರು ಗಂಟೆಗಳ ಪರಿಯಂತ ಇತರರ ನೀರಸ ಭಾಷಣಗಳನ್ನು ಕೇಳುತ್ತ ತಾಳ್ಮೆಯಿಂದ ನಿರೀಕ್ಷಿಸುತ್ತಾರೆ. ಎಲ್ಲ ಮುಗಿದ ಮೇಲೆ ಬೆಲ್ಲ ಇರಬೇಕೆಂಬ ನೀತಿ ಅಧ್ಯಕ್ಷರಿಗೆ ಚೆನ್ನಾಗಿ ಗೊತ್ತು.” 

 ಸ್ವಾಮೀಜಿ ಅವರು ಹದಿನೈದನೇ ತಾರೀಖು ‘ನಮ್ಮಲ್ಲಿ ಏಕೆ ಒಮ್ಮತವಿಲ್ಲ’ ಎಂಬುದರ ಮೇಲೆ ಒಂದು ಸ್ವಾರಸ್ಯವಾಗಿ ಉಪನ್ಯಾಸ ಮಾಡಿದರು. ಪ್ರತಿಯೊಂದು ಧರ್ಮದವರು ತಮ್ಮ ಧರ್ಮವೆಂಬ ಬಾವಿಯಲ್ಲಿರುವ ಕಪ್ಪೆಗಳಂತೆ ಹೊರಗೆ ಬಂದು ಇತರ ಕಪ್ಪೆಗಳೊಡನೆ ಹೋಲಿಸಿಕೊಳ್ಳುವುದಿಲ್ಲ. ಎಲ್ಲಾ ಬಾವಿಗೂ ನೀರನ್ನು ಒದಗಿಸುವ ಮಳೆ, ಆ ಮಳೆಗೆ ನೀರನ್ನು ಒದಗಿಸುವ ಮಹಾಸಾಗರ ಇವುಗಳ ಕಡೆ ಗಮನವೇ ಕೊಡದೇ\break ಇರುವುದರಿಂದ ಈ ಮತಾಂಧತೆ ಇರುವುದು. ಇದನ್ನು ವಿವರಿಸಲು ಒಂದು ಸುಂದರವಾದ ಕಥೆಯನ್ನು ಹೇಳಿದರು: 

 ಒಂದು ಕಪ್ಪೆ ಬಾವಿಯಲ್ಲಿ ವಾಸವಾಗಿತ್ತು. ಅಲ್ಲಿ ಅದು ಬಹು ಕಾಲದಿಂದಲೂ ಇತ್ತು. ಅದು ಅಲ್ಲೇ ಹುಟ್ಟಿ ಅಲ್ಲೇ ಬೆಳೆಯಿತು. ಅದಿನ್ನೂ ಸಣ್ಣ ಕಪ್ಪೆಯಾಗಿತ್ತು..... ಒಂದು ದಿನ ಸಮುದ್ರದಿಂದ ಒಂದು ಕಪ್ಪೆ ಅಲ್ಲಿಗೆ ಬಂದು ಬಿತ್ತು. 

 ಬಾವಿಯ ಕಪ್ಪೆ: “ನೀನು ಎಲ್ಲಿಂದ ಬಂದೆ? 

 ಸಮುದ್ರದ ಕಪ್ಪೆ: “ನಾನು ಸಮುದ್ರದಿಂದ ಬಂದೆ.” 

 ಬಾವಿಯ ಕಪ್ಪೆ: “ಸಮುದ್ರವೇ! ಅದೆಷ್ಟು ದೊಡ್ಡದು? ಅದು ನನ್ನ ಬಾವಿಯಷ್ಟು ದೊಡ್ಡದೆ? ಎಂದು ಹೇಳಿ ಬಾವಿಯ ಒಂದು ಕಡೆಯಿಂದ ಮತ್ತೊಂದು ಕಡೆಗೆ ನೆಗೆಯಿತು.”

 ಸಮುದ್ರದ ಕಪ್ಪೆ: “ಅಯ್ಯಾ ನನ್ನ ಸ್ನೇಹಿತನೆ, ಸಮುದ್ರವನ್ನು ಬಾವಿಗೆ ಹೇಗೆ\break ಹೋಲಿಸಬಲ್ಲೆ?” 

 ಬಾವಿಯ ಕಪ್ಪೆ ಇನ್ನೊಂದು ಸಲ ನೆಗೆದು, “ನಿನ್ನ ಸಮುದ್ರ ಇಷ್ಟು ದೊಡ್ಡದೋ” ಎಂದು ಕೇಳಿತು. 

 ಸಮುದ್ರದ ಕಪ್ಪೆ: “ಎಂತಹ ಹುಚ್ಚುತನ ನಿನ್ನದು! ಸಮುದ್ರವನ್ನು ನಿನ್ನ ಬಾವಿಗೆ ಹೋಲಿಸುವೆಯಲ್ಲ!” 

 ಬಾವಿಯ ಕಪ್ಪೆ: “ಅದು ಹೇಗಾದರೂ ಇರಲಿ. ನನ್ನ ಬಾವಿಗಿಂತ ದೊಡ್ಡದು ಯಾವುದೂ ಇಲ್ಲ. ಇದಕ್ಕಿಂತ ದೊಡ್ಡದು ಯಾವುದೂ ಇರಲಾರದು. ಇವನೊಬ್ಬ ಸುಳ್ಳುಗಾರ. ಇವನನ್ನು ಹೊರಗೆ ನೂಕಿ.” ಎಂದಿತು. 

 “ನಮ್ಮ ಈ ಸಣ್ಣ ಪ್ರಪಂಚದ ಮೇರೆಯನ್ನು ಒಡೆದುಹಾಕುವುದಕ್ಕೆ ಪ್ರಯತ್ನಿಸುತ್ತಿರುವ ಅಮೇರಿಕಾ ದೇಶೀಯರೆ ನಿಮಗೆ ನನ್ನ ಕೃತಜ್ಞತೆಗಳು. ಈ ಗುರಿಯನ್ನು ಮುಂದೆ ಸಾಧಿಸುವುದಕ್ಕೆ ಭಗವಂತನು ಸಹಾಯ ಮಾಡುವನೆಂದು ಆಶಿಸುತ್ತೇನೆ” ಎಂದರು ಸ್ವಾಮಿಗಳು. 

 ಸ್ವಾಮೀಜಿಯವರು ಹತ್ತೊಂಭತ್ತನೇ ತಾರೀಖು ಹಿಂದೂಧರ್ಮದ ಮೇಲೆ ಒಂದು ಲೇಖನವನ್ನು ಓದಿದರು. ಆ ಲೇಖನವನ್ನು ಕೇಳುವುದಕ್ಕೆ ಸಹಸ್ರಾರು ಜನ ಕಾದು ಕುಳಿತಿದ್ದರು. ಹಿಂದೂಧರ್ಮದ ಭಾವನೆಯನ್ನೆಲ್ಲ ಒಂದು ಸಣ್ಣ ಲೇಖನದಲ್ಲಿ ಚಿತ್ತಾಕರ್ಷಕವಾದ ರೀತಿ ಇಟ್ಟಿದ್ದು ಸ್ವಾಮೀಜಿಯವರೊಬ್ಬರೆ. ಇದುವರೆಗೆ ಯಾರೂ ಇಂತಹ ಸಾಹಸವನ್ನು ಮಾಡಿರಲಿಲ್ಲ. ಕನ್ನಡಿಯೊಳಗೆ ಸಮುದ್ರವನ್ನು ತೋರುವಂತಿದೆ ಆ ಲೇಖನ. ಅದರ ಸಾರಾಂಶವನ್ನು ಕೆಳಗೆ ಕೊಟ್ಟಿದೆ: 

 ಹಿಂದೂ ಧರ್ಮ ಅತ್ಯಂತ ಪುರಾತನವಾದುದು, ಕ್ರಿಸ್ತ, ಬೌದ್ಧ, ಮಹಮ್ಮದೀಯ, ಜ಼ರಾತೂಷ್ಟ್ರ ಧರ್ಮಗಳೆಲ್ಲ ಮಕ್ಕಳಂತೆ ಆ ಧರ್ಮದ ಎದುರಿಗೆ. ಅದನ್ನು ನಾಶಮಾಡಲು ಬಂದ ಎಲ್ಲರನ್ನು ಎದುರಿಸಿ ನಿಂತಿದೆ. ಅವರಲ್ಲಿರುವ ಒಳ್ಳೆಯದನ್ನು ತೆಗೆದುಕೊಂಡು ಅದನ್ನು ಜೀರ್ಣಿಸಿಕೊಂಡು ನಿಂತಿದೆ. 

 ಹಿಂದೂಗಳ ಶಾಸ್ತ್ರವಾದರೋ ವೇದ, ಇದು ಅಪೌರುಷೇಯ. ಇದಕ್ಕೆ ಆದಿ ಅಂತ್ಯಗಳಿಲ್ಲ. ಇಲ್ಲಿ ವೇದ ಒಂದು ಪುಸ್ತಕವಲ್ಲ. ಇದು ಆಧ್ಯಾತ್ಮಿಕ ನಿಯಮಗಳ ಒಂದು ಸಂಕಲನ. ಈ ನಿಯಮಗಳು ಸದಾ ಜಾಗ್ರತವಾಗಿವೆ. ಅವು ಮನುಷ್ಯ ಕಂಡುಹಿಡಿದ ಮೇಲೆ ಜಾರಿಗೆ ಬರಲಿಲ್ಲ. ಅವನು ಹುಟ್ಟುವುದಕ್ಕೆ ಮುಂಚೆಯೇ ಇತ್ತು. ಅವನು ಸತ್ತಾದ ಮೇಲೆಯೂ ಇರುವುದು. ಗುರುತ್ವಾಕರ್ಷಣ ನಿಯಮ ಹೇಗೆ ಅದನ್ನು ಕಂಡುಹಿಡಿಯುವ ಮೊದಲು ಇದ್ದಿತೊ, ಮುಂದೆ ಮಾನವ ಮರೆತರೂ ಅದು ಇದ್ದೇ ಇರುತ್ತದೆಯೋ ಹಾಗೆಯೆ ಆಧ್ಯಾತ್ಮಿಕ ಜಗತ್ತನ್ನು ಆಳುತ್ತಿರುವ ನಿಯಮಗಳೂ ಕೂಡ. 

 ಇದರಂತೆಯೇ ಹಿಂದೂವು ಜಗತ್ತಿಗೆ ಆದಿ ಅಂತ್ಯವಿಲ್ಲ ಎನ್ನುತ್ತಾನೆ. ಅವನು ಉಪಯೋಗಿಸುವ ಸೃಷ್ಟಿ, ಸ್ಥಿತಿ, ಪ್ರಳಯಗಳೆಲ್ಲ ಒಂದು ಅಲೆಯ ಗತಿಯನ್ನು ವಿವರಿಸುವುದು. ಅಲೆ ಏಳುವುದಕ್ಕೆ ಮುಂಚೆ ಸೂಕ್ಷ್ಮವಾಗಿತ್ತು. ಹಾಗೆಯೆ ಬಿದ್ದ ಮೇಲೆ ಸೂಕ್ಷ್ಮಾವಸ್ಥೆಗೆ ಹೋಗುತ್ತದೆ. ಸ್ಥೂಲದ ಹಿಂದೆ ಸೂಕ್ಷ್ಮ ಇದ್ದರೆ ಸೂಕ್ಷ್ಮದ ಹಿಂದೆ ಸ್ಥೂಲ ಇದೆ. ಇದೊಂದು ಸರಪಳಿ. ಇದು ಯಾವಾಗಲೂ ಇರುವುದು. ಅವಸ್ಥಾಭೇದವನ್ನು ಮಾತ್ರ ಹಿಂದೂಧರ್ಮ ಒಪ್ಪುವುದು. 

 ಅನಂತರವೇ ಜೀವ. ಈ ಜೀವಗಳು ಕೂಡ ಅನಾದಿ. ಯಾವುದೋ ಒಂದು ಕಾಲದಲ್ಲಿ ಈ ಜೀವಿಗಳೆಲ್ಲ ಸೃಷ್ಟಿಸಲ್ಪಟ್ಟುವು ಎಂಬುದು ಸರಿಯಲ್ಲ. ನ್ಯಾಯವಾದ ದಯಾಳುವಾದ ದೇವರು ಕೆಲವರನ್ನು ಸುಖದಲ್ಲಿ ಮತ್ತೆ ಕೆಲವರನ್ನು ದುಃಖದಲ್ಲಿ ಏತಕ್ಕೆ ಹುಟ್ಟಿಸಬೇಕು! ಅಥವಾ ಯಾರು ಈ ಜನ್ಮದಲ್ಲಿ ದುಃಖಪಡುವರೊ ಅವರು ಮುಂದೆ ಸುಖವನ್ನು ಅನುಭವಿಸುವರು ಎಂದರೆ ಸಮಸ್ಯೆಯು ಸ್ವಲ್ಪವೂ ಪರಿಹಾರವಾಗುವುದಿಲ್ಲ. ಪರಮ ದಯಾಳುವಾದ ಧಾರ್ಮಿಕ ದೇವರ ಆಳ್ವಿಕೆಯಲ್ಲಿರುವಾಗ ಇಲ್ಲಿಯಾದರೂ ತಾನೆ ಕೆಲವರು ಏತಕ್ಕೆ ಸಂಕಟದಲ್ಲಿ ನರಳಬೇಕು? ಇದು ದೇವರನ್ನು ಒಬ್ಬ ಪಕ್ಷಪಾತಿಯನ್ನಾಗಿ ಮಾಡುವುದು. ಅವನು ಕ್ರೂರಿ, ಮತ್ತು ನಿರ್ದಯನಾಗುತ್ತಾನೆ. ಹಿಂದೂಗಳಾದರೋ ಜೀವ ಅನಾದಿ ಎನ್ನುವರು. ಎಷ್ಟು ಹಿಂದೆ ಹೊದರೂ ಅದಕ್ಕೆ ಮತ್ತೊಂದು ಹಿಂದೆ ಇದೆ. ಕಾಲದ ಬೇಲಿಯೊಳಗೆ ಆದಿಯನ್ನು ಕಂಡುಹಿಡಿಯುವುದು ಕತ್ತಲಲ್ಲಿ ಕಪ್ಪು ಬೆಕ್ಕನ್ನು ಅದು ಅಲ್ಲಿರದಿರುವಾಗ ಹುಡುಕಿದಂತೆ! ಆದಕಾರಣವೇ ಹಿಂದುಗಳು ದೇವರನ್ನು ದೂರದೆ ತಮ್ಮ ಸ್ಥಿತಿಯ ವ್ಯತ್ಯಾಸಕ್ಕೆ ತಕ್ಕ ಕರ್ಮ ಸಿದ್ಧಾಂತವನ್ನು ತರುವರು. ಹಿಂದೂ ತಾನು ಹಿಂದೆ ಇದ್ದೆ ಮತ್ತು ಮುಕ್ತಿ ಸಿಕ್ಕುವವರೆಗೆ ಮಂದೆಯು ಬರುತ್ತಿರುವೆನು ಎಂದು ನಂಬುತ್ತಾನೆ. ನಿಮ್ಮ ಹಿಂದಿನದು ನಿಮಗೆ ಗೊತ್ತಿದ್ದರೆ ಹೇಳಿ ಎಂದು ಇತರರು ಹಿಂದೂಗಳನ್ನು ಕೇಳಬಹುದು. ಆದರೆ ಅದು ಗೊತ್ತಿಲ್ಲ ಎಂದರೆ ಇಲ್ಲ. ಆದರೆ ನೀವು ಅದನ್ನು ಮಾಡಲೇ ಇಲ್ಲವೆ! ಪ್ರಯತ್ನಮಾಡಿದರೆ ತನ್ನ ಹಿಂದಿನದನ್ನು ತಿಳಿದುಕೊಳ್ಳಲು ಸಾಧ್ಯ ಎಂಬುದನ್ನು ಹಿಂದೂ ಧರ್ಮ ಹೇಳುತ್ತದೆ. 

 ಜೀವ ದೇಹವಲ್ಲ; ಅದನ್ನು ಉಪಯೋಗಿಸುವ ಚೈತನ್ಯವಷ್ಟೆ. ಹಿಂದೂಗಳು ಅವನನ್ನು ಆತ್ಮ ಎನ್ನುತ್ತಾರೆ. ಅದನ್ನು ಕತ್ತಿ ಛೇದಿಸಲಾರದು, ನೀರು ಕರಗಿಸಲಾರದು, ಗಾಳಿ ಒಣಗಿಸಲಾರದು. ಆತ್ಮ ಎಂದರೆ ಸುತ್ತಳತೆ ಇಲ್ಲದ ಒಂದು ವೃತ್ತವೆಂದೂ ಕೇಂದ್ರ ಮಾತ್ರ\break ದೇಹದಲ್ಲಿರುವುದೆಂದೂ ನಂಬುವರು. ಸಾವು ಎಂದರೆ ಒಂದು ದೇಹದಿಂದ ಮತ್ತೊಂದು ದೇಹಕ್ಕೆ ಈ ಕೇಂದ್ರದ ಬದಲಾವಣೆ. ಪ್ರಕೃತಿಯ ಸ್ವಭಾವದಿಂದ ಆತ್ಮ ಬಂಧಿತವಲ್ಲ. ಅದು ಸ್ವತಂತ್ರ, ಬಂಧನದಿಂದ ದೂರವಾದುದು, ಪವಿತ್ರವಾದುದು, ಶುದ್ಧವಾದುದು, ಪೂರ್ಣವಾದುದು. ಅಂತೂ ಹೇಗೋ ಪ್ರಕೃತಿಯ ಬಂಧನಕ್ಕೆ ಸಿಕ್ಕಿಕೊಂಡು ತಾನೂ ಪ್ರಕೃತಿ ಎಂದು ಭಾವಿಸುವುದು. ಪರಿಪೂರ್ಣವಾದ ಆತ್ಮ ಹೇಗೆ ಬಂಧನಕ್ಕೆ ಒಳಗಾಯಿತು ಎಂಬುವುದು ನಮಗೆ ಗೊತ್ತಿಲ್ಲ. ಅಂತೂ ಅದು ಈಗ ಸಿಕ್ಕಿಕೊಂಡಿರುವಂತಿದೆ ಎಂದು ಹೇಳುತ್ತಾರೆ. ಬಂಧನದಿಂದ ಪಾರಾಗಲು ಮಾರ್ಗವನ್ನು ಹಿಂದೂಧರ್ಮ ತೋರುತ್ತದೆ. 

 ಮಾನವನನ್ನು ವೇದ ಪಾಪಿ ಎನ್ನುವುದಿಲ್ಲ. ಅವನನ್ನು ಅಮೃತಪುತ್ರ ಎನ್ನುವದು. ನಾವು ಭಗವಂತನ ಮಕ್ಕಳು, ಸಚ್ಚಿದಾನಂದರಲ್ಲಿ ಭಾಗಿಗಳು, ಪರಿಶುದ್ಧಾತ್ಮರು. ನಾವು ನೀವೆಲ್ಲರೂ (ಹಿಂದೂಗಳಲ್ಲದವರು) ಜಗದ ಪವಿತ್ರಾತ್ಮರು. ಪಾಪಿಗಳೆ! ಹಾಗೆ ಮಾನವನನ್ನು ಕರೆಯುವುದೇ ಒಂದು ಮಹಾ ಪಾತಕ, ಮಾನವ ಸ್ವಭಾವದಲ್ಲಿರುವ ದೊಡ್ಡ ಕಳಂಕ. ಹೇ ಕೇಸರಿಗಳೇ! ನೀವು ಕುರಿಗಳು ಎಂಬ ಭ್ರಾಂತಿಯನ್ನು ಕೊಡವಿ ಬನ್ನಿ. ಅಮೃತಾತ್ಮರು ನೀವು! ಪ್ರಕೃತಿಯಲ್ಲ, ನೀವು ದೇಹವಲ್ಲ, ಪ್ರಕೃತಿ ನಿಮ್ಮ ಸೇವಕ, ನೀವು ಪ್ರಕೃತಿಯ ಸೇವಕರಲ್ಲ. 

 ವೇದ ಕಾರ‍್ಯಕಾರಣ ಸಂಬಂಧಗಳ ಒಂದು ಭಯಾನಕ ಸಂಯೋಗವನ್ನು ಸಾರುವುದಿಲ್ಲ. ಈ ಪ್ರಕೃತಿ ನಿಯಮಗಳನ್ನು ಮೆಟ್ಟಿ ನಿಂತ ಒಂದು ಮೂಲ ಚೈತನ್ಯವಿದೆ. ಅದೇ ಭಗವಂತ: “ಅವನಾಜ್ಞೆಯಿಂದ ಗಾಳಿ ಬೀಸುವುದು, ಬೆಂಕಿ ಉರಿಯುವುದು, ಮೋಡಗಳು ಮಳೆಗರೆಯುವುವು, ಮೃತ್ಯು ತನ್ನ ಕಾರ‍್ಯವನ್ನು ಮಾಡುವುದು.” ಅವನು ಸರ್ವಾಂತರ್ಯಾಮಿ, ಪರಿಶುದ್ಧ ನಿರಾಕಾರ ಸರ್ವಶಕ್ತ ಮತ್ತು ದಯಾಸಿಂಧು. ಅಂತಹ ದೇವರನ್ನು ಪ್ರೀತಿಗೋಸುಗ ಪ್ರೀತಿಸಬೇಕು, ಅದೇ ಶ್ರೇಷ್ಠ. ಯಾವುದಾದರೂ ಒಂದು ವಸ್ತುವನ್ನು ಪಡೆಯಬೇಕೆಂದು ಪ್ರೀತಿಸುವುದು ಒಂದು ವ್ಯಾಪಾರ, ಅದು ಪ್ರೀತಿಯಲ್ಲ. ಅಂತಹ ದೇವರನ್ನು ಸಾಕ್ಷಾತ್ಕಾರ ಮಾಡಿಕೊಂಡಿರುವೆ ಎಂಬುದೇ ಧರ‍್ಮಕ್ಕೆ ಪ್ರಮಾಣ. ಹಿಂದೂ ಧರ್ಮ ಯಾವುದೋ ಕೆಲವು ಸಿದ್ಧಾಂತಗಳನ್ನು ಅಥವಾ ಮೂಢನಂಬಿಕೆಗಳನ್ನು ನಂಬುವುದಕ್ಕೆ ಮಾಡುವ ಪ್ರಯತ್ನವಲ್ಲ. ಅವುಗಳನ್ನು ಸಾಕ್ಷಾತ್ಕಾರ ಮಾಡಿಕೊಳ್ಳಬೇಕು, ಅದರಂತೆ ಇರಬೇಕು, ಅದರಂತೆ ಆಗಬೇಕು. 

 ಅನವರತ ಮುಕ್ತಾತ್ಮರಾಗುವುದು, ಪವಿತ್ರಾತ್ಮರಾಗುವುದು, ದೇವರನ್ನು ಸೇರುವುದು ಮತ್ತು ದೇವರನ್ನು ನೋಡುವುದು ಅವರ ಸಿದ್ಧಾಂತದ ಗುರಿ. ಇಲ್ಲಿಂದ ಮುಂದಕ್ಕೆ ಮತ್ತೊಂದು ಮೆಟ್ಟಲು ಇದೆ. ಕೆಲವರಿಗೆ ದೇವರನ್ನು ಸೇರಬೇಕು ಎಂಬ ಆಸೆಯಿದೆ. ಇದನ್ನು ದ್ವೈತ ಎಂದು ಹೇಳುವರು. ಕೆಲವರಿಗೆ ದೇವರ ಅಂಶವಾಗಬೇಕು ಎಂಬ ಭಾವವಿದೆ. ಅವರು ವಿಶಿಷ್ಟಾದ್ವೈತಿಗಳು. ಮತ್ತೆ ಕೆಲವರು ದೇವರಲ್ಲಿ ಸಂಪೂರ್ಣ ತಮ್ಮ ವ್ಯಕ್ತಿತ್ವವನ್ನು ಅಳಿಸಿಬಿಟ್ಟು ಅವನಲ್ಲಿ ಒಂದಾಗಲು ಯತ್ನಿಸುವರು. ಅದು ಅದ್ವೈತ. ಹಿಂದೂಧರ್ಮದಲ್ಲಿ ಈ ಮೆಟ್ಟಲುಗಳೆಲ್ಲಾ ಇವೆ. 

\newpage

 ಭರತಖಂಡದಲ್ಲಿ ವಿಗ್ರಹಾರಾಧನೆಗೆ ಅವಕಾಶವಿದೆ. ಇದು ಸರ್ವವ್ಯಾಪಿಯಾದ\break ಭಗವಂತನನ್ನು ಚಿಂತಿಸುವುದಕ್ಕೆ ಒಂದು ಆಶ್ರಯ. ವಿಗ್ರಹಗಳೇ ದೇವರಲ್ಲ. ವಿಗ್ರಹದ ಮೂಲಕ ಹಿಂದೂ ದೇವರನ್ನು ನೋಡುವನು. ಪ್ರಪಂಚದಲ್ಲಿ ಎಲ್ಲಾ ಧರ್ಮದಲ್ಲಿಯೂ ದೇವರನ್ನು ಜ್ಞಾಪಕಕ್ಕೆ ತರುವುದಕ್ಕೆ ಒಂದೊಂದು ಚಿಹ್ನೆಗಳಿವೆ. ಆದರೆ ಅನೇಕ ವೇಳೆ ಆಯಾ ಮತಕ್ಕೆ ಸೇರಿದವರು ತಮ್ಮ ಚಿಹ್ನೆಯನ್ನು ನಿಜವಾದ ಚಿಹ್ನೆ ಎಂದು ಭಾವಿಸಿ, ಇತರರ ಚಿಹ್ನೆಯನ್ನು ತಪ್ಪು ಎಂದು ಅವರನ್ನು ದೂರುವರು. ಆದರೆ ಹಿಂದುವಿನ ದೃಷ್ಟಿ ಉದಾರವಾದುದು. ಅವನು ಯಾರನ್ನೂ ಹಳಿಯುವುದಕ್ಕೆ ಹೋಗುವುದಿಲ್ಲ. ದೇವರು ಯಾವುದೋ ಒಂದು ಧರ್ಮಕ್ಕೆ ಮಾತ್ರ ಎಲ್ಲಾ ಸತ್ಯವನ್ನೂ ಕೊಟ್ಟುಬಿಟ್ಟು ಉಳಿದವುಗಳಿಗೆ ಸುಳ್ಳನ್ನು ಹಂಚಿಲ್ಲ. ಶ‍್ರೀಕೃಷ್ಣ ಗೀತೆಯಲ್ಲಿ ಹೀಗೆ ಹೇಳುವನು: “ಹಲವು ಮಣಿಗಳನ್ನು ಕೂಡಿಸುವ ಒಂದು ದಾರದಂತೆ ನಾನೇ ಎಲ್ಲಾ ಧರ್ಮದಲ್ಲಿಯೂ ಇರುವೆನು. ಎಲ್ಲಿ ಅಸಾಧಾರಣವಾದ ಪವಿತ್ರತೆ ಮತ್ತು ಮಹಾಶಕ್ತಿ ಉದಿಸಿ ಮಾನವ ಜನಾಂಗವನ್ನು ಪವಿತ್ರ ಮಾಡುತ್ತಿರುವುದೊ ಅಲ್ಲೆಲ್ಲ ನಾನೇ ಇರುವೆನು ಎಂದು ತಿಳಿ.” 

 ಇದರಿಂದ ಮಹಾತ್ಮರು ಇತರ ಧರ್ಮಗಳಲ್ಲಿಯೂ ಇರುವರು, ಅವರು ಮುಕ್ತಿಗೆ ಅರ್ಹರು ಎಂಬ ಉದಾರಭಾವನೆ ಹಿಂದೂವಿಗೆ ಬಾಲ್ಯಾರಭ್ಯದಿಂದಲೇ ಬರುವುದು. ಅವನು ಆ ವಾತಾವರಣದಲ್ಲಿ ಬೆಳೆಯುವನು. ಅವನು ತಾನು ಹುಟ್ಟು ಹಿಂದೂವಾಗಿದ್ದರೂ ಎಲ್ಲಾ ಧರ್ಮಗಳ ಮೇಲೆಯೂ ಅವನಿಗೆ ಭಕ್ತಿ ಗೌರವಗಳಿವೆ. ಇವನು ಎಲ್ಲಾ ಧರ್ಮಗಳೂ ಒಂದೇ ಪರಮ ಸತ್ಯದ ಕಡೆಗೆ ಒಯ್ಯುವ ಮಾರ್ಗಗಳು ಎಂದು ನಂಬುವನು. 

 ಕೊನೆಗೆ ಸ್ವಾಮೀಜಿ ಹೀಗೆ ಹೇಳುವರು: “ಯಾರು ಹಿಂದೂಗಳ ಬ್ರಹ್ಮನೋ, ಯಾರು ಜ಼ರತೂಷ್ಟ್ರ ಸಂಪ್ರದಾಯದ ಅಹುರಮಾಜ್ದನೊ, ಬೌದ್ಧರ ಬುದ್ಧನೋ, ಯಹೂದ್ಯರ ಜೇಹೋವನೋ, ಕ್ರೈಸ್ತರ ಸ್ವರ್ಗದಲ್ಲಿರುವ ತಂದೆಯೋ ಅವನು ನಿಮ್ಮ ಈ ಘನ ಉದ್ದೇಶವನ್ನು ಕಾರ‍್ಯರೂಪಕ್ಕೆ ತರಲು ಶಕ್ತಿಯನ್ನು ಅನುಗ್ರಹಿಸಲಿ. ಪೂರ್ವ ದಿಗಂತದಲ್ಲಿ ತಾರೆ ಉದಿಸಿತು. ಕ್ರಮೇಣ ಅದು ಪಶ್ಚಿಮದ ಕಡೆ ಪ್ರಯಾಣ ಮಾಡಿತು. ಕೆಲವು ವೇಳೆ ಅದು ಮಬ್ಬು ಮಬ್ಬಾಗಿ ಕೆಲವು ವೇಳೆ ಜಾಜ್ವಲ್ಯಮಾನವಾಗಿ ಬೆಳಗುತ್ತ ಪ್ರಪಂಚವನ್ನು ಒಂದು ಪ್ರದಕ್ಷಿಣೆ ಮಾಡಿ ಪುನಃ ಅದೇ ಪೂರ್ವ ದಿಗಂತದಲ್ಲಿ ಸಾನ್ಪೋ ನದಿಯ ತೀರದಲ್ಲಿ ಹಿಂದಿಗಿಂತಲೂ ಸಾವಿರಪಾಲು ಕಾಂತಿಯುತವಾಗಿ ಉದಯಿಸುತ್ತಿದೆ. 

 “ಹೇ ಕೊಲಂಬಿಯ! ಸ್ವಾತಂತ್ರ್ಯದ ತವರೂರೆ, ನಿನಗೆ ವಿಜಯವಾಗಲಿ, ನೆರೆಯವರ ರಕ್ತದಲ್ಲಿ ನಿನ್ನ ಕೈಯನ್ನು ಎಂದೂ ಅದ್ದಲಿಲ್ಲ. ಶ‍್ರೀಮಂತರಾಗುವುದಕ್ಕೆ ಅತಿ ಶೀಘ್ರವಾದ ಹಾದಿಯೆ ತಮ್ಮ ನೆರೆಹೊರೆಯವರನ್ನು ಸೂರೆಮಾಡುವುದು ಎಂಬುದನ್ನು ಎಂದೂ ಅರಿಯದ ನಿನಗೆ ಸಮನ್ವಯ ಪತಾಕೆಯನ್ನು ಹಿಡಿದು ನಾಗರೀಕತೆಯ ಮುಂದಾಳಾಗಿ ನಡೆವ ಒಂದು ಮಹಾ ಗೌರವ ಸಲ್ಲುತ್ತದೆ.” 

 ಯಾವ ಸಮನ್ವಯ ವಾಣಿಯನ್ನು ದಕ್ಷಿಣೇಶ್ವರ ಗುರುವರ‍್ಯನ ಅಡಿದಾವರೆಯಲ್ಲಿ ಕುಳಿತು ಸ್ವಾಮೀಜಿ ಕಲಿತಿದ್ದರೋ ಅದನ್ನು ಅವರು ಪಾಶ್ಚಾತ್ಯ ದೇಶಗಳಲ್ಲಿ ಮೊಳಗಿದರು. ಹಿಂದೂಧರ್ಮದಷ್ಟು ವಿಶಾಲವಾದ, ಉದಾರವಾದ ಧರ್ಮ ಮತ್ತೊಂದು ಇಲ್ಲ ಎನ್ನುವುದು ಎಲ್ಲರಿಗೂ ಕಣ್ಣಿಗೆ ಕಟ್ಟಿದಂತೆ ಆಯಿತು. ಪಾಶ್ಚಾತ್ಯ ಜನರು ಇಂಡಿಯಾ ದೇಶದವರನ್ನು ಅರೆನಾಗರೀಕರು, ಮಣ್ಣು ಕಲ್ಲುಗಳನ್ನು ಪೂಜಿಸುವವರು, ಅವರಿಗೆ ತಾವು ನಾಗರಿಕತೆಯನ್ನು ಕೊಡಬೇಕಾಗಿದೆ ಎಂದು ಭಾವಿಸಿದ್ದರು. ಆದರೆ ಸ್ವಾಮೀಜಿ ನುಡಿಗಳನ್ನು ಕೇಳಿದಾಗ ಪ್ರಪಂಚದಲ್ಲಿ, ಅಧ್ಯಾತ್ಮ ಮತ್ತು ತತ್ತ್ವ ಕ್ಷೇತ್ರದ ಶ್ರೇಷ್ಠತಮ ಭಾವನೆಗಳನ್ನು ಅರಿಯಬೇಕಾದರೆ ಭರತಖಂಡದ ಕಡೆ ತಿರುಗಬೇಕೆಂದು ಭಾವಿಸತೊಡಗಿದರು. ಹೊಟ್ಟೆಗಿಲ್ಲದೆ ಬಟ್ಟೆಗಿಲ್ಲದೆ ದಟ್ಟ ದಾರಿದ್ರ್ಯದಲ್ಲಿ ನರಳುವ ಜನಾಂಗವಾದರೂ ಅದಕ್ಕೆ ಮತ್ತೊಬ್ಬರಿಗೆ ಕೊಡುವುದಕ್ಕೆ ಒಂದು ಇದೆ. ಅದೇ ಅಧ್ಯಾತ್ಮ. ಸ್ವಾಮೀಜಿ ಹೊರಗಿನಿಂದ ನೋಡಿದರೆ ಭಿಕ್ಷುಕನಂತೆ ಹೋದರು. ಆದರೆ ಅಮೇರಿಕಾ ದೇಶದಲ್ಲಿ ಚಕ್ರವರ್ತಿಯಂತೆ ಅನರ್ಘ್ಯರತ್ನಗಳನ್ನು ಕೊಟ್ಟರು. ಜೀವನಕ್ಕೆ ಶಾಂತಿ ತರುವ, ಭಿನ್ನಾಭಿಪ್ರಾಯದ ಘರ್ಷಣೆಗಳಿಂದ ಪಾರುಮಾಡುವ ವೈವಿಧ್ಯತೆಯ ಹಿಂದೆ ಇರುವ ಐಕ್ಯತೆಯನ್ನು ನೋಡುವ ಸಂದೇಶವನ್ನು ಕೊಟ್ಟರು. ಇದರಿಂದ ಸ್ವಾಮೀಜಿಗೆ ಕೀರ್ತಿ ಬಂದಿತು. ಆದರೆ ತಮ್ಮ ವ್ಯಕ್ತಿತ್ವಕ್ಕೆ ಈ ಕೀರ್ತಿ ಸಲ್ಲುತ್ತದೆ ಎಂದು ಅವರು ಭಾವಿಸಲಿಲ್ಲ. ಭರತಖಂಡದ ಪುರಾತನ ಮಹರ್ಷಿಗಳಿಗೆ, ಗೀತಾ ಉಪನಿಷತ್ತನ್ನು ಕೊಟ್ಟ ಮಹಾನ್ ವ್ಯಕ್ತಿಗಳಿಗೆ ಸಲ್ಲುವ ಗೌರವವೆಂದು ಅವರ ಪರವಾಗಿ ಅದನ್ನು ಸ್ವೀಕರಿಸಿದರು. 

 ಸ್ವಾಮೀಜಿ ಸೆಪ್ಟೆಂಬರ್ ೨೦ರಂದು ಕ್ರೈಸ್ತರಿಗೆ ಒಂದು ಕಟುವಾದ ಮಾತನ್ನು ಆಡಿದರು. ಇದಕ್ಕೆ ಮುಂಚೆ ಇಂಡಿಯಾದೇಶದಿಂದ ಪಶ್ಚಿಮದೇಶಗಳಿಗೆ ಹೋದವರು ಅವರನ್ನು ಅಂಗಲಾಚಿ ಬೇಡಿಕೊಳ್ಳುತ್ತಿದ್ದರು. ತಮ್ಮಲ್ಲಿ ಯೋಗ್ಯವಾಗಿರುವುದು ಏನೂ ಇಲ್ಲ, ನೀವು ಸರಿ, ನೀವು ಬಂದು ನಮ್ಮನ್ನು ಉದ್ಧಾರ ಮಾಡಬೇಕೆಂದು ಅವರನ್ನು ಹೊಗಳುತ್ತ ತಮ್ಮನ್ನು ನಿಂದಿಸಿಕೊಳ್ಳುತ್ತಿದ್ದರು. ತಮ್ಮ ಜನರಲ್ಲೆ ತಪ್ಪನ್ನು ಕಂಡುಹಿಡಿಯುತ್ತಿದ್ದರು. ಕ್ರೈಸ್ತದೇಶದಲ್ಲಿ ಜನರು ಇತರ ದೇಶದ ಜನರ ಆರ್ಥಿಕ ಸಹಾಯಕ್ಕೆ ಬರುವುದರ ಬದಲು ಅವರಿಗೆ ಧರ‍್ಮವನ್ನು ಕೊಡುವುದಕ್ಕೆ ಬರುತ್ತಿದ್ದುದನ್ನು ನೋಡಿ ಸ್ವಾಮೀಜಿಗೆ ಸಹಿಸಲಸದಳವಾಯಿತು. ಇದು ಕಲ್ಲಿದ್ದಲಿನ ಊರಿಗೆ ಎಲ್ಲಿಂದಲೋ ಕಲ್ಲಿದ್ದಲನ್ನು ಸಾಗಿಸಿದಂತೆ. ಅವರು ಆ ತುಂಬಿದ ಸಭೆಯಲ್ಲಿ ನಿರ್ಭೀತಿಯಿಂದ ಮಾತನಾಡಿದರು: 

 “ಕ್ರೈಸ್ತರು ಯಾವಾಗಲೂ ಹಿತಕಾರಿಯಾದ ವಿಮರ್ಶೆಯನ್ನು ಸ್ವೀಕರಿಸಲು ಯೋಗ್ಯರಾಗಿರಬೇಕು. ನಾನು ನಿಮ್ಮಲ್ಲಿ ಅಲ್ಪದೋಷವನ್ನು ತೋರಿದರೆ ಅದರಿಂದ ನಿಮಗೂ ಏನೂ ಬಾಧಕವಿಲ್ಲವೆಂದು ತಿಳಿಯುತ್ತೇನೆ. ಕ್ರೈಸ್ತಮತಾನುಯಾಯಿಗಳೇ, ನೀವು ಕ್ರೈಸ್ತರಲ್ಲದವರ ಆತ್ಮವನ್ನು ಉದ್ಧರಿಸಲು ಅಷ್ಟೊಂದು ಕಾತರರಾಗುವಿರಲ್ಲ, ಉಪವಾಸದಿಂದ ನಶಿಸಿಹೋಗುತ್ತಿರುವ ಅವರ ದೇಹವನ್ನು ಕಾಪಾಡುವುದಕ್ಕೆ ಏತಕ್ಕೆ ಪ್ರಯತ್ನ ಮಾಡಬಾರದು? ಭರತಖಂಡದಲ್ಲಿ ಭಯಂಕರ ಕ್ಷಾಮದಿಂದ ಸಾವಿರಾರು ಜನ ಸತ್ತರು. ಆದರೂ ಕ್ರೈಸ್ತರು ಏನನ್ನೂ ಮಾಡಲಿಲ್ಲ. ಭರತಖಂಡದ ಮೂಲೆ ಮೂಲೆಗಳಲ್ಲಿ ನೀವು ಚರ್ಚುಗಳನ್ನು ಕಟ್ಟುತ್ತೀರಿ. ಪೌರಸ್ತ್ಯದೇಶದಲ್ಲಿ ನಾವು ಅತಿ ಮುಖ್ಯವಾಗಿ ಗಮನದಲ್ಲಿಡಬೇಕಾಗುವುದು ಧಾರ್ಮಿಕ ಬರಗಾಲವನ್ನಲ್ಲ, ಅವರಿಗೆ ಧರ್ಮ ಹೇರಳವಾಗಿದೆ. ಭರತಖಂಡದಲ್ಲಿ ಹಸಿವಿನಿಂದ ನರಳುತ್ತಿರುವ ಕೋಟ್ಯಂತರ ಜನರು ದುಃಖದಿಂದ ಒಣಗಿದ ಗಂಟಲಿನಲ್ಲಿ ಒಂದು ತುತ್ತು ಕೂಳಿಗಾಗಿ ಅಳುತ್ತಿರುವರು. ಅವರು ನಮ್ಮನ್ನು ಕೇಳುವುದು ಅನ್ನ. ಆದರೆ ನಾವು ಅವರಿಗೆ ಕೊಡುವುದು ಕಲ್ಲು. ಉಪವಾಸದಿಂದ ನರಳುತ್ತಿರುವವರಿಗೆ ತತ್ತ್ವೋಪದೇಶ ಮಾಡುವುದೊಂದು ಅಪಮಾನ! ಭರತಖಂಡದಲ್ಲಿ ಹಣಕ್ಕೋಸ್ಕರ ಯಾರು ಧರ್ಮವನ್ನು ಬೋಧಿಸುವರೋ ಅವರು ತಮ್ಮ ಜಾತಿಯನ್ನು ಕಳೆದುಕೊಳ್ಳುವರು. ಜನರು ಅವರ ಮುಖಕ್ಕೆ ಉಗುಳುವರು. ನನ್ನ ದೇಶದಲ್ಲಿ ದಾರಿದ್ರ್ಯದಿಂದ ನರಳುತ್ತಿರುವವರಿಗೆ ಸಹಾಯ ಬೇಡುವುದಕ್ಕೆ ನಾನು ಬಂದೆ. ಕ್ರೈಸ್ತರಿಂದ ಕ್ರೈಸ್ತರಲ್ಲದವರಿಗೆ ಸಹಾಯವನ್ನು ಹೊಂದುವುದು ಎಷ್ಟು ಕಷ್ಟ ಎಂಬುದು ಈಗ ನನಗೆ ಅರಿವಾಗಿದೆ.” 

 ೨೬ನೇ ತಾರೀಖು ‘ಬೌದ್ಧಧರ್ಮ ಹಿಂದೂಧರ್ಮದ ಪೂರೈಕೆ’ ಎಂಬ ವಿಷಯದಲ್ಲಿ ಉಪನ್ಯಾಸ ಮಾಡಿದರು. ಹಿಂದೂಧರ್ಮ ಬೌದ್ಧ ಧರ್ಮವಿಲ್ಲದೆ ಬಾಳಲಾರದು. ಹಿಂದೂಧರ್ಮದ ಅದ್ಭುತವಾದ ಧೀಶಕ್ತಿಯ ಜೊತೆಗೆ ಭಗವಾನ್ ಬುದ್ಧನ ವಿಶ್ವಾನುಕಂಪ ಒಂದುಗೂಡಬೇಕು, ಅದಕ್ಕೆ ನಾವು ಯತ್ನಿಸಬೇಕು ಎಂದರು. 

 ೨೭ನೆಯ ತಾರೀಖು ಅಧಿವೇಶನದ ಕೊನೆಯ ಭಾಷಣವನ್ನು ಸ್ವಾಮೀಜಿ ಮಾಡಿ ಕೊನೆಯಲ್ಲಿ ಹೀಗೆಂದರು: 

 “ಜಗತ್ತಿಗೆ ವಿಶ್ವಧರ್ಮ ಸಮ್ಮೇಳನ ಏನಾದರೂ ತೋರಿದ್ದರೆ ಅದು ಇದು: ಪವಿತ್ರತೆ ದಯೆ ಮುಂತಾದುವು ಜಗತ್ತಿನ ಯಾವ ಧರ್ಮಸಂಸ್ಥೆಗೂ ಮೀಸಲಲ್ಲ. ಪ್ರತಿಯೊಂದು ಧರ್ಮವೂ ಕೂಡ ಅತ್ಯುತ್ತಮವಾದ ಸ್ತ್ರೀ–ಪುರುಷರನ್ನು ಕೊಟ್ಟಿರುವುದು. ಇಷ್ಟೊಂದು ನಿದರ್ಶನಗಳು ಇರುವಾಗಲೂ, ಯಾರಾದರೂ ತಮ್ಮ ಧರ್ಮ ಒಂದೇ ಬಾಳುವುದು, ಉಳಿದ ಧರ್ಮಗಳಾವುವೂ ಇರುವುದಿಲ್ಲ ಎಂದು ಭಾವಿಸಿದರೆ ನಾನು ಅವರಿಗಾಗಿ\break ಹೃತ್ಪೂರ್ವಕ ಮರುಗುವೆ. ಪ್ರತಿಯೊಂದು ಧರ್ಮದ ಧ್ವಜದ ಮೇಲೂ ಅವರು ಎಷ್ಟು ವಿರೋಧಿಸಿದರೂ ‘ಹೋರಾಟವಲ್ಲ, ಸಹಾಯ, ನಾಶವಲ್ಲ ಸ್ವೀಕಾರ, ವೈಮನಸ್ಯವಲ್ಲ ಶಾಂತಿ ಮತ್ತು ಸಮನ್ವಯ’ ಎಂಬುದನ್ನು ಬೇಗ ಬರೆಯಲಾಗುವುದು.” 

 ಸ್ವಾಮಿ ವಿವೇಕಾನಂದರು ವಿಶ್ವಧರ್ಮ ಸಮ್ಮೇಳನದಿಂದ ಜಗದ್ವಿಖ್ಯಾತರಾದರು, ಭರತಖಂಡದ ಧರ್ಮಕ್ಕೆ ಸಂಸ್ಕೃತಿಗೆ ವಿಶ್ವವೇ ಮಣಿಯುವಂತೆ ಮಾಡಿದರು. ಬಂದವರಲ್ಲಿ ಅಚ್ಚರಿ ಮೂಡುವಂತೆ ಮಾಡಿದರು. ಅವರು ಕೀರ್ತಿಯಿಂದ ಸಂತೋಷಪಡಲಿಲ್ಲ. ಪರಿವ್ರಾಜಕರಾಗಿ ಅನಾಮಧೇಯರಾಗಿ ಅಲೆಯುತ್ತಿದ್ದಾಗ ಇದ್ದ ಶಾಂತಿ ಇನ್ನುಮೇಲೆ ತನ್ನ ಪಾಲಿಗೆ ಎಂದೆಂದಿಗೂ ಸಿಕ್ಕುವಂತಿಲ್ಲ ಎಂದು ವ್ಯಥೆಪಟ್ಟರು. ಇನ್ನು ಅವರಿಗೆ ಏಕಾಂತ ಜೀವನವಿಲ್ಲ. ಕೀರ್ತಿ, ಬೆಳಕು ಸದಾ ಅವರನ್ನು ಹಿಂಬಾಲಿಸುತ್ತಿರುವುದು. ಅವರು ಎಲ್ಲಿ ಕುಳಿತುಕೊಳ್ಳುವರು, ಎಲ್ಲಿ ಮಲಗುವರು, ಏನು ತಿನ್ನುವರು, ಏನು ಮಾಡುವರು ಎಂಬುದನ್ನು ಅರಿಯುವುದಕ್ಕೆ ವೃತ್ತಪತ್ರಿಕಾ ಬಾತ್ಮೀದಾರರು ಮುತ್ತಿಗೆ ಹಾಕುವರು. 

\newpage

 ಇನ್ನುಮೇಲೆ ಸ್ವಾಮೀಜಿಗೆ ಬಿಡುವಿಲ್ಲ. ಉಪನ್ಯಾಸದ ಸುರಿಮಳೆ ಅವರಿಗೆ ಕಾದಿದ್ದುವು. ಬಿಡುವಿಲ್ಲದೆ ಅಮೇರಿಕಾದ ಒಂದು ಮೂಲೆಯಿಂದ ಮತ್ತೊಂದು ಮೂಲೆಗೆ ಅಲೆದರು. ಜನರು ಕೇಳುವ ಹಲವು ಹಾಸ್ಯಾಸ್ಪದ ಪ್ರಶ್ನೆಗಳಿಗೆ ಉತ್ತರ ಕೊಡುವುದು, ಅವರ ನಿಂದೆಯನ್ನು ಎದುರಿಸುವುದು, ಎಲ್ಲಕ್ಕಿಂತ ಹೆಚ್ಚಾಗಿ ವೇದಾಂತದ ಗಹನಭಾವನೆಯನ್ನು ವೈಜ್ಞಾನಿಕ ರೀತಿ ಎಲ್ಲರೂ ಗ್ರಹಿಸುವಂತಹ ರೀತಿಯಲ್ಲಿ ಬೋಧಿಸುವುದು – ಇದರಲ್ಲೇ ಅವರ ಅಮೇರಿಕಾ ದೇಶದ ಕಾಲವೆಲ್ಲ ಕಳೆಯುವುದು. 

 ತಮಗೆ ಎಷ್ಟೇ ಸೌಲಭ್ಯಗಳು ಬಂದರೂ ಭರತಖಂಡದ ದುರ್ದೆಸೆಯನ್ನು ಸ್ವಾಮೀಜಿ ಮರೆಯಲಿಲ್ಲ. ಬಿರುಗಾಳಿಯಂತೆ ಅಮೇರಿಕಾ ದೇಶದಲ್ಲಿ ಸಂಚರಿಸುತ್ತಿದ್ದರೂ ಭರತಖಂಡವನ್ನು ಮೇಲೆತ್ತಬೇಕಾದರೆ ಏನು ಮಾಡಬೇಕೆಂಬುದೊಂದು ಅವರ ಹೃದಯದಲ್ಲಿ ಕುದಿಯುತ್ತಿತ್ತು. ಆ ಸಮಯದಲ್ಲಿಯೇ ಭರತಖಂಡದ ಅನೇಕ ಭಕ್ತರಿಗೆ, ಉದ್ಧಾರಕ್ಕೆ ಸೊಂಟಕಟ್ಟಿ ನಿಲ್ಲಬೇಕೆಂದೂ ಆ ಮಹಾಸೇವೆಯಲ್ಲಿ ಬಾಳನ್ನು ಅರ್ಪಣ ಮಾಡಲು ದೀಕ್ಷೆಯನ್ನು ತೊಡಬೇಕೆಂದೂ ಬೆಂಕಿಯಂತಹ ಭಾಷೆಯಲ್ಲಿ ಪತ್ರಗಳನ್ನು ಬರೆಯುತ್ತಿದ್ದರು. 

 ಸ್ವಾಮಿ ವಿವೇಕಾನಂದರು ಕೀರ್ತಿಯನ್ನು ಪಡೆದ ದಿನದಿಂದಲೇ ಇಂಡಿಯಾ ದೇಶದಿಂದ ಹೋದ ಕೆಲವರಿಗೆ ಅಸೂಯೆ ಹುಟ್ಟಿತು. ವಿವೇಕಾನಂದರ ಕೀರ್ತಿ ಎದುರಿಗೆ ಅವರೆಲ್ಲ ಮ್ಲಾನರಾಗಿ ಹೋದರು. ಅವರನ್ನು ಯಾರೂ ಕೇಳದಂತೆ ಆದರು. ಸೂರ್ಯನೆದುರಿಗೆ ಇರುವ ನಕ್ಷತ್ರಗಳಂತೆ ಮಾಯವಾಗಿ ಹೋದರು. ಕ್ರೈಸ್ತ ಪಾದ್ರಿಗಳು ವಿವೇಕಾನಂದರ ಉದಾರ ಸಂದೇಶವನ್ನು ಸಹಿಸಲಿಲ್ಲ. ಸ್ವಾಮೀಜಿ ಮುಚ್ಚುಮರೆಯಿಲ್ಲದೆ ಪಾದ್ರಿಗಳನ್ನು ಬಹಿರಂಗದಲ್ಲಿ ಖಂಡಿಸತೊಡಗಿದರು. ಪಾದ್ರಿಗಳು ಮತ್ತು ಇಂಡಿಯಾ ದೇಶದಿಂದ ಹೋದ ಕೆಲವರು ಸೇರಿ ಸ್ವಾಮಿಗಳ ಮೇಲೆ ಅಪಪ್ರಚಾರವನ್ನು ಆರಂಭಿಸಿದರು. ಅವರು ಉತ್ತಮ ಕುಲಕ್ಕೆ ಸೇರಿದವರಲ್ಲವೆಂದೂ, ಅವರೊಬ್ಬ ಭಿಕ್ಷುಕರ ಗುಂಪಿಗೆ ಸೇರಿದವರೆಂದೂ ಸಾರತೊಡಗಿದರು. ಪಾದ್ರಿಗಳು ಅನೇಕ ಸ್ತ್ರೀಯರಿಗೆ ಹಣದ ಆಸೆಯನ್ನು ತೋರಿ ವಿವೇಕಾನಂದರನ್ನು ಕೆಡಿಸುವಂತೆ ಪ್ರಚೋದಿಸಿದರು. ಕೆಲವು ಮನೆಗಳಿಗೆ ಸ್ವಾಮಿ ವಿವೇಕಾನಂದರನ್ನು ಕರೆಯುತ್ತಿದ್ದರು. ಸ್ವಾಮಿಗಳು ಅಲ್ಲಿಗೆ ಹೋದರೆ ಮನೆಯವರು ಬೀಗವನ್ನು ಹಾಕಿಕೊಂಡು ಎಲ್ಲಿಗೋ ಹೊರಟು ಹೋಗಿಬಿಡುತ್ತಿದ್ದರು. ಏಕೆಂದರೆ ಸ್ವಾಮಿಗಳು ಬರುವುದಕ್ಕೆ ಮುಂಚೆ ಯಾರೋ ಹೆಸರಿಲ್ಲದ ಕಾಗದವನ್ನು ಬರೆದು ವಿವೇಕಾನಂದರ ಚಾರಿತ್ರ್ಯ ಚೆನ್ನಾಗಿಲ್ಲವೆಂದೂ ಅವರನ್ನು ಗೌರವಸ್ಥರು ಮನೆಗೆ ಸೇರಿಸಬಾರದೆಂದೂ ಹೇಳುತ್ತಿದ್ದರು. ಕೆಲವು ಸ್ತ್ರೀಯರ ಕೈಯಿಂದ ತಮ್ಮನ್ನು ವಿವೇಕಾನಂದರು ಮಾನಭಂಗ ಮಾಡಿದರೆಂದು ಪತ್ರಿಕೆಗಳಲ್ಲಿ ಪಾದ್ರಿಗಳು ಬರೆಸಿದರು. ವಿವೇಕಾನಂದರಿಗೆ ಬಂದ ಕೀರ್ತಿ ಒಂದು ಕಂಟಕಮಯದ ಹಾಸಿಗೆಯಂತೆ ಆಯಿತು. ಆದರೆ ಸ್ವಾಮೀಜಿ ಇಂತಹ ಗೊಡ್ಡು ಅಪವಾದಗಳಿಗೆ ಅಂಜುವವರಲ್ಲ. ಧೈರ್ಯವಾಗಿ ಅದನ್ನು ಎದುರಿಸಿದರು. ಅನೇಕ ವೇಳೆ ಮುಂಚೆ ದೂರಿ ಬರೆದ ಹೆಂಗಸರೇ ಅನಂತರ ಸ್ವಾಮೀಜಿ ಅವರ ಕ್ಷಮಾಪಣೆಯನ್ನು ಕೋರಿ ಕಾಗದ ಬರೆದರು. ಇಂತಹ ಸಮಯದಲ್ಲಿ ಬರೆದ ಒಂದು ಪತ್ರದಲ್ಲಿ ಸ್ವಾಮೀಜಿ “ನಾನು ಈಜುವುದಕ್ಕೆ ನೀರಿಗೆ ಇಳಿದಿರುವೆ, ಚೆನ್ನಾಗಿ ಸ್ನಾನ ಮಾಡಿಯೇ ಬರುತ್ತೇನೆ. ಯಾರಿಗೂ ಅಂಜುವುದಿಲ್ಲ” ಎಂದು ಬರೆದಿರುವರು. ಆಗಲೇ ಒಂದು ಸಾರಿ ಮಾತನಾಡುವಾಗ ಕ್ರೈಸ್ತರನ್ನು ಕುರಿತು ಹೀಗೆ ಹೇಳುವರು: 

 “ಒಂದನ್ನು ನಾನು ನಿಮಗೆ ಹೇಳಬೇಕೆಂದು ಇರುವೆ. ಅದನ್ನು ನಿರ್ದಯವಾದ ಟೀಕೆ ಎಂದು ಭಾವಿಸಬೇಡಿ. ನೀವು ನಿಮ್ಮ ಪಾದ್ರಿಗಳಿಗೆ ವಿದ್ಯಾಭ್ಯಾಸ ಕೊಟ್ಟು ತರಬೇತು ಕೊಟ್ಟು ಅವರಿಗೆ ಸಂಬಳ ಕೊಟ್ಟು ಕಳುಹಿಸುತ್ತೀರಿ. ಇದು ಏತಕ್ಕೆ? ನಮ್ಮ ದೇಶಕ್ಕೆ ಬಂದು ನಮ್ಮ ಪೂರ್ವಿಕರ ಧರ್ಮ ಆಚಾರ ವ್ಯವಹಾರ ಮುಂತಾದುವುಗಳನ್ನು ಟೀಕಿಸಿ ಅವಹೇಳನ ಮಾಡುವುದಕ್ಕೆ. ಅವರು ಒಂದು ದೇವಸ್ಥಾನದ ಹತ್ತಿರ ಹೋಗಿ ‘ಎಲೈ ವಿಗ್ರಹಾರಾಧಕರೆ, ನೀವು ನರಕಕ್ಕೆ ಹೋಗುತ್ತೀರಿ’ ಎನ್ನುತ್ತಾರೆ. ಆದರೆ ಹಿಂದು, ಸಾಧು ಸ್ವಭಾವದವನು. ಅವರು ಇದನ್ನು ನೋಡುವುದು ಹೀಗೆ. ನೀವು ನಮ್ಮನ್ನು ದೂರುವ ಮನುಷ್ಯರನ್ನು ತರಬೇತು ಮಾಡಿ ಕಳುಹಿಸುತ್ತೀರಿ ಆದರೆ ನಾನೇನಾದರೂ ನಿಮ್ಮನ್ನು ಸ್ವಲ್ಪ ಮುಟ್ಟಿದರೂ, ಒಳ್ಳೆಯ ಉದ್ದೇಶದಿಂದ ನಿಮಗೆ ಹೇಳಲು ಹೋದರೂ ನೀವೆಲ್ಲ ಹೀಗೆ ಕೂಗಿಕೊಳ್ಳುತ್ತೀರಿ: ‘ನೀವು ನಮ್ಮನ್ನು ಮುಟ್ಟಬೇಡಿ. ನಾವೆಲ್ಲ ಅಮೇರಿಕಾ ದೇಶದವರು, ಪ್ರಪಂಚದಲ್ಲಿ ಕ್ರೈಸ್ತರಲ್ಲದ ಇತರರನ್ನೆಲ್ಲ ನಾವು ಟೀಕಿಸುತ್ತೇವೆ, ನಿಂದಿಸುತ್ತೇವೆ. ಆದರೆ ನೀವು ನಮ್ಮನ್ನು ಮುಟ್ಟಕೂಡದು. ನಾವೆಲ್ಲ ಲಜ್ಜಾವತಿಯಂತೆ’ ಎಂದು ಹೇಳಿಕೊಳ್ಳುತ್ತೀರಿ. ನಿಮಗೆ ತೋಚಿದುದನ್ನು ನೀವು ಮಾಡಬಹುದು. ಆದರೆ ನಾವು ಎಂದಿನಂತೆಯೇ ಇರುವೆವು. ಒಂದು ದೃಷ್ಟಿಯಲ್ಲಿ ನಾವು ನಿಮಗಿಂತ ಮೇಲು. ನಾವು ನಮ್ಮ ಮಕ್ಕಳಿಗೆ ಇಲ್ಲದ ಸಲ್ಲದ ಕಥೆಗಳನ್ನೆಲ್ಲ ನಂಬಿ ಎಂದು ಹೇಳುವುದಿಲ್ಲ. ನಿಮ್ಮ ಪಾದ್ರಿಗಳು ಯಾವಾಗಲಾದರೂ ನಮ್ಮನ್ನು ಟೀಕಿಸುವಾಗ ಇದನ್ನು ಗಮನದಲ್ಲಿಡಲಿ. ಭರತಖಂಡದವರೆಲ್ಲರೂ ಎದ್ದು ನಿಂತು ಹಿಂದೂ ಸಾಗರದ ಅಡಿಯಲ್ಲಿರುವ ಕೆಸರನ್ನು ತೆಗೆದುಕೊಂಡು ಪಾಶ್ಚಾತ್ಯರಿಗೆಲ್ಲ ಬಳಿದರೂ ನೀವು ನಮಗೆ ಮಾಡುತ್ತಿರುವ ಅಪಮಾನಕ್ಕೆ ಕೋಟಿಗೆ ಒಂದು ಪಾಲೂ ಆಗುವುದಿಲ್ಲ. ಇದೆಲ್ಲ ಏತಕ್ಕೆ? ಪಾಶ್ಚಾತ್ಯ ದೇಶದ ಜನರ ಮತವನ್ನು ಬದಲಾಯಿಸಲು ನಾವು ಯಾರನ್ನಾದರೂ ಒಬ್ಬ ಸಂನ್ಯಾಸಿಯನ್ನು ಕಳುಹಿಸಿರುವೆವೆ? ನೀವು ನಿಮ್ಮ ಧರ್ಮವನ್ನು ಇಟ್ಟುಕೊಳ್ಳಿ. ನಮ್ಮದನ್ನು ನಾವು ಇಟ್ಟುಕೊಳ್ಳಲು ಬಿಡಿ. ನೀವು ನಿಮ್ಮ ಧರ್ಮವನ್ನು ಇತರರ ಮೇಲೆ ಬಿದ್ದು ಆಕ್ರಮಿಸುವ ಧರ್ಮ ಎನ್ನುತ್ತೀರಿ. ನೀವೇನೋ ಮೇಲೆ ಬಿದ್ದು ಹೋಗುತ್ತೀರಿ. ಆದರೆ ಅದರಿಂದ ಎಷ್ಟು ಜನರನ್ನು ಮತಾಂತರಗೊಳಿಸಿರುವಿರಿ? ಪ್ರಪಂಚದ ಆರನೇ ಒಂದು ಭಾಗ ಚೈನಾ. ಅದರಲ್ಲಿರುವ ಪ್ರತಿಯೊಬ್ಬರೂ ಬೌದ್ಧರು. ಇದನ್ನು ಅವರು (ಬೌದ್ಧರು) ಹೇಗೆ ಸಾಧಿಸಿದರು? ಇದಕ್ಕಾಗಿ ಅವರು ಒಂದು ಬಿಂದು ರಕ್ತವನ್ನೂ ಕೂಡ ಹರಿಸಲಿಲ್ಲ. ನೀವು ಎಷ್ಟೇ ಹೆಮ್ಮೆ ಕೊಚ್ಚಿಕೊಂಡರೂ ಬಲಾತ್ಕಾರವಿಲ್ಲದೆ ನಿಮ್ಮ ಕ್ರೈಸ್ತಧರ್ಮಎಲ್ಲಿ ಮುಂದೆ ಬಂದಿದೆ? ಅವರು ಮತಾಂತರಗೊಳ್ಳಬೇಕಾಗಿತ್ತು, ಇಲ್ಲವೆ ತಮ್ಮ ಪ್ರಾಣವನ್ನು ಕಳೆದುಕೊಳ್ಳಬೇಕಾಗಿತ್ತು. ದೇವರ ಕೃಪೆಗೆ ಪಾತ್ರರಾಗಿರುವವರು ನೀವೊಬ್ಬರೇ ಎಂದು ಭಾವಿಸುವಿರಿ. ಏಕೆಂದರೆ ನೀವು ಕೊಲ್ಲಬಲ್ಲಿರಿ. ಅರಬ್ಬಿಯವರೂ ಇದನ್ನೇ ಮಾಡಿದರು. ಇದಕ್ಕೇ ಹೆಮ್ಮೆ ಕೊಚ್ಚಿಕೊಳ್ಳುತ್ತಿದ್ದರು. ಈಗ ಅವರೆಲ್ಲಿರುವರು? ಮರಳು ಕಾಡಿನಲ್ಲಿ ಹೊಟ್ಟೆಗಿಲ್ಲದೆ ಅಲೆಯುತ್ತಿರುವರು. ರೋಮನ್ನರೂ ಕೂಡ ಇದನ್ನೇ ಹೇಳುತ್ತಿದ್ದರು. ಆದರೆ ಅವರು ಈಗ ಎಲ್ಲಿರುವರು? ಆದರೆ ನಾವು ನಮ್ಮ ಕಲ್ಲು ಬಂಡೆಯ ಮೇಲೆ ಕುಳಿತುಕೊಂಡಿರುವೆವು. ‘ಶಾಂತಿ ಸಮಾಧಾನವನ್ನು ನೀಡುವವರೇ ಧನ್ಯರು. ಅವರೇ ಭಗವಂತನ ಮಕ್ಕಳು’ ಎಂದು ಬೈಬಲ್ ಸಾರುವುದು. ಉಳಿದವರೆ ನಿರ್ನಾಮವಾಗುವರು. ಯಾವುದು ಸ್ವಾರ್ಥ ಮೂಲವೋ, ಸ್ಪರ್ಧೆಯೇ ಅದಕ್ಕೆ ನ್ಯಾಯವೋ, ಭೋಗವೇ ಅದರ ಗುರಿಯೋ ಅದು ಈಗಲೋ ಅಥವಾ ಸ್ವಲ್ಪ ಕಾಲವಾದ ಮೇಲೆ ನಾಶವಾಗಲೇಬೇಕು. ನೀವು ನಿಜವಾಗಿ ಬಾಳಬೇಕಾದರೆ ಕ್ರಿಸ್ತನ ಬಳಿಗೆ ಹೋಗಿ, ಯಾರಿಗೆ ಎಲ್ಲಿ ತಂಗುವುದಕ್ಕೆ ಸ್ಥಳವಿರಲಿಲ್ಲವೊ ಅವನ ಬಳಿಗೆ ಹೋಗಿ. ಭೋಗದ ಹೆಸರಿನಲ್ಲಿ ನಿಮ್ಮ ಧರ್ಮವನ್ನು ಬೋಧಿಸಿರುವಿರಿ. ಇದು ಎಂತಹ ವಿಚಿತ್ರ! ನೀವು ಬದುಕಬೇಕಾದರೆ ಅದನ್ನು ತಲೆ ಕೆಳಗೆ ಮಾಡಬೇಕು. ನಾನು ಈ ದೇಶದಲ್ಲಿ ಕೇಳಿರುವುದೆಲ್ಲ ಆಷಾಢಭೂತಿತನ. ನಿಮ್ಮ ಜನಾಂಗ ಬದುಕಬೇಕಾದರೆ ಅವನ ಬಳಿಗೆ ಹೋಗಿ. ದೇವರು ಮತ್ತು ಪ್ರಪಂಚ ಇವೆರಡನ್ನೂ ನೀವು ಒಟ್ಟಿಗೆ ಸೇವಿಸಲಾರಿರಿ. ಈ ಸಂಪತ್ತು ಇವುಗಳೆಲ್ಲ ಕ್ರಿಸ್ತನಿಂದ ಬಂದಿವೆ ಎಂದು ನೀವು ತಿಳಿದುಕೊಂಡಿರುವಿರೇನು? ಕ್ರಿಸ್ತ ಅವುಗಳನ್ನೆಲ್ಲ ನಿರಾಕರಿಸಿದ್ದ. ಈ ಪ್ರಪಂಚದ ಧನಕನಕಾದಿ ಐಶ್ವರ್ಯಗಳಿದ್ದರೂ ಅವನನ್ನು ಬಿಡದೆ ಇದ್ದರೆ ಅದು ಒಳ್ಳೆಯದೇ. ಆದರೆ ಯಾವಾಗ ಇದು ಸಾಧ್ಯವಿಲ್ಲವೋ ಅವನೆಡೆಗೆ ಹೋಗಿ ಭೋಗಾನ್ವೇಷಣೆಯನ್ನೆಲ್ಲ ತ್ಯಜಿಸಿ ಕ್ರಿಸ್ತನೊಡನೆ ಚಿಂದಿಯಲ್ಲಿರುವುದು ಅವನಿಲ್ಲದ ಅರಮನೆಯಲ್ಲಿರುವುದಕ್ಕಿಂತ ಮೇಲು.” 

 ಕ್ರೈಸ್ತ ಪಾದ್ರಿಗಳು ಹಿಂದೂ ಜನಾಂಗದ ಮೇಲೆ ಮಾಡುತ್ತಿದ್ದ ಅಪಪ್ರಚಾರಗಳನ್ನು ನೋಡಿ ಸ್ವಾಮೀಜಿ ಕೆರಳಿದ್ದರು. ಅದಕ್ಕೇ ಹಾಗೆ ರೋಸಿ ಮಾತನಾಡಿದರು. ಅಮೇರಿಕದ ಕ್ರೈಸ್ತರಿಂದ ಇಂಡಿಯಾ ದೇಶದಲ್ಲಿ ಕೆಲಸ ಮಾಡುವುದಕ್ಕೆ ಹಣವನ್ನು ವಸೂಲಿ ಮಾಡಲು ಭಯಾನಕವಾದ ಸುಳ್ಳನ್ನು ಪಾದ್ರಿಗಳು ಹರಡುತ್ತಿದ್ದರು. ಭರತಖಂಡದ ಮೇಲೆ ಸುಳ್ಳು ಪುಸ್ತಕಗಳನ್ನು ಬರೆದು, ಕಾಲ್ಪನಿಕ ಚಿತ್ರಗಳನ್ನು ಹಂಚಿ ಅವರು ಪ್ರಚಾರ ಮಾಡುತ್ತಿದ್ದರು. ಅದರಲ್ಲಿ ಕೆಲವು ಇವು: ಭರತಖಂಡದ ಜನ ಹೆಚ್ಚು ಮಕ್ಕಳನ್ನು ಗಂಗಾನದಿಗೆ ಬಿಸಾಡುವರು. ಗಂಗಾನದಿಯ ಕೆಲವು ಕಡೆ ಮಕ್ಕಳ ಮೂಳೆಗಳೇ ತುಂಬಿಹೋಗಿವೆ. ನದಿ ಕೂಡ ಸರಾಗವಾಗಿ ಹರಿಯುವಂತೆ ಇಲ್ಲ! ಅಲ್ಲಿ ಮೊಸಳೆಗಳ ಬಾಯಿಗೆ ಹೆಣ್ಣುಮಕ್ಕಳನ್ನು ಕೊಡುತ್ತಾರೆ. ಗಂಡನು ಬದುಕಿರುವ ತನ್ನ ಹೆಂಡತಿಯನ್ನೇ ಚಿತೆಯ ಮೇಲೆ ಬಲಾತ್ಕಾರವಾಗಿ ಸುಡುವನು. ಏಕೆಂದರೆ ಅವಳು ಅನಂತರ ಒಂದು ಪಿಶಾಚಿಯಾಗಿ ಗಂಡನ ಶತ್ರುಗಳನ್ನು ಕಾಡಲಿ ಎಂದು. ಪುರಿ ರಥದ ಕೆಳಗೆ ಸಹಸ್ರಾರು ಜನ ಬಿದ್ದು ಆತ್ಮಹತ್ಯೆಯನ್ನು ಮಾಡಿಕೊಳ್ಳುವರು. ಸ್ತ್ರೀಯನ್ನು ಗಂಡನ ಶವಕ್ಕೆ ಕಟ್ಟಿ ಬಲಾತ್ಕಾರವಾಗಿ ಸುಡುತ್ತಾರೆ ಇತ್ಯಾದಿ. ಇಂತಹ ಅಪಪ್ರಚಾರದ ಕಥೆಗಳನ್ನು ಓದಿದ ಜನ ಸ್ವಾಮೀಜಿಯವರನ್ನು ಅನೇಕ ವೇಳೆ ಆ ವಿಷಯವಾಗಿ ಪ್ರಶ್ನೆ ಕೇಳುತ್ತಿದ್ದರು. ಸ್ವಾಮೀಜಿ ಕೆಲವು ವೇಳೆ ಬಹಳ ಹಾಸ್ಯವಾಗಿ, ಕೆಲವು ವೇಳೆ ಅವರಿಗೆ ತಾಕುವಂತೆ ಉತ್ತರ ಕೊಡುತ್ತಿದ್ದರು. ಒಬ್ಬ ಹೆಂಗಸು ಇಂಡಿಯಾ ದೇಶದಲ್ಲಿ ಮಗುವನ್ನು ಗಂಗೆಗೆ ಎತ್ತಿ ಹಾಕುತ್ತಾರೆಯೇ ಎಂದು ಕೇಳಿದಾಗ, ಸ್ವಾಮೀಜಿ, ನಗುತ್ತಾ ತಮ್ಮನ್ನೂ ಅವರ ತಾಯಿ ನದಿಗೆ ಹಾಕಿದಳು, ಆದರೆ ತಾನು ಈಜಿಕೊಂಡು ಬಂದುಬಿಟ್ಟೆ ಎಂದರು. ಮೊಸಳೆ ಬಾಯಿಗೆ ಏತಕ್ಕೆ ಅಲ್ಲಿ ಜನ ಹೆಣ್ಣು ಮಕ್ಕಳನ್ನೇ ಹಾಕುತ್ತಾರೆ? ಎಂಬ ಪ್ರಶ್ನೆಗೆ ಸ್ವಾಮೀಜಿ, ಬಹುಶಃ ಹೆಣ್ಣುಮಕ್ಕಳ ಮಾಂಸ ಮೃದುವಾಗಿದೆ, ಮೊಸಳೆಗೆ ಕಷ್ಟವಾಗದೆ ಇರಲಿ ತಿನ್ನುವುದಕ್ಕೆ ಎಂದು ಇರಬಹುದು, ಎಂದರು. ಹೆಂಗಸನ್ನು ಚಿತೆಯಲ್ಲಿ ಗಂಡನೊಡನೆ ಸುಡುತ್ತಾರೆಯೇ? ಎಂಬುದಕ್ಕೆ ನಿಮ್ಮ ದೇಶದಲ್ಲಿ ಮಾಟಗಾತಿಯರನ್ನು ನೀವು ಸುಡುತ್ತಿದ್ದುದು ನಿಮಗೆ ಮರೆತುಹೋಗಿದೆ ಎಂದು ಕಾಣುತ್ತಿದೆ, ನಾವು ಹಾಗೆ ಸುಡುವುದಿಲ್ಲ, ಅದು ಧರ್ಮಕ್ಕೆ ವಿರೋಧ ಎಂದರು. ಜಗನ್ನಾಥರಥದ ಚಕ್ರದ ಕೆಳಗೆ ಸಾವಿರಾರು ಜನ ಪ್ರಾಣ ಬಿಡುತ್ತಾರೆಯೆ ಎಂಬ ಪ್ರಶ್ನೆಗೆ, ಆ ನೂಕುನುಗ್ಗಲಲ್ಲಿ ಯಾರೋ ಒಬ್ಬಿಬ್ಬರು ಸಿಕ್ಕಿಕೊಂಡು ಸತ್ತಿರಬಹುದು. ಅದನ್ನು ನಿಮ್ಮ ಪ್ರಾದ್ರಿಗಳು ವಿಪರೀತಮಾಡಿ ಹೇಳಿರುವರು ಎಂದರು. ಒಬ್ಬ ಹೆಂಗಸು ಒಂದು ಸಲ ಸ್ವಾಮೀಜಿಯವರನ್ನು ಭರತಖಂಡದಲ್ಲಿ ಪ್ರಾಯದ ಹುಡುಗಿಯರನ್ನು ಮುದುಕರಿಗೆ ಕೊಟ್ಟು ಮದುವೆ ಮಾಡಿಕೊಡುತ್ತಾರಂತೆ ಏತಕ್ಕೆ ಅಲ್ಲಿ ಮದುವೆ ಮಾಡಿಕೊಳ್ಳುವುದಕ್ಕೆ ಯುವಕರೇನಾಗಿ ಹೋಗಿರುವರು ಎಂದು ಕೇಳಿದಳು. ಅದಕ್ಕೆ ಸ್ವಾಮೀಜಿ ನಗುತ್ತ ಯುವಕರೆಲ್ಲ ಸಂನ್ಯಾಸಿಗಳಾಗಿ ಹೋಗಿರುವುದರಿಂದ ಕೊನೆಗೆ ಮುದುಕರೇ ಮದುವೆ ಮಾಡಿಕೊಳ್ಳುವುದಕ್ಕೆ ಉಳಿದುಕೊಳ್ಳುವರು ಎಂದರು. 

 ಸ್ವಾಮೀಜಿಯವರನ್ನು ಕೆಲವರು ಟೀಕಿಸುತ್ತಿದ್ದರು. ಆದರೆ ಬಹುಜನ ವಿದ್ವಾಂಸರು ಕ್ರೈಸ್ತ ಪಂಗಡದ ಪಾದ್ರಿಗಳು ಕೂಡ ಮೆಚ್ಚುತ್ತಿದ್ದರು. ಅಮೇರಿಕಾ ದೇಶದಲ್ಲಿ ಹಿಂದೂಗಳ ವಿಚಾರದಲ್ಲಿ ಹರಡಿರುವ ತಪ್ಪು ಅಭಿಪ್ರಾಯಗಳನ್ನು ಬಿಡಿಸಲು ಸ್ವಾಮೀಜಿ ಏಕಾಂಗಿಯಾಗಿ ಪ್ರಯತ್ನಮಾಡಿದರು. ಆ ಕಾಲದಲ್ಲಿ ಭರತಖಂಡ ಗುಲಾಮಗಿರಿಯಲ್ಲಿತ್ತು. ಅಂತಹ ದೇಶವನ್ನು ಹಿಂದೆ ಕಟ್ಟಿಕೊಂಡು ಅಮೇರಿಕಾ ದೇಶದಲ್ಲಿ ಅಲ್ಲಿಯ ಸಂಕುಚಿತ ಮನೋಭಿಪ್ರಾಯವುಳ್ಳವರೊಡನೆ ಹೋರಾಡಬೇಕಾದರೆ ಇದಕ್ಕೆ ಬರೀ ಮಾನವ ಸಾಹಸವೇ ಸಾಲದು. ಇದೊಂದು ಮಾನವಾತೀತ ಸಾಹಸ. ಸ್ವಾಮೀಜಿ ಇದನ್ನು ಸಾರ್ಥಕವಾಗಿಸಿದರು. ಭರತಖಂಡಕ್ಕೆ ಗೌರವ ಬರುವ ರೀತಿಯಲ್ಲಿ ಸಾಧಿಸಿದರು. ಸ್ವಾಮೀಜಿಯವರ ಉಪನ್ಯಾಸಕ್ಕೆ ಜನ ಕಿಕ್ಕಿರಿದು ಬರುತ್ತಿರುವುದನ್ನು ನೋಡಿ ಉಪನ್ಯಾಸವನ್ನು ಅಣಿಮಾಡುವ ಒಂದು ಸಂಸ್ಥೆ ಸ್ವಾಮೀಜಿಯವರನ್ನು ಅಮೇರಿಕಾ ದೇಶದಲ್ಲಿ ಉಪನ್ಯಾಸ ಮಾಡುವುದಕ್ಕೆ ಕರಾರು ಬರೆಸಿಕೊಂಡಿತು. ಉಪನ್ಯಾಸದಿಂದ ಸಂಗ್ರಹವಾಗುವ ಹಣದಲ್ಲಿ ಒಂದು ಪಾಲನ್ನು ಸ್ವಾಮಿಗಳಿಗೆ ಕೊಡುವಂತೆ ಒಪ್ಪಿಗೆಯಾಯಿತು. ಅವರ ಉಪನ್ಯಾಸಕ್ಕೆ ಜನ ಬೇಕಾದಷ್ಟು ಬರುತ್ತಿದ್ದರೂ, ಟಿಕೇಟೆಲ್ಲ ಮುಗಿದುಹೋಗಿವೆ ಎಂದು ಹೇಳುತ್ತಿದ್ದರೂ, ಸ್ವಾಮೀಜಿ ಪಾಲಿಗೆ ಬರುತ್ತಿದ್ದ ದುಡ್ಡು ಬಹಳ ಅಲ್ಪವಾಗಿತ್ತು. ಹಣದ ಬಹುಪಾಲನ್ನು ಆ ಸಂಸ್ಥೆಯವರು ತಿಂದುಹಾಕಿ ಬಿಡುತ್ತಿದ್ದರು. ಸ್ವಾಮೀಜಿ ಕೆಲವು ಕಾಲ ಅದರ ಪರವಾಗಿ ಉಪನ್ಯಾಸಮಾಡಿ ಅನಂತರ ಆ ಕರಾರಿನಿಂದ ಬಿಡಿಸಿಕೊಂಡರು. ಅದರಿಂದ ಸ್ವಾಮೀಜಿಯವರಿಗೆ ಆರ್ಥಿಕದೃಷ್ಟಿಯಿಂದ ನಷ್ಟವಾದರೂ ತಾನು ಸ್ವತಂತ್ರವಾಗಿ ಉಪನ್ಯಾಸ ಮಾಡುತ್ತೇನೆ ಎಂದು ಅದರ ಪಾಶದಿಂದ ತಪ್ಪಿಸಿಕೊಡರು. 

