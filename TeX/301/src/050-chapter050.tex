
\chapter{ಅಮೇರಿಕಾದಲ್ಲಿ}

 ಸ್ವಾಮೀಜಿ ಜುಲೈ ೩೧ನೇ ತಾರೀಖು ಲಂಡನ್ನಿಗೆ ಬಂದರು. ಅನೇಕ ಹಳೆಯ ಸ್ನೇಹಿತರು ಅವರನ್ನು ಸ್ವಾಗತಿಸಿದರು. ಅಮೇರಿಕಾದ ಡೆಟ್ರಾಯಿಟ್‍ನಿಂದ ಸ್ವಾಮೀಜಿಯವರನ್ನು ನೋಡಲು ಇಬ್ಬರು ಸ್ತ್ರೀಯರು ಕೂಡ ಬಂದಿದ್ದರು. ಸ್ವಾಮೀಜಿ ವಿಂಬಲ್‍ಡನ್ನಿನಲ್ಲಿ ಎರಡು ವಾರಗಳನ್ನು ಮಾತುಕತೆಗಳಲ್ಲಿ ಕಳೆದರು. ಬಹಿರಂಗ ಉಪನ್ಯಾಸವನ್ನು ಮಾತ್ರ ಕೊಡಲಿಲ್ಲ. ಆಗಸ್ಟ್ ೧೬ನೇ ತಾರೀಖು ತಮ್ಮ ಗುರುಭಾಯಿ ಮತ್ತು ಅಮೇರಿಕಾದಿಂದ ಬಂದ ಶಿಷ್ಯರೊಡನೆ ಅಮೇರಿಕಾಕ್ಕೆ ಹೊರಟರು. ಸೋದರಿ ನಿವೇದಿತಾ ತತ್ಕಾಲದಲ್ಲಿ ಇಂಗ್ಲೆಂಡಿನಲ್ಲಿಯೇ ಉಳಿದಳು. ಸಮುದ್ರ ಪ್ರಶಾಂತವಾಗಿತ್ತು. ರಾತ್ರಿಯ ಹೊತ್ತು ಬೆಳುದಿಂಗಳಿಗೆ ನಲಿಯುತ್ತಿದ್ದ ಸಾಗರವನ್ನು ನೋಡಿ, “ಈ ಮಾಯೆಯೇ ಇಷ್ಟು ಚೆನ್ನಾಗಿದ್ದರೆ ಇದರ ಹಿಂದೆ ಇರುವ ಸತ್ಯ ಇನ್ನೆಷ್ಟು ಚೆನ್ನಾಗಿರಬೇಕು!” ಎಂದರು ಸ್ವಾಮೀಜಿ. ಮತ್ತೊಂದು ದಿನ ಹುಣ್ಣಿಮೆಯ ದಿನ ಸೂರ್ಯ ಮುಳುಗಿ ಪೂರ್ಣಚಂದ್ರ ಉದಯಿಸುತ್ತಿರುವಾಗ ಸ್ತಬ್ಧರಾಗಿ ಕೆಲವು ಕಾಲ ಅದನ್ನೇ ನೋಡುತ್ತ ಅನಂತರ ಸುತ್ತಲಿದ್ದವರಿಗೆ, “ಕವನವನ್ನು ಓದುವುದೇಕೆ, (ಆಕಾಶ ಮತ್ತು ಸಾಗರವನ್ನು ತೋರುತ್ತ) ಕಾವ್ಯದ ಸಾರವೇ ಸುತ್ತಲೂ ಇರುವಾಗ?” ಎಂದರು. 

 ಸ್ವಾಮೀಜಿ ನ್ಯೂಯಾರ್ಕ್ ಮುಟ್ಟಿದ ದಿನ ಮಧ್ಯಾಹ್ನವೇ ಅಲ್ಲಿಂದ ಸುಮಾರು ನೂರು ಐವತ್ತು ಮೈಲಿ ದೂರದಲ್ಲಿ ಹಡ್‍ಸನ್ ನದೀ ತೀರದಲ್ಲಿ ರಿಡ್ಜ್ ಲೆ ಮೇನರ್ ಎಂಬ ಗ್ರಾಮದ ಮನೆಗೆ ಲೆಗೆಟ್ ದಂಪತಿಗಳೊಡನೆ ಹೊರಟರು. ಅಲ್ಲಿಗೆ ಒಂದು ತಿಂಗಳಾದ ಮೇಲೆ ಸೋದರಿ ನಿವೇದಿತಾ ಇಂಗ್ಲೆಂಡಿನಿಂದ ಬಂದಳು. ಸ್ವಾಮೀಜಿ ಅವಳು ಈ ಹಳ್ಳಿಯ ಮನೆಯಲ್ಲಿ ನವೆಂಬರ್ ೫ನೇ ತಾರೀಖಿನವರೆಗೆ ಇದ್ದರು. ಅಮೇರಿಕಾದಲ್ಲಿ ಉಪನ್ಯಾಸ ಮಾಡುತ್ತಿದ್ದ ಅಭೇದಾನಂದರಿಗೂ ಇಲ್ಲಿಗೆ ಬರುವಂತೆ ಕಾಗದ ಬರೆದು ಅವರನ್ನು ಕರೆಸಿಕೊಂಡರು. ಅಭೇದಾನಂದರು ಅಲ್ಲಿ ಸುಮಾರು ಹತ್ತು ದಿನಗಳಿದ್ದು ವೇದಾಂತ ಪ್ರಚಾರದ ಕಾರ‍್ಯ ಹೇಗೆ ಹಬ್ಬುತ್ತಿದೆ ಎಂಬುದನ್ನು ವಿವರಿಸಿದರು. ಈಗ ಆ ಸಂಘಕ್ಕೆ ನ್ಯೂಯಾರ್ಕಿನಲ್ಲಿ ತನ್ನದೇ ಆದ ಕಟ್ಟಡವಿತ್ತು. 

 ಸ್ವಾಮಿ ತುರಿಯಾನಂದರನ್ನು ನ್ಯೂಯಾರ್ಕಿನ ಹತ್ತಿರ ಇರುವ ಮೌಂಟ್‍ಕ್ಲೇರ್ ಎಂಬಲ್ಲಿಗೆ ಉಪನ್ಯಾಸ ಮಾಡಲು ಕಳುಹಿಸಿದರು. ವೇದಾಂತ ಸೊಸೈಟಿಯಲ್ಲಿಯೂ ಅಭೇದಾನಂದರ ಕೆಲಸಗಳೊಡನೆ ಅವರು ಭಾಗಿಯಾದರು. ಅನಂತರ ಕೇಂಬ್ರಿಡ್ಜ್ ಮಸಾಚುಸೆಟ್ಸ್ ಅಲ್ಲಿಗೆ ಹೋದರು. ಡಿಸೆಂಬರ್ ೧೦ನೇ ತಾರೀಖು ಕೇಂಬ್ರಿಡ್ಜ್ ಕಾನ್ಫರೆನ್ಸಿನಲ್ಲಿ ಶಂಕರಾಚಾರ‍್ಯರ ಮೇಲೆ ಒಂದು ಲೇಖನವನ್ನು ಓದಿದರು. ಅದನ್ನು ಹಾರ್‍ವರ್ಡ್ ವಿಶ್ವವಿದ್ಯಾನಿಲಯದ ಪ್ರಾಧ್ಯಾಪಕರೆಲ್ಲಾ ಬಹಳ ಮೆಚ್ಚಿದರು. 

 ಸ್ವಾಮಿ ವಿವೇಕಾನಂದರು ನ್ಯೂಯಾರ್ಕಿನಲ್ಲಿ ನವೆಂಬರ್ ೮ನೇ ತಾರೀಖು ಒಂದು ಪ್ರಶ್ನೋತ್ತರ ಮಾಲಿಕೆಯಲ್ಲಿ ಭಾಗಿಗಳಾದರು. ೧೦ನೇ ತರೀಖು ನ್ಯೂಯಾರ್ಕಿನ ವೇದಾಂತ ಸಂಘದವರು ಸ್ವಾಮೀಜಿಗೆ ಒಂದು ಸತ್ಕಾರವನ್ನು ಏರ್ಪಡಿಸಿದ್ದರು. ಅಲ್ಲಿಗೆ ಸ್ವಾಮೀಜಿಯವರ ಹಿಂದಿನ ಅನೇಕ ಸ್ನೇಹಿತರು ಬಂದಿದ್ದರು. ಸ್ವಾಮೀಜಿ ಅಮೇರಿಕಾದಲ್ಲಿ ಪ್ರಕಟಪಡಿಸಿದ ಪುಸ್ತಕಗಳನ್ನು ಓದಿ ಅನೇಕರು ಈಗ ಅವರೆಡೆಗೆ ಆಕರ್ಷಿತರಾದರು. ಅವರು ನ್ಯೂಯಾರ್ಕಿನಲ್ಲಿ ಹದಿನೈದು ದಿನಗಳಿದ್ದು ಅನಂತರ ಕ್ಯಾಲಿಫೋರ್ನಿಯಾಕ್ಕೆ ಹೊರಟರು. ಮಧ್ಯದಲ್ಲಿ ಹಲವು ಜನ ಸ್ನೇಹಿತರು ಕೋರಿಕೊಂಡುದರಿಂದ ಚಿಕಾಗೋ ನಗರದಲ್ಲಿ ಕೆಲವು ದಿನಗಳಿದ್ದು ತಮ್ಮ ಹಳೆಯ ಸ್ನೇಹಿತರನ್ನು ನೋಡಿದರು. ಇವರಿಗಾಗಿ ಏರ್ಪಡಿಸಿದ ಹಲವು ಸತ್ಕಾರ ಕೂಟಗಳಲ್ಲಿ ಭಾಗವಹಿಸಿದರು. ಅನಂತರ ಸ್ವಾಮೀಜಿ ಕ್ಯಾಲಿಫೋರ್ನಿಯಾದಿಂದ ಲಾಸ್‍ಏಂಜಲೀಸ್ ಎಂಬಲ್ಲಿಗೆ ಹೋಗಿ ಶ‍್ರೀಮತಿ ಬ್ಲಾಡ್‍ಗೆಟ್ ಎಂಬುವರ ಅತಿಥಿಗಳಾಗಿದ್ದರು. ಆ ಸಮಯದಲ್ಲಿ ಸ್ವಾಮೀಜಿ ಹಲವು ಉಪನ್ಯಾಸಗಳನ್ನು ಕೊಟ್ಟರು. ವೇದಾಂತ ತತ್ತ್ವ, ಬ್ರಹ್ಮಾಂಡ, ಕರ್ಮ ಮತ್ತು ಅದರ ರಹಸ್ಯ, ಮನಸ್ಸಿನ ಶಕ್ತಿ, ತೆರೆದ ರಹಸ್ಯ ಮುಂತಾದ ವಿಷಯಗಳ ಮೇಲೆ ಮಾತನಾಡಿದರು. ಹತ್ತಿರದಲ್ಲಿರುವ ಪಾಸದೀನ ಎಂಬಲ್ಲಿಯೂ ದೇವದೂತ ಏಸುಕ್ರಿಸ್ತ, ವಿಶ್ವಧರ್ಮ ಸಾಕ್ಷಾತ್ಕಾರ ಮುಂತಾದ ವಿಷಯಗಳ ಮೇಲೆ ಮಾತನಾಡಿದರು. ಇದಲ್ಲದೆ ರಾಮಾಯಣ ಮಹಾಭಾರತದ ಕಥೆಗಳ ಮೇಲೂ ಮಾತನಾಡಿದರು. 

 ಲಾಸ್ ಏಂಜಲೀಸ್‍ನ \enginline{Home of Truth} ಎಂಬ ಸಂಘದಿಂದ ನಿಮಂತ್ರಣ ಬಂದಿತು. ಸ್ವಾಮೀಜಿ ಅಲ್ಲಿ ಸುಮಾರು ಒಂದು ತಿಂಗಳಿದ್ದು ಪ್ರವಚನಗಳನ್ನು ಕೊಟ್ಟರು. ಪ್ರತಿದಿನವೂ ಒಂದು ಸಾವಿರ ಮಂದಿಯವರೆಗೂ ಸೇರುತ್ತಿದ್ದರು. ಸ್ವಾಮೀಜಿ ಇಲ್ಲಿ ಕೆಲವು ಕಾಲ ಮಿಸ್ ಸ್ಪೆನ್‍ಸರ್ ಎಂಬುವರ ಅತಿಥಿಗಳಾಗಿದ್ದರು. ಅವರ ಮನೆಯಲ್ಲಿದ್ದಾಗ ಮಿಸ್ ಸ್ಪೆನ್‍ಸರ್ ಅವರ ಅಂಧಳಾದ ತಾಯಿ ತನ್ನ ಕೊನೆಯ ದಿನಗಳಲ್ಲಿ ಇದ್ದಳು. ಸ್ವಾಮೀಜಿ ಅವಕಾಶವಾದಾಗಲೆಲ್ಲ ವೃದ್ಧ ತಾಯಿಯ ಸಮೀಪದಲ್ಲಿ ಅವರಿಗೆ ಸಮಾಧಾನ ಹೇಳುತ್ತಿದ್ದರು. 

 ಸ್ವಾಮೀಜಿ ಲಾಸ್ ಏಂಜಲೀಸ್‍ನಿಂದ ಓಕ್‍ಲೆಂಡಿಗೆ ಹೋದರು. ಅಲ್ಲಿ ರೆವರೆಂಡ್ ಡಾಕ್ಟರ್ ಬೆಂಜಮಿನ್ ಫೆ ಎಂಬುವರ ಅತಿಥಿಗಳಾಗಿದ್ದರು. ಅಲ್ಲಿಯ ಚರ್ಚಿನಲ್ಲಿ ಎಂಟು ಉಪನ್ಯಾಸಗಳನ್ನು ಕೊಟ್ಟರು. ಆ ಉಪನ್ಯಾಸಕ್ಕೆ ಸಹಸ್ರಾರು ಜನ ಬರುತ್ತಿದ್ದರು. ಅವರ ಉಪನ್ಯಾಸಗಳ ಸಾರಾಂಶವನ್ನೆಲ್ಲ ವೃತ್ತಪತ್ರಿಕೆಗಳು ಕೊಡುತ್ತಿದ್ದವು. ಅನಂತರ ಹಲವು ಕೋರಿಕೆಯ ಮೇಲೆ ಸ್ಯಾನ್‍ಫ್ರಾನ್ಸಿಸ್‍ಕೋಗೆ ಹೋದರು. ಅಲ್ಲಿ ಬಹಿರಂಗ ಉಪನ್ಯಾಸವನ್ನು ಕೊಟ್ಟರು. ಬೆಳಗಿನ ಹೊತ್ತು ರಾಜಯೋಗ ಮತ್ತು ಧ್ಯಾನದ ಮೇಲೆ ಪ್ರವಚಾನಾದಿಗಳನ್ನು ತೆಗೆದುಕೊಂಡರು. ಅಲ್ಲಿ ಮಾರ್ಚ್ ಮತ್ತು ಏಪ್ರಿಲ್ ತಿಂಗಳುಗಳಲ್ಲಿ ಪ್ರತಿ ಭಾನುವಾರವೂ ಬಹಿರಂಗ ಉಪನ್ಯಾಸಗಳನ್ನು ಕೂಡ ಕೊಟ್ಟರು. ಇನ್ನೂ ಹಲವು ಕಡೆ ಉಪನ್ಯಾಸಗಳನ್ನು ಕೊಟ್ಟು ಜನರಲ್ಲಿ ವೇದಾಂತದ ಮೇಲೆ ಆಸಕ್ತಿಯನ್ನು ಕೆರಳಿಸಿದರು. 

 ಸ್ವಾಮೀಜಿ ಕ್ಯಾಲಿಫೋರ್ನಿಯಾ ಬಿಡುವುದಕ್ಕೆ ಮುಂಚೆ ಅವರ ಶಿಷ್ಯೆಯಾದ ಮಿಸ್ ಮಿನ್ನಿಸಿ ಬುಕ್ ಎಂಬಾಕೆ ವೇದಾಂತದ ಆಶ್ರಮಕ್ಕೆಂದು ಊರಿಗೆ ದೂರದಲ್ಲಿ ಸುಮಾರು ೧೬೦ ಎಕರೆಗಳಷ್ಟು ವಿಸ್ತಾರವಾದ ಸ್ಥಳವನ್ನು ಕೊಟ್ಟರು. ಸ್ವಾಮೀಜಿ ಅದಕ್ಕೆ ಶಾಂತಿ ಆಶ್ರಮ ಎಂದು ಹೆಸರಿಟ್ಟು ತುರಿಯಾನಂದರನ್ನು ಆಗಸ್ಟ್ ೨ನೇ ತಾರೀಖು ಅಲ್ಲಿಗೆ ಕಳುಹಿಸಿದರು. ಸ್ವಾಮೀಜಿಗೆ ಇಷ್ಟು ಹೊತ್ತಿಗೆ ಮಾತನಾಡಿ ಸಾಕಾಗಿಹೋಗಿತ್ತು. ಅವರಿಗೆ ಕರ್ಮವೇ ಬೇಸರವಾಗಿತ್ತು. ತಾವು ಗಂಟುಮೂಟೆ ಕಟ್ಟಿಕೊಂಡು ಬಂದೆಡೆಗೆ ಹೋಗಬೇಕೆಂದು ಹಾತೊರೆಯುವಂತೆ ಇದೆ, ಅವರು ಆ ಸಮಯದಲ್ಲಿ ಬರೆದ ಒಂದು ಪತ್ರ. ಅಲ್ಲಿ ಹೀಗೆ ಹೇಳುವರು: 

 “ಕೆಲಸ ಯಾವಾಗಲೂ ಕಷ್ಟ. ನನ್ನ ಕೆಲಸ ಸಂಪೂರ್ಣವಾಗಿ ನಿಂತುಹೋಗಲಿ, ನನ್ನ ಮನಸ್ಸು ಜಗನ್ಮಯಿಯಲ್ಲಿ ತಲ್ಲೀನವಾಗಲೆಂದು, ಜೋ, ನನಗಾಗಿ ಪ್ರಾರ್ಥಿಸು. ಅವಳ ಕೆಲಸ ಅವಳಿಗೆ ಗೊತ್ತಿದೆ. ಲಂಡನ್ನಿನಲ್ಲಿ ನೀನು ಮತ್ತೊಮ್ಮೆ ಇರುವುದಕ್ಕೆ ಸಂತೋಷಿಸಬಹುದು. ಹಳೆಯ ಸ್ನೇಹಿತರುಗಳು - ಅವರೆಲ್ಲರಿಗೂ ನನ್ನ ಪ್ರೀತಿಯನ್ನು ಕೃತಜ್ಞತೆಯನ್ನು ಹೇಳು. ನಾನು ಆರೋಗ್ಯದಿಂದ ಇರುವೆನು. ಮನಸ್ಸೂ ಕೂಡ ಸಮಾಧಾನದಲ್ಲಿರುವುದು. ದೇಹದ ಶಾಂತಿಗಿಂತ ಹೆಚ್ಚಾಗಿ ಆತ್ಮದ ಶಾಂತಿಯೇ ತೋರುತ್ತಿದೆ. ಕದನಗಳಲ್ಲಿ ಸೋಲುವೆನು ಮತ್ತು ಗೆಲ್ಲುವೆನು. ನನ್ನ ಸಾಮಾನನ್ನು ಮೂಟೆಕಟ್ಟಿ ಇಟ್ಟಿರುವೆನು. ಆ ಪರಮ ಬಂಧನವಿಮೋಚಕನಿಗಾಗಿ ಕಾಯುತ್ತಿರುವೆನು.” 

 “ಶಿವ, ಶಿವ, ಸಾಗಿಸು ಆಚೆಯ ತೀರಕ್ಕೆ ನನ್ನ ದೋಣಿಯನ್ನು.” 

 “ಏನಾದರೂ ಆಗಲಿ ಜೋ, ದಕ್ಷಿಣೇಶ್ವರದ ಆಲದ ಮರದ ಕೆಳಗೆ\break ಶ‍್ರೀರಾಮಕೃಷ್ಣರ ಅದ್ಭುತ ಉಪದೇಶಾಮೃತವನ್ನು ಆಶ್ಚರ‍್ಯಚಕಿತನಾಗಿ ಕೇಳುತ್ತಿದ್ದ ಬಾಲಕನೇ ಇನ್ನೂ ನಾನು. ಅದು ನನ್ನ ಸಹಜಸ್ವಭಾವ. ಕೆಲಸ ಪರೋಪಕಾರ ಮುಂತಾದುವುಗಳೆಲ್ಲ ತೋರಿಕೆಗೆ ಮಾತ್ರ. ಈಗ ಪುನಃ ಅವರ ವಾಣಿಯನ್ನು ಕೇಳುತ್ತಿರುವೆನು. ನನ್ನ ಆತ್ಮೋದ್ದೀಪನ ಮಾಡುತ್ತಿದ್ದ ಅದೇ ಆ ಹಳೆಯ ವಾಣಿ, ಬಂಧನಗಳು ಕಳಚಿ ಬೀಳುತ್ತಿವೆ. ಪ್ರೇಮ ಕುಗ್ಗುತ್ತಿದೆ. ಕೆಲಸ ಅರುಚಿಯಾಗುತ್ತಿದೆ - ಜೀವನದ ಆಕರ್ಷಣೆ ಹೋಯಿತು. ಈಗ ಗುರುದೇವನ ವಾಣಿಯೊಂದೇ ಕರೆಯುತ್ತಿದೆ. ಬರುತ್ತೇನೆ, ಹೇ ಗುರುದೇವ! ಬರುತ್ತೇನೆ ಹೇ ಗುರುದೇವ! - ‘ಸತ್ತವರು ಸತ್ತವರನ್ನು ಮಣ್ಣುಪಾಲು ಮಾಡಲಿ- ನೀನು ನನ್ನನ್ನು ಹಿಂಬಾಲಿಸು.’ ಹೇ ನನ್ನ ಪ್ರೇಮೇಶ್ವರನೇ ನಾನು ಬರುತ್ತೇನೆ, ನಾನು ಬರುತ್ತೇನೆ! ಹೌದು ನಾನು ಬರುತ್ತೇನೆ. ನಿರ್ವಾಣ ನನ್ನ ಮುಂದಿದೆ. ಕೆಲವುವೇಳೆ ನಾನದನ್ನು ಅನುಭವಿಸುತ್ತೇನೆ. ಸ್ವಲ್ಪವೂ ಕದಲದ ಉಸಿರಾಡದ ಅದೇ ಅನಂತಸಾಗರ.” 

 “ನಾನು ಹುಟ್ಟಿದೆ, ಸುಖ, ಕಷ್ಟಗಳನ್ನು ಅನುಭವಿಸಿದೆ, ಸಂತೋಷ. ದೊಡ್ಡ ತಪ್ಪುಗಳನ್ನು ಮಾಡಿದೆ, ಸಂತೋಷ. ಶಾಂತಿಯಿಂದ ಹೊರಡುವುದಕ್ಕೆ ಸಂತೋಷ. ಯಾರನ್ನೂ ನಾನು ಬಂಧಿಸಿ ಹೋಗುವುದಿಲ್ಲ. ನಾನು ಬಂಧನಗಳನ್ನೂ ತೆಗೆದುಕೊಂಡು ಹೋಗುವುದಿಲ್ಲ. ಈ ದೇಹ ಪತನವಾಗಿ ನನ್ನನ್ನು ಬಿಡುಗಡೆ ಮಾಡುವುದೋ ಅಥವಾ ನಾನು ಜೀವನ್ಮುಕ್ತಿಯನ್ನು ಪಡೆಯುವೆನೋ? ಹಳೆಯ ಮನುಷ್ಯನು ಹೋದ. ಎಂದೋ ಹೋದ. ಪುನಃ ಹಿಂದಿರುಗುವುದಿಲ್ಲ.” 

 “ಮಾರ್ಗದರ್ಶನ, ಗುರು, ನಾಯಕ, ಬೋಧಕ ಇವು ಹೊರಟುಹೋದವು. ಹುಡುಗ; ವಿದ್ಯಾರ್ಥಿ, ಸೇವಕಮಾತ್ರ ಹಿಂದೆ ಉಳಿದಿದೆ.” 

 “ಜೋ! ನಾನು ಏತಕ್ಕೆ ಅವರ ತಂಟೆಗೆ ಹೋಗುವುದಿಲ್ಲ ಗೊತ್ತಾಯಿತೆ? ಮತ್ತೊಬ್ಬರ ಕೆಲಸದಲ್ಲಿ ಕೈಹಾಕುವುದಕ್ಕೆ ನಾನು ಯಾರು? ನಾಯಕನ ಸ್ಥಾನವನ್ನು ಎಂದೋ ತ್ಯಜಿಸಿದೆ. ಈಗ ನನಗೆ ಮಾತನಾಡುವುದಕ್ಕೆ ಅಧಿಕಾರವಿಲ್ಲ. ಈ ವರ್ಷದ ಆದಿಯಿಂದ ಇಂಡಿಯಾದೇಶದಲ್ಲಿ ನಾನು ಏನನ್ನೂ ಹೇಳಿ ಮಾಡಿಸಲಿಲ್ಲ. ಅದು ನಿಮಗೆ ಗೊತ್ತಿದೆ. ಹಿಂದೆ ನೀನು ಮತ್ತು ಶ‍್ರೀಮತಿ ಬ-ಇವರುಗಳು ತೋರಿದ ಅನೇಕ ಪ್ರೀತಿಗೆ ವಂದನೆಗಳು. ಅನವರತವೂ ಎಲ್ಲಾ ಆಶೀರ್ವಾದಗಳೂ ನಿನಗೆ. ನನ್ನ ಜೀವನದ ಅತ್ಯಂತ ಮಧುರವಾದ ಸಮಯವೇ ನಾನು ಏನೂ ಕೆಲಸಮಾಡದೆ ಸುಮ್ಮನೆ ತೇಲುತ್ತಿದ್ದಾಗ. ಪುನಃ ನಾನೀಗ ತೇಲುತ್ತಿರುವೆನು. ಮೇಲೆ ಬೆಳಗುವ ಬೆಚ್ಚಗಿರುವ ಸೂರ‍್ಯ. ಸುತ್ತಲೂ ನಾನಾವಿಧದ ತರುಲತೆಗಳು, ಬಿಸಿಲಲ್ಲಿ ಎಲ್ಲವೂ ನಿಶ್ಯಬ್ದವಾಗಿದೆ, ಶಾಂತವಾಗಿದೆ - ನಾನು ತೇಲುತ್ತಿರುವೆನು, ಮಂದವಾಗಿ ನದಿಯ ವಕ್ಷಸ್ಥಳದಲ್ಲಿ ತೇಲುತ್ತಿರವೆನು. ನನ್ನ ಕೈಕಾಲುಗಳಿಂದ ನೀರನ್ನು ಅಲ್ಲಾಡಿಸಿ ಸ್ವಲ್ಪವೂ ಶಬ್ದಮಾಡುವುದಕ್ಕೆ ನನಗೆ ಇಷ್ಟವಿಲ್ಲ. ಅದ್ಭುತವಾದ ಮೌನವೆಲ್ಲಿ ಭಗ್ನವಾಗುವುದೋ ಎಂಬ ಅಂಜಿಕೆ. ಇದು ನಿಜವಾಗಿಯೂ ಒಂದು ಕನಸು ಎಂಬ ಭಾವವನ್ನು ಹೆಚ್ಚಿಸುವ ಮೌನ.” 

 “ನನ್ನ ಕೆಲಸ ಹಿಂದೆ ಆಸೆ ಇತ್ತು. ನನ್ನ ಪ್ರೀತಿಯ ಹಿಂದೆ ವ್ಯಕ್ತಿತ್ವವಿತ್ತು. ನನ್ನ ಪವಿತ್ರತೆಯ ಹಿಂದೆ ಅಂಜಿಕೆ ಇತ್ತು. ನನ್ನ ನಾಯಕತ್ವದ ಹಿಂದೆ ಅಧಿಕಾರದ ಆಸೆ ಇತ್ತು. ಈಗ ಅವುಗಳೆಲ್ಲ ಮಾಯವಾಗುತ್ತಿವೆ. ನಾನು ಸುಮ್ಮನೆ ಸಿಕ್ಕಿದ ಕಡೆ ತೇಲುತ್ತೇನೆ. ನಾನು ಬರುತ್ತೇನೆ. ನಿನ್ನ ಪ್ರೇಮಾಲಿಂಗನದಲ್ಲಿ ತೇಲುತ್ತ ನೀನು ಬಯಸಿದ ಕಡೆ ಕರೆದೊಯ್ಯಿ. ನಿಶ್ಯಬ್ದವಾದ ಆಶ್ಚರ‍್ಯಕರವಾದ, ಕಾತುರದ ಲೋಕಕ್ಕೆ ನಾನು ಬರುತ್ತೇನೆ. ಸಭಿಕನಾಗಿ, ಮತ್ತೆಂದಿಗೂ ಪಾತ್ರಧಾರಿಯಾಗಿ ಅಲ್ಲ.” 

 “ಓಂ! ಎಷ್ಟು ಶಾಂತವಾಗಿದೆ! ನನ್ನ ಆಲೋಚನೆಗಳು ಎಲ್ಲೋ ದೂರದಿಂದ, ಬಹಳ ದೂರದಿಂದ ನನ್ನ ಹೃದಯಾಂತರಾಳದಿಂದ ಬರುವಂತೆ ತೋರುತ್ತಿದೆ. ಅಸ್ಪಷ್ಟವಾದಂತೆ ದೂರದಿಂದ ಬರುವ ಪಿಸುಮಾತಿನಂತಿದೆ. ಮಧುರ, ಮಧುರ ಶಾಂತಿ ಎಲ್ಲೆಲ್ಲಿಯೂ. ಇದರಂತೆ ನಿದ್ರಾ ಪ್ರಪಂಚಕ್ಕೆ ಜಾರಿಹೋಗುವಾಗ ವಿಷಯಗಳನ್ನು ನೋಡಿ ನೆರಳಿನಂತೆ ತೋರುವ ಕೆಲವು ಕ್ಷಣಗಳು ಕಾಣುವುವು- ಅಂಜಿಕೆ ಇಲ್ಲ - ಪ್ರೀತಿ ಇಲ್ಲ, ಉದ್ವೇಗವಿಲ್ಲ. ಮೂಕ ವಿಗ್ರಹ, ಚಿತ್ರಪಟ, ಇವುಗಳ ಮಧ್ಯದಲ್ಲಿ ಒಬ್ಬನಿರುವಾಗ ತೋರುವ ಶಾಂತಿ- ನಾನು ಬರುತ್ತೇನೆ.” 

 “ಪ್ರಪಂಚವಿದೆ. ಆದರೆ ಸುಂದರವಾಗಿಯೂ ಇಲ್ಲ, ಕುರೂಪವಾಗಿಯೂ ಇಲ್ಲ. ಭಾವೋದ್ರೇಕವನ್ನು ಮಾಡಿದ ಇಂದ್ರಿಯ ಜ್ಞಾನದಂತೆ \enginline{(Sensation)}. ಓ! ಜೋ! ಅದರ ಆನಂದ! ಸರ್ವವೂ ಒಳ್ಳೆಯದಾಗಿರುವುದು, ಸುಂದರವಾಗಿರುವುದು. ಏಕೆಂದರೆ ನನಗೆ ವಸ್ತುಗಳ ಮೇಲೆ ಆಸಕ್ತಿ ತಗ್ಗುತ್ತಿದೆ- ಇವುಗಳಲ್ಲಿ ನನ್ನ ದೇಹವೇ ಮೊದಲನೆಯದು - ಓಂ ತತ್‍ಸತ್.” 

 ಕ್ಯಾಲಿಫೋರ್ನಿಯಾ ಪ್ರದೇಶದಲ್ಲಿ ಸ್ವಾಮೀಜಿ ಬಹಳ ಕಷ್ಟಪಟ್ಟು ಕೆಲಸ ಮಾಡಿದರು. ಅಲ್ಲಿ ಸುಮಾರು ನೂರು ಉಪನ್ಯಾಸಗಳನ್ನು ಕೊಟ್ಟರು. ಇದಲ್ಲದೆ ವ್ಯಕ್ತಿಗಳಿಗೆ ಕೊಟ್ಟ ಭೇಟಿಗಳು, ಇವುಗಳಿಂದ ಅವರ ದಿನ ಬಿಡುವಿಲ್ಲದೆ ಇತ್ತು. ಲೆಗೆಟ್ ದಂಪತಿಗಳು ಆಗ ಲಂಡನ್ನಿನಲ್ಲಿ ಇದ್ದರು. ಅವರು ಸ್ವಾಮೀಜಿಯವರಿಗೆ ಜುಲೈ ಹೊತ್ತಿಗೆ ಪ್ಯಾರಿಸ್ಸಿಗೆ ಬರಬೇಕೆಂದು ಕೋರಿಕೊಂಡು ಒಂದು ಪತ್ರವನ್ನು ಬರೆದರು. ಅದಕ್ಕಾಗಿ ಸ್ವಾಮೀಜಿ ನ್ಯೂಯಾರ್ಕಿಗೆ ಬಂದರು. ಜುಲೈ ೨೦ನೇ ತಾರೀಖು ಸ್ವಾಮೀಜಿ ಪ್ಯಾರಿಸ್ಸಿಗೆ ತೆರಳಿದರು. 

