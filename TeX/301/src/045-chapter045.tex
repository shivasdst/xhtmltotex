
\chapter{ಸೂರ‍್ಯಗ್ರಹಣದ ದಿನ }

ಸ್ವಾಮೀಜಿ ಬಾಗ್‍ಬಜಾರಿನ ಬಲರಾಮ ವಸುಗಳ ಮನೆಯಲ್ಲಿ ವಾಸಮಾಡುತ್ತಿರುವರು. ಇಂದು ಪೂರ್ಣ ಸೂರ‍್ಯಗ್ರಹಣದ ದಿನ. ಜ್ಯೋತಿಷ್ಯರು ಗ್ರಹಗಳನ್ನು ನೋಡುವುದಕ್ಕೆ ನಾನಾ ಸ್ಥಳಗಳಿಗೆ ಹೋಗಿದ್ದಾರೆ, ಆಚಾರಶೀಲರಾದ ಸ್ತ್ರೀ, ಪುರುಷರು ಗಂಗಾಸ್ನಾನಕ್ಕಾಗಿ ಬಹುದೂರದಿಂದ ಬಂದು ಉತ್ಸುಕರಾಗಿ ಗ್ರಹಣವಾಗುವ ಕಾಲವನ್ನು ಎದುರು ನೋಡುತ್ತಿದ್ದರು. ಸ್ವಾಮೀಜಿಗೆ ಮಾತ್ರ ಗ್ರಹಣದ ವಿಚಾರವಾಗಿ ವಿಶೇಷವಾದ ಉತ್ಸಾಹವೇನೂ ಇಲ್ಲ. ಶಿಷ್ಯರು (ಶರತ್‍ಚಂದ್ರನು) ಈ ದಿನ ಸ್ವಾಮೀಜಿಗೆ ತನ್ನ ಕೈಯಿಂದಲೇ ಅಡಿಗೆಮಾಡಿ ಬಡಿಸುತ್ತಾನೆ. ಹಾಗೆ ಸ್ವಾಮಿಗಳ ಅಪ್ಪಣೆ ಆಗಿದೆ. ತರಕಾರಿ ಮತ್ತು ಅಡಿಗೆಗೆ ಬೇಕಾದ ಇತರ ಪದಾರ್ಥಗಳನ್ನು ತಂದು ಅವನು ಸುಮಾರು ಎರಡು ಗಂಟೆಯ ಹೊತ್ತಿಗೆ ಬಲರಾಮ ಬಾಬುಗಳ ಮನೆಗೆ ಬಂದಿದ್ದಾನೆ. ಅವನನ್ನು ನೋಡಿ ಸ್ವಾಮೀಜಿ, “ನಿಮ್ಮ ದೇಶದ (ಪೂರ್ವ ಬಂಗಾಳ) ಅಡಿಗೆ ಮಾಡಬೇಕು ಮತ್ತು ಗ್ರಹಣವಾಗುವುದಕ್ಕೆ ಮುಂಚೆಯೇ ತಿನ್ನುವುದು ಕೊಡುವುದು ಎಲ್ಲಾ ಮುಗಿದು ಹೋಗಬೇಕು” ಎಂದರು. 

 ಬಲರಾಮ ಬಾಬುಗಳ ಮನೆ ಹೆಂಗಸರು ಯಾರೂ ಈಗ ಕಲ್ಕತ್ತೆಯಲ್ಲಿ ಇಲ್ಲ. ಆದ್ದರಿಂದ ಮನೆ ಈಗ ಪೂರ್ತಿಯಾಗಿ ಬಿಡುವಾಗಿತ್ತು. ಶಿಷ್ಯ ಮನೆಯ ಒಳಗೆ ಹೋಗಿ ಅಡಿಗೆ ಮಾಡಲು ಮೊದಲು ಮಾಡಿದನು. ಶ‍್ರೀರಾಮಕೃಷ್ಣರ ಗತಪ್ರಾಣಳಾದ ಯೋಗಿನ್ ಮಾತೆಯು ಹತ್ತಿರ ನಿಂತುಕೊಂಡು ಬೇಕಾದ ಪದಾರ್ಥಗಳನ್ನು ಶಿಷ್ಯನಿಗೆ ಒದಗಿಸಿಕೊಡುತ್ತ ಸಹಾಯ ಮಾಡುತ್ತಿದ್ದಳು. ಸ್ವಾಮೀಜಿ ಮಧ್ಯೆ ಮಧ್ಯೆ ಒಳಕ್ಕೆ ಬಂದು ಅಡಿಗೆಯನ್ನು ನೋಡಿ ಅವನಿಗೆ ಉತ್ಸಾಹ ಕೊಡುತ್ತಿದ್ದರು. ಮತ್ತು ಆಗಾಗ್ಯೆ “ನೋಡು ಅಡಿಗೆ ಪೂರ್ವಬಂಗಾಳದ ಕಡೆಯವರ ಹಾಗೆ ಇರಬೇಕು” ಎಂದು ಹೇಳಿದರು. 

 ಅನ್ನ ಸಾರು ತೊವ್ವೆ ಪಲ್ಯ ಮುಂತಾದ ಅಡಿಗೆ ಇನ್ನೇನು ಮುಗಿದಂತೆ ಇತ್ತು. ಆ ಸಮಯದಲ್ಲಿ ಸ್ವಾಮೀಜಿ ಸ್ನಾನ ಮಾಡಿಕೊಂಡು ಬಂದು ತಾವೇ ಎಲೆಹಾಕಿಕೊಂಡು ಊಟಕ್ಕೆ ಕುಳಿತರು. “ಇನ್ನೂ ಸ್ವಲ್ಪ ಅಡಿಗೆ ಆಗುವುದು ಇದೆ” ಎಂದು ಹೇಳಿದರೂ ಕೇಳದೆ ಹಟ ಮಾಡಿಕೊಂಡು ಹುಡುಗರ ಹಾಗೆ “ಏನಾಗಿದೆಯೊ ಅದನ್ನೇ ಬೇಗ ತಂದು ಹಾಕಿಬಿಡು. ಇನ್ನು ನಾನು ಕಾದುಕೊಂಡಿರಲಾರೆ. ಹೊಟ್ಟೆ ಹಸಿವಿನಿಂದ ಉರಿದುಹೋಗುತ್ತಿದ್ದೇನೆ;” ಎಂದರು. ಆದ್ದರಿಂದ ಶಿಷ್ಯನು ಗಡಿಬಿಡಿಯಿಂದ ಮೊದಲು ಪಲ್ಯ, ಅನ್ನವನ್ನು ಬಡಿಸಿದನು. ಸ್ವಾಮೀಜಿ ತಕ್ಷಣವೇ ಊಟಕ್ಕೆ ಮೊದಲು ಮಾಡಿದರು. ಆಮೇಲೆ ಶಿಷ್ಯನು ಮಿಕ್ಕ ಅಡಿಗೆಗಳನ್ನೆಲ್ಲ ಸ್ವಾಮೀಜಿಗೆ ಬಡಿಸಿದಮೇಲೆ ಯೋಗಾನಂದ ಪ್ರೇಮಾನಂದ ಮುಂತಾದ ಮಿಕ್ಕ ಸಂನ್ಯಾಸಿವೃಂದಕ್ಕೆ ಬಡಿಸಲು ಹೊರಟನು. ಶಿಷ್ಯನು ಯಾವತ್ತೂ ಅಡಿಗೆಯಲ್ಲಿ ಬುದ್ಧಿವಂತನೆನಿಸಿಕೊಂಡಿರಲಿಲ್ಲ. ಆದರೆ ಸ್ವಾಮೀಜಿ ಇಂದು ಅವನ ಅಡಿಗೆಯನ್ನು ಬಹಳವಾಗಿ ಹೊಗಳುವುದಕ್ಕೆ ಮೊದಲು ಮಾಡಿದರು. ಕಲ್ಕತ್ತೆಯ ಜನರು ಆತ ಮಾಡಿದ್ದ ಪಲ್ಯದ ಹೆಸರನ್ನು ಕೇಳಿದೊಡನೆಯೆ ಹಾಸ್ಯಮಾಡಿಕೊಂಡು ನಗುವರು. ಆದರೆ ಸ್ವಾಮೀಜಿ ಮಾತ್ರ ಆ ಪಲ್ಯವನ್ನು ತಿಂದು ಸಂತೋಷಪಟ್ಟು, “ಇಂಥಾದನ್ನು ಎಂದೂ ತಿಂದಿರಲಿಲ್ಲ!” ಎಂದರು. ಆಮೇಲೆ ಮೊಸರು ಹಾಕಿಸಿಕೊಂಡು ಊಟ ಮಾಡಿ ಮುಗಿಸಿ ಆಚಮನಾನಂತರದಲ್ಲಿ ಒಳಗೆ ಇದ್ದ ಮಂಚದ ಮೇಲೆ ಹೋಗಿ ಕುಳಿತುಕೊಂಡರು. ಶಿಷ್ಯನು ಸ್ವಾಮೀಜಿ ಎದುರಿಗೆ ನಡುಮನೆಯಲ್ಲಿ ಪ್ರಸಾದವನ್ನು ತೆಗೆದುಕೊಳ್ಳುತ್ತ ಕುಳಿತ. ಸ್ವಾಮೀಜಿ ತಂಬಾಕನ್ನು ತೀಡುತ್ತ ತೀಡುತ್ತ ಅವನನ್ನು ಕುರಿತು “ಯಾರು ಚೆನ್ನಾಗಿ ಅಡಿಗೆ ಮಾಡಲಾರರೊ ಅವರು ಒಳ್ಳೆಯ ಸಾಧುಗಳಾಗಲಾರರು. ಮನಸ್ಸು ಶುದ್ಧವಾಗಿಲ್ಲದೆ ಇದ್ದರೆ ಒಳ್ಳೆಯ ರುಚಿಯಾದ ಅಡಿಗೆ ಆಗುವುದಿಲ್ಲ” ಎಂದು ಹೇಳಿದರು. 

 ಸ್ವಲ್ಪ ಹೊತ್ತಿನ ಮೇಲೆ ನಾಲ್ಕು ಕಡೆಯಲ್ಲಿಯೂ ಶಂಖದ ಮತ್ತು ಘಂಟೆಯ ಧ್ವನಿ ಎದ್ದಿತು. ಮತ್ತು ಸ್ತ್ರೀಯರ ‘ಉಲು’ ಧ್ವನಿಯು (ಮಂಗಳಧ್ವನಿ) ಕೇಳುವುದಕ್ಕೆ ಮೊದಲಾಯಿತು. ಸ್ವಾಮೀಜಿ “ಅಯ್ಯಾ, ಗ್ರಹಣ ಹಿಡಿಯಿತು, ನಾನು ಮಲಗಿಕೊಳ್ಳುತ್ತೇನೆ, ಸ್ವಲ್ಪ ಕಾಲು ಹಿಸುಕು” ಎಂದು ಹೇಳಿ ನಿದ್ರೆ ಮಾಡತೊಡಗಿದರು. ಶಿಷ್ಯನು ಅವರ ಪಾದಸೇವೆ ಮಾಡುತ್ತ ಈ ಪುಣ್ಯಕಾಲದಲ್ಲಿ ಗುರು ಪಾದಸೇವೆಯೆ ತನಗೆ ಗಂಗಾಸ್ನಾನ ಮತ್ತು ಜಪ ಎಂದು ಭಾವಿಸತೊಡಗಿದನು. ಹೀಗೆಂದುಕೊಂಡು ಶಿಷ್ಯನು ಶಾಂತ ಮನಸ್ಕನಾಗಿ ಸ್ವಾಮೀಜಿ ಪಾದಸೇವೆ ಮಾಡತೊಡಗಿದನು. ಗ್ರಹಣ ಪೂರ್ಣ ಗ್ರಾಸವಾಗಿ ಕ್ರಮೇ‌ಣ ನಾಲ್ಕು ದಿಕ್ಕುಗಳಲ್ಲಿಯೂ ಸಾಯಂಕಾಲದ ಹಾಗೆ ಕತ್ತಲೆ ಕವಿದುಕೊಂಡಿತ್ತು. 

ಗ್ರಹಣ ಬಿಡುವುದಕ್ಕೆ ಇನ್ನು ೧೫-೨೦ ನಿಮಿಷವಿದೆ ಎನ್ನುವಾಗ ಸ್ವಾಮೀಜಿ ಎದ್ದು ಕೈಕಾಲು ಮುಖ ತೊಳೆದುಕೊಂಡು ತಂಬಾಕನ್ನು ಸೇವಿಸುತ್ತ ಶಿಷ್ಯನನ್ನು ಕುರಿತು ಹಾಸ್ಯಮಾಡುತ್ತ “ಗ್ರಹಣಕಾಲದಲ್ಲಿ ಯಾರು ಏನು ಮಾಡುತ್ತಾರೆಯೊ ಅವರು ಅದನ್ನು ಕೋಟಿ ಪಾಲಿನಷ್ಟು ಪಡೆಯುತ್ತಾರೆ ಎಂದು ಜನ ಹೇಳುತ್ತಾರೆ. ಮಹಾಮಾಯೆ ಈ ಶರೀರದಲ್ಲಿ ಒಳ್ಳೆಯ ನಿದ್ದೆಯನ್ನು ಕೊಡಲಿಲ್ಲ. ಈ ಸಮಯದಲ್ಲಿ ಒಂದಿಷ್ಟು ನಿದ್ದೆ ಮಾಡಿದರೆ ಆಮೇಲೆ ಚೆನ್ನಾಗಿ ನಿದ್ದೆ ಬರುವದು ಎಂದು ಭಾವಿಸಿದೆ. ಆದರೆ ಅದು ಆಗಲಿಲ್ಲ. ಹೆಚ್ಚೆಂದರೆ ಹದಿನೈದು ನಿಮಿಷ ನಿದ್ದೆಯಾಗಿರಬಹುದು.” 

 ಅನಂತರ ಎಲ್ಲರೂ ಸ್ವಾಮೀಜಿ ಹತ್ತಿರ ಬಂದು ಕುಳಿತುಕೊಂಡರು. ಸ್ವಲ್ಪಕಾಲ ಇತರ ಮಾತುಕತೆಗಳಾದ ಮೇಲೆ ಶುದ್ಧಾನಂದ ಸ್ವಾಮಿಗಳು, ಸ್ವಾಮೀಜಿಯವರನ್ನು ಧ್ಯಾನದ ಸ್ವರೂಪವೇನು ಎಂದು ಪ್ರಶ್ನೆ ಮಾಡಿದರು. 

 ಸ್ವಾಮೀಜಿ: “ಯಾವುದಾದರೂ ಒಂದು ವಸ್ತುವಿನಮೇಲೆ ಮನಸ್ಸನ್ನು ಕೇಂದ್ರೀಕರಿಸುವುದಕ್ಕೆ ಧ್ಯಾನ ಎಂದು ಹೆಸರು. ಒಂದು ವಿಷಯದ ಮೇಲೆ ಮನಸ್ಸನ್ನು ಏಕಾಗ್ರಮಾಡಲು ಸಮರ್ಥನಾದರೆ ಆ ಮನಸ್ಸನ್ನು ಯಾವ ವಿಷಯದ ಮೇಲೆ ಬೇಕಾದರೂ ಏಕಾಗ್ರ ಮಾದಬಹುದು.” 

 ಶಿಷ್ಯ: “ಶಾಸ್ತ್ರದಲ್ಲಿ ವಿಷಯ ಮತ್ತು ನಿರ್ವಿಷಯ ಎಂದು ಭೇದವನ್ನು ಇಟ್ಟುಕೊಂಡು ಎರಡು ವಿಧವಾದ ಧ್ಯಾನ ಹೇಳಲ್ಪಟ್ಟಿದೆಯಲ್ಲಾ ಅದರ ಅರ್ಥವೇನು? ಅವುಗಳಲ್ಲಿ ಯಾವುದು ಶ್ರೇಷ್ಠ?” 

 ಸ್ವಾಮೀಜಿ: “ಮೊದಲು ಯಾವುದಾದರೂ ವಿಷಯವನ್ನು ತೆಗೆದುಕೊಂಡು ಧ್ಯಾನವನ್ನು ಅಭ್ಯಾಸ ಮಾಡಬೇಕು. ಒಂದು ಕಾಲದಲ್ಲಿ ನಾನು ಒಂದು ಕರಿಯ ಬಟ್ಟಿನಲ್ಲಿ ಮನಸ್ಸನ್ನು ನಿಲ್ಲಿಸುತ್ತಿದ್ದೆನು. ಆಗ ಕೊನೆಯಲ್ಲಿ ಬಟ್ಟು ಕಾಣದೇ ಹೋಗುತ್ತಿತ್ತು ಅಥವಾ ಮುಂದೆ ಏನಿದೆ ಅದನ್ನು ತಿಳಿದುಕೊಳ್ಳಲಾರದೆ ಹೋಗುತ್ತಿದ್ದೆನು. ಮನಸ್ಸು ನಿರುದ್ಧವಾಗಿಬಿಡುತ್ತಿತ್ತು. ಯಾವ ವೃತ್ತಿಯ ತರಂಗವೂ ಉಂಟಾಗುತ್ತಿರಲಿಲ್ಲ. ಗಾಳಿಯಿಲ್ಲದ ಸಾಗರದಂತೆ ಇರುತ್ತಿತ್ತು. ಈ ಅವಸ್ಥೆಯಲ್ಲಿ ಅತೀಂದ್ರಿಯ ಸತ್ಯದ ಛಾಯೆಯನ್ನು ಸ್ವಲ್ಪಸ್ವಲ್ಪ ನೋಡಬಹುದಾಗಿತ್ತು. ಅದಕ್ಕೋಸ್ಕರವೇ ಯಾವುದಾದರೂ ಒಂದು ಬಾಹ್ಯ ವಿಷಯವನ್ನು ಇಟ್ಟುಕೊಂಡು ಧ್ಯಾನವನ್ನು ಅಭ್ಯಾಸ ಮಾಡಿದರೂ ಮನಸ್ಸು ಏಕಾಗ್ರ ಅಥವಾ ಧ್ಯಾನಸ್ಥವಾಗುತ್ತದೆ ಎಂದು ತೋರುವುದು. ಆದರೆ ಯಾವುದರಲ್ಲಿ ಯಾವ ಮನಸ್ಸು ನಿಲ್ಲುತ್ತದೆಯೋ ಅದನ್ನು ಅವರು ಅವಲಂಬಿಸಿಕೊಂಡು ಧ್ಯಾನವನ್ನು ಅಭ್ಯಾಸ ಮಾದಿದರೆ ಮನಸ್ಸು ಶೀಘ್ರದಲ್ಲಿ ಸ್ಥಿರವಾಗಿ ಹೋಗುತ್ತದೆ. ಅದಕ್ಕೋಸ್ಕರವೇ ಈ ದೇಶದಲ್ಲಿ ಇಷ್ಟೊಂದು ದೇವದೇವಿಯರ ಮೂರ್ತಿಗಳ ಪೂಜೆ. ಅಲ್ಲದೆ ಈ ದೇವ ದೇವಿಯರ ಪೂಜೆಯಿಂದ ಎಂತಹ ಶಿಲ್ಪಕಲೆ ಉತ್ಪನ್ನವಾಯಿತು! ಈಗ ಆ ಮಾತು ಹಾಗಿರಲಿ. ಧ್ಯಾನದ ಬಾಹ್ಯಾಲಂಬನ ಎಲ್ಲರಿಗೂ ಸಮಾನ ಅಥವಾ ಒಂದೇ ಆಗುವುದು ಸಾಧ್ಯವಿಲ್ಲ. ಯಾರು ಯಾವ ವಿಷಯಗಳನ್ನು ಅವಲಂಬಿಸಿಕೊಂಡು ಧ್ಯಾನಸಿದ್ಧನಾಗುತ್ತಾನೆಯೊ ಆತನು ಆ ಬಾಹ್ಯಾಲಂಬನವನ್ನೇ ಕೀರ್ತನೆ ಮಾಡುತ್ತಲೂ ಪ್ರಚಾರ ಮಾಡುತ್ತಲೂ ಹೋಗುವನು. ಆಮೇಲೆ ಕ್ರಮೇಣ ಅದರಿಂದ ಮನಸ್ಸನ್ನು ಸ್ಥಿರಪಡಿಸಿಕೊಳ್ಳಬೇಕೆಂಬ ಸಂಗತಿ ಮರೆತು ಬಾಹ್ಯಾಲಂಬನವೇ ದೊಡ್ಡದಾಗಿ ಕುಳಿತುಕೊಳ್ಳುತ್ತದೆ. ಉಪಾಯವನ್ನು ಹಿಡಿದುಕೊಂಡೆ ಜನರು ಒದ್ದಾಡುತ್ತ ಕುಳಿತಿದ್ದಾರೆ. ಉದ್ದೇಶದ ಕಡೆಗೆ ಲಕ್ಷ್ಯವೇ ಕಡಿಮೆಯಾಗಿ ಹೋಗಿದೆ. ಉದ್ದೇಶ ಮನಸ್ಸನ್ನು ವೃತ್ತಿ ಶೂನ್ಯ ಮಾಡುವುದು. ಯಾವುದಾದರೂ ಒಂದು ವಿಷಯವಾಗದೇ ಹೋದರೆ ಇದು ಸಾಧ್ಯವಿಲ್ಲ.” 

 ಶಿಷ್ಯ: “ಮನೋವೃತ್ತಿ ವಿಷಯಾಕಾರವನ್ನು ಪಡೆದರೆ ಅದರಿಂದ ಬ್ರಹ್ಮಾನುಭವ ಬರುವುದು ಹೇಗೆ?” 

 ಸ್ವಾಮೀಜಿ: “ವೃತ್ತಿ ಮೊದಲು ವಿಷಯಾಕಾರವಾಗಿರುವುದೇನೋ ನಿಜ. ಆದರೆ ಆಮೇಲೆ ಈ ವಿಷಯದ ಜ್ಞಾನ ಇರುವುದಿಲ್ಲ. ‘ಆಸ್ತಿ’ (ಇದೆ) ಎಂಬಷ್ಟು ಜ್ಞಾನ ಮಾತ್ರ ಇರುತ್ತದೆ.” 

 ಶಿಷ್ಯ: “ಮಹರಾಜ್, ಮನಸ್ಸಿಗೆ ಏಕಾಗ್ರತೆ ಉಂಟಾದರೂ ಕಾಮವಾಸನಾದಿಗಳು ಬರುತ್ತವೆ. ಏಕೆ?” 

 ಸ್ವಾಮೀಜಿ: “ಅವು ಹಿಂದಿನ ಸಂಸ್ಕಾರದಿಂದ ಬರುತ್ತವೆ. ಬುದ್ಧದೇವನು ಸಮಾಧಿಸ್ಥನಾಗಲು ಹೋಗಲು ಮಾರನು ತಲೆ ಹಾಕುತ್ತಿದ್ದನು. ಮಾರನೆಂದರೆ ಹೊರಗೆ ಏನೂ ಇರುತ್ತಿರಲಿಲ್ಲ. ಮನಸ್ಸಿನ ಪೂರ್ವಸಂಸ್ಕಾರವೇ ಛಾಯಾರೂಪವಾಗಿ ಹೊರಗೆ ಕಾಣುತ್ತಿತ್ತು.” 

 ಶಿಷ್ಯ: “ಹಾಗಾದರೆ ಸಿದ್ಧನಾಗುವುದಕ್ಕೆ ಮುಂಚೆ ನಾನಾ ಹೆದರಿಕೆಯನ್ನು ಉಂಟುಮಾಡುವ ನಾನಾ ರೂಪಗಳು ಕಂಡುಬರುತ್ತವೆ ಎಂದು ಕೇಳಿದ್ದೇವಲ್ಲ, ಅದೇನು ಮನಃಕಲ್ಪಿತವೆ?” 

 ಸ್ವಾಮೀಜಿ: “ಅಲ್ಲದೆ ಮತ್ತೇನು? ಸಾಧಕನು ಆಗ ಇವು ತನ್ನ ಮನಸ್ಸಿನ ಬಹಿಃಪ್ರಕಾಶ ಎಂಬುದನ್ನು ತಿಳಿದುಕೊಳ್ಳಲಾರನು. ಆದರೆ ಹೊರಗೆ ಏನೂ ಇರುವುದಿಲ್ಲ. ನೀನು ನೋಡುತ್ತಿದ್ದೀಯಲ್ಲಾ ಈ ಜಗತ್ತು ಇರುವುದಿಲ್ಲ. ಎಲ್ಲಾ ಮನಸ್ಸಿನ ಕಲ್ಪನೆ. ಮನಸ್ಸು ಯಾವಾಗ ವೃತ್ತಿಶೂನ್ಯವಾಗುವುದೋ ಆವಾಗ ಅದರಿಂದ ಬ್ರಹ್ಮಭಾವದ ದರ್ಶನವಾಗುತ್ತದೆ. ಆಗ ‘ಯಂ ಯಂ ಲೋಕಂ ಮನಸಾ ಸಂವಿಭಾತಿ’ ಆಯಾ ಲೋಕದ ದರ್ಶನವಾಗುತ್ತದೆ, ಯಾವುದನ್ನು ಸಂಕಲ್ಪಮಾಡಿದರೆ ಅದು ಸಿದ್ಧವಾಗುತ್ತದೆ. ಈ ವಿಧವಾದ ಸತ್ಯ ಸಂಕಲ್ಪಾವಸ್ಥೆಯುಂಟಾದರೆ, ಮನಸ್ಸನ್ನು ಅಧೀನದಲ್ಲಿಟ್ಟುಕೊಂಡು ಯಾವ ಆಸೆಗೂ ದಾಸನಾಗದೆ ಇರಬಲ್ಲವನಾದರೆ ಅಂಥವನೆ ಬ್ರಹ್ಮಜ್ಞಾನವನ್ನು ಪಡೆಯುತ್ತಾನೆ. ಯಾರು ಚಂಚಲರಾಗುತ್ತಾರೆಯೋ ಅವರು ನಾನಾಸಿದ್ಧಿಗಳನ್ನು ಪಡೆದು ಪರಮಾರ್ಥದಿಂದ ಬ್ರಷ್ಟರಾಗುತ್ತಾರೆ.” 

 ಆ ಮಾತನ್ನು ಹೇಳುತ್ತ ಹೇಳುತ್ತ ಸ್ವಾಮೀಜಿ, ‘ಶಿವ, ಶಿವ’ ಎಂದು ಶಿವನಾಮವನ್ನು ಉಚ್ಚಾರ ಮಾಡತೊಡಗಿದರು. ಕೊನೆಗೆ ಹೀಗೆ ಹೇಳಿದರು: “ತ್ಯಾಗವನ್ನು ಬಿಟ್ಟು ಮತ್ತಾವುದರಿಂದಲೂ ಈ ಗಂಭೀರವಾದ ಜೀವನ ಸಮಸ್ಯೆಯ ರಹಸ್ಯವನ್ನು ಭೇದಿಸುವುದಕ್ಕೆ ಆಗುವುದಿಲ್ಲ. ತ್ಯಾಗ, ತ್ಯಾಗ, ತ್ಯಾಗ! ಇದೇ ನಿಮ್ಮ ಜೀವನದ ಮೂಲಮಂತ್ರವಾಗಲಿ. ‘ಸರ್ವಂ ವಸ್ತು ಭಯಾನ್ವಿತಂ ಭುವಿ ನೃಣಾಂ ವೈರಾಗ್ಯಮೇವಾ ಭಯಂ.’ ಈ ಪ್ರಪಂಚದಲ್ಲಿ ಎಲ್ಲಾ ಭಯದಿಂದ ಕೂಡಿದೆ. ವೈರಾಗ್ಯ ಒಂದೇ ಅಭಯವನ್ನು ನೀಡುವುದು.” 

