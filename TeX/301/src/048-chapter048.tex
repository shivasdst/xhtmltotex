
\chapter{ಯೂರೋಪಿನ ಕಡೆಗೆ }

 ವೈದ್ಯರು ಸ್ವಾಮೀಜಿಯವರಿಗೆ ಸಾಧ್ಯವಾದಷ್ಟು ಬೇಗ ಇಂಡಿಯಾ ದೇಶದಿಂದ\break ಯೂರೋಪಿನ ಕಡೆಗೆ ಹವಾ ಬದಲಾವಣೆಗೆ ಹೋದರೆ ಮೇಲೆಂದು ಸೂಚಿಸಿದರು. ಸ್ವಾಮೀಜಿಯವರಿಗೂ ಅಮೇರಿಕಾದಿಂದ ಮೇಲಿಂದ ಮೇಲೆ ಆ ದೇಶಕ್ಕೆ ಪುನಃ ಬರಬೇಕು ಎಂದು ಪತ್ರಗಳು ಬರುತ್ತಿದ್ದವು. ಸ್ವಾಮೀಜಿಗೆ ಇಂಗ್ಲೆಂಡ್ ಮತ್ತು ಅಮೇರಿಕಾದೇಶದಲ್ಲಿ ಮತ್ತೊಬ್ಬ ಸಂನ್ಯಾಸಿ ಬೋಧಕರು ಬೇಕಾಗಿತ್ತು. ಅದಕ್ಕಾಗಿ ತುರಿಯಾನಂದರಿಗೆ ತಮ್ಮೊಡನೆ ಬರಲು ಹೇಳಿದರು. ತುರಿಯಾನಂದರು ಮೊದಲು ಒಪ್ಪಲಿಲ್ಲ. ತಮಗೆ ಇಂಗ್ಲೀಷಿನಲ್ಲಿ ಉಪನ್ಯಾಸಾದಿಗಳು ಮಾಡುವುದಕ್ಕೆ ಬರುವುದಿಲ್ಲ ಎಂದರು. ಆಗ ಸ್ವಾಮೀಜಿ ಉಪನ್ಯಾಸವನ್ನು ತಾವು ಸಾಕಷ್ಟು ಮಾಡಿರುವೆನೆಂದೂ, ಈಗ ಬೇಕಾಗಿರುವುದು ಆಧ್ಯಾತ್ಮಿಕ ಜೀವನವನ್ನು ನಡೆಸುವವರ ಒಂದು ಆದರ್ಶವೆಂದೂ ಹೇಳಿದರು. ಸ್ವಾಮೀಜಿಗೆ ತುರಿಯಾನಂದರನ್ನು ಒಪ್ಪಿಸುವುದೇ ಒಂದು ದೊಡ್ಡ ಕಷ್ಟವಾಗಿತ್ತು. ಸ್ವಾಮೀಜಿ ತುರಿಯಾನಂದರನ್ನು ತಬ್ಬಿಕೊಂಡು ಕಣ್ಣೀರು ಸುರಿಸುತ್ತ ತಾವು ಆಧ್ಯಾತ್ಮಿಕ ವಿಷಯಗಳನ್ನು ಹರಡುವ ಕೆಲಸದಲ್ಲಿ ಸಾಯುತ್ತಿರುವೆನು, ಸ್ವಲ್ಪ ಸಹಾಯಕ್ಕೆ ಬರಲಾರೆಯ ಎಂದು ಕೇಳಿಕೊಂಡಾಗ ತುರಿಯಾನಂದರು ಒಪ್ಪಿಕೊಳ್ಳದೆ ವಿಧಿಯೇ ಇರಲಿಲ್ಲ. ಸ್ವಾಮೀಜಿ ನಿವೇದಿತಾಳನ್ನು ಹಣಸಂಗ್ರಹಕ್ಕಾಗಿ ಪಶ್ಚಿಮಕ್ಕೆ ಬರಬೇಕೆಂದು ಕೇಳಿಕೊಂಡರು. ಹೀಗೆ ಮೂರು ಜನರೂ ೨೦ನೇ ತಾರೀಖು ೧೮೯೯ರಲ್ಲಿ ಕಲ್ಕತ್ತೆಯಿಂದ ‘ಗೋಲ್‍ಕೊಂಡ’ ಎಂಬ ಹಡಗಿನಲ್ಲಿ ಯೂರೋಪಿನ ಕಡೆ ಹೊರಟರು. 

 ಸ್ವಾಮೀಜಿ ಹಡಗು ೨೪ನೇ ತಾರೀಖು ಮದ್ರಾಸನ್ನು ಮುಟ್ಟಿತು. ಆದರೆ ಹಡಗಿನಲ್ಲಿದ್ದ ಯಾರನ್ನೂ ಕಲ್ಕತ್ತೆಯಲ್ಲಿ ಪ್ಲೇಗಿನ ನಿಮಿತ್ತ ಹೊರಗಡೆ ಬರಲು ಬಿಡಲಿಲ್ಲ. ಮದ್ರಾಸಿನ ಅನೇಕ ಭಕ್ತರು ಮತ್ತು ರಾಮಕೃಷ್ಣಾನಂದ ಸ್ವಾಮಿಗಳು ಹೊರಗಡೆ ನಿಂತುಕೊಂಡೇ ಮಾತನಾಡಿದರು. ಭಕ್ತರು ಹೂವು ಹಣ್ಣು ಮುಂತಾದುವುಗಳನ್ನು ಸ್ವಾಮೀಜಿಗೆ ಕಳುಹಿಸಿದರು. ಅಳಸಿಂಗ ಪೆರುಮಾಳ್ ಕೊಲಂಬೋಗೆ ಹೋಗುವುದಕ್ಕೆ\break ಸ್ವಾಮೀಜಿಯೊಂದಿಗೆ ಹಡಗು ಹತ್ತಿದರು. ಅವರು ಬ್ರಹ್ಮಾವಾದಿನಿಗೆ ಸೇರಿದ ಸಮಸ್ಯೆಗಳನ್ನು ಅವರೊಂದಿಗೆ ಚರ್ಚಿಸಬೇಕಾಗಿತ್ತು. 

 ಕೊಲಂಬೊ ನಗರವನ್ನು ಮುಟ್ಟಿದಾಗ ಹಲವು ಭಕ್ತರು ಸ್ವಾಮೀಜಿಯವರನ್ನು ನೋಡುವುದಕ್ಕೆ ಬಂದಿದ್ದರು. ಅವರಲ್ಲಿ ಸರ್ ಕುಮಾರಸ್ವಾಮಿ, ಅರುಣಾಚಲಂ ಮುಂತಾದವರೂ ಇದ್ದರು. ಶ‍್ರೀಮತಿ ಹಿಗಿನ್ಸ್ ಎಂಬಾಕೆ ಬೌದ್ಧ ಹುಡುಗಿಯರಿಗೆ ನಡೆಸುತ್ತಿದ್ದ ಒಂದು ಬೋರ್ಡಿಂಗ್ ಸ್ಕೂಲನ್ನು ಸ್ವಾಮೀಜಿ ನೋಡಿಕೊಂಡು ಬಂದರು. ಜೂನ್ ೨೮ನೇ ತಾರೀಖು ಹಡಗು ಕೊಲಂಬೊ ರೇವನ್ನು ಬಿಟ್ಟು ಅರಬ್ಬಿ ಸಮುದ್ರವನ್ನು ಪ್ರವೇಶಿಸಿತು. ಆಗ ಮಳೆಯಿಂದ ತಾಡಿತವಾದ ಸಮುದ್ರದಲ್ಲಿ ಹಡಗು ಮುಂದುವರಿಯುತ್ತ ಹತ್ತು ದಿನಗಳಾದ ಮೇಲೆ ಜುಲೈ ೮ನೇ ತಾರೀಖು ಏಡನ್ ನಗರವನ್ನು ತಲುಪಿತು. ಅನಂತರ ಸಾಗರ ಶಾಂತವಾಗಿತ್ತು. ಕೆಂಪು ಸಮುದ್ರ ಸೂಯೇಜ್ ಕಾಲುವೆ ದಾಟಿ ಮೆಡಿಟರೇನಿಯನ್ ಸಮುದ್ರವನ್ನು ಪ್ರವೇಶ ಮಾಡಿತು. ನೇಪಲ್ಸ್ ಮಾರ್‍ಸೆಲ್ಸ್ ಮುಖಾಂತರ ಲಂಡನ್ ಅನ್ನು ಜುಲೈ ೩೧ನೇ ತಾರೀಖು ಸೇರಿತು. 

 ಸ್ವಾಮೀಜಿ ಸಮುದ್ರದಲ್ಲಿ ಹೋಗುತ್ತಿದ್ದಾಗಲೇ ಉದ್ಬೋಧನಕ್ಕಾಗಿ ‘ಪರಿವ್ರಾಜಕ’ ಎಂಬ ಶಿರೋನಾಮೆಯಲ್ಲಿ ಪರ್ಯಟನದ ಒಂದು ಪುಸ್ತಕವನ್ನು ಬರೆಯತೊಡಗಿದರು. ಕಲ್ಕತ್ತೆಯಿಂದ ಗಂಗಾನದಿಯ ಮೇಲೆ ಹೊರಟು ಚಂಡಮಾರುತದಿಂದ ತಾಡಿತವಾದ ಬಂಗಾಳಕೊಲ್ಲಿ ಅನಂತರ ಅರಬ್ಬಿಸಮುದ್ರ, ಕೆಂಪುಸಮುದ್ರದ ಮೂಲಕ ಯೂರೋಪಿಗೆ ಹೋಗುವಾಗ ನೋಡಿರುವುದನ್ನು ತಮ್ಮದೇ ಆದ ಅನನುಕರಣೀಯ ಶೈಲಿಯಲ್ಲಿ ಹೇಳುವರು. ಸಾಹಿತಿ ಸ್ವಾಮಿ ವಿವೇಕಾನಂದರ ಪರಿಚಯವಾದುದು ನಮಗೆ ಈ ಪುಸ್ತಕದಲ್ಲಿ. ಇದರಲ್ಲಿ ಬರುವ ಲಘು ಹಾಸ್ಯ, ಕಥನ ಶೈಲಿ, ವರ್ಣನೆ, ವಿಮರ್ಶಾತ್ಮಕ ಐತಿಹಾಸಿಕ ದೃಷ್ಟಿ ಓದುಗರನ್ನು ಮುಗ್ಧರನ್ನಾಗಿ ಮಾಡುವುದು. ಮಧ್ಯೆ ಮಧ್ಯೆ ತೇಜಃಪುಂಜವಾದ ಸ್ಫೂರ್ತಿದಾಯಕ ನುಡಿಗಳು ಜ್ವಾಲಾಮುಖಿಯಿಂದ ಸಿಡಿದ ಶಿಲಾಪ್ರವಾಹದಂತೆ ಇವೆ. ಪ್ರವಾಸ ಸಾಹಿತ್ಯದಲ್ಲಿ ಇದೊಂದು ಅತ್ಯಮೂಲ್ಯ ಗ್ರಂಥ. ಇದರಿಂದ ಓದುಗರಿಗೆ ಸ್ವಲ್ಪ ಭಾಗವನ್ನು ರುಚಿ ನೋಡುವುದಕ್ಕೆ ಕೊಡುವೆವು. ಗಂಗಾನದಿಯನ್ನು ಕುರಿತು ಮಾಡಿರುವ ವರ್ಣನೆ ಎಷ್ಟು ಕಾವ್ಯಮಯವಾಗಿ ಇದೆ: 

 “ಹೃಷೀಕೇಶದ ಗಂಗೆ ನಿನಗೆ ಜ್ಞಾಪಕವಿದೆಯೆ? ಈ ನಿರ್ಮಲ ನೀಲಾಭ ಜಲ, ಹತ್ತು ಮಾರು ನೀರಿನ ಒಳಗೆ ಓಡಾಡುವ ಮೀನಿನ ಮೇಲಿರುವ ಪೊರೆ ಕೂಡ ಕಾಣುವುದು. ಆ ಅಪೂರ್ವ ಸುಸ್ವಾದ ಹಿಮಶೀತಲ ‘ಗಂಗಾವಾರಿ ಮನೋಹಾರಿ’ ಅದ್ಭುತ ತರಂಗೋತ್ತ ‘ಹರಹರ’ ಧ್ವನಿ. ಹತ್ತಿರದ ಗಿರಿಕಂದರಗಳಿಂದ ಹರಹರ ನಿನಾದದ ಪ್ರತಿದಿಧ್ವನಿ. ಆ ಹೃಷಿಕೇಶದ ಅರಣ್ಯದಲ್ಲಿ ಮಧುಕರೀ ಭಿಕ್ಷೆಯ ಜೀವನ, ಗಂಗಾದ್ವೀಪದ ಮೇಲೆ ಭೋಜನ, ಕೈಗಳಿಂದ ಯಥೇಚ್ಛವಾದ ಆ ಶೀತಲವಾರಿಯ ಪಾನ, ಸುತ್ತಲೂ ನಿರ್ಭಯದಿಂದ ತೇಲುವ ರೊಟ್ಟಿಯ ಚೂರಿಗೆ ಧಾವಿಸುವ ಮೀನಿನ ಬಳಗ ಇವೆಲ್ಲ ನಿನಗೆ ಜ್ಞಾಪಕವಿದೆಯೆ? ಈ ಗಂಗಾಜಲ ಪ್ರೀತಿ, ಗಂಗಾಮಹಿಮೆ, ಈ ಗಂಗಾವಾರಿಯ ವೈರಾಗ್ಯಪ್ರದ ಸ್ಪರ್ಶ! ಈ ಹಿಮಾಲಯವಾಹಿನಿ ಗಂಗಾ, ಶ‍್ರೀನಗರದ ಟೆಹರಿ ಉತ್ತರಕಾಶಿ ಗಂಗೋತ್ರಿ ಮೂಲಕ ಹರಿದು ಬಂದಿದೆ.” 

 “ಗೀತಾ ಮತ್ತು ಗಂಗಾಜಲದಲ್ಲಿ ಹಿಂದೂಗಳ ಹಿಂದುತ್ವವಿದೆ. ನಾನು ಕಳೆದ ವೇಳೆ ಪಾಶ್ಚಾತ್ಯದೇಶಕ್ಕೆ ಹೋದಾಗ ಸಮಯದಲ್ಲಿ ಬೇಕಾಗಬಹುದೆಂದು ಸ್ವಲ್ಪ ಗಂಗಾಜಲವನ್ನು ತೆಗೆದುಕೊಂಡು ಹೋಗಿದ್ದೆ. ಆವಶ್ಯಕವಿದ್ದಾಗ ಕೆಲವು ಬಿಂದು ಗಂಗಾವಾರಿಯನ್ನು ತೆಗೆದುಕೊಳ್ಳುತ್ತಿದ್ದೆ. ಸೇವಿಸಿದ ತಕ್ಷಣವೇ ಆ ಪಾಶ್ಚಾತ್ಯ ಜನ ಸ್ರೋತ ನಾಗರಿಕತೆಯ ಕಲ್ಲೋಲ, ಕೋಟಿ ಕೋಟಿ ಮಾನವನ ಉನ್ಮಾದ ಪ್ರಾಯ ಪದಸಂಘಾತದ ಮಧ್ಯೆ ಮನಸ್ಸು ಸ್ಥಿರವಾಗುತ್ತಿತ್ತು. ಆ ಜನಸ್ತೋಮ, ರಜೋಗುಣೋನ್ಮಾದ, ಹೆಜ್ಜೆ ಹೆಜ್ಜೆಗೆ ಅಲ್ಲಿಯ ಜನರ ಪೋಟಾಪೋಟಿ, ಸಂಘರ್ಷಣೆ, ಆ ವಿಲಾಸಭೂಮಿ, ಅಮರಾವತಿ ಸದೃಶ ಪ್ಯಾರಿಸ್, ಲಂಡನ್, ನ್ಯೂಯಾರ್ಕ್, ಬರ್ಲಿನ್, ರೋಮ್ ಎಲ್ಲಾ ಮಾಯವಾಗುತ್ತಿತ್ತು. ‘ಹರ ಹರ ಹರ’ ಎಂಬ ಧ್ವನಿಯೊಂದೇ ಕೇಳುತ್ತಿತ್ತು. ಜನಶೂನ್ಯ ಹಿಮಾಲಯ ಪ್ರಾಂತ್ಯದಲ್ಲಿ ಹರಿವ ಸುರತರಿಂಗಿನಿ, ದೇಹದ ಅಂಗೋಪಾಂಗದಲ್ಲಿ ಸಂಚರಿಸಿ ಹರಹರ ನಿನಾದ ಮಾಡುವುದು ಕೇಳಿಬರುತ್ತಿತ್ತು.” 

 ಗಂಗಾನದಿಯ ತೀರದ ಸೌಂದರ್ಯವನ್ನು ಹೀಗೆ ವಿವರಿಸುವರು, ಗೈರಿಕವಸನ ಧಾರಿಯಾದ ಸಂನ್ಯಾಸಿಯ ಹೃದಯದಲ್ಲಿ ಹುದುಗಿರುವ ಕವಿಯ ಪರಿಚಯ ನಮಗೆ ಆಗುವುದು: 

 “ವಿದೇಶದಿಂದ ಹಿಂತಿರುಗುವಾಗ ಡೈಮಂಡ್ ಹಾರ್ಬರ್ ಮೂಲಕ ಕಲ್ಕತ್ತ ಪ್ರವೇಶಮಾಡಿದಲ್ಲದೆ ಗಂಗಾನದಿ ತೀರದ ಸೌಂದರ್ಯವನ್ನು ಅನುಭವಿಸಲು ಆಗುವುದಿಲ್ಲ. ನೀಲಿ ಆಕಾಶ, ಅದರ ಬಸಿರೊಳಗೆ ಕಾರ್ಮೋಡಗಳ ತಂಡ, ಅದರ ಕೆಳಗೆ ಹೊಂಬಣ್ಣದಂಚಿನ ಬಿಳಿಯ ಮೋಡ, ಅದರ ಕೆಳಗೆ ಗರಿಗೆದರಿ ನೂರಾರು ಚಾಮರಗಳಂತೆ ಬೀಸುತ್ತಿರುವ ತೆಂಗು, ಖರ್ಜೂರದ ಮರಗಳು. ಅದಕ್ಕಿಂತ ಸ್ವಲ್ಪ ಕೆಳಗೆ ಹಲವು ಬಗೆಯ ಹಳದಿ ಬಣ್ಣದ ಮಾವು, ಲಿಚಿ, ನೇರಳೆ, ಹಲಸಿನ ಮರಗಳು. ಎಲೆಗಳಿಂದ ತುಂಬಿ ಗಿಡದ ರೆಂಬೆಕೊಂಬೆಗಳನ್ನೇ ಕಾಣದಂತೆ ಮಾಡಿವೆ ಅವು. ಪಕ್ಕದಲ್ಲೆ ಬೊಂಬಿನ ಮರಗಳು ಚಾಮರದಂತೆ ಬೀಸುತ್ತಿವೆ. ಅದರ ಕೆಳಗೆ ಹಾಸಿದಂತೆ ಇರುವ ಅತಿ ಮೃದುವಾದ ಹಸಿರು ಹುಲ್ಲು, ಯಾರ್ಕಾಡ್, ಪರ್ಸಿಯ, ಟರ್ಕಿಸ್ಥಾನದ ರತ್ನಕಂಬಳಿಗಳನ್ನು ಕೂಡ ಇದರೊಂದಿಗೆ ಹೋಲಿಸುವುದಕ್ಕೆ ಆಗುವುದಿಲ್ಲ. ಕಣ್ಣು ನೋಡುವಷ್ಟು ದೂರಹೋದರೂ ಅಚ್ಚ ಹಳದಿಯ ಹುಲ್ಲು. ಅದನ್ನು ಯಾರೋ ಕತ್ತರಿಸಿ ಹಾಸಿದಂತಿದೆ. ನದಿಯ ತೀರದವರೆಗೆ ಇದು ಹಬ್ಬಿದೆ. ನದಿಯ ನೀರಿನ ಅಲೆಗಳು ಹಸಿರಿನೊಂದಿಗೆ ಆಡುವಂತೆ ಇವೆ. ನೆಲವೆಲ್ಲ ಹಸಿರುಮಯ. ಅದರ ಅಂಚಿನಲ್ಲಿ ಗಂಗಾನದಿಯ ಮಂದಮಧುರ ಪ್ರವಾಹ. ಒಂದು ದಿಗಂತದಿಂದ ಮತ್ತೊಂದು ದಿಗಂತದವರೆಗೆ ನಿನ್ನ ಕಣ್ಣನ್ನು ಹೊರಳಿಸಿದರೆ ಒಂದು ರೇಖೆಯಲ್ಲಿ ಅಷ್ಟೊಂದು ಬಣ್ಣಗಳ ಸಂತೆ ಕಾಣುವದು. ಒಂದೇ ಬಣ್ಣದ ಹಲವು\break ತರತಮಗಳು ಕಾಣುವುವು. ಬಣ್ಣಗಳ ಆಕರ್ಷಣೆಗೆ ನೀನು ಎಂದಾದರೂ\break ವಶನಾಗಿರುವೆಯಾ? ಪತಂಗ ದೀಪದಲ್ಲಿ ಬಿದ್ದು ಸಾಯುವ ಆಕರ್ಷಣೆಗೆ, ದುಂಬಿ ಹೂವಿನ ಸೆರೆಯೊಳಗೆ ಉಪವಾಸದಿಂದ ಪ್ರಾಣವನ್ನಾದರೂ ಬಿಡಲು ಯತ್ನಿಸುವ ಆಕರ್ಷಣೆಗೆ ನೀನು ಒಳಗಾಗಿರುವೆಯಾ?” 

 ಹೀಗೆಯೆ ಸಮುದ್ರ ಹಡಗು ಅದರಲ್ಲಿರುವ ಜನರು, ದಾರಿಯ ರೇವು ಪಟ್ಟಣಗಳು ಇವುಗಳನ್ನೆಲ್ಲ ಅಷ್ಟು ಸುಂದರವಾಗಿ ವರ್ಣಿಸುವರು. ಅದನ್ನು ನಾವು ಓದಿಯೇ ತೃಪ್ತರಾಗಬೇಕು. ಸ್ವಾಮೀಜಿ ಆ ಸಮಯದಲ್ಲಿ ಹಡಗಿನ ಯಜಮಾನ, ಅಪಾಯಗಳಿಗೆ ಹಡಗು ಸಿಕ್ಕಿರುವಾಗ ಅವನು ಹೇಗೆ ಅದನ್ನು ಎದುರಿಸಿವನು ಎಂಬುದನ್ನು ವಿವರಿಸುವಾಗ ನಾಯಕನ ಶೀಲವನ್ನೇ ನಮ್ಮ ಕಣ್ಣಮುಂದೆ ತಂದಿರುವಂತೆ ಒಂದು ವಾಕ್ಯವನ್ನು ಬರೆಯುವರು: “ಸಿರ್‍ದಾರ್ ಸರ್‍ದಾರ್”– ತಲೆಕೊಡುವವನೇ ನಾಯಕ. ನಾವೆಲ್ಲ ಬಲಿಕೊಡದೆ ಸರ್‍ದಾರ್ ಆಗಬೇಕೆಂದು ಬಯಸುವೆವು. ಇದರಿಂದ ಏನೂ ಪ್ರಯೋಜನವಾಗುವುದಿಲ್ಲವೆಂದು ಹೇಳಿ, ಓದಿದರೆ ನಮ್ಮ ರಕ್ತ ಕುದಿಯಬೇಕು, ಅಂತಹ ಜ್ವಾಲಾಮಯವಾದ ಭಾಷೆಯಲ್ಲಿ ಹೀಗೆ ಬರೆಯುವರು: 

 “ಪುರಾತನ ಆರ್ಯ ಮಹರ್ಷಿ ಕುಲಸಂಭೂತರೆಂದು ನೀವೆನಿತು ಹಿಗ್ಗಿದರೂ ಪುರಾತನ ಆರ್ಯಾವರ್ತದ ಮಹಿಮೆ ಗೌರವಗಳನ್ನು ಹಗಲಿರುಳು ನೀವೆನಿತು ಕೀರ್ತಿಸಿದರೂ, ಉಚ್ಚಕುಲಪ್ರಸೂತರೆಂದು ನೀವೆನಿತು ಉಬ್ಬಿದರೂ, ಆರ್ಯಾವರ್ತದಲ್ಲಿ ಉತ್ತಮ ವರ್ಗದವರು ಎಂದೆನಿಸಿಕೊಂಡಿರುವ ನೀವೆಲ್ಲ ಸಪ್ರಾಣರೆಂದು ತಿಳಿದಿರುವಿರೇನು? ನೀವೆಲ್ಲ ಹೆಣಗಳಾಗಿದ್ದೀರಿ! ಶತಮಾನಗಳ ಮಮ್ಮಿಗಳಾಗಿದ್ದೀರಿ! ಯಾರನ್ನು ನಿಮ್ಮ ಪೂರ್ವಿಕರು ಚಲಮಾನ ಸ್ಮಶಾನಗಳೆಂದು ತಿರಸ್ಕರಿಸಿದರೋ, ಅವರಲ್ಲಿ ಮಾತ್ರ ಭಾರತವರ್ಷದ ಪ್ರಾಣ ಸ್ವಲ್ಪವಾದರೂ ಇನ್ನೂ ಸಂಚರಿಸುತ್ತಿದೆ. ನೀವು ಮಾತ್ರ ನಿಜವಾಗಿಯೂ ಜೀವ ಶವಗಳಾಗಿದ್ದೀರಿ. ನಿಮ್ಮ ನಿವಾಸ, ಅಲ್ಲಿರುವ ಸಾಮಾನನ್ನು ನೋಡಿದರೆ, ಪ್ರಾಕ್ತನ ವಸ್ತು ಸಂಗ್ರಹಶಾಲೆಯ ನೆನಪಾಗುತ್ತದೆ. ಜೀವವಿಲ್ಲ, ನವ್ಯತೆಯಿಲ್ಲ. ನಿಮ್ಮ ನೀತಿ, ರೀತಿ ಆಚಾರ ವ್ಯವಹಾರಗಳಂತೂ ನೋಡಿದವರಿಗೆ ಅಡುಗೂಲಜ್ಜಿಯ ಕಥೆಯನ್ನು ನೆನಪಿಗೆ ತರುತ್ತವೆ. ಮುಖತಃ ನಿಮ್ಮ ಪರಿಚಯವನ್ನು ಮಾಡಿಕೊಂಡು ಹಿಂತಿರುಗಿದವನಿಗೆ ಯಾವುದೋ ಬಹು ಪುರಾತನ ಚಿತ್ರಶಾಲೆಯೊಂದನ್ನು ಸಂದರ್ಶಿಸಿ ಬಂದಂತೆ ಭಾಸವಾಗುತ್ತದೆ. ಈ ಮಾಯಾ ಪ್ರಪಂಚದಲ್ಲಿ ಮಾಯೆಗಳೆಂದರೆ ನೀವೆ! ಉಚ್ಚವರ್ಗದವರೆಂದು ಕೂಗಿಕೊಳ್ಳುವ ಭಾರತೀಯರಿರಾ, ನೀವೆಲ್ಲ ಛಾಯೆಗಳು, ಮರುಮರೀಚಿಕೆಗಳು! ನಿಮಗಿಂತ ಹೆಚ್ಚಾದ ಛಾಯೆಗಳಿಲ್ಲ; ಮರುಮರೀಚಿಕೆಗಳಿಲ್ಲ. ನೀವೆಲ್ಲ ಭೂತಕಾಲದ ಭೂತಗಳಾಗಿದ್ದೀರಿ. ನಿಮ್ಮ ವರ್ತಮಾನತೆ ಒಂದು ಹುಸಿಗನಸು, ಬರಿಯ ಭ್ರಾಂತಿ. ಪಿತ್ತದಿಂದ ಉತ್ಪತ್ತಿಯಾದ ಚಿತ್ತಗ್ಲಾನಿಗೆ ಉದಾಹರಣೆ. ನೀವು ಸೊನ್ನೆಗಳು. ಭವಿತವ್ಯದ ದೃಷ್ಟಿಗೆ ನೀವೆಲ್ಲ ಶೂನ್ಯ! ಸ್ವಪ್ನಲೋಕದ ಛಾಯಾಮೂರ್ತಿಗಳಿರಾ, ಏಕಿನ್ನೂ ಉಳಿದಿರುವಿರಿ? ಮೃತಗತಭಾರತದ ರಕ್ತಮಾಂಸ ವಿಹೀನ ಅಸ್ಥಿಪಂಜರಗಳಿರಾ, ನೀವೇಕೆ ಆದಷ್ಟು ಬೇಗನೆ ಹುಡಿಯಲ್ಲಿ ಹುಡಿಯಾಗಿ, ಗಾಳಿಯಲ್ಲಿ ಗಾಳಿಯಾಗಿ ಹೋಗಬಾರದು? ಹೌದು ನಿಮ್ಮ ಎಲುಬಿನ ಬೆರಳಿನಲ್ಲಿ ನಿಮ್ಮ ಪೂರ್ವಿಕರು ತೊಡಿಸಿದ ಅತ್ಯಮೂಲ್ಯ ರತ್ನದುಂಗರವಿನ್ನೂ ಬಿದ್ದುಹೋಗದೆ ಸಿಕ್ಕಿಕೊಂಡಿದೆ. ಕೊಳೆತುಹೋಗುತ್ತಿರುವ ನಿಮ್ಮ ಹೆಣದ ಅಸಹ್ಯ ಆಲಿಂಗನೆಯಲ್ಲಿ ಪೂರ್ವಿಕರಿತ್ತ ಐಶ್ವರ‍್ಯಮಂಜೂಷೆಯಿನ್ನೂ ನಾಶವಾಗದೆ ಇದೆ. ಇದುವರೆಗೂ ಅದನ್ನು ದಾನಮಾಡುವ ಪುಣ್ಯಾವಕಾಶ ನಿಮಗೆ ಲಭಿಸಿರಲಿಲ್ಲ. ಉದಾರ ವಿದ್ಯಾಪ್ರಚಾರವಾಗುತ್ತಿರುವ ಇಂದು, ಹೊಸಬೆಳಕು ಹಬ್ಬುತ್ತಿರುವ ಇಂದು ನಿಮಗೆ ಶುಭಾವಕಾಶ ದೊರೆತಿದೆ. ನಿಮ್ಮ ಸಂತಾನದವರಿಗೆ ಅದನ್ನು ದಾನ ಮಾಡಿ. ಆದಷ್ಟು ಬೇಗನೆ ದಾನ ಮಾಡಿ. ಶೂನ್ಯದಲ್ಲಿ ಐಕ್ಯವಾಗಿ ಕಾಣದೆ ತೊಲಗಿ ಹೋಗಿ. ನಿಮ್ಮ ಸ್ಥಾನದಲ್ಲಿ ನವೀನ ಭಾರತವೇಳಲಿ, ಕೃಷಿಕನ ದರಿದ್ರ ನಿವಾಸದಿಂದ ಹಲಹಸ್ತೆಯಾಗಿ ನವೀನ ಭಾರತಾಂಬೆ ಮೈದೋರಲಿ. ಬೆಸ್ತನ ಜೋಪಡಿಯಿಂದಾಕೆ ಮೂಡಲಿ. ಚಮ್ಮಾರ, ಜಾಡಮಾಲಿಯ ಬಡಗುಡಿಸಲಿನಿಂದ ಆಕೆ ಹೊರಹೊಮ್ಮಲಿ. ಮಳಿಗೆಯಿಂದ, ಕಾರ್ಖಾನೆಯಿಂದ, ಅಂಗಡಿಯಿಂದ, ಸಂತೆಯಿಂದ ಆಕೆ ಪ್ರತ್ಯಕ್ಷವಾಗಲಿ. ಪರ್ವತ ಕಾನನದಿಂದ, ಕಂದರವನಗಳಿಂದ ಆಕೆಯ ಮೂರ್ತಿ ರೂಪುಗೊಳ್ಳಲಿ. ಸಾಮಾನ್ಯ ಜನರು ಸಹಸ್ರಾರು ವರ್ಷದಿಂದ ಪದದಲಿತರಾಗಿದ್ದಾರೆ. ಉಚ್ಚವರ್ಗದವರ ಕ್ರೂರ ಪದಾಘಾತವನ್ನು ಗೊಣಗಾಡದೆ, ಸಹಿಸಿ ಸಹಿಸಿ ಅವರಲ್ಲೊಂದು ಅದ್ಭುತ ಸಹಿಷ್ಣುತೆ ತಲೆದೋರಿದೆ. ಅನಂತ ದುಃಖಭಾಜನರಾದುದರಿಂದ ಅಮಿತ ಶಕ್ತಿಶಾಲಿಗಳಾಗಿದ್ದಾರೆ. ಬೊಗಸೆ ಗಂಜಿಯನ್ನು ಕುಡಿದು ಜೀವಿಸುತ್ತಿದ್ದರೂ, ಅವರು ಲೋಕವನ್ನೇ ಅಲುಗಿಸಲು ಸಮರ್ಥರಾಗಿದ್ದಾರೆ. ಇನ್ನೊಂದು ಮುಷ್ಟಿ ಅನ್ನವನ್ನು ಅವರಿಗೆ ನೀಡಿ. ಅವರ ಶಕ್ತಿಯನ್ನು ಒಳಕೊಳ್ಳಲು ಜಗತ್ತು ಸಾಲದಾಗುತ್ತದೆ. ರಕ್ತ ಬೀಜನಿಗೆ ಇದ್ದಂತಹ ಅಕ್ಷಯವಿದ್ಯೆ ಅವರಲ್ಲಿದೆ. ಅದಲ್ಲದೆ ನಿರ್ಮಲ ಜೀವನದಿಂದ ಮಾತ್ರ ಸಾಧ್ಯವಾಗುವ ಅದ್ಭುತ ಓಜಸ್ಸು ಅವರಲ್ಲಿದೆ. ಅಂತಹ ಓಜಸ್ಸು ಪ್ರಪಂಚದ ಬೇರಾವ ಭಾಗದಲ್ಲಿಯೂ ಅಲಭ್ಯ. ಅಂತಹ ಶಾಂತಿ, ಅಂತಹ ಪ್ರೀತಿ, ಅಂತಹ ತೃಪ್ತಿ, ಅಂತಹ ಮೌನ, ನಿರಂತರ ಕಾರ‍್ಯದಕ್ಷತೆ, ಕಾರ್ಯ ನಿರ್ವಾಹ, ಸಮಯದಲ್ಲಿ ಅವಶ್ಯಕವಾದ ಸಿಂಹಸದೃಶ ಸಬಲತೆ– ಇದನ್ನು ಇನ್ನೆಲ್ಲಿ ಕಾಣುವಿರಿ? ಗತಕಾಲದ ಅಸ್ಥಿಪಂಜರಗಳಿರಾ, ನಿಮ್ಮ ಮುಂದೆ ನಿಂತಿದ್ದಾರೆ ನಿಮ್ಮ ಸಂತಾನದವರು– ಆವಿರ್ಭವಿಸಲಿರುವ ಭಾರತವರ್ಷದೋಪಾದಿಯಲ್ಲಿ! ನಿಮ್ಮ ಐಶ್ವರ‍್ಯ ಮಂಜೂಷೆಯನ್ನೂ ರತ್ನದ ಉಂಗುರವನ್ನೂ ಆದಷ್ಟು ಬೇಗನೆ ಅವರಿಗೆ ದಾನಮಾಡಿ, ತರುವಾಯ ನೀವು ಗಾಳಿಯಾಗಿ, ಮಾಯವಾಗಿ; ಕಿವಿಗೊಟ್ಟು ಮಾತ್ರ ಆಲಿಸುತ್ತಿರಿ. ನೀವು ಮಾಯವಾದ ಕೂಡಲೆ ಸದ್ಯಃ ಪ್ರಸೂತ ನವಭಾರತದ ಮಹಾಪ್ರಾರಂಭಗಾನ, ಮೇಘ ಗಂಭೀರ ಧ್ವನಿಯಂತೆ ವಿಶ್ವದ ದಿಕ್ತಟಗಳಿಂದ ಅನುರಣಿತವಾಗುವುದು. ಗುರುದೇವನಿಗೆ ಜಯವಾಗಲಿ.” 

