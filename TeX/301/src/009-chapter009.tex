
\chapter{ದಕ್ಷಿಣೇಶ್ವರದ ದಿನಗಳು}

ನರೇಂದ್ರ ಶ‍್ರೀರಾಮಕೃಷ್ಣರ ನೇತೃತ್ವದಲ್ಲಿ ಸಾಧನೆಯನ್ನು ಮಾಡತೊಡಗಿದ.\break ಶ‍್ರೀರಾಮಕೃಷ್ಣರು ಶಿಷ್ಯರನ್ನು ತರಬೇತು ಮಾಡುತ್ತಿದ್ದುದು ಅಪೂರ್ವವಾದ ರೀತಿ. ಅವರು ಶಾಸ್ತ್ರ ಪುರಾವೆಗಳ ಮೇಲೆ ಪ್ರವಚನಾದಿಗಳನ್ನು ಕೊಟ್ಟು ಶಿಷ್ಯರಿಗೆ ಉಪದೇಶ ಮಾಡುತ್ತಿರಲಿಲ್ಲ. ಅವರು ಸಾಕ್ಷಾತ್ತಾಗಿ ಬ್ರಹ್ಮದರ್ಶನವನ್ನು ಪಡೆದವರು. ಹೇಗೆ ಶಿಷ್ಯರನ್ನು ಬ್ರಹ್ಮದರ್ಶನ ಪಡೆಯುವುದಕ್ಕೆ ಯೋಗ್ಯರನ್ನಾಗಿ ಮಾಡಬೇಕೋ ಅದು ಅವರಿಗೆ ಚೆನ್ನಾಗಿ ಗೊತ್ತಿತ್ತು. ಕೇವಲ ದರ್ಶನ ಸ್ಪರ್ಶನ ಇಚ್ಛೆಯಿಂದಲೇ ಅವರು ಒಬ್ಬನಲ್ಲಿ ಆಧ್ಯಾತ್ಮಿಕತೆಯನ್ನು ಜಾಗ್ರತಗೊಳಿಸಬಲ್ಲವರಾಗಿದ್ದರು. ಆದರೂ ಶಿಷ್ಯರಿಗೆ ಸಾಧನೆ ಮಾಡಿ ಸ್ವಪ್ರಯತ್ನದ ಮೇಲೆ ನಿಲ್ಲಿ ಎಂದು ಹೇಳುತ್ತಿದ್ದರು. ಆಧ್ಯಾತ್ಮಿಕ ಜೀವನದಲ್ಲಿ ಕೆಲವು ವೇಳೆ ಗುರುವಿನ ಇಚ್ಛಾಮಾತ್ರದಿಂದ ಕೆಲವು ಅನುಭವಗಳಾಗಬಹುದು. ಆದರೆ ಅದು ಮಿಂಚಿನಂತೆ ಬಂದು ಹೋಗುವುದು. ಅದನ್ನು ನಾವೇ ಸ್ವತಃ ಪಡೆಯಬೇಕಾದರೆ, ತಾತ್ಕಾಲಿಕ ಅನುಭವವನ್ನು ನಮ್ಮ ಜೀವನದ ಅನುಗಾಲ ಅನುಭವವನ್ನಾಗಿ ಮಾಡಿಕೊಳ್ಳಬೇಕಾದರೆ, ನಾವು ಅದಕ್ಕೆ ಸಾಧನೆ ಮಾಡಬೇಕು. ತಮ್ಮ ನೇತೃತ್ವದಲ್ಲಿ ಅವರಿಂದ ಸಾಧನೆ ಮಾಡಿಸಿದರು. ಧ್ಯಾನ ಜಪತಪಾದಿಗಳನ್ನು ಮಾಡುವಾಗ ಅವರಿಗೆ ಬಂದೊದಗುವ ಅಡಚಣೆಗಳನ್ನು ಅರಿತು ಅವುಗಳಿಂದ ಪಾರಾಗುವ ಸಲಹೆಗಳನ್ನು ಕೊಡುತ್ತಿದ್ದರು.

ನರೇಂದ್ರನ ಮನೆಯ ಹತ್ತಿರ ಒಂದು ಜೂಟ್ ಮಿಲ್ಲು ಇತ್ತು. ಅದರಿಂದ ಆಗುವ ಗದ್ದಲದಿಂದ ಮನಸ್ಸಿನ ಏಕಾಗ್ರತೆಗೆ ಭಂಗ ಬರುತ್ತಿತ್ತು. ಏನು ಮಾಡಬೇಕು ಎಂದು ಶ‍್ರೀರಾಮಕೃಷ್ಣರನ್ನು ಕೇಳಿದನು. ಶ‍್ರೀರಾಮಕೃಷ್ಣರು ಆ ಗದ್ದಲದ ಮೇಲೆಯೇ ಏಕಾಗ್ರತೆ ಮಾಡು ಎಂದರು. ನರೇಂದ್ರನಿಗೆ ಆ ಸಲಹೆ ಫಲಕಾರಿಯಾಗಿ ಕಂಡಿತು. ಒಂದು ಸಲ ಧ್ಯಾನದಲ್ಲಿ ನರೇಂದ್ರ ತನ್ಮಯನಾಗಿ ದೇಹದ ಅರಿವಿಲ್ಲದೇ ಇರಬೇಕೆಂದು ಆಶಿಸಿದಾಗ ಅದು ಸಾಧ್ಯವಾಗಲಿಲ್ಲ. “ಇದಕ್ಕೆ ಏನು ಮಾಡಬೇಕು?” ಎಂದು ಶ‍್ರೀರಾಮಕೃಷ್ಣರನ್ನು ಕೇಳಿದಾಗ ಅವರು ನರೇಂದ್ರನ ಭ್ರೂಮಧ್ಯೆ ಚುಚ್ಚಿ “ಇಲ್ಲಿ ಧ್ಯಾನ ಮಾಡು” ಎಂದರು. ಅವರು ತೋರಿದ ಕಡೆಯಲ್ಲಿಯೇ ಆ ವೇದನೆಯ ಮೇಲೆ ಧ್ಯಾನಮಾಡಿ ತನ್ನ ಇಷ್ಟಾರ್ಥ ಸಿದ್ಧಿಯನ್ನು ಪಡೆದ.

ಶ‍್ರೀರಾಮಕೃಷ್ಣರು ಒಮ್ಮೆ ದೇವಿಯನ್ನು ಪ್ರಾರ್ಥನೆ ಮಾಡುತ್ತಿದ್ದಾಗ ತಮಗೆ ಆದ ಅನುಭವಗಳನ್ನು ಅನುಮಾನಿಸುವಂತಹವನನ್ನು ಕಳುಹಿಸು ಎಂದು ಬೇಡಿಕೊಂಡಿದ್ದರು. ಜಗನ್ಮಾತೆ ಅದಕ್ಕಾಗಿಯೇ ನರೇಂದ್ರನನ್ನು ಕಳುಹಿಸಿದಂತೆ ಇತ್ತು. ಶ‍್ರೀರಾಮಕೃಷ್ಣರ ಶಿಷ್ಯರಲ್ಲಿ ನರೇಂದ್ರನೇ ಒಂದು ಪಂಗಡಕ್ಕೆ ಸೇರಿದವನು. ಅವರ ಇತರ ಶಿಷ್ಯರು ಒಂದು ಪಂಗಡಕ್ಕೆ ಸೇರಿದವರು. ಇತರರು ಶ‍್ರೀರಾಮಕೃಷ್ಣರು ಹೇಳಿದ ಅನುಭವಗಳನ್ನು ಆಡಿದ ಮಾತುಗಳನ್ನು ಎಳ್ಳಷ್ಟೂ ಅನುಮಾನಿಸುತ್ತಿರಲಿಲ್ಲ. ಏಕೆಂದರೆ ಶ‍್ರೀರಾಮಕೃಷ್ಣರು ಭ್ರಾಂತಿಪುರುಷರಲ್ಲ, ಜೀವರಿಗೆ ಏನೇನೊ ಹೇಳಿ ತಪ್ಪು ದಾರಿಗೆ ಎಳೆಯುವವರಲ್ಲ ಎಂಬ ಭಾವನೆ ಬಂದು ಹೋಗುತ್ತಿತ್ತು. ನರೇಂದ್ರನಿಗಾದರೊ ಶ‍್ರೀರಾಮಕೃಷ್ಣರ ಮೇಲೆ ಭಕ್ತಿ ವಿಶ್ವಾಸ ಇತರರಿಗೆ ಇರುವಂತೆಯೇ ಇತ್ತು. ಆದರೆ ಅವನು ಬೌದ್ಧಿಕವಾಗಿ ತಿಳಿಯಲೆತ್ನಿಸಿದನು, ಅದಕ್ಕೆ ಒಂದು ವಿವರಣೆಯನ್ನು ಕೊಡಲು ಯತ್ನಿಸಿದನು. ಆತ್ಮತೃಪ್ತಿಗೆ ಯಾವ ವಿಚಾರ ತರ್ಕವೂ ಬೇಕಿಲ್ಲ. ಆದರೆ ಇನ್ನೊಬ್ಬರಿಗೆ ಆದನ್ನು ಹೇಳಬೇಕಾದರೆ ಅದನ್ನು ಎಲ್ಲಾ ದೃಷ್ಟಿಕೋನಗಳಿಂದಲೂ ನೋಡಬೇಕಾಗುವುದು. ಶ‍್ರೀರಾಮಕೃಷ್ಣರೇ ಹೇಳುತ್ತಿದ್ದರು, ಒಬ್ಬನು ಆತ್ಮಹತ್ಯೆಯನ್ನು ಮಾಡಿಕೊಳ್ಳಬೇಕಾದರೆ ಒಂದು ಸೂಜಿ ಸಾಕು, ಒಂದು ತುಂಡು ಹಗ್ಗ ಸಾಕು. ಆದರೆ ಅವನು ಯುದ್ಧಕ್ಕೆ ಹೋಗಿ ಇತರರನ್ನು ಕೊಲ್ಲಬೇಕಾದರೆ, ಕತ್ತಿ ಗುರಾಣಿ ಮತ್ತು ಇನ್ನೂ ಯಾವ ಯಾವ ಅಸ್ತ್ರಶಸ್ತ್ರಗಳಿವೆಯೋ ಅವೆಲ್ಲಾ ಬೇಕಾಗುವುದು ಎಂದು. ನರೇಂದ್ರ ಅನಂತರ ಶ‍್ರೀರಾಮಕೃಷ್ಣರ ಅನುಭವವನ್ನು ಯುಕ್ತಿಪೂರ್ವಕವಾಗಿ ಎಲ್ಲರಿಗೂ ವಿವರಿಸಬೇಕಾಗಿತ್ತು. ವಿಧಿ ಅವನನ್ನು ಇದಕ್ಕೆ ಸನ್ನದ್ಧನಾಗಿರುವಂತೆ ಮಾಡಿತು. ಶ‍್ರೀರಾಮಕೃಷ್ಣರು ನರೇಂದ್ರನ ಪ್ರವೃತ್ತಿಯನ್ನು ನೋಡಿ ಸಹನೆಗೆಡಲಿಲ್ಲ. ಅವನಿಗೆ ಆ ದಾರಿಯಲ್ಲಿಯೇ ಮುಂದುವರಿಯುವಂತೆ ಉತ್ತೇಜನ ಕೊಟ್ಟರು.

ನರೇಂದ್ರನ ಮಹಿಮೆ ನರೇಂದ್ರನಿಗೇ ಗೊತ್ತಿಲ್ಲದೇ ಇದ್ದರೂ ಶ‍್ರೀರಾಮಕೃಷ್ಣರು ಅದನ್ನು ಸ್ಪಷ್ಟವಾಗಿ ನೋಡುತ್ತಿದ್ದರು. ಒಂದು ಸಲ ಶಿಷ್ಯನೊಬ್ಬನಿಗೆ ನರೇಂದ್ರನ ಶಕ್ತಿಯನ್ನು ಕುರಿತು ಹೀಗೆ ಹೇಳುವರು: “ಇಲ್ಲಿ ನೋಡು ನರೇಂದ್ರನನ್ನು! ಎಂತಹ ಜ್ಞಾನ ಅವನಲ್ಲಿದೆ! ಇದೊಂದು ಮೇರೆಯಿಲ್ಲದ ಕಾಂತಿಯಂತೆ ಇದೆ. ಮಹಾಮಾಯೆಯೇ ನರೇಂದ್ರನ ಸಮೀಪಕ್ಕೆ ಹತ್ತು ಅಡಿಗಳಿಗಿಂತ ಹೆಚ್ಚಾಗಿ ಬರಲಾರಳು. ಅವಳು ಯಾವ ಮಹಿಮೆಯನ್ನು ಅವನಲ್ಲಿ ಇಟ್ಟಿರುವಳೋ ಅದರಿಂದಲೇ ಅವಳು ಬದ್ಧಳಾಗಿರುವಳು.” ಆಗ ಶ‍್ರೀರಾಮಕೃಷ್ಣರೇ ದೇವಿಗೆ ಪ್ರಾರ್ಥನೆ ಮಾಡಿದರು, ಕಾಂತಿಯನ್ನು ಸ್ವಲ್ಪ ತಗ್ಗಿಸೆಂದು. ಅದು ಜಾಸ್ತಿಯಾಗಿ ನರೇಂದ್ರ ತನ್ನ ವ್ಯಕ್ತಿತ್ವವನ್ನು ಮರೆತು ಪರಬ್ರಹ್ಮನಲ್ಲಿ ನಿರತನಾದರೆ ಅದರಿಂದ ಪ್ರಪಂಚಕ್ಕೆ ಬಹಳ ನಷ್ಟವಾಗುವುದೆಂದು ಭಾವಿಸಿದರು.

ಒಂದು ದಿನ ನರೇಂದ್ರ ಮತ್ತು ಇತರರಿಗೆ ದೇವರು ಸಗುಣನೆ, ನಿರ್ಗುಣನೆ, ಅವನು ಅವತಾರ ಮಾಡುವುದು ನಿಜವೆ ಅಲ್ಲವೆ ಎಂಬ ವಿಷಯದಲ್ಲಿ ಬಿರುಸಾಗಿ ವಾದ ನಡೆಯಿತು. ನರೇಂದ್ರ ಅವರನ್ನೆಲ್ಲ ವಾದದಲ್ಲಿ ಸದೆ ಬಡಿದಿದ್ದ. ಆ ಸಮಯಕ್ಕೆ ಶ‍್ರೀರಾಮಕೃಷ್ಣರು ಅಲ್ಲಿಗೆ ಬಂದರು. ಇವರ ವಾದದ ವಾತಾವರಣವನ್ನು ಗ್ರಹಿಸಿ ಒಂದು ಹಾಡನ್ನು ಹೇಳಿದರು. ಅದರ ಭಾವ “ಎಲೈ ಮನಸ್ಸೇ, ಆ ಪರಮಾತ್ಮನನ್ನು ಕುಡುಕ ಕತ್ತಲಲ್ಲಿ ಹುಡುಕುತ್ತಿರುವಂತಿದೆಯಲ್ಲ” ಎಂಬುದು. ಆ ಹಾಡು ಮುಂದುವರಿಯುವುದು: “ಷಡ್‍ದರ್ಶನಗಳನ್ನು ಹುಡುಕಿದರೂ ಅವನು ಸಿಕ್ಕುವುದಿಲ್ಲ. ಅವನು ತಂತ್ರದಲ್ಲಿಯೂ ಇಲ್ಲ, ವೇದದಲ್ಲಿಯೂ ಇಲ್ಲ. ಪರಮ ಪ್ರೇಮವೆಲ್ಲಿರುವುದೊ ಅವನು ಅಲ್ಲಿರುವನು.” ಇದನ್ನು ಹೇಳಿ ಶ‍್ರೀರಾಮಕೃಷ್ಣರು ಸಮಾಧಿಮಗ್ನರಾದರು. ವಾದದ ಧೂಳಿನಲ್ಲಿ ಸಿಕ್ಕದೇ ಇದ್ದುದನ್ನು ಪರಮಹಂಸರ ಮೊಗದ ಮಂದಹಾಸ ಸೂಚಿಸುತ್ತಿತ್ತು.

ಶ‍್ರೀರಾಮಕೃಷ್ಣರು ತಮ್ಮ ಶಿಷ್ಯರಿಗೆ ಆಧ್ಯಾತ್ಮಿಕ ಜೀವನದ ವಿಷಯದಲ್ಲಿ ಚರ್ಚೆಯನ್ನು ಮಾಡಲು ಪ್ರೋತ್ಸಾಹ ಕೊಡುತ್ತಿದ್ದರು. ಅದರಿಂದ ಶಿಷ್ಯರ ಮನಸ್ಸಿನಲ್ಲಿ ಇರುವುದೇನು, ಅವರು ಯಾವ ಮೆಟ್ಟಲಿನಲ್ಲಿ ಇರುವರು ಎಂಬುದನ್ನು ತಿಳಿದುಕೊಳ್ಳಲು ಇವರಿಗೆ ಸಾಧ್ಯವಾಗುತ್ತಿತ್ತು. ಅನಂತರ ಶಿಷ್ಯರು ಪ್ರಪಂಚದಲ್ಲಿ ಬೋಧಿಸಬೇಕಾದರೆ ಜನರ ಪ್ರಶ್ನೆಗಳನ್ನು ಎದುರಿಸುವುದಕ್ಕೆ ಅದರಿಂದ ಸಹಾಯಕವಾಗುವುದು ಎಂದು ತಿಳಿದಿದ್ದರು. ಶಿಷ್ಯರು ಚರ್ಚೆಮಾಡುವಾಗ ಯಾವಾಗ ಅಡ್ಡದಾರಿಗೆ ಇಳಿಯುತ್ತಿದ್ದರೊ ಆಗ ಅವರನ್ನು ತಿದ್ದುತ್ತಿದ್ದರು. ಒಂದು ಸಲ ನರೇಂದ್ರ ಶ್ರದ್ಧೆಯನ್ನು ಎರಡು ಭಾಗ ಮಾಡಿ ಚರ್ಚಿಸುತ್ತಿದ್ದ: ಒಂದು ಅಂಧಶ್ರದ್ಧೆ ಮತ್ತೊಂದು ವಿಚಾರದಿಂದ ಕೂಡಿರುವ ಶ್ರದ್ಧೆ ಎಂದು. ಶ‍್ರೀರಾಮಕೃಷ್ಣರು ಇದನ್ನು ಕೇಳಿ ನರೇಂದ್ರನ ತಪ್ಪನ್ನು ತಿದ್ದಿದರು. ಶ್ರದ್ಧೆ ಯಾವಾಗಲೂ ಅಂಧವೇ. ವಿಚಾರದಿಂದ ಅದು ಶ್ರದ್ಧೆಯಾಗುವುದಿಲ್ಲ. ಅದು ಜ್ಞಾನವಾಗುತ್ತದೆ ಎಂದರು. ಆದರೆ ತರ್ಕದ ಮೂಲಕ ಇದು ಇದೆ ಅಥವಾ ಇರಬೇಕು ಎಂದು ನಿರ್ಧರಿಸಬಹುದು ಅಷ್ಟೆ. ಸುಮ್ಮನೆ ಯುಕ್ತಿಯ ಆಧಾರದ ಮೇಲೆ ಇದೆ ಎಂದು ಊಹಿಸಿದವನಿಗೂ ಮತ್ತು ಅದನ್ನು ಯಾವ ವಿಚಾರವೂ ಇಲ್ಲದೆ ನಂಬುವವನಿಗೂ ಶ‍್ರೀರಾಮಕೃಷ್ಣರ ದೃಷ್ಟಿಯಲ್ಲಿ ವ್ಯತ್ಯಾಸ ಅಷ್ಟು ಇರಲಿಲ್ಲ. ಅವರು ಬೆಲೆ ಕೊಡುತ್ತಿದ್ದುದು ಅನುಭವಕ್ಕೆ.

ಶ‍್ರೀರಾಮಕೃಷ್ಣರು ಜಾತಿಗೆ ವಿರೋಧವಾಗಿ ಆಗಲಿ, ಪರವಾಗಿ ಆಗಲಿ ಬೋಧಿಸುತ್ತಿರಲಿಲ್ಲ. ಅವರೇ ಶ್ರೇಷ್ಠ ಬ್ರಾಹ್ಮಣರ ಕುಲದಿಂದ ಬಂದರೂ, ಪರೆಯನ ಮನೆಗೆ ಹೋಗಿ ಅವರು ಅವನ ಮನೆಯ ಕಕ್ಕಸನ್ನು ಗುಡಿಸಿದರು. ಆಧ್ಯಾತ್ಮಿಕ ಜೀವನಕ್ಕೂ ಜಾತಿಗೂ ಯಾವ ಸಂಬಂಧವೂ ಇರಲಿಲ್ಲ. ಅವರ ದೃಷ್ಟಿಯಲ್ಲಿ ಜಾತಿ ಕೇವಲ ಒಂದು ಸಮಾಜ ರಚನೆಗೆ ಸಂಬಂಧಪಟ್ಟದ್ದು. ಅವರ ಶಿಷ್ಯರು ಹಿಂದೂ ಧರ್ಮಕ್ಕೆ ಸೇರಿದ ಎಲ್ಲಾ ಕಾರ‍್ಯಕ್ಷೇತ್ರಗಳಿಂದಲೂ ಬಂದಿದ್ದವರು. ಅವರ ಪಟ್ಟಶಿಷ್ಯನಾದ ನರೇಂದ್ರ ಕಾಯಸ್ತ ಕುಲಕ್ಕೆ ಸೇರಿದವನು, ಬ್ರಾಹ್ಮಣನಲ್ಲ. ಅವರ ದೃಷ್ಟಿಯಲ್ಲಿ ಒಬ್ಬ ಆಧ್ಯಾತ್ಮಿಕ ಜೀವನದಲ್ಲಿ ಬೆಳೆಯುತ್ತ ಹೋದಂತೆ ಜಾತಿ ಮತಗಳಿಗೆ ಸೇರಿದ ಶೃಂಖಲೆಗಳೆಲ್ಲ ತಾವೇ ಬಿದ್ದು ಹೋಗುತ್ತವೆ ಎಂಬುದು. ತೆಂಗಿನ ಮರ ಬೆಳೆದು ಮೇಲೆ ಮೇಲಕ್ಕೆ ಹೋದಂತೆಲ್ಲ ಕೆಳಗಿರುವ ಗರಿಗಳು ಬಿದ್ದು ಬಿದ್ದು ಹೋಗುತ್ತವೆ. ನಾವು ಮರವನ್ನು ಬೆಳೆಸುವ ಕಡೆಗೆ ಗಮನ ಕೊಡಬೇಕೇ ಹೊರತು ಗರಿಯನ್ನು ಕಿತ್ತು ಹಾಕಿದರೆ ಬೇಗ ಬೆಳೆಯುವುದಿಲ್ಲ. ಅನೇಕ ವೇಳೆ ಕೇವಲ ಸಮಾಜ ಸುಧಾರಕರು ಬಾಹ್ಯದ ಕಡೆಗೆ ಗಮನ ಕೊಟ್ಟು ಅದಕ್ಕಾಗಿ ಕಾದಾಡುತ್ತಾರೆ. ಅದರಿಂದ ಕಾವು ಧೂಳು, ಆದರೆ ಬೆಳಕಿಲ್ಲ. ಶ‍್ರೀರಾಮಕೃಷ್ಣರ ದೃಷ್ಟಿಯಾದರೋ ಸಮಸ್ಯೆಯ ಆಳಕ್ಕೆ ಹೋಗುತ್ತಿತ್ತು. ನರೇಂದ್ರ ತನ್ನ ಕಣ್ಣ ಮುಂದೆ ಶ‍್ರೀರಾಮಕೃಷ್ಣರ ಮನಸ್ಸು ಹೇಗೆ ಕೆಲಸ ಮಾಡುತ್ತಿತ್ತು ಎಂಬುದನ್ನು ನೋಡುತ್ತಿದ್ದ, ಇದಕ್ಕಿಂತ ದೊಡ್ಡ ವಿದ್ಯಾಭ್ಯಾಸವಿಲ್ಲ.

ಶ‍್ರೀರಾಮಕೃಷ್ಣರು ಆತ್ಮಸಾಕ್ಷಾತ್ಕಾರದ ವಿಷಯದಲ್ಲಿ ಹೇಳಿದ ಮಾತುಗಳು ಉದಾರವಾಗಿವೆ, ವಿಶಾಲವಾಗಿವೆ. ದೇವರನ್ನು ಹೇಗೆ ಪ್ರಾರ್ಥಿಸಬೇಕು ಎಂದು ಯಾರೋ ಕೇಳಿದರು. ಅದಕ್ಕೆ ಶ‍್ರೀರಾಮಕೃಷ್ಣರು “ನಿನಗೆ ಮನಸ್ಸು ತೋರಿದ ರೀತಿಯಲ್ಲಿ ಅವನನ್ನು ಪ್ರಾರ್ಥಿಸು. ಏಕೆಂದರೆ ದೇವರು ಇರುವೆಯ ಕಾಲಿನ ಸಪ್ಪಳವನ್ನು ಕೂಡ ಆಲಿಸಬಲ್ಲ” ಎಂದರು. ದೇವರನ್ನು ಹೇಗೆ ಪಡೆಯುವುದು ಎಂದು ಕೇಳಿದ್ದಕ್ಕೆ, ಕಾಮಕಾಂಚನ ತ್ಯಾಗದಿಂದ ಎಂದು ಹೇಳಿದರು. ದೇವರನ್ನು ಯಾವುದೋ ಆಕಾರದಲ್ಲಿ ಚಿಂತನೆ ಮಾಡಿದರೆ, ಯಾವುದೋ ಹೆಸರಿನಿಂದ ಅವನನ್ನು ಕರೆದರೆ, ಯಾವುದೋ ಮಹಾತ್ಮನಲ್ಲಿ ಶರಣಾಗತನಾದರೆ, ಅವನು ಸಿಕ್ಕಿಬಿಡುವನು ಎಂದು ಹೇಳುತ್ತಿರಲಿಲ್ಲ. ಸಮಸ್ಯೆಯ ಆಳಕ್ಕೆ ಹೋಗುತ್ತಿದ್ದರು ಅವರು. ಜೀವಿಯನ್ನು ಪ್ರಪಂಚಕ್ಕೆ ಕಟ್ಟಿಹಾಕಿರುವುದು ಅವನಲ್ಲಿರುವ ಕಾಮಕಾಂಚನಾಸಕ್ತಿ. ದೇವರನ್ನು ಕಾಣದಂತೆ ಮಾಡಿರುವುದೇ ಇದು. ಹೇಗೋ ಇದರಿಂದ ಪಾರಾದರೆ ಸಾಕು, ಅವನಿಗೆ ಸಾಕ್ಷಾತ್ಕಾರ ದೊರಕುತ್ತದೆ ಎಂಬುದು ಅವರ ಮತ. ನರೇಂದ್ರ ಮತ್ತು ಇತರ ಶಿಷ್ಯರಿಗೆ ಶ‍್ರೀರಾಮಕೃಷ್ಣರು, ತಾವು ಮಾಡಿದ ಸಾಧನೆಯಲ್ಲಿ ಹದಿನಾರನೆ ಒಂದು ಪಾಲನ್ನು ಮಾಡಿದರೆ ಸಾಕು ಭಗವಂತನನ್ನು ಪಡೆಯಬಹುದು ಎಂದು ಹೇಳುತ್ತಿದ್ದರು. ದೇವರು ಸಾಕಾರನೆ ಅಥವಾ ನಿರಾಕಾರನೆ ಎಂದು ಕೇಳಿದಾಗ ಶ‍್ರೀರಾಮಕೃಷ್ಣರು ವಿಚಾರದ ಉಗ್ರಾಣಕ್ಕೆ ಹೋಗಿ ಯಾವುದೋ ಕೆಲವು ಅಣಿಮಾಡಿದ ಸಿದ್ಧಾಂತ ತರುತ್ತಿರಲಿಲ್ಲ. ಅವರು ಅನುಭವದ ಆಳಕ್ಕೆ ಮುಳುಗುತ್ತಿದ್ದರು. ಅಲ್ಲಿಂದ ತಂದುದನ್ನು ಶಿಷ್ಯರ ಎದುರಿಗೆ ಇಡುತ್ತಿದ್ದರು. ದೇವರು ಸಾಕಾರನೂ ಹೌದು, ನಿರಾಕಾರನೂ ಹೌದು, ಅವನು ಇವುಗಳಿಗೆ ಅತೀತನೂ ಹೌದು. ಅವನು ನಮಗೆ ತಿಳಿದಿರುವ ಎಲ್ಲ ಯುಕ್ತಿ ಮತ್ತು ಸಿದ್ಧಾಂತಗಳಿಗೆ ಅತೀತ. ಅವನು ಜೀವಿಯ ಹೃದಯಾಂತರದಲ್ಲೆ ವ್ಯಕ್ತವಾಗುತ್ತಾನೆ. ಭಕ್ತರನ್ನು ಉದ್ಧಾರ ಮಾಡಲು ಅವನು ಅವರಿಗೆ ಪ್ರಿಯವಾದ ಯಾವ ಆಕಾರವನ್ನು ಬೇಕಾದರೂ ಧರಿಸುತ್ತಾನೆ. ಅವನನ್ನು ಯಾವುದೋ ಒಂದು ಶಾಸ್ತ್ರದಿಂದ ಅಥವಾ ಒಂದು ದೇವಸ್ಥಾನದಿಂದ ಬಂಧಿಸಲು ಆಗುವುದಿಲ್ಲ ಎಂದು ಹೇಳುತ್ತಿದ್ದರು. ವಿಗ್ರಹಾರಾಧನೆ ಸರಿಯೆ ತಪ್ಪೆ ಎಂದು ಕೇಳುವುದು ಕೆಲಸಕ್ಕೆ ಬಾರದ ಪ್ರಶ್ನೆಗಳು ಅವರ ದೃಷ್ಟಿಯಲ್ಲಿ. ವಿಗ್ರಹಾರಾಧನೆ ಭಗವತ್ಸಾಕ್ಷಾತ್ಕಾರಕ್ಕೆ ಸಹಾಯ ಮಾಡಿದರೆ ಒಳ್ಳೆಯದು. ಶ‍್ರೀರಾಮಕೃಷ್ಣರು ಮಾರ್ಗದ ಕಡೆಗೆ ಅಷ್ಟು ಗಮನವನ್ನು ಕೊಡುತ್ತಿರಲಿಲ್ಲ. ಒಬ್ಬ ಯಾವ ಮಾರ್ಗವನ್ನಾದರೂ ಅನುಸರಿಸಲಿ, ಚಿಂತೆಯಿಲ್ಲ. ಆದರೆ ಅದರ ಹಿಂದೆ ತಾನು ಭಗವಂತನನ್ನು ಪಡೆಯಬೇಕೆಂಬ ತೀವ್ರ ಆಕಾಂಕ್ಷೆ ಇರಬೇಕು. ಈ ಆಕಾಂಕ್ಷೆಯೇ ಅವನನ್ನು ಗುರಿಯ ಕಡೆಗೆ ಮುಂದು ಮುಂದಕ್ಕೆ ತೆಗೆದುಕೊಂಡು ಹೋಗುವುದು. ಒಬ್ಬನು ತಪ್ಪು ದಾರಿಯನ್ನು ಹಿಡಿದಿದ್ದರೂ, ಅವನಲ್ಲಿ ಗುರಿಯನ್ನು ಸೇರಬೇಕೆಂಬ ಆಕಾಂಕ್ಷೆ ತೀವ್ರವಾಗಿದ್ದರೆ ಅವನು ಸರಿಯಾದ ಹಾದಿಗೆ ಕ್ರಮೇಣ ಬಂದು ಗುರಿ ಸೇರುತ್ತಾನೆ ಎಂಬುದನ್ನು ವಿವರಿಸಲು ಒಂದು ಉದಾಹರಣೆಯನ್ನು ಕೊಡುತ್ತಿದ್ದರು. ಒಬ್ಬ ಕಾಶಿಗೆ ಹೋಗಬೇಕು ಎಂದು ಮನೆಬಿಟ್ಟು ಹೊರಡುವನು. ಕಾಶಿ ಉತ್ತರದಲ್ಲಿದ್ದರೆ ಇವನು ದಕ್ಷಿಣಕ್ಕೆ ಹೋಗುವನು. ಆದರೆ ಸ್ವಲ್ಪ ಕಾಲ ನಡೆದ ಮೇಲೆ ಯಾರೊ ಪಥಿಕರು ಅವನನ್ನು ಸರಿಯಾದ ದಾರಿಗೆ ಬಿಡುವರು. ಅದರಂತೆಯೇ ಶ‍್ರೀರಾಮಕೃಷ್ಣರು, ಒಬ್ಬ ಯಾವ ಮಾರ್ಗವನ್ನಾದರೂ ಅನುಸರಿಸಲಿ, ಆ ಮಾರ್ಗದಲ್ಲಿ ನಡೆಯುವವನಿಗೆ ಅತ್ಯಂತ ಮುಖ್ಯವಾಗಿ ಬೇಕಾದುದು, ದಾರಿಯನ್ನು ನಡೆದು ಏನಾದರೂ ಆಗಲಿ ತಾನು ಗುರಿಯನ್ನು ಸೇರಬೇಕು ಎಂಬ ತೀವ್ರ ಆಕಾಂಕ್ಷೆ, ಎಂಬುದನ್ನು ಒತ್ತಿ ಹೇಳುತ್ತಿದ್ದರು.

ಶ‍್ರೀರಾಮಕೃಷ್ಣರು ಪಾಪವನ್ನು ಒತ್ತಿ ಹೇಳುತ್ತಿರಲಿಲ್ಲ. ಜೀವಿಗೆ ಹುಟ್ಟುವಾಗ ಒಂದು ಮಿತಿ ಇದೆ. ಅವನೊಡನೆ ಹೀನ ಸಂಸ್ಕಾರಗಳು ಬರುತ್ತವೆ. ಆದರೆ ಅದು ಅವನ ತಾತ್ಕಾಲಿಕ ಅವಸ್ಥೆ. ಅದರ ಕಡೆಗೆ ಏತಕ್ಕೆ ಗಮನ ಕೊಡಬೇಕು? ಆ ತಾತ್ಕಾಲಿಕ ಅವಸ್ಥೆಯ ಹಿಂದೆ ಇರುವ ಪವಿತ್ರವಾದ ಅವಸ್ಥೆಯ ಕಡೆ ಗಮನ ಕೊಡಬೇಕು. ಶ‍್ರೀರಾಮಕೃಷ್ಣರು ಅದಕ್ಕೇ, ಒಬ್ಬನು ತಾನು ಪಾಪಿ ಎನ್ನುತ್ತಿದ್ದರೆ ಅವನು ಪಾಪಿಯೇ ಆಗುತ್ತಾನೆ; ನಾನು ಭಗವಂತನ ಮಗು, ನನ್ನನ್ನು ಯಾವುದು ಬಂಧಿಸಬಲ್ಲದು ಎಂದು ಭಗವಂತನ ಮಂಗಳ ನಾಮವನ್ನು ನೆನೆಯುತ್ತಿದ್ದರೆ ಅವನು ಕೂಡ ದೇವರಂತೆಯೇ ಆಗುತ್ತಾನೆ ಎಂದು ಹೇಳುತ್ತಿದ್ದರು. ಎರಡೂ ನಿಜವೇ ಇರಬಹುದು. ಜೀವಿಯನ್ನು ಪಾಪ ಮುತ್ತಿದೆ. ಆ ಪಾಪದ ಹಿಂದೆ ಪುಣ್ಯವಿದೆ. ಪಾಪವನ್ನೇ ಒತ್ತಿ ಹೇಳುತ್ತಿದ್ದರೆ ಅದು ಬಿಟ್ಟು ಹೋಗುವುದಿಲ್ಲ. ಪುಣ್ಯದ ಮೇಲೆ ನಮ್ಮ ಗಮನವನ್ನು ತಂದರೆ ಪಾಪ ತಾನಾಗಿಯೇ ಬಿಟ್ಟುಹೋಗುತ್ತದೆ. ಶ‍್ರೀರಾಮಕೃಷ್ಣರು ಯಾವಾಗಲೂ ಒತ್ತಿ ಹೇಳುತ್ತಿದ್ದುದು ಈ ಪುಣ್ಯದ ಭಾವವನ್ನು. ಒಂದು ಕೋಣೆಯಲ್ಲಿ ಹಲವಾರು ವರುಷಗಳಿಂದ ಕತ್ತಲೆ ಇದ್ದರೆ, ಕತ್ತಲೆ ಎಂದು ಹೇಳುತ್ತಿದ್ದರೆ ಬೆಳಕು ಬರುವುದಿಲ್ಲ. ಕದ ತೆಗೆಯಬೇಕು, ಆಗ ಬೆಳಕು ಬರುವಾಗ ನಿಧಾನವಾಗಿ ಬರುವುದಿಲ್ಲ, ಒಂದೇ ಸಲ ಬರುತ್ತದೆ.

ಒಂದು ಸಲ ಶ‍್ರೀರಾಮಕೃಷ್ಣರು ವೈಷ್ಣವ ಸಂಪ್ರದಾಯವನ್ನು ಕುರಿತು ‘ಅದರಲ್ಲಿ ಮೂರು ವಿಷಯಗಳಿವೆ; ಮೊದಲನೆಯದು ದೇವರ ಹೆಸರಿನಲ್ಲಿ ಆನಂದ ಪಡುವುದು, ಎರಡನೆಯದು ಜೀವಿಗಳಿಗೆ ದಯೆ ತೋರುವುದು, ಮೂರನೆಯದು ವೈಷ್ಣವರಿಗೆ ಸೇವೆ ಮಾಡುವುದು’ ಎಂದು ಹೇಳಿದರು. ಜೀವರಿಗೆ ದಯೆ ತೋರುವುದೆಂಬ ಪದವನ್ನು ಉಚ್ಚರಿಸಿದೊಡನೆಯೇ ಅವರು ಸಮಾಧಿಮಗ್ನರಾಗಿ ಹಿಂತಿರುಗಿ ಬಂದಮೇಲೆ “ಜೀವರಿಗೆ ದಯೆ! ಜೀವರಿಗೆ ದಯೆ!” ಎಂದು ಅವಾಕ್ಕಾದರು. “ಮಾನವ ತೃಣದಂತೆ ನೆಲದ ಮೇಲೆ ಹರಿದಾಡುತ್ತಿರುವನು, ಇವನು ಮತ್ತೊಬ್ಬ ಜೀವಿಗೆ ದಯೆ ತೋರುವನೆ! ಇಲ್ಲ, ಎಂದಿಗೂ ಇದು ಸಾಧ್ಯವಾಗಲಾರದು. ಪ್ರತಿಯೊಂದು ಜೀವಿಯಲ್ಲಿಯೂ ಈಶ್ವರನಿರುವನು ಎಂದು ತಿಳಿದು ಅವನಿಗೆ ಸೇವೆಯನ್ನು ಮಾತ್ರ ಮಾಡಬಹುದು” ಎಂದರು. ಅವರ ದೃಷ್ಟಿಯಲ್ಲಿ ಭಗವಂತನೊಬ್ಬನೇ ಜೀವರಿಗೆ ದಯೆ ತೋರಬಲ್ಲ. ಜೀವಿಗಳಾದರೋ ಮತ್ತೊಬ್ಬರಿಗೆ ಸೇವೆಯನ್ನು ಮಾತ್ರ ಮಾಡಬಹುದು. ನರೇಂದ್ರ ಇದನ್ನು ಕೇಳಿದ. ಅನಂತರ ಸ್ನೇಹಿತರಿಗೆ, “ಇಂದು ನಾನೊಂದು ಹೊಸ ಬೆಳಕನ್ನು ಕಂಡೆ. ಬದುಕಿದ್ದರೆ ಜಗತ್ತಿಗೆಲ್ಲ ಇದನ್ನು ಸಾರುತ್ತೇನೆ” ಎಂದು ಹೇಳಿದನು. ಅನಂತರ ವಿವೇಕಾನಂದರಾದ ಮೇಲೆ ಅವರು ಮಾಡಿದ ಅಮೋಘವಾದ ಕರ್ಮಯೋಗದ ಉಪನ್ಯಾಸವೆಲ್ಲ ಶ‍್ರೀರಾಮಕೃಷ್ಣರ ಬಾಯಿಯಿಂದ ಕೇಳಿದ ಒಂದು ಸೂತ್ರದಂತಹ ಮಾತಿಗೆ ಮಾಡಿದ ವ್ಯಾಖ್ಯಾನ.

ಯಾವಾಗಲೂ ಜೀವಿಗೆ ದೇವರನ್ನು ರುಚಿ ನೋಡಲು ಆಸೆ, ಅವನೊಂದಿಗೆ ಒಂದಾಗಲು ಆಸೆಯಿಲ್ಲ. ಮನುಷ್ಯನಿಗೆ ತನ್ನ ವ್ಯಕ್ತಿತ್ವವನ್ನು ಕಾಪಾಡಿಕೊಂಡು ದೇವರನ್ನು ಆಸ್ವಾದಿಸಲು ಆಸೆ. ಆದರೆ, ಅವನೊಂದಿಗೆ ಒಂದಾಗುವುದು ಆಧ್ಯಾತ್ಮಿಕ ಜೀವನದಲ್ಲಿ ಚರಮಗುರಿ. ಶ‍್ರೀರಾಮಕೃಷ್ಣರು ಒಂದು ದಿನ ನರೇಂದ್ರನನ್ನು “ನೀನೊಂದು ನೊಣ ಎಂದು ತಿಳಿದುಕೊ. ಎದುರಿಗೊಂದು ಸಕ್ಕರೆ ಪಾನಕದ ಪಾತ್ರೆಯಿದೆ. ಅದೇ ದೇವರು. ನೀನು ಪಾನಕವನ್ನು ಎಲ್ಲಿ ಕುಳಿತುಕೊಂಡು ಕುಡಿಯುವೆ?” ಎಂದು ಕೇಳಿದರು. ಅದಕ್ಕೆ ನರೇಂದ್ರ, “ನಾನು ಪಾತ್ರೆಯ ಅಂಚಿನಲ್ಲಿ ಕುಳಿತುಕೊಂಡು ಕುಡಿಯುತ್ತೇನೆ. ಮುಳುಗಿದರೆ ಸತ್ತು ಹೋಗಬೇಕಾಗುವುದು” ಎಂದ. ಅದಕ್ಕೆ ಶ‍್ರೀರಾಮಕೃಷ್ಣರು “ಸಚ್ಚಿದಾನಂದವೆಂಬ ಸರೋವರದಲ್ಲಿ ಮುಳುಗಿದರೆ ಯಾರೂ ಸಾಯುವುದಿಲ್ಲ. ಅವರೂ ಅದರಂತೆ ಆಗುವರು” ಎಂದರು. ನಾವು ಯಾವಾಗಲೂ ನಮ್ಮ ಕ್ಷುದ್ರ ವ್ಯಕ್ತಿತ್ವವನ್ನು ಉಳಿಸಿಕೊಂಡಿರಲು ಬಯಸುವೆವು. ಅದು ದೇವರಿಗಾಗಿ ಹೋದರೆ ಒಂದು ಅನಂತ ವ್ಯಕ್ತಿತ್ವ ಬರುವುದು. ಅದಕ್ಕೆ ಅಂಜಬೇಕಾಗಿಲ್ಲ. ನಮಗೆ ಕಳೆದುಕೊಳ್ಳುವುದಕ್ಕೆ ಅಷ್ಟು ಯೋಚನೆ. ಆದರೆ ದೇವರಿಗಾಗಿ ಕಳೆದುಕೊಂಡರೆ ಅಲ್ಪ ಹೋಗಿ ಭೂಮ ಬರುವುದು. ಶ‍್ರೀರಾಮಕೃಷ್ಣರು ಇಂತಹ ಗಹನ ವಿಷಯಗಳನ್ನು ಸಾಧಾರಣ ಒಂದು ಉಪಮಾನದ ಮೂಲಕ ವಿವರಿಸಿಬಿಡುತ್ತಿದ್ದರು.

ನರೇಂದ್ರ ದಕ್ಷಿಣೇಶ್ವರವನ್ನು ಬಿಟ್ಟು ಮನೆಗೆ ಹೋದರೂ, ಕೆಲವು ವೇಳೆ ಜಾಗ್ರದವಸ್ಥೆಯಲ್ಲಿ ಮತ್ತೆ ಕೆಲವು ವೇಳೆ ಸ್ವಪ್ನಾವಸ್ಥೆಯಲ್ಲಿ ಅವನಿಗೆ ಹಲವು ಅನುಭವಗಳಾದವು. ಒಂದು ಸಲ ನರೇಂದ್ರನಿಗೆ ಒಂದು ಕನಸಾಯಿತು. ಆ ಕನಸಿನಲ್ಲಿ ಶ‍್ರೀರಾಮಕೃಷ್ಣರನ್ನು ನೋಡಿದ. ಅವರು ನರೇಂದ್ರನಿಗೆ “ಬಾ ನನ್ನೊಡನೆ, ನಿನಗೆ ರಾಧೆಯನ್ನು ತೋರಿಸುತ್ತೇನೆ” ಎಂದು ಕರೆದುಕೊಂಡುಹೋದರು. ಸ್ವಲ್ಪ ದೂರ ಹೋದಮೇಲೆ, ತಾವೇ ಶ‍್ರೀಕೃಷ್ಣನ ಪ್ರೇಮಕ್ಕೆ ಸರ್ವಸ್ವವನ್ನೂ ಅರ್ಪಿಸಿದ್ದ ರಾಧೆಯಂತಾದರು. ನರೇಂದ್ರ ಇದನ್ನು ನೋಡಿದಮೇಲೆ ರಾಧಾಪ್ರೇಮವನ್ನು ಟೀಕಿಸುತ್ತಿದ್ದದು ತಪ್ಪಿತು. ಭಗವಂತನನ್ನು ಹೇಗೆ ಪ್ರೀತಿಸಬೇಕು, ಅದಕ್ಕೆ ತನ್ನ ಸರ್ವಸ್ವವನ್ನೂ ಹೇಗೆ ತೆರಬೇಕು, ಪ್ರಪಂಚ ಏನು ಅನ್ನುವುದೋ ಎಂಬುದನ್ನು ಮರೆಯಬೇಕು ಎಂಬ ಭಾವನೆಗಳಿಗೆ ರಾಧೆ ಒಂದು ಚಿಹ್ನೆಯಾಗಿರುವಳು ಎಂಬುದನ್ನು ಅರಿತನು. ಶ‍್ರೀರಾಮಕೃಷ್ಣರು ಜಾಗ್ರದವಸ್ಥೆಯಲ್ಲಿ ಮಾತ್ರ ಬೋಧಿಸುತ್ತಿರಲಿಲ್ಲ. ಕನಸಿನಲ್ಲಿಯೂ ನರೇಂದ್ರನ ಮನಸ್ಸನ್ನು ಪ್ರವೇಶಿಸಿದ್ದರು. ಅಲ್ಲಿಯೂ ಅವನಿಗೆ ಗುರುವಾಗಿದ್ದರು.

ಶ‍್ರೀರಾಮಕೃಷ್ಣರ ಶಿಷ್ಯರಾದ ನಿತ್ಯಗೋಪಾಲ, ಮನಮೋಹನ ಮುಂತಾದವರು ಭಕ್ತಿಯಿಂದ ಭಜನೆ ಮಾಡುವಾಗ ಭಾವದಿಂದ ಪರವಶರಾಗಿ ನೆಲದ ಮೇಲೆ ಬೀಳುತ್ತಿದ್ದುದನ್ನು ನರೇಂದ್ರ ನೋಡುತ್ತಿದ್ದ. ಅಯ್ಯೋ, ನನಗೆ ಹಾಗೆ ಭಾವ ಪರವಶನನ್ನಾಗಿ ಮಾಡುವುದಿಲ್ಲವಲ್ಲ ಎಂದು ಶ‍್ರೀರಾಮಕೃಷ್ಣರನ್ನು ಕೇಳಿಕೊಂಡ. ಅದಕ್ಕೆ ಶ‍್ರೀರಾಮಕೃಷ್ಣರು ಸಂತೈಸಿದರು: “ಇದಕ್ಕೆ ಏತಕ್ಕೆ ಚಿಂತಿಸುವುದು? ಆನೆಯೊಂದು ಸಣ್ಣ ಕೊಳಕ್ಕೆ ನುಗ್ಗಿದರೆ ಆ ಕೊಳದ ನೀರೆಲ್ಲ ಅಲ್ಲೋಲಕಲ್ಲೋಲವಾಗುವುದು. ಆದರೆ ಅದೇ ಆನೆ ದೊಡ್ಡ ಸರೋವರಕ್ಕೆ ನುಗ್ಗಿದರೆ ಆ ಸರೋವರ ಅಲ್ಲೋಲಕಲ್ಲೋಲವಾಗುವುದೇ? ಅದರಂತೆ ನೀನು ಮಹಾ ಸರೋವರ.”

ಒಂದು ದಿನ ಕಲ್ಕತ್ತೆಯಲ್ಲಿ ನರೇಂದ್ರನ ಶ‍್ರೀಮಂತ ಸ್ನೇಹಿತರು ಅವನನ್ನು ಸಂತೋಷ ಕೂಟಕ್ಕೆ ಕರೆದುಕೊಂಡು ಹೋದರು. ದೊಡ್ಡ ಮನೆ ಒಂದರ ಮುಂದೆ ಇಳಿದು ಒಳಗೆ ಹೋದರು. ಅಲ್ಲಿ ಊಟ ತಿಂಡಿಯಾದ ಮೇಲೆ ಉದ್ಯಾನವನದಲ್ಲಿ ಸಂಗೀತಾದಿಗಳು ಆದವು. ನರೇಂದ್ರನೂ ಹಾಡಿದ. ಸ್ವಲ್ಪ ಹೊತ್ತಾದ ಮೇಲೆ ನರೇಂದ್ರ ಆಲಸ್ಯದಿಂದ ವಿಶ್ರಾಂತಿ ಪಡೆಯಬೇಕೆಂದು ಬಯಸಿದ. ಅವನನ್ನು ಸುಸಜ್ಜಿತವಾದ ಒಂದು ಕೋಣೆಗೆ ಬಿಟ್ಟರು. ಅಲ್ಲಿ ಅವನು ಇದ್ದಾಗ ಒಬ್ಬ ನರ್ತಕಿಯನ್ನು ಅವನ ಮನರಂಜನೆ ಮಾಡುವುದಕ್ಕೆ ಕಳುಹಿಸಿದರು. ನರೇಂದ್ರನಾಥನಿಗೆ ಅದು ಗೊತ್ತಿರಲಿಲ್ಲ. ಆ ನರ್ತಕಿ ಬಂದು ಇವನ ಹತ್ತಿರ ಕುಳಿತು ಮಾತನಾಡತೊಡಗಿದಳು. ಆಕೆ ತನ್ನ ಕೊಳಕು ಬಾಳಿನ ಆತ್ಮಕಥೆಯನ್ನು ಹೇಳಿಕೊಂಡಳು. ನರೇಂದ್ರ ಅನುಕಂಪೆಯಿಂದ ಅವಳಿಗೆ ಸಹಾನುಭೂತಿಯನ್ನು ತೋರಿದ. ಪಾಪ ಆ ನರ್ತಕಿ ನರೇಂದ್ರ ತೋರಿದ ಸಹಾನುಭೂತಿಯನ್ನು ಲೌಕಿಕವಾಗಿ ತಿಳಿದುಕೊಂಡಳು. ತಾನು ಯಾವ ಉದ್ದೇಶದಿಂದ ಇವನ ಬಳಿಗೆ ಬಂದಿರುವೆ ಎಂಬುದನ್ನು ವಿವರಿಸಿದಳು. ನರೇಂದ್ರನಾದರೋ ಅವಳ ಇಚ್ಛೆಯನ್ನು ಅರಿತೊಡನೆಯೆ ಜಾಗ್ರತನಾಗಿ “ನಾನೀಗ ಹೋಗಬೇಕಾಗಿದೆ, ದಯವಿಟ್ಟು ಕ್ಷಮಿಸು. ನಿನಗೆ ನನ್ನ ಹೃತ್ಪೂರ್ವಕ ಸಹಾನುಭೂತಿಯಿದೆ. ನೀನು ಒಳ್ಳೆಯವಳಾಗು ಎಂದು ನಾನು ಆಶಿಸುತ್ತೇನೆ. ಇಂತಹ ಜೀವನವನ್ನು ನಡೆಸುವುದು ಅಯೋಗ್ಯ ಎಂದು ನೀನು ಅರಿತರೆ, ಒಂದು ದಿನ ನೀನು ಇದರಿಂದ ಪಾರಾಗುವೆ” ಎಂದು ಹೇಳಿ ಅಲ್ಲಿಂದ ಹೊರಟುಹೋದ. ಆ ನರ್ತಕಿ ಅನಂತರ ನರೇಂದ್ರನ ಸ್ನೇಹಿತರ ಹತ್ತಿರ ಹೋಗಿ: “ನೀವು ಎಂತಹ ಮೋಸ ಮಾಡಿದಿರಿ ನನಗೆ. ಆ ಸಾಧುಪುರುಷನನ್ನು ಮರುಳು ಮಾಡುವುದಕ್ಕೆ ಕಳಿಸಿದಿರಲ್ಲ” ಎಂದಳು.

ನರೇಂದ್ರ ಶ‍್ರೀರಾಮಕೃಷ್ಣರ ಸಮೀಪಕ್ಕೆ ಬರುವಾಗ ಕೇವಲ ವಿಚಾರವಾದಿಯಂತೆ ಬಂದ. ಆದರೆ ಶ‍್ರೀರಾಮಕೃಷ್ಣರ ಸಂಪರ್ಕದಿಂದ ಭಕ್ತಿಯೆಂದರೆ ಏನು ಎಂಬುದನ್ನು ಮನಗಂಡ. ತನ್ನ ಒಳಗೇ ಇತ್ತು ಅದು. ಆದರೆ ಅದು ವಿಚಾರದಿಂದ ಕಣ್ಣಿಗೆ ಕಾಣುತ್ತಿರಲಿಲ್ಲ. ಶ‍್ರೀರಾಮಕೃಷ್ಣರು ಅವನಿಗೆ ಅವನ ವಿಚಾರದ ಹಿಂದೆ ಎಂತಹ ಭಾವುಕ ಭಕ್ತನ ಸ್ಥಿತಿ ಇದೆ ಎಂಬುದನ್ನು ತೋರಿದರು. ಅನಂತರ ನರೇಂದ್ರ, “ಶ‍್ರೀರಾಮಕೃಷ್ಣರು ನೋಡುವುದಕ್ಕೆ ಭಕ್ತರು. ಒಳಗೆಲ್ಲಾ ಜ್ಞಾನವಿದೆ. ನಾನು ಹೊರಗಿನಿಂದ ಜ್ಞಾನಿ, ಒಳಗೆಲ್ಲ ಭಕ್ತ” ಎನ್ನುತ್ತಿದ್ದನು. ಜ್ಞಾನ ಮತ್ತು ಭಕ್ತಿ ಸಮತೂಕವಾಗಿ ನರೇಂದ್ರನ ಜೀವನದಲ್ಲಿ ವೃದ್ಧಿಯಾಗುವಂತೆ ಶ‍್ರೀರಾಮಕೃಷ್ಣರು ಮಾಡಿದರು. ಜೊತೆಗೆ ಪ್ರಾರ್ಥನೆ ಮತ್ತು ಧ್ಯಾನ ಎರಡೂ ಜೀವಿ ಮೇಲೇಳಬೇಕಾದರೆ ಅತ್ಯಾವಶ್ಯಕ ಎಂದರು. ಧ್ಯಾನ ಪ್ರಾರ್ಥನೆಗಳೇ ರೆಕ್ಕೆ. ಇವು ಬಲವಾಗಿದ್ದರೇನೆ ಆಧ್ಯಾತ್ಮಿಕ ಪ್ರಪಂಚದಲ್ಲಿ ಯಶಸ್ವಿಯಾಗಬೇಕಾದರೆ ಬರೀ ಒಣ ತರ್ಕದಿಂದ ಅಲ್ಲ, ಕಣ್ಣೀರನ್ನು ಮಾತ್ರ ಸುರಿಸುವ ಉದ್ವೇಗದಿಂದಲೂ ಅಲ್ಲ.

ಶ‍್ರೀರಾಮಕೃಷ್ಣರು ನರೇಂದ್ರನಿಗೆ ಏನನ್ನು ಬೇಕಾದರೂ ಮಾಡಲು ಸಿದ್ಧವಾಗಿದ್ದರು. ಅಷ್ಟೊಂದು ಪ್ರೀತಿ ಅವನ ಮೇಲೆ. ಒಂದು ದಿನ ಒಬ್ಬ ಶ‍್ರೀಮಂತ ಶಿಷ್ಯನ ಹತ್ತಿರ ಮಾತನಾಡುತ್ತಿದ್ದಾಗ ನರೇಂದ್ರನ ವಿಷಯವಾಗಿ “ಈಗ ಅವನ ತಂದೆ ತೀರಿ ಹೋಗಿರುವನು. ಮನೆಯವರು ಉಪವಾಸದಿಂದ ನರಳುತ್ತಿರುವರು. ಅವನ ಸ್ನೇಹಿತರು ಈ ಆಪತ್ಕಾಲದಲ್ಲಿ ಸಹಾಯಮಾಡಿದರೆ ಒಳ್ಳೆಯದು” ಎಂದರು. ನರೇಂದ್ರ ಇದನ್ನು ಕೇಳಿದ. ಆ ದೊಡ್ಡ ಮನುಷ್ಯ ಹೊರಟುಹೋದಮೇಲೆ “ಮಹಾಶಯರೆ, ನೀವು ಏತಕ್ಕೆ ಅವನಿಗೆ ನಮ್ಮ ಕಷ್ಟವನ್ನು ಹೇಳಿದ್ದು?” ಎಂದು ಆಕ್ಷೇಪಿಸಿದ. ಅದಕ್ಕೆ ಶ‍್ರೀರಾಮಕೃಷ್ಣರು “ನರೇನ್, ನಿನಗೆ ಗೊತ್ತಿಲ್ಲವೆ ನಾನು ನಿನಗಾಗಿ ಏನನ್ನು ಬೇಕಾದರೂ ಮಾಡುತ್ತೇನೆ ಎಂಬುದು. ನಿನಗಾಗಿ ನಾನು ಮನೆಯಿಂದ ಮನೆಗೆ ಭಿಕ್ಷೆ ಎತ್ತಲು ಬೇಕಾದರೂ ಹೋಗುತ್ತೇನೆ” ಎಂದರು. ನರೇಂದ್ರ ತನ್ನ ಗುರುಗಳಿಗೆ ಅವನ ಮೇಲಿರುವ ಪ್ರೇಮದ ಆಳವನ್ನು ನೋಡಿ ಮೂಕನಂತಾದನು. ಅವರು ತಮ್ಮ ಪ್ರೇಮದಿಂದ ನಮ್ಮನ್ನು ದಾಸನನ್ನಾಗಿ ಮಾಡಿಕೊಂಡರು ಎಂದು ನರೇಂದ್ರ ಅನಂತರ ಹೇಳುತ್ತಿದ್ದ.

ಶ‍್ರೀರಾಮಕೃಷ್ಣರು ನರೇಂದ್ರನನ್ನು ಪ್ರಥಮ ಬಾರಿ ನೋಡಿದಂದಿನಿಂದ ಅವನು ಸಂನ್ಯಾಸಿಯಾಗುವುದಕ್ಕೆ ಹುಟ್ಟಿರುವನು ಎಂಬುದನ್ನು ಅರಿತರು. ಆದಕಾರಣವೇ ಅವನಿಗೆ ಮದುವೆ ಸನ್ನಾಹಗಳು ಆಗುತ್ತಿವೆ ಎಂಬುದನ್ನು ಕೇಳಿದಾಗ ದೇವರಿಗೆ ಕಂಬನಿದುಂಬಿ ಅದು ನಿಂತುಹೋಗುವಂತೆ ಪ್ರಾರ್ಥಿಸಿದರು. ನರೇಂದ್ರ ಮದುವೆಯಾಗಿ ಒಂದು ಸಣ್ಣ ಸಂಸಾರಕ್ಕೆ ಸುಖಕೊಡಲು ಜನ್ಮವೆತ್ತಿಲ್ಲ. ಅವನೊಂದು ಮಹಾ ವೃಕ್ಷವಾಗಿ ಅಸಂಖ್ಯಾತ ಜೀವಿಗಳಿಗೆ ಆಶ್ರಯ ಕೊಡಬೇಕಾಗಿದೆ ಎನ್ನುತ್ತಿದ್ದರು. ನರೇಂದ್ರ ಇತರ ರಾಮಕೃಷ್ಣರ ಗೃಹಸ್ಥ ಶಿಷ್ಯರೊಡನೆ ಹೆಚ್ಚಾಗಿ ಬೆರೆತು ಎಲ್ಲಿ ಅವನಲ್ಲಿಯೂ ಪ್ರಾಪಂಚಿಕತೆ ಚಿಗುರುವುದೋ ಎಂದು ಅಂಜಿದರು. ನರೇಂದ್ರ ಶ‍್ರೀರಾಮಕೃಷ್ಣರ ಶಿಷ್ಯನಾದ ಗಿರೀಶ್ ಚಂದ್ರಘೋಷ್‍ನೊಡನೆ ಹೆಚ್ಚು ಬೆರೆಯುತ್ತಿದ್ದ. ಅದಕ್ಕೆ ಶ‍್ರೀರಾಮಕೃಷ್ಣರು “ಅವನೊಡನೆ ಅಷ್ಟು ಹೆಚ್ಚು ಬೆರೆಯಬೇಡ. ಅವನು ಪ್ರಾಪಂಚಿಕ ಜೀವನವನ್ನು ಮಿತಿಮೀರಿ ಅನುಭವಿಸಿದವನು” ಎಂದು ಎಚ್ಚರಿಕೆ ಕೊಟ್ಟರು. ನರೇಂದ್ರ “ಗಿರೀಶ ಈಗ ಅದನ್ನೆಲ್ಲ ಬಿಟ್ಟುಬಿಟ್ಟಿರುವನು” ಎಂದನು. ಅದಕ್ಕೆ ಶ‍್ರೀರಾಮಕೃಷ್ಣರು “ಎಷ್ಟೇ ಬಿಟ್ಟಿದ್ದರೂ ಹಿಂದಿನ ವಾಸನೆ ಇದ್ದೇ ಇರುತ್ತದೆ. ಬೆಳ್ಳುಳ್ಳಿ ರಸವನ್ನು ಇಟ್ಟ ಪಾತ್ರೆಯನ್ನು ಎಷ್ಟು ತೊಳೆದರೂ ಸ್ವಲ್ಪ ವಾಸನೆ ಇದ್ದೇ ಇರುತ್ತದಲ್ಲ ಹಾಗೆ” ಎಂದರು. ಪ್ರಾಪಂಚಿಕ ಸುಖವನ್ನು ಅನುಭವಿಸಿದ ಮನುಷ್ಯ ಕಾಗೆ ಕುಕ್ಕಿದ ಹಣ್ಣಿನಂತೆ. ಅದನ್ನು ದೇವರಿಗೂ ಕೊಡುವುದಕ್ಕೆ ಆಗುವುದಿಲ್ಲ, ತಾನೂ ತಿನ್ನಲು ಸಾಧ್ಯವಿಲ್ಲ ಎನ್ನುತ್ತಿದ್ದರು ಅವರು. ಪ್ರಾಪಂಚಿಕ ವಿಷಯಗಳನ್ನು ಅನುಭವಿಸಿದವರು ಬೇರೊಂದು ಬಗೆಯ ಗುಂಪಿಗೆ ಸೇರಿದವರು. ಒಂದು ಸಂನ್ಯಾಸಿಗಳ ಗುಂಪು ದೇವರ ಚಿಂತನೆಯನ್ನು ಮಾಡುತ್ತ ಕುಳಿತಿತ್ತು. ಅವರ ಮುಂದೆ ಒಬ್ಬ ಸ್ತ್ರೀ ಹೋದಳು. ಯಾರೂ ಅತ್ತಕಡೆ ನೋಡಲಿಲ್ಲ. ಅವರಲ್ಲಿ ಒಬ್ಬ ಮಾತ್ರ ಅವಳ ಕಡೆ ನೋಡಿದ. ಅವನು ಹಿಂದೆ ಸಂಸಾರದಲ್ಲಿದ್ದ. ನಾಲ್ಕು ಮಕ್ಕಳ ತಂದೆಯಾಗಿ ಸಂಸಾರವನ್ನು ತ್ಯಜಿಸಿದವನು. ಆದರೆ ನರೇಂದ್ರನ ಆದರ್ಶ ಅದಲ್ಲ. ಎಲ್ಲಾ ಗೃಹಸ್ಥರನ್ನೂ ದೂರಕೂಡದು. ಎಲ್ಲರಲ್ಲಿಯೂ ಭಗವಂತನೇ ಇರುವನು ಎಂಬುದನ್ನು ನೋಡಬೇಕು. ಅದರಂತೆಯೇ ಸ್ತ್ರೀ ಎಷ್ಟು ಪತಿತಳಾಗಿರಲಿ, ಅವರಲ್ಲಿಯೂ ಜಗನ್ಮಾತೆಯೇ ಇರುವಳು ಎಂದು ನೋಡಬೇಕು. ಆದರೆ ನಿಕಟತೆಯಿಂದ ದೂರವಿರಬೇಕು. ಮನುಷ್ಯ ಒಮ್ಮೆ ಕಾಮಿನಿಯ ಪಾಶಕ್ಕೆ ಬಿದ್ದರೆ ಮತ್ತೆ ಏಳುವಂತಿಲ್ಲ. ಎಷ್ಟೇ ಭಕ್ತಿಯಿಂದ ಸ್ತ್ರೀ ನೆಲದ ಮೇಲೆ ಬಿದ್ದು ಭಾವದಿಂದ ಉನ್ಮತ್ತಳಾಗಿದ್ದರೂ ಆಕೆಯಿಂದ ಜೋಪಾನವಾಗಿರು ಎಂದು ನರೇಂದ್ರನಿಗೆ ಎಚ್ಚರಿಕೆ ಕೊಡುತ್ತಿದ್ದರು. ಚಿತ್ತ ಶುದ್ಧವಾಗಿಲ್ಲದೆ ಇದ್ದರೆ ಭಕ್ತಿ ಇರಲಾರದು. ಬೇಕಾದರೆ ಅಲ್ಲಿ ಉದ್ವೇಗ ಇರಬಹುದು. ಬರೀ ಉದ್ವೇಗವೇ ಭಕ್ತಿಯಲ್ಲ. ಚಿತ್ತಶುದ್ಧಿಯಾಗಿಲ್ಲದೇ ಇದ್ದರೆ ಏಕಮುಖವಾದ ಭಕ್ತಿ ಇರಲಾರದು. ಹಲವು ವಸ್ತುಗಳ ಮೇಲೆ ಅವನ ಮನಸ್ಸು ಹರಿದು ಹಂಚಿ ಹೋಗಿರುತ್ತದೆ. ಪ್ರಾಪಂಚಿಕ ಹಲವು ವಸ್ತುಗಳಿಗೆ, ಹೆಂಡತಿಗೆ, ಮಕ್ಕಳಿಗೆ, ಕೀರ್ತಿಗೆ, ಲಾಭಕ್ಕೆ ಬದ್ಧ; ದೇವರಿಗೆ ಕೊಡುವುದಕ್ಕೆ ಅವನಲ್ಲಿರುವುದು ಬಹಳ ಅಲ್ಪ. ಕಾಮಕಾಂಚನಕ್ಕೆ ದಾಸನಾದರೆ ದೇವರು ನಿನಗೆ ಎಂದೂ ದೊರಕುವುದಿಲ್ಲವೆಂದು ನರೇಂದ್ರನಿಗೆ ಹೇಳುತ್ತಿದ್ದರು. ಶ‍್ರೀರಾಮಕೃಷ್ಣರು ಅದ್ಭುತ ಸಾಧನೆಯಿಂದ ಯಾವ ಯಾವ ಅನುಭವಗಳನ್ನು ಪಡೆದಿದ್ದರೋ ಆ ಅನುಭವದ ಬೀಜಗಳನ್ನೆಲ್ಲ ನರೇಂದ್ರನ ಎದೆಯಲ್ಲಿ ಬಿತ್ತಿದರು. ಅದು ಚೆನ್ನಾಗಿ ಮೊಳೆತು ಫಲಕಾರಿಯಾಗಲು ಎಲ್ಲಾ ಶ್ರಮವನ್ನೂ ತೆಗೆದುಕೊಂಡರು. ಏಕೆಂದರೆ ಯಾವುದನ್ನು ನರೇಂದ್ರನಲ್ಲಿ ಬಿಟ್ಟುಹೋಗುವರೊ ಅದು ಜಗತ್ತಿನ ಕಲ್ಯಾಣಕ್ಕೆ ಎಂಬುದು ಅವರಿಗೆ ಮೊದಲಿನಿಂದಲೂ ವೇದ್ಯವಾಗಿತ್ತು.

ಕ್ರಿ.ಶ. ೧೮೮೫ರ ಬೇಸಗೆ ಕಾಲವಾದ ಮೇಲೆ ಶ‍್ರೀರಾಮಕೃಷ್ಣರ ಆರೋಗ್ಯ ಶಿಥಿಲವಾಗತೊಡಗಿತು. ಬಂದ ಭಕ್ತರೊಡನೆ ಹೊತ್ತು ಗೊತ್ತಿಲ್ಲದೆ ಮಾತನಾಡುವುದು, ಭಜನೆ ನೃತ್ಯ, ಸಮಾಧಿಮಗ್ನರಾಗುವುದು ಇವುಗಳೆಲ್ಲ ಅವರ ದೇಹದ ಮೇಲೆ ತಮ್ಮ ಪ್ರಭಾವವನ್ನು ಬೀರತೊಡಗಿದುವು. ಗಂಟಲಿನ ನೋವು ಮೊದಲು ಕಾಣಿಸಿಕೊಂಡಿತು. ಅದಕ್ಕೆ ಚಿಕಿತ್ಸೆಯನ್ನು ಮಾಡಿದರು. ಆದರೆ ನೋವು ಕಡಿಮೆಯಾಗಲಿಲ್ಲ. ಕ್ರಮೇಣ ಉಲ್ಬಣವಾಗುತ್ತ ಬಂದಿತು. ಕಲ್ಕತ್ತೆಯಿಂದ ಬಂದ ವೈದ್ಯರು ಅವರನ್ನು ಪರೀಕ್ಷಿಸಿ, ಭಕ್ತರೊಡನೆ ಹೆಚ್ಚು ಮಾತನಾಡಕೂಡದೆಂದೂ, ಭಕ್ತಿ ಪರವಶರಾಗಿ ಸಮಾಧಿಗೆ ಹೋಗಕೂಡದೆಂದೂ ಹೇಳಿದರು. ಆದರೆ ಇವೆರಡನ್ನೂ ಅವರು ನೆರವೇರಿಸುವಂತಿರಲಿಲ್ಲ. ಯಾರಾದರೂ ಭವಜೀವಿಗಳು ದೇವರಿಗೆ ಸಂಬಂಧಪಟ್ಟ ಯಾವುದಾದರೂ ಪ್ರಶ್ನೆಯನ್ನು ಹಾಕಿದರೆ, ತಮ್ಮ ಗಂಟಲಿನ ನೋವು ಮತ್ತು ಕಾಲ ಪರಿಮಿತಿ ಎಲ್ಲವನ್ನು ಮರೆತು ಮಾತನಾಡುತ್ತಿದ್ದರು. ಯಾರಾದರೂ ಎಚ್ಚರಿಕೆ ಕೊಟ್ಟರೆ, ಭವಜೀವಿಗಳಿಗೆ ಸಂತೋಷವಾದರೆ ಸಾಕು ಎಂದು ತಮಗೆ ಎಷ್ಟು ಕಷ್ಟವಾದರೂ ಗಮನಿಸುತ್ತಿರಲಿಲ್ಲ. ಅದರಂತೆಯೇ ಭಕ್ತಿಗೀತೆಗಳನ್ನು ಕೇಳಿದರೆ ಸಾಕು ಅವರ ಮನಸ್ಸು ಊರ್ಧ್ವಮುಖವಾಗಿ ಸಮಾಧಿಮಗ್ನವಾಗಿಬಿಡುತ್ತಿತ್ತು. ಅವರ ಗಂಟಲಿನ ನೋವು ಯಾವಾಗ ಕಡಿಮೆಯಾಗಲಿಲ್ಲವೋ ಆಗ ಅವರನ್ನು ಚಿಕಿತ್ಸೆ ಮಾಡುತ್ತಿದ್ದ ವೈದ್ಯರು ಇದು ಕ್ಯಾನ್ಸರ್ ಎನ್ನುವ ನಿರ್ಧಾರಕ್ಕೆ ಬಂದರು.

ಶ‍್ರೀರಾಮಕೃಷ್ಣರ ಖಾಯಿಲೆ ಕ್ಯಾನ್ಸರ್ ಎಂದು ನಿರ್ಧಾರಕ್ಕೆ ಬಂದಮೇಲೆ ನರೇಂದ್ರ ವೈದ್ಯಕೀಯ ಪುಸ್ತಕಗಳನ್ನು ಓದಿ, ಕೆಲವು ನುರಿತ ವೈದ್ಯರುಗಳಲ್ಲಿ ಅದರ ವಿಷಯವಾಗಿ ಚರ್ಚಿಸಿದ. ಆ ಖಾಯಿಲೆಯನ್ನು ಗುಣಮಾಡುವ ಮದ್ದು ಇನ್ನೂ ಕಂಡುಹಿಡಿದಿಲ್ಲವೆಂದೂ, ಅದು ಮೃತ್ಯುವಿನಲ್ಲಿಯೇ ಪರ‍್ಯವಸಾನವಾಗುವುದೆಂದೂ ತಿಳಿದುಕೊಂಡನು. ತನ್ನ ಗುರುಭಾಯಿಗಳಿಗೆ ಈ ಸಮಾಚಾರವನ್ನು ಕೊಟ್ಟ. ಗುರುಗಳ ಅಂತಿಮ ಸೇವೆಯನ್ನು ಮಾಡಿ, ಉಗ್ರಸಾಧನೆ ಮಾಡಿ ತಾವು ಈ ಜೀವನದಲ್ಲಿ ಗುರಿಯನ್ನು ಕಾಣಬೇಕೆಂದು ಹೇಳಿದ. ಅದಕ್ಕಾಗಿ ಕೆಲವು ಭಕ್ತರಿಂದ ಹಣವನ್ನು ಚಂದಾ ಎತ್ತಿದನು. ಶ‍್ರೀರಾಮಕೃಷ್ಣರ ಸೇವೆ ಮಾಡಲು ನರೇಂದ್ರನೇ ಮುಂದೆ ನಿಂತ. ಇತರ ಶಿಷ್ಯರಿಗೆಲ್ಲ ಸೇವೆಗೆ ಕಾಲವನ್ನು ನಿಯಮಿಸಿದ. ಆ ಸಮಯದಲ್ಲಿ ನರೇಂದ್ರ ಕೊನೆಯ ಲಾ ತರಗತಿಯಲ್ಲಿದ್ದ, ಇತರ ಸ್ನೇಹಿತರೂ ಕೂಡ ಕಾಲೇಜುಗಳಲ್ಲಿ ಓದುತ್ತಿದ್ದರು. ಆದರೆ ತಮಗೆ ಜೀವನದಲ್ಲಿ ಕಣ್ಣನ್ನು ತೆರೆಸಿ, ದೇವರೆಡೆಗೆ ಕೈಹಿಡಿದು ನಡೆಸಿದ ಆ ಪರಮ ಗುರುಗಳ ಕೊನೆಯ ಸೇವೆಯನ್ನು ಮಾಡುವುದಕ್ಕೆ ಅವರು ಎಲ್ಲವನ್ನೂ ಮರೆತರು. ಕಲ್ಕತ್ತೆಯಲ್ಲಿ ಶ‍್ರೀರಾಮಕೃಷ್ಣರಿಗಾಗಿ ಒಂದು ಮನೆಯನ್ನು ಮಾಡಿದರು. ಶ‍್ರೀಶಾರದಾದೇವಿಯವರು ಅವರಿಗೆ ಅಡಿಗೆ ಮಾಡಿ ಹಾಕುವುದಕ್ಕೆ ಅಲ್ಲಿಗೆ ಬಂದರು. ಹಗಲು ರಾತ್ರಿ ಅವರ ಸೇವೆಗೆ ತರುಣ ಶಿಷ್ಯರು ಸನ್ನದ್ಧರಾದರು. ಶ‍್ರೀರಾಮಕೃಷ್ಣರಿಗೆ ಆ ಮನೆ ಹಿಡಿಸಲಿಲ್ಲ. ಅಲ್ಲಿ ಸಾಕಾದಷ್ಟು ಗಾಳಿ ಬೆಳಕು ಇರಲಿಲ್ಲ. ಆದಕಾರಣವೇ ಸ್ವಲ್ಪ ದಿನಗಳಾದ ಮೇಲೆ ಬಾಗ್‍ಬಜಾರಿನಲ್ಲಿರುವ ಬಲರಾಮ ಬೋಸರ ಮನೆಗೆ ಹೋದರು. ಅನಂತರ ಶ್ಯಾಮಪುಕುರದಲ್ಲಿ ಒಂದು ಮನೆಯನ್ನು\break ಗೊತ್ತುಮಾಡಿ ಶ‍್ರೀರಾಮಕೃಷ್ಣರನ್ನು ಅಲ್ಲಿಗೆ ಕರೆದುಕೊಂಡುಹೋದರು. ನರೇಂದ್ರ ಮತ್ತು ಇತರ ಶಿಷ್ಯರು, ಶ‍್ರೀರಾಮಕೃಷ್ಣರ ಸೇವೆ ಮತ್ತು ಅದನ್ನು ಮಾಡದೆ ಇದ್ದ ಕಾಲದಲ್ಲಿ ಸಾಧನೆ ಭಜನೆ ಇವುಗಳಲ್ಲಿ ನಿರತರಾದರು.

ಶ‍್ರೀರಾಮಕೃಷ್ಣರಿಗೆ ಬಂದ ಖಾಯಿಲೆಯನ್ನು ಹಲವು ಜನ ಹಲವು ದೃಷ್ಟಿಯಿಂದ ನೋಡತೊಡಗಿದರು. ಇದು ಭಗವದಿಚ್ಛೆ ಎಂದರು ಕೆಲವರು. ಮತ್ತೆ ಕೆಲವರು,\break ಶ‍್ರೀರಾಮಕೃಷ್ಣರೇ ಒಂದು ಅವತಾರ, ಅವರ ಇಚ್ಛೆ ಭಗವದಿಚ್ಛೆ ಎಂದು ಎರಡು ಬೇರೆ ಹೇಗೆ ಇರಬಲ್ಲದು ಎಂದರು. ಶಿಷ್ಯರು ಸೇವೆ ಮಾಡಿ ಉದ್ಧಾರವಾಗಲೆಂದು ಅವರು ಈ ಖಾಯಿಲೆಯನ್ನು ತಾವೇ ಬರಮಾಡಿಕೊಂಡಿರುವರು ಎಂದರು ಮೂರನೆಯ ಗುಂಪಿನವರು. ಶ‍್ರೀರಾಮಕೃಷ್ಣರು ಯಾರಾದರೂ ಆಗಲಿ ಚಿಂತೆಯಿಲ್ಲ. ಅವರ ದೇಹ ಎಲ್ಲಾ ದೇಹಗಳಂತೆ ವೃದ್ಧಿ ಕ್ಷಯವನ್ನು ಅನುಭವಿಸಬೇಕು. ಈ ಸಮಯದಲ್ಲಿ ಅವರಿಗೆ ಖಾಯಿಲೆ ಏತಕ್ಕೆ ಬಂತು ಎಂಬ ವಿಚಾರ ಮಾಡುವುದಲ್ಲ ಮುಖ್ಯ, ಅವರಿಗೆ ಸಾಧ್ಯವಾದಷ್ಟು ಸೇವೆ ಮಾಡಿ ಉಪಶಮನ ಮಾಡುವುದಕ್ಕೆ ಪ್ರಯತ್ನಿಸುವುದು ನಮ್ಮ ಕರ್ತವ್ಯ ಎಂದು ಕೆಲವರು ಹೇಳಿದರು. ನರೇಂದ್ರನೇ ಈ ಕೊನೆಯ ಗುಂಪಿನ ಮುಂದಾಳು.

ಹಲವು ಜನ ಶಿಷ್ಯರು ಭಕ್ತಿಯಿಂದ ಪರವಶರಾಗಿ ನೆಲದ ಮೇಲೆ ಬೀಳುವುದು ಕಂಬನಿ ಸುರಿಸುವುದು ಇವುಗಳನ್ನೆಲ್ಲ ಮಾಡುತ್ತಿದ್ದುದನ್ನು ನರೇಂದ್ರ ನೋಡಿದ.\break ಶ‍್ರೀರಾಮಕೃಷ್ಣರಲ್ಲಾದರೋ ಶುದ್ಧ ಚಾರಿತ್ರ್ಯವಿತ್ತು, ಕಾಮಕಾಂಚನಾಸಕ್ತಿ ಲವಲೇಶವಾದರೂ ಇರಲಿಲ್ಲ. ಇವುಗಳ ಹಿನ್ನೆಲೆ ಇಲ್ಲದೆ ಭಾವೋದ್ರೇಕಕ್ಕೆ ಸಿಕ್ಕಿದವರನ್ನು ನರೇಂದ್ರ ಚೆನ್ನಾಗಿ ಹಂಗಿಸುತ್ತಿದ್ದ. ಇಂತಹ ಉದ್ವೇಗಪರವಶತೆ ಆಧ್ಯಾತ್ಮಿಕ ಜೀವನದಲ್ಲಿ ಒಳ್ಳೆಯದಲ್ಲ. ಕೆಲವು ವೇಳೆ ಮನಸ್ಸು ಮೇಲಕ್ಕೆ ಹೋಗುವುದು. ಆದರೆ ಎಷ್ಟು ಬೇಗ ಹೋಗುವುದೋ ಅಷ್ಟೇ ಬೇಗ ಕೆಳಕ್ಕೆ ಇಳಿಯುವುದು. ಹಿಂದಿಗಿಂತ ಹೆಚ್ಚಾಗಿ ಅವರು ಕೆಳಕ್ಕೆ ಹೋಗುವರು. ಇಂತಹ ಹಲವು ನಿದರ್ಶನಗಳನ್ನು ಅವನು ನೋಡಿದ್ದನು, ಈ ವಿಷಯದಲ್ಲಿ ಎಚ್ಚರಿಕೆ ಕೊಡುತ್ತಿದ್ದನು. ಮತ್ತೊಂದು ಅಭಿಪ್ರಾಯವೆಂದರೆ ಶ‍್ರೀರಾಮಕೃಷ್ಣರಲ್ಲಿ ಶರಣಾದರೆ ಸಾಕು, ಅವರೊಬ್ಬ ಅವತಾರ ವ್ಯಕ್ತಿಗಳು, ಅವರು ನಮ್ಮನ್ನು ಉದ್ಧಾರ ಮಾಡುವುದರಲ್ಲಿ ಸಂದೇಹವಿಲ್ಲ ಎಂಬುದು. ಮತ್ತೊಬ್ಬರಲ್ಲಿ ಶರಣಾಗುವುದು ಅಷ್ಟು ಸುಲಭವಲ್ಲ. ಬಾಯಿಯಲ್ಲಿ ಹೇಳುವಷ್ಟು ಅದು ಸುಲಭವಲ್ಲ. ಎಲ್ಲಿಯವರೆಗೆ ನಮ್ಮಲ್ಲಿ ಅಹಂಕಾರ ಇರುವುದೋ ಅಲ್ಲಿಯವರೆಗೆ ಅದು ನಮಗೆ ಶರಣಾಗತರಾಗಲು ಅವಕಾಶ ಕೊಡುವುದಿಲ್ಲ. ನಾನು ಶರಣಾಗತನಾಗಿರುವೆನು ಎಂದು ಅದಕ್ಕೇ ಹೆಮ್ಮೆ ಕೊಚ್ಚಿಕೊಳ್ಳುವನು! ಸಣ್ಣ ಹೆಮ್ಮೆಯ ಬದಲು ದೊಡ್ಡದೊಂದು ಹೆಮ್ಮೆಗೆ ಅವಕಾಶ ಕೊಟ್ಟಂತೆ. ಆಧ್ಯಾತ್ಮಿಕ ಜೀವನದಲ್ಲಿ ನಟಿಸುವುದು ಸುಲಭ. ಎಲ್ಲಾ ಕಾರ‍್ಯಕ್ಷೇತ್ರಕ್ಕಿಂತ ಈ ಕಾರ‍್ಯಕ್ಷೇತ್ರದಲ್ಲಿ ಮೋಸ ಹೆಚ್ಚು. ಒಂದು ಸಾಚಾ ನಾಣ್ಯವಿದ್ದರೆ ಅದಕ್ಕೆ ನೂರರಷ್ಟು ಖೋಟಾ ನಾಣ್ಯಗಳು ಇರುತ್ತವೆ. ಆದಕಾರಣವೇ ಉದ್ವೇಗಕ್ಕೆ ವಶರಾಗುವುದು, ನಾನೇನೋ ಒಬ್ಬ ದೊಡ್ಡ ಭಕ್ತ, ಅಂಥವರಲ್ಲಿ ಶರಣಾಗಿರುವೆನು ಎಂದು ಹೆಮ್ಮೆ ಕೊಚ್ಚಿಕೊಳ್ಳುವುದು ಮತ್ತು ಹಲವು ದೃಶ್ಯಗಳನ್ನು ನೋಡುವುದು ಇವುಗಳನ್ನೆಲ್ಲ\break ಗಮನಿಸಕೂಡದು. ಎಲ್ಲಕ್ಕಿಂತ ಹೆಚ್ಚಾಗಿ ಶುದ್ಧ ಚಾರಿತ್ರ್ಯ ಈ ಜೀವನದಲ್ಲಿ ಆವಶ್ಯಕ. ಅದರ ಕಡೆ ಗಮನ ಕೊಡುವುದು ಅತ್ಯಂತ ಮುಖ್ಯ ಎಂದು ನರೇಂದ್ರ ಇತರ ಶಿಷ್ಯರಿಗೆ ಒತ್ತಿ ಒತ್ತಿ ಹೇಳುತ್ತಿದ್ದ. ಈ ಆಧ್ಯಾತ್ಮಿಕ ಜೀವನದಲ್ಲಿ ನೂರಕ್ಕೆ ಎಂಭತ್ತು ಜನ ಕಪಟಿಗಳಾಗುವರು, ಹದಿನೈದು ಜನ ಹುಚ್ಚರಾಗುವರು, ಎಲ್ಲೋ ಉಳಿದೈದು ಮಂದಿ ದೇವರೆಡೆಗೆ ನೇರವಾಗಿ ಹೋಗಲು ಯೋಗ್ಯರಾಗುವರು ಎಂದು ಹೇಳುತ್ತಿದ್ದನು.

ಶ‍್ರೀರಾಮಕೃಷ್ಣರ ಸ್ಥಿತಿ ಉತ್ತಮಗೊಳ್ಳಲಿಲ್ಲ. ವೈದ್ಯರು ಕಲ್ಕತ್ತೆಗೆ ಸ್ವಲ್ಪ ದೂರದಲ್ಲಿ ತೋಟದ ಮಧ್ಯದಲ್ಲಿ ಯಾವುದಾದರೊಂದು ಮನೆ ನೋಡಿ ಅಲ್ಲಿಗೆ ಹೋದರೆ ಸ್ಥಳ ಬದಲಾವಣೆಯಿಂದ ಸ್ವಲ್ಪ ಉತ್ತಮಗೊಳ್ಳಬಹುದೆಂದು ಸೂಚಿಸಿದರು. ಅದರಂತೆಯೇ ಕಾಶೀಪುರದ ಉದ್ಯಾನದ ಮನೆಗೆ ಶ‍್ರೀರಾಮಕೃಷ್ಣರನ್ನು ಕರೆದುಕೊಂಡು ಹೋದರು. ಸ್ಥಳ ವಿಶಾಲವಾಗಿತ್ತು. ಸುತ್ತಲೂ ಹಲವು ಕೊಳಗಳು ಮತ್ತು ಉದ್ಯಾನ. ಇವುಗಳ ಮಧ್ಯದಲ್ಲಿದ್ದಾಗ ಶ‍್ರೀರಾಮಕೃಷ್ಣರು ಸ್ವಲ್ಪ ಚೇತರಿಸಿಕೊಂಡಂತೆ ತೋರಿದರು.

