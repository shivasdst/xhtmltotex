
\chapter{ಭರತಖಂಡದ ಕಡೆಗೆ }

 ಸ್ವಾಮೀಜಿ ಮತ್ತು ಇಂಡಿಯಾದೇಶದಲ್ಲಿ ನೆಲಸಲು ಹೊರಟ ಸೇವಿಯರ್ಸ್‍‍ ದಂಪತಿಗಳು ಇಂಗ್ಲೆಂಡನ್ನು ಡಿಸೆಂಬರ್ ಹದಿನೈದನೇ ತಾರೀಖು ಬಿಟ್ಟರು. ಇಂಗ್ಲೆಂಡ್ ದೇಶವನ್ನು ಬಿಡುವುದಕ್ಕೆ ಮುಂಚೆ ಒಬ್ಬ ಆಂಗ್ಲೇಯ ಶಿಷ್ಯರು ಸ್ವಾಮೀಜಿ ಅವರನ್ನು “ಸ್ವಾಮೀಜಿ, ನೀವು ನಾಲ್ಕು ವರುಷಗಳು ಭೋಗಭೂಮಿಯೆನಿಸಿದ ಬಲಾಢ್ಯವಾದ ಮಹಿಮಾನ್ವಿತವಾದ ಪಾಶ್ಚಾತ್ಯದೇಶದಲ್ಲಿ ವಾಸಮಾಡಿದ ಮೇಲೆ ಈಗ ಭರತಖಂಡವನ್ನು ಹೇಗೆ ನೋಡುತ್ತೀರಿ?” ಎಂದು ಪ್ರಶ್ನೆಮಾಡಿದರು. ಅದಕ್ಕೆ ಸ್ವಾಮೀಜಿ, “ನಾನು ಬರುವುದಕ್ಕೆ ಮುಂಚೆ ಇಂಡಿಯಾ ದೇಶವನ್ನು ಪ್ರೀತಿಸುತ್ತಿದ್ದೆ. ಈಗ ಭರತಖಂಡದ ಧೂಳು ಕೂಡ ನನಗೆ ಪವಿತ್ರವಾಗಿದೆ. ಅದರ ಮೇಲೆ ಬೀಸುವ ಗಾಳಿ ಪವಿತ್ರವಾಗಿದೆ. ಈಗ ಅದು ಪುಣ್ಯಭೂಮಿಯಾಗಿದೆ. ತೀರ್ಥಕ್ಷೇತ್ರವಾಗಿದೆ ” ಎಂದರು. 

 ಸ್ವಾಮೀಜಿ ಮತ್ತು ಅವರ ಜತೆಗಾರರು ಡೋವರ್, ಕಲೆಸ್, ಮೌಂಟ್ ಸೀನಿಸ್ ಮೂಲಕ ಮಿಲಾನ್ ನಗರಕ್ಕೆ ಹೋದರು. ಸ್ವಾಮೀಜಿಯವರ ಮನಸ್ಸಿನಲ್ಲಿ ತಾವು ಇಂಡಿಯಾ ದೇಶಕ್ಕೆ ಹಿಂತಿರುಗಿ ಹೋಗುತ್ತಿರುವೆ ಎಂಬುದೇ ಆನಂದವನ್ನು ಕೊಡುತ್ತಿತ್ತು. ಪ್ರಯಾಣವನ್ನು ಆನಂದಿಸಿದರು. ಸುತ್ತಲಿರುವುದನ್ನೆಲ್ಲ ಮಕ್ಕಳಂತೆ ನೋಡಿ ನಲಿದರು. ಮುಂದೆ ತಾವು ಭರತಖಂಡದಲ್ಲಿ ಏನನ್ನು ಮಾಡಬೇಕೆಂದಿರುವೆ ಎಂಬ ವಿಷಯದಲ್ಲಿ ಸೇವಿಯರ್ಸ್‍‍ ದಂಪತಿಗಳೊಡನೆ ಮಾತನಾಡುತ್ತಿದ್ದರು. ಮಿಲಾನ್ ನಗರವನ್ನು ಸೇರಿದ ಮೇಲೆ ಅಲ್ಲಿಯ ಪ್ರಮುಖ ಚರ‍್ಚಿನ ಸಮೀಪದಲ್ಲಿ ಒಂದು ಹೋಟಲಿನಲ್ಲಿ ತಂಗಿದ್ದರು. ಲಿಯನಾರ್ಡೊ ಡ ವಿನ್ಸಿ ಎಂಬ ಶ್ರೇಷ್ಠ ಕಲೆಗಾರ ಬರೆದ \enginline{Last Supper} ಎಂಬ ಚಿತ್ರವನ್ನು ನೋಡಿ ಆನಂದಿಸಿದರು. ಮಿಲಾನ್ ನಗರವನ್ನು ನೋಡಿಕೊಂಡು ಪೀಸಾ ನಗರಕ್ಕೆ ಹೋದರು. ಅಲ್ಲಿ ವಾಲಿರುವ ಗೋಪುರ (\enginline{Leaning tower of Pisa}) ಚರ‍್ಚು ಪಾದ್ರಿಗಳು ಇರುವ ಮಠಗಳು ಮುಂತಾದವುಗಳನ್ನೆಲ್ಲ ನೋಡಿದರು. ಮಿಲಾನ್ ಮತ್ತು ಪೀಸಾನಗರಗಳಲ್ಲಿರುವ ಬಿಳಿಯ ಮತ್ತು ಕಪ್ಪು ಅಮೃತ ಶಿಲೆಯಿಂದ ಮಾಡಿದ ಕೆತ್ತನೆಯ ಕೆಲಸಗಳನ್ನು ಮೆಚ್ಚಿದರು. ಅನಂತರ ಫ್ಲಾರೆನ್ಸ್ ನಗರಕ್ಕೆ ಬಂದರು. ಆರ‍್ನೋ ನದಿಯ ತೀರದಲ್ಲಿ ಬೆಟ್ಟಗಳಿಂದ ಆವೃತವಾದ ನಗರ ಬಹಳ ಸುಂದರವಾಗಿತ್ತು. ಅಲ್ಲಿ ಇತಿಹಾಸ ಪ್ರಸಿದ್ಧವಾದ ಹಲವಾರು ಸ್ಥಳಗಳನ್ನು ನೋಡಿದರು. ಅಲ್ಲಿರುವ ಕಲಾಭವನವನ್ನು ನೋಡಿದರು. ಉದ್ಯಾನವನಗಳಿಗೆ ಹೋಗಿ ಬಂದರು. ಅಲ್ಲೆ ಕ್ರೈಸ್ತ ಸಾಧು ಸವನರೋಲಾ ಎಂಬುವನು ಇದ್ದದ್ದು. ಅವನ ವಿಷಯವನ್ನು ಮನನಮಾಡತೊಡಗಿದರು. 

 ಫ್ಲಾರೆನ್ಸ್ ನಲ್ಲಿ ಅವರಿಗೆ ಒಂದು ವಿಸ್ಮಯ ಕಾದುಕೊಂಡಿತ್ತು. ಅವರು ಉದ್ಯಾನವನದಲ್ಲಿ ಹೋಗುತ್ತಿದ್ದಾಗ ಚಿಕಾಗೊ ನಗರದ ಹೇಲ್ಸ್ ದಂಪತಿಗಳನ್ನು ಅಕಸ್ಮಾತ್ತಾಗಿ ಕಂಡರು. ಸ್ವಾಮೀಜಿ ವಿಶ್ವಧರ್ಮ ಸಮ್ಮೇಳನಕ್ಕೆ ಹೋಗುವುದಕ್ಕೆ ಮುಂಚೆ ದಾರಿ ತಪ್ಪಿದಾಗ ಅವರ ಮನೆಯಲ್ಲಿ ತಂಗಿದ್ದರು ಮತ್ತು ಅನಂತರವೂ ಚಿಕಾಗೊ ನಗರಕ್ಕೆ ಬಂದಾಗಲೆಲ್ಲ ಅವರ ಮನೆಯಲ್ಲಿ ಬಹಳ ಕಾಲ ಇರುತ್ತಿದ್ದರು. ಅವರನ್ನು ಫಾದರ್ ಪೋಪ್ ಎಂತಲೂ ಶ‍್ರೀಮತಿ ಅವರನ್ನು ಮದರ್ ಚರ್ಚ್ ಎಂತಲೂ ಸ್ವಾಮೀಜಿ ಹಾಸ್ಯದಿಂದ ಕರೆಯುತ್ತಿದ್ದರು. ಕೆಲವು ಗಂಟೆಗಳ ಕಾಲ ಎಲ್ಲಾ ಒಟ್ಟಿಗೆ ಸೇರಿಕೊಂಡು ಮಾತನಾಡಿದರು. ತಮ್ಮ ಹಿಂದಿನದನ್ನೆಲ್ಲ ಸ್ಮರಿಸಿಕೊಂಡರು. 

 ಫ್ಲಾರೆನ್ಸ್ ನಿಂದ ಸ್ವಾಮೀಜಿಯವರು ರೋಮ್ ನಗರಕ್ಕೆ ಬಂದರು. ಸ್ವಾಮೀಜಿ ಆ ಪುರಾತನ ಪ್ರಖ್ಯಾತವಾದ ನಗರವನ್ನು ನೋಡಬೇಕೆಂದು ಬಹಳ ಆಶಿಸುತ್ತಿದ್ದರು. ಅವರು ಒಂದು ವಾರ ಇಲ್ಲಿ ಕಳೆದರು. ಲಂಡನ್ ಬಿಡುವುದಕ್ಕೆ ಮುಂಚೆ ಶ‍್ರೀಮತಿ ಸೇವಿಯರ್ಸ್‍‍ ಅವರು ಮಿಸ್ ಮ್ಯಾಕ್ಲಿಯಾಡ್ ಮೂಲಕ ರೋಮ್ ನಗರದಲ್ಲಿ ಆಕೆಗೆ ಪರಿಚಯವಿರುವ ಒಬ್ಬರ ವಿಳಾಸವನ್ನು ತೆಗೆದುಕೊಂಡು ಬಂದಿದ್ದರು. ಅವಳ ಹೆಸರು ಮಿಸ್ ಎಡ್‍ವರ್ಡ್ಸ್. ಆಕೆಯೊಡನೆ ಮ್ಯಾಕ್ಲಿಯಾಡ್ ಸಹೋದರಿಯ ಮಗಳಾದ ಮಿಸ್ ಆಲ್‍ಬರ‍್ಟಾ ಸ್ಟರ್‍ಗಿಸ್ ಎಂಬುವಳು ಇದ್ದಳು. ಇವರಿಬ್ಬರೂ ಕೂಡ ರೋಮ್ ನಗರವನ್ನು ನೋಡುವ ಪರ್ಯಟನೆಯಲ್ಲಿ ಜೊತೆಗೂಡಿದರು. ಮಿಸ್ ಎಡ್‍ವರ್ಡ್ಸ್ ಸ್ವಾಮೀಜಿಯವರ ವೇದಾಂತ ತತ್ತ್ವವನ್ನು ಮೆಚ್ಚಿದಳು. ಸ್ವಾಮೀಜಿಯವರಿಗೆ ರೋಮನ್ ಚರಿತ್ರೆಗೆ ಸಂಬಂಧಪಟ್ಟ ಹಲವು ವಿಷಯಗಳ ಪರಿಚಯ ಅಗಾಧವಾಗಿತ್ತು. ಆಕೆ ಇದನ್ನು ಸ್ವಾಮಿಗಳಲ್ಲಿ ಕಂಡು ವಿಸ್ಮಿತಳಾದಳು. 

 ರೋಮಿನಲ್ಲಿ ಕಂಡುದನ್ನೆಲ್ಲ ಸ್ವಾಮೀಜಿ ಮೆಚ್ಚಿದರು. ದೊಡ್ಡ ಗೋಪುರವನ್ನೊಳಗೊಂಡ ಸೆಂಟ್ ಪೀಟರ‍್ಸ ಚರ್ಚ್ ನೋಡಿದರು. ಆ ಚರ್ಚಿನಲ್ಲಿ ಪೂಜೆ ಸಮಯದಲ್ಲಿ ರೂಢಿಯಲ್ಲಿರುವ ಆಚಾರಗಳನ್ನೆಲ್ಲ ಗಮನಿಸಿದರು. ಅವುಗಳೆಲ್ಲ ಇಂಡಿಯಾದೇಶದ ಆಚಾರಗಳಂತೆಯೇ ಇವೆ ಎಂದು ಹೇಳಿದರು. ಭಕ್ತರು “ಸ್ವಾಮೀಜಿ ನೀವು ಈ ಆಚಾರಗಳನ್ನು ಮೆಚ್ಚುತ್ತೀರಾ?” ಎಂದು ಕೇಳಿದರು. ಅದಕ್ಕೆ ಸ್ವಾಮೀಜಿ, “ನೀವು ದೇವರ ಸಾಕಾರವನ್ನು ಒಪ್ಪಿಕೊಂಡರೆ ಹೂವು ಗಂಧ ಹಣ್ಣು ಬಟ್ಟೆ ಎಲ್ಲವನ್ನೂ ಕೊಡಿ. ಅವನಿಗೆ ಅರ್ಪಣೆ ಮಾಡಲು ಯೋಗ್ಯವಾಗಿರುವಂಥದು ಈ ಪ್ರಪಂಚದಲ್ಲೆ ಇಲ್ಲ” ಎಂದರು. ರೋಮ್ ದೇಶದ ಹಿಂದಿನ ಅವಶೇಷಗಳನ್ನು ನೋಡಿದರು. ಅದು ಕ್ರಿಸ್‍ಮಸ್ಸಿನ ಸಮಯ. ಎಲ್ಲೆಲ್ಲಿಯೂ ಮೇರಿ ಮತ್ತು ಬಾಲಕ್ರಿಸ್ತನ ಅಲಂಕಾರಗಳನ್ನು ಮಾಡುತ್ತಿದ್ದರು. ಅದೊಂದು ಕ್ರಿಸ್ತನ ವಾತಾವರಣದಿಂದ ತುಂಬಿ ತುಳುಕುತ್ತಿದ್ದ ಕಾಲ. ಸ್ವಾಮೀಜಿ ಅದರಲ್ಲಿ ಒಂದಾದರು. 

 ರೋಮ್ ಸುತ್ತಮುತ್ತಲಿರುವ ಅರಮನೆ, ಪೋರಂ, ಪಲಟೈನ್ ಬೆಟ್ಟ, ವೆಸ್ಟ್ ದೇವಾಲಯ, ಪುರಾತನ ರೋಮನ್ನರ ಈಜುವ ಸ್ಥಳಗಳು, ದೊಡ್ಡ ದೊಡ್ಡ ಕಂಬಗಳು ಕ್ಯಾಪಿಡಿಲೈನ್ ಗುಡ್ಡ, ಮೆರಿಯಾ ಚರ್ಚ್ ಮತ್ತು ಪೋಪುಗಳು ಇರುವ ಸ್ಥಳ ಇವುಗಳನ್ನೆಲ್ಲ ನೋಡಿದರು. ಸ್ವಾಮೀಜಿ ಟ್ರೋಜನ್ ಕಂಬಗಳನ್ನು ನೋಡಿದರು. ಅವು ೧೧೭ ಅಡಿ ಎತ್ತರವಾಗಿವೆ. ಅಲ್ಲಿ ಟ್ರೋಜನ್ ಜನರು ತಮ್ಮ ಸಾಹಸವನ್ನೆಲ್ಲಾ ಕೆತ್ತಿರುವರು. ಸುಮಾರು ಎರಡು ಸಾವಿರ ಮನುಷ್ಯಾಕೃತಿಗಳು ಅಲ್ಲಿವೆ. ಜೆರುಸಲೆಂ ನಗರವನ್ನು ಗೆದ್ದ ಜ್ಞಾಪಕಾರ್ಥವಾಗಿ ಟೈಟಸ್ ಕಟ್ಟಿಸಿದ ವಿಜಯದ್ವಾರವನ್ನು ನೋಡಿದರು. ಅವು ಆಗಲೂ ಸರಿಯಾದ ಸ್ಥಿತಿಯಲ್ಲೇ ಇಡಲ್ಪಟ್ಟಿದ್ದವು. ರೋಮ್ ಹಿಂದಿನ ಕಾಲದಲ್ಲಿ ಎಂತಹ ಉಚ್ಛ್ರಾಯಸ್ಥಿತಿಗೆ ಹೋಗಿತ್ತು ಎಂತಹ ದೊಡ್ಡ ಕಟ್ಟಡಗಳು ಸೇತುವೆಗಳು ರಾಜಮಾರ್ಗಗಳು ಇವುಗಳನ್ನು ನಿರ್ಮಿಸಿದರು, ಅವರ ಭೋಗ ಐಶ್ವರ‍್ಯ, ಅನಂತರ ಅವರ ಅವನತಿ, ಚಕ್ರವರ್ತಿಗಳಿದ್ದ ಕಡೆ ಈಗ ಜೇಡರ ಹುಳು ಬಲೆಯನ್ನು ಕಟ್ಟುತ್ತಿರುವುದು ಇವನೆಲ್ಲ ನೋಡಿದರು. ಕಾಲದ ಮಹಾ ಪ್ರವಾಹದಲ್ಲಿ ಚಕ್ರಾಧಿಪತ್ಯಗಳು ನೀರಿನ ಗುಳ್ಳೆಗಳಂತೆ ಕೆಲವು ಕಾಲ ತೇಲಿ ಅನಂತರ ಭಗ್ನವಾಗಿರುವುದನ್ನು ಮನಗಂಡರು. 

 ಕ್ರಿಸ್‍ಮಸ್ ದಿನಗಳಾದುದರಿಂದ ಮೇರಿಯ ಚರ್ಚಿನ ಮುಂದೆ ಒಂದು ಜಾತ್ರೆಯಂತೆ ಇತ್ತು. ಚರ್ಚಿನ ಸುತ್ತಲೂ ಅಂಗಡಿಗಳು. ಅದರಲ್ಲಿ ಹುಡುಗರಿಗೆ ಬೇಕಾಗುವ ಆಟದ ಸಾಮಾನು ಮತ್ತು ತಿಂಡಿತೀರ್ಥಗಳು, ಮೇರಿ ಮತ್ತು ಎಳೆಯ ಮಗು ಕ್ರಿಸ್ತನ ಚಿತ್ರಗಳು ಇವುಗಳನ್ನೆಲ್ಲ ಮಾರುತ್ತಿದ್ದರು. ಸ್ವಾಮೀಜಿಯವರಿಗೆ ಇದನ್ನು ನೋಡಿದಾಗ ಇಂಡಿಯಾ ದೇಶದ ಜಾತ್ರೆಯ ದಿನಗಳು ನೆನಪಿಗೆ ಬಂದವು. 

 ರೋಮ್ ನಗರವನ್ನು ಬಿಟ್ಟಾದಮೇಲೆ ನೇಪಲ್ಸ್ ನಗರವನ್ನು ತಲುಪಿದರು. ಅಲ್ಲಿಂದ ಹಡಗು ಹೊರಡುವುದಕ್ಕೆ ಇನ್ನೂ ಹಲವು ದಿನಗಳು ಇದ್ದವು. ನೇಪಲ್ಸ್ ಸುತ್ತಮುತ್ತಲೂ ನೋಡುವುದಕ್ಕೆ ಹೊರಟರು. ವೆಸೂವಿಯಸ್ ಜ್ವಾಲಾಮುಖಿಯನ್ನು ನೋಡುವುದಕ್ಕೆ ಒಂದು ಬೆಟ್ಟದ ರೈಲ್ವೆಯಲ್ಲಿ ಹೋದರು. ಅವರು ಅದರ ನೆತ್ತಿಯ ಮೇಲೆ ಇದ್ದಾಗ, ಜ್ವಾಲಾಮುಖಿಯ ಗರ್ಭದಿಂದ ಹೊರಗೆ ಕಲ್ಲುಗಳು ಹಾರಿ ಬೀಳುತ್ತಿರುವುದನ್ನು ಕಂಡರು. ಮತ್ತೊಂದು ದಿನ ಜ್ವಾಲಾಮುಖಿಯಿಂದ ನಿಮಿಷಾರ್ಧದಲ್ಲಿ ಹಿಂದೆ ನಾಶವಾಗಿ ಹೋದ ಪಾಂಪೆಯ ನಗರವನ್ನು ನೋಡಲು ಹೋದರು. ಅಲ್ಲಿ ಒಂದು ಮನೆಯನ್ನು ಮಣ್ಣಿನಿಂದ ಬಿಡಿಸಿ ಅದರ ಒಳಗೆ ಹಿಂದಿನ ಕಾಲದಲ್ಲಿ ಯಾವ ಯಾವ ವಸ್ತುಗಳು ಹೇಗೆ ಇದ್ದುವೋ ಅದನ್ನೆಲ್ಲ ಹಾಗೆಯೇ ಜೋಡಿಸಿದ್ದರು. ಸ್ವಾಮೀಜಿ ಈ ಪಟ್ಟಣವನ್ನು ನೋಡಿ ಆದಮೇಲೆ “ಆಧುನಿಕ ನಾಗರಿಕತೆಯ ಮೇಲೆ ನನಗಿದ್ದ ಗೌರವಗಳೆಲ್ಲ ಪಲಾಯನವಾಯಿತು. ಆವಿ, ವಿದ್ಯುತ್ ಎರಡನ್ನು ಬಿಟ್ಟರೆ ಮಿಕ್ಕ ಎಲ್ಲವು ಅವರಿಗೆ ಇತ್ತು. ಆಧುನಿಕ ಕಾಲಕ್ಕಿಂತ ಹೆಚ್ಚು ಕಲಾಭಾವನೆಗಳು ಮತ್ತು ಅದನ್ನು ಕಾರ್ಯಗತಮಾಡುವ ಶಕ್ತಿ ಅವರಲ್ಲಿತ್ತು” ಎಂದು ಕಾಗದ ಬರೆದಿರುವರು. ಅಲ್ಲಿರುವ ಆಕ್ವೇರಿಯಂ (ಜಲಚರ ಪ್ರಾಣಿಗಳನ್ನಿಟ್ಟಿರುವ ಪ್ರದರ್ಶನಾಲಯ) ಮತ್ತು ಮ್ಯೂಸಿಯಂಗಳನ್ನು ನೋಡಿದರು. ಇಂಡಿಯಾ ದೇಶಕ್ಕೆ ಅವರು ಹೊರಡುವ ಹಡಗು ಮೂವತ್ತನೆಯ ತಾರೀಖು ಬಂದಿತು. ಅದರಲ್ಲಿ ಇವರೊಡನೆ ಇಂಡಿಯಾ ದೇಶಕ್ಕೆ ಬರುವ ಜೆ. ಜೆ. ಗುಡ್‍ವಿನ್ ಕೂಡ ಇಂಗ್ಲೆಂಡಿನಿಂದ ಬಂದಿದ್ದರು. ಹಡಗು ಜನವರಿ ಹದಿನೈದನೆಯ ತಾರೀಖು ಕೊಲಂಬೊ ನಗರವನ್ನು ಸೇರುವುದಿತ್ತು. ಸುಮಾರು ಹದಿನೈದು ದಿನಗಳು ಸಮುದ್ರ ಪ್ರಯಾಣವನ್ನು ಸ್ವಾಮೀಜಿ ಆನಂದದಿಂದ ಕಳೆದರು. 

 ಒಮ್ಮೆ ಹಡಗು ಕ್ರೀಟ್ ದ್ವೀಪದ ಸಮೀಪದಲ್ಲಿ ಅರ್ಧರಾತ್ರಿಯ ಸಮಯದಲ್ಲಿ ಹೋಗುತ್ತಿದ್ದಾಗ ಸ್ವಾಮೀಜಿಯವರಿಗೆ ಒಂದು ವಿಚಿತ್ರವಾದ ಕನಸಾಯಿತು. ಆ ಕನಸಿನಲ್ಲಿ ಒಬ್ಬ ನೀಳವಾದ ಜಡೆಗಳಿಂದ ಕೂಡಿದ ಗೌರವ ವ್ಯಕ್ತಿ ಕಾಣಿಸಿಕೊಂಡು ಹೀಗೆ ಹೇಳಿದ: “ನಾನು ನಿನಗೆ ತೋರುವ ಈ ಸ್ಥಳವನ್ನು ಚೆನ್ನಾಗಿ ಗಮನದಲ್ಲಿಡು. ನೀನು ಈಗ ಕ್ರೀಟ್ ದ್ವೀಪದಲ್ಲಿರುವೆ. ಇಲ್ಲಿಯೇ ಕ್ರೈಸ್ತಧರ್ಮ ಪ್ರಾರಂಭವಾಯಿತು. ನಾನೊಬ್ಬ ಇಲ್ಲಿ ವಾಸಿಸುತ್ತಿದ್ದ ಥೆರಪುಟ (ಅಂದರೆ ಬೌದ್ಧರ ಸಂಘಕ್ಕೆ ಸೇರಿದ ಭಿಕ್ಷು). ನಾವು ಬೋಧಿಸಿದ ಸಂದೇಶಗಳನ್ನೇ ಜೀಸಸ್ ಬೋಧಿಸಿದನು ಎಂದು ಅನಂತರ ಪ್ರಚಾರ ಮಾಡಿರುತ್ತಾರೆ. ಕ್ರಿಸ್ತನೆನ್ನುವ ಚಾರಿತ್ರಿಕ ವ್ಯಕ್ತಿ ಇರಲಿಲ್ಲ. ಇಲ್ಲಿರುವ ಭೂಮಿಯನ್ನು ಶೋಧನೆ ಮಾಡಿದರೆ ಅದೆಲ್ಲ ವ್ಯಕ್ತವಾಗುವುದು”. ಸ್ವಾಮೀಜಿ ಕನಸಿನಿಂದ ಎದ್ದರು. ತಮ್ಮ ಕೋಣೆಯಿಂದ ಹೊರಗೆ ಹೋಗಿ ನೋಡಿದಾಗ ಹಡಗಿನ ಒಬ್ಬ ಅಧಿಕಾರಿ ತಾರಾಡುತ್ತಿದ್ದನು. ಅವನನ್ನು ಹತ್ತಿರದಲ್ಲಿರುವುದು ಯಾವ ಊರು ಎಂದು ಕೇಳಿದರು. ಅದಕ್ಕೆ ಆತ, ಅದು ಕ್ರೀಟ್ ದ್ವೀಪವೆಂದೂ, ಸುಮಾರು ನಲವತ್ತು ಐವತ್ತು ಮೈಲಿಗಳ ದೂರದಲ್ಲಿದೆ ಎಂದೂ ಹೇಳಿದನು. ಸ್ವಾಮೀಜಿಗೆ ಈ ಕನಸಾದ ಮೇಲೆ ಕ್ರಿಸ್ತ ನಿಜವಾದ ಒಬ್ಬ ಚಾರಿತ್ರಿಕ ವ್ಯಕ್ತಿ ಆಗಿದ್ದ ಎಂಬ ನಂಬಿಕೆ ಜಾರಿಹೋಯಿತು. ಆದರೆ ಕ್ರಿಸ್ತನ ಆದರ್ಶವನ್ನು ಹೃತ್ಪೂರ್ವಕ ಗೌರವಿಸುತ್ತಿದ್ದರು. ಅನಂತರ ಅಲ್ಲಿ ಆರ್ಕಿಯಾಲಜಿ ಡಿಪಾರ್ಟ್‍ಮೆಂಟಿನವರು ಸಂಶೋಧನೆ ಮಾಡಿದಾಗ ಬೌದ್ಧ ಧರ್ಮದ ಅವಶೇಷಗಳು ದೊರೆತುದಾಗಿ ವರದಿಯಾದವು. ಇದು ಸ್ವಾಮೀಜಿ ಕಾಲಾನಂತರ ಬೆಳಕಿಗೆ ಬಂದ ಸುದ್ದಿ. 

 ಹಡಗಿನಲ್ಲಿ ಹೋಗುತ್ತಿದ್ದಾಗ ಒಂದು ತಮಾಷೆಯ ಪ್ರಸಂಗ ಒಂದು ದಿನ ನಡೆಯಿತು. ಆ ಹಡಗಿನಲ್ಲಿ ಇಬ್ಬರು ಕ್ರೈಸ್ತ ಪಾದ್ರಿಗಳು ಇಂಡಿಯಾದೇಶಕ್ಕೆ ಬೋಧಿಸುವುದಕ್ಕೆ ಹೋಗುತ್ತಿದ್ದರು. ಅವರು ಸ್ವಾಮೀಜಿ ಹತ್ತಿರ ಪ್ರತಿದಿನವೂ ಧಾರ್ಮಿಕ ವಿಷಯಗಳ ಚರ್ಚೆಯನ್ನು ಮಾಡುತ್ತಿದ್ದರು. ಕ್ರೈಸ್ತಧರ್ಮದಲ್ಲಿರುವ ತತ್ತ್ವ ಮತ್ತು ವಿಚಾರಗಳ ಚರ್ಚೆಯನ್ನು ಮಾಡುತ್ತಿದ್ದರು. ಕ್ರೈಸ್ತಧರ್ಮದ ತತ್ತ್ವಗಳು ಹಿಂದೂ ದರ್ಶನದ ಎದುರಿಗೆ ನಿಲ್ಲಲಾರವು. ಪ್ರತಿಸಲವೂ ಆ ಪಾದ್ರಿಗಳು ಸ್ವಾಮೀಜಿಯವರ ಹತ್ತಿರ ವಾದಮಾಡಿ ಸೋತಾಗ ಸುಮ್ಮನೆ ಇದ್ದರು. ಅನಂತರ ಒಂದು ದಿನ ಒಬ್ಬ ಪಾದ್ರಿಯು ಹಡಗಿನ ಅಂಚಿನಲ್ಲಿದ್ದಾಗ ಸ್ವಾಮೀಜಿ ಅವನ ಸಮೀಪಕ್ಕೆ ಹೋಗಿ. ಅವನನ್ನು ಬಿಗಿಯಾಗಿ ಹಿಡಿದು ಅರ್ಧ ಹಾಸ್ಯವಾಗಿ ಅರ್ಧ ಗಂಭೀರವಾಗಿ “ನೋಡು, ನೀನು ನನ್ನ ಧರ್ಮವನ್ನು ಬಯ್ಯುವುದನ್ನು ಬಿಡದೆ ಇದ್ದರೆ ನಿನ್ನನ್ನು ಹೀಗೆಯೇ ಸಮುದ್ರಕ್ಕೆ ನೂಕಿ ಬಿಡುವೆ!” ಎಂದು ಹೆದರಿಸಿದರು. ಆ ಪಾದ್ರಿಯ ಕೈಕಾಲುಗಳು ಅಂಜಿಕೆಯಿಂದ ನಡುಗತೊಡಗಿದವು. “ದಯವಿಟ್ಟು ಬಿಟ್ಟುಬಿಡಿ, ನಾನು ಇನ್ನುಮೇಲೆ ಹಾಗೆ ಮಾಡುವುದಿಲ್ಲ” ಎಂದು ಬೇಡಿಕೊಂಡನು. ಅನಂತರ ಸ್ವಾಮೀಜಿ ಕೈಸಡಿಲಿಸಿದರು. ಅಂದಿನಿಂದ ಆತ ಹಡಗಿನಲ್ಲಿರುವ ತನಕ ಸ್ವಾಮೀಜಿಗೆ ಗೌರವವನ್ನು ತೋರಿಸುತ್ತಿದ್ದ. 

 ಸ್ವಾಮೀಜಿ ಕಲ್ಕತ್ತೆಯಲ್ಲಿದ್ದಾಗ ಈ ವಿಷಯವನ್ನು ಕುರಿತು ತಮ್ಮ ಶಿಷ್ಯನೊಂದಿಗೆ ಮಾತನಾಡುತ್ತಿದ್ದಾಗ ಸಂಭಾಷಣೆ ಹೀಗೆ ಬರುವುದು: 

 ಸ್ವಾಮೀಜಿ: “ಪ್ರಿಯ ಸಿನ್ಹ, ಯಾರಾದರೂ ನಿನ್ನ ತಾಯಿಯನ್ನು ನಿಂದಿಸಿದರೆ ನೀನೇನು ಮಾಡುವೆ?” 

 ಶಿಷ್ಯ: “ಅವನಮೇಲೆ ಬಿದ್ದು ಅವನಿಗೆ ಚೆನ್ನಾಗಿ ಬುದ್ಧಿ ಕಲಿಸುತ್ತೇನೆ.” 

 ಸ್ವಾಮೀಜಿ: “ಸರಿ ನೀನು ಹೇಳಿದ್ದು. ಅದರಂತೆಯೇ ನಮ್ಮ ದೇಶದ ನಿಜವಾದ ತಾಯಿಯೇ ನಮ್ಮ ಧರ್ಮ. ಅದರ ಮೇಲೆ ಅಂತಹ ಅಭಿಮಾನ ಇರಬೇಕು. ಆಗ ಹಿಂದೂ ಕ್ರೈಸ್ತನಾಗಿ ಮತಾಂತರವಾಗುವುದನ್ನು ನಿನಗೆ ಸಹಿಸಲು ಅಸಾಧ್ಯವಾಗುವುದು. ಪ್ರತಿದಿನ ಇದು ಆಗುತ್ತಿರುವುದನ್ನು ನೀನು ನೋಡುತ್ತಿರುವೆ. ಆದರೂ ಸುಮ್ಮನೆ ಇರುವೆ. ನಿನ್ನ ಧರ್ಮಶ್ರದ್ಧೆಯೆಲ್ಲಿ? ನಿನ್ನ ದೇಶಭಕ್ತಿಯೆಲ್ಲಿ? ನಿಮ್ಮ ಕಣ್ಣೆದುರಿಗೇ ಪ್ರತಿದಿನ ಕ್ರೈಸ್ತಪಾದ್ರಿಗಳು ನಿಮ್ಮ ಧರ್ಮವನ್ನು ಅಲ್ಲಗಳೆಯುವರು. ಅದನ್ನು ರಕ್ಷಿಸುವುದಕ್ಕೆ ಎಷ್ಟು ಜನ ನಿಮ್ಮಲ್ಲಿ ಮುಂದೆ ಬರುವರು? ಇಂತಹ ದ್ರೋಹವನ್ನು ಮಾಡುತ್ತಿರುವಾಗ ನಿಮ್ಮ ರಕ್ತ ಕುದಿಯುವುದಿಲ್ಲವೆ?” 

 ಸ್ವಾಮೀಜಿ ಸರ್ವಸಂಗ ಪರಿತ್ಯಾಗಮಾಡಿದ ಸಾಧು. ಆದರೆ ಅವರು ಧರ್ಮ ಮತ್ತು ದೇಶವನ್ನು ಯಾರಾದರೂ ಹೀಯಾಳಿಸಿದರೆ ಸರ್ಪದಂತೆ ಹೆಡೆ ಎತ್ತಿ ಅವರ ಮೇಲೆ ಪ್ರತಿವಾದಗಳ ಶರವನ್ನು ಕರೆಯುತ್ತಿದ್ದರು. 

 ಏಡನ್ನಿಗೆ ಹಡಗು ಬಂದಿತು. ಅಲ್ಲಿ ಸ್ವಲ್ಪ ಕಾಲ ನಿಂತಿತು. ಏಡನ್ ನಗರವನ್ನು ನೋಡಲು ಸ್ವಾಮೀಜಿ ಸೇವಿಯರ‍್ಸ ದಂಪತಿಗಳೊಡನೆ ಹೊರಟರು. ಅಲ್ಲಿ ಪೇಟೆ ಬೀದಿಯೊಂದರಲ್ಲಿ ಒಬ್ಬ ಹಿಂದೂಸ್ತಾನಿ ಮನುಷ್ಯನು ಸಣ್ಣದೊಂದು ವೀಳ್ಯದೆಲೆ ಅಡಿಕೆಯ ಅಂಗಡಿಯನ್ನು ಇಟ್ಟುಕೊಂಡಿದ್ದನು. ಆತ ಒಂದು ಗುಡಿಗುಡಿಯಿಂದ ತಂಬಾಕನ್ನು ಸೇದುತ್ತಿದ್ದ. ಸ್ವಾಮೀಜಿ ಅದನ್ನು ನೋಡಿದೊಡನೆ ಅವನ ಹತ್ತಿರ ಹೋದರು. “ತಮ್ಮ, ನನಗೆ ಸ್ವಲ್ಪ ಸೇದಲು ಕೊಡು” ಎಂದು ಕೇಳಿ ಅವನಿಂದ ಅದನ್ನು ತೆಗೆದುಕೊಂಡು ಸೇದಿ ಬಂದರು. 

 ಜನವರಿ ಹದಿನೈದನೆಯ ತಾರೀಖು ಬೆಳಿಗ್ಗೆ ದೂರದಲ್ಲಿ ಸಿಂಹಳದ್ವೀಪದ ತೆಂಗಿನಮರಗಳು ಕಾಣತೊಡಗಿದವು. ಸುಮಾರು ನಾಲ್ಕು ವರ್ಷಗಳಾದ ಮೇಲೆ ಭರತಖಂಡಕ್ಕೆ ಸಮೀಪದಲ್ಲಿರುವ ಒಂದು ಮೂಲೆಯ ದರ್ಶನ ಸ್ವಾಮೀಜಿಗೆ ಆಗಲಿದೆ. ಅತ್ಯಂತ ಉತ್ಸಾಹದಿಂದ ಹೊರಗೆ ಬಂದು ನೋಡತೊಡಗಿದರು. ಆಗ ಅವರ ಆನಂದ ವರ್ಣನಾತೀತವಾಗಿತ್ತು. ಇಡೀ ಭರತಖಂಡ ಸ್ವಾಮಿಗಳನ್ನು ಸ್ವಾಗತಿಸುವುದಕ್ಕೆ ಅಣಿಯಾಗುತ್ತಿತ್ತು. ಒಂದು ನಗರಿ ಮತ್ತೊಂದು ನಗರಿಯೊಡನೆ ಸ್ಪರ್ಧಿಸುತ್ತಿತ್ತು. ಭರತಖಂಡ ಹಿಂದೆ ಮತ್ತಾವ ದಿಗ್ವಿಜಯಿಗೂ ಇಂತಹ ಸಂಭ್ರಮದ ಸ್ವಾಗತವನ್ನು ನೀಡಿರಲಿಲ್ಲ. ಸ್ವಾಮೀಜಿ ಕನಸಿನಲ್ಲಿಯೂ ಕಲ್ಪಿಸಿಕೊಂಡಿರಲಿಲ್ಲ ತಮಗೆ ಇಂತಹ ಸಂಭ್ರಮದ ಸ್ವಾಗತ ಕಾದಿದೆ ಎಂಬುದನ್ನು. 

