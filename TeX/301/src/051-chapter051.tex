
\chapter{ಪ್ಯಾರಿಸ್ಸಿನ ಧರ್ಮ ಇತಿಹಾಸಗಳ ಸಮ್ಮೇಳನ }

 ಸ್ವಾಮೀಜಿಯವರು ಆಗಸ್ಟ್ ಮೊದಲನೆ ತಾರೀಖಿನಿಂದ ಡಿಸೆಂಬರ್ ಮಧ್ಯದವರೆಗೆ ಪ್ಯಾರಿಸ್ಸಿನಲ್ಲಿ ಇದ್ದರು. ಮಧ್ಯೆ ಮಧ್ಯೆ ಯಾವಾಗಲಾದರೊಮ್ಮೆ ಸುತ್ತಮುತ್ತಲಿರುವ ಊರನ್ನು ನೋಡುವುದಕ್ಕೆ ಹೋಗುತ್ತಿದ್ದರೂ ಪ್ಯಾರಿಸ್ ಅವರ ಕೇಂದ್ರವಾಗಿತ್ತು. ಪ್ಯಾರಿಸ್‍ನಲ್ಲಿ ಮೊದಲು ಲೆಗೆಟ್ ದಂಪತಿಗಳ ಅತಿಥಿಗಳಾಗಿ ಒಂದು ಚೆನ್ನಾದ ಮನೆಯಲ್ಲಿದ್ದರು. ಮಿಸ್ಟರ್ ಲೆಗೆಟ್ ಅವರ ಸತ್ಕಾರಕ್ಕೆ ಆಕರ್ಷಿತರಾಗಿ ಹೆಸರಾಂತ ಕವಿಗಳು ತತ್ತ್ವಜ್ಞಾನಿಗಳು, ವಿಜ್ಞಾನಿಗಳು, ಧಾರ್ಮಿಕ ವ್ಯಕ್ತಿಗಳು, ರಾಜಕಾರಣ ಪಟುಗಳು, ಸಂಗೀತಗಾರರು, ಪ್ರಾಧ್ಯಾಪಕರು ಹಲವು ಬಗೆಯ ಲಲಿತಕಲಾವಿದರು ನೆರೆಯುತ್ತಿದ್ದರು. ಬಿಡುವಿಲ್ಲದ ವಾಕ್ ಪ್ರವಾಹ ಬೆಟ್ಟದ ಝರಿಯಂತೆ ಅವಿರಳವಾಗಿ ಬೀಳುವ ಶುಭ್ರ ತಿಳಿನೀರಿನಂತೆ ಇತ್ತು. ಅಗ್ನಿಕುಂಡದಿಂದ ಸಿಡಿದ ಕಿಡಿಯಂತೆ ಎಲ್ಲರ ಹೃದಯದಿಂದಲೂ ಭಾವನೆಗಳು ಏಳುತ್ತಿದ್ದವು. ಮೈಮರೆಯುವಂತೆ ಮಾಡುವ ಸಂಗೀತ, ಪ್ರಚಂಡ ಮೇಧಾವಿಗಳಿಂದ ಬರುವ ಭಾವನಾತರಂಗಗಳ ಘರ್ಷಣೆಯ ಸಮ್ಮೋಹ, ಕುಳಿತವರಿಗೆ ಕಾಲ ದೇಶವನ್ನು ಮರೆಯುವಂತೆ ಮಾಡುತ್ತಿತ್ತು. 

 ಸ್ವಾಮೀಜಿಯವರು ಫ್ರೆಂಚ್ ಭಾಷೆಯನ್ನು ಕಲಿಯುತ್ತಿದ್ದರು. ಈ ಸಮಯದಲ್ಲಿ ಸ್ವಾಮೀಜಿ ಪ್ಯಾರಿಸ್ಸಿನಲ್ಲಿ ನಡೆಯುವ ಧರ್ಮ ಇತಿಹಾಸದ ಸಮ್ಮೇಳನದಲ್ಲಿ ಭಾಗವಹಿಸುವುದಕ್ಕಾಗಿ ಬಂದಿದ್ದರು. ಆಗ ಫ್ರೆಂಚ್ ಭಾಷೆಯಲ್ಲಿಯೇ ಮಾತನಾಡಬೇಕೆಂದು ಪ್ರಯತ್ನಪಟ್ಟು ಬಹುಮಟ್ಟಿಗೆ ಯಶಸ್ವಿಯಾಗಿದ್ದರು. ಹಿಂದೆ ಚಿಕಾಗೋ ನಗರದಲ್ಲಿ ನಡೆದಂತೆಯೇ ಒಂದು ವಸ್ತುಪ್ರದರ್ಶನಾಲಯ ಮತ್ತು ಧರ್ಮಗಳ ಇತಿಹಾಸದ ಸಮ್ಮೇಳನ ಕೆಲವು ದಿನಗಳವರೆಗೆ ಜರುಗಿತು. ಸಮ್ಮೇಳನದಲ್ಲಿ ಯಾವ ಧರ್ಮದ ವಿಷಯವಾಗಿಯೂ ಚರ್ಚೆಗೆ ಅವಕಾಶ ಕೊಡಲಿಲ್ಲ. ಅದರ ಉದ್ದೇಶ ಹಲವು ವ್ಯವಸ್ಥಿತ ಧರ್ಮಗಳು ಹೇಗೆ ಬೆಳೆದವು ಎಂಬುದನ್ನು ಚಾರಿತ್ರಿಕವಾಗಿ ಪರಿಶೀಲಿಸುವುದಾಗಿತ್ತು. ಅದರ ಜೊತೆಗೆ ಸಂಬಂಧಪಟ್ಟ ಕೆಲವು ವಿಷಯಗಳನ್ನು ಚರ್ಚಿಸಿದರು. ಆದಕಾರಣ ಬೇರೆ ಬೇರೆ ಧರ್ಮಕ್ಕೆ ಸೇರಿದ ಮಿಷನರಿ ಪಂಗಡಗಳ ಪ್ರತಿನಿಧಿಗಳಿಗೆ ಇಲ್ಲಿ ಸ್ಥಳವಿರಲಿಲ್ಲ. ಚಿಕಾಗೋ ನಗರದ ಧರ್ಮಸಮ್ಮೇಳನ ಒಂದು ಅದ್ಭುತವಾದ ಘಟನೆ ಆಗಿತ್ತು. ಪ್ರಪಂಚದ ಎಲ್ಲಾ ಧರ್ಮದ ಪ್ರತಿನಿಧಿಗಳೂ ಅಲ್ಲಿ ನೆರೆದಿದ್ದರು. ಆದರೆ ಪ್ಯಾರಿಸ್ಸಿನ ಸಮ್ಮೇಳನದಲ್ಲಿ ಆದರೋ ಬಂದಿದ್ದವರೆಲ್ಲ, ಧರ್ಮ ಹೇಗೆ ಬೆಳೆಯಿತು, ವೃದ್ಧಿಯಾಯಿತು ಎಂಬುದನ್ನು ವಿಚಾರಮಾಡುವವರು ಮಾತ್ರ. ಚಿಕಾಗೋ ಸಮ್ಮೇಳನದಲ್ಲಿ ಕ್ಯಾಥೊಲಿಕ್ ಪ್ರಭಾವ ಅಧಿಕವಾಗಿತ್ತು. ಅವರು ಸಮ್ಮೇಳನದಿಂದ ತಮ್ಮ ಪಂಗಡಕ್ಕೆ ತುಂಬಾ ಸಹಾಯವಾಗಬಹುದೆಂದು ಭಾವಿಸಿದರು. ಅಲ್ಲಿ ತಮ್ಮ ವಿರೋಧಿಗಳಾದ ಪ್ರಾಟಿಸ್‍ಟೆಂಟರಿಗಿಂತ ತಾವು ತುಂಬಾ ಮೇಲು ಎಂಬುದನ್ನು ಹೆಚ್ಚು ಪ್ರತಿಭಟನೆ ಇಲ್ಲದೆ ಸಾಧಿಸಬಹುದೆಂದು ಭಾವಿಸಿದ್ದರು. ಅಲ್ಲಿ ನೆರೆದ ಕ್ರೈಸ್ತರು, ಹಿಂದೂಗಳು, ಬೌದ್ಧರು, ಮುಸಲ್ಮಾನರು ಮತ್ತು ಇತರ ಧರ್ಮಗಳೆದುರಿಗೆ ತಮ್ಮ ಮತದ ಒಳ್ಳೆಯ ಭಾಗಗಳನ್ನು ತೋರಿ ಇತರ ಧರ್ಮದ ದೌರ್ಬಲ್ಯಗಳನ್ನು ಎತ್ತಿ ಹೇಳಿ, ತಮ್ಮ ಧರ್ಮವನ್ನು ಊರ್ಜಿತಗೊಳಿಸಿಕೊಳ್ಳಬಹುದೆಂದು ಆಶಿಸಿದ್ದರು. ಆದರೆ ಪರಿಣಾಮ ಬೇರೆ ರೂಪ ತಾಳಿತು. ಹಲವು ಧರ್ಮಗಳಲ್ಲಿ ಸೌಹಾರ್ದ ಭಾವನೆಯನ್ನು ತರುವುದಕ್ಕೆ ಕ್ರೈಸ್ತರ ಕೈಯಲ್ಲಿ ಸಾಧ್ಯವಾಗಲಿಲ್ಲ. ಆದಕಾರಣವೇ ರೋಮನ್ ಕ್ಯಾಥೊಲಿಕ್ ಜನರು ಇನ್ನೊಂದು ಅಂತಹ ಧರ್ಮ ಸಮ್ಮೇಳನವನ್ನು ನಡೆಸಲು ವಿರೋಧಿಸಿದರು. ಫ್ರಾನ್ಸ್, ರೋಮನ್ ಕ್ಯಾಥೋಲಿಕ್ ದೇಶ. ಸಮ್ಮೇಳನಕ್ಕೆ ಸಂಬಂಧಪಟ್ಟ ಅಧಿಕಾರಿ ವರ್ಗದಲ್ಲಿ ಧರ್ಮಗಳ ಸಮ್ಮೇಳನವನ್ನು ನಡೆಸಬೇಕೆಂಬ ಇಚ್ಛೆ ಇದ್ದರೂ ರೋಮನ್ ಕ್ಯಾಥೋಲಿಕ್ ಜನರು ಪ್ರತಿಭಟಿಸಿದುದರಿಂದ ಅದನ್ನು ನಡೆಸಲು\break ಆಗಲಿಲ್ಲ. 

 ಪ್ಯಾರಿಸ್ಸಿನಲ್ಲಿ ನಡೆದ ಧರ್ಮದ ಇತಿಹಾಸ ಸಮ್ಮೇಳನ ಪೌರಾತ್ಯ ವಿದ್ಯೆಯಲ್ಲಿ ಆಸಕ್ತರಾದವರು ಕಾಲಕಾಲಕ್ಕೆ ನಡೆಸುವ ಒಂದು ಸಮ್ಮೇಳನದಂತೆ ಇತ್ತು. ಅಲ್ಲಿಗೆ ಸಂಸ್ಕೃತ, ಪಾಳಿ, ಅರಬ್ಬಿ ಮತ್ತು ಇತರ ಪೌರಾತ್ಯ ಭಾಷೆಗಳನ್ನು ಬಲ್ಲ ವಿದ್ವಾಂಸರು ಮಾತ್ರ ಬಂದಿದ್ದರು. ಇವರ ಜೊತೆಗೆ ಪುರಾತನ ಕ್ರೈಸ್ತಧರ್ಮಕ್ಕೆ ಸಂಬಂಧಪಟ್ಟಿದ್ದನ್ನೂ ಪ್ಯಾರಿಸ್ ಸಮ್ಮೇಳನದಲ್ಲಿ ಸೇರಿಸಿದ್ದರು. 

 ಏಷ್ಯಾ ದೇಶದಿಂದ ಸಮ್ಮೇಳನಕ್ಕೆ ಮೂರು ಜನ ಜಪಾನಿ ಪಂಡಿತರು ಮಾತ್ರ ಬಂದಿದ್ದರು. ಇಂಡಿಯಾ ದೇಶದಿಂದ ಸ್ವಾಮಿ ವಿವೇಕಾನಂದರು ಬಂದಿದ್ದರು. 

 ಪಾಶ್ಚಾತ್ಯ ಸಂಸ್ಕೃತ ವಿದ್ವಾಂಸರ ದೃಷ್ಟಿಯಲ್ಲಿ ವೈದಿಕಧರ್ಮವು ಪ್ರಕೃತಿಯಲ್ಲಿ ಭಯಭಕ್ತಿಯನ್ನು ಹುಟ್ಟಿಸುವ ಶಕ್ತಿ ಮತ್ತು ವಸ್ತುಗಳ ಆರಾಧನೆಯಿಂದ ಪ್ರಾರಂಭವಾಯಿತು ಎಂಬುದು. 

 ಮೇಲಿನ ಅಭಿಪ್ರಾಯವನ್ನು ಖಂಡಿಸಿ ಮಾತನಾಡುವಂತೆ ಪ್ಯಾರಿಸ್ಸಿನ ಸಮ್ಮೇಳನ ಸ್ವಾಮಿ ವಿವೇಕಾನಂದರಿಗೆ ನಿಮಂತ್ರಣ ಕಳುಹಿಸಿತ್ತು. ಸ್ವಾಮಿಯವರು ಆ ‌ವಿಷಯದ ಮೇಲೆ ಒಂದು ಲೇಖನವನ್ನು ಓದುತ್ತೇನೆ ಎಂದು ಹೇಳಿದರು. ಆದರೆ ಅವರಿಗೆ ಆರೋಗ್ಯ ಸರಿಯಿಲ್ಲದ ನಿಮಿತ್ತ ಅದನ್ನು ಮಾಡಲು ಆಗಲಿಲ್ಲ. ಅವರು ಬಹಳ ಕಷ್ಟಪಟ್ಟು ಸಮ್ಮೇಳನದಲ್ಲಿ ಭಾಗವಹಿಸಲು ಬಂದರು. ಅಲ್ಲಿ ಎಲ್ಲಾ ಪಾಶ್ಚಾತ್ಯ ಸಂಸ್ಕೃತ ವಿದ್ವಾಂಸರು ಅವರನ್ನು ಆದರದಿಂದ ಬರಮಾಡಿಕೊಂಡರು. ಅವರು ಈಗಾಗಲೆ ಸ್ವಾಮೀಜಿಯವರ ವೇದಾಂತ ಉಪನ್ಯಾಸಗಳನ್ನು ಓದಿದ್ದುದರಿಂದ ಅವರಿಗೆ ಸ್ವಾಮೀಜಿಯವರ ಮೇಲಿದ್ದ ಶ್ಲಾಘನೆ ಹೆಚ್ಚಿತು. 

 ಸಮ್ಮೇಳನದಲ್ಲಿ ಗಸ್ಟೇವ್ ಓಪರ್ಸ್ ಎಂಬ ಒಬ್ಬ ಜರ್ಮನ್ ವಿದ್ವಾಂಸ “ಸಾಲಿಗ್ರಾಮ ಶಿಲೆಯ ಮೂಲ” ಎಂಬ ವಿಷಯವಾಗಿ ಒಂದು ಲೇಖನವನ್ನು ಓದಿದನು. ಸಾಲಿಗ್ರಾಮ ಪೂಜೆಯನ್ನು ಯೋನಿಪೂಜೆಯ ಚಿಹ್ನೆ ಎಂಬುದನ್ನು ತೋರಿಸಲು ಯತ್ನಿಸಿದನು. ಅವನ ದೃಷ್ಟಿಯಲ್ಲಿ ಲಿಂಗಪೂಜೆ ಪುರುಷನ ಲಿಂಗಪೂಜೆ. ಶಿವಲಿಂಗ ಪುರುಷನ ಲಿಂಗದ ಚಿಹ್ನೆ. ಸಾಲಿಗ್ರಾಮ ಸ್ತ್ರೀಯ ಯೋನಿಯ ಚಿಹ್ನೆ. ಆತ ಹೀಗೆ ಸಾಲಿಗ್ರಾಮ ಪೂಜೆ ಮತ್ತು ಲಿಂಗಪೂಜೆ ಇವನ್ನು ಯೋನಿಪೂಜೆ ಮತ್ತು ಶಿಶ್ನಪೂಜೆ ಎಂದು ಸ್ಥಿರಪಡಿಸಲು ಯತ್ನಿಸಿದನು. ಸ್ವಾಮೀಜಿಯವರು ಮೇಲಿನ ಎರಡು ದೃಷ್ಟಿಗಳನ್ನೂ ಖಂಡಿಸಿದರು. ಶಿವಲಿಂಗದ ವಿಷಯವಾಗಿ ಹಾಸ್ಯಾಸ್ಪದವಾದ ವಿವರಣೆಯನ್ನು ತಾವು ಹಿಂದೆ ಕೇಳಿದ್ದರೂ, ಸಾಲಿಗ್ರಾಮ ಶಿಲೆಯ ಸಿದ್ಧಾಂತವಾದರೊ ಅತಿ ವಿಚಿತ್ರವಾದುದು ಮತ್ತು ಅದಕ್ಕೆ ಯಾವ ಆಧಾರವೂ ಇಲ್ಲ, ಎಂದರು. 

 ಶಿವಲಿಂಗದ ಉಪಾಸನೆ ಪ್ರಾರಂಭವಾದದ್ದು, ಅಥರ್ವವೇದ ಸಂಹಿತೆಯಲ್ಲಿ\break ಯೂಪಸ್ತಂಭವನ್ನು ಸ್ತೋತ್ರಮಾಡುವ ಪ್ರಮುಖವಾದ ಮಂತ್ರದಿಂದ ಎಂದು ಸ್ವಾಮೀಜಿ ಹೇಳಿದರು. ಆ ಮಂತ್ರದಲ್ಲಿ ಯೂಪಸ್ತಂಭವನ್ನು (ಅಥವಾ ಸ್ಕಂಬ) ಅನಾದಿ ಅನಂತವೆಂದು ಸ್ತೋತ್ರಮಾಡುವರು. ಅಲ್ಲಿ ಸ್ತಂಭವನ್ನು ಸನಾತನ ಬ್ರಹ್ಮನ ಚಿಹ್ನೆಯಾಗಿ ಉಪಯೋಗಿಸುವರು. ಅನಂತರ ಯಜ್ಞದ ಬೆಂಕಿ, ಅಲ್ಲಿಂದ ಏಳುವ ಹೊಗೆ, ಜ್ವಾಲೆ, ಬೂದಿ, ಸೋಮಬಳ್ಳಿ, ಯಜ್ಞಕ್ಕೆ ಕಟ್ಟಿಗೆಯನ್ನು ಹೊರುವ ಎತ್ತು ಇವುಗಳೆಲ್ಲ ಶಿವನಿಗೆ ಸಂಬಂಧಪಟ್ಟ ಭಾವನೆಗಳು. ಶಿವನ ದೇಹದ ಕಾಂತಿ, ಅವನ ಜಟಾಜೂಟ, ಅವನ ನೀಲಕಂಠ, ವೃಷಭಾರೂಢ ಮುಂತಾದ ಭಾವನೆಗಳು ಅಲ್ಲಿಂದ ಬರುವುವು. ಇದರಂತೆಯೇ ಯೂಪಸ್ತಂಭ ಕಾಲಕ್ರಮೇಣ ಲಿಂಗದ ಆಕಾರವನ್ನು ತಾಳಿತು. ಅದನ್ನು ಶ‍್ರೀಶಂಕರ ಎಂತಲೇ ಜನ ನೋಡತೊಡಗಿದರು. ಅಥರ್ವವೇದ ಸಂಹಿತೆಯಲ್ಲಿ ಬ್ರಹ್ಮನ ಜೊತೆಯಲ್ಲಿ ಯಜ್ಞಕ್ಕೆ ಹಾಕುವ ಆಹುತಿಗಳನ್ನೂ ಹೊಗಳುವರು. 

 ಲಿಂಗಪುರಾಣದಲ್ಲಿ, ಅದೇ ಸ್ತೋತ್ರವನ್ನು ಕಥಾರೂಪದಲ್ಲಿ ವಿಸ್ತರಿಸಿರುವರು. ಸ್ತಂಬದ ಮಹಿಮೆ ಮತ್ತು ಮಹಾದೇವನು ಎಲ್ಲರಿಗಿಂತ ದೊಡ್ಡವನು ಎಂಬುದನ್ನು ತೋರುವುದಕ್ಕೆ ಅಲ್ಲಿ ಯತ್ನಿಸುವರು. 

 ನಾವು ಮತ್ತೊಂದು ವಿಷಯವನ್ನು ಇಲ್ಲಿ ಗಮನಿಸಬೇಕಾಗಿದೆ. ಬುದ್ಧನ ಸ್ಮಾರಕವಾಗಿ ಸ್ತೂಪಗಳನ್ನು ಬೌದ್ಧರು ಸ್ಥಾಪಿಸುತ್ತಿದ್ದರು. ದೊಡ್ಡ ಸ್ತೂಪವನ್ನು ಕಟ್ಟುವುದಕ್ಕೆ ಆಗದವರು ಅದರಂತೆ ಇರುವ ಒಂದು ಸಣ್ಣ ಕಟ್ಟಡವನ್ನು ಕಟ್ಟಿ ತಮ್ಮ ಭಕ್ತಿಯನ್ನು ವ್ಯಕ್ತಪಡಿಸುತ್ತಿದ್ದರು. ಹಿಂದೂಗಳ ದೇವಸ್ಥಾನದ ವಿಷಯದಲ್ಲಿ ಕೂಡ ಬೆನಾರಸ್ ಮತ್ತು ಇನ್ನೂ ಕೆಲವು ಕಡೆ ಇವುಗಳನ್ನು ನೋಡುತ್ತೇವೆ. ಯಾರಿಗೆ ದೇವಸ್ಥಾನವನ್ನು ಕಟ್ಟಲು ಸಾಧ್ಯವಿಲ್ಲವೊ ಅವರು ಒಂದು ಸಣ್ಣ ದೇವಸ್ಥಾನವನ್ನು ಹೋಲುವ ಕಟ್ಟಡವನ್ನು ಕಟ್ಟುತ್ತಾರೆ. ಇದರಂತೆಯೇ ಬೌದ್ಧರ ಕಾಲದಲ್ಲಿ ಹಿಂದೂಗಳು ದೊಡ್ಡ ಸ್ತಂಭವನ್ನು ಕಟ್ಟುತ್ತಿದ್ದರು. ಬಡವರಾದವರು ಸಣ್ಣದನ್ನು ಕಟ್ಟುತ್ತಿದ್ದರು. ಅನಂತರ ಸಣ್ಣ ಚಿಹ್ನೆಯೇ ಒಂದು ಹೊಸ ಸ್ತಂಭವಾಯಿತು. 

 ಬೌದ್ಧರ ಸ್ತೂಪಗಳ ಮತ್ತೊಂದು ಹೆಸರೇ ಧಾತುಗರ್ಭ ಎಂಬುದು. ಧಾತುಗರ್ಭದೊಳಗೆ ಸಾಲಿಗ್ರಾಮದಂತೆ ಇರುವ ಸಣ್ಣ ಪೆಟ್ಟಿಗೆ ಬರುವುದು. ಅದರೊಳಗೆ ಪ್ರಖ್ಯಾತರಾದ ಬೌದ್ಧ ಶಿಷ್ಯರ ಮೂಳೆ ಬೂದಿ ಮುಂತಾದ ಅವಶೇಷಗಳನ್ನು ಇಡುತ್ತಿದ್ದರು. ಜೊತೆಗೆ ಚಿನ್ನ ಬೆಳ್ಳಿ ಮುಂತಾದ ಲೋಹಗಳೂ ಇರುತ್ತಿದ್ದವು. ಬೌದ್ಧರ ಧಾತುಗರ್ಭದಂತೆ ಮಾಡಿದ ಕೃತಕಶಿಲೆಯನ್ನು ಸಾಲಿಗ್ರಾಮ ಹೋಲುವುದು. ಮೊದಲು ಇದನ್ನು ಬೌದ್ಧರು ಪೂಜೆಮಾಡುತ್ತಿದ್ದರು. ಅನಂತರ ಈ ರೂಢಿ, ಹಲವು ಬೌದ್ಧ ಧರ್ಮಗಳು ಆಚಾರಗಳು ಹಿಂದೂಧರ್ಮಕ್ಕೆ ಪ್ರವೇಶಿಸಿದಂತೆ ಇದೂ ಕೂಡ ವೈಷ್ಣವ ಸಂಪ್ರದಾಯದಲ್ಲಿ ಸೇರಿಕೊಂಡಿತು. ನರ್ಮದಾ ತೀರ ಮತ್ತು ನೇಪಾಳದಲ್ಲಿ ಬೌದ್ಧರ ಪ್ರಭಾವ ಬೇರೆ ಎಲ್ಲಾ ಕಡೆಗಳಿಗಿಂತಲೂ ಬಹಳ ಕಾಲವಿತ್ತು. ನರ್ಮದಾತೀರದಲ್ಲಿ ಸಿಕ್ಕಿದ ಲಿಂಗಕ್ಕೆ ನರ್ಮದೇಶ್ವರ ಶಿವಲಿಂಗ ಎಂದು ಹೆಸರು. ಇದೊಂದು ವಿಚಿತ್ರ ಸಂಘಟನೆ. ನೇಪಾಳದಲ್ಲಿ ಸಿಕ್ಕುವ ಸಾಲಿಗ್ರಾಮ ಶಿಲೆ ಇಂಡಿಯಾ ದೇಶದಲ್ಲಿ ಬೇರೆ ಕಡೆ ಸಿಕ್ಕುವ ಸಾಲಿಗ್ರಾಮ ಶಿಲೆಗಿಂತ ಪವಿತ್ರವಾದುದು ಎಂಬ ಭಾವನೆ ಬಂದು ಹೋಗಿದೆ. ನಾವು ಈ ವಿಷಯವನ್ನು ಹೆಚ್ಚು ವಿಚಾರ ಮಾಡಬೇಕಾಗಿದೆ. 

 ಸಾಲಿಗ್ರಾಮ ಶಿಲೆಯನ್ನು ಯೋನಿ ಚಿಹ್ನೆ ಎಂಬ ವಿವರಣೆ ಕಾಲ್ಪನಿಕ ಸೃಷ್ಟಿ. ಪ್ರಾರಂಭದಿಂದಲೂ ಇದು ಅದನ್ನು ಸರಿಯಾಗಿ ವಿವರಿಸಲಾರದು. ಶಿವಲಿಂಗ ಶಿಶ್ನದ ಚಿಹ್ನೆ ಎಂಬುದು ವಿಚಾರಮತಿಗಳಲ್ಲದವರು ತಂದ ವಿವರಣೆ. ಭರತಖಂಡ ಅವನತಿಯ ಆಳಕ್ಕೆ ಇಳಿದಾಗ, ಬೌದ್ಧಧರ್ಮದ ಅಧೋಗತಿಯ ಕಾಲಕ್ಕೆ ಅಂತಹ ವಿವರಣೆ ಬಂದಿತು. ಅಂದಿನ ಕಾಲದ ಅಸಹ್ಯಕರವಾದ ಬೌದ್ಧರ ತಾಂತ್ರಿಕ ಶಾಸ್ತ್ರಗಳು ನೇಪಾಳ ಮತ್ತು ಟಿಬೆಟ್ಟಿನಲ್ಲಿ ಈಗಲೂ ಕೂಡ ಸಿಕ್ಕುತ್ತವೆ ಮತ್ತು ಅಸಹ್ಯಕರವಾದ ಆಚಾರಗಳೂ ಅಲ್ಲಿ ಈಗಲೂ ಬಳಕೆಯಲ್ಲಿವೆ. 

 ಸ್ವಾಮೀಜಿಯವರು ಮತ್ತೊಂದು ಉಪನ್ಯಾಸವನ್ನು ಕೊಟ್ಟರು. ಅಲ್ಲಿ ಭರತ ಖಂಡದ ಧಾರ್ಮಿಕ ಭಾವನೆಗಳ ಐತಿಹಾಸಿಕ ವಿಕಾಸ ಎಂಬ ವಿಷಯದ ಮೇಲೆ ಮಾತನಾಡಿದರು. ಅಲ್ಲಿ ಭಿನ್ನ ಭಿನ್ನ ಹಿಂದೂ ಪಂಗಡಗಳ, ಬೌದ್ಧಧರ್ಮದ ಮತ್ತು ಇನ್ನೂ ಇತರ ಧಾರ್ಮಿಕ ಭಾವನೆಗಳಿಗೆ ವೇದಗಳೇ ಸಾಮಾನ್ಯ ಮೂಲ ಎಂದರು. ಭರತಖಂಡದ ಧಾರ್ಮಿಕ ಭಾವನೆಗಳ ಬೀಜಗಳೆಲ್ಲ ವೇದಗಳಲ್ಲಿ ಅಡಗಿವೆ. ಬೌದ್ಧಧರ್ಮ ಮತ್ತು ಇತರ ಭಾರತೀಯ ಧಾರ್ಮಿಕ ಭಾವನೆಗಳೆಲ್ಲ ಆ ಬೀಜಗಳು ವಿಕಾಸವಾಗಿ ಬೆಳೆದ ಸ್ಥಿತಿಗಳು. ಹಿಂದೂಧರ್ಮ ಚೆನ್ನಾಗಿ ಬೆಳೆದು ಪ್ರಾಪ್ತವಯಸ್ಸಿಗೆ ಬಂದ ಸ್ಥಿತಿ. ಸಮಾಜ ವಿಕಾಸವಾದಂತೆ ಅಥವಾ ಸಂಕೋಚವಾದಂತೆ ಆ ಬೀಜಗಳೂ ಕೂಡ ಸಂಕೋಚ ವಿಕಾಸವಾಗುತ್ತಾ ಬಂದವು. 

 ಶ‍್ರೀಕೃಷ್ಣ ಬುದ್ಧನಿಗಿಂತ ಮುಂಚೆ ಬಂದವನು ಎಂಬ ವಿಷಯವನ್ನು ಕುರಿತು ಅವರು ಸ್ವಲ್ಪ ಮಾತನಾಡಿದರು. ವಿಷ್ಣು ಪುರಾಣದಲ್ಲಿ ಬರುವ ರಾಜರ ಪರಂಪರೆಯ ಹೆಸರನ್ನು ಕ್ರಮೇಣ ಒಪ್ಪಿಕೊಳ್ಳುತ್ತಿರುವರು, ಮತ್ತು ಹಿಂದಿನ ಕಾಲದ ಘಟನೆಯ ಮೇಲೆ ಅದು ಬೆಳಕನ್ನು ಬೀರುತ್ತದೆ ಎಂಬ ವಿಷಯವನ್ನು ಪಾಶ್ಚಾತ್ಯ ವಿದ್ವಾಂಸರಿಗೆ ಹೇಳಿದರು. ಇದರಂತೆಯೇ ಭರತಖಂಡದಲ್ಲಿರುವ ಪರಂಪರೆಯ ವಿಷಯವೂ ಸತ್ಯ. ಪಾಶ್ಚಾತ್ಯ ವಿದ್ವಾಂಸರು ಇವುಗಳ ವಿಷಯವಾಗಿ ಮನಸ್ಸಿಗೆ ತೋಚಿದಂತೆ ಲೇಖನಗಳನ್ನು ಬರೆಯುವ ಬದಲು ಪರಂಪರೆಯಲ್ಲಿ ಸುಪ್ತವಾಗಿರುವ ಸತ್ಯಗಳನ್ನು ಕಂಡು ಹಿಡಿಯಲು ಯತ್ನಿಸಬೇಕು ಎಂದರು. 

 ಪ್ರೊಫೆಸರ್ ಮ್ಯಾಕ್ಸ್ ಮುಲ್ಲರ್ ತಮ್ಮ ಒಂದು ಪುಸ್ತಕದಲ್ಲಿ ಹೀಗೆ ಹೇಳುವರು: “ಗ್ರೀಕ್ ಮತ್ತು ಭಾರತೀಯ ವಿಷಯಗಳಲ್ಲಿ ಎಷ್ಟೇ ಸಾಮಾನ್ಯವಾಗಿರುವುದಿದ್ದರೂ ಯಾರೋ ಒಬ್ಬ ಗ್ರೀಕರಿಗೆ ಸಂಸ್ಕೃತ ಗೊತ್ತಿತ್ತು ಎಂಬುದನ್ನು ಪ್ರಮಾಣಪಡಿಸುವವರೆಗೆ ಭರತಖಂಡ ಗ್ರೀಸಿಗೆ ಸಹಾಯ ಮಾಡಿತು ಎಂಬುದನ್ನು ಒಪ್ಪಿಕೊಳ್ಳಲು ಸಾಧ್ಯವಿಲ್ಲ.” ಕೆಲವು ಪಾಶ್ಚಾತ್ಯ ವಿದ್ವಾಂಸರು ಗ್ರೀಕ್ ಖಗೋಳ ಶಾಸ್ತ್ರದಲ್ಲಿರುವ ಹಲವು ಹೆಸರುಗಳು ಭಾರತೀಯ ಖಗೋಳ ಶಾಸ್ತ್ರದಲ್ಲಿಯೂ ಇರುವುದನ್ನು ನೋಡಿ ಮತ್ತು ಗ್ರೀಕರು ಇಂಡಿಯಾದೇಶದ ಗಡಿ ಸಮೀಪದಲ್ಲಿ ತಮ್ಮ ಸಣ್ಣದೊಂದು ರಾಜ್ಯವನ್ನು ಸ್ಥಾಪಿಸಿದ್ದರು ಎಂದು ತಿಳಿದು, ಅದರ ಆಧಾರದ ಮೇಲೆಯೇ ಭರತಖಂಡದ ಇತಿಹಾಸ, ಖಗೋಳಶಾಸ್ತ್ರ, ಗಣಿತ ಇವುಗಳ ಹಿಂದೆಲ್ಲಾ ಗ್ರೀಕರ ಪ್ರಭಾವವೇ ಇದೆ ಎಂದು ಹೇಳುವರು. ಅಷ್ಟು ಮಾತ್ರವಲ್ಲ, ಅವರು ಒಂದು ಹೆಜ್ಜೆ ಮುಂದೆ ಹೋಗಿ ಭರತಖಂಡದ ವಿಜ್ಞಾನವೆಲ್ಲಾ ಗ್ರೀಕ್ ವಿಜ್ಞಾನದ ಒಂದು ಪ್ರತಿದ್ವನಿ ಎಂದು ಸಾಧಿಸುವ ಪ್ರಯತ್ನವನ್ನು ಮಾಡಿರುವರು! 

 ಒಂದು ಸಣ್ಣ ಸಂಸ್ಕೃತ ಶ್ಲೋಕದ ಉದಾಹರಣೆ:

\begin{verse}
“ಮ್ಲೇಚ್ಛಾ ವೈ ಯವನಾಃ ತೇಷು ಏಷಾ ವಿದ್ಯಾ ಪ್ರತಿಷ್ಠಿತಾ~।\\ಋಷಿವತ್ ತೇಽಪಿ ಪೂಜ್ಯನ್ತೇ…” 
\end{verse}

 “ಈ ವಿದ್ಯೆ ಮ್ಲೇಚ್ಛರಲ್ಲಿಯೂ ಯವನರಲ್ಲಿಯೂ ಬೇರೂರಿದೆ. ಅವರನ್ನು ಕೂಡ ಋಷಿಗಳಂತೆ ಗೌರವಿಸಬೇಕು.” ಪಾಶ್ಚಾತ್ಯ ವಿದ್ವಾಂಸರ ಕಲ್ಪನೆ ಎಷ್ಟು ನಿರಾತಂಕವಾಗಿ ಕೆಲಸ ಮಾಡುವುದು! ಈ ಮೇಲಿನ ಶ್ಲೋಕ ಮ್ಲೇಚ್ಛರು ಆರ‍್ಯರಿಗೆ ಕಲಿಸಿದರು ಎಂಬುದನ್ನು ಎಲ್ಲಿ ಹೇಳುವುದು? ಆರ್ಯರುಗಳ ಶಿಷ್ಯರಾದ ಮ್ಲೇಚ್ಛರನ್ನು ಹೊಗಳಲು ಮತ್ತು ಆರ್ಯರ ಜ್ಞಾನವನ್ನು ಕಲಿಯುವಂತೆ ಅವರಿಗೆ ಪ್ರೋತ್ಸಾಹವೀಯಲು ಅವರು ಹೀಗೆ ಹೇಳಿರಬಹುದು. 

 ಎರಡನೆಯದಾಗಿ ಆರ್ಯರ ವಿಜ್ಞಾನ, ಶಾಸ್ತ್ರಗಳ ಮೂಲವೆಲ್ಲ, ವೇದಗಳಲ್ಲಿ ದೊರಕುವುದರಿಂದ ಮತ್ತು ಅಲ್ಲಿಂದ ಇಲ್ಲಿಯವರೆಗೆ ಅದು ಹೆಜ್ಜೆಹೆಜ್ಜೆಗೆ ಹೇಗೆ ಮುಂದುವರಿಯಿತು ಎಂಬುದನ್ನು ನಾವು ಕಂಡುಹಿಡಿಯಲು ಸಾಧ್ಯವಾಗಿರುವುದರಿಂದ ಅವುಗಳ ಮೇಲೆ ಗ್ರೀಕರು ತಮ್ಮ ಪ್ರಭಾವವನ್ನು ಬೀರಿದ್ದರು, ಎಂಬ ಕ್ಲಿಷ್ಟವಾದ ಊಹೆಯನ್ನು ಏತಕ್ಕೆ ಅಲ್ಲಿ ತರಬೇಕು? “ಜೇನುತುಪ್ಪ ಮನೆಯಲ್ಲೆ ಸಿಕ್ಕುವಾಗ ಅದಕ್ಕಾಗಿ ಬೆಟ್ಟದ ಮೇಲಕ್ಕೆ ಏಕೆ ಹೋಗಬೇಕು” ಎಂಬ ಸಂಸ್ಕೃತಗಾದೆ ಇದೆ. 

 ಪುನಃ ಆರ್ಯರ ಖಗೋಳ ಶಾಸ್ತ್ರದಲ್ಲಿರುವ ಗ್ರೀಕ್ ಪದಗಳನ್ನು ಸಂಸ್ಕೃತದಿಂದ ಬಹಳ ಸುಲಭವಾಗಿ ಪಡೆಯಬಹುದು. ಪಾಶ್ಚಾತ್ಯ ವಿದ್ವಾಂಸರಿಗೆ ಈ ಪದಗಳಿಗೆ ಗ್ರೀಕ್ ಮೂಲ ಎಂದು ಹೇಳಲು ಯಾವ ಅಧಿಕಾರವಿದೆಯೋ ನಮಗೆ ಗೊತ್ತಿಲ್ಲ. ನೇರವಾಗಿ ಆ ಪದಗಳು ಬಂದ ಮೂಲವನ್ನೇ ಅವರು ತಾತ್ಸಾರದಿಂದ ಕಾಣುವರು. 

 ಇದರಂತೆಯೇ ಕಾಳಿದಾಸ ಮತ್ತು ಇತರ ಕವಿಗಳಲ್ಲಿ ‘ಯವನಿಕಾ’ ಎಂಬ ಪದ ಬಂದಿದ್ದರೆ ಇಡೀ ಭಾರತೀಯ ನಾಡಿನ ಸಾಹಿತ್ಯವೆಲ್ಲಾ ಗ್ರೀಕರ ನಾಟಕವನ್ನು ಹೋಲುತ್ತದೆಯೇ ಎಂಬುದನ್ನು ಕುರಿತು ಮೊದಲು ವಿಚಾರಿಸಬೇಕು. ಯಾರು ಎರಡು ಭಾಷೆಗಳ ನಾಟಕ ಮತ್ತು ಅಭಿನಯದ ರೀತಿಗಳನ್ನು ಅಧ್ಯಯನ ಮಾಡಿರುವರೋ ಅವರಿಗೆ ಏನಾದರೂ ಪರಸ್ಪರ ಸಾಮ್ಯಗಳು ಕಂಡರೆ ಅದು ಕೇವಲ ಹಾಗೆ ಭಾವಿಸುವ ಮೊಂಡನಾದವನ ಕನಸಿನಲ್ಲಿದೆಯೇ ವಿನಃ ಬೇರೆ ಇನ್ನು ಎಲ್ಲಿಯೂ ನಿಜವಾಗಿಯೂ ಇಲ್ಲ ಎಂಬುದನ್ನು ತಿಳಿಯಬೇಕು. ಗ್ರೀಕರ ಮೇಳಗೀತೆಗಳೆಲ್ಲಿ? ಗ್ರೀಸಿನ ಯವನಿಕಾ ರಂಗಭೂಮಿಯ ಒಂದು ಕಡೆ ಇದೆ. ಆರ್ಯರದು ಅದಕ್ಕೆ ವಿರುದ್ಧವಾಗಿದೆ. ಗ್ರೀಕ್ ನಾಟಕಗಳನ್ನು ವ್ಯಕ್ತಪಡಿಸುವ ರೀತಿಯೇ ಒಂದು, ಆರ್ಯರ ರೀತಿಯೇ ಒಂದು. ಆರ್ಯ ಮತ್ತು ಗ್ರೀಕರ ನಾಟಕಗಳಲ್ಲಿ ಸ್ವಲ್ಪವೂ ಸಾಮ್ಯ ಇಲ್ಲ. ಅದರ ಬದಲು ಶೇಕ್ಸ್​ಪಿಯರ್ ನಾಟಕಗಳು ಇಂಡಿಯಾದ ನಾಟಕಗಳನ್ನು ಬಹುಮಟ್ಟಿಗೆ ಹೋಲುತ್ತವೆ. ಇದರಿಂದ ಶೇಕ್ಸ್ ಪಿಯರ್ ಕಾಳಿದಾಸ ಮತ್ತು ಇತರ ಪುರಾತನ ಭಾರತೀಯ ನಾಟಕಕರ್ತೃಗಳಿಗೆ ಋಣಿಯಾಗಿದ್ದನು, ಪಾಶ್ಚಾತ್ಯ ಸಾಹಿತ್ಯವೆಲ್ಲ ಇಂಡಿಯಾ ಸಾಹಿತ್ಯದ ಒಂದು ಅನುಕರಣೆ ಎಂಬ ನಿರ್ಧಾರಕ್ಕೂ ಬರಬಹುದು. 

 ಕೊನೆಯದಾಗಿ ಪಾಶ್ಚಾತ್ಯ ವಿದ್ವಾಂಸರ ಊಹೆಯನ್ನು ಅವರ ಮೇಲೆಯೇ ತಿರುಗಿಸಬಹುದು. ಯಾರಾದರೂ ಒಬ್ಬ ಹಿಂದೂಗಳಿಗೆ ಗ್ರೀಕ್ ಎಂದಾದರೂ‌ ತಿಳಿದು ಇತ್ತು ಎಂಬುದನ್ನು ಸಮರ್ಥಿಸುವವರೆಗೆ ಗ್ರೀಕರ ಪ್ರಭಾವವನ್ನು ಕುರಿತು ನಾವು ಮಾತನಾಡಕೂಡದು. ಇದರಂತೆಯೇ ಗ್ರೀಕರ ಪ್ರಭಾವವನ್ನು ಭಾರತೀಯ ಶಿಲ್ಪಕಲೆಯ ಮೇಲೆ ಆರೋಪಿಸುವುದೂ ಕೂಡ ನಿರಾಧಾರವಾದುದು. 

 ಸ್ವಾಮೀಜಿಯವರು ಶ‍್ರೀಕೃಷ್ಣನ ಪೂಜೆ ಬುದ್ಧನ ಪೂಜೆಗಿಂತ ಪುರಾತನವಾದುದು ಎಂದು ಹೇಳಿದರು. ಗೀತೆ ಮಹಾಭಾರತದ ಕಾಲಕ್ಕೆ ಸೇರದೆ ಇದ್ದರೆ, ಅದು\break ಮಹಾಭಾರತಕ್ಕಿಂತ ಮುಂಚಿನದಾಗಿರಬೇಕೆ ವಿನಃ ಅನಂತರ ಬಂದುದಲ್ಲ. ಗೀತೆಯ ಶೈಲಿ ಮಹಾಭಾರತದ ಶೈಲಿಯಂತೆಯೇ ಇದೆ. ಗೀತೆಯಲ್ಲಿ ಆಧ್ಯಾತ್ಮಿಕ ವಿಷಯಗಳನ್ನು\break ವಿವರಿಸಲು ಉಪಯೋಗಿಸಿರುವ ಗುಣವಾಚಕಗಳೆಲ್ಲ ಮಹಾಭಾರತದ ವನಪರ್ವ ಮತ್ತು ಇತರ ಪರ್ವಗಳಲ್ಲಿ ಉಪಯೋಗಿಸಲ್ಪಟ್ಟಿವೆ. ಹೀಗೆ ಪದಗಳು ಎರಡುಕಡೆಯೂ ಬಂದಿರಬೇಕಾದರೆ ಅದು ಒಂದೇ ಕಾಲದಲ್ಲಿ ಬರೆದ ಹೊರತು ಸಾಧ್ಯವಿಲ್ಲ. ಗೀತೆಯಲ್ಲಿ ಬರುವ ವಿಚಾರ ಸರಣಿ ಮಹಾಭಾರತದಲ್ಲಿ ಬರುವಂತೆಯೇ ಇದೆ. ಗೀತೆ ಆಗಿನ ಕಾಲದ ಧಾರ್ಮಿಕ ಪಂಗಡಗಳನ್ನು ಹೇಳುವಾಗ ಬೌದ್ಧಧರ್ಮದ ಮಾತನ್ನೇ ಏತಕ್ಕೆ ಎತ್ತುವುದಿಲ್ಲ? 

 ಬುದ್ಧನ ಅನಂತರ ಬಂದ ಬರಹಗಾರರು ಎಷ್ಟೇ ಪ್ರಯತ್ನಪಡಲಿ, ಬೌದ್ಧಧರ್ಮ ವಿಷಯವನ್ನು ಮರೆಮಾಡಿಲ್ಲ. ಅದು ಎಲ್ಲಿಯಾದರೂ ಒಂದು ಕಡೆ ಬರುವುದು. ಪ್ರತ್ಯಕ್ಷವಾಗಿಯೇ ಬುದ್ಧ ಮತ್ತು ಬೌದ್ಧಧರ್ಮದ ವಿಷಯ ಬಂದೇ ಬರುವುದು. ಗೀತೆಯಲ್ಲಿ ಅಂತಹ ಸೂಚನೆಯನ್ನು ಯಾರಾದರೂ ತೋರಬಲ್ಲರೆ? ಗೀತೆ ಎಲ್ಲಾ ವಿಧವಾದ ಧಾರ್ಮಿಕ ಭಾವನೆಗಳಿಗೂ ಸೌಹಾರ್ದವನ್ನು ಕಲ್ಪಿಸುತ್ತದೆ. ಅದರಲ್ಲಿ ಯಾವುದನ್ನೂ ದೂರುವುದಿಲ್ಲ. ಗೀತಾಕರ್ತೃವು ಬೌದ್ಧಧರ್ಮವನ್ನು ಮಾತ್ರ ಏತಕ್ಕೆ ಬಿಟ್ಟಿರುವನು ಎಂಬುದನ್ನು ನಾವು ವಿವರಿಸಬೇಕಾಗಿದೆ.

 ಗೀತೆ ಯಾರನ್ನೂ ಬೇಕೆಂದು ನಿಂದಿಸುವುದಿಲ್ಲ. ಅಂಜಿಕೆಯಿಂದ ಹೀಗೆ ಮಾಡಿದೆಯೆ? ಅದು ಅಲ್ಲಿ ಪ್ರಮುಖವಾಗಿ ಇಲ್ಲವೇ ಇಲ್ಲ. ಸ್ವಯಂ ಭಗವಂತನೇ ವೇದಗಳನ್ನು\break ಸೃಷ್ಟಿಸಿದವನು, ಅದನ್ನು ವಿವರಿಸಿದವನು. ಅವನೇ ವೈದಿಕ ಭಾವನೆಗಳು ಮಿತಿಮೀರಿ ಅಧಿಕ ಪ್ರಸಂಗಕ್ಕೆ ಕೈಹಾಕಿದರೆ ಅವುಗಳನ್ನು ಕೂಡ ನಿಗ್ರಹಿಸದೆ ಇಲ್ಲ. ಇಂಥವನು ಬೌದ್ಧರಿಗೆ ಏತಕ್ಕೆ ಹೆದರಬೇಕಾಗಿತ್ತು? 

 ಹೇಗೆ ಪಾಶ್ಚಾತ್ಯ ವಿದ್ವಾಂಸರು ಯಾವುದಾದರೂ ಒಂದು ಕೃತಿಗೆ ತಮ್ಮ ಇಡೀ ಜೀವನವನ್ನೇ ಅರ್ಪಣ ಮಾಡುವರೋ ಹಾಗೆಯೇ ಯಾವುದಾದರೂ ಒಂದು ಸಂಸ್ಕೃತ ಕೃತಿಗೆ ಅರ್ಪಣೆ ಮಾಡಿದರೆ ಅದರಿಂದ ಪ್ರಪಂಚಕ್ಕೆ ಎಷ್ಟೋ ಪ್ರಯೋಜನವಾಗುವುದು. ಭಾರತೀಯರ ಇತಿಹಾಸದಲ್ಲಿ ಮಹಾಭಾರತ ಅತ್ಯಂತ ಮುಖ್ಯವಾದ ಗ್ರಂಥ. ಪಾಶ್ಚಾತ್ಯರು ಚೆನ್ನಾಗಿ ಇದನ್ನು ಇನ್ನೂ ಓದಿಲ್ಲ ಎಂದರೆ ಅದು ಅತಿಶಯವಲ್ಲ. 

 ಉಪನ್ಯಾಸವಾದ ಮೇಲೆ ಅದಕ್ಕೆ ಪರವಾಗಿ ಮತ್ತು ವಿರೋಧವಾಗಿ ನೆರೆದವರು ತಮ್ಮ ಅಭಿಪ್ರಾಯವನ್ನು ವ್ಯಕ್ತಪಡಿಸಿದರು. ಅಧ್ಯಕ್ಷರು ಸ್ವಾಮೀಜಿಯವರು ಹೇಳಿದ ಬಹುಭಾಗವನ್ನು ತಾವು ಒಪ್ಪಿಕೊಳ್ಳುತ್ತೇವೆ ಎಂದು ಹೇಳಿದರು. ಹಿಂದಿನ ಕಾಲದ ಪುರಾತನ ವಸ್ತುಗಳನ್ನು ವಿಮರ್ಶಿಸುವ ರೀತಿ ಈಗ ಇಲ್ಲ ಎಂದು ಅವರು ಸ್ವಾಮೀಜಿಗಳಿಗೆ ಹೇಳಿದರು. ಆಧುನಿಕ ಸಂಸ್ಕೃತ ವಿದ್ವಾಂಸರ ಅಭಿಪ್ರಾಯಗಳು ಬಹುಮಟ್ಟಿಗೆ ಸ್ವಾಮೀಜಿಯವರ ಅಭಿಪ್ರಾಯಗಳನ್ನೇ ಹೋಲುತ್ತವೆ. ಪುರಾಣ ಮತ್ತು ಸಾಂಪ್ರದಾಯಿಕವಾಗಿ ಬಂದ ಅಭಿಪ್ರಾಯಗಳಲ್ಲಿ ನಿಜವಾದ ಚಾರಿತ್ರಿಕ ಅಂಶಗಳು ಬಹುವಾಗಿವೆ ಎಂಬುದನ್ನು ಅವರು ಒಪ್ಪಿಕೊಂಡರು. 

 ಕೊನೆಯದಾಗಿ ವಿದ್ವಾಂಸರಾದ ಅಧ್ಯಕ್ಷರು ಸ್ವಾಮೀಜಿಯವರ ಉಪನ್ಯಾಸಗಳಲ್ಲಿ ಒಂದು ವಿಷಯ ವಿನಃ ಉಳಿದ ಎಲ್ಲಾ ವಿಷಯಗಳನ್ನೂ ಒಪ್ಪಿಕೊಂಡರು. ಅವರು ಒಪ್ಪದೇ ಇದ್ದುದೇ ಗೀತೆ ಮತ್ತು ಮಹಾಭಾರತ ಒಂದೇ ಕಾಲದಲ್ಲಿತ್ತು ಎಂಬುದು. ಅವರು ಕೊಟ್ಟ ಕಾರಣ ಪಾಶ್ಚಾತ್ಯ ವಿದ್ವಾಂಸರು ಇದನ್ನು ಒಪ್ಪಿಕೊಳ್ಳುವುದಿಲ್ಲ ಎಂಬುದಲ್ಲದೆ ಬೇರೆ ಇಲ್ಲ. 

