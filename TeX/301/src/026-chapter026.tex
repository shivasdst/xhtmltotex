
\chapter{ಸಹಸ್ರ ದೀಪೋದ್ಯಾನದಲ್ಲಿ}

 ಸ್ವಾಮೀಜಿ ಉಪನ್ಯಾಸದ ಬ್ಯೂರೋ ಮೂಲಕ ಒಂದು ವರುಷ ಅಮೇರಿಕಾ ದೇಶದ ಪ್ರಮುಖ ನಗರಗಳಲ್ಲೆಲ್ಲ ಉಪನ್ಯಾಸವನ್ನು ಮಾಡಿದರು. ಸಣ್ಣ ಸಣ್ಣ ಕೂಟಗಳಲ್ಲಿಯೂ ಮಾತನಾಡಿದರು. ಡೆಟ್ರಾಯಿಟ್‍ನಲ್ಲಿ ಗೌರ್ನರ್ ಪತ್ನಿಯಾದ ಶ‍್ರೀಮತಿ ಜಾನ್ ಜೀ ಬ್ಯಾಗ್ಲಿ ಎಂಬುವರ ಮನೆಯಲ್ಲಿ ನಾಲ್ಕು ವಾರಗಳು ಇದ್ದರು. ಆಕೆ ವಿದ್ಯಾವಂತಳಾದ ಸುಸಂಸ್ಕೃತಳಾದ ಆಧ್ಯಾತ್ಮಿಕ ಜೀವಿ. ಸ್ವಾಮೀಜಿ ತಮ್ಮ ಮನೆಯಲ್ಲಿರುವವರೆಗೆ ಮಾತು ಮತ್ತು ವ್ಯವಹಾರದಲ್ಲಿ ಶ್ರೇಷ್ಠತಮವಾದುದನ್ನು ವ್ಯಕ್ತಪಡಿಸುತ್ತಿದ್ದರು; ಅವರ ಸಾನ್ನಿಧ್ಯವೇ ಅನವರತ ಆಶೀರ್ವಾದದಂತೆ ಇತ್ತು ಎಂದು ಆಕೆ ಅನಂತರ ಹೇಳುತ್ತಿದ್ದಳು. ಬ್ಯಾಗ್ಲಿ ಅವರ ಮನೆಯನ್ನು ಬಿಟ್ಟು ಆದಮೇಲೆ \enginline{Worlds Fair Commission}ನ ಪ್ರೆಸಿಡೆಂಟರಾದ ಶ‍್ರೀ ಪಾಮರ್ ಎಂಬುವರ ಮನೆಯಲ್ಲಿದ್ದರು. ಬೇರೆ ಕಡೆ ಮಾತನಾಡುವುದಕ್ಕೆ ಹೋಗದೇ ಇದ್ದಾಗ ಚಿಕಾಗೋ ನಗರದ ಶ‍್ರೀ ಹೇಲ್ಸ್ ಮನೆಯಲ್ಲಿ ಇರುತ್ತಿದ್ದರು. ಡೆಟ್ರಾಯಿಟ್‍ನಲ್ಲಿ ಫೆಬ್ರವರಿ ತಿಂಗಳಿನಲ್ಲಿ ಯೂನಿಟೇರಿಯನ್ ಚರ್ಚಿನ ಆಶ್ರಯದಲ್ಲಿ ಕೆಲವು ಉಪನ್ಯಾಸಗಳನ್ನು ಕೊಟ್ಟಮೇಲೆ ಸ್ವಾಮೀಜಿ ೧೮೯೪ನೇ ಮಾರ್ಚ್ ಏಪ್ರಿಲ್ ಮೇ ತಿಂಗಳನ್ನು ಚಿಕಾಗೊ ನಗರದಲ್ಲಿ ಕಳೆದರು. ಬೇಸಿಗೆಯ ಮಧ್ಯಭಾಗದಲ್ಲಿ ನ್ಯೂ ಇಂಗ್ಲೆಂಡಿನಲ್ಲಿ ಗ್ರೀನ್‍ಏಕರ್ ಎಂಬ ಸ್ಥಳದಲ್ಲಿ ಒಂದು ಸಮ್ಮೇಳನವನ್ನು ಅಣಿ ಮಾಡಿದ್ದರು. ಸ್ವಾಮೀಜಿ ಅಲ್ಲಿ ಕೆಲವು ಉಪನ್ಯಾಸಗಳನ್ನು ಕೊಟ್ಟರು. ಅಲ್ಲಿ ಸ್ವಾಮೀಜಿ ಒಂದು ವಿಶಾಲವಾದ ಪೈನ್‍ಮರದ ಕೆಳಗೆ ಕುಳಿತುಕೊಂಡು ವಿದ್ಯಾರ್ಥಿಗಳಿಗೆ ವೇದಾಂತಬೋಧನೆ ಮಾಡಿದರು. ಆ ಮರವನ್ನು ಅಂದಿನಿಂದ ಸ್ವಾಮಿಗಳ ಪೈನ್‍ಮರ ಎನ್ನುತ್ತಾರೆ. ಇಲ್ಲಿಯ ಉಪನ್ಯಾಸಗಳನ್ನು ಡಾಕ್ಟರ್ ಲೂಯಿಸ್ ಜಿ. ಜೇನ್ಸ್ ಎಂಬುವರು ವ್ಯವಸ್ಥೆಗೊಳಿಸಿದರು. ಸ್ವಾಮೀಜಿ ಗ್ರೀನ್‍ಏಕರ್ ಬಿಟ್ಟಮೇಲೆ ಬಾಸ್ಟನ್ ಚಿಕಾಗೊ ನ್ಯೂಯಾರ್ಕ್‍ಗಳಿಗೆ ಹೋಗಿದ್ದರು. ಅಕ್ಟೋಬರ್ ಕೊನೆಯ ಹೊತ್ತಿಗೆ ಬಾಲ್ಟಿಮೋರ್ ಮತ್ತು ವಾಷಿಂಗ್‍ಟನ್‍ಗಳನ್ನು ಸಂದರ್ಶಿಸಿದರು. ನವೆಂಬರ್‍ನಲ್ಲಿ ಪುನಃ ನ್ಯೂಯಾರ್ಕ್‍ಗೆ ಹೋದರು. ಇಲ್ಲಿ ಕೆಲವು ವೇಳೆ ತಾತ್ಕಾಲಿಕವಾಗಿದ್ದು ಕೆಲವು ಉಪನ್ಯಾಸಗಳನ್ನು ಮಾತ್ರ ಕೊಟ್ಟಿದ್ದರು. ಸರಿಯಾಗಿ ಯಾವುದೂ ವ್ಯವಸ್ಥೆಗೊಂಡಿರಲಿಲ್ಲ. ಬ್ರುಕ್‍ಲಿನ್ ಎಥಿಕಲ್ ಅಸೋಷಿಯೇಶನ್ನಿನ ಆಶ್ರಯದಲ್ಲಿ ಹಿಂದೂ ಧರ್ಮದ ಮೇಲೆ ಕೆಲವು ಉಪನ್ಯಾಸಗಳನ್ನು ಕೊಡಲು ಡಾಕ್ಟರ್ ಲೂಯಿಸ್ ಜಿ. ಜೇನ್ಸ್ ಅವರು ಏರ್ಪಾಡು ಮಾಡಿದರು. ಸ್ವಾಮೀಜಿ ಈ ನಿಮಂತ್ರಣವನ್ನು ಒಪ್ಪಿಕೊಂಡರು. ಅನಂತರ ಜೀವಾವಧಿ ಅವರಿಬ್ಬರೂ ಸ್ನೇಹಿತರಾಗಿ ಉಳಿದುಕೊಂಡಿದ್ದರು. 

 ಸ್ವಾಮೀಜಿ ಇಷ್ಟು ಹೊತ್ತಿಗೆ ಉಪನ್ಯಾಸದ ಬ್ಯೂರೊವನ್ನು ಬಿಟ್ಟುಬಿಟ್ಟಿದ್ದರು. ಅವರ ಉಪನ್ಯಾಸವನ್ನು ಕೇಳಲು ಬರುವ ಮಂದಿಗಳನ್ನು ನೋಡಿ ಅವರಿಗೆ ಬೇಜಾರಾಯಿತು. ಸುಮ್ಮನೆ ಕುತೂಹಲದಿಂದ ಪ್ರೇರಿತರಾಗಿ ಬರುತ್ತಿದ್ದವರೆ ಹೆಚ್ಚು. ಆದಕಾರಣ ಬಹಿರಂಗ ಉಪನ್ಯಾಸವನ್ನು ನಿಲ್ಲಿಸಿ, ಆರಿಸಿದ ಕೆಲವು ಶಿಷ್ಯರಿಗೆ ವೇದಾಂತ ಬೋಧನೆಯನ್ನು ಮಾಡಬೇಕೆಂದು ಬಯಸಿದರು. ಅದಕ್ಕಾಗಿ ತಾವು ಇನ್ನೊಬ್ಬರ ಅತಿಥಿಗಳಾಗಿ ಇರುವುದಕ್ಕಿಂತ ಬೇರೆ ತಮ್ಮದೇ ಸ್ವಂತ ಮನೆಯಲ್ಲಿದ್ದರೆ ಅನುಕೂಲ ಎಂದು ಭಾವಿಸಿದರು. ಇದಕ್ಕಾಗಿ ಒಂದು ಮನೆಯನ್ನು ಬಾಡಿಗೆಗೆ ಮಾಡಿದರು. ಪ್ರವಚನಗಳನ್ನು ಉಚಿತವಾಗಿ ಕೊಡತೊಡಗಿದರು. ಭಕ್ತಾದಿಗಳು ಕೊಡುವ ಹಣ ಬಾಡಿಗೆಗೆ ಮತ್ತು ಜೀವನಕ್ಕೆ ಸಾಲದೆ ಇದ್ದರೆ, ಹೊರಗೆ ಸ್ವಾಮೀಜಿ ಹಣಕ್ಕಾಗಿ ಕೆಲವು ಉಪನ್ಯಾಸಗಳನ್ನು ಮಾಡಿ, ಅದರಿಂದ ಬಂದ ಹಣದಿಂದ ಬಾಡಿಗೆಯನ್ನು ಕೊಡುತ್ತಿದ್ದರು. ಒಂದು ಸಣ್ಣ ಕೋಣೆಯಲ್ಲಿ ಪ್ರವಚನವನ್ನು ಪ್ರಾರಂಭ ಮಾಡಿದರು. ಆ ಸಣ್ಣ ಕೋಣೆಯಲ್ಲಿ ಬರುವ ಜನರು ಹಿಡಿಸುತ್ತಿರಲಿಲ್ಲ. ನೆಲದ ಮೇಲೆ ಸ್ಥಳವಿಲ್ಲದೇ ಇದ್ದರೆ ಕಿಟಕಿಯ ಬಾಗಿಲು, ಮೆಟ್ಟಲು ಎಲ್ಲೆಲ್ಲಿ ಸ್ಥಳವಿತ್ತೊ ಅಲ್ಲೆಲ್ಲ ಕುಳಿತುಕೊಂಡು ಕೇಳುತ್ತಿದ್ದರು. ಆ ಪ್ರವಚನಗಳು ಬಹಳ ಶ್ರೇಷ್ಠ ವರ್ಗಕ್ಕೆ ಸೇರಿದವು. ಏಕೆಂದರೆ ಅಲ್ಲಿಗೆ ಬರುವವರು ಕೇವಲ ವಿಷಯಗಳನ್ನು ತಿಳಿದುಕೊಳ್ಳಬೇಕೆಂಬ ಆಸಕ್ತಿಯಿಂದ ಮಾತ್ರ ಪ್ರೇರೇಪಿತರಾಗಿದ್ದರು. ಅದನ್ನು ಕೇಳಿದವರು ಎಂದಿಗೂ ಮರೆಯುವಂತೆ ಇರಲಿಲ್ಲ. 

 ಬರಬರುತ್ತ ಜನ ಹೆಚ್ಚಾದಾಗ ಸ್ವಲ್ಪ ದೊಡ್ಡ ಪ್ರಾಂಗಣವನ್ನೇ ಬಾಡಿಗೆಗೆ ತೆಗೆದುಕೊಳ್ಳಬೇಕಾಯಿತು. ಇಲ್ಲಿ ಪ್ರವಚನಗಳು ೧೮೯೫ನೇ ಜೂನ್‍ವರೆಗೆ ನಡೆದವು. ಪ್ರವಚನಾದಿಗಳನ್ನು ಕೊಡುವುದರ ಜೊತೆಗೆ ವಿದ್ಯಾರ್ಥಿಗಳನ್ನು ಧ್ಯಾನದಲ್ಲಿಯೂ ತರಬೇತು ಮಾಡಿದರು. ವಿದ್ಯಾರ್ಥಿಗಳಿಗೆ ಧ್ಯಾನವನ್ನು ಹೇಳಿಕೊಡುತ್ತಿದ್ದಾಗ ಅವರೇ ಧ್ಯಾನಮಗ್ನರಾಗಿ ಪ್ರಪಂಚವನ್ನೇ ಮರೆತುಬಿಡುತ್ತಿದ್ದರು. ಆ ಸಮಯಗಳಲ್ಲಿ ಸ್ವಾಮೀಜಿ ತಾವು ಹಾಗೆ ಧ್ಯಾನದಲ್ಲಿ ತನ್ಮಯರಾದರೆ ಅವರನ್ನು ಪ್ರಕೃತಿಸ್ಥರಾಗುವಂತೆ ಮಾಡುವುದು ಹೇಗೆ ಎಂಬುದನ್ನು ಕೆಲವು ಶಿಷ್ಯರಿಗೆ ಹೇಳಿಕೊಟ್ಟರು. ಈ ಕಾಲದಲ್ಲೇ ಸ್ವಾಮೀಜಿ ರಾಜಯೊಗವನ್ನು ಬರೆದರು. ಶಿಷ್ಯರಿಗೆ ಬೋಧಿಸಿದುದರ ಜೊತೆಗೆ ಅದನ್ನು ಪುಸ್ತಕ ರೂಪದಲ್ಲಿಯೂ ತಂದರು. 

 ಈ ಸಮಯದಲ್ಲಿ ಸ್ವಾಮೀಜಿಯವರಿಗೆ ಅವರ ಅತ್ಯಂತ ನಿಕಟ ಸ್ನೇಹಿತರ ಪರಿಚಯವಾಯಿತು. ಪ್ರಸಿದ್ಧ ಪಿಟೀಲುವಾದಕನ ಪತ್ನಿಯಾದ ಶ‍್ರೀಮತಿ ಓಲ್‍ಬುಲ್, ಡಾಕ್ಟರ್ ಅಲನ್ ಡೆಮಿಸ್, ಎಸ್. ಇ. ವಾಲ್ಡೊ, ಪ್ರೊಫೆಸರ್ ವೈಮೇನ್ ಮತ್ತು ರೈಟ್, ಡಾಕ್ಟರ್ ಸ್ಟ್ರೀಟ್ ಮತ್ತು ಚರ್ಚಿನ ಕೆಲವು ಪಾದ್ರಿಗಳೂ ಚಿರಸ್ನೇಹಿತರಾದರು. 

 ಫ್ರಾನ್‍ಸಿಸ್ ಲೆಗೆಟ್, ಅವರ ಪತ್ನಿ ಮತ್ತು ಸೋದರಿಯಾದ ಮಾಕ್ಲಿಯಾಡ್ ಸ್ವಾಮೀಜಿ ಅವರ ಪರಮಾಪ್ತ ಮಿತ್ರರಾದರು. ಸ್ವಾಮೀಜಿಯವರಿಗೆ ಹಲವು ವಿಧದಲ್ಲಿ ಅವರು ಸಹಾಯ ಮಾಡುತ್ತಿದ್ದರು. ಡಿಕನ್ಸ್ ಸೊಸೈಟಿಯವರು ಸ್ವಾಮೀಜಿಯವರನ್ನು ತಮ್ಮ ಸೊಸೈಟಿ ಆಶ್ರಯದಲ್ಲಿ ಮಾತನಾಡಲು ಕರೆದರು. ಈ ಸಮಯದಲ್ಲೆ ಪ್ರಖ್ಯಾತ ವಿದ್ಯುತ್ ಶಾಸ್ತ್ರಜ್ಞನಾದ ನಿಕಾಲಸ್ ಟೆಲ್‍ಸಾ ಎಂಬುವನು ಸ್ವಾಮೀಜಿಯವರು ಕೊಟ್ಟ ಸಾಂಖ್ಯತತ್ತ್ವದ ಉಪನ್ಯಾಸವನ್ನು ಬಹಳ ಮೆಚ್ಚಿದನು. 

 ೧೮೯೫ನೇ ಜೂನ್ ಹೊತ್ತಿಗೆ ಸ್ವಾಮೀಜಿ ತಮ್ಮ ಕಾರ‍್ಯಕ್ಕೆ ಚೆನ್ನಾಗಿ ತಳಹದಿಯನ್ನು ಹಾಕಿದರು. ಪ್ರವಚನಾದಿಗಳ ಸುಂಟರ ಗಾಳಿಯಲ್ಲಿದ್ದಾಗ ಅನೇಕ ವೇಳೆ ತಮಗೇ ಅದು ಬೇಜಾರು ಆಗಿಹೋಗುತ್ತಿತ್ತು. “ಉಡಲು ಚಿಂದಿಯ ಬಟ್ಟೆ; ಬೋಳು ತಲೆ, ಊಟಕ್ಕೆ ಭಿಕ್ಷಾಪಾತ್ರೆ, ಮಲಗಲು ಮರದ ನೆರಳು - ಇದಕ್ಕಾಗಿ ನಾನು ಹಾತೊರೆಯುತ್ತಿರುವೆ” ಎಂದು ಬರೆಯುವರು. 

 ನ್ಯೂಯಾರ್ಕಿನ ಪ್ರವಚನಗಳನ್ನು ಪೂರೈಸಿದ ಮೇಲೆ ನ್ಯೂಹ್ಯಾಮ್ ಶೈರಿನಲ್ಲಿ ಪ್ರಶಾಂತವನದಲ್ಲಿದ್ದ ತಮ್ಮ ಭಕ್ತರೊಬ್ಬರ ಮನೆಗೆ ವಿಶ್ರಾಂತಿಗೆ ಹೋದರು. ಸೇಂಟ್‍ಲಾರೆನ್ಸ್ ನದಿಯ ಮಧ್ಯದಲ್ಲಿ ಸಹಸ್ರಾರು ದ್ವೀಪಗಳಿವೆ. ಅದರಲ್ಲಿ ಒಂದು ದ್ವೀಪದಲ್ಲಿದ್ದ ಮನೆಯನ್ನು ಸ್ವಾಮೀಜಿ ಇಳಿದುಕೊಳ್ಳುವುದಕ್ಕೆ ಮತ್ತು ಪ್ರವಚನಾದಿಗಳನ್ನು ನಡೆಸುವುದಕ್ಕೆ ಅವರಿಗೆ ಮಿಸ್ ಡೆಚರ್ ಎಂಬಾಕೆ ಉಚಿತವಾಗಿ ಕೊಟ್ಟಳು. 

 ಶ‍್ರೀಮತಿ ಡೆಚರ್ ಸ್ವಾಮೀಜಿಯವರಿಗೆ ಇಳಿದುಕೊಳ್ಳುವುದಕ್ಕಾಗಿ ಒಂದು ಭಾಗವನ್ನು ಬೇರೆಯಾಗಿಯೇ ಕಟ್ಟಿದರು. ಅದು ಮೇಲೆ ಇತ್ತು. ಅದರ ಕೆಳಗೆ ಒಂದು ದೊಡ್ಡ ಪ್ರಾಂಗಣ. ಸ್ವಾಮೀಜಿ ಅಲ್ಲೆ ಪ್ರವಚನಾದಿಗಳನ್ನು ಅಲ್ಲಿರುವ ತನಕ ನಡೆಸುತ್ತ್ತಿದ್ದರು. ಅದರ ಕೆಳಗೆ ಪ್ರವಚನಗಳನ್ನು ಕೇಳುವುದಕ್ಕೆ ಬಂದಿದ್ದವರು ವಾಸವಾಗಿದ್ದರು. ಇಡೀ ಮನೆ ಒಂದು ಪರ್ವತದ ಹಿನ್ನೆಲೆಯಲ್ಲಿ ಇತ್ತು. ಸುತ್ತಲೂ ಅರಣ್ಯದಿಂದ ಪರಿವೃತವಾಗಿತ್ತು. ಅಲ್ಲಿಂದ ಸೇಂಟ್ ಲಾರೆನ್ಸ್ ನದಿ ಹರಿದು ಹೋಗುತ್ತಿರುವುದು, ಮತ್ತು ಅದರಲ್ಲಿದ ಹಲವು ದ್ವೀಪಗಳು ಕಾಣುತ್ತಿದ್ದವು. ಮರದ ಮರ್ಮರ ನಿನಾದ, ಹಕ್ಕಿಗಳ ಇಂಚರ ಮತ್ತು ಮಂದಗಮನದಿಂದ ಮುಂದೆ ಸಾಗುವ ಸೇಂಟ್ ಲಾರೆನ್ಸ್ ನದಿಯ ಮೊರೆ - ಇವು ಆ ಸಹಸ್ರ ದ್ವೀಪೋದ್ಯಾನದ ಮನೆಗೆ ಹಿನ್ನೆಲೆಯಾಗಿದ್ದುವು. ಬಹುಶಃ ಉಪನಿಷತ್ತಿನ ಕಾಲದಲ್ಲಿ ಗುರು-ಶಿಷ್ಯರು ಇಂತಹ ವಾತಾವರಣದಲ್ಲಿ ವಿದ್ಯೆಯನ್ನು ಕಲಿಯುತ್ತಿದ್ದರು ಎಂದು ಕಾಣುವುದು. ಏಳು ವಾರಗಳು ವಿದ್ಯಾರ್ಥಿಗಳು ಇಲ್ಲಿ ತಮ್ಮ ಗೌರವಕ್ಕೆ ಪಾತ್ರರಾದ ಗುರುಗಳ ಸಾನ್ನಿಧ್ಯದಲ್ಲಿ ಕಳೆದರು. ಸ್ವಾಮಿಗಳಾದರೋ ಇಲ್ಲಿ ಉನ್ನತ ಮಟ್ಟದ ಆಧ್ಯಾತ್ಮಿಕ ವಿಷಯಗಳನ್ನು ಕೇಳಲು ಕುತೂಹಲಿಗಳಾದ ಕೆಲವು ಮಂದಿ ಶಿಷ್ಯರಿಗೆ ಬೋಧಿಸಿದರು. ಗುರುವಿಗೆ ನಿಜವಾದ ಇಂತಹ ವಾತಾವರಣ ಬೇಕು. ಒಂದು ದೊಡ್ಡ ಜನಸಮುದಾಯದಲ್ಲಿ ಸೂಕ್ಷ್ಮವಾದ ಆಧ್ಯಾತ್ಮಿಕ ವಿಷಯಗಳನ್ನೆಲ್ಲ ಹೇಳಲಾಗುವುದಿಲ್ಲ. ಅದಕ್ಕೆ ಆರಿಸಿದ ಕೆಲವೇ ಮಂದಿ ಶಿಷ್ಯರಿರಬೇಕು. ಸ್ವಾಮೀಜಿ ಬೇರೆ ಕಡೆ ದೊಡ್ಡ ದೊಡ್ಡ ಉಪನ್ಯಾಸಗಳಲ್ಲಿ ಏನನ್ನು ಹೇಳುವರೋ ಅದನ್ನು ಸರಳವಾಗಿ ಇಲ್ಲಿ ಹೇಳುವರು. ಇಲ್ಲಿ ಸ್ವಾಮೀಜಿ ಕೊಟ್ಟ ಪ್ರವಚನವನ್ನು ಅನಂತರ ಬರೆದಿಟ್ಟು ಕೊಂಡಿದ್ದು “\enginline{Inspired talks}” (ಸ್ಫೂರ್ತಿ ವಾಣಿ) ಎಂಬ ಪುಸ್ತಕರೂಪದಲ್ಲಿ ಪ್ರಕಟಿಸಿರುವರು. 

 ಆ ಸಮಯದಲ್ಲಿ ಸಹಸ್ರದ್ವೀಪೋದ್ಯಾನ ಮನೆಯಲ್ಲಿ ಸ್ವಮೀಜಿಯವರ ಪ್ರವಚನವನ್ನು ಕೇಳುತ್ತಿದ್ದವರೆ ಅನಂತರ ತಮ್ಮ ಅನುಭವವನ್ನು ವ್ಯಕ್ತಪಡಿಸಿರುವರು. ಅದನ್ನು ಅವರ ಮಾತಿನಿಂದಲೇ ಕೇಳುವುದು ಉತ್ತಮ. ಮಿಸ್ ಎಸ್. ಇ. ವಾಲ್ಡೊ ಅವರು ಹೀಗೆ ಬರೆಯುತ್ತಾರೆ: 

 “ಸ್ವಾಮೀಜಿಯವರೊಂದಿಗೆ ಆಗ ಅಲ್ಲಿರುವುದಕ್ಕೆ ಅದೃಷ್ಟ ಪಡೆದವರಿಗೆ ಎಂದೆಂದಿಗೂ ಮರೆಯದ ಪುಣ್ಯಸ್ಮೃತಿಯ ದಿನಗಳು ಅವು. ಆಧ್ಯಾತ್ಮಿಕ ಜೀವನ ವಿಕಾಸಕ್ಕೆ ಅಸಾಧಾರಣವಾದ ಅವಕಾಶ ಒದಗಿತ್ತು. ನ್ಯೂಯಾರ್ಕಿನಿಂದ ಸಹಸ್ರ ದ್ವೀಪೋದ್ಯಾನಕ್ಕೆ ಸ್ವಾಮೀಜಿಯವರನ್ನು ಅನುಸರಿಸಿ ಬಂದ ಶಿಷ್ಯರ ಆ ಸಣ್ಣ ಭಕ್ತವೃಂದ ಎಂತಹ ಧ್ಯಾನಾವಸ್ಥೆಯಲ್ಲಿದ್ದರೆಂಬುದನ್ನು ವಿವರಿಸಲು ಸಾಧ್ಯವಿಲ್ಲ. ಅವರು ಪ್ರತಿದಿನವೂ ಆದರದಿಂದ ಸ್ವಾಮೀಜಿಯವರನ್ನು ಸೇವೆ ಮಾಡುತ್ತಿದ್ದರು. ಅದಕ್ಕೆ ಬದಲಾಗಿ ಸ್ವಾಮೀಜಿ ನೀಡುತ್ತಿದ್ದ ಬೋಧನೆಯನ್ನು ಕೃತಜ್ಞತೆಯಿಂದ ಕೇಳುತ್ತಿದ್ದರು. ಸ್ವಾಮೀಜಿ ತಮ್ಮ ಬೋಧನೆಯ ಕರ್ಮದಲ್ಲೆ ತನ್ಮಯರಾಗಿದ್ದರು. ಒಬ್ಬ ಸ್ಫೂರ್ತಿಗೊಂಡ ವ್ಯಕ್ತಿಯಂತೆ ಅವರು ನಮ್ಮೊಡನೆ ಮಾತನಾಡಿದರು. ಅವರ ಪ್ರವಚನಗಳು ಕೇಳಿದವರ ಹೃದಯದಲ್ಲಿ ಚಿರಮುದ್ರಿತವಾಗಿವೆ. ನಮ್ಮ ಮನಸ್ಸು ಆಗ ಎಷ್ಟು ಉತ್ತಮ ಭೂಮಿಕೆಗೆ ಹೋಗುತ್ತಿತ್ತು ಎಂತಹ ತೀವ್ರವಾದ ಆಧ್ಯಾತ್ಮಿಕ ವಾತಾವರಣದಲ್ಲಿ ಇರುತ್ತಿತ್ತು ಎಂಬುದನ್ನು ನಮ್ಮಲ್ಲಿ ಯಾರೂ ಮರೆಯುವಂತೆ ಇರಲಿಲ್ಲ. ಆ ಸಮಯದಲ್ಲಿ ಸ್ವಾಮೀಜಿ ತಮ್ಮ ಹೃದಯಾಂತರಾಳದ ಭಾವನೆಗಳನ್ನೆಲ್ಲ ವ್ಯಕ್ತಪಡಿಸಿದರು. ಅವರು ತಮ್ಮ ಜೀವನದ ಹೋರಾಟವನ್ನೆಲ್ಲ ನಮ್ಮ ಕಣ್ಣ ಮುಂದೆ ಮತ್ತೊಮ್ಮೆ ಸುಳಿಯುವಂತೆ ಮಾಡಿದರು. ನಮ್ಮ ಅನುಮಾನಗಳನ್ನು ಬಹೆಗರಿಸುವುದಕ್ಕೆ, ಪ್ರಶ್ನೆಗಳಿಗೆ ಉತ್ತರ ಕೊಡುವುದಕ್ಕೆ ನಮ್ಮ ಅಂಜಿಕೆಗಳನ್ನು ಪರಿಹರಿಸುವುದಕ್ಕೆ ಸ್ವಾಮೀಜಿಯವರ ಗುರುದೇವರ ಆತ್ಮವೇ ಅವರ ಮೂಲಕ ಮಾತನಾಡುವಂತೆ ಇತ್ತು. ಅನೇಕ ವೇಳೆ ಸ್ವಾಮೀಜಿ ನಮ್ಮ ಇರವನ್ನೇ ಗಮನಿಸುವಂತೆ ತೋರುತ್ತಿರಲಿಲ್ಲ. ಆಗ ನಾವು ಅವರಿಗೆ ಜ್ಞಾಪಕ ಕೊಡುವುದಕ್ಕೆ ಮನಸ್ಸೂ ಬರುತ್ತಿರಲಿಲ್ಲ. ಅವರ ಭಾವನೆಯ ಅವಿಚ್ಛಿನ್ನ ಪ್ರವಾಹಕ್ಕೆ ಆತಂಕವನ್ನು ತಂದೊಡ್ಡಲು ಯತ್ನಿಸಲಿಲ್ಲ.” 

 “ಸ್ವಾಮೀಜಿ ನೇರವಾಗಿ ಮಾತನಾಡುವಂತೆ ತೋರಲಿಲ್ಲ. ಅವರು ತಮಗೆ ತಾವೇ ಬೆಂಕಿಯಂತಹ ವಾಕ್ಯಗಳ ಮೂಲಕ ಮಾತನಾಡಿಕೊಳ್ಳುವಂತೆ ಇತ್ತು. ಈ ಮಾತುಗಳು ಅಷ್ಟು ತೀವ್ರವಾಗಿದ್ದವು, ಅಷ್ಟು ತೃಪ್ತಿದಾಯಕವಾಗಿದ್ದವು. ಕೇಳುವವರೆದೆಯಲ್ಲಿ ಎಂದೆಂದಿಗೂ ಅಚ್ಚಳಿಯದೆ ಚಿರಸ್ಥಾಯಿಯಾಗಿ ಅವು ನಿಂತವು.” 

 ಸ್ವಾಮೀಜಿ ಆಗ ಇದ್ದಷ್ಟು ಮೃದುವಾಗಿ ಯಾವಾಗಲೂ ಇರಲಿಲ್ಲ. ಇದು ಅವರ ಶ‍್ರೀಗುರುದೇವರು ತಮ್ಮ ಶಿಷ್ಯರನ್ನು ತರಬೇತು ಮಾಡಿದ ರೀತಿಯೇ ಇರಬೇಕು. ತಾನು ತನ್ನ ಪವಿತ್ರತಮ ಭಾವನೆಯನ್ನು ಆನಂದಿಸುತ್ತಿರುವಾಗ ಇತರರು ಅದನ್ನು ಕೇಳುವಂತೆ ಇತ್ತು.” 

 “ಸ್ವಾಮಿ ವಿವೇಕಾನಂದರಂತಹ ವ್ಯಕ್ತಿಯೊಡನೆ ಇರುವುದೇ ನಿರಂತರವಾಗಿ ಸ್ಫೂರ್ತಿಗೊಂಡಂತೆ. ಬೆಳಗಿನಿಂದ ಸಾಯಂಕಾಲದವರೆಗೆ ಒಂದೇ ಸಮನಾಗಿ, ತೀವ್ರವಾಗಿ ಆಧ್ಯಾತ್ಮಿಕ ವಾತಾವರಣದಲ್ಲಿ ನಾವು ಇದ್ದೆವು. ಅನೇಕ ವೇಳೆ ತಮಾಷೆ ಮಾಡುತ್ತಿದ್ದರು, ಹಾಸ್ಯ ಮಾಡುತ್ತಿದ್ದರು. ಬಹಳ ಚಮತ್ಕಾರವಾಗಿ ಮಾರುತ್ತರವನ್ನು ಕೊಡುತ್ತಿದ್ದರು. ಹಾಗಿದ್ದರೂ ಅವರ ಜೀವನದ ಮುಖ್ಯ ಪಲ್ಲವಿಯಾದ ಆಧ್ಯಾತ್ಮಿಕತೆಯಿಂದ ದೂರವಿರಲಿಲ್ಲ. ಪ್ರತಿಯೊಂದು ಘಟನೆಯ ಮೂಲಕವಾಗಿಯೂ, ಏನನ್ನಾದರೂ ಬೋಧಿಸುತ್ತಿದ್ದರು. ತಮಾಷೆಯಿಂದ ಕೂಡಿದ ಯಾವುದಾದರೂ ಪೌರಾಣಿಕ ಕಥೆಗಳಿಗೆ ಕ್ಷಣಾರ್ಧದಲ್ಲಿ ಗಹನವಾದ ತತ್ತ್ವದ ಆಳಕ್ಕೆ ಇಳಿದುಹೋಗುತ್ತಿದ್ದರು. ಸ್ವಾಮೀಜಿಯವರಿಗೆ ಗೊತ್ತಿದ್ದ ಪೌರಾಣಿಕ ಕಥೆಗಳಿಗೆ ಒಂದು ಕೊನೆಮೊದಲಿಲ್ಲ. ಪುರಾತನ ಆರ್ಯ ಜನಾಂಗದಷ್ಟು ಮತ್ತಾವ ಜನಾಂಗವೂ ಇಷ್ಟೊಂದು ಕಥೆಗಳನ್ನು ಪ್ರಪಂಚಕ್ಕೆ ಒದಗಿಸಿರಲಾರದು. ಸ್ವಾಮೀಜಿ ನಮಗೆ ಅವನ್ನು ಹೇಳಲು ಸಂತೋಷಪಡುತ್ತಿದ್ದರು. ನಮಗಂತೂ ಅವನ್ನು ಕೇಳಿದಾಗ ಪರಮಾನಂದ ಆಗುತ್ತಿತ್ತು. ಆ ಕಥೆಯ ಹಿಂದೆ ಇರುವ ತತ್ತ್ವವನ್ನು ಅವರು ನಮಗೆ ತೋರುವುದನ್ನು ಎಂದೂ ಮರೆಯುತ್ತಿರಲಿಲ್ಲ. ಅವುಗಳಿಂದ ಅಮೋಘವಾದ ಆಧ್ಯಾತ್ಮಿಕ ನೀತಿಗಳನ್ನು ನಮಗೆ ತೋರುವುದನ್ನು ಮರೆಯುತ್ತಿರಲಿಲ್ಲ. ಇಂತಹ ಶ್ರೇಷ್ಠಗುರು ತನಗೆ ದೊರೆತ ಅದೃಷ್ಟಕ್ಕೆ ಶಿಷ್ಯ ತನ್ನನ್ನು ಎಷ್ಟು ಅಭಿನಂದಿಸಿಕೊಂಡರೂ ಸಾಲದಾಗಿದೆ.” 

 “ಸ್ವಾಮೀಜಿಯವರ ಭಾವನೆಗಳು ನಮಗೆ ಹೊಸದಾಗಿ ಕಂಡವು. ವಿಚಿತ್ರವಾಗಿ ಕಂಡವು. ನಾವು ಅವನ್ನು ಬಹಳ ನಿಧಾನವಾಗಿ ಮಾತ್ರ ಅರ್ಥಮಾಡಿಕೊಳ್ಳಬಹುದಾಗಿತ್ತು. ಆದರೆ ಸ್ವಾಮೀಜಿಯ ಸಹನೆ ಎಂದಿಗೂ ಬತ್ತಲಿಲ್ಲ. ಅವರ ಉತ್ಸಾಹ ಎಂದಿಗೂ ಕುಂಠಿತವಾಗಿಲ್ಲ.” 

 “ಒಂದು ಅಪೂರ್ವ ಘಟನೆಯಂತೆ ಹನ್ನೆರಡು ಜನ ಶಿಷ್ಯರು ಮಾತ್ರ ಸ್ವಾಮೀಜಿಯವರನ್ನು ಸಹಸ್ರದ್ವೀಪೋದ್ಯಾನಕ್ಕೆ ಅನುಸರಿಸಿದುದು ಕೇವಲ ಆಕಸ್ಮಿಕವೆ! ಸ್ವಾಮೀಜಿ ನಮ್ಮನ್ನು ತಮ್ಮ ಶಿಷ್ಯರಂತೆ ಸ್ವೀಕರಿಸುವೆನು ಎಂದರು. ಆದಕಾರಣವೆ ಅವರು ಯಾವಾಗಲೂ ನಮಗೆ ಉಚಿತವಾಗಿ ಬೋಧಿಸಿದರು, ತಮ್ಮ ಶ್ರೇಷ್ಠಭಾವನೆಗಳನ್ನು ಕೊಟ್ಟರು. ಹನ್ನೆರಡು ಜನವೂ ಯಾವಾಗಲೂ ಇರುತ್ತಿರಲಿಲ್ಲ. ಒಂದು ಸಲಕ್ಕೆ ಹತ್ತು ಜನವೇ ಜಾಸ್ತಿ ಇದ್ದುದು. ನಮ್ಮಲ್ಲಿ ಇಬ್ಬರು ಅನಂತರ ಸಂನ್ಯಾಸಿಗಳಾದರು.” 

 “ನಮಗೆ ಮಂತ್ರೋಪದೇಶವನ್ನು ಅನುಗ್ರಹಿಸಿದ ಸಮಾರಂಭವಾದರೊ, ಅತ್ಯಂತ ಸರಳವಾದ ಅದ್ಭುತ ಪರಿಣಾಮಕಾರಿಯಾಗುವ ರೀತಿಯಲ್ಲಿತ್ತು. ಒಂದು ಸಣ್ಣ ಯಜ್ಞ ವೇದಿಕೆ, ಅದರ ಸಮೀಪದಲ್ಲಿ ಅಂದವಾದ ಹೂಗಳು. ಗುರುವಿನ ಶ್ರದ್ಧಾಪೂರ್ವಕವಾದ ಬೊಧನೆಗಳಂತೆಯೇ ಅವು ಅಲ್ಲಿದ್ದವು. ಅದು ಒಂದು ದಿನ ಬೇಸಿಗೆಯ ಅರುಣೋದಯದಲ್ಲಿ ಆಯಿತು. ಆ ದೃಶ್ಯ ನಮ್ಮ ಮನಸ್ಸಿನಲ್ಲಿ ಇನ್ನೂ ಅಚ್ಚಳಿಯದೆ ಇದೆ.” 

 ಶ‍್ರೀಮತಿ ಫಂಕಿ ಎಂಬ ಮತ್ತೊಬ್ಬ ಶಿಷ್ಯೆ ತನ್ನ ನೆನಪನ್ನು ಈ ರೀತಿ ಚಿತ್ರಿಸುವಳು: “ಮೊದಲನೆ ವೇಳೆ ಬಂದಾಗ ಅವರನ್ನು ಪರಿಚಯ ಮಾಡಿಕೊಳ್ಳಲು ಆಗಲಿಲ್ಲ. ಆದರೆ ಅವರು ಏನನ್ನು ಹೇಳಿದ್ದರೋ ಅದನ್ನು ಮನನ ಮಾಡುತ್ತಿದ್ದೆವು. ಮತ್ತೊಮ್ಮೆ ಅವರನ್ನು ಎಲ್ಲಿಯಾದರೂ ಯಾವಾಗಲಾದರೂ ನೋಡಬೇಕು, ಅದಕ್ಕಾಗಿ ಭೂಪ್ರದಕ್ಷಿಣೆ ಮಾಡಬೇಕಾಗಿ ಬಂದರೂ ಚಿಂತೆಯಿಲ್ಲ ಎಂದು ಭಾವಿಸಿಕೊಂಡೆವು. ಸುಮಾರು ಒಂದೂವರೆ ವರುಷದವರೆಗೆ ಅವರ ಸಮಾಚಾರವೇ ನಮಗೆ ತಿಳಿಯಲಿಲ್ಲ. ಬಹುಶಃ ಸ್ವಾಮೀಜಿ ಇಂಡಿಯಾ ದೇಶಕ್ಕೆ ಹಿಂತಿರುಗಿ ಹೋಗಿಬಿಟ್ಟಿರಬಹುದೆಂದು ಭಾವಿಸಿದೆವು. ಒಂದು ದಿನ ಮಧ್ಯಾಹ್ನ ನಮ್ಮ ಸ್ನೇಹಿತರೊಬ್ಬರು ಸ್ವಾಮೀಜಿ ಇನ್ನೂ ಅಮೆರಿಕಾ ದೇಶದಲ್ಲಿಯೇ ಇರುವರೆಂದೂ ಅವರು ಸಹಸ್ರ ದ್ವೀಪೋಧ್ಯಾನದಲ್ಲಿ ಬೇಸಿಗೆಯನ್ನು ಕಳೆಯುವರೆಂದೂ ಕೇಳಿದೆವು. ನಾವು ಮಾರನೆ ದಿನವೇ ಅವರನ್ನು ಹುಡುಕುವುದಕ್ಕೆ ಹೊರಟೆವು. ನಮಗೆ ಬೋಧಿಸಿ ಎಂದು ಅವರನ್ನು ಕೇಳಬೇಕೆಂದು ನಿರ್ಧರಿಸಿದೆವು.” 

 “ಬಹಳ ಹುಡುಕಾಡಿದ ಮೇಲೆ ಅವರು ನಮಗೆ ಸಿಕ್ಕಿದರು. ಅವರು ಒಬ್ಬರೇ ಇದ್ದಾಗ ಅವರ ಶಾಂತಿಗೆ ಭಂಗ ಬರುವಂತೆ ನಾವು ಅವರ ಬಳಿಗೆ ಹೋದರೆ ಅವರು ಏನನ್ನುವರೋ ಎಂದು ಭಾವಿಸಿದೆವು. ಆದರೆ ಅವರು ಹಿಂದೆ ನಮ್ಮ ಹೃದಯದಲ್ಲಿ ಒಂದು ಜ್ಯೋತಿಯನ್ನು ಹಚ್ಚಿಸಿದ್ದರು. ಅದನ್ನು ಯಾರೂ ಆರಿಸುವುದಕ್ಕೆ ಆಗಲಿಲ್ಲ. ನಾವು ಆ ಅದ್ಭುತ ವ್ಯಕ್ತಿಯ ಜೀವನ ಮತ್ತು ಸಂದೇಶವನ್ನು ಮತ್ತೂ ಕೇಳಬೇಕು ಎಂದು ಆಶಿಸಿದೆವು. ಅಂದು ರಾತ್ರಿಯಾಗಿತ್ತು, ಮಳೆ ಬರುತ್ತಿತ್ತು. ನಾವು ಬಹಳ ದೂರ ನಡೆದು ನಡೆದು ಸಾಕಾಗಿತ್ತು. ಆದರೆ ಸ್ವಾಮೀಜಿಯವರನ್ನು ಕಣ್ಣಾರೆ ನೋಡಿದಲ್ಲದೆ ನಮಗೆ ಶಾಂತಿ ಇರಲಿಲ್ಲ. ಅವರು ನಮ್ಮನ್ನು ಸ್ವೀಕರಿಸುವರೆ! ಅವರು ಸ್ವೀಕರಿಸದೇ ಇದ್ದರೆ ನಾವೇನು ಮಾಡಬೇಕು? ನಮ್ಮ ವಿಷಯ ಏನನ್ನೂ ಅರಿಯದ ಒಬ್ಬನನ್ನು ನೋಡಲು ಹಲವು ನೂರು ಮೈಲಿಗಳು ಹೋಗುವುದು ಮೌಢ್ಯತೆ ಎಂದು ಒಮ್ಮೆ ನನಗೆ ಹೊಳೆಯಿತು. ಆದರೆ ನಾವು ಬೆಟ್ಟದ ಕಡೆಗೆ ನಡೆದುಕೊಂಡು ಹೋದೆವು. ಯಾರನ್ನೋ ದೀಪವನ್ನು ಹಿಡಿದುಕೊಂಡು ದಾರಿಯನ್ನು ತೋರಿಸಲು ಗೊತ್ತು ಮಾಡಿದ್ದೆವು. ಅನಂತರ ಸ್ವ್ವಾಮೀಜಿ ನಮ್ಮನ್ನು ಕುರಿತು ‘ಈ ನನ್ನ ಶಿಷ್ಯರು ನನ್ನನ್ನು ಹುಡುಕಿಕೊಂಡು ನೂರಾರು ಮೈಲಿಗಳು ರಾತ್ರಿ ಮಳೆಯಲ್ಲಿ ಬಂದರು’ ಎಂದು ಹೇಳುತ್ತಿದ್ದರು. ನಾವು ಅವರಿಗೆ ಏನು ಹೇಳಬೇಕು ಎಂಬುದನ್ನು ಕುರಿತು ಆಲೋಚಿಸಿದ್ದೆವು. ಆದರೆ ಅವರ ಸಮ್ಮುಖದಲ್ಲಿ ನಿಂತಾಗ ನಾವು ಏನನ್ನು ಹೇಳಬೇಕು ಎಂದು ಮನಸ್ಸು ಮಾಡಿಕೊಂಡಿದ್ದೆವೊ ಅದೆಲ್ಲ ಮರೆತುಹೋಯಿತು. ನಮ್ಮಲ್ಲಿ ಒಬ್ಬರು ‘ನಾವು ಡೆಟ್ರಾಯಿಟ್‍ನಿಂದ ಬಂದೆವು. ಶ‍್ರೀಮತಿ...ವ ಅವರು ನಮ್ಮನ್ನು ನಿಮ್ಮೆಡೆಗೆ ಕಳುಹಿಸಿದರು’ ಎಂದರು. ಮತ್ತೊಬ್ಬರು ‘ಕ್ರಿಸ್ತನೇನಾನದೂ ಬದುಕಿದ್ದರೆ ಅವನ ಬಳಿಗೆ ಹೋಗುವಂತೆ ನಾವು ನಿಮ್ಮೆಡೆಗೆ, ನಮಗೆ ಬೋಧಿಸಿ ಎಂದು ಬೇಡುವುದಕ್ಕೆ ಬಂದಿರುವೆವು’ ಎಂದರು. ‘ನಿಮ್ಮನ್ನು ಬಿಡುಗಡೆ ಮಾಡಲು ಕ್ರಿಸ್ತನಲ್ಲಿದ್ದ ಶಕ್ತಿ ನನ್ನಲ್ಲಿ ಇದ್ದಿದ್ದರೆ!’ ಎಂದರು ಸ್ವಮೀಜಿ. ಸ್ವಲ್ಪ ಹೊತ್ತು ಏನನ್ನೋ ಯೋಚಿಸುತ್ತಿರುವಂತೆ ನಮ್ಮನ್ನು ನೋಡಿ ಅನಂತರ ಆ ಮನೆಯ ಯಜಮಾನಿಯ ಕಡೆ ತಿರುಗಿ ‘ಈ ಸ್ತ್ರೀಯರು ಡೆಟ್ರಾಯಿಟ್‍ನಿಂದ ಬಂದಿರುವರು. ಮನೆಯಲ್ಲಿ ಅವರಿಗೆ ಸ್ವಲ್ಪ ಅವಕಾಶ ಕೊಡಿ. ಇವತ್ತು ರಾತ್ರಿ ಅವರು ನಮ್ಮೊಡನೆ ಕಾಲ ಕಳೆಯುವರು’ ಎಂದರು. ಅಂದು ರಾತ್ರಿ ಸ್ವಾಮೀಜಿಯ ಪ್ರವಚನವನ್ನು ಬಹಳ ಕಾಲದವರೆಗೂ ಕೇಳಿದೆವು. ಅನಂತರ ಅವರು ನಮ್ಮನ್ನು ಗಮನಿಸಲಿಲ್ಲ. ನಾವು ಅವರಿಗೆ ಹಿಂತಿರುಗಿ ಹೋಗುತ್ತೇವೆ, ಎಂದು ಹೇಳಿದಾಗ ಮಾರನೆ ದಿನ ಬೆಳಿಗ್ಗೆ ಒಂಭತ್ತು ಗಂಟೆಗೆ ಬನ್ನಿ ಎಂದು ಹೇಳಿದರು. ನಾವು ಸರಿಯಾಗಿ ಅಷ್ಟು ಹೊತ್ತಿಗೆ ಮಾರನೆ ದಿನ ಬಂದೆವು. ಗುರುದೇವರು ನಮ್ಮನ್ನು ಸ್ವೀಕರಿಸುವುದಕ್ಕೆ ಮನಸ್ಸು ಮಾಡಿದ್ದರು. ಆ ಮನೆಯ ಶಿಷ್ಯರ ಸಂಸಾರದಲ್ಲಿ ನಮ್ಮನ್ನೂ ಒಬ್ಬರನ್ನಾಗಿ ಸ್ವೀಕರಿಸಿದರು.” 

 ಈ ಸಮಯದಲ್ಲಿ ಆಕೆ ತನ್ನ ಸ್ನೇಹಿತನಿಗೆ ಬರೆದ ಪತ್ರವೊಂದರಲ್ಲಿ ಹೀಗೆ ವಿವರಿಸುವಳು: 

 “ವಿವೇಕಾನಂದರು ಇರುವ ಮನೆಯಲ್ಲೇ ಈಗ ನಾವು ಇರುವೆವು. ಬೆಳಿಗ್ಗೆ ಎಂಟು ಗಂಟೆಯಿಂದ ರಾತ್ರಿ ಅವೇಳೆ ಹೊತ್ತಿನವರೆಗೆ ಅವರ ಬೋಧನೆಯನ್ನು ಕೇಳುತ್ತಿರುವೆವು. ನನ್ನ ಅತ್ಯಂತ ಹುಟ್ಟು ಕಲ್ಪನೆಯಲ್ಲಿಯೂ ಇಷ್ಟು ಅದ್ಭುತವಾಗಿರುವುದನ್ನು ಇಷ್ಟು ಪರಿಪೂರ್ಣವಾಗಿರುವುದನ್ನು ಚಿತ್ರಿಸಿಕೊಂಡಿರಲಿಲ್ಲ. ಸ್ವಾಮಿ ವಿವೇಕಾನಂದರ ಜೊತೆಯಲ್ಲಿರುವುದು! ಅವರಿಂದ ಸ್ವೀಕರಿಸಲ್ಪಡುವುದು!” 

 “ಸ್ವಾಮಿ ವಿವೇಕಾನಂದರ ಆ ಪವಿತ್ರವಾದ ಬೋಧನೆಗಳು! ಅವರು ಭೂತ ಪ್ರೇತಗಳು ಯಾವುದನ್ನೂ ಮಾತನಾಡುವುದಿಲ್ಲ. ದೇವರು ಬುದ್ಧ ಜೀಸಸ್ ಇವರ ವಿಷಯವನ್ನು ಮಾತ್ರ ಮಾತನಾಡುವರು. ನಾನು ಇನ್ನು ಮೇಲೆ ಹಿಂದಿನಂತೆ ಇರುವುದಕ್ಕೆ ಆಗುವುದಿಲ್ಲ ಎನಿಸುವುದು. ನನಗೆ ಈಗ ಸತ್ಯದ ಒಂದು ಮಿಂಚಿನ ನೋಟ ಸಿಕ್ಕಿದೆ.” 

 “ಪ್ರತಿದಿನ ಭೋಜನ ಸಮಯದಲ್ಲಿ ಸ್ವಾಮಿ ವಿವೇಕಾನಂದರ ವಾಣಿಯನ್ನು ಕೇಳುವುದೆಂದರೆ ಏನು! ಬೆಳಿಗ್ಗೆ ಮತ್ತು ರಾತ್ರಿ ಮನೆ ಅಂಗಳದ ಮೇಲೆ ಚಿನ್ನದ ಹಣತೆಗಳಂತೆ ಉರಿಯುತ್ತಿರುವ ತಾರಾವಳಿಗಳು ಆಕಾಶದಲ್ಲಿ, ಅಂತಹ ಸಮಯದಲ್ಲಿ ವಿವೇಕಾನಂದರ ಪ್ರವಚನವನ್ನು ಕೇಳುವುದು! ಅವರು ನಿಜವಾಗಿಯೂ ಹರಿಯುತ್ತಿರುವ ಝರಿಗಳಲ್ಲಿ ಪುಸ್ತಕವನ್ನು, ಕಲ್ಲುಗಳಲ್ಲಿ ಸಂದೇಶವನ್ನು ಮತ್ತು ಎಲ್ಲದರಲ್ಲಿಯೂ ಒಳ್ಳೆಯದನ್ನು ನೋಡುವರು. ಇದೇ ಸ್ವಾಮಿಗಳು ಹಾಸ್ಯಪ್ರಿಯರು, ಸಂತೋಷಲೋಲರು. ಕೆಲವು ವೇಳೆ ನಾವು ಸಂತೋಷದಿಂದ ಹುಚ್ಚರಾಗಿ ಹೋಗುವೆವು.” 

 “ಕೆಲವು ದಿನಗಳಿಂದ ನಾವು ತತ್ತ್ವದ ಆಕಾಶದಲ್ಲಿ ಹಾರಾಡುತ್ತಿರುವೆವು. ಸದ್ಯಕ್ಕೆ ಡೆಟ್ರಾಯಿಟ್ ಅನ್ನು ಮರೆಯಬೇಕೆಂದು ಸ್ವಾಮೀಜಿ ಹೇಳಿರುವರು. ಅವರ ಬೋಧನೆಯನ್ನು ಕೇಳುತ್ತಿರುವಾಗ ಲೌಕಿಕವಾದುದನ್ನೆಲ್ಲಾ ಮರೆಯಿರಿ ಎನ್ನುವರು. ಒಂದು ಹುಲ್ಲಿನೆಸಳಿನಿಂದ ಮನುಷ್ಯನವರೆಗೆ ಎಲ್ಲರಲ್ಲೂ ದೇವರನ್ನು ನೋಡಿ ಎಂದು ಅವರು ನಮಗೆ ಬೋಧಿಸುತ್ತಿರುವರು. ಅತ್ಯಂತ ಅಸುರೀ ವ್ಯಕ್ತಿಯಲ್ಲಿ ಕೂಡ ಅವನನ್ನೇ ನೋಡಿ ಎನ್ನುವರು!” 

 “ಕಾಗದ ಬರೆಯುವುದಕ್ಕೆ ನಿಜವಾಗಿ ಕಾಲವೇ ಸಿಕ್ಕುತ್ತಿಲ್ಲ. ಇಲ್ಲಿ ಜನ ಜಾಸ್ತಿ ಇರುವುದರಿಂದ ಸ್ವಲ್ಪ ಅನುಕೂಲವಾಗಿದೆ, ಇಲ್ಲಿ ಮನರಂಜನೆಗೆ ವಿಶ್ರಾಂತಿಗೆ ಸಮಯವೇ ಇಲ್ಲ. ಏಕೆಂದರೆ ಸ್ವಾಮೀಜಿ ಬೇಗನೆ ಇಂಗ್ಲೆಂಡಿಗೆ ಹೋಗುವವರಾಗಿರುವವರು. ನಾವೆಲ್ಲ ಅಣಿಯಾಗುವುದಕ್ಕೆ ಸ್ವಲ್ಪ ಕಾಲ ತೆಗೆದುಕೊಳ್ಳುತ್ತೇವೆ. ಎಲ್ಲಿ ಹೊತ್ತಾಗಿಹೋದರೆ ಅನರ್ಘ್ಯ ರತ್ನಗಳು ನಮ್ಮ ಪಾಲಿಗೆ ದೊರೆಯುವುದಿಲ್ಲವೋ ಎಂದು ಕಾತರರಾಗಿರುವೆವು. ಅವರ ಮಾತುಗಳು ರತ್ನದಂತೆ. ಅವರು ಹೇಳುವುದೆಲ್ಲ ಒಂದು ಸುಂದರವಾದ ಕೆತ್ತನೆಯ ಕೆಲಸದಲ್ಲಿ ಅಡಕವಾಗಿ ಕುಳಿತುಕೊಳ್ಳುವುದು. ಅವರು ಬೋಧಿಸುತ್ತಿರುವಾಗ ಯಾವ ಯಾವುದೋ ದೂರದ ವಿಷಯಗಳ ಕಡೆ ಹೋದರೂ ಒಂದು ಮುಖ್ಯ ಸಿದ್ಧಾಂತದ ಕಡೆ ಬರುವರು. ಅದೇ ದೇವರನ್ನು ಸಾಕ್ಷಾತ್ಕರಿಸಿಕೊಳ್ಳಿ, ಮಿಕ್ಕಿರುವುದೆಲ್ಲ ಗೌಣ ಎಂಬುದು.” 

 “ಈ ಮನೆಯಲ್ಲಿರುವವರೆಲ್ಲ ಒಳ್ಳೆಯವರೇ. ಆದರೂ ನಾನು ಮಿಸ್ ವಾಲ್ಡೊ ಮತ್ತು ಮಿಸ್ ಎಲ್ಲಿಸ್ ಇವರನ್ನು ತುಂಬ ಮೆಚ್ಚುತ್ತೇನೆ. ಅವರಲ್ಲಿ ಕೆಲವರು ಅಸಾಧಾರಣ ವ್ಯಕ್ತಿಗಳು ಇರುವರು. ಅವರಲ್ಲಿ ಒಬ್ಬರೇ ಕೇಂಬ್ರಿಡ್ಜ್ ನ ಡಾಕ್ಟರ್ ರೈಟ್ ಎನ್ನುವವರು. ಸುಸಂಸ್ಕೃತ ಮನುಷ್ಯ. ಕೆಲವು ವೇಳೆ ತುಂಬಾ ಹಾಸ್ಯ ಮಾಡುತ್ತಾನೆ. ಆತ ಸ್ವಾಮೀಜಿ ಅವರ ಬೋಧನೆಯಲ್ಲಿ ತಲ್ಲೀನನಾಗಿ ಪ್ರತಿದಿನ ಪ್ರವಚನದ ಕೊನೆಯಲ್ಲಿ ‘ಸ್ವಾಮೀಜಿ, ಕೊನೆಗೆ ಇದೆಲ್ಲ ನಾನೇ ಬ್ರಹ್ಮ, ಅಖಂಡ ಅದ್ವಿತೀಯ ಎಂಬುದಕ್ಕೆ ಬರುತ್ತದೆ ಅಲ್ಲವೆ?’ ಎಂದು ಕೇಳುವನು. ಸ್ವಾಮೀಜಿ ಆಗ ತಮಾಮಾಷೆಯಿಂದ ನಗುತ್ತ ಹೇಳುವುದನ್ನು ಕೇಳಬೇಕು! ‘ಹೌದೂ ಡಾಕ್ಟರ್, ನಿನ್ನ ನಿಜವಾದ ಸ್ವಭಾವದ ದೃಷ್ಟಿಯಿಂದ ನೀನು ಬ್ರಹ್ಮನೆ, ಅಖಂಡ ಅದ್ವಿತೀಯನೆ’ ಎನ್ನುವರು. ಅನಂತರ ಆ ವಿದ್ಯಾವಂತನಾದ ಡಾಕ್ಟರ್ ಊಟಕ್ಕೆ ಸ್ವಲ್ಪ ಹೊತ್ತಾಗಿ ಬಂದರೆ, ಸ್ವಾಮೀಜಿ ಕಣ್ಣಿನಲ್ಲಿ ಹಾಸ್ಯವನ್ನು ಪ್ರಕಟಿಸುತ್ತಾ ಸ್ವಲ್ಪ ಗಂಭೀರವಾಗಿಯೇ ‘ನೋಡಿ, ಇಲ್ಲಿ ಬ್ರಹ್ಮ ಬಂದ, ಅಖಂಡ ಬಂದ’ ಎನ್ನುವರು” 

 “ಸ್ವಾಮೀಜಿ ವಿನೋದಪ್ರಿಯರು. ಅವರು ಕೆಲವು ವೇಳೆ ನಾನು ನಿಮಗೆ ಅಡಿಗೆ ಮಾಡುತ್ತೇನೆ ಎನ್ನುತ್ತಾರೆ. ಅವರು ಚೆನ್ನಾಗಿ ಅಡಿಗೆ ಮಾಡಬಲ್ಲರು. ಮಾಡಿದುದನ್ನು ಸಹೋದರ ಕೂಟಕ್ಕೆಲ್ಲ ಬಡಿಸಲು ಅವರಿಗೆ ತುಂಬಾ ಸಂತೋಷ. ಅವರ ಅಡಿಗೆ ರುಚಿಕರವಾಗಿರುವುದು. ಆದರೆ ಬಹಳ ಖಾರ ಮತ್ತು ಬೇಕಾದಷ್ಟು ಮಸಾಲೆ ಹಾಕುವರು. ಆದರೆ ನಾನೇನೋ ಊಟ ಮಾಡಲು ಸಂಕಲ್ಪ ಮಾಡಿರುವೆನು. ಅದು ನನ್ನನ್ನು ಸಾಯಿಸಿದರೂ ಚಿಂತೆಯಿಲ್ಲ. ವಿವೇಕಾನಂದರಂತಹವರು ಅಡಿಗೆ ಮಾಡಿದರೆ, ನಮ್ಮಂತಹವರು ಊಟವನ್ನಾದರೂ ಮಾಡಬೇಕು. ಅವರನ್ನು ದೇವರು ಆಶೀರ್ವದಿಸಲಿ!” 

 “ಅಂತಹ ಸಮಯದಲ್ಲಿ ನಾವು ಒಂದು ನಡಗುವ ಸುಂಟರಗಾಳಿಯಲ್ಲಿಯೇ ಸಿಕ್ಕಿಹಾಕಿಕೊಳ್ಳುವೆವು. ಸ್ವಾಮೀಜಿ ಅವರು ಅಂತಹ ಸಮಯದಲ್ಲಿ ನಮ್ಮ ಕೈಮೇಲೆ ಒಂದು ಬಿಳಿಯ ಟವಲನ್ನು ಹಾಕಿಕೊಂಡು ರೈಲಿನ ಊಟದ ಬಂಡಿಯ ಬಟ್ಲರ್‍ನಂತೆ ‘ಊಟದ ಬಂಡಿಗೆ ಬರುವುದಕ್ಕೆ ಇದೇ ಕೊನೆಯ ಕರೆ. ಊಟವನ್ನೆಲ್ಲ ಆಗಲೆ ಬಡಿಸಿ ಆಗಿದೆ!’ ಎನ್ನುವರು. ಊಟಮಾಡುವ ಮೇಜಿನ ಮೇಲಂತೂ ಅಷ್ಟೊಂದು ಹಾಸ್ಯ ತಮಾಷೆ! ಅಲ್ಲಿ ಯಾರ ನಡತೆಯಲ್ಲಾದರೂ ಇರುವ ಏನಾದರೂ ವೈಚಿತ್ರತೆಯನ್ನು ಹಾಸ್ಯಮಯವಾಗಿ ಬಣ್ಣಿಸುವರು. ಆದರೆ ಎಂದಿಗೂ ಕಟುವಾಗಿ ಟೀಕಿಸುತ್ತಿರಲಿಲ್ಲ. ಬರೀ ತಮಾಷೆಗಾಗಿ” 

 “ನಾನು ನಿನಗೆ ಸ್ವಾಮೀಜಿಯವರ ಹಾಸ್ಯ ಪ್ರವೃತ್ತಿಯ ವಿಷಯವಾಗಿ ಕಾಗದ ಬರೆದಾದಮೇಲೆ ಎಷ್ಟೋ ಸಣ್ಣ ಘಟನೆಗಳು ಜರುಗಿವೆ. ಅದರಲ್ಲಿ ಸ್ವಾಮೀಜಿಯವರ ಶೀಲದ ಹಲವು ಭಾಗಗಳು ನಮಗೆ ವ್ಯಕ್ತವಾಗುವುವು. ಅವರು ಹೇಳುವುದನ್ನೆಲ್ಲ ಟಿಪ್ಪಣಿ ಬರೆದಿಡಬೇಕೆಂದು ಯೋಚಿಸುವೆನು. ಆದರೆ ನಾನು ಕೇಳುವುದರಲ್ಲೆ ಮೈಮರೆತು ಹೋಗುವೆ. ಟಿಪ್ಪಣಿ ಮರೆತೇ ಹೋಗುವುದು. ಅವರ ಧ್ವನಿಯೋ ಅತ್ಯಂತ ಇಂಪಾಗಿದೆ. ಆ ಧ್ವನಿಯ ಇಂಪಿನಲ್ಲೆ ಒಬ್ಬ ತನ್ಮಯನಾಗಬಹುದು. ಏನಾದರೂ ಆಗಲಿ, ಮಿಸ್ ವಾಲ್ಡೊ ಸ್ವಾಮೀಜಿ ಹೇಳಿದುದನ್ನೆಲ್ಲ ಪೂರ್ತಿ ಬರೆದು ಇಡುತ್ತಿರುವಳು. ಈ ರೀತಿಯಾಗಿ ಅವುಗಳು ಶಾಶ್ವತವಾಗಿ ಉಳಿಯುತ್ತವೆ.” 

 “ನಾನು ಮತ್ತು ಸೋದರಿ ಕ್ರಿಸ್ಟೈನ್ ಜನಿಸಿದ ಸಮಯದಲ್ಲಿ ಯಾರೋ ಒಬ್ಬ ಒಳ್ಳೆಯ ದೇವತೆಯ ಆಶೀರ್ವಾದ ನಮ್ಮ ಪಾಲಿಗೆ ಬಂದಿರಬೇಕು ಎಂದು ಭಾವಿಸುವೆನು. ನಮಗೆ ಇನ್ನೂ ಕರ್ಮ ಮತ್ತು ಪುನರ್ಜನ್ಮಗಳು ಹೆಚ್ಚಾಗಿ ಗೊತ್ತಿಲ್ಲ. ಆದರೆ ಸ್ವಾಮೀಜಿಯವರ ಸಂಪರ್ಕ ನಮಗೆ ಸಿಕ್ಕಿರುವುದನ್ನು ನೋಡಿದರೆ ಎರಡೂ ನಿಜವಿರಬೇಕು ಎನ್ನಿಸುವುದು.” 

 “ನಾನು ಕೆಲವು ವೇಳೆ ಅವರಿಗೆ ನೇರವಾದ ಪ್ರಶ್ನೆಗಳನ್ನೇ ಹಾಕುತ್ತೇನೆ. ಏಕೆಂದರೆ ಅವರು ಅದನ್ನು ಹೇಗೆ ಸಾವಧಾನದಿಂದ ನೋಡುವರು! ನಾನು ಕೆಲವು ವೇಳೆ ಉದ್ವೇಗದಿಂದ ‘ದೇವತೆಗಳೇ ನಡೆಯುವುದಕ್ಕೆ ಎಲ್ಲಿ ಅಂಜುವರೊ ಅಲ್ಲಿ ಮೂರ್ಖರು ಧಾವಿಸುವರು’ ಎಂಬಂತೆ ವರ್ತಿಸುತ್ತಿದ್ದೆ. ಒಂದು ಸಲ ಅವರು ‘ಮಿಸ್ ಫಂಕಿ ನನ್ನನ್ನು ಪರೀಕ್ಷಿಸುತ್ತಾಳೆ. ಅವಳು ಎಷ್ಟು ಸರಳ ಹೃದಯಳು’ ಎಂದು ಹೇಳಿದರು. ಅವರು ಹಾಗೆ ಹೇಳಿದ್ದು ಎಷ್ಟು ವಿಶ್ವಾಸದಿಂದ ಕೂಡಿತ್ತು!” 

 ಒಂದು ದಿನ ಸಾಯಂಕಾಲ ಮಳೆ ಬರುತ್ತಿದ್ದಾಗ ನಾವೆಲ್ಲ ಕೋಣೆಯಲ್ಲಿ ಕುಳಿತಿದ್ದೆವು. ಸ್ವಾಮೀಜಿಯವರು ಪಾತಿವ್ರತ್ಯದ ಮೇಲೆ ಮಾತನಾಡುತ್ತಿದ್ದರು. ಆಗ ಸೀತೆಯ ಕಥೆಯನ್ನು ಹೇಳಿದರು. ಅವರು ಹೇಗೆ ಕಥೆಯನ್ನು ಹೇಳಬಲ್ಲರು! ಅವರು ಕಥೆ ಹೇಳುತ್ತಿದ್ದರೆ ನಮ್ಮ ಕಣ್ಣೆದುರಿಗೆ ಅದು ಕಾಣುವುದು, ಅಲ್ಲಿರುವ ವ್ಯಕ್ತಿಗಳು ಸಜೀವವಾಗುವುವು. ಆ ಸಮಯದಲ್ಲಿ ಪಾಶ್ಚಾತ್ಯದೇಶದ ಸುಂದರವಾದ ಸಮಾಜ ನಾಯಿಕೆಯರು, ಮತ್ತೊಬ್ಬರನ್ನು ವಶಮಾಡಿಕೊಳ್ಳುವುದರಲ್ಲಿ ನಿಪುಣರಾದರವರು ಅವರಿಗೆ ಕಾಣಿಸಬಹುದೊ ಎಂಬ ಆಲೋಚನೆ ಮನಸ್ಸಿನಲ್ಲಿ ಹೊಳೆಯಿತು. ಆ ಭಾವನೆ ಮನಸ್ಸಿನಲ್ಲಿ ಮೂಡಿದೊಡನೆಯೆ ಸ್ವಾಮೀಜಿ ಹೀಗೆ ಹೇಳತೊಡಗಿದರು: ‘ಈ ಪ್ರಪಂಚದ ಅತ್ಯಂತ ಸ್ಫುರದ್ರೂಪಿಯು ಅಯೋಗ್ಯವಾದ ರೀತಿಯಲ್ಲಿ, ನಾರಿಯರಿಗೆ ಭೂಷಣವಲ್ಲದ ರೀತಿಯಲ್ಲಿ ನನ್ನನ್ನು ನೋಡಿದರೆ ಅವಳೊಂದು ಭಯಂಕರ ಹಳದಿಯ ಕಪ್ಪೆಯಂತೆ ನನಗೆ ತೋರುವಳು. ಅಂತಹ ಕಪ್ಪೆಗಳನ್ನು ಯಾರೂ ಮೆಚ್ಚುವುದಿಲ್ಲ ತಾನೆ?’ ಎಂದರು.” 

 “ಇವತ್ತು ಸ್ವಾಮೀಜಿ ಬೆಳಗಿನ ಪ್ರವಚನವನ್ನು ಮುಗಿಸಿದರು. ‘ಶ‍್ರೀಮತಿ ಫಂಕಿ ನೀನೊಂದು ತಮಾಷೆಯ ಕಥೆಯನ್ನು ಹೇಳು. ನಾವು ಬೇಗ ಒಬ್ಬರನ್ನೊಬ್ಬರು ಅಗಲಿ ಹೋಗುವೆವು. ಆಗ ತಮಾಷೆಯಾಗಿ ಮಾತುಕತೆ ಆಡಬೇಕು ಅಲ್ಲವೆ’. ಅಯ್ಯೊ ಅವರು ಸೋಮವಾರ ನಮ್ಮನ್ನು ಬಿಟ್ಟು ಹೋಗುವರು.” 

 “ನಾವು ಪ್ರತಿದಿನ ಸಾಯಂಕಾಲ ದೂರ ನಡೆದುಕೊಂಡು ಹೋಗುವೆವು. ನಾವು ಸಾಧಾರಣವಾಗಿ ಮನೆಯ ಹಿಂಬದಿಯ ಇಳಿಜಾರಿನಲ್ಲಿ ಹೋಗಿ ನದೀ ಸಮೀಪಕ್ಕೆ ಒಂದು ಹಳೆಯ ರಸ್ತೆಯ ಮೂಲಕ ಹೋಗುವೆವು. ಕೆಲವು ವೇಳೆ ಮಧ್ಯದಲ್ಲಿ ಎಲ್ಲಿಯಾದರೂ ಹಸುರಿನ ಮೇಲೆ ಕುಳಿತುಕೊಂಡು, ಸ್ವಾಮೀಜಿಯವರ ಅಮೊಘವಾದ ವಾಣಿಯನ್ನು ಕೇಳುತ್ತೇವೆ. ಒಂದು ಹಕ್ಕಿಯೋ ಹೂವೋ ಚಿಟ್ಟೆಯೊ ಇದರಿಂದ ಮಾತು ಪ್ರಾರಂಭವಾಗಿ ವೇದಗಳ ಕಥೆಯನ್ನು ಹೇಳುವರು. ಅನಂತರ ಯಾವುದಾದರೂ ಶ್ಲೋಕವನ್ನು ಹೇಳುವರು. ಅವರ ಒಂದು ಶ್ಲೋಕ ಹೀಗೆ ಪ್ರಾರಂಭವಾಗುವುದು: ‘ಅವಳ ಕಣ್ಣುಗಳು ತಾವರೆಯ ಮೇಲೆ ಕುಳಿತ ಕಪ್ಪು ದುಂಬಿಯಂತೆ ಇದ್ದವು’.” 

 “ಕೊನೆಯ ದಿನ ನನಗೆ ಅಮೋಘವಾಗಿತ್ತು. ಪವಿತ್ರವಾಗಿತ್ತು. ಈ ದಿನ ಬೆಳಿಗ್ಗೆ ಪ್ರವಚನ ಇರಲಿಲ್ಲ. ಸ-ಮತ್ತು ನನ್ನನ್ನು ತಮ್ಮ ಜೊತೆಯಲ್ಲಿ ಸಂಚಾರಕ್ಕೆ ಕರೆದುಕೊಂಡು ಹೋದರು. ಅವರು ನಮ್ಮೊಡನೆ ಏನನ್ನೊ ಮಾತನಾಡಬೇಕೆಂದು ಬಯಸಿದರು. ಮನೆಯ ಹಿಂದೆ ಅರ್ಧ ಮೈಲಿ ದೂರದಲ್ಲಿದ್ದ ಬೆಟ್ಟ ಹತ್ತಿ ಹೋದೆವು. ಎಲ್ಲಾ ಕಡೆಯಲ್ಲಿಯೂ ನಿರ್ಜನ ಅರಣ್ಯ. ಕೊನೆಗೆ ಕೆಳಗೆ ಬಾಗಿದ್ದ ಒಂದು ಕೊಂಬೆಯನ್ನು ಆರಿಸಿಕೊಂಡು ಅದರ ಕೆಳಗೆ ಕುಳಿತುಕೊಂಡೆವು. ಅವರೇನೊ ಮಾತನಾಡುತ್ತಾರೆ ಎಂದು ನಾವು ಭಾವಿಸಿದ್ದರೆ ಅವರು ತಕ್ಷಣವೇ ‘ಈಗ ಧ್ಯಾನ ಮಾಡೋಣ. ಆಲದ ಮರದ ಕೆಳಗೆ ಕುಳಿತ ಬುದ್ಧನಂತೆ ಇರೋಣ’ ಎಂದರು. ಅವರು ಧ್ಯಾನಕ್ಕೆ ಕುಳಿತೊಡನೆ ಒಂದು ಕಂಚಿನ ವಿಗ್ರಹದಂತೆ ಆದರು; ಅಷ್ಟು ಸ್ಥಿರವಾಗಿದ್ದರು. ಆಗ ಜೋರಾಗಿ ಗುಡುಗು ಮಿಂಚುಗಳು ಪ್ರಾರಂಭವಾಗಿ ದೊಡ್ಡ ಮಳೆ ಬಂತು. ಅವರು ಅದನ್ನು ಗಮನಿಸಲೇ ಇಲ್ಲ. ನಾನು ಅವರಿಗೆ ಒಂದು ಛತ್ರಿಯನ್ನು ಹಿಡಿದು ಸಾಕಷ್ಟು ಮಳೆ ಬೀಳದಂತೆ ನೋಡಿಕೊಂಡೆ. ಅವರು ಧ್ಯಾನದಲ್ಲೇ ತಲ್ಲೀನರಾಗಿ ಹೋಗಿದ್ದರು. ಎಲ್ಲವನ್ನೂ ಸಂಪೂರ್ಣ ಮರೆತಿದ್ದರು. ಆಗ ಸ್ವಲ್ಪ ದೂರದಿಂದ ಶಬ್ದ ಕೇಳಿ ಬಂತು. ಇತರರು ಮಳೆಯ ಅಂಗಿ ಛತ್ರಿಗಳನ್ನು ತೆದುಕೊಂಡು ನಮ್ಮನ್ನು ಹುಡುಕಿಕೊಂಡು ಬಂದರು. ಸ್ವಾಮೀಜಿ, ಅಯ್ಯೋ ಹೋಗಬೇಕಲ್ಲ ಎಂದು ಎದ್ದರು. ನಾನು ಪುನಃ ಕಲ್ಕತ್ತೆಯಲ್ಲಿ ಮಳೆಗಾಲದಲ್ಲಿರುವಂತೆ ಭಾಸವಾಗುವುದು ಎಂದರು.” 

 “ಆ ಕೊನೆಯ ದಿನವಂತೂ ನಮ್ಮ ಮೇಲೆಲ್ಲ ಅಷ್ಟೊಂದು ಪ್ರೀತಿ ಮತ್ತು ಮಾಧುರ‍್ಯವನ್ನು ತೋರಿಸಿದ್ದರು. ಸ್ಟೀಮರ್ ನದಿಯಲ್ಲಿ ಬಳಸನ್ನು ಸುತ್ತುಹಾಕುತ್ತಿದ್ದಾಗ ಸ್ವಾಮಿಜಿ ಅಲ್ಲಿಂದ ಹುಡುಗನಂತೆ ಸಂತೋಷದಿಂದ ತಮ್ಮ ಟೋಪಿಯನ್ನು ಎತ್ತಿ ನಮ್ಮಿಂದ ಬೀಳ್ಕೊಂಡರು. ಅವರು ನಿಜವಾಗಿಯು ನಮ್ಮನ್ನು ಅಗಲಿ ಹೊರಟು ಹೋದರು.” 

\delimiter

 ಬುಧವಾರ ಜೂನ್ ಹತ್ತೊಂಭತ್ತನೆಯ ತಾರೀಖಿನಿಂದ ಪ್ರವಚನ ಪ್ರಾರಂಭವಾಯಿತು. ಸ್ವಾಮೀಜಿ ಕೈಯಲ್ಲಿ ಬೈಬಲ್ಲನ್ನು ತೆಗೆದುಕೊಂಡು ಬಂದರು. ಅದರಲ್ಲಿ ಸಂತಜಾನ್ ಭಾಗವನ್ನು ಓದಿದರು. ನೀವುಗಳೆಲ್ಲ ಕ್ರೈಸ್ತರಾದುದರಿಂದ ಕ್ರೈಸ್ತರ ಶಾಸ್ತ್ರದಿಂದಲೇ ಪ್ರಾರಂಭ ಮಾಡುವುದು ಒಳ್ಳೆಯದು ಎಂದರು. ಈ ಪ್ರವಚನಗಳು ಆಗಸ್ಟ್ ಆರನೆಯ ತಾರೀಖು ಕೊನೆಗೊಂಡವು. ಇವುಗಳ ಅನುವಾದ ‘ಸ್ಫೂರ್ತಿವಾಣಿ’ ಎಂಬ ಗ್ರಂಥ. ಅವುಗಳಿಂದ ಸ್ವಲ್ಪ ಭಾಗವನ್ನು ಮಾತ್ರ ಕೆಳಗೆ ಕೊಡುವೆವು. ಇದರಿಂದ ಅವರ ಭಾವನೆಯ ಕಿಡಿಗಳ ಬಿಸಿ ನಮಗೆ ಅರಿವಾಗುವುದು:” 

 “ಈ ಪ್ರಪಂಚದಲ್ಲಿ ಯಾವಾಗಲೂ ದಾನಿಯ ಸ್ಥಾನದಲ್ಲಿ ನಿಲ್ಲಿ. ಸಹಾಯ ನೀಡಿ, ಸೇವೆ ಮಾಡಿ. ನಿಮಗೆ ಸಾಧ್ಯವಾದ ಯಾವುದಾದರೂ ಸಣ್ಣಪುಟ್ಟದನ್ನು ಕೊಡಿ. ಆದರೆ ವ್ಯಾಪಾರದ ಭಾವನೆಯಿಂದ ಪಾರಾಗಿ. ಯಾವ ಶರತ್ತನ್ನೂ ಹಾಕಬೇಡಿ. ಯಾವುದನ್ನೂ ಬಲಾತ್ಕರಿಸಬೇಡಿ. ದೇವರು ನಮಗೆ ಕೊಟ್ಟಂತೆ ಔದಾರ‍್ಯದಿಂದ ನಮ್ಮಲ್ಲಿರುವುದನ್ನು ಇತರರಿಗೆ ಕೊಡುವ.” 

 “ಭಗವಂತನೊಬ್ಬನೇ ದಾನಿ. ಪ್ರಪಂಚದಲ್ಲಿರುವವರೆಲ್ಲ ವ್ಯಾಪಾರಿಗಳು. ಅವನಿಂದ ಚೆಕ್ಕು ಬಂದರೆ ಎಲ್ಲರೂ ಅದನ್ನು ಮಾನ್ಯ ಮಾಡಲೇಬೇಕು.” 

 “ಈಶ್ವರ ಅನಿರ್ವಚನೀಯ ಪ್ರೇಮಸ್ವರೂಪ. ಅದನ್ನು ಅನುಭವಿಸಬಹುದು. ಎಂದಿಗೂ ವಿವರಿಸುವುದಕ್ಕೆ ಆಗಲಾರದವನು ಅವನು.” 

 “ನಾವು ದುಃಖ ಮತ್ತು ಹೋರಾಟದಲ್ಲಿದ್ದಾಗ ಪ್ರಪಂಚ ಒಂದು ಭಯಾನಕವಾದ ಸ್ಥಳದಂತೆ ಗೋಚರಿಸುವುದು. ಎರಡು ನಾಯಿಮರಿಗಳು ಆಡುತ್ತ ಕಚ್ಚಾಡುತ್ತಿರುವುದನ್ನು ನೋಡುವಾಗ ನಾವು ಅದಕ್ಕೆ ಅಷ್ಟು ಗಮನವನ್ನು ಕೊಡುವುದಿಲ್ಲ. ಏಕೆಂದದರೆ ಇದು ಬರೀ ಆಟವೆಂದು ನಮಗೆ ಗೊತ್ತಿದೆ. ಕೆಲವು ವೇಳೆ ಒಂದು ಮತ್ತೊಂದನ್ನು ಕಚ್ಚಿದರೂ ಅದರಿಂದ ಯಾವ ಅಪಾಯವೂ ಆಗುವುದಿಲ್ಲವೆಂದು ನಮಗೆ ಗೊತ್ತಿದೆ. ಇದರಂತೆ ನಮ್ಮ ಹೋರಾಟವೆಲ್ಲ ದೇವರ ಕಣ್ಣಿಗೆ ಒಂದು ಆಟದಂತೆ. ಈ ಪ್ರಪಂಚ ಇರುವುದೇ ಆಡುವುದಕ್ಕೆ. ಇದರಿಂದ ದೇವರಿಗೆ ತಮಾಷೆ. ಇಲ್ಲಿರುವುದಾವುದೂ ದೇವರಿಗೆ ಕೋಪವನ್ನು ತರಲಾರದು.” 

 “ಪೋಲೀಸಿನವನು ನಮ್ಮನ್ನು ಅಟ್ಟಿಸಿಕೊಂಡು ಬರುತ್ತಿರುವನೇನೋ ಎಂಬಂತೆ ನಾವು ಪ್ರಪಂಚದಲ್ಲಿ ಓಡುತ್ತಿರುವೆವು. ಪ್ರಪಂಚದಲ್ಲಿರುವ ಸೌಂದರ್ಯದ ಕ್ಷಣಿಕ ನೋಟವೊಂದೇ ನಮ್ಮ ಪಾಲಿಗೆ ದೊರಕುವುದು. ನಮ್ಮನ್ನು ಕಾಡುತ್ತಿರುವ ಅಂಜಿಕೆಯೆಲ್ಲ ಪಂಚಭೂತಗಳನ್ನು ನಂಬುವುದರಿಂದ ಬಂದಿದೆ. ಪಂಚಭೂತಗಳ ಹಿಂದೆ ಇರುವ ಮನಸ್ಸಿನಿಂದ ಮಾತ್ರ ಅದಕ್ಕೆ ಅಸ್ತಿತ್ವ ಬಂದಿದೆ. ನಾವು ನೋಡುತ್ತಿರುವುದೇ ಪ್ರಕೃತಿಯ ಮೂಲಕ ವ್ಯಕ್ತವಾಗುತ್ತಿರುವ ಈಶ್ವರನನ್ನು.” 

 “ಸಾಗರದ ಕಡೆ ನೋಡಿ. ಅಲೆಯ ಕಡೆ ನೋಡಬೇಡಿ. ಒಂದು ಇರುವೆಗೂ ದೇವದೂತನಿಗೂ ಯಾವ ವ್ಯತ್ಯಾಸವನ್ನೂ ಮಾಡಬೇಡಿ. ಪ್ರತಿಯೊಂದು ಕೀಟವೂ ಕ್ರಿಸ್ತನ ಸಹೋದರನೆ. ಹಿಗಿರುವಾಗ ಒಂದನ್ನು ದೊಡ್ಡದು ಮತ್ತೊಂದನ್ನು ಸಣ್ಣದು ಎಂದು ಹೇಗೆ ಹೇಳುವುದು? ಪ್ರತಿಯೊಂದೂ ತನ್ನ ಸ್ಥನದಲ್ಲಿ ದೊಡ್ಡದೇ. ನಾವು ಇಲ್ಲಿರುವಷ್ಟೇ ಸೂರ್ಯನಲ್ಲಿರುವೆವು, ನಕ್ಷತ್ರಗಳಲ್ಲಿರುವೆವು. ಆತ್ಮ ಕಾಲ ದೇಶಾತೀತವಾದುದು, ಸರ್ವವ್ಯಾಪಿ. ಭಗವಂತನನ್ನು ಸ್ತುತಿಸುತ್ತಿರುವ ಪ್ರತಿಯೊಂದು ಬಾಯಿಯೂ ನಾನೇ. ನಾವು ದೇಹವಲ್ಲ, ವಿಶ್ವವೇ ನಮ್ಮ ದೇಹ. ಮಾಯಾದಂಡಗಳನ್ನು ತಿರುಗಿಸುತ್ತ ನಮ್ಮ ಇಚ್ಛೆಯಂತೆ ದೃಶ್ಯಗಳನ್ನು ಸೃಷ್ಟಿಸುವ ಮಂತ್ರವಾದಿಗಳು ನಾವು. ದೊಡ್ಡ ಬಲೆಯಲ್ಲಿರುವ ಜೇಡರ ಹುಳು ನಾವು. ಅಲ್ಲಿ ಯಾವ ತಂತುವಿನ ಮೇಲೆ ಬೇಕಾದರೂ ಹೋಗಬಲ್ಲೆವು. ಆ ಜೇಡರ ಹುಳುವಿಗೆ ಈಗ ತಾನು ಇರುವ ಸ್ಥಳ ಮಾತ್ರ ಗೊತ್ತಿದೆ. ಆದರೆ ಕ್ರಮೇಣ ಬಲೆಯೆಲ್ಲ ಅದಕ್ಕೆ ಗೊತ್ತಾಗುವುದು. ಈಗ ನಮಗೆ ನಮ್ಮ ದೇಹ ಮಾತ್ರ ಗೊತ್ತಿದೆ. ಈಗ ನಾವು ಅತೀಂದ್ರಿಯ ಶಕ್ತಿಯನ್ನು ಪಡೆದರೆ ನಮಗೆ ಎಲ್ಲಾ ಗೊತ್ತಾಗುವುದು. ಎಲ್ಲರ ಮೆದುಳನ್ನೂ ಉಪಯೋಗಿಸಿಕೊಳ್ಳಬಲ್ಲೆವು. ಈಗಲೂ ನಾವು ಜಾಗ್ರತಾವಸ್ಥೆಯಲ್ಲಿ ಇರುವಾಗಲೂ ಅದನ್ನು ಮೇಲಕ್ಕೆ ತಳ್ಳಬಹುದು. ಆದರೆ ಅದು ಅತಿ ಪ್ರಜ್ಞೆಯ ಸ್ಥಳದಿಂದ ಕೆಲಸ ಮಾಡುವುದು.” 

 “ಭಕ್ತಿಯೇ ಪರಮಪ್ರೇಮಸ್ವರೂಪ. ಇದು ಲಭಿಸಿದರೆ ನಿತ್ಯ ತೃಪ್ತನಾಗುವನು. ಏನು ಹೋದರೂ ದುಃಖಪಡುವುದಿಲ್ಲ. ಅಸೂಯೆ ಪಡುವುದಿಲ್ಲ. ಅದನ್ನು ಪಡೆದಮೇಲೆ ಉನ್ಮತ್ತನಾಗುವನು.” 

 “ನನ್ನ ಗುರುದೇವರು ಹೀಗೆ ಹೇಳುತ್ತಿದ್ದರು: ಈ ಪ್ರಪಂಚ ಒಂದು ದೊಡ್ಡ ಹುಚ್ಚರ ಆಸ್ಪತ್ರೆಯಂತೆ. ಇಲ್ಲಿರುವವರೆಲ್ಲ ಹುಚ್ಚರೇ. ಕೆಲವರು ಹೊನ್ನಿಗೆ ಹುಚ್ಚರು, ಕೆಲವರು ಹೆಣ್ಣಿಗೆ ಹುಚ್ಚರು, ಕೆಲವರು ಕೀರ್ತಿಗೆ ಹುಚ್ಚರು, ಕೆಲವರು ಯಶಸ್ಸಿಗೆ ಹುಚ್ಚರು. ಇದರಲ್ಲಿ ಎಲ್ಲೋ ಕೆಲವರು ಮಾತ್ರ ದೇವರಿಗೆ ಹುಚ್ಚರು. ನನಗೆ ದೇವರ ಹುಚ್ಚು ಇಷ್ಟ. ದೇವರು ಸ್ಪರ್ಶಮಣಿ, ನಮ್ಮನ್ನು ಕ್ಷಣಮಾತ್ರದಲ್ಲಿ ಚಿನ್ನವನ್ನಾಗಿ ಮಾಡಬಲ್ಲ. ಬಾಹ್ಯ ಆಕಾರ ಹಾಗೆಯೇ ಇರುವದು. ಒಳಗಿನ ಸ್ವಭಾವ ಮಾತ್ರ ಬದಲಾಯಿಸಿರುವುದು. ಮಾನವಾಕಾರ ಇರುವದು. ಆದರೆ ನಾವು ಇನ್ನು ಮೇಲೆ ಯಾರಿಗೂ ತೊಂದರೆ ಕೊಡುವುದಿಲ್ಲ, ಪಾಪ ಮಾಡುವುದಿಲ್ಲ.” 

 “ಭಗವಂತನನ್ನು ಚಿಂತಿಸುತ್ತಿರುವಾಗ, ಕೆಲವರು ಅಳುತ್ತಾರೆ, ಕೆಲವರು ಹಾಡುತ್ತಾರೆ, ಕೆಲವರು ನಗುತ್ತಾರೆ, ಕೆಲವರು ಕುಣಿಯುತ್ತಾರೆ, ಕೆಲವರು ಅದ್ಭುತ ವಿಷಯಗಳನ್ನು ಮಾತನಾಡುತ್ತಾರೆ. ಆದರೆ ಯಾರೂ ದೇವರ ವಿಷಯವಲ್ಲದೆ ಬೇರೇನನ್ನೂ ಮಾತನಾಡುವುದಿಲ್ಲ.” 

 “ಪ್ರವಾದಿಗಳು ಬೋಧಿಸುತ್ತಾರೆ. ಆದರೆ ಜೀಸಸ್ ಬುದ್ಧ ಶ‍್ರೀರಾಮಕೃಷ್ಣರಂತಹವರು ಧರ್ಮವನ್ನು ಪ್ರತ್ಯಕ್ಷ ಕೊಡಬಲ್ಲರು. ಅವರ ಒಂದು ನೋಟ ಸಾಕು, ಒಂದು ಸ್ಪರ್ಶ ಸಾಕು, ಇದನ್ನು ನೀಡುವುದಕ್ಕೆ.” 

 “ಪ್ರತಿಯೊಂದು ಸುಖವಾದ ಮೇಲೂ ದುಃಖ ಬರುವುದು. ಕೆಲವು ವೇಳೆ ಬೇಗ ಬರುವುದು, ಕೆಲವು ವೇಳೆ ನಿಧಾನವಾಗಿ ಬರುವುದು. ಜೀವ ಮುಂದುವರಿದಷ್ಟೂ ಅವು ಬೇಗ ಬೇಗ ಬರುತ್ತವೆ. ನಮಗೆ ಬೇಕಾಗಿರುವುದು ಸುಖವೂ ಅಲ್ಲ, ದುಃಖವೂ ಅಲ್ಲ. ಇವೆರಡು ನಮ್ಮ ನೈಜಸ್ವಭಾವವನ್ನು ಮರೆಮಾಚುವುವು. ಇವೆರಡೂ ಸರಪಳೆಯೇ. ಒಂದು ಕಬ್ಬಿಣದ್ದು, ಮತ್ತೊಂದು ಚಿನ್ನದ್ದು. ಇವುಗಳ ಹಿಂದೆ ಸುಖ ದುಃಖಗಳಿಗೆ ಅತೀತನಾದ ಆತ್ಮನಿರುವನು.” 

 “ಪ್ರಪಂಚ ನನಗಾಗಿ ಇರುವುದು, ನಾನು ಪ್ರಪಂಚಕ್ಕೆ ಅಲ್ಲ. ಪಾಪ ಪುಣ್ಯಗಳು ಎರಡೂ ನಮ್ಮ ಸೇವಕರು. ನಾವು ಅವರ ಸೇವಕರಲ್ಲ. ಇದ್ದಕಡೆಯೇ ಇರುವುದು ಮೂಢನ ಸ್ಥಿತಿ. ಮನುಷ್ಯನ ಸ್ವಭಾವವೇ ಪುಣ್ಯವನ್ನು ಅರಸಿ ಪಾಪದಿಂದ ಪಾರಾಗುವುದು. ದೇವನ ಸ್ವಭಾವವಾದರೊ ಅವನು ಯಾವುದನ್ನೂ ಅರಸುವುದಿಲ್ಲ. ನಿತ್ಯ ಧನ್ಯನಾಗಿರುವನು. ನಾವು ದೇವರಾಗೋಣ.” 

 “ಅನಂತಾಕಾಶವೇ ಭಗವಂತನ ಧೂಪದಾನಿ. ಸೂರ್ಯಚಂದ್ರರು ಹಣತೆಗಳಂತೆ. ಮತ್ತಾವ ದೇವಾಲಯ ನಮಗೆ ಬೇಕು?” 

 “ಜೀವನದಲ್ಲಿ ಯಾವುದನ್ನೂ ಆಶಿಸಲೂ ಬೇಡಿ, ತ್ಯಜಿಸಲೂ ಬೇಡಿ. ಬಂದದನ್ನು ಸ್ವೀಕರಿಸಿ. ಯಾವುದರ ಪ್ರಭಾವಕ್ಕೂ ಬೀಳದೆ ಇರುವುದೇ ಸ್ವಾತಂತ್ರ್ಯ.” 

 “ನಾಮರೂಪಗಳಲ್ಲಿ ಎಂದಿಗೂ ಸ್ವಾತಂತ್ರ್ಯವಿಲ್ಲ. ಈ ನಾಮರೂಪದ ಜೇಡಿಮಣ್ಣಿನಿಂದಲೇ ನಮ್ಮಂತಹ ಮಡಿಕೆ ಕುಡಿಕೆಗಳು ಆಗಿರುವುದು. ಆದಕಾರಣ ಇದಕ್ಕೆ ಒಂದು ಪರಿಮಿತಿ ಇದೆ. ಇದು ಸ್ವತಂತ್ರವಾಗಲಾರದು. ಸಾಪೇಕ್ಷ ವಸ್ತುವಿಗೆ ಎಂದಿಗೂ ಸ್ವಾತಂತ್ರ್ಯವಿಲ್ಲ. ಒಂದು ಮಡಿಕೆ, ಮಡಿಕೆ ಆಗಿರುವಾಗ ನಾನು ಸ್ವತಂತ್ರ ಎಂದು ಹೇಳಲಾರದು. ಅದು ತನ್ನ ನಾಮರೂಪಗಳನ್ನು ಕಳೆದುಕೊಂಡಮೇಲೆ ಮಾತ್ರ ಸ್ವತಂತ್ರವಾಗುವುದು. ಪ್ರಪಂಚವೆಲ್ಲ ನಾನಾಕಾರಗಳನ್ನು ಧರಿಸಿರುವ ಆತ್ಮನೆ. 

 “ವಾಸ್ತವಿಕವಾಗಿ ಕೆಟ್ಟ ಆಲೋಚನೆಗಳೇ ಸಾಂಕ್ರಾಮಿಕ ಕ್ರಿಮಿಗಳು.” 

 “ಪ್ರಪಂಚದಲ್ಲಿರುವ ಒಳ್ಳೆಯ ಆಲೋಚನೆಗೆಲ್ಲ ನಾವು ಹಕ್ಕುದಾರರು, ಅದನ್ನು ನಾವು ಸ್ವೀಕರಿಸಲು ಯೋಗ್ಯವಾದಾಗ ಮಾತ್ರ.” 

 “ಎಲ್ಲಾ ಜ್ಞಾನದ ಮೂಲವೂ ನಮ್ಮಲ್ಲಿ ಪ್ರತಿಯೊಬ್ಬರಲ್ಲಿಯೂ ಇರುವುದು.” 

 “ನಾವೇ ಹಾಳಾಗಿಹೋಗಿರುವೆವು. ಪ್ರಪಂಚ ಹಾಳಾಗಿದೆ ಎಂದು ಭಾವಿಸುವೆವು.” 

 “ಅಹಂಕಾರದ ಭಾವನೆ ಇಲ್ಲದೆ ಇರುವಾಗಲೇ ನಾವು ಅತ್ಯಂತ ಶ್ರೇಷ್ಠ ಕೆಲಸವನ್ನು ಮಾಡುವೆವು, ನಮ್ಮ ಅತ್ಯಂತ ಶ್ರೇಷ್ಠ ಪ್ರಭಾವವನ್ನು ಬೀರುವೆವು. ಎಲ್ಲಾ ಮಹಾಪುರುಷರಿಗೂ ಇದು ಗೊತ್ತು. ಲೀಲಾನಾಟಕ ಸೂತ್ರಧಾರನ ಪ್ರಭಾವಕ್ಕೆ ನಮ್ಮ ಹೃದಯವನ್ನು ತೆರೆಯೋಣ. ಅವನೇ ಕೆಲಸ ಮಾಡಲಿ. ನಾನು ಏನನ್ನೂ ಮಾಡಬೇಕಾಗಿಲ್ಲ. 

 “ಅಹಂಕಾರವನ್ನು ಕಂಡುಹಿಡಿದು ಅದನ್ನು ಕಿತ್ತೊಗೆಯಿರಿ. ಅದನ್ನು ಮರೆಯಿರಿ, ದೇವರು ಕೆಲಸಮಾಡಲಿ. ಇದೆಲ್ಲ ಅವನಿಗೆ ಸಂಬಂಧಪಟ್ಟದ್ದು. ನಾವು ಬದಿಗೆ ಸರಿದು ದೇವರು ಕೆಲಸ ಮಾಡುವುದನ್ನು ನೋಡಬೇಕು. ನಮಗೆ ಇನ್ನೇನೂ ಕೆಲಸವಿಲ್ಲ. ನಾನು ಎಂಬುದು ಎಷ್ಟು ಹೋಗುವುದೋ ದೇವರು ಅಷ್ಟು ಬರುವನು, ಅಲ್ಪ ಅಹಂಕಾರದಿಂದಾಗಿ ಪಾರಾಗಿ. ಭೂಮದ ನಾನು ಮಾತ್ರ ಇರಲಿ.” 

 “ನಾವು ನಮ್ಮ ಆಲೋಚನೆಗಳು ಮಾಡಿದಂತೆ ಇರುವೆವು. ಆದಕಾರಣ ನೀವು ಏನನ್ನು ಆಲೋಚಿಸುವಿರೋ ಅದರ ವಿಷಯವಾಗಿ ಬಹಳ ಜೋಪಾನವಾಗಿರಿ. ಪದಗಳು ಗೌಣ. ಆಲೋಚನೆಯೆ ಸಜೀವವಾಗಿರುವುದು. ಅದೇ ಬಹಳ ದೂರದವರೆಗೂ ಚಲಿಸುವುದು. ನಾವು ಮಾಡುವ ಪ್ರತಿಯೊಂದು ಆಲೋಚನೆಯೂ ನಮ್ಮ ಶೀಲದ ಸ್ವಭಾವವನ್ನು ತಾಳುವುದು. ಆದಕಾರಣವೇ ಪವಿತ್ರನಾದ ಶುದ್ಧಾ ತ್ಮ ವ್ಯಕ್ತಪಡಿಸುವ ಹಾಸ್ಯ ಮತ್ತು ನಿಂದೆಯ ಹಿಂದೆಯೂ ಅವನ ಪ್ರೀತಿ ಮತ್ತು ಪವಿತ್ರತೆ ಇರುವುದು. ಅದು ನಮ್ಮ ಕಲ್ಯಾಣಕ್ಕೆ ಕಾರಣವಾಗುವುದು.” 

\delimiter

 ಸ್ವಾಮೀಜಿ ಸಹಸ್ರ ದ್ವೀಪೋದ್ಯಾನದಲ್ಲಿ ಪ್ರವಚನಗಳನ್ನು ಪೂರೈಸಿ ಅಲ್ಲಿಂದ ಗ್ರೀನ್‍ಏಕರ್ ಸಮ್ಮೇಳನಕ್ಕೆ ಕೆಲವು ಉಪನ್ಯಾಸಗಳನ್ನು ಕೊಡಲು ಹೋದರು. ಅಲ್ಲಿಂದ ನ್ಯೂಯಾರ್ಕಿಗೆ ಬಂದರು. 

 (ಸ್ವಾಮೀಜಿ ಸಹಸ್ರ ದ್ವೀಪ ಉದ್ಯಾನದಲ್ಲಿ ವಾಸವಾಗಿದ್ದ ಮನೆಯನ್ನು ಈಗ ನ್ಯೂಯಾರ್ಕ್ ಕೇಂದ್ರದವರು ತೆಗೆದುಕೊಂಡು ಅದನ್ನು ಸ್ವಾಮೀಜಿಯರ ಜ್ಞಾಪಕಾರ್ಥವಾಗಿ ಒಂದು ಆಶ್ರಮವನ್ನಾಗಿ ಮಾಡಿರುವರು.) 

