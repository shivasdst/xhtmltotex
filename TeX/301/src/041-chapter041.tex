
\chapter{ಅಮರನಾಥ  }

 ಸ್ವಾಮೀಜಿ ಕಾಶ್ಮೀರದ ಕಡೆಗೆ ಹೋಗಬೇಕೆಂದು ೧೧ನೇ ತಾರೀಖು ಆಲ್ಮೋರವನ್ನು ಬಿಟ್ಟರು. ದಾರಿಯಲ್ಲಿ ಕೆಲವು ಬೆಟ್ಟಗಳನ್ನು ತೋರಿಸಿ ಶಿಷ್ಯರಿಗೆ ಇದೇ ಹಿಂದೆ ಕಿಂಪುರುಷರು ವಾಸಿಸುತ್ತಿದ್ದ ಸ್ಥಳ ಎಂದು ಹೇಳಿದರು. ಸ್ವಾಮೀಜಿ ಇಂತಹ ಭ್ರಾಮ್ಯಕ ದೃಶ್ಯಗಳನ್ನು ಹಿಂದೆ ತಾವು ಈ ಸ್ಥಳದಲ್ಲಿ ಕಂಡಿದ್ದೇನೆಂದು ಹೇಳಿದರು. ೧೨ನೇ ತಾರೀಖು ಸುಂದರವಾದ ಭೀಮ ಸರೋವರದ ತೀರದಲ್ಲಿ ತಂಗಿದ್ದರು. ಮಾರನೆ ದಿನ ಅಲ್ಲಿಂದ ರಾವಲ್ಪಿಂಡಿಗೆ ಹೋದರು. ರಾವಲ್ಪಿಂಡಿಯಿಂದ ಮುರ‍್ರಿಗೆ ಟಾಂಗಾದಲ್ಲಿ ಹೋದರು. ಅಲ್ಲಿ ಮೂರು ದಿನಗಳಿದ್ದು ಅಲ್ಲಿಂದ ಬಾರಾಮುಲ್ಲಕ್ಕೆ ಹೋದರು. 

 ಸ್ವಾಮೀಜಿಯವರಿಗೆ ಮೊದಲಿನಿಂದಲೂ ಶಿವನ ಮೇಲೆ ಪ್ರೀತಿ. ಈಗ ಶಿವನ ತೌರುಮನೆಯಂತಿರುವ ಹಿಮಾಲಯದಲ್ಲಿರುವಾಗಲಂತೂ ಅವನ ಭಾವನೆಯಲ್ಲಿಯೇ ಮುಳುಗಿಹೋಗಿದ್ದರು. ಶಿವಯೋಗಿಗಳ ಆದರ್ಶ. ಹಿಮಾಲಯದಲ್ಲಿ ಧುಮುಕಿಕೊಂಡು ಹರಿಯುತ್ತಿರುವ ಗಿರಿಝರಿಗಳು ‘ಬೋಂಮ್, ಬೋಂಮ್, ಹರ, ಹರ’ ಎಂಬ ನಿನಾದವನ್ನು ಮಾಡಿಕೊಂಡು ಹರಿಯುವಂತೆ ತೋರುತ್ತವೆ. 

 ಬಾರಾಮುಲ್ಲದಿಂದ ಮೂರು ಮನೆಯಂತಿರುವ ದೋಣಿಗಳನ್ನು ಮಾಡಿಕೊಂಡು ಜೀಲಂ ನದಿಯ ಮೇಲೆ ಶ‍್ರೀನಗರಕ್ಕೆ ಹೊರಟರು. ಹೋಗುವಾಗ ದಾರಿಯಲ್ಲೆ ಸಿಕ್ಕುವ ಒಂದು ಹಳ್ಳಿಯ ಮಹಮ್ಮದೀಯನ ಮನೆಗೆ ಸ್ವಾಮೀಜಿ ತಮ್ಮ ಶಿಷ್ಯರನ್ನು ಕರೆದುಕೊಂಡು ಹೋದರು. ಒಂದು ಗುಡಿಸಿಲಿನ ಮುಂದೆ ಕಡುಗೆಂಪಿನ ಲಂಗ ತೊಟ್ಟು ಬಿಳಿಯ ಮುಸುಕನ್ನು ಹಾಕಿಕೊಂಡಿದ್ದ ನೋಡಲು ಸುಂದರವಾದ ಮಹಮ್ಮದೀಯ ರೈತರ ಮುದುಕಿಯೊಬ್ಬಳು ಕುಳಿತಿದ್ದಳು. ಆಕೆ ಜೀನಾರ್ ಮರದ ಕೆಳಗೆ ಕುಳಿತುಕೊಂಡು ಚರಕದಿಂದ ನೂಲನ್ನು ತೆಗೆಯುತ್ತಿದ್ದಳು. ಸುತ್ತಲೂ ಅವಳ ಸೊಸೆಯರು ಕುಳಿತಿದ್ದರು. ಆ ದಾರಿಯಲ್ಲಿ ಹೋಗುತ್ತಿದ್ದಾಗ ಆಕೆಯನ್ನು ನೋಡಲು ಕೆಳಗಿಳಿದರು. ಸ್ವಾಮೀಜಿ ಎರಡನೆ ವೇಳೆ ಆಕೆಯನ್ನು ನೋಡಲು ಹೋಗುತ್ತಿರುವುದು. ಕಳೆದ ವರ್ಷ ವಿಶ್ವಾಸದಿಂದ ಆಕೆ ಸ್ವಾಮಿಗಳನ್ನು ಕಂಡು ಮಾತನಾಡಿದ್ದಳು. ತಾಯಿ ನೀವು ಯಾವ ಧರ್ಮಕ್ಕೆ ಸೇರಿದವರು ಎಂದು ಸ್ವಾಮೀಜಿ ಅವಳನ್ನು ಕೇಳಿದ್ದರು. ಈ ಮಾತನ್ನು ಕೇಳಿದೊಡನೆಯೆ ಸಂತೋಷ ಮತ್ತು ಹೆಮ್ಮೆಯಿಂದ ಆಕೆಯ ಮುಖ ಅರಳಿತು. ಸ್ಪಷ್ಟವಾಗಿ ಗಟ್ಟಿಯಾಗಿ “ದೇವರ ದಯೆಯಿಂದ ನಾನು ಮುಸಲ್ಮಾನಳು” ಎಂದು ಹೇಳಿದಳು. ಈ ಘಟನೆಯನ್ನು ಸ್ವಾಮೀಜಿ ಎಷ್ಟೋ‌ ವೇಳೆ ಸಂಭಾಷಣೆಯಲ್ಲಿ ಮತ್ತು ಉಪನ್ಯಾಸದಲ್ಲಿ ಹೇಳುತ್ತಿದ್ದರು. ಈಗ ಸ್ವಾಮೀಜಿ ತಮ್ಮ ಶಿಷ್ಯರನ್ನು ಆಕೆಗೆ ಪರಿಚಯ ಮಾಡಿಸಿದಾಗ ಅವರನ್ನೆಲ್ಲ ಆದರದಿಂದ ಬರಮಾಡಿಕೊಂಡು ಅತಿಥಿ ಸತ್ಕಾರವನ್ನು ಮಾಡಿದಳು. 

 ಸ್ವಾಮೀಜಿ ಜೂನ್ ೨೧ನೇ ತಾರೀಖು ಶ‍್ರೀನಗರವನ್ನು ಸೇರಿದರು. ಅಲ್ಲಿ ದೋಣಿಯ ಮನೆಯಲ್ಲಿಯೇ ಜೂನ್ ೨೫ರವರೆಗೂ ಇದ್ದರು. ನದೀ ತೀರದಲ್ಲಿರುವ ಸುತ್ತಮುತ್ತಲ ಹಳ್ಳಿಗಳಿಗೆ ಪ್ರತಿದಿನ ಸ್ವಾಮೀಜಿ ಶಿಷ್ಯರನ್ನು ಕರೆದುಕೊಂಡು ಹೋಗುತ್ತಿದ್ದರು. ಹಿಂದೂಧರ್ಮ ಮತ್ತು ಸಂಸ್ಕೃತಿಗೆ ಸಂಬಂಧಪಟ್ಟ ಹಲವು ಮಾತುಕತೆಗಳನ್ನು ಆಡುತ್ತಿದ್ದರು. ಆಗಲೆ ಒಮ್ಮೆ ಸೋದರಿ ನಿವೇದಿತಾ, ಕಲ್ಕತ್ತೆಯಲ್ಲಿರುವ ಕಾಳಿಘಾಟಿನ ವಿಗ್ರಹದ ಮುಂದೆ ಜನ ಏತಕ್ಕೆ ನೆಲವನ್ನು ಮುತ್ತಿಡುತ್ತಾರೆ ಎಂದು ಕೇಳಿದಳು. ಸ್ವಾಮೀಜಿ ಈ ಪರ್ವತ ಶಿಖರಗಳೆದುರು ಬಾಗಿ ನೆಲವನ್ನು ಮುತ್ತಿಡುವುದು, ದೇವರ ವಿಗ್ರಹದ ಮುಂದೆ ನೆಲವನ್ನು ಮುತ್ತಿಡುವುದು, ಎರಡೂ ಒಂದೇ ಅಲ್ಲವೆ? ಎಂದರು. ಎರಡೂ ಒಬ್ಬನಲ್ಲಿ ವಿಸ್ಮಯ ಮತ್ತು ಭಕ್ತಿಯ ಭಾವನೆಯನ್ನು ಉದ್ದೀಪನೆ ಗೊಳಿಸುವುವು. 

 ಸ್ವಾಮೀಜಿ ಅನೇಕ ವೇಳೆ ಇದ್ದಕ್ಕೆ ಇದ್ದಂತೆಯೇ ಕೆಲವು ದಿನಗಳು ಹೊರಟು ಹೋಗಿಬಿಡುತ್ತಿದ್ದರು. ಅನಿರೀಕ್ಷಿತವಾಗಿ ಪುನಃ ಬರುತ್ತಿದ್ದರು. ಇತರರೊಂದಿಗೆ ಇರುವುದು ಒಂದು ಯಾತನೆ ಎಂಬಂತೆ ಅವರು ಅನುಭವಿಸುತ್ತಿದ್ದರು. ಅವರ ಪ್ರಖ್ಯಾತಿಯನ್ನು ಕೇಳಿ ಅವರ ದರ್ಶನಾರ್ಥವಾಗಿ ಬರುವ ಜನರನ್ನು ನೋಡಿದಾಗ ಅವರಿಗೆ ಬೇಜಾರಾಗುತ್ತಿತ್ತು. ದರ್ಶನಾರ್ಥಿಗಳು ಅವರು ದೋಣಿಯ ಒಳಗೆ ಬಂದು ಅವರೆದುರಿಗೆ ಸುಮ್ಮನೆ ಬಹಳ ಕಾಲ ಕುಳಿತುಕೊಳ್ಳುತ್ತಿದ್ದರು. ಸ್ವಲ್ಪವೂ ಅವರ ಏಕಾಂತಕ್ಕೆ ಅವಕಾಶವೇ ಇರುತ್ತಿರಲಿಲ್ಲ. ಪ್ರಿಯತಮ ತನ್ನ ಪ್ರೇಯಸಿಯನ್ನು ನಿರೀಕ್ಷಿಸುವಂತೆ ಮೌನವಾಗಿ ಅನಾಮಧೇಯರಾಗಿ ಗೈರಿಕವಸನಧಾರಿಯಾಗಿ ಅಲೆಯುತ್ತಿರುವ ಅಥವಾ ಗೋಪ್ಯವಾಗಿರುವ ಜೀವನವನ್ನು ಮನನಮಾಡುತ್ತಿದ್ದರು. 

 ಸ್ವಾಮೀಜಿ ಕೆಲವು ಕಾಲ ನಿರ್ಜನಪ್ರದೇಶದಲ್ಲಿ ಧ್ಯಾನಮಾಡಿ ಬಂದಾಗ ಅಲೌಕಿಕ ತೇಜಸ್ಸು ಅವರ ಮುಖದಲ್ಲಿ ವ್ಯಕ್ತವಾಗುತ್ತಿತ್ತು. ಆ ಸಮಯದಲ್ಲಿ ಬಹಳ ಗಾಢವಾದ ಭಾವನೆಗಳನ್ನು ತಮ್ಮ ಮಾತಿನಲ್ಲಿ ವ್ಯಕ್ತಪಡಿಸುತ್ತಿದ್ದರು; ಅವೇ ಇಲ್ಲಿ ಬರುವ ಕೆಲವು: 

 “ದೇಹವನ್ನು ಕುರಿತು ಚಿಂತಿಸುವುದೂ ಒಂದು ಪಾಪವೆ.” 

 “ಸಿದ್ಧಿಯನ್ನು ಪ್ರದರ್ಶಿಸುವುದು ಒಂದು ತಪ್ಪು.” 

 “ಪರಿಸ್ಥಿತಿ ಏನೂ ಉತ್ತಮಗೊಳ್ಳುವುದಿಲ್ಲ. ಅವು ಎಂದಿನಂತೆಯೇ ಇರುವುವು. ಅದರಲ್ಲಿ ಮಾಡಿಕೊಳ್ಳುವ ಬದಲಾವಣೆಯಿಂದ ನಾವು ಉತ್ತಮರಾಗುವೆವು.” 

 “ಯೋಜನೆ, ಯೋಜನೆ, ಪಾಶ್ಚಾತ್ಯರಾದ ನೀವು ಯಾವ ಧರ್ಮವನ್ನೂ ಸೃಷ್ಟಿಸಲಾರಿರಿ. ನಿಮ್ಮಲ್ಲಿ ಯಾರಾದರೂ ಅದನ್ನು ಮಾಡಿದ್ದರೆ ಅದೇ ಯಾವ ಯೋಜನೆಯೂ ಇಲ್ಲದ ಕ್ಯಾಥೋಲಿಕ್ ಸಾಧುಗಳು. ಯೋಜನಾಬದ್ಧ ಜೀವಿಗಳೆಂದಿಗೂ ಧರ್ಮವನ್ನು ಸಾಧಿಸಲಾರರು.” 

 “ಪಾಶ್ಚಾತ್ಯರದು ಯಾವಾಗಲೂ ಗೋಳುಮುಖವೇ. ನೀವು ದುಃಖವನ್ನು ಆರಾಧಿಸುತ್ತೀರಿ. ನಿಮ್ಮ ದೇಶದಲ್ಲೆಲ್ಲ ನಾನು ಇದನ್ನು ನೋಡಿದೆ. ಪಾಶ್ಚಾತ್ಯ ಸಮಾಜದಲ್ಲಿ ಹೊರಗೆ ನಗುವೋ ನಗು. ಆದರೆ ಒಳಗೆ ಗೋಳು. ಎಲ್ಲಾ ದುಃಖದಲ್ಲಿ ಪರ‍್ಯವಸಾನವಾಗುವುದು. ತಮಾಷೆ ಹಾಸ್ಯ ಎಲ್ಲ ಹೊರಗೆ, ಆದರೆ ನಿಜವಾಗಿಯೂ ದುಃಖ ಬೇಯುತ್ತಿರುವುದು. ಆದರೆ ಭರತಖಂಡದಲ್ಲಿ ಹೊರಗೆ ವ್ಯಸನ, ಮಂಕು ಕವಿದಿದೆ. ಆದರೆ ಒಳಗೆ ಆನಂದ, ಸ್ವಚ್ಛಂದ.” 

 ಸ್ವಾಮೀಜಿ ರಮ್ಯ ನೈಸರ್ಗಿಕ ದೃಶ್ಯಗಳ ಮಧ್ಯೆ ಇದ್ದ ಹಿಂದೂ ದೇವಾಲಯಗಳಿಗೆ ಶಿಷ್ಯರನ್ನು ಕರೆದುಕೊಂಡು ಹೋಗಿ, ಹಿಂದೂ ಹೇಗೆ ದೇವಸ್ಥಾನವನ್ನು ಕಟ್ಟುವುದಕ್ಕೆ ಸುಂದರವಾದ ಸ್ಥಳವನ್ನು ಆರಿಸುತ್ತಾನೆ ಎಂದು ಹೇಳಿದರು. ಅದು ಬೆಟ್ಟದ ಮೇಲೆ ಇರಬಹುದು, ಸರೋವರದ ಮಧ್ಯದಲ್ಲಿ, ನದೀ ತೀರದಲ್ಲಿ, ಎಲ್ಲಾ ಕಡೆಯಲ್ಲಿಯೂ ಹಿಂದೂ ನೈಸರ್ಗಿಕ ಸೌಂದರ‍್ಯವನ್ನು ಭಗವಂತನ ಕಡೆಗೆ ಒಯ್ಯುವ ಮಹಾದ್ವಾರಗಳಂತೆ ಪ್ರೀತಿಸುವನು. 

 ಸ್ವಾಮೀಜಿ ಇಲ್ಲಿ ಇದ್ದಾಗಲೆ ಜುಲೈ ನಾಲ್ಕನೇ ತಾರೀಖು ಬಂದಿತು. ಅಂದು ಅಮೇರಿಕಾ ದೇಶದವರಿಗೆ ದೊಡ್ಡ ಹಬ್ಬ, ಸ್ವಾತಂತ್ರ್ಯವನ್ನು ಗಳಿಸಿದ ದಿನ. ಸ್ವಾಮೀಜಿ ಶಿಷ್ಯರಲ್ಲಿ ಒಬ್ಬ ಮಹಿಳೆ ವಿನಃ ಉಳಿದವರೆಲ್ಲ ಅಮೇರಿಕಾ ದೇಶಕ್ಕೆ ಸೇರಿದವರು. ಅಮೇರಿಕಾ ಮಹಿಳೆಯರಿಗೆ ಗೊತ್ತಾಗದಂತೆ ಒಬ್ಬ ದರ್ಜಿಯವರ ಕೈಯಲ್ಲಿ ಅಮೇರಿಕಾ ದೇಶದ ಧ್ವಜವನ್ನು ಮಾಡಿಸಿ ನಾಲ್ಕನೇ ಜುಲೈ ದಿನ ಆ ಧ್ವಜವನ್ನು ದೋಣಿಯಲ್ಲಿ ಕಟ್ಟಿಸಿದರು. ಹೂವು ಎಲೆ ಕೊಂಬೆಗಳಿಂದ ದೋಣಿಯನ್ನು ಶೃಂಗಾರ ಮಾಡಿದರು. ಇಂಗ್ಲೀಷಿನಲ್ಲಿ ನಾಲ್ಕನೇ ಜುಲೈಗೆ ಎಂದು ಬರೆದ ಪದ್ಯವನ್ನು ಓದಿದರು. ಅದರಲ್ಲಿ ಸ್ವಾತಂತ್ರ್ಯೋತ್ಸವವನ್ನು ಜೀವಿಯ ಮುಕ್ತಿಯ ಚಿಹ್ನೆಯಾಗಿ ತೆಗೆದುಕೊಂಡು ಬಣ್ಣಿಸಿದ್ದರು. 

 ಸ್ವಾಮೀಜಿ ಒಮ್ಮೆ ಜನಕನ ವಿಷಯವಾಗಿ ಹೀಗೆ ಹೇಳಿದರು: “ಜನಕನಂತೆ ಆಗುವುದು ಸುಲಭವೇ? ಸಿಂಹಾಸನದ ಮೇಲೆ ಕುಳಿತಿರುವಾಗಲೂ ಅನಾಸಕ್ತರಾಗಿರುವುದು ಸಾಧ್ಯವೆ? ಐಶ್ವರ್ಯ ಕೀರ್ತಿ ಹೆಂಡತಿ ಮಕ್ಕಳು ಇವುಗಳನ್ನು ಗಣಿಸದೆ ಇರುವುದು ಸಾಧ್ಯವೆ? ಪಾಶ್ಚಾತ್ಯ ದೇಶದಲ್ಲಿ ಹಲವರು ಇಂತಹ ಸ್ಥಿತಿಯನ್ನು ಮುಟ್ಟಿರುವೆನು ಎಂದು ಹೇಳಿದ್ದರು; ಅದಕ್ಕೆ ನಾನು ಇಂತಹ ಮಹಾಪುರುಷರು ಇಂಡಿಯಾ ದೇಶದಲ್ಲಿ ಇಲ್ಲ ಎಂದು ಮಾತ್ರ ಹೇಳುತ್ತಿದ್ದೆ. ನಿಮಗೆ ಮತ್ತು ನಿಮ್ಮ ಮಕ್ಕಳಿಗೆ ಹೀಗೆ ಹೇಳಿಕೊಳ್ಳುವುದನ್ನು ಮರೆಯಬೇಡಿ: ಒಂದು ಮಿಂಚಿನ ಹುಳುವಿಗೂ ಸೂರ್ಯನಿಗೂ ಯಾವ ವ್ಯತ್ಯಾಸ ಇರುವುದೊ, ಒಂದು ಹಳ್ಳಕ್ಕೂ ಅನಂತ ಸಾಗರಕ್ಕೂ ಯಾವ ವ್ಯತ್ಯಾಸ ಇರುವುದೊ, ಸಾಸಿವೆಕಾಳಿಗೂ ಮತ್ತು ಮೇರು ಪರ್ವತಕ್ಕೂ ಯಾವ ವ್ಯತ್ಯಾಸ ಇರುವುದೋ ಅದೇ ಗೃಹಸ್ಥನಿಗೂ ಸಂನ್ಯಾಸಿಗೂ ಇರುವ ವ್ಯತ್ಯಾಸ.”‌ 

 ತಮ್ಮ ಆದರ್ಶದಿಂದ ಚ್ಯುತರಾದ ಅಥವಾ ಕಪಟಿಗಳಾದ ಸಂನ್ಯಾಸಿಗಳ ಮುಂದೆ ಕೂಡ ಒಂದು ಮಹದಾದರ್ಶವಿದೆ. ಹಲವರ ಸೋಲು ಕೆಲವರ ಜಯಕ್ಕೆ ಕಾರಣವಾಗಿದೆ. ಇದನ್ನು ಮರೆಯಕೂಡದು ಎನ್ನುತ್ತಿದ್ದರು. 

 ಹದಿನೆಂಟನೇ ತಾರೀಖು ಸ್ವಾಮೀಜಿಯವರ ವೃಂದ ಇಸ್ಲಾಮಾಬಾದಿಗೆ ಹೋಯಿತು. ಮಾರನೆ ದಿನ ಜೀಲಂ ನದೀ ತೀರದ ಒಂದು ಕಾಡಿನಲ್ಲಿ ಪಾಚೀ ನೀರಿನಲ್ಲಿ ಅರ್ಧ ಮುಳುಗಿಹೋಗಿದ್ದ ಪಂಡರಿನಾಥ ಎಂಬ ದೇವಸ್ಥಾನವನ್ನು ಕಂಡುಹಿಡಿದರು. ಬಹುಶಃ ಈ ದೇವಸ್ಥಾನ ಕಾನಿಷ್ಕನ ಕಾಲದಿಂದಲೂ (ಕ್ರಿ. ಶ. ೧೫೦) ಇರಬಹುದು ಎಂದರು. ಸ್ವಾಮೀಜಿ ಒಳಗೆ ಕೊರೆದಿರುವ ಚಿತ್ರಗಳು ಮತ್ತು ಅಲಂಕಾರಗಳನ್ನು ಇತರರಿಗೆ ತೋರಿದರು. ಅಲ್ಲಿ ಬುದ್ಧನ ವಿಗ್ರಹ ಮತ್ತು ಮಾಯಾದೇವಿಯ ವಿಗ್ರಹಗಳು ಇದ್ದವು. 

 ಇಸ್ಲಾಮಾಬಾದಿನಲ್ಲಿದ್ದಾಗ ಒಂದು ಸಲ ಸ್ವಾಮೀಜಿ ಹೀಗೆ ಮಾತನಾಡಿದರು: “ಮೃತ್ಯು ನನ್ನನ್ನು ಸಮೀಪಿಸಿದಾಗಲೆಲ್ಲ ನನ್ನ ದೌರ್ಬಲ್ಯ ಮಾಯವಾಗುವುದು. ನನಗೆ ಅಂಜಿಕೆಯೂ ಇಲ್ಲ, ಅನುಮಾನವೂ‌ ಇಲ್ಲ. ನಾನು ಬಾಹ್ಯಪ್ರಪಂಚವನ್ನು ಲೆಕ್ಕಿಸುವುದಿಲ್ಲ. ಆಗ ನಾನು ಸುಮ್ಮನೆ ಸಾಯುವುದಕ್ಕೆ ಅಣಿಯಾಗುವೆ. (ಕೈಯಲ್ಲಿ ಎರಡು ಕಲ್ಲುಗಳನ್ನು ತೆಗೆದುಕೊಂಡು ಒಂದನ್ನು ಮತ್ತೊಂದರ ಮೇಲೆ ಕುಟ್ಟುತ್ತಾ) ನಾನು ಇದರಂತೆ ಗಟ್ಟಿಯಾಗಿರುವೆನು. ಏಕೆಂದರೆ ನಾನು ಭಗವಂತನ ಪಾದಪದ್ಮಗಳನ್ನು ಸ್ಪರ್ಶಿಸಿರುವೆನು.” 

 ೨೩ನೇ ತಾರೀಖು ಮಾರ್ತಾಂಡ ದೇವಾಲಯದ ಅವಶೇಷಗಳನ್ನು ನೋಡಿದರು. ೨೫ನೇ ತಾರೀಖು ಸುಂದರವಾದ ಮಾರ್ಗದ ಮೂಲಕ ಅಚ್‍ಬಲ್ ಸೇರಿದರು. ಅಲ್ಲಿಯೇ ಸ್ವಾಮೀಜಿ ಅಮರನಾಥಕ್ಕೆ ಹೋಗುತ್ತಿದ್ದ ಎರಡು ಮೂರು ಸಾವಿರ ಯಾತ್ರಿಕರೊಡನೆ ತಾವೂ ಅಮರನಾಥಕ್ಕೆ ಹೋಗುವೆ ಎಂದರು. ಸೋದರಿ ನಿವೇದಿತಾಳನ್ನು ಮಾತ್ರ ತಮ್ಮ ಜೊತೆಯಲ್ಲಿ ಕರೆದುಕೊಂಡು ಹೋಗಲು ಒಪ್ಪಿದರು. ಪಹಿಲ್‍ಗಾಮಿನವರೆಗೆ ಉಳಿದವರೆಲ್ಲ ಸ್ವಾಮೀಜಿ ಜೊತೆಯಲ್ಲಿ ಹೋಗಿ ಅಮರನಾಥದಿಂದ ಹಿಂತಿರುಗಿ ಬರುವವರೆಗೆ ಅಲ್ಲೆ ಇರುವುದಾಗಿ ನಿರ್ಧಾರವಾಯಿತು. ಜೊತೆಯಲ್ಲಿ ಹೋದ ಸೋದರಿ ನಿವೇದಿತೆಯ ವರ್ಣನೆಯಿಂದಲೇ ಅವರ ಅಮರನಾಥ ಯಾತ್ರೆಯನ್ನು ವಿವರಿಸುತ್ತೇವೆ. 

 “ಆ ಕಾಲದಲ್ಲಿ ಕಾಶ್ಮೀರ ಯಾತ್ರಾರ್ಥಿಗಳಿಂದ ತುಂಬಿ ತುಳುಕಾಡುವಂತೆ ಇತ್ತು. ಸ್ವಾಮೀಜಿಯವರ ಯಾತ್ರೆಗೆ ಅಣಿಮಾಡಲು, ಅಚ್‍ಬಲ್ ಬಿಟ್ಟು ಇಸ್ಲಾಮಾಬಾದಿಗೆ ಬಂದರು. ಎಲ್ಲೆಲ್ಲಿಯೂ ಸಂಚರಿಸುತ್ತಿರುವ ಯಾತ್ರಿಕರ ತಂಡವೇ ಕಾಣಿಸುತ್ತಿತ್ತು. ಇವರಲ್ಲಿ ಯಾವ ಸದ್ದುಗದ್ದಲವೂ ಇರಲಿಲ್ಲ. ಒಂದು ಕ್ರಮವಿತ್ತು. ನೋಡುವುದಕ್ಕೆ ದೃಶ್ಯ ಅಂದವಾಗಿ ಕಾಣಿತ್ತಿತ್ತು. ಎರಡು ಮೂರು ಸಾವಿರ ಜನ ರಾತ್ರಿ ಒಂದು ಬದಿಯಲ್ಲಿ ತಂಗುವರು; ಬೆಳಿಗ್ಗೆ ಅಷ್ಟು ಹೊತ್ತಿಗೇ ಎದ್ದು ಹೊರಡುತ್ತಿದ್ದರು. ನೆನ್ನೆ ಅವರು ಅಲ್ಲಿ ತಂಗಿದ್ದರು ಎಂಬುದಕ್ಕೆ ಅವರು ಬಿಟ್ಟ ಒಲೆ ಬೂದಿಯಲ್ಲದೆ ಮತ್ತೇನೂ ಅಲ್ಲಿ ಕಾಣುತ್ತಿರಲಿಲ್ಲ. ಅವರ ಜೊತೆಯಲ್ಲಿ ಒಂದು ಅಂಗಡಿಬೀದಿಯೇ ಹೋಗುತ್ತಿತ್ತು. ತಂಗುವ ಕಡೆಯಲ್ಲೆ ಡೇರೆಯನ್ನು ಹೂಡಿ ಅಂಗಡಿಯನ್ನು ಅಷ್ಟೇ ಶೀಘ್ರದಲ್ಲಿ ತೆಗೆದುಬಿಡುತ್ತಿದ್ದರು. ಇಂತಹ ಒಂದು ವ್ಯವಸ್ಥೆ ಅವರೊಡನೆ ಜನ್ಮತಃ ಬಂದಂತೆ ಇತ್ತು. ಶಿಬಿರದ ಮಧ್ಯಭಾಗದಲ್ಲಿ ದೊಡ್ಡದೊಂದು ರೋಡು ಇರುತ್ತಿತ್ತು. ಪಕ್ಕದಲ್ಲಿದ್ದ ಅಂಗಡಿಯಲ್ಲಿ ಒಣಗಿದ ಹಣ್ಣು ಹಾಲು ಬೇಳೆ ಅಕ್ಕಿ ಮೊದಲಾದುವುಗಳನ್ನು ಕೊಳ್ಳಬಹುದಾಗಿತ್ತು. ಯಾವುದಾದರೊಂದು ಎತ್ತರವಾದ ಕಡೆ ತಹಶೀಲ್‍ದಾರ್ ಮತ್ತು ಅವರ ಎಡಬಲಗಳಲ್ಲಿ ನನ್ನ (ಸೋದರಿ ನಿವೇದಿತಾ) ಮತ್ತು ಸ್ವಾಮೀಜಿ ಡೇರೆಯನ್ನು ಹೂಡುತ್ತಿದ್ದರು. ಇಲ್ಲಿ ಜನರು ನೆರೆಯಲು ಮತ್ತು ಮಾತುಕತೆಯಾಡಲು ಸೇರುತ್ತಿದ್ದರು.” 

 “ಎಲ್ಲಾ ಪಂಥಗಳಿಗೆ ಸೇರಿದ ಸಾವಿರಾರು ಸಾಧುಗಳು ಇದ್ದರು. ಎಲ್ಲರಿಗೂ ಗೈರಿಕವಸನದ ಡೇರೆ. ಕೆಲವರ ಡೇರೆಯಂತೂ ಛತ್ರಿಗಿಂತ ಸ್ವಲ್ಪ ದೊಡ್ಡದಾಗಿರುತ್ತಿತ್ತು. ಸಾಧುಗಳ ಮೇಲೆ ಸಾಮೀಜಿ ಪ್ರಭಾವ ಅದ್ಭುತವಾಗಿತ್ತು. ಇವರಲ್ಲಿ ಹೆಚ್ಚು ಬುದ್ಧಿವಂತರಾದವರು ಪ್ರತಿಯೊಂದು ತಂಗುವ ಸ್ಥಳದಲ್ಲಿಯೂ ಸ್ವಾಮೀಜಿ ಡೇರೆಗೆ ಹೋಗುತ್ತಿದ್ದರು. ಅನಂತರ ಸ್ವಾಮೀಜಿ ಹೇಳಿದರು, ಅವರು ಮಾತನಾಡುತ್ತಿದ್ದುದೆಲ್ಲ ಶಿವನಿಗೆ ಸಂಬಂಧಪಟ್ಟದ್ದು ಎಂದು. ಸ್ವಾಮೀಜಿ ವಿಶಾಲ ಜಗತ್ತನ್ನು ನೋಡಿ ತಮ್ಮ ದೃಷ್ಟಿಯನ್ನು ವಿಶಾಲಗೊಳಿಸಬೇಕೆಂದಾಗ ಸಾಧುಗಳು ಒಪ್ಪುತ್ತಿರಲಿಲ್ಲ. ಪರದೇಶಿಯೂ ಮನುಷ್ಯರೇ ಎನ್ನುತ್ತಿದ್ದರು ಸ್ವಾಮೀಜಿ. ಆದರೆ ಸ್ವದೇಶೀ ಎಂಬ ಭಾವನೆ ಏಕೆ? ಅವರಲ್ಲಿ ಹಲವರು ಸ್ವಾಮೀಜಿಗೆ ಮಹಮ್ಮದೀಯರ ಮೇಲೆ ಇದ್ದ ವಿಪರೀತ ವಿಶ್ವಾಸವನ್ನು ಮೆಚ್ಚುತ್ತಿರಲಿಲ್ಲ. ಹಿಂದೂಗಳು ಮತ್ತು ಮಹಮ್ಮದೀಯರು ಬೇರೆ ಬೇರೆ ಎಂದು ಭಾವಿಸುತ್ತಿದ್ದುದೇ ಹೊರತು ಇವರಿಬ್ಬರೂ ಒಂದಾಗಬಲ್ಲರು ಎಂಬುದು ಅವರ ಭಾವನೆಗೂ ಸಿಲುಕದ ವಸ್ತುವಾಗಿತ್ತು. ಪಂಜಾಬು ಹಿಂದೂ ಧರ್ಮಕ್ಕಾಗಿ ಸತ್ತವರ ನೆತ್ತರಿಂದ ಒಣಗಿದೆ. ಇಲ್ಲಿಯಾದರೂ ಸ್ವಲ್ಪ ಮತಾಂಧತೆಯನ್ನು ಅಭ್ಯಾಸಮಾಡಬೇಕೆಂಬ ಅವರ ಇಚ್ಛೆ ಮುಂದೆ ಅದರ ಸುಳಿವಿಲ್ಲದೇ ಹೊರಟುಹೋಗುವುದು ಎಂಬುದನ್ನು ಅರಿತ ಸ್ವಾಮೀಜಿ ತತ್ಕಾಲಕ್ಕೆ ಕೆಲವು ರಿಯಾಯಿತಿಗಳನ್ನು ತೋರಿದರು. ಅವರಿಗೆ ತಮ್ಮ ಸಹೋದರರ ಮೇಲೆ ಇದ್ದ ವಿಶ್ವಾಸವನ್ನು ಇದು ತೋರುವುದು. ಇದರಿಂದ ತಮ್ಮ ಭಾವನೆಯನ್ನು ಹೆಚ್ಚು ಪ್ರಭಾವಕಾರಿಯಾಗಿ ಇತರರಿಗೆ ತಿಳಿಸಲು ಸಾಧ್ಯವಾಯಿತು… ನಮ್ಮ ಜೊತೆಯಲ್ಲಿ ಬರುತ್ತಿದ್ದ ತಹಶೀಲ್‍ದಾರ್ ಮತ್ತು ಹಲವು ಉದ್ಯೋಗಸ್ಥರೆಲ್ಲ ಮಹಮ್ಮದೀಯರಾಗಿದ್ದರು. ಹಿಂದೂ ಯಾತ್ರಿಕರೊಡನೆ ಇವರು ಕೊನೆಗೆ ಶಿವನ ಗುಹೆಯೊಳಗೆ ಪ್ರವೇಶಮಾಡಿದರೆ ಯಾರೂ ಅವರನ್ನು ವಿರೋಧಿಸುತ್ತಿರಲಿಲ್ಲ. ಕೊನೆಗೆ ತಹಶೀಲ್‍ದಾರರು ಕೆಲವು ಸ್ನೇಹಿತರೊಡನೆ ಸ್ವಾಮೀಜಿಯವರ ಬಳಿಗೆ ಬಂದು ತಮ್ಮನ್ನು ಶಿಷ್ಯರನ್ನಾಗಿ ಸ್ವೀಕರಿಸಬೇಕೆಂದು ಕೇಳಿಕೊಂಡರು. ಯಾರಿಗೂ ಇದರಲ್ಲಿ ಯಾವ ಅಸಮಂಜಸವಾಗಲಿ ವಿಚಿತ್ರವಾಗಲಿ ತೋರಲಿಲ್ಲ.” 

 “ನಾವು ಇಸ್ಲಾಮಾಬಾದನ್ನು ಬಿಟ್ಟಮೇಲೆ ಯಾತ್ರಿಕರನ್ನು ಭವನ್ ಎಂಬ ಕಡೆ ಪುನಃ ಸಂಧಿಸಿದೆವು. ಇದು ಹಲವು ತೀರ್ಥಗಳಿಗೆ ಪ್ರಖ್ಯಾತವಾದ ಸ್ಥಳ. ರಾತ್ರಿ ಆ ಕೊಳದ ಶುಭ್ರವಾದ ನೀರಿನಲ್ಲಿ ಪ್ರತಿಬಿಂಬಿಸುತ್ತಿದ್ದ ದೀಪದ ಕಾಂತಿ, ದೇವಸ್ಥಾನದಿಂದ ದೇವಸ್ಥಾನಕ್ಕೆ ಗುಂಪುಗುಂಪಾಗಿ ಹೋಗುತ್ತಿದ್ದ ಯಾತ್ರಿಕರ ತಂಡ ಇನ್ನೂ ಕಣ್ಣಿಗೆ ಕಟ್ಟಿದಂತೆ ಇದೆ.” 

 “ಪಹಿಲ್‍ಗಾಮ್ ಎಂಬುದು ಕುರುಬರ ಒಂದು ಗ್ರಾಮ. ಅಂದು ಏಕಾದಶಿ ಆಗಿದ್ದುದರಿಂದ ಯಾತ್ರಿಕರು ಅಲ್ಲೇ ತಂಗಿದ್ದರು. ಇದೊಂದು ಸುಂದರವಾದ ಬೆಟ್ಟದ ಕೊರಕಲು ಪ್ರದೇಶದಲ್ಲಿತ್ತು. ಮೇಲಿನ ಬೆಟ್ಟದಿಂದ ನದಿಯೊಂದು ಹರಿದು ಬರುತ್ತಿತ್ತು. ಅದರಲ್ಲಿ ಮಧ್ಯೆ ಮಧ್ಯೆ ಮರಳ ದಿಬ್ಬಗಳ ದ್ವೀಪಗಳು. ನದಿ ಕೆಳಗಾದರೊ ಮೇಲಿಂದ ಹರಿದುಬಂದ ನುಣುಪಾದ ಕಲ್ಲಿನ ರಾಶಿ. ಅದರ ತೀರದಲ್ಲಿದ್ದ ಬೆಟ್ಟವಂತು ಪೈನ್ ಮರಗಳಿಂದ ತುಂಬಿಹೋಗಿತ್ತು. ಸೂರ‍್ಯಾಸ್ತಮಯದಸಮಯದಲ್ಲಿ ಬೆಟ್ಟದ ಮೇಲೆ ನೋಡಿದಾಗ ತ್ರಯೋದಶಿಯ ಚಂದ್ರ ಶೋಭಿಸುತ್ತಿದ್ದ. ನಾರ್ವೆ ಮತ್ತು ಸ್ವಿಟ್ಸರ್‍ಲೆಂಡ್ ದೇಶಗಳ ಅತಿ ಸುಂದರವಾದ ಪ್ರಶಾಂತವಾದ ದೃಶ್ಯದಂತೆ ಇತ್ತು ಇದು. ಇಲ್ಲಿಯೇ ಮನುಷ್ಯರ ಕೊನೆಯ ಮನೆಗಳು, ಸೇತುವೆ, ಒಂದು ತೋಟದ ಮನೆ, ಸುತ್ತಲೂ ಉತ್ತ ಹೊಲ ಮತ್ತು ಅಲ್ಲೊಂದು ಇಲ್ಲೊಂದು ಗುಡಿಸಲುಗಳು ಇದ್ದುವು.”

 “ಅವರ್ಣನೀಯವಾದ ನಿಸರ್ಗ ಸೌಂದರ‍್ಯದಲ್ಲಿ ಮೂರು ಸಾವಿರ ಯಾತ್ರಿಕರು ತಮ್ಮ ಮುಂದಿರುವ ಕಣಿವೆಯಲ್ಲಿ ಮುಂದುವರಿದರು. ಮೊದಲನೆಯ ದಿನ ನಾವು ಪೈನ್ ಮರಗಳ ನಡುವೆ ಶಿಬಿರ ಮಾಡಿದೆವು. ಎರಡನೆ ದಿನ ಹಿಮಪಂಕ್ತಿಗಳನ್ನು ದಾಟಿ ಒಂದು ಘನೀಭೂತವಾದ ನದೀ ತೀರದಲ್ಲಿ ಡೇರೆಯನ್ನು ಹೂಡಿದೆವು. ಅಂದು ರಾತ್ರಿ ಶಿಬಿರ ಧೂನಿಯನ್ನು ಜ್ಯೂನಿಪರ್ ಗಿಡಗಳಿಂದ ಮಾಡಿದರು. ಮಾರನೆ ದಿನ ಇನ್ನೂ ಮೇಲೆ ಶಿಬಿರವಾಯಿತು. ಅಂದು ಆಳುಗಳು ಆ ಗಿಡಗಳನ್ನು ಹುಡುಕಿಕೊಂಡು ತುಂಬಾ ಅಲೆದಾಡಿ ಕೊನೆಗೆ ಸ್ವಲ್ಪವನ್ನು ತಂದು ಬೆಂಕಿ ಹಚ್ಚಿದರು. ಮುಂದೆ ರಸ್ತೆ ಕೊನೆಗೊಂಡಿತು. ಕಾಡು ದಾರಿಯಲ್ಲಿ ಕಡಿದಾದ ಸ್ಥಳದಲ್ಲಿ ಸ್ವಲ್ಪ ದೂರ ನಡೆದುಕೊಂಡು ಹೋದಮೇಲೆ ಗುಹೆ ಇರುವ ಒಂದು ಕಣಿವೆಯೊಳಗೆ ಬಂದೆವು. ನಾವು ಇಲ್ಲಿ ಮುಂದುವರಿದು ಹೋದಂತೆ ಬೆಟ್ಟಗಳು ಹೊಸದಾಗಿ ಬಿದ್ದ ಮಂಜಿನಿಂದ ಸಿಂಗರಿಸಿಕೊಂಡಂತೆ ಕಂಡವು.” 

 “ಅನಂತರ ದೊಡ್ಡದೊಂದು ಗುಹೆ ಕಾಣಿಸಿಕೊಂಡಿತು. ಅದೇ ಅಮರನಾಥ ಗುಹೆ. ೧೩,೦೦೦ ಅಡಿಗಳ ಎತ್ತರದಲ್ಲಿದೆ. ಗುಹೆ ಐವತ್ತು ಅಡಿ ಎತ್ತರ ಐವತ್ತೈದು ಅಡಿ ಅಗಲವಿದೆ. ಒಳಗೆ ಸುಮಾರು ಐವತ್ತು ಅಡಿಗಳಷ್ಟು ಉದ್ದವಿದೆ. ಗುಹೆಯ ಹೊರಗಡೆ ಅದರ ಪಕ್ಕದಲ್ಲಿಯೇ ಸಣ್ಣ ಒಂದು ಹಿಮ ಝರಿ ಮೇಲಿಂದ ಹರಿದು ಬರುತ್ತಿದೆ. ಅದನ್ನು ಅಮರಗಂಗಾ ಎನ್ನುವರು. ಯಾತ್ರಿಕರು ಅಲ್ಲಿ ಸ್ನಾನಮಾಡಿ ಗುಹೆ ಒಳಗೆ ಪ್ರವೇಶಿಸುವರು. ಗುಹೆಯ ಒಂದು ಮೂಲೆಯಲ್ಲಿ ಶಿವಲಿಂಗದ ಆಕಾರದ ದೊಡ್ಡ ಹಿಮರಾಶಿ ಬಿದ್ದಿದೆ. ಇದನ್ನೇ ಅಮರನಾಥ ಶಿವ ಎನ್ನುವರು. ಇದು ಪ್ರಕೃತಿ ನಿರ್ಮಿತ, ದೇವಾಲಯದ ಗುಹೆಯನ್ನು ಯಾರೂ ಕೊರೆದಿಲ್ಲ. ಪ್ರಕೃತಿಯೇ ತನ್ನ ಚಳಿಗಾಳಿ ಬಿಸಿಲುಮಳೆಯೆಂಬ ಆಯುಧದಿಂದ ಇದನ್ನು ಬಿಡಿಸಿದೆ. ಇಲ್ಲಿರುವ ಲಿಂಗ ಕೂಡ ಪ್ರಕೃತಿ ನಿರ್ಮಿತ. ಮನುಷ್ಯನ ಯಾವ ಸುತ್ತಿಗೆಯ ಪೆಟ್ಟೂ ಇದಕ್ಕೆ ಬಿದ್ದಿಲ್ಲ.” 

 “ಯಾತ್ರಿಕರು ಏನು ಮಾಡಬೇಕೊ ಅದನ್ನೆಲ್ಲ ಸ್ವಾಮೀಜಿ ನೆರವೇರಿಸಿದರು. ಜಪಮಾಡಿದರು, ಉಪವಾಸವಿದ್ದರು, ಹತ್ತಿರ ಹರಿಯುತ್ತಿದ್ದ ಹಿಮ ನದಿಯಲ್ಲಿ ಸ್ನಾನ ಮಾಡಿದರು. ಗುಹೆಯನ್ನು ಪ್ರವೇಶ ಮಾಡಿದ ಮೇಲೆ ಶಿವನೇ ಅವರಿಗೆ ಕಂಡಂತೆ ಇತ್ತು. ಯಾತ್ರಿಕರ ಗುಜುಗುಂಪು, ಮೇಲೆ ಹಾರಾಡುತ್ತಿದ್ದ ಪಾರಿವಾಳಗಳು, ಇವುಗಳ ನಡುವೆ ಸ್ವಾಮೀಜಿ ಶಿವನಿಗೆ ನಮಿಸಿದರು. ಅನಂತರ ಎಲ್ಲಿ ಭಾವಪರವಶನಾಗುವೆನೋ ಎಂದು ಅಂಜಿ ಎದ್ದು ಮೌನವಾಗಿ ಹೊರಟು ಬಂದರು. ಆ ಅಲ್ಪ ಸಮಯದಲ್ಲಿ ಶಿವ ಇವರಿಗೆ ಇಚ್ಛಾಮರಣವನ್ನು ಅನುಗ್ರಹಿಸಿದ ಎಂದು ಹೇಳುತ್ತಿದ್ದರು. ಅಂದು ಹುಣ್ಣಿಮೆ. ಯಾತ್ರೆಯ ಕೊನೆಯ ದಿನ. ಕೆಲವು ಕಾಲದವರೆಗೆ ಸ್ವಾಮೀಜಿ ಕೇವಲ ಶಿವನಿಗೆ ಸಂಬಂಧಪಟ್ಟ ಮಾತುಕತೆಯಲ್ಲದೆ ಬೇರೆ ಮಾತನಾಡುತ್ತಲೇ ಇರಲಿಲ್ಲ, ಶಿವಭಾವ ಅವರಲ್ಲಿ ಓತಪ್ರೋತವಾಗಿತ್ತು.” 

