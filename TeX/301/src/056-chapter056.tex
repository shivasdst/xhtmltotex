
\chapter{ಜೀವನ ಸಂಧ್ಯೆಯ ಕೆಲವು ಚಿತ್ರಗಳು}

 ಸ್ವಾಮೀಜಿಯವರ ಆರೋಗ್ಯ ಚೆನ್ನಾಗಿಲ್ಲ. ಸ್ವಾಮಿ ನಿರಂಜನಾನಂದರ ಒತ್ತಾಯದ ಪ್ರಾರ್ಥನೆಯಿಂದ ಆರೇಳು ದಿನಗಳಿಂದಲೂ ಕವಿರಾಜರ ಔಷಧಿಯನ್ನು ಸ್ವಾಮಿಗಳು ಸೇವಿಸುತ್ತಿದ್ದಾರೆ. ಈ ಚಿಕಿತ್ಸೆಯ ಪ್ರಕಾರ ನೀರು ಕುಡಿಯುವುದು ಪೂರ್ಣ ನಿಷಿದ್ಧ. ಹಾಲು ಕುಡಿದೇ ಅವರ ಬಾಯಾರಿಕೆ ಇಂಗಬೇಕು. 

 ಶಿಷ್ಯ ಮಠಕ್ಕೆ ಇಂದು ಬಹು ಬೇಗ ಬೆಳಿಗ್ಗೆ ಹೊತ್ತಿಗೆ ಮುಂಚೆ ಬಂದಿದ್ದ. ಅವನನ್ನು ನೋಡಿ ಪ್ರೇಮದಿಂದ ಸ್ವಾಮಿಗಳು “ಓ!‌ ನೀನು ಬಂದಿರುವೆಯಾ? ಒಳ್ಳೆಯದಾಯಿತು. ನಾನು ನಿನ್ನನ್ನೇ ಕುರಿತು ಆಲೋಚಿಸುತ್ತಿದ್ದೆ” ಎಂದರು. 

 ಶಿಷ್ಯ: “ಸುಮಾರು ಆರೇಳು ದಿನಗಳಿಂದಲೂ ನೀವು ಹಾಲಿನಲ್ಲೇ ಇರುವಿರೆಂದು ಕೇಳಿದೆ.” 

 ಸ್ವಾಮೀಜಿ: “ಹೌದು, ನಿರಂಜನನ ತೀವ್ರ ಒತ್ತಾಯದಿಂದ ನಾನೀ ಚಿಕಿತ್ಸೆಗೆ ಒಪ್ಪಬೇಕಾಯಿತು. ನಾನು ಅವರ ಪ್ರಾರ್ಥನೆಗೆ ಕಿವಿಗೊಡದಿರಲಾರೆ.” 

 ಶಿಷ್ಯ: “ನೀವು ನೀರನ್ನು ಬಹಳ ಸಾರಿ ಕುಡಿಯುತ್ತಿದ್ದಿರಲ್ಲ. ನೀವು ಹೇಗೆ ಈಗ ಅದನ್ನು ಸಂಪೂರ್ಣವಾಗಿ ಬಿಡಲು ಸಾಧ್ಯವಾಯಿತು?” 

 ಸ್ವಾಮೀಜಿ: “ನನಗೆ ಯಾವಾಗ ಈ ಚಿಕಿತ್ಸೆಯಿಂದ ನೀರು ಕುಡಿಯುವುದನ್ನು ಸಂಪೂರ್ಣ ಬಿಡಬೇಕೆಂದು ಗೊತ್ತಾಯಿತೋ ಆಗಲೇ ದೃಢಮನಸ್ಸಿನಿಂದ ಕುಡಿಯಬಾರದೆಂದು ನಿರ್ಧರಿಸಿದೆ. ಈಗ ಕುಡಿಯಬೇಕೆಂಬ ಯೋಚನೆ ಒಮ್ಮೆಯೂ ಮನಸ್ಸಿಗೆ ಬರುವುದಿಲ್ಲ.” 

 ಶಿಷ್ಯ: “ಈ ಚಿಕಿತ್ಸೆಯಿಂದ ನಿಮಗೆ ಗುಣವಾಗುತ್ತಿರಬೇಕಲ್ಲವೆ?” 

 ಸ್ವಾಮೀಜಿ: “ನಾನರಿಯೆ, ಕೇವಲ ನನ್ನ ಗುರುಭಾಯಿಗಳ ಅಪ್ಪಣೆಯನ್ನು ನಾನು ಪಾಲಿಸುತ್ತಿದ್ದೇನೆ.” 

 ಶಿಷ್ಯ: “ ಈ ವೈದ್ಯರು ಉಪಯೋಗಿಸುವ ಹಸಿರು ಔಷಧಿಗಳು ನಿಮ್ಮ ದೇಹ ಪ್ರಕೃತಿಗೆ ಚೆನ್ನಾಗಿ ಒಗ್ಗುವುದೆಂದು ನಾನು ಕೇಳಿದ್ದೇನೆ.” 

 ಸ್ವಾಮೀಜಿ: “ಇದರಲ್ಲಿ ನನ್ನ ಭಾವನೆ ಏನೆಂದರೆ, ಪ್ರವೀಣನಾದ ವೈದ್ಯನ ಕೈಯಲ್ಲಿ ಸತ್ತಾದರೂ ಸಾಯಬಹುದು. ಎಲ್ಲೋ ಕೆಲವು ಪುರಾತನ ಪುಸ್ತಕಗಳನ್ನು ಓದಿ, ಆ ವಿಷಯದಲ್ಲಿ ಸಂಪೂರ್ಣ ತರಬೇತಾಗದೆ ಎಲ್ಲೋ ಕೆಲವು ರೋಗಗಳನ್ನು ವಾಸಿಮಾಡಿರುವ ಒಬ್ಬ ಪ್ರಾಪಂಚಿಕ ವ್ಯಕ್ತಿಯ ಕೈಯಲ್ಲಿ ರೋಗ ವಾಸಿಯಾಗುವುದನ್ನು ನಿರೀಕ್ಷಿಸುವುದಕ್ಕಿಂತ ಒಬ್ಬ ನಿಷ್ಣಾತ ವೈದ್ಯನ ಕೈಯಲ್ಲಿ ಸಾಯುವುದೇ ಮೇಲು.” 

 ಸ್ವಾಮೀಜಿ ಕೆಲವು ಭಕ್ಷ್ಯಗಳನ್ನು ತಯಾರಿಸಿದ್ದರು. ಅದರಲ್ಲಿ ಒಂದು ಶಾಮಿಗೆಯಿಂದ ಮಾಡಿದ್ದಾಗಿತ್ತು. ಅದನ್ನು ತಿಂದ ಶಿಷ್ಯ ಅದೇನೆಂದು ಕೇಳಿದಾಗ ಅವು ಲಂಡನ್ನಿನಿಂದ ತಂದ ಕೆಲವು ಒಣಗಿಸಿದ ಮಣ್ಣು ಹುಳುಗಳು ಎಂದರು. ಎಲ್ಲರೂ ನಗತೊಡಗಿದರು. ಶಿಷ್ಯ ಕಕ್ಕಾಬಿಕ್ಕಿಯಾದ. ಸ್ವಾಮೀಜಿ ಅಷ್ಟು ಮಿತಾಹಾರಿಯಾಗಿ ಕಡಿಮೆ ನಿದ್ದೆ ಮಾಡುತ್ತಿದ್ದರೂ ಕೂಡ ಬಹಳ ಲವಲವಿಕೆಯಿಂದಿದ್ದಾರೆ, ಕೆಲವು ದಿನಗಳ ಹಿಂದೆ ಒಂದು ಹೊಸ ಎನ್‌ಸೈಕ್ಲೋಪೀಡಿಯಾ ಸಂಪುಟಗಳನ್ನು ಮಠಕ್ಕೆ ತರಿಸಿದ್ದರು. ಹೊಳೆಯುತ್ತಿದ್ದ ಆ ಪುಸ್ತಕಗಳನ್ನು ನೋಡಿ ಶಿಷ್ಯ “ಇಡೀ ಜೀವನವೆಲ್ಲಾ ಓದಿದರೂ ಈ ಸಂಪುಟಗಳು ಮುಗಿಯುವ ಮಟ್ಟಿಗೆ ಕಾಣೆ” ಎಂದು ಸ್ವಾಮೀಜಿಗೆ ಹೇಳಿದ. ಅವನಿಗೆ ಸ್ವಾಮೀಜಿ ಅಷ್ಟು ಹೊತ್ತಿಗಾಗಲೇ ಹತ್ತು ಸಂಪುಟಗಳನ್ನು ಮುಗಿಸಿ ಹನ್ನೊಂದನೆಯದನ್ನಾಗಲೇ ಓದಲು ಪ್ರಾರಂಭಿಸಿದ್ದಾರೆಂದು ಗೊತ್ತಿರಲಿಲ್ಲ. 

 ಸ್ವಾಮೀಜಿ: “ನೀನು ಏನು ಹೇಳುವೆ? ಈ ಹತ್ತು ಸಂಪುಟಗಳಲ್ಲಿ ಯಾವುದನ್ನು ಬೇಕಾದರೂ ಕೇಳು, ನಾನೆಲ್ಲಕ್ಕೂ ಉತ್ತರ ಹೇಳುವೆ.” 

 ಶಿಷ್ಯ: (ಆಶ್ಚರ್ಯದಿಂದ) “ನೀವು ಈ ಪುಸ್ತಕಗಳನ್ನೆಲ್ಲಾ ಓದಿರುವಿರಾ?” 

 ಸ್ವಾಮೀಜಿ: “ಮತ್ತೆ, ಇಲ್ಲದಿದ್ದಲ್ಲಿ ನಿನಗೇಕೆ ಪ್ರಶ್ನೆ ಕೇಳೆಂದು ಹೇಳುತ್ತಿದ್ದೆ?” 

 ಪ್ರಶ್ನೆ ಕೇಳಿದಾಗ ಸ್ವಾಮೀಜಿ ಕೇಳಿದ್ದಕ್ಕೆಲ್ಲ ಉತ್ತರ ಹೇಳಿದ್ದಲ್ಲದೆ ಕ್ಲಿಷ್ಟ ವಿಷಯಗಳಿಗೆ ಸಂಬಂಧಿಸಿದ ಕೆಲವು ಭಾಗಗಳನ್ನು ಪುಸ್ತಕದಲ್ಲಿದ್ದಂತೆಯೆ ಪುನರುಚ್ಚರಿಸಿದರು. ಅವಾಕ್ಕಾದ ಶಿಷ್ಯ ಪುಸ್ತಕಗಳನ್ನು ಒತ್ತಟ್ಟಿಗಿಟ್ಟು ಹೇಳಿದ: “ಇದು ಖಂಡಿತ ಮಾನುಷ ಶಕ್ತಿ ಅಲ್ಲ.” 

 ಸ್ವಾಮೀಜಿ: “ಗೊತ್ತಾಯಿತೆ? ತೀವ್ರ ಬ್ರಹ್ಮಚರ್ಯಾಭ್ಯಾಸದಿಂದ ವಿದ್ಯೆಯನ್ನೆಲ್ಲಾ ಬಹು ಸ್ವಲ್ಪ ಕಾಲದಲ್ಲೇ ಕಲಿತು ಪೂರ್ಣಜ್ಞಾನ ಪಡೆಯಬಹುದು. ಒಮ್ಮೆ ಓದಿದ ಅಥವಾ ಯಾರಿಂದಲಾದರೂ ಒಮ್ಮೆ ಕೇಳಿದುದನ್ನು ಮರೆಯದೆ ಇರುವಂಥಾ ಜ್ಞಾಪಕಶಕ್ತಿ ಅದರಿಂದ ಬರುವುದು. ನಮ್ಮ ದೇಶದಲ್ಲಿ ಬ್ರಹ್ಮಚರ‍್ಯದ ಅಭಾವವಿರುವುದರಿಂದಲೇ ಈ ದೇಶ ಇಷ್ಟೊಂದು ಅಧೋಗತಿಗಿಳಿದಿದೆ.” 

 ಶಿಷ್ಯ: “ನೀವೇನು ಬೇಕಾದರೂ ಹೇಳಿ, ಇಂಥ ಅಮಾನುಷಿಕ ಶಕ್ತಿ ಕೇವಲ ಬ್ರಹ್ಮಚರ‍್ಯಾಭ್ಯಾಸ ಮಾಡಿದ ಮಾತ್ರದಿಂದ ಬರುವ ಫಲವಲ್ಲ. ಬೇರೇನೋ ಇದ್ದೇ ಇರಬೇಕು.” 

 ಸ್ವಾಮೀಜಿ ಅದಕ್ಕೆ ಉತ್ತರವನ್ನೇನೂ ಕೊಡಲಿಲ್ಲ. 

 ಅನಂತರ ಸ್ವಾಮೀಜಿ ಎಲ್ಲಾ ಬಗೆಯ ತತ್ತ್ವಗಳಲ್ಲೂ ಇರುವ ಬಗೆಬಗೆಯ ಕ್ಲಿಷ್ಟ ವಿಷಯಗಳನ್ನು ಕುರಿತು ದೀರ್ಘವಾಗಿ ಉಜ್ವಲವಾಗಿ ಚರ್ಚಿಸತೊಡಗಿದರು.\break ಸಂಭಾಷಣೆಯಾಗುತ್ತಿರುವಾಗ ಸ್ವಾಮಿ ಬ್ರಹ್ಮಾನಂದರು ಆ ಕೊಠಡಿಗೆ ಪ್ರವೇಶಿಸಿ, ಶಿಷ್ಯನಿಗೆ ಹೇಳಿದರು: “ನೀನೊಳ್ಳೆ ಹುಡುಗನಯ್ಯ! ಸ್ವಾಮೀಜಿಗೆ ಮೈಸರಿಯಿಲ್ಲ, ಏನಾದರೂ ತಮಾಷೆಯಾಗಿ ಮಾತನಾಡಿ ಅವರನ್ನು ನಗಿಸುವುದನ್ನು ಬಿಟ್ಟು ಅತ್ಯಂತ ಕಠಿಣವಾದ ಈ ವಿಷಯಗಳನ್ನು ಕುರಿತು ಎಡಬಿಡದೆ ಅಷ್ಟುಹೊತ್ತಿನಿಂದಲೂ ಮಾತನಾಡುವಂತೆ ಮಾಡಿರುವೆ.” ಶಿಷ್ಯನಿಗೆ ನಾಚಿಕೆಯಾಯಿತು. ಆದರೆ ಸ್ವಾಮೀಜಿ ಸ್ವಾಮಿ ಬ್ರಹ್ಮಾನಂದರಿಗೆ, “ನಿನ್ನ ಕವಿರಾಜರ ಚಿಕಿತ್ಸೆಯನ್ನು ಬದಿಗಿಡು. ಇವರೆಲ್ಲಾ ನನ್ನ ಮಕ್ಕಳು. ಅವರಿಗೆ ಬೋಧಿಸುವುದರಿಂದ ಈ ದೇಹ ಬಿದ್ದುಹೋಗುವ ಹಾಗಿದ್ದರೆ ಹೋಗಲಿ, ಅದಕ್ಕೆ ನಾನು ಸ್ವಲ್ಪವೂ ಬೆಲೆ ಕೊಡುವುದಿಲ್ಲ” ಎಂದರು. 

 ಅನಂತರ ಏನೇನೋ ಲೋಕಾಭಿರಾಮವಾಗಿ ಮಾತನಾಡುತ್ತಿದ್ದರು. 

 ನಂತರ ಬಂಗಾಳಿ ಸಾಹಿತಿ ಭರತಚಂದ್ರರ ವಿಚಾರ ಬಂದಿತು. ಮೊದಲಿನಿಂದಲೂ ಸ್ವಾಮೀಜಿ ಭರತಚಂದ್ರರನ್ನು ಅನೇಕ ಬಗೆಯಲ್ಲಿ ಟೀಕಿಸುತ್ತಿದ್ದರು. ಆತನ ಕಾಲದಲ್ಲಿ ಜನಜೀವನ ನಡವಳಿಕೆಗಳು, ವಿವಾಹಪದ್ಧತಿಗಳು ಮತ್ತು ಸಮಾಜದ ಇತರ ಸಂಪ್ರದಾಯಗಳನ್ನು ಕುರಿತು ವಿಡಂಬನೆ ಮಾಡಿದರು. ಭರತಚಂದ್ರ ಬಾಲ್ಯವಿವಾಹ ಪಕ್ಷಪಾತಿ. ಆತನ ಪದ್ಯಗಳು ಬಹಳ ಕೀಳುಮಟ್ಟದ್ದು ಮತ್ತು ಅಶ್ಲೀಲತೆಯಿಂದ ತುಂಬಿದ್ದವು. ಬಂಗಾಳವೊಂದನ್ನು ಬಿಟ್ಟರೆ ಮತ್ತಾವ ಸುಸಂಸ್ಕೃತ ಸಮಾಜವೂ ಅದನ್ನು ಇಷ್ಟ ಪಡುವುದಿಲ್ಲ ಎಂದು ಅಭಿಪ್ರಾಯಪಟ್ಟರು. ಅವರು ಹೇಳಿದ್ದೇನೆಂದರೆ “ಅಂತಹ ಪುಸ್ತಕಗಳು ಹುಡುಗರ ಕೈಗೆ ಸಿಗದಂತೆ ಎಚ್ಚರಿಕೆಯಿಂದಿರಬೇಕು.” ಅನಂತರ ಮೈಕೇಲ್ ಮಧುಸೂದನನ ವಿಚಾರವಾಗಿ ಅವರು ಹೇಳಿದರು. “ನಿಮ್ಮ ಸಂಸ್ಥಾನದಲ್ಲಿ ಅವನೊಬ್ಬ ಅದ್ಭುತ ಪ್ರತಿಭಾಶಾಲಿ. ಬಂಗಾಳಿ ಸಾಹಿತ್ಯದಲ್ಲಿ ಅವನ ‘ಮೇಘನಾದವಧ’ ದಂತಹ ಮಹಾ ಕಾವ್ಯ ಮತ್ತೊಂದಿಲ್ಲ. ಈಗಿನ ಆಂಗ್ಲ ಸಾಹಿತ್ಯದಲ್ಲಿ ಕೂಡ ಅಂತಹ ಕಾವ್ಯ ಕಾಣಬರುವುದು ಅಪರೂಪ.” 

 ಶಿಷ್ಯ: “ಆದರೆ ಸ್ವಾಮೀಜಿ, ಮೈಕೇಲ್‌ನಿಗೆ ಬಹಳ ಆಡಂಬರದ ಶೈಲಿ ಇಷ್ಟ.” 

 ಸ್ವಾಮೀಜಿ: “ನಿಮ್ಮ ದೇಶದಲ್ಲಿ ಹೊಸದಾಗಿ ಯಾರು ಏನನ್ನು ಮಾಡಿದರೂ ಅದನ್ನು ಛೀಮಾರಿ ಮಾಡುವಿರಿ. ಅವನು ಏನು ಹೇಳುತ್ತಾನೆಂಬುದನ್ನು ಮೊದಲು ಪರೀಕ್ಷಿಸಿ. ಅದನ್ನು ಬಿಟ್ಟು ನಿಮ್ಮ ದೇಶೀಯರು ಯಾವುದು ಪುರಾತನವಾಗಿಲ್ಲವೋ ಅದನ್ನೆಲ್ಲಾ ತುಚ್ಛೀಕರಿಸುವರು. ಉದಾಹರಣೆಗೆ ಬಂಗಾಳಿ ಸಾಹಿತ್ಯದ ಶಿರೋರತ್ನವಾದ ಈ ಮೇಘನಾದವಧ ಕಾವ್ಯವನ್ನು ಹಳಿಯುವುದಕ್ಕೋಸ್ಕರವೇ “ಚುಚುಂದರವಧ” ಎಂಬ ವಿಡಂಬನ ಕಾವ್ಯ ಬರೆಯಲ್ಪಟ್ಟಿತು. ಅವರು ಎಷ್ಟೇ ವ್ಯಂಗ್ಯವಾಗಿ ಬರೆಯಲಿ, ಅದರಿಂದೇನೂ ಫಲವಿಲ್ಲ. ಆ ಮೇಘನಾದವಧ ಕಾವ್ಯ ಹಿಮಾಲಯದಂತೆ ಆಚಂದ್ರಾರ್ಕವಾದ ಕೀರ್ತಿಯನ್ನು ಹೊಂದಿದೆ. ಈ ಟೀಕೆ ಮಾಡುವ ವಿಮರ್ಶಕರ ಅಭಿಪ್ರಾಯಗಳೆಲ್ಲ ಮೂಲೆಗೆ ಬಿದ್ದಿವೆ. ಮೈಕೇಲನು ಸ್ವಂತ ಶೈಲಿಯಲ್ಲಿ ಎಂತಹ ವೀರ‍್ಯವತ್ತಾದ ಪದಪ್ರಯೋಗ ಮಾಡಿ ಈ ಮಹಾಕಾವ್ಯ ಬರೆದಿದ್ದಾನೆಂಬುದು\break ಜನಸಾಮಾನ್ಯರಿಗೆ ಹೇಗೆ ತಾನೆ ಗೊತ್ತಾಗುವುದು? ಈಗ ತಾನೆ ಗಿರೀಶಬಾಬು ನಿಮ್ಮ ಪಂಡಿತರು ಅಷ್ಟೊಂದು ಅವಹೇಳನ ಮಾಡಿ ಟೀಕಿಸುತ್ತಿರುವ ಈ ನವಶೈಲಿಯಲ್ಲಿ ಅದ್ಭುತವಾದ ಗ್ರಂಥಗಳನ್ನು ಬರೆಯುತ್ತಿದ್ದಾನೆ. ಆದರೆ ಗಿರೀಶಬಾಬು ಅದಕ್ಕೆ ಕಿವಿಗೊಡುವನೇನು? ಜನರು ಕೆಲವು ಕಾಲದ ನಂತರ ಅದನ್ನು ಮೆಚ್ಚುವರು. 

 ಮೈಕೇಲನ ವಿಚಾರ ಹೀಗೆ ಕೆಲವು ಕಾಲ ಮಾತನಾಡುತ್ತ ಸ್ವಾಮೀಜಿ ಹೇಳಿದರು: “ಕೆಳಗಣ ಲೈಬ್ರರಿಯಲ್ಲಿ ಮೇಘನಾದವಧ ಕಾವ್ಯವನ್ನು ತೆಗೆದುಕೊಂಡು ಬಾ”. ಶಿಷ್ಯನು ತಂದಾಗ ಅವರು ಹೇಳಿದರು: “ಈಗ ಓದು, ನೀನು ಹೇಗೆ ಓದುವೆ ನೋಡೋಣ.” 

 ಶಿಷ್ಯ ಒಂದು ಭಾಗವನ್ನು ಓದಿದ. ಅದು ಸ್ವಾಮೀಜಿಗೆ ಒಗ್ಗದೆ ಅವರೇ ಪುಸ್ತಕವನ್ನು ತೆಗೆದುಕೊಂಡು ಅದನ್ನು ಓದುವುದು ಹೇಗೆಂದು ತೋರಿಸಿಕೊಟ್ಟು ಪುನಃ ಓದುವಂತೆ ಹೇಳಿದರು. ಅನಂತರ ಅವರು “ಈ ಕಾವ್ಯದ ಯಾವ ಭಾಗ ಅತ್ಯುತ್ತಮವಾದುದೆಂದು ನೀನು ಬಲ್ಲೆಯಾ?” ಎಂದರು. ಶಿಷ್ಯ ನಿರುತ್ತರನಾದಾಗ ಸ್ವಾಮಿಗಳು ಯಾವಭಾಗದಲ್ಲಿ ಇಂದ್ರಜಿತ್ತು ಯುದ್ಧದಲ್ಲಿ ಮಡಿದಿದ್ದಾನೆ, ಶೋಕಭರಿತಳಾಗಿದ್ದರೂ ಮಂಡೋದರಿ ರಾವಣನನ್ನು ಯುದ್ಧವಿಮುಖನಾಗುವಂತೆ ಪ್ರಯತ್ನಿಸುತ್ತಿದ್ದಾಳೆ, ರಾವಣ ತನ್ನ ಮನಸ್ಸಿನಿಂದ ಬಲವಂತವಾಗಿ ಪುತ್ರಶೋಕವನ್ನು ತಳ್ಳಿ ಮಹಾವೀರನಂತೆ ಯುದ್ಧಕ್ಕೆ ಹೋಗಲು ನಿರ್ಧರಿಸಿದ್ದಾನೆ, ತೀವ್ರಕೋಪ, ದ್ವೇಷಗಳಲ್ಲಿ ಪತ್ನೀಪುತ್ರರನ್ನೂ ಮರೆತು ಯುದ್ಧಕ್ಕೆ ನುಗ್ಗಲು ಸಿದ್ಧನಾಗಿದ್ದಾನೆ, –ಇದೇ ಈ ಪುಸ್ತಕದಲ್ಲಿ ಅತ್ಯುತ್ತಮ ಭಾಗ. ಏನು ಬೇಕಾದರೂ ಬರಲಿ, ನನ್ನ ಕರ್ತವ್ಯವನ್ನು ಮರೆಯುವುದಿಲ್ಲ – ಈ ಪ್ರಪಂಚ ಇದ್ದರೆಷ್ಟು, ಹೋದರೆಷ್ಟು? ಇದು ಒಬ್ಬ ಮಹಾವೀರನಾದವನ ಬಾಯಲ್ಲಿ ಬರುವಂತಹ ಮಾತುಗಳು. ಇಂತಹ ಭಾವನೆಗಳಿಂದ ಸ್ಫೂರ್ತಿಗೊಂಡು ಮೈಕೇಲನು ಆ ಭಾಗವನ್ನು ಬರೆದಿದ್ದಾನೆ.” ಹೀಗೆ ಹೇಳುತ್ತಾ ಸ್ವಾಮೀಜಿ ಪುಸ್ತಕದಲ್ಲಿ ಆ ಭಾಗವನ್ನು ತೆಗೆದುಕೊಂಡು ರಸವತ್ತಾಗಿ ಓದಲು ಮೊದಲು ಮಾಡಿದರು. 

\delimiter

 ಕವಿರಾಜನ ಚಿಕಿತ್ಸೆಯಲ್ಲಿ ಸ್ವಾಮೀಜಿಗೆ ಗುಣವಾಗಿದೆ. ಶಿಷ್ಯ ಈಗ ಮಠದಲ್ಲಿಯೇ ಇದ್ದಾನೆ. ಸ್ವಾಮೀಜಿ ಸೇವೆ ಮಾಡುತ್ತಿದ್ದಾಗ ಶಿಷ್ಯ “ಆತ್ಮ ಸರ್ವವ್ಯಾಪಿ, ಎಲ್ಲಾ ಜೀವಿಗಳ ಜೀವ, ತೀರ ಹತ್ತಿರದಲ್ಲಿದ್ದಾನೆ. ಆದರೂ ಅವನನ್ನೇಕೆ ನೋಡಲಾಗುವುದಿಲ್ಲ?” ಎಂದು ಕೇಳಿದ. 

 ಸ್ವಾಮೀಜಿ: “ನಿನಗೆ ಕಣ್ಣಿದೆ ಎಂಬುದನ್ನು ನೀನೇ ನೋಡುವಿಯೇನು? ಇತರರು ಕಣ್ಣುಗಳ ವಿಚಾರ ಮಾತನಾಡಿದಾಗ ನಿನಗೂ ಕಣ್ಣಿದೆ ಎಂಬ ನೆನಪಾಗುವುದು. ಅಲ್ಲದೆ ಧೂಳು ಮಣ್ಣೇನಾದರೂ ನಿನ್ನ ಕಣ್ಣಿನೊಳಗೆ ಬಿದ್ದು ವೇದನೆಯನ್ನುಂಟು ಮಾಡಿದಾದ ನಿನಗೂ ಕಣ್ಣಿದೆ ಎಂದು ಚೆನ್ನಾಗಿ ಗೊತ್ತಾಗುವುದು. ಹಾಗೆಯೇ ವಿಶ್ವಾತ್ಮನ ಸಾಕ್ಷಾತ್ಕಾರವೂ ಸುಲಭವಾಗಿ ಸಿಕ್ಕುವಂತಹುದಲ್ಲ. ಧರ್ಮಶಾಸ್ತ್ರಗಳನ್ನೋದುವುದರಿಂದ, ಮತ್ತೊಬ್ಬರಿಂದ ಉಪದೇಶ ಕೇಳುವುದರಿಂದ ಇತರರ ಸ್ವಭಾವ ಸ್ವಲ್ಪಮಟ್ಟಿಗೆ ನಿನಗೆ ತಿಳಿಯುವುದು. ಆದರೆ ಯಾವಾಗ ಈ ಪ್ರಪಂಚದ ಕಹಿವೇದನೆ, ಕಷ್ಟಗಳ ಪೆಟ್ಟಿನಿಂದ ನಿನ್ನ ಹೃದಯ ಜರ್ಝರಿತವಾಗುವುದೋ, ನಮ್ಮ ಹತ್ತಿರದ ಪ್ರಿಯ ಬಂಧುಗಳು ಮರಣವನ್ನೈದುವರೋ ಆಗ ಮಾನವ ತಾನು ದಿಕ್ಕಿಲ್ಲದ ಅನಾಥನೆಂದು ತಿಳಿಯುತ್ತಾನೆ. ದಾಟಲಾಗದ ಅಭೇದ್ಯವಾದ ಶೂನ್ಯತೆ ಅವನ ಮನಸ್ಸನ್ನೆಲ್ಲಾ ಆವರಿಸಿದಾಗ ಜೀವನು ಆತ್ಮಸಾಕ್ಷಾತ್ಕಾರಕ್ಕೆ ಹಾತೊರೆಯುತ್ತಾನೆ. ಅದಕ್ಕೇ ದುಃಖ ಆತ್ಮಜ್ಞಾನಕ್ಕೆ ಸಹಕಾರಿ. ಆದರೆ ಈ ಅನುಭವದ ಕಹಿನೆನಪನ್ನು ನಾವು ಯಾವಾಗಲೂ ನೆನಪಿನಲ್ಲಿಟ್ಟಿರಬೇಕು. ಯಾರು ಕೇವಲ ನಾಯಿ ಬೆಕ್ಕುಗಳಂತೆ ಜೀವನದ ಅಳಲನ್ನು ಅನುಭವಿಸುತ್ತಾ ಸಾಯುವನೋ ಅವನು ಮನುಷ್ಯನೇನು? ಯಾವ ಮನುಷ್ಯ ಸುಖ ದುಃಖದ ತೀಕ್ಷ್ಣ ಪ್ರಹಾರಗಳಿಂದ ಜರ್ಝರಿತನಾದಾಗಲೂ ಯುಕ್ತಾಯುಕ್ತ ವಿಚಾರ ಮಾಡುವನೋ ಅವನು ಮಾತ್ರ ಮನುಷ್ಯ. ಅಂಥವನು ಇವೆಲ್ಲಾ ಕ್ಷಣಭಂಗುರಗಳೆಂದು ತಿಳಿದು ಆತ್ಮದಲ್ಲಿ ಗಾಢವಾದ ಭಕ್ತಿಯುಳ್ಳವನಾಗುವನು. ಇದೇ ಮನುಷ್ಯರಿಗೂ ಪ್ರಾಣಿಗಳಿಗೂ ಇರುವ ವ್ಯತ್ಯಾಸ. ಯಾವುದು ಅತ್ಯಂತ ಹತ್ತಿರವಿದೆಯೋ ಅದು ಬೇಗ ದೃಷ್ಟಿ ಗೆ ಗೋಚರವಾಗುವುದಿಲ್ಲ. ಆತ್ಮನು ನಮಗೆ ಅತ್ಯಂತ ಹತ್ತಿರದವನು. ಅದಕ್ಕೇ ಉದಾಸೀನ. ಅಸ್ಥಿರವಾದ ಮನಸ್ಸಿಗೆ ಅದರ ಅರಿವೇ ಆಗುವುದಿಲ್ಲ. ಯಾವ ಮನುಷ್ಯ ಚಟುವಟಿಕೆಯುಳ್ಳವನಾಗಿ, ಶಾಂತನಾಗಿ, ಆತ್ಮನಿಗ್ರಹ ಮತ್ತು ವಿಮರ್ಶಾಜ್ಞಾನವುಳ್ಳವನಾಗಿರುವನೋ ಅವನು ಈ ಬಾಹ್ಯಜಗತ್ತನ್ನು ನಿರ್ಲಕ್ಷಿಸಿ ಅತಂರ್‌ಜ್ಞಾನದಲ್ಲೇ ಹೆಚ್ಚು ಹೆಚ್ಚು ಮುಳುಗುತ್ತಾನೆ. ಆತ್ಮನ ಮಹಾತ್ಮೆಯನ್ನರಿತು ತಾನೂ ಮಹಾತ್ಮನಾಗುವನು. ಆಗ ಮಾತ್ರ ಅವನಿಗೆ ಆತ್ಮಜ್ಞಾನ ಲಭಿಸಿ ಶಾಸ್ತ್ರದಲ್ಲಿ ಹೇಳುವ ‘ನಾನೇ ಆತ್ಮ, ನೀನೆ ಅದು ಆಗಿರುವೆ, ಓ! ಶ್ವೇತಕೇತು’ ಮುಂತಾದುವುಗಳಲ್ಲಿರುವ ಸತ್ಯ ವೇದ್ಯವಾಗುತ್ತದೆ. ಅರ್ಥವಾಯಿತೆ?” 

 ಶಿಷ್ಯ: “ಆಯಿತು ಸ್ವಾಮೀಜಿ, ಹೀಗೆ ದುಃಖ ಸಂಕಟದ ಹಾದಿಯಿಂದ ಆತ್ಮಜ್ಞಾನ ಪಡೆದುಕೊಳ್ಳುವ ಮಾರ್ಗವೇಕೆ? ಇದಕ್ಕೆ ಬದಲಾಗಿ ಸೃಷ್ಟಿಯೇ ಇಲ್ಲದಿದ್ದಲ್ಲಿ ಎಲ್ಲವೂ ಸರಿಹೋಗುತ್ತಿತ್ತು. ನಾವೆಲ್ಲಾ ಬ್ರಹ್ಮನೊಡನೆ ಐಕ್ಯವಾಗಿದ್ದೆವು. ಹೀಗಿದ್ದ ಮೇಲೆ ಬ್ರಹ್ಮನಿಗೇಕೆ ಈ ಸೃಷ್ಟಿಸುವ ಆಸೆ? ಬ್ರಹ್ಮನೇ ಆದ ಜೀವನು ಹುಟ್ಟುಸಾವುಗಳ ಪರಸ್ಪರ ದ್ವಂದ್ವಗಳಲ್ಲಿ ಹೋರಾಡುವುದು ಏಕೆ?” 

 ಸ್ವಾಮೀಜಿ: “ಯಾವಾಗ ಮನುಷ್ಯನಿಗೆ ಮತ್ತೇರಿದೆಯೋ ಅವನಿಗೆ ಅನೇಕ ಬಗೆಯ ಭ್ರಮೆಯುಂಟಾಗುವುದು. ಆದರೆ ಯಾವಾಗ ಮತ್ತಿಳಿಯುವುದೋ ಆಗ ಅವನಿಗೆ ಅವೆಲ್ಲಾ ಕಾವೇರಿದ ಮೆದುಳಿನ ಕಲ್ಪನೆಗಳೆಂದು ಗೊತ್ತಾಗುವುದು. ನೀನೀಗ ಆದಿಯೇ ಇಲ್ಲದ ಸೃಷ್ಟಿಯಲ್ಲಿ ಏನೇನು ಮಾಡುವಿಯೋ ಯಾವುದಕ್ಕೆ ಕೊನೆ ಇದೆ ಎನ್ನಿಸುವುದೋ ಅವೆಲ್ಲಾ ನಿನ್ನ ಮತ್ತೇರಿದ ಸ್ಥಿತಿಯ ಪರಿಣಾಮ. ಯಾವಾಗ ಈ ಸ್ಥಿತಿಯನ್ನು ಮೀರಿಹೋಗುವೆಯೋ ಆಗ ಈ ಪ್ರಶ್ನೆಗಳಿಗೆ ಎಡೆಯೇ ಇರುವುದಿಲ್ಲ. 

 ಶಿಷ್ಯ: “ಹಾಗಾದರೆ ಈ ಭೂಮಂಡಲದ ಸೃಷ್ಟಿಸ್ಥಿತಿಗಳಲ್ಲಿ ಯಾವ ಸತ್ಯವೂ ಇಲ್ಲವೆ?” 

 ಸ್ವಾಮೀಜಿ: “ಏಕಿರಬಾರದು? ಎಲ್ಲಿಯವರೆಗೆ ನಿನ್ನಲ್ಲಿ ದೇಹಭಾವನೆ ಇದ್ದು ಅಹಂಭಾವವಿರುವುದೋ ಅಲ್ಲಿಯವರೆಗೂ ಇವೆಲ್ಲಾ ಇರುವುದು. ಆದರೆ ದೇಹ ಭಾವನೆ ಹೋಗಿ ನೀನು ಆತ್ಮನಲ್ಲಿ ಭಕ್ತಿಯುಳ್ಳವನಾಗಿ, ಆತ್ಮನಲ್ಲಿ ಜೀವಿಸುವೆಯೋ ಆಗ ನಿನಗೆ ಇವುಗಳೊಂದೂ ಇರುವುದಿಲ್ಲ. ಸೃಷ್ಟಿ, ಹುಟ್ಟು, ಸಾವು ಇದೆಯೋ ಇಲ್ಲವೋ ಮುಂತಾದ ಪ್ರಶ್ನೆಗಳಿಗೆ ಎಡೆಯೇ ಇರುವುದಿಲ್ಲ. ಆಗ ನೀನು ಹೀಗೆ ಹೇಳಬೇಕಾಗುವುದು:

\begin{verse}
ಕ್ವ ಗತಂ ಕೇನ ವಾ ನೀತಂ ಕುತ್ರಲೀನಮಿದಂ ಜಗತ್~।\\ಆಧುನೈನ ಮಯಾ ದೃಷ್ಟಂ ನಾಸ್ತಿ ಕಿಂ ಮಹದ್ಭುತಂ~॥ 
\end{verse}

 ಪ್ರಪಂಚ ಎಲ್ಲಿ ಹೋಯಿತು? ಯಾರಿಂದ ಹೋಯಿತು? ಅದೆಲ್ಲಿ ಅವಿತುಕೊಂಡಿತು? ಈಗತಾನೆ ಅದನ್ನು ನೋಡಿದೆ, ಮರುಕ್ಷಣವೇ ಇಲ್ಲವಾಗಿದೆ. ಎಂತಹ ಅದ್ಭುತ!” 

 ಶಿಷ್ಯ: “ಪ್ರಪಂಚ ಇರುವುದನ್ನೇ ಅರಿಯದೆ, ‘ಭೂಮಂಡಲವೆಲ್ಲಾ ಎಲ್ಲಿಲಯವಾಯಿತು?’ ಎಂದು ಕೇಳುವುದು!” 

 ಸ್ವಾಮೀಜಿ: “ಏಕೆಂದರೆ ನಮ್ಮ ಭಾವನೆಯನ್ನು ಭಾಷೆಯ ಮೂಲಕ ವಿವರಿಸಬೇಕು. ಅದಕ್ಕೇ ಹಾಗೆ ವಿವರಿಸುವುದು. ಆ ಶ್ಲೋಕದ ಕವಿಭಾವ. ಭಾಷೆಗೆ ಮೀರಿದ ಸ್ಥಿತಿಯನ್ನು ಭಾವ ಮತ್ತು ಭಾಷೆಯ ಮೂಲಕ ವಿವರಿಸತೊಡಾಗಿದ್ದಾನೆ. ಅದಕ್ಕೆ ಅವನು ಜಗತ್ತು ಮಾಯೆ, ಆಕಾಶದಂತೆ ಸಾಪೇಕ್ಷ ಎಂದು ವಿವರಿಸುತ್ತಾನೆ. ಜಗತ್ತು ನಿರಪೇಕ್ಷವಾಗಿ ಸತ್ಯವಲ್ಲ. ಮನಸ್ಸು ಮತ್ತು ಭಾಷೆಗೆ ಮೀರಿದ ಬ್ರಹ್ಮನು ಮಾತ್ರ ಸತ್ಯ. ನೀನೇನು ಕೇಳಬೇಕೆಂದಿದ್ದೀಯೋ ಕೇಳು. ಇಂದು ನಿನ್ನ ಸಂಶಯಗಳನ್ನೆಲ್ಲ ಬಗೆಹರಿಸುತ್ತೇನೆ.” 

 ಪೂಜಾಮಂದಿರದಿಂದ ಸಂಜೆ ಆರತಿಗಾಗಿ ಘಂಟೆಯ ನಿನಾದ ಕೇಳಿಬಂತು. ಎಲ್ಲರೂ ಮಂಗಳಾರತಿಗೆ ಹೋದರು. ಶಿಷ್ಯ ಸ್ವಾಮೀಜಿಯವರ ಕೊಠಡಿಯಲ್ಲೇ ಉಳಿದ. ಇದನ್ನು ನೋಡಿ ಸ್ವಾಮೀಜಿ “ಪೂಜಾಮಂದಿರಕ್ಕೆ ನೀನು ಹೋಗುವುದಿಲ್ಲವೋ!” ಎಂದು ಕೇಳಿದರು. 

 ಶಿಷ್ಯ: “ನನಗೆ ಇಲ್ಲಿಯೇ ಇರಲು ಇಚ್ಛೆ” 

 ಸ್ವಾಮೀಜಿ: “ಹಾಗೆಮಾಡು.” 

 ಸ್ವಲ್ಪ ಕಾಲಾನಂತರ ಶಿಷ್ಯ ಕೊಠಡಿಯ ಹೊರಗೆ ನೋಡಿ “ಇಂದು ಅಮಾವಾಸ್ಯೆ ರಾತ್ರಿ. ಎಲ್ಲಾ ಭಾಗಗಳೂ ಅಂಧಕಾರದಿಂದಾವೃತವಾಗಿವೆ. ಇಂದು ಕಾಳಿಕಾಮಾತೆಯನ್ನು ಪೂಜಿಸುವ ರಾತ್ರಿ” ಎಂದ. 

 ಸ್ವಾಮೀಜಿ ಮೌನವಾಗಿ ಸ್ವಲ್ಪಹೊತ್ತು ಪೂರ್ವದಿಗಂತದ ಕಡೆ ದಿಟ್ಟಿಸಿ ನೋಡಿದರು. ನಂತರ ಹೇಳಿದರು: “ಈ ಕತ್ತಲೆಯಲ್ಲಿ ಎಂತಹ ಗುಪ್ತವಾದ ಪ್ರಶಾಂತ ಸಂದರ್ಯವಿದೆ” – ಹೀಗೆ ಹೇಳುತ್ತಾ ಸ್ವಾಮಿಗಳು ದಟ್ಟವಾದ ಆ ಅಂಧಕಾರವನ್ನೇ ದಿಟ್ಟಿಸುತ್ತಾ ಅದರಲ್ಲೇ ಪರವಶರಾದರು. ಕೆಲವು ನಿಮಿಷಗಳಾದ ನಂತರ ಅವರು ನಿಧಾನವಾಗಿ ಒಂದು ಬಂಗಾಳಿ ಗೀತೆಯಲ್ಲಿ ಹಾಡಲು ಪ್ರಾರಂಭಿಸಿದರು: “ಓ! ಮಾತೆ, ತೀವ್ರ ಅಂಧಕಾರದಲ್ಲಿ ನಿನ್ನ ಸೌಂದರ‍್ಯ ಪ್ರಕಾಶಿಸುವುದು…” ಹಾಡುಮುಗಿದ ಮೇಲೆ ಸ್ವಾಮೀಜಿ ಕೊಠಡಿಯನ್ನು ಪ್ರವೇಶಿಸಿ ಬಾಯಲ್ಲಿ, ‘ಮಾ, ಮಾ, ಕಾಳಿ,’ ಎಂದು ಆಗಾಗ ಹೇಳುತ್ತಾ ಕುಳಿತುಕೊಂಡರು. ನೋಡಿ ಕಳವಳಗೊಂಡು ಶಿಷ್ಯ ಕೇಳಿದ, “ಸ್ವಾಮೀಜಿ, ದಯವಿಟ್ಟು ನನ್ನೊಡನೆ ಮಾತನಾಡಿ.” 

 ಸ್ವಾಮೀಜಿ: (ನಗುಮುಖದಿಂದ ಹೇಳಿದರು) “ಬಾಹ್ಯದಲ್ಲಿ ಇಷ್ಟು ಸುಂದರವಾಗಿ ಮಧುರವಾಗಿರುವ ಆತ್ಮನ ಅಗಾಧವಾದ ಸೌಂದರ‍್ಯದ ಆಳವನ್ನು ಅರಿಯಬಲ್ಲೆಯಾ?” ಎಂದರು. ಶಿಷ್ಯ ಆ ಸಂಭಾಷಣೆಯನ್ನು ಬೇರೆ ಕಡೆ ತಿರುಗಿಸಲು ಇಚ್ಛಿಸಿದಾಗ ಅದನ್ನು ನೋಡಿ ಸ್ವಾಮೀಜಿ ಕಾಳಿಯ ಮೇಲೆ ಮತ್ತೊಂದು ಹಾಡನ್ನು ಎತ್ತಿದರು: ‘ಹೇ ಮಾತೆ, ಅಮೃತಪ್ರವಾಹ ನೀನು, ಎಷ್ಟು ಭಾವದಲ್ಲಿ ಆಕಾರದಲ್ಲಿ ನೀನು ವ್ಯಕ್ತವಾಗುತ್ತಿರುವೆ!’ ಹಾಡು ಮುಗಿದ ಮೇಲೆ ಸ್ವಾಮೀಜಿ “ಕಾಳಿ ಬ್ರಹ್ಮನ ವ್ಯಕ್ತಸ್ವರೂಪ. ಶ‍್ರೀರಾಮಕೃಷ್ಣರು ಹೇಳುತ್ತಿದ್ದ ಚಲಿಸುವ ಚಲಿಸದೆ ಇರುವ ಹಾವಿನ ಉಪಮಾನ ಗೊತ್ತಿಲ್ಲವೆ? (ಒಂದು ವ್ಯಕ್ತ ಮತ್ತೊಂದು ಅವ್ಯಕ್ತಸ್ವರೂಪ)” 

 ಶಿಷ್ಯ: “ಹೌದು ಸ್ವಾಮೀಜಿ.” 

 ಸ್ವಾಮೀಜಿ: “ಈ ಬಾರಿ ನನಗೆ ಗುಣವಾದ ಮೇಲೆ ನಾನು ಮಾತೆಯನ್ನು ನನ್ನ ಹೃದಯದ ರಕ್ತದಿಂದ ಪೂಜಿಸುವೆ. ಆಗ ಮಾತ್ರ ಆಕೆಗೆ ಸಂತೋಷವಾಗುವುದು. ನಿನ್ನ ರಘುನಂದನ ಕೂಡ ಅದನ್ನು ಹೇಳುತ್ತಾನೆ. ಮಾತೆಯ ಮಗು ವೀರನಾಗಬೇಕು. ವ್ಯಥೆ, ದುಃಖ, ಸಾವುಗಳಿಗೀಡಾದರೂ ನಿರ್ಗತಿಕನಾದರೂ ಮಾತೆಯ ಪುತ್ರ ನಿರ್ಭಯನಾಗಿರಬೇಕು.” 

 ಸ್ವಾಮೀಜಿ ಈಗ ಮಠದಲ್ಲೇ ಇದ್ದಾರೆ. ಅವರ ಆರೋಗ್ಯ ಅಷ್ಟೇನೂ ಗುಣಮುಖವಾಗಿಲ್ಲ. ಆದರೂ ಅವರು ಬೆಳಿಗ್ಗೆ ಮತ್ತು ಸಂಜೆ ಎರಡೂ ಹೊತ್ತೂ ಗಾಳಿ ಸಂಚಾರ ಹೊರಡುತ್ತಾರೆ. ಶಿಷ್ಯ ಸ್ವಾಮೀಜಿಗೆ ಪ್ರಣಾಮ ಮಾಡಿ ಅವರ ಆರೋಗ್ಯದ ವಿಷಯವಾಗಿ ಪ್ರಶ್ನೆಮಾಡಿದ. 

 ಸ್ವಾಮೀಜಿ: “ಸರಿ, ಈ ದೇಹ ಅತ್ಯಂತ ಶೋಚನೀಯಾವಸ್ಥೆಯಲ್ಲಿದೆ. ಆದರೆ ನೀವಾದರೂ ನನ್ನ ಕೆಲಸಕ್ಕೆ ಸಹಾಯಮಾಡಲು ಒಂದು ಹೆಜ್ಜೆಯನ್ನೂ ಮುಂದಿಟಿಲ್ಲ. ನಾನೊಬ್ಬ ಏನುತಾನೆ ಮಾಡಬಲ್ಲೆ? ಈ ಬಾರಿ ಶರೀರ ಬಂಗಾಳದೇಶದಲ್ಲಿ ಜನಿಸಿದುದರಿಂದ ಅದು ಹೇಗೆ ತಾನೆ ಹೆಚ್ಚು ಶ್ರಮವನ್ನು ಸಹಿಸಬಲ್ಲುದು? ಇಲ್ಲಿಗೆ ಬರುವವರೆಲ್ಲಾ ಪವಿತ್ರಾತ್ಮರು. ನನ್ನ ಕೆಲಸಕ್ಕೆ ನೀವು ಸಹಾಯಕರಾಗದಿದ್ದಲ್ಲಿ ನಾನೊಬ್ಬನೇ ಏನು ಮಾಡಬಲ್ಲೆ? 

 ಶಿಷ್ಯ: “ಸ್ವಾಮೀಜಿ, ಈ ಆತ್ಮತ್ಯಾಗನಿರತರಾದ ಬ್ರಹ್ಮಚಾರಿಗಳು, ಸಂನ್ಯಾಸಿಗಳೆಲ್ಲಾ ನಿಮ್ಮ ಹಿಂದೆ ಇದ್ದಾರೆ. ಅವರಲ್ಲಿ ಪ್ರತಿಯೊಬ್ಬರೂ ತಮ್ಮ ಜೀವನವನ್ನು ನಿಮ್ಮ ಕೆಲಸಕ್ಕೆ ಅರ್ಪಿಸಲು ಸಿದ್ಧರಾಗಿದ್ದಾರೆ – ಆದರೂ ನೀವೇಕೆ ಹೀಗೆ ಹೇಳುವುರಿ?” 

 ಸ್ವಾಮೀಜಿ: “ನನಗೆ ಬಂಗಾಳಿ ಯುವಕರ ಒಂದು ಸೇನೆಯೇ ಬೇಕು. ಅವರೇ ದೇಶಕ್ಕೆ ಭರವಸೆ ನೀಡುವರು. ಒಳ್ಳೆಯ ಶೀಲವಂತರಾದ ಯುವಕರ, ಬುದ್ಧಿವಂತರಾಗಿ ತಮ್ಮ ಸರ್ವಸ್ವವನ್ನೂ ತ್ಯಾಗಮಾಡುವಂತಹ, ವಿಧೇಯರಾಗಿ ನನ್ನ ಭಾವನೆಗಳನ್ನು ಕಾರ್ಯರೂಪಕ್ಕೆ ತರಲು ತಮ್ಮ ಪ್ರಾಣವನ್ನೇ ಅರ್ಪಿಸಬಲ್ಲವರ ಮೇಲೆ ನನ್ನ ಭವಿಷ್ಯದ ಹಾರೈಕೆಯೆಲ್ಲಾ ನಿಂತಿದೆ. ಅದರಿಂದ ದೇಶಕ್ಕೂ ಅವರಿಗೂ ಕಲ್ಯಾಣವಾಗುವುದು. ಇಲ್ಲದಿದ್ದಲ್ಲಿ ಸಾಮಾನ್ಯ ಯುವಕರೂ ಬರುತ್ತಿದ್ದಾರೆ– ಬರುತ್ತಿರುತ್ತಾರೆ– ಅವರ ಮುಖದಲ್ಲಿ ಗೋಳು ಹರಿಯುತ್ತಿದೆ. ಅವರ ಹೃದಯ ವೀರ್ಯಹೀನವಾಗಿದೆ. ಅವರ ದೇಹ ದುರ್ಬಲವಾಗಿದೆ. ಕೆಲಸಕ್ಕೆ ಅನರ್ಹವಾಗಿದೆ. ಅವರ ಮನಸ್ಸು ಧೈರ್ಯಹೀನವಾಗಿದೆ. ಅವರಿಂದ ಏನು ಕೆಲಸ ತಾನೆ ಸಾಧ್ಯ? ನಚಿಕೇತನಿಗಿದ್ದಂತಹ ಶ್ರದ್ಧೆಯುಳ್ಳ ೧೦–೧೨ ಜನ ಹುಡುಗರು ನನಗೆ ಸಿಕ್ಕಿದ್ದರೆ ದೇಶದ ಯೋಜನೆ ವೃತ್ತಿಯನ್ನೆಲ್ಲಾ ಹೊಸ ಮಾರ್ಗದಲ್ಲಿ ಹರಿಯುವಂತೆ ಮಾಡಬಲ್ಲೆ.” 

 ಶಿಷ್ಯ: “ಅಷ್ಟೊಂದು ಮಂದಿ ಯುವಕರು ನಿಮ್ಮೆಡೆಗೆ ಬರುತ್ತಾರೆ. ಅವರಲ್ಲಿ ಯಾರೊಬ್ಬರಿಗೂ ನೀವು ಹೇಳಿದ ಗುಣಗಳಿಲ್ಲವೆ?” 

 ಸ್ವಾಮೀಜಿ: “ಅವರಲ್ಲಿ ಯಾರು ನನಗೆ ಒಳ್ಳೆಯವರೆಂದು ತೋರುವರೋ ಅವರಲ್ಲಿ ಕೆಲವರು ಮದುವೆಯ ಬಂಧನಕ್ಕೆ ಸಿಲುಕಿದ್ದಾರೆ. ಕೆಲವರು ಪ್ರಾಪಂಚಿಕ ಹೆಸರು, ಕೀರ್ತಿ, ಐಶ್ವರ್ಯಕ್ಕೆ ತಮ್ಮನ್ನು ಮಾರಿಕೊಂಡಿದ್ದಾರೆ–, ಕೆಲವರು ನಿಶ್ಯಕ್ತರು. ಉಳಿದವರು, ಅವರೇ ಹೆಚ್ಚು ಮಂದಿ, ಯಾವ ಉತ್ತಮ ಆದರ್ಶಗಳನ್ನು ಪಡೆಯಲು ಯೋಗ್ಯರಲ್ಲ. ನೀನೇನೋ ನನ್ನ ಆದರ್ಶ ಭಾಗವನ್ನು ಹೊಂದಲು ಅರ್ಹ. ಆದರೆ ಅದನ್ನು ವ್ಯವಹಾರದಲ್ಲಿ ತರಲು ನಿನಗೆ ಸಾಧ್ಯವಿಲ್ಲ. ಈ ಕಾರಣಗಳಿಂದ ಒಮ್ಮೊಮ್ಮೆ ನನ್ನ ಮನಸ್ಸು ರೊಚ್ಚಿಗೇಳುವುದು. ಈ ಮಾನವ ದೇಹ ಧರಿಸುವುದರಿಂದ ಇದು ಅಡ್ಡಿ ಬಂದು ನಾನು ಹೆಚ್ಚು ಕೆಲಸ ಮಾಡಲಾಗುವುದಿಲ್ಲ. ಆದರೂ ನಾನು ಸಂಪೂರ್ಣ ನಂಬಿಕೆ ಕಳೆದುಕೊಂಡಿಲ್ಲ. ಏಕೆಂದರೆ ದೇವರ ಇಚ್ಛೆಯಿಂದ ಈ ಕೆಲವು ಹುಡುಗರಿಂದಲೇ ಮುಂದೆ ಅನೇಕ ದೊಡ್ಡ ಕರ್ಮವೀರರು, ಆಧ್ಯಾತ್ಮಿಕ ವೀರರು ಹುಟ್ಟಿಬಂದು ನನ್ನ ಉದ್ದೇಶಗಳನ್ನು ಭವಿಷ್ಯದಲ್ಲಿ ಕಾರ್ಯರೂಪಕ್ಕೆ ತರಬಹುದು.” 

 ಶಿಷ್ಯ: “ನಿಮ್ಮ ವಿಶಾಲವಾದ ಮತ್ತು ಸರಳವಾದ ಆದರ್ಶಗಳನ್ನು ಒಂದಲ್ಲ ಒಂದು ದಿನ ಇಡೀ ವಿಶ್ವವೇ ಸ್ವೀಕರಿಸುವುದೆಂದು ನನಗೆ ದೃಢವಾದ ನಂಬಿಕೆ ಇದೆ. ಏಕೆಂದರೆ ಅವು ಸರ್ವಭಾಗಗಳನ್ನೂ ಒಳಗೊಂಡಿವೆ, ಎಲ್ಲಾ ಆದರ್ಶಗಳಿಗೂ ಶಕ್ತಿ ಒದಗಿಸುತ್ತವೆ. ದೇಶದ ಜನರೆಲ್ಲಾ ಪ್ರತ್ಯಕ್ಷವಾಗಿ ಆಗಲಿ ಪರೋಕ್ಷವಾಗಿ ಆಗಲಿ ನಿಮ್ಮ ಅಭಿಪ್ರಾಯಗಳನ್ನು ಒಪ್ಪಿಕೊಂಡು ಜನರಿಗೆ ಬೋಧಿಸುತ್ತಿದ್ದಾರೆ.” 

 ಸ್ವಾಮೀಜಿ: “ಅವರು ನನ್ನ ಹೆಸರನ್ನು ಗೌರವಿಸಿದರೇನು ಬಿಟ್ಟರೇನು? ಅವರು ನನ್ನ ಅಭಿಪ್ರಾಯಗಳನ್ನು ಒಪ್ಪಿಕೊಂಡರೆ ಸಾಕು. ಕಾಮಿನಿ ಕಾಂಚನಗಳನ್ನು ತ್ಯಾಗ ಮಾಡಿದ ಮೇಲೂ ಶೇಕಡಾ ೯೯ ಮಂದಿ ಸಾಧುಗಳು ಹೆಸರು ಕೀರ್ತಿಗಳ ಬಲೆಗೆ ಸಿಕ್ಕಿಬೀಳುವರು. “ಕೀರ್ತಿ…. ಉದಾತ್ತ ಮನಸ್ಸಿನ ಕಟ್ಟಕಡೆಯ ದುರ್ಬಲತೆ…" ನೀನಿದನ್ನು ಓದಿಲ್ಲವೆ? ಕರ್ಮಫಲದಾಸೆಯನ್ನೆಲ್ಲಾ ತ್ಯಜಿಸಿ ನಾವು ಕೆಲಸಮಾಡಬೇಕು. ಜನರು ನಮ್ಮನ್ನು ಒಳ್ಳೆಯವರು, ಕೆಟ್ಟವರು ಎಂದು ಎರಡನ್ನೂ ಹೇಳುತ್ತಾರೆ. ಆದರೆ ನಾವು ನಮ್ಮ ಗುರಿಯನ್ನು ಮುಂದಿಟ್ಟುಕೊಂಡು ಜನರು ನಮ್ಮನ್ನು ಹೊಗಳಲಿ ಬಿಡಲಿ ಯಾವುದಕ್ಕೂ ಸೊಪ್ಪುಹಾಕದೆ ಸಿಂಹಸದೃಶರಾಗಿ ಕೆಲಸ ಮಾಡಬೇಕು.” 

 ಶಿಷ್ಯ: “ನಾವೀಗ ಯಾವ ಗುರಿಯನ್ನು ಅನುಸರಿಸಬೇಕು?” 

 ಸ್ವಾಮೀಜಿ: “ಮಹಾವೀರನನ್ನು ನಮ್ಮ ಗುರಿಯಾಗಿಟ್ಟುಕೊಳ್ಳಬೇಕು. ರಾಮಚಂದ್ರನ ಅಣತಿಯಂತೆ ಅವನು ಸಮುದ್ರವನ್ನು ದಾಟಿದ: ಅವನಿಗೆ ಜೀವನ ಮರಣ ಯಾವುದರ ಮೇಲೂ ಗಮನವಿರಲಿಲ್ಲ. ಅವನು ಇಂದ್ರಿಯಗಳನ್ನು ಸಂಪೂರ್ಣ ತನ್ನ ವಶದಲ್ಲಿಟ್ಟುಕೊಂಡಿದ್ದ ತೀಕ್ಷ್ಣ ಮತಿಯಾಗಿದ್ದನು. ನೀವೀಗ ಈ ಸೇವೆಯ ಆದರ್ಶದ ಮೇಲೆ ನಿಮ್ಮ ಜೀವನವನ್ನು ಕಟ್ಟಬೇಕು. ಅದರ ಮೂಲಕ ಇತರ ಎಲ್ಲಾ ಆದರ್ಶಗಳೂ ಕ್ರಮೇಣ ವಿಕಸಿಸುವುವು. ಎರಡನೇ ಮಾತಿಲ್ಲದೆ ಗುರು ಮಾತಿಗೆ ವಿಧೇಯತೆ ಮತ್ತು ಬ್ರಹ್ಮಚರ್ಯ ಇದೇ ಜಯದ ರಹಸ್ಯ. ಹನುಮಂತ ಒಂದು ಕಡೆ ಸೇವೆಯ ಆದರ್ಶವನ್ನು ಇನ್ನೊಂದು ಕಡೆ ಇಡೀ ಜಗತ್ತೇ ಬೆರಗಾಗುವಂತಹ ಸಿಂಹಸದೃಶ ಧೈರ್ಯವನ್ನು ಪ್ರತಿಬಿಂಬಿಸುತ್ತಾನೆ. ರಾಮನ ಕಲ್ಯಾಣಕ್ಕಾಗಿ ಪ್ರಾಣ ಅರ್ಪಿಸಲು ಅವನು ಸ್ವಲ್ಪವೂ ಅನುಮಾನಿಸಲಿಲ್ಲ. ರಾಮನ ಸೇವೆಯ ಹೊರತು ಉಳಿದುದೆಲ್ಲಾ ಅವನಿಗೆ ತಾತ್ಸಾರ. ಬ್ರಹ್ಮ, ಶಿವ ಮತ್ತು ಈ ಜಗತ್ತಿನ ದೇವತೆಗಳ ಪಡೆಯೆಲ್ಲಾ ಅವನಿಗೆ ಬೇಸರ. ಶ‍್ರೀರಾಮನ ಆಣತಿಯಂತೆ ನಡೆಯುವುದೊಂದೇ ಅವನ ಜೀವನದ ವ್ರತ. ಅಂತಹ ಹೃತ್ಪೂರ್ವಕ ಭಕ್ತಿ ಬೇಕಾಗಿದೆ. ತಾಳಕ್ಕೆ ಸರಿಯಾಗಿ ಕುಣಿಯುತ್ತಾ ಜನರೆಲ್ಲಾ ಅಧೋಗತಿಗಿಳಿದಿದ್ದಾರೆ. ಮೊದಲೇ ಅಜೀರ್ಣ ವ್ಯಾಧಿಗೆ ಈಡಾದ ಜನಾಂಗ, ಜೊತೆಗೆ ಹೀಗೆ ಕುಣಿತಕ್ಕೂ‌ ಪ್ರಾರಂಭಿಸಿದರೆ ಅವರು ಹೇಗೆತಾನೆ ಇಷ್ಟೊಂದು ಕಷ್ಟವನ್ನು ಸಹಿಸಬಲ್ಲರು? ಯಾವುದರ ಪ್ರಥಮ ಗುಣವೇ ಪವಿತ್ರತೆಯಾಗಿದೆಯೋ ಅಂತಹ ಉಚ್ಚಸಾಧನೆಯನ್ನು ಅನುಕರಿಸಹೋಗಿ ಸಂಪೂರ್ಣ ತಾಮಸದಲ್ಲಿ ಮುಳುಗಿ ಹೋಗಿರುವರು. ನೀನು ಪ್ರತಿಯೊಂದು ಗ್ರಾಮ ಮತ್ತು ಪಟ್ಟಣಕ್ಕೂ ಹೋಗಿ ನೋಡು. ಎಲ್ಲೆಲ್ಲೂ ನಿನಗೆ ಈ ಕುಣಿತವೆ ಕಂಡುಬರುವುದು. ದೇಶದಲ್ಲಿ ತಮಟೆಗಳನ್ನು ತಯಾರಿಸುವುದಿಲ್ಲವೇ? ಭರತಖಂಡದಲ್ಲಿ ತುತ್ತೂರಿ, ಡಮರುಗಳು ಸಿಕ್ಕುವುದಿಲ್ಲವೇ? ಹುಡುಗರೆಲ್ಲಾ ಈ ತೂರ್ಯವಾಣಿಯನ್ನು ಕೇಳುವಂತೆ ಮಾಡಿ. ಬಾಲ್ಯದಿಂದ ಈ ಕೋಮಲಸ್ವಭಾವದ ಸಂಗೀತವನ್ನು ಕೇಳಿ ಕೇಳಿ, ಕೀರ್ತನೆಗಳನ್ನು ಕೇಳಿ ದೇಶವೆಲ್ಲಾ ನಾರಿಯರ ದೇಶವಾಗಿ ಪರಿಣಮಿಸಿದೆ. ಇದಕ್ಕಿಂತ ಬೇರೆ ಅಧೋಗತಿ ಯಾವುದಿದೆ? ಕವಿಯ ಕಲ್ಪನೆ ಕೂಡ ಈ ಚಿತ್ರವನ್ನು ಕಲ್ಪಿಸಲಾರದು. ತಮಟೆ ಮತ್ತು ಕಹಳೆಗಳನ್ನು ಬಾರಿಸಬೇಕು. ರಣರಂಗದಲ್ಲಿ ಬಾರಿಸಿದಂತೆ ಬಾರಿಸಬೇಕು. ‘ಮಹಾವೀರ, ಮಹಾವೀರ’ ಎಂಬ ಗರ್ಜನೆ ನಮ್ಮ ಬಾಯಿಂದ ಮೊರೆಯುತ್ತಿರಬೇಕು. ‘ಹರ, ಹರ, ಓಂ, ಓಂ.!’ ಎಂಬ ನಿನಾದದಿಂದ ದಶದಿಕ್ಕುಗಳೂ ಪ್ರತಿಧ್ವನಿತವಾಗುವಂತೆ ಮಾಡಬೇಕು. ಮಾನವನ ಮೃದುಹೃದಯವನ್ನು ಮಿಡಿಯುವಂತಹ ಸಂಗೀತ ಇಂದು ನಮಗೆ ಬೇಕಾಗಿಲ್ಲ – ಗೆಜ್ಜೆ ಧ್ವನಿಗಳ ನಾಟ್ಯ ಸಂಗೀತವನ್ನು ನಿಲ್ಲಿಸಿ, ದ್ರುಪದ ಸಂಗೀತಕ್ಕೆ ಕಿವಿಗೊಡುವಂತೆ ಮಾಡಬೇಕು. ಗಂಭೀರವಾದ ವೇದಘೋಷದ ಗರ್ಜನೆಯಿಂದ ದೇಶವನ್ನೆಲ್ಲಾ ತುಂಬಬೇಕು. ಪ್ರತಿಯೊಬ್ಬರಲ್ಲಿಯೂ ಪುರುಷಸಿಂಹ ಕೆಚ್ಚೆದೆ ವ್ಯಕ್ತವಾಗುವಂತೆ ಮಾಡಬೇಕು. ಈ ಧ್ಯೇಯದ ಮೇಲೆ ನಿನ್ನ ಶೀಲವನ್ನು ಕಟಬಲ್ಲೆಯಾದರೆ ಸಾವಿರಾರು ಜನ ನಿನ್ನನ್ನು ಹಿಂಬಾಲಿಸುವರು. ಆದರೆ ನಿನ್ನ ಧ್ಯೇಯದಿಂದ ಒಂದು ಅಂಗುಲವೂ ಹಿಮ್ಮೆಟ್ಟಬೇಡ. ಎದೆಗೆಡಬೇಡ, ಊಟಮಾಡುವಾಗ ಉಡುಪು ಧರಿಸುವಾಗ, ಮಲಗಿರುವಾಗ, ಹಾಡುವಾಗ, ಆಟವಾಡುವಾಗ, ಸುಖದುಃಖ ಎಲ್ಲಾ ಅವಸ್ಥೆಗಳಲ್ಲಿಯೂ ಅತ್ಯುಚ್ಚ ಆಂತರಿಕ ಶಕ್ತಿ ನಿನ್ನಲ್ಲಿ ವ್ಯಕ್ತವಾಗಲಿ. ಆಗಮಾತ್ರ ನೀನು ಮಹಾಶಕ್ತಿ ಜಗನ್ಮಾತೆಯ ಕೃಪೆಗೆ ಪಾತ್ರನಾಗುವೆ.” 

 ಶಿಷ್ಯ: “ಕೆಲವು ವೇಳೆ ಹೇಗೋ ನನಗೆ ಗೊತ್ತಿಲ್ಲದೆ ಮನಸ್ಸು ಅಧೋಗತಿಗಿಳಿಯುವುದು, ಸ್ವಾಮೀಜಿ.” 

 ಸ್ವಾಮೀಜಿ: “ಹಾಗಿದ್ದಲ್ಲಿ ಹೀಗೆ ಯೋಚಿಸು– ‘ನಾನು ಯಾರ ಶಿಷ್ಯ! ಅವರ ಜೊತೆ ನಾನು ವ್ಯವಹರಿಸುವೆ– ನನಗೆ ಅಂತಹ ದೌರ್ಬಲ್ಯ, ಮನಸ್ಸು ಅಧೋಗತಿಗಿಳಿಯುವುದು ಸರಿಯೇ?’ ಇಂತಹ ಮನಸ್ಸಿನ ದೌರ್ಬಲ್ಯವನ್ನು ಮೆಟ್ಟಿ ಎದ್ದು ನಿಲ್ಲು. ನನಗೆ ಶೌರ‍್ಯವಿದೆ– ನನಗೆ ಶಾಶ್ವತ ಬುದ್ಧಿ ಇದೆ – ನಾನು ಬ್ರಹ್ಮಜ್ಞಾನಿ, ಆತ್ಮಸಾಕ್ಷಾತ್ಕಾರ ಪಡೆದವನು’ ಹೀಗೆ ನಿನ್ನ ಸ್ಥಾನದ ಹಿರಿಮೆಯನ್ನು ಯೋಚಿಸು. ‘ಕಾಮಿನಿ ಕಾಂಚನ ತ್ಯಾಗಿಗಳಾದ ಶ‍್ರೀರಾಮಕೃಷ್ಣ ಪರಮಹಂಸರ ಒಡನಾಡಿಯಾಗಿದ್ದಂಥವರ ಶಿಷ್ಯ ನಾನು’ ಎಂದು ಚೆನ್ನಾಗಿ ಮನನಮಾಡಿಕೊ. ಇದರಿಂದ ಒಳ್ಳೆಯ ಪರಿಣಾಮ ಉಂಟಾಗುವುದು. ಯಾರಿಗೆ ಈ ಹೆಮ್ಮೆ ಇಲ್ಲವೊ ಅವನಲ್ಲಿ ಬ್ರಹ್ಮನ ಜಾಗೃತಿ ಆಗಿಲ್ಲ. ರಾಮಪ್ರಸಾದನ ಹಾಡನ್ನು ಕೇಳಿಲ್ಲವೆ? ಅವನು ಹೇಳುತ್ತಿದ್ದ “ಜಗನ್ಮಾತೆ ಆಳುತ್ತಿರುವ ಈ ಜಗತ್ತಿನಲ್ಲಿ ನಾನು ಅದನ್ನು ಕಂಡು ಅಂಜುವುದೆ.” ಇಂತಹ ಹೆಮ್ಮೆಯನ್ನು ಯಾವಾಗಲೂ ಮನಸ್ಸಿನಲ್ಲಿ ಮೆಲಕು ಹಾಕುತ್ತಿತ್ತು. ಅನಂತರ ಹೃದಯ ದೌರ್ಬಲ್ಯ ಮನೋದೌರ್ಬಲ್ಯ ಯಾವುದೂ ನಿನ್ನ ಬಳಿ ಸುಳಿಯಲಾರವು. ನಿನ್ನ ಮನಸ್ಸು ಎಂದೂ ದುರ್ಬಲವಾಗಲು ಅವಕಾಶ ಕೊಡಬೆಡ. ಮಹಾವೀರನನ್ನು ನೆನೆ. ಜಗನ್ಮಾತೆಯನ್ನು ಸ್ಮರಿಸು. ಆಗ ನಿನ್ನಿಂದ ಎಲ್ಲಾ ದೌರ್ಬಲ್ಯವೂ ಹೇಡಿತನವೂ ತಕ್ಷಣವೇ ಮಾಯವಾಗುವುವು. 

 ಈ ಮಾತುಗಳನ್ನು ಹೇಳುತ್ತಾ ಸ್ವಾಮೀಜಿ ಮಹಡಿಯಿಂದ ಕೆಳಗಿಳಿದುಬಂದು ಅಂಗಳದಲ್ಲಿ ತಾವು ದಿನವೂ ಕುಳಿತುಕೊಳ್ಳುತ್ತಿದ್ದ ಮಂಚದ ಮೇಲೆ ಕುಳಿತರು. ಅನಂತರ ಅಲ್ಲಿ ನೆರೆದಿದ್ದ ಸಂನ್ಯಾಸಿಗಳು ಬ್ರಹ್ಮಚಾರಿಗಳನ್ನೆಲ್ಲಾ ಕುರಿತು ಈ ರೀತಿ ಹೇಳಿದರು: “ಇಲ್ಲಿ ತೆರೆ ಕಳಚಿದ ಬ್ರಹ್ಮನಿದ್ದಾನೆ. ಅವನನ್ನು ನಂಬದೆ ಇತರ ವಸ್ತುಗಳ ಮೇಲೆ ಮನಸ್ಸಿಟ್ಟವರಿಗೆ ಧಿಃಕಾರ! ಓ! ಇಲ್ಲೇ ಬ್ರಹ್ಮ ಅಂಗೈ ಮೇಲಣ ನೆಲ್ಲಿಕಾಯಂತೆ ಪ್ರತ್ಯಕ್ಷವಾಗಿ ಇದ್ದಾನೆ– ನಿಮಗೆ ಕಾಣುವುದಿಲ್ಲವೇ? ಇಲ್ಲೆ!” 

 ಈ ಮಾತನ್ನು ಹೇಗೆ ಮನಮುಟ್ಟುವಂತೆ ಹೇಳಿದರೆಂದರೆ, ಅಲ್ಲಿದ್ದವರೆಲ್ಲಾ ಪರದೆಯ ಮೇಲೆ ಚಿತ್ರಿಸಿದ ಗೊಂಬೆಗಳಂತೆ ಅಲ್ಲಾಡದೆ ನಿಂತರು. ಎಲ್ಲರನ್ನೂ ಗಾಢಧ್ಯಾನ ಆವರಿಸಿದಂತೆ ಇತ್ತು. ಸ್ವಲ್ಪಕಾಲದ ಮೇಲೆ ಆ ತೀವ್ರತೆ ಕಡಿಮೆಯಾಗಿ ಪ್ರಜ್ಞೆಗೆ ಬಂದರು. ಸ್ವಲ್ಪ ನಂತರ ನಡೆಯುತ್ತಿದ್ದಾಗ ಸ್ವಾಮಿಗಳು ಶಿಷ್ಯನಿಗೆ ಹೇಳಿದರು: “ಇಂದು ನೋಡಿದೆಯಾ? ಅವರೆಲ್ಲಾ ಹೇಗೆ ಏಕಮನಸ್ಕರಾದರೆಂದು? – ಅವರೆಲ್ಲಾ ಶ‍್ರೀರಾಮಕೃಷ್ಣರ ಮಕ್ಕಳು. ಕೇವಲ ಆ ಮಾತುಗಳನ್ನು ಕೇಳಿದುದರಿಂದಲೇ ಅವರಿಗೆ ಸತ್ಯದ ಅರಿವಾಯಿತು.” 

 ಶಿಷ್ಯ: “ಅವರ ವಿಷಯ ಹಾಗಿರಲಿ, ನನ್ನ ಹೃದಯವೂ ಅನಿರ್ವಚನೀಯವಾದ ಆನಂದದಿಂದ ಪುಲಕಿತವಾಯಿತು. ಈಗ ಅದೊಂದು ಮಾಯವಾದ ಸ್ವಪ್ನದಂತಿದೆ.” 

 ಸ್ವಾಮೀಜಿ: “ಎಲ್ಲ ಸಕಾಲದಲ್ಲಿ ಬರುವುದು – ಈಗ ಕೆಲಸ ಮಾಡುತ್ತಾ ಹೋಗು. ಮೌಢ್ಯತೆ ಮತ್ತು ಅಜ್ಞಾನದಲ್ಲಿ ಮುಳುಗಿರುವವರನ್ನು ಎತ್ತಲು ಏನಾದರೂ ಕೊಂಚ ಕೆಲಸ ಮಾಡು. ಆಗ ನಿಮಗೆ ಅಂತಹ ಅನುಭವಗಳು ಬರುವುವು!” 

 ಶಿಷ್ಯ: “ಆ ಕರ್ಮಕೋಟೆಯೊಳಗೆ ಹೋಗಲು ಹೆದರುತ್ತೇನೆ – ಅಷ್ಟು ಶಕ್ತಿಯೂ ಇಲ್ಲ– ಶಾಸ್ತ್ರಗಳೂ ‘ಕರ್ಮದ ಹಾದಿ ಗಹನ’ ಎಂದು ಹೇಳುತ್ತವೆ.” 

 ಸ್ವಾಮೀಜಿ: “ಹಾಗಾದರೆ ನೀನೇನು ಮಾಡಬಯಸುವೆ.” 

 ಶಿಷ್ಯ: “ಎಲ್ಲಾ ಗ್ರಂಥಗಳ ಸತ್ಯವನ್ನೂ ಸಾಕ್ಷಾತ್ಕರಿಸಿಕೊಂಡಿರುವ ನಿಮ್ಮಂತಹ ಪೂಜ್ಯರೊಡನೆ ವಾಸಮಾಡುತ್ತಾ, ಚರ್ಚಿಸುತ್ತಾ ಅವನ್ನು ಕೇಳುವುದರಿಂದ, ಅವುಗಳ ಮನನ ಧ್ಯಾನದಿಂದ ಬ್ರಹ್ಮನನ್ನು ಈ ಜೀವನದಲ್ಲೇ ಸಾಕ್ಷಾತ್ಕಾರ ಮಾಡಿಕೊಳ್ಳಬೇಕು. ಇದಲ್ಲದೆ ಬೆರಾವುದಕ್ಕೂ ನನಗೆ ಉತ್ಸಾಹವೂ ಇಲ್ಲ, ಶಕ್ತಿಯೂ ಇಲ್ಲ.” 

 ಸ್ವಾಮೀಜಿ: “ನೀನಿದನ್ನು ಇಚ್ಛೆಪಟ್ಟಲ್ಲಿ ಹಾಗೆಯೇ ಮಾಡು. ಶಾಸ್ತ್ರಗಳ ಮೇಲೆ ನಿನ್ನ ಆಲೋಚನೆ ಮತ್ತು ನಿರ್ಧಾರಗಳನ್ನು ಇತರರಿಗೂ ಹೇಳು– ಅದರಿಂದ ಅವರಿಗೂ ಪ್ರಯೋಜನವಾಗುವುದು. ಎಲ್ಲಿಯವರೆಗೆ ದೇಹವಿರುವುದೋ ಅಲ್ಲಿಯವರೆಗೂ ಏನಾದರೊಂದು ಕೆಲಸ ಮಾಡುತ್ತಲೇ ಇರಬೇಕು. ಆದ್ದರಿಂದ ಇತರರಿಗೆ ಒಳ್ಳೆಯದಾಗುವಂತಹ ಕೆಲಸಗಳನ್ನು ಮಾಡು. ಶಾಸ್ತ್ರಗಳ ಮೇಲೆ ನಿನ್ನ ಸಾಕ್ಷಾತ್ಕಾರ ಮತ್ತು ನಿರ್ಧಾರಗಳು ಸತ್ಯವನ್ನು ಅರಸುತ್ತಿರುವ ಇನ್ನೊಬ್ಬನಿಗೆ ಉಪಯೋಗವಾಗಬಹುದು. ಅವುಗಳನ್ನು ಬರೆದಿಡುವುದರಿಂದ ಇತರರಿಗೆ ಸಹಾಯವಾಗುವುದು.” 

 ಶಿಷ್ಯ: “ಮೊದಲು ನನಗೆ ಸತ್ಯ ಸಾಕ್ಷಾತ್ಕಾರವಾಗಲಿ, ನಂತರ ನಾನು ಬರೆಯುವೆ. ಶ‍್ರೀರಾಮಕೃಷ್ಣರು, ‘ಅಧಿಕಾರದ ಮುದ್ರೆ ಇಲ್ಲದಿದ್ದಲ್ಲಿ ಯಾರೂ ನಿನ್ನ ಮಾತಿಗೆ ಕಿವಿಗೊಡುವುದಿಲ್ಲ’ ಎಂದು ಹೇಳುತ್ತಿದ್ದರು.” 

 ಸ್ವಾಮೀಜಿ: “ಈ ಪ್ರಪಂಚದಲ್ಲಿ ನೀನಿರುವ ಈ ಅವಸ್ಥೆಯಲ್ಲೇ ಆಧ್ಯಾತ್ಮಿಕ ಸಾಧನೆ ಮತ್ತು ತರ್ಕಾವಸ್ಥೆಯಲ್ಲೇ ಅನೇಕ ಮಂದಿ ಸಿಕ್ಕಿಕೊಂಡಿದ್ದಾರೆ. ಅ ಅವಸ್ಥೆಯಿಂದ ಅವರಿಗೆ ಮೀರಿ ಹೋಗಲಾಗುತ್ತಿಲ್ಲ. ನಿನ್ನ ಅನುಭವ ಮತ್ತು ಯೋಚನಾತರಂಗ ಕಡೆಯಪಕ್ಷ ಅವರಿಗಾದರೂ ಸಹಾಯವಾದೀತು. ನೀನು ಈ ಮಠದ ಸಾಧುಗಳೊಡನೆ ನಡೆಸಿದ ಸಂಭಾಷಣೆಯ ಸಾರವನ್ನು ಸರಳ ಭಾಷೆಯಲ್ಲಿ ಬರೆದರೆ ಅದೇ ಅನೇಕರಿಗೆ ಸಹಾಯವಾಗುವುದು.” 

 ಶಿಷ್ಯ: “ನೀವು ಹೀಗೆ ಇಚ್ಛೆಪಡುವುದರಿಂದ ಹಾಗೆ ಮಾಡುವೆ.” 

 ಸ್ವಾಮೀಜಿ: “ಇತರರಿಗೆ ಉಪಯೋಗವಾಗದ, ಅಜ್ಞಾನ ಕೂಪದಲ್ಲಿ ಮುಳುಗಿರುವ ಜನರ ಕಲ್ಯಾಣಕ್ಕೆ ನೆರವಾಗದ, ಕಾಮಿನಿಕಾಂಚನದ ಬಂಧನದಿಂದ ಅವರನ್ನು ಪಾರಾಗಲು ಸಹಾಯ ಮಾಡದ ಆಧ್ಯಾತ್ಮಿಕ ಸಾಧನೆ ಮತ್ತು ಸಾಕ್ಷಾತ್ಕಾರದಿಂದ ಪ್ರಯೋಜನವೇನು? ಎಲ್ಲಿಯವರೆಗೆ ಒಬ್ಬ ಜೀವಿ ಬಂಧನದಲ್ಲಿದ್ದಾನೋ ಅಲ್ಲಿಯವರೆಗೂ ನಿನಗೆ ಮೋಕ್ಷ ದೊರಕುವುದೆಂದು ತಿಳಿದಿರುವೆಯಾ? ಅವನಿಗೆ ಮುಕ್ತಿಯಾಗುವವರೆಗೂ– ಅದಕ್ಕೆ ಹಲವು ಜನ್ಮಗಳು ಬೇಕಾಗಬಹುದು– ನೀನು ಅವನಿಗೆ ಬ್ರಹ್ಮಸಾಕ್ಷಾತ್ಕಾರವಾಗುವವರೆಗೂ ಸಹಾಯ ಮಾಡಲು ಹುಟ್ಟುತ್ತಿರಬೇಕು. ಪ್ರತಿಯೊಬ್ಬ ಜೀವಿಯೂ ನಿನ್ನ ಭಾಗ. ಇತರರಿಗಾಗಿ ಮಾಡುವ ಎಲ್ಲಾ ಕೆಲಸದ ತತ್ತ್ವಾಧಾರ ಇದೇ. ನಿನ್ನ ಹೆಂಡತಿ ಮಕ್ಕಳು ನಿನ್ನವರೆಂದು ತಿಳಿದು ಅವರಿಗೆ ಹೃತ್ಪೂರ್ವಕವಾಗಿ ಕಲ್ಯಾಣವನ್ನು ಬಯಸುವಂತೆಯೇ ಪ್ರತಿಯೊಬ್ಬ ಜೀವಿಗೂ ಅದೇ ಬಗೆಯ ಪ್ರೇಮ ಎಂದಿಗೆ ನಿನ್ನಲ್ಲಿ ಉದಿಸುವುದೋ ಅಂದೆ ನಿನ್ನಲ್ಲಿ ಬ್ರಹ್ಮ ಜಾಗ್ರತನಾಗಿರುವನೆಂದು ತಿಳಿಯುವೆ. ಅಲ್ಲಿಯವರೆಗೂ ಅಲ್ಲ. ಎಂದು ಯಾವ ಜಾತಿಗಳನ್ನೂ ಲೆಕ್ಕಿಸದೆ ಎಲ್ಲರಿಗೂ ಒಳ್ಳೆಯದಾಗಲೆಂದು ಬಯಸುವ ಭಾವನೆ ನಿನ್ನಲ್ಲಿ ಉದಿಸುವುದೋ ಅಂದು ನೀನು ನಿನ್ನ ಗುರಿಯೆಡೆಗೆ ಹೋಗುತ್ತಿರುವೆ ಎಂದು ಭಾವಿಸುವೆ.” 

 ಶಿಷ್ಯ: “ಸ್ವಾಮೀಜಿ, ಇದೆಂತಹ ಪ್ರಚಂಡ ನಿರೂಪಣೆ! ಸರ್ವರಿಗೂ ಮುಕ್ತಿಯಾದಲ್ಲದೆ ವ್ಯಕ್ತಿಯೊಬ್ಬನಿಗೆ ಮುಕ್ತಿಯಿಲ್ಲ– ನಾನೆಂದೂ ಇಂತಹ ಅದ್ಭುತ ಪ್ರಸ್ತಾಪವನ್ನು ಕೇಳಿರಲಿಲ್ಲ.” 

 ಸ್ವಾಮೀಜಿ: “ಒಂದು ಬಗೆಯ ವೇದಾಂತಿಗಳು ಈ ಭಾವನೆ ಹೊಂದಿದ್ದಾರೆ. ಅವರ ಪ್ರಕಾರ ವ್ಯಕ್ತಿಯ ಮೋಕ್ಷ ನಿಜವಾದ ಪರಿಪೂರ್ಣವಾದ ಮೋಕ್ಷವಲ್ಲ. ಮೋಕ್ಷದಲ್ಲಿ ಸಾರ್ವತ್ರಿಕ ಮೋಕ್ಷವೇ ನಿಜವಾದ ಮುಕ್ತಿ. ಈ ಪಂಥದ ಒಳ್ಳೆಯದು ಕೆಟ್ಟದ್ದು ಎರಡನ್ನೂ ತೋರಿಸಬಹುದು.” 

 ಶಿಷ್ಯ: “ವೇದಾಂತದ ಪ್ರಕಾರ ವ್ಯಕ್ತಿಯ ಅವಸ್ಥೆಯೇ ಎಲ್ಲಾ ಬಂಧನದ ಮೂಲ. ಅನಂತರ ಕರ್ಮದ ಆಸೆ ಪರಿಣಾಮದ ಫಲವಾಗಿ ಆ ಎಲ್ಲೆಯಿರುವ ಅವಸ್ಥೆಯಲ್ಲಿ ಇರುವಂತೆ ತೋರುತ್ತದೆ. ವಿವೇಚನಾಶಕ್ತಿಯಿಂದ ಆ ಎಲ್ಲಾ ಅವಸ್ಥೆ ಮಾಯವಾದಾಗ ಜೀವ ಎಲ್ಲಾ ಬಗೆಯ ಬಂಧನಗಳಿಂದಲೂ ಬಿಡುಗಡೆ ಹೊಂದುವುದು. ಅಂದಮೇಲೆ ಜ್ಞಾನಾತೀತವಾದ ಆತ್ಮನಿಗೆ ಬಂಧನವೆಂದರೇನು? ಜೀವ ಜಗತ್ತು ಯಾರಿಗೆ ಸತ್ಯವಾಗಿದೆಯೋ ಅವರು ಎಲ್ಲರಿಗೂ ಮುಕ್ತಿ ದೊರೆತ ಹೊರತು ತನಗೆ ಸಿಕ್ಕುವುದಿಲ್ಲವೆಂದು ಯೋಚಿಸಬಹುದು. ಆದರೆ ಯಾವಾಗ ಮನಸ್ಸು ನಾಮರೂಪದ ಎಲ್ಲೆ ಕಟ್ಟನ್ನು ಮೀರಿ\break ಬ್ರಹ್ಮನಲ್ಲಿ ಲೀನವಾಗುವುದೋ ಆಗ ಅವನಿಗೆ ಭೇದಮಾಡಲು ಹೇಗೆ ಸಾಧ್ಯ? ಆದ್ದರಿಂದ ಯಾವುದೂ ಅವನ ಮುಕ್ತಿಗೆ ಅಡ್ಡಿ ಬರುವುದಿಲ್ಲ.” 

 ಸ್ವಾಮೀಜಿ: “ಹೌದು, ನೀನು ಹೇಳುವುದು ನಿಜ, ಮುಕ್ಕಾಲು ಪಾಲು ವೇದಾಂತಿಗಳೆಲ್ಲಾ ಹಾಗೆಯೇ ಅಭಿಪ್ರಾಯಪಡುತ್ತಾರೆ. ಅದು ಒಳ್ಳೆಯದೇ – ಈ ಅಭಿಪ್ರಾಯದಲ್ಲಿ ವ್ಯಕ್ತಿಯ ಮುಕ್ತಿಗೆ ಅಡ್ಡಿಯಿಲ್ಲ. ಆದರೆ ಯಾವ ಮನುಷ್ಯ ತನ್ನ ಜೊತೆಯಲ್ಲಿ ಇಡೀ ವಿಶ್ವವನ್ನೇ ಮುಕ್ತಿಗೆ ಒಯ್ಯುವನೆಂದು ಯೋಚಿಸುತ್ತಾನೋ ಅವನೆಂತಹ ಮಹಾತ್ಮನಿರಬೇಕು!” 

 ಶಿಷ್ಯ: “ಸ್ವಾಮೀಜಿ, ಇದು ನಮ್ಮ ಧೈರ್ಯವನ್ನು ಸೂಚಿಸಬಹುದೇ ಹೊರತು ಇದಕ್ಕೆ ಶಾಸ್ತ್ರ ಸಮ್ಮತವಿಲ್ಲ.” 

 ಸ್ವಾಮಿಗಳು ಬೇರಾವುದೋ ಯೋಚನೆಯಲ್ಲಿ ಮಗ್ನರಾದುದರಿಂದ ನನ್ನ ಮಾತುಗಳನ್ನು ಕೇಳಿಸಿಕೊಳ್ಳಲಿಲ್ಲ. ಅವರು ಸ್ವಲ್ಪಹೊತ್ತಿನ ಮೇಲೆ ಹೇಳಿದರು: “ಹಗಲೂ ರಾತ್ರಿ ಭ್ರಹ್ಮನನ್ನು ಕುರಿತು ಯೋಚಿಸು, ಧ್ಯಾನಿಸು, ಒಂದೇ ಏಕಾಗ್ರತೆಯಿಂದ ಧ್ಯಾನಿಸು. ಬಾಹ್ಯಜೀನದಲ್ಲಿ ಎಚ್ಚೆತ್ತಿರುವಾಗ ಇತರರಿಗಾಗಿ ಕೆಲಸಮಾಡು ಅಥವಾ ನಿನ್ನ ಮನಸ್ಸಿನಲ್ಲಿ ‘ಜಗತ್ತಿಗೂ ಜೀವನಿಗೂ ಕಲ್ಯಾಣವಾಗಲಿ.’ ಎಂದು ಮರಳಿ ಮರಳಿ ಪ್ರಾರ್ಥಿಸು. ‘ಎಲ್ಲರ ಮನಸ್ಸೂ ಬ್ರಹ್ಮನೆಡೆಗೆ ಹರಿಯಲಿ,’ ಎಂದು ಪುನಃ ಪುನಃ ಯೋಚಿಸು. ಇಂತಹ ನಿರರ್ಗಳ ಯೋಚನಾಪ್ರವಾಹದಿಂದ ಜಗತ್ತಿಗೆ ಕಲ್ಯಾಣವಾಗುವುದು. ಪ್ರಪಂಚದಲ್ಲಿ ಒಳ್ಳೆಯದಾವುದೂ, ಅದು ಕೆಲಸವಾಗಲಿ, ಯೋಚನೆಯಾಗಲಿ, ಯಾವುದೂ ನಿರರ್ಥಕವಲ್ಲ. ನಿನ್ನ ಆಲೋಚನಾಪ್ರವಾಹ ಬಹುಶಃ ಅಮೇರಿಕಾದಲ್ಲಿರುವವನೊಬ್ಬನ ಧಾರ್ಮಿಕ ಭಾವನೆಯನ್ನು ಜಾಗ್ರತಗೊಳಿಸಬಹುದು.” 

 ಶಿಷ್ಯ: “ಸ್ವಾಮೀಜಿ, ದಯವಿಟು ನನ್ನನ್ನು ಹರಸಿ, ನನ್ನ ಮನಸ್ಸು ಸತ್ಯದಲ್ಲಿ ಕೇಂದ್ರೀಕೃತವಾಗುವಂತೆ ಹರಸಿ.” 

 ಸ್ವಾಮೀಜಿ: “ಹಾಗೇ ಆಗಲಿ, ನಿನಗೆ ಉತ್ಕಟ ಆಕಾಂಕ್ಷೆ ಇದ್ದರೆ ಖಂಡಿತ ಆಗುವುದು.” 

\delimiter

 ಇಂದು ಶ‍್ರೀರಾಮಕೃಷ್ಣರ ವಾರ್ಷಿಕೋತ್ಸವ. ಸ್ವಾಮೀಜಿ ಕಂಡ ಕಟ್ಟಕಡೆಯ ಉತ್ಸವ. ಶಿಷ್ಯ ಶ‍್ರೀರಾಮಕೃಷ್ಣರ ಮೇಲೆ ಕಟ್ಟಿದ ಪ್ರಾರ್ಥನಾಶ್ಲೋಕವೊಂದನ್ನು ಸ್ವಾಮೀಜಿಗೆ ಒಪ್ಪಿಸಿದನಂತರ ಸ್ವಾಮೀಜಿ ಶಿಷ್ಯನಿಗೆ “ಪಾದಗಳು ಬಹು ನೋಯುತ್ತಿರುವುದರಿಂದ ಮೃದುವಾಗಿ ಒತ್ತು” ಎಂದು ಹೇಳಿದರು. 

 ಪದ್ಯವನ್ನೋದಿದ ಮೇಲೆ ಸ್ವಾಮೀಜಿ “ಚೆನ್ನಾಗಿ ಬರೆದಿರುವೆ” ಎಂದರು. 

\newpage

 ಸ್ವಾಮೀಜಿ ಖಾಯಿಲೆ ಬಹು ದಾರುಣವಾದುದನ್ನು ನೋಡಿ ಶಿಷ್ಯನ ಹೃದಯ ಹಿಂಡಿದಂತಾಯಿತು. ಅವನ ಅಂತರಂಗದ ಭಾವನೆಗಳನ್ನು ತಿಳಿದು ಸ್ವಾಮೀಜಿ “ನೀನೇನು ಯೋಚಿಸುತ್ತಿರುವೆ? ದೇಹ ಹುಟ್ಟಿ ಆಯಿತು – ಅದು ಸತ್ತೇ ತೀರಬೇಕು. ನನ್ನ ಅಭಿಪ್ರಾಯಗಳಲ್ಲಿ ಕೆಲವನ್ನಾದರೂ ನಿಮ್ಮಲ್ಲಿ ಬೇರೂರುವ ಹಾಗೆ ಮಾಡಿದ್ದರೆ ನಾನು ಹುಟಿದ್ದು ನಿರರ್ಥಕವಾಗಿಲ್ಲವೆಂದು ಭಾವಿಸುವೆ” ಎಂದರು. 

 ಶಿಷ್ಯ: “ನಾವು ನಿಮ್ಮ ಕೃಪೆಗೆ ಅರ್ಹರೇ? ನೀವು ನನ್ನ ಅರ್ಹತೆಯನ್ನು ಗಣನೆಗೆ ತಾರದೆ ನನ್ನನ್ನು ಆಶೀರ್ವದಿಸಿದರೆ ನಾನು ಧನ್ಯನೆಂದು ಭಾವಿಸುವೆ.” 

 ಸ್ವಾಮೀಜಿ: “ಯಾವಾಗಲೂ ತ್ಯಾಗವೇ ತಳಹದಿ ಎಂಬುದನ್ನು ನೆನಪಿನಲ್ಲಿಡು, ಎಲ್ಲಿಯವರೆಗೆ ಈ ಭಾವನೆ ಬೇರೂರುವುದಿಲ್ಲವೋ ಅಲ್ಲಿಯವರೆಗೆ ಬ್ರಹ್ಮ ಅಥವಾ ಇತರ ಭೂಮಂಡಲದ ಯಾವ ದೇವತೆಯೂ ಮುಕ್ತಿ ಪಡೆಯಲು ಸಾಧ್ಯವಿಲ್ಲ.” 

 ಶಿಷ್ಯ: “ಪ್ರತಿದಿನವೂ ನಿಮ್ಮ ಬಾಯಿಂದ ಇದನ್ನು ಕೇಳುತ್ತಿದ್ದರೂ ನಾನು ಸಾಕ್ಷಾತ್ಕರಿಸಿಕೊಳ್ಳಲಾರದೆ ಹೋದೆನೆಂದು ನನಗೆ ತೀವ್ರ ವೇದನೆಯುಂಟಾಯಿತು.” 

 ಸ್ವಾಮೀಜಿ: “ತ್ಯಾಗ ಬಂದೇ ಬರುವುದು. ಆದರೆ ಸಕಾಲದಲ್ಲಿ ಬರುವುದು. ‘ಕಾಲೇನಾತ್ಮನಿ ವಿದಂತಿ’ – ಸಕಾಲದಲ್ಲಿ ನಮ್ಮಲ್ಲಿಯೇ ನಮಗೆ ಮುಕ್ತಿ ದೊರೆಯುವುದು. ಹಿಂದಿನ ಜನ್ಮದ ಸಂಸ್ಕಾರಗಳು ತೀರಿದ ಮೇಲೆ ತ್ಯಾಗ ಎದೆಯಲ್ಲಿ ಚಿಗುರುವುದು.” 

 ಸ್ವಲ್ಪ ಹೊತ್ತಿನನಂತರ – “ನೀನು ಹೊರಗೇಕೆ ಜನರ ಆ ದೊಡ್ಡ ದೊಂಬಿಗೆ ಹೋಗಬೇಕು. ನೀನು ನನ್ನೊಡನೆಯೇ ಇರು. ನಿರಂಜನನಿಗೆ ಬಾಗಿಲಲ್ಲಿಯೇ ಇರಲು ಹೇಳು. ಯಾರೂ ನನ್ನನ್ನು ಇಂದು ತೊಂದರೆ ಕೊಡದೆ ಇರಲಿ” ಎಂದರು. ಅನಂತರ ಕೆಳಗಣ ಸಂಭಾಷಣೆ ಶಿಷ್ಯನಿಗೂ ಸ್ವಾಮೀಜಿಗೂ ನಡೆಯಿತು. 

 ಸ್ವಾಮೀಜಿ: “ಇನ್ನು ಮುಂದೆ ವಾರ್ಷಿಕೋತ್ಸವ ಬೇರೆ ವಿಧವಾಗಿ ನಡೆದರೆ ವಾಸಿ ಎಂದು ನನಗೆ ತೋರುತ್ತದೆ. ಒಂದು ದಿನಕ್ಕೆ ಬದಲು ಉತ್ಸವ ಐದಾರು ದಿನ ನಡೆದರೆ ವಾಸಿ. ಮೊದಲನೆಯ ದಿನ ಧರ್ಮ ಶಾಸ್ತ್ರಗಳನ್ನು ಓದುವುದು, ಅದಕ್ಕೆ ಅರ್ಥ ಹೇಳುವುದು. ಎರಡನೆಯ ದಿನ ವೇದವೇದಾಂತಗಳ ಮೇಲೆ, ಅವುಗಳ ಸಮಸ್ಯೆ ಮತ್ತು ಪರಿಹಾರಗಳ ಮೇಲೆ ಸಂಭಾಷಣೆ. ಮೂರನೆಯ ದಿನ ಪ್ರಶ್ನೆಯ ತರಗತಿಗಳನ್ನೇರ್ಪಡಿಸುವುದು. ನಾಲ್ಕನೆಯ ದಿನ ಉಪನ್ಯಾಸಗಳನ್ನಿಡುವುದು. ಕಡೆಯ ದಿನ ಈಗ ನಡೆಯುವಂತೆ ಹಬ್ಬವನ್ನಾಚರಿಸುವುದು. ದುರ್ಗಾ ಪೂಜೆಯಂತೆಯೇ ನಾಲ್ಕೈದು ದಿನ ನಡೆಯಬಹುದು. ಮೇಲೆ ಹೇಳಿದ ಹಾಗೆ ಮಾಡಿದರೆ ಶ‍್ರೀರಾಮಕೃಷ್ಣರ ಶಿಷ್ಯರ ಹೊರತು ಮತ್ತಾರೂ ಕಡೆಯ ದಿನ ಬಿಟ್ಟು ಉಳಿದ ದಿನಗಳಲ್ಲಿ ಅದಕ್ಕೆ ಸಹಕರಿಸಲಾಗದು. ಆದರೆ ಅದರಿಂದೇನೂ ಭಾದಕವಿಲ್ಲ. ಒಂದು ದೊಡ್ಡ ಜನ ದೊಂಬಿ ಸೇರಿದರೆ ಶ‍್ರೀರಾಮಕಷ್ಣರ ಸಂದೇಶ ಸಾರಿದಂತಾಗುವುದಿಲ್ಲ.” 

 ಶಿಷ್ಯ: “ಎಂತಹ ಸುಂದರವಾದ ಆಲೋಚನೆ. ಮುಂದಿನ ಬಾರಿ ತಮ್ಮ ಇಚ್ಛೆಯಂತೆಯೆ ಮಾಡಬಹುದು.” 

 ಸ್ವಾಮೀಜಿ: “ನೋಡು ಮಗು, ನೀವೆಲ್ಲಾ ಅದರಂತೆ ಮಾಡಿ. ನನಗೆ ಅದರ ಮೇಲೆ ಅಂತಹ ಮನಸ್ಸೇನೂ ಇಲ್ಲ.” 

 ಶಿಷ್ಯ: “ಸ್ವಾಮೀಜಿ. ಈ ಬಾರಿ ಅನೇಕ ಸಂಗೀತಗಾರರ ಗುಂಪು ಬಂದಿದೆ.” 

 ಈ ಮಾತನ್ನು ಕೇಳಿ ಸ್ವಾಮೀಜಿ ಎದ್ದುನಿಂತು ಕಿಟಕಿಯ ಕಂಬಿಗಳನ್ನು ಹಿಡಿದುಕೊಂಡು ಗುಂಪುಕೂಡಿದ ಭಕ್ತಗಣವನ್ನು ನೋಡಿದರು. ಸ್ವಲ್ಪಹೊತ್ತಿನ ಮೇಲೆ ಕುಳಿತುಕೊಂಡರು. 

 ಸ್ವಾಮೀಜಿ: “ಶ‍್ರೀರಾಮಕೃಷ್ಣರ ಲೀಲಾ ನಾಟಕದಲ್ಲಿ ನೀವೆಲ್ಲಾ ನಟರು. ಇದಾದ ಮೇಲೆ ಇವರ ಮಾತೇ ಏಕೆ, ಜನರು ನಿಮ್ಮ ಹೆಸರನ್ನೂ ಪವಿತ್ರವಾಗಿ ಭಾವಿಸುವರು. ನೀನು ಬರೆಯುತ್ತಿರುವ ಈ ಶ್ಲೋಕಗಳನ್ನು ಪ್ರೇಮ ಮತ್ತು ಜ್ಞಾನ ಸಂಪಾದನೆಗಾಗಿ ಜನರಿಂದ ಓದಲ್ಪಡುತ್ತದೆ. ಆತ್ಮಜ್ಞಾನ ಪಡೆಯುವುದೇ ಜೀವನದ ಮಹೋದ್ದೇಶವೆಂದು ತಿಳಿ. ಜಗದ್ಗುರುಗಳಾದ ಅವತಾರಪುರುಷರಲ್ಲಿ ನಿನಗೆ ಭಕ್ತಿಯಿದ್ದರೆ ಅದು ತನ್ನಷ್ಟಕ್ಕೆ ತಾನೆ ಸಕಾಲದಲ್ಲಿ ಆವಿರ್ಭಾವವಾಗುವುದು.” 

 ಶಿಷ್ಯ: “ಸ್ವಾಮೀಜಿ, ನನಗೆ ಜ್ಞಾನಲಾಭವಾಗುವುದೆ?” 

 ಸ್ವಾಮೀಜಿ: “ಶ‍್ರೀರಾಮಕೃಷ್ಣರ ಆಶೀರ್ವಾದದಿಂದ ನಿನಗೆ ಜ್ಞಾನಲಾಭವಾಗುವುದು. ನಿನಗೆ ಪ್ರಾಪಂಚಿಕ ಜೀವನದಲ್ಲಿ ಹೆಚ್ಚು ಸುಖ ಸಿಕ್ಕುವುದಿಲ್ಲ.” 

 ಶಿಷ್ಯ: ನೀವು ಮನಸ್ಸಿನ ದುರ್ಬಲತೆಯನ್ನು ನಾಶಮಾಡಲು ಅನುಗ್ರಹಿಸಿದರೆ ಮಾತ್ರ ನನಗೆ ಭರವಸೆಯುಂಟಾಗುವುದು.” 

 ಸ್ವಾಮೀಜಿ: “ಅಂಜಿಕೆಯೇಕೆ? ನೀನು ಅನಿರೀಕ್ಷಿತವಾಗಿ ಇಲ್ಲಿಗೆ ಬಂದಿರುವೆ. ನೀನು ಮುಕ್ತಿ ಹೊಂದೇ ತೀರುವೆ.” 

 ಶಿಷ್ಯ: (ದೈನ್ಯದಿಂದ) “ನೀವು ನನ್ನನ್ನು ಕಾಪಾಡಬೇಕು. ಈ ಜನ್ಮದಲ್ಲೇ ಅಜ್ಞಾನದಿಂದ ನನ್ನನ್ನು ಉದ್ಧರಿಸಬೇಕು.” 

 ಸ್ವಾಮೀಜಿ: “ಯಾರು ಯಾರನ್ನು ಕಾಪಾಡಲು ಸಾಧ್ಯ ಹೇಳು? ಗುರು ಕೆಲವು ತೆರೆಗಳನ್ನು ತೆಗೆದುಹಾಕಲು ಮಾತ್ರ ಸಾಧ್ಯ. ಯಾವಾಗ ಉಪಾಧಿಗಳೆಲ್ಲ ಹೋಗುವುವೋ ಆಗ ನಮ್ಮ ಆತ್ಮ ಸ್ವಯಂಪ್ರಕಾಶಮಾನವಾಗಿ ಸೂರ‍್ಯನಂತೆ ಪ್ರಜ್ವಲಿಸುವುದು.” 

 ಶಿಷ್ಯ: “ಹಾಗಾದರೆ ಶಾಸ್ತ್ರದಲ್ಲಿ ನಾವೇಕೆ ಕೃಪೆ ಎಂಬುದನ್ನು ಓದುತ್ತೇವಲ್ಲ?” 

 ಸ್ವಾಮೀಜಿ: “ಕೃಪೆ ಎಂದರೆ ಇದು: ಯಾರಿಗೆ ಆತ್ಮಸಾಕ್ಷಾತ್ಕಾರವಾಗಿದೆಯೊ ಅವನು ದೊಡ್ಡ ಶಕ್ತಿಯ ಭಂಡಾರವಾಗಿರುವನು. ಆತನನ್ನು ಕೇಂದ್ರವಾಗಿಟ್ಟುಕೊಂಡು ಒಂದು ವೃತ್ತವನ್ನೆಳೆದರೆ ಯಾರು ಆ ವೃತ್ತದೊಳಗೆ ಬರುವರೋ ಅವರು ಆತನ ಅಭಿಪ್ರಾಯಗಳ ಆಕರ್ಷಣೆಗೊಳಗಾಗುವರು. ಆಗ ಹೆಚ್ಚು ತಪಸ್ಸಿಲ್ಲದೆ ಆತನ ಅದ್ಭುತ ಆಧ್ಯಾತ್ಮಿಕತೆಯ ಫಲವನ್ನು ಹೊಂದುವರು. ಇದನ್ನು ಕೃಪೆ ಎಂದು ನೀನು ಕರೆಯುವುದಾದರೆ ಹಾಗೇ ಕರೆ.” 

 ಶಿಷ್ಯ: “ಇದಕ್ಕಿಂತ ಹೆಚ್ಚಿನ ಪದವಿ ಇಲ್ಲವೆ?” 

 ಸ್ವಾಮೀಜಿ: “ಇದೆ, ಅವತಾರ ಪುರುಷರು ಬಂದಾಗ ಅವರೊಡನೆ ವಿಶ್ವ ನಾಟಕದ ಸಹಾಯಾರ್ಥವಾಗಿ ಅನೇಕ ಮುಕ್ತಾತ್ಮರೂ ಬರುವರು. ಲಕ್ಷಾಂತರ ಜೀವಿಗಳನ್ನು ಅಜ್ಞಾನದಿಂದ ಉದ್ಧಾರಮಾಡಿ, ಇವರಿಗೆ ಅದೇ ಜನ್ಮದಲ್ಲೇ ಮುಕ್ತಿಕೊಡಲು ಅವತಾರ ಪುರುಷನಿಗೆ ಮಾತ್ರ ಸಾಧ್ಯ. ಇದಕ್ಕೆ ಕೃಪೆ ಎನ್ನುವರು. ನಿನಗೆ ಅರ್ಥವಾಯಿತೆ?” 

 ಶಿಷ್ಯ: “ಆಯಿತು ಸ್ವಾಮೀಜಿ. ಅಂತಹವರ ದರ್ಶನ ಲಾಭದಿಂದ ಧನ್ಯರಾಗದವರ ಪಾಡೇನು? 

 ಸ್ವಾಮೀಜಿ: “ಅದಕ್ಕೆ ದಾರಿ ಆತನನ್ನು ಪ್ರಾರ್ಥಿಸುವುದೊಂದೆ– ಆತನನ್ನು ಪ್ರಾರ್ಥಿಸುವುದರಿಂದ ಅನೇಕ ಮಂದಿ ಆತನ ದರ್ಶನ ಲಾಭ ಪಡೆದಿದ್ದಾರೆ. ಮಾನವ ಶರೀರದಲ್ಲೇ ನೋಡಿ ಆತನ ಕೃಪೆಗೆ ಪಾತ್ರರಾಗಿದ್ದಾರೆ.” 

 ಶಿಷ್ಯ: “ಶ‍್ರೀರಾಮಕೃಷ್ಣರ ನಿರ‍್ಯಾಣವಾದಮೇಲೆ ನೀವೆಂದಾದರೂ ಅವರನ್ನು ನೋಡಿದ್ದೀರಾ?” 

 ಸ್ವಾಮೀಜಿ ಪವಾಹಾರಿ ಬಾಬರವರಿಂದ ದೀಕ್ಷೆ ಪಡೆಯುವ ನಿರ್ಧಾರ ಮಾಡಿದಾಗ ಶ‍್ರೀರಾಮಕೃಷ್ಣರು ತಮಗೆ ಮೂರುಬಾರಿ ಕಾಣಿಸಿಕೊಂಡ ಘಟನೆಯನ್ನು ವಿವರಿಸಿದರು. 

\delimiter

 ಪೂರ್ವಬಂಗಾಳದಿಂದ ಹಿಂತಿರುಗಿ ಬಂದಾಗಿನಿಂದ ಸ್ವಾಮೀಜಿ ಮಠದಲ್ಲಿಯೇ ಇದ್ದು ಮಗುವಿನಂತೆಯೇ ಜೀವನ ನಡೆಸುತ್ತಿದ್ದರು. ಪ್ರತಿವರ್ಷವೂ ಕೆಲವು ಸಂತಾಲ ಕೆಲಸಗಾರರು ಮಠದಲ್ಲಿ ಕೆಲಸ ಮಾಡುತ್ತಿದ್ದರು. ಸ್ವಾಮೀಜಿ ಅವರೊಂದಿಗೆ ಹಾಸ್ಯ ಮಾಡುತ್ತಾ ಅವರ ಸುಖದುಃಖಗಳನ್ನು ವಿಚಾರಿಸುವುದರಲ್ಲಿ ಸಂತೋಷಿಸುತ್ತಿದ್ದರು. ಒಂದು ದಿನ ಕೆಲವು ಮಂದಿ ದೊಡ್ಡ ಮನುಷ್ಯರು ಕಲ್ಕತ್ತೆಯಿಂದ ಸ್ವಾಮೀಜಿಯನ್ನು ಸಂದರ್ಶಿಸಲು ಬಂದರು. ಅವರು ಸ್ವಾಮೀಜಿ ಸಂತಾಲರೊಡಾನೆ ಸಂತೋಷದಿಂದ ಮಾತನಾಡುತ್ತಿದ್ದಾಗ ಆ ದೊಡ್ಡಮನುಷ್ಯರು ಬಂದಿರುವ ಸಮಾಚಾರವನ್ನು ಕೇಳಿ “ನನಗೆ ಈಗ ಹೋಗಲಾಗುವುದಿಲ್ಲ. ನಾನು ಈ ಜನರೊಡನೆ ಸಂತೋಷವಾಗಿದ್ದೇನೆ” ಎಂದರು. ಅಂದು ನಿಜವಾಗಿಯೂ ಸ್ವಾಮೀಜಿ ಸಂತಾಲರನ್ನು ಬಿಟ್ಟು ಆ ಮನುಷ್ಯರನ್ನು ನೋಡಲು ಹೋಗಲೇ ಇಲ್ಲ. 

 ಆ ಸಂತಾಲರಲ್ಲಿ ಒಬ್ಬನ ಹೆಸರು “ಕಿಷ್ಟ” ಎಂದು. ಸ್ವಾಮೀಜಿ ಆ ಕಿಷ್ಟನನ್ನು ತುಂಬಾ ಪ್ರೀತಿಸುತ್ತಿದ್ದರು. ಸ್ವಾಮೀಜಿ ಅವರೊಡನೆ ಮಾತನಾಡಲು ಹೋದಾಗ ಕಿಷ್ಟನು ಸ್ವಾಮೀಜಿಗೆ ಹೇಳುತ್ತಿದ್ದ: “ಓ ಸ್ವಾಮೀಜಿ, ನಾನು ಕೆಲಸಮಾಡುತ್ತಿರುವಾಗ ಬರಬೇಡಿ. ನಿಮ್ಮೊಡನೆ ಮಾತನಾಡುತ್ತಾ ಸಮಯವೆಲ್ಲಾ ಕಳೆದುಹೋಗಿ ಕೆಲಸವೇ ಸಾಗುವುದಿಲ್ಲ. ಕೊನೆಗೆ ಇದರ ಮೇಲ್ವಿಚಾರಕರಾದ ಸ್ವಾಮೀಜಿ ಬಂದಾಗ ನಮಗೆ ಚೆನ್ನಾಗಿ ಬೈಗುಳ ಬೀಳುತ್ತದೆ.” ಸ್ವಾಮೀಜಿ ಇದರಿಂದ ಎದೆಕರಗಿ ಹೇಳಿದರು: “ಇಲ್ಲ ಇಲ್ಲ, ಅವರೇನನ್ನೂ ಬೈಯುವುದಿಲ್ಲ. ನಿಮ್ಮ ಊರಿನ ವಿಚಾರ ಸ್ವಲ್ಪ ಹೇಳು.” ಮಾತನಾಡುತ್ತಾ ಹಾಗೇ ಅವರ ಸಂಸಾರಜೀವನದ ವಿಚಾರವನ್ನೂ ಕೇಳುತ್ತಿದ್ದರು. 

 ಒಂದು ದಿನ ಸ್ವಾಮೀಜಿ ಕಿಷ್ಟನಿಗೆ ಹೇಳಿದರು: “ನೀವು ಒಂದು ದಿನ ಇಲ್ಲಿ ಊಟಮಾಡುವಿರಾ?” ಕಿಷ್ಟ ಹೇಳಿದ: “ನಾವು ನೀವು ಮುಟ್ಟಿದ ಪದಾರ್ಥ ತಿನ್ನುವುದಿಲ್ಲ, ನೀವು ನಮ್ಮ ಆಹಾರಕ್ಕೆ ಉಪ್ಪನ್ನು ಹಾಕಿದರೆ ನಮ್ಮ ಜಾತಿ ಹೋಗುವುದು!” ಸ್ವಾಮೀಜಿ “ನೀವು ಉಪ್ಪನ್ನೇಕೆ ತಿನ್ನಬೇಕು, ಉಪ್ಪು ಹಾಕದೆ ಪಲ್ಯ ಮಾಡಿದರೆ ನೀವು ಉಣ್ಣಬಹುದಲ್ಲವೆ?” ಎಂದು ಕೇಳಿದರು. ಕಿಷ್ಟ ಒಪ್ಪಿಕೊಂಡ. ಅನಂತರ ಸ್ವಾಮೀಜಿ ಅಪ್ಪಣೆಯಂತೆ ಚಪಾತಿ, ಪಲ್ಯ, ಮಿಠಾಯಿ, ಮೊಸರು ಮುಂತಾದುವನ್ನು ತಂದು ಸಂತಾಲರ ಊಟಕ್ಕಾಗಿ ಸಿದ್ಧಪಡಿಸಲಾಯಿತು. ಸ್ವಾಮೀಜಿ ಅವರೆಲ್ಲಾ ತಮ್ಮ ಮುಂದೆಯೇ ಊಟಮಾಡುವಂತೆ ಮಾಡಿದರು. ಊಟಮಾಡುತ್ತಿದ್ದಾಗ ಕಿಶ್ಟ ಕೇಳಿದ: “ನಿಮಗೆ ಹೇಗೆ ಈ ಪದಾರ್ಥ ಬಂತು? ಇದರಂತಹ ರುಚಿಯನ್ನು ನಾನೆಂದೂ ಕಂಡೇ ಇರಲಿಲ್ಲ.” ಅವರಿಗೆ ಹೊಟ್ಟೆತುಂಬಾ ಊಟಮಾಡಿಸುತ್ತಾ ಸ್ವಾಮೀಜಿ ಹೇಳಿದರು “ನೀವೆಲ್ಲಾ ನಾರಾಯಣರು, ದೇವರ ಆವಿರ್ಭಾವ. ಇಂದು ನಾನು ನಾರಾಯಣನಿಗೆ ಊಟವನ್ನರ್ಪಿಸಿದ್ದೇನೆ.” ದರಿದ್ರನಾರಾಯಣನಿಗೆ ಸೇವೆ, ಬಡವರಲ್ಲಿ ದೇವರನ್ನು ಕಾಣುವುದೆನ್ನುತ್ತಿದ್ದುದನ್ನು ಅಂದು ಸ್ವಾಮೀಜಿ ಕಾರ್ಯರೂಪದಲ್ಲಿ ನೆರವೇರಿಸಿದರು. 

 ಊಟವಾದ ನಂತರ ಸಂತಾಲರು ವಿಶ್ರಾಂತಿಗೆ ತೆರಳಿದರು. ಸ್ವಾಮೀಜಿ ಶಿಷ್ಯನನ್ನುದ್ದೇಶಿಸಿ ಹೇಳಿದರು: “ಅವರಲ್ಲಿ ಎಷ್ಟು ಸರಳತೆ ಇದೆ ನೋಡಿದೆಯಾ? ಅವರ ಕಷ್ಟವನ್ನು ಸ್ವಲ್ಪವಾದರೂ ಕಡಿಮೆಮಾಡಬಲ್ಲಿರಾ? ಇಲ್ಲದಿದ್ದಲ್ಲಿ ಕಾವಿಬಟ್ಟೆ ತೊಟ್ಟು ಪ್ರಯೋಜನವೇನು? ಇತರರ ಕಲ್ಯಾಣಕ್ಕೋಸುಗ ಎಲ್ಲವನ್ನೂ ತ್ಯಾಗಮಾಡುವುದೇ ನಿಜವಾದ ಸಂನ್ಯಾಸ. ಅವರು ಜೀವನದಲ್ಲಿ ಏನೊಂದು ಒಳ್ಳೆಯದನ್ನೂ ಅನುಭವಿಸಿಲ್ಲ. ಕೆಲವುವೇಳೆ ನನಗೆ ಈ ಮಠವನ್ನೆಲ್ಲಾ ಮಾರಿ ಬಂದ ದುಡ್ಡನ್ನು ಬಡಬಗ್ಗರಿಗೆ ಹಂಚಿ ಬಿಡಬೇಕೆನ್ನಿಸುವುದು. ನಾವು ಮಠವನ್ನು ನಮ್ಮ ಆಶ್ರಮವನ್ನಾಗಿ ಮಾಡಿಕೊಂಡೆವು. ಅಯ್ಯೊ! ಈ ದೇಶದ ಜನರಿಗೆ ತಿನ್ನಲಿಕ್ಕೆ ಹಿಟ್ಟಿಲ್ಲ. ನಮಗೆ ತುತ್ತೆತ್ತಲಿಕ್ಕಾದರೂ ಹೇಗೆ ಮನಸ್ಸು ಬರುವುದು? ನಾನು ಪಾಶ್ಚಾತ್ಯದೇಶಗಳಲ್ಲಿದ್ದಾಗ ಜಗನ್ಮಾತೆಗೆ ಪ್ರಾರ್ಥಿಸುತಿದ್ದೆ. ‘ಇಲ್ಲಿಯ ಜನರು ಹೂವಿನ ಸುಪ್ಪತ್ತಿಗೆಯಲ್ಲಿ ಮಲಗುತ್ತಿದ್ದಾರೆ. ಎಲ್ಲಾ ಬಗೆಯ ರುಚಿಕರವಾಗ ವಸ್ತುಗಳನ್ನೂ ತಿನ್ನುತ್ತಾರೆ. ಅವರು ಅನುಭವಿಸದ ಸುಖವೇನಿದೆ? ನಮ್ಮ ದೇಶದ ಜನಾಂಗದವರೋ ಹೊಟ್ಟೆಗಿಲ್ಲದೆ ಸಾಯುತ್ತಿದ್ದಾರೆ. ತಾಯಿ, ಅವರಿಗೆ ಬೇರೆ ಮಾರ್ಗವೇ ಇಲ್ಲವೆ?’ ನಾನು ಪರದೇಶಕ್ಕೆ ಧರ‍್ಮ ಬೋಧಿಸಲು ಹೋದುದಕ್ಕೆ ಒಂದು ಮುಖ್ಯ ಉದ್ದೇಶ ನಮ್ಮ ದೇಶದ ಜನರ ಹೊಟ್ಟೆ ತುಂಬಿಸಲು ಅಲ್ಲೇನಾದರೂ ಸಹಾಯವಾಗಬಹುದೆಂದು. ನನ್ನ ದೇಶದ ಬಡಜನರು ಉಪವಾಸದಿಂದ ನರಳುತ್ತಿರುವುದನ್ನು ನೋಡಿದಾಗ ನನಗೆ ಈ ವಿಧಿಯುಕ್ತ ಪೂಜೆ, ಪಾಂಡಿತ್ಯವೆಲ್ಲವನ್ನೂ ಕಿತ್ತೊಗೆದು ಗ್ರಾಮಗ್ರಾಮಗಳಿಗೂ ಹೋಗಿ ನಮ್ಮ ಸಾಧನೆ ಮತ್ತು ಶೀಲದ ಬಲದಿಂದ ಅಲ್ಲಿಯ ಹಣವಂತರ ಮೇಲೆ ಪ್ರಭಾವವನ್ನು ಬೀರಿ ಹಣವನ್ನು ಕೂಡಿಸಿ ಬಡವರಿಗೆ ಸೇವೆ ಮಾಡುತ್ತಾ ಇಡೀ ಜೀವನವನ್ನೆಲ್ಲಾ ಕಳೆಯಬೇಕೆನ್ನಿಸುವುದು. 

 “ಅಯ್ಯೋ! ಯಾರೂ ದೇಶದ ಬಡಬಗ್ಗರ ವಿಚಾರವಾಗಿ ಯೋಚಿಸುವುದಿಲ್ಲ. ಅವರೇ ಪಟ್ಟಣದ ಮೂಲಾಧಾರ. ಅವರ ಶ್ರಮದಿಂದ ಆಹಾರ ಉತ್ಪತ್ತಿಯಾಗುತ್ತಿದೆ. ಈ ಬಡಜನರು ಈ ಗುಡಿಸುವವರು ಕೂಲಿಕಾರರು ಒಂದು ದಿನ ತಮ್ಮ ಕೆಲಸವನ್ನು ನಿಲ್ಲಿಸಲಿ. ಇಡೀ ಪಟ್ಟಣವೇ ಹಾಹಾಕಾರವೇಳುವುದು. ಆದರೆ ಯಾರೂ ಅವರಿಗೆ ಕನಿಕರ ತೋರುವುದಿಲ್ಲ. ಅವರ ಕಷ್ಟಗಳಲ್ಲಿ ಸಂತೈಸುವವರಿಲ್ಲ. ಸ್ವಲ್ಪನೋಡು, ಹಿಂದೂಗಳು ಸಹಾನುಭೂತಿ ತೋರದುದರ ಫಲವಾಗಿ ಮದ್ರಾಸಿನಲ್ಲಿ ಸಾವಿರಾರು ಜನ ಹಿಂದುಳಿದವರು ಕ್ರೈಸ್ತರಾಗುತ್ತಿದ್ದಾರೆ. ಇದಕ್ಕೆ ಕೇವಲ ಹಸಿವೇ ಮುಖ್ಯ ಕಾರಣವಲ್ಲ. ನಮ್ಮಿಂದ ಅವರಿಗೆ ಸ್ವಲ್ಪವೂ ಸಹಾನುಭೂತಿ ದೊರಕದುದೇ ಇದಕ್ಕೆ ಕಾರಣ. ಹಗಲೂ ರಾತ್ರಿ ನಾವು ಅವರಿಗೆ ‘ಮುಟ್ಟಬೇಡಿ, ಮುಟ್ಟಬೇಡಿ’ ಎಂದು ಕೂಗುತ್ತಿದ್ದೇವೆ. ದೇಶದಲ್ಲಿ ಸ್ವಲ್ಪವಾದರೂ ಕನಿಕರ, ಸಹಾನುಭೂತಿ ಉಳಿದಿರುವುದೇನು? ಒಮ್ಮೊಮ್ಮೆ ನನಗೆ ಈ ‘ಮುಟ್ಟಬೇಡಿ’ ಎಂಬ ಧರ್ಮದ ಎಲ್ಲೆಯನ್ನು ಕತ್ತರಿಸಿ ಹೊರಗೆ ನಿಂತು ಎಲ್ಲರಿಗೂ ‘ಯಾರು ಬಡವರೋ ದುಃಖಿಗಳೋ, ದೀನರೋ, ದಲಿತರೋ ಎಲ್ಲರೂ ಬನ್ನಿ’ ಎಂದು ಕರೆದು ಎಲ್ಲರನ್ನೂ ಶ‍್ರೀರಾಮಕೃಷ್ಣರ ಹೆಸರಿನಲ್ಲಿ ಒಂದುಗೂಡಿಸಬೇಕೆನ್ನಿಸುವುದು. ಅವರು ಮುಂದಕ್ಕೆ ಬಂದ ಹೊರತು ಮಾತೆಯೂ ಎಚ್ಚರಗೊಳ್ಳುವುದಿಲ್ಲ. ಇವರಿಗೆಲ್ಲಾ ಹೊಟ್ಟೆಬಟ್ಟೆಗಾಗುವಷ್ಟನ್ನು ನಾವು ಮಾಡಲಾರದೆ ಹೋದೆವು‌! ನಾವೇನು ಮಾಡಿದ್ದೇವೆ? ಅಯ್ಯೋ! ಅವರಿಗೆ ಸುಖ ಏನೆಂಬುದೇ ಗೊತ್ತಿಲ್ಲ. ರಾತ್ರಿ ಹಗಲು ದುಡಿದರೂ ಸಾಕಾಗುವಷ್ಟು ಆಹಾರ ಬಟ್ಟೆ ಹೊಂದಲು ಶಕ್ತರಾಗಿಲ್ಲ. ನಾವು ಅವರನ್ನು ಕಣ್ತೆರೆಯುವಂತೆ ಮಾಡೋಣ. ಈ ಹಗಲಿನಷ್ಟೇ ಸ್ಪಷ್ಟವಾಗಿ ನೋಡುತ್ತಿರು– ಎಲ್ಲರಲ್ಲೂ ಇರುವುದು ಒಬ್ಬನೇ ಬ್ರಹ್ಮ – ಅವರಲ್ಲಿ – ನನ್ನಲ್ಲಿ – ಒಂದೇ ಶಕ್ತಿ ನೆಲೆಸಿದೆ. ವ್ಯತ್ಯಾಸ ಅದರ ಅವಿರ್ಭಾವನೆಯ ಹೆಚ್ಚು ಕಡಿಮೆಯಲ್ಲಿದೆ. ಇಡೀ ದೇಹದಲ್ಲೆಲ್ಲಾ ರಕ್ತ ಸಂಚರಿಸದ ಹೊರತು ಯಾವ ದೇಹವೂ ಸರಿಯಾಗಿ ಕೆಲಸ ಮಾಡದು. ಒಂದು ಅಂಗಕ್ಕೆ ಪಾರ್ಶ್ವವಾಯು ಬಂದರೆ ಉಳಿದ ಎಲ್ಲಾ ಅಂಗಗಳು ಸರಿಯಾಗಿದ್ದರೂ ಆ ದೇಹದಿಂದ ಹೆಚ್ಚೇನೂ ಮಾಡಲಾಗುವುದಿಲ್ಲ. ಇದನ್ನು ಚೆನ್ನಾಗಿ ನೆನಪಿನಲ್ಲಿಡು.” 

 ಶಿಷ್ಯ: “ಸ್ವಾಮೀಜಿ, ದೇಶದಲ್ಲಿ ಇಷ್ಟೊಂದು ಬಗೆಯ ಪಂಗಡ ಆದರ್ಶಗಳು ಇರುವುದರಿಂದಲೇ ಅವರನ್ನೆಲ್ಲಾ ಒಂದುಗೂಡಿಸಿ ಶಾಂತಿ ಬರುವಂತೆ ಮಾಡುವುದು ಬಹು ಕಷ್ಟದ ಕೆಲಸ.” 

 ಸ್ವಾಮೀಜಿ: (ಕೋಪದಿಂದ) “ನೀನು ಯಾವ ಕೆಲಸವನ್ನೇ ಆಗಲಿ ಕಷ್ಟವೆಂದು ತಿಳಿದಿದ್ದರೆ ಇಲ್ಲಿಗೆ ಬರಬೇಡ. ಭಗವತ್ಕೃಪೆಯಿಂದ ಎಲ್ಲಾ ಸುಗಮವಾಗುವುದು. ನಿನ್ನ ಕೆಲಸ ಜಾತಿ ಮತ ವರ್ಣಗಳನ್ನು ಲೆಕ್ಕಿಸದೆ ಬಡಬಗ್ಗರಿಗೆ ಸೇವೆ ಮಾಡುವುದಾಗಿದೆ. ಅದರ ಪರಿಣಾಮವನ್ನು ಯೋಚಿಸುವ ಆವಶ್ಯಕತೆಯೇನಿಲ್ಲ. ಕೇವಲ ಕೆಲಸ ಮಾಡುತ್ತಾ ಹೋಗುವುದು ನಿನ್ನ ಕರ್ತವ್ಯ. ಅನಂತರ ಎಲ್ಲವೂ ತನ್ನಷ್ಟಕ್ಕೆ ತಾನೇ ಹಿಂಬಾಲಿಸುವುದು. ನನ್ನ ಕೆಲಸ ನಿರ್ಮಿಸುವುದು, ಧ್ವಂಸವಲ್ಲ. ವಿಶ್ವದ ಚರಿತ್ರೆಯನ್ನು ಓದಿ ನೋಡು. ಒಂದು ದೇಶದ ಸಂದಿಗ್ಧ ಕಾಲದಲ್ಲಿ ಒಬ್ಬ ಮಹಾತ್ಮನಾದ ವ್ಯಕ್ತಿ ಅದರ ಜೀವನಾಡಿಯಾಗಿರುತ್ತಾನೆ. ಆತನ ಆದರ್ಶಗಳಿಂದ ನೂರಾರು ಜನ ದೀಪ್ತಿಗೊಂಡು ಜಗತ್ತಿಗೆ ಒಳ್ಳೆಯದನ್ನು ಮಾಡಿದ್ದಾರೆ. ನೀವೆಲ್ಲಾ ಬುದ್ಧಿವಂತರಾದ ಹುಡುಗರು ಅನೇಕ ದಿನಗಳಿಂದ ಇಲ್ಲಿಗೆ ಬರುತ್ತಿರುವಿರಿ. ಹೇಳಿ, ಇತರರ ಸೇವೆಗಾಗಿ ಒಂದು ಜನ್ಮವನ್ನು ಅರ್ಪಿಸಲಾಗುವುದಿಲ್ಲವೆ? ಮುಂದಿನ ಜನ್ಮದಲ್ಲಿ ವೇದಾಂತ ಮತ್ತು ಧರ್ಮಶಾಸ್ತ್ರಗಳನ್ನೆಲ್ಲಾ ಓದಬಹುದು. ಈ ಜನ್ಮವನ್ನು ಇತರರ ಸೇವೆಗಾಗಿ ಅರ್ಪಿಸಿ. ಆಗ ನೀವು ಇಲ್ಲಿಗೆ ಬಂದುದು ವ್ಯರ್ಥವಾಗಲಿಲ್ಲವೆಂದು ನನಗೆ ಗೊತ್ತಾಗುವುದು.” 

 ಈ ಮಾತುಗಳನ್ನಾಡುತ್ತಾ ಸ್ವಾಮೀಜಿ ಮೌನವಾಗಿ ಗಾಢ ಯೋಚನಾಮಗ್ನರಾದರು. ಸ್ವಲ್ಪ ಹೊತ್ತಿನ ಮೇಲೆ ಅವರು ಹೇಳಿದರು: “ಅಷ್ಟೊಂದು ಕಠಿಣ ಸಾಧನೆ ಮಾಡಿದ ಮೇಲೆ ಇದು ನಿಜವಾಗಿ ಸತ್ಯ – ದೇವರು ಎಲ್ಲಾ ಜೀವಿಗಳಲ್ಲೂ ಇದ್ದಾನೆ. ಅವನಿಗಿಂತ ಬೇರೆ ದೇವರಿಲ್ಲ. ಯಾರು ಜೀವರನ್ನು ಸೇವಿಸುತ್ತಾನೋ ಅವನು ದೇವರನ್ನು ಸೇವಿಸಿದಂತೆ ಎಂಬುದು ಮನದಟ್ಟಾಯಿತು, ಸ್ವಲ್ಪ ಹೊತ್ತು ಸುಮ್ಮನಿದ್ದು ಶಿಷ್ಯನನ್ನುದ್ದೇಶಿಸಿ “ಇಂದು ನಾನು ಹೇಳಿದುದನ್ನು ನಿನ್ನ ಹೃದಯದಲ್ಲಿ ಚೆನ್ನಾಗಿ ಬರೆದಿಡು. ನೀನಿದನ್ನು ಮರೆಯದಂತೆ ನೋಡಿಕೊ ಎಂದರು.” 

 ಶಿಷ್ಯ ಸ್ವಾಮೀಜಿ ಕೊಠಡಿಯಲ್ಲೇ ಹಿಂದಿನ ರಾತ್ರಿಯನ್ನು ಕಳೆದಿದ್ದ, ಬೆಳಗಿನ ಝಾವ ನಾಲ್ಕು ಗಂಟೆಗೆ ಸ್ವಾಮೀಜಿ ಶಿಷ್ಯನನ್ನೆಬ್ಬಿಸಿ “ಘಂಟೆಯನ್ನು ಹೊಡೆದು ಬ್ರಹ್ಮಚಾರಿಗಳನ್ನೂ ಸಾಧುಗಳನ್ನೂ ನಿದ್ರೆಯಿಂದ ಎಬ್ಬಿಸು” ಎಂದರು. ಅಪ್ಪಣೆಯ ಮೇಲೆ ಶಿಷ್ಯ ನಿದ್ರಿಸುತ್ತಿದ್ದ ಸಾಧುಗಳ ಹತ್ತಿರ ಘಂಟೆ ಬಾರಿಸಿದ. ಆಶ್ರಮದ ನಿವಾಸಿಗಳೆಲ್ಲಾ ಪೂಜಾ ಮಂದಿರಕ್ಕೆ ತ್ವರೆಯಿಂದ ಧ್ಯಾನ ಮಾಡಲು ಹೊರಟರು. ಸ್ವಾಮೀಜಿ ಅಪ್ಪಣೆಯಂತೆ ಶಿಷ್ಯ ಸ್ವಾಮಿ ಬ್ರಹ್ಮಾನಂದರ ಹಾಸಿಗೆ ಹತ್ತಿರ ಜೋರಾಗಿ ಗಂಟೆ ಬಾರಿಸಿದ. ಅವರು ಗಟ್ಟಿಯಾಗಿ ಒದರಿದರು, “ಅಯ್ಯೋ ರಾಮ! ಆ ಬಾಂಗ್ಲಾನಿಂದ ನಮಗೆ ಈ ಮಠದಲ್ಲಿ ಉಳಿಗಾಲವಿಲ್ಲ,” ಎಂದು. ಶಿಷ್ಯನಿಂದ ಇದನ್ನು ಕೇಳಿದ ಸ್ವಾಮೀಜಿ “ಒಳ್ಳೆಯ ಕೆಲಸ ಮಾಡಿದೆ” ಎಂದು ಹೇಳುತ್ತಾ ಹೊಟ್ಟೆ ಹುಣ್ಣಾಗುವಂತೆ ನಕ್ಕರು. 

 ಸಂನ್ಯಾಸಿಗಳೆಲ್ಲಾ, ಸ್ವಾಮಿ ಬ್ರಹ್ಮಾನಂದರೂ ಕೂಡ ಅಷ್ಟು ಹೊತ್ತಿಗಾಗಲೇ ಧ್ಯಾನಾರೂಢರಾಗಿದ್ದರು. ಸ್ವಾಮೀಜಿಗೆ ಒಂದು ಪ್ರತ್ಯೇಕ ಆಸನವನ್ನೇರ್ಪಡಿಸಲಾಗಿತ್ತು. ಸ್ವಾಮೀಜಿ ಪೂರ್ವಾಭಿಮುಖವಾಗಿ ಕುಳಿತುಕೊಂಡು ತಮ್ಮ ಮುಂದಿದ್ದ ಆಸನವನ್ನು ಶಿಷ್ಯನಿಗೆ ತೋರಿಸಿ “ಹೋಗು ಅಲ್ಲಿ ಕುಳಿತು ಧ್ಯಾನ ಮಾಡು” ಎಂದರು. 

\newpage

 ಆಸನಾರೂಢರಾದ ಸ್ವಲ್ಪ ಹೊತ್ತಿನಲ್ಲಿಯೇ ಸ್ವಾಮೀಜಿ ಸಂಪೂರ್ಣ ಶಾಂತ ಚಿತ್ತರಾಗಿ ಪ್ರತಿಮೆಯಂತೆ ಸ್ಥಿರವಾದರು. ಅವರ ಉಸಿರಾಡುವಿಕೆ ಬಹು ನಿಧಾನವಾಗುತ್ತಾ ಬಂತು. ಎಲ್ಲರೂ ತಮ್ಮ ತಮ್ಮ ಸ್ಥಳಗಳಲ್ಲಿ ಕುಳಿತಿದ್ದರು. 

 ಸುಮಾರು ಒಂದೂವರೆ ಗಂಟೆಯಾದ ಮೇಲೆ ಸ್ವಾಮೀಜಿ “ಶಿವ, ಶಿವ” ಎಂದು ಹೇಳುತ್ತಾ ಧ್ಯಾನದಿಂದ ಮೇಲೆದ್ದರು. ಅವರ ಕಣ್ಣುಗಳು ಪ್ರಜ್ವಲಿಸುತ್ತಿದ್ದವು. ಮುಖಭಾವ ಪ್ರಶಾಂತವಾಗಿ, ಸೌಮ್ಯವಾಗಿ, ಗಂಭೀರವಾಗಿದ್ದಿತು. ಶ‍್ರೀರಾಮಕೃಷ್ಣರಿಗೆ ಪ್ರಣಾಮ ಮಾಡಿ, ಕೆಳಗೆ ಇಳಿದು ಬಂದು ಮಠದ ವರಾಂಡದಲ್ಲಿ ಶತಪಥ ತಿರುಗುತ್ತಿದ್ದರು. ಹೊತ್ತಿನ ತರುವಾಯ ಶಿಷ್ಯನಿಗೆ ಹೇಳಿದರು, “ನೋಡಿದೆಯಾ, ಈಗ ಮಠದ ಸಾಧುಗಳು ಧ್ಯಾನ ಮುಂತಾದುವನ್ನು ಹೇಗೆ ಮಾಡುತ್ತಿರುವರೆಂದು. ಧ್ಯಾನ ಗಾಢವಾಗುತ್ತಾ ಹೋದ ಹಾಗೆಲ್ಲಾ ಅನೇಕ ಅದ್ಭುತವಾದ ಅನುಭವಗಳಾಗುತ್ತವೆ. ಬಾರಾನಗರದ ಮಠದಲ್ಲಿ ಧ್ಯಾನಿಸುತ್ತಿದ್ದಾಗ ನಾನು ಇಡ ಮತ್ತು ಪಿಂಗಳ ಎಂಬ ನರಗಳನ್ನು ನೋಡಿದೆ. ಸ್ವಲ್ಪ ಸಾಧನೆಯಿಂದಲೆ ನಾವದನ್ನು ನೋಡಬಹುದು. ಅನಂತರ ಸುಷುಮ್ನಾವನ್ನು ಯಾರು ನೋಡುವರೋ ಅವರು ತಮಗಿಷ್ಟ ಬಂದುದನ್ನು ನೋಡಬಹುದು. ಯಾರಿಗೆ ಗುರುವಿನಲ್ಲಿ ಅವಿಚ್ಛಿನ್ನವಾದ ಭಕ್ತಿ ಇದೆಯೋ ಅವನಿಗೆ ಜಪ ಧ್ಯಾನ ಮುಂತಾದವು ತಾವಾಗಿಯೇ ಬರುವುವು. ಅದಕ್ಕಾಗಿ ಅವನು ಹೋರಾಡಬೇಕಾಗಿಲ್ಲ. ಗುರುವೇ ಬ್ರಹ್ಮ, ಗುರುವೇ ವಿಷ್ಣು, ಗುರುವೇ ಮಹೇಶ್ವರ.” 

 ಅನಂತರ ಶಿಷ್ಯ ಸ್ವಾಮಿಗಳಿಗೆ ತಂಬಾಕನ್ನು ಸಿದ್ಧಪಡಿಸಲು, ಸ್ವಾಮಿಗಳು ಅದನ್ನು ಸೇದುತ್ತಾ ಹೇಳಿದರು: “ಆಂತರ್ಯದಲ್ಲಿರುವುದು ಸಿಂಹ–ನಿರಂತರ ಪವಿತ್ರ ಜ್ಯೋತಿರ್ಮಯ ಮುಕ್ತಾತ್ಮ– ಅವನನ್ನು ಪ್ರತ್ಯಕ್ಷವಾಗಿ ಧ್ಯಾನ ಮತ್ತು ಚಿತ್ತೈಕಾಗ್ರತೆಯಿಂದ ಯಾರು ಸಾಕ್ಷಾತ್ಕಾರಿಸಿಕೊಳ್ಳುವರೋ ಅವರಿಗೆ ಈ ಮಾಯಾಪ್ರಪಂಚ ಮಾಯವಾಗುವುದು. ಅವನು ಸರ್ವರಲ್ಲಿಯೂ ಮೂರ್ತಿಭವಿಸಿದ್ದಾನೆ. ಹೆಚ್ಚು ಸಾಧನೆ ಮಾಡಿದಷ್ಟೂ ಬೇಗ ಕುಂಡಲಿನಿ ಜಾಗೃತಿಗೊಳ್ಳುವುದು. ಯಾವಾಗ ಈ ಶಕ್ತಿ ಶಿರವನ್ನು ಸೇರುವುದೋ ಆಗ ಅವನ ದೃಷ್ಟಿಗೆ ಯಾವುದೂ ಆಡಚಣೆಯಾಗುವುದಿಲ್ಲ – ಅವನಿಗೆ ಆತ್ಮಸಾಕ್ಷಾತ್ಕಾರವಾಗುವುದು.” 

 ಶಿಷ್ಯ: “ಸ್ವಾಮೀಜಿ, ನಾನು ಇವುಗಳನ್ನೆಲ್ಲಾ ಧರ್ಮಗ್ರಂಥಗಳಲ್ಲಿ ಓದಿದ್ದೇನೆ ಅಷ್ಟೆ. ಯಾವುದೂ ಸಾಕ್ಷಾತ್ಕಾರವಾಗಿಲ್ಲ.” 

 ಸ್ವಾಮೀಜಿ: “ಯೋಗ್ಯಕಾಲದಲ್ಲಿ ಅದು ಬಂದೇ ಬರುವುದು. ಕೆಲವರಿಗೆ ನಿಧಾನವಾಗಿ ಕೆಲವರಿಗೆ ಬೇಗ ಬರುವುದು. ಆದರೆ ಎಂದಿಗೂ ಬಿಡುವುದಿಲ್ಲವೆಂಬ ನಿರ್ಧಾರದಿಂದ ಅದನ್ನೇ ಗಟ್ಟಿಯಾಗಿ ಹಿಡಿದಿರಬೇಕು. ಇದೇ ನಿಜವಾದ ಪುರುಷಕಾರ. ತಡೆಯಿಲ್ಲದೆ ಎಣ್ಣೆಯ ಪ್ರವಾಹದಂತೆ ಮನಸ್ಸು ಒಂದು ವಸ್ತುವಿನಲ್ಲೆ ಏಕಾಗ್ರ ಮಾಡಬೇಕು. ಸಾಧಾರಣ ಮನುಷ್ಯನ ಮನಸ್ಸು ನಾನಾ ವಸ್ತುಗಳ ಮೇಲೆ ಹರಿದು ಹಂಚಿಹೋಗಿರುತ್ತದೆ. ಧ್ಯಾನಕಾಲದಲ್ಲಿ ಪ್ರಾರಂಭದಲ್ಲಿ ಮನಸ್ಸು ಚಂಚಲವಾಗೇ ಆಗುವುದು. ಆದರೆ ಮನಸ್ಸಿನಲ್ಲಿ ಯಾವ ಆಸೆ ಬೇಕಾದರೂ ಬರಲಿ. ಶಾಂತನಾಗಿ ಕುಳಿತು ಯಾವ ಭಾವನೆಗಳು ಬರುತ್ತವೆಂಬುದನ್ನು ನೋಡು. ಹೀಗೆ ಪರೀಕ್ಷಿಸುತ್ತಾಹೋದ ಮೇಲೆ ಮನಸ್ಸೂ ಶಾಂತವಾಗಿ ಇತರ ಯೋಚನಾತರಂಗಗಳು ಬರುವುದಿಲ್ಲ. ಈ ತರಂಗಗಳು ಮನಸ್ಸಿನ ಯೋಚನಾಶಕ್ತಿಯ ಪ್ರತಿನಿಧಿಗಳು. ನೀನು ಹಿಂದೆ ಯಾವ ಯಾವ ಗಾಢ ಆಲೋಚನೆಯಲ್ಲಿ ಮುಳುಗಿದ್ದೆಯೋ ಅವೆಲ್ಲಾ ನಿನ್ನ ಸುಪ್ತಾವಸ್ಥೆಯ ತರಂಗಗಳಾಗಿ ಮಾರ್ಪಟ್ಟಿರುತ್ತವೆ – ಅವೇ ಧ್ಯಾನದ ಸಮಯದಲ್ಲಿ ಮೇಲೆ ಬರುತ್ತವೆ. ಧ್ಯಾನದಲ್ಲಿರುವಾಗ ಮನಸ್ಸಿನಲ್ಲಿ ಈ ರೀತಿ ಅಲೆಗಳು ಅಥವಾ ಆಲೋಚನೆಗಳು ಏಳುವುದರಿಂದಲೇ ನಿನ್ನ ಮನಸ್ಸು ಅನೇಕ ಭಾವನೆಗಳ ಮೇಲೆ ಕೇಂದ್ರೀಕೃತವಾಗಿರುತ್ತದೆ. ಇದಕ್ಕೆ ವಿಕಲ್ಪ ಅಥವಾ ಆಂದೋಲನದ ಮೇಲಿನ ಧ್ಯಾನ ಎಂದು ಹೆಸರು. ಆದರೆ ಯಾವಾಗ ಮನಸ್ಸು ಎಲ್ಲಾ ಬಗೆಯ ವೃತ್ತಿಗಳಿಂದಲೂ ಬಿಡುಗಡೆ ಹೊಂದುವುದೋ ಆಗ ಅದು ಆಂತರಿಕ ಶಕ್ತಿಯಲ್ಲಿ ಲೀನವಾಗುವುದು. ಇದೇ ಅಖಂಡ – ನಂತರ ಜ್ಞಾನ. ತನ್ನ ನೆಲೆ ತಾನೇ ಆಗಿರುವುದು. ಇದೇ ನಿರ್ವಿಕಲ್ಪ ಸಮಾಧಿ – ಎಲ್ಲಾ ಬಗೆಯ ಕರ್ಮಗಳಿಂದಲೂ ವಿಮುಕ್ತವಾಗಿರುವುದು. ಶ‍್ರೀರಾಮಕೃಷ್ಣರಲ್ಲಿ ನಾವು ಅನೇಕ ಬಾರಿ ಈ ಎರಡು ಬಗೆಯ ಸಮಾಧಿಗಳನ್ನೂ ನೋಡುತ್ತಿದ್ದೆವು. ಅವರು ಈ ಸ್ಥಿತಿ ಹೊಂದಲು ಹೋರಾಡಬೇಕಾಗಿರಲಿಲ್ಲ. ಅವು ಸಹಜವಾಗಿಯೇ ಅವರಿಗೆ ಯಾವಾಗೆಂದರೆ ಆಗ ಬರುತ್ತಿದ್ದವು. ಅದೊಂದು ಅದ್ಭುತ ಪ್ರಸಂಗ. ಅವರನ್ನು ಪ್ರತ್ಯಕ್ಷ ನೋಡಿದ ಮೇಲೆ ನಮಗೆ ಈ ವಿಷಯಗಳನ್ನು ಗ್ರಹಿಸಲು ಸಾಧ್ಯವಾಯಿತು. ಪ್ರತಿದಿನವೂ ಏಕಾಂಗಿಯಾಗಿ ಧ್ಯಾನ ಮಾಡು. ಆಗ ಎಲ್ಲವೂ ತಮ್ಮಷ್ಟಕ್ಕೆ ತಾವೇ ಗೋಚರವಾಗುವುವು, ಜಗನ್ಮಾತೆ ಜ್ಯೋತಿರ್ಮಯ ಮೂರ್ತಿ ನಿನ್ನಲ್ಲಿ ನಿದ್ರಿಸುತ್ತಿರುವಳು. ಅದಕ್ಕೇ ನೀನೇನೂ ತಿಳಿದುಕೊಳ್ಳಲಾರೆ. ಅವಳೇ ಕುಂಡಲಿನಿ. ಧ್ಯಾನಕ್ಕೆ ಮುಂಚೆ ನೀನು ನರಶುದ್ಧಿ ಮಾಡುವಾಗ ಮಾನಸಿಕವಾಗಿ ನಿನ್ನ ಮೂಲಾಧಾರದಲ್ಲಿರುವ ಕುಂಡಲಿನಿಯನ್ನು ಹೊಡೆದೆಬ್ಬಿಸು– ‘ಏಳು ತಾಯಿ ಏಳು’ ಎನ್ನು. ನಿಧಾನವಾಗಿ ಈ ಸಾಧನೆ ಮಾಡಬೇಕು. ಧ್ಯಾನದ ವೇಳೆಯಲ್ಲಿ ನಿನ್ನ ಉದ್ರೇಕ ಭಾವನೆಗಳನ್ನು ಆದಷ್ಟು ನಿಗ್ರಹಿಸು. ಅವು ಬಹಳ ಅಪಾಯಕಾರಿ. ಯಾರು ಹೆಚ್ಚು ಉದ್ವೇಗಪರರೋ ಅವರಲ್ಲಿ ಕುಂಡಲಿನಿ ಬಹುಬೇಗ ಜಾಗೃತಗೊಳ್ಳುವುದು. ಹಾಗೆ ಅಷ್ಟೇ ಬೇಗ ಅದು ಕೆಳಕ್ಕೂ ಹೋಗುವುದು. ಅದು ಕೆಳಕ್ಕೆ ಬಂದಾಗ ಭಕ್ತನನ್ನು ತುಂಬಾ ಶೋಚನೀಯಾವಸ್ಥೆಯಲ್ಲಿ ಬಿಟ್ಟು ಹೋಗುವುದು. ಅದಕ್ಕೇ ಕೀರ್ತನೆ ಮುಂತಾದ ಈ ಉದ್ರೇಕಪರ ಭಾವಕ್ಕೆ ಪರವಶರಾಗುವುದರಲ್ಲಿ ಒಂದು ದೊಡ್ಡ ಕೊರತೆ ಇದೆ. ಕುಣಿತ ಮುಂತಾದುವುಗಳಿಂದ ಆ ಕ್ಷಣಕ್ಕೆ ಈ ಶಕ್ತಿ ಮೇಲೇರುವುದೇನೋ ನಿಜ. ಆದರೆ ಅದೊಂದೂ ಸ್ಥಿರವಲ್ಲ. ಅದಕ್ಕೆ ಬದಲು ಅದು ತನ್ನ ಸ್ಥಾನಕ್ಕೆ ಹಿಂತಿರುಗಿ ಬಂದು ವ್ಯಕ್ತಿಯ ಕಾಮವನ್ನು ಉದ್ರೇಕಿಸುವುದು. ಅಮೇರಿಕಾದಲ್ಲಿ ನನ್ನ ಉಪನ್ಯಾಸಗಳನ್ನು ಕೆಳಿ ತಾತ್ಕಾಲಿಕ ಉದ್ವೇಗದಿಂದ ಪ್ರೇಕ್ಷಕರಲ್ಲಿ ಹಲವರು ಭಾವವಶರಾಗುತ್ತಿದ್ದರು. ಕೆಲವರು ವಿಗ್ರಹದಂತೆ ನಿಶ್ಚಲರಾಗಿಬಿಡುತ್ತಿದ್ದರು. ಅನಂತರ ವಿಚಾರಿಸಿದ ಮೇಲೆ ಅವರಲ್ಲಿ ಅನೆಕರಿಗೆ ಈ ಅವಸ್ಥೆ ಕಳೆದ ಕೂಡಲೆ ಕಾಮಾಸಕ್ತಿ ಹೆಚ್ಚಾಗುತ್ತಿತ್ತೆಂದು ತಿಳಿಯಿತು. ಆದರೆ ಸರಿಯಾಗಿ ಜಪಧ್ಯಾನ ಮಾಡದ ಕಾರಣ ಹೀಗಾಗುವುದು.” 

 ಶಿಷ್ಯ: “ಸ್ವಾಮೀಜಿ ಯಾವ ಧರ್ಮಗ್ರಂಥಗಳಲ್ಲೂ ನಾನು ಈ ಆಧ್ಯಾತ್ಮಿಕ ಸಾಧನೆಯ ರಹಸ್ಯವನ್ನು ಓದಿರಲಿಲ್ಲ – ಇಂದು ಅನೇಕ ಹೊಸ ವಿಷಯಗಳನ್ನು ಕೇಳಿದೆ.” 

 ಸ್ವಾಮೀಜಿ: “ಧರ್ಮ ಶಾಸ್ತ್ರಗಳು ಆಧ್ಯಾತ್ಮಿಕ ಸಾಧನೆಯ ಎಲ್ಲಾ ರಹಸ್ಯವನ್ನೂ ಒಳಗೊಂಡಿರುವವೆಂದು ತಿಳಿದೆಯಾ? ಇವೆಲ್ಲಾ ಗುರುಶಿಷ್ಯ ಪರಂಪರೆಯಿಂದ ರಹಸ್ಯವಾಗಿ ಕೊಡಲ್ಪಡುತ್ತವೆ. ಬಹು ಎಚ್ಚರಿಕೆಯಿಂದ ಧ್ಯಾನ ಜಪವನ್ನು ಮಾಡು. ಸುವಾಸನಾ ಪುಷ್ಪಗಳನ್ನು ಇಟ್ಟು ಊದಿನ ಕಡ್ಡಿ ಹಚ್ಚಿಸು. ಪ್ರಾರಂಭದಲ್ಲಿ ಮನಸ್ಸು ಪರಿಶುದ್ಧವಾಗಲು ಈ ಬಾಹ್ಯ ಸಹಾಯವನ್ನು ತೆಗೆದುಕೊ; ನೀನು ನಿನ್ನ ಗುರು ಮತ್ತು ಇಷ್ಟ ದೇವತೆಯ ಮಂತ್ರವನ್ನು ಜಪಿಸುತ್ತಿರುವಾಗ ಹೇಳು: ಸರ್ವ ಜೀವಿಗಳಿಗೂ ಶಾಂತಿ ಬರಲಿ, ವಿಶ್ವದಲ್ಲೆಲ್ಲಾ ಶಾಂತಿ ನೆಲಸಲಿ. ಮೊದಲು ಈ ಶುಭಾಶಯಗಳನ್ನು ಉತ್ತರ, ದಕ್ಷಿಣ, ಪೂರ್ವ ಪಶ್ಚಿಮ, ಮೇಲೆ ಕೆಳಗೆ–ಎಲ್ಲಾ ಕಡೆಗೂ ಕಳುಹಿಸು. ಅನಂತರ ಧ್ಯಾನಕ್ಕೆ ಕುಳಿತುಕೊ. ಪ್ರಥಮ ಮಟ್ಟದಲ್ಲಿ ಎಲ್ಲರೂ ಹೀಗೆ ಮಾಡಬೇಕು. ಅನಂತರ ಸ್ಥಿರವಾಗಿ ಕುಳಿತು (ಯಾವ ದಿಕ್ಕಿಗೆ ಬೇಕಾದರೂ ಕುಳಿತುಕೊ) ನಾನು ನಿನಗೆ ದೀಕ್ಷೆ ಕೊಟ್ಟ ರೀತಿ ಧ್ಯಾನ ಮಾಡು. ಒಂದು ದಿನವನ್ನೂ ಹಾಗೆಯೇ ಬಿಡಬೇಡ. ನಿನಗೆ ತುಂಬಾ ಜರೂರಾದ ಕೆಲಸವಿದ್ದಲ್ಲಿ ಒಂದು ಕಾಲು ಗಂಟೆಯಾದರೂ ನಿನ್ನ ಆಧ್ಯಾತ್ಮಿಕ ಸಾಧನೆ ಮಾಡು. ಮಗು, ನಿನ್ನಲ್ಲಿ ನಿಶ್ಚಲ ಭಕ್ತಿಯಿಲ್ಲದಿದ್ದಲ್ಲಿ ನಿನ್ನ ಗುರಿಯನ್ನು ಹೇಗೆ ಸೇರಬಲ್ಲೆ?” 

 ಸ್ವಾಮೀಜಿ ಮಹಡಿಗೆ ಹೋದರು, ಹೋಗುವಾಗ ಹೇಳಿದರು: “ನೀವೆಲ್ಲಾ ಹೆಚ್ಚು ಶ್ರಮವಿಲ್ಲದೆ ಆಧ್ಯಾತ್ಮಿಕ ಜೀವನದಲ್ಲಿ ಮುಂದುವರಿಯುವಿರಿ. ನೀವಿಲ್ಲಿಗೆ ಬರುವಷ್ಟು ಅದೃಷ್ಟಶಾಲಿಗಳಾದ್ದರಿಂದ ನಿಮಗೆ ಮುಕ್ತಿ ಸಮೀಪದಲ್ಲೆ ಇದೆ. ಈಗ ನಿಮ್ಮ ಧ್ಯಾನ ಮುಂತಾದುವುಗಳಿಂದ ಗೋಳಿನಿಂದ ತುಂಬಿರುವ ವಿಶ್ವಜನತೆಯ ಕಷ್ಟ ನಷ್ಟಗಳನ್ನು ನಿವಾರಿಸಲು ಪ್ರಯತ್ನಿಸಿ. ಕಠೋರ ತಪಸ್ಸಿನಿಂದ ನಾನು ಈ ದೇಹವನ್ನು ಹಾಳು ಮಾಡಿಕೊಂಡಿರುವೆ. ನನ್ನ ಈ ಮೂಳೆ ಮಾಂಸಗಳ ಚೀಲದಲ್ಲಿ ಸ್ವಲ್ಪ ಕೂಡ ಶಕ್ತಿ ಇಲ್ಲ. ಈಗ ನೀವು ಕೆಲಸಕ್ಕೆ ಹೊರಡಿ. ನನಗೆ ವಿಶ್ರಾಂತಿ ಸ್ವಲ್ಪ ಕೊಡಿ. ನೀವು ಇನ್ನೇನನ್ನು ಮಾಡದೇ ಇದ್ದರೂ ಚಿಂತೆಯಿಲ್ಲ. ಇಲ್ಲಿಯವರೆಗೂ ವ್ಯಾಸಂಗ ಮಾಡಿರುವ ಆಧ್ಯಾತ್ಮಿಕ ಸತ್ಯಗಳನ್ನು ಜನತೆಗೆ ಬೋಧಿಸಿ. ಇದಕ್ಕಿಂತ ಹೆಚ್ಚಿನ ಧ್ಯಾನ ವಿಲ್ಲ. ಏಕೆಂದರೆ ಪ್ರಪಂಚದಲ್ಲಿ ಜ್ಞಾನ ದಾನಕ್ಕಿಂತ ಮಿಗಿಲಾದ ದಾನವಿಲ್ಲ.” 

