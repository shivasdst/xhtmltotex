
\chapter{ಗುರುದೇವರ ನಿರ್ಯಾಣ}

ಶ‍್ರೀರಾಮಕೃಷ್ಣರು ಕಾಶೀಪುರದ ತೋಟದ ಮನೆಗೆ ಬಂದರು. ಅವರ ಲೀಲಾ ನಾಟಕದ ಕೊನೆಯ ಅಂಕ ಇದು. ಅವರ ದಿವ್ಯಚಕ್ಷುಸ್ಸಿಗೆ ತಾನು ಬೇಗನೆ ಪ್ರಪಂಚವನ್ನು ಬಿಡುವೆ ಎಂಬುದು ಗೊತ್ತಾಯಿತು. ಅದಕ್ಕೆ ಮುಂಚೆ ನರೇಂದ್ರನ ವ್ಯಕ್ತಿತ್ವವನ್ನು ಎಲ್ಲಾ ವಿಧದಿಂದಲೂ ರೂಪಿಸತೊಡಗಿದರು. ತಾವು ಪ್ರಪಂಚವನ್ನು ಬಿಟ್ಟು ಹೋಗಬೇಕಲ್ಲ ಎಂದು ವ್ಯಥೆಪಡಲಿಲ್ಲ. ತಮ್ಮ ತರುವಾಯ ನರೇಂದ್ರ ತಾವು ಆಧ್ಯಾತ್ಮಿಕ ಜೀವನದಲ್ಲಿ ಸಂಪಾದಿಸಿದುದನ್ನೆಲ್ಲ ಬಹು ಜನರ ಹಿತಕ್ಕೆ ಮತ್ತು ಸುಖಕ್ಕೆ ಹಂಚಲು ಯೋಗ್ಯನಾಗುತ್ತಿರುವನು ಎಂಬುದನ್ನು ನೋಡಿ ಆನಂದಿಸಿದರು. ತಾನು ಎಂದು ಈ ಪ್ರಪಂಚವನ್ನು ಬಿಟ್ಟು ಹೊರಟುಹೋಗುತ್ತೇನೆ ಎಂಬ ವಿಷಯದಲ್ಲಿ ಶ‍್ರೀರಾಮಕೃಷ್ಣರು ತಮ್ಮ ಶಿಷ್ಯರಿಗೆ ಸುಳಿವು ಕೊಟ್ಟರು: “ಬಹಳ ಜನ ಈ ದೇಹದಲ್ಲಿರುವ ಚೈತನ್ಯದ ವಿಷಯವಾಗಿ ಯಾವಾಗ ಹೊಗಳುವರೊ ಆಗ ತಾಯಿ ಇದನ್ನು ತೆಗೆದುಕೊಂಡು ಹೋಗಿಬಿಡುವಳು.”

ಶ‍್ರೀರಾಮಕೃಷ್ಣರ ಆರೋಗ್ಯ ದಿನಕಳೆದಂತೆ ಕ್ಷೀಣವಾಗುತ್ತ ಬಂದಿತು.ನರೇಂದ್ರನ ನೇತೃತ್ವದಲ್ಲಿ ಇತರ ಶಿಷ್ಯರೆಲ್ಲ ಹೃತ್ಪೂರ್ವಕವಾಗಿ ಅವರ ಶುಶ್ರೂಷೆ ಮಾಡತೊಡಗಿದರು. ಇದಕ್ಕಾಗಿ ಹಗಲು ರಾತ್ರಿ ಅವರು ಕಾಶೀಪುರದ ಮನೆಯಲ್ಲಿಯೇ ಇರಬೇಕಾಯಿತು. ಹುಡುಗರ ತಂದೆ ತಾಯಿಗಳು ಇದನ್ನು ಮೆಚ್ಚಲಿಲ್ಲ. ನರೇಂದ್ರ ಲಾ ಪರೀಕ್ಷೆಗೆ ಓದುತ್ತಿದ್ದ. ಇದೇ ಕಾಲದಲ್ಲಿ ಅವರ ಮನೆಯ ಮೇಲೆ ಒಂದು ಕೇಸು ಕೂಡ ಕೋರ್ಟಿನಲ್ಲಿತ್ತು. ಇದಕ್ಕಾಗಿ ಅವನು ಕಲ್ಕತ್ತೆಗೆ ಹೋಗಿ ಬರುತ್ತಿದ್ದನು. ಕಾಶೀಪುರಕ್ಕೆ ಹೋಗಿ ಶ‍್ರೀರಾಮಕೃಷ್ಣರನ್ನು ಸೇವೆ ಮಾಡಿ ಮಿಗುವ ಕಾಲದಲ್ಲಿ ಮಾತ್ರ ತಾನು ಓದುವೆನು ಎಂದು ಶಪಥ ಮಾಡಿದನು. ನರೇಂದ್ರನ ಕೋರಿಕೆಯ ಮೇಲೆ ತಮ್ಮ ವಿದ್ಯಾಭ್ಯಾಸವನ್ನು ಕೂಡ ಮರೆತ ಯುವಶಿಷ್ಯರು ಶ‍್ರೀರಾಮಕೃಷ್ಣರ ಸುತ್ತಲೂ ಇದ್ದರು. ಅವರಿಗೆ ವಿರಾಮವಾದಾಗಲೆಲ್ಲ ಅಧ್ಯಯನ, ಭಜನೆ, ಪ್ರವಚನ ಮುಂತಾದವುಗಳಲ್ಲಿ ಕಾಲ ಕಳೆಯುತ್ತಿದ್ದರು. ನರೇಂದ್ರನ ನೇತೃತ್ವದಲ್ಲಿ ಅವರೆಲ್ಲ ಅಣ್ಣ-ತಮ್ಮಂದಿರಂತೆ ಬೆಳೆದರು. ಆ ಆಧ್ಯಾತ್ಮಿಕ ಸಹೋದರ ವರ್ಗಕ್ಕೆ ಸೇರಿದವರೇ ನರೇಂದ್ರ, ರಾಖಾಲ, ಬಾಬುರಾಮ, ನಿರಂಜನ, ಯೋಗಿನ್, ಲಟು, ತಾರಕ, ಗೋಪಾಲದಾದ, ಕಾಳಿ, ಶಶಿ, ಶರತ್ ಮತ್ತು ಚೋಟಗೋಪಾಲ. ಶಾರದ, ಹರಿ, ಗಂಗಾಧರ ಮುಂತಾದವರು ಮನೆಯಿಂದ ಮಧ್ಯೆ ಮಧ್ಯೆ ಅಲ್ಲಿಗೆ ಬಂದು ಹೋಗುತ್ತಿದ್ದರು.

ಶ‍್ರೀರಾಮಕೃಷ್ಣರ ಕೊನೆಯ ದಿನಗಳು ಸಮೀಪಿಸಿದಂತೆಲ್ಲ ನರೇಂದ್ರನ ಆಧ್ಯಾತ್ಮಿಕ ಅಭೀಪ್ಸೆ ಮತ್ತೂ ತೀವ್ರವಾಗತೊಡಗಿತು. ಒಂದು ದಿನ ರಾತ್ರಿ ನರೇಂದ್ರನಿಗೆ ನಿದ್ರೆ ಬರಲಿಲ್ಲ. ಆಗ ಶರತ್, ಚೋಟಗೋಪಾಲ ಮುಂತಾದವರನ್ನು ಕರೆದು “ಬನ್ನಿ ಸ್ವಲ್ಪ ತೋಟದಲ್ಲಿ ಅಡ್ಡಾಡಿಕೊಂಡು ಬರುವ” ಎಂದು ಹೊರಟನು. ಹೋಗುತ್ತಿದ್ದಾಗ “ಶ‍್ರೀಗುರುದೇವರ ಖಾಯಿಲೆ ಉಲ್ಬಣಾವಸ್ಥೆಗೆ ಬಂದಿದೆ. ಅವರು ತಮ್ಮ ದೇಹವನ್ನು ತ್ಯಜಿಸಿಬಿಡಬಹುದು. ಕಾಲ ಇರುವಾಗಲೇ ಅವರ ಸೇವೆ, ಪ್ರಾರ್ಥನೆ ಮತ್ತು ಧ್ಯಾನಗಳಿಂದ ಆಧ್ಯಾತ್ಮಿಕ ಜೀವನದಲ್ಲಿ ಮುಂದುವರಿಯುವುದಕ್ಕೆ ಪ್ರಯತ್ನಿಸಬೇಕು. ಅವರ ಕಾಲಾನಂತರ ಎಷ್ಟು ಪಶ್ಚಾತ್ತಾಪ ಪಟ್ಟರೂ ಪ್ರಯೋಜನವಿಲ್ಲ. ಯಾವ ಯಾವುದೋ ಕೆಲಸಗಳನ್ನೆಲ್ಲ ಮಾಡಿ ಆಮೇಲೆ ನಾವು ಸಾಧನೆ ಮಾಡುತ್ತೇವೆ ಎಂದು ಭಾವಿಸಿ ಕಾಲವನ್ನು ವ್ಯರ್ಥಮಾಡುತ್ತಿರುವೆವು. ಅದರಿಂದ ನಮ್ಮ ಆಸೆ ಆಕಾಂಕ್ಷೆಗಳು ಮತ್ತೂ ವೃದ್ಧಿಯಾಗುವುದು. ಅದು ವೃದ್ಧಿ ಆದರೆ ಮರಣಕ್ಕೆ ಸಮ. ಆಸೆಗಳನ್ನು ನಿರ್ಮೂಲ ಮಾಡಬೇಕು” ಎಂದನು.

ಅಂದು ಚಳಿಗಾಲದ ರಾತ್ರಿ. ತೋಟದ ಮಧ್ಯದಲ್ಲಿ ಹತ್ತಿರ ಬಿದ್ದಿದ್ದ ಹುಲ್ಲನ್ನು ಹತ್ತಿಸಿ, ಧ್ಯಾನಕ್ಕೆ ಕುಳಿತರು. ಧೂನಿಯನ್ನು ಹಚ್ಚಿಸುವಾಗ, “ಇದರಿಂದ ನಮ್ಮ ಆಸೆಗೆ ಬೆಂಕಿ ಹತ್ತಿಸುತ್ತಿರುವೆವು ಎಂದು ಭಾವಿಸಿ. ಇಂತಹ ಸಮಯದಲ್ಲೆ ಸಾಧುಗಳು ಧೂನಿಯನ್ನು ಹಚ್ಚುವುದು. ಇದರಂತೆಯೇ ನಾವೂ ಕೂಡ ನಮ್ಮ ಆಸೆಗೆ ಬೆಂಕಿಯನ್ನು ಇಡುವ” ಎಂದನು. ಆ ಉರಿಯುತ್ತಿರುವ ಬೆಂಕಿಯ ಸುತ್ತ ಕುಳಿತು ಧ್ಯಾನ ಮಗ್ನರಾದರು.

ಒಂದು ದಿನ ಶ‍್ರೀರಾಮಕೃಷ್ಣರು ಶ‍್ರೀರಾಮ ಮಂತ್ರವನ್ನು ನರೇಂದ್ರನಿಗೆ ಉಪದೇಶ ಮಾಡಿದರು. ಈ ಮಂತ್ರವನ್ನೇ ತಾನು ತಮ್ಮ ಗುರುಗಳಿಂದ ಪಡೆದುಕೊಂಡದ್ದು ಎಂದು ಹೇಳಿದನು. ಈ ಮಂತ್ರವನ್ನು ಸ್ವೀಕರಿಸಿದ ನಂತರ ನರೇಂದ್ರ ಭಾವೋನ್ಮತ್ತನಾದ. ಸಂಜೆ ರಾಮ ರಾಮ ಎಂದು ಹೇಳುತ್ತ ಪ್ರಪಂಚವನ್ನೇ ಮರೆತು ಮನೆಯ ಸುತ್ತಲೂ ಪ್ರದಕ್ಷಿಣೆ ಮಾಡಲು ಆರಂಭಿಸಿದ. ಶ‍್ರೀರಾಮಕೃಷ್ಣರಿಗೆ ಯಾರೋ ಈ ಪ್ರಸಂಗವನ್ನು ಹೇಳಿದಾಗ “ಚಿಂತೆಯಿಲ್ಲ. ಕ್ರಮೇಣ ಅವನು ತನ್ನ ಸಹಜ ಸ್ಥಿತಿಗೆ ಬರುವನು” ಎಂದರು.

ಒಂದು ದಿನ ನರೇಂದ್ರನಿಗೂ ಮಹೇಂದ್ರನಿಗೂ ಈ ಸಂಭಾಷಣೆ ಜರುಗಿತು:

ನರೇಂದ್ರ: “ಕಳೆದ ಶನಿವಾರ (೨ನೇ ಜನವರಿ ೧೮೮೬) ನಾನು ಇಲ್ಲಿ ಧ್ಯಾನ ಮಾಡುತ್ತಿದ್ದೆ. ತತ್‍ಕ್ಷಣವೇ ನನ್ನ ಹೃದಯದಲ್ಲಿ ಯಾವುದೋ ವಸ್ತು ಚಲಿಸುತ್ತಿರುವಂತೆ ಭಾಸವಾಯಿತು.”

ಮಹೇಂದ್ರ: “ಅದು ಕುಂಡಲಿನಿ ಶಕ್ತಿಯ ಜಾಗ್ರತಿ ಇರಬಹುದು”

ನರೇಂದ್ರ: “ಬಹುಶಃ ಇರಬಹುದು. ನನಗೆ ಚೆನ್ನಾಗಿ ಇಡ ಮತ್ತು ಪಿಂಗಳ ಭಾಸವಾಯಿತು. ನಾನು ಹಾಜರನಿಗೆ ನನ್ನ ಹೃದಯದ ಮೇಲೆ ಕೈಯನ್ನು ಇಡುವಂತೆ ಹೇಳಿದೆ. ನಿನ್ನೆ ಶ‍್ರೀರಾಮಕೃಷ್ಣರನ್ನು ಮಹಡಿಯ ಮೇಲೆ ಕಂಡೆ. ನಾನು ಅವರಿಗೆ ಹೀಗೆ ಹೇಳಿದೆ: ‘ಎಲ್ಲರಿಗೂ ಯಾವುದಾದರೂ ಒಂದು ಆಧ್ಯಾತ್ಮಿಕ ಅನುಭವ ಆಗಿದೆ. ನನಗೂ ನೀವು ಏನನ್ನಾದರೂ ಅನುಗ್ರಹಿಸಿ. ಎಲ್ಲರಿಗೂ ಸಿಕ್ಕಿರುವಾಗ ನನಗೊಬ್ಬನಿಗೆ ಇಲ್ಲವೆ?’ ಅದಕ್ಕೆ ಶ‍್ರೀರಾಮಕೃಷ್ಣರು ‘ನಿಮ್ಮ ಮನೆಯವರಿಗೆ ಏನಾದರೂ ಅಣಿ ಮಾಡಿ ಬಾ. ಅನಂತರ ನಿನಗೆ ಎಲ್ಲಾ ಸಿಕ್ಕುವುದು. ನಿನಗೆ ಏನು ಬೇಕು ಹೇಳು?’ ಎಂದು ಕೇಳಿದರು. ಅದಕ್ಕೆ ನಾನು ‘ಒಂದೊಂದು ಸಲವೂ ಮೂರು ನಾಲ್ಕು ದಿನಗಳವರೆಗೆ ಸಮಾಧಿಯಲ್ಲಿ ನಿರತನಾಗಿರಬೇಕು. ಮಧ್ಯೆ ಊಟ ಮಾಡುವುದಕ್ಕೆ ಮಾತ್ರ ಅದು ಅವಕಾಶ ಕೊಡಬೇಕು’ ಎಂದೆ. ಅದಕ್ಕೆ ಅವರು, ‘ನೀನು ಮೂರ್ಖ. ಅದಕ್ಕಿಂತ ಮೇಲಿರುವ ಒಂದು ಅವಸ್ಥೆ ಇದೆ. “ಇರುವುದೆಲ್ಲ ನೀನೆ” ಎಂದು ಹಾಡುವವನು ನೀನೆ ಅಲ್ಲವೆ? ನಿಮ್ಮ ಮನೆಯವರಿಗೆ ಏನನ್ನಾದರೂ ಅಣಿ ಮಾಡಿ ಬಾ. ಆಗ ಸಮಾಧಿಗಿಂತ ಮೇಲಿರುವುದನ್ನು ನೀನು ಪಡೆಯುವೆ’ ಎಂದರು.”

“ಈ ದಿನ ಬೆಳಗ್ಗೆ ನಾನು ಮನೆಗೆ ಹೋದೆ. ಪರೀಕ್ಷೆ ಹತ್ತಿರ ಬರುತ್ತಿರುವಾಗ ಓದುವುದನ್ನು ಉಪೇಕ್ಷೆ ಮಾಡುತ್ತಿರುವೆ ಎಂದು ಮನೆಯವರು ನನ್ನನ್ನು ದೂರಿದರು. ನಾನು ನನ್ನ ಅಜ್ಜಿಯ ಮನೆಯಲ್ಲಿ ಓದುವುದಕ್ಕೆ ಹೋದೆ. ನಾನು ಇನ್ನೇನು ಓದುವುದಕ್ಕೆ ಪ್ರಾರಂಭಿಸಬೇಕು ಎನ್ನುವುದರೊಳಗೆ ನನಗೆ ಅದರ ಮೇಲೆ ಜುಗುಪ್ಸೆ ಬಂದಿತು. ಪ್ರಪಂಚದಲ್ಲಿ ಓದುವುದೇ ಒಂದು ಕೆಟ್ಟ ಕೆಲಸ ಎಂಬಂತೆ ತೋರಿತು. ನನ್ನ ಮನಸ್ಸಿನಲ್ಲಿ ಒಂದು ತುಮುಲ ಯುದ್ಧ ಪ್ರಾರಂಭವಾಯಿತು. ನಾನು ಹಾಗೆ ಎಂದೂ ಜೀವನದಲ್ಲಿ ಅತ್ತಿರಲಿಲ್ಲ. ನಾನು ನಂತರ ಪುಸ್ತಕಗಳನ್ನೆಲ್ಲ ಹಾಗೆಯೇ ಬಿಟ್ಟು ಇಲ್ಲಿಗೆ ಓಡಿಬಂದೆ. ನನ್ನ ಕಾಲಿನ ಜೋಡು ಬರುವಾಗ ಎಲ್ಲಿಯೋ ಕಳೆದುಹೋಯಿತು. ನಾನೊಂದು ಹುಲ್ಲಿನ ಬಣವೆಯ ಸಮೀಪವಾಗಿ ಓಡಿಬರುತ್ತಿದ್ದೆ. ಆದ್ದರಿಂದ ಕೆಲವು ಹುಲ್ಲುಗಳು ಹಾರಿಬಂದು ನನಗೆ ಅಂಟಿಕೊಂಡವು. ನನಗಾವುದರ ಪರಿವೆಯೂ ಇರಲಿಲ್ಲ. ನಾನು ಇಲ್ಲಿಗೆ ಬರುವವರೆಗೆ ಸುಮ್ಮನೆ ಓಡಿದೆ.”

ಅಂದಿನ ರಾತ್ರಿಯೇ ಒಂಭತ್ತು ಗಂಟೆಯ ಹೊತ್ತಿಗೆ ಎಷ್ಟೇ ಯಾತನೆಯಾಗಿದ್ದರೂ ಮೆಲ್ಲಗೆ ಮತ್ತೆ ಕೆಲವು ವೇಳೆ ಸನ್ನೆಯ ಮೂಲಕ ನರೇಂದ್ರನ ವಿಷಯವಾಗಿ ಶ‍್ರೀರಾಮಕೃಷ್ಣರು ಹೇಳಿದರು: “ನರೇಂದ್ರನ ಅದ್ಭುತ ಸ್ಥಿತಿಯನ್ನು ನೋಡು! ಅವನು ಒಂದು ಸಮಯದಲ್ಲಿ ದೇವರನ್ನು ನಂಬುತ್ತಿರಲಿಲ್ಲ. ಈಗ ಅವನ ಸಾಕ್ಷಾತ್ಕಾರಕ್ಕೆ ಹೇಗೆ ತವಕ ಪಡುತ್ತಿರುವನು!” ಅನಂತರ ನರೇಂದ್ರ ಬೇಗ ಗುರಿಯನ್ನು ಮುಟ್ಟುತ್ತಾನೆ ಎಂದು ಸೂಚನೆಯನ್ನು ಕೊಟ್ಟರು. ಅಂದಿನ ರಾತ್ರಿಯೇ ನರೇಂದ್ರ ಇನ್ನೂ ಕೆಲವು ಶಿಷ್ಯರೊಡನೆ ಧ್ಯಾನ ಮಾಡುವುದಕ್ಕೆ ದಕ್ಷಿಣೇಶ್ವರಕ್ಕೆ ಹೋದನು.

ಒಂದು ದಿನ ಶ‍್ರೀರಾಮಕೃಷ್ಣರು “ನೀನು ಏತಕ್ಕೆ ಕಾಲೇಜಿನ ವಿದ್ಯಾಭ್ಯಾಸವನ್ನು ಮುಂದುವರಿಸಬಾರದು?” ಎಂದು ಕೇಳಿದರು. ಅದಕ್ಕೆ ನರೇಂದ್ರ, “ಮಹಾಶಯರೆ, ನಾನು ಕಲಿತಿರುವುದನ್ನೆಲ್ಲಾ ಒಂದೇ ಸಲ ಮರೆತುಬಿಡುವಂತಹ ಒಂದು ಔಷಧಿ ಸಿಕ್ಕಿದರೆ ಸಾಕು” ಎಂದನು. ನರೇಂದ್ರ ಈ ಸಮಯದಲ್ಲಿ ಉಗ್ರ ತಪಸ್ಸಿನಲ್ಲಿ ನಿರತನಾಗಿರುತ್ತಿದ್ದ. ಶ‍್ರೀರಾಮಕೃಷ್ಣರು ಆಧ್ಯಾತ್ಮಿಕ ಜೀವನದ ಹಲವು ಸೂಕ್ಷ್ಮ ಸಲಹೆಗಳನ್ನು ನರೇಂದ್ರನಿಗೆ ಕೊಟ್ಟಿದ್ದರು. ನರೇಂದ್ರ ಅವುಗಳನ್ನೆಲ್ಲ ಅಭ್ಯಾಸ ಮಾಡಿ ಅದರ ಪ್ರತಿಫಲವನ್ನು ಪಡೆದಿದ್ದ. ತಮ್ಮ ನಂತರ ನರೇಂದ್ರನನ್ನು ತಮ್ಮ ಶಿಷ್ಯರಿಗೆಲ್ಲ ನಾಯಕನನ್ನಾಗಿ ಮಾಡಲು ಅವನಿಗೆ ತರಬೇತನ್ನು ಕೊಡುತ್ತಿದ್ದರು. ಒಂದು ದಿನ ಅವರು ನರೇಂದ್ರನಿಗೆ, “ನಾನು ಅವರನ್ನು ನಿನ್ನ ವಶಕ್ಕೆ ಬಿಟ್ಟುಹೋಗುತ್ತೇನೆ. ನನ್ನ ಕಾಲಾನಂತರ ಅವರು ಸಾಧನೆ, ಭಜನೆಯಲ್ಲಿ ನಿರತರಾಗಿ, ಮನೆಗೆ ಹೋಗದಂತೆ ನೋಡಿಕೊ” ಎಂದು ಹೇಳಿದರು.

ಒಂದು ದಿನ ಶ‍್ರೀರಾಮಕೃಷ್ಣರು ತಮ್ಮ ಶಿಷ್ಯರಿಗೆ, “ಹೊರಗೆ ಹೋಗಿ ಭಿಕ್ಷೆಯನ್ನು ಬೇಡಿಕೊಂಡು ಬನ್ನಿ” ಎಂದು ಕಳುಹಿಸಿದರು. ಶ‍್ರೀರಾಮಕೃಷ್ಣರ ಅಪ್ಪಣೆ ಮೇರೆಗೆ ಅವರ ಶಿಷ್ಯರು ಮನೆಮನೆಗೆ ಭಿಕ್ಷೆ ಎತ್ತಲು ಹೊದರು. ಕೆಲವು ಮನೆಯ ಯಜಮಾನರು, ತಮ್ಮ ವಿದ್ಯಾಭ್ಯಾಸವನ್ನು ಮುಂದುವರಿಸುವುದನ್ನು ಮರೆತು ಇಂತಹ ಕೆಲಸದಲ್ಲಿ ನಿರತರಾಗಿದ್ದ ಅವರನ್ನು ಜರೆದರು. ಅಂತೂ ಭಿಕ್ಷೆಯನ್ನು ತಂದಮೇಲೆ ಅದನ್ನು ಅಡಿಗೆ ಮಾಡಿ ಶ‍್ರೀರಾಮಕೃಷ್ಣರಿಗೆ ಸ್ವಲ್ಪ ಕೊಟ್ಟರು. ಶ‍್ರೀರಾಮಕೃಷ್ಣರು “ಇದು ಅತ್ಯಂತ ಶುದ್ಧವಾದ ಆಹಾರ” ಎಂದು ಹೇಳುತ್ತ ಸ್ವೀಕರಿಸಿದರು.

ಒಂದು ದಿನ ಶಶಿಧರ ತರ್ಕಚೂಡಾಮಣಿ ಎಂಬ ಕಲ್ಕತ್ತೆಯ ಘನ ವಿದ್ವಾಂಸನೊಬ್ಬ ಶ‍್ರೀರಾಮಕೃಷ್ಣರನ್ನು ನೋಡಲು ಬಂದ. ಆತ ಶ‍್ರೀರಾಮಕೃಷ್ಣರಿಗೆ, “ನಿಮ್ಮಂತಹ ಯೋಗಿಗಳು ಗಾಯದ ಮೇಲೆ ಮನಸ್ಸಿಟ್ಟು, ಅದು ಗುಣವಾಗಲಿ ಎಂದು ಇಚ್ಛಿಸಿದರೆ ಅದು ಗುಣವಾಗಬಲ್ಲುದು ಎಂದು ನಮ್ಮ ಶಾಸ್ತ್ರಗಳು ಹೇಳುತ್ತವೆ. ನೀವು ಏತಕ್ಕೆ ಹಾಗೆ ಮಾಡಬಾರದು?” ಎಂದು ಕೇಳಿದ. ಅದಕ್ಕೆ ಶ‍್ರೀರಾಮಕೃಷ್ಣರು, “ನೀನು ವಿದ್ವಾಂಸ. ಆದರೂ ಇಂತಹ ಸಲಹೆಯನ್ನು ಕೊಡುತ್ತಿರುವೆಯಲ್ಲ? ನಾನು ನನ್ನ ಮನಸ್ಸನ್ನು ಒಂದೇ ಸಲ ದೇವರಿಗೆ ಕೊಟ್ಟುಬಿಟ್ಟಿರುವೆನು. ಅದನ್ನು ದೇವರಿಂದ ತೆಗೆದು, ಕೊಳೆತು ನಾರುತ್ತಿರುವ ಈ ದೇಹದ ಮೇಲೆ ಇಡಲು ಸಾಧ್ಯವೆ?” ಎಂದರು. ಪಂಡಿತನಿಗೆ ಏನೂ ಮಾತನಾಡಲು ತೋರದೆಹೋಯಿತು. ಆತ ಹೊರಟುಹೋದ ಮೇಲೆ ನರೇಂದ್ರ ಇನ್ನೂ ಕೆಲವು ಶಿಷ್ಯರೊಡನೆ ಶ‍್ರೀರಾಮಕೃಷ್ಣರನ್ನು ತಮಗಾಗಿಯಾದರೂ ಈ ರೋಗದಿಂದ ಪಾರಾಗುವಂತೆ ದೇವರಿಗೆ ಬೇಡಿಕೊಳ್ಳಬೇಕೆಂದು ಬಲಾತ್ಕರಿಸಿದನು. ಶ‍್ರೀರಾಮಕೃಷ್ಣರು “ನಾನು ಬೇಕು ಎಂದು ಈ ಯಾತನೆಯನ್ನು ಅನುಭವಿಸುತ್ತಿರುವೆನು ಎಂದು ಭಾವಿಸಿದಿರೇನು? ನನಗೂ ರೋಗದಿಂದ ಗುಣವಾಗಬೇಕೆಂದು ಆಸೆ, ಆದರೂ ಅದು ನನ್ನನ್ನು ಬಿಟ್ಟುಹೋಗಿಲ್ಲ. ಇದೆಲ್ಲ ಜಗನ್ಮಯಿಯ ಇಚ್ಛೆಯ ಮೇಲಿದೆ.”

ನರೇಂದ್ರ: “ಹಾಗಾದರೆ ಜಗನ್ಮಯಿಗೆ ಗುಣಮಾಡು ಎಂದು ದಯವಿಟ್ಟು ಕೇಳಿಕೊಳ್ಳಿ.”

ಶ‍್ರೀರಾಮಕೃಷ್ಣ: “ನೀನು ಹಾಗೆ ಮಾತನಾಡುವುದು ಬಹಳ ಸುಲಭ. ಆದರೆ ನಾನು ಹಾಗೆ ಕೇಳುವುದಕ್ಕೆ ಆಗುವುದಿಲ್ಲ.”

ನರೇಂದ್ರ: “ಇಲ್ಲ, ಹಾಗಾಗುವುದಿಲ್ಲ. ನಮ್ಮಗಳಿಗಾಗಿಯಾದರೂ ನೀವು ತಾಯಿಯನ್ನು ಕೇಳಿಕೊಳ್ಳಬೇಕು.”

ಶ‍್ರೀರಾಮಕೃಷ್ಣ: “ನೋಡೋಣ, ಹೇಗಾಗುವುದೊ!”

ಸ್ವಲ್ಪ ಹೊತ್ತಾದ ಮೇಲೆ ನರೇಂದ್ರ ಬಂದು ಶ‍್ರೀರಾಮಕೃಷ್ಣರನ್ನು, “ನೀವು ದೇವಿಯನ್ನು ಕೇಳಿದಿರಾ? ಅವಳು ಏನೆಂದಳು?” ಎಂದು ಕೇಳಿದ. ಅದಕ್ಕೆ ಶ‍್ರೀರಾಮಕೃಷ್ಣರು “ಅವಳಿಗೆ ನನ್ನ ಗಂಟಲನ್ನು ತೋರಿಸಿ, ‘ಇಲ್ಲಿ ವ್ರಣವಾಗಿರುವುದರಿಂದ ನಾನು ಏನನ್ನೂ ತಿನ್ನಲಾರೆ. ನನಗೆ ಸ್ವಲ್ಪ ತಿನ್ನಲು ಸಾಧ್ಯವಾಗುವಂತೆ ಮಾಡು’ ಎಂದೆ. ತಾಯಿ ನಿಮ್ಮನ್ನೆಲ್ಲ ತೋರಿ, ‘ಏಕೆ, ನೀನು ಇಷ್ಟೊಂದು ಬಾಯಿಗಳ ಮೂಲಕ ತಿನ್ನುತ್ತಿಲ್ಲವೆ?’ ಎಂದು ಕೇಳಿದಳು. ಇದನ್ನು ಕೇಳಿ ನನಗೆ ನಾಚಿಕೆ ಆಯಿತು” ಎಂದರು.

ನರೇಂದ್ರ ಆಧ್ಯಾತ್ಮಿಕ ಜೀವನದಲ್ಲಿ ಮುಂದುವರಿಯುತ್ತಿದ್ದನು. ತನ್ನಲ್ಲಿ ಶಕ್ತಿ ಸಂಗ್ರಹವಾಗುತ್ತಿರುವುದು ಕೆಲವು ವೇಳೆ ಅವನಿಗೆ ಗೋಚರವಾಗುತ್ತಿತ್ತು. ಅದನ್ನು ಒಂದು ದಿನ ಪರೀಕ್ಷಿಸಲು ಯತ್ನಿಸಿದನು. ೧೮೮೬ನೇ ಮಾರ್ಚಿ ತಿಂಗಳಿನಲ್ಲಿ ಶಿವರಾತ್ರಿ ಬಂದಿತು. ನರೇಂದ್ರ ಉಪವಾಸವಿದ್ದು ದಿನವನ್ನು ಧ್ಯಾನ ಪಾರಾಯಣದಲ್ಲಿ ಕಳೆದನು. ರಾತ್ರಿ ಮೊದಲನೆ ಜಾವದ ಪೂಜೆಯೂ ಆಯಿತು. ಅನಂತರ ಪೂಜಾಮಂದಿರದಿಂದ ನೆರೆದವರೆಲ್ಲ ಹೊರಗೆ ಹೋದರು. ಆಗ ನರೇಂದ್ರ ಕಾಳಿಗೆ, “ನಾನು ಧ್ಯಾನಕ್ಕೆ ಕುಳಿತುಕೊಳ್ಳುವೆ. ಕೆಲವು ಕಾಲದ ಮೇಲೆ ನನ್ನನ್ನು ಮುಟ್ಟು, ಆಗ ನಿನಗೆ ಏನು ಆಗುವುದೊ ಅದನ್ನು ಹೇಳು” ಎಂದನು. ನರೇಂದ್ರ ಗಾಢಧ್ಯಾನದಲ್ಲಿ ತಲ್ಲೀನನಾದಾಗ ಕಾಳಿ ಅವನನ್ನು ಮುಟ್ಟಲು, ವಿದ್ಯುತ್‍ಶಕ್ತಿಯ ತಂತಿಯನ್ನು ಮುಟ್ಟಿದಂತೆ ಅವನಿಗೆ ಆಯಿತು. ಇದನ್ನು ನರೇಂದ್ರನಿಗೆ ಅನಂತರ ಕಾಳಿ ಹೇಳಿದ. ಪೂಜೆಯೆಲ್ಲ ಆದಮೇಲೆ ನರೇಂದ್ರ ಶ‍್ರೀರಾಮಕೃಷ್ಣರನ್ನು ನೋಡಲು ಹೋದಾಗ ಅವರು ನರೇಂದ್ರನಿಗೆ ಹೀಗೆ ಹೇಳಿದರು, “ನೀನು ಶಕ್ತಿಯನ್ನು ಸಂಗ್ರಹಿಸುವುದಕ್ಕೆ ಮುಂಚೆ ಅದನ್ನು ಆಗಲೇ ವ್ಯಯಮಾಡುತ್ತಿರುವೆ. ಮೊದಲು ಅದನ್ನು ಸಂಗ್ರಹಿಸು. ಅನಂತರ ಹೇಗೆ ಎಷ್ಟನ್ನು ಖರ್ಚು ಮಾಡುವುದು ಎಂಬುದು ಗೊತ್ತಾಗುವುದು. ತಾಯಿಯೇ ನಿನಗೆ ಅದನ್ನು ಹೇಳುವಳು. ನಿನ್ನ ಭಾವನೆಯನ್ನು ಆ ಹುಡುಗನಿಗೆ (ಕಾಳಿಗೆ) ಕೊಟ್ಟು ಅವನಿಗೆ ಎಷ್ಟು ಅಪಾಯ ತಂದಿರುವೆ ಗೊತ್ತೆ? ಅವನು ಇದುವರೆಗೆ ಯಾವುದೋ ಒಂದು ದಾರಿಯಲ್ಲಿ ಹೋಗುತ್ತಿದ್ದ. ಈಗ ಅದೆಲ್ಲ ಹಾಳಾಯಿತು. ಆಗಿಹೋದುದನ್ನು ಕುರಿತು ಚಿಂತಿಸಿ ಫಲವಿಲ್ಲ. ಇನ್ನು ಮೇಲೆ ಹಾಗೆ ಮಾಡಬೇಡ”. ಶ‍್ರೀರಾಮಕೃಷ್ಣರು ತಮ್ಮ ಕೋಣೆಯಲ್ಲಿಯೇ ಇದ್ದರೂ ಅವರಿಗೆ ಇದೆಲ್ಲವೂ ಗೊತ್ತಾಗಿತ್ತು. ಅವನು ಆಶ್ಚರ್ಯದಿಂದ ಮೂಕನಂತಾದ.

ಗೃಹಸ್ಥ ಶಿಷ್ಯರು ಶ‍್ರೀರಾಮಕೃಷ್ಣರಿಗೆ ತಗುಲುವ ವೆಚ್ಚವನ್ನೆಲ್ಲ ಸಂತೋಷದಿಂದ ವಹಿಸುತ್ತಿದ್ದರು. ಯುವಕ ಶಿಷ್ಯರು ಹಗಲು ರಾತ್ರಿ ಶ‍್ರೀರಾಮಕೃಷ್ಣರ ಶುಶ್ರೂಷೆ ಮಾಡುತ್ತಿದ್ದರು. ಆದರೆ ಶ‍್ರೀರಾಮಕೃಷ್ಣರ ಸ್ಥಿತಿ ಉತ್ತಮವಾಗಲಿಲ್ಲ. ಕೆಲವು ವೇಳೆ ವ್ರಣದಿಂದ ರಕ್ತ ಧಾರಾಕಾರವಾಗಿ ಹರಿಯುತ್ತಿತ್ತು, ನೋವು ಸಹಿಸಲಸದಳವಾಗಿರುತ್ತಿತ್ತು. ಆಗಲೂ ಶ‍್ರೀರಾಮಕೃಷ್ಣರು ಮಂದಹಾಸದಿಂದ “ದೇಹ ರೋಗವನ್ನು ಗಮನಿಸಲಿ. ಮನಸ್ಸೇ, ನೀನು ಯಾವಾಗಲೂ ಆನಂದದಲ್ಲಿರು” ಎಂದು ಹೇಳುತ್ತಿದ್ದರು. ಒಂದು ದಿನ ಶ‍್ರೀರಾಮಕೃಷ್ಣರು ಮಹೇಂದ್ರನಿಗೆ “ನೋಡು ನಾನು ಇದನ್ನೆಲ್ಲ ಸಹಿಸುತ್ತಿರುವೆ. ಇದೂ ಕೂಡ (ತಮ್ಮನ್ನು ತೋರಿ) ಅಂತಹ ಒಂದು ಆಕಾರ” ಎಂದರು.

೧೫ನೇ ಮಾರ್ಚಿ ೧೮೮೬. ಗಂಟೆ ಪ್ರಾತಃಕಾಲ ಏಳು. ಶ‍್ರೀರಾಮಕೃಷ್ಣರಿಗೆ ಸ್ವಲ್ಪ ಉತ್ತಮವಾಗಿರುವಂತೆ ಕಂಡುಬರುತ್ತಿದೆ. ಅವರು ಭಕ್ತರೊಡನೆ ಕೆಲವು ವೇಳೆ ಮೆಲ್ಲಗೆ ಮತ್ತೆ ಕೆಲವು ವೇಳೆ ಸಂಜ್ಞೆಗಳ ಮೂಲಕ ಮಾತನಾಡುತ್ತಿದ್ದರು. ನರೇಂದ್ರ ಮತ್ತು ಇತರ ಶಿಷ್ಯರುಗಳೂ ಇರುವರು.

ಶ‍್ರೀರಾಮಕೃಷ್ಣರು ಭಕ್ತರಿಗೆ: “ನನಗೇನು ಕಾಣುತ್ತಿದೆ ಗೊತ್ತೆ? ಭಗವಂತನೇ ಎಲ್ಲವೂ ಆಗಿರುವುದಾಗಿ ಕಾಣಬರುತ್ತಿದೆ. ಮನುಷ್ಯ ಮತ್ತು ಉಳಿದ ಜೀವಜಂತುಗಳೆಲ್ಲವೂ ಚರ್ಮದಿಂದ ಆಗಿರುವಂತೆಯೂ ಭಗವಂತ ಅವುಗಳೊಳಗೆ ಕುಳಿತು ಕೈಕಾಲು ತಲೆ ಇವನ್ನು ಅಲ್ಲಾಡಿಸುತ್ತಿರುವಂತೆಯೂ ಕಂಡುಬರುತ್ತಿದೆ. ಹಿಂದೆ ನನಗೆ ಒಮ್ಮೆ ಇದೇ ತರಹದ ಅನುಭವ ಆಗಿತ್ತು. ಮನೆ ತೋಟ ರಸ್ತೆ ಮನುಷ್ಯರು ದನ ಕರು ಎಲ್ಲವೂ ಒಂದೇ ವಸ್ತುವಿನಿಂದ ಎಂದರೆ ಮೇಣದಿಂದ ನಿರ್ಮಿತವಾದಂತೆ ಕಂಡು ಬಂದವು”.

“ಭಗವಂತನೇ ಬಲಿಯಾಗಿಯೂ, ಬಲಿಪೀಠವಾಗಿಯೂ, ಬಲಿಕೊಡತಕ್ಕವನಾಗಿಯೂ ಕಂಡುಬರುತ್ತಿದ್ದಾನೆ.”

ಶ‍್ರೀರಾಮಕೃಷ್ಣರು ತಮಗಾಗಿರುವ ಈ ಅನುಭವಗಳನ್ನು ವರ್ಣಿಸುತ್ತಿದ್ದ ಹಾಗೆಯೇ ಭಾವದಿಂದ ತುಂಬಿತುಳುಕಾಡುತ್ತಿದ್ದಾರೆ. ಶ‍್ರೀರಾಮಕೃಷ್ಣರು ತಮಗೆ ಈಗ ಯಾವ ನೋವು ಇಲ್ಲ; ಮತ್ತೆ ಎಂದಿನಂತೆ ಆಗಿಬಿಟ್ಟಿದ್ದೇನೆ ಎಂದರು. ಲಾಟುವನ್ನು ನೋಡಿ “ಅಲ್ಲಿ ನೋಡಿ ಲಾಟೂನ, ತಲೆಗೆ ಕೈಕೊಟ್ಟು ಕುಳಿತಿದ್ದಾನೆ. ಭಗವಂತನೇ ಹಾಗೆ ಕೈಕೊಟ್ಟು ಕುಳಿತಿದ್ದಾನೆ” ಎಂದರು. ಪುನಃ “ಈ ಶರೀರ ಇನ್ನೂ ಸ್ವಲ್ಪ ಕಾಲ ಇರುವುದಾದರೆ ಇನ್ನೂ ಅನೇಕರ ಆತ್ಮ ಜಾಗೃತಿ ಉಂಟಾಗಿಬಿಡುತ್ತಿತ್ತು” ಎಂದು ಹೇಳಿದರು. “ಇನ್ನು ಭಗವಂತ ಈ ಶರೀರವನ್ನು ಉಳಿಸಿಕೊಡುವುದಿಲ್ಲ. ನಾನು ಸರಳ ಮತ್ತು ಏನೂ ತಿಳಿಯದಿರುವವನೆಂಬುದನ್ನು ನೋಡಿ, ನನ್ನಿಂದ ಎಲ್ಲವನ್ನೂ ಜನರು ಕಸಿದುಕೊಂಡುಬಿಡಬಹುದೆಂದು, ನಾನು ಎಲ್ಲವನ್ನೂ ಎಲ್ಲರಿಗೂ ಹಂಚಿ ಬಿಡಬಹುದೆಂದು, ಈ ಶರೀರವನ್ನು ಅವನು ಇನ್ನು ಉಳಿಸುವುದಿಲ್ಲ.”

ರಾಖಾಲ: “ನಿಮ್ಮ ಶರೀರವನ್ನು ಇನ್ನು ಸ್ವಲ್ಪ ಕಾಲ ಬದುಕಿಸಿ ಇಡುವಂತೆ ನೀವು ಭಗವಂತನಿಗೆ ಹೇಳಿ.”

ಶ‍್ರೀರಾಮಕೃಷ್ಣ: “ಅದು ಭಗವಂತನ ಇಚ್ಛೆಯನ್ನು ಅವಲಂಬಿಸಿದೆ”

ನರೇಂದ್ರ: “ನಿಮ್ಮ ಇಚ್ಛೆ ಮತ್ತು ಭಗವಂತನ ಇಚ್ಛೆ ಎರಡೂ ಒಂದೇ ಆಗಿಹೋಗಿದೆ.”

ಶ‍್ರೀರಾಮಕೃಷ್ಣ: “ನಾನು ಭಗವಂತನಿಗೆ ಹೇಳಿದರೆ ಏನೂ ನಡೆಯುವ ಹಾಗಿಲ್ಲ. ಈಗ ನೋಡುತ್ತಾ ಇರುವೆ, ನಾನೂ ಭಗವತಿಯೂ ಒಂದೇ ಆಗಿರುವ ಹಾಗೆ. ರಾಧೆ ತನ್ನ ಅತ್ತಿಗೆಯ ಭಯಕ್ಕಾಗಿ ಶ‍್ರೀಕೃಷ್ಣನಿಗೆ ತನ್ನ ಒಳಗೇ ಇದ್ದುಬಿಡು ಎಂದು ಹೇಳಿದಳು. ಆದರೆ ಮತ್ತೆ ಅವಳು ಅವನ ದರ್ಶನಕ್ಕಾಗಿ ವ್ಯಾಕುಲಪಟ್ಟಾಗ ಕೃಷ್ಣ ಹೃದಯವನ್ನು ಬಿಟ್ಟು ಹೊರಗೆ ಬರಲು ಇಚ್ಛಿಸಲಿಲ್ಲ.”

ನರೇಂದ್ರಾದಿಗಳಿಗೆ ಹೇಳುತ್ತಾರೆ: “ಇದರೊಳಗೆ ಇಬ್ಬರು ಇದ್ದಾರೆ: ಒಬ್ಬ ವ್ಯಕ್ತಿ ಭಗವತಿ, ಇನ್ನೊಬ್ಬ ವ್ಯಕ್ತಿ ಆಕೆಯ ಭಕ್ತ. ಹಿಂದೆ ಕೈಮುರಿದುಕೊಂಡಿದ್ದವ ಆಕೆಯ ಭಕ್ತ, ಆ ಭಕ್ತನಿಗೆ ಈಗ ಖಾಯಿಲೆ ಆಗಿದೆ. ಅರ್ಥವಾಯಿತೆ? ಅಯ್ಯೋ, ಇದನ್ನೆಲ್ಲ ನಾನು ಯಾರಿಗೆ ಹೇಳಿಕೊಳ್ಳಲಿ? ಯಾರಿಗೆ ತಾನೆ ಅರ್ಥಮಾಡಿಕೊಳ್ಳಲು ಶಕ್ತಿ ಇದೆ? ಭಗವಂತ ಮನುಷ್ಯದೇಹಧಾರಣೆ ಮಾಡಿಕೊಂಡು ಅವತಾರ ಪುರುಷನಾಗಿ ಧರೆಗೆ ಬರುತ್ತಾನೆ. ಆತ ಹಿಂದಿರುಗುವಾಗ ಭಕ್ತರೂ ಆತನೊಡನೆ ಹಿಂತಿರುಗಿ ಹೋಗಿ ಬಿಡುತ್ತಾರೆ.”

ರಾಖಾಲ: “ಅದಕ್ಕಾಗಿಯೇ ನಾವು ಪ್ರಾರ್ಥನೆ ಮಾಡಿಕೊಳ್ಳುವುದು, ನಮ್ಮನ್ನು ಇಲ್ಲೆ ಬಿಟ್ಟು ನೀವು ಹೊರಟು ಹೋಗಕೂಡದು ಅಂತ”.

ಶ‍್ರೀರಾಮಕೃಷ್ಣ: “ಭಜನೆಯ ಒಂದು ಗುಂಪು ಇದ್ದಕ್ಕೆ ಇದ್ದಹಾಗೆ ಬಂತು, ನರ್ತಿಸಿತು, ಹಾಡಿತು ಮತ್ತು ಇದ್ದಕ್ಕೆ ಇದ್ದಹಾಗೆ ಹೊರಟುಹೋಯಿತು. ಅವರು ಬಂದರು ಮತ್ತೆ ಹೊರಟುಹೋದರು. ಅವರು ಯಾರು ಏನು ಎಂಬುದು ಯಾರಿಗೂ ಗೊತ್ತಾಗಲಿಲ್ಲ. ದೇಹಧಾರಣೆ ಮಾಡಿಕೊಂಡರೆ ಕಷ್ಟವನ್ನು ಅನುಭವಿಸಲೇಬೇಕಾಗುವುದು. ಒಮ್ಮೊಮ್ಮೆ ನನಗೆ ನಾನೇ ಹೇಳಿಕೊಳ್ಳುತ್ತೇನೆ, ಇನ್ನೊಮ್ಮೆ ನಾನು ಈ ಧರೆಗೆ ಬರದೆ ಇರುವಂತಾಗಲಿ ಎಂದು. ಆದರೆ ಒಂದು ವಿಷಯ, ಔತಣದ ಊಟ ಮಾಡಿ ಮಾಡಿ ಮನೆಯ ಗೊಡ್ಡುಸಾರು, ಅನ್ನ ರುಚಿಸದು. ಈ ದೇಹಧಾರಣೆ ಭಕ್ತರಿಗಾಗಿ.”

ಶ‍್ರೀರಾಮಕೃಷ್ಣರು ನರೇಂದ್ರನಿಗೆ ಹೇಳಿದರು: “ಒಬ್ಬ ಚಾಂಡಾಲ ಮಾಂಸವನ್ನು ಹೊತ್ತುಕೊಂಡು ಹೋಗುತ್ತಿದ್ದ. ಶಂಕರಾಚಾರ‍್ಯರು ಗಂಗಾಸ್ನಾನ ಮಾಡಿ ಬೀದಿಯಲ್ಲಿ ಹೋಗುತ್ತಿದ್ದರು. ಚಂಡಾಲ ಇದ್ದಕ್ಕೆ ಇದ್ದಹಾಗೆ ಅವರನ್ನು ಮುಟ್ಟಿಬಿಟ್ಟ. ಅವರು ಕೋಪಿಸಿಕೊಂಡು ಹೇಳಿದರು: ‘ಏನಿದು, ನೀನು ನನ್ನನ್ನು ಮುಟ್ಟಿಬಿಟ್ಟೆಯಲ್ಲ’ ಎಂದು. ಅದಕ್ಕೆ ಆತ ಹೇಳಿದ: ‘ಪೂಜ್ಯರೇ, ನಾನೂ ನಿಮ್ಮನ್ನು ಮುಟ್ಟಲಿಲ್ಲ. ನೀವೂ ನನ್ನನ್ನು ಮುಟ್ಟಲಿಲ್ಲ. ವಿಚಾರಮಾಡಿ ನೋಡಿ. ನೀವು ದೇಹವೆ? ಮನಸ್ಸೆ? ಬುದ್ಧಿಯೆ? ನೀವು ಏನು ಎಂಬುದನ್ನು ವಿಚಾರಮಾಡಿ ನೋಡಿ. ನೀವು ನಿರ್ಲಿಪ್ತ ಶುದ್ಧ ಆತ್ಮ. ಸತ್ತ್ವ, ರಜಸ್ಸು, ತಮಸ್ಸು ಈ ಯಾವ ಗುಣಗಳಿಂದಲೂ ಅದು ಲಿಪ್ತವಾಗಿಲ್ಲ’. ಬ್ರಹ್ಮ ಯಾವರೀತಿ ಇರುವನು ಎಂಬುದು ನಿಮಗೆ ಗೊತ್ತೆ? ಉದಾಹರಣೆಗೆ ಗಾಳಿ ಸುಗಂಧ ದುರ್ಗಂಧ ಎರಡನ್ನೂ ಹೊತ್ತು ತರುತ್ತದೆ. ಆದರೆ ಅದು ಮಾತ್ರ ಅವುಗಳಿಂದ ನಿರ್ಲಿಪ್ತ”

ನರೇಂದ್ರ: “ಹೌದು.”

ಶ‍್ರೀರಾಮಕೃಷ್ಣ: “ಆತ ಗುಣಾತೀತ ಮಾಯಾತೀತ. ವಿದ್ಯಾಮಾಯೆ, ಅವಿದ್ಯಾಮಾಯೆ ಇವೆರಡಕ್ಕೂ ಅತೀತ. ಕಾಮಕಾಂಚನವೇ ಅವಿದ್ಯಾಮಾಯೆ. ಜ್ಞಾನ ವೈರಾಗ್ಯ ಭಕ್ತಿ ಇವು ವಿದ್ಯಾಮಾಯೆಯ ಸಂಪತ್ತು. ಶಂಕರಾಚಾರ‍್ಯರು ವಿದ್ಯಾಮಾಯೆಯನ್ನು ಆಶ್ರಯಿಸಿದ್ದರು. ನೀವು ಮತ್ತು ಇವರೆಲ್ಲರೂ ನನಗಾಗಿ ಚಿಂತಿಸುತ್ತ ಇದ್ದೀರಿ. ಈ ಭಾವನೆ ವಿದ್ಯಾಮಾಯೆಗೆ ಸೇರಿದ್ದು. ವಿದ್ಯಾಮಾಯೆಯೇ ಮೆಟ್ಟಲಿನ ಸಾಲಿನಲ್ಲಿ ಕೊನೆಯದು. ಅದನ್ನು ಹತ್ತಿದೊಡನೆಯೇ ಚಾವಣಿ. ಕೆಲವರು ಚಾವಣಿಗೆ ಹತ್ತಿದ ಮೇಲೆ ಮೆಟ್ಟಲಿನ ಮೂಲಕ ಕೆಳಕ್ಕೆ ಮೇಲಕ್ಕೆ ಬಂದು ಹೋಗುತ್ತಿರುವರು. ಬ್ರಹ್ಮಜ್ಞಾನ ದೊರೆತ ನಂತರವೂ ವಿದ್ಯಾಮಾಯೆಯನ್ನು ಇಟ್ಟುಕೊಂಡಿರುವರು. ಅವರು ಅದನ್ನು ಲೋಕಶಿಕ್ಷಣಕ್ಕಾಗಿಯೂ ಭಕ್ತಿಯನ್ನು ಆಸ್ವಾದಿಸಲೋಸುಗವಾಗಿಯೂ ಇಟ್ಟುಕೊಂಡಿರುತ್ತಾರೆ.”

ನರೇಂದ್ರ: “ಕೆಲವರು ತ್ಯಾಗದ ವಿಷಯವನ್ನು ಎತ್ತಿದೆನೆಂದರೆ ನನ್ನ ಮೇಲೆ ಕೋಪಿಸಿಕೊಳ್ಳುತ್ತಾರೆ.”

ಶ‍್ರೀರಾಮಕೃಷ್ಣರು ಮೃದುಸ್ವರದಿಂದ: “ತ್ಯಾಗ ಆವಶ್ಯಕ. ಒಂದು ವಸ್ತುವನ್ನು ಇನ್ನೊಂದರ ಮೇಲೆ ಇಟ್ಟಿದ್ದರೆ, ಮೊದಲನೆಯದು ಬೇಕಾದರೆ ಎರಡನೆಯದನ್ನು ಸರಿಸಬೇಡವೇ? ಒಂದನ್ನು ತೆಗೆದಿಡದೆ ಇನ್ನೊಂದು ದೊರಕುವುದೇ!”

ನರೇಂದ್ರ: “ಹೌದು ನಿಜ.”

ಶ‍್ರೀರಾಮಕೃಷ್ಣ: “ಎಲ್ಲವೂ ಭಗವನ್ಮಯವಾಗಿ ಕಾಣುವಾಗ, ಬೇರೆ ಇನ್ನಾವುದನ್ನಾದರೂ ನೋಡಲು ಸಾಧ್ಯವೆ?”

ನರೇಂದ್ರ: “ಸಂಸಾರವನ್ನು ತ್ಯಾಗಮಾಡಲೇಬೇಕಾಗುತ್ತದೆ, ಅಲ್ಲವೇನು?”

ಶ‍್ರೀರಾಮಕೃಷ್ಣ: “ಈಗ ತಾನೆ ಹೇಳಿದೆನಲ್ಲ, ಎಲ್ಲವೂ ಭಗವನ್ಮಯವಾಗಿ ಕಾಣುವಾಗ ಬೇರೆ ಏನನ್ನಾದರೂ ನೋಡಲು ಸಾಧ್ಯವೆ? ಆಗ ಸಂಸಾರ ಮುಂತಾದುವನ್ನು ನೋಡಲು ಸಾಧ್ಯವೆ?”

“ಆದರೆ ಮಾನಸಿಕವಾಗಿ ತ್ಯಜಿಸಬೇಕು. ಇಲ್ಲಿಗೆ ಯಾರು ಬರುತ್ತಾರೊ ಅವರಲ್ಲಿ ಯಾರೂ ಸಂಸಾರಿಗಳಲ್ಲ. ಹೆಂಗಸಿನ ಸಹವಾಸ ಮಾಡಬೇಕು ಎಂದು ಎಲ್ಲೋ ಕೆಲವರಿಗೆ ಸ್ವಲ್ಪ ಇಚ್ಛೆ ಇತ್ತು. ಅದೊಂದಿಷ್ಟು ಇಚ್ಛೆ ಆಗಲೇ ಅವರಿಗೆ ಪೂರ್ಣವಾಗಿ ಹೋಗಿದೆ.”

ಶ್ರಿರಾಮಕೃಷ್ಣರು ನರೇಂದ್ರನನ್ನು ನೋಡುತ್ತ “ಭೇಷ್” ಎಂದರು. ನರೇಂದ್ರ ‘ಭೇಷ್ ಯಾವುದು?’ ಎಂದು ಕೇಳಿದ. ಅದಕ್ಕೆ ಶ‍್ರೀರಾಮಕೃಷ್ಣರು “ಮಹಾತ್ಯಾಗಕ್ಕೆ ಸಿದ್ಧತೆ ನಡೆಯುತ್ತಿದೆ” ಎಂದರು.

ರಾಖಾಲ: “ನರೇಂದ್ರನಿಗೆ ಈಗ ನೀವು ಹೇಳುವುದೆಲ್ಲ ಚೆನ್ನಾಗಿ ಅರ್ಥವಾಗುತ್ತಿದೆ.”

ಶ‍್ರೀರಾಮಕೃಷ್ಣ: “ಹೌದು, ನರೇಂದ್ರನೊಬ್ಬನಿಗೇ ಏತಕ್ಕೆ, ಉಳಿದವರಿಗೂ ಅಲ್ಲವೇನು? ನೋಡುತ್ತ ಇದ್ದೇನೆ ಎಲ್ಲಾ ಇದರಿಂದಾಗಿ ಬಂದಿರುವುದಾಗಿ.”

ನರೇಂದ್ರನಿಗೆ ಸಂಜ್ಞೆಯ ಮೂಲಕ: “ನೀನು ಏನು ಅರ್ಥಮಾಡಿಕೊಂಡೆ ಹೇಳು ನೋಡೋಣ” ಎಂದರು.

ನರೇಂದ್ರ: “ಸೃಷ್ಟಿಯಾಗಿರುವುದೆಲ್ಲ ನಿಮ್ಮಿಂದಲೇ ಬಂದಿದೆ.”

ಶ‍್ರೀರಾಮಕೃಷ್ಣರ ಮುಖ ಆನಂದದಿಂದ ಅರಳಿತು. ರಾಖಾಲನಿಗೆ “ನರೇಂದ್ರ ಹೇಳಿದ್ದನ್ನು ಕೇಳಿದೆಯಾ?” ಎಂದರು. ನರೇಂದ್ರನಿಗೆ ಒಂದು ಹಾಡನ್ನು ಹೇಳುವಂತೆ ಹೇಳಿದರು. ಶ‍್ರೀರಾಮಕೃಷ್ಣರು ಅದನ್ನು ಕೇಳುತ್ತಿದ್ದಂತೆ ಭಾವಾವಿಷ್ಟರಾದರು.

ನರೇಂದ್ರನಿಗೆ ಮೊದಲಿನಿಂದಲೂ ಬುದ್ಧನ ಜೀವನದ ಮೇಲೆ ಅತ್ಯಂತ ಪ್ರೀತಿ, ಅವನ ತ್ಯಾಗ ಮುಂತಾದುವನ್ನು ಮುಕ್ತಕಂಠದಿಂದ ಹೊಗಳುತ್ತಿದ್ದ. ಶ‍್ರೀರಾಮಕೃಷ್ಣರ ಶಿಷ್ಯರು ತಾವು ಧ್ಯಾನ ಮಾಡುವ ಕೋಣೆಯ ಗೋಡೆಯ ಮೇಲೆ ಬುದ್ಧನ ಈ ಪ್ರತಿಜ್ಞೆಯ ನುಡಿಗಳನ್ನು ಬರೆದಿದ್ದರು: “ನನ್ನ ದೇಹ ಈ ಆಸನದ ಮೇಲೆ ಒಣಗಿ ಹೋದರೂ ಚಿಂತೆಯಿಲ್ಲ. ಹಲವು ಜನ್ಮಗಳು ಸಾಧನೆ ಮಾಡಿದರೂ ದುರ್ಲಭವಾದ ಆ ಅರಿವನ್ನು ಕಾಣುವವರೆಗೆ, ಈ ಸ್ಥಳದಿಂದ ಬಿಟ್ಟು ಏಳುವುದಿಲ್ಲ.” ನರೇಂದ್ರನು ತಾರಕ ಮತ್ತು ಕಾಳಿಯೊಡನೆ ಬುದ್ಧ ತಪಸ್ಸು ಮಾಡಿದ ಗಯೆಗೆ ಹೋಗಬೇಕೆಂದು ಮನಸ್ಸು ಮಾಡಿದನು. ತಾರಕ ಹೋಗುವುದಕ್ಕೆ ಬೇಕಾದ ಹಣವನ್ನು ಒದಗಿಸಿದನು. ಕಾಶೀಪುರದಲ್ಲಿ ಯಾರಿಗೂ ಹೇಳದೆ ಮೂರು ಜನರೂ ಬುದ್ಧಗಯೆಗೆ ಹೋದರು. ಹೋಗುವಾಗ ಮೂರು ಜನರೂ ಗೈರಿಕ ವಸನವನ್ನು ಧರಿಸಿದ್ದರು.

ಗಯೆಯ ರೈಲ್ವೆ ನಿಲ್ದಾಣದಿಂದ ಏಳು ಮೈಲಿಗಳು ನಡೆದುಕೊಂಡು ಹೋಗಿ ಬುದ್ಧ ಗಯೆಯನ್ನು ಸೇರಿದರು. ಒಂದು ದಿನ ಸಾಯಂಕಾಲ ಬುದ್ಧ ಕುಳಿತು ತಪಸ್ಸು ಮಾಡಿದ ಆಲದ ಮರದ ಕೆಳಗೆ ಮೂವರೂ ಧ್ಯಾನಕ್ಕೆ ಕುಳಿತರು. ನರೇಂದ್ರ ಬುದ್ಧನನ್ನು ಕುರಿತು ಧ್ಯಾನ ಮಾಡುತ್ತಿದ್ದಂತೆ ಅಶ್ರುಲೋಚನನಾಗಿ ಪಕ್ಕದಲ್ಲಿ ಕುಳಿತಿದ್ದ ತಾರಕನನ್ನು ಅಪ್ಪಿಕೊಂಡನು. ತಾರಕ ಬೆಚ್ಚು ಬೆರಗಾಗಿ ಅದಕ್ಕೆ ಕಾರಣವನ್ನು ಕೇಳಿದನು. “ಬುದ್ಧನ ಅನುಕಂಪವನ್ನು ಕುರಿತು ಚಿಂತಿಸುತ್ತಿದ್ದಂತೆ ತನಗೆ ಅದನ್ನು ತಡೆಯಲು ಸಾಧ್ಯವಾಗಲಿಲ್ಲ” ಎಂದನು ನರೇಂದ್ರ.

ಕಾಶೀಪುರದಲ್ಲಿ ನರೇಂದ್ರ ಇಲ್ಲದಿರುವುದು ಎಲ್ಲರಿಗೂ ಬೇಜಾರಾಯಿತು. ಯಾರಾದರೂ ಬೋಧಗಯೆಗೆ ಹೋಗಿ ನರೇಂದ್ರನನ್ನು ಕರೆದುಕೊಂಡು ಬರಬೇಕೆಂದು ಇದ್ದರು.\break ಶ‍್ರೀರಾಮಕೃಷ್ಣರಿಗೆ ಇದು ಗೊತ್ತಾದಾಗ “ನೀವು ಏತಕ್ಕೆ ಇಷ್ಟೊಂದು ಚಿಂತಾಮಗ್ನರಾಗಿರುವುದು? ಅವನು ಎಲ್ಲಿಗೆ ಹೋಗುತ್ತಾನೆ, ಅವನು ಎಷ್ಟು ದಿವಸ ಇಲ್ಲಿಂದ ಹೋಗಿ ಇರುತ್ತಾನೆ, ಅವನು ಶೀಘ್ರದಲ್ಲಿಯೇ ಬರುತ್ತಾನೆ” ಎಂದರು. ಅನಂತರ ಹತ್ತಿರ ಇದ್ದ ಭಕ್ತರನ್ನು ಕುರಿತು ಹೀಗೆ ಹೇಳಿದರು: “ನೀವು ಪ್ರಪಂಚವನ್ನೆಲ್ಲ ಸುತ್ತಾಡಿಕೊಂಡು ಬನ್ನಿ, ನಿಜವಾದ ಧರ್ಮ ಬೇರೆ ಇನ್ನೆಲ್ಲಿಯೂ ಇಲ್ಲ ಎಂಬುದು ನಿಮಗೆ ಗೊತ್ತಾಗುವುದು. ಇರುವ ಆಧ್ಯಾತ್ಮಿಕತೆಯೆಲ್ಲ ಇಲ್ಲಿಯೇ ಇದೆ.” ಶ‍್ರೀರಾಮಕೃಷ್ಣರು ‘ಇಲ್ಲಿ’ ಎನ್ನುವುದನ್ನು ಎರಡು ದೃಷ್ಟಿಯಿಂದ ನೊಡಬಹುದು. ಇಲ್ಲಿ ಎಂದರೆ ಒಬ್ಬನ ಆಂತರ‍್ಯದಲ್ಲಿ ಅದು ಇದ್ದರೆ ಬೇರೆ ಕಡೆಯೂ ಅದು ಕಾಣುವುದು. ಎರಡನೆಯದೆ, ಪ್ರಚಂಡ ಸಾಧನೆಯನ್ನು ಮಾಡಿ ಯಾವ ಸಾಕ್ಷಾತ್ಕಾರವನ್ನು ಇಲ್ಲಿ ಎಂದರೆ ಅವರ ದೇಹದ ಮೂಲಕ ಪಡೆದುಕೊಂಡರೊ ಅದು ಬೇರೆಲ್ಲಿಯೂ ಅಷ್ಟು ಉತ್ಕಟವಾಗಿ, ತೀವ್ರವಾಗಿ ವ್ಯಕ್ತವಾಗಿಲ್ಲ ಎಂಬುದು.

ನರೇಂದ್ರ ಮತ್ತು ಅವನ ಇಬ್ಬರು ಸ್ನೇಹಿತರು ಭೋದಗಯೆಯಲ್ಲಿದ್ದ ಹಿಂದೂ ಮಠದ ಮಠಾಧಿಪತಿಗಳ ಅತಿಥಿಗಳಾಗಿ ನಾಲ್ಕು ದಿನಗಳು ಇದ್ದರು. ಅದಾದ ಮೇಲೆ ಶ‍್ರೀರಾಮಕೃಷ್ಣರು ಈಗ ಹೇಗಿರುವರೋ ಎಂಬುದನ್ನು ನೋಡಬೇಕೆಂದು ಬಯಸಿದರು. ದಾರಿಯಲ್ಲಿ ಖರ್ಚಿನ ಸ್ವಲ್ಪ ಭಾಗವನ್ನು ಆ ಮಠದಿಂದ ಪಡೆದು ಗಯಾ ನಗರಿಗೆ ಬಂದರು. ಅಲ್ಲಿ ನರೇಂದ್ರನ ತಂದೆಯ ಸ್ನೇಹಿತನೊಬ್ಬ ವಕೀಲನಾಗಿದ್ದನು. ನರೇಂದ್ರನಿಗೆ ಆತನ ಪರಿಚಯವಿತ್ತು. ಆತ ಇವನು ಮತ್ತು ಇವನ ಸಂಗಡಿಗರನ್ನು ಕಂಡೊಡನೆ ತನ್ನ ಮನೆಗೆ ಕರೆದುಕೊಂಡುಹೋದನು, ಅಲ್ಲಿ ಅವರಿಗೆ ಊಟ ಕೊಟ್ಟನು. ಭಜನೆ ಮುಂತಾದವುಗಳಲ್ಲಿ ಕೆಲವು ಗಂಟೆಗಳನ್ನು ಕಳೆದಾದ ಮೇಲೆ ಕಲ್ಕತ್ತೆಗೆ ಮುಟ್ಟಬೇಕಾದರೆ ಇನ್ನುಳಿದ ದಾರಿಯ ಖರ್ಚನ್ನೆಲ್ಲ ಅವನು ಕೊಟ್ಟು ಕಳುಹಿಸಿದನು. ನರೇಂದ್ರ ಕಾಶೀಪುರಕ್ಕೆ ಬಂದಮೇಲೆ ಎಲ್ಲರಿಗೂ ಆನಂದವಾಯಿತು. ಶ‍್ರೀರಾಮಕೃಷ್ಣರು ಬುದ್ಧಗಯೆಗೆ ಸಂಬಂಧಪಟ್ಟ ವಿಷಯವನ್ನೆಲ್ಲ ವಿವರಿಸುವಂತೆ ನರೇಂದ್ರನಿಗೆ ಹೇಳಿದರು.

ಶ‍್ರೀರಾಮಕೃಷ್ಣರು ಮಹೇಂದ್ರನಿಗೆ: “ನರೇಂದ್ರ ಬುದ್ಧಗಯೆಗೆ ಹೋಗಿದ್ದ.”

ಮಹೇಂದ್ರ ನರೇಂದ್ರನಿಗೆ: “ಬುದ್ಧನ ತತ್ತ್ವವೇನು?”

ನರೇಂದ್ರ: “ಆತ ತನ್ನ ತಪಸ್ಸಿನ ಕೊನೆಯಲ್ಲಿ ಏನನ್ನು ಸಾಕ್ಷಾತ್ಕಾರ ಮಾಡಿಕೊಂಡನೋ ಅದನ್ನು ಬಾಯಿಂದ ಹೇಳಲು ಅವನಿಗೆ ಸಾಧ್ಯವಾಗಲಿಲ್ಲ. ಅದಕ್ಕಾಗಿ ಜನ ಅವನನ್ನು ನಾಸ್ತಿಕ ಎಂದು ಕರೆದರು.”

ಶ‍್ರೀರಾಮಕೃಷ್ಣರು ಸಂಜ್ಞೆಯ ಮೂಲಕ: “ಅವನೇಕೆ ನಾಸ್ತಿಕನಾಗುತ್ತಾನೆ? ನಾಸ್ತಿಕನಲ್ಲ. ತಾನು ಪಡೆದ ಅನುಭವವನ್ನು ಬಾಯಿಂದ ಇತರರಿಗೆ ಹೇಳಲು ಆಗಲಿಲ್ಲ ಅಷ್ಟೆ. ಬುದ್ಧ ಎಂದರೆ ಏನು ಗೊತ್ತೆ? ಬೋಧಸ್ವರೂಪವನ್ನು ಚಿಂತನೆ ಮಾಡುತ್ತ ಮಾಡುತ್ತ ಅದೇ ಆಗುವಿಕೆ, ಬೋಧಸ್ವರೂಪನಾಗುವಿಕೆ.”

ನರೇಂದ್ರ: “ಹೌದು ನಿಶ್ಚಯ. ಇದರಲ್ಲಿ ಮೂರು ಅಂತಸ್ತಿನವರಿದ್ದಾರೆ. ಬುದ್ಧ ಅರ್ಹತ್ ಮತ್ತು ಬೋಧಿಸತ್ವ.”

ಶ‍್ರೀರಾಮಕೃಷ್ಣ: “ಇದೂ ಆತನ ಕ್ರೀಡೆಯೆ. ಒಂದು ಹೊಸ ಲೀಲೆ.” “ಬುದ್ಧ ಹೇಗೆ ನಾಸ್ತಿಕನಾಗುತ್ತಾನೆ? ಸ್ವಸ್ವರೂಪದ ಬೋಧೆಯಾದ ನಂತರ, ಅಸ್ತಿ ನಾಸ್ತಿ ಇವುಗಳ ಮಧ್ಯದಲ್ಲಿ ವರ್ತಿಸುವ ಸ್ಥಿತಿಯಲ್ಲಿ ಇರಬೇಕಾಗುತ್ತದೆ.”

ನರೇಂದ್ರ ಮಹೇಂದ್ರನಿಗೆ: “ಆ ಅವಸ್ಥೆಯಲ್ಲಿ ಅಸಂಗತವಾದುವು ಸಂಗತವಾಗುತ್ತವೆ. ಯಾವ ಆಮ್ಲಜನಕ ಜಲಜನಕಗಳ ಸಂಯೋಗದಿಂದ ತಣ್ಣಗಿರುವ ನೀರು ಉತ್ಪನ್ನವಾಗುವುದೋ, ಅದೇ ಆಮ್ಲಜನಕ, ಜಲಜನಕಗಳ ಸಂಯೋಗದಿಂದ ತಪಿಸುವ ಶಾಖವೂ ಉತ್ಪನ್ನವಾಗುತ್ತದೆ.”

“ಆ ಅವಸ್ಥೆಯಲ್ಲಿ ಕರ್ಮ ಮತ್ತು ಕರ್ಮ ತ್ಯಾಗ ಎರಡೂ ಸಾಧ್ಯ. ಅಂದರೆ ನಿಷ್ಕಾಮಕರ್ಮ ಸಾಧ್ಯ.”

“ಇಂದ್ರಿಯ ಸುಖಗಳಲ್ಲಿಯೇ ಮುಳುಗಿರುವ ಸಾಂಸಾರಿಕರು ಎಲ್ಲವೂ ‘ಅಸ್ತಿ’ ಎಂದು ಹೇಳುತ್ತಾರೆ. ಮಾಯಾವಾದಿಗಳು ಎಲ್ಲವೂ ನಾಸ್ತಿ ಎಂದು ಹೇಳುತ್ತಾರೆ. ಬುದ್ಧನ ಅನುಭವ ‘ಅಸ್ತಿ ನಾಸ್ತಿ’ಗಳಿಗೆ ಅತೀತವಾದದ್ದು.”

ಶ‍್ರೀರಾಮಕೃಷ್ಣ: ‘ಅಸ್ತಿ ನಾಸ್ತಿ ಪ್ರಕೃತಿಯ ಗುಣಗಳು. ಆದರೆ ಸತ್ಯ ಇವೆರಡಕ್ಕೂ ಅತೀತವಾದುದು. (ನರೇಂದ್ರನಿಗೆ) ಬುದ್ಧದೇವ ಕೊಟ್ಟ ಉಪದೇಶಗಳೇನು?’

ನರೇಂದ್ರ: “ಭಗವಂತ ಇರುವನೆ ಇಲ್ಲವೆ ಎಂಬ ವಿಷಯದಲ್ಲಿ ಅವನು ಮಾತನಾಡುತ್ತಲೇ ಇರಲಿಲ್ಲ. ಆದರೆ ತನ್ನ ಇಡೀ ಜೀವನವನ್ನೆಲ್ಲ ದಯಾಮೂರ್ತಿಯಾಗಿ ಕಳೆದ. ಒಂದು ಗಿಡುಗ, ಹಕ್ಕಿಯನ್ನು ಹಿಡಿದು ಅದನ್ನು ಇನ್ನು ಏನು ತಿನ್ನುವುದರಲ್ಲಿತ್ತು. ಆ ಹಕ್ಕಿಯನ್ನು ಉಳಿಸಲು ಬುದ್ಧ ತನ್ನ ದೇಹದಿಂದ ಮಾಂಸವನ್ನು ಕತ್ತರಿಸಿ ಕೊಟ್ಟ. ಎಂಥ ವೈರಾಗ್ಯ! ರಾಜಕುಮಾರನಾಗಿದ್ದರೂ ಎಲ್ಲವನ್ನೂ ತ್ಯಜಿಸಿದ. ಯಾರಿಗೆ ಏನು ಇಲ್ಲವೋ ಅವನು ತ್ಯಜಿಸುವುದು ತಾನೆ ಏನನ್ನು? ಬುದ್ಧನಾದ ಮೇಲೆ ನಿರ್ವಾಣವನ್ನು ಹೊಂದಿದಮೇಲೆ ಆತ ತನ್ನ ಮನೆಗೆ ಹಿಂದಿರುಗುವಾಗ, ಹೆಂಡತಿ, ಮಗ ಮತ್ತು ರಾಜಮನೆತನದ ಇತರರಿಗೆ ವೈರಾಗ್ಯವನ್ನು ಅವಲಂಬಿಸುವಂತೆ ಉಪದೇಶವಿತ್ತ. ಎಂಥ ಅದ್ಭುತ ವೈರಾಗ್ಯ ಆತನದು! ಹಾಗೇನೇ ವ್ಯಾಸದೇವನ ವರ್ತನೆಯನ್ನು ನೋಡಿ! ಆತ ತನ್ನ ಮಗ ಶುಕದೇವನಿಗೆ ಹೇಳಿದ: ‘ಮಗು ಸಂಸಾರದಲ್ಲಿದ್ದೇ ತಪಸ್ಸನ್ನು ಮಾಡು’ ಎಂದು. ಸಂಸಾರವನ್ನು ತ್ಯಜಿಸಲು ಅವನಿಗೆ ಅನುಮತಿಯನ್ನು ಕೊಡಲಿಲ್ಲ.”

“ಬುದ್ಧ ಪವಾಡಗಳನ್ನು ಒಪ್ಪುತ್ತಿರಲಿಲ್ಲ. ಆತನದು ಎಂತಹ ಅದ್ಭುತ ವೈರಾಗ್ಯ! ಆಲದ ಮರದ ಕೆಳಗೆ ತಪಸ್ಸಿಗೆ ಕುಳಿತುಕೊಂಡು ಹೇಳಿದ. ‘ನನಗೆ ನಿರ್ವಾಣ ದೊರೆಯದೇ ಹೋದರೆ ಈ ಶರೀರ ಇಲ್ಲೇ ಒಣಗಿ ಬತ್ತಿಹೋಗಲಿ.’ ಎಂಥ ದೃಢ ಪ್ರತಿಜ್ಞೆ ಆತನದು!”

“ಈ ಶರೀರವೇ ನಮ್ಮ ದೊಡ್ಡ ಶತ್ರು. ಇದನ್ನು ಅಂಕೆಯಲ್ಲಿಟ್ಟುಕೊಳ್ಳದೆ ಇದ್ದರೆ ಏನಾದರೂ ದೊರಕುವುದೆ?”

ಶ‍್ರೀರಾಮಕೃಷ್ಣ: “ಬುದ್ಧನ ತಲೆಜುಟ್ಟನ್ನು ನೋಡಿದೀಯೇನು?” ಎಂದು ನರೇಂದ್ರನನ್ನು ಕೇಳಿದರು.

ನರೇಂದ್ರ: “ಅದು ಜುಟ್ಟಲ್ಲ, ಕಿರೀಟ ಇರಬೇಕು. ಅದು ಅನೇಕ ರುದ್ರಾಕ್ಷಿಗಳನ್ನು ಸೇರಿಸಿ ಮಾಡಿದ ಕಿರೀಟದ ಹಾಗೆ ಕಾಣುವುದು.”

ಶ‍್ರೀರಾಮಕೃಷ್ಣ: “ಆತನ ಕಣ್ಣುಗಳು?”

ನರೇಂದ್ರ: “ಸಮಾಧಿಯಲ್ಲಿದ್ದ ಹಾಗೆ.”

ಶ‍್ರೀರಾಮಕೃಷ್ಣ: “ಒಳ್ಳೆಯದು, ಇಲ್ಲಿ ಎಲ್ಲಾ ದೊರೆಯುತ್ತದೆ ಅಲ್ಲವೆ? ತೊಗರಿಬೇಳೆ ಕಡಲೇಬೇಳೆ ಹುಣಸೇಹಣ್ಣು ಕೂಡ.”

ನರೇಂದ್ರ: “ನೀವು ಎಲ್ಲ ಅಂತಸ್ತುಗಳನ್ನು ಹತ್ತಿ, ಈಗ ಕೆಳಗಿನ ಹಂತದಲ್ಲಿ ಓಡಾಡುತ್ತಿರುವಿರಿ.”

ಶ‍್ರೀರಾಮಕೃಷ್ಣ: “ಯಾರೋ ನನ್ನನ್ನು ಈ ಕೆಳಗಿನ ಅಂತಸ್ತಿನಲ್ಲಿ ಹಿಡಿದು ನಿಲ್ಲಿಸಿರುವ ಹಾಗೆ ಕಾಣುತ್ತದೆ.”

ಶ‍್ರೀರಾಮಕೃಷ್ಣರು ಮಹೇಂದ್ರನ ಕೈಯಿಂದ ಬೀಸಣಿಗೆಯನ್ನು ತೆಗೆದುಕೊಂಡು ಹೇಳುತ್ತಿದ್ದಾರೆ: “ಈ ಬೀಸಣಿಗೆಯನ್ನು ಪ್ರತ್ಯಕ್ಷ ನೋಡುವ ರೀತಿಯಲ್ಲಿ, ಭಗವಂತನನ್ನು ನೋಡಿದ್ದೇನೆ.”

ಹೀಗೆಂದು ಹೇಳಿ ತಮ್ಮ ಹೃದಯದ ಮೇಲೆ ಕೈಯನ್ನು ಇಟ್ಟು ಸಂಜ್ಞೆ ಮೂಲಕ ಹೇಳುತ್ತಿದ್ದಾರೆ: “ನಾನು ಹೇಳಿದ್ದು ಏನು ಹೇಳು ನೋಡೋಣ.”

ನರೇಂದ್ರ: “ತಿಳಿದುಕೊಂಡುಬಿಟ್ಟೆ.”

ಶ‍್ರೀರಾಮಕೃಷ್ಣ: “ಏನು ತಿಳಿದುಕೊಂಡೆ ಹೇಳು ನೋಡೋಣ?”

ನರೇಂದ್ರ: “ಚೆನ್ನಾಗಿ ಕಿವಿಗೆ ಬೀಳಲಿಲ್ಲ.”

ಶ‍್ರೀರಾಮಕೃಷ್ಣರು ಮತ್ತೆ ಸಂಜ್ಞೆಯ ಮೂಲಕ ಸೂಚಿಸಿದರು: “ಭಗವಂತ ಮತ್ತು ನನ್ನ ಹೃದಯದಲ್ಲಿ ವಾಸಿಸುತ್ತಿರುವವನು ಇಬ್ಬರೂ ಒಂದೇ” ಎಂದು.

ನರೇಂದ್ರ: “ಹೌದು, ಹೌದು, ಸೋಽಹಂ.”

ಶ‍್ರೀರಾಮಕೃಷ್ಣ: “ಇಬ್ಬರಿಗೂ ಒಂದು ಗೆರೆಯ ಅಂತರ ಮಾತ್ರ ಇದೆ. ಭಕ್ತನ ಅಹಂ ಭಗವದಾನಂದವನ್ನು ಪಡೆಯಲೋಸುಗ ಇದೆ.”

ನರೇಂದ್ರ ಮಹೇಂದ್ರನಿಗೆ: “ಮಹಾಪುರುಷರು ತಮ್ಮ ಆತ್ಮ ಸಾಕ್ಷಾತ್ಕಾರವನ್ನು ಪಡೆದ ನಂತರವೂ ಜೀವನ ಉದ್ಧಾರಕ್ಕೆ ‘ಅಹಂ’ ಎಂಬುದನ್ನು ಉಳಿಸಿಕೊಂಡು ದೇಹದ ಸುಖದುಃಖಗಳನ್ನು ಅನುಭವಿಸುತ್ತ ಜೀವಿಸಿರುತ್ತಾರೆ.”

“ನಾವು ಚಾವಟಿ ಏಟಿನಿಂದ ಕೆಲಸ ಮಾಡುತ್ತೇವೆ; ಅವರು ಸ್ವೇಚ್ಛೆಯಿಂದ ಮಾಡುತ್ತಾರೆ.”

ಶ‍್ರೀರಾಮಕೃಷ್ಣರಿಗೆ ನರೇಂದ್ರ: “ಮೇಲ್ಛಾವಣಿಯೇನೋ ಕಣ್ಣಿಗೆ ಕಾಣುತ್ತದೆ. ಆದರೆ ಅದರ ಮೇಲಕ್ಕೆ ಹತ್ತಿ ಹೋಗಬೇಕಾದರೆ ಬಹಳ ಪ್ರಯಾಸ ಪಡಬೇಕು.”

ಶ‍್ರೀರಾಮಕೃಷ್ಣ: “ಆದರೆ ಆಗಲೆ ಯಾರಾದರೂ ಮೇಲಕ್ಕೆ ಹತ್ತಿ ಹೊಗಿಬಿಟ್ಟಿದ್ದರೆ ಆತ ಅಲ್ಲಿಂದ ಕೆಳಗೆ ಹಗ್ಗವನ್ನು ಬಿಟ್ಟು ಮೇಲಕ್ಕೆ ಎಳೆದುಕೊಳ್ಳಬಲ್ಲ.”

“ಒಮ್ಮೆ ಹೃಷೀಕೇಶದ ಒಬ್ಬ ಸಾಧು ಇಲ್ಲಿಗೆ ಬಂದಿದ್ದ. ಆತ ನನಗೆ ಹೇಳಿದ, ಏನು ಆಶ್ಚರ‍್ಯ! ನಿಮ್ಮಲ್ಲಿ ಐದು ವಿಧವಾದ ಸಮಾಧಿಗಳೂ ವ್ಯಕ್ತವಾಗಿವೆಯಲ್ಲ!”

“ಕೆಲವು ವೇಳೆ ಕಪಿಯಂತೆ: ಯಾವ ರೀತಿ ಕಪಿ ಕೊಂಬೆಯಿಂದ ಕೊಂಬೆಗೆ ನೆಗೆಯುತ್ತ ಹೇಗೆ ಮರವನ್ನು ಹತ್ತುವುದೋ ಅದೇ ರೀತಿ ಮಹಾವಾಯು ದೇಹದಲ್ಲಿರುವ ಚಕ್ರದಿಂದ ಚಕ್ರಕ್ಕೆ ನೆಗೆಯುತ್ತ ಮುಂದುವರಿಯುತ್ತದೆ. ಕೊನೆಗೆ ಸಮಾಧಿ.”

“ಕೆಲವು ವೇಳೆ ಹಕ್ಕಿಯಂತೆ: ಯಾವ ರೀತಿ ಹಕ್ಕಿ ರೆಂಬೆಯಿಂದ ರೆಂಬೆಗೆ ಹಾರಿ ಮುಂದುವರಿಯುವುದೊ ಅದೇ ರೀತಿ ಮಹಾವಾಯು ದೇಹವೆಂಬ ವೃಕ್ಷದಲ್ಲಿ ಮುಂದುವರಿಯುತ್ತದೆ. ಕೊನೆಗೆ ಸಮಾಧಿ.”

“ಕೆಲವು ವೇಳೆ ಇರುವೆಯಂತೆ: ಮಹಾವಾಯು ಇರುವೆಯೋಪಾದಿಯಲ್ಲಿ ದೇಹದಲ್ಲಿ ಸ್ವಲ್ಪ ಸ್ವಲ್ಪವಾಗಿ ಮುಂದುವರಿಯುತ್ತದೆ. ಅದು ಸಹಸ್ರಾರವನ್ನು ಮುಟ್ಟುವುದೇ ತಡ ಸಮಾಧಿ.”

“ಕೆಲವು ವೇಳೆ ಸರ್ಪದಂತೆ: ಮಹಾವಾಯು ಸರ್ಪದ ಹಾಗೆ ಡೊಂಕುಡೊಂಕಾಗಿ ಮುಂದುವರಿಯುತ್ತದೆ. ಸಹಸ್ರಾರವನ್ನು ಮುಟ್ಟಿದಾಗ ಸಮಾಧಿ.”

ಶ‍್ರೀರಾಮಕೃಷ್ಣರು ಅಂದು ಅಲ್ಲಿಗೆ ತಮ್ಮ ಸಂಭಾಷಣೆಯನ್ನು ನಿಲ್ಲಿಸಿದರು.

ಶ‍್ರೀರಾಮಕೃಷ್ಣರು ಹಿಂದೆ ನರೇಂದ್ರನಿಗೆ ಅಷ್ಟಾವಕ್ರ ಗೀತೆಯನ್ನು ಹೇಳುತ್ತಿದ್ದಾಗ ಅವನು ಅದನ್ನು ಟೀಕಿಸುತ್ತಿದ್ದ. ಒಂದು ದಿನ ಇವನ ಟೀಕೆ ಬಲವಾದಾಗ ಶ‍್ರೀರಾಮಕೃಷ್ಣರು ನಗುತ್ತ ಅವನನ್ನು ಮುಟ್ಟಿದಾಗ ಎಲ್ಲಾ ಹೇಗೆ ಒಂದೇ ಪರಬ್ರಹ್ಮ ವಸ್ತುವಿನಿಂದ ಕೂಡಿದೆಯೋ ಅದನ್ನು ಅನುಭವಿಸಿದ. ಆದರೆ ಅಲ್ಲಿಯೂ ನಾಮರೂಪದ ಅಲೆಗಳು ಆ ಪರಬ್ರಹ್ಮ ವಸ್ತುವಿನ ಮೇಲೆ ಇದ್ದುದನ್ನು ನೋಡಿದ್ದ, ನಾಮರೂಪಾತೀತವಾದ ಅನುಭವವನ್ನು ಪಡೆಯಬೇಕೆಂದು ಹಂಬಲಿಸಿದ್ದ, ಶ‍್ರೀರಾಮಕೃಷ್ಣ\-ರಿಗೆ ತನಗೆ ಅದನ್ನು ಅನುಗ್ರಹಿಸಬೇಕೆಂದು ಬೇಡಿಕೊಂಡಿದ್ದ. ಆದರೆ ಶ‍್ರೀರಾಮಕೃಷ್ಣರು ಅದನ್ನು ಅನುಗ್ರಹಿಸಲಿಲ್ಲ. ಅವನು ಅದರ ಸವಿಯನ್ನು ರುಚಿ ನೋಡಿದರೆ, ಈ ಪ್ರಪಂಚವನ್ನು ಬಿಟ್ಟು ಓಡಿಹೋಗಬಹುದು, ಇವನಿಂದ ಆಗ ಯಾವ ಕಾರ್ಯವೂ ಸಾಧ್ಯವಾಗುವುದಿಲ್ಲ, ಇವನು ಪ್ರಪಂಚಕ್ಕೆ ಬಂದದ್ದು ನಷ್ಟವಾಗುವುದು ಎಂದು ಭಾವಿಸಿದ್ದರು. ಆದರೆ ನರೇಂದ್ರನಿಗಾದರೊ ಅದನ್ನು ಅನುಭವಿಸುವವರೆಗೆ ತೃಪ್ತಿಯಿಲ್ಲ. ಒಂದು ದಿನ ಅವನು ಧ್ಯಾನ ಮಾಡುತ್ತಿದ್ದಾಗ ಮನಸ್ಸು ದೇಶ ಕಾಲಾತೀತವಾದ ಅವಸ್ಥೆಗೆ ಹೋಯಿತು. ಕೆಲವು ಕಾಲವಾದಮೇಲೆ ನರೇಂದ್ರನ ಬಾಯಿಯಿಂದ ಗೋಪಾಲದಾದನಿಗೆ “ಗೋಪಾಲದಾದ ನನ್ನ ದೇಹ ಎಲ್ಲಿ?” ಎಂಬ ಧ್ವನಿ ಕೇಳಿ ಬಂತು. ಆತ “ಅದೇಕೆ ಅದು ಇಲ್ಲೇ ಇದೆ” ಎಂದು ಹೇಳಿ ನರೇಂದ್ರನ ಸ್ಥಿರವಾದ ದೇಹವನ್ನು ಮುಟ್ಟಿ ತೋರಿದ. ಆದರೂ ಆ ದೇಹಕ್ಕೆ ಪ್ರಜ್ಞೆಯೇ ಇಲ್ಲದಂತೆ ಕಂಡಿತು. ತತ್‍ಕ್ಷಣವೇ ಶ‍್ರೀರಾಮಕೃಷ್ಣರ ಕೋಣೆಗೆ ಗೋಪಾಲದಾದ ಓಡಿಹೋಗಿ ಆದುದನ್ನು ವಿವರಿಸಿದ. ಶ‍್ರೀರಾಮಕೃಷ್ಣರು ಉದ್ವೇಗಗೊಳ್ಳದೆ, “ಅವನು ನನ್ನನ್ನು ಕಾಡುತ್ತಿದ್ದ, ಆ ಅವಸ್ಥೆಯನ್ನು ಪಡೆಯಬೇಕು ಎಂದು. ಆ ಸ್ಥಿತಿಯಲ್ಲಿಯೇ ಸ್ವಲ್ಪ ಕಾಲ ಇರಲಿ” ಎಂದರು.

ರಾತ್ರಿ ಒಂಭತ್ತು ಗಂಟೆಯಾದಮೇಲೆ ನರೇಂದ್ರ ಕ್ರಮೇಣ ಪ್ರಕೃತಿಸ್ಥನಾದ. ಅವನ ಮನಸ್ಸು ಅವರ್ಣನೀಯವಾದ ಆನಂದದಿಂದ ತುಂಬಿ ತುಳುಕುತ್ತಿತ್ತು. ನಿರಪೇಕ್ಷವಾದ ಅದ್ವೈತಾನುಭವಕ್ಕೆ ಮಾತ್ರ ಎಲ್ಲಾ ಸಿದ್ಧಾಂತಗಳಲ್ಲಿ ಒಂದು ಸಾಮರಸ್ಯ ಬರುವಂತೆ ಮಾಡಲು ಸಾಧ್ಯ ಎಂದು ಅನಂತರ ಅರಿಯತೊಡಗಿದ. ಇದನ್ನೇ ಶ‍್ರೀರಾಮಕೃಷ್ಣರು ಅದ್ವೈತವನ್ನು ಸೊಂಟದಲ್ಲಿ ಕಟ್ಟಿಕೊಂಡು ಒಬ್ಬ ಯಾವುದನ್ನಾದರೂ ನಂಬಲಿ ಚಿಂತೆಯಿಲ್ಲ ಎಂದು ಹೇಳುತ್ತಿದ್ದರು. ನರೇಂದ್ರ ಸ್ವಲ್ಪಕಾಲವಾದ ಮೇಲೆ ಶ‍್ರೀರಾಮಕೃಷ್ಣರ ಸಮೀಪಕ್ಕೆ ಬಂದ. ಆಗ ಶ‍್ರೀರಾಮಕೃಷ್ಣರು ಹೀಗೆ ಹೇಳಿದರು: “ಈಗ ತಾಯಿ ನಿನಗೆ ಎಲ್ಲವನ್ನೂ ತೋರಿರುವಳು. ಹೇಗೆ ನಿಧಿಯನ್ನು ಒಂದು ಪೆಟ್ಟಿಗೆಯಲ್ಲಿಟ್ಟು ಬೀಗವನ್ನು ಹಾಕುತ್ತಾರೆಯೊ ಹಾಗೆ, ಈ ಅನುಭವಕ್ಕೆ ಈಗ ಬೀಗ ಹಾಕಿದೆ. ಅದರ ಕೀಲಿಕೈ ನನ್ನ ಕೈಯಲ್ಲಿರುವುದು. ನಿನಗೆ ಮಾಡುವುದಕ್ಕೆ ಕೆಲಸವಿದೆ. ನೀನು ನನ್ನ ಕೆಲಸವನ್ನು ಮಾಡಿದ ಮೇಲೆ, ಆ ನಿಧಿಯನ್ನು ನಿನಗೆ ತಂದುಕೊಡುವೆ. ಆಗ ಎಲ್ಲವನ್ನೂ ಅರಿಯುವೆ. ಇನ್ನು ಕೆಲವು ಕಾಲದವರೆಗೆ ದೇಹದ ವಿಷಯದಲ್ಲಿ ಜೋಪಾನವಾಗಿರು. ಆಹಾರವನ್ನು ತೆಗೆದುಕೊಳ್ಳುವಾಗ ಅತ್ಯಂತ ಪರಿಶುದ್ಧವಾದುದನ್ನೇ ತೆಗೆದುಕೊಳ್ಳಬೇಕು, ಬಹಳ ಒಳ್ಳೆಯವರೊಡನೆ ಮಾತ್ರ ಬೆರೆಯಬೇಕು.”

ಶ‍್ರೀರಾಮಕೃಷ್ಣರು ನರೇಂದ್ರ ಹೋದ ತರುವಾಯ ಹತ್ತಿರವಿದ್ದ ಇತರ ಶಿಷ್ಯರಿಗೆ ಹೀಗೆ ಹೇಳಿದರು: “ನರೇಂದ್ರ ತನ್ನ ಸ್ವೇಚ್ಛೆಯಿಂದಲೇ ಮರಣವನ್ನು ಹೊಂದುತ್ತಾನೆ. ತಾನು ಯಾರು ಎಂಬುದನ್ನು ಅರಿತೊಡನೆಯೇ ಈ ದೇಹದಲ್ಲಿ ಒಂದು ಅರೆಗಳಿಗೆಯೂ ಜಾಸ್ತಿ ಇರುವುದಿಲ್ಲ. ನರೇಂದ್ರ ತನ್ನ ಬುದ್ಧಿ ಮತ್ತು ಆಧ್ಯಾತ್ಮಿಕ ಶಕ್ತಿಯಿಂದ ಪ್ರಪಂಚವನ್ನೇ ಅಲ್ಲಾಡಿಸಿಬಿಡುವ ಸಮಯ ಬರುವುದು. ಅದ್ವೈತದ ಅನುಭೂತಿಯನ್ನು ಅವನಿಂದ ಬಚ್ಚಿಡುವಂತೆ ನಾನು ತಾಯಿಗೆ ಪ್ರಾರ್ಥಿಸಿರುವೆನು. ಅವನು ಬಹಳ ಕೆಲಸವನ್ನು ಮಾಡಬೇಕಾಗಿದೆ. ಅವನ ಮೇಲೆ ಇರುವ ಉಪಾಧಿ ಬಹಳ ತೆಳ್ಳಗೆ ಇದೆ. ಅದು ಯಾವಾಗ ಬೇಕಾದರು ಹರಿದುಹೋಗಬಹುದು.”

ನರೇಂದ್ರನಿಗೆ ನಿರ್ವಿಕಲ್ಪ ಸಮಾಧಿಯನ್ನು ಅನುಭವಿಸಬೇಕೆಂಬ ಬಯಕೆ ಉತ್ಕಟವಾಗಿದ್ದುದರಿಂದ ಶ‍್ರೀರಾಮಕೃಷ್ಣರು ಅದನ್ನು ಅವನಿಗೆ ಅನುಗ್ರಹಿಸಿದರು. ಆದರೆ ಅವನು ಅಲ್ಲಿಯೇ ಇರಕೂಡದು. ಇದ್ದರೆ ಈ ಪ್ರಪಂಚದಲ್ಲಿ ಅವನು ಮಾಡಬೇಕಾಗಿರುವ ಕೆಲಸ ನಡೆಯುವ ಹಾಗಿಲ್ಲ. ಶ‍್ರೀರಾಮಕೃಷ್ಣರಂತೆ ಇತರರು ನಿರ್ವಿಕಲ್ಪ ಸಮಾಧಿಗೆ ಹೋಗಿಬರುವುದಕ್ಕೆ ಆಗುವುದಿಲ್ಲ. ಇತರರು ಹೋದರೆ ಬರುವುದಕ್ಕೆ ಆಗುವುದಿಲ್ಲ. ಆದಕಾರಣವೇ ಶ‍್ರೀರಾಮಕೃಷ್ಣರು ನರೇಂದ್ರನಿಗೆ ಪುನಃ ಅದು ಬರದಂತೆ ಆ ಅನುಭವದ ಬಾಗಿಲನ್ನು ಹಾಕಿ ಬೀಗದ ಕೈಗಳನ್ನು ತಾವು ಇಟ್ಟುಕೊಂಡಿದ್ದರು.

ಒಂದು ದಿನ ಗೋಪಾಲದಾದ ಎಂಬ ಶಿಷ್ಯನು ಕಾವಿಯ ಬಟ್ಟೆ ಮತ್ತು ರುದ್ರಾಕ್ಷಿ ಮಾಲೆಗಳನ್ನು ತಂದು ಶ‍್ರೀರಾಮಕೃಷ್ಣರ ಸಮಕ್ಷಮದಲ್ಲಿಟ್ಟು ಅವನ್ನು ಸಾಧುಗಳಿಗೆ ಹಂಚಿ ಎಂದು ಕೇಳಿಕೊಂಡನು. ಅದಕ್ಕೆ ಶ‍್ರೀರಾಮಕೃಷ್ಣರು “ಇಲ್ಲಿರುವ ಹುಡುಗರು ತ್ಯಾಗ ಮತ್ತು ವೈರಾಗ್ಯದಿಂದ ತುಂಬಿ ತುಳುಕಾಡುತ್ತಿರುವರು. ಬೇರೆಲ್ಲೂ ಇವರಿಗಿಂತ ಶ್ರೇಷ್ಠರಾದ ಸಾಧುಗಳು ನಿನಗೆ ದೊರಕಲಾರರು, ಇವರಿಗೆ ಬಟ್ಟೆ ಮತ್ತು ಜಪಮಾಲೆಗಳನ್ನು ಹಂಚು” ಎಂದರು. ಒಂದು ದಿನ ಸಾಯಂಕಾಲ ಶ‍್ರೀರಾಮಕೃಷ್ಣರೇ ತಮ್ಮ ಶಿಷ್ಯರನ್ನು ಕರೆದು ಅವರ ಕೈಯಲ್ಲಿ ಆ ಕಾವಿಯ ಬಟ್ಟೆ ಮತ್ತು ರುದ್ರಾಕ್ಷಿಮಾಲೆಗಳನ್ನು ಕೊಟ್ಟರು. ಅನಂತರ ಅವರು, ಯಾವ ಜಾತಿಮತಗಳ ಭೇದವೂ ಇಲ್ಲದೆ ಎಲ್ಲರಿಂದಲೂ ಭಿಕ್ಷೆಯನ್ನು ತೆಗೆದುಕೊಳ್ಳಬಹುದು ಎಂದರು. ಆಗ ಸಂನ್ಯಾಸವನ್ನು ತೆಗೆದುಕೊಂಡವರೆ ಮುಂದೆ ಶ‍್ರೀರಾಮಕೃಷ್ಣ ಸಂಸ್ಥೆಯ ಮೂಲಪುರುಷರಾದರು. ಅವರುಗಳೇ ನರೇಂದ್ರ, ರಾಖಾಲ, ಬಾಬುರಾಮ, ಯೋಗಿನ್, ನಿರಂಜನ, ತಾರಕ, ಶರತ್, ಶಶಿ, ಲಟು, ಕಾಳಿ ಮತ್ತು ಗೋಪಾಲದಾದ ಎಂಬುವರು.

ಶ‍್ರೀರಾಮಕೃಷ್ಣರು ನರೇಂದ್ರನನ್ನು ಜಗನ್ಮಾತೆಗೆ ಅರ್ಪಿಸಿ ಎರಡೂವರೆ ವರ್ಷಗಳು ಆದವು. ಅವನನ್ನು ಜಗನ್ಮಾತೆಗೆ ಅರ್ಪಿಸಿದ ದಿನದಿಂದಲೂ ಶ‍್ರೀರಾಮಕೃಷ್ಣರ ಆರೋಗ್ಯ ಚೆನ್ನಾಗಿರಲಿಲ್ಲ. ಅವರು ಯಾವುದಾದರೂ ಒಂದು ಅನಾರೋಗ್ಯದಿಂದ ನರಳುತ್ತಲೇ ಇದ್ದರು. ನರೇಂದ್ರನು ಶ‍್ರೀರಾಮಕೃಷ್ಣರ ನೇತೃತ್ವದಲ್ಲಿ ಆಧ್ಯಾತ್ಮಿಕ ಜೀವನದಲ್ಲಿ ಮುಂದುವರಿದನು. ಅವನ ಧ್ಯಾನ ಎಷ್ಟು ಗಾಢವಾಗಿರುತ್ತಿತ್ತು ಎಂದರೆ, ಅವನ ಮೈಮೇಲೆ ಸೊಳ್ಳೆಗಳು ಸ್ವಲ್ಪವೂ ಜಾಗ ಬಿಡದೆ ಕುಳಿತು ರಕ್ತ ಹೀರುವಾಗಲೂ ಅವನಿಗೆ ಬಾಹ್ಯಪ್ರಜ್ಞೆ ಬರುತ್ತಿರಲಿಲ್ಲ ಎಂದು ಅವನು ಧ್ಯಾನ ಮಾಡುವುದನ್ನು ನೋಡಿದವರು ಹೇಳುವರು.

ಒಂದು ದಿನ ಶ‍್ರೀರಾಮಕೃಷ್ಣರು ನರೇಂದ್ರನನ್ನು ತಮ್ಮ ಬಳಿಗೆ ಕರೆದು ಕೂಡಿಸಿಕೊಂಡರು. ಆಗ ದಾರುಣ ಯಾತನೆಯನ್ನು ಅನುಭವಿಸುತ್ತಿದ್ದರು. ಆ ಸಮಯದಲ್ಲಿ ಶ‍್ರೀರಾಮಕೃಷ್ಣರು, “ನರೇಂದ್ರ, ಇತರರಿಗೆ ಬೋಧಿಸುವನು” ಎಂದು ಬರೆದರು. ನರೇಂದ್ರ “ಇಲ್ಲ, ಅದನ್ನು ನಾನು ಮಾಡುವುದಿಲ್ಲ” ಎಂದು ಹೇಳಿದನು. ಅದಕ್ಕೆ ಶ‍್ರೀರಾಮಕೃಷ್ಣರು “ನೀನು ಮಾಡಲೇಬೇಕಾಗುವುದು” ಎಂದರು. ಕೆಲವು ದಿನಗಳ ಹಿಂದೆ ಶ‍್ರೀರಾಮಕೃಷ್ಣರು ತಮ್ಮ ತಪಸ್ಸು ನರೇಂದ್ರನ ಮೂಲಕ ಪ್ರಪಂಚದಲ್ಲಿ ಕೆಲಸ ಮಾಡುವುದು ಎಂದರು.\break ಶ‍್ರೀರಾಮಕೃಷ್ಣರು ಪ್ರತಿದಿನ ಸಾಯಂಕಾಲ ನರೇಂದ್ರನನ್ನು ತಮ್ಮ ಕೋಣೆಗೆ ಕರೆದು ಆಧ್ಯಾತ್ಮಿಕ ಜೀವನದ ವಿಷಯಗಳನ್ನೆಲ್ಲ ಹೇಳುತ್ತಿದ್ದರು. ಗುರುಭಾಯಿಗಳನ್ನು ಒಂದು ಕಡೆ ಸೇರಿಸಿ, ಅವರು ಆಧ್ಯಾತ್ಮಿಕ ಜೀವನದಲ್ಲಿ ಮುಂದುವರಿಯುವಂತೆ ನೋಡಿಕೊ ಎಂದು ಅವರ ಜವಾಬ್ದಾರಿಯನ್ನೆಲ್ಲ ನರೇಂದ್ರನಿಗೆ ಕೊಟ್ಟರು.

ಶ‍್ರೀರಾಮಕೃಷ್ಣರ ಮಹಾಸಮಾಧಿಗೆ ಇನ್ನು ಮೂರು ನಾಲ್ಕು ದಿನಗಳು ಮಾತ್ರ ಉಳಿದಿವೆ. ಅವರು ನರೇಂದ್ರನನ್ನು ತಮ್ಮ ಬಳಿಗೆ ಸಾಯಂಕಾಲ ಕರೆದರು. ಅವನನ್ನೇ ದಿಟ್ಟಿಸಿ ನೋಡುತ್ತಿದ್ದ ಹಾಗೆಯೇ ಶ‍್ರೀರಾಮಕೃಷ್ಣರು ಗಾಢ ಧ್ಯಾನಾಸಕ್ತರಾದರು. ಆ ಸಮಯದಲ್ಲಿ ಅವರ ದೇಹದಿಂದ ಯಾವುದೋ ಶಕ್ತಿಯ ಪ್ರವಾಹವೊಂದು ನರೇಂದ್ರನ ದೇಹದ ಒಳಗೆ ಪ್ರವೇಶಿಸಿದಂತೆ ಆಯಿತು. ನರೇಂದ್ರನಿಗೆ ಬಾಹ್ಯಪ್ರಜ್ಞೆ ತಪ್ಪಿತು. ಸ್ವಲ್ಪ ಹೊತ್ತಾದ ಮೇಲೆ ನರೇಂದ್ರ ಕಣ್ಣನ್ನು ತೆರೆದಾಗ ಶ‍್ರೀರಾಮಕೃಷ್ಣರು ಕಂಬನಿದುಂಬಿ ಅಳುತ್ತಿರುವುದನ್ನು ನೋಡಿದನು. ನರೇಂದ್ರ “ಏತಕ್ಕೆ” ಎಂದು ಕೇಳಿದಾಗ: “ನರೇಂದ್ರ, ಇಂದು ನನ್ನ ಸರ್ವಸ್ವವನ್ನೂ ನಿನಗೆ ಧಾರೆ ಎರೆದು ನಾನು ಭಿಕ್ಷುಕನಾಗಿರುವೆನು. ನಾನು ನಿನಗೆ ಕೊಟ್ಟಿರುವ ಶಕ್ತಿಯ ಮೂಲಕ ನೀನು ಅದ್ಭುತ ಕಾರ‍್ಯಗಳನ್ನು ಮಾಡುವೆ. ಅನಂತರವೇ ನೀನು ಎಲ್ಲಿಂದ ಬಂದೆಯೋ ಅಲ್ಲಿಗೆ ಹೋಗುವೆ” ಎಂದರು. ಶ‍್ರೀರಾಮಕೃಷ್ಣರು ಹಲವು ವರ್ಷಗಳು ಉಗ್ರ ತಪಸ್ಸನ್ನು ಮಾಡಿ ಯಾವ ಯಾವ ಸಾಕ್ಷಾತ್ಕಾರಗಳನ್ನು ಪಡೆದಿದ್ದರೋ ಅವನ್ನೆಲ್ಲ ನರೇಂದ್ರನಿಗೆ ಕೊಟ್ಟುಬಿಟ್ಟರು. ತಮ್ಮ ತಪೋಶಕ್ತಿಯ ಭಾರವನ್ನು ನರೇಂದ್ರನಿಗೆ ಕೊಟ್ಟುಬಿಟ್ಟು ಅವರ ಮನಸ್ಸು ಹಗುರವಾಯಿತು. ಲೋಕಕಲ್ಯಾಣಕ್ಕಾಗಿ ಅವನು ತಮ್ಮ ನಂತರ ಅದನ್ನು ಉಪಯೋಗಿಸುತ್ತಾನೆ ಎಂಬುದರಲ್ಲಿ ಅವರಿಗೆ ಯಾವ ಸಂದೇಹವೂ ಇರಲಿಲ್ಲ. ಶ‍್ರೀರಾಮಕೃಷ್ಣರ ಕಾಂತಿ ಇವರ ತರುವಾಯ ನರೇಂದ್ರನ ಹೃದಯದಲ್ಲಿ ಬೆಳೆಯತೊಡಗಿತು.

ಶ‍್ರೀರಾಮಕೃಷ್ಣರ ಮಹಾಸಮಾಧಿಗೆ ಒಂದೆರಡು ದಿನ ಮುಂಚೆ ನರೇಂದ್ರ ಅವರ ಹಾಸಿಗೆಯ ಬಳಿ ನಿಂತಿದ್ದನು. ಶ‍್ರೀರಾಮಕೃಷ್ಣರಿಗೆ ದಾರುಣಯಾತನೆ ಆಗ. ಸ್ವಲ್ಪವೂ ತ್ರಾಣವಿರಲಿಲ್ಲ, ಒಂದು ರಾಶಿ ದಿಂಬಿನ ಮಧ್ಯೆ ಅವರ ದೇಹ ಬಿದ್ದಿತ್ತು. ಆಗ ನರೇಂದ್ರನ ಮನಸ್ಸಿನಲ್ಲಿ ಒಂದು ಯೋಚನೆ ಮೂಡಿತು: “ಅವರು ಅನೇಕ ವೇಳೆ ತಾವು ಈಶ್ವರನ ಅವತಾರ ಎಂದು ಹೇಳಿರುವರು. ಈಗ ಮೃತ್ಯು ಸಮ್ಮುಖದಲ್ಲಿ, ಯಮಯಾತನೆಯನ್ನು ಅನುಭವಿಸುತ್ತಿರುವಾಗಲೂ, ಅವರು ತಾವು ಅವತಾರ ಎಂದು ಹೇಳಿದರೆ ನಂಬುತ್ತೇನೆ.” ಈ ಆಲೋಚನೆ ನರೇಂದ್ರನ ಮನಸ್ಸಿನಲ್ಲಿ ಮೂಡಿದೊಡನೆಯೇ ಶ‍್ರೀರಾಮಕೃಷ್ಣರು ನರೇಂದ್ರನ ಕಡೆ ನೋಡುತ್ತಾ ತಮ್ಮ ಶಕ್ತಿಯನ್ನೆಲ್ಲ ಸಂಗ್ರಹಿಸಿಕೊಂಡು ಸ್ಪಷ್ಟವಾಗಿ “ನರೇಂದ್ರ! ನಿನಗೆ ಇನ್ನೂ ಸಂದೇಹ ಹೋಗಲಿಲ್ಲವೆ? ಯಾರು ಹಿಂದೆ ರಾಮನಾಗಿದ್ದನೊ ಮತ್ತು ಕೃಷ್ಣನಾಗಿದ್ದನೊ ಅವನೇ ಈಗ ರಾಮಕೃಷ್ಣನಾಗಿ ಬಂದಿರುವನು. ಆದರೆ ನಿನ್ನ ವೇದಾಂತದ ದೃಷ್ಟಿಯಿಂದಲ್ಲ” ಎಂದರು. ನರೇಂದ್ರನಿಗೆ ಶ‍್ರೀರಾಮಕೃಷ್ಣರಲ್ಲಿ ಇದುವರೆಗೆ ಅಂತಹ ಆಧ್ಯಾತ್ಮಿಕ ಆವಿರ್ಭಾವಗಳನ್ನು ನೋಡಿದ್ದರೂ ಇನ್ನೂ ಅವರನ್ನು ಅನುಮಾನಿಸಿದೆನಲ್ಲ ಎಂದು ನಾಚಿಕೆಯಾಯಿತು.

ಶ‍್ರೀರಾಮಕೃಷ್ಣರ ಕೊನೆಯ ಎರಡು ದಿನಗಳು ಶಿಷ್ಯರಿಗೆ ತುಂಬಾ ದಾರುಣವಾದ ದಿನಗಳಾದವು. ಬೇಗನೆ ಅವರು ದೇಹವನ್ನು ವಿಸರ್ಜಿಸುವರು ಎಂದು ಅವರೆಲ್ಲ ಅರಿತಿದ್ದರು. ಅವರ ನಂತರ ಪ್ರಪಂಚದಲ್ಲಿ ತಾವು ಅನಾಥರಾಗುತ್ತೇವೆ ಎಂದು ಭಾವಿಸಿದರು. ಕೊನೆಯ ದಿನವಂತೂ ಶ‍್ರೀರಾಮಕೃಷ್ಣರ ಯಾತನೆ ಹೃದಯ ವಿದ್ರಾವಕವಾಗುವಂತೆ ಇತ್ತು. ವೈದ್ಯರನ್ನು ಕರೆದರು. ಆದರೆ ಅವರಿಗೆ ಏನನ್ನು ಮಾಡಲೂ ಆಗಲಿಲ್ಲ. ಸಂಜೆಗೆ ಸ್ವಲ್ಪ ಹೊತ್ತಿಗೆ ಮುಂಚೆ ಶ‍್ರೀರಾಮಕೃಷ್ಣರು ತಮಗೆ ಉಸಿರಾಡಲು ತುಂಬಾ ಕಷ್ಟವಾಗಿದೆ ಎಂದರು. ಸ್ವಲ್ಪ ಹೊತ್ತಾದ ಮೇಲೆ ಸಮಾಧಿಗೆ ಹೋದರು. ಶಿಷ್ಯರು ಅಳಲು ಮೊದಲು ಮಾಡಿದರು. ಅರ್ಧ ರಾತ್ರಿ ಆದಮೇಲೆ ಶ‍್ರೀರಾಮಕೃಷ್ಣರಿಗೆ ಪ್ರಜ್ಞೆ ಬಂತು. ಆಗ ತಮಗೆ ಹಸಿವಾಗುತ್ತಿದೆ ಎಂದರು. ಸ್ವಲ್ಪ ಪಾಯಸವನ್ನು ಕುಡಿದರು. ನಾಲ್ಕೈದು ದಿಂಬುಗಳನ್ನು ಒರಗಿಕೊಂಡು ಶಶಿಯನ್ನು ಹಿಡಿದುಕೊಂಡು ಕೊನೆಗಳಿಗೆಯವರೆಗೂ ನರೇಂದ್ರನೊಡನೆ ಮಾತನಾಡಿ ತಮ್ಮ ಕೊನೆಯ ಸಲಹೆಯನ್ನು ಅವನಿಗೆ ಕೊಟ್ಟರು. ಅನಂತರ “ಜೈಕಾಳಿ, ಜೈಕಾಳಿ, ಜೈಕಾಳಿ” ಎಂದು ಹೇಳಿ ಹಾಸಿಗೆಯ ಮೇಲೆ ಮೆಲ್ಲಗೆ ಮಲಗಿದರು. ರಾತ್ರಿ ಒಂದು ಗಂಟೆಯಾಗಿ ಎರಡು ನಿಮಿಷವಾಗಿತ್ತು. ಅಂದು ಹದಿನಾರನೇ ಆಗಸ್ಟ್ ೧೮೮೬. ಶ‍್ರೀರಾಮಕೃಷ್ಣರ ದೇಹದಲ್ಲಿ ಯಾವುದೋ ಶಕ್ತಿ ಸಂಚರಿಸಿತು. ಅವರ ದೇಹದ ರೊಮಗಳು ನಿಮಿರಿ ನಿಂತವು. ಕಣ್ಣು ಭ್ರೂಮಧ್ಯದಲ್ಲಿ ನೆಲೆಸಿತು. ಮೊಗದ ಮೇಲೆ ಒಂದು ದೈವೀ ಮಂದಹಾಸ ಮೂಡಿತು. ಶ‍್ರೀರಾಮಕೃಷ್ಣರು ಮಹಾಸಮಾಧಿಯನ್ನು ಪಡೆದರು. ಹಿಂತಿರುಗಿ ಈ ದೇಶ ಕಾಲಕ್ಕೆ ಬರದ ಸಮಾಧಿ ಅದು. ಶ‍್ರೀರಾಮಕೃಷ್ಣರ ಲೀಲಾ ನಾಟಕದ ಕೊನೆಯ ಅಂಕ ಪೂರೈಸಿತು. ಪರಬ್ರಹ್ಮನ ಕಡಲಿನಿಂದ ಬಂದು, ಈ ಪ್ರಪಂಚದಲ್ಲಿ ತಮ್ಮ ಕಾರ್ಯವನ್ನು ಮಾಡಿ, ತಮ್ಮ ನಂತರ ಕೆಲಸ ಮಾಡುವುದಕ್ಕೆ ಮತ್ತೊಬ್ಬನನ್ನು ಅಣಿಮಾಡಿ ಅದೇ ಪರಬ್ರಹ್ಮ ಮಹಾಸಾಗರದಲ್ಲಿ ಶ‍್ರೀರಾಮಕೃಷ್ಣರೆಂಬ ಅಲೆ ಐಕ್ಯವಾಯಿತು.

ಶ‍್ರೀರಾಮಕೃಷ್ಣರು ಯಾವ ಮಂಚದ ಮೇಲೆ ಮಲಗಿದ್ದರೋ ಆ ಮಂಚವನ್ನು ಬೆಳಗ್ಗೆ ಅವರ ದೇಹದ ಸಮೇತ ತೆಗೆದುಕೊಂಡು ಬಂದರು. ಅದನ್ನು ಗಂಗಾಜಲದಿಂದ ತೊಳೆದು, ಗೈರಿಕವಸನಗಳಿಂದ ಅಲಂಕರಿಸಿದರು. ಗಂಧ ಹೂವುಗಳಿಂದ ಪೂಜಿಸಿದರು. ಕೆಲವು ಕಾಲ ಕಾಶೀಪುರದ ಉದ್ಯಾನ ಮನೆಯಲ್ಲಿಯೇ ಕಳೇಬರವಿತ್ತು. ಇಲ್ಲೆ ಶ‍್ರೀರಾಮಕೃಷ್ಣರು ಹಲವು ಜೀವಿಗಳಿಗೆ ತಮ್ಮ ಜೀವನದ ಕೊನೆಯ ಬೋಧನೆಯನ್ನು ಕೊಟ್ಟದ್ದು. ಅವರ ವ್ಯಕ್ತಿತ್ವದಿಂದ ಓತಪ್ರೋತವಾಗಿತ್ತು ಆ ವಾತಾವರಣ. ಅನಂತರ ಸ್ವಲ್ಪ ದೂರದಲ್ಲಿರುವ ಗಂಗಾ ತೀರದ ಸ್ಮಶಾನಕ್ಕೆ ಕಳೇಬರವನ್ನು ಭಜನಾದಿಗಳನ್ನು ಮಾಡಿಕೊಂಡು ತೆಗೆದುಕೊಂಡು ಹೋದರು; ಭಕ್ತರು ಮತ್ತು ಶಿಷ್ಯರು ಎಲ್ಲರ ಕಣ್ಣುಗಳೂ ಕಂಬನಿಯಿಂದ ತೊಯ್ದುಹೋಗಿದ್ದವು. ಕೊನೆಯ ಪ್ರಯಾಣಕ್ಕೆ ಎಲ್ಲರೂ ಕ್ರಮೇಣ ತಮ್ಮ ಹೆಗಲನ್ನು ಕೊಟ್ಟರು. ಅದನ್ನು ನೋಡಿದವರೆಲ್ಲ ಬಾಗಿದರು. ಕಾಶೀಪುರದ ಸ್ಮಶಾನಘಟ್ಟದಲ್ಲಿ ದೇಹವನ್ನು ಚಿತೆಯ ಮೇಲಿಟ್ಟರು. ಅಗ್ನಿದೇವನ ಸಂಪರ್ಕವನ್ನು ಅದಕ್ಕೆ ಮಾಡಿದರು. ಧೂಪದಿಂದ ಆಜ್ಯಗಳನ್ನು ಅದಕ್ಕೆ ನಿವೇದಿಸಿದರು. ಕೆಲವು ಗಂಟೆಗಳಲ್ಲಿ ಎಲ್ಲ ಬೊಗಸೆ ಬೂದಿಯಾಯಿತು. ಅದನ್ನು ಒಂದು ಜಾಡಿಯಲ್ಲಿಟ್ಟು ಕಾಶೀಪುರದ ತೋಟದ ಮನೆಗೆ ತೆಗೆದುಕೊಂಡು ಹೋದರು. ಈಗ ಆ ಮನೆ ಶೂನ್ಯಾಗಾರದಂತೆ ಇತ್ತು, ದೇವರಿಲ್ಲದ ದೇಗುಲದಂತೆ ಇತ್ತು. ಶ‍್ರೀರಾಮಕೃಷ್ಣರು ಇನ್ನು ಮೇಲೆ ತಮ್ಮ ಶಿಷ್ಯರ ನೆನಹಿನ ಗುಡಿಯಲ್ಲಿ ವಾಸಿಸತೊಡಗಿದರು.

