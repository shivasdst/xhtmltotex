
\chapter*{೪೬. ಉದ್ಬೋಧನ }

 ಸ್ವಾಮೀಜಿ ಪರಮಹಂಸರ ಭಾವವನ್ನು ಜನಸಾಧಾರಣರಲ್ಲಿ ಪ್ರಚಾರಮಾಡುವುದಕ್ಕಾಗಿ ಬಂಗಾಳಿ ಭಾಷೆಯಲ್ಲಿ ಒಂದು ಪತ್ರಿಕೆಯನ್ನು ಹೊರಡಿಸಬೇಕೆಂದು ತಮ್ಮ ಗುರುಭಾಯಿಗಳೊಡನೆ ಪ್ರಸ್ತಾಪ ಮಾಡಿದರು. ಸ್ವಾಮೀಜಿ ತ್ರಿಗುಣಾತೀತರಿಗೆ ಆ ಪತ್ರಿಕೆಯ ಜವಾಬ್ದಾರಿಯನ್ನು ಕೊಟ್ಟರು. ಸ್ವಾಮಿಗಳ ಹತ್ತಿರ ಒಂದು ಸಾವಿರ ರೂಪಾಯಿ ಇತ್ತು. ಪರಮಹಂಸರ ಗೃಹಸ್ಥ ಭಕ್ತರೊಬ್ಬರು ಒಂದು ಸಾವಿರ ರೂಪಾಯಿಗಳನ್ನು ಸಾಲವಾಗಿ ಕೊಟ್ಟರು. ಈ ಹಣದಿಂದ ಕಾರ‍್ಯಾರಂಭವಾಯಿತು. ಒಂದು ಮುದ್ರಣ ಯಂತ್ರವನ್ನು ಕೊಂಡು ಶ್ಯಾಮಬಜಾರಿನ ರಾಮಚಂದ್ರ ಮಿತ್ರರ ಗಲ್ಲಿಯಲ್ಲಿರುವ ಶ‍್ರೀಯುತ ಗಿರೀಂದ್ರನಾಥ ಬಸಾಕರ ಮನೆಯಲ್ಲಿ ಮುದ್ರಾಲಯವನ್ನು ತೆರೆದದ್ದಾಯಿತು. ತ್ರಿಗುಣಾತೀತರು ಕಾರ‍್ಯಭಾರವನ್ನು ಸ್ವೀಕರಿಸಿ ಕ್ರಿ.ಶ. ೧೮೯೮ನೇ ಮಾಘ ಶುದ್ಧ ಪಾಡ್ಯಮಿ ದಿನ ಈ ಪತ್ರಿಕೆಯನ್ನು ಪ್ರಕಾಶ ಪಡಿಸಿದರು. ಸ್ವಾಮೀಜಿ ಈ ಪತ್ರಿಕೆಗೆ ಉದ್ಬೋಧನ ಎಂಬ ಹೆಸರನ್ನು ಕೊಟ್ಟು ಅದರ ಅಭಿವೃದ್ಧಿಗಾಗಿ ತ್ರಿಗುಣಾತೀತರಿಗೆ ಅಶೀರ್ವದಿಸಿದರು. ಸರ್ವಶ್ರಮ ಸಹಿಷ್ಣುಗಳಾದ ತ್ರಿಗುಣಾತೀತರು ಸ್ವಾಮೀಜಿ ಅಪ್ಪಣೆ ಮೇರೆಗೆ ಅದರ ಮುದ್ರಣ ಮತ್ತು ಪ್ರಚಾರಕ್ಕಾಗಿ ಎಷ್ಟು ಪರಿಶ್ರಮಪಟ್ಟರೆಂಬುದನ್ನು ತಿಳಿಸುವುದಕ್ಕೆ ಮತ್ತೊಂದು ದೃಷ್ಟಾಂತವು ಹುಡುಕಿದರೂ ಸಿಕ್ಕುವುದಿಲ್ಲ. ತ್ರಿಗುಣಾತೀತರು ಈ ಪತ್ರಿಕೆಯ ಪ್ರಚಾರಕ್ಕಾಗಿ ಪ್ರಾಣವನ್ನಾದರೂ ಕೊಡುವುದಕ್ಕೆ ಹಿಂದೆಗೆಯದೆ ಕೆಲಸ ಮಾಡುತ್ತಿದ್ದರು. ಅಲ್ಲದೆ ಆಗ ಕೆಲಸಮಾಡುವವರನ್ನು ದುಡ್ಡು ಕೊಟ್ಟು ಇಟ್ಟುಕೊಳ್ಳುವುದಕ್ಕೆ ಅನುಕೂಲವಿರಲಿಲ್ಲ. ಪತ್ರಿಕೆಗಾಗಿ ಗೊತ್ತುಮಾಡಿ ಇಟ್ಟಿದ್ದ ಹಣದಲ್ಲಿ ಪತ್ರಿಕೆಗೆ ಹೊರತು ಮತ್ತಾವುದಕ್ಕೂ ಒಂದೂ ಕಾಸನ್ನು ಉಪಯೋಗಿಸಕೂಡದೆಂದು ಸ್ವಾಮೀಜಿ ಕಟ್ಟಪ್ಪಣೆ ಮಾಡಿದ್ದರು. ಆದುದರಿಂದ ತ್ರಿಗುಣಾತೀತರು ಭಕ್ತಾದಿಗಳ ಮನೆಯಲ್ಲಿ ಭಿಕ್ಷೆಯನ್ನು ಮಾಡಿ ತಮ್ಮ ಹೊಟ್ಟೆಯ ಪಾಡನ್ನು ಹೇಗೆ ಹೇಗೋ ಕಳೆದುಕೊಳ್ಳುತ್ತ ಆ ಅಪ್ಪಣೆಯನ್ನು ಅಕ್ಷರಶಃ ಪಾಲಿಸಿದರು. 

 ಪತ್ರಿಕೆಯ ಪ್ರಸ್ತಾವನೆಯನ್ನು ಸ್ವಾಮಿಗಳೇ ಸ್ವಂತವಾಗಿ ಬರೆದುಕೊಟ್ಟರು. ಪರಮಹಂಸರ ಸಂನ್ಯಾಸಿ ಮತ್ತು ಗೃಹಸ್ಥ ಭಕ್ತರೇ ಈ ಪತ್ರಿಕೆಯಲ್ಲಿ ಲೇಖನಗಳನ್ನು ಬರೆಯಬೇಕೆಂದೂ ಅಶ್ಲೀಲ ವ್ಯಂಜಕವಾದ ಯಾವ ಪ್ರಕಟಣೆ ಮುಂತಾದವುಗಳನ್ನೂ ಈ ಪತ್ರಿಕೆಯಲ್ಲಿ ಹಾಕಕೂಡದೆಂದೂ ಸ್ವಾಮೀಜಿ ಹೇಳಿದ್ದರು. ಸ್ವಾಮೀಜಿ ಸಂಘರೂಪವಾಗಿ ಪರಿಣತವಾದ ರಾಮಕೃಷ್ಣ ಮಿಶನ್ ಸಭ್ಯರನ್ನು ಈ ಪತ್ರಿಕೆಯಲ್ಲಿ ಲೇಖನವನ್ನು ಬರೆಯುವಂತೆಯೂ, ಪತ್ರಿಕೆಯ ಸಹಾಯದಿಂದ ಪರಮಹಂಸರ ಧರ‍್ಮ ಸಂಬಂಧವಾದ ಅಭಿಪ್ರಾಯಗಳನ್ನು ಜನಸಾಧಾರಣರಲ್ಲಿ ಹರಡುವಂತೆಯೂ ಹೇಳಿದರು. ಪತ್ರಿಕೆಯ ಮೊದಲ ಸಂಚಿಕೆ ಪ್ರಕಾಶವಾಗಲು ಶಿಷ್ಯನು ಒಂದು ದಿನ ಮಠಕ್ಕೆ ಹೋದನು. ಸ್ವಾಮಿಜಿಗೆ ನಮಸ್ಕಾರ ಮಾಡಿ ಕುಳಿತುಕೊಳ್ಳಲು ಸ್ವಾಮೀಜಿ ಅವನನ್ನು ಉದ್ಬೋಧನವನ್ನು ನೋಡಿದೆಯಾ? ಎಂದು ವಿಚಾರಿಸಿದರು. 

 ಶಿಷ್ಯ: “ನೋಡಿದೆ, ಚೆನ್ನಾಗಿದೆ!” 

 ಸ್ವಾಮೀಜಿ: “ಈ‌ಪತ್ರಿಕೆಯ ಭಾವ ಭಾಷೆ ಎಲ್ಲ ಹೊಸ ಮಾದರಿಯಲ್ಲಿ ಇರಬೇಕು.” 

 ಶಿಷ್ಯ: “ಹೇಗೆ?” 

 ಸ್ವಾಮೀಜಿ: “ಪರಮಹಂಸರ ಭಾವವನ್ನು ಎಲ್ಲರಿಗೂ ಕೊಡಲೇ ಬೇಕು. ಅಲ್ಲದೆ ಬಂಗಾಳಿ ಭಾಷೆಗೆ ನೂತನವಾದ ಓಜಸ್ಸನ್ನು ತಂದುಕೊಡಬೇಕು. ಹೆಚ್ಚಾಗಿ ಕ್ರಿಯಾಪದದ ಉಪಯೋಗವನ್ನು ಕಡಿಮೆ ಮಾಡಬೇಕು. ಲೇಖನಗಳನ್ನು ನಾನು ಮೊದಲು ನೋಡಿ ಆಮೇಲೆ ಉದ್ಬೋಧನಕ್ಕೆ ಅಚ್ಚಿಗೆ ಕೊಡುತ್ತೇನೆ.” 

 ಶಿಷ್ಯ: “ತ್ರಿಗುಣಾತೀತರು ಈ ಪತ್ರಿಕೆಗೆ ಪಡುತ್ತಿರುವ ಶ್ರಮ ಅನ್ಯರಿಗೆ ಸಾಧ್ಯವಿಲ್ಲ.” 

 ಸ್ವಾಮೀಜಿ: “ಪರಮಹಂಸರ ಈ ಸಂನ್ಯಾಸೀ ಸಂತಾನರೆಲ್ಲ ಸುಮ್ಮನೆ ಮರದ ಕೆಳಗೆ ಬೆಂಕಿಯನ್ನು ಹತ್ತಿಸಿ ಕುಳಿತುಕೊಳ್ಳುವುದಕ್ಕಾಗಿ ಹುಟ್ಟಿದ್ದಾರೆಂದು ನೀನು ತಿಳಿದುಕೊಂಡಹಾಗಿದೆ. ಇವರಲ್ಲಿ ಯಾರು, ಯಾವಾಗ, ಯಾವ ಕ್ಷೇತ್ರಕ್ಕೆ ಹೋದರೂ ಆಗ ಅವರ ಉದ್ಯಮವನ್ನು ನೋಡಿ ಜನರು ಬೆರಗಾಗುವರು. ಕಾರ‍್ಯ ಹೇಗೆ ಮಾಡಬೇಕೆಂಬುದನ್ನು ಇವರಿಂದ ಕಲಿತುಕೊ. ನೋಡು ನನ್ನ ಮಾತನ್ನು ನಡೆಸುವುದಕ್ಕಾಗಿ ಸಾಧನೆ ಭಜನೆಯನ್ನು ಧ್ಯಾನ ಧಾರಣವನ್ನು ಬಿಟ್ಟು ಕಾರ‍್ಯವನ್ನು ಕೈಕೊಂಡಿದ್ದಾನೆ. ಇದೇನು ಕಡಿಮೆ ತ್ಯಾಗದ ಮಾತೆ? ನನ್ನಲ್ಲಿ ಎಷ್ಟು ವಿಶ್ವಾಸದಿಂದ ಈ ಕರ್ಮಪ್ರವೃತ್ತಿ ಉಂಟಾಗಿದೆ ಬಲ್ಲೆಯ? ಕಾರ್ಯಸಿದ್ಧಿ ಮಾಡಿ ಆಮೇಲೆ ಬಿಡುತ್ತಾನೆ. ನಿನಗೆ ಇಷ್ಟೊಂದು ದಾರ್ಢ್ಯವಿದೆಯೆ?” 

 ಶಿಷ್ಯ: “ಮಹಾರಾಜ್, ಕಾವಿಯನ್ನುಟ್ಟ ಸಂನ್ಯಾಸಿಗಳು ಗೃಹಸ್ಥರ ಮನೆ ಮನೆ ಬಾಗಿಲಿಗೆ ಅಲೆಯುವುದು ಹೇಗೋ ಕಾಣುತ್ತದೆ.” 

 ಸ್ವಾಮೀಜಿ: “ಏಕೆ ಪತ್ರಿಕೆಯ ಪ್ರಚಾರ ಗೃಹಸ್ಥರ ಕಲ್ಯಾಣಕ್ಕಾಗಿ ಇದೆ. ದೇಶದಲ್ಲಿ ಹೊಸ ಭಾವಗಳನ್ನು ಪ್ರಚಾರಗೊಳಿಸುವುದರ ಮೂಲಕ ಜನಸಾಧಾರಣರ ಕಲ್ಯಾಣ ಸಾಧಿತವಾಗಬೇಕು. ಫಲಾಕಾಂಕ್ಷೆ ಇಲ್ಲದೆ ಮಾಡುವ ಈ ಕರ್ಮ ಸಾಧನೆ ಭಜನೆಗಳಿಗಿಂತ ಕಡಿಮೆ ಎಂದು ಯೋಚಿಸಿಕೊಂಡಿದ್ದೀಯಾ? ನಮ್ಮ ಉದ್ದೇಶ ಜೀವರ ಹಿತ ಸಾಧನೆ. ನಮಗೆ ಈ ಪತ್ರಿಕೆಯ ಆದಾಯದಿಂದ ದುಡ್ಡು ಕೂಡಿಹಾಕುವ ಯೋಜನೆ ಇಲ್ಲ. ನಾವು ಸರ್ವವನ್ನೂ ತ್ಯಾಗ ಮಾಡಿದ ಸಂನ್ಯಾಸಿಗಳು. ನಮಗೇನು ಹೆಂಡತಿ ಮಕ್ಕಳು ಇಲ್ಲ. ಅವರಿಗೋಸ್ಕರ ನಾವು ಏನನ್ನೂ ಕೂಡಿಡಬೇಕಾಗಿಲ್ಲ. ಕಾರ್ಯ ಸಾಧನೆ ಮತ್ತು ಸಂಪಾದನೆ. ಆದರೆ ಅದರ ಆದಾಯವೆಲ್ಲವೂ ಜೀವಸೇವೆಗಾಗಿ ಉಪಯೋಗಿಸಲ್ಪಡುತ್ತವೆ. ಅಲ್ಲಲ್ಲಿ ಸಂಘಗಳ ಏರ್ಪಾಡು, ಸೇವಾಶ್ರಮ ಸ್ಥಾಪನೆ ಮತ್ತು ಇನ್ನೂ ನಾನಾ ಹಿತಕರವಾದ ಕಾರ‍್ಯಗಳಲ್ಲಿ ಇದರಿಂದ ಬಂದ ಹಣ ಸದ್ವಿನಿಯೋಗವಾಗುತ್ತದೆ. ನಾವೇನು ಗೃಹಸ್ಥರ ಹಾಗೆ ಸ್ವಂತ ಸಂಪಾದನೆಯ ಉದ್ದೇಶವನ್ನಿಟ್ಟುಕೊಂಡು ಈ ಕೆಲಸ ಮಾಡುತ್ತಿಲ್ಲ. ಕೇವಲ ಪರಹಿತಕ್ಕಾಗಿಯೇ ನಮ್ಮ ಈ ಕಾರ್ಯವೆಲ್ಲ – ತಿಳಿದುಕೊ.” 

 ಶಿಷ್ಯ: “ಹಾಗಿದ್ದರೂ ಎಲ್ಲರೂ ಈ ಭಾವವನ್ನು ಗ್ರಹಿಸಲಾರರು.” 

 ಸ್ವಾಮೀಜಿ: “ಗ್ರಹಿಸಲಾರದೆ ಇದ್ದರೆ ಬಿಡಲಿ. ನಮಗೆ ಅದರಿಂದ ಬರುವುದೇನು ಹೋಗುವುದೇನು? ನಾನು ನಿಂದೆ ಸ್ತುತಿ ಇವನ್ನು ಲೆಕ್ಕಮಾಡಿಕೊಂಡು ಕೆಲಸಕ್ಕೆ ಪ್ರವೇಶ ಮಾಡುವುದಿಲ್ಲ.” 

 ಶಿಷ್ಯ:‌“ಆ ದಿವಸ ನೋಡಿದೆ, ಸ್ವಾಮಿ ತ್ರಿಗುಣಾತೀತರು ಪರಮಹಂಸರ ಚಿತ್ರಪಟವನ್ನು ಮುದ್ರಣಾಲಯದಲ್ಲಿ ಪೂಜಿಸಿದ ಮೇಲೆ ಕಾರ್ಯವನ್ನು ಪ್ರಾರಂಭಿಸಿದರು ಮತ್ತು ಕಾರ್ಯ ಸಫಲವಾಗುವುದಕ್ಕೋಸ್ಕರ ತಮ್ಮ ಕೃಪೆಯನ್ನು ಪ್ರಾರ್ಥಿಸಿದರು.” 

 ಸ್ವಾಮೀಜಿ: “ನಮ್ಮ ಕೇಂದ್ರ ಪರಮಹಂಸರೆ, ನಮ್ಮಲ್ಲಿ ಪ್ರತಿಯೊಬ್ಬರೂ ಆ ಕೇಂದ್ರ ಜ್ಯೋತಿಯ ಒಂದೊಂದು ಕಿರಣ. ಪರಮಹಂಸರ ಪೂಜೆಯನ್ನು ಮಾಡಿ ಕೆಲಸವನ್ನು ಆರಂಭಿಸಿದನೊ, ಒಳ್ಳೆಯದು ಮಾಡಿದ. ನನಗೆ ಮಾತ್ರ ಯಾರೂ ಪೂಜೆ ಸಂಗತಿಯನ್ನು ಕುರಿತು ಹೇಳಲಿಲ್ಲ.” 

 ಶಿಷ್ಯ: “ಮಹಾರಾಜ್, ಅವರು ತಮಗೆ ಹೆದರುವರು. ತ್ರಿಗುಣಾತೀತರು ‘ನೀನು ಮೊದಲು ಸ್ವಾಮಿಗಳ ಹತ್ತಿರ ಹೋಗಿ ತಿಳಿದುಕೊಂಡು ಬಾ, ಮೊದಲನೆ ಸಂಚಿಕೆ ವಿಚಾರವಾಗಿ ಅವರು ಏನು ಅಭಿಪ್ರಾಯ ಪಡುತ್ತಾರೆಂದು. ಆಮೇಲೆ ನಾನು ಅವರನ್ನು ದರ್ಶನ ಮಾಡುತ್ತೇನೆ’ ಎಂದು. ಅವರು ಈ ಬೆಳಿಗ್ಗೆ ಹೇಳಿದರು.” 

 ಸ್ವಾಮೀಜಿ: “ಹೋಗಿ ಹೇಳು, ಅವನ ಕೆಲಸದಿಂದ ನನಗೆ ಬಹಳ ತೃಪ್ತಿ ಆಗಿದೆ ಎಂದು. ಅವನಿಗೆ ನನ್ನ ಸ್ನೇಹಾಶೀರ್ವಾದಗಳನ್ನು ತಿಳಿಸು. ಅಲ್ಲದೆ ನಿಮ್ಮಲ್ಲಿ ಪ್ರತಿಯೊಬ್ಬರೂ ಸಾಧ್ಯವಾದಷ್ಟು ಅವನಿಗೆ ಸಹಾಯ ಮಾಡಿ. ಅದರಿಂದ ದೇವರ ಕೆಲಸವೆ ಆಗುತ್ತದೆ.” 

 ಈ ಮಾತುಗಳನ್ನು ಹೇಳಿ ಸ್ವಾಮೀಜಿ, ಬ್ರಹ್ಮಾನಂದ ಸ್ವಾಮಿಗಳನ್ನು ಹತ್ತಿರಕ್ಕೆ ಕರೆದು ಆವಶ್ಯಕತೆ ಬಿದ್ದರೆ ಮುಂದೆ ಉದ್ಭೋಧನಕ್ಕಾಗಿ ಮತ್ತಷ್ಟು ದುಡ್ಡನ್ನು ಕೊಡುವಂತೆ ಹೇಳಿದರು. ಆ ದಿವಸ ರಾತ್ರಿ ಊಟವಾದ ಮೇಲೆ ಸ್ವಾಮೀಜಿ ಪುನಃ ಶಿಷ್ಯನೊಡನೆ ‘ಉದ್ಬೋಧನ’ ಪತ್ರಿಕೆಯ ಸಂಬಂಧವಾಗಿ ಆಲೋಚನೆ ಮಾಡಿದರು. ಈ ಸಂದರ್ಭದಲ್ಲಿ ಇದನ್ನೂ ನಾವು ಇಲ್ಲಿ ಪಾಠಕರಿಗೆ ಹೇಳುತ್ತೇವೆ. 

 ಸ್ವಾಮೀಜಿ: “ಉದ್ಬೋಧನವು ಸಾಧಾರಣರಿಗೆ ಸಕಲ ವಿಚಾರಗಳಲ್ಲಿಯೂ ಜೀವನ ಪೋಷಕ ಆದರ್ಶಗಳನ್ನು ಕೊಡಬೇಕು. ಅಲ್ಲ, ಇಲ್ಲ, ಎಂಬ ಭಾವ ಮನುಷ್ಯನನ್ನು ನಿರ್ಜೀವ ಮಾಡಿಬಿಡುತ್ತದೆ ನೋಡುತ್ತಿಲ್ಲವೆ. ತಾಯಿತಂದೆಗಳು ಹುಡುಗರನ್ನು ಹಗಲು ರಾತ್ರಿ ಓದಿ ಬರೆಯಬೇಕೆಂದು ಹೊಡೆಯುತ್ತ, ‘ಇವನಿಗೆ ಏನೂ ಆಗುವುದಿಲ್ಲ, ಶುದ್ಧ ಕತ್ತೆ’ ಎಂದು ಬಯ್ಯುತ್ತಿದ್ದರೆ ಅವರ ಹುಡುಗರು ಅನೇಕ ಕಡೆಗಳಲ್ಲಿ ಹಾಗೇ ಆಗಿಬಿಡುತ್ತಾರೆ. ಹುಡುಗರಿಗೆ ಒಳ್ಳೆಯದನ್ನು ಹೇಳಿದರೆ ಉತ್ಸಾಹವನ್ನು ಕೊಟ್ಟರೆ, ಕಾಲಕ್ರಮದಲ್ಲಿ ನಿಜವಾಗಿಯೂ ಒಳ್ಳೆಯದಾಗುತ್ತದೆ. ಹುಡುಗರ ವಿಚಾರದಲ್ಲಿ ಯಾವ ನಿಯಮವೋ ಭಾವ ಪ್ರಪಂಚದ ಉಚ್ಚ ಅಧಿಕಾರ ದೃಷ್ಟಿಯಿಂದ ಯಾರು ಮಕ್ಕಳ ಹಾಗೆ ಇರುತ್ತಾರೊ ಅವರ ವಿಚಾರದಲ್ಲಿಯೂ ಹೀಗೆಯೆ. ಜೀವ ಪೋಷಕವಾದ ಭಾವಗಳನ್ನು ಕೊಡುತ್ತ ಬಂದರೆ ಸಾಧಾರಣರು ಮನುಷ್ಯರಾಗಿ ತಮ್ಮ ಕಾಲ ಮೇಲೆ ತಾವು ನಿಲ್ಲುವುದನ್ನು ಕಲಿಯುವರು. ಭಾಷೆ ಸಾಹಿತ್ಯ ದರ್ಶನ ಕವಿತೆ ಶಿಲ್ಪ ಇವುಗಳಲ್ಲಿಯೂ ಮನುಷ್ಯನು ಆಲೋಚಿಸುವ ಅಥವಾ ಮಾಡುವ ಸಮಸ್ತ ವಿಚಾರಗಳಲ್ಲಿಯೂ ತಪ್ಪನ್ನು ತೋರಿಸದೆ ಇವುಗಳನ್ನು ಹೇಗೆ ಕ್ರಮಕ್ರಮವಾಗಿ ಮತ್ತೂ ಚೆನ್ನಾಗಿ ಮಾಡಬಹುದು ಅದನ್ನೇ ಹೇಳಿಕೊಡಬೇಕು. ಭ್ರಮ ಪ್ರಮಾದಗಳನ್ನೇ ತೋರಿಸಿದರೆ ಮನುಷ್ಯನ ಮನಸ್ಸಿಗೆ ನೋವಾಗುತ್ತದೆ. ಪರಮಹಂಸರಲ್ಲಿ ನಾನು ಇದನ್ನೇ ನೋಡಿದ್ದು. ಯಾರನ್ನು ನಾವು ಹೇಯ ಎಂದು ತಿಳಿದಿದ್ದೆವೊ ಅಂಥವರಿಗೂ ಅವರು ಪ್ರೋತ್ಸಾಹವನ್ನು ಕೊಟ್ಟು ಜೀವನದಲ್ಲಿ ಅವರ ಮತಿಗತಿಗಳನ್ನು ತಿರುಗಿಸಿ ಬಿಡುತ್ತಿದ್ದರು. ಅವರು ಶಿಕ್ಷಣ ಕೊಡುತ್ತಿದ್ದ ರೀತಿಯೇ ಒಂದು ಅದ್ಭುತ ವ್ಯಾಪಾರ!” 

 ಈ ಮಾತುಗಳನ್ನು ಹೇಳಿ ಸ್ವಾಮೀಜಿ ಸ್ವಲ್ಪ ಸ್ಥಿರವಾದರು. ಸ್ವಲ್ಪ ಹೊತ್ತಿನ ಮೇಲೆ ಪುನಃ ಹೇಳತೊಡಗಿದರು: 

 “ಧರ್ಮಪ್ರಚಾರವೆಂದರೆ ಕೇವಲ ಯಾವುವೆಂದರೆ ಅದನ್ನು ತೆಗೆದುಕೊಂಡು ಯಾರೆಂದರೆ ಅವರ ಮೇಲೆ ಮುಸುಡಿಯನ್ನು ತಿರುಗಿಸುವ ಕೆಲಸವೆಂದು ತಿಳಿದುಕೊಳ್ಳಬೇಡ. ದೇಹ ಮನಸ್ಸು ಆತ್ಮ ಎಲ್ಲಕ್ಕೂ ಪೋಷಕ ಭಾವಗಳನ್ನು ಕೊಡಬೇಕು. ಹಾಗಲ್ಲದೆ ದ್ವೇಷಿಸಬಾರದು. ಒಬ್ಬರನ್ನೊಬ್ಬರು ದ್ವೇಷಿಸಿ ದ್ವೇಷಿಸಿಯೇ ನಮಗೆ ಅಧಃಪತನ ಉಂಟಾಗಿದೆ. ಈಗ ಕೇವಲ ಬಲಪ್ರದವಾಗುವ ಮತ್ತು ಜೀವಪೋಷಕವಾಗುವ ಭಾವವನ್ನೇ ಕೊಟ್ಟು ಜನರನ್ನು ಮೇಲಕ್ಕೆ ಎತ್ತಬೇಕು. ಮೊದಲು ಸಮಸ್ತ ಹಿಂದೂ ಜನರನ್ನು ಹೀಗೆ ಎತ್ತಬೇಕು. ಆಮೇಲೆ ಜಗತ್ತನ್ನು ಎತ್ತಬೇಕು. ಪರಮಹಂಸರು ಅವತಾರಮಾಡುವುದಕ್ಕೆ ಕಾರಣವೇ ಇದು. ಅವರು ಜಗತ್ತಿನಲ್ಲಿ ಯಾರಭಾವವನ್ನೂ ನಾಶಪಡಿಸಲಿಲ್ಲ. ಮಹಾ ಅಧಃಪತಿತನಾದ ಮನುಷ್ಯನಿಗೂ ಅವರು ಅಭಯವನ್ನು ಕೊಟ್ಟು ಮೇಲೆತ್ತುತ್ತಿದ್ದರು. ನಾವೂ ಅವರ ಮಾರ್ಗವನ್ನು ಅನುಸರಿಸಿ ಎಲ್ಲರನ್ನೂ ಮೇಲಕ್ಕೆ ಎತ್ತಬೇಕು”. 

 “ನಿಮ್ಮ ಚರಿತ್ರೆ ಸಾಹಿತ್ಯ ಪುರಾಣ ಮುಂತಾದ ಶಾಸ್ತ್ರಗ್ರಂಥಗಳೆಲ್ಲ ಮನುಷ್ಯನಿಗೆ ಕೇವಲ ಭಯವನ್ನೇ ತೋರಿಸುತ್ತವೆ. ಜನರಿಗೆ ನೀವು ನರಕಕ್ಕೇ ಹೋಗುತ್ತೀರಿ, ನಿಮಗೆ ಮಾರ್ಗವಿಲ್ಲ, ಎಂದು ಮಾತ್ರ ಹೇಳುತ್ತವೆ. ಅದರಿಂದಲೇ ಈ ಉತ್ಸಾಹಹೀನ ಸಾಹಿತ್ಯವೂ ದೌರ್ಬಲ್ಯವೂ ಭರತಖಂಡಕ್ಕೆ ಅಸ್ತಿಗತವಾಗಿಬಿಟ್ಟಿದೆ. ಆದಕಾರಣ ವೇದ ವೇದಾಂತಗಳ ಉಚ್ಚ ಉಚ್ಚ ಭಾವಗಳನ್ನು ಸುಲಭವಾದ ಮಾತಿನಲ್ಲಿ ಜನರಿಗೆ ತಿಳಿಯಪಡಿಸಬೇಕಾಗಿದೆ. ಸದಾಚಾರ ಸದ್ವ್ಯವಹಾರ ವಿದ್ಯೆ ಶಿಕ್ಷಣ ಇವುಗಳನ್ನು ಕೊಟ್ಟು ಬ್ರಾಹ್ಮಣನನ್ನೂ ಚಂಡಾಲನನ್ನೂ ಒಂದು ಸಮಕ್ಕೆ ತಂದು ನಿಲ್ಲಿಸಬೇಕು. ‘ಉದ್ಬೋಧನ' ಪತ್ರಿಕೆಯಲ್ಲಿ ಇದನ್ನೆಲ್ಲ ಬರೆದು ಹುಡುಗರು ಮಕ್ಕಳು ಗಂಡಸರು ಹೆಂಗಸರು ಮುದುಕರು ಎಲ್ಲರನ್ನೂ ಉದ್ಧಾರಮಾಡಿ, ನೋಡೋಣ. ಹಾಗಾದರೆ ನಿಮ್ಮ ವೇದ ವೇದಾಂತ ವ್ಯಾಸಂಗ ಸಾರ್ಥಕವಾಯಿತೆಂದು ತಿಳಿಯುತ್ತೇನೆ. ಏನನ್ನುತ್ತೀಯ, ಕೈಲಾಗುತ್ತದೆಯೋ?” 

 ಶಿಷ್ಯ: “ತಮ್ಮ ಆಶೀರ್ವಾದ ಮತ್ತು ಆದೇಶಗಳಾದರೆ ಎಲ್ಲ ಸಂಗತಿಗಳಲ್ಲಿಯೂ ಜಯಶೀಲನಾಗುವೆನೆಂದು ತೋರುತ್ತದೆ.” 

 ಸ್ವಾಮೀಜಿ: “ಮತ್ತೊಂದು ಮಾತು. ಶರೀರವನ್ನು ಬಲವಾಗಿ ಇಟ್ಟುಕೊಳ್ಳುವುದನ್ನು ನೀನು ಕಲಿಯಬೇಕು, ಮತ್ತು ಇತರರಿಗೆ ಕಲಿಸಬೇಕು. ನೋಡಿಲ್ಲವೆ, ಈಗಲೂ ನಾನು ನಿತ್ಯವೂ ಡಂಬಲ್ ಸಾಧನೆ ಮಾಡುತ್ತಿರುವುದನ್ನು? ನಿತ್ಯವೂ ಬೆಳಿಗ್ಗೆ ಸಾಯಂಕಾಲ ಸಂಚಾರ ಮಾಡು. ದೇಹಶ್ರಮಪಡು. ದೇಹವೂ ಮನಸ್ಸೂ ಸರಿಸಮನಾಗಿ ಅಭಿವೃದ್ಧಿಯಾಗುತ್ತ ಹೋಗಬೇಕು. ಎಲ್ಲದಕ್ಕೂ ಮತ್ತೊಬ್ಬರನ್ನು ನೆಚ್ಚಿಕೊಂಡು ಏತಕ್ಕೆ ಹೋಗುತ್ತಿರಬೇಕು. ಶರೀರವನ್ನು ಬಲಪಡಿಸಿಕೊಳ್ಳುವುದರ ಉಪಯೋಗವನ್ನು ತಿಳಿದುಕೊಂಡರೆ ಆಮೇಲೆ ತಾವೇ ಇದರಲ್ಲಿ ಪ್ರವರ್ತಿಸುತ್ತಾರೆ. ಉಪಯೋಗವನ್ನು ತಿಳಿಸುವುದಕ್ಕೆ ಈ ಶಿಕ್ಷಣ ಬೇಕಾಗಿರುವುದು.” 

