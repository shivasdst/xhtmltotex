
\chapter{ಬಾಲ್ಯದ ಆಟಪಾಠಗಳು}

ನರೇಂದ್ರನಾಥನಿಗೆ ಎಂಟು ವರುಷಗಳು ತುಂಬುವ ಹೊತ್ತಿಗೆ ಅವನನ್ನು ಈಶ್ವರಚಂದ್ರ ವಿದ್ಯಾಸಾಗರನ ಶಾಲೆಗೆ ಹಾಕಿದರು. ಅಲ್ಲಿ ನರೇಂದ್ರ ಚೆನ್ನಾಗಿ ಓದಲು ಉಪಕ್ರಮಿಸಿದನು. ತನ್ನ ಬುದ್ಧಿ ಚತುರತೆಯಿಂದ ಎಲ್ಲಾ ಉಪಾಧ್ಯಾಯರನ್ನು ಒಲಿಸಿಕೊಂಡನು. ಆದರೆ ನರೇಂದ್ರ ಕ್ಲಾಸಿನ ರೂಮಿನಲ್ಲಿರುವಾಗಲೂ ಶಾಂತಿಯಿಂದ ಇರುತ್ತಿರಲಿಲ್ಲ. ಅವನ ಮನಸ್ಸನ್ನೆಲ್ಲ ಕೇಂದ್ರೀಕರಿಸಲು ಉಪಾಧ್ಯಾಯರು ಹೇಳುವ ಪಾಠವೇ ಸಾಲುತ್ತಿರಲಿಲ್ಲ ನರೇಂದ್ರನದು ಅಂತಹ ಪ್ರಚಂಡ ಬುದ್ಧಿಶಕ್ತಿ. ಉಪಾಧ್ಯಾಯರು ಹೇಳುವುದನ್ನು ಕ್ಷಣಾರ್ಧದಲ್ಲಿ ಗ್ರಹಿಸಿಬಿಡುತ್ತಿದ್ದ. ಮಿಕ್ಕ ಕಾಲವನ್ನು ಸ್ನೇಹಿತರೊಡನೆ ಹಾಸ್ಯ ಪರಿಹಾಸ್ಯಗಳನ್ನು ಮಾಡುವುದು, ತಾನು ಓದಿದ ಅಥವಾ ಕೇಳಿದ ಕಥೆಗಳನ್ನು ಹುಡುಗರಿಗೆ ಹೇಳುವುದು ಇದರಲ್ಲಿ ಕಳೆಯುತ್ತಿದ್ದ. ಒಂದು ದಿನ ಉಪಾಧ್ಯಾಯರು ಪಾಠ ಹೇಳುತ್ತಿದ್ದಾಗ ಹಿಂದಿನ ಬೆಂಚಿನಲ್ಲಿ ಹುಡುಗರು ಗಮನವಿಟ್ಟು ಪಾಠವನ್ನು ಕೇಳದೆ ಇರುವುದು ಗೊತ್ತಾಯಿತು. ಉಪಾಧ್ಯಾಯರು ಹುಡುಗರನ್ನು “ನಾನು ಈಗ ಏನು ಹೇಳುತ್ತಿದ್ದೆ ಹೇಳಿ” ಎಂದು ಕೇಳಿದರು. ಆದರೆ ಹುಡುಗರಿಗೆ ಚೇಷ್ಟೆಮಾಡುವುದು ಪಾಠವನ್ನು ಗಮನವಿಟ್ಟು ಕೇಳುವುದು ಒಟ್ಟಿಗೆ ಹೇಗೆ ಬರಬೇಕು? ಅವರು ಹೇಳುವುದಕ್ಕೆ ಆಗಲಿಲ್ಲ. ಉಪಾಧ್ಯಾಯರು ಅವರನ್ನು ಬೆಂಚಿನ ಮೇಲೆ ನಿಲ್ಲಿಸಿದರು. ಅದೇ ಬೆಂಚಿನಲ್ಲಿ ಕುಳಿತಿದ್ದ ನರೇಂದ್ರನನ್ನು ಪಾಠದ ವಿಷಯ ಕೇಳಿದರು. ನರೇಂದ್ರ ಎಲ್ಲವನ್ನು ಹೇಳಿಬಿಟ್ಟ. ಅವನು ಬೆಂಚಿನ ಮೇಲೆ ನಿಂತುಕೊಳ್ಳಬೇಕಾಗಿಲ್ಲ ಎಂದು ಉಪಾಧ್ಯಾಯರು ಹೇಳಿದರು. ಆದರೆ ನರೇಂದ್ರನಾದರೋ ಬೇರೆ ಗುಂಪಿಗೆ ಸೇರಿದ ವಿದ್ಯಾರ್ಥಿ. ತಾನೂ ಕೂಡ ಇತರ ಹುಡುಗರೊಡನೆ ಸೇರಿ ಗಲಾಟೆ ಮಾಡುತ್ತಿದ್ದ; ಅವರಿಗೆ ಆದ ಶಿಕ್ಷೆ ತನಗೂ ಆಗಬೇಕೆಂದು ಉಪಾಧ್ಯಾಯರು ಬೇಡ ಎಂದರೂ ತನ್ನ ಸ್ನೇಹಿತರೊಂದಿಗೆ ಬೆಂಚಿನ ಮೇಲೆ ನಿಂತನು. ಬೀಸುವ ದೊಣ್ಣೆ ಪೆಟ್ಟು ತಪ್ಪಿಸಿಕೊಂಡರೆ ಸಾಕು ಎನ್ನುವ ಸ್ವಭಾವ ಮನುಷ್ಯನದು. ಆದರೆ ನರೇಂದ್ರನಾದರೊ ತಾನೇ ಶಿಕ್ಷೆಗೆ ಮುಂದೆ ಬಂದು ಅದನ್ನು ಬಲಾತ್ಕಾರವಾಗಿ ತೆಗೆದುಕೊಂಡು ಅನುಭವಿಸುವುದನ್ನು ನೋಡುವೆವು.

ನರೇಂದ್ರ ಮೇಲಿನ ತರಗತಿಗೆ ಬಂದ ಮೇಲೆ ಅವನು ಇಂಗ್ಲೀಷನ್ನು ಕಲಿಯಬೇಕಾಯಿತು. ಇಂಗ್ಲೀಷ್ ಮ್ಲೇಚ್ಛರ ಭಾಷೆ, ತಾನು ಏತಕ್ಕೆ ಅದನ್ನು ಕಲಿಯಲು ಹೋಗಬೇಕು, ತನಗೆ ಅದು ಬೇಡ ಎಂದು ಹಟಹಿಡಿದ. ಶಾಲೆಯಲ್ಲಿ ಉಪಾಧ್ಯಾಯರು ಅವನಿಗೆ ಕಲಿಯಲು ಒಪ್ಪಿಸಲು ಪ್ರಯತ್ನಿಸಿ ನಿರಾಶರಾದರು. ಮನೆಗೆ ಹೋದರೆ ಅಲ್ಲಿ ತಾಯಿ ತಂದೆ ಮತ್ತು ಇತರ ನೆಂಟರುಗಳೆಲ್ಲ ನರೇಂದ್ರ ಇಂಗ್ಲೀಷ್ ಓದಲೇಬೇಕೆಂದೂ ಅದೊಂದು ಆಗಿನ ಕಾಲದಲ್ಲಿ ಎಲ್ಲಾ ವಿದ್ಯೆಗೂ ಇರುವ ಏಕಮಾತ್ರ ಕೀಲಿಕೈ ಎಂದೂ ಒತ್ತಾಯ ಮಾಡಿದ ಮೇಲೆ ನರೇಂದ್ರ ಇಂಗ್ಲೀಷನ್ನು ಕಲಿಯಲು ಪ್ರಾರಂಭಿಸಿದನು. ಕಲಿಯುವುದಕ್ಕೆ ಮುಂಚೆ ನರೇಂದ್ರ ತುಂಬಾ ಅನುಮಾನಿಸುತ್ತಿದ್ದರೂ ಒಮ್ಮೆ ಇಂಗ್ಲೀಷನ್ನು ಕಲಿಯಲು ಯಾವಾಗ ಮನಸ್ಸು ಮಾಡಿದನೋ ಆಗ ಎಲ್ಲರಿಗಿಂತ ಬೇಗ ಬೇಗ ಕಲಿತುಕೊಳ್ಳುತ್ತ ಹೋಗುತ್ತಿದ್ದ.

ನರೇಂದ್ರ ಆ ಪಾಠಶಾಲೆಯಲ್ಲಿ ಓದುತ್ತಿದ್ದಾಗ ಒಂದು ಘಟನೆ ನಡೆಯಿತು. ಅಲ್ಲಿ ಒಬ್ಬ ಉಪಾಧ್ಯಾಯರು ಬಹು ಕೋಪಿಷ್ಠರು. ಕೋಪ ಬಂದಾಗ ಹುಡುಗರನ್ನು ಚೆನ್ನಾಗಿ ಹಿಡಿದುಕೊಂಡು ಹೊಡೆಯುತ್ತಿದ್ದರು. ಒಮ್ಮೆ ಹಾಗೆ ಕೋಪದಿಂದ ವಿಕಾರವಾದ ಮುಖವನ್ನು ನೋಡಿ ನರೇಂದ್ರನಿಗೆ ಸುಮ್ಮನಿರಲು ಆಗಲಿಲ್ಲ. ಅವನು ಜೋರಾಗಿ ನಕ್ಕುಬಿಟ್ಟ. ಆ ಉಪಾಧ್ಯಾಯರ ಕೋಪ ನರೇಂದ್ರನ ಮೇಲೆ ತಿರುಗಿತು. ನರೇಂದ್ರನನ್ನು ಹಿಡಿದುಕೊಂಡು ಚೆನ್ನಾಗಿ ಬಡಿದು, “ಇನ್ನುಮೇಲೆ ನಗುವುದಿಲ್ಲ ಎಂದು ಹೇಳು ಮತ್ತು ನಕ್ಕಿದ್ದಕ್ಕೆ ಕ್ಷಮಾಪಣೆ ಕೇಳು” ಎಂದು ಗರ್ಜಿಸಿದರು. ನರೇಂದ್ರನಾದರೋ ಬೆದರಿಕೆಗೆ ಬಗ್ಗುವವನಲ್ಲ. “ನಾನು ಹಾಗೆ ಕ್ಷಮಾಪಣೆ ಕೇಳುವುದಿಲ್ಲ” ಎಂದನು. ಉಪಾಧ್ಯಾಯರ ಕೋಪ ಇನ್ನೂ ಉಕ್ಕಿತು. ಹೊಡೆದಾದ ಮೇಲೆ ನರೇಂದ್ರನ ಕಿವಿಗಳನ್ನು ಹಿಡಿದೆಳೆದು, ಅವನನ್ನು ಬೆಂಚಿನ ಮೇಲೆ ನಿಲ್ಲಿಸಿದರು. ಆ ಸಮಯದಲ್ಲಿ ಕಿವಿ ಹರಿದು ರಕ್ತ ಬರಲಾರಂಭಿಸಿತು. ನರೇಂದ್ರ ಅಳುತ್ತ ಉಪಾಧ್ಯಾಯರನ್ನು “ನೀವು ಯಾರು ನನ್ನನ್ನು ಹೊಡೆಯುವುದಕ್ಕೆ? ನನ್ನನ್ನು ಮುಟ್ಟೀರಿ ಜೋಕೆ” ಎಂದು ಕಿರುಚಿಕೊಳ್ಳುತ್ತಿದ್ದ. ಕ್ಲಾಸಿನ ರೂಮಿನಲ್ಲಿ ದೊಡ್ಡ ಆಂದೋಳನವೇ ಆಯಿತು. ಆ ಶಾಲೆಯ ಮುಖ್ಯಸ್ಥರಾದ ಈಶ್ವರಚಂದ್ರ ವಿದ್ಯಾಸಾಗರ ಆಗ ಕ್ಲಾಸಿನ ರೂಮಿನ ಒಳಕ್ಕೆ ಬಂದರು. ನರೇಂದ್ರ ಅವರಿಗೆ ನಡೆದಿರುವುದನ್ನೆಲ್ಲ ಹೇಳಿ ಕೈಯಲ್ಲಿ ಪುಸ್ತಕಗಳನ್ನು ಹಿಡಿದುಕೊಂಡು “ಇನ್ನುಮೇಲೆ ನಾನು ನಿಮ್ಮ ಸ್ಕೂಲಿಗೆ ಬರುವುದಿಲ್ಲ; ಹೊರಟುಹೋಗುತ್ತೇನೆ” ಎಂದು ಹೊರಟನು. ಈಶ್ವರಚಂದ್ರ ನರೇಂದ್ರನನ್ನು ತಮ್ಮ ಕೋಣೆಗೆ ಕರೆದುಕೊಂಡು ಹೋಗಿ ಅವನನ್ನು ಸಮಾಧಾನ ಮಾಡಿದರು. ಅನಂತರ ಹುಡುಗರನ್ನು ಹಾಗೆಲ್ಲ ಹೊಡೆಯ ಕೂಡದೆಂದು ಉಪಾಧ್ಯಾಯರಿಗೆ ತಿಳಿಸಿದರು. ಮನೆಗೆ ಹೋಗಿ ನರೇಂದ್ರ ತಾಯಿಗೆ ಸ್ಕೂಲಿನಲ್ಲಿ ನಡೆದ ಘಟನೆಯನ್ನು ತಿಳಿಸಿದನು. ತಾಯಿ ಮಗನಿಗೆ ಆದ ಗಾಯವನ್ನು ನೋಡಿ ಇನ್ನು ಮೇಲೆ ನೀನು ಸ್ಕೂಲಿಗೆ ಹೋಗಬೇಕಾಗಿಲ್ಲ ಎಂದು ಹೇಳಿದಳು. ಆದರೆ ಮಾರನೇ ದಿನ ನರೇಂದ್ರ ಎಂದಿನಂತೆ ಪುಸ್ತಕಗಳನ್ನು ಹಿಡಿದುಕೊಂಡು ಶಾಲೆಗೆ ಹೊರಟ. ಹಿಂದಿನ ದಿನ ಆದುದನ್ನೆಲ್ಲ ಸಂಪೂರ್ಣ ಮರೆತೇಬಿಟ್ಟಿದ್ದ. ಆದ ಅನ್ಯಾಯವನ್ನೆಲ್ಲ ಮೆಲುಕು ಹಾಕುವವನಲ್ಲ. ಸೇಡನ್ನು ತೀರಿಸಿಕೊಳ್ಳುವ ಸ್ವಭಾವದವನಲ್ಲ ನರೇಂದ್ರ. ಅದೂ ಅಲ್ಲದೆ ಆ ತರಗತಿಯ ತನ್ನ ಸ್ನೇಹಿತರನ್ನು ಬಿಟ್ಟಿರಲಾರ ಅವನು.

ಪಾಠ ಪ್ರವಚನ ನಡೆಯುವ ಕ್ಲಾಸಿನ ರೂಮು ನರೇಂದ್ರನ ಜೀವನದ ಯಾವುದೋ ಒಂದು ಸಣ್ಣ ಅಂಶ. ನಿಜವಾಗಿ ಅವನ ಪ್ರಾಣವಿದ್ದುದು ಹೊರಗೆ. ಪಾಠ ಪ್ರವಚನಗಳನ್ನು ತಿಳಿದುಕೊಳ್ಳುವುದರಲ್ಲಿ ಹೇಗೆ ಅವನಿಗೆ ಕುಶಾಗ್ರ ಬುದ್ಧಿಯೋ ಹಾಗೆಯೇ ಆಟವಾಡುವುದರಲ್ಲಿ ಅಷ್ಟೇ ಚುರುಕು. ಜೀವನದಲ್ಲಿ ಆಟಪಾಠಗಳೆರಡೂ ಒಬ್ಬ ಹುಡುಗನಲ್ಲಿ ಒಟ್ಟಿಗೇ ಸೇರುವುದು ಬಹಳ ಅಪರೂಪ. ಒಂದರಲ್ಲಿ ಮುಂದೆ ಇದ್ದರೆ ಮತ್ತೊಂದರಲ್ಲಿ ಹಿಂದೆ. ಯಾವಾಗಲೂ ಪುಸ್ತಕ ಪಿಶಾಚಿಯಂತೆ ಓದುವ ಹುಡುಗರಿಗೆ ಆಟ ಎಂದರೆ ತಲೆನೋವು. ಅವರಿಗೆ ಹಾಗೆ ಆಡುವುದು ಕಾಲವನ್ನು ವ್ಯರ್ಥಮಾಡಿದಂತೆ. ಅದರಂತೆಯೇ ಆಟದಲ್ಲಿ ಮುಂದೆ ಇರುವವನು ಪಾಠದಲ್ಲಿ ಮುಂದೆ ಇರುವುದು ಅಪರೂಪ. ಆದರೆ ಒಂದು ಜೀವನದ ಶೀಲ ನಿರ್ಮಾಣದ ದೃಷ್ಟಿಯಿಂದ ನೋಡಿದರೆ, ಪಾಠಕ್ಕೆ ಇರುವ ಪ್ರಾಮುಖ್ಯತೆ ಆಟಕ್ಕೂ ಇದೆ. ಆಟ ನಮ್ಮಲ್ಲಿ ಹಲವಾರು ಗುಣಗಳನ್ನು ಪೋಷಿಸುತ್ತದೆ. ಮುಂದಾಳು ಆಗುವುದಕ್ಕೆ ಅವಕಾಶ ಕೊಡುವುದು, ಅಂಜಿಕೆಯನ್ನು ಹೊರದೂಡುವುದು. ಎಲ್ಲಕ್ಕಿಂತ ಹೆಚ್ಚಾಗಿ ಒಂದು ಕ್ರೀಡಾ ಮನೋಭಾವ ಬೆಳೆಯುವಂತೆ ಮಾಡುವುದು. ಸೋಲು ಗೆಲುವುಗಳನ್ನು ಮರೆತು ಆಟಗಾರ ಎರಡನ್ನೂ ಸಮಚಿತ್ತದಿಂದ ಸ್ವೀಕರಿಸುವನು. ಏಕೆಂದರೆ ಆಟದಲ್ಲಿ ಇಂದು ಸೋತವನು ನಾಳೆ ಗೆಲ್ಲುತ್ತಾನೆ, ಇಂದು ಗೆದ್ದವನು ನಾಳೆ ಸೋಲುವ ಸಂಭವವೂ ಇದೆ. ಆಟ ಎಂದರೆ ಇದೇ. ಈ ಸೋಲು ಗೆಲುವುಗಳನ್ನು ಹೇಗೆ ನಾವು ನೋಡಬೇಕು ಎಂಬುದನ್ನು ನಮಗೆ ಕಲಿಸುವುದು ಕ್ರೀಡೆ. ಆದಕಾರಣವೇ ನಾವು ಆಟಕ್ಕಾಗಿ ದೇವರು ಸೃಷ್ಟಿಸಿರುವನು ಎಂದು ಹೇಳುತ್ತೇವೆ. ಸೃಷ್ಟಿಯೇ ಒಂದು ದೊಡ್ಡ ಆಟ, ಅದರಲ್ಲಿ ಪ್ರತಿಜೀವಿಯದೂ ಒಂದು ಸಣ್ಣ ಆಟ. ಇದೇ ಸಂಸಾರ. ಚೆನ್ನಾಗಿರುವ ಆಟಗಾರ ಈ ಸಂಸಾರದಲ್ಲಿ ಚೆನ್ನಾಗಿ ಈಜುತ್ತಾನೆ. ಆಟ ನಮ್ಮ ಶರೀರವನ್ನು ಗಟ್ಟಿಮುಟ್ಟಾಗಿ ಮಾಡುವುದು ನಿಜ. ಆದರೆ ಅದು ನಮ್ಮ ಮನಸ್ಸಿನ ಬೆಳವಣಿಗೆಗೆ ಅದಕ್ಕಿಂತಲೂ ಹೆಚ್ಚು ಅವಕಾಶವನ್ನು ಕೊಡುವುದು.

ಯಾವ ಆಟವಾದರೂ ಆಗಲಿ ನರೇಂದ್ರ ಅಲ್ಲಿ ಮುಂದಾಳಾಗಿರುವುದನ್ನು ನೋಡುವೆವು. ಆಟಗಾರರ ಮಧ್ಯೆ ಮನಸ್ತಾಪಗಳಿದ್ದರೆ ಬಗೆಹರಿಸುವುದಕ್ಕೆ ನರೇಂದ್ರ ಮುಂದೆ ಹೋಗುವುದನ್ನು ನೋಡುವೆವು. ಒಂದು ಆಟ ಬೇಜಾರಾದರೆ ಮತ್ತೊಂದು ಆಟವನ್ನು ಕಂಡುಹಿಡಿಯುವುದರಲ್ಲಿ ಅವನ ಮನಸ್ಸು ನಿರತವಾಗಿರುವುದನ್ನು ನೋಡುವೆವು.

ಗೋಲಿ, ಕುಣಿಯುವುದು, ನೆಗೆಯುವುದು, ಬಾಕ್ಸಿಂಗ್ ಮುಂತಾದುವುಗಳು ನರೇಂದ್ರನಿಗೆ ಪ್ರಿಯವಾದ ಆಟಗಳು. ಹಲವು ಹೊಸ ಆಟಗಳನ್ನು ಕಂಡುಹಿಡಿದು ತನ್ನ ಸ್ನೇಹಿತರನ್ನು ಸಂತೋಷಪಡಿಸುತ್ತಿದ್ದ. ಕಲ್ಕತ್ತಾನಗರಕ್ಕೆ ಆಗತಾನೆ ಸೋಡಾ ವಾಟರ್ ಮಾಡುವ ಯಂತ್ರಗಳು ಬಂದಿದ್ದವು. ನರೇಂದ್ರ ಆಟದ ಸೋಡಾ ಮಾಡುವ ಯಂತ್ರವನ್ನು ಕಂಡುಹಿಡಿಯುತ್ತಾನೆ. ಆಗತಾನೆ ಕಲ್ಕತ್ತೆಯ ಬೀದಿಗಳನ್ನು ಬೆಳಗಲು ಗ್ಯಾಸ್ ಲೈಟ್ ಬಂದಿತ್ತು. ಆಟದ ಗ್ಯಾಸ್ ಲ್ಯಾಂಪನ್ನು ಮಾಡುತ್ತಿದ್ದ. ಆಟದ ಟ್ರಾಮ್, ರೈಲುಗಾಡಿಗಳನ್ನು ಮಾಡುವುದರಲ್ಲಿ ನಿರತನಾಗಿರುತ್ತಿದ್ದ. ಕೆಲವು ವೇಳೆ ಹುಡುಗರಲ್ಲಿ ಎರಡು ಪಾರ್ಟಿಗಳನ್ನು ಮಾಡಿ ಒಬ್ಬರು ಮತ್ತೊಬ್ಬರ ಮೇಲೆ ಯುದ್ಧ ಮಾಡುವಂತೆ ಮಾಡುತ್ತಿದ್ದ. ಸೋಲುವ ಘಟ್ಟಕ್ಕೆ ಬಂದಾಗ, ಇನ್ನೇನು ಆಟ ಜಗಳಕ್ಕೆ ತಿರುಗುವ ಸಮಯ ಬರುವ ಹೊತ್ತಿಗೆ, ಇವನು ಮಧ್ಯದಲ್ಲಿ ಬಂದು ಶಾಂತಿ ನೆಲಸುವಂತೆ ಮಾಡುತ್ತಿದ್ದ.

ನರೇಂದ್ರನಿಗೆ ಒಂದೇ ಆಟ ಎಷ್ಟೇ ಚೆನ್ನಾಗಿದ್ದರೂ ಸ್ವಲ್ಪ ಕಾಲದ ಮೇಲೆ ಬೇಜಾರಾಗಿ ಹೋಗುತ್ತಿತ್ತು. ಮತ್ತೆ ಯಾವುದಾದರೂ ಹೊಸದನ್ನು ಯೋಚಿಸುತ್ತಿದ್ದ. ನರೇಂದ್ರ ತನ್ನ ಜೊತೆಗಾರರನ್ನೆಲ್ಲ ಸೇರಿಸಿಕೊಂಡು ನಾಟಕದ ಪಾರ‍್ಟಿಯನ್ನು ಮಾಡಿದ. ಕೆಲವು ನಾಟಕಗಳನ್ನು ಅಭ್ಯಾಸಮಾಡಿ ತಮ್ಮ ಮನೆಯ ಪೂಜಾ ಮಂದಿರದ ಪ್ರಾಂಗಣದಲ್ಲಿ ಆಡಿದ. ನಾಟಕಕ್ಕೆ ಹುಡುಗರು ಜಮಾಯಿಸಿದ ಸಾಮಾನುಗಳಿಂದ ಅದು ತುಂಬಿ ಹೋಯಿತು. ಅದಕ್ಕಿಂತ ಹೆಚ್ಚಾಗಿ ಅವರ ಗದ್ದಲವಂತೂ ಮನೆಯಲ್ಲಿದ್ದ ಅವನ ಸೋದರಮಾವನಿಗೆ ಹಿಡಿಸಲಿಲ್ಲ. ಅವನು ಒಂದು ದಿನ ಈ ಹುಡುಗರ ರಂಗಭೂಮಿ, ಸ್ಕ್ರೀನ್ ಮುಂತಾದವನ್ನೆಲ್ಲ ಕಿತ್ತು ಬಿಸುಟ. ನಾಟಕದ ಕಂಪನಿ ಮೂಲೆಗೆ ಕುಳಿತುಕೊಂಡಿತು. ಇನ್ನುಮೇಲೆ ಅದಕ್ಕಂತೂ ಉಳಿಗಾಲವಿಲ್ಲ ಇವರ ಮನೆಯಲ್ಲಿ. ಅನಂತರ ಮನೆಯ ಅಂಗಳದಲ್ಲಿ ಒಂದು ಗರಡಿ ಮನೆಯನ್ನು ನರೇಂದ್ರ ಪ್ರಾರಂಭ ಮಾಡಿದ. ಅಂಗಸಾಧನೆಗೆ ಸಣ್ಣ ಪುಟ್ಟ ವಸ್ತುಗಳನ್ನೆಲ್ಲ ಅಲ್ಲಿ ಸಂಗ್ರಹಿಸಿಟ್ಟನು. ಅವನ ಸ್ನೇಹಿತರೆಲ್ಲ ಅಲ್ಲಿ ನೆರೆದು ಪುನಃ ಗಲಾಟೆ ಆರಂಭವಾಯಿತು. ಅಂಗಸಾಧನೆ ಮಾಡುತ್ತಿದ್ದ ಒಬ್ಬ ಹುಡುಗನಿಗೆ ಕೈಮುರಿದು ಹೋಯಿತು. ಅವನು ಅದಕ್ಕಾಗಿ ಬ್ಯಾಂಡೇಜ್ ಕಟ್ಟಿಕೊಂಡು ಹಲವು ತಿಂಗಳು ನರಳಬೇಕಾಯಿತು. ಪುನಃ ನರೇಂದ್ರನ ಮಾವ ಗರಡಿ ಮನೆಗೆ ಸೇರಿದ ವಸ್ತುಗಳನ್ನೆಲ್ಲ ಆಚೆಗೆ ಬಿಸುಟು ಇನ್ನು ಮೇಲೆ ಮನೆಯಲ್ಲಿ ಏನನ್ನೂ ಆಡಕೂಡದೆಂದು ಕಟ್ಟಪ್ಪಣೆ ಮಾಡಿದ.

ನರೇಂದ್ರ ಇನ್ನು ಮೇಲೆ ಮನೆಯಲ್ಲಿ ಏನು ಮಾಡಿದರೂ ಉಳಿಗಾಲವಿಲ್ಲ ಎಂದು ಆಲೋಚನೆ ಮಾಡಿ ಹತ್ತಿರದಲ್ಲಿದ್ದ ನವಗೋಪಾಲ ಮಿತ್ರನ ಗರಡಿ ಮನೆಗೆ ಸೇರಿದ. ಅಲ್ಲಿ ಕುಸ್ತಿ, ಬೈಟಕ್, ಲಾಠಿ, ಈಜುವುದು, ದೋಣಿ ನಡೆಸುವುದು ಇವುಗಳನ್ನೆಲ್ಲ ಕಲಿಯುತ್ತಿದ್ದನು. ನರೇಂದ್ರನಿಗೆ ಅಲ್ಲಿಯ ವಾರ್ಷಿಕೋತ್ಸವದಲ್ಲಿ ಮೊದಲನೆಯ ಬಹುಮಾನವೂ ಬಂದಿತು.

ನರೇಂದ್ರ ಪಾಕಶಾಸ್ತ್ರದಲ್ಲಿಯೂ ಪ್ರಾವೀಣ್ಯತೆಯನ್ನು ಪಡೆಯಲು ಯತ್ನಿಸುವನು. ಹುಡುಗರಿಗೆಲ್ಲ ಚಂದಾ ತರುವಂತೆ ಹೇಳಿ ತಾನೇ ಬಹುಪಾಲನ್ನು ಹಾಕಿ ರುಚಿರುಚಿಕರವಾದ ಅಡಿಗೆಗಳನ್ನು ಮಾಡಿ ಹುಡುಗರಿಗೆಲ್ಲ ಬಡಿಸುತ್ತಿದ್ದನು.

ಶಾಲೆಗೆ ರಜಾ ಇದ್ದಾಗಲೆಲ್ಲ ನರೇಂದ್ರ ಹುಡುಗರನ್ನು ಬೇರೆ ಬೇರೆ ಕಡೆ ನೋಡುವುದಕ್ಕೆ ಕರೆದುಕೊಂಡು ಹೋಗುತ್ತಿದ್ದ. ಒಂದು ಸಲ ಅವರನ್ನೆಲ್ಲ ಬೊಟಾನಿಕಲ್ ತೋಟಕ್ಕೆ ಗಂಗಾನದಿಯ ಮೂಲಕ ಕರೆದುಕೊಂಡು ಹೋದನು. ಅದನ್ನೆಲ್ಲ ತೋರಿಸಿ ಹಿಂತಿರುಗುವ ಸಮಯದಲ್ಲಿ ಒಬ್ಬ ಹುಡುಗನಿಗೆ ವಾಂತಿ ಶುರುವಾಯಿತು. ಅವನಿಗೆ ತುಂಬಾ ನಿತ್ರಾಣವಾಗಿ ದೋಣಿಯಲ್ಲಿದ್ದನು. ದೋಣಿಯವನು ಹುಡುಗರಿಗೆ, “ನೀವು ದೋಣಿಯನ್ನೆಲ್ಲ ತೊಳೆದು ಹೋಗಬೇಕು ಇಲ್ಲದೇ ಇದ್ದರೆ ಇಳಿಯುವುದಕ್ಕೆ ಬಿಡುವುದಿಲ್ಲ” ಎಂದು ಹೆದರಿಸಿದನು. ಇಳಿಯುವ ಸ್ಥಳ ಬಂದಾಗಲೂ ಇಳಿಯುವುದಕ್ಕೆ ಅವಕಾಶವನ್ನು ಕೊಡಲಿಲ್ಲ. ನರೇಂದ್ರ ಹೆಚ್ಚು ಕೂಲಿಯನ್ನು ಕೊಡುತ್ತೇವೆ ನೀವೇ ಆ ಕೆಲಸವನ್ನು ಮಾಡಿ ಎಂದು ದೋಣಿಯವನಿಗೆ ಎಷ್ಟು ಹೇಳಿದರೂ ಅವನು ಕೇಳಲಿಲ್ಲ. ಹೀಗೆ ಹುಡುಗರೊಂದಿಗೆ ಚರ್ಚೆಯಾಗುತ್ತಿದ್ದಾಗ ದೋಣಿಯ ಒಂದು ಮೂಲೆಯಿಂದ ನರೇಂದ್ರ ಧುಮುಕಿ ತೀರದ ಕಡೆ ಹೋದ. ದೂರದಲ್ಲಿ ಇಬ್ಬರು ಇಂಗ್ಲೀಷ್ ಸಿಪಾಯಿಗಳು ಬರುತ್ತಿರುವುದನ್ನು ನೋಡಿ, ಅವರ ಬಳಿಗೆ ಓಡಿದ. ಅವರ ಕೈಯನ್ನು ಹಿಡಿದುಕೊಂಡು ದೋಣಿಯವನ ಕಡೆ ತೋರಿಸಿ, ಅವನು ನಮಗೆ ತೊಂದರೆ ಕೊಡುತ್ತಿದ್ದಾನೆ ಎಂದು ಹೇಳಿದ. ಆ ಇಂಗ್ಲೀಷ್ ಸಿಪಾಯಿಗಳಿಬ್ಬರೂ ದೋಣಿಯವನನ್ನು ಹೆದರಿಸಿದರು. ದೋಣಿಯವನು ಮರುಮಾತಿಲ್ಲದೆ ಹುಡುಗರಿಗೆ ಇಳಿಯುವುದಕ್ಕೆ ಅವಕಾಶ ಮಾಡಿಕೊಟ್ಟನು. ಸಿಪಾಯಿಗಳು ನರೇಂದ್ರನನ್ನು ಕಂಡು ತುಂಬಾ ಸಂತೋಷಪಟ್ಟರು. “ನಾವು ನಾಟಕಕ್ಕೆ ಹೋಗುತ್ತಿರುವೆವು. ನೀನು ಬಂದರೆ ಕರೆದುಕೊಂಡು ಹೋಗುತ್ತೇವೆ” ಎಂದರು. ಆದರೆ ನರೇಂದ್ರ ಹುಡುಗರ ಮೇಲ್ವಿಚಾರಣೆಯಲ್ಲಿದ್ದುದರಿಂದ ಹೋಗದೆ ಅವರು ಮಾಡಿದ ಉಪಕಾರಕ್ಕೆ ವಂದನೆಗಳನ್ನು ಸಲ್ಲಿಸಿದ.

ಮತ್ತೊಂದು ಸಲ ಕಲ್ಕತ್ತ ರೇವಿಗೆ ಒಂದು ಯುದ್ಧದ ಹಡಗು ಬಂದಿತು. ಅದನ್ನು ನೋಡಲು ಹೋಗುವುದಕ್ಕೆ ಕ್ಯಾಪ್ಟನ್ ಅಪ್ಪಣೆ ಚೀಟಿಯನ್ನು ಕೊಡುತ್ತಿದ್ದ. ಕಲ್ಕತ್ತೆಯ ಅನೇಕರು ನೋಡಿಕೊಂಡು ಬಂದು ನರೇಂದ್ರನ ಸ್ನೇಹಿತರಿಗೆ ಅದನ್ನು ಹೇಳಿದರು. ಆ ಯುದ್ಧದ ಹಡಗನ್ನು ನೋಡುವುದಕ್ಕೆ ಅಪ್ಪಣೆಯನ್ನು ಪಡೆದುಕೊಂಡು ಬಾ ಎಂದು ಸ್ನೇಹಿತರು ನರೇಂದ್ರನನ್ನು ಕಾಡಿದರು. ನರೇಂದ್ರ ತಾನೆ ಒಂದು ಅರ್ಜಿಯನ್ನು ತಯಾರುಮಾಡಿ ಹಡಗಿನ ಕ್ಯಾಪ್ಟನ್ ಇದ್ದಕಡೆ ಹೋದ. ಬಾಗಿಲಿನ ಬಳಿ ನಿಂತಿದ್ದ ದರ್ವಾನ್ ಈ ಹುಡುಗನನ್ನು ಬಿಡಲಿಲ್ಲ. ಆದರೆ ನರೇಂದ್ರ ಹತಾಶನಾಗಲಿಲ್ಲ. ಆ ಕಟ್ಟಡದ ಸುತ್ತಲೂ ಹೋಗಿ ನೋಡಲು ಪ್ರಾರಂಭಿಸಿದ. ಅದರ ಹಿಂದೆ ಒಂದು ಬಾಗಿಲು ಇತ್ತು. ಅಲ್ಲಿ ಯಾರೂ ಇರಲಿಲ್ಲ. ನರೇಂದ್ರ ಮುಂದೆ ಹೋದಾಗ ಅಲ್ಲೇ ಹಡಗಿನ ಕ್ಯಾಪ್ಟನ್ ಕುಳಿತಿದ್ದುದನ್ನು ನೋಡಿದ. ಅನೇಕರು ಅವನಿಂದ ಅಪ್ಪಣೆ ಚೀಟಿಯನ್ನು ಪಡೆದುಕೊಳ್ಳುತ್ತಿದ್ದರು. ನರೇಂದ್ರನೂ ಅವರ ಹಿಂದೆ ನಿಂತುಕೊಂಡ. ತನ್ನ ಸರದಿ ಬಂದಾಗ ತಾನೂ ತನ್ನ ಪತ್ರವನ್ನು ಅವನ ಮುಂದೆ ಇಟ್ಟಾಗ ಹಡಗನ್ನು ನೋಡಲು ಇವನಿಗೂ ಅಪ್ಪಣೆ ಸಿಕ್ಕಿತು. ಅದನ್ನು ತೆಗೆದುಕೊಂಡು ಬರುವಾಗ ಮುಂದಿನ ಬಾಗಿಲಿನಿಂದಲೇ ಹೊರಗೆ ಬಂದ. ಆಗ ಇವನನ್ನು ಒಳಗೆ ಬಿಡುವುದಿಲ್ಲ ಎಂದು ಹೇಳಿದ ದರ್ವಾನ್ ನೋಡುತ್ತಾನೆ ಈ ಹುಡುಗ ಆಗಲೇ ಒಳಗಿನಿಂದ ಬರುತ್ತಿದ್ದ! ಅದನ್ನು ಕಂಡು ಆಶ್ಚರ್ಯದಿಂದ “ನೀನು ಹೇಗೆ ಒಳಗೆ ಹೋದೆ?” ಎಂದು ಅವನು ಕೇಳಿದ. ಅದಕ್ಕೆ ನರೇಂದ್ರ ತಮಾಷೆಯಿಂದ “ಓ ನಾನೊಬ್ಬ ಮಂತ್ರವಾದಿ” ಎಂದು ಹೇಳಿ ಹೊರಟುಹೋದ.

ಒಂದು ದಿನ ಹುಡುಗರೆಲ್ಲಾ ಕಲಿಯುತ್ತಿದ್ದ ಗರಡಿಯ ವಾರ್ಷಿಕೋತ್ಸವದ ದಿನ ಬಂತು. ಆ ಗರಡಿಯ ಮಾಲೀಕನು ಹುಡುಗರಿಗೇ ಅದನ್ನೆಲ್ಲ ಮಾಡಿಕೊಳ್ಳುವುದಕ್ಕೆ ಸ್ವಾತಂತ್ರ್ಯವನ್ನು ಕೊಟ್ಟುಬಿಟ್ಟಿದ್ದನು. ಆ ಗರಡಿಯ ಮುಂದಿನ ರಸ್ತೆಯಲ್ಲಿ ಅವರ ವಾರ್ಷಿಕೋತ್ಸವ ಪ್ರಾರಂಭವಾಯಿತು. ಹುಡುಗರು ತಾವು ಕಲಿತ ಸಾಹಸವನ್ನೆಲ್ಲ ಪ್ರದರ್ಶಿಸುತ್ತಿದ್ದರು. ಕೊನೆಗೆ ಟ್ರಪೀಜ್​ಜಿನ ಮೂಲಕ ಪ್ರದರ್ಶಿಸುತ್ತಿದ್ದಾಗ, ಅದು ಅಜಾಗರೂಕತೆಯಿಂದ ನೋಡಲು ಬಂದಿದ್ದ ಒಬ್ಬ ನಾವಿಕನ ಮೇಲೆ ಬಿತ್ತು. ಅವನು ಪ್ರಜ್ಞೆಯಿಲ್ಲದೇ ಕೆಳಗೆ ಬಿದ್ದನು. ಅವನು ಸತ್ತುಹೋಗಿಬಿಟ್ಟಿರಬಹುದು, ಹತ್ತಿರ ಇದ್ದರೆ ನಾವೆಲ್ಲ ತೊಂದರೆಗೆ ಸಿಕ್ಕಬಹುದೆಂದು ಎಲ್ಲರೂ ಪರಾರಿಯಾದರು. ನರೇಂದ್ರ ಮತ್ತು ಅವನ ಸ್ನೇಹಿತರು ಕೆಲವರು ಮಾತ್ರ ಉಳಿದುಕೊಂಡರು. ತತ್‍ಕ್ಷಣವೇ ನರೇಂದ್ರ ಬಿದ್ದವನನ್ನು ಉಪಚರಿಸಿ ಹತ್ತಿರ ಇದ್ದ ವೈದ್ಯನನ್ನು ಕರೆದುಕೊಂಡು ಬಂದು ಅವನಿಂದ ನಾವಿಕನಿಗೆ ಪ್ರಥಮ ಚಿಕಿತ್ಸೆಯನ್ನು ಮಾಡಿಸಿದ. ಕೆಲವು ದಿನ ಅವನು ವಿರಾಮವನ್ನು ಪಡೆಯಬೇಕೆಂದು ವೈದ್ಯರು ಹೇಳಿದರು. ನರೇಂದ್ರನು ತನಗೆ ತಿಳಿದ ಒಬ್ಬ ಪರಿಚಯಸ್ಥರ ಮನೆಯಲ್ಲಿ ಅಣಿ ಮಾಡಿದನು. ನಾವಿಕನಿಗೆ ಪೂರ್ಣ ಗುಣವಾಗಿ ಹೋಗುವಾಗ, ನರೇಂದ್ರ ತನ್ನ ಸ್ನೇಹಿತರಿಂದ ಸ್ವಲ್ಪ ಹಣವನ್ನು ವಸೂಲಿ ಮಾಡಿ ಅದನ್ನೆಲ್ಲ ಅವನಿಗೆ ಬಹುಮಾನವಾಗಿ ಕೊಟ್ಟು ಕಳುಹಿಸಿದನು.

ವಯಸ್ಸಾಗುತ್ತ ಬಂದಂತೆಲ್ಲ ನರೇಂದ್ರ ಹೆಚ್ಚು ಹೆಚ್ಚು ವಿಚಾರ ಪ್ರಚೋದಿಸುವಂತಹ ಪುಸ್ತಕಗಳು, ವೃತ್ತಪತ್ರಿಕೆಗಳು ಇವುಗಳನ್ನು ಓದಲುಪಕ್ರಮಿಸಿದ. ವಿದ್ವಾಂಸರುಗಳು ಕೊಡುತ್ತಿದ್ದ ಉಪನ್ಯಾಸಗಳಿಗೆ ಹೋಗುತ್ತಿದ್ದ. ಅವರು ಹೇಳಿದುದರ ಸಾರವನ್ನೆಲ್ಲ ಚೆನ್ನಾಗಿ ಗ್ರಹಿಸಿ ತನ್ನದೇ ವಿಮರ್ಶೆಯನ್ನೂ ಕೊಡುತ್ತಿದ್ದ. ಒಂದು ಸಲ ಸಂಗೀತ ಕೇಳುವುದಕ್ಕೆ ಹೋಗಿದ್ದ. ಆ ವಿದ್ವಾಂಸನಾದರೊ ಕಸರತ್ತನ್ನು ತೋರಿಸುವುದರಲ್ಲೆ ಮುಳುಗಿ ಭಾವದ ಕಡೆ ಗಮನವನ್ನೇ ಕೊಟ್ಟಿರಲಿಲ್ಲ. ಅದನ್ನು ಕೇಳಿದ ಮೇಲೆ ನರೇಂದ್ರ “ಸುಮ್ಮನೆ ಕಸರತ್ತಿನಿಂದ ತುಂಬಿದರೆ ಬಂತೆ? ಸಂಗೀತ ಹಾಡಿನಲ್ಲಿರುವ ಭಾವವನ್ನು ಪ್ರತಿಬಿಂಬಿಸಿ ಕೇಳುವವರ ಎದೆಗೆ ತಾಕುವಂತಿರಬೇಕು” ಎಂದನು.

ಕ್ರಿ.ಶ. ೧೮೭೭ರಲ್ಲಿ ವಿಶ್ವನಾಥದತ್ತನಿಗೆ ರಾಯಪುರದಲ್ಲಿ ಕೆಲಸವಾಯಿತು. ಮುಂಚೆ ತಂದೆ ಹೋದನು. ಅನಂತರ ತನ್ನ ಮನೆಯವರನ್ನೆಲ್ಲಾ ಕರೆದುಕೊಂಡು ಬರುವಂತೆ ನರೇಂದ್ರನಿಗೆ ಹೇಳಿಹೋದ. ಕಲ್ಕತ್ತೆಯಿಂದ ಅಲಹಾಬಾದ್, ಜಬ್ಬಲ್‍ಪುರ ಇವುಗಳ ಮೇಲೆ ರಾಯಪುರಕ್ಕೆ ಹೋಗಬೇಕಾಗಿತ್ತು. ಹಲವು ದಿನಗಳು ಗಾಡಿಯಲ್ಲಿ ಪ್ರಯಾಣ ಮಾಡಬೇಕಾಗಿ ಬಂತು. ಗಾಡಿ ಅರಾವಳಿ ಬೆಟ್ಟಗಳ ಪ್ರದೇಶದಲ್ಲಿ ಹೋಗುತ್ತಿರುವಾಗ ನರೇಂದ್ರ ಸುತ್ತಲಿರುವ ನಿಸರ್ಗದ ಸೌಂದರ್ಯಕ್ಕೆ ಮುಗ್ಧನಾಗಿ ಹೋಗುತ್ತಿದ್ದ. ಹಾಗೆ ಒಂದು ದೊಡ್ಡ ಬೆಟ್ಟದ ಸಮೀಪದಲ್ಲಿ ಗಾಡಿ ಹೋಗುತ್ತಿದ್ದಾಗ ಮೇಲಿರುವ ಒಂದು ದೊಡ್ಡ ಬಂಡೆಯ ಅಂಚಿನಲ್ಲಿ ಕಾಡುಜೇನು ಮಾಡಿಕೊಂಡಿದ್ದ ಒಂದು ದೊಡ್ಡ ಗೂಡನ್ನು ನೋಡಿದ. ಅದನ್ನು ನೋಡುತ್ತ ನೋಡುತ್ತ ಇದ್ದಂತೆ ಮನಸ್ಸು ಭಾವಮುಖವಾಯಿತು. ಪ್ರಕೃತಿಯಲ್ಲಿ ಜೇನು ಮುಂತಾದುವು ಹೇಗೆ ಕೆಲಸ ಮಾಡುತ್ತಿವೆ. ತಮಗೆ ಗೊತ್ತಿಲ್ಲದೆ ಎಂತಹ ಒಂದು ಅದ್ಭುತವಾದ ಸಮಾಜವನ್ನು ರಚಿಸಿಕೊಂಡು, ಸಾಮೂಹಿಕ ಜೀವನಕ್ಕೆ ತಮ್ಮ ಸರ್ವಸ್ವವನ್ನೂ ಧಾರೆಯೆರೆದುಕೊಂಡಿರುವುದು ಅನುಭವವಾಯಿತು. ಈ ವಿರಾಟ್ ವಿಶ್ವದ ಹಿಂದೆಲ್ಲ ಕೆಲಸ ಮಾಡುತ್ತಿರುವ ಭಗವಂತನ ನಿಯಮ, ಅವನ ಯೋಜನೆ, ಇವುಗಳೆಲ್ಲ ನರೇಂದ್ರನ ಮನಸ್ಸಿಗೆ ತಾಕಿ ಅವನಿಗೆ ಭೂಮಾನುಭೂತಿಯಾಯಿತು. ಜೀವಿ ಸಂಸ್ಕಾರ ಪಡೆದಿದ್ದರೆ, ಪ್ರಕೃತಿಯಲ್ಲಿರುವ ಘಟನೆಗಳು ದೃಶ್ಯಗಳು ಅವನನ್ನು ಕೆಲವು ವೇಳೆ ತತ್‍ಕ್ಷಣವೇ ಅತೀಂದ್ರಿಯ ಅನುಭವಕ್ಕೆ ಒಯ್ಯುತ್ತವೆ. ಇಂತಹ ಒಂದು ಅನುಭವ ನರೇಂದ್ರನಿಗೆ ಆಗ ಆಯಿತು. ಕೆಲವು ಕಾಲದವರೆಗೆ ಸುತ್ತಲಿರುವುದನ್ನೆಲ್ಲ ಮರೆತನು, ಒಂದು ಅವರ್ಣನೀಯ ಆನಂದದಲ್ಲಿ ತಲ್ಲೀನನಾದನು. ಕ್ರಮೇಣ ಅವನ ಮನಸ್ಸು ವ್ಯವಹಾರ ಭೂಮಿಕೆಗೆ ಬಂದರೂ ಸವಿದ ಅತೀಂದ್ರಿಯ ಆನಂದದ ನೆನಪು ಮಾತ್ರ ಹೋಗಲಿಲ್ಲ.

ರಾಯಪುರವನ್ನು ತಲುಪಿದ ಮೇಲೆ ನರೇಂದ್ರ ಅಲ್ಲಿರುವ ತನಕ ಅಲ್ಲಿರುವ ಶಾಲೆಗೆ ಸೇರಲು ಆಗಲಿಲ್ಲ. ಆದಕಾರಣ ಮನೆಯಲ್ಲಿ ತಂದೆ ತಾನೇ ನರೇಂದ್ರನ ಬುದ್ಧಿಯ ವಿಕಾಸಕ್ಕೆ ಸಹಾಯಕನಾದನು. ಕಲ್ಕತ್ತೆಯಲ್ಲಿದ್ದವರೆಗೆ ನರೇಂದ್ರನಿಗೆ ತನ್ನ ತಂದೆಯ ನಿಕಟ ಪರಿಚಯ ಆಗಿರಲಿಲ್ಲ. ನರೇಂದ್ರ ತಾನು ಯಾವಾಗಲೂ ಶಾಲೆ ಮತ್ತು ಸ್ನೇಹಿತರು ಆಟ ಇವುಗಳಲ್ಲಿ ನಿರತನಾಗಿರುತ್ತಿದ್ದ. ಈಗ ತಂದೆಯ ಬುದ್ಧಿ ಮತ್ತು ಅವನ ಹೃದಯದ ಔದಾರ್ಯ ಅವನಿಗೆ ಅರ್ಥವಾಗುತ್ತ ಬಂತು. ತಂದೆಯೇ ಮಗನಿಗೆ ಹಲವು ವಿಷಯಗಳ ಮೇಲೆ ಮಾತನಾಡುತ್ತಿದ್ದ. ಮನೆಯಲ್ಲೇ ಬೇಕಾದಷ್ಟು ಪುಸ್ತಕಗಳನ್ನು ಸಂಗ್ರಹಿಸಿಟ್ಟಿದ್ದ. ವಿರಾಮವಾದ ಕಾಲದಲ್ಲೆಲ್ಲ ನರೇಂದ್ರ ಓದಿ ವಿಷಯಗಳನ್ನು ಸಂಗ್ರಹಿಸುತ್ತಿದ್ದ. ನರೇಂದ್ರನ ತಂದೆ ರಾಯಪುರದಲ್ಲಿ ಬಹಳ ಪ್ರಖ್ಯಾತನಾಗಿದ್ದ. ಅವನನ್ನು ನೋಡಲು ಅನೇಕ ಸುಪ್ರಸಿದ್ಧ ವ್ಯಕ್ತಿಗಳು ಬರುತ್ತಿದ್ದರು. ಆ ಸಮಯದಲ್ಲಿ ಯಾವುದಾದರೂ ಆಕರ್ಷಕ ವಿಷಯಗಳನ್ನು ಮಾತನಾಡುವಾಗ, ತಂದೆ ನರೇಂದ್ರನನ್ನು ಕರೆದು ಚರ್ಚೆಯಲ್ಲಿ ಭಾಗವಹಿಸುವಂತೆ ಹೇಳುತ್ತಿದ್ದ. ಒಂದು ದಿನ ವಿಶ್ವನಾಥನು ತನ್ನ ಮನೆಯಲ್ಲಿ ಒಬ್ಬ ಪ್ರಖ್ಯಾತ ಸಾಹಿತಿಯೊಂದಿಗೆ ಮಾತನಾಡುತ್ತಿದ್ದ. ಆಗ ಯಾವುದೋ ತುರ್ತುಕರೆ ಬಂದುದರಿಂದ ಹೊರಗೆ ಹೋಗಬೇಕಾಯಿತು. ಒಳಗೆ ಓದುತ್ತ ಕುಳಿತಿದ್ದ ನರೇಂದ್ರನನ್ನು ಕರೆದು “ಇವರೊಡನೆ ನೀನು ಮಾತನಾಡುತ್ತಿರು; ನಾನು ಬೇಗ ಬರುತ್ತೇನೆ” ಎಂದು ಹೇಳಿ ಹೊರಟುಹೋದ. ಹೋದ ಕೆಲವುಗಂಟೆಗಳ ಮೇಲೆ ವಿಶ್ವನಾಥದತ್ತ ಹಿಂತಿರುಗಿ ಬಂದ. ಆಗ ನರೇಂದ್ರನೊಡನೆ ಮಾತನಾಡುತ್ತಿದ್ದವನು ಅವನ ತಂದೆಯ ಎದುರಿಗೆ, ನರೇಂದ್ರನ ಬುದ್ಧಿಶಕ್ತಿಯನ್ನೂ ಆ ವಯಸ್ಸಿಗೇ ಅವನು ಅಷ್ಟೊಂದು ಪುಸ್ತಕಗಳನ್ನು ಓದಿರುವುದನ್ನೂ ಬಹುವಾಗಿ ಪ್ರಶಂಸಿಸಿ ಈ ಹುಡುಗ ಸಾಹಿತ್ಯಕ್ಕೆ ಕೈಹಾಕಿದರೆ ಅಮರವಾದ ಕೀರ್ತಿಯನ್ನು ಗಳಿಸುತ್ತಾನೆ ಎಂದು ಭವಿಷ್ಯವನ್ನು ನುಡಿದನು.

ಎರಡು ವರುಷಗಳಾದ ಮೇಲೆ ೧೮೭೯ರಲ್ಲಿ ವಿಶ್ವನಾಥದತ್ತ ಕಲ್ಕತ್ತೆಗೆ ಹಿಂತಿರುಗಿದ. ನರೇಂದ್ರ ಓದುತ್ತಿದ್ದ ಹಿಂದಿನ ಸ್ಕೂಲಿನಲ್ಲಿ ಮೇಲಿನ ತರಗತಿಗೆ ಅವನನ್ನು ಸೇರಿಸುವ ವಿಷಯದಲ್ಲಿ ಸ್ವಲ್ಪ ತೊಂದರೆಯಾಯಿತು. ಆದರೆ ಶಾಲೆಯ ಉಪಾಧ್ಯಾಯರುಗಳಿಗೆಲ್ಲ ನರೇಂದ್ರ ಬೇಕಾಗಿದ್ದುದರಿಂದ ಅವನನ್ನು ಮೇಲಿನ ತರಗತಿಗೆ ಸೇರಿಸಿದರು. ನರೇಂದ್ರ ಎರಡು ವರುಷಗಳಿಂದ ಪಾಠವನ್ನು ಓದಿಯೇ ಇರಲಿಲ್ಲ. ಈಗ ಹಟಮಾಡಿ ಕುಳಿತ ಆ ಪಾಠಗಳನ್ನೆಲ್ಲ ಓದುವುದಕ್ಕೆ. ಅದ್ಭುತವಾದ ಇಚ್ಛಾಶಕ್ತಿ ಇದ್ದುದರಿಂದ ಕಷ್ಟಪಟ್ಟು ಮೂರು ವರುಷಗಳ ಪಾಠವನ್ನು ಕೆಲವು ತಿಂಗಳಲ್ಲಿ ಓದಲು ಕುಳಿತ. ಪರೀಕ್ಷೆ ಮೂರು ನಾಲ್ಕು ದಿನಗಳಿವೆ ಎನ್ನುವಾಗ ನರೇಂದ್ರನಿಗೆ ಹೊಳೆಯಿತು ತಾನು ರೇಖಾ ಗಣಿತವನ್ನೇ ನೋಡಿಕೊಂಡಿಲ್ಲ ಎಂದು. ಕೆಲವು ದಿನಗಳಲ್ಲಿ ರೇಖಾ ಗಣಿತದ ನಾಲ್ಕು ಭಾಗಗಳನ್ನು ಗ್ರಹಿಸಿದ. ಎನ್‍ಟ್ರೆನ್ಸ್ ಪರೀಕ್ಷೆಯಲ್ಲಿ ಪ್ರಥಮ ದರ್ಜೆಯಲ್ಲಿ ಉತ್ತೀರ್ಣನಾದ. ಆ ಶಾಲೆಯಿಂದ ಅಂತಹ ಪ್ರಶಸ್ತಿಯನ್ನು ಪಡೆದವನು ಇವನೊಬ್ಬನೇ. ವಿಶ್ವನಾಥದತ್ತನಿಗೆ ಸಂತೋಷವಾಗಿ ಮಗನಿಗೆ ಬಹುಮಾನವಾಗಿ ಒಂದು ಗಡಿಯಾರವನ್ನು ತೆಗೆದುಕೊಟ್ಟನು.

ನರೇಂದ್ರ ಎನ್‍ಟ್ರೆನ್ಸ್ ಪರೀಕ್ಷೆಯನ್ನು ಪಾಸುಮಾಡುವ ಹೊತ್ತಿಗೆ ಬೇಕಾದಷ್ಟು ಪುಸ್ತಕಗಳನ್ನು ಓದಿದ್ದ. ಬಂಗಾಳಿ ಮತ್ತು ಇಂಗ್ಲೀಷ್ ಭಾಷೆಯಲ್ಲಿರುವ ಹಲವು ಪ್ರೌಢ ಸಾಹಿತ್ಯಗ್ರಂಥಗಳನ್ನು ಓದಿದ್ದ. ಚರಿತ್ರೆಗೆ ಸಂಬಂಧಪಟ್ಟ ಹಲವು ಗ್ರಂಥಗಳನ್ನು ಓದಿದ್ದ. ಭರತಖಂಡದ ಇತಿಹಾಸಕ್ಕೆ ಸಂಬಂಧಪಟ್ಟ ಮಾರ್ಶ್‍ಮ್ಯಾನ್, ಎಲೆಫೆನ್‍ಸ್ಟೋನ್ ಮುಂತಾದವರು ಬರೆದ ಪ್ರೌಢ ಗ್ರಂಥಗಳನ್ನು ಓದಿದ್ದ. ನರೇಂದ್ರನ ಸ್ವಭಾವ ಯಾವಾಗಲೂ ಪರೀಕ್ಷೆಗೆ ಮುಂಚೆ ಮಾತ್ರ ಪಠ್ಯ ಪುಸ್ತಕಗಳನ್ನು ಓದುವುದು. ಶಾಲೆಯಲ್ಲಿ ಉಪಾಧ್ಯಾಯರು ಹೇಳುತ್ತಿದ್ದುದನ್ನು ಗಮನವಿಟ್ಟು ಕೇಳುತ್ತಿದ್ದ. ಇತರ ಕಾಲದಲ್ಲಿ ತನ್ನ ಜ್ಞಾನದಾಹವನ್ನು ತಣಿಸಿಕೊಳ್ಳುವುದಕ್ಕೆ ಹಲವು ಗ್ರಂಥಗಳನ್ನು ಓದುತ್ತಿದ್ದ. ನರೇಂದ್ರ ಪುಸ್ತಕಗಳನ್ನು ಓದಿ ವಿಷಯ ಸಂಗ್ರಹಿಸುವುದೊಂದು ನಮಗೆ ಅಲೌಕಿಕವಾಗಿ ಕಾಣುವುದು. ನರೇಂದ್ರ ಹೇಳುತ್ತಿದ್ದ, ಒಂದು ಪ್ಯಾರಾಗ್ರಾಫಿನ ಮೊದಲನೆ ಕೆಲವು ಪದಗಳನ್ನು ನೋಡಿದರೆ ಅದರಲ್ಲಿ ಮುಂದೆ ಏನು ಬರುತ್ತದೆ ಎಂಬುದು ಗೊತ್ತಾಗಿಬಿಡುತ್ತದೆ; ಅನಂತರ ಹಾಳೆಯ ಕೆಲವು ಪದಗಳನ್ನು ನೋಡಿದರೆ ಅಲ್ಲಿ ಯಾವ ವಿಷಯಗಳಿವೆ ಎಂಬುದು ಗೊತ್ತಾಗಿಬಿಡುತ್ತದೆ ಎಂದು. ಸಾಧಾರಾಣ ಓದುಗರು ಕೀಟದಂತೆ ಪದದಿಂದ ಪದಕ್ಕೆ ತೆವಳುತ್ತ ಹೋಗಬೇಕು. ನರೇಂದ್ರನ ಬುದ್ಧಿಶಕ್ತಿಯಾದರೋ ವಿದ್ಯುತ್ ವೇಗದಲ್ಲಿ ಸಂಚರಿಸುತ್ತಿತ್ತು. ರಾತ್ರಿಯಲ್ಲಿ ಮಿಂಚಿದರೆ ಕತ್ತಲಲ್ಲಿರುವುದೆಲ್ಲ ಏನು ಎಂಬುದು ಹೇಗೆ ಅರಿವಾಗಿ ಬಿಡುತ್ತದೆಯೋ ಹಾಗೆ ನರೇಂದ್ರನ ಬುದ್ಧಿಶಕ್ತಿ.

