
\chapter*{ಮುನ್ನುಡಿ}

\vskip -6pt

ವಿಜ್ಞಾನ, ಕಲೆ ಮತ್ತು ಸಾಹಿತ್ಯಗಳಲ್ಲಿ ಉತ್ತಮ ಸಾಧನೆಗೈದ ಸ್ತ್ರೀ ಪುರುಷರ ಬಗ್ಗೆ ಜನಸಾಮಾನ್ಯರಿಗೆ ವಿಶೇಷ ಆದರವಿರುತ್ತದೆ.

ಅತ್ಯುತ್ತಮ ಹಸ್ತ ಕೌಶಲ್ಯವಿದ್ದಾಗಲೂ, ಅದನ್ನು ಮೀರಿದ ಸೃಜನಶೀಲ ಸಾಧನೆಯು,\break ಬೆರಗುಗೊಳಿಸುವ ಸಂಗತಿಯಾಗಿರುತ್ತದೆ. ಅತ್ಯುತ್ತಮ ಸಂಶೋಧಕರೂ ಸಹ ತಮ್ಮ ಜೀವಿತಾವಧಿಯಲ್ಲಿ ಕೆಲವೊಮ್ಮೆ ಮಾತ್ರ ಸೃಜನಶೀಲತೆಯನ್ನು ಮೆರೆಯಬಲ್ಲರು. ಆಗ ಅವರಿಗುಂಟಾಗುವ\break ಭಾವತೀವ್ರತೆಯು, ಬೀದಿಗಳಲ್ಲಿ ~“ಯುರೇಕಾ” ಎಂದು ಕೂಗುತ್ತಾ ಓಡಿದ ಅರ್ಕಿಮಿಡೀಸ್‍ನ ಕಥೆಯನ್ನು ನೆನಪಿಸುತ್ತದೆ.

ವಿಜ್ಞಾನ ವಿಷಯಗಳ ಬಗ್ಗೆ ನಾವು ಓದುವ ವರದಿಗಳಲ್ಲಿ ಇದಾವುದೂ ಕಾಣುವುದಿಲ್ಲ. ಒಂದು ಆವಿಷ್ಕಾರದ ಹಿಂದಿನ ನಾಟಕೀಯ ಅಂಶಗಳು\enginline{-}ಪ್ರಾರಂಭಿಕ ಉತ್ತೇಜನ, ವಿಷಯದ ಬೆನ್ನುಹತ್ತಿದ ತೀವ್ರ ನಡೆ, ಹುಸಿ ಜಾಡುಗಳಲ್ಲಿ ಪಯಣ, ಹತಾಷೆಗಳು ಇದರ ಬಳಿಕವಷ್ಟೇ ಬರುವ ಆವಿಷ್ಕಾರದ ಅಮೃತ ಘಳಿಗೆ, ವಿಜ್ಞಾನದ ವಿಷಯಗಳಲ್ಲಿ ಉಂಟಾಗುವ ಭಿನ್ನಮತಗಳೂ ಸಹ ಪತ್ರಿಕೆಗಳಲ್ಲಿ ನೀರಸವಾಗಿ ಪ್ರಕಟಗೊಳ್ಳುತ್ತವೆ. ಮ್ಯಾಗಜೀ಼ನ್ ಸಂಪಾದಕರ ಕೈಚಳಕದಿಂದಲೇ ಇದಾಗುವುದು. ವೈಜ್ಞಾನಿಕ ಸಂಶೋಧನೆಯ ಉನ್ಮಾದ, ವೈಜ್ಞಾನಿಕ ರಂಗದಲ್ಲಿನ ವರ್ಣರಂಜಿತ ವ್ಯಕ್ತಿಗಳು ವ್ಯಕ್ತಿತ್ವಗಳ ಘರ್ಷಣೆಗಳು–ಇದಾವುದೂ ಪ್ರಕಟಿತ ವಾಙ್ಮಯದಲ್ಲಿ ಸಿಗುವುದಿಲ್ಲ. ಇವು ವೈಜ್ಞಾನಿಕ ರಂಗದ ಜಾನಪದವಾಗುತ್ತವೆ. ಪ್ರಸಿದ್ಧ ವಿಜ್ಞಾನಿಗಳ ಆತ್ಮಕಥೆಗಳೂ ಸಹ ಬರೀ ಮೌಲ್ಯಯುತ ದಾಖಲೆಗಳೇ ಆಗಿರುತ್ತವೆ. ಹಾಗಾಗಿ ವಿಜ್ಞಾನಿಗಳ ಸಮಕಾಲೀನರು, ಅದರಲ್ಲೂ ಅವರ ಸನಿಹದವರ ಬರವಣಿಗೆಯು ಮೌಲಿಕವಾಗುತ್ತದೆ.

ಕಳೆದ ಶತಮಾನದಲ್ಲಿ ಆಧುನಿಕ ಭೌತಶಾಸ್ತ್ರದಲ್ಲಿನ ದ್ಯುತಿ ವಿಜ್ಞಾನ ಮತ್ತು ಧ್ವನಿ ವಾಸ್ತುಶಾಸ್ತ್ರಕ್ಕೆ, ಸರ್ ಸಿ.ವಿ. ರಾಮನ್‍ರವರ ಕೊಡುಗೆಯು, “ರಾಮನ್ ಪರಿಣಾಮ”ದಷ್ಟೇ ಮೌಲಿಕವಾದವು. ಅವರ ವೈಜ್ಞಾನಿಕ ಶೋಧಗಳೂ ಅವರು ದೇಶೀಯ ವಿಜ್ಞಾನಿಗಳಿಗೆ ನೀಡಿದ ಮಾರ್ಗದರ್ಶನವೂ ನಿರಂತರ ಪರಿಣಾಮ ಬೀರಿದವು.

ಡಾ.ಎ. ಜಯರಾಮನ್‍ರವರು ಜಗತ್ ವಿಖ್ಯಾತ ವಿಜ್ಞಾನಿಗಳು. ಕಂಡೆನ್ಸಡ್ ಮ್ಯಾಟರ್ ಮತ್ತು ವಸ್ತುಗಳನ್ನು ಒತ್ತಡದ ವೈಪರೀತ್ಯಕ್ಕೆ ಒಳಪಡಿಸಿದಾಗಿನ ಭೌತವಿಜ್ಞಾನಕ್ಕೆ ಅವರ ಕೊಡುಗೆ ಅಪಾರ. ರಾಮನ್ ಇನ್ಸ್ಟಿಟ್ಯೂಟ್ 1949 ರಲ್ಲಿ ಸ್ಥಾಪನೆಗೊಂಡಾಗಿನಿಂದಲೂ, ಅದರ ಜೊತೆಗೇ ಬೆಳೆದವರು ಅವರು.

ಹನ್ನೊಂದು ವರ್ಷಗಳ ಕಾಲ ರಾಮನ್‍ರವರ ಜೊತೆಗೆ ಕೆಲಸಮಾಡಿ, ದಿನನಿತ್ಯವೂ ಅವರ ವೈಜ್ಞಾನಿಕ ಕಾರ್ಯಶೈಲಿ, ಅವರಿಗಿದ್ದ ಪ್ರಚೋದನೆಗಳು ಅವರ ತಾತ್ವಿಕ ಮನೋಭಾವಗಳನ್ನು\break ಗಮನಿಸುತ್ತಿದ್ದರು. ಭಾರತದ ವಿಶಿಷ್ಠ ವಿಜ್ಞಾನಿಯ ಹತ್ತಿರದ ನೋಟವೂ, ಅವರ ಬಗೆಗಿನ ಅಭಿಮಾನವೂ ಜೊತೆಗೆ ವಿವೇಚನೆಯ ತೀರ್ಮಾನಗಳೂ ಈ ವ್ಯಕ್ತಿ ಚರಿತ್ರೆಯ ವಿಶೇಷಗಳಾಗಿವೆ. ಒಬ್ಬ ಶ್ರೇಷ್ಠ ವ್ಯಕ್ತಿಯ ಮಾನವ ಸಹಜ ದೌರ್ಬಲ್ಯಗಳೂ ಪ್ರಾಮಾಣಿಕವಾಗಿ ಬಿಂಬಿತವಾಗಿವೆ. ಹಾಗಾಗಿ ಈ ಬರವಣಿಗೆಯಲ್ಲಿ ಸತ್ಯನಿಷ್ಠತೆಯಿದೆ. ಬರಹಗಾರರಿಗೆ ಬರವಣಿಗೆಯಲ್ಲಿ ಹಿಡಿತವಿದೆ. ಜಯರಾಮನ್‍ರವರು ವಿಜ್ಞಾನ ಚರಿತ್ರೆಯನ್ನು ಈ ಮೂಲಕ ಶ‍್ರೀಮಂತಗೊಳಿಸಿದ್ದಾರೆ.

\begin{flushright}
 \textbf{ಎ.ಕೆ. ರಾಮದಾಸ್}\\
 ಭೌತಶಾಸ್ತ್ರ ಪ್ರಾಧ್ಯಾಪಕರು\\
 ಪರ್ಡ್ಯೂ ವಿಶ್ವವಿದ್ಯಾಲಯ
\end{flushright}

\begin{flushleft}
 \enginline{7} ಫೆಬ್ರವರಿ \enginline{1989}\\
 ವೆಸ್ಟ್ ಲಫಾ಼ಯತ್\\
 ಐಎನ್ \enginline{47907}\\
 ಯು.ಎಸ್.ಎ.
\end{flushleft}

\begin{flushright}
\end{flushright}
‌  
