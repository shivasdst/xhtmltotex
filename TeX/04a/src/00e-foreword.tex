
\chapter*{ಮುನ್ನುಡಿ}

ಇಂಡಿಯನ್ ಅಕಾಡೆಮಿ ಆಫ್ ಸೈನ್ಸಸ್‍ನ (\enginline{1934}) ಸಂಸ್ಥಾಪಕರಾದ ಫ್ರೊಫೆಸರ್ ಸಿ. ವಿ. ರಾಮನ್‍ರವರ ಬಗೆಗಿನ ಈ ಪುಸ್ತಕವು ಈ ವರ್ಷ ಪುನಃ ಪ್ರಕಟಿತಗೊಳ್ಳುತ್ತಿರುವುದು ಸಂತೋಷದ ಸಂಗತಿ. ಈ ಪುಸ್ತಕದ ಮೊದಲ ಆವೃತ್ತಿಯು \enginline{1988}ರಲ್ಲಿ ಅಲೈಡ್ ಈಸ್ಟ್\enginline{-}ವೆಸ್ಟ್ ಪ್ರೆಸ್‍ನವರಿಂದ ಪ್ರಕಟಗೊಂಡಿದ್ದು, ಪ್ರತಿಗಳು ಮುಗಿದಿದ್ದವು. ಕಳೆದ ಮೂರು ದಶಕಗಳಿಂದೀಚೆಗೆ ಉಂಟಾದ ಪುಸ್ತಕ ಪ್ರಕಟಣೆಯ ತಂತ್ರಜ್ಞಾನದ ಲಾಭ ಪಡೆಯಲು ಇಡೀ ಆವೃತ್ತಿಯನ್ನು ಪರಿಷ್ಕರಿಸಲಾಗಿದೆ. ಇಂಡಿಯನ್ ಅಕಾಡೆಮಿ ಆಫ್ ಸೈನ್ಸಸ್‍ನ ಪದಾಧಿಕಾರಿಗಳು ಈ ಪುಸ್ತಕದ ಬಗ್ಗೆ ತೀವ್ರ ಆಸಕ್ತಿ ತೋರಿದ್ದಾರೆ, ಅವರಿಗೆ ಅಭಿನಂದನೆಗಳು. ರಾಮನ್ ರಿಸರ್ಚ್ ಇನ್ಸ್ಟಿಟ್ಯೂಟ್ ಮತ್ತು ಶ‍್ರೀಮತಿ ಡೊಮಿನಿಕ್ ರಾಧಾಕೃಷ್ಣನ್ ಅವರು ಅನೇಕ ಹೊಸ ಫೋಟೋಗಳನ್ನು ನೀಡಿ ಈ ಪುಸ್ತಕದ ಮೌಲ್ಯವನ್ನು ಹೆಚ್ಚಿಸಿದ್ದಾರೆ. ಹಾಗೆಯೇ ಡಾ।। ಮೋಹನ್ ನಾರಾಯಣನ್ ಮತ್ತು ಡಾ।। ವಿನೋದ್ ನಾರಾಯಣರವರು ಮಾಡಿದ ಸಹಾಯಕ್ಕೆ ನಾನು ಆಭಾರಿ.

ಅಕಾಡೆಮಿ ಆಫ್ ಸೈನ್ಸಸ್ ಮೂಲಕ ಈ ಆವೃತ್ತಿಯು ಹೊರಬರುತ್ತಿರುವುದು, ಪುಸ್ತಕದ ಮೂಲ ಆಶಯಕ್ಕೆ ಸಮೀಪವಾಗಿದೆ. ಪ್ರಕೃತಿಯ ಸೂಕ್ಷ್ಮ ವೀಕ್ಷಣೆ ಮತ್ತು ತಾಳ್ಮೆಯಿಂದ ಕೂಡಿದ ಸತ್ಯ ಶೋಧನೆಗಳಿಂದಲೇ ಅತಿ ಶ್ರೇಷ್ಠ ಸಂಶೋಧನೆಗಳು ಸಾಧ್ಯವಾಗುತ್ತವೆಂಬುದು ಈ ಪುಸ್ತಕದ ತಿರುಳು. ಈ ಸಂಗತಿಯು ಭಾರತದ ಹಾಗೂ ವಿದೇಶದ ಯುವ ಜನಾಂಗಕ್ಕೆ ಮನದಟ್ಟಾಗಬಹುದೆಂದು ಆಶಿಸುತ್ತೇನೆ. ಸೂರ್ಯನ ಬೆಳಕು ಮತ್ತು ಅತಿ ಸರಳ ಉಪಕರಣಗಳನ್ನು ಬಳಸಿಕೊಂಡು ರಾಮನ್‍ರವರು ಅನೇಕ ಪ್ರಮುಖ ಸಂಶೋಧನೆಗಳನ್ನು (ರಾಮನ್ ಪರಿಣಾಮವನ್ನೂ ಸಹ) ಸಾಧ್ಯವಾಗಿಸಿದರು. ಈಗ ರಾಮನ್ ರೋಹಿತವನ್ನು \enginline{(Raman spectrum)} ಲೇಸರ್ ಬಳಸಿಕೊಂಡು ಕೆಲವೇ ನಿಮಿಷಗಳಲ್ಲಿ ಪಡೆಯಬಹುದು. ಆದರೆ ರಾಮನ್‍ರವರ ಕಾಲದಲ್ಲಿ ಇದು ದೊಡ್ಡ ಸಾಹಸ ಕಾರ್ಯವೇ ಆಗಿತ್ತು. ಕಲ್ಕತ್ತದಲ್ಲಿದ್ದ ದಿನಗಳಲ್ಲಿ ಬೆಳಕಿನ ಚದರಿಕೆಗೆ ಸಂಬಂಧಿಸಿದ ಪ್ರಯೋಗಗಳನ್ನು ಸೂರ್ಯನ ಬೆಳಕನ್ನು ಬಳಸಿಕೊಂಡೇ ಮಾಡತೊಡಗಿದರು. ಇದರ ಅಂತಿಮ ಸಾಧನೆಯೇ “ರಾಮನ್ ಪರಿಣಾಮ”.

ಸೂರ್ಯನಿಂದ ಬರುವ ಬೆಳಕನ್ನು ಅನೇಕ ಬಣ್ಣಗಳ ಫಿಲ್ಟರುಗಳಿಂದ ಬೇರ್ಪಡಿಸಿ, ದ್ರವಗಳ ಮೂಲಕ ಹಾಯಿಸಿದಾಗ ಅದು ದ್ರವದ ಅಣುಗಳೊಡನೆ ಅಂತರಕ್ರಿಯೆಗೊಂಡು ಅದರ ತರಂಗಗಳ ಆವರ್ತವು ಬದಲಾಗುತ್ತದೆ. ಹೀಗೆ ಬದಲಾದ ಕ್ಷೀಣ ತರಂಗಗಳನ್ನು ಬರಿಗಣ್ಣಿನಿಂದ ರಾಮನ್ ನೋಡಿದರು. ಇದಾದ ಬಳಿಕ, ರೋಹಿತ ಪಟ್ಟಿಯಲ್ಲಿ ಕರಿಯ ಬ್ಯಾಂಡ್‍ನಿಂದ ದೂರವಿದ್ದು ಕ್ಷೀಣವಾಗಿ ಹೊಳೆಯುವ ಬೆಳಕಿನ ರೇಖೆಯನ್ನು ದೃಢಪಡಿಸಿಕೊಳ್ಳಲು, ಪಾದರಸದ ಆರ್ಕ್ ಲ್ಯಾಂಪನ್ನು ಬಳಸಿಕೊಂಡರು. ಆಗ, ಈ ಕ್ಷೀಣ ಬೆಳಕು ಅನೇಕ ಗೆರೆಗಳಿಂದ ಕೂಡಿದ್ದು ಅಣುಗಳ ಕಂಪನದಿಂದ ಚದರಿಕೆಗೆ ಒಳಗಾಗಿರುವ ಅಂಶ ಹೊರಬಿತ್ತು. ಆಗಿನ ಕಾಲದಲ್ಲಿ ಈ ಸಂಶೋಧನೆಗೆ ಬಳಸಿದ ಉಪಕರಣಗಳ ಬೆಲೆ ಕೇವಲ \enginline{500} ರೂಪಾಯಿಗಳೆಂದು ಅವರು ಅನೇಕ ಬಾರಿ ಹೇಳಿದ್ದಾರೆ. ರಾಮನ್‍ರವರು ಬರಿಗಣ್ಣಿನಿಂದ ಗುರುತಿಸಿದ ಬೆಳಕಿನ ಚದರಿಕೆಯೇ ಇಡೀ ಪ್ರಯೋಗಕ್ಕೆ ಮೂಲಾಧಾರ ಎಂಬ ಅಂಶದ ಬಗ್ಗೆ ಅನೇಕರು ಯೋಚಿಸುವುದಿಲ್ಲ.

ನಾನು ರಾಮನ್ ಇನ್ಸ್ಟಿಟ್ಯೂಟ್ ಅನ್ನು ಸೇರಿದಾಗ ಅಲ್ಲಿ ವಿದ್ಯುತ್ ಸಂಪರ್ಕವಿರಲಿಲ್ಲ. ಹದಿನೆಂಟು ತಿಂಗಳ ದೀರ್ಘಕಾಲದ ಬಳಿಕ ವಿದ್ಯುತ್ ಬಂದಿತು. ಈ ಅವಧಿಯಲ್ಲಿ ರಾಮನ್‍ರವರು ತಲೆ ಕೆಡಸಿಕೊಳ್ಳಲಿಲ್ಲ. ಹೊರಗೆ ಬಿಸಿಲಿನಲ್ಲಿ ಕನ್ನಡಿಯನ್ನಿಟ್ಟು ಅದರಿಂದ ಪ್ರತಿಫಲನಗೊಂಡ ಬೆಳಕನ್ನು ಉಪಯೋಗಿಸಿ ಪ್ರಯೋಗ ಮಾಡಲು ಹೇಳಿದರು. ಅವರ ಸಂಗ್ರಹಾಲಯದಲ್ಲಿದ್ದ ಹಲವಾರು ಖನಿಜಗಳು ಹಲವು ಸುಂದರ ಪ್ರಯೋಗಗಳಿಗೆ ಆಕರದ್ರವ್ಯವಾದವು. ಅನೇಕ ಸಂಶೋಧನೆಗಳೂ ಆದವು. ಇನ್ನಾವುದೋ ಪ್ರಯೋಗ ಮಾಡುತ್ತಿದ್ದಾಗ, “ರಾಮನ್ ಪರಿಣಾಮ”ವು ಸೂರ್ಯನ ಬೆಳಕಿನಲ್ಲಿ ಬರಿಗಣ್ಣಿಗೆ ಕಾಣುವ ಸಂಗತಿಯನ್ನು ಅವರೇ ಒಮ್ಮೆ ತೋರಿಸಿದರು. ಅಲ್ಲಿಯವರೆಗೂ, ಈ ವಿಷಯವು ನನ್ನ ಅರಿವಿಗೆ ಬಂದಿರಲಿಲ್ಲ.

ರಾಮನ್ ಅವರು ವರ್ಣರಂಜಿತ ವ್ಯಕ್ತಿತ್ವವುಳ್ಳವರು. ಬಹಳ ಉತ್ತೇಜನ ನೀಡುವ ವ್ಯಕ್ತಿ. ಭಾರತದ ವಿಜ್ಞಾನ ಕ್ಷೇತ್ರಕ್ಕೆ ಅವರು ನೀಡಿದ ಕೊಡುಗೆಯು ಚಿರಂತನವಾಗಿರುತ್ತದೆ. ಒಬ್ಬ ವ್ಯಕ್ತಿಯಾಗಿ ಮತ್ತು ಒಬ್ಬ ವಿಜ್ಞಾನಿಯಾಗಿ ರಾಮನ್‍ರವರ ವಿಭಿನ್ನ ವ್ಯಕ್ತಿತ್ವವನ್ನು ಈ ಕೃತಿಯು ಸಾದರಪಡಿಸುತ್ತದೆಂದು ಭಾವಿಸುತ್ತೇನೆ.

\begin{flushright}
\textbf{ಎ. ಜಯರಾಮನ್}\\ಫೀನಿಕ್ಸ್, ಫೆಬ್ರವರಿ \enginline{2017}
\end{flushright}

