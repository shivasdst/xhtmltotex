
\chapter{ಶ್ರೀ ಶಿವಸಹಸ್ರನಾಮಸ್ತೋತ್ರ ಪ್ರಾರಂಭಃ}

\begin{center}
॥ ಪೂರ್ವಪೀಠಿಕಾ ॥
\end{center}

\begin{center}
॥ ಓಂ ನಮಃ ಶಿವಾಯ ॥
\end{center}

\begin{center}
\textbf{ವಾಸುದೇವ ಉವಾಚ}
\end{center}

\begin{verse}
ತತಃ ಸ ಪ್ರಯತೋ ಭೂತ್ವಾ ಮಮ ತಾತ ಯಧಿಷ್ಠಿರ ।\\ಪ್ರಾಂಜಲಿಃ ಪ್ರಾಹ ವಿಪ್ರರ್ಷಿರ್ನಾಮಸಂಗ್ರಹಮಾದಿತಃ \num{॥ ೧ ॥}
\end{verse}

\begin{center}
\textbf{ಉಪಮನ್ಯುರುವಾಚ}
\end{center}

\begin{verse}
ಬ್ರಹ್ಮಪ್ರೋಕ್ತೈರ್ಪುಷಿಪ್ರೋಕ್ತೈರ್ವೇದವೇದಾಂಗಸಂಭವೈಃ ।\\ಸರ್ವಲೋಕೇಷು ವಿಖ್ಯಾತಂ ಸ್ತುತ್ಯಂ ಸ್ತೋಷ್ಯಾಮಿ ನಾಮಭಿಃ \num{॥ ೨ ॥}
\end{verse}

\begin{verse}
ಮಹದ್ಭಿರ್ವಿಹಿತೈಃ ಸತ್ಯೈಃ ಸಿದ್ಧೈಃ ಸರ್ವಾರ್ಥಸಾಧಕೈಃ ।\\ಪುಷಿಣಾ ತಂಡಿನಾ ಭಕ್ತ್ಯಾ ಕೃತೈರ್ವೇದಕೃತಾತ್ಮನಾ \num{॥ ೩ ॥}
\end{verse}

\begin{verse}
ಯಥೋಕ್ತೈಃ ಸಾಧುಭಿಃ ಖ್ಯಾತೈರ್ಮುನಿಭಿಸ್ತತ್ತ್ವದರ್ಶಿಭಿಃ ।\\ಪ್ರವರಂ ಪ್ರಥಮಂ ಸ್ವರ್ಗ್ಯಂ ಸರ್ವಭೂತಹಿತಂ ಶುಭಮ್ \num{॥ ೪ ॥}
\end{verse}

\begin{verse}
ಶ್ರುತೈಃ ಸರ್ವತ್ರ ಜಗತಿ ಬ್ರಹ್ಮಲೋಕಾವತಾರಿತೈಃ ।\\ಸತ್ಯೈಸ್ತತ್ ಪರಮಂ ಬ್ರಹ್ಮ ಬ್ರಹ್ಮಪ್ರೋಕ್ತಂ ಸನಾತನಮ್ ।\\ವಕ್ಷ್ಯೇ ಯದುಕುಲಶ್ರೇಷ್ಠ ಶೃಣುಷ್ವಾವಹಿತೋ ಮಮ \num{॥ ೫ ॥}
\end{verse}

\begin{verse}
ವರಯೈನಂ ಭವಂ ದೇವಂ ಭಕ್ತಸ್ತ್ವಂ ಪರಮೇಶ್ವರಮ್ ।\\ತೇನ ತೇ ಶ್ರಾವಯಿಷ್ಯಾಮಿ ಯತ್ತದ್ಬ್ರಹ್ಮ ಸನಾತನಮ್ \num{॥ ೬ ॥}
\end{verse}

\begin{verse}
ನ ಶಕ್ಯಂ ವಿಸ್ತರಾತ್ ಕೃತ್ಸ್ನಂ ವಕ್ತುಂ ಸರ್ವಸ್ಯ ಕೇನಚಿತ್ ।\\ಯುಕ್ತೇನಾಪಿ ವಿಭೂತೀನಾಮಪಿ ವರ್ಷಶತೈರಪಿ \num{॥ ೭ ॥}
\end{verse}

\begin{verse}
ಯಸ್ಯಾದಿರ್ಮಧ್ಯಮಂತಂ ಚ ಸುರೈರಪಿ ನ ಗಮ್ಯತೇ ।\\ಕಸ್ತಸ್ಯ ಶಕ್ನುಯಾದ್ವಕ್ತುಂ ಗುಣಾನ್ ಕಾರ್ತೆ್ಸ್ನ್ಯ|ನ ಮಾಧವ \num{॥ ೮ ॥}
\end{verse}

\begin{verse}
ಕಿಂ ತು ದೇವಸ್ಯ ಮಹತಃ ಸಂಕ್ಷಿಪ್ತಾರ್ಥಪದಾಕ್ಷರಮ್ ।\\ಶಕ್ತಿತಶ್ಚರಿತಂ ವಕ್ಷ್ಯೇ ಪ್ರಸಾದಾತ್ತಸ್ಯ ಧೀಮತಃ \num{॥ ೯ ॥}
\end{verse}

\begin{verse}
ಅಪ್ರಾಪ್ಯ ತು ತತೋಽನುಜ್ಞಾಂ ನ ಶಕ್ಯಃ ಸ್ತೋತುಮೀಶ್ವರಃ ।\\ಯದಾ ತೇನಾಭ್ಯನುಜ್ಞಾತಃ ಸ್ತುತೋ ವೈ ಸ ತದಾ ಮಯಾ \num{॥ ೧೦ ॥}
\end{verse}

\begin{verse}
ಅನಾದಿನಿಧನಸ್ಯಾಹಂ ಜಗದ್ಯೋನೇರ್ಮಹಾತ್ಮನಃ ।\\ನಾಮ್ನಾಂ ಕಂಚಿತ್ ಸಮುದ್ದೇಶಂ ವಕ್ಷ್ಯಾಮ್ಯವ್ಯಕ್ತಯೋನಿನಃ \num{॥ ೧೧ ॥}
\end{verse}

\begin{verse}
ವರದಸ್ಯ ವರೇಣ್ಯಸ್ಯ ವಿಶ್ವರೂಪಸ್ಯ ಧೀಮತಃ ।\\ಶ್ರುಣು ನಾಮ್ನಾಂ ಚಯಂ ಕೃಷ್ಣ ಯದುಕ್ತಂ ಪದ್ಮಯೋನಿನಾ \num{॥ ೧೨ ॥}
\end{verse}

\begin{verse}
ದಶ ನಾಮಸಹಸ್ರಾಣಿ ಯಾನ್ಯಾಹ ಪ್ರಪಿತಾಮಹಃ ।\\ತಾನಿ ನಿರ್ಮಥ್ಯ ಮನಸಾ ದಧ್ನೋ ಘೃತಮಿವೋದ್ಧೃತಮ್ \num{॥ ೧೩ ॥}
\end{verse}

\begin{verse}
ಗಿರೇಃ ಸಾರಂ ಯಥಾ ಹೇಮ ಪುಷ್ಪಸಾರಂ ಯಥಾ ಮಧು ।\\ಘೃತಾತ್ಸಾರಂ ಯಥಾ ಮಂಡಸ್ತಥೈತತ್ಸಾರಮುದ್ಧೃತಮ್ \num{॥ ೧೪ ॥}
\end{verse}

\begin{verse}
ಸರ್ವಪಾಪಾಪಹಮಿದಂ ಚತುರ್ವೇದಸಮನ್ವಿತಮ್ ।\\ಪ್ರಯತ್ನೇನಾಧಿಗಂತವ್ಯಂ ಧಾರ್ಯಂ ಚ ಪ್ರಯತಾತ್ಮನಾ \num{॥ ೧೫ ॥}
\end{verse}

\begin{verse}
ಮಾಂಗಲ್ಯಂ ಪೌಷ್ಟಿಕಂ ಚೈವ ರಕ್ಷೋಘ್ನಂ ಪಾವನಂ ಮಹತ್ \num{॥ ೧೬ ॥}
\end{verse}

\begin{verse}
ಇದಂ ಭಕ್ತಾಯ ದಾತವ್ಯಂ ಶ್ರದ್ದಧಾನಾಸ್ತಿಕಾಯ ಚ ।\\ನಾಶ್ರದ್ದಧಾನರೂಪಾಯ ನಾಸ್ತಿಕಾಯಾಜಿತಾತ್ಮನೇ \num{॥ ೧೭ ॥}
\end{verse}

\begin{verse}
ಯಶ್ಚಾಭ್ಯಸೂಯತೇ ದೇವಂ ಕಾರಣಾತ್ಮಾನಮೀಶ್ವರಮ್ ।\\ಸ ಕೃಷ್ಣ ನರಕಂ ಯಾತಿ ಸಹಪೂರ್ವೈಃ ಸಹಾತ್ಮಜೈಃ \num{॥ ೧೮ ॥}
\end{verse}

\begin{verse}
ಇದಂ ಧ್ಯಾನಮಿದಂ ಯೋಗಮಿದಂ ಧ್ಯೇಯಮನುತ್ತಮಮ್ ।\\ಇದಂ ಜಪ್ಯಮಿದಂ ಜ್ಞಾನಂ ರಹಸ್ಯಮಿದಮುತ್ತಮಮ್ \num{॥ ೧೯ ॥}
\end{verse}

\begin{verse}
ಯಂ ಜ್ಞಾತ್ವಾ ಅಂತಕಾಲೇಽಪಿ ಗಚ್ಛೇತ ಪರಮಾಂ ಗತಿಮ್ ।\\ಪವಿತ್ರಂ ಮಂಗಲಂ ಮೇಧ್ಯಂ ಕಲ್ಯಾಣಮಿದಮುತ್ತಮಮ್ \num{॥ ೨೦ ॥}
\end{verse}

\begin{verse}
ಇದಂ ಬ್ರಹ್ಮಾ ಪುರಾ ಕೃತ್ವಾ ಸರ್ವಲೋಕಪಿತಾಮಹಃ ।\\ಸರ್ವಸ್ತವಾನಾಂ ರಾಜತ್ವೇ ದಿವ್ಯಾನಾಂ ಸಮಕಲ್ಪಯತ್ \num{॥ ೨೧ ॥}
\end{verse}

\begin{verse}
ತದಾಪ್ರಭೃತಿ ಚೈವಾಯಮೀಶ್ವರಸ್ಯ ಮಹಾತ್ಮನಃ ।\\ಸ್ತವರಾಜ ಇತಿ ಖ್ಯಾತೋ ಜಗತ್ಯಮರಪೂಜಿತಃ \num{॥ ೨೨ ॥}
\end{verse}

\begin{verse}
ಬ್ರಹ್ಮಲೋಕಾದಯಂ ಸ್ವರ್ಗೇ ಸ್ತವರಾಜೋಽವತಾರಿತಃ ।\\ಯತಸ್ತಂಡಿಃ ಪುರಾ ಪ್ರಾಪ ತೇನ ತಂಡಿಕೃತೋಽಭವತ್ \num{॥ ೨೩ ॥}
\end{verse}

\begin{verse}
ಸ್ವರ್ಗಾಚ್ಚೈವಾತ್ರ ಭೂರ್ಲೋಕಂ ತಂಡಿನಾ ಹ್ಯವತಾರಿತಃ ।\\ಸರ್ವಮಂಗಲಮಾಂಗಲ್ಯಂ ಸರ್ವಪಾಪಪ್ರಣಾಶನಮ್ ।\\ನಿಗದಿಷ್ಯೇ ಮಹಾಬಾಹೋ ಸ್ತವಾನಾಮುತ್ತಮಂ ಸ್ತವಮ್ \num{॥ ೨೪ ॥}
\end{verse}

\begin{verse}
ಬ್ರಹ್ಮಣಾಮಪಿ ಯದ್ಬ್ರಹ್ಮ ಪರಾಣಾಮಪಿ ಯತ್ಪರಮ್ \num{॥ ೨೫ ॥}
\end{verse}

\begin{verse}
ತೇಜಸಾಮಪಿ ಯತ್ತೇಜಸ್ತಪಸಾಮಪಿ ಯತ್ತಪಃ ।\\ಶಾಂತೀನಾಮಪಿ ಯಾ ಶಾಂತಿರ್ದ್ಯುತೀನಾಮಪಿ ಯಾ ದ್ಯುತಿಃ \num{॥ ೨೬ ॥}
\end{verse}

\begin{verse}
ದಾಂತಾನಾಮಪಿ ಯೋ ದಾಂತೋ ಧೀಮತಾಮಪಿ ಯಾ ಚ ಧೀಃ ।\\ದೇವಾನಾಮಪಿ ಯೋ ದೇವ ಪುಷೀಣಾಮಪಿ ಯಸ್ತ್ವ ೃಷಿಃ \num{॥ ೨೭ ॥}
\end{verse}

\begin{verse}
ಯಜ್ಞಾನಾಮಪಿ ಯೋ ಯಜ್ಞಃ ಶಿವಾನಾಮಪಿ ಯಃ ಶಿವಃ ।\\ರುದ್ರಾಣಾಮಪಿ ಯೋ ರುದ್ರಃ ಪ್ರಭಾ ಪ್ರಭವತಾಮಪಿ \num{॥ ೨೮ ॥}
\end{verse}

\begin{verse}
ಯೋಗಿನಾಮಪಿ ಯೋ ಯೋಗೀ ಕಾರಣಾನಾಂ ಚ ಕಾರಣಮ್ ।\\ಯತೋ ಲೋಕಾಃ ಸಂಭವಂತಿ ನ ಭವಂತಿ ಯತಃ ಪುನಃ \num{॥ ೨೯ ॥}
\end{verse}

\begin{verse}
ಸರ್ವಭೂತಾತ್ಮಭೂತಸ್ಯ ಹರಸ್ಯಾಮಿತತೇಜಸಃ ।\\ಅಷ್ಟೋತ್ತರಸಹಸ್ರಂ ತು ನಾಮ್ನಾಂ ಶರ್ವಸ್ಯ ಮೇ ಶ್ರುಣು ।\\ಯಚ್ಛ್ರುತ್ವಾ ಮನುಜವ್ಯಾಘ್ರ ಸರ್ವಾನ್ಕಾಮಾನವಾಪ್ಸ್ಯಸಿ \num{॥ ೩೦ ॥}
\end{verse}

\begin{center}
\textbf{॥ ಪುಷ್ಯಾದಿನ್ಯಾಸಃ ॥}
\end{center}

\begin{verse}
ಅಸ್ಯ ಶ್ರೀಸದಾಶಿವಸಹಸ್ರನಾಮಸ್ತೋತ್ರಮಂತ್ರಸ್ಯ ನಾರಾಯಣ ಪುಷಿಃ, ಶ್ರೀ ಸದಾಶಿವೋ ದೇವತಾ, ಅನುಷ್ಟುಪ್ ಛಂದಃ, ಸದಾಶಿವೋ ಬೀಜಮ್, ಗೌರೀ ಶಕ್ತಿಃ ಶ್ರೀ ಸದಾಶಿವಪ್ರೀತ್ಯರ್ಥಂ ಜಪೇ ವಿನಿಯೋಗಃ॥
\end{verse}

\begin{center}
\textbf{॥ ಧ್ಯಾನಮ್ ॥}
\end{center}

\begin{verse}
ಧ್ಯಾಯೇನ್ನಿತ್ಯಂ ಮಹೇಶಂ ರಜತಗಿರಿನಿಭಂ ಚಾರುಚಂದ್ರಾವತಂಸಂ\\ರತ್ನಾಕಲ್ಪೋಜ್ಜ್ವಲಾಂಗಂ ಪರಶುಮೃಗವರಾಭೀತಿಹಸ್ತಂ ಪ್ರಸನ್ನಮ್ ।\\ಪದ್ಮಾಸೀನಂ ಸಮಂತಾತ್ ಸ್ತುತಮಮರಗಣೈರ್ವ್ಯಾಘ್ರಕೃತ್ತಿಂವಸಾನಂ\\ವಿಶ್ವಾದ್ಯಂ ವಿಶ್ವವಂದ್ಯಂ ನಿಖಿಲಭಯಹರಂ ಪಂಚವಕ್ತ್ರಂ ತ್ರಿನೇತ್ರಮ್​॥
\end{verse}

