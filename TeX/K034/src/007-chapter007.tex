
\chapter{ಸ್ತ್ರೀಯರ ಉದ್ಧಾರ}

\textbf{ಹಿಂದೂ ಸ್ತ್ರೀಯರಆದರ್ಶ}: ಸೀತೆಯು ನಿಜವಾದ ಹಿಂದೂ ಸ್ತ್ರೀಯರ ಆದರ್ಶ. ಸೀತೆಯ ಜೀವನದಿಂದಲೇ ಪ್ರಖ್ಯಾತರಾದ ನಾರೀ ಶಿರೋಮಣಿಗಳ ಆದರ್ಶವೆಲ್ಲ ಬಂದಿದೆ. ಅವಳು ಸಾವಿರಾರು ವರುಷಗಳಿಂದಲೂ ಭರತಖಂಡದ ಸ್ತ್ರೀ ಪುರುಷ ಬಾಲಕರ ಗೌರವಕ್ಕೆ ಪಾತ್ರಳಾಗಿರುವಳು. ಅವಳು ಆ ಸ್ಥಾನದಲ್ಲಿ ಯಾವಾಗಲೂ ಇರುವಳು. ಈ ಮಹಾಮಹಿಮಳಾದ ಸೀತೆ ಪವಿತ್ರತೆಗಿಂತ ಪವಿತ್ರಳು. ತಾಳ್ಮೆ ಮತ್ತು ದುಃಖವೇ ಮೈವೆತ್ತಂತೆ ಇದ್ದಳು. ಅವಳು ಗೊಣಗಾಡದೆ ತನಗೆ ಬಂದು ದನ್ನೆಲ್ಲ ಅನುಭವಿಸಿದಳು. ಅವಳು ಪರಮ ಪವಿತ್ರಳು ಮತ್ತು ಪತಿವ್ರತಳು ಆಗಿ ದ್ದಳು. ಅವಳು ನಮ್ಮ ಜನರ ಆದರ್ಶ, ನಮ್ಮ ದೇವತೆಗಳ ಆದರ್ಶ. ಆ ನಾರಿ ಶಿರೋಮಣಿಯೇ ಯಾವಾಗಲೂ ನಮ್ಮ ದೇಶದ ಭಾಗ್ಯದೇವತೆಯಾಗಿ ಇರುವಳು. ನಮ್ಮ ಪುರಾಣಗಳೆಲ್ಲ ಇಲ್ಲದೇಹೋಗಬಹುದು, ನಮ್ಮ ವೇದಗಳೇ ಮಾಯವಾಗ ಬಹುದು, ಎಂದೆಂದಿಗೂ ನಮ್ಮ ಸಂಸ್ಕೃತ ಭಾಷೆಯೇ ನಾಶವಾಗಬಹುದು. ಆದರೆ ಎಲ್ಲಿಯವರೆಗೆ ಐದು ಹಿಂದೂಗಳಾದರೂ ಇರುವರೋ, ಅವರಿಗೆ ಏನೂ ತಿಳಿಯದೆ ತಮ್ಮ ಗ್ರಾಮ್ಯಭಾಷೆಯಲ್ಲಿ ಮಾತನಾಡುತ್ತಿದ್ದರೂ ಸೀತೆಯ ಕಥೆ ಮಾತ್ರ ಉಳಿಯು ವುದು. ಇದನ್ನು ಗಮನಿಸಿ. ಅವಳು ನಮ್ಮ ಜನಾಂಗದ ಆದರ್ಶದ ಅಂತರಾಳವನ್ನು ಪ್ರವೇಶಿಸಿರುವಳು! ಭರತಖಂಡದ ಪ್ರತಿಯೊಬಪ್ ನರನಾರಿಯರ ರಕ್ತದಲ್ಲಿಯೂ ಅವಳು ಹರಿಯುತ್ತಿರುವಳು. ನಾವೆಲ್ಲ ಸೀತೆಯ ಮಕ್ಕಳು.

ಪಾಶ್ಚಾತ್ಯ ಸ್ತ್ರೀ ಆದರ್ಶಕ್ಕಿಂತ ಭರತಖಂಡದ ನಾರಿಯರ ಆದರ್ಶ ಎಷ್ಟು ವ್ಯತ್ಯಾಸವಾಗಿರುವುದು! ಪಾಶ್ಚಾತ್ಯರು ನಿಮ್ಮ ಶಕ್ತಿಯನ್ನು ಏನನ್ನಾದರೂ ಸಾಹಸ ವನ್ನು ಮಾಡಿ ತೋರಿ ಎನ್ನುವರು. ಇಂಡಿಯಾದೇಶದಲ್ಲಿ ಜೀವನದಲ್ಲಿ ಅನುಭವಿಸಿ ತೋರಿ ಎನ್ನುವರು. ಮನುಷ್ಯ ಎಷ್ಟೊಂದನ್ನು ಪಡೆಯಬಹುದು ಎಂಬುದನ್ನು ಪಾಶ್ಚಾತ್ಯ ಸಾಧಿಸಿದೆ, ಭರತಖಂಡವಾದರೊ ಒಬ್ಬ ಬದುಕುವುದಕ್ಕೆ ಎಷ್ಟು ಕನಿಷ್ಟ ಸಾಕಾಗುವುದು ಎಂಬುದನ್ನು ಸಾಧಿಸಿದೆ. ಇವೆರಡೂ ಒಂದು ಅತಿರೇಕ. ಸೀತೆ ನಮ್ಮ ಭರತಖಂಡಕ್ಕೆ ಸೇರಿದವಳು, ಅವಳು ಭರತಖಂಡದ ನಾರೀ ಕುಲಕ್ಕೆ ಆದರ್ಶ. ಅವಳೇನಾದರೂ ಇದ್ದಳೆ, ಅವಳಿಗೆ ಸಂಬಂಧಪಟ್ಟ ಕಥೆಗಳು ನಿಜವಾಗಿ ನಡೆದದ್ದೆ ಅಥವಾ ಬರೀ ಕಲ್ಪನೆಯೊ ಇದನ್ನು ನಾವು ಗಮನಿಸಬೇಕಾಗಿಲ್ಲ. ಆದರೆ ಸೀತೆಯ ಆದರ್ಶ ನಮ್ಮ ಜನಾಂಗದಲ್ಲಿ ಬೇರೂರಿದೆ. ನಮ್ಮ ಜನಾಂಗದ ಜೀವನವನ್ನೆಲ್ಲ ವ್ಯಾಪಿಸಿ ನಮ್ಮಲ್ಲೆಲ್ಲ ಓತಪ್ರೋತವಾಗಿ, ನಮ್ಮ ರಕ್ತದ ಧಮನಿಧಮನಿಯಲ್ಲಿ ಸೀತಾದೇವಿಯ ಆದರ್ಶದಂತೆ ಅನುರಣಿತವಾಗುತ್ತಿರುವ ಆದರ್ಶ ಮತ್ತೊಂದು ಇಲ್ಲ. ಸೀತೆ ಪಾತಿವ್ರತ್ಯವೇ ರೂಪುವೆತ್ತಂತೆ ಇದ್ದಳು. ತನ್ನ ಗಂಡನಲ್ಲದ ಯಾವ ಪರಪುರುಷನ ದೇಹವನ್ನೂ ಮುಟ್ಟಿದವಳಲ್ಲ. ಭರತಖಂಡದಲ್ಲಿ ಪ್ರತಿಯೊಂದು ಒಳ್ಳೆಯದಕ್ಕೂ ಪವಿತ್ರವಾಗಿರುವುದಕ್ಕೂ ಸೀತೆ ಎಂದು ಹೆಸರು. ಬ್ರಾಹ್ಮಣ ಮತ್ತೊಬ್ಬ ಸ್ತ್ರೀಯನ್ನು ಹರಸಬೇಕಾದರೆ ಸೀತೆಯಂತಾಗು ಎನ್ನುವನು. ಒಂದು ಹುಡುಗಿಯನ್ನು ಹರಸಬೇಕಾದರೆ ಸೀತೆಯಂತಾಗು ಎನ್ನುವನು. ಅವರೆಲ್ಲ ಸೀತೆಯ ಮಕ್ಕಳು. ಸೀತೆ ಆಜನ್ಮದುಃಖಿನಿ, ಸಹಿಷ್ಣುತಾಮೂರ್ತಿ, ಪವಿತ್ರಾತ್ಮಳು. ಇಲ್ಲಿಯ ಸ್ತ್ರೀಯರು ಪತಿವ್ರತಾಶಿರೋಮಣಿ ಸೀತೆಯಂತಾಗಲು ಪ್ರಯತ್ನಿಸುತ್ತಿರುವರು. ಅವಳು ಜೀವನದಲ್ಲಿ ಅಷ್ಟೊಂದು ಕಷ್ಟವನ್ನು ಅನುಭವಿಸಿದರೂ ರಾಮನ ವಿಷಯ ವಾಗಿ ಒಂದು ಬಿರುನುಡಿಯೂ ಇಲ್ಲ. ಅವಳು ತನ್ನ ಪಾಲಿಗೆ ಬಂದ ಕಷ್ಟವನ್ನು ಕರ್ತವ್ಯದಂತೆ ಸ್ವೀಕರಿಸಿ ಅದನ್ನು ಅನುಭವಿಸುವಳು. ಅವಳನ್ನು ಕಾಡಿಗೆ ಕಳುಹಿಸಿದ ಅನ್ಯಾಯವಾದ ಘಟನೆಯನ್ನು ಕುರಿತು ಯೋಚಿಸಿ ನೋಡಿ. ಆದರೆ ಸೀತೆ ಯಾವು ದನ್ನೂ ದೂರುವುದಿಲ್ಲ. ಇದು ಪುನಃ ಭಾರತೀಯರ ಆದರ್ಶ. ಸೀತೆ ಸ್ವಭಾವತಃ ನಿಜವಾದ ಭಾರತೀಯಳು. ಅವಳು ಅಪಕಾರಕ್ಕೆ ಅಪಕಾರವನ್ನು ಮಾಡಲಿಲ್ಲ.

ಯಾವ ಜನಾಂಗ ಸೀತೆಯನ್ನು ಸೃಷ್ಟಿಮಾಡಿತೊ, ಅದು ಕೇವಲ ಅಂತಹ ವ್ಯಕ್ತಿ ಯನ್ನು ಕಲ್ಪಿಸಿಕೊಂಡಿದ್ದರೂ, ಪ್ರಪಂಚದ ನಾರಿಕುಲಕ್ಕೆ ಮತ್ತಾರೂ ತೋರದ ಗೌರವವನ್ನು ತೋರಿರುವುದು. ಪಾಶ್ಚಾತ್ಯ ಸ್ತ್ರೀಯರಿಗೆ ಕಾನೂನಿನ ಪ್ರಕಾರ ಬೇಕಾ ದಷ್ಟು ಅಡ್ಡಿ ಆತಂಕಗಳಿವೆ. ನಮಗೆ ಇವುಗಳು ಯಾವುವೂ ಇಲ್ಲ. ನಮ್ಮಲ್ಲಿಯೂ ಲೋಪದೋಷಗಳಿವೆ, ವಿನಾಯಿತಿಗಳಿವೆ, ಅದರಂತೆಯೇ ಪಾಶ್ಚಾತ್ಯದಲ್ಲಿ ಕೂಡ. ಪ್ರಪಂಚದಲ್ಲೆಲ್ಲ ಮಾನವನು ದಯೆ ಮತ್ತು ಪುಜುತ್ವವನ್ನು ವ್ಯಕ್ತಪಡಿಸಲು ಪ್ರಯತ್ನಿಸುತ್ತಿರುವನು ಎಂಬುದನ್ನು ಮರೆಯಕೂಡದು. ನಮ್ಮ ಜನಾಂಗದ ಆಚಾರ ವ್ಯವಹಾರಗಳು ಅದನ್ನು ವ್ಯಕ್ತಪಡಿಸುವ ಅತಿ ಸಮೀಪದ ಮಧ್ಯವರ್ತಿಗಳು ಅಷ್ಟೆ. ಕುಟುಂಬಕ್ಕೆ ಸಂಬಂಧಪಟ್ಟ ಗುಣಗಳಲ್ಲಿ ಭಾರತೀಯನ ಮಾರ್ಗ ಇತರರಿ ಗಿಂತ ಮೇಲು ಎಂದು ಹೇಳುವುದಕ್ಕೆ ಯಾವ ಅನುಮಾನವನ್ನು ಪಡುವುದಿಲ್ಲ.

ಸೀತಾ ಸಾವಿತ್ರಿಯರ ದೇಶವಾದ ಈ ಪವಿತ್ರ ಭರತಖಂಡದ ಸ್ತ್ರೀಯರಲ್ಲಿ ಮಾತ್ರ ಬೇರೆಲ್ಲೂ ಇಲ್ಲದ ಸೇವಾ ಮನೋಭಾವ ಪ್ರೀತಿ ದಯೆ ತೃಪ್ತಿ ಗೌರವ ಇವು ಗಳನ್ನು ನೋಡುತ್ತೇವೆ. ಪಾಶ್ಚಾತ್ಯ ದೇಶದಲ್ಲಿ ಸ್ತ್ರೀಯರು ಸ್ತ್ರೀಯರಂತೆ ಕಾಣುತ್ತಿರ ಲಿಲ್ಲ. ಅವರು ಬರೀ ಪುರುಷನಂತೆ ಇರುವರು. ಗಾಡಿ ಹೊಡೆಯುವರು, ಆಫೀಸಿ ನಲ್ಲಿ ಕೆಲಸ ಮಾಡುವರು, ಶಾಲೆಗೆ ಹೋಗುವರು, ಇನ್ನು ಹಲವು ಜೀವನೋ ಪಾಯದ ಕೆಲಸಗಳನ್ನು ಮಾಡುವರು. ಭರತಖಂಡದ ಸ್ತ್ರೀಯರ ಗಾಂಭೀರ್ಯ ಮತ್ತು ಲಜ್ಜೆ ಇವನ್ನು ನೋಡಿದಾಗ ನನ್ನ ಕಣ್ಣಿಗೆ ಸಮಾಧಾನವನ್ನು ತರುವುದು.

ಸ್ತ್ರೀಯರ ವಿಷಯದಲ್ಲಿ ನಾವು ಸುಧಾರಣೆಗಳನ್ನು ಮಾಡುವಾಗ, ಸೀತೆಯ ಆದರ್ಶದಿಂದ ಕದಲಿಸಲು ನಾವು ಮಾಡುವ ಪ್ರಯತ್ನವೆಲ್ಲ ನಿಷ್ಪ್ರಯೋಜನವಾಗು ವುದು. ನಾವು ದಿನ ನಿತ್ಯವೂ ಇದನ್ನು ನೋಡುತ್ತಿರುವೆವು. ಭರತಖಂಡದ ನಾರಿ ಯರು ಸೀತೆಯ ಮೇಲ್ಪಂಕ್ತಿಯಲ್ಲಿ ಬೆಳೆಯಬೇಕು. ಇದೊಂದೇ ನಮಗೆ ಯೋಗ್ಯ ವಾದ ಮಾರ್ಗ.

\textbf{ಪೂರ್ವ ಮತ್ತು ಆಧುನಿಕ ಕಾಲದಲ್ಲಿ ಅವರ ಸ್ಥಾನ} ಸುಮ್ಮನೆ ಪಾಶ್ಚಾತ್ಯರು ನಮ್ಮನ್ನು ದೂರುವುದನ್ನು ಅಂಧರಾಗಿ ಅನುಕರಿಸಕೂಡದು. ನಮ್ಮ ಸ್ತ್ರೀಯರಿಗೆ ಯಾವಾಗಲೂ ನಾವು ಏನೂ ಸಮಾನತೆಯನ್ನು ಕೊಟ್ಟಿಲ್ಲ ಎಂದು ಭಾವಿಸಬಾರದು. ಹಲವು ಶತಮಾನಗಳಿಂದ ಸ್ತ್ರೀಯರನ್ನು ನಾವು ರಕ್ಷಿಸಬೇಕಾದ ಪ್ರಸಂಗ ಬಂತು. ಆದಕಾರಣ ನಾವು ಅವರನ್ನು ಆಶ್ರಿತರನ್ನಾಗಿ ಮಾಡಬೇಕಾಯಿತೆ ಹೊರತು ಅವರಿಗೆ ನಮ್ಮಷ್ಟು ಯೋಗ್ಯತೆ ಇಲ್ಲ ಎಂದಲ್ಲ. ಅವಳೇನು ನಮ್ಮ ಪುರುಷನಿಗಿಂತ ಕಡಿಮೆ ಯಲ್ಲ. ನಮ್ಮ ಹಿಂದಿನ ಆಚಾರ ವ್ಯವಹಾರಗಳನ್ನು ನೋಡಿದರೆ ಇದು ಗೊತ್ತಾಗು ವುದು. ನಮ್ಮ ಪೂರ್ವ ಕಾಲದ ಪುಷ್ಯಾಶ್ರಮದ ವಿದ್ಯಾಭ್ಯಾಸ ಕಾಲದಲ್ಲಿ ಬಾಲಕ ಬಾಲಕಿಯರಿಗೆ ಇದ್ದ ಸ್ವಾತಂತ್ರ್ಯವನ್ನು ಕುರಿತು ಯೋಚಿಸಿ ನೋಡಿ. ನಮ್ಮ ಸಂಸ್ಕೃತ ನಾಟಕಗಳನ್ನು ಓದಿ, ಶಂಕುತಳೋಪಾಖ್ಯಾನವನ್ನು ಓದಿ. ಟೆನಿಸನ್ ಮಹಾಕವಿ ಬರೆದ \eng{Princess} ಎಂಬ ಕವಿತೆ ನಮಗೆ ಏನಾದರೂ ಕೊಡಬಲ್ಲದೆ ನೋಡಿ.

ಮಲಬಾರಿನಲ್ಲಿ ಸ್ತ್ರೀಯರೇ ಪ್ರತಿಯೊಂದು ಕ್ಷೇತ್ರದಲ್ಲಿಯೂ ಮುಂದಾಳು ಗಳು. ಎಲ್ಲಾ ಕಡೆಯಲ್ಲಿಯೂ ಜನರು ಅಷ್ಟು ಶುಭ್ರವಾಗಿರುವರು. ವಿದ್ಯಾಭ್ಯಾಸಕ್ಕೆ ಜನರಲ್ಲಿ ಅಷ್ಟೊಂದು ಕುತೂಹಲ. ನಾನು ಆ ದೇಶದಲ್ಲಿದ್ದಾಗ ಸಂಸ್ಕೃತ ಭಾಷೆ ಯಲ್ಲಿ ಮಾತನಾಡಬಲ್ಲ ಹಲವು ಸ್ತ್ರೀಯರನ್ನು ನೋಡಿದೆ. ಭರತಖಂಡದ ಇತರ ಕಡೆಗಳಲ್ಲಿ ಲಕ್ಷಕ್ಕೆ ಒಬ್ಬನೂ ಹೀಗೆ ಮಾತಾಡಲಾರ. ಯಜಮಾನ್ಯ ನಮ್ಮನ್ನು ಮೇಲಕ್ಕೆ ಎತ್ತುವುದು, ಗುಲಾಮಗಿರಿ ನಮ್ಮನ್ನು ಕೆಳಕ್ಕೆ ತರುವುದು. ಮಲಬಾರನ್ನು ಪೋರ್ಚುಗೀಸರಾಗಲಿ ಮುಸಲ್ಮಾನರಾಗಲಿ ಗೆಲ್ಲಲಿಲ್ಲ. ದ್ರಾವಿಡರು ಮಧ್ಯ ಏಷ್ಯಾ ಖಂಡಕ್ಕೆ ಸೇರಿದ ಆರ್ಯೇತರರು. ಅವರು ನಮ್ಮ ಭರತಖಂಡಕ್ಕೆ ಆರ್ಯರಿಗಿಂತ ಮುಂಚೆ ಬಂದರು. ದೇಶದಲ್ಲಿರುವ ದ್ರಾವಿಡರು ನಾಗರಿಕತೆಯಲ್ಲಿ ಬಹಳ ಮುಂದುವರಿದವರು. ಅವರಲ್ಲಿ ಸ್ತ್ರೀಯರು ಪುರುಷರಿಗಿಂತ ಮೇಲಿದ್ದರು.

ಆರ್ಯ ಮತ್ತು ಸೆಮಿಟೆಕ್ ಜನಾಂಗಗಳ ಸ್ತ್ರೀಯರ ಆದರ್ಶ ಬೇರೆ ಬೇರೆ, ಪರಸ್ಪರ ವಿರೋಧವಾಗಿಯೇ ಇವೆ (ಸೆಮಿಟೆಕ್ ಜನಾಂಗಗಳು ಎಂದರೆ, ಯಹೂದ್ಯ ಅರಬ್ಬಿ ಮೊದಲಾದವರು). ಸೆಮಿಟೆಕ್ ಜನಾಂಗಗಳಲ್ಲಿ ಸ್ತ್ರೀ ಭಕ್ತಿಜೀವನಕ್ಕೆ ಆತಂಕ ಪ್ರಾಯಳು. ಅವಳು ಯಾವ ಧಾರ್ಮಿಕ ಪೂಜೆ ವ್ರತಾದಿಗಳನ್ನು ಮಾಡುವಂತಿಲ್ಲ. ಅವಳು ಊಟಕ್ಕೆ ಒಂದು ಹಕ್ಕಿಯನ್ನೂ ಕೊಲ್ಲಬಾರದು. ಆದರೆ ಆರ್ಯರಲ್ಲಿ ಆದರೋ ಗಂಡನು ಹೆಂಡತಿಯ ಸಹಾಯವಿಲ್ಲದೆ ಯಾವ ಧಾರ್ಮಿಕ ಕೆಲಸಗಳನ್ನೂ ಮಾಡುವಂತಿಲ್ಲ.

ಪರಮಸತ್ಯವಾದ ಪರಬ್ರಹ್ಮನ ದೃಷ್ಟಿಯಿಂದ ಲಿಂಗಭೇದಗಳಿಗೆ ಅವಕಾಶವೇ ಇಲ್ಲ. ಅದನ್ನು ಇನ್ನೂ ಕೆಳಮಟ್ಟದಲ್ಲಿ ಮಾತ್ರ ನೋಡುತ್ತೇವೆ. ಮನಸ್ಸು ಅಂತ ರ್ಮುಖವಾದಂತೆ ಭೇದಭಾವನೆ ಕಡಿಮೆಯಾಗುತ್ತ ಬರುವುದು. ಕೊನೆಗೆ ಮನಸ್ಸು ಅಖಂಡ ಪರಬ್ರಹ್ಮನಲ್ಲಿ ಐಕ್ಯವಾದಾಗ, ಇವರು ಗಂಡಸರು ಇವರು ಹೆಂಗಸರು ಎಂಬ ಭಾವನೆಯೇ ಇರುವುದಿಲ್ಲ. ನಾವು ಇದನ್ನು ಶ್ರೀರಾಮಕೃಷ್ಣರ ಜೀವನದಲ್ಲಿ ನಿಜವಾಗಿ ನೋಡಿರುವೆವು. ಹಲವು ದೃಷ್ಟಿಯಿಂದ ಸ್ತ್ರೀ ಪುರುಷರಿಗೆ ವ್ಯತ್ಯಾಸಗಳಿ ದ್ದರೂ ಆತ್ಮನ ದೃಷ್ಟಿಯಿಂದ ಯಾವ ವ್ಯತ್ಯಾಸವೂ ಇಲ್ಲ. ಆದಕಾರಣ ಪುರುಷನಿಗೆ ಬ್ರಹ್ಮಜ್ಞಾನ ಸಿಕ್ಕಿದರೆ ಸ್ತ್ರೀಗೂ ಏತಕ್ಕೆ ಸಿಕ್ಕಬಾರದು?

ಭಕ್ತಿ ಮತ್ತು ಜ್ಞಾನಕ್ಕೆ ಸ್ತ್ರೀಯರು ಯೋಗ್ಯರಲ್ಲ ಎಂದು ನೀವು ಯಾವ ಶಾಸ್ತ್ರ ದಲ್ಲಿ ನೋಡುತ್ತೀರಿ? ನಮ್ಮ ಅವನತಿಯ ಕಾಲದಲ್ಲಿ ಹಿಂದೆ ಪುರೋಹಿತರು ಬ್ರಾಹ್ಮಣೇತರರಿಗೆ ವೇದಾಧ್ಯಯನಕ್ಕೆ ಅವಕಾಶವನ್ನು ತಪ್ಪಿಸಿದಾಗಲೆ ಸ್ತ್ರೀಯರಿಗೂ ಅದನ್ನು ಓದುವ ಅವಕಾಶವನ್ನು ತೆಗೆದುಹಾಕಿದರು. ಆಧುನಿಕ ಹಿಂದೂಧರ್ಮ ಪೌರಾಣಿಕವಾದುದು. ಇದು ಬೌದ್ಧಧರ್ಮದ ನಂತರ ಬಂದದ್ದು. ವೇದಗಳಲ್ಲಿ ಬರುವ ನಮ್ಮ ಗೃಹಕೃತ್ಯಕ್ಕೆ ಸಂಬಂಧಪಟ್ಟ ಯಾಗ ಯಜ್ಞಗಳಲ್ಲಿ ಹೆಂಗಸರು ಅತ್ಯಂತ ಆವಶ್ಯಕ ಎಂದು ದಯಾನಂದರು ಸಾರಿದರು, ಅವಳಿಗೆ ಮನೆಯಲ್ಲಿ ಸಾಲಿ ಗ್ರಾಮವನ್ನು ಮುಟ್ಟುವ ಯೋಗ್ಯತೆ ಇರಲಿಲ್ಲ ಎಂಬುದು ಪೌರಾಣಿಕ ಅಭಿಪ್ರಾಯ ಎಂಬುದನ್ನು ದಯಾನಂದ ಸರಸ್ವತಿಗಳು ಶೃತಪಡಿಸಿರುವರು. ವೇದಗಳ ಅಥವಾ ಉಪನಿಷತ್ತಿನ ಕಾಲದಲ್ಲಿ ಸ್ಮೃತಪುಣ್ಯಳಾದ ಗಾರ್ಗಿ ಮೈತ್ರೇಯಿ ಮುಂತಾದವರು ಪುಷಿಗಳಿಗೆ ಸರಿಸಮನಾಗಿ ಬ್ರಹ್ಮವಿದ್ಯೆಯನ್ನು ಚರ್ಚಿಸಬಲ್ಲ ಯೋಗ್ಯತೆಯನ್ನು ಪಡೆದಿರುವುದನ್ನು ನೋಡುತ್ತೇವೆ. ವೇದಶಾಸ್ತ್ರದಲ್ಲಿ ಪಾರಂಗತರಾದ ಸಾವಿರ ಬ್ರಾಹ್ಮಣರು ಇದ್ದ ಮಹಾಸಭೆಯಲ್ಲಿ ಗಾರ್ಗಿ ಧೈರ್ಯದಿಂದ ಯಾಜ್ಞವಲ್ಕ್ಯನನ್ನು ಬ್ರಹ್ಮನ ವಿಷಯವಾಗಿ ಪ್ರಶ್ನಿಸುವಳು.

ಸ್ತ್ರೀಯರ ಉದ್ಧಾರ ಅತ್ಯಂತ ಆವಶ್ಯಕ. ಹಿಂದೆ ಅಂತಹ ಆದರ್ಶನಾರಿಯರಿಗೆ ಬ್ರಹ್ಮಜ್ಞಾನಕ್ಕೆ ಹಕ್ಕಿದ್ದರೆ, ಈಗ ಏತಕ್ಕೆ ಅದೇ ಹಕ್ಕು ಇರಬಾರದು? ಹಿಂದೆ ಯಾವುದು ಸಾಧ್ಯವಾಗಿತ್ತೋ ಅದು ಈಗಲೂ ಸಾಧ್ಯವಾಗಬೇಕು. ಇತಿಹಾಸದಲ್ಲಿ ಹಿಂದಿನಂತೆ ಮುಂದೆಯೂ ಆಗುವುದನ್ನು ನೋಡುವೆವು.

ನೀವು ಯಾವಾಗಲೂ ಸ್ತ್ರೀಯರನ್ನು ದೂರುತ್ತೀರಿ. ಆದರೆ ನೀವು ಅವರ ಮೇಲ್ಮೆಗೆ ಏನು ಮಾಡಿದ್ದೀರಿ? ನೀವು ಹಲವಾರು ಸ್ಮೃತಿಶಾಸ್ತ್ರಗಳನ್ನು ಬರೆದು ಅವರಿಗೆ ಅಷ್ಟದಿಗ್​ಬಂಧನಗಳನ್ನು ಒಡ್ಡಿರುವಿರಿ. ಹೆಂಗಸರನ್ನು ಬರೀ ಮಕ್ಕಳನ್ನು ತಯಾರುಮಾಡುವ ಯಂತ್ರಗಳನ್ನಾಗಿ ಮಾಡಿದ್ದಾರೆ ಗಂಡಸರು. ಯಾವಾಗಲೂ ಇನ್ನೊಬ್ಬರನ್ನು ಆಶ್ರಯಿಸಿಕೊಂಡು, ಯಾವಾಗಲೂ ಪರರನ್ನು ನೆಚ್ಚಿಕೊಂಡಿರು ವಂತೆ ಹೆಂಗಸರನ್ನು ಮಾಡಿರುವೆವು. ಆದಕಾರಣವೇ ಪ್ರಪಂಚದಲ್ಲಿ ಸ್ವಲ್ಪ ಕಷ್ಟ ಮತ್ತು ಅಪಾಯಬಂದರೂ ಅವಳಿಗೆ ಸುಮ್ಮನೆ ಅಳುವುದೊಂದೇ ಗೊತ್ತಿರುವುದು.

ಸ್ತ್ರೀಯರಿಗೆ ಯೋಗ್ಯವಾದ ಗೌರವವನ್ನು ತೋರಿದ ಜನಾಂಗಗಳೆಲ್ಲ ಮುಂದೆ ಬಂದಿವೆ. ಯಾವ ದೇಶವೇ ಆದರೂ ಸ್ತ್ರೀಯರನ್ನು ಗೌರವಿಸಿಲ್ಲವೊ ಅದು ಹಾಗೆ ಮುಂದೆ ಬಂದಿಲ್ಲ ಮತ್ತು ಮುಂದೆ ಬರಲಾದರು. ನಿಮ್ಮ ಜನಾಂಗ ಇಷ್ಟೊಂದು ಹಿಂದೆ ಉಳಿದಿರುವುದಕ್ಕೆ ಮುಖ್ಯ ಕಾರಣ, ಈ ಶಕ್ತಿ ಸ್ವರೂಪಿಣಿಯನ್ನು ನಾವು ಗೌರವಿಸದೆ ಇದ್ದುದು. ಮನು ಎಲ್ಲಿ ನಾರಿಯರನ್ನು ಗೌರವಿಸುವರೊ ಅಲ್ಲಿ ದೇವತೆ ಗಳು ಸುಪ್ರೀತರಾಗುವರು, ಎಲ್ಲಿ ಅವರನ್ನು ಅಗೌರವದಿಂದ ಕಾಣುವರೊ ಅಲ್ಲಿ ಪುರುಷನು ಮಾಡಿದ ಸತ್​ಕರ್ಮಗಳೆಲ್ಲ ನಿಷ್ಫಲವಾಗುವುದು ಎಂದು ಸಾರುತ್ತಾನೆ. ಯಾವ ಮನೆಯಲ್ಲೇ ಆಗಲಿ, ದೇಶದಲ್ಲೇ ಆಗಲಿ ಸ್ತ್ರೀಯರನ್ನು ಗೌರವದಿಂದ ನೋಡುವುದಿಲ್ಲವೊ, ಎಲ್ಲಿ ಸ್ತ್ರೀಯರು ಕೊರಗಿನ ಜೀವನವನ್ನು ಬಾಳುವರೊ ಅಲ್ಲಿ ಪುರುಷರಿಗೆ ಶ್ರೇಯಸ್ಸಾಗಲಾರದು.

ನಿಜವಾಗಿ ಶಕ್ತಿ ಉಪಾಸಕ ಯಾರು ಗೊತ್ತೆ? ಯಾರು ದೇವರನ್ನು ಸರ್ವವ್ಯಾಪಿ ಯಾದ ಚೈತನ್ಯವೆಂದು ಭಾವಿಸುವನೊ ಮತ್ತು ಸ್ತ್ರೀಯರಲ್ಲಿ ಅದರ ಆವಿರ್ಭಾವ ವನ್ನು ನೋಡುವನೊ ಅವನೇ ನಿಜವಾದ ಶಕ್ತಿ ಆರಾಧಕ. ನಿಮ್ಮ ಸ್ತ್ರೀಯರ ಸ್ಥಿತಿ ಯನ್ನು ಉತ್ತಮ ಪಡಿಸಬಲ್ಲಿರಾ? ಆಗ ನಿಮಗೆ ಶ್ರೇಯಸ್ಕರವಾಗುವುದು. ಇಲ್ಲದೆ ಇದ್ದರೆ ನೀವು ಈಗ ಇರುವಂತೆಯೇ ಹಿಂದೆಯೇ ಉಳಿಯುವಿರಿ. ನಮ್ಮ ಸ್ತ್ರೀಯ ರನ್ನು ಮೇಲೆತ್ತಬೇಕು. ಜನಸಾಮಾನ್ಯರನ್ನು ಮೊದಲು ಜಾಗ್ರತಗೊಳಿಸಬೇಕು. ಆಗ ಮಾತ್ರ ನಮ್ಮ ಭರತಖಂಡಕ್ಕೆ ಮಂಗಳವಾಗಬಹುದು. ಸ್ತ್ರೀಯರನ್ನು ಮೇಲಕ್ಕೆ ಎತ್ತಿದರೆ, ಆಗ ಅವರ ಮಕ್ಕಳು ತಮ್ಮ ಸತ್ಕರ್ಮಗಳಿಂದ ನಮ್ಮ ದೇಶದ ಮಹಿಮೆ ಯನ್ನು ಹೆಚ್ಚಿಸುವರು. ಆಗ ನಮ್ಮಲ್ಲಿ ಸಂಸ್ಕೃತಿ ಜ್ಞಾನ ಶಕ್ತಿ ಭಕ್ತಿಗಳು ವ್ಯಕ್ತವಾಗು ವುವು.

\textbf{ಅವರ ಸಮಸ್ಯೆಯನ್ನು ಹೇಗೆ ನಾವು ಬಗೆಹರಿಸಬೇಕು}: ನಿಮ್ಮಲ್ಲಿ ಯಾರಾದರೂ ನಾನು ಈ ಹೆಂಗಸಿನ ಅಥವಾ ಮಗುವಿನ ಅದೃಷ್ಟವನ್ನು ರೂಪಿಸುತ್ತೇನೆ ಎಂದು ಭಾವಿಸಿದರೆ ಅದು ಸುಳ್ಳು. ಸಾವಿರ ಸಲ ಸುಳ್ಳು. ಅನೇಕರು ಹಲವು ವೇಳೆ ನನ್ನನ್ನು ವಿಧವಾ ವಿವಾಹ ಸಂಬಂಧದ ಮೇಲೆ ನನ್ನ ಅಭಿಪ್ರಾಯಗಳನ್ನು ಕೇಳುತ್ತಾರೆ. 'ನಾನೇನು ವಿಧವೆಯೆ? ನನಗೆ ಈ ಪ್ರಶ್ನೆಯನ್ನು ಹಾಕುತ್ತೀರಲ್ಲ' ಎಂದು ಕೇಳಿ ಬಿಡು ತ್ತೇನೆ. ನಾನೇನು ಹೆಂಗಸೆ, ನನಗೇತಕ್ಕೆ ಈ ಪ್ರಶ್ನೆಯನ್ನು ಹಾಕುತ್ತೀರಿ? ಸ್ತ್ರೀಯರ ಸಮಸ್ಯೆಯನ್ನು ಬಗೆಹರಿಸುವುದಕ್ಕೆ ನೀವು ಯಾರು? ನೀವೇನು ಪ್ರತ್ಯಕ್ಷ ದೇವರೆ, ಪ್ರತಿಯೊಬಪ್ ವಿಧವೆ ಮತ್ತು ಪ್ರತಿಯೊಬಪ್ ಹೆಂಗಸನ್ನು ಆಳುವುದಕ್ಕೆ? ನೀವು ಕೈ ತೆಗೆಯಿರಿ. ಅವಳು ತನ್ನ ಸಮಸ್ಯೆಗಳನ್ನು ತಾನೇ ಬಗೆಹರಿಸಿಕೊಳ್ಳುವಳು.

ನನ್ನ ದೃಷ್ಟಿಯಲ್ಲಿ ಪ್ರತಿಯೊಂದು ದೇಶದ ಸಮಾಜವೂ ತನ್ನದೇ ರೀತಿಯಲ್ಲಿ ಅದನ್ನು ಬಗೆಹರಿಸುವುದು. ಆದಕಾರಣ ನಾವು ಈಗಲೇ ಬಾಲ್ಯವಿವಾಹ, ವಿಧವಾ ವಿವಾಹ ಮೊದಲಾದ ಪ್ರಶ್ನೆಗಳಿಂದ ನಮ್ಮ ತಲೆಯನ್ನು ಕೆಡಿಸಿಕೊಳ್ಳಬೇಕಾಗಿಲ್ಲ. ನಮ್ಮ ದೇಶದಲ್ಲಿರುವ ಪ್ರತಿಯೊಬ್ಬ ಸ್ತ್ರೀಪುರುಷರಿಗೂ ಯೋಗ್ಯವಾದ ವಿದ್ಯಾ ಭ್ಯಾಸವನ್ನು ಕೊಡುವುದು ನಮ್ಮ ಕರ್ತವ್ಯ. ಇಂತಹ ವಿದ್ಯಾಭ್ಯಾಸದ ಪರಿಣಾಮ ವಾಗಿ ಅವರಿಗೆ ಯಾವುದು ಸರಿ ಯಾವುದು ತಪ್ಪು ಎಂದು ಗೊತ್ತಾಗಿ ಕೆಟ್ಟದ್ದನ್ನು ಬಿಡುವರು. ಆಗ ಸಮಾಜದಲ್ಲಿ ನಾವು ಪರವಾಗಿ ಏನನ್ನೂ ಎತ್ತಿ ಕಟ್ಟಬೇಕಾಗಿಲ್ಲ, ವಿರೋಧವಾಗಿ ಏನನ್ನೂ ಅಲ್ಲಗಳೆಯಬೇಕಾಗಿಲ್ಲ. ಅವರಿಗೆ ಯೋಗ್ಯವಾದ ವಿದ್ಯಾ ಭ್ಯಾಸವನ್ನು ಕೊಡುವುದಷ್ಟೇ ನಾವು ಮಾಡುವ ಕೆಲಸ. ಸ್ತ್ರೀಯರನ್ನು ತಮ್ಮ ಸಮಸ್ಯೆಯನ್ನು ತಾವೇ ಬಗೆಹರಿಸಿಕೊಳ್ಳುವ ಸ್ಥಿತಿಗೆ ತರಬೇಕು. ಇತರರಾರೂ ಈ ಕೆಲಸವನ್ನು ಅವರಿಗೆ ಮಾಡಬಾರದು. ಭರತಖಂಡದ ಸ್ತ್ರೀಯರು ಪ್ರಪಂಚದ ಇತರ ಸ್ತ್ರೀಯರಷ್ಟೇ ಯೋಗ್ಯರು ತಮ್ಮ ಜವಾಬ್ದಾರಿಯನ್ನು ತಾವು ನಿರ್ವಹಿಸುವು ದರಲ್ಲಿ.

\textbf{ಅವರ ಸಮಸ್ಯೆಯನ್ನು ಪರಿಹರಿಸಿಕೊಳ್ಳಲು ಬೇಕಾದ ವಿದ್ಯಾಭ್ಯಾಸ}: ಇಷ್ಟೊಂದು ಆಶಾಜನಕವಾದ ವ್ಯಕ್ತಿಗಳಿದ್ದರೂ ಅವರನ್ನು ನೀವು ಮೇಲಕ್ಕೆ ಎತ್ತಲಿಲ್ಲ. ನೀವು ಅವರಿಗೆ ಜ್ಞಾನವನ್ನು ನೀಡಲಿಲ್ಲ. ಅವರಿಗೆ ಸರಿಯಾದ ವಿದ್ಯಾಭ್ಯಾಸ ಸಿಕ್ಕಿದರೆ ಪ್ರಪಂಚದಲ್ಲೆಲ್ಲ ಅವರು ಆದರ್ಶನಾರಿಯರಾಗಬಹುದು. ಅವರಲ್ಲಿ ಬೇಕಾದಷ್ಟು ಸಮಸ್ಯೆಗಳೇನೊ ಇವೆ. ಆದರೆ ವಿದ್ಯಾಭ್ಯಾಸ ಎಂಬ ಮಂತ್ರದಿಂದ ಬಗೆಹರಿಸ ಲಾಗದ ಸಮಸ್ಯೆಯೇ ಇಲ್ಲ.

ಸ್ತ್ರೀಯರಿಗೆ ವಿದ್ಯಾಭ್ಯಾಸವನ್ನು ಕೊಡಲು ನಾವು ಪ್ರಾರಂಭಿಸಬೇಕು. ನಮ್ಮ ಹಿಂದೂ ನಾರಿಯರು ಪಾತಿವ್ರತ್ಯವೆಂದರೆ ಏನು ಎಂಬುದನ್ನು ಸುಲಭವಾಗಿ ಅರ್ಥ ಮಾಡಿಕೊಳ್ಳಬಲ್ಲರು. ಏಕೆಂದರೆ ಅವರಿಗೆ ಇದು ವಂಶಾನುಗತವಾಗಿ ಬಂದದ್ದು. ಮೊದಲು ಅವರಲ್ಲಿ ಈ ಶೀಲವನ್ನು ಎಲ್ಲಕ್ಕಿಂತ ಹೆಚ್ಚಾಗಿ ಬೆಳೆಯುವಂತೆ ಮಾಡಿ. ಇದರಿಂದ ಅವರು ಒಂಟಿಯಾಗಿರಲಿ, ಮದುವೆ ಆಗಲಿ, ತಮ್ಮ ಆದರ್ಶಕ್ಕೆ ಪ್ರಾಣ ವನ್ನಾದರೂ ತೆರುವಂತಾಗಲಿ. ಆದರೆ ಎಂದಿಗೂ ಆದರ್ಶದಿಂದ ಚ್ಯುತರಾಗ ಕೂಡದು. ಒಬ್ಬನು ಆದರ್ಶಕ್ಕೆ ಅದು ಏನಾದರೂ ಚಿಂತೆಯಿಲ್ಲ, ತನ್ನ ಜೀವನವನ್ನೇ ಧಾರೆಯೆರೆಯುವುದು ಏನು ಅಲ್ಪವೆ?

ಅವರಿಗೆ ಚರಿತ್ರೆ ಪುರಾಣ ಧರ್ಮ ಕಲೆ ವಿಜ್ಞಾನ ಮನೆಯನ್ನು ನೋಡಿಕೊಳ್ಳು ವುದು, ಅಡಿಗೆ ಮಾಡುವುದು, ಹೊಲಿಯುವುದು ಮತ್ತು ದೇಹದ ಆರೋಗ್ಯಕ್ಕೆ ಸಂಬಂಧಪಟ್ಟ ಮುಖ್ಯ ವಿಷಯಗಳನ್ನು ಹೇಳಿಕೊಡಬೇಕು. ಕಲಿಸಿದರೆ ಸಾಲದು. ಅವರಿಗೆ ವ್ಯವಹಾರ ಜ್ಞಾನವೂ ಇರಬೇಕು. ಅವರೆದುರಿಗೆ ತ್ಯಾಗಭಕ್ತಿ ಮುಂತಾದವು ಗಳಿಂದ ಕೂಡಿದ ಮಹಾವ್ಯಕ್ತಿಗಳ ಆದರ್ಶವನ್ನು ಇಡಬೇಕು. ಇದನ್ನು ಅವರು ಜೀವನದಲ್ಲಿ ಅನುಸರಿಸುವಂತೆ ಇರಬೇಕು. ಸೀತಾ, ಸಾವಿತ್ರಿ, ದಮಯಂತಿ, ಲೀಲಾ ವತಿ, ಖಾನ, ಮೀರ ಮುಂತಾದವರ ಜ್ವಲಂತ ಆದರ್ಶವನ್ನು ಅವರೆದುರಿಗೆ ಇಡ ಬೇಕು. ಇವರ ಆದರ್ಶಗಳಿಂದ ಅವರು ತಮ್ಮ ಜೀವನವನ್ನು ರೂಪಿಸಿಕೊಳ್ಳಬೇಕು. ಇದರ ಜೊತೆಗೆ ಅವರು ಧೈರ್ಯ ಮತ್ತು ಸಾಹಸಗಳನ್ನು ಕೂಡ ಅಭ್ಯಾಸ ಮಾಡ ಬೇಕು. ಈಗಿನ ಕಾಲದಲ್ಲಿ ಅವರು ತಮ್ಮ ಆತ್ಮರಕ್ಷಣೆ ಮಾಡಿಕೊಳ್ಳುವುದನ್ನು ಕಲಿಯಬೇಕಾಗಿದೆ. ಝಾನ್ಸಿರಾಣಿ ಎಷ್ಟು ಅದ್ಭುತವಾಗಿದ್ದಳು ನೋಡಿ. ಇಂತಹ ವಿದ್ಯಾಭ್ಯಾಸವನ್ನು ಅವರಿಗೆ ಕೊಟ್ಟರೆ ಅವರು ತಮ್ಮ ಸಮಸ್ಯೆಗಳನ್ನು ತಾವೇ ಬಗೆ ಹರಿಸಿಕೊಳ್ಳುವರು.

ಮನೆಗಳಲ್ಲಿ ಅವಳು ಆದರ್ಶ ತಾಯಂದಿರಂತೆ ಇರಬೇಕು. ಇಂತಹವರ ಮಕ್ಕಳು ತಮ್ಮ ತಾಯಂದಿರ ಆದರ್ಶಗಳಲ್ಲಿ ಮತ್ತೂ ಮುಂದೆ ಬರುವರು. ಸುಸಂಸ್ಕೃತರಾದ ಸಾಧ್ವಿಗಳ ಮನೆಯಲ್ಲಿಯೇ ಮಹಾಪುರುಷರು ಜನ್ಮವೆತ್ತುವುದು.

ಈಗಿನ ಕಾಲಕ್ಕೆ ಅತ್ಯಂತ ಆವಶ್ಯಕವಾಗಿ ಕೆಲವು ತ್ಯಾಗ ಮನೋಭಾವದ ವ್ಯಕ್ತಿ ಗಳು ಬೇಕಾಗಿದ್ದಾರೆ. ಅಂತಹ ಕೆಲವು ವ್ಯಕ್ತಿಗಳನ್ನು ನಾವು ತಯಾರು ಮಾಡಬೇಕು. ಅನಂತರ ಅವರು ಆಜನ್ಮ ಬ್ರಹ್ಮಚಾರಿಗಳಾಗಿ ಹಿಂದಿನಿಂದ ಬಂದ ಅವರ ರಕ್ತಗತ ವಾದ ಪವಿತ್ರತೆಯ ಮೂಲಕ ಸೇವಾಕಾರ್ಯವನ್ನು ಕೈಕೊಳ್ಳಬಹುದು. ಇದರ ಜೊತೆಗೆ ವಿಜ್ಞಾನ ಮತ್ತು ಇತರ ವಿಷಯಗಳನ್ನು ಅವರಿಗೆ ಹೇಳಿಕೊಡಬೇಕು. ಇದರಿಂದ ಇತರರಿಗೂ ಪ್ರಯೋಜನವಾಗುವುದು. ಇದನ್ನು ತಾವು ಕಲಿತು ಸಂತೋಷದಿಂದ ಇತರರಿಗೂ ಕಲಿಸಬೇಕು. ನಮ್ಮ ತಾಯ್ನಾಡಿನ ಹಿತರಕ್ಷಣೆಗೆ ಕೆಲವು ಸ್ತ್ರೀಯರು ಪರಿ ಶುದ್ಧರಾದ ಬ್ರಹ್ಮಚಾರಿಣಿಯರಾಗಬೇಕು.

ಜನರೆದುರಿಗೆ ಇಂತಹ ಆದರ್ಶವನ್ನು ಇಟ್ಟು, ಇದನ್ನು ಅನುಷ್ಠಾನಕ್ಕೆ ತರಲು ಯತ್ನಿಸಿದರೆ, ಜನರ ಆಲೋಚನೆ ಮತ್ತು ಆದರ್ಶಗಳಲ್ಲಿ ಒಂದು ಕ್ರಾಂತಿಯಾಗು ವುದರಲ್ಲಿ ಸಂದೇಹವಿಲ್ಲ. ಈಗ ಪರಿಸ್ಥಿತಿ ಹೇಗಿದೆ! ಹುಡುಗಿಗೇನಾದರೂ ಒಂಭತ್ತು ಹತ್ತು ವರ್ಷಗಳಾಯಿತು ಎಂದರೆ ತಂದೆತಾಯಿಗಳು ಹೇಗೊ ಅವಳಿಗೆ ಮದುವೆ ಮಾಡಿ ಕೈತೊಳೆದುಕೊಂಡುಬಿಡಬೇಕೆಂದು ಆಲೋಚಿಸುವರು. ಆ ಹುಡುಗಿಗೆ ತನ್ನ ಹದಿಮೂರನೆಯ ವಯಸ್ಸಿನಲ್ಲಿ ಒಂದು ಮಗುವಾದರೆ ಕುಟುಂಬದ ಆನಂದವನ್ನು ಬಣ್ಣಿಸುವುದು ಅಸದಳ! ಪರಿಸ್ಥಿತಿ ಬದಲಾವಣೆ ಆದರೆ ಮಾತ್ರ ಆಶಾದಾಯಕವಾಗು ವುದು. ಆಗ ಹಿಂದಿನ ಶುದ್ಧಗುಣ ನಮ್ಮಲ್ಲಿ ಮೂಡಬಹುದು. ಬ್ರಹ್ಮಚರ್ಯವನ್ನು ಪಾಲನೆ ಮಾಡುವವರಲ್ಲಿ ಎಂತಹ ಆತ್ಮಶ್ರದ್ಧೆ ಮತ್ತು ಧೈರ್ಯ ಮೂಡುವುದು! ಇದರಿಂದ ನಮ್ಮ ದೇಶಕ್ಕೆ ಎಷ್ಟು ಒಳ್ಳೆಯದಾಗುವುದು!

ಭರತಖಂಡದ ಉದ್ಧಾರಕ್ಕೆ ನಾವು ಧೀರರಾದ ಸ್ತ್ರೀಯರನ್ನು ಮುಂದೆ ತರಬೇಕು. ಅವರು ಸಂಗಮಿತ್ರ, ಲೀಲಾವತಿ, ಅಹಲ್ಯಾಬಾಯಿ ಮತ್ತು ಮೀರಾಬಾಯಿ ಮುಂತಾ ದವರ ಆದರ್ಶವನ್ನು ಮುಂದುವರಿಸಬಲ್ಲಂತಹ ವ್ಯಕ್ತಿಗಳಾಗಬೇಕು. ಭಗವಂತನ ಪಾದಸ್ಪರ್ಶದಿಂದ ಮಾತ್ರ ಬರುವಂತಹ ಶಕ್ತಿ ಮತ್ತು ಪವಿತ್ರತೆಯಿಂದ ಕೂಡಿದ, ಮಹಾಪುರುಷರಿಗೆ ಜನ್ಮ ನೀಡಬಲ್ಲಂತಹ ಸ್ತ್ರೀಯರು ನಮಗೆ ಬೇಕು.

ಸ್ತ್ರೀಯರು ಆದಿಶಕ್ತಿಯ ಆವಿರ್ಭಾವ. ಪುರುಷ ತನ್ನ ಇಂದ್ರಿಯಗಳಿಗೆ ಪ್ರಿಯ ವಾದ ಸ್ತ್ರೀಯರ ಬಾಹ್ಯ ಆಕರ್ಷಣೆಗೆ ಉನ್ಮತ್ತನಾಗಿ ಹೋಗಿರುವನು. ಆದರೆ ಅಂತರಾಳದಲ್ಲಿರುವ ಜ್ಞಾನ ಭಕ್ತಿ ವಿವೇಕ ವೈರಾಗ್ಯ ಇವುಗಳನ್ನು ಅರಿತರೆ ಪುರುಷ ಸರ್ವಜ್ಞನಾಗುವನು, ಬ್ರಹ್ಮಜ್ಞಾನಿಯಾಗುವನು. ಅವನು ಹಿಡಿದ ಯಾವ ಕೆಲಸವೂ ವಿಫಲವಾಗುವುದಿಲ್ಲ. “ಅವಳು ತೃಪ್ತಳಾಗಿ, ಅನುಗ್ರಹಿಸಿದರೆ ಮನುಷ್ಯ ಮುಕ್ತಿ ಯನ್ನು ಪಡೆಯುತ್ತಾನೆ,” (ದುರ್ಗಾಸಪ್ತಶತಿ). ತಪಸ್ಸು ಮತ್ತು ಪೂಜೆಯಿಂದ ಜಗನ್ಮಾತೆಯನ್ನು ತೃಪ್ತಿಪಡಿಸಿದಲ್ಲದೆ ಬ್ರಹ್ಮ ವಿಷ್ಣುಗಳಿಗೇ ಅವಳ ಆಕರ್ಷಣೆ ಯಿಂದ ಪಾರಾಗುವ ಸಂಭವವಿಲ್ಲ. ಅವರಲ್ಲಿರುವ ಬ್ರಹ್ಮ ವ್ಯಕ್ತವಾಗುವುದಕ್ಕಾಗಿ, ಗೃಹದೇವಿಯ ಪೂಜೆಗಾಗಿ ನಾನು ಸ್ತ್ರೀ ಮಠವನ್ನು ಸ್ಥಾಪಿಸುತ್ತೇನೆ.

ಈ ಮಠಕ್ಕೆ ಸಂಬಂಧಪಟ್ಟಂತೆ ಒಂದು ಬಾಲಿಕೆಯರ ಶಾಲೆ ಇರುವುದು. ಅಲ್ಲಿ ಹುಡುಗಿಯರಿಗೆ ಧರ್ಮ ಸಾಹಿತ್ಯ ಸಂಸ್ಕೃತ ವ್ಯಾಕರಣ ಸ್ವಲ್ಪ ಇಂಗ್ಲೀಷು ಇವುಗಳ ನ್ನೆಲ್ಲಾ ಹೇಳಿಕೊಡುತ್ತಾರೆ. ನಿತ್ಯಜೀವನದ ಬಳಕೆಗೆ ಬೇಕಾದ ಹೊಲಿಗೆ ಅಡಿಗೆ ಗೃಹಕೃತ್ಯ ಮಕ್ಕಳ ಪೋಷಣೆ ಮುಂತಾದುವನ್ನು ಹೇಳಿಕೊಡುವರು. ಜಪ ಧ್ಯಾನ ಪೂಜೆ ಇವು ಬೋಧನೆಯ ಮುಖ್ಯವಾದ ಅಂಗವಾಗಿರಬೇಕು. ಶಾಲೆಯಲ್ಲಿ ಪಾಠ ವನ್ನು ಕೊಡುವ ಜವಾಬ್ದಾರಿಯು ವಿದ್ಯಾವಂತರಾದ ವಿಧವೆಯರು ಮತ್ತು ಬ್ರಹ್ಮ ಚಾರಿಣಿಯರಿಗೆ ಸೇರುವುದು. ಬಾಲಕಿಯರ ಶಾಲೆಯೊಡನೆ ಈ ದೇಶದಲ್ಲಿ ಗಂಡಸರ ಸಂಪರ್ಕವಿಲ್ಲದೇ ಇರುವುದೇ ಮೇಲು. ಹಿಂದೂ ಬ್ರಹ್ಮಚಾರಿಣಿಯರು ಶಾಲೆಯ ಬಾಲಕಿಯರಿಗೆ ಬ್ರಹ್ಮಚರ್ಯದ ವಿಷಯವಾಗಿ ಬೋಧನೆಯನ್ನು ಕೊಡುವರು.

ಈ ಮಠದಲ್ಲಿ ಐದಾರು ವರುಷಗಳು ತರಬೇತನ್ನು ಪಡೆದ ಮೇಲೆ ಬಾಲಕಿಯರ ಪೋಷಕರು ಅವರಿಗೆ ಬೇಕಾದರೆ ಮದುವೆಯನ್ನು ಮಾಡಬಹುದು. ತ್ಯಾಗಜೀವನಕ್ಕೆ ಅವರು ಯೋಗ್ಯರೆಂದು ಕಂಡುಬಂದರೆ ಅವರ ಪೋಷಕರಿಂದ ಅಪ್ಪಣೆ ಪಡೆದು ಬ್ರಹ್ಮಚರ್ಯವ್ರತವನ್ನು ಸ್ವೀಕರಿಸಿ ಮಠದಲ್ಲಿ ಬೇಕಾದರೆ ಇರಬಹುದು. ಈ ಬ್ರಹ್ಮ ಚಾರಿಣಿಯಾದ ಸಂನ್ಯಾಸಿಯರು ಕಾಲಕ್ರಮೇಣ ಮಠದ ಪ್ರಚಾರಕರು ಮತ್ತು ಉಪಾ ಧ್ಯಾಯರು ಆಗುವರು. ಗ್ರಾಮಗಳಲ್ಲಿ ಮತ್ತು ಊರುಗಳಲ್ಲಿ ಅವರು ಕೇಂದ್ರವನ್ನು ತೆರೆದು ಸ್ತ್ರೀಯರಿಗೆ ವಿದ್ಯಾಭ್ಯಾಸಕ್ಕೆ ಪ್ರಯತ್ನಿಸುವರು. ಇಂತಹ ಶುದ್ಧ ಚಾರಿತ್ರ್ಯದ ಪ್ರಚಾರಕರಿಂದ ದೇಶದಲ್ಲಿ ನಿಜವಾದ ವಿದ್ಯಾಭ್ಯಾಸ ಹರಡುವುದು.

ವಿದ್ಯಾರ್ಥಿನಿಯರು ಮಠದಲ್ಲಿ ಇರುವ ತನಕ ಅವರು ಬ್ರಹ್ಮಚರ್ಯ ವ್ರತವನ್ನು ಪಾಲಿಸಬೇಕು. ಇದೇ ಮಠದ ಮೂಲ ಆದರ್ಶ. ಈ ಮಠದ ವಿದ್ಯಾರ್ಥಿನಿಯರಿಗೆ ತಪಸ್ಸು ತ್ಯಾಗ ಇಂದ್ರಿಯನಿಗ್ರಹವೇ ಆದರ್ಶವಾಗಿರಬೇಕು. ಜೀವಾವಧಿ ಅವರು ಸೇವಾಧರ್ಮವನ್ನು ಪಾಲಿಸಬೇಕು. ಇಂತಹ ಆದರ್ಶ ಜೀವನವನ್ನು ನಡೆಸಿದರೆ ಯಾರು ಅವರನ್ನು ಗೌರವಿಸುವುದಿಲ್ಲ? ಯಾರಿಗೆ ಅವರಲ್ಲಿ ಶ್ರದ್ಧೆ ಇರುವುದಿಲ್ಲ? ಈ ದೇಶದ ಸ್ತ್ರೀಯರ ಶೀಲವನ್ನು ಇಂತಹ ಆದರ್ಶದ ಮೂಲಕ ರೂಪಿಸಿದರೆ ಮಾತ್ರ ಸೀತಾ ಸಾವಿತ್ರಿ ಗಾರ್ಗಿ ಮುಂತಾದ ಆದರ್ಶ ವ್ಯಕ್ತಿಗಳು ಭರತಖಂಡದಲ್ಲಿ ಪುನಃ ಉದಯಿಸಬಹುದು.

