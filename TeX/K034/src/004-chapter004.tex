
\chapter{ವಿದ್ಯಾಭ್ಯಾಸ}

\section{ನಮ್ಮ ಸಮಾಜದ ನ್ಯೂನತೆಗೆಲ್ಲಾ ದಿವ್ಯೌಷಧಿ}

\textbf{ಆಧುನಿಕ ರೀತಿ}: ಈಗ ನಮಗೆ ದೊರಕುತ್ತಿರುವ ವಿದ್ಯಾಭ್ಯಾಸದಲ್ಲಿ ಕೆಲವು ಒಳ್ಳೆಯ ಅಂಶಗಳಿವೆ. ಆದರೆ ಇದರಲ್ಲಿ ದೊಡ್ಡ ನ್ಯೂನತೆಯೊಂದು ಇದೆ. ಇದ ರಿಂದ ಒಳ್ಳೆಯ ಅಂಶಗಳು ಗೌಣವಾಗುವುವು. ಮೊದಲನೆಯದಾಗಿ, ಇದು ಪುರುಷ ಸಿಂಹರನ್ನು ಮಾಡುವ ವಿದ್ಯಾಭ್ಯಾಸವಲ್ಲ. ಇದು ಕೇವಲ ನಿಷೇಧಮಯವಾದುದು. ನಿಷೇಧಮಯ ವಿದ್ಯಾಭ್ಯಾಸವಾದರೂ ಆಗಲಿ ಅಥವಾ ನಿಷೇಧದ ಮೇಲೆ ನಿಂತ ಯಾವ ತರಬೇತೆ ಆಗಲಿ ಅದು ಮೃತ್ಯುಗಿಂತ ಘೋರವಾದುದು. ಮಗುವನ್ನು ಶಾಲೆಗೆ ಹಾಕುವರು. ಆ ಮಗು ಕಲಿಯುವ ಮೊದಲನೇ ಪಾಠವೇ ತನ್ನ ತಂದೆ ಮೂರ್ಖ ಎಂದು ತಿಳಿಯುವುದು. ಎರಡನೆಯದೆ ತನ್ನ ಅಜ್ಜ ಹುಚ್ಚ ಎಂದು ತಿಳಿಯುವುದು. ಮೂರನೆಯದೇ ಗುರುಗಳೆಲ್ಲ ಮಿಥ್ಯಾಚಾರಿಗಳು ಎಂಬುದು. ನಾಲ್ಕನೆಯದೆ ನಮ್ಮ ಶಾಸ್ತ್ರಗಳೆಲ್ಲ ಸುಳ್ಳಿನ ಕಂತೆ ಎಂಬುದು. ಮಗುವಿಗೆ ಹದಿ ನಾರು ವರುಷಗಳು ಆಗುವ ಹೊತ್ತಿಗೆ ಅವನೇ ಒಂದು ಕೆಲಸಕ್ಕೆ ಬಾರದ ಕಂತೆ ಆಗುವನು. ನಿರ್ಜೀವನಾಗಿ ನಿತ್ರಾಣನಾಗುವನು. ಇದರ ಪರಿಣಾಮವಾಗಿ ಕಳೆದ ಐವತ್ತು ವರುಷಗಳಿಂದಲೂ ಇಂತಹ ವಿದ್ಯಾಭ್ಯಾಸದಿಂದ ಮೂರು ಪ್ರಾಂತ್ಯ ಗಳಲ್ಲಿಯೂ ಒಬಪ್ ಸ್ವತಂತ್ರವಾಗಿ ವಿಚಾರ ಮಾಡುವ ಮನುಷ್ಯ ತರಬೇತಾಗಿಲ್ಲ. ಸ್ವತಂತ್ರವಾಗಿ ಆಲೋಚನೆ ಮಾಡುವ ಪ್ರತಿಯೊಬಪ್ನೂ ಬೇರೆ ಎಲ್ಲೊ ವಿದ್ಯಾ ಭ್ಯಾಸವನ್ನು ಪಡೆದವನು, ಈ ದೇಶದಲ್ಲಿ ಅಲ್ಲ: ಅಥವಾ ಈಗಿನ ಶಿಕ್ಷಣದ ಮೌಢ್ಯತೆಯಿಂದ ಪಾರಾಗಲು ಹಿಂದಿನ ಕಾಲದ ಶಾಲೆಗಳಿಗೆ ಹೋದನು.

ಹುಡುಗರಾದಾಗಿನಿಂದಲೂ ನಮಗೆ ನಿಷೇಧಮಯವಾದ ವಿದ್ಯಾಭ್ಯಾಸವೇ ಆಗಿದೆ. ನಾವು ಯಾವ ಕೆಲಸಕ್ಕೂ ಬಾರದವರೆಂಬುದನ್ನು ಮಾತ್ರ ತಿಳಿದುಕೊಂಡಿರು ವೆವು. ನಮ್ಮ ದೇಶದಲ್ಲಿ ಮಹಾಪುರುಷರು ಜನಿಸಿದ್ದರು ಎಂಬುದನ್ನು ತಿಳಿದು ಕೊಳ್ಳುವುದಕ್ಕೆ ಅವಕಾಶವೇ ಇಲ್ಲ. ನಿರ್ವಿಧವಾದ ಕಲ್ಯಾಣಪ್ರದವಾದ ಯಾವ ಭಾವನೆಗಳನ್ನೂ ನಮಗೆ ಕೊಟ್ಟಿಲ್ಲ. ನಾವೇ ಸ್ವತಂತ್ರವಾಗಿ ಕೆಲಸ ಮಾಡುವುದನ್ನು ಕಲಿತಿಲ್ಲ. ನಾವು ಬರೀ ದೌರ್ಬಲ್ಯವನ್ನು ಕಲಿತಿರುವೆವು.

ಈಗಿನ ವಿದ್ಯಾಭ್ಯಾಸ ಕೇವಲ ಗುಮಾಸ್ತರನ್ನು ತಯಾರುಮಾಡುವ ಕಾರ್ಖಾನೆ ಯಂತಿದೆ. ಇದು ಬರೀ ಇಷ್ಟೇ ಆಗಿದ್ದರೆ ನಾನು ಎಷ್ಟೋ ಸಂತೋಷಪಡುತ್ತಿದ್ದೆ. ಆದರೆ ನೋಡಿ, ಜನ ಹೇಗೆ ಶ್ರದ್ಧಾವಿಹೀನರಾಗುತ್ತಿರುವರು. ಭಗವದ್ಗೀತೆಯನ್ನು ಅದೊಂದು ಪ್ರಕ್ಷಿಪ್ತಭಾಗವೆಂದು ಹೇಳುವರು. ವೇದಗಳಲ್ಲಿ ಬರುವ ಮಂತ್ರ ಗಳನ್ನು ಬರೀ ಹಳ್ಳಿಯ ಹಾಡುಗಳೆಂದು ಹೇಳುವರು. ಭರತಖಂಡಕ್ಕೆ ಹೊರಗಿರುವ ಪ್ರತಿಯೊಂದು ಘಟನೆ ಮತ್ತು ವ್ಯಕ್ತಿಗೆ ಸಂಬಂಧಪಟ್ಟದ್ದನ್ನೆಲ್ಲ ಚೆನ್ನಾಗಿ ತಿಳಿದು ಕೊಳ್ಳಲು ಇಚ್ಛಿಸುವರು. ಆದರೆ ತಮ್ಮ ದೇಶದ ಹಿಂದಿನ ಏಳು ತಲೆಮಾರಿನವರೆಗೆ ಇರುವವರ ಹೆಸರೇ ಗೊತ್ತಿಲ್ಲ. ಇನ್ನು ಹದಿನಾಲ್ಕು ತಲೆಮಾರಿನ ಹೆಸರನ್ನು ಬಿಟ್ಟು ಬಿಡಿ!

ನಮ್ಮ ಶಿಕ್ಷಣಪಂಡಿತರು ನಮ್ಮ ಹುಡುಗರನ್ನು ಅರಗಿಳಿಗಳಂತೆ ಮಾಡುತ್ತಿರು ವರು. ಹಲವಾರು ವಿಷಯಗಳನ್ನು ಅರ್ಥವಾಗದೆ ಅವರ ತಲೆಗೆ ತುಂಬಿ ಅವರ ತಿಳಿವಳಿಕೆಯನ್ನು ಕೆಡಿಸುತ್ತಿರುವರು. ಅಯ್ಯೊ ದೇವರೆ! ಪದವೀಧರನಾಗುವುದಕ್ಕೆ ಎಷ್ಟೊಂದು ಅವಸರ, ಗಲಾಟೆ! ಅದಾದಮೇಲೆ ಎಲ್ಲಾ ತಣ್ಣಗಾಗುವುದು. ಇವತ್ತೆಲ್ಲ ಅವರೇನು ಕಲಿಯುವರು? ನಮ್ಮಲ್ಲಿರುವ ಆಚಾರ ವ್ಯವಹಾರವೆಲ್ಲ ಕೆಲಸಕ್ಕೆ ಬಾರದುದು; ಪಾಶ್ಚಾತ್ಯರ ಆಚಾರ ವ್ಯವಹಾರಗಳೆಲ್ಲ ಒಳ್ಳೆಯವು ಎಂಬುದು. ಇಷ್ಟೊಂದನ್ನು ಕಲಿತರೂ ಬಡತನದಿಂದ ಪಾರಾಗಲಾರರು. ಇಂತಹ ಕಾಲೇಜಿನ ಶಿಕ್ಷಣ ಇದ್ದರೆ ಎಷ್ಟು, ಹೋದರೆ ಎಷ್ಟು. ನಮ್ಮ ಜನರಿಗೆ ಸ್ವಲ್ಪ ಯಾಂತ್ರಿಕ ಶಿಕ್ಷಣ ಸಿಕ್ಕಿದರೆ ಅದರಿಂದ ಜೀವನೋಪಾಯವನ್ನಾದರೂ ಮಾಡಿ ಕೊಳ್ಳಬಹುದು. ಬರೀ ಪದವೀಧರನಾಗುವ ವಿದ್ಯಾಭ್ಯಾಸದಿಂದ ಚಾಕರಿಗೆ ಗೋಗರೆ ಯುವುದೇ ಆಗಿದೆ.

ಯಾವುದರ ಪರಿಣಾಮವಾಗಿ ಹಲವು ತಲೆಮಾರಿನಿಂದ ಇಚ್ಛಾಶಕ್ತಿಯ ವಿಕಾಸಕ್ಕೆ ಅಡಚಣೆಯುಂಟಾಗಿದೆಯೋ, ಇಲ್ಲ ಅದು ಸಂಪೂರ್ಣ ನಿರ್ನಾಮವಾಗಿದೆಯೋ ಅದನ್ನು ವಿದ್ಯಾಭ್ಯಾಸವೆಂದು ಕರೆಯಬಹುದೆ? ಯಾವುದರ ಪ್ರಭಾವದಿಂದ ಹೊಸ ಭಾವನೆಗಳನ್ನು ಬಿಡಿ, ಹಳೆಯ ಭಾವನೆಗಳು ಕೂಡ ಕ್ರಮೇಣ ನಶಿಸಿಹೋಗು ತ್ತವೆಯೊ ಅದನ್ನು ವಿದ್ಯಾಭ್ಯಾಸವೆಂದು ಕರೆಯಬಹುದೆ? ಮನುಷ್ಯನನ್ನು ಒಂದು ಯಂತ್ರಸದೃಶವನ್ನಾಗಿ ಮಾಡುವುದು ವಿದ್ಯಾಭ್ಯಾಸವೆ? ನನ್ನ ದೃಷ್ಟಿಯಲ್ಲಿ ಒಬ್ಬ ಇಚ್ಛಾನುಸಾರ ಸ್ವಂತ ಬುದ್ಧಿವಂತಿಕೆಯಿಂದ ತಪ್ಪುಮಾಡುವುದು ಕೂಡ ಯಂತ್ರ ದಂತೆ ಒಳ್ಳೆಯದಾಗಿರುವುದಕ್ಕಿಂತ ಮೇಲು.

ಒಬ್ಬ ಕೆಲವು ಪರೀಕ್ಷೆಗಳನ್ನು ಪಾಸುಮಾಡಿ ಚೆನ್ನಾಗಿ ಮಾತನಾಡಿದರೆ ವಿದ್ಯಾ ವಂತನೆಂದು ಭಾವಿಸುತ್ತೀರಿ. ಯಾವ ವಿದ್ಯಾಭ್ಯಾಸ ಜನಸಾಧಾರಣರಿಗೆ ಜೀವನೋ ಪಾಯಕ್ಕೆ ಸಹಾಯಮಾಡಲಾರದೊ, ಚಾರಿತ್ರ್ಯಶುದ್ಧಿಗೆ ಸಹಾಯ ಮಾಡಲಾರದೊ, ಜೀವಿಯ ಹೃದಯದಲ್ಲಿ ಪರೋಪಕಾರದ ಭಾವನೆಯನ್ನು ಮತ್ತು ಸಿಂಹಸದೃಶ ಧೈರ್ಯವನ್ನು ತುಂಬಲಾರದೊ ಅದರಿಂದ ಏನು ಪ್ರಯೋಜನ? ಶಾಲಾಕಾಲೇಜು ಗಳಲ್ಲಿ ನಿಮಗೆ ದೊರಕುತ್ತಿರುವ ವಿದ್ಯಾಭ್ಯಾಸದಿಂದ ಈಗ ನೀವು ಅಜೀರ್ಣದಿಂದ ನರಳುತ್ತಿರುವಿರಿ. ನೀವೀಗ ಯಂತ್ರದಂತೆ ಕೆಲಸಮಾಡುತ್ತಿರುವಿರಿ.

\textbf{ನಿಜವಾದ ವಿದ್ಯಾಭ್ಯಾಸ}: ನಿಜವಾದ ವಿದ್ಯಾಭ್ಯಾಸವೆಂದರೇನು? ಅದೇನು ಪುಸ್ತಕ ಪಾಂಡಿತ್ಯವೆ? ಅಲ್ಲ. ಅದೇನು ಹಲವು ವಿಷಯಗಳನ್ನು ತಿಳಿದುಕೊಂಡಿರುವುದೆ? ಅದೂ ಅಲ್ಲ. ಯಾವುದರಿಂದ ನಮ್ಮ ಇಚಾಊಇ್ಕ86ದ್ಶಕ್ತಿ ವ್ಯಕ್ತವಾಗುವುದೊ ಮತ್ತು ಅದನ್ನು ಫಲಪ್ರದವಾಗುವಂತೆ ಮಾಡಬಲ್ಲದೊ ಆ ತರಬೇತನ್ನೇ ವಿದ್ಯಾಭ್ಯಾಸ ಎನ್ನಬಹುದು. ನಿಜವಾದ ವಿದ್ಯಾಭ್ಯಾಸ ನಮ್ಮಲ್ಲಿ ಸುಪ್ತವಾಗಿರುವ ಸಾಧ್ಯತೆಗಳನ್ನು ವ್ಯಕ್ತವಾಗುವಂತೆ ಮಾಡಬೇಕು. ವ್ಯಕ್ತಿಯು ಸರಿಯಾಗಿರುವುದನ್ನು ಪರಿಣಾಮಕಾರಿ ಯಾದ ರೀತಿಯಲ್ಲಿ ಇಚಿಊಇ್ಕ86ದ್ಸುವುದಕ್ಕೆ ಸಾಧ್ಯವಾಗುವಂತೆ ಮಾಡಬೇಕು. ಅದು ಬರಿಯ ವಿಷಯ ಸಂಗ್ರಹವಲ್ಲ. ಎಲ್ಲಾ ಶಿಕ್ಷಣದ ಉದ್ದೇಶವೂ, ತರಬೇತಿನ ಉದ್ದೇಶವೂ ಪುರುಷಸಿಂಹರನ್ನು ಮಾಡುವುದಾಗಿರಬೇಕು. ವಿದ್ಯಾಭ್ಯಾಸ ಎಂದರೆ ಅದು ನಿಮ್ಮ ತಲೆಗೆ ತುರುಕಿದ ಸರಕಲ್ಲ. ಅದು ಅಲ್ಲಿ ಜೀವಾವಧಿ ಅಜೀರ್ಣವಾಗಿ ಚೆಲ್ಲಾಪಿಲ್ಲಿಯಾಗಿ ಇರುವುದಲ್ಲ. ನಿಜವಾದ ವಿದ್ಯಾಭ್ಯಾಸ ನಮ್ಮ ಜೀವನವನ್ನು ರೂಪಿಸಬೇಕು; ಪುರುಷಸಿಂಹರನ್ನು ಮಾಡಬೇಕು; ಶುದ್ಧ ಚಾರಿತ್ರ್ಯದವರನ್ನಾಗಿ ಮಾಡಬೇಕು. ಭಾವನೆಗಳನ್ನು ರಕ್ತಗತಮಾಡಿಕೊಳ್ಳುವಂತೆ ಮಾಡಬೇಕು. ನೀವು ಐದು ಭಾವನೆಗಳನ್ನು ಚೆನ್ನಾಗಿ ತಿಳಿದುಕೊಂಡು ನಿಮ್ಮ ಜೀವನದಲ್ಲಿ ಅವನ್ನು ವ್ಯಕ್ತ ಪಡಿಸಿದರೆ ಆಗ ಒಂದು ಪುಸ್ತಕಭಂಡಾರವನ್ನೇ ಕಂಠಪಾಠ ಮಾಡಿಕೊಂಡಿರುವವನಿ ಗಿಂತ ಹೆಚ್ಚು ವಿದ್ಯಾವಂತರು ನೀವು. "ಗಂಧದ ತುಂಡುಗಳನ್ನು ಹೊರುವ ಕತ್ತೆಗೆ ಅದರ ಭಾರ ಮಾತ್ರ ಗೊತ್ತು". ಅದರ ಬೆಲೆ ಗೊತ್ತಿಲ್ಲ. ವಿಷಯ ಸಂಗ್ರಹವೇ ವಿದ್ಯಾಭ್ಯಾಸವಾದರೆ ಪ್ರಪಂಚದಲ್ಲಿ ಪುಸ್ತಕಾಲಯಗಳೇ ಮಹಾಮುನಿಗಳಾಗುತ್ತಿ ದ್ದವು, ವಿಶ್ವಕೋಶಗಳೇ ಮಹಾಪುಷಿಗಳಾಗುತ್ತಿದ್ದವು!

ವಿದ್ಯಾಭ್ಯಾಸ ಎಂದರೆ, ನಾನು ಈಗ ಜಾರಿಯಲ್ಲಿರುವುದನ್ನು ಕುರಿತು ಹೇಳು ತ್ತಿಲ್ಲ. ಸ್ಪಷ್ಟವಾದ ನಿರ್ದಿಷ್ಟವಾದ ವಿಷಯವನ್ನು ಬೋಧಿಸಬೇಕಾಗಿದೆ. ಬರೀ ಪುಸ್ತಕ ಪಾಂಡಿತ್ಯ ಏನೂ ಪ್ರಯೋಜನವಿಲ್ಲ. ಯಾವ ವಿದ್ಯಾಭ್ಯಾಸದಿಂದ ನಮ್ಮಲ್ಲಿ ಶುದ್ಧಚಾರಿತ್ರ್ಯ ಮೂಡುವುದೊ, ನಮ್ಮ ಮಾನಸಿಕ ಶಕ್ತಿ ವೃದ್ಧಿಯಾಗುವುದೊ, ಬುದ್ಧಿವಿಕಾಸವಾಗುವುದೊ, ವ್ಯಕ್ತಿ ಸ್ವತಂತ್ರನಾಗಿ ಬಾಳತಕ್ಕ ಸ್ಥಿತಿಗೆ ಬರಬಲ್ಲನೊ ಅಂತಹ ತರಬೇತು ಬೇಕು. ಪಾಶ್ಚಾತ್ಯವಿಜ್ಞಾನ ಶಾಸ್ತ್ರಗಳು ವೇದಾಂತದೊಂದಿಗೆ ಮಿಲನವಾಗಿರಬೇಕು. ನಮ್ಮಲ್ಲಿ ಬ್ರಹ್ಮಚರ್ಯದ ಆದರ್ಶವಿರಬೇಕು. ನಮ್ಮಲ್ಲಿ ಆತ್ಮಶ್ರದ್ಧೆ ಇರಬೇಕು.

ಮೇಲಿನ ಮಟ್ಟದ ಶಿಕ್ಷಣ ಎಂದರೆ ಕೇವಲ ಭೌತಿಕ ಶಾಸ್ತ್ರವನ್ನು ತಿಳಿದುಕೊಂಡು ನಮ್ಮ ನಿತ್ಯಜೀವನಕ್ಕೆ ಬೇಕಾದ ಹಲವು ವಸ್ತುಗಳನ್ನು ಯಂತ್ರದಿಂದ ತಯಾರು ಮಾಡುವುದೇನು? ನಮ್ಮ ಜೀವನ ಸಮಸ್ಯೆಯನ್ನು ಪರಿಹರಿಸುವುದೆ ಮೇಲಿನ ಮಟ್ಟದ ಶಿಕ್ಷಣದಿಂದ ಆಗುವ ಪ್ರಯೋಜನ. ಆಧುನಿಕ ನಾಗರಿಕ ಜನಾಂಗದ ಮಹಾಮೇಧಾವಿಗಳು ಇದನ್ನು ಕುರಿತು ಚಿಂತಿಸುತ್ತಿರುವರು. ಆದರೆ ನಮ್ಮ ದೇಶ ದಲ್ಲಿ ಸಾವಿರಾರು ವರುಷಗಳ ಹಿಂದೆಯೇ ಇದನ್ನು ಬಗೆಹರಿಸಿದರು.

ವಿದ್ಯಾಭ್ಯಾಸ ಎಂದರೆ ಆಗಲೆ ಸುಪ್ತಾವಸ್ಥೆಯಲ್ಲಿರುವ ಪರಿಪೂರ್ಣತೆಯನ್ನು ವ್ಯಕ್ತಪಡಿಸುವುದು. ಧರ್ಮವೇ ಶಿಕ್ಷಣದ ಸಾರ ಎಂದು ಭಾವಿಸುತ್ತೇನೆ. ಆದರೆ ಅದು ನನ್ನ ಅಥವಾ ನಿಮ್ಮ ಧಾರ್ಮಿಕ ಭಾವನೆಗಳಲ್ಲ. ನಾವು ಊಟಮಾಡುವಾಗ ಅನ್ನದಂತೆ ಇರುವುದೇ ಧರ್ಮ. ಉಳಿದವುಗಳೆಲ್ಲ ಪಲ್ಯ ಉಪ್ಪಿನಕಾಯಿಗಳಂತೆ. ಬರೀ ಪಲ್ಯವನ್ನು ತಿಂದರೆ ಅಜೀರ್ಣವಾಗುವುದು. ಅದರಂತೆಯೇ ಬರೀ ಅನ್ನವನ್ನು ತಿಂದರೂ ಕೂಡ.


\section{ಆದರ್ಶಮಾರ್ಗ}

\textbf{ಏಕಾಗ್ರತೆ ಮತ್ತು ಅನಾಸಕ್ತಿ}: ನಮಗೆ ಜ್ಞಾನಸಂಪಾದನೆಗೆ ಒಂದೇ ಮಾರ್ಗ ಇರು ವುದು. ಅತ್ಯಂತ ಕನಿಷ್ಠಪಕ್ಷದ ಮನುಷ್ಯನಿಂದ ಹಿಡಿದು ಮಹಾಯೋಗಿಯ ವರೆಗೆ ಎಲ್ಲರೂ ಒಂದೇ ಮಾರ್ಗವನ್ನು ಉಪಯೋಗಿಸಬೇಕಾಗಿದೆ. ಅದನ್ನೇ ಏಕಾಗ್ರತೆ ಎನ್ನುವುದು. ಪ್ರಯೋಗಶಾಲೆಯಲ್ಲಿ ಕೆಲಸಮಾಡುತ್ತಿರುವ ರಸಾಯನಶಾಸ್ತ್ರಜ್ಞ ತನ್ನ ಮನಸ್ಸನ್ನೆಲ್ಲ ಏಕಾಗ್ರಮಾಡಿ ತಾನು ವಿಭಜನೆ ಮಾಡುತ್ತಿರುವ ವಸ್ತುವಿನ ಮೇಲೆ ಬೀರುವನು. ಇದರಿಂದ ಅವನಿಗೆ ಅವುಗಳ ರಹಸ್ಯ ಗೊತ್ತಾಗುವುದು. ಖಗೋಳ ಶಾಸ್ತ್ರಜ್ಞ ತನ್ನ ಮಾನಸಿಕ ಶಕ್ತಿಯನ್ನೆಲ್ಲಾ ಏಕಾಗ್ರಮಾಡಿ ದೂರದರ್ಶಕ ಯಂತ್ರದ ಮೂಲಕ ಆಕಾಶವನ್ನು ಭೇದಿಸುವನು. ಆಗ ಸೂರ್ಯಚಂದ್ರ ತಾರಾವಳಿಗಳು ತಮ್ಮ ರಹಸ್ಯವನ್ನು ಅವನಿಗೆ ಹೇಳುವುವು.

ಮನೋಶಕ್ತಿಯ ಏಕಾಗ್ರತೆಯಿಂದಲೆ ಪ್ರಪಂಚದಲ್ಲಿರುವ ಜ್ಞಾನರಾಶಿಯನ್ನೆಲ್ಲ ಸಂಗ್ರಹಿಸಿದ್ದು. ಪ್ರಪಂಚದ ಬಾಗಿಲುಗಳನ್ನು ಹೇಗೆ ತಟ್ಟುವುದು ಎಂಬುದು ನಮಗೆ ಗೊತ್ತಿದ್ದರೆ, ಅದು ತನ್ನ ರಹಸ್ಯಗಳನ್ನೆಲ್ಲ ನಮಗೆ ನೀಡುವುದು. ಪೆಟ್ಟಿನ ಶಕ್ತಿ ಮತ್ತು ತೀವ್ರತೆ ನಮ್ಮ ಏಕಾಗ್ರತೆಯ ಮೇಲೆ ನಿಂತಿರುವುದು. ಮನುಷ್ಯನ ಶಕ್ತಿಗೆ ಒಂದು ಮೇರೆ ಇಲ್ಲ. ನಾವು ಅದನ್ನು ಏಕಾಗ್ರ ಮಾಡಿದಷ್ಟು ಪ್ರಬಲವಾಗು ವುದು. ಇದೇ ರಹಸ್ಯ.

ಬೂಡ್ಸುಗಳಿಗೆ ಬಣ್ಣ ಹಾಕುವ ಅತ್ಯಂತ ಕೆಳಮಟ್ಟದ ಕೆಲಸ ಮಾಡುವವನು ಕೂಡ, ಅವನ ಏಕಾಗ್ರತೆ ಹೆಚ್ಚಾದಂತೆ ಆ ಕೆಲಸವನ್ನು ಮತ್ತೂ ಚೆನ್ನಾಗಿ ಮಾಡು ವನು. ಅಡಿಗೆ ಮಾಡುವವನಿಗೆ ಏಕಾಗ್ರತೆ ಇದ್ದರೆ ಅವನು ಚೆನ್ನಾಗಿ ಅಡಿಗೆ ಮಾಡು ವನು. ದ್ರವ್ಯಸಂಪಾದನೆಯ ವೃತ್ತಿಯಲ್ಲಾಗಲಿ, ದೇವರಪೂಜೆಯ ವಿಷಯದಲ್ಲಾ ಗಲಿ ಅಥವಾ ಮತ್ತಾವುದೇ ಕೆಲಸದಲ್ಲಿ ನಿರತನಾಗಿದ್ದರೂ, ಮನೋ ಏಕಾಗ್ರತೆ ತೀವ್ರವಾದಂತೆ ಅವನ ಕೆಲಸವೂ ಅಷ್ಟು ಚೆನ್ನಾಗುವುದು. ಈ ಏಕಾಗ್ರತೆಯ ಕರೆ ಯೊಂದೇ, ಏಕಾಗ್ರತೆಯ ಪೆಟ್ಟೊಂದೇ ಪ್ರಕೃತಿಯ ಬಾಗಿಲುಗಳನ್ನು ತೆರೆದು ಅಲ್ಲಿ ಇರುವ ರಹಸ್ಯವನ್ನು ನಮಗೆ ನೀಡುವುದು. ಮನೋ ಏಕಾಗ್ರತೆಯೇ ಜ್ಞಾನ ಭಂಡಾರಕ್ಕೆ ಇರುವ ಏಕಮಾತ್ರ ಕೀಲಿಕೈ.

ನಾವು ಏಕಾಗ್ರತೆಯನ್ನು ವೃದ್ಧಿಮಾಡಿಕೊಂಡಂತೆ ಅನಾಸಕ್ತಿಯನ್ನು ಕೂಡ ವೃದ್ಧಿ ಮಾಡಿಕೊಳ್ಳಲೇ ಬೇಕು. ನಾವು ಮನಸ್ಸನ್ನು ಒಂದು ವಸ್ತುವಿನ ಮೇಲೆ ಏಕಾಗ್ರ ಮಾಡುವುದನ್ನು ಮಾತ್ರ ಕಲಿಯುವುದಲ್ಲ, ಅದನ್ನು ಮರುಕ್ಷಣದಲ್ಲಿಯೇ ಅಲ್ಲಿಂದ ತೆಗೆದು ಬೇರೊಂದು ವಸ್ತುವಿನ ಮೇಲೆ ಇಡಬೇಕು. ಇವೆರಡನ್ನು ನಾವು ಒಟ್ಟಿಗೆ ಅಭ್ಯಾಸ ಮಾಡಬೇಕು. ಆಗಲೇ ನಾವು ಸುರಕ್ಷಿತವಾಗಿರುವೆವು.

ಮನಸ್ಸನ್ನು ಒಂದು ರೀತಿ ಅಭಿವೃದ್ಧಿ ಮಾಡುವುದೆಂದರೆ ಇದೇ. ನನ್ನ ದೃಷ್ಟಿ ಯಲ್ಲಿ ವಿದ್ಯಾಭ್ಯಾಸದ ಸಾರವೇ ಮನೋ ಏಕಾಗ್ರತೆ, ಸುಮ್ಮನೆ ವಿಷಯಗಳನ್ನು ಸಂಗ್ರಹಿಸುವುದಲ್ಲ. ನಾನೇನಾದರೂ ಪುನಃ ವಿದ್ಯಾಭ್ಯಾಸವನ್ನು ಮಾಡಬೇಕಾದ ಅವಕಾಶ ಒದಗಿದರೆ, ಅದರಲ್ಲಿ ನನಗೇನಾದರೂ ಸ್ವಾತಂತ್ರ್ಯವಿದ್ದರೆ, ನಾನು ವಿಷಯಗಳನ್ನು ಸಂಗ್ರಹಿಸುವುದರ ಕಡೆಗೆ ಗಮನ ಕೊಡುವುದಿಲ್ಲ. ಅದರ ಬದಲು ನಾನು ಮನಸ್ಸಿನ ಏಕಾಗ್ರತೆ ಮತ್ತು ಅನಾಸಕ್ತಿಯನ್ನು ರೂಢಿಸಿ ಅನಂತರ ಅಂತಹ ಪೂರ್ಣವಾದ ಮಾನಸಿಕ ಯಂತ್ರದಿಂದ ನನ್ನ ಇಚ್ಛಾನುಸಾರ ವಿಷಯಗಳನ್ನು ಸಂಗ್ರಹಿಸುವೆನು. ಮಗುವಿಗೆ ಜೊತೆಜೊತೆಯಲ್ಲಿಯೇ ಒಂದು ವಸ್ತುವಿನ ಮೇಲೆ ಮನಸ್ಸನ್ನು ಇಡುವುದನ್ನು ಮತ್ತು ಅದರಿಂದ ಮನಸ್ಸನ್ನು ತೆಗೆದುಕೊಳ್ಳುವುದನ್ನು ಕಲಿಸಬೇಕು.

\textbf{ಬ್ರಹ್ಮಚಾರ್ಯ}: ಪ್ರತಿಯೊಬಪ್ ಹುಡುಗನಿಗೂ ಬ್ರಹ್ಮಚರ್ಯವನ್ನು ಚೆನ್ನಾಗಿ ಪಾಲಿಸು ವುದನ್ನು ಕಲಿಸಬೇಕು. ಆಗ ಮಾತ್ರ ಶ್ರದ್ಧಾ ಮತ್ತು ಭಕ್ತಿಗಳು ಬರುವುವು. ಕಾಯಾ ವಾಚಾ ಮನಸಾ ಎಲ್ಲಾ ಸ್ಥಿತಿಗಳಲ್ಲಿಯೂ ಪರಿಶುದ್ಧವಾಗಿರುವುದೇ ಬ್ರಹ್ಮಚರ್ಯ. ಬ್ರಹ್ಮಚರ್ಯದ ಅಭಾವದಿಂದಲೇ ನಮ್ಮಲ್ಲಿ ಎಲ್ಲವೂ ಇಂತಹ ಕ್ಷೀಣಸ್ಥಿತಿಗೆ ಬಂದಿ ರುವುದು. ಬ್ರಹ್ಮಚರ್ಯದ ನಿಷ್ಠೆಯಿಂದ ಅತ್ಯಲ್ಪ ಕಾಲದಲ್ಲಿ ನಾವು ವಿಷಯಗಳನ್ನು ಸಂಗ್ರಹಿಸಬಹುದು. ಒಂದು ಸಲ ನೋಡಿದರೆ ಸಾಕು; ಅದನ್ನು ಮರೆಯದಂತೆ ಜ್ಞಾಪಕದಲ್ಲಿಟ್ಟುಕೊಳ್ಳಬಹುದು. ಬ್ರಹ್ಮಚರ್ಯದೀಕ್ಷಿತನಾದವನಿಗೆ ಅದ್ಭುತವಾದ ಇಚ್ಚಾಶಕ್ತಿ ಕ್ರಿಯಾಶಕ್ತಿಗಳು ಬರುವುವು. ಯಾವಾಗ ನಾವು ಆಸೆಯನ್ನು ನಿಗ್ರಹಿಸು ತ್ತೇವೆಯೊ ಆಗ ಮನಸ್ಸು ತುಂಬ ಮೇಲುಮಟ್ಟಕ್ಕೆ ಬರುವುದು. ಲೈಂಗಿಕಶಕ್ತಿಯನ್ನು ಆಧ್ಯಾತ್ಮಿಕ ಶಕ್ತಿಯನ್ನಾಗಿ ಪರಿವರ್ತಿಸಿ. ಆ ಶಕ್ತಿ ಎಷ್ಟು ಬಲವಾಗಿದ್ದರೆ ಅದರಿಂದ ಅಷ್ಟೊಂದು ಹೆಚ್ಚು ಕೆಲಸವನ್ನು ಮಾಡಬಹುದು. ವೇಗದಿಂದ ಹರಿಯುವ ನೀರಿನ ಪ್ರವಾಹದಿಂದ ಮಾತ್ರ ವಿದ್ಯುಚಊಇ್ಕ86ದ್ಕ್ತಿಯನ್ನು ಉತ್ಪತ್ತಿ ಮಾಡಬಹುದು.

\textbf{ಶ್ರದ್ಧೆ}: ಭಾವನೆಯನ್ನು ನಾವು ಮತ್ತೊಮ್ಮೆ ತರಬೇಕಾಗಿದೆ. ಒಬಪ್ ನಿಗೂ ಮತ್ತೊಬಪ್ನಿಗೂ ಇರುವ ವ್ಯತ್ಯಾಸಕ್ಕೆಲ್ಲ ಅವರಲ್ಲಿರುವ ಶ್ರದ್ಧೆಯ ಹೆಚ್ಚು ಕಡಿಮೆಯಲ್ಲದೆ ಬೇರೆಯಲ್ಲ. ಯಾವುದು ಒಬಪ್ನನ್ನು ಮಹಾಪುರುಷನನ್ನಾಗಿ ಮಾಡುವುದೊ, ಮತ್ತೊಬಪ್ನನ್ನು ಅತಿ ಕನಿಷ್ಠ ವ್ಯಕ್ತಿಯನ್ನಾಗಿ ಮಾಡುವುದೊ ಅದೇ ಶ್ರದ್ಧೆ. ಯಾರು ತಾನು ಯಾವ ಕೆಲಸಕ್ಕೂ ಪ್ರಯೋಜನವಿಲ್ಲದವನೆಂದು ಹಗಲು ರಾತ್ರಿ ಚಿಂತಿಸುತ್ತಿರುವರೊ ಅವರು ಅಪ್ರಯೋಜಕರಾಗಿಯೇ ಆಗುವರು. ಒಬಪ್ ಹಗಲು ರಾತ್ರಿ ತಾನು ಅತಿ ದುಃಖಿ, ದೀನ ನಿಷ್ಪ್ರಯೋಜಕ ಎಂದು ಭಾವಿಸುತ್ತಿದ್ದರೆ ಹಾಗೆಯೇ ಆಗುವನು. ನಾವು ಸರ್ವಶಕ್ತನಾದ ಭಗವಂತನ ಮಕ್ಕಳು, ನಾವು ಅನಂತಾತ್ಮನ ಪಾವಿತ್ರ್ಯದ ಕಿಡಿಗಳು. ನಾವು ಹೇಗೆ ನಿಷ್ಪ್ರಯೋಜಕರಾಗಬಲ್ಲೆವು? ನಾವೇ ಸರ್ವವೂ. ನಾವು ಏನನ್ನು ಬೇಕಾದರೂ ಮಾಡಲು ಸಿದ್ಧವಾಗಿರುವೆವು. ನಾವೆಲ್ಲವನ್ನೂ ಸಾಧಿಸುವೆವು. ನಮ್ಮ ಪೂರ್ವಿಕರಲ್ಲಿ ಇಂತಹ ಆತ್ಮಶ್ರದ್ಧೆ ಇತ್ತು. ನಾಗರಿಕತೆಯ ಮೆರವಣಿಗೆಯಲ್ಲಿ ಅವರನ್ನು ಮುಂದೆ ನೂಕಿದ್ದು ಈ ಆತ್ಮಾಭಿ ಮಾನ. ಈಗ ಅವನತಿಗೆ ಬಂದಿದ್ದರೆ, ನಮ್ಮಲ್ಲಿ ಲೋಪದೋಷಗಳಿದ್ದರೆ, ಎಂದು ನಮ್ಮವರು ನಮ್ಮಲ್ಲಿ ವಿಶ್ವಾಸವನ್ನು ಕಳೆದುಕೊಂಡರೊ ಅಂದಿನಿಂದ ಅದು ಪ್ರಾರಂಭವಾಯಿತು. ಆದಕಾರಣವೇ ಜೀವದಾನ ಮಾಡುವಂತಹ, ಮಹೋನ್ನತ ವಾದ, ನಮಗೆ ಶ್ರೇಯಸ್ಕರವಾದ ಶ್ರದ್ಧೆಯ ಸಂದೇಶವನ್ನು ನಿಮ್ಮ ಮಕ್ಕಳಿಗೆ ಬಾಲ್ಯ ದಿಂದಲೂ ಸಾರಿ.

\textbf{ಶೀಲ}: ನಿಮಗೆ ಇಂದು ಬೇಕಾಗಿರುವುದು ನಿಮ್ಮ ಇಚಾಊಇ್ಕ86ದ್ಶಕ್ತಿಯನ್ನು ವೃದ್ಧಿಮಾಡ ಬಲ್ಲಂತಹ ಚಾರಿತ್ರ್ಯ. ನೀವು ನಿಮ್ಮ ಇಚಾಊಇ್ಕ86ದ್ಶಕ್ತಿಯನ್ನು ರೂಢಿಸಿಕೊಂಡಂತೆ ಅದು ನಿಮ್ಮನ್ನು ಮೇಲೆ ಮೇಲಕ್ಕೆ ಒಯ್ಯುವುದು. ವಜ್ರದಂತಹ ಕಷ್ಟದ ಕೋಟೆಗಳನ್ನು ಕೂಡ ಸೀಳಿಕೊಂಡು ಹೋಗುವುದು ಚಾರಿತ್ರ್ಯ. ಮನುಷ್ಯನ ಶೀಲ ಎಂದರೆ ಅವನ ಸ್ವಭಾವದ ಮೊತ್ತ. ನಾವು ನಮ್ಮ ಆಲೋಚನೆ ಮಾಡಿದಂತೆ ಇರುವೆವು. ಆದ ಕಾರಣವೆ ನೀವು ಏನನ್ನು ಆಲೋಚಿಸುವಿರೋ ಅದರ ವಿಷಯದಲ್ಲಿ ಬಹಳ ಜೋಪಾನವಾಗಿರಿ. ನಾವು ಮಾಡುವ ಪ್ರತಿಯೊಂದು ಕೆಲಸವೂ, ನಮ್ಮ ದೇಹದ ಪ್ರತಿಯೊಂದು ಚಲನವಲನವೂ, ನಾವು ಮಾಡುವ ಪ್ರತಿಯೊಂದು ಆಲೋ ಚನೆಯೂ, ನಮ್ಮ ಮನಸ್ಸಿನ ಮೇಲೆ ಒಂದು ಪ್ರಭಾವವನ್ನು ಬಿಡುವುದು. ಈ ಪ್ರಭಾವಗಳ ಮೊತ್ತವೇ ನಾವು ಪ್ರತಿ ಕ್ಷಣ ಹೇಗಿರುವೆವೊ ಅದಕ್ಕೆ ಕಾರಣ. ಪ್ರತಿಯೊಬಪ್ನ ಶೀಲವೂ ಅವನ ಮನಸ್ಸಿನಲ್ಲಿರುವ ಪ್ರತಿಕ್ರಿಯೆಯ ಮೊತ್ತದಿಂದ ನಿರ್ಧರಿಸಲ್ಪಟ್ಟಿದೆ. ಒಳ್ಳೆಯ ಅನುಭವಗಳಿದ್ದರೆ ಶೀಲ ಶುದ್ಧವಾಗುವುದು. ಕೆಟ್ಟ ಅನುಭವಗಳಿದ್ದರೆ ಶೀಲ ಕೆಡುವುದು.

ಮನಸ್ಸಿನ ಮೇಲೆ ಹಲವು ಭಾವನೆಗಳು ನಾಟಿದ್ದರೆ ಅವುಗಳೆಲ್ಲ ಒಟ್ಟಿಗೆ ಸೇರಿ ಒಂದು ಸ್ವಭಾವವಾಗುವುದು. ಕೆಟ್ಟ ಸ್ವಭಾವದಿಂದ ಪಾರಾಗಬೇಕಾದರೆ ಒಳ್ಳೆಯ ಸ್ವಭಾವವನ್ನು ರೂಢಿಸಿಕೊಳ್ಳಬೇಕು. ನಮ್ಮಲ್ಲಿರುವ ಕೆಟ್ಟ ಅಭ್ಯಾಸಗಳನ್ನೆಲ್ಲ ಒಳ್ಳೆಯ ಅಭ್ಯಾಸದಿಂದ ಗೆಲ್ಲಬೇಕಾಗಿದೆ. ನಮ್ಮ ಮನಸ್ಸಿನಲ್ಲಿರುವ ಹೀನ ಭಾವನೆ ಗಳನ್ನು ಪರಿವರ್ತನೆ ಮಾಡಬೇಕಾದರೆ ಯಾವಾಗಲೂ ಒಳ್ಳೆಯ ಕೆಲಸವನ್ನು ಮಾಡಿ, ಒಳ್ಳೆಯ ಆಲೋಚನೆಯನ್ನು ಮಾಡಿ.

ನಮ್ಮ ಶೀಲವನ್ನು ರೂಪಿಸುವುದರಲ್ಲಿ ಸುಖ ದುಃಖ ಎರಡಕ್ಕೂ ಸಮನಾದ ಪಾತ್ರವಿದೆ. ಕೆಲವು ವೇಳೆ ದುಃಖ ಸುಖಕ್ಕಿಂತ ದೊಡ್ಡ ಗುರು. ಪ್ರಪಂಚದ ಶ್ರೇಷ್ಠ ವ್ಯಕ್ತಿಗಳನ್ನು ಪರಿಶೀಲಿಸಿದರೆ, ಅವರಲ್ಲಿ ಬಹುಪಾಲು ಜನರಿಗೆ ಸುಖಕ್ಕಿಂತ ಹೆಚ್ಚಾಗಿ ದುಃಖ, ಐಶ್ವರ್ಯಕ್ಕಿಂತ ಹೆಚ್ಚಾಗಿ ಬಡತನ, ಹೊಗಳಿಕೆಗಿಂತ ಹೆಚ್ಚಾಗಿ ತೆಗಳಿಕೆ, ಬುದ್ಧಿ ಕಲಿಸಿವೆ ಎಂಬುದು ಗೊತ್ತಾಗುವುದು. ಭೋಗದ ತೊಡೆಯಲ್ಲಿ, ಗುಲಾಬಿಯ ಹಾಸಿಗೆಯ ಮೇಲೆ, ಒಂದು ಕಣ್ಣೀರನ್ನು ಹಾಕದೆ ಯಾರು ಮಹಾಪುರುಷರಾಗಿರು ವರು.

\textbf{ನಿಸರ್ಗದ ನಿಕಟ ಸಂಬಂಧ}: ನೀವು ಉಪನಿಷತ್ತಿನಲ್ಲಿ ಬರುವ ಕಥೆಗಳನ್ನು ಓದಿ? ನಾನು ಒಂದು ಕಥೆಯನ್ನು ನಿಮಗೆ ಹೇಳುತ್ತೇನೆ. ಸತ್ಯಕಾಮನೆಂಬ ಶಿಷ್ಯ ತನ್ನ ಗುರುವಿನ ಮನೆಗೆ ಗುರುಕುಲ ವಾಸಕ್ಕಾಗಿ ಹೋದ. ಗುರು ಅವನಿಗೆ ಕೆಲವು ದನಗಳನ್ನು ಕೊಟ್ಟು ಮೇಯಿಸಲು ಕಾಡಿಗೆ ಕಳುಹಿಸಿದನು. ಹಲವು ತಿಂಗಳುಗಳಾ ಯಿತು. ದನಗಳ ಸಂಖ್ಯೆ ಮೊದಲಿಗೆ ಎರಡರಷ್ಟು ಆಯಿತು. ಆಗ ಸತ್ಯಕಾಮ ಗುರು ವಿನ ಬಳಿಗೆ ಹೋಗುವುದಕ್ಕೆ ಸಮಯವಾಯಿತೆಂದು ಭಾವಿಸಿದನು. ಶಿಷ್ಯ ಗುರುವಿನ ಸಮೀಪಕ್ಕೆ ಬಂದ. ಅವನ ಮೊಗದಲ್ಲಿ ಬ್ರಹ್ಮಸಾಕ್ಷಾತ್ಕಾರ ಮಾಡಿಕೊಂಡವನ ತೇಜಸ್ಸನ್ನು ನೋಡಿದ. ಈ ಕಥೆಯಿಂದ ನಾವು ಕಲಿಯುವ ನೀತಿಯೆ ನಿಸರ್ಗದ ನಿಕಟ ಸಂಬಂಧದಿಂದಲೇ ಅವನು ಇವುಗಳನ್ನೆಲ್ಲ ಕಲಿತಿದ್ದ ಎಂಬುದು.

\textbf{ಗುರುಕುಲ ವಾಸ}: ಗುರುಗೃಹ ವಾಸವೇ ಶಿಕ್ಷಣದ ನನ್ನ ಆದರ್ಶ. ಗುರುವಿನ ಜೀವನದ ನಿಕಟ ಪರಿಚಯವಿಲ್ಲದೆ ಯಾವ ವಿದ್ಯಾಭ್ಯಾಸವೂ ಸಾಧ್ಯವಿಲ್ಲ. ಒಬಪ್ ತನ್ನ ಬಾಲ್ಯದಿಂದಲೂ ಪರಿಶುದ್ಧವಾದ ಜೀವನ ಯಾರಲ್ಲಿ ನಂದಾದೀವಿಗೆಯಂತೆ ಬೆಳಗುತ್ತಿದೆಯೊ ಅಂತಹ ಗುರುವಿನ ಸಮೀಪದಲ್ಲಿ ಇರಬೇಕು. ನಮ್ಮ ದೇಶದಲ್ಲಿ ವಿದ್ಯೆಯನ್ನು ದಾನ ಮಾಡುವವನು ಯಾವಾಗಲೂ ತ್ಯಾಗಿಯಾಗಿದ್ದನು. ಈಗಲೂ ಕೂಡ ವಿದ್ಯಾದಾನವನ್ನು ಮಾಡುವ ಕೆಲಸ ತ್ಯಾಗಿಗಳಿಗೇ ಬರಬೇಕಾಗಿದೆ.

ಭರತಖಂಡದ ಹಿಂದಿನ ವಿದ್ಯಾಭ್ಯಾಸದ ಕ್ರಮ ಈಗಿನಂತೆ ಇರಲಿಲ್ಲ. ವಿದ್ಯಾರ್ಥಿ ಯಾವ ಫೀಜನ್ನೂ ಕೊಡಬೇಕಾಗಿರಲಿಲ್ಲ. ಜ್ಞಾನ ಅತ್ಯಂತ ಪವಿತ್ರ ವಾದುದು, ಅದನ್ನು ಯಾರೂ ಮಾರಬಾರದು ಎಂದು ಭಾವಿಸಿದ್ದರು. ಜ್ಞಾನವನ್ನು ಉದಾರವಾಗಿ ಏನನ್ನೂ ಕೇಳದೆ ಕೊಡಬೇಕು, ಗುರುಗಳು ವಿದ್ಯಾರ್ಥಿಗಳನ್ನು ಉಚಿತ ವಾಗಿ ಸ್ವೀಕರಿಸುತ್ತಿದ್ದರು. ಅನೇಕ ಗುರುಗಳು ಶಿಷ್ಯರಿಗೆ ಅನ್ನಾಹಾರಾದಿಗಳನ್ನು ಕೂಡ ಕೊಡುತ್ತಿದ್ದರು. ಇಂತಹ ಗುರುಗಳಿಗೆ ಸಹಾಯ ಮಾಡಲು ಶ್ರೀಮಂತರು ದಾನ ಮಾಡುತ್ತಿದ್ದರು. ಇದರಿಂದ ವಿದ್ಯಾರ್ಥಿಗಳನ್ನು ಅವರೇ ನೋಡಿಕೊಳ್ಳಬೇಕಾ ಗಿತ್ತು. ಗುರು ಶಿಷ್ಯನಿಂದ ದ್ರವ್ಯ ಹೆಸರು ಕೀರ್ತಿ ಮುಂತಾದ ಯಾವುದನ್ನೂ ನಿರೀಕ್ಷಿಸಬಾರದಾಗಿತ್ತು. ಕೇವಲ ಪ್ರೀತಿಯಿಂದ, ಮಾನವಕೋಟಿಯ ಮೇಲಿನ ಪ್ರೀತಿಯಿಂದ ಮಾತ್ರ ಅದು ಪ್ರೇರಕವಾಗಬೇಕಾಗಿತ್ತು.

ಹಿಂದಿನ ಕಾಲದ ಶಿಷ್ಯರು ಗುರುವಿನ ಮನೆಗೆ ಕೈಯಲ್ಲಿ ಸೌದೆಯನ್ನು ತೆಗೆದು ಕೊಂಡು ಹೋಗುತ್ತಿದ್ದರು. ಗುರು ಅವನ ಯೋಗ್ಯತೆಯನ್ನು ಪರೀಕ್ಷಿಸಿದ ಮೇಲೆ ಅವನಿಗೆ ವೇದವನ್ನು ಹೇಳಿಕೊಡುತ್ತಿದ್ದನು. ಶಿಷ್ಯನಿಗೆ ಗುರುವಿನ ಮೇಲೆ ಶ್ರದ್ಧೆ, ದೈನ್ಯತೆ, ವಿಧೇಯತೆ ಗೌರವ ಭಕ್ತಿಗಳಿಲ್ಲದೇ ಇದ್ದರೆ ಯಾವ ಜ್ಞಾನವೂ ಶಿಷ್ಯನಲ್ಲಿ ವೃದ್ಧಿಯಾಗಲಾರದು. ಎಲ್ಲಿ ಈ ಆದರ್ಶವನ್ನು ಮರೆತಿರುವರೋ ಅಲ್ಲಿ ಗುರು ಕೇವಲ ಉಪನ್ಯಾಸಕನಾಗಿರುವನು. ಪಾಠ ಹೇಳುವವನು ತನ್ನ ಫೀಜನ್ನು ನಿರೀಕ್ಷಿಸು ವನು. ಶಿಷ್ಯ ಗುರು ಹೇಳುವ ವಿಷಯವನ್ನು ತನ್ನ ತಲೆಯಲ್ಲಿ ತುಂಬಿಕೊಳ್ಳುವನು. ಇದಾದ ಮೇಲೆ ಅವರವರ ದಾರಿಯನ್ನು ಅವರು ಹಿಡಿದು ಹೋಗುವರು.

ನಿಜವಾದ ಗುರು ಶಿಷ್ಯನಿರುವ ಮೆಟ್ಟಲಿಗೆ ತಕ್ಷಣವೇ ಇಳಿದುಬರುವನು. ಶಿಷ್ಯನ ಹೃದಯದಲ್ಲಿರುವುದನ್ನು ತಿಳಿದುಕೊಳ್ಳುವನು. ಇಂತಹ ಗುರು ಮಾತ್ರ ನಿಜವಾಗಿ ಬೋಧಿಸಬಲ್ಲ, ಇತರರಲ್ಲ. ಶಿಷ್ಯನಲ್ಲಿ ಹೃದಯ ಪರಿಶುದ್ಧವಾಗಿರಬೇಕು, ಜ್ಞಾನಾ ಕಾಂಕ್ಷೆ ಇರಬೇಕು, ಸತತ ಪ್ರಯತ್ನವಿರಬೇಕು. ಕಾಯಾ ವಾಚಾ ಮನಸಾ ಅವನು ಪರಿಶುದ್ಧನಾಗಿರಬೇಕು. ಜ್ಞಾನಾಕಾಂಕ್ಷೆ, ಅದೊಂದು ಪೂರ್ವದಿಂದ ಬಂದ ನಿಯಮ. ನಮಗೆ ಏನು ಬೇಕೋ ಅದು ದೊರಕುವುದು. ನಮ್ಮ ಹೃದಯದಲ್ಲಿ ಯಾವುದರ ಮೇಲೆ ನಮಗೆ ಅಭೀಪ್ಸೆ ಇದೆಯೋ ಅದಲ್ಲದೆ ಬೇರೆ ಯಾವುದೂ ನಮಗೆ ಬರುವುದಿಲ್ಲ. ಯಾವ ವಿದ್ಯಾರ್ಥಿ ಸತತ ಪ್ರಯತ್ನ ಮಾಡುವನೊ ಅವನು ಜಯ ಗಳಿಸುವುದರಲ್ಲಿ ಸಂದೇಹವಿಲ್ಲ.

\textbf{ಮಾನಸಿಕ ಶಾಸ್ತ್ರದ ಮಾರ್ಗ}: ಬಲಾತ್ಕಾರವಾಗಿ ಒಂದು ವಿಷಯವನ್ನು ಒಬಪ್ನಿಗೆ ಇಚೆಊಇ್ಕ86ದ್ಯಿಲ್ಲದೇ ಇದ್ದರೂ ಕಲಿಸುವುದನ್ನು ಬಿಡಬೇಕಾಗಿದೆ. ಕತ್ತೆಯನ್ನು ಎಷ್ಟು ತೊಳೆದರೂ ಅದು ಕುದುರೆಯಾಗುವುದಿಲ್ಲ. ಯಾರೂ ಮತ್ತೊಬಪ್ರಿಗೆ ಕಲಿಸ ಲಾರರು. ಗುರು ತಾನು ಶಿಷ್ಯನಿಗೆ ಕಲಿಸುತ್ತೇನೆ ಎಂದು ಎಲ್ಲವನ್ನೂ ಹಾಳುಮಾಡು ವನು. ಪ್ರತಿಯೊಬಪ್ನಲ್ಲಿಯೂ ಆಗಲೆ ಜ್ಞಾನವಿದೆಯೆಂದು ವೇದಾಂತ ಸಾರುವುದು. ಒಬಪ್ ಸಣ್ಣ ಹುಡುಗನಲ್ಲಿ ಕೂಡ ಅದು ಆಗಲೆ ಇದೆ. ಅದನ್ನು ಜಾಗ್ರತಗೊಳಿಸ ಬೇಕಾಗಿದೆ. ಗುರು ಮಾಡುವುದು ಇಷ್ಟೇ ಕೆಲಸವನ್ನು. ನೀವು ಒಂದು ಗಿಡವನ್ನು ಬೆಳೆಸುವುದಕ್ಕಿಂತ ಹೆಚ್ಚಾದ ರೀತಿಯಲ್ಲಿ ಮಗುವಿಗೆ ಹೇಳಿಕೊಡಲಾರಿರಿ. ಆ ಗಿಡ ಹಾಳಾಗದಂತೆ ನೋಡಿಕೊಳ್ಳಬಹುದು. ಅದಕ್ಕೆ ನೀವು ಸಹಾಯವನ್ನು ಮಾತ್ರ ಮಾಡಬಲ್ಲಿರಿ. ಇರುವ ಆತಂಕಗಳನ್ನು ತೆಗೆದರೆ ಜ್ಞಾನ ಸ್ವಭಾವತಃ ಬರುವುದು. ಗಿಡದ ಸುತ್ತಲೂ ಮಣ್ಣನ್ನು ಸ್ವಲ್ಪ ಅಗತೆ ಮಾಡಿ. ಗಿಡದ ಸುತ್ತಲೂ ಒಂದು ಬೇಲಿಯನ್ನು ಹಾಕಿ. ಇತರರು ಅದನ್ನು ಕೀಳದಂತೆ ನೋಡಿಕೊಳ್ಳಿ. ಇಲ್ಲಿಗೆ ನಿಮ್ಮ ಕೆಲಸ ಮುಗಿಯಿತು. ನೀವು ಜಾಸ್ತಿ ಮಾಡಲಾರಿರಿ. ಉಳಿದದ್ದು ಆ ಗಿಡದಲ್ಲಿ ಯಾವ ಚೈತನ್ಯ ಆಗಲೆ ಇದೆಯೊ ಅದರಿಂದ ವ್ಯಕ್ತವಾಗುವುದು. ಇದರಂತೆಯೇ ಮಗುವಿನ ವಿದ್ಯಾಭ್ಯಾಸ ಕೂಡ. ಮಗು ತನಗೆ ತಾನೇ ಕಲಿಯುವುದು.

\textbf{ಇಂದಿನ ಆವಶ್ಯಕತೆಗೆ ಸ್ವಾಮೀಜಿ ಯೋಜನೆ}: ಹೊರಗಿನವರ ಆಳ್ವಿಕೆಗೆ ಒಳ ಗಾಗದೆ, ನಮಗೆ ಸೇರಿದ ಹಲವು ಶಾಸ್ತ್ರಗಳನ್ನು ಕಲಿಯಬೇಕು. ಇದರ ಜೊತೆಗೆ ಇಂಗ್ಲಿಷ್ಭಾಷೆ ಮತ್ತು ಪಾಶ್ಚಾತ್ಯ ವಿಜ್ಞಾನವನ್ನು ನಾವು ತಿಳಿಯಬೇಕು. ನಮಗೆ ಯಾಂತ್ರಿಕಶಿಕ್ಷಣ ಬೇಕು. ನಮ್ಮ ಕೈಗಾರಿಕೆ ಅಭಿವೃದ್ಧಿಯಾಗುವುದಕ್ಕೆ ಎಲ್ಲಾ ಸಲ ಕರಣೆಗಳು ಬೇಕು. ಸುಮ್ಮನೆ ಸರ್ಕಾರಿ ಕೆಲಸವನ್ನು ಹುಡುಕಿಕೊಂಡು ಹೋಗುವ ಬದಲು ಇತರ ಉದ್ಯಮಗಳ ಮೂಲಕ ತನ್ನ ಜೀವನೋಪಾಯಕ್ಕೆ ಸಾಕಾಗುವಷ್ಟನ್ನು ಈಗ ಸಂಪಾದಿಸಿ, ಕಷ್ಟಕಾಲಕ್ಕೆ ಸ್ವಲ್ಪ ಹಣವನ್ನು ಕೂಡಿಡುವುದಕ್ಕೂ ಸಹಾಯ ಆಗಬೇಕು.

ಹುಡುಗರು ಓದುವಂತಹ ಯಾವ ಪುಸ್ತಕವೂ ನಮ್ಮಲ್ಲಿ ಇಲ್ಲ. ರಾಮಾಯಣ ಮಹಾಭಾರತ ಮತ್ತು ಉಪನಿಷತ್ತುಗಳಿಂದ ಕಥೆಗಳನ್ನು ಸಂಗ್ರಹಿಸಬೇಕು, ಇವನ್ನು ಸುಲಭವಾದ ಭಾಷೆಯಲ್ಲಿ ಬರೆದು ಮಕ್ಕಳಿಗೆ ಕೊಡಬೇಕು.

ಪ್ರತಿಯೊಬ್ಬನ ಮನೆಯ ಬಾಗಿಲಿಗೂ ಶ್ರೇಷ್ಠವಾದ ಭಾವನೆಗಳನ್ನು ಕೊಡು ವಂತಹ ಒಂದು ಯಂತ್ರ ಕೆಲಸ ಮಾಡುವಂತೆ ಮಾಡಬೇಕೆಂಬುದೇ ನನ್ನ ಜೀವನದ ಹಿರಿಯಾಸೆ. ಅನಂತರ ಸ್ತ್ರೀ ಪುರುಷರು ತಮಗೆ ತೋಚಿದಂತೆ ಮಾಡಲಿ. ಜೀವನದ ಅತಿ ಗಾಢ ಸಮಸ್ಯೆಗಳ ಮೇಲೆ ನಮ್ಮ ಹಿರಿಯರು ಮತ್ತು ಪ್ರಪಂಚದ ಇತರರು ಏನು ವಿಚಾರ ಮಾಡಿರುವರು ಎಂಬುದನ್ನು ತಿಳಿಯಲಿ. ಈಗ ಇತರರು ಏನು ಮಾಡುತ್ತಿರುವರು ಎಂಬುದನ್ನು ವಿಶೇಷವಾಗಿ ನೋಡಲಿ. ಅನಂತರ ತಾವು ಏನು ಮಾಡಬೇಕೆಂಬುದನ್ನು ನಿಶ್ಚಯಿಸಲಿ. ನಾವು ರಾಸಾಯನಿಕ ವಸ್ತುಗಳನ್ನು ಸೇರಿಸಬೇಕಾಗಿದೆ. ಅನಂತರ ಪ್ರಕೃತಿ ತನ್ನ ನಿಯಮಾನುಸಾರ ಹರಳುಗಳನ್ನಾಗಿ ಮಾಡುವುದು.

ಜನಾಂಗದ ಆಧ್ಯಾತ್ಮಿಕ ಮತ್ತು ಲೌಕಿಕ ಶಿಕ್ಷಣದ ಮೇಲೆ ನಮಗೆ ಒಂದು ಹತೋಟಿ ಇರಬೇಕು. ಇದನ್ನು ನೀವು ಕನಸು ಕಾಣಬೇಕು, ಈ ವಿಷಯವನ್ನು ಕುರಿತು ಮಾತನಾಡಬೇಕು, ನೀವು ಈ ವಿಷಯವನ್ನು ಕುರಿತು ಆಲೋಚಿಸಬೇಕು, ಇದನ್ನು ಅನುಷ್ಠಾನಕ್ಕೆ ತರಬೇಕು. ಅಲ್ಲಿಯವರೆಗೆ ನಮ್ಮ ಜನಾಂಗಕ್ಕೆ ವಿಮೋಚನೆ ಇಲ್ಲ. ನಮ್ಮ ಕೈಯಲ್ಲಿ ಇಡೀ ಜನಾಂಗದ ಆಧ್ಯಾತ್ಮಿಕ ಮತ್ತು ಲೌಕಿಕ ಶಿಕ್ಷಣ ಇರಬೇಕು, ಅದು ನಮ್ಮ ರೀತಿಯಲ್ಲಿ ಇರಬೇಕು. ಸಾಧ್ಯವಾದ ಮಟ್ಟಿಗೆ ಅವರ ಸ್ವಭಾವವನ್ನು ಅನುಸರಿಸಬೇಕು. ಇದೇನೋ ಒಂದು ಬೃಹತ್ ಯೋಜನೆ. ಇದೆಂದಾದರೂ ಕಾರ್ಯ ಕಾರಿಯಾಗುವುದೊ ನಮಗೆ ಗೊತ್ತಿಲ್ಲ. ಆದರೆ ನಾವು ಇದರ ಪ್ರಾರಂಭವನ್ನು ಮಾಡಬೇಕು.

ನಮಗೆ ಒಂದು ದೇವಸ್ಥಾನ ಇರಬೇಕು. ಹಿಂದುಗಳಿಗೆ ಯಾವ ಕೆಲಸವನ್ನು ಮಾಡಬೇಕಾದರೂ ಧರ್ಮ ಮೊದಲು. ಅದನ್ನು ಒಂದು ಜಾತ್ಯತೀತವಾದ ದೇವ ಸ್ಥಾನವನ್ನಾಗಿ ಮಾಡಬೇಕು. ಅಲ್ಲಿ ಎಲ್ಲಾ ಪಂಗಡಗಳ ಮುಖ್ಯ ಚಿಹ್ನೆಯಾದ ‘ಓಂಕಾರ’ ಒಂದೇ ಇರಬೇಕು. ಅಲ್ಲಿ ಎಲ್ಲಾ ಮತಗಳಲ್ಲಿಯೂ ಇರುವ ಸಾಮಾನ್ಯ ತತ್ತ್ವಗಳನ್ನು ಮಾತ್ರ ಬೋಧಿಸಬೇಕು. ಇತರರು ಬಂದು ತಮ್ಮ ಸಂಪ್ರದಾಯಕ್ಕೆ ಸಂಬಂಧಪಟ್ಟ ಸಿದ್ಧಾಂತಗಳನ್ನು ಬೇಕಾದರೆ ಬೋಧಿಸಬಹುದು. ಆದರೆ ಇತರ ರನ್ನು ಎಂದಿಗೂ ಖಂಡಿಸಕೂಡದು. ಎರಡನೆಯದಾಗಿ ಲೌಕಿಕ ಮತ್ತು ಧಾರ್ಮಿಕ ವಿಷಯಗಳನ್ನು ಜನರಿಗೆ ಬೋಧಿಸುವ ಪ್ರಚಾರಕರನ್ನು ಇದು ತರಬೇತು ಮಾಡ ಬೇಕು. ನಾವು ಇದುವರೆಗೆ ಧರ್ಮವನ್ನು ಮನೆಯಿಂದ ಮನೆಗೆ ಕೊಟ್ಟೆವು. ಈಗ ಅದರ ಜೊತೆಗೆ ಲೌಕಿಕ ಶಿಕ್ಷಣವನ್ನು ಕೂಡ ಕೊಡೋಣ. ಅದನ್ನು ಸುಲಭವಾಗಿ ಮಾಡಬಹುದು. ಇಂತಹ ಪ್ರಚಾರಕರ ಮೂಲಕ ವಿಸ್ತಾರವಾಗಿ ಹಬ್ಬುವುದು. ಕ್ರಮೇಣ ದೇಶದ ಇತರ ಕಡೆಯೂ ಇಂತಹ ದೇವಸ್ಥಾನಗಳನ್ನು ಮಾಡುವೆವು. ಕೊನೆಗೆ ಭರತಖಂಡದಲ್ಲೆಲ್ಲ ಇದು ವ್ಯಾಪಿಸಬೇಕು. ಇದು ನನ್ನ ಯೋಜನೆ. ಇದು ಬೃಹತ್ತಾಗಿ ಕಾಣುವುದು. ಆದರೆ ಇದರ ಆವಶ್ಯಕತೆಯಿದೆ.

