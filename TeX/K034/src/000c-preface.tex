
\chapter*{ರಾಷ್ಟ್ರನಿರ್ಮಾಪಕ ಸ್ವಾಮಿ ವಿವೇಕಾನಂದ}

ಸ್ವಾಮಿ ವಿವೇಕಾನಂದರ ಹೆಸರನ್ನು ಕೇಳಿದಾಗ ನಮ್ಮ ಮನಸ್ಸಿನಲ್ಲಿ ಒಂದು ಚಿತ್ರ ಮೂಡುವುದು. ಅದೇ ದ್ವಾಪರ ಯುಗದ ಅಂತ್ಯದ ಕುರುಕ್ಷೇತ್ರ ಸಮರಾಂ ಗಣದ ದೃಶ್ಯ. ಅರ್ಜುನನ ಕೈಯಿಂದ ಗಾಂಢೀವ ಜಾರಿಹೋಗಿದೆ, ಕಣ್ಣುಗಳಲ್ಲಿ ನೀರು, ಅವನು ಕಿಂಕರ್ತವ್ಯಮೂಢನಾಗಿ ವಿಷಾದದಲ್ಲಿ ಮುಳುಗಿ ಹೋಗಿದ್ದಾನೆ. ಅವನ ಕ್ಷಾತ್ರಪೌರುಷದ ಅಗ್ನಿಕುಂಡದ ಮೇಲೆ ವೈರಾಗ್ಯದ ಬೂದಿ ಮುಸುಕಿ ಕೊಂಡಿದೆ. ಶ್ರೀಕೃಷ್ಣ ಪಾಂಚಜನ್ಯದಿಂದ ಊದಿ, ಅಗ್ನಿಕುಂಡವನ್ನು ಕಲಕುವಂತಹ ಪೌರುಷದ ಸಲಾಕೆಯನ್ನು ಉಪಯೋಗಿಸಿ ಅವನನ್ನು ಕಾರ್ಯೋನ್ಮುಖನನ್ನಾಗಿ ಮಾಡುವ ಈ ಮುಂದಿನ ಶ್ಲೋಕ ನಮ್ಮ ನೆನಪಿಗೆ ಬರುವುದು:

\begin{verse}
ಕುತಸ್ತ್ವಾ ಕಶ್ಮಲಮಿದಂ ವಿಷಮೇ ಸಮುಪಸ್ಥಿತಮ್​\\ಅನಾರ್ಯಜುಷ್ಟಮಸ್ವರ್ಗ್ಯಮಕೀರ್ತಿಕರಮರ್ಜುನ ॥
\end{verse}

\begin{verse}
ಕ್ಲೈಬ್ಯಂ ಮಾ ಸ್ಮ ಗಮಃ ಪಾರ್ಥ ನೈತತ್ತ್ವಯ್ಯುಪಪದ್ಯತೇ \\ಕ್ಷುದ್ರಂ ಹೃದಯ ದೌರ್ಬಲ್ಯಂ ತ್ಯಕ್ತ್ವೋತ್ತಿಷ್ಠ ಪರಂತಪ ॥
\end{verse}

“ಆರ್ಯರಿಗೆ ಅಯೋಗ್ಯವೂ ಸ್ವರ್ಗಗತಿಗೆ ವಿರೋಧವೂ, ಅಪಕೀರ್ತಿಯ ನ್ನುಂಟುಮಾಡುವುದೂ ಆದ ಈ ಮೋಹವು ಇಂತಹ ವಿಷಮ ಸಮಯದಲ್ಲಿ ನಿನಗೆ ಹೇಗೆ ಬಂತು? ಎಲೈ ಅರ್ಜುನ! ಷಂಡತನವನ್ನು ಹೊಂದಬೇಡ. ಇದು ನಿನಗೆ ಯೋಗ್ಯವಲ್ಲ. ಎಲೈ ಶತ್ರುತಾಪನನೆ ತುಚ್ಛವಾದ ಮನಸ್ಸಿನ ದೌರ್ಬಲ್ಯವನ್ನು ಬಿಟ್ಟು ಏಳು.”

ಅಲ್ಲಿ ಶ್ರೀಕೃಷ್ಣ ಅರ್ಜುನನೊಬ್ಬನನ್ನು ಕಾರ್ಯೋನ್ಮುಖನನ್ನಾಗಿ ಮಾಡುವನು ತನ್ನ ಗೀತೆಯ ಬೋಧನೆಯಿಂದ. ಸ್ವಾಮಿ ವಿವೇಕಾನಂದರು ಅರ್ಜುನನ ಸ್ಥಿತಿಗೆ ಇಳಿ ದಿದ್ದ ಭಾರತೀಯರನ್ನೆಲ್ಲ ತಮ್ಮ ಪಾಂಚಜನ್ಯಸದೃಶವಾದ ವಾಣಿಯಿಂದ ಜಾಗ್ರತರ ನ್ನಾಗಿ ಮಾಡಬೇಕಾಯಿತು.

ಎಂತಹ ಹೀನಸ್ಥಿತಿಗೆ ಇಳಿದಿದ್ದರು ನಮ್ಮ ಭಾರತೀಯರು ಆ ಕಾಲದಲ್ಲಿ! ಮೂವತ್ತು ಕೋಟಿ ಭಾರತೀಯರು ತಮಸ್ಸಿನ ಕೆಸರಿನಲ್ಲಿ ಸಿಕ್ಕಿಹೋಗಿದ್ದರು. ಇಂದಿನದನ್ನು ಎದುರಿಸುವುದಕ್ಕೆ ಧೈರ್ಯವಿಲ್ಲ; ಮುಂದಿನದನ್ನು ನಿರ್ಮಿಸುವುದಕ್ಕೆ ಉತ್ಸಾಹವಿಲ್ಲ; ನಮ್ಮನ್ನು ಮುತ್ತಿ ಕಿತ್ತು ತಿನ್ನುತ್ತಿರುವ ಕಷ್ಟಪರಂಪರೆಗಳೊಂದಿಗೆ ಹೋರಾಡುವ ಛಲವಿಲ್ಲ, ನಮಗಾಗುತ್ತಿರುವ ಅವಮಾನಗಳನ್ನು ಇದು ನಮ್ಮ ಹಣೆಯಬರಹ ಎಂದು ಕೈ ಕಟ್ಟಿಕೊಂಡು ಸ್ವೀಕರಿಸುವ ಷಂಡ ಮನೋಭಾವ, ತಮ್ಮಲ್ಲಿ ಶ್ರದ್ಧೆಯಿಲ್ಲ, ಭರತಖಂಡದ ಭವಿಷ್ಯದಲ್ಲಿ ಶ್ರದ್ಧೆಯಿಲ್ಲ. ಇಂತಹ ಪ್ರಕೃತಿಯ ಜನರ ಮಂದೆಯನ್ನು ಜಾಗ್ರತಗೊಳಿಸುವುದಕ್ಕೆ, ನರಕುರಿಗಳೆದೆಯಲ್ಲಿ ಸಿಂಹಪರಾಕ್ರಮವನ್ನು ಬೀರುವುದಕ್ಕೆ, ಜೀವನವನ್ನು ಒಂದು ವರದಂತೆ ಭಾವಿಸಿ ಸಾಹಸದಿಂದ ಬಾಳುವುದಕ್ಕೆ, ಸೃಷ್ಟಿಕರ್ತನೇ ಶಕ್ತಿ ಸಮುದ್ರದಿಂದ ಒಂದು ಭೀಮ ಅಲೆಯನ್ನು ಎಸೆದಂತಿದೆ ವಿವೇಕಾನಂದರ ವ್ಯಕ್ತಿತ್ವ.

ಭರತಖಂಡದಲ್ಲಿ ತತ್ತ್ವಕ್ಕೆ ಬರಗಾಲವಿಲ್ಲ, ಸಿದ್ಧಾಂತಗಳಿಗೆ ಕೊರತೆಯಿಲ್ಲ. ಹಿಂದಿನಿಂದ ಬಂದ ಆಚಾರ್ಯ ಪರಂಪರೆ ಶ್ರೇಷ್ಠವರ್ಗಕ್ಕೆ ಸೇರಿದ ತಾತ್ತ್ವಿಕ ಗ್ರಂಥಗಳನ್ನು ರಚಿಸಿದೆ. ನಮ್ಮ ಒಂದು ಕೊರತೆ ಎಂದರೆ ಅದು ಅನುಷ್ಠಾನ ಪ್ರಪಂಚದಲ್ಲಿ ಪ್ರವೇಶಿಸದೇ ಇದ್ದುದು. ಗೀತೆ ಉಪನಿಷತ್ತು ಮುಂತಾದುವುಗಳು ಪರಲೋಕದ ಸಂಧಾನಕ್ಕೆ ಎಂಬ ಭಾವ ಮೂಡಿಹೋಗಿತ್ತು ನಮ್ಮ ಜನರಲ್ಲಿ. ಸ್ವಾಮಿ ವಿವೇಕಾನಂದರು ಅದನ್ನು ಬಿಡಿಸಿದರು. ಗೀತೋಪನಿಷತ್ತಿನ ಸಂದೇಶ ನಮ್ಮ ಮುಕ್ತಿಗೆ ಎಷ್ಟು ಮುಖ್ಯವೋ ಅಷ್ಟೇ ಮುಖ್ಯ ಪ್ರಪಂಚದಲ್ಲಿ ಗೌರವದಿಂದ ಬಾಳಬೇಕೆಂದು ಆಶಿಸುವವನಿಗೆ, ನಮ್ಮ ಕಣ್ಣೆದುರಿಗೆ ಇರುವ ಪ್ರಪಂಚದಲ್ಲಿ ಹೇಗೆ ಧೀರರಾಗಿರಬೇಕು, ಸಾಹಸದಿಂದ ಬಾಳಬೇಕು, ಪೌರುಷದಿಂದ ಇರಬೇಕು ಎಂಬು ದನ್ನು ಅದು ಬೋಧಿಸುವುದು. ಗೀತೆ ಹುಟ್ಟಿದ್ದೇ ಇದಕ್ಕೆ. ಅರ್ಜುನನಿಗೆ ವೈರಾಗ್ಯ ಬಂದಾಗ ಶ್ರೀಕೃಷ್ಣ ಅದನ್ನು ಶ್ಲಾಘಿಸಿ ಹೋಗು ಹಿಮಾಲಯಕ್ಕೆ ತಪಸ್ಸು ಮಾಡುವು ದಕ್ಕೆ ಎನ್ನಲಿಲ್ಲ. ಅದಕ್ಕೆ ವ್ಯತಿರಿಕ್ತವಾಗಿ ಬಿಸುಟ ಗಾಂಢೀವವನ್ನು ಧರಿಸುವಂತೆ ಮಾಡುವನು, ಬಾಣಗಳ ಮಳೆಯನ್ನು ಅದರ ಮೂಲಕ ಸುರಿಸುವಂತೆ ಮಾಡು ವನು. ಬಂಧುಬಾಂಧವರು ಗುರುಹಿರಿಯರು ಯಾರಾದರೂ ಆಗಲಿ ಅಧರ್ಮಕ್ಕೆ ಸಹಾಯಕರಾಗಿ ನಿಂತರೆ ನಿಸ್ಸಂಕೋಚವಾಗಿ ಅವರನ್ನು ತರಿದುಬಿಡುವಂತೆ ಮಾಡು ವನು. ಇಂತಹ ಕರ್ಮವನ್ನು ಪ್ರೋತ್ಸಾಹಿಸುವಂತಹ ತತ್ತ್ವ ನಮಗೆ ಗೀತೆಯಲ್ಲಿ ದೊರಕುವುದು.

ಸ್ವಾಮಿ ವಿವೇಕಾನಂದರು, ಭವಜೀವಿಗಳಿಗೆ ದಾರಿ ಬೆಳಕಾಗಬಲ್ಲಂತಹ ಆಧ್ಯಾ ತ್ಮಿಕ ಸತ್ಯಗಳನ್ನು ನಮ್ಮ ಭರತಖಂಡ ನೀಡಿದ್ದರೂ, ನಮ್ಮ ನಾಡು ಈಗ ಇಂತಹ ಅಧೋಗತಿಗೆ ಬರುವುದಕ್ಕೆ ಕಾರಣಗಳನ್ನು ಕುರಿತು ಆಲೋಚಿಸತೊಡಗಿದರು. ಅದನ್ನು ಕಂಡುಹಿಡಿದರು, ಅದಕ್ಕೆ ಮದ್ದನ್ನೂ ಕಂಡು ಹಿಡಿದರು. ಶಸ್ತ್ರಚಿಕಿತ್ಸಕನ ಕೈಯಲ್ಲಿರುವ ಅಸ್ತ್ರದಂತಿದೆ ಅವರ ಮಾತು; ಹರಿತವಾಗಿ, ಕೆಲವು ವೇಳೆ ನೋವನ್ನು ಉಂಟುಮಾಡುತ್ತದೆ. ಆದರೆ ಇದು ರೋಗಿಯನ್ನು ಗುಣಮಾಡುವುದಕ್ಕೆ ವಿನಾ ವೃಥಾ ನೋವನ್ನು ಕೊಟ್ಟು ತಾವು ಆನಂದಿಸಬೇಕು ಎಂದಲ್ಲ.

ತಾತ್ತ್ವಿಕ ಪ್ರಪಂಚದಲ್ಲಿ ನಮ್ಮಷ್ಟು ದೂರಹೋದವರು ಮತ್ತೊಬ್ಬರಿಲ್ಲ. ಸರ್ವವೂ ಬ್ರಹ್ಮಮಯ ಎಂದು ಸಾರಿದೆವು. ಎಲ್ಲರಲ್ಲಿಯೂ ಒಂದೇ ಪರಮಾತ್ಮ ನೆಲೆಸಿರುವನು ಎಂದು ಸಾರಿದೆವು. ಆದರೆ ವ್ಯವಹಾರದಲ್ಲಿ ನೋಡಿದರೆ ಎಷ್ಟು ವ್ಯತ್ಯಾಸ! ಶೂದ್ರ ಪರೆಯ ಹತ್ತಿರ ಬಂದರೆ ನಾವು ಮೈಲಿಗೆ ಆಗಿಹೋಗುತ್ತೇವೆ ಎಂದು ಭಾವಿಸಿದೆವು. ಕೀಳುವರ್ಗದವರಿಗೆ ಯಾವ ಉತ್ತಮ ಬೋಧನೆಯನ್ನೂ ಕೊಡಲಿಲ್ಲ. ರಸಾನ್ನವನ್ನು ನಾವುಂಡು ಕದಾನ್ನವನ್ನು ಅವರಿಗೆ ಕೊಟ್ಟೆವು. ಹಾಗೆ ಮಾಡುವುದಕ್ಕೆ ನಾವು ವಿಚಿತ್ರವಾದ ವಿವರಣೆಗಳನ್ನು ಕಂಡುಹಿಡಿದೆವು. ನಮಗೂ ಅವರಿಗೂ ಇರುವ ಅಂತರ ಹೆಚ್ಚುವುದಕ್ಕೆ ಏನುಬೇಕೊ ಅದನ್ನು ಮಾಡಿದೆವು. ಅನ್ಯಧರ್ಮದವರು ಬಂದು ನಮ್ಮವರನ್ನೇ ತಮ್ಮ ಧರ್ಮಕ್ಕೆ ಸೇರಿಸಿಕೊಳ್ಳುತ್ತಿದ್ದರೆ ಅದನ್ನು ತಡೆಗಟ್ಟುವುದಕ್ಕೆ ಯಾವ ಪ್ರಯತ್ನವನ್ನೂ ಮಾಡಲಿಲ್ಲ. ನಮ್ಮ ಧರ್ಮ ಬಿಟ್ಟು ಅನ್ಯಧರ್ಮಕ್ಕೆ ಸೇರಿದ ಪ್ರತಿಯೊಬ್ಬ ವ್ಯಕ್ತಿಯಿಂದ ನಮ್ಮ ಧರ್ಮಕ್ಕೆ ಅವನ ಒಂದು ಸಂಖ್ಯೆ ಕಡಿಮೆ ಆಗಿರುವುದು ಮಾತ್ರವಲ್ಲ, ನಮ್ಮ ಧರ್ಮವನ್ನು ವಿರೋಧಿ ಸುವ ಒಬ್ಬ ವೈರಿ ಹೆಚ್ಚಿರುವನು ಎಂಬುದನ್ನು ಗಮನಿಸಿ ಎಂದು ಸ್ವಾಮಿಜಿಯವರು ನಮಗೆ ಅಂದು ಜಾಗರೂಕತೆ ಕೊಟ್ಟರು. ನಮ್ಮ ಅಸಹಿಷ್ಣುತೆ ನಿರ್ದಯತೆಗಳಿಂ ದಲೇ ನಮ್ಮ ದೇಶದಲ್ಲಿ ಬೃಹದಾಕಾರದ ಸಮಸ್ಯೆ ಏಳುವಂತೆ ಮಾಡಿಕೊಂಡಿರು ವೆವು. ಮೇಲಿರುವವರು ಕೆಳಗೆ ಇರುವವರನ್ನು ಮೇಲೆತ್ತಲು ಬರಬೇಕು. ಅನ್ಯ ಧರ್ಮಕ್ಕೆ ಪಾಲಾದವರು ಪುನಃ ನಮ್ಮ ಧರ್ಮಕ್ಕೆ ಬರಲು ಯತ್ನಿಸಿದರೆ ಅವರಿಗಾಗಿ ನಮ್ಮ ಧರ್ಮದ ಬಾಗಿಲುಗಳನ್ನು ತೆರೆಯಬೇಕು ಎಂದರು ವಿವೇಕಾನಂದರು.

ನಮ್ಮ ಶಾಸ್ತ್ರಗಳಲ್ಲಿ ಮುಖ್ಯಾಂಶಗಳಿವೆ ಮತ್ತು ಗೌಣ ಅಂಶಗಳಿವೆ. ಯಾವು ದಕ್ಕೆ ಹೆಚ್ಚು ಬೆಲೆ ಕೊಡಬೇಕು, ಯಾವುದಕ್ಕೆ ಕಡಿಮೆ ಬೆಲೆಯನ್ನು ಕೊಡಬೇಕು ಎಂಬುದನ್ನು ಮರೆತೆವು. ಕರಟವನ್ನು ಹಿಡಿದುಕೊಂಡೆವು. ಅದು ಯಾವುದನ್ನು ರಕ್ಷಿಸುವುದಕ್ಕಾಗಿ ಇದ್ದಿತೊ ಆ ತಿರುಳನ್ನು ಮರೆತೆವು. ಒಂದು ಸಲ ಸ್ವಾಮೀಜಿ ಹೀಗೆ ಹೇಳುತ್ತಾರೆ: “ಇಂದು ನಮ್ಮ ಧರ್ಮವೆಲ್ಲಾ ಅಡಿಗೆ ಮನೆ ಪ್ರವೇಶಿಸಿದೆ. ಅಡಿಗೆ ಮಾಡುವ ಪಾತ್ರೆಗಳೇ ನಮ್ಮ ಸಾಲಿಗ್ರಾಮಗಳಾಗಿವೆ. ನಾನು ಮಡಿ ನೀವು ಮೈಲಿಗೆ. ನೀವು ಯಾರೂ ನನ್ನನ್ನು ಮುಟ್ಟಬೇಡಿ ಎಂಬುದೇ ಗಾಯತ್ರಿ ಮಂತ್ರವಾಗಿದೆ. ಇದು ಹೀಗೆಯೇ ಮುಂದುವರಿದರೆ ನಾವೆಲ್ಲ ಹುಚ್ಚರ ಆಸ್ಪತ್ರೆಗೆ ಸೇರಬೇಕಾಗುವುದು. ಮಂಗಳಾರತಿಯನ್ನು ಮೂರುಬಾರಿ ಮಾಡುವುದೇ, ಐದುಬಾರಿಯೆ, ಏಳುಬಾರಿಯೆ ಎಂಬ ವಿಷಯದಲ್ಲಿ ಮಹಾ ಚರ್ಚೆಗಳಾಗಿವೆ. ಹೊರಗೆ ಹೋಗಿಬಂದರೆ ಕೈಯ್ಯನ್ನು ಎಷ್ಟು ಬಾರಿ ತೊಳೆದುಕೊಳ್ಳಬೇಕು, ಎಂತಹ ಮಣ್ಣನ್ನು ಆಗ ಉಪಯೋಗಿಸಬೇಕು ಎಂಬ ವಿಷಯದಲ್ಲಿ ಉದ್​ಗ್ರಂಥಗಳು ಹುಟ್ಟಿವೆ. ಯಾರು ಸತ್ತರೆ ಎಷ್ಟು ದಿವಸ ಸೂತಕ ಇರಬೇಕು, ಆಗ ಏನು ತಿನ್ನಬೇಕು, ಏನು ಕುಡಿಯಬೇಕು ಎಂಬ ವಿಷಯ ದಲ್ಲಿ ದೊಡ್ಡ ವಾದಗಳೇ ಆಗಿವೆ. ಈ ಗಲಾಟೆಯಲ್ಲಿ ನಿರತರಾಗಿ ಗಹನವಾದ ಆಧ್ಯಾ ತ್ಮಿಕ ವಿಷಯಗಳ ಕಡೆ ನಮ್ಮ ಮನಸ್ಸು ಹೋಗಲಿಲ್ಲ.”

ನಮ್ಮಲ್ಲಿರುವ ಮತ್ತೊಂದು ದೊಡ್ಡ ದೋಷವೇ ಸಹಕಾರ ಮನೋಭಾವದ ಅಭಾವ. ಪ್ರತ್ಯೇಕ ವ್ಯಕ್ತಿಗಳಲ್ಲಿ ಎಷ್ಟೋ ಒಳ್ಳೆಯ ಗುಣಗಳಿವೆ. ಆದರೆ ಹಲವು ಜನ ಒಟ್ಟು ಕಲೆತು ಒಂದು ಸಂಘವನ್ನು ಬಹುದಿನ ನಡೆಯುವಂತೆ ಮಾಡಲಾರೆವು. ಕಲೆತ ಕೆಲವು ದಿನಗಳಲ್ಲೇ ಭಿನ್ನಾಭಿಪ್ರಾಯ ಘರ್ಷಣೆ ಪ್ರಾರಂಭವಾಗಿ ಅದು ನಾಮಾವಶೇಷವಾಗಿ ಹೋಗುವುದು. ಹುಟ್ಟುವಷ್ಟೇ ಬೇಗ ಅದು ಸಾಯುವುದು. ನಾವು ಸಂಘದಿಂದ ನಮಗೆ ಬರುವ ಕೀರ್ತಿ ಅಧಿಕಾರದ ಲಾಭಗಳನ್ನು ಗಮನಿಸು ವೆವೇ ಹೊರತು ಸಂಸ್ಥೆಯನ್ನಲ್ಲ. ವ್ಯಕ್ತಿ ಎಷ್ಟೇ ಪ್ರಖ್ಯಾತನಾಗಿದ್ದರೂ ಅಲ್ಪಾಯು, ಸಂಸ್ಥೆ ಮಾತ್ರ ದೀರ್ಘಕಾಲ ಬಾಳಬಲ್ಲದು. ಹಲವು ಜನರಿಗೆ ನಂತರವೂ ಪ್ರಯೋ ಜನವಾಗಬಲ್ಲದು. ರಾಜಕೀಯವೇ ಆಗಲಿ ವ್ಯಾಪಾರ ಉದ್ಯಮವೇ ಆಗಲಿ, ಯಾಂತ್ರಿಕ ವಸ್ತುವನ್ನು ಉತ್ಪಾದನೆ ಮಾಡುವುದೇ ಆಗಲಿ ಇವುಗಳನ್ನೆಲ್ಲ ಸಾಧಿಸ ಬೇಕಾದರೆ ನಾವು ದೊಡ್ಡ ಸಂಸ್ಥೆಯನ್ನು ಕಟ್ಟಬೇಕಾಗುವುದು. ಒಂದು ಸಂಸ್ಥೆಗೆ ಶಕ್ತಿ ಬರಬೇಕಾದರೆ ಹಲವು ವ್ಯಕ್ತಿಗಳು ಅದಕ್ಕೆ ತ್ಯಾಗ ಮಾಡಬೇಕು. ಅದಕ್ಕೆ ಕೊಡ ಬೇಕು. ಅದರಿಂದ ಕಸಿದುಕೊಳ್ಳುವುದಲ್ಲ. ಎಲ್ಲಿ ಕೊಡುವುದಕ್ಕಿಂತ ಹೆಚ್ಚಾಗಿ ತೆಗೆದುಕೊಳ್ಳುವೆವೊ ಅಲ್ಲಿ ಸಂಸ್ಥೆ ಬಹಳ ಬೇಗ ನಿರ್ನಾಮವಾಗುವುದು. ಎಲ್ಲಿ ಕೊಡುವುದರ ಕಡೆಗೆ ನಮಗೆ ಗಮನವಿರುವುದೋ ಅಲ್ಲಿ ಆ ಸಂಸ್ಥೆ ವ್ಯಕ್ತಿಯ ಕಾಲಾ ನಂತರವೂ ಬಹುಜನರ ಹಿತಕ್ಕೆ ಬಹುಜನರ ಸುಖಕ್ಕೆ ದೀರ್ಘಕಾಲ ಬಾಳಬಲ್ಲದು.

ಗುಲಾಮಗಿರಿಯಲ್ಲಿ ಬಹುಕಾಲ ಬಾಳಿದವರಲ್ಲಿರುವ ಮತ್ತೊಂದು ಅವ ಗುಣವೇ ಅಸೂಯೆ. ನಮ್ಮಲ್ಲಿ ಯಾರಾದರೂ ಮುಂದೆ ಬಂದರೆ ಅದನ್ನು ಸಹಿಸುವು ದಕ್ಕೆ ಆಗುವುದಿಲ್ಲ. ಹೇಗೋ ಅವನನ್ನು ಪದಚ್ಯುತನನ್ನಾಗಿ ಮಾಡಲು ಯತ್ನಿಸು ವೆವು. ಅಸೂಯೆ ಎಂಬುದು ನಮ್ಮಲ್ಲಿ ರಕ್ತಗತವಾದ ವಿಷಯ. ಅದರಿಂದ ಪಾರಾ ಗಲು ನಾವು ಯತ್ನಿಸಬೇಕು. ಎಲ್ಲಿಯವರೆಗೆ ಇದು ನಮ್ಮಲ್ಲಿರುವುದೋ ಅಲ್ಲಿಯ ವರೆಗೆ ನಾವು ಜೀವನದಲ್ಲಿ ಏನನ್ನೂ ಸಾಧಿಸಲಾರೆವು.

ಮತ್ತೊಂದು ನಮ್ಮಲ್ಲಿರುವ ಅವಗುಣವೇ ಅಧಿಕಾರಲಾಲಸೆ. ಎಲ್ಲರಿಗೂ ಅಪ್ಪಣೆ ಕೊಡಲು ಆಸೆ, ಆಳಲು ಆಸೆ. ಅಪ್ಪಣೆಯನ್ನು ಪರಿಪಾಲಿಸುವುದನ್ನು ಕಲಿಯದೆ ಅಪ್ಪಣೆಯನ್ನು ಕೊಡುವುದು ಬರಲಾರದು ಎಂಬುದನ್ನು ನಾವೆಲ್ಲ ಇನ್ನೂ ಅರಿಯಬೇಕಾಗಿದೆ. ಎಲ್ಲರಿಗೂ ನಾಯಕರಾಗಿರಬೇಕು ಎನ್ನುವ ಕುತೂಹಲ. ಬರೀ ಹಂಬಲದಿಂದಲೇ ಒಂದು ಸ್ಥಿತಿ ಸಿಕ್ಕುವುದೇನು ಜೀವಿಗೆ. ಅದಕ್ಕೆ ತಕ್ಕ ಯೋಗ್ಯತೆ ಯನ್ನು ಪಡೆದಿರಬೇಕು. ಯಾರಿಗೆ ಯೋಗ್ಯತೆ ಇದೆಯೊ ಅವನು ನಾಯಕನಾಗ ಬೇಕೆಂದು ಇಚ್ಛಿಸುವುದಿಲ್ಲ. ಆ ಪದವಿ ಅವನನ್ನು ಹುಡುಕಿಕೊಂಡು ಬರುವುದು. ಸ್ವಾಮಿ ವಿವೇಕಾನಂದರು “ಯಾರು ಸಿರ್​ದಾರನೊ ಅವನು ಸರ್​ದಾರ್​” ಎಂದು ಹೇಳುತ್ತಿದ್ದರು; ಎಂದರೆ ಯಾರು ತಮ್ಮ ತಲೆಯನ್ನು ಬಲಿಕೊಡಲು ಸಿದ್ಧನಾಗಿರು ವನೊ ಅವನು ನಾಯಕನಾಗಲು ಯೋಗ್ಯ. ಅನೇಕ ವೇಳೆ ಯೋಗ್ಯತೆಯನ್ನು ಸಂಪಾ ದಿಸುವ ಕಡೆಗೆ ಗಮನವನ್ನೇ ಕೊಡದೆ ನಾಯಕಗಿರಿಗೆ ಆಸೆಪಡುವೆವು. ಇದೊಂದು ದೊಡ್ಡ ನ್ಯೂನತೆ ನಮ್ಮಲ್ಲಿ. ಜೀವನದಲ್ಲಿ ಪ್ರತಿಯೊಬ್ಬನೂ ನಾಯಕನೇ ಆಗ ಬೇಕೇ? ಆಣತಿಯನ್ನು ಪರಿಪಾಲಿಸುವುದರಲ್ಲಿ ಒಂದು ಆನಂದ ವಿಲ್ಲವೇ? ಶ್ರೀರಾಮ ಸೇತುವೆಯನ್ನು ಕಟ್ಟುತ್ತಿದ್ದಾಗ ಅಳಿಲು ಬಂದು ಮರಳನ್ನು ಕೊಡವಿ ಹೋಗುತ್ತಿದ್ದಂತಹ ಕೆಲಸದಿಂದಲೂ ಧನ್ಯರಾಗಲು ಸಾಧ್ಯ ಎಂದರು ಸ್ವಾಮಿ ವಿವೇಕಾನಂದರು. ಇಂತಹ ಮನೋಭಾವವನ್ನು ನಾವು ರೂಢಿಸಬೇಕಾಗಿದೆ.

ಭರತಖಂಡ ಅನ್ಯರಾಷ್ಟ್ರಗಳೊಂದಿಗೆ ಸಂಬಂಧವನ್ನು ದೀರ್ಘಕಾಲ ಕಳೆದು ಕೊಂಡುದು ಅದರ ಅವನತಿಗೊಂದು ದೊಡ್ಡ ಕಾರಣ. ನಾವು ಕೂಪಮಂಡೂಕ ಗಳಾದೆವು. ಹೊರಗೆ ಹೋಗಿ ಬಂದವರಿಗೆ ಬಹಿಷ್ಕಾರವನ್ನು ಹಾಕಿದರು. ಯಾವ ರಾಷ್ಟ್ರವೇ ಆಗಲಿ, ನಾಗರಿಕತೆಯೇ ಆಗಲಿ, ತನ್ನ ಬಿಲದೊಳಗೆ ತಾನು ಇರುತ್ತೇನೆ ಎಂದರೆ ಬಹಳ ಬೇಗ ಕುಂಠಿತವಾಗಿ ಹೋಗುವುದು. ಅನ್ಯರಲ್ಲಿರುವ ಒಳ್ಳೆಯ ದನ್ನು ಹೀರಿಕೊಳ್ಳಲಾರದು, ತನ್ನಲ್ಲಿರುವ ನ್ಯೂನತೆಯನ್ನು ತಿಳಿದುಕೊಳ್ಳಲಾರದು. ಸಾಧ್ಯವಿದ್ದಷ್ಟು ಜನ ಭರತಖಂಡದಿಂದ ಹೊರಗೆ ಹೋಗಿ, ಅನ್ಯರಾಷ್ಟ್ರಗಳು ಹೇಗೆ ಮುಂದುವರಿಯುತ್ತಿವೆ ಎಂಬುದನ್ನು ನೋಡಲಿ; ಆಗ ಕಣ್ಣು ತೆರೆಯುವುದು ಎನ್ನುತ್ತಿದ್ದರು ಸ್ವಾಮೀಜಿ.

ನಮ್ಮ ಮತ್ತೊಂದು ನ್ಯೂನತೆ ಕರ್ಮ ಸಿದ್ಧಾಂತವನ್ನು ಸರಿಯಾಗಿ ಅರ್ಥಮಾಡಿ ಕೊಳ್ಳದೇ ಇದ್ದುದು. ಯಾವ ಕರ್ಮಸಿದ್ಧಾಂತ ನಮ್ಮನ್ನು ಕಾರ್ಯೋನ್ಮುಖರನ್ನಾಗಿ ಮಾಡಬೇಕೊ ಅದನ್ನು ತಪ್ಪು ತಿಳಿದುಕೊಂಡು, ಶುದ್ಧ ಸೋಮಾರಿಯ ಜೀವನಕ್ಕೆ ಉಪಯೋಗಿಸಿಕೊಂಡೆವು. ನಮಗೆ ಯಾವ ಕೆಟ್ಟದ್ದಾಗಲಿ ಅನ್ಯಾಯವಾಗಲಿ ಇದು ನಮ್ಮ ಕರ್ಮ, ಇದು ನಮ್ಮ ಹಣೆಯ ಬರಹ ಎಂದು ಅದಕ್ಕೆ ಬಾಗುವೆವು. ಅದೇ ಕರ್ಮ ಸಿದ್ಧಾಂತ ಈಗ ಹೊಸ ಉತ್ತಮ ಕರ್ಮಗಳನ್ನು ಮಾಡುವುದಕ್ಕೆ, ಭವ್ಯವಾದ ಭವಿಷ್ಯದ ತಳಹದಿಯನ್ನು ಹಾಕುವುದಕ್ಕೆ ಪ್ರಚೋದನಕಾರಿಯಾಗಬಲ್ಲದು. ನಾವು ಈಗ ಇರುವುದಕ್ಕೆ ಹಿಂದಿನದು ಕಾರಣವಾದರೆ, ಮುಂದಿನದು ಈಗ ನಾನು ಏನು ಮಾಡುತ್ತಿರುವೆನೊ ಅದರ ಮೇಲೆ ನಿಂತಿದೆ ಎಂಬ ಜವಾಬ್ದಾರಿಯನ್ನು ಕೊಡು ವುದು. ದೇವರೂ ಕೂಡ ಸೋಮಾರಿಗೆ ಸಹಾಯಮಾಡುವುದಿಲ್ಲ. ನಾವು ಯಾವಾಗ ಸೊಂಟಕಟ್ಟುತ್ತೇವೆಯೋ ಆಗ ಅವನ ಶಕ್ತಿ ನಮ್ಮ ನೆರವಿಗೆ ಬರುವುದು.

ನಮ್ಮ ಮತ್ತೊಂದು ದೌರ್ಬಲ್ಯವೇ ನಮ್ಮ ಜನಾಂಗದ ಅರ್ಧಭಾಗವಾದ ಸ್ತ್ರೀಯರನ್ನು ನಾವು ಕಡೆಗಣಿಸಿದ್ದು. ವಿದ್ಯೆ ಐಶ್ವರ್ಯ ಅಧಿಕಾರ ಎಲ್ಲವನ್ನೂ ಪುರುಷ ನಿಗೆ ಮಾತ್ರ ಕೊಟ್ಟೆವು. ಅವನು ತಿಂದು ಮಿಕ್ಕಿದ್ದು, ಅವನ ಎಂಜಲು ಮಾತ್ರ ಉಳಿದ ಅರ್ಧಭಾಗಕ್ಕೆ ಮೀಸಲು. ಸ್ತ್ರೀ ಮತ್ತು ಪುರುಷರು ಹಕ್ಕಿಗೆ ಇರುವ ಎರಡು ರೆಕ್ಕೆಗಳಂತೆ. ಒಂದು ಹಕ್ಕಿಗೆ ಒಂದೇ ರೆಕ್ಕೆ ಎಷ್ಟೇ ಬಲವಾಗಿರಲಿ ಮೇಲೆ ಹೇಗೆ ಹಾರಲಾರದೊ ಹಾಗೆ ಸಮಾಜ. ನಾವು ಮತ್ತೊಂದು ಅಂಗವಾದ ಸ್ತ್ರೀಯರ ಸ್ಥಾನ ಮಾನಗಳನ್ನು ಅಭಿವೃದ್ಧಿಪಡಿಸಬೇಕು. ಅವರಿಗೆ ವಿದ್ಯೆ ಕೊಡಬೇಕು. ಅವರನ್ನು ಪೂಜ್ಯದೃಷ್ಟಿಯಿಂದ ನೋಡಬೇಕು. ಅವರ ಆತ್ಮೋನ್ನತಿಗೆ ಎಲ್ಲಾ ಅವಕಾಶ ಗಳನ್ನೂ ಒದಗಿಸಿಕೊಡಬೇಕು. ಪುರುಷ ಕಲಿತಿದ್ದು ಹೊರಗೆ ಸಂಪಾದನೆ ಮಾಡು ವುದರಲ್ಲಿಯೇ ಕೊನೆಗೊಳ್ಳುವುದು. ಆದರೆ ಯಾವ ವಿದ್ಯೆಯನ್ನು ಸಂಸ್ಕೃತಿಯನ್ನು ಮನೆಯಲ್ಲಿ ತಾಯಾಗಿರುವವಳು ಪಡೆದಿರುವಳೊ ಅದರಿಂದಲೇ ಬೆಳೆಯುವ ಮಕ್ಕ ಳಿಗೆ ಪ್ರಯೋಜನ. ಯಾವ ಕೈಗಳು ಈಗ ತೊಟ್ಟಿಲನ್ನು ತೂಗುತ್ತಿವೆಯೋ ಆ ಕೈಗಳೆ ನಾಳೆ ಪುರುಷಸಿಂಹರನ್ನು ಜಗತ್ತಿಗೆ ದಾನಮಾಡುವುದು. ಸ್ತ್ರೀಯರಿಗೆ ಸರಿಯಾದ ವಿದ್ಯೆಯನ್ನು ಕೊಡಬೇಕು. ನಂತರ ಅವರು ತಮ್ಮ ಸಮಸ್ಯೆಗಳನ್ನು ತಾವೇ ಪರಿಹಾರ ಮಾಡಿಕೊಳ್ಳುವರು ಎನ್ನುತ್ತಿದ್ದರು ಸ್ವಾಮೀಜಿ.

ಭರತಖಂಡ ಮೇಲೇಳಬೇಕಾದರೆ ತನ್ನ ಆರ್ಥಿಕ ಸಂಪತ್ತನ್ನು ವೃದ್ಧಿಮಾಡಿ ಕೊಳ್ಳಬೇಕು. ಏನೊ ಸಿಕ್ಕಿದ್ದನ್ನು ತಿಂದು ಕಾಲಯಾಪನೆ ಮಾಡಿ ಎಂದು ಜನರಿಗೆ ಸ್ವಾಮೀಜಿ ಹೇಳುತ್ತಿರಲಿಲ್ಲ. ನಾವು ಆರೋಗ್ಯಕರವಾದ ಬಲಿಷ್ಠವಾದ ಆಹಾರ ವನ್ನು ಸೇವಿಸಬೇಕು. ಗಟ್ಟಿಮುಟ್ಟಾಗಿ ಬೆಳೆಯಬೇಕು. ಕಬ್ಬಿಣದಂತಹ ಮಾಂಸ ಖಂಡಗಳು, ಉಕ್ಕಿನಂತಹ ನರಗಳು ಬೇಕು ನಮ್ಮ ಜನರಿಗೆ ಎನ್ನುತ್ತಿದ್ದರು. ಹಾಗೆ ಆಗಬೇಕಾದರೆ ನಾವು ಹೆಚ್ಚು ಆಹಾರವನ್ನು ಉತ್ಪಾದನೆ ಮಾಡುವುದನ್ನು ಕಲಿಯ ಬೇಕು. ಪಾಶ್ಚಾತ್ಯ ದೇಶಗಳು ತಮ್ಮ ಯಂತ್ರ ನಾಗರಿಕತೆಯಿಂದ ಇದನ್ನು ಹೇಗೆ ಸಾಧಿಸುತ್ತಿವೆ ಎಂಬುದನ್ನು ಅರಿಯಬೇಕು. ಅದನ್ನು ಸಾಧ್ಯವಾದಷ್ಟು ಭರತ ಖಂಡದಲ್ಲಿ ಜಾರಿಗೆ ತರಬೇಕು.

ಸ್ವಾಮೀಜಿಯವರಿಗೆ ಅಳು ಕಂಡರೆ ಆಗುತ್ತಿರಲಿಲ್ಲ. ದೇವರ ಹೆಸರಿನಲ್ಲಿಯೂ ನಾವು ಅಳಕೂಡದು ಎನ್ನುತ್ತಿದ್ದರು. ಅಲ್ಲಿ ಅಳುವುದಕ್ಕೆ ಏನಿದೆ? ಸಚ್ಚಿದಾನಂದ ಸ್ವರೂಪ ಭಗವಂತ. ಅವನಿಂದ ಸಿಡಿದುಬಂದ ಕಿಡಿಗಳು ಜೀವಿಗಳು. ಅವನ ಮುಖದಲ್ಲಿರುವ ಮಂದಹಾಸ ಆನಂದ ನಮ್ಮಲ್ಲಿಯೂ ವ್ಯಕ್ತವಾಗುತ್ತಿರಬೇಕು. ಅಳುಮೋರೆಯವನಿಗೆ ಮನೆಬಿಟ್ಟು ಹೊರಗೆ ಹೋಗಬೇಡ ಎನ್ನುತ್ತಿದ್ದರು ಸ್ವಾಮೀಜಿ. ಮನೆಯಿಂದ ಹೋಗುವುದಕ್ಕೆ ಮುಂಚೆ ಕನ್ನಡಿಯಲ್ಲಿ ನಿನ್ನ ಮುಖ ವನ್ನು ನೋಡಿಕೊ. ಅಳುಮೋರೆ ಇದ್ದರೆ ಆವತ್ತು ಹೊರಗೆ ಹೋಗಬೇಡ. ಅದೊಂದು ಸಾಂಕ್ರಾಮಿಕ ಜಾಡ್ಯ. ಅದನ್ನು ಇತರರಿಗೆ ಕೊಡಬೇಡ. ಸಾಧ್ಯವಿದ್ದರೆ ಆನಂದ ಕೊಡು, ಪ್ರೋತ್ಸಾಹ ಕೊಡು, ಗೆಲವು ಕೊಡು. ಅದಿಲ್ಲದೇ ಇದ್ದರೆ ತೆಪ್ಪಗೆ ನೀನೊಬ್ಬನೇ ಮನೆಯಲ್ಲಿದ್ದರೆ ದೊಡ್ಡ ಉಪಕಾರಮಾಡುವೆ ಜಗತ್ತಿಗೆ! ಮುಂದೆ ಬರುವ ಅವರ ವಾಣಿಯನ್ನು ಕೇಳೋಣ.

“ಬಹುಕಾಲ ನಾವು ಅತ್ತಿರುವೆವು. ಇನ್ನು ಹೆಚ್ಚು ಅಳಕೂಡದು. ನಿಮ್ಮ ಕಾಲ ಮೇಲೆ ನಿಂತು ಮನುಷ್ಯರಾಗಿ. ನಮಗೆ ಇಂದು ಪುರುಷಸಿಂಹರನ್ನು ಮಾಡುವ ಧರ್ಮ ಬೇಕು. ಪುರುಷಸಿಂಹರನ್ನು ಮಾಡುವ ಸಿದ್ಧಾಂತಗಳು ಬೇಕು. ಪುರುಷ ಸಿಂಹರನ್ನಾಗಿ ಮಾಡುವ ಸರ್ವತೋಮುಖಿಗಳನ್ನಾಗಿ ಮಾಡುವ ವಿದ್ಯಾಭ್ಯಾಸ ಬೇಕು. ಸತ್ಯದ ಪರೀಕ್ಷೆ ಇಲ್ಲಿದೆ. ಯಾವುದು ನಿಮ್ಮನ್ನು ದೈಹಿಕವಾಗಿ ಆಗಲಿ, ಆಧ್ಯಾತ್ಮಿಕವಾಗಿ ಆಗಲಿ, ಮಾನಸಿಕವಾಗಿ ಆಗಲಿ, ದುರ್ಬಲರನ್ನಾಗಿ ಮಾಡುವುದೊ ಅದನ್ನು ವಿಷದಂತೆ ತ್ಯಜಿಸಿ. ಅದರಲ್ಲಿ ಚೇತನವಿಲ್ಲ. ಅದು ಸತ್ಯವೆನಿಸಲಾರದು. ಸತ್ಯ ಶಕ್ತಿವರ್ಧಕ. ಸತ್ಯ ಪರಿಶುದ್ಧವಾದುದು. ಸತ್ಯವೇ ಅನಂತ ಜ್ಞಾನ. ಸತ್ಯವು ಶಕ್ತಿ ಯನ್ನು ನೀಡಬೇಕು, ಬೆಳಕನ್ನು ನೀಡಬೇಕು, ಹೊಸಚೇತನವನ್ನು ತುಂಬಬೇಕು.”

ಸ್ವಾಮೀಜಿಯವರು ಪದೇ ಪದೇ ಬೋಧಿಸುತ್ತಿದ್ದ ಮತ್ತೊಂದು ಭಾವನೆಯೇ ಆತ್ಮಶ್ರದ್ಧೆ. ನಮಗೆ ನಮ್ಮಲ್ಲಿ ಒಂದು ನಂಬಿಕೆ ಇರಬೇಕು. ಏಕೆಂದರೆ ನಾವು ದೇವರ ಮಕ್ಕಳು. ಅವನಲ್ಲಿ ಯಾವ ಪವಿತ್ರತೆ ಇದೆಯೋ ಜ್ಞಾನವಿದೆಯೋ, ಶಕ್ತಿ ಇದೆಯೋ ಅವೆಲ್ಲ ನಮ್ಮಲ್ಲಿಯೂ ಒಂದು ಪ್ರಮಾಣದಲ್ಲಿದೆ. ನಾವೀಗ ಅದನ್ನು ಮರೆತಿರುವೆವು. ತಾತ್ಕಾಲಿಕವಾದ ಅದರ ಮೇಲೆ ಕುಳಿತ ಧೂಳನ್ನೇ ಸತ್ಯವೆಂದು ಭ್ರಮಿಸಿ, ನಾವು ಪಾಪಿಗಳು ಮೂಢರು ಎಂದು ಅವಹೇಳನಮಾಡಿಕೊಳ್ಳುತ್ತಿರು ವೆವು. ನಾವೇ ಪಾಪಿಗಳಾದರೆ ಇನ್ನು ಪುಣ್ಯಾತ್ಮರು ಯಾರು? ನಾವೇ ದಡ್ಡರಾದರೆ ಇನ್ನು ಸುಜ್ಞಾನಿಗಳು ಯಾರು? ನಾವು ಜೀವನದಲ್ಲಿ ನೆಚ್ಚುಗೆಡಕೂಡದು. ದುಃಖ ಸಂಕಟಗಳು ಬಂದಾಗ ಪಲಾಯನ ಹೇಳಕೂಡದು. ಅವುಗಳೊಂದಿಗೆ ಹೋರಾಡ ಬೇಕು. ಆಗಲೇ ಅವುಗಳು ಓಡಿಹೋಗಬೇಕಾದರೆ. ದೇವರು ನಮಗೆ ಕಷ್ಟ ಕೊಡುವು ದರಲ್ಲಿ ಒಂದು ರಹಸ್ಯವಿದೆ. ಅವನು ನಮ್ಮ ಅಂತಶ್ಶಕ್ತಿಯನ್ನು ಉದ್ದೀಪನಗೊಳಿ ಸುವುದಕ್ಕೆ ಇದನ್ನು ಕೊಡುವನು. ಅದನ್ನು ಆರಿಸುವುದಕ್ಕಲ್ಲ. ಹೋರಾಡಿದಷ್ಟು ಆ ಶಕ್ತಿ ಬಲಿಷ್ಠವಾಗುತ್ತಾ ಬರುವುದು.

ಸ್ವಾಮೀಜಿಯವರು ಈ ಪ್ರಪಂಚವನ್ನು ಒಂದು ಗರಡಿಯ ಮನೆ ಎನ್ನುತ್ತಿ ದ್ದರು. ಇಲ್ಲಿರುವ ಹಲವು ಅಂಗಸಾಧನೆಯ ಯಂತ್ರಗಳೇ ಕಷ್ಟ ನಷ್ಟ ಸುಖ ದುಃಖ ಟೀಕೆ ಹೊಗಳಿಕೆ ಶತ್ರು ಮಿತ್ರ ಮುಂತಾದುವು. ಇವುಗಳೊಂದಿಗೆ ನಾವು ಹೋರಾಡಿ ದಾಗಲೇ ನಮ್ಮ ಶಕ್ತಿ ಬಲವಾಗುವುದು.

ಸ್ವಾಮೀಜಿಯವರು ಭರತಖಂಡದ ಪುನರುದ್ಧಾರಕ್ಕೆ ನಿಂತವರಲ್ಲಿ ಕೆಳಗೆ ಬರುವ ಮೂರು ಗುಣಗಳು ಇರಬೇಕೆಂದು ಮದ್ರಾಸಿನಲ್ಲಿ ಮಾಡಿದ ಉಪನ್ಯಾಸದಲ್ಲಿ ವಿವರಿಸುವರು:

“ಕೋಟ್ಯಂತರ ದೇವ ಮತ್ತು ಪುಷಿಸಂತಾನರು ಪಶುಸಂತಾನರಾಗಿರುವರೆಂದು ನಿಮಗೆ ಮನಸ್ಸಿನಲ್ಲಿ ವ್ಯಥೆ ಇದೆಯೆ? ಕೋಟ್ಯಂತರ ಜನ ಈಗ ಉಪವಾಸದಲ್ಲಿ ನರಳುತ್ತಿರುವರು, ಹಿಂದಿನಿಂದಲೂ ನರಳುತ್ತಿದ್ದರು ಎಂಬುದು ಗೊತ್ತೆ? ಅಜ್ಞಾನ ಕಾರ್ಮೋಡದಂತೆ ಭಾರತಾವನಿಯನ್ನೆಲ್ಲ ಆವರಿಸಿದೆ ಎಂಬುದು ನಿಮಗೆ ಗೊತ್ತೆ? ಇದರಿಂದ ನೀವು ವಿಹ್ವಲರಾಗಿರುವಿರಾ? ಇದರಿಂದ ನಿಮ್ಮ ನಿದ್ರೆಗೆ ಭಂಗ ಬಂದಿ ದೆಯೆ? ಇದು ನಿಮ್ಮ ರಕ್ತದಲ್ಲಿ ವ್ಯಾಪಿಸಿದೆಯೆ? ನಾಡಿಯಲ್ಲಿ ಸಂಚರಿಸಿ ಹೃದಯದ ಸ್ಪಂದನದೊಂದಿಗೆ ಸ್ಪಂದಿಸುತ್ತಿದೆಯೆ? ಇದು ನಿಮ್ಮನ್ನು ಉನ್ಮತ್ತರನ್ನಾಗಿ ಮಾಡಿ ದೆಯೆ? ಈ ಸರ್ವನಾಶದ ದುಃಖ ನಿಮ್ಮನ್ನು ವ್ಯಾಪಿಸಿ ನಿಮ್ಮ ಕೀರ್ತಿ ಯಶಸ್ಸು, ಹೆಂಡತಿ ಮಕ್ಕಳು ಆಸ್ತಿ ಮತ್ತು ನಿಮ್ಮ ದೇಹವನ್ನೇ ನೀವು ಮರೆತಿರುವಿರಾ? ನೀವು ಇದನ್ನು ಮಾಡಿರುವಿರಾ? ದೇಶಭಕ್ತರಾಗುವುದಕ್ಕೆ ಇದೇ ಮೊದಲನೇ ಹೆಜ್ಜೆ.

“ಈ ಉದ್ದೇಶವನ್ನು ಬರೀ ಬಾಯಿಮಾತಿನಲ್ಲಿ ವ್ಯಕ್ತಗೊಳಿಸದೆ, ಈ ಸಮಸ್ಯೆ ಯಿಂದ ಪಾರಾಗುವುದಕ್ಕೆ ಯಾವುದಾದರೊಂದು ಮಾರ್ಗವನ್ನು ಕಂಡುಹಿಡಿದಿರು ವಿರಾ? ಅವರನ್ನು ದೂರದೆ ಸಹಾಯವನ್ನು ನೀಡುವ, ವರ್ತಮಾನ ಮೃತ್ಯುಸ್ಥಿತಿ ಯಿಂದ ಅವರನ್ನು ಮೇಲೆತ್ತಿ ಅವರ ದುಃಖವನ್ನು ಶಮನಗೊಳಿಸುವ ಸಹಾನು ಭೂತಿಯ ನುಡಿಗಳನ್ನು ಉಚ್ಚರಿಸಬಲ್ಲಿರಾ? ಇದು ಮಾತ್ರವಲ್ಲ ಪರ್ವತದಂತಹ ಆತಂಕವನ್ನು ದಾಟಬಲ್ಲ ಇಚ್ಛಾಶಕ್ತಿ ನಿಮ್ಮಲ್ಲಿದೆಯೆ? ಇಡೀ ಪ್ರಪಂಚವೇ ಹಿರಿದ ಕತ್ತಿಯಿಂದ ನಿಮ್ಮನ್ನು ಎದುರಿಸಿದರೂ, ಸರಿ ಎಂದು ತೋರಿದುದನ್ನು ಮಾಡುವ ಧೈರ್ಯ ನಿಮಗೆ ಇದೆಯೆ? ಹೆಂಡತಿ ಮಕ್ಕಳು ಅದನ್ನು ವಿರೋಧಿಸಿದರೂ, ನಿಮ್ಮ ದ್ರವ್ಯ ಹೋದರೂ, ಕೀರ್ತಿ ನಾಶವಾದರೂ ನೀವು ಅದನ್ನು ಬಿಡದೆ ಇರಬಲ್ಲಿರಾ? ಅದನ್ನೇ ಹಂಬಲಿಸುತ್ತ ಎಡಬಿಡದೆ ಗುರಿ ಎಡೆಗೆ ಧಾವಿಸಬಲ್ಲಿರಾ? ಈ ದೃಢತೆ ನಿಮ್ಮಲ್ಲಿದೆಯೆ? ಈ ಮೂರು ಗುಣಗಳಿದ್ದರೆ ನಿಮ್ಮಲ್ಲಿ ಪ್ರತಿಯೊಬ್ಬರೂ ಅದ್ಭುತ ಕಾರ್ಯವನ್ನು ಸಾಧಿಸಬಲ್ಲಿರಿ.”

ಸ್ವಾಮೀಜಿಯವರು ಭರತಖಂಡದ ಭವಿಷ್ಯದಲ್ಲಿ ನೆಚ್ಚುಗೆಡಲಿಲ್ಲ. ತಮ್ಮ ಸ್ವರ್ಣಯುಗ ಹಿಂದೆಯೇ ಆಗಿಹೋಯಿತು ಎಂದು ಗತಕಾಲದ ವೈಭವವನ್ನು ಮೆಲಕುಹಾಕುವ ಸೋಮಾರಿಗಳಲ್ಲ ಸ್ವಾಮೀಜಿ. ಹಿಂದಿನದನ್ನು ಗೌರವಿಸಿದರು. ಹಿಂದೆ ಇದ್ದ ಪವಿತ್ರ ಅಂಶಗಳನ್ನೆಲ್ಲ ಹೀರಿಕೊಂಡರು. ಆದರೆ ಅಲ್ಲಿಗೆ ನಿಲ್ಲಲಿಲ್ಲ. ಹಿಂದಿನದು ಈಗ ಮಾಡುವ ಕಾರ್ಯಕ್ಕೆ ಸಹಾಯಕವಾಗಿ ಆಗಬೇಕು. ಮತ್ತೆ ಭವಿಷ್ಯ ದಲ್ಲಿ ಹೆಚ್ಚು ನೆಚ್ಚುಗೆಯನ್ನು ಇಟ್ಟಿದ್ದರು. ಹಿಂದಿನದನ್ನೆಲ್ಲ ಮೀರಿಸುವ ಭವಿಷ್ಯ ನಮಗೆ ಕಾದಿದೆ ಎಂದು ಆಶಾವಾದಿಗಳಾಗಿದ್ದರು. ಹಿಂದಿನ ಶಾಸ್ತ್ರಗಳನ್ನು ಮೀರುವ ಶಾಸ್ತ್ರಗಳು ಬರುತ್ತವೆ, ಹಿಂದಿನ ಪುಷಿಗಳನ್ನು ಮೀರಿದ ಪುಷಿಗಳು ಬರುವರು. ಹಿಂದಿನ ಉಚ್ಛ್ರಾಯ ಸ್ಥಿತಿಯನ್ನು ಅಣಕಿಸುವಂತಹ ಸುದಿನ ನಮಗೆ ಕಾದಿದೆ ಎನ್ನುತ್ತಿದ್ದರು ಸ್ವಾಮೀಜಿ. ಅದನ್ನು ಸ್ವಾಗತಿಸುವುದಕ್ಕೆ ಈಗಿನವರು ಅಣಿಯಾಗ ಬೇಕು. ಅದಕ್ಕಾಗಿ ದುಡಿಯಬೇಕು. ಭವಿಷ್ಯದ ಸಸಿಗೆ ನಾವು ಹಾಕಬೇಕಾದ ನೀರು ಮತ್ತು ಗೊಬ್ಬರವೇ ನಮ್ಮ ಶ್ರಮ ನಮ್ಮ ದುಡಿತ. ದ್ರಷ್ಟಾರನ ವಾಣಿಯಲ್ಲಿ ನಮ್ಮ ಭಾರತಮಾತೆ ಹೇಗೆ ನಿದ್ರೆಯಿಂದೇಳುತ್ತಿರುವಳು ಎಂಬುದನ್ನು ಅವರ ಒಂದು ವಾಣಿಯ ಮೂಲಕವೇ ಕೇಳೋಣ:

“ಓ ನನ್ನ ಭಾರತೀಯ ಭ್ರಾತೃಗಳಿರಾ, ನಿದ್ರೆಯಿಂದೆದ್ದೇಳಿ. ನವೋದಯವಾಗು ತ್ತಿದೆ. ಕಣ್ದೆರೆದು ನೋಡಿ, ಅದೋ ಸುದೀರ್ಘ ರಾತ್ರಿ ಕಡೆಗಿಂದು ಕೊನೆಗಾಣುತ್ತಿದೆ. ಬಹುಕಾಲದ ಶೋಕಪಾತಗಳು ಕಡೆಗಿಂದು ಮಾಯವಾಗುತ್ತಿವೆ. ಇದುವರೆಗೆ ಶವ ದಂತೆ ಬಿದ್ದಿದ್ದ ಶರೀರವಿಂದು ಸಚೇತನವಾಗುತ್ತಿದೆ. ಅದೋ ಕಿವಿಗೊಡಿ, ವಾಣಿ ಯೊಂದು ಕೇಳಿಬರುತ್ತಿದೆ–ಬಹು ಪುರಾತನಕಾಲದ ಇತಿಹಾಸದ ಕಾಲಗರ್ಭದಿಂದ ಹೊಮ್ಮಿ, ಪರ್ವತ ಶಿಖರಗಳಿಂದ ಮರುದನಿಯಾಗಿ ಚಿಮ್ಮಿ, ಕಾನನ ಗಿರಿ ಕಂದರ ಗಳಲ್ಲಿ ಸಂಚರಿಸಿ, ಬರಬರುತ್ತ ಪ್ರಬಲವಾಗಿ, ಬಂದಂತೆಲ್ಲ ಅಪ್ರತಿಹತವಾಗಿ ನಮ್ಮೀ ಪುಣ್ಯಭೂಮಿಯನ್ನು ನಿದ್ರೆಯಿಂದ ಹೊಡೆದೆಬ್ಬಿಸಿ, ಜ್ಞಾನ ಭಕ್ತಿ ವೈರಾಗ್ಯ ಸೇವಾತತ್ತ್ವಗಳನ್ನು ಉಚ್ಚಕಂಠದಿಂದ ಸಾರುವ ತೂರ್ಯವಾಣಿಯೊಂದು ಕೇಳಿಬರು ತ್ತಿದೆ. ಹಿಮಾಲಯದಿಂದ ಬೀಸಿ ಬರುವ ಪುಣ್ಯ ಸಮೀರಣದಂತೆ, ನಿರ್ಜೀವದಂತೆ ಇದ್ದ ಅಸ್ಥಿಮಾಂಸಗಳಿಗೆ ಜೀವದಾನ ಮಾಡುತ್ತಿದೆ. ಜಡನಿದ್ರೆಯನ್ನು ಪರಿಹರಿಸು ತ್ತಿದೆ. ಕಾರ್ಯೋತ್ಸಾಹ ಸ್ಥೈರ್ಯ ಧೈರ್ಯಗಳನ್ನು ಉದ್ರೇಕಿಸುತ್ತಿದೆ. ಕುರುಡರಿಗೆ ಕಾಣದು. ಮೂರ್ಖರಿಗೆ ತಿಳಿಯದು. ನಮ್ಮೀ ಭಾರತಭೂಮಿ ಯುಗಯುಗಗಳ ನಿದ್ರೆಯಿಂದೇಳುತ್ತಿದೆ. ಆಕೆಯನ್ನು ಇನ್ನು ಯಾರೂ ತಡೆಯಬಲ್ಲವರಿಲ್ಲ. ಇನ್ನಾಕೆ ನಿದ್ರೆ ಮಾಡುವುದಿಲ್ಲ. ಯಾವ ಶಕ್ತಿಯೂ ಆಕೆಯನ್ನು ಬಗ್ಗಿಸಲಾರದು. ಏಕೆಂದರೆ ಅದೋ ನೋಡಿ! ಮಹಾಕಾಳಿ ಮತ್ತೊಮ್ಮೆ ಎಚ್ಚೆತ್ತು ಮೈ ಕೊಡಹಿ ಉಸಿರೆಳೆದು ನಿಲ್ಲುತ್ತಿರುವಳು.”

ಜೀವನವನ್ನು ಒಂದು ವರದಂತೆ ಸ್ವೀಕರಿಸಿ ಅವರು ಪ್ರಪಂಚಕ್ಕೆ ಬಂದರು. ಬಾಳನ್ನು ಹಳಿಯಲಿಲ್ಲ, ತುಚ್ಛವಾಗಿ ಕಾಣಲಿಲ್ಲ. ಇಲ್ಲಿ ಧೈರ್ಯದಿಂದಲೂ ಸಾಹಸ ದಿಂದಲೂ ಬಾಳಿದರು. ಹಾಗೆ ಬಾಳುವಂತೆ ಎಲ್ಲರ ಮೇಲೆಯೂ ತಮ್ಮ ಪ್ರಭಾವ ವನ್ನು ಬೀರಿದರು. ತ್ಯಾಗ ಸೇವೆ ಮತ್ತು ಪ್ರೇಮದ ಪರಿಮಳವನ್ನು ದಿಕ್ಕುದಿಕ್ಕಿಗೂ ಬೀರಿದರು. ತಮ್ಮ ಮುಕ್ತಿಯನ್ನು ಬದಿಗೊತ್ತಿ “ನಾನು ಎಷ್ಟು ಬಾರಿ ಬೇಕಾದರೂ ಪ್ರಪಂಚಕ್ಕೆ ಬರಲು ಸಿದ್ಧನಾಗಿರುವೆನು, ಪಡಬಾರದ ಕಷ್ಟವನ್ನು ಅನುಭವಿಸಲು ಸಿದ್ಧನಾಗಿರುವೆನು, ಇದರಿಂದ ಯಾವುದಾದರೊಂದು ಜೀವಿ ಭವಬಂಧನದಿಂದ ಪಾರಾದರೆ ಸಾಕು, ಅದಕ್ಕೆ ಶಾಶ್ವತ ಶಾಂತಿ ಸಿಕ್ಕಿದರೆ ಸಾಕು” ಎಂಬ ವಿಶ್ವಾನುಕಂಪ ದಿಂದ ತುಂಬಿ ತುಳಕಾಡುತ್ತಿದ್ದ ಹೃದಯ ಅವರದು.

ಸ್ವಾಮಿ ವಿವೇಕಾನಂದರ ವಾಣಿ ಸ್ವಾತಂತ್ರ್ಯಪೂರ್ವದಲ್ಲಿ ಎಷ್ಟು ಅವಶ್ಯ ವಾಗಿತ್ತೋ, ಸ್ವಾತಂತ್ರ್ಯ ಉತ್ತರದಲ್ಲಿ ಅದಕ್ಕಿಂತಲೂ ಹೆಚ್ಚು ಆವಶ್ಯಕವಾಗಿದೆ. ಅವರ ವಾಣಿ ಭೈರವನ ತಂಬಟೆಯಂತೆ. ನಮ್ಮನ್ನು ಸುಪ್ತಸ್ಥಿತಿಯಿಂದ ಜಾಗ್ರತ ಗೊಳಿಸುವುದು. ಅದರಲ್ಲಿ ಮಿಂಚಿನಗೊಂಚಲಿನಲ್ಲಿರುವ ಕಾಂತಿಯಿದೆ. ಅದು ನಮ್ಮ ಜಡತನವನ್ನು ಪರಿಹರಿಸುವುದು. ಈಗ ಪತನದಿಂದ ಅಭ್ಯುದಯದ ಕಡೆಗೆ ಜಾಗ್ರತ ಭರತಖಂಡದ ರಥ ಸಾಗುತ್ತಿದೆ. ದಾರಿ ಕಡಿದು. ಆದರೆ ಸರ್ವರೂ ಮನಸ್ಸು ಮಾಡಿ ನೂಕಿದರೆ ಎಂತಹ ದುರ್ಗಮವಾದ ಸ್ಥಳವನ್ನಾದರೂ ಏರ ಬಲ್ಲದು. ಸ್ವಾಮೀಜಿಯವರ ವಾಣಿ ಹಾಗೆ ನೂಕಲು ಸೊಂಟಕಟ್ಟಿ ನಿಂತವರ ಹೃದಯದಲ್ಲಿ ಶ್ರದ್ಧೆ, ಅದಮ್ಯ ಉತ್ಸಾಹ, ಛಲ ಇವುಗಳೆಲ್ಲ ಸ್ಪಂದಿಸುವಂತೆ ಮಾಡಬಲ್ಲದು.

\begin{flushright}
ಸೋಮನಾಥಾನಂದ
\end{flushright}

