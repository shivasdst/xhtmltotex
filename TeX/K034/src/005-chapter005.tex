
\chapter{ಜನಸಾಧಾರಣರನ್ನು ಮೇಲೆತ್ತುವುದು}

\textbf{ಅವರ ಯೋಗ್ಯತೆ}: ನಮ್ಮ ಜನಸಾಧಾರಣರಿಗೆ ಈ ಪ್ರಪಂಚದ ವಿಷಯ ತಿಳಿ ಯದು. ನಮ್ಮ ಜನಸಾಧಾರಣರು ತುಂಬಾ ಒಳ್ಳೆಯವರು. ಏಕೆಂದರೆ ಇಲ್ಲಿ ಬಡತನ ಒಂದು ಅಪರಾಧವಲ್ಲ. ನಮ್ಮ ಜನಸಾಧಾರಣರು ಉದ್ರೇಕಗೊಳ್ಳುವು ದಿಲ್ಲ. ಅನೇಕ ವೇಳೆ ನಾನು ಅಮೆರಿಕಾ ಮತ್ತು ಇಂಗ್ಲೆಂಡುಗಳಲ್ಲಿ ಇದ್ದಾಗ ನನ್ನ ವೇಷಭೂಷಣಗಳಿಗಾಗಿ ಜನ ನನ್ನನ್ನು ಮುತ್ತುತ್ತಿದ್ದರು. ಆದರೆ ಭಾರತ ದೇಶದಲ್ಲಿ ಒಬಪ್ ವಿಚಿತ್ರವಾಗಿ ಬಟ್ಟೆ ತೊಡುತ್ತಾನೆ ಎಂದು ಯಾರೂ ಅವನನ್ನು ಮುತ್ತುವು ದಿಲ್ಲ. ಯೂರೋಪಿನ ಜನಸಾಧಾರಣರಿಗಿಂತ ಇನ್ನು ಎಲ್ಲಾ ವಿಧದಲ್ಲಿಯೂ ನಮ್ಮ ವರು ಬಹಳ ಒಳ್ಳೆಯವರು. ಅನುಭವದಿಂದ ನಾನು ಹೇಳಬಲ್ಲೆ. ಅವರು ದಡ್ಡರಲ್ಲ ಮೂರ್ಖರಲ್ಲ. ಅವರಿಗೆ ವಿಷಯವನ್ನು ತಿಳಿದುಕೊಳ್ಳಬೇಕೆಂದು ಇತರರಿಗೆ ಇರುವಂತೆಯೇ ಆಸಕ್ತಿ ಇದೆ.

ಭಾರತದೇಶದ ನಿಮ್ನವರ್ಗದವರು, ಕೂಲಿಗಳು, ನೆಯ್ಗೆಯವರು ಮುಂತಾದವ ರನ್ನು ಹೊರಗಿನಿಂದ ಬಂದ ಜನರು ಗೆದ್ದಿರುವರು. ಇಲ್ಲಿರುವ ಜನ ಅವರನ್ನು ನಿಕೃಷ್ಟದೃಷ್ಟಿಯಿಂದ ನೋಡುವರು. ಇವರೇ ಸಾವಿರಾರು ವರುಷಗಳಿಂದ ಮೌನ ವಾಗಿ ದೇಶದಲ್ಲಿ ದುಡಿಯುತ್ತಿರುವರು. ಅವರಿಗೆ ತಮ್ಮ ಶ್ರಮಕ್ಕೆ ತಕ್ಕ ಫಲ ದೊರಕುವುದಿಲ್ಲ. ನಿಮಗಿಂತ, ವಿದ್ಯಾವಂತರಿಗಿಂತ ಚಮ್ಮಾರ ಕೂಲಿ ಜಾಡಮಾಲಿ ಮುಂತಾದವರಲ್ಲಿ ಹೆಚ್ಚು ಕೆಲಸ ಮಾಡುವ ಪ್ರವೃತ್ತಿ ಇದೆ ಮತ್ತು ಅವರು ವಿದ್ಯಾ ವಂತರಿಗಿಂತ ಹೆಚ್ಚು ಸ್ವಾವಲಂಬಿಗಳು. ಸಾವಿರಾರು ವರುಷಗಳಿಂದ ಅವರು ಕಷ್ಟ ಪಟ್ಟು ಕೆಲಸಮಾಡುತ್ತ ಗೊಣಗಾಡದೆ ದೇಶದ ಐಶ್ವರ್ಯವನ್ನು ಸೃಷ್ಟಿಮಾಡುತ್ತಿರು ವರು.

ನಿಮ್ಮಂತೆ ಅವರು ಕೆಲವು ಪುಸ್ತಕಗಳನ್ನು ಓದದೆ ಇದ್ದರೆ ಚಿಂತೆ ಇಲ್ಲ. ನಿಮ್ಮಂತೆ ಅವರು ಇತರರನ್ನು ಅನುಕರಿಸದೇ ಇದ್ದರೆ ಚಿಂತೆಯಿಲ್ಲ. ಆದರೆ ಇವು ಗಳಿಂದ ಏನು ಪ್ರಯೋಜನ? ಎಲ್ಲಾ ದೇಶಗಳಲ್ಲಿಯೂ ಅವರೇ ಜನಾಂಗದ ಬೆನ್ನು ಮೂಳೆ. ಈ ನಿಮ್ನ ವರ್ಗದವರು ಕೆಲಸಮಾಡುವುದನ್ನು ಬಿಟ್ಟರೆ ನಿಮಗೆ ಊಟ ಬಟ್ಟೆಗಳು ಹೇಗೆ ಸಿಕ್ಕಬಲ್ಲವು? ಕಲ್ಕತ್ತೆಯ ಜಾಡಮಾಲಿಗಳು ಒಂದು ದಿನ ಕೆಲಸ ನಿಲ್ಲಿಸಿದರೆ ಇಡೀ ಊರೇ ಅಸ್ತವ್ಯಸ್ತವಾಗುವುದು. ಅವರು ಮೂರುದಿನ ಕೆಲಸ ನಿಲ್ಲಿಸಿದರೆ ಊರಿನಲ್ಲಿ ಸಾಂಕ್ರಾಮಿಕ ಜಾಡ್ಯ ತಲೆದೋರಿ ಊರಿಗೆ ಊರೇ ಹಾಳಾಗು ವುದು. ಕೂಲಿಯವರು ಕೆಲಸ ನಿಲ್ಲಿಸಿದರೆ ನಮಗೆ ಸಿಕ್ಕುವ ಆಹಾರ ಬಟ್ಟೆ ಬರೆಗಳು ನಿಂತಂತೆ. ನೀವು ಅವರನ್ನು ನಿಮ್ನವರ್ಗದವರೆಂದು ಅಲ್ಲಗಳೆಯುತ್ತೀರಿ. ನಿಮ್ಮ ಸಮಾನ ನಾಗರಿಕರಿಲ್ಲ ಎಂದು ಮೆರೆಯುತ್ತೀರಿ.

ಭರತಖಂಡದ ಶ್ರಮಜೀವಿಗಳ ಮೌನವಾದ ದುಡಿತದಿಂದ ಬ್ಯಾಬಿಲೋನಿಯಾ, ಪರ್ಶಿಯಾ, ಅಲೆಗ್ಸಾಂಡ್ರಿಯ, ಗ್ರೀಸ್, ರೋಮ್, ವೆನಿಸ್, ಜಿನೀವಾ, ಬಾಗ್ ದಾದ್, ಸಾಮರ್​ಖಂಡ, ಸ್ಪೆಯಿನ್, ಪೋರ್ಚುಗಲ್, ಫ್ರಾನ್ಸ್, ಡೆನ್​ಮಾರ್ಕ್, ಹಾಲೆಂಡ್, ಇಂಗ್ಲೆಂಡ್ ಇವುಗಳೆಲ್ಲ ಪ್ರವರ್ಧಮಾನಕ್ಕೆ ಬಂದುವು. ಆದರೆ ಇಂದು ಇವರನ್ನು ಯಾರು ಚಿಂತಿಸುವರು? ಪ್ರಪಂಚದ ಪ್ರಗತಿಗೆ ಕಾರಣವಾದ ಇವರನ್ನು ಯಾರು ಹೊಗಳುವರು? ಆಧ್ಯಾತ್ಮಿಕ ಜೀವನದ ವೀರರು, ಯುದ್ಧದಲ್ಲಿ ಜಯವನ್ನು ಸಂಪಾದಿಸಿದವನು, ಸಾಹಿತ್ಯ ಪ್ರಪಂಚದಲ್ಲಿ ಹೆಸರಾಂತ ವ್ಯಕ್ತಿಗಳು ಇವರನ್ನು ಪ್ರಪಂಚ ಕೊಂಡಾಡುವುದು. ಆದರೆ ನಮ್ಮ ಕೆಲಸಗಾರರಾದರೊ ಹಗಲು ರಾತ್ರಿ ತಾಳ್ಮೆಯಿಂದ ಪ್ರೀತಿಯಿಂದ ಕೌಶಲದಿಂದ ಸ್ವಲ್ಪವೂ ಗೊಣಗಾಡದೆ ಕೆಲಸ ಮಾಡುತ್ತಿದ್ದರೂ ಇವರನ್ನು ಯಾರೂ ಲೆಕ್ಕಿಸುವುದಿಲ್ಲ. ಇವರನ್ನು ಯಾರೂ ಪ್ರೋತ್ಸಾಹಿಸುವುದಿಲ್ಲ. ಅನೇಕರು ಮಹತ್​ಕಾರ್ಯವನ್ನು ಮಾಡಿ ದಾಗ ದೊಡ್ಡ ವೀರ ರಾಗುವರು. ಜನಸಂದಣಿ ಯಾವಾಗ ಒಬ್ಬನನ್ನು ನೋಡುತ್ತದೆಯೊ ಆಗ ಹೇಡಿಯೂ ತನ್ನ ಪ್ರಾಣವನ್ನು ಸುಲಭವಾಗಿ ಕೊಡುವನು, ಅತ್ಯಂತ ಸ್ವಾರ್ಥನೂ ಅನಾಸಕ್ತನಾಗಿರಲು ಯತ್ನಿಸುವನು. ಇತರರು ನೋಡದೆ ಇರುವಾಗ ತನ್ನ ಪಾಲಿಗೆ ಬಂದ ಯಾವ ಕೆಲಸವನ್ನಾದರೂ ನಿಃಸ್ವಾರ್ಥವಾಗಿ ದಕ್ಷತೆಯಿಂದ ಮಾಡುವವನೇ ಧನ್ಯ. ಭರತಖಂಡದ ಶ್ರಮಜೀವಿಗಳೆ! ಇತರರು ನಿಮ್ಮನ್ನು ತುಳಿಯುತ್ತಿದ್ದರೂ, ನೀವು ನಿಮ್ಮ ಕರ್ತವ್ಯ ಮಾಡಿ. ನಿಮ್ಮನ್ನು ಗೌರವಿಸುವೆನು.

\textbf{ಅವರ ಈಗಿನ ದುಃಸ್ಥಿತಿ ಮತ್ತು ಅದಕ್ಕೆ ಕಾರಣಗಳು}: ಸಮಾಜದ ನಾಯಕತ್ವ ವಿದ್ಯಾವಂತರ ಕೈಯಲ್ಲಿರಲಿ, ಐಶ್ವರ್ಯವಂತನ ಕೈಯಲ್ಲಿರಲಿ ಕ್ಷತ್ರಿಯನ ಕೈಯ ಲ್ಲಿರಲಿ, ಅದರ ಶಕ್ತಿಯ ಮೂಲ ಜನಸಾಧಾರಣರು. ಯಾವುದಾದರೂ ಈ ಮೂಲ ದಿಂದ ಬೇರಾದರೆ ಅದು ಅಷ್ಟೂ ದುರ್ಬಲವಾಗುವುದು. ಆದರೆ ಇದೊಂದು ವಿಧಿಯ ಲೀಲೆ, ಇದೊಂದು ಮಾಯೆ. ಯಾವ ನಾಯಕರು ಪ್ರತ್ಯಕ್ಷವಾಗಿಯೋ ಅಪ್ರತ್ಯಕ್ಷವಾಗಿಯೋ ನ್ಯಾಯವಾಗಿಯೋ ಅನ್ಯಾಯವಾಗಿಯೋ ಯಾರಿಂದ ಈ ಅಧಿಕಾರವನ್ನು ಮೌನದಿಂದಲೋ ಉಪಾಯದಿಂದಲೋ ಬಲಾತ್ಕಾರದಿಂದಲೋ ಪಡೆದುಕೊಂಡರೊ ಅವರನ್ನು ಸ್ವಲ್ಪವೂ ಗಮನಿಸಲಿಲ್ಲ.

ವೇದಾಂತದ ತೌರುಮನೆಯಾದ ಈ ನಮ್ಮ ದೇಶದಲ್ಲಿ ನಮ್ಮ ಜನ ಸಾಧಾರಣರು ಸಾವಿರಾರು ವರುಷಗಳಿಂದ ತಾವು ಕೆಲಸಕ್ಕೆ ಬಾರದವರು ಎಂದು ಭಾವಿಸುವರು. ಅವರನ್ನು ಮುಟ್ಟಿದರೆ ಮೈಲಿಗೆ, ಅವರೊಡನೆ ಕುಳಿತರೆ ಮೈಲಿಗೆ. ಅವರು ಹುಟ್ಟಿದಾಗ ಶೋಚನೀಯ ಅವಸ್ಥೆಯಲ್ಲಿದ್ದರೂ, ಅನಂತರವೂ ಅವರು ಹಾಗೆಯೆ ಇರಬೇಕು. ಇದರ ಪರಿಣಾಮವಾಗಿ ಅವರು ಅಧಃಪಾತಾಳಕ್ಕೆ ಇಳಿಯು ತ್ತಿರುವರು. ಮನುಷ್ಯ ಎಷ್ಟು ಕೆಳಗೆ ಹೋಗಬಹುದೊ ಅಷ್ಟು ಕೆಳಗೆ ಬಂದಿರುವರು. ಪ್ರಪಂಚದ ಮತ್ತಾವ ದೇಶದಲ್ಲಿ ಜನ ದನದೊಂದಿಗೆ ಮಲಗುತ್ತಾರೆ? ಇದಕ್ಕೆ ಯಾರನ್ನೂ ಬೈಯ್ಯಬೇಡಿ. ಏನೂ ತಿಳಿಯದವರು ಮಾಡುವ ತಪ್ಪನ್ನು ನೀವು ಮಾಡ ಬೇಡಿ. ಇದರ ಪರಿಣಾಮ ಇಲ್ಲಿದೆ. ನಾವೇ ಅದಕ್ಕೆ ಹೊಣೆ. ಧೈರ್ಯವಾಗಿ ಎದ್ದು ನಿಂತು ಜವಾಬ್ದಾರಿಯನ್ನು ತೆಗೆದುಕೊಳ್ಳಿ. ಇತರರನ್ನು ಸುಮ್ಮನೆ ದೂರ ಬೇಡಿ. ನಿಮ್ಮಲ್ಲಿರುವ ನ್ಯೂನತೆಗೆಲ್ಲ ನೀವೇ ಕಾರಣ.

ನಮ್ಮ ಪೂರ್ವಕಾಲದ ಕುಲೀನವಂಶಸ್ಥರು ನಮ್ಮ ಜನಸಾಧಾರಣರನ್ನು ತುಳಿ ಯುತ್ತಾ ಹೋದರು. ಈ ಅನಾಚಾರದಿಂದ ಪಾಪ ಆ ಜನ ತಾವು ಮನುಷ್ಯರು ಎಂಬುದನ್ನು ಮರೆತರು. ಅತ್ಯಂತ ಕನಿಷ್ಠ ಕೆಲಸಗಳೇ ಅವರ ಪಾಲಿಗೆ ಬಂದಿವೆ. ಭರತಖಂಡದಲ್ಲಿ ಬಡವರು ನಿಮ್ನವರ್ಗದವರು ಪಾಪಿಗಳು ಇವರಿಗೆ ಯಾರೂ ಸ್ನೇಹಿತರಿಲ್ಲ. ಇವರಿಗೆ ಯಾವ ಸಹಾಯವೂ ಬರುವುದಿಲ್ಲ. ಅವರು ತಾವು ಎಷ್ಟೇ ಪ್ರಯತ್ನಪಟ್ಟರೂ ಮೇಲೇಳಲಾರರು. ಪ್ರತಿದಿನವೂ ಅವರು ಆಳಆಳಕ್ಕೆ ಮುಳುಗು ತ್ತಿರುವರು. ಈ ನಿರ್ದಯ ಸಮಾಜ ಅವರಿಗೆ ಕೊಡುವ ಪೆಟ್ಟು ತಾಕುತ್ತಿದೆ. ಆದರೆ ಅದು ಎಲ್ಲಿಂದ ಬರುತ್ತಿದೆ ಎಂಬುದು ಅವರಿಗೆ ಗೊತ್ತಿಲ್ಲ. ಅವರು ತಾವು ಕೂಡ ಮನುಷ್ಯರು ಎಂಬುದನ್ನು ಮರೆತುಬಿಟ್ಟಿರುವರು. ಗುಲಾಮಗಿರಿಯೆ ಇದರ ಪರಿ ಣಾಮ.

ಕೆಲವು ಮೇಧಾವಿಗಳು ಕೆಲವು ವರುಷಗಳಿಂದ ಇದನ್ನು ನೋಡಿರುವರು. ಆದರೆ ಇದಕ್ಕಾಗಿ ಸುಮ್ಮನೆ ಧರ್ಮವನ್ನು ದೂರುವರು. ಅವರಿಗೆ ಸಾಧಾರಣ ಜನರನ್ನು ಮೇಲಕ್ಕೆ ತರಲು ತೋರುವ ಮಾರ್ಗ ಒಂದೇ ಕಾಣುವುದು. ಅದೇ ಪ್ರಪಂಚದ ಶ್ರೇಷ್ಠತಮ ಧರ್ಮವನ್ನು ನಾಶಮಾಡುವುದು. ಭಗವಂತನ ದಯೆಯಿಂದ ನನಗೆ ಅದರ ರಹಸ್ಯ ಗೊತ್ತಾಗಿದೆ. ಧರ್ಮ ಅಲ್ಲ ಅದಕ್ಕೆ ಕಾರಣ. ನಿಮ್ಮ ಧರ್ಮ ನೀನೆ ಹಲವದರಂತೆ ಕಾಣುತ್ತಿರುವೆ ಎಂದು ಸಾರುವುದು. ಆದರೆ ಅದನ್ನು ಅನುಷ್ಠಾನಕ್ಕೆ ತರಲಿಲ್ಲ. ನಮ್ಮಲ್ಲಿ ದಯೆಯಿಲ್ಲ, ಹೃದಯದ ಔದಾರ್ಯತೆ ಇಲ್ಲ. ಭಗವಂತ ಮತ್ತೊಮ್ಮೆ ಬುದ್ಧನಂತೆ ಬಂದು ದೀನರಿಗೆ ದಲಿತರಿಗೆ ಪಾಪಿಗಳಿಗೆ ಹೇಗೆ ಮರುಕ ತೋರುವುದು ಎಂಬುದನ್ನು ತೋರಿದ. ತಮಗೊಂದು ಮನೆಯಿಲ್ಲದೆ ಎಲ್ಲಾ ದೇಶದ ಗುಲಾಮರಾಗಿ ಹೊಟ್ಟೆಗಿಲ್ಲದೆ ಅಲೆಯುತ್ತಿರುವುದೇ ಅವರ ಪಾಡಾಗಿದೆ. ರಾಕ್ಷಸರೆ! ನಿಮಗೆ ಗೊತ್ತಿಲ್ಲ, ನಾಣ್ಯದ ಒಂದು ಕಡೆ ಪೀಡಿಸುವ ರಾಕ್ಷಸತ್ವ, ಅದರ ಮತ್ತೊಂದು ಕಡೆಯೇ ಗುಲಾಮಗಿರಿ. ರಾಕ್ಷಸ ಮತ್ತು ಗುಲಾಮ ಇವೆರಡೂ ಒಟ್ಟಿಗೇ ಹೋಗುವುವು. ಒಂದಿದ್ದರೆ ಮತ್ತೊಂದು ಇರುವುದು.

ಹಿಂದೂಧರ್ಮದಷ್ಟು ಶ್ರೇಷ್ಠವಾಗಿ ಮಾನವನ ಮಹಿಮೆಯನ್ನು ಸಾರುವ ಧರ್ಮ ಪ್ರಪಂಚದಲ್ಲಿ ಮತ್ತೊಂದು ಇಲ್ಲ. ಆದರೆ ಹಿಂದುಗಳಷ್ಟು ಕ್ರೂರವಾಗಿ ದೀನದಲಿತರನ್ನು ಪೀಡಿಸುವಷ್ಟು ಮತ್ತೊಂದು ಜನಾಂಗವಿಲ್ಲ. ಇದಕ್ಕೆಲ್ಲ ಧರ್ಮ ಕಾರಣವಲ್ಲ ಎಂಬುದನ್ನು ದೇವರು ನನಗೆ ತೋರಿರುವನು. ನಮ್ಮ ಹಿಂದೂ ಧರ್ಮದಲ್ಲಿರುವ ಕಪಟಿಗಳು, ವಂಚಕರೇ ಇದಕ್ಕೆಲ್ಲ ಕಾರಣ. ಜನರನ್ನು ಸುಲಿಯು ವುದಕ್ಕೆ ಎಲ್ಲಾ ವಿಧವಾದ ಉಪಾಯಗಳನ್ನು ಅವರು ಕಂಡುಹಿಡಿದಿರುವರು. ಪರೆಯರನ್ನು ಮತ್ತು ಜಾಡಮಾಲಿಗಳನ್ನು ಯಾರು ಈ ದುಃಸ್ಥಿತಿಗೆ ತಂದವರು? ನಾವು ದೊಡ್ಡ ವೇದಾಂತವನ್ನು ಬೋಧಿಸುತ್ತಿದ್ದರೂ ಹೃದಯ ನಿರ್ದಯವಾಗಿದೆ. ಇದು ಗಾಯದ ಮೇಲೆ ಉಪ್ಪು ನೀರು ಎರಚಿದಂತೆ. ಪ್ರಪಂಚದಲ್ಲೆಲ್ಲ ಶ್ರೇಷ್ಠತಮ ವಾದ ಧರ್ಮ ನಿಮ್ಮಲ್ಲಿದೆ. ಆದರೆ ನೀವು ಜನಸಾಧಾರಣರಿಗೆ ಕೆಲಸಕ್ಕೆ ಬಾರದ ವಸ್ತು ವನ್ನು ಧರ್ಮದ ಹೆಸರಿನಲ್ಲಿ ಕೊಡುತ್ತೀರಿ. ನಿಮ್ಮಲ್ಲಿ ಎಂದಿಗೂ ಬತ್ತದ ಅಮೃತ ಪ್ರವಾಹ ಹರಿಯುತ್ತಿದೆ. ಆದರೆ ನೀವು ಅವರಿಗೆ ಕೊಡುವುದು ಬಚ್ಚಲ ನೀರು. ನಿಮ್ಮ ಪದವೀಧರನಿಗೆ ಆ ನೀಚ ಕುಲದವನನ್ನು ಮುಟ್ಟಲು ಆಗುವುದಿಲ್ಲ. ಆದರೆ ಅವನು ತನ್ನ ವಿದ್ಯಾಭ್ಯಾಸಕ್ಕೆ ಅವನಿಂದ ದುಡ್ಡನ್ನು ತೆಗೆದುಕೊಳ್ಳುವನು.

\textbf{ಇದರ ಪರಿಹಾರೋಪಾಯ ಮತ್ತು ನಮ್ಮ ಜವಾಬ್ದಾರಿ}: ಜೀವನೋಪಾಯದಲ್ಲೆ ಅವರು ಮುಳುಗಿದ್ದುದರಿಂದ ಅವರಿಗೆ ಜ್ಞಾನವನ್ನು ಪಡೆಯುವುದಕ್ಕೆ ಅವಕಾಶವೇ ಇರಲಿಲ್ಲ. ಇಷ್ಟು ಕಾಲವು ಅವರು ಬುದ್ಧಿವಂತರಾದ ಯಂತ್ರಗಳಂತೆ ಕೆಲಸ ಮಾಡಿರುವರು. ಸಮಾಜದಲ್ಲಿ ಕೆಲವು ವಿದ್ಯಾವಂತರಾದ ಬುದ್ಧಿವಂತರು ಅವರ ಶ್ರಮದ ಫಲವನ್ನೆಲ್ಲ ಕಸಿದುಕೊಂಡಿರುವರು. ಪ್ರತಿಯೊಂದು ದೇಶದಲ್ಲಿಯೂ ಹೀಗೆಯೆ. ಆದರೆ ಈಗ ಕಾಲ ಬದಲಾಯಿಸಿದೆ. ನಿಮ್ನ ವರ್ಗದವರಿಗೆ ಇದು ಕ್ರಮೇಣ ಗೊತ್ತಾಗುತ್ತಿದೆ. ತಮ್ಮ ಹಕ್ಕು ಬಾಧ್ಯತೆಗಳನ್ನು ಪಡೆಯುವುದಕ್ಕಾಗಿ ಐಕಮತ್ಯದಿಂದ ಅವರು ಪ್ರಯತ್ನ ಪಡುತ್ತಿರುವರು. ಅಮೆರಿಕ ಮತ್ತು ಯೂರೋಪಿನ ಜನಸಾಧಾರಣರು ಮೊದಲು ಜಾಗ್ರತರಾದರು. ಅವರು ಆಗಲೆ ಇದಕ್ಕೆ ಹೋರಾಡುತ್ತಿರುವರು. ಭರತಖಂಡದಲ್ಲಿ ಕೂಡ ಈ ಕೆಳಮಟ್ಟದಲ್ಲಿ ಕೆಲಸ ಮಾಡುತ್ತಿರುವವರ ಮುಷ್ಕರಗಳನ್ನು ನೋಡಿದರೆ ಇಲ್ಲಿಯೂ ಅವರು ಜಾಗ್ರತ ರಾಗುತ್ತಿರುವುದು ಕಾಣುವುದು. ಮೇಲಿನವರು ಎಷ್ಟೇ ಪ್ರಯತ್ನಪಟ್ಟರೂ ಕೆಳಗಿನವ ರನ್ನು ಇನ್ನು ಮೇಲೆ ತುಳಿಯುವುದಕ್ಕೆ ಆಗುವುದಿಲ್ಲ. ಮೇಲಿರುವವರು ತಮ್ಮ ಹಿತರಕ್ಷಣೆಯ ದೃಷ್ಟಿಯಿಂದಲೇ ಕೆಳಗಿರುವವರಿಗೆ ತಮ್ಮ ಹಕ್ಕುಬಾಧ್ಯತೆಗಳನ್ನು ಪಡೆಯಲು ಸಹಾಯ ಮಾಡಬೇಕು.

ಅಮೆರಿಕಾ ದೇಶದಲ್ಲಿ ಪ್ರತಿಯೊಬ್ಬನೂ ಮೇಲಕ್ಕೆ ಬರಲು ಸಾಧ್ಯ. ಅವನಿಗೆ ಬೇಕಾದಷ್ಟು ಅವಕಾಶಗಳಿವೆ. ಅವನು ಈಗ ಬಡವನಾಗಿರಬಹುದು, ಮುಂದೆ ಶ್ರೀಮಂತನಾಗಿ ವಿದ್ಯಾವಂತನಾಗಿ ಗೌರವಸ್ಥನಾಗಬಹುದು. ಅಮೆರಿಕಾದೇಶದಲ್ಲಿ ಪ್ರತಿಯೊಬ್ಬರೂ ಬಡವರಿಗೆ ಸಹಾಯ ಮಾಡಲು ಕಾತರರಾಗಿರುತ್ತಾರೆ. ಭರತ ಖಂಡದಲ್ಲಿ ನಾವು ಬಡವರು ಎಂಬ ಕೂಗೇನೊ ಕೇಳಿಸುತ್ತಿದೆ. ಆದರೆ ಬಡವರ ಹಿತರಕ್ಷಣೆಗೆ ಎಷ್ಟು ಸಂಸ್ಥೆಗಳಿವೆ? ಭರತಖಂಡದಲ್ಲಿರುವ ಕೋಟ್ಯಂತರ ಬಡ ಬಗ್ಗರಿಗಾಗಿ ಎಷ್ಟು ಜನ ನಿಜವಾಗಿ ವ್ಯಥೆ ಪಡುವರು? ನಾವು ನಿಜವಾಗಿಯೂ ಮನುಷ್ಯರೆ? ನಾವು ಬಡವರ ಜೀವನೋಪಾಯಕ್ಕೆ, ಅವರನ್ನು ಮುಂದೆ ತರುವುದಕ್ಕೆ ಏನು ಮಾಡುತ್ತಿರುವೆವು? ನಾವು ಅವರನ್ನು ಮುಟ್ಟುವುದಿಲ್ಲ. ಅವರೊಡನೆ ಬೆರೆಯು ವುದಿಲ್ಲ. ನಾವು ನಿಜವಾಗಿಯೂ ಮಾನವರೆ?

ಯೂರೋಪ್ ದೇಶದಲ್ಲಿ ಹಲವು ನಗರಗಳನ್ನು ನೋಡಿದ ಮೇಲೆ, ಅಲ್ಲಿ ಅತಿ ದರಿದ್ರರಿಗೂ ಸಿಕ್ಕುವ ವಿದ್ಯಾಭ್ಯಾಸ ಮತ್ತು ಸೌಕರ್ಯಗಳನ್ನು ಕಂಡು ನಮ್ಮ ಜನರ ದುಃಸ್ಥಿತಿ ನನ್ನ ಮನಸ್ಸಿಗೆ ತಾಕಿ ನಾನು ಕಣ್ಣೀರಿಟ್ಟೆ. ಇದಕ್ಕೆ ಕಾರಣ ಯಾವುದು? ಇದಕ್ಕೆ ದೊರೆತ ಉತ್ತರವೇ ವಿದ್ಯಾಭ್ಯಾಸ. ವಿದ್ಯಾಭ್ಯಾಸದಿಂದ ಆತ್ಮಶ್ರದ್ಧೆ ಜಾಗ್ರತ ವಾಗುವುದು. ಆತ್ಮಶ್ರದ್ಧೆಯಿಂದ ಬ್ರಹ್ಮಸಾಕ್ಷಾತ್ಕಾರವಾಗುವುದು. ಆದರೆ ನಮ್ಮ ಲ್ಲಾದರೊ ಬ್ರಹ್ಮನು ಕ್ರಮೇಣ ಸುಪ್ತಾವಸ್ಥೆಗೆ ಹೋಗುತ್ತಿರುವನು. ನಾನು ನ್ಯೂ ಯಾರ್ಕಿನಲ್ಲಿದ್ದಾಗ ಐರ್​ಲೆಂಡ್ ದೇಶದಿಂದ ವಲಸೆ ಬಂದ ಜನರನ್ನು ನೋಡು ತ್ತಿದ್ದೆ. ಮನೆಯಲ್ಲಿದ್ದಾಗ ದಬ್ಬಾಳಿಕೆಗೆ ತುತ್ತಾಗಿ ನಿತ್ರಾಣರಾಗಿ, ಮನೆಯಲ್ಲಿ ಒಂದು ಚೂರು ಸಾಮಾನಿಲ್ಲದೆ ಏನು ಮಾಡಬೇಕೆಂದು ಗೊತ್ತಿಲ್ಲದೆ ಕಂಗಾಲಾಗಿದ್ದರು. ಅವರು ಅಮೆರಿಕಾದೇಶಕ್ಕೆ ಬರುವಾಗ ಬರೀ ಒಂದು ಕೋಲು ಮತ್ತು ಅದರ ಕೊನೆಯಲ್ಲಿ ನೇತಾಡುತ್ತಿದ್ದ ಚಿಂದಿಯ ಗಂಟು ಮಾತ್ರ ಇತ್ತು. ಅವರ ನಡೆಯಲ್ಲಿ ಅಂಜಿಕೆ, ನೋಟದಲ್ಲಿ ಕಳವಳ ತುಂಬಿತ್ತು. ಅಮೆರಿಕಾದೇಶಕ್ಕೆ ಬಂದ ಆರು ತಿಂಗಳಿ ನಲ್ಲಿಯೇ ಚಿತ್ರ ಬದಲಾಯಿಸುವುದು. ಮನುಷ್ಯ ಧೈರ್ಯವಾಗಿ ದಾರಿಯಲ್ಲಿ ನಡೆ ಯುವನು. ಅವನುಟ್ಟ ಬಟ್ಟೆ ಬದಲಾಯಿಸಿರುವುದು. ಅವನ ನೋಟ ಮತ್ತು ನಡತೆಯಲ್ಲಿ ಅಂಜಿಕೆ ಇಲ್ಲ. ಇದಕ್ಕೆ ಕಾರಣವೇನು? ನಮ್ಮ ವೇದಾಂತ ಹೀಗೆ ಹೇಳು ವುದು: ಐರ್​ಲೆಂಡಿನವನು ತನ್ನ ದೇಶದಲ್ಲಿದ್ದಾಗ ಸುತ್ತಲೂ ತಾತ್ಸಾರ ವಾತಾವರಣ ವಿತ್ತು–ಇಡೀ ದೇಶವೇ “ಪ್ಯಾಟ್, ನಿನಗೆ ಬೇರೆ ಗತಿಯೇ ಇಲ್ಲ. ನೀನು ಗುಲಾಮ ನಾಗಿ ಹುಟ್ಟಿದೆ, ನೀನು ಗುಲಾಮನಾಗಿಯೇ ಇರಬೇಕಾಗುವುದು” ಎಂದು ಹೇಳು ತ್ತಿತ್ತು. ಬಾಲ್ಯಾರಭ್ಯ ಅವನು ಇದನ್ನು ಕೇಳಿ ಕೇಳಿ ಇದರಲ್ಲಿ ನಂಬತೊಡಗಿದನು. ತಾನು ತುಂಬಾ ನೀಚ ಎಂದು ಆಲೋಚಿಸತೊಡಗಿದನು. ಅವನಲ್ಲಿರುವ ಬ್ರಹ್ಮ ಆಳಕ್ಕೆ ಮುಳುಗಿಹೋಯಿತು. ಅವನು ಅಮೆರಿಕಾದೇಶಕ್ಕೆ ಬಂದೊಡನೆಯೇ ಸುತ್ತಲೂ–“ಪ್ಯಾಟ್, ನೀನೂ ನಮ್ಮಂತೆಯೇ ಮನುಷ್ಯ. ಮನುಷ್ಯನೇ ಇದನ್ನೆಲ್ಲ ಮಾಡಿದ್ದು. ನನ್ನಂತೆ ನಿನ್ನಂತೆ ಇರುವ ಮನುಷ್ಯ ಏನನ್ನು ಬೇಕಾದರೂ ಸಾಧಿಸ ಬಹುದು. ಭರವಸೆ ಇರಲಿ” ಎಂಬ ಧ್ವನಿಯನ್ನು ಕೇಳಿದನು. ಪ್ಯಾಟ್ ಧೈರ್ಯವಾಗಿ ಕತ್ತೆತ್ತಿ ನೋಡಿದ. ನಿಜವಾಗಿಯೂ ಹಾಗೆಯೇ ಇತ್ತು. ಅವನಲ್ಲಿ ಸುಪ್ತವಾಗಿದ್ದ ಬ್ರಹ್ಮ ವ್ಯಕ್ತವಾಗತೊಡಗಿದನು. ಎದ್ದೇಳು ಜಾಗ್ರತನಾಗು, ಗುರಿ ಸೇರುವವರೆಗೂ ನಿಲ್ಲಬೇಡ ಎಂದು ಪ್ರಕೃತಿಯೇ ಹೇಳುವಂತೆ ಇತ್ತು.

ಮೇಲಿನವರಿಂದ ಕೆಳಗಿನವರಿಗೆ ಯಾವಾಗ ವಿದ್ಯೆ ಮತ್ತು ಸಂಸ್ಕೃತಿ ಹರಿದು ಬರಲು ಪ್ರಾರಂಭವಾಯಿತೊ ಅಂದಿನಿಂದ ಪಾಶ್ಚಾತ್ಯ ದೇಶದ ಇಂದಿನ ನಾಗರಿ ಕತೆಗೂ ಮತ್ತು ಇಂಡಿಯಾ, ಈಜಿಪ್ಟ್, ರೋಮ್ ಮುಂತಾದ ಪುರಾತನ ಕಾಲದ ನಾಗರಿಕತೆಗೂ ವ್ಯತ್ಯಾಸ ಪ್ರಾರಂಭವಾಯಿತು. ಯಾವ ದೇಶದಲ್ಲಿ ಜನಸಾಧಾರಣ ರಲ್ಲಿ ಹೆಚ್ಚು ವಿದ್ಯೆ ಮತ್ತು ಬುದ್ಧಿ ಹರಡುವುದೋ ಅದಕ್ಕೆ ಸರಿಸಮನಾಗಿ ಆ ದೇಶ ಮುಂದೆ ಬರುವುದನ್ನು ನನ್ನ ಕಣ್ಣೆದುರಿಗೇ ನೋಡುತ್ತಿರುವೆನು. ಭರತಖಂಡದ ಅವನತಿಗೆ ಮುಖ್ಯ ಕಾರಣ ದೇಶದ ವಿದ್ಯಾಬುದ್ಧಿಗಳನ್ನೆಲ್ಲ ಕೆಲವು ವ್ಯಕ್ತಿಗಳು, ರಾಜರ ಶಾಸನ ಸಹಾಯದಿಂದಲೂ ದುರಹಂಕಾರದಿಂದಲೂ ತಮ್ಮ ವಶಮಾಡಿ ಕೊಂಡದ್ದು. ನಾವು ಮೇಲೇಳಬೇಕಾದರೆ ಜನಸಾಧಾರಣರು ವಿದ್ಯಾವಂತರಾಗುವು ದಕ್ಕೆ ಅವಕಾಶವನ್ನು ಕೊಡಬೇಕಾಗಿದೆ. ಅತ್ಯಂತ ದೀನದರಿದ್ರರ ಮನೆಯ ಬಾಗಿ ಲಿಗೂ ಭರತಖಂಡದಲ್ಲಿ ಮತ್ತು ಇತರ ಕಡೆ ಅತ್ಯಂತ ಪ್ರತಿಭಾವಂತರು ಮಾಡಿದ ಶ್ರೇಷ್ಠತಮವಾದ ಆಲೋಚನೆಗಳು ತಲುಪುವಂತೆ ಮಾಡುವುದೇ ನನ್ನ ಆದರ್ಶ. ಅನಂತರ ಅವರು ತಮಗೆ ಸೂಕ್ತ ತೋರಿದ ರೀತಿಯಲ್ಲಿ ಆಲೋಚಿಸಲಿ.

ಯಾರು ಅವರಿಗೆ ಬೆಳಕನ್ನು ಕೊಡುವರು? ಮನೆಯಿಂದ ಮನೆಗೆ ಯಾರು ಅವ ರಿಗೆ ಜ್ಞಾನವನ್ನು ದಾನಮಾಡುತ್ತ ಹೋಗುವರು? ಈ ಜನರೇ ನಿಮ್ಮ ದೇವರಾಗಲಿ. ಅನವರತ ಅವರಿಗಾಗಿ ಕೆಲಸ ಮಾಡಿ, ಅವರಿಗಾಗಿ ಆಲೋಚಿಸಿ, ಅವರಿಗಾಗಿ ಪ್ರಾರ್ಥನೆ ಮಾಡಿ. ದೇವರು ದಾರಿ ತೋರುವನು. ಯಾರ ಹೃದಯ ದೀನದಲಿತರಿಗೆ ಮರುಗುವುದೊ ಆ ಮನುಷ್ಯನನ್ನು ನಾನು ಮಹಾತ್ಮನೆಂದು ಕರೆಯುತ್ತೇನೆ. ಅಲ್ಲದೇ ಇದ್ದರೆ ಅವನೊಬ್ಬ ದುರಾತ್ಮ. ನಾವೆಲ್ಲ ಅವರ ಕಲ್ಯಾಣಕ್ಕಾಗಿ ಒಟ್ಟು ಗೂಡಿ ಪ್ರಾರ್ಥಿಸೋಣ. ನಾವು ಏನನ್ನೂ ಸಾಧಿಸದೆ, ಅಜ್ಞಾತರಾಗಿ ಸಾಯಬಹುದು. ನಮಗೆ ಯಾರೂ ಮರುಕತೋರದೆ ಇರಬಹುದು. ಆದರೆ ಯಾವ ಒಂದು ಒಳ್ಳೆಯ ಆಲೋಚನೆಯೂ ವ್ಯರ್ಥವಾಗುವುದಿಲ್ಲ. ಈಗಲೋ ಇನ್ನು ಸ್ವಲ್ಪ ಕಾಲವಾದ ಮೇಲೊ ಅದು ಪರಿಣಾಮಕಾರಿ ಆಗಿಯೇ ಆಗುವುದು. ನನ್ನ ಹೃದಯದಲ್ಲಿರುವ ಭಾವನೆಗಳನ್ನೆಲ್ಲ ನಿಮಗೆ ಹೇಳುತ್ತಿಲ್ಲ. ಅದು ನಿಮಗೆ ಗೊತ್ತು. ನೀವು ಅದನ್ನು ಊಹಿಸಬಹುದು. ಎಲ್ಲಿಯವರೆಗೂ ಲಕ್ಷಾಂತರ ಜನರು ಉಪವಾಸ ಮತ್ತು ಅಜ್ಞಾನ ದಲ್ಲಿ ನರಳುತ್ತಿರುವರೊ ಅಲ್ಲಿಯವರೆಗೆ ಆ ಜನರ ಸಹಾಯದಿಂದಲೆ ವಿದ್ಯಾವಂತ ರಾಗಿ ಅವರನ್ನು ಸ್ವಲ್ಪವೂ ಲೆಕ್ಕಿಸದವರನ್ನು ದೇಶದ್ರೋಹಿಗಳೆನ್ನುತ್ತೇನೆ. ಇಂದು ವೇಷಭೂಷಣಗಳಿಂದ ವೈಯ್ಯಾರ ಮಾಡುತ್ತಿರುವವರಿಗೆ ಅವರಿಗೆ ಬೇಕಾದ ಹಣ ವೆಲ್ಲ ಈ ದೌರ್ಭಾಗ್ಯರನ್ನು ಸುಲಿದು ಬಂದದ್ದು. ಇಂದು ಉಪವಾಸದಿಂದ ನರಳು ತ್ತಿರುವ ಕಾಡುಮನುಷ್ಯನಿಗಿಂತ ಮೇಲಲ್ಲದ ಇಪ್ಪತ್ತು ಕೋಟಿ ಜನರ ಉದ್ಧಾರಕ್ಕೆ ಕೈಯೆತ್ತಬೇಕೆಂದು ಕೇಳಿಕೊಳ್ಳುತ್ತೇನೆ.

ನಾವು ಅವರಿಗೆ ಲೌಕಿಕವಾದ ಶಿಕ್ಷಣವನ್ನು ಕೊಡಬೇಕು. ನಮ್ಮ ಪೂರ್ವಿಕರ ಯೋಜನೆಯನ್ನು ನಾವು ಅನುಸರಿಸಬೇಕಾಗಿದೆ. ಅಂದರೆ ಎಲ್ಲ ಜನರಿಗೂ ಜ್ಞಾನ ವನ್ನು ನೀಡಬೇಕಾಗಿದೆ. ಅವರನ್ನು ಕ್ರಮೇಣ ಮೇಲಕ್ಕೆ ಎತ್ತಿ, ಅವರನ್ನು ಸರಿಸಮಾನ ರನ್ನಾಗಿ ಮಾಡಿ. ಧರ್ಮದ ಮೂಲಕ ಲೋಕ ಶಿಕ್ಷಣವನ್ನು ಕೂಡ ಕೊಡಿ.

ದೇಶದ ಒಂದು ಕಡೆಯಿಂದ ಮತ್ತೊಂದು ಕಡೆಗೆ ಹಳ್ಳಿಯಿಂದ ಹಳ್ಳಿಗೆ ಹೋಗಿ ಅಲ್ಲಿಯ ಜನರಿಗೆ ಸುಮ್ಮನೆ ಕುಳಿತುಕೊಂಡರೆ ಪ್ರಯೋಜನವಿಲ್ಲವೆಂದು ಹೇಳ ಬೇಕಾಗಿದೆ. ಅವರಿಗೆ ಅವರ ಸ್ಥಿತಿಯನ್ನು ವಿವರಿಸಿ “ಎದ್ದೇಳಿ ಸಹೋದರರೆ, ಜಾಗ್ರತ ರಾಗಿ, ಇನ್ನೆಷ್ಟು ಕಾಲ ನೀವು ನಿದ್ರಿಸುವುದು!” ಎಂದು ಹೇಳಿ ತಮ್ಮ ಸ್ಥಿತಿಯನ್ನು ಹೇಗೆ ಉತ್ತಮಪಡಿಸಿಕೊಳ್ಳಬೇಕೆಂಬುದನ್ನು ವಿವರಿಸಿ. ಶಾಸ್ತ್ರದ ಗಹನವಾದ ವಿಷಯಗಳನ್ನು ಅವರಿಗೆ ಅರ್ಥವಾಗುವ ರೀತಿಯಲ್ಲಿ ಸುಲಭವಾಗಿ ಹೇಳಿ. ಇಲ್ಲಿಯ ವರೆಗೆ ಧರ್ಮ ಬ್ರಾಹ್ಮಣರಿಗೆ ಮಾತ್ರ ಮೀಸಲಾಗಿತ್ತು. ಆದರೆ ಇನ್ನು ಮೇಲೆ ಹಾಗಿರ ಲಾರದು. ಕಾಲ ಬದಲಾಯಿಸಿದೆ. ಈಗ ಎಲ್ಲರಿಗೂ ಧರ್ಮ ದೊರಕುವಂತೆ ನೀವು ಅದರ ಸ್ವಾಧೀನವನ್ನು ತೆಗೆದುಕೊಳ್ಳಿ. ಬ್ರಾಹ್ಮಣರಿಗೆ ಇರುವಷ್ಟೇ ಅಧಿಕಾರ ಅವ ರಿಗೂ ಧರ್ಮಕ್ಕೆ ಇದೆ ಎಂದು ಹೇಳಿ. ಪರೆಯವನಾದರೂ ಚಿಂತೆಯಿಲ್ಲ. ಎಲ್ಲರಿಗೂ ಮಂತ್ರೋಪದೇಶವನ್ನು ಕೊಡಿ. ಜೀವನಕ್ಕೆ ಆವಶ್ಯಕವಾದ ವ್ಯಾಪಾರ, ವಾಣಿಜ್ಯ, ವ್ಯವಸಾಯ ಮುಂತಾದವುಗಳ ಮೇಲೆಯೂ ಅವರಿಗೆ ಬೋಧನೆಯನ್ನು ಕೊಡಿ.

ಒಂದು ಕೇಂದ್ರ ಕಾಲೇಜಿನ ಮೂಲಕ ಜನಸಾಮಾನ್ಯರನ್ನು ಮೇಲಕ್ಕೆ ಎತ್ತುವು ದನ್ನು ಪ್ರಚಾರಮಾಡಿ. ಈ ಕಾಲೇಜಿನಲ್ಲಿ ತರಬೇತನ್ನು ಪಡೆದ ಪ್ರಚಾರಕರಿಂದ ವಿದ್ಯಾಭ್ಯಾಸ ಮತ್ತು ಧರ್ಮವಿಷಯಗಳನ್ನು ದರಿದ್ರರ ಮನೆಮನೆ ಬಾಗಿಲಿಗೆ ಒಯ್ಯಿರಿ. ಕೆಲವರು ಅನಾಸಕ್ತರಾದ ಸಂನ್ಯಾಸಿಗಳು ಪರಹಿತಾಕಾಂಕ್ಷಿಗಳಾಗಿ ಭೂಪಟ, ಮಾಯಾದೀಪ, ಗ್ಲೋಬ್ ಮುಂತಾದ ಸಲಕರಣೆಗಳ ಮೂಲಕ ಚಂಡಾಲನವರೆಗೆ ಎಲ್ಲ ಜನಸಾಧಾರಣರಿಗೆ ಬೋಧನೆ ಮಾಡಿದರೆ ಕಾಲಕ್ರಮೇಣ ಇದರಿಂದ ಒಳ್ಳೆಯದಾಗುವುದಿಲ್ಲವೆ? ಬೆಟ್ಟ ಮಹಮ್ಮದನ ಸಮೀಪಕ್ಕೆ ಬರದೇ ಇದ್ದರೆ ಮಹಮ್ಮದ್ ಬೆಟ್ಟದ ಸಮೀಪಕ್ಕೆ ಹೋಗಬೇಕಾಗಿದೆ. ಬಡವರು ಪಾಠಶಾಲೆ ಮುಂತಾದುವುಗಳಿಗೆ ಬರಲಾರರು.

ದೇಶ ಗುಡಿಸಲಿನಲ್ಲಿ ಜೀವಿಸುತ್ತಿದೆ ಎಂಬುದನ್ನು ಮರೆಯದಿರಿ. ಆದರೆ ಅವರಿಗೆ ಯಾರೂ ಏನನ್ನೂ ಮಾಡಲಿಲ್ಲ. ಆಧುನಿಕ ಸಮಾಜ ಸುಧಾರಕರು ಈಗ ವಿಧವಾ ವಿವಾಹದಲ್ಲಿ ನಿರತರಾಗಿರುವರು. ನನಗೆ ಎಲ್ಲಾ ಬಗೆಯ ಸುಧಾರಣೆಗಳ ಮೇಲೂ ಸಹಾನುಭೂತಿ ಇದೆ. ಆದರೆ ಒಂದು ದೇಶದ ಅದೃಷ್ಟ ಅಲ್ಲಿರುವ ವಿಧವೆಯರಿಗೆ ಸಿಕ್ಕುವ ಗಂಡಂದಿರ ಸಂಖ್ಯೆಯ ಮೇಲಿಲ್ಲ. ಅಲ್ಲಿಯ ಜನಸಾಮಾನ್ಯರ ಮಟ್ಟವನ್ನು ಮೇಲಕ್ಕೆ ಎತ್ತುವುದರ ಮೇಲಿದೆ. ನೀವು ಅವರನ್ನು ಮೇಲಕ್ಕೆ ಎತ್ತಬಲ್ಲಿರಾ? ಆಜನ್ಮದಿಂದ ಬಂದ ಆಧ್ಯಾತ್ಮಿಕತೆಯನ್ನು ಅವರು ಕಳೆದುಕೊಳ್ಳದಂತೆ ಅವರಿಗೆ ತಾವು ಕಳೆದುಕೊಂಡ ವ್ಯಕ್ತಿತ್ವವನ್ನು ನೀವು ಕೊಡಬಲ್ಲಿರಾ? ಎಲ್ಲರಲ್ಲಿಯೂ ಸಮಾ ನತೆ, ಎಲ್ಲರಿಗೂ ಸ್ವಾತಂತ್ರ್ಯ ಮತ್ತು ಅದಮ್ಯ ಕಾರ್ಯೋತ್ಸಾಹ ಇವುಗಳಲ್ಲಿ ನೀವು ಪಾಶ್ಚಾತ್ಯರಿಗೆ ಪಾಶ್ಚಾತ್ಯರಾಗಿ, ನಮ್ಮ ಧರ್ಮ ಮತ್ತು ಸ್ವಭಾವಗಳಲ್ಲಿ ಸಂಪೂರ್ಣ ಹಿಂದೂವಾಗಿ ಇರಬಲ್ಲಿರಾ? ಇದನ್ನು ನಾವು ಸಾಧಿಸಬೇಕಾಗಿದೆ. ಇದನ್ನು ಸಾಧಿಸಿಯೇ ಸಾಧಿಸುತ್ತೇವೆ.

