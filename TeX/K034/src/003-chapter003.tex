
\chapter{ಪುನರುದ್ಧಾರಕ್ಕೆ ಮುಖ್ಯವಾದ ಸಾಧನೆಗಳು}

\textbf{ಶ್ರದ್ಧಾವಂತ ಕೆಲಸಗಾರರನ್ನು ತರಬೇತು ಮಾಡುವುದು}: ರಾಜಕೀಯ ಅಥವಾ ಸಾಮಾಜಿಕ ವ್ಯವಸ್ಥೆಗಳ ಮೂಲವೆಲ್ಲ ಮಾನವನ ಒಳ್ಳೆಯ ಸ್ವಭಾವದ ಮೇಲೆ ನಿಂತಿದೆ. ಪಾರ್ಲಿಮೆಂಟ್ ಒಳ್ಳೆಯ ಶಾಸನಗಳು ಮಾಡುತ್ತದೆ ಎಂದು ಯಾವ ದೇಶವೂ ದೊಡ್ಡದಾಗುವುದಿಲ್ಲ, ಅಥವಾ ಒಳ್ಳೆಯದಾಗುವುದಿಲ್ಲ. ಅಲ್ಲಿಯ ಜನ ಒಳ್ಳೆಯವರಾಗಿರುವುದರಿಂದ ಆ ದೇಶ ಒಳ್ಳೆಯದಾಗಿದೆ. ಪ್ರಪಂಚದ ಐಶ್ವರ್ಯ ಗಳೆಲ್ಲಕ್ಕಿಂತ ಮನುಷ್ಯ ಶ್ರೇಷ್ಠ. ಭಾರತದೇಶ ತನ್ನನ್ನು ತನ್ನಿಂದಲೇ ಉದ್ಧಾರ ಮಾಡಿಕೊಳ್ಳಲು ಮಾತ್ರ ಇಂಗ್ಲೀಷರು ನಮಗೆ ಸಹಾಯ ಮಾಡಬಲ್ಲರು. ಭಾರತ ದೇಶದ ಪ್ರಾಣ ಹಿಂಡುತ್ತಿರುವವರ ಆಜ್ಞೆಗೆ ಒಳಗಾಗಿ ಆದ ಪ್ರಗತಿಯೆಲ್ಲ ನನ್ನ ದೃಷ್ಟಿಯಿಂದ ನಿಷ್ಪ್ರಯೋಜನ. ಗುಲಾಮರು ಒಂದು ಕೆಲಸವನ್ನು ಬಲಾತ್ಕಾರವಾಗಿ ಮಾಡಿದರೆ ಎಂತಹ ಒಳ್ಳೆಯ ಕೆಲಸವಾದರೂ ಅಧೋಗತಿಗೆ ಬರುವುದು. ಇದು ನಿಮ್ಮ ಇಚೆಊಇ್ಕ86ದ್ಯೇ ಅಥವಾ ನಿಮ್ಮನ್ನು ಆಳುವವರ ಇಚೆಊಇ್ಕ86ದ್ಯೇ ಎಂಬುದನ್ನು ಹೇಳಿ. ಇದು ನಿಮ್ಮದಾದರೆ, ನಿಮ್ಮನ್ನು ಆಳುವವರು ಅದನ್ನು ನಿಮಗೆ ಒದಗಿಸುವರೆ, ಅಥವಾ, ನೀವೇ ಅದನ್ನು ಮಾಡಿಕೊಳ್ಳುತ್ತೀರೊ? ಭಿಕ್ಷುಕರ ಬೇಡಿಕೆ ಎಂದಿಗೂ ನೆರೆವೇರುವುದಿಲ್ಲ. ನಿಮಗೆ ಬೇಕಾದುದನ್ನೆಲ್ಲ ಸರ್ಕಾರ ಕೊಟ್ಟರೆ ಅದನ್ನು ಸರಿಯಾಗಿ ನೋಡಿಕೊಳ್ಳುವವರು ಯಾರು? ಆದಕಾರಣವೇ ಅಂತಹ ವ್ಯಕ್ತಿಗಳಿಗೆ ಮೊದಲು ತರಬೇತು ಕೊಡಿ.

ನಮ್ಮ ದೇಶಕ್ಕೆ ತಮ್ಮ ಸರ್ವಸ್ವವನ್ನೂ ಧಾರೆ ಎರೆಯಬಲ್ಲಂತಹ ವ್ಯಕ್ತಿಗಳಿ ದ್ದರೆ, ಆಪಾದಮಸ್ತಕ ಪರಿಯಂತರ ನಿಃಸ್ಪೃಹರಾಗಿದ್ದರೆ, ಆಗ ಭರತಖಂಡ ಸರ್ವತೋಮುಖವಾಗಿ ಮಹಿಮಾನ್ವಿತವಾಗುವುದು. ನೂರಾರು ಜನ ಉದಾರ ಹೃದಯಿಗಳಾದ ಸ್ತ್ರೀಪುರುಷರು, ತಮ್ಮ ಕೋಟ್ಯಂತರ ದೇಶಬಾಂಧವರ ಅಜ್ಞಾನ ಮತ್ತು ದಾರಿದ್ರ್ಯದ ಸುಳಿಯಲ್ಲಿ ಮುಳುಗಿ ಹೋಗುತ್ತಿರುವವರ ಉದ್ಧಾರಕ್ಕಾಗಿ ತಮ್ಮ ಸುಖ ಭೋಗಗಳನ್ನು ತ್ಯಜಿಸಿ, ಬರುವ ಎಲ್ಲಾ ಕಷ್ಟಗಳನ್ನು ಸಹಿಸಲು ಅಣಿ ಯಾದರೆ ಮಾತ್ರ ನಮ್ಮ ದೇಶ ಉದ್ಧಾರವಾಗುವುದು.

ಭರತಖಂಡದ ಜನಸಾಮಾನ್ಯರನ್ನು ಮೇಲಕ್ಕೆ ಎತ್ತುವುದಕ್ಕೆ ಬದ್ಧಕಂಕಣ ರಾದವರೊಂದಿಗೆ ಕೆಲಸಮಾಡಿ. ಅವರನ್ನು ಜಾಗ್ರತಗೊಳಿಸಿ, ಅವರನ್ನು ಒಂದು ಗೂಡಿಸಿ. ಅವರಲ್ಲಿ ತ್ಯಾಗದ ಭಾವನೆ ಮೂಡುವಂತೆ ಮಾಡಿ. ಇಂತಹ ಯುವಕರ ಮೇಲೆಯೇ ಭರತಖಂಡದ ಭವಿಷ್ಯ ನಿಂತಿರುವುದು. ಇಂತಹ ಪವಿತ್ರಭಾವನೆ ಯಿಂದ ಕೂಡಿದ, ಈಶ್ವರನಲ್ಲಿ ಶ್ರದ್ಧಾನ್ವಿತರಾದ, ದೀನದಲಿತರಿಗಾಗಿ ಅನುಕಂಪೆ ಯಿಂದ ಕೂಡಿದ ಸಿಂಹಸದೃಶ ಪರಾಕ್ರಮವುಳ್ಳ ಒಂದು ಲಕ್ಷ ಮಂದಿ ಭರತಖಂಡ ದಲ್ಲೆಲ್ಲ ಮುಕ್ತಿಸೇವೆ ಸರ್ವಸಮಾನತ್ವ ಮುಂತಾದ ಭಾವನೆಗಳನ್ನು ಸಾರುತ್ತ ಹೋಗಲಿ.

ನನಗೆ ಇಂದು ಬೇಕಾಗಿರುವುದು ಸೇವಾಭಾವನೆಯಿಂದ ಸ್ಫೂರ್ತಿಗೊಂಡ ಒಂದು ತಂಡ. ನಮ್ಮ ಶಾಸ್ತ್ರದ ಸನಾತನ ತತ್ತ್ವಗಳನ್ನು ಭರತಖಂಡದಲ್ಲಿ ಮತ್ತು ಹೊರಗೆ ಬೋಧಿಸುವಂತಹ ಪ್ರಚಾರಕರನ್ನು ತರಬೇತು ಮಾಡುವಂತಹ ಕೇಂದ್ರವನ್ನು ಭರತಖಂಡದಲ್ಲಿ ತೆರೆಯಬೇಕೆಂಬುದೇ ನನ್ನ ಯೋಜನೆ.

ಮೊದಲು ಆಧ್ಯಾತ್ಮಿಕ ಭಾವನೆಯಿಂದ ಭರತಖಂಡವನ್ನು ತುಂಬಬೇಕು. ಯಾವ ರಾಜಕೀಯ ಸಂಸ್ಥೆಗಳಿಗಾಗಿ ಭರತಖಂಡದಲ್ಲಿ ಈಗ ಪ್ರಯತ್ನಿಸುತ್ತಿರು ವೆವೊ ಅವು ಯುರೋಪಿನಲ್ಲಿ ನೂರಾರು ವರ್ಷಗಳಿಂದ ಇವೆ. ಶತಮಾನಗಳಿಂದ ಅವುಗಳನ್ನು ಪ್ರಯೋಗಿಸಿ ನಿಷ್ಪ್ರಯೋಜನವೆಂದು ಅವರು ಅರಿತಿರುವರು. ರಾಜ ಕೀಯ ಸರ್ಕಾರಕ್ಕೆ ಸಂಬಂಧಪಟ್ಟ ಸಂಸ್ಥೆಗಳ ಪದ್ಧತಿಗಳನ್ನು ಒಂದಾದ ಮೇಲೊಂದಾಗಿ ಕೆಲಸಕ್ಕೆ ಬಾರದುದೆಂದು ಆಚೆಗೆ ಎಸೆದಿರುವರು. ಯುರೋಪಿನ ಚಿತ್ತ ಚಂಚಲವಾಗಿದೆ. ಎತ್ತ ತಿರುಗಬೇಕೊ ಗೊತ್ತಿಲ್ಲ. ಮಾನವಕೋಟಿಯನ್ನು ಕೇವಲ ಕತ್ತಿಯ ಬಲದಿಂದ ಆಳುವುದು ನಿಷ್ಪ್ರಯೋಜನ. ಅದರಿಂದ ಏನನ್ನೂ ಸಾಧಿಸಲಾಗುವುದಿಲ್ಲ. ಬಲಾತ್ಕಾರದಿಂದ ಜನರನ್ನು ಆಳಬೇಕು ಎಂಬ ಭಾವನೆ ಬಂದ ದೇಶಗಳೇ ಅಧೋಗತಿಗಿಳಿದು ಚೂರು ಚೂರಾಗುವುದಕ್ಕೆ ಉಪಕ್ರಮಿಸಿವೆ. ಬಾಹ್ಯಶಕ್ತಿಯ ಕೇಂದ್ರವಾದ ಯುರೋಪು ಇನ್ನು ಐವತ್ತು ವರುಷಗಳಲ್ಲಿ ತನ್ನ ನಿಲು ವನ್ನು ಬದಲಾಯಿಸದೆ ಇದ್ದರೆ, ಅಧ್ಯಾತ್ಮವನ್ನು ಕೇಂದ್ರವನ್ನಾಗಿ ಮಾಡಿಕೊಳ್ಳದೇ ಇದ್ದರೆ, ನಿರ್ನಾಮವಾಗಿ ಹೋಗುವುದು.

ಮುಂದೆ ಜನಗಣರಾಜ್ಯವೋ ಅಥವಾ ಮತ್ತಾವುದೋ ಸಾಧಾರಣ ಜನರು ರಾಜ್ಯ ಭಾರ ಮಾಡುವಂತಹ ಸ್ಥಿತಿ ಬರುವುದು ಎಂದು ಈಗ ಆಗುತ್ತಿರುವ ಬದಲಾವಣೆ ಯಿಂದ ಊಹಿಸಬಹುದು. ಜನರಿಗೆ ತಮ್ಮ ಜೀವನೋಪಾಯಕ್ಕೆ ತಕ್ಕ ಅನುಕೂಲಗ ಳಿರಬೇಕೆಂದು ಆಶಿಸುವುದು ಸ್ವಾಭಾವಿಕವಾಗಿಯೇ ಇದೆ. ಅವರಿಗೆ ಕಡಿಮೆ ಕೆಲಸ ಬೇಕು. ಅವರ ಮೇಲೆ ದಬ್ಬಾಳಿಕೆ ನಿಲ್ಲಬೇಕು. ಯುದ್ಧವಿರಕೂಡದು. ಎಲ್ಲರಿಗೂ ಹೆಚ್ಚು ಆಹಾರ ಬೇಕು ಎಂದು ಆಶಿಸುವುದು ಸಹಜವೇ ಆಗಿದೆ. ಧರ್ಮ ಮತ್ತು ಮನುಷ್ಯನ ಒಳ್ಳೆಯ ಸ್ವಭಾವದ ಮೇಲೆ ನಿಲ್ಲದೆ ಯಾವ ನಾಗರಿಕತೆಯೂ ಬಹಳ ಕಾಲ ಇರಬಲ್ಲದು ಎಂದು ಯಾರು ಊಹಿಸಬಲ್ಲರು? ಧರ್ಮ ಸಮಸ್ಯೆಯ ಮೂಲಕ್ಕೆ ಹೋಗುವುದು. ಅದು ಸರಿಯಾಗಿದ್ದರೆ ಎಲ್ಲವೂ ಸರಿಯಾಗಿರುವುದು.

ಶಾಸನ, ಸರ್ಕಾರ, ರಾಜಕೀಯ ಇವುಗಳೆಲ್ಲ ಒಂದು ಮಾರ್ಗವೇ ಹೊರತು ಇವೇ ಗುರಿಯಲ್ಲ. ಇವುಗಳಾಚೆ ಗುರಿ ಇರುವುದು. ಅಲ್ಲಿ ಯಾವ ಶಾಸನವೂ ಬೇಕಿಲ್ಲ. ತಳಹದಿ ಶಾಸನವಲ್ಲ, ನೀತಿ ಮತ್ತು ಹೃದಯ ಪರಿಶುದ್ಧಿ ಎಂಬುದನ್ನು ಕ್ರಿಸ್ತ ಮನಗಂಡ. ಯಾವ ಪಾರ್ಲಿಮೆಂಟಿನ ಶಾಸನವೂ ಮನುಷ್ಯರನ್ನು ನೀತಿವಂತ ರನ್ನಾಗಿ ಮಾಡಲಾರದು. ಇದು ಸಮಸ್ಯೆಯ ಮೂಲಕ್ಕೆ ಹೋಗಿ ಅವರ ಚಾರಿತ್ರ್ಯ ವನ್ನು ರೂಢಿಸುವ ಪ್ರಯತ್ನ ಮಾಡಬೇಕು.

ಹಿಂದೂ ಸಮಾಜವನ್ನು ಮೇಲೆತ್ತಬೇಕಾದರೆ ಧರ್ಮವನ್ನು ನಾಶಮಾಡಬೇಕಾ ಗಿಲ್ಲ ಎಂದು ನಾನು ಹೇಳುತ್ತೇನೆ. ಧರ್ಮದಿಂದ ಅಲ್ಲ ನಮ್ಮ ಸಮಾಜದಲ್ಲಿ ಈಗ ಕಾಣುತ್ತಿರುವ ನ್ಯೂನತೆಗಳು ಇರುವುದು. ಆದರೆ ಧರ್ಮವನ್ನು ಸರಿಯಾದ ರೀತಿ ಸಮಾಜದಲ್ಲಿ ಹರಡಿಲ್ಲದೆ ಇರುವುದೇ ಕಾರಣವಾಗಿದೆ. ನಾನು ಈಗ ಹೇಳಿರುವು ದನ್ನು ನಮ್ಮ ಶಾಸ್ತ್ರಗಳ ಆಧಾರದಿಂದಲೇ ಸಮರ್ಥಿಸುವುದಕ್ಕೆ ಸಿದ್ಧವಾಗಿರುವೆನು. ನಾನು ಬೋಧಿಸುವುದೇ ಇದನ್ನು. ಇದನ್ನು ಅನುಷ್ಠಾನಕ್ಕೆ ತರುವುದಕ್ಕಾಗಿ ಜೀವನ ದಲ್ಲಿ ಹೋರಾಡಬೇಕಾಗಿದೆ. ನಾವು ಗಮನಿಸಬೇಕಾದ ಮೊದಲನೆ ಕರ್ತವ್ಯವೇ ನಮ್ಮ ವೇದ ಉಪನಿಷತ್ತು ಪುರಾಣಗಳಲ್ಲಿರುವ ಅದ್ಭುತವಾದ ಸತ್ಯಗಳನ್ನು ಗ್ರಂಥ ಗಳ ಗುಹೆಯಿಂದ ಹೊರಗೆ ತೆಗೆದು ಜನರಿಗೆಲ್ಲಾ ಅವನ್ನು ಪ್ರಚಾರ ಮಾಡಬೇಕು. ಭರತಖಂಡದಲ್ಲಿ ಮೇಲೆ ಪ್ರತಿಯೊಂದು ಮೆಟ್ಟಲು ಹತ್ತಬೇಕಾದರೂ ನಮ್ಮಲ್ಲಿ ಇರುವ ಧಾರ್ಮಿಕ ಪ್ರವೃತ್ತಿ ಹೆಚ್ಚಬೇಕಾಗುವುದು. ಸೋಷಿಯಾಲಿಸಂ ಅಥವಾ ರಾಜಕೀಯ ಭಾವನೆಗಳಿಂದ ದೇಶವನ್ನು ತುಂಬುವುದಕ್ಕೆ ಮುಂಚೆ ಆಧ್ಯಾತ್ಮಿಕ ಭಾವನೆಗಳ ಪ್ರವಾಹದಿಂದ ತುಂಬಿ. ದಾನಕ್ಕೆ ಹೆಸರಾಂತ ಈ ದೇಶದಲ್ಲಿ ಮೊದಲನೆ ದಾನವಾದ ಆಧ್ಯಾತ್ಮಿಕದಾನವನ್ನು ಕೈಕೊಳ್ಳೋಣ. ಇದು ಕೇವಲ ಭರತಖಂಡ ದಲ್ಲೇ ಪರ್ಯವಸಾನವಾಗಕೂಡದು. ಆಧ್ಯಾತ್ಮಿಕ ಪ್ರಚಾರದಿಂದ ಇತರ ಲೌಕಿಕ ವಿದ್ಯೆಯು ಕೂಡ ಅದರೊಡನೆ ಬರುವುದು. ಆದರೆ ನೀವು ಅಧ್ಯಾತ್ಮವನ್ನು ತೊರೆದು ಕೇವಲ ಲೌಕಿಕ ವಿದ್ಯೆಗೆ ಮಾತ್ರ ಗಮನ ಕೊಟ್ಟರೆ ಅದರಿಂದ ಏನೂ ಪ್ರಯೋಜನ ವಾಗುವುದಿಲ್ಲ. ಜನರ ಮೇಲೆ ನೀವು ಸ್ವಲ್ಪವೂ ಪ್ರಭಾವವನ್ನು ಬೀರಲಾರಿರಿ ಎಂದು ನಾನು ಹೇಳುತ್ತೇನೆ.

ಭವಿಷ್ಯ ಭರತಖಂಡದ ನಿರ್ಮಾಣಕ್ಕೆ ನಾವು ಪ್ರಥಮದಲ್ಲಿ ಮಾಡಬೇಕಾದುದೇ ಧಾರ್ಮಿಕ ಭಾವನೆಯನ್ನು ಒಂದುಗೂಡಿಸುವುದು. ಹಿಂದೂಗಳಲ್ಲೆಲ್ಲ ಕೆಲವು ಸಾಮಾನ್ಯ ಭಾವನೆಗಳು ಇವೆ ಎಂಬುದನ್ನು ಎಲ್ಲರಿಗೂ ಪ್ರಚಾರಮಾಡಬೇಕಾಗಿದೆ. ನಮ್ಮ ಆತ್ಮೋದ್ಧಾರಕ್ಕಾಗಿ ನಮ್ಮ ನಮ್ಮಲ್ಲೇ ಕಾದಾಡುವುದನ್ನು ನಾವು ಮರೆಯ ಬೇಕಾದ ಕಾಲ ಬಂದಿದೆ. ಭರತಖಂಡದ ರಾಷ್ಟ್ರದ ಐಕ್ಯತೆ ಎಂದರೆ ಚದುರಿಹೋಗಿ ರುವ ಆಧ್ಯಾತ್ಮಿಕ ಶಕ್ತಿಯನ್ನು ಸಂಗ್ರಹಿಸುವುದಾಗಿದೆ. ಭರತಖಂಡದಲ್ಲಿ ಜನಾಂಗ ವೆಂದರೆ ಒಂದೇ ಆದರ್ಶಕ್ಕೆ ಹೃದಯ ಸ್ಪಂದಿಸುತ್ತಿರುವ ವ್ಯಕ್ತಿಗಳ ಐಕ್ಯತೆ.

ನಮ್ಮ ದೇಶಕ್ಕೆ ಇಂದು ಬೇಕಾಗಿರುವುದೇ ಕಬ್ಬಿಣದಂತಹ ಮಾಂಸಖಂಡಗಳು, ಉಕ್ಕಿನಂತಹ ನರಗಳು ಇರುವ ವ್ಯಕ್ತಿಗಳು. ದುರ್ದಮ್ಯವಾದ ಪ್ರಪಂಚದ ರಹಸ್ಯ ವನ್ನೆಲ್ಲ ಭೇದಿಸಿ ತಮ್ಮ ಇಚ್ಛೆಯನ್ನು ನೆರವೇರಿಸಿಕೊಳ್ಳಬಲ್ಲವರಾಗಿರಬೇಕು. ಸಮುದ್ರದ ಆಳಕ್ಕೆ ಬೇಕಾದರೂ ಹೋಗಿ ಮೃತ್ಯುವನ್ನಾದರೂ ಎದುರಿಸಲು ಅವರು ಸಿದ್ಧರಾಗಿರಬೇಕು. ನಮಗೆ ಇಂದು ಬೇಕಾಗಿರುವುದೇ ಅದು. ಅದ್ವೈತಭಾವನೆ ಯನ್ನು, ಎಲ್ಲಾ ಕಡೆಯಲ್ಲಿಯೂ ಒಂದೇ ಪರಬ್ರಹ್ಮನಿರುವನು ಎಂಬ ಭಾವನೆ ಯನ್ನು ತಿಳಿದುಕೊಂಡು ಅದನ್ನು ಸಾಕ್ಷಾತ್ಕಾರ ಮಾಡಿಕೊಳ್ಳುವುದರಿಂದ ಮಾತ್ರ ಅಂತಹ ವ್ಯಕ್ತಿಗಳನ್ನು ನಾವು ಸೃಷ್ಟಿಸಬಲ್ಲೆವು. ಅಖಂಡ ಪರಬ್ರಹ್ಮನೇ ಎಲ್ಲ ರಲ್ಲಿಯೂ ಇರುವನು ಎಂಬ ಭಾವನೆಯನ್ನು ಕಾರ್ಯತಃ ತರದೆ ನಮ್ಮ ದೇಶವನ್ನು ಉದ್ಧಾರಮಾಡಲು ಆಗುವುದಿಲ್ಲ. ಉಪನಿಷತ್ತಿನ ಸತ್ಯಗಳು ನಿಮ್ಮ ಎದುರಿಗೆ ಇವೆ. ಅವುಗಳನ್ನು ಸ್ವೀಕರಿಸಿ ಅವುಗಳಂತೆ ಬಾಳಿ. ಭರತಖಂಡದ ವಿಮೋಚನೆ ಶೀಘ್ರ ದಲ್ಲಿಯೇ ಆಗುವುದು.

ನಮಗಿಂದು ಶಕ್ತಿ, ಶಕ್ತಿ ಬೇಕು ಎಂದು ಪದೇ ಪದೇ ಹೇಳುವೆನು. ಉಪನಿಷತ್ತು ಗಳು ಶಕ್ತಿಯ ಮಹಾಗಣಿ. ಪ್ರಪಂಚಕ್ಕೆಲ್ಲ ದಾನ ಮಾಡುವಷ್ಟು ಶಕ್ತಿ ಅದರಲ್ಲಿದೆ. ಇಡೀ ಪ್ರಪಂಚವನ್ನೇ ಅದರಿಂದ ಜಾಗೃತಗೊಳಿಸಬಹುದು. ಇಡೀ ಪ್ರಪಂಚಕ್ಕೆ ಶಕ್ತಿಯನ್ನು ನೀಡಬಹುದು ಅದರಿಂದ. ಜಗದ ಎಲ್ಲಾ ದುರ್ಬಲರಿಗೆ ದುಃಖಿಗಳಿಗೆ ದಬ್ಬಾಳಿಕೆಗೆ ತುತ್ತಾದವರಿಗೆ, ಎಲ್ಲಾ ಮತೀಯರಿಗೆ ಧರ್ಮದ ಪಂಥದವರಿಗೆ ತಮ್ಮ ಕಾಲ ಮೇಲೆ ತಾವು ನಿಂತು ಮುಕ್ತರಾಗಿ ಎಂದು ಉಪನಿಷತ್ತು ತೂರ್ಯವಾಣಿಯಿಂದ ಬೋಧಿಸುವುದು. ದೈಹಿಕ, ಮಾನಸಿಕ ಆಧ್ಯಾತ್ಮಿಕ ಸ್ವಾತಂತ್ರ್ಯವೇ ಉಪನಿಷತ್ತಿನ ಬೋಧನೆಯ ಪಲ್ಲವಿ.

ಉಪನಿಷತ್ತಿನಲ್ಲಿ ಜನರ ಮೌಢ್ಯದ ಮೇಲೆ ಒಂದು ಸಿಡಿಮದ್ದಿನಂತೆ ಬೀಳುವ ಒಂದು ಪದವಿದ್ದರೆ ಅದೇ ‘ಅಭೀಃ’ (ನಿರ್ಭಯತೆ) ಎಂಬ ಪದ. ಜನರಿಗೆ ಬೋಧಿಸ ಬೇಕಾದ ಏಕಮಾತ್ರ ಧರ್ಮವೇ ನಿರ್ಭಯತೆ ಎಂಬುದು. ಜಾಗ್ರತರಾಗಿ, ಎದ್ದು ನಿಲ್ಲಿ. ದೌರ್ಬಲ್ಯದ ಸಮ್ಮೋಹನಾಸ್ತ್ರದ ಪಾಶದಿಂದ ಜಾಗ್ರತರಾಗಿ. ಯಾರೂ ನಿಜ ವಾಗಿಯೂ ದುರ್ಬಲರಲ್ಲ. ಆತ್ಮ ಅನಂತವಾದುದು, ಸರ್ವವ್ಯಾಪಿಯಾದುದು, ಸರ್ವಶಕ್ತವಾದುದು. ಎದ್ದು ನಿಂತು ನಿಮ್ಮಲ್ಲಿರುವ ಪರಬ್ರಹ್ಮನನ್ನು ವ್ಯಕ್ತಗೊಳಿಸಿ. ಅವನನ್ನು ಅಲ್ಲಗಳೆಯಬೇಡಿ. ನಮ್ಮ ಜನಾಂಗ ತುಂಬಾ ಸೋಮಾರಿತನ, ದೌರ್ಬಲ್ಯ ಮತ್ತು ಸಮ್ಮೋಹನಾಸ್ತ್ರಕ್ಕೆ ತುತ್ತಾಗಿದೆ. ಆಧುನಿಕ ಹಿಂದುಗಳೇ, ಸೋಮಾರಿತನದ ಪರವಶತೆಯಿಂದ ಜಾಗ್ರತರಾಗಿ. ಆತ್ಮನ ನೈಜ ಸ್ವಭಾವವನ್ನು ನೀವು ಅರಿತುಕೊಳ್ಳಿ, ಪ್ರತಿಯೊಬ್ಬರಿಗೂ ಅದನ್ನೇ ಬೋಧಿಸಿ. ಸುಪ್ತಾವಸ್ಥೆಯ ಲ್ಲಿರುವ ಆತ್ಮನನ್ನು ಕೂಗಿ ಕರೆಯಿರಿ. ನೋಡಿ ಅವನು ಹೇಗೆ ಏಳುತ್ತಾನೆ ಎಂಬು ದನ್ನು. ಆಗ ಶಕ್ತಿ ಬರುವುದು, ಪರಾಕ್ರಮ ಬರುವುದು, ಸಾಧು ಸ್ವಭಾವ ಬರುವುದು, ಪರಿಶುದ್ಧತೆ ಬರುವುದು. ಸುಪ್ತಾವಸ್ಥೆಯಲ್ಲಿರುವ ಆತ್ಮ ಯಾವಾಗ ಜಾಗ್ರತನಾಗು ವನೋ ಆಗ ಎಲ್ಲಾ ಒಳ್ಳೆಯ ಗುಣಗಳೂ ಬರುವುವು.

‘ಸೋಽಹಂ’ ಎಂದು ಯಾವಾಗಲೂ ಹೇಳಿ. ಈ ಮಾತುಗಳು ನಮ್ಮ ಮನಸ್ಸಿ ನಲ್ಲಿರುವ ಕೊಳೆಯನ್ನು ದಹಿಸುವುವು. ಇವು ಆಗಲೆ ನಿಮ್ಮ ಹೃದಯದಲ್ಲಿ ನಿದ್ರಿಸು ತ್ತಿರುವ ಅನಂತ ಶಕ್ತಿಯನ್ನು ಜಾಗ್ರತಗೊಳಿಸುವುವು.

ಸಮಾಜವನ್ನು ಜಾಗ್ರತಗೊಳಿಸುತ್ತಿರುವವರೆಲ್ಲ ವಿಶೇಷವಾಗಿ ಅವರ ನಾಯಕರು ತಮ್ಮ ಕಮ್ಯುನಿಸಂ ಅಥವಾ ಸರ್ವಸಮಾನತೆಯ ಭಾವನೆಗೆ ಒಂದು ಆಧ್ಯಾತ್ಮಿಕ ತಳಹದಿ ಇರಬೇಕೆಂದು ಮನಗಾಣುತ್ತಿರುವರು. ವೇದಾಂತವೇ ಅದಕ್ಕೆ ಆಧ್ಯಾತ್ಮಿಕ ತಳಹದಿಯನ್ನು ಒದಗಿಸಬಲ್ಲದು. ಸರ್ಕಾರದ ಅಥವಾ ಶಾಸನಗಣದ ಯಾವ ಬಲಾ ತ್ಕಾರವೂ ಜನರ ಸ್ವಭಾವವನ್ನು ಬದಲಾಯಿಸಲಾರದು. ಆಧ್ಯಾತ್ಮಿಕ ಮತ್ತು ನೈತಿಕ ಶಿಕ್ಷಣ ಮಾತ್ರ ನಮ್ಮಲ್ಲಿರುವ ದೋಷಗಳನ್ನು ನಿರ್ಮೂಲಮಾಡಬಲ್ಲದು. ನಮ್ಮಲ್ಲಿ ಶಕ್ತಿ ಮತ್ತು ಸ್ಫೂರ್ತಿ ಇದ್ದ ಹಿಂದಿನ ಕಾಲಕ್ಕೆ ಹೋಗಿ ಮತ್ತೊಮ್ಮೆ ನೀವು ಬಲಶಾಲಿಗಳಾಗಿ. ಪೂರ್ವಕಾಲದ ಶಕ್ತಿಯ ಚಿಲುಮೆಯಿಂದ ಪಾನಮಾಡಿ. ಭರತ ಖಂಡದಲ್ಲಿ ಬಾಳುವೆಯ ರೀತಿಯೇ ಇದು.

\textbf{ಸಮಾಜ ಸುಧಾರಣೆಗೆ ಮಾರ್ಗ}: ನಮ್ಮ ಸಮಾಜದಲ್ಲಿರುವ ಲೋಪದೋಷ ಗಳನ್ನು ತೋರುವ ಸಾವಿರಾರು ಉಪನ್ಯಾಸಗಳು ವೇದಿಕೆಯ ಮೇಲಿನಿಂದ ಆಗಿವೆ. ಹಿಂದೂ ಜನಾಂಗ ಮತ್ತು ಅದರ ನಾಗರಿಕತೆಯನ್ನು ಅವಹೇಳನ ಮಾಡುವ ಬೇಕಾ ದಷ್ಟು ಗ್ರಂಥಗಳನ್ನು ಬರೆದು ಆಗಿದೆ. ಆದರೂ ಇವುಗಳಿಂದ ಯಾವ ನಿಜವಾದ ಪ್ರಯೋಜನವೂ ಆಗಿಲ್ಲ. ಇದಕ್ಕೆ ಕಾರಣವೇನು? ಕಾರಣವನ್ನು ಕಂಡುಹಿಡಿಯುವು ದೇನು ಅಷ್ಟು ಕಷ್ಟವಲ್ಲ. ಸಮ್ಮನೆ ದೋಷಾರೋಪಣೆ ಮಾಡಿರುವರು. ಒಳ್ಳೆಯ ದನ್ನು ಮಾಡುವುದಕ್ಕೆ ದೋಷಾರೋಪಣೆಯಲ್ಲ ಮಾರ್ಗ.

ಅತ್ಯಂತ ಮೂಢಾಚಾರದಿಂದ ಕೂಡಿದ ಅಯುಕ್ತವಾದ ಸಂಸ್ಥೆಯನ್ನೂ ದೂರ ಬೇಡಿ. ಏಕೆಂದರೆ ಅವುಗಳೆಲ್ಲ ಹಿಂದೆ ಒಂದು ಒಳ್ಳೆಯ ಕೆಲಸವನ್ನು ಮಾಡಿವೆ. ಪ್ರಪಂಚದ ಬೇರೆ ದೇಶಗಳಲ್ಲೆಲ್ಲಿಯೂ ಭರತಖಂಡದಲ್ಲಿದ್ದಂತೆ ಸದುದ್ದೇಶ ಮತ್ತು ಆದರ್ಶಗಳುಳ್ಳ ಸಂಸ್ಥೆಗಳು ಇಲ್ಲ. ಈಗ ನಮಗೆ ದೋಷಪೂರಿತವಾಗಿ ಕಾಣುತ್ತಿರುವ ಕೆಲವು ಆಚಾರ ವ್ಯವಹಾರಗಳು ಕೂಡ ಹಿಂದೆ ಒಂದು ಒಳ್ಳೆಯ ಕೆಲಸವನ್ನು ಮಾಡಿದ್ದುವು. ನಾವು ಅವುಗಳನ್ನು ನಿವಾರಿಸಬೇಕಾದರೆ ಶಾಪವನ್ನು ಕೊಡಬೇಡಿ. ನಮ್ಮ ಜನಾಂಗವನ್ನು ಇದುವರೆಗೆ ಸಂರಕ್ಷಿಸಿ ಇಟ್ಟಿದ್ದಕ್ಕೆ ಅವುಗಳಿಗೆ ಕೃತಜ್ಞತೆಯನ್ನು ತೋರಿ ಧನ್ಯವಾದವನ್ನು ಅರ್ಪಿಸಿ ಅವುಗಳನ್ನು ತೆಗೆಯಿರಿ.

ನಾನು ಸುಧಾರಣೆಯಲ್ಲಿ ನಂಬುವುದಿಲ್ಲ. ಬೆಳವಣಿಗೆಯಲ್ಲಿ ನಂಬುತ್ತೇನೆ. ನಾನು ದೇವರ ಸ್ಥಾನದಲ್ಲಿ ನಿಂತು ಸಮಾಜಕ್ಕೆ ಹೀಗೆ ಮಾಡು ಎಂದು ವಿಧಿಸಲಾರೆ. ನನ್ನ ಆದರ್ಶ ಬೆಳವಣಿಗೆ, ವಿಕಾಸ, ನಮ್ಮ ಜನಾಂಗದ ರೀತಿಯಲ್ಲಿ ಮುಂದುವರಿ ಯುವುದು. ಪ್ರತಿಯೊಂದು ವ್ಯಕ್ತಿಯೂ ತನ್ನ ಉದ್ಧಾರವನ್ನು ತಾನೇ ಮಾಡಿಕೊಳ್ಳ ಬೇಕಾಗಿದೆ, ಬೇರೆ ಮಾರ್ಗವೇ ಇಲ್ಲ. ಇದರಂತೆಯೇ ರಾಷ್ಟ್ರಗಳೂ ಕೂಡ. ಉತ್ತಮ ಸಂಸ್ಥೆಗಳನ್ನು ನಿರ್ಮಾಣ ಮಾಡುವುದಕ್ಕೆ ಮುಂಚೆ ಹಳೆಯದನ್ನು ಧ್ವಂಸಮಾಡು ವುದು ಬಹಳ ಅಪಾಯಕಾರಿ. ಬೆಳವಣಿಗೆ ಯಾವಾಗಲೂ ನಿಧಾನವಾಗಿ ಆಗುವುದು. ಮನುಷ್ಯ ಎಲ್ಲಿರುವನೊ ಅಲ್ಲಿಂದ ಅವನನ್ನು ತೆಗೆದುಕೊಂಡು ಮೇಲಕ್ಕೆ ಬಿಡಿ.

ನಮ್ಮ ಆಧುನಿಕ ಕಾಲದ ಹಲವು ಸಮಾಜ ಸುಧಾರಣೆಗಳು ಬರಿಯ ಪಾಶ್ಚಾತ್ಯರ ಕ್ರಮಗಳನ್ನು ಅಂಧರಾಗಿ ಅನುಕರಿಸುವುದಾಗಿದೆ ಎಂದು ಹೇಳಲು ನನಗೆ ವ್ಯಥೆ ಯಾಗುವುದು. ಇದು ನಿಜವಾಗಿಯೂ ಭರತಖಂಡಕ್ಕೆ ಶ್ರೇಯಸ್ಕರವಲ್ಲ. ಭರತ ಖಂಡದ ಸುಧಾರಕರೆಲ್ಲ ನಮ್ಮ ಅವನತಿ ಮತ್ತು ನಮ್ಮಲ್ಲಿರುವ ಪೌರೋಹಿತ್ಯರ ಉಪಟಳಕ್ಕೆ ಧರ್ಮವೇ ಕಾರಣವೆಂದರು. ಅವಿನಾಶವಾದ ಕಟ್ಟಡಗಳನ್ನು ಕೆಳಗೆಳೆದು ಧ್ವಂಸಮಾಡಲು ಯತ್ನಿಸಿದರು. ಪರಿಣಾಮವಾಗಿ ಎಲ್ಲ ನಿಷ್ಪ್ರಯೋಜನವಾಯಿತು.

ನೀವು ಸಮಸ್ಯೆಯ ಮೂಲಕ್ಕೆ ಹೋಗಬೇಕಾಗಿದೆ. ಅದನ್ನೇ ನಾನು ಆಮೂಲಾಗ್ರ ವಾದ ಸುಧಾರಣೆ ಎನ್ನುವುದು. ಅಲ್ಲಿ ಬೆಂಕಿಯನ್ನು ಇಡಿ. ಅದು ಅಲ್ಲಿಂದ ಉರಿದು ಮೇಲೇಳಲಿ. ಅದರಿಂದ ಭರತಖಂಡದ ರಾಷ್ಟ್ರ ಮೂಡುವುದು. ನನ್ನ ಮಾರ್ಗವೇ ವ್ಯಾಧಿಯ ಮೂಲಕ್ಕೆ ಹೋಗಿ ಅದರ ಬೇರುಗಳನ್ನು ನಾಶಮಾಡುವುದು. ಸುಮ್ಮನೆ ಅವನ್ನು ಮೇಲಕ್ಕೆ ಏಳದಂತೆ ಅದುಮಿ ಹಿಡಿದಿರುವುದಲ್ಲ. ಎಲ್ಲಾ ವಿಧವಾದ ಸಾಮಾಜಿಕ ಸುಧಾರಣೆಗಳ ಹಿಂದೆಯೂ ಆಧ್ಯಾತ್ಮಿಕ ಶಕ್ತಿ ಕೆಲಸ ಮಾಡುತ್ತಿರುವುದು ವ್ಯಕ್ತವಾಗುವುದು. ಆಧ್ಯಾತ್ಮಿಕತೆ ಬಲವಾಗಿದ್ದರೆ ಸಮಾಜ ಸರಿಯಾಗಿ ವ್ಯವಸ್ಥಿತ ವಾಗುವುದು. ಸುಮ್ಮನೆ ಸಮಾಜ ಸುಧಾರಣೆಗೆ ಕೈ ಹಾಕಬೇಡಿ. ಆಧ್ಯಾತ್ಮಿಕ ಸುಧಾ ರಣೆ ಇಲ್ಲದೆ ಯಾವ ವಿಧವಾದ ಸಾಮಾಜಿಕ ಸುಧಾರಣೆಯೂ ಆಗಲಾರದು.

ನಮಗೆ ಸಮಾಜ ಸುಧಾರಣೆಯ ಆವಶ್ಯಕತೆಯಿದೆ. ಕೆಲವು ವೇಳೆ ಮಹಾ ಪುರುಷರು ಪ್ರಗತಿಗೆ ಸಹಾಯಕವಾದ ಕೆಲವು ಭಾವನೆಗಳನ್ನು ಕಂಡುಹಿಡಿಯು ತ್ತಿದ್ದರು. ರಾಜರು ಅದನ್ನು ಶಾಸನ ರೂಪಕ್ಕೆ ತರುತ್ತಿದ್ದರು. ಹಿಂದೆ ಸಮಾಜ ಸುಧಾರಣೆ ಆಗಿದ್ದು ಹೀಗೆ. ಆಧುನಿಕ ಕಾಲದಲ್ಲಿ ಈಗಿನ ಕಾಲಕ್ಕೆ ತಕ್ಕಂತೆ ಸುಧಾರಣೆ ಗಳನ್ನು ಜಾರಿಗೆ ತರಬೇಕಾದರೆ ಅಂತಹ ಅಧಿಕಾರವನ್ನು ಪಡೆದ ವ್ಯಕ್ತಿಗಳನ್ನು ಸೃಷ್ಟಿಸಬೇಕಾಗಿದೆ. ರಾಜರು ಹೋದರು. ಈಗ ಜನರ ಕೈಯಲ್ಲಿ ಅಧಿಕಾರವಿದೆ. ಸಾಧಾರಣ ಜನರು ವಿದ್ಯಾವಂತರಾಗಿ ತಮ್ಮ ಆವಶ್ಯಕತೆಗಳನ್ನು ಮನಗಂಡು, ತಮ್ಮ ಸಮಸ್ಯೆಗಳನ್ನು ಪರಿಹರಿಸುವ ಸ್ಥಿತಿಗೆ ಬರುವವರೆಗೆ ನಾವು ಕಾಯಬೇಕು. ಪ್ರಪಂಚದಲ್ಲಿ ಅಲ್ಪಸಂಖ್ಯಾತರ ದೌರಾತ್ಮ್ಯ ತುಂಬಾ ಭಯಾನಕವಾದುದು. ಕಾರ್ಯ ಗತ ಮಾಡಲಾಗದ ಸುಧಾರಣೆಯ ವಿಷಯಗಳಲ್ಲಿ ನಮ್ಮ ಶಕ್ತಿಯನ್ನು ವ್ಯಯ ಮಾಡುವುದಕ್ಕಿಂತ ಸಮಸ್ಯೆಯ ಮೂಲಕ್ಕೆ ಹೋಗೋಣ. ಎಂದರೆ ಒಂದು ಶಾಸನ ಗಣವನ್ನು ನಿರ್ಮಾಣ ಮಾಡೋಣ. ಹಾಗೆ ಮಾಡಬೇಕಾದರೆ ನಾವು ಜನರಲ್ಲಿ ವಿದ್ಯಾಭ್ಯಾಸವನ್ನು ಹರಡಬೇಕು. ಇದರಿಂದ ಅವರು ತಮ್ಮ ಸಮಸ್ಯೆಗಳನ್ನು ತಾವೇ ಬಗೆಹರಿಸಿಕೊಳ್ಳುವರು. ಇದನ್ನು ಮಾಡುವವರೆಗೆ ಸುಧಾರಣೆಯ ಭಾವನೆಗಳೆಲ್ಲ ಕೇವಲ ಭಾವನಾ ಮಾತ್ರವಾಗಿ ಉಳಿಯುವುವು. ಜನ ತಮ್ಮ ಸಮಸ್ಯೆಯನ್ನು ತಾವೇ ಬಗೆಹರಿಸಿಕೊಳ್ಳಬೇಕು. ಇದೇ ಹೊಸ ಕ್ರಮ. ಭರತಖಂಡದಲ್ಲಿ ಇದನ್ನು ಅನು ಷ್ಠಾನಕ್ಕೆ ತರುವುದಕ್ಕೆ ಕಾಲ ಹಿಡಿಯುವುದು. ಏಕೆಂದರೆ ಇಲ್ಲಿ ಹಿಂದೆ ಯಾವಾಗಲೂ ರಾಜರು ಆಳುತ್ತಿದ್ದರು.

ನಾವು ನಿಜವಾದ ಸುಧಾರಕರಾಗಬೇಕಾದರೆ ಮೂರು ಗುಣಗಳು ಆವಶ್ಯಕ. ಮೊದಲು ಹೃದಯದಲ್ಲಿ ಅನುಕಂಪ ಇರಬೇಕು. ನಿಮ್ಮ ದೇಶಬಾಂಧವರಿಗೆ ನೀವು ನಿಜವಾಗಿಯೂ ಪರಿತಪಿಸುತ್ತೀರೇನು? ಈ ಪ್ರಪಂಚದಲ್ಲಿ ಇಷ್ಟೊಂದು ದುಃಖ ವಿದೆ, ಅಜ್ಞಾನವಿದೆ, ಮೌಢ್ಯತೆ ಇದೆ ಎಂದು ನೀವು ವ್ಯಸ್ಥರಾಗಿರುವಿರಾ? ಮನುಷ್ಯರೆಲ್ಲ ನಿಮ್ಮ ಸಹೋದರರೆಂದು ನೀವು ಭಾವಿಸುತ್ತೀರೇನು? ಈ ಭಾವನೆ ನಿಮ್ಮ ವ್ಯಕ್ತಿತ್ವವನ್ನೆಲ್ಲ ವ್ಯಾಪಿಸಿಕೊಂಡಿರುವುದೆ? ಇದು ನಿಮ್ಮ ರಕ್ತದಲ್ಲಿ ಹರಿಯು ತ್ತಿದೆಯೆ? ನಿಮ್ಮ ನಾಡಿಯಲ್ಲಿ ಮಿಡಿಯುತ್ತಿದೆಯೆ? ಇವುಗಳೆಲ್ಲ ಇದ್ದರೆ ಇದು ಬರೀ ಮೊದಲನೆಯ ಹೆಜ್ಜೆ ಮಾತ್ರ. ಅನಂತರವೇ ಈ ಸಮಸ್ಯೆಯ ಪರಿಹಾರಕ್ಕೆ ಯಾವುದಾ ದರೂ ಮಾರ್ಗವನ್ನು ನೀವು ಕಂಡುಹಿಡಿದಿರುವಿರಾ? ಹಿಂದಿನವೆಲ್ಲ ಕೇವಲ ಮೂಢಾ ಚಾರಗಳು ಇರಬಹುದು. ಆದರೆ ಈ ಮೂಢಾಚಾರದ ಸುತ್ತಮುತ್ತಲು ಸತ್ಯದ ಘಟ್ಟಿ ಇತ್ತು. ಚಿನ್ನವನ್ನು ಉಳಿಸಿಕೊಂಡು ಕೊಳೆಯನ್ನು ಮಾತ್ರ ನಿರ್ಮೂಲ ಮಾಡುವ ಯಾವುದಾದರೂ ಮಾರ್ಗವನ್ನು ಕಂಡು ಹಿಡಿದಿರುವಿರಾ? ಮತ್ತೊಂದು ವಿಷಯ ಆವಶ್ಯಕ. ನಿಮ್ಮ ಉದ್ದೇಶವೇನು? ಹಣದ ಆಸೆ, ಅಧಿಕಾರದ ಲಾಲಸೆ, ಕೀರ್ತಿಯ ಆಸೆ, ಇವುಗಳಿಂದ ಪ್ರೇರೇಪಿತರಾಗಿಲ್ಲ ಎಂದು ನಿಸ್ಸಂದೇಹವಾಗಿ ತಿಳಿದುಕೊಂಡಿರುವಿರಾ?

