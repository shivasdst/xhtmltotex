
\chapter{ನಮ್ಮ ಮಾತೃಭೂಮಿ}

\textbf{ಅದರ ಹಿರಿಮೆ}: ಈ ಪ್ರಪಂಚದಲ್ಲಿ ಯಾವುದಾದರೂ ಪುಣ್ಯಭೂಮಿಯೆಂದು ಕರೆಸಿಕೊಳ್ಳಲು ಅರ್ಹತೆ ಪಡೆದಿದ್ದರೆ, ಜೀವಿಗಳು ತಮ್ಮ ಬಾಳಿನ ಅಂತ್ಯಕರ್ಮ ವನ್ನು ಸಮೆಸಲು ಬರಬೇಕಾದ ಸ್ಥಳವೊಂದಿದ್ದರೆ, ಭಗವಂತನೆಡೆಗೆ ಸಂಚರಿಸು ತ್ತಿರುವ ಪ್ರತಿಯೊಂದು ಜೀವಿಯೂ ತನ್ನ ಕೊನೆಯ ಯಾತ್ರೆಯನ್ನು ಪೂರೈಸುವು ದಕ್ಕೆ ಒಂದು ಕರ್ಮ ಭೂಮಿಗೆ ಬರಬೇಕಾದರೆ, ಯಾವುದಾದರೂ ದೇಶದಲ್ಲಿ ಮಾನವಕೋಟಿ, ಮಾಧುರ್ಯತೆ ಔದಾರ್ಯತೆ ಪವಿತ್ರತೆ ಶಾಂತಿ, ಎಲ್ಲಕ್ಕಿಂತ ಹೆಚ್ಚಾಗಿ ಧ್ಯಾನ ಮತ್ತು ಅಂತರ್ಮುಖ ಜೀವನದಲ್ಲಿ ತನ್ನ ಪರಾಕಾಷ್ಠೆಯನ್ನು ಮುಟ್ಟಿದ್ದರೆ ಅದೇ ಈ ಭರತಖಂಡ. ಅನಾದಿಕಾಲದಿಂದಲೂ ಇಲ್ಲಿಂದ ಧರ್ಮಸಂಸ್ಥಾಪನಾ ಚಾರ್ಯರು ಪೃಥ್ವಿಯನ್ನೆಲ್ಲ ತಮ್ಮ ಪವಿತ್ರವಾದ, ಎಂದೆಂದಿಗೂ ಬತ್ತದ ಆಧ್ಯಾತ್ಮಿಕ ಮಹಾಸತ್ಯದ ಪ್ರವಾಹದಿಂದ ತೋಯಿಸಿರುವರು. ಇಲ್ಲಿಂದ ಎದ್ದ ಆಧ್ಯಾತ್ಮಿಕ ಮಹಾಪ್ರವಾಹದ ಅಲೆ, ಪೂರ್ವಪಶ್ಚಿಮ ಉತ್ತರದಕ್ಷಿಣವೆನ್ನದೆ ಜಗತ್ತನ್ನೇ ಆವರಿ ಸಿರುವುದು. ಇಂದು ಜಡನಾಗರಿಕತೆಯ ಪ್ರಪಂಚಕ್ಕೆ ಅಧ್ಯಾತ್ಮವನ್ನು ಧಾರೆಯೆರೆ ಯುವ ಮಹಾಪ್ರವಾಹವೂ ಇಲ್ಲಿಂದ ಉದಿಸಬೇಕಾಗಿದೆ.

ಈ ಪ್ರಾಚೀನ ದೇಶದಲ್ಲಿ ತತ್ತ್ವಜ್ಞಾನ ಇತರ ದೇಶಗಳಿಗಿಂತ ಮುಂಚೆ ಉದಿ ಸಿತು. ಆಧ್ಯಾತ್ಮಿಕ ಪ್ರವಾಹದ ಸ್ಥೂಲಪ್ರತೀಕದಂತೆ ಸಮುದ್ರ ಸ್ಪರ್ಧಿಗಳಂತೆ ಇರುವ ನದಿಗಳು ಒಂದೆಡೆ ಹರಿಯುತ್ತಿವೆ. ಶ್ರೇಣಿಶ್ರೇಣಿಯಾಗಿ ಶಾಶ್ವತವಾದ ತುಷಾರ ಭೂಷಣ ಹಿಮಾಲಯ ಪರ್ವತಸ್ತೋಮ ವ್ಯಾಪಿಸಿಕೊಂಡು ಗಗನರಹಸ್ಯ ವನ್ನು ಭೇದಿಸುವಂತೆ ಇದೆ. ಪುಣ್ಯ ಪುಷಿಗಳ ಮಹಾಮುನಿಗಳ ಪಾದಧೂಳಿ ಸೋಕಿದ ಪವಿತ್ರಭೂಮಿ ಇದು. ಬಾಹ್ಯಪ್ರಕೃತಿ ಮತ್ತು ಮನುಷ್ಯನನ್ನು ಕುರಿತು ಜಿಜ್ಞಾಸೆ ಮೊದಲಾದುದು ಇಲ್ಲಿ. ಆತ್ಮದ ಅಮರತ್ವ, ಅಂತರ್ಯಾಮಿಯಾದ ಈಶ್ವರ, ಪ್ರಪಂಚ ಮತ್ತು ಜೀವಿಯಲ್ಲಿ ಓತಪ್ರೋತನಾಗಿರುವ ಪರಮಾತ್ಮ ಇಂತಹ ಭಾವನೆ ಗಳ ಪ್ರಥಮೋದಯ ಸ್ಥಾನವಿದು. ಇಲ್ಲಿ ಧರ್ಮ ಮತ್ತು ತತ್ತ್ವಗಳು ತಮ್ಮ ಪರಾ ಕಾಷ್ಠೆಯನ್ನು ಮುಟ್ಟಿವೆ.

ನಮ್ಮ ಪುಣ್ಯಭೂಮಿ ಧರ್ಮ ಮತ್ತು ತತ್ತ್ವಗಳ ತೌರುಮನೆ, ಆಧ್ಯಾತ್ಮಿಕ ವೀರರ ಜನ್ಮಸ್ಥಾನ, ತ್ಯಾಗಿಗಳ ನಾಡು ಇದು. ಇಲ್ಲಿ ಮಾತ್ರ ಅತಿ ಪುರಾತನ ಕಾಲದಿಂದ ಇತ್ತೀಚಿನವರೆಗೆ ಮಾನವನೆದುರಿಗೆ ಅತ್ಯುಚ್ಚ ಆದರ್ಶವನ್ನು ಬೆಳಗಿರುವರು. ಇದು ದರ್ಶನಗಳ ಆಧ್ಯಾತ್ಮಿಕತೆಯ ನೀತಿಯ ಪ್ರೀತಿ ಮಾಧುರ್ಯ ಮಾರ್ದವಗಳ ತೌರೂರು. ಈ ಘನ ಆದರ್ಶಗಳು ಇನ್ನೂ ಜೀವಂತವಾಗಿವೆ. ನನಗೆ ಆಗಿರುವ ಪ್ರಾಪಂಚಿಕ ಅನುಭವದಿಂದ, ಲವಲೇಶವೂ ಸಂದೇಹವಿಲ್ಲದೆ, ಪ್ರಪಂಚದಲ್ಲೆಲ್ಲ ಮೇಲಿನ ಆದರ್ಶಗಳಿಗೆ ಭರತಖಂಡ ಪ್ರಥಮ ಸ್ಥಾನದಲ್ಲಿದೆ ಎಂದು ಘಂಟಾ ಘೋಷವಾಗಿ ಸಾರುತ್ತೇನೆ.

ಹಲವು ಶತಮಾನಗಳ ಘರ್ಷಣೆ, ಭರತವರ್ಷದಮೇಲೆ ಮಾಡಿದ ನೂರಾರು ಆಕ್ರಮಣಗಳು, ಆಚಾರ ವ್ಯವಹಾರಗಳಲ್ಲಿ ಆದ ನೂರಾರು ಬದಲಾವಣೆಗಳನ್ನು ಎದುರಿಸಿ ನಿಂತ ಭರತಖಂಡವೇ ಇದು. ಕೊನೆಯಿಲ್ಲದ ಸ್ಫೂರ್ತಿ ಮತ್ತು ಅವಿನಾಶಿ ಯಾದ ಆತ್ಮನಿಂದ ಇಂದಿಗೂ ಯಾವ ಬಂಡೆಯೂ ಅಷ್ಟು ಸ್ಥಿರವಾಗಿಲ್ಲ, ಹಾಗೆ ನಿಂತಿರುವುದು ಈ ದೇಶ. ಇಲ್ಲಿಯ ಜೀವ ಆತ್ಮನ ಸ್ವಭಾವದಂತೆಯೇ ಆದಿ ಅಂತ್ಯ ವಿಲ್ಲದೆ ಅಮರವಾಗಿದೆ. ನಾವು ಇಂತಹ ದೇಶದ ಮಕ್ಕಳು. ಗ್ರೀಸ್ ದೇಶ ಹುಟ್ಟುವು ದಕ್ಕೆ ಮುಂಚೆಯೇ, ರೋಮ್ ದೇಶ ತಲೆದೋರುವುದಕ್ಕೆ ಮುಂಚೆಯೇ, ಆಧುನಿಕ ಯೂರೋಪಿನ ಪೂರ್ವಿಕರು ಕಾಡುಗಳಲ್ಲಿ ವಾಸಿಸುತ್ತ ಮೈಗೆ ಬಣ್ಣ ಬಳಿದುಕೊಳ್ಳು ತ್ತಿದ್ದ ಕಾಲಕ್ಕೆ ಮುಂಚೆಯೇ, ಭಾರತವು ಕಾರ್ಯೋನ್ಮುಖವಾಗಿತ್ತು. ಇದಕ್ಕೂ ಮುಂಚೆ ಇತಿಹಾಸದ ಬಗೆಗೂ ನಿಲುಕದ, ದಂತಕತೆಯ ಆಧಾರಕ್ಕೂ ಅತೀತವಾದ ಪೂರ್ವದಿಂದಲೂ ಇಂದಿನವರೆಗೆ ಒಂದು ಭಾವನಾಪರಂಪರೆ ಇಲ್ಲಿಂದ ಹರಿದು ಹೋಗಿದೆ. ಪ್ರತಿಯೊಂದು ಭಾವನೆಯ ಹಿಂದೆಯೂ ಆಶೀರ್ವಾದವನ್ನು ಮುಂದೆ ಶಾಂತಿಯನ್ನು ಪೋಣಿಸಿ ಇದನ್ನು ವ್ಯಕ್ತಗೊಳಿಸಿರುವರು. ನೀವು ಪ್ರಪಂಚದ ಇತಿಹಾಸವನ್ನೆಲ್ಲ ಅಧ್ಯಯನಮಾಡಿ, ನಿಮಗೆ ಗೋಚರಿಸುವ ಪ್ರತಿಯೊಂದು ಪವಿತ್ರ ಆದರ್ಶದ ಉಗಮ ಸ್ಥಾನವೂ ಭರತಖಂಡ ಎಂಬುದು ವ್ಯಕ್ತವಾಗುವುದು; ಅತಿ ಪ್ರಾಚೀನಕಾಲದಿಂದಲೂ ಭರತಖಂಡ ಮಾನವಕೋಟಿಯ ಪರಮಪವಿತ್ರ ಭಾವನೆಗೆ ನಿಧಿ ಎಂಬುದು ಗೊತ್ತಾಗುವುದು; ತಾನೇ ಈ ಭಾವನೆಗಳನ್ನು ಸೃಷ್ಟಿ ಮಾಡಿದ್ದರೂ ಅದನ್ನು ಪ್ರಪಂಚಕ್ಕೆಲ್ಲ ನಿರ್ವಂಚನೆಯಿಂದ ಕೊಟ್ಟಿದೆ ಎಂಬುದೂ ಧಾರ್ಮಿಕ ಕ್ಷೇತ್ರದಲ್ಲಿ ಮಾಡಿದ ಸಂಶೋಧನೆಗಳಲ್ಲಿ ಅನ್ಯ ರಾಷ್ಟ್ರಗಳು ಭರತ ಖಂಡದಿಂದ ಎರವಲಾಗಿ ತೆಗೆದುಕೊಳ್ಳದ ಭಾವನೆಗಳಿಲ್ಲ ಎಂಬುದೂ ತೋರು ವುದು. ಅದರಂತೆಯೇ ಯಾವುದಾದರೂ ಒಂದು ಧರ್ಮದಲ್ಲಿ ಆತ್ಮನ ಅಮರತ್ವದ ವಿಷಯವಾಗಿ ಒಳ್ಳೆಯ ಭಾವನೆಗಳಿದ್ದರೆ ಅವೆಲ್ಲ ಭರತಖಂಡಕ್ಕೆ ಪುಣಿ. ಹಿಂದೆ ಈ ದೇಶದಲ್ಲಿ ಮೂಡಿದ ಆಧ್ಯಾತ್ಮಿಕತೆ ಮತ್ತು ತತ್ತ್ವ ಭಾವನೆಗಳು ಉಬ್ಬರದ ಅಲೆ ಗಳಂತೆ ಪ್ರಪಂಚವನ್ನೆಲ್ಲ ವ್ಯಾಪಿಸಿರುವುದು. ಪುನಃ ಇದರಿಂದಲೇ ಆ ಮಹಾಭಾವನೆ ಗಳು ಮೇಲೆದ್ದು ಕ್ಷಯಿಸುತ್ತಿರುವ ಮಾನವ ಜನಾಂಗಕ್ಕೆ ನವಜೀವನವನ್ನು ಸ್ಫೂರ್ತಿಯನ್ನು ಕೊಡಬೇಕಾಗಿದೆ.

ಭರತಖಂಡಕ್ಕೆ ಪ್ರಪಂಚದ ಪುಣ ಮಹತ್ತರವಾದುದು. ಪ್ರತಿಯೊಂದು ದೇಶ ವನ್ನು ತೆಗೆದುಕೊಂಡರೆ ಪ್ರಪಂಚದ ಮತ್ತಾವ ಜನಾಂಗವೂ ಇಂತಹ ಸಹಿಷ್ಣು ಹಿಂದುವಿನಷ್ಟು ವಿಶ್ವಕ್ಕೆ ಶ್ರೇಯಸ್ಸನ್ನು ಮಾಡಿಲ್ಲ. ಯಾರ ಕಣ್ಣಿಗೂ ಕಾಣದೆ, ಯಾರ ಕಿವಿಗೂ ಬೀಳದೆ ಇದ್ದರೂ ಮೆಲ್ಲಗೆ ಬೀಳುವ ಹಿಮಮಣಿ ಶ್ರೇಷ್ಠವಾದ ಗುಲಾಬಿಯ ಹೂವುಗಳನ್ನು ಅರಳಿಸುವಂತಿದೆ ವಿಶ್ವದ ಭವ್ಯ ವಿಕಾಸಕ್ಕೆ ಭಾರತೀಯರ ಪ್ರಭಾವ. ಮೌನವಾಗಿ ಯಾರಿಗೂ ಗೋಚರಿಸದೆ ಇದ್ದರೂ ಅಪ್ರತಿಹತವಾಗಿದೆ ಇದರ ಪ್ರಭಾವ. ಇದು ವಿಶ್ವದ ಭಾವನಾ ತರಂಗವನ್ನೇ ಕ್ರಾಂತಿಗೊಳಿಸಿದೆ. ಆದರೂ ಇದು ಯಾವಾಗ ಆಯಿತು ಎಂದು ಯಾರಿಗೂ ಗೊತ್ತಿಲ್ಲ.

ಹಿಂದಿನಕಾಲದಲ್ಲಿ ಮತ್ತು ಈಗಲೂ ಬಲಾಢ್ಯ ಜನಾಂಗಗಳಿಂದ ಮಹಾಭಾವನೆ ಗಳು ಉತ್ಪನ್ನವಾಗಿವೆ. ಹಿಂದೆ ಮತ್ತು ಈಗಿನ ಕಾಲದಲ್ಲಿ ಒಂದು ಮಹಾ ಶಕ್ತಿ ದಾಯಕವಾದ ಮತ್ತು ಸತ್ಯವಾದ ಭಾವನೆಗಳ ಬೀಜವನ್ನು ಮುಂದೆ ಬರುತ್ತಿರುವ ಜನಾಂಗದ ಭೀಮ ಅಲೆಗಳು ಹರಡಿವೆ. ಆದರೆ ಇದನ್ನು ಗಮನಿಸಿ; ಹಾಗೆ ಭಾವನೆಗಳನ್ನು ಹರಡುವಾಗ ರಣ ಕಹಳೆಗಳ ಭೀಕರ ಧ್ವನಿಯಿತ್ತು, ಪ್ರತಿಯೊಂದು ಭಾವನೆಯನ್ನೂ ರಕ್ತದ ಕಡಲಿನಲ್ಲಿ ಅದ್ದಬೇಕಾಯಿತು. ಪ್ರತಿಯೊಂದು ಶಕ್ತಿಯ ಭಾವನೆಯ ಹಿಂದೆಯೂ ಕೋಟ್ಯನುಕೋಟಿ ಜೀವಿಗಳ ಸಂಕಟ, ಅನಾಥರ ಗೋಳು, ವಿಧವೆಯರ ಕಂಬನಿಯ ಕೋಡಿ ಛಾಯೆಯಂತೆ ಹಿಂಬಾಲಿಸಿತು. ಇತರ ರಾಷ್ಟ್ರ ಗಳು ಬೋಧಿಸಿದ್ದು ಹೀಗೆ. ಆದರೆ ಭರತಖಂಡ ಶಾಂತವಾಗಿ ಸಾವಿರಾರು ವರ್ಷಗಳಿಂದ ಬಾಳಿದೆ.

ಪಾಶ್ಚಾತ್ಯರು, ಬಲಶಾಲಿಯೇ ಕೊನೆಗೆ ಉಳಿಯುವನು ಎಂಬ ವಿಷಯವಾಗಿ ಮಾತನಾಡುವರು. ದೇಹದ ಬಲವೆ ಉಳಿಯುವುದಕ್ಕೆ ಯೋಗ್ಯವಾದುದು ಎಂದು ಅವರು ಭಾವಿಸುವರು. ಇದು ನಿಜವಾಗಿದ್ದರೆ ಹಿಂದೆ ಇನ್ನೊಬ್ಬರನ್ನು ಆಳುವುದ ರಲ್ಲಿ ಪ್ರಖ್ಯಾತವಾದ ಜನಾಂಗಗಳಲ್ಲಿ ಯಾವುದಾದರೂ ಒಂದಾದರೂ ಇಂದು ಪ್ರಖ್ಯಾತವಾಗಿರಬೇಕಾಗಿತ್ತು. ಯಾವ ಒಂದು ಬೇರೆದೇಶವನ್ನೂ ಆಕ್ರಮಣ ಮಾಡದ ದುರ್ಬಲನಾದ ಹಿಂದು ಎಂದೋ ನಿರ್ನಾಮವಾಗಿಹೋಗಬೇಕಾಗಿತ್ತು. ಆದರೂ ಮೂವತ್ತು ಕೋಟಿ ಜನ ಇಂದೂ ಬದುಕಿರುವೆವು. ಪ್ರಪಂಚದ ಜನಾಂಗ ಗಳಲ್ಲಿ ನಾವೊಬ್ಬರು ಮಾತ್ರ ಇನ್ನೊಬ್ಬರನ್ನು ಜಯಿಸಿಲ್ಲ. ಈ ಆಶೀರ್ವಾದ ನಮ್ಮನ್ನು ಕಾಯುತ್ತಿದೆ. ಅದಕ್ಕಾಗಿ ನಾವಿನ್ನೂ ಜೀವಂತವಾಗಿರುವೆವು.

ಪುರಾತನ ಗ್ರೀಸ್ ಪ್ರಪಂಚದಿಂದ ಕಣ್​ಮರೆಯಾಗಿಯೇ ಹೋಯಿತು. ಅದರ ವಿಷಯವಾಗಿ ಹೇಳುವ ಒಂದು ವ್ಯಕ್ತಿಯೂ ಉಳಿಯದೆ ಹೋಯಿತು. ಈ ಪ್ರಪಂಚ ದಲ್ಲಿ ಪಡೆಯುವುದಕ್ಕೆ ಯೋಗ್ಯವಾದ ವಸ್ತುವಿನ ಮೇಲೆಲ್ಲ ರೋಮನ್ ಧ್ವಜ ಹಾರಾಡುತ್ತಿದ್ದ ಒಂದು ಕಾಲವಿತ್ತು. ಪ್ರಪಂಚದಲ್ಲಿ ಎಲ್ಲರೂ ರೋಮಿನ ಶಕ್ತಿ ಪ್ರಭಾವಕ್ಕೆ ಬಿದ್ದರು. ಎಲ್ಲರ ಮೇಲೆಯೂ ಅದು ತನ್ನ ಅಧಿಕಾರವನ್ನು ಚಲಾಯಿಸುತ್ತಿತ್ತು. ಭೂಮಿ ತಲ್ಲಣಿಸುತ್ತಿತ್ತು ರೋಮಿನ ಹೆಸರನ್ನು ಕೇಳಿದರೆ. ಆದರೆ ಈಗ ರೋಮನ್ ಚಕ್ರಾಧಿಪತ್ಯದ ಕೇಂದ್ರ ಪಾಳುಬಿದ್ದಿದೆ. ಸೀಸರ್ ಚಕ್ರ ವರ್ತಿಗಳು ಸಿಂಹಾಸನದ ಮೇಲೆ ಕುಳಿತಕಡೆ ಜೇಡ ಈಗ ತನ್ನ ಬಲೆಯನ್ನು ನೇಯುತ್ತಿರುವುದು. ಇದರಂತೆಯೇ ಕೋರೈಸುವ ಹಲವು ಚಕ್ರಾಧಿಪತ್ಯಗಳು ಮೇಲೆದ್ದು, ಕೆಲವು ಕಾಲ ಮಿರುಗಿ ಮಾಯವಾಗಿವೆ. ಕೆಲವುಕಾಲ ದೌರ್ಜನ್ಯದಿಂದ ಬಾಳಿ ನೀರಿನಮೇಲೆ ಸಣ್ಣ ಅಲೆಯಂತೆ ನಾಮಾವಶೇಷವಾಗಿವೆ. ಈ ರಾಷ್ಟ್ರಗಳು ಮಾನವಕೋಟಿಯ ಮೇಲೆ ತಮ್ಮ ಪ್ರಭಾವವನ್ನು ಬಿಟ್ಟ ರೀತಿ ಇದು. ಆದರೆ ನಾವು ಇನ್ನೂ ಬದುಕಿರುವೆವು. ಮನು ಏನಾದರೂ ಬಂದು ಇಂದು ನಮ್ಮನ್ನು ನೋಡಿದರೆ ಅವನಿಗೇನೂ ಆಶ್ಚರ್ಯವಾಗುವುದಿಲ್ಲ, ತಾನೊಂದು ಪರದೇಶದಲ್ಲಿರುವೆನು ಎಂದು ಅವನು ಭಾವಿಸುವುದಿಲ್ಲ. ಸಾವಿರಾರು ವರ್ಷಗಳಿಂದ ಆಲೋಚಿಸಿ ನಮ್ಮ ಸಮಾಜಕ್ಕೆ ಹೊಂದಿಸಿಕೊಂಡ ಅದೇ ನಿಯಮಗಳು ಈಗಲೂ ಇವೆ. ಹಲವು ಶತ ಮಾನಗಳ ಅನುಭವದ ಮೂಸೆಗೆ ಸಿಕ್ಕಿ ಬಂದ ಅನುಭವಗಳು ಈಗಲೂ ಇವೆ. ಇವುಗಳಿಗೆ ಅಂತ್ಯವೇ ಇಲ್ಲದಂತೆ ಕಾಣುವುದು. ಕಾಲಕಳೆದಂತೆ ನಮ್ಮ ದುರ ದೃಷ್ಟದ ಗಾಳಿ ಅವುಗಳ ಮೇಲೆ ಬೀಸಿದಂತೆಲ್ಲ ಅದು ಒಂದೇ ಒಂದು ಕೆಲಸವನ್ನು ಮಾಡಿರುವಂತೆ ತೋರುವುದು. ಅದೇ ನಮ್ಮನ್ನು ಮತ್ತೂ ಬಲಾಢ್ಯರನ್ನಾಗಿ ಮಾಡಿ ರುವುದು.

ನೀವು ಯಾವುದಾದರೂ ದೇಶದಲ್ಲಿ ಪ್ರಖ್ಯಾತರಾದ ರಾಜರುಗಳು ತಮ್ಮ ಮೂಲ ಪುರುಷರು ರಾಜರಲ್ಲ, ಕೋಟೆ ಕೊತ್ತಲಗಳಲ್ಲಿ ವಾಸಿಸುತ್ತ ಜನರನ್ನು ಕೊಳ್ಳೆ ಹೊಡೆಯುತ್ತಿದ್ದ ಪಾಳೆಯಗಾರರಲ್ಲ, ಕಾಡಿನಲ್ಲಿ ವಾಸಿಸುತ್ತಿದ್ದ ಅರ್ಧ ಬೆತ್ತಲೆಯ ಪುಷಿಗಳು ಎಂಬುದನ್ನು ಕೇಳಿರುವಿರಾ? ಇದೇ ಆ ದೇಶ. ಇಲ್ಲಿ ಜನ್ಮ ತಾಳಿದ ಮಹಿಮಾವಂತರಲ್ಲಿ ನಾನೊಬ್ಬನೆಂದು ಭಾವಿಸುತ್ತೇನೆ. ಇದು ಕೇವಲ ನನ್ನ ವ್ಯಕ್ತಿತ್ವ ಕ್ಕಾಗಿ ಅಲ್ಲ, ನನ್ನ ದೇಶಕ್ಕಾಗಿ. ನಾನು ಹಿಂದಿನದನ್ನು ಹೆಚ್ಚು ಹೆಚ್ಚು ತಿಳಿದಂತೆಲ್ಲಾ ಅದನ್ನು ಹೆಚ್ಚು ಹೆಚ್ಚಾಗಿ ನೋಡಿದಂತೆಲ್ಲಾ ಹೆಚ್ಚು ಹೆಚ್ಚಾಗಿ ಈ ಅಭಿಮಾನ ನನ್ನಲ್ಲಿ ಬರುತ್ತಿದೆ. ಇದು ನನಗೆ ಶಕ್ತಿಯನ್ನು ಕೊಟ್ಟಿದೆ, ನನ್ನ ಭಾವನೆಗೆ ಧೃಡತೆಯನ್ನು ಕೊಟ್ಟಿದೆ. ನನ್ನನ್ನು ನೀಚ ಅವಸ್ಥೆಯಿಂದ ಮೇಲೆತ್ತಿದೆ. ನಮ್ಮ ಪೂರ್ವಿಕರು ನಮ್ಮ ಮುಂದೆ ಇಟ್ಟ ಮಹಾ ಆದರ್ಶಗಳನ್ನು ಕಾರ್ಯಗತಮಾಡುವುದಕ್ಕೆ ಶಕ್ತಿಯನ್ನು ಇತ್ತಿದೆ. ಪುರಾತನ ಆರ್ಯ ಮಹರ್ಷಿಗಳ ಸಂತಾನರೆ, ಪರಮೇಶ್ವರನ ದಯೆಯಿಂದ ನಿಮ್ಮಲ್ಲಿಯೂ ನನ್ನಲ್ಲಿರುವ ಅಭಿಮಾನವಿರಲಿ. ನಿಮ್ಮ ಪೂರ್ವಿಕರ ಮೇಲಿರುವ ಶ್ರದ್ಧಾಭಕ್ತಿಗಳು ನಿಮ್ಮ ರಕ್ತಗತವಾಗಲಿ. ಅದು ನಿಮ್ಮ ಜೀವನದಲ್ಲಿ ಓತಪ್ರೋತ ವಾಗಲಿ. ಇದು ಪ್ರಪಂಚದ ಮೋಕ್ಷಕ್ಕೆ ಸಹಾಯಕವಾಗಲಿ. ಇಲ್ಲಿದೆ ಮಾನವನಿಗೆ ಜೀವನವನ್ನು ಕೊಡುವ ಅಮೃತಪ್ರವಾಹ–ಇತರ ದೇಶಗಳಲ್ಲಿ ಕೋಟ್ಯಂತರ ಜೀವಿ ಗಳ ಎದೆಯನ್ನು ಸುಡುತ್ತಿರುವ ಜಡಸಿದ್ಧಾಂತದ ದಳ್ಳುರಿ ಶಮನಕ್ಕೆ ಅಮೃತವಾರಿ.

\textbf{ಭಾರತದ ಜೀವನ ಕೇಂದ್ರ}: ಪ್ರತಿಯೊಂದು ದೇಶಕ್ಕೂ ತನ್ನದೇ ಆದ ಪಾತ್ರವಿದೆ. ಅದರಂತೆಯೇ ಪ್ರತಿಯೊಂದು ದೇಶಕ್ಕೂ ತನ್ನದೇ ಒಂದು ರೀತಿ ಇದೆ, ತನ್ನದೇ ಒಂದು ವ್ಯಕ್ತಿತ್ವವಿದೆ. ವಿಶ್ವಸಮನ್ವಯ ವಾಣಿಯಲ್ಲಿ ಪ್ರತಿ ದೇಶಕ್ಕೂ ಯಾವುದೋ ಒಂದು ಸ್ವರವಿದೆ. ಇದೇ ಅದರ ಕೇಂದ್ರ, ಅದರ ಜೀವಾಳ. ಇದೇ ಆ ಜನಾಂಗದ ಬೆನ್ನೆಲುಬು, ತಳಪಾಯ. ಒಂದು ದೇಶದಲ್ಲಿ ಇಂಗ್ಲೆಂಡಿನಲ್ಲಿರುವಂತೆ ಅದು ರಾಜ ಕೀಯ ಶಕ್ತಿ ಇರಬಹುದು. ಮತ್ತೊಂದು ದೇಶಕ್ಕೆ ಕಲಾಜೀವನವೇ ಕೇಂದ್ರ ಇರ ಬಹುದು. ಇದರಂತೆಯೇ ಬೇರೆ ಬೇರೆ ರಾಷ್ಟ್ರಗಳಿಗೆ ಬೇರೆ ಬೇರೆ ಕೇಂದ್ರಗಳಿರು ತ್ತವೆ. ನಾನು ಅಮೇರಿಕನ್ನರಿಗೆ ಧರ್ಮವನ್ನು ಬೋಧಿಸಬೇಕಾದರೂ, ಅದರಿಂದ ಜನಾಂಗಕ್ಕೆ ಎಷ್ಟು ಪ್ರಯೋಜನವಾಗುವುದು ಎಂಬುದನ್ನು ಹೇಳಿದಲ್ಲದೇ ಸಾಧ್ಯ ವಿಲ್ಲವೆಂಬುದನ್ನು ನೋಡಿರುವೆನು. ವೇದಾಂತ ಬೋಧನೆಯಿಂದ ಎಂತಹ ಅದ್ಭುತ ರಾಜಕೀಯ ಬದಲಾವಣೆಗಳಾಗಬಹುದು ಎಂಬುದನ್ನು ತೋರಿದಲ್ಲದೆ ಇಂಗ್ಲೀಷಿನವರಿಗೆ ನಾನು ಧರ್ಮವನ್ನು ಬೋಧಿಸಲು ಸಾಧ್ಯವಾಗಲಿಲ್ಲ.

ಈ ನಮ್ಮ ಪುಣ್ಯಭೂಮಿಯಲ್ಲಿ ರಾಷ್ಟ್ರಜೀವನದ ತಳಪಾಯ ಬೆನ್ನೆಲುಬು, ಜೀವನದ ಕೇಂದ್ರ ಧರ್ಮವಲ್ಲದೆ ಬೇರಲ್ಲ. ಭರತಖಂಡದಲ್ಲಿ ಧಾರ್ಮಿಕ ಜೀವನವೇ ಅದರ ಕೇಂದ್ರ. ಜನಾಂಗದ ಜೀವನದ ಗಾನದ ಮುಖ್ಯ ಸ್ವರವೇ ಅದು. ಇತರರು ರಾಜಕೀಯ ವಿಷಯಗಳನ್ನು ಮಾತನಾಡಲಿ, ವ್ಯಾಪಾರದಿಂದ ನಾವು ಎಷ್ಟು ಧನ ಸಂಗ್ರಹಿಸಬಹುದು ಎಂಬುದನ್ನು ಕುರಿತು ಮಾತನಾಡಲಿ, ಆರ್ಥಿಕ ಉದ್ಯಮ ಗಳಿಂದ ನಾವು ಎಷ್ಟೊಂದು ಐಶ್ವರ್ಯವನ್ನು ಸಂಗ್ರಹಿಸಬಹುದು, ಇತರರ ಮೇಲೆ ಎಂತಹ ಪ್ರಭಾವವನ್ನು ಬೀರಬಹುದು ಎಂಬ ವಿಷಯವನ್ನು ಕುರಿತು ಮಾತ ನಾಡಲಿ. ಆದರೆ ಭಾರತೀಯನ ಮನಸ್ಸಿಗೆ ಇದು ಹಿಡಿಯುವುದಿಲ್ಲ. ಇದನ್ನು ತಿಳಿದು ಕೊಳ್ಳಲು ಅವನು ಇಚ್ಛಿಸುವುದಿಲ್ಲ. ಅವನಿಗೆ ಆಧ್ಯಾತ್ಮಿಕ ವಿಷಯವನ್ನು ಕುರಿತು ಹೇಳಿ. ಈಶ್ವರ ಜೀವ ಅನಂತ ಮುಕ್ತಿ ಮುಂತಾದ ವಿಷಯಗಳ ಮೇಲೆ ಹೇಳಿ. ಭರತಖಂಡದ ಅತ್ಯಂತ ಕನಿಷ್ಠ ರೈತಾಪಿ ಜನಕ್ಕೂ ಈ ವಿಷಯಗಳ ಮೇಲೆ ಇತರ ದೇಶದ ನಾಸ್ತಿಕರಿಗಿಂತ ಹೆಚ್ಚು ತಿಳಿದಿದೆ ಎಂದು ನಾನು ಭರವಸೆ ನೀಡಬಲ್ಲೆ.

ಆದಕಾರಣ ಸಾಮಾಜಿಕ ಸುಧಾರಣೆಯನ್ನು ಕೂಡ ಇದರಿಂದ ನಮ್ಮ ಆಧ್ಯಾತ್ಮಿಕ ಜೀವನಕ್ಕೆ ಎಂತಹ ಸಹಾಯವಾಗುವುದು ಎಂಬ ದೃಷ್ಟಿಯಿಂದ ಮಾತ್ರ ಬೋಧಿಸ ಬೇಕಾಗಿದೆ. ರಾಜಕೀಯವನ್ನು ಕೂಡ, ಇದು ನಮ್ಮ ಆಧ್ಯಾತ್ಮಿಕ ಜೀವನಕ್ಕೆ ಎಷ್ಟು ಸಹಾಯಮಾಡುವುದು ಎಂಬ ದೃಷ್ಟಿಯಿಂದಲೇ ಬೋಧಿಸಬೇಕಾಗಿದೆ. ಪ್ರತಿ ಯೊಬ್ಬನೂ ತನ್ನ ಆದರ್ಶವನ್ನು ಆರಿಸಿಕೊಳ್ಳಬೇಕಾಗಿದೆ, ಅದರಂತೆಯೇ ಪ್ರತಿ ಯೊಂದು ದೇಶವೂ ಕೂಡ.

ನಾವು ಬಹಳ ಹಿಂದೆಯೇ ನಮ್ಮ ಆದರ್ಶವನ್ನು ಆರಿಸಿಕೊಂಡಿರುವೆವು. ಅದೇನು ಅಂತಹ ಕೆಟ್ಟ ಆಯ್ಕೆಯಲ್ಲ. ಜಡದ್ರವ್ಯವನ್ನು ತ್ಯಜಿಸಿ ಅಧ್ಯಾತ್ಮವನ್ನು ಕುರಿತು ಚಿಂತಿಸುವುದು, ಮನುಷ್ಯನನ್ನು ಬಿಟ್ಟು ದೇವರನ್ನು ಕುರಿತು ಚಿಂತಿಸುವುದು, ಏನು ಕೆಟ್ಟ ಆಯ್ಕೆಯೆ? ಪರಲೋಕದ ವಿಷಯದಲ್ಲಿ ತೀವ್ರ ಶ್ರದ್ಧೆ, ಇಹಲೋಕದ ಮೇಲೆ ತೀವ್ರ ಅಸಡ್ಡೆ, ತೀವ್ರ ತ್ಯಾಗ ಮನೋಭಾವ, ದೇವರ ಮೇಲೆ ಅಪೂರ್ವ ಶ್ರದ್ಧೆ, ಭಕ್ತಿ, ಆತ್ಮದ ಅಮರತ್ವದಲ್ಲಿ ಅಚಲವಾದ ನಂಬಿಕೆ ನಮ್ಮಗಳಲ್ಲಿ ಇವೆ. ಇದನ್ನು ಸಾಧ್ಯವಾದರೆ ಬಿಡಿಸಿ ನೋಡೋಣ ಎಂದು ನಿಮಗೆಲ್ಲ ನಾನು ಸವಾಲು ಹಾಕುತ್ತೇನೆ. ನೀವೇನೋ ನನ್ನ ಎದುರಿಗೆ ತಾವು ಜಡವಾದಿಗಳೆಂದು ತೋರಿಸಿಕೊಳ್ಳಬಹುದು. ಕೆಲವು ಕಾಲ ಅದನ್ನು ಕುರಿತು ಮಾತನಾಡಬಹುದು. ಆದರೆ ನಿಮ್ಮ ಸ್ವಭಾವ ಏನು ಎಂಬುದು ನನಗೆ ಗೊತ್ತಿದೆ. ನಾನೇನಾದರೂ ನಿಮ್ಮ ಕೈಯನ್ನು ಹಿಡಿದು ಕರೆದರೆ ನೀವು ಒಬ್ಬ ದೊಡ್ಡ ದೇವರ ಭಕ್ತನಂತೆ ಬರುವಿರಿ. ನೀವು ನಿಮ್ಮ ಸ್ವಭಾವವನ್ನು ಹೇಗೆ ಬದಲಾಯಿಸುವುದು ಸಾಧ್ಯ? ನಿಮ್ಮ ರಾಜಕೀಯ ವಿಷಯ, ಸಮಾಜ ಸುಧಾ ರಣೆ, ಹಣ ಸಂಪಾದನೆ ಮಾಡುವುದು, ವ್ಯಾಪಾರೋದ್ಯಮಗಳು ಇವುಗಳ ವಿಷಯ ವಾಗಿ ಬೇಕಾದಷ್ಟು ಹರಟೆ ಹೊಡೆಯಬಹುದು. ಇವುಗಳೆಲ್ಲ ಬಾತಿನ ಮೇಲಿರುವ ನೀರಿನಂತೆ ಹರಿದುಕೊಂಡು ಹೋಗಿಬಿಡುವುದು.

ಒಳ್ಳೆಯದಕ್ಕೋ ಕೆಟ್ಟದಕ್ಕೋ ಅಂತೂ ನಮ್ಮ ಜೀವನ ಧರ್ಮದಲ್ಲಿ ಕೇಂದ್ರೀ ಕೃತವಾಗಿದೆ. ನೀವು ಅದನ್ನು ಬದಲಾಯಿಸಲಾರಿರಿ. ಅದನ್ನು ನಾಶ ಮಾಡಿ ಬೇರೊಂ ದನ್ನು ಆ ಸ್ಥಳದಲ್ಲಿ ಇಡಲಾರಿರಿ. ನೀವು ಆಗಲೇ ಬೆಳೆದುನಿಂತ ದೊಡ್ಡ ಮರವನ್ನು ತೆಗೆದು ಬೇರೊಂದು ಸ್ಥಳದಲ್ಲಿ ಹಾಕಿ ತಕ್ಷಣವೆ ಅಲ್ಲಿ ಬೇರು ಬಿಡುವಂತೆ ಮಾಡ ಲಾರಿರಿ. ಒಳ್ಳೆಯದಕ್ಕೋ ಕೆಟ್ಟದಕ್ಕೋ ಸಾವಿರಾರು ವರುಷಗಳಿಂದ ಧಾರ್ಮಿಕ ಭಾವನೆ ಭರತಖಂಡದಲ್ಲಿ ಹರಿಯುತ್ತಿದೆ. ಒಳ್ಳೆಯದಕ್ಕೋ ಕೆಟ್ಟದಕ್ಕೋ ನೂರಾರು ಶತಮಾನಗಳಿಂದ ಭರತಖಂಡದ ವಾತಾವರಣ ಧಾರ್ಮಿಕ ಭಾವನೆಗಳಿಂದ ಓತ ಪ್ರೋತವಾಗಿದೆ. ಒಳ್ಳೆಯದಕ್ಕೋ ಕೆಟ್ಟದಕ್ಕೋ ನಾವು ಈ ಭಾವನೆಗಳ ಮಧ್ಯದಲ್ಲಿ ಬೆಳೆದಿರುವೆವು. ಈ ಭಾವನೆಗಳು ನಮ್ಮ ರಕ್ತಗತವಾಗಿ ಧಮನಿ ಧಮನಿಯಲ್ಲಿ ಅನುರಣಿತವಾಗುತ್ತಿವೆ. ಅದು ನಮ್ಮ ದೇಹಗತವಾಗಿದೆ. ನಮ್ಮ ಜೀವನದ ಜೀವಾಳ ವಾಗಿದೆ. ಇದಕ್ಕೆ ಸರಿಯಾದ ಪ್ರತಿಕ್ರಿಯೆಯಾಗದಂತೆ ಇವುಗಳನ್ನು ನೀವು ತ್ಯಜಿಸ ಬಲ್ಲಿರಾ? ಸಾವಿರಾರು ವರುಷಗಳಿಂದ ಒಂದು ಕಡೆ ಹರಿದ ನದಿಯ ಮುಖವನ್ನೇ ಬದಲಾಯಿಸಬಲ್ಲಿರಾ! ನೀವು ಗಂಗಾನದಿ ಹಿಮಾಲಯದಲ್ಲಿ ತನ್ನ ಮೂಲಕ್ಕೆ ಹೋಗಿ ಬೇರೊಂದು ಮುಖದಲ್ಲಿ ಹರಿಯಬೇಕೆಂದು ಬಯಸುವಿರಾ? ಇದು ಸಾಧ್ಯ ವಾದರೂ ಭರತಖಂಡ ತನ್ನ ಧಾರ್ಮಿಕ ಭೂಮಿಯನ್ನು ಬಿಟ್ಟು ರಾಜಕೀಯವೊ ಅಥವಾ ಮತ್ತಾವುದಾದರೂ ಮಾರ್ಗದಲ್ಲಿ ಹರಿಯಲಾರದು. ಎಲ್ಲಿ ವಿರೋಧ ಕಡಿಮೆಯೊ ಅಲ್ಲಿಂದ ಕೆಲಸ ಮಾಡಲು ಸಾಧ್ಯ. ಭರತಖಂಡದಲ್ಲಿ ಧಾರ್ಮಿಕ ಕ್ಷೇತ್ರ ಅತ್ಯಂತ ಸುಲಭವಾದ ಮಾರ್ಗ. ಧಾರ್ಮಿಕ ಕ್ಷೇತ್ರದಲ್ಲಿ ಹರಿಯುವುದೇ ಜೀವನದ ಗತಿ, ಬೆಳವಣಿಗೆಯ ಮಾರ್ಗ, ಯೋಗ್ಯವಾದ ರೀತಿ.

ಯಾವುದಾದರೂ ದೇಶ ತನ್ನ ಜನಾಂಗದ ಮೂಲಕೇಂದ್ರವನ್ನು, ಸಾವಿರಾರು ವರುಷಗಳಿಂದಲೂ ಹರಿದುಬಂದ ಮಾರ್ಗವನ್ನು ತೊರೆದು ಬೇರೆ ಕಡೆ ಹರಿಯಲು ಸಾಧ್ಯವಾದರೆ ಆ ದೇಶ ನಿರ್ನಾಮವಾಗುವುದು. ಆದಕಾರಣವೇ ನೀವು ಧರ್ಮವನ್ನು ತೊರೆದು, ರಾಜಕೀಯವನ್ನು ಸಮಾಜಸುಧಾರಣೆಯನ್ನು ಅಥವಾ ಮತ್ತಾವುದಾ ದರೂ ಆದರ್ಶವನ್ನು ನಿಮ್ಮ ಜೀವನದ ಕೇಂದ್ರವನ್ನಾಗಿ ಮಾಡಿಕೊಂಡರೆ ನೀವು ಸರ್ವನಾಶವಾಗುವಿರಿ. ನೀವೂ ಆ ಜೀವದಾನ ಮಾಡುವ ಪ್ರವಾಹದೊಡನೆಯೆ ನಾಶ ವಾಗುವಿರಿ. ನೀವು ಸಾಯುವಿರಿ, ನಾಮಾವಶೇಷವಾಗುವಿರಿ. ಅದೊಂದೇ ಇದರಿಂ ದಾಗುವ ಪರಿಣಾಮ. ಧರ್ಮ, ಧರ್ಮವೊಂದೇ ಭರತಖಂಡದ ಪ್ರಾಣ. ಅದು ಹೋದರೆ ಈ ದೇಶ ನಾಶವಾಗುವುದು. ನೀವು ದೇಶವನ್ನು ಬೇಕಾದಷ್ಟು ರಾಜಕೀಯದಿಂದ ತುಂಬಬಹುದು. ಬೇಕಾದಷ್ಟು ಸಮಾಜಸುಧಾರಣೆಗಳನ್ನು ತೆಗೆದುಕೊಂಡು ಬರಬಹುದು. ಅಲ್ಲಿರುವ ಪ್ರತಿಯೊಬ್ಬ ಮಗುವಿಗೂ ಕುಬೇರನ ಐಶ್ವರ್ಯವನ್ನೇ ಕೊಡಬಹುದು. ಆದರೆ ಭಾರತೀಯನ ಮನಸ್ಸು ಮೊದಲು ಧಾರ್ಮಿಕದ ಮೇಲೆ, ಅನಂತರ ಇತರ ವಸ್ತುಗಳ ಮೇಲೆ ಹೋಗುವುದು. ಆದ ಕಾರಣ ನಾವು ಧರ್ಮವನ್ನು ಬಲಗೊಳಿಸಬೇಕು. ಎಲ್ಲರೂ ಪ್ರತಿಯೊಂದು ಕೆಲಸ ವನ್ನೂ ಧಾರ್ಮಿಕದೃಷ್ಟಿಯಿಂದ ಮಾಡುವುದನ್ನು ಕಲಿಯಬೇಕು. ಧರ್ಮವನ್ನು ಮೂಲವಾಗಿಟ್ಟುಕೊಂಡು ಉಳಿದವುಗಳೆಲ್ಲ ಅದನ್ನು ಅನುಸರಿಸಲಿ. ಭರತಖಂಡದ ಜನಾಂಗದ ಆದರ್ಶವೇ ತ್ಯಾಗ ಮತ್ತು ಸೇವೆ. ಅವುಗಳನ್ನು ಬಲಗೊಳಿಸಿ. ಉಳಿದವು ಗಳೆಲ್ಲ ಅದನ್ನು ಅನುಸರಿಸುವುವು. ಈ ದೇಶದಲ್ಲಿ ಧರ್ಮವೇ ಶ್ರೇಷ್ಠ ಆದರ್ಶ. ನಮ್ಮ ಬಿಡುಗಡೆ ಇರುವುದು ಅದರಲ್ಲಿಯೆ.

ಭರತಖಂಡವನ್ನು ನಾಶಮಾಡುವುದಕ್ಕೆ ಆಗುವುದಿಲ್ಲ. ಅದು ಅಮರವಾಗಿ ನಿಲ್ಲುವುದು. ಈ ಆದರ್ಶ ಆ ಜನಾಂಗದ ಹಿನ್ನೆಲೆಯಲ್ಲಿರುವತನಕ, ಅಲ್ಲಿಯ ಜನ ಆಧ್ಯಾತ್ಮಿಕತೆಯನ್ನು ತ್ಯಜಿಸದೇ ಇದ್ದರೆ ಅದು ಜೀವಂತವಾಗಿರುವುದು. ದರಿದ್ರ ಭಿಕ್ಷುಕರಾಗಿ ಅವರು ಇರಬಹುದು. ಬದುಕಿರುವತನಕ, ಸುತ್ತಲೂ ಕೊಳೆ ತುಂಬಿರ ಬಹುದು. ಆದರೆ ಅವರು ದೇವರನ್ನು ಮರೆಯದಿರಲಿ, ತಾವು ಮಹರ್ಷಿಗಳ ಸಂತಾನರು ಎಂಬುದನ್ನು ಮರೆಯದಿರಲಿ.

\textbf{ಭರತಖಂಡದ ಉದ್ದೇಶ}: ಸಾವಿರಾರು ವರುಷಗಳಿಂದಲೂ ಕಷ್ಟ ತಾಪತ್ರಯ ಗಳಿಗೆ ಸಿಕ್ಕಿದರೂ ಹಿಂದೂ ಜನಾಂಗ ಏತಕ್ಕೆ ನಾಶವಾಗದೆ ಉಳಿದಿದೆ? ನಮ್ಮ ಆಚಾರ ವ್ಯವಹಾರಗಳು ಅಷ್ಟು ಕೆಟ್ಟಿದ್ದರೆ, ಈ ಪ್ರಪಂಚದಲ್ಲಿ ನಾವು ಏತಕ್ಕೆ ಇನ್ನೂ ನಾಮಾವಶೇಷವಾಗಿಲ್ಲ. ಹೊರಗಿನಿಂದ ಬಂದು ನಮ್ಮನ್ನು ಗೆದ್ದ ಜನ ನಮ್ಮನ್ನು ಧ್ವಂಸಮಾಡಲು ಸರ್ವಪ್ರಯತ್ನ ಮಾಡಲಿಲ್ಲವೆ? ಇತರ ನಾಗರಿಕ ಜನಾಂಗಗಳು ನಾಶವಾದಂತೆ ಹಿಂದೂ ಜನಾಂಗ ಏತಕ್ಕೆ ಧ್ವಂಸವಾಗದೆ ಉಳಿಯಿತು? ಭರತಖಂಡ ಕ್ಷೀಣವಾಗಿ ಏತಕ್ಕೆ ಕಾಡುಪಾಲು ಆಗಲಿಲ್ಲ? ಆಗ ಅನ್ಯ ದೇಶೀಯರು ಅಮೆರಿಕ, ಆಸ್ಟ್ರೇಲಿಯ, ಆಫ್ರಿಕಾ ದೇಶಗಳಲ್ಲಿ ಮಾಡಿದಂತೆ ಇಂಡಿಯಾ ದೇಶಕ್ಕೆ ಬಂದು ಇಲ್ಲಿಯ ಫಲವತ್ತಾದ ಭೂಮಿಯನ್ನು ಉಳುವುದಕ್ಕೆ ಹಿಂದೆಗೆಯುತ್ತಿರಲಿಲ್ಲ. ಭರತಖಂಡಕ್ಕೂ ತನ್ನ ಶಕ್ತಿ ಇದೆಯೆಂಬುದನ್ನು ಅರಿಯಿರಿ. ಅದಕ್ಕೂ ತನ್ನ ಆದರ್ಶ ವೇನೆಂಬುದು ಗೊತ್ತು. ಭರತಖಂಡ ಇನ್ನೂ ಬದುಕಿದೆ. ಏತಕ್ಕೆಂದರೆ ವಿಶ್ವದ ನಾಗರಿಕತೆಯ ಸಮಷ್ಟಿಗೆ ಇಂಡಿಯಾದೇಶವೂ ತನ್ನ ಪಾಲಿನದನ್ನು ನೀಡಬೇಕಾಗಿದೆ. 

ಪ್ರತಿಯೊಬ್ಬ ಮನುಷ್ಯನಿಗೂ ಒಂದು ಆದರ್ಶವಿದೆ. ಹೊರಗಿನ ಮನುಷ್ಯ ತನ್ನ ಆದರ್ಶದ ಬಾಹ್ಯರೂಪವಷ್ಟೆ. ಇದರಂತೆಯೇ ಪ್ರತಿಯೊಂದು ಜನಾಂಗಕ್ಕೂ ಒಂದು ಆದರ್ಶವಿದೆ. ಈ ಆದರ್ಶ ವಿಶ್ವಕ್ಕೆ ಸಹಾಯಮಾಡುತ್ತಿದೆ. ಪ್ರಪಂಚ ಬದುಕಿರಬೇಕಾದರೆ ಅದು ಆವಶ್ಯಕ. ವಿಶ್ವದ ಬಾಳುವೆಗೆ ಆ ಭಾವನೆ ಅನಾವಶ್ಯಕ ಎಂದು ಅರಿತೊಡನೆಯೆ, ಆ ಭಾವನೆಯ ಕೇಂದ್ರ ವ್ಯಕ್ತಿಯಾಗಲಿ ರಾಷ್ಟ್ರವಾಗಲಿ ನಾಶವಾಗುವುದು. ಇಷ್ಟೊಂದು ಒಳಗೆ ಮತ್ತು ಹೊರಗಿನ ದಬ್ಬಾಳಿಕೆ, ಆಕ್ರಮಣ, ದಾರಿದ್ರ್ಯ, ದುಃಖ ಇವುಗಳಿಗೆ ತುತ್ತಾದರೂ ಭಾರತೀಯರು ಇನ್ನೂ ಬದುಕಿರುವುದ ರಿಂದ ನಮಗೊಂದು ಭಾವನೆ ಇದೆ, ಇದು ವಿಶ್ವದ ಸಮಷ್ಟಿಯ ಹಿತಕ್ಕೆ ಆವಶ್ಯಕ ಎಂಬುದನ್ನು ಅರಿಯಬೇಕು.

ಪ್ರತಿಯೊಂದು ದೇಶಕ್ಕೂ ಅದರದೇ ಒಂದು ಸ್ವಭಾವ ಇದೆ. ಅದರದೇ ಒಂದು ಉದ್ದೇಶವಿದೆ. ಪ್ರಪಂಚದಲ್ಲಿ ತನ್ನ ಪಾಲಿಗೆ ಬಂದ ಯಾವುದೋ ಕರ್ತವ್ಯವನ್ನು ಅದು ನಿರ್ವಹಿಸಬೇಕಾಗಿದೆ. ಪ್ರತಿಯೊಂದು ಜನಾಂಗವೂ ತನ್ನ ಉದ್ದೇಶವನ್ನು ಸಫಲಗೊಳಿಸಿಕೊಳ್ಳಲು ತನ್ನದೇ ರೀತಿಯಲ್ಲಿ ಕೆಲಸ ಮಾಡಬೇಕಾಗಿದೆ. ನಮ್ಮ ಜನಾಂಗದ ಉದ್ದೇಶ ಎಂದಿಗೂ ರಾಜಕೀಯ ಮಹತ್ವ ಆಗಿರಲಿಲ್ಲ, ಸೇನಾಬಲ ಆಗಿರಲಿಲ್ಲ. ಇದು ಹಿಂದೆ ಎಂದೂ ಆಗಿರಲಿಲ್ಲ. ಮತ್ತು ಮುಂದೆ ಎಂದಿಗೂ ಆಗಲಾರದು, ಇದನ್ನು ಗಮನಿಸಿ. ಆದರೆ ಜೀವನದಲ್ಲಿ ನಮ್ಮ ಪಾಲಿಗೆ ಬಂದ ಕರ್ತವ್ಯವೇ ಬೇರೆ. ಅದೇ ದೇಶದ ಆಧ್ಯಾತ್ಮಿಕ ಶಕ್ತಿ ತರಂಗಗಳನ್ನೆಲ್ಲ ಸಂಗ್ರಹಿಸಿ, ಅದನ್ನು ಒಂದು ದೊಡ್ಡ ಶಕ್ತಿ ಸಮುದ್ರವನ್ನಾಗಿಮಾಡಿ, ಅವಕಾಶ ಒದಗಿದಾಗ ಅದನ್ನು ಪ್ರಪಂಚಕ್ಕೆಲ್ಲಾ ಧಾರೆಯೆರೆಯುವುದು. ಪ್ರಪಂಚ ಈ ಅಮೃತ ಸಂಜೀವಿನಿ ಗಾಗಿ ಕಾಯುತ್ತಿದೆ. ನಮ್ಮ ಪೂರ್ವಿಕರು ಸಂಗ್ರಹಿಸಿಟ್ಟ ಈ ಅದ್ಭುತ ನಿಧಿಗೆ ಹೊರಗಿನ ಪ್ರಪಂಚ ಹೇಗೆ ಪರಿತಪಿಸುತ್ತಿದೆ ಎಂಬುದು ನಿಮಗೆ ಗೊತ್ತಿಲ್ಲ. ನಾವಿಲ್ಲಿ ಸುಮ್ಮನೆ ಹರಟೆಹೊಡೆಯುತ್ತೇವೆ. ಒಬ್ಬರೊಡನೊಬ್ಬರು ಕಲಹವಾಡುತ್ತೇವೆ. ಪವಿತ್ರವಾಗಿರುವುದನ್ನೆಲ್ಲ ನಾವು ಅಣಕಿಸಿ ಟೀಕಿಸುತ್ತೇವೆ. ನಮ್ಮ ಪೂರ್ವಿಕರು ಈ ಪವಿತ್ರ ದೇಶದಲ್ಲಿ ಸಂಗ್ರಹಿಸಿಟ್ಟಿರುವ ಅಮೃತ ಸರೋವರದಿಂದ ಒಂದು ಗುಟುಕು ಪಾನಮಾಡುವುದಕ್ಕೆ ನಮ್ಮ ದೇಶದ ಹೊರಗಿನ ಜನ ಹೇಗೆ ತಮ್ಮ ಕೈಗಳನ್ನು ನಮ್ಮ ಕಡೆ ಚಾಚುತ್ತಿರುವರು ಎಂಬುದು ನಮಗೆ ಸ್ವಲ್ಪವೂ ತಿಳಿಯದು.

\textbf{ಭರತಖಂಡದ ಭವಿಷ್ಯ}: ನಿಮಗೆ ಆಧ್ಯಾತ್ಮಿಕತೆಯಲ್ಲಿ ನಂಬಿಕೆ ಇರಲಿ ಇಲ್ಲದಿರಲಿ ನಮ್ಮ ರಾಷ್ಟ್ರದ ಹಿತದೃಷ್ಟಿಯಿಂದಲಾದರೂ ಆಧ್ಯಾತ್ಮಿಕ ಜೀವನವನ್ನು ಗ್ರಹಿಸಿ ಅದನ್ನು ರಕ್ಷಿಸಿಕೊಂಡು ಬರಬೇಕಾಗಿದೆ. ಇತರ ದೇಶಗಳಿಂದ ಏನೇನನ್ನು ಪಡೆಯ ಬಹುದೊ ಅವುಗಳನ್ನೆಲ್ಲಾ ಪಡೆಯಿರಿ. ಆದರೆ ಇವುಗಳೆಲ್ಲ ಆಧ್ಯಾತ್ಮಿಕ ಆದರ್ಶಕ್ಕೆ ತಲೆಬಾಗಬೇಕು. ಇಂತಹ ಒಂದು ಸಮ್ಮಿಲನದಿಂದ ಅದ್ಭುತವಾದ ಮಹಿಮಾನ್ವಿತ ವಾದ ಭವಿಷ್ಯಭಾರತ ಮೂಡುವುದು. ಅದು ಮೂಡುವುದರಲ್ಲಿ ನನಗೆ ಯಾವ ಸಂದೇಹವೂ ಇಲ್ಲ. ಹಿಂದಿಗಿಂತ ಮಹಿಮಾನ್ವಿತವಾದ ಭರತಖಂಡ ಬರುವುದ ರಲ್ಲಿ ನನಗೆ ಯಾವ ಸಂಶಯವೂ ಇಲ್ಲ. ಹಿಂದಿನ ಪುಷಿಗಳಿಗಿಂತ ಮಹಾ ಪುಷಿ ಗಳು ಮುಂದೆ ಉದಪ್‡ವಿಸುವರು. ನಿಮ್ಮ ಪೂರ್ವಜರಿಗೆ ತೃಪ್ತಿಮಾತ್ರ ಅಲ್ಲ ಆಗು ವುದು. ಅವರು ಅತ್ಯಂತ ಅಭಿಮಾನದಿಂದ ಪಿತೃಲೋಕದಿಂದ ಇಷ್ಟೊಂದು ಮಹಿಮಾನ್ವಿತರಾದ ಅವರ ಸಂತತಿಯವರನ್ನು ನೋಡುವರು. ನನ್ನ ಸಹೋದರರೆ, ನಾವೆಲ್ಲ ಕಷ್ಟಪಟ್ಟು ಕೆಲಸಮಾಡೋಣ, ಇದು ನಿದ್ರಿಸುವುದಕ್ಕೆ ಸಮಯವಲ್ಲ. ನಮ್ಮ ಕರ್ತವ್ಯ ನಿರ್ವಹಣೆಯಮೇಲೆ ಭಾರತದ ಭವಿಷ್ಯ ನಿಂತಿರುವುದು. ಅವಳು ನಮಗಾಗಿ ಕಾಯುತ್ತಿರುವಳು. ಅವಳೀಗ ಸುಮ್ಮನೆ ನಿದ್ರಿಸುತ್ತಿರುವಳು. ನಮ್ಮ ಮಾತೃಭೂಮಿ ಹಿಂದಿಗಿಂತ ಮಹಿಮಾನ್ವಿತಳಾಗಿ ನವೋತ್ಸಾಹದಿಂದ ಸನಾತನ ಸಿಂಹಾಸನದ ಮೇಲೆ ಮಂಡಿಸಿರುವುದನ್ನು ನೋಡಿ.

ಒಂದು ದೊಡ್ಡ ವೃಕ್ಷ ಸುಂದರವಾದ ಒಂದು ಫಲವನ್ನು ನೀಡುವುದು. ಆ ಹಣ್ಣು ಭೂಮಿಯ ಮೇಲೆ ಬಿದ್ದು ಕೊಳೆಯುವುದು. ಆ ನಾಶದಿಂದ ಮುಂದಿನ ಮರದ ಬೇರುಗಳು ಜನಿಸುವುವು. ಅದು ಬಹುಶಃ ಹಿಂದಿನ ಮರಕ್ಕಿಂತ ದೊಡ್ಡ ವೃಕ್ಷವಾಗಬಹುದು. ನಾವೆಲ್ಲ ಸಾಗಿಬಂದ ಈ ಅವನತಿ ಆವಶ್ಯಕವಾಗಿತ್ತು. ಈ ಅವನತಿಯಿಂದ ಭವಿಷ್ಯ ಭರತಖಂಡ ಮೂಡುವುದು. ಅದು ಈಗ ತಾನೆ ಮೊಳೆಯುತ್ತಿದೆ. ಅದರ ಮೂಲ ತಳಿರುಗಳು ಆಗಲೆ ವ್ಯಕ್ತವಾಗಿವೆ. ಒಂದು ಮೇರು ಸದೃಶ ಭೀಮವೃಕ್ಷ ಆಗಲೆ ಬೇರು ಬಿಡುತ್ತಿದೆ.

ನಿದ್ರೆಯಿಂದೆದ್ದೇಳಿ, ನವೋದಯವಾಗುತ್ತಿದೆ. ಕಣ್ದೆರೆದು ನೋಡಿ, ಸುದೀರ್ಘ ರಾತ್ರಿ ಕಡೆಗಿಂದು ಕೊನೆಗಾಣುತ್ತಿದೆ. ಬಹುಕಾಲದ ಶೋಕತಾಪಗಳು ಕಡೆಗಿಂದು ಮಾಯವಾಗುತ್ತಲಿವೆ. ಇದುವರೆಗೆ ಶವದಂತೆ ಬಿದ್ದಿದ್ದ ಶರೀರವಿಂದು ಸಚೇತನ ವಾಗುತ್ತಿದೆ. ಅದೋ ಕಿವಿಗೊಡಿ. ವಾಣಿಯೊಂದು ಕೇಳಿಬರುತ್ತಿದೆ. ಬಹು ಪುರಾತನ ಕಾಲದ ಇತಿಹಾಸ ಕಾಲದ ಗರ್ಭದಿಂದ ಹೊಮ್ಮಿ ಪರ್ವತಶಿಖರಗಳಿಂದ ಮರುದನಿ ಯಾಗಿ ಚಿಮ್ಮಿ, ಅರಣ್ಯಾರಣ್ಯ ಗಿರಿಕಂದರಗಳಲ್ಲಿ ಸಂಚರಿಸಿ, ಬರಬರುತ್ತ ಪ್ರಬಲ ವಾಗಿ, ಬಂದಂತೆಲ್ಲ ಅಪ್ರತಿಹತವಾಗಿ, ನಮ್ಮೀ ಪುಣ್ಯಭೂಮಿಯನ್ನು ನಿದ್ದೆಯಿಂದ ಬಡಿದೆಬ್ಬಿಸಿ, ಜ್ಞಾನ, ಭಕ್ತಿ, ವೈರಾಗ್ಯ ಸೇವಾತತ್ತ್ವಗಳನ್ನು ಉಚ್ಚಕಂಠದಿಂದ ಸಾರುವ ತೂರ್ಯವಾಣಿಯೊಂದು ಕೇಳಿಬರುತ್ತಿದೆ. ಹಿಮಾಲಯದಿಂದ ಬೀಸಿಬರುವ ಪುಣ್ಯ ಸಮೀರದಂತೆ ನಿರ್ಜೀವದಂತಿದ್ದ ಅಸ್ಥಿಮಾಂಸಗಳಿಗೆ ಜೀವದಾನ ಮಾಡುತ್ತಿದೆ. ಜಡನಿದ್ರೆಯನ್ನು ಪರಿಹರಿಸುತ್ತಿದೆ. ಕಾರ್ಯೋತ್ಸಾಹ, ಸ್ಥೈರ್ಯ, ಧೈರ್ಯಗಳನ್ನು ಉದ್ರೇಕಿಸುತ್ತಿದೆ. ಕುರುಡರಿಗೆ ಕಾಣದು ಮೂರ್ಖರಿಗೆ ತಿಳಿಯದು. ನಮ್ಮೀ ಭಾರತ ಭೂಮಿ ಯುಗಯುಗಗಳ ನಿದ್ರೆಯಿಂದ ಮೇಲೇಳುತ್ತಿದೆ. ಆಕೆಯನ್ನು ಇನ್ನು ಯಾರೂ ತಡೆಯಬಲ್ಲವರಿಲ್ಲ. ಇನ್ನು ಆಕೆ ನಿದ್ದೆ ಮಾಡುವುದಿಲ್ಲ. ಯಾವ ಶಕ್ತಿಯೂ ಆಕೆಯನ್ನು ಬಗ್ಗಿಸಲಾರದು. ಏಕೆಂದರೆ ಅದೋ ನೋಡಿ! ಮಹಾಕಾಳಿ ಮತ್ತೊಮ್ಮೆ ಎಚ್ಚೆತ್ತು ಮೈ ಕೊಡಹಿ ಉಸಿರೆಳೆದು ನಿಲ್ಲುತ್ತಿರುವಳು.

ಏಳಿ, ಏಳಿ, ಸುಧೀರ್ಘರಾತ್ರಿ ಕಡೆಗಿಂತು ಸಾಗುತ್ತಿದೆ. ಅರುಣೋದಯವಾಗು ತ್ತಿದೆ. ಉಬ್ಬರದ ಅಲೆ ಮೇಲೆದ್ದಿದೆ. ಅದನ್ನು ಇನ್ನು ಯಾರೂ ತಡೆಗಟ್ಟಲಾರರು. ನಂಬಿ, ನಂಬಿ ಈಶ್ವರೇಚ್ಛೆಯಿದು. ಭರತಖಂಡ ಮೇಲೇಳಬೇಕು. ಬಡವರನ್ನು ಮತ್ತು ಜನಸಾಮಾನ್ಯರನ್ನು ಸಂತೋಷಪಡಿಸಬೇಕಾಗಿದೆ. ಆನಂದಪಡಿ. ಆಧ್ಯಾ ತ್ಮಿಕ ಮಹಾಪ್ರವಾಹ ಮೇಲೆದ್ದಿದೆ. ಅದು ದೇಶದ ಮೇಲೆಲ್ಲ ಅಪ್ರತಿಹತವಾಗಿ ಮೇರೆ ಇಲ್ಲದೆ ಉರುಳುತ್ತಿದೆ. ಪ್ರತಿಯೊಬ್ಬರೂ ಮುಂದೆ ಬನ್ನಿ, ನಾವು ಮಾಡುವ ಪ್ರತಿಯೊಂದು ಒಳ್ಳೆಯ ಕೆಲಸವೂ ಅದರ ಶಕ್ತಿಯನ್ನು ಹೆಚ್ಚಿಸುವುದು. ಪ್ರತಿ ಯೊಬ್ಬನ ಶ್ರಮವೂ ಮಾರ್ಗವನ್ನು ಸುಗಮ ಮಾಡುವುದು. ಭಗವಂತನಿಗೆ ಜಯವಾಗಲಿ.

