
\chapter{ವರ್ಣದ ಸಮಸ್ಯೆ}

\textbf{ವರ್ಣವು ಸಮಾಜದಲ್ಲಿ ಇರುವುದು, ಧರ್ಮದಲ್ಲಿ ಅಲ್ಲ}: ನಮ್ಮ ವರ್ಣ ಮತ್ತು ಆಶ್ರಮಗಳನ್ನು ಸಾಧಾರಣವಾಗಿ ಧರ್ಮದೊಂದಿಗೆ ಸೇರಿಸಿಬಿಟ್ಟಿದ್ದರೂ ಇದು ಸರಿ ಯಲ್ಲ. ಈ ವರ್ಣಗಳು ನಮ್ಮ ಜನಾಂಗವನ್ನು ರಕ್ಷಿಸುವುದಕ್ಕೆ ಒಂದು ಕಾಲದಲ್ಲಿ ಆವಶ್ಯಕವಾಗಿತ್ತು. ಎಂದು ಈ ಆವಶ್ಯಕತೆ ಹೋಗುವುದೊ ಈ ಸಂಸ್ಥೆಗಳು ಸ್ವಾಭಾ ವಿಕವಾಗಿ ಮಾಯವಾಗುವುವು. ಧರ್ಮದಲ್ಲಿ ಯಾವ ವರ್ಣವೂ ಇಲ್ಲ. ಅತ್ಯಂತ ಮೇಲಿನ ವರ್ಣದಲ್ಲಿರುವವನು ಮತ್ತು ಅತ್ಯಂತ ಕೆಳಗಿನ ವರ್ಣದಲ್ಲಿರುವವನು ಕೂಡ ಭಾರತದೇಶದಲ್ಲಿ ಸಂನ್ಯಾಸಿಯಾಗಬಹುದು. ಆಗ ಇಬಪ್ರೂ ಒಂದು ಆಗು ವರು. ಈ ವರ್ಣ ಭಾವನೆ ವೇದಾಂತದ ಧರ್ಮಕ್ಕೆ ವಿರುದ್ಧವಾದುದು.

ವರ್ಣ ಕೇವಲ ಸಮಾಜಕ್ಕೆ ಸಂಬಂಧಪಟ್ಟದ್ದು. ನಮ್ಮ ಆಚಾರ್ಯರುಗಳೆಲ್ಲ ಅದನ್ನು ಧ್ವಂಸಮಾಡುವುದಕ್ಕೆ ಪ್ರಯತ್ನಪಟ್ಟಿರುವರು. ಬೌದ್ಧ ಕಾಲದಿಂದ ಇಂದಿನವರೆಗೆ ಪ್ರತಿಯೊಂದು ಪಂಥವೂ ವರ್ಣಕ್ಕೆ ವಿರೋಧವಾಗಿ ಬೋಧಿಸಿದೆ. ಆದರೂ ಇದರಿಂದ ಮತ್ತೂ ಗೋಜಾಗಿದೆ. ಬುದ್ಧನಿಂದ ಹಿಡಿದು ರಾಮಮೋಹನ ರಾಯನವರೆಗೆ ವರ್ಣವನ್ನು ಧರ್ಮ ಎಂದು ತಪ್ಪು ತಿಳಿದರು. ಅದರಿಂದ ಧರ್ಮ ಮತ್ತು ವರ್ಣವನ್ನು ನೆಲಸಮ ಮಾಡಲು ಪ್ರಯತ್ನಿಸಿ ಅದರಲ್ಲಿ ವಿಫಲರಾದರು.

ಪುರೋಹಿತರು ಎಷ್ಟೇ ಕೂಗಾಡಲಿ, ವರ್ಣ ಕೇವಲ ಸಮಾಜಕ್ಕೆ ಸಂಬಂಧ ಪಟ್ಟದ್ದು. ಈಗ ಅದು ಜಡವಾಗಿದೆ. ಹಿಂದೆ ಒಂದು ಕಾಲದಲ್ಲಿ ತನಗೆ ಇದ್ದ ಕೆಲಸವನ್ನು ಮಾಡಿ ಈಗ ಅದರ ಆವಶ್ಯಕತೆ ಇಲ್ಲದಿದ್ದರೂ ತನ್ನ ದುರ್ಗಂಧದಿಂದ ವಾತಾವರಣವನ್ನೆಲ್ಲ ತುಂಬಿದೆ. ವರ್ಣವನ್ನು ತೆಗೆಯಬೇಕಾದರೆ ಜನರಿಗೆ ಅವರು ಕಳೆದುಕೊಂಡ ವ್ಯಕ್ತಿತ್ವವನ್ನು ಕೊಡಬೇಕು. ಭರತಖಂಡದ ರಾಜಕೀಯ ಶಾಖೆಗಳ ವಿಸ್ತರಣೆ ಅಷ್ಟೆ ವರ್ಣಗಳು. ಇದೊಂದು ಆನುವಂಶಿಕವಾಗಿ ಬಂದ ಕುಲಕಸುಬು. ಯುರೋಪಿನೊಂದಿಗೆ ಮಾಡುತ್ತಿರುವ ವಾಣಿಜ್ಯ ಸ್ಪರ್ಧೆ ಎಲ್ಲಕ್ಕಿಂತ ಹೆಚ್ಚಾಗಿ ವರ್ಣಗಳನ್ನು ಶಿಥಿಲಗೊಳಿಸಿದೆ.

\textbf{ವರ್ಣದ ಮೂಲಭಾವನೆ}: ನನಗೆ ವಯಸ್ಸಾದಂತೆಲ್ಲ ಹಿಂದಿನಿಂದ ಬಂದ ವರ್ಣ ಮುಂತಾದ ಸಂಸ್ಥೆಗಳನ್ನು ಗೌರವದಿಂದ ನೋಡಲು ಉಪಕ್ರಮಿಸಿರುವೆನು. ಅವು ಗಳಲ್ಲಿ ಹಲವು ಕೆಲಸಕ್ಕೆ ಬಾರದವು ಎಂದು ನಾನು ಭಾವಿಸುತ್ತಿದ್ದ ಕಾಲ ಒಂದಿತ್ತು. ಆದರೆ ನನಗೆ ವಯಸ್ಸಾದಂತೆಲ್ಲ ಅವುಗಳಲ್ಲಿ ಯಾವುದನ್ನೂ ಕಟುವಾಗಿ ದೂರ ಲಾರೆ. ಏಕೆಂದರೆ ಅವುಗಳಲ್ಲಿ ಪ್ರತಿಯೊಂದೂ ಹಲವು ಶತಮಾನಗಳ ಪ್ರಯೋಗದ ಪ್ರತಿಫಲವಾಗಿದೆ.

ನೆನ್ನೆ ಹುಟ್ಟಿದ ಮಗು, ಇನ್ನೊಂದು ಎರಡು ದಿನಗಳಾದ ಮೇಲೆ ಅದು ಸಾಯ ಬಹುದು; ಅದು ಬಂದು ನನ್ನ ಕೆಲಸ ಮಾಡುವ ರೀತಿಯನ್ನೆಲ್ಲ ಬದಲಾಯಿಸು ಎನ್ನು ವುದು. ಆ ಮಗುವಿನ ಬುದ್ಧಿವಾದವನ್ನು ಕೇಳಿ, ಅದಕ್ಕೆ ತಕ್ಕಂತೆ ನಾನು ಎಲ್ಲವನ್ನೂ ಬದಲಾಯಿಸಿದರೆ ನಾನೇ ಮೂರ್ಖನಾಗುವೆನು, ಇತರರು ಅಲ್ಲ. ಹೀಗೆ ನನಗೆ ಬುದ್ಧಿವಾದವನ್ನು ಕೊಡಲು ಬರುವ ತಿಳಿಗೇಡಿಗಳಿಗೆ ನಾನು ಹೀಗೆ ಹೇಳುತ್ತೇನೆ: “ನೀವೆ ಒಂದು ಭದ್ರವಾದ ಸಮಾಜವನ್ನು ರಚಿಸಿದ ಮೇಲೆ ನಿಮ್ಮ ಮಾತನ್ನು ಕೇಳು ತ್ತೇನೆ. ನೀವು ಎರಡು ದಿನಗಳು ಒಂದು ಭಾವನೆಯನ್ನು ಇಟ್ಟುಕೊಂಡಿರಲಾರಿರಿ. ನಿಮ್ಮ ನಿಮ್ಮೊಳಗೆ ಜಗಳಕಾದು ತೆಪ್ಪಗಾಗುವಿರಿ. ನೀವು ವಸಂತಪುತುವಿನಲ್ಲಿ ಹುಟ್ಟಿದ ಚಿಟ್ಟೆಗಳಂತೆ ಇರುವಿರಿ. ನೀವು ಅವುಗಳಂತೆಯೆ ಐದು ನಿಮಿಷಗಳಲ್ಲಿ ಸಾಯುವಿರಿ. ನೀವು ನೀರಿನ ಗುಳ್ಳೆಗಳಂತೆ ಮೇಲೆ ಬಂದು ಗುಳ್ಳೆಗಳಂತೆಯೇ ಒಡೆದು ಹೋಗುವಿರಿ. ಮೊದಲು ನಮ್ಮಂತಹ ಭದ್ರವಾದ ಸಮಾಜವನ್ನು ರಚಿಸಿ. ಅನಂತರ ನಿಮ್ಮೊಡನೆ ಆ ವಿಷಯದ ಮೇಲೆ ಮಾತನಾಡಬಹುದು. ನನ್ನ ಸ್ನೇಹಿತನೆ, ಅಲ್ಲಿಯ ವರೆಗೆ ನೀನೊಬ್ಬ ಏನೂ ಅರಿಯದ ಬಾಲಕ.”

ವರ್ಣ ನಿಜವಾಗಿಯೂ ಒಳ್ಳೆಯದು. ವರ್ಣದ ಯೋಜನೆಯನ್ನು ನಾವು ಅನು ಸರಿಸಬೇಕಾಗಿದೆ. ನಿಜವಾಗಿ ವರ್ಣ ಎಂದರೆ ಏನು ಎಂಬುದು ಹತ್ತು ಲಕ್ಷದಲ್ಲಿ ಒಬ್ಬನಿಗೂ ತಿಳಿಯದು. ವರ್ಣವಿಲ್ಲದ ದೇಶವೇ ಪ್ರಪಂಚದಲ್ಲಿ ಇಲ್ಲ. ಭರತ ಖಂಡದಲ್ಲಿ ವರ್ಣದ ಮೂಲಕ ವರ್ಣಾತೀತರಾಗಿ ಹೋಗುತ್ತೇವೆ. ವರ್ಣ ನಿಂತಿರು ವುದೇ ಈ ಭಾವನೆಯ ಮೇಲೆ. ಭರತಖಂಡದಲ್ಲಿ ಪ್ರತಿಯೊಬ್ಬನನ್ನು ಬ್ರಾಹ್ಮಣನ ನ್ನಾಗಿ ಮಾಡಬೇಕು. ಅದೇ ಗುರಿ. ಬ್ರಾಹ್ಮಣನೇ ಮಾನವಕೋಟಿಯ ಆದರ್ಶ. ನೀವು ಭರತಖಂಡದ ಇತಿಹಾಸವನ್ನು ನೋಡಿದರೆ ನಿಮ್ನವರ್ಗದವರನ್ನು ಮೇಲಕ್ಕೆ ಎತ್ತಲು ಬಹಳ ಪ್ರಯತ್ನಗಳಾಗಿವೆ ಎಂಬುದು ಗೊತ್ತಾಗುವುದು. ಹಲವರನ್ನು ತಾವಿದ್ದ ಸ್ಥಳದಿಂದ ಮೇಲಕ್ಕೆ ಎತ್ತಿರುವರು. ಎಲ್ಲರೂ ಬ್ರಾಹ್ಮಣರಾಗುವವರೆಗೆ ಈ ಕೆಲಸ ಇನ್ನೂ ಹೆಚ್ಚಾಗಿ ಆಗುವುದು. ಅದೇ ಗುರಿ.

ತ್ಯಾಗ ಮತ್ತು ತಪಸ್ಸಿನ ಮೇಲೆ ನಿಂತ ಬ್ರಾಹ್ಮಣನೇ ನಮ್ಮ ಆದರ್ಶ. ಬ್ರಾಹ್ಮಣನ ಆದರ್ಶ ಯಾವುದು? ಯಾವನಲ್ಲಿ ಪ್ರಾಪಂಚಿಕತೆ ಲವಲೇಶವೂ ಇಲ್ಲವೊ, ಯಾರಲ್ಲಿ ಜ್ಞಾನ ಮತ್ತು ಸಾತ್ವಿಕ ಜೀವನ ತುಂಬಿ ತುಳುಕಾಡುತ್ತಿದೆಯೋ ಅವನೇ ಬ್ರಾಹ್ಮಣ. ಇದೇ ಹಿಂದೂ ಜನಾಂಗದ ಆದರ್ಶ. ಬ್ರಾಹ್ಮಣನ ವಿಷಯ ವಾಗಿ ಏನು ಹೇಳುತ್ತಾರೆ ನಿಮಗೆ ಗೊತ್ತಿಲ್ಲವೆ? ಬ್ರಾಹ್ಮಣ ಎಲ್ಲಾ ಶಾಸನಗಳಿಗೂ ಅತೀತ. ಅವನಿಗೆ ಯಾವ ಶಾಸನವೂ ಇಲ್ಲ. ರಾಜರು ಅವನನ್ನು ಆಳಲಾರರು. ಅವನನ್ನು ಯಾರೂ ಹಿಂಸಿಸಲಾರರು. ಇದೆಲ್ಲ ಸತ್ಯ. ಸ್ವಾರ್ಥರಾದ ಮೂರ್ಖರು ಕೊಡುವ ವಿವರಣೆಗಳ ಮೂಲಕ ಇದನ್ನು ನೋಡಬಾರದು, ಆದರೆ ಅವುಗಳನ್ನು ಹಿಂದೆ ವೇದಾಂತ ಯಾವ ದೃಷ್ಟಿಯಿಂದ ಹೇಳಿತೋ ಆ ದೃಷ್ಟಿಯಲ್ಲಿ ನೋಡಿ. ಯಾವ ಬ್ರಾಹ್ಮಣ ತನ್ನ ಸ್ವಾರ್ಥವನ್ನೆಲ್ಲ ಅಳಿಸಿರುವನೊ, ಯಾವ ಜ್ಞಾನ ಮತ್ತು ಪ್ರೀತಿಯನ್ನು ಸಂಪಾದಿಸಲು ಮತ್ತು ಅದನ್ನು ಇತರರಿಗೆ ಪ್ರಚಾರಮಾಡಲು ಜೀವಿ ಸಿರುವನೊ, ಇಂತಹವರಿಂದ ದೇಶ ತುಂಬಿದ್ದರೆ, ನೀತಿವಂತರು ಸಾಧುಗಳು ಮತ್ತು ತಪೋಮಹಿಮರು ಆದ ಸ್ತ್ರೀ ಪುರುಷರಿಂದ ಕೂಡಿದ್ದರೆ, ಅಂತಹವರು ಶಾಸನಕ್ಕೆ ಅತೀತರಾಗಿರುವರು ಎಂದು ಭಾವಿಸುವುದರಲ್ಲಿ ಆಶ್ಚರ್ಯವೇನೂ ಇಲ್ಲ. ಅಂತಹವ ರನ್ನು ಆಳಲು ಪೋಲೀಸರು ಏಕೆ? ಮಿಲಿಟರಿ ಏಕೆ? ಯಾರಾದರೂ ಅವರನ್ನು ಏತಕ್ಕೆ ಆಳಬೇಕು? ಅವರು ಒಂದು ಸರ್ಕಾರಕ್ಕೆ ಅಧೀನರಾಗಿ ಏತಕ್ಕೆ ಬಾಳಬೇಕು? ಅವರು ಪವಿತ್ರಾತ್ಮರು, ಸಾಧು ಸ್ವಭಾವದವರು. ಅವರು ನಿಜವಾಗಿ ದೈವೀಸಂಪತ್ತಿನಿಂದ ಕೂಡಿದವರು. ಇವರೇ ನಮ್ಮ ಆದರ್ಶ ಬ್ರಾಹ್ಮಣರು. ಸತ್ಯಯುಗದಲ್ಲಿ ಒಂದೇ ವರ್ಣ ಇರುವುದು ಎಂಬುದನ್ನು ನಾವು ಓದುತ್ತೇವೆ. ಅದೇ ಬ್ರಾಹ್ಮಣವರ್ಣ. ಆದಿ ಯಲ್ಲಿ ಪ್ರಪಂಚದ ಎಲ್ಲರೂ ಬ್ರಾಹ್ಮಣರಾಗಿದ್ದರು ಎಂಬುದನ್ನು ನಾವು ಮಹಾ ಭಾರತದಲ್ಲಿ ಓದುತ್ತೇವೆ. ಅವರು ಅವನತಿಗೆ ಬಂದಂತೆ ಹಲವು ವರ್ಣಗಳಾಗಿ ಕವಲೊಡೆದರು. ಪುನಃ ಹೊಸಕಲ್ಪ ಪ್ರಾರಂಭವಾದಾಗ ಅವರೆಲ್ಲ ಬ್ರಾಹ್ಮಣರಾಗು ವರು ಎಂದು ಹೇಳುವುದು.

ಬ್ರಾಹ್ಮಣನ ಮಗ ಯಾವಾಗಲೂ ಬ್ರಾಹ್ಮಣನೇ ಆಗಿರಬೇಕಾಗಿಲ್ಲ. ಆದರೆ ಅವನು ಬ್ರಾಹ್ಮಣನಾಗಲು ಎಲ್ಲಾ ಸಾಧ್ಯತೆಗಳೂ ಇವೆ. ಆದರೆ ಅವನು ಆಗದೆ ಇರ ಬಹುದು. ಬ್ರಾಹ್ಮಣನ ಜಾತಿ ಮತ್ತು ಬ್ರಾಹ್ಮಣನ ಗುಣಗಳು ಇವೆರಡೂ ಬೇರೆ.

ಹೇಗೆ ಸತ್ತ್ವ ರಜಸ್ ತಮಸ್ ಎಂಬ ಮೂರು ಗುಣಗಳಿವೆಯೋ ಪ್ರತಿಯೊಬ್ಬ ನಲ್ಲಿಯೂ ಈ ಮೂರು ಗುಣಗಳು ಸ್ವಲ್ಪ ಹೆಚ್ಚು ಕಡಿಮೆಯಾಗಿ ವ್ಯಕ್ತವಾಗು ತ್ತದೆಯೊ, ಅದರಂತೆಯೆ ಒಬ್ಬ ಬ್ರಾಹ್ಮಣ ಕ್ಷತ್ರಿಯ ವೈಶ್ಯ ಶೂದ್ರನಾಗುವ ಸ್ವಭಾವ ಕೂಡ ಪ್ರತಿಯೊಬ್ಬರಲ್ಲಿಯೂ ಸ್ವಲ್ಪ ಹೆಚ್ಚು ಕಡಿಮೆ ಇದೆ. ಒಬ್ಬೊಬ್ಬ ರಲ್ಲಿ ಒಂದೊಂದು ಹೆಚ್ಚುಕಡಿಮೆಯಾಗಿ ವ್ಯಕ್ತವಾಗುವುದು. ಉದಾಹರಣೆಗೆ ಮನುಷ್ಯ ನಿರತನಾಗಿರುವ ವೃತ್ತಿಯನ್ನು ಗಮನಿಸಿ. ಅವನು ಸಂಬಳಕ್ಕೆ ಇನ್ನೊಬ್ಬರಿ ಗಾಗಿ ದುಡಿಯುತ್ತಿರುವಾಗ ಅವನು ಶೂದ್ರ. ಅವನು ಲಾಭಕ್ಕಾಗಿ ಇತರರೊಡನೆ ವ್ಯವಹಾರ ನಡೆಸುತ್ತಿರುವಾಗ ಅವನು ವೈಶ್ಯ. ಅವನು ಧರ್ಮದ ಪರವಾಗಿ ಅಧರ್ಮಕ್ಕೆ ವಿರೋಧವಾಗಿ ಯಾವಾಗ ಹೋರಾಡುತ್ತಿರುವನೋ ಆಗ ಕ್ಷತ್ರಿಯ. ಯಾವಾಗ ಅವನು ಭಗವಂತನ ಧ್ಯಾನದಲ್ಲಿ ಮತ್ತು ಅವನನ್ನು ಇತರರಿಗೆ ಬೋಧಿಸುವುದರಲ್ಲಿ ಕಾಲ ಕಳೆಯುವನೊ ಅವನು ಬ್ರಾಹ್ಮಣ. ಒಬ್ಬ ಒಂದು ವರ್ಣದಿಂದ ಮತ್ತೊಂದು ವರ್ಣಕ್ಕೆ ಸ್ವಾಭಾವಿಕವಾಗಿ ಬದಲಾಯಿಸಬಹುದು. ಇಲ್ಲದೇ ಇದ್ದರೆ ವಿಶ್ವಾಮಿತ್ರ ಹೇಗೆ ಬ್ರಾಹ್ಮಣನಾದ? ಪರಶುರಾಮ ಹೇಗೆ ಕ್ಷತ್ರಿಯನಾದ?

ಪಾಶ್ಚಾತ್ಯ ನಾಗರಿಕತೆ ಹಬ್ಬಿದ್ದು ಕತ್ತಿಯ ಮೂಲಕ. ಆರ್ಯರು ಸಮಾಜವನ್ನು ವಿವಿಧವರ್ಣಗಳಾಗಿ ಮಾರ್ಪಡಿಸಿ ಅದನ್ನು ಒಂದು ಸೋಪಾನ ಪಂಕ್ತಿಯನ್ನಾಗಿ ಮಾಡಿದರು. ಒಬ್ಬನ ಸಂಸ್ಕೃತಿ ಮತ್ತು ಪಾಂಡಿತ್ಯಕ್ಕೆ ತಕ್ಕಂತೆ ಅವನು ಮೇಲ ಮೇಲಕ್ಕೆ ಹೋಗಬಹುದಾಗಿತ್ತು. ಯುರೋಪಿನಲ್ಲಿ ಬಲಶಾಲಿಯೆ ಯಾವಾಗಲೂ ಗೆಲ್ಲುವನು, ದುರ್ಬಲ ಯಾವಾಗಲೂ ಸಾಯುವನು. ಆದರೆ ಭರತಖಂಡದಲ್ಲಿ ಆದರೋ ಸಮಾಜದಲ್ಲಿರುವ ಪ್ರತಿಯೊಂದು ಕಾನೂನು ದುರ್ಬಲರ ರಕ್ಷಣೆಗೆ.

ಇದೇ ನಮ್ಮ ವರ್ಣದ ಆದರ್ಶ. ಇದೇ ಮಾನವಕೋಟಿಯನ್ನು ಕ್ರಮೇಣ ಅಹಿಂಸೆ, ಶಾಂತಿ, ಅಧ್ಯಯನ ಮತ್ತು ಧ್ಯಾನಾವಸ್ಥೆಯ ಸ್ಥಿತಿಗೆ ಒಯ್ಯುವುದು. ಈ ಆದರ್ಶದಲ್ಲಿ ದೇವರು ಇರುವನು.

ದೇವರು ಮಾನವನಿಗೆ ಕೊಟ್ಟಿರುವ ಅತ್ಯದ್ಭುತವಾದ ಸಂಸ್ಥೆಗಳಲ್ಲಿ ವರ್ಣಗಳು ಒಂದು ಎಂದು ಭಾವಿಸುತ್ತೇನೆ. ಕಾಲಕ್ರಮೇಣ ಅದರಿಂದ ಬಂದ ನ್ಯೂನತೆಗಳು, ಪರದೇಶದವರ ದಬ್ಬಾಳಿಕೆ ಮತ್ತು ಎಲ್ಲಕ್ಕಿಂತ ಹೆಚ್ಚಾಗಿ ಬ್ರಾಹ್ಮಣರೆಂದು ಕರೆಸಿ ಕೊಳ್ಳುವುದಕ್ಕೆ ಯೋಗ್ಯರಲ್ಲದ ವ್ಯಕ್ತಿಗಳ ಅಜ್ಞಾನ ಮತ್ತು ಅಹಂಕಾರ ಇವುಗಳಿಂದ ಅದ್ಭುತವಾದ ವರ್ಣಸಂಸ್ಥೆ ತನ್ನ ಕೆಲಸವನ್ನು ಸರಿಯಾಗಿ ನಿರ್ವಹಿಸಲು ಆಗಲಿಲ್ಲ ಎಂದು ನಾವು ನಂಬುವೆವು. ಇದಾಗಲೇ ಭರತಖಂಡಕ್ಕೆ ಎಷ್ಟೋ ಒಳ್ಳೆಯದನ್ನು ಮಾಡಿದೆ. ಮುಂದೆಯೂ ಭಾರತೀಯರನ್ನು ಅವರ ಗುರಿಯೆಡೆಗೆ ತೆಗೆದುಕೊಂಡು ಹೋಗುವುದರಲ್ಲಿ ಸಂದೇಹವಿಲ್ಲ.

ವರ್ಣ ಹೋಗಕೂಡದು. ಆದರೆ ಅದನ್ನು ಈಗಿನ ಕಾಲಕ್ಕೆ ಹೊಂದಿಸಿಕೊಳ್ಳ ಬೇಕಾಗಿದೆ. ಆ ಹಳೆಯ ಸಂಸ್ಥೆಯಲ್ಲಿ ಸಾವಿರಾರು ಸಂಸ್ಥೆಗಳಿಗೆ ಜೀವದಾನ ಮಾಡ ಬಲ್ಲಂತಹ ಚೈತನ್ಯವಿದೆ.

ವರ್ಣಗಳು ಇರಬೇಕು, ಹಕ್ಕುಗಳು ಹೋಗಬೇಕು. ಸಮಾಜ ಸಾಧಾರಣವಾಗಿ ಪಂಗಡವಾಗಿ ಕವಲೊಡೆಯುವುದು. ಆದರೆ ಈ ಹಕ್ಕುಗಳು ಹೋಗಬೇಕು. ವರ್ಣ ಒಂದು ಸ್ವಾಭಾವಿಕವಾದ ವರ್ಗೀಕರಣ. ನಾನು ಸಮಾಜದಲ್ಲಿ ಒಂದು ಕೆಲಸ ಮಾಡುತ್ತೇನೆ, ನೀನು ಇನ್ನೊಂದು ಕೆಲಸ ಮಾಡುತ್ತೀಯೆ. ನೀನೊಂದು ರಾಜ್ಯವ ನ್ನಾಳಬಹುದು. ನಾನೊಂದು ಪಾದರಕ್ಷೆಯನ್ನು ರಿಪೇರಿಮಾಡುತ್ತೇನೆ. ಆದರೆ ಇದ ರಿಂದ ನೀನು ಹೇಗೆ ನನಗಿಂತ ಮೇಲಾದೆ. ನೀನು ನನ್ನ ಪಾದರಕ್ಷೆಯನ್ನು ರಿಪೇರಿ ಮಾಡಬಲ್ಲೆಯಾ? ನಾನು ರಾಜ್ಯವಾಳಬಲ್ಲೆನೆ? ನನಗೆ ಎಕ್ಕಡ ಹೊಲೆಯುವುದು ಗೊತ್ತಿದೆ, ನೀನು ವೇದಗಳನ್ನು ಚೆನ್ನಾಗಿ ಓದಿ ತಿಳಿದುಕೊಳ್ಳಬಲ್ಲೆ. ಆದರೆ ಅದರಿಂದ ನೀನು ನನ್ನ ಮೇಲೆ ಅಧಿಕಾರ ಚಲಾಯಿಸಲಾಗುವುದಿಲ್ಲ. ಒಬ್ಬ ಕೊಲೆಯನ್ನು ಮಾಡಿ ದರೆ ಅವನನ್ನು ಏತಕ್ಕೆ ಹೊಗಳಬೇಕು? ಇನ್ನೊಬ್ಬ ಒಂದು ಸೇಬಿನ ಹಣ್ಣನ್ನು ಕದ್ದರೆ ಅವನನ್ನು ಏತಕ್ಕೆ ಗಲ್ಲಿಗೆ ಹಾಕಬೇಕು? ಈ ಪರಿಸ್ಥಿತಿ ಹೋಗಬೇಕಾಗಿದೆ.

\textbf{ವರ್ಣ ಒಳ್ಳೆಯದು}: ಜೀವನದ ಸಮಸ್ಯೆಗಳನ್ನು ಎದುರಿಸಲು ಸ್ವಾಭಾವಿಕವಾದ ಮಾರ್ಗವೇ ಇದು. ಜನಗಳು ಪಂಗಡಗಳಾಗಿ ವಿಭಾಗ ಆಗಲೇಬೇಕಾಗಿದೆ. ನೀವು ಅದರಿಂದ ಪಾರಾಗಲಾರಿರಿ. ನೀವು ಎಲ್ಲಿ ಹೋದರೂ ಅಲ್ಲಿ ಜಾತಿ ಇದ್ದೇ ಇರು ವುದು. ಆದರೆ ಈ ಹಕ್ಕುಗಳು ಇರಬೇಕೆಂದು ಅರ್ಥವಲ್ಲ. ಈ ಹಕ್ಕುಗಳನ್ನು ಹೊಡೆದು ಓಡಿಸಬೇಕು. ನೀವು ಬೆಸ್ತನಿಗೆ ವೇದಾಂತವನ್ನು ಬೋಧಿಸಿದರೆ, ಅವನು, 'ನಾನು ನಿನ್ನಂತೆಯೇ ಮನುಷ್ಯ, ನಾನು ಬೆಸ್ತ, ನೀನು ತತ್ತ್ವಜ್ಞಾನಿನ್; ಆದರೆ ಇಬಪ್ ರಲ್ಲಿಯೂ ಒಂದೇ ಬ್ರಹ್ಮನಿರುವನು' ಎಂದು ಹೇಳುವನು. ನಮಗೆ ಬೇಕಾಗಿರು ವುದು ಅದು. ಯಾರಿಗೂ ವಿಶೇಷ ಹಕ್ಕುಗಳಿಲ್ಲ. ಎಲ್ಲರಿಗೂ ಸಮಾನವಾದ ಅವ ಕಾಶ ಇರಬೇಕು. ಪ್ರತಿಯೊಬಪ್ರಿಗೂ ಪರಮಾತ್ಮ ಎಲ್ಲರಲ್ಲಿಯೂ ಇರುವನೆಂದು ಸಾರಿ. ಪ್ರತಿಯೊಬಪ್ನೂ ತನ್ನ ಉದ್ಧಾರವನ್ನು ತಾನೇ ಮಾಡಿಕೊಳ್ಳುತ್ತಾನೆ. ಪ್ರತ್ಯೇಕ ಹಕ್ಕುಬಾಧ್ಯತೆಗಳ ಕಾಲ ಆಗಿಹೋಯಿತು. ಭರತಖಂಡದಲ್ಲಿ ಅದು ಪುನಃ ಬರು ವಂತೆ ಇಲ್ಲ.

\textbf{ಅಸ್ಪೃಶ್ಯತೆ ಒಂದು ಮೌಢ್ಯತೆ}: ಹಿಂದೆ ಯಾರನ್ನು ಮಹಾತ್ಮ ಎಂದು ಕರೆಯುತ್ತಿ ದ್ದರು ಎಂದರೆ ಯಾರು ಅನೇಕ ಒಳ್ಳೆಯ ಕೆಲಸಗಳಿಂದ ವಿಶ್ವಕ್ಕೆ ಉಪಕಾರವನ್ನು ಮಾಡುವರೊ ಅವರನ್ನು. ಆದರೆ ಈಗ ನಾನು ಮಡಿ, ಇಡಿ ಪ್ರಪಂಚ ಮೈಲಿಗೆ ಎಂದು ಹೇಳುವುದಾಗಿದೆ. ನನ್ನನ್ನು ಮುಟ್ಟಬೇಡಿ, ನನ್ನನ್ನು ಮುಟ್ಟಬೇಡಿ ಎಂದು ಹಾರಾಡುತ್ತಾರೆ. ಪ್ರಪಂಚವೆಲ್ಲ ಮೈಲಿಗೆ, ನಾನೊಬಪ್ನೇ ಮಡಿ ಎಂದು ಭಾವಿಸು ತ್ತಾರೆ. ಇದೊಳ್ಳೆ ಬ್ರಹ್ಮಜ್ಞಾನ. ಅಯ್ಯೋ ದೇವರೆ, ಎಷ್ಟು ವಿಚಿತ್ರ ಇದು! ಈಗಿನ ಕಾಲದಲ್ಲಿ ಬ್ರಹ್ಮ ನಮ್ಮ ಹೃದಯದ ಅಂತರಾಳದಲ್ಲಿಯೂ ಇಲ್ಲ, ಮೇಲಿನ ಸ್ವರ್ಗದಲ್ಲಿಯೂ ಇಲ್ಲ, ಇತರರಲ್ಲಿ ಅಂತರ್ಯಾಮಿಯಾಗಿಯೂ ಇಲ್ಲ. ಈಗ ಅವನು ಅಡಿಗೆಯ ಪಾತ್ರೆಯನ್ನು ಹೊಕ್ಕಿರುವನು.

ನಾವು ಆಚಾರಶೀಲ ಹಿಂದುಗಳು. ಆದರೆ ನಾವು ನಮ್ಮನ್ನು ಮುಟ್ಟಬೇಡ ಎನ್ನು ವವರ ಗುಂಪಿಗೆ ಸೇರಿಲ್ಲ. ಅದು ಹಿಂದೂಧರ್ಮವಲ್ಲ. ಅದು ನಮ್ಮ ಶಾಸ್ತ್ರ ದಲ್ಲಿಯೂ ಇಲ್ಲ. ಇದೊಂದು ಸಂಪ್ರದಾಯವಂತರ ಮೂಢಾಚಾರ. ಇದು ಯಾವಾಗಲೂ ನಮ್ಮ ದೇಶದ ಏಳ್ಗೆಗೆ ಆತಂಕವಾಗಿದೆ. ಧರ್ಮ ಅಡಿಗೆ ಮಾಡುವ ಪಾತ್ರೆಗೆ ಪ್ರವೇಶಿಸಿದೆ. ಈಗಿನ ಹಿಂದೂಗಳು ವೈದಿಕರೂ ಅಲ್ಲ, ಜ್ಞಾನಮಾರ್ಗಿಗಳೂ ಅಲ್ಲ. ಬರೀ ‘ನನ್ನನ್ನು ಮುಟ್ಟಬೇಡಿ’ ಎಂಬುದೇ ಅವರ ಧರ್ಮದ ಸರ್ವಸ್ವ.

‘ಮುಟ್ಟಬೇಡಿ’ ಎಂಬುದು ಒಂದು ಬಗೆಯ ಮಾನಸಿಕ ಜಾಡ್ಯ. ಜೋಪಾನ! ಎಲ್ಲಾ ವಿಕಾಸವೂ ಜೀವನ, ಸಂಕೋಚವೇ ಮರಣ. ಪ್ರೀತಿ ನಮ್ಮ ಹೃದಯವನ್ನು ವಿಕಾಸ ಮಾಡುವುದು. ಸ್ವಾರ್ಥತೆ ಹೃದಯವನ್ನು ಸಂಕೋಚ ಮಾಡುವುದು. ಪ್ರೀತಿ ಯೊಂದೇ ಜೀವನದ ಏಕಮಾತ್ರ ನಿಯಮ. ನನ್ನನ್ನು ಮುಟ್ಟಬೇಡಿ ಎಂಬ ನಾಸ್ತಿಕತೆ ಯಲ್ಲಿ ನಿಮ್ಮ ಧರ್ಮ ಕೊನೆಗಾಣದಿರಲಿ. “ಆತ್ಮವತ್ ಸರ್ವಭೂತೇಷು” ಎಂಬುದು ಬರೀ ಶಾಸ್ತ್ರದಲ್ಲಿ ಮಾತ್ರ ಉಳಿಯಬೇಕೇನು? ಹಸಿದ ಹೊಟ್ಟೆಗೆ ಒಂದು ತುತ್ತು ಕೂಳು ಕೊಡದವನಿಗೆ ಮುಕ್ತಿ ಹೇಗೆ ದೊರಕಬಲ್ಲುದು? ಇನ್ನೊಬ್ಬರ ಗಾಳಿ ಸೋಕಿದರೆ ಮೈಲಿಗೆ ಆಗಿಹೋಗುವವರು ಇನ್ನೊಬ್ಬರನ್ನು ಶುದ್ಧ ಮಾಡುವುದು ಹೇಗೆ?

ಇತರರನ್ನು ನಾವು ಪೀಡಿಸುವುದನ್ನು ಬಿಡಬೇಕು. ನಾವು ಎಂತಹ ವಿಚಿತ್ರ ಸ್ಥಿತಿಗೆ ಬಂದಿರುವೆವು! ಮೋಚಿ ಮೋಚಿಯಂತೆ ಬಂದರೆ ಯಾರೂ ಅವನನ್ನು ಹತ್ತಿರ ಸೇರಿ ಸುವುದಿಲ್ಲ. ಆದರೆ ಅವನ ತಲೆಯ ಮೇಲೆ ಪಾದ್ರಿ ನೀರು ಸುರಿದು ಮಂತ್ರ ಹೇಳಿ ಅವನನ್ನು ಕ್ರೈಸ್ತನನ್ನಾಗಿ ಮಾಡಿದರೆ, ಅವನೊಂದು ಚಿಂದಿಯನ್ನು ತೊಟ್ಟಿದ್ದರೂ ಯಾರಿಗೂ ಅವನಿಗೆ ಹಸ್ತಲಾಘವ ಕೊಡದೆ ಇರುವಷ್ಟು ಧೈರ್ಯವಿಲ್ಲ. ಯಾವುದೂ ಇದಕ್ಕಿಂತ ಹಾಸ್ಯಾಸ್ಪದವಾಗಿರುವುದಿಲ್ಲ.

ಹಿಂದೂಗಳು ಸಹಾನುಭೂತಿ ತೋರದೆ ಇದ್ದುದರಿಂದ ಸಾವಿರಾರು ಜನ ಪರೆ ಯರು ಮದ್ರಾಸಿನಲ್ಲಿ ಕ್ರೈಸ್ತಧರ್ಮಕ್ಕೆ ಸೇರಿದರು. ಇದು ಬರೀ ಹಸಿವಿಗಾಗಿ ಮಾಡಿದ್ದು ಎಂದು ಭಾವಿಸಬೇಡಿ. ಅವರಿಗೆ ನಮ್ಮಿಂದ ಯಾವ ಸಹಾನು ಭೂತಿಯೂ ದೊರಕುತ್ತಿಲ್ಲ. ಹಗಲು ರಾತ್ರಿ ನಾವು ಅವರಿಗೆ ‘ನಮ್ಮನ್ನು ಮುಟ್ಟ ಬೇಡಿ, ನಮ್ಮನ್ನು ಮುಟ್ಟಬೇಡಿ’ ಎಂದು ಅರಚಿಕೊಳ್ಳುತ್ತಿರುವೆವು. ಈ ದೇಶದಲ್ಲಿ ಏನಾದರೂ ದಯೆ ಮತ್ತು ಮರುಕ ಎಂಬುದು ಉಳಿದಿರುವುದೆ? ನಾವೆಲ್ಲ ‘ಮುಟ್ಟ ಬೇಡಿ’ ಎಂಬುವವರ ಗುಂಪಿಗೆ ಸೇರಿದವರು. ಇಂತಹ ಆಚಾರವನ್ನು ಕಿತ್ತುಹಾಕಿ. ಈ ‘ಮುಟ್ಟಬೇಡಿ’ ಎಂಬ ಗೋಡೆಯನ್ನು ಒಡೆದುಹಾಕಿ, ಎಲ್ಲರೂ ಬನ್ನಿ, ದೀನರೆ, ದಲಿತರೆ ದುಃಖಿಗಳೆ ಎಂದು ಕರೆಯಬೇಕೆಂದು ಕೆಲವು ವೇಳೆ ಅನ್ನಿಸುವುದು. ಅವರು ಜಾಗ್ರತರಾಗುವವರೆಗೆ ಜಗನ್ಮಾತೆ ಜಾಗ್ರತಳಾಗುವುದಿಲ್ಲ.

ಪ್ರತಿಯೊಬ್ಬ ಹಿಂದೂ ಮತ್ತೊಬ್ಬನಿಗೆ ಸಹೋದರ ಎಂದು ನಾನು ಹೇಳುತ್ತೇನೆ. ನಾವೇ ಅವರಿಗೆ ಮುಟ್ಟಬೇಡಿ ಎಂದು ಅಧೋಗತಿಗೆ ಒಯ್ದವರು. ಆದಕಾರಣ ಇಡೀ ದೇಶ ಅಜ್ಞಾನ ಹೇಡಿತನ ನೀಚತನ ಇವುಗಳ ಆಳಕ್ಕೆ ಮುಳುಗಿದೆ. ಅವರನ್ನು ನಾವು ಮೇಲಕ್ಕೆ ಎತ್ತಬೇಕಾಗಿದೆ. ಅವರಿಗೆ ಭರವಸೆ ಮತ್ತು ಶ್ರದ್ಧೆಯ ಮಾತನ್ನು ಆಡಬೇಕಾಗಿದೆ. ಅವರಿಗೆ ನಾವು ‘ನಮ್ಮಂತೆಯೇ ನೀವೂ ಮನುಷ್ಯರು. ನಮಗೆ ಇರುವ ಹಕ್ಕುಬಾಧ್ಯತೆಗಳೆಲ್ಲ ನಿಮಗೆ ಇವೆ’ ಎಂದು ಹೇಳಬೇಕಾಗಿದೆ.

ವರ್ಣದ ಸಮಸ್ಯೆಯ ಪರಿಹಾರ ವರ್ಣದಲ್ಲಿ ಮೇಲಿರುವವರನ್ನು ಕೆಳಗೆ ತರುವು ದಲ್ಲ. ಏನು ಸಿಕ್ಕಿದರೆ ಅದನ್ನೆಲ್ಲ ತಿನ್ನುವುದು ಕುಡಿಯುವುದಲ್ಲ, ಎಲ್ಲಾ ವಿಧಿ ನಿಷೇಧಗಳನ್ನು ಗಾಳಿಗೆ ಎಸೆದು ಸ್ವಚ್ಛಂದ ಜೀವನ ನಡೆಸುವುದಲ್ಲ. ನಮ್ಮಲ್ಲಿ ಪ್ರತಿ ಯೊಬ್ಬರೂ ವೇದಾಂತದ ಬೋಧನೆಯನ್ನು ಅನುಷ್ಠಾನಕ್ಕೆ ತಂದು ನಾವು ಆಧ್ಯಾ ತ್ಮಿಕ ಸಂಪನ್ನರಾಗಿ ಆದರ್ಶ ಬ್ರಾಹ್ಮಣರಾಗುವುದೇ ಪರಿಹಾರದ ಮಾರ್ಗ. ಪ್ರತಿ ಯೊಬ್ಬರಿಗೂ ನಿಮ್ಮ ಪೂರ್ವಿಕರು ವಿಧಿನಿಯಮಗಳನ್ನು ರಚಿಸಿರುವರು–ಅವರು ಆರ್ಯರಾಗಬಹುದು, ಅನಾರ್ಯರಾಗಬಹುದು. ನಿಮ್ಮಗಳಿಗೆಲ್ಲ ಒಂದೇ ಆಜ್ಞೆಯನ್ನು ವಿಧಿಸಿರುವರು–ಅದಾವುದೆಂದರೆ ಬ್ರಾಹ್ಮಣನಾಗುವುದಕ್ಕೆ ಪ್ರಯತ್ನಿಸುವುದು. ಈ ವೇದಾಂತದ ಆದರ್ಶವನ್ನು ಭರತಖಂಡದಲ್ಲಿ ಮಾತ್ರ ಜಾರಿಗೆ ತರುವುದಲ್ಲ, ಇಡೀ ಪ್ರಪಂಚದಲ್ಲಿ ಜಾರಿಗೆ ತರಬೇಕು.

ಶಂಕರಾಚಾರ್ಯರು ಗೀತಾ ಭಾಷ್ಯದ ಪ್ರಾರಂಭದಲ್ಲಿ ಬ್ರಾಹ್ಮಣ್ಯವೇ ಮಾನವ ಕೋಟಿಯ ಆದರ್ಶ ಎಂದು ಬಹಳ ಸುಂದರವಾಗಿ ಹೇಳುವರು. ಶ್ರೀಕೃಷ್ಣನು ಬ್ರಾಹ್ಮಣರ ಬ್ರಾಹ್ಮಣ್ಯವನ್ನು ಕಾಪಾಡುವುದಕ್ಕಾಗಿ ಅವತಾರ ಮಾಡಿದ ಎಂದು ಹೇಳುವರು. ಇದೇ ಪರಮಾದರ್ಶ. ಈ ಬ್ರಾಹ್ಮಣ ಭಗವಂತನಲ್ಲೆ ಸದಾ ನೆಲಸಿರು ವವನು, ಬ್ರಹ್ಮಜ್ಞಾನವನ್ನು ಪಡೆದವನು. ಈ ಆದರ್ಶ ಮನುಷ್ಯ, ಪರಿಪೂರ್ಣ ಮನುಷ್ಯ, ಪ್ರಪಂಚದಿಂದ ಕಣ್ಮರೆಯಾಗಕೂಡದು. ಈಗ ಇರುವ ವರ್ಣಗಳಲ್ಲಿ ಎಷ್ಟೇ ಲೋಪದೋಷಗಳಿದ್ದರೂ ನಾವು ಆ ಬ್ರಾಹ್ಮಣನಿಗೆ ಈ ಗೌರವವನ್ನು ಕೊಡ ಬೇಕು. ಈ ವರ್ಣದಿಂದ ನಿಜವಾದ ಆದರ್ಶಬ್ರಾಹ್ಮಣರು ಇತರ ವರ್ಣಗಳೆಲ್ಲ ಕ್ಕಿಂತ ಹೆಚ್ಚಾಗಿ ಬಂದಿರುವರು ಅದು ನಿಜ. ಇತರ ವರ್ಣದವರೆಲ್ಲ ಅವರಿಗೆ ಈ ಗೌರವವನ್ನು ತೋರಬೇಕಾಗಿದೆ. ಅವರಲ್ಲಿರುವ ಲೋಪದೋಷಗಳನ್ನು ಧೈರ್ಯ ವಾಗಿ ತೋರಬೇಕು. ಆದರೆ ಅದೇ ಕಾಲದಲ್ಲಿ ಅವರಿಗೆ ಸಲ್ಲುವ ಗೌರವವನ್ನೂ ಕೂಡ ಕೊಡಬೇಕು.

ಸುಮ್ಮನೆ ವರ್ಣಗಳ ವಿಷಯದಲ್ಲಿ ಕಾದಾಡಿದರೆ ಪ್ರಯೋಜನವಿಲ್ಲ. ಇದು ನಮ್ಮನ್ನು ಮತ್ತೂ ದುರ್ಬಲರನ್ನಾಗಿ ಮಾಡುವುದು. ಇನ್ನೂ ಅವನತಿಯ ಆಳಕ್ಕೆ ಹೋಗುವಂತೆ ಮಾಡುವುದು. ಮೇಲಿನವರನ್ನು ಕೆಳಕ್ಕೆ ಎಳೆಯುವುದಲ್ಲ ಪರಿಹಾರ, ಕೆಳಗಿನವರನ್ನು ಮೇಲಕ್ಕೆ ಎತ್ತಬೇಕು. ನಮ್ಮ ಶಾಸ್ತ್ರದಲ್ಲೆಲ್ಲ ತೋರುವ ಮಾರ್ಗ ಇದೆ. ಇನ್ನು ನಮ್ಮ ಶಾಸ್ತ್ರದ ವಿಷಯವಾಗಿ ಚೆನ್ನಾಗಿ ಪರಿಚಯವಿಲ್ಲದವರು, ಹಿಂದಿ ನವರು ಎಂತಹ ಒಂದು ಅದ್ಭುತವಾದ ಯೋಜನೆಯನ್ನು ನಮ್ಮ ಮುಂದೆ ಇಟ್ಟಿರು ವರು ಎಂಬುದನ್ನು ಅರ್ಥಮಾಡಿಕೊಳ್ಳುವುದಕ್ಕೆ ಯೋಗ್ಯತೆ ಇಲ್ಲದವರು, ಇದಕ್ಕೆ ವಿರೋಧವಾಗಿ ಏನು ಬೇಕಾದರೂ ಹೇಳಬಹುದು. ಅದಕ್ಕೆ ಬೆಲೆಯಿಲ್ಲ. ಅವರ ಯೋಜನೆ ಏನು? ಆದರ್ಶದ ಗುರಿ ಬ್ರಾಹ್ಮಣ, ಅದರ ಅತ್ಯಂತ ಕೆಳಮೆಟ್ಟಲು ಚಂಡಾಲ. ಅವರ ಉದ್ದೇಶವೆ ಚಂಡಾಲನನ್ನು ಬ್ರಾಹ್ಮಣನನ್ನಾಗಿ ಮಾಡುವುದು. ಕ್ರಮೇಣ ಅವನಿಗೆ ಹೆಚ್ಚು ಹೆಚ್ಚಾಗಿ ಹಕ್ಕುಗಳು ಬರುವುದನ್ನು ನೋಡುವಿರಿ.

ಈಗಿನ ಕಾಲದಲ್ಲಿ ವರ್ಣದ ವಿಷಯದಲ್ಲಿ ಇಷ್ಟೊಂದು ವಾದವಿವಾದಗಳನ್ನು ನೋಡಿದಾಗ ನನಗೆ ವ್ಯಥೆಯಾಗುವುದು. ಇದು ನಿಲ್ಲಬೇಕು. ಇದರಿಂದ ಇಬ್ಬರಿಗೂ ಪ್ರಯೋಜನವಿಲ್ಲ. ಬ್ರಾಹ್ಮಣನಿಗಂತೂ ಪ್ರಯೋಜನವಿಲ್ಲ. ಏಕೆಂದರೆ ಬ್ರಾಹ್ಮಣರಿಗೆ ಮಾತ್ರ ಮೀಸಲಾದ ಹಕ್ಕು ಬಾಧ್ಯತೆಗಳು ಹೊರಟು ಹೋದವು. ಪ್ರತಿಯೊಬ್ಬ ಮೇಲಿರುವವನೂ ತನ್ನ ಸಮಾಧಿಗೆ ತಾನೇ ಸಿದ್ಧನಾಗಿರಬೇಕು. ಇವರು ಇದನ್ನು ಎಷ್ಟು ಬೇಗ ಮಾಡಿಕೊಂಡರೆ ಅಷ್ಟು ಮೇಲು. ಇವರು ಇದನ್ನು ನಿಧಾನ ಮಾಡಿದಂತೆಲ್ಲ ಕೊಳೆತು ನಾರುವುದು ಜಾಸ್ತಿಯಾಗುವುದು, ಮರಣ ಮತ್ತೂ ಭಯಾನಕವಾಗುವುದು. ಭರತಖಂಡದಲ್ಲಿ ಇತರರ ಉದ್ಧಾರಕ್ಕೆ ಬ್ರಾಹ್ಮಣ ದುಡಿಯಬೇಕು. ಇದು ಅವನ ಕರ್ತವ್ಯ. ಅವನು ಇದನ್ನು ಮಾಡಿದರೆ ಮಾತ್ರ ಬ್ರಾಹ್ಮಣ, ಮಾಡುವ ತನಕ ಮಾತ್ರ ಬ್ರಾಹ್ಮಣ.

ಯಾರು ತಾವು ಬ್ರಾಹ್ಮಣರು ಎಂದು ಭಾವಿಸುತ್ತಾರೆಯೊ ಅವನು ಮೊದಲು ತನ್ನಲ್ಲಿ ಆ ಬ್ರಾಹ್ಮಣ್ಯವನ್ನು ವ್ಯಕ್ತಗೊಳಿಸಲಿ, ಅನಂತರ ಇತರರನ್ನು ಆ ಸ್ಥಿತಿಗೆ ತೆಗೆದುಕೊಂಡು ಬರಲಿ. ಭರತಖಂಡದ ಆದರ್ಶ ಬ್ರಾಹ್ಮಣರನ್ನು ತಯಾರು ಮಾಡುವುದು. ಪವಿತ್ರದಂತೆ ಪವಿತ್ರವಾಗಿ ದೇವರಂತೆ ಸಾಧುವಾಗಿರುವ ಬ್ರಾಹ್ಮಣನ ಆದರ್ಶವನ್ನು ಮರೆಯಬೇಡಿ ಎಂದು ಬ್ರಾಹ್ಮಣರಿಗೆ ಬಿನ್ನವಿಸಿಕೊಳ್ಳು ತ್ತೇವೆ. ಆದಿಯಲ್ಲಿ ಎಲ್ಲರೂ ಬ್ರಾಹ್ಮಣರಾಗಿದ್ದರು. ಕೊನೆಗೆ ಎಲ್ಲರೂ ಬ್ರಾಹ್ಮಣ ರಾಗುತ್ತಾರೆ ಎಂದು ಮಹಾಭಾರತ ಸಾರುವುದು.

ಈಗಿನ ಅನೇಕ ಬ್ರಾಹ್ಮಣರಿಗೆ ತಮ್ಮ ಜಾತಿಯಲ್ಲಿ ಒಂದು ಹೆಮ್ಮೆ ಇರುವಂತೆ ಕಾಣುವುದು. ಭರತಖಂಡದ ವಿದ್ಯಾವಂತನೋ, ಹೊರಗಿನ ಪಂಡಿತನೋ ಯಾರಾ ದರಾಗಲಿ ಅವರಲ್ಲಿರುವ ಈ ಹೆಮ್ಮೆಗೆ ನೀರೆರೆದರೆ ಅವನಿಗೆ ತುಂಬಾ ಸಂತೋಷ ವಾಗುವುದು.

ಬ್ರಾಹ್ಮಣರೆ, ಇದು ಮರಣದ ಚಿಹ್ನೆ. ಜೋಪಾನವಾಗಿರಿ. ಏಳಿ ನಿಮ್ಮ ಬ್ರಾಹ್ಮಣ್ಯದ ಹಿರಿಮೆ ಸುತ್ತಲಿರುವ ಅಬ್ರಾಹ್ಮಣರನ್ನು ಮೇಲೆತ್ತಿ ಅದನ್ನು ವ್ಯಕ್ತ ಪಡಿಸಿ. ದೊಡ್ಡ ಯಜಮಾನನಂತೆ ಈ ಕೆಲಸ ಮಾಡಬೇಡಿ. ದುರಹಂಕಾರ ಮತ್ತು ಪೌರಪಾಶ್ಚಾತ್ಯಗಳ ಮೂಢನಂಬಿಕೆಗೆ ತುತ್ತಾಗಿ ಈ ಕಾರ್ಯವನ್ನು ಮಾಡಬೇಡಿ. ಸೇವಾ ದೃಷ್ಟಿಯಿಂದ ಈ ಕೆಲಸವನ್ನು ಮಾಡಿ. ಬ್ರಾಹ್ಮಣರನ್ನು ನಾನು ಕೇಳಿಕೊಳ್ಳುತ್ತೇನೆ. ತಮಗೆ ಗೊತ್ತಿರುವುದನ್ನು, ಶತಶತಮಾನಗಳಿಂದ ಸಂಗ್ರಹಿಸಿಟ್ಟುಕೊಂಡಿರುವ ಸಂಸ್ಕೃತಿಯನ್ನು, ಇತರ ಭಾರತೀಯರಿಗೆ ಹೇಳಿ ಅವರನ್ನು ಮೇಲೆತ್ತುವುದಕ್ಕೆ ಕಷ್ಟ ಪಡಿ. ಭರತಖಂಡದಲ್ಲಿ ನಿಜವಾದ ಬ್ರಾಹ್ಮಣ್ಯವೆಂದರೆ ಏನು ಎಂಬುದನ್ನು ಅರಿತು ಕೊಳ್ಳುವುದು ಬ್ರಾಹ್ಮಣನ ಕರ್ತವ್ಯ. ಮನು ಬ್ರಾಹ್ಮಣನಲ್ಲಿ ಎಲ್ಲಾ ಧರ್ಮ ನಿಧಿ ಇರುವುದರಿಂದ ಅವನಿಗೆ ಅಷ್ಟೊಂದು ಹಕ್ಕು ಬಾಧ್ಯತೆಗಳನ್ನು ಕೊಟ್ಟಿದೆ ಎಂದು ಸಾರುತ್ತಾನೆ. ಅವನು ಆ ನಿಧಿಯನ್ನು ತೆರೆದು ಎಲ್ಲರಿಗೂ ಹಂಚಬೇಕು.

ಬ್ರಾಹ್ಮಣ ಭರತಖಂಡದ ಜನಗಳಿಗೆ ಪ್ರಥಮದಲ್ಲಿ ಬೋಧಿಸಿದವನು ನಿಜ. ಜೀವನದಲ್ಲಿ ಪರಮಸತ್ಯವನ್ನು ಅರಿಯುವುದಕ್ಕಾಗಿ ಎಲ್ಲವನ್ನು ಇತರರಿಗಿಂತ ಮೊದಲು ತ್ಯಜಿಸಿದವನು ಇವನು. ಇತರರಿಗಿಂತ ಮುಂದೆ ಹೋದದ್ದು ಅವನ ತಪ್ಪಲ್ಲ. ಅವನು ಮಾಡಿದಂತೆ ಇತರ ವರ್ಣದವರು ಏತಕ್ಕೆ ಮಾಡಲಿಲ್ಲ? ಶುದ್ಧ ಸೋಮಾರಿಗಳಾಗಿ ಕಾಲಕಳೆಯುತ್ತ ಬ್ರಾಹ್ಮಣರು ಮುಂದೆ ಹೋಗುವುದಕ್ಕೆ ಏತಕ್ಕೆ ಬಿಟ್ಟಿರಿ?

ಆದರೆ ಜೀವನದಲ್ಲಿ ಪ್ರಯೋಜನವನ್ನು ಪಡೆಯುವುದು ಒಂದು, ಆದರೆ ಅದನ್ನು ಸ್ವಾರ್ಥಕ್ಕಾಗಿ ಬಳಸುವುದು ಬೇರೊಂದು. ಯಾವಾಗ ಅಧಿಕಾರವನ್ನು ಕೆಟ್ಟದ್ದಕ್ಕೆ ಉಪಯೋಗಿಸುವೆವೊ ಆಗ ಅದು ಕ್ರೂರವಾಗುವುದು. ಅದನ್ನು ಕೇವಲ ಇತರರ ಕಲ್ಯಾಣಕ್ಕಾಗಿ ಮಾತ್ರ ಉಪಯೋಗಿಸಬೇಕು. ಶತಮಾನಗಳಿಂದ ಸಂಗ್ರಹಿಸಿ ಬಂದ ಈ ಸಂಸ್ಕೃತಿಯ ನಿಧಿಗೆ ಬ್ರಾಹ್ಮಣ ಅಧಿಕಾರಿಯಾಗಿರುವನು. ಇದನ್ನು ಈಗ ಎಲ್ಲರಿಗೂ ಅವನು ಕೊಡಬೇಕಾಗಿದೆ. ಇದನ್ನೇ ಅವನು ಹಿಂದೆ ಕೊಡದ ಕಾರಣ ಮಹಮ್ಮದೀಯರು ನಮ್ಮನ್ನು ಆಕ್ರಮಿಸಲು ಸಾಧ್ಯವಾಯಿತು. ಮೊದಲಿನಿಂದಲೂ ಬ್ರಾಹ್ಮಣ ಈ ನಿಧಿಯನ್ನು ಇತರರಿಗೆ ನೀಡಲಿಲ್ಲ. ಆದಕಾರಣವೇ ಕಳೆದ ಸಾವಿರಾರು ವರ್ಷಗಳಿಂದ ಹೊರದೇಶದಿಂದ ಬಂದವರೆಲ್ಲರೂ ನಮ್ಮನ್ನು ಆಳಿರುವರು. ಇದ ರಿಂದಲೇ ನಾವು ಅವನತಿಗೆ ಬಂದದ್ದು. ನಮ್ಮ ಪೂರ್ವಿಕರು ಸಂಗ್ರಹಿಸಿಟ್ಟ ನಿಧಿಯ ಕೋಣೆಯನ್ನೊಡೆದು ಅದರಲ್ಲಿರುವುದನ್ನೆಲ್ಲ ಇತರರಿಗೆ ನೀಡಬೇಕು. ಬಂಗಾಳ ದೇಶದಲ್ಲಿ ಒಂದು ಗಾದೆ ಇದೆ. ಯಾವ ಹಾವು ಒಬ್ಬನನ್ನು ಕಚ್ಚಿದೆಯೋ ಅದೇ ಹಾವು ಬಂದು ಕಚ್ಚಿದವನಿಂದ ವಿಷವನ್ನು ಹೀರಿದರೆ ಅವನು ಬದುಕುತ್ತಾನೆ, ಎಂದು ಹೇಳುವರು. ಈಗ ಬ್ರಾಹ್ಮಣ ಆ ವಿಷವನ್ನು ತಾನೇ ಹೀರಬೇಕಾಗಿದೆ.

ಬ್ರಾಹ್ಮಣೇತರರಿಗೆ “ಸ್ವಲ್ಪ ತಾಳಿ, ಅವಸರಪಡಬೇಡಿ” ಎಂದು ಹೇಳುತ್ತೇನೆ. ಬ್ರಾಹ್ಮಣರೊಂದಿಗೆ ಜಗಳಕಾಯುವುದಕ್ಕೆ ಪ್ರತಿಯೊಂದು ಅವಕಾಶವನ್ನೂ ತೆಗೆದು ಕೊಳ್ಳಬೇಡಿ. ಏಕೆಂದರೆ ನಾನು ತೋರಿದಂತೆ ನಿಮ್ಮ ಸ್ಥಿತಿಗೆ ನೀವೇ ಕಾರಣ. ಆಧ್ಯಾ ತ್ಮಿಕ ಜೀವನ ಮತ್ತು ಸಂಸ್ಕೃತ ಭಾಷಾ ಅಧ್ಯಯನವನ್ನು ಕಲಿಯಬೇಡಿ ಎಂದು ಯಾರು ನಿಮಗೆ ಹೇಳಿದರು? ಇದುವರೆಗೂ ನೀವು ಏನು ಮಾಡುತ್ತಿದ್ದಿರಿ? ನೀವು ಏತಕ್ಕೆ ಅದನ್ನು ಗಮನಕ್ಕೆ ತೆಗೆದುಕೊಳ್ಳಲಿಲ್ಲ? ನಿಮಗಿಂತ ಇತರರಿಗೆ ಹೆಚ್ಚು ಬುದ್ಧಿ ಶಕ್ತಿ ಧೈರ್ಯ, ಉತ್ಸಾಹ, ಮುಂದೆ ಬರುವ ಛಲ ಇವೆಲ್ಲ ಇದೆ ಎಂದು ಈಗ ಏಕೆ ಗೊಣಗುತ್ತೀರಿ? ವೃತ್ತಪತ್ರಿಕೆಯಲ್ಲಿ ಸುಮ್ಮನೆ ವಾದವಿವಾದಗಳಲ್ಲಿ ನಿರತರಾಗು ವುದಕ್ಕಿಂತ, ನಿಮ್ಮ ಮನೆಯಲ್ಲಿಯೇ ಪರಸ್ಪರ ಜಗಳಕಾಯುವುದಕ್ಕಿಂತ (ಇದು ಪಾಪ) ಬ್ರಾಹ್ಮಣನಲ್ಲಿರುವ ಸಂಸ್ಕೃತಿಯನ್ನು ಪಡೆಯಲು ಯತ್ನಿಸಿ. ಆಗ ಸಮಸ್ಯೆ ಬಗೆಹರಿಯುವುದು. ನೀವೇಕೆ ಸಂಸ್ಕೃತದಲ್ಲಿ ವಿದ್ಯಾವಂತರಾಗಬಾರದು? ಭರತ ಖಂಡದಲ್ಲಿ ಎಲ್ಲಾ ವರ್ಣದವರಿಗೂ ಸಂಸ್ಕೃತವನ್ನು ಕಲಿಸುವುದಕ್ಕೆ ಏತಕ್ಕೆ ಲಕ್ಷಾಂತರ ರೂಪಾಯಿಗಳನ್ನು ಖರ್ಚುಮಾಡಬಾರದು? ಇದೇ ಪ್ರಶ್ನೆ. ನೀವು ಇದನ್ನು ಮಾಡಿದೊಡನೆಯೆ ನೀವು ಬ್ರಾಹ್ಮಣರಿಗೆ ಸಮನಾಗುವಿರಿ. ಭರತಖಂಡ ದಲ್ಲಿ ಶಕ್ತಿಯ ಮೂಲವೇ ಇದು.

ಕೆಳಗಿನ ವರ್ಣದಲ್ಲಿರುವವರಿಗೆ ನಾನು ಹೇಳುವುದೇ ಇದು. ನೀವು ಮೇಲೇಳ ಬೇಕಾದರೆ ಸಂಸ್ಕೃತವನ್ನು ಅಭ್ಯಾಸ ಮಾಡಬೇಕು. ಇದರಲ್ಲಿಯೇ ನಿಮ್ಮ ಕ್ಷೇಮ ಇರುವುದು. ಸುಮ್ಮನೆ ಮೇಲಿನ ವರ್ಣದವರೊಡನೆ ಜಗಳ ಕಾಯುವುದು, ಅವರ ಮೇಲೆ ಆರೋಪಮಾಡಿ ಬರೆಯುವುದು, ಇದರಿಂದ ಏನೂ ಪ್ರಯೋಜನವಿಲ್ಲ. ಇದರಿಂದ ಯಾವ ಒಳ್ಳೆಯದೂ ಆಗಲಾರದು. ಇದರಿಂದ ಜಗಳ ಕದನ ಹೆಚ್ಚು ವುದು. ಈ ಜನಾಂಗ ದುರ್ದೈವದಿಂದ ಆಗಲೇ ಒಡೆದುಹೋಗಿದೆ. ಇದರಿಂದ ಮತ್ತೂ ಚೂರುಚೂರಾಗಿ ಹೋಗುವುದು. ನೀವು ವರ್ಣಗಳನ್ನು ಒಂದೇ ಸಮನಾಗಿ ಮಾಡಬೇಕಾದರೆ ಮೇಲಿನ ವರ್ಣದವರಿಗೆ ಶಕ್ತಿಯನ್ನು ನೀಡಿರುವ ಶಿಕ್ಷಣವನ್ನು ಮತ್ತು ಸಂಸ್ಕೃತಿಯನ್ನು ಪಡೆಯುವುದೊಂದೇ ಕ್ಷೇಮದ ಹಾದಿ.

