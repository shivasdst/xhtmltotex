
\chapter{ಈಗಿನ ಹೀನಸ್ಥಿತಿ—ಅದಕ್ಕೆ ಕಾರಣಗಳು}

\textbf{ನಾವೇ ಅದಕ್ಕೆ ಹೊಣೆ}: ನಾವೆಲ್ಲ ವೇದಾಂತಿಗಳು. ನಾವೇ ನಮ್ಮ ನಾಶಕ್ಕೆ ಕಾರಣ ರಾಗುವವರೆಗೆ ಪ್ರಪಂಚದಲ್ಲಿ ಯಾವುದೂ ನಮ್ಮನ್ನು ನಾಶಮಾಡಲಾರದು ಎಂಬು ದನ್ನು ಮರೆಯದಿರಿ. ನಾವು ಯಾರನ್ನೂ ದೂರಬೇಕಾಗಿಲ್ಲ. ನಮ್ಮ ಕರ್ಮ ಇದು. ಹೊರಗಿನಿಂದ ಬಂದ ಯಾವ ವಿಷಕ್ರಿಮಿಯೂ, ದೇಹವು ದುರಭ್ಯಾಸ ಕೆಟ್ಟ ಆಹಾರ ಇವಕ್ಕೆ ತುತ್ತಾಗಿ ದುರ್ಬಲವಾಗುವವರೆಗೆ ಅದನ್ನು ಆಶ್ರಯಿಸಲಾರದು. ಆರೋಗ್ಯ ವಾಗಿರುವವನು ಸಾವಿರಾರು ವಿಷಕ್ರಿಮಿಗಳಿದ್ದರೂ ಅದರಿಂದ ಪಾರಾಗಅುನು. ಕಾರಣ ಇಲ್ಲಿದೆ, ಪರಿಣಾಮವೂ ಇಲ್ಲಿದೆ, ನಾವೇ ಅದಕ್ಕೆ ಹೊಣೆ. ಎದ್ದು ನಿಲ್ಲಿ, ಧೀರರಾಗಿ, ಹೊಣೆಗಾರಿಕೆ ನಿಮ್ಮದು ಎಂಬುದನ್ನು ಒಪ್ಪಿಕೊಳ್ಳಿ, ಇತರರನ್ನು ದೂರಿ ಪ್ರಯೋಜನವಿಲ್ಲ. ನಿಮ್ಮಲ್ಲಿರುವ ನ್ಯೂನತೆಗಳಿಗೆಲ್ಲ ನೀವೇ ಕಾರಣಕರ್ತರು.

\textbf{ಹಿಂದಿನದನ್ನು ನಾವು ಅಲ್ಲಗಳೆದೆವು}: ಈಗಿನ ಕಾಲದಲ್ಲಿ ಪ್ರತಿಯೊಬಪ್ರೂ ಹಿಂದಿ ನದನ್ನು ಸ್ಮರಿಸಿಕೊಳ್ಳುತ್ತಿರುವವರನ್ನು ದೂರುವರು. ಗತಕಾಲದ ಅನುಭವವನ್ನು ಮೆಲುಕುಹಾಕುವುದೇ ನಮ್ಮ ಇಂದಿನ ದುರವಸ್ಥೆಗೆ ಬಹುಪಾಲು ಕಾರಣ ಎನ್ನು ವರು. ಆದರೆ ನನಗೆ ಸತ್ಯ ಇದಕ್ಕೆ ವಿರೋಧವಾಗಿ ಕಾಣುವುದು. ಎಲ್ಲಿಯವರೆಗೆ ಭರತಖಂಡ ತನ್ನ ಹಿಂದಿನದನ್ನು ಮರೆತಿತ್ತೊ ಅಲ್ಲಿಯವರೆಗೆ ಹಿಂದೂದೇಶ ತಟಸ್ಥ ವಾಗಿತ್ತು. ತಮ್ಮ ಹಿಂದಿನದನ್ನು ಸ್ಮರಿಸಿಕೊಳ್ಳಲು ಪ್ರಾರಂಭ ಮಾಡಿದಂದಿನಿಂದ ಪ್ರತಿಯೊಂದು ಕಾರ್ಯಕ್ಷೇತ್ರದಲ್ಲಿ ನವಜೀವನ ಮೂಡುತ್ತಿದೆ.

ಪ್ರತಿಯೊಬ್ಬ ವಿಮರ್ಶಾತ್ಮಕ ವಿದ್ಯಾರ್ಥಿಗೂ ಭರತಖಂಡದಲ್ಲಿದ್ದ ಸಾಮಾಜಿಕ ನಿಯಮಗಳು ಕಾಲಕಾಲಕ್ಕೆ ಬದಲಾಯಿಸಿದೆ ಎನ್ನುವುದು ಗೊತ್ತಿದೆ. ಆದಿಯಲ್ಲಿ ದೊಡ್ಡ ಉದ್ದೇಶವನ್ನು ಇಟ್ಟುಕೊಂಡು ಇದನ್ನು ಅನುಷ್ಠಾನಕ್ಕೆ ತಂದರು. ಕಾಲ ಕಳೆದಂತೆ ಇದು ವ್ಯಕ್ತವಾಗುವುದನ್ನು ಮನಗಂಡರು. ಅವರನ್ನು ಮೆಚ್ಚಬೇಕಾದರೆ ಪ್ರಪಂಚ ಇನ್ನೂ ಕೆಲವು ಶತಮಾನಗಳು ತಾಳಬೇಕು. ನಮ್ಮ ಪೂರ್ವಿಕರ ಮಹಿಮೆ ಯನ್ನು ಅರ್ಥಮಾಡಿಕೊಳ್ಳುವುದಕ್ಕೆ ಇನ್ನೂ ಯೋಗ್ಯತೆಯನ್ನು ಸಂಪಾದನೆ ಮಾಡಿ ಕೊಳ್ಳದೆ ಇರುವುದೇ ನಮ್ಮ ಅವನತಿಯ ಮೂಲಕಾರಣ.

\textbf{ನಮ್ಮ ದೃಷ್ಟಿ ಸಂಕೋಚವಾಯಿತು}: ಈ ಅವನತಿಗೆ ಇರುವ ಕಾರಣಗಳಲ್ಲಿ ಒಂದು ನಮ್ಮ ದೃಷ್ಟಿಯನ್ನು ಸಂಕೋಚಮಾಡಿಕೊಂಡದ್ದು. ಇತರ ಜನಾಂಗ ಗಳೊಂದಿಗೆ ಹೋಲಿಸಿ ನೋಡಲು ನಾವು ಪರದೇಶಗಳಿಗೆ ಹೋಗಲಿಲ್ಲ. ನಮ್ಮ ಸುತ್ತಲೂ ಏನು ಆಗುತ್ತಿದೆ ಎಂಬುದನ್ನು ಗಮನಿಸಲಿಲ್ಲ. ಇದೇ ಭಾರತೀಯನ ಶೋಚನೀಯ ಸ್ಥಿತಿಗೆ ಒಂದು ಕಾರಣ.

ಭರತಖಂಡದ ದುಸ್ಥಿತಿಗೆ ಒಂದು ಮುಖ್ಯವಾದ ಕಾರಣ ಅದು ಕೂಪಮಂಡೂಕ ವಾಗಿದ್ದು. ಕಪ್ಪೆ ತನ್ನ ಚಿಪ್ಪಿನೊಳಗೆ ಹೋಗಿ ಮುಚ್ಚಿಕೊಳ್ಳುವಂತೆ ಮುಚ್ಚಿ ಕೊಂಡಿತು. ತನ್ನಲ್ಲಿರುವ ಮಾಣಿಕ್ಯವನ್ನು ಇತರ ದೇಶಗಳಿಗೆ ಕೊಡಲಿಲ್ಲ. ಆರ್ಯ ಜನಾಂಗದ ಹೊರಗೆ ಜನ ಆಧ್ಯಾತ್ಮಿಕ ಸತ್ಯಕ್ಕಾಗಿ ತವಕಪಡುತ್ತಿದ್ದಾಗ, ನಮ್ಮ ಲ್ಲಿರುವ ಅಮೃತವನ್ನು ಅವರಿಗೆ ಕೊಡಲಿಲ್ಲ. ಇತರರನ್ನು ಬಿಟ್ಟು ಯಾವ ವ್ಯಕ್ತಿಯೇ ಆಗಲಿ, ಜನಾಂಗವೇ ಆಗಲಿ ಬಾಳಲಾರದು ಎಂಬುದು ನನಗೆ ಚೆನ್ನಾಗಿ ಮನ ದಟ್ಟಾಗಿದೆ. ನಮ್ಮ ಸಮಾನ ಇಲ್ಲ ಎಂಬ ದುರಹಂಕಾರದಿಂದಲೋ ಅಥವಾ ಇನ್ನು ಯಾವ ಉದ್ದೇಶದಿಂದಲೋ ನಾವು ಸಂಕುಚಿತವಾಗುವುದಕ್ಕೆ ಪ್ರಯತ್ನ ಪಟ್ಟಾಗ ಲೆಲ್ಲ ಹಾಗೆ ಮಾಡಿ ದುರಂತಕ್ಕೆ ಈಡಾಗಿರುವೆವು. ನನ್ನ ಮನಸ್ಸಿಗೆ ಹೊಳೆಯುವ ಭರತಖಂಡದ ಅವನತಿಗೆ ಮತ್ತು ಅದರ ದುರ್ದೆಸೆಗೆ ಕಾರಣವೇ, ಅದು ತನ್ನ ಸುತ್ತಲೂ ಇತರರನ್ನು ದ್ವೇಷಿಸುವ ಗೋಡೆಯನ್ನು ಕಟ್ಟಿ ಅದರಿಂದ ಆವೃತ ವಾಗಿದ್ದು. ಹಿಂದಿನವರು ಇದನ್ನು ಬೌದ್ಧರೊಡನೆ ಬೆರೆಯದಿರಲಿ ಎಂದು ಮಾಡಿ ದರು. ಇದಕ್ಕೆ ಹಿಂದಿನವರಾಗಲಿ ಅಥವಾ ಇಂದಿನವರಾಗಲಿ ಯಾವ ಕಾರಣವನ್ನು ಕೊಟ್ಟರೂ ಚಿಂತೆಯಿಲ್ಲ, ಇದರಿಂದ ಏನು ಆಗಬೇಕಾಯಿತೋ ಅದು ಆಯಿತು. ಅದೇ ಇತರರನ್ನು ದ್ವೇಷಿಸಿ ನಾವು ಎಂದಿಗೂ ಉದ್ಧಾರವಾಗಲಾರೆವು ಎಂಬುದು ಚೆನ್ನಾಗಿ ವ್ಯಕ್ತವಾಯಿತು. ಹಿಂದಿನ ಕಾಲದಲ್ಲಿ ಜನಾಂಗದ ಪ್ರಗತಿಯಲ್ಲಿ ಅಗ್ರಭಾಗ ದಲ್ಲಿದ್ದ ದೇಶವನ್ನು ಈಗ ಪ್ರಪಂಚದ ಜನರು ತಿರಸ್ಕಾರದೃಷ್ಟಿಯಿಂದ ನೋಡು ವರು. ಅವರನ್ನು ಲೆಕ್ಕಿಸುವುದೇ ಇಲ್ಲ. ನಮ್ಮ ಪೂರ್ವಿಕರು ಯಾವ ನಿಯಮವನ್ನು ಕಂಡುಹಿಡಿದು ಅದನ್ನು ಅನುಷ್ಠಾನಕ್ಕೆ ತಂದರೋ ಅದನ್ನು ವಿರೋಧಿಸಿದವರಿಗೆ ನಾವೇ ಒಂದು ಉದಾಹರಣೆಯಾಗಿರುವೆವು.

\textbf{ಧರ್ಮದ ಅಪಪ್ರಯೋಗ}: ಯಾವ ದೇಶದಲ್ಲಿ ಅಲ್ಲಿಯ ದೊಡ್ಡ ವ್ಯಕ್ತಿಗಳು ಕಳೆದ ಎರಡು ಸಾವಿರ ವರ್ಷಗಳಿಂದಲೂ ಆಹಾರವನ್ನು ಎಡಗೈಯಿಂದ ತೆಗೆದು ಕೊಳ್ಳುವುದೆ ಬಲಗೈಯಿಂದ ತೆಗೆದುಕೊಳ್ಳುವುದೆ, ನೀರನ್ನು ಎಡಗಡೆಯಿಂದ ಕುಡಿ ಯುವುದೆ ಬಲಗಡೆಯಿಂದ ಕುಡಿಯುವುದೆ ಎಂಬುದನ್ನು ಚರ್ಚಿಸುತ್ತಿರುವರೊ ಅಂತಹ ದೇಶ ನಾಶವಾಗದಿದ್ದರೆ ಇನ್ನಾವ ದೇಶ ನಾಶವಾಗುವುದು? ಕಳೆದ ಆರೇಳು ಶತಮಾನಗಳಿಂದಲೂ ನಮ್ಮ ಅವನತಿಯ ಕಾಲದಲ್ಲಿ ನೂರಾರು ಜನ ಪಂಡಿತರು, ನಾವು ನೀರನ್ನು ಎಡಗೈಯಿಂದ ಕುಡಿಯುವುದೆ ಬಲಗೈಯಿಂದ ಕುಡಿಯುವುದೆ? ಕೈಯನ್ನು ಮೂರೂ ಸಲ ತೊಳೆಯುವುದೆ ನಾಲ್ಕು ಸಲ ತೊಳೆಯುವುದೆ? ನಾವು ಬಾಯನ್ನು ಐದು ಸಲಿಯೇ ಆರು ಸಲಿಯೇ ಮುಕ್ಕಳಿಸುವುದು? ಎಂಬ ವಿಷಯಗಳನ್ನು ಚರ್ಚಿಸುತ್ತಿರುವು ದನ್ನು ನೋಡಿ! ಇಂತಹ ಕೆಲಸಕ್ಕೆ ಬಾರದ ವಿಷಯಗಳನ್ನು ಬಹಳ ಮುಖ್ಯವೆಂದು ಚರ್ಚಿಸುತ್ತ ಅದರ ಮೇಲೆ ವಿದ್ವತ್ ಪೂರ್ಣವಾದ ಪುಸ್ತಕಗಳನ್ನು ಬರೆಯುವ ಜನರಿಂದ ಇನ್ನೇನನ್ನು ನೀವು ನಿರೀಕ್ಷಿಸಬಲ್ಲಿರಿ!

ನಮ್ಮ ಧರ್ಮ ಅಡಿಗೆ ಮನೆಯ ಒಳಗೆ ಪ್ರವೇಶ ಮಾಡುವ ಅಪಾಯವಿದೆ. ನಮ್ಮಲ್ಲಿ ಹಲವರು ಈಗ ವೇದಾಂತಿಗಳೂ ಅಲ್ಲ, ಪೌರಾಣಿಕರೂ ಅಲ್ಲ, ತಾಂತ್ರಿ ಕರೂ ಅಲ್ಲ. “ನಮ್ಮನ್ನು ಮುಟ್ಟಬೇಡಿ”, ಎನ್ನುವವರು ಮಾತ್ರ ಆಗಿರುವೆವು. ನಮ್ಮ ಧರ್ಮ ಅಡಿಗೆಯ ಮನೆಯಲ್ಲಿದೆ. ನಮ್ಮ ದೇವರೆ ಅಡಿಗೆ ಮಾಡುವ ಪಾತ್ರೆ. ನಮ್ಮ ಧರ್ಮವೇ “ನಮ್ಮನ್ನು ಮುಟ್ಟಬೇಡಿ, ನಾವು ಮಡಿ” ಎಂದು ಹೇಳಿಕೊಳ್ಳು ವುದು. ಇನ್ನೊಂದು ಶತಮಾನ ಹೀಗೆ ಆದರೆ ನಮ್ಮಲ್ಲಿ ಪ್ರತಿಯೊಬ್ಬರೂ ಹುಚ್ಚರ ಆಸ್ಪತ್ರೆಗೆ ಸೇರಬೇಕಾಗುವುದು. ಬುದ್ಧಿ ಮಂದವಾಗುತ್ತ ಬರುವುದರ ಚಿಹ್ನೆ ಇದು. ಆಗ ಗಹನವಾದ ಯಾವುದನ್ನೂ ಅದು ಗ್ರಹಿಸಲಾರದು. ಆಗ ಎಲ್ಲಾ ನವೀನತೆಯೂ ನಾಶವಾಗಿ ಮನಸ್ಸು ತನ್ನ ಶಕ್ತಿಯನ್ನು ಕಳೆದುಕೊಳ್ಳುತ್ತದೆ. ಹೊಸದಾಗಿ ಏನನ್ನೂ ಆಲೋಚಿಸಲಾರದಾಗುತ್ತದೆ. ಸುಮ್ಮನೆ ಯಾವುದಾದರೂ ಅರ್ಥವಿಲ್ಲದ ಆಚಾರದ ಪ್ರದಕ್ಷಿಣೆ ಮಾಡುತ್ತಿರುತ್ತದೆ.

\textbf{ಜನಸಾಮಾನ್ಯರನ್ನು ಪೀಡಿಸಿದ್ದು}: ನಮ್ಮ ಜನಾಂಗದ ಒಂದು ಮಹಾ ಪಾಪವೇ ಇಲ್ಲಿಯ ಜನಸಾಧಾರಣರನ್ನು ನಿರ್ಲಕ್ಷಿಸಿದ್ದು. ಇದೇ ಭರತಖಂಡದ ಅವನತಿಗೆ ಒಂದು ಮುಖ್ಯ ಕಾರಣ ಎಂದು ಭಾವಿಸುತ್ತೇನೆ. ಭರತಖಂಡದ ಜನಸಾಧಾರಣರಿಗೆ ಪುನಃ ವಿದ್ಯೆ ಕೊಟ್ಟು ಊಟ ಕೊಟ್ಟು ಅವರನ್ನು ಚೆನ್ನಾಗಿ ನೋಡಿಕೊಳ್ಳುವ ತನಕ ಯಾವ ರಾಜಕೀಯದಿಂದಲೂ ಏನೂ ಪ್ರಯೋಜನವಿಲ್ಲ. ಜನಸಾಧಾರಣರು ನಮ್ಮ ವಿದ್ಯಾಭ್ಯಾಸಕ್ಕೆ ದುಡ್ಡು ಕೊಡುವರು, ಅವರು ನಮ್ಮ ದೇವಸ್ಥಾನಗಳನ್ನು ಕಟ್ಟುವರು. ಅವರಿಗೆ ನಮ್ಮಿಂದ ಸಿಕ್ಕುವ ಫಲವೇ ನಮ್ಮ ಕಾಲಿನ ಒದೆತ. ಅವರು ನಿಜವಾಗಿ ನಮ್ಮ ಆಳಾಗಿರುವರು. ನಾವು ಭರತಖಂಡವನ್ನು ಮೇಲಕ್ಕೆ ಎತ್ತಬೇಕಾ ದರೆ, ಜನಸಾಧಾರಣರಿಗಾಗಿ ಕೆಲಸ ಮಾಡಬೇಕು.

\textbf{ಸ್ತ್ರೀಯರನ್ನು ನಿರ್ಲಕ್ಷಿಸಿದ್ದು}: "ಯತ್ರ ನಾರ್ಯಸ್ತು ಪೂಜ್ಯಂತೇ ರಮಂತೇ ತತ್ರ ದೇವತಾಃ" ನಾರಿಯರನ್ನು ಎಲ್ಲಿ ಪೂಜ್ಯದೃಷ್ಟಿಯಿಂದ ನೋಡುವರೊ ಅಲ್ಲಿ ದೇವತೆ ಗಳು ತೃಪ್ತರಾಗುವರು ಎನ್ನುವನು ಮನು. ನಾವು ಮಹಾಪಾಪಿಗಳು. ಅದಕ್ಕೆಲ್ಲ ಕಾರಣ, ಸ್ತ್ರೀಯರನ್ನು ಕೆಲಸಕ್ಕೆ ಬಾರದ ಕೀಟಗಳು, ನರಕಕ್ಕೆ ಅವಳೇ ಮಾರ್ಗ ಎಂದು ಕರೆದದ್ದು. ಹಿಂದಿನ ಖುಷಿಗಳು "ನೀನೇ ಸ್ತ್ರೀ, ನೀನೇ ಪುರುಷ, ನೀನೇ ಕುಮಾರ, ನೀನೇ ಕುಮಾರಿ, ಎಂದಿರುವರು. ಆದರೆ ನಾವಾದರೋ ಈ ನಾರಿ ಎಂಬುವ ಮೋಹಿನಿಯನ್ನು ಯಾರು ಸೃಷ್ಟಿಸಿದರು ಎಂದು ಮೂದಲಿಸುತ್ತೇವೆ.

ಈ ದೇಶದಲ್ಲಿ ಗಂಡಸರಿಗೂ ಹೆಂಗಸರಿಗೂ ಅಷ್ಟೊಂದು ವ್ಯತ್ಯಾಸವನ್ನು ಏತಕ್ಕೆ ಮಾಡುತ್ತಾರೋ ನನಗೆ ಗೊತ್ತಿಲ್ಲ. ಆದರೆ ನಮ್ಮ ವೇದಾಂತವಾದರೂ ಒಂದೇ ಆತ್ಮ ಎಲ್ಲರಲ್ಲಿಯೂ ಇದೆಯೆಂದು ಹೇಳುವುದು. ನೀವು ಯಾವಾಗಲೂ ಹೆಂಗಸ ರನ್ನು ದೂರುತ್ತೀರಿ. ನೀವು ಅವರ ಉದ್ಧಾರಕ್ಕೆ ಏನು ಮಾಡಿರುವಿರಿ? ಹಲವಾರು ಸ್ಮೃತಿಶಾಸ್ತ್ರಗಳನ್ನು ಬರೆದು ಸ್ತ್ರೀಯರುಗಳಿಗೆ ಅಷ್ಟ ಬಂಧನಗಳನ್ನು ನಿರ್ಮಿಸಿದಿರಿ. ಪುರುಷರು ಸ್ತ್ರೀಯರನ್ನು ಒಂದು ಮಕ್ಕಳು ಹೆರುವ ಕಾರ್ಖಾನೆಗಳಾಗಿ ಮಾಡಿದ್ದಾರೆ. ಸಾಕ್ಷಾತ್ ಜಗಜ್ಜನನಿಯ ಜೀವಂತ ಪ್ರತೀಕದಂತಿರುವ ಸ್ತ್ರೀಯರನ್ನು ನೀವು ಮೇಲಕ್ಕೆ ಎತ್ತದಿದ್ದರೆ, ನಿಮ್ಮ ಉದ್ಧಾರಕ್ಕೆ ಬೇರೆ ಮಾರ್ಗವೇ ಇಲ್ಲ.

\begin{center}
\textbf{ಇತರ ಕಾರಣಗಳು, ಪರಿಣಾಮಗಳು ಮತ್ತು ಪರಿಹಾರ}
\end{center}

\textbf{ಸಾಂಸ್ಕೃತಿಕ ಪಾಷಂಡತನ ಮತ್ತು ಮತಭ್ರಾಂತಿ}: ಭರತಖಂಡದಲ್ಲಿ ಪ್ರಗತಿಗೆ ಹಲವು ಆತಂಕಗಳಿವೆ. ಅವುಗಳಲ್ಲಿ ಎರಡು ಮುಖ್ಯ. ಇತ್ತ ಪುಲಿ ಅತ್ತ ದರಿ ಎಂಬಂತೆ ಒಂದು ಜಡವಾದ, ನಾಸ್ತಿಕವಾದ; ಮತ್ತೊಂದು ಅದಕ್ಕೆ ವಿರುದ್ಧವಾದ ಬರಿ ಮೂಢ ನಂಬಿಕೆ. ಇವುಗಳಿಂದ ನಾವು ಪಾರಾಗಬೇಕಾಗಿದೆ.

ಪಾಶ್ಚಾತ್ಯ ಜ್ಞಾನಾಮೃತವನ್ನು ಪಾನಮಾಡಿದ ಆಧುನಿಕ ತಾನು ಸರ್ವಜ್ಞ ಎಂದು ಭಾವಿಸುವನು. ಅವನು ನಮ್ಮ ಪೂರ್ವ ಕಾಲದ ಜ್ಞಾನಿಗಳನ್ನು ತಾತ್ಸಾರವಾಗಿ ಕಾಣುವನು. ಭಾರತೀಯ ಭಾವನೆಗಳನ್ನು ಒಂದು ಪೊಳ್ಳು ಹರಟೆ ಎಂದು, ಇಲ್ಲಿಯ ತತ್ತ್ವಶಾಸ್ತ್ರವನ್ನು ಒಂದು ಹುಡುಗಾಟಿಕೆ ಎಂದು, ಇಲ್ಲಿಯ ಧರ್ಮವನ್ನು ಒಂದು ಮೂರ್ಖರ ಮೂಢನಂಬಿಕೆ ಎಂದು ಭಾವಿಸುವನು. ಅನುಕರಣೆ ನಾಗರಿಕತೆ ಯಲ್ಲ. ನಾನು ರಾಜನ ವೇಷಭೂಷಣಗಳಿಂದ ಅಲಂಕರಿಸಿಕೊಳ್ಳಬಹುದು. ಆದರೆ ಇದ ರಿಂದ ನಾನೊಬ್ಬ ರಾಜನಾಗುತ್ತೇನೆಯೇ? ಕತ್ತೆ ಸಿಂಹದ ಚರ್ಮ ಹೊದ್ದ ಮಾತ್ರಕ್ಕೆ ಸಿಂಹವಾಗಲಾರದು. ಅನುಕರಣೆ, ಅಂಧ ಅನುಕರಣೆ, ಇವುಗಳಿಂದ ನಾವು ಸ್ವಲ್ಪವೂ ಮುಂದೆ ಹೋಗಲಾರೆವು. ನಿಜವಾಗಿ ಮನುಷ್ಯ ಅಧೋಗತಿಗೆ ಇಳಿದ ಚಿಹ್ನೆ ಇದು. ಯಾವಾಗ ಒಬ್ಬನಲ್ಲಿ ಆತ್ಮನಿಂದೆ ತಲೆದೋರುವುದೊ ಅವನು ಅವನತಿಯ ಪರಮಾವಧಿಯನ್ನು ಮುಟ್ಟಿರುವನು. ಯಾವಾಗ ಒಬ್ಬನಿಗೆ ತನ್ನ ಪೂರ್ವಿಕರ ವಿಷಯದಲ್ಲಿ ನಾಚಿಕೆ ಆಗುವುದೊ ಆಗ ಅವನ ಅಂತ್ಯ ಸಮೀಪಿಸಿದೆ ಎಂದು ತಿಳಿಯಬೇಕು. ಭಾರತೀಯರ ಜೀವನದೃಷ್ಟಿಯಿಂದ ಅತ್ತ ಸರಿಯದಿರಿ. ಇಲ್ಲಿರುವ ಹಿಂದುಗಳೆಲ್ಲ ಮತ್ತೊಬ್ಬರಂತೆಯೇ ವೇಷಭೂಷಣಗಳನ್ನು ಹಾಕಿಕೊಂಡು ಅವ ರಂತೆ ಮೆರೆದರೆ ಇದರಿಂದ ಭರತಖಂಡ ಉದ್ಧಾರವಾಗುವುದು ಎಂದು ಕ್ಷಣ ಕಾಲವೂ ಭಾವಿಸಬೇಡಿ.

ಮೇಲಿನವನಿಗೆ ವಿರೋಧವಾಗಿ ಮತ್ತೊಬ್ಬ ವಿದ್ಯಾವಂತನಿರುವನು. ಅವನೊಬ್ಬ ಹುಚ್ಚನಂತೆ. ಅವನು ಮತ್ತೊಂದು ಅತಿರೇಕಕ್ಕೆ ಹೋಗಿ ಪ್ರತಿಯೊಂದು ಕೆಲಸಕ್ಕೆ ಬಾರದ ಘಟನೆಗಳಿಗೆ ಒಂದು ವಿವರಣೆಯನ್ನು ಕೊಡಲು ಪ್ರಯತ್ನಿಸುವನು. ಈ ಜನಾಂಗಕ್ಕೆ ಸೇರಿದ ಪ್ರತಿಯೊಂದು ವಿಚಿತ್ರವಾದ ಮೂಢನಂಬಿಕೆಗೂ ವಿಚಿತ್ರವಾದ ದೇವರುಗಳಿಗೂ ಪ್ರತಿಯೊಂದು ಗ್ರಾಮದ ಮೂಢಾಚಾರಕ್ಕೂ ಒಂದು ತಾತ್ತ್ವಿಕ ವಾದ ವಿವರಣೆಯನ್ನು ಕೊಡಬಯಸುವನು. ದೇವರಿಗೇ ಗೊತ್ತು ಇನ್ನೂ ಏನೇನು ಅವನ ಮನಸ್ಸಿನಲ್ಲಿದೆಯೊ! ಅವನಿಗೆ ಪ್ರತಿಯೊಂದು ಗ್ರಾಮದ ಮೂಢ ನಡ ವಳಿಕೆಯೂ ಒಂದು ದೇವರ ಸಂದೇಶ, ಇದನ್ನು ನೆರವೇರಿಸುವುದರ ಮೇಲೆಯೇ ಜನಾಂಗದ ಭವಿಷ್ಯ ನಿಂತಿದೆ ಎಂದು ಅವನು ಭಾವಿಸುವನು. ನೀವು ಇವುಗಳ ವಿಷಯದಲ್ಲಿ ಬಹಳ ಜೋಪಾನವಾಗಿರಬೇಕು.

ಸತ್ಯಾಂಶವೇನೆಂದರೆ ನಮ್ಮಲ್ಲಿ ಹಲವು ಮೂಢಾಚಾರಗಳಿವೆ. ಹಲವು ಗಾಯ ಗಳು, ಕುರುಗಳು ಇವೆ. ಇವುಗಳನ್ನು ನಿರ್ಮೂಲ ಮಾಡಬೇಕು. ಹೆಸರಿಲ್ಲದಂತೆ ಮಾಡಬೇಕು. ಆದರೆ ಇವು ನಮ್ಮ ಧರ್ಮವನ್ನು, ಜನಾಂಗದ ಜೀವನವನ್ನು ಅಥವಾ ನಮ್ಮ ಆಧ್ಯಾತ್ಮಿಕತೆಯನ್ನು ಹಾಳುಮಾಡಬಾರದು. ಧರ್ಮದ ಮೂಲ ನಿಯಮ ಗಳೆಲ್ಲ ಭದ್ರವಾಗಿವೆ. ಈ ಕೆಲಸಕ್ಕೆ ಬಾರದ ಕಳೆಗಳನ್ನು ಎಷ್ಟು ಬೇಗ ನಿರ್ಮೂಲ ಮಾಡಿದರೆ ಅಷ್ಟು ಬೇಗ ಮೂಲ ಸತ್ಯಗಳು ಚೆನ್ನಾಗಿ ಬೆಳಕಿಗೆ ಬರುತ್ತವೆ. ಆ ಮೂಲ ಸತ್ಯಗಳನ್ನು ಹಿಡಿದುಕೊಳ್ಳಿ.

\textbf{ಶಾರೀರಿಕ ದೌರ್ಬಲ್ಯ}: ನಮ್ಮ ಉಪನಿಷತ್ತು ಎಷ್ಟು ಅಮೋಘವಾಗಿದ್ದರೂ, ನಾವು ಪುರಾತನ ಆರ್ಯಮಹರ್ಷಿಗಳ ಸಂಭೂತರೆಂದು ಎಷ್ಟು ಹೆಮ್ಮೆ ಕೊಚ್ಚಿ ಕೊಂಡರೂ, ನಮ್ಮನ್ನು ಇತರ ಜನಾಂಗಗಳೊಂದಿಗೆ ಹೋಲಿಸಿ ನೋಡಿದರೆ ನಾವು ತುಂಬಾ ದುರ್ಬಲರಾಗಿರುವೆವು ಎಂಬುದನ್ನು ನಿಮಗೆ ಹೇಳಬೇಕಾಗಿದೆ. ಮೊದಲ ನೆಯ ಕಾರಣವೇ ಶಾರೀರಿಕ ದೌರ್ಬಲ್ಯವಾಗಿದೆ.

ನಮ್ಮ ಯುವಕರು ಬಲಶಾಲಿಗಳಾಗಬೇಕು. ಧರ್ಮ ಅನಂತರ ಬರುವುದು. ನನ್ನ ಯುವಕರೆ, ಬಲಾಢ್ಯರಾಗಿ. ಅದೇ ನಾನು ನಿಮಗೆ ಕೊಡುವ ಸಲಹೆ. ನೀವು ಚೆಂಡಾಟ ವಾಡುವುದರಿಂದ ಗೀತೆಯನ್ನು ಓದುವುದಕ್ಕಿಂತ ಹೆಚ್ಚಾಗಿ ಮುಕ್ತಿಗೆ ಸಮೀಪಿಸುವಿರಿ. ನಿಮ್ಮ ಭುಜಬಲ ಮಾಂಸಖಂಡ ಬಲವಾಗಿದ್ದರೆ ಭಗವದ್ಗೀತೆಯನ್ನು ಅಷ್ಟು ಚೆನ್ನಾಗಿ ಅರ್ಥಮಾಡಿಕೊಳ್ಳಬಲ್ಲಿರಿ. ನಿಮ್ಮ ನಾಡಿಯಲ್ಲಿ ಬಲವಾದ ಬಿಸಿರಕ್ತ ಹರಿಯುತ್ತಿದ್ದರೆ ಶ್ರೀಕೃಷ್ಣನ ಅಪೂರ್ವ ಮಹಿಮೆ ಮತ್ತು ಪೌರುಷವನ್ನು ಚೆನ್ನಾಗಿ ಗ್ರಹಿಸಬಲ್ಲಿರಿ. ನೀವು ಧೈರ್ಯವಾಗಿ ನಿಮ್ಮ ಕಾಲಮೇಲೆ ನಿಲ್ಲುವುದನ್ನು ಕಲಿತರೆ, ನೀವು ಕೂಡ ಪುರುಷರು ಎಂಬ ಧೈರ್ಯ ಬಂದರೆ, ನೀವು ಉಪನಿಷತ್ತನ್ನು, ಆತ್ಮನ ಮಹಿಮೆಯನ್ನು ಚೆನ್ನಾಗಿ ಅರ್ಥಮಾಡಿಕೊಳ್ಳಬಲ್ಲಿರಿ. ನನಗೆ ಕಬ್ಬಿಣದಂತಹ ಮಾಂಸಖಂಡಗಳು ಬೇಕು. ಉಕ್ಕಿನಂತಹ ನರಗಳು ಬೇಕು. ಅದರ ಹಿಂದೆ ಸಿಡಿಲಿ ನಂತಿರುವ ಮನಸ್ಸಿರಬೇಕು. ನಮಗಿಂದು ಬೇಕಾಗಿರುವುದು ಬಲ, ಪೌರುಷ ಕ್ಷಾತ್ರವೀರ್ಯ ಮತ್ತು ಬ್ರಹ್ಮತೇಜಸ್ಸು.

\textbf{ನಮ್ಮಲ್ಲಿರುವ ಆತ್ಮಶ್ರದ್ಧೆಯ ಅಭಾವ}: ನಾವು ಆತ್ಮಶ್ರದ್ಧೆಯನ್ನು ಕಳೆದು ಕೊಂಡಿರುವೆವು. ಆಂಗ್ಲೇಯ ಮಹಿಳೆ ಮಹನೀಯರಲ್ಲಿ ಇರುವಷ್ಟು ಶ್ರದ್ಧೆ ಕೂಡ ನಮ್ಮಲ್ಲಿ ಇಲ್ಲ. ಅವರಲ್ಲಿರುವ ಶ್ರದ್ಧೆಯಲ್ಲಿ ಸಾವಿರದ ಒಂದು ಪಾಲು ಕೂಡ ನಮ್ಮಲ್ಲಿ ಇಲ್ಲ ಎಂದರೆ ನೀವು ನಂಬುವಿರೇನು? ಮೂವತ್ತೈದುಕೋಟಿ ಜನರನ್ನು ಸಾವಿರ ವರುಷಗಳಿಂದಲೂ ಹೊರಗಿನಿಂದ ಬಂದ ಪರದೇಶಿಯರು ತುಳಿದು ಆಳು ತ್ತಿರುವರು! ಏಕೆಂದರೆ ಅವರಿಗೆ ಆತ್ಮಶ್ರದ್ಧೆ ಇತ್ತು. ನಮ್ಮಲ್ಲಿ ಇಲ್ಲ. ನಮಗೆ ಇಂದು ನಮ್ಮಲ್ಲಿ ಒಂದು ಶ್ರದ್ಧೆ ಬೇಕು. ಬಲವೇ ಜೀವನ, ದೌರ್ಬಲ್ಯವೇ ಮರಣ.

ನಾವು ಹೊರಗಿನವರಿಗೆ ಸೋತಿರುವುದರಿಂದ, ನಾವು ದುರ್ಬಲರು. ನಮಗೆ ಯಾವುದಕ್ಕೂ ಸ್ವಾತಂತ್ರ್ಯವಿಲ್ಲ ಎಂದು ಭಾವಿಸುವೆವು. ಇದರಿಂದಲೇ ನಾವು ಶ್ರದ್ಧೆಯನ್ನು ಕಳೆದುಕೊಂಡು ಇರುವುದು. ನಿಜವಾದ ಆತ್ಮಶ್ರದ್ಧೆಯನ್ನು ನಮ್ಮಲ್ಲಿ ಉಜ್ಜೀವನಗೊಳಿಸಬೇಕಾಗಿದೆ. ಆತ್ಮಶ್ರದ್ಧೆಯ ಹಣತೆಯನ್ನು ಹಚ್ಚಬೇಕಾಗಿದೆ. ಆಗ ಮಾತ್ರ ನಮ್ಮ ದೇಶವನ್ನು ಆವರಿಸಿರುವ ಸಮಸ್ಯೆಗಳನ್ನು ನಾವೇ ಬಗೆಹರಿಸಲು ಸಾಧ್ಯವಾಗುವುದು.

ನಿಮ್ಮಲ್ಲಿ ಶ್ರದ್ಧೆ ಇರಲಿ. ನೀವು ಆ ಶ್ರದ್ಧೆಯ ಮೇಲೆ ನಿಂತು ಧೀರರಾಗಿ. ನಮಗೆ ಬೇಕಾಗಿರುವುದು ಅದೇ. ಆತ್ಮ ನಿತ್ಯಮುಕ್ತ, ಅಮೃತಸ್ವರೂಪ. ಸ್ವಭಾವತಃ ನಾವು ಪರಿಶುದ್ಧರು. ಇಂತಹ ಶ್ರದ್ಧೆ ನಮಗೆ ಬೇಕು. ಇಂತಹ ಶ್ರದ್ಧೆ ನಮ್ಮನ್ನು ಪುರುಷ ರನ್ನಾಗಿ ಮಾಡುವುದು, ದೇವತೆಗಳನ್ನಾಗಿ ಮಾಡುವುದು. ನೀವು ಮಹಾಕಾರ್ಯಗಳನ್ನು ಮಾಡುವುದಕ್ಕೆ ಜನ್ಮವೆತ್ತಿರುವಿರಿ ಎಂಬ ಶ್ರದ್ಧೆ ಇರಲಿ. ಪ್ರತಿಯೊಬ್ಬನೂ, ಇತರ ರೆಲ್ಲ ತಮ್ಮ ಪಾಲಿನ ಕರ್ತವ್ಯವನ್ನು ಮಾಡಿರುವರು, ನಾವು ಮಾತ್ರ ನಮ್ಮ ಪಾಲಿನ ದನ್ನು ಅರ್ಪಣೆ ಮಾಡಬೇಕು ಎಂದು ಭಾವಿಸಬೇಕು. ಈ ಜವಾಬ್ದಾರಿಯನ್ನು ನಾವೇ ವಹಿಸಿಕೊಳ್ಳಬೇಕಾಗಿದೆ. ನಮ್ಮ ಜನಾಂಗದ ರಕ್ತಕ್ಕೆ ಪ್ರವೇಶಿಸುತ್ತಿರುವ, ಪ್ರತಿ ಯೊಂದನ್ನೂ ಅಣಕಿಸುವ, ಪ್ರತಿಯೊಂದನ್ನೂ ಲಘುವಾಗಿ ಕಾಣುವ ಸ್ವಭಾವ ದಿಂದ ಪಾರಾಗಿ, ತ್ಯಜಿಸಿ ಧೀರರಾಗಿ ಶ್ರದ್ಧಾವಂತರಾಗಿ. ಅನಂತರ ಉಳಿದವುಗಳೆಲ್ಲ ಅದನ್ನು ಅನುಸರಿಸಿ ಬರುವುವು.

\textbf{ನಮ್ಮಲ್ಲಿರುವ ಸ್ವಪ್ರಯತ್ನದ ಅಭಾವ}: ನಮ್ಮ ಜನಾಂಗದ ಸ್ವಭಾವವೇ ಅನ್ಯ ರನ್ನು ಮಗುವಿನಂತೆ ಆಶ್ರಯಿಸುವುದಾಗಿದೆ. ತಮ್ಮ ಬಾಯಿಯ ಸಮೀಪಕ್ಕೆ ಯಾರಾ ದರೂ ತಿಂಡಿಯನ್ನು ತಂದು ಹಿಡಿದರೆ ಆಗ ಎಲ್ಲರೂ ತಿನ್ನುವುದಕ್ಕೆ ಸಿದ್ಧರಾಗಿರು ವರು. ಕೆಲವರು ಅದನ್ನು ಬಾಯಿಯ ಒಳಗೂ ತುರುಕಬೇಕೆಂದು ಹೇಳುವರು. ಸ್ವ ಪ್ರಯತ್ನವಿಲ್ಲದೇ ಇದ್ದರೆ ನೀವು ಬಾಳುವುದಕ್ಕೆ ಯೋಗ್ಯರಲ್ಲ. ಪ್ರತಿಯೊಂದು ದೇಶವೂ ತನ್ನಿಂದ ತಾನೇ ಉದ್ಧಾರ ಆಗಬೇಕಾಗಿದೆ ಎಂಬುದನ್ನು ನೀವುಗಳೆಲ್ಲ ತಿಳಿದುಕೊಳ್ಳಬೇಕು. ಅದರಂತೆಯೇ ಪ್ರತಿಯೊಂದು ವ್ಯಕ್ತಿ ಕೂಡ. ಇನ್ನೊಬಪ್ರ ಸಹಾಯಕ್ಕೆ ನೆಚ್ಚಿಕೊಳ್ಳಬೇಡಿ. ಹೊರಗಿನವರನ್ನು ನಮ್ಮ ಸಹಾಯಕ್ಕೆ ಕಾಯಬೇಡಿ. ವ್ಯಕ್ತಿಗಳಂತೆ ದೇಶಗಳು ಕೂಡ ತಮ್ಮನ್ನು ತಾವೇ ಉದ್ಧಾರ ಮಾಡಿಕೊಳ್ಳಬೇಕಾಗಿದೆ. ಇದೇ ನಿಜವಾದ ದೇಶಭಕ್ತಿ. ದೇಶ ಇದನ್ನು ಮಾಡದೇ ಇದ್ದರೆ, ಇನ್ನೂ ಅದರ ವಿಮೋಚನೆಯ ಕಾಲ ಬಂದಿಲ್ಲ, ಅದಿನ್ನೂ ಕಾಯಬೇಕಾಗಿದೆ.

\textbf{ವಿಧೇಯತೆಯ ಅಭಾವ}: ಪ್ರತಿಯೊಬಪ್ನೂ ಅಪ್ಪಣೆ ಮಾಡುವವನೇ ಆಗಿದ್ದಾನೆ. ಯಾರೂ ಆಣತಿಯನ್ನು ಪರಿಪಾಲಿಸುವುದಿಲ್ಲ. ಹಿಂದಿನ ಕಾಲದ ಬ್ರಹ್ಮಚರ್ಯ ಆಶ್ರಮದ ಅಭಾವವೇ ಇದಕ್ಕೆಲ್ಲ ಕಾರಣ. ಮೊದಲು ಆಣತಿಯನ್ನು ಪಾಲಿಸುವು ದನ್ನು ಕಲಿಯಿರಿ. ಅನಂತರ ಅಪ್ಪಣೆ ಕೊಡುವ ಅವಕಾಶ ಬರುವುದು. ಮೊದಲು ಯಾವಾಗಲೂ ಆಳಾಗುವುದನ್ನು ಕಲಿಯಿರಿ. ಅನಂತರ ನೀವು ಅರಸರಾಗಬಹುದು. ನಿಮ್ಮ ಹಿರಿಯರು ನಿಮಗೆ ನದಿಗೆ ಬಿದ್ದು ಮೊಸಳೆ ಹಿಡಿ ಎಂದರೆ, ಮೊದಲು ಮಾಡಿ ಅನಂತರ ಪ್ರಶ್ನೆಯನ್ನು ಕೇಳಿ. ಅಪ್ಪಣೆ ತಪ್ಪಾಗಿದ್ದರೂ ಮೊದಲು ಅದನ್ನು ಪಾಲಿಸಿ ಅನಂತರ ಅದನ್ನು ಪ್ರಶ್ನಿಸಿ. ವಿಧೇಯತೆಯ ಗುಣವನ್ನು ಅಭ್ಯಾಸ ಮಾಡಿ. ಆದರೆ ಅದಕ್ಕೆ ನಿಮ್ಮ ಶ್ರದ್ಧೆಯನ್ನು ಬಲಿಕೊಡಬಾರದು. ಹಿರಿಯರಿಗೆ ವಿಧೇಯತೆಯನ್ನು ತೋರದೆ ಇದ್ದರೆ ಯಾವ ಶಕ್ತಿಯೂ ಕೇಂದ್ರೀಕೃತವಾಗುವುದಿಲ್ಲ. ವ್ಯಕ್ತಿಯ ಶಕ್ತಿ ಗಳೆಲ್ಲ ಕೇಂದ್ರೀಕೃತವಾಗುವವರೆಗೂ ಯಾವ ಮಹತ್ಕಾರ್ಯವೂ ಸಾಧ್ಯವಾಗುವು ದಿಲ್ಲ. ಸೋಮಾರಿತನ ಸ್ವಾರ್ಥತೆ ಅಸೂಯೆ ಇವೇ ನಮ್ಮಲ್ಲಿರುವುದು. ನಮ್ಮಲ್ಲಿ ಬರೀ ಮಾತು, ಮಾತು, ಮಾತು. ನಾವೆಲ್ಲ ತುಂಬ ದೊಡ್ಡವರು ಎಂದು ಭಾವಿಸು ತ್ತೇವೆ. ಇದಕ್ಕೆಲ್ಲ ಅರ್ಥವಿಲ್ಲ. ನಾವು ಕೆಲಸಕ್ಕೆ ಬಾರದವರು. ನಾವು ನಿಜವಾಗಿ ಅದೇ ಆಗಿರುವುದು. ನಾವು ಅರಗಿಳಿಯಂತೆ ಏನೇನೋ ಮಾತಾಡುವೆವು. ಆದರೆ ಕಾರ್ಯತಃ ಏನೂ ಇಲ್ಲ. ಸುಮ್ಮನೆ ಮಾತು, ಕೆಲಸವಿಲ್ಲ, ಇದೇ ನಮ್ಮ ಸ್ವಭಾವ ಆಗಿರುವುದು. ಇಂತಹ ದುರ್ಬಲವಾದ ಬುದ್ಧಿ ಏನನ್ನೂ ಸಾಧಿಸಲಾರದು. ನಾವು ನಮ್ಮ ಬುದ್ಧಿ ಯನ್ನು ಬಲಗೊಳಿಸಿಕೊಳ್ಳಬೇಕು.

ನಾವು ಬರೀ ಸೋಮಾರಿಗಳು. ನಮ್ಮ ಕೈಯಲ್ಲಿ ಕೆಲಸ ಸಾಧ್ಯವಿಲ್ಲ. ನಾವು ಮತ್ತೊಬ್ಬರೊಡನೆ ಸಹಕರಿಸಲಾರೆವು. ನಾವು ಮತ್ತೊಬ್ಬರನ್ನು ಪ್ರೀತಿಸಲಾರೆವು. ಮೂರು ಜನ ಕೂಡ ಒಬ್ಬರು ಮತ್ತೊಬ್ಬರನ್ನು ದ್ವೇಷಿಸದೆ ಒಟ್ಟಿಗೆ ಕಲೆಯಲಾರೆವು. ನಾವಿರುವ ನಿಜಸ್ಥಿತಿಯೇ ಇದು. ಸ್ವಲ್ಪವಾದರೂ ವ್ಯವಸ್ಥೆ ಅರಿಯದ ದೊಂಬಿಯ ಜನ, ಬರೀ ಸ್ವಾರ್ಥಪರರು. ಹಣೆಯ ಮೇಲೆ ಮತದ ಚಿಹ್ನೆಯನ್ನು ಹೀಗೆ ಇಡು ವುದೇ ಹಾಗೆ ಇಡುವುದೇ ಎಂದು ನೂರಾರು ವರುಷಗಳಿಂದ ನಮ್ಮ ನಮ್ಮಲ್ಲಿಯೇ ಹೋರಾಡುತ್ತಿರುವೆವು. ನಾನು ಊಟ ಮಾಡುತ್ತಿರುವಾಗ ಮತ್ತೊಬ್ಬ ನೋಡಿದರೆ ಅದು ಕೆಡುವುದೆ ಇಲ್ಲವೆ ಎಂಬ ಗಹನವಾದ ಪ್ರಶ್ನೆಗಳ ಮೇಲೆ ಬೇಕಾದಷ್ಟು ಗ್ರಂಥ ಗಳನ್ನು ಬರೆದಿರುವೆವು.

\textbf{ನಮ್ಮಲ್ಲಿರುವ ವ್ಯವಸ್ಥೆಯ ಅಭಾವ}: ಒಂದು ವ್ಯವಸ್ಥಿತ ಬಾಳ್ವೆ ನಮ್ಮ ಜೀವನಕ್ಕೆ ಹೊಸದು. ಆದರೆ ಇದನ್ನು ನಮ್ಮ ಜೀವನದಲ್ಲಿ ತರಬೇಕು. ಅಸೂಯೆಯ ಅಭಾವವೇ ಇದರ ರಹಸ್ಯ. ನಿನ್ನ ಸಹೋದರರ ಅಭಿಪ್ರಾಯಕ್ಕೆ ಮನ್ನಣೆಯನ್ನು ಕೊಡು. ಯಾವಾಗಲೂ ಸೌಹಾರ್ದ ಬಾಳುವೆಗೆ ಪ್ರಯತ್ನಪಡು. ಸಂಸ್ಥೆಗಳಲ್ಲಿ ಏತಕ್ಕೆ ಅಷ್ಟೊಂದು ಶಕ್ತಿ ಇದೆ ಗೊತ್ತೆ? ಒಂದು ಸ್ಥೂಲವಾಗಿರುವ ಉದಾಹರಣೆ ಯನ್ನು ತೆಗೆದುಕೊಂಡರೆ, ನಾಲ್ಕು ಕೋಟಿ ಇಂಗ್ಲೀಷರು ಮೂವತ್ತು ಕೋಟಿ ಭಾರತೀಯರನ್ನು ಹೇಗೆ ಆಳಲು ಸಾಧ್ಯ? ಇದಕ್ಕೆ ಮಾನಸಿಕವಾಗಿ ಏನು ಕಾರಣ? ನಾಲ್ಕುಕೋಟಿ ಆಂಗ್ಲೇಯರು ಒಂದು ರೀತಿ ಆಲೋಚಿಸುವರು. ಇದರಿಂದ ಅದ್ಭುತವಾದ ಶಕ್ತಿ ಉತ್ಪನ್ನವಾಗುವುದು. ನಿಮ್ಮ ಮೂವತ್ತುಕೋಟಿ ಜನರಾದರೊ ಒಬೊಪ್ಬಪ್ನೂ ಒಂದೊಂದು ರೀತಿ ಆಲೋಚಿಸುತ್ತಾನೆ. ನಾವೊಂದು ಮಹಿಮಾ ಮಯವಾದ ಭವಿಷ್ಯ ಭರತಖಂಡವನ್ನು ಸೃಷ್ಟಿಸಬೇಕಾದರೆ ಇದರ ರಹಸ್ಯವೆಲ್ಲ ಸಂಸ್ಥಾಬದ್ಧ ಜೀವನದಲ್ಲಿದೆ, ಶಕ್ತಿ ಸಂಗ್ರಹದಲ್ಲಿದೆ, ನಾವೆಲ್ಲರೂ ಒಮ್ಮತದವ ರಾಗುವುದರಲ್ಲಿದೆ. ನನ್ನ ಮನಸ್ಸಿನಲ್ಲಿ ಅಥರ್ವವೇದದ ಮುಂದಿನ ಶ್ಲೋಕ ಜ್ಞಾಪಕಕ್ಕೆ ಬರುವುದು: "ನೀವೆಲ್ಲರೂ ಒಂದೇ ಮನಸ್ಸಿನವರಾಗಿ, ಒಂದೇ ಆಲೋ ಚನೆ ಮಾಡುವವರಾಗಿ. ಹಿಂದೆ ದೇವತೆಗಳೆಲ್ಲ ಒಮ್ಮತದವರಾದುದರಿಂದ ಹವಿ ಸ್ಸನ್ನು ಸ್ವೀಕರಿಸಲು ಸಾಧ್ಯವಾಯಿತು." ಆ ದೇವತೆಗಳನ್ನು ಮನುಷ್ಯನು ಪೂಜಿಸು ವುದಕ್ಕೆ ಸಾಧ್ಯವಾದುದು ಮನುಷ್ಯರೆಲ್ಲ ಒಂದೇ ರೀತಿ ಆಲೋಚಿಸಿದುದರಿಂದ. ಒಂದು ಸಮಾಜದಲ್ಲಿರುವವರೆಲ್ಲ ಒಂದು ರೀತಿ ಆಲೋಚಿಸುವುದರಲ್ಲಿಯೇ ಆ ಸಮಾಜದ ಶಕ್ತಿ ರಹಸ್ಯವಿರುವುದು. ನೀವು ನಿಮ್ಮ ನಿಮ್ಮಲ್ಲಿ ಬ್ರಾಹ್ಮಣ ಬ್ರಾಹ್ಮ ಣೇತರ ದ್ರಾವಿಡ ಆರ್ಯ ಮುಂತಾದ ಕೆಲಸಕ್ಕೆ ಬಾರದ ವಿಷಯಗಳನ್ನು ಕುರಿತು ಕಾದಾಡುವವರೆಗೆ ನೀವು ಭವಿಷ್ಯ ಭಾರತ ನಿರ್ಮಾಣಕ್ಕೆ ಬೇಕಾಗುವ ಶಕ್ತಿ ಸಂಚಯ ವನ್ನು ಮಾಡಲು ಆಗುವುದಿಲ್ಲ. ಇದನ್ನು ಗಮನಿಸಿ. ಭವಿಷ್ಯ ಭರತಖಂಡ ಇದರ ಮೇಲೆ ಮಾತ್ರ ನಿಂತಿರುವುದು. ಇದೇ ಇಚಾಊಇ್ಕ86ದ್ಶಕ್ತಿಯ ಸಂಗ್ರಹ, ಪರಸ್ಪರ ಸೌಹಾರ್ದ. ಎಲ್ಲವನ್ನು ಒಂದುಗೂಡಿಸುವ ರಹಸ್ಯ ಇಲ್ಲಿದೆ.

\textbf{ನಮ್ಮಲ್ಲಿ ವ್ಯವಹಾರ ಪುಜುತ್ವವಿಲ್ಲ}: ವ್ಯಾವಹಾರಿಕ ವಿಷಯಗಳಲ್ಲಿ ಹಿಂದೂ ಗಳು ಶುದ್ಧ ಸೋಮಾರಿಗಳು. ಏಕೆಂದರೆ ಅವರ ಕಾರ್ಯದಲ್ಲಿ ಒಂದು ಶಕ್ತಿ ಇಲ್ಲ. ಲೆಕ್ಕಾಚಾರಗಳನ್ನು ಸರಿಯಾಗಿ ಇಡಲಾರರು. ನಮ್ಮಗಳೆಲ್ಲರ ಪ್ರಯತ್ನವೂ ಒಂದು ಗುಣದ ಅಭಾವದಿಂದ ವ್ಯರ್ಥವಾಗುವುದು. ಅದೇ ನಾವು ವ್ಯವಹಾರದಲ್ಲಿ ಒಂದು ಪುಜುಮಾರ್ಗವನ್ನು ಅಭ್ಯಾಸಮಾಡದೆ ಇರುವುದು. ವ್ಯವಹಾರವು ವ್ಯವಹಾರವಾ ಗಿರಬೇಕು. ಅಲ್ಲಿ ನಮ್ಮವನು ನಿಮ್ಮವನು ಎಂಬ ಭಾವನೆ ಇರಕೂಡದು. ಹಿಂದೂ ಗಳು ಹೇಳುವಂತೆ ಇಲ್ಲಿ ವಂಚನೆ ಇರಕೂಡದು. ಒಬಪ್ನು ತನ್ನ ವಶದಲ್ಲಿರುವ ಎಲ್ಲದರ ಲೆಕ್ಕಾಚಾರವನ್ನು ಸರಿಯಾಗಿಡಬೇಕು. ಒಂದಕ್ಕೆ ಮೀಸಲಾಗಿಟ್ಟಿರುವ ದುಡ್ಡನ್ನು ಮತ್ತೊಂದಕ್ಕೆ ತಾನು ಉಪವಾಸವಿದ್ದರೂ ಚಿಂತೆಯಿಲ್ಲ ಉಪಯೋಗಿಸ ಕೂಡದು. ಇದೇ ವ್ಯವಹಾರ ಪುಜುತ್ವ. ಅನಂತರವೆ, ನಮ್ಮಲ್ಲಿ ಒಂದು ಕೆಲಸ ಮಾಡುವುದಕ್ಕೆ ಅದ್ಭುತವಾದ ಉತ್ಸಾಹವಿರಬೇಕು. ನೀವು ಏನು ಮಾಡಿದರೂ ತತ್ಕಾಲಕ್ಕೆ ನಿಮಗೆ ಅದು ಪೂಜೆಯಂತಿರಲಿ.

\textbf{ನಮ್ಮಲ್ಲಿ ಪ್ರೀತಿಯಿಲ್ಲ}: ಯಾವ ವ್ಯಕ್ತಿಯೇ ಆಗಲಿ, ಜನಾಂಗವೇ ಆಗಲಿ ಇನ್ನೊಬಪ್ನನ್ನು ದ್ವೇಷಿಸಿ ತಾನು ಬಾಳಲಾರದು. ಭರತಖಂಡ ಎಂದು ಮ್ಲೇಚಊಇ್ಕ86ದ್ ಎಂಬ ಪದವನ್ನು ಉಚ್ಚರಿಸಿ ಇತರರೊಡನೆ ಬೆರೆಯುವುದನ್ನು ತ್ಯಜಿಸಿತೋ, ಅಂದಿ ನಿಂದಲೇ ನಮ್ಮ ದೇಶದ ದುರದೃಷ್ಟ ಪ್ರಾರಂಭವಾಯಿತು. ಪ್ರೀತಿ ಎಂದಿಗೂ ನಿರರ್ಥಕವಾಗುವುದಿಲ್ಲ. ಇಂದೋ ನಾಳೆಯೋ ಅಥವಾ ಹಲವು ಶತಮಾನಗಳಾದ ಮೇಲೆಯೋ ಪ್ರೀತಿ ಜಯಿಸಲೇಬೇಕು. ಪ್ರೀತಿಗೆ ಜಯ ಸಿಕ್ಕೇ ಸಿಕ್ಕುವುದು. ನೀವು ನಿಮ್ಮ ದೇಶದವರನ್ನು ಪ್ರೀತಿಸುವಿರೇನು?

\textbf{ದೇಶಸೇವಕರ ಕರ್ತವ್ಯ}: ನೀವು ದೇವರನ್ನು ಹುಡುಕಿಕೊಂಡು ಎಲ್ಲಿ ಹೋಗುವಿರಿ? ದೀನರು ದರಿದ್ರರು ದುರ್ಬಲರು ಇವರೆಲ್ಲ ದೇವರಲ್ಲವೆ? ಮೊದಲು ಏತಕ್ಕೆ ನೀವು ಅವರನ್ನು ಪೂಜಿಸಬಾರದು? ಗಂಗಾನದೀ ತೀರದಲ್ಲಿ ಏತಕ್ಕೆ ಒಂದು ಬಾವಿಯನ್ನು ತೋಡಲು ಹೋಗುತ್ತೀರಿ? ಪ್ರೇಮದ ಅದ್ಭುತವಾದ ಶಕ್ತಿಯಲ್ಲಿ ನಂಬಿ. ನಿಮ್ಮಲ್ಲಿ ಪ್ರೀತಿ ಇದೆಯೆ? ಇದ್ದರೆ ನೀವು ಸರ್ವಶಕ್ತರು. ನೀವು ಸಂಪೂರ್ಣವಾಗಿ ನಿಃಸ್ವಾರ್ಥಿಗಳೆ? ಹಾಗಿದ್ದರೆ ನಿಮ್ಮನ್ನು ಯಾರೂ ತಡೆಯುವವರಿಲ್ಲ. ಶುದ್ಧ ಚಾರಿತ್ರ್ಯದವನೇ ಎಲ್ಲ ಕಡೆಯೂ ಗೆಲ್ಲಬೇಕಾದರೆ. ಸಮುದ್ರದ ಆಳದಲ್ಲಿಯೂ ದೇವರು ತನ್ನ ಮಕ್ಕಳನ್ನು ಸಂರಕ್ಷಿಸುವನು. ನಮ್ಮ ದೇಶಕ್ಕೆ ಮಹಾವೀರರು ಬೇಕಾಗಿದ್ದರೆ, ವೀರರಾಗಿ ನೀವು.?

ಹೃತ್ಪೂರ್ವಕವಾಗಿ ಮತ್ತೊಬ್ಬನಿಗೆ ಮರುಕಪಡಿ. ತರ್ಕ ವಿಚಾರ ಇವುಗಳಲ್ಲೆಲ್ಲ ಏನು ಇದೆ? ಅದು ಸ್ವಲ್ಪ ದೂರ ಹೋಗಿ ನಿಲ್ಲುವುದು. ಆದರೆ ಸ್ಫೂರ್ತಿ ಬರುವುದು ಹೃದಯದ ಅಂತರಾಳದಿಂದ. ಪ್ರೀತಿ ತೆರೆಯದ ಬಾಗಿಲೇ ಇಲ್ಲ. ವಿಶ್ವರಹಸ್ಯವನ್ನು ತಿಳಿದುಕೊಳ್ಳಲು ಪ್ರೀತಿಯೇ ಮೂಲ. ನಮ್ಮ ಭಾವಿ ಸುಧಾರಕರೆ, ದೇಶಭಕ್ತರೆ, ಮೊದಲು ನಿಮ್ಮಲ್ಲಿ ಅನುಕಂಪೆ ಇರಲಿ. ನಿಮ್ಮಲ್ಲಿ ಆ ಮರುಕವಿದೆಯೆ? ಕೋಟ್ಯಂ ತರ ದೇವತೆಗಳು, ಮಹಾಪುಷಿಗಳ ಕುಲಸಂಜಾತರು ಇಂದು ಮೃಗಸಮಾನರಾಗಿರು ವರೆಂದು ನಿಮಗೆ ತೋರುವುದೆ? ಇಂದು ಕೋಟ್ಯಂತರ ಜನ ಉಪವಾಸದಿಂದ ನರಳುತ್ತಿರುವರು, ಹಿಂದಿನಿಂದಲೂ ಅವರು ನರಳುತ್ತಿದ್ದರು ಎಂಬುದು ನಿಮ್ಮ ಎದೆಗೆ ತಾಕುವುದೇ? ಅಜ್ಞಾನವೆಂಬ ಕಾರ್ಮೋಡ ಭಾರತಾವನಿಯನ್ನೆಲ್ಲ ವ್ಯಾಪಿಸಿದೆ ಎಂಬುದು ನಿಮಗೆ ತೋರುವುದೆ? ಇದರಿಂದ ನಿಮಗೆ ವ್ಯಾಕುಲವಾಗಿದೆಯೆ? ಇದ ರಿಂದ ನಿಮಗೆ ನಿದ್ರಾಭಂಗ ಆಗಿದೆಯೆ? ಇದು ನಿಮ್ಮ ರಕ್ತಗತವಾಗಿ ನಾಡಿಯಲ್ಲಿ ಹೃದಯಸ್ಪಂದನದೊಂದಿಗೆ ಸ್ಪಂದಿಸುತ್ತಿದೆಯೆ? ಈ ವ್ಯಾಕುಲತೆ ನಿಮ್ಮನ್ನು ಹುಚ್ಚ ರನ್ನಾಗಿ ಮಾಡಿದೆಯೆ? ನಾವು ಎಂತಹ ದುರ್ದೆಸೆಗೆ ಬಂದಿರುವೆವು ಎಂಬುದನ್ನು ಗ್ರಹಿಸಿರುವಿರಾ? ಇದರಿಂದ ನಿಮ್ಮ ಕೀರ್ತಿ ಯಶಸ್ಸು ಹೆಂಡತಿ ಮಕ್ಕಳು ಆಸ್ತಿ ಐಶ್ವರ್ಯ ಎಲ್ಲವನ್ನೂ ಮರೆತಿರುವಿರಾ? ನೀವು ಇದನ್ನು ಮಾಡಿರುವಿರೇನು? ದೇಶಭಕ್ತ ನಾಗಬೇಕಾದರೆ ಇದೇ ಮೊದಲು ನಾವು ಮಾಡಬೇಕಾದ ಕೆಲಸ.

ನೀವು ವಾಸಮಾಡುತ್ತಿರುವ ಸಣ್ಣ ಬಿಲಗಳಿಂದ ಹೊರಗೆ ಬಂದು ಸುತ್ತಲೂ ಹೊರಗೆ ನೋಡಿ. ಹೇಗೆ ಹೊರಗೆ ಜನಾಂಗಗಳು ಮುಂದುವರಿಯುತ್ತಿವೆ ಎಂಬು ದನ್ನು ಗಮನಿಸಿ. ನೀವು ಮನುಷ್ಯನನ್ನು ಪ್ರೀತಿಸುವಿರೇನು? ದೇಶವನ್ನು ಪ್ರೀತಿಸು ವಿರೇನು? ಹಾಗಾದರೆ ಬನ್ನಿ. ಉತ್ತಮವಾದ ಆದರ್ಶಗಳಿಗೆ ಹೋರಾಡೋಣ. ನಿಮ್ಮ ಅತ್ಯಂತ ಪ್ರೀತಿಪಾತ್ರರ ಮತ್ತು ಅತಿ ನಿಕಟಬಂಧುಗಳು ಕರೆದರೂ ಹಿಂದಿರುಗಿ ನೋಡದಿರಿ. ಹಿಂದಿರುಗದಿರಿ, ಯಾವಾಗಲೂ ಮುಂದೆ, ಮುಂದೆ!

ಪುರುಷಸಿಂಹರು, ಪುರುಷಸಿಂಹರು, ನಮಗೆ ಬೇಕು. ನಮಗೆ ಇಂದು ಬೇಕಾ ಗಿರುವುದು ಆಶಿಷ್ಠ ಧ್ರಡಿಷ್ಠ ಬಲಿಷ್ಠ ಶ್ರದ್ಧಾವಂತರಾದ ಯುವಕರು. ಅನಂತರ ಉಳಿದವುಗಳೆಲ್ಲ ನಮಗೆ ಬರುವುದು. ಇಂತಹ ನೂರುಮಂದಿ ಸಿಕ್ಕಿದರೆ ಪ್ರಪಂಚ ವನ್ನೆಲ್ಲ ಬದಲಾಯಿಸಬಹುದು.

ಮುಖ್ಯವಾಗಿ ಬೇಕಾಗಿರುವುದು ತ್ಯಾಗ. ತ್ಯಾಗವಿಲ್ಲದೆ ಮತ್ತೊಬ್ಬರಿಗೆ ತನ್ನನ್ನು ಪೂರ್ಣ ಅರ್ಪಿಸಿಕೊಳ್ಳಲಾರ. ತ್ಯಾಗಿಗಳು ಎಲ್ಲರನ್ನೂ ಒಂದೇ ಸಮನಾಗಿ ನೋಡು ವರು. ಅವನು ಎಲ್ಲರ ಸೇವೆಗೂ ಬದ್ಧಕಂಕಣನಾಗಿರುವನು. ಸತ್ಯ ಪ್ರೀತಿ ಪುಜುತ್ವ ಇವುಗಳನ್ನು ಯಾವುದೂ ತಡೆಯಲಾರದು. ಪ್ರಾಣ ಹೋದರೂ ಚಿಂತೆಯಿಲ್ಲ ನೀನು ಪ್ರಾಮಾಣಿಕನೆ, ನಿಃಸ್ವಾರ್ಥಿಯೇ, ಇನ್ನೊಬ್ಬರನ್ನು ಪ್ರೀತಿಸುವೆಯಾ? ಆಗ ನೀನು ಮೃತ್ಯುವಿಗೂ ಅಂಜಬೇಕಾಗಿಲ್ಲ. ಅದಕ್ಕೆ ಸರಿಸಮನಾಗಿ ಜ್ಞಾನದಲ್ಲಿಯೂ ಮುಂದುವರಿಯುವೆ. ನನ್ನ ಧೀರ ಗಂಭೀರ ಸಾಧುತ್ವಭಾವದ ಆತ್ಮೀಯರೆ, ಈ ಭಾವನೆಯನ್ನು ಕಾರ್ಯಗತ ಮಾಡಿ ಕೆಲಸಕ್ಕೆ ಕೈಹಾಕಿ, ನಿಮ್ಮ ಸೊಂಟವನ್ನು ಕಟ್ಟಿ. ಹೆಸರು ಕೀರ್ತಿ ಮುಂತಾದ ಕೆಲಸಕ್ಕೆ ಬಾರದ ವಸ್ತುಗಳನ್ನು ನೋಡುವುದಕ್ಕೆ ಹಿಂತಿರುಗಬೇಡಿ. ನಿಮ್ಮ ಸ್ವಾರ್ಥವನ್ನು ತ್ಯಜಿಸಿ ಕಾರ್ಯೋನ್ಮುಖರಾಗಿ.

ಎಲ್ಲಾ ಸೋಮಾರಿತನದಿಂದ ಪಾರಾಗಿ. ಇಹಪರಭೋಗಗಳನ್ನೆಲ್ಲ ತ್ಯಜಿಸಿ. ಬೆಂಕಿ ಒಳಗೆ ಪ್ರವೇಶಿಸಿ ಜನರನ್ನು ದೇವರ ಕಡೆಗೆ ತನ್ನಿ. ನೀವು ಕೆಲಸಕ್ಕೆ ಕೈಹಾಕಿದರೆ ನಿಮ್ಮಲ್ಲಿ ಅಂತಹ ಶಕ್ತಿ ಸಂಚಾರವಾಗುವುದು. ಅದನ್ನು ಹೊರುವುದೇ ನಿಮಗೆ ಕಷ್ಟವಾಗುವುದು. ನಿಃಸ್ವಾರ್ಥತೆಯಿಂದ ನಾವು ಯಾವಾಗ ಇತರರಿಗೆ ಒಂದು ಸ್ವಲ್ಪ ಒಳ್ಳೆಯದನ್ನು ಮಾಡಿದರೂ ಅದರಿಂದ ನಮ್ಮ ಆಂತರ್ಯದಲ್ಲಿರುವ ಶಕ್ತಿ ಜಾಗ್ರತ ವಾಗುವುದು. ಇನ್ನೊಬ್ಬನಿಗೆ ಸ್ವಲ್ಪ ಒಳ್ಳೆಯದಾಗುವುದನ್ನು ಸ್ವಲ್ಪ ಆಲೋಚಿಸಿ ದರೂ ಅದರಿಂದ ಸಿಂಹಸದೃಶ ಶಕ್ತಿ ಅರಿವಾಗುವುದು. ನಾನು ನಿಮ್ಮನ್ನೆಲ್ಲ ಅಷ್ಟೊಂದು ಪ್ರೀತಿಸುತ್ತೇನೆ. ಆದರೂ ನೀವು ಇತರರ ಕಲ್ಯಾಣಕ್ಕೆ ದುಡಿದು ಸಾಯಿರಿ ಎಂದು ಹಾರೈಸುವೆನು. ನೀವು ಹಾಗೆ ಮಾಡುವುದನ್ನು ನೋಡಿದಾಗ ನನಗೆ ಸಂತೋಷವಾಗುವುದು.

ಸತ್ಯ ಅಸತ್ಯಕ್ಕಿಂತ ಕೋಟಿಪಾಲು ಮೇಲು. ಇದರಂತೆಯೇ ಸಾಧು ಸ್ವಭಾವ. ನಿಮ್ಮಲ್ಲಿ ಇವಿದ್ದರೆ ಇವುಗಳ ಸಹಾಯದಿಂದಲೆ ನೀವು ಜೀವನದಲ್ಲಿ ಮಾರ್ಗವನ್ನು ಮಾಡಿಕೊಳ್ಳುವಿರಿ. ಪ್ರತಿಯೊಂದು ಕೆಲಸವೂ ಜಯಪ್ರದವಾಗುವುದಕ್ಕೆ ಮುಂಚೆ ಹಲವು ಕಷ್ಟಗಳನ್ನು ಎದುರಿಸಬೇಕು. ಯಾರು ಬಿಡದೆ ಕೆಲಸ ಮಾಡುವರೊ ಅವರು ಒಂದಲ್ಲ ಒಂದು ದಿನ ಜಯಪ್ರದರಾಗುವರು. ನಾವು ಒಂದು ಸತ್ಕಾರ್ಯವನ್ನು ಸಾಧಿಸಬೇಕಾದರೆ ಅತ್ಯಂತ ತಾಳ್ಮೆ, ಪವಿತ್ರತೆ ಮತ್ತು ಛಲ ಇವು ಆವಶ್ಯಕ. ಶ್ರದ್ಧಾ ಭಕ್ತಿಯಿಂದ ಹಿಡಿದ ಕೆಲಸವನ್ನು ಬಿಡಬೇಡಿ. ಸತ್ಯವಂತ ನಾನು, ಪುಜತ್ವದವ ನಾನು, ಪರಿಶುದ್ಧ ನಾನು ಎಂದು ಭಾವಿಸಿ. ಮೊದಲು ದೊಡ್ಡದೊಡ್ಡ ಯೋಜನೆ ಗಳನ್ನು ಕಲ್ಪಿಸಿಕೊಳ್ಳಬೇಡಿ. ನಿಧಾನವಾಗಿ ಸ್ವಲ್ಪ ಸ್ವಲ್ಪವನ್ನು ಮಾಡುತ್ತಾ ಮುಂದೆ ಮುಂದಕ್ಕೆ ಹೋಗಿ.

ಇವೆರಡೂ ನಿಮ್ಮಲ್ಲಿ ಬರದಂತೆ ನೋಡಿಕೊಳ್ಳಿ–ಅದೇ ಅಧಿಕಾರ ಲಾಲಸೆ ಮತ್ತು ಅಸೂಯಾಭಾವನೆ. ನೀವು ನಾಯಕನಂತೆ ಮುಂದೆ ನಿಂತರೆ ನಿಮ್ಮ ಸಹಾಯಕ್ಕೆ ಯಾರೂ ಬರುವುದಿಲ್ಲ. ನೀವು ಜಯಪ್ರದರಾಗಬೇಕಾದರೆ ನಾನೆಂಬು ದನ್ನು ಮೊದಲು ಮರೆಯಿರಿ. ನಿಮ್ಮ ಸಹೋದರರಿಗೆ ನಾಯಕನಾಗಬೇಡಿ. ಅವರಿಗೆ ಸೇವೆಮಾಡಿ. ನಾನು ನಾಯಕನಾಗಬೇಕೆಂಬ ಭಾವನೆ ಮೆಟ್ಟಿಕೊಂಡು ಅನೇಕ ಜನ ಜೀವನದಲ್ಲಿ ದುರಂತಕ್ಕೆ ಈಡಾಗಿರುವರು. ಈ ಅವಗುಣದಿಂದ ಪಾರಾಗಿ, ಇತರರ ಮೇಲೆ ಅಮೆರಿಕಾ ದೇಶೀಯರು ಹೇಳುವಂತೆ ‘ಬಾಸ್​’ ಮಾಡುವುದಕ್ಕೆ ಹೋಗ ಬೇಡಿ. ಅಧಿಕಾರ ಚಲಾಯಿಸುವುದಕ್ಕೆ ಹೋಗಬೇಡಿ. ಎಲ್ಲರ ಸೇವಕರಾಗಿ. ಆಳುವು ದಕ್ಕೆ ಪ್ರಯತ್ನಿಸಬೇಡಿ. ಇತರರನ್ನು ಚೆನ್ನಾಗಿ ಸೇವಿಸುವವನೆ ಚೆನ್ನಾಗಿ ಆಳುವವನು.

ಎಲ್ಲರೊಂದಿಗೂ ತಾಳ್ಮೆಯಿಂದಿರಿ. ನೀವು ವಾದ ವಿವಾದಗಳಲ್ಲಿ ಏತಕ್ಕೆ ಸೇರಬೇಕು? ಭಿನ್ನಾಭಿಪ್ರಾಯಗಳನ್ನು ಸಹನೆಯಿಂದ ನೋಡಬೇಕು. ತಾಳ್ಮೆ, ಪರಿ ಶುದ್ಧತೆ ಮತ್ತು ಹಿಡಿದ ಕೆಲಸವನ್ನು ಬಿಡದೆ ಇರುವ ಸ್ವಭಾವ, ಇವುಗಳಿದ್ದರೆ ಕೊನೆಗೆ ಯಾವ ಕೆಲಸವನ್ನು ಹಿಡಿದರೂ ಜಯಪ್ರದವಾಗಿಯೇ ಆಗುವುದು. ಪ್ರತಿಯೊಬ್ಬ ರನ್ನು ತೃಪ್ತಿಪಡಿಸುವುದಕ್ಕೆ ಪ್ರಯತ್ನಿಸಿ. ಆದರೆ ಒಳಗೊಂದು ಹೊರಗೊಂದು ಬೇಡ. ಪರಿಶುದ್ಧರಾಗಿ ಬಲವಾಗಿ ನೀವು ನಂಬಿರುವುದನ್ನು ಹಿಡಿಯಿರಿ. ಈಗ ನಿಮ್ಮ ಮಾರ್ಗದಲ್ಲಿ ಯಾವ ಆತಂಕಗಳಿದ್ದರೂ ಕೊನೆಗೆ ಇತರರು ನಿಮ್ಮ ಮಾತಿಗೆ ಮನ್ನಣೆ ಯನ್ನು ಕೊಡಲೇ ಬೇಕು. ನಿಮ್ಮ ಭಾವನೆಗಳಲ್ಲಿ ನಿರ್ದಿಷ್ಟರಾಗಿರಿ. ಇನ್ನೊಬ್ಬರನ್ನು ದೂರಬೇಡಿ. ನಿಮ್ಮ ಸಂದೇಶವನ್ನು ಸಾರಿ. ನೀವು ಹೇಳಬೇಕಾಗಿರುವುದನ್ನು ಹೇಳಿ ಸುಮ್ಮನಿರಿ. ಮುಂದಿನದು ಭಗವಂತನ ಕೈಯ್ಯಲ್ಲಿದೆ.

ನಿಮ್ಮ ಹೃದಯವನ್ನು ಆದರ್ಶಗಳನ್ನು ವಿಶ್ವಪರ್ಯಂತ ವಿಶಾಲ ಮಾಡಿ. ನನಗೆ ಮತಭ್ರಾಂತನಲ್ಲಿರುವ ತೀವ್ರತೆ ಜೊತೆಗೆ ಜಡವಾದಿಯಲ್ಲಿರುವಷ್ಟು ವೈಶಾಲ್ಯತೆ ಬೇಕು. ನಮ್ಮ ಹೃದಯ ಕಡಲಿನಷ್ಟು ಆಳ, ಆಕಾಶದಷ್ಟು ವಿಶಾಲವಾಗಿರಬೇಕು.

ನನ್ನ ಸಹೋದರರೆ, ನಾವು ಬಡವರು, ನಗಣ್ಯರು. ಆದರೆ ಭಗವಂತ ಯಾವಾ ಗಲೂ ತನ್ನ ಕೆಲಸವನ್ನು ಮಾಡಿದ್ದೇ ಇಂತಹವರ ಮೂಲಕ. ನನಗೆ ಹೊರಗಿನಿಂದ ಯಾರೊ ಸಹಾಯ ಮಾಡುತ್ತಾರೆ ಎಂದು ನೆಚ್ಚಿ ಕುಳಿತಿರಬೇಡಿ. ಮನುಷ್ಯಸಹಾಯ ಗಳೆಲ್ಲಕ್ಕಿಂತ ದೇವರು ದೊಡ್ಡವನಲ್ಲವೆ? ಪರಿಶುದ್ಧರಾಗಿ, ಭಗವಂತನಲ್ಲಿ ಶ್ರದ್ಧೆ ಇಡಿ. ಯಾವಾಗಲೂ ಅವನನ್ನು ನಂಬಿ. ಆಗ ನೀವು ಯಾವಾಗಲೂ ಸರಿಯಾದ ಮಾರ್ಗದಲ್ಲಿರುವಿರಿ. ಯಾವುದೂ ನಿಮ್ಮನ್ನು ಕದಲಿಸಲಾರದು. ಕರುಣಾಳು ಬಾ ಬೆಳಕೆ ದಾರಿ ತೋರೆಂದು ಪ್ರಾರ್ಥಿಸಿ. ಗಾಢಾಂಧಕಾರದಿಂದ ಬೆಳಕಿನ ಕಿರಣ ವೊಂದು ಬರುವುದು. ನಮಗೆ ದಾರಿತೋರಲು ಕೈಯೊಂದು ಎತ್ತುವುದು. ಭಗವಂತ ನಿಗೆ ಜಯವಾಗಲಿ, ಮುಂದುವರಿಯಿರಿ. ಭಗವಂತನೇ ನಮ್ಮ ಸೇನಾನಿ. ಯಾರು ಬಿದ್ದರೆಂದು ನೋಡುವುದಕ್ಕೆ ಹಿಂದಿರುಗಬೇಡಿ. ಮುಂದೆ, ಎಂದೆಂದೂ ಮುಂದೆ. ಸಹೋದರರೆ, ನಾವು ಹೀಗೆ ಮುಂದುವರಿಯುವೆವು. ಒಬ್ಬ ಬೀಳುವನು ಇನ್ನೊಬ್ಬ ಅವನ ಸ್ಥಳಕ್ಕೆ ಬರುವನು. ಧೀರರೆ, ಕೆಲಸಕ್ಕೆ ಕೈ ಹಾಕಿ. ಎದೆಗುಂದಬೇಡಿ. ನನಗೆ ಆಗುವುದಿಲ್ಲ ಎಂದು ಹೇಳಬೇಡಿ. ಕೆಲಸ ಮಾಡಿ. ಅದರ ಹಿಂದೆ ಪರಮೇ ಶ್ವರನೇ ಇರುವನು. ನಿಮ್ಮಲ್ಲಿ ಮಹಾಶಕ್ತಿ ಇದೆ.

