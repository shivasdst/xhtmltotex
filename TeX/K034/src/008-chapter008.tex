
\chapter{ನಮ್ಮ ಸಾಂಸ್ಕೃತಿಕ ಜೀವನವನ್ನು ಉಜ್ಜೀವನಗೊಳಿಸಬೇಕು}

\textbf{ನಮ್ಮ ಸಾಂಸ್ಕೃತಿಕ ಜೀವನದ ಹಿನ್ನೆಲೆಯನ್ನು ರಕ್ಷಿಸಿಕೊಳ್ಳಬೇಕು}: ನಾವು ನಮ್ಮ ಪ್ರಕೃತಿಗೆ ಅನುಸಾರವಾಗಿ ಬೆಳೆಯಬೇಕು. ಪರದೇಶದ ಸಂಸ್ಥೆಗಳು ನಮಗೆ ತೋರಿಸಿ ಕೊಟ್ಟ ಮಾರ್ಗದಿಂದ ಪ್ರಯೋಜನವಿಲ್ಲ. ಅದರಂತೆ ಮಾಡುವುದಕ್ಕೆ ನಮಗೆ ಸಾಧ್ಯವೇ ಇಲ್ಲ. ಹೀಗೆ ಆಗದೆ ಇರುವುದರಿಂದ ದೇವರಿಗೆ ಧನ್ಯವಾದಗಳು. ನಾವು ಇತರರು ನಮ್ಮನ್ನು ಹೇಗೆ ಮಾಡಿಸಿದರೆ, ಹಾಗೆ ಆಗದೆ ಇರುವುದು ಮೇಲು. ನಾನು ಇತರ ಜನಾಂಗದ ಸಂಸ್ಥೆಗಳನ್ನು ದೂರುವುದಿಲ್ಲ. ಅವು ಆ ದೇಶಕ್ಕೆ ಒಳ್ಳೆಯವು, ಆದರೆ ನಮಗಲ್ಲ. ಪಾಶ್ಚಾತ್ಯ ದೇಶಗಳ ವಿಜ್ಞಾನ ಮತ್ತು ಇತರ ಸಂಸ್ಥೆಗಳು, ಮತ್ತು ಹಿಂದಿನಿಂದ ಪಾರಂಪರ್ಯವಾಗಿ ಬಂದ ಹಲವು ವಿಷಯಗಳಿಂದ ಈಗಿನದು ಅವ ರಲ್ಲಿ ರೂಪುಗೊಂಡಿದೆ. ನಮಗೆ ನಮ್ಮದೇ ಆದ ಹಿಂದಿನಿಂದ ಬಂದ ಸಂಸ್ಕೃತಿ ಇದೆ. ಸಾವಿರಾರು ವರುಷಗಳ ಕರ್ಮ ನಮ್ಮ ಹಿಂದೆ ಇದೆ. ಆದಕಾರಣ ಸ್ವಾಭಾವಿಕ ವಾಗಿ ನಾವು ಅವುಗಳ ಜಾಡಿನಲ್ಲೇ ಹೋಗಬೇಕಾಗಿದೆ. ನಾವು ಹೀಗೆಯೆ ಮಾಡ ಬೇಕಾಗಿದೆ.

ನಾವು ಪಾಶ್ಚಾತ್ಯರಾಗಲಾರೆವು. ನಾವು ಪಾಶ್ಚಾತ್ಯರನ್ನು ಅನುಕರಿಸಿ ಪ್ರಯೋಜನ ವಿಲ್ಲ. ಪಾಶ್ಚಾತ್ಯರನ್ನು ಅನುಕರಿಸುವುದಕ್ಕೆ ನಿಮಗೆ ಸಾಧ್ಯವಾಯಿತು ಎಂದು ಭಾವಿ ಸೋಣ. ಆಗ ನೀವು ಆ ಕ್ಷಣ ನಾಶವಾಗುವಿರಿ. ನಿಮ್ಮಲ್ಲಿ ಜೀವವಿರುವುದಿಲ್ಲ. ಮಾನವನ ಇತಿಹಾಸದ ಬಗೆಗೆ ನಿಲುಕದ ಅನಾದಿಕಾಲದಿಂದಲೂ ಒಂದು ಪ್ರವಾಹ ಹರಿಯುತ್ತಿದೆ. ನೀವು ಅದನ್ನು ಹಿಂದಕ್ಕೆ ತಳ್ಳಿ ಹಿಮಾಲಯದ ಶಿಖರಗಳ ಮಧ್ಯದಲ್ಲಿ ನೀರ್ಗಲ್ಲಿನ ಮೂಲರೂಪಕ್ಕೆ ಅದನ್ನು ಕಳುಹಿಸುವುದಕ್ಕೆ ಆಗುವುದೆ? ಒಂದು ವೇಳೆ ಇದು ಸಾಧ್ಯವಾದರೂ ನಿಮಗೆ ಐರೋಪ್ಯರಾಗಲು ಸಾಧ್ಯವಿಲ್ಲ. ಪಾಶ್ಚಾತ್ಯರಿಗೇ ಅವರ ಕೆಲವು ಶತಮಾನಗಳ ಸಂಸ್ಕೃತಿಯನ್ನೆ ಮರೆಯುವುದಕ್ಕೆ ಆಗದೆ ಇರುವಾಗ ಹಲವು ಶತಮಾನಗಳಿಂದ ಬಂದಿರುವ ನಮ್ಮ ಸಂಸ್ಕೃತಿಯನ್ನು ಮರೆಯಬಲ್ಲೆವೆ? ಇದು ಸಾಧ್ಯವೇ ಇಲ್ಲ. ಆದಕಾರಣ ನಾವು ಭರತಖಂಡವನ್ನೇ ಪಾಶ್ಚಾತ್ಯ ದೇಶದಂತೆ ಮಾರ್ಪಡಿಸಲು ಆಗುವುದಿಲ್ಲ. ಹಾಗೆ ಮಾಡಲೆತ್ನಿಸುವುದು ಕೂಡ ಮೂರ್ಖತನ.

ಭರತಖಂಡದಲ್ಲಿ ಪ್ರಗತಿಯ ದಾರಿಯಲ್ಲಿ ಎರಡು ಆತಂಕಗಳಿವೆ. ಅದೇ ಪೂರ್ವಾಚಾರಪರಾಯಣತೆ ಒಂದು, ಅತ್ಯಾಧುನಿಕ ಯುರೋಪಿನ ಅನುಕರಣೆ ಮತ್ತೊಂದು. ಇವೆರಡರಲ್ಲಿ ನಾನು ಪೂರ್ವದಿಂದ ಬಂದ ಸಂಪ್ರದಾಯವಂತ ರನ್ನು ಮೆಚ್ಚುತ್ತೇನೆ, ಯುರೋಪಿನ ಅಂಧ ಅನುಕರಣೆಯನ್ನಲ್ಲ. ಹಿಂದಿನ ಸಂಪ್ರ ದಾಯಸ್ಥ ಅಜ್ಞಾನಿಯಾಗಿರಬಹುದು, ಒರಟನಾಗಿರಬಹುದು. ಆದರೆ ಅವನೊಬ್ಬ ಮನುಷ್ಯ. ಅವನಿಗೆ ಶ್ರದ್ಧೆ ಇದೆ, ಶಕ್ತಿ ಇದೆ, ಅವನು ತನ್ನ ಕಾಲಮೇಲೆ ತಾನು ನಿಲ್ಲುವನು. ಆದರೆ ಪಾಶ್ಚಾತ್ಯರನ್ನು ಅನುಕರಿಸುವ ಮನುಷ್ಯನಿಗಾದರೊ ಶಕ್ತಿಯೇ ಇಲ್ಲ. ಅವನು ಎಲ್ಲಾ ಕಡೆಯಿಂದಲೂ ಸಿಕ್ಕುವ ಕೆಲವು ಭಾವನೆಗಳನ್ನು ಅಸ್ತವ್ಯಸ್ತ ವಾಗಿ ತಲೆಯಲ್ಲಿ ತುಂಬಿಕೊಂಡಿರುವನು. ಇದನ್ನು ಅರಗಿಸಿಕೊಂಡಿಲ್ಲ. ತನ್ನದ ನ್ನಾಗಿ ಮಾಡಿಕೊಂಡಿಲ್ಲ. ಅವನು ಸುಮ್ಮನೇ ಸುತ್ತುತ್ತಿರುವನು. ಅವನು ಕೆಲಸ ಮಾಡುವುದಕ್ಕೆ ಮೂಲ ಉತ್ತೇಜನ ಎಲ್ಲಿದೆ? ಕೆಲವು ಇಂಗ್ಲೀಷರು ಚೆನ್ನಾಗಿ ಮಾತ ನಾಡಿ ಅವನ ಭುಜತಟ್ಟಿದರು ಅಷ್ಟೆ. ಅವನ ಸುಧಾರಣೆಯ ಯೋಜನೆಗಳಿಗೆ ಕಾರಣ, ಸಮಾಜದಲ್ಲಿರುವ ಕೆಲವು ಆಚಾರವನ್ನು ಅವನು ಕಟುವಾಗಿ ಟೀಕಿಸುವುದಕ್ಕೆ ಕಾರಣ ಹಿಂದೆ, ಯಾರೋ ಕೆಲವು ಐರೋಪ್ಯರು ಅವುಗಳನ್ನು ಟೀಕಿಸುವುದೇ ಆಗಿದೆ. ನಮ್ಮಲ್ಲಿ ಕೆಲವು ಆಚಾರಗಳನ್ನು ಏತಕ್ಕೆ ಕೆಟ್ಟದೆಂದು ಭಾವಿಸುತ್ತಾರೆ? ಯುರೋಪಿ ಯನ್ನರು ಹಾಗೆ ಹೇಳುತ್ತಾರೆ ಎಂಬುದೇ ಕಾರಣ. ನಾನು ಇದನ್ನು ಒಪ್ಪಿಕೊಳ್ಳುವು ದಿಲ್ಲ. ನಿಮ್ಮ ಶಕ್ತಿಯ ಆಧಾರದ ಮೇಲೆ ನಿಂತು ಸಾಯಿರಿ. ನಿಮ್ಮಲ್ಲಿ ಏನಾದರೂ ಪಾಪವಿದ್ದರೆ ಅದೇ ದೌರ್ಬಲ್ಯ. ಎಲ್ಲ ವಿಧವಾದ ದೌರ್ಬಲ್ಯಗಳಿಂದಲೂ ಪಾರಾಗಿ. ದೌರ್ಬಲ್ಯವೇ ಪಾಪ, ದೌರ್ಬಲ್ಯವೇ ಮರಣ. ಚಿತ್ತಸ್ವಾಸ್ಥ್ಯವಿಲ್ಲದ ಈ ಮನು ಷ್ಯರು ಇನ್ನೂ ಒಂದು ವ್ಯಕ್ತಿಯಾಗಿಲ್ಲ. ನಾವು ಅವರನ್ನು ಏನೆಂದು ಕರೆಯೋಣ? ಹೆಂಗಸರೆಂದೆ, ಗಂಡಸರೆಂದೆ, ಅಥವಾ ಮೃಗಗಳೆಂದೆ? ಆದರೆ ಹಿಂದಿನ ಕಾಲದ ಸಂಪ್ರದಾಯವಂತರಲ್ಲಾದರೊ ಒಂದು ಛಲವಿತ್ತು, ಅವರು ಮನುಷ್ಯರಾಗಿದ್ದರು.

ಆದಕಾರಣ ಇವರಿಬ್ಬರಲ್ಲಿ ಸಂಪ್ರದಾಯವಂತನಲ್ಲಾದರೂ ಜನಾಂಗದ ಮೂಲಚೈತನ್ಯವಾದ ಆಧ್ಯಾತ್ಮಿಕತೆ ಇದೆ. ಮತ್ತೊಬ್ಬನ ಕೈಯಲ್ಲಾದರೋ ಪಾಶ್ಚಾ ತ್ಯರ ಕೆಲವು ನಕಲು ಆಭರಣಗಳಿವೆ, ಚೇತನವನ್ನು ನೀಡುವ ಆಧ್ಯಾತ್ಮಿಕತೆ ಇಲ್ಲ. ಇವರಿಬ್ಬರಲ್ಲಿ ನೀವೆಲ್ಲ ಸಂಪ್ರದಾಯವಂತನನ್ನು ಮೆಚ್ಚುತ್ತೀರಿ ಎಂದು ನಾನು ಭಾವಿಸುತ್ತೇನೆ. ಅವನಲ್ಲಿ ಸ್ವಲ್ಪ ಭರವಸೆ ಇದೆ. ಅವನಲ್ಲಿ ಜನಾಂಗದ ಸ್ವಭಾವ ವಿದೆ. ಹಿಡಿದುಕೊಳ್ಳುವುದಕ್ಕೆ ಏನೋ ಸಿಕ್ಕಿದೆ. ಆದಕಾರಣ ಅವನು ಬದುಕುವನು, ಇತರರಾದರೊ ಸಾಯುವರು. ನೀವು ಆಧ್ಯಾತ್ಮಿಕತೆಯನ್ನು ಬಿಟ್ಟು, ಪ್ರಪಂಚವನ್ನೇ ಸರ್ವಸ್ವ ಎಂದು ಭಾವಿಸುವ ಪಾಶ್ಚಾತ್ಯ ನಾಗರಿಕತೆಯನ್ನು ಅನುಕರಿಸಿದರೆ ಇದರ ಪರಿಣಾಮವಾಗಿ ಇನ್ನು ಮೂರು ತಲೆಮಾರಿನಲ್ಲಿ ನೀವು ನಾಶವಾಗುವಿರಿ. ಏಕೆಂದರೆ ಜನಾಂಗದ ಜೀವಾಳವೇ ನಾಶವಾಗುವುದು, ಯಾವುದರ ಮೇಲೆ ನಮ್ಮ ಜನಾಂಗದ ಸೌಧ ನಿಂತಿರುವುದೋ ಆ ಸೌಧವೇ ಕುಸಿದುಬೀಳುವುದು. ಕೊನೆಗೆ ಸರ್ವನಾಶ ವಾಗುವುದು.

ಭರತಖಂಡದಲ್ಲಿ ಯಾರಾದರೂ ಜನರಿಗೆ ತಿನ್ನುವುದು ಕುಡಿಯುವುದು ಮಜಾ ಮಾಡುವುದು ಈ ಆದರ್ಶವನ್ನು ಬೋಧಿಸಿದರೆ, ಈಗಿನ ಭೌತಿಕ ಪ್ರಪಂಚವೆ ಸರ್ವಸ್ವ ಇದೇ ದೇವರೆಂದು ಬೋಧಿಸಿದರೆ, ಅವನೊಬ್ಬ ಸುಳ್ಳುಗಾರ. ಸಮಾಜ ದಲ್ಲಿ ಅವನಿಗೆ ಆ ಪವಿತ್ರದೇಶದಲ್ಲಿ ಸ್ಥಳವಿಲ್ಲ ಎಂದು ತಿಳಿಯಿರಿ. ಭಾರತೀಯ ಅಂಥವನನ್ನು ಕೇಳುವುದಕ್ಕೆ ಇಚ್ಛೆಪಡುವುದಿಲ್ಲ. ಪಾಶ್ಚಾತ್ಯ ನಾಗರಿಕತೆ ಎಷ್ಟೇ ಮಿರುಗುತ್ತಿದ್ದರೂ, ಅದು ಎಷ್ಟೇ ನಾಜೂಕಾಗಿ ಕಂಡರೂ, ಅದರಲ್ಲಿ ಬೇಕಾದಷ್ಟು ಶಕ್ತಿ ಬಾಹುಳ್ಯ ಇದ್ದರೂ ಅವುಗಳಿಂದ ನಮಗೆ ಏನೂ ಪ್ರಯೋಜನ ಇಲ್ಲ ಎಂದು ಅವರೆದುರಿಗೇ ಹೇಳುತ್ತೇನೆ. ಇವುಗಳೆಲ್ಲ ಕ್ಷಣಿಕ. ದೇವರೊಬ್ಬನೇ ಸತ್ಯ, ಆತ್ಮ ನೊಬ್ಬನೇ ಸತ್ಯ, ಆಧ್ಯಾತ್ಮಿಕತೆಯೊಂದೇ ಸತ್ಯ. ನೀವು ಅವುಗಳನ್ನು ಬಿಡದೆ ಹಿಡಿದು ಕೊಳ್ಳಿ.

ನಮ್ಮ ಸಾಂಸ್ಕೃತಿಕ ದೃಷ್ಟಿಯನ್ನು ವಿಶಾಲಗೊಳಿಸಬೇಕು. ಭರತಖಂಡ ಹೊರಗಿ ನವರ ಸಹಾಯವಿಲ್ಲದೆ ಇರಲಾರದು. ನಾವು ಹಾಗೆ ಇರಬಲ್ಲೆವು ಎಂದು ಭಾವಿಸಿದ್ದೇ ಮೂರ್ಖತನ. ಇದರ ಪರಿಣಾಮವಾಗಿ ನಾವು ಒಂದು ಸಾವಿರ ವರುಷ ಗುಲಾಮಗಿರಿ ಯಲ್ಲಿ ನರಳಿರುವೆವು. ಇದಕ್ಕೆ ನಾವು ಬೇಕಾದಷ್ಟು ಅನುಭವಿಸಿ ಆಗಿದೆ. ಇನ್ನು ಮೇಲೆ ಅದನ್ನು ಮಾಡದೆ ಇರೋಣ. ಭಾರತೀಯ ಹೊರಗೆ ಹೋಗಬಾರದು ಎಂಬ ಭಾವನೆಯೆಲ್ಲ ಕೆಲಸಕ್ಕೆ ಬಾರದ್ದು. ಇಂತಹ ಭಾವನೆಗಳನ್ನು ಹೊಡೆದಟ್ಟಬೇಕು. ನೀವು ಹೊರಗೆ ಹೋಗಿ ಇತರ ದೇಶಗಳನ್ನು ಹೆಚ್ಚು ನೋಡಿ ಬಂದಷ್ಟೂ ಅದರಿಂದ ನಿಮಗೆ ಮತ್ತು ನಿಮ್ಮ ದೇಶಕ್ಕೆ ಮೇಲು. ನೀವು ಹಿಂದೆ ನೂರುವರುಷಗಳಿಂದ ಇದನ್ನು ಮಾಡಿದ್ದರೆ, ಈಗ ನೀವು ಎಲ್ಲರ ಪದತಳದಲ್ಲಿಯೂ ಇರುತ್ತಿರಲಿಲ್ಲ. ಜೀವನದ ಪ್ರಥಮ ಚಿಹ್ನೆಯೇ ವಿಕಾಸ. ನೀವು ಜೀವಿಸಬೇಕಾದರೆ ವಿಕಾಸವಾಗಬೇಕು. ವಿಕಾಸವಾಗುವುದು ನಿಂತೊಡನೆಯೆ ಮೃತ್ಯು ನಿಮ್ಮ ಸಮೀಪದಲ್ಲಿರುವುದು, ಅಪಾಯ ನಿಮ್ಮ ಸಮೀಪದಲ್ಲಿರುವುದು.

ಹಲವು ಅಪಾಯಗಳು ನಮ್ಮ ಮಾರ್ಗದಲ್ಲಿವೆ. ಅವುಗಳಲ್ಲಿ ಒಂದೇ ನಮ್ಮ ಸಮಾನ ಇಲ್ಲ ಈ ಪ್ರಪಂಚದಲ್ಲಿ, ಎಂದು ಭಾವಿಸುವುದು. ನನಗೆ ಭರತಖಂಡದ ಮೇಲೆ ಬೇಕಾದಷ್ಟು ಪ್ರೀತಿ ಇದ್ದರೂ, ನಮ್ಮ ಪೂರ್ವಿಕರನ್ನು ನಾನು ಎಷ್ಟೇ ಗೌರವಿಸಿದರೂ, ಎಷ್ಟೇ ಕೊಂಡಾಡಿದರೂ, ನಾವು ಇತರರಿಂದ ಹಲವು ವಿಷಯ ಗಳನ್ನು ಕಲಿಯಬೇಕಾಗಿದೆ ಎಂದು ನಂಬುತ್ತೇನೆ. ಎಲ್ಲರ ಹತ್ತಿರವೂ ಕುಳಿತು ಕೊಂಡು ಕಲಿಯಲು ನಾವು ಸಿದ್ಧರಾಗಿರಬೇಕು. ಎಲ್ಲರೂ ನಮಗೆ ಏನನ್ನಾದರೂ ಒಳ್ಳೆಯದನ್ನು ಕಲಿಸಬಹುದು. ಆದಕಾರಣವೇ ನಮಗೆ ಧರ್ಮಶಾಸ್ತ್ರವನ್ನು ಕೊಟ್ಟ ಮನು ಹೀಗೆ ಹೇಳುವನು: “ಅಂತ್ಯಜನಾದರೂ ಚಿಂತೆಯಿಲ್ಲ ಅವನಿಂದ ಒಂದು ಒಳ್ಳೆಯ ವಿಷಯವನ್ನು ಕಲಿತುಕೊಳ್ಳಿ. ಸೇವೆಯ ಮೂಲಕ ಮುಕ್ತಿ ಮಾರ್ಗವನ್ನು ಅರಿಯಿರಿ.” ಆದಕಾರಣ ನಾವು ಮನುವಿನ ನಿಜವಾದ ಮಕ್ಕಳಾದುದರಿಂದ ಅವನ ಆಜ್ಞೆಯನ್ನು ಪಾಲಿಸಬೇಕು. ಇಹ ಮತ್ತು ಪರಲೋಕದ ವಿಷಯವಾಗಿ ಯಾರು ಏನನ್ನು ಹೇಳಿದರೂ ಅದನ್ನು ಕಲಿತುಕೊಳ್ಳುವುದಕ್ಕೆ ಸಿದ್ಧರಾಗಿರಬೇಕು.

ಯುರೋಪಿನಲ್ಲಿ ಪ್ರತಿಯೊಂದರ ಮೂಲಕ ವ್ಯಕ್ತವಾಗುವುದು ಗ್ರೀಸ್. ಪ್ರತಿ ಯೊಂದು ಕಟ್ಟಡ, ಅದರೊಳಗೆ ಉಪಯೋಗಿಸುವ ಪ್ರತಿಯೊಂದು ವಸ್ತುವಿನ ಮೇಲೆಯೂ ಗ್ರೀಸಿನ ಛಾಯೆ ಇದೆ. ಯುರೋಪಿನ ವಿಜ್ಞಾನ ಮತ್ತು ಕಲೆ ಗ್ರೀಸಲ್ಲದೆ ಬೇರಲ್ಲ. ಇಂದು ಭರತಖಂಡದಲ್ಲಿ ಪುರಾತನ ಗ್ರೀಸ್ ಪುರಾತನ ಭಾರತೀಯರನ್ನು ಸಂಧಿಸುತ್ತಿದೆ. ಕ್ರಮೇಣ ಮೌನವಾಗಿ ಈ ಪ್ರಭಾವ ವ್ಯಕ್ತವಾಗುತ್ತದೆ. ಈಗ ನಮ್ಮ ಸುತ್ತಲೂ ಕಾಣುತ್ತಿರುವ ವೈಶಾಲ್ಯತೆ, ನವಜೀವನ ಮತ್ತು ನಮ್ಮ ರಾಷ್ಟ್ರದಲ್ಲಿ ಆಗುತ್ತಿರುವ ಉತ್ಥಾನ ಈ ಸಂಗಮದ ಪರಿಣಾಮವಾಗಿದೆ. ವಿಶಾಲವಾದ ಉದಾರ ವಾದ ಒಂದು ಜೀವನದ ಭಾವನೆ ನಮ್ಮ ಮುಂದೆ ಇದೆ. ಮೊದಮೊದಲು ನಾವು ಅದನ್ನು ಸಂಕುಚಿತಮಾಡುವುದಕ್ಕೆ ಪ್ರಯತ್ನಿಸಿದರೆ ಈ ಔದ್ಧಾರ್ಯದ ವಿಶಾಲವಾದ ಭಾವನೆಗಳು ಅವಕ್ಕೆ ಎಡೆಗೊಡುವುದಿಲ್ಲ. ನಮ್ಮ ಪೂರ್ವಿಕರು ಏನನ್ನು ಮಾಡ ಬೇಕೆಂದು ಬಯಸಿದ್ದರೊ, ಅದೇ ಇದು. ನಮ್ಮ ಹೃದಯ ವಿಶಾಲವಾಗಬೇಕು, ಅಲ್ಲಿರುವ ಒಳ್ಳೆಯದನ್ನು ಹೀರಿಕೊಳ್ಳಬೇಕು, ಎಲ್ಲವನ್ನೂ ವ್ಯಾಪಿಸಿಕೊಳ್ಳುವ ಸಾಮಾನ್ಯ ವರ್ಗಕ್ಕೆ ತರಬೇಕು ಎಂಬುದೇ ಗುರಿ.

ನಾವು ಪಾಶ್ಚಾತ್ಯರಿಂದ ಎಷ್ಟೋ ವಿಷಯಗಳನ್ನು ಕಲಿಯಬೇಕಾಗಿದೆ. ಅವರ ಕಲೆ ಮತ್ತು ವಿಜ್ಞಾನವನ್ನು ಅವರಿಂದ ಕಲಿಯಬೇಕು. ಲೌಕಿಕ ವ್ಯವಹಾರದಲ್ಲಿ ನಾವು ಅವರಿಂದ ಕಲಿಯಬೇಕಾದುದು ಎಷ್ಟೊ ಇವೆ. ಸಂಘಶಕ್ತಿಯ ಅಧಿಕಾರವನ್ನು ಉಪ ಯೋಗಿಸುವ ರೀತಿ, ಇತರರ ಕೆಲಸಗಳನ್ನು ಉಪಯೋಗಿಸಿಕೊಳ್ಳುವ ರೀತಿ, ಅತ್ಯಲ್ಪ ಸಾಧನಗಳಿಂದ ಅತ್ಯುತ್ತಮ ಫಲವನ್ನು ಪಡೆಯುವುದು, ಇವುಗಳನ್ನೆಲ್ಲ ನಾವು ಅವರಿಂದ ಕಲಿಯಬೇಕಾಗಿದೆ. ಇವುಗಳನ್ನೆಲ್ಲ ನಾವು ಪಾಶ್ಚಾತ್ಯರಿಂದ ಕಲಿಯ ಬಹುದು.

ನಾವು ಪ್ರಯಾಣ ಮಾಡಬೇಕು. ದೂರದೇಶಗಳಿಗೆ ಹೋಗಿ ನೋಡಬೇಕು. ಇತರ ದೇಶಗಳಲ್ಲಿ ಸಮಾಜದ ಯಂತ್ರ ಹೇಗೆ ಕೆಲಸಮಾಡುತ್ತಿದೆ ಎಂಬುದನ್ನು ನೋಡ ಬೇಕು. ನಾವು ಮತ್ತೊಮ್ಮೆ ಒಂದು ರಾಷ್ಟ್ರವಾಗಬೇಕಾದರೆ ಇತರ ದೇಶಗಳಲ್ಲಿ ಏನು ಆಗುತ್ತಿದೆ ಎಂಬುದನ್ನು ಚೆನ್ನಾಗಿ ತಿಳಿದುಕೊಂಡು ಅವರೊಂದಿಗೆ ಸಂಬಂಧವನ್ನು ಬೆಳೆಸಿಕೊಳ್ಳಬೇಕು. ನಿಮ್ಮ ನೆಲೆಯ ಮೇಲೆ ನೀವು ನಿಂತು ಹೊರಗಿನಿಂದ ಬರು ವುದನ್ನೆಲ್ಲ ಸ್ವೀಕರಿಸಿ. ಪ್ರತಿಯೊಂದು ದೇಶದಿಂದಲೂ ಕಲಿಯಿರಿ. ನಿಮಗೆ ಏನು ಪ್ರಯೋಜನವೊ ಅದನ್ನು ಸ್ವೀಕರಿಸಿ.

ಆದರೆ ಇದನ್ನು ಜೋಪಾನವಾಗಿ ಗಮನಿಸಿ. ನಾವು ಹಿಂದೂಗಳಾದುದರಿಂದ ಪ್ರತಿಯೊಂದನ್ನೂ ನಮ್ಮ ಜನಾಂಗದ ಆದರ್ಶಕ್ಕೆ ಅಧೀನಗೊಳಿಸಬೇಕು. ಪಾಶ್ಚಾತ್ಯರ ವಿಜ್ಞಾನ ಪಾಂಡಿತ್ಯ ಮತ್ತು ನಮ್ಮದೇ ಆದ ಐಶ್ವರ್ಯ ಸ್ಥಾನಮಾನಗಳು ಪ್ರತಿಯೊಬ್ಬ ಹಿಂದೂವಿನಲ್ಲಿಯೂ ಆಜನ್ಮವಾಗಿ ಬಂದ ಆಧ್ಯಾತ್ಮಿಕತೆ ಮತ್ತು ಜನಾಂಗದ ಪವಿತ್ರತೆ ಇವುಗಳಿಗೆ ಅಡಿಯಾಳಾಗಿರಬೇಕು. ಇದೇ ಭಾರತೀಯನ ರಹಸ್ಯ.

\textbf{ಪೌರಾತ್ಯ ಮತ್ತು ಪಾಶ್ಚಾತ್ಯ ದೃಷ್ಟಿಗಳು}: ಏಷ್ಯಾದೇಶದ ವಾಣಿ ಯಾವಾಗಲೂ ಧರ್ಮದ ವಾಣಿ ಆಗಿರುವುದು. ಯುರೋಪಿನ ವಾಣಿ ಯಾವಾಗಲೂ ರಾಜಕೀಯ ಆಗಿರುವುದು. ಪ್ರತಿಯೊಂದು ತನ್ನ ಸ್ಥಾನದಲ್ಲಿ ದೊಡ್ಡದೇ. ಯುರೋಪಿನ ವಾಣಿ ಪುರಾತನ ಗ್ರೀಸಿನ ವಾಣಿ. ಗ್ರೀಸಿನ ಮನಸ್ಸಿಗೆ ತನ್ನ ನೆರೆಹೊರೆಯ ಸಮಾಜವೇ ಸರ್ವಸ್ವ. ಇವುಗಳ ಹೊರಗೆ ಇರುವುದೆಲ್ಲ ಬರ್ಬರ. ಗ್ರೀಸಿನವರಿಗೆ ಮಾತ್ರ ಪ್ರಪಂಚದಲ್ಲಿ ಬಾಳಲು ಅಧಿಕಾರ ಎಂದು ಭಾವಿಸುವರು. ಗ್ರೀಸಿನವರು ಏನು ಮಾಡುತ್ತಾರೋ ಅವೆಲ್ಲ ಸರಿ. ಬೇರೆ ಕಡೆ ಇನ್ನು ಏನಿದೆಯೊ ಅದೆಲ್ಲ ತಪ್ಪು, ಮತ್ತು ಅವುಗಳಿಗೆ ಉಳಿಯಲು ಅಧಿಕಾರವಿಲ್ಲ. ಆದಕಾರಣ ಗ್ರೀಕರಲ್ಲಿ ಮಾನವಸಹಜ ವಾದ ಸಹಾನುಭೂತಿ, ತೀವ್ರ ಸ್ವಾಭಾವಿಕತೆ ಮತ್ತು ಕಲಾಪ್ರಿಯತೆಯನ್ನು ನೋಡು ವೆವು. ಗ್ರೀಕನು ಈ ಪ್ರಪಂಚದಲ್ಲಿ ಮಾತ್ರ ಬಾಳುವನು. ಅವನು ಕನಸುಣಿಯಲ್ಲ. ಅವನ ಕಾವ್ಯಕೂಡ ವ್ಯವಹಾರ ಪ್ರಧಾನವಾದುದು. ಗ್ರೀಕರ ದೇವದೇವತೆಗಳು ಮಾನವ ಸಹಜವಾಗಿರುವುದು ಮಾತ್ರವಲ್ಲ; ನಮ್ಮಲ್ಲಿರುವಂತೆ ಕೋಪತಾಪ ಗಳಿಂದಲೂ ಕೂಡಿರುವರು.

ಗ್ರೀಕರು ತಮಗೆ ಯಾವುದು ಸುಂದರವಾಗಿ ಕಾಣುವುದೊ ಅದನ್ನು ಪ್ರೀತಿಸು ವರು. ಆದರೆ ಇದನ್ನು ಗಮನಿಸಿ, ಇದು ಯಾವಾಗಲೂ ಬಾಹ್ಯ ಪ್ರಕೃತಿಗೆ ಸಂಬಂಧಿ ಸಿದುದು. ಗಿರಿನದಿಗಳು, ಹೂವುಗಳು, ಸುಂದರವಾದ ಆಹಾರಗಳು, ಮನುಷ್ಯನ ದೇಹ ಅದರ ಸೌಂದರ್ಯ ಇವುಗಳನ್ನು ಅವರು ಮೆಚ್ಚುವುದು. ಅವರೆ ಅನಂತರ ಯುರೋಪಿಗೆಲ್ಲ ಗುರುವಾದರು. ಆದಕಾರಣ ಯುರೋಪಿನ ಧ್ವನಿ ಗ್ರೀಕ್.

ಏಷ್ಯಾಖಂಡದಲ್ಲಿ ಮತ್ತೊಂದು ಆದರ್ಶವಿದೆ. ಪೌರಾತ್ಯನ ಸೌಂದರ್ಯ ಮತ್ತು ಭವ್ಯತೆ ಬೇರೊಂದು ದಿಕ್ಕಿನಲ್ಲಿ ಬೆಳೆಯಿತು. ಅದು ಅರ್ಂತಮುಖ, ಬಹಿರ್ಮುಖ ವಲ್ಲ. ಇಲ್ಲಿಯೂ ಪ್ರಕೃತಿಯ ಮೇಲೆ ಅದೇ ಆಸಕ್ತಿ ಇದೆ. ಇಲ್ಲಿಯೂ ಬರ್ಬರ ಮತ್ತು ಗ್ರೀಕ್ ಎಂಬ ಭಾವನೆ ಇದೆ.

ಆದರೆ ಅದು ಇನ್ನೂ ವಿಸ್ತಾರವಾಗುವುದು. ಏಷ್ಯಾಖಂಡದಲ್ಲಿ ಈಗಲೂ ಕೂಡ ಕುಲ ಬಣ್ಣ ಭಾಷೆಯಿಂದಲೇ ಒಬ್ಬರು ಒಂದು ಜನಾಂಗವಾಗಲಾರರು, ಅದನ್ನು ಒಂದು ಜನಾಂಗವನ್ನಾಗಿ ಮಾಡುವುದು ಧರ್ಮ. ಇಬ್ಬರು ಬೌದ್ಧರಾದರೆ ಒಬ್ಬ ಚೈನಾದಿಂದ ಬಂದಿರಬಹುದು, ಮತ್ತೊಬ್ಬ ಪರ್ಶಿಯಾದಿಂದ ಬಂದಿರಬಹುದು. ಆದರೆ ಅವರಿಬ್ಬರೂ ಒಂದೇ ಧರ್ಮಕ್ಕೆ ಸೇರಿರುವುದರಿಂದ ತಾವು ಸಹೋದರರು ಎಂದು ಭಾವಿಸುವರು. ಮಾನವಕೋಟಿಯನ್ನು ಒಟ್ಟಿಗೆ ಬಂಧಿಸಿರುವ ಸೂತ್ರವೇ ಧರ್ಮ. ಆದಕಾರಣದಿಂದಲೇ ಪೌರಾತ್ಯ ಒಬ್ಬ ಕಲ್ಪನಾಜೀವಿ, ಅವನೊಬ್ಬ ಹುಟ್ಟು ಕನಸುಣಿ. ಜಲಪಾತದ ಗರ್ಜನೆ, ಹಕ್ಕಿಗಳ ಗಾನದಿಂಚರ, ಸೂರ್ಯಚಂದ್ರ ತಾರಾವಳಿ ಮತ್ತು ಈ ವಿಶ್ವದ ಸೌಂದರ್ಯವನ್ನು ಅವನು ಮೆಚ್ಚಿದರೂ ಇಷ್ಟೇ ಸಾಲದು ಅವನಿಗೆ. ಇವುಗಳನ್ನು ಮೀರಿದ ಒಂದು ಕನಸನ್ನು ಕಾಣಲೆತ್ನಿಸುವನು ಅವನು. ಅವನು ಇಹವನ್ನು ಮೀರಲೆತ್ನಿಸುವನು. ಅವನು ಕಾಲವನ್ನು ಅಷ್ಟು ಗಣನೆಗೇ ತರುವುದಿಲ್ಲ.

ಯುಗಯುಗಗಳಿಂದ ಮಾನವಕೋಟಿಯ ಮೂಲಸ್ಥಾನವಾಗಿದೆ ಪೌರಾತ್ಯ ದೇಶ. ಜೀವನದ ಎಲ್ಲಾ ಕಷ್ಟಕಾರ್ಪಣ್ಯಗಳನ್ನು ಅವನು ಸಹಿಸಿರುವನು. ಇಲ್ಲಿ ಚಕ್ರಾಧಿಪತ್ಯಗಳಾದ ಮೇಲೆ ಚಕ್ರಾಧಿಪತ್ಯಗಳು ಬಂದಿವೆ. ಎಷ್ಟೋ ರಾಜ್ಯಗಳು ಎದ್ದು ಬಿದ್ದು ಹೋದುವು. ಮಾನವನ ಶಕ್ತಿ ಮಹಿಮೆ ಐಶ್ವರ್ಯವೆಲ್ಲ ಇಲ್ಲಿ ಇತ್ತು. ಪಾಂಡಿತ್ಯ ಮತ್ತು ಅಧಿಕಾರದ ಮಹಾವೈಭವವೇ ಇಲ್ಲಿತ್ತು. ಅಧಿಕಾರದ ಪರಮಾ ವಧಿ, ರಾಜವೈಭವಗಳು, ಪಾಂಡಿತ್ಯದ ಪರಾಕಾಷ್ಠೆ ಎಲ್ಲ ಇತ್ತು. ಆದಕಾರಣವೇ ಪೌರಾತ್ಯನು ನಿಕೃಷ್ಟವಾಗಿ ಇವುಗಳನ್ನೆಲ್ಲ ಕಾಣುವನು. ಯಾವುದು ಬದಲಾಯಿಸುವು ದಿಲ್ಲವೋ, ಯಾವುದು ನಾಶವಾಗುವುದಿಲ್ಲವೋ, ಈ ದುಃಖಸಾವುಗಳಿಂದ ತುಂಬಿದ ಲೋಕದಲ್ಲಿ ಯಾವುದು ನಿತ್ಯವಾಗಿದೆಯೊ ಸನಾತವಾಗಿದೆಯೊ ಆನಂದ ಮಯವಾಗಿರುವುದೊ ಅದನ್ನು ನೋಡಲು ಆಶಿಸುವನು. ಪೌರಾತ್ಯದೇಶದ ಮಹಾತ್ಮನಿಗೆ ಇವುಗಳನ್ನು ಎಷ್ಟು ಹೇಳಿದರೂ ತೃಪ್ತಿಇಲ್ಲ. ಮಹಾತ್ಮರೆಲ್ಲ ವಿನಾಯಿತಿ ಇಲ್ಲದೆ ಏಷ್ಯಾಖಂಡದಿಂದ ಬಂದವರು.

ಯಾರ ಕಣ್ಣುಗಳು ಪಾಶ್ಚಾತ್ಯ ಭೋಗ ಸಾಮಗ್ರಿಗಳ ಕಾಂತಿಯಿಂದ ಕುರುಡಾ ಗಿದೆಯೊ, ಯಾರು ತಿನ್ನುವುದು ಕುಡಿಯುವುದು ವಿಷಯವಸ್ತುಗಳನ್ನು ಅನುಭವಿ ಸುವುದೇ ತಮ್ಮ ಜೀವನದ ಪರಮ ಉದ್ದೇಶ ಎಂದು ಭಾವಿಸುವರೊ, ಹೊನ್ನು ಮತ್ತು ಮಣ್ಣು ಇವೇ ಪಡೆಯುವುದಕ್ಕೆ ಯೋಗ್ಯವಾದುದೆಂದು ಭಾವಿಸುವರೊ, ಯಾವನ ಸುಖ ಇಂದ್ರಿಯದಲ್ಲಿದೆಯೊ, ಯಾರ ದೇವರು ದ್ರವ್ಯವೊ, ಯಾರ ಜೀವನದ ದೃಷ್ಟಿ ಬದುಕಿರುವಾಗ ಸುಖವಾಗಿ ಆರಾಮವಾಗಿ ಬಾಳಿ ಅನಂತರ ಸಾಯು ವುದಾಗಿದೆಯೊ, ಯಾರ ಮನಸ್ಸು ಮುಂದೇನಾಗುವುದು ಎಂಬುದರ ಕಡೆಗೆ ಸ್ವಲ್ಪವೂ ಗಮನವನ್ನು ಕೊಡುವುದಿಲ್ಲವೊ, ಯಾರು ತಮ್ಮ ಸುತ್ತಲಿರುವ ಇಂದ್ರಿಯ ಪ್ರಪಂಚ ವಿನಃ ಬೇರೇನನ್ನು ಚಿಂತಿಸುವುದು ಅಪರೂಪವೋ ಅಂತಹ ಜನ ಭರತಖಂಡಕ್ಕೆ ಬಂದರೆ ಇಲ್ಲಿ ಏನು ಕಾಣುತ್ತಾರೆ? ಬಡತನ ಕೊಳೆ ಮೂಢ ನಂಬಿಕೆ ಅಜ್ಞಾನ ಎಲ್ಲೆಲ್ಲೂ ಕವಿದುಕೊಂಡಿರುವುದನ್ನು ನೋಡುತ್ತಾನೆ. ಏತಕ್ಕೆ ಹೀಗಿರುವುದು? ಪಾಶ್ಚಾತ್ಯನ ದೃಷ್ಟಿಯಲ್ಲಿ ಜ್ಞಾನ ಎಂದರೆ ವ್ಯಕ್ತಿಯಲ್ಲಿರುವ ಉಡುಪು ತೊಡುಪುಗಳು, ವಿದ್ಯಾಭ್ಯಾಸ ಮತ್ತು ಸಮಾಜದಲ್ಲಿ ತೋರುವ ನಾಜೂ ಕಾಗಿದೆ. ಪಾಶ್ಚಾತ್ಯ ದೇಶಗಳು ತಮ್ಮ ಲೌಕಿಕ ಸ್ಥಿತಿಯನ್ನು ಉತ್ತಮಗೊಳಿಸಿ ಕೊಳ್ಳಲು ಬೇಕಾದಷ್ಟು ಪ್ರಯತ್ನ ಮಾಡಿವೆ.

ಮಾನವ ಇತಿಹಾಸವನ್ನು ನೋಡಿದರೆ, ಹೊರಗೆ ಹೋಗಿ ಇತರ ದೇಶಗಳನ್ನು ಆಕ್ರಮಿಸದವರು ಭಾರತೀಯರೊಬ್ಬರೇ. ಇತರರ ಆಸ್ತಿಯನ್ನು ಅವರು ದೋಚಲು ಹೋಗಲಿಲ್ಲ. ಈ ದೇಶ ಫಲವತ್ತಾಗಿದ್ದುದರಿಂದ ಕಷ್ಟಪಟ್ಟು ಕೆಲಸಮಾಡಿ ಐಶ್ವರ್ಯ ವನ್ನು ಸಂಪಾದಿಸಿದರು. ಇದೇ ಅವರ ತಪ್ಪು! ಹೊರಗಿನವರಿಗೆ ಇವರ ಆಸ್ತಿಯ ಮೇಲೆ ಕಣ್ಣು ಬಿತ್ತು, ದೋಚಲು ಮನಸ್ಸಾಯಿತು. ಅವರ ಆಸ್ತಿಯನ್ನು ಇನ್ನೊ ಬ್ಬರು ಕೊಳ್ಳೆಹೊಡೆದುಕೊಂಡು ಹೋದರೂ ಚಿಂತೆಯಿಲ್ಲ, ಅವರನ್ನು ಬರ್ಬರ ಎಂದು ಕರೆದರೂ ಚಿಂತೆಯಿಲ್ಲ ಇದಕ್ಕೆ ಬದಲಾಗಿ ಅವರು ಅನ್ಯದೇಶೀಯರಿಗೆ ಪರ ಬ್ರಹ್ಮನ ವಿಷಯವನ್ನು ಬೋಧಿಸಲು ಪ್ರಚಾರಕರನ್ನು ಕಳುಹಿಸುವರು. ಮಾನವನ ಅಂತರಾಳದಲ್ಲಿ ಹುದುಗಿರುವ ರಹಸ್ಯಗಳನ್ನು ಎಲ್ಲರಿಗೂ ಕೊಡಲು, ತೋರಿಕೆಯ ಮಾನವನ ಹಿಂದೆ ಹೇಗೆ ಪರಮಾತ್ಮ ಅವಿತಿರುವನು ಎಂಬುದನ್ನು ಬೋಧಿಸಲು ಸಿದ್ಧರಾಗಿರುವರು. ಭಾರತೀಯನಿಗೆ ಈ ಪ್ರಪಂಚ ಒಂದು ಕನಸು. ಭೋಗ ಪ್ರಪಂಚದ ಹಿಂದೆ, ನಿಜವಾದ ಪವಿತ್ರವಾದ ಮಾನವನ ಸ್ವಭಾವ ಇರುವುದು. ಆ ಆತ್ಮನನ್ನು ಯಾವ ಪಾಪವೂ ಮುತ್ತಲಾರದು, ಯಾವ ಕೆಟ್ಟ ಕರ್ಮಗಳೂ ಅವನನ್ನು ಕೊಳೆಮಾಡಲಾರವು, ಯಾವ ಕಾಮವೂ ಅದನ್ನು ಕುಲಗೆಡಿಸಲಾರದು. ಅದನ್ನು ಬೆಂಕಿ ಸುಡಲಾರದು. ನೀರು ತೋಯಿಸಲಾರದು, ಶಾಖ ಒಣಗಿಸಲಾರದು, ಮೃತ್ಯು ಸಂಹರಿಸಲಾರದು. ಭಾರತೀಯನಿಗೆ ಮಾನವನ ಅಂತರಾಳದಲ್ಲಿರುವ ಈ ನೈಜ ಸ್ವಭಾವ, ಪಾಶ್ಚಾತ್ಯನಿಗೆ ಇಂದ್ರಿಯ ಪ್ರಪಂಚಗಳು ಎಷ್ಟು ಸತ್ಯವೋ ಅಷ್ಟೇ ಸತ್ಯ.

ಈ ಪ್ರಪಂಚವೆಲ್ಲ ಭಾವನಾಮಯ, ಇದೊಂದು ಕನಸು ಎಂದು ಸಿದ್ಧವಾದರೆ, ಇದನ್ನು ಸತ್ಯ ಎಂದು ಪ್ರತಿಪಾದಿಸಲು ತನ್ನ ಬಟ್ಟೆಬರೆ ಐಶ್ವರ್ಯಗಳನ್ನೆಲ್ಲ ತ್ಯಜಿಸಲು ಅವನು ಸಿದ್ಧನಾಗಿರುವನು. ಆತ್ಮ ನಿತ್ಯವಾದುದು ಎಂಬುದನ್ನು ಅರಿತ ಮೇಲೆ ತನ್ನ ದೇಹವನ್ನು ಒಂದು ತೃಣಕ್ಕಿಂತ ಕಡೆಯಾಗಿ ವಿಸರ್ಜಿಸುವುದಕ್ಕೆ ಅವನು ನದೀತೀರ ದಲ್ಲಿ ಧ್ಯಾನಕ್ಕೆ ಕುಳಿತುಕೊಳ್ಳಲು ಸಿದ್ಧನಾಗಿರುವನು. ಇಲ್ಲೇ ಅವರ ಸಾಹಸ ಇರು ವುದು. ಮೃತ್ಯುವನ್ನು ಒಬ್ಬ ಸಹೋದರನಂತೆ ಅವನು ಸ್ವಾಗತಿಸುವನು. ಏಕೆಂದರೆ ಅವನಿಗೆ ನಿಜವಾಗಿ ಸಾವಿಲ್ಲ ಎಂಬುದು ಚೆನ್ನಾಗಿ ಗೊತ್ತಿದೆ. ಸಾವಿರಾರು ವರುಷಗಳು ಅವರು ಅನ್ಯರ ಆಕ್ರಮಣಕ್ಕೆ ತುತ್ತಾಗಿ ಪಡಬಾರದ ಕಷ್ಟಸಂಕಟಗಳನ್ನೆಲ್ಲ ಅನು ಭವಿಸಿದರೂ ಅವರು ಇನ್ನೂ ಬದುಕಿರುವರು. ಆ ದೇಶ ಇಂದಿಗೂ ಜೀವಂತ ವಾಗಿರುವುದು. ಆ ದೇಶದಲ್ಲಿ ಅತ್ಯಂತ ಕಷ್ಟಕಾಲದಲ್ಲಿ ಕೂಡ ಮಹಾತ್ಮರು ಜನ್ಮ ವೆತ್ತಿರುವರು. ಏಷ್ಯಖಂಡ ಆಧ್ಯಾತ್ಮಿಕ ವೀರರನ್ನು ಸೃಷ್ಟಿಸಿರುವುದು. ಯುರೋಪು ರಾಜಕೀಯ ಪ್ರಪಂಚದಲ್ಲಿ ವೈಜ್ಞಾನಿಕ ಪ್ರಪಂಚದಲ್ಲಿ ಶೂರಾಧಿಶೂರರನ್ನು ಸೃಷ್ಟಿ ಸಿರುವುದು.

ಪಾಶ್ಚಾತ್ಯರಾದ ನೀವು ನಿಮ್ಮ ಕರ್ಮಭೂಮಿಯಾದ ಸಮರ ಕ್ಷೇತ್ರ, ರಾಜಕೀಯ ಕ್ಷೇತ್ರ ಇವುಗಳಲ್ಲಿ ವ್ಯವಹಾರ ಪಟುಗಳು. ಬಹುಶಃ ಈ ಕ್ಷೇತ್ರಗಳಲ್ಲಿ ಭಾರತೀಯ ಅಷ್ಟು ಚುರುಕು ಇಲ್ಲದೇ ಇರಬಹುದು. ಆದರೆ ಅವನು ತನ್ನ ಕ್ಷೇತ್ರದಲ್ಲಿ ಕರ್ಮ ಪಟು. ನೀವು ಒಂದು ಜಯಕಾರವನ್ನು ಹೇಳಿ ಪಿರಂಗಿಯ ಬಾಯಿಗೆ ಎದೆಯನ್ನು ಹೇಗೆ ಒಡ್ಡಬಲ್ಲಿರೋ, ದೇಶಭಕ್ತರಾಗಿ ನೀವು ಹೇಗೆ ನಿಮ್ಮ ಪ್ರಾಣವನ್ನು ತೃಣಕ್ಕಿಂತ ಕಡೆಯಾಗಿ ಅರ್ಪಣ ಮಾಡಬಲ್ಲಿರೋ, ಹಾಗೆಯೇ ಭಾರತೀಯ ದೇವರ ವಿಷಯ ದಲ್ಲಿ ಧೈರ್ಯವನ್ನು ತೋರುವನು. ಯಾರಾದರೂ ಒಂದು ತತ್ತ್ವವನ್ನು ಇಂದು ಬೋಧಿಸಿದರೆ ಅದನ್ನು ಕಾರ್ಯರೂಪಕ್ಕೆ ತರಲೆತ್ನಿಸುವ ನೂರಾರು ಮಂದಿ ನಾಳೆ ಸಿಕ್ಕು ವರು. ಒಂದು ಕಾಲಿನ ಮೇಲೆ ನಿಂತರೆ ಮುಕ್ತಿ ಸಿಕ್ಕುವುದು ಎಂದು ಯಾರಾದರೂ ಬೋಧಿಸಿದರೆ ಅದನ್ನು ಅನುಸರಿಸಲು ಒಂದು ಐನೂರು ಜನರಾದರೂ ತಕ್ಷಣವೇ ಸಿಕ್ಕುವರು. ಇದೊಂದು ಹಾಸ್ಯಾಸ್ಪದ ಎಂದು ನೀವು ಹೇಳಬಹುದು. ಆದರೆ ಇದರ ಹಿಂದೆ ಅವರ ತತ್ತ್ವವಿದೆ. ತತ್ತ್ವವನ್ನು ಅನುಷ್ಠಾನ ಮಾಡಲು ಅವರಲ್ಲಿ ತೀವ್ರ ಆಸಕ್ತಿ ಇದೆ.

ಪಾಶ್ಚಾತ್ಯ ದೇಶಗಳಲ್ಲಿ ಒಬ್ಬನ ಉದ್ಧಾರಕ್ಕೆ ಯೋಜನೆಗಳು ಎಂದರೆ ಒಂದು ಬುದ್ಧಿಯ ಕಸರತ್ತು ಆಗಿದೆ. ಆ ಯೋಜನೆಯನ್ನು ಕಾರ್ಯಗತ ಮಾಡುವುದೇ ಇಲ್ಲ. ಜೀವನಕ್ಕೂ ಅದಕ್ಕೂ ಸಂಬಂಧವೇ ಇಲ್ಲ. ಪಾಶ್ಚಾತ್ಯ ದೇಶದಲ್ಲಿ ಚೆನ್ನಾಗಿ ಮಾತ ನಾಡುವವನೆ ದೊಡ್ಡ ಬೋಧಕ. ಪಾಶ್ಚಾತ್ಯರಿಗೆ ಗಹನವಾದ ಆಧ್ಯಾತ್ಮಿಕ ಸತ್ಯಗಳನ್ನು ತಿಳಿದುಕೊಳ್ಳಲು ತುಂಬಾ ಕಾಲ ಬೇಕಾಗುವುದು. ಪ್ರತಿಯೊಂದು ಅವರಿಗೆ ರೂಪಾಯಿ ಆಣೆ ಪೈಗಳು. ಯಾವುದಾದರೂ ಧರ್ಮ ಅವರಿಗೆ ಐಶ್ವರ್ಯ ಆರೋಗ್ಯ ಸೌಂದರ್ಯ ದೀರ್ಘಾಯುಸ್ಸು ಇವುಗಳನ್ನೇ ಕೊಟ್ಟರೆ ಎಲ್ಲರೂ ಅತ್ತ ಕಡೆ ಧಾವಿಸು ವರು. ಇಲ್ಲದೇ ಇದ್ದರೆ ಯಾರೂ ಅತ್ತ ಕಡೆ ಸುಳಿಯುವುದಿಲ್ಲ. ಹೇಗೆ ಪಾಶ್ಚಾತ್ಯ ತನ್ನ ಜೀವನದಲ್ಲಿ ಸುಖವನ್ನು ಕಾಪಾಡಿಕೊಳ್ಳಲು ಯತ್ನಿಸುವನೊ ಹಾಗೆಯೇ ಭಾರತೀಯ ತನ್ನ ಜೀವನದಲ್ಲಿ ಉನ್ನತ ಮಟ್ಟದ ಆಧ್ಯಾತ್ಮಿಕ ಆದರ್ಶವನ್ನು ವ್ಯಕ್ತಗೊಳಿಸಲು ಯತ್ನಿಸುವನು. ಧರ್ಮವೆಂದರೆ ಬರೀ ಮಾತಲ್ಲ. ನಿತ್ಯಜೀವನ ದಲ್ಲಿ ಪ್ರತಿಯೊಂದು ಚೂರನ್ನು ಅನುಷ್ಠಾನಕ್ಕೆ ತರಲು ಸಾಧ್ಯ ಎಂಬುದನ್ನು ಅವನು ತೋರಲೆತ್ನಿಸುವನು.

ಜೀವನ ವಿಕಾಸವಾಗಬೇಕಾದರೆ ಸ್ವಾತಂತ್ರ್ಯ ಅತ್ಯಂತ ಆವಶ್ಯಕ. ನಿಮ್ಮ ಪೂರ್ವಿ ಕರು ಆತ್ಮನ ಬೆಳವಣಿಗೆಗೆ ಎಲ್ಲಾ ಸ್ವಾತಂತ್ರ್ಯವನ್ನು ಕೊಟ್ಟರು. ಇದರಿಂದ ಧರ್ಮ ಬೆಳೆಯಿತು. ಸಮಾಜವನ್ನು ಅಷ್ಟದಿಗ್ ಬಂಧನಗಳಿಂದ ಕಟ್ಟಿದರು. ಆದ ಕಾರಣ ಅದು ಬೆಳೆಯಲಿಲ್ಲ. ಪಾಶ್ಚಾತ್ಯ ದೇಶದಲ್ಲಿ ಇದಕ್ಕೆ ವಿರೋಧವಾಗಿದೆ. ಸಮಾಜದ ಬೆಳವಣಿಗೆಗೆ ಎಲ್ಲಾ ಸ್ವಾತಂತ್ರ್ಯವನ್ನು ಕೊಟ್ಟರು. ಧರ್ಮಕ್ಕೆ ಯಾವ ಸ್ವಾತಂತ್ರ್ಯ ವನ್ನೂ ಕೊಡಲಿಲ್ಲ. ಪಾಶ್ಚಾತ್ಯ ದೇಶಕ್ಕೆ ಸಮಾಜ ಸುಧಾರಣೆ ಮೂಲಕ ಅಧ್ಯಾತ್ಮ ಬೇಕಾಗಿದೆ. ಭಾರತಕ್ಕೆ ಅಧ್ಯಾತ್ಮದ ಮೂಲಕ ಸಮಾಜ ಸುಧಾರಣೆ ಬೇಕಾಗಿದೆ.

ಅಧ್ಯಾತ್ಮಜ್ಞಾನವೊಂದೇ ನಮ್ಮ ಅಜ್ಞಾನವನ್ನು ಎಂದೆಂದಿಗೂ ಪರಿಹರಿಸ ಬಲ್ಲದು. ಇತರ ಜ್ಞಾನಗಳು ನಮ್ಮ ತಾತ್ಕಾಲಿಕ ಬಯಕೆಗಳನ್ನು ಮಾತ್ರ ತೃಪ್ತಿಪಡಿಸ ಬಲ್ಲವು. ಅಧ್ಯಾತ್ಮ ವಿದ್ಯೆಯಿಂದ ಮಾತ್ರ ನಮ್ಮ ಆಸೆ ಕೈಗೂಡಬೇಕಾದರೆ. ನಮ್ಮ ದೇಹದ ಮೂಲಕ ಎಷ್ಟೋ ಸಾಹಸಗಳನ್ನು ಮಾಡಬಹುದು. ವೈಜ್ಞಾನಿಕ ಪ್ರಯೋಗ ದಿಂದ ಬುದ್ಧಿಯು ಯಂತ್ರಗಳ ಮೂಲಕ ಕಂಡುಹಿಡಿದಿರುವ ವಸ್ತುಗಳು ಅದ್ಭುತ ವಾಗಿವೆ. ಆದರೆ ಇವುಗಳಾವುವೂ ಆತ್ಮ ಪ್ರಭಾವಕ್ಕೆ ಸರಿದೂಗಲಾರವು.

ಯಂತ್ರಗಳು ಎಂದಿಗೂ ಮನುಷ್ಯನನ್ನು ಸುಖಿಯಾಗಿ ಮಾಡಿಲ್ಲ. ಮುಂದೆಯೂ ಮಾಡಲಾರವು. ಯಾರು ಇದನ್ನು ನಂಬುವರೊ ಅವರು ಸಂತೋಷವು ಯಂತ್ರದಲ್ಲಿಲ್ಲ, ಅದು ಮನಸ್ಸಿನಲ್ಲಿದೆ ಎನ್ನುವರು. ಯಾರು ತನ್ನ ಮನಸ್ಸನ್ನು ನಿಗ್ರಹಿಸಿರುವನೊ ಅವನು ಮಾತ್ರ ಸುಖಿಯಾಗಿರಬಲ್ಲ. ಇತರರಲ್ಲ. ಇವತ್ತೆಲ್ಲ ಯಂತ್ರ ಏನು ಮಾಡಬಹುದು! ಯಾರು ಒಂದು ತಂತಿಯ ಮೂಲಕ ವಿದ್ಯುತ್​ಶಕ್ತಿಯನ್ನು ಕಳುಹಿಸಬಲ್ಲನೊ ಅವನನ್ನು ಒಬ್ಬ ಬಹಳ ಬುದ್ಧಿವಂತ ಮತ್ತು ಮಹಾವ್ಯಕ್ತಿ ಎಂದು ಏತಕ್ಕೆ ಕರೆಯಬೇಕು? ಪ್ರತಿ ಕ್ಷಣವೂ ಪ್ರಕೃತಿ ಇದ ಕ್ಕಿಂತ ಕೋಟಿಪಾಲು ಹೆಚ್ಚಾಗಿ ಇದನ್ನು ಮಾಡುತ್ತಿಲ್ಲವೆ? ಪ್ರಕೃತಿಗೆ ಏತಕ್ಕೆ ಅಡ್ಡ ಬಿದ್ದು ನಮಸ್ಕಾರಮಾಡಬಾರದು? ಪ್ರಪಂಚವೆಲ್ಲ ನಿಮ್ಮ ಕೈಕೆಳಗೆ ಇದ್ದರೇನು? ಅದರಲ್ಲಿರುವ ಪ್ರತಿಯೊಂದು ಕಣವೂ ನಿಮ್ಮ ಅಧೀನದಲ್ಲಿದ್ದರೇನು? ನಿಮ್ಮ ನಿಮ್ಮನ್ನೇ ಗೆದ್ದು ನಿಮ್ಮ ಹೃದಯದಲ್ಲಿ ಆನಂದವನ್ನು ಪಡೆಯದೆ ಇದ್ದರೆ ಇವು ಗಳಾವುವೂ ನಿಮಗೆ ಆನಂದವನ್ನು ನೀಡಲಾರವು.

ಮನುಷ್ಯ ಪ್ರಕೃತಿಯನ್ನು ಗೆಲ್ಲುವುದಕ್ಕೆ ಹೊರಟಿರುವನು ಎಂಬುದೇನೊ ನಿಜ. ಪಾಶ್ಚಾತ್ಯನು ಪ್ರಕೃತಿ ಎಂದರೆ ಭೌತಿಕವಾದ ಹೊರಗಿರುವ ಪ್ರಪಂಚ ಎಂದು ಭಾವಿಸುವನು. ಬಾಹ್ಯಪ್ರಕೃತಿಯೇನೊ ಬೆಟ್ಟಗುಡ್ಡಗಳು ನದಿಸಾಗರಗಳು ಮುಂತಾದ ವೈವಿಧ್ಯತೆಗಳಿಂದ ಅತಿ ಮನೋಹರವಾಗಿದೆ. ಆದರೆ ಭವ್ಯತೆಯಲ್ಲಿ ಇದನ್ನು ಮೀರಿಸಿದ ಮನುಷ್ಯನ ಅಂತರ್ಮುಖ ಜೀವನವಿದೆ. ಇದು ಸೂರ್ಯಚಂದ್ರ ತಾರಾವಳಿಗಿಂತ ದೊಡ್ಡದು. ಇದು ನಮ್ಮ ಪೃಥ್ವಿಗಿಂತ ದೊಡ್ಡದು. ನಮ್ಮ ಈಗಿನ ಬಾಳುವೆಯನ್ನು ಮೀರಿ ನಿಂತಿದೆ ಇದು. ಇದೇ ಬೇರೊಂದು ಪ್ರಪಂಚದ ಅಧ್ಯ ಯನ. ಬಾಹ್ಯಪ್ರಕೃತಿಯ ಅಧ್ಯಯನದಲ್ಲಿ ಪಾಶ್ಚಾತ್ಯ ಹೇಗೆ ಮೀರಿಹೋಗುವನೊ ಹಾಗೆಯೇ ಭಾರತೀಯ ಅಂತರ್​ಮುಖ ಜೀವನದ ಸಾಹಸದಲ್ಲಿ ಮೀರಿ ಹೋಗು ವನು.

\textbf{ಮಾನವನ ಪ್ರಗತಿಗೆ ಇವೆರಡೂ ಆವಶ್ಯಕ} ಪಾಶ್ಚಾತ್ಯನಿಗೆ ಬಾಹ್ಯಜಗತ್ತು ಎಷ್ಟು ಸತ್ಯವೊ ಭಾರತೀಯನಿಗೆ ಅಧ್ಯಾತ್ಮ ಜಗತ್ತು ಅಷ್ಟು ಸತ್ಯ. ಭಾರತೀಯನಿಗೆ ಅಧ್ಯಾತ್ಮದಲ್ಲಿ ತನಗೆ ಬೇಕಾದುದೆಲ್ಲ ದೊರಕುವುದು. ಜೀವನವನ್ನು ಸಾರವಸ್ತು ವನ್ನಾಗಿ ಮಾಡುವುದೆಲ್ಲ ಅವನಿಗೆ ಅಲ್ಲಿ ಸಿಕ್ಕುವುದು. ಪಾಶ್ಚಾತ್ಯನಿಗೆ ಭಾರತೀಯ ಒಬಪ್ ಕನಸುಣಿ. ಭಾರತೀಯನಿಗೆ ಪಾಶ್ಚಾತ್ಯ ಒಬಪ್ ಕನಸುಣಿಯಂತೆ ಕಾಣುವನು. ಪ್ರಾಪ್ತವಯಸ್ಸಿಗೆ ಬಂದ ಸ್ತ್ರೀಪುರುಷರು ಇಂದೊ ನಾಳೆಯೋ ಬಿಟ್ಟು ಹೋಗಬೇಕಾ ಗಿರುವ ಕೆಲಸಕ್ಕೆ ಬಾರದ ಇಂದ್ರಿಯವಸ್ತುಗಳಿಗೆ ಇಷ್ಟೊಂದು ಪ್ರಾಮುಖ್ಯತೆಯನ್ನು ಕೊಡುವುದನ್ನು ನೋಡಿ ನಗುವನು. ಪ್ರತಿಯೊಬಪ್ರೂ ಮತ್ತೊಬಪ್ರನ್ನು ಕನಸುಣಿ ಗಳು ಎಂದು ಕರೆಯುವರು. ಮಾನವಕೋಟಿಯ ಪ್ರಗತಿಗೆ ಭಾರತೀಯನ ಭಾವನೆ ಅಷ್ಟೇ ಮುಖ್ಯ. ಅದೊಂದೇ ಅಲ್ಲ ಅದಕ್ಕಿಂತ ಮುಖ್ಯ ಎಂದು ಭಾವಿಸುತ್ತೇನೆ.

ಆದಕಾರಣ ಯಾವಾಗಲಾದರೂ ಆಧ್ಯಾತ್ಮಿಕ ಬದಲಾವಣೆ ಬೇಕಾದಾಗ ಅದು ಯಾವಾಗಲೂ ಭರತಖಂಡದಿಂದ ಬಂದಿದೆ. ಇದರಂತೆಯೇ ಭಾರತೀಯರು ಯಂತ್ರ ಮುಂತಾದುವನ್ನು ಕಲಿಯಬೇಕಾದರೆ ಪಾಶ್ಚಾತ್ಯರ ಶಿಷ್ಯರಾಗಬೇಕಾಗಿದೆ. ಪಾಶ್ಚಾತ್ಯನು ಅಧ್ಯಾತ್ಮ, ದೇವರು, ಜೀವ, ಈ ಸಂಸಾರದ ರಹಸ್ಯ ಮುಂತಾದು ವನ್ನು ಕಲಿಯಬೇಕಾದರೆ ಭಾರತೀಯರ ಶಿಷ್ಯರಾಗಬೇಕಾಗಿದೆ.

ಈ ನಮ್ಮ ಪ್ರಪಂಚ ಶ್ರಮದ ವಿಭಾಗದ ಯೋಜನೆಯ ಮೇಲೆ ನಿಂತಿದೆ. ಒಬ್ಬನಿಗೆ ಎರಡೂ ಸಿಕ್ಕಬೇಕೆಂದು ಬಯಸುವುದು ಅಸಾಧ್ಯ. ಆದರೂ ನಮಗೆ ಎಳ್ಳಷ್ಟೂ ಬುದ್ಧಿಯಿಲ್ಲ. ಮಗು, ಪಾಪ ಅಜ್ಞಾನದಿಂದ ಈ ಪ್ರಪಂಚದಲ್ಲಿ ಬಹಳ ಜೋಪಾನದಿಂದ ರಕ್ಷಿಸಬೇಕಾದ ವಸ್ತು ಎಂದರೆ ತನ್ನ ಗೊಂಬೆ ಎಂದು ತಿಳಿಯು ವುದು. ಇದರಂತೆಯೇ ಯಾವ ದೇಶ ಬಾಳಿನ ಸೌಲಭ್ಯಗಳಲ್ಲಿ ಮುಂದಿದೆಯೋ ಅದು ಇದೊಂದನ್ನೇ ರಕ್ಷಿಸಬೇಕು ಎಂದು ಭಾವಿಸುವುದು. ಪ್ರಗತಿ ಎಂದರೆ ಇವುಗಳನ್ನು ಬೇಕಾದಷ್ಟು ಹೊಂದಿರುವುದು ಎಂದು ತಿಳಿಯುವುದು. ಮುಂದುವರಿ ದಿರುವುದು ಎಂದರೆ ಇವುಗಳಲ್ಲಿ ಮುಂದಿರುವುದು ಎಂದು ಭಾವಿಸುವುದು. ಯಾವ ದೇಶಕ್ಕೆ ಇವುಗಳನ್ನು ಪಡೆಯಬೇಕೆಂಬ ಆಸೆ ಇಲ್ಲವೊ, ಅದನ್ನು ಪಡೆಯುವುದಕ್ಕೆ ಶಕ್ತಿ ಇಲ್ಲವೊ, ಅವರು ಬದುಕುವುದಕ್ಕೆ ಯೋಗ್ಯರಲ್ಲ. ಅವರ ಬಾಳು ವ್ಯರ್ಥ ಎಂದು ಭಾವಿಸುವುದು. ಭೌತಿಕ ಸಂಪತ್ತಿನಲ್ಲಿ ಮುಂದಿರುವ ದೇಶವನ್ನು ನೋಡಿ, ಮತ್ತೊಂದು ದೇಶ ಅದರಿಂದ ಏನೂ ಪ್ರಯೋಜನವಿಲ್ಲ ಎಂದು ಭಾವಿಸುವುದು. ಭರತಖಂಡದಿಂದ ಪ್ರಪಂಚಕ್ಕೆ ಹಿಂದೆ ಈ ಸಂದೇಶ ಬಂತು. ಪ್ರಪಂಚದಲ್ಲಿರುವು ದನ್ನೆಲ್ಲ ಒಬ್ಬ ಹೊಂದಿದ್ದರೂ ಆಧ್ಯಾತ್ಮಿಕ ಸಂಪತ್ತಿಲ್ಲದೆ ಇದ್ದರೆ ಇದರಿಂದ ಏನು ಪ್ರಯೋಜನ? ಇದೇ ಪೌರಾತ್ಯನ ದೃಷ್ಟಿ. ಮತ್ತೊಂದು ಪಾಶ್ಚಾತ್ಯನ ದೃಷ್ಟಿ. ಈ ಎರಡು ಆದರ್ಶಗಳಲ್ಲಿಯೂ ಒಂದು ಆಕರ್ಷಣೆ ಇದೆ. ಒಂದು ಮಹಿಮೆ ಇದೆ. ಈ ಎರಡು ಆದರ್ಶಗಳ ಸಮನ್ವಯವೇ ನಮಗಿಂದು ಬೇಕಾಗಿರುವುದು.

\textbf{ಭಾರತೀಯ ಸಂಸ್ಕೃತಿಯ ಪ್ರಸಾರ}: ನಮ್ಮ ಜಾಗ್ರತ ಜನಾಂಗದ ಜೀವನದ ಉದ್ದೇಶವೇ ಭರತಖಂಡದ ಭಾವನೆ ವಿಶ್ವವನ್ನು ಗೆಲ್ಲಬೇಕಾಗಿರುವುದು. ಆಧ್ಯಾ ತ್ಮಿಕ ಮತ್ತು ತಾತ್ತ್ವಿಕ ಭಾವನೆ ಮತ್ತೊಮ್ಮೆ ಭರತಖಂಡದಿಂದ ಹೊರಗೆ ಹೋಗಿ ಪ್ರಪಂಚವನ್ನು ಗೆಲ್ಲಬೇಕು. ಪ್ರಪಂಚದಲ್ಲಿ ಎಷ್ಟೋ ಮತ್ತೊಬಪ್ರನ್ನು ಆಕ್ರಮಿ ಸಿದ ಜನಾಂಗಗಳಿವೆ. ನಾವೂ ಕೂಡ ಹಿಂದೆ ದೊಡ್ಡ ದಿಗ್ವಿಜಯಿಗಳಾಗಿದ್ದೆವು. ನಮ್ಮ ದಿಗ್ವಿಜಯವನ್ನು ಆ ಪ್ರಖ್ಯಾತನಾದ ಅಶೋಕಚಕ್ರವರ್ತಿ ಧರ್ಮಜಯವೆಂದು ಬಣ್ಣಿಸಿರುವನು. ಮತ್ತೊಮ್ಮೆ ಭರತಖಂಡ ಪ್ರಪಂಚವನ್ನು ಗೆಲ್ಲಬೇಕು.

ಕೆಲವರು ಮುಂಚೆ ನಮ್ಮ ಸಮಸ್ಯೆಗಳನ್ನು ಪರಿಹರಿಸಿಕೊಂಡು ಅನಂತರ ಬೇರೆಯವರಿಗೆ ಬೇಕಾದರೆ ಹೇಳೋಣ ಎನ್ನುವರು. ನೀವು ಯಾವಾಗ ಇತರರಿಗಾಗಿ ಕೆಲಸ ಮಾಡುವಿರೊ ಆಗಲೆ ನೀವು ಚೆನ್ನಾಗಿ ಕೆಲಸ ಮಾಡುವಿರಿ ಎಂದು ನಾನು ಹೇಳುತ್ತೇನೆ. ಪರದೇಶದಲ್ಲಿ ಪರಭಾಷೆಯಲ್ಲಿ ಎಂದು ನೀವು ನಿಮ್ಮ ಭಾವನೆಯನ್ನು ವ್ಯಕ್ತಪಡಿಸಿದಿರೋ ಅಂದೇ ನೀವು ಚೆನ್ನಾಗಿ ಕೆಲಸ ಮಾಡಿದ್ದು ಎಂದು ನಾನು ಹೇಳು ತ್ತೇನೆ.

ಪರದೇಶದವರು ಬಂದು ನಮ್ಮ ದೇಶವನ್ನು ಬೇಕಾದರೆ ಸೈನ್ಯದಿಂದ ತುಂಬಲಿ ಚಿಂತೆಯಿಲ್ಲ. ಭರತಖಂಡವೇ, ನೀನು ಮೇಲೆದ್ದು ಇತರರನ್ನು ಅಧ್ಯಾತ್ಮದಿಂದ ಗೆಲ್ಲು. ಈ ಪವಿತ್ರ ದೇಶದಲ್ಲಿ ಪ್ರೀತಿಯು ದ್ವೇಷವನ್ನು ಗೆಲ್ಲಬೇಕು, ದ್ವೇಷ ದ್ವೇಷ ವನ್ನು ಗೆಲ್ಲುವುದಲ್ಲ. ಚಾರ್ವಾಕವಾದ ಮತ್ತು ಅದರ ಪರಿಣಾಮವಾಗಿ ಬರುವ ದುಃಖವನ್ನೆಲ್ಲ ನಾವು ಚಾರ್ವಾಕದಿಂದಲೇ ಗೆಲ್ಲುವುದಕ್ಕಾಗುವುದಿಲ್ಲ. ಸೈನ್ಯಗಳು ಯಾವಾಗ ಇತರ ದೇಶದ ಸೈನ್ಯಗಳನ್ನು ಗೆಲ್ಲಲು ಯತ್ನಿಸುವುವೋ ಆಗ ಇಬ್ಬರೂ ಮುರ್ಖರಾಗಿ ಹೋಗುವರು. ಅಧ್ಯಾತ್ಮವೇ ಪಾಶ್ಚಾತ್ಯವನ್ನು ಗೆಲ್ಲಬೇಕು. ಕ್ರಮೇಣ ಪಾಶ್ಚಾತ್ಯ ದೇಶಗಳು ತಮ್ಮನ್ನು ಜನಾಂಗದ ದೃಷ್ಟಿಯಿಂದ ರಕ್ಷಿಸುವುದೇ ಅಧ್ಯಾತ್ಮ ಎಂಬುದನ್ನು ಮನಗಾಣುತ್ತಿರುವರು. ಅವರು ಅದಕ್ಕಾಗಿ ಕಾಯುತ್ತಿರುವರು. ಅದಕ್ಕಾಗಿ ತವಕಪಡುತ್ತಿರುವರು. ಅದು ಎಲ್ಲಿಂದ ಬರಬೇಕು? ನಮ್ಮ ಪೂರ್ವ ಕಾಲದ ಪುಷಿಗಳ ಸಂದೇಶವನ್ನು ಪರದೇಶಗಳಿಗೆ ಸಾರಬಲ್ಲ ವ್ಯಕ್ತಿಗಳೆಲ್ಲಿ? ಪ್ರಪಂಚದದಲ್ಲಿ ಮೂಲೆ ಮೂಲೆಗೆ ಸಾರುವುದಕ್ಕೆ ಅದಕ್ಕಾಗಿ ತಮ್ಮ ಸರ್ವಸ್ವವನ್ನು ಅರ್ಪಣೆ ಮಾಡಬಲ್ಲ ವ್ಯಕ್ತಿಗಳೆಲ್ಲಿ? ಸನಾತನ ಸತ್ಯಗಳನ್ನು ಪ್ರಚಾರ ಮಾಡಲು ವೀರರು ಬೇಕಾಗಿರುವರು.

ಪ್ರಪಂಚಕ್ಕೆ ಇದು ಬೇಕಾಗಿದೆ. ಇದಿಲ್ಲದೇ ಇದ್ದರೆ ಪ್ರಪಂಚದ ಸರ್ವನಾಶ ವಾಗುವುದು. ಪಾಶ್ಚಾತ್ಯ ದೇಶವೆಲ್ಲ ಒಂದು ಜ್ವಾಲಾಮುಖಿಯ ನೆತ್ತಿಯ ಮೇಲೆ ಇದೆ. ಅದು ನಾಳೆಯೆ ನುಚ್ಚು ನೂರಾಗಬಹುದು. ಅದರ ನಿವಾರಣೋಪಾಯಕ್ಕೆ ಪ್ರಪಂಚವನ್ನೆಲ್ಲ ಹುಡುಕಿ ನೋಡಿರುವರು. ಆದರೆ ದಾರಿ ಕಾಣದಿದೆ, ಭೋಗದ ಪರಮಾವಧಿಯನ್ನು ಅವರು ಮುಟ್ಟಿರುವರು. ಆದರೆ ಅವರಿಗೆ ಅದರ ಕ್ಷಣಿಕತೆ ವ್ಯಕ್ತವಾಗಿದೆ. ಈಗ ಭರತಖಂಡದ ಭಾವನೆ ಅಲ್ಲಿ ಬೇರು ಬಿಡಲು ಸಾಧ್ಯ. ಅದಕ್ಕಾಗಿ ಪ್ರಯತ್ನ ಪಡಿ. ಇದೇ ಅದಕ್ಕೆ ಸಕಾಲ.

ಆದಕಾರಣ ನಾವು ಹೊರಗೆ ಹೋಗಿ ನಮಗೆ ತಿಳಿದಿರುವ ಆಧ್ಯಾತ್ಮಿಕ ವಿಷಯ ಗಳನ್ನು ಅವರಿಗೆ ಹೇಳೋಣ. ಅವರಿಂದ ಕಲಿಯಬೇಕಾಗಿರುವುದನ್ನು ಕಲಿಯೋಣ. ಆಧ್ಯಾತ್ಮಿಕ ಮಹಾ ನಿಧಿಗಳನ್ನು ಕೊಟ್ಟು ಇಹಲೋಕದ ಮಹಾನಿಧಿಯನ್ನು ಅವ ರಿಂದ ಪಡೆಯೋಣ. ನಾವು ಯಾವಾಗಲೂ ಇತರರಿಂದ ಕಲಿಯುವುದೇ ಅಲ್ಲ. ಇತರರಿಗೆ ನಾವು ಕಲಿಸಲೂ ಬೇಕು. ಸರಿಸಮಾನತೆ ಇಲ್ಲದೆ ಸ್ನೇಹ ಇರಲಾರದು. ಒಬ್ಬ ಯಾವಾಗಲೂ ಕಲಿಸುತ್ತ, ಮತ್ತೊಬ್ಬ ಯಾವಾಗಲೂ ಅವನ ಪದತಳದಲ್ಲಿ ಕುಳಿತುಕೊಂಡು ಕಲಿಯುತ್ತ ಇದ್ದರೆ ಸರಿಸಮಾನತೆ ಇರಲಾರದು. ನೀವು ಕಲಿಸ ಬೇಕು, ಕಲಿತುಕೊಳ್ಳಲೂ ಬೇಕು. ನೀವು ಹಲವು ಶತಮಾನಗಳವರೆಗೆ ಇನ್ನೊಬ್ಬರಿಗೆ ಬೇಕಾದಷ್ಟು ಕಲಿಸುವುದಕ್ಕೆ ಸಾಕಾದಷ್ಟು ನಿಮ್ಮಲ್ಲಿದೆ.

ಪ್ರಪಂಚವನ್ನು ಭರತಖಂಡದ ಆಧ್ಯಾತ್ಮಿಕ ಭಾವನೆಯಿಂದ ಗೆಲ್ಲಬೇಕು ಎಂದರೆ ಏನು ಎಂಬುದನ್ನು ನೀವು ಮರೆಯಕೂಡದು. ಅವರಿಗೆ ನಾವು ಜೀವದಾನ ಮಾಡುವಂತಹ ಸತ್ಯಗಳನ್ನು ಹೇಳಿಕೊಡಬೇಕು. ಬಹು ಕಾಲದಿಂದ ನಾವು ಬಾಚಿ ತಬ್ಬಿಕೊಂಡಿರುವ ಮೂಢಾಚಾರಗಳನ್ನು ಅಲ್ಲ. ಇಲ್ಲಿಯೂ ಕೂಡ ನಾವು ಮೂಢಾ ಚಾರಗಳನ್ನು ಬೇರು ಸಹಿತ ಕಿತ್ತು ಆಚೆಗೆ ಎಸೆಯಬೇಕು. ಅದು ಎಂದೆಂದಿಗೂ ನಿರ್ನಾಮವಾಗಬೇಕು.

ಪಾಶ್ಚಾತ್ಯ ದೇಶಗಳಿಗೆ ನಾವು ಗಹನವಾದ ವೇದಾಂತ ತತ್ತ್ವಗಳನ್ನು ಸಾರಿದರೆ ಆ ದೇಶಗಳ ಗೌರವ ಮತ್ತು ಸಹಾನುಭೂತಿಯನ್ನು ಸಂಪಾದಿಸಬಹುದು. ಆಧ್ಯಾತ್ಮಿಕ ಜೀವನದಲ್ಲಿ ನಾವು ಎಂದೆಂದಿಗೂ ಅವರ ಗುರುಗಳಾಗಬಹುದು. ಭೌತಿಕ ವಿಷಯಗಳಲ್ಲಿ ಅವರು ನಮ್ಮ ಗುರುಗಳಾಗುವರು, ಧಾರ್ಮಿಕ ವಿಷಯಗಳನ್ನು ಕೂಡ ನಾವು ಅವರ ಪದತಳದಲ್ಲಿ ಕುಳಿತುಕೊಂಡು ಕಲಿಯುವಂತಹ ದಿನ ಬಂತು ಎಂದರೆ ನಮ್ಮ ಜನಾಂಗದ ಸರ್ವನಾಶವಾದಂತೆಯೆ ಸರಿ.

ಹಗಲು ರಾತ್ರಿ ಪಾಶ್ಚಾತ್ಯರನ್ನು ನಮಗೆ ಅದನ್ನು ಕೊಡಿ, ಇದನ್ನು ಕೊಡಿ ಎಂದು ಗೋಗರೆಯುತ್ತಿರಬೇಕಾಗಿಲ್ಲ. ಕೊಟ್ಟು ತೆಗೆದುಕೊಳ್ಳುವುದರಿಂದ ಯಾವಾಗ ನಮ್ಮಲ್ಲಿ ಒಂದು ಸಹೃದಯತೆ ಮೂಡುವುದೊ ಆಗ ನಾವು ಇಷ್ಟೊಂದು ಗಲಾಟೆ ಯನ್ನು ಎಬ್ಬಿಸಬೇಕಾಗಿಲ್ಲ. ಅವರು ತಾವೇ ನಮಗೆ ಎಲ್ಲವನ್ನು ಮಾಡುವರು. ನಾವು ವೇದಾಂತ ಧರ್ಮವನ್ನು ಚೆನ್ನಾಗಿ ಅರ್ಥಮಾಡಿಕೊಂಡು ಅನುಭವಿಸಿ ಯಾವಾಗ ಇತರರಿಗೆ ಹೇಳುತ್ತೇವೆಯೊ ಆಗ ಭಾರತೀಯ ಮತ್ತು ಪಾಶ್ಚಾತ್ಯರಿಗೆ ಇಬ್ಬರಿಗೂ ಮೇಲಾಗುವುದು. ಇದರೊಂದಿಗೆ ಹೋಲಿಸಿ ನೋಡಿದರೆ ರಾಜಕೀಯದ ಪ್ರಯತ್ನ ಗಳಿಂದ ನಮಗೆ ಬರುವುದು ಅತ್ಯಲ್ಪ. ನಾನು ಇದನ್ನು ಎಲ್ಲರಿಗೂ ಪ್ರಚಾರ ಮಾಡುವುದಕ್ಕೆ ನನ್ನ ಪ್ರಾಣವನ್ನೇ ಅರ್ಪಿಸುತ್ತೇನೆ.

