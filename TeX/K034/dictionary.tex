\sethyphenation{kannada}{
ಅಂಗ-ವಾ-ಗಿ-ರ-ಬೇಕು
ಅಂಗ-ವಾದ
ಅಂಗ-ಸಾ-ಧ-ನೆಯ
ಅಂಜ-ಬೇ-ಕಾ-ಗಿಲ್ಲ
ಅಂಜಿಕೆ
ಅಂತ
ಅಂತರ
ಅಂತ-ರಾ-ಳ-ದಲ್ಲಿ
ಅಂತ-ರಾ-ಳ-ದ-ಲ್ಲಿಯೂ
ಅಂತ-ರಾ-ಳ-ದ-ಲ್ಲಿ-ರುವ
ಅಂತ-ರಾ-ಳ-ದಿಂದ
ಅಂತ-ರಾ-ಳ-ವನ್ನು
ಅಂತ-ರ್ಮುಖ
ಅಂತ-ರ್ಯಾ-ಮಿ-ಯಾಗಿಯೂ
ಅಂತ-ರ್ಯಾ-ಮಿ-ಯಾದ
ಅಂತ-ಶ್ಶಕ್ತಿ-ಯನ್ನು
ಅಂತಹ
ಅಂತ-ಹವ
ಅಂತ-ಹ-ವರು
ಅಂತೂ
ಅಂತ್ಯ
ಅಂತ್ಯ-ಕರ್ಮ
ಅಂತ್ಯ-ಜ-ನಾ-ದರೂ
ಅಂತ್ಯದ
ಅಂತ್ಯವೇ
ಅಂಥವ-ನನ್ನು
ಅಂದರೆ
ಅಂದಿ
ಅಂದಿ-ನಿಂದ
ಅಂದು
ಅಂದೇ
ಅಂಧ
ಅಂಧ-ರಾಗಿ
ಅಂಶ-ಗಳನ್ನೆಲ್ಲ
ಅಂಶ-ಗ-ಳಿವೆ
ಅಂಶ-ಗಳು
ಅಖಂಡ
ಅಗತೆ
ಅಗೌ-ರ-ವ-ದಿಂದ
ಅಗ್ನಿ-ಕುಂ-ಡದ
ಅಗ್ನಿ-ಕುಂ-ಡ-ವನ್ನು
ಅಗ್ರ-ಭಾಗ
ಅಚ-ಲ-ವಾದ
ಅಜೀ-ರ್ಣ-ದಿಂದ
ಅಜೀ-ರ್ಣ-ವಾಗಿ
ಅಜೀ-ರ್ಣ-ವಾ-ಗು-ವುದು
ಅಜ್ಜ
ಅಜ್ಞಾ-ತ-ರಾಗಿ
ಅಜ್ಞಾನ
ಅಜ್ಞಾನ-ದಿಂದ
ಅಜ್ಞಾನ-ವನ್ನು
ಅಜ್ಞಾನ-ವಿದೆ
ಅಜ್ಞಾನ-ವೆಂಬ
ಅಜ್ಞಾನಿ-ಯಾ-ಗಿ-ರ-ಬ-ಹುದು
ಅಡ-ಚ-ಣೆ-ಯುಂ-ಟಾ-ಗಿ-ದೆಯೋ
ಅಡಿಗೆ
ಅಡಿ-ಗೆಯ
ಅಡಿ-ಯಾ-ಳಾ-ಗಿ-ರ-ಬೇಕು
ಅಡ್ಡ
ಅಡ್ಡಿ
ಅಣ-ಕಿಸಿ
ಅಣ-ಕಿ-ಸುವ
ಅಣ-ಕಿ-ಸು-ವಂ-ತಹ
ಅಣಿ
ಅಣಿ-ಯಾಗ
ಅತಿ
ಅತಿ-ರೇಕ
ಅತಿ-ರೇ-ಕಕ್ಕೆ
ಅತೀತ
ಅತೀ-ತ-ರಾ-ಗಿ-ರು-ವರು
ಅತೀ-ತ-ವಾದ
ಅತ್ತ
ಅತ್ತಿ-ರು-ವೆವು
ಅತ್ಯಂತ
ಅತ್ಯ-ದ್ಭು-ತ-ವಾದ
ಅತ್ಯಲ್ಪ
ಅತ್ಯಾ-ಧು-ನಿಕ
ಅತ್ಯುಚ್ಚ
ಅತ್ಯು-ತ್ತಮ
ಅಥ-ರ್ವ-ವೇ-ದದ
ಅಥವಾ
ಅದ-ಕ್ಕಾಗಿ
ಅದ-ಕ್ಕಿಂತ
ಅದ-ಕ್ಕಿಂ-ತಲೂ
ಅದಕ್ಕೂ
ಅದಕ್ಕೆ
ಅದ-ಕ್ಕೆಲ್ಲ
ಅದನ್ನು
ಅದ-ನ್ನೆಲ್ಲ
ಅದನ್ನೇ
ಅದಮ್ಯ
ಅದರ
ಅದ-ರಂತೆ
ಅದ-ರಂ-ತೆಯೆ
ಅದ-ರಂ-ತೆಯೇ
ಅದ-ರದೇ
ಅದ-ರಲ್ಲಿ
ಅದ-ರ-ಲ್ಲಿದೆ
ಅದ-ರ-ಲ್ಲಿಯೆ
ಅದ-ರ-ಲ್ಲಿ-ರುವ
ಅದ-ರ-ಲ್ಲಿ-ರು-ವು-ದ-ನ್ನೆಲ್ಲ
ಅದ-ರಿಂದ
ಅದ-ರಿಂ-ದಲೇ
ಅದ-ರೊ-ಡನೆ
ಅದ-ರೊ-ಳಗೆ
ಅದ-ಲ್ಲದೆ
ಅದಾ-ದ-ಮೇಲೆ
ಅದಿನ್ನೂ
ಅದಿ-ಲ್ಲದೇ
ಅದು
ಅದುಮಿ
ಅದೂ
ಅದೃಷ್ಟ
ಅದೃ-ಷ್ಟ-ವನ್ನು
ಅದೆಲ್ಲ
ಅದೇ
ಅದೇನು
ಅದೊಂದು
ಅದೊಂದೇ
ಅದೋ
ಅದ್ದ-ಬೇ-ಕಾ-ಯಿತು
ಅದ್ಭುತ
ಅದ್ಭು-ತ-ವಾ-ಗಿ-ದ್ದಳು
ಅದ್ಭು-ತ-ವಾದ
ಅದ್ವೈ-ತ-ಭಾ-ವನೆ
ಅಧಃ-ಪಾ-ತಾ-ಳಕ್ಕೆ
ಅಧ-ರ್ಮಕ್ಕೆ
ಅಧಿ-ಕಾರ
ಅಧಿ-ಕಾ-ರದ
ಅಧಿ-ಕಾ-ರ-ಲಾ-ಲಸೆ
ಅಧಿ-ಕಾ-ರ-ವನ್ನು
ಅಧಿ-ಕಾ-ರ-ವಿದೆ
ಅಧಿ-ಕಾ-ರ-ವಿಲ್ಲ
ಅಧಿ-ಕಾ-ರಿ-ಯಾ-ಗಿ-ರು-ವನು
ಅಧೀ-ನ-ಗೊ-ಳಿ-ಸ-ಬೇಕು
ಅಧೀ-ನ-ದ-ಲ್ಲಿ-ದ್ದ-ರೇನು
ಅಧೀ-ನ-ರಾಗಿ
ಅಧೋ-ಗ-ತಿ-ಗಿ-ಳಿದು
ಅಧೋ-ಗ-ತಿಗೆ
ಅಧ್ಯ
ಅಧ್ಯ-ಕ್ಷರು
ಅಧ್ಯ-ಯನ
ಅಧ್ಯ-ಯ-ನ-ದಲ್ಲಿ
ಅಧ್ಯ-ಯ-ನ-ಮಾಡಿ
ಅಧ್ಯ-ಯ-ನ-ವನ್ನು
ಅಧ್ಯಾತ್ಮ
ಅಧ್ಯಾ-ತ್ಮ-ಜ್ಞಾ-ನ-ವೊಂದೇ
ಅಧ್ಯಾ-ತ್ಮದ
ಅಧ್ಯಾ-ತ್ಮ-ದಲ್ಲಿ
ಅಧ್ಯಾ-ತ್ಮ-ದಿಂದ
ಅಧ್ಯಾ-ತ್ಮ-ವನ್ನು
ಅಧ್ಯಾ-ತ್ಮವೇ
ಅನಂತ
ಅನಂ-ತರ
ಅನಂ-ತ-ರವೂ
ಅನಂ-ತ-ರವೆ
ಅನಂ-ತ-ರವೇ
ಅನಂ-ತ-ವಾ-ದುದು
ಅನಂ-ತಾ-ತ್ಮನ
ಅನ-ವ-ರತ
ಅನಾ-ಚಾ-ರ-ದಿಂದ
ಅನಾ-ಥರ
ಅನಾ-ದಿ-ಕಾ-ಲ-ದಿಂ-ದಲೂ
ಅನಾ-ರ್ಯ-ಜು-ಷ್ಟ-ಮ-ಸ್ವ-ರ್ಗ್ಯ-ಮ-ಕೀ-ರ್ತಿ-ಕ-ರ-ಮ-ರ್ಜುನ
ಅನಾ-ರ್ಯ-ರಾ-ಗ-ಬ-ಹುದು
ಅನಾ-ವ-ಶ್ಯಕ
ಅನಾ-ಸ-ಕ್ತ-ನಾ-ಗಿ-ರಲು
ಅನಾ-ಸ-ಕ್ತ-ರಾದ
ಅನಾ-ಸಕ್ತಿ
ಅನಾ-ಸ-ಕ್ತಿ-ಯನ್ನು
ಅನು
ಅನು-ಕಂಪ
ಅನು-ಕಂಪೆ
ಅನು-ಕ-ರಣೆ
ಅನು-ಕ-ರ-ಣೆ-ಯ-ನ್ನಲ್ಲ
ಅನು-ಕ-ರಿ-ಸ-ಕೂ-ಡದು
ಅನು-ಕ-ರಿ-ಸದೇ
ಅನು-ಕ-ರಿಸಿ
ಅನು-ಕ-ರಿ-ಸಿ-ದರೆ
ಅನು-ಕ-ರಿ-ಸುವ
ಅನು-ಕ-ರಿ-ಸು-ವು-ದಕ್ಕೆ
ಅನು-ಕ-ರಿ-ಸು-ವು-ದಾ-ಗಿದೆ
ಅನು-ಕೂ-ಲಗ
ಅನು-ಗ್ರ-ಹಿ-ಸಿ-ದರೆ
ಅನು-ಭ-ವ-ಗ-ಳಿ-ದ್ದರೆ
ಅನು-ಭ-ವ-ಗಳು
ಅನು-ಭ-ವದ
ಅನು-ಭ-ವ-ದಿಂದ
ಅನು-ಭ-ವ-ವನ್ನು
ಅನು-ಭವಿ
ಅನು-ಭ-ವಿ-ಸಲು
ಅನು-ಭ-ವಿಸಿ
ಅನು-ಭ-ವಿ-ಸಿ-ದರೂ
ಅನು-ಭ-ವಿ-ಸಿ-ದಳು
ಅನು-ಭ-ವಿ-ಸು-ವಳು
ಅನು-ಮಾ-ನ-ವನ್ನು
ಅನು-ರ-ಣಿ-ತ-ವಾ-ಗು-ತ್ತಿ-ರುವ
ಅನು-ರ-ಣಿ-ತ-ವಾ-ಗು-ತ್ತಿವೆ
ಅನು-ಷ್ಠಾನ
ಅನು-ಷ್ಠಾ-ನಕ್ಕೆ
ಅನು-ಸ-ರಿ-ಸ-ಬೇ-ಕಾ-ಗಿದೆ
ಅನು-ಸ-ರಿ-ಸ-ಬೇಕು
ಅನು-ಸ-ರಿ-ಸಲಿ
ಅನು-ಸ-ರಿ-ಸಲು
ಅನು-ಸ-ರಿಸಿ
ಅನು-ಸ-ರಿ-ಸು-ವಂತೆ
ಅನು-ಸ-ರಿ-ಸು-ವುವು
ಅನು-ಸಾ-ರ-ವಾಗಿ
ಅನೇಕ
ಅನೇ-ಕರು
ಅನ್ನ-ದಂತೆ
ಅನ್ನ-ವನ್ನು
ಅನ್ನಾ-ಹಾ-ರಾ-ದಿ-ಗಳನ್ನು
ಅನ್ನಿ-ಸು-ವುದು
ಅನ್ಯ
ಅನ್ಯ-ದೇ-ಶೀ-ಯ-ರಿಗೆ
ಅನ್ಯ-ಧ-ರ್ಮಕ್ಕೆ
ಅನ್ಯ-ಧ-ರ್ಮ-ದ-ವರು
ಅನ್ಯರ
ಅನ್ಯ-ರ-ಲ್ಲಿ-ರುವ
ಅನ್ಯ-ರಾ-ಷ್ಟ್ರ-ಗಳು
ಅನ್ಯ-ರಾ-ಷ್ಟ್ರ-ಗ-ಳೊಂ-ದಿಗೆ
ಅನ್ಯಾ-ಯ-ವಾ-ಗಲಿ
ಅನ್ಯಾ-ಯ-ವಾ-ಗಿಯೋ
ಅನ್ಯಾ-ಯ-ವಾದ
ಅಪ-ಕಾ-ರಕ್ಕೆ
ಅಪ-ಕಾ-ರ-ವನ್ನು
ಅಪ-ಕೀ-ರ್ತಿಯ
ಅಪ-ಪ್ರ-ಯೋಗ
ಅಪ-ರಾ-ಧ-ವಲ್ಲ
ಅಪ-ರೂ-ಪವೋ
ಅಪಾಯ
ಅಪಾ-ಯ-ಕಾರಿ
ಅಪಾ-ಯ-ಗಳು
ಅಪಾ-ಯ-ಬಂ-ದರೂ
ಅಪಾ-ಯ-ವಿದೆ
ಅಪೂರ್ವ
ಅಪ್ಪಣೆ
ಅಪ್ಪ-ಣೆ-ಯನ್ನು
ಅಪ್ರ-ತಿ-ಹ-ತ-ವಾಗಿ
ಅಪ್ರ-ತಿ-ಹ-ತ-ವಾ-ಗಿದೆ
ಅಪ್ರ-ತ್ಯ-ಕ್ಷ-ವಾ-ಗಿಯೋ
ಅಪ್ರ-ಯೋ-ಜ-ಕ-ರಾ-ಗಿಯೇ
ಅಬ್ರಾ-ಹ್ಮ-ಣ-ರನ್ನು
ಅಭಾವ
ಅಭಾ-ವ-ದಿಂದ
ಅಭಾ-ವ-ದಿಂ-ದಲೇ
ಅಭಾ-ವವೇ
ಅಭಿ-ಪ್ರಾಯ
ಅಭಿ-ಪ್ರಾ-ಯಕ್ಕೆ
ಅಭಿ-ಪ್ರಾ-ಯ-ಗಳನ್ನು
ಅಭಿ-ಮಾನ
ಅಭಿ-ಮಾ-ನ-ದಿಂದ
ಅಭಿ-ಮಾ-ನ-ವಿ-ರಲಿ
ಅಭಿ-ವೃದ್ಧಿ
ಅಭಿ-ವೃ-ದ್ಧಿ-ಪ-ಡಿ-ಸ-ಬೇಕು
ಅಭಿ-ವೃ-ದ್ಧಿ-ಯಾ-ಗು-ವು-ದಕ್ಕೆ
ಅಭೀಃ
ಅಭೀಪ್ಸೆ
ಅಭ್ಯಾಸ
ಅಭ್ಯಾ-ಸ-ಗಳನ್ನೆಲ್ಲ
ಅಭ್ಯಾ-ಸ-ದಿಂದ
ಅಭ್ಯಾ-ಸ-ಮಾ-ಡದೆ
ಅಭ್ಯು-ದ-ಯದ
ಅಮ-ರತ್ವ
ಅಮ-ರ-ತ್ವದ
ಅಮ-ರ-ತ್ವ-ದಲ್ಲಿ
ಅಮ-ರ-ವಾಗಿ
ಅಮ-ರ-ವಾ-ಗಿದೆ
ಅಮೃತ
ಅಮೃ-ತ-ಪ್ರ-ವಾ-ಹ-ಇ-ತರ
ಅಮೃ-ತ-ವನ್ನು
ಅಮೃ-ತ-ವಾರಿ
ಅಮೃ-ತ-ಸ್ವ-ರೂಪ
ಅಮೆ-ರಿಕ
ಅಮೆ-ರಿಕಾ
ಅಮೆ-ರಿ-ಕಾ-ದೇ-ಶಕ್ಕೆ
ಅಮೆ-ರಿ-ಕಾ-ದೇ-ಶ-ದಲ್ಲಿ
ಅಮೇ-ರಿ-ಕ-ನ್ನ-ರಿಗೆ
ಅಮೋ-ಘ-ವಾ-ಗಿ-ದ್ದರೂ
ಅಯು-ಕ್ತ-ವಾದ
ಅಯೋ-ಗ್ಯವೂ
ಅಯ್ಯೊ
ಅಯ್ಯೋ
ಅರ-ಗಿಳಿ-ಗ-ಳಂತೆ
ಅರ-ಗಿಳಿ-ಯಂತೆ
ಅರ-ಗಿ-ಸಿ-ಕೊಂ-ಡಿಲ್ಲ
ಅರ-ಚಿ-ಕೊ-ಳ್ಳು-ತ್ತಿ-ರು-ವೆವು
ಅರ-ಣ್ಯಾ-ರಣ್ಯ
ಅರಬ್ಬಿ
ಅರ-ಳಿ-ಸು-ವಂ-ತಿದೆ
ಅರ-ಸ-ರಾ-ಗ-ಬ-ಹುದು
ಅರಿತ
ಅರಿ-ತರೆ
ಅರಿ-ತಿ-ರು-ವರು
ಅರಿತು
ಅರಿ-ತು-ಕೊಳ್ಳಿ
ಅರಿ-ತೊ-ಡ-ನೆಯೆ
ಅರಿ-ಯದ
ಅರಿ-ಯ-ಬೇ-ಕಾ-ಗಿದೆ
ಅರಿ-ಯ-ಬೇಕು
ಅರಿ-ಯಿರಿ
ಅರಿ-ಯು-ವು-ದ-ಕ್ಕಾಗಿ
ಅರಿ-ವಾ-ಗು-ವುದು
ಅರು-ಣೋ-ದ-ಯ-ವಾಗು
ಅರ್ಂ-ತ-ಮುಖ
ಅರ್ಜುನ
ಅರ್ಜು-ನನ
ಅರ್ಜು-ನ-ನಿಗೆ
ಅರ್ಜು-ನ-ನೊ-ಬ್ಬ-ನನ್ನು
ಅರ್ಥ
ಅರ್ಥ-ಮಾಡಿ
ಅರ್ಥ-ಮಾ-ಡಿ-ಕೊಂಡು
ಅರ್ಥ-ಮಾ-ಡಿ-ಕೊ-ಳ್ಳ-ಬ-ಲ್ಲಿರಿ
ಅರ್ಥ-ಮಾ-ಡಿ-ಕೊ-ಳ್ಳು-ವು-ದಕ್ಕೆ
ಅರ್ಥ-ವಲ್ಲ
ಅರ್ಥ-ವಾ-ಗದೆ
ಅರ್ಥ-ವಾ-ಗುವ
ಅರ್ಥ-ವಿಲ್ಲ
ಅರ್ಥ-ವಿ-ಲ್ಲದ
ಅರ್ಧ
ಅರ್ಧ-ಭಾ-ಗಕ್ಕೆ
ಅರ್ಧ-ಭಾ-ಗ-ವಾದ
ಅರ್ಪಣ
ಅರ್ಪಣೆ
ಅರ್ಪಿಸಿ
ಅರ್ಪಿ-ಸಿ-ಕೊ-ಳ್ಳ-ಲಾರ
ಅರ್ಪಿ-ಸು-ತ್ತೇನೆ
ಅರ್ಹತೆ
ಅಲಂ-ಕ-ರಿ-ಸಿ-ಕೊ-ಳ್ಳ-ಬ-ಹುದು
ಅಲೆ
ಅಲೆ-ಗಳು
ಅಲೆ-ಗ್ಸಾಂ-ಡ್ರಿಯ
ಅಲೆ-ಯಂತೆ
ಅಲೆ-ಯನ್ನು
ಅಲೆ-ಯು-ತ್ತಿ-ರು-ವುದೇ
ಅಲ್ಪವೆ
ಅಲ್ಪ-ಸಂ-ಖ್ಯಾ-ತರ
ಅಲ್ಪಾಯು
ಅಲ್ಲ
ಅಲ್ಲ-ಗ-ಳೆ-ದೆವು
ಅಲ್ಲ-ಗ-ಳೆ-ಯ-ಬೇ-ಕಾ-ಗಿಲ್ಲ
ಅಲ್ಲ-ಗ-ಳೆ-ಯ-ಬೇಡಿ
ಅಲ್ಲ-ಗ-ಳೆ-ಯು-ತ್ತೀರಿ
ಅಲ್ಲದೇ
ಅಲ್ಲಿ
ಅಲ್ಲಿಂದ
ಅಲ್ಲಿಗೆ
ಅಲ್ಲಿಯ
ಅಲ್ಲಿ-ಯ-ವ-ರೆಗೆ
ಅಲ್ಲಿ-ರುವ
ಅಳ-ಕೂ-ಡದು
ಅಳಿಲು
ಅಳಿ-ಸಿ-ರು-ವನೊ
ಅಳು
ಅಳು-ಮೋರೆ
ಅಳು-ಮೋ-ರೆ-ಯ-ವ-ನಿಗೆ
ಅಳು-ವು-ದಕ್ಕೆ
ಅಳು-ವು-ದೊಂದೇ
ಅವ
ಅವ-ಕಾಶ
ಅವ-ಕಾ-ಶ-ಗ-ಳಿವೆ
ಅವ-ಕಾ-ಶ-ವನ್ನು
ಅವ-ಕಾ-ಶ-ವನ್ನೂ
ಅವ-ಕಾ-ಶವೇ
ಅವಕ್ಕೆ
ಅವ-ಗು-ಣ-ದಿಂದ
ಅವ-ಗು-ಣವೇ
ಅವ-ತಾರ
ಅವನ
ಅವ-ನತಿ
ಅವ-ನ-ತಿಗೆ
ಅವ-ನ-ತಿ-ಗೊಂದು
ಅವ-ನ-ತಿಯ
ಅವ-ನ-ತಿ-ಯಿಂದ
ಅವ-ನನ್ನು
ಅವ-ನಲ್ಲಿ
ಅವ-ನ-ಲ್ಲಿ-ರುವ
ಅವ-ನಿಂದ
ಅವ-ನಿಗೆ
ಅವ-ನಿ-ಗೇನೂ
ಅವನು
ಅವ-ನುಟ್ಟ
ಅವನೇ
ಅವ-ನೊಂದು
ಅವ-ನೊಬ್ಬ
ಅವನ್ನು
ಅವ-ಮಾ-ನ-ಗಳನ್ನು
ಅವರ
ಅವ-ರದು
ಅವ-ರನ್ನು
ಅವ-ರಲ್ಲಿ
ಅವ-ರ-ಲ್ಲಿ-ರುವ
ಅವ-ರ-ವರ
ಅವ-ರಿಂದ
ಅವ-ರಿ-ಗಾಗಿ
ಅವ-ರಿಗೂ
ಅವ-ರಿಗೆ
ಅವ-ರಿ-ಬ್ಬರೂ
ಅವರು
ಅವರೆ
ಅವರೆ-ದು-ರಿಗೆ
ಅವರೆ-ದು-ರಿಗೇ
ಅವರೆಲ್ಲ
ಅವರೇ
ಅವ-ರೇನು
ಅವ-ರೊಂ-ದಿಗೆ
ಅವ-ರೊ-ಡನೆ
ಅವಳ
ಅವ-ಳನ್ನು
ಅವ-ಳಿಗೆ
ಅವ-ಳೀಗ
ಅವಳು
ಅವಳೇ
ಅವ-ಳೇ-ನಾ-ದರೂ
ಅವ-ಳೇನು
ಅವಶ್ಯ
ಅವ-ಸರ
ಅವ-ಸ-ರ-ಪ-ಡ-ಬೇಡಿ
ಅವ-ಸ್ಥೆ-ಯ-ಲ್ಲಿ-ದ್ದರೂ
ಅವ-ಸ್ಥೆ-ಯಿಂದ
ಅವ-ಹೇ-ಳನ
ಅವ-ಹೇ-ಳ-ನ-ಮಾ-ಡಿ-ಕೊ-ಳ್ಳು-ತ್ತಿರು
ಅವಿ-ತಿ-ರು-ವನು
ಅವಿ-ನಾ-ಶ-ವಾದ
ಅವಿ-ನಾಶಿ
ಅವು
ಅವು-ಗಳ
ಅವು-ಗ-ಳಂತೆ
ಅವು-ಗ-ಳಂ-ತೆಯೆ
ಅವು-ಗಳನ್ನು
ಅವು-ಗಳನ್ನೆಲ್ಲಾ
ಅವು-ಗಳಲ್ಲಿ
ಅವು-ಗಳಿಂದ
ಅವು-ಗ-ಳಿಗೆ
ಅವು-ಗಳು
ಅವು-ಗ-ಳೆಲ್ಲ
ಅವು-ಗ-ಳೊಂ-ದಿಗೆ
ಅವೆಲ್ಲ
ಅಶೋ-ಕ-ಚ-ಕ್ರ-ವರ್ತಿ
ಅಷ್ಟ
ಅಷ್ಟ-ದಿಗ್
ಅಷ್ಟ-ದಿ-ಗ್ಬಂ-ಧ-ನ-ಗಳನ್ನು
ಅಷ್ಟು
ಅಷ್ಟೂ
ಅಷ್ಟೆ
ಅಷ್ಟೇ
ಅಷ್ಟೊಂದು
ಅಸಡ್ಡೆ
ಅಸ-ತ್ಯ-ಕ್ಕಿಂತ
ಅಸ-ದಳ
ಅಸ-ಹಿ-ಷ್ಣುತೆ
ಅಸಾಧ್ಯ
ಅಸೂ-ಯಾ-ಭಾ-ವನೆ
ಅಸೂಯೆ
ಅಸೂ-ಯೆಯ
ಅಸ್ತ-ವ್ಯಸ್ತ
ಅಸ್ತ-ವ್ಯ-ಸ್ತ-ವಾ-ಗು-ವುದು
ಅಸ್ತ್ರ-ದಂ-ತಿದೆ
ಅಸ್ಥಿ-ಮಾಂ-ಸ-ಗ-ಳಿಗೆ
ಅಸ್ಪೃ-ಶ್ಯತೆ
ಅಹಂ-ಕಾರ
ಅಹ-ಲ್ಯಾ-ಬಾಯಿ
ಅಹಿಂಸೆ
ಆ
ಆಂಗ್ಲೇಯ
ಆಂಗ್ಲೇ-ಯರು
ಆಂತ-ರ್ಯ-ದ-ಲ್ಲಿ-ರುವ
ಆಕ-ರ್ಷಣೆ
ಆಕ-ರ್ಷ-ಣೆಗೆ
ಆಕಾ-ಶ-ದಷ್ಟು
ಆಕಾ-ಶ-ವನ್ನು
ಆಕೆ
ಆಕೆ-ಯನ್ನು
ಆಕ್ರ-ಮಣ
ಆಕ್ರ-ಮ-ಣಕ್ಕೆ
ಆಕ್ರ-ಮ-ಣ-ಗಳು
ಆಕ್ರಮಿ
ಆಕ್ರ-ಮಿ-ಸ-ದ-ವರು
ಆಕ್ರ-ಮಿ-ಸಲು
ಆಗ
ಆಗದೆ
ಆಗ-ಬೇ-ಕಾ-ಗಿದೆ
ಆಗ-ಬೇ-ಕಾ-ದರೆ
ಆಗ-ಬೇ-ಕಾ-ಯಿತೋ
ಆಗ-ಬೇಕು
ಆಗ-ಲಾ-ರದು
ಆಗಲಿ
ಆಗ-ಲಿಲ್ಲ
ಆಗಲೆ
ಆಗಲೇ
ಆಗ-ಲೇ-ಬೇ-ಕಾ-ಗಿದೆ
ಆಗಿ
ಆಗಿದೆ
ಆಗಿ-ದೆಯೆ
ಆಗಿ-ದ್ದರೆ
ಆಗಿ-ದ್ದಾನೆ
ಆಗಿದ್ದು
ಆಗಿಯೇ
ಆಗಿ-ರ-ಬೇ-ಕಾ-ಗಿಲ್ಲ
ಆಗಿ-ರ-ಲಿಲ್ಲ
ಆಗಿ-ರುವ
ಆಗಿ-ರು-ವುದು
ಆಗಿ-ರು-ವೆವು
ಆಗಿಲ್ಲ
ಆಗಿವೆ
ಆಗಿ-ಹೋ-ಗು-ತ್ತೇವೆ
ಆಗಿ-ಹೋ-ಗು-ವ-ವರು
ಆಗಿ-ಹೋ-ಯಿತು
ಆಗು
ಆಗು-ತ್ತಿದೆ
ಆಗು-ತ್ತಿ-ರ-ಲಿಲ್ಲ
ಆಗು-ತ್ತಿ-ರುವ
ಆಗುವ
ಆಗು-ವನು
ಆಗು-ವರು
ಆಗು-ವುದನ್ನು
ಆಗು-ವು-ದಿಲ್ಲ
ಆಗು-ವುದು
ಆಗು-ವುದೆ
ಆಗು-ವುದೊ
ಆಚಾರ
ಆಚಾ-ರ-ಗಳನ್ನು
ಆಚಾ-ರದ
ಆಚಾ-ರ-ವನ್ನು
ಆಚಾ-ರ-ಶೀಲ
ಆಚಾರ್ಯ
ಆಚಾ-ರ್ಯ-ರು-ಗ-ಳೆಲ್ಲ
ಆಚೆಗೆ
ಆಜನ್ಮ
ಆಜ-ನ್ಮ-ದಿಂದ
ಆಜ-ನ್ಮ-ದುಃ-ಖಿನಿ
ಆಜ-ನ್ಮ-ವಾಗಿ
ಆಜ್ಞೆಗೆ
ಆಜ್ಞೆ-ಯನ್ನು
ಆಡ-ಬೇ-ಕಾ-ಗಿದೆ
ಆಣ-ತಿ-ಯನ್ನು
ಆಣೆ
ಆತಂಕ
ಆತಂ-ಕ-ಗಳನ್ನು
ಆತಂ-ಕ-ಗ-ಳಿ-ದ್ದರೂ
ಆತಂ-ಕ-ಗ-ಳಿವೆ
ಆತಂ-ಕ-ವನ್ನು
ಆತಂ-ಕ-ವಾ-ಗಿದೆ
ಆತ್ಮ
ಆತ್ಮದ
ಆತ್ಮನ
ಆತ್ಮ-ನನ್ನು
ಆತ್ಮ-ನಿಂದ
ಆತ್ಮ-ನಿಂದೆ
ಆತ್ಮ-ರ-ಕ್ಷಣೆ
ಆತ್ಮ-ವತ್
ಆತ್ಮ-ಶ್ರದ್ಧೆ
ಆತ್ಮ-ಶ್ರ-ದ್ಧೆಯ
ಆತ್ಮ-ಶ್ರ-ದ್ಧೆ-ಯನ್ನು
ಆತ್ಮ-ಶ್ರ-ದ್ಧೆ-ಯಿಂದ
ಆತ್ಮಾಭಿ
ಆತ್ಮೀ-ಯರೆ
ಆತ್ಮೋ-ದ್ಧಾ-ರ-ಕ್ಕಾಗಿ
ಆತ್ಮೋ-ನ್ನ-ತಿಗೆ
ಆದ
ಆದ-ಕಾ-ರಣ
ಆದ-ಕಾರಣ-ದಿಂ-ದಲೇ
ಆದ-ಕಾರಣವೇ
ಆದರೂ
ಆದರೆ
ಆದರೋ
ಆದರ್ಶ
ಆದ-ರ್ಶ-ಕ್ಕಿಂತ
ಆದ-ರ್ಶಕ್ಕೆ
ಆದ-ರ್ಶ-ಗಳ
ಆದ-ರ್ಶ-ಗಳನ್ನು
ಆದ-ರ್ಶ-ಗಳಲ್ಲಿ
ಆದ-ರ್ಶ-ಗ-ಳ-ಲ್ಲಿಯೂ
ಆದ-ರ್ಶ-ಗಳಿಂದ
ಆದ-ರ್ಶ-ಗ-ಳಿಗೆ
ಆದ-ರ್ಶ-ಗಳು
ಆದ-ರ್ಶ-ಗ-ಳುಳ್ಳ
ಆದ-ರ್ಶದ
ಆದ-ರ್ಶ-ದಂತೆ
ಆದ-ರ್ಶ-ದಲ್ಲಿ
ಆದ-ರ್ಶ-ದಿಂದ
ಆದ-ರ್ಶ-ನಾ-ರಿ-ಯ-ರಾ-ಗ-ಬ-ಹುದು
ಆದ-ರ್ಶ-ನಾ-ರಿ-ಯ-ರಿಗೆ
ಆದ-ರ್ಶ-ಬ್ರಾ-ಹ್ಮ-ಣರು
ಆದ-ರ್ಶ-ಮಾರ್ಗ
ಆದ-ರ್ಶ-ವನ್ನು
ಆದ-ರ್ಶ-ವಾ-ಗಿ-ರ-ಬೇಕು
ಆದ-ರ್ಶ-ವಿದೆ
ಆದ-ರ್ಶ-ವಿ-ರ-ಬೇಕು
ಆದ-ರ್ಶ-ವೆಲ್ಲ
ಆದ-ರ್ಶವೇ
ಆದಿ
ಆದಿ-ಯಲ್ಲಿ
ಆದಿ-ಶ-ಕ್ತಿಯ
ಆಧಾ-ರಕ್ಕೂ
ಆಧಾ-ರದ
ಆಧಾ-ರ-ದಿಂ-ದಲೇ
ಆಧು-ನಿಕ
ಆಧ್ಯಾ
ಆಧ್ಯಾ-ತ್ಮಿಕ
ಆಧ್ಯಾ-ತ್ಮಿ-ಕತೆ
ಆಧ್ಯಾ-ತ್ಮಿ-ಕ-ತೆಯ
ಆಧ್ಯಾ-ತ್ಮಿ-ಕ-ತೆ-ಯನ್ನು
ಆಧ್ಯಾ-ತ್ಮಿ-ಕ-ತೆ-ಯಲ್ಲಿ
ಆಧ್ಯಾ-ತ್ಮಿ-ಕ-ತೆ-ಯೊಂದೇ
ಆಧ್ಯಾ-ತ್ಮಿ-ಕ-ದಾ-ನ-ವನ್ನು
ಆಧ್ಯಾ-ತ್ಮಿ-ಕ-ವಾಗಿ
ಆನಂದ
ಆನಂ-ದ-ಪಡಿ
ಆನಂ-ದ-ವನ್ನು
ಆನಂ-ದಿ-ಸ-ಬೇಕು
ಆನು-ವಂ-ಶಿ-ಕ-ವಾಗಿ
ಆಪಾ-ದ-ಮ-ಸ್ತಕ
ಆಫೀಸಿ
ಆಫ್ರಿಕಾ
ಆಭ-ರ-ಣ-ಗ-ಳಿವೆ
ಆಮೂ-ಲಾಗ್ರ
ಆಯಿತು
ಆಯ್ಕೆ-ಯಲ್ಲ
ಆಯ್ಕೆಯೆ
ಆರಾ-ಧಕ
ಆರಾ-ಮ-ವಾಗಿ
ಆರಿ-ಸಿ-ಕೊಂ-ಡಿ-ರು-ವೆವು
ಆರಿ-ಸಿ-ಕೊ-ಳ್ಳ-ಬೇ-ಕಾ-ಗಿದೆ
ಆರಿ-ಸು-ವು-ದ-ಕ್ಕಲ್ಲ
ಆರು
ಆರೇಳು
ಆರೋಗ್ಯ
ಆರೋ-ಗ್ಯ-ಕ-ರ-ವಾದ
ಆರೋ-ಗ್ಯಕ್ಕೆ
ಆರೋ-ಪ-ಮಾಡಿ
ಆರ್ಥಿಕ
ಆರ್ಯ
ಆರ್ಯ-ಮ-ಹ-ರ್ಷಿ-ಗಳ
ಆರ್ಯ-ರಲ್ಲಿ
ಆರ್ಯ-ರಾ-ಗ-ಬ-ಹುದು
ಆರ್ಯ-ರಿ-ಗಿಂತ
ಆರ್ಯ-ರಿಗೆ
ಆರ್ಯರು
ಆರ್ಯೇ-ತ-ರರು
ಆಲೋ
ಆಲೋ-ಚನೆ
ಆಲೋ-ಚ-ನೆ-ಗಳು
ಆಲೋ-ಚ-ನೆ-ಯನ್ನು
ಆಲೋ-ಚ-ನೆಯೂ
ಆಲೋ-ಚಿ-ಸ-ತೊ-ಡ-ಗಿ-ದನು
ಆಲೋ-ಚಿ-ಸ-ತೊ-ಡ-ಗಿ-ದರು
ಆಲೋ-ಚಿ-ಸ-ಬೇಕು
ಆಲೋ-ಚಿ-ಸ-ಲಾ-ರ-ದಾ-ಗು-ತ್ತದೆ
ಆಲೋ-ಚಿ-ಸಲಿ
ಆಲೋ-ಚಿಸಿ
ಆಲೋ-ಚಿ-ಸಿ-ದು-ದ-ರಿಂದ
ಆಲೋ-ಚಿ-ಸು-ತ್ತಾನೆ
ಆಲೋ-ಚಿ-ಸು-ವರು
ಆಲೋ-ಚಿ-ಸು-ವಿರೋ
ಆಲೋ-ಚಿ-ಸು-ವು-ದ-ರ-ಲ್ಲಿಯೇ
ಆಳ
ಆಳ-ಆ-ಳಕ್ಕೆ
ಆಳಕ್ಕೆ
ಆಳ-ದ-ಲ್ಲಿಯೂ
ಆಳ-ಬೇಕು
ಆಳ-ಲಾ-ರರು
ಆಳಲು
ಆಳಾ-ಗಿ-ರು-ವರು
ಆಳಾ-ಗು-ವುದನ್ನು
ಆಳಿ-ರು-ವರು
ಆಳು
ಆಳು-ತ್ತಿ-ದ್ದರು
ಆಳು-ವ-ವನು
ಆಳು-ವ-ವರ
ಆಳು-ವ-ವರು
ಆಳುವು
ಆಳು-ವುದ
ಆಳು-ವು-ದಕ್ಕೆ
ಆಳು-ವುದು
ಆಳ್ವಿ-ಕೆಗೆ
ಆವತ್ತು
ಆವರಿ
ಆವ-ರಿ-ಸಿದೆ
ಆವ-ರಿ-ಸಿ-ರುವ
ಆವ-ಶ್ಯಕ
ಆವ-ಶ್ಯ-ಕತೆ
ಆವ-ಶ್ಯ-ಕ-ತೆ-ಗಳನ್ನು
ಆವ-ಶ್ಯ-ಕ-ತೆಗೆ
ಆವ-ಶ್ಯ-ಕ-ತೆ-ಯಿದೆ
ಆವ-ಶ್ಯ-ಕ-ವಾಗಿ
ಆವ-ಶ್ಯ-ಕ-ವಾ-ಗಿತ್ತು
ಆವ-ಶ್ಯ-ಕ-ವಾ-ಗಿದೆ
ಆವ-ಶ್ಯ-ಕ-ವಾದ
ಆವಿ-ರ್ಭಾವ
ಆವೃತ
ಆಶಾ-ಜ-ನ-ಕ-ವಾದ
ಆಶಾ-ದಾ-ಯ-ಕ-ವಾಗು
ಆಶಾ-ವಾ-ದಿ-ಗ-ಳಾ-ಗಿ-ದ್ದರು
ಆಶಿಷ್ಠ
ಆಶಿ-ಸು-ವನು
ಆಶಿ-ಸು-ವ-ವ-ನಿಗೆ
ಆಶಿ-ಸು-ವುದು
ಆಶೀ-ರ್ವಾದ
ಆಶೀ-ರ್ವಾ-ದ-ವನ್ನು
ಆಶ್ಚ-ರ್ಯ-ವಾ-ಗು-ವು-ದಿಲ್ಲ
ಆಶ್ಚ-ರ್ಯ-ವೇನೂ
ಆಶ್ರಮ
ಆಶ್ರ-ಮ-ಗಳನ್ನು
ಆಶ್ರ-ಮದ
ಆಶ್ರ-ಯಿ-ಸ-ಲಾ-ರದು
ಆಶ್ರ-ಯಿ-ಸಿ-ಕೊಂಡು
ಆಶ್ರ-ಯಿ-ಸು-ವು-ದಾ-ಗಿದೆ
ಆಶ್ರಿ-ತ-ರ-ನ್ನಾಗಿ
ಆಸಕ್ತಿ
ಆಸೆ
ಆಸೆ-ಪ-ಡು-ವೆವು
ಆಸೆ-ಯನ್ನು
ಆಸ್ಟ್ರೇ-ಲಿಯ
ಆಸ್ತಿ
ಆಸ್ತಿಯ
ಆಸ್ತಿ-ಯನ್ನು
ಆಸ್ಪ-ತ್ರೆಗೆ
ಆಹಾರ
ಆಹಾ-ರ-ಗಳು
ಆಹಾ-ರ-ವನ್ನು
ಇಂಗ್ಲಿ-ಷ್ಭಾಷೆ
ಇಂಗ್ಲೀ-ಷರು
ಇಂಗ್ಲೀ-ಷಿ-ನ-ವ-ರಿಗೆ
ಇಂಗ್ಲೀಷು
ಇಂಗ್ಲೆಂ-ಡಿ-ನ-ಲ್ಲಿ-ರು-ವಂತೆ
ಇಂಗ್ಲೆಂ-ಡು-ಗಳಲ್ಲಿ
ಇಂಗ್ಲೆಂಡ್
ಇಂಡಿಯಾ
ಇಂಡಿ-ಯಾ-ದೇ-ಶ-ದಲ್ಲಿ
ಇಂಡಿ-ಯಾ-ದೇ-ಶವೂ
ಇಂತಹ
ಇಂತ-ಹ-ವರ
ಇಂತ-ಹ-ವ-ರಿಂದ
ಇಂದಿಗೂ
ಇಂದಿನ
ಇಂದಿ-ನ-ದನ್ನು
ಇಂದಿ-ನ-ವ-ರಾ-ಗಲಿ
ಇಂದಿ-ನ-ವ-ರೆಗೆ
ಇಂದು
ಇಂದೂ
ಇಂದೊ
ಇಂದೋ
ಇಂದ್ರಿಯ
ಇಂದ್ರಿ-ಯ-ಗ-ಳಿಗೆ
ಇಂದ್ರಿ-ಯ-ದ-ಲ್ಲಿ-ದೆಯೊ
ಇಂದ್ರಿ-ಯ-ನಿ-ಗ್ರ-ಹವೇ
ಇಂದ್ರಿ-ಯ-ವ-ಸ್ತು-ಗ-ಳಿಗೆ
ಇಚಾ-ಊ-ಇ-್ಕ-ದ್ಶಕ್ತಿ
ಇಚಾ-ಊ-ಇ-್ಕ-ದ್ಶ-ಕ್ತಿಯ
ಇಚಾ-ಊ-ಇ-್ಕ-ದ್ಶ-ಕ್ತಿ-ಯನ್ನು
ಇಚಿ-ಊ-ಇ-್ಕ-ದ್ಸು-ವು-ದಕ್ಕೆ
ಇಚೆ-ಊ-ಇ-್ಕ-ದ್ಯಿ-ಲ್ಲದೇ
ಇಚೆ-ಊ-ಇ-್ಕದ್ಯೇ
ಇಚ್ಚಾ-ಶಕ್ತಿ
ಇಚ್ಛಾ-ನು-ಸಾರ
ಇಚ್ಛಾ-ಶಕ್ತಿ
ಇಚ್ಛಾ-ಶ-ಕ್ತಿಯ
ಇಚ್ಛಿ-ಸು-ವರು
ಇಚ್ಛಿ-ಸು-ವು-ದಿಲ್ಲ
ಇಚ್ಛೆ-ಪ-ಡು-ವು-ದಿಲ್ಲ
ಇಚ್ಛೆ-ಯನ್ನು
ಇಟ್ಟ
ಇಟ್ಟಿ-ದ್ದಕ್ಕೆ
ಇಟ್ಟಿ-ದ್ದರು
ಇಟ್ಟಿರು
ಇಟ್ಟು
ಇಟ್ಟು-ಕೊಂ-ಡಿ-ರ-ಲಾ-ರಿರಿ
ಇಟ್ಟು-ಕೊಂಡು
ಇಡ
ಇಡ-ಬೇಕು
ಇಡ-ಲಾ-ರರು
ಇಡ-ಲಾ-ರಿರಿ
ಇಡಿ
ಇಡೀ
ಇಡು
ಇಡು-ವುದನ್ನು
ಇಡು-ವುದೇ
ಇತರ
ಇತ-ರರ
ಇತ-ರ-ರನ್ನು
ಇತ-ರ-ರಲ್ಲ
ಇತ-ರ-ರಲ್ಲಿ
ಇತ-ರ-ರಾ-ದರೊ
ಇತ-ರ-ರಾರೂ
ಇತ-ರರಿ
ಇತ-ರ-ರಿಂದ
ಇತ-ರ-ರಿ-ಗಾಗಿ
ಇತ-ರ-ರಿ-ಗಿಂತ
ಇತ-ರ-ರಿಗೂ
ಇತ-ರ-ರಿಗೆ
ಇತ-ರರು
ಇತ-ರ-ರೊ-ಡನೆ
ಇತಿ-ಹಾಸ
ಇತಿ-ಹಾ-ಸದ
ಇತಿ-ಹಾ-ಸ-ದಲ್ಲಿ
ಇತಿ-ಹಾ-ಸ-ವನ್ನು
ಇತಿ-ಹಾ-ಸ-ವ-ನ್ನೆಲ್ಲ
ಇತ್ತ
ಇತ್ತಿದೆ
ಇತ್ತೀ-ಚಿ-ನ-ವ-ರೆಗೆ
ಇತ್ತು
ಇದ
ಇದ-ಕ್ಕಾಗಿ
ಇದ-ಕ್ಕಿಂತ
ಇದಕ್ಕೂ
ಇದಕ್ಕೆ
ಇದ-ಕ್ಕೆಲ್ಲ
ಇದನ್ನು
ಇದ-ನ್ನೆಲ್ಲ
ಇದನ್ನೇ
ಇದರ
ಇದ-ರಂ-ತೆಯೇ
ಇದ-ರಲ್ಲಿ
ಇದ-ರ-ಲ್ಲಿಯೇ
ಇದ-ರಿಂ
ಇದ-ರಿಂದ
ಇದ-ರಿಂ-ದಲೇ
ಇದ-ರೊಂ-ದಿಗೆ
ಇದಾ-ಗಲೇ
ಇದಾದ
ಇದಿ-ಲ್ಲದೇ
ಇದು
ಇದು-ವ-ರೆಗೂ
ಇದು-ವ-ರೆಗೆ
ಇದೆ
ಇದೆಂ-ದಾ-ದರೂ
ಇದೆಯೆ
ಇದೆ-ಯೆಂದು
ಇದೆ-ಯೆಂ-ಬು-ದನ್ನು
ಇದೆಯೊ
ಇದೆಯೋ
ಇದೆಲ್ಲ
ಇದೇ
ಇದೇನೋ
ಇದೊಂ-ದನ್ನೇ
ಇದೊಂದು
ಇದೊಂದೇ
ಇದೊಳ್ಳೆ
ಇದ್ದ
ಇದ್ದರೂ
ಇದ್ದರೆ
ಇದ್ದ-ರೇನು
ಇದ್ದಳು
ಇದ್ದಳೆ
ಇದ್ದಾಗ
ಇದ್ದಿತೊ
ಇದ್ದು-ದ-ರಿಂದ
ಇದ್ದುದು
ಇದ್ದೇ
ಇನ್ನಾಕೆ
ಇನ್ನಾವ
ಇನ್ನು
ಇನ್ನೂ
ಇನ್ನೆಷ್ಟು
ಇನ್ನೇ-ನನ್ನು
ಇನ್ನೊ
ಇನ್ನೊಂದು
ಇನ್ನೊ-ಬ-ಪ್ನನ್ನು
ಇನ್ನೊ-ಬಪ್ರ
ಇನ್ನೊಬ್ಬ
ಇನ್ನೊ-ಬ್ಬ-ನಿಗೆ
ಇನ್ನೊ-ಬ್ಬರ
ಇನ್ನೊ-ಬ್ಬ-ರನ್ನು
ಇನ್ನೊ-ಬ್ಬರಿ
ಇನ್ನೊ-ಬ್ಬ-ರಿಗೆ
ಇಪ್ಪತ್ತು
ಇಬಪ್
ಇಬಪ್ರೂ
ಇಬ್ಬ-ರಿಗೂ
ಇಬ್ಬರು
ಇಬ್ಬರೂ
ಇರ
ಇರ-ಕೂ-ಡದು
ಇರ-ಬ-ಲ್ಲದು
ಇರ-ಬ-ಲ್ಲಿರಾ
ಇರ-ಬ-ಲ್ಲೆವು
ಇರ-ಬ-ಹುದು
ಇರ-ಬಾ-ರದು
ಇರ-ಬೇ-ಕಾ-ಗು-ವುದು
ಇರ-ಬೇಕು
ಇರ-ಬೇ-ಕೆಂದು
ಇರ-ಲಾ-ರದು
ಇರಲಿ
ಇರ-ಲಿಲ್ಲ
ಇರು
ಇರು-ತ್ತಿ-ರ-ಲಿಲ್ಲ
ಇರು-ತ್ತೇನೆ
ಇರುವ
ಇರು-ವಂತೆ
ಇರು-ವಂ-ತೆಯೇ
ಇರು-ವನು
ಇರು-ವ-ನೆಂದು
ಇರು-ವರು
ಇರು-ವರೋ
ಇರು-ವಳು
ಇರು-ವ-ವರ
ಇರು-ವ-ವ-ರನ್ನು
ಇರು-ವಷ್ಟು
ಇರು-ವಷ್ಟೇ
ಇರು-ವಾಗ
ಇರು-ವಿರಿ
ಇರು-ವು-ದಕ್ಕೆ
ಇರು-ವು-ದ-ರಿಂದ
ಇರು-ವು-ದಲ್ಲ
ಇರು-ವು-ದಿಲ್ಲ
ಇರು-ವುದು
ಇರು-ವು-ದೆಲ್ಲ
ಇರು-ವುದೇ
ಇರು-ವೆವು
ಇರೋಣ
ಇಲ್ಲ
ಇಲ್ಲದ
ಇಲ್ಲ-ದಂತೆ
ಇಲ್ಲ-ದ-ವರು
ಇಲ್ಲ-ದಿ-ದ್ದರೂ
ಇಲ್ಲ-ದಿ-ರಲಿ
ಇಲ್ಲದೆ
ಇಲ್ಲದೇ
ಇಲ್ಲ-ದೇ-ಹೋ-ಗ-ಬ-ಹುದು
ಇಲ್ಲವೆ
ಇಲ್ಲವೊ
ಇಲ್ಲಿ
ಇಲ್ಲಿಂದ
ಇಲ್ಲಿಗೆ
ಇಲ್ಲಿತ್ತು
ಇಲ್ಲಿದೆ
ಇಲ್ಲಿಯ
ಇಲ್ಲಿಯೂ
ಇಲ್ಲಿ-ರುವ
ಇಲ್ಲೇ
ಇಳಿ
ಇಳಿದ
ಇಳಿ-ದಿ-ದ್ದರು
ಇಳಿ-ದು-ಬ-ರು-ವನು
ಇಳಿಯು
ಇವಕ್ಕೆ
ಇವ-ತ್ತೆಲ್ಲ
ಇವನು
ಇವನ್ನು
ಇವರ
ಇವ-ರನ್ನು
ಇವ-ರಿಗೆ
ಇವ-ರಿ-ಬ್ಬ-ರಲ್ಲಿ
ಇವರು
ಇವರೆಲ್ಲ
ಇವರೇ
ಇವಿ-ದ್ದರೆ
ಇವು
ಇವು-ಗಳ
ಇವು-ಗಳನ್ನು
ಇವು-ಗಳನ್ನೆಲ್ಲ
ಇವು-ಗ-ಳನ್ನೇ
ಇವು-ಗಳಲ್ಲಿ
ಇವು-ಗ-ಳ-ಲ್ಲೆಲ್ಲ
ಇವು-ಗ-ಳಾಚೆ
ಇವು-ಗ-ಳಾ-ವುವೂ
ಇವು-ಗಳಿಂದ
ಇವು-ಗ-ಳಿಗೆ
ಇವು-ಗ-ಳಿ-ದ್ದರೆ
ಇವು-ಗಳು
ಇವು-ಗ-ಳೆಲ್ಲ
ಇವು-ಗ-ಳೊಂ-ದಿಗೆ
ಇವೆ
ಇವೆ-ರ-ಡನ್ನು
ಇವೆ-ರ-ಡ-ರಲ್ಲಿ
ಇವೆ-ರಡೂ
ಇವೆಲ್ಲ
ಇವೇ
ಇಷ್ಟು
ಇಷ್ಟೇ
ಇಷ್ಟೊಂ-ದನ್ನು
ಇಷ್ಟೊಂದು
ಇಹ
ಇಹ-ಪ-ರ-ಭೋ-ಗ-ಗಳನ್ನೆಲ್ಲ
ಇಹ-ಲೋ-ಕದ
ಇಹ-ವನ್ನು
ಈ
ಈಗ
ಈಗಲೂ
ಈಗಲೇ
ಈಗಲೋ
ಈಗಿನ
ಈಗಿ-ನಂತೆ
ಈಗಿ-ನದು
ಈಗಿ-ನ-ವರು
ಈಜಿ-ಪ್ಟ್
ಈಡಾ-ಗಿ-ರು-ವರು
ಈಡಾ-ಗಿ-ರು-ವೆವು
ಈಶ್ವರ
ಈಶ್ವ-ರ-ನಲ್ಲಿ
ಈಶ್ವ-ರೇ-ಚ್ಛೆ-ಯಿದು
ಉಂಟು-ಮಾ-ಡು-ತ್ತದೆ
ಉಕ್ಕಿ-ನಂ-ತಹ
ಉಗಮ
ಉಚಿತ
ಉಚ್ಚ-ಕಂ-ಠ-ದಿಂದ
ಉಚ್ಚ-ರಿ-ಸ-ಬ-ಲ್ಲಿರಾ
ಉಚ್ಚ-ರಿಸಿ
ಉಚ್ಛ್ರಾಯ
ಉಜ್ಜೀ-ವ-ನ-ಗೊ-ಳಿ-ಸ-ಬೇ-ಕಾ-ಗಿದೆ
ಉಜ್ಜೀ-ವ-ನ-ಗೊ-ಳಿ-ಸ-ಬೇಕು
ಉಡುಪು
ಉತ್ತಮ
ಉತ್ತ-ಮ-ಗೊ-ಳಿಸಿ
ಉತ್ತ-ಮ-ಪ-ಡಿ-ಸಿ-ಕೊ-ಳ್ಳ-ಬೇ-ಕೆಂ-ಬು-ದನ್ನು
ಉತ್ತ-ಮ-ವಾದ
ಉತ್ತ-ರ-ದ-ಕ್ಷಿ-ಣ-ವೆ-ನ್ನದೆ
ಉತ್ತ-ರ-ದಲ್ಲಿ
ಉತ್ತ-ರವೇ
ಉತ್ತೇ-ಜನ
ಉತ್ಥಾನ
ಉತ್ಪತ್ತಿ
ಉತ್ಪ-ನ್ನ-ವಾ-ಗಿವೆ
ಉತ್ಪ-ನ್ನ-ವಾ-ಗು-ವುದು
ಉತ್ಪಾ-ದನೆ
ಉತ್ಸಾಹ
ಉತ್ಸಾ-ಹ-ವಿ-ರ-ಬೇಕು
ಉತ್ಸಾ-ಹ-ವಿಲ್ಲ
ಉದ-ಪ್ವಿ-ಸು-ವರು
ಉದ-ಯಿ-ಸ-ಬ-ಹುದು
ಉದಾರ
ಉದಾ-ರ-ವಾಗಿ
ಉದಾ-ಹ-ರಣೆ
ಉದಾ-ಹ-ರ-ಣೆಗೆ
ಉದಾ-ಹ-ರ-ಣೆ-ಯಾ-ಗಿ-ರು-ವೆವು
ಉದಿ
ಉದಿ-ಸ-ಬೇ-ಕಾ-ಗಿದೆ
ಉದ್ಗ್ರಂ-ಥ-ಗಳು
ಉದ್ದೀ-ಪ-ನ-ಗೊಳಿ
ಉದ್ದೇಶ
ಉದ್ದೇ-ಶ-ದಿಂ-ದಲೋ
ಉದ್ದೇ-ಶ-ವನ್ನು
ಉದ್ದೇ-ಶ-ವಿದೆ
ಉದ್ದೇ-ಶವೂ
ಉದ್ದೇ-ಶವೆ
ಉದ್ದೇ-ಶವೇ
ಉದ್ದೇ-ಶ-ವೇನು
ಉದ್ಧಾರ
ಉದ್ಧಾ-ರ-ಕ್ಕಾಗಿ
ಉದ್ಧಾ-ರಕ್ಕೆ
ಉದ್ಧಾ-ರ-ಮಾ-ಡಲು
ಉದ್ಧಾ-ರ-ವನ್ನು
ಉದ್ಧಾ-ರ-ವಾ-ಗ-ಲಾ-ರೆವು
ಉದ್ಧಾ-ರ-ವಾ-ಗು-ವುದು
ಉದ್ಯಮ
ಉದ್ಯ-ಮ-ಗಳ
ಉದ್ಯ-ಮವೇ
ಉದ್ರೇ-ಕ-ಗೊ-ಳ್ಳುವು
ಉದ್ರೇ-ಕಿ-ಸು-ತ್ತಿದೆ
ಉನ್ನತ
ಉನ್ಮ-ತ್ತ-ನಾಗಿ
ಉನ್ಮ-ತ್ತ-ರ-ನ್ನಾಗಿ
ಉಪ
ಉಪ-ಕಾ-ರ-ಮಾ-ಡುವೆ
ಉಪ-ಕಾ-ರ-ವನ್ನು
ಉಪ-ಕ್ರ-ಮಿ-ಸಿ-ರು-ವೆನು
ಉಪ-ಕ್ರ-ಮಿ-ಸಿವೆ
ಉಪ-ಟ-ಳಕ್ಕೆ
ಉಪ-ನಿ-ಷ-ತ್ತನ್ನು
ಉಪ-ನಿ-ಷ-ತ್ತಿನ
ಉಪ-ನಿ-ಷ-ತ್ತಿ-ನಲ್ಲಿ
ಉಪ-ನಿ-ಷತ್ತು
ಉಪ-ನಿ-ಷ-ತ್ತು-ಗಳಿಂದ
ಉಪ-ನ್ಯಾ-ಸ-ಕ-ನಾ-ಗಿ-ರು-ವನು
ಉಪ-ನ್ಯಾ-ಸ-ಗಳು
ಉಪ-ನ್ಯಾ-ಸ-ದಲ್ಲಿ
ಉಪ-ಯೋ-ಗಿಸ
ಉಪ-ಯೋ-ಗಿ-ಸ-ಬೇ-ಕಾ-ಗಿದೆ
ಉಪ-ಯೋ-ಗಿ-ಸ-ಬೇಕು
ಉಪ-ಯೋ-ಗಿಸಿ
ಉಪ-ಯೋ-ಗಿ-ಸಿ-ಕೊಂ-ಡೆವು
ಉಪ-ಯೋ-ಗಿ-ಸಿ-ಕೊ-ಳ್ಳುವ
ಉಪ-ಯೋ-ಗಿ-ಸುವ
ಉಪ-ಯೋ-ಗಿ-ಸು-ವೆವೊ
ಉಪ-ವಾಸ
ಉಪ-ವಾ-ಸ-ದಲ್ಲಿ
ಉಪ-ವಾ-ಸ-ದಿಂದ
ಉಪ-ವಾ-ಸ-ವಿ-ದ್ದರೂ
ಉಪಾ
ಉಪಾ-ಯ-ಗಳನ್ನು
ಉಪಾ-ಯ-ದಿಂ-ದಲೋ
ಉಪಾ-ಸಕ
ಉಪ್ಪಿ-ನ-ಕಾ-ಯಿ-ಗ-ಳಂತೆ
ಉಪ್ಪು
ಉಬ್ಬ-ರದ
ಉರಿದು
ಉರು-ಳು-ತ್ತಿದೆ
ಉಳಿದ
ಉಳಿ-ದದ್ದು
ಉಳಿ-ದವು
ಉಳಿ-ದ-ವು-ಗ-ಳೆಲ್ಲ
ಉಳಿ-ದಿದೆ
ಉಳಿ-ದಿ-ರು-ವು-ದಕ್ಕೆ
ಉಳಿ-ದಿ-ರು-ವುದೆ
ಉಳಿ-ಯದೆ
ಉಳಿ-ಯ-ಬೇ-ಕೇನು
ಉಳಿ-ಯಲು
ಉಳಿ-ಯಿತು
ಉಳಿಯು
ಉಳಿ-ಯು-ವನು
ಉಳಿ-ಯು-ವಿರಿ
ಉಳಿ-ಯು-ವು-ದಕ್ಕೆ
ಉಳಿ-ಯು-ವುವು
ಉಳಿ-ಸಿ-ಕೊಂಡು
ಉಳು-ವು-ದಕ್ಕೆ
ಉಸಿ-ರೆ-ಳೆದು
ಊಟ
ಊಟಕ್ಕೆ
ಊಟ-ಮಾ-ಡು-ವಾಗ
ಊದಿ
ಊರಿಗೆ
ಊರಿ-ನಲ್ಲಿ
ಊರು-ಗಳಲ್ಲಿ
ಊರೇ
ಊಹಿ-ಸ-ಬ-ಲ್ಲರು
ಊಹಿ-ಸ-ಬ-ಹುದು
ಎಂಜಲು
ಎಂತಹ
ಎಂದರು
ಎಂದರೆ
ಎಂದಲ್ಲ
ಎಂದಿಗೂ
ಎಂದಿ-ರು-ವರು
ಎಂದು
ಎಂದೂ
ಎಂದೆಂ-ದಿಗೂ
ಎಂದೆಂದೂ
ಎಂದೋ
ಎಂಬ
ಎಂಬಂತೆ
ಎಂಬು
ಎಂಬು-ದನ್ನು
ಎಂಬು-ದರ
ಎಂಬುದು
ಎಂಬುದೂ
ಎಂಬುದೇ
ಎಂಬು-ದೇನೊ
ಎಂಬುವ
ಎಂಬು-ವ-ವರ
ಎಕ್ಕಡ
ಎಚ್ಚೆತ್ತು
ಎಡ-ಗ-ಡೆ-ಯಿಂದ
ಎಡ-ಗೈ-ಯಿಂದ
ಎಡ-ಬಿ-ಡದೆ
ಎಡೆಗೆ
ಎಡೆ-ಗೊ-ಡು-ವು-ದಿಲ್ಲ
ಎತ್ತ
ಎತ್ತ-ದಿ-ದ್ದರೆ
ಎತ್ತ-ಬ-ಲ್ಲಿರಾ
ಎತ್ತ-ಬೇಕಾ
ಎತ್ತ-ಬೇ-ಕಾ-ಗಿದೆ
ಎತ್ತ-ಬೇಕು
ಎತ್ತ-ಲಿಲ್ಲ
ಎತ್ತಲು
ಎತ್ತಿ
ಎತ್ತಿ-ದರೆ
ಎತ್ತಿ-ರು-ವರು
ಎತ್ತುವು
ಎತ್ತು-ವು-ದಕ್ಕೆ
ಎತ್ತು-ವು-ದರ
ಎತ್ತು-ವುದು
ಎದು-ರಿಗೆ
ಎದು-ರಿ-ಸ-ಬೇಕು
ಎದು-ರಿ-ಸಲು
ಎದು-ರಿಸಿ
ಎದು-ರಿ-ಸಿ-ದರೂ
ಎದು-ರಿ-ಸು-ವು-ದಕ್ಕೆ
ಎದೆ-ಗುಂ-ದ-ಬೇಡಿ
ಎದೆಗೆ
ಎದೆ-ಯನ್ನು
ಎದ್ದ
ಎದ್ದು
ಎದ್ದೇಳಿ
ಎದ್ದೇಳು
ಎನ್ನ-ಬ-ಹುದು
ಎನ್ನ-ಲಿಲ್ಲ
ಎನ್ನು
ಎನ್ನುತ್ತಿ
ಎನ್ನು-ತ್ತಿ-ದ್ದರು
ಎನ್ನುವ
ಎನ್ನು-ವನು
ಎನ್ನು-ವರು
ಎನ್ನು-ವ-ವರು
ಎನ್ನು-ವುದು
ಎಬ್ಬಿ-ಸ-ಬೇ-ಕಾ-ಗಿಲ್ಲ
ಎರ-ಚಿ-ದಂತೆ
ಎರ-ಡಕ್ಕೂ
ಎರ-ಡ-ನೆ-ಯ-ದಾಗಿ
ಎರ-ಡ-ನೆ-ಯದೆ
ಎರ-ಡ-ರಷ್ಟು
ಎರಡು
ಎರಡೂ
ಎರ-ವ-ಲಾಗಿ
ಎರೆ-ಯ-ಬ-ಲ್ಲಂ-ತಹ
ಎಲೈ
ಎಲ್ಲ
ಎಲ್ಲ-ಕ್ಕಿಂತ
ಎಲ್ಲ-ದರ
ಎಲ್ಲರ
ಎಲ್ಲ-ರನ್ನೂ
ಎಲ್ಲ-ರ-ಲ್ಲಿಯೂ
ಎಲ್ಲ-ರಿಗೂ
ಎಲ್ಲರೂ
ಎಲ್ಲ-ರೊಂ-ದಿಗೂ
ಎಲ್ಲ-ವನ್ನು
ಎಲ್ಲ-ವನ್ನೂ
ಎಲ್ಲವೂ
ಎಲ್ಲಾ
ಎಲ್ಲಿ
ಎಲ್ಲಿಂದ
ಎಲ್ಲಿದೆ
ಎಲ್ಲಿ-ಯ-ವ-ರೆಗೂ
ಎಲ್ಲಿ-ಯ-ವ-ರೆಗೆ
ಎಲ್ಲಿ-ರು-ವನೊ
ಎಲ್ಲೆಲ್ಲೂ
ಎಲ್ಲೊ
ಎಳೆ-ಯು-ವು-ದಲ್ಲ
ಎಳ್ಳಷ್ಟೂ
ಎಷ್ಟು
ಎಷ್ಟೇ
ಎಷ್ಟೊ
ಎಷ್ಟೊಂ-ದನ್ನು
ಎಷ್ಟೊಂದು
ಎಷ್ಟೋ
ಎಸೆ-ದಂ-ತಿದೆ
ಎಸೆ-ದಿ-ರು-ವರು
ಎಸೆದು
ಎಸೆ-ಯ-ಬೇಕು
ಏಕ-ಮಾತ್ರ
ಏಕಾಗ್ರ
ಏಕಾ-ಗ್ರತೆ
ಏಕಾ-ಗ್ರ-ತೆಯ
ಏಕಾ-ಗ್ರ-ತೆ-ಯನ್ನು
ಏಕಾ-ಗ್ರ-ತೆ-ಯಿಂ-ದಲೆ
ಏಕಾ-ಗ್ರ-ತೆಯೇ
ಏಕಾ-ಗ್ರ-ಮಾಡಿ
ಏಕೆ
ಏಕೆಂ-ದರೆ
ಏತಕ್ಕೆ
ಏತ-ಕ್ಕೆಂ-ದರೆ
ಏನ-ನ್ನಾ-ದರೂ
ಏನನ್ನು
ಏನನ್ನೂ
ಏನಾ-ದರೂ
ಏನಿದೆ
ಏನಿ-ದೆಯೊ
ಏನು
ಏನು-ಬೇಕೊ
ಏನೂ
ಏನೆಂದು
ಏನೇ-ನನ್ನು
ಏನೇನು
ಏನೇನೋ
ಏನೊ
ಏನೋ
ಏರ
ಏಳ-ದಂತೆ
ಏಳಿ
ಏಳು
ಏಳು-ತ್ತಾನೆ
ಏಳು-ಬಾ-ರಿಯೆ
ಏಳು-ವಂತೆ
ಏಳ್ಗೆಗೆ
ಏಷ್ಯ-ಖಂಡ
ಏಷ್ಯಾ
ಏಷ್ಯಾ-ಖಂ-ಡ-ದಲ್ಲಿ
ಏಷ್ಯಾ-ಖಂ-ಡ-ದಿಂದ
ಏಷ್ಯಾ-ದೇ-ಶದ
ಐಕ-ಮ-ತ್ಯ-ದಿಂದ
ಐಕ್ಯತೆ
ಐಕ್ಯ-ವಾ-ದಾಗ
ಐದಾರು
ಐದು
ಐದು-ಬಾ-ರಿಯೆ
ಐನೂರು
ಐರೋ-ಪ್ಯ-ರಾ-ಗಲು
ಐರೋ-ಪ್ಯರು
ಐರ್ಲೆಂ-ಡಿ-ನ-ವನು
ಐರ್ಲೆಂಡ್
ಐವತ್ತು
ಐಶ್ವರ್ಯ
ಐಶ್ವ-ರ್ಯ-ಕ್ಕಿಂತ
ಐಶ್ವ-ರ್ಯ-ಗಳನ್ನೆಲ್ಲ
ಐಶ್ವ-ರ್ಯ-ವಂ-ತನ
ಐಶ್ವ-ರ್ಯ-ವನ್ನು
ಐಶ್ವ-ರ್ಯ-ವನ್ನೇ
ಐಶ್ವ-ರ್ಯ-ವೆಲ್ಲ
ಒಂಟಿ-ಯಾ-ಗಿ-ರಲಿ
ಒಂದಕ್ಕೆ
ಒಂದಲ್ಲ
ಒಂದಾದ
ಒಂದಾ-ದರೂ
ಒಂದಿತ್ತು
ಒಂದಿ-ದ್ದರೆ
ಒಂದು
ಒಂದು-ಗೂ-ಡಿ-ಸುವ
ಒಂದು-ಗೂ-ಡಿ-ಸು-ವುದು
ಒಂದೆಡೆ
ಒಂದೇ
ಒಂದೊಂದು
ಒಂಭತ್ತು
ಒಟ್ಟಿಗೆ
ಒಟ್ಟಿಗೇ
ಒಟ್ಟು
ಒಡೆದು
ಒಡೆ-ದು-ಹಾಕಿ
ಒಡೆ-ದು-ಹೋ-ಗಿದೆ
ಒಡ್ಡ-ಬ-ಲ್ಲಿರೋ
ಒಡ್ಡಿ-ರು-ವಿರಿ
ಒಣ-ಗಿ-ಸ-ಲಾ-ರದು
ಒದ-ಗಿ-ದರೆ
ಒದ-ಗಿ-ದಾಗ
ಒದ-ಗಿ-ಸ-ಬ-ಲ್ಲದು
ಒದ-ಗಿ-ಸಿ-ಕೊ-ಡ-ಬೇಕು
ಒದ-ಗಿ-ಸು-ವರೆ
ಒದೆತ
ಒಪ್ಪಿ-ಕೊಳ್ಳಿ
ಒಪ್ಪಿ-ಕೊ-ಳ್ಳುವು
ಒಬಪ್
ಒಬ-ಪ್ನನ್ನು
ಒಬ-ಪ್ನಿಗೆ
ಒಬಪ್ನು
ಒಬೊ-ಪ್ಬಪ್ನೂ
ಒಬ್ಬ
ಒಬ್ಬನ
ಒಬ್ಬ-ನನ್ನು
ಒಬ್ಬ-ನಲ್ಲಿ
ಒಬ್ಬ-ನಿಗೂ
ಒಬ್ಬ-ನಿಗೆ
ಒಬ್ಬನು
ಒಬ್ಬನೂ
ಒಬ್ಬರು
ಒಬ್ಬ-ರೊ-ಡ-ನೊ-ಬ್ಬರು
ಒಬ್ಬೊಬ್ಬ
ಒಮ್ಮ-ತ-ದವ
ಒಮ್ಮ-ತ-ದ-ವ-ರಾ-ದು-ದ-ರಿಂದ
ಒಯ್ದ-ವರು
ಒಯ್ಯಿರಿ
ಒಯ್ಯು-ವುದು
ಒರ-ಟ-ನಾ-ಗಿ-ರ-ಬ-ಹುದು
ಒಳ
ಒಳ-ಗಾಗಿ
ಒಳಗೂ
ಒಳಗೆ
ಒಳ-ಗೊಂದು
ಒಳ್ಳೆಯ
ಒಳ್ಳೆ-ಯ-ದಕ್ಕೂ
ಒಳ್ಳೆ-ಯ-ದಕ್ಕೋ
ಒಳ್ಳೆ-ಯ-ದನ್ನು
ಒಳ್ಳೆ-ಯ-ದಾ-ಗಿದೆ
ಒಳ್ಳೆ-ಯ-ದಾ-ಗಿ-ರು-ವು-ದ-ಕ್ಕಿಂತ
ಒಳ್ಳೆ-ಯ-ದಾ-ಗು-ವುದನ್ನು
ಒಳ್ಳೆ-ಯ-ದಾ-ಗು-ವು-ದಿಲ್ಲ
ಒಳ್ಳೆ-ಯ-ದಾ-ಗು-ವು-ದಿ-ಲ್ಲವೆ
ಒಳ್ಳೆ-ಯ-ದಾ-ಗು-ವುದು
ಒಳ್ಳೆ-ಯದು
ಒಳ್ಳೆ-ಯದೂ
ಒಳ್ಳೆ-ಯ-ವ-ರಾ-ಗಿ-ರು-ವು-ದ-ರಿಂದ
ಒಳ್ಳೆ-ಯ-ವರು
ಒಳ್ಳೆ-ಯವು
ಓ
ಓಂಕಾರ
ಓಡಿ-ಸ-ಬೇಕು
ಓಡಿ-ಹೋ-ಗ-ಬೇ-ಕಾ-ದರೆ
ಓತ
ಓತ-ಪ್ರೋತ
ಓತ-ಪ್ರೋ-ತ-ನಾ-ಗಿ-ರುವ
ಓತ-ಪ್ರೋ-ತ-ವಾಗಿ
ಓದದೆ
ಓದಿ
ಓದು-ತ್ತೇವೆ
ಓದುವ
ಓದು-ವಂ-ತಹ
ಓದು-ವು-ದ-ಕ್ಕಿಂತ
ಔದಾ-ರ್ಯತೆ
ಔದ್ಧಾ-ರ್ಯದ
ಕಂಗಾ-ಲಾ-ಗಿ-ದ್ದರು
ಕಂಠ-ಪಾಠ
ಕಂಡರೂ
ಕಂಡರೆ
ಕಂಡು
ಕಂಡು-ಬಂ-ದರೆ
ಕಂಡು-ಹಿ-ಡಿ-ದರು
ಕಂಡು-ಹಿ-ಡಿ-ದಿರು
ಕಂಡು-ಹಿ-ಡಿ-ದಿ-ರುವ
ಕಂಡು-ಹಿ-ಡಿ-ದಿ-ರು-ವರು
ಕಂಡು-ಹಿ-ಡಿ-ದಿ-ರು-ವಿರಾ
ಕಂಡು-ಹಿ-ಡಿದು
ಕಂಡು-ಹಿ-ಡಿ-ದೆವು
ಕಂಡು-ಹಿ-ಡಿಯು
ಕಂಡು-ಹಿ-ಡಿ-ಯುವು
ಕಂತೆ
ಕಂದರ
ಕಂಬ-ನಿಯ
ಕಚ್ಚಿ-ದ-ವ-ನಿಂದ
ಕಚ್ಚಿ-ದೆಯೋ
ಕಟು-ವಾಗಿ
ಕಟ್ಟಡ
ಕಟ್ಟ-ಡ-ಗಳನ್ನು
ಕಟ್ಟ-ಬೇ-ಕಾ-ಗಿಲ್ಲ
ಕಟ್ಟ-ಬೇ-ಕಾ-ಗು-ವುದು
ಕಟ್ಟಿ
ಕಟ್ಟಿ-ಕೊಂಡು
ಕಟ್ಟಿ-ದರು
ಕಟ್ಟು-ತ್ತಿ-ದ್ದಾಗ
ಕಟ್ಟು-ವರು
ಕಡ-ಲಿ-ನಲ್ಲಿ
ಕಡ-ಲಿ-ನಷ್ಟು
ಕಡಿದು
ಕಡಿಮೆ
ಕಡಿ-ಮೆ-ಯ-ಲ್ಲದೆ
ಕಡಿ-ಮೆ-ಯಾಗಿ
ಕಡಿ-ಮೆ-ಯಾ-ಗುತ್ತ
ಕಡಿ-ಮೆಯೊ
ಕಡೆ
ಕಡೆ-ಗ-ಣಿ-ಸಿದ್ದು
ಕಡೆ-ಗಳಲ್ಲಿ
ಕಡೆ-ಗಿಂತು
ಕಡೆ-ಗಿಂದು
ಕಡೆಗೆ
ಕಡೆ-ಯ-ಲ್ಲಿಯೂ
ಕಡೆ-ಯಾಗಿ
ಕಡೆ-ಯಿಂದ
ಕಡೆ-ಯಿಂ-ದಲೂ
ಕಡೆಯೂ
ಕಡೆಯೇ
ಕಣವೂ
ಕಣ್ಣಿಗೂ
ಕಣ್ಣಿಗೆ
ಕಣ್ಣೀ-ರನ್ನು
ಕಣ್ಣೀ-ರಿಟ್ಟೆ
ಕಣ್ಣು
ಕಣ್ಣು-ಗಳಲ್ಲಿ
ಕಣ್ಣು-ಗಳು
ಕಣ್ಣೆ-ದು-ರಿಗೆ
ಕಣ್ಣೆ-ದು-ರಿಗೇ
ಕಣ್ದೆ-ರೆದು
ಕಣ್ಮ-ರೆ-ಯಾ-ಗ-ಕೂ-ಡದು
ಕಣ್ಮ-ರೆ-ಯಾ-ಗಿಯೇ
ಕತೆಗೂ
ಕತ್ತಿಯ
ಕತ್ತಿ-ಯಿಂದ
ಕತ್ತೆ
ಕತ್ತೆಗೆ
ಕತ್ತೆತ್ತಿ
ಕತ್ತೆ-ಯನ್ನು
ಕಥೆ
ಕಥೆ-ಗಳನ್ನು
ಕಥೆ-ಗಳು
ಕಥೆ-ಯನ್ನು
ಕಥೆ-ಯಿಂದ
ಕದನ
ಕದ-ಲಿ-ಸ-ಲಾ-ರದು
ಕದ-ಲಿ-ಸಲು
ಕದಾ-ನ್ನ-ವನ್ನು
ಕದ್ದರೆ
ಕನ-ಸನ್ನು
ಕನಸು
ಕನ-ಸುಣಿ
ಕನ-ಸು-ಣಿ-ಯಂತೆ
ಕನ-ಸು-ಣಿ-ಯಲ್ಲ
ಕನಿಷ್ಟ
ಕನಿಷ್ಠ
ಕನಿ-ಷ್ಠ-ಪ-ಕ್ಷದ
ಕನ್ನ-ಡಿ-ಯಲ್ಲಿ
ಕಪ-ಟಿ-ಗಳು
ಕಪ್ಪೆ
ಕಬ್ಬಿ-ಣ-ದಂ-ತಹ
ಕಮ್ಯು-ನಿಸಂ
ಕರ-ಟ-ವನ್ನು
ಕರ-ಣೆ-ಗಳು
ಕರು
ಕರು-ಣಾಳು
ಕರೂ
ಕರೆ
ಕರೆ-ದದ್ದು
ಕರೆ-ದರೂ
ಕರೆ-ದರೆ
ಕರೆ-ಯ-ಬ-ಹುದೆ
ಕರೆ-ಯ-ಬೇಕು
ಕರೆ-ಯ-ಬೇ-ಕೆಂದು
ಕರೆ-ಯಿರಿ
ಕರೆ-ಯುತ್ತಿ
ಕರೆ-ಯು-ತ್ತೇನೆ
ಕರೆ-ಯು-ವರು
ಕರೆ-ಯೋಣ
ಕರೆಸಿ
ಕರೆ-ಸಿ-ಕೊ-ಳ್ಳಲು
ಕರ್ತವ್ಯ
ಕರ್ತ-ವ್ಯ-ದಂತೆ
ಕರ್ತ-ವ್ಯ-ವನ್ನು
ಕರ್ತ-ವ್ಯವೇ
ಕರ್ಮ
ಕರ್ಮ-ಗಳನ್ನು
ಕರ್ಮ-ಗಳೂ
ಕರ್ಮ-ಭೂ-ಮಿ-ಯಾದ
ಕರ್ಮ-ವನ್ನು
ಕರ್ಮ-ಸಿ-ದ್ಧಾಂತ
ಕಲ-ಕು-ವಂ-ತಹ
ಕಲ-ಹ-ವಾ-ಡು-ತ್ತೇವೆ
ಕಲಾ-ಜೀ-ವ-ನವೇ
ಕಲಾ-ಪ್ರಿ-ಯ-ತೆ-ಯನ್ನು
ಕಲಿ-ತರೂ
ಕಲಿ-ತರೆ
ಕಲಿ-ತಿದ್ದ
ಕಲಿ-ತಿದ್ದು
ಕಲಿ-ತಿ-ರು-ವೆವು
ಕಲಿ-ತಿಲ್ಲ
ಕಲಿತು
ಕಲಿ-ತು-ಕೊ-ಳ್ಳಲೂ
ಕಲಿ-ತು-ಕೊಳ್ಳಿ
ಕಲಿ-ತು-ಕೊ-ಳ್ಳು-ವು-ದಕ್ಕೆ
ಕಲಿಯ
ಕಲಿ-ಯದೆ
ಕಲಿ-ಯ-ಬೇ-ಕಾ-ಗಿದೆ
ಕಲಿ-ಯ-ಬೇ-ಕಾ-ಗಿ-ರು-ವುದನ್ನು
ಕಲಿ-ಯ-ಬೇ-ಕಾ-ದರೆ
ಕಲಿ-ಯ-ಬೇ-ಕಾ-ದುದು
ಕಲಿ-ಯ-ಬೇಕು
ಕಲಿ-ಯ-ಬೇಡಿ
ಕಲಿ-ಯಲು
ಕಲಿ-ಯಿರಿ
ಕಲಿ-ಯುತ್ತ
ಕಲಿ-ಯುವ
ಕಲಿ-ಯು-ವಂ-ತಹ
ಕಲಿ-ಯು-ವರು
ಕಲಿ-ಯು-ವು-ದಲ್ಲ
ಕಲಿ-ಯು-ವುದು
ಕಲಿ-ಯು-ವುದೇ
ಕಲಿ-ಯೋಣ
ಕಲಿಸ
ಕಲಿ-ಸ-ಬ-ಹುದು
ಕಲಿ-ಸ-ಬೇಕು
ಕಲಿ-ಸಲೂ
ಕಲಿ-ಸಿ-ದರೆ
ಕಲಿ-ಸಿವೆ
ಕಲಿ-ಸುತ್ತ
ಕಲಿ-ಸು-ತ್ತೇನೆ
ಕಲಿ-ಸು-ವು-ದಕ್ಕೆ
ಕಲಿ-ಸು-ವುದನ್ನು
ಕಲೆ
ಕಲೆತ
ಕಲೆತು
ಕಲೆ-ಯ-ಲಾ-ರೆವು
ಕಲ್ಕ-ತ್ತೆಯ
ಕಲ್ಪ-ನಾ-ಜೀವಿ
ಕಲ್ಪ-ನೆಯೊ
ಕಲ್ಪಿ-ಸಿ-ಕೊಂ-ಡಿ-ದ್ದರೂ
ಕಲ್ಪಿ-ಸಿ-ಕೊ-ಳ್ಳ-ಬೇಡಿ
ಕಲ್ಯಾ-ಣ-ಕ್ಕಾಗಿ
ಕಲ್ಯಾ-ಣಕ್ಕೆ
ಕಲ್ಯಾ-ಣ-ಪ್ರ-ದ-ವಾದ
ಕಳ-ವಳ
ಕಳು-ಹಿ-ಸ-ಬ-ಲ್ಲನೊ
ಕಳು-ಹಿ-ಸಿದ
ಕಳು-ಹಿ-ಸಿ-ದನು
ಕಳು-ಹಿ-ಸು-ವರು
ಕಳು-ಹಿ-ಸು-ವು-ದಕ್ಕೆ
ಕಳೆ-ಗಳನ್ನು
ಕಳೆದ
ಕಳೆ-ದಂತೆ
ಕಳೆದು
ಕಳೆ-ದು-ಕೊಂಡ
ಕಳೆ-ದು-ಕೊಂ-ಡರೊ
ಕಳೆ-ದು-ಕೊಂಡು
ಕಳೆ-ದು-ಕೊ-ಳ್ಳ-ದಂತೆ
ಕಳೆ-ದು-ಕೊ-ಳ್ಳು-ತ್ತದೆ
ಕಳೆ-ಯು-ವನೊ
ಕವ-ಲೊ-ಡೆ-ದರು
ಕವ-ಲೊ-ಡೆ-ಯು-ವುದು
ಕವಿತೆ
ಕವಿ-ದು-ಕೊಂ-ಡಿ-ರು-ವುದನ್ನು
ಕಶ್ಮ-ಲ-ಮಿದಂ
ಕಷ್ಟ
ಕಷ್ಟ-ಕಾ-ರ್ಪ-ಣ್ಯ-ಗಳನ್ನು
ಕಷ್ಟ-ಕಾ-ಲಕ್ಕೆ
ಕಷ್ಟ-ಕಾ-ಲ-ದಲ್ಲಿ
ಕಷ್ಟ-ಗಳನ್ನು
ಕಷ್ಟದ
ಕಷ್ಟ-ಪಟ್ಟು
ಕಷ್ಟ-ಪ-ರಂ-ಪ-ರೆ-ಗ-ಳೊಂ-ದಿಗೆ
ಕಷ್ಟ-ವನ್ನು
ಕಷ್ಟ-ವಲ್ಲ
ಕಷ್ಟ-ವಾ-ಗು-ವುದು
ಕಷ್ಟ-ಸಂ-ಕ-ಟ-ಗಳನ್ನೆಲ್ಲ
ಕಸ-ರತ್ತು
ಕಸಿ-ದು-ಕೊಂ-ಡಿ-ರು-ವರು
ಕಸಿ-ದು-ಕೊ-ಳ್ಳು-ವು-ದಲ್ಲ
ಕಹ-ಳೆ-ಗಳ
ಕಾಂಕ್ಷೆ
ಕಾಂತಿ-ಯಿಂದ
ಕಾಂತಿ-ಯಿದೆ
ಕಾಗಿದೆ
ಕಾಡಿಗೆ
ಕಾಡಿ-ನಲ್ಲಿ
ಕಾಡು-ಗಳಲ್ಲಿ
ಕಾಡು-ಪಾಲು
ಕಾಡು-ಮ-ನು-ಷ್ಯ-ನಿ-ಗಿಂತ
ಕಾಣ-ದಿದೆ
ಕಾಣದು
ಕಾಣದೆ
ಕಾಣ-ಬೇಕು
ಕಾಣ-ಲಿಲ್ಲ
ಕಾಣ-ಲೆ-ತ್ನಿ-ಸು-ವನು
ಕಾಣು-ತ್ತಾರೆ
ಕಾಣು-ತ್ತಿರ
ಕಾಣು-ತ್ತಿ-ರುವ
ಕಾಣು-ತ್ತಿ-ರುವೆ
ಕಾಣುವ
ಕಾಣು-ವನು
ಕಾಣು-ವರೊ
ಕಾಣು-ವುದು
ಕಾಣು-ವುದೊ
ಕಾತ-ರ-ರಾ-ಗಿ-ರು-ತ್ತಾರೆ
ಕಾದಾ-ಡಿ-ದರೆ
ಕಾದಾ-ಡು-ವ-ವ-ರೆಗೆ
ಕಾದಾ-ಡು-ವುದನ್ನು
ಕಾದಿದೆ
ಕಾನನ
ಕಾನೂ-ನಿನ
ಕಾನೂನು
ಕಾಪಾ-ಡಿ-ಕೊ-ಳ್ಳಲು
ಕಾಪಾ-ಡು-ವು-ದ-ಕ್ಕಾಗಿ
ಕಾಮವೂ
ಕಾಯ-ಬೇ-ಕಾ-ಗಿದೆ
ಕಾಯ-ಬೇಕು
ಕಾಯ-ಬೇಡಿ
ಕಾಯಾ
ಕಾಯು-ತ್ತಿದೆ
ಕಾಯು-ತ್ತಿ-ರು-ವರು
ಕಾಯು-ತ್ತಿ-ರು-ವಳು
ಕಾಯು-ವುದು
ಕಾರಣ
ಕಾರ-ಣ-ಕ-ರ್ತರು
ಕಾರ-ಣ-ಗಳನ್ನು
ಕಾರ-ಣ-ಗಳಲ್ಲಿ
ಕಾರ-ಣ-ಗಳು
ಕಾರ-ಣ-ವನ್ನು
ಕಾರ-ಣ-ವಲ್ಲ
ಕಾರ-ಣ-ವಾ-ಗಿದೆ
ಕಾರ-ಣ-ವಾದ
ಕಾರ-ಣ-ವಾ-ದರೆ
ಕಾರ-ಣವೆ
ಕಾರ-ಣ-ವೆಂ-ದರು
ಕಾರ-ಣವೇ
ಕಾರ-ಣ-ವೇನು
ಕಾರಿ-ಯಾ-ಗು-ವುದೊ
ಕಾರ್ಖಾನೆ
ಕಾರ್ಖಾ-ನೆ-ಗ-ಳಾಗಿ
ಕಾರ್ಮೋಡ
ಕಾರ್ಮೋ-ಡ-ದಂತೆ
ಕಾರ್ಯ
ಕಾರ್ಯಕ್ಕೆ
ಕಾರ್ಯ-ಕ್ಷೇ-ತ್ರ-ದಲ್ಲಿ
ಕಾರ್ಯ-ಗತ
ಕಾರ್ಯ-ಗ-ತ-ಮಾ-ಡು-ವು-ದಕ್ಕೆ
ಕಾರ್ಯತಃ
ಕಾರ್ಯ-ದಲ್ಲಿ
ಕಾರ್ಯ-ರೂ-ಪಕ್ಕೆ
ಕಾರ್ಯ-ವನ್ನು
ಕಾರ್ಯೋ-ತ್ಸಾಹ
ಕಾರ್ಯೋ-ನ್ಮು-ಖ-ನ-ನ್ನಾಗಿ
ಕಾರ್ಯೋ-ನ್ಮು-ಖ-ರ-ನ್ನಾಗಿ
ಕಾರ್ಯೋ-ನ್ಮು-ಖ-ರಾಗಿ
ಕಾರ್ಯೋ-ನ್ಮು-ಖ-ವಾ-ಗಿತ್ತು
ಕಾಲ
ಕಾಲ-ಕ-ಳೆ-ದಂತೆ
ಕಾಲ-ಕ-ಳೆ-ಯುತ್ತ
ಕಾಲ-ಕಾ-ಲಕ್ಕೆ
ಕಾಲಕ್ಕೆ
ಕಾಲ-ಕ್ರ-ಮೇಣ
ಕಾಲ-ಗ-ರ್ಭ-ದಿಂದ
ಕಾಲದ
ಕಾಲ-ದಲ್ಲಿ
ಕಾಲ-ದಿಂದ
ಕಾಲ-ಮೇಲೆ
ಕಾಲ-ಯಾ-ಪನೆ
ಕಾಲ-ವನ್ನು
ಕಾಲ-ವಾದ
ಕಾಲ-ವಿತ್ತು
ಕಾಲವು
ಕಾಲವೂ
ಕಾಲಾ
ಕಾಲಿನ
ಕಾಲೇ-ಜಿನ
ಕಾಲೇ-ಜಿ-ನಲ್ಲಿ
ಕಾವ್ಯ-ಕೂಡ
ಕಾಶ
ಕಾಷ್ಠೆ-ಯನ್ನು
ಕಿಂಕ-ರ್ತ-ವ್ಯ-ಮೂ-ಢ-ನಾಗಿ
ಕಿಡಿ-ಗಳು
ಕಿತ್ತು
ಕಿತ್ತು-ಹಾಕಿ
ಕಿರಣ
ಕಿವಿಗೂ
ಕಿವಿ-ಗೊಡಿ
ಕೀಟ-ಗಳು
ಕೀಯ
ಕೀರ್ತಿ
ಕೀರ್ತಿಯ
ಕೀಲಿಕೈ
ಕೀಳ-ದಂತೆ
ಕೀಳು-ವ-ರ್ಗ-ದ-ವ-ರಿಗೆ
ಕುಂಠಿ-ತ-ವಾಗಿ
ಕುಟುಂ-ಬಕ್ಕೆ
ಕುಟುಂ-ಬದ
ಕುಡಿ
ಕುಡಿ-ಯ-ಬೇಕು
ಕುಡಿ-ಯು-ವು-ದಲ್ಲ
ಕುಡಿ-ಯು-ವುದು
ಕುಡಿ-ಯು-ವುದೆ
ಕುತಸ್ತ್ವಾ
ಕುತೂ-ಹಲ
ಕುದು-ರೆ-ಯಾ-ಗು-ವು-ದಿಲ್ಲ
ಕುಬೇ-ರನ
ಕುಮಾರ
ಕುಮಾರಿ
ಕುರಿತು
ಕುರು-ಕ್ಷೇತ್ರ
ಕುರು-ಗಳು
ಕುರು-ಡ-ರಿಗೆ
ಕುರುಡಾ
ಕುಲ
ಕುಲ-ಕ-ಸುಬು
ಕುಲಕ್ಕೆ
ಕುಲ-ಗೆ-ಡಿ-ಸ-ಲಾ-ರದು
ಕುಲ-ದ-ವ-ನನ್ನು
ಕುಲ-ಸಂ-ಜಾ-ತರು
ಕುಲೀ-ನ-ವಂ-ಶ-ಸ್ಥರು
ಕುಳಿತ
ಕುಳಿ-ತ-ಕಡೆ
ಕುಳಿ-ತರೆ
ಕುಳಿ-ತಿ-ರ-ಬೇಡಿ
ಕುಳಿತು
ಕುಳಿ-ತು-ಕೊಂ-ಡರೆ
ಕುಳಿ-ತು-ಕೊಂಡು
ಕುಳಿ-ತು-ಕೊ-ಳ್ಳಲು
ಕುಸಿ-ದು-ಬೀ-ಳು-ವುದು
ಕೂಗಾ-ಡಲಿ
ಕೂಗಿ
ಕೂಗೇನೊ
ಕೂಡ
ಕೂಡದು
ಕೂಡಿ-ಡು-ವು-ದಕ್ಕೂ
ಕೂಡಿದ
ಕೂಡಿ-ದ-ವರು
ಕೂಡಿ-ದ್ದರೆ
ಕೂಡಿ-ರು-ವರು
ಕೂಪ-ಮಂ-ಡೂಕ
ಕೂಲಿ
ಕೂಲಿ-ಗಳು
ಕೂಲಿ-ಯ-ವರು
ಕೂಳು
ಕೃತ-ಜ್ಞ-ತೆ-ಯನ್ನು
ಕೃತ-ವಾ-ಗಿದೆ
ಕೆಟ್ಟ
ಕೆಟ್ಟ-ದಕ್ಕೋ
ಕೆಟ್ಟ-ದೆಂದು
ಕೆಟ್ಟ-ದ್ದಕ್ಕೆ
ಕೆಟ್ಟ-ದ್ದನ್ನು
ಕೆಟ್ಟ-ದ್ದಾ-ಗಲಿ
ಕೆಟ್ಟಿ-ದ್ದರೆ
ಕೆಡಿ-ಸಿ-ಕೊ-ಳ್ಳ-ಬೇ-ಕಾ-ಗಿಲ್ಲ
ಕೆಡಿ-ಸು-ತ್ತಿ-ರು-ವರು
ಕೆಡು-ವುದು
ಕೆಡು-ವುದೆ
ಕೆಲ-ವರು
ಕೆಲವು
ಕೆಲ-ವು-ಕಾಲ
ಕೆಲಸ
ಕೆಲ-ಸಕ್ಕೂ
ಕೆಲ-ಸಕ್ಕೆ
ಕೆಲ-ಸ-ಗಳನ್ನು
ಕೆಲ-ಸ-ಗಳನ್ನೂ
ಕೆಲ-ಸ-ಗಳಿಂದ
ಕೆಲ-ಸ-ಗಳೇ
ಕೆಲ-ಸ-ಗಾ-ರ-ರನ್ನು
ಕೆಲ-ಸ-ಗಾ-ರ-ರಾ-ದರೊ
ಕೆಲ-ಸ-ದಲ್ಲಿ
ಕೆಲ-ಸ-ದಿಂ-ದಲೂ
ಕೆಲ-ಸ-ಮಾಡಿ
ಕೆಲ-ಸ-ಮಾ-ಡುತ್ತ
ಕೆಲ-ಸ-ಮಾ-ಡು-ತ್ತಿದೆ
ಕೆಲ-ಸ-ಮಾ-ಡು-ತ್ತಿ-ರುವ
ಕೆಲ-ಸ-ಮಾ-ಡು-ತ್ತಿ-ರು-ವಿರಿ
ಕೆಲ-ಸ-ಮಾ-ಡು-ವುದನ್ನು
ಕೆಲ-ಸ-ಮಾ-ಡೋಣ
ಕೆಲ-ಸ-ವ-ನ್ನಾ-ದರೂ
ಕೆಲ-ಸ-ವನ್ನು
ಕೆಲ-ಸ-ವಾ-ದರೂ
ಕೆಲ-ಸ-ವಿಲ್ಲ
ಕೆಲ-ಸವೂ
ಕೆಳಕ್ಕೆ
ಕೆಳ-ಗಿನ
ಕೆಳ-ಗಿ-ನವ
ಕೆಳ-ಗಿ-ನ-ವ-ರನ್ನು
ಕೆಳ-ಗಿ-ನ-ವ-ರಿಗೆ
ಕೆಳ-ಗಿ-ರು-ವ-ವ-ರಿಗೆ
ಕೆಳಗೆ
ಕೆಳ-ಗೆ-ಳೆದು
ಕೆಳ-ಮ-ಟ್ಟದ
ಕೆಳ-ಮ-ಟ್ಟ-ದಲ್ಲಿ
ಕೆಳ-ಮೆ-ಟ್ಟಲು
ಕೆಸ-ರಿ-ನಲ್ಲಿ
ಕೇಂದ್ರ
ಕೇಂದ್ರ-ಗ-ಳಿರು
ಕೇಂದ್ರ-ವ-ನ್ನಾಗಿ
ಕೇಂದ್ರ-ವನ್ನು
ಕೇಂದ್ರ-ವಾದ
ಕೇಂದ್ರೀ
ಕೇಂದ್ರೀ-ಕೃ-ತ-ವಾ-ಗು-ವ-ವ-ರೆಗೂ
ಕೇಂದ್ರೀ-ಕೃ-ತ-ವಾ-ಗು-ವು-ದಿಲ್ಲ
ಕೇಳದೆ
ಕೇಳಿ
ಕೇಳಿ-ಕೊ-ಳ್ಳು-ತ್ತೇನೆ
ಕೇಳಿ-ದನು
ಕೇಳಿ-ದರೆ
ಕೇಳಿ-ದಾಗ
ಕೇಳಿ-ಬರು
ಕೇಳಿ-ಬ-ರು-ತ್ತಿದೆ
ಕೇಳಿ-ಬ-ರು-ತ್ತಿ-ದೆ-ಬಹು
ಕೇಳಿ-ರು-ವಿರಾ
ಕೇಳಿ-ಸು-ತ್ತಿದೆ
ಕೇಳು
ಕೇಳು-ತ್ತಾರೆ
ಕೇಳು-ವು-ದಕ್ಕೆ
ಕೇಳೋಣ
ಕೇವಲ
ಕೈ
ಕೈಕೆ-ಳಗೆ
ಕೈಕೊ-ಳ್ಳ-ಬ-ಹುದು
ಕೈಕೊ-ಳ್ಳೋಣ
ಕೈಗಳನ್ನು
ಕೈಗಳು
ಕೈಗಳೆ
ಕೈಗಾ-ರಿಕೆ
ಕೈಗೂ-ಡ-ಬೇ-ಕಾ-ದರೆ
ಕೈತೊ-ಳೆ-ದು-ಕೊಂ-ಡು-ಬಿ-ಡ-ಬೇ-ಕೆಂದು
ಕೈಯ
ಕೈಯನ್ನು
ಕೈಯ-ಲ್ಲಾ-ದರೋ
ಕೈಯಲ್ಲಿ
ಕೈಯ-ಲ್ಲಿ-ರಲಿ
ಕೈಯ-ಲ್ಲಿ-ರುವ
ಕೈಯಿಂದ
ಕೈಯೆ-ತ್ತ-ಬೇ-ಕೆಂದು
ಕೈಯೊಂದು
ಕೈಯ್ಯನ್ನು
ಕೈಯ್ಯ-ಲ್ಲಿದೆ
ಕೈಹಾಕಿ
ಕೈಹಾ-ಕಿ-ದರೆ
ಕೊಂಡದ್ದು
ಕೊಂಡರೂ
ಕೊಂಡಾ-ಡಿ-ದರೂ
ಕೊಂಡಾ-ಡು-ವುದು
ಕೊಂಡಿತು
ಕೊಂಡಿದೆ
ಕೊಂಡಿ-ರು-ವೆವು
ಕೊಂಡು
ಕೊಂಡುದು
ಕೊಚ್ಚಿ
ಕೊಟ್ಟ
ಕೊಟ್ಟರು
ಕೊಟ್ಟರೂ
ಕೊಟ್ಟರೆ
ಕೊಟ್ಟಿದೆ
ಕೊಟ್ಟಿ-ರುವ
ಕೊಟ್ಟಿಲ್ಲ
ಕೊಟ್ಟು
ಕೊಟ್ಟೆವು
ಕೊಡ
ಕೊಡದ
ಕೊಡ-ದ-ವ-ನಿಗೆ
ಕೊಡದೆ
ಕೊಡ-ಬ-ಯ-ಸು-ವನು
ಕೊಡ-ಬ-ಲ್ಲದೆ
ಕೊಡ-ಬ-ಲ್ಲಿರಾ
ಕೊಡ-ಬ-ಹುದು
ಕೊಡ-ಬೇ-ಕಾ-ಗಿದೆ
ಕೊಡ-ಬೇ-ಕಾ-ಗಿ-ರ-ಲಿಲ್ಲ
ಕೊಡ-ಬೇಕು
ಕೊಡ-ಬೇಡ
ಕೊಡ-ಬೇಡಿ
ಕೊಡ-ಲಿಲ್ಲ
ಕೊಡಲು
ಕೊಡಲೇ
ಕೊಡವಿ
ಕೊಡಹಿ
ಕೊಡಿ
ಕೊಡು
ಕೊಡು-ತ್ತಿ-ದ್ದರು
ಕೊಡು-ತ್ತೀರಿ
ಕೊಡುವ
ಕೊಡು-ವನು
ಕೊಡು-ವರು
ಕೊಡುವು
ಕೊಡು-ವು-ದ-ಕ್ಕಿಂತ
ಕೊಡು-ವುದನ್ನು
ಕೊಡು-ವು-ದರ
ಕೊಡು-ವು-ದಷ್ಟೇ
ಕೊಡು-ವು-ದಿಲ್ಲ
ಕೊಡು-ವು-ದಿ-ಲ್ಲವೊ
ಕೊಡು-ವುದು
ಕೊಡೋಣ
ಕೊತ್ತ-ಲ-ಗಳಲ್ಲಿ
ಕೊನೆ-ಗಾ-ಣ-ದಿ-ರಲಿ
ಕೊನೆ-ಗಾ-ಣು-ತ್ತಿದೆ
ಕೊನೆಗೆ
ಕೊನೆ-ಗೊ-ಳ್ಳು-ವುದು
ಕೊನೆಯ
ಕೊನೆ-ಯಲ್ಲಿ
ಕೊನೆ-ಯಿ-ಲ್ಲದ
ಕೊರ-ಗಿನ
ಕೊರತೆ
ಕೊರ-ತೆ-ಯಿಲ್ಲ
ಕೊಲೆ-ಯನ್ನು
ಕೊಲ್ಲ-ಬಾ-ರದು
ಕೊಳೆ
ಕೊಳೆತು
ಕೊಳೆ-ಮಾ-ಡ-ಲಾ-ರವು
ಕೊಳೆ-ಯನ್ನು
ಕೊಳೆ-ಯು-ವುದು
ಕೊಳ್ಳದೆ
ಕೊಳ್ಳದೇ
ಕೊಳ್ಳ-ಬ-ಹುದು
ಕೊಳ್ಳ-ಬೇಕು
ಕೊಳ್ಳ-ಬೇಡಿ
ಕೊಳ್ಳಲು
ಕೊಳ್ಳಿ
ಕೊಳ್ಳು-ವು-ದಕ್ಕೆ
ಕೊಳ್ಳು-ವುದು
ಕೊಳ್ಳು-ವುದೆ
ಕೊಳ್ಳೆ
ಕೊಳ್ಳೆ-ಹೊ-ಡೆ-ದು-ಕೊಂಡು
ಕೋಟಿ
ಕೋಟಿ-ಪಾಲು
ಕೋಟಿಯ
ಕೋಟೆ
ಕೋಟೆ-ಗಳನ್ನು
ಕೋಟ್ಯಂ
ಕೋಟ್ಯಂ-ತರ
ಕೋಟ್ಯ-ನು-ಕೋಟಿ
ಕೋಡಿ
ಕೋಣೆ-ಯ-ನ್ನೊ-ಡೆದು
ಕೋಪ-ತಾಪ
ಕೋರೈ-ಸುವ
ಕೋಲು
ಕೌಶ-ಲ-ದಿಂದ
ಕ್ಕಾಗಿ
ಕ್ಕಿಂತ
ಕ್ರಮ
ಕ್ರಮ-ಗಳನ್ನು
ಕ್ರಮೇಣ
ಕ್ರಾಂತಿ-ಗೊ-ಳಿ-ಸಿದೆ
ಕ್ರಾಂತಿ-ಯಾಗು
ಕ್ರಿಯಾ-ಶ-ಕ್ತಿ-ಗಳು
ಕ್ರಿಸ್ತ
ಕ್ರೂರ-ವಾಗಿ
ಕ್ರೂರ-ವಾ-ಗು-ವುದು
ಕ್ರೈಸ್ತ-ಧ-ರ್ಮಕ್ಕೆ
ಕ್ರೈಸ್ತ-ನ-ನ್ನಾಗಿ
ಕ್ಲೈಬ್ಯಂ
ಕ್ಷಣ
ಕ್ಷಣವೂ
ಕ್ಷಣಿಕ
ಕ್ಷಣಿ-ಕತೆ
ಕ್ಷತ್ರಿಯ
ಕ್ಷತ್ರಿ-ಯನ
ಕ್ಷತ್ರಿ-ಯ-ನಾದ
ಕ್ಷಯಿ-ಸು-ತ್ತಿ-ರುವ
ಕ್ಷಾತ್ರ-ಪೌ-ರು-ಷದ
ಕ್ಷಾತ್ರ-ವೀರ್ಯ
ಕ್ಷೀಣ-ವಾಗಿ
ಕ್ಷೀಣ-ಸ್ಥಿ-ತಿಗೆ
ಕ್ಷುದ್ರಂ
ಕ್ಷೇತ್ರ
ಕ್ಷೇತ್ರ-ಗಳಲ್ಲಿ
ಕ್ಷೇತ್ರ-ದಲ್ಲಿ
ಕ್ಷೇತ್ರ-ದ-ಲ್ಲಿಯೂ
ಕ್ಷೇಮ
ಕ್ಷೇಮದ
ಖಂಡಕ್ಕೆ
ಖಂಡ-ಗಳು
ಖಂಡದ
ಖಂಡ-ದಲ್ಲಿ
ಖಂಡ-ದಿಂದ
ಖಂಡಿ-ಸ-ಕೂ-ಡದು
ಖಗೋಳ
ಖರ್ಚು-ಮಾ-ಡ-ಬಾ-ರದು
ಖಾನ
ಖುಷಿ-ಗಳು
ಗಂಗಾ-ನದಿ
ಗಂಗಾ-ನದೀ
ಗಂಟು
ಗಂಡಂ-ದಿರ
ಗಂಡ-ನ-ಲ್ಲದ
ಗಂಡನು
ಗಂಡ-ಸರ
ಗಂಡ-ಸ-ರಿಗೂ
ಗಂಡ-ಸರು
ಗಂಡ-ಸ-ರೆಂದೆ
ಗಂಧದ
ಗಂಭೀರ
ಗಗ-ನ-ರ-ಹಸ್ಯ
ಗಟ್ಟಿ-ಮು-ಟ್ಟಾಗಿ
ಗಣದ
ಗಣ-ನೆಗೇ
ಗಣ-ವನ್ನು
ಗತ
ಗತ-ಕಾ-ಲದ
ಗತಿ
ಗತಿಯೇ
ಗಮಃ
ಗಮನ
ಗಮ-ನಕ್ಕೆ
ಗಮ-ನ-ವನ್ನು
ಗಮ-ನ-ವನ್ನೇ
ಗಮ-ನ-ವಿ-ರು-ವುದೋ
ಗಮ-ನಿ-ಸ-ಬೇ-ಕಾ-ಗಿಲ್ಲ
ಗಮ-ನಿ-ಸ-ಬೇ-ಕಾದ
ಗಮ-ನಿ-ಸ-ಲಿಲ್ಲ
ಗಮ-ನಿಸಿ
ಗಮ-ನಿಸು
ಗರ-ಡಿಯ
ಗರ್ಜನೆ
ಗರ್ಭ-ದಿಂದ
ಗಲಾಟೆ
ಗಲಾ-ಟೆ-ಯಲ್ಲಿ
ಗಲಿ
ಗಲು
ಗಲೂ
ಗಲ್ಲಿಗೆ
ಗಳ
ಗಳಂತೆ
ಗಳನ್ನು
ಗಳನ್ನೂ
ಗಳಲ್ಲಿ
ಗಳ-ಲ್ಲಿಯೂ
ಗಳಾ-ದೆವು
ಗಳಾ-ವುವೂ
ಗಳಿಂದ
ಗಳಿಂ-ದಲೂ
ಗಳಿಗೆ
ಗಳಿ-ಸು-ವು-ದ-ರಲ್ಲಿ
ಗಳು
ಗಳೆಲ್ಲ
ಗಳೆ-ಲ್ಲ-ಕ್ಕಿಂತ
ಗಳೊಂ-ದಿಗೆ
ಗಹ-ನ-ವಾದ
ಗಾಂಢೀವ
ಗಾಂಢೀ-ವ-ವನ್ನು
ಗಾಂಭೀರ್ಯ
ಗಾಗದೆ
ಗಾಗಿ
ಗಾಡಿ
ಗಾಢ
ಗಾಢಾಂ-ಧ-ಕಾ-ರ-ದಿಂದ
ಗಾದೆ
ಗಾನದ
ಗಾನ-ದಿಂ-ಚರ
ಗಾಯ
ಗಾಯತ್ರಿ
ಗಾಯದ
ಗಾರ್ಗಿ
ಗಾಳಿ
ಗಾಳಿಗೆ
ಗಿಂತ
ಗಿಡ
ಗಿಡದ
ಗಿಡ-ದಲ್ಲಿ
ಗಿಡ-ವನ್ನು
ಗಿತ್ತು
ಗಿದೆಯೊ
ಗಿರ-ಬೇಕು
ಗಿರಿ
ಗಿರಿ-ಕಂ-ದ-ರ-ಗಳಲ್ಲಿ
ಗಿರಿ-ನ-ದಿ-ಗಳು
ಗಿರುವ
ಗಿರು-ವುದು
ಗಿಲ್ಲ
ಗೀತಾ
ಗೀತೆ
ಗೀತೆಯ
ಗೀತೆ-ಯನ್ನು
ಗೀತೆ-ಯಲ್ಲಿ
ಗೀತೋ-ಪ-ನಿ-ಷ-ತ್ತಿನ
ಗುಂಪಿಗೆ
ಗುಟುಕು
ಗುಡಿ-ಸ-ಲಿ-ನಲ್ಲಿ
ಗುಣ-ಗಳಲ್ಲಿ
ಗುಣ-ಗ-ಳಿ-ದ್ದರೆ
ಗುಣ-ಗ-ಳಿವೆ
ಗುಣ-ಗ-ಳಿ-ವೆಯೋ
ಗುಣ-ಗಳು
ಗುಣ-ಗಳೂ
ಗುಣದ
ಗುಣ-ಮಾ-ಡು-ವು-ದಕ್ಕೆ
ಗುಣ-ವನ್ನು
ಗುಣವೇ
ಗುಮಾ-ಸ್ತ-ರನ್ನು
ಗುರಿ
ಗುರಿ-ಯಲ್ಲ
ಗುರಿ-ಯೆ-ಡೆಗೆ
ಗುರು
ಗುರು-ಕುಲ
ಗುರು-ಗ-ಳಾ-ಗ-ಬ-ಹುದು
ಗುರು-ಗ-ಳಾ-ಗು-ವರು
ಗುರು-ಗ-ಳಿಗೆ
ಗುರು-ಗಳು
ಗುರು-ಗ-ಳೆಲ್ಲ
ಗುರು-ಗೃಹ
ಗುರು-ವಾ-ದರು
ಗುರು-ವಿನ
ಗುರು-ಹಿ-ರಿ-ಯರು
ಗುಲಾ-ಬಿಯ
ಗುಲಾಮ
ಗುಲಾ-ಮ-ಗಿರಿ
ಗುಲಾ-ಮ-ಗಿ-ರಿ-ಯಲ್ಲಿ
ಗುಲಾ-ಮ-ಗಿ-ರಿಯೆ
ಗುಲಾ-ಮ-ನಾ-ಗಿಯೇ
ಗುಲಾ-ಮ-ರಾಗಿ
ಗುಲಾ-ಮರು
ಗುಳ್ಳೆ-ಗ-ಳಂತೆ
ಗುಳ್ಳೆ-ಗ-ಳಂ-ತೆಯೇ
ಗುಹೆ-ಯಿಂದ
ಗೂಡಿ
ಗೂಡಿಸಿ
ಗೃಹ-ಕೃತ್ಯ
ಗೃಹ-ಕೃ-ತ್ಯಕ್ಕೆ
ಗೃಹ-ದೇ-ವಿಯ
ಗೆದ್ದ
ಗೆದ್ದಿ-ರು-ವರು
ಗೆದ್ದು
ಗೆಲವು
ಗೆಲ್ಲ-ಬೇ-ಕಾ-ಗಿದೆ
ಗೆಲ್ಲ-ಬೇ-ಕಾ-ಗಿ-ರು-ವುದು
ಗೆಲ್ಲ-ಬೇ-ಕಾ-ದರೆ
ಗೆಲ್ಲ-ಬೇಕು
ಗೆಲ್ಲ-ಲಿಲ್ಲ
ಗೆಲ್ಲಲು
ಗೆಲ್ಲು
ಗೆಲ್ಲು-ವನು
ಗೆಲ್ಲು-ವು-ದ-ಕ್ಕಾ-ಗು-ವು-ದಿಲ್ಲ
ಗೆಲ್ಲು-ವು-ದಕ್ಕೆ
ಗೆಲ್ಲು-ವು-ದಲ್ಲ
ಗೊಂಬೆ
ಗೊಣ-ಗಾ-ಡದೆ
ಗೊಣ-ಗು-ತ್ತೀರಿ
ಗೊತ್ತಾಗಿ
ಗೊತ್ತಾ-ಗಿದೆ
ಗೊತ್ತಾಗು
ಗೊತ್ತಾ-ಗು-ತ್ತಿದೆ
ಗೊತ್ತಾ-ಗು-ವುದು
ಗೊತ್ತಿದೆ
ಗೊತ್ತಿ-ದ್ದರೆ
ಗೊತ್ತಿ-ರು-ವುದನ್ನು
ಗೊತ್ತಿ-ರು-ವುದು
ಗೊತ್ತಿಲ್ಲ
ಗೊತ್ತಿ-ಲ್ಲದೆ
ಗೊತ್ತಿ-ಲ್ಲವೆ
ಗೊತ್ತು
ಗೊತ್ತೆ
ಗೊಬ್ಬ-ರವೇ
ಗೊಳಿ-ಸು-ವುದು
ಗೋಗರೆ
ಗೋಗ-ರೆ-ಯು-ತ್ತಿ-ರ-ಬೇ-ಕಾ-ಗಿಲ್ಲ
ಗೋಚ-ರಿ-ಸದೆ
ಗೋಚ-ರಿ-ಸುವ
ಗೋಜಾ-ಗಿದೆ
ಗೋಡೆ-ಯನ್ನು
ಗೋಳು
ಗೌಣ
ಗೌಣ-ವಾ-ಗು-ವುವು
ಗೌರವ
ಗೌರ-ವಕ್ಕೆ
ಗೌರ-ವ-ದಿಂದ
ಗೌರ-ವ-ವನ್ನು
ಗೌರ-ವ-ವನ್ನೂ
ಗೌರ-ವ-ಸ್ಥ-ನಾ-ಗ-ಬ-ಹುದು
ಗೌರ-ವಿ-ಸದೆ
ಗೌರ-ವಿ-ಸಿ-ದರು
ಗೌರ-ವಿ-ಸಿ-ದರೂ
ಗೌರ-ವಿ-ಸಿ-ಲ್ಲವೊ
ಗೌರ-ವಿ-ಸು-ವರೊ
ಗೌರ-ವಿ-ಸು-ವು-ದಿಲ್ಲ
ಗೌರ-ವಿ-ಸು-ವೆನು
ಗ್ರಂಥ
ಗ್ರಂಥ-ಗಳನ್ನು
ಗ್ರಹಿ-ಸ-ಬ-ಲ್ಲಿರಿ
ಗ್ರಹಿ-ಸ-ಲಾ-ರದು
ಗ್ರಹಿಸಿ
ಗ್ರಹಿ-ಸಿ-ರು-ವಿರಾ
ಗ್ರಾಮ-ಗಳಲ್ಲಿ
ಗ್ರಾಮದ
ಗ್ರಾಮ-ವನ್ನು
ಗ್ರಾಮ್ಯ-ಭಾ-ಷೆ-ಯಲ್ಲಿ
ಗ್ರೀಕನು
ಗ್ರೀಕರ
ಗ್ರೀಕ-ರಲ್ಲಿ
ಗ್ರೀಕರು
ಗ್ರೀಕ್
ಗ್ರೀಸ-ಲ್ಲದೆ
ಗ್ರೀಸಿನ
ಗ್ರೀಸಿ-ನ-ವ-ರಿಗೆ
ಗ್ರೀಸಿ-ನ-ವರು
ಗ್ರೀಸ್
ಗ್ಲೋಬ್
ಘಂಟಾ
ಘಟನೆ
ಘಟ-ನೆ-ಗ-ಳಿಗೆ
ಘಟ-ನೆ-ಯನ್ನು
ಘಟ್ಟಿ
ಘನ
ಘರ್ಷಣೆ
ಘೋರ-ವಾ-ದುದು
ಘೋಷ-ವಾಗಿ
ಚಂಚ-ಲ-ವಾ-ಗಿದೆ
ಚಂಡಾಲ
ಚಂಡಾ-ಲ-ನನ್ನು
ಚಂಡಾ-ಲ-ನ-ವ-ರೆಗೆ
ಚಕ್ರ
ಚಕ್ರಾ-ಧಿ-ಪ-ತ್ಯ-ಗ-ಳಾದ
ಚಕ್ರಾ-ಧಿ-ಪ-ತ್ಯ-ಗಳು
ಚಕ್ರಾ-ಧಿ-ಪ-ತ್ಯದ
ಚದು-ರಿ-ಹೋಗಿ
ಚನೆ
ಚನೆಯೂ
ಚಮ್ಮಾರ
ಚರಿತ್ರೆ
ಚರ್ಚಿ-ಸ-ಬಲ್ಲ
ಚರ್ಚಿ-ಸುತ್ತ
ಚರ್ಚಿ-ಸು-ತ್ತಿ-ರು-ವರೊ
ಚರ್ಚಿ-ಸು-ತ್ತಿ-ರುವು
ಚರ್ಚೆ-ಗ-ಳಾ-ಗಿವೆ
ಚರ್ಮ
ಚಲ-ನ-ವ-ಲ-ನವೂ
ಚಲಾ-ಯಿ-ಸ-ಲಾ-ಗು-ವು-ದಿಲ್ಲ
ಚಲಾ-ಯಿ-ಸು-ತ್ತಿತ್ತು
ಚಲಾ-ಯಿ-ಸು-ವು-ದಕ್ಕೆ
ಚಾಕ-ರಿಗೆ
ಚಾಚು-ತ್ತಿ-ರು-ವರು
ಚಾರ-ಗಳನ್ನು
ಚಾರ-ಗಳು
ಚಾರಿ-ಣಿ-ಯ-ರಿಗೆ
ಚಾರಿ-ಣಿ-ಯಾದ
ಚಾರಿತ್ರ್ಯ
ಚಾರಿ-ತ್ರ್ಯದ
ಚಾರಿ-ತ್ರ್ಯ-ದ-ವನೇ
ಚಾರಿ-ತ್ರ್ಯ-ದ-ವ-ರ-ನ್ನಾಗಿ
ಚಾರಿ-ತ್ರ್ಯ-ಶು-ದ್ಧಿಗೆ
ಚಾರ್ಯರು
ಚಾರ್ವಾ-ಕ-ದಿಂ-ದಲೇ
ಚಾರ್ವಾ-ಕ-ವಾದ
ಚಿಂತಿ-ಸು-ತ್ತಿ-ರು-ವರು
ಚಿಂತಿ-ಸು-ತ್ತಿ-ರು-ವರೊ
ಚಿಂತಿ-ಸು-ವರು
ಚಿಂತಿ-ಸು-ವುದು
ಚಿಂತೆ
ಚಿಂತೆ-ಯಿಲ್ಲ
ಚಿಂದಿಯ
ಚಿಂದಿ-ಯನ್ನು
ಚಿಟ್ಟೆ-ಗ-ಳಂತೆ
ಚಿತ್ತ
ಚಿತ್ತ-ಸ್ವಾ-ಸ್ಥ್ಯ-ವಿ-ಲ್ಲದ
ಚಿತ್ರ
ಚಿನ್ನ-ವನ್ನು
ಚಿಪ್ಪಿ-ನೊ-ಳಗೆ
ಚಿಮ್ಮಿ
ಚಿಲು-ಮೆ-ಯಿಂದ
ಚಿಹ್ನೆ
ಚಿಹ್ನೆ-ಯನ್ನು
ಚಿಹ್ನೆ-ಯಾದ
ಚಿಹ್ನೆಯೇ
ಚುರುಕು
ಚೂರನ್ನು
ಚೂರಾ-ಗು-ವು-ದಕ್ಕೆ
ಚೂರು
ಚೂರು-ಚೂ-ರಾಗಿ
ಚೆಂಡಾಟ
ಚೆನ್ನಾಗಿ
ಚೆನ್ನಾ-ಗು-ವುದು
ಚೆಲ್ಲಾ-ಪಿ-ಲ್ಲಿ-ಯಾಗಿ
ಚೇತ-ನ-ವನ್ನು
ಚೇತ-ನ-ವಿಲ್ಲ
ಚೈತನ್ಯ
ಚೈತ-ನ್ಯ-ವಿದೆ
ಚೈತ-ನ್ಯ-ವೆಂದು
ಚೈನಾ-ದಿಂದ
ಚ್ಯುತ-ರಾಗ
ಛಲ
ಛಲ-ವಿತ್ತು
ಛಲ-ವಿಲ್ಲ
ಛಾಯೆ
ಛಾಯೆ-ಯಂತೆ
ಜಗ-ಜ್ಜ-ನ-ನಿಯ
ಜಗ-ತ್ತನ್ನೇ
ಜಗ-ತ್ತಿಗೆ
ಜಗತ್ತು
ಜಗದ
ಜಗ-ನ್ಮಾತೆ
ಜಗ-ನ್ಮಾ-ತೆ-ಯನ್ನು
ಜಗಳ
ಜಗ-ಳ-ಕಾದು
ಜಗ-ಳ-ಕಾ-ಯು-ವು-ದ-ಕ್ಕಿಂತ
ಜಗ-ಳ-ಕಾ-ಯು-ವು-ದಕ್ಕೆ
ಜಡ-ತ-ನ-ವನ್ನು
ಜಡ-ದ್ರ-ವ್ಯ-ವನ್ನು
ಜಡ-ನಾ-ಗ-ರಿ-ಕ-ತೆಯ
ಜಡ-ನಿ-ದ್ರೆ-ಯನ್ನು
ಜಡ-ವಾ-ಗಿದೆ
ಜಡ-ವಾದ
ಜಡ-ವಾ-ದಿ-ಗ-ಳೆಂದು
ಜಡ-ವಾ-ದಿ-ಯ-ಲ್ಲಿ-ರು-ವಷ್ಟು
ಜಡ-ಸಿ-ದ್ಧಾಂ-ತದ
ಜನ
ಜನಕ್ಕೂ
ಜನ-ಗ-ಣ-ರಾ-ಜ್ಯವೋ
ಜನ-ಗ-ಳಿಗೆ
ಜನ-ಗಳು
ಜನರ
ಜನ-ರನ್ನು
ಜನ-ರಲ್ಲಿ
ಜನ-ರಾ-ದರೂ
ಜನ-ರಾ-ದರೊ
ಜನ-ರಿಂದ
ಜನ-ರಿಗೂ
ಜನ-ರಿಗೆ
ಜನ-ರಿ-ಗೆಲ್ಲಾ
ಜನರು
ಜನ-ರೆ-ದು-ರಿಗೆ
ಜನರೇ
ಜನ-ವಾ-ಗ-ಬ-ಲ್ಲದು
ಜನ-ಸಂ-ದಣಿ
ಜನ-ಸಾ-ಧಾ-ರಣ
ಜನ-ಸಾ-ಧಾ-ರ-ಣ-ರನ್ನು
ಜನ-ಸಾ-ಧಾ-ರ-ಣ-ರಿ-ಗಾಗಿ
ಜನ-ಸಾ-ಧಾ-ರ-ಣ-ರಿ-ಗಿಂತ
ಜನ-ಸಾ-ಧಾ-ರ-ಣ-ರಿಗೆ
ಜನ-ಸಾ-ಧಾ-ರ-ಣರು
ಜನ-ಸಾ-ಮಾ-ನ್ಯರ
ಜನ-ಸಾ-ಮಾ-ನ್ಯ-ರನ್ನು
ಜನಾಂಗ
ಜನಾಂ-ಗಕ್ಕೂ
ಜನಾಂ-ಗಕ್ಕೆ
ಜನಾಂ-ಗ-ಗಳ
ಜನಾಂ-ಗ-ಗಳಲ್ಲಿ
ಜನಾಂ-ಗ-ಗಳಿಂದ
ಜನಾಂ-ಗ-ಗ-ಳಿವೆ
ಜನಾಂ-ಗ-ಗಳು
ಜನಾಂ-ಗ-ಗ-ಳೆಲ್ಲ
ಜನಾಂ-ಗ-ಗ-ಳೊಂ-ದಿಗೆ
ಜನಾಂ-ಗದ
ಜನಾಂ-ಗ-ದಲ್ಲಿ
ಜನಾಂ-ಗ-ವ-ನ್ನಾಗಿ
ಜನಾಂ-ಗ-ವನ್ನು
ಜನಾಂ-ಗ-ವಾ-ಗ-ಲಾ-ರರು
ಜನಾಂ-ಗ-ವಿಲ್ಲ
ಜನಾಂ-ಗವೂ
ಜನಾಂ-ಗವೇ
ಜನಿ-ಸಿ-ದ್ದರು
ಜನಿ-ಸು-ವುವು
ಜನ್ಮ
ಜನ್ಮ-ವೆ-ತ್ತಿ-ರು-ವಿರಿ
ಜನ್ಮ-ವೆ-ತ್ತು-ವುದು
ಜನ್ಮ-ಸ್ಥಾನ
ಜಪ
ಜಯ
ಜಯ-ಕಾ-ರ-ವನ್ನು
ಜಯ-ಪ್ರ-ದ-ರಾ-ಗ-ಬೇ-ಕಾ-ದರೆ
ಜಯ-ಪ್ರ-ದ-ರಾ-ಗು-ವರು
ಜಯ-ಪ್ರ-ದ-ವಾ-ಗಿಯೇ
ಜಯ-ಪ್ರ-ದ-ವಾ-ಗು-ವು-ದಕ್ಕೆ
ಜಯ-ವನ್ನು
ಜಯ-ವಾ-ಗಲಿ
ಜಯಿ-ಸ-ಲೇ-ಬೇಕು
ಜಯಿ-ಸಿಲ್ಲ
ಜಲ-ಪಾ-ತದ
ಜವಾ-ಬ್ದಾರಿ
ಜವಾ-ಬ್ದಾ-ರಿ-ಯನ್ನು
ಜವಾ-ಬ್ದಾ-ರಿಯು
ಜಾಗ-ರೂ-ಕತೆ
ಜಾಗೃ-ತ-ಗೊ-ಳಿ-ಸ-ಬ-ಹುದು
ಜಾಗ್ರತ
ಜಾಗ್ರ-ತ-ಗೊ-ಳಿಸ
ಜಾಗ್ರ-ತ-ಗೊ-ಳಿ-ಸ-ಬೇಕು
ಜಾಗ್ರ-ತ-ಗೊ-ಳಿಸಿ
ಜಾಗ್ರ-ತ-ಗೊ-ಳಿ-ಸು-ತ್ತಿ-ರು-ವ-ವರೆಲ್ಲ
ಜಾಗ್ರ-ತ-ಗೊ-ಳಿ-ಸು-ವು-ದಕ್ಕೆ
ಜಾಗ್ರ-ತ-ಗೊ-ಳಿ-ಸು-ವುವು
ಜಾಗ್ರ-ತ-ನಾಗು
ಜಾಗ್ರ-ತರ
ಜಾಗ್ರ-ತ-ರಾಗಿ
ಜಾಗ್ರ-ತ-ರಾ-ಗು-ವ-ವ-ರೆಗೆ
ಜಾಗ್ರ-ತ-ರಾ-ದರು
ಜಾಗ್ರ-ತ-ಳಾ-ಗು-ವು-ದಿಲ್ಲ
ಜಾಡ-ಮಾಲಿ
ಜಾಡ-ಮಾ-ಲಿ-ಗಳನ್ನು
ಜಾಡ-ಮಾ-ಲಿ-ಗಳು
ಜಾಡಿ-ನಲ್ಲೇ
ಜಾಡ್ಯ
ಜಾತಿ
ಜಾತಿ-ಯಲ್ಲಿ
ಜಾತ್ಯ-ತೀ-ತ-ವಾದ
ಜಾರಿಗೆ
ಜಾರಿ-ಯ-ಲ್ಲಿ-ರು-ವುದನ್ನು
ಜಾರಿ-ಹೋ-ಗಿದೆ
ಜಾಸ್ತಿ
ಜಾಸ್ತಿ-ಯಾ-ಗು-ವುದು
ಜಿಜ್ಞಾಸೆ
ಜಿನೀವಾ
ಜೀವ
ಜೀವಂತ
ಜೀವಂ-ತ-ವಾ-ಗಿ-ರು-ವುದು
ಜೀವಂ-ತ-ವಾ-ಗಿ-ರು-ವೆವು
ಜೀವಂ-ತ-ವಾ-ಗಿವೆ
ಜೀವ-ದಾನ
ಜೀವನ
ಜೀವ-ನಕ್ಕೂ
ಜೀವ-ನಕ್ಕೆ
ಜೀವ-ನದ
ಜೀವ-ನ-ದಲ್ಲಿ
ಜೀವ-ನ-ದ-ಲ್ಲಿದೆ
ಜೀವ-ನ-ದಿಂ-ದಲೇ
ಜೀವ-ನ-ದೃ-ಷ್ಟಿ-ಯಿಂದ
ಜೀವ-ನ-ವನ್ನು
ಜೀವ-ನ-ವ-ನ್ನೆಲ್ಲ
ಜೀವ-ನ-ವನ್ನೇ
ಜೀವ-ನ-ವಿದೆ
ಜೀವ-ನವೇ
ಜೀವನೋ
ಜೀವ-ನೋ-ಪಾ-ಯಕ್ಕೆ
ಜೀವ-ನೋ-ಪಾ-ಯ-ದಲ್ಲೆ
ಜೀವ-ನೋ-ಪಾ-ಯ-ವ-ನ್ನಾ-ದರೂ
ಜೀವ-ವಿ-ರು-ವು-ದಿಲ್ಲ
ಜೀವಾಳ
ಜೀವಾ-ಳವೇ
ಜೀವಾ-ವಧಿ
ಜೀವಿ
ಜೀವಿ-ಗಳ
ಜೀವಿ-ಗಳು
ಜೀವಿಗೆ
ಜೀವಿಯ
ಜೀವಿ-ಯಲ್ಲಿ
ಜೀವಿಯೂ
ಜೀವಿ-ಸ-ಬೇ-ಕಾ-ದರೆ
ಜೀವಿ-ಸು-ತ್ತಿದೆ
ಜೇಡ
ಜೊತೆಗೆ
ಜೊತೆ-ಜೊ-ತೆ-ಯ-ಲ್ಲಿಯೇ
ಜೋಪಾನ
ಜೋಪಾ-ನ-ದಿಂದ
ಜೋಪಾ-ನ-ವಾಗಿ
ಜೋಪಾ-ನ-ವಾ-ಗಿ-ರ-ಬೇಕು
ಜೋಪಾ-ನ-ವಾ-ಗಿರಿ
ಜ್ಞಾನ
ಜ್ಞಾನಕ್ಕೆ
ಜ್ಞಾನ-ಗಳು
ಜ್ಞಾನ-ದ-ಲ್ಲಿಯೂ
ಜ್ಞಾನ-ಮಾ-ರ್ಗಿ-ಗಳೂ
ಜ್ಞಾನ-ರಾ-ಶಿ-ಯ-ನ್ನೆಲ್ಲ
ಜ್ಞಾನ-ವನ್ನು
ಜ್ಞಾನ-ವಿ-ದೆ-ಯೆಂದು
ಜ್ಞಾನ-ವಿ-ದೆಯೋ
ಜ್ಞಾನವೂ
ಜ್ಞಾನ-ಸಂ-ಪಾ-ದ-ನೆಗೆ
ಜ್ಞಾನಾ
ಜ್ಞಾನಾ-ಕಾಂಕ್ಷೆ
ಜ್ಞಾನಾ-ಮೃ-ತ-ವನ್ನು
ಜ್ಞಾನಿ-ಗಳನ್ನು
ಜ್ಞಾಪ-ಕಕ್ಕೆ
ಜ್ಞಾಪ-ಕ-ದ-ಲ್ಲಿ-ಟ್ಟು-ಕೊ-ಳ್ಳ-ಬ-ಹುದು
ಜ್ವಲಂತ
ಜ್ವಾಲಾ-ಮು-ಖಿಯ
ಝಾನ್ಸಿ-ರಾಣಿ
ಟೀಕಿ-ಸು-ತ್ತೇವೆ
ಟೀಕಿ-ಸು-ವು-ದಕ್ಕೆ
ಟೀಕಿ-ಸು-ವುದೇ
ಟೀಕೆ
ಟೆನಿ-ಸನ್
ಡೆನ್ಮಾ-ರ್ಕ್
ಣಾಮ
ಣೇತರ
ತಂಡ
ತಂತಿಯ
ತಂದರು
ತಂದರೋ
ತಂದ-ವರು
ತಂದು
ತಂದೆ
ತಂದೆ-ತಾ-ಯಿ-ಗಳು
ತಂಬ-ಟೆ-ಯಂತೆ
ತಕ್ಕ
ತಕ್ಕಂತೆ
ತಕ್ಷ-ಣವೆ
ತಕ್ಷ-ಣವೇ
ತಟಸ್ಥ
ತಟ್ಟು-ವುದು
ತಡೆ-ಗ-ಟ್ಟ-ಲಾ-ರರು
ತಡೆ-ಗ-ಟ್ಟು-ವು-ದಕ್ಕೆ
ತಡೆ-ಯ-ಬ-ಲ್ಲ-ವ-ರಿಲ್ಲ
ತಡೆ-ಯ-ಲಾ-ರದು
ತಡೆ-ಯು-ವ-ವ-ರಿಲ್ಲ
ತಣ್ಣ-ಗಾ-ಗು-ವುದು
ತತ್ಕಾ-ಲಕ್ಕೆ
ತತ್ತ್ವ
ತತ್ತ್ವಕ್ಕೆ
ತತ್ತ್ವ-ಗಳ
ತತ್ತ್ವ-ಗಳನ್ನು
ತತ್ತ್ವ-ಗಳು
ತತ್ತ್ವ-ಜ್ಞಾನ
ತತ್ತ್ವ-ಜ್ಞಾ-ನಿನ್
ತತ್ತ್ವ-ವನ್ನು
ತತ್ತ್ವ-ವಿದೆ
ತತ್ತ್ವ-ಶಾ-ಸ್ತ್ರ-ವನ್ನು
ತತ್ರ
ತನಕ
ತನಗೆ
ತನ್ನ
ತನ್ನದ
ತನ್ನದೇ
ತನ್ನನ್ನು
ತನ್ನಲ್ಲಿ
ತನ್ನ-ಲ್ಲಿ-ರುವ
ತನ್ನಿ
ತನ್ನಿಂದ
ತನ್ನಿಂ-ದಲೇ
ತಪ-ಸ್ಸಿನ
ತಪಸ್ಸು
ತಪೋ-ಮ-ಹಿ-ಮರು
ತಪ್ಪನ್ನು
ತಪ್ಪಲ್ಲ
ತಪ್ಪಾ-ಗಿ-ದ್ದರೂ
ತಪ್ಪಿ-ಸಿ-ದಾ-ಗಲೆ
ತಪ್ಪು
ತಪ್ಪು-ಮಾ-ಡು-ವುದು
ತಬ್ಬಿ-ಕೊಂ-ಡಿ-ರುವ
ತಮಗೆ
ತಮ-ಗೊಂದು
ತಮಸ್
ತಮ-ಸ್ಸಿನ
ತಮ್ಮ
ತಮ್ಮನ್ನು
ತಮ್ಮಲ್ಲಿ
ತಯಾರು
ತಯಾ-ರು-ಮಾ-ಡುವ
ತರ
ತರಂ-ಗ-ಗಳನ್ನೆಲ್ಲ
ತರಂ-ಗ-ವನ್ನೇ
ತರದೆ
ತರ-ಬೇ-ಕಾ-ಗಿದೆ
ತರ-ಬೇ-ಕಾ-ದರೆ
ತರ-ಬೇಕು
ತರ-ಬೇ-ತನ್ನು
ತರ-ಬೇ-ತನ್ನೇ
ತರ-ಬೇ-ತಾ-ಗಿಲ್ಲ
ತರ-ಬೇ-ತಿನ
ತರ-ಬೇತು
ತರ-ಬೇತೆ
ತರ-ಲಿಲ್ಲ
ತರಲು
ತರ-ಲೆ-ತ್ನಿ-ಸುವ
ತರಿ-ದು-ಬಿ-ಡು-ವಂತೆ
ತರು-ತ್ತಿ-ದ್ದರು
ತರುವು
ತರು-ವು-ದ-ಕ್ಕಾಗಿ
ತರು-ವು-ದಕ್ಕೆ
ತರು-ವು-ದಲ್ಲ
ತರು-ವು-ದಿಲ್ಲ
ತರು-ವುದು
ತರ್ಕ
ತಲು-ಪು-ವಂತೆ
ತಲೆಗೆ
ತಲೆ-ದೋರಿ
ತಲೆ-ದೋ-ರು-ವು-ದಕ್ಕೆ
ತಲೆ-ದೋ-ರು-ವುದೊ
ತಲೆ-ಬಾ-ಗ-ಬೇಕು
ತಲೆ-ಮಾ-ರಿನ
ತಲೆ-ಮಾ-ರಿ-ನಲ್ಲಿ
ತಲೆ-ಮಾ-ರಿ-ನ-ವ-ರೆಗೆ
ತಲೆ-ಮಾ-ರಿ-ನಿಂದ
ತಲೆಯ
ತಲೆ-ಯನ್ನು
ತಲೆ-ಯಲ್ಲಿ
ತಲ್ಲ-ಣಿ-ಸು-ತ್ತಿತ್ತು
ತಳ-ಪಾಯ
ತಳ-ಹದಿ
ತಳ-ಹ-ದಿ-ಯನ್ನು
ತಳಿ-ರು-ಗಳು
ತಳ್ಳಿ
ತವ-ಕ-ಪ-ಡು-ತ್ತಿ-ದ್ದಾಗ
ತವ-ಕ-ಪ-ಡು-ತ್ತಿ-ರು-ವರು
ತಾಂತ್ರಿ
ತಾಕಿ
ತಾಕು-ತ್ತಿದೆ
ತಾಕು-ವುದೇ
ತಾತ್ಕಾ-ಲಿಕ
ತಾತ್ಕಾ-ಲಿ-ಕ-ವಾದ
ತಾತ್ತ್ವಿಕ
ತಾತ್ಸಾರ
ತಾತ್ಸಾ-ರ-ವಾಗಿ
ತಾನು
ತಾನೆ
ತಾನೇ
ತಾನೊಂದು
ತಾಪ-ತ್ರಯ
ತಾಯಂ-ದಿರ
ತಾಯಂ-ದಿ-ರಂತೆ
ತಾಯಾ-ಗಿ-ರು-ವ-ವಳು
ತಾಯ್ನಾ-ಡಿನ
ತಾರಾ-ವಳಿ
ತಾರಾ-ವ-ಳಿ-ಗಳು
ತಾರಾ-ವ-ಳಿ-ಗಿಂತ
ತಾಳ-ಬೇಕು
ತಾಳಿ
ತಾಳಿದ
ತಾಳ್ಮೆ
ತಾಳ್ಮೆ-ಯಿಂದ
ತಾಳ್ಮೆ-ಯಿಂ-ದಿರಿ
ತಾವಿದ್ದ
ತಾವು
ತಾವೇ
ತಿಂಗಳಿ
ತಿಂಗ-ಳು-ಗಳಾ
ತಿಂಡಿ-ಯನ್ನು
ತಿಂದರೂ
ತಿಂದರೆ
ತಿಂದು
ತಿನ್ನ-ಬೇಕು
ತಿನ್ನು-ತ್ತಿ-ರುವ
ತಿನ್ನು-ವು-ದಕ್ಕೆ
ತಿನ್ನು-ವುದು
ತಿರ-ಸ್ಕಾ-ರ-ದೃ-ಷ್ಟಿ-ಯಿಂದ
ತಿರು-ಗ-ಬೇಕೊ
ತಿರು-ಳನ್ನು
ತಿಳಿ
ತಿಳಿ-ಗೇ-ಡಿ-ಗ-ಳಿಗೆ
ತಿಳಿ-ದಂ-ತೆಲ್ಲಾ
ತಿಳಿ-ದರು
ತಿಳಿ-ದಿದೆ
ತಿಳಿ-ದಿ-ರುವ
ತಿಳಿದು
ತಿಳಿ-ದು-ಕೊಂ-ಡಿರು
ತಿಳಿ-ದು-ಕೊಂ-ಡಿ-ರು-ವಿರಾ
ತಿಳಿ-ದು-ಕೊಂ-ಡಿ-ರು-ವುದೆ
ತಿಳಿ-ದು-ಕೊಂಡು
ತಿಳಿ-ದು-ಕೊ-ಳ್ಳ-ಬಲ್ಲೆ
ತಿಳಿ-ದು-ಕೊ-ಳ್ಳ-ಬೇಕು
ತಿಳಿ-ದು-ಕೊ-ಳ್ಳ-ಬೇ-ಕೆಂದು
ತಿಳಿ-ದು-ಕೊ-ಳ್ಳ-ಲಾ-ರದು
ತಿಳಿ-ದು-ಕೊ-ಳ್ಳಲು
ತಿಳಿ-ದು-ಕೊ-ಳ್ಳು-ವನು
ತಿಳಿ-ಯ-ದ-ವರು
ತಿಳಿ-ಯದು
ತಿಳಿ-ಯದೆ
ತಿಳಿ-ಯ-ಬೇಕು
ತಿಳಿ-ಯಲಿ
ತಿಳಿ-ಯಿರಿ
ತಿಳಿಯು
ತಿಳಿ-ಯು-ವುದು
ತಿಳಿ-ವ-ಳಿ-ಕೆ-ಯನ್ನು
ತೀರ-ದಲ್ಲಿ
ತೀವ್ರ
ತೀವ್ರತೆ
ತೀವ್ರ-ವಾ-ದಂತೆ
ತುಂಡು-ಗಳನ್ನು
ತುಂಬ
ತುಂಬ-ಬ-ಹುದು
ತುಂಬ-ಬೇಕು
ತುಂಬ-ಲಾ-ರದೊ
ತುಂಬಲಿ
ತುಂಬಾ
ತುಂಬಿ
ತುಂಬಿ-ಕೊಂ-ಡಿ-ರು-ವನು
ತುಂಬಿ-ಕೊ-ಳ್ಳು-ವನು
ತುಂಬಿತ್ತು
ತುಂಬಿದ
ತುಂಬಿದೆ
ತುಂಬಿ-ದ್ದರೆ
ತುಂಬಿರ
ತುಂಬು-ವು-ದಕ್ಕೆ
ತುಚ್ಛ-ವಾಗಿ
ತುಚ್ಛ-ವಾದ
ತುತ್ತಾಗಿ
ತುತ್ತಾ-ಗಿದೆ
ತುತ್ತಾ-ದರೂ
ತುತ್ತಾ-ದ-ವ-ರಿಗೆ
ತುತ್ತು
ತುರು-ಕ-ಬೇ-ಕೆಂದು
ತುರು-ಕಿದ
ತುಳ-ಕಾ-ಡು-ತ್ತಿದ್ದ
ತುಳಿ
ತುಳಿದು
ತುಳಿ-ಯು-ತ್ತಿ-ದ್ದರೂ
ತುಳಿ-ಯು-ವು-ದಕ್ಕೆ
ತುಳು-ಕಾ-ಡು-ತ್ತಿ-ದೆಯೋ
ತುಷಾರ
ತೂಗು-ತ್ತಿ-ವೆಯೋ
ತೂರ್ಯ-ವಾ-ಣಿ-ಯಿಂದ
ತೂರ್ಯ-ವಾ-ಣಿ-ಯೊಂದು
ತೃಣ-ಕ್ಕಿಂತ
ತೃಪ್ತ-ರಾ-ಗು-ವರು
ತೃಪ್ತ-ಳಾಗಿ
ತೃಪ್ತಿ
ತೃಪ್ತಿ-ಇಲ್ಲ
ತೃಪ್ತಿ-ಪ-ಡಿಸ
ತೃಪ್ತಿ-ಪ-ಡಿ-ಸಿ-ದ-ಲ್ಲದೆ
ತೃಪ್ತಿ-ಪ-ಡಿ-ಸು-ವು-ದಕ್ಕೆ
ತೃಪ್ತಿ-ಮಾತ್ರ
ತೆಗ-ಳಿಕೆ
ತೆಗೆ-ದರೆ
ತೆಗೆದು
ತೆಗೆ-ದು-ಕೊಂ-ಡರೆ
ತೆಗೆ-ದು-ಕೊಂಡು
ತೆಗೆ-ದು-ಕೊ-ಳ್ಳದ
ತೆಗೆ-ದು-ಕೊ-ಳ್ಳ-ಲಿಲ್ಲ
ತೆಗೆ-ದು-ಕೊಳ್ಳಿ
ತೆಗೆ-ದು-ಕೊ-ಳ್ಳು-ವನು
ತೆಗೆ-ದು-ಕೊ-ಳ್ಳು-ವುದನ್ನು
ತೆಗೆ-ದು-ಕೊ-ಳ್ಳು-ವು-ದ-ರಿಂದ
ತೆಗೆ-ದು-ಕೊ-ಳ್ಳು-ವುದೆ
ತೆಗೆ-ದು-ಕೊ-ಳ್ಳು-ವೆವೊ
ತೆಗೆ-ದು-ಹಾ-ಕಿ-ದರು
ತೆಗೆ-ಯ-ಬೇ-ಕಾ-ದರೆ
ತೆಗೆ-ಯಿರಿ
ತೆಪ್ಪ-ಗಾ-ಗು-ವಿರಿ
ತೆಪ್ಪಗೆ
ತೆರು-ವಂ-ತಾ-ಗಲಿ
ತೆರೆದು
ತೆರೆ-ಯದ
ತೆರೆ-ಯ-ಬೇಕು
ತೆರೆ-ಯ-ಬೇ-ಕೆಂ-ಬುದೇ
ತೆರೆ-ಯು-ವುದು
ತೇಜ-ಸ್ಸನ್ನು
ತೊಟ್ಟಿ-ದ್ದರೂ
ತೊಟ್ಟಿ-ಲನ್ನು
ತೊಡು-ತ್ತಾನೆ
ತೊಡು-ಪು-ಗಳು
ತೊಡೆ-ಯಲ್ಲಿ
ತೊರೆದು
ತೊಳೆ-ದರೂ
ತೊಳೆ-ದು-ಕೊ-ಳ್ಳ-ಬೇಕು
ತೊಳೆ-ಯು-ವುದೆ
ತೋಚಿ-ದಂತೆ
ತೋಡಲು
ತೋಯಿ-ಸ-ಲಾ-ರದು
ತೋಯಿ-ಸಿ-ರು-ವರು
ತೋರದ
ತೋರದೆ
ತೋರ-ಬೇ-ಕಾ-ಗಿದೆ
ತೋರ-ಬೇಕು
ತೋರ-ಲೆ-ತ್ನಿ-ಸು-ವನು
ತೋರಿ
ತೋರಿ-ಕೆಯ
ತೋರಿದ
ತೋರಿ-ದಂತೆ
ತೋರಿ-ದ-ಲ್ಲದೆ
ತೋರಿ-ದು-ದನ್ನು
ತೋರಿ-ರು-ವನು
ತೋರಿ-ರು-ವುದು
ತೋರಿಸಿ
ತೋರಿ-ಸಿ-ಕೊ-ಳ್ಳ-ಬ-ಹುದು
ತೋರು
ತೋರುವ
ತೋರು-ವನು
ತೋರು-ವುದು
ತೋರು-ವುದೆ
ತೋರೆಂದು
ತೌರು-ಮನೆ
ತೌರು-ಮ-ನೆ-ಯಾದ
ತೌರೂರು
ತ್ಕಾರವೂ
ತ್ತದೆಯೊ
ತ್ತವೆ
ತ್ತವೆಯೊ
ತ್ತಾರೆ
ತ್ತಿತ್ತು
ತ್ತಿದೆ
ತ್ತಿದೆಯೆ
ತ್ತಿದ್ದ
ತ್ತಿದ್ದರು
ತ್ತಿದ್ದೆ
ತ್ತಿರುವ
ತ್ತಿರು-ವರು
ತ್ತಿಲ್ಲ
ತ್ತೇನೆ
ತ್ತೇವೆ
ತ್ತೇವೆಯೊ
ತ್ಮಿಕ
ತ್ಯಕ್ತ್ವೋ-ತ್ತಿಷ್ಠ
ತ್ಯಜಿಸ
ತ್ಯಜಿ-ಸದೇ
ತ್ಯಜಿ-ಸಲು
ತ್ಯಜಿಸಿ
ತ್ಯಜಿ-ಸಿತೋ
ತ್ಯಜಿ-ಸಿ-ದ-ವನು
ತ್ಯರ
ತ್ಯಾಗ
ತ್ಯಾಗ-ಜೀ-ವ-ನಕ್ಕೆ
ತ್ಯಾಗದ
ತ್ಯಾಗ-ಭಕ್ತಿ
ತ್ಯಾಗ-ವಿ-ಲ್ಲದೆ
ತ್ಯಾಗಿ-ಗಳ
ತ್ಯಾಗಿ-ಗ-ಳಿಗೇ
ತ್ಯಾಗಿ-ಗಳು
ತ್ಯಾಗಿ-ಯಾ-ಗಿ-ದ್ದನು
ದಂತ-ಕ-ತೆಯ
ದಂತೆ
ದಕ್ಕೆ
ದಕ್ಷ-ತೆ-ಯಿಂದ
ದಟ್ಟಾ-ಗಿದೆ
ದಡ್ಡ-ರಲ್ಲ
ದಡ್ಡ-ರಾ-ದರೆ
ದನ-ಗಳ
ದನ-ಗಳನ್ನು
ದನ-ದೊಂ-ದಿಗೆ
ದನ್ನು
ದನ್ನೂ
ದನ್ನೆಲ್ಲ
ದಬ್ಬಾ-ಳಿಕೆ
ದಬ್ಬಾ-ಳಿ-ಕೆಗೆ
ದಮ-ಯಂತಿ
ದಯಾ-ನಂದ
ದಯಾ-ನಂ-ದರು
ದಯೆ
ದಯೆ-ಯಿಂದ
ದಯೆ-ಯಿಲ್ಲ
ದರಲ್ಲಿ
ದರಾ-ಗಲಿ
ದರಿ
ದರಿದ್ರ
ದರಿ-ದ್ರರ
ದರಿ-ದ್ರ-ರಿಗೂ
ದರಿ-ದ್ರರು
ದರು
ದರೂ
ದರೆ
ದರ್ಶ-ನ-ಗಳ
ದಲಿ-ತ-ರಿಗೆ
ದಲಿ-ತರೆ
ದಲೇ
ದಲ್ಲ
ದಲ್ಲಿ
ದಲ್ಲಿದ್ದ
ದಲ್ಲಿಯೂ
ದಲ್ಲಿಯೇ
ದಲ್ಲೆಲ್ಲ
ದಲ್ಲೇ
ದಳ್ಳುರಿ
ದವರ
ದಷ್ಟು
ದಹಿ-ಸು-ವುವು
ದಾಗ
ದಾಗಲೇ
ದಾಗುವ
ದಾಟ-ಬಲ್ಲ
ದಾದ್
ದಾನ
ದಾನಕ್ಕೆ
ದಾನ-ಮಾ-ಡುತ್ತ
ದಾನ-ಮಾ-ಡು-ವುದು
ದಾನ-ವಾದ
ದಾಯ-ಕ-ವಾದ
ದಾಯಸ್ಥ
ದಾರಿ
ದಾರಿ-ತೋ-ರಲು
ದಾರಿದ್ರ್ಯ
ದಾರಿ-ದ್ರ್ಯದ
ದಾರಿ-ಯನ್ನು
ದಾರಿ-ಯಲ್ಲಿ
ದಿಂದ
ದಿಂದಲೂ
ದಿಕ್ಕಿ-ನಲ್ಲಿ
ದಿಕ್ಕು-ದಿ-ಕ್ಕಿಗೂ
ದಿಗ್ವಿ-ಜ-ಯ-ವನ್ನು
ದಿಗ್ವಿ-ಜ-ಯಿ-ಗ-ಳಾ-ಗಿ-ದ್ದೆವು
ದಿದ್ದ
ದಿನ
ದಿನ-ಗ-ಳಲ್ಲೇ
ದಿನ-ಗ-ಳಾದ
ದಿನ-ಗಳು
ದಿರು-ವುದು
ದಿಲ್ಲ
ದಿಲ್ಲವೋ
ದಿವಸ
ದಿವ್ಯೌ-ಷಧಿ
ದಿಸುವ
ದೀನ
ದೀನ-ದ-ರಿ-ದ್ರರ
ದೀನ-ದ-ಲಿ-ತ-ರನ್ನು
ದೀನ-ದ-ಲಿ-ತ-ರಿ-ಗಾಗಿ
ದೀನ-ದ-ಲಿ-ತ-ರಿಗೆ
ದೀನ-ರಿಗೆ
ದೀನರು
ದೀನರೆ
ದೀರ್ಘ-ಕಾಲ
ದೀರ್ಘಾ-ಯುಸ್ಸು
ದುಃಖ
ದುಃಖ-ವನ್ನು
ದುಃಖ-ವ-ನ್ನೆಲ್ಲ
ದುಃಖವೇ
ದುಃಖ-ಸಾ-ವು-ಗಳಿಂದ
ದುಃಖಿ
ದುಃಖಿ-ಗ-ಳಿಗೆ
ದುಃಖಿ-ಗಳೆ
ದುಃಸ್ಥಿತಿ
ದುಃಸ್ಥಿ-ತಿಗೆ
ದುಡಿತ
ದುಡಿ-ತ-ದಿಂದ
ದುಡಿದು
ದುಡಿ-ಯ-ಬೇಕು
ದುಡಿ-ಯು-ತ್ತಿ-ರು-ವರು
ದುಡಿ-ಯು-ತ್ತಿ-ರು-ವಾಗ
ದುಡ್ಡನ್ನು
ದುಡ್ಡು
ದುರ
ದುರಂ-ತಕ್ಕೆ
ದುರ-ದೃಷ್ಟ
ದುರ-ಭ್ಯಾಸ
ದುರ-ವ-ಸ್ಥೆಗೆ
ದುರ-ಹಂ-ಕಾರ
ದುರ-ಹಂ-ಕಾ-ರ-ದಿಂ-ದಲೂ
ದುರ-ಹಂ-ಕಾ-ರ-ದಿಂ-ದಲೋ
ದುರಾತ್ಮ
ದುರ್ಗಂ-ಧ-ದಿಂದ
ದುರ್ಗ-ಮ-ವಾದ
ದುರ್ಗಾ-ಸ-ಪ್ತ-ಶತಿ
ದುರ್ದ-ಮ್ಯ-ವಾದ
ದುರ್ದೆ-ಸೆಗೆ
ದುರ್ದೈ-ವ-ದಿಂದ
ದುರ್ಬಲ
ದುರ್ಬ-ಲ-ನಾದ
ದುರ್ಬ-ಲರ
ದುರ್ಬ-ಲ-ರ-ನ್ನಾಗಿ
ದುರ್ಬ-ಲ-ರಲ್ಲ
ದುರ್ಬ-ಲ-ರಾ-ಗಿ-ರು-ವೆವು
ದುರ್ಬ-ಲ-ರಿಗೆ
ದುರ್ಬ-ಲರು
ದುರ್ಬ-ಲ-ವಾ-ಗು-ವ-ವ-ರೆಗೆ
ದುರ್ಬ-ಲ-ವಾ-ಗು-ವುದು
ದುರ್ಬ-ಲ-ವಾದ
ದುಸ್ಥಿ-ತಿಗೆ
ದೂರ
ದೂರ-ದ-ರ್ಶಕ
ದೂರದೆ
ದೂರ-ದೇ-ಶ-ಗ-ಳಿಗೆ
ದೂರ-ಬೇ-ಕಾ-ಗಿಲ್ಲ
ದೂರ-ಬೇಡಿ
ದೂರ-ಹೋ-ದ-ವರು
ದೂರಿ
ದೂರು-ತ್ತೀರಿ
ದೂರು-ವರು
ದೂರು-ವುದನ್ನು
ದೂರು-ವು-ದಿಲ್ಲ
ದೃಢತೆ
ದೃಶ್ಯ
ದೃಷ್ಟದ
ದೃಷ್ಟಿ
ದೃಷ್ಟಿ-ಗಳು
ದೃಷ್ಟಿ-ಯನ್ನು
ದೃಷ್ಟಿ-ಯಲ್ಲಿ
ದೃಷ್ಟಿ-ಯಿಂದ
ದೃಷ್ಟಿ-ಯಿಂ-ದಲೇ
ದೆಯೆ
ದೇನು
ದೇವ
ದೇವ-ತಾಃ
ದೇವತೆ
ದೇವ-ತೆ-ಗಳ
ದೇವ-ತೆ-ಗ-ಳ-ನ್ನಾಗಿ
ದೇವ-ತೆ-ಗಳನ್ನು
ದೇವ-ತೆ-ಗಳು
ದೇವ-ತೆ-ಗ-ಳೆಲ್ಲ
ದೇವ-ದೇ-ವ-ತೆ-ಗಳು
ದೇವರ
ದೇವ-ರಂತೆ
ದೇವ-ರನ್ನು
ದೇವ-ರ-ಪೂ-ಜೆಯ
ದೇವ-ರ-ಲ್ಲವೆ
ದೇವ-ರಾ-ಗಲಿ
ದೇವ-ರಿಗೆ
ದೇವ-ರಿಗೇ
ದೇವರು
ದೇವ-ರು-ಗ-ಳಿಗೂ
ದೇವರೂ
ದೇವರೆ
ದೇವ-ರೆಂದು
ದೇವ-ರೊ-ಬ್ಬನೇ
ದೇವ-ಸ್ಥಾನ
ದೇವ-ಸ್ಥಾ-ನ-ಗಳನ್ನು
ದೇಶ
ದೇಶ-ಕ್ಕಾಗಿ
ದೇಶಕ್ಕೂ
ದೇಶಕ್ಕೆ
ದೇಶ-ಗಳ
ದೇಶ-ಗಳನ್ನು
ದೇಶ-ಗಳಲ್ಲಿ
ದೇಶ-ಗ-ಳ-ಲ್ಲಿಯೂ
ದೇಶ-ಗ-ಳ-ಲ್ಲೆ-ಲ್ಲಿಯೂ
ದೇಶ-ಗಳಿಂದ
ದೇಶ-ಗ-ಳಿ-ಗಿಂತ
ದೇಶ-ಗ-ಳಿಗೆ
ದೇಶ-ಗಳು
ದೇಶ-ಗಳೇ
ದೇಶದ
ದೇಶ-ದಂತೆ
ದೇಶ-ದಲ್ಲಿ
ದೇಶ-ದ-ಲ್ಲಿ-ದ್ದಾಗ
ದೇಶ-ದ-ಲ್ಲಿಯೂ
ದೇಶ-ದ-ಲ್ಲಿ-ರುವ
ದೇಶ-ದಲ್ಲೇ
ದೇಶ-ದ-ವ-ರನ್ನು
ದೇಶ-ದಿಂದ
ದೇಶ-ದಿಂ-ದಲೂ
ದೇಶ-ದ್ರೋ-ಹಿ-ಗ-ಳೆ-ನ್ನು-ತ್ತೇನೆ
ದೇಶ-ಬಾಂ-ಧ-ವರ
ದೇಶ-ಬಾಂ-ಧ-ವ-ರಿಗೆ
ದೇಶ-ಭಕ್ತ
ದೇಶ-ಭ-ಕ್ತ-ರಾಗಿ
ದೇಶ-ಭ-ಕ್ತ-ರಾ-ಗು-ವು-ದಕ್ಕೆ
ದೇಶ-ಭ-ಕ್ತರೆ
ದೇಶ-ಭಕ್ತಿ
ದೇಶ-ವನ್ನು
ದೇಶ-ವಾದ
ದೇಶವೂ
ದೇಶ-ವೆಲ್ಲ
ದೇಶವೇ
ದೇಶ-ಸೇ-ವ-ಕರ
ದೇಶೀ-ಯರು
ದೇಹ
ದೇಹ-ಗ-ತ-ವಾ-ಗಿದೆ
ದೇಹದ
ದೇಹ-ವನ್ನು
ದೇಹ-ವನ್ನೂ
ದೇಹ-ವನ್ನೇ
ದೇಹವು
ದೈನ್ಯತೆ
ದೈವೀ-ಸಂ-ಪ-ತ್ತಿ-ನಿಂದ
ದೈಹಿಕ
ದೈಹಿ-ಕ-ವಾಗಿ
ದೊಂಬಿಯ
ದೊಡ್ಡ
ದೊಡ್ಡ-ದಾ-ಗು-ವು-ದಿಲ್ಲ
ದೊಡ್ಡದು
ದೊಡ್ಡದೇ
ದೊಡ್ಡ-ದೊಡ್ಡ
ದೊಡ್ಡ-ವ-ನ-ಲ್ಲವೆ
ದೊಡ್ಡ-ವರು
ದೊರ-ಕ-ಬ-ಲ್ಲುದು
ದೊರ-ಕು-ತ್ತಿ-ರುವ
ದೊರ-ಕು-ತ್ತಿಲ್ಲ
ದೊರ-ಕು-ವಂತೆ
ದೊರ-ಕು-ವು-ದಿಲ್ಲ
ದೊರ-ಕು-ವುದು
ದೊರೆತ
ದೋಚಲು
ದೋಷ-ಗಳನ್ನು
ದೋಷ-ಪೂ-ರಿ-ತ-ವಾಗಿ
ದೋಷವೇ
ದೋಷಾ-ರೋ-ಪಣೆ
ದೋಷಾ-ರೋ-ಪ-ಣೆ-ಯಲ್ಲ
ದೌರಾತ್ಮ್ಯ
ದೌರ್ಜ-ನ್ಯ-ದಿಂದ
ದೌರ್ಬಲ್ಯ
ದೌರ್ಬಲ್ಯಂ
ದೌರ್ಬ-ಲ್ಯ-ಗ-ಳಿಂ-ದಲೂ
ದೌರ್ಬ-ಲ್ಯದ
ದೌರ್ಬ-ಲ್ಯ-ವನ್ನು
ದೌರ್ಬ-ಲ್ಯ-ವಾ-ಗಿದೆ
ದೌರ್ಬ-ಲ್ಯವೇ
ದೌರ್ಭಾ-ಗ್ಯ-ರನ್ನು
ದ್ದರು
ದ್ದರೂ
ದ್ದರೆ
ದ್ದಳು
ದ್ದವು
ದ್ರವ್ಯ
ದ್ರವ್ಯವೊ
ದ್ರವ್ಯ-ಸಂ-ಪಾ-ದ-ನೆಯ
ದ್ರಷ್ಟಾ-ರನ
ದ್ರಾವಿಡ
ದ್ರಾವಿ-ಡರು
ದ್ವಾಪರ
ದ್ವೇಷ
ದ್ವೇಷ-ವನ್ನು
ದ್ವೇಷಿ-ಸದೆ
ದ್ವೇಷಿಸಿ
ದ್ವೇಷಿ-ಸುವ
ಧನ
ಧನ್ಯ
ಧನ್ಯ-ರಾ-ಗಲು
ಧನ್ಯ-ವಾ-ದ-ಗಳು
ಧನ್ಯ-ವಾ-ದ-ವನ್ನು
ಧಮನಿ
ಧಮ-ನಿ-ಧ-ಮ-ನಿ-ಯಲ್ಲಿ
ಧಮ-ನಿ-ಯಲ್ಲಿ
ಧರಿ-ಸು-ವಂತೆ
ಧರ್ಮ
ಧರ್ಮಕ್ಕೆ
ಧರ್ಮ-ಜ-ಯ-ವೆಂದು
ಧರ್ಮದ
ಧರ್ಮ-ದಲ್ಲಿ
ಧರ್ಮ-ದ-ಲ್ಲಿ-ರುವ
ಧರ್ಮ-ದಿಂದ
ಧರ್ಮ-ದೊಂ-ದಿಗೆ
ಧರ್ಮ-ವನ್ನು
ಧರ್ಮ-ವ-ಲ್ಲದೆ
ಧರ್ಮ-ವಿ-ಷ-ಯ-ಗಳನ್ನು
ಧರ್ಮ-ವೆಂದರೆ
ಧರ್ಮ-ವೆಲ್ಲಾ
ಧರ್ಮವೇ
ಧರ್ಮ-ವೊಂದೇ
ಧರ್ಮ-ಶಾ-ಸ್ತ್ರ-ವನ್ನು
ಧರ್ಮ-ಸಂ-ಸ್ಥಾ-ಪನಾ
ಧಾರೆ
ಧಾರೆ-ಯೆರೆ
ಧಾರೆ-ಯೆ-ರೆ-ಯು-ವುದು
ಧಾರ್ಮಿಕ
ಧಾರ್ಮಿ-ಕದ
ಧಾರ್ಮಿ-ಕ-ದೃ-ಷ್ಟಿ-ಯಿಂದ
ಧಾವಿ-ಸ-ಬ-ಲ್ಲಿರಾ
ಧಾವಿಸು
ಧೀರ
ಧೀರ-ರಾಗಿ
ಧೀರ-ರಾ-ಗಿ-ರ-ಬೇಕು
ಧೀರ-ರಾದ
ಧೀರರೆ
ಧೂಳನ್ನೇ
ಧೃಡ-ತೆ-ಯನ್ನು
ಧೈರ್ಯ
ಧೈರ್ಯ-ಗಳನ್ನು
ಧೈರ್ಯ-ದಿಂದ
ಧೈರ್ಯ-ದಿಂ-ದಲೂ
ಧೈರ್ಯ-ವನ್ನು
ಧೈರ್ಯ-ವಾಗಿ
ಧೈರ್ಯ-ವಿಲ್ಲ
ಧ್ಯಾನ
ಧ್ಯಾನಕ್ಕೆ
ಧ್ಯಾನ-ದಲ್ಲಿ
ಧ್ಯಾನಾ-ವ-ಸ್ಥೆಯ
ಧ್ಯಾಯರು
ಧ್ರಡಿಷ್ಠ
ಧ್ವಂಸ-ಮಾ-ಡಲು
ಧ್ವಂಸ-ಮಾಡು
ಧ್ವಂಸ-ಮಾ-ಡು-ವು-ದಕ್ಕೆ
ಧ್ವಂಸ-ವಾ-ಗದೆ
ಧ್ವಜ
ಧ್ವನಿ
ಧ್ವನಿ-ಯನ್ನು
ಧ್ವನಿ-ಯಿತ್ತು
ನಂತರ
ನಂತ-ರವೂ
ನಂತಿ-ರುವ
ನಂದಾ-ದೀ-ವಿ-ಗೆ-ಯಂತೆ
ನಂಬ-ತೊ-ಡ-ಗಿ-ದನು
ನಂಬಿ
ನಂಬಿಕೆ
ನಂಬಿ-ರು-ವುದನ್ನು
ನಂಬು-ತ್ತೇನೆ
ನಂಬು-ವರೊ
ನಂಬು-ವಿ-ರೇನು
ನಂಬು-ವು-ದಿಲ್ಲ
ನಂಬು-ವೆವು
ನಕಲು
ನಗ-ಣ್ಯರು
ನಗ-ರ-ಗಳನ್ನು
ನಗು-ವನು
ನಡ
ನಡ-ತೆ-ಯಲ್ಲಿ
ನಡೆ
ನಡೆ-ದದ್ದೆ
ನಡೆ-ಯಲ್ಲಿ
ನಡೆ-ಯು-ವಂತೆ
ನಡೆ-ಸಿ-ದರೆ
ನಡೆ-ಸು-ತ್ತಿ-ರು-ವಾಗ
ನಡೆ-ಸು-ವು-ದಲ್ಲ
ನತೆ
ನದನ್ನು
ನದಿ-ಗಳು
ನದಿಗೆ
ನದಿಯ
ನದಿ-ಸಾ-ಗ-ರ-ಗಳು
ನದೀ-ತೀರ
ನನ-ಗಿಂತ
ನನಗೆ
ನನ-ಗೇ-ತಕ್ಕೆ
ನನ-ಗೇ-ನಾ-ದರೂ
ನನ್ನ
ನನ್ನಂತೆ
ನನ್ನನ್ನು
ನನ್ನಲ್ಲಿ
ನನ್ನ-ಲ್ಲಿ-ರುವ
ನಮ-ಗಲ್ಲ
ನಮ-ಗಾಗಿ
ನಮ-ಗಾ-ಗು-ತ್ತಿ-ರುವ
ನಮ-ಗಿಂದು
ನಮಗೂ
ನಮಗೆ
ನಮ-ಗೊಂದು
ನಮ-ಸ್ಕಾ-ರ-ಮಾ-ಡ-ಬಾ-ರದು
ನಮ್ಮ
ನಮ್ಮಂ-ತಹ
ನಮ್ಮಂ-ತೆಯೇ
ನಮ್ಮ-ಗಳಲ್ಲಿ
ನಮ್ಮ-ಗ-ಳೆ-ಲ್ಲರ
ನಮ್ಮದೇ
ನಮ್ಮನ್ನು
ನಮ್ಮಲ್ಲಿ
ನಮ್ಮ-ಲ್ಲಿಯೂ
ನಮ್ಮ-ಲ್ಲಿಯೇ
ನಮ್ಮ-ಲ್ಲಿ-ರುವ
ನಮ್ಮ-ಲ್ಲಿ-ರು-ವಂತೆ
ನಮ್ಮ-ಲ್ಲಿ-ರು-ವುದು
ನಮ್ಮ-ಲ್ಲಿ-ರು-ವುದೋ
ನಮ್ಮ-ಲ್ಲೆಲ್ಲ
ನಮ್ಮಲ್ಲೇ
ನಮ್ಮ-ವನು
ನಮ್ಮ-ವ-ರನ್ನೇ
ನಮ್ಮ-ವರು
ನಮ್ಮಷ್ಟು
ನಮ್ಮಿಂದ
ನಮ್ಮೀ
ನರ-ಕಕ್ಕೆ
ನರ-ಕು-ರಿ-ಗ-ಳೆ-ದೆ-ಯಲ್ಲಿ
ನರ-ಗಳು
ನರ-ನಾ-ರಿ-ಯರ
ನರ-ಳಿ-ರು-ವೆವು
ನರಳು
ನರ-ಳು-ತ್ತಿ-ದ್ದರು
ನರ-ಳು-ತ್ತಿ-ರು-ವರು
ನರ-ಳು-ತ್ತಿ-ರು-ವರೊ
ನರ-ಳು-ತ್ತಿ-ರು-ವಿರಿ
ನಲ್ಲಿ
ನಲ್ಲಿಯೂ
ನಲ್ಲಿಯೇ
ನಲ್ಲಿ-ರುವ
ನವ-ಜೀ-ವನ
ನವ-ಜೀ-ವ-ನ-ವನ್ನು
ನವರ
ನವರು
ನವೀ-ನ-ತೆಯೂ
ನವೋ-ತ್ಸಾ-ಹ-ದಿಂದ
ನವೋ-ದ-ಯ-ವಾಗು
ನವೋ-ದ-ಯ-ವಾ-ಗು-ತ್ತಿದೆ
ನಶಿ-ಸಿ-ಹೋಗು
ನಷ್ಟ
ನಾಗ-ಬೇ-ಕಾ-ದರೆ
ನಾಗರಿ
ನಾಗ-ರಿಕ
ನಾಗ-ರಿ-ಕತೆ
ನಾಗ-ರಿ-ಕ-ತೆಗೂ
ನಾಗ-ರಿ-ಕ-ತೆಯ
ನಾಗ-ರಿ-ಕ-ತೆ-ಯನ್ನು
ನಾಗ-ರಿ-ಕ-ತೆ-ಯಲ್ಲಿ
ನಾಗ-ರಿ-ಕ-ತೆ-ಯಿಂದ
ನಾಗ-ರಿ-ಕ-ತೆಯೂ
ನಾಗ-ರಿ-ಕ-ತೆಯೇ
ನಾಗ-ರಿ-ಕ-ರಿಲ್ಲ
ನಾಗಿ
ನಾಚಿಕೆ
ನಾಜೂ
ನಾಜೂ-ಕಾಗಿ
ನಾಟ-ಕ-ಗಳನ್ನು
ನಾಟಿ-ದ್ದರೆ
ನಾಡಲಿ
ನಾಡಿ
ನಾಡಿ-ಯಲ್ಲಿ
ನಾಡು
ನಾಡು-ವ-ವನೆ
ನಾಣ್ಯದ
ನಾನು
ನಾನೆಂಬು
ನಾನೇ
ನಾನೇ-ನಾ-ದರೂ
ನಾನೇನು
ನಾನೊಂದು
ನಾನೊ-ಬಪ್ನೇ
ನಾನೊಬ್ಬ
ನಾನೊ-ಬ್ಬ-ನೆಂದು
ನಾಮಾ-ವ-ಶೇ-ಷ-ವಾಗಿ
ನಾಮಾ-ವ-ಶೇ-ಷ-ವಾ-ಗಿಲ್ಲ
ನಾಮಾ-ವ-ಶೇ-ಷ-ವಾ-ಗಿವೆ
ನಾಮಾ-ವ-ಶೇ-ಷ-ವಾ-ಗು-ವಿರಿ
ನಾಯ-ಕ-ಗಿ-ರಿಗೆ
ನಾಯ-ಕತ್ವ
ನಾಯ-ಕ-ನಂತೆ
ನಾಯ-ಕ-ನಾಗ
ನಾಯ-ಕ-ನಾ-ಗ-ಬೇ-ಕೆಂಬ
ನಾಯ-ಕ-ನಾ-ಗ-ಬೇಡಿ
ನಾಯ-ಕ-ನಾ-ಗಲು
ನಾಯ-ಕನೇ
ನಾಯ-ಕ-ರಾ-ಗಿ-ರ-ಬೇಕು
ನಾಯ-ಕರು
ನಾರಿ
ನಾರಿ-ಕು-ಲಕ್ಕೆ
ನಾರಿ-ಯರ
ನಾರಿ-ಯ-ರನ್ನು
ನಾರಿ-ಯರು
ನಾರೀ
ನಾರು
ನಾರು-ವುದು
ನಾರ್ಯಸ್ತು
ನಾಲ್ಕ-ನೆ-ಯದೆ
ನಾಲ್ಕು
ನಾಲ್ಕು-ಕೋಟಿ
ನಾಳೆ
ನಾಳೆಯೆ
ನಾಳೆಯೋ
ನಾವಾ-ದರೋ
ನಾವಿನ್ನೂ
ನಾವಿ-ರುವ
ನಾವಿಲ್ಲಿ
ನಾವೀಗ
ನಾವು
ನಾವುಂಡು
ನಾವೂ
ನಾವೆಲ್ಲ
ನಾವೆ-ಲ್ಲರೂ
ನಾವೆ-ಲ್ಲ-ವನ್ನೂ
ನಾವೇ
ನಾವೊಂದು
ನಾವೊ-ಬ್ಬರು
ನಾಶ
ನಾಶಕ್ಕೆ
ನಾಶ-ದಿಂದ
ನಾಶ-ಮಾ-ಡ-ಬೇಕಾ
ನಾಶ-ಮಾ-ಡ-ಲಾ-ರದು
ನಾಶ-ಮಾ-ಡು-ವು-ದಕ್ಕೆ
ನಾಶ-ಮಾ-ಡು-ವುದು
ನಾಶ-ವಾ-ಗ-ದಿ-ದ್ದರೆ
ನಾಶ-ವಾ-ಗದೆ
ನಾಶ-ವಾ-ಗ-ಬ-ಹುದು
ನಾಶ-ವಾಗಿ
ನಾಶ-ವಾ-ಗು-ವಿರಿ
ನಾಶ-ವಾ-ಗು-ವು-ದಿ-ಲ್ಲವೋ
ನಾಶ-ವಾ-ಗು-ವುದು
ನಾಶ-ವಾ-ದಂತೆ
ನಾಶ-ವಾ-ದರೂ
ನಾಸ್ತಿ-ಕತೆ
ನಾಸ್ತಿ-ಕ-ರಿ-ಗಿಂತ
ನಾಸ್ತಿ-ಕ-ವಾದ
ನಿಂತ
ನಿಂತಂತೆ
ನಿಂತರೆ
ನಿಂತ-ವರ
ನಿಂತ-ವ-ರಲ್ಲಿ
ನಿಂತಿದೆ
ನಿಂತಿರು
ನಿಂತಿ-ರು-ವುದು
ನಿಂತಿ-ರು-ವುದೋ
ನಿಂತು
ನಿಂತೊ-ಡ-ನೆಯೆ
ನಿಂದಲೇ
ನಿಃಸ್ಪೃ-ಹ-ರಾ-ಗಿ-ದ್ದರೆ
ನಿಃಸ್ವಾ-ರ್ಥ-ತೆ-ಯಿಂದ
ನಿಃಸ್ವಾ-ರ್ಥ-ವಾಗಿ
ನಿಃಸ್ವಾ-ರ್ಥಿ-ಗಳೆ
ನಿಃಸ್ವಾ-ರ್ಥಿಯೇ
ನಿಕಟ
ನಿಕ-ಟ-ಬಂ-ಧು-ಗಳು
ನಿಕೃ-ಷ್ಟ-ದೃ-ಷ್ಟಿ-ಯಿಂದ
ನಿಕೃ-ಷ್ಟ-ವಾಗಿ
ನಿಗೂ
ನಿಗೆ
ನಿಗ್ರ-ಹಿ-ಸಿ-ರು-ವನೊ
ನಿಗ್ರ-ಹಿಸು
ನಿಜ
ನಿಜ-ವಾಗಿ
ನಿಜ-ವಾ-ಗಿ-ದ್ದರೆ
ನಿಜ-ವಾ-ಗಿಯೂ
ನಿಜ-ವಾದ
ನಿಜ-ಸ್ಥಿ-ತಿಯೇ
ನಿತ್ಯ-ಜೀ-ವನ
ನಿತ್ಯ-ಜೀ-ವ-ನಕ್ಕೆ
ನಿತ್ಯ-ಜೀ-ವ-ನದ
ನಿತ್ಯ-ಮುಕ್ತ
ನಿತ್ಯ-ವಾ-ಗಿ-ದೆಯೊ
ನಿತ್ಯ-ವಾ-ದುದು
ನಿತ್ಯವೂ
ನಿತ್ರಾ-ಣ-ನಾ-ಗು-ವನು
ನಿತ್ರಾ-ಣ-ರಾಗಿ
ನಿದ್ದೆ
ನಿದ್ದೆ-ಯಿಂದ
ನಿದ್ರಾ-ಭಂಗ
ನಿದ್ರಿಸು
ನಿದ್ರಿ-ಸು-ತ್ತಿ-ರು-ವಳು
ನಿದ್ರಿ-ಸು-ವು-ದಕ್ಕೆ
ನಿದ್ರಿ-ಸು-ವುದು
ನಿದ್ರೆ
ನಿದ್ರೆಗೆ
ನಿದ್ರೆ-ಯಿಂದ
ನಿದ್ರೆ-ಯಿಂ-ದೆ-ದ್ದೇಳಿ
ನಿದ್ರೆ-ಯಿಂ-ದೇ-ಳು-ತ್ತಿದೆ
ನಿದ್ರೆ-ಯಿಂ-ದೇ-ಳು-ತ್ತಿ-ರು-ವಳು
ನಿಧಾನ
ನಿಧಾ-ನ-ವಾಗಿ
ನಿಧಿ
ನಿಧಿ-ಗಳನ್ನು
ನಿಧಿಗೆ
ನಿಧಿಯ
ನಿಧಿ-ಯನ್ನು
ನಿನಗೆ
ನಿನ್ನ
ನಿನ್ನಂತೆ
ನಿನ್ನಂ-ತೆಯೇ
ನಿಮ-ಗಿಂತ
ನಿಮಗೆ
ನಿಮ-ಗೆಲ್ಲ
ನಿಮಿ-ಷ-ಗಳಲ್ಲಿ
ನಿಮ್ನ
ನಿಮ್ನ-ವ-ರ್ಗ-ದ-ವ-ರನ್ನು
ನಿಮ್ನ-ವ-ರ್ಗ-ದ-ವರು
ನಿಮ್ನ-ವ-ರ್ಗ-ದ-ವ-ರೆಂದು
ನಿಮ್ಮ
ನಿಮ್ಮಂತೆ
ನಿಮ್ಮ-ಗ-ಳಿ-ಗೆಲ್ಲ
ನಿಮ್ಮ-ದಾ-ದರೆ
ನಿಮ್ಮದು
ನಿಮ್ಮನ್ನು
ನಿಮ್ಮ-ನ್ನೆಲ್ಲ
ನಿಮ್ಮನ್ನೇ
ನಿಮ್ಮಲ್ಲಿ
ನಿಮ್ಮ-ಲ್ಲಿದೆ
ನಿಮ್ಮ-ಲ್ಲಿ-ದೆಯೆ
ನಿಮ್ಮ-ಲ್ಲಿಯೂ
ನಿಮ್ಮ-ಲ್ಲಿ-ರುವ
ನಿಮ್ಮ-ವನು
ನಿಮ್ಮೊ-ಡನೆ
ನಿಮ್ಮೊ-ಳಗೆ
ನಿಯಮ
ನಿಯ-ಮ-ಗಳು
ನಿಯ-ಮ-ವನ್ನು
ನಿಯ-ಮಾ-ನು-ಸಾರ
ನಿರ-ತ-ನಾ-ಗಿ-ದ್ದರೂ
ನಿರ-ತ-ನಾ-ಗಿ-ರುವ
ನಿರ-ತ-ರಾಗಿ
ನಿರ-ತ-ರಾ-ಗಿ-ರು-ವರು
ನಿರ-ತ-ರಾಗು
ನಿರ-ರ್ಥ-ಕ-ವಾ-ಗು-ವು-ದಿಲ್ಲ
ನಿರೀ-ಕ್ಷಿ-ಸ-ಬ-ಲ್ಲಿರಿ
ನಿರೀ-ಕ್ಷಿ-ಸ-ಬಾ-ರ-ದಾ-ಗಿತ್ತು
ನಿರೀ-ಕ್ಷಿಸು
ನಿರ್ಜೀ-ವ-ದಂ-ತಿದ್ದ
ನಿರ್ಜೀ-ವ-ದಂತೆ
ನಿರ್ಜೀ-ವ-ನಾಗಿ
ನಿರ್ದಯ
ನಿರ್ದ-ಯ-ತೆ-ಗ-ಳಿಂ
ನಿರ್ದ-ಯ-ವಾ-ಗಿದೆ
ನಿರ್ದಿ-ಷ್ಟ-ರಾ-ಗಿರಿ
ನಿರ್ದಿ-ಷ್ಟ-ವಾದ
ನಿರ್ಧ-ರಿ-ಸ-ಲ್ಪ-ಟ್ಟಿದೆ
ನಿರ್ನಾ-ಮ-ವಾ-ಗ-ಬೇಕು
ನಿರ್ನಾ-ಮ-ವಾಗಿ
ನಿರ್ನಾ-ಮ-ವಾ-ಗಿ-ದೆಯೋ
ನಿರ್ನಾ-ಮ-ವಾ-ಗಿ-ಹೋ-ಗ-ಬೇ-ಕಾ-ಗಿತ್ತು
ನಿರ್ನಾ-ಮ-ವಾ-ಗು-ವುದು
ನಿರ್ಭ-ಯತೆ
ನಿರ್ಮಾಣ
ನಿರ್ಮಾ-ಣಕ್ಕೆ
ನಿರ್ಮಿ-ಸಿ-ದಿರಿ
ನಿರ್ಮಿ-ಸು-ವು-ದಕ್ಕೆ
ನಿರ್ಮೂಲ
ನಿರ್ಮೂ-ಲ-ಮಾ-ಡ-ಬ-ಲ್ಲದು
ನಿರ್ಲ-ಕ್ಷಿ-ಸಿದ್ದು
ನಿರ್ವಂ-ಚ-ನೆ-ಯಿಂದ
ನಿರ್ವ-ಹ-ಣೆ-ಯ-ಮೇಲೆ
ನಿರ್ವ-ಹಿ-ಸ-ಬೇ-ಕಾ-ಗಿದೆ
ನಿರ್ವ-ಹಿ-ಸಲು
ನಿರ್ವ-ಹಿ-ಸುವು
ನಿರ್ವಿ-ಧ-ವಾದ
ನಿಲು
ನಿಲು-ಕದ
ನಿಲ್ಲದೆ
ನಿಲ್ಲ-ಬೇಕು
ನಿಲ್ಲ-ಬೇಡ
ನಿಲ್ಲ-ಲಿಲ್ಲ
ನಿಲ್ಲಿ
ನಿಲ್ಲಿ-ಸಿ-ದರೆ
ನಿಲ್ಲು-ತ್ತಿ-ರು-ವಳು
ನಿಲ್ಲು-ವನು
ನಿಲ್ಲು-ವುದನ್ನು
ನಿಲ್ಲು-ವುದು
ನಿವಾ-ರ-ಣೋ-ಪಾ-ಯಕ್ಕೆ
ನಿವಾ-ರಿ-ಸ-ಬೇ-ಕಾ-ದರೆ
ನಿಶ್ಚ-ಯಿ-ಸಲಿ
ನಿಷೇ-ಧ-ಗಳನ್ನು
ನಿಷೇ-ಧದ
ನಿಷೇ-ಧ-ಮಯ
ನಿಷೇ-ಧ-ಮ-ಯ-ವಾದ
ನಿಷೇ-ಧ-ಮ-ಯ-ವಾ-ದುದು
ನಿಷ್ಠೆ-ಯಿಂದ
ನಿಷ್ಪ್ರ-ಯೋ-ಜಕ
ನಿಷ್ಪ್ರ-ಯೋ-ಜ-ಕ-ರಾ-ಗ-ಬ-ಲ್ಲೆವು
ನಿಷ್ಪ್ರ-ಯೋ-ಜನ
ನಿಷ್ಪ್ರ-ಯೋ-ಜ-ನ-ವಾಗು
ನಿಷ್ಪ್ರ-ಯೋ-ಜ-ನ-ವಾ-ಯಿತು
ನಿಷ್ಪ್ರ-ಯೋ-ಜ-ನ-ವೆಂದು
ನಿಷ್ಫ-ಲ-ವಾ-ಗು-ವುದು
ನಿಸ-ರ್ಗದ
ನಿಸ್ಸಂ-ಕೋ-ಚ-ವಾಗಿ
ನಿಸ್ಸಂ-ದೇ-ಹ-ವಾಗಿ
ನೀಚ
ನೀಚ-ತನ
ನೀಡ-ಬ-ಲ್ಲಂ-ತಹ
ನೀಡ-ಬಲ್ಲೆ
ನೀಡ-ಬ-ಹುದು
ನೀಡ-ಬೇ-ಕಾ-ಗಿದೆ
ನೀಡ-ಬೇಕು
ನೀಡ-ಲಾ-ರವು
ನೀಡ-ಲಿಲ್ಲ
ನೀಡಿ-ದ್ದರೂ
ನೀಡಿ-ರುವ
ನೀಡುವ
ನೀಡು-ವುದು
ನೀತಿ
ನೀತಿಯ
ನೀತಿಯೆ
ನೀತಿ-ವಂತ
ನೀತಿ-ವಂ-ತರು
ನೀನು
ನೀನೂ
ನೀನೆ
ನೀನೇ
ನೀನೊಂದು
ನೀನೊಬ್ಬ
ನೀನೊ-ಬ್ಬನೇ
ನೀರನ್ನು
ನೀರಿನ
ನೀರಿ-ನಂತೆ
ನೀರಿ-ನ-ಮೇಲೆ
ನೀರು
ನೀರೆ-ರೆ-ದರೆ
ನೀರ್ಗ-ಲ್ಲಿನ
ನೀವೀಗ
ನೀವು
ನೀವು-ಗ-ಳೆಲ್ಲ
ನೀವೂ
ನೀವೆ
ನೀವೆಲ್ಲ
ನೀವೆ-ಲ್ಲರೂ
ನೀವೇ
ನೀವೇಕೆ
ನೀವೇನು
ನೀವೇನೋ
ನುಚ್ಚು
ನುಡಿ-ಗಳನ್ನು
ನೂಕಲು
ನೂಕಿ-ದರೆ
ನೂಕಿದ್ದು
ನೂರಾ-ಗ-ಬ-ಹುದು
ನೂರಾರು
ನೂರು-ಮಂದಿ
ನೂರು-ವ-ರು-ಷ-ಗಳಿಂದ
ನೆಚ್ಚಿ
ನೆಚ್ಚಿ-ಕೊಂ-ಡಿರು
ನೆಚ್ಚಿ-ಕೊ-ಳ್ಳ-ಬೇಡಿ
ನೆಚ್ಚು-ಗೆ-ಡ-ಕೂ-ಡದು
ನೆಚ್ಚು-ಗೆ-ಡ-ಲಿಲ್ಲ
ನೆಚ್ಚು-ಗೆ-ಯನ್ನು
ನೆತ್ತಿಯ
ನೆನ-ಪಿಗೆ
ನೆನ್ನೆ
ನೆಯ
ನೆಯ್ಗೆ-ಯ-ವರು
ನೆರ-ವಿಗೆ
ನೆರ-ವೇ-ರಿ-ಸಿ-ಕೊ-ಳ್ಳ-ಬ-ಲ್ಲ-ವ-ರಾ-ಗಿ-ರ-ಬೇಕು
ನೆರ-ವೇ-ರಿ-ಸು-ವು-ದರ
ನೆರೆ-ವೇ-ರು-ವು-ದಿಲ್ಲ
ನೆರೆ-ಹೊ-ರೆಯ
ನೆಲ-ಸಮ
ನೆಲ-ಸಿರು
ನೆಲೆಯ
ನೆಲೆ-ಸಿ-ರು-ವನು
ನೇತಾ-ಡು-ತ್ತಿದ್ದ
ನೇಯು-ತ್ತಿ-ರು-ವುದು
ನೈಜ
ನೈತ-ತ್ತ್ವ-ಯ್ಯು-ಪ-ಪ-ದ್ಯತೇ
ನೈತಿಕ
ನೊಬ್ಬನೇ
ನೋಟ
ನೋಟ-ದಲ್ಲಿ
ನೋಡ
ನೋಡ-ದಿರಿ
ನೋಡದೆ
ನೋಡ-ಬಾ-ರದು
ನೋಡ-ಬೇಕು
ನೋಡಲಿ
ನೋಡಲು
ನೋಡಿ
ನೋಡಿಕೊ
ನೋಡಿ-ಕೊ-ಳ್ಳ-ಬ-ಹುದು
ನೋಡಿ-ಕೊ-ಳ್ಳ-ಬೇಕಾ
ನೋಡಿ-ಕೊಳ್ಳಿ
ನೋಡಿ-ಕೊ-ಳ್ಳಿ-ಅದೇ
ನೋಡಿ-ಕೊಳ್ಳು
ನೋಡಿ-ಕೊ-ಳ್ಳುವ
ನೋಡಿ-ಕೊ-ಳ್ಳು-ವ-ವರು
ನೋಡಿದ
ನೋಡಿ-ದಂ-ತೆಲ್ಲಾ
ನೋಡಿ-ದರೆ
ನೋಡಿ-ದಾಗ
ನೋಡಿದೆ
ನೋಡಿ-ರು-ವರು
ನೋಡಿ-ರು-ವೆನು
ನೋಡಿ-ರು-ವೆವು
ನೋಡು
ನೋಡು-ತ್ತ-ದೆಯೊ
ನೋಡು-ತ್ತಾನೆ
ನೋಡು-ತ್ತಿ-ರು-ವೆನು
ನೋಡು-ತ್ತಿ-ರು-ವೆವು
ನೋಡು-ತ್ತೀರಿ
ನೋಡು-ತ್ತೇವೆ
ನೋಡು-ವನೊ
ನೋಡು-ವರು
ನೋಡು-ವರೊ
ನೋಡು-ವಿರಿ
ನೋಡು-ವು-ದಕ್ಕೆ
ನೋಡು-ವು-ದಿ-ಲ್ಲವೊ
ನೋಡು-ವೆವು
ನೋಡೋಣ
ನೋವನ್ನು
ನ್ನಾಗಿ
ನ್ನಾಳ-ಬ-ಹುದು
ನ್ನುಂಟು-ಮಾ-ಡು-ವುದೂ
ನ್ನೆಲ್ಲಾ
ನ್ಯಾಯ-ವಾ-ಗಿಯೋ
ನ್ಯೂ
ನ್ಯೂನತೆ
ನ್ಯೂನ-ತೆ-ಗ-ಳಿ-ಗೆಲ್ಲ
ನ್ಯೂನ-ತೆ-ಗಳು
ನ್ಯೂನ-ತೆ-ಗೆಲ್ಲ
ನ್ಯೂನ-ತೆ-ಗೆಲ್ಲಾ
ನ್ಯೂನ-ತೆ-ಯನ್ನು
ನ್ಯೂನ-ತೆ-ಯೊಂದು
ಪಂಕ್ತಿ-ಯ-ನ್ನಾಗಿ
ಪಂಗ-ಡ-ಗಳ
ಪಂಗ-ಡ-ಗ-ಳಾಗಿ
ಪಂಗ-ಡ-ವಾಗಿ
ಪಂಡಿ-ತನೋ
ಪಂಡಿ-ತರು
ಪಂಥ-ದ-ವ-ರಿಗೆ
ಪಂಥವೂ
ಪಟು
ಪಟು-ಗಳು
ಪಟ್ಟದ್ದು
ಪಟ್ಟಾಗ
ಪಟ್ಟು
ಪಡ-ಬಾ-ರದ
ಪಡಿ
ಪಡಿ-ಸ-ಬ-ಲ್ಲಿರಾ
ಪಡಿಸಿ
ಪಡಿ-ಸಿ-ದರೆ
ಪಡು-ತ್ತಿ-ರು-ವರು
ಪಡು-ವರು
ಪಡು-ವು-ದಿಲ್ಲ
ಪಡೆದ
ಪಡೆ-ದ-ವನು
ಪಡೆ-ದಿ-ದ್ದರೆ
ಪಡೆ-ದಿ-ರ-ಬೇಕು
ಪಡೆ-ದಿ-ರು-ವಳೊ
ಪಡೆ-ದಿ-ರು-ವುದನ್ನು
ಪಡೆದು
ಪಡೆ-ದು-ಕೊಂ-ಡರೊ
ಪಡೆಯ
ಪಡೆ-ಯದೆ
ಪಡೆ-ಯ-ಬ-ಹುದು
ಪಡೆ-ಯ-ಬೇ-ಕೆಂಬ
ಪಡೆ-ಯಲು
ಪಡೆ-ಯಿರಿ
ಪಡೆ-ಯು-ತ್ತಾನೆ
ಪಡೆ-ಯು-ವು-ದ-ಕ್ಕಾಗಿ
ಪಡೆ-ಯು-ವು-ದಕ್ಕೆ
ಪಡೆ-ಯು-ವುದು
ಪಡೆ-ಯು-ವು-ದೊಂದೇ
ಪಡೆ-ಯೋಣ
ಪತ-ನ-ದಿಂದ
ಪತಿ-ವ್ರ-ತಳು
ಪತಿ-ವ್ರ-ತಾ-ಶಿ-ರೋ-ಮಣಿ
ಪದ
ಪದ-ಚ್ಯು-ತ-ನ-ನ್ನಾಗಿ
ಪದ-ತ-ಳ-ದಲ್ಲಿ
ಪದ-ತ-ಳ-ದ-ಲ್ಲಿಯೂ
ಪದ-ವನ್ನು
ಪದವಿ
ಪದ-ವಿ-ದ್ದರೆ
ಪದ-ವೀ-ಧ-ರ-ನಾ-ಗುವ
ಪದ-ವೀ-ಧ-ರ-ನಾ-ಗು-ವು-ದಕ್ಕೆ
ಪದ-ವೀ-ಧ-ರ-ನಿಗೆ
ಪದೇ
ಪದ್ಧ-ತಿ-ಗಳನ್ನು
ಪರ
ಪರಂ-ತಪ
ಪರಂ-ಪರೆ
ಪರ-ದೇ-ಶ-ಗ-ಳಿಗೆ
ಪರ-ದೇ-ಶದ
ಪರ-ದೇ-ಶ-ದಲ್ಲಿ
ಪರ-ದೇ-ಶ-ದ-ಲ್ಲಿ-ರು-ವೆನು
ಪರ-ದೇ-ಶ-ದ-ವರ
ಪರ-ದೇ-ಶ-ದ-ವರು
ಪರ-ದೇ-ಶಿ-ಯರು
ಪರ-ಪು-ರು-ಷನ
ಪರ-ಬ್ರ-ಹ್ಮನ
ಪರ-ಬ್ರ-ಹ್ಮ-ನನ್ನು
ಪರ-ಬ್ರ-ಹ್ಮ-ನಲ್ಲಿ
ಪರ-ಬ್ರ-ಹ್ಮ-ನಿ-ರು-ವನು
ಪರ-ಬ್ರ-ಹ್ಮನೇ
ಪರ-ಭಾ-ಷೆ-ಯಲ್ಲಿ
ಪರಮ
ಪರ-ಮ-ಪ-ವಿತ್ರ
ಪರ-ಮ-ಸ-ತ್ಯ-ವನ್ನು
ಪರ-ಮ-ಸ-ತ್ಯ-ವಾದ
ಪರಮಾ
ಪರ-ಮಾತ್ಮ
ಪರ-ಮಾ-ದರ್ಶ
ಪರ-ಮಾ-ವ-ಧಿ-ಯನ್ನು
ಪರಮೇ
ಪರ-ಮೇ-ಶ್ವ-ರನ
ಪರ-ರನ್ನು
ಪರ-ಲೋ-ಕದ
ಪರ-ವ-ಶ-ತೆ-ಯಿಂದ
ಪರ-ವಾಗಿ
ಪರ-ಶು-ರಾಮ
ಪರ-ಸ್ಪರ
ಪರ-ಹಿ-ತಾ-ಕಾಂ-ಕ್ಷಿ-ಗ-ಳಾಗಿ
ಪರಾ
ಪರಾ-ಕಾಷ್ಠೆ
ಪರಾ-ಕಾ-ಷ್ಠೆ-ಯನ್ನು
ಪರಾ-ಕ್ರಮ
ಪರಾ-ಕ್ರ-ಮ-ವುಳ್ಳ
ಪರಿ
ಪರಿ-ಚ-ಯ-ವಿ-ಲ್ಲ-ದ-ವರು
ಪರಿ-ಚ-ಯ-ವಿ-ಲ್ಲದೆ
ಪರಿ-ಣಾಮ
ಪರಿ-ಣಾ-ಮ-ಕಾರಿ
ಪರಿ-ಣಾ-ಮ-ಗಳು
ಪರಿ-ಣಾ-ಮ-ವಾಗಿ
ಪರಿ-ಣಾ-ಮ-ವಾ-ಗಿದೆ
ಪರಿ-ಣಾ-ಮವೂ
ಪರಿ-ತ-ಪಿ-ಸು-ತ್ತಿದೆ
ಪರಿ-ತ-ಪಿ-ಸು-ತ್ತೀ-ರೇನು
ಪರಿ-ಪಾ-ಲಿ-ಸು-ವುದನ್ನು
ಪರಿ-ಪಾ-ಲಿ-ಸು-ವು-ದ-ರಲ್ಲಿ
ಪರಿ-ಪಾ-ಲಿ-ಸು-ವು-ದಿಲ್ಲ
ಪರಿ-ಪೂರ್ಣ
ಪರಿ-ಪೂ-ರ್ಣ-ತೆ-ಯನ್ನು
ಪರಿ-ಮ-ಳ-ವನ್ನು
ಪರಿ-ಯಂ-ತರ
ಪರಿ-ವ-ರ್ತನೆ
ಪರಿ-ವ-ರ್ತಿಸಿ
ಪರಿ-ಶೀ-ಲಿ-ಸಿ-ದರೆ
ಪರಿ-ಶುದ್ಧ
ಪರಿ-ಶು-ದ್ಧತೆ
ಪರಿ-ಶು-ದ್ಧ-ನಾ-ಗಿ-ರ-ಬೇಕು
ಪರಿ-ಶು-ದ್ಧ-ರಾಗಿ
ಪರಿ-ಶು-ದ್ಧರು
ಪರಿ-ಶು-ದ್ಧ-ವಾ-ಗಿ-ರ-ಬೇಕು
ಪರಿ-ಶು-ದ್ಧ-ವಾ-ಗಿ-ರು-ವುದೇ
ಪರಿ-ಶು-ದ್ಧ-ವಾದ
ಪರಿ-ಶು-ದ್ಧ-ವಾ-ದುದು
ಪರಿ-ಶುದ್ಧಿ
ಪರಿ-ಸ್ಥಿತಿ
ಪರಿ-ಹ-ರಿಸ
ಪರಿ-ಹ-ರಿ-ಸಿ-ಕೊಂಡು
ಪರಿ-ಹ-ರಿ-ಸಿ-ಕೊ-ಳ್ಳಲು
ಪರಿ-ಹ-ರಿಸು
ಪರಿ-ಹ-ರಿ-ಸು-ತ್ತಿದೆ
ಪರಿ-ಹ-ರಿ-ಸುವ
ಪರಿ-ಹ-ರಿ-ಸು-ವುದು
ಪರಿ-ಹ-ರಿ-ಸು-ವುದೆ
ಪರಿ-ಹಾರ
ಪರಿ-ಹಾ-ರಕ್ಕೆ
ಪರಿ-ಹಾ-ರದ
ಪರಿ-ಹಾ-ರೋ-ಪಾಯ
ಪರೀ-ಕ್ಷಿ-ಸಿದ
ಪರೀಕ್ಷೆ
ಪರೀ-ಕ್ಷೆ-ಗಳನ್ನು
ಪರೆ
ಪರೆಯ
ಪರೆ-ಯ-ರನ್ನು
ಪರೆ-ಯ-ವ-ನಾ-ದರೂ
ಪರೋ-ಪ-ಕಾ-ರದ
ಪರ್ಯ-ವ-ಸಾ-ನ-ವಾ-ಗ-ಕೂ-ಡದು
ಪರ್ವತ
ಪರ್ವ-ತ-ದಂ-ತಹ
ಪರ್ವ-ತ-ಶಿ-ಖ-ರ-ಗಳಿಂದ
ಪರ್ವ-ತ-ಸ್ತೋಮ
ಪರ್ಶಿಯಾ
ಪರ್ಶಿ-ಯಾ-ದಿಂದ
ಪಲಾ-ಯನ
ಪಲ್ಯ
ಪಲ್ಯ-ವನ್ನು
ಪಲ್ಲವಿ
ಪವಿತ್ರ
ಪವಿ-ತ್ರತೆ
ಪವಿ-ತ್ರ-ತೆ-ಗಿಂತ
ಪವಿ-ತ್ರ-ತೆಯ
ಪವಿ-ತ್ರ-ತೆ-ಯಿಂದ
ಪವಿ-ತ್ರ-ದಂತೆ
ಪವಿ-ತ್ರ-ದೇ-ಶ-ದಲ್ಲಿ
ಪವಿ-ತ್ರ-ಭಾ-ವನೆ
ಪವಿ-ತ್ರ-ಭೂಮಿ
ಪವಿ-ತ್ರಳು
ಪವಿ-ತ್ರ-ವಾಗಿ
ಪವಿ-ತ್ರ-ವಾ-ಗಿ-ರು-ವು-ದಕ್ಕೂ
ಪವಿ-ತ್ರ-ವಾ-ಗಿ-ರು-ವು-ದ-ನ್ನೆಲ್ಲ
ಪವಿ-ತ್ರ-ವಾದ
ಪವಿ-ತ್ರಾ-ತ್ಮರು
ಪವಿ-ತ್ರಾ-ತ್ಮಳು
ಪಶು-ಸಂ-ತಾ-ನ-ರಾ-ಗಿ-ರು-ವ-ರೆಂದು
ಪಾಂಚ-ಜ-ನ್ಯ-ದಿಂದ
ಪಾಂಚ-ಜ-ನ್ಯ-ಸ-ದೃ-ಶ-ವಾದ
ಪಾಂಡಿತ್ಯ
ಪಾಂಡಿ-ತ್ಯಕ್ಕೆ
ಪಾಂಡಿ-ತ್ಯದ
ಪಾಂಡಿ-ತ್ಯವೆ
ಪಾಠ
ಪಾಠವೇ
ಪಾಠ-ಶಾಲೆ
ಪಾಡಾ-ಗಿದೆ
ಪಾತಿ-ವ್ರ-ತ್ಯ-ವೆಂದರೆ
ಪಾತಿ-ವ್ರ-ತ್ಯವೇ
ಪಾತ್ರ-ಳಾ-ಗಿ-ರು-ವಳು
ಪಾತ್ರ-ವಿದೆ
ಪಾತ್ರೆ
ಪಾತ್ರೆ-ಗಳೇ
ಪಾತ್ರೆಗೆ
ಪಾತ್ರೆ-ಯನ್ನು
ಪಾದ-ಧೂಳಿ
ಪಾದ-ರ-ಕ್ಷೆ-ಯನ್ನು
ಪಾದ-ಸ್ಪ-ರ್ಶ-ದಿಂದ
ಪಾದ್ರಿ
ಪಾನ-ಮಾಡಿ
ಪಾನ-ಮಾ-ಡಿದ
ಪಾನ-ಮಾ-ಡು-ವು-ದಕ್ಕೆ
ಪಾಪ
ಪಾಪ-ವಿ-ದ್ದರೆ
ಪಾಪವೂ
ಪಾಪವೇ
ಪಾಪಿ-ಗ-ಳಾ-ದರೆ
ಪಾಪಿ-ಗ-ಳಿಗೆ
ಪಾಪಿ-ಗಳು
ಪಾಯಕ್ಕೆ
ಪಾಯದ
ಪಾರಂ-ಗ-ತ-ರಾದ
ಪಾರಂ-ಪ-ರ್ಯ-ವಾಗಿ
ಪಾರಾ
ಪಾರಾ-ಗ-ಅುನು
ಪಾರಾ-ಗ-ಬೇ-ಕಾ-ಗಿದೆ
ಪಾರಾ-ಗ-ಬೇ-ಕಾ-ದರೆ
ಪಾರಾ-ಗ-ಲಾ-ರರು
ಪಾರಾ-ಗ-ಲಾ-ರಿರಿ
ಪಾರಾ-ಗಲು
ಪಾರಾಗಿ
ಪಾರಾ-ಗುವ
ಪಾರಾ-ಗು-ವು-ದಕ್ಕೆ
ಪಾರಾ-ದರೆ
ಪಾರ್ಥ
ಪಾರ್ಲಿ-ಮೆಂ-ಟಿನ
ಪಾರ್ಲಿ-ಮೆಂಟ್
ಪಾಲನೆ
ಪಾಲಾ-ದ-ವರು
ಪಾಲಿಗೆ
ಪಾಲಿನ
ಪಾಲಿ-ನ-ದನ್ನು
ಪಾಲಿ-ಸ-ಬೇಕು
ಪಾಲಿಸಿ
ಪಾಲಿಸು
ಪಾಲಿ-ಸುವು
ಪಾಲು
ಪಾಳು-ಬಿ-ದ್ದಿದೆ
ಪಾಳೆ-ಯ-ಗಾ-ರ-ರಲ್ಲ
ಪಾವಿ-ತ್ರ್ಯದ
ಪಾಶ-ದಿಂದ
ಪಾಶ್ಚಾ
ಪಾಶ್ಚಾತ್ಯ
ಪಾಶ್ಚಾ-ತ್ಯ-ದಲ್ಲಿ
ಪಾಶ್ಚಾ-ತ್ಯನ
ಪಾಶ್ಚಾ-ತ್ಯ-ನಿಗೆ
ಪಾಶ್ಚಾ-ತ್ಯನು
ಪಾಶ್ಚಾ-ತ್ಯರ
ಪಾಶ್ಚಾ-ತ್ಯ-ರನ್ನು
ಪಾಶ್ಚಾ-ತ್ಯ-ರಾ-ಗ-ಲಾ-ರೆವು
ಪಾಶ್ಚಾ-ತ್ಯ-ರಾಗಿ
ಪಾಶ್ಚಾ-ತ್ಯ-ರಾದ
ಪಾಶ್ಚಾ-ತ್ಯ-ರಿಂದ
ಪಾಶ್ಚಾ-ತ್ಯ-ರಿಗೆ
ಪಾಶ್ಚಾ-ತ್ಯ-ರಿಗೇ
ಪಾಶ್ಚಾ-ತ್ಯರು
ಪಾಶ್ಚಾ-ತ್ಯ-ವನ್ನು
ಪಾಶ್ಚಾ-ತ್ಯ-ವಿ-ಜ್ಞಾನ
ಪಾಷಂ-ಡ-ತನ
ಪಾಸು-ಮಾಡಿ
ಪಿತೃ-ಲೋ-ಕ-ದಿಂದ
ಪಿರಂ-ಗಿಯ
ಪೀಡಿ-ಸಿದ್ದು
ಪೀಡಿ-ಸುವ
ಪೀಡಿ-ಸು-ವಷ್ಟು
ಪೀಡಿ-ಸು-ವುದನ್ನು
ಪುಜ-ತ್ವ-ದವ
ಪುಜುತ್ವ
ಪುಜು-ತ್ವ-ವನ್ನು
ಪುಜು-ತ್ವ-ವಿಲ್ಲ
ಪುಜು-ಮಾ-ರ್ಗ-ವನ್ನು
ಪುಣ
ಪುಣಿ
ಪುಣ್ಯ
ಪುಣ್ಯ-ಭೂಮಿ
ಪುಣ್ಯ-ಭೂ-ಮಿ-ಯನ್ನು
ಪುಣ್ಯ-ಭೂ-ಮಿ-ಯಲ್ಲಿ
ಪುಣ್ಯ-ಭೂ-ಮಿ-ಯೆಂದು
ಪುಣ್ಯಾ-ತ್ಮರು
ಪುನಃ
ಪುನ-ರು-ದ್ಧಾ-ರಕ್ಕೆ
ಪುರಾಣ
ಪುರಾ-ಣ-ಗ-ಳ-ಲ್ಲಿ-ರುವ
ಪುರಾ-ಣ-ಗ-ಳೆಲ್ಲ
ಪುರಾ-ತನ
ಪುರಾ-ತ-ನ-ಕಾ-ಲದ
ಪುರುಷ
ಪುರು-ಷ-ನಂತೆ
ಪುರು-ಷ-ನಿ-ಗಿಂತ
ಪುರು-ಷ-ನಿಗೆ
ಪುರು-ಷನು
ಪುರು-ಷ-ರಿಂದ
ಪುರು-ಷ-ರಿ-ಗಿಂತ
ಪುರು-ಷ-ರಿಗೆ
ಪುರು-ಷರು
ಪುರು-ಷ-ಸಿಂ-ಹ-ರನ್ನು
ಪುರು-ಷ-ಸಿಂ-ಹರು
ಪುರೋ-ಹಿ-ತರು
ಪುಲಿ
ಪುಷಿ
ಪುಷಿ-ಗಳ
ಪುಷಿ-ಗಳನ್ನು
ಪುಷಿ-ಗ-ಳಿ-ಗಿಂತ
ಪುಷಿ-ಗ-ಳಿಗೆ
ಪುಷಿ-ಗಳು
ಪುಷಿ-ಸಂ-ತಾ-ನರು
ಪುಷ್ಯಾ-ಶ್ರ-ಮದ
ಪುಸ್ತಕ
ಪುಸ್ತ-ಕ-ಗಳನ್ನು
ಪುಸ್ತ-ಕ-ಭಂ-ಡಾ-ರ-ವನ್ನೇ
ಪುಸ್ತ-ಕವೂ
ಪುಸ್ತ-ಕಾ-ಲ-ಯ-ಗಳೇ
ಪೂಜಿ-ಸ-ಬಾ-ರದು
ಪೂಜಿಸು
ಪೂಜೆ
ಪೂಜೆ-ಗಾಗಿ
ಪೂಜೆ-ಯಂ-ತಿ-ರಲಿ
ಪೂಜೆ-ಯಿಂದ
ಪೂಜ್ಯಂತೇ
ಪೂಜ್ಯ-ದೃ-ಷ್ಟಿ-ಯಿಂದ
ಪೂರೈ-ಸುವು
ಪೂರ್ಣ
ಪೂರ್ಣ-ವಾದ
ಪೂರ್ವ
ಪೂರ್ವ-ಕಾ-ಲದ
ಪೂರ್ವ-ಜ-ರಿಗೆ
ಪೂರ್ವ-ದಿಂದ
ಪೂರ್ವ-ದಿಂ-ದಲೂ
ಪೂರ್ವ-ಪ-ಶ್ಚಿಮ
ಪೂರ್ವಾ-ಚಾ-ರ-ಪ-ರಾ-ಯ-ಣತೆ
ಪೂರ್ವಿ
ಪೂರ್ವಿ-ಕರ
ಪೂರ್ವಿ-ಕ-ರನ್ನು
ಪೂರ್ವಿ-ಕ-ರಲ್ಲಿ
ಪೂರ್ವಿ-ಕರು
ಪೃಥ್ವಿ-ಗಿಂತ
ಪೃಥ್ವಿ-ಯ-ನ್ನೆಲ್ಲ
ಪೆಟ್ಟಿನ
ಪೆಟ್ಟು
ಪೆಟ್ಟೊಂದೇ
ಪೈಗಳು
ಪೊಳ್ಳು
ಪೋಣಿಸಿ
ಪೋರ್ಚು-ಗಲ್
ಪೋರ್ಚು-ಗೀ-ಸ-ರಾ-ಗಲಿ
ಪೋಲೀ-ಸರು
ಪೋಷ-ಕ-ರಿಂದ
ಪೋಷ-ಕರು
ಪೋಷಣೆ
ಪೌರ-ಪಾ-ಶ್ಚಾ-ತ್ಯ-ಗಳ
ಪೌರಾ-ಣಿಕ
ಪೌರಾ-ಣಿ-ಕರೂ
ಪೌರಾ-ಣಿ-ಕ-ವಾ-ದುದು
ಪೌರಾತ್ಯ
ಪೌರಾ-ತ್ಯ-ದೇ-ಶದ
ಪೌರಾ-ತ್ಯನ
ಪೌರಾ-ತ್ಯನು
ಪೌರುಷ
ಪೌರು-ಷದ
ಪೌರು-ಷ-ದಿಂದ
ಪೌರು-ಷ-ವನ್ನು
ಪೌರೋ-ಹಿ-ತ್ಯರ
ಪ್ಯಾಟ್
ಪ್ರಕಾರ
ಪ್ರಕಾ-ಶ-ಕ-ರಿಗೆ
ಪ್ರಕಾ-ಶ-ಕರು
ಪ್ರಕೃತಿ
ಪ್ರಕೃ-ತಿಗೆ
ಪ್ರಕೃ-ತಿಯ
ಪ್ರಕೃ-ತಿ-ಯನ್ನು
ಪ್ರಕೃ-ತಿಯೇ
ಪ್ರಕ್ಷಿ-ಪ್ತ-ಭಾ-ಗ-ವೆಂದು
ಪ್ರಖ್ಯಾ-ತ-ನಾ-ಗಿ-ದ್ದರೂ
ಪ್ರಖ್ಯಾ-ತ-ನಾದ
ಪ್ರಖ್ಯಾ-ತ-ರಾದ
ಪ್ರಖ್ಯಾ-ತ-ವಾ-ಗಿ-ರ-ಬೇ-ಕಾ-ಗಿತ್ತು
ಪ್ರಖ್ಯಾ-ತ-ವಾದ
ಪ್ರಗತಿ
ಪ್ರಗ-ತಿಗೆ
ಪ್ರಗ-ತಿಯ
ಪ್ರಗ-ತಿ-ಯಲ್ಲಿ
ಪ್ರಗ-ತಿ-ಯೆಲ್ಲ
ಪ್ರಚಾರ
ಪ್ರಚಾ-ರ-ಕರ
ಪ್ರಚಾ-ರ-ಕ-ರನ್ನು
ಪ್ರಚಾ-ರ-ಕ-ರಿಂದ
ಪ್ರಚಾ-ರ-ಕರು
ಪ್ರಚಾ-ರ-ದಿಂದ
ಪ್ರಚಾ-ರ-ಮಾ-ಡ-ಬೇ-ಕಾ-ಗಿದೆ
ಪ್ರಚಾ-ರ-ಮಾ-ಡಲು
ಪ್ರಚಾ-ರ-ಮಾಡಿ
ಪ್ರಚೋ-ದ-ನ-ಕಾ-ರಿ-ಯಾ-ಗ-ಬ-ಲ್ಲದು
ಪ್ರತಿ
ಪ್ರತಿ-ಕ್ರಿ-ಯೆಯ
ಪ್ರತಿ-ಕ್ರಿ-ಯೆ-ಯಾ-ಗ-ದಂತೆ
ಪ್ರತಿ-ದಿ-ನವೂ
ಪ್ರತಿ-ಪಾ-ದಿ-ಸಲು
ಪ್ರತಿ-ಫ-ಲ-ವಾ-ಗಿದೆ
ಪ್ರತಿ-ಭಾ-ವಂ-ತರು
ಪ್ರತಿ-ಯೊಂ-ದನ್ನೂ
ಪ್ರತಿ-ಯೊಂ-ದರ
ಪ್ರತಿ-ಯೊಂದು
ಪ್ರತಿ-ಯೊಂದೂ
ಪ್ರತಿ-ಯೊ-ಬಪ್
ಪ್ರತಿ-ಯೊ-ಬಪ್ನ
ಪ್ರತಿ-ಯೊ-ಬ-ಪ್ನ-ಲ್ಲಿಯೂ
ಪ್ರತಿ-ಯೊ-ಬಪ್ನೂ
ಪ್ರತಿ-ಯೊ-ಬ-ಪ್ರಿಗೂ
ಪ್ರತಿ-ಯೊ-ಬಪ್ರೂ
ಪ್ರತಿ-ಯೊಬ್ಬ
ಪ್ರತಿ-ಯೊ-ಬ್ಬನ
ಪ್ರತಿ-ಯೊ-ಬ್ಬ-ನನ್ನು
ಪ್ರತಿ-ಯೊ-ಬ್ಬನೂ
ಪ್ರತಿ-ಯೊ-ಬ್ಬ-ರ-ಲ್ಲಿಯೂ
ಪ್ರತಿ-ಯೊ-ಬ್ಬ-ರಿಗೂ
ಪ್ರತಿ-ಯೊ-ಬ್ಬರೂ
ಪ್ರತೀ-ಕ-ದಂ-ತಿ-ರುವ
ಪ್ರತ್ಯಕ್ಷ
ಪ್ರತ್ಯ-ಕ್ಷ-ವಾ-ಗಿಯೋ
ಪ್ರತ್ಯೇಕ
ಪ್ರಥಮ
ಪ್ರಥ-ಮ-ದಲ್ಲಿ
ಪ್ರಥ-ಮೋ-ದಯ
ಪ್ರದ-ಕ್ಷಿಣೆ
ಪ್ರಧಾ-ನ-ವಾ-ದುದು
ಪ್ರಪಂಚ
ಪ್ರಪಂ-ಚಕ್ಕೆ
ಪ್ರಪಂ-ಚ-ಕ್ಕೆಲ್ಲ
ಪ್ರಪಂ-ಚ-ಕ್ಕೆಲ್ಲಾ
ಪ್ರಪಂ-ಚ-ಗಳು
ಪ್ರಪಂ-ಚದ
ಪ್ರಪಂ-ಚ-ದ-ದಲ್ಲಿ
ಪ್ರಪಂ-ಚ-ದಲ್ಲಿ
ಪ್ರಪಂ-ಚ-ದ-ಲ್ಲಿ-ರುವ
ಪ್ರಪಂ-ಚ-ದ-ಲ್ಲಿ-ರುವು
ಪ್ರಪಂ-ಚ-ದಲ್ಲೆಲ್ಲ
ಪ್ರಪಂ-ಚ-ದಿಂದ
ಪ್ರಪಂ-ಚ-ವನ್ನು
ಪ್ರಪಂ-ಚ-ವ-ನ್ನೆಲ್ಲ
ಪ್ರಪಂ-ಚ-ವನ್ನೇ
ಪ್ರಪಂ-ಚವೆ
ಪ್ರಪಂ-ಚ-ವೆಲ್ಲ
ಪ್ರಪಂ-ಚವೇ
ಪ್ರಬಲ
ಪ್ರಬ-ಲ-ವಾಗಿ
ಪ್ರಬ-ಲ-ವಾಗು
ಪ್ರಭಾವ
ಪ್ರಭಾ-ವಕ್ಕೆ
ಪ್ರಭಾ-ವ-ಗಳ
ಪ್ರಭಾ-ವ-ದಿಂದ
ಪ್ರಭಾ-ವ-ವನ್ನು
ಪ್ರಮಾ-ಣ-ದ-ಲ್ಲಿದೆ
ಪ್ರಯತ್ನ
ಪ್ರಯ-ತ್ನ-ಗ-ಳಾ-ಗಿವೆ
ಪ್ರಯ-ತ್ನ-ಪ-ಟ್ಟರೂ
ಪ್ರಯ-ತ್ನ-ಪ-ಟ್ಟಿ-ರು-ವರು
ಪ್ರಯ-ತ್ನ-ಪಡು
ಪ್ರಯ-ತ್ನ-ವನ್ನೂ
ಪ್ರಯ-ತ್ನ-ವಿ-ರ-ಬೇಕು
ಪ್ರಯ-ತ್ನ-ವಿ-ಲ್ಲದೇ
ಪ್ರಯ-ತ್ನವೂ
ಪ್ರಯ-ತ್ನ-ವೆಲ್ಲ
ಪ್ರಯ-ತ್ನಿ-ಸ-ಬೇಡಿ
ಪ್ರಯ-ತ್ನಿಸಿ
ಪ್ರಯ-ತ್ನಿ-ಸಿ-ದರೆ
ಪ್ರಯ-ತ್ನಿ-ಸು-ತ್ತಿರು
ಪ್ರಯ-ತ್ನಿ-ಸು-ತ್ತಿ-ರು-ವನು
ಪ್ರಯ-ತ್ನಿ-ಸು-ತ್ತಿ-ರು-ವರು
ಪ್ರಯ-ತ್ನಿ-ಸು-ವನು
ಪ್ರಯ-ತ್ನಿ-ಸು-ವರು
ಪ್ರಯ-ತ್ನಿ-ಸು-ವುದು
ಪ್ರಯಾಣ
ಪ್ರಯೋ
ಪ್ರಯೋಗ
ಪ್ರಯೋ-ಗದ
ಪ್ರಯೋ-ಗ-ಶಾ-ಲೆ-ಯಲ್ಲಿ
ಪ್ರಯೋ-ಗಿಸಿ
ಪ್ರಯೋ-ಜನ
ಪ್ರಯೋ-ಜ-ನ-ವನ್ನು
ಪ್ರಯೋ-ಜ-ನ-ವಾ-ಗು-ವುದು
ಪ್ರಯೋ-ಜ-ನ-ವಿಲ್ಲ
ಪ್ರಯೋ-ಜ-ನ-ವಿ-ಲ್ಲ-ದ-ವ-ನೆಂದು
ಪ್ರಯೋ-ಜ-ನ-ವಿ-ಲ್ಲ-ವೆಂದು
ಪ್ರಯೋ-ಜ-ನವೂ
ಪ್ರಯೋ-ಜ-ನವೊ
ಪ್ರವ-ರ್ಧ-ಮಾ-ನಕ್ಕೆ
ಪ್ರವಾಹ
ಪ್ರವಾ-ಹದ
ಪ್ರವಾ-ಹ-ದಿಂದ
ಪ್ರವಾ-ಹ-ದೊ-ಡ-ನೆಯೆ
ಪ್ರವೃತ್ತಿ
ಪ್ರವೇಶ
ಪ್ರವೇ-ಶಿ-ಸದೇ
ಪ್ರವೇ-ಶಿಸಿ
ಪ್ರವೇ-ಶಿ-ಸಿದೆ
ಪ್ರವೇ-ಶಿ-ಸಿ-ರು-ವಳು
ಪ್ರವೇ-ಶಿ-ಸು-ತ್ತಿ-ರುವ
ಪ್ರಶ್ನಿಸಿ
ಪ್ರಶ್ನಿ-ಸು-ವಳು
ಪ್ರಶ್ನೆ
ಪ್ರಶ್ನೆ-ಗಳ
ಪ್ರಶ್ನೆ-ಗಳಿಂದ
ಪ್ರಶ್ನೆ-ಯನ್ನು
ಪ್ರಸಂಗ
ಪ್ರಸಾರ
ಪ್ರಾಂತ್ಯ
ಪ್ರಾಚೀನ
ಪ್ರಾಚೀ-ನ-ಕಾ-ಲ-ದಿಂ-ದಲೂ
ಪ್ರಾಣ
ಪ್ರಾಣ-ವನ್ನು
ಪ್ರಾಣ-ವನ್ನೇ
ಪ್ರಾಪಂ-ಚಿಕ
ಪ್ರಾಪಂ-ಚಿ-ಕತೆ
ಪ್ರಾಪ್ತ-ವ-ಯ-ಸ್ಸಿಗೆ
ಪ್ರಾಮಾ-ಣಿ-ಕನೆ
ಪ್ರಾಮು-ಖ್ಯ-ತೆ-ಯನ್ನು
ಪ್ರಾಯಳು
ಪ್ರಾರಂಭ
ಪ್ರಾರಂ-ಭ-ದಲ್ಲಿ
ಪ್ರಾರಂ-ಭ-ವನ್ನು
ಪ್ರಾರಂ-ಭ-ವಾಗಿ
ಪ್ರಾರಂ-ಭ-ವಾ-ದಾಗ
ಪ್ರಾರಂ-ಭ-ವಾ-ಯಿತು
ಪ್ರಾರಂ-ಭ-ವಾ-ಯಿತೊ
ಪ್ರಾರಂ-ಭಿ-ಸ-ಬೇಕು
ಪ್ರಾರ್ಥನೆ
ಪ್ರಾರ್ಥಿಸಿ
ಪ್ರಾರ್ಥಿ-ಸೋಣ
ಪ್ರಿಯ
ಪ್ರೀತಿ
ಪ್ರೀತಿಗೆ
ಪ್ರೀತಿ-ಪಾ-ತ್ರರ
ಪ್ರೀತಿ-ಯನ್ನು
ಪ್ರೀತಿ-ಯಿಂದ
ಪ್ರೀತಿ-ಯಿಲ್ಲ
ಪ್ರೀತಿಯು
ಪ್ರೀತಿಯೇ
ಪ್ರೀತಿ-ಸ-ಲಾ-ರೆವು
ಪ್ರೀತಿಸು
ಪ್ರೀತಿ-ಸು-ತ್ತೇನೆ
ಪ್ರೀತಿ-ಸು-ವಿ-ರೇನು
ಪ್ರೀತಿ-ಸು-ವೆಯಾ
ಪ್ರೇಮದ
ಪ್ರೇರ-ಕ-ವಾ-ಗ-ಬೇ-ಕಾ-ಗಿತ್ತು
ಪ್ರೇರೇ-ಪಿ-ತ-ರಾ-ಗಿಲ್ಲ
ಪ್ರೋತ-ವಾ-ಗಿದೆ
ಪ್ರೋತ್ಸಾಹ
ಪ್ರೋತ್ಸಾ-ಹಿ-ಸು-ವಂ-ತಹ
ಪ್ರೋತ್ಸಾ-ಹಿ-ಸು-ವು-ದಿಲ್ಲ
ಫಲ
ಫಲ-ಪ್ರ-ದ-ವಾ-ಗು-ವಂತೆ
ಫಲ-ವ-ತ್ತಾ-ಗಿ-ದ್ದು-ದ-ರಿಂದ
ಫಲ-ವ-ತ್ತಾದ
ಫಲ-ವನ್ನು
ಫಲ-ವ-ನ್ನೆಲ್ಲ
ಫಲವೇ
ಫೀಜನ್ನು
ಫೀಜನ್ನೂ
ಫ್ರಾನ್ಸ್
ಬಂಗಾಳ
ಬಂಡೆಯೂ
ಬಂತು
ಬಂದ
ಬಂದಂತೆ
ಬಂದಂ-ತೆಲ್ಲ
ಬಂದದ್ದು
ಬಂದರು
ಬಂದರೆ
ಬಂದ-ವ-ರಿಗೆ
ಬಂದ-ವರು
ಬಂದ-ವರೆ-ಲ್ಲರೂ
ಬಂದಷ್ಟೂ
ಬಂದಾಗ
ಬಂದಿ
ಬಂದಿದೆ
ಬಂದಿ-ದ್ದರೆ
ಬಂದಿ-ರ-ಬ-ಹುದು
ಬಂದಿ-ರುವ
ಬಂದಿ-ರು-ವರು
ಬಂದಿ-ರು-ವೆವು
ಬಂದಿಲ್ಲ
ಬಂದಿವೆ
ಬಂದು
ಬಂದುವು
ಬಂದೊ-ಡ-ನೆಯೇ
ಬಂಧ-ನ-ಗಳನ್ನು
ಬಂಧ-ನ-ಗಳಿಂದ
ಬಂಧಿ-ಸಿ-ರುವ
ಬಂಧು-ಬಾಂ-ಧ-ವರು
ಬಗೆ
ಬಗೆಗೂ
ಬಗೆಗೆ
ಬಗೆಯ
ಬಗೆ-ಹ-ರಿ-ಯು-ವುದು
ಬಗೆ-ಹ-ರಿಸ
ಬಗೆ-ಹ-ರಿ-ಸ-ಬೇಕು
ಬಗೆ-ಹ-ರಿ-ಸಲು
ಬಗೆ-ಹ-ರಿ-ಸಿ-ಕೊ-ಳ್ಳ-ಬೇಕು
ಬಗೆ-ಹ-ರಿ-ಸಿ-ಕೊ-ಳ್ಳುವ
ಬಗೆ-ಹ-ರಿ-ಸಿ-ಕೊ-ಳ್ಳು-ವರು
ಬಗೆ-ಹ-ರಿ-ಸಿ-ಕೊ-ಳ್ಳು-ವಳು
ಬಗೆ-ಹ-ರಿ-ಸಿ-ದರು
ಬಗೆ-ಹ-ರಿ-ಸು-ವು-ದಕ್ಕೆ
ಬಗೆ-ಹ-ರಿ-ಸು-ವುದು
ಬಗ್ಗ-ರಿ-ಗಾಗಿ
ಬಗ್ಗಿ-ಸ-ಲಾ-ರದು
ಬಚ್ಚಲ
ಬಟ್ಟೆ
ಬಟ್ಟೆ-ಗಳು
ಬಟ್ಟೆ-ಬರೆ
ಬಡ
ಬಡ-ತನ
ಬಡ-ತ-ನ-ದಿಂದ
ಬಡ-ವ-ನಾ-ಗಿ-ರ-ಬ-ಹುದು
ಬಡ-ವರ
ಬಡ-ವ-ರನ್ನು
ಬಡ-ವ-ರಿಗೆ
ಬಡ-ವರು
ಬಡಿ-ದೆ-ಬ್ಬಿಸಿ
ಬಣ್ಣ
ಬಣ್ಣಿ-ಸಿ-ರು-ವನು
ಬಣ್ಣಿ-ಸು-ವುದು
ಬತ್ತದ
ಬದ-ಲಾಗಿ
ಬದ-ಲಾ-ಯಿ-ಸದೆ
ಬದ-ಲಾ-ಯಿ-ಸ-ಬ-ಲ್ಲಿರಾ
ಬದ-ಲಾ-ಯಿ-ಸ-ಬ-ಹುದು
ಬದ-ಲಾ-ಯಿ-ಸ-ಲಾ-ರದು
ಬದ-ಲಾ-ಯಿ-ಸ-ಲಾ-ರಿರಿ
ಬದ-ಲಾ-ಯಿ-ಸಿ-ದರೆ
ಬದ-ಲಾ-ಯಿ-ಸಿದೆ
ಬದ-ಲಾ-ಯಿ-ಸಿ-ರು-ವುದು
ಬದ-ಲಾ-ಯಿಸು
ಬದ-ಲಾ-ಯಿ-ಸುವು
ಬದ-ಲಾ-ಯಿ-ಸು-ವುದು
ಬದ-ಲಾ-ವಣೆ
ಬದ-ಲಾ-ವ-ಣೆ-ಗಳನ್ನು
ಬದ-ಲಾ-ವ-ಣೆ-ಗ-ಳಾ-ಗ-ಬ-ಹುದು
ಬದಲು
ಬದಿ-ಗೊತ್ತಿ
ಬದು-ಕಿದೆ
ಬದು-ಕಿ-ರ-ಬೇ-ಕಾ-ದರೆ
ಬದು-ಕಿ-ರು-ವ-ತ-ನಕ
ಬದು-ಕಿ-ರು-ವರು
ಬದು-ಕಿ-ರು-ವಾಗ
ಬದು-ಕಿ-ರು-ವುದ
ಬದು-ಕಿ-ರು-ವೆವು
ಬದು-ಕು-ತ್ತಾನೆ
ಬದು-ಕು-ವನು
ಬದು-ಕು-ವು-ದಕ್ಕೆ
ಬದ್ಧ-ಕಂ-ಕಣ
ಬದ್ಧ-ಕಂ-ಕ-ಣ-ನಾ-ಗಿ-ರು-ವನು
ಬನ್ನಿ
ಬಯ-ಕೆ-ಗಳನ್ನು
ಬಯ-ಸಿ-ದ್ದರೊ
ಬಯ-ಸು-ವಿರಾ
ಬಯ-ಸು-ವುದು
ಬರ-ಗಾ-ಲ-ವಿಲ್ಲ
ಬರ-ದಂತೆ
ಬರದೇ
ಬರ-ಬ-ರುತ್ತ
ಬರ-ಬ-ಲ್ಲನೊ
ಬರ-ಬ-ಹುದು
ಬರ-ಬೇ-ಕಾ-ಗಿದೆ
ಬರ-ಬೇ-ಕಾದ
ಬರ-ಬೇ-ಕಾ-ದರೆ
ಬರ-ಬೇಕು
ಬರ-ಲಾ-ದರು
ಬರ-ಲಾ-ರದು
ಬರ-ಲಾ-ರರು
ಬರಲಿ
ಬರಲು
ಬರಹ
ಬರಿ
ಬರಿಯ
ಬರೀ
ಬರು
ಬರು-ತ್ತವೆ
ಬರು-ತ್ತಿದೆ
ಬರು-ತ್ತಿ-ರುವ
ಬರುವ
ಬರು-ವಂ-ತಹ
ಬರು-ವನು
ಬರು-ವರು
ಬರು-ವ-ವ-ರೆಗೆ
ಬರು-ವಾಗ
ಬರು-ವಿರಿ
ಬರು-ವುದ
ಬರು-ವು-ದಕ್ಕೆ
ಬರು-ವುದನ್ನು
ಬರು-ವು-ದರ
ಬರು-ವು-ದಿಲ್ಲ
ಬರು-ವುದು
ಬರು-ವುವು
ಬರೆ-ಗಳು
ಬರೆದ
ಬರೆ-ದಿ-ರು-ವೆವು
ಬರೆದು
ಬರೆ-ಯುವ
ಬರೆ-ಯು-ವುದು
ಬರ್ಬರ
ಬಲ
ಬಲ-ಗ-ಡೆ-ಯಿಂದ
ಬಲ-ಗೈ-ಯಿಂದ
ಬಲ-ಗೊ-ಳಿ-ಸ-ಬೇಕು
ಬಲ-ಗೊ-ಳಿಸಿ
ಬಲ-ಗೊ-ಳಿ-ಸಿ-ಕೊ-ಳ್ಳ-ಬೇಕು
ಬಲ-ದಿಂದ
ಬಲ-ವಾಗಿ
ಬಲ-ವಾ-ಗಿ-ದ್ದರೆ
ಬಲ-ವಾ-ಗಿ-ರಲಿ
ಬಲ-ವಾ-ಗು-ವುದು
ಬಲ-ವಾದ
ಬಲವೆ
ಬಲವೇ
ಬಲ-ಶಾ-ಲಿ-ಗ-ಳಾ-ಗ-ಬೇಕು
ಬಲ-ಶಾ-ಲಿ-ಗ-ಳಾಗಿ
ಬಲ-ಶಾ-ಲಿಯೆ
ಬಲ-ಶಾ-ಲಿಯೇ
ಬಲಾ
ಬಲಾಢ್ಯ
ಬಲಾ-ಢ್ಯ-ರ-ನ್ನಾಗಿ
ಬಲಾ-ಢ್ಯ-ರಾಗಿ
ಬಲಾ-ತ್ಕಾ-ರ-ದಿಂದ
ಬಲಾ-ತ್ಕಾ-ರ-ದಿಂ-ದಲೋ
ಬಲಾ-ತ್ಕಾ-ರ-ವಾಗಿ
ಬಲಿ-ಕೊ-ಡ-ಬಾ-ರದು
ಬಲಿ-ಕೊ-ಡಲು
ಬಲಿಷ್ಠ
ಬಲಿ-ಷ್ಠ-ವಾ-ಗುತ್ತಾ
ಬಲಿ-ಷ್ಠ-ವಾದ
ಬಲೆ-ಯನ್ನು
ಬಲ್ಲಂ-ತಹ
ಬಲ್ಲದು
ಬಲ್ಲವು
ಬಲ್ಲಿರಾ
ಬಳ-ಕೆಗೆ
ಬಳ-ಸು-ವುದು
ಬಳಿಗೆ
ಬಳಿ-ದು-ಕೊಳ್ಳು
ಬಹಳ
ಬಹಿ-ರ್ಮುಖ
ಬಹಿ-ಷ್ಕಾ-ರ-ವನ್ನು
ಬಹು
ಬಹು-ಕಾಲ
ಬಹು-ಕಾ-ಲದ
ಬಹು-ಜ-ನರ
ಬಹು-ದಿನ
ಬಹುದು
ಬಹುದೊ
ಬಹು-ಪಾಲು
ಬಹುಶಃ
ಬಾ
ಬಾಗಿ
ಬಾಗಿ-ಲಿಗೂ
ಬಾಗಿ-ಲಿಗೆ
ಬಾಗಿ-ಲು-ಗಳನ್ನು
ಬಾಗಿಲೇ
ಬಾಗು-ವೆವು
ಬಾಗ್
ಬಾಚಿ
ಬಾಣ-ಗಳ
ಬಾತಿನ
ಬಾಧ್ಯ-ತೆ-ಗಳನ್ನು
ಬಾಧ್ಯ-ತೆ-ಗಳು
ಬಾಯನ್ನು
ಬಾಯಿಗೆ
ಬಾಯಿ-ಮಾ-ತಿ-ನಲ್ಲಿ
ಬಾಯಿಯ
ಬಾರದ
ಬಾರ-ದ-ವರು
ಬಾರ-ದ-ವ-ರೆಂ-ಬು-ದನ್ನು
ಬಾರ-ದವು
ಬಾರ-ದುದು
ಬಾರ-ದು-ದೆಂದು
ಬಾರದ್ದು
ಬಾರಿ
ಬಾಲಕ
ಬಾಲ-ಕರ
ಬಾಲ-ಕಿ-ಯರ
ಬಾಲ-ಕಿ-ಯ-ರಿಗೆ
ಬಾಲಿ-ಕೆ-ಯರ
ಬಾಲ್ಯ
ಬಾಲ್ಯ-ದಿಂ-ದಲೂ
ಬಾಲ್ಯ-ವಿ-ವಾಹ
ಬಾಲ್ಯಾ-ರಭ್ಯ
ಬಾಳ-ತಕ್ಕ
ಬಾಳನ್ನು
ಬಾಳ-ಬ-ಲ್ಲದು
ಬಾಳ-ಬೇಕು
ಬಾಳ-ಬೇ-ಕೆಂದು
ಬಾಳ-ಲಾ-ರದು
ಬಾಳಲು
ಬಾಳಿ
ಬಾಳಿ-ದರು
ಬಾಳಿ-ದ-ವ-ರ-ಲ್ಲಿ-ರುವ
ಬಾಳಿದೆ
ಬಾಳಿನ
ಬಾಳು
ಬಾಳು-ವಂತೆ
ಬಾಳು-ವನು
ಬಾಳು-ವರೊ
ಬಾಳು-ವು-ದಕ್ಕೆ
ಬಾಳು-ವೆಗೆ
ಬಾಳು-ವೆಯ
ಬಾಳು-ವೆ-ಯನ್ನು
ಬಾಳ್ವೆ
ಬಾವಿ-ಯನ್ನು
ಬಾಸ್
ಬಾಹುಳ್ಯ
ಬಾಹ್ಯ
ಬಾಹ್ಯ-ಜ-ಗತ್ತು
ಬಾಹ್ಯ-ಪ್ರ-ಕೃತಿ
ಬಾಹ್ಯ-ಪ್ರ-ಕೃ-ತಿಯ
ಬಾಹ್ಯ-ಪ್ರ-ಕೃ-ತಿ-ಯೇನೊ
ಬಾಹ್ಯ-ರೂ-ಪ-ವಷ್ಟೆ
ಬಾಹ್ಯ-ಶ-ಕ್ತಿಯ
ಬಿಟ್ಟ
ಬಿಟ್ಟರೆ
ಬಿಟ್ಟಿರಿ
ಬಿಟ್ಟು
ಬಿಡದೆ
ಬಿಡ-ಬೇ-ಕಾ-ಗಿದೆ
ಬಿಡ-ಬೇಕು
ಬಿಡ-ಬೇಡಿ
ಬಿಡಲು
ಬಿಡಿ
ಬಿಡಿಸಿ
ಬಿಡಿ-ಸಿ-ದರು
ಬಿಡು
ಬಿಡು-ಗಡೆ
ಬಿಡು-ತ್ತಿದೆ
ಬಿಡು-ವಂತೆ
ಬಿಡು-ವರು
ಬಿಡು-ವುದು
ಬಿತ್ತು
ಬಿದ್ದರು
ಬಿದ್ದ-ರೆಂದು
ಬಿದ್ದಿದ್ದ
ಬಿದ್ದು
ಬಿನ್ನ-ವಿ-ಸಿ-ಕೊಳ್ಳು
ಬಿರು-ನು-ಡಿಯೂ
ಬಿಲ-ಗಳಿಂದ
ಬಿಲ-ದೊ-ಳಗೆ
ಬಿಸಿ-ರಕ್ತ
ಬಿಸುಟ
ಬೀಜ-ವನ್ನು
ಬೀರ-ಬ-ಹುದು
ಬೀರ-ಲಾ-ರಿರಿ
ಬೀರಿ-ದರು
ಬೀರು-ವನು
ಬೀರು-ವು-ದಕ್ಕೆ
ಬೀಳದೆ
ಬೀಳುವ
ಬೀಳು-ವನು
ಬೀಸಿ
ಬೀಸಿ-ದಂ-ತೆಲ್ಲ
ಬೀಸಿ-ಬ-ರುವ
ಬುದ್ಧ-ನಂತೆ
ಬುದ್ಧ-ನಿಂದ
ಬುದ್ಧಿ
ಬುದ್ಧಿಯ
ಬುದ್ಧಿ-ಯಿಲ್ಲ
ಬುದ್ಧಿಯು
ಬುದ್ಧಿ-ವಂತ
ಬುದ್ಧಿ-ವಂ-ತ-ರಾದ
ಬುದ್ಧಿ-ವಂ-ತರು
ಬುದ್ಧಿ-ವಂ-ತಿ-ಕೆ-ಯಿಂದ
ಬುದ್ಧಿ-ವಾ-ದ-ವನ್ನು
ಬುದ್ಧಿ-ವಿ-ಕಾ-ಸ-ವಾ-ಗು-ವುದೊ
ಬೂಡ್ಸು-ಗ-ಳಿಗೆ
ಬೂದಿ
ಬೃಹತ್
ಬೃಹ-ತ್ತಾಗಿ
ಬೃಹ-ದಾ-ಕಾ-ರದ
ಬೆಂಕಿ
ಬೆಂಕಿ-ಯನ್ನು
ಬೆಟ್ಟ
ಬೆಟ್ಟ-ಗು-ಡ್ಡ-ಗಳು
ಬೆಟ್ಟದ
ಬೆತ್ತ-ಲೆಯ
ಬೆನ್ನು
ಬೆನ್ನೆ-ಲುಬು
ಬೆರೆ-ಯ-ದಿ-ರಲಿ
ಬೆರೆಯು
ಬೆರೆ-ಯು-ವುದನ್ನು
ಬೆಲೆ
ಬೆಲೆ-ಯನ್ನು
ಬೆಲೆ-ಯಿಲ್ಲ
ಬೆಳ-ಕನ್ನು
ಬೆಳ-ಕಾ-ಗ-ಬ-ಲ್ಲಂ-ತಹ
ಬೆಳ-ಕಿಗೆ
ಬೆಳ-ಕಿನ
ಬೆಳಕೆ
ಬೆಳ-ಗಿ-ರು-ವರು
ಬೆಳ-ಗು-ತ್ತಿ-ದೆಯೊ
ಬೆಳ-ವ-ಣಿಗೆ
ಬೆಳ-ವ-ಣಿ-ಗೆಗೆ
ಬೆಳ-ವ-ಣಿ-ಗೆಯ
ಬೆಳ-ವ-ಣಿ-ಗೆ-ಯಲ್ಲಿ
ಬೆಳೆ-ದಿ-ರು-ವೆವು
ಬೆಳೆ-ದು-ನಿಂತ
ಬೆಳೆ-ಯ-ಬೇಕು
ಬೆಳೆ-ಯ-ಲಿಲ್ಲ
ಬೆಳೆ-ಯಿತು
ಬೆಳೆ-ಯುವ
ಬೆಳೆ-ಯು-ವಂತೆ
ಬೆಳೆ-ಸಿ-ಕೊ-ಳ್ಳ-ಬೇಕು
ಬೆಳೆ-ಸು-ವು-ದ-ಕ್ಕಿಂತ
ಬೆಸ್ತ
ಬೆಸ್ತ-ನಿಗೆ
ಬೇಕಾ
ಬೇಕಾ-ಗಿದೆ
ಬೇಕಾ-ಗಿ-ದ್ದರೆ
ಬೇಕಾ-ಗಿ-ದ್ದಾರೆ
ಬೇಕಾ-ಗಿರು
ಬೇಕಾ-ಗಿ-ರು-ವರು
ಬೇಕಾ-ಗಿ-ರು-ವುದು
ಬೇಕಾ-ಗಿ-ರು-ವುದೇ
ಬೇಕಾ-ಗುವ
ಬೇಕಾ-ಗು-ವುದು
ಬೇಕಾದ
ಬೇಕಾ-ದರೂ
ಬೇಕಾ-ದರೆ
ಬೇಕಾ-ದಷ್ಟು
ಬೇಕಾ-ದಾಗ
ಬೇಕಾ-ದು-ದ-ನ್ನೆಲ್ಲ
ಬೇಕಾ-ದು-ದೆಲ್ಲ
ಬೇಕಿಲ್ಲ
ಬೇಕು
ಬೇಕೆಂದು
ಬೇಕೇ
ಬೇಕೋ
ಬೇಗ
ಬೇಡ
ಬೇಡಿ
ಬೇಡಿಕೆ
ಬೇರಲ್ಲ
ಬೇರಾ-ದರೆ
ಬೇರು
ಬೇರು-ಗಳನ್ನು
ಬೇರು-ಗಳು
ಬೇರೂ-ರಿದೆ
ಬೇರೆ
ಬೇರೆ-ದೇ-ಶ-ವನ್ನೂ
ಬೇರೆ-ಯಲ್ಲ
ಬೇರೆ-ಯ-ವ-ರಿಗೆ
ಬೇರೆಲ್ಲೂ
ಬೇರೇ-ನನ್ನು
ಬೇರೊಂ
ಬೇರೊಂದು
ಬೇಲಿ-ಯನ್ನು
ಬೈಯ್ಯ-ಬೇಡಿ
ಬೋಧಕ
ಬೋಧನೆ
ಬೋಧ-ನೆಯ
ಬೋಧ-ನೆ-ಯನ್ನು
ಬೋಧ-ನೆ-ಯನ್ನೂ
ಬೋಧ-ನೆ-ಯಿಂದ
ಬೋಧಿಸ
ಬೋಧಿ-ಸ-ಬಲ್ಲ
ಬೋಧಿ-ಸ-ಬ-ಹುದು
ಬೋಧಿ-ಸ-ಬೇ-ಕಾ-ಗಿದೆ
ಬೋಧಿ-ಸ-ಬೇ-ಕಾ-ದರೂ
ಬೋಧಿ-ಸ-ಬೇಕು
ಬೋಧಿ-ಸಲು
ಬೋಧಿಸಿ
ಬೋಧಿ-ಸಿ-ದರೆ
ಬೋಧಿ-ಸಿ-ದ-ವನು
ಬೋಧಿ-ಸಿದೆ
ಬೋಧಿ-ಸಿದ್ದು
ಬೋಧಿ-ಸು-ತ್ತಿದ್ದ
ಬೋಧಿ-ಸು-ತ್ತಿ-ದ್ದರೂ
ಬೋಧಿ-ಸುವ
ಬೋಧಿ-ಸು-ವಂ-ತಹ
ಬೋಧಿ-ಸು-ವು-ದ-ರಲ್ಲಿ
ಬೋಧಿ-ಸು-ವುದು
ಬೋಧಿ-ಸು-ವುದೇ
ಬೌದ್ಧ
ಬೌದ್ಧ-ಧ-ರ್ಮದ
ಬೌದ್ಧ-ರಾ-ದರೆ
ಬೌದ್ಧ-ರೊ-ಡನೆ
ಬ್ಬರು
ಬ್ಯಾಬಿ-ಲೋ-ನಿಯಾ
ಬ್ರಹ್ಮ
ಬ್ರಹ್ಮ-ಚರ್ಯ
ಬ್ರಹ್ಮ-ಚ-ರ್ಯದ
ಬ್ರಹ್ಮ-ಚ-ರ್ಯ-ದೀ-ಕ್ಷಿ-ತ-ನಾ-ದ-ವ-ನಿಗೆ
ಬ್ರಹ್ಮ-ಚ-ರ್ಯ-ವನ್ನು
ಬ್ರಹ್ಮ-ಚ-ರ್ಯ-ವ್ರ-ತ-ವನ್ನು
ಬ್ರಹ್ಮ-ಚಾ-ರಿ-ಗ-ಳಾಗಿ
ಬ್ರಹ್ಮ-ಚಾ-ರಿ-ಣಿ-ಯ-ರಾ-ಗ-ಬೇಕು
ಬ್ರಹ್ಮ-ಚಾ-ರಿ-ಣಿ-ಯರು
ಬ್ರಹ್ಮ-ಚಾರ್ಯ
ಬ್ರಹ್ಮ-ಜ್ಞಾನ
ಬ್ರಹ್ಮ-ಜ್ಞಾ-ನಕ್ಕೆ
ಬ್ರಹ್ಮ-ಜ್ಞಾ-ನ-ವನ್ನು
ಬ್ರಹ್ಮ-ಜ್ಞಾ-ನಿ-ಯಾ-ಗು-ವನು
ಬ್ರಹ್ಮ-ತೇ-ಜಸ್ಸು
ಬ್ರಹ್ಮನ
ಬ್ರಹ್ಮ-ನಿ-ರು-ವನು
ಬ್ರಹ್ಮನು
ಬ್ರಹ್ಮ-ಮಯ
ಬ್ರಹ್ಮ-ವಿ-ದ್ಯೆ-ಯನ್ನು
ಬ್ರಹ್ಮ-ಸಾ-ಕ್ಷಾ-ತ್ಕಾರ
ಬ್ರಹ್ಮ-ಸಾ-ಕ್ಷಾ-ತ್ಕಾ-ರ-ವಾ-ಗು-ವುದು
ಬ್ರಾಹ್ಮ
ಬ್ರಾಹ್ಮಣ
ಬ್ರಾಹ್ಮ-ಣನ
ಬ್ರಾಹ್ಮ-ಣ-ನ-ನ್ನಾಗಿ
ಬ್ರಾಹ್ಮ-ಣ-ನಲ್ಲಿ
ಬ್ರಾಹ್ಮ-ಣ-ನ-ಲ್ಲಿ-ರುವ
ಬ್ರಾಹ್ಮ-ಣ-ನಾ-ಗಲು
ಬ್ರಾಹ್ಮ-ಣ-ನಾ-ಗು-ವು-ದಕ್ಕೆ
ಬ್ರಾಹ್ಮ-ಣ-ನಾದ
ಬ್ರಾಹ್ಮ-ಣ-ನಿ-ಗಂತೂ
ಬ್ರಾಹ್ಮ-ಣ-ನಿಗೆ
ಬ್ರಾಹ್ಮ-ಣನೇ
ಬ್ರಾಹ್ಮ-ಣರ
ಬ್ರಾಹ್ಮ-ಣ-ರನ್ನು
ಬ್ರಾಹ್ಮ-ಣ-ರಾ-ಗಿ-ದ್ದರು
ಬ್ರಾಹ್ಮ-ಣ-ರಾಗು
ಬ್ರಾಹ್ಮ-ಣ-ರಾ-ಗು-ವ-ವ-ರೆಗೆ
ಬ್ರಾಹ್ಮ-ಣ-ರಾ-ಗು-ವುದೇ
ಬ್ರಾಹ್ಮ-ಣ-ರಿಗೆ
ಬ್ರಾಹ್ಮ-ಣರು
ಬ್ರಾಹ್ಮ-ಣರೆ
ಬ್ರಾಹ್ಮ-ಣ-ರೆಂದು
ಬ್ರಾಹ್ಮ-ಣ-ರೊಂ-ದಿಗೆ
ಬ್ರಾಹ್ಮ-ಣ-ವರ್ಣ
ಬ್ರಾಹ್ಮ-ಣೇ-ತ-ರ-ರಿಗೆ
ಬ್ರಾಹ್ಮ-ಣ್ಯದ
ಬ್ರಾಹ್ಮ-ಣ್ಯ-ವನ್ನು
ಬ್ರಾಹ್ಮ-ಣ್ಯ-ವೆಂದರೆ
ಬ್ರಾಹ್ಮ-ಣ್ಯವೇ
ಭಂಗ
ಭಂಡಾ-ರಕ್ಕೆ
ಭಕ್ತ-ನಂತೆ
ಭಕ್ತಿ
ಭಕ್ತಿ-ಗ-ಳಿ-ಲ್ಲದೇ
ಭಕ್ತಿ-ಗಳು
ಭಕ್ತಿ-ಜೀ-ವ-ನಕ್ಕೆ
ಭಕ್ತಿ-ಯಿಂದ
ಭಗ-ವಂತ
ಭಗ-ವಂ-ತನ
ಭಗ-ವಂ-ತ-ನಲ್ಲಿ
ಭಗ-ವಂ-ತ-ನಲ್ಲೆ
ಭಗ-ವಂ-ತ-ನಿಗೆ
ಭಗ-ವಂ-ತ-ನೆ-ಡೆಗೆ
ಭಗ-ವಂ-ತನೇ
ಭಗ-ವ-ದ್ಗೀ-ತೆ-ಯನ್ನು
ಭದ್ರ-ವಾ-ಗಿವೆ
ಭದ್ರ-ವಾದ
ಭಯಾ-ನ-ಕ-ವಾ-ಗು-ವುದು
ಭಯಾ-ನ-ಕ-ವಾ-ದುದು
ಭರತ
ಭರ-ತ-ಖಂಡ
ಭರ-ತ-ಖಂ-ಡಕ್ಕೂ
ಭರ-ತ-ಖಂ-ಡಕ್ಕೆ
ಭರ-ತ-ಖಂ-ಡದ
ಭರ-ತ-ಖಂ-ಡ-ದಲ್ಲಿ
ಭರ-ತ-ಖಂ-ಡ-ದ-ಲ್ಲಿದ್ದ
ಭರ-ತ-ಖಂ-ಡ-ದ-ಲ್ಲಿ-ದ್ದಂತೆ
ಭರ-ತ-ಖಂ-ಡ-ದ-ಲ್ಲಿ-ರುವ
ಭರ-ತ-ಖಂ-ಡ-ದಲ್ಲೆಲ್ಲ
ಭರ-ತ-ಖಂ-ಡ-ದಿಂದ
ಭರ-ತ-ಖಂ-ಡ-ವನ್ನು
ಭರ-ತ-ಖಂ-ಡ-ವನ್ನೇ
ಭರ-ತ-ಖಂ-ಡ-ವಾ-ದರೊ
ಭರ-ತ-ಖಂ-ಡವೇ
ಭರ-ತ-ವ-ರ್ಷ-ದ-ಮೇಲೆ
ಭರ-ವಸೆ
ಭವ-ಜೀ-ವಿ-ಗ-ಳಿಗೆ
ಭವ-ಬಂ-ಧ-ನ-ದಿಂದ
ಭವಿಷ್ಯ
ಭವಿ-ಷ್ಯದ
ಭವಿ-ಷ್ಯ-ದಲ್ಲಿ
ಭವಿ-ಷ್ಯ-ಭಾ-ರತ
ಭವಿ-ಸಿ-ದರೂ
ಭವ್ಯ
ಭವ್ಯತೆ
ಭವ್ಯ-ತೆ-ಯಲ್ಲಿ
ಭವ್ಯ-ವಾದ
ಭಾಗ್ಯ-ದೇ-ವ-ತೆ-ಯಾಗಿ
ಭಾರ
ಭಾರತ
ಭಾರ-ತಕ್ಕೆ
ಭಾರ-ತದ
ಭಾರ-ತ-ದಲ್ಲಿ
ಭಾರ-ತ-ದೇಶ
ಭಾರ-ತ-ದೇ-ಶದ
ಭಾರ-ತ-ದೇ-ಶ-ದಲ್ಲಿ
ಭಾರ-ತ-ಭೂಮಿ
ಭಾರ-ತ-ಮಾತೆ
ಭಾರ-ತವು
ಭಾರ-ತಾ-ವ-ನಿ-ಯ-ನ್ನೆಲ್ಲ
ಭಾರ-ತೀಯ
ಭಾರ-ತೀ-ಯನ
ಭಾರ-ತೀ-ಯ-ನಿಗೆ
ಭಾರ-ತೀ-ಯರ
ಭಾರ-ತೀ-ಯ-ರನ್ನು
ಭಾರ-ತೀ-ಯ-ರ-ನ್ನೆಲ್ಲ
ಭಾರ-ತೀ-ಯ-ರಿಗೆ
ಭಾರ-ತೀ-ಯರು
ಭಾರ-ತೀ-ಯ-ರೊ-ಬ್ಬರೇ
ಭಾರ-ತೀ-ಯಳು
ಭಾವ
ಭಾವನಾ
ಭಾವ-ನಾ-ಪ-ರಂ-ಪರೆ
ಭಾವ-ನಾ-ಮಯ
ಭಾವನೆ
ಭಾವ-ನೆ-ಗಳ
ಭಾವ-ನೆ-ಗಳನ್ನು
ಭಾವ-ನೆ-ಗಳನ್ನೂ
ಭಾವ-ನೆ-ಗಳನ್ನೆಲ್ಲ
ಭಾವ-ನೆ-ಗ-ಳಲ್ಲ
ಭಾವ-ನೆ-ಗಳಲ್ಲಿ
ಭಾವ-ನೆ-ಗಳಿಂದ
ಭಾವ-ನೆ-ಗ-ಳಿ-ದ್ದರೆ
ಭಾವ-ನೆ-ಗ-ಳಿಲ್ಲ
ಭಾವ-ನೆ-ಗಳು
ಭಾವ-ನೆ-ಗ-ಳೆಲ್ಲ
ಭಾವ-ನೆಗೆ
ಭಾವ-ನೆಯ
ಭಾವ-ನೆ-ಯನ್ನು
ಭಾವ-ನೆ-ಯನ್ನೂ
ಭಾವ-ನೆ-ಯಿಂದ
ಭಾವ-ನೆ-ಯೆಲ್ಲ
ಭಾವ-ನೆಯೇ
ಭಾವಿ
ಭಾವಿ-ಸ-ಬಾ-ರದು
ಭಾವಿ-ಸ-ಬೇಕು
ಭಾವಿ-ಸ-ಬೇಡಿ
ಭಾವಿಸಿ
ಭಾವಿ-ಸಿ-ದನು
ಭಾವಿ-ಸಿ-ದರೆ
ಭಾವಿ-ಸಿ-ದೆವು
ಭಾವಿ-ಸಿ-ದ್ದರು
ಭಾವಿ-ಸಿದ್ದೇ
ಭಾವಿಸು
ಭಾವಿ-ಸು-ತ್ತಾರೆ
ಭಾವಿ-ಸು-ತ್ತಾ-ರೆಯೊ
ಭಾವಿ-ಸು-ತ್ತಿದ್ದ
ಭಾವಿ-ಸು-ತ್ತಿ-ದ್ದರೆ
ಭಾವಿ-ಸು-ತ್ತೀರಿ
ಭಾವಿ-ಸು-ತ್ತೀ-ರೇನು
ಭಾವಿ-ಸು-ತ್ತೇನೆ
ಭಾವಿ-ಸುವ
ಭಾವಿ-ಸು-ವನು
ಭಾವಿ-ಸು-ವನೊ
ಭಾವಿ-ಸು-ವರು
ಭಾವಿ-ಸು-ವರೊ
ಭಾವಿ-ಸು-ವು-ದ-ರಲ್ಲಿ
ಭಾವಿ-ಸು-ವು-ದಿಲ್ಲ
ಭಾವಿ-ಸು-ವುದು
ಭಾವಿ-ಸು-ವೆವು
ಭಾಷಾ
ಭಾಷೆ
ಭಾಷೆ-ಯಲ್ಲಿ
ಭಾಷೆ-ಯಿಂ-ದಲೇ
ಭಾಷೆಯೇ
ಭಾಷ್ಯದ
ಭಿಕ್ಷು-ಕರ
ಭಿಕ್ಷು-ಕ-ರಾಗಿ
ಭಿನ್ನಾ-ಭಿ-ಪ್ರಾಯ
ಭಿನ್ನಾ-ಭಿ-ಪ್ರಾ-ಯ-ಗಳನ್ನು
ಭೀಕರ
ಭೀಮ
ಭೀಮ-ವೃಕ್ಷ
ಭುಜ-ತ-ಟ್ಟಿ-ದರು
ಭುಜ-ಬಲ
ಭೂತಿಯ
ಭೂತಿಯೂ
ಭೂಪಟ
ಭೂಮಿ
ಭೂಮಿಗೆ
ಭೂಮಿಯ
ಭೂಮಿ-ಯನ್ನು
ಭೂಷಣ
ಭೇದ-ಭಾ-ವನೆ
ಭೇದಿಸಿ
ಭೇದಿ-ಸು-ವಂತೆ
ಭೇದಿ-ಸು-ವನು
ಭೈರ-ವನ
ಭೋಗ
ಭೋಗ-ಗಳನ್ನು
ಭೋಗದ
ಭೌತಿಕ
ಭೌತಿ-ಕ-ವಾದ
ಭ್ಯಾಸ-ವನ್ನು
ಭ್ರಮಿಸಿ
ಭ್ರಾತೃ-ಗ-ಳಿರಾ
ಮಂಗ-ಳ-ವಾ-ಗ-ಬ-ಹುದು
ಮಂಗ-ಳಾ-ರ-ತಿ-ಯನ್ನು
ಮಂಡಿ-ಸಿ-ರು-ವುದನ್ನು
ಮಂತ್ರ
ಮಂತ್ರ-ದಿಂದ
ಮಂತ್ರ-ವಾ-ಗಿದೆ
ಮಂತ್ರೋ-ಪ-ದೇ-ಶ-ವನ್ನು
ಮಂದ-ವಾ-ಗುತ್ತ
ಮಂದ-ಹಾಸ
ಮಂದಿ
ಮಂದೆ-ಯನ್ನು
ಮಕ್ಕ
ಮಕ್ಕಳ
ಮಕ್ಕ-ಳನ್ನು
ಮಕ್ಕ-ಳಾ-ದು-ದ-ರಿಂದ
ಮಕ್ಕ-ಳಿಗೆ
ಮಕ್ಕಳು
ಮಗ
ಮಗು
ಮಗು-ವನ್ನು
ಮಗು-ವಾ-ದರೆ
ಮಗು-ವಿಗೂ
ಮಗು-ವಿಗೆ
ಮಗು-ವಿನ
ಮಗು-ವಿ-ನಂತೆ
ಮಜಾ
ಮಟ್ಟದ
ಮಟ್ಟ-ವನ್ನು
ಮಟ್ಟಿಗೆ
ಮಠಕ್ಕೆ
ಮಠದ
ಮಠ-ದಲ್ಲಿ
ಮಠ-ವನ್ನು
ಮಡಿ
ಮಣ್ಣನ್ನು
ಮಣ್ಣು
ಮತ-ಗ-ಳ-ಲ್ಲಿಯೂ
ಮತದ
ಮತ-ಭ್ರಾಂ-ತ-ನ-ಲ್ಲಿ-ರುವ
ಮತ-ಭ್ರಾಂತಿ
ಮತೀ-ಯ-ರಿಗೆ
ಮತ್ತಾರೂ
ಮತ್ತಾವ
ಮತ್ತಾ-ವುದಾ
ಮತ್ತಾ-ವು-ದಾ-ದರೂ
ಮತ್ತಾ-ವುದೇ
ಮತ್ತಾ-ವುದೋ
ಮತ್ತು
ಮತ್ತೂ
ಮತ್ತೆ
ಮತ್ತೊಂ-ದಕ್ಕೆ
ಮತ್ತೊಂದು
ಮತ್ತೊ-ಬ-ಪ್ನನ್ನು
ಮತ್ತೊ-ಬ-ಪ್ನಿಗೂ
ಮತ್ತೊ-ಬ-ಪ್ರನ್ನು
ಮತ್ತೊ-ಬ-ಪ್ರಿಗೆ
ಮತ್ತೊಬ್ಬ
ಮತ್ತೊ-ಬ್ಬನ
ಮತ್ತೊ-ಬ್ಬ-ನಿಗೆ
ಮತ್ತೊ-ಬ್ಬ-ರಂ-ತೆಯೇ
ಮತ್ತೊ-ಬ್ಬ-ರನ್ನು
ಮತ್ತೊ-ಬ್ಬ-ರಿಗೆ
ಮತ್ತೊ-ಬ್ಬ-ರಿಲ್ಲ
ಮತ್ತೊ-ಬ್ಬ-ರೊ-ಡನೆ
ಮತ್ತೊಮ್ಮೆ
ಮದುವೆ
ಮದು-ವೆ-ಯನ್ನು
ಮದ್ದನ್ನೂ
ಮದ್ರಾ-ಸಿ-ನಲ್ಲಿ
ಮಧ್ಯ
ಮಧ್ಯ-ದಲ್ಲಿ
ಮಧ್ಯ-ವ-ರ್ತಿ-ಗಳು
ಮನ
ಮನ-ಗಂಡ
ಮನ-ಗಂ-ಡರು
ಮನ-ಗಂಡು
ಮನ-ಗಾ-ಣು-ತ್ತಿ-ರು-ವರು
ಮನಸಾ
ಮನ-ಸ್ಸನ್ನು
ಮನ-ಸ್ಸ-ನ್ನೆಲ್ಲ
ಮನ-ಸ್ಸಾ-ಯಿತು
ಮನಸ್ಸಿ
ಮನ-ಸ್ಸಿಗೆ
ಮನ-ಸ್ಸಿನ
ಮನ-ಸ್ಸಿ-ನಲ್ಲಿ
ಮನ-ಸ್ಸಿ-ನ-ಲ್ಲಿದೆ
ಮನ-ಸ್ಸಿ-ನ-ಲ್ಲಿ-ದೆಯೊ
ಮನ-ಸ್ಸಿ-ನ-ಲ್ಲಿ-ರುವ
ಮನ-ಸ್ಸಿ-ನ-ವ-ರಾಗಿ
ಮನ-ಸ್ಸಿ-ರ-ಬೇಕು
ಮನಸ್ಸು
ಮನು
ಮನು-ವಿನ
ಮನುಷ್ಯ
ಮನು-ಷ್ಯನ
ಮನು-ಷ್ಯ-ನನ್ನು
ಮನು-ಷ್ಯ-ನಿಂದ
ಮನು-ಷ್ಯ-ನಿ-ಗಾ-ದರೊ
ಮನು-ಷ್ಯ-ನಿಗೂ
ಮನು-ಷ್ಯನು
ಮನು-ಷ್ಯನೇ
ಮನು-ಷ್ಯ-ರನ್ನು
ಮನು-ಷ್ಯ-ರಾಗಿ
ಮನು-ಷ್ಯ-ರಾ-ಗಿ-ದ್ದರು
ಮನು-ಷ್ಯರು
ಮನು-ಷ್ಯರೆ
ಮನು-ಷ್ಯ-ರೆಲ್ಲ
ಮನು-ಷ್ಯ-ಸ-ಹಾಯ
ಮನೆ
ಮನೆ-ಗಳಲ್ಲಿ
ಮನೆಗೆ
ಮನೆ-ಬಿಟ್ಟು
ಮನೆ-ಮನೆ
ಮನೆಯ
ಮನೆ-ಯನ್ನು
ಮನೆ-ಯಲ್ಲಿ
ಮನೆ-ಯ-ಲ್ಲಿದೆ
ಮನೆ-ಯ-ಲ್ಲಿ-ದ್ದರೆ
ಮನೆ-ಯ-ಲ್ಲಿ-ದ್ದಾಗ
ಮನೆ-ಯ-ಲ್ಲಿಯೇ
ಮನೆ-ಯಲ್ಲೇ
ಮನೆ-ಯಿಂದ
ಮನೆ-ಯಿ-ಲ್ಲದೆ
ಮನೋ
ಮನೋ-ಭಾವ
ಮನೋ-ಭಾ-ವದ
ಮನೋ-ಭಾ-ವ-ವನ್ನು
ಮನೋ-ಶ-ಕ್ತಿಯ
ಮನೋ-ಹ-ರ-ವಾ-ಗಿದೆ
ಮನ್ನಣೆ
ಮನ್ನ-ಣೆ-ಯನ್ನು
ಮಯ-ವಾ-ಗಿ-ರು-ವುದೊ
ಮಯ-ವಾದ
ಮರ-ಕ್ಕಿಂತ
ಮರಣ
ಮರ-ಣದ
ಮರದ
ಮರ-ಳನ್ನು
ಮರ-ವನ್ನು
ಮರುಕ
ಮರು-ಕ-ತೋ-ರದೆ
ಮರು-ಕ-ಪಡಿ
ಮರು-ಕ-ವಿ-ದೆಯೆ
ಮರು-ಕ್ಷ-ಣ-ದ-ಲ್ಲಿಯೇ
ಮರು-ಗು-ವುದೊ
ಮರು-ದನಿ
ಮರು-ದ-ನಿ-ಯಾಗಿ
ಮರೆ-ತರು
ಮರೆ-ತಿತ್ತೊ
ಮರೆ-ತಿ-ರು-ವರೋ
ಮರೆ-ತಿ-ರು-ವಿರಾ
ಮರೆ-ತಿ-ರು-ವೆವು
ಮರೆ-ತು-ಬಿ-ಟ್ಟಿ-ರು-ವರು
ಮರೆ-ತೆವು
ಮರೆಯ
ಮರೆ-ಯ-ಕೂ-ಡದು
ಮರೆ-ಯ-ದಂತೆ
ಮರೆ-ಯ-ದಿ-ರಲಿ
ಮರೆ-ಯ-ದಿರಿ
ಮರೆ-ಯ-ಬ-ಲ್ಲೆವೆ
ಮರೆ-ಯ-ಬೇಡಿ
ಮರೆ-ಯಿರಿ
ಮರೆ-ಯು-ವು-ದಕ್ಕೆ
ಮಲ-ಗು-ತ್ತಾರೆ
ಮಲ-ಬಾ-ರನ್ನು
ಮಲ-ಬಾ-ರಿ-ನಲ್ಲಿ
ಮಳೆ-ಯನ್ನು
ಮಹ-ತ್ಕಾ-ರ್ಯ-ವನ್ನು
ಮಹ-ತ್ಕಾ-ರ್ಯವೂ
ಮಹ-ತ್ತ-ರ-ವಾ-ದುದು
ಮಹತ್ವ
ಮಹ-ನೀ-ಯ-ರಲ್ಲಿ
ಮಹ-ಮ್ಮ-ದನ
ಮಹ-ಮ್ಮ-ದೀ-ಯರು
ಮಹ-ಮ್ಮದ್
ಮಹ-ರ್ಷಿ-ಗಳ
ಮಹಾ
ಮಹಾ-ಕವಿ
ಮಹಾ-ಕಾ-ರ್ಯ-ಗಳನ್ನು
ಮಹಾ-ಕಾಳಿ
ಮಹಾ-ಗಣಿ
ಮಹಾತ್ಮ
ಮಹಾ-ತ್ಮ-ನಿಗೆ
ಮಹಾ-ತ್ಮ-ನೆಂದು
ಮಹಾ-ತ್ಮರು
ಮಹಾ-ತ್ಮ-ರೆಲ್ಲ
ಮಹಾ-ನಿ-ಧಿ-ಯನ್ನು
ಮಹಾ-ಪಾ-ಪಿ-ಗಳು
ಮಹಾ-ಪು-ರು-ಷ-ನ-ನ್ನಾಗಿ
ಮಹಾ-ಪು-ರು-ಷ-ರಾ-ಗಿರು
ಮಹಾ-ಪು-ರು-ಷ-ರಿಗೆ
ಮಹಾ-ಪು-ರು-ಷರು
ಮಹಾ-ಪು-ಷಿ-ಗಳ
ಮಹಾ-ಪು-ಷಿ-ಗ-ಳಾ-ಗು-ತ್ತಿ-ದ್ದವು
ಮಹಾ-ಪ್ರ-ವಾಹ
ಮಹಾ-ಪ್ರ-ವಾ-ಹದ
ಮಹಾ-ಪ್ರ-ವಾ-ಹವೂ
ಮಹಾ-ಭಾ-ರತ
ಮಹಾ-ಭಾ-ವನೆ
ಮಹಾ-ಮ-ಹಿ-ಮ-ಳಾದ
ಮಹಾ-ಮು-ನಿ-ಗಳ
ಮಹಾ-ಮು-ನಿ-ಗ-ಳಾ-ಗುತ್ತಿ
ಮಹಾ-ಮೇ-ಧಾ-ವಿ-ಗಳು
ಮಹಾ-ಯೋ-ಗಿಯ
ಮಹಾ-ವೀ-ರರು
ಮಹಾ-ವೈ-ಭ-ವವೇ
ಮಹಾ-ವ್ಯಕ್ತಿ
ಮಹಾ-ವ್ಯ-ಕ್ತಿ-ಗಳ
ಮಹಾ-ಶಕ್ತಿ
ಮಹಾ-ಸ-ತ್ಯದ
ಮಹಾ-ಸ-ಭೆ-ಯಲ್ಲಿ
ಮಹಿಮಾ
ಮಹಿ-ಮಾ-ನ್ವಿತ
ಮಹಿ-ಮಾ-ನ್ವಿ-ತ-ರಾದ
ಮಹಿ-ಮಾ-ನ್ವಿ-ತ-ಳಾಗಿ
ಮಹಿ-ಮಾ-ನ್ವಿ-ತ-ವಾ-ಗು-ವುದು
ಮಹಿ-ಮಾ-ನ್ವಿ-ತ-ವಾದ
ಮಹಿ-ಮಾ-ವಂ-ತ-ರಲ್ಲಿ
ಮಹಿಮೆ
ಮಹಿ-ಮೆ-ಯನ್ನು
ಮಹಿಳೆ
ಮಹೋ-ನ್ನತ
ಮಾ
ಮಾಂಸ
ಮಾಂಸ-ಖಂಡ
ಮಾಂಸ-ಖಂ-ಡ-ಗಳು
ಮಾಡ
ಮಾಡದ
ಮಾಡದೆ
ಮಾಡದೇ
ಮಾಡ-ಬಲ್ಲ
ಮಾಡ-ಬ-ಲ್ಲದು
ಮಾಡ-ಬ-ಲ್ಲದೊ
ಮಾಡ-ಬ-ಲ್ಲರು
ಮಾಡ-ಬ-ಲ್ಲಿರಿ
ಮಾಡ-ಬ-ಲ್ಲಿರೋ
ಮಾಡ-ಬ-ಲ್ಲೆಯಾ
ಮಾಡ-ಬ-ಹುದು
ಮಾಡ-ಬಾ-ರದು
ಮಾಡ-ಬೇ-ಕಾ-ಗಿದೆ
ಮಾಡ-ಬೇ-ಕಾದ
ಮಾಡ-ಬೇ-ಕಾ-ದರೂ
ಮಾಡ-ಬೇ-ಕಾ-ದರೆ
ಮಾಡ-ಬೇ-ಕಾ-ದುದೇ
ಮಾಡ-ಬೇ-ಕಾ-ಯಿತು
ಮಾಡ-ಬೇ-ಕಾ-ಯಿತೆ
ಮಾಡ-ಬೇಕು
ಮಾಡ-ಬೇ-ಕೆಂದು
ಮಾಡ-ಬೇ-ಕೆಂ-ಬು-ದನ್ನು
ಮಾಡ-ಬೇ-ಕೆಂ-ಬುದೇ
ಮಾಡ-ಬೇಕೊ
ಮಾಡ-ಬೇಡಿ
ಮಾಡ-ಲಾ-ಗದ
ಮಾಡ-ಲಾ-ರದು
ಮಾಡ-ಲಾ-ರದೊ
ಮಾಡ-ಲಾ-ರವು
ಮಾಡ-ಲಾ-ರಿರಿ
ಮಾಡ-ಲಾ-ರೆವು
ಮಾಡಲಿ
ಮಾಡ-ಲಿಲ್ಲ
ಮಾಡ-ಲಿ-ಲ್ಲವೆ
ಮಾಡಲು
ಮಾಡ-ಲೆ-ತ್ನಿ-ಸು-ವುದು
ಮಾಡಿ
ಮಾಡಿ-ಕೊಂ-ಡರೆ
ಮಾಡಿ-ಕೊಂ-ಡ-ವನ
ಮಾಡಿ-ಕೊಂ-ಡಿರು
ಮಾಡಿ-ಕೊಂ-ಡಿ-ರು-ವ-ವನಿ
ಮಾಡಿ-ಕೊಂ-ಡಿಲ್ಲ
ಮಾಡಿ-ಕೊಳ್ಳ
ಮಾಡಿ-ಕೊ-ಳ್ಳದೇ
ಮಾಡಿ-ಕೊ-ಳ್ಳ-ಬ-ಲ್ಲರು
ಮಾಡಿ-ಕೊ-ಳ್ಳ-ಬೇ-ಕಾ-ಗಿದೆ
ಮಾಡಿ-ಕೊ-ಳ್ಳಲು
ಮಾಡಿ-ಕೊ-ಳ್ಳಲೇ
ಮಾಡಿ-ಕೊ-ಳ್ಳು-ತ್ತಾನೆ
ಮಾಡಿ-ಕೊ-ಳ್ಳು-ತ್ತೀರೊ
ಮಾಡಿ-ಕೊ-ಳ್ಳು-ವರು
ಮಾಡಿ-ಕೊ-ಳ್ಳು-ವಿರಿ
ಮಾಡಿ-ಕೊ-ಳ್ಳು-ವುದನ್ನು
ಮಾಡಿ-ಕೊ-ಳ್ಳು-ವು-ದ-ರಿಂದ
ಮಾಡಿದ
ಮಾಡಿ-ದಂತೆ
ಮಾಡಿ-ದಂ-ತೆಲ್ಲ
ಮಾಡಿ-ದಂ-ದಿ-ನಿಂದ
ಮಾಡಿ-ದರು
ಮಾಡಿ-ದರೂ
ಮಾಡಿ-ದರೆ
ಮಾಡಿ-ದಷ್ಟು
ಮಾಡಿದೆ
ಮಾಡಿ-ದೆಯೆ
ಮಾಡಿ-ದೆವು
ಮಾಡಿ-ದೊ-ಡ-ನೆಯೆ
ಮಾಡಿ-ದ್ದರೂ
ಮಾಡಿ-ದ್ದರೆ
ಮಾಡಿ-ದ್ದಾರೆ
ಮಾಡಿ-ದ್ದೀರಿ
ಮಾಡಿದ್ದು
ಮಾಡಿ-ದ್ದುವು
ಮಾಡಿದ್ದೇ
ಮಾಡಿ-ರು-ವಂತೆ
ಮಾಡಿ-ರು-ವರು
ಮಾಡಿ-ರು-ವಿರಾ
ಮಾಡಿ-ರು-ವಿರಿ
ಮಾಡಿ-ರು-ವಿ-ರೇನು
ಮಾಡಿ-ರು-ವೆವು
ಮಾಡಿಲ್ಲ
ಮಾಡಿವೆ
ಮಾಡಿ-ಸಿ-ದರೆ
ಮಾಡು
ಮಾಡು-ತ್ತದೆ
ಮಾಡುತ್ತಾ
ಮಾಡು-ತ್ತಾರೆ
ಮಾಡು-ತ್ತಾರೋ
ಮಾಡು-ತ್ತಿದೆ
ಮಾಡು-ತ್ತಿ-ದ್ದರು
ಮಾಡು-ತ್ತಿ-ದ್ದರೂ
ಮಾಡು-ತ್ತಿ-ದ್ದಿರಿ
ಮಾಡು-ತ್ತಿರು
ಮಾಡು-ತ್ತಿ-ರು-ತ್ತದೆ
ಮಾಡು-ತ್ತಿ-ರುವ
ಮಾಡು-ತ್ತಿ-ರು-ವರು
ಮಾಡು-ತ್ತಿ-ರು-ವ-ವರ
ಮಾಡು-ತ್ತಿ-ರು-ವ-ವ-ರಿಗೆ
ಮಾಡು-ತ್ತಿ-ರು-ವಾಗ
ಮಾಡು-ತ್ತಿ-ರು-ವುದು
ಮಾಡು-ತ್ತಿ-ರು-ವೆನೊ
ಮಾಡು-ತ್ತಿ-ರು-ವೆವು
ಮಾಡು-ತ್ತಿ-ಲ್ಲವೆ
ಮಾಡು-ತ್ತೀಯೆ
ಮಾಡು-ತ್ತೇನೆ
ಮಾಡುವ
ಮಾಡು-ವಂ-ತಹ
ಮಾಡು-ವಂ-ತಿಲ್ಲ
ಮಾಡು-ವಂತೆ
ಮಾಡು-ವನು
ಮಾಡು-ವನೊ
ಮಾಡು-ವರು
ಮಾಡು-ವರೊ
ಮಾಡು-ವ-ವ-ನಿಗೆ
ಮಾಡು-ವ-ವನು
ಮಾಡು-ವ-ವನೇ
ಮಾಡು-ವ-ವ-ರಲ್ಲಿ
ಮಾಡು-ವ-ವ-ರಾಗಿ
ಮಾಡು-ವ-ವ-ರೆಗೆ
ಮಾಡು-ವಷ್ಟು
ಮಾಡು-ವಾಗ
ಮಾಡು-ವಿರಿ
ಮಾಡು-ವಿರೊ
ಮಾಡುವು
ಮಾಡು-ವು-ದ-ಕ್ಕಿಂತ
ಮಾಡು-ವು-ದಕ್ಕೆ
ಮಾಡು-ವುದನ್ನು
ಮಾಡು-ವು-ದಾ-ಗಿ-ರ-ಬೇಕು
ಮಾಡು-ವು-ದಿಲ್ಲ
ಮಾಡು-ವುದು
ಮಾಡು-ವು-ದೆಂ-ದರೆ
ಮಾಡು-ವು-ದೆಲ್ಲ
ಮಾಡು-ವುದೇ
ಮಾಡು-ವು-ದೇನು
ಮಾಡು-ವುದೊ
ಮಾಡು-ವೆವು
ಮಾಡೋಣ
ಮಾಣಿ-ಕ್ಯ-ವನ್ನು
ಮಾತ
ಮಾತ-ನಾ-ಡ-ಬಲ್ಲ
ಮಾತ-ನಾ-ಡ-ಬ-ಹುದು
ಮಾತ-ನಾ-ಡ-ಬೇಕು
ಮಾತ-ನಾ-ಡಲಿ
ಮಾತ-ನಾ-ಡಿ-ದರೆ
ಮಾತ-ನಾ-ಡು-ತ್ತಿ-ದ್ದರೂ
ಮಾತ-ನಾ-ಡು-ವರು
ಮಾತನ್ನು
ಮಾತಲ್ಲ
ಮಾತಾ-ಡ-ಲಾರ
ಮಾತಾ-ಡು-ವೆವು
ಮಾತಿಗೆ
ಮಾತು
ಮಾತು-ಗಳು
ಮಾತೃ-ಭೂಮಿ
ಮಾತ್ರ
ಮಾತ್ರಕ್ಕೆ
ಮಾತ್ರ-ವಲ್ಲ
ಮಾತ್ರ-ವಾಗಿ
ಮಾಧುರ್ಯ
ಮಾಧು-ರ್ಯತೆ
ಮಾನ
ಮಾನ-ಗಳ
ಮಾನ-ಗಳನ್ನು
ಮಾನವ
ಮಾನ-ವ-ಕೋಟಿ
ಮಾನ-ವ-ಕೋ-ಟಿಯ
ಮಾನ-ವ-ಕೋ-ಟಿ-ಯನ್ನು
ಮಾನ-ವನ
ಮಾನ-ವ-ನಿಗೆ
ಮಾನ-ವನು
ಮಾನ-ವ-ನೆ-ದು-ರಿಗೆ
ಮಾನ-ವರೆ
ಮಾನ-ವ-ಸ-ಹಜ
ಮಾನ-ಸಿಕ
ಮಾನ-ಸಿ-ಕ-ವಾಗಿ
ಮಾಯ-ವಾಗ
ಮಾಯ-ವಾ-ಗಿವೆ
ಮಾಯ-ವಾ-ಗು-ತ್ತ-ಲಿವೆ
ಮಾಯ-ವಾ-ಗು-ತ್ತಿವೆ
ಮಾಯ-ವಾ-ಗು-ವುವು
ಮಾಯಾ-ದೀಪ
ಮಾಯೆ
ಮಾರ-ಬಾ-ರದು
ಮಾರ್ಗ
ಮಾರ್ಗ-ದಲ್ಲಿ
ಮಾರ್ಗ-ದ-ಲ್ಲಿ-ರು-ವಿರಿ
ಮಾರ್ಗ-ದ-ಲ್ಲಿವೆ
ಮಾರ್ಗ-ದಿಂದ
ಮಾರ್ಗ-ವನ್ನು
ಮಾರ್ಗವೇ
ಮಾರ್ದ-ವ-ಗಳ
ಮಾರ್ಪ-ಡಿ-ಸಲು
ಮಾರ್ಪ-ಡಿಸಿ
ಮಿಂಚಿ-ನ-ಗೊಂ-ಚ-ಲಿ-ನ-ಲ್ಲಿ-ರುವ
ಮಿಕ್ಕಿದ್ದು
ಮಿಡಿ-ಯು-ತ್ತಿ-ದೆಯೆ
ಮಿತ್ರ
ಮಿಥ್ಯಾ-ಚಾ-ರಿ-ಗಳು
ಮಿರುಗಿ
ಮಿರು-ಗು-ತ್ತಿ-ದ್ದರೂ
ಮಿಲ-ನ-ವಾ-ಗಿ-ರ-ಬೇಕು
ಮಿಲಿ-ಟರಿ
ಮೀರ
ಮೀರ-ಲೆ-ತ್ನಿ-ಸು-ವನು
ಮೀರಾ-ಬಾಯಿ
ಮೀರಿ
ಮೀರಿದ
ಮೀರಿ-ಸಿದ
ಮೀರಿ-ಸುವ
ಮೀರಿ-ಹೋ-ಗು-ವನೊ
ಮೀರುವ
ಮೀಸ-ಲಾ-ಗಿ-ಟ್ಟಿ-ರುವ
ಮೀಸ-ಲಾ-ಗಿತ್ತು
ಮೀಸ-ಲಾದ
ಮೀಸಲು
ಮುಂಚೆ
ಮುಂಚೆಯೇ
ಮುಂತಾ
ಮುಂತಾದ
ಮುಂತಾ-ದವ
ಮುಂತಾ-ದ-ವರ
ಮುಂತಾ-ದ-ವ-ರಲ್ಲಿ
ಮುಂತಾ-ದ-ವರು
ಮುಂತಾ-ದವು
ಮುಂತಾ-ದ-ವು-ಗಳ
ಮುಂತಾದು
ಮುಂತಾ-ದು-ವನ್ನು
ಮುಂತಾ-ದುವು
ಮುಂತಾ-ದು-ವು-ಗ-ಳಿಗೆ
ಮುಂತಾ-ದು-ವು-ಗಳು
ಮುಂದಕ್ಕೆ
ಮುಂದಾಳು
ಮುಂದಿ-ದೆಯೋ
ಮುಂದಿನ
ಮುಂದಿ-ನ-ದನ್ನು
ಮುಂದಿ-ನದು
ಮುಂದಿ-ರುವ
ಮುಂದಿ-ರು-ವುದು
ಮುಂದು-ವರಿ
ಮುಂದು-ವ-ರಿ-ದರೆ
ಮುಂದು-ವ-ರಿ-ದ-ವರು
ಮುಂದು-ವ-ರಿ-ಯಿರಿ
ಮುಂದು-ವ-ರಿ-ಯು-ತ್ತಿವೆ
ಮುಂದು-ವ-ರಿ-ಯುವೆ
ಮುಂದು-ವ-ರಿ-ಯು-ವೆವು
ಮುಂದು-ವ-ರಿ-ಸ-ಬ-ಲ್ಲಂ-ತಹ
ಮುಂದೆ
ಮುಂದೆಯೂ
ಮುಂದೇ-ನಾ-ಗು-ವುದು
ಮುಕ್ಕ-ಳಿ-ಸು-ವುದು
ಮುಕ್ತ-ರಾಗಿ
ಮುಕ್ತಿ
ಮುಕ್ತಿಗೆ
ಮುಕ್ತಿ-ಯನ್ನು
ಮುಕ್ತಿ-ಸೇವೆ
ಮುಖ
ಮುಖ-ದಲ್ಲಿ
ಮುಖ-ದ-ಲ್ಲಿ-ರುವ
ಮುಖ-ವನ್ನೇ
ಮುಖ್ಯ
ಮುಖ್ಯ-ವಾಗಿ
ಮುಖ್ಯ-ವಾದ
ಮುಖ್ಯ-ವೆಂದು
ಮುಖ್ಯವೋ
ಮುಖ್ಯಾಂ-ಶ-ಗ-ಳಿವೆ
ಮುಗಿ-ಯಿತು
ಮುಚ್ಚಿ
ಮುಚ್ಚಿ-ಕೊ-ಳ್ಳು-ವಂತೆ
ಮುಟ್ಟ
ಮುಟ್ಟ-ಬೇಡ
ಮುಟ್ಟ-ಬೇಡಿ
ಮುಟ್ಟಲು
ಮುಟ್ಟಿ-ದರೆ
ಮುಟ್ಟಿ-ದ-ವ-ಳಲ್ಲ
ಮುಟ್ಟಿ-ದ್ದರೆ
ಮುಟ್ಟಿ-ರು-ವನು
ಮುಟ್ಟಿ-ರು-ವರು
ಮುಟ್ಟಿವೆ
ಮುಟ್ಟುವ
ಮುಟ್ಟು-ವು-ದಿಲ್ಲ
ಮುತ್ತ-ಲಾ-ರದು
ಮುತ್ತಿ
ಮುತ್ತು-ತ್ತಿ-ದ್ದರು
ಮುತ್ತುವು
ಮುರ್ಖ-ರಾಗಿ
ಮುಳುಗಿ
ಮುಳು-ಗಿದೆ
ಮುಳು-ಗಿ-ದ್ದು-ದ-ರಿಂದ
ಮುಳು-ಗಿ-ಹೋ-ಯಿತು
ಮುಳುಗು
ಮುಷ್ಕ-ರ-ಗಳನ್ನು
ಮುಸ-ಲ್ಮಾ-ನ-ರಾ-ಗಲಿ
ಮುಸುಕಿ
ಮೂಡ-ಬ-ಹುದು
ಮೂಡಿದ
ಮೂಡಿ-ಹೋ-ಗಿತ್ತು
ಮೂಡು-ತ್ತಿದೆ
ಮೂಡು-ವಂತೆ
ಮೂಡು-ವು-ದ-ರಲ್ಲಿ
ಮೂಡು-ವುದು
ಮೂಡು-ವುದೊ
ಮೂಢ
ಮೂಢ-ನಂ-ಬಿಕೆ
ಮೂಢ-ನಂ-ಬಿ-ಕೆಗೂ
ಮೂಢ-ನಂ-ಬಿ-ಕೆಗೆ
ಮೂಢರು
ಮೂಢಾ
ಮೂಢಾ-ಚಾರ
ಮೂಢಾ-ಚಾ-ರಕ್ಕೂ
ಮೂಢಾ-ಚಾ-ರ-ಗಳನ್ನು
ಮೂಢಾ-ಚಾ-ರ-ಗ-ಳಿವೆ
ಮೂಢಾ-ಚಾ-ರದ
ಮೂಢಾ-ಚಾ-ರ-ದಿಂದ
ಮೂದ-ಲಿ-ಸು-ತ್ತೇವೆ
ಮೂರ-ನೆ-ಯದೇ
ಮೂರು
ಮೂರು-ದಿನ
ಮೂರು-ಬಾರಿ
ಮೂರೂ
ಮೂರ್ಖ
ಮೂರ್ಖ-ತನ
ಮೂರ್ಖ-ನಾ-ಗು-ವೆನು
ಮೂರ್ಖರ
ಮೂರ್ಖ-ರಲ್ಲ
ಮೂರ್ಖ-ರಿಗೆ
ಮೂರ್ಖರು
ಮೂಲ
ಮೂಲಕ
ಮೂಲ-ಕವೇ
ಮೂಲ-ಕಾ-ರಣ
ಮೂಲ-ಕೇಂ-ದ್ರ-ವನ್ನು
ಮೂಲಕ್ಕೆ
ಮೂಲ-ಚೈ-ತ-ನ್ಯ-ವಾದ
ಮೂಲ-ಭಾ-ವನೆ
ಮೂಲ-ರೂ-ಪಕ್ಕೆ
ಮೂಲ-ವಾ-ಗಿ-ಟ್ಟು-ಕೊಂಡು
ಮೂಲ-ವೆಲ್ಲ
ಮೂಲವೇ
ಮೂಲ-ಸ್ಥಾ-ನ-ವಾ-ಗಿದೆ
ಮೂಲೆ
ಮೂಲೆಗೆ
ಮೂಳೆ
ಮೂವತ್ತು
ಮೂವ-ತ್ತು-ಕೋಟಿ
ಮೂವ-ತ್ತೈ-ದು-ಕೋಟಿ
ಮೂಸೆಗೆ
ಮೃಗ-ಗ-ಳೆಂದೆ
ಮೃಗ-ಸ-ಮಾ-ನ-ರಾ-ಗಿರು
ಮೃತ್ಯು
ಮೃತ್ಯು-ಗಿಂತ
ಮೃತ್ಯು-ವ-ನ್ನಾ-ದರೂ
ಮೃತ್ಯು-ವನ್ನು
ಮೃತ್ಯು-ವಿಗೂ
ಮೃತ್ಯು-ಸ್ಥಿತಿ
ಮೆಚ್ಚ-ಬೇ-ಕಾ-ದರೆ
ಮೆಚ್ಚಿ-ದರೂ
ಮೆಚ್ಚು-ತ್ತೀರಿ
ಮೆಚ್ಚು-ತ್ತೇನೆ
ಮೆಚ್ಚು-ವುದು
ಮೆಟ್ಟ-ಲಿಗೆ
ಮೆಟ್ಟಲು
ಮೆಟ್ಟಿ-ಕೊಂಡು
ಮೆರ-ವ-ಣಿ-ಗೆ-ಯಲ್ಲಿ
ಮೆರೆ-ದರೆ
ಮೆರೆ-ಯು-ತ್ತೀರಿ
ಮೆಲ-ಕು-ಹಾ-ಕುವ
ಮೆಲು-ಕು-ಹಾ-ಕು-ವುದೇ
ಮೆಲ್ಲಗೆ
ಮೇಧಾ-ವಿ-ಗಳು
ಮೇಯಿ-ಸಲು
ಮೇರು
ಮೇರೆ
ಮೇಲ
ಮೇಲಕ್ಕೆ
ಮೇಲ-ಲ್ಲದ
ಮೇಲಾ-ಗು-ವುದು
ಮೇಲಾದೆ
ಮೇಲಿದೆ
ಮೇಲಿ-ದ್ದರು
ಮೇಲಿನ
ಮೇಲಿ-ನ-ವ-ನಿಗೆ
ಮೇಲಿ-ನ-ವ-ರನ್ನು
ಮೇಲಿ-ನ-ವ-ರಿಂದ
ಮೇಲಿ-ನ-ವರು
ಮೇಲಿ-ನಿಂದ
ಮೇಲಿ-ರುವ
ಮೇಲಿ-ರು-ವ-ವನೂ
ಮೇಲಿ-ರು-ವ-ವ-ರನ್ನು
ಮೇಲಿ-ರು-ವ-ವರು
ಮೇಲಿಲ್ಲ
ಮೇಲು
ಮೇಲು-ಮ-ಟ್ಟಕ್ಕೆ
ಮೇಲೂ
ಮೇಲೆ
ಮೇಲೆ-ತ್ತ-ಬೇ-ಕಾ-ದರೆ
ಮೇಲೆ-ತ್ತ-ಬೇಕು
ಮೇಲೆ-ತ್ತಲು
ಮೇಲೆತ್ತಿ
ಮೇಲೆ-ತ್ತಿದೆ
ಮೇಲೆ-ತ್ತು-ವು-ದಕ್ಕೆ
ಮೇಲೆ-ತ್ತು-ವುದು
ಮೇಲೆ-ದ್ದಿದೆ
ಮೇಲೆದ್ದು
ಮೇಲೆಯೂ
ಮೇಲೆಯೇ
ಮೇಲೆಯೋ
ಮೇಲೆಲ್ಲ
ಮೇಲೇಳ
ಮೇಲೇ-ಳ-ಬೇ-ಕಾ-ದರೆ
ಮೇಲೇ-ಳ-ಬೇಕು
ಮೇಲೇ-ಳ-ಲಾ-ರರು
ಮೇಲೇ-ಳಲಿ
ಮೇಲೇ-ಳು-ತ್ತಿದೆ
ಮೇಲೊ
ಮೇಲೊಂ-ದಾಗಿ
ಮೇಲ್ಪಂ-ಕ್ತಿ-ಯಲ್ಲಿ
ಮೇಲ್ಮೆಗೆ
ಮೈ
ಮೈಗೆ
ಮೈತ್ರೇಯಿ
ಮೈಲಿಗೆ
ಮೈವೆ-ತ್ತಂತೆ
ಮೈಸೂರು
ಮೊಗ-ದಲ್ಲಿ
ಮೊತ್ತ
ಮೊತ್ತ-ದಿಂದ
ಮೊತ್ತವೇ
ಮೊದ-ಮೊ-ದಲು
ಮೊದಲ
ಮೊದ-ಲನೆ
ಮೊದ-ಲ-ನೆಯ
ಮೊದ-ಲ-ನೆ-ಯ-ದಾಗಿ
ಮೊದ-ಲನೇ
ಮೊದ-ಲಾದ
ಮೊದ-ಲಾ-ದ-ವರು
ಮೊದ-ಲಾ-ದುದು
ಮೊದ-ಲಿಗೆ
ಮೊದ-ಲಿ-ನಿಂ-ದಲೂ
ಮೊದಲು
ಮೊಳೆ-ಯು-ತ್ತಿದೆ
ಮೊಸಳೆ
ಮೋಕ್ಷಕ್ಕೆ
ಮೋಚಿ
ಮೋಚಿ-ಯಂತೆ
ಮೋಹವು
ಮೋಹಿ-ನಿ-ಯನ್ನು
ಮೌಢ್ಯತೆ
ಮೌಢ್ಯ-ತೆ-ಯಿಂದ
ಮೌಢ್ಯದ
ಮೌನ
ಮೌನ-ದಿಂ-ದಲೋ
ಮೌನ-ವಾಗಿ
ಮೌನ-ವಾದ
ಮ್ಲೇಚ-ಊ-ಇ-್ಕದ್
ಯಂತಿದೆ
ಯಂತ್ರ
ಯಂತ್ರ-ಗಳ
ಯಂತ್ರ-ಗ-ಳಂತೆ
ಯಂತ್ರ-ಗ-ಳ-ನ್ನಾಗಿ
ಯಂತ್ರ-ಗಳು
ಯಂತ್ರ-ಗಳೇ
ಯಂತ್ರದ
ಯಂತ್ರ-ದಂತೆ
ಯಂತ್ರ-ದ-ಲ್ಲಿಲ್ಲ
ಯಂತ್ರ-ದಿಂದ
ಯಂತ್ರ-ಸ-ದೃ-ಶ-ವ-ನ್ನಾಗಿ
ಯಜ-ಮಾ-ನ-ನಂತೆ
ಯಜ-ಮಾನ್ಯ
ಯಜ್ಞ-ಗಳಲ್ಲಿ
ಯತ್ನಿ-ಸ-ಬೇಕು
ಯತ್ನಿಸಿ
ಯತ್ನಿ-ಸಿ-ದರು
ಯತ್ನಿ-ಸಿ-ದರೆ
ಯತ್ನಿಸು
ಯತ್ನಿ-ಸು-ವನು
ಯತ್ನಿ-ಸು-ವನೊ
ಯತ್ನಿ-ಸು-ವುವೋ
ಯತ್ರ
ಯದು
ಯನ
ಯನ್ನರು
ಯನ್ನು
ಯರು
ಯಲ್ಲ
ಯಲ್ಲಿ
ಯಶಸ್ಸು
ಯಹೂದ್ಯ
ಯಾಂತ್ರಿಕ
ಯಾಂತ್ರಿ-ಕ-ಶಿ-ಕ್ಷಣ
ಯಾಗ
ಯಾಗಿ
ಯಾಗು-ವುದು
ಯಾಜ್ಞ-ವ-ಲ್ಕ್ಯ-ನನ್ನು
ಯಾತ್ರೆ-ಯನ್ನು
ಯಾದ
ಯಾದರೆ
ಯಾದ-ವ-ಗಿರಿ
ಯಾರ
ಯಾರನ್ನು
ಯಾರನ್ನೂ
ಯಾರಲ್ಲಿ
ಯಾರಾ
ಯಾರಾ-ದರೂ
ಯಾರಿಂದ
ಯಾರಿಗೂ
ಯಾರಿಗೆ
ಯಾರು
ಯಾರೂ
ಯಾರೊ
ಯಾರೋ
ಯಾರ್ಕಿ-ನ-ಲ್ಲಿ-ದ್ದಾಗ
ಯಾವ
ಯಾವನ
ಯಾವ-ನಲ್ಲಿ
ಯಾವಾ
ಯಾವಾಗ
ಯಾವಾ-ಗ-ಲಾ-ದರೂ
ಯಾವಾ-ಗಲೂ
ಯಾವು
ಯಾವು-ದಕ್ಕೂ
ಯಾವು-ದಕ್ಕೆ
ಯಾವುದನ್ನು
ಯಾವು-ದನ್ನೂ
ಯಾವು-ದರ
ಯಾವು-ದ-ರಿಂದ
ಯಾವುದಾ
ಯಾವು-ದಾ-ದರೂ
ಯಾವು-ದಾ-ದ-ರೊಂದು
ಯಾವುದು
ಯಾವುದೂ
ಯಾವುದೋ
ಯಾವುವೂ
ಯಿಂದ
ಯಿತು
ಯುಗದ
ಯುಗ-ಯು-ಗ-ಗಳ
ಯುಗ-ಯು-ಗ-ಗಳಿಂದ
ಯುತ್ತಾ
ಯುದ್ಧ-ದಲ್ಲಿ
ಯುದ್ಧ-ವಿ-ರ-ಕೂ-ಡದು
ಯುರೋಪಿ
ಯುರೋ-ಪಿ-ಗೆಲ್ಲ
ಯುರೋ-ಪಿನ
ಯುರೋ-ಪಿ-ನಲ್ಲಿ
ಯುರೋ-ಪಿ-ನೊಂ-ದಿಗೆ
ಯುರೋಪು
ಯುವ
ಯುವ-ಕರ
ಯುವ-ಕರು
ಯುವ-ಕರೆ
ಯುವನು
ಯುವುದು
ಯುವುದೆ
ಯುವುದೇ
ಯೂರೋ-ಪಿನ
ಯೂರೋಪ್
ಯೊಂದನ್ನೂ
ಯೊಂದು
ಯೊಂದೇ
ಯೊಬ್ಬನ
ಯೊಬ್ಬನೂ
ಯೊಬ್ಬ-ರಿಗೂ
ಯೊಬ್ಬರೂ
ಯೋಗಿ-ಸುವ
ಯೋಗ್ಯ
ಯೋಗ್ಯತೆ
ಯೋಗ್ಯ-ತೆ-ಯನ್ನು
ಯೋಗ್ಯ-ರಲ್ಲ
ಯೋಗ್ಯ-ರ-ಲ್ಲದ
ಯೋಗ್ಯರು
ಯೋಗ್ಯ-ರೆಂದು
ಯೋಗ್ಯ-ವಲ್ಲ
ಯೋಗ್ಯ-ವಾದ
ಯೋಗ್ಯ-ವಾ-ದುದು
ಯೋಗ್ಯ-ವಾ-ದು-ದೆಂದು
ಯೋಚಿಸಿ
ಯೋಜನೆ
ಯೋಜ-ನೆ-ಗ-ಳಿಗೆ
ಯೋಜ-ನೆ-ಗಳು
ಯೋಜ-ನೆಯ
ಯೋಜ-ನೆ-ಯನ್ನು
ರಂತೆ
ರಕ್ತಕ್ಕೆ
ರಕ್ತ-ಗತ
ರಕ್ತ-ಗ-ತ-ಮಾ-ಡಿ-ಕೊ-ಳ್ಳು-ವಂತೆ
ರಕ್ತ-ಗ-ತ-ವಾ-ಗಲಿ
ರಕ್ತ-ಗ-ತ-ವಾಗಿ
ರಕ್ತ-ಗ-ತ-ವಾದ
ರಕ್ತದ
ರಕ್ತ-ದಲ್ಲಿ
ರಕ್ತ-ದ-ಲ್ಲಿಯೂ
ರಕ್ಷ-ಣೆಗೆ
ರಕ್ಷಿ-ಸ-ಬೇ-ಕಾದ
ರಕ್ಷಿ-ಸ-ಬೇಕು
ರಕ್ಷಿ-ಸಿ-ಕೊಂಡು
ರಕ್ಷಿ-ಸಿ-ಕೊ-ಳ್ಳ-ಬೇಕು
ರಕ್ಷಿ-ಸು-ವು-ದ-ಕ್ಕಾಗಿ
ರಕ್ಷಿ-ಸು-ವು-ದಕ್ಕೆ
ರಕ್ಷಿ-ಸು-ವುದೇ
ರಚಿಸಿ
ರಚಿ-ಸಿದ
ರಚಿ-ಸಿದೆ
ರಚಿ-ಸಿ-ರು-ವ-ರು-ಅ-ವರು
ರಜಸ್
ರಣ
ರಣೆ
ರಥ
ರನ್ನಾಗಿ
ರನ್ನು
ರಮಂತೇ
ರಲ್ಲಿ
ರಲ್ಲಿಯೂ
ರಸಾ-ನ್ನ-ವನ್ನು
ರಸಾ-ಯ-ನ-ಶಾ-ಸ್ತ್ರಜ್ಞ
ರಹಸ್ಯ
ರಹ-ಸ್ಯ-ಗಳನ್ನು
ರಹ-ಸ್ಯ-ಗಳನ್ನೆಲ್ಲ
ರಹ-ಸ್ಯ-ವನ್ನು
ರಹ-ಸ್ಯ-ವಿದೆ
ರಹ-ಸ್ಯ-ವಿ-ರು-ವುದು
ರಹ-ಸ್ಯ-ವೆಲ್ಲ
ರಾಕ್ಷಸ
ರಾಕ್ಷ-ಸತ್ವ
ರಾಕ್ಷ-ಸರೆ
ರಾಗಿ
ರಾಗು-ತ್ತಾರೆ
ರಾಗು-ತ್ತಿ-ರು-ವುದು
ರಾಗು-ವರು
ರಾಗು-ವ-ವ-ರೆಗೆ
ರಾಗು-ವು-ದ-ರ-ಲ್ಲಿದೆ
ರಾಜ
ರಾಜ-ಕೀಯ
ರಾಜ-ಕೀ-ಯದ
ರಾಜ-ಕೀ-ಯ-ದಿಂದ
ರಾಜ-ಕೀ-ಯ-ದಿಂ-ದಲೂ
ರಾಜ-ಕೀ-ಯ-ವನ್ನು
ರಾಜ-ಕೀ-ಯವೇ
ರಾಜ-ಕೀ-ಯವೊ
ರಾಜನ
ರಾಜ-ನಾ-ಗು-ತ್ತೇ-ನೆಯೇ
ರಾಜರ
ರಾಜ-ರಲ್ಲ
ರಾಜರು
ರಾಜ-ರು-ಗಳು
ರಾಜ-ವೈ-ಭ-ವ-ಗಳು
ರಾಜ್ಯ
ರಾಜ್ಯ-ಗಳು
ರಾಜ್ಯವ
ರಾಜ್ಯ-ವಾ-ಳ-ಬ-ಲ್ಲೆನೆ
ರಾತ್ರಿ
ರಾದ-ವ-ರೊಂ-ದಿಗೆ
ರಾಮನ
ರಾಮ-ಮೋ-ಹನ
ರಾಮಾ-ಯಣ
ರಾಯ-ನ-ವ-ರೆಗೆ
ರಾಷ್ಟ್ರ
ರಾಷ್ಟ್ರ-ಗ-ಳಿಗೆ
ರಾಷ್ಟ್ರ-ಗಳು
ರಾಷ್ಟ್ರ-ಗಳೂ
ರಾಷ್ಟ್ರ-ಜೀ-ವ-ನದ
ರಾಷ್ಟ್ರದ
ರಾಷ್ಟ್ರ-ದಲ್ಲಿ
ರಾಷ್ಟ್ರ-ನಿ-ರ್ಮಾಣ
ರಾಷ್ಟ್ರ-ನಿ-ರ್ಮಾ-ಪಕ
ರಾಷ್ಟ್ರ-ವಾ-ಗ-ಬೇ-ಕಾ-ದರೆ
ರಾಷ್ಟ್ರ-ವಾ-ಗಲಿ
ರಾಷ್ಟ್ರವೇ
ರಾಸಾ-ಯ-ನಿಕ
ರಿಂದ
ರಿಂದಲೇ
ರಿಗೂ
ರಿಗೆ
ರಿಪೇರಿ
ರಿಪೇ-ರಿ-ಮಾ-ಡು-ತ್ತೇನೆ
ರೀತಿ
ರೀತಿ-ಯ-ನ್ನೆಲ್ಲ
ರೀತಿ-ಯಲ್ಲಿ
ರೀತಿಯೇ
ರುವ
ರುವುದು
ರೂಢಿ-ಸ-ಬೇ-ಕಾ-ಗಿದೆ
ರೂಢಿಸಿ
ರೂಢಿ-ಸಿ-ಕೊಂ-ಡಂತೆ
ರೂಢಿ-ಸಿ-ಕೊ-ಳ್ಳ-ಬೇಕು
ರೂಢಿ-ಸುವ
ರೂಪಕ್ಕೆ
ರೂಪಾಯಿ
ರೂಪಾ-ಯಿ-ಗಳನ್ನು
ರೂಪಿ-ಸ-ಬೇಕು
ರೂಪಿ-ಸಿ-ಕೊ-ಳ್ಳ-ಬೇಕು
ರೂಪಿ-ಸಿ-ದರೆ
ರೂಪಿ-ಸು-ತ್ತೇನೆ
ರೂಪಿ-ಸು-ವು-ದ-ರಲ್ಲಿ
ರೂಪು-ಗೊಂ-ಡಿದೆ
ರೂಪು-ವೆ-ತ್ತಂತೆ
ರೆಕ್ಕೆ
ರೆಕ್ಕೆ-ಗ-ಳಂತೆ
ರೆಲ್ಲ
ರೈತಾಪಿ
ರೋಗಿ-ಯನ್ನು
ರೋಮನ್
ರೋಮಿನ
ರೋಮ್
ರ್ಮುಖ-ವಾ-ದಂತೆ
ಲಕ್ಷ
ಲಕ್ಷಕ್ಕೆ
ಲಕ್ಷ-ದಲ್ಲಿ
ಲಕ್ಷಾಂ-ತರ
ಲಘು-ವಾಗಿ
ಲಜ್ಜೆ
ಲವ-ಲೇ-ಶವೂ
ಲಾಗದ
ಲಾಭ-ಕ್ಕಾಗಿ
ಲಾಭ-ಗಳನ್ನು
ಲಾರದು
ಲಾರರು
ಲಾರಿರಿ
ಲಾರೆ
ಲಾಲಸೆ
ಲಿಂಗ-ಭೇ-ದ-ಗ-ಳಿಗೆ
ಲಿಗೂ
ಲಿಲ್ಲ
ಲೀಲಾ
ಲೀಲಾ-ವತಿ
ಲೀಲೆ
ಲೆಕ್ಕಾ-ಚಾ-ರ-ಗಳನ್ನು
ಲೆಕ್ಕಾ-ಚಾ-ರ-ವನ್ನು
ಲೆಕ್ಕಿ-ಸ-ದ-ವ-ರನ್ನು
ಲೆಕ್ಕಿ-ಸು-ವು-ದಿಲ್ಲ
ಲೆಕ್ಕಿ-ಸು-ವುದೇ
ಲೆಲ್ಲ
ಲೈಂಗಿ-ಕ-ಶ-ಕ್ತಿ-ಯನ್ನು
ಲೋಕ
ಲೋಕ-ದಲ್ಲಿ
ಲೋಪ-ದೋಷ
ಲೋಪ-ದೋ-ಷ-ಗಳನ್ನು
ಲೋಪ-ದೋ-ಷ-ಗ-ಳಿ-ದ್ದರೂ
ಲೋಪ-ದೋ-ಷ-ಗ-ಳಿ-ದ್ದರೆ
ಲೋಪ-ದೋ-ಷ-ಗ-ಳಿವೆ
ಲೌಕಿಕ
ಲೌಕಿ-ಕ-ವಾದ
ಲ್ಲಾದರೊ
ಲ್ಲಿರಲಿ
ಲ್ಲಿರುವ
ಳಿಗೆ
ಳಿರ-ಬೇ-ಕೆಂದು
ವಂಚ-ಕರೇ
ವಂಚನೆ
ವಂತ-ನೆಂದು
ವಂತ-ರಿ-ಗಿಂತ
ವಂತಹ
ವಂತೆ
ವಂಶಾ-ನು-ಗ-ತ-ವಾಗಿ
ವಜ್ರ-ದಂ-ತಹ
ವತಿ
ವಧಿ
ವನು
ವನೊ
ವನೋ
ವನ್ನಾಗಿ
ವನ್ನಾ-ದರೂ
ವನ್ನು
ವನ್ನೂ
ವನ್ನೆಲ್ಲ
ವಯ-ಸ್ಸಾ-ದಂ-ತೆಲ್ಲ
ವಯ-ಸ್ಸಿ-ನಲ್ಲಿ
ವರ-ದಂತೆ
ವರು
ವರುಷ
ವರು-ಷ-ಗಳ
ವರು-ಷ-ಗಳಲ್ಲಿ
ವರು-ಷ-ಗಳಿಂದ
ವರು-ಷ-ಗ-ಳಿಂ-ದಲೂ
ವರು-ಷ-ಗಳು
ವರೆಂದು
ವರೆಗೆ
ವರ್ಗಕ್ಕೆ
ವರ್ಗ-ದ-ವ-ರಿಗೆ
ವರ್ಗ-ದ-ವರು
ವರ್ಗೀ-ಕ-ರಣ
ವರ್ಣ
ವರ್ಣಕ್ಕೆ
ವರ್ಣ-ಗಳ
ವರ್ಣ-ಗಳನ್ನು
ವರ್ಣ-ಗಳಲ್ಲಿ
ವರ್ಣ-ಗ-ಳಾಗಿ
ವರ್ಣ-ಗಳು
ವರ್ಣ-ಗ-ಳೆಲ್ಲ
ವರ್ಣದ
ವರ್ಣ-ದಲ್ಲಿ
ವರ್ಣ-ದ-ಲ್ಲಿ-ರು-ವ-ವನು
ವರ್ಣ-ದ-ಲ್ಲಿ-ರು-ವ-ವ-ರಿಗೆ
ವರ್ಣ-ದ-ವ-ರಿಗೂ
ವರ್ಣ-ದ-ವ-ರಿಗೆ
ವರ್ಣ-ದ-ವರು
ವರ್ಣ-ದ-ವರೆಲ್ಲ
ವರ್ಣ-ದ-ವ-ರೊ-ಡನೆ
ವರ್ಣ-ದಿಂದ
ವರ್ಣ-ವನ್ನು
ವರ್ಣ-ವಿ-ಲ್ಲದ
ವರ್ಣವು
ವರ್ಣವೂ
ವರ್ಣ-ಸಂಸ್ಥೆ
ವರ್ಣಾ-ತೀ-ತ-ರಾಗಿ
ವರ್ತ-ಮಾನ
ವರ್ತಿ-ಗಳು
ವರ್ಷ-ಗ-ಳಾ-ಯಿತು
ವರ್ಷ-ಗಳಿಂದ
ವರ್ಷ-ಗ-ಳಿಂ-ದಲೂ
ವಲಸೆ
ವಲ್ಲ
ವಳಿ-ಕೆಯೂ
ವವನು
ವವರ
ವಶ-ದ-ಲ್ಲಿ-ರುವ
ವಶ-ಮಾಡಿ
ವಸಂ-ತ-ಪು-ತು-ವಿ-ನಲ್ಲಿ
ವಸ್ತು
ವಸ್ತು-ಗಳ
ವಸ್ತು-ಗಳನ್ನು
ವಸ್ತು-ಗಳು
ವಸ್ತು-ವನ್ನು
ವಸ್ತು-ವಿನ
ವಹಿ-ಸಿ-ಕೊ-ಳ್ಳ-ಬೇ-ಕಾ-ಗಿದೆ
ವಾಗಲಿ
ವಾಗಿ
ವಾಗಿತ್ತು
ವಾಗಿತ್ತೋ
ವಾಗಿದೆ
ವಾಗಿದ್ದು
ವಾಗಿಯೂ
ವಾಗಿ-ರು-ವ-ವನು
ವಾಗಿ-ರು-ವುದು
ವಾಗಿವೆ
ವಾಗು-ತ್ತಿದೆ
ವಾಗು-ವಿರಿ
ವಾಗು-ವು-ದಿಲ್ಲ
ವಾಗು-ವುದು
ವಾಚಾ
ವಾಡು-ವು-ದ-ರಿಂದ
ವಾಣಿ
ವಾಣಿಜ್ಯ
ವಾಣಿಯ
ವಾಣಿ-ಯನ್ನು
ವಾಣಿ-ಯಲ್ಲಿ
ವಾಣಿ-ಯಿಂದ
ವಾಣಿ-ಯೊಂದು
ವಾತಾ-ವ-ರಣ
ವಾತಾ-ವ-ರ-ಣ-ವ-ನ್ನೆಲ್ಲ
ವಾದ
ವಾದ-ಗಳೇ
ವಾದರೂ
ವಾದ-ವಿ-ವಾ-ದ-ಗಳನ್ನು
ವಾದ-ವಿ-ವಾ-ದ-ಗಳಲ್ಲಿ
ವಾದುದು
ವಾಸ
ವಾಸ-ಕ್ಕಾಗಿ
ವಾಸ-ಮಾ-ಡು-ತ್ತಿ-ರುವ
ವಾಸವೇ
ವಾಸಿ-ಸುತ್ತ
ವಾಸಿ-ಸು-ತ್ತಿದ್ದ
ವಿಕ-ವಾಗಿ
ವಿಕಾಸ
ವಿಕಾ-ಸಕ್ಕೆ
ವಿಕಾ-ಸ-ವಾ-ಗ-ಬೇ-ಕಾ-ದರೆ
ವಿಕಾ-ಸ-ವಾ-ಗ-ಬೇಕು
ವಿಕಾ-ಸ-ವಾ-ಗು-ವುದು
ವಿಕಾ-ಸವೂ
ವಿಚಾರ
ವಿಚಿತ್ರ
ವಿಚಿ-ತ್ರ-ವಾಗಿ
ವಿಚಿ-ತ್ರ-ವಾದ
ವಿಜ್ಞಾನ
ವಿಜ್ಞಾ-ನ-ವನ್ನು
ವಿತ್ತು-ಇಡೀ
ವಿದೆ
ವಿದ್ಯಾ
ವಿದ್ಯಾ-ದಾ-ನ-ವನ್ನು
ವಿದ್ಯಾ-ಬು-ದ್ಧಿ-ಗಳನ್ನೆಲ್ಲ
ವಿದ್ಯಾ-ಭ್ಯಾಸ
ವಿದ್ಯಾ-ಭ್ಯಾ-ಸಕ್ಕೆ
ವಿದ್ಯಾ-ಭ್ಯಾ-ಸದ
ವಿದ್ಯಾ-ಭ್ಯಾ-ಸ-ದಲ್ಲಿ
ವಿದ್ಯಾ-ಭ್ಯಾ-ಸ-ದಿಂದ
ವಿದ್ಯಾ-ಭ್ಯಾ-ಸ-ವನ್ನು
ವಿದ್ಯಾ-ಭ್ಯಾ-ಸ-ವಲ್ಲ
ವಿದ್ಯಾ-ಭ್ಯಾ-ಸ-ವಾ-ದರೂ
ವಿದ್ಯಾ-ಭ್ಯಾ-ಸ-ವಾ-ದರೆ
ವಿದ್ಯಾ-ಭ್ಯಾ-ಸವೂ
ವಿದ್ಯಾ-ಭ್ಯಾ-ಸವೆ
ವಿದ್ಯಾ-ಭ್ಯಾ-ಸ-ವೆಂ-ದ-ರೇನು
ವಿದ್ಯಾ-ಭ್ಯಾ-ಸ-ವೆಂದು
ವಿದ್ಯಾ-ಭ್ಯಾ-ಸವೇ
ವಿದ್ಯಾರ್ಥಿ
ವಿದ್ಯಾ-ರ್ಥಿ-ಗಳನ್ನು
ವಿದ್ಯಾ-ರ್ಥಿಗೂ
ವಿದ್ಯಾ-ರ್ಥಿ-ನಿ-ಯ-ರಿಗೆ
ವಿದ್ಯಾ-ರ್ಥಿ-ನಿ-ಯರು
ವಿದ್ಯಾ-ವಂತ
ವಿದ್ಯಾ-ವಂ-ತ-ನಾಗಿ
ವಿದ್ಯಾ-ವಂ-ತ-ನಿ-ರು-ವನು
ವಿದ್ಯಾ-ವಂ-ತನೋ
ವಿದ್ಯಾ-ವಂ-ತರ
ವಿದ್ಯಾ-ವಂ-ತ-ರಾ-ಗ-ಬಾ-ರದು
ವಿದ್ಯಾ-ವಂ-ತ-ರಾಗಿ
ವಿದ್ಯಾ-ವಂ-ತ-ರಾ-ಗುವು
ವಿದ್ಯಾ-ವಂ-ತ-ರಾದ
ವಿದ್ಯಾ-ವಂ-ತ-ರಿ-ಗಿಂತ
ವಿದ್ಯಾ-ವಂ-ತರು
ವಿದ್ಯು-ಚ-ಊ-ಇ-್ಕ-ದ್ಕ್ತಿ-ಯನ್ನು
ವಿದ್ಯು-ತ್ಶ-ಕ್ತಿ-ಯನ್ನು
ವಿದ್ಯೆ
ವಿದ್ಯೆಗೆ
ವಿದ್ಯೆ-ಯನ್ನು
ವಿದ್ಯೆ-ಯಿಂದ
ವಿದ್ಯೆಯು
ವಿದ್ವತ್
ವಿಧ-ದ-ಲ್ಲಿಯೂ
ವಿಧವಾ
ವಿಧ-ವಾದ
ವಿಧವೆ
ವಿಧ-ವೆ-ಯರ
ವಿಧ-ವೆ-ಯ-ರಿಗೆ
ವಿಧ-ವೆ-ಯರು
ವಿಧ-ವೆಯೆ
ವಿಧಿ
ವಿಧಿ-ನಿ-ಯ-ಮ-ಗಳನ್ನು
ವಿಧಿಯ
ವಿಧಿ-ಸ-ಲಾರೆ
ವಿಧಿ-ಸಿ-ರು-ವ-ರು-ಅ-ದಾ-ವು-ದೆಂ-ದರೆ
ವಿಧೇ-ಯತೆ
ವಿಧೇ-ಯ-ತೆಯ
ವಿಧೇ-ಯ-ತೆ-ಯನ್ನು
ವಿನ
ವಿನಃ
ವಿನಾ
ವಿನಾ-ಯಿತಿ
ವಿನಾ-ಯಿ-ತಿ-ಗ-ಳಿವೆ
ವಿಫ-ಲ-ರಾ-ದರು
ವಿಫ-ಲ-ವಾ-ಗು-ವು-ದಿಲ್ಲ
ವಿಭ-ಜನೆ
ವಿಭಾಗ
ವಿಭಾ-ಗದ
ವಿಮ-ರ್ಶಾ-ತ್ಮಕ
ವಿಮೋ-ಚನೆ
ವಿಮೋ-ಚ-ನೆಯ
ವಿರಾ
ವಿರು-ದ್ಧ-ವಾದ
ವಿರು-ದ್ಧ-ವಾ-ದುದು
ವಿರೇನು
ವಿರೋಧ
ವಿರೋ-ಧ-ವಾಗಿ
ವಿರೋ-ಧ-ವಾ-ಗಿದೆ
ವಿರೋ-ಧ-ವಾ-ಗಿಯೇ
ವಿರೋ-ಧವೂ
ವಿರೋಧಿ
ವಿರೋ-ಧಿ-ಸಿ-ದರೂ
ವಿರೋ-ಧಿ-ಸಿ-ದ-ವ-ರಿಗೆ
ವಿಲ್ಲ
ವಿಲ್ಲದೆ
ವಿಲ್ಲ-ವೆಂ-ಬು-ದನ್ನು
ವಿಲ್ಲವೇ
ವಿವ-ರ-ಣೆ-ಗಳ
ವಿವ-ರ-ಣೆ-ಗಳನ್ನು
ವಿವ-ರ-ಣೆ-ಯನ್ನು
ವಿವ-ರಿಸಿ
ವಿವ-ರಿ-ಸು-ವರು
ವಿವಾ-ದ-ಗಳಲ್ಲಿ
ವಿವಾಹ
ವಿವಾ-ಹ-ದಲ್ಲಿ
ವಿವಿ-ಧ-ವ-ರ್ಣ-ಗ-ಳಾಗಿ
ವಿವೇಕ
ವಿವೇಕಾನಂದ
ವಿವೇಕಾನಂದರ
ವಿವೇಕಾನಂದರು
ವಿಶಾಲ
ವಿಶಾ-ಲ-ಗೊ-ಳಿ-ಸ-ಬೇಕು
ವಿಶಾ-ಲ-ವಾ-ಗ-ಬೇಕು
ವಿಶಾ-ಲ-ವಾ-ಗಿ-ರ-ಬೇಕು
ವಿಶಾ-ಲ-ವಾದ
ವಿಶೇಷ
ವಿಶೇ-ಷ-ವಾಗಿ
ವಿಶ್ವ-ಕೋ-ಶ-ಗಳೇ
ವಿಶ್ವಕ್ಕೆ
ವಿಶ್ವದ
ವಿಶ್ವ-ಪ-ರ್ಯಂತ
ವಿಶ್ವ-ರ-ಹ-ಸ್ಯ-ವನ್ನು
ವಿಶ್ವ-ವನ್ನು
ವಿಶ್ವ-ಸ-ಮ-ನ್ವಯ
ವಿಶ್ವಾ-ನು-ಕಂಪ
ವಿಶ್ವಾ-ಮಿತ್ರ
ವಿಶ್ವಾ-ಸ-ವನ್ನು
ವಿಷ-ಕ್ರಿ-ಮಿ-ಗ-ಳಿ-ದ್ದರೂ
ವಿಷ-ಕ್ರಿ-ಮಿಯೂ
ವಿಷ-ದಂತೆ
ವಿಷಮ
ವಿಷಮೇ
ವಿಷಯ
ವಿಷ-ಯ-ಗಳ
ವಿಷ-ಯ-ಗಳನ್ನು
ವಿಷ-ಯ-ಗಳಲ್ಲಿ
ವಿಷ-ಯ-ಗಳಿಂದ
ವಿಷ-ಯದ
ವಿಷ-ಯ-ದಲ್ಲಾ
ವಿಷ-ಯ-ದಲ್ಲಿ
ವಿಷ-ಯ-ವನ್ನು
ವಿಷ-ಯ-ವ-ಸ್ತು-ಗಳನ್ನು
ವಿಷ-ಯ-ವಾಗಿ
ವಿಷ-ವನ್ನು
ವಿಷಾ-ದ-ದಲ್ಲಿ
ವಿಷ್ಣು-ಗ-ಳಿಗೇ
ವಿಸ-ರ್ಜಿ-ಸು-ವು-ದಕ್ಕೆ
ವಿಸ್ತ-ರಣೆ
ವಿಸ್ತಾ-ರ-ವಾಗಿ
ವಿಸ್ತಾ-ರ-ವಾ-ಗು-ವುದು
ವಿಹ್ವ-ಲ-ರಾ-ಗಿ-ರು-ವಿರಾ
ವೀರ
ವೀರರ
ವೀರ-ರನ್ನು
ವೀರ-ರಾಗಿ
ವೀರರು
ವುದ-ಕ್ಕಿಂತ
ವುದಕ್ಕೆ
ವುದನ್ನು
ವುದ-ನ್ನೆಲ್ಲ
ವುದ-ರಲ್ಲಿ
ವುದ-ರ-ಲ್ಲಿಯೇ
ವುದಾ-ಗಿ-ದೆಯೊ
ವುದಿಲ್ಲ
ವುದು
ವುದೇ
ವುವು
ವೃಕ್ಷ
ವೃಕ್ಷ-ವಾ-ಗ-ಬ-ಹುದು
ವೃತ್ತ-ಪ-ತ್ರಿ-ಕೆ-ಯಲ್ಲಿ
ವೃತ್ತಿ-ಯನ್ನು
ವೃತ್ತಿ-ಯ-ಲ್ಲಾ-ಗಲಿ
ವೃಥಾ
ವೃದ್ಧಿ
ವೃದ್ಧಿ-ಮಾಡ
ವೃದ್ಧಿ-ಮಾಡಿ
ವೃದ್ಧಿ-ಮಾ-ಡಿ-ಕೊಂ-ಡಂತೆ
ವೃದ್ಧಿ-ಯಾ-ಗ-ಲಾ-ರದು
ವೃದ್ಧಿ-ಯಾ-ಗು-ವುದೊ
ವೆಂದರೆ
ವೆತ್ತಿ-ರು-ವರು
ವೆನಿಸ್
ವೆಲ್ಲ
ವೆವು
ವೆವೇ
ವೆವೊ
ವೇಗ-ದಿಂದ
ವೇದ
ವೇದ-ಗಳ
ವೇದ-ಗಳನ್ನು
ವೇದ-ಗಳಲ್ಲಿ
ವೇದ-ಗಳೇ
ವೇದ-ವನ್ನು
ವೇದ-ಶಾ-ಸ್ತ್ರ-ದಲ್ಲಿ
ವೇದಾಂತ
ವೇದಾಂ-ತದ
ವೇದಾಂ-ತ-ದೊಂ-ದಿಗೆ
ವೇದಾಂ-ತ-ವನ್ನು
ವೇದಾಂ-ತ-ವಾ-ದರೂ
ವೇದಾಂ-ತವೇ
ವೇದಾಂ-ತಿ-ಗಳು
ವೇದಾಂ-ತಿ-ಗಳೂ
ವೇದಾ-ಧ್ಯ-ಯ-ನಕ್ಕೆ
ವೇದಿ-ಕೆಯ
ವೇನೆಂ-ಬುದು
ವೇಳೆ
ವೇಷ-ಭೂ-ಷ-ಣ-ಗಳನ್ನು
ವೇಷ-ಭೂ-ಷ-ಣ-ಗಳಿಂದ
ವೇಷ-ಭೂ-ಷ-ಣ-ಗ-ಳಿ-ಗಾಗಿ
ವೈಜ್ಞಾ-ನಿಕ
ವೈದಿ-ಕರೂ
ವೈಭ-ವ-ವನ್ನು
ವೈಯ್ಯಾರ
ವೈರಾಗ್ಯ
ವೈರಾ-ಗ್ಯದ
ವೈರಿ
ವೈವಿ-ಧ್ಯ-ತೆ-ಗಳಿಂದ
ವೈಶಾ-ಲ್ಯತೆ
ವೈಶ್ಯ
ವೊಂದು
ವ್ಯಕ್ತ
ವ್ಯಕ್ತ-ಗೊ-ಳಿ-ಸದೆ
ವ್ಯಕ್ತ-ಗೊ-ಳಿ-ಸಲಿ
ವ್ಯಕ್ತ-ಗೊ-ಳಿ-ಸಲು
ವ್ಯಕ್ತ-ಗೊ-ಳಿಸಿ
ವ್ಯಕ್ತ-ಗೊ-ಳಿ-ಸಿ-ರು-ವರು
ವ್ಯಕ್ತ-ಪ-ಡಿ-ಸಲು
ವ್ಯಕ್ತ-ಪ-ಡಿ-ಸಿ-ದಿರೋ
ವ್ಯಕ್ತ-ಪ-ಡಿ-ಸುವ
ವ್ಯಕ್ತ-ಪ-ಡಿ-ಸು-ವುದು
ವ್ಯಕ್ತ-ವಾ-ಗ-ತೊ-ಡ-ಗಿ-ದನು
ವ್ಯಕ್ತ-ವಾ-ಗಿದೆ
ವ್ಯಕ್ತ-ವಾ-ಗಿವೆ
ವ್ಯಕ್ತ-ವಾಗು
ವ್ಯಕ್ತ-ವಾ-ಗು-ತ್ತದೆ
ವ್ಯಕ್ತ-ವಾ-ಗು-ತ್ತಿ-ರ-ಬೇಕು
ವ್ಯಕ್ತ-ವಾ-ಗು-ವಂತೆ
ವ್ಯಕ್ತ-ವಾ-ಗು-ವು-ದ-ಕ್ಕಾಗಿ
ವ್ಯಕ್ತ-ವಾ-ಗು-ವುದನ್ನು
ವ್ಯಕ್ತ-ವಾ-ಗು-ವುದು
ವ್ಯಕ್ತ-ವಾ-ಗು-ವುದೊ
ವ್ಯಕ್ತ-ವಾ-ಯಿತು
ವ್ಯಕ್ತಿ
ವ್ಯಕ್ತಿ-ಗಳ
ವ್ಯಕ್ತಿ-ಗ-ಳಂತೆ
ವ್ಯಕ್ತಿ-ಗಳನ್ನು
ವ್ಯಕ್ತಿ-ಗಳಲ್ಲಿ
ವ್ಯಕ್ತಿ-ಗ-ಳಾ-ಗ-ಬೇಕು
ವ್ಯಕ್ತಿ-ಗಳಿ
ವ್ಯಕ್ತಿ-ಗ-ಳಿಗೆ
ವ್ಯಕ್ತಿ-ಗ-ಳಿ-ದ್ದರೂ
ವ್ಯಕ್ತಿ-ಗಳು
ವ್ಯಕ್ತಿ-ಗ-ಳೆಲ್ಲಿ
ವ್ಯಕ್ತಿಗೆ
ವ್ಯಕ್ತಿತ್ವ
ವ್ಯಕ್ತಿ-ತ್ವ-ವನ್ನು
ವ್ಯಕ್ತಿ-ತ್ವ-ವ-ನ್ನೆಲ್ಲ
ವ್ಯಕ್ತಿ-ತ್ವ-ವಿದೆ
ವ್ಯಕ್ತಿಯ
ವ್ಯಕ್ತಿ-ಯ-ನ್ನಾಗಿ
ವ್ಯಕ್ತಿ-ಯ-ಲ್ಲಿ-ರುವ
ವ್ಯಕ್ತಿ-ಯಾ-ಗಲಿ
ವ್ಯಕ್ತಿ-ಯಾ-ಗಿಲ್ಲ
ವ್ಯಕ್ತಿ-ಯಿಂದ
ವ್ಯಕ್ತಿಯು
ವ್ಯಕ್ತಿಯೂ
ವ್ಯಕ್ತಿಯೇ
ವ್ಯತಿ-ರಿ-ಕ್ತ-ವಾಗಿ
ವ್ಯತ್ಯಾಸ
ವ್ಯತ್ಯಾ-ಸ-ಕ್ಕೆಲ್ಲ
ವ್ಯತ್ಯಾ-ಸ-ಗಳಿ
ವ್ಯತ್ಯಾ-ಸ-ವನ್ನು
ವ್ಯತ್ಯಾ-ಸ-ವಾ-ಗಿ-ರು-ವುದು
ವ್ಯತ್ಯಾ-ಸವೂ
ವ್ಯಥೆ
ವ್ಯಥೆ-ಯಾ-ಗು-ವುದು
ವ್ಯಯ
ವ್ಯರ್ಥ
ವ್ಯರ್ಥ-ವಾ-ಗು-ವು-ದಿಲ್ಲ
ವ್ಯರ್ಥ-ವಾ-ಗು-ವುದು
ವ್ಯವ-ಸಾಯ
ವ್ಯವ-ಸ್ಥಿತ
ವ್ಯವಸ್ಥೆ
ವ್ಯವ-ಸ್ಥೆ-ಗಳ
ವ್ಯವ-ಸ್ಥೆಯ
ವ್ಯವ-ಹಾರ
ವ್ಯವ-ಹಾ-ರ-ಗಳನ್ನು
ವ್ಯವ-ಹಾ-ರ-ಗಳಲ್ಲಿ
ವ್ಯವ-ಹಾ-ರ-ಗಳು
ವ್ಯವ-ಹಾ-ರ-ಗ-ಳೆಲ್ಲ
ವ್ಯವ-ಹಾ-ರ-ದಲ್ಲಿ
ವ್ಯವ-ಹಾ-ರವಾ
ವ್ಯವ-ಹಾ-ರವು
ವ್ಯವ-ಹಾ-ರ-ವೆಲ್ಲ
ವ್ಯಸ್ಥ-ರಾ-ಗಿ-ರು-ವಿರಾ
ವ್ಯಾಕ-ರಣ
ವ್ಯಾಕು-ಲತೆ
ವ್ಯಾಕು-ಲ-ವಾ-ಗಿ-ದೆಯೆ
ವ್ಯಾಧಿಯ
ವ್ಯಾಪಾರ
ವ್ಯಾಪಾ-ರ-ದಿಂದ
ವ್ಯಾಪಾ-ರೋ-ದ್ಯ-ಮ-ಗಳು
ವ್ಯಾಪಿ-ಸ-ಬೇಕು
ವ್ಯಾಪಿಸಿ
ವ್ಯಾಪಿ-ಸಿ-ಕೊಂ-ಡಿ-ರು-ವುದೆ
ವ್ಯಾಪಿ-ಸಿ-ಕೊಂಡು
ವ್ಯಾಪಿ-ಸಿ-ಕೊ-ಳ್ಳುವ
ವ್ಯಾಪಿ-ಸಿದೆ
ವ್ಯಾಪಿ-ಸಿ-ದೆಯೆ
ವ್ಯಾಪಿ-ಸಿ-ರು-ವುದು
ವ್ಯಾವ-ಹಾ-ರಿಕ
ವ್ರತ-ವನ್ನು
ವ್ರತಾ-ದಿ-ಗಳನ್ನು
ಶಂಕ-ರಾ-ಚಾ-ರ್ಯರು
ಶಂಕು-ತ-ಳೋ-ಪಾ-ಖ್ಯಾ-ನ-ವನ್ನು
ಶಕ್ತಿ
ಶಕ್ತಿಗೆ
ಶಕ್ತಿಯ
ಶಕ್ತಿ-ಯ-ನ್ನಾಗಿ
ಶಕ್ತಿ-ಯನ್ನು
ಶಕ್ತಿ-ಯ-ನ್ನೆಲ್ಲಾ
ಶಕ್ತಿ-ಯಲ್ಲಿ
ಶಕ್ತಿಯೂ
ಶಕ್ತಿಯೇ
ಶಕ್ತಿ-ವ-ರ್ಧಕ
ಶತ
ಶತ-ಮಾನ
ಶತ-ಮಾ-ನ-ಗಳ
ಶತ-ಮಾ-ನ-ಗ-ಳ-ವ-ರೆಗೆ
ಶತ-ಮಾ-ನ-ಗ-ಳಾದ
ಶತ-ಮಾ-ನ-ಗಳಿಂದ
ಶತ-ಮಾ-ನ-ಗ-ಳಿಂ-ದಲೂ
ಶತ-ಮಾ-ನ-ಗಳು
ಶತ-ಶ-ತ-ಮಾ-ನ-ಗಳಿಂದ
ಶತ್ರು
ಶತ್ರು-ತಾ-ಪ-ನನೆ
ಶಮ-ನಕ್ಕೆ
ಶಮ-ನ-ಗೊ-ಳಿ-ಸುವ
ಶರೀ-ರ-ವಿಂದು
ಶವ
ಶವ-ದಂತೆ
ಶಸ್ತ್ರ-ಚಿ-ಕಿ-ತ್ಸ-ಕನ
ಶಾಂತ-ವಾಗಿ
ಶಾಂತಿ
ಶಾಂತಿ-ಯನ್ನು
ಶಾಖ
ಶಾಖೆ-ಗಳ
ಶಾಪ-ವನ್ನು
ಶಾರೀ-ರಿಕ
ಶಾಲಾ-ಕಾ-ಲೇಜು
ಶಾಲೆ
ಶಾಲೆ-ಗ-ಳಿಗೆ
ಶಾಲೆಗೆ
ಶಾಲೆಯ
ಶಾಲೆ-ಯಲ್ಲಿ
ಶಾಲೆ-ಯೊ-ಡನೆ
ಶಾಶ್ವತ
ಶಾಶ್ವ-ತ-ವಾದ
ಶಾಸನ
ಶಾಸ-ನಕ್ಕೆ
ಶಾಸ-ನ-ಗ-ಣದ
ಶಾಸ-ನ-ಗ-ಳಿಗೂ
ಶಾಸ-ನ-ಗಳು
ಶಾಸ-ನ-ವಲ್ಲ
ಶಾಸ-ನವೂ
ಶಾಸ್ತ್ರ
ಶಾಸ್ತ್ರ-ಗಳ
ಶಾಸ್ತ್ರ-ಗಳನ್ನು
ಶಾಸ್ತ್ರ-ಗಳಲ್ಲಿ
ಶಾಸ್ತ್ರ-ಗಳು
ಶಾಸ್ತ್ರ-ಗ-ಳೆಲ್ಲ
ಶಾಸ್ತ್ರಜ್ಞ
ಶಾಸ್ತ್ರದ
ಶಾಸ್ತ್ರ-ದಲ್ಲಿ
ಶಾಸ್ತ್ರ-ದಲ್ಲೆಲ್ಲ
ಶಾಸ್ತ್ರ-ವನ್ನು
ಶಿಕ್ಷಣ
ಶಿಕ್ಷ-ಣದ
ಶಿಕ್ಷ-ಣ-ದಿಂದ
ಶಿಕ್ಷ-ಣ-ಪಂ-ಡಿ-ತರು
ಶಿಕ್ಷ-ಣ-ವನ್ನು
ಶಿಖ-ರ-ಗಳ
ಶಿಖ-ರ-ಗಳಿಂದ
ಶಿಥಿ-ಲ-ಗೊ-ಳಿ-ಸಿದೆ
ಶಿರೋ-ಮ-ಣಿ-ಗಳ
ಶಿರೋ-ಮ-ಣಿಯೇ
ಶಿಷ್ಯ
ಶಿಷ್ಯನ
ಶಿಷ್ಯ-ನಲ್ಲಿ
ಶಿಷ್ಯ-ನಿಂದ
ಶಿಷ್ಯ-ನಿಗೆ
ಶಿಷ್ಯ-ನಿ-ರುವ
ಶಿಷ್ಯ-ರಾ-ಗ-ಬೇ-ಕಾ-ಗಿದೆ
ಶಿಷ್ಯ-ರಿಗೆ
ಶಿಷ್ಯರು
ಶೀಘ್ರ
ಶೀಲ
ಶೀಲ-ವನ್ನು
ಶೀಲವೂ
ಶುದ್ಧ
ಶುದ್ಧ-ಗುಣ
ಶುದ್ಧ-ಚಾ-ರಿತ್ರ್ಯ
ಶುದ್ಧತೆ
ಶುದ್ಧ-ರಾದ
ಶುದ್ಧ-ವಾ-ಗು-ವುದು
ಶುಭ್ರ-ವಾ-ಗಿ-ರು-ವರು
ಶೂದ್ರ
ಶೂದ್ರ-ನಾ-ಗುವ
ಶೂರಾ-ಧಿ-ಶೂ-ರ-ರನ್ನು
ಶೃತ-ಪ-ಡಿ-ಸಿ-ರು-ವರು
ಶೋಕ-ತಾ-ಪ-ಗಳು
ಶೋಕ-ಪಾ-ತ-ಗಳು
ಶೋಚ-ನೀಯ
ಶ್ರದ್ಧಾ
ಶ್ರದ್ಧಾ-ನ್ವಿ-ತ-ರಾದ
ಶ್ರದ್ಧಾ-ಭ-ಕ್ತಿ-ಗಳು
ಶ್ರದ್ಧಾ-ವಂತ
ಶ್ರದ್ಧಾ-ವಂ-ತ-ರಾಗಿ
ಶ್ರದ್ಧಾ-ವಂ-ತ-ರಾದ
ಶ್ರದ್ಧಾ-ವಿ-ಹೀ-ನ-ರಾ-ಗು-ತ್ತಿ-ರು-ವರು
ಶ್ರದ್ಧೆ
ಶ್ರದ್ಧೆಯ
ಶ್ರದ್ಧೆ-ಯನ್ನು
ಶ್ರದ್ಧೆ-ಯಲ್ಲಿ
ಶ್ರದ್ಧೆ-ಯಿಲ್ಲ
ಶ್ರಮ
ಶ್ರಮಕ್ಕೆ
ಶ್ರಮ-ಜೀ-ವಿ-ಗಳ
ಶ್ರಮ-ಜೀ-ವಿ-ಗಳೆ
ಶ್ರಮದ
ಶ್ರಮವೂ
ಶ್ರೀಕೃಷ್ಣ
ಶ್ರೀಕೃ-ಷ್ಣನ
ಶ್ರೀಕೃ-ಷ್ಣನು
ಶ್ರೀಮಂ-ತ-ನಾಗಿ
ಶ್ರೀಮಂ-ತರು
ಶ್ರೀರಾಮ
ಶ್ರೀರಾ-ಮ-ಕೃಷ್ಣ
ಶ್ರೀರಾ-ಮ-ಕೃ-ಷ್ಣರ
ಶ್ರೇಣಿ-ಶ್ರೇ-ಣಿ-ಯಾಗಿ
ಶ್ರೇಯ-ಸ್ಕ-ರ-ವಲ್ಲ
ಶ್ರೇಯ-ಸ್ಕ-ರ-ವಾ-ಗು-ವುದು
ಶ್ರೇಯ-ಸ್ಕ-ರ-ವಾದ
ಶ್ರೇಯ-ಸ್ಸನ್ನು
ಶ್ರೇಯ-ಸ್ಸಾ-ಗ-ಲಾ-ರದು
ಶ್ರೇಷ್ಠ
ಶ್ರೇಷ್ಠ-ತಮ
ಶ್ರೇಷ್ಠ-ತ-ಮ-ವಾದ
ಶ್ರೇಷ್ಠ-ವ-ರ್ಗಕ್ಕೆ
ಶ್ರೇಷ್ಠ-ವಾಗಿ
ಶ್ರೇಷ್ಠ-ವಾದ
ಶ್ಲಾಘಿಸಿ
ಶ್ಲೋಕ
ಶ್ವರನೇ
ಷಂಡ
ಷಂಡ-ತ-ನ-ವನ್ನು
ಷ್ಠಾನಕ್ಕೆ
ಷ್ಯರು
ಸಂಕಟ
ಸಂಕ-ಟ-ಗಳು
ಸಂಕು-ಚಿ-ತ-ಮಾ-ಡು-ವು-ದಕ್ಕೆ
ಸಂಕು-ಚಿ-ತ-ವಾ-ಗು-ವು-ದಕ್ಕೆ
ಸಂಕೋಚ
ಸಂಕೋ-ಚ-ಮಾ-ಡಿ-ಕೊಂ-ಡದ್ದು
ಸಂಕೋ-ಚ-ವಾ-ಯಿತು
ಸಂಕೋ-ಚವೇ
ಸಂಖ್ಯೆ
ಸಂಖ್ಯೆಯ
ಸಂಗ-ಮದ
ಸಂಗ-ಮಿತ್ರ
ಸಂಗ್ರಹ
ಸಂಗ್ರ-ಹ-ದ-ಲ್ಲಿದೆ
ಸಂಗ್ರ-ಹ-ವಲ್ಲ
ಸಂಗ್ರ-ಹವೇ
ಸಂಗ್ರ-ಹಿ-ಸ-ಬ-ಹುದು
ಸಂಗ್ರ-ಹಿ-ಸ-ಬೇಕು
ಸಂಗ್ರ-ಹಿಸಿ
ಸಂಗ್ರ-ಹಿ-ಸಿಟ್ಟ
ಸಂಗ್ರ-ಹಿ-ಸಿ-ಟ್ಟಿ-ರುವ
ಸಂಗ್ರ-ಹಿ-ಸಿ-ಟ್ಟು-ಕೊಂ-ಡಿ-ರುವ
ಸಂಗ್ರ-ಹಿ-ಸಿದ್ದು
ಸಂಗ್ರ-ಹಿ-ಸು-ವು-ದರ
ಸಂಗ್ರ-ಹಿ-ಸು-ವು-ದಲ್ಲ
ಸಂಗ್ರ-ಹಿ-ಸು-ವು-ದಾ-ಗಿದೆ
ಸಂಗ್ರ-ಹಿ-ಸು-ವೆನು
ಸಂಘ-ದಿಂದ
ಸಂಘ-ವನ್ನು
ಸಂಘ-ಶ-ಕ್ತಿಯ
ಸಂಚಯ
ಸಂಚ-ರಿಸಿ
ಸಂಚ-ರಿಸು
ಸಂಚಾ-ರ-ವಾ-ಗು-ವುದು
ಸಂಜೀ-ವಿನಿ
ಸಂತ-ತಿ-ಯ-ವ-ರನ್ನು
ಸಂತಾ-ನರು
ಸಂತಾ-ನರೆ
ಸಂತೋಷ
ಸಂತೋ-ಷ-ದಿಂದ
ಸಂತೋ-ಷ-ಪ-ಡಿ-ಸ-ಬೇ-ಕಾ-ಗಿದೆ
ಸಂತೋ-ಷ-ಪ-ಡು-ತ್ತಿದ್ದೆ
ಸಂತೋ-ಷ-ವಾ-ಗು-ವುದು
ಸಂತೋ-ಷವು
ಸಂದೇಶ
ಸಂದೇ-ಶ-ವನ್ನು
ಸಂದೇ-ಹ-ವಿಲ್ಲ
ಸಂದೇ-ಹ-ವಿ-ಲ್ಲದೆ
ಸಂದೇ-ಹವೂ
ಸಂಧಾ-ನಕ್ಕೆ
ಸಂಧಿ-ಸು-ತ್ತಿದೆ
ಸಂನ್ಯಾ-ಸಿ-ಗಳು
ಸಂನ್ಯಾ-ಸಿ-ಯರು
ಸಂನ್ಯಾ-ಸಿ-ಯಾ-ಗ-ಬ-ಹುದು
ಸಂಪ-ತ್ತನ್ನು
ಸಂಪ-ತ್ತಿ-ನಲ್ಲಿ
ಸಂಪ-ತ್ತಿ-ಲ್ಲದೆ
ಸಂಪ-ನ್ನ-ರಾಗಿ
ಸಂಪ-ರ್ಕ-ವಿ-ಲ್ಲದೇ
ಸಂಪಾ
ಸಂಪಾ-ದನೆ
ಸಂಪಾ-ದಿ-ಸ-ಬ-ಹುದು
ಸಂಪಾ-ದಿ-ಸಲು
ಸಂಪಾ-ದಿಸಿ
ಸಂಪಾ-ದಿ-ಸಿ-ದರು
ಸಂಪಾ-ದಿ-ಸಿ-ದ-ವನು
ಸಂಪೂರ್ಣ
ಸಂಪೂ-ರ್ಣ-ವಾಗಿ
ಸಂಪ್ರ
ಸಂಪ್ರ-ದಾ-ಯಕ್ಕೆ
ಸಂಪ್ರ-ದಾ-ಯ-ವಂತ
ಸಂಪ್ರ-ದಾ-ಯ-ವಂ-ತ-ನನ್ನು
ಸಂಪ್ರ-ದಾ-ಯ-ವಂ-ತ-ನ-ಲ್ಲಾ-ದರೂ
ಸಂಪ್ರ-ದಾ-ಯ-ವಂ-ತರ
ಸಂಪ್ರ-ದಾ-ಯ-ವಂ-ತ-ರ-ಲ್ಲಾ-ದರೊ
ಸಂಬಂಧ
ಸಂಬಂ-ಧದ
ಸಂಬಂ-ಧ-ದಿಂ-ದಲೇ
ಸಂಬಂ-ಧ-ಪಟ್ಟ
ಸಂಬಂ-ಧ-ಪ-ಟ್ಟಂತೆ
ಸಂಬಂ-ಧ-ಪ-ಟ್ಟ-ದ್ದ-ನ್ನೆಲ್ಲ
ಸಂಬಂ-ಧ-ಪ-ಟ್ಟದ್ದು
ಸಂಬಂ-ಧ-ವನ್ನು
ಸಂಬಂ-ಧವೇ
ಸಂಬಂಧಿ
ಸಂಬ-ಳಕ್ಕೆ
ಸಂಭ-ವ-ವಿಲ್ಲ
ಸಂಭೂ-ತ-ರೆಂದು
ಸಂರ-ಕ್ಷಿಸಿ
ಸಂರ-ಕ್ಷಿ-ಸು-ವನು
ಸಂಶ-ಯವೂ
ಸಂಶೋ-ಧ-ನೆ-ಗಳಲ್ಲಿ
ಸಂಸಾ-ರದ
ಸಂಸ್ಕೃತ
ಸಂಸ್ಕೃ-ತ-ದಲ್ಲಿ
ಸಂಸ್ಕೃ-ತ-ವನ್ನು
ಸಂಸ್ಕೃತಿ
ಸಂಸ್ಕೃ-ತಿಯ
ಸಂಸ್ಕೃ-ತಿ-ಯನ್ನು
ಸಂಸ್ಕೃ-ತಿ-ಯನ್ನೆ
ಸಂಸ್ಥಾ-ಬದ್ಧ
ಸಂಸ್ಥೆ
ಸಂಸ್ಥೆ-ಗಳ
ಸಂಸ್ಥೆ-ಗಳನ್ನು
ಸಂಸ್ಥೆ-ಗಳಲ್ಲಿ
ಸಂಸ್ಥೆ-ಗ-ಳಿ-ಗಾಗಿ
ಸಂಸ್ಥೆ-ಗ-ಳಿಗೆ
ಸಂಸ್ಥೆ-ಗ-ಳಿವೆ
ಸಂಸ್ಥೆ-ಗಳು
ಸಂಸ್ಥೆಗೆ
ಸಂಸ್ಥೆ-ಯ-ನ್ನಲ್ಲ
ಸಂಸ್ಥೆ-ಯನ್ನು
ಸಂಸ್ಥೆ-ಯನ್ನೂ
ಸಂಸ್ಥೆ-ಯಲ್ಲಿ
ಸಂಹ-ರಿ-ಸ-ಲಾ-ರದು
ಸಕಾಲ
ಸಚೇ-ತನ
ಸಚೇ-ತ-ನ-ವಾ-ಗು-ತ್ತಿದೆ
ಸಚ್ಚಿ-ದಾ-ನಂದ
ಸಣ್ಣ
ಸತತ
ಸತ್ಕ-ರ್ಮ-ಗಳಿಂದ
ಸತ್ಕ-ರ್ಮ-ಗ-ಳೆಲ್ಲ
ಸತ್ಕಾ-ರ್ಯ-ವನ್ನು
ಸತ್ತರೆ
ಸತ್ತ್ವ
ಸತ್ಯ
ಸತ್ಯ-ಕಾಮ
ಸತ್ಯ-ಕಾ-ಮ-ನೆಂಬ
ಸತ್ಯ-ಕ್ಕಾಗಿ
ಸತ್ಯ-ಗಳನ್ನು
ಸತ್ಯ-ಗಳು
ಸತ್ಯದ
ಸತ್ಯ-ಯು-ಗ-ದಲ್ಲಿ
ಸತ್ಯ-ವಂತ
ಸತ್ಯ-ವಾದ
ಸತ್ಯವು
ಸತ್ಯ-ವೆಂದು
ಸತ್ಯ-ವೆ-ನಿ-ಸ-ಲಾ-ರದು
ಸತ್ಯವೇ
ಸತ್ಯವೊ
ಸತ್ಯವೋ
ಸತ್ಯಾಂ-ಶ-ವೇನೆಂದರೆ
ಸದಾ
ಸದು-ದ್ದೇಶ
ಸದೃಶ
ಸನಾ-ತನ
ಸನಾ-ತ-ವಾ-ಗಿ-ದೆಯೊ
ಸಫ-ಲ-ಗೊ-ಳಿ-ಸಿ-ಕೊ-ಳ್ಳಲು
ಸಮ-ನಾಗಿ
ಸಮ-ನಾ-ಗು-ವಿರಿ
ಸಮ-ನಾದ
ಸಮ-ನ್ವ-ಯವೇ
ಸಮ-ಯ-ದಲ್ಲಿ
ಸಮ-ಯ-ವಲ್ಲ
ಸಮ-ಯ-ವಾ-ಯಿ-ತೆಂದು
ಸಮರ
ಸಮ-ರಾಂ
ಸಮ-ರ್ಥಿ-ಸು-ವು-ದಕ್ಕೆ
ಸಮ-ಷ್ಟಿಗೆ
ಸಮ-ಷ್ಟಿಯ
ಸಮಸ್ಯೆ
ಸಮ-ಸ್ಯೆ-ಗಳ
ಸಮ-ಸ್ಯೆ-ಗಳನ್ನು
ಸಮ-ಸ್ಯೆ-ಗ-ಳೇನೊ
ಸಮ-ಸ್ಯೆಯ
ಸಮ-ಸ್ಯೆ-ಯನ್ನು
ಸಮ-ಸ್ಯೆಯೇ
ಸಮಾ
ಸಮಾಜ
ಸಮಾ-ಜಕ್ಕೆ
ಸಮಾ-ಜದ
ಸಮಾ-ಜ-ದಲ್ಲಿ
ಸಮಾ-ಜ-ದ-ಲ್ಲಿ-ರುವ
ಸಮಾ-ಜ-ದ-ಲ್ಲಿ-ರು-ವ-ವರೆಲ್ಲ
ಸಮಾ-ಜ-ವನ್ನು
ಸಮಾ-ಜವೂ
ಸಮಾ-ಜವೇ
ಸಮಾ-ಜ-ಸು-ಧಾ-ರ-ಣೆ-ಗಳನ್ನು
ಸಮಾ-ಜ-ಸು-ಧಾ-ರ-ಣೆ-ಯನ್ನು
ಸಮಾ-ಧಾ-ನ-ವನ್ನು
ಸಮಾ-ಧಿಗೆ
ಸಮಾನ
ಸಮಾ-ನ-ತೆ-ಯನ್ನು
ಸಮಾ-ನ-ವಾದ
ಸಮೀ-ಪಕ್ಕೆ
ಸಮೀ-ಪದ
ಸಮೀ-ಪ-ದಲ್ಲಿ
ಸಮೀ-ಪ-ದ-ಲ್ಲಿ-ರು-ವುದು
ಸಮೀ-ಪಿ-ಸಿದೆ
ಸಮೀ-ಪಿ-ಸು-ವಿರಿ
ಸಮೀ-ರ-ಣ-ದಂತೆ
ಸಮೀ-ರ-ದಂತೆ
ಸಮುದ್ರ
ಸಮು-ದ್ರದ
ಸಮು-ದ್ರ-ದಿಂದ
ಸಮು-ದ್ರ-ವ-ನ್ನಾ-ಗಿ-ಮಾಡಿ
ಸಮು-ಪ-ಸ್ಥಿ-ತಮ್
ಸಮೆ-ಸಲು
ಸಮ್ಮನೆ
ಸಮ್ಮಿ-ಲ-ನ-ದಿಂದ
ಸಮ್ಮೋ-ಹ-ನಾ-ಸ್ತ್ರಕ್ಕೆ
ಸಮ್ಮೋ-ಹ-ನಾ-ಸ್ತ್ರದ
ಸರ-ಕಲ್ಲ
ಸರ-ಸ್ವ-ತಿ-ಗಳು
ಸರಿ
ಸರಿ-ದೂ-ಗ-ಲಾ-ರವು
ಸರಿ-ಯ-ದಿರಿ
ಸರಿ-ಯಾಗಿ
ಸರಿ-ಯಾ-ಗಿ-ಡ-ಬೇಕು
ಸರಿ-ಯಾ-ಗಿ-ದ್ದರೆ
ಸರಿ-ಯಾ-ಗಿ-ರು-ವುದನ್ನು
ಸರಿ-ಯಾ-ಗಿ-ರು-ವುದು
ಸರಿ-ಯಾದ
ಸರಿ-ಸ-ಬೇ-ಕಾ-ಗಿದೆ
ಸರಿ-ಸ-ಮ-ನಾಗಿ
ಸರಿ-ಸ-ಮಾನ
ಸರಿ-ಸ-ಮಾ-ನತೆ
ಸರೋ-ವ-ರ-ದಿಂದ
ಸರ್ಕಾರ
ಸರ್ಕಾ-ರಕ್ಕೆ
ಸರ್ಕಾ-ರದ
ಸರ್ಕಾರಿ
ಸರ್ದಾರ್
ಸರ್ವಜ್ಞ
ಸರ್ವ-ಜ್ಞ-ನಾ-ಗು-ವನು
ಸರ್ವ-ತೋ-ಮು-ಖ-ವಾಗಿ
ಸರ್ವ-ತೋ-ಮು-ಖಿ-ಗ-ಳ-ನ್ನಾಗಿ
ಸರ್ವ-ನಾಶ
ಸರ್ವ-ನಾ-ಶದ
ಸರ್ವ-ನಾ-ಶ-ವಾ-ಗು-ವಿರಿ
ಸರ್ವ-ನಾ-ಶ-ವಾ-ದಂ-ತೆಯೆ
ಸರ್ವ-ಪ್ರ-ಯತ್ನ
ಸರ್ವ-ಭೂ-ತೇಷು
ಸರ್ವರೂ
ಸರ್ವವೂ
ಸರ್ವ-ವ್ಯಾಪಿ
ಸರ್ವ-ವ್ಯಾ-ಪಿ-ಯಾ-ದುದು
ಸರ್ವ-ಶ-ಕ್ತ-ನಾದ
ಸರ್ವ-ಶ-ಕ್ತರು
ಸರ್ವ-ಶ-ಕ್ತ-ವಾ-ದುದು
ಸರ್ವ-ಸ-ಮಾ-ನ-ತೆಯ
ಸರ್ವ-ಸ-ಮಾ-ನತ್ವ
ಸರ್ವಸ್ವ
ಸರ್ವ-ಸ್ವ-ವನ್ನು
ಸರ್ವ-ಸ್ವ-ವನ್ನೂ
ಸಲ
ಸಲ-ಕ-ರ-ಣೆ-ಗಳ
ಸಲಹೆ
ಸಲಾ-ಕೆ-ಯನ್ನು
ಸಲಿಯೇ
ಸಲ್ಲುವ
ಸವಾಲು
ಸಸಿಗೆ
ಸಹ-ಕ-ರಿ-ಸ-ಲಾ-ರೆವು
ಸಹ-ಕಾರ
ಸಹ-ಜ-ವಾ-ಗಿ-ರು-ವುದು
ಸಹ-ಜವೇ
ಸಹ-ನೆ-ಯಿಂದ
ಸಹಾನು
ಸಹಾ-ನು-ಭೂತಿ
ಸಹಾ-ನು-ಭೂ-ತಿ-ಯನ್ನು
ಸಹಾಯ
ಸಹಾ-ಯ-ಕ-ರಾಗಿ
ಸಹಾ-ಯ-ಕ-ವಾ-ಗಲಿ
ಸಹಾ-ಯ-ಕ-ವಾಗಿ
ಸಹಾ-ಯ-ಕ-ವಾದ
ಸಹಾ-ಯಕ್ಕೆ
ಸಹಾ-ಯ-ದಿಂ-ದಲೂ
ಸಹಾ-ಯ-ದಿಂ-ದಲೆ
ಸಹಾ-ಯ-ಮಾ-ಡ-ಲಾ-ರದೊ
ಸಹಾ-ಯ-ಮಾ-ಡು-ತ್ತಿದೆ
ಸಹಾ-ಯ-ಮಾ-ಡು-ವು-ದಿಲ್ಲ
ಸಹಾ-ಯ-ಮಾ-ಡು-ವುದು
ಸಹಾ-ಯ-ವನ್ನು
ಸಹಾ-ಯ-ವಾ-ಗು-ವುದು
ಸಹಾ-ಯ-ವಿ-ಲ್ಲದೆ
ಸಹಾ-ಯವೂ
ಸಹಿತ
ಸಹಿಷ್ಣು
ಸಹಿ-ಷ್ಣು-ತಾ-ಮೂರ್ತಿ
ಸಹಿ-ಸಲು
ಸಹಿ-ಸಿ-ರು-ವನು
ಸಹಿ-ಸುವು
ಸಹೃ-ದ-ಯತೆ
ಸಹೋ-ದರ
ಸಹೋ-ದ-ರ-ನಂತೆ
ಸಹೋ-ದ-ರರ
ಸಹೋ-ದ-ರ-ರಿಗೆ
ಸಹೋ-ದ-ರರು
ಸಹೋ-ದ-ರರೆ
ಸಹೋ-ದ-ರ-ರೆಂದು
ಸಾಂಕ್ರಾ-ಮಿಕ
ಸಾಂಸ್ಕೃ-ತಿಕ
ಸಾಕಾ-ಗು-ವ-ಷ್ಟನ್ನು
ಸಾಕಾ-ಗು-ವುದು
ಸಾಕಾ-ದಷ್ಟು
ಸಾಕು
ಸಾಕ್ಷಾತ್
ಸಾಕ್ಷಾ-ತ್ಕಾರ
ಸಾಗಿ-ಬಂದ
ಸಾಗು-ತ್ತಿದೆ
ಸಾತ್ವಿಕ
ಸಾಧ-ನ-ಗಳಿಂದ
ಸಾಧ-ನೆ-ಗಳು
ಸಾಧಾ-ರಣ
ಸಾಧಾ-ರ-ಣರು
ಸಾಧಾ-ರ-ಣ-ವಾಗಿ
ಸಾಧಿಸ
ಸಾಧಿ-ಸದೆ
ಸಾಧಿ-ಸ-ಬ-ಲ್ಲಿರಿ
ಸಾಧಿ-ಸ-ಬೇ-ಕಾ-ಗಿದೆ
ಸಾಧಿ-ಸ-ಬೇ-ಕಾ-ದರೆ
ಸಾಧಿ-ಸ-ಲಾ-ಗು-ವು-ದಿಲ್ಲ
ಸಾಧಿ-ಸ-ಲಾ-ರದು
ಸಾಧಿ-ಸ-ಲಾ-ರೆವು
ಸಾಧಿ-ಸಿದೆ
ಸಾಧಿ-ಸಿಯೇ
ಸಾಧಿ-ಸು-ತ್ತಿವೆ
ಸಾಧಿ-ಸು-ತ್ತೇವೆ
ಸಾಧಿ-ಸು-ವೆವು
ಸಾಧು
ಸಾಧು-ಗಳು
ಸಾಧು-ತ್ವ-ಭಾ-ವದ
ಸಾಧು-ವಾ-ಗಿ-ರುವ
ಸಾಧ್ಯ
ಸಾಧ್ಯ-ತೆ-ಗಳನ್ನು
ಸಾಧ್ಯ-ತೆ-ಗಳೂ
ಸಾಧ್ಯ-ವಾ-ಗ-ಬೇಕು
ಸಾಧ್ಯ-ವಾ-ಗ-ಲಿಲ್ಲ
ಸಾಧ್ಯ-ವಾ-ಗಿತ್ತೋ
ಸಾಧ್ಯ-ವಾ-ಗು-ವಂತೆ
ಸಾಧ್ಯ-ವಾ-ಗುವು
ಸಾಧ್ಯ-ವಾ-ಗು-ವುದು
ಸಾಧ್ಯ-ವಾದ
ಸಾಧ್ಯ-ವಾ-ದರೂ
ಸಾಧ್ಯ-ವಾ-ದರೆ
ಸಾಧ್ಯ-ವಾ-ದಷ್ಟು
ಸಾಧ್ಯ-ವಾ-ದುದು
ಸಾಧ್ಯ-ವಾ-ಯಿತು
ಸಾಧ್ಯ-ವಿ-ದ್ದರೆ
ಸಾಧ್ಯ-ವಿ-ದ್ದಷ್ಟು
ಸಾಧ್ಯ-ವಿಲ್ಲ
ಸಾಧ್ಯವೇ
ಸಾಧ್ವಿ-ಗಳ
ಸಾಮ-ಗ್ರಿ-ಗಳ
ಸಾಮ-ರ್ಖಂಡ
ಸಾಮಾ-ಜಿಕ
ಸಾಮಾ-ನಿ-ಲ್ಲದೆ
ಸಾಮಾನ್ಯ
ಸಾಯ
ಸಾಯ-ಬ-ಹುದು
ಸಾಯಿರಿ
ಸಾಯು
ಸಾಯು-ವನು
ಸಾಯು-ವರು
ಸಾಯು-ವಿರಿ
ಸಾಯು-ವುದು
ಸಾರ
ಸಾರ-ಬಲ್ಲ
ಸಾರ-ವಸ್ತು
ಸಾರವೇ
ಸಾರಿ
ಸಾರಿ-ದರು
ಸಾರಿ-ದರೆ
ಸಾರಿ-ದೆವು
ಸಾರುತ್ತ
ಸಾರು-ತ್ತಾನೆ
ಸಾರು-ತ್ತೇನೆ
ಸಾರುವ
ಸಾರು-ವು-ದಕ್ಕೆ
ಸಾರು-ವುದು
ಸಾಲದು
ಸಾಲಿ
ಸಾಲಿ-ಗ್ರಾ-ಮ-ಗ-ಳಾ-ಗಿವೆ
ಸಾವಿತ್ರಿ
ಸಾವಿ-ತ್ರಿ-ಯರ
ಸಾವಿರ
ಸಾವಿ-ರದ
ಸಾವಿ-ರಾರು
ಸಾವಿಲ್ಲ
ಸಾಹಸ
ಸಾಹ-ಸ-ಗಳನ್ನು
ಸಾಹ-ಸ-ದಲ್ಲಿ
ಸಾಹ-ಸ-ದಿಂದ
ಸಾಹಿತ್ಯ
ಸಿಂಹದ
ಸಿಂಹ-ಪ-ರಾ-ಕ್ರ-ಮ-ವನ್ನು
ಸಿಂಹ-ರ-ನ್ನಾಗಿ
ಸಿಂಹ-ರನ್ನು
ಸಿಂಹ-ವಾ-ಗ-ಲಾ-ರದು
ಸಿಂಹ-ಸ-ದೃಶ
ಸಿಂಹಾ-ಸ-ನದ
ಸಿಕ್ಕ-ಬ-ಲ್ಲವು
ಸಿಕ್ಕ-ಬಾ-ರದು
ಸಿಕ್ಕ-ಬೇ-ಕೆಂದು
ಸಿಕ್ಕಿ
ಸಿಕ್ಕಿ-ದರೂ
ಸಿಕ್ಕಿ-ದರೆ
ಸಿಕ್ಕಿದೆ
ಸಿಕ್ಕಿ-ದ್ದನ್ನು
ಸಿಕ್ಕಿ-ಹೋ-ಗಿ-ದ್ದರು
ಸಿಕ್ಕು
ಸಿಕ್ಕುವ
ಸಿಕ್ಕು-ವರು
ಸಿಕ್ಕು-ವುದು
ಸಿಕ್ಕು-ವು-ದೇನು
ಸಿಕ್ಕೇ
ಸಿಡಿ-ದು-ಬಂದ
ಸಿಡಿ-ಮ-ದ್ದಿ-ನಂತೆ
ಸಿಡಿಲಿ
ಸಿತು
ಸಿದ
ಸಿದುದು
ಸಿದ್ಧ-ನಾ-ಗಿ-ರ-ಬೇಕು
ಸಿದ್ಧ-ನಾ-ಗಿರು
ಸಿದ್ಧ-ನಾ-ಗಿ-ರು-ವನು
ಸಿದ್ಧ-ನಾ-ಗಿ-ರು-ವೆನು
ಸಿದ್ಧ-ರಾ-ಗಿ-ರ-ಬೇಕು
ಸಿದ್ಧ-ರಾ-ಗಿರು
ಸಿದ್ಧ-ರಾ-ಗಿ-ರು-ವರು
ಸಿದ್ಧ-ವಾ-ಗಿ-ರು-ವೆನು
ಸಿದ್ಧ-ವಾ-ಗಿ-ರು-ವೆವು
ಸಿದ್ಧ-ವಾ-ದರೆ
ಸಿದ್ಧಾಂತ
ಸಿದ್ಧಾಂ-ತ-ಗಳನ್ನು
ಸಿದ್ಧಾಂ-ತ-ಗ-ಳಿಗೆ
ಸಿದ್ಧಾಂ-ತ-ಗಳು
ಸಿದ್ಧಾಂ-ತ-ವನ್ನು
ಸಿರು-ವನೊ
ಸಿರು-ವುದು
ಸಿರ್ದಾ-ರನೊ
ಸೀತಾ
ಸೀತಾ-ದೇ-ವಿಯ
ಸೀತೆ
ಸೀತೆಯ
ಸೀತೆ-ಯಂ-ತಾ-ಗಲು
ಸೀತೆ-ಯಂ-ತಾಗು
ಸೀತೆ-ಯನ್ನು
ಸೀತೆಯು
ಸೀಳಿ-ಕೊಂಡು
ಸೀಸರ್
ಸುಂದ-ರ-ವಾಗಿ
ಸುಂದ-ರ-ವಾದ
ಸುಖ
ಸುಖ-ಕ್ಕಿಂತ
ಸುಖಕ್ಕೆ
ಸುಖ-ವನ್ನು
ಸುಖ-ವಾಗಿ
ಸುಖಿ-ಯಾಗಿ
ಸುಖಿ-ಯಾ-ಗಿ-ರ-ಬಲ್ಲ
ಸುಗಮ
ಸುಜ್ಞಾ-ನಿ-ಗಳು
ಸುಡ-ಲಾ-ರದು
ಸುಡು-ತ್ತಿ-ರುವ
ಸುತ್ತ-ಮು-ತ್ತಲು
ಸುತ್ತ-ಲಿ-ರುವ
ಸುತ್ತಲೂ
ಸುತ್ತ-ಲೂ-ಪ್ಯಾಟ್
ಸುತ್ತು-ತ್ತಿ-ರು-ವನು
ಸುದಿನ
ಸುದೀರ್ಘ
ಸುಧಾ
ಸುಧಾ-ರ-ಕ-ರಾ-ಗ-ಬೇ-ಕಾ-ದರೆ
ಸುಧಾ-ರ-ಕರು
ಸುಧಾ-ರ-ಕರೆ
ಸುಧಾ-ರ-ಕ-ರೆಲ್ಲ
ಸುಧಾ-ರಣೆ
ಸುಧಾ-ರ-ಣೆ-ಗಳ
ಸುಧಾ-ರ-ಣೆ-ಗಳನ್ನು
ಸುಧಾ-ರ-ಣೆ-ಗಳು
ಸುಧಾ-ರ-ಣೆಗೆ
ಸುಧಾ-ರ-ಣೆಯ
ಸುಧಾ-ರ-ಣೆ-ಯನ್ನು
ಸುಧಾ-ರ-ಣೆ-ಯಲ್ಲಿ
ಸುಧಾ-ರ-ಣೆಯೂ
ಸುಧೀ-ರ್ಘ-ರಾತ್ರಿ
ಸುಪ್ತ-ವಾ-ಗಿದ್ದ
ಸುಪ್ತ-ವಾ-ಗಿ-ರುವ
ಸುಪ್ತ-ಸ್ಥಿ-ತಿ-ಯಿಂದ
ಸುಪ್ತಾ-ವ-ಸ್ಥೆಗೆ
ಸುಪ್ತಾ-ವ-ಸ್ಥೆಯ
ಸುಪ್ತಾ-ವ-ಸ್ಥೆ-ಯ-ಲ್ಲಿ-ರುವ
ಸುಪ್ರೀ-ತ-ರಾ-ಗು-ವರು
ಸುಮ್ಮ-ನಿರಿ
ಸುಮ್ಮನೆ
ಸುಮ್ಮನೇ
ಸುರ-ಕ್ಷಿ-ತ-ವಾ-ಗಿ-ರು-ವೆವು
ಸುರಿದು
ಸುರಿ-ಸು-ವಂತೆ
ಸುಲ-ಭ-ವಾಗಿ
ಸುಲ-ಭ-ವಾದ
ಸುಲಿದು
ಸುಲಿಯು
ಸುಳಿ-ಯಲ್ಲಿ
ಸುಳಿ-ಯು-ವು-ದಿಲ್ಲ
ಸುಳ್ಳಿನ
ಸುಳ್ಳು
ಸುಳ್ಳು-ಗಾರ
ಸುವ
ಸುವು-ದಕ್ಕೆ
ಸುವು-ದಿಲ್ಲ
ಸುವುದೇ
ಸುಸಂ-ಸ್ಕೃ-ತ-ರಾದ
ಸೂಕ್ತ
ಸೂತಕ
ಸೂತ್ರವೇ
ಸೂರ್ಯ-ಚಂದ್ರ
ಸೃಷ್ಟಿ
ಸೃಷ್ಟಿ-ಕ-ರ್ತನೇ
ಸೃಷ್ಟಿ-ಮಾ-ಡಿತೊ
ಸೃಷ್ಟಿ-ಮಾ-ಡು-ತ್ತಿರು
ಸೃಷ್ಟಿ-ಸ-ಬ-ಲ್ಲೆವು
ಸೃಷ್ಟಿ-ಸ-ಬೇ-ಕಾ-ಗಿದೆ
ಸೃಷ್ಟಿ-ಸ-ಬೇ-ಕಾ-ದರೆ
ಸೃಷ್ಟಿ-ಸಿ-ದರು
ಸೃಷ್ಟಿ-ಸಿ-ರು-ವುದು
ಸೆಮಿ-ಟೆಕ್
ಸೇತು-ವೆ-ಯನ್ನು
ಸೇನಾನಿ
ಸೇನಾ-ಬಲ
ಸೇಬಿನ
ಸೇರ-ಬೇ-ಕಾ-ಗು-ವುದು
ಸೇರ-ಬೇಕು
ಸೇರಿ
ಸೇರಿದ
ಸೇರಿ-ದರು
ಸೇರಿ-ದ-ವರು
ಸೇರಿ-ದ-ವಳು
ಸೇರಿ-ರು-ವು-ದ-ರಿಂದ
ಸೇರಿಲ್ಲ
ಸೇರಿವೆ
ಸೇರಿ-ಸ-ಬೇ-ಕಾ-ಗಿದೆ
ಸೇರಿ-ಸಿ-ಕೊ-ಳ್ಳು-ತ್ತಿ-ದ್ದರೆ
ಸೇರಿ-ಸಿ-ಬಿ-ಟ್ಟಿ-ದ್ದರೂ
ಸೇರು-ವ-ವ-ರೆಗೂ
ಸೇರು-ವುದು
ಸೇವ-ಕ-ರಾಗಿ
ಸೇವಾ
ಸೇವಾ-ಕಾ-ರ್ಯ-ವನ್ನು
ಸೇವಾ-ತ-ತ್ತ್ವ-ಗಳನ್ನು
ಸೇವಾ-ಧ-ರ್ಮ-ವನ್ನು
ಸೇವಾ-ಭಾ-ವ-ನೆ-ಯಿಂದ
ಸೇವಿ-ಸ-ಬೇಕು
ಸೇವಿ-ಸು-ವ-ವನೆ
ಸೇವೆ
ಸೇವೆಗೂ
ಸೇವೆ-ಮಾಡಿ
ಸೇವೆಯ
ಸೈನ್ಯ-ಗಳನ್ನು
ಸೈನ್ಯ-ಗಳು
ಸೈನ್ಯ-ದಿಂದ
ಸೊಂಟ-ಕಟ್ಟಿ
ಸೊಂಟ-ಕ-ಟ್ಟು-ತ್ತೇ-ವೆಯೋ
ಸೊಂಟ-ವನ್ನು
ಸೋಕಿದ
ಸೋಕಿ-ದರೆ
ಸೋಣ
ಸೋತಿ-ರು-ವು-ದ-ರಿಂದ
ಸೋಪಾನ
ಸೋಮ-ನಾ-ಥಾ-ನಂದ
ಸೋಮಾ-ರಿ-ಗ-ಳಲ್ಲ
ಸೋಮಾ-ರಿ-ಗ-ಳಾಗಿ
ಸೋಮಾ-ರಿ-ಗಳು
ಸೋಮಾ-ರಿಗೆ
ಸೋಮಾ-ರಿ-ತನ
ಸೋಮಾ-ರಿ-ತ-ನದ
ಸೋಮಾ-ರಿ-ತ-ನ-ದಿಂದ
ಸೋಮಾ-ರಿಯ
ಸೋಷಿ-ಯಾ-ಲಿಸಂ
ಸೋಽಹಂ
ಸೌಂದರ್ಯ
ಸೌಂದ-ರ್ಯ-ವನ್ನು
ಸೌಕ-ರ್ಯ-ಗಳನ್ನು
ಸೌದೆ-ಯನ್ನು
ಸೌಧ
ಸೌಧವೇ
ಸೌಲ-ಭ್ಯ-ಗಳಲ್ಲಿ
ಸೌಹಾರ್ದ
ಸ್ತ್ರೀ
ಸ್ತ್ರೀಗೂ
ಸ್ತ್ರೀಪು-ರು-ಷ-ರಿಗೂ
ಸ್ತ್ರೀಪು-ರು-ಷರು
ಸ್ತ್ರೀಯ
ಸ್ತ್ರೀಯನ್ನು
ಸ್ತ್ರೀಯರ
ಸ್ತ್ರೀಯ-ರಂತೆ
ಸ್ತ್ರೀಯ-ರ-ಆ-ದರ್ಶ
ಸ್ತ್ರೀಯ-ರನ್ನು
ಸ್ತ್ರೀಯ-ರಲ್ಲಿ
ಸ್ತ್ರೀಯ-ರಷ್ಟೇ
ಸ್ತ್ರೀಯ-ರಿಗೂ
ಸ್ತ್ರೀಯ-ರಿಗೆ
ಸ್ತ್ರೀಯರು
ಸ್ತ್ರೀಯ-ರು-ಗ-ಳಿಗೆ
ಸ್ತ್ರೀಯರೇ
ಸ್ಥಳಕ್ಕೆ
ಸ್ಥಳ-ದಲ್ಲಿ
ಸ್ಥಳ-ದಿಂದ
ಸ್ಥಳ-ವ-ನ್ನಾ-ದರೂ
ಸ್ಥಳ-ವಿಲ್ಲ
ಸ್ಥಳ-ವೊಂ-ದಿ-ದ್ದರೆ
ಸ್ಥಾನ
ಸ್ಥಾನ-ದಲ್ಲಿ
ಸ್ಥಾನ-ದ-ಲ್ಲಿದೆ
ಸ್ಥಾನ-ಮಾ-ನ-ಗಳು
ಸ್ಥಾನ-ವ-ನ್ನಾಗಿ
ಸ್ಥಾನ-ವಿದು
ಸ್ಥಾನವೂ
ಸ್ಥಾಪಿ-ಸು-ತ್ತೇನೆ
ಸ್ಥಿತಿ
ಸ್ಥಿತಿ-ಗ-ಳ-ಲ್ಲಿಯೂ
ಸ್ಥಿತಿಗೆ
ಸ್ಥಿತಿ-ಯನ್ನು
ಸ್ಥಿರ-ವಾ-ಗಿಲ್ಲ
ಸ್ಥೂಲ-ಪ್ರ-ತೀ-ಕ-ದಂತೆ
ಸ್ಥೂಲ-ವಾ-ಗಿ-ರುವ
ಸ್ಥೈರ್ಯ
ಸ್ನೇಹ
ಸ್ನೇಹಿ-ತನೆ
ಸ್ನೇಹಿ-ತ-ರಿಲ್ಲ
ಸ್ಪಂದ-ನ-ದೊಂ-ದಿಗೆ
ಸ್ಪಂದಿ-ಸು-ತ್ತಿ-ದೆಯೆ
ಸ್ಪಂದಿ-ಸು-ತ್ತಿ-ರುವ
ಸ್ಪಂದಿ-ಸು-ವಂತೆ
ಸ್ಪರ್ಧಿ-ಗ-ಳಂತೆ
ಸ್ಪರ್ಧೆ
ಸ್ಪಷ್ಟ-ವಾದ
ಸ್ಪೆಯಿನ್
ಸ್ಫೂರ್ತಿ
ಸ್ಫೂರ್ತಿ-ಗೊಂಡ
ಸ್ಫೂರ್ತಿ-ಯನ್ನು
ಸ್ಮ
ಸ್ಮರಿ-ಸಿ-ಕೊ-ಳ್ಳಲು
ಸ್ಮರಿ-ಸಿ-ಕೊ-ಳ್ಳು-ತ್ತಿ-ರು-ವ-ವ-ರನ್ನು
ಸ್ಮೃತ-ಪು-ಣ್ಯ-ಳಾದ
ಸ್ಮೃತಿ-ಶಾ-ಸ್ತ್ರ-ಗಳನ್ನು
ಸ್ವ
ಸ್ವಂತ
ಸ್ವಚ್ಛಂದ
ಸ್ವತಂ-ತ್ರ-ನಾಗಿ
ಸ್ವತಂ-ತ್ರ-ವಾಗಿ
ಸ್ವಪ್ರ-ಯ-ತ್ನದ
ಸ್ವಭಾವ
ಸ್ವಭಾ-ವ-ಗಳಲ್ಲಿ
ಸ್ವಭಾ-ವತಃ
ಸ್ವಭಾ-ವದ
ಸ್ವಭಾ-ವ-ದಂ-ತೆಯೇ
ಸ್ವಭಾ-ವ-ದ-ವರು
ಸ್ವಭಾ-ವ-ದಿಂದ
ಸ್ವಭಾ-ವ-ವನ್ನು
ಸ್ವಭಾ-ವ-ವಾ-ಗು-ವುದು
ಸ್ವಭಾ-ವವೇ
ಸ್ವರ-ವಿದೆ
ಸ್ವರವೇ
ಸ್ವರೂಪ
ಸ್ವರೂ-ಪಿ-ಣಿ-ಯನ್ನು
ಸ್ವರ್ಗ-ಗ-ತಿಗೆ
ಸ್ವರ್ಗ-ದ-ಲ್ಲಿಯೂ
ಸ್ವರ್ಣ-ಯುಗ
ಸ್ವಲ್ಪ
ಸ್ವಲ್ಪ-ವನ್ನು
ಸ್ವಲ್ಪ-ವಾ-ದರೂ
ಸ್ವಲ್ಪವೂ
ಸ್ವಾಗ-ತಿ-ಸು-ವನು
ಸ್ವಾಗ-ತಿ-ಸು-ವು-ದಕ್ಕೆ
ಸ್ವಾತಂತ್ರ್ಯ
ಸ್ವಾತಂ-ತ್ರ್ಯ-ಪೂ-ರ್ವ-ದಲ್ಲಿ
ಸ್ವಾತಂ-ತ್ರ್ಯ-ವನ್ನು
ಸ್ವಾತಂ-ತ್ರ್ಯ-ವಿ-ದ್ದರೆ
ಸ್ವಾತಂ-ತ್ರ್ಯ-ವಿಲ್ಲ
ಸ್ವಾತಂ-ತ್ರ್ಯವೇ
ಸ್ವಾಧೀ-ನ-ವನ್ನು
ಸ್ವಾಭಾ
ಸ್ವಾಭಾ-ವಿಕ
ಸ್ವಾಭಾ-ವಿ-ಕತೆ
ಸ್ವಾಭಾ-ವಿ-ಕ-ವಾಗಿ
ಸ್ವಾಭಾ-ವಿ-ಕ-ವಾ-ಗಿಯೇ
ಸ್ವಾಭಾ-ವಿ-ಕ-ವಾದ
ಸ್ವಾಮಿ
ಸ್ವಾಮಿ-ಜಿ-ಯ-ವರು
ಸ್ವಾಮೀಜಿ
ಸ್ವಾಮೀ-ಜಿ-ಯ-ವರ
ಸ್ವಾಮೀ-ಜಿ-ಯ-ವ-ರಿಗೆ
ಸ್ವಾಮೀ-ಜಿ-ಯ-ವರು
ಸ್ವಾರ್ಥ-ಕ್ಕಾಗಿ
ಸ್ವಾರ್ಥತೆ
ಸ್ವಾರ್ಥನೂ
ಸ್ವಾರ್ಥ-ಪ-ರರು
ಸ್ವಾರ್ಥ-ರಾದ
ಸ್ವಾರ್ಥ-ವನ್ನು
ಸ್ವಾರ್ಥ-ವ-ನ್ನೆಲ್ಲ
ಸ್ವಾವ-ಲಂ-ಬಿ-ಗಳು
ಸ್ವೀಕ-ರಿ-ಸಲು
ಸ್ವೀಕ-ರಿಸಿ
ಸ್ವೀಕ-ರಿ-ಸು-ತ್ತಿ-ದ್ದರು
ಸ್ವೀಕ-ರಿ-ಸುವ
ಸ್ಸನ್ನು
ಹಂಚ-ಬೇಕು
ಹಂಬ-ಲ-ದಿಂ-ದಲೇ
ಹಂಬ-ಲಿ-ಸುತ್ತ
ಹಕ್ಕಿ-ಗಳ
ಹಕ್ಕಿಗೆ
ಹಕ್ಕಿ-ದ್ದರೆ
ಹಕ್ಕಿ-ಯನ್ನೂ
ಹಕ್ಕು
ಹಕ್ಕು-ಗಳನ್ನು
ಹಕ್ಕು-ಗ-ಳಿಲ್ಲ
ಹಕ್ಕು-ಗಳು
ಹಕ್ಕು-ಗಳೂ
ಹಕ್ಕು-ಬಾ-ಧ್ಯ-ತೆ-ಗಳ
ಹಕ್ಕು-ಬಾ-ಧ್ಯ-ತೆ-ಗಳನ್ನು
ಹಕ್ಕು-ಬಾ-ಧ್ಯ-ತೆ-ಗ-ಳೆಲ್ಲ
ಹಗಲು
ಹಚ್ಚ-ಬೇ-ಕಾ-ಗಿದೆ
ಹಣ
ಹಣ-ತೆ-ಯನ್ನು
ಹಣದ
ಹಣ-ವನ್ನು
ಹಣೆಯ
ಹಣೆ-ಯ-ಬ-ರಹ
ಹಣ್ಣನ್ನು
ಹಣ್ಣು
ಹತೋಟಿ
ಹತ್ತ-ಬೇ-ಕಾ-ದರೂ
ಹತ್ತಿರ
ಹತ್ತಿ-ರವೂ
ಹತ್ತು
ಹದಿ
ಹದಿ-ನಾಲ್ಕು
ಹದಿ-ಮೂ-ರ-ನೆಯ
ಹಬ್ಬಿದ್ದು
ಹಬ್ಬು-ವುದು
ಹರಟೆ
ಹರ-ಟೆ-ಹೊ-ಡೆ-ಯು-ತ್ತೇವೆ
ಹರ-ಡ-ಬೇಕು
ಹರ-ಡಿ-ಲ್ಲದೆ
ಹರ-ಡಿವೆ
ಹರ-ಡು-ವಾಗ
ಹರ-ಡು-ವುದು
ಹರ-ಡು-ವುದೋ
ಹರ-ಳು-ಗ-ಳ-ನ್ನಾಗಿ
ಹರ-ಸ-ಬೇ-ಕಾ-ದರೆ
ಹರಿ-ತ-ವಾಗಿ
ಹರಿದ
ಹರಿದು
ಹರಿ-ದು-ಕೊಂಡು
ಹರಿ-ದು-ಬಂದ
ಹರಿ-ಯ-ಬೇ-ಕೆಂದು
ಹರಿ-ಯ-ಲಾ-ರದು
ಹರಿ-ಯಲು
ಹರಿಯು
ಹರಿ-ಯು-ತ್ತಿದೆ
ಹರಿ-ಯು-ತ್ತಿ-ದ್ದರೆ
ಹರಿ-ಯು-ತ್ತಿ-ರು-ವಳು
ಹರಿ-ಯು-ತ್ತಿವೆ
ಹರಿ-ಯುವ
ಹರಿ-ಯು-ವುದೇ
ಹರಿ-ಸಿ-ಕೊ-ಳ್ಳು-ವರು
ಹಲ-ವ-ದ-ರಂತೆ
ಹಲ-ವ-ರನ್ನು
ಹಲ-ವರು
ಹಲ-ವಾರು
ಹಲವು
ಹಳಿ-ಯ-ಲಿಲ್ಲ
ಹಳೆಯ
ಹಳೆ-ಯ-ದನ್ನು
ಹಳ್ಳಿಗೆ
ಹಳ್ಳಿಯ
ಹಳ್ಳಿ-ಯಿಂದ
ಹವಿ
ಹಸಿದ
ಹಸಿ-ವಿ-ಗಾಗಿ
ಹಸ್ತ-ಲಾ-ಘವ
ಹಾಕದೆ
ಹಾಕ-ಬೇ-ಕಾದ
ಹಾಕ-ಬೇಕು
ಹಾಕ-ಬೇಡಿ
ಹಾಕಿ
ಹಾಕಿ-ಕೊಂಡು
ಹಾಕಿ-ದರು
ಹಾಕು-ತ್ತೀ-ರಲ್ಲ
ಹಾಕು-ತ್ತೀರಿ
ಹಾಕು-ತ್ತೇನೆ
ಹಾಕುವ
ಹಾಕು-ವರು
ಹಾಕು-ವು-ದಕ್ಕೆ
ಹಾಗಾ-ದರೆ
ಹಾಗಿ-ದ್ದರೆ
ಹಾಗಿರ
ಹಾಗೆ
ಹಾಗೆಯೆ
ಹಾಗೆಯೇ
ಹಾಡು-ಗ-ಳೆಂದು
ಹಾದಿ
ಹಾರ-ಲಾ-ರದೊ
ಹಾರಾ-ಡು-ತ್ತಾರೆ
ಹಾರಾ-ಡು-ತ್ತಿದ್ದ
ಹಾರೈ-ಸು-ವೆನು
ಹಾಲೆಂಡ್
ಹಾಳಾ-ಗ-ದಂತೆ
ಹಾಳಾಗು
ಹಾಳು-ಮಾ-ಡ-ಬಾ-ರದು
ಹಾಳು-ಮಾಡು
ಹಾವು
ಹಾಸಿ-ಗೆಯ
ಹಾಸ್ಯಾ-ಸ್ಪದ
ಹಾಸ್ಯಾ-ಸ್ಪ-ದ-ವಾ-ಗಿ-ರು-ವು-ದಿಲ್ಲ
ಹಿಂಡು-ತ್ತಿ-ರು-ವ-ವರ
ಹಿಂತಿ-ರು-ಗ-ಬೇಡಿ
ಹಿಂದಕ್ಕೆ
ಹಿಂದಿ
ಹಿಂದಿ-ಗಿಂತ
ಹಿಂದಿನ
ಹಿಂದಿ-ನಂತೆ
ಹಿಂದಿ-ನ-ಕಾ-ಲ-ದಲ್ಲಿ
ಹಿಂದಿ-ನ-ದನ್ನು
ಹಿಂದಿ-ನ-ದ-ನ್ನೆಲ್ಲ
ಹಿಂದಿ-ನದು
ಹಿಂದಿ-ನ-ವ-ರಾ-ಗಲಿ
ಹಿಂದಿ-ನ-ವರು
ಹಿಂದಿ-ನ-ವೆಲ್ಲ
ಹಿಂದಿ-ನಿಂದ
ಹಿಂದಿ-ನಿಂ-ದಲೂ
ಹಿಂದಿ-ರು-ಗ-ದಿರಿ
ಹಿಂದಿ-ರು-ಗ-ಬೇಡಿ
ಹಿಂದಿ-ರುಗಿ
ಹಿಂದು
ಹಿಂದು-ಗ-ಳಷ್ಟು
ಹಿಂದು-ಗ-ಳಿಗೆ
ಹಿಂದು-ಗಳು
ಹಿಂದು-ಗ-ಳೆಲ್ಲ
ಹಿಂದು-ಗಳೇ
ಹಿಂದು-ವಿ-ನಷ್ಟು
ಹಿಂದೂ
ಹಿಂದೂ-ಗ-ಳ-ಲ್ಲೆಲ್ಲ
ಹಿಂದೂ-ಗ-ಳಾ-ದರೂ
ಹಿಂದೂ-ಗ-ಳಾ-ದು-ದ-ರಿಂದ
ಹಿಂದೂ-ಗಳು
ಹಿಂದೂ-ದೇಶ
ಹಿಂದೂ-ಧರ್ಮ
ಹಿಂದೂ-ಧ-ರ್ಮ-ದಷ್ಟು
ಹಿಂದೂ-ಧ-ರ್ಮ-ವಲ್ಲ
ಹಿಂದೂ-ವಾಗಿ
ಹಿಂದೂ-ವಿ-ನ-ಲ್ಲಿಯೂ
ಹಿಂದೆ
ಹಿಂದೆ-ಗೆ-ಯು-ತ್ತಿ-ರ-ಲಿಲ್ಲ
ಹಿಂದೆಯೂ
ಹಿಂದೆಯೇ
ಹಿಂಬಾ-ಲಿ-ಸಿತು
ಹಿಂಸಿ-ಸ-ಲಾ-ರರು
ಹಿಡಿ
ಹಿಡಿದ
ಹಿಡಿ-ದರು
ಹಿಡಿ-ದರೂ
ಹಿಡಿ-ದರೆ
ಹಿಡಿ-ದಿ-ರು-ವಿರಾ
ಹಿಡಿ-ದಿ-ರು-ವು-ದಲ್ಲ
ಹಿಡಿದು
ಹಿಡಿ-ದು-ಕೊಂ-ಡೆವು
ಹಿಡಿ-ದು-ಕೊಳ್ಳಿ
ಹಿಡಿ-ದು-ಕೊ-ಳ್ಳು-ವು-ದಕ್ಕೆ
ಹಿಡಿ-ಯಿರಿ
ಹಿಡಿ-ಯು-ವು-ದಿಲ್ಲ
ಹಿಡಿ-ಯು-ವುದು
ಹಿತಕ್ಕೆ
ಹಿತ-ದೃ-ಷ್ಟಿ-ಯಿಂ-ದ-ಲಾ-ದರೂ
ಹಿತ-ರ-ಕ್ಷ-ಣೆಗೆ
ಹಿತ-ರ-ಕ್ಷ-ಣೆಯ
ಹಿನ್ನೆ-ಲೆ-ಯನ್ನು
ಹಿನ್ನೆ-ಲೆ-ಯ-ಲ್ಲಿ-ರು-ವ-ತ-ನಕ
ಹಿಮ-ಮಣಿ
ಹಿಮಾ-ಲಯ
ಹಿಮಾ-ಲ-ಯಕ್ಕೆ
ಹಿಮಾ-ಲ-ಯದ
ಹಿಮಾ-ಲ-ಯ-ದಲ್ಲಿ
ಹಿಮಾ-ಲ-ಯ-ದಿಂದ
ಹಿರಿದ
ಹಿರಿಮೆ
ಹಿರಿ-ಯ-ರಿಗೆ
ಹಿರಿ-ಯರು
ಹಿರಿ-ಯಾಸೆ
ಹೀಗಿ-ರು-ವುದು
ಹೀಗೆ
ಹೀಗೆಯೆ
ಹೀಗೆಯೇ
ಹೀನ
ಹೀನ-ಸ್ಥಿ-ತಿ-ಅ-ದಕ್ಕೆ
ಹೀನ-ಸ್ಥಿ-ತಿಗೆ
ಹೀರ-ಬೇ-ಕಾ-ಗಿದೆ
ಹೀರಿ-ಕೊಂ-ಡರು
ಹೀರಿ-ಕೊ-ಳ್ಳ-ಬೇಕು
ಹೀರಿ-ಕೊ-ಳ್ಳ-ಲಾ-ರದು
ಹೀರಿ-ದರೆ
ಹುಚ್ಚ
ಹುಚ್ಚ-ನಂತೆ
ಹುಚ್ಚರ
ಹುಟ್ಟಿದ
ಹುಟ್ಟಿ-ದಾಗ
ಹುಟ್ಟಿದೆ
ಹುಟ್ಟಿದ್ದೇ
ಹುಟ್ಟಿವೆ
ಹುಟ್ಟು
ಹುಟ್ಟು-ವಷ್ಟೇ
ಹುಟ್ಟುವು
ಹುಡುಕಿ
ಹುಡು-ಕಿ-ಕೊಂಡು
ಹುಡು-ಗ-ನಲ್ಲಿ
ಹುಡು-ಗ-ನಿಗೂ
ಹುಡು-ಗ-ರನ್ನು
ಹುಡು-ಗ-ರಾ-ದಾ-ಗಿ-ನಿಂ-ದಲೂ
ಹುಡು-ಗರು
ಹುಡು-ಗಾ-ಟಿಕೆ
ಹುಡು-ಗಿಗೆ
ಹುಡು-ಗಿ-ಗೇ-ನಾ-ದರೂ
ಹುಡು-ಗಿ-ಯನ್ನು
ಹುಡು-ಗಿ-ಯ-ರಿಗೆ
ಹುದು-ಗಿ-ರುವ
ಹೂವು-ಗಳನ್ನು
ಹೂವು-ಗಳು
ಹೃತ್ಪೂ-ರ್ವ-ಕ-ವಾಗಿ
ಹೃದಯ
ಹೃದ-ಯದ
ಹೃದ-ಯ-ದಲ್ಲಿ
ಹೃದ-ಯ-ದ-ಲ್ಲಿ-ರುವ
ಹೃದ-ಯ-ದ-ಲ್ಲಿ-ರು-ವುದನ್ನು
ಹೃದ-ಯ-ವನ್ನು
ಹೃದ-ಯ-ಸ್ಪಂ-ದ-ನ-ದೊಂ-ದಿಗೆ
ಹೃದ-ಯಿ-ಗ-ಳಾದ
ಹೆಂಗಸ
ಹೆಂಗ-ಸನ್ನು
ಹೆಂಗ-ಸ-ರನ್ನು
ಹೆಂಗ-ಸ-ರಿಗೂ
ಹೆಂಗ-ಸರು
ಹೆಂಗ-ಸ-ರೆಂದೆ
ಹೆಂಗ-ಸಿನ
ಹೆಂಗಸೆ
ಹೆಂಡತಿ
ಹೆಂಡ-ತಿಯ
ಹೆಚ್ಚ-ಬೇ-ಕಾ-ಗು-ವುದು
ಹೆಚ್ಚಾಗಿ
ಹೆಚ್ಚಾದ
ಹೆಚ್ಚಾ-ದಂತೆ
ಹೆಚ್ಚಿ-ರು-ವನು
ಹೆಚ್ಚಿ-ಸು-ವರು
ಹೆಚ್ಚಿ-ಸು-ವುದು
ಹೆಚ್ಚು
ಹೆಚ್ಚು-ಕ-ಡಿ-ಮೆ-ಯಾಗಿ
ಹೆಚ್ಚು-ವು-ದಕ್ಕೆ
ಹೆಜ್ಜೆ
ಹೆಮ್ಮೆ
ಹೆಮ್ಮೆಗೆ
ಹೆರುವ
ಹೆಸ-ರನ್ನು
ಹೆಸ-ರಾಂತ
ಹೆಸ-ರಿ-ನಲ್ಲಿ
ಹೆಸ-ರಿ-ನ-ಲ್ಲಿಯೂ
ಹೆಸ-ರಿ-ಲ್ಲ-ದಂತೆ
ಹೆಸರು
ಹೆಸರೇ
ಹೇಗಿದೆ
ಹೇಗಿ-ರು-ವೆವೊ
ಹೇಗೆ
ಹೇಗೊ
ಹೇಗೋ
ಹೇಡಿ-ತನ
ಹೇಡಿಯೂ
ಹೇಳ
ಹೇಳ-ಕೂ-ಡದು
ಹೇಳ-ಬಲ್ಲೆ
ಹೇಳ-ಬ-ಹುದು
ಹೇಳ-ಬೇ-ಕಾ-ಗಿದೆ
ಹೇಳ-ಬೇ-ಕಾ-ಗಿ-ರು-ವುದನ್ನು
ಹೇಳ-ಬೇಡಿ
ಹೇಳಲು
ಹೇಳಿ
ಹೇಳಿ-ಕೊ-ಡ-ಬೇಕು
ಹೇಳಿ-ಕೊ-ಡ-ಲಾ-ರಿರಿ
ಹೇಳಿ-ಕೊ-ಡು-ತ್ತಾರೆ
ಹೇಳಿ-ಕೊ-ಡು-ತ್ತಿ-ದ್ದನು
ಹೇಳಿ-ಕೊ-ಡು-ವರು
ಹೇಳಿ-ಕೊಳ್ಳು
ಹೇಳಿತೋ
ಹೇಳಿ-ದರು
ಹೇಳಿ-ದರೂ
ಹೇಳಿ-ದ-ಲ್ಲದೇ
ಹೇಳಿ-ರುವು
ಹೇಳು
ಹೇಳು-ತ್ತಾರೆ
ಹೇಳು-ತ್ತಿ-ದ್ದರು
ಹೇಳು-ತ್ತಿ-ರ-ಲಿಲ್ಲ
ಹೇಳು-ತ್ತಿಲ್ಲ
ಹೇಳು-ತ್ತೇನೆ
ಹೇಳು-ತ್ತೇ-ವೆಯೊ
ಹೇಳುವ
ಹೇಳು-ವಂತೆ
ಹೇಳು-ವನು
ಹೇಳು-ವರು
ಹೇಳು-ವ-ವನು
ಹೇಳು-ವು-ದಕ್ಕೆ
ಹೇಳು-ವು-ದಾ-ಗಿದೆ
ಹೇಳು-ವುದು
ಹೇಳು-ವುದೇ
ಹೇಳು-ವುವು
ಹೇಳು-ವೆನು
ಹೇಳೋಣ
ಹೊಂದ-ಬೇಡ
ಹೊಂದಿ-ದ್ದರೂ
ಹೊಂದಿ-ರು-ವುದು
ಹೊಂದಿ-ಸಿ-ಕೊಂಡ
ಹೊಂದಿ-ಸಿ-ಕೊಳ್ಳ
ಹೊಕ್ಕಿ-ರು-ವನು
ಹೊಗ-ಳ-ಬೇಕು
ಹೊಗ-ಳಿಕೆ
ಹೊಗ-ಳಿ-ಕೆ-ಗಿಂತ
ಹೊಗ-ಳು-ವರು
ಹೊಟ್ಟೆ-ಗಿ-ಲ್ಲದೆ
ಹೊಟ್ಟೆಗೆ
ಹೊಡೆ-ದ-ಟ್ಟ-ಬೇಕು
ಹೊಡೆದು
ಹೊಡೆ-ದೆ-ಬ್ಬಿಸಿ
ಹೊಡೆ-ಯ-ಬ-ಹುದು
ಹೊಡೆ-ಯು-ತ್ತಿದ್ದ
ಹೊಡೆ-ಯು-ವರು
ಹೊಣೆ
ಹೊಣೆ-ಗಾ-ರಿಕೆ
ಹೊತ್ತಿಗೆ
ಹೊದ್ದ
ಹೊನ್ನು
ಹೊಮ್ಮಿ
ಹೊರಗಿ
ಹೊರ-ಗಿನ
ಹೊರ-ಗಿ-ನ-ವರ
ಹೊರ-ಗಿ-ನ-ವ-ರನ್ನು
ಹೊರ-ಗಿ-ನ-ವ-ರಿಗೆ
ಹೊರ-ಗಿ-ನಿಂದ
ಹೊರ-ಗಿ-ರುವ
ಹೊರಗೆ
ಹೊರ-ಗೊಂದು
ಹೊರ-ಟಿ-ರು-ವನು
ಹೊರಟು
ಹೊರತು
ಹೊರ-ದೇ-ಶ-ದಿಂದ
ಹೊರುವ
ಹೊರು-ವುದೇ
ಹೊಲಿಗೆ
ಹೊಲಿ-ಯು-ವುದು
ಹೊಲೆ-ಯು-ವುದು
ಹೊಳೆ-ಯುವ
ಹೊಸ
ಹೊಸ-ಕಲ್ಪ
ಹೊಸ-ಚೇ-ತ-ನ-ವನ್ನು
ಹೊಸ-ದಾಗಿ
ಹೊಸದು
ಹೋಗ
ಹೋಗ-ಕೂ-ಡದು
ಹೋಗ-ಬ-ಹು-ದಾ-ಗಿತ್ತು
ಹೋಗ-ಬ-ಹುದೊ
ಹೋಗ-ಬಾ-ರದು
ಹೋಗ-ಬೇಕಾ
ಹೋಗ-ಬೇ-ಕಾ-ಗಿದೆ
ಹೋಗ-ಬೇಕು
ಹೋಗ-ಬೇಡ
ಹೋಗ-ಬೇಡಿ
ಹೋಗ-ಲಾ-ರೆವು
ಹೋಗಲಿ
ಹೋಗ-ಲಿಲ್ಲ
ಹೋಗಿ
ಹೋಗಿದೆ
ಹೋಗಿ-ದ್ದಾನೆ
ಹೋಗಿ-ಬಂ-ದರೆ
ಹೋಗಿ-ಬಿ-ಡು-ವುದು
ಹೋಗಿ-ರು-ವನು
ಹೋಗು
ಹೋಗು-ತ್ತಿ-ದ್ದಂ-ತಹ
ಹೋಗು-ತ್ತಿ-ದ್ದರು
ಹೋಗು-ತ್ತಿ-ರು-ವನು
ಹೋಗು-ತ್ತಿ-ರು-ವ-ವರ
ಹೋಗು-ತ್ತೀರಿ
ಹೋಗು-ತ್ತೇವೆ
ಹೋಗುವ
ಹೋಗು-ವಂತೆ
ಹೋಗು-ವರು
ಹೋಗು-ವಿರಿ
ಹೋಗು-ವು-ದಕ್ಕೆ
ಹೋಗು-ವು-ದ-ರಲ್ಲಿ
ಹೋಗು-ವುದು
ಹೋಗು-ವುದೊ
ಹೋಗು-ವುವು
ಹೋಗೋಣ
ಹೋದ
ಹೋದದ್ದು
ಹೋದನು
ಹೋದರು
ಹೋದರೂ
ಹೋದರೆ
ಹೋದವು
ಹೋದುವು
ಹೋಯಿತು
ಹೋರಾಡ
ಹೋರಾ-ಡ-ಬೇ-ಕಾ-ಗಿದೆ
ಹೋರಾಡಿ
ಹೋರಾ-ಡಿ-ದಷ್ಟು
ಹೋರಾ-ಡು-ತ್ತಿ-ರು-ವನೋ
ಹೋರಾ-ಡು-ತ್ತಿ-ರು-ವರು
ಹೋರಾ-ಡು-ತ್ತಿ-ರು-ವೆವು
ಹೋರಾ-ಡುವ
ಹೋರಾ-ಡೋಣ
ಹೋಲಿಸಿ
}
