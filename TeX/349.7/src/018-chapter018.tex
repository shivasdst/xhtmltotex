
\chapter[ಭರತಖಂಡ ಅಜ್ಞಾನ ಕೂಪದಲ್ಲಿದೆಯೇ? ]{ಭರತಖಂಡ ಅಜ್ಞಾನ ಕೂಪದಲ್ಲಿದೆಯೇ? \protect\footnote{\engfoot{C.W. Vol. IV, P. 198}}}

ಕೆಳಗೆ ಬರುವುದು ಸ್ವಾಮಿ ವಿವೇಕಾನಂದರು ಅಮೆರಿಕಾದ ಸಂಯುಕ್ತ ಸಂಸ್ಥಾನಗಳ\break ಡೆಟ್ರಾಯಿಟ್​ನಲ್ಲಿ ಮಾಡಿದ ಉಪನ್ಯಾಸದ ಸಾರಾಂಶ. ಇದು ಬಾಸ್ಟ್ರನ್​ ಈವಿನಿಂಗ್​ ಟ್ರಾನ್ಸ್​ಕ್ರಿಪ್ಟ್ ​ಸಂಪಾದಕರ ಟೀಕೆಗಳನ್ನು ಒಳಗೊಂಡಿದೆ. ಇದು ೧೮೯೪ನೇ ಇಸವಿಯ ೫ನೇ ಏಪ್ರಿಲ್​ ತಾರೀಖಿನ ಪತ್ರಿಕೆಯಿಂದ ತೆಗೆದುಕೊಂಡದ್ದಾಗಿರುತ್ತದೆ.

ಇತ್ತೀಚೆಗೆ ಸ್ವಾಮಿ ವಿವೇಕಾನಂದರು ಡೆಟ್ರಾಯಿಟ್​ನಲ್ಲಿ ಇದ್ದರು. ಅಲ್ಲಿಯ ಜನರ ಮೇಲೆ ಒಂದು ಅಭೂತಪೂರ್ವವಾದ ಪರಿಣಾಮವನ್ನು ಉಂಟು ಮಾಡಿದರು. ಎಲ್ಲಾ ಬಗೆಯ ಜನರು ಅವರನ್ನು ಕೇಳುವುದಕ್ಕೆ ಕಿಕ್ಕಿರಿದಿದ್ದರು. ಅವರಲ್ಲಿಯೂ ಮುಖ್ಯವಾಗಿ ವಿದ್ಯಾವಂತರು ಸ್ವಾಮೀಜಿಯವರ ಉಪನ್ಯಾಸದಲ್ಲಿದ್ದ ತರ್ಕಸರಣಿ ಮತ್ತು ವಿಚಾರಗಳ ಗಾಂಭೀರ್ಯವನ್ನು ಬಹಳ ಮೆಚ್ಚಿದ್ದರು. ಅವರ ಉಪನ್ಯಾಸಕ್ಕೆ ಬಂದ ಜನಸಮೂಹವನ್ನು ಒಳಗೊಳ್ಳಲು ಅಪೇರಾ ಹೌಸಿಗೆ ಮಾತ್ರ ಸಾಧ್ಯವಾಗಿತ್ತು. ಅವರು ಇಂಗ್ಲಿಷನ್ನು ತುಂಬಾ ಚೆನ್ನಾಗಿ ಮಾತನಾಡುವರು. ಅವರು ಎಷ್ಟು ಒಳ್ಳೆಯವರೊ ನೋಡಲು ಅಷ್ಟೇ ಸುಂದರವಾಗಿರುವರು. ಡೆಟ್ರಾಯಿಟ್​ ವೃತ್ತ ಪತ್ರಿಕೆಗಳು ಅವರ ಉಪನ್ಯಾಸವನ್ನು\break ಪ್ರಕಟಿಸಲು ಬೇಕಾದಷ್ಟು ಸ್ಥಳವನ್ನು ಕೊಟ್ಟಿವೆ. ಡೆಟ್ರಾಯಿಟ್​ ಈವಿನಿಂಗ್​ ಎಂಬ\break ಪತ್ರಿಕೆಯ ಸಂಪಾದಕೀಯವು ಹೀಗೆ ಹೇಳುತ್ತದೆ: ಈ ಊರಿನಲ್ಲಿ ಸ್ವಾಮಿ ವಿವೇಕಾನಂದರು ನೀಡಿದ ಎಲ್ಲ ಉಪನ್ಯಾಸಗಳಿಗಿಂತಲೂ ಅವರು ನಿನ್ನೆ ರಾತ್ರಿ ಅಪೇರಾ ಹೌಸ್​ನಲ್ಲಿ ನೀಡಿದ ಉಪನ್ಯಾಸವು ಉತ್ತಮವಾಗಿತ್ತು ಎಂಬುದು ಬಹುಜನರ ಅಭಿಪ್ರಾಯ. ಈ ಹಿಂದೂ ಉಪನ್ಯಾಸಕರ ಶ್ರೇಷ್ಠತೆಗೆ ಕಾರಣ ಅವರಲ್ಲಿದ್ದ ಭಾವನೆಯ ಸ್ಪಷ್ಟತೆ. ಅವರು ಕ್ರೈಸ್ತಧರ್ಮದ ಅನುಯಾಯಿಗಳಲ್ಲೇ ಇರುವ ತೀವ್ರವಾದ ಭಿನ್ನತೆಯನ್ನು ಚೆನ್ನಾಗಿ ತೋರಿಸಿದರು. ಅವರು ತಾವು ಒಂದು ಅರ್ಥದಲ್ಲಿ ಕ್ರೈಸ್ತರೆಂದೂ ಮತ್ತೊಂದು ಅರ್ಥದಲ್ಲಿ ಕ್ರೈಸ್ತರಲ್ಲವೆಂದೂ\break ಸಭಿಕರಿಗೆ ಸ್ಪಷ್ಟಪಡಿಸಿದರು. ಅದರಂತೆಯೇ ಹಿಂದೂಗಳಲ್ಲಿಯೂ ಇರುವ ಭಿನ್ನತೆಯನ್ನು ತೋರಿಸಿದರು. ತಾವು ಆ ಪದದ ಉತ್ತಮಾರ್ಥದಲ್ಲಿ ಹಿಂದೂಗಳು ಎಂಬುದನ್ನು ವ್ಯಕ್ತಪಡಿಸಿದರು. “ನನಗೆ ನಿಜವಾದ ಕ್ರೈಸ್ತ ಮಿಷನರಿಗಳು ಬೇಕು. ಅಂತಹ ನೂರಾರು, ಸಾವಿರಾರು ಮಿಷನರಿಗಳು ಭರತಖಂಡಕ್ಕೆ ಬರಲಿ. ಕ್ರಿಸ್ತನ ಜೀವನವನ್ನು ನಮಗೆ ತನ್ನಿ, ನಮ್ಮ ಸಮಾಜದಲ್ಲಿ ಇದು ಓತಪ್ರೋತವಾಗಲಿ. ಭರತಖಂಡದ ಪ್ರತಿಯೊಂದು ಗ್ರಾಮದಲ್ಲಿಯೂ ಮೂಲೆ ಮೂಲೆಯಲ್ಲಿಯೂ ಬೇಕಾದರೆ ಅವನನ್ನು ಸಾರಲಿ.” ಹೀಗೆ ಸ್ವಾಮಿ ವಿವೇಕಾನಂದರು\break ಹೇಳಿದಾಗ ಯಾರೂ ಅವರನ್ನು ಟೀಕಿಸಲಾರರು. ಅಂತಹ ಉಚ್ಚಸ್ಥಿತಿಯಲ್ಲಿ ಅವರು\break ನಿಂತುಕೊಂಡಿರುವರು.

ಮುಖ್ಯ ಪ್ರಶ್ನೆಯ ವಿಷಯದಲ್ಲಿ ವ್ಯಕ್ತಿಯು ಅಷ್ಟೊಂದು ಪರಿಪೂರ್ಣರಾಗಿರುವಾಗ, ಅವರು ಹೇಳುವ ಇತರ ವಿಷಯಗಳೆಲ್ಲ ಗೌಣವಾಗುತ್ತವೆ. ಈ ಹಿಂದೂ ಸಂನ್ಯಾಸಿ ಯಾರು ಗ್ರೀನ್​ಲ್ಯಾಂಡಿನ ಹಿಮಾವೃತ ಪರ್ವತಗಳ ಮತ್ತು ಭರತಖಂಡದ ತೀರ ಪ್ರದೇಶದ\break ಆಧ್ಯಾತ್ಮಿಕ ಹಿತರಕ್ಷಣೆಯನ್ನು ತಾವು ವಹಿಸಿಕೊಂಡಿರುವೆವು ಎಂದು ಭಾವಿಸುವರೊ\break ಅವರಿಗೆ ಜೀವನ ಮತ್ತು ಶೀಲಕ್ಕೆ ಸಂಬಂಧಿಸಿದ್ದುದನ್ನು ಬೋಧಿಸುವಾಗ, ಅವರಲ್ಲಿ\break ಅತ್ಯಂತ ವಿನಯಭಾವ ಇರುವುದನ್ನು ನೋಡುತ್ತೇವೆ. ಆದರೆ ಜಗತ್ತಿನ ಎಲ್ಲಾ ಸುಧಾರಕರ ಸಂಸ್ಥೆಗಳ ಸಾರವೇ ವಿನಯ. ಕ್ರೈಸ್ತ ಧರ್ಮದ ಮೂಲ ಪುರುಷನ ಮಹಾತ್ಮೆಯನ್ನು\break ವಿವರಿಸಿ ಆದ ಮೇಲೆ ತಾವು ಮಾತನಾಡಿದ ರೀತಿಯಲ್ಲಿ ಮಾತನಾಡುವುದಕ್ಕೆ ವಿವೇಕಾನಂದ\-ರಿಗೆ ಅಧಿಕಾರವಿದೆ. ಪ್ರಪಂಚದ ಬೇರೆ ಬೇರೆ ಕಡೆ ತಾವು ಇಂತಹ ಧರ್ಮಕ್ಕೆ ಸೇರಿದವರು ಎಂದು ಅಭಿಮಾನಪಡುವವರಿಗೆ ವಿವೇಕಾನಂದರು ಹೇಳುವುದು ಸೂಕ್ತವಾಗಿದೆ. ಅವರ ನುಡಿಗಳು ಏಸುವಿನ ಈ ಮುಂದಿನ ನುಡಿಗಳನ್ನು ನಮ್ಮ ನೆನಪಿಗೆ ತರುತ್ತವೆ: “ನಿಮ್ಮ\break ಜೇಬಿನಲ್ಲಿ ಚಿನ್ನ, ಬೆಳ್ಳಿ, ಹಿತ್ತಾಳೆ ಯಾವುದೂ ಇರಬೇಕಾಗಿಲ್ಲ. ಪ್ರಯಾಣಕ್ಕೆ ಜೋಳಿಗೆಯೂ ಬೇಕಾಗಿಲ್ಲ. ನಿಮಗೆ ಎರಡು ಅಂಗಿಗಳ ಅಥವಾ ಪಾದರಕ್ಷೆಗಳ ದಂಡಗಳ ಆವಶ್ಯಕತೆಯೂ ಇಲ್ಲ. ಕೆಲಸಗಾರನಿಗೆ ಊಟ ದೊರಕಿಯೇ ದೊರಕುತ್ತದೆ.” ವಿವೇಕಾನಂದರು ಬರುವುದಕ್ಕೆ ಮುಂಚೆ ಭಾರತೀಯ ಧಾರ್ಮಿಕ ಭಾವನೆಗಳನ್ನು ಸ್ವಲ್ಪ ತಿಳಿದವರು, ಪೌರಸ್ತ್ಯರು\break ಪಾಶ್ಚಾತ್ಯರ ವಾಣಿಜ್ಯ ಮನೋಭಾವವನ್ನು ಯಾವ ಬಗೆಯ ಜುಗುಪ್ಸೆಯಿಂದ\break ನೋಡುತ್ತಾರೆ ಎಂಬುದನ್ನು ಅರ್ಥಮಾಡಿಕೊಳ್ಳಬಲ್ಲರು. ಇದನ್ನೇ ವಿವೇಕಾನಂದರು\break ವರ್ತಕನ ದೃಷ್ಟಿ ಎನ್ನುವರು. ನಾವು ಮಾಡುವ ಕೆಲಸದಲ್ಲಿ ಮಾತ್ರವಲ್ಲ ನಮ್ಮ ಧರ್ಮದಲ್ಲಿ ಕೂಡ ಇದೇ ದೃಷ್ಟಿ ಇರುವುದು.

ಮಿಷನರಿಗಳು ಮರೆಯದೆ ಇರಬೇಕಾದ ಒಂದು ವಿಷಯವು ಇಲ್ಲಿದೆ. ಯಾರು ಪೂರ್ವದೇಶಗಳಲ್ಲಿ ಕ್ರೈಸ್ತರಲ್ಲದವರನ್ನು ತಮ್ಮ ಮತಕ್ಕೆ ಸೇರಿಸಿಕೊಳ್ಳುವುದಕ್ಕೆ ಪ್ರಯತ್ನಿಸುತ್ತಿರುವರೊ ಅವರು ತಾವು ಬೋಧಿಸಿದಂತೆ ಬದುಕುವುದನ್ನು ಕಲಿತು ಈ ಪ್ರಪಂಚದ ವೈಭವವನ್ನೆಲ್ಲ ನಿಕೃಷ್ಟ ದೃಷ್ಟಿಯಿಂದ ನೋಡುವುದನ್ನು ಅಭ್ಯಾಸ ಮಾಡಬೇಕು.

ಸೋದರ ವಿವೇಕಾನಂದರು ಭರತಖಂಡ ಜಗತ್ತಿನ ಅತ್ಯಂತ ನೈತಿಕವಾದ ದೇಶ ಎಂದು ನಂಬುವರು. ಅದು ಈಗ ಬಂಧನದಲ್ಲಿದ್ದರೂ ಆಧ್ಯಾತ್ಮಿಕತೆ ಅಲ್ಲಿಂದ ಇನ್ನೂ\break ಮಾಯವಾಗಿಲ್ಲ. ಇತ್ತೀಚೆಗೆ ಅವರು ಡೆಟ್ರಾಯಿಟ್​ನಲ್ಲಿ ಮಾಡಿದ ಉಪನ್ಯಾಸಗಳಿಂದ ಉದ್ಧೃತ ಭಾಗಗಳು ಕೆಳಗೆ ಬರುವುವು. ಈ ಹಂತದಲ್ಲಿ ಉಪನ್ಯಾಸಕರು ತಮ್ಮ\break ಭಾಷಣದಲ್ಲಿ ನೈತಿಕ ಭಾವನೆಯನ್ನು ಒತ್ತಿಹೇಳಿದರು. ಅವರ ದೇಶದ ಜನರು ಎಲ್ಲಾ ಸ್ವಾರ್ಥವನ್ನೂ ಪಾಪವೆಂದು ಎಣಿಸುವರು, ನಿಃಸ್ವಾರ್ಥವನ್ನು ಪುಣ್ಯವೆಂದು ಭಾವಿಸುವರು. ಉಪನ್ಯಾಸದಲ್ಲೆಲ್ಲ ಈ ಒಂದು ವಿಷಯವನ್ನೇ ಒತ್ತಿ ಹೇಳಿದರು. ಇದನ್ನೇ ಉಪನ್ಯಾಸದ ಪಲ್ಲವಿ ಎಂದು ಬೇಕಾದರೆ ಹೇಳಬಹುದು. “ಹಿಂದೂಗಳು ಒಂದು ಮನೆಯನ್ನು ಕಟ್ಟುವುದನ್ನು ಸ್ವಾರ್ಥವೆಂದು ಭಾವಿಸುತ್ತಾರೆ. ಆದಕಾರಣವೇ ಅವರು ದೇವರ ಪೂಜೆಗಾಗಿ ಮತ್ತು ಅತಿಥಿಗಳಿಗಾಗಿ ಮನೆಯನ್ನು ಕಟ್ಟುವರು. ತಮಗಾಗಿ ಅಡಿಗೆ ಮಾಡಿಕೊಳ್ಳುವುದು ಸ್ವಾರ್ಥ, ಅದಕ್ಕಾಗಿ ಬಡವರಿಗೆ ಕೊಡಲು ಅಡಿಗೆ ಮಾಡುವರು. ಊಟ ಮಾಡುವ ಸಮಯದಲ್ಲಿ\break ಯಾರಾದರೂ ಉಪವಾಸವಿರುವ ಅಪರಿಚಿತ ಬಂದರೆ ಅವರು ತಮ್ಮ ಪಾಲಿನದನ್ನು\break ಅವನಿಗೆ ಕೊಡುವರು. ಈ ಭಾವನೆಯು ಭರತಖಂಡದಲ್ಲೆಲ್ಲಾ ವ್ಯಾಪಿಸಿರುವುದು.\break ಯಾರಾದರೂ ಇಳಿದುಕೊಳ್ಳುವುದಕ್ಕೆ ಮತ್ತು ಊಟಕ್ಕೆ ಯಾವ ಮನೆಯಲ್ಲಾದರೂ ಅವಕಾಶವನ್ನು ಕೇಳಬಹುದು. ಯಾವ ಮನೆಯವರಾದರೂ ಅವನನ್ನು ಸ್ವೀಕರಿಸುವರು.

“ಜಾತಿಪದ್ಧತಿಗೂ ಧರ್ಮಕ್ಕೂ ಯಾವ ಸಂಬಂಧವೂ ಇಲ್ಲ. ಒಬ್ಬನ ಕಸಬು\break ಅನುವಂಶಿಕವಾಗಿ ಬಂದದ್ದು. ಬಡಗಿ ಹುಟ್ಟು ಬಡಗಿಯಾಗಿರುವನು. ಅಕ್ಕಸಾಲಿಗ ಹುಟ್ಟು ಅಕ್ಕಸಾಲಿಗನಾಗಿರುವನು. ಕೆಲಸಗಾರ ಹುಟ್ಟು ಕೆಲಸಗಾರನಾಗಿರುವನು. ಪುರೋಹಿತ ಹುಟ್ಟು ಪುರೋಹಿತನಾಗಿರುವನು.

“ಅಲ್ಲಿನ ಜನರು ಎರಡು ದಾನಗಳನ್ನು ಬಹಳ ಮೆಚ್ಚುವರು. ಅದೇ ವಿದ್ಯಾದಾನ ಮತ್ತು ಪ್ರಾಣದಾನ. ಆದರೆ, ವಿದ್ಯಾದಾನ ಎಲ್ಲಕ್ಕಿಂತ ಶ್ರೇಷ್ಠ. ಒಬ್ಬನು ಮತ್ತೊಬ್ಬನಿಗೆ\break ಪ್ರಾಣದಾನ ಮಾಡಬಹುದು. ಅದು ಒಳ್ಳೆಯದು. ಆದರೆ ವಿದ್ಯಾದಾನ ಅದಕ್ಕಿಂತ ಮೇಲು. ಹಣಕ್ಕಾಗಿ ವಿದ್ಯೆಯನ್ನು ಹೇಳಿಕೊಡುವುದು ಪಾಪ. ಯಾರು ಹೀಗೆ ಹಣಕ್ಕಾಗಿ ವಿದ್ಯೆಯನ್ನು ವಿಕ್ರಯಮಾಡುವರೊ ಅವರು ಪಾಪವನ್ನು ಮಾಡಿದಂತೆ. ಸರ್ಕಾರ ಕಾಲಕಾಲಕ್ಕೆ ಗುರುಗಳಿಗೆ ದಾನವನ್ನು ಕೊಡುವುದು. ನಾಗರೀಕರು ಎಂದು ಕರೆಯಿಸಿಕೊಳ್ಳುವ ದೇಶಗಳಲ್ಲಿ ಇರುವ ಸ್ಥಿತಿಗಿಂತ ಇದು ಉತ್ತಮ. “ಉಪನ್ಯಾಸಕರು ಇತರ ಅನೇಕ ದೇಶಗಳಲ್ಲಿ ನಾಗರಿಕತೆ ಎಂದರೆ ಏನು ಎಂಬ ಪ್ರಶ್ನೆಯನ್ನು ಹಾಕಿದ್ದರು. ಕೆಲವು ವೇಳೆ “ನಾವಿರುವ ಸ್ಥಿತಿಯೇ ನಾಗರಿಕತೆ” ಎಂಬ ಉತ್ತರ ಬಂದಿತ್ತು. ನಾನು ಇದನ್ನು ಒಪ್ಪಲಾರೆ ಎಂದು ಉಪನ್ಯಾಸಕರು ಹೇಳಿದರು. ಒಂದು ದೇಶ ಸಮುದ್ರಗಳನ್ನೇ ಗೆಲ್ಲಬಹುದು. ಪಂಚಭೂತಗಳನ್ನೇ ನಿಗ್ರಹಿಸಬಹುದು. ನಮ್ಮ ಜೀವನಕ್ಕೆ ಬೇಕಾದ ಸೌಲಭ್ಯಗಳನ್ನೆಲ್ಲ ಬೇಕಾದರೂ ಉತ್ಪತ್ತಿ ಮಾಡಿಕೊಳ್ಳುವ ಸ್ಥಿತಿಗೆ\break ಬರಬಹುದು. ಆದರೂ ಯಾರು ತನ್ನನ್ನು ತಾನು ಗೆದ್ದಿರುವನೋ ಅಂತಹವನ ಹೃದಯದಲ್ಲಿ ಶ್ರೇಷ್ಠವಾದ ನಾಗರಿಕತೆ ಇದೆ ಎಂಬುದನ್ನು ತಿಳಿಯದೇ ಇರಬಹುದು. ಪ್ರಪಂಚದಲ್ಲಿ ಇತರ ದೇಶಗಳೆಲ್ಲಕ್ಕಿಂತಲೂ ತಮ್ಮನ್ನು ತಾವು ಗೆಲ್ಲುವ ಪ್ರಯತ್ನ ಭರತಖಂಡದಲ್ಲಿ ಹೆಚ್ಚಾಗಿ\break ನಡೆಯುತ್ತಿದೆ. ಏಕೆಂದರೆ ಅಲ್ಲಿ ಲೌಕಿಕವು ಆಧ್ಯಾತ್ಮಕ್ಕೆ ಅಡಿಯಾಳು. ಅಲ್ಲಿ ಪ್ರತಿಯೊಬ್ಬನೂ ಆತ್ಮವು ಹೇಗೆ ಎಲ್ಲರಲ್ಲಿಯೂ ವಿಕಾಸವಾಗುತ್ತದೆ ಎಂಬುದನ್ನು ನೋಡುವನು. ಅವನು ಪ್ರಕೃತಿಯನ್ನು ಅಧ್ಯಯನ ಮಾಡುವುದು ಈ ದೃಷ್ಟಿಯಿಂದ. ಆದಕಾರಣವೇ ಜೀವನದ ದುರದೃಷ್ಟಗಳನ್ನೆಲ್ಲ ಅಷ್ಟು ತಾಳ್ಮೆಯಿಂದ ಸಹಿಸಬಲ್ಲ ಸೌಮ್ಯ ಪ್ರಕೃತಿ ಅವನಿಗಿದೆ. ಇತರ ದೇಶಗಳ ಭರತಖಂಡ ಅಜ್ಞಾನ ಕೂಪದಲ್ಲಿದೆಯೇ? ೧೭೫ ಜನಗಳಿಗಿಂತ ಅಧಿಕ\break ಪ್ರಮಾಣದಲ್ಲಿ ಹಿಂದೂವಿನಲ್ಲಿ ಆಧ್ಯಾತ್ಮಿಕ ಶಕ್ತಿ ಪ್ರಬುದ್ಧವಾಗಿದೆ. ಆದಕಾರಣವೇ\break ಇಂತಹ ದಿವ್ಯದೃಷ್ಟಿಯ ಅಮೃತವು ಹರಿಯುತ್ತಿರುವ ದೇಶ ಮತ್ತು ಜನಾಂಗ ಇನ್ನೂ ಬದುಕಿರುವುದು ಮತ್ತು ಅದು ಹತ್ತಿರದ ಮತ್ತು ದೂರದ ವಿಚಾರವಂತರನ್ನೆಲ್ಲ ಆಕರ್ಷಿಸುತ್ತಿರುವುದು. ಹಾಗೆ ಬಂದ ವಿಚಾರವಂತರು ತಮ್ಮ ಜೀವನದ ಹೊರೆಯನ್ನು ಹಗುರ ಮಾಡಿಕೊಳ್ಳುತ್ತಿರುವರು.

ಉಪನ್ಯಾಸದ ಪ್ರಾರಂಭದಲ್ಲಿಯೇ ಭಾಷಣಕಾರರು, ಶೋತೃಗಳು ತಮ್ಮನ್ನು\break ಹಲವು ಪ್ರಶ್ನೆಗಳನ್ನು ಕೇಳಿರುವರು ಎಂದು ಹೇಳಿದರು. ಇವುಗಳಲ್ಲಿ ಕೆಲವು ಪ್ರಶ್ನೆಗಳಿಗೆ\break ಏಕಾಂತವಾಗಿ ಉತ್ತರ ಹೇಳುವೆ ಎಂದರು. ಅವುಗಳಲ್ಲಿ ಮೂರಕ್ಕೆ ಮಾತ್ರ ಬಹಿರಂಗದಲ್ಲಿ ಉತ್ತರ ಹೇಳುತ್ತೇನೆಂದು ಆರಿಸಿಕೊಂಡಿದ್ದರು. ಆ ಪ್ರಶ್ನೆಗಳು “ಭಾರತೀಯರು\break ತಮ್ಮ ಮಕ್ಕಳನ್ನು ಮೊಸಳೆಗಳಿಗೆ ಎಸೆಯುವರೇ?” “ಜಗನ್ನಾಥ ರಥದ ಅಡಿಯಲ್ಲಿ\break ಅವರು ಆತ್ಮಹತ್ಯೆ ಮಾಡಿಕೊಳ್ಳುವರೇ?” “ಸತ್ತ ಗಂಡನೊಂದಿಗೆ ವಿಧವೆಯನ್ನು ಜೀವಸಹಿತ ಸುಡುವರೇ?”

ಇವುಗಳಲ್ಲಿ ಮೊದಲನೆ ಪ್ರಶ್ನೆಗೆ, ದೂರ ದೇಶದಲ್ಲಿ ಸಂಚಾರ ಮಾಡುತ್ತಿರುವ\break ಅಮೆರಿಕಾ ದೇಶೀಯನು ಹೇಗೆ ಉತ್ತರ ಕೊಡುವನೋ ಹಾಗೆಯೇ ಉತ್ತರ ಕೊಟ್ಟರು. ನ್ಯೂಯಾರ್ಕ್​ ನಗರದಲ್ಲಿ ಬೀದಿಯಲ್ಲಿ ರೆಡ್​ ಇಂಡಿಯನ್ನರು ಅಂಜಿಕೊಂಡು ಓಡಾಡು\-ತ್ತಿರುವರು ಮತ್ತು ಇಂತಹ ಹಲವು ಕಟ್ಟುಕಥೆಗಳನ್ನು ಯೂರೋಪ್​ ದೇಶದ ಜನ\break ಅಮೆರಿಕ ದೇಶೀಯರ ವಿಷಯದಲ್ಲಿ ಈಗಲೂ ನಂಬಿರುವರೆ? ಅಮೆರಿಕ ದೇಶೀಯನು ಇದಕ್ಕೆ ಉತ್ತರ ಕೊಡುವಂತೆಯೇ ಸ್ವಾಮೀಜಿಯೂ ಕೊಟ್ಟರು. ಮೊದಲನೆಯದಾಗಿ\break ಪ್ರಶ್ನೆಯೇ ಹಾಸ್ಯಾಸ್ಪದವಾದುದು, ನಾವು ಅದನ್ನು ಗಮನಕ್ಕೆ ತೆಗೆದುಕೊಳ್ಳಬೇಕಾಗಿಲ್ಲ. ಕೆಲವು ಜನರು ಸದ್ಭಾವನೆಯಿಂದಲೇ, ಆದರೆ ಅಜ್ಞಾನದಿಂದ, ಜನ ಏತಕ್ಕೆ ಹೆಣ್ಣುಮಕ್ಕಳನ್ನು ಮೊಸಳೆಗೆ ಕೊಡುತ್ತಾರೆ ಎಂದು ಕೇಳಿದರು. ಅದಕ್ಕೆ ಸ್ವಾಮೀಜಿ ಹಾಸ್ಯವಾಗಿಯೇ ಉತ್ತರ ಕೊಟ್ಟರು. ಕತ್ತಲೆಯ ದೇಶದ ನದಿಗಳಲ್ಲಿರುವ ಮೊಸಳೆಗಳಿಗೆ ತಿನ್ನಲು ಅದು ಮೃದುವಾಗಿರಬಹುದು, ಅವರು ಮೊಸಳೆಗೆ ಚೆನ್ನಾಗಿ ಅಗಿಯಲು ಸುಲಭವಾಗಿರಬಹುದೆಂದು ಹಾಗೆ ಕೊಡಬಹುದು ಎಂದರು! ಜಗನ್ನಾಥ ರಥಕ್ಕೆ ಸಿಕ್ಕಿ ಜನರು ಸಾಯುವುದನ್ನು ವಿವರಿಸುತ್ತಾ, ಅವರು ಜಗನ್ನಾಥ ರಥವನ್ನು ಎಳೆಯುವ ಸಂಭ್ರಮದಲ್ಲಿ ಹಲವರು ಹಗ್ಗವನ್ನು ಹಿಡಿದುಕೊಳ್ಳುವುದಕ್ಕಾಗಿ ಪ್ರಯತ್ನಿಸಿದಾಗ ಚಕ್ರಕ್ಕೆ ಸಿಕ್ಕಿ ಸತ್ತಿರಬಹುದು ಎಂದರು. ಬಹುಶಃ ಇಂತಹ ಕೆಲವು ಆಕಸ್ಮಿಕಗಳನ್ನು ಉತ್ಪ್ರೇಕ್ಷಿಸಿ ಹೀಗೆ ಹೇಳಿರಬೇಕು. ಅನ್ಯದೇಶದ ಒಳ್ಳೆಯ ಜನರು ಇಂತಹ ವಿಷಯಗಳನ್ನು ಕೇಳಿ ಅಂಜಿಕೆಯಿಂದ ಕಂಪಿಸಿರಬಹುದು. ಜನರು ವಿಧವೆಯರನ್ನು ಸುಡುವುದಿಲ್ಲ ಎಂದು ವಿವೇಕಾನಂದರು ಒತ್ತಿ ಹೇಳಿದರು. ಕೆಲವು ವಿಧವೆಯರು ತಾವೇ ಚಿತೆಗೆ ಬಿದ್ದು ಸತ್ತಿರುವುದು ನಿಜವಿರಬಹುದು. ಆದರೆ ಹಾಗೆ ಮಾಡುವಾಗಲೂ ಸಂತರು ಅದಕ್ಕೆ ಸಮ್ಮತಿ ಕೊಟ್ಟಿರಲಿಲ್ಲ. ಆತ್ಮಹತ್ಯೆಯನ್ನು ಅವರು ಮಹಾಪಾಪವೆಂದು ಭಾವಿಸುವರು. ಕೆಲವು ವಿಧವೆಯರು ತಮ್ಮ ಗಂಡಂದಿರನ್ನು ಕೊನೆಯತನಕ ಅನುಸರಿಸುತ್ತೇವೆ ಎಂದು ಒತ್ತಾಯಿಸಿದಾಗ ಅವರಿಗೆ ಒಂದು ಪರೀಕ್ಷೆಯನ್ನು ಕೊಡುತ್ತಿದ್ದರು – ಉರಿಯುವ ಜ್ವಾಲೆಗೆ ತಮ್ಮ ಕೈಯನ್ನು ಒಡ್ಡುವುದು. ಅದು ಉರಿದು ಹೋಗುವ ತನಕ ಕೈಯನ್ನು ಅಲ್ಲಿಯೇ ಇಟ್ಟಿರುವುದಕ್ಕೆ ಸಾಧ್ಯವಾಗಿದ್ದರೆ ಅವರು ಗಂಡನೊಂದಿಗೆ ಬೆಂಕಿಗೆ ಬೀಳಲು ಯಾವ\break ಆತಂಕವನ್ನೂ ಒಡ್ಡುತ್ತಿರಲಿಲ್ಲ. ಆದರೆ ಗಂಡ ಸತ್ತರೆ ಮೃತ್ಯುವಿನಲ್ಲಿಯೂ ಅವನನ್ನೇ\break ಅನುಸರಿಸಿ ಹೋದ ಸ್ತ್ರೀಯರು ಇಂಡಿಯಾ ದೇಶ ಒಂದರಲ್ಲೇ ಅಲ್ಲ ಇರುವುದು. ಇದು ಎಲ್ಲಾ ದೇಶಗಳಲ್ಲಿಯೂ ನಡೆದಿದೆ. ಇದು ಯಾವ ದೇಶದಲ್ಲಿ ಆದರೂ ಒಂದು ಅಸಾಧಾರಣವಾದ ಧರ್ಮಾಂಧತೆ. ಇದು ಎಲ್ಲಾ ಕಡೆಯಂತೆ ಇಲ್ಲಿಯೂ ಕೂಡ\break ಅಸ್ವಾಭಾವಿಕ. “ಇಲ್ಲ, ಭರತಖಂಡದಲ್ಲಿ ಜನ ಎಂದಿಗೂ ಸ್ತ್ರೀಯರನ್ನು ಸುಡುವುದಿಲ್ಲ.\break ಅಲ್ಲಿ ಮಾಟಗಾತಿಯರನ್ನೂ ಎಂದಿಗೂ ಜನರು ಸುಟ್ಟಿಲ್ಲ” ಎಂದು ಉಪನ್ಯಾಸಕರು ಮತ್ತೆ ಹೇಳಿದರು.

ಅವರು ಕೊನೆಗೆ ಹೇಳಿದ್ದು ಸ್ವಲ್ಪ ಕಟುವಾಗಿಯೇ ಇತ್ತು. ಹಿಂದೂ ಸಂನ್ಯಾಸಿಯ ತಾತ್ತ್ವಿಕ ಭಾವನೆಗಳ ವಿಶ್ಲೇಷಣೆಯನ್ನು ಇಲ್ಲಿ ಮಾಡುವ ಅವಶ್ಯಕತೆ ಇಲ್ಲ. ಪ್ರತಿಯೊಂದು ಜೀವಿಯೂ ಮುಕ್ತನಾಗಲು ಹೋರಾಡುವುದೇ ಅದರ ಉದ್ದೇಶ. ಈ ವರುಷ ಒಬ್ಬ ಪ್ರಖ್ಯಾತನಾದ ಹಿಂದು ಲೊವೆಲ್​ ಇನ್​ಸ್ಟಿಟ್ಯೂಟ್​ ಪ್ರಾರಂಭಿಸಿದನು. ಯಾವುದನ್ನು ಮಿಸ್ಟರ್​ ಮುಜುಂದಾರ್​ ಪ್ರಾರಂಭಿಸಿದನೋ ಅದನ್ನು ಸೋದರ ವಿವೇಕಾನಂದ ಯೋಗ್ಯವಾಗಿಯೇ ಕೊನೆಗಾಣಿಸಿದರು. ಹಿಂದೂ ದರ್ಶನದಲ್ಲಿ ವ್ಯಕ್ತಿತ್ವಕ್ಕೆ ಪ್ರಾಧಾನ್ಯತೆಯನ್ನು ಕೊಡುವುದಿಲ್ಲ. ಈ ಪೂರ್ಣ ಆಗಂತುಕನು ಅತ್ಯಂತ ಆಸಕ್ತಿಯನ್ನು ಕೆರಳಿಸುವ ವ್ಯಕ್ತಿ. ವಿಶ್ವಧರ್ಮ ಸಮ್ಮೇಳನದಲ್ಲಿ ವಿವೇಕಾನಂದರನ್ನು ಕೊನೆಗೆ ಮಾತನಾಡಲು ಹೇಳುತ್ತಿದ್ದರು. ಇದು ಜನರನ್ನು ಕೊನೆಯ ತನಕ ಕುಳಿತುಕೊಳ್ಳುವಂತೆ ಮಾಡುವುದಕ್ಕಾಗಿ. ಕೆಲವು ದಿನ ಮಧ್ಯಾಹ್ನದ ಸಮಯದಲ್ಲಿ ನೀರಸ ಉಪನ್ಯಾಸಕರು ದೀರ್ಘ ಭಾಷಣಗಳನ್ನು ಮಾಡಿದಾಗ, ನೂರಾರು ಜನ ಮನೆಗೆ ಹೋಗಲು ಎದ್ದಾಗ, ಅಧ್ಯಕ್ಷರು ಎದ್ದು ಕೊನೆಗೆ ವಿವೇಕಾನಂದರು ಸ್ವಲ್ಪ ಮಾತಾಡುವರು ಎಂದು ಹೇಳುತ್ತಿದ್ದರು. ಆಗ ನೂರಾರು ಜನ ಸದ್ದು ಗದ್ದಲವಿಲ್ಲದೆ ಕುಳಿತುಕೊಳ್ಳುತ್ತಿದ್ದರು. ಕೊಲಂಬಿಯಾದ ಹಾಲಿನಲ್ಲಿ ನಾಲ್ಕು ಸಾವಿರ ಜನ ಬೀಸಣಿಗೆಯಿಂದ ಬೀಸಿಕೊಳ್ಳುತ್ತ\break ಒಂದೆರಡು ಗಂಟೆಗಳ ಕಾಲ ನೀರಸವಾದ ಉಪನಾಯಸಗಳನ್ನು ಕೇಳುತ್ತ ಕುಳಿತುಕೊಳ್ಳುತ್ತಿದ್ದರು, ಕೊನೆಗೆ ಸ್ವಾಮಿ ವಿವೇಕಾನಂದರು ಮಾಡುವ ಹದಿನೈದು ನಿಮಿಷಗಳ ಉಪನ್ಯಾಸವನ್ನು ಕೇಳುವುದಕ್ಕಾಗಿ. ಅಧ್ಯಕ್ಷರಿಗೆ ಶ್ರೇಷ್ಠವಾದುದನ್ನು ಕೊನೆಗೆ ಇಟ್ಟಿರಬೇಕು ಎಂಬ ಹಳೆಯ ನಿಯಮವು ಚೆನ್ನಾಗಿ ಗೊತ್ತಿತ್ತು.

