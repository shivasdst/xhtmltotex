
\chapter[ಜಗತ್ತಿನ ಮಹಾಗುರುಗಳು]{ಜಗತ್ತಿನ ಮಹಾಗುರುಗಳು \protect\footnote{\engfoot{C.W. Vol. IV P. 120}}}

\begin{center}
(೧೯೦೦ರ ಫೆಬ್ರವರಿ ೩ನೇ ತಾರೀಖು ಕ್ಯಾಲಿಫೋರ್ನಿಯಾದ ಪಸದೆನಾದಲ್ಲಿ ನೀಡಿದ ಉಪನ್ಯಾಸ)
\end{center}

ಹಿಂದುಗಳ ಸಿದ್ಧಾಂತದಂತೆ ವಿಶ್ವವು ಅಲೆಗಳಂತೆ ಕಲ್ಪಾಂತರಗಳ ಮೂಲಕ ಸಂಚರಿಸುತ್ತಿದೆ. ಅದು ಮೇಲೆದ್ದು ಶಿಖರವನ್ನು ಮುಟ್ಟಿ ಕೆಳಗೆ ಬೀಳುವುದು. ಅಲ್ಲೇ ಕೆಲವು ಕಾಲ ಇರುವುದು. ಅನಂತರ ಅಲೆಯಾಗಿ ಪುನಃ ಮೇಲೇಳುವುದು, ಕೆಳಗೆ ಬೀಳುವುದು. ಒಟ್ಟಿನಲ್ಲಿ ಯಾವುದು ಸಮಷ್ಟಿಯಲ್ಲಿ ಸತ್ಯವೋ ಅದೇ ಅದರ ಅಂಶ ಗಳಿಗೆಲ್ಲ ಅನ್ವಯಿಸುವುದು. ಮಾನವನ ಪ್ರಗತಿ ಇರುವುದು ಹೀಗೆ, ರಾಷ್ಟ್ರಗಳ ಇತಿಹಾಸ ಇರುವುದು ಹೀಗೆ. ಅವು ಏಳುವುವು, ಬೀಳುವುವು. ಎದ್ದ ಮೇಲೆ ಬೀಳುವುವು, ಪುನಃ ಹೆಚ್ಚು ರಭಸದಿಂದ ಏಳುವುದು, ಈ ಚಲನೆ ಯಾವಾಗಲೂ ಹೀಗೆಯೇ ಮುಂದುವರಿಯುತ್ತಿರುವುದು. ಧಾರ್ಮಿಕ ಜಗತ್ತಿನಲ್ಲಿಯೂ ಇದೇ ಚಲನೆ ಇರುವುದು.ಪ್ರತಿಯೊಂದು ದೇಶದ ಆಧ್ಯಾತ್ಮಿಕ ಇತಿಹಾಸದಲ್ಲಿಯೂ ಒಂದು ಏಳು ಬೀಳು ಇದ್ದೇ ಇದೆ. ದೇಶವು ಅವನತಿ ಕಡೆ ಧಾವಿಸುವುದು, ಎಲ್ಲಾ ಚೂರು ಚೂರಾಗುವಂತೆ ತೋರುವುದು. ಪುನಃ ಅದು ಶಕ್ತಿಯನ್ನು ಸಂಗ್ರಹಿಸಿ ಮೇಲೇಳುವುದು. ಮತ್ತೊಂದು ದೊಡ್ಡ ಅಲೆ ಬರುವುದು. ಅದು ಉಬ್ಬರದ ಅಲೆ ಇರಬಹುದು. ಆ ಅಲೆಯ ತುತ್ತ ತುದಿಯಲ್ಲಿ ಭಗವಂತನ ಸಂದೇಶವಾಹಕನೊಬ್ಬ ಮಹಾಪುರುಷನು ಯಾವಾಗಲೂ ಇರುವನು. ಅವನೇ ಸೃಷ್ಟಿಕರ್ತ. ಅವನೇ ಪುನಃ ಸೃಷ್ಟಿಸಿದ ವಸ್ತು ಕೂಡ ಆಗುವನು. ಅಲೆ ಮೇಲೇಳುವಂತೆ ಪ್ರಚೋದಿಸುವವನು ಅವನೇ, ದೇಶ ಮೇಲೇಳುವಂತೆ ಪ್ರಚೋದಿಸುವವನು ಅವನೇ, ಆದರೆ ಯಾವುದು ಅಲೆಯನ್ನು ಸೃಷ್ಟಿಸುವುದೋ ಅದೇ ಶಕ್ತಿ ಈ ಮಹಾವ್ಯಕ್ತಿಯನ್ನೂ ಸೃಷ್ಟಿಸುವುದು. ಇವೆರಡಕ್ಕೂ ಕ್ರಿಯೆ ಮತ್ತು ಪ್ರತಿಕ್ರಿಯೆಗಳಾಗುತ್ತಿರುತ್ತವೆ. ಅವನು ತನ್ನ ಅದ್ಭುತ ಶಕ್ತಿಯನ್ನೆಲ್ಲಾ ಸಮಾಜದ ಮೇಲೆ ಬೀರುವನು, ಅನಂತರ ಸಮಾಜ ಅವನನ್ನು ಮಹಾವ್ಯಕ್ತಿಯನ್ನಾಗಿ ಮಾಡುವುದು. ಇವರೇ ವಿಶ್ವವಿಖ್ಯಾತ ಚಿಂತಕರು. ಇವರೇ ಪ್ರವಾದಿಗಳು, ದೇವದೂತರು, ಭಗವಂತನ ಅವತಾರಗಳು.

ಪ್ರಪಂಚದಲ್ಲಿ ಒಂದೇ ಧರ್ಮ ಇರುವುದಕ್ಕೆ ಮಾತ್ರ ಸಾಧ್ಯ. ಒಬ್ಬನೇ ಪ್ರವಾದಿ ಇರುವುದಕ್ಕೆ ಸಾಧ್ಯ, ಒಬ್ಬನೇ ಅವತಾರ ಇರುವುದಕ್ಕೆ ಮಾತ್ರ ಸಾಧ್ಯ ಎಂದು ಜನ ಆಲೋಚಿಸುವರು. ಆದರೆ ಆ ಭಾವನೆ ಸತ್ಯವಲ್ಲ. ಈ ವಿಖ್ಯಾತ ದೇವದೂತರ ಜೀವನವನ್ನು ನೋಡಿದರೆ ಪ್ರತಿಯೊಬ್ಬರೂ ಯಾವುದೋ ಒಂದು ನಿರ್ದಿಷ್ಟ ಕೆಲಸ ವನ್ನು ಮಾಡುವುದಕ್ಕೆ, ಅದೊಂದನ್ನು ಮಾತ್ರ ಮಾಡುವುದಕ್ಕೆ ಬಂದಂತೆ ತೋರು ವುದು. ಇವುಗಳ ಮೊತ್ತದಲ್ಲಿ ಮಾತ್ರ ಸಾಮರಸ್ಯವಿದೆಯೆ ಹೊರತು ಒಂದರಲ್ಲಿ ಇಲ್ಲ. ಜನಾಂಗಗಳ ಜೀವನದಲ್ಲಿರುವಂತೆಯೇ ಯಾವುದೋ ಒಂದು ಜನಾಂಗ ಮಾತ್ರ ಸುಖ ಅನುಭವಿಸಲು ಬಂದಿಲ್ಲ. ಯಾರಿಗೂ ಹೀಗೆ ಸಾಧಿಸುವುದಕ್ಕೆ ಧೈರ್ಯವಿಲ್ಲ. ಜನಾಂಗಗಳ ದೈವೀ ಸಾಮರಸ್ಯದಲ್ಲಿ ಪ್ರತಿಯೊಂದು ಜನಾಂಗಕ್ಕೂ ಒಂದೊಂದು ಪಾತ್ರವಿದೆ. ಪ್ರತಿಯೊಂದು ಜನಾಂಗಕ್ಕೂ ತಾನು ಮಾಡುವ ಒಂದು ಕರ್ತವ್ಯವಿದೆ, ಸಾಧಿಸುವ ಒಂದು ಉದ್ದೇಶವಿದೆ. ಇವುಗಳ ಮೊತ್ತವೇ ಮಹಾ ಸಾಮರಸ್ಯ.

ಆದಕಾರಣ ಯಾವನೋ ಒಬ್ಬನೇ ದೇವದೂತ ಎಂದೆಂದಿಗೂ ಜಗತ್ತನ್ನು ಆಳಲಾರ. ಇದುವರೆಗೆ ಹೀಗೆ ಯಾರೂ ಜಯಪ್ರದರಾಗಿಲ್ಲ. ಮುಂದೆ ಹಾಗೆ ಆಗು ವಂತೆಯೂ ಇಲ್ಲ. ಪ್ರತಿಯೊಬ್ಬನೂ ಒಂದು ಅಂಶವನ್ನು ಮಾತ್ರ ತನ್ನ ಕಾಣಿಕೆಯಾಗಿ ಕೊಡಬಲ್ಲ. ಆ ಅಂಶದ ದೃಷ್ಟಿಯಿಂದ ಪ್ರತಿಯೊಬ್ಬ ದೇವದೂತನೂ ಪರ್ಯಾಯ ವಾಗಿ ಜಗತ್ತನ್ನೂ ಅದರ ಭವಿಷ್ಯವನ್ನೂ ಆಳುತ್ತಾನೆ.

ನಮ್ಮಲ್ಲಿ ಬಹುಪಾಲು ಹುಟ್ಟು ಸಗುಣೋಪಾಸಕರು. ನಾವೇನೋ ಸಿದ್ಧಾಂತ ಗಳನ್ನು ಕುರಿತು ಮಾತನಾಡುತ್ತೇವೆ. ತತ್ತ್ವವನ್ನು ಕುರಿತು ಮಾತನಾಡುತ್ತೇವೆ. ಅದು ಸರಿ. ನಮ್ಮ ಪ್ರತಿಯೊಂದು ಆಲೋಚನೆ ಮತ್ತು ಕ್ರಿಯೆ ಕೂಡ, ನಮಗೆ ಒಂದು ವ್ಯಕ್ತಿಯ ಮೂಲಕ ಸಿದ್ಧಾಂತವು ಬಂದಾಗ ಮಾತ್ರ ಅದು ಅರ್ಥವಾಗುವುದೆಂಬುದನ್ನು ತೋರಿಸುತ್ತದೆ. ಸ್ಥೂಲವಾದ ಒಬ್ಬ ಆದರ್ಶ ವ್ಯಕ್ತಿಯ ಮೂಲಕ ಅದು ಬಂದಾಗ ಮಾತ್ರ ಅದರಲ್ಲಿರುವ ಭಾವನೆಯನ್ನು ನಾವು ಗ್ರಹಿಸಬಹುದು. ಉದಾಹರಣೆಯ ಮೂಲಕ ಅದರ ಹಿಂದೆ ಇರುವ ಬೋಧನೆ ನಮಗೆ ಅರ್ಥವಾಗುವುದು. ನಮ್ಮಲ್ಲಿ ಎಲ್ಲರೂ ಉದಾಹರಣೆ ಇಲ್ಲದೆ, ವ್ಯಕ್ತಿಗಳಿಲ್ಲದೆ, ಸಿದ್ಧಾಂತವನ್ನು ಅರ್ಥ ಮಾಡಿ ಕೊಳ್ಳುವಷ್ಟು ಮುಂದುವರಿದು ಹೋಗಿದ್ದರೆ ಚೆನ್ನಾಗಿತ್ತು. ಆದರೆ ನಾವು ಹಾಗಿಲ್ಲ. ಆದಕಾರಣವೇ ಮಾನವಕೋಟಿಯ ಬಹುಪಾಲು ಜನರು ಬುದ್ಧ, ಕ್ರಿಸ್ತರೆಂದು ಆರಾಧಿಸಲ್ಪಡುವ ಮಹಾ ವ್ಯಕ್ತಿಗಳಡಿಯಲ್ಲಿ, ಅವತಾರ ಅಥವಾ ದೇವದೂತರಡಿ ಯಲ್ಲಿ ಶರಣಾಗಿರುವರು. ಮಹಮ್ಮದೀಯರು ಮೊದಲಿನಿಂದಲೂ ಅಂತಹ ಆರಾ ಧನೆಗೆ ವಿರೋಧವಾಗಿದ್ದರು. ದೇವದೂತರು ಮತ್ತು ಸಂದೇಶಕರನ್ನು ಪೂಜಿಸು ವುದಾಗಲಿ, ಅವರಿಗೆ ಗೌರವ ತೋರುವುದಾಗಲೀ, ಯಾವುದನ್ನೂ ಅವರು ಲಕ್ಷಿಸು ವುದಿಲ್ಲ. ಆದರೆ ಕಾರ್ಯತಃ ಬಂದಾಗ ಒಬ್ಬ ದೇವದೂತನ ಬದಲು ಸಾವಿರಾರು ಜನ ಮಹಾತ್ಮರನ್ನು ಪೂಜಿಸಬೇಕಾಯಿತು. ನಾವು ವಾಸ್ತವಾಂಶಕ್ಕೆ ವಿರೋಧವಾಗಿ ಹೋಗಲಾರೆವು. ನಾವು ವ್ಯಕ್ತಿಗಳನ್ನು ಪೂಜಿಸದೆ ವಿಧಿಯಿಲ್ಲ. ಅದು ಒಳ್ಳೆಯದೆ. “ದೇವರೆ, ನಮಗೆ ತಂದೆಯನ್ನು ತೋರು” ಎಂದು ಜನ ನಿಮ್ಮ ದೇವದೂತನನ್ನು ಕೇಳಿದಾಗ ಅವನು ಕೊಟ್ಟ ಉತ್ತರವನ್ನು ನೆನೆಸಿಕೊಳ್ಳಿ. “ಯಾರು ನನ್ನನ್ನು ನೋಡು ವನೋ ಅವನು ತಂದೆಯನ್ನು ನೋಡಿರುವನು” ಎಂದನು. ದೇವರನ್ನು ಮನುಷ್ಯ ಎಂದಲ್ಲದೆ ಬೇರೆ ರೀತಿ ನಮ್ಮಲ್ಲಿ ಯಾರು ಊಹಿಸಿಕೊಳ್ಳಲು ಸಾಧ್ಯ? ನಾವು ಅವ ನನ್ನು ಮಾನವರಲ್ಲಿ ಮತ್ತು ಅವನ ಮೂಲಕ ಮಾತ್ರ ತಿಳಿದುಕೊಳ್ಳಬಲ್ಲೆವು. ಈ ಕೋಣೆಯಲ್ಲಿ ಬೆಳಕಿನ ಸ್ಪಂದನ ಎಲ್ಲಕಡೆಯಲ್ಲಿಯೂ ಇರುವುದು. ಆದರೆ ಏತಕ್ಕೆ ನಾವು ಎಲ್ಲಾ ಕಡೆಯಲ್ಲಿಯೂ ನೋಡಲಾರೆವು? ನೀವು ಆ ದೀಪದಲ್ಲಿ ಮಾತ್ರ ನೋಡಬಲ್ಲಿರಿ. ಭಗವಂತ ಸರ್ವಾಂತರ್ಯಾಮಿ ಎಂಬುದೊಂದು ಸಿದ್ಧಾಂತ. ಆದರೆನಾವು ಈಗ ಇರುವ ರೀತಿಯಲ್ಲಿ ಮಾನವನಲ್ಲಿ ಮಾತ್ರ ಅವನನ್ನು ನೋಡಲು ಸಾಧ್ಯ. ಅವನಲ್ಲಿ ಮಾತ್ರ ಗ್ರಹಿಸಲು ಸಾಧ್ಯ. ಈ ಮಹಾ ವ್ಯಕ್ತಿಗಳು ಬಂದಾಗ ಮಾನವನು ದೇವರನ್ನು ನೋಡುತ್ತಾನೆ. ಅವರು ನಮ್ಮಂತೆ ಬರದೆ ಬೇರೆ ವಿಧದಲ್ಲಿ ಬರುವರು. ನಾವು ಭಿಕ್ಷುಕರಂತೆ ಬರುವೆವು. ಅವರು ಚಕ್ರವರ್ತಿಗಳಂತೆ ಬರುವರು. ನಾವು ಗತಿಕೆಟ್ಟ ಅನಾಥರಂತೆ ದಾರಿ ಗೊತ್ತಿಲ್ಲದೆ ಬರುವೆವು. ನಾವೇನು ಮಾಡಬೇಕು? ನಮ್ಮ ಜೀವನದ ಅರ್ಥವೇ ನಮಗೆ ತಿಳಿಯದು. ನಾವು ಅದನ್ನು ಗ್ರಹಿಸಲಾರೆವು. ಇಂದು ಒಂದನ್ನು ಮಾಡುತ್ತಿರುವೆವು, ನಾಳೆ ಮತ್ತೊಂದು. ನಾವು ನೀರಿನಲ್ಲಿ ಅತ್ತ ಇತ್ತ ಚಲಿಸುತ್ತಿರುವ ಹುಲ್ಲಿನ ಎಸಳಿನಂತೆ, ಬಿರುಗಾಳಿಗೆ ಸಿಕ್ಕಿದ ಗರಿಯಂತೆ ಚಲಿಸುತ್ತಿರುವೆವು.

ಮಾನವ ಇತಿಹಾಸವನ್ನು ಪರಿಗಣಿಸಿದರೆ, ಈ ದೇವದೂತರು ಬರುವುದನ್ನು ಮತ್ತು ಅವರು ಬರುವಾಗಲೆ ತಾವು ಏನು ಮಾಡಬೇಕು, ಹೇಗೆ ಮಾಡಬೇಕೆಂಬುದನ್ನು ನಿಶ್ಚಯಿಸಿಕೊಂಡು ಬಂದಿರುವುದನ್ನು ನೋಡುವೆವು; ಯೋಜನೆಯೆಲ್ಲ ಅವರ ಮುಂದೆ ಇರುವುದು. ಅದರಿಂದ ಅವರು ಒಂದು ಕೂದಲೆಳೆಯಷ್ಟೂ ಕದಲುವುದಿಲ್ಲ. ಅವರು ಒಂದು ಉದ್ದೇಶದಿಂದ ಬರುವುದರಿಂದ ಅವರೊಂದು ಸಂದೇಶವನ್ನು ತರುವರು. ಅವರು ತರ್ಕಿಸಬೇಕಿಲ್ಲ. ಈ ಮಹಾಪುರುಷರು ತಾವು ಬೋಧಿಸುವುದನ್ನು ಎಂದಾ ದರೂ ತರ್ಕಿಸಿದರು ಎಂಬುದನ್ನು ಎಂದಾದರೂ ನೀವು ಕೇಳಿರುವಿರಾ ಅಥವಾ ನೋಡಿರುವಿರಾ? ಇಲ್ಲ, ಅವರಲ್ಲಿ ಯಾರೊಬ್ಬರೂ ಹಾಗೆ ಮಾಡಲಿಲ್ಲ. ಅವರು ನೇರವಾಗಿ ಮಾತನಾಡುವರು.ಅವರು ಏಕೆ ಆಲೋಚಿಸಬೇಕು? ಅವರು ಸತ್ಯವನ್ನು ನೋಡುವರು. ಅವರು ಸತ್ಯವನ್ನು ನೋಡುವುದು ಮಾತ್ರವಲ್ಲ, ಅದನ್ನು ತೋರು ತ್ತಾರೆ. ದೇವರಿರುವನೆ ಎಂದು ನೀವು ನನ್ನನ್ನು ಕೇಳಿದರೆ, ಹೌದು ಎನ್ನುತ್ತೇನೆ ನಾನು. ತಕ್ಷಣ ನೀವು ನಿನ್ನ ನಿಲುವಿಗೆ ಆಧಾರವೇನು ಎಂದು ಪ್ರಶ್ನಿಸುವಿರಿ. ನಿಮಗೆ ಅದನ್ನು ಯುಕ್ತಿಪೂರ್ವಕವಾಗಿ ದೃಢಪಡಿಸಬೇಕಾದರೆ, ಪಾಪ, ನಾನು ನನ್ನ ಸಾಮರ್ಥ್ಯ ವನ್ನೆಲ್ಲಾ ಉಪಯೋಗಿಸಬೇಕು. ನೀವು ಕ್ರಿಸ್ತನನ್ನು ದೇವರಿರುವನೆ ಎಂದು ಕೇಳಿದರೆ, ಹೌದು ಎನ್ನುತ್ತಿದ್ದ ಅವನು. ಅದಕ್ಕೆ ಏನಾದರೂ ಪ್ರಮಾಣವಿದೆಯೆ, ಎಂದು ಕೇಳಿದ್ದರೆ “ನೋಡಿ ದೇವನನ್ನು” ಎಂದು ಅವನು ತೋರುತ್ತಿದ್ದ. ನೋಡಿ, ಅವರಿಗೆ ಇದೊಂದು ಪ್ರತ್ಯಕ್ಷ ಅನುಭವ. ಯುಕ್ತಿಯ ಕಸರತ್ತು ಅಲ್ಲ. ಅವರೆಂದಿಗೂ ಕತ್ತಲೆ ಯಲ್ಲಿ ಅಲೆಯುವುದಿಲ್ಲ. ಪ್ರತ್ಯಕ್ಷ ಅನುಭವದ ದೃಢತೆ ಅವರಲ್ಲಿರುವುದು. ನಾನು ಎದುರಿಗಿರುವ ಮೇಜನ್ನು ನೋಡುತ್ತೇನೆ. ಅದಕ್ಕೆ ವಿರೋಧವಾಗಿ ಎಷ್ಟೇ ವಾದಿಸಿ ದರೂ ಈ ನಂಬಿಕೆಯನ್ನು ಅದು ಕಳೆಯಲಾರದು. ಇದು ಪ್ರತ್ಯಕ್ಷ ಅನುಭವ. ಅವರ ಆದರ್ಶದಲ್ಲಿ, ಎಲ್ಲಕ್ಕಿಂತ ಹೆಚ್ಚಾಗಿ ಅವರಲ್ಲಿ ಅವರ ಜೀವಿತೋದ್ದೇಶದಲ್ಲಿ, ಎಲ್ಲಕ್ಕಿಂತ ಹೆಚ್ಚಾಗಿ ಸ್ವಶಕ್ತಿಯಲ್ಲಿ ಅವರಿಗಿರುವ ನಂಬಿಕೆ ಅಂಥದು.ಈ ಮಹಾ ಪುರುಷರಿಗೆ ಇತರರೆಲ್ಲರಿಗಿಂತ ಆತ್ಮಶ್ರದ್ಧೆ ಹೆಚ್ಚು. ಸಾಧಾರಣ ಜನರು “ನೀನು ದೇವರನ್ನು ನಂಬುವೆಯಾ? ಪುನರ್ಜನ್ಮ ನಂಬುವೆಯಾ? ನೀನು ಈ ಸಿದ್ಧಾಂತವನ್ನು ಅಥವಾ ಆ ಮತತತ್ತ್ವವನ್ನು ನಂಬುವೆಯಾ?” ಎಂದು ಕೇಳುವರು. ಆದರೆ ಇಂತಹ ನಂಬಿಕೆಗಳಲ್ಲಿ ಮೂಲಭೂತವಾದ ಆತ್ಮಶ್ರದ್ಧೆಯೇ ಇರುವುದಿಲ್ಲ.ಅಯ್ಯೋ, ಯಾರಿಗೆ ಆತ್ಮವಿಶ್ವಾಸವಿಲ್ಲವೋ ಅವರು ಇತರರನ್ನು ಹೇಗೆ ನಂಬ ಬಲ್ಲರು? ನಾನಿರುವುದೇ ನನಗೆ ಅನುಮಾನ. ಈ ಕ್ಷಣ ನಾನು ಇರುವೆನು, ಯಾವುದೂ ನನ್ನನ್ನು ನಾಶಮಾಡಲಾರದು ಎಂದು ಭಾವಿಸುವೆನು. ಮರುಕ್ಷಣವೇ ಮೃತ್ಯುಭಯದಿಂದ ಕಂಪಿಸುವೆನು. ಈ ಕ್ಷಣ ನಾನು ಅಮೃತಾತ್ಮ ಎಂದು ಭಾವಿಸುವೆನು. ಮರುಕ್ಷಣ ಪೆಟ್ಟೊಂದು ಬಿದ್ದಾಗ ಅಸಹಾಯಕತೆಯಿಂದ ಒದ್ದಾಡುವೆನು. ನಾನು ಬದುಕಿರುವೆನೋ ಇಲ್ಲವೋ ಅದು ಗೊತ್ತಿಲ್ಲ. ಒಂದು ಕ್ಷಣ ನಾನು ಆಧ್ಯಾತ್ಮಿಕನೆಂದೂ ನೀತಿವಂತನೆಂದೂ ನಂಬುತ್ತೇನೆ. ಮರುಕ್ಷಣದಲ್ಲಿ ಒಂದು ಪೆಟ್ಟು ಬಿದ್ದಾಗ ತತ್ತರಿಸುತ್ತೇನೆ. ಇದು ಏತಕ್ಕೆ? ನನ್ನಲ್ಲಿ ಆತ್ಮಶ್ರದ್ಧೆಯೇ ಇಲ್ಲ, ನನ್ನ ನೈತಿಕ ಜೀವನದ ತಳಹದಿ ಚೂರು ಚೂರಾಗಿದೆ.

ಆದರೆ ಈ ಮಹಾಪುರುಷರಲ್ಲಿ ಯಾವಾಗಲೂ ಈ ಚಿಹ್ನೆಯನ್ನು ನೋಡುವಿರಿ– ಅವರಲ್ಲಿ ಅದಮ್ಯ ಆತ್ಮಶ್ರದ್ಧೆ ಇರುವುದು. ಇಂತಹ ಅದ್ಭುತ ಆತ್ಮಶ್ರದ್ಧೆ ಬಹಳ ಅಪರೂಪ. ನಮಗೆ ಇದು ಅರ್ಥವಾಗುವುದಿಲ್ಲ. ಆದಕಾರಣವೇ ಈ ಮಹಾವ್ಯಕ್ತಿಗಳು ತಮ್ಮ ವಿಷಯವಾಗಿ ಏನು ಹೇಳುತ್ತಾರೊ ಅದನ್ನು ಹೇಗೋ ವಿವರಿಸಲು ಯತ್ನಿಸು ತ್ತೇವೆ. ಅವರು ತಮ್ಮ ಸಾಕ್ಷಾತ್ಕಾರದ ವಿಷಯದಲ್ಲಿ ಏನು ಹೇಳುವರೋ ಅದನ್ನು ವಿವರಿಸುವುದಕ್ಕೆ ಜನರು ಇಪ್ಪತ್ತುಸಾವಿರ ಸಿದ್ಧಾಂತಗಳನ್ನು ಮಂಡಿಸುತ್ತಾರೆ. ನಾವು ನಮ್ಮನ್ನು ಆ ದೃಷ್ಟಿಯಿಂದ ನೋಡುತ್ತಿಲ್ಲ. ಆದಕಾರಣವೇ ನಾವು ಅವರನ್ನು ಅರ್ಥ ಮಾಡಿಕೊಳ್ಳಲಾರೆವು.

ಈ ಮಹಾವ್ಯಕ್ತಿಗಳು ಮಾತನಾಡಿದರೆ ಜಗತ್ತು ಕೇಳಲೇಬೇಕಾಗುವುದು. ಅವರು ಆಡುವ ಪ್ರತಿಯೊಂದು ಶಬ್ದವೂ ನೇರ. ಅದೊಂದು ಫಿರಂಗಿಯ ಗುಂಡಿನಂತೆ ಸ್ಪೋಟಿಸು ವುದು. ಮಾತಿನ ಹಿಂದೆ ಶಕ್ತಿ ಇಲ್ಲದೆ ಇದ್ದರೆ ಬರೀ ಮಾತಿನಲ್ಲಿ ಏನಿದೆ? ನೀವು ಯಾವ ಭಾಷೆಯಲ್ಲಿ ಮಾತನಾಡುವಿರೊ, ಹೇಗೆ ವಿಷಯಗಳನ್ನು ಅಣಿಮಾಡಿಕೊಂಡಿ ರುವೆಯೋ, ಅದರಿಂದೇನು ಪ್ರಯೋಜನ?ನೀವು ವ್ಯಾಕರಣ ಶುದ್ಧವಾಗಿ ಮಾತನಾಡು ವಿರೋ, ಸುಂದರಪದಗಳನ್ನು ಜೋಡಿಸಿರುವಿರೋ ಇದನ್ನು ಕಟ್ಟಿಕೊಂಡು ಪ್ರಯೋ ಜನವೇನು? ಇತರರಿಗೆ ಕೊಡುವುದಕ್ಕೆ ನಿಮ್ಮಲ್ಲಿ ಏನಾದರೂ ಇದೆಯೆ, ಇಲ್ಲವೆ, ಅದೇ ಪ್ರಶ್ನೆ. ಇದೊಂದು ಕೊಡುವ, ತೆಗೆದುಕೊಳ್ಳುವ ಪ್ರಸಂಗ, ಸುಮ್ಮನೆ ಕೇಳುವುದಲ್ಲ. ಕೊಡುವುದಕ್ಕೆ ಏನಾದರೂ ನಿಮ್ಮಲ್ಲಿ ಇದೆಯೇ? ಅದೇ ಮೊದಲನೆ ಪ್ರಶ್ನೆ. ಇದ್ದರೆ ಕೊಡಿ. ಮಾತು ಭಾವನೆಯನ್ನು ನೀಡುವುದು ಅಷ್ಟೆ. ಹಲವು ವಿಧದಲ್ಲಿ ನಾವು ಇದನ್ನು ಮಾತುತ್ತೇವೆ. ಕೆಲವು ವೇಳೆ ನಾವು ಮಾತನ್ನೇ ಆಡುವುದಿಲ್ಲ. ಸಂಸ್ಕೃತದ ಹಳೆಯ ಶ್ಲೋಕವೊಂದರಲ್ಲಿ ಹೀಗಿದೆ: “ಮರದ ಕೆಳಗೆ ಗುರು ಕುಳಿತಿ ರುವುದನ್ನು ನೋಡಿದೆ. ಗುರು ಹದಿನಾರು ವರುಷದ ಯುವಕ, ಶಿಷ್ಯ ಎಂಬತ್ತು ವರುಷದ ವೃದ್ಧ. ಗುರು ಮೌನವಾಗಿ ವ್ಯಾಖ್ಯಾನಮಾಡುತ್ತಿದ್ದನು. ಶಿಷ್ಯನ ಸಂಶಯ ಗಳೆಲ್ಲ ದೂರವಾದವು.”

ಕೆಲವು ವೇಳೆ ಅವರು ಮಾತನ್ನೇ ಆಡುವುದಿಲ್ಲ. ಆದರೂ ಸತ್ಯವನ್ನು ಒಬ್ಬರು ಇನ್ನೊಬ್ಬರಿಗೆ ತಿಳಿಸುವರು ಅವರು ಕೊಡುವುದಕ್ಕೆ ಬರುವರು. ಅವರು ಆಜ್ಞಾಪಿಸುವರು. ಅವರು ದೇವದೂತರು. ನಾವು ಅವರ ಆಜ್ಞೆಯನ್ನು ಪಾಲಿಸಬೇಕಾಗಿದೆ. ಏಸುವು ಯಾವ ಅಧಿಕಾರವಾಣಿಯಿಂದ ಮಾತನಾಡುತ್ತಾನೆ ಎಂಬುದನ್ನು ನೀವು ನಿಮ್ಮ ಶಾಸ್ತ್ರದಲ್ಲಿಯೇ ನೋಡುವುದಿಲ್ಲವೆ? “ಹೋಗಿ ಜಗತ್ತಿಗೆಲ್ಲ ಸಾರಿ. ನಾನು ನಿಮಗೆ ಆಜ್ಞಾಪಿಸಿರುವಂತೆ ನಡೆಯಬೇಕೆಂದು ಅವರಿಗೆಲ್ಲ ಹೇಳಿ.” ತನ್ನ ಸಂದೇಶ ದಲ್ಲಿದ್ದ ಶ್ರದ್ಧೆ ಅವನ ಬೋಧನೆಯಲ್ಲೆಲ್ಲಾ ಅನುರಣಿತವಾಗುತ್ತಿದೆ. ಜಗತ್ತು ಯಾರನ್ನು ದೇವದೂತರೆಂದು ಆರಾಧಿಸುವುದೋ ಅಂತಹ ಆಧ್ಯಾತ್ಮಿಕ ವ್ಯಕ್ತಿಗಳಲ್ಲೆಲ್ಲ ನೀವು ಇದನ್ನು ನೋಡುವಿರಿ.

ಇಂತಹ ಮಹಾಗುರುಗಳೇ ಜೀವಂತ ದೇವರು ಜಗತ್ತಿನಲ್ಲಿ. ಇವರನ್ನು ಅಲ್ಲದೆ ಇನ್ನಾರನ್ನು ನಾವು ಪೂಜಿಸಬೇಕು? ನನ್ನ ಮನಸ್ಸಿನಲ್ಲಿ ಭಗವಂತನ ಭಾವನೆಯನ್ನು ಕಲ್ಪಿಸಿಕೊಳ್ಳಲು ಯತ್ನಿಸುವೆನು. ನಾನು ಕಲ್ಪಿಸಿಕೊಳ್ಳುವುದೆಷ್ಟು ಅಲ್ಪ, ಎಷ್ಟು ಕೃತ್ರಿಮ ಎಂದು ಕಾಣುವುದು. ಅಂತಹ ದೇವರನ್ನು ಪೂಜಿಸುವುದು ಪಾಪ. ಜಗತ್ತಿನಲ್ಲಿ ಬಾಳಿದ ಇಂತಹ ಮಹಾವ್ಯಕ್ತಿಗಳ ಜೀವನವನ್ನು ಕಣ್ಣುಬಿಟ್ಟು ನೋಡುವೆನು. ನಾನು ಕಲ್ಪಿಸಿಕೊಳ್ಳುವ ಭಗವಂತನ ಭಾವನೆಗಳೆಲ್ಲಕ್ಕಿಂತ ಇವರು ಮೇಲು. ನನ್ನಿಂದ ಏನಾದರೂ ಕದ್ದವನನ್ನು ಜೈಲಿಗೆ ಕಳುಹಿಸುವಂತಹ ನನ್ನಂಥವನು ಎಂತಹ ದಯೆಯ ಭಾವನೆಯನ್ನು ಕಲ್ಪಿಸಿಕೊಳ್ಳಬಲ್ಲ? ನನ್ನ ಶ್ರೇಷ್ಠ ಕ್ಷಮಾ ಭಾವನೆಯಾದರೂ ಏನು? ಸ್ವಾರ್ಥವಲ್ಲದೆ ಬೇರೇನೂ ಅಲ್ಲ. ನಿಮ್ಮಲ್ಲಿ ಯಾರು ನಿಮ್ಮ ದೇಹವನ್ನು ಬಿಟ್ಟು ನೆಗೆದು ಹೋಗಬಲ್ಲಿರಿ? ಮನಸ್ಸನ್ನು ಬಿಟ್ಟು ನೆಗೆದು ಹೋಗಬಲ್ಲಿರಿ? ನಿಮ್ಮಲ್ಲಿ ಒಬ್ಬರಿಗೂ ಇದು ಸಾಧ್ಯವಿಲ್ಲ. ನೀವು ಹೇಗೆ ಬಾಳುತ್ತಿರುವಿರೋ ಅದಲ್ಲದೆ, ಮತ್ತಾವ ಪರಮ ಪ್ರೇಮದ ಭಾವನೆಯನ್ನು ಕಲ್ಪಿಸಿಕೊಳ್ಳಬಲ್ಲಿರಿ? ನಾವು ಯಾವುದನ್ನು ಅನು ಭವಿಸಿಲ್ಲವೋ ಅದರ ಭಾವನೆಯನ್ನು ಕಲ್ಪಿಸಿಕೊಳ್ಳಲಾರೆವು. ಆದಕಾರಣ ಭಗ ವಂತನನ್ನು ಕಲ್ಪಿಸಿಕೊಳ್ಳುವ ಪ್ರಯತ್ನವೆಲ್ಲ ನಿಷ್ಪ್ರಯೋಜನವೇ ಸರಿ. ಈ ಮಹಾ ವ್ಯಕ್ತಿಗಳಲ್ಲಿ ಕಣ್ಣಿಗೆ ಕಾಣುವ ಅಂಶಗಳಿವೆ, ಬರೀ ಆದರ್ಶವಲ್ಲ; ನನ್ನ ಭಾವನೆಗೂ ನಿಲುಕದ ಪ್ರೀತಿ, ದಯೆ, ಪಾವಿತ್ರ್ಯ ಮುಂತಾದುವು ಈ ಮಹಾವ್ಯಕ್ತಿಗಳಲ್ಲಿ ಇವೆ. ಈ ಪವಿತ್ರಾತ್ಮರ ಪಾದಗಳಿಗೆ ನಮಿಸಿ ಇವರನ್ನು ದೇವರೆಂದು ಪೂಜಿಸುವುದರಲ್ಲಿ ಆಶ್ಚರ್ಯವೇನಿದೆ? ಇತರರು ತಾನೆ ಮತ್ತೆ ಬೇರೆ ಏನು ಮಾಡಬಲ್ಲರು?ಒಬ್ಬ ಬೇಕಾ ದಷ್ಟು ಮಾತನಾಡಬಹುದು. ಆದರೆ ಕಾರ್ಯತಃ ಏನು ಮಾಡಬಲ್ಲ ಎಂಬುದನ್ನು ನೋಡಲಾಶಿಸುವೆನು. ಮಾತು ಅನುಷ್ಠಾನವಲ್ಲ. ದೇವರು, ನಿರಾಕಾರ ಮುಂತಾದು ವನ್ನೆಲ್ಲಾ ಕುರಿತು ಮಾತನಾಡುವುದೇನೋ ಒಳ್ಳೆಯದು. ಆದರೆ ಈ ಮಾನವದೇವರೆ ಎಲ್ಲಾ ದೇಶದ, ಎಲ್ಲಾ ಜನಾಂಗದ ನಿಜವಾದ ದೇವರು. ಎಲ್ಲಿಯವರೆಗೆ ಮನುಷ್ಯ ಮನುಷ್ಯನಾಗಿರುವನೋ ಅಲ್ಲಿಯವರೆಗೆ ಈ ದೇವಾಂಶ ಸಂಭೂತರನ್ನು ಹಿಂದೆ ಪೂಜಿಸುತ್ತಿದ್ದನು, ಮುಂದೆಯೂ ಪೂಜಿಸುವನು. ನಿಜವಾಗಿ ಒಂದು ಸತ್ಯವಿದೆ ಎನ್ನುವುದರ ಶ್ರದ್ಧೆ ಅಲ್ಲಿರುವುದು, ಭರವಸೆ ಅಲ್ಲಿರುವುದು. ಸುಮ್ಮನೆ ಅಸ್ಪಷ್ಟವಾದ ಸಿದ್ಧಾಂತದಿಂದ ಏನು ಪ್ರಯೋಜನ?

ನಾನು ನಿಮಗೆ ಹೇಳಬೇಕೆಂಬುದು ಇದು: ಇಂತಹ ಮಹಾಪುರುಷರನ್ನೆಲ್ಲಾ ಪೂಜಿಸುವುದಕ್ಕೆ ನನಗೆ ಸಾಧ್ಯವಾಯಿತು, ಮುಂದೆ ಬರುವವರನ್ನು ಪೂಜಿಸುವುದಕ್ಕೂನಾನು ಸಿದ್ಧನಾಗಿರುವೆನು. ಮಗನು ಯಾವ ಬಟ್ಟೆ ಹಾಕಿಕೊಂಡು ಎದುರಿಗೆ ಬಂದರೂ ತಾಯಿಗೆ ಗೊತ್ತಾಗುವುದು. ಅದು ಸಾಧ್ಯವಾಗದ್ದರೆ ಆಕೆ ನಿಜವಾಗಿ ಅವನ ತಾಯಿಯಲ್ಲ ಎನ್ನಬೇಕಾಗುವುದು. ಯಾರೋ ಒಬ್ಬ ದೇವದೂತನಲ್ಲಿ ಮಾತ್ರ ನೀವು ಸತ್ಯ, ಪಾವಿತ್ರ್ಯ ದೈವತ್ವವನ್ನು ತಿಳಿದುಕೊಳ್ಳಲು ಸಾಧ್ಯ, ಇತರರಲ್ಲಿ ಇಲ್ಲ ಎಂದರೆ, ನಾನು ನಿರ್ಣಯಿಸುವುದೇನೆಂದರೆ, ನೀವು ಯಾರಲ್ಲಿಯೂ ದೈವತ್ವವನ್ನು ನೋಡಲಾರಿರಿ. ನೀವು ಕೇವಲ ಪದಗಳನ್ನು ನುಂಗಿರುವಿರಿ; ಹೇಗೆ ಯಾವುದಾದ ರೊಂದು ರಾಜಕೀಯ ಪಕ್ಷಕ್ಕೆ ಸೇರಿರುವಿರೋ ಹಾಗೆಯೋ ಯಾವುದೋ ಒಂದು ಧರ್ಮಪಕ್ಷಕ್ಕೂ ಸೇರಿರುವಿರಿ ಎಂದು. ಆದರೆ ಇದು ಧರ್ಮವಲ್ಲ. ಕೆಲವು ಮೂಢ ರಿರುವರು, ಅವರು ಹತ್ತಿರದಲ್ಲಿ ಚೆನ್ನಾದ ಸಿಹಿನೀರಿದ್ದರೂ ಉಪ್ಪು ನೀರನ್ನು ಬಳಸುವರು. ಏಕೆ ಹಾಗೆ ಮಾಡುತ್ತೀಯೆ ಎಂದು ಕೇಳಿದರೆ ಆ ಉಪ್ಪುನೀರಿನ ಬಾವಿ ಯನ್ನು ತನ್ನ ಅಪ್ಪ ತೋಡಿಸಿದ್ದ ಎನ್ನುವರು. ನನ್ನ ಅಲ್ಪ ಅನುಭವದಲ್ಲಿ ನಾನು ಈ ಜ್ಞಾನವನ್ನು ಶೇಖರಿಸಿರುವೆನು. ಯಾವ ಭೀಕರ ಕೃತ್ಯಕ್ಕೆ ನಾವು ಧರ್ಮವನ್ನು ದೂರುತ್ತೇವೋ ಅದಕ್ಕೆ ಧರ್ಮ ಕಾರಣವೇ ಅಲ್ಲ. ಯಾವ ಧರ್ಮವೂ ಜನರನ್ನು ಹಿಂಸಿಸಲಿಲ್ಲ. ಯಾವ ಧರ್ಮವೂ ಮಂತ್ರವಾದಿಗಳನ್ನು ಸುಡಲಿಲ್ಲ. ಇಂತಹ ಕೆಲಸ ವನ್ನು ಯಾವ ಧರ್ಮವೂ ಎಂದಿಗೂ ಮಾಡಲಿಲ್ಲ. ಇಂತಹ ಭೀಕರ ಕೃತ್ಯಗಳನ್ನು ಮಾಡುವಂತೆ ಜನರನ್ನು ಯಾವುದು ಉದ್ರೇಕಿಸಿದ್ದು? ರಾಜಕೀಯವಲ್ಲದೆ ಎಂದೂ ಧರ್ಮವಲ್ಲ. ಇಂತಹ ರಾಜಕೀಯವನ್ನು ಧರ್ಮವೆಂದು ಕರೆದರೆ ಅದಕ್ಕೆ ಯಾರು ಹೊಣೆ?

ಪ್ರತಿಯೊಬ್ಬನೂ ತಮ್ಮ ದೇವದೂತನೊಬ್ಬನೇ ನಿಜವಾದ ದೇವದೂತನೆಂದರೆ ಅವನು ಹೇಳುವುದು ಸರಿಯಲ್ಲ. ಅವನಿಗೆ ಧರ್ಮದ ತಿಳುವಳಿಕೆಯೇ ಇಲ್ಲ. ಧರ್ಮ ಮಾತಲ್ಲ, ಸಿದ್ಧಾಂತವಲ್ಲ, ಅಥವಾ ಬೌದ್ಧಿಕ ಒಪ್ಪಿಗೆಯೂ ಅಲ್ಲ. ಅದು ನಮ್ಮ ಹೃದಯಾಂತರಾಳದಲ್ಲಿ ಭಗವಂತನನ್ನು ಸಾಕ್ಷಾತ್ಕಾರ ಮಾಡಿಕೊಳ್ಳುವುದು, ಅವನನ್ನು ಸ್ಪರ್ಶಿಸುವುದು. ವಿಶ್ವಾತ್ಮ ಮತ್ತು ಅವನ ಆವಿರ್ಭಾವಗಳೊಡನೆ ಒಂದಾದ ಆತ್ಮ ನಾನು ಎಂಬುದನ್ನು ಅನುಭವಿಸುವುದೇ ಧರ್ಮ. ನೀವು ನಿಜವಾಗಿಯೂ ತಂದೆಯ ಮನೆಗೆ ಪ್ರವೇಶಿಸಿದ್ದರೆ, ನೀವು ಅವನ ಮಕ್ಕಳನ್ನೇ ನೋಡಿಯೂ ಹೇಗೆ ನಿಮಗೆ ಅವರ ಪರಿಚಯವಾಗದೆ ಇರುವುದು? ನೀವು ಅವರನ್ನು ತಿಳಿಯಲಾರದೆ ಇದ್ದರೆ ನೀವು ತಂದೆಯ ಮನೆಗೇ ಹೋಗಿಲ್ಲ ಎಂದಾಯಿತು. ತಾಯಿಗೆ ಮಗು ಯಾವ ವೇಷದಲ್ಲಿದ್ದರೂ ಗೊತ್ತಾಗುವುದು. ಅವನು ಎಷ್ಟೇ ಬೇರೆ ವೇಷದಲ್ಲಿದ್ದರೂ ಕಂಡು ಹಿಡಿಯಬಲ್ಲಳು. ಪ್ರತಿಯೊಂದು ದೇಶದಲ್ಲಿ ಮತ್ತು ಪ್ರತಿಯೊಂದು ಕಾಲದಲ್ಲಿ ಇರುವ ಮಹಾವ್ಯಕ್ತಿಗಳನ್ನು ಒಪ್ಪಿಕೊಳ್ಳಿ. ಅವರಲ್ಲಿ ಪರಸ್ಪರ ಭೇದವಿಲ್ಲ ಎಂಬು ದನ್ನು ಅರಿಯಿರಿ. ಎಲ್ಲಿ ನಿಜವಾದ ಧರ್ಮವಿದೆಯೋ, ಜೀವಿಗೆ ಭಗವಂತನ ಸ್ಪರ್ಶ ವಾಗಿದೆಯೋ, ಭಗವಂತನೊಂದಿಗೆ ಸಂಬಂಧ ವುಂಟಾಗಿದೆಯೋ – ಅಲ್ಲೆಲ್ಲಾ ಹೃದಯ ವಿಶಾಲವಾಗುವುದು, ಎಲ್ಲರ ಹೃದಯಾಂತರಾಳದಲ್ಲಿ ಬೆಳಗುತ್ತಿರುವ ದಿವ್ಯಜ್ಯೋತಿ ಯನ್ನು ನೋಡಲು ಸಾಧ್ಯವಾಗುವುದು. ಈ ದೃಷ್ಟಿಯಲ್ಲಿ ಕೆಲವು ಮಹಮ್ಮದೀಯರು ಇನ್ನೂ ಹಿಂದುಳಿದಿರುವರು. ಪಂಥೀಯತೆ ಹೆಚ್ಚು ಅವರಲ್ಲಿ.ಅವರ ಪಲ್ಲವಿಯೆ “ಅಲ್ಲಾ ಒಬ್ಬನೆ ದೇವರು, ಅವನ ದೂತನೇ ಮಹಮ್ಮದ” ಎಂಬುದು. ಇದರಾಚೆ ಇರುವುದೆಲ್ಲ ಕೆಟ್ಟದ್ದು ಮಾತ್ರವಲ್ಲ ತಕ್ಷಣ ಅದನ್ನು ನಾಶ ಮಾಡಬೇಕೆಂದು ಭಾವಿಸು ವರು. ಸ್ತ್ರೀಪುರುಷರು ಯಾರಾದರೂ ಆಗಲಿ, ಯಾರು ತಾವು ನಂಬುವುದನ್ನು ಒಪ್ಪುವುದಿಲ್ಲವೊ ಅವರನ್ನೆಲ್ಲಾ ಈ ಕ್ಷಣವೇ ಕೊಲ್ಲಬೇಕು. ಈ ಧರ್ಮಕ್ಕೆ ಸೇರದೇ ಇರುವುದನ್ನೆಲ್ಲಾ ಧ್ವಂಸ ಮಾಡಬೇಕು. ಇದನ್ನಲ್ಲದೆ ಬೇರೆಯದನ್ನು ಬೋಧಿಸುವ ಗ್ರಂಥಗಳನ್ನೆಲ್ಲಾ ದಹಿಸಬೇಕು. ಪೆಸಿಫಿಕ್​ ಸಾಗರದಿಂದ ಅಟ್ಲಾಂಟಿಕ್​ ಸಾಗರದವರೆಗೆ ಐನೂರು ವರ್ಷಗಳ ಕಾಲ ರಕ್ತಕಾಲುವೆ ಹರಿಸಿದರು ಇವರು. ಇದೇ ಇಸ್ಲಾಂ ಮತ. ಆದರೂ ಮಹಮ್ಮದೀಯರಲ್ಲಿಯೂ ಎಲ್ಲಿಯಾದರೂ ಒಬ್ಬ ತತ್ತ್ವಜ್ಞಾನಿ ಇದ್ದರೆ ಅವನು ನಿಜವಾಗಿ ಈ ಭೀಕರ ಕೃತ್ಯವನ್ನು ವಿರೋಧಿಸುತ್ತಿದ್ದನು. ಹಾಗೆ ಮಾಡಿ ಭಗವಂತ ತನ್ನನ್ನು ಸ್ಪರ್ಶಿಸಿದ್ದಾನೆ, ತಾನು ಸತ್ಯಾಂಶವನ್ನು ಸಾಕ್ಷಾತ್ಕಾರ ಮಾಡಿಕೊಂಡಿ ದ್ದೇನೆ, ಎಂಬುದನ್ನು ತೋರಿಸುತ್ತಿದ್ದನು.ಅವನು ಧರ್ಮದೊಂದಿಗೆ ಸರಸವಾಡು ತ್ತಿರಲಿಲ್ಲ. ಅವನು ಮಾತನಾಡುತ್ತಿದ್ದುದು ಯಾವುದೋ ತನ್ನ ವಂಶದವರ ಧರ್ಮವನ್ನಲ್ಲ. ಸ್ವಂತ ಅನುಭವದ ಸತ್ಯವನ್ನುಹೇಳುತ್ತಿದ್ದನು.

ಆಧುನಿಕ ವಿಕಾಸವಾದದ ಜೊತೆ ಜೊತೆಗೇ ಪುರಾತನಪ್ರವಣತೆಯೂ \enginline{(atavism)} ಇದೆ; ಧರ್ಮದ ಹಳೆಯ ಭಾವನೆಗಳನ್ನು ಸ್ವೀಕರಿಸುವ ಪ್ರವೃತ್ತಿಯೂ ಇದೆ. ತಪ್ಪಾದರೂ ಚಿಂತೆಯಿಲ್ಲ. ಯಾವುದಾದರೂ ಹೊಸ ಭಾವನೆಗಳನ್ನು ಆಲೋಚಿ ಸೋಣ. ಹಾಗೆ ಮಾಡುವುದು ಮೇಲು. ಗುರಿಯನ್ನು ಮುಟ್ಟಲು ಏಕೆ ಯತ್ನಿಸ ಬಾರದು? ಸೋಲಿನಿಂದ ನಾವು ಹೆಚ್ಚು ಬುದ್ಧಿವಂತರಾಗುವೆವು. ಕಾಲ ಅನಂತವಾಗಿದೆ. ಗೋಡೆಯನ್ನು ನೋಡು. ಗೋಡೆ ಎಂದಾದರೂ ಸುಳ್ಳು ಹೇಳೀತೆ? ಅದು ಯಾವಾ ಗಲೂ ಒಂದು ಗೋಡೆಯೇ ಆಗಿರುತ್ತದೆ. ಮನುಷ್ಯ ಸುಳ್ಳನ್ನು ಹೇಳುತ್ತಾನೆ. ಆದರೆ ಅವನು ಒಬ್ಬ ದೇವನೂ ಆಗಬಲ್ಲ. ಏನನ್ನಾದರೂ ಮಾಡುವುದು ಮೇಲು. ಅದು ತಪ್ಪಾದರೂ ಚಿಂತೆಯಿಲ್ಲ. ಇದು ಏನೂ ಮಾಡದೇ ಇರುವುದಕ್ಕಿಂತ ಮೇಲು. ಹಸು ಎಂದಿಗೂ ಸುಳ್ಳು ಹೇಳುವುದಿಲ್ಲ. ಆದರೆ ಅದು ಯಾವಾಗಲೂ ಹಸುವಾಗಿಯೇ ಉಳಿಯುವುದು. ಏನಾದರೂ ಮಾಡಿ, ಏನಾದರೂ ಯೋಚಿಸಿ ಅದು ತಪ್ಪೋ, ಸರಿಯೋ ಚಿಂತೆಯಿಲ್ಲ. ಹೇಗೋ ಏನಾದರೂ ಆಲೋಚಿಸಿ. ನಮ್ಮ ಪೂರ್ವಿಕರು ಹೀಗೆ ಆಲೋಚಿಸಲಿಲ್ಲ ಎಂದು ಸುಮ್ಮನೆ ಕುಳಿತುಕೊಂಡು ಅನುಭವಿಸುವ ಸ್ವಭಾವ ವನ್ನೆಲ್ಲ, ಆಲೋಚಿಸುವ ಸ್ವಭಾವವನ್ನೆಲ್ಲ ಕಳೆದುಕೊಳ್ಳಬೇಕೆ? ಇದಕ್ಕಿಂತ ನಾವು ಮೃತ್ಯುವಶರಾಗುವುದು ಮೇಲು! ಸಜೀವ ಭಾವನೆಗಳಿಲ್ಲದೆ ಇದ್ದರೆ, ಧಾರ್ಮಿಕ ವಿಷಯದಲ್ಲಿ ನಮ್ಮಲ್ಲಿ ನಿಶ್ಚಿತ ಅಭಿಪ್ರಾಯಗಳಿಲ್ಲದೆ ಇದ್ದರೆ ನಾವು ಬಾಳಿ ಪ್ರಯೋ ಜನವೇನು? ನಾಸ್ತಿಕರಿಗಾದರೂ ಒಂದು ಭರವಸೆ ಇದೆ. ಏಕೆಂದರೆ ಇತರರ ಆಲೋ ಚನೆಗಳನ್ನು ಒಪ್ಪದೇ ಇದ್ದರೂ ತಾವೇ ಸ್ವತಃ ಆಲೋಚಿಸುವರು. ತಾವೇ ಸ್ವತಃ ಏನನ್ನೂ ಆಲೋಚನೆ ಮಾಡದವರು ಇನ್ನೂ ಧಾರ್ಮಿಕ ಪ್ರಪಂಚದಲ್ಲಿ ಕಣ್ಣನ್ನೇ ತೆರೆದಿಲ್ಲ. ಅವರು ಹೇಗೋ ಇದ್ದ ಕಡೆ ಇದ್ದುಕೊಂಡು ಬದುಕಿರುವರು. ಅವರು ಆಲೋಚಿಸುವುದಿಲ್ಲ. ಅವರು ಧರ್ಮವನ್ನು ಲಕ್ಷಿಸುವುದಿಲ್ಲ. ಆದರೆ ನಂಬದವನು, ನಾಸ್ತಿಕನು ಧರ್ಮದ ಬಗ್ಗೆ ಆಳವಾಗಿ ಯೋಚಿಸುತ್ತಾನೆ. ಹೋರಾಡುತ್ತಾನೆ. ಆದ ಕಾರಣ ಏನನ್ನಾದರೂ ಆಲೋಚಿಸಿ. ಧರ್ಮಕ್ಕಾಗಿ ಹೋರಾಡಿ. ನೀವು ಸೋತರೂ ಚಿಂತೆಯಿಲ್ಲ. ಯಾವುದೋ ಒಂದು ವಿಚಿತ್ರ ಸಿದ್ಧಾಂತ ನಿಮಗೆ ಸಿಕ್ಕಿದರೂ ಚಿಂತೆಯಿಲ್ಲ. ನಿಮ್ಮನ್ನು ಒಬ್ಬ ವಿಚಿತ್ರ ಮನುಷ್ಯ ಎಂದು ಕರೆಯುತ್ತಾರೆ ಎಂದು ಅಂಜಿದರೆ ಅದನ್ನು ನಿಮ್ಮಲ್ಲಿಯೇ ಇಟ್ಟುಕೊಳ್ಳಿ, ನೀವು ಅದನ್ನು ಇತರರಿಗೆ ಹೋಗಿ ಬೋಧಿಸ ಬೇಕಾಗಿಲ್ಲ. ಆದರೆ ಏನನ್ನಾದರೂ ಸಾಧಿಸಿ, ದೇವರಿಗಾಗಿ ಹೋರಾಡಿ. ಬೆಳಕು ಬಂದೇ ತೀರಬೇಕು. ಪ್ರತಿದಿನ ನನಗೆ ಯಾರಾದರೂ ಊಟ ಮಾಡಿಸಿದರೆ ಕೊನೆಗೆ ನನ್ನ ಕೈಗಳ ಉಪಯೋಗವೇ ನಿಂತು ಹೋಗುವುದು. ಕುರಿಯ ಮಂದೆಯಂತೆಯೇ ಒಬ್ಬರು ಮತ್ತೊಬ್ಬರನ್ನು ಅನುಸರಿಸುವುದರಿಂದಲೇ ಮಾನವನು ಧರ್ಮ ಜೀವನಕ್ಕೆ ಸತ್ತಂತೆ ಆಗು ವನು. ಜಡವಾಗಿ ಉಳಿದರೆ ಅದೇ ಮೃತ್ಯು. ಕಾರ್ಯೋನ್ಮುಖರಾಗಿ. ಎಲ್ಲಿ ಚಟುವಟಿಕೆ ಇದೆಯೊ ಅಲ್ಲಿ ಭಿನ್ನಾಭಿಪ್ರಾಯಗಳಿದ್ದೇ ಇರುವುವು. ಭಿನ್ನಾಭಿ ಪ್ರಾಯದಿಂದಲೇ ಜೀವನದಲ್ಲಿ ರಸ ಸೃಷ್ಟಿ. ಇದೇ ಸೌಂದರ್ಯ, ಇದೇ ಎಲ್ಲದರ ಕಲೆ. ವೈವಿಧ್ಯ ಇರುವುದರಿಂದಲೇ ಎಲ್ಲಾ ಸುಂದರವಾಗಿರುವುದು. ವೈವಿಧ್ಯವೇ ಜೀವನದ ಮೂಲ, ಜೀವನದ ಚಿಹ್ನೆ. ನಾವೇಕೆ ಇದಕ್ಕೆ ಅಂಜಬೇಕು?

ಮಹಾಪುರುಷರನ್ನು ಅರ್ಥಮಾಡಿಕೊಳ್ಳುವ ಸ್ಥಿತಿಗೆ ನಾವೀಗ ಬರುತ್ತಿ ರುವೆವು. ಧರ್ಮದಲ್ಲಿ ಹೇಳಿದುದನ್ನು ಸುಮ್ಮನೆ ಒಪ್ಪಿಕೊಳ್ಳುವ ಸ್ಥಿತಿಯನ್ನು ಬಿಟ್ಟು, ನಿಜವಾದ ಚಿಂತನೆ ಎಂಬುದು ನಡೆದಿದ್ದರೆ, ನಿಜವಾದ ಭಗವತ್​ಪ್ರೀತಿ ಇದ್ದರೆ, ಜೀವನು ದೇವರೆಡೆಗೆ ಹೋಗಿ ಕೆಲವು ತಾತ್ಕಾಲಿಕ ಅನುಭವವನ್ನಾದರೂ ಪಡೆದಿದ್ದರೆ – ಅದೇ ಐತಿಹಾಸಿಕ ಪ್ರಮಾಣ. ಆಗ ತಕ್ಷಣ “ಎಲ್ಲ ಸಂಶಯಗಳೂ ಎಂದೆಂದಿಗೂ ನಾಶವಾಗುವುವು. ನಮ್ಮ ಹೃದಯದ ವಕ್ರತೆಯೆಲ್ಲಾ ನೇರವಾಗುವುದು. ನಮ್ಮ ಬಂಧನಗಳೆಲ್ಲಾ ಕಳಚಿ ಹೋಗುವುವು. ನಮ್ಮ ಕರ್ಮವೆಲ್ಲಾ ದಗ್ಧವಾಗುವುದು” ಏಕೆಂದರೆ ಯಾರು ಅಂತರತಮನೋ, ದೂರತಮನೋ, ಅವನನ್ನು ಅಂಥವನು ಕಂಡಿರುವನು. ಇದೇ ಧರ್ಮ, ಧರ್ಮದ ಸರ್ವಸ್ವವೇ ಇದು. ಉಳಿದುವೆಲ್ಲ – ಸಿದ್ಧಾಂತ, ನಂಬಿಕೆ, ಇತ್ಯಾದಿಗಳು ಪ್ರತ್ಯಕ್ಷಾನುಭವಕ್ಕೆ ಒಯ್ಯುವ ಹಲವು ಮಾರ್ಗಗಳು ಅಷ್ಟೆ. ನಾವೀಗ ಹೋರಾಡುತ್ತಿರುವುದು ಬರೀ ಬುಟ್ಟಿಗಾಗಿ, ಅದರಲ್ಲಿರುವ ಹಣ್ಣುಗಳೆಲ್ಲಾ ಚರಂಡಿ ಪಾಲಾಗಿವೆ.

ದೇವರ ವಿಷಯದಲ್ಲಿ ಇಬ್ಬರು ಜಗಳ ಕಾಯುತ್ತಿದ್ದರೆ ಅವರನ್ನು “ನೀನು ದೇವರನ್ನು ಕಂಡಿರುವೆಯಾ? ಆಧ್ಯಾತ್ಮಿಕ ಅನುಭವ ನಿನಗೆ ಆಗಿದೆಯೆ” ಎಂದು ಪ್ರಶ್ನೆ ಹಾಕಿ. ಒಬ್ಬನು ಕ್ರಿಸ್ತನೊಬ್ಬನೇ ದೇವದೂತನೆನ್ನುವನು. ಆಗಲಿ ಅವನು ಕ್ರಿಸ್ತನನ್ನು ನೋಡಿರುವನೇ? “ನಿನ್ನ ತಂದೆಯಾದರೂ ನೋಡಿರುವನೇ?” “ಇಲ್ಲ ಸ್ವಾಮಿ.” “ನಿನ್ನ ತಾತ ನೋಡಿದ್ದನೇ?” “ಇಲ್ಲ ಸ್ವಾಮಿ.” “ನೀನು ನೋಡಿ ದ್ದೀಯ?” “ಇಲ್ಲ ಸ್ವಾಮಿ” “ಹಾಗಾದರೆ ನೀವೆಲ್ಲ ಜಗಳ ಕಾಯುವುದೇತಕ್ಕೆ? ಹಣ್ಣುಗಳೆಲ್ಲ ಚರಂಡಿಯಲ್ಲಿ ಬಿದ್ದುಹೋಗಿವೆ. ನೀವು ಬುಟ್ಟಿಯ ವಿಷಯದಲ್ಲಿ ಹೋರಾಡುತ್ತಿರುವಿರಿ.” ವಿವೇಚನೆಯುಳ್ಳ ಸ್ತ್ರೀಪುರುಷರು ಯಾರೇ ಆಗಲಿ ಹೀಗೆ ಜಗಳ ಕಾಯುವುದಕ್ಕೆ ನಾಚಿಕೆ ಪಡಬೇಕು.

ಈ ದೇವದೂತರು, ಪ್ರವಾದಿಗಳು ಎಲ್ಲರೂ ಮಹಾವ್ಯಕ್ತಿಗಳು ಸತ್ಯ. ಏಕೆಂದರೆ ಪ್ರತಿಯೊಬ್ಬರೂ ಒಂದು ಉದಾತ್ತ ಭಾವನೆಯನ್ನು ಬೋಧಿಸಲು ಬಂದರು. ಉದಾ ಹರಣೆಗೆ ಭರತಖಂಡದ ದೇವದೂತರನ್ನು ತೆಗೆದುಕೊಳ್ಳಿ. ಅವರು ಧರ್ಮಸಂಸ್ಥಾಪಕ ರಲ್ಲೆಲ್ಲಾ ಪ್ರಾಚೀನರಾದವರು. ಮೊದಲು ನಾವು ಕೃಷ್ಣನನ್ನು ತೆಗೆದುಕೊಳ್ಳೋಣ. ನೀವು ಗೀತೆಯನ್ನು ಓದಿರಬಹುದು.ಆ ಪುಸ್ತಕದ ಉದ್ದಕ್ಕೂ ಇರುವ ಒಂದು ಭಾವನೆಯೇ ಅನಾಸಕ್ತಿ. ಅನಾಸಕ್ತರಾಗಿರಿ. ನಮ್ಮ ಹೃದಯದ ಪ್ರೇಮ ಒಂದು ವ್ಯಕ್ತಿಗೆ ಮಾತ್ರ ಮೀಸಲು. ಯಾರು ಅದು? ಯಾರು ಬದಲಾಯಿಸುವುದಿಲ್ಲವೋ ಅವನು. ಅವನಾರು? ಅವನೇ ದೇವರು. ಬದಲಾಯಿಸುತ್ತಿರುವ ಯಾರಿಗೂ ನಿಮ್ಮ ಹೃದಯವನ್ನು ತೆರುವ ತಪ್ಪು ಮಾಡಬೇಡಿ. ಏಕೆಂದರೆ ಅದರಿಂದ ದುಃಖಪ್ರಾಪ್ತಿ. ನೀವು ಅದನ್ನು ಒಂದು ವ್ಯಕ್ತಿಗೆ ಕೊಡಬಹುದು. ಆದರೆ ಆ ವ್ಯಕ್ತಿಯು ಕಾಲವಾದರೆ ದುಃಖವೇ ಪ್ರತಿಫಲ. ನೀವು ಅದನ್ನು ಒಬ್ಬ ಸ್ನೇಹಿತನಿಗೆ ಕೊಡಬಹುದು. ಆದರೆ ನಾಳೆ ಅವನು ನಿಮ್ಮ ವೈರಿಯಾಗಬಹುದು. ಅದನ್ನೇ ನಿಮ್ಮ ಪತಿಗೆ ಕೊಟ್ಟರೆ ಒಂದು ದಿನ ಅವನು ನಿಮ್ಮೊಂದಿಗೆ ಜಗಳ ಕಾಯಬಹುದು. ನೀವು ಅದನ್ನು ನಿಮ್ಮ ಸತಿಗೆ ಕೊಡಬಹುದು. ಆದರೆ ಅವಳು ನಾಳೆಯಲ್ಲ, ನಾಡಿದ್ದೇ ಕಾಲವಾಗಬಹುದು. ಪ್ರಪಂಚ ಸಾಗುತ್ತಿ ರುವುದೇ ಹೀಗೆ; ಆದಕಾರಣವೇ ಕೃಷ್ಣ ಗೀತೆಯಲ್ಲಿ ಹೇಳುವುದು, ದೇವರೊಬ್ಬನೇ ಬದಲಾವಣೆ ಆಗದಿರುವವನು ಎಂದು.ಅವನ ಪ್ರೇಮ ಎಂದಿಗೂ ನಿಷ್ಪ್ರಯೋಜನವಾಗುವುದಿಲ್ಲ. ನಾವೆಲ್ಲೇ ಇರಲಿ, ಏನನ್ನೇ ಮಾಡುತ್ತಿರಲಿ, ಅವನೆಂದಿಗೂ ದಯಾ ಮಯನಾದ, ಪ್ರೇಮಮಯನಾದ ಮೂರ್ತಿಯೆ. ನಾವೇನು ಮಾಡಿದರೂ ಅವನೆಂದಿಗೂ ಬದಲಾಯಿಸುವುದಿಲ್ಲ. ಅವನೆಂದಿಗೂ ಕೋಪ ಗೊಳ್ಳುವುದಿಲ್ಲ. ದೇವರು ಹೇಗೆ ನಮ್ಮ ಮೇಲೆ ಕೋಪಗೊಳ್ಳಬಲ್ಲನು? ನಿಮ್ಮ ಮಗು ಎಷ್ಟೋ ತುಂಟತನ ಮಾಡುವುದು. ಆದರೆ ನಿಮಗೆ ಅದರ ಮೇಲೆ ಕೋಪವೇ? ನಾವು ಏನಾಗಬೇಕೋ ಅದು ದೇವರಿಗೆ ಗೊತ್ತಿಲ್ಲವೆ? ನಾವೆಲ್ಲ ಇಂದೋ ನಾಳೆಯೋ ಪರಿಪೂರ್ಣರಾಗುವೆವು ಎಂಬುದು ಅವನಿಗೆ ಗೊತ್ತಿದೆ. ಅವನಿಗೆ ತಾಳ್ಮೆ ಇದೆ. ಅನಂತ ತಾಳ್ಮೆ ಇದೆ. ನಾವು ಅವನನ್ನೇ ಪ್ರೀತಿಸಬೇಕು. ಇರುವವರೆಲ್ಲಾ ಅವನಲ್ಲಿ ಮತ್ತು ಅವನ ದೆಸೆಯಿಂದ ಮಾತ್ರ ಬದುಕಿರಲು ಸಾಧ್ಯ. ಇದೇ ಮುಖ್ಯ ಸಂದೇಶ. ನೀವು ಸತಿಯನ್ನು ಪ್ರೀತಿಸಬೇಕು, ಆದರೆ ಸತಿಗಾಗಿ ಅಲ್ಲ. “ಪ್ರಿಯೆ, ಪತಿಗಾಗಿ ಎಂದೂ ಯಾರೂ ಪತಿಯನ್ನು ಪ್ರೀತಿಸಲಿಲ್ಲ, ಆದರೆ ದೇವರು ಅವನಲ್ಲಿರುವುದರಿಂದ ಪ್ರೀತಿಸುವರು.” ಸತಿಪತಿಯರ ಪ್ರೀತಿಯಲ್ಲಿಯೂ, ಸತಿ ಪತಿಯನ್ನು ಪ್ರೀತಿಸುವೆ ಎಂದು ಭಾವಿಸು ವಾಗಲೂ ಅಲ್ಲಿರುವ ನಿಜವಾದ ಆಕರ್ಷಣೆ ದೇವರು ಎಂದು ವೇದಾಂತ ತತ್ತ್ವ ಸಾರುವುದು. ಅವನೊಬ್ಬನೇ ಏಕ ಮಾತ್ರ ಆಕರ್ಷಕ, ಬೇರೊಬ್ಬರಿಲ್ಲ. ಆದರೆ ಅನೇಕ ವೇಳೆ ಸತಿಗೆ ಹೀಗೆಂದು ಗೊತ್ತಿಲ್ಲ. ಅಜ್ಞಾನದಿಂದ ಅವಳು ಸರಿಯಾದುದನ್ನೇ ಮಾಡುತ್ತಿರುವಳು, ಅದು ಭಗವಂತನನ್ನು ಪ್ರೀತಿಸುವುದು. ಅಜ್ಞಾನದಿಂದ ನಾವು ಅದನ್ನು ಮಾಡಿದಾಗ ದುಃಖಪ್ರಾಪ್ತಿ ಅಷ್ಟೇ. ಅದನ್ನು ತಿಳಿದು ಮಾಡಿದರೆ ಅದೇ ಮುಕ್ತಿ. ನಮ್ಮ ಶಾಸ್ತ್ರ ಹೇಳುವುದು ಇದನ್ನು, ಎಲ್ಲಿಯಾದರೂ ಪ್ರೇಮವಿದ್ದರೆ, ಎಲ್ಲಿಯಾದರೂ ಆನಂದದ ಒಂದು ಕಣವಿದ್ದರೆ, ಅದು ಆ ಭಗವಂತನ ಆವಿಭಾರ್ವವದ ಕಿರಣವೆಂದು ಅರಿಯಿರಿ. ಏಕೆಂದರೆ, ಅವನೇ ಸಂತೋಷ, ಆನಂದ, ಪ್ರೇಮ ಸ್ವರೂಪನು. ಅವನಿಲ್ಲದೆ ಇದ್ದರೆ ಯಾವ ಪ್ರೇಮವು ಇಲ್ಲ.

ಕೃಷ್ಣನು ಎಲ್ಲ ಕಡೆಯಲ್ಲಿ ಬೋಧಿಸುವ ರೀತಿಯೇ ಇದು. ಅವನು ಈ ವರ್ಚ ಸ್ಸನ್ನೇ ಇಡೀ ಜನಾಂಗದ ಮೇಲೆ ಬೀರಿರುವನು. ಹಿಂದೂ ಏನನ್ನಾದರೂ ಮಾಡಿದರೆ, ಅವನು ಒಂದು ತೊಟ್ಟು ನೀರು ಕುಡಿದರೂ “ಇದರಲ್ಲಿ ಪುಣ್ಯವಿದ್ದರೆ ಅದು ದೇವರಿಗೆ ಹೋಗಲಿ” ಎನ್ನುವನು. ಬೌದ್ಧರು ತಾವು ಯಾವುದಾದರೂ ಧರ್ಮದ ಕೆಲಸವನ್ನು ಮಾಡಿದರೆ “ಆ ಧರ್ಮಕಾರ್ಯದ ಪುಣ್ಯ ಪ್ರಪಂಚಕ್ಕೆ ಹೋಗಲಿ, ನಾನು ಮಾಡುವು ದರಲ್ಲಿ ಏನಾದರೂ ಪುಣ್ಯ ಬರುವ ಹಾಗಿದ್ದರೆ ಅದು ಪ್ರಪಂಚಕ್ಕೆ ಹೋಗಲಿ, ಪ್ರಪಂಚದ ಪಾಪ ನನಗೆ ಬರಲಿ” ಎನ್ನುವರು. ಹಿಂದೂ ತನಗೆ ದೇವರಲ್ಲಿ ಹೆಚ್ಚು ನಂಬಿಕೆ ಇದೆ ಎಂದು ಹೇಳುವನು. ದೇವರು ಸರ್ವಶಕ್ತ, ಸರ್ವಾಂತರ್ಯಾಮಿ, ಸರ್ವವ್ಯಾಪಿ ಎನ್ನುವನು. “ನಾನು ನನ್ನ ಪುಣ್ಯವನೆಲ್ಲಾ ಅವನಿಗೆ ಧಾರೆ ಎರೆದರೆ ಅದೇ ಮಹಾಯಜ್ಞ. ಅದು ಇಡೀ ವಿಶ್ವಕ್ಕೇ ಅರ್ಪಣೆಯಾಗುವುದು” ಎನ್ನುವನು ಹಿಂದು.

ಇದು ಅವನ ಸಂದೇಶದ ಒಂದು ಮುಖ. ಕೃಷ್ಣನ ಸಂದೇಶದ ಮತ್ತೊಂದು ಮುಖವು ಯಾವುದು? ಯಾರು ಪ್ರಪಂಚದ ಮಧ್ಯೆ ಇದ್ದು ಕೆಲಸಮಾಡುತ್ತಾರೆಯೋ, ಅದರ ಫಲವನ್ನೆಲ್ಲಾ ಭಗವಂತನಿಗೆ ಅರ್ಪಿಸುತ್ತಾರೆಯೋ, ಅವರನ್ನು ಜಗತ್ತಿನ ಪಾಪ ಮುಟ್ಟಲಾರದು. ಕಮಲವು ಹೇಗೆ ನೀರಿನಲ್ಲಿ ಹುಟ್ಟಿದರೂ ಅದರಿಂದ ಮೇಲೆದ್ದು ವಿಕಾಸವಾಗುವುದೋ ಹಾಗೆಯೇ ಕರ್ಮಫಲವನ್ನೆಲ್ಲಾ ಭಗವಂತನಿಗೆ ಅರ್ಪಿಸಿ ಕೆಲಸ ಮಾಡುವ ವನು ಕೂಡ. (ಗೀತೆ, \enginline{V.10}).

ಕೃಷ್ಣನು ತೀವ್ರವಾದ ಕರ್ಮವನ್ನು ಬೋಧಿಸುವ ಗುರುವಾಗಿ ಮತ್ತೊಂದು ಸಂದೇಶವನ್ನು ಸಾರುವನು. ಕರ್ಮ, ಕರ್ಮ, ಹಗಲು ರಾತ್ರಿ ಕರ್ಮ ಮಾಡಿ ಎನ್ನುವುದು ಗೀತೆ. ಹಾಗಾದರೆ ಶಾಂತಿ ಎಲ್ಲಿ? ನಾನು ಬದುಕಿರುವ ಪರ್ಯಂತ ಗಾಡಿಗೆ ಕಟ್ಟಿದ ಕುದುರೆಯಂತೆ ದುಡಿದು ಮಡಿದರೆ ನಾವಿಲ್ಲಿರುವುದು ಏತಕ್ಕೆ? ಕೃಷ್ಣ ಹೇಳುತ್ತಾನೆ, “ಹೌದು, ನಿನಗೆ ಶಾಂತಿ ಸಿಕ್ಕುವುದು. ಕರ್ಮಕ್ಕೆ ವಿಮುಖವಾಗುವುದಲ್ಲ, ಶಾಂತಿಗೆ ಮಾರ್ಗ.” ಸಾಧ್ಯವಾದರೆ ನಿಮ್ಮ ಕರ್ತವ್ಯಗಳನ್ನೆಲ್ಲಾ ಆಚೆಗೊಗೆದು ಗಿರಿಶಿಖರದ ಮೇಲಿರಿ. ಅಲ್ಲಿಯೂ ಮನಸ್ಸು ಯಾವಾಗಲೂ ಕೆಲಸ ಮಾಡುತ್ತಲೇ ಇರುವುದು. ಸುತ್ತುತ್ತಿರುವುದು, ಅಲೆಯುತ್ತಿರುವುದು. ಯಾರೋ ಸಂನ್ಯಾಸಿಯೊಬ್ಬರನ್ನು ಕೇಳಿದರು: “ಸ್ವಾಮಿ, ನಿಮಗೆ ಒಂದು ಸರಿಯಾದ ಸ್ಥಳ ದೊರಕಿತೆ? ನೀವು ಎಷ್ಟು ವರುಷಗಳಿಂದ ಹಿಮಾಲಯದಲ್ಲಿ ಅಲೆಯುತ್ತಿರುವಿರಿ?” ಎಂದು. “ನಲವತ್ತು ವರುಷ ಗಳಿಂದ” ಎಂದ ಸಂನ್ಯಾಸಿ. “ಆರಿಸಿಕೊಳ್ಳುವುದಕ್ಕೆ, ತಂಗುವುದಕ್ಕೆ ಎಷ್ಟೋ ಸುಂದರ ವಾದ ಸ್ಥಳಗಳಿವೆ, ಆದರೂ ನೀವು ಏತಕ್ಕೆ ಹಾಗೆ ಮಾಡಲಿಲ್ಲ?” ಎಂದುದಕ್ಕೆ: “ಏಕೆಂದರೆ ಈ ನಲವತ್ತು ವರುಷಗಳು ಮನಸ್ಸು ನನಗೆ ಆಸ್ಪದ ಕೊಡಲಿಲ್ಲ” ಎಂದ. “ಶಾಂತಿಯನ್ನು ಹುಡುಕೋಣ” ಎಂದು ನಾವೆಲ್ಲ ಹೇಳುತ್ತೇವೆ. ಆದರೆ ಹಾಗೆ ಮಾಡಲು ನಮ್ಮ ಮನಸ್ಸು ಅವಕಾಶ ಕೊಡುವುದಿಲ್ಲ.

ಟಾರ್ಟರ್​ ಜನಾಂಗದ ಒಬ್ಬನನ್ನು ಹಿಡಿದವನ ಕಥೆ ನಿಮಗೆ ಗೊತ್ತೆ? ಸಿಪಾಯಿಯೊಬ್ಬ ಊರ ಹೊರಗೆ ಹೋಗಿದ್ದ. ಸಿಪಾಯಿಗಳು ಇಳಿದು ಕೊಳ್ಳುವ ಸ್ಥಳದ ಸಮೀಪಕ್ಕೆ ಬಂದಾಗ ಅವನು “ನಾನೊಬ್ಬ ಟಾರ್ಟರ್​ನನ್ನು ಹಿಡಿದಿರುವೆನು” ಎಂದು ಅರಚಿಕೊಂಡ. “ಅವನನ್ನು ಕರೆದುಕೊಂಡು ಬಾ” ಯಾರೋ ಹೇಳಿದರು. “ಅವನು ಬರುವುದಿಲ್ಲ ಸ್ವಾಮಿ” ಎಂದ. “ಹಾಗಾದರೆ ನೀನೇ ಬಾ” ಎಂದರು. “ಅವನು ನನ್ನನ್ನು ಬರಲು ಬಿಡುವುದಿಲ್ಲ” ಎಂದ. ಹೀಗೆ ನಮ್ಮ ಮನಸ್ಸಿನಲ್ಲಿ ನಾವೊಂದು ಟಾರ್ಟರ್​ ಮನುಷ್ಯನನ್ನು ಹಿಡಿದಿರುವೆವು. ನಾವು ಅವನನ್ನು ಸುಮ್ಮನಿರಿಸಲಾರೆವು. ಅವನೂ ನಮ್ಮನ್ನು ಸುಮ್ಮನಿರಲು ಬಿಡುವುದಿಲ್ಲ. ನಾವೆಲ್ಲ ಸುಮ್ಮನಿರಿ, ಶಾಂತರಾಗಿ ಎಂದು ಏನೇನೋ ಹೇಳುವೆವು.ಪ್ರತಿಯೊಂದು ಮಗುವೂ ಬೇಕಾದರೆ ಹಾಗೆ ಹೇಳಬಲ್ಲುದು ಮತ್ತು ಹಾಗೆ ಮಾಡಬಲ್ಲೆ ಎಂದು ಭಾವಿಸುವುದು. ಆದರೆ ಅದು ಬಹಳ ಕಷ್ಟ. ನಾನು ಅದನ್ನು ಪ್ರಯತ್ನಿಸಿರುವೆನು. ನನ್ನ ಕರ್ತವ್ಯವನ್ನೆಲ್ಲ ನಾನು ಆಚೆಗೆಸೆದು ಬೆಟ್ಟದ ಮೇಲಕ್ಕೆ ಓಡಿಹೋದೆ. ನಾನು ಗುಹೆಯಲ್ಲಿದ್ದೆ. ದಟ್ಟ ಕಾನನದಲ್ಲಿದ್ದೆ. ಆದರೂ ನಾನೊಂದು ಟಾರ್ಟರ್​ ಹಿಡಿದೆ. ಏಕೆಂದರೆ ಸದಾ ಕಾಲ ದಲ್ಲಿ ನನ್ನದೇ ಪ್ರಪಂಚ ಯಾವಾಗಲೂ ನನ್ನಲ್ಲಿತ್ತು. ನನ್ನ ಮನಸ್ಸಿನಲ್ಲಿರುವುದೇ ಟಾರ್ಟರ್​. ಸುಮ್ಮನೆ ಹೊರಗಿನವರನ್ನು ದೂರಿ ಪ್ರಯೋಜನವಿಲ್ಲ. ನಮ್ಮಲ್ಲಿ ಟಾರ್ಟರ್​ ಇರುವಾಗ ಈ ಸನ್ನಿವೇಶ ಒಳ್ಳೆಯದು. ಈ ಸನ್ನಿವೇಶ ಕೆಟ್ಟದ್ದು ಎಂದು ಹೇಳುವೆವು. ನಾವು ಅವನನ್ನು ಸುಮ್ಮನಿರಿಸಿದರೆ ಶಾಂತರಾಗುವೆವು.

ಆದಕಾರಣವೇ ಕೃಷ್ಣನು, ಕರ್ತವ್ಯವಿಮುಖರಾಗಬೇಡಿ, ಅದನ್ನು ವೀರರಂತೆ ಸ್ವೀಕರಿಸಿ. ಅದರ ಫಲಾಪೇಕ್ಷೆಯ ಕಡೆ ಮನಸ್ಸಿಲ್ಲದಿರಲಿ ಎನ್ನುವನು. ಆಳಿಗೆ ಪ್ರಶ್ನೆ ಹಾಕುವ ಅಧಿಕಾರವಿಲ್ಲ. ಯೋಧನಿಗೆ ಆಲೋಚಿ ಸುವ ಹಕ್ಕಿಲ್ಲ. ಮುಂದೆ ಹೋಗಿ, ನೀವು ಮಾಡಬೇಕಾದ ಕೆಲಸದ ಸ್ವಭಾವವನ್ನು ಕುರಿತು ಹೆಚ್ಚು ಚಿಂತಿಸದಿರಿ.ನೀವು ನಿಃಸ್ವಾರ್ಥರೆ ಎಂದು ನಿಮ್ಮ ಮನಸ್ಸನ್ನು ಕೇಳಿ. ನೀವು ನಿಃಸ್ವಾರ್ಥರಾಗಿದ್ದರೆ, ಏನನ್ನೂ ಚಿಂತಿಸಬೇಕಾಗಿಲ್ಲ. ಯಾರೂ ನಿಮ್ಮನ್ನು ಹಿಡಿಯುವಂತಿಲ್ಲ. ಕಾರ್ಯೋನ್ಮುಖರಾಗಿರಿ. ಹತ್ತಿರ ಇರುವ ಕರ್ತವ್ಯವನ್ನು ಮಾಡಿ. ನೀವು ಇದನ್ನು ಮಾಡಿದರೆ ಕ್ರಮೇಣ “ಯಾರು ಪ್ರಚಂಡ ಕರ್ಮದ ಮಧ್ಯದಲ್ಲಿ ಅನಂತ ಶಾಂತಿಯನ್ನು ಪಡೆಯುವರೋ, ಯಾರು ಅನಂತ ಶಾಂತಿಯ ಮಧ್ಯದಲ್ಲಿ ಪ್ರಚಂಡ ಕರ್ಮವನ್ನು ಕಾಣುವರೋ ಅವರೇ ಯೋಗಿ, ಮಹಾತ್ಮ, ಪೂರ್ಣಾತ್ಮ” ಎಂಬ ಸತ್ಯ ಅರಿವಾಗುವುದು.

ಈ ಬೋಧನೆಯ ಪರಿಣಾಮ ಏನೆಂದರೆ, ಕರ್ಮಗಳೆಲ್ಲಾ ಪವಿತ್ರ ಕರ್ಮ ಗಳಾಗುವುವು. ಯಾವ ಕರ್ತವ್ಯವನ್ನೂ ಬರೀ ಕೀಳು ಕೆಲಸ ಎಂದು ಕರೆಯುವುದಕ್ಕೆ ನಮಗೆ ಅಧಿಕಾರವಿಲ್ಲ. ಪ್ರತಿಯೊಬ್ಬನ ಕರ್ಮವೂ ಸಿಂಹಾಸನದ ಮೇಲೆ ಕುಳಿತು ಆಳುವ ಚಕ್ರವರ್ತಿಯ ಕರ್ಮದಷ್ಟೇ ಶ್ರೇಷ್ಠ.

ಬುದ್ಧನ ಸಂದೇಶವನ್ನು ಕೇಳಿ. ಅದ್ಭುತವಾದ ಸಂದೇಶ ಅದು. ನಮ್ಮ ಹೃದಯ ದಲ್ಲಿ ಅದಕ್ಕೊಂದು ಸ್ಥಳ ಇದೆ. “ಎಲ್ಲಾ ಸ್ವಾರ್ಥವನ್ನೂ, ಸ್ವಾರ್ಥಕ್ಕೆ ಕಾರಣವಾಗು ವುದೆಲ್ಲವನ್ನೂ ನಿರ್ಮೂಲ ಮಾಡಿ. ಹೆಂಡತಿ, ಮಕ್ಕಳು ಮತ್ತು ಯಾವ ಸಂಸಾರವೂ ಇಲ್ಲದಿರಲಿ. ಪ್ರಾಪಂಚಿಕರಾಗಬೇಡಿ. ಸಂಪೂರ್ಣ ನಿಃಸ್ವಾರ್ಥರಾಗಿ” ಎನ್ನುತ್ತಾನೆ ಬುದ್ಧ. ಪ್ರಾಪಂಚಿಕ ತಾನು ನಿಃಸ್ವಾರ್ಥನಾಗಿರುವೆನು ಎಂದು ಭಾವಿಸುವನು. ಆದರೆ ಅವನು ತನ್ನ ಹೆಂಡತಿಯ ಮುಖವನ್ನು ನೋಡಿದರೆ ಅದು ಅವನನ್ನು ಸ್ವಾರ್ಥಿಯನ್ನಾಗಿ ಮಾಡುವುದು. ತಾಯಿ ತಾನು ಸಂಪೂರ್ಣ ನಿಃಸ್ವಾರ್ಥಳಾಗಿರುವೆನು ಎಂದು ಭಾವಿಸು ವಳು. ಆದರೆ ತನ್ನ ಮಗುವಿನ ಮುಖವನ್ನು ನೋಡಿದೊಡನೆ ಸ್ವಾರ್ಥ ತಲೆದೋರು ವುದು. ಸ್ವಾರ್ಥದ ಆಲೋಚನೆ ಎದ್ದೊಡನೆಯೆ, ಯಾವುದಾದರೂ ಸ್ವಾರ್ಥವನ್ನು ನಾವು ಮಾಡಲೆತ್ನಿಸಿದೊಡನೆಯೆ, ನಿಜವಾದ ಮನುಷ್ಯ, ಪೂರ್ಣ ಮನುಷ್ಯ ಹೋದಂತೆ, ಅವನೊಂದು ಮೃಗ, ಅವನೊಬ್ಬ ಗುಲಾಮ. ತನ್ನ ನೆರೆಹೊರೆಯವರನ್ನು ಮರೆಯುವನು. ಅವನು ಇನ್ನು ಮೊದಲು ನೀನು ನಂತರ ನಾನು ಎಂದು ಹೇಳುವುದಿಲ್ಲ. “ಮೊದಲು ನಾನು, ಅನಂತರ ಉಳಿದವರೆಲ್ಲ ತಮ್ಮ ತಮ್ಮನ್ನು ತಾವೇ ನೋಡಿಕೊಳ್ಳಲಿ” ಎನ್ನುವನು.

ಕೃಷ್ಣನ ಸಂದೇಶಕ್ಕೂ ನಮ್ಮಲ್ಲಿ ಒಂದು ಸ್ಥಳವಿದೆ ಎಂಬುದನ್ನು ನೋಡಿ ಆಯಿತು. ಆ ಸಂದೇಶವಿಲ್ಲದೇ ಇದ್ದರೆ ನಾವು ಚಲಿಸುವಂತೆಯೇ ಇಲ್ಲ. ನಾವು ಕೃಷ್ಣನ ಈ ಸಂದೇಶಕ್ಕೆ ಮನಗೊಡದೇ ಇದ್ದರೆ ಹೃತ್ಪೂರ್ವಕ ಶಾಂತಿ ಸಂತೋಷದಿಂದ ನಮ್ಮ ಪಾಲಿಗೆ ಬಂದ ಕರ್ತವ್ಯವನ್ನು ನಿರ್ವಹಿಸಲಾರೆವು: “ನಿಮ್ಮ ಕೆಲಸದಲ್ಲಿ ಸ್ವಲ್ಪ ದೋಷವಿದ್ದರೂ ಚಿಂತೆಯಿಲ್ಲ. ದೋಷವಿಲ್ಲದೆ ಯಾವ ಕೆಲಸವೂ ಇಲ್ಲ.” “ದೇವರ ಪಾಲಿಗೆ ಬಿಡು, ಅದರ ಫಲದಲ್ಲಿ ಆಸಕ್ತನಾಗಬೇಡ.”

ಮತ್ತೊಂದು ಸಂದೇಶಕ್ಕೂ ನಮ್ಮ ಹೃದಯದಲ್ಲಿ ಎಡೆ ಇದೆ. ಕಾಲ ಓಡುತ್ತಿದೆ. ಪ್ರಪಂಚ ಅನಿತ್ಯ. ಎಲ್ಲಾ ದುಃಖಮಯ. ನಿಮ್ಮ ಚೆನ್ನಾದ ಊಟ, ಉಪಚಾರ, ಬಟ್ಟೆಬರೆಗಳೊಂದಿಗೆ ಮನೆಯಲ್ಲಿ ಹಾಯಾಗಿ ನಿದ್ರಿಸುತ್ತಿರುವ ಸ್ತ್ರೀಪುರುಷರೆ, ಕೋಟ್ಯಂತರ ಜನ ಕೂಳಿಗಿಲ್ಲದೆ ಸಾಯುತ್ತಿರುವುದನ್ನು ನೀವು ಗಮನಿಸುವಿರಾ? ಇದೆಲ್ಲ ದುಃಖ ದುಃಖ ದುಃಖ ಎಂಬ ಮಹಾಸತ್ಯವನ್ನು ಕುರಿತು ಯೋಚಿಸಿ ನೋಡಿ! ಪ್ರಪಂಚಕ್ಕೆ ಮಗು ಪ್ರವೇಶಿಸುವಾಗ ಏನು ಮಾಡುವುದು ಗಮನಿಸಿ. ಅದು ಅಳುವುದು. ಮಗು ಅಳುವುದು. ಇದು ಸತ್ಯ. ಇದೊಂದು ಅಳುವ ಸ್ಥಳ. ನಾವು ಆ ದೇವದೂತನ ಮಾತನ್ನು ಕೇಳಿದ್ದರೆ ಸ್ವಾರ್ಥರಾಗಕೂಡದು.

ನಜರತ್ತಿನ ಪ್ರವಾದಿಯ ಮತ್ತೊಂದು ಕರೆಯನ್ನು ಕೇಳಿ. “ಅಣಿಯಾಗಿ. ಸ್ವರ್ಗವು ಸಮೀಪದಲ್ಲೇ ಇದೆ.” ಎನ್ನುವನು. ಕೃಷ್ಣನ ಸಂದೇಶವನ್ನು ಕುರಿತು ಚಿಂತನೆ ಮಾಡಿರು ವೆನು. ನಾನು ಆಸಕ್ತಿ ಇಲ್ಲದೇ ಕೆಲಸ ಮಾಡಲು ಯತ್ನಿಸುತ್ತಿರುವೆನು. ಆದರೆ ಕೆಲವು ವೇಳೆ ನಾನು ಅದನ್ನು ಮರೆಯುವೆನು. ಆಗ ತಕ್ಷಣ ಬುದ್ಧನ ಸಂದೇಶ ಬರುವುದು: “ಜೋಪಾನವಾಗಿರಿ, ಪ್ರಪಂಚದಲ್ಲಿ ಎಲ್ಲವೂ ಅನಿತ್ಯ. ಪ್ರಪಂಚವು ದುಃಖದಿಂದ ತುಂಬಿದೆ.” ನಾನು ಅದನ್ನೂ ಕೇಳುತ್ತೇನೆ. ಯಾವುದನ್ನು ಒಪ್ಪಿಕೊಳ್ಳಬೇಕೋ ನನಗೆಗೊತ್ತಿಲ್ಲ. ಅನಂತರ ಪುನಃ ವಜ್ರಾಘಾತದಂತೆ ಮತ್ತೊಂದು ಸಂದೇಶ ಬರುವುದು. “ಅಣಿಯಾಗಿ ಸ್ವರ್ಗವು ಸಮೀಪದಲ್ಲಿಯೇ ಇರುವುದು”, ಒಂದು ಕ್ಷಣವೂ ತಡ ಮಾಡಬೇಡಿ. ನಾಳೆಗೆ ಎಂದು ಏನನ್ನೂ ಬಿಡಬೇಡಿ. ಈ ಕ್ಷಣವೇ ನೀವು ಮೃತ್ಯುವಶ ರಾಗಬಹುದು. ಅದಕ್ಕೆ ಅಣಿಯಾಗಿ. ಈ ಸಂದೇಶಕ್ಕೂ ಅವಕಾಶವಿದೆ. ನಾವು ಅದನ್ನು ಒಪ್ಪಿಕೊಳ್ಳುತ್ತೇವೆ. ಆ ದೇವದೂತನನ್ನು ನಮಿಸುತ್ತೇವೆ, ಆ ಭಗವಂತನನ್ನು ನಮಿಸುತ್ತೇವೆ.

ಸಮತ್ವದ ಸಂದೇಶವನ್ನು ಸಾರಿದ ಮಹಮ್ಮದನಿಗೆ ನಾವು ಅನಂತರ ಬರುತ್ತೇವೆ. ಈ ಧರ್ಮದಿಂದ ಪ್ರಯೋಜನವೇನು ಎಂದು ನೀವು ಕೇಳಬಹುದು. ಅದರಲ್ಲಿ ಏನೂ ಪ್ರಯೋಜನವಿಲ್ಲದೆ ಇದ್ದರೆ ಅದು ಇಷ್ಟು ದಿನ ಹೇಗೆ ಇರುತಿತ್ತು? ಒಳ್ಳೆ ಯದು ಮಾತ್ರ ಇರುವುದು. ಅದೊಂದೆ ಕೊನೆಯವರೆಗೂ ಇರುವುದು. ಒಳ್ಳೆಯದು ಮಾತ್ರ ಬಲಾಢ್ಯವಾದುದು. ಆದಕಾರಣವೇ ಅದು ಕೊನೆಯವರೆಗೂ ಇರುವುದು. ಈ ಜೀವನದಲ್ಲಿ ದುರಾಚಾರಿ ತಾನೇ ಎಷ್ಟು ದಿನ ಬಾಳಬಲ್ಲ? ಸದಾಚಾರಶೀಲನ ಆಯಸ್ಸು ಅದಕ್ಕಿಂತ ಹೆಚ್ಚಲ್ಲವೇ? ನಿಜವಾಗಿಯೂ ಪಾವಿತ್ರ್ಯವೇ ಶಕ್ತಿ, ಚಾರಿತ್ರ್ಯ ಶುದ್ಧಿಯೇ ಶಕ್ತಿ. ಮಹಮ್ಮದನ ಬೋಧನೆಯಲ್ಲಿ ಒಳ್ಳೆಯದಿಲ್ಲದೇ ಇದ್ದರೆ ಅದು ಹೇಗೆ ಇದುವರೆಗೂ ಉಳಿಯುತ್ತಿತ್ತು? ಅದರಲ್ಲಿ ಎಷ್ಟೋ ಒಳ್ಳೆಯದಿದೆ. ಮಹಮ್ಮದನು ಸಮತ್ವದ ಪ್ರವಾದಿ, ಮಾನವನ ಸಹೋದರತ್ವದ, ಇಡೀ ಮಹಮ್ಮದೀಯರ ಸಹೋದರತ್ವದ ಪ್ರವಾದಿ.

ಪ್ರತಿಯೊಬ್ಬ ದೇವದೂತನೂ, ಪ್ರವಾದಿಯೂ, ಜಗತ್ತಿಗೆ ಒಂದು ನಿರ್ದಿಷ್ಟ ಸಂದೇಶವನ್ನು ತಂದನು ಎಂಬುದನ್ನು ನೋಡಿದೆವು. ಮೊದಲು ಅವನ ಸಂದೇಶವನ್ನು ಕೇಳಿ ಅನಂತರ ಅವನ ಜೀವನವನ್ನು ನೋಡಿ. ಆಗ ಅವನ ಇಡೀ ಜೀವನ ವಿವರಿಸ ಲ್ಪಟ್ಟಂತಾಗುವುದು. ಉಜ್ವಲವಾಗಿ ಬೆಳಗಲ್ಪವುಡುದು.

ಬುದ್ಧಿಯಿಲ್ಲದ ಮೂಢರು ಇಪ್ಪತ್ತು ಸಾವಿರ ಸಿದ್ಧಾಂತಗಳನ್ನು ಪ್ರಚಾರಕ್ಕೆ ತರುವರು. ತಮ್ಮ ದೃಷ್ಟಿಗೆ ತೋರಿದಂತೆ ಅದನ್ನು ವಿವರಿಸಿ ಅದನ್ನೆಲ್ಲ ತಮ್ಮ ದೇವ ದೂತರು ಹೇಳಿದರೆಂದು ಸಾರುವರು. ಅವರು ಪ್ರವಾದಿಗಳ ಸಂದೇಶವನ್ನು ತೆಗೆದು ಕೊಂಡು ತಮ್ಮ ತಪ್ಪು ವ್ಯಾಖ್ಯಾನಗಳನ್ನು ಕೊಡುವರು. ಪ್ರತಿಯೊಬ್ಬ ದೇವದೂತನಿಗೂ ಅವನ ಜೀವನವೇ ಅದಕ್ಕೆ ಬರೆದ ಭಾಷ್ಯ. ಅವರ ಜೀವನವನ್ನು ನೋಡಿ. ಅವರೇ ನೇನು ಮಾಡಿದರೂ ಅದು ಅವರ ಶಾಸ್ತ್ರದಲ್ಲಿದೆ. ಗೀತೆಯನ್ನು ಓದಿ. ಗೀತಾಚಾರ್ಯನ ಜೀವನದಲ್ಲಿ ಇದೆಲ್ಲ ಉದಾಹರಿತವಾಗಿರುವುದು ನಿಮಗೆ ತೋರುವುದು.

ಮಹಮ್ಮದನು ತನ್ನ ಜೀವನದಲ್ಲಿ ಮಹಮ್ಮದೀಯರಲ್ಲಿ ಸಂಪೂರ್ಣ ಸಮತ್ವ ವಿರಬೇಕು. ಸಹೋದರಭಾವ ಇರಬೇಕು ಎಂಬುದನ್ನು ತೋರಿಸಿದ. ಅಲ್ಲಿ ಜನಾಂಗ ಜಾತಿ ಬಣ್ಣ ಲಿಂಗ ಇವುಗಳ ವ್ಯತ್ಯಾಸವೇ ಇಲ್ಲ, ಟರ್ಕಿಯ ಸುಲ್ತಾನ ಆಫ್ರಿಕಾದ ಸಂತೆಯಲ್ಲಿ ಒಬ್ಬನನ್ನು ಗುಲಾಮನನ್ನಾಗಿ ಕೊಂಡು ಅವನಿಗೆ ಸರಪಳಿಯಲ್ಲಿ ಬಿಗಿದು ಟರ್ಕಿಗೆ ತರಬಹುದು. ಆದರೆ ಅವನು ಮಹಮ್ಮದೀಯನಾದರೆ ಅವನಲ್ಲಿ ಸಾಕಷ್ಟು ವಿದ್ಯಾಬುದ್ಧಿಗಳಿದ್ದರೆ ಸುಲ್ತಾನನ ಮಗಳನ್ನೇ ಬೇಕಾದರೆ ಮದುವೆಯಾಗಬಹುದು. ಇದನ್ನು ಈ ದೇಶದಲ್ಲಿ (ಅಮೆರಿಕಾ) ನೀಗ್ರೋಗಳು ಮತ್ತು ರೆಡ್​ ಇಂಡಿಯನ್ನರನ್ನು ಹೇಗೆ ನೋಡುತ್ತಿರುವರು ಎಂಬುವುದರೊಂದಿಗೆ ಹೋಲಿಸಿ ನೋಡಿ. ಹಿಂದೂಗಳು ಏನು ಮಾಡುತ್ತಾರೆ? ನಿಮ್ಮ ಪಾದ್ರಿಗಳಾರಾದರೂ ಆಚಾರಶೀಲನ ಆಹಾರವನ್ನು ಮುಟ್ಟಿದರೆ ಅದನ್ನು ಆಚೆಗೆ ಎಸೆಯುತ್ತಾರೆ. ನಮ್ಮಲ್ಲಿ ಎಷ್ಟೇ ದೊಡ್ಡ ತತ್ತ್ವವಿದ್ದರೂ ಅನುಷ್ಠಾನದಲ್ಲಿ ನಮ್ಮ ದುರ್ಬಲತೆಯನ್ನು ನೋಡಿ. ಆದರೆ ಮಹಮ್ಮದೀಯರಲ್ಲಿ ನೋಡಿ. ವರ್ಣ ಅಥವಾ ಜನಾಂಗಭೇದವಿಲ್ಲದೆ ಅವರು ಎಲ್ಲರನ್ನೂ ಸಮಾನವಾಗಿ ಕಾಣುತ್ತಾರೆ. ಈ ಸಮಾನತೆ ಬೇರೆ ಧರ್ಮದಲ್ಲಿ ಕಂಡುಬರುವುದಿಲ್ಲ.

ಇವರಿಗಿಂತ ಮೇಲಾದ ಇತರ ದೇವದೂತರು ಬರುವರೆ? ನಿಜವಾಗಿ ಅವರು ಈ ಪ್ರಪಂಚಕ್ಕೆ ಬರುವರು. ಆದರೆ ಇದನ್ನೆ ಎದುರು ನೋಡಬೇಡಿ. ನಿಮ್ಮಲ್ಲಿ ಪ್ರತಿಯೊಬ್ಬರು ಈ ನಿಜವಾದ, ಎಲ್ಲ ಹಳೆಯ ಒಡಂಬಡಿಕೆಗಳನ್ನೂ ಸೇರಿಸಿ ಮಾಡಿದ ಹೊಸ ಒಡಂಬಡಿಕೆಯ ಪ್ರವಾದಿಗಳಾಗಬೇಕು. ಹಳೆಯ ಸಂದೇಶಗಳನ್ನೆಲ್ಲಾ ತೆಗೆದುಕೊಳ್ಳಿ. ಅವಕ್ಕೆ ನಿಮ್ಮ ಅನುಭವವನ್ನೆಲ್ಲಾ ಸೇರಿಸಿ, ಇತರರಿಗೆ ನೀವೊಬ್ಬ ಪ್ರವಾದಿಯಾಗಿ. ಪ್ರತಿಯೊಬ್ಬ ಗುರುಗಳೂ ಮಹಾವ್ಯಕ್ತಿಗಳಾಗಿದ್ದರು, ಪ್ರತಿ ಯೊಬ್ಬರೂ ನಮಗೆ ಸ್ವಲ್ಪ ಬಿಟ್ಟು ಹೋಗಿರುವರು. ಅವರು ನಮ್ಮ ದೇವರಾಗಿರುವರು. ನಾವು ಅವರಿಗೆ ನಮಿಸುವೆವು. ನಾವು ಅವರ ಸೇವಕರು. ಆದರೆ ಜೊತೆಗೆ ನಮ್ಮನ್ನೂ ನಾವು ಗೌರವಿಸಿಕೊಳ್ಳುವೆವು. ಏಕೆಂದರೆ ಅವರು ದೇವದೂತರಾದರೆ, ದೇವರ ಮಕ್ಕಳಾದ ನಾವೂ ಕೂಡ ಹಾಗೆಯೇ ಇರುವೆವು. ಅವರು ಪೂರ್ಣತೆಯನ್ನು ಪಡೆದರು. ನಾವು ಕೂಡ ನಮ್ಮ ಪೂರ್ಣತೆಯನ್ನು ಈಗ ಮುಟ್ಟುವೆವು. “ಸ್ವರ್ಗ ಸಮೀಪದಲ್ಲೇ ಇದೆ” ಎಂಬ ಕ್ರಿಸ್ತನ ವಚನವನ್ನು ಗಮನಿಸಿ. ಈ ಕ್ಷಣ ನಮ್ಮಲ್ಲಿ ಪ್ರತಿಯೊಬ್ಬರೂ ಶಪಥ ಮಾಡೋಣ: “ನಾನೊಬ್ಬ ದೇವದೂತನಾಗುತ್ತೇನೆ, ಜ್ಯೋತಿಯ ಸಂದೇಶ ವಾಹಕನಾಗುತ್ತೇನೆ. ನಾನು ದೇವರ ಮಗುವಾಗುತ್ತೇನೆ. ಇಲ್ಲ, ನಾನೇ ದೇವ ನಾಗುತ್ತೇನೆ.”

