
\chapter[ಇಂಡಿಯಾ ]{ಇಂಡಿಯಾ \protect\footnote{\engfoot{C.W. Vol. VIII, p. 204}}}

\centerline{(ಅಮೆರಿಕದ ಪತ್ರಿಕೆಯಿಂದ ತೆಗೆದ ಒಂದು ಉಪನ್ಯಾಸದ ಸಾರಾಂಶ)}

ಡೆಟ್ರಾಯಿಟ್​ನ ಯೂನಿಟೇರಿಯನ್​ ಚರ್ಚಿನಲ್ಲಿ ೧೮೯೪ರ ಫೆಬ್ರವರಿ ೧೫ರಂದು ಸ್ವಾಮಿ ವಿವೇಕಾನಂದರು ಹಿಂದೂಗಳ ಆಚಾರವ್ಯವಹಾರಗಳನ್ನು ಕುರಿತು ಒಂದು ತುಂಬಿದ ಸಭೆಯಲ್ಲಿ ಭಾಷಣ ಮಾಡಿದರು. ಸ್ವಾಮೀಜಿ ಅವರ ಗಾಂಭೀರ್ಯ, ಅವರ ವಾಗ್ಮಿತೆಗಳು ಸಭಿಕರನ್ನು ಸಂತೋಷಗೊಳಿಸಿದುವು. ಸಭಿಕರು ಸ್ವಾಮೀಜಿಯವರ ಭಾಷಣವನ್ನು ಮೊದಲಿನಿಂದ ಕೊನೆಯವರೆಗೆ ತುಂಬ ಗಮನವಿಟ್ಟು ಕೇಳಿದರು. ಮಧ್ಯೆ ಮಧ್ಯೆ ತಮ್ಮ ಮೆಚ್ಚಿಗೆಯನ್ನು ಸೂಚಿಸಲು ಕರತಾಡನಗಳನ್ನೂ ಮಾಡುತ್ತಿದ್ದರು. ಅವರು ನಿನ್ನೆ ಮಾಡಿದ ಉಪನ್ಯಾಸವು ವಿಶ್ವಧರ್ಮ ಸಮ್ಮೇಳನದಲ್ಲಿ ಮಾಡಿದ ಪ್ರಸಿದ್ಧ ಉಪನ್ಯಾಸಕ್ಕಿಂತ ಹೆಚ್ಚು ಜನಪ್ರಿಯವಾಗಿತ್ತು. ಅದರಲ್ಲಿಯೂ ಉಪನ್ಯಾಸಕರು ಉಪದೇಶಾತ್ಮಕ ವಿಷಯಗಳನ್ನು ಬಿಟ್ಟು ಕೆಲವು ವೇಳೆ ತಮ್ಮ ಜನರ ಆಧ್ಯಾತ್ಮಿಕ ವಿಷಯಗಳನ್ನು ಹೇಳುವಾಗಲಂತೂ ಕೇಳುವುದಕ್ಕೆ ಹರ್ಷದಾಯಕವಾಗಿತ್ತು. ಪೂರ್ವ ದೇಶದ ಸೋದರರು ಮತ್ತು ಧಾರ್ಮಿಕ ವಿಷಯಗಳನ್ನು ಕುರಿತು ಮಾತನಾಡುವಾಗಲಂತೂ ತಮ್ಮ ಅದ್ಭುತ ತಾತ್ತ್ವಿಕ ವಾಗ್ಮಿತೆಯನ್ನು ತೋರುವರು. ಪ್ರಕೃತಿಯ ಮಹಾನ್​ ನೈತಿಕ ನಿಯಮವನ್ನು ಆತ್ಮಸಾಕ್ಷಿಯಾಗಿ ಅನುಸರಿಸುವ ಸಂದರ್ಭದಲ್ಲಿ ಸಹಜ ವಾಗಿಯೇ ಪಾಲಿಸಬೇಕಾದ ಕರ್ತವ್ಯಗಳನ್ನು ಅವರು ವಿವರಿಸುವಾಗ ತಮ್ಮ ದೇಶದ ಜನರಿಗೇ ವಿಶಿಷ್ಟವಾದ ಮೃದು ಮಂಜುಳ ಧ್ವನಿಯಲ್ಲಿ, ರೋಮಾಂಚಕರ ವಾಗಿ ಹೇಳುವಾಗಲಂತೂ ಒಬ್ಬ ಪ್ರವಾದಿ ಮಾತನಾಡುತ್ತಿದ್ದಾನೋ ಎಂಬಂತೆ ಇತ್ತು. ಮಾತನಾಡುವಾಗ ಗಂಭೀರವಾಗಿ ವಿಷಯಗಳನ್ನು ಪ್ರತಿಪಾದಿಸುವರು. ಆದರೆ ಕೆಲವು ವೇಳೆ ಕೆಲವು ನೈತಿಕ ಸತ್ಯವನ್ನು ಹೇಳುವಾಗಲಂತೂ ಅವರ ವಾಗ್ಮಿತೆ ಅತ್ಯುತ್ತಮ ಮಟ್ಟಕ್ಕೆ ಏರುತ್ತಿತ್ತು.

ಈ ಪ್ರಾಚ್ಯ ಸಂನ್ಯಾಸಿಯು ಇಂಡಿಯಾದೇಶದಲ್ಲಿ ಕ್ರೈಸ್ತಪಾದ್ರಿಗಳ ಚಟುವಟಿಕೆಗಳ ವಿಷಯದಲ್ಲಿ ತಮ್ಮ ಅಸಮಾಧಾನವನ್ನು ಸ್ಪಷ್ಟವಾಗಿ ವ್ಯಕ್ತಪಡಿಸುತ್ತಾರೆ. ಭಾರತದ ನೈತಿಕತೆಯು ಜಗತ್ತಿನಲ್ಲಿಯೇ ಅತ್ಯಂತ ಉನ್ನತ ಮಟ್ಟದ್ದು ಎಂಬುದು ಅವರ ಅಭಿಪ್ರಾಯ. ಇಂತಹವರನ್ನು ಸದ್ಯದಲ್ಲಿಯೇ ಚೀನಾಕ್ಕೆ ಮಿಷನರಿ ಕೆಲಸಕ್ಕಾಗಿ ಹೋಗುತ್ತಿರುವ ಬಿಷಪ್​ ನೀಂಡೆ ಎನ್ನುವವರು ಪರಿಚಯ ಮಾಡಿಕೊಟ್ಟುದು ಒಂದು ವಿಶೇಷ. ನೀಂಡೆಯವರು ಡಿಸೆಂಬರ್​ವರೆಗೆ ಚೀಣಾದಲ್ಲಿರುವರು. ಇನ್ನೂ ಹೆಚ್ಚುಕಾಲ ಅಲ್ಲಿದ್ದರೆ ಭಾರತಕ್ಕೆ ಹೋಗುವರು. ಬಿಷಪ್ಪರು ಇಂಡಿಯಾ ದೇಶದಲ್ಲಿರುವ ಅಪೂರ್ವ ವಿಷಯಗಳನ್ನು ಹೇಳಿ, ಅಲ್ಲಿಯ ವಿದ್ಯಾವಂತರು ಎಂತಹ ಬುದ್ಧಿಶಾಲಿಗಳು ಎಂದು ಹೊಗಳಿ, ಸ್ವಾಮಿ ವಿವೇಕಾ ನಂದರ ಪರಿಚಯವನ್ನು ಸಂತೋಷಕರವಾದ ರೀತಿಯಲ್ಲಿ ಮಾಡಿಕೊಟ್ಟರು. ಆ ಕಂದು ಬಣ್ಣದ ಸಂನ್ಯಾಸಿ ತಲೆಗೆ ರುಮಾಲನ್ನು ಕಟ್ಟಿಕೊಂಡು, ಒಂದು ಹೊಳೆಯುವ ನಿಲುವಂಗಿಯನ್ನು ತೊಟ್ಟಿದ್ದನು. ಸುಂದರವಾದ ಮುಖವನ್ನು ಮತ್ತು ಬುದ್ಧಿಶಕ್ತಿಯನ್ನು ಸೂಚಿಸುವ ಆ ಹೊಳೆಯುವ ಕಣ್ಣುಗಳನ್ನು ನೋಡಿದಾಗ ಎಲ್ಲರೂ ಆಕರ್ಷಿತರಾದರು. ಬಿಷಪ್ಪರಿಗೆ ಧನ್ಯವಾದಗಳನ್ನು ಅರ್ಪಿಸಿ ತಮ್ಮ ದೇಶದಲ್ಲಿರುವ ವರ್ಣಗಳು, ಅವರ ಆಚಾರ ವ್ಯವಹಾರಗಳು, ಅಲ್ಲಿರುವ ಭಿನ್ನಭಿನ್ನ ಭಾಷೆಗಳು ಇವನ್ನು ಕುರಿತು ಹೇಳತೊಡಗಿದರು. ಕೆಳಗಿನದು ಅವರ ಭಾಷಣದ ಸಾರಾಂಶ.

ಮುಖ್ಯವಾಗಿ ಉತ್ತರದಲ್ಲಿ ನಾಲ್ಕು ಭಾಷೆಗಳು, ದಕ್ಷಿಣದಲ್ಲಿ ನಾಲ್ಕು ಭಾಷೆ ಗಳು ಇವೆ. ಆದರೆ ಎಲ್ಲರಿಗೂ ಒಂದೇ ಸಾಮಾನ್ಯ ಧರ್ಮವಿದೆ. ಮೂವತ್ತುಕೋಟಿ ಜನರಲ್ಲಿ ಐದರಲ್ಲಿ ನಾಲ್ಕು ಜನರು ಹಿಂದೂಗಳು. ಹಿಂದೂಗಳು ಒಂದು ವಿಚಿತ್ರ ಜನಾಂಗ. ಹಿಂದೂ ಪ್ರತಿಯೊಂದನ್ನೂ ಒಂದು ಧಾರ್ಮಿಕ ದೃಷ್ಟಿಯಿಂದ ಮಾಡುವನು. ಅವನು ಧಾರ್ಮಿಕವಾಗಿ ಊಟಮಾಡುವನು, ಧಾರ್ಮಿಕವಾಗಿ ಮಲಗುವನು, ಧಾರ್ಮಿಕವಾಗಿ ಏಳುವನು. ಅವನು ಪುಣ್ಯಕೆಲಸಗಳನ್ನು ಧಾರ್ಮಿಕವಾಗಿ ಮಾಡುವನು, ಹಾಗೆಯೇ ಪಾಪಕೆಲಸಗಳನ್ನೂ ಧಾರ್ಮಿಕವಾಗಿ ಮಾಡುವನು. ಈ ಸಂದರ್ಭದಲ್ಲಿ ಉಪನ್ಯಾಸಕರು ಹಿಂದೂಧರ್ಮದ ಒಂದು ದೊಡ್ಡ ನೀತಿಯನ್ನು ವಿವರಿಸಿದರು. ಎಲ್ಲ ಬಗೆಯ ಸ್ವಾರ್ಥವೇ ಪಾಪ, ನಿಃಸ್ವಾರ್ಥವೆ ಪುಣ್ಯ ಎಂಬುದೇ ಅದು. ಉಪನ್ಯಾಸದಲ್ಲೆಲ್ಲಾ ಅವರು ಇದನ್ನು ಮತ್ತೆ ಮತ್ತೆ ಉದಾಹರಿಸುತ್ತಿದ್ದರು. ಇದೊಂದೇ ಅವರ ಉಪನ್ಯಾಸದ ಮುಖ್ಯ ಪಲ್ಲವಿ ಎಂಬಂತೆ ಇತ್ತು. ಕೇವಲ ವಾಸಕ್ಕಾಗಿ ಒಂದು ಮನೆಯನ್ನು ಕಟ್ಟುವುದು ಸ್ವಾರ್ಥ ಎನ್ನುವನು ಹಿಂದೂ. ಆದ ಕಾರಣವೇ ಅವನು ದೇವರ ಪೂಜೆಗೆ ಮತ್ತು ಅತಿಥಿ ಸೇವೆಗೆ ಎಂದು ಒಂದು ಮನೆಯನ್ನು ಕಟ್ಟುವನು. ತನ್ನ ಹೊಟ್ಟೆಯ ಹಸಿವನ್ನು ತಣಿಸುವುದಕ್ಕಾಗಿ ಮಾತ್ರ ಅಡಿಗೆ ಮಾಡಿಕೊಳ್ಳುವುದು ಸ್ವಾರ್ಥ. ಆದಕಾರಣವೇ ಅವನು ಬಡವರಿಗೆ ಊಟಹಾಕಿ ಕೊನೆಗೆ ಮಿಕ್ಕಿರುವುದನ್ನು ತಾನು ತಿನ್ನುವನು. ಈ ಭಾವನೆ ಇಂಡಿಯಾ ದೇಶದಲ್ಲೆಲ್ಲಾ ಹಬ್ಬಿರುವುದು. ಯಾರಾದರೂ ತಂಗುವುದಕ್ಕೆ ಮತ್ತು ಭೋಜನಕ್ಕೆ ಯಾರ ಮನೆಯಲ್ಲಾದರೂ ಕೇಳಬಹುದು. ಪ್ರತಿಯೊಂದು ಮನೆಯೂ ಅವನಿಗೆ ತೆರೆದಿದೆ.

ಜಾತಿ ಪದ್ಧತಿಗೂ ಧರ್ಮಕ್ಕೂ ಏನೂ ಸಂಬಂಧವಿಲ್ಲ. ಒಬ್ಬನ ವೃತ್ತಿ ವಂಶಾನುಗತವಾಗಿ ಬಂದುದು. ಬಡಗಿ ತನ್ನ ಹುಟ್ಟಿನಿಂದ ಬಡಗಿ. ಅಕ್ಕಸಾಲಿಗ ತನ್ನ ಹುಟ್ಟಿನಿಂದ ಅಕ್ಕಸಾಲಿಗ, ಕೂಲಿಯೂ ಅಷ್ಟೆ; ಪುರೋಹಿತರೂ ಹಾಗೆಯೇ. ಆದರೆ ಅದು ಇತ್ತೀಚೆಗೆ ಬಂದ ಒಂದು ದೋಷ. ಒಂದು ಸಾವಿರ ವರುಷಗಳಿಂದ ಇದು ಇದೆ. ಒಂದು ಜನಾಂಗದ ಇತಿಹಾಸದಲ್ಲಿ, ಈ ದೇಶ ಮತ್ತು ಇತರ ದೇಶಗಳಿಗೆ ಹೋಲಿಸಿದರೆ ಇಂಡಿಯಾದಲ್ಲಿ, ಇದೇನೂ ಅಷ್ಟೊಂದು ದೀರ್ಘ ಕಾಲವಾಗಿ ಕಾಣಿಸುವುದಿಲ್ಲ. ಎರಡು ದಾನಗಳನ್ನು ಜನರು ಕೊಂಡಾಡುವರು. ಒಂದು ಜ್ಞಾನದಾನ ಮತ್ತೊಂದು ಪ್ರಾಣದಾನ. ಆದರೆ ಜ್ಞಾನದಾನವು ಪ್ರಾಣದಾನ ಕ್ಕಿಂತ ಹೆಚ್ಚು. ಒಬ್ಬ ಮತ್ತೊಬ್ಬನ ಪ್ರಾಣವನ್ನು ಉಳಿಸಿಬಹುದು. ಅದು ಒಳ್ಳೆಯದು. ಒಬ್ಬ ಮತ್ತೊಬ್ಬನಿಗೆ ಜ್ಞಾನವನ್ನು ಕೊಡಬಹುದು, ಇದು ಉತ್ತಮ. ಹಣಕ್ಕಾಗಿ ವಿದ್ಯೆಯನ್ನು ಕಲಿಸುವುದು ಪಾಪ. ವಿದ್ಯೆ ಎಂಬುದು ಹಣಕ್ಕೆ ಮಾರು ವಂತಹ ವಸ್ತುವಲ್ಲ. ಹಾಗೆ ಮಾಡಿದರೆ ಅದೊಂದು ಅಧರ್ಮ. ಸರ್ಕಾರವು ಆಗಾಗ ಗುರುಗಳಿಗೆ ದಾನಗಳನ್ನು ನೀಡುತ್ತದೆ. ಈ ರೀತಿಯ ವಿದ್ಯಾದಾನ ಪದ್ಧತಿಯ ನೈತಿಕ ಪರಾಣಾಮವು ನಾಗರಿಕ ದೇಶಗಳೆನಿಸಿಕೊಳ್ಳುವ ಕಡೆಗಳಲ್ಲಿ ಇರುವ ವಿದ್ಯಾಭ್ಯಾಸ ಪದ್ಧತಿಗಿಂತ ಉತ್ತಮ.

ನಾಗರಿಕತೆ ಎಂದರೆ ಏನು ಎಂದು ಈ ದೇಶದ ಉದ್ದಗಲಗಳಲ್ಲೆಲ್ಲಾ ಈ ಉಪನ್ಯಾಸಕರು ಪ್ರಶ್ನೆ ಕೇಳಿರುವರು. ಇದೇ ಪ್ರಶ್ನೆಯನ್ನು ಹಲವು ದೇಶಗಳಲ್ಲಿ ಕೇಳಿರುವರು. ನಾವೇನಾಗಿರುವೆವೋ ಅದು ನಾಗರಿಕತೆ ಎಂದು ಅನೇಕ ಸಲ ಅವರಿಗೆ ಉತ್ತರಗಳು ದೊರಕಿವೆ. ಆದರೆ ತಾವು ಅದಕ್ಕೆ ಕೊಡುವ ಅರ್ಥವೇ ಬೇರೆ ಎಂದರು ಉಪನ್ಯಾಸಕರು. ಒಂದು ದೇಶದವರು ಪ್ರಕೃತಿಯ ಮೇಲೆ ಪ್ರಭುತ್ವವನ್ನು ಸಾಧಿಸಿ ಅದರ ಮೂಲಕ ಬೇಕಾದಷ್ಟು ಸುಖ ಸಂಪತ್ತುಗಳನ್ನು ಹೆಚ್ಚಿಸಿಕೊಳ್ಳಬಹುದು. ಆದರೂ ಯಾರು ತನ್ನನ್ನು ಗೆದ್ದಿರುವನೊ ಆ ವ್ಯಕ್ತಿಯಲ್ಲಿ ಅತ್ಯುನ್ನತ ನಾಗರಿಕತೆಯನ್ನು ಕಾಣುತ್ತೇವೆ ಎಂಬುದನ್ನು ಅರಿಯದೇ ಇದ್ದಿರಬಹುದು. ಭರತಖಂಡದಲ್ಲಿ ಈ ಸ್ಥಿತಿ ಜಗತ್ತಿನ ಇತರ ದೇಶಗಳಿಗಿಂತ ಹೆಚ್ಚಾಗಿ ವ್ಯಕ್ತವಾಗಿದೆ. ಏಕೆಂದರೆ ಅಲ್ಲಿ ಪ್ರಾಪಂಚಿಕ ವಿಷಯಗಳೆಲ್ಲ ಆಧ್ಯಾತ್ಮಿಕಕ್ಕೆ ಅಧೀನ. ಅಲ್ಲಿ ಅವರು ಪ್ರಕೃತಿಯನ್ನು ನೋಡುವುದು ಎಷ್ಟರ ಮಟ್ಟಿಗೆ ಅದು ತಮ್ಮಲ್ಲಿ ಸುಪ್ತವಾಗಿರುವ ಆತ್ಮವನ್ನು ವ್ಯಕ್ತಗೊಳಿಸುತ್ತಿದೆ ಎಂಬ ದೃಷ್ಟಿಯಿಂದ. ಆದಕಾರಣವೇ ಅವರು ಪ್ರಪಂಚದ ಕಷ್ಟನಷ್ಟಗಳನ್ನೆಲ್ಲಾ ಅಷ್ಟು ಸಮಾಧಾನದಿಂದ ಸಹಿಸಬಲ್ಲರು. ಇತರರೆಲ್ಲರಿಗಿಂತ ಹೆಚ್ಚಾಗಿ ಆಧ್ಯಾತ್ಮಿಕ ಶಕ್ತಿ ತಮ್ಮಲ್ಲಿದೆ ಎಂಬುದು ಅವರಿಗೆ ಸಂಪೂರ್ಣವಾಗಿ ಗೊತ್ತು, ಆದಕಾರಣ ಇಂತಹ ದೇಶದಿಂದ ಇಂತಹ ಜನಾಂಗದಿಂದ ಎಂದಿಗೂ ಬತ್ತದ ಆಧ್ಯಾತ್ಮಿಕ ಪ್ರವಾಹವು ಹಿಂದಿನಿಂದಲೂ ಹರಿಯುತ್ತಿದೆ. ಈ ಪ್ರವಾಹವು ಹಲವು ಪ್ರಾಚ್ಯ ಮತ್ತು ಪಾಶ್ಚಾತ್ಯ ಮೇಧಾವಿಗಳನ್ನು ಆಕರ್ಷಿಸಿದೆ; ಪ್ರಾಪಂಚಿಕ ಹೊರೆ ಗಳನ್ನು ಅವರ ಹೆಗಲಿನಿಂದ ಇಳಿಸಿದೆ.

ಕ್ರಿಸ್ತಪೂರ್ವ ೨೬೦ರಲ್ಲಿ ಆಗಿನ ರಾಜನು ದೇಶದಲ್ಲಿ ಇನ್ನು ಮೇಲೆ ಯುದ್ಧ ವಾಗಲೀ ರಕ್ತಪಾತವಾಗಲೀ ಇರಕೂಡದು ಎಂದು, ಯೋಧರ ಬದಲು ಧಾರ್ಮಿಕ ಬೋಧಕರ ಪಡೆಯನ್ನು ಹೊರದೇಶಗಳಿಗೆ ಕಳುಹಿಸಿದನು. ಇದರಿಂದ ದೇಶ ಆರ್ಥಿಕ ದುಃಸ್ಥಿತಿಗೆ ಬಂದರೂ ಒಳ್ಳೆಯ ಕೆಲಸವನ್ನೇ ಮಾಡಿದನು. ಭರತಖಂಡವು ಶಸ್ತ್ರಸನ್ನದ್ಧರಾದ ಪರಕೀಯರಿಗೆ ಆಳಾದರೂ ಭಾರತೀಯ ಆಧ್ಯಾತ್ಮಿಕತೆ ಎಂದೆಂದಿಗೂ ಉಳಿದಿದೆ. ಜಗತ್ತಿನಲ್ಲಿ ಯಾವುದೂ ಅದನ್ನು ಕಸಿದುಕೊಳ್ಳುವಂತೆ ಕಾಣಿಸುವುದಿಲ್ಲ. ಜೀವ ದೇವರ ಕಡೆಗೆ ಹೋಗುವಾಗ ವಿಧಿಯು ಇವರ ಮೇಲೆ ಎಸೆಯುತ್ತಿರುವ ತೀಕ್ಷ್ಣವಾದ ಬಾಣಗಳನ್ನು ಸಹಿಸುವುದರಲ್ಲಿ ಕ್ರಿಸ್ತನಲ್ಲಿದ್ದಂತಹ ಒಂದು ದೈನ್ಯ ಇವರಲ್ಲಿ ಇದೆ. ಇಂತಹ ದೇಶಕ್ಕೆ ಧರ್ಮವನ್ನು ಬೋಧಿಸುವ ಯಾವ ಪಾದ್ರಿಗಳೂ ಬೇಕಾಗಿಲ್ಲ. ಏಕೆಂದರೆ ಅವರ ಧರ್ಮವು ಎಲ್ಲರೂ ಮೃದುವಾಗಿರು ವಂತೆ, ಎಲ್ಲರೊಡನೆಯೂ ಮಧುರವಾಗಿ ವರ್ತಿಸುವಂತೆ, ಎಲ್ಲರಿಗೂ – ಮಾನವ ರಾಗಲಿ, ಪ್ರಾಣಿಗಳಾಗಲಿ – ದಯೆ ತೋರುವಂತೆ, ವಿಶ್ವಾಸವನ್ನು ತೋರುವಂತೆ ಮಾಡುತ್ತದೆ. ನೈತಿಕವಾಗಿ ಭರತಖಂಡ ಅಮೆರಿಕಕ್ಕಿಂತಲೂ ಮತ್ತು ಇತರ ಎಲ್ಲಾ ರಾಷ್ಟ್ರಗಳಿಗಿಂತಲೂ ಎಷ್ಟೋ ಎತ್ತರದಲ್ಲಿದೆ ಎಂದರು ಉಪನ್ಯಾಸಕರು. ಪಾದ್ರಿಗಳು ಅಲ್ಲಿಗೆ ಬಂದು ಅಲ್ಲಿಯ ಪವಿತ್ರ ವಾತಾವರಣದಲ್ಲಿದ್ದು, ಅಲ್ಲಿಯ ಅಸಂಖ್ಯಾತ ಮಹಾತ್ಮರು ಎಂತಹ ಪ್ರಭಾವವನ್ನು ಜನರ ಮೇಲೆ ಬೀರಿರುವರು ಎಂಬುದನ್ನು ನೋಡುವುದು ಮೇಲು.

ಅನಂತರ ಅಲ್ಲಿಯ ವಿವಾಹದ ಪ್ರಶ್ನೆಯನ್ನು ತೆಗೆದುಕೊಂಡರು. ಹಿಂದಿನ ಕಾಲದಲ್ಲಿ ಸ್ತ್ರೀಪುರುಷರು ಒಟ್ಟಿಗೆ ಅಧ್ಯಯನ ಮಾಡುತ್ತಿದ್ದಾಗ ಸ್ತ್ರೀಯರಿಗೆ ಇದ್ದ ಹಕ್ಕುಗಳನ್ನು ಉಪನ್ಯಾಸಕರು ವಿವರಿಸಿದರು. ಅಲ್ಲಿನ ಸಂತರ ಪಂಕ್ತಿಯಲ್ಲಿ ಎಷ್ಟೋಮಂದಿ ಮಹಿಳೆಯರೂ ಇದ್ದರು. ಕ್ರೈಸ್ತರಲ್ಲಿ ಇರುವವರೆಲ್ಲ ಪ್ರವಾದಿಗಳು ಆದರೆ ಹಿಂದೂಗಳ ಪವಿತ್ರ ಗ್ರಂಥಗಳಲ್ಲಿ ಮಹಿಳೆಯರಿಗೆ ವಿಶಿಷ್ಟ ಸ್ಥಾನವಿದೆ. ಗೃಹಸ್ಥರು ಪಂಚಯಜ್ಞಗಳನ್ನು ಮಾಡಬೇಕಾಗಿದೆ. ಅದರಲ್ಲಿ ಒಂದು ಅಧ್ಯಯನ ಮತ್ತು ಪ್ರವಚನ. ಮತ್ತೊಂದು ಭೂತಯಜ್ಞ, ಪ್ರಾಣಿಗಳನ್ನು ಪೂಜಿಸುವುದು. ಅಮೆರಿಕಾ ದೇಶದವರೆಗೆ ಇದನ್ನು ಅರ್ಥಮಾಡಿಕೊಳ್ಳುವುದು ಕಷ್ಟವಾಗಬಹುದು. ಐರೋಪ್ಯರಿಗೂ ಇದು ಕಷ್ಟವೇ. ಇತರ ದೇಶಗಳು ಪ್ರಾಣಿಗಳ ಮಂದೆಯನ್ನೇ ಕೊಲ್ಲುವುವು. ಅಲ್ಲಿನ ಜನರು ಪರಸ್ಪರರನ್ನೂ ಕೊಲ್ಲುವರು. ಒಂದು ರಕ್ತದ ಸಾಗರದಲ್ಲಿ ಅವರು ವಾಸಿಸುವರು. ಇಂಡಿಯಾ ದೇಶದಲ್ಲಿ ಏತಕ್ಕೆ ಪ್ರಾಣಿಗಳನ್ನು ಕೊಲ್ಲುವುದಿಲ್ಲ ಎಂಬುದಕ್ಕೆ ಒಬ್ಬ ಯೂರೋಪಿಯನ್ನನು ಕೊಟ್ಟ ಕಾರಣ ಏನೆಂದರೆ, ಅಲ್ಲಿಯ ಜನರು ಪ್ರಾಣಿಗಳಲ್ಲಿ ತಮ್ಮ ಪಿತೃಗಳ ಜೀವವಿದೆ ಎಂದು ಭಾವಿಸುವರು ಎಂಬುದು. ಈ ಕಾರಣವು ಮೃಗದೋಪಾದಿಯಲ್ಲಿರುವ ಒಂದು ಕಾಡು ಜನಾಂಗಕ್ಕೆ ಅನ್ವಯಿಸುವುದೇ ಹೊರತು ಭರತಖಂಡಕ್ಕೆ ಅನ್ವಯಿಸುವುದಿಲ್ಲ. ಇದನ್ನು ಹೇಳಿದವರು ಅಹಿಂಸೆಯನ್ನು ಮತ್ತು ಪುನರ್​ಜನ್ಮವನ್ನು ಬೋಧಿಸಿದ ವೇದವನ್ನು ಒಪ್ಪದ ಕೆಲವು ನಾಸ್ತಿಕರು. ಇದೆಂದಿಗೂ ಧರ್ಮವಾಗಿರಲಿಲ್ಲ. ಇದು ಚಾರ್ವಾಕರ ಮತ. ಮೂಕ ಪ್ರಾಣಿಗಳ ಪೂಜೆಯನ್ನು ಅತಿ ವಿಸ್ತಾರವಾಗಿ ವಿವರಿಸಿದರು. ಆತಿಥ್ಯವೇ ಭಾರತೀಯರ ಒಂದು ಮಹಾವ್ರತ. ಇದನ್ನು ಒಂದು ಕಥೆಯ ಮೂಲಕ ಉಪನ್ಯಾಸಕರು ವಿವರಿಸಿದರು. ಬರಗಾಲದ ನಿಮಿತ್ತ ಒಬ್ಬ ಬ್ರಾಹ್ಮಣ ಅವನ ಹೆಂಡತಿ ಅವನ ಮಗ ಮತ್ತು ಸೊಸೆ ಹಲವು ದಿನಗಳಿಂದ ಊಟ ಮಾಡಿರಲಿಲ್ಲ. ಮನೆಯ ಯಜಮಾನ ಹೊರಗೆಹೋಗಿ ಅಲ್ಲಿ ಇಲ್ಲಿ ಹುಡುಕಾಡಿದಾಗ ಸ್ವಲ್ಪ ಅಕ್ಕಿ ಸಿಕ್ಕಿತು. ಇದನ್ನು ತಂದು ಬೇಯಿಸಿ ನಾಲ್ಕು ಭಾಗ ಮಾಡಿ ಇನ್ನೇನು ಊಟಮಾಡುವುದರಲ್ಲಿದ್ದರು. ಆ ಸಮಯದಲ್ಲಿ ಯಾರೋ ಬಾಗಿಲನ್ನು ತಟ್ಟಿದರು ನೋಡಿದಾಗ ಅಲ್ಲಿ ಒಬ್ಬ ಅತಿಥಿ ಇದ್ದ. ನಾಲ್ಕು ಜನರೂ ತಮ್ಮ ಅನ್ನವನ್ನು ಅವನ ಮುಂದೆ ಇಟ್ಟರು. ಅವನು ಉಂಡು ಇವರನ್ನು ಹರಸಿ ಹೊರಟು ಹೋದನು. ಇತ್ತ ನಾಲ್ವರೂ ಸತ್ತುಹೋದರು. ಅತಿಥಿಗಳಿಗಾಗಿ ನಾವು ಏನು ಮಾಡಬೇಕೆಂಬುದನ್ನು ವಿವರಿಸುವುದಕ್ಕೆ ಭರತಖಂಡದಲ್ಲಿ ಈ ಕಥೆ ಹೇಳುತ್ತಾರೆ.

ಉಪನ್ಯಾಸಕರು ಅತ್ಯಂತ ಉಜ್ವಲವಾಗಿ ತಮ್ಮ ಭಾಷಣವನ್ನು ಪೂರೈಸಿದರು. ಮೊದಲಿನಿಂದ ಕೊನೆಯವರೆಗೂ ಉಪನ್ಯಾಸ ಸರಳವಾಗಿತ್ತು. ಆದರೆ ಕೆಲವು ವೇಳೆ ರೂಪಕಗಳನ್ನು ವಿವರಿಸುವಾಗಲಂತೂ ಭಾಷೆ ಕಾವ್ಯಮಯವಾಗುತ್ತಿತ್ತು. ಪ್ರಕೃತಿಯ ಸೌಂದರ್ಯವನ್ನು ಅವರು ಎಷ್ಟು ಚೆನ್ನಾಗಿ ಆಸ್ವಾದಿಸುತ್ತಿದ್ದರು ಎಂಬುದನ್ನು ಇದು ತೋರುವುದು. ಅವರ ಅತ್ಯುನ್ನತ ಆಧ್ಯಾತ್ಮಿಕತೆಯು ಸಭಿಕರೆಲ್ಲರ ಮೇಲೂ ತನ್ನ ಪ್ರಭಾವವನ್ನು ಬೀರುತ್ತಿತ್ತು. ಏಕೆಂದರೆ ಅದು ಚರಾಚರವಸ್ತುಗಳಿಗೆಲ್ಲ ಪ್ರೇಮವನ್ನು ಸೂಸುತ್ತಿತ್ತು. ಪ್ರಪಂಚದಲ್ಲೆಲ್ಲಾ ಹೇಗೆ ಒಂದು ಮಹಾಶಕ್ತಿ ಪರಸ್ಪರ ಸೌಹಾರ್ದಕ್ಕೆ, ವಿಶ್ವಾಸಕ್ಕೆ ಕೆಲಸಮಾಡುತ್ತಿದೆ ಎಂಬುದನ್ನು ಅವರ ವಾಣಿಯಲ್ಲೆಲ್ಲಾ ಗಮನಿಸಬಹುದಾಗಿತ್ತು.

