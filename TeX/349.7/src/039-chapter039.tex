
\chapter[ನಾರದ ಮತ್ತು ಸೂತ್ರಗಳು ]{ನಾರದ ಮತ್ತು ಸೂತ್ರಗಳು \protect\footnote{\engfoot{C.W. Vol. VI, P 150}}}

\centerline{ಸ್ವಾಮೀಜಿ ಅವರು ಅಮೆರಿಕಾ ದೇಶದಲ್ಲಿ ಮಾಡಿದ ಭಾಷಾಂತರ}

\begin{center}
೧
\end{center}

\begin{enumerate}
\item ಭಕ್ತಿಯು ಭಗವಂತನ ಮೇಲೆ ಇಡುವ ತೀವ್ರವಾದ ಪ್ರೇಮ.

 \item ಭಕ್ತಿಯು ಅಮೃತಸ್ವರೂಪವಾದುದು.

 \item ಅದನ್ನು ಪಡೆದ ಮೇಲೆ ಮನುಷ್ಯನು ಸಿದ್ಧನಾಗುವನು, ಅಮೃತನಾಗುವನು, ಎಂದೆಂದಿಗೂ ತೃಪ್ತನಾಗುವನು.

 \item ಅದನ್ನು ಪಡೆದ ಮೇಲೆ ಮತ್ತಾವುದನ್ನೂ ಅವನು ಆಶಿಸುವುದಿಲ್ಲ. ಮತ್ತಾವುದನ್ನೂ ದ್ವೇಷಿಸುವುದಿಲ್ಲ, ಮತ್ತಾವುದರಲ್ಲಿಯೂ ರಮಿಸುವುದಿಲಿಲ್ಲ.

 \item ಅದನ್ನು ಅರಿತ ಮೇಲೆ, ಭಗವದ್​ ಭಕ್ತಿಯಿಂದ ಉನ್ಮತ್ತನಾಗುವನು, ಸ್ತಬ್ಧನಾಗುವನು, ಆತ್ಮಾರಾಮನಾಗುವನು.

 \item ಅದು ಕಾಮನೆಯ ರೂಪವಾದುದಲ್ಲ. ಏಕೆಂದರೆ ಅದು ನಿರೋಧ ರೂಪವಾದದ್ದು.

 \item ಸಂನ್ಯಾಸ ಎಂದರೆ ಲೌಕಿಕ ಮತ್ತು ವೈದಿಕ ವ್ಯವಹಾರಗಳನ್ನು ತ್ಯಜಿಸುವುದು.

 \item ಆ ಭಕ್ತ ಸಂನ್ಯಾಸಿಯು ಅನನ್ಯವಾಗಿ ಭಗವಂತನನ್ನು ಚಿಂತಿಸುವನು, ಅದಕ್ಕೆ ವಿರೋಧವಾಗಿರುವುದನ್ನೆಲ್ಲ ತ್ಯಜಿಸುವನು.

 \item ಅವನು ಅನ್ಯ ಆಶ್ರಯಗಳನ್ನು ತ್ಯಜಿಸಿ ಭಗವಂತನಲ್ಲಿ ಮಾತ್ರ ಆಶ್ರಯವನ್ನು ಪಡೆಯುವನು.

 \item ಎಲ್ಲಿಯವರೆಗೆ ಭಕ್ತಿ ದೃಢವಾಗಿಲ್ಲವೋ ಅಲ್ಲಿಯವರೆಗೆ ಶಾಸ್ತ್ರಗಳನ್ನು ಅನುಸರಿಸಬೇಕು.

 \item ಇಲ್ಲದೇ ಇದ್ದರೆ ಸ್ವಾತಂತ್ರ್ಯದ ಹೆಸರಿನಲ್ಲಿ ಪತಿತರಾಗುವ ಸಂಭವ ಇದೆ.

 \item ಭಕ್ತಿಯು ದೃಢವಾದ ಮೇಲೆ ಲೌಕಿಕ ವ್ಯವಹಾರಗಳನ್ನು ಕೂಡ ತ್ಯಜಿಸುವನು, ಶರೀರಧಾರಣೆಗೆ ಬೇಕಾದುದನ್ನು ಮಾತ್ರ ಮಾಡುವನು.

 \item ಭಕ್ತಿಯನ್ನು ಕುರಿತಂತೆ ಹಲವು ವಿವರಣೆಗಳಿವೆ. ಆದರೆ ನಾರದರು ಕೆಳಗಿನ ವಿವರಣೆಯನ್ನು ಕೊಡುವರು. ಕಾಯೇನ, ವಾಚಾ, ಮನಸಾ ಕರ್ಮ ಗಳನ್ನೆಲ್ಲ ಭಗವಂತನಿಗೆ ಅರ್ಪಿಸುವುದು. ಅವನನ್ನು ಕ್ಷಣಕಾಲ ಮರೆತರೂ ಭಕ್ತನು ಪರಮ ವ್ಯಾಕುಲಕ್ಕೆ ತುತ್ತಾಗುವನು. ಆಗ ಪರಮಪ್ರೇಮ ಪ್ರಾರಂಭವಾಗಿದೆ ಎಂದು ಅರಿಯಬೇಕು.

 \item ವ್ರಜಗೋಪಿಯರಿಗೆ (ಕೃಷ್ಣನ ಮೇಲೆ) ಇದ್ದ ಪ್ರೇಮದಂತೆ.

 \item ಅವರು ದೇವರನ್ನು ತಮ್ಮ ಪ್ರಿಯತಮನೆಂದು ಪ್ರೀತಿಸುತ್ತಿದ್ದರೂ ಅವನ ಮಹಾತ್ಮೆಯ ಜ್ಞಾನವನ್ನು ಎಂದಿಗೂ ಮರೆತಿರಲಿಲ್ಲ.

 \item ಹಾಗೆ ಮರೆತಿದ್ದರೆ ಅವರು ಜಾರೆಯರಂತೆ ಆಗುತ್ತಿದ್ದರು.

 \item ಅದೇ (ಗೋಪಿಯರ ಪ್ರೀತಿಯೇ) ಪರಮಾದರ್ಶ. ಅಲ್ಲಿ ಮಾನವ ಪ್ರೇಮದಲ್ಲಿ ಸಾಧಾರಣವಾಗಿರುವ ಕೊಟ್ಟು ತೆಗೆದುಕೊಳ್ಳುವ-ಸ್ವಭಾವವಿಲ್ಲ.

\end{enumerate}

\begin{center}
೨
\end{center}

\begin{enumerate}
\item ಭಕ್ತಿಯು, ಕರ್ಮ, ಜ್ಞಾನ, ಯೋಗಗಳಿಗಿಂತಲೂ ಅಧಿಕತರವಾದುದು. ಏಕೆಂದರೆ ಭಕ್ತಿಯೇ ಫಲರೂಪವಾದುದು, ಭಕ್ತಿಯೇ ಸಾಧನ ಮತ್ತು ಸಾಧ್ಯ.

 \item ಆಹಾರದ ವಿಷಯವನ್ನು ಒಬ್ಬ ಅರಿತಿದ್ದರೆ ಅಥವಾ ನೋಡಿದರೆ ಹೇಗೆ ತೃಪ್ತಿಯುಂಟಾಗುವುದಿಲ್ಲವೋ ಅದರಂತೆಯೇ ಒಬ್ಬನಿಗೆ ದೇವರ ವಿಷಯವನ್ನು ಅರಿತರೆ ಅಥವಾ ದೇವರನ್ನು ನೋಡಿದರೆ ತೃಪ್ತಿಯಾಗುವುದಿಲ್ಲ. ಪರಮಪ್ರೇಮ ದೊರಕಬೇಕು. ಆದಕಾರಣ ಭಕ್ತಿಯೇ ಶ್ರೇಷ್ಠ.

\end{enumerate}

\begin{center}
೩
\end{center}

\begin{enumerate}
\item ಭಕ್ತಿಯ ವಿಷಯವಾಗಿ ಆಚಾರ್ಯರು ಹೀಗೆ ಹೇಳುವರು;

 \item ಯಾರಿಗೆ ಆ ಭಕ್ತಿ ಬೇಕೊ ಅವರು ವಿಷಯತ್ಯಾಗವನ್ನು ಮತ್ತು ಸಂಗತ್ಯಾಗವನ್ನು ಮಾಡಬೇಕು.

 \item ಹಗಲು ರಾತ್ರಿ ಅವರು ಭಕ್ತಿಯ ವಿಷಯವನ್ನು ಮಾತ್ರ ಕುರಿತು ಚಿಂತಿಸುತ್ತಿರಬೇಕು.

 \item ಎಲ್ಲಿ ಭಗವಂತನ ವಿಷಯಗಳನ್ನು ಗಾಯನಮಾಡುತ್ತಿರುವರೊ, ಎಲ್ಲಿ ಅವನ ವಿಷಯವನ್ನು ಮಾತನಾಡುತ್ತಿರುವರೊ ಅಲ್ಲಿಗೆ ಹೋಗಬೇಕು.

 \item ಭಕ್ತಿಗೆ ಮುಖ್ಯವಾದ ಕಾರಣ ಮಹಾತ್ಮರ ಕೃಪೆ.

 \item ಮಹಾತ್ಮರ ಸಂಗ ಪ್ರಪಂಚದಲ್ಲಿ ದುರ್ಲಭ; ಅದು ನಮ್ಮನ್ನು ಉದ್ಧರಿಸುವುದರಲ್ಲಿ ಸಂದೇಹವಿಲ್ಲ.

 \item ಭಗವಂತನ ಕೃಪೆಯಿಂದ ಅಂತಹ ಗುರುಗಳು ನಮಗೆ ಲಭಿಸುವರು.

 \item ಭಗವಂತನಿಗೂ ಅವನ ಭಕ್ತರಿಗೂ ಯಾವ ವ್ಯತ್ಯಾಸವೂ ಇಲ್ಲ.

 \item ಆದಕಾರಣ ಅಂಥವರನ್ನು ಮಾತ್ರ ಆರಸಿ.

 \item ದುಸ್ಸಂಗವನ್ನು ಸರ್ವಥಾ ತ್ಯಜಿಸಬೇಕು.

 \item ಏಕೆಂದರೆ ಇದರಿಂದಲೇ ಕಾಮ ಕ್ರೋಧ ಮೋಹ ಮದ ಬುದ್ಧಿನಾಶ ಕೊನೆಗೆ ಸರ್ವನಾಶ.

 \item ಅದು ಮೊದಲು ಅಲೆಯಂತೆ ಎದ್ದರೂ, ದುಸ್ಸಂಗ ಅದನ್ನು ಒಂದು ಸಾಗರದಂತೆ ಮಾಡುವುದು.

 \item ಯಾರು ಮೋಹವನ್ನು ತ್ಯಜಿಸುವರೋ, ಮಹಾನುಭಾವರಿಗೆ ಸೇವೆ ಸಲ್ಲಿಸುವರೋ, ಯಾರು ನಿರ್ಮಲರಾಗಿರುವರೋ, ಯಾರು ನಿರ್ಜನ ಪ್ರದೇಶದಲ್ಲಿ ವಾಸಿಸುವರೋ ಲೋಕಬಂಧನಗಳನ್ನು ನಾಶಮಾಡುವರೋ, ತ್ರಿಗುಣಗಳಿಗೆ ಅತೀತರಾಗಿರುವರೊ, ಯೋಕ್ಷೇಮವನ್ನು ತ್ಯಜಿಸುವರೋ, ಅವರು ಮಾಯೆಯನ್ನೂ ದಾಟುವರು.

 \item ಯಾರು ಕರ್ಮಫಲವನ್ನು ತ್ಯಜಿಸುವರೊ, ಕರ್ಮಗಳನ್ನು ತ್ಯಜಿಸುವರೊ, ಸುಖದುಃಖಗಳನ್ನು ತ್ಯಜಿಸುವರೊ, ಯಾರು ಶಾಸ್ತ್ರಗಳನ್ನು ಕೂಡ ತ್ಯಜಿಸುವರೊ, ಅಂಥವರಿಗೆ ಕೇವಲವಾದ ಅವಿಚ್ಛಿನ್ನವಾದ ಅನುರಾಗ ಪ್ರಾಪ್ತವಾಗುವುದು.

 \item ಅವನು (ಮಾಯೆಯನ್ನು) ದಾಟುತ್ತಾನೆ. ಇತರರನ್ನೂ ದಾಟಿಸುತ್ತಾನೆ.

\end{enumerate}

\begin{center}
೪
\end{center}

\begin{enumerate}
\item ಪ್ರೇಮವು ಅನಿರ್ವಚನೀಯವಾದುದು

 \item ಮೂಗ ತಾನು ರುಚಿನೋಡಿರುವುದನ್ನು ವಿವರಿಸುವುದಕ್ಕೆ ಹೇಗೆ ಆಗುವುದಿಲ್ಲವೋ, ಆದರೆ ಅವನ ಸನ್ನೆಗಳು ಅದನ್ನು ವ್ಯಕ್ತಪಡಿಸುವುವೊ, ಹಾಗೆಯೇ ಪರಮಭಕ್ತಿಯನ್ನು ಮಾತಿನ ಮೂಲಕ ವಿವರಿಸಲು ಅಸಾಧ್ಯ. ಆದರೆ ಅವನ ಜೀವನವೇ ಅದನ್ನು ನಿದರ್ಶಿಸುವುದು.

 \item ಕೆಲವು ಅಪೂರ್ವ ವ್ಯಕ್ತಿಗಳಲ್ಲಿ ಇದು ಪ್ರಕಾಶವಾಗುವುದು.

 \item ಆ ಪರಮಪ್ರೇಮವು ಗುಣಗಳಿಗೆ ಅತೀತವಾದುದು, ಕಾಮನೆಗಳಿಲ್ಲ ದುದು, ಪ್ರತಿ ಕ್ಷಣವೂ ವೃದ್ಧಿಯಾಗುತ್ತಿರುವುದು, ಅವಿಚ್ಛಿನ್ನವಾದುದು, ಸೂಕ್ಷ್ಮ ತರವಾದುದು, ಅನುಭವರೂಪವಾದುದು.

 \item ಪರಮಪ್ರೇಮವು ಪ್ರಾಪ್ತವಾದರೆ, ಅದನ್ನೇ ಎಲ್ಲಾ ಕಡೆಗಳಲ್ಲಿಯೂ ಓದುವನು, ಅದರ ವಿಷಯಗಳನ್ನೇ ಎಲ್ಲಾ ಕಡೆಗಳಲ್ಲಿಯೂ ಹೇಳುವನು, ಅದರ ವಿಷಯವನ್ನೇ ಎಲ್ಲಾ ಕಡೆಗಳಲ್ಲಿಯೂ ಮಾತನಾಡುವನು, ಅದನ್ನೇ ಕುರಿತು ಎಲ್ಲಾ ಕಡೆಗಳಲ್ಲಿಯೂ ಚಿಂತಿಸುವನು.

 \item ಗುಣಭೇದದಿಂದ ಅಥವಾ ಸ್ಥಿತಿಭೇದದಿಂದ ಈ ಪ್ರೇಮವು ಬೇರೆ ಬೇರೆಯಾಗಿ ವ್ಯಕ್ತವಾಗುವುದು.

 \item ಗುಣಭೇದವೆಂದರೆ, ತಮಸ್ಸು, ರಜಸ್ಸು, ಸತ್ತ್ವಗುಣಗಳು, ಸ್ಥಿತಿಭೇದ ವೆಂದರೆ ಆರ್ತ, ಅರ್ಥಾರ್ಥಿ, ಜಿಜ್ಞಾಸು, ಜ್ಞಾನಿ.

 \item ಅವುಗಳಲ್ಲಿ ಹಿಂದಿನದಕ್ಕಿಂತ ಮುಂದಿನದೇ ಶ್ರೇಯಸ್ಕರವಾದುದು.

 \item ಭಕ್ತಿಯು ಉಪಾಸನೆಗೆ ಬಹಳ ಸುಲಭವಾದ ಮಾರ್ಗ.

 \item ಇದು ಸ್ವಯಂಪ್ರಮಾಣರೂಪವಾಗಿರುವುದು. ಇದಕ್ಕೆ ಬೇರೆ ಯಾವ ಪ್ರಮಾಣವೂ ಬೇಕಿಲ್ಲ.

 \item ಇದು ಶಾಂತಿರೂಪವಾದುದು, ಪರಮಾನಂದರೂಪವಾದುದು.

 \item ಭಕ್ತಿ ಯಾರಿಗೂ, ಯಾವುದಕ್ಕೂ ಹಾನಿಯನ್ನು ತರುವುದಿಲ್ಲ. ಸಾಮಾನ್ಯ ಪೂಜಾವಿಧಾನಗಳನ್ನೂ ಅಲ್ಲಗಳೆಯುವುದಿಲ್ಲ.

 \item ಕಾಮಕ್ಕೆ ಸಂಬಂಧಪಟ್ಟದ್ದನ್ನು, ದೇವರ ವಿಷಯದಲ್ಲಿ ಅನುಮಾನವನ್ನು ಹುಟ್ಟಿಸುವಂತಹದನ್ನು ಮತ್ತು ತಮ್ಮ ವೈರಿಗಳ ಬಗ್ಗೆ ಸಂಭಾಷಣೆ ಇತ್ಯಾದಿಗಳನ್ನು ಕೇಳಿಸಿಕೊಳ್ಳಕೂಡದು.

 \item ಅಹಂಭಾವ, ಡಂಭ ಮುಂತಾದವುಗಳನ್ನು ತ್ಯಜಿಸಬೇಕು.

 \item ಆ ಕಾಮನೆಗಳನ್ನು ನಿಗ್ರಹಿಸಲು ಸಾಧ್ಯವಿಲ್ಲದೇ ಇದ್ದರೆ, ಅವುಗಳನ್ನು ದೇವರಿಗೆ ಅರ್ಪಿಸಬೇಕು, ಸರ್ವ ಕರ್ಮಗಳನ್ನು ಅವನಿಗೆ ಅರ್ಪಿಸಬೇಕು.

 \item ಭಕ್ತಿ, ಭಕ್ತ, ಭಗವಾನ್​ ಎಂದು ತ್ರಿರೂಪವಾಗಿರುವ ಭಕ್ತಿಯನ್ನು ಒಂದಾಗಿಸಿ ನಿತ್ಯದಾಸನಂತೆ, ನಿತ್ಯಕಾಂತೆಯಂತೆ ಪ್ರೇಮವನ್ನೇ ಏಕೈಕ ಉದ್ದೇಶ ವಾಗಿಟ್ಟುಕೊಂಡು ಭಜಿಸಬೇಕು.

\end{enumerate}

\begin{center}
೫
\end{center}

\begin{enumerate}
\item ಭಗವಂತನ ಮೇಲೆ ಕೇಂದ್ರೀಕೃತವಾದ ಭಕ್ತಿಯುಳ್ಳವರೇ ಶ್ರೇಷ್ಠರು.

 \item ಅವರು ಭಗವಂತನ ವಿಷಯವನ್ನು ಇತರರೊಡನೆ ಗದ್ಗದ ಕಂಠದಿಂದ ಮಾತನಾಡುತ್ತಾರೆ. ಮತ್ತು ಗಟ್ಟಿಯಾಗಿ ಅಳುತ್ತಾರೆ. ಅವರು ತೀರ್ಥಗಳನ್ನು ಮತ್ತೂ ಪವಿತ್ರವಾದ ತೀರ್ಥಗಳನ್ನಾಗಿ ಮಾಡುವರು. ಕರ್ಮಗಳನ್ನು ಸುಕರ್ಮಗಳನ್ನಾಗಿ ಮಾಡುವರು. ಶಾಸ್ತ್ರಗಳನ್ನು ಸತ್​ಶಾಸ್ತ್ರಗಳನ್ನಾಗಿ ಮಾಡುವರು. ಏಕೆಂದರೆ ಅವರು ಭಗವಂತನಲ್ಲಿ ತನ್ಮಯರಾಗಿರುವರು.

 \item ಯಾವಾಗ ಭಕ್ತನು ದೇವರನ್ನು ಇಷ್ಟು ಪ್ರೀತಿಸುವನೊ ಆಗ, ಅವನ ಪಿತೃಗಳು ಸಂತೋಷಪಡುವರು, ದೇವತೆಗಳು ನೃತ್ಯಮಾಡುವರು, ಈ ಭೂಮಿಗೆ ಒಬ್ಬ ರಕ್ಷಕ ದೊರೆಯುವನು.

 \item ಅಂತಹ ಭಕ್ತರಲ್ಲಿ ಜಾತಿ ಲಿಂಗ ವಿದ್ಯಾ ಕುಲ ರೂಪ ಧನ ಕ್ರಿಯಾ ಭೇದಗಳಿಲ್ಲ.

 \item ಏಕೆಂದರೆ ಎಲ್ಲರೂ ಭಗವಂತನಿಗೆ ಸೇರಿದವರು.

 \item ವಾದಗಳನ್ನು ತ್ಯಜಿಸಬೇಕು.

 \item ಏಕೆಂದರೆ ಇದಕ್ಕೆ ಒಂದು ಕೊನೆಯಿಲ್ಲ, ಇದರಿಂದ ಯಾವ ತೃಪ್ತಿಕರ ವಾದ ಪರಿಣಾಮವೂ ಉಂಟಾಗುವುದಿಲ್ಲ.

 \item ಭಕ್ತಿಶಾಸ್ತ್ರವನ್ನು ಮನನಮಾಡಬೇಕು, ಅದನ್ನು ವೃದ್ಧಿ ಮಾಡುವ ಕರ್ಮಗಳನ್ನು ಮಾಡಬೇಕು.

 \item ಸುಖ ದುಃಖ ಲಾಭ ನಷ್ಟಗಳನ್ನು ತ್ಯಜಿಸಿ ಹಗಲು ರಾತ್ರಿ ಭಗವಂತನನ್ನು ಭಜಿಸಬೇಕು. ಕ್ಷಣಾರ್ಧವನ್ನು ಕೂಡ ವ್ಯರ್ಥ ಮಾಡಕೂಡದು.

 \item ಅಹಿಂಸೆ ಸತ್ಯ ಶೌಚ ದಯೆ ಆಸ್ತಿಕ್ಯಬುದ್ಧಿ ಮುಂತಾದವುಗಳನ್ನು ರೂಢಿಸಬೇಕು.

 \item ಉಳಿದ ಎಲ್ಲಾ ಭಾವನೆಗಳನ್ನೂ ತ್ಯಜಿಸಿ ಸರ್ವದಾ ಸರ್ವಭಾವಗಳಿಂದಲೂ ಭಗವಂತನನ್ನೇ ಭಜಿಸಬೇಕು. ಹೀಗೆ ಭಜಿಸಿದಾಗ ಭಗವಂತನು ಕಾಣಿಸಿಕೊಳ್ಳು ವನು, ಭಕ್ತರಿಗೆ ಅನುಭವವನ್ನು ಕೊಡುವನು.

 \item ಭೂತ, ವರ್ತಮಾನ, ಭವಿಷ್ಯತ್ತುಗಳಲ್ಲೂ ಭಕ್ತಿಯೇ ಶ್ರೇಷ್ಠ.

 \item ಪ್ರಪಂಚದ ಟೀಕೆಗೆ ಮನಗೊಡದೆ, ಪುರಾತನ ಋಷಿಗಳನ್ನು ಅನುಸರಿಸಿ ನಾವು ಭಗವದ್​ ಭಕ್ತಿಯ ಸಿದ್ಧಾಂತವನ್ನು ಧ್ಯೆರ್ಯದಿಂದ ಹೇಳಿರುವೆವು.

\end{enumerate}

