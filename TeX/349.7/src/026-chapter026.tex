
\chapter[ಭಕ್ತಿಯೋಗವನ್ನು ಕುರಿತು ]{ಭಕ್ತಿಯೋಗವನ್ನು ಕುರಿತು \protect\footnote{\engfoot{C.W. Vol. VI, P. 137}}}

ರಾಜಯೋಗವನ್ನು ಮತ್ತು ಆಸನಗಳನ್ನು ಕುರಿತು ನಾವು ಇದುವರೆಗೆ ವಿಚಾರಮಾಡುತ್ತಿದ್ದೆವು. ಈಗ ಭಕ್ತಿಯೋಗವನ್ನು ತೆಗೆದುಕೊಳ್ಳೋಣ. ಈ ಯೋಗಗಳಲ್ಲಿ ಯಾವುದಾದರೂ ಒಂದು ಎಲ್ಲರಿಗೂ ಆವಶ್ಯಕ ಎಂದಲ್ಲ. ಹಲವು ಆದರ್ಶಗಳನ್ನು ನಿಮ್ಮ ಮುಂದೆ ಇರಿಸುವೆನು. ಅವುಗಳಲ್ಲಿ ಒಂದಲ್ಲದೆ ಇದ್ದರೆ ಮತ್ತೊಂದು ನಿಮಗೆ ಹಿಡಿಸಬಹುದು.

\vskip 6pt

ನಮ್ಮಲ್ಲಿ ಸಾಮರಸ್ಯವಿರಬೇಕು. ಜ್ಞಾನ, ಭಕ್ತಿ, ಕರ್ಮ, ಧ್ಯಾನ ಎಲ್ಲವೂ ನಮ್ಮಲ್ಲಿ ಸಮ\break ತೂಕದಲ್ಲಿ ಸೇರಿರಬೇಕು. ವ್ಯಕ್ತಿಗಳು ಮತ್ತು ಜನಾಂಗಗಳು ಇವುಗಳಲ್ಲಿ ಯಾವುದಾದರೂ ಒಂದನ್ನೇ ತೆಗೆದುಕೊಂಡು ಉಳಿದವುಗಳನ್ನು ಅರ್ಥಮಾಡಿಕೊಳ್ಳುವುದಿಲ್ಲ. ನಾವು ಬಹು\break ಮುಖರಾಗಬೇಕು, ಇದೇ ಆದರ್ಶ. ಜಗತ್ತಿನ ದುಃಖಕ್ಕೆಲ್ಲಾ ಮುಖ್ಯಕಾರಣ ನಾವೆಲ್ಲ ಯಾವುದೋ ಒಂದು ಮಾರ್ಗದಲ್ಲಿ ಮಾತ್ರ ಬೆಳೆದು, ಇತರ ಮಾರ್ಗಗಳನ್ನು ಅನುಸರಿಸು\break ವವರ ವಿಷಯದಲ್ಲಿ ಸಹಾನುಭೂತಿಯನ್ನು ತೋರದೆ ಇರುವುದು. ಭೂಮಿಯೊಳಗಿದ್ದು ಅಲ್ಲಿಂದ ಒಂದು ರಂಧ್ರದ ಮೂಲಕ ಸೂರ್ಯನನ್ನು ನೋಡುತ್ತಿರುವ ಒಬ್ಬನನ್ನು ತೆಗೆದುಕೊಳ್ಳಿ, ಅವನು ಸೂರ್ಯನ ಒಂದು ಅಂಶವನ್ನು ನೋಡುತ್ತಾನೆ. ಮತ್ತೊಬ್ಬನು ಭೂಮಿಯ ಮೇಲೆ ನಿಂತುಕೊಂಡು ಸೂರ್ಯನನ್ನು ನೋಡುತ್ತಿರುವನು. ಮತ್ತೊಬ್ಬ ಮಂಜಿನ ಮೂಲಕ ಅವನನ್ನು ನೋಡುತ್ತಿರುವನು. ಮತ್ತೊಬ್ಬನು ಬೆಟ್ಟದ ಮೇಲಿಂದ ನೋಡುತ್ತಾನೆ. ಪ್ರತಿಯೊಬ್ಬರಿಗೂ ಸೂರ್ಯ ಬೇರೆ ಬೇರೆ ರೀತಿಯಲ್ಲಿ ಕಾಣಿಸುವನು. ಹಾಗೆಯೇ ಎಷ್ಟೋ ದೃಶ್ಯಗಳಿವೆ. ಆದರೆ ನಿಜವಾಗಿರುವುದು ಒಂದೇ ಸೂರ್ಯ. ಹಲವರಿಗೆ ಹಲವು ವಿಧಗಳಲ್ಲಿ ಕಾಣಿಸುತ್ತಿದ್ದರೂ ಇರುವ ವಸ್ತು ಒಂದೇ. ಅದೇ ಸೂರ್ಯ.

\vskip 6pt

ಪ್ರತಿಯೊಬ್ಬನೂ ತನ್ನ ಸ್ವಭಾವಕ್ಕೆ ಅನುಗುಣವಾಗಿ ಯಾವುದಾದರೂ ಒಂದು ಆದರ್ಶವನ್ನು ತೆಗೆದುಕೊಂಡು ಒಂದು ಮಾರ್ಗದಲ್ಲಿ ಮುಂದುವರಿಯುವನು. ಆದರೆ ಎಲ್ಲರ ಗುರಿಯೂ ಒಂದೇ. ರೋಮನ್​ ಕ್ಯಾಥೊಲಿಕ್​ರಲ್ಲಿ ಆಳವಿದೆ, ಆಧ್ಯಾತ್ಮವಿದೆ. ಆದರೆ ವೈಶಾಲ್ಯ ಇಲ್ಲ. ಯೂನಿಟೇರಿಯನ್ನರಲ್ಲಿ ವೈಶಾಲ್ಯ ಇದೆ. ಆದರೆ ಅವರಲ್ಲಿ ಆಧ್ಯಾತ್ಮವಿಲ್ಲ. ಧರ್ಮ ಗೌಣವಾಗಿದೆ. ನಮಗೆ ಬೇಕಾಗಿರುವುದು ರೋಮನ್​ ಕ್ಯಾಥೊಲಿಕರ ಆಳ, ಯೂನಿಟೇರಿಯನ್ನರ ವೈಶಾಲ್ಯ; ನಾವು ಆಕಾಶದಷ್ಟು ವಿಶಾಲವಾಗಿರಬೇಕು, ಸಾಗರದಷ್ಟು ಆಳವಾಗಿರಬೇಕು. ಅಜ್ಞೇಯತಾವಾದಿಯ ವೈಶಾಲ್ಯ ಇರಬೇಕು. ಸಹಿಷ್ಣುತೆ ಎಂಬುದಕ್ಕೆ ಅಹಂಕಾರದಿಂದ ಕೂಡಿದ ಮಾನವನು ಬೇರೆ ಒಂದು ಅರ್ಥವನ್ನು ಕಲ್ಪಿಸಿದ್ದಾನೆ. ಅದು ತಾನು ಇತರರಿಗಿಂತ ಎತ್ತರದಲ್ಲಿ ಇದ್ದೇನೆ ಎಂದು ಭಾವಿಸಿ ಇತರರನ್ನು ಕನಿಕರ ದೃಷ್ಟಿಯಿಂದ ನೋಡುವುದು. ಇದು ಮನಸ್ಸಿನ ಶೋಚನೀಯ ಸ್ಥಿತಿ. ನಾವೆಲ್ಲ ಒಂದೇ ಗುರಿಯೆಡೆಗೆ ಹೋಗುತ್ತಿರುವೆವು. ಆದರೆ ನಮ್ಮ ನಮ್ಮ ಸ್ವಭಾವಕ್ಕೆ ಅನುಗುಣವಾದ ಬೇರೆ ಬೇರೆ ದಾರಿಗಳಲ್ಲಿ ಹೋಗುತ್ತಿರುವೆವು, ಅಷ್ಟೇ. ನಮ್ಮ ಮನಸ್ಸು ಬಹುಮುಖಿಯಾಗಬೇಕು, ನಮ್ಮ ಶೀಲದಲ್ಲಿ ವೈವಿಧ್ಯ ಇರಬೇಕು. ನಾವು ಇನ್ನೊಬ್ಬನನ್ನು ಸಹಿಸುವುದು ಮಾತ್ರವಲ್ಲ, ಅದಕ್ಕಿಂತ ಕಷ್ಟವಾಗಿರುವುದನ್ನು ಮಾಡಬೇಕು; ಅವನೊಡನೆ ಸಹಾನುಭೂತಿಯನ್ನು ತೋರಿ, ಅವನ ಮಾರ್ಗದಲ್ಲಿ ನಡೆದು ಅವನು ಹೇಗೆ ದೇವರನ್ನು ಅನುಭವಿಸುತ್ತಾನೊ ಹಾಗೆ ಅನುಭವಿಸಬೇಕು. ಪ್ರತಿಯೊಂದು ಧರ್ಮದಲ್ಲಿಯೂ ಎರಡು ಭಾಗಗಳಿವೆ. ಒಂದು ಇತಿಮಾರ್ಗ, ಮತ್ತೊಂದು ನೇತಿಮಾರ್ಗ. ನೀವು ತಂದೆ, ಹೋಲಿಘೋಸ್ಟ್​, ಮತ್ತು ಕ್ರಿಸ್ತ ಇವರ ಮೂಲಕ ಮುಕ್ತಿ ಎಂದು ಹೇಳುವಾಗ ನಾನು ನಿಮ್ಮ ಅಭಿಪ್ರಾಯವನ್ನು ಒಪ್ಪುತ್ತೇನೆ. ಒಳ್ಳೆಯದು, ನಾನು ಕೂಡ ಇದನ್ನು ಒಪ್ಪುತ್ತೇನೆ ಎನ್ನುವೆನು. ಆದರೆ ನೀವು, ಬೇರೆ ನಿಜವಾದ ಧರ್ಮವೇ ಇಲ್ಲ, ಬೇರೆ ಅವತಾರವೇ ಇಲ್ಲ ಎಂದಾಗ “ನಿಲ್ಲಿ, ನೀವು ಇತರರನ್ನು ಅಲ್ಲಗಳೆದಾಗ, ಇತರರನ್ನು ನೀವು ನಿಮ್ಮೊಡನೆ ಸೇರಿಸದೆ ಇದ್ದಾಗ ನಾನು ನಿಮ್ಮ ಅಭಿಪ್ರಾಯವನ್ನು ಒಪ್ಪುವುದಿಲ್ಲ” ಎನ್ನುವೆನು. ಪ್ರತಿಯೊಂದು ಧರ್ಮವೂ ಒಂದು ಸಂದೇಶವನ್ನು ಜಗತ್ತಿಗೆ ಕೊಡಬೇಕಾಗಿದೆ; ಮಾನವನಿಗೆ ಒಂದು ಉಪದೇಶವನ್ನು ಬೋಧಿಸಬೇಕಾಗಿದೆ. ಆದರೆ ಒಂದು ಧರ್ಮ ಇತರರನ್ನು ಅಲ್ಲಗಳೆದಾಗ, ಇತರರಿಗೆ ಅಶಾಂತಿಯನ್ನುಂಟುಮಾಡುವಾಗ ಅದು ನಿಷೇಧಾತ್ಮಕವಾದ ಅಪಾಯಕರವಾದ ಮಾರ್ಗವನ್ನು ಅನುಸರಿಸುವುದು, ಅದಕ್ಕೆ ಎಲ್ಲಿ ಮೊದಲು ಮಾಡಬೇಕು ಎಲ್ಲಿ ಕೊನೆಗಾಣಿಸಬೇಕು ಎಂಬುದು ಗೊತ್ತಿರುವುದಿಲ್ಲ.

\vskip 6pt

ಪ್ರತಿಯೊಂದು ಶಕ್ತಿಯೂ ವೃತ್ತವನ್ನು ಪೂರೈಸುವುದು. ಮಾನವ ಎಂಬ ಶಕ್ತಿ ತರಂಗ ದೇವರಿಂದ ಬಂದಿದೆ. ಪುನಃ ಅದು ದೇವರನ್ನು ಸೇರುವುದು. ದೇವರಲ್ಲಿಗೆ ಹಿಂತಿರುಗಿ ಹೋಗುವುದಕ್ಕೆ ಎರಡು ಮಾರ್ಗಗಳಲ್ಲಿ ಯಾವುದಾದರೂ ಒಂದನ್ನು ಅನುಸರಿಸಬೇಕಾಗಿದೆ. ಒಂದು ಪ್ರಕೃತಿಯೊಡನೆ ನಿಧಾನವಾಗಿ ಹಿಂದಿರುಗುವುದು, ಇಲ್ಲವೆ ನಮ್ಮ ಸ್ವಂತ ಶಕ್ತಿಯಿಂದ ಪ್ರವಾಹಕ್ಕೆ ಎದುರಾಗಿ ಹೋಗಿ ದೇವರನ್ನು ನೋಡುವುದು; ಅಂದರೆ ಬಲಾತ್ಕಾರದಿಂದ ಸಾಧನೆ ಮಾಡಿ ಹತ್ತಿರದ ಹಾದಿಯಲ್ಲಿ ದೇವರನ್ನು ಕಾಣುವುದು. ಇದನ್ನೇ ಯೋಗಿ\break ಮಾಡುವುದು.

\vskip 6pt

ಪ್ರತಿಯೊಬ್ಬನೂ ತನ್ನ ಸ್ವಭಾವಕ್ಕೆ ಅನುಗುಣವಾದ ಆದರ್ಶವನ್ನು ತೆಗೆದುಕೊಳ್ಳಬೇಕಾಗಿದೆ ಎಂದು ಹೇಳಿದ್ದೇನೆ. ಅದನ್ನೇ ಇಷ್ಟ ಎನ್ನುವುದು. ಅದನ್ನು ನೀವು ರಹಸ್ಯವಾಗಿಟ್ಟಿರಬೇಕು. ನೀವು ಭಗವಂತನನ್ನು ಪೂಜಿಸುವಾಗ ನಿಮ್ಮ ಇಷ್ಟದ ಮೂಲಕ ಪೂಜಿಸಿ. ದಾರಿಯನ್ನು ನಾವು ಕಂಡುಹಿಡಿಯುವುದು ಹೇಗೆ? ಅದು ಬಹಳ ಕಷ್ಟ. ಆದರೆ ಸಾಧನೆಯನ್ನು ಪಟ್ಟಾಗಿ ಮುಂದುವರಿಸಿದರೆ ಅದು ತಾನಾಗಿಯೇ ಗೊತ್ತಾಗುತ್ತದೆ. ದೇವರು ಮಾನವನಿಗೆ ಕೊಡುವ ಬಹು ಮುಖ್ಯವಾದ ಮೂರು ವರಗಳಿವೆ-ಮಾನವದೇಹ, ಮುಕ್ತನಾಗಬೇಕೆಂಬ ಆಕಾಂಕ್ಷೆ, ಮತ್ತು ಮುಕ್ತನಾದ ಮಹಾಪುರುಷನೊಬ್ಬನಿಂದ ಸಹಾಯ ಪಡೆಯುವುದು. ಸಗುಣ ದೇವರಿಲ್ಲದೇ ಇದ್ದರೆ ನಮಗೆ ಭಕ್ತಿ ದೊರಕಲಾರದು. ಭಗವಂತ ಮತ್ತು ಭಕ್ತ ಇಬ್ಬರೂ ಇರಬೇಕು. ದೇವರು ಎಂದರೆ ಅನಂತರೂಪವನ್ನು ತಾಳಿದ ಮಾನವನೇ. ಹೀಗಿಲ್ಲದೇ ಸಾಧ್ಯವಿಲ್ಲ. ನಾವು ಎಲ್ಲಿಯವರೆಗೆ ಮಾನವರಾಗಿರುವೆವೋ ಅಲ್ಲಿಯವರೆಗೆ ನಮಗೆ ಮಾನವ ದೇವರು ಬೇಕು. ನಾವು ಒಬ್ಬ ಸಾಕಾರ ದೇವರನ್ನು ಮಾತ್ರ ನೋಡಲು ಸಾಧ್ಯ. ಬೇರೆ ಸಾಧ್ಯವೇ ಇಲ್ಲ. ಈ ಪ್ರಪಂಚದಲ್ಲಿ ಕಾಣಿಸುವುದೆಲ್ಲ ವಸ್ತು ಮಾತ್ರವಲ್ಲ. ಅದು ವಸ್ತು + ನಮ್ಮ ಮನಸ್ಸು ಆಗಿರುವುದು. ಕುರ್ಚಿ ಮತ್ತು ಆ ಕುರ್ಚಿಯು ನಮ್ಮ ಮನಸ್ಸಿನ ಮೇಲೆ ಮಾಡಿರುವ ಪರಿಣಾಮವೇ ಈಗ ಕಾಣಿಸುವ ಕುರ್ಚಿ. ನೀವು ಪ್ರತಿಯೊಂದನ್ನೂ ನಿಮ್ಮ ಮನಸ್ಸಿನಿಂದ ಆವರಿಸಬೇಕು. ಆಗ ಮಾತ್ರ ನೀವು ಅದನ್ನು ನೋಡಬಲ್ಲಿರಿ. ಉದಾಹರಣೆಗೆ ಒಂದು ಬಿಳಿಯ, ಚೌಕಾಕಾರದ, ಹೊಳೆಯುತ್ತಿರುವ, ಮುಚ್ಚಲು ಗಟ್ಟಿಯಾದ ಒಂದು ಪೆಟ್ಟಿಗೆ ಇದೆ. ಇದನ್ನು ಮೂರು ಇಂದ್ರಿಯಗಳು ಇರುವವನು ನೋಡುತ್ತಾನೆ. ಬಳಿಕ ನಾಲ್ಕು, ಅನಂತರ ಐದು ಇಂದ್ರಿಯಗಳಿರುವವನು ನೋಡುತ್ತಾನೆ. ಐದು ಇಂದ್ರಿಯಗಳು ಇರುವವನು ನೋಡಿದಾಗ ಮಾತ್ರ ಅದು ತನ್ನ ಎಲ್ಲ ಗುಣಗಳಿಂದಲೂ ಕೂಡಿ ಕಾಣಿಸಿಕೊಳ್ಳುವುದು. ಪ್ರತಿಯೊಬ್ಬನೂ ಹಿಂದಿನವನಿಗಿಂತ ಒಂದು ಹೆಚ್ಚಿನ ಗುಣವನ್ನು ಅದರಲ್ಲಿ ಕಾಣುತ್ತಾನೆ. ಆರು ಇಂದ್ರಿಯಗಳು ಉಳ್ಳವನು ಅದೇ ಪೆಟ್ಟಿಗೆಯನ್ನು ನೋಡುತ್ತಾನೆ ಎಂದು ಇಟ್ಟುಕೊಳ್ಳೋಣ. ಆಗ ಅವನಿಗೆ ಅದರಲ್ಲಿ ಮತ್ತೂ ಒಂದು ಗುಣ ಕಾಣುತ್ತದೆ.

ನಾನು ಪ್ರೀತಿಯನ್ನು ಮತ್ತು ಜ್ಞಾನವನ್ನು ನೋಡುವುದರಿಂದ ವಿಶ್ವಚೈತನ್ಯ ಪ್ರೀತಿಯಂತೆ ಮತ್ತು ಜ್ಞಾನದಂತೆ ಕಾಣಿಸುತ್ತಿದೆ ಎಂದು ಭಾವಿಸುತ್ತೇನೆ. ಯಾವುದು ನನ್ನಲ್ಲಿ ಪ್ರೀತಿಯನ್ನು ಹುಟ್ಟಿಸುವುದೋ ಅದು ಹೇಗೆ ಪ್ರೀತಿ ಶೂನ್ಯವಾಗಿರಬಲ್ಲದು? ವಿಶ್ವಚೈತನ್ಯವನ್ನು ಮಾನವ ಗುಣಗಳಿಲ್ಲದೆ ಊಹಿಸಿಕೊಳ್ಳುವುದಕ್ಕೆ ಆಗುವುದಿಲ್ಲ. ಪ್ರಾರಂಭದಲ್ಲಿ ನಮ್ಮಿಂದ ದೇವರು ಪ್ರತ್ಯೇಕವಾಗಿರುವನು ಎಂದು ಆಲೋಚಿಸುವುದು ಅಗತ್ಯ. ಮೂರು ದೇವರ ಆದರ್ಶಗಳಿವೆ. ಅತಿ ಕೆಳಗಿರುವುದೇ, ದೇವರು ನಮ್ಮಂತೆ ದೇಹಧಾರಿಯಾಗಿರುವನು ಎಂದು ಭಾವಿಸುವುದು; ಅನಂತರವೇ ಅವನಲ್ಲಿ ಮಾನವೀಯ ಗುಣಗಳನ್ನು ಆರೋಪ ಮಾಡುವುದು. ಅನಂತರವೇ ಮುಂದೆ ಮುಂದೆ ಹೋದಂತೆ ನಾವು ಭಗವಂತನ ಶ್ರೇಷ್ಠಭಾವನೆಗೆ ಬರುವೆವು.

ಆದರೆ ನಾವು ಈ ಹಂತಗಳಲ್ಲಿ ದೇವರನ್ನಲ್ಲದೆ ಬೇರೆ ಯಾವುದನ್ನೂ ನೋಡುತ್ತಿಲ್ಲ ಎಂಬುದನ್ನು ಗಮನಿಸಿ. ಇದರಲ್ಲಿ ಭ್ರಾಂತಿ ಇಲ್ಲ, ತಪ್ಪಿಲ್ಲ. ನಾವು ಸೂರ್ಯನನ್ನು ಬೇರೆ ಬೇರೆ ಕಡೆಗಳಿಂದ ನೋಡಿದಾಗ ಬೇರೆ ಬೇರೆಯಾಗಿ ಕಂಡರೂ ಅವನು ಸೂರ್ಯನೇ ಆಗಿದ್ದನು; ಅವನೇನೂ ಚಂದ್ರನಾಗಿರಲಿಲ್ಲ, ಬೇರೆ ಏನೂ ಆಗಿರಲಿಲ್ಲ.

ನಾವು ನಮ್ಮಂತಲ್ಲದೆ ಬೇರೆ ರೀತಿಯಾಗಿ ದೇವರನ್ನು ನೋಡಲಾರೆವು. ಅವನು ಅನಂತವಾದರೂ ನಮ್ಮಂತೆಯೇ ಇರುವನು ಎಂದು ಭಾವಿಸುತ್ತೇವೆ. ನಾವು ದೇವರನ್ನು ನಿರ್ಗುಣಬ್ರಹ್ಮ ಎಂದು ಭಾವಿಸಿದರೂ ಅವನನ್ನು ಪ್ರೀತಿಸಬೇಕಾದರೆ, ಅನುಭವಿಸಬೇಕಾದರೆ, ಸಾಪೇಕ್ಷ ಪ್ರಪಂಚಕ್ಕೆ ಅವನು ಬರಲೇಬೇಕಾಗಿದೆ.

ಪ್ರತಿಯೊಂದು ಧರ್ಮದಲ್ಲಿ ನಾವು ನೋಡುವ ಭಕ್ತಿಯನ್ನು ಎರಡು ಭಾಗಗಳಾಗಿ ಮಾಡುವರು. ಒಂದು ಆಚಾರ, ಮಂತ್ರ ಮುಂತಾದವುಗಳ ಮೂಲಕ ಇರುವುದು; ಮತ್ತೊಂದು ಪ್ರೇಮಮಯವಾಗಿರುವುದು. ಈ ಜಗತ್ತಿನಲ್ಲಿ ನಾವು ನಿಯಮಗಳಿಗೆ\break ಬದ್ಧರಾಗಿರುವೆವು; ನಾವು ಯಾವಾಗಲೂ ಇವುಗಳಿಂದ ಪಾರಾಗಲು ಯತ್ನಿಸುತ್ತಿರುವೆವು. ಪ್ರಕೃತಿಯ ಆಜ್ಞೆಯನ್ನು ಉಲ್ಲಂಘಿಸಲು ಯತ್ನಿಸುತ್ತಿರುವೆವು. ಪ್ರಕೃತಿಯನ್ನು ಮೆಟ್ಟಿ ನಿಲ್ಲಲು ಯತ್ನಿಸುತ್ತೇವೆ. ಪ್ರಕೃತಿ ನಮ್ಮನ್ನು ಬೆತ್ತಲೆಯಾಗಿ ಪ್ರಪಂಚಕ್ಕೆ ತಂದಿತು. ಆದರೆ ನಾವು ಬಟ್ಟೆಯನ್ನು ಹಾಕಿಕೊಳ್ಳುವೆವು. ಮಾನವನು ಸ್ವತಂತ್ರನಾಗಿರಲು ಇಚ್ಛಿಸುತ್ತಾನೆ. ನಾವು ಎಲ್ಲಿಯವರೆಗೆ ಪ್ರಕೃತಿಯ ಆಜ್ಞೆಯನ್ನು ಮೀರಿ ಹೋಗಲಾರೆವೋ ಅಲ್ಲಿಯವರೆಗೆ ವ್ಯಥೆಪಡುವೆವು. ನಾವು ನಿಯಮಾತೀತರಾಗುವುದಕ್ಕೋಸ್ಕರವೇ ನಿಯಮವನ್ನು ಪಾಲಿಸುವೆವು. ನಾವು ಜೀವನದಲ್ಲಿ ಹೋರಾಡುವುದೇ ಪ್ರಕೃತಿಯ ಆಜ್ಞೆಯನ್ನು ಉಲ್ಲಂಘಿಸುವುದಕ್ಕಾಗಿ. (ಆದ್ದರಿಂದಲೇ ನನ್ನ ಸಹಾನುಭೂತಿ ಕ್ರಿಶ್ಚಿಯನ್​ ಸೈಂಟಿಸ್ಟರ ಪರವಾಗಿದೆ. ಏಕೆಂದರೆ ಅವರು ಮಾನವನ ಸ್ವಾತಂತ್ರ್ಯವನ್ನು, ಆತ್ಮನ ದಿವ್ಯತೆಯನ್ನು ಬೋಧಿಸುತ್ತಾರೆ.) ಆತ್ಮವು ಸುತ್ತುಮುತ್ತಲಿನ ಸನ್ನಿವೇಶಕ್ಕಿಂತ ಮಿಗಿಲಾದುದು. “ಈ ವಿಶ್ವವೇ ನನ್ನ ತಂದೆಯ ರಾಜ್ಯ. ನಾನೇ ಮುಂದಿನ ಉತ್ತರಾಧಿಕಾರಿ.” ಮಾನವ ಈ ದೃಷ್ಟಿಯಿಂದ ನೋಡಬೇಕು. “ನನ್ನ ಆತ್ಮ ಎಲ್ಲವನ್ನೂ ಹತೋಟಿಯಲ್ಲಿಡಬಲ್ಲದು.”

\vskip 6pt

ನಾವು ಸ್ವತಂತ್ರರಾಗುವುದಕ್ಕೆ ಮುಂಚೆ ನಿಯಮದ ಮೂಲಕ ಹೋಗಬೇಕಾಗಿದೆ. ಬಾಹ್ಯಪೂಜೆ, ಆಚಾರ, ಮತ, ಸಿದ್ಧಾಂತ ಇವುಗಳೆಲ್ಲ ಆಯಾಯಾ ಮೆಟ್ಟಿಲಿನಲ್ಲಿರುವ ಜೀವಿಗಳಿಗೆ ಪ್ರಯೋಜನಕರವಾಗಿವೆ. ನಾವು ಬಲಾಢ್ಯರಾಗುವವರೆಗೆ ಇವು ನಮ್ಮನ್ನು ಸಂರಕ್ಷಿಸುವುವು. ಅನಂತರ ಇವು ಬೇಕಾಗಿಲ್ಲ. ಇವೆಲ್ಲ ನಮ್ಮನ್ನು ಸಲಹುವ ದಾದಿಯರಂತೆ. ನಾವು ಬಾಲಕರಾಗಿರುವಾಗ ಇವುಗಳಿಲ್ಲದೇ ಇದ್ದರೆ ಆಗುವುದಿಲ್ಲ. ಪುಸ್ತಕಗಳು ಕೂಡ ದಾದಿಯರಂತೆ. ಔಷಧಿಗಳು ಕೂಡ ದಾದಿಯರಂತೆ. ಒಬ್ಬನು ತಾನು ತನ್ನ ದೇಹಕ್ಕೆ ಯಜಮಾನನಾಗುವ ಕಾಲ ಬರುವವರೆಗೆ ಪ್ರಯತ್ನಪಡಬೇಕು. ಔಷಧಿ ಮತ್ತು ಮೂಲಿಕೆಗಳು ಕೂಡ ನಮಗೆ ಸಹಾಯ ಮಾಡುತ್ತವೆ. ಆದರೆ ನಾವು ಬಲಾಢ್ಯರಾದ ಮೇಲೆ ಅವುಗಳ ಅಗತ್ಯವಿಲ್ಲ.

\vskip 6pt

ದೇಹವು ಸ್ಥೂಲರೂಪದಲ್ಲಿರುವ ಮನಸ್ಸಷ್ಟೆ. ಮನಸ್ಸು ಸೂಕ್ಷ್ಮಪದರಗಳಿಂದಾಗಿದೆ; ದೇಹ ಸ್ಥೂಲಪದರಗಳಿಂದಾಗಿದೆ. ಒಬ್ಬನಿಗೆ ಮನಸ್ಸಿನ ಮೇಲಿನ ಸ್ವಾಧೀನತೆ ಪೂರ್ಣ\-ವಾದರೆ ದೇಹದ ಮೇಲಿರುವ ಸ್ವಾಧೀನತೆಯೂ ಪೂರ್ಣವಾಗುವುದು. ಪ್ರತಿಯೊಂದು ಮನಸ್ಸಿಗೂ ತನ್ನದೇ ಆದ ಒಂದು ದೇಹವಿರುವಂತೆ, ಪ್ರತಿಯೊಂದು ಶಬ್ದವೂ ಒಂದು ನಿರ್ದಿಷ್ಟವಾದ ಭಾವನೆಗೆ ಸೇರಿವೆ. ಕೋಪ ಬಂದಾಗ ಮೂರ್ಖ, ದಡ್ಡ ಎಂದು ಗಟ್ಟಿಯಾಗಿ ಕೂಗುವೆವು. ದುಃಖವಾದಾಗ “ಅಯ್ಯೊ” ಎಂದು ಮೃದುವಾಗಿ ಹೇಳುವೆವು. ಇವೆಲ್ಲಾ ತಾತ್ಕಾಲಿಕ ಭಾವನೆಗಳೇನೊ ನಿಜ. ಆದರೆ ನಿತ್ಯವಾದ ಕೆಲವು ಭಾವನೆಗಳು ಇರುತ್ತವೆ. ಅವೇ ಪ್ರೀತಿ ಶಾಂತಿ ಸಂತೋಷ ಪಾವಿತ್ರ ಇತ್ಯಾದಿ. ಎಲ್ಲಾ ಧರ್ಮಗಳಲ್ಲಿಯೂ ಇವನ್ನು ವ್ಯಕ್ತಗೊಳಿಸುವ ಪದಗಳು ಇರುತ್ತವೆ. ಪದಗಳಲ್ಲಿ ಮಾನವನ ಅತಿ ಶ್ರೇಷ್ಠ ಭಾವನೆಗಳೆಲ್ಲ ಅಂತರ್ಗತವಾಗಿವೆ. ಆಲೋಚನೆಯು ಪದವನ್ನು ಸೃಷ್ಟಿಸಿದೆ. ಪದಗಳು ಆಲೋಚನೆಯನ್ನು ಮತ್ತು ಭಾವನೆಯನ್ನು ಪ್ರಚೋದಿಸಬಲ್ಲವು. ಇಲ್ಲೇ ಶಬ್ದಗಳ ಸಹಾಯ ನಮಗೆ ಬರುವುದು.\break ಪ್ರತಿಯೊಂದು ಪದದಲ್ಲಿಯೂ ಒಂದೊಂದು ಆದರ್ಶವಿದೆ. ಈ ಪವಿತ್ರ ಮಂತ್ರಗಳು ನಮಗೆಲ್ಲ ಗೊತ್ತಿವೆ. ಆದರೆ ನಾವು ಅವನ್ನು ಬರಿಯ ಪುಸ್ತಕದಲ್ಲಿ ಓದಿದರೆ ಅದರಿಂದ ಏನೂ ಪ್ರಯೋಜನವಿಲ್ಲ. ಅವು ಪರಿಣಾಮಕಾರಿಯಾಗಬೇಕಾದರೆ ಅವುಗಳಲ್ಲಿ ಚೇತನವಿರಬೇಕು. ಅವುಗಳಲ್ಲಿ ಹಾಗೆ ಚೇತನವಿರಬೇಕಾದರೆ ಭಗವತ್ಸಾಕ್ಷಾತ್ಕಾರವಾದವನು ಆ ಮಂತ್ರಗಳನ್ನು ಬಳಸಿರಬೇಕು. ಅವನು ಈಗ ಜೀವಂತನಾಗಿರಬೇಕು. ಅವನು ಮಾತ್ರ ಮಂತ್ರಗಳಿಗೆ ಜೀವವನ್ನು ತುಂಬಬಲ್ಲ. ಕ್ರಿಸ್ತನು ಜಗತ್ತಿಗೆ ಯಾವ ಶಕ್ತಿಯನ್ನು ಕೊಟ್ಟನೋ ಅದು ಪರಂಪರಾನುಗತವಾಗಿ “ಹಸ್ತವನ್ನು ಇರಿಸುವುದು” ಎಂಬುದರ ಮೂಲಕ ಜಗತ್ತಿನಲ್ಲಿ ನಿರಂತರವಾಗಿ ಹರಿದು ಬರುತ್ತಿದೆ. ಈ ಶಕ್ತಿಯನ್ನು ಮತ್ತೊಬ್ಬನಿಗೆ ದಾನಮಾಡುವ ಸಾಮರ್ಥ್ಯ ಯಾರಿಗೆ ಇದೆಯೋ ಅವನೇ ಗುರು. ಏಸುವಿನಂತಹ ಮಹಾ ಗುರುಗಳಿಗೆ ಮಂತ್ರದ ಅವಶ್ಯಕತೆಯೇ ಇರುವುದಿಲ್ಲ. ಆದರೆ ಸಾಧಾರಣ ಗುರುಗಳು ಈ ಮಂತ್ರಗಳ ಮೂಲಕ ಆಧ್ಯಾತ್ಮಿಕ ಶಕ್ತಿಯನ್ನು ನೀಡಬೇಕಾಗಿದೆ.

\vskip 6pt

ಇತರರ ದೋಷವನ್ನು ಎಂದಿಗೂ ನೋಡಬೇಡಿ. ನೀವು ಜನರನ್ನು ಅವರ ದೋಷದಿಂದ ಅಳೆಯಲಾರಿರಿ. ನಾವು ಒಂದು ಮರವನ್ನು ಪರೀಕ್ಷೆ ಮಾಡಬೇಕಾದರೆ ಅದರ\break ಕೆಳಗೆ ಬಿದ್ದಿರುವ ಕೊಳೆತ ಒಂದು ಹಣ್ಣನ್ನು ತೆಗೆದುಕೊಳ್ಳಕೂಡದು. ಇದರಂತೆಯೇ\break ಮನುಷ್ಯನು ಮಾಡಿದ ತಪ್ಪುಗಳು ಅವನ ಸ್ವಭಾವವನ್ನು ವ್ಯಕ್ತಗೊಳಿಸಲಾರವು. ಜಗತ್ತಿನಾದ್ಯಂತ ದುರ್ಜನರು ಒಂದೇ ಎಂಬುದನ್ನು ಗಮನದಲ್ಲಿಡಿ. ಕಳ್ಳರು, ಕೊಲೆಪಾತಕರು ಏಷ್ಯಾ, ಯೂರೋಪು, ಅಮೆರಿಕಾ ದೇಶಗಳಲ್ಲೆಲ್ಲ ಒಂದೇ. ಅವರದೇ ಒಂದು ಜನಾಂಗ. ಒಳ್ಳೆಯವರಲ್ಲಿ, ಪರಿಶುದ್ಧರಲ್ಲಿ, ಬಲಾಢ್ಯರಲ್ಲಿ ಮಾತ್ರ ನೀವು ವೈವಿಧ್ಯವನ್ನು ನೋಡುವಿರಿ. ಇತರರಲ್ಲಿ ದೌರ್ಜನ್ಯವನ್ನು ಒಪ್ಪಿಕೊಳ್ಳಬೇಡಿ. ದೌರ್ಜನ್ಯವು ದೌರ್ಬಲ್ಯದಿಂದ, ಅಜ್ಞಾನದಿಂದ ಆದುದು. ಜನರಿಗೆ ಅವರು ದುರ್ಬಲರು ಎಂದು ಹೇಳಿ ಪ್ರಯೋಜನವೇನು?\break ಟೀಕೆ ಮಾಡುವುದು, ಧ್ವಂಸಮಾಡುವುದು ಇವುಗಳಿಂದ ಪ್ರಯೋಜನವಿಲ್ಲ. ನಾವು\break ಅವರಿಗೆ ಮತ್ತಾವುದಾದರೂ ಉತ್ತಮವಾಗಿರುವುದನ್ನು ಕೊಡಬೇಕಾಗಿದೆ. ಪವಿತ್ರತೆ ನಿಮ್ಮ ಆಜನ್ಮಸಿದ್ಧ ಹಕ್ಕು ಎಂದು ಹೇಳಬೇಕು. ಹೆಚ್ಚು ಜನರು ಏತಕ್ಕೆ ದೇವರ ಸಮೀಪಕ್ಕೆ\break ಬರುವುದಿಲ್ಲ? ಏಕೆಂದರೆ ಮುಕ್ಕಾಲು ಪಾಲು ಜನರು ಪಂಚೇಂದ್ರಿಯಗಳನ್ನು ಬಿಟ್ಟರೆ\break ತಮಗೆ ಆನಂದವೇ ಇಲ್ಲ ಎಂದು ಭಾವಿಸುವರು. ಮುಕ್ಕಾಲುಪಾಲು ಜನರು ತಮ್ಮ\break ಆಂತರಿಕ ಜಗತ್ತಿನಲ್ಲಿ ಆಗುತ್ತಿರುವುದನ್ನು ತಮ್ಮ ಕಣ್ಣಿನಿಂದ ನೋಡುವುದಿಲ್ಲ, ತಮ್ಮ ಕಿವಿಯಿಂದ ಕೇಳುವುದಿಲ್ಲ.

\vskip 6pt

ನಾವು ಈ ಪ್ರೀತಿಪೂರ್ವಕವಾದ ಪೂಜೆಯ ವಿಚಾರಕ್ಕೆ ಬರೋಣ. “ಚರ್ಚಿನಲ್ಲಿ ಹುಟ್ಟುವುದು ಒಳ್ಳೆಯದು. ಆದರೆ ಅಲ್ಲೇ ಸಾಯುವುದು ಕೆಟ್ಟದ್ದು” ಎಂಬ ಒಂದು ನುಡಿ ಇದೆ. ಮರ ಇನ್ನೂ ಸಸಿಯಾಗಿರುವಾಗ ಸುತ್ತಲಿರುವ ಬೇಲಿಯಿಂದ ರಕ್ಷಣೆ ಪಡೆಯುತ್ತದೆ. ಆದರೆ ಮರ ದೊಡ್ಡದಾದಂತೆಲ್ಲಾ ಬೇಲಿಯನ್ನು ತೆಗೆಯದೆ ಅದು ಚೆನ್ನಾಗಿ ಬೆಳೆಯಲಾರದು. ನಾವು ನೋಡಿದಂತೆ ಬಾಹ್ಯ ಪೂಜೆ ಒಂದು ಅವಶ್ಯಕವಾದ ಸ್ಥಿತಿ ನಿಜ. ಆದರೆ ಕ್ರಮೇಣ ನಾವು ಬೆಳೆದಂತೆಲ್ಲ ಈ ಸ್ಥಿತಿಯಿಂದ ಪಾರಾಗಿ ಮೇಲಿನ ಸ್ಥಿತಿಗೆ ಹೋಗುತ್ತೇವೆ. ಭಗವಂತನ ಮೇಲೆ ಪ್ರೇಮ ನಮ್ಮಲ್ಲಿ ದೃಢವಾದ ಮೇಲೆ ಅವನು ಸರ್ವಶಕ್ತ, ಸರ್ವವ್ಯಾಪಿ, ಸರ್ವಜ್ಞ ಮುಂತಾದುವುಗಳನ್ನು ಗಮನಕ್ಕೆ ತಂದುಕೊಳ್ಳುವುದಿಲ್ಲ. ದೇವರಿಂದ ನಮಗೆ ಬೇಕಾಗಿರುವುದು ಭಗವಂತನ ಪ್ರೀತಿಯೊಂದೇ. ಇನ್ನೂ ದೇವರಿಗೆ ಮಾನವ ರೂಪವನ್ನು ಕಲ್ಪಿಸುವ\break ಸ್ವಭಾವ ನಮ್ಮಲ್ಲಿದೆ. ಅದನ್ನು ನಾವು ಬಿಡಲಾರೆವು. ಆದಕಾರಣ ನಾವು ದೇಹದಿಂದ\break ಹೊರಕ್ಕೆ ಹಾರಲಾರೆವು. ನಾವು ಮಾನವರನ್ನು ಪ್ರೀತಿಸುವಂತೆಯೇ ದೇವರನ್ನೂ ಪ್ರೀತಿಸಬೇಕಾಗಿದೆ.

\vskip 6pt

ಮಾನವ ಪ್ರೀತಿಯಲ್ಲಿ ಐದು ವಿಧಗಳಿವೆ. ಮೊದಲನೆಯದು ಅತ್ಯಂತ ಕೆಳಗಿನದು, ಅತ್ಯಂತ ಸಾಧಾರಣವಾಗಿರುವುದು ಶಾಂತ, ನಮ್ಮ ಹೊಟ್ಟೆಬಟ್ಟೆ ರಕ್ಷಣೆಗಳಿಗಾಗಿ ದೇವರನ್ನು ನೆಚ್ಚಿಕೊಂಡಿರುವುದು. ಎರಡನೆಯದೇ ದಾಸ್ಯ ಇಲ್ಲಿ ನಮಗೆ ಭಗವಂತನ ಸೇವೆಯಲ್ಲಿ ಆಸಕ್ತಿ. ಇಲ್ಲಿ ಭಕ್ತ ಭಗವಂತನನ್ನು ಸ್ವಾಮಿಯಂತೆ ನೋಡುವನು, ಸೇವಾಭಾವನೆಯೊಂದೇ ಅವನಲ್ಲಿ ಎಲ್ಲಕ್ಕಿಂತ ಮೇಲಾಗಿರುವುದು. ಇಲ್ಲಿ ಯಜಮಾನ ಒಳ್ಳೆಯವನೇ, ಕ್ರೂರಿಯೇ, ದಯಾಮಯನೇ ಎಂಬ ಯಾವುದನ್ನೂ ಭಕ್ತನು ಗಣನೆಗೆ ತೆಗೆದುಕೊಳ್ಳುವುದಿಲ್ಲ. ಮೂರನೆಯದೇ ಸಖ್ಯ. ದೇವರನ್ನು ತನ್ನ ಸ್ನೇಹಿತನಂತೆ ನೋಡುವುದು; ಸಮಾನ ಅಂತಸ್ತಿನಲ್ಲಿ ಇರುವವರ ಪ್ರೀತಿ ಇದು. ದೇವರು ತಮ್ಮ ಜೊತೆಗಾರ ಎಂದು ಭಕ್ತ ಇಲ್ಲಿ ಭಾವಿಸುವನು. ನಾಲ್ಕನೆಯದು ವಾತ್ಸಲ್ಯ. ಇಲ್ಲಿ ಭಕ್ತ ದೇವರನ್ನು ಮಗುವೆಂದು ಭಾವಿಸುವನು. ಭಾರತದಲ್ಲಿ ಇದನ್ನು ಹಿಂದೆ ಹೇಳಿದವುಗಳಿಗಿಂತ ಉನ್ನತವಾದುದೆಂದು ಪರಿಗಣಿಸುತ್ತಾರೆ. ಏಕೆಂದರೆ ಇಲ್ಲಿ ಸ್ವಲ್ಪವೂ ಅಂಜಿಕೆಯಿಲ್ಲ. ಐದನೆಯದು ಮಧುರಭಾವ-ಸತಿಪತಿಯರ ಪ್ರೀತಿ. ಪರಿಪೂರ್ಣನಾದ, ಪ್ರಿಯತಮನಾದ ಭಗವಂತನನ್ನು ಪ್ರೀತಿಗಾಗಿ ಪ್ರೀತಿಸುವುದು.

\vskip 6pt

ಇದನ್ನು ಬಹಳ ಸುಂದರವಾಗಿ ಈ ರೀತಿ ವಿವರಿಸುವರು: “ನಾಲ್ಕು ಕಣ್ಣುಗಳು ಸಂಧಿಸಿದುವು. ಎರಡು ಆತ್ಮಗಳಲ್ಲಿ ಒಂದು ಬದಲಾವಣೆ ಪ್ರಾರಂಭವಾಯಿತು. ಈ ಎರಡು ಆತ್ಮಗಳ ಮಧ್ಯೆ ಪ್ರೇಮ ಬಂದು ಇಬ್ಬರನ್ನೂ ಒಂದು ಮಾಡಿತು.”

\vskip 6pt

ಭಕ್ತನಿಗೆ ಪರಿಪೂರ್ಣವಾದ ಕೊನೆಯ ಸ್ವರೂಪದ ಪ್ರೀತಿ ಪ್ರಾಪ್ತವಾದರೆ ಅವನಿಂದ ಎಲ್ಲಾ ಆಸೆಗಳೂ ತೊಲಗುವುವು. ಬಾಹ್ಯಾಚಾರ, ಸಿದ್ಧಾಂತ, ಗುಡಿ ಗೋಪುರಗಳು ಇವುಗಳು ಉದುರಿಹೋಗುತ್ತವೆ. ಮುಕ್ತಿಯ ಆಸೆಯನ್ನು ಕೂಡ ಅವನು ತ್ಯಜಿಸುವನು. ಪರಮಪ್ರೇಮದಲ್ಲಿ ಲೈಂಗಿಕ ಭಾವನೆ ಇರುವುದಿಲ್ಲ ಪರಮ ಪ್ರೀತಿಯಲ್ಲಿ ಪರಮ ಏಕತೆಯನ್ನು ಸಾಧಿಸುವರು. ಎಲ್ಲಿ ಲೈಂಗಿಕ ಭಾವನೆ ಇದೆಯೋ ಅಲ್ಲಿ ದೇಹಭಾವನೆ ಇರುವುದು. ಆದಕಾರಣ ಆತ್ಮನಲ್ಲಿ ಮಾತ್ರ ಐಕ್ಯತೆ ಸಾಧ್ಯ. ದೇಹಭಾವನೆ ಎಷ್ಟು ಕಡಮೆ ಇದ್ದರೆ ಅಷ್ಟೂ ನಮ್ಮ ಪ್ರೇಮ ಪೂರ್ಣವಾಗುವುದು. ಕೊನೆಗೆ ದೇಹದ ಭಾವನೆಯೆಲ್ಲ ಸಂಪುರ್ಣ ಮಾಯವಾಗಿ ಎರಡು ಆತ್ಮಗಳು ಐಕ್ಯವಾಗುವುವು. ನಾವು ಯಾವಾಗಲೂ ಪ್ರೀತಿಸುತ್ತಲೇ ಇರುವೆವು. ಪ್ರೇಮವು ಬಾಹ್ಯ ಆಕಾರವನ್ನು ತೂರಿಹೋಗಿ ಅತೀತವಾಗಿರುವುದನ್ನು ನೋಡುವುದು. “ಎಂತಹ ಕುರೂಪಿಯಲ್ಲಿಯೂ ಪ್ರಿಯತಮನು ರತಿಯನ್ನೇ ಕಾಣುವನು” ಎಂಬ ನಾಣ್ನುಡಿ ಇದೆ. ಹೊರಗೆ ಕಾಣಿಸುವುದು ಬರಿಯ ಒಂದು ಸೂಚನೆ. ಅದರ ಮೇಲೆ ಇವನು ತನ್ನ ಪ್ರೇಮವನ್ನೆಲ್ಲ ಆರೋಪಿಸುವನು. ಮುತ್ತಿನಹುಳು ತನಗೆ ಒತ್ತುತ್ತಿರುವ ಮರಳು\break ಕಣವನ್ನು ಮುಚ್ಚುವುದಕ್ಕಾಗಿ ತನ್ನ ದೇಹದಿಂದ ದ್ರವವನ್ನು ಸ್ರವಿಸಿ ಆ ಮರಳ ಕಣವನ್ನು ಮುಚ್ಚುವುದು. ಅದೇ ಅನಂತರ ಒಂದು ಮುತ್ತಾಗುವುದು. ಹಾಗೆಯೇ ಮನುಷ್ಯ ಹೊರಗೆ ಪ್ರೇಮವನ್ನು ಚೆಲ್ಲಿ, ತನ್ನ ಆದರ್ಶವನ್ನೇ ಇತರರ ಮೇಲೆ ಆರೋಪ ಮಾಡಿ, ಅಲ್ಲಿ ತನ್ನ ಪ್ರೇಮದ ಪರಮಾದರ್ಶವನ್ನು ನೋಡುವನು. ಪ್ರೀತಿಯ ಪರಮಾದರ್ಶವೇ ನಿಃಸ್ವಾರ್ಥ, ಭಗವಂತನೇ ಪ್ರೇಮ. ನಾವು ದೇವರನ್ನು ಪ್ರೀತಿಸುತ್ತೇವೆ ಅಥವಾ ಪ್ರೀತಿಯನ್ನು ಪ್ರೀತಿಸುತ್ತೇವೆ ಅಷ್ಟೆ. ಪ್ರೀತಿಯನ್ನು ವಿವರಿಸುವುದಕ್ಕೆ ಆಗುವುದಿಲ್ಲ. “ಮೂಕನು ಬೆಣ್ಣೆಯ ರುಚಿ ನೋಡಿದಂತೆ.” ಅವನು ಬೆಣ್ಣೆ ಹೇಗಿದೆ ಎಂದು ವಿವರಿಸಲಾರ. ಬೆಣ್ಣೆ ಬೆಣ್ಣೆಯೇ. ಅದರ ರುಚಿ ಗೊತ್ತಿಲ್ಲದವರಿಗೆ ಅದನ್ನು ವಿವರಿಸುವುದಕ್ಕೆ ಆಗುವುದಿಲ್ಲ. ಪ್ರೀತಿಗೋಸುಗ ಪ್ರೀತಿ ಎಂಬುದನ್ನು ಯಾರು ಅನುಭವಿಸಿಲ್ಲವೋ ಅವರಿಗೆ ಯಾರೂ ಇದನ್ನು ವಿವರಿಸುವುದಕ್ಕೆ ಆಗುವುದಿಲ್ಲ.

\vskip 0.2cm

ಪ್ರೀತಿಯನ್ನು ಒಂದು ತ್ರಿಭುಜಕ್ಕೆ ಹೋಲಿಸಬಹುದು. ಮೊದಲನೆಯ ಭುಜವೇ ಪ್ರೀತಿ ಎಂದಿಗೂ ಬೇಡುವುದಿಲ್ಲ. ಏನನ್ನೂ ಕೇಳುವುದಿಲ್ಲ ಎಂಬುದು. ಎರಡನೆಯದೇ ಆ ಪ್ರೀತಿಯಲ್ಲಿ ಅಂಜಿಕೆ ಇಲ್ಲ ಎಂಬುದು. ಮೂರನೆಯದೇ ಪ್ರೀತಿಗೋಸ್ಕರ ಪ್ರೀತಿ ಎಂಬುದು. ಪ್ರೀತಿಯ ಶಕ್ತಿಯಿಂದ ಇಂದ್ರಿಯಗಳು ಸೂಕ್ಷ್ಮವಾಗುತ್ತ ಬರುವುವು, ಉನ್ನತವಾಗುತ್ತಾ ಬರುವುವು. ಮಾನವ ಸಂಬಂಧದಲ್ಲಿ ಪರಿಪೂರ್ಣವಾದ ಪ್ರೀತಿ ಬಹಳ ಅಪರೂಪ. ಏಕೆಂದರೆ ಮಾನವನ ಪ್ರೀತಿ ಹೆಚ್ಚಾಗಿ ಅನ್ಯೋನ್ಯವಾದ ಪ್ರೀತಿ. ಆದರೆ ಭಗವಂತನ ಪ್ರೀತಿ ಅಖಂಡವಾಗಿ ಹರಿಯುವ ಪ್ರವಾಹದಂತೆ, ಯಾವುದೂ ಅದನ್ನು ತಡೆಯಲಾರದು. ಮಾನವ ಭಗವಂತನನ್ನು ತನ್ನ ಶ್ರೇಷ್ಠ ಆದರ್ಶವೆಂದು ಭಾವಿಸಿದಾಗ, ಅವನಿಂದ ಏನನ್ನೂ ಬೇಡದೆ ಇದ್ದಾಗ, ಏನನ್ನೂ ಇಚ್ಛಿಸದೇ ಇದ್ದಾಗ ಅವನ ಪ್ರೀತಿ ಅತ್ಯಂತ ಪವಿತ್ರ ಸ್ಥಾನಕ್ಕೆ ಏರುವುದು. ಆಗ ಪ್ರಪಂಚದಲ್ಲಿ ಅದು ಒಂದು ಮಹಾ ಶಕ್ತಿಯಾಗುತ್ತದೆ. ಅದನ್ನು ಪಡೆಯಬೇಕಾದರೆ ಬಹಳ ಕಾಲ ಹಿಡಿಯುವುದು. ನಮ್ಮ ಸ್ವಭಾವಕ್ಕೆ ಯಾವುದು ಅತ್ಯಂತ ಸಮೀಪದಲ್ಲಿರುವುದೋ ಅದರಿಂದ ನಾವು ಪ್ರಾರಂಭ ಮಾಡಬೇಕು. ಕೆಲವರು ಸೇವೆ ಮಾಡುವುದಕ್ಕಾಗಿಯೇ ಹುಟ್ಟಿರುತ್ತಾರೆ, ಕೆಲವರ ಸ್ವಭಾವ ತಾಯಿಯಂತೆ ಪ್ರೀತಿಸುವುದಾಗಿರುತ್ತದೆ. ಇದರ ಪ್ರತಿಫಲವೆಲ್ಲ ದೇವರಿಗೆ ಸೇರಿರುವುದು. ನಾವು ಮಾನವ ಸ್ವಭಾವದ ಪ್ರಯೋಜನವನ್ನು ಪಡೆಯಬೇಕಾಗಿದೆ.

\vskip 0.2cm

ನಿಮ್ಮ ಧರ್ಮದಿಂದ ನಮ್ಮ ಸಮಾಜಕ್ಕೆ ಏನು ಪ್ರಯೋಜನವೆಂದು ಕೇಳುವರು. ಸತ್ಯವನ್ನು ಪರೀಕ್ಷಿಸುವುದಕ್ಕೆ ಸಮಾಜ ಒಂದು ಒರೆಗಲ್ಲೆಂದು ಅವರು ಭಾವಿಸುವರು. ಇದು ಕುತರ್ಕ, ಸಮಾಜವೆಂಬುದು ನಾವು ಸಾಗಿಹೋಗುತ್ತಿರುವಾಗ ಸಿಕ್ಕುವ ಒಂದು ತಾತ್ಕಾಲಿಕ ಅವಸ್ಥೆ. ನಾವೊಂದು ಮಗುವಿನ ಪ್ರಯೋಜನ ದೃಷ್ಟಿಯಿಂದ ವಿಜ್ಞಾನದ ಯಾವುದಾದರೂ ಅನ್ವೇಷಣೆಯನ್ನು ಪರೀಕ್ಷಿಸಿದಂತೆ ಇದು ಭಯಂಕರವಾದುದು. ಸಮಾಜದ ಸ್ಥಿತಿಯು ಶಾಶ್ವತ ಸ್ವರೂಪದ್ದಾಗಿದ್ದರೆ ಮಗುವು ಯಾವಾಗಲೂ ಮಗುವಿನಂತೆಯೇ ಇರುವಂತಾಗುತ್ತಿತ್ತು. ಎಂದಿಗೂ ಪೂರ್ಣವಾದ ಮಾನವ ಮಗು ಇರುವುದಿಲ್ಲ. ಅದು ವಿರೋಧಾಭಾಸ.\break ಆದ್ದರಿಂದ ಪರಿಪೂರ್ಣವಾದ ಸಮಾಜ ಎಂಬುದು ಇರುವುದಿಲ್ಲ. ಮಾನವನು ತನ್ನ ಪೂರ್ವಸ್ಥಿತಿಯನ್ನು ಮೀರಿ ಬೆಳೆದೇ ಬೆಳೆಯುತ್ತಾನೆ. ಸಮಾಜ ಒಂದು ಸ್ಥಿತಿಯಲ್ಲಿ ಸರಿ. ಆದರೆ ಅದೇ ನಮ್ಮ ಆದರ್ಶವಾಗಲಾರದು. ಅದು ಯಾವಾಗಲೂ ಬದಲಾಗುತ್ತಿರುವುದು. ಈಗಿರುವ ಈ ವೈಶ್ಯ ಬುದ್ಧಿಯ ಸಂಸ್ಕೃತಿ, ಅದರ ನಟನೆ ತೋರಿಕೆ ಮುಂತಾದುವುಗಳೊಂದಿಗೆ ನಾಶವಾಗಲೇ ಬೇಕಾಗಿದೆ. ಇದೆಲ್ಲ ಒಂದು ನಾಟಕ. ಜಗತ್ತಿಗೆ ಬೇಕಾಗಿರುವುದು ವ್ಯಕ್ತಿಗಳ ಮೂಲಕ ಅದ್ಭುತ ಆಲೋಚನಾ ಶಕ್ತಿ. ನನ್ನ ಗುರುಗಳು ಹೀಗೆ ಹೇಳುತ್ತಿದ್ದರು: “ನೀನು ನಿನ್ನ ಹೃದಯ ಕಮಲವು ಅರಳುವಂತೆ ಏತಕ್ಕೆ ಮಾಡಕೂಡದು? ಆಗ ದುಂಬಿಗಳು ತಮಗೆ ತಾವೇ ಅದರೆಡೆಗೆ ಬರುವುವು” ಭಗವಂತನ ಪ್ರೇಮದಲ್ಲಿ ಹುಚ್ಚರಾಗಿ ತನ್ಮಯರಾಗಿ ಹೋಗಿರುವವರು ಜಗತ್ತಿಗೆ ಬೇಕಾಗಿದ್ದಾರೆ. ಮೊದಲು ನಿಮ್ಮಲ್ಲಿ ನಿಮಗೆ ಭರವಸೆಯಿದ್ದರೆ ಅನಂತರ ದೇವರ ಮೇಲೆ ಭರವಸೆ ಹುಟ್ಟುವುದು. (ಜಗತ್ತಿನ ಇತಿಹಾಸವು ಇಂತಹ ಆರು ಜನ ಶ್ರದ್ಧಾಳುಗಳ ಮೇಲೆ ನಿಂತಿದೆ; ಶುದ್ಧ ಚಾರಿತ್ರ್ಯವುಳ್ಳವರ ಮೇಲೆ ನಿಂತಿದೆ. ನಮಗೆ ಮೂರು ವಸ್ತುಗಳು ಬೇಕಾಗಿವೆ: ಅನುಭವಿಸುವ ಹೃದಯ, ಯೋಚಿಸುವ ಮಿದುಳು, ಕೆಲಸ ಮಾಡುವುದಕ್ಕೆ ಕೈಗಳು. ಮೊದಲು ನಾವು ಜಗತ್ತಿನಿಂದ ಹೊರಗೆ ಹೋಗಿ ಭಗವಂತನ ಯೋಗ್ಯ ಉಪಕರಣಗಳಾಗಬೇಕು. ಒಂದು ಅದ್ಭುತಶಕ್ತಿಯ ಕೇಂದ್ರವಾಗಿರಿ. ಮೊದಲು ಪ್ರಪಂಚದ ಮೇಲೆ ಅನುಕಂಪವಿರಲಿ. ಎಲ್ಲರೂ ಕೆಲಸಕ್ಕೆ ಸಿದ್ಧರಾಗಿರುವಾಗ ಅನುಕಂಪವನ್ನು ತೋರಬಲ್ಲವರು ಯಾರು ಇರುವರು? ಒಬ್ಬ ಇಗ್ನೇಷಿಯಸ್​ ಲಯೋಲನನ್ನು ಸೃಷ್ಟಿ ಮಾಡಿದ ಭಾವತೀವ್ರತೆ ಎಲ್ಲಿ? ನಿಮ್ಮ ಪ್ರೀತಿಯನ್ನು ಮತ್ತು ದೈನ್ಯವನ್ನು ಪರೀಕ್ಷೆ ಮಾಡಿ. ಯಾರಲ್ಲಿ ಅಸೂಯೆ ಇದೆಯೋ ಅವನಲ್ಲಿ ದೈನ್ಯವಿಲ್ಲ, ಪ್ರೀತಿಯಿಲ್ಲ. ಅಸೂಯೆ ಭಯಾನಕವಾದುದು, ಮಹಾಪಾತಕ, ಇದು ನಮಗೆ ತಿಳಿಯದೆ ಹೇಗೋ ನಮ್ಮೊಳಗೆ ಪ್ರವೇಶಿಸುವುದು. ನಿಮ್ಮ ಮನಸ್ಸಿನಲ್ಲಿ ದ್ವೇಷ ಮತ್ತು ಅಸೂಯಾ ಭಾವನೆಗಳು ಏಳುವುದೇ ಎಂದು ಪ್ರಶ್ನಿಸಿ. ಹಲವು ಒಳ್ಳೆಯ ಕೆಲಸಗಳು ಜಗತ್ತಿನ ಮೇಲೆ ಜನರು ಸುರಿಸುತ್ತಿರುವ ಟನ್ನುಗಟ್ಟಲೆ\break ದ್ವೇಷ-ಕೋಪ ಇವುಗಳಿಂದ ನಾಶವಾಗುತ್ತಿವೆ. ನೀವು ಪರಿಶುದ್ಧರಾಗಿದ್ದರೆ, ಧೀರರಾಗಿದ್ದರೆ, ನೀವೊಬ್ಬರೇ ಜಗತ್ತಿಗೆ ಸಮ.)

ಜಗತ್ತಿನಲ್ಲಿ ನಾವು ಒಳ್ಳೆಯ ಕೆಲಸವನ್ನು ಮಾಡಬೇಕಾದರೆ ನಮಗೆ ಬೇಕಾಗಿರುವುದೇ ಭಾವನೆಯಿಂದ ಕೂಡಿದ ಬುದ್ಧಿಶಕ್ತಿ. ಅದರ ಹಿಂದೆ ಭಾವವಿಲ್ಲದೇ ಇದ್ದರೆ ಅದೇನನ್ನೂ ಸಾಧಿಸಲಾರದು. ಎಂದಿಗೂ ವ್ಯರ್ಥವಾಗದ ಪ್ರೀತಿಯನ್ನು ತೆಗೆದುಕೊಳ್ಳಿ. ಆಗ ಮಿದುಳು ಆಲೋಚಿಸುವುದು; ಯಾವುದು ಸರಿಯೋ ಆ ಕೆಲಸವನ್ನು ಕೈಗಳು ಮಾಡುವುವು. ಋಷಿಗಳು ದೇವರನ್ನೇ ಸದಾ ಚಿಂತಿಸುತ್ತಿದ್ದರು. ಅವರು ಅವನ ದರ್ಶನವನ್ನು ಪಡೆದರು. “ಯಾರು ಪರಿಶುದ್ಧಾತ್ಮರೋ ಅವರು ದೇವರನ್ನು ಕಾಣುವರು.” ಎಲ್ಲಾ ಮಹಾಪುರುಷರೂ ತಾವು ದೇವರನ್ನು ಕಂಡೆವೆಂದು ಸಾರುವರು. ಸಾವಿರಾರು ವರುಷಗಳ ಹಿಂದೆ ಅವರು ಈ ದೃಶ್ಯವನ್ನು ಕಂಡರು. ಮನಸ್ಸಿಗೆ ಅತೀತವಾದ ಏಕತೆಯನ್ನು ಅವರು ಮನಗಂಡರು. ನಾವು ಈಗ ಮಾಡಬೇಕಾಗಿರುವುದೇ ಪುನಃ ಅವರ ಆದರ್ಶವನ್ನು ಅನುಸರಿಸುವುದು.

