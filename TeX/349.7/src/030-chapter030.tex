
\vspace{-0.6cm}

\chapter[ವೇದ-ಉಪನಿಷತುಗಳನ್ನು ಕುರಿತು ಕೆಲವು ಭಾವನೆಗಳು ]{ವೇದ-ಉಪನಿಷತುಗಳನ್ನು ಕುರಿತು ಕೆಲವು ಭಾವನೆಗಳು \protect\footnote{\engfoot{C.W. Vol. VI, P. 86}}}

ವೇದಗಳ ಯಜ್ಞವೇದಿಕೆಯೆ ರೇಖಾಗಣಿತಕ್ಕೆ \enginline{(Geometry)} ಮೂಲ.

ದೇವತೆಗಳನ್ನು ಆಹ್ವಾನಿಸುತ್ತಿದ್ದುದೇ ಪೂಜೆಗೆ ಆದಿ. ಅಂದರೆ ಬರಮಾಡಿಕೊಂಡ ದೇವರಿಗೆ ನಾವು ಹವಿಸ್ಸನ್ನು ಕೊಡುವೆವು; ಅವರು ನಮಗೆ ಸಹಾಯವನ್ನು ಮಾಡುವರು.

ಮಂತ್ರಗಳು ಕೇವಲ ಸ್ತೋತ್ರಮಾಲೆಯಲ್ಲ. ಅವನ್ನು ಸರಿಯಾದ ಮನೋಭಾವದಿಂದ ಉಚ್ಚರಿಸಿದರೆ ಅದ್ಭುತ ಶಕ್ತಿ ಅವುಗಳಲ್ಲಿ ಅಂತರ್ಗತವಾಗಿರುವುದು ಗೊತ್ತಾಗುವುದು.

ಸ್ವರ್ಗವೂ ಕೂಡ ಈ ಲೋಕದಂತೆಯೇ; ಅಲ್ಲಿ ಇನ್ನೂ ಹೆಚ್ಚು ಇಂದ್ರಿಯಗಳು ಇದ್ದು ಅವು ಇನ್ನೂ ಹೆಚ್ಚು ತೀವ್ರವಾಗಿ ಇರುತ್ತವೆ.

ಈ ನಮ್ಮ ದೇಹ ನಾಶವಾಗುವಂತೆಯೇ ಎಂತಹ ಉನ್ನತವಾದ ದೇಹವಾದರೂ ಅದು ನಾಶವಾಗಲೇಬೇಕು. ಎಲ್ಲ ದೇಹಗಳಿಗೂ, ಈ ಜನ್ಮದಲ್ಲೂ, ಮುಂದಿನ ಜನ್ಮದಲ್ಲೂ ಮೃತ್ಯು ಬಂದೇ ಬರುವುದು. ದೇವತೆಗಳು ಕೂಡ ಮರ್ತ್ಯರೇ. ಅವರು ಭೋಗವನ್ನು ಮಾತ್ರ ಕೊಡಬಲ್ಲರು.

ಈ ದೇಹದ ಹಿಂದೆ ನೋಡುವ ಮತ್ತು ಅರಿಯುವ ಒಬ್ಬನಿರುವಂತೆ ಎಲ್ಲ ದೇವತೆಗಳ ಹಿಂದೆ ದೇವರೆಂಬ ಒಬ್ಬ ಸರ್ವೇಶ್ವರನಿರುವನು.

ಸೃಷ್ಟಿ-ಸ್ಥಿತಿ-ಲಯಗಳನ್ನು ಉಂಟುಮಾಡುವ ಶಕ್ತಿ, ಸರ್ವವ್ಯಾಪಕತ್ವ, ಸರ್ವಜ್ಞತ್ವ ಮತ್ತು ಸರ್ವಶಕ್ತಿತ್ವ ಇವುಗಳೆಲ್ಲ ಸರ್ವೇಶ್ವರನಿಗೆ ಸೇರಿದುವು.

\vskip 5pt

“ಅಮೃತ ಪುತ್ರರೇ ಕೇಳಿ! ಮೇಲಿನ ಲೋಕದಲ್ಲಿರುವ ದೇವತೆಗಳೆ, ನೀವು ಕೂಡ ಕೇಳಿ! ಸಂಶಯದ ಆಚೆ, ಅಜ್ಞಾನದ ಆಚೆ ನಾನೊಂದು ಜ್ಯೋತಿಯನ್ನು ಕಂಡಿರುವೆನು, ಸನಾತನ ಪುರುಷನನ್ನು ಕಂಡಿರುವೆನು.” ಉಪನಿಷತ್ತಿನಲ್ಲಿ ಇದಕ್ಕೆ ಮಾರ್ಗವಿದೆ.

\vskip 5pt

ಭೂಲೋಕದಲ್ಲಿ ನಾವು ಸಾಯುವೆವು. ಸ್ವರ್ಗಲೋಕದಲ್ಲಿ ಸಾಯುವೆವು. ಅದಕ್ಕಿಂತ ಮಿಗಿಲಾದ ಲೋಕದಲ್ಲಿಯೂ ನಾವು ಸಾಯುವೆವು. ನಾವು ದೇವರನ್ನು ಪಡೆದಾಗ ಮಾತ್ರ ಅಮೃತರಾಗುವೆವು.

\vskip 5pt

ಉಪನಿಷತ್ತು ಅಮೃತತ್ವಕ್ಕೆ ಒಯ್ಯುವ ಪಥವನ್ನು ಮಾತ್ರ ಹೇಳುವುದು. ಉಪನಿಷತ್ತಿನ ಮಾರ್ಗ ಪರಿಶುದ್ಧವಾದ ಮಾರ್ಗ. ಅಲ್ಲಿ ಹೇಳಿರುವ ಹಲವು ಲೋಕಾಚಾರ ದೇಶಾಚಾರಗಳು ಇಂದು ನಮಗೆ ಅರ್ಥವಾಗುವುದಿಲ್ಲ. ಅವುಗಳ ಮೂಲಕವಾಗಿ ಸಾಗಿಹೋದ ಮೇಲೆ ನಮಗೆ ಸತ್ಯ ಸ್ಪಷ್ಟವಾಗಿ ಕಾಣಿಸುವುದು. ಸತ್ಯವನ್ನು ಪಡೆಯುವುದಕ್ಕಾಗಿ ಭೂಲೋಕ ಸ್ವರ್ಗಲೋಕಗಳನ್ನೆಲ್ಲ ಪರಿತ್ಯಜಿಸಿರುವರು.

\vskip 5pt

ಉಪನಿಷತ್ತು ಹೀಗೆ ಹೇಳುವುದು:

“ಆ ಭಗವಂತನು ಪ್ರಪಂಚದಲ್ಲಿ ಹಾಸುಹೊಕ್ಕಾಗಿರುವನು. ಇದೆಲ್ಲ ಅವನದೇ.”

\vskip 6pt

“ಸರ್ವವ್ಯಾಪಿಯಾದ, ಅದ್ವಿತೀಯನಾದ, ನಿರ್ದೇಹಿಯಾದ, ಪರಿಶುದ್ಧನಾದ ವಿಶ್ವದ ಸನಾತನ ಕವಿಯೇ ಅವನು. ಸೂರ್ಯ ನಕ್ಷತ್ರಗಳೇ ಅವನ ಛಂದಸ್ಸು. ಅವನು ಪ್ರತಿಯೊಬ್ಬರನ್ನೂ ಅವರವರ ಕರ್ಮಾನುಸಾರ ಆಳುತ್ತಿರುವನು.”

\vskip 6pt

“ಕರ್ಮದ ಮೂಲಕ ಜ್ಞಾನವನ್ನು ಪಡೆಯಲೆತ್ನಿಸುವವರು ಗಾಢಾಂಧಕಾರದಲ್ಲಿ ತೊಳಲುತ್ತಿರುವರು. ಯಾರು ಈ ಪ್ರಕೃತಿಯೇ ಸರ್ವಸ್ವ ಎಂದು ಅರಿತಿರುವರೋ ಅವರೂ ಅಂಧಕಾರದಲ್ಲಿರುವರು. ಪ್ರಕೃತಿಯಿಂದ ಪಾರಾಗಲು ಯತ್ನಿಸುವವರು ಮತ್ತೂ ಗಾಢವಾದ ಅಂಧಕಾರದಲ್ಲಿರುವರು.”

\vskip 6pt

ಹಾಗಾದರೆ ಕರ್ಮ ಕೆಟ್ಟದ್ದೆ? ಇಲ್ಲ. ಹಿಂದೆ ಬರುತ್ತಿರುವವರಿಗೆ ಇದು ಸಹಾಯ ಮಾಡುವುದು.

\vskip 6pt

ಒಂದು ಉಪನಿಷತ್ತಿನಲ್ಲಿ ನಚಿಕೇತನೆಂಬ ಯುವಕ ಈ ಪ್ರಶ್ನೆಯನ್ನು ಹಾಕುತ್ತಾನೆ:\break “ಕೆಲವರು ಮೃತನು ಎಂದೆಂದಿಗೂ ಹೋದ ಎನ್ನುವರು. ಮತ್ತೆ ಕೆಲವರು ಅವನಿನ್ನೂ ಜೀವಿಸುತ್ತಿರುವನು ಎನ್ನುವರು. ನೀನು ಮೃತ್ಯುವಾದ ಯಮ, ನಿನಗೆ ಸತ್ಯ ಗೊತ್ತಿದೆ. ಯಾವುದು ನಿಜ ಎಂಬುದನ್ನು ಹೇಳು.” ಆಗ ಯಮ, “ಅನೇಕ ಮಂದಿ ದೇವತೆಗಳಿಗೇ ಇದು ಗೊತ್ತಿಲ್ಲ. ಮನುಷ್ಯರಿಗೆ ಮೊದಲೇ ಗೊತ್ತಿಲ್ಲ. ಈ ಪ್ರಶ್ನೆಯನ್ನು ಕೇಳಬೇಡ” ಎನ್ನುವರು. ಆದರೆ ನಚಿಕೇತ ಬಿಡುವುದಿಲ್ಲ. ಯಮ ಪುನಃ, “ದೇವತೆಗಳ ಭೋಗವನ್ನೆಲ್ಲ ನಿನಗೆ ಕೊಡುತ್ತೇನೆ. ಈ ಪ್ರಶ್ನೆಯನ್ನು ಮಾತ್ರ ಕೇಳಬೇಡ” ಎನ್ನುತ್ತಾನೆ. ಆದರೆ ನಚಿಕೇತ ಬಿಡುವುದೇ ಇಲ್ಲ. ವಜ್ರಮನಸ್ಕನಾಗಿ ನಿಲ್ಲುವನು. ಆಗ ಯಮ, “ಮಗು, ನೀನು ಮೂರನೆಯ ವೇಳೆಯೂ ಐಶ್ವರ್ಯ, ಅಧಿಕಾರ, ದೀರ್ಘಾಯುಸ್ಸು, ಕೀರ್ತಿ, ಸಂಸಾರ ಮುಂತಾದುವನ್ನು ತ್ಯಜಿಸಿರುವೆ. ನೀನು ಪರಮ ಸತ್ಯವನ್ನು ಕೇಳುವಷ್ಟು ಧೈರ್ಯವುಳ್ಳವನಾಗಿರುವೆ. ಎರಡು ಪಥಗಳಿವೆ. ಒಂದು ಪ್ರೇಯಸ್ಸಿನ ಮಾರ್ಗ, ಮತ್ತೊಂದು ಪೇಯಸ್ಸಿನ ಮಾರ್ಗ. ನೀನು ಶ್ರೇಯಸ್ಸಿನ ಮಾರ್ಗವನ್ನು ಆರಿಸಿಕೊಂಡಿರುವೆ” ಎನ್ನುವನು.

\vskip 6pt

ಸತ್ಯವನ್ನು ಬೋಧಿಸಬೇಕಾದರೆ ಯಾವ ನಿಯಮ ಇದೆ ಎಂಬುದನ್ನು ಗಮನಿಸಿ. ಮೊದಲನೆಯದೇ ಪಾವಿತ್ರ್ಯ ಚಿತ್ತವಿಕಾರವಿಲ್ಲದ ಪರಿಶುದ್ಧನಾದ ಯುವಕ ವಿಶ್ವದ ರಹಸ್ಯವನ್ನು ಕೇಳುತ್ತಿರುವನು. ಎರಡನೆಯದೇ ಸತ್ಯಕ್ಕಾಗಿ ಸತ್ಯವನ್ನು ಒಬ್ಬ ಅರಸಬೇಕು.

\vskip 6pt

ಸಾಕ್ಷಾತ್ಕಾರವನ್ನು ಪಡೆದವನ ಮೂಲಕ, ಅದನ್ನು ಪ್ರತ್ಯಕ್ಷ ಅನುಭವಿಸಿದವನ ಮೂಲಕ, ಸತ್ಯವು ಬಂದರೆ ಮಾತ್ರ, ಅದು ಫಲಕಾರಿಯಾಗುವುದು. ಶಾಸ್ತ್ರಗಳು ಇದನ್ನು ಕೊಡಲಾರವು, ವಾದ ಇದನ್ನು ಸ್ಥಿರಪಡಿಸಲಾರದು. ಇದರ ರಹಸ್ಯವನ್ನು ಅರಿತವನಿಗೆ ಸತ್ಯ\break ಪ್ರಾಪ್ತಿಯಾಗುವುದು.

\vskip 6pt

ಇದನ್ನು ಪಡೆದ ಮೇಲೆ ಮೌನವಾಗಿರು. ಸುಮ್ಮನೆ ವಾದದಿಂದ ವಿಚಲಿತನಾಗಬೇಡ. ನೀನು ಸ್ವಂತ ಅನುಭವವನ್ನು ಪಡೆಯಬೇಕಾಗಿದೆ. ನೀನು ಮಾತ್ರ ಇದನ್ನು ಮಾಡಬಲ್ಲೆ.

\vskip 6pt

ಅದು ಸುಖವೂ ಅಲ್ಲ, ದುಃಖವೂ ಅಲ್ಲ, ಪಾಪವೂ ಅಲ್ಲ, ಪುಣ್ಯವೂ ಅಲ್ಲ, ಜ್ಞಾನವೂ ಅಲ್ಲ, ಅಜ್ಞಾನವೂ ಅಲ್ಲ. ನೀವು ಇದನ್ನು ಸಾಕ್ಷಾತ್ಕಾರ ಮಾಡಿಕೊಳ್ಳಬೇಕಾಗಿದೆ. ನಾನು ಇದನ್ನು ನಿಮಗೆ ಹೇಗೆ ವಿವರಿಸಬಲ್ಲೆ?

\vskip 6pt

ಯಾರು ಹೃತ್ಪೂರ್ವಕವಾಗಿ “ದೇವರೆ, ನನಗೆ ನೀನೊಬ್ಬನೇ ಬೇಕು” ಎಂದು ಪ್ರಾರ್ಥಿಸುವರೋ ಅವರಿಗೆ ದೇವರು ದೊರಕುವನು. ಪರಿಶುದ್ಧನಾಗಿರು, ಶಾಂತನಾಗಿರು. ಮನಸ್ಸು ಚಂಚಲವಾದಾಗ ಭಗವಂತನನ್ನು ಅದು ಪ್ರತಿಬಿಂಬಿಸಲಾರದು.”

\vskip 6pt

“ಯಾವುದನ್ನು ವೇದಗಳು ಸಾರುವುವೋ, ನಾವು ಯಾವುದನ್ನು ಪಡೆಯಲು ಯಾಗಯಜ್ಞಗಳನ್ನು ಮಾಡುವೆವೋ ಆ ಅವ್ಯಕ್ತವಾದುದೇ ಓಂ,” ಇದೇ ಪವಿತ್ರತಮ ಮಂತ್ರ. ಯಾರು ಈ ಮಂತ್ರದ ರಹಸ್ಯವನ್ನು ಅರಿತಿರುವರೋ ಅವರಿಗೆ ಅವರು ಇಚ್ಛಿಸಿದುದೆಲ್ಲ ಪ್ರಾಪ್ತವಾಗುವುದು. ಈ ಮಂತ್ರಕ್ಕೆ ಶರಣಾಗಿ. ಯಾರು ಇದಕ್ಕೆ ಶರಣಾಗುವರೋ ಅವರಿಗೆ ಮುಂದಿನ ಪಥ ಗೋಚರಿಸುವುದು.

