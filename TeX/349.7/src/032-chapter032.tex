
\vspace{-0.6cm}

\chapter[ಆತ್ಮ ಸ್ವಾತಂತ್ರ್ಯ ]{ಆತ್ಮ ಸ್ವಾತಂತ್ರ್ಯ \protect\footnote{\engfoot{C.W. Vol. VI, P. 84}}}

ನಾವು ನಮಗೆ ಕಣ್ಣಿದೆ ಎಂಬುದನ್ನು ಪರಿಣಾಮದ ಮೂಲಕ ನೋಡಿದಲ್ಲದೆ ಹೇಗೆ\break ಅರಿಯಲಾರವೊ ಹಾಗೆಯೇ ಪರಿಣಾಮದ ಮೂಲಕವಾಗಿ ಅಲ್ಲದೆ ಆತ್ಮನನ್ನು ಅರಿಯಲಾಗುವುದಿಲ್ಲ. ಅದನ್ನು ಇಂದ್ರಿಯಗಳು ಅರಿಯುವ ಸ್ಥಿತಿಗೆ ತರಲಾಗುವುದಿಲ್ಲ. ನಾವು ಆತ್ಮ ಎಂದು ಅರಿತಾಗ ಮುಕ್ತರಾಗುವೆವು. ಆತ್ಮ ಎಂದಿಗೂ ಬದಲಾಗುವುದಿಲ್ಲ. ಅದು ಮತ್ತಾವ ಕಾರಣಕ್ಕೂ ಒಳಗಾಗುವುದಿಲ್ಲ. ಏಕೆಂದರೆ ಅದೇ ಕಾರಣವಾಗಿರುವುದು. ಅದು ಸ್ವಯಂಭೂ. ಮತ್ತಾವ ಕಾರಣಕ್ಕೂ ಒಳಗಾಗದು. ಯಾವುದಾದರೂ ಒಂದನ್ನು ನಿಮ್ಮಲ್ಲಿ ನೀವು ಕಂಡು\break ಹಿಡಿದರೆ ಆಗ ನೀವು ಆತ್ಮನನ್ನು ಅರಿತಂತೆ.

ಸ್ವಾತಂತ್ರ್ಯ ಮತ್ತು ಮುಕ್ತಿ ಇವುಗಳಲ್ಲಿ ಒಂದನ್ನು ಬಿಟ್ಟು ಮತ್ತೊಂದು ಇರಲಾರದು. ಮುಕ್ತರಾಗಬೇಕಾದರೆ ಪ್ರಕೃತಿಯ ನಿಯಮಗಳಿಗೆ ಅತೀತರಾಗಬೇಕು. ನಾವು ಅಜ್ಞಾನಿಗಳಾಗಿರುವವರೆಗೆ ನಿಯಮವಿರುವುದು. ಜ್ಞಾನ ಬಂದೊಡನೆಯೆ ನಿಯಮವೆಂಬುವುದಿಲ್ಲ. ನಾವು ಸ್ವತಂತ್ರರು ಎಂದು ಗೊತ್ತಾಗುವುದು. ಇಚ್ಛೆ ಎಂದಿಗೂ ಸ್ವತಂತ್ರವಾಗಲಾರದು. ಇದು ಕಾರ್ಯಕಾರಣಗಳ ನಿಯಮಕ್ಕೆ ಒಳಪಟ್ಟಿದೆ. ಇಚ್ಛೆಯ ಹಿಂದೆ ಇರುವ ನಾನು ಸ್ವತಂತ್ರನು. ಅದೇ ಆತ್ಮ. ನಾನು ಮುಕ್ತ-ಇದರ ಆಧಾರದ ಮೇಲೆಯೇ ನಾವು ಬಾಳಬೇಕಾಗಿದೆ. ಮುಕ್ತಿ ಎಂದರೆ ಅಮೃತತ್ವ.

