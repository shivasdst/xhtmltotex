
\chapter[ಸಾಕ್ಷಾತ್ಕಾರ: ಮಾರ್ಗ ಮತ್ತು ಗುರಿ ]{ಸಾಕ್ಷಾತ್ಕಾರ: ಮಾರ್ಗ ಮತ್ತು ಗುರಿ \protect\footnote{\engfoot{C.W. Vol. V, P 291}}}

ಪ್ರಪಂಚದವರೆಲ್ಲ ಒಂದೇ ಧರ್ಮವನ್ನು, ಒಂದೇ ಬಗೆಯ ಪೂಜೆಯನ್ನು, ಒಂದೇ ಬಗೆಯ ನೀತಿಯನ್ನು ಒಪ್ಪಿಕೊಂಡರೆ ಅದೇ ಪ್ರಪಂಚಕ್ಕೆ ಒದಗಬಹುದಾದ ಮಹಾ ದುರ್ಗತಿ. ಇದು ಎಲ್ಲಾ ಧಾರ್ಮಿಕ ಮತ್ತು ಆಧ್ಯಾತ್ಮಿಕ ಪ್ರಗತಿಗೆ ಕುಠಾರ ಪ್ರಾಯವಾಗುವುದು. ಯಾವುದನ್ನು ಶ್ರೇಷ್ಠಧರ್ಮವೆಂದು ನಾವು ಭಾವಿಸುವೆವೋ ಅದನ್ನು ನ್ಯಾಯವಾದ ಮಾರ್ಗದಿಂದಲೋ ಅನ್ಯಾಯವಾದ ಮಾರ್ಗದಿಂದಲೋ ಎಲ್ಲರೂ ಒಪ್ಪಿಕೊಳ್ಳುವಂತೆ ಮಾಡಿ, ಧರ್ಮವು ಶೀಘ್ರವಾಗಿ ನಾಶವಾಗುವಂತೆ ಮಾಡುವುದಕ್ಕಿಂತ ಪ್ರತಿಯೊಬ್ಬರೂ ತಾವು ಯಾವುದನ್ನು ಶ್ರೇಷ್ಠ ಆದರ್ಶ ಎಂದು ಭಾವಿಸುವರೋ ಅದಕ್ಕೆ ಇರುವ ಆತಂಕಗಳನ್ನು ನಿವಾರಿಸಿ, ಒಂದೇ ಧರ್ಮವು ಜಗತ್ತಿನಲ್ಲಿ ಸ್ಥಾಪಿತವಾಗದಂತೆ ನೋಡಿಕೊಳ್ಳಲು ಪ್ರಯತ್ನಿಸಬೇಕು.

ಭಗವಂತನಲ್ಲಿ ಅಥವಾ ಪ್ರತಿಯೊಬ್ಬ ಮಾನವನ ನೈಜ ಸ್ವಭಾವವಾದ ದಿವ್ಯತೆಯಲ್ಲಿ ಒಂದಾಗುವುದೇ ಮಾನವಕೋಟಿಯ ಏಕಮಾತ್ರ ಗುರಿ ಮತ್ತು ಎಲ್ಲಾ ಧರ್ಮಗಳ ಗಂತವ್ಯ ಗುರಿ ಒಂದೇ ಆದರೂ ಮಾನವರ ಸ್ವಭಾವಕ್ಕೆ ತಕ್ಕಂತೆ ಮಾರ್ಗಗಳು ಬೇರೆಬೇರೆಯಾಗಿರ\-ಬಹುದು.

ಈ ಸ್ಥಿತಿಯ ಗುರಿ ಮತ್ತು ಅದನ್ನು ತಲುಪುವ ಮಾರ್ಗ ಎರಡನ್ನೂ ಯೋಗ ಎಂದು ಕರೆಯುತ್ತಾರೆ. ಇಂಗ್ಲಿಷಿನ ಯೋಕ್​ ಎಂದರೆ ಸೇರಿಸು, ಕೂಡಿಸು ಎಂಬ ಧಾತುವಿಗೆ\break ಸಮಾನವಾದ ಯುಜಿರ್​ ಎಂಬ ಸಂಸ್ಕೃತ ಧಾತುವಿನಿಂದ ಯೋಗ ಎಂಬ ಪದವು ಬಂದಿದೆ.\break ನಮ್ಮನ್ನು ಸತ್ಯದೊಂದಿಗೆ ಕೂಡಿಸುವುದೇ ಯೋಗ. ಹಲವು ಯೋಗಗಳಿವೆ. ಇವುಗಳಲ್ಲಿ ಮುಖ್ಯವಾಗಿರುವುವು ಕರ್ಮಯೋಗ, ಭಕ್ತಿಯೋಗ, ರಾಜಯೋಗ ಮತ್ತು ಜ್ಞಾನಯೋಗ.

ಪ್ರತಿಯೊಬ್ಬರೂ ತಮ್ಮ ಸ್ವಭಾವಕ್ಕೆ ಅನುಗುಣವಾಗಿ ಬೆಳೆಯಬೇಕಾಗಿದೆ. ಪ್ರತಿಯೊಂದು ವಿಜ್ಞಾನಕ್ಕೂ ಒಂದು ಮಾರ್ಗವಿರುವಂತೆ ಪ್ರತಿಯೊಂದು ಧರ್ಮಕ್ಕೂ ಒಂದೊಂದು\break ಮಾರ್ಗವಿದೆ. ಆಧ್ಯಾತ್ಮಿಕ ಜೀವನದಲ್ಲಿ ಗುರಿಯನ್ನು ಮುಟ್ಟುವ ಮಾರ್ಗವನ್ನೇ ಯೋಗ ಎನ್ನುವುದು. ಮಾನವರ ಸ್ವಭಾವಕ್ಕೆ ತಕ್ಕಂತೆ ಹಲವು ಬಗೆಯ ಯೋಗಗಳು ಇವೆ. ಈ\break ಯೋಗಗಳನ್ನು ಪ್ರತ್ಯೇಕವಾಗಿ ನಾಲ್ಕು ಮಾರ್ಗಗಳನ್ನಾಗಿ ಮಾಡುವೆವು:–

\begin{enumerate}
\item ಕರ್ಮಯೋಗ:– ಕರ್ಮದ ಮತ್ತು ಕರ್ತವ್ಯದ ಮೂಲಕ ಒಬ್ಬನು ತನ್ನ ದಿವ್ಯತೆಯನ್ನು ಅರಿಯುವುದು.

 \item ಭಕ್ತಿಯೋಗ:– ಈಶ್ವರನ ಮೇಲೆ ಇಡುವ ಪ್ರೀತಿಯ ಮತ್ತು ಭಕ್ತಿಯ ಮೂಲಕ ತನ್ನ ದಿವ್ಯತೆಯನ್ನು ಅರಿಯುವುದು.

 \item ರಾಜಯೋಗ:– ಮನಸ್ಸಿನ ನಿಗ್ರಹದಿಂದ ತನ್ನ ದಿವ್ಯತೆಯನ್ನು ಅರಿಯುವುದು.

 \item ಜ್ಞಾನಯೋಗ:– ಜ್ಞಾನದ ಮೂಲಕ ತನ್ನ ದಿವ್ಯತೆಯನ್ನು ಅರಿಯುವುದು.

\end{enumerate}

ಇವುಗಳೆಲ್ಲ ದೇವರೆಂಬ ಒಂದೇ ಗುರಿಯೆಡೆಗೆ ಒಯ್ಯುವ ಹಲವು ಮಾರ್ಗಗಳು. ಎಲ್ಲಾ ಧರ್ಮಗಳೂ ಜನರನ್ನು ಧಾರ್ಮಿಕ ಪ್ರವೃತ್ತಿಯವರನ್ನಾಗಿ ಮಾಡಲು ಸಹಕಾರಿಗಳಾಗಿರುವುದರಿಂದ ಹೆಚ್ಚು ಧರ್ಮಗಳಿರುವುದು ಅನುಕೂಲವೇ ಆಗಿದೆ. ಮಾನವನಲ್ಲಿ ಸುಪ್ತವಾಗಿರುವ ಪಾವಿತ್ರ್ಯವನ್ನು ಜಾಗೃತಗೊಳಿಸಲು ಎಷ್ಟು ಹೆಚ್ಚು ಪಂಗಡಗಳಿದ್ದರೆ ಅಷ್ಟೂ ಹೆಚ್ಚು\break ಅವಕಾಶಗಳು ದೊರೆಯುತ್ತವೆ.

