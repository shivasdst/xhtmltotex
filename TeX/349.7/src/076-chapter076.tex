
\chapter[ಸಂನ್ಯಾಸಿ ಮತ್ತು ಗೃಹಸ್ಥ ]{ಸಂನ್ಯಾಸಿ ಮತ್ತು ಗೃಹಸ್ಥ \protect\footnote{\engfoot{C.W. Vol. V, P. 260}}}

ಗೃಹಸ್ಥರಿಗೆ ಸಂನ್ಯಾಸಿಗಳ ವಿಷಯದಲ್ಲಿ ಯಾವ ಅಧಿಕಾರವೂ ಇರಕೂಡದು. ಸಂನ್ಯಾಸಿಯು ಶ‍್ರೀಮಂತನ ಹತ್ತಿರ ಯಾವ ವ್ಯವಹಾರವನ್ನೂ ಇಟ್ಟುಕೊಳ್ಳಕೂಡದು. ಅವನ ಕರ್ತವ್ಯ ಬಡವರ ಹತ್ತಿರ. ಬಡವರನ್ನು ಪ್ರೀತಿ ವಿಶ್ವಾಸದಿಂದ ಕಂಡು ಸಾಧ್ಯವಾದಷ್ಟು ಅವರಿಗೆ ಸೇವೆ ಮಾಡಬೇಕು. ಶ‍್ರೀಮಂತರನ್ನು ಆದರಿಸಿ ಅವರನ್ನು ಅನುಸರಿಸಿಕೊಂಡಿರುವುದು ಭರತಖಂಡದ ಸಂನ್ಯಾಸಿಗಳ ಪಾಡಾಗಿದೆ. ನಿಜವಾದ ಸಂನ್ಯಾಸಿ ಇದನ್ನು ಎಲ್ಲಾ ವಿಧಗಳಿಂದಲೂ\break ತ್ಯಜಿಸಬೇಕು. ಶ‍್ರೀಮಂತರನ್ನು ಆಶ್ರಯಿಸಿಕೊಂಡಿರುವುದು ವೇಶ್ಯೆಗೆ ಯೋಗ್ಯವಾದ\break ಕೆಲಸವೇ ಹೊರತು ಸಂನ್ಯಾಸಿಗೆ ಯೋಗ್ಯವಾದುದಲ್ಲ. ಕಾಮಕಾಂಚನಗಳಲ್ಲಿ\break ನಿರತನಾಗಿರುವವನು, ಯಾರ ಜೀವನದ ಪಲ್ಲವಿಯೇ ಕಾಮಕಾಂಚನ ತ್ಯಾಗವಾಗಿದೆಯೋ ಅಂತಹವರ ಅನುಯಾಯಿ ಹೇಗೆ ಆಗಬಲ್ಲ? ಶ‍್ರೀರಾಮಕೃಷ್ಣರು ಕಂಬನಿದುಂಬಿ ಜಗನ್ಮಾತೆಗೆ ಕಾಮಕಾಂಚನ ಸೋಂಕದವರನ್ನು ತನ್ನೆಡೆಗೆ ಮಾತನಾಡುವುದಕ್ಕೆ ಕಳುಹಿಸೆಂದು ಪ್ರಾರ್ಥಿಸುತ್ತಿದ್ದರು. ಪ್ರಾಪಂಚಿಕರೊಂದಿಗೆ ಮಾತನಾಡಿ ತನ್ನ ತುಟಿಯೆಲ್ಲ ಸುಟ್ಟುಹೋಗಿದೆ ಎನ್ನುತ್ತಿದ್ದರು. ಪ್ರಾಪಂಚಿಕರ ಮತ್ತು ಹೀನ ಜೀವನವನ್ನು ನಡೆಸಿದವರ ಸ್ಪರ್ಶವನ್ನು ಕೂಡ ತಮಗೆ ಸಹಿಸುವುದಕ್ಕೆ ಆಗುವುದಿಲ್ಲ ಎನ್ನುತ್ತಿದ್ದರು. ಯತಿರಾಜರಾದ ಶ‍್ರೀರಾಮಕೃಷ್ಣರ ಸಂದೇಶವನ್ನು ಪ್ರಾಪಂಚಿಕರು ಎಂದಿಗೂ ಬೋಧಿಸಲಾರರು. ಪ್ರಾಪಂಚಿಕರು ಎಂದಿಗೂ ನಿಸ್ಪೃಹಿಗಳಾಗಿರಲಾರರು. ಅವರಿಗೆ ಯಾವುದಾದರೂ ಸ್ವಲ್ಪ ಸ್ವಾರ್ಥ ಇದ್ದೇ ಇರುವುದು. ದೇವರೇ ಗೃಹಸ್ಥನಾಗಿ ಅವತಾರವೆತ್ತಿ ಬಂದರೂ ಅವನು ನಿಸ್ಪೃಹಿಯೆಂದು ನಾನು\break ನಂಬುವುದಿಲ್ಲ. ಗೃಹಸ್ಥನು ಧಾರ್ಮಿಕ ನಾಯಕರ ಪಾತ್ರವನ್ನು ವಹಿಸಿದರೆ ಧರ್ಮದ\break ಹೆಸರಿನಲ್ಲಿ ತನ್ನ ಸ್ವಾರ್ಥವನ್ನು ಸಾಧಿಸುವನು. ಇದರ ಪರಿಣಾಮವಾಗಿಯೆ ಆ ಪಂಥ ಕುಲಗೆಟ್ಟು ಹೋಗುವುದು. ಗೃಹಸ್ಥರ ಮುಂದಾಳುತನದಲ್ಲಿರುವ ಧರ್ಮಗಳಿಗೆಲ್ಲಾ ಇದೇ ಗತಿಯಾಗಿದೆ. ತ್ಯಾಗವಿಲ್ಲದೆ ಧರ್ಮ ನಿಲ್ಲಲಾರದು.

ಕಾಂಚನವನ್ನು ಸಂನ್ಯಾಸಿಗಳು ತ್ಯಜಿಸುವುದು ಎಂದರೆ ಹೇಗೆ ಎಂದು ತಿಳಿಸಬೇಕೆಂದು ಯಾರೋ ಸ್ವಾಮಿಗಳನ್ನು ಕೇಳಿದರು. ಅದಕ್ಕೆ ಸ್ವಾಮೀಜಿ ಹೀಗೆ ಉತ್ತರ ಕೊಟ್ಟರು:

ಒಂದು ಗುರಿ ಸಾಧನೆಗೆ ಒಂದು ಮಾರ್ಗವನ್ನು ನಾವು ಅನುಸರಿಸಬೇಕಾಗಿದೆ. ಈ ಮಾರ್ಗವು ಕಾಲ ದೇಶ ವ್ಯಕ್ತಿಗಳಿಗೆ ತಕ್ಕಂತೆ ಬದಲಾಗುವುದು. ಆದರೆ ಗುರಿ ಮಾತ್ರ ಬದಲಾಗುವುದಿಲ್ಲ. ಸಂನ್ಯಾಸಿಯ ಗುರಿ ತನ್ನ ಆತ್ಮನ ಮೋಕ್ಷ ಮತ್ತು ಜಗತ್ತಿನ ಹಿತ. (ಆತ್ಮನೋ ಮೋಕ್ಷಾರ್ಥಂ ಜಗದ್ಧಿತಾಯ ಚ) ಇದನ್ನು ಸಾಧಿಸುವುದಕ್ಕೆ ಕಾಮಕಾಂಚನಗಳ ತ್ಯಾಗ ಅತ್ಯಾವಶ್ಯಕ. ತ್ಯಾಗವೆಂದರೆ ಎಲ್ಲಾ ಸ್ವಾರ್ಥದ ನಾಶ. ತನ್ನ ದುಡ್ಡನ್ನು ಇನ್ನೊಬ್ಬರ ಹತ್ತಿರ ಇಟ್ಟು ಅದನ್ನು ಮುಟ್ಟದೆ ಅದರ ಪ್ರಯೋಜನವನ್ನೆಲ್ಲಾ ಪಡೆಯುವುದಲ್ಲ. ಇದು ತ್ಯಾಗವಾಗಬಲ್ಲುದೆ? ಮೇಲಿನ ಎರಡು ಆದರ್ಶಗಳ ಸಾಧನೆಗೆ (ಆತ್ಮನ ಮೋಕ್ಷ, ಜಗತ್ತಿನ\break ಹಿತ) ಭಿಕ್ಷಾ ಜೀವನ ಬಹಳ ಸಹಕಾರಿಯಾಗಿರುತ್ತಿತ್ತು. ಹಿಂದಿನ ಕಾಲದಲ್ಲಿ ಮನು\break ಮುಂತಾದವರು ಹೇಳಿದಂತೆ ಗೃಹಸ್ಥರು ಸಂನ್ಯಾಸಿಗಳಿಗೆ ಪ್ರತಿದಿನವೂ ತಾವು ಮಾಡಿದ ಅಡಿಗೆಯಲ್ಲಿ ಸ್ವಲ್ಪ ಭಾಗವನ್ನು ತೆಗೆದು ಇಡುತ್ತಿದ್ದಾಗ ಇದು ಸರಿಯಾಗುತ್ತಿತ್ತು. ಆದರೆ ಈಗ, ಅದರಲ್ಲೂ ಬಂಗಾಳ ದೇಶದಲ್ಲಿ ಮಧುಕರಿ ವೃತ್ತಿ ಜಾರಿಯಲ್ಲಿ ಇಲ್ಲ. ಇಲ್ಲಿ ಮಧುಕರಿಯಿಂದ ಜೀವನ ನಡೆಸುವುದು ವೃಥಾ ಶಕ್ತಿವ್ಯಯ ಭಿಕ್ಷೆ ಬೇಡುವುದು ಮೇಲಿನ ಎರಡು ಗುರಿಗಳನ್ನು ಸಾಧಿಸುವುದಕ್ಕೆ ಸಹಕಾರಿ. ಆದರೆ ಈಗ ಅದರಿಂದ ಪ್ರಯೋಜನವಿಲ್ಲ. ಇಂತಹ ಸ್ಥಿತಿಯಲ್ಲಿ ಸಂನ್ಯಾಸಿ ಕೇವಲ ತನ್ನ ಜೀವನೋಪಾಯಕ್ಕೆ ದ್ರವ್ಯವನ್ನು ಸ್ವೀಕರಿಸಿ ಉಳಿದ ಕಾಲವನ್ನು ಆದರ್ಶ ಸಾಧನೆಗೆ ಉಪಯೋಗಿಸಿದರೆ ಬಾಧಕವಿಲ್ಲ. ಬರಿಯ ಮಾರ್ಗಕ್ಕೇ ಹೆಚ್ಚು ಪ್ರಾಮುಖ್ಯ ಕೊಡುವುದರಿಂದ ಭ್ರಾಂತರಾಗುವೆವು. ಗುರಿಯನ್ನು ಎಂದಿಗೂ ಮರೆಯಕೂಡದು.

\vskip -0.5cm

