
\chapter[ತರಗತಿಯ ಟಿಪ್ಪಣಿಗಳು - 1 ]{ತರಗತಿಯ ಟಿಪ್ಪಣಿಗಳು - 1 \protect\footnote{\engfoot{C.W. Vol. VII, P. 407-15}}}

\centerline{\textbf{ಕಲೆ}}

ಕಲೆಯಲ್ಲಿ ಮುಖ್ಯ ವಿಷಯದ ಮೇಲೆ ಆಸಕ್ತಿಯನ್ನು ಕೇಂದ್ರೀಕರಿಸಬೇಕು. ನಾಟಕ\break ಎಲ್ಲಕ್ಕಿಂತಲೂ ಬಹಳ ಕಷ್ಟವಾದ ಕಲೆ. ಅದರಲ್ಲಿ ನಾವು ಎರಡನ್ನು ತೃಪ್ತಿಪಡಿಸಬೇಕಾಗಿದೆ. ಮೊದಲನೆಯದು ಕಿವಿ, ಎರಡನೆಯದು ಕಣ್ಣು. ಒಂದು ದೃಶ್ಯವನ್ನು ಚಿತ್ರಿಸಬೇಕಾದರೆ\break ಒಂದು ವಸ್ತುವನ್ನು ಚಿತ್ರಿಸುವುದು ಸುಲಭ. ಬೇರೆ ಬೇರೆ ವಿಷಯಗಳನ್ನು ಚಿತ್ರಿಸಿ\break ಮುಖ್ಯವಾದ ವಿಷಯದ ಮೇಲೆ ಕೇಂದ್ರೀಕರಿಸುವುದು ಕಷ್ಟ. ಬರದಂತೆ ಉಳಿದವುಗಳನ್ನೆಲ್ಲ ಅಣಿಗೊಳಿಸಬೇಕಾಗಿದೆ.

\centerline{\textbf{ಸಂಗೀತ}}

ದ್ರುಪದ್​ ಮತ್ತು ಖ್ಯಾಲ್​ ಮುಂತಾದವುಗಳಲ್ಲಿ ಶಾಸ್ತ್ರವಿದೆ. ಆದರೆ ಕೀರ್ತನೆ ಮತ್ತು ವಿರಹ ಮುಂತಾದವುಗಳಲ್ಲಿ ನಿಜವಾದ ಸಂಗೀತ ಇರುವುದು. ಏಕೆಂದರೆ ಅಲ್ಲಿ ಭಾವನೆ ಇದೆ. ಭಾವವೇ ಪ್ರತಿಯೊಂದರ ಸಾರ, ರಹಸ್ಯ. ಜನಸಾಮಾನ್ಯರ ಹಾಡುಗಳಲ್ಲಿ ಬೇಕಾದಷ್ಟು ಸಂಗೀತವಿದೆ. ಅವುಗಳನ್ನೆಲ್ಲ ಸಂಗ್ರಹಿಸಬೇಕು. ದ್ರುಪದ್​ ಮುಂತಾದವುಗಳ ಶಾಸ್ತ್ರವನ್ನು, ಕೀರ್ತನದ ಸಂಗೀತಕ್ಕೆ ಸೇರಿಸಿದರೆ ಪೂರ್ಣವಾದ ಸಂಗೀತವಾಗುವುದು.

\centerline{\textbf{ಮಂತ್ರ ಮತ್ತು ಮಂತ್ರಚೈತನ್ಯ}}

ಮಂತ್ರಶಾಸ್ತ್ರಜ್ಞರು, ಹಿಂದಿನ ಕಾಲದಿಂದಲೂ ಕೆಲವು ಮಂತ್ರಗಳು ಗುರು ಶಿಷ್ಯರ, ಪರಂಪರೆಯ ಮೂಲಕ ಬಂದಿವೆ, ಅವುಗಳನ್ನು ಸುಮ್ಮನೆ ಉಚ್ಚಾರ ಮಾಡಿದರೆ ಸಾಕು, ಏನಾದರೂ ಒಂದು ರೀತಿಯ ಸಾಕ್ಷಾತ್ಕಾರ ಆಗುವುದು, ಎಂದು ನಂಬುತ್ತಾರೆ. ಮಂತ್ರಚೈತನ್ಯ ಎಂಬ ಪದಕ್ಕೆ ಎರಡು ಅರ್ಥಗಳಿವೆ. ಕೆಲವರ ಅಭಿಪ್ರಾಯದ ಪ್ರಕಾರ ಕೆಲವು ಮಂತ್ರಗಳನ್ನು\break ಜಪಮಾಡಿದರೆ ಆ ಮಂತ್ರದ ದೇವರು ಪ್ರತ್ಯಕ್ಷವಾಗುವುದು. ಆದರೆ ಇನ್ನು ಕೆಲವರ\break ಅಭಿಪ್ರಾಯದಲ್ಲಿ, ಅರ್ಹನಲ್ಲದೇ ಇರುವ ಗುರುವಿನ ಮಂತ್ರವನ್ನು ಪಡೆದು ಅದನ್ನೇ ಜಪಮಾಡಿದರೆ ಸಿದ್ಧಿಸುವುದಿಲ್ಲ. ಅದು ಸಿದ್ಧಿಸಬೇಕಾದರೆ ಅದಕ್ಕಾಗಿ ಯಾವುದಾದರೂ ಪರಿಹಾರವನ್ನು ಮಾಡಿಕೊಳ್ಳಬೇಕು, ಆಗ ಮಂತ್ರದಲ್ಲಿ ಚೈತನ್ಯ ಉದಯಿಸುವುದು ಎಂದು ಹೇಳುವರು. ಅನಂತರ ಅದನ್ನು ಜಪಮಾಡಿದರೆ ಸಿದ್ಧಿಸುವುದು. ಚೈತನ್ಯದಿಂದ ಕೂಡಿದ ಹಲವು ಮಂತ್ರಗಳಿಗೆ ಹಲವು ಚಿಹ್ನೆಗಳಿವೆ. ಆದರೆ ಅವುಗಳಲ್ಲೆಲ್ಲಾ ಸರ್ವಸಾಮಾನ್ಯವಾಗಿರುವುದೇ, ಯಾವ ಕಷ್ಟವೂ ಇಲ್ಲದೆ ಸಾಧಕನು ಯಾವ ಮಂತ್ರವನ್ನು ಬಹಳ ಕಾಲದವರೆಗೆ ಉಚ್ಚರಿಸಬಲ್ಲನೋ ಮತ್ತು ಯಾವುದು ಅವನ ಮನಸ್ಸನ್ನು ಬೇಗ ಆದರ್ಶದ ಮೇಲೆ ಏಕಾಗ್ರವಾಗುವಂತೆ ಮಾಡಬಲ್ಲುದೋ ಅದು. ಇವು ತಾಂತ್ರಿಕ ಮಂತ್ರಗಳಿಗೆ ಸಂಬಂಧಪಟ್ಟವುಗಳು.

ವೇದಗಳ ಕಾಲದಿಂದಲೂ ಮಂತ್ರದ ವಿಷಯದಲ್ಲಿ ಎರಡು ಅಭಿಪ್ರಾಯಗಳು ಇದ್ದುವು. ಯಾಸ್ಕ ಮುಂತಾದವರು ವೇದಮಂತ್ರಗಳಲ್ಲಿ ಅರ್ಥವಿದೆ ಎನ್ನುವರು. ಆದರೆ ಪುರಾತನ ಮಂತ್ರಶಾಸ್ತ್ರಜ್ಞರಾದರೋ ಅವಕ್ಕೆ ಅರ್ಥವಿಲ್ಲ ಎನ್ನುವರು. ಅವುಗಳನ್ನು ಕೆಲವು ಯಜ್ಞಗಳಲ್ಲಿ ಉಚ್ಚರಿಸಿದರೆ, ಅವು ಲೌಕಿಕ ಮತ್ತು ಅಧ್ಯಾತ್ಮಿಕ ವಸ್ತುಗಳನ್ನು ಕೊಡಬಲ್ಲುವು ಎಂದು ಹೇಳುತ್ತಾರೆ. ಉಪನಿಷತ್ತಿನ ಉಚ್ಚಾರಣೆಯಿಂದ ಆಧ್ಯಾತ್ಮಿಕ ಜ್ಞಾನ ಲಭಿಸುತ್ತದೆ.

\centerline{\textbf{ಈಶ್ವರನ ಭಾವನೆಗಳು}}

ಮನುಷ್ಯನ ಅಂತರಾಳದಲ್ಲಿ, ಯಾರಾದರೂ ಒಬ್ಬ ಮುಕ್ತನಾದವನನ್ನು, ಪ್ರಕೃತಿಯ ನಿಯಮಗಳಿಗೆ ಅತೀತವಾಗಿ ಹೋಗಿರುವವನನ್ನು, ಕಾಣಬೇಕೆಂಬ ಬಯಕೆ ಇದೆ. ವೇದಾಂತಿಗಳು ಇಂತಹ ನಿತ್ಯನಾಗಿರುವ ಈಶ್ವರನನ್ನು ನಂಬುತ್ತಾರೆ. ಆದರೆ ಬೌದ್ಧರು ಮತ್ತು ಸಾಂಖ್ಯರು ಜನ್ಯೇಶ್ವರನಲ್ಲಿ ಮಾತ್ರ ನಂಬುತ್ತಾರೆ. ಅಂದರೆ ಸೃಷ್ಟಿಯಾದ ಈಶ್ವರ, ಮುಂಚೆ ಮನುಷ್ಯರಾಗಿದ್ದು ತಮ್ಮ ತಪಃಪ್ರಭಾವದಿಂದ ಈಗ ಆ ಸ್ಥಿತಿಗೆ ಏರಿರುವರು. ಪುರಾಣಗಳು ಇವೆರಡು ಸಿದ್ಧಾಂತಗಳಿಗೂ ಅವತಾರ ವಾದದ ಮೂಲಕ ರಾಜಿಮಾಡಿಸುತ್ತವೆ. ಜನ್ಯೇಶ್ವರನೆಂದರೆ, ನಿತ್ಯ ಈಶ್ವರನೇ ಮಾಯೆಯ ಮೂಲಕ ಜನ್ಯೇಶ್ವರನಂತೆ ಆಗುತ್ತಾನೆ ಎನ್ನುವರು. ನಿತ್ಯೇಶ್ವರನಾದಂತಹ ಮುಕ್ತನು ಹೇಗೆ ಸೃಷ್ಟಿಸಬಲ್ಲ ಎಂಬ ಸಾಂಖ್ಯರ ವಾದವು ತಪ್ಪು ತರ್ಕದ ಮೇಲೆ ನಿಂತಿದೆ. ಅವನು ಮುಕ್ತ. ಆದಕಾರಣ ಅವನು ತನಗೆ ತೋಚಿದುದನ್ನು ಮಾಡಬಲ್ಲ. ವೇದಾಂತದ ದೃಷ್ಟಿಯಲ್ಲಿ ಜನ್ಯೇಶ್ವರರು, ಪ್ರಪಂಚದ ಸೃಷ್ಟಿ ಸ್ಥಿತಿ ಪ್ರಳಯಗಳನ್ನು\break ಮಾಡಲಾರರು.

\centerline{\textbf{ಆಹಾರವನ್ನು ಕುರಿತು}}

ನೀವು ಇತರರು ಪುರುಷಸಿಂಹರಾಗಬೇಕು ಎಂದು ಹೇಳುತ್ತೀರಿ. ಆದರೆ ಅವರಿಗೆ\break ಸರಿಯಾದ ಆಹಾರವನ್ನು ಕೊಡುವುದಿಲ್ಲ. ನಾನು ಕಳೆದ ನಾಲ್ಕು ವರುಷಗಳಿಂದಲೂ ಈ ಸಮಸ್ಯೆಯನ್ನು ಕುರಿತು ವಿಚಾರ ಮಾಡುತ್ತಿರುವೆನು. ಗೋಧಿಯಿಂದ ಅವಲಕ್ಕಿಯಂತಹ ಯಾವುದಾದರೂ ವಸ್ತುವನ್ನು ತಯಾರು ಮಾಡಲು ಸಾಧ್ಯವೇ ಎಂಬುದನ್ನು ನಾನು ವಿಚಾರಿಸಬೇಕೆಂದು ಇರುವೆನು. ಆಗ ಪ್ರತಿದಿನವೂ ಬೇರೆ ಬೇರೆ ಆಹಾರವನ್ನು ತಿನ್ನಬಹುದು. ಕುಡಿಯುವ ನೀರನ್ನು ಶುದ್ಧ ಮಾಡುವುದಕ್ಕಾಗಿ ನಮ್ಮ ದೇಶಕ್ಕೆ ಹೊಂದುವ ಜಲಶೋಧಕವನ್ನು ಹುಡುಕಿದೆ. ಒಂದು ಪೋರ್ಸಿಲೈನ್​ ಬಾಂಡಲೆಯ ಮೂಲಕ ನೀರನ್ನು ಶೋಧಿಸಿದರೆ ಅದರಲ್ಲಿ ಕ್ರಿಮಿಗಳೆಲ್ಲ ಸಂಗ್ರಹವಾಗುವುದನ್ನು ನಾನು ನೋಡಿದೆ. ಕ್ರಿಮಿಗಳೆಲ್ಲ ಆ ಪೋರ್ಸಿಲೈನ್​ ಪಾತ್ರೆಯ ಒಳಗೆಯೇ ಉಳಿದವು. ಕ್ರಮೇಣ ಅಂತಹ ಜಲಶೋಧಕವೇ ಕ್ರಿಮಿಗಳ ಬೀಡಾಗುವುದು. ಜಲಶೋಧಕಯಂತ್ರದ ಒಂದು ಅಪಾಯ ಇದು. ಬೇಕಾದಷ್ಟು ಶೋಧನೆಗಳನ್ನು ಮಾಡಿದ ಮೇಲೆ ನಾನು ಒಂದು ಮಾರ್ಗವನ್ನು ಕಂಡುಹಿಡಿದೆ. ಅದು ನೀರನ್ನು ಆವಿಮಾಡಿ ಅದನ್ನು ತಂಪು ಮಾಡುವುದು, ಆ ನೀರಿಗೆ ಆಮ್ಲಜನಕವನ್ನು ಸೇರಿಸುವುದು. ಇದರಿಂದ ನೀರು ಬಹಳ ಶುದ್ಧವಾಗುವುದು. ಇದು ಆರೋಗ್ಯಕ್ಕೆ ತುಂಬಾ ಸಹಾಯ ಮಾಡುವುದು.

\centerline{\textbf{ಸಂನ್ಯಾಸ ಮತ್ತು ಗೃಹಸ್ಥ ಜೀವನ}}

ಇವರಿಬ್ಬರ ಕರ್ತವ್ಯಗಳನ್ನು ಕುರಿತು ಸ್ವಾಮೀಜಿ ಅವರು ಹೀಗೆ ಹೇಳಿದರು:\break ಸಂನ್ಯಾಸಿಯು ತನ್ನ ಆಧ್ಯಾತ್ಮಿಕ ಪುರೋಗಮನದ ದೃಷ್ಟಿಯಿಂದ ಗೃಹಸ್ಥರು ಮುಟ್ಟಿದ ಆಹಾರ ಹಾಸಿಗೆ ಬಟ್ಟೆಬರೆ ಮುಂತಾದುವುಗಳನ್ನು ಉಪಯೋಗಿಸಕೂಡದು. ಹಾಗೆ ತೆಗೆದುಕೊಳ್ಳಬಾರದು ಎನ್ನುವುದು ಗೃಹಸ್ಥರ ಮೇಲಿನ ದ್ವೇಷದಿಂದಲ್ಲ. ಸಂನ್ಯಾಸಿಯು ತಾನು ಪರಮಹಂಸನಾಗುವವರೆಗೆ ಈ ನಿಯಮವನ್ನು ಅನುಸರಿಸಬೇಕು. ಗೃಹಸ್ಥರು ಸಂನ್ಯಾಸಿಗಳಿಗೆ “ನಮೋ ನಾರಾಯಣಾಯ” ಎಂದು ಹೇಳಿ ನಮಸ್ಕರಿಸಬೇಕು. ಸಂನ್ಯಾಸಿಯು\break ಗೃಹಸ್ಥರನ್ನು ಆಶೀರ್ವದಿಸಬೇಕು.

\begin{verse}
 ಮೇರುಸರ್ಷಪಯೋರ್ಯದ್​ ಯತ್​ ಸೂರ್ಯಖದ್ಯೋತಯೋರಿವ~।\\
 ಸರಿತ್​ ಸಾಗರಯೋರ್ಯದ್​ ಯತ್​ ತಥಾ ಭಿಕ್ಷುರ್ಗೃಹಸ್ಥಯೋಃ~॥
\end{verse}

ಮೇರುಪರ್ವತಕ್ಕೂ ಸಾಸವೆಕಾಳಿಗೂ ಇರುವ ವ್ಯತ್ಯಾಸದಂತೆ, ಸೂರ್ಯನಿಗೂ ಮಿಂಚುಹುಳುವಿಗೂ ಇರುವ ವ್ಯತ್ಯಾಸದಂತೆ ಸಂನ್ಯಾಸಿ ಮತ್ತು ಗೃಹಸ್ಥ ಇವರೊಳಗಿರುವ ವ್ಯತ್ಯಾಸ.

ಸ್ವಾಮಿ ವಿವೇಕಾನಂದರು ಎಲ್ಲರಿಗೂ ಇದನ್ನು ಉಚ್ಚರಿಸುವಂತೆ ಹೇಳಿದರು. ವೇದಾಂತಸ್ತೋತ್ರ ಒಂದನ್ನು ಪಠಿಸುತ್ತ, ನೀವುಗಳೆಲ್ಲ ಇದನ್ನು ಪಠಿಸುತ್ತಿರಬೇಕು ಎಂದರು. ಶ್ರವಣ ಎಂದರೆ ಗುರುಮುಖೇನ ಕೇಳುವುದು ಮಾತ್ರವಲ್ಲ, ನಾವು ಕೂಡ ಇದನ್ನು ಪಠಿಸುತ್ತಿರಬೇಕು. “ಆವೃತ್ತಿರಸಕೃದುಪದೇಶಾತ್​,” ಶಾಸ್ತ್ರವನ್ನು ಅನೇಕ ವೇಳೆ ನಾವು ಓದುತ್ತಿರಬೇಕು ಎಂದು ಹೇಳುತ್ತಾರೆ. ವ್ಯಾಸರು ಈ ಸೂತ್ರದಲ್ಲಿ ಪುನಃ ಪುನಃ ಅದನ್ನು ಪಠಿಸಬೇಕು\break ಎನ್ನುತ್ತಾರೆ.

\centerline{\textbf{ಗುರುವಿನ ಯೋಗ್ಯತೆಯನ್ನು ಕುರಿತು}}

ಒಂದು ಸಲ ಸಂಭಾಷಣೆ ಆಗುತ್ತಿರುವಾಗ ಸ್ವಾಮೀಜಿ ಅವರು ಸ್ಫೂರ್ತಿಯಿಂದ ಹೀಗೆ ಹೇಳಿದರು: “ನಿಮ್ಮ ಲೆಕ್ಕಾಚಾರದ ಬುದ್ಧಿಯನ್ನು ಬಿಡಿ. ನೀವು ಯಾವುದಾದರೂ ಒಂದೇ ಒಂದು ವಸ್ತುವಿನ ಮೇಲೆ ನಿಮಗೆ ಇರುವ ವ್ಯಾಮೋಹವನ್ನು ತ್ಯಜಿಸಿದರೂ ನೀವು ಮುಕ್ತಿಯ ಕಡೆಗೆ ಹೋಗುತ್ತಿರುವಿರಿ. ಸಾಧು, ಪಾಪಿ ಎಲ್ಲರನ್ನು ಸಮಾನವಾಗಿ ಕಾಣಿರಿ. ಆ ಜಾರಿಣಿ ಸ್ತ್ರೀ ಕೂಡ ಸಾಕ್ಷಾತ್​ ದೇವಿಯೆ. ಸಂನ್ಯಾಸಿ ಒಂದೆರಡು ಸಲ ಅವಳು ತಾಯಿ ಎಂದು ಹೇಳುವನು. ಅನಂತರ ಅವನು ಪುನಃ ಮೋಹಕ್ಕೆ ಬಿದ್ದು ‘ಹೇ ಪಾಪಿಯಾದ ಜಾರೆಯೆ!’ ಎನ್ನುವನು. ಒಂದೇ ಸಲ ಅಜ್ಞಾನವೆಲ್ಲ ಮಾಯವಾಗಬಹುದು. ಅಜ್ಞಾನವು ಕರ್ಮೇಣ ಹೋಗುವುದು ಎನ್ನುವುದು ಅವಿವೇಕದ ಮಾತು. ಗುರು ಭ್ರಷ್ಟನಾದರೂ ಅವನನ್ನು ಬಿಡದೆ ಇರುವ ಶಿಷ್ಯರು\break ಇರುವರು. ರಜಪುಟಾಣದಲ್ಲಿ ಒಬ್ಬ ಗುರು ಕ್ರೈಸ್ತ ಮತಕ್ಕೆ ಸೇರಿದ್ದ. ಆದರೂ ಶಿಷ್ಯ ತನ್ನ\break ಗುರುದಕ್ಷಿಣೆಯನ್ನು ಕೊಡುತ್ತ ಹೋದ. ನಿಮ್ಮ ಪಾಶ್ಚಾತ್ಯ ಭಾವನೆಗಳನ್ನು ಬಿಡಿ. ಯಾವಾಗ ಒಬ್ಬ ಗುರುವಿನಲ್ಲಿ ಶರಣಾಗಿರುವಿರೋ ಕೊನೆಯವರೆಗೂ ಅಲ್ಲೇ ನಿಲ್ಲಿ. ವೇದಾಂತದಲ್ಲಿ ನೀತಿ ಇಲ್ಲ ಎಂದು ಹೇಳುವವರು ಏನೂ ಅರಿಯದವರು. ಹೌದು, ಒಂದು ದೃಷ್ಟಿಯಿಂದ ಅವರು ಹೇಳುವುದು ನಿಜ. ವೇದಾಂತ ನೀತಿಗೆ ಅತೀತವಾಗಿರುವುದು. ನೀವು ಸಂನ್ಯಾಸಿಗಳಾಗಿರುವುದರಿಂದ ಉತ್ತಮ ವಿಷಯಗಳನ್ನು ಕುರಿತು ಮಾತನಾಡಿ.”

\vskip 5pt

\centerline{\textbf{ಶ‍್ರೀರಾಮಕೃಷ್ಣರು ಮತ್ತು ಅವರ ಭಾವನೆಗಳು}}

\vskip 5pt

“ಬಲಾತ್ಕಾರವಾಗಿ ಯಾವುದಾದರೂ ಒಂದನ್ನಾದರೂ ಬ್ರಹ್ಮ ಎಂದು ಆಲೋಚಿಸಿ. ಶ‍್ರೀರಾಮಕೃಷ್ಣರನ್ನು ದೇವರೆಂದು ನೋಡುವುದು ಸುಲಭ. ಆದರೆ ಅದರಲ್ಲಿರುವ ಅಪಾಯವೇ ಇತರರಲ್ಲಿ ನಾವು ಈಶ್ವರ ಭಕ್ತಿಯನ್ನು ಅಭ್ಯಾಸ ಮಾಡಲಾಗುವುದಿಲ್ಲ ಎಂಬುದು.\break ದೇವರು ನಿತ್ಯ, ಅವನಿಗೆ ಯಾವ ಆಕಾರವೂ ಇಲ್ಲ, ಅವನು ಸರ್ವಾಂತರ್ಯಾಮಿ.\break ಅವನಿಗೆ ಯಾವುದಾದರೂ ಆಕಾರವಿದೆ ಎಂದು ಭಾವಿಸುವುದು ಈಶ್ವರನಿಂದೆ. ನೀವು ಯಾವುದಾದರೂ ಒಂದು ವಸ್ತುವಿನಲ್ಲಿ ಮಾತ್ರ ದಿವ್ಯತೆಯನ್ನು ಕಾಣಲು ಪ್ರಯತ್ನಿಸುವಿರಿ. ಅದೇ ವಿಗ್ರಹಾರಾಧನೆಯ ರಹಸ್ಯ.”

\vskip 5pt

ಶ‍್ರೀರಾಮಕೃಷ್ಣರು ಅಕ್ಷರಶಃ ತಾವು ಅವತಾರ ಎಂದು ನಂಬುತ್ತಿದ್ದರು. ಅದು ನನಗೆ ಅರ್ಥವಾಗದೆ ಇರಬಹುದು. ನಾನು ವೇದಾಂತ ದೃಷ್ಟಿಯಿಂದ ಅವರನ್ನು ಬ್ರಹ್ಮ ಎಂದು ಹೇಳುತ್ತಿದ್ದೆ. ಆದರೆ ಅವರು ಕಾಲವಾಗುವುದಕ್ಕೆ ಮುಂಚೆ, ಅವರು ಉಸಿರಾಡುವುದಕ್ಕೆ ತುಂಬಾ ವ್ಯಥೆಪಡುತ್ತಿದ್ದಾಗ, ನಾನು ನನ್ನ ಮನಸ್ಸಿನಲ್ಲೆ ಅವರು ಈ ಸ್ಥಿತಿಯಲ್ಲಿಯೂ ತಾವು ಅವತಾರ ಎಂದು ಹೇಳಿದರೆ ನಾನು ಅದನ್ನು ನಂಬುತ್ತೇನೆ ಎಂದುಕೊಂಡೆ. ತಕ್ಷಣವೇ ಶ‍್ರೀರಾಮಕೃಷ್ಣರು “ಯಾರು ರಾಮನಾಗಿದ್ದನೋ, ಮತ್ತು ಯಾರು ಕೃಷ್ಣನಾಗಿದ್ದನೋ ಅವನೇ ಈಗ ಶ‍್ರೀರಾಮಕೃಷ್ಣನಾಗಿರುವುದು. ಆದರೆ ನಿನ್ನ ವೇದಾಂತ ದೃಷ್ಟಿಯಿಂದ ಅಲ್ಲ” ಎಂದರು. ಅವರು ನನ್ನನ್ನು ಬಹಳವಾಗಿ ಪ್ರೀತಿಸುತ್ತಿದ್ದರು. ಇದರಿಂದ ಅನೇಕರಿಗೆ\break ನನ್ನ ಮೇಲೆ ಅಸೂಯೆ ಉಂಟಾಯಿತು. ಅವರು ಒಬ್ಬನನ್ನು ನೋಡಿದ ಕೂಡಲೆ ಅವನ ಶೀಲವನ್ನೆಲ್ಲ ತಿಳಿದುಕೊಳ್ಳಬಲ್ಲವರಾಗಿದ್ದರು. ಎಂದಿಗೂ ತಮ್ಮ ಅಭಿಪ್ರಾಯವನ್ನು ಅವರು ಅನಂತರ ಬದಲಾಯಿಸುತ್ತಿರಲಿಲ್ಲ. ನಾವು ವಿಚಾರದ ಮೂಲಕ ಒಬ್ಬನನ್ನು ತಿಳಿದುಕೊಳ್ಳಲು ಯತ್ನಿಸುತ್ತಿದ್ದರೆ ಅವರು ಯಾವುದೋ ಅಲೌಕಿಕ ಶಕ್ತಿಯ ಮೂಲಕ ತಿಳಿದುಕೊಳ್ಳುತ್ತಿದ್ದರು. ನಮ್ಮ ಅಭಿಪ್ರಾಯಗಳಾದರೊ ಅನೇಕ ವೇಳೆ ತಪ್ಪಾಗುತ್ತಿತ್ತು. ಅವರು ಕೆಲವರನ್ನು ತಮ್ಮ ಅಂತರಂಗಿಗಳೆಂದು ಕರೆಯುತ್ತಿದ್ದರು. ಅವರಿಗೆ ತಮ್ಮ ಜೀವನದ ರಹಸ್ಯ ಮತ್ತು ಯೋಗ ಮುಂತಾದುವುಗಳನ್ನು ಹೇಳುತ್ತಿದ್ದರು. ಬಹಿರಂಗಿಗಳಿಗೆ ಅವರು ಈಗ ನಾವು ಯಾವುದನ್ನು ಅವರ ಉಪದೇಶ ಎಂದು ಹೇಳುವೆವೊ ಅಂತಹ ದೃಷ್ಟಾಂತ ಕಥೆಗಳನ್ನು ಬೋಧಿಸುತ್ತಿದ್ದರು.\break ಅವರ ಅಂತರಂಗಿಗಳನ್ನು ತಮ್ಮ ಕೆಲಸಕ್ಕೆ ಅಣಿಗೊಳಿಸುತ್ತಿದ್ದರು. ಅನೇಕರು ಅವರ ವಿಷಯದಲ್ಲಿ ದೂರು ಹೇಳಿದರೂ ಅವರು ಅದನ್ನು ಗಮನಿಸುತ್ತಿರಲಿಲ್ಲ. ಅವನು ಮಾಡುವ ಕೆಲಸಗಳ ಮೂಲಕ ಬಹಿರಂಗಿಯ ಸ್ವಭಾವ ಎಂತಹದು ಎಂಬುದು ನನಗೆ ಚೆನ್ನಾಗಿ ಗೊತ್ತಿರಬಹುದು. ಆದರೆ ಅಂತರಂಗಿಗಳ ಮೇಲೆ ನನಗೆ ಒಂದು ವಿಧವಾದ ಮೂಢಭಕ್ತಿಯ ಗೌರವವಿದೆ. “ನನ್ನನ್ನು ಪ್ರೀತಿಸಿದರೆ, ನನ್ನ ನಾಯಿಯನ್ನೂ ಪ್ರೀತಿಸು” ಎಂಬ ಗಾದೆ ಇದೆ. ನಾನು ಆ\break ಬ್ರಾಹ್ಮಣ ಪೂಜಾರಿಯನ್ನು ಅತಿ ಉತ್ಕಟವಾಗಿ ಪ್ರೀತಿಸುತ್ತೇನೆ. ಅವರು ಯಾವುದನ್ನು ಗೌರವಿಸುತ್ತಿದ್ದರೋ ನಾನು ಅದನ್ನು ಗೌರವಿಸುತ್ತೇನೆ. ನನ್ನನ್ನು ನನ್ನ ಪಾಡಿಗೆ ಬಿಟ್ಟಿದ್ದರೆ ನಾನೊಂದು ಪಂಥವನ್ನು ಕಟ್ಟುತ್ತಿದ್ದೆ ಎಂದು ಅವರು ನನ್ನ ವಿಷಯದಲ್ಲಿ ಕಳವಳವನ್ನು ವ್ಯಕ್ತಪಡಿಸಿದ್ದರು.

\vskip 5pt

ಅವರು ಕೆಲವರಿಗೆ ನೀವು ಈ ಜನ್ಮದಲ್ಲಿ ಆಧ್ಯಾತ್ಮಿಕ ಸಂಪತ್ತನ್ನು ಗಳಿಸಲಾರಿರಿ ಎಂದು ಹೇಳುತ್ತಿದ್ದರು. ಅವರು ಪ್ರತಿಯೊಂದನ್ನೂ ತಿಳಿದುಕೊಳ್ಳಬಲ್ಲವರಾಗಿದ್ದರು. ಅದಕ್ಕಾಗಿಯೇ ಕೆಲವರ ಮೇಲೆ ಅವರಿಗೆ ಪಕ್ಷಪಾತ. ಅವರು ವೈದ್ಯರಂತೆ, ಒಬ್ಬೊಬ್ಬರಿಗೆ ಒಂದೊಂದು ಬಗೆಯ ಚಿಕಿತ್ಸೆ ಆವಶ್ಯಕ ಎಂಬುದನ್ನು ಮನಗಂಡಿದ್ದರು. ಅಂತರಂಗಿಗಳಿಗಲ್ಲದೆ ಉಳಿದವರಾರಿಗೂ ಅವರ ಕೋಣೆಯಲ್ಲಿ ಮಲಗಲು ಅವಕಾಶವಿರಲಿಲ್ಲ. ಯಾರು ಅವರನ್ನು ನೋಡಿಲ್ಲವೋ ಅವರು ಮುಕ್ತಿಯನ್ನು ಪಡೆಯುವುದಿಲ್ಲ ಎಂಬುದು ಸತ್ಯವಲ್ಲ. ಅಥವಾ ಯಾರು ಅವರನ್ನು ಮೂರು ಬಾರಿ ನೋಡಿರುವರೋ ಅವರು ಮುಕ್ತಿಯನ್ನು ಹೊಂದುತ್ತಾರೆ ಎಂಬುದೂ ಸತ್ಯವಲ್ಲ.

\vskip 5pt

ಯಾರಿಗೆ ಗಹನವಾದ ವೇದಾಂತ ವಿಷಯಗಳನ್ನು ತಿಳಿದುಕೊಳ್ಳಲು ಸಾಧ್ಯವಿಲ್ಲವೋ ಅಂತಹ ಜನಸಾಧಾರಣರಿಗೆ ನಾರದೋಕ್ತ ಭಕ್ತಿಯನ್ನು ಅವರು ಬೋಧಿಸುತ್ತಿದ್ದರು.

\vskip 5pt

ಅವರು ಸಾಧಾರಣವಾಗಿ ದ್ವೈತವನ್ನು ಬೋಧಿಸುತ್ತಿದ್ದರು. ಅವರು ಎಂದಿಗೂ ಅದ್ವೈತವನ್ನು ಬೋಧಿಸುತ್ತಿರಲಿಲ್ಲ. ಆದರೆ ನನಗೆ ಅವರು ಅದನ್ನು ಬೋಧಿಸಿದರು. ನಾನು ಮುಂಚೆ ದ್ವೈತಿಯಾಗಿದ್ದೆ.

\vskip 5pt

\centerline{\textbf{ಶ‍್ರೀರಾಮಕೃಷ್ಣರು ರಾಷ್ಟ್ರದ ಆದರ್ಶ}}

\vskip 5pt

ಒಂದು ಜನಾಂಗ ಉದ್ಧಾರವಾಗಬೇಕಾದರೆ ಅದಕ್ಕೊಂದು ಭವ್ಯವಾದ ಆದರ್ಶವಿರಬೇಕು. ಈಗ ಆ ಆದರ್ಶವೇ ಪರಬ್ರಹ್ಮ. ಆದರೆ ಒಂದು ನಿರ್ಗುಣ ಆದರ್ಶದಿಂದ ಸ್ಫೂರ್ತಿಪಡೆಯಲು ನಿಮಗೆ ಸಾಧ್ಯವಿಲ್ಲದೆ ಇರುವುದರಿಂದ ಯಾವುದಾದರೂ ಒಂದು ವ್ಯಕ್ತಿಯ ಆದರ್ಶ ಬೇಕಾಗುವುದು. ನಿಮಗೆ ಶ‍್ರೀ ರಾಮಕೃಷ್ಣರಲ್ಲಿ ಅಂತಹ ಆದರ್ಶವಿದೆ. ಇತರ ವ್ಯಕ್ತಿಗಳು ನಮ್ಮ ಆದರ್ಶ ಆಗದೆ ಇರುವುದಕ್ಕೆ ಕಾರಣ, ಅವರುಗಳ ಕಾಲ ಈಗ ಆಗಿಹೋಗಿರುವುದು. ವೇದಾಂತವು ಪ್ರತಿಯೊಬ್ಬರಿಗೂ ತಿಳಿಯಬೇಕಾಗಿರುವುದರಿಂದ, ಈಗಿನ ಕಾಲದ ಜನರಿಗೆ ಸಹಾನುಭೂತಿಯನ್ನು ತೋರುವ ಒಂದು ವ್ಯಕ್ತಿ ಬೇಕಾಗಿದೆ. ಅದು ಶ‍್ರೀರಾಮಕೃಷ್ಣರಲ್ಲಿ ಈಡೇರುವುದು. ಆದಕಾರಣ ನೀವು ಈಗ ಅವರ ಆದರ್ಶವನ್ನು ಎಲ್ಲರ ಮುಂದೆ ಇರಿಸಬೇಕು. ಒಬ್ಬನು ಅವರನ್ನು ಕೇವಲ ಸಾಧುವೆಂದು ಸ್ವೀಕರಿಸುವನೆ, ಅಥವಾ ಅವತಾರವೆಂದು ಸ್ವೀಕರಿಸುವನೆ ಎಂಬುದು ಗೌಣ.

\eject

ಅವರು ಇಲ್ಲಿಗೆ ಪುನಃ ಬರುತ್ತೇನೆ ಎಂದು ಹೇಳುತ್ತಿದ್ದರು. ಆಗ ಅವರು ವಿದೇಹ\break ಮುಕ್ತಿಯನ್ನು ಪಡೆಯುವರು. ನೀವು ಕೆಲವು ಮಾಡಬೇಕಾದರೆ ನಿಮಗೆ ಅಂತಹ ಇಷ್ಟದೇವತೆ ಇರಬೇಕು, ಕ್ರೈಸ್ತ ಜನಾಂಗ ಹೇಳುವಂತೆ ರಕ್ಷಾದೇವತೆ ಇರಬೇಕು. ಪ್ರತಿಯೊಂದು ರಾಷ್ಟ್ರಕ್ಕೂ ಬೇರೆ ಬೇರೆ ಇಷ್ಟದೇವತೆಗಳು ಇವೆ. ಇವರಲ್ಲಿ ಪ್ರತಿಯೊಬ್ಬರೂ ತಮ್ಮದೇ ಶ್ರೇಷ್ಠ ಎಂದು ಸಾಧಿಸಲು ಪ್ರಯತ್ನಿಸುತ್ತಿರುವರು ಎಂದು ನಾನು ಕೆಲವು ವೇಳೆ ಯೋಚಿಸುತ್ತೇನೆ. ಕೆಲವು ವೇಳೆ ಅಂತಹ ಇಷ್ಟದೇವತೆಗಳಿಗೆ ರಾಷ್ಟ್ರಕ್ಕೆ ಸಹಾಯ ಮಾಡಲು ಶಕ್ತಿ ಇರುವುದಿಲ್ಲ ಎಂದೂ ನಾನು ಊಹಿಸುತ್ತೇನೆ.

