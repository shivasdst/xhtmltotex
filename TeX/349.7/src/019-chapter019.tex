
\chapter[ಭಾರತದಲ್ಲಿ ಕಲೆ ಮತ್ತು ವಿಜ್ಞಾನ ]{ಭಾರತದಲ್ಲಿ ಕಲೆ ಮತ್ತು ವಿಜ್ಞಾನ \protect\footnote{\engfoot{C.W. Vol. IV, P. 196}}}

ಸ್ಯಾನ್​ಫ್ರಾನ್ಸ್​ಸಿಸ್ಕೊ ನಗರದ ಹಾಲಿನಲ್ಲಿ ಸ್ವಾಮಿ ವಿವೇಕಾನಂದರು ಭಾರತದಲ್ಲಿ ಕಲೆ ಮತ್ತು ವಿಜ್ಞಾನವೆಂಬ ವಿಷಯದ ಮೇಲೆ ಮಾತನಾಡುವರು ಎಂದು ಪರಿಚಯ ಮಾಡಿಸಿದರು. ಸಭಿಕರನ್ನೆಲ್ಲ ಸ್ವಾಮಿ ವಿವೇಕಾನಂದರು ತಮ್ಮ ಭಾಷಣದಿಂದ ಆಕರ್ಷಿಸಿದರು. ಸ್ವಾಮೀಜಿಯವರು ಉಪನ್ಯಾಸವಾದ ಮೇಲೆ ಸಭಿಕರು ಕೇಳಿದ ಹಲವು ಪ್ರಶ್ನೆಗಳು ಅವರು ತೋರಿದ ಉತ್ಸಾಹಕ್ಕೆ ಸಾಕ್ಷಿಯಾಗಿವೆ.

ಸ್ವಾಮೀಜಿಯವರು ಉಪನ್ಯಾಸದ ಕೆಲವು ಅಂಶಗಳು ಇವು: ರಾಷ್ಟ್ರಗಳ ಇತಿಹಾಸದ ಆದಿಯಲ್ಲಿ ಸರಕಾರವು ಯಾವಾಗಲೂ ಪುರೋಹಿತರ ಕೈಯಲ್ಲಿತ್ತು. ಎಲ್ಲಾ ಪಾಂಡಿತ್ಯವೂ ಪುರೋಹಿತರಿಂದ ಬರಬೇಕಾಗಿತ್ತು. ಪುರೋಹಿತರ ನಂತರ ಸರ್ಕಾರ ಬೇರೆಯವರ ಕೈಸೇರಿತು. ಅನಂತರ ಕ್ಷತ್ರಿಯರ ಆಳ್ವಿಕೆ ಆರಂಭವಾಗುವುದು. ಆಗ ಸೈನ್ಯದ ಶಕ್ತಿಯ ಆಡಳಿತವೇ ಜಯಪ್ರದವಾಗುವುದು. ಇದು ಯಾವಾಗಲೂ ನಿಜವಾಗಿದೆ. ಕೊನೆಗೆ\break ವಿಷಯಾಸಕ್ತಿಯಲ್ಲಿ ಜನರು ಪ್ರವೃತ್ತರಾಗುವರು. ಇದರ ಪರಿಣಾಮವಾಗಿ ಜನರು ಅವನತಿ ಹೊಂದುತ್ತಾರೆ. ಆಗ ಮತ್ತೂ ಬಲಿಷ್ಠವಾದ ಮತ್ತು ಹೆಚ್ಚು ಅನಾಗರಿಕವಾದ ಜನಾಂಗಗಳು ಅವರನ್ನು ಆಳುತ್ತಾರೆ. ಅರೆನಾಗರಿಕ ಜನಾಂಗಗಳಿಗೆ ಕ್ಷತ್ರಿಯ ಶಕ್ತಿ ಬಲಿಯಾಗುವುದು.

ಬಹಳ ಹಿಂದಿನ ಕಾಲದಿಂದಲೂ ಜಗತ್ತಿನ ಜನಾಂಗಗಳೆಲ್ಲ ಭರತಖಂಡವನ್ನು\break ಜ್ಞಾನಭೂಮಿ ಎಂದು ಕರೆಯುತ್ತಿದ್ದವು. ಇತರರನ್ನು ಗೆಲ್ಲುವುದಕ್ಕೆ ಭರತಖಂಡ ಹೊರಗೆ ಹೋಗಿಲ್ಲ. ಈ ಜನ ಹೋರಾಟಗಾರರಲ್ಲ. ನಿಮ್ಮ ಪಾಶ್ಚಾತ್ಯ ದೇಶದವರಂತೆ ಅವರು ಮಾಂಸಾಹಾರಿಗಳಲ್ಲ. ಮಾಂಸಾಹಾರ ಒಬ್ಬನನ್ನು ಹೋರಾಟಗಾರನನ್ನಾಗಿ ಮಾಡುವುದು. ಪ್ರಾಣಿಗಳ ರಕ್ತ ನಿಮ್ಮನ್ನು ಶಾಂತರಾಗಿರಲು ಬಿಡುವುದಿಲ್ಲ. ಏನನ್ನಾದರೂ ಮಾಡಬೇಕೆಂದು ನೀವು ಹಾತೊರೆಯುತ್ತೀರಿ.

ಎಲಿಜಬೆತ್ತಿನ ಕಾಲದ ಇಂಡಿಯ ಮತ್ತು ಇಂಗ್ಲೆಂಡುಗಳನ್ನು ಹೋಲಿಸಿ ನೋಡಿ ಆಗ ಅದು ನಿಮಗೆ (ಐರೋಪ್ಯರಿಗೆ) ಎಂತಹ ದುರ್ದಿನವಾಗಿತ್ತು. ಆಗ ನಾವು ತುಂಬಾ ಮುಂದುವರಿದಿದ್ದೆವು. ಆಂಗ್ಲೊಸ್ಯಾಕ್ಸನ್​ ಜನಾಂಗ ಯಾವಾಗಲೂ ಕಲೆಗೆ ಯೋಗ್ಯರಾಗಿರಲಿಲ್ಲ. ಅವರಲ್ಲಿ ಒಳ್ಳೆಯ ಕಾವ್ಯವಿದೆ. ಉದಾಹರಣೆಗೆ ಶೇಕ್ಸ್​ಪಿಯರ್​ ಕವಿಯ ಸರಳ\break ರಗಳೆಗಳು ಎಷ್ಟು ಸುಂದರವಾಗಿವೆ! ಪದಗಳಲ್ಲಿ ಪ್ರಾಸವಿರುವುದೇ ಶ್ರೇಷ್ಠವಲ್ಲ. ಈ\break ಪ್ರಪಂಚದಲ್ಲಿ ಇದೇನು ಅಷ್ಟು ನಾಗರಿಕವಾದುದಲ್ಲ.

ಇಂಡಿಯಾ ದೇಶದ ಸಂಗೀತದಲ್ಲಿ ಸಪ್ತಸ್ವರಗಳು ಚೆನ್ನಾಗಿ ಅಭಿವೃದ್ಧಿಯಾಗಿದ್ದವು. ಬಹು ಹಿಂದೆಯೇ ಅರ್ಧ ಕಾಲು ಸ್ವರಗಳು ಕೂಡ ಅಭಿವೃದ್ಧಿಯಾಗಿದ್ದವು. ಇಂಡಿಯಾ ದೇಶ ಸಂಗೀತದಲ್ಲಿ, ನಾಟಕದಲ್ಲಿ, ಶಿಲ್ಪಕಲೆಗಳಲ್ಲಿಯೂ ಮುಂದುವರಿದಿತ್ತು. ಈಗ ಏನಾದರೂ ಅದೆಲ್ಲ ಬರೀ ಅನುಕರಣೆ. ಈಗ ಭರತ ಖಂಡದಲ್ಲಿ ಪ್ರತಿಯೊಂದೂ ಒಬ್ಬ ವ್ಯಕ್ತಿಗೆ ಬದುಕಲು ನಮಗೆ ಎಷ್ಟು ಕಡಮೆ ಸಾಕು ಎಂಬ ಭಾವನೆಯ ಮೇಲೆ ನಿಂತಿದೆ.

