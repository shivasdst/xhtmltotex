
\chapter[ನಿಃಸ್ವಾರ್ಥ ಸೇವೆಯೇ ನಿಜವಾದ ತ್ಯಾಗ ]{ನಿಃಸ್ವಾರ್ಥ ಸೇವೆಯೇ ನಿಜವಾದ ತ್ಯಾಗ \protect\footnote{\engfoot{C.W. Vol. VI. p83}}}

ಪ್ರಪಂಚ ಹೇಡಿಗಳಿಗೆ ಅಲ್ಲ, ಓಡಿಹೋಗಲೆತ್ನಿಸಬೇಡಿ. ಸೋಲು ಗೆಲುವುಗಳನ್ನು ಗಣನೆಗೆ ತರಬೇಡಿ. ನಿಃಸ್ವಾರ್ಥರಾಗಿ ಕೆಲಸ ಮಾಡಿ. ಯಾರು ಗೆಲ್ಲುವುದಕ್ಕಾಗಿ ಜನ್ಮತಾಳಿರುವರೋ ಅವರು ಸತ್ಯ ಸಂಕಲ್ಪ ಮಾಡಿಕೊಂಡು ಎಂದಿಗೂ ಅದರಿಂದ ವಿಚಲಿತರಾಗುವುದಿಲ್ಲ. ನಿಮಗೆ ಕೆಲಸ ಮಾಡಲು ಅಧಿಕಾರವಿದೆ. ಆದರೆ ಕರ್ಮಫಲ ವನ್ನು ಇಚ್ಛಿಸುವ ಕೀಳುಸ್ಥಿತಿಗೆ ಇಳಿಯಬೇಡಿ. ಬಿಡುವಿಲ್ಲದೆ ಕರ್ಮದಲ್ಲಿ ನಿರತರಾಗಿ. ಆದರೆ ಕೆಲಸದ ಹಿಂದೆ ನಿಷ್ಕಾಮ ದೃಷ್ಟಿ ಇರಲಿ. ಸತ್ಯರ್ಮಗಳು ಕೂಡ ಕೆಲವು ವೇಳೆ ಮಹಾಬಂಧನಕ್ಕೆ ಕಾರಣವಾಗುವುವು. ಆದಕಾರಣ ಸತ್ಕರ್ಮಗಳ ಬಂಧನಕ್ಕೆ ಅಥವಾ ಕೀರ್ತಿಯ ಹೆಸರಿನ ಆಸೆಗೆ ಬಲಿಯಾಗಬೇಡಿ. ಯಾರು ಈ ರಹಸ್ಯವನ್ನು ಅರಿತಿರುವರೊ ಅವರು ಜನನ ಮರಣಗಳಿಂದ ಪಾರಾಗಿ ಅಮೃತಾತ್ಮರಾಗುವರು.

ಸಾಧಾರಣ ಸಂನ್ಯಾಸಿ ಪ್ರಪಂಚವನ್ನು ತ್ಯಜಿಸಿ ಹೊರಗೆ ಹೋಗಿ ಭಗವಂತನನ್ನು ಚಿಂತಿಸುವನು. ಆದರೆ ನಿಜವಾದ ಸಂನ್ಯಾಸಿ ಪ್ರಪಂಚದಲ್ಲಿರುವನು; ಆದರೆ ಪ್ರಪಂಚಕ್ಕೆ ಸೇರಿರುವುದಿಲ್ಲ. ತಮ್ಮ ಬಯಕೆಯನ್ನು ತೃಪ್ತಿಪಡಿಸಿಕೊಂಡಿಲ್ಲದವರು ಕಾಡಿನಲ್ಲಿದ್ದೂ ಪ್ರಾಪಂಚಿಕ ಆಸೆಯನ್ನು ಮೆಲುಕು ಹಾಕುತ್ತಿರುವರು. ಅವರು ನಿಜವಾದ ಸಂನ್ಯಾಸಿಗಳಲ್ಲ. ಜೀವನದ ಹೋರಾಟದ ಮಧ್ಯದಲ್ಲಿರಿ. ನಿದ್ರಿಸುವಾಗ ಅಥವಾ ಒಂದು ಗುಹೆಯಲ್ಲಿ ಇರುವಾಗ ಯಾರು ಬೇಕಾದರೂ ಶಾಂತಚಿತ್ತ ರಾಗಿರಬಹುದು. ಉನ್ಮತ್ತ ಕರ್ಮದ ಪ್ರಚಂಡ ಸುಂಟರಗಾಳಿಯಲ್ಲಿದ್ದೂ ಕೇಂದ್ರ ವನ್ನು ಸೇರಬೇಕು. ನೀವು ಕೇಂದ್ರವನ್ನು ಕಂಡಿದ್ದರೆ ವಿಚಲಿತರಾಗುವುದಿಲ್ಲ.

