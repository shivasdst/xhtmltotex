
\chapter[ಭಾರತ ಮಹಿಳೆ ]{ಭಾರತ ಮಹಿಳೆ \protect\footnote{\engfoot{C.W. Vol. VIII, P. 54}}}

\centerline{\textbf{(೧೯೦೦ರ ಜನವರಿ ೧೮ರಂದು ಕ್ಯಾಲಿಫೋರ್ನಿಯಾದ ಪಸದೆನಾದಲ್ಲಿ ನೀಡಿದ ಉಪನ್ಯಾಸ)}}

\vskip 0.3cm

ನಾನು ಮಾತನಾಡುವುದಕ್ಕೆ ಮುಂಚೆ ನಿಮ್ಮ ಸಹಿಷ್ಣುತೆಯನ್ನು ಕೋರುತ್ತೇನೆ. ನಾನು ಮದುವೆಯಾಗದವರ ಪಂಥಕ್ಕೆ ಸೇರಿದವನು. ತಾಯಿ, ಸತಿ, ಮಗಳು ಮತ್ತು ಸೋದರಿ ಇವರ ಸ್ಥಾನದಲ್ಲಿರುವ ಸ್ತ್ರೀಯರ ವಿಷಯದಲ್ಲಿ ನನಗಿರುವ ಅನುಭವ ಇತರರಿಗೆ\break ಇರುವಷ್ಟು ಪೂರ್ಣವಾಗಿ ಇಲ್ಲದೆ ಇರಬಹುದು. ಇಂಡಿಯಾ ದೇಶ ದೊಡ್ಡದೊಂದು ಖಂಡ. ಅದು ಬರಿಯ ಒಂದು ದೇಶವಲ್ಲ; ಅಲ್ಲಿ ಹಲವಾರು ಜನಾಂಗಗಳಿವೆ. ಇದನ್ನು ನೀವು ಜ್ಞಾಪಕದಲ್ಲಿಡಬೇಕಾಗಿದೆ. ಯೂರೋಪಿನ ಜನಾಂಗಗಳು ಇಂಡಿಯಾದ ಜನಾಂಗ\-ಗಳಿಗಿಂತ ಪರಸ್ಪರ ಹೆಚ್ಚು ಸಮೀಪದಲ್ಲಿವೆ, ಅವುಗಳಲ್ಲಿ ಹೆಚ್ಚು ಸಾಮ್ಯವಿದೆ. ಭರತಖಂಡದಲ್ಲೆಲ್ಲಾ ಮುಖ್ಯವಾದ ಎಂಟು ಭಾಷೆಗಳಿವೆ ಎಂದರೆ ನಿಮಗೆ ಈ ವಿಷಯದ ಸ್ಥೂಲ ಪರಿಚಯವಾಗಬಹುದು. ಇವೆಲ್ಲ ಬೇರೆ ಬೇರೆ ಭಾಷೆಗಳು; ಬರಿಯ ಉಪಭಾಷೆಗಳಲ್ಲ. ಪ್ರತಿಯೊಂದು ಭಾಷೆಗೂ ತನ್ನದೇ ಆದ ಒಂದು ಸಾಹಿತ್ಯ ಮತ್ತು ಒಂದು ಇತಿಹಾಸವಿದೆ. ಹಿಂದಿಯನ್ನು ಸುಮಾರು ಹತ್ತು ಕೋಟಿ ಜನರು ಮಾತನಾಡುವರು. ಬಂಗಾಳಿಯನ್ನು ಸುಮಾರು ಆರು ಕೋಟಿ ಜನರು ಮಾತನಾಡುವರು. ಹೀಗೆಯೇ ಇತರ ಭಾಷೆಗಳು. ಉತ್ತರದಲ್ಲಿರುವ\break ನಾಲ್ಕು ಭಾಷೆಗಳಿಗೂ ದಕ್ಷಿಣದಲ್ಲಿರುವ ನಾಲ್ಕು ಭಾಷೆಗಳಿಗೂ ಇರುವ ವ್ಯತ್ಯಾಸ\break ಯುರೋಪಿನಲ್ಲಿ ಎರಡು ಭಾಷೆಗಳಿಗೆ ಇರುವ ವ್ಯತ್ಯಾಸಕ್ಕಿಂತ ಹೆಚ್ಚು. ಅವೆಲ್ಲ ಬೇರೆ ಬೇರೆ ಭಾಷೆಗಳು. ನಿಮ್ಮ ಭಾಷೆಗೂ ಜಪಾನಿ ಭಾಷೆಗೂ ಇರುವಷ್ಟು ವ್ಯತ್ಯಾಸ ಅವುಗಳಲ್ಲಿ ಒಂದಕ್ಕೂ ಇನ್ನೊಂದಕ್ಕೂ ಇದೆ. ನಾನು ದಕ್ಷಿಣ ಇಂಡಿಯಾಕ್ಕೆ ಹೋದರೆ, ಸಂಸ್ಕೃತ ಮಾತನಾಡುವವರು ಸಿಕ್ಕದೆ ಇದ್ದರೆ, ಇಂಗ್ಲಿಷ್​ ಮಾತನಾಡಬೇಕಾಗುವುದೆಂದು ಕೇಳಿದಾಗ ನಿಮಗೆ ಆಶ್ಚರ್ಯವಾಗಬಹುದು. ಆಚಾರ, ವ್ಯವಹಾರ, ಉಡಿಗೆ, ತೊಡಿಗೆ, ಆಹಾರ, ಚಿಂತನಾ ವಿಧಾನ ಇವುಗಳಲ್ಲಿ ಒಂದು ಜನಾಂಗಕ್ಕೂ ಮತ್ತೊಂದು ಜನಾಂಗಕ್ಕೂ ಎಷ್ಟೋ ಭಿನ್ನತೆಗಳು ಇವೆ.

ಅನಂತರ ಜಾತಿ ಇದೆ. ಪ್ರತಿಯೊಂದು ಜಾತಿಯೂ ಬೇರೆಯೇ ಒಂದು ಜನಾಂಗವಾಗಿದೆ. ಒಬ್ಬನು ಭರತಖಂಡದಲ್ಲಿ ಬಹಳ ದಿನಗಳಿದ್ದರೆ ಜನರ ಲಕ್ಷಣಗಳನ್ನು ನೋಡಿ ಅವರು ಯಾವ ಜಾತಿಗೆ ಸೇರಿದವರೆಂಬುದನ್ನು ಹೇಳಬಹುದು. ಒಂದೇ ಜಾತಿಯಲ್ಲಿ ಕೂಡ ಆಚಾರ ವ್ಯವಹಾರಗಳಲ್ಲಿ ಎಷ್ಟೋ ಭೇದವಿದೆ. ಈ ಜಾತಿಗಳೆಲ್ಲ ಪ್ರತ್ಯೇಕ ಪ್ರತ್ಯೇಕವಾಗಿವೆ. ಸಾಮಾಜಿಕವಾಗಿ ಒಟ್ಟಿಗೆ ಕಲೆಯುವರು. ಆದರೆ ಸಹಪಂಕ್ತಿ ಇಲ್ಲ, ಹೆಣ್ಣು ಕೊಡುವ ಅಥವಾ ತರುವ ವಾಡಿಕೆ ಇಲ್ಲ. ಈ ವಿಷಯದಲ್ಲಿ ಅವರು ಸಂಪೂರ್ಣ ಪ್ರತ್ಯೇಕ ಪ್ರತ್ಯೇಕವಾಗಿರುವರು. ಒಬ್ಬರಿಗೂ ಮತ್ತೊಬ್ಬರಿಗೂ ವ್ಯವಹಾರವಿರುವುದು, ಸ್ನೇಹವಿರುವುದು, ಅಲ್ಲಿಗೆ ಕೊನೆ.

ಸಾಧಾರಣ ಮನುಷ್ಯರು ಸ್ತ್ರೀಯರ ವಿಷಯವನ್ನು ತಿಳಿದುಕೊಳ್ಳುವುದಕ್ಕಿಂತ ನನಗೆ ಹೆಚ್ಚು ಅವಕಾಶವಿದೆ. ಏಕೆಂದರೆ ನಾನು ಬೋಧಕನಾಗಿರುವುದರಿಂದ ಊರಿಂದೂರಿಗೆ ಸದಾ ಸಂಚರಿಸುತ್ತ ಎಲ್ಲ ಬಗೆಯ ಜನರೊಡನೆ ಬೇರೆಯಬೇಕಾಗುವುದು. ಉತ್ತರ ಭಾರತದಲ್ಲಿ ಪರಪುರುಷನೆದುರಿಗೆ ಸ್ತ್ರೀ ಬಾರದೇ ಇದ್ದರೂ, ನಮ್ಮ ಬೋಧನೆಯನ್ನು ಕೇಳುವುದಕ್ಕೆ ಮತ್ತು ನಮ್ಮೊಡನೆ ಮಾತನಾಡುವುದಕ್ಕೆ ಬರುವರು. ಆದರೂ ಕೂಡ ಭಾರತದ ಮಹಿಳೆಯರ ವಿಷಯವನ್ನೆಲ್ಲ ನಾನು ತಿಳಿದುಕೊಂಡಿರುವೆನು ಎಂದರೆ ತಪ್ಪಾಗಬಹುದು.

ಆದರ್ಶವನ್ನು ನಿಮ್ಮ ಮುಂದಿಡಲು ಯತ್ನಿಸುತ್ತೇನೆ. ಪ್ರತಿಯೊಂದು ಜನಾಂಗ\break ದಲ್ಲಿಯೂ ಅಲ್ಲಿಯ ಸ್ತ್ರೀಪುರುಷರು ಒಂದು ಆದರ್ಶವನ್ನು ಪ್ರತಿನಿಧಿಸುವರು. ಅದನ್ನು\break ತಿಳಿದೋ, ತಿಳಿಯದೆಯೋ, ಅನುಷ್ಠಾನಕ್ಕೆ ತರುತ್ತಿರುವರು. ವ್ಯಕ್ತಿಯು ಒಂದು ಆದರ್ಶದ\break ಬಾಹ್ಯ ಅಭಿವ್ಯಕ್ತಿ. ಇಂತಹ ವ್ಯಕ್ತಿಗಳ ಸಮುದಾಯವೇ ಒಂದು ಜನಾಂಗ. ಈ ಜನಾಂಗ ಒಂದು ಉದಾತ್ತ ಆದರ್ಶದ ಪ್ರತಿನಿಧಿ; ಅದರೆಡೆಗೆ ಇದು ಸಂಚರಿಸುತ್ತಿದೆ. ಆದಕಾರಣವೇ, ನಾವು ಒಂದು ಜನಾಂಗವನ್ನು ತಿಳಿದುಕೊಳ್ಳಬೇಕಾದರೆ, ಆ ಜನಾಂಗದ ಆದರ್ಶವನ್ನು ತಿಳಿದುಕೊಳ್ಳಬೇಕೆಂದು ನ್ಯಾಯವಾಗಿಯೇ ಹೇಳುವರು. ಪ್ರತಿಯೊಂದು ಜನಾಂಗವನ್ನು ಅದರ ಆದರ್ಶದಿಂದ ಮಾತ್ರ ಪರೀಕ್ಷಿಸಬಹುದೇ ಹೊರತು ಅನ್ಯಥಾ ಅಲ್ಲ.

ಎಲ್ಲಾ ಬೆಳವಣಿಗೆ, ಅಭಿವೃದ್ಧಿ, ಉನ್ನತಿ, ಅವನತಿ ಇವು ಕೇವಲ ಸಾಪೇಕ್ಷ ಶಬ್ದಗಳು. ಇವು ಒಂದು ಆದರ್ಶವನ್ನು ಸೂಚಿಸುವುವು. ನಾವು ತಿಳಿದುಕೊಳ್ಳಬೇಕೆಂದಿರುವ\break ವ್ಯಕ್ತಿಯನ್ನು ಈ ಆದರ್ಶದೊಂದಿಗೆ ಹೋಲಿಸಬೇಕಾಗಿದೆ. ದೇಶಗಳ ವಿಷಯದಲ್ಲಿ\break ಇದು ಹೆಚ್ಚು ಸ್ಪಷ್ಟವಾಗಿ ಕಂಡುಬರುತ್ತದೆ. ಯಾವುದನ್ನು ಒಂದು ದೇಶ ಒಳ್ಳೆಯದೆಂದು ಭಾವಿಸುವುದೋ, ಅದನ್ನು ಮತ್ತೊಂದು ದೇಶ ಹಾಗೆ ಭಾವಿಸುವುದಿಲ್ಲ. ಈ ದೇಶದಲ್ಲಿ\break ಸೋದರಮಾವನ ಮಗನನ್ನು ಅಥವಾ ಮಗಳನ್ನು ಮದುವೆಯಾಗುವುದಕ್ಕೆ ಯಾವ\break ಅಭ್ಯಂತರವೂ ಇಲ್ಲ. ಆದರೆ ಇಂಡಿಯಾ ದೇಶದ ಕೆಲವು ಕಡೆಗಳಲ್ಲಿ ಇದು ನ್ಯಾಯಕ್ಕೆ ವಿರುದ್ಧ. ಇದನ್ನು ಮಹಾಪಾತಕವೆಂದು ಪರಿಗಣಿಸುವರು. ಈ ದೇಶದಲ್ಲಿ ವಿಧವಾ\break ವಿವಾಹ ನ್ಯಾಯಬದ್ಧವಾಗಿದೆ. ಆದರೆ ಹಿಂದೂ ಸಮಾಜದ ಮೇಲಿನ ವರ್ಗದವರಲ್ಲಿ ಹೆಂಗಸು ಹಾಗೆ ವಿವಾಹ ಮಾಡಿಕೊಳ್ಳುವುದು ನಿಂದಾಸ್ಪದ. ಪ್ರತಿಯೊಬ್ಬರಿಗೂ ಬೇರೆ ಬೇರೆ ಆದರ್ಶಗಳಿವೆ. ಒಬ್ಬರನ್ನು ಮತ್ತೊಬ್ಬರ ಆದರ್ಶದಿಂದ ಅಳೆಯುವುದು ಸಾಧುವೂ ಅಲ್ಲ, ಸಾಧ್ಯವೂ ಅಲ್ಲ. ಆದಕಾರಣ, ಮೊದಲು ಒಂದು ರಾಷ್ಟ್ರವು ಯಾವ ಆದರ್ಶವನ್ನು ಇಟ್ಟುಕೊಂಡಿದೆ ಎಂಬುದನ್ನು ನೋಡಬೇಕಾಗಿದೆ. ಬೇರೆ ಬೇರೆ ರಾಷ್ಟ್ರಗಳನ್ನು ಕುರಿತು\break ನಾವು ಮಾತನಾಡುವಾಗ ಎಲ್ಲರಿಗೂ ಒಂದು ಸರ್ವಸಾಮಾನ್ಯ ನೀತಿ ಮತ್ತು ಒಂದೇ\break ಬಗೆಯ ಆದರ್ಶವಿದೆ ಎಂದು ಭಾವಿಸುವೆವು. ಮತ್ತೊಬ್ಬರನ್ನು ನಾವು ಪರೀಕ್ಷಿಸುವಾಗ ಯಾವುದು ನಮಗೆ ಒಳ್ಳೆಯದೋ ಅದು ಇತರರಿಗೂ ಒಳ್ಳೆಯದಾಗಿರಬೇಕು, ನಾವು\break ಮಾಡುವುದೇ ಸರಿ, ನಾವು ಯಾವುದನ್ನು ಮಾಡುವುದಿಲ್ಲವೋ ಅದು ಇತರರಿಗೂ\break ನಿಷೇಧ ಎಂದು ಭಾವಿಸುವೆವು. ನಾನು ಇದನ್ನು ಟೀಕಿಸುತ್ತಿಲ್ಲ. ಸತ್ಯವನ್ನು ವ್ಯಕ್ತಪಡಿಸು\-ವುದಕ್ಕೆ ಪ್ರಯತ್ನಿಸುತ್ತಿರುವೆ, ಅಷ್ಟೆ. ಪಾಶ್ಚಾತ್ಯರು ಚೀನಿ ಹೆಂಗಸರ ಪಾದಗಳನ್ನು ನೋಡಿ\break ನಗುವಾಗ ತಮ್ಮದೇ ಸ್ತ್ರೀಯರ ಒಳಕುಪ್ಪಸದಿಂದ ಜನಾಂಗಕ್ಕೆ ಎಷ್ಟು ಹಾನಿ ಇದೆ\break ಎಂಬುದನ್ನು ಭಾವಿಸುವುದೇ ಇಲ್ಲ. ಇದು ಕೇವಲ ಒಂದು ಉದಾಹರಣೆ ಮಾತ್ರ.\break ಪಾದಗಳನ್ನು ಸಣ್ಣದಾಗಿ ಮಾಡುವುದರಿಂದ ಆಗುವ ಅಪಾಯ, ಬಿಗಿಯಾದ ಒಳಕುಪ್ಪಸದಿಂದ ಇಡಿಯ ದೇಹಕ್ಕೆ ಆಗುತ್ತಿರುವ ಅಪಾಯದ ಲಕ್ಷಪಾಲುಗಳಲ್ಲಿ ಒಂದು ಪಾಲಿಲ್ಲ.\break ಇದರಿಂದ ದೇಹದ ಅಂಗೋಪಾಂಗಗಳು ಸಂಕುಚಿತವಾಗಿ ಬೆನ್ನೆಲುಬು ಹಾವಿನಂತೆ\break ವಕ್ರವಾಗುವುದು. ಅಳತೆಯನ್ನು ತೆಗೆದುಕೊಳ್ಳುವಾಗ ಇವು ಎಷ್ಟು ಬಾಗಿವೆ ಎಂಬುದು ಗೊತ್ತಾಗುವುದು. ನಾನು ಇದನ್ನು ನಿಂದಾದೃಷ್ಟಿಯಿಂದ ಹೇಳುತ್ತಿಲ್ಲ. ಆದರೆ ನೀವು\break ನಿಮ್ಮ ಹೆಂಗಸರೇ ಶ್ರೇಷ್ಠ, ಉಳಿದವರು ಕೀಳು, ಏಕೆಂದರೆ ಅವರು ನಿಮ್ಮ ಆಚಾರ\break ವ್ಯವಹಾರಗಳನ್ನು ಅನುಸರಿಸುತ್ತಿಲ್ಲವೆಂದು ಹೇಗೆ ಭಾವಿಸುತ್ತೀರೋ, ಹಾಗೆಯೇ ಅವರೂ ನಿಮ್ಮನ್ನು ಅದೇ ದೃಷ್ಟಿಯಿಂದ ನೋಡುವರು.

ಇಬ್ಬರಲ್ಲಿಯೂ ಪರಸ್ಪರ ತಪ್ಪು ತಿಳುವಳಿಕೆ ಇದೆ. ಒಂದು ಸರ್ವೇಸಾಮಾನ್ಯ ವಾಡಿಕೆ ಇದೆ, ತಿಳುವಳಿಕೆಯ ಒಂದು ಸಾಮಾನ್ಯ ನೆಲೆಯಿದೆ, ಒಂದು ಸಾಮಾನ್ಯ ಮಾನವಕೋಟಿ ಇದೆ. ಇದೇ ನಮ್ಮ ಕಾರ್ಯದ ತಳಹದಿಯಾಗಬೇಕು. ಎಲ್ಲೋ ಒಂದೊಂದು ಕಡೆ ಮಾತ್ರ ಅಲ್ಪ ಸ್ವಲ್ಪ ಕಾಣುತ್ತಿರುವ ಮಾನವ ಸ್ವಭಾವದ ಪೂರ್ಣತೆಯನ್ನು ನಾವು ಹುಡುಕಬೇಕಾಗಿದೆ. ಒಬ್ಬನೆ ಪ್ರತಿಯೊಂದರಲ್ಲಿಯೂ ಪೂರ್ಣತೆಯನ್ನು ಪಡೆಯಲಾರ. ನೀವು ಒಂದು ಕೆಲಸವನ್ನು ಮಾಡಬೇಕಾಗಿದೆ. ನಾನು ನನ್ನ ಪಾಲಿಗೆ ಬಂದ ಕೆಲಸವನ್ನು ದೈನ್ಯದಿಂದ ಮಾಡಬೇಕಾಗಿದೆ. ಒಬ್ಬೊಬ್ಬರು ಒಂದೊಂದು ಪಾತ್ರವನ್ನು ವಹಿಸಬೇಕಾಗಿದೆ. ಈ ಅಂಶಗಳ ಮೊತ್ತವೇ ಪೂರ್ಣತೆ. ವ್ಯಕ್ತಿ ಹೇಗೋ ಹಾಗೆಯೇ ಜನಾಂಗ.\break ಪ್ರತಿಯೊಂದು ಜನಾಂಗವೂ ಒಂದು ಕೆಲಸವನ್ನು ಮಾಡಬೇಕಾಗಿದೆ. ಪ್ರತಿಯೊಂದು ಜನಾಂಗವೂ ಮಾನವ ಸ್ವಭಾವದ ಒಂದು ಭಾಗವನ್ನು ಅಭಿವೃದ್ಧಿಗೊಳಿಸಬೇಕಾಗಿದೆ. ಬಹುಶಃ ಭವಿಷ್ಯದಲ್ಲಿ ಪ್ರತಿಯೊಂದು ಅದ್ಭುತ ಜನಾಂಗವೂ ಯಾವುದನ್ನು ಪೂರ್ಣಗೊಳಿಸಿದೆಯೋ ಇದನ್ನೆಲ್ಲವನ್ನು ಹೀರಿಕೊಂಡು ಬೇರೊಂದು ಜನಾಂಗ ಉದಯಿಸಬಹುದು. ಇಂತಹ ಜನಾಂಗವನ್ನು ಪ್ರಪಂಚ ಇದುವರೆವಿಗೂ ಕನಸಿನಲ್ಲಿಯೂ ನೋಡದೆ ಇರಬಹುದು. ಇಷ್ಟು ಹೇಳುವುದಲ್ಲದೆ, ಯಾರನ್ನು ಕುರಿತೂ ನಾನು ವಿಮರ್ಶೆ ಮಾಡಲಾರೆ. ನಾನು ಜೀವನದಲ್ಲಿ ಕಡಿಮೆ ಸಂಚಾರಮಾಡಿಲ್ಲ. ಕಣ್ಣನ್ನು ತೆರೆದಿಟ್ಟು ಕಂಡಿರುವೆನು. ಹೆಚ್ಚು ಸಂಚಾರಮಾಡಿ ನೋಡಿದಷ್ಟೂ ಮೂಕನಾಗಬೇಕಾಗಿದೆ. ನಾನು ಯಾರನ್ನೂ ದೂರುವುದಿಲ್ಲ.

ಭರತಖಂಡದಲ್ಲಿ ಆದರ್ಶ ಮಹಿಳೆ ತಾಯಿ. ಮಹಿಳೆ ಮೊದಲು ತಾಯಿ, ಅನಂತರವೂ ತಾಯಿಯೇ. ಹಿಂದೂ ಮನಸ್ಸಿನಲ್ಲಿ ಮಹಿಳೆ ಎಂಬ ಶಬ್ದ ತಾಯಿತನವನ್ನು ನೆನಪಿಗೆ\break ತರುವುದು. ದೇವರನ್ನು ತಾಯಿ ಎಂದು ಕರೆಯುವರು. ನಾವು ಹುಡುಗರಾಗಿದ್ದಾಗ\break ಪ್ರತಿದಿನ ಬೆಳಗ್ಗೆ ಒಂದು ಬಟ್ಟಲು ನೀರನ್ನು ತಾಯಿಯ ಹತ್ತಿರ ತೆಗೆದುಕೊಂಡು\break ಹೋಗುತ್ತಿದ್ದೆವು. ಅವಳು ತನ್ನ ಕಾಲಿನ ಹೆಬ್ಬೆಟ್ಟನ್ನು ಅದರಲ್ಲಿ ಅದ್ದಿದ ಮೇಲೆ ನಾವು\break ಅದನ್ನು ಕುಡಿಯುತ್ತಿದ್ದೆವು.

ಪಾಶ್ಚಾತ್ಯರಲ್ಲಿ ಮಹಿಳೆ ಸತಿ. ಸ್ತ್ರೀತ್ವದ ಆದರ್ಶ ಸತಿಯಲ್ಲಿ ಕೇಂದ್ರೀಕೃತವಾಗಿದೆ. ಭರತಖಂಡದಲ್ಲಿ ಸಾಧಾರಣ ಮನುಷ್ಯನಿಗೆ ಸ್ತ್ರೀತ್ವದ ಶಕ್ತಿಯೆಲ್ಲಾ ಮಾತೃತ್ವದಲ್ಲಿ ಕೇಂದ್ರಿಕೃತವಾಗಿದೆ. ಪಾಶ್ಚಾತ್ಯ ಗೃಹದಲ್ಲಿ ಸತಿ ಆಳುವಳು. ಹಿಂದೂ ಗೃಹದಲ್ಲಿ ತಾಯಿ ಆಳುವಳು. ಪಾಶ್ಚಾತ್ಯರ ಮನೆಗೆ ತಾಯಿ ಬಂದರೆ ಆಕೆ ತನ್ನ ಸೊಸೆಯ ಅಧೀನದಲ್ಲಿರಬೇಕಾಗಿದೆ. ಇಲ್ಲಿ ಮನೆ ಸೊಸೆಗೆ ಸೇರಿದ್ದು. ಭರತ ಖಂಡದಲ್ಲಿ ತಾಯಿ ಯಾವಾಗಲೂ ಮನೆಯಲ್ಲಿರುತ್ತಾಳೆ. ಹೆಂಡತಿ ತಾಯಿಯ ಅಧೀನದಲ್ಲಿರಬೇಕು. ನೋಡಿ ಆದರ್ಶಗಳ ಭಿನ್ನತೆಯನ್ನು.

ನಾನು ಕೆಲವು ಹೋಲಿಕೆಗಳನ್ನು ಮಾತ್ರ ಸೂಚಿಸುತ್ತೇನೆ. ಕೆಲವು ವಿಷಯಗಳನ್ನು ಮಾತ್ರ ತಿಳಿಸುತ್ತೇನೆ. ಇದರಿಂದ ಇವೆರಡನ್ನೂ ಹೋಲಿಸಬಹುದು. ನೀವು “ಭಾರತ ಮಹಿಳೆಯ ಸತಿ ಸ್ಥಾನವೆಲ್ಲಿ?” ಎಂದು ಪ್ರಶ್ನಿಸಿದರೆ, ಭಾರತೀಯನು “ಅಮೆರಿಕಾ ನಾರಿಯ ಮಾತೃ ಸ್ಥಾನವೆಲ್ಲಿ?” ಎನ್ನುವನು. “ನಮಗೆ ಜನ್ಮಕೊಟ್ಟ ಮಹಾ ಮಾತೆಯೆಲ್ಲಿ? ನವಮಾಸಗಳವರೆಗೆ ನಮ್ಮನ್ನು ತನ್ನ ದೇಹದಲ್ಲಿ ಪೋಷಿಸಿದವಳೆಲ್ಲಿ? ನಮಗೆ ಅವಶ್ಯಕವಾದರೆ ಇಪ್ಪತ್ತು\break ವೇಳೆಯಾದರೂ ತನ್ನ ಜೀವನವನ್ನು ತೆರಲು ಸಿದ್ಧಳಾಗಿರುವ ಆಕೆಯ ಸ್ಥಾನವೆಂತಹುದು? ನಾವು ಎಷ್ಟು ಪಾಪಿಗಳಾದರೂ, ದುರಾತ್ಮರಾದರೂ, ಯಾರ ಪ್ರೇಮಚಿಲುಮೆ ಎಂದೂ ಬತ್ತುವುದಿಲ್ಲವೋ ಅವಳೆಲ್ಲಿ? ಯಾರೊಂದಿಗೆ ಹೋಲಿಸಿದರೆ, ಸ್ವಲ್ಪ ಒರಟಾಗಿ ಮಾತನಾಡಿದರೂ ದಾಂಪತ್ಯ ವಿಚ್ಛೇದನಕ್ಕಾಗಿ ನ್ಯಾಯಸ್ಥಾನಕ್ಕೆ ನುಗ್ಗುವ ನಿಮ್ಮ ಹೆಂಗಸಿನ ಸ್ಥಾನವೆಲ್ಲಿ? ಅಮೆರಿಕಾ ಮಹಿಳೆಯರೆ! ಅವಳ ಸ್ಥಾನವೆಲ್ಲಿ?” ಎಂದು ಪ್ರಶ್ನಿಸುವನು. ಆಕೆ ನಿಮ್ಮ ದೇಶದಲ್ಲಿ ಕಾಣಿಸುತ್ತಿಲ್ಲವಲ್ಲ! ಮೊದಲು ತಾಯಿಯನ್ನು ಗೌರವಿಸುವ ಮಗನನ್ನು ನಾನು ಇನ್ನೂ ಇಲ್ಲಿ ಕಂಡಿಲ್ಲ. ನಾವು ಸಾಯುವಾಗಲೂ ಹೆಂಡತಿ ಮಕ್ಕಳು ತಾಯಿಯ ಸ್ಥಾನವನ್ನು ಆಕ್ರಮಿಸುವುದನ್ನು ಇಚ್ಛಿಸುವುದಿಲ್ಲ. ನಮ್ಮ ತಾಯಿಯ ಕಣ್ಣಮುಂದೆ ನಾವು ಕಾಲವಾದರೆ ಆಕೆಯ ತೊಡೆಯ ಮೇಲೆ ತಲೆ ಇಟ್ಟು ಕೊನೆಯುಸಿರನ್ನು ಬಿಡಲು ಇಚ್ಛಿಸುವೆವು. ಎಲ್ಲಿ ಅಂತಹ ತಾಯಿ? ಹೆಂಗಸು ಎಂಬ ಹೆಸರು ಕೇವಲ ದೇಹಕ್ಕೆ ಮಾತ್ರ ಸಂಬಂಧಿಸಿರುವುದೆ?\break ಭಗವಂತ! ಮಾಂಸವು ಮಾಂಸದಾಕರ್ಷಣೆಗೆ ಸಿಕ್ಕಬೇಕೆಂಬ ಆದರ್ಶವನ್ನು ಕಂಡರೆ ಹಿಂದೂ ಕಂಪಿಸುವನು. ಇಲ್ಲ! ಇಲ್ಲ! ಮಹಿಳೆಯೆ, ದೇಹಕ್ಕೆ ಸಂಬಂಧಪಟ್ಟ ಯಾವುದರೊಂದಿಗೂ ನೀನು ಅನ್ವಯಿಸುವುದಿಲ್ಲ. ನಿನ್ನ ಹೆಸರು ಪರಮ ಪವಿತ್ರವೆಂದು ಎಂದೋ ಸಾರಿದರು. ಕಾಮ ಯಾವುದನ್ನು ಸಮೀಪಿಸಲಾರದೋ ಅದು ‘ತಾಯಿ’ ಎಂಬ ಒಂದು ಪದವಲ್ಲದೆ ಮತ್ತಾವುದು? ಭಾರತದ ಆದರ್ಶವೇ ಇದು.

ಕ್ಯಾಥೊಲಿಕ್​ ಪಂಗಡಕ್ಕೆ ಸೇರಿದ ಫಕೀರರ ಗುಂಪನ್ನು ಹೋಲುವ ಸಂಘಕ್ಕೆ ಸೇರಿದವನು\break ನಾನು. ಬಟ್ಟೆಬರೆಗಳ ಮೇಲೆ ಆಸೆ ಇಲ್ಲದೆ ಮನೆಯಿಂದ ಮನೆಗೆ ಹೋಗಿ, ಭಿಕ್ಷವನ್ನೆತ್ತಿ\break ಜನರಿಗೆ ಬೇಕಾದಾಗ ಬೋಧನೆ ಮಾಡಿ, ಸ್ಥಳ ಸಿಕ್ಕಿದೆಡೆ ತಂಗುವುದು ನಮ್ಮ ಜೀವನದ ನಿಯಮ. ಈ ಪಂಗಡಕ್ಕೆ ಸೇರಿದವರೆಲ್ಲರೂ ಸ್ತ್ರೀಯರನ್ನು ತಾಯಿ ಎಂದು ಕರೆಯಬೇಕು. ಪ್ರತಿಯೊಬ್ಬ ಸ್ತ್ರೀಯನ್ನೂ, ಸಣ್ಣ ಹುಡುಗಿಯನ್ನೂ ತಾಯಿ ಎಂದು ಕರೆಯುವುದು ರೂಢಿ. ಪಾಶ್ಚಾತ್ಯ ದೇಶಕ್ಕೆ ಬಂದ ಮೇಲೆ ನನ್ನ ಹಳೆಯ ಅಭ್ಯಾಸ ಬಲದಿಂದ, ಹೆಂಗಸರಿಗೆ “ಆಗಲಿ ತಾಯಿ” ಎಂದಾಗ ಅವರು ಭಯಚಿಕಿತರಾಗುತ್ತಿದ್ದರು. ಅವರು ಏತಕ್ಕೆ ಭಯ ಚಕಿತರಾಗುತ್ತಿದ್ದರೋ ನನಗೆ ಆಗ ತಿಳಿಯಲಿಲ್ಲ; ಅನಂತರ ಅದಕ್ಕೆ ಕಾರಣ ತಿಳಿಯಿತು. ಇದರಿಂದ ಅವರಿಗೆ ತುಂಬಾ ವಯಸ್ಸಾಗಿದೆ ಎಂದು ತಿಳಿಯುತ್ತಿದ್ದರು. ಭಾರತ ಮಹಿಳೆಯ ಆದರ್ಶ ತಾಯಿತನ. ಅತ್ಯಮೂಲ್ಯ ನಿಃಸ್ವಾರ್ಥದ ಸಹಿಷ್ಣುತೆಯ ಮೂರ್ತಿ ಕ್ಷಮಾಮಯಿ ತಾಯಿ. ಹೆಂಡತಿ ನೆರಳಿನಂತೆ ಹಿಂಬಾಲಿಸುವಳು. ತಾಯಿಯ ಜೀವನವನ್ನು ಆಕೆ ಅನುಸರಿಸಬೇಕು. ಅದೇ ಅವಳ ಕರ್ತವ್ಯ. ತಾಯಿ ಪ್ರೇಮದ ಆದರ್ಶ. ಆಕೆ ಗೃಹವನ್ನು ಆಳುವಳು. ಗೃಹ ಅವಳ ಅಧೀನ. ಭರತಖಂಡದಲ್ಲಿ ಮಗು ಏನಾದರೂ ತಪ್ಪು ಮಾಡಿದರೆ ಗದರಿಸಿ ಹೊಡೆಯುವವನು ತಂದೆ. ತಂದೆ ಮತ್ತು ಮಕ್ಕಳ ಮಧ್ಯೆ ನಿಲ್ಲುವವಳು ತಾಯಿ. ಪಾಶ್ಚಾತ್ಯರಲ್ಲಿ ಸಂಪೂರ್ಣ ವಿರೋಧವಾಗಿದೆ. ಹುಡುಗರನ್ನು ಶಿಕ್ಷಿಸುವವಳು ತಾಯಿ. ಪಾಪ, ತಂದೆ ಮಧ್ಯೆ ಬರಬೇಕಾಗಿದೆ. ನೋಡಿ, ಆದರ್ಶಗಳು ಬೇರೆ ಬೇರೆ. ಇದನ್ನು ನಾನು ದೂರುತ್ತಿಲ್ಲ. ನೀವು ಇಲ್ಲಿ ಮಾಡುವುದು ಸರಿ. ಆದರೆ ನಮ್ಮ ರೀತಿ, ಯಾವುದನ್ನು ಹಲವು ಶತಮಾನಗಳಿಂದ ಕಲಿತಿರುವೆವೋ ಅದು ಬೇರೆ. ತಾಯಿ ಮಗುವಿಗೆ ಶಾಪ ಕೊಡುವುದನ್ನು ನೀವು ಎಂದಿಗೂ ಕೇಳಲಾರಿರಿ. ಆಕೆ ಕ್ಷಮಿಸುವಳು. ಅವಳು ಅನವರತ ಕ್ಷಮಾಗಣಿ. “ಸ್ವರ್ಗದಲ್ಲಿರುವ ನಮ್ಮ ತಂದೆ” ಎನ್ನುವ ಬದಲು ಸದಾ ಕಾಲದಲ್ಲಿಯೂ ತಾಯಿ ಎಂದು ಕರೆಯುತ್ತೇವೆ. ಹಿಂದೂ ಮನಸ್ಸಿನಲ್ಲಿ ಆ ಪದ, ಆ ಭಾವನೆ, ಅನಂತ ಪ್ರೇಮಕ್ಕೆ ಪ್ರತೀಕವಾಗಿದೆ. ನಮ್ಮ ಕ್ಷಣಭಂಗುರ ಜಗತ್ತಿನಲ್ಲಿ ತಾಯಿಯ ಪ್ರೇಮ, ಭಗವಂತನ ಪ್ರೇಮಕ್ಕೆ ಅತಿ ಸಮೀಪದಲ್ಲಿರುವುದು. “ಹೇ ತಾಯಿ, ದಯೆ ತೋರು. ನಾನು ಪಾಪಿ. ಮಕ್ಕಳೆಷ್ಟೋ ಕೆಟ್ಟವರಿರುವರು. ಆದರೆ ಕೆಟ್ಟ ತಾಯಿ ಎಂದಿಗೂ ಇಲ್ಲ” ಎಂದು ರಾಮಪ್ರಸಾದನೆಂಬ ಪ್ರಸಿದ್ಧ ಕವಿ ಹಾಡುವನು.

ಹಿಂದೂ ತಾಯಿಯ ಸ್ಥಾನ ಅದು, ಮಗನ ಹೆಂಡತಿ ಅವಳ ಮಗಳಂತೆ ಮನೆಗೆ\break ಬರುವಳು. ಹೇಗೆ ತನ್ನ ಮಗಳು ಮದುವೆಮಾಡಿಕೊಂಡು ಹೋದಳೋ, ಹಾಗೆಯೇ ತನ್ನ ಮಗ ಮದುವೆಯಾಗಿ ಮತ್ತೊಬ್ಬ ಮಗಳನ್ನು ಮನೆಗೆ ತರುವನು. ರಾಣಿಯ ರಾಣಿಯಾದ ತನ್ನ ತಾಯಿಯ ಅಧೀನದಲ್ಲಿ ಅವಳು ಇರಬೇಕು. ನಾನಾದರೋ ಮದುವೆ ಆಗದ ಆಶ್ರಮಕ್ಕೆ ಸೇರಿದ್ದರಿಂದ ಮದುವೆ ಆಗದಿದ್ದರೂ, ಒಂದು ವೇಳೆ ಮದುವೆ ಆಗಿದ್ದರೆ, ನನ್ನ ಹೆಂಡತಿ ನನ್ನ ತಾಯಿಯ ಮನಸ್ಸು ನೋಯುವಂತೆ ವರ್ತಿಸಿದರೆ, ಆಕೆಯನ್ನು ಗೌರವಿಸಲಾರೆ, ನನಗೆ ಜುಗುಪ್ಸೆ ಹುಟ್ಟುತ್ತಿತ್ತು. ಏತಕ್ಕೆ? ನಾನು ತಾಯಿಯನ್ನು ಪೂಜಿಸುವುದಿಲ್ಲವೆ? ಅವಳ ಸೊಸೆ ಏತಕ್ಕೆ ಪೂಜಿಸಬಾರದು? ನಾನು ಯಾರನ್ನು ಪೂಜಿಸುತ್ತೇನೆಯೋ ಅವರನ್ನು ಅವಳು ಏತಕ್ಕೆ ಪ್ರೀತಿಸಬಾರದು? ನನ್ನನ್ನು ಗಮನಕ್ಕೆ ತಾರದೆ ನನ್ನ ತಾಯಿಯನ್ನು ಆಳಲು ಅವಳು ಯಾರು? ಆಕೆ ತನ್ನ ಸ್ತ್ರೀತ್ವ ಫಲಕಾರಿಯಾಗುವವರೆಗೆ ಕಾಯಬೇಕು? ಯಾವುದು\break ಸ್ತ್ರೀತ್ವವನ್ನು ಪೂರ್ಣಮಾಡುವುದೋ, ಸ್ತ್ರೀಯರಲ್ಲಿ ಯಾವುದು ಸ್ತ್ರೀತ್ವವೊ, ಅದೇ ಮಾತೃತ್ವ. ಅವಳೂ ತಾಯಿಯಾಗುವವರೆಗೆ ಕಾಯಲಿ. ಆಗ ಅವಳಿಗೂ ಅದೇ ಹಕ್ಕು ಸಿಕ್ಕುವುದು. ಅದೇ ಹಿಂದೂಗಳ ರೀತಿ. ಎಲ್ಲಾ ನಾರಿಯರ ಪಾತ್ರ ತಾಯಿಯಾಗುವುದು. ಆದರೆ ಎಷ್ಟು ವ್ಯತ್ಯಾಸ! ಎಷ್ಟು ವ್ಯತ್ಯಾಸ! ಹಲವು ವರ್ಷಗಳ ಕಾಲ ನನ್ನ ಜನನಕ್ಕಾಗಿ ತಾಯಿತಂದೆಗಳು ಉಪವಾಸವಿದ್ದು ದೇವರನ್ನು ಪ್ರಾರ್ಥಿಸಿದರು. ಮಗು ಹುಟ್ಟುವುದಕ್ಕೆ ಮುಂಚೆ ಅದಕ್ಕಾಗಿ ಪ್ರಾರ್ಥಿಸುವರು. ಧರ್ಮಶಾಸ್ತ್ರ ಕರ್ತೃ ಮನು ಆರ್ಯನಾರು ಎಂದು ವಿವರಿಸುವಾಗ, “ಯಾರು ಪ್ರಾರ್ಥನಾ ಬಲದಿಂದ ಜನಿಸುವನೋ ಅವನೇ ಆರ್ಯನು” ಎನ್ನುವನು. ಯಾವ ಮಗು ಪ್ರಾರ್ಥನಾ ಬಲವಿಲ್ಲದೆ ಜನಿಸಿತೊ, ಅದು ಜಾರಜ, ಅವನ ದೃಷ್ಟಿಯಲ್ಲಿ. ಮಗುವಿಗಾಗಿ\break ಪ್ರಾರ್ಥಿಸಬೇಕು. ಯಾವ ಮಕ್ಕಳು ನಿಂದೆಯ ಸುರಿಮಳೆಯಿಂದ ಜನಿಸುವುವೋ,\break ದಂಪತಿಗಳು ಅಜಾಗರೂಕರಾಗಿರುವಾಗ ನಿರ್ವಾಹವಿಲ್ಲದೆ ಅವು ಪ್ರಪಂಚಕ್ಕೆ ಜಾರುವುವೋ ಅಂತಹ ಸಂತಾನದಿಂದ ನೀವು ಏನನ್ನು ಆಶಿಸಬಲ್ಲಿರಿ? ಅಮೆರಿಕಾ ತಾಯಂದಿರೆ, ನೀವು ಇದನ್ನು ಅಲೋಚಿಸಿ ನೋಡಿ. ನಿಮ್ಮ ಹೃದಯಾಂತರಾಳದಲ್ಲಿ ಆಲೋಚಿಸಿ ನೋಡಿ. ನೀವು ನಿಜವಾದ ಸ್ತ್ರೀಯರಾಗಿರುವುದಕ್ಕೆ ಸಿದ್ಧರಾಗಿರುವಿರಾ? ಇದು ಜಾತಿ ದೇಶ ಅಥವಾ ರಾಷ್ಟ್ರದ ಅಭಿಮಾನದ ಪ್ರಶ್ನೆಯಲ್ಲ. ಸಂಕಟದುಃಖಗಳಿಂದ ತುಂಬಿರುವ ಈ ಪ್ರಪಂಚದಲ್ಲಿರುವ ಯಾರಿಗೆ ಅಭಿಮಾನಪಡುವ ಧ್ಯೆರ್ಯವಿದೆ? ಜಗದೀಶ್ವರನ ಅದ್ಭುತ ಶಕ್ತಿಯ ಎದುರಿಗೆ ನಾವಾರು? ಆದರೆ ನಾನು ನಿಮಗೆ ಇಂದು ಈ ಪ್ರಶ್ನೆಯನ್ನು ಹಾಕುತ್ತೇನೆ: “ನೀವೆಲ್ಲ ಮಕ್ಕಳ ಜನನಕ್ಕಾಗಿ ಪ್ರಾರ್ಥಿಸುವಿರೇನು? ತಾಯಿಯಾಗುವುದಕ್ಕೆ ನಿಮಗೆ ಸಂತೋಷವೆ ಇಲ್ಲವೆ? ತಾಯ್ತನದಿಂದ ನೀವು ಪವಿತ್ರರಾದೆವೆಂದು ಭಾವಿಸುವಿರೋ ಇಲ್ಲವೊ?” ಈ ಪ್ರಶ್ನೆಯನ್ನು ನೀವೇ ಹಾಕಿಕೊಳ್ಳಿ, ಹಾಗಿಲ್ಲದೆ ಇದ್ದರೆ ನಿಮ್ಮ ಮದುವೆ ಸುಳ್ಳು, ನಿಮ್ಮ ಸ್ತ್ರೀತ್ವ ಸುಳ್ಳು, ನಿಮ್ಮ ವಿದ್ಯೆ ಬರಿಯ ಮೂಢನಂಬಿಕೆ. ಪ್ರಾರ್ಥನೆಯಿಲ್ಲದೆ ನಿಮ್ಮ ಮಕ್ಕಳು ಪ್ರಪಂಚಕ್ಕೆ ಕಾಲಿಟ್ಟರೆ ಅದೊಂದು ಮಾನವಕೋಟಿಗೆ ಕೊಟ್ಟ ಮಹಾಶಾಪ.

ನಮ್ಮೆದುರಿಗೆ ಬರುವ ಆದರ್ಶಗಳನ್ನು ಈಗ ನೋಡಿ. ತಾಯಿತನದಿಂದ ಗುರುತರವಾದ ಜವಾಬ್ದಾರಿ ಬರುತ್ತದೆ. ಇದೇ ಮೂಲ. ಇಲ್ಲಿಂದ ಮುಂದುವರಿಯಬೇಕು. ಏತಕ್ಕೆ ನಾವು ತಾಯಿಯನ್ನು ಅಷ್ಟು ಪೂಜಿಸಬೇಕು? ಏಕೆಂದರೆ ನಮ್ಮ ಶಾಸ್ತ್ರ, ಜನನಕ್ಕೆ ಮುಂಚೆ ಆಗುವ ಪ್ರಭಾವವೇ ಮಗುವನ್ನು ಒಳ್ಳೆಯದನ್ನಾಗಿಯೋ ಕೆಟ್ಟದನ್ನಾಗಿಯೋ ಮಾಡುವುದೆಂದು ಸಾರುವುದು. ಅನಂತರ ಸಾವಿರಾರು ವಿಶ್ವವಿದ್ಯಾನಿಲಯಗಳಿಗೆ ಹೋಗಬಹುದು. ಲಕ್ಷಾಂತರ ಗ್ರಂಥಗಳನ್ನು ಓದಬಹುದು, ಪ್ರಪಂಚದ ಪಂಡಿತರೊಂದಿಗೆಲ್ಲಾ ಸಂಬಂಧವನ್ನು ಬೆಳೆಸಬಹುದು, ಆದರೆ ಇವೆಲ್ಲಕ್ಕಿಂತಲೂ ಯಾರು ಒಳ್ಳೆಯ ಸಂಸ್ಕಾರದೊಂದಿಗೆ ಹುಟ್ಟಿರುವರೋ ಅವರೇ ಮೇಲು. ನೀವು ಒಳ್ಳೆಯದಕ್ಕೋ ಕೆಟ್ಟದಕ್ಕೋ ಜನ್ಮ ತಾಳುವಿರಿ. ಮಗು ಹುಟ್ಟು ದಾನವ, ಇಲ್ಲವೆ ದೇವ ಎಂದು ಶಾಸ್ತ್ರ ಸಾರುವುದು. ವಿದ್ಯೆ ಮುಂತಾದುವೆಲ್ಲ ಅನಂತರ\break ಬರುವುವು. ಅವೆಲ್ಲ ಅಲ್ಪ ವಿಷಯ. ನೀವು ಹುಟ್ಟುವಾಗ ಏನಾಗಿದ್ದಿರೋ ಹಾಗೆಯೇ ಇರುವಿರಿ. ಹುಟ್ಟುವಾಗ ಅನಾರೋಗ್ಯವಾಗಿದ್ದು ಅನಂತರ ಎಷ್ಟೊಂದು ಔಷಧ ಸೇವಿಸಿದರೂ ಆರೋಗ್ಯವಾಗಬಲ್ಲಿರಾ? ನಿತ್ರಾಣಿ ತಾಯಿತಂದೆಗಳಿಗೆ, ರೋಗ ರುಜಿನಗಳಿಂದ\break ನರಳುತ್ತಿರುವವರಿಗೆ, ಎಷ್ಟು ಜನ ದೃಢಕಾಯರಾದ ಮಕ್ಕಳಾಗಿವೆ? ಎಷ್ಟು ಜನ? ಯಾರೂ ಇಲ್ಲ. ಒಳ್ಳೆಯದಕ್ಕೊ ಕೆಟ್ಟದಕ್ಕೊ ಒಂದು ಮಹಾ ಶಕ್ತಿಯೊಂದಿಗೆ ನಾವು ಬರುವೆವು. ನಾವೆಲ್ಲ ಹುಟ್ಟು ಅಮರರು, ಇಲ್ಲವೇ ಅಸುರರು. ವಿದ್ಯೆ, ತರಬೇತು ಮುಂತಾದುವೆಲ್ಲ ಅಷ್ಟು ಪ್ರಾಮುಖ್ಯವಲ್ಲ.

ನಮ್ಮ ಶಾಸ್ತ್ರಗಳು “ಜನನಕ್ಕೆ ಮುಂಚಿನ ಪ್ರಭಾವವನ್ನು ನಿಯಂತ್ರಿಸಿ” ಎಂದು\break ಹೇಳುತ್ತವೆ. ತಾಯಿಯನ್ನು ನಾವೇಕೆ ಆರಾಧಿಸಬೇಕು? ಆಕೆ ನಮ್ಮ ಜನನಕ್ಕಾಗಿ ಪರಿಶುದ್ಧಳಾದಳು. ಪರಿಶುದ್ಧತೆಯಷ್ಟೇ ಪರಿಶುದ್ಧಳಾಗಲು ಅವಳು ಕಠಿಣ ತಪಸ್ಸನ್ನು ಆಚರಿಸಿದಳು. ಇದನ್ನು ಗಮನದಲ್ಲಿಡಿ. ಭರತಖಂಡದಲ್ಲಿ ಯಾವ ಸ್ತ್ರೀಯೂ ತನ್ನ ದೇಹವನ್ನು\break ಯಾವನಿಗೇ ಆದರು ಸಮರ್ಪಿಸುವುದಿಲ್ಲ. ದೇಹ ಆಕೆಯದು. “ದಾಂಪತ್ಯದ ಹಕ್ಕು”\break ಎಂಬ ಕಾನೂನನ್ನು ಆಂಗ್ಲೇಯರು ಭಾರತದಲ್ಲಿ ಜಾರಿಗೆ ತಂದಿರುವರು. ಆದರೆ ಯಾವ ಭಾರತೀಯನೂ ಅದರ ಸಹಾಯವನ್ನು ಕೋರಿಲ್ಲ. ಪುರುಷನು ತನ್ನ ಸತಿಯ ದೇಹ ಸಂಬಂಧವನ್ನು ಆಶಿಸುವಾಗ, ಆಕೆ ಪ್ರಾರ್ಥನೆ ವ್ರತ ನಿಯಮಾದಿಗಳ ಮೂಲಕ ಆ ಸಮಯವನ್ನು ತನ್ನ ಸ್ವಾಧೀನದಲ್ಲಿಟ್ಟುಕೊಂಡಿರುವಳು. ಏಕೆಂದರೆ ಯಾವುದು ಸಂತಾನೋತ್ಪತ್ತಿಗೆ ಕಾರಣವಾಗುವುದೊ ಅದು ಭಗವಂತನ ಪವಿತ್ರ ಚಿಹ್ನೆ. ಒಳ್ಳೆಯದಕ್ಕೋ ಕೆಟ್ಟದಕ್ಕೋ, ಅದ್ಭುತ ಶಕ್ತಿಯಿಂದ ಸನ್ನದ್ಧವಾಗಿ ಬರುವ ಮತ್ತೊಂದು ಜೀವಿಯ ಜನನಕ್ಕೆ ಸ್ತ್ರೀಪುರುಷರು ಮಾಡುವ ಪರಮ ಪ್ರಾರ್ಥನೆ ಅದು. ಇದೊಂದು ತಮಾಷೆಯೆ? ಇದು ಕೇವಲ ಇಂದ್ರಿಯತೃಪ್ತಿ ಮಾತ್ರವೇನು? ಇದು ದೇಹದ ಕೇವಲ ಮೃಗೀಯ ತೃಪ್ತಿ ಮಾತ್ರವೆ? ಇಲ್ಲ, ಸಾವಿರವೇಳೆ ಇಲ್ಲವೆಂದು ಹಿಂದೂ ಸಾರುವನು.

ಇದನ್ನು ಅನುಸರಿಸಿಕೊಂಡು ಮತ್ತೊಂದು ಭಾವನೆ ಬರುವುದು. ಆದರ್ಶಮಾತೃ ಪ್ರೇಮದ ಮೂರ್ತಿ, ಸಹಿಷ್ಣುತಾಮೂರ್ತಿ, ಕ್ಷಮಾಮೂರ್ತಿ ಅವಳು. ಮಾತೃ ಪೂಜೆಗೆ ಕಾರಣ ಅಲ್ಲಿರುವುದು. ಆಕೆ ನನ್ನ ಜನನಕ್ಕಾಗಿ ತಪಸ್ವಿನಿಯಾದಳು. ನಾನು ಜನಿಸುತ್ತೇನೆಂದು ಹಲವಾರು ವರುಷಗಳಿಂದ ದೇಹವನ್ನು ಶುದ್ಧವಾಗಿಟ್ಟಳು; ಆಹಾರ, ವಸನ, ಆಲೋಚನೆ\-ಯೆಲ್ಲವನ್ನು ಪರಿಶುದ್ಧವಾಗಿಟ್ಟಳು. ಇದನ್ನೆಲ್ಲ ಆಕೆ ಮಾಡಿದಳು. ಆದಕಾರಣವೇ ಆಕೆ\break ಪೂಜಾರ್ಹಳು. ಅನಂತರ ಬರುವುದಾವುದು? ತಾಯಿತನದೊಂದಿಗೆ ಸತೀತ್ವ ಬರುವುದು.

ನೀವು ಪಾಶ್ಚಾತ್ಯ ಜನರು ವೈಯಕ್ತಿಕ ದೃಷ್ಟಿಯುಳ್ಳವರು. ನನಗೆ ಇದರ ಮೇಲೆ ಇಚ್ಛೆ, ಆದಕಾರಣ ಇದನ್ನು ನಾನು ಮಾಡುತ್ತೇನೆ; ಇದಕ್ಕಾಗಿ ಉಳಿದವರನ್ನೆಲ್ಲಾ ನೂಕುತ್ತೇನೆ.\break ಏಕೆಂದರೆ ನನಗೆ ಅದೇ ಇಚ್ಛೆ. ನನ್ನ ಸ್ವಂತ ತೃಪ್ತಿ ಬೇಕಾಗಿದೆ. ನಾನು ಈ ಹೆಂಗಸನ್ನು ಮದುವೆಯಾಗುತ್ತೇನೆ. ಏಕೆ? ನಾನು ಅವಳನ್ನು ಪ್ರೀತಿಸುತ್ತೇನೆ ಅದಕ್ಕಾಗಿ. ಈ ಹೆಂಗಸು ನನ್ನನ್ನು ಮದುವೆಯಾಗುವಳು. ಏಕೆ? ಆಕೆ ನನ್ನನ್ನು ಪ್ರೀತಿಸುವಳು. ಇಲ್ಲಿಗೆ ಮುಕ್ತಾಯವಾಗುವುದು. ಈ ಪ್ರಪಂಚದಲ್ಲೆಲ್ಲಾ ನಾನು ಅವಳು ಇಬ್ಬರೇ ಇರುವುದು. ನಾನು ಅವಳನ್ನು ಮದುವೆಯಾಗುವೆನು. ಅವಳು ನನ್ನನ್ನು ಮದುವೆಯಾಗುವಳು. ಇದರಿಂದ ಯಾರಿಗೂ ಅಪಾಯವಿಲ್ಲ. ಇದಕ್ಕೆ ಯಾರೂ ಜವಾಬ್ದಾರರಲ್ಲ. ನಿಮ್ಮ ಜಾನ್​ ಮತ್ತು ಜೇನಿ ಹೀಗೆ\break ಆದರೆ ಕಾಡಿಗೆ ಹೋಗಿ ಬೇಕಾದರೆ ವಾಸಮಾಡಬಹುದು. ಆದರೆ ಅವರು ಸಮಾಜದಲ್ಲಿ\break ಇರಬೇಕಾದುದರಿಂದ ಅವರ ಮದುವೆಯಿಂದ ನಮಗೆ ಮಹಾವಿಪತ್ತು ಬರಬಹುದು ಅಥವಾ ಸುಖ ಬರಬಹುದು; ಅವರ ಸಂತಾನ ಪ್ರತ್ಯಕ್ಷ ರಾಕ್ಷಸರಾಗಬಹುದು, ಕಳ್ಳರು ಕೊಲೆಪಾತಕಿಗಳು ದರೋಡೆಗಾರರು ಅಥವಾ ಕುಡುಕರಾಗಬಹುದು.

ಭಾರತದ ಸಾಮಾಜಿಕ ತಳಹದಿ ಯಾವುದು? ವರ್ಣವ್ಯವಸ್ಥೆ. ನಾನು ವರ್ಣದಲ್ಲಿ ಹುಟ್ಟಿರುವುದು, ವರ್ಣಕ್ಕೇ ಬದುಕಿರುವುದು. ಇಲ್ಲಿ ನನ್ನನ್ನು ಕುರಿತು ಹೇಳುತ್ತಿಲ್ಲ. ನಾನು ಸಂನ್ಯಾಸಿಯಾದುದರಿಂದ ವರ್ಣಾತೀತ. ಯಾರು ಸಮಾಜದಲ್ಲಿರುವರೋ ಅವರಿಗೆ ಇದು ಅನ್ವಯಿಸುವುದು. ಒಂದು ವರ್ಣದಲ್ಲಿ ಹುಟ್ಟಿದ ಮೇಲೆ ಇಡೀ ಜೀವನವನ್ನೆಲ್ಲ ಅದರ ನಿಯಮಾನುಸಾರ ಕಳೆಯಬೇಕು. ನಿಮ್ಮ ಮಾತಿನಲ್ಲಿ ಹೇಳುವುದಾದರೆ ಪಾಶ್ಚಾತ್ಯರದು ವೈಯಕ್ತಿಕ ದೃಷ್ಟಿ, ಭಾರತೀಯರದು ಸಾಮಾಜಿಕ ದೃಷ್ಟಿ. ನಿಮ್ಮ ಇಚ್ಛೆಗೆ ಅನುಸಾರವಾಗಿ ಗಂಡಸು ಯಾವ ಹೆಂಗಸನ್ನಾದರೂ, ಹೆಂಗಸು ಯಾವ ಗಂಡಸನ್ನಾದರೂ ಮದುವೆ\break ಮಾಡಿಕೊಳ್ಳಲು ಸ್ವಾತಂತ್ರ್ಯ ಕೊಟ್ಟರೆ ಏನಾಗುವುದೆಂದು ಶಾಸ್ತ್ರ ಹೇಳುವುದು. ನೀವು\break ಒಬ್ಬಳನ್ನು ಪ್ರೀತಿಸುವಿರಿ ಎಂದು ಭಾವಿಸೋಣ. ಆ ಹೆಂಗಸಿನ ತಂದೆ ಹುಚ್ಚನಾಗಿರಬಹುದು, ಕ್ಷಯರೋಗ ಪೀಡಿತನಾಗಿರಬಹುದು. ಹೆಂಗಸು ಒಬ್ಬ ಗಂಡಸನ್ನು ಪ್ರೀತಿಸುವಳು. ಆ\break ಗಂಡಸಿನ ತಂದೆ ಭಯಂಕರ ಕುಡುಕನಾಗಿರಬಹುದು. ಆಗ ಶಾಸ್ತ್ರ ಏನು ಹೇಳುವುದು? ಇಂತಹ ಮದುವೆ ಕಾನೂನಿಗೆ ವಿರೋಧ ಎನ್ನುವುದು. ಕುಡುಕರ, ಕ್ಷಯರೋಗ ಪೀಡಿತರ, ಹುಚ್ಚರ ಮಕ್ಕಳನ್ನು ಮದುವೆಯಾಗಕೂಡದೆಂದು ಹೇಳುವುದು. ಅಂಗಹೀನರಿಗೆ, ಗೂನು ಬೆನ್ನಿನವರಿಗೆ, ಪೆದ್ದರಿಗೆ, ಉನ್ಮತ್ತರಿಗೆ ಮದುವೆಯಿಲ್ಲವೆಂದು ಶಾಸ್ತ್ರ ಸಾರುವುದು.

ಅರೇಬಿಯಾ ದೇಶದಿಂದ ಮಹಮ್ಮದೀಯನು ಬರುವನು. ಅವನಿಗೆ ಅರಬ್ಬೀ ದೇಶದ ಕಾನೂನಿದೆ. ಅರೇಬಿಯಾ ಮರಳುಕಾಡಿನ ಕಾನೂನನ್ನು ನಮ್ಮ ಮೇಲೆ ಬಲಾತ್ಕಾರವಾಗಿ ಹೇರುವನು. ಆಂಗ್ಲೇಯರು ತಮ್ಮ ಕಾನೂನನ್ನು ತಂದು ಸಾಧ್ಯವಾದಷ್ಟೂ ನಮ್ಮ ಮೇಲೆ ಜಾರಿಮಾಡಲು ಯತ್ನಿಸುವರು. ಅವರು ನಮ್ಮನ್ನು ಗೆದ್ದಿರುವರು. ನಾಳೆ ಒಬ್ಬ ಬಂದು “ನಿನ್ನ ಸಹೋದರಿಯನ್ನು ಮದುವೆಯಾಗುತ್ತೇನೆ” ಎನ್ನುವನು. ನಾವು ಮಾಡುವುದೇನು? “ಯಾರು ಒಂದೇ ಮನೆತನದವರೋ ಅವರು ನೂರು ತಲೆಯ ಪರಿಯಂತರದಷ್ಟು\break ದೂರವಿದ್ದರೂ ಮದುವೆಯಾಗಕೂಡದು. ಇದು ನ್ಯಾಯವಿರುದ್ಧ. ಇದರಿಂದ ಜನಾಂಗ ಕ್ಷೀಣಿಸುವುದು, ದುರ್ಬಲವಾಗುವುದು” ಎಂದು ಶಾಸ್ತ್ರ ಹೇಳುವುದು. ಅದು ಅಲ್ಲೇ\break ನಿಲ್ಲುವುದು. ಆದಕಾರಣವೇ ಮದುವೆಯ ವಿಷಯದಲ್ಲಿ ನನಗಾಗಲೀ ನನ್ನ ಸಹೋದರಿಗಾಗಲೀ ಅಧಿಕಾರವಿಲ್ಲ. ವರ್ಣ ಇದನ್ನೆಲ್ಲ ನಿರ್ಣಯಿಸುವುದು. ಕೆಲವು ವೇಳೆ ಶೈಶವಾವಸ್ಥೆಯಲ್ಲಿಯೇ ನಮಗೆ ಮದುವೆ ಆಗಿರುವುದು. ಅದಕ್ಕೆ ಕಾರಣವೇನು? ಹೇಗಿದ್ದರೂ\break ಹುಡುಗ ಹುಡುಗಿಯರ ಒಪ್ಪಿಗೆಯಿಲ್ಲದೆ ಮದುವೆಯಾಗಬೇಕಾಗಿರುವಾಗ, ಪ್ರೀತಿ\break ಹುಟ್ಟುವುದಕ್ಕೆ ಮುಂಚೆ, ಅವರು ಸಣ್ಣವರಾಗಿರುವಾಗಲೇ ಮದುವೆಯಾಗುವುದು\break ಒಳ್ಳೆಯದು ಎನ್ನುವುದು ಶಾಸ್ತ್ರ. ಅವರು ಬೇರೆ ಬೇರೆ ಬೆಳೆಯುವವರೆವಿಗೂ ಅವಕಾಶ\break ಕೊಟ್ಟರೆ, ಹುಡುಗಿ ಮತ್ತೊಬ್ಬ ಹುಡುಗನನ್ನು ಪ್ರೀತಿಸಬಹುದು. ಇದರಿಂದ ಏನಾದರೂ ಅನಿಷ್ಟ ಪ್ರಾಪ್ತವಾಗಬಹುದು. ಅದಕ್ಕೆ ವರ್ಣ ‘ಇಲ್ಲೇ ನಿಲ್ಲಿ’ ಎನ್ನುವುದು. ನನ್ನ\break ಸಹೋದರಿಯು ಕುರೂಪಿ ಅಂಗಹೀನೆಯಾದರೂ ಪ್ರೀತಿಸುತ್ತೇನೆ. ಹಾಗೆಯೇ ಅವಳು ಕೂಡ. ನನ್ನ ಸೋದರನ ವಿಷಯದಲ್ಲೂ ಹಾಗೇ. ಇದರಂತೆಯೇ ಬಾಲ್ಯ ದಂಪತಿಗಳು ಒಬ್ಬರು ಮತ್ತೊಬ್ಬರನ್ನು ಪ್ರೀತಿಸುವರು. ನೀವು “ಅವರು ಎಷ್ಟೋ ಆನಂದವನ್ನು ಕಳೆದುಕೊಳ್ಳುವರು. ಪುರುಷ ಸ್ತ್ರೀಯನ್ನು, ಸ್ತ್ರೀ ಪುರುಷನನ್ನು ತಾನಾಗಿ ಪ್ರೀತಿಸುವಾಗ ಏಳುವ ಪ್ರಾಥಮಿಕ ಸುಂದರ ಭಾವನಾತರಂಗಗಳೇ ಇರುವುದಿಲ್ಲವಲ್ಲ! ವಿಧಿಯಿಲ್ಲದೆ ಅಣ್ಣ ತಂಗಿಯರಂತೆ ಪ್ರೀತಿಸುವುದರಲ್ಲಿ ಉದ್ವೇಗವೇ ಇರುವುದಿಲ್ಲ” ಎನ್ನಬಹುದು. ಹಿಂದೂ “ಆದರೂ ಚಿಂತೆಯಿಲ್ಲ. ನಮ್ಮದು ಸಾಮಾಜಿಕ ದೃಷ್ಟಿ. ಒಬ್ಬಳು ಹೆಂಗಸಿನ ಅಥವಾ ಗಂಡಸಿನ ಸುಂದರ ಪ್ರೇಮ ಭಾವನೆಗಾಗಿ, ನೂರಾರು ಜನರ ಮೇಲೆ ದುಃಖವನ್ನು ಹೇರಲು ನನಗೆ ಇಚ್ಛೆಯಿಲ್ಲ” ಎನ್ನುವನು.

ಅವರು ಮದುವೆಯಾಗುವರು. ಹೆಂಡತಿ ಗಂಡನೊಂದಿಗೆ ಮನೆಗೆ ಬರುವಳು. ಇದನ್ನೇ ಎರಡನೆಯ ಮದುವೆ ಎನ್ನುವರು. ಬಾಲ್ಯದಲ್ಲಿಯೆ ಆಗುವ ಮದುವೆ ಮೊದಲನೆಯ ಮದುವೆ. ಹೆಣ್ಣು ತನ್ನ ತಾಯಿ ತಂದೆಯರ ಮನೆಯಲ್ಲಿ ಬೆಳೆಯುವಳು. ಅವರು ವಯಸ್ಸಿಗೆ ಬಂದಮೇಲೆ ಎರಡನೆಯ ಮದುವೆ ಎಂಬ ಶಾಸ್ತ್ರವನ್ನು ಮಾಡುವರು. ಅನಂತರ ಒಂದೇ ಮನೆಯಲ್ಲಿ ಗಂಡನ ತಾಯಿ ತಂದೆಯರೊಂದಿಗೆ ಇರುವಳು. ಅವಳು ತಾಯಿಯಾದ ಮೇಲೆ ಮನೆಯ ರಾಣಿಯಂತೆ ತನ್ನ ಸ್ಥಾನವನ್ನು ಅಲಂಕರಿಸುವಳು.

ಈಗ ಭಾರತದ ಮತ್ತೊಂದು ವಿಚಿತ್ರ ಸಂಸ್ಥೆಗೆ ಬರುವೆವು. ಮೇಲಿನ ಎರಡು ಮೂರು ವರ್ಣಗಳಲ್ಲಿ ವಿಧವೆಯರು ಪುನಃ ಮದುವೆ ಆಗಲಾರರೆಂಬುದನ್ನು ನಾನು ಆಗಲೇ\break ಹೇಳಿರುವೆನು. ಅವರು ಇಚ್ಛಿಸಿದರೂ ಮದುವೆಯಾಗಲಾರರು. ಇದರಿಂದ ಅನೇಕರಿಗೆ ತೊಂದರೆಯೇನೋ ನಿಜ. ಇದನ್ನು ವಿಧವೆಯರೆಲ್ಲರೂ ಒಪ್ಪುತ್ತಾರೆ ಎಂದಲ್ಲ. ಯಾವಾಗ ಅವರು ಮದುವೆಯಾಗುವುದಿಲ್ಲವೋ ಆಗ ಬ್ರಹ್ಮಚಾರಿಣಿಯರಾಗಿರಬೇಕಾಗುವುದು.\break ಅವರು ಮೀನು ಮಾಂಸಗಳನ್ನು ತಿನ್ನಬಾರದು, ಮದ್ಯವನ್ನು ಕುಡಿಯಬಾರದು, ಬಿಳಿಯ ಬಟ್ಟೆಯನ್ನು ಅಲ್ಲದೆ ಯಾವ ಅಲಂಕಾರ ವಸ್ತ್ರವನ್ನೂ ಉಡಬಾರದು. ಇಂತಹ ನಿಯಮಗಳು ಎಷ್ಟೋ ಇವೆ. ನಾವೊಂದು ತಪಸ್ವಿಗಳ ಜನಾಂಗ. ಯಾವಾಗಲೂ ತಪಸ್ಸು ಮಾಡುವುದೆಂದರೆ ನಮಗೆ ಇಚ್ಛೆ. ಹೆಂಗಸು ಎಂದಿಗೂ ಮದ್ಯಮಾಂಸಗಳನ್ನು ಸೇವಿಸುವುದಿಲ್ಲ. ನಾವು ವಿದ್ಯಾರ್ಥಿಗಳಾಗಿದ್ದಾಗ ಹೀಗೆ ಇರುವುದಕ್ಕೆ ನಮಗೆ ಕಷ್ಟವಾಗುತ್ತಿತ್ತು. ಆದರೆ ಹುಡುಗಿಯರಿಗೆ ಹಾಗಾಗುತ್ತಿರಲಿಲ್ಲ. ಮಾಂಸಾಹಾರವನ್ನು ತುಂಬಾ ಹೀನವೆಂದು ನಮ್ಮ ಹೆಂಗಸರು ಭಾವಿಸುವರು. ಕೆಲವು ವರ್ಣಗಳಲ್ಲಿ ಗಂಡಸರು ಮಾಂಸವನ್ನು ಸೇವಿಸುವರು. ಆದರೆ\break ಹೆಂಗಸರು ಎಂದಿಗೂ ಸೇವಿಸುವುದಿಲ್ಲ. ಪುನಃ ಮದುವೆಯಾಗುವುದಕ್ಕೆ ಅವಕಾಶವಿಲ್ಲದೆ ಇರುವುದರಿಂದ ಅನೇಕರಿಗೆ ತೊಂದರೆಯಾಗಬಹುದೆಂದು ನನಗೆ ಗೊತ್ತಿದೆ.

ಅವರು ಉತ್ಕಟ ಸಾಮಾಜಿಕ ದೃಷ್ಟಿಯುಳ್ಳವರು ಎಂಬ ಭಾವನೆಯನ್ನು ಮರೆಯಬಾರದು. ಎಲ್ಲಾ ಕಡೆಗಳಲ್ಲಿಯೂ ಮೇಲಿನ ವರ್ಗದಲ್ಲಿರುವ ಸಮಾಜದ ಹೆಂಗಸರ ಸಂಖ್ಯೆ ಗಂಡಸರ ಸಂಖ್ಯೆಗಿಂತ ಹೆಚ್ಚಾಗಿರುವುದು. ಇದಕ್ಕೆ ಕಾರಣವೇನು? ಮೇಲಿನ ವರ್ಗದಲ್ಲಿ ಹಲವಾರು ತಲೆತಲಾಂತರಗಳಿಂದ ಹೆಂಗಸರು ಸುಖಜೀವನವನ್ನು ನಡೆಯಿಸುತ್ತಿರುವರು. ಅವರು “ಉಳುವುದಿಲ್ಲ, ನೇಯುವುದಿಲ್ಲ. ಆದರೂ ಅವರಂತೆ ವೈಭವಯುತ ಸಾಲೊಮನ್​ ಕೂಡ ಅಲಂಕರಿಸಿಕೊಂಡಿಲ್ಲ.” ಪಾಪ, ಹುಡುಗರು ನೊಣಗಳಂತೆ ಸಾಯುವರು. ಹುಡುಗಿ ಬೆಕ್ಕಿನಂತೆ ಒಂಭತ್ತು ಜನ್ಮಗಳುಳ್ಳವಳು ಎಂದು ಹಿಂದೂಗಳು ಹೇಳುವರು. ಜನಗಣತಿಯನ್ನು ನೋಡಿದರೆ ಹುಡುಗರಿಗಿಂತ ಅವರು ಬೇಗ ಹೆಚ್ಚುವುದು ಕಾಣುವುದು. ಆದರೆ ಈಗ ಅವರೂ ಹುಡುಗರಂತೆ ಕಷ್ಟಪಟ್ಟು ಕೆಲಸ ಮಾಡುವುದರಿಂದ ಅವರ ಸಂಖ್ಯೆಯೂ ಕಡಿಮೆಯಾಗುತ್ತಿರುವುದು. ಮೇಲಿನ ವರ್ಗದಲ್ಲಿ ಹೆಂಗಸರ ಸಂಖ್ಯೆ ಗಂಡಸರ ಸಂಖ್ಯೆಗಿಂತ ಹೆಚ್ಚು. ಕೆಳಗಿನ ವರ್ಗದಲ್ಲಿ ಸ್ಥಿತಿ ಬದಲಾಗುವುದು. ಅಲ್ಲಿ ಅವರೆಲ್ಲರೂ ಕಷ್ಟಪಟ್ಟು ಕೆಲಸ ಮಾಡುವರು. ಕೆಲವು ವೇಳೆ ಹೆಂಗಸರು ಗಂಡಸರಿಗಿಂತ ಹೆಚ್ಚು ಕಷ್ಟಪಟ್ಟು ಕೆಲಸ ಮಾಡುವರು. ಜೊತೆಗೆ ಅವರಿಗೆ ಮನೆಯ ಕೆಲಸ ಬೇರೆ ಇರುವುದು. ಆದರೆ ಜ್ಞಾಪಕದಲ್ಲಿಡಿ. ಇದು ನನಗೆ ಹೊಳೆಯುತ್ತಲೇ ಇರಲಿಲ್ಲ. ಮಾರ್ಕ್​ಟ್ವೈನ್​ ಎಂಬ ಅಮೆರಿಕದ ಪರ್ಯಟನಕಾರನೇ ಇದನ್ನು ಹೀಗೆ ಬರೆಯುವನು: “ಪಾಶ್ಚಾತ್ಯರು ಹಿಂದೂ ಆಚಾರ ವ್ಯವಹಾರಗಳನ್ನು ಎಷ್ಟು ದೂರಿದರೂ, ಪಾಶ್ಚಾತ್ಯ ದೇಶಗಳಲ್ಲಿ ಮಾಡುವಂತೆ ಯಾವ ಹೆಂಗಸನ್ನೂ ನೇಗಿಲಿನೊಂದಿಗೆ ಆಗಲಿ, ಗಾಡಿಯೊಂದಿಗೆ ಆಗಲಿ, ಕೆಲಸ ಮಾಡಿಸುವುದನ್ನು ನಾನು ನೋಡಿಲ್ಲ. ಇಂಡಿಯಾ ದೇಶದ ಹೊಲದಲ್ಲಿ ಯಾವ ಹೆಂಗಸು ಅಥವಾ ಹುಡುಗಿ ಕೆಲಸ ಮಾಡುವುದನ್ನು ನಾನು ನೋಡಿಲ್ಲ. ಎರಡು ಕಡೆಯೂ ರೈಲ್ವೆ ಲೈನಿನ ಪಕ್ಕದಲ್ಲಿ, ಕಂದುಬಣ್ಣದ ಅರ್ಧ ಬೆತ್ತಲೆ ಗಂಡಸು ಹೊಲದಲ್ಲಿ ಕೆಲಸ ಮಾಡುತ್ತಿರುವರು. ಆದರೆ ಒಬ್ಬ ಹೆಂಗಸೂ ಇಲ್ಲ. ಈ ಎರಡು ಗಂಟೆಗಳಲ್ಲಿ ಯಾವ ಹೆಂಗಸಾಗಲೀ ಹುಡುಗಿಯಾಗಲೀ ಕೆಲಸ ಮಾಡುವುದನ್ನು ನಾನು ನೋಡಲಿಲ್ಲ. ಭರತಖಂಡದಲ್ಲಿ ಬಹಳ ಕೆಳಗಿನ ವರ್ಗದಲ್ಲಿ ಇರುವ ಹೆಂಗಸರೂ ಕೂಡ\break ಕಷ್ಟವಾದ ಕೆಲಸವನ್ನು ಮಾಡುವುದಿಲ್ಲ. ಇತರ ದೇಶದಲ್ಲಿರುವ ಅದೇ ವರ್ಗದ ಜನರೊಂದಿಗೆ ಹೋಲಿಸಿ ನೋಡಿದರೆ ಅವರು ಸುಖಜೀವಿಗಳೆನ್ನಬಹುದು. ಉಳುವ ಕೆಲಸವನ್ನಂತೂ ಅವರು ಮಾಡುವುದೇ ಇಲ್ಲ.”

ಈಗ ನೋಡಿ, ಕೆಳಗಿನ ವರ್ಗದಲ್ಲಿ ಗಂಡಸರ ಸಂಖ್ಯೆ ಹೆಂಗಸರ ಸಂಖ್ಯೆಗಿಂತ ಹೆಚ್ಚು. ಅಲ್ಲಿ ನೀವು ಏನನ್ನು ಸ್ವಾಭಾವಿಕವಾಗಿ ನಿರೀಕ್ಷಿಸುತ್ತೀರಿ? ಗಂಡಸರು ಹೆಚ್ಚು ಜನರಿರುವುದರಿಂದ ಹೆಂಗಸು ಹೆಚ್ಚು ಮದುವೆ ಮಾಡಿಕೊಳ್ಳುವುದಕ್ಕೆ ಅವಕಾಶವಿದೆ.

ಮದುವೆಯಾಗದ ವಿಧವೆಗೆ ಸಂಬಂಧಪಟ್ಟ ಸಮಸ್ಯೆ ಇದು. ಮೊದಲಿನ ಎರಡು ವರ್ಗಗಳಲ್ಲಿ ಹೆಂಗಸರ ಸಂಖ್ಯೆ ಗಂಡಸರಿಗಿಂತ ಹೆಚ್ಚು. ಇಲ್ಲಿ ದೊಡ್ಡ ತೊಡಕು ಬರುವುದು. ಒಂದು ಕಡೆ ಮದುವೆಯಾಗದ ವಿಧವಾ ಸಮಸ್ಯೆ. ಇನ್ನೊಂದು ಕಡೆ ಗಂಡ ಸಿಕ್ಕದ ಹುಡುಗಿಯ ಸಮಸ್ಯೆ! ಎರಡರಲ್ಲಿ ಯಾವುದಾದರೊಂದು ಇರುವುದು. ಹಿಂದೂವಿನ ದೃಷ್ಟಿ ಸಾಮಾಜಿಕ ದೃಷ್ಟಿ. ಅವನು ಹೀಗೆ ಹೇಳುವನು: “ವಿಧವೆಯ ಸಮಸ್ಯೆಯನ್ನು ಅಷ್ಟು ಮುಖ್ಯವಾಗಿ ಪರಿಗಣಿಸುವುದಿಲ್ಲ”. ಏತಕ್ಕೆ? “ಅವರಿಗೆ ಆಗಲೇ ಒಂದು ಅವಕಾಶ ದೊರಕಿದೆ. ಶಾಂತರಾಗಿ ಕುಳಿತುಕೊಳ್ಳಿ. ಉಳಿದ ನಿರ್ಭಾಗ್ಯ ಹುಡುಗಿಯರ ಗತಿಯನ್ನು ಕುರಿತು ಚಿಂತಿಸಿ. ಅವರಿಗೆ ಮದುವೆಯಾಗುವುದಕ್ಕೆ ಒಂದು ಸಾರಿಯಾದರೂ ಅವಕಾಶ ಸಿಕ್ಕಿಲ್ಲ. ದೇವರು ನಿಮಗೆ ಒಳ್ಳೆಯದನ್ನು ಮಾಡಲಿ!” ಎನ್ನುವರು. ಒಮ್ಮೆ ಆಕ್ಸ್​ಫರ್ಡ್​ ಬೀದಿಯಲ್ಲಿ ಸಾವಿರಾರು ಮಂದಿ ಹೆಂಗಸರು ಹತ್ತು ಗಂಟೆಯಾದ ಮೇಲೆ ವ್ಯಾಪಾರಕ್ಕೆ ಅಂಗಡಿಗೆ ಬಂದಿದ್ದರು. ಒಬ್ಬ ಅಮೆರಿಕದ ಗಂಡಸು ಇದನ್ನು ನೋಡಿ. “ಇವರಲ್ಲಿ ಎಷ್ಟು ಮಂದಿಗೆ ಗಂಡ ದೊರಕುತ್ತಾನೆಯೋ ದೇವರೆ ಬಲ್ಲ!” ಎಂದನು. ಹೀಗೆಯೇ ಹಿಂದೂ ತಮ್ಮ ವಿಧವೆಯರಿಗೆ ಹೇಳುವನು: “ನಿಮಗೆ ಆಗಲೇ ಒಂದು ಅವಕಾಶ ದೊರಕಿತ್ತು. ಆದರೆ ನಿಮಗೆ ಈಗ ಇಂತಹ ವೈಧವ್ಯ ಬಂದುದಕ್ಕೆ ನಾವು ವ್ಯಥೆಪಡುವೆವು. ಆದರೆ ನಾವು ಏನನ್ನೂ ಮಾಡಲಾಗುವುದಿಲ್ಲ. ಉಳಿದವರು ಕಾದು ಕುಳಿತಿರುವರು.”

ಹಿಂದೂಧರ್ಮ ಆಗ ಅವರ ನೆರವಿಗೆ ಬರುವುದು. ಅವರಿಗೆ ಸಹಾನುಭೂತಿಯನ್ನು ತೋರುವುದು. ಇದನ್ನು ಜ್ಞಾಪಕದಲ್ಲಿಡಬೇಕು. ನಮ್ಮ ಧರ್ಮ, ಮದುವೆ ಎನ್ನುವುದು ಅಷ್ಟು ದೊಡ್ಡ ಆದರ್ಶವಲ್ಲ, ಅದು ದುರ್ಬಲರಿಗೆ ಮಾತ್ರ ಎನ್ನುವುದು. ಅಧ್ಯಾತ್ಮಿಕ ಪಿಪಾಸೆಯುಳ್ಳ ಗಂಡಸರಾಗಲೀ, ಹೆಂಗಸರಾಗಲೀ ಮದುವೆಯಾಗುವುದಿಲ್ಲ. ಆಕೆ “ಆಗಲಿ, ದೇವರು ನನಗೆ ಒಂದು ಸದವಕಾಶವನ್ನು ಕೊಟ್ಟಿರುವನು. ಮದುವೆಯಿಂದ ಪ್ರಯೋಜನವೇನು? ದೇವರನ್ನು ಪೂಜಿಸಿ” ಎನ್ನುವಳು. ಎಲ್ಲರೂ ದೇವರ ಮೇಲೆ ಮನಸ್ಸನ್ನು ಇಡಲಾರರು. ಇದೇನೋ ನಿಜ. ಕೆಲವರಿಗೆ ಇದು ಅಸಾಧ್ಯ. ಅವರು ಅನುಭವಿಸಬೇಕು. ಉಳಿದವರು ಅವರಿಗಾಗಿ ದುಃಖಪಡಬೇಕಾಗಿಲ್ಲ. ನಾನು ಇದನ್ನು ನಿಮ್ಮ ತೀರ್ಪಿಗೆ ಬಿಡುತ್ತೇನೆ. ಇದು ಭಾರತದ ಆದರ್ಶ.

ಅನಂತರ ಮಗಳ ಸ್ಥಾನದಲ್ಲಿರುವ ಸ್ತ್ರೀಯ ವಿಷಯ. ಹಿಂದೂ ಕುಟುಂಬದಲ್ಲಿ\break ದೊಡ್ಡ ಸಮಸ್ಯೆಯೇ ಮಗಳು. ಮಗಳ ಜೊತೆಗೆ ವರ್ಣ ಸೇರಿ ಹಿಂದೂವನ್ನು ನಾಶಮಾಡುವುದು. ಏಕೆಂದರೆ, ಅವಳು ಅದೇ ಜಾತಿಯಲ್ಲಿ ಮದುವೆಯಾಗಬೇಕು; ಅದೇ ಜಾತಿಯಲ್ಲಿ ಇರುವ ಅದೇ ಒಳಪಂಗಡದಲ್ಲಿ, ಪಾಪ, ತಂದೆ ಮಗಳಿಗೆ ಮದುವೆ ಮಾಡುವುದಕ್ಕೆ ಪ್ರಯತ್ನಿಸಿ ಅನೇಕ ವೇಳೆ ಭಿಕಾರಿಯಾಗಬೇಕಾಗುವುದು. ಹುಡುಗನ ತಂದೆ ದೊಡ್ಡ ಮೊತ್ತದ ವರದಕ್ಷಿಣೆಯನ್ನು ಕೇಳುತ್ತಾನೆ. ಹುಡುಗಿಯ ತಂದೆ ಕೆಲವೊಮ್ಮೆ ಅವಳಿಗೆ ಗಂಡನನ್ನು ಸಂಪಾದಿಸಲು ತನ್ನ ಸರ್ವಸ್ವವನ್ನು ಮಾರಬೇಕಾಗುತ್ತದೆ. ಹಿಂದೂ ಜೀವನದ ದೊಡ್ಡದೊಂದು ತೊಂದರೆಯೆ ಮಗಳು. ಆಶ್ಚರ್ಯವೇನೆಂದರೆ ಸಂಸ್ಕೃತದಲ್ಲಿ ಮಗಳಿಗೆ “ದುಹಿತಾ” ಎಂದು ಹೆಸರು. ಈ ಪದೋತ್ಪತ್ತಿ ಹೀಗಾಯಿತು. ಹಿಂದಿನ ಕಾಲದಲ್ಲಿ ಮಗಳು ಹಸುವಿನ ಹಾಲನ್ನು ಕರೆಯುವ ವಾಡಿಕೆ ಇತ್ತು. ಇದರಿಂದ “ದೋಹ” ಹಾಲುಕರೆ ಎಂಬ ಪದ ಬಂದು “ದುಹಿತಾ” ಎಂದಾಗಿದೆ. ಮಗಳು ಎನ್ನುವುದಕ್ಕೆ ನಿಜವಾದ ಅರ್ಥ, ಹಾಲು ಕರೆಯುವವಳು ಎಂದು. ಕಾಲಾನಂತರ ದುಹಿತಾ ಎನ್ನುವುದಕ್ಕೆ ಬೇರೆ ಅರ್ಥವನ್ನು ಕಲ್ಪಿಸಿದರು. ಯಾರು ಮನೆಯ ಹಾಲನ್ನೆಲ್ಲ (ಐಶ್ವರ್ಯವನ್ನೆಲ್ಲ) ಹೊರಕ್ಕೆ ಕರೆಯುತ್ತಾಳೋ ಅವಳು ಎಂದು! ಇದು ಎರಡನೆಯ ಅರ್ಥ.

ನಮ್ಮ ಸ್ತ್ರೀಯರ ಹಲವು ಸ್ಥಾನಗಳಿವು. ನಾನು ಹೇಳಿದಂತೆ ತಾಯಿಯದು ಅತ್ಯುತ್ತಮ ಸ್ಥಾನ. ಅನಂತರ ಹೆಂಡತಿ, ಅನಂತರ ಮಗಳು. ಇದು ಗಹನವಾದ ಜಟಿಲ ವರ್ಗೀಕರಣ, ಪರಕೀಯರು ಹಲವು ವರ್ಷಗಳವರೆಗೆ ಆ ದೇಶದಲ್ಲಿದ್ದರೂ ಇದನ್ನು ತಿಳಿದುಕೊಳ್ಳಲಾರರು. ಉದಾಹರಣೆಗೆ ನಮ್ಮಲ್ಲಿ ಮೂರು ಬಗೆಯ ಪುರುಷ ವಾಚಕ ಸರ್ವನಾಮಗಳಿವೆ. ಅವು ಒಂದು ಬಗೆಯ ಕ್ರಿಯಾಪದಗಳೆ ಆಗಿವೆ ನಮ್ಮ ಭಾಷೆಯಲ್ಲಿ. ಒಂದು ಅತೀ ಗೌರವಯುತವಾದದು, ಇನ್ನೊಂದು ಮಧ್ಯಮ ವರ್ಗದ್ದು, ಮತ್ತೊಂದು ಕನಿಷ್ಠ. “ನೀನು” ಎನ್ನುವುದನ್ನು ಮಕ್ಕಳಿಗೆ ಮತ್ತು ಆಳುಗಳಿಗೆ ಉಪಯೋಗಿಸುತ್ತೇವೆ. ಮಧ್ಯಮವರ್ಗದ ಪದವನ್ನು ನಮಗೆ ಸಮಾನರಾದವರ ವಿಷಯದಲ್ಲಿ ಉಪಯೋಗಿಸಬೇಕು. ಇದನ್ನು ನಮ್ಮ ಜೀವನದ ಎಲ್ಲಾ ಗಹನ ಸಂಬಂಧಗಳಲ್ಲಿಯೂ ಉಪಯೋಗಿಸಬೇಕು. ನನ್ನ ಅಕ್ಕನನ್ನು ಯಾವಾಗಲೂ “ನೀವು” ಎಂದು ಕರೆಯುತ್ತೇನೆ. ಆದರೆ ಅವಳು ನನ್ನೊಂದಿಗೆ ಮಾತನಾಡುವಾಗ “ನೀನು” ಎಂದು ಸಂಬೋಧಿಸುವಳು. ಅವಳು ಬಾಯಿತಪ್ಪಿಯಾದರೂ ಮಾತನಾಡುವಾಗ “ನೀವು” ಎಂದು ಉಪಯೋಗಿಸಕೂಡದು. ಅದೊಂದು ಶಾಪದಂತೆ. ನಮಗಿಂತ ಹಿರಿಯರಿಗೆ ಪ್ರೀತಿಯನ್ನು ನಾವು ಯಾವಾಗಲೂ ಈ ಬಗೆಯ ಭಾಷೆಯಲ್ಲಿ ವ್ಯಕ್ತಗೊಳಿಸಬೇಕು. ಇದು ರೂಢಿ. ಇದರಂತೆಯೇ ನನ್ನ ಅಕ್ಕ, ಅಣ್ಣ, ತಾಯಿ ತಂದೆಯವರನ್ನು ಕೂಗುವಾಗ “ನೀನು” ಎಂದು ಕರೆಯಕೂಡದು. ನಾವು ನಮ್ಮ ತಂದೆತಾಯಿಗಳನ್ನು ಹೆಸರು ಹಿಡಿದು ಕೂಗುವುದಿಲ್ಲ. ಈ ದೇಶದ ಅಭ್ಯಾಸ ನನಗೆ ಗೊತ್ತಾಗುವುದಕ್ಕೆ ಮುಂಚೆ ಅತಿ ಗೌರವಸ್ಥರ ಮನೆಯಲ್ಲಿ, ಮಗ ತಾಯಿಯನ್ನು ಹೆಸರು ಹಿಡಿದು ಕರೆದಾಗ ನನಗೆ ಆಶ್ಚರ್ಯವಾಯಿತು. ಅನಂತರ ಇದು ನನಗೆ ರೂಢಿಯಾಯಿತು. ದೇಶಾಚಾರ ಇದು. ಆದರೆ ನಮ್ಮಲ್ಲಿ ತಂದೆತಾಯಿಗಳು ಎದುರಿಗೆ ಇರುವಾಗ ಎಂದಿಗೂ ಅವರ ಹೆಸರನ್ನು ಉಚ್ಚರಿಸುವುದಿಲ್ಲ. ಅವರನ್ನು ಯಾವಾಗಲೂ ಪ್ರಥಮ ಪುರುಷ ಬಹುವಚನದಿಂದಲೇ ಸಂಬೋಧಿಸುವುದು.

ನಮ್ಮ ಸ್ತ್ರೀಪುರುಷರ ಸಾಮಾಜಿಕ ಜೀವನದಲ್ಲಿ ಮತ್ತು ಅವರಲ್ಲಿ ಇರುವ ಪರಸ್ಪರ ಬಾಂಧವ್ಯದ ಅನುಕ್ರಮದಲ್ಲಿ ಅತಿ ಸೂಕ್ಷ್ಮವಾದ ಆಚಾರ ವ್ಯವಹಾರಗಳನ್ನು ನೋಡುತ್ತೇವೆ. ನಮ್ಮ ಹಿರಿಯರು ಎದುರಿಗೆ ಇರುವಾಗ ಕಿರಿಯರು ತಮ್ಮ ಹೆಂಡತಿಯೊಂದಿಗೆ ಮಾತನಾಡುವುದಿಲ್ಲ. ಅವರು ಮಾತ್ರ ಇರುವಾಗ ಅಥವಾ ಅವರಿಗಿಂತ ಕಿರಿಯರು ಇರುವಾಗ ಮಾತನಾಡಬಹುದು. ನನಗೆ ಮದುವೆಯಾಗಿದ್ದರೆ ನನ್ನ ತಂಗಿ, ಸೋದರಳಿಯ\break ಅಥವಾ ಸೋದರಸೊಸೆ ಇವರೆದುರಿಗೆ ನನ್ನ ಹೆಂಡತಿಯೊಂದಿಗೆ ಮಾತನಾಡಬಹುದು; ನನ್ನ ಅಕ್ಕ ಅಥವಾ ತಾಯಿತಂದೆಗಳಿರುವಾಗ ಮಾತನಾಡುವುದಿಲ್ಲ. ನಾನು ನನ್ನ ಸಹೋದರಿಯರೆದುರಿಗೆ ಅವರ ಗಂಡಂದಿರ ವಿಚಾರ ಮಾತನಾಡಲೇಕೂಡದು. ಇದರ ಹಿಂದೆ ಇರುವ ಒಂದು ಭಾವನೆ ಇದು- ನಾವೊಂದು ತಪಸ್ವಿಗಳ ಜನಾಂಗ, ಸಮಾಜದ ವರ್ಣಾಶ್ರಮ\-ಗಳೆದುರಿಗೆ ಈ ಒಂದು ಆದರ್ಶವಿದೆ. ಮದುವೆ ಎಂಬುದು ಅಪರಿಶುದ್ಧವಾದ, ಅಷ್ಟು ಉತ್ತಮವಲ್ಲದ ವಿಷಯ ಎಂದು ಭಾವಿಸುವರು. ಆದಕಾರಣವೇ ಪ್ರೀತಿಯ ವಿಚಾರವನ್ನು ಯಾವಾಗಲೂ ಮಾತನಾಡಕೂಡದು. ನಾನು ಕಾದಂಬರಿಗಳನ್ನು ಸಹೋದರ ಸಹೋದರಿ ತಾಯಿ ಮತ್ತು ಇನ್ನು ಕೆಲವರೆದುರಿಗೆ ಓದಕೂಡದು. ಅವರಿರುವಾಗ ಪುಸ್ತಕವನ್ನು\break ಮುಚ್ಚುತ್ತೇನೆ.

ತಿನ್ನುವುದು ಕುಡಿಯುವುದು ಇವೆಲ್ಲ ಅದೇ ಗುಂಪಿಗೆ ಸೇರಿವೆ. ಹಿರಿಯರ ಮುಂದೆ ನಾವು ಊಟ ಮಾಡುವುದಿಲ್ಲ. ನಮ್ಮ ಹೆಂಗಸರು ಗಂಡಸರ ಮುಂದೆ ಎಂದೂ ಊಟ ಮಾಡುವುದಿಲ್ಲ. ಮಕ್ಕಳು ಅಥವಾ ತಮಗಿಂತ ಕಿರಿಯವರ ಮುಂದೆ ಮಾಡಬಹುದು. ಗಂಡನ ಮುಂದೆ ಅಗಿಯುವುದಕ್ಕಿಂತ ಹೆಂಡತಿ ಪ್ರಾಣ ಬೇಕಾದರೆ ಬಿಟ್ಟಾಳು. ಕೆಲವು ವೇಳೆ ಸಹೋದರಿ ಮತ್ತು ಸಹೋದರರು ಊಟ ಮಾಡುತ್ತಿರುವಾಗ, ಅವಳ ಗಂಡ ಬಂದರೆ, ಅವಳು ಊಟ ಮಾಡುವುದನ್ನು ಬಿಡುವಳು. ಪಾಪ ಗಂಡ ಬಾಗಿಲಿನಿಂದ ಆಚೆ ಓಡಬೇಕಾಗುವುದು!

ಇವು ಆ ದೇಶದಲ್ಲಿ ಮಾತ್ರ ರೂಢಿಯಲ್ಲಿರುವ ಆಚಾರ ವ್ಯವಹಾರಗಳು\break ಅನ್ಯ ದೇಶಗಳಲ್ಲಿಯೂ ಇವುಗಳಲ್ಲಿ ಕೆಲವನ್ನು ಗಮನಿಸಿರುವೆನು. ನಾನು ಎಂದೂ\break ಮದುವೆಯಾಗದವನಾದುದರಿಂದ ಹೆಂಡತಿಗೆ ಸಂಬಂಧಪಟ್ಟ ಅನುಭವಗಳನ್ನು ಪೂರ್ಣವಾಗಿ ಹೇಳಲಾರೆ. ತಾಯಿ ಸಹೋದರಿ ಇವರ ವಿಷಯ ನನಗೆ ಗೊತ್ತಿದೆ. ಇತರರ ಹೆಂಡತಿಯರನ್ನು ನೋಡಿರುವೆನು. ಇವುಗಳ ಸಹಾಯದಿಂದ ನಾನು ನಿಮಗೆ ಹೇಳಿರುವುದನ್ನು ಸಂಗ್ರಹಿಸಿರುವೆನು.

ಸಂಸ್ಕೃತಿ, ವಿದ್ಯೆ ಇವೆಲ್ಲ ಪುರುಷರ ಜವಾಬ್ದಾರಿ. ಎಲ್ಲಿ ಪುರುಷರು ಸುಸಂಸ್ಕೃತರೋ ಅಲ್ಲಿ ಸ್ತ್ರೀಯರೂ ಸುಸಂಸ್ಕೃತರು. ಎಲ್ಲಿ ಪುರುಷರು ಹಾಗೆ ಇಲ್ಲವೋ ಅಲ್ಲಿ ಸ್ತ್ರೀಯರೂ ಹಾಗೆ ಇಲ್ಲ. ಹಿಂದೂ ಪದ್ಧತಿಯ ಪ್ರಕಾರ ಪ್ರಾಥಮಿಕ ವಿದ್ಯಾಭ್ಯಾಸ ಗ್ರಾಮಪಂಚಾಯತಿಗೆ ಸೇರಿದುದು. ಅನಾದಿಕಾಲದಿಂದಲೂ ಜಮೀನನ್ನೆಲ್ಲ ರಾಷ್ಟ್ರೀಯ ಸ್ವತ್ತನ್ನಾಗಿ ಮಾಡಿರುವರು. ಅದು ಸರ್ಕಾರಕ್ಕೆ ಸೇರಿದ್ದು. ಮಾಲೀಕನಿಗೆ ನೆಲದ ಮೇಲೆ ತನ್ನ ಸ್ವಂತ ಹಕ್ಕು ಎಂಬುದಿಲ್ಲ. ಭರತಖಂಡದಲ್ಲಿ ವರಮಾನವು ಭೂ ಕಂದಾಯದಿಂದ ಬರುವುದು. ಏಕೆಂದರೆ\break ಪ್ರತಿಯೊಬ್ಬನೂ ಸರ್ಕಾರದಿಂದ ಭೂಮಿಯನ್ನು ಪಡೆದುಕೊಂಡಿರುವನು. ನೆಲ\break ಕೂಡೊಕ್ಕಲಿಗೆ ಸೇರಿದೆ. ಅಲ್ಲಿ ಐದು, ಹತ್ತು, ಇಪ್ಪತ್ತು, ನೂರು ಸಂಸಾರಗಳಿರಬಹುದು. ಅವರೇ ನೆಲವನ್ನೆಲ್ಲಾ ಆಳುವರು. ಸರ್ಕಾರಕ್ಕೆ ಗೊತ್ತಾದ ಕಂದಾಯ ತೆರುವರು, ವೈದ್ಯ, ಉಪಾಧ್ಯಾಯ ಮುಂತಾದವರನ್ನು ನೇಮಿಸುವರು.

ಹರ್ಬರ್ಟ್​ ಸ್ಪೆನ್ಸರನ ಪುಸ್ತಕವನ್ನು ಓದಿದವರಿಗೆ, ಯೂರೋಪ್​ ದೇಶದಲ್ಲಿ ಪ್ರಯೋಗಮಾಡಿ ನೋಡಿದ ಸಂನ್ಯಾಸೀಮಠ ಪದ್ಧತಿಯ ವಿದ್ಯಾಭ್ಯಾಸ ನೆನಪಿನಲ್ಲಿರಬಹುದು. ಕೆಲವು ಕಡೆಗಳಲ್ಲಿ ಅದು ಯಶಸ್ವಿಯಾಗಿ ನೆರವೇರಿತು. ಇಂಡಿಯಾದಲ್ಲಿ ಹಳ್ಳಿಯವರು\break ನೇಮಿಸುವ ಉಪಾಧ್ಯಾಯನಿರುವನು. ಈ ಪ್ರಾಥಮಿಕ ಶಾಲೆಗಳು ಇನ್ನೂ ಆರಂಭದ ಸ್ಥಿತಿಯಲ್ಲಿಯೇ ಇವೆ. ನಮ್ಮ ರೀತಿ ಕೂಡ ಅತೀ ಸರಳ. ಪ್ರತಿಯೊಬ್ಬ ಹುಡುಗನು\break ಕುಳಿತುಕೊಳ್ಳುವುದಕ್ಕೆ ಚಾಪೆಯನ್ನು, ಬರೆಯುವುದಕ್ಕೆ ಓಲೆಗರಿಯನ್ನು ತರುವನು.\break ಮೊದಲು ಓಲೆಗರಿ, ಏಕೆಂದರೆ ಕಾಗದದ ಬೆಲೆ ಹೆಚ್ಚು. ಹುಡುಗರು ಚಾಪೆ ಹಾಸಿ ಅದರ ಮೇಲೆ ಕುಳಿತುಕೊಳ್ಳುವರು. ಮಸಿಕುಡಿಕೆ ಮತ್ತು ಪುಸ್ತಕವನ್ನು ತಂದು ಬರೆಯಲು ಮೊದಲು ಮಾಡುವರು. ಸ್ವಲ್ಪ ಲೆಕ್ಕ, ವ್ಯಾಕರಣ, ಭಾಷಾಜ್ಞಾನ ಮತ್ತು ಲೆಕ್ಕಾಚಾರ ಇವನ್ನು ಪ್ರಾಥಮಿಕ ವಿದ್ಯಾಭ್ಯಾಸದಲ್ಲಿ ಕಲಿಸುವರು.

ಒಬ್ಬ ವೃದ್ಧ ಉಪಾಧ್ಯಾಯರು ನನಗೆ ನೀತಿಯ ಒಂದು ಪಾಠವನ್ನು ಕಲಿಸಿದರು. ಅದನ್ನು ನಾವು ಹುಡುಗರಾಗಿದ್ದಾಗ ಕಂಠಪಾಠ ಮಾಡಿದೆವು. ಅದರ ಒಂದು ಪಾಠ\break ನನಗಿನ್ನೂ ನೆನಪಿನಲ್ಲಿರುವುದು.

\begin{myquote}
ಹಳ್ಳಿಯ ಹಿತಕ್ಕಾಗಿ ಒಬ್ಬ ತನ್ನ ಮನೆಯ ಹಿತವನ್ನು ತೊರೆಯಬೇಕು.\\ದೇಶದ ಹಿತಕ್ಕಾಗಿ ಹಳ್ಳಿಯ ಹಿತವನ್ನು ತೊರೆಯಬೇಕು.\\ಜನಾಂಗದ ಹಿತಕ್ಕೆ ತನ್ನ ದೇಶದ ಹಿತವನ್ನು ಬೇಕಾದರೆ ತೊರೆಯಬೇಕು.\\ಪ್ರಪಂಚದ ಕಲ್ಯಾಣಕ್ಕೆ ತನ್ನ ಸರ್ವಸ್ವವನ್ನೂ ತ್ಯಜಿಸಬೇಕು.
\end{myquote}

ಪುಸ್ತಕದಲ್ಲಿ ಈ ಅರ್ಥವನ್ನು ಕೊಡುವ ಹಲವು ಶ್ಲೋಕಗಳಿವೆ. ನಾವು ಇವನ್ನು ಕಂಠಪಾಠ ಮಾಡುವೆವು, ಉಪಾಧ್ಯಾಯರು ಇವನ್ನು ವಿವರಿಸುತ್ತಾರೆ. ಬಾಲಕ ಬಾಲಕಿಯರೆಲ್ಲರೂ ಈ ವಿಷಯಗಳನ್ನು ಕಲಿಯುತ್ತಾರೆ. ಅನಂತರ ವಿದ್ಯಾಭ್ಯಾಸ ಬೇರೆಯಾಗುವುದು. ಹಿಂದಿನ ಕಾಲದ ಸಂಸ್ಕೃತ ವಿಶ್ವವಿದ್ಯಾನಿಲಯಗಳಲ್ಲಿ ಹುಡುಗರೇ ಹೆಚ್ಚು. ಹುಡುಗಿಯರು ಅಲ್ಲಿಗೆ ಹೋಗುತ್ತಿದ್ದುದು ಅಪರೂಪ. ಆದರೆ ಕೆಲವು ಅಪವಾದಗಳಿವೆ.

ಈ ಆಧುನಿಕ ಕಾಲದಲ್ಲಿ ಯೂರೋಪ್​ ದೇಶದ ಮಾದರಿಯ ಪ್ರೌಢವಿದ್ಯಾಭ್ಯಾಸಕ್ಕೆ \break ಹೆಚ್ಚು ಪ್ರೋತ್ಸಾಹವಿದೆ. ಹೆಂಗಸರಲ್ಲಿ ಇದನ್ನು ಪಡೆಯುವವರ ಸಂಖ್ಯೆ ಹೆಚ್ಚುತ್ತಿದೆ. ಅದನ್ನು ವಿರೋಧಿಸಿದವರೂ ಇದ್ದರು. ಆದರೆ ಯಾರಿಗೆ ಅದರ ಮೇಲೆ ಅಭಿಮಾನವಿತ್ತೋ ಅವರೇ ಗೆದ್ದರು. ಆಕ್ಸ್​ಫರ್ಡ್​, ಕೇಂಬ್ರಿಡ್ಜ್​, ಹಾರವರ್ಡ್​ ಮತ್ತು ಏಲ್​ ವಿಶ್ವವಿದ್ಯಾನಿಲಯಗಳು ಇಂದಿಗೂ ಕೂಡ ಸ್ತ್ರೀಯರಿಗೆ ಪ್ರವೇಶ ಕೊಡದೆ ಇರುವುದು ಸೋಜಿಗ. ಆದರೆ ಕಲ್ಕತ್ತಾ ವಿಶ್ವವಿದ್ಯಾನಿಲಯ, ಇಪ್ಪತ್ತು ವರ್ಷಕ್ಕೆ ಮುಂಚೆಯೇ ಸ್ತ್ರೀಯರಿಗೆ ಪ್ರವೇಶ ಕೊಟ್ಟಿತು. ನಾನು ಪದವೀಧರನಾದ ವರ್ಷವೇ, ಹಲವು ಸ್ತ್ರೀಯರೂ ಅದನ್ನು ಗಳಿಸಿದರು. ಅವರಿಗೂ ಹುಡುಗರ ಪಾಠವೆ, ಅದೇ ದರ್ಜೆಯೆ. ಅವರು ಚೆನ್ನಾಗಿ ಯಶಸ್ಸನ್ನು ಗಳಿಸಿದರು. ಸ್ತ್ರೀ ವಿದ್ಯಾಭ್ಯಾಸಕ್ಕೆ ನಮ್ಮ ಧರ್ಮ ಆತಂಕವನ್ನು ತರುವುದಿಲ್ಲ. ಹೆಂಗಸರಿಗೆ ಈ ರೀತಿ ವಿದ್ಯೆ ಕೊಡಬೇಕು, ತರಬೇತಿ ಕೊಡಬೇಕು. ಹಳೆಯ ಗ್ರಂಥಗಳಲ್ಲಿ ವಿಶ್ವವಿದ್ಯಾ ನಿಲಯಗಳಿಗೆ ಪುರುಷರಂತೆ ಸ್ತ್ರೀಯರೂ ಬರುತ್ತಿದ್ದರು ಎನ್ನುವುದನ್ನು ಓದುತ್ತೇವೆ. ಅನಂತರ ಇಡೀ ದೇಶದಲ್ಲಿ ವಿದ್ಯಾಭ್ಯಾಸಕ್ಕೆ ಅಷ್ಟು ಗಮನ ಕೊಡಲಿಲ್ಲ. ಪರಕೀಯರ ಆಳ್ವಿಕೆಯಲ್ಲಿ ನೀವು ಮತ್ತೇನನ್ನು ನಿರೀಕ್ಷಿಸಬಲ್ಲಿರಿ? ಹೊರಗಿನಿಂದ ಬಂದು ಗೆದ್ದು ನಮ್ಮನ್ನು ಆಳುವುದು ನಮಗೆ ಒಳ್ಳೆಯದನ್ನು ಮಾಡುವುದಕ್ಕೆ ಅಲ್ಲ. ಅವರಿಗೆ ಧನ ಬೇಕಾಗಿದೆ. ನಾನು ಹನ್ನೆರಡು ವರ್ಷ ಕಷ್ಟಪಟ್ಟು ಕಲ್ಕತ್ತಾ ವಿಶ್ವವಿದ್ಯಾನಿಲಯದಲ್ಲಿ ಪದವೀಧರನಾದೆ. ಆದರೆ ನಾನು ಆಗ ನನ್ನ ದೇಶದಲ್ಲಿ ತಿಂಗಳಿಗೆ ಐದು ಡಾಲರುಗಳನ್ನು ಸಂಪಾದಿಸುವುದೂ ಕಷ್ಟವಾಗಿತ್ತು. ನೀವು ಇದನ್ನು ನಂಬಬಲ್ಲಿರಾ? ಇದು ನಿಜವಾಗಿಯೂ ಸತ್ಯ. ಅವರು ನಮ್ಮಲ್ಲಿ ಸ್ಥಾಪಿಸಿರುವ ವಿದ್ಯಾಸಂಸ್ಥೆಗಳು ಕಡಿಮೆ ಸಂಬಳದ ಕೆಲಸಗಾರರನ್ನು ತಯಾರು ಮಾಡುವುದಕ್ಕೆ, ಗುಮಾಸ್ತರ ಪಡೆ, ಅಂಚೆ ಇಲಾಖೆಯ ನೌಕರರು ಇತ್ಯಾದಿಗಳ ಉತ್ಪಾದನೆ, ಇಷ್ಟೆ ಅವುಗಳ ಗುರಿ.

ಹೀಗಾಗಿ ಹುಡುಗ ಹುಡುಗಿಯರ ವಿದ್ಯಾಭ್ಯಾಸದ ಕಡೆಗೆ ಗಮನವನ್ನೇ ಕೊಟ್ಟಿಲ್ಲ. ಸಂಪೂರ್ಣ ಅದನ್ನು ನಿರ್ಲಕ್ಷಿಸಿರುವರು. ಅಲ್ಲಿ ಎಷ್ಟೋ ಸುಧಾರಣೆಗಳನ್ನು ಮಾಡಬೇಕಾಗಿದೆ. ಆದರೆ ನೀವು ದಯವಿಟ್ಟು ಇದನ್ನು ನೆನಪಿನಲ್ಲಿಡಬೇಕು. ನಿಮ್ಮ ಒಂದು ಗಾದೆಯನ್ನೆ ಉಪಯೋಗಿಸುವುದಕ್ಕೆ ಅನುಮತಿ ಕೊಡಿ: “ಗಂಡುಜಾತಿಗೆ ಅನ್ವಯಿಸುವುದೇ ಹೆಣ್ಣುಜಾತಿಗೂ ಅನ್ವಯಿಸುತ್ತದೆ.” ಹೊರಗಿನ ಹೆಂಗಸರು ಭಾರತದ ಮಹಿಳೆಯರ ಕಷ್ಟವನ್ನು ಕುರಿತು ಯಾವಾಗಲೂ ಮಾತನಾಡುತ್ತಿರುವರು. ಹಿಂದೂ ಪುರುಷರ ಕಷ್ಟದ ನೆನಪೇ ಅವರಿಗೆ ಇರುವುದಿಲ್ಲ. ಅವರೆಲ್ಲ ಕಂಬನಿ ಕರೆಯುತ್ತಿರುವರು. ಸಣ್ಣ ಹುಡುಗಿಯರನ್ನು ಯಾರಿಗೆ ಮದುವೆ ಮಾಡುತ್ತಾರೆ? ಅಲ್ಲಿರುವ ಎಲ್ಲ ಹುಡುಗಿಯರನ್ನೂ ಮುದುಕರಿಗೆ ಮದುವೆ ಮಾಡುತ್ತಾರೆಯೆ-ಎಂದು ಕೇಳಿದ ಒಬ್ಬ ಪಾಶ್ಚಾತ್ಯನು, “ಏನು ಅಲ್ಲಿ ಹುಡುಗರು ಏನು ಮಾಡುತ್ತಾ ಇರುತ್ತಾರೆ? ಎಲ್ಲಾ ಹುಡುಗಿಯರನ್ನು ಮುದುಕರಿಗೆ ಮಾತ್ರ ಮದುವೆ ಮಾಡುತ್ತಾರೆಯೇ” ಎಂದನು. ನಾವು ಹುಟ್ಟು ವಯೋವೃದ್ಧರು. ಬಹುಶಃ ಗಂಡಸರೆಲ್ಲಾ ಅಲ್ಲಿ ಹಾಗೆಯೇ ಇರಬಹುದು!

ಭಾರತ ಜನಾಂಗದ ಆದರ್ಶ ಮುಕ್ತಿ. ಈ ಪ್ರಪಂಚದಲ್ಲಿ ಏನೂ ಹುರುಳಿಲ್ಲ.\break ಇದೊಂದು ದೃಶ್ಯ, ಕನಸು. ಈ ಒಂದು ಜನ್ಮ ಕಳೆದ ಕೋಟ್ಯಂತರ ಜನ್ಮಗಳಂತೆ ಇದೆ.\break ಪ್ರಕೃತಿಯೆಲ್ಲ ಮಾಯೆ, ಭ್ರಾಂತಿ, ಕೋಟಲೆ ಕೊಡುವ ಚಿತ್ರ ವಿಚಿತ್ರಾಕೃತಿಗಳು ತುಂಬಿ ತುಳುಕಾಡುತ್ತಿರುವ ಪ್ರದೇಶವಿದು. ಇದೇ ಹಿಂದೂ ತತ್ತ್ವ. ಮಕ್ಕಳು ಜೀವನವನ್ನು\break ನೋಡಿ ನಗುವುವು. ಎಷ್ಟು ಸುಂದರವಾಗಿದೆ ಸುಖವಾಗಿದೆ ಎಂದು ಭಾವಿಸುವುವು.\break ಆದರೆ ಕೆಲವು ವರ್ಷಗಳಅನುಭವವಾದ ಮೇಲೆ ಅವರು ಹೊರಟುಬಂದ ಸ್ಥಳಕ್ಕೆ\break ಪುನಃ ಹೋಗಬೇಕಾಗಿದೆ. ಅಳುತ್ತಾ ಜೀವನ ಪ್ರಾರಂಭಿಸಿದರು, ಅಳುತ್ತಲೇ ಜೀವನ\-ವನ್ನು ತೊರೆಯುವರು. ರಾಷ್ಟ್ರಗಳು ತಮ್ಮ ಯೌವನಾಧಿಕ್ಯ-ಶಕ್ತಿ-ಭ್ರಾಂತಿಯಿಂದ, ತಾವು ಏನುಬೇಕಾದರೂ ಸಾಧಿಸಬಲ್ಲೆವೆಂದು ಭಾವಿಸುವುವು. ನಾವೇ ಭೂಲೋಕದ ದೇವತೆಗಳು, ದೇವರ ಪ್ರೀತಿಗೆ ಪಾತ್ರರಾದವರು ಎಂದು ತಿಳಿಯುವರು. ದೇವರು ಅವರಿಗೆ ಪ್ರಪಂಚವನ್ನೆಲ್ಲಾ ಆಳುವಂತೆ, ಅವನ ನಿಯೋಜನೆಗಳನ್ನು ಕಾರ್ಯರೂಪಕ್ಕೆ ತರುವಂತೆ ತಮ್ಮ ಮನಸ್ಸಿಗೆ ಬಂದಂತೆ ಮಾಡುವುದಕ್ಕೆ, ಪ್ರಪಂಚವನ್ನು ತಲೆಕೆಳಗೆ ಮಾಡುವುದಕ್ಕೆ ಅಪ್ಪಣೆ ಕೊಟ್ಟಿರುವನು ಎಂದು ಭಾವಿಸುವರು. ದೇವರು ಅವರಿಗೆ, ದರೋಡೆಗೆ, ಕೊಲೆಗೆ ಸರ್ವಾಧಿಕಾರವನ್ನು ಕೊಟ್ಟಿರುವನೆಂದು ಭ್ರಮಿಸಿ ಹಾಗೆ ಮಾಡುವರು. ಪಾಪ, ಇನ್ನೂ ಅವರು ಮಕ್ಕಳು! ಚಕ್ರಾಧಿಪತ್ಯಗಳಾದ ಮೇಲೆ ಚಕ್ರಾಧಿಪತ್ಯಗಳು ಮೇಲೆದ್ದವು. ಕಣ್​ಮನಗಳನ್ನು\break ಅಪಹರಿಸುವ ಉಚ್ಛ್ರಾಯ ಸ್ಥಿತಿಗೆ ಏರಿದವು. ಈಗ ಮಾಯವಾದವು. ಎಲ್ಲಿಗೆ ಹೋದವೋ, ಅವುಗಳ ಸುಳಿವೇ ಇಲ್ಲ. ಅವುಗಳ ಅವನತಿಯೂ ಅಷ್ಟೇ ಬೀಭತ್ಸವಾಗಿರಬಹುದು.

ಕಮಲದ ಎಲೆಯ ಮೇಲೆ ಜಲಬಿಂದುವೊಂದು ಚಲಿಸಿ ಕ್ಷಣದಲ್ಲಿ ನೀರಿಗೆ ಬೀಳುವಂತೆ ಮಾನವರ ಜೀವನ ನಾವು ನೋಡಿದೆಡೆಯಲ್ಲಿ ನಾಶ. ಈಗ ದಟ್ಟಕಾನನವಿರುವ ಕಡೆ\break ಒಂದಾನೊಂದು ಕಾಲದಲ್ಲಿ ಪ್ರಬಲ ನಗರಗಳನ್ನೊಳಗೊಂಡ ಚಕ್ರಾಧಿಪತ್ಯವಿತ್ತು. ಇದೇ ಭಾರತೀಯ ಆಲೋಚನೆಯ ಧ್ವನಿ, ಪಲ್ಲವಿ. ನಿಮ್ಮ ಪಾಶ್ಚಾತ್ಯ ಜನಾಂಗದ ನಾಡಿಯಲ್ಲಿ ಇನ್ನೂ ಹೊಸ ರಕ್ತ ಹರಿಯುತ್ತಿದೆ. ರಾಷ್ಟ್ರಗಳಿಗೂ ವ್ಯಕ್ತಿಗೆ ಇರುವಂತೆ ಒಂದು ಕಾಲವಿದೆ. ಇಂದು ಗ್ರೀಸ್​ ಎಲ್ಲಿ? ರೋಮ್​ ಎಲ್ಲಿ? ಬಲಾಢ್ಯ ಸ್ಪೈಯಿನ್​ ಜನಾಂಗವೆಲ್ಲಿ? ಈ ಘಟನಾವಳಿಗಳು ಆಗುತ್ತಿರುವಾಗ ಭರತಖಂಡ ಏನಾಗಿತ್ತು? ರಾಷ್ಟ್ರಗಳು ಹುಟ್ಟುವುವು, ಸಾಯುವುವು, ಏಳುವುವು, ಬೀಳುವುವು. ಮಗುವಾಗಿರುವಾಗಲೇ ಹಿಂದೂವಿಗೆ ಪ್ರಪಂಚದಲ್ಲಿ ಯಾರೂ ತಡೆಗಟ್ಟಲಾರದ ಮೊಗಲ ಯೋಧನ ದಾಳಿಯ ನೆನಪಿದೆ. ಅದು “ಟಾರ್ಟರ್​” ಎಂಬ ಭಯಾನಕ ಪದವನ್ನು ನಿಮ್ಮ ಭಾಷೆಗೆ ಸೇರಿಸಿದೆ. ಹಿಂದೂ, ಬುದ್ಧಿವಾದವನ್ನು ಕಲಿತುಕೊಂಡಿರುವನು. ಈಗಿನ ಹುಡುಗರಂತೆ ಅವನು ಜಂಭಕೊಚ್ಚಿಕೊಳ್ಳಲು ಬಯಸುವುದಿಲ್ಲ. ಪಾಶ್ಚಾತ್ಯ ಜನರೆ! ನಿಮಗೆ ತೋರಿದುದನ್ನು ಹೇಳಿ. ಇದು ನಿಮ್ಮ ಕಾಲ. ಮಕ್ಕಳೇ, ಮುಂದೆ ಹೋಗಿ, ನಿಮ್ಮ ಹರಟೆಯನ್ನು ಮುಗಿಸಿ. ಹುಡುಗರು ಹರಟುವ ಕಾಲ ಇದು. ನಾವು ಬುದ್ಧಿ ಕಲಿತು ತೆಪ್ಪಗಿರುವೆವು. ನಿಮ್ಮಲ್ಲಿ ಈಗ ಸ್ವಲ್ಪ ಐಶ್ವರ್ಯವಿದೆ. ಅದಕ್ಕೆ ನಮ್ಮನ್ನು ನಿಕೃಷ್ಟ ದೃಷ್ಟಿಯಿಂದ ನೋಡುವಿರಿ. ಆಗಲಿ, ಇದು ನಿಮ್ಮ ಕಾಲ. ಮಕ್ಕಳ ಒಣಹರಟೆ, ಬಾಲಭಾಷೆ. ಇದೇ ಹಿಂದೂವಿನ ಮನೋಭಾವ.

ಜಗದೀಶ್ವರನನ್ನು ಶಬ್ದಜಾಲದಿಂದ ಪಡೆಯಲು ಅಸಾಧ್ಯ. ದೇವಾಧಿದೇವನನ್ನು ಬುದ್ಧಿಶಕ್ತಿಯಿಂದ ಪಡೆಯಲಾರೆವು. ಅನ್ಯರನ್ನು ಎದುರಿಸುವ ಶಕ್ತಿಯಿಂದಲೂ ಅವನನ್ನು ಪಡೆಯಲಾರೆವು. ಸೃಷ್ಟಿಯ ಆದಿರಹಸ್ಯವನ್ನು ಯಾರು ತಿಳಿದಿರುವರೋ, ಪ್ರಪಂಚದಲ್ಲಿ ಎಲ್ಲ ನಶ್ವರವೆಂಬ ಭಾವನೆ ಯಾರ ಎದೆಗೆ ತಟ್ಟಿದೆಯೋ, ಅವರಿಗೆ ಭಗವಂತ ದೊರಕುವುದು. ಇತರರಿಗೆ ಅಲ್ಲ. ಆನಾದಿಕಾಲದ ಅನುಭವದಿಂದ ಭರತಖಂಡ ಈ ಪಾಠವನ್ನು ಕಲಿತಿರುವುದು. ಅವನೆಡೆಗೆ ತನ್ನ ಮೊಗವನ್ನು ತಿರುಗಿಸಿರುವುದು. ಎಷ್ಟೋ ತಪ್ಪುಗಳನ್ನು ಆಕೆ ಮಾಡಿರುವಳು. ಜನಾಂಗದ ಮೇಲೆ ರಾಶಿ ರಾಶಿ ಕಸ ಕವಿದಿರುವುದು. ಚಿಂತೆಯಿಲ್ಲ. ಅದರಿಂದ ತೊಂದರೆ ಏನು? ಕಸವನ್ನು ಗುಡಿಸುವುದು, ನಗರವನ್ನು ನಿರ್ಮಲವಾಗಿಡುವುದೆಂದರೆ ಏನು? ಇದು ಚೇತನವನ್ನು ತುಂಬಬಲ್ಲದೆ? ಎಲ್ಲಿ ಸುವ್ಯವಸ್ಥಿತ ಸಂಸ್ಥೆಗಳಿವೆಯೋ ಅವೂ ನಾಶವಾಗುವುವು. ಸಂಸ್ಥೆಯಿಂದ ಪ್ರಯೋಜನವೇನು? ಪಾಶ್ಚಾತ್ಯ ನಾಜೂಕಿನ ಸಂಸ್ಥೆಗಳು ಐದು ದಿನಗಳಲ್ಲಿ ತಯಾರಾಗಿ ಆರನೆಯ ದಿನ ಒಡೆದುಹೋಗುವುವು. ಇಂತಹ ಅಲ್ಪ ಜನಾಂಗಗಳು ಎರಡು ಶತಮಾನಗಳಾದರೂ ಒಟ್ಟಿಗೆ ಕಲೆತು ಬಾಳಲಾರವು. ನಮ್ಮ ಸಂಸ್ಥೆಗಳಾದರೋ ಕಾಲದ ಪರೀಕ್ಷೆಯನ್ನು ಎದುರಿಸಿ ನಿಂತಿವೆ. ಹಿಂದೂ ಹೀಗೆ ಹೇಳುವರು: “ಹೌದು, ಪ್ರಪಂಚದಲ್ಲಿ ಕಳೆದುಹೋದ ಜನಾಂಗಗಳಿಗೆಲ್ಲಾ ಅಂತ್ಯಕ್ರಿಯೆಯನ್ನು ಮಾಡಿರುವೆವು. ಆಧುನಿಕ ಜನಾಂಗದ ಶವಸಂಸ್ಕಾರಕ್ಕೂ ಕಾದುಕುಳಿತಿರುವೆವು. ಇದಕ್ಕೆ ಕಾರಣ. ನಮ್ಮ ಗುರಿ ಇಹಲೋಕವಲ್ಲ, ಪರಲೋಕ. ನಿಮ್ಮ ಆದರ್ಶವಿದ್ದಂತೆ ನೀವು ಆಗುವಿರಿ ನಿಮ್ಮ ಆದರ್ಶ ನಶ್ವರವಾದರೆ, ಈ ಭೌತಿಕ ಪ್ರಪಂಚಕ್ಕೆ ಸಂಬಂಧಪಟ್ಟಿದ್ದರೆ, ನೀವೂ ಅದರಂತೆಯೆ ಆಗುವಿರಿ. ನಿಮ್ಮ ಆದರ್ಶ ಜಡ ವಸ್ತುವಾದರೆ, ನೀವೂ ಜಡವಾಗುವಿರಿ. ನೋಡಿ! ನಮ್ಮ ಆದರ್ಶ ಆತ್ಮ, ಅದೊಂದೇ ಶಾಶ್ವತ, ಉಳಿದವು ಯಾವುದೂ ಅಲ್ಲ. ಅದರಂತೆಯೇ ನಾವೂ ಶಾಶ್ವತವಾಗಿ ಬಾಳುವೆವು.”

