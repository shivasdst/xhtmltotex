
\chapter[ಸುಗುಣ ಮತ್ತು ನಿರ್ಗುಣ ಬ್ರಹ್ಮ ]{ಸುಗುಣ ಮತ್ತು ನಿರ್ಗುಣ ಬ್ರಹ್ಮ \protect\footnote{\engfoot{C.W. Vol. VIII, P. 188}}}

ನೀವು ಯಾರನ್ನು ಸಗುಣ ಪರಮೇಶ್ವರ ಎನ್ನುವಿರೋ ಅವನೇ ನಿರ್ಗುಣ ಬ್ರಹ್ಮ. ಅವನು ಏಕಕಾಲದಲ್ಲಿ ಸಗುಣ ಮತ್ತು ನಿರ್ಗುಣ ಆಗಿರುವನು. ನಾವೆಲ್ಲಾ ಸಾಕಾರವೆತ್ತ ನಿರಾಕಾರಗಳು. ನೀವು ನಿರಪೇಕ್ಷ ದೃಷ್ಟಿಯಿಂದ ಹೇಳಿದಾಗ ನಿರ್ಗುಣ, ಸಾಪೇಕ್ಷ ದೃಷ್ಟಿಯಿಂದ ಹೇಳಿದಾಗ ಸಗುಣ. ನಿಮ್ಮಲ್ಲಿ ಪ್ರತಿಯೊಬ್ಬರೂ ವಿಶ್ವ ಮಾನವರು, ಸರ್ವವ್ಯಾಪಿಗಳು. ನಿಮಗೆ ಮೊದಲು ಇದು ಊಹಿಸುವುದಕ್ಕೂ ಅಸಾಧ್ಯವಾಗಬಹುದು, ಆದರೆ ಅದು ನಾನು ನಿಮ್ಮ ಎದುರಿಗೆ ಇರುವಷ್ಟೇ ಸತ್ಯ. ಆತ್ಮವು ಹೇಗೆ ಸರ್ವವ್ಯಾಪಿಯಾಗಿಲ್ಲದೇ ಇರಬಲ್ಲದು? ಅದಕ್ಕೆ ಉದ್ದ ಇಲ್ಲ, ಅಗಲ ಇಲ್ಲ, ಗಾತ್ರ ಇಲ್ಲ; ದ್ರವ್ಯಕ್ಕೆ ಅನ್ವಯಿಸುವ ಯಾವ ಗುಣವೂ ಅದಕ್ಕೆ ಇಲ್ಲ. ನಾವೆಲ್ಲಾ ಆತ್ಮವಾಗಿದ್ದರೆ ಆಕಾಶವು ನಮಗೆ ಮಿತಿಯನ್ನು ಕಲ್ಪಿಸಲಾರದು. ದೇಶವು ಮಾತ್ರ ದೇಶವನ್ನು ಮಿತಗೊಳಿಸಬಲ್ಲದು, ದ್ರವ್ಯ ಮಾತ್ರ ದ್ರವ್ಯವನ್ನು ಮಿತಿಗೊಳಿಸಬಲ್ಲದು, ನಾವು ಈ ದೇಹಕ್ಕೆ ಮಾತ್ರ ಸೇರಿದ್ದರೆ ನಾವು ಬರಿಯ ದ್ರವ್ಯವಾಗುವೆವು. ದೇಹ ಆತ್ಮ ಎಲ್ಲಾ ದ್ರವ್ಯಮಯವಾಗುವುದು. ದೇಹದಲ್ಲಿ ವಾಸಮಾಡುವುದು, ದೇಹವನ್ನು ಧರಿಸುವುದು, ಮುಂತಾದುವೆಲ್ಲ ಕೇವಲ ಅನುಕೂಲಕ್ಕಾಗಿ ಉಪಯೋಗಿಸುತ್ತಿರುವ ಪದಗಳಾಗುವುವು. ಇದಲ್ಲದೆ ಅದಕ್ಕೆ ಬೇರೆ ಏನೂ ಅರ್ಥ ಇರುವುದಿಲ್ಲ. ನಿಮ್ಮಲ್ಲಿ ಅನೇಕರಿಗೆ ನಾನು ಕೊಟ್ಟ ಆತ್ಮನ ವಿವರಣೆ ಜ್ಞಾಪಕದಲ್ಲಿರಬಹುದು. ಪ್ರತಿಯೊಂದು ಜೀವವೂ ಒಂದು ವೃತ್ತದಂತೆ, ಅದರ ಕೇಂದ್ರ ಒಂದು ಕಡೆ ಇರುವುದು, ಪರಿಧಿ ಎಲ್ಲಿಯೂ ಇಲ್ಲ. ದೇಹವಿರುವ ಕಡೆ ಕೇಂದ್ರವಿದೆ. ಚಟುವಟಿಕೆ ಅಲ್ಲಿ ನಡೆಯುತ್ತಿರುವುದು. ನೀನು ವಿಶ್ವವ್ಯಾಪಿ, ಆದರೆ ಯಾವುದೋ ಒಂದು ಸ್ಥಳದಲ್ಲಿ ಮಾತ್ರ ‘ನಾನು ಇರುವೆನು’ ಎಂದು ಭಾವಿಸುತ್ತೀಯೆ. ಆ ಸ್ಥಳವು ದ್ರವ್ಯಗಳನ್ನು ಸೆಳೆದುಕೊಂಡು ತಾನು ವ್ಯಕ್ತವಾಗು ವುದಕ್ಕೆ ಒಂದು ಯಂತ್ರವನ್ನು ಮಾಡಿಕೊಂಡಿರುವುದು. ಯಾವುದರ ಮೂಲಕ ಇದು ವ್ಯಕ್ತವಾಗುವುದೋ ಅದನ್ನೇ ದೇಹ ಎನ್ನುವುದು. ಆದಕಾರಣ ನೀನು ಎಲ್ಲಾ ಕಡೆಯೂ ಇರುವೆ. ಒಂದು ದೇಹ ಅಥವಾ ಯಂತ್ರ ಅಪ್ರಯೋಜಕವಾದರೆ ಕೇಂದ್ರ ಬದಲಾಗುವುದು. ಬೇರೊಂದು ಕೇಂದ್ರವನ್ನು ತೆಗೆದುಕೊಂಡು ಸ್ಥೂಲ ಅಥವಾ ಸೂಕ್ಷ್ಮ ದ್ರವ್ಯದ ಮೂಲಕ ಮತ್ತೊಂದು ಯಂತ್ರವನ್ನು ಮಾಡಿ ಕೊಳ್ಳುವುದು. ಅದೇ ಮನುಷ್ಯ. ದೇವರೆಂದರೇನು? ದೇವರು ಅಂದರೆ ಒಂದು ವೃತ್ತ. ಅದಕ್ಕೆ ಪರಿಧಿ ಎಲ್ಲಿಯೂ ಇಲ್ಲ; ಕೇಂದ್ರ ಎಲ್ಲೆಲ್ಲಿಯೂ ಇರುವುದು. ಆ ವೃತ್ತದಲ್ಲಿ ಪ್ರತಿಯೊಂದು ಬಿಂದುವು ಸಚೇತನವಾಗಿರುವುದು, ಜೀವಿಸಿರುವುದು, ಕೆಲಸ ಮಾಡುತ್ತಿರುವುದು. ಮಿತಿಯುಳ್ಳ ಜೀವಿಗಳಾದ ನಮ್ಮಲ್ಲಿ ಒಂದು ಕೇಂದ್ರ ಮಾತ್ರ ಸಚೇತನವಾಗಿರುವುದು. ಅದು ಮುಂದಕ್ಕೂ ಹಿಂದಕ್ಕೂ ಚಲಿಸುತ್ತಿರು ವುದು. ವಿಶ್ವದ ದೃಷ್ಟಿಯಿಂದ ನೋಡಿದರೆ ದೇಹದ ಅಸ್ತಿತ್ವವು ಅತ್ಯಂತ ಕಿರಿ ದಾದದ್ದು. ಇದರಂತೆಯೇ ಇಡೀ ವಿಶ್ವವನ್ನು ದೇವರೊಡನೆ ಹೋಲಿಸಿದಾಗ ಅದು ಏನೇನೂ ಅಲ್ಲ. ದೇವರು ಮಾತನಾಡುವನು ಎಂದಾಗ ಅವನು ತನ್ನ ವಿಶ್ವದ ಮೂಲಕ ಮಾತನಾಡುವನು ಎನ್ನುವೆವು. ಅವನು ಕಾಲದೇಶಾತೀತ ಎಂದಾಗ ಅವನನ್ನು ನಿರ್ಗುಣ ಎನ್ನುವೆವು. ಆದರೂ ಅವನು ಒಬ್ಬನೇ ಆಗಿರುವನು.

ಒಂದು ಉದಾಹರಣೆಯನ್ನು ಕೊಡೋಣ. ನಾವು ಇಲ್ಲಿ ನಿಂತುಕೊಂಡು ಸೂರ್ಯನನ್ನು ನೋಡುತ್ತಿರುವೆವು. ನೀವು ಸೂರ್ಯನ ಕಡೆಗೆ ಹೋಗಬೇಕು ಎಂದು ಇಟ್ಟುಕೊಳ್ಳಿ. ನೀವು ಸಾವಿರ ಮೈಲುಗಳಷ್ಟು ಸೂರ್ಯನ ಹತ್ತಿರ ಹೋದರೆ ಅಲ್ಲಿ ನಿಮಗೆ ಬೇರೊಂದು ದೊಡ್ಡದಾಗಿರುವ ಸೂರ್ಯ ಕಾಣಿಸುವನು. ನೀವು ಅಲ್ಲಿಂದ ಇನ್ನೂ ಮುಂದಕ್ಕೆ ಹೋದರೆ ಮತ್ತೂ ದೊಡ್ಡ ಸೂರ್ಯನನ್ನು ನೋಡುವಿರಿ. ನೀವು ಈ ಪ್ರಯಾಣವನ್ನು ಹಲವು ಘಟ್ಟಗಳಾಗಿ ಮಾತಿ ಪ್ರತಿಯೊಂದು ಸ್ಥಳದಿಂದಲೂ ಒಂದು ಸೂರ್ಯನ ಫೋಟೋವನ್ನು ತೆಗೆದುಕೊಂಡು, ಕೊನೆಗೆ ನಿಜವಾದ ಸೂರ್ಯನ ಫೋಟೋವನ್ನು ತೆಗೆದುಕೊಂಡು, ಒಂದನ್ನೊಂದು ಹೋಲಿಸಿ ನೋಡಿದರೆ ಅವೆಲ್ಲ ಬೇರೆ ಬೇರೆಯಾಗಿ ಕಾಣಿಸುವುವು. ಮೊದಲನೆ ಯದು ಕೆಂಪು ಚೆಂಡಿನ ತರಹ ಇತ್ತು. ಕೊನೆಯದು ಲಕ್ಷಾಂತರ ಮೈಲಿ ವಿಸ್ತೀರ್ಣ ಉಳ್ಳದ್ದು. ಆದರೂ ಎಲ್ಲವೂ ಒಂದೇ ಸೂರ್ಯನಿಗೆ ಸಂಬಂಧಪಟ್ಟವುಗಳು. ಇದರಂತೆಯೇ ದೇವರೂ ಕೂಡ. ಪರಬ್ರಹ್ಮನನ್ನು ಬೇರೆ ಬೇರೆ ದೃಷ್ಟಿಗಳಿಂದ ನೋಡುತ್ತಿರುವೆವು. ಬೇರೆ ಬೇರೆ ಮನೋಕ್ಷೇತ್ರಗಳಿಂದ ನೋಡುತ್ತಿರುವೆವು. ಅತಿ ಪ್ರಾಚೀನ ಮಾನವ ಅವನನ್ನು ತನ್ನ ಪಿತೃಗಳಂತೆ ಕಾಣುವನು; ಅವನ ದೃಷ್ಟಿ ವಿಶಾಲವಾದಂತೆ ವಿಶ್ವೇಶ್ವರನಂತೆ ನೋಡುವನು. ಶ್ರೇಷ್ಠತಮನಾದ ಮಾನವನು ತಾನೇ ಅವನು ಎಂದು ಭಾವಿಸುವನು. ಇರುವವನು ಒಬ್ಬನೇ ದೇವರು, ನಮಗೆ ಕಾಣಿಸುವ ಬೇರೆ ಬೇರೆ ದೃಶ್ಯಗಳೆಲ್ಲಾ ಒಂದೇ ದೇವರ ವಿಭಿನ್ನ ನೋಟಗಳು ಅಷ್ಟೆ.

