
\chapter[ಭಾಗವದ್ಗೀತಾ (೩) ]{ಭಾಗವದ್ಗೀತಾ (೩) \protect\footnote{\engfoot{C.W. Vol. I, P. 467}}}

\centerline{(೧೯೦೦ರ ಮೇ ೨೯ ರಂದು ಸ್ಯಾನ್​ ಫ್ರಾನ್ಸಿ ಸ್ಕೋದಲ್ಲಿ ನೀಡಿದ ಉಪನ್ಯಾಸ)}

ಅರ್ಜುನ: “ನೀನು ಈಗ ತಾನೇ ಕರ್ಮವನ್ನು ಮಾಡು ಎಂದು ಹೇಳಿದೆ. ಆದರೂ ಬ್ರಹ್ಮಜ್ಞಾನವನ್ನೇ ಶ್ರೇಷ್ಠ ಎನ್ನುವೆ. ಜ್ಞಾನವೇ ಕರ್ಮಕ್ಕಿಂತ ಮೇಲು ಎಂದು ಭಾವಿಸಿ ದರೆ ಕರ್ಮವನ್ನು ಮಾಡು ಎಂದು ಏಕೆ ಹೇಳುವೆ?” (\enginline{III, 1})

ಶ‍್ರೀಕೃಷ್ಣ: “ಹಿಂದಿನಿಂದ ಈ ಎರಡು ಸಿದ್ಧಾಂತಗಳು ಬಂದಿವೆ. ಸಾಂಖ್ಯರು ಜ್ಞಾನವನ್ನು ಹೇಳುತ್ತಾರೆ. ಯೋಗಿಗಳು ಕರ್ಮವನ್ನು ಹೇಳುತ್ತಾರೆ. ಆದರೆ ಸುಮ್ಮನೆ ಕರ್ಮವನ್ನು ತ್ಯಜಿಸಿದರೆ ಯಾರಿಗೂ ಶಾಂತಿ ದೊರಕಲಾರದು. ಈ ಪ್ರಪಂಚದಲ್ಲಿ ಯಾರೂ ಒಂದು ನಿಮಿಷವೂ ಕರ್ಮವನ್ನು ತ್ಯಜಿಸಲಾರರು. ಪ್ರಕೃತಿಯ ಗುಣಗಳೇ ಅವನನ್ನು ಕೆಲಸ ಮಾಡುವಂತೆ ಪ್ರೇರೇಪಿಸುತ್ತವೆ. ಯಾರು ಕೆಲಸವನ್ನು ಬಿಡುತ್ತಾರೆಯೋ, ಆದರೆ ಅವುಗಳನ್ನು ಕುರಿತು ಮನಸ್ಸಿನಲ್ಲಿ ಚಿಂತಿಸುತ್ತಿರುವರೋ ಅವರು ಮಿಥ್ಯಾಚಾರಿಗಳಾಗುತ್ತಾರೆ. ಆದರೆ ಯಾರು ತಮ್ಮ ಮನೋಶಕ್ತಿಯಿಂದ ಇಂದ್ರಿಯಗಳನ್ನು ನಿಗ್ರಹಿಸಿ ಕರ್ಮವನ್ನು ಮಾಡುತ್ತಾರೆಯೋ ಅವರೇ ಯೋಗ್ಯರು. ಆದಕಾರಣ ಕರ್ಮವನ್ನೇ ಮಾಡು.” (\enginline{II, 2-8})

“ನಿನಗೆ ಯಾವ ಕರ್ತವ್ಯಗಳೂ ಇಲ್ಲ. ನೀನು ಮುಕ್ತ ಎಂದು ಅರಿತಿದ್ದರೂ ಲೋಕಕಲ್ಯಾಣಕ್ಕಾಗಿ ಕರ್ಮ ಮಾಡಬೇಕಾಗಿದೆ. ಏಕೆಂದರೆ ಶ್ರೇಷ್ಠರು ಯಾವುದನ್ನು ಮಾಡುವರೊ ಅದನ್ನೇ ಇತರರು ಅನುಸರಿಸುವರು. ಯಾವ ಶ್ರೇಷ್ಠನಿಗೆ ಶಾಂತಿ ಲಭಿಸಿದೆಯೋ ಯಾರು ಮುಕ್ತಾತ್ಮನಾಗಿರುವನೋ ಅವನು ಕೆಲಸವನ್ನು ಮಾಡದೇ ಇದ್ದರೆ ಶಾಂತಿ ಮತ್ತು ಜ್ಞಾನವಿಲ್ಲದ ಇತರರು ಅವನನ್ನು ಅನುಕರಿಸಲು ಯತ್ನಿಸುವರು. ಇದರಿಂದ ಲೋಕಕ್ಕೆ ಹಾನಿಯಾಗುವುದು.” (\enginline{II, 20-24})

“ನೋಡು ಅರ್ಜುನ, ನನಗೆ ಇಲ್ಲದೇ ಇರುವುದು ಯಾವುದೂ ಇಲ್ಲ, ಬೇಕಾಗಿರುವುದು ಯಾವುದೂ ಇಲ್ಲ, ಆದರೂ ನಾನು ಕರ್ಮ ಮಾಡುತ್ತಿರುವೆನು. ನಾನು ಒಂದು ಕ್ಷಣ ಕರ್ಮವನ್ನು ಬಿಟ್ಟರೆ ಲೋಕವೇ ನಾಶವಾಗುವುದು. ಅಜ್ಞರು ಯಾವ ಕೆಲಸವನ್ನು ಫಲಾಪೇಕ್ಷೆ ಮತ್ತು ಲಾಭದಾಸೆಯಿಂದ ಮಾಡುವರೋ ಅದನ್ನೇ ಜ್ಞಾನಿಗಳು ಯಾವ ಆಸಕ್ತಿಯೂ ಇಲ್ಲದೇ, ಫಲ ಮತ್ತು ಲಾಭದ ಆಸೆಯಿಲ್ಲದೆ ಮಾಡಲಿ.” (\enginline{III, 25})

“ನಿನಗೆ ಜ್ಞಾನವಿದ್ದರೂ ಅಜ್ಞಾನಿಗಳ ಬಾಲಬುದ್ಧಿಯನ್ನು ಕೆಡಿಸಬೇಡ. ಅದರ ಬದಲು ಅವರು ಯಾವ ಮೆಟ್ಟಿಲಲ್ಲಿ ಇರುವರೋ ಅಲ್ಲಿಗೆ ಹೋಗಿ ಅವರನ್ನು ಮೇಲೆತ್ತಲು ಯತ್ನಿಸು.” (\enginline{III, 26-29}) ಇದೊಂದು ಅತಿ ಪ್ರಬಲವಾದ ಭಾವನೆ. ಇದು ಭರತಖಂಡದ ಆದರ್ಶವಾಗಿದೆ. ಆದ ಕಾರಣವೇ ಮಹಾ ತತ್ತ್ವಜ್ಞಾನಿಯೂ ದೇವಸ್ಥಾನಕ್ಕೆ ಹೋಗಿ ವಿಗ್ರಹಗಳನ್ನು ಪೂಜಿಸುವುದನ್ನು ನೋಡುವಿರಿ. ಇದೊಂದು ನಟನೆ ಅಲ್ಲ.

“ಯಾರು ಇತರ ದೇವತೆಗಳನ್ನು ಪೂಜಿಸುತ್ತಿರುವರೋ ಅವರೂ ಕೂಡ ನನ್ನನ್ನೇ ನಿಜವಾಗಿಯೂ ಪೂಜಿಸುತ್ತಿರುವರು” (\enginline{IX, 23}) ಎಂದು ಅನಂತರ ಶ‍್ರೀಕೃಷ್ಣ ಹೇಳುವುದನ್ನು ಓದುತ್ತೇವೆ. ಮನುಷ್ಯ ಪೂಜೆ ಮಾಡುತ್ತಿರುವುದು ದೇವರನ್ನು. ಬೇರೆ ಹೆಸರಿನಿಂದ ಕರೆದರೆ ದೇವರಿಗೆ ಕೋಪ ಬರುವುದೆ? ಅವನಿಗೆ ಹಾಗೆ ಕೋಪ ಬಂದರೆ ಅವನು ದೇವರೆ ಅಲ್ಲ. ಮಾನವನ ಹೃದಯದಲ್ಲಿ ಏನಿದೆಯೋ ಅದೇ ದೇವರು ಎಂಬುದು ನಮಗೆ ಗೊತ್ತಾಗುವುದಿಲ್ಲವೇ? ಅವನು ಒಂದು ಕಲ್ಲನ್ನಾದರೂ ಪೂಜಿಸಲಿ! ಅದರಿಂದೇನು?

ಧರ್ಮ ಎಂದರೆ ಕೆಲವು ಸಿದ್ಧಾಂತಗಳು ಎಂಬ ಭಾವನೆಯಿಂದ ಒಮ್ಮೆ ನಾವು ಪಾರಾದರೆ ಇದನ್ನು ಸ್ಪಷ್ಟವಾಗಿ ತಿಳಿದುಕೊಳ್ಳಬಲ್ಲೆವು. ಕ್ರೈಸ್ತ ಧರ್ಮದ ಒಂದು ಭಾವನೆಯೇ - ಆಡಮ್​ ಸೇಬಿನ ಹಣ್ಣನ್ನು ತಿಂದನು, ಅದಕ್ಕಾಗಿ ಪ್ರಪಂಚ ಬಂದಿತು, ಇದರಿಂದ ನಾವು ಪಾರಾಗಲಾರೆವು ಎಂಬುದು. ಕ್ರಿಸ್ತನನ್ನು ನಂಬಿ! ಯಾರೋ ಒಬ್ಬ ಮನುಷ್ಯ ಸತ್ತ ಎಂಬುದನ್ನು ನಂಬಿ! ಆದರೆ ಇಂಡಿಯಾ ದೇಶದ ಭಾವನೆ ಸಂಪೂರ್ಣವಾಗಿ ಬೇರೆ. ಅಲ್ಲಿ ಧರ್ಮ ಎಂದರೆ ಸಾಕ್ಷಾತ್ಕಾರವಲ್ಲದೆ ಬೇರೇನೂ ಅಲ್ಲ. ಒಬ್ಬ ಗುರಿಯನ್ನು ನಾಲ್ಕು ಕುದುರೆಯ ಸಾರೋಟಿನ ಮೇಲೆ ಸೇರುತ್ತಾನೆಯೋ, ವಿದ್ಯುತ್​ ವಾಹನದಲ್ಲಿ ಸೇರುತ್ತಾನೆಯೋ ಅಥವಾ ಉರುಳಿಕೊಂಡು ಹೋಗಿ ಸೇರುತ್ತಾನೆಯೊ ಎಂಬುದಲ್ಲ ಮುಖ್ಯ. ಹೇಗೆ ಹೋದರೂ ಗುರಿ ಒಂದೇ. ಕ್ರೈಸ್ತರಿಗೆ ಭಯಾನಕವಾದ ದೇವರ ಕೋಪದಿಂದ ಹೇಗೆ ಪಾರಾಗಬೇಕೆಂಬುದೇ ಪ್ರಶ್ನೆ. ಆದರೆ ಹಿಂದೂಗಳಿಗಾದರೊ ತಾವು ತಾತ್ತ್ವಿಕವಾಗಿ ಏನಾಗಿರುವರೋ ಅದರಂತೆ ಸಹಜವಾಗಿ ಆಗುವುದು. ತಾವು ಕಳೆದುಕೊಂಡ ಆತ್ಮನ ಸ್ಥಾನವನ್ನು ಪಡೆಯುವುದಾಗಿದೆ.

ನೀವು ಆತ್ಮ ಎಂಬುದನ್ನು ಸಾಕ್ಷಾತ್ಕಾರ ಮಾಡಿಕೊಂಡಿರುವಿರಾ? ಹೌದು ನನಗೆ ಆಗಿದೆ ಎಂದರೆ, ಅದರ ಅರ್ಥವೇನು? ಅತ್ಮನು ಈ ದೇಹವೆಂಬ ಮಾಂಸದ ಮುದ್ದೆಯೇ ಅಥವಾ ಅನಂತವಾದ ನಿತ್ಯ ಮಂಗಳವಾದ ಸ್ವಯಂಪ್ರಭೆಯಿಂದ ಕೂಡಿದ ಅಮೃತಾತ್ಮನೆ? ನೀವು ಪ್ರಪಂಚದಲ್ಲೆ ಶ್ರೇಷ್ಠ ತತ್ತ್ವಜ್ಞಾನಿ ಆಗಿರಬಹುದು. ಆದರೆ ಎಲ್ಲಿಯವರೆಗೆ ನಾನೆ ದೇಹ ಎಂಬ ಭಾವನೆ ಇದೆಯೋ ಅಲ್ಲಿಯವರೆಗೆ ನೀವು ನಿಮ್ಮ ಕಾಲ ಕೆಳಗೆ ಹರಿದಾಡುವ ಸಣ್ಣ ಕೀಟಕ್ಕಿಂತ ಮೇಲಲ್ಲ. ನಿಮಗೆ ಕ್ಷಮೆ ಇಲ್ಲ. ನಿಮ್ಮದು ಮಹಾಪರಾಧ. ನಿಮಗೆ ತತ್ತ್ವವೆಲ್ಲ ಗೊತ್ತಿದೆ, ಆದರೂ ನೀವು ದೇಹವೆಂದು ಪರಿಗಣಿಸುವಿರಿ! ನೀವು ದೇಹ! ದೇವತೆಗಳು ಆಗಿರುವಿರಿ! ಅದೊಂದು ಧರ್ಮವೇ?

ಧರ್ಮವೆಂದರೆ ಆತ್ಮನನ್ನು ಅತ್ಮನಂತೆ ಅರಿಯುವುದು. ಈಗ ನಾವೇನು ಮಾಡುತ್ತಿರುವೆವು? ಅದಕ್ಕೆ ವಿರುದ್ಧವಾದುದನ್ನು. ಆತ್ಮನನ್ನು ದೇಹವೆಂದು ಭಾವಿಸುತ್ತಿರುವೆವು. ಆ ಅಮೃತಾತ್ಮನಿಂದ ನಾವು ಮೃತ್ಯು ಮತ್ತು ಪಂಚಭೂತಗಳನ್ನು ನಿರ್ಮಿಸುತ್ತೇವೆ; ಜಡವಾದ ಅಚೇತನವಾದ ಪಂಚಭೂತಗಳಿಂದ ನಾವು ಆತ್ಮನನ್ನು ನಿರ್ಮಿಸುತ್ತಿರುವೆವು. ತಲೆಯ ಮೇಲೆ ನಿಂತುಕೊಳ್ಳುವುದರಿಂದ ಅಥವಾ ಒಂದು ಕಾಲಿನ ಮೇಲೆ ನಿಂತುಕೊಳ್ಳುವುದರಿಂದ ಅಥವಾ ಮೂರು ತಲೆಗಳಿರುವ ಐದು ಸಾವಿರ ದೇವರ ಪೂಜೆಯಿಂದ ನಿಮಗೆ ಬ್ರಹ್ಮಜ್ಞಾನ ಲಭಿಸುವುದಾದರೆ ಒಳ್ಳೆಯದು...! ನಿಮಗೆ ಹೇಗೆ ಬೇಕೊ ಹಾಗೆ ಮಾಡಿ. ಅದಕ್ಕೆ ವಿರೋಧವಾಗಿ ಹೇಳುವುದಕ್ಕೆ ಯಾರಿಗೂ ಏನೂ ಅಧಿಕಾರವಿಲ್ಲ. ಅದಕ್ಕೇ ಕೃಷ್ಣ ಹೇಳುವುದು, ನಿನ್ನ ಮಾರ್ಗ ಉತ್ತಮವಾಗಿದ್ದರೆ, ಮತ್ತೊಬ್ಬನ ಮಾರ್ಗ ಎಷ್ಟೇ ಹೀನವಾಗಿದ್ದರೂ, ಅದು ತಪ್ಪು ಎಂದು ಹೇಳುವುದಕ್ಕೆ ನಿನಗೆ ಅಧಿಕಾರವಿಲ್ಲ ಎಂದು.

ಧರ್ಮ ಎಂದರೆ ಒಂದು ವಿಧವಾದ ಬೆಳವಣಿಗೆ, ಅದು ಕೆಲಸಕ್ಕೆ ಬಾರದ ಪದಪುಂಜವಲ್ಲ ಎಂಬುದನ್ನು ಗಮನಿಸಬೇಕಾಗಿದೆ. ಎರಡು ಸಾವಿರ ವರ್ಷಗಳ ಹಿಂದೆ ಒಬ್ಬ ದೇವರನ್ನು ನೋಡಿದ - ಮೋಸಸ್​ ಉರಿಯುತ್ತಿರುವ ಪೊದೆಯಲ್ಲಿ ದೇವರನ್ನು ನೋಡಿದ. ಅವನು ದೇವರನ್ನು ನೋಡಿದಾಗ ಏನು ಮಾಡಿದನೋ ಅದು ನಿಮ್ಮನ್ನು ಉದ್ಧರಿಸಬಲ್ಲದೇ? ಯಾರು ದೇವರನ್ನು ನೋಡಿದರೂ ಅದರಿಂದ ನಿಮಗೆ ಎಳ್ಳಿನಷ್ಟೂ ಪ್ರಯೋಜನವಾಗದು. ಅದು ನಿಮ್ಮನ್ನೂ ಕೂಡ ಹಾಗೆ ಮಾಡುವಂತೆ ಪ್ರಚೋದಿಸ ಬಹುದೇ ಹೊರತು ಬೇರೇನೂ ಇಲ್ಲ. ದಾರಿಯಲ್ಲಿರುವ ಕೈಮರದಂತೆ ಇವು. ಒಬ್ಬ ಊಟ ಮಾಡಿದರೆ ಮತ್ತೊಬ್ಬನಿಗೆ ಹೊಟ್ಟೆ ತುಂಬುವುದಿಲ್ಲ. ನೀವೇ ದೇವರನ್ನು ನೋಡಬೇಕಾಗಿದೆ. ದೇವರ ಸ್ವಭಾವ ಏನು? ಅವನಿಗೆ ಒಂದು ದೇಹದಲ್ಲಿ ಮೂರು ತಲೆಗಳಿವೆಯೇ, ಅಥವಾ ಆರು ದೇಹಗಳಲ್ಲಿ ಐದು ತಲೆಗಳಿವೆಯೆ ಎಂಬುದಕ್ಕಾಗಿ ಹೋರಾಡುತ್ತಿರುವರು. ನೀವು ದೇವರನ್ನು ನೋಡಿರುವಿರಾ? ಇಲ್ಲ. ಅವನನ್ನು ಎಂದಾದರೂ ನೋಡುತ್ತೇವೆ ಎಂಬುದನ್ನೂ ಅವರು ನಂಬುವುದಿಲ್ಲ ನಾವು. ಎಂತಹ ತಿಳಿಗೇಡಿಗಳು ನಾವು! ನಿಜವಾಗಿಯೂ ನಾವು ಹುಚ್ಚರೇ!

ಒಬ್ಬ ದೇವರು ಇದ್ದರೆ, ಅವನು ನಿಮ್ಮ ದೇವರು ಆಗಿರಬೇಕು, ನಮ್ಮ ದೇವರೂ ಆಗಿರಬೇಕು ಎಂಬುದು ಭರತಖಂಡದಲ್ಲಿ ಹಿಂದಿನಿಂದ ಬಂದ ಒಂದು ಸಂಪ್ರದಾಯ. ಸೂರ್ಯ ಯಾರಿಗೆ ಸೇರಿರುವನು? ನೀವು ಅಂಕಲ್​ ಸ್ಯಾಮ್​ ಎಲ್ಲರಿಗೂ ಚಿಕ್ಕಪ್ಪ ಎನ್ನುವಿರಿ. ಒಬ್ಬ ದೇವರಿದ್ದರೆ ನೀವು ಅವನನ್ನು ನೋಡಲು ಸಾಧ್ಯವಾಗಬೇಕು. ಇಲ್ಲದೇ ಇದ್ದರೆ ಅವನು ಹೋಗಲಿ.

ಪ್ರತಿಯೊಬ್ಬರೂ ತಮ್ಮ ಮಾರ್ಗವೇ ಶ್ರೇಷ್ಠ ಎಂದು ಭಾವಿಸುವರು. ಒಳ್ಳೆಯದು! ಆದರೆ ಇದನ್ನು ಜ್ಞಾಪಕದಲ್ಲಿಡಿ, ಅದು ನಿಮಗೆ ಮಾತ್ರ ಒಳ್ಳೆಯದಿರಬಹುದು. ಒಂದು ಆಹಾರ ಒಬ್ಬನಿಗೆ ಅಜೀರ್ಣ ತರುವಂತಹುದು ಮತ್ತೊಬ್ಬನಿಗೆ ಚೆನ್ನಾಗಿ ಜೀರ್ಣವಾಗಬಹುದು. ನಿಮ್ಮ ಮಾರ್ಗ ನಿಮಗೆ ಸರಿಯಾಗಿದ್ದರೆ ಪ್ರತಿಯೊಬ್ಬರೂ ಇದನ್ನೇ ಅನುಕರಿಸಬೇಕೆಂಬ ನಿರ್ಣಯಕ್ಕೆ ಬರಬೇಡಿ. ಜಾಕ್​ನ ಕೋಟು ಜಾನ್​ ಮತ್ತು ಮೇರಿಗೆ ಸರಿಹೋಗುವುದೆಂದು ಆಲೋಚಿಸಬೇಡಿ. ಎಲ್ಲಾ ಅವಿದ್ಯಾವಂತ ಅಸಂಸ್ಕೃತ, ಅವಿವೇಕಿಗಳನ್ನೆಲ್ಲ ಈ ರೀತಿ ಚಿಂತಿಸುವಂತೆ ಮಾಡಿರುವೆವು. ನಿಮಗೆ ನೀವೇ ವಿಚಾರ ಮಾಡಿ ನಾಸ್ತಿಕರಾಗಿ, ಜಡವಾದಿಗಳಾಗಿ; ಅದಾದರೂ ಮೇಲು. ಬುದ್ಧಿಯನ್ನು ಉಪಯೋಗಿಸಿ. ಈ ವ್ಯಕ್ತಿಯು ಅನುಸರಿಸುತ್ತಿರುವ ಮಾರ್ಗ ಸರಿಯಿಲ್ಲ ಎಂದು ಹೇಳಲು ನಿಮಗೆ ಏನು ಅಧಿಕಾರವಿದೆ? ಅದು ನಿಮಗೆ ಸರಿಯಿಲ್ಲದೇ ಇರಬಹುದು, ಎಂದರೆ ನೀವು ಆ ಮಾರ್ಗವನ್ನು ಅನುಸರಿಸಿದರೆ ನೀವೂ ಅಧೋಗತಿಗೆ ಬರಬಹುದು. ಆದರೆ ಅವನೂ ಅಧೋಗತಿಗೆ ಬರುವನು ಎಂದು ಹೇಳುವುದಕ್ಕೆ ಆಗುವುದಿಲ್ಲ. ಅದಕ್ಕೆ ಶ‍್ರೀಕೃಷ್ಣ ಹೀಗೆ ಹೇಳುವುದು: ನೀವು ಜ್ಞಾನಿಗಳಾಗಿದ್ದೂ ಇನ್ನೊಬ್ಬನು ದುರ್ಬಲನೆಂದು ಕಂಡಾಗ ಅವನನ್ನು ಖಂಡಿಸಬೇಡಿ. ಅವನ ಮಟ್ಟಕ್ಕೆ ಹೋಗಿ ಸಾಧ್ಯಾವಾದರೆ ಅವನಿಗೆ ಸಹಾಯಮಾಡಿ. ಅವನು ಕ್ರಮೇಣ ಬೆಳೆಯಬೇಕಾಗಿದೆ. ನಾನು ಐದು ಗಂಟೆಯೊಳಗೆ ಅವನ ತಲೆಯೊಳಗೆ ಐದು ಬಕೆಟ್​ ಜ್ಞಾನವನ್ನು ಸುರಿಯಬಹುದು. ಇದರಿಂದ ಏನು ಪ್ರಯೋಜನವಾಯಿತು? ಅವನು ಹಿಂದಿಗಿಂತ ಮತ್ತೂ ಕೆಡುವನು.

ಈ ಕರ್ಮ ಬಂಧನವೆಲ್ಲ ಎಲ್ಲಿಂದ ಬಂದಿತು? ಇದಕ್ಕೆ ಕಾರಣವೇ ನಾವು ಆತ್ಮನನ್ನು ಕರ್ಮದಿಂದ ಬಂಧಿಸಿರುವುದು. ಹಿಂದೂತತ್ತ್ವದ ಪ್ರಕಾರ ಎರಡು ಅಸ್ತಿತ್ತ್ವಗಳಿವೆ. ಪ್ರಕೃತಿ ಒಂದು ಕಡೆ, ಆತ್ಮ ಮತ್ತೊಂದು ಕಡೆ. ಪ್ರಕೃತಿ ಎಂದರೆ ಬಾಹ್ಯಪ್ರಪಂಚ ಮಾತ್ರವಲ್ಲ. ನಮ್ಮ ದೇಹ ಮನಸ್ಸು ಇಚ್ಚೆ ಮತ್ತು ನಾವು ಯಾವುದನ್ನು ಅಹಂಕಾರ ಎನ್ನುತ್ತೇವೆಯೋ ಅವೆಲ್ಲ ಸೇರಿವೆ. ಅವೆಲ್ಲದರ ಆಚೆಗೆ ಅನಂತ ಜೀವ, ಅನಂತ ಆತ್ಮಜ್ಯೋತಿ ಇದೆ. ಈ ತತ್ತ್ವದ ಪ್ರಕಾರ ಆತ್ಮ ಪ್ರಕೃತಿಯಿಂದ ಬೇರೆ. ಅವನು ಹಿಂದೆ ಯಾವಾಗಲೂ ಬೇರೆಯಾಗಿಯೇ ಇದ್ದನು. ಮುಂದೆ ಬೇರೆಯಾಗಿಯೇ ಇರುವನು. ಆತ್ಮ ಯಾವ ಕಾಲದಲ್ಲಿಯೂ ಮನಸ್ಸಿನೊಂದಿಗೂ ಕೂಡ ಐಕ್ಯತಾ ಭಾವನೆಯನ್ನು ಹೊಂದಿರಲಿಲ್ಲ.

ನೀವು ತಿನ್ನುವ ಆಹಾರವೇ ಮನಸ್ಸಾಗುತ್ತಿದೆ ಎಂಬುದು ಸ್ವತಃ ಸಿದ್ಧವಾದುದು. ಅದು ಜಡವಸ್ತು. ಆತ್ಮನಿಗೂ ಆಹಾರಕ್ಕೂ ಎಂದಿಗೂ ಸಂಬಂಧವಿಲ್ಲ. ನೀವು ಊಟ ಮಾಡುವುದು ಮಾಡದೇ ಇರುವುದು ಅದಕ್ಕೆ ಸಂಬಂಧಿಸಿಲ್ಲ.ನೀವು ಆಲೋಚಿಸು ವುದು ಬಿಡುವುದು ಅದಕ್ಕೆ ಸಂಬಂಧಿಸಿಲ್ಲ. ಅದು ಅನಂತ ಜ್ಯೋತಿ. ಅದರ ಜ್ಯೋತಿ ಯಾವಾಗಲೂ ಒಂದೇ ಸಮನಾಗಿರುವುದು. ಬೆಳಕಿನ ಮುಂದೆ ನೀಲಿ ಅಥವಾ ಹಸುರು ಗಾಜನ್ನು ಇಟ್ಟರೆ ಅದು ಬೆಳಕನ್ನು ಏನು ಮಾಡಬಲ್ಲುದು? ಅದರ ಬೆಳಕನ್ನು ಯಾರೂ ಬದಲಾಯಿಸಲಾರರು. ಮನಸ್ಸು ಬದಲಾಯಿಸುವುದು, ಹಲವು ಬಣ್ಣಗಳ ಭಾವನೆಯನ್ನು ಕೊಡುವುದು. ಆತ್ಮ ಈ ದೇಹವನ್ನು ತ್ಯಜಿಸಿದ ಒಡನೆಯೇ ಇದೆಲ್ಲ ಚೂರಾಗುವುದು.

ಪ್ರಕೃತಿಯಲ್ಲಿ ಇರುವ ಸತ್ಯವೇ ಆತ್ಮ. ಸತ್ಯವೇ, ಆತ್ಮದ ಜ್ಯೋತಿಯೇ ನಮ್ಮ ದೇಹ ಮನಸ್ಸು ಮುಂತಾದುವುಗಳ ಮೂಲಕ ಚಲಿಸುವುದು ಮತ್ತು ಮಾತನಾಡುವುದು. ಆತ್ಮನ ಶಕ್ತಿ ಮತ್ತು ಚೈತನ್ಯವೇ ಪ್ರಕೃತಿಯ ಮೂಲಕ ಹಲವು ರೀತಿ ಕಾಣಿಸುತ್ತಿದೆ.ನಮ್ಮ ಆಲೋಚನೆ, ದೇಹಗಳ ಚಲನವಲನ ಇವುಗಳ ಹಿಂದೆಲ್ಲ ಆತ್ಮನಿರುವುದು. ಆದರೆ ಅದು, ಪಾಪ ಪುಣ್ಯ ಸುಖ ದುಃಖ ಶೀತೋಷ್ಣ ಮುಂತಾದ ಪ್ರಕೃತಿಯ ದ್ವಂದ್ವ ಭಾವನೆಗಳಿಗೆ ಸಿಲುಕುವುದಿಲ್ಲ. ಆದರೆ ಅದು ಮಾತ್ರ ಎಲ್ಲದಕ್ಕೂ ಬೆಳಕನ್ನು ನೀಡುತ್ತದೆ.

“ಆದಕಾರಣವೇ ಅರ್ಜುನ, ಈ ಕರ್ಮವೆಲ್ಲ ಪ್ರಕೃತಿಯಲ್ಲಿ ಇವೆ. ಪ್ರಕೃತಿಯೇ ನಮ್ಮ ದೇಹ ಮತ್ತು ಮನಸ್ಸಿನಲ್ಲಿ ತನ್ನ ನಿಯಮಾನುಸಾರ ಕೆಲಸಗಳನ್ನು ಮಾಡುತ್ತಿದೆ. ನಾವು ಪ್ರಕೃತಿಯೊಂದಿಗೆ ತಾದಾತ್ಮ್ಯ ಭಾವವನ್ನು ತಾಳಿ ನಾನು ಇದನ್ನು ಮಾಡು ತ್ತಿರುವೆನು ಎಂದು ಹೇಳುತ್ತೇವೆ. ಮೋಹ ನಮ್ಮನ್ನು ಹಿಡಿಯುವುದು ಹೀಗೆ.” (\enginline{III, 27})

ನಾವು ಯಾವಾಗಲೂ ಯಾವುದಾದರೂ ಬಲಾತ್ಕಾರಕ್ಕೆ ಒಳಗಾಗಿ ಕೆಲಸ ಮಾಡುವೆವು. ಹಸಿವು ಬಲಾತ್ಕರಿಸಿದಾಗ ಊಟ ಮಾಡುವೆವು. ದುಃಖವನ್ನು ಅನುಭವಿಸುವುದು ಅದಕ್ಕಿಂತ ಘೋರವಾದ ದಾಸ್ಯ. ನಿಜವಾದ “ನಾನು” ನಿತ್ಯಮುಕ್ತ. ಯಾವ ಕೆಲಸವನ್ನಾದರೂ ಮಾಡುವಂತೆ ಅದನ್ನು ಯಾರು ಬಲಾತ್ಕರಿಸ ಬಲ್ಲರು? ಕಷ್ಟವನ್ನು ಅನುಭವಿಸುವವನು ಪ್ರಕೃತಿಯಲ್ಲಿರುವನು. ನಾವೇ ದೇಹವೆಂದು ಭ್ರಮಿಸುವಾಗ ಮಾತ್ರ, ನಾವು ವ್ಯಥೆಪಡುತ್ತಿದ್ದೇವೆ, ನಾನು ಇಂಥವನು ಅಂಥವನು ಮುಂತಾದ ಕೆಲಸಕ್ಕೆ ಬಾರದವುಗಳನ್ನೆಲ್ಲ ಹೇಳುತ್ತೇವೆ. ಅದರೆ ಯಾರು ಸತ್ಯವನ್ನೇ ಅರಿತಿರುವನೊ ಅವನು ಸಾಕ್ಷಿಯಂತೆ ನಿಲ್ಲುವನು. ಅವನ ದೇಹ ಏನನ್ನಾದರೂ ಮಾಡಲಿ, ಮನಸ್ಸು ಏನನ್ನಾದರೂ ಮಾಡಲಿ, ಅವನು ಗಮನಿಸುವುದೇ ಇಲ್ಲ. ಆದರೆ ಇದನ್ನು ಜ್ಞಾಪಕದಲ್ಲಿಡಿ, ಬಹುಪಾಲು ಜನರು ಯಾವಾಗಲೂ ಭ್ರಮೆಯಲ್ಲಿರುವರು. ಯಾವಾಗಲಾದರೂ ಏನನ್ನಾದರೂ ಮಾಡಿದಾಗ ತಾವು ಮಾಡಿದೆವು ಎಂದು ಭಾವಿಸುವರು. ಅವರಿನ್ನೂ ಶ್ರೇಷ್ಠವಾದ ತತ್ತ್ವವನ್ನು ತಿಳಿದುಕೊಳ್ಳುವ ಸ್ಥಿತಿಗೆ ಬಂದಿಲ್ಲ. ಅಂತಹವರ ಶ್ರದ್ಧೆಯನ್ನು ಕೆಡಿಸಬೇಡಿ. ಅವರು ಪಾಪವನ್ನು ತ್ಯಜಿಸಿ ಪುಣ್ಯವನ್ನು ಮಾಡುತ್ತಿರುವರು. ಇದು ಬಹಳ ಒಳ್ಳೆಯ ಭಾವನೆ! ಅವರು ಅದನ್ನು ಮಾಡಿಕೊಂಡು ಹೋಗಲಿ. ಅವರು ಒಂದು ಒಳ್ಳೆಯ ಕೆಲಸವನ್ನು ಮಾಡುತ್ತಿರುವರು. ಕ್ರಮೇಣ ಒಳ್ಳೆಯದನ್ನು ಮಾಡುವುದಕ್ಕಿಂತ ಶ್ರೇಷ್ಠವಾಗಿರುವುದೊಂದು ಇದೆ ಎಂದು ಅರಿಯುವರು. ಅವರು ಸುಮ್ಮನೆ ಸಾಕ್ಷಿಯಾಗಿ ನಿಲ್ಲುವರು. ಕೆಲಸಗಳು ಸಾಗುತ್ತವೆ. ಕ್ರಮೇಣ ಅವರು ಇದನ್ನು ಅರಿಯುತ್ತಾರೆ. ಅವರು ಪಾಪವನ್ನೆಲ್ಲ ತ್ಯಜಿಸಿದ ಮೇಲೆ, ಒಳ್ಳೆಯದನ್ನು ಮಾಡಿದ ಮೇಲೆ, ಅವರು ತಾವು ಪ್ರಕೃತಿಗೆ ಅತೀತರೆಂಬುದನ್ನು ಅರಿಯುವರು. ಅವರು ಕರ್ಮ ಮಾಡುವವರಲ್ಲ, ದೂರವಿರುವವರು. ಅವರು ಕೇವಲ ಸಾಕ್ಷಿಗಳು. ಅವರು ಸುಮ್ಮನೆ ನಿಂತು ನೋಡುವರು. ಪ್ರಕೃತಿಯೇ ವಿಶ್ವವನ್ನೆಲ್ಲ ಸೃಷ್ಟಿಸುತ್ತಿರುವುದು. ಅವರು ಅದಕ್ಕೆ ವಿಮುಖರಾಗುವರು. “ಸೌಮ್ಯನೆ, ಆದಿಯಲ್ಲಿ ಸತ್ಯವೊಂದೆ ಇತ್ತು, ಅದಲ್ಲದೇ ಬೇರೆ ಇರಲಿಲ್ಲ. ಅದು ತಪಸ್ಸನ್ನು ಮಾಡಿ ಎಲ್ಲವನ್ನೂ ಸೃಷ್ಟಿಸಿತು” (ಛಾಂದೋಗ್ಯ, \enginline{VI, 2, 2-3}).

“ಯಾರಿಗೆ ಸರಿಯಾದ ದಾರಿ ಗೊತ್ತಿದೆಯೋ ಅವರೂ ಕೂಡ ತಮ್ಮ ಸಂಸ್ಕಾರಕ್ಕೆ ತಕ್ಕಂತೆ ಕರ್ಮವನ್ನು ಮಾಡುವರು. ಪ್ರತಿಯೊಬ್ಬರೂ ತಮ್ಮ ಸ್ವಭಾವಕ್ಕೆ ಅನುಗುಣ ವಾಗಿ ಕರ್ಮ ಮಾಡುವರು. ಯಾರೂ ಅದರಿಂದ ತಪ್ಪಿಸಿಕೊಳ್ಳಲಾರರು. ಒಂದು ಪರಮಾಣುವೂ ನಿಯಮವನ್ನು ಮೀರಲಾರದು. ಮಾನಸಿಕವಾಗಿರಲಿ, ಭೌತಿಕ ವಾಗಿರಲಿ, ಪ್ರತಿಯೊಂದೂ ನಿಯಮವನ್ನು ಪಾಲಿಸಲೇಬೇಕು. ಬಾಹ್ಯನಿಗ್ರಹದಿಂದ ಏನು ಪ್ರಯೋಜನ?” (\enginline{III, 33})

ಜೀವನದಲ್ಲಿ ನಮಗೆ ಯಾವುದರಿಂದ ಪ್ರಯೋಜನವಾಗುವುದು? ಭೋಗದಿಂದ ಅಲ್ಲ; ಸಂಗ್ರಹದಿಂದ ಅಲ್ಲ. ಪ್ರತಿಯೊಂದನ್ನು ವಿಶ್ಲೇಷಣೆ ಮಾಡಿ. ನಿಜವಾದ ಮೌಲ್ಯವಿರುವುದು ಅನುಭವದಲ್ಲಿ, ಅದೇ ನಮಗೆ ಬೋಧಿಸುವುದು. ಅನೇಕ ವೇಳೆ ನಾವು ಪಟ್ಟಿರುವ ಕಷ್ಟವು ಸುಖಕ್ಕಿಂತ ಹೆಚ್ಚು ಬುದ್ಧಿಯನ್ನು ಕಲಿಸಿರುವುದು. ಪ್ರಕೃತಿ ನಮಗೆ ಕೊಡುವ ಶಹಭಾಸ್​ಗಿರಿಗಿಂತ ಅದರಿಂದ ಬೀಳುವ ಪ್ರಹಾರಗಳು ನಮಗೆ ಚೆನ್ನಾಗಿ ಬುದ್ಧಿಯನ್ನು ಕಲಿಸಿವೆ. ಬರಗಾಲಕ್ಕೂ ಒಂದು ಸ್ಥಳವಿದೆ, ಅದಕ್ಕೂ ಒಂದು ಬೆಲೆ ಇದೆ.

ಶ‍್ರೀಕೃಷ್ಣನ ಪ್ರಕಾರ ಈ ಪ್ರಪಂಚಕ್ಕೆ ನಾವೇನೂ ಹೊಸತಾಗಿ ಬಂದಿಲ್ಲ. ನಮ್ಮ ಮನಸ್ಸೇನೂ ಹೊಸ ಮನಸ್ಸುಗಳಲ್ಲ. ಆಧುನಿಕ ಕಾಲದಲ್ಲಿ ಪ್ರತಿಯೊಂದು ಮಗುವೂ ಮಾನವ ಜನಾಂಗದ ಒಟ್ಟು ಅನುಭವವನ್ನು ತರುವುದು ಎಂಬುದು ನಮಗೆಲ್ಲ ಗೊತ್ತಿದೆ. ಹಾಗೆ ತರುವ ಹಳೆಯ ಅನುಭವಗಳು ಮಾನವ ಜನ್ಮದ ಅನುಭವಗಳು ಮಾತ್ರವಲ್ಲ, ಸಸ್ಯ ಜೀವನದ ಅನುಭವಗಳೂ ಕೂಡ. ಇವೆಲ್ಲ ಹಿಂದಿನ ಅಧ್ಯಾಯ ಗಳು, ಈಗಿನದು ಇಂದಿನ ಅಧ್ಯಾಯ, ಅವನ ಮುಂದೆ ಬೇಕಾದಷ್ಟು ಭವಿಷ್ಯದ ಅಧ್ಯಾಯಗಳಿವೆ. ಪ್ರತಿಯೊಬ್ಬನಿಗೂ ಅವನ ಮಾರ್ಗ ಆಗಲೇ ನಿಯೋಜಿತವಾಗಿದೆ; ರಚಿತವಾಗಿದೆ. ನಮಗೆ ಅವುಗಳ ವಿಷಯದಲ್ಲಿ ಎಷ್ಟೇ ಅಜ್ಞಾನವಿರಲಿ ಯಾವುದೂ -ಯಾವ ಘಟನೆಯೂ, ಯಾವ ಪರಿ ಸ್ಥಿತಿಯೂ ಕಾರಣವಿಲ್ಲದೆ ಇಲ್ಲ. ಕಾರ್ಯಕಾರಣ ಸಂಬಂಧದ ಅನಂತ ಸರಪಳಿ ಕೊನೆಗೆ ಪ್ರಕೃತಿಯನ್ನು ಹೋಗಿ ಮುಟ್ಟುತ್ತದೆ.ಇಡೀ ವಿಶ್ವವೇ ಆ ಶೃಂಖಲೆಯಿಂದ ಬದ್ಧವಾಗಿದೆ. ಕಾರ್ಯಕಾರಣ ಸಂಬಂಧದ ವಿಶ್ವ ಸರಪಣಿಯಿಂದ ಜಗತ್ತೆಲ್ಲವೂ ಬದ್ಧವಾಗಿದೆ. ನಿಮಗೆ ಒಂದು ಕೊಂಡಿ ಸಿಕ್ಕುವುದು, ನನಗೆ ಒಂದು ಕೊಂಡಿ ಸಿಕ್ಕುವುದು. ಅದೇ ನನ್ನ ಸ್ವಭಾವ.

“ನಿನ್ನ ಧರ್ಮದಲ್ಲಿ ಸಾಯುವುದು ಮತ್ತೊಬ್ಬರ ಧರ್ಮವನ್ನು ಅನುಸರಿಸು ವುದಕ್ಕಿಂತ ಮೇಲು” (\enginline{III, 35}) ಎನ್ನುವನು ಶ‍್ರೀಕೃಷ್ಣ. ಇದು ನನ್ನ ದಾರಿ, ನಾನು ಕೆಳಗೆ ಇಲ್ಲಿರುವೆನು, ನೀನು ಅಲ್ಲಿ ಮೇಲಿರುವೆ. ಅನೇಕ ವೇಳೆ ನಾನು ನನ್ನ ದಾರಿಯನ್ನು ತೊರೆದು, ಮೇಲೆ ಮತ್ತೊಬ್ಬನೊಡನೆ ಇರಬೇಕೆಂದು ಆಶಿಸುವೆನು. ನಾನು ಮೇಲೆ ಹೋದರೆ, ಅಲ್ಲಿಯೂ ಇಲ್ಲ, ಇಲ್ಲಿಯೂ ಇಲ್ಲ. ನಾವು ಈ ಸಿದ್ಧಾಂತವನ್ನು ಎಂದಿಗೂ ಮರೆಯಬಾರದು.ಇದೆಲ್ಲ ಒಂದು ಬೆಳವಣಿಗೆ, ಸಾವಕಾಶವಾಗಿ ಬೆಳೆಯಿರಿ.ನಿಮಗೆ ಎಲ್ಲಾ ಸಿದ್ಧಿಸುವುದು. ಇಲ್ಲದೇ ಇದ್ದರೆ ಬಹಳ ಆಧ್ಯಾತ್ಮಿಕ ಅಪಾಯವಿರುತ್ತದೆ. ಧರ್ಮವನ್ನು ಬೋಧಿಸುವ ವಿಷಯದಲ್ಲಿರುವ ಮುಖ್ಯವಾದ ರಹಸ್ಯವೇ ಇದು.

“ಜನರನ್ನು ಉದ್ಧಾರಮಾಡುವುದು ಮತ್ತು ಎಲ್ಲರೂ ಒಂದೇ ಸಿದ್ಧಾಂತವನ್ನು ನಂಬುವುದು” ಎಂದರೆ ಏನು ಅರ್ಥ? ಅದೆಂದಿಗೂ ಸಾಧ್ಯವಿಲ್ಲ. ಎಲ್ಲರಿಗೂ ಬೋಧಿಸಲ್ಪಡಬಹುದಾದ ಸಾಮಾನ್ಯ ಭಾವನೆಗಳಿವೆ ನಿಜ. ಆದರೆ ನಿಜವಾದ ಗುರು ನಿಮ್ಮ ಸ್ವಭಾವ ಏನು ಎಂಬುದನ್ನು ನಿರ್ಧರಿಸಬಲ್ಲ. ಅದು ನಿಮಗೆ ಗೊತ್ತಿಲ್ಲದೇ ಇರಬಹುದು. ಯಾವುದು ನಿಮ್ಮ ಸ್ವಭಾವ ಎಂದು ಭಾವಿಸುತ್ತಿರುವಿರೊ ಅದೇ ತಪ್ಪಿರಬಹುದು. ಅದು ನಿಮಗೆ ಚೆನ್ನಾಗಿ ಗೊತ್ತಿಲ್ಲ. ಗುರುವಿಗೆ ಇದು ಗೊತ್ತಾಗಿರಬೇಕು. ನಿಮ್ಮನ್ನು ನೋಡಿದ ತಕ್ಷಣ ಅದು ಗುರುವಿಗೆ ಗೊತ್ತಾಗಬೇಕು. ಅವನು ನಿಮ್ಮನ್ನು ಸರಿ ಯಾದ ದಾರಿಯಲ್ಲಿ ನಡೆಸುವನು. ನಾವು ಅಲ್ಲಿ ಇಲ್ಲಿ ಒದ್ದಾಡಿ ಹೋರಾಡಿ ಏನೇನೋ ಮಾಡುವೆವು. ಆದರೂ ಮುಂದುವರಿಯುವುದಿಲ್ಲ. ಆದರೆ ಸಕಾಲ ಬಂದಾಗ, ನಾವು ಜೀವನದ ಪ್ರವಾಹದ ಕೇಂದ್ರಕ್ಕೆ ಧುಮುಕಿ ಮುಂದೆ ತೇಲಿಕೊಂಡು ಹೋಗುವೆವು. ಆ ಪ್ರವಾಹದಲ್ಲಿ ನಾವು ಬಿದ್ದಾಗ ತೇಲುವೆವು. ಇದೇ ಶುಭ ಚಿಹ್ನೆ. ಅನಂತರ ಇನ್ನು ಮೇಲೆ ಯಾವ ಹೋರಾಟವೂ ಇರುವುದಿಲ್ಲ. ಇದನ್ನು ನಾವು ಕಂಡುಹಿಡಿಯ ಬೇಕಾಗಿದೆ. ಆದುದರಿಂದ ನಮ್ಮ ಮಾರ್ಗವನ್ನೇ ತೊರೆದು ಅನ್ಯರ ಮಾರ್ಗವನ್ನು ಅನುಸರಿಸುವುದಕ್ಕಿಂತ ನಮ್ಮ ಮಾರ್ಗದಲ್ಲಿ ಮಡಿಯುವುದು ಲೇಸು.

ಇದರ ಬದಲು ನಾವು ಒಂದು ಧರ್ಮವನ್ನು ಸ್ಥಾಪಿಸಿ ಕೆಲವು ಮೂಢ ನಂಬಿಕೆ ಗಳನ್ನು ಜಾರಿಗೆ ತರುವೆವು. ಮಾನವ ಕೋಟೆಯ ಗುರಿಯನ್ನು ಲೆಕ್ಕಕ್ಕೆತಾರದೆ, ಪ್ರತಿಯೊಬ್ಬರಿಗೂ ಒಂದೇ ಸ್ವಭಾವ ಇದೆ ಎಂದು ಭಾವಿಸುವೆವು. ಯಾವ ಇಬ್ಬರಿಗೂ ಒಂದೇ ದೇಹವಾಗಲಿ, ಮನಸ್ಸಾಗಲಿ ಇಲ್ಲ. ಯಾವ ಇಬ್ಬರಿಗೂ ಒಂದೇ ಧರ್ಮವಿಲ್ಲ.

ನೀವು ಧಾರ್ಮಿಕರಾಗಬೇಕೆಂದು ಆಶಿಸಿದರೆ ಯಾವ ಸಂಸ್ಥಾಬದ್ಧ ಧರ್ಮ ದ್ವಾರದ ಕಡೆಗೂ ಹೋಗಬೇಡಿ. ಅದು ಒಳ್ಳೆಯದಕ್ಕಿಂತ ನೂರು ಪಾಲು ಕೆಟ್ಟದ್ದನ್ನು ಮಾಡುವುದು. ಏಕೆಂದರೆ ಪ್ರತಿಯೊಂದು ವ್ಯಕ್ತಿಯ ವೈಯಕ್ತಿಕ ಬೆಳವಣಿಗೆಯನ್ನು ಅದು ತಡೆಯುವುದು. ಎಲ್ಲವನ್ನೂ ತಿಳಿದುಕೊಳ್ಳಿ. ಆದರೆ ನೀವು ನಿಮ್ಮ ಆದರ್ಶದಲ್ಲಿ ಭದ್ರವಾಗಿರಿ. ನೀವು ನನ್ನ ಮಾತನ್ನು ಕೇಳುವುದಾದರೆ ಯಾವುದರ ಬಲೆಗೂ ಬೀಳಬೇಡಿ. ಅವು ನಿಮ್ಮ ಮುಂದೆ ಪಾಶವನ್ನು ಬೀಸಿದೊಡನೆಯೆ, ನೀವು ಅವುಗಳಿಂದ ಪಾರಾಗಿ ಬೇರೆಕಡೆ ಹೋಗಿ. ಜೇನುಹುಳು ಹಲವು ಹೂಗಳಿಂದ ಮಧುವನ್ನು ಹೀರುವಂತೆ ಸ್ವತಂತ್ರರಾಗಿರಿ; ಬದ್ಧರಾಗಬೇಡಿ. ಯಾವುದೇ ಸಂಸ್ಥಾಬದ್ಧ ಧರ್ಮದ ಬಾಗಿಲಿಗೂ ಪ್ರವೇಶ ಮಾಡಬೇಡಿ. ಧರ್ಮ ನಿಮಗೆ ಮತ್ತು ದೇವರಿಗೆ ಮಾತ್ರ ಸಂಬಂಧಿಸಿರುವುದು. ಮೂರನೆ ಯವರು ಅದರ ಮಧ್ಯದಲ್ಲಿ ಬರಕೂಡದು. ಈ ಸಂಸ್ಥಾಬದ್ಧ ಧರ್ಮಗಳು ಎಂತಹ ಹಾವಳಿಯನ್ನು ಮಾಡಿವೆ ಎಂಬುದನ್ನು ಕುರಿತು ಯೋಚಿಸಿ ನೋಡಿ. ಈ ಧಾರ್ಮಿಕ ಕ್ರೂರಿಗಳಿಗಿಂತ ಯಾವ ನೆಪೋಲಿಯನ್​ಕ್ರೂರಿಯಾಗಿದ್ದನು? ನೀವು ನಾವು ಸೇರಿ ಒಂದು ಸಂಸ್ಥೆಯನ್ನು ಮಾಡಿದರೆ, ಅದರಿಂದ ಹೊರಗಿರುವವರನ್ನು ದ್ವೇಷಿಸುವೆವು. ಪ್ರೀತಿಸುವುದು ಎಂದರೆ ಇನೊಬ್ಬರನ್ನು ದ್ವೇಷಿಸುವುದಾದರೆ ಹಾಗೆ ಪ್ರೀತಿಸದೆ ಇರುವುದೇ ಮೇಲು. ಅದು ಪ್ರೀತಿಯೇ ಅಲ್ಲ. ಅದೊಂದು ನರಕ. ನಿಮ್ಮ ಜನರನ್ನು ಪ್ರೀತಿಸುವುದು ಎಂದರೆ ಇತರರನ್ನೆಲ್ಲ ದ್ವೇಷಿಸುವುದು ಎಂದಾದರೆ, ಇದೇ ಸ್ವಾರ್ಥದ ಸಾರ, ಪಾಶವೀಯತೆ. ಇದು ನಮ್ಮನ್ನು ಮೃಗಸದೃಶರನ್ನಾಗಿ ಮಾಡುವುದು. ಆದಕಾರಣವೇ ಮತ್ತೊಬ್ಬರ ಸಹಜ ಧರ್ಮ ಶ್ರೇಷ್ಠ ವೆಂದು ಕಂಡರೂ ಅದನ್ನು ಅನುಸರಿಸುವುದಕ್ಕಿಂತ ನಿಮ್ಮ ಸಹಜಧರ್ಮವನ್ನು ಅನುಸರಿಸುತ್ತ ಸಾಯುವುದು ಮೇಲು.

“ಅರ್ಜುನ, ಎಚ್ಚರಿಕೆ! ಕಾಮ ಮತ್ತು ಕ್ರೋಧವೇ ಮಹಾಶತ್ರುಗಳು. ಅವನ್ನು ನಿಗ್ರಹಿಸಬೇಕಾಗಿದೆ. ಯಾರು ಜ್ಞಾನಿಗಳೊ ಅವರನ್ನು ಕೂಡ ಈ ಕಾಮಕ್ರೋಧಗಳು ಆವರಿಸುವುವು. ಅವನ್ನು ತೃಪ್ತಿಪಡಿಸಲು ಆಗುವುದಿಲ್ಲ. ಅವು ಇಂದ್ರಿಯ ಮತ್ತು ಮನಸ್ಸಿನಲ್ಲಿವೆ. ಆತ್ಮ ಯಾವುದನ್ನೂ ಇಚ್ಚಿಸುವುದಿಲ್ಲ” (\enginline{III, 37-40}).

“ನಾನು ಈ ಯೋಗವನ್ನು ಹಿಂದೆ ವಿವಸ್ವಂತನಿಗೆ ಹೇಳಿದೆನು. ವಿವಸ್ವಂತ ಮನುವಿಗೆ ಇದನ್ನು ಹೇಳಿದನು... ಹೀಗೆ ಜ್ಞಾನ ಒಬ್ಬ ರಾಜನಿಂದ ಮತ್ತೊಬ್ಬನಿಗೆ ಪಾರಂಪರ್ಯವಾಗಿ ಬಂದಿತು. ಆದರೆ ಕಾಲಕ್ರಮೇಣ ಈ ಮಹಾಯೋಗ ನಷ್ಟ ವಾಯಿತು. ಅದಕಾರಣವೇ ನಾನು ಇಂದು ನಿನಗೆ ಇದನ್ನು ಹೇಳುತ್ತಿರುವೆನು” (\enginline{IV, 1-3}).

ಅರ್ಜುನ ಆಗ ಕೇಳುತ್ತಾನೆ: “ನೀನು ಹೀಗೆ ಏತಕ್ಕೆ ಮಾತನಾಡುವೆ? ನೀನು ಈಗ ಎಲ್ಲೋ ಕೆಲವು ವರುಷಗಳ ಹಿಂದೆ ಹುಟ್ಟಿದವನು. ವಿವಸ್ವಂತ ನಿನಗಿಂತ ಮುಂಚೆ ಹುಟ್ಟಿದವನು. ನೀನು ಇದನ್ನು ಅವನಿಗೆ ಬೋಧಿಸಿದೆ ಎಂದರೆ ಅರ್ಥವೇನು?” (\enginline{IV-4})

ಆಗ ಶ‍್ರೀಕೃಷ್ಣನು ಉತ್ತರವನ್ನು ಕೊಡುವನು: “ಹೇ ಅರ್ಜುನ, ನೀನೂ ನಾನೂ ಎಷ್ಟೋ ಜನ್ಮಗಳನ್ನು ಧರಿಸಿ ಬಂದಿರುವೆವು. ಆದರೆ ನಿನಗೆ ಅದರ ಜ್ಞಾಪಕವಿಲ್ಲ. ನಾನು ಜನನ ಮರಣಾತೀತನು. ಪ್ರಪಂಚಕ್ಕೆಲ್ಲ ಪರಮೇಶ್ವರ. ನಾನು ನನ್ನ ಪ್ರಕೃತಿಯನ್ನು ಆಶ್ರಯಿಸಿ ದೇಹವನ್ನು ಧರಿಸುತ್ತೇನೆ. ಎಂದು ಧರ್ಮ ನಾಶವಾಗುವುದೋ, ಅಧರ್ಮ ವೃದ್ಧಿಯಾಗುವುದೋ, ಆಗ ನಾನು ಮಾನವಕೋಟಿಯನ್ನು ಉದ್ಧಾರ ಮಾಡಲು ಬರುವೆನು. ಸಾಧುಗಳನ್ನು ಉದ್ಧಾರಮಾಡುವುದಕ್ಕೆ, ದುರ್ಜನರನ್ನು ನಿಗ್ರಹಿಸುವುದಕ್ಕೆ ಧರ್ಮವನ್ನು ಸಂಸ್ಥಾಪನೆ ಮಾಡುವುದಕ್ಕೆ ನಾನು ಪುನಃ ಪುನಃ ಬರುವೆನು. ಯಾರು ಯಾವ ಮಾರ್ಗದ ಮೂಲಕವಾಗಿ ಬರಲು ಇಚ್ಛಿಸಲಿ ನಾನು ಅವರಿಗೆ ಆ ಮಾರ್ಗದಲ್ಲಿ ದೊರಕುತ್ತೇನೆ. ಆದರೂ ಅರ್ಜುನ, ಯಾರೂ ನನ್ನ ಧರ್ಮವನ್ನು ಅತಿಕ್ರಮಿಸಿ ಹೋಗಲಾರರು.” ಹಿಂದೆ ಯಾರೂ ಅದನ್ನೂ ಮಾಡಲಿಲ್ಲ. ನಾವು ಹೇಗೆ ಅದನ್ನು ಮಾಡಬಲ್ಲೆವು? ಯಾರೂ ಅವರವರ ಮಾರ್ಗವನ್ನು ಬಿಟ್ಟು ಹೋಗಲಾರರು. (\enginline{(IV, 5-8, 11)})

ಸಮಾಜಗಳೆಲ್ಲ ನಿಂತಿರುವುದು ಅಸಮರ್ಪಕವಾದ ಸಾಮಾನ್ಯೀಕರಣದ ಮೇಲೆ. ಸಾಮಾನ್ಯೀಕರಣ ಯಾವಾಗ ಪರಿಪೂರ್ಣವಾಗಿರುವುದೊ ಆಗ ಮಾತ್ರ ನಿಯಮ ಗಳನ್ನು ರಚಿಸಲು ಸಾಧ್ಯ. “ಪ್ರತಿಯೊಂದು ನಿಯಮಕ್ಕೂ ಒಂದು ವಿನಾಯಿತಿ ಇದೆ” ಎಂಬ ಹಳೆಯ ನಾಣ್ನುಡಿಯ ಅರ್ಥವೇನು? ಅದೊಂದು ನಿಯಮವಾದರೆ ಯಾರೂ ಅದನ್ನು ಮುರಿಯಲಾರರು, ಯಾರಿಗೂ ಅದನ್ನು ಮುರಿಯುವುದು ಸಾಧ್ಯವಿಲ್ಲ. ಸೇಬಿನ ಹಣ್ಣು ಆಕರ್ಷಣ ಸಿದ್ಧಾಂತವನ್ನು ವಿರೋಧಿಸಬಲ್ಲುದೆ! ನೀವು ನಿಯಮವನ್ನು ಉಲ್ಲಂಘಿಸಿದ ತಕ್ಷಣ ಪ್ರಪಂಚವೇ ಇರುವುದಿಲ್ಲ. ನೀವು ನಿಯಮವನ್ನು ಮೀರುವ ಒಂದು ಸಮಯ ಬರುವುದು. ಆಗ ನಿಮ್ಮ ಪ್ರಜ್ಞೆ ಮನಸ್ಸು ದೇಹಗಳೆಲ್ಲ ಕರಗಿ ಹೋಗುವುವು.

ಅಲ್ಲೊಬ್ಬ ಮನುಷ್ಯ ಕದಿಯುತ್ತಿರುವನು, ಅವನು ಏತಕ್ಕೆ ಕದಿಯುವನು? ನೀವು ಅವನನ್ನು ಶಿಕ್ಷಿಸುತ್ತೀರಿ. ನೀವು ಏತಕ್ಕೆ ಅವನಿಗೆ ಅವಕಾಶ ಕೊಟ್ಟು ಅವನ ಶಕ್ತಿಯನ್ನು ಉಪಯೋಗಿಸಿಕೊಳ್ಳಬಾರದು? “ನೀನೊಬ್ಬ ಪಾಪಿ” ಎಂದು ನೀವು ಹೇಳುತ್ತೀರಿ, ಹಲವರು ಅವನು ಕಾನೂನನ್ನು ಉಲ್ಲಂಘಿಸಿರುವನು ಎಂದು ಹೇಳುತ್ತಾರೆ. ಮಾನವರ ಮಂದೆಯಲ್ಲಿ ಎಲ್ಲರನ್ನೂ ಒಂದು ರೀತಿ ಆಲೋಚಿಸುವಂತೆ ಬಲಾತ್ಕಾರಿಸುವನು. ಆದ ಕಾರಣವೇ ಪಾಪ, ದೌರ್ಬಲ್ಯ ಮತ್ತು ಇತರ ತೊಂದರೆಗಳು. ನೀವು ಯೋಚಿಸು ವಷ್ಟು ಪ್ರಪಂಚ ಕೆಟ್ಟದ್ದಲ್ಲ. ಅವಿವೇಕಿಗಳಾದ ನಾವೇ ಅದನ್ನು ಪಾಪಮಯವನ್ನಾಗಿ ಮಾಡಿರುವೆವು. ನಾವೇ ನಮ್ಮ ದೈತ್ಯರು ಮತ್ತು ದೆವ್ವಗಳನ್ನು ಸೃಷ್ಟಿಸುವೆವು. ನಂತರ ನಾವು ಅವುಗಳಿಂದ ಪಾರಾಗ ಲಾರೆವು. ನಮ್ಮ ಕೈಗಳಿಂದ ಕಣ್ಣುಗಳನ್ನು ಮುಚ್ಚಿಕೊಂಡು ಯಾರಾದರೂ ನಮಗೆ ದಾರಿ ತೋರಿ ಎಂದು ಅಳುವವರು ನಾವೇ. ಮೂಢರೇ, ಕೈಗಳನ್ನು ತೆಗೆಯಿರಿ ನಿಮ್ಮ ಕಣ್ಣುಗಳ ಮೇಲಿನಿಂದ. ನಾವು ಇಷ್ಟನ್ನೇ ಮಾಡಬೇಕಾಗಿ ರುವುದು. ನಮ್ಮನ್ನು ಉದ್ಧರಿಸಬೇಕೆಂದು ದೇವತೆಗಳನ್ನು ಕರೆಯುತ್ತೇವೆ. ಯಾರೂ ತಮ್ಮಲ್ಲೇ ತಪ್ಪನ್ನು ಕಂಡುಹಿಡಿಯುವುದೇ ಇಲ್ಲ. ಇದೇ ಶೋಚನೀಯ ಸ್ಥಿತಿ. ಸಮಾಜದಲ್ಲಿ ಏತಕ್ಕೆ ಇಷ್ಟೊಂದು ಪಾಪವಿದೆ? ಅವರು ಏನು ಹೇಳುವುದು? ಇಂದ್ರಿಯಗಳು, ಸೈತಾನ್​, ಸ್ತ್ರೀ - ಇವುಗಳನ್ನು ಏತಕ್ಕೆ ನಾವು ಸೃಷ್ಟಿಸುವುದು! ಅವುಗಳನ್ನು ಸೃಷ್ಟಿಸಿ ಎಂದು ಯಾರೂ ನಿಮಗೆ ಹೇಳುವುದಿಲ್ಲ. “ಅರ್ಜುನ, ಯಾರೂ ನನ್ನ ಮಾರ್ಗದಿಂದ ವಿಚಲಿತರಾಗಲಾರರು” (\enginline{VI, - 11}). ನಾವು ಅವಿವೇಕಿಗಳು. ನಮ್ಮ ಮಾರ್ಗ ಅವಿವೇಕಿಗಳ ಮಾರ್ಗ. ನಾವು ಈ ಮಾಯೆಯ ಮೂಲಕ ಹೋಗ ಬೇಕಾಗಿದೆ. ದೇವರು ಸ್ವರ್ಗವನ್ನು ಮಾಡಿದನು. ಅನಂತರ ಮನುಷ್ಯನು ತನಗಾಗಿ ನರಕವನ್ನು ಮಾಡಿಕೊಂಡನು.

“ಯಾವ ಕರ್ಮವೂ ನನ್ನನ್ನು ಮುಟ್ಟಲಾರದು. ಕರ್ಮಫಲಕ್ಕೆ ನನಗೆ ಯಾವ ವಿಧವಾದ ಆಸೆಯೂ ಇಲ್ಲ. ಯಾರು ನನ್ನನ್ನು ಹೀಗೆ ಅರಿತಿರುವರೊ ಅವರಿಗೆ ರಹಸ್ಯ ಗೊತ್ತಿದೆ. ಅವರು ಕರ್ಮದಿಂದ ಬಾಧಿತರಾಗುವುದಿಲ್ಲ. ಈ ರಹಸ್ಯವನ್ನು ತಿಳಿದ ಹಿಂದಿನಕಾಲದ ಜ್ಞಾನಿಗಳು ಯಾವ ಅಪಾಯವೂ ಇಲ್ಲದೆ ಕರ್ಮದಲ್ಲಿ ನಿರತರಾಗಿದ್ದರು. ನೀನು ಕೂಡ ಹಾಗೆಯೇ ಕೆಲಸ ಮಾಡು” (\enginline{IV, 14-15}).

“ಯಾರು ಕರ್ಮದಲ್ಲಿ ಅಕರ್ಮವನ್ನೂ, ಅಕರ್ಮದಲ್ಲಿ ಕರ್ಮವನ್ನೂ ನೋಡು ವರೋ ಅವರೇ ಜ್ಞಾನಿಗಳು” (\enginline{IV-18}). ಇದೇ ಸಮಸ್ಯೆ. ಪ್ರತಿಯೊಂದು ಇಂದ್ರಿಯಗಳ ಮೂಲಕವೂ ಕರ್ಮ ಆಗುತ್ತಿರುವುದು. ಆಗಲೂ ಯಾವುದೂ ನಿನ್ನನ್ನು ವಿಚಲಿತ ನನ್ನಾಗಿ ಮಾಡದಂತಹ ಅದ್ಭುತ ಶಾಂತಿ ಇದೆಯೇ? ಪೇಟೆಯ ದಾರಿಯಲ್ಲಿ ನಿಂತು ಬೇಕಾದಷ್ಟು ಗಡಿಬಿಡಿಯ ಮಧ್ಯೆ ಬಸ್ಸಿಗೆ ಕಾಯುವಾಗ ನೀನು ಧ್ಯಾನದಲ್ಲಿರುವೆಯಾ? ಶಾಂತ ಚಿತ್ತನಾಗಿ ಅನುದ್ವಿಗ್ನನಾಗಿರುವೆಯಾ? ಗುಹೆಯಲ್ಲಿರುವಾಗ ಪ್ರಶಾಂತಚಿತ್ತ ನಾಗಿರುವಂತೆ ತೀವ್ರವಾದ ಕರ್ಮದ ಬಿರುಗಾಳಿ ಇರುವಾಗಲೂ ಇರುವೆಯಾ? ನೀನು ಹಾಗೆ ಇದ್ದರೆ ಯೋಗಿ. ಇಲ್ಲದೆ ಇದ್ದರೆ ಅಲ್ಲ.

“ಯಾರು ಕೆಲಸವನ್ನು ಸ್ವೇಚ್ಛೆಯಿಂದ ಮಾಡಿರುವರೊ, ಯಾರಿಗೆ ಯಾವ ಲಾಭದ ಆಸೆಯೂ ಇಲ್ಲವೋ, ಸ್ವಾರ್ಥದ ಆಸೆ ಇಲ್ಲವೋ ಅವನನ್ನು ಜ್ಞಾನಿಗಳು ಪಂಡಿತನೆಂದು ಹೇಳುತ್ತಾರೆ” (\enginline{(IV-19)}). ನಾವು ಎಲ್ಲಿಯವರೆಗೆ ಸ್ವಾರ್ಥಪರರಾಗಿರುವೆವೊ ಅಲ್ಲಿಯವರೆಗೆ ಸತ್ಯ ನಮಗೆ ದೊರಕಲಾರದು. ನಾವು ಪ್ರತಿಯೊಂದನ್ನೂ ನಮ್ಮ ದೃಷ್ಟಿಯಿಂದ ನೋಡುವೆವು. ವಸ್ತುಗಳು ಹೇಗೆ ಇರುವುವೋ ಹಾಗೆಯೇ ಕಾಣಿಸಿಕೊಳ್ಳುವವು. ಅವೇನೂ ನಮ್ಮನ್ನು ವಂಚಿಸುವುದಿಲ್ಲ. ಎಂದಿಗೂ ಇಲ್ಲ. ನಾವೇ ಅದನ್ನು ಮರೆ ಮಾಡುವುದು, ನಮ್ಮಲ್ಲಿ ಕುಂಚ ಇದೆ. ಒಂದು ವಸ್ತು ಕಣ್ಣೆದುರಿಗೆ ಬರುವುದು, ನಾವು ಅದನ್ನು ಒಪ್ಪುವುದಿಲ್ಲ. ಅನಂತರ ಅದಕ್ಕೆ ಸ್ವಲ್ಪ ಬಣ್ಣ ಹಾಕಿ ನೋಡುವೆವು. ಸತ್ಯವನ್ನು ತಿಳಿಯಬೇಕೆಂಬ ಆಸೆ ನಮಗೆ ಇಲ್ಲ. ನಾವು ಪ್ರತಿ ಯೊಂದನ್ನೂ ನಮ್ಮ ದೃಷ್ಟಿಯಿಂದ ಲೇಪನ ಮಾಡುವೆವು.ಎಲ್ಲಾ ಕರ್ಮಗಳ ಹಿಂದೆಯೂ ಇರುವ ಕ್ರಿಯೋತ್ತೇಜಕ ಶಕ್ತಿಯೇ ಸ್ವಾರ್ಥ, ನಾವೇ ಎಲ್ಲವನ್ನೂ ಮರೆ ಮಾಡುವೆವು. ನಾವು ರೇಷ್ಮೆಹುಳುವಿನಂತೆ. ಅದು ತನ್ನ ದೇಹದಿಂದಲೇ ತಂತುವನ್ನು ತಯಾರುಮಾಡಿ, ಅದರಿಂದ ಗೂಡನ್ನು ಮಾಡಿಕೊಳ್ಳುವುದು. ಅದು ಅನಂತರ ಬಂಧನಕ್ಕೆ ಬೀಳುವುದು. ಅದು ತನ್ನ ಕೆಲಸದಿಂದ ತಾನೇ ಬಂಧನಕ್ಕೆ ಬೀಳುವುದು. ನಾವೆಲ್ಲ ಮಾಡುತ್ತಿರುವುದು ಇದನ್ನೇ. ‘ನನ್ನದು’ ಎಂದು ಹೇಳಿದೊಡನೆಯೇ ಬಂಧನದ ನೂಲು ಸುತ್ತಿಕೊಳ್ಳುತ್ತದೆ. ನಾನು ನನ್ನದು ಎಂಬುದೇ ಮತ್ತೊಂದು ಸುತ್ತು. ಹೀಗೆ ನನ್ನನ್ನು ಬಿಗಿಯುತ್ತಾ ಹೋಗುವುದು.

ನಾವು ಒಂದು ಕ್ಷಣವೂ ಕೆಲಸ ಮಾಡದೆ ಇರಲಾರೆವು. ಕೆಲಸಮಾಡಿ. ನಿಮ್ಮ ನೆರೆಹೊರೆಯವರು ‘ಸ್ವಲ್ಪ ಬಂದು ಸಹಾಯಮಾಡಿ’ ಎಂದರೆ ನೀವು ಹೇಗೆ ಹೋಗುತ್ತೀರೋ ಹಾಗೆಯೇ ನಿಮ್ಮ ಸಹಾಯಕ್ಕೂ ಹೋಗಿ, ಅದಕ್ಕಿಂತ ಹೆಚ್ಚು ಅಲ್ಲ. ಜಾನನ ದೇಹಕ್ಕಿಂತ ನಿಮ್ಮ ದೇಹವೇನೂ ಹೆಚ್ಚಲ್ಲ. ಜಾನನ ದೇಹಕ್ಕೆ ಮಾಡುವುದಕ್ಕಿಂತ ಹೆಚ್ಚನ್ನೇನೂ ನಿಮ್ಮ ದೇಹಕ್ಕೆ ಮಾಡಬೇಡಿ. ಇದೇ ಧರ್ಮ.

“ಯಾರ ಕರ್ಮಗಳು ಅಸೆ, ಸ್ವಾರ್ಥಗಳಿಂದ ಮುಕ್ತವಾಗಿವೆಯೋ, ಅವನು ಜ್ಞಾನಾಗ್ನಿಯಿಂದ ಎಲ್ಲ ಕರ್ಮಬಂಧನಗಳನ್ನು ಸುಟ್ಟಿರುವ ಜ್ಞಾನಿ” (\enginline{IV-19}). ಗ್ರಂಥ ಗಳನ್ನು ಓದುವುದರಿಂದ ಇದು ಸಾಧ್ಯವಾಗಲಾರದು. ಕತ್ತೆ ಒಂದು ಪುಸ್ತಕಾಲಯವನ್ನೇ ಹೊರಬಹುದು. ಇದರಿಂದ ಅದೇನು ಬುದ್ಧಿವಂತನಾಗುವುದಿಲ್ಲ. ಸುಮ್ಮನೆ ಹಲವು ಪುಸ್ತಕಗಳನ್ನು ಓದಿ ಏನು ಪ್ರಯೋಜನ? “ಕರ್ಮಫಲದಲ್ಲಿ ಆಸಕ್ತಿಯನ್ನು ಬಿಟ್ಟು ನಿತ್ಯತೃಪ್ತನೂ ನಿರಾಶ್ರಯನೂ ಆಗಿ ಕರ್ಮದಲ್ಲಿ ಪ್ರವೃತ್ತನಾಗಿದ್ದರೂ ಅವನು ಏನನ್ನೂ ಮಾಡುತ್ತಿಲ್ಲ” (\enginline{IV-20}).

ನಾನು ತಾಯಿಯ ಗರ್ಭದಿಂದ ಬೆತ್ತಲೆ ಬಂದೆ, ಬೆತ್ತಲೆಯೇ ಹೋಗುತ್ತೇನೆ. ನಿಸ್ಸಹಾಯಕನಾಗಿ ನಾನು ಬಂದೆ, ನಿಸ್ಸಹಾಯಕನಾಗಿ ಹೋಗುತ್ತಿರುವೆನು. ಈಗ ನಿಸ್ಸಹಾಯಕನಾಗಿರುವೆನು. ನಮಗೆ ಗುರಿ ಗೊತ್ತಿಲ್ಲ. ಇದನ್ನು ಕುರಿತು ಆಲೋಚಿಸು ವುದಕ್ಕೂ ನಮಗೆ ಭಯವಾಗುವುದು. ನಮಗೆ ಇಂತಹ ವಿಚಿತ್ರವಾದ ಭಾವನೆಗಳು ಬರುತ್ತವೆ. ನಾವೊಂದು ಪ್ರೇತ ಆವಾಹಕನ ಬಳಿಗೆ ಹೋಗಿ, ಪ್ರೇತ ಏನಾದರೂ ನಮಗೆ ಸಹಾಯ ಮಾಡಬಲ್ಲದೆ ಎಂದು ನೋಡುತ್ತೇವೆ. ಈ ದೌರ್ಬಲ್ಯವನ್ನು ನೋಡಿ! ದೇವರೊ ದೆವ್ವವೊ ಪಿಶಾಚಿಯೊ ಯಾವುದಾದರೂ ಸಹಾಯಕ್ಕೆ ಬರಲಿ! ಎಲ್ಲಾ ಪುರೋಹಿತರೂ, ಎಲ್ಲಾ ಆಷಾಢಭೂತಿಗಳೂ ಬರಲಿ! ನಾವು ದುರ್ಬಲರಾದೊಡನೆಯೇ ಅವು ನಮ್ಮನ್ನು ಅಂಟಿಕೊಳ್ಳುವುವು, ಆಗ ಎಲ್ಲ ದೇವತೆಗಳನ್ನೂ ಕರೆತರುವೆವು.

ನನ್ನ ದೇಶದಲ್ಲಿ ಒಬ್ಬ ವ್ಯಕ್ತಿಯು ಬಲಾಢ್ಯನಾಗಿ ಬುದ್ಧಿವಂತನಾಗಿ ತತ್ತ್ವಜ್ಞಾನಿ ಯಾಗುವನು. ಆಗ ಅವನು ಈ ರೀತಿ ಪ್ರಾರ್ಥಿಸುವುದು, ಸ್ನಾನಾದಿಗಳನ್ನು ಮಾಡು ವುದು ಇವೆಲ್ಲ ಕೆಲಸಕ್ಕೆ ಬಾರದವು ಎಂದು ಭಾವಿಸುವನು. ಅವನ ತಂದೆ ಕಾಲವಾಗು ವನು, ತಾಯಿ ಕಾಲವಾಗುವಳು. ಹಿಂದೂವಿಗೆ ಜೀವನದಲ್ಲಿ ಬೀಳುವ ಮಹಾಪ್ರಹಾರ ಇದು. ಆಗ ಅವನು ಎಲ್ಲಾ ಕೊಳಚೆ ನೀರಿನಲ್ಲಿಯೂ ಸ್ನಾನ ಮಾಡುತ್ತ, ಎಲ್ಲಾ ದೇವಾಲಯಗಳಿಗೂ ಹೋಗಿ ಧೂಳನ್ನು ನೆಕ್ಕುವನು. ಯಾರಾದರೂ ಅವನಿಗೆ ಸಹಾಯವನ್ನು ಮಾಡಬಲ್ಲರೆ? ಆದರೆ ನಾವು ಹತಾಶರು. ನಮಗೆ ಯಾರಿಂದಲೂ ಸಹಾಯ ಬರುವುದಿಲ್ಲ. ಇದೇ ಸತ್ಯ. ಮನುಷ್ಯರ ಸಂಖ್ಯೆಗಿಂತ ಹೆಚ್ಚಾಗಿ ದೇವರುಗಳು ಇರುವರು. ಆದರೂ ಯಾವ ಸಹಾಯವೂ ಒದಗಲಿಲ್ಲ. ನಾವು ನಾಯಿಗಳಂತೆ ಸಾಯುವೆವು. ಆದರೂ ಯಾವ ಸಹಾಯವೂ ಇಲ್ಲ. ಆದರೂ ನಿರಾಶೆಯ ಮಧ್ಯದಲ್ಲಿಯೂ ಎಂದಾದರೂ ಸಹಾಯ ಸಿಕ್ಕೀತೆಂದು ಸಹಾಯಕ್ಕಾಗಿ ನೋಡುವೆವು. ಆದರೆ ಎಲ್ಲ ಕಡೆಯೂ ಮೃಗೀಯತೆ, ಕ್ಷಾಮ, ರೋಗ, ಶೋಕ, ಕೆಡುಕು. ಎಲ್ಲರೂ ನೆರವಿಗಾಗಿ ಅಳುತ್ತಿರುವರು. ಆದರೆ ಸಹಾಯ ಬರುತ್ತಿಲ್ಲ. ಸಹಾಯಕ್ಕಾಗಿ ಹಲುಬುತ್ತಿರುವೆವು. ಓ ಎಂತಹ ದಾರುಣವಾದ ಸ್ಥಿತಿ! ಎಂತಹ ಭಯಾನಕವಾದ ಸ್ಥಿತಿ, ನಿಮ್ಮ ಹೃದಯವನ್ನೇ ಪರೀಕ್ಷೆ ಮಾಡಿ ಕೊಳ್ಳಿ, ಕಷ್ಟದ ಅರ್ಧಪಾಲಿಗೆ ಕಾರಣ, ತಂದೆ ತಾಯಿಗಳು. ಈ ದೌರ್ಬಲ್ಯದೊಂದಿಗೆ ಹುಟ್ಟಿದ ಮೇಲೆ ಅದನ್ನೇ ನಮ್ಮಲ್ಲಿ ತುಂಬುವರು. ಕ್ರಮೇಣ ನಾವು ಅದರಲ್ಲಿ (ದೌರ್ಬಲ್ಯದಲ್ಲಿ) ಬೆಳೆಯುತ್ತಾ ಹೋಗುವೆವು.

ಅಯ್ಯೋ! ನನ್ನ ಸಹಾಯಕ್ಕೆ ಯಾರೂ ಬರಲಿಲ್ಲವಲ್ಲ ಎಂದು ಭಾವಿಸುವುದು ಒಂದು ಮಹಾಪರಾಧ. ಯಾರಿಂದಲೂ ಸಹಾಯವನ್ನು ಆಶಿಸಬೇಡಿ. ನಮಗೆ ನಾವೇ ಸಹಾಯ ಮಾಡಿಕೊಳ್ಳಬೇಕು. ನಮಗೆ ನಾವೇ ಸಹಾಯ ಮಾಡಿಕೊಳ್ಳದೇ ಹೋದರೆ ಮತ್ತಾರೂ ಸಹಾಯ ಮಾಡಲಾರರು. “ನಿನಗೆ ನೀನೇ ಮಿತ್ರ ಮತ್ತು ನಿನಗೆ ನೀನೇ ಶತ್ರು. ನಿನ್ನ ಆತ್ಮವಲ್ಲದೆ ಬೇರೆ ಶತ್ರುವಿಲ್ಲ. ನಿನ್ನ ಆತ್ಮನಲ್ಲದೆ ಬೇರೆ ಮಿತ್ರನಿಲ್ಲ” (\enginline{VI-5}). ಇದೇ ಶೇಷ್ಠವಾದ ಕಟ್ಟಕಡೆಯ ಬೋಧನೆ. ಓ, ಆದರೆ ಇದನ್ನು ಕಲಿಯಲು ಎಷ್ಟೊಂದು ಕಾಲ ಹಿಡಿಯುವುದು! ಅದು ಇನ್ನೇನು ಸಿಕ್ಕಿದಂತೆ ಕಾಣುವುದು. ಮರುಕ್ಷಣವೇ ಹಿಂದಿನ ಸಂಸ್ಕಾರಗಳು ಅಪ್ಪಳಿಸುವುವು. ಬೆನ್ನೆಲುಬು ಮುರಿಯುತ್ತದೆ. ನಾವು ದುರ್ಬಲರಾಗಿ ಪುನಃ ಹೊರಗಿನಿಂದ ಬರುವ ಸಹಾಯ ಮತ್ತು ಮೂಢನಂಬಿಕೆಯನ್ನು ಹಿಡಿದು ಕೊಳ್ಳುವೆವು. ಸಹಾಯವನ್ನು ಹುಡುಕಿಕೊಂಡು ಹೋಗುವುದರಿಂದ ಬರುವ ದಾರುಣ ದುಃಖವನ್ನು ಕುರಿತು ಯೋಚಿಸಿ ನೋಡಿ!

ಪುರೋಹಿತರು ಮಾಮೂಲಿನಂತೆ ಏನನ್ನೊ ಹೇಳಿ ಏನನ್ನೊ ನಿರೀಕ್ಷಿಸುವರು. ಅರವತ್ತು ಸಹಸ್ರ ಜನರು ಆಕಾಶವನ್ನೇ ನೋಡುತ್ತಾ ಪ್ರಾರ್ಥಿಸುವರು. ಪುರೋಹಿತ ರಿಗೆ ದಕ್ಷಿಣೆ ಕೂಡುವರು. ಮಾಸಗಳು ಕಳೆದಂತೆ ಅವರು ಇನ್ನೂ ಆಕಾಶವನ್ನು ನೋಡು ತ್ತಿರುವರು, ದಕ್ಷಿಣೆ ಕೊಡುವರು, ಪ್ರಾರ್ಥಿಸುವರು. ಇದನ್ನು ಕುರಿತು ಯೋಚಿಸಿ ನೋಡಿ! ಇದೊಂದು ಹುಚ್ಚಲ್ಲವೇ? ಮತ್ತೇನು? ಇದಕ್ಕೆ ಕಾರಣ ಯಾರು? ನೀವು ಧರ್ಮವನ್ನೇನೋ ಬೋಧಿಸಬಹುದು. ಆದರೆ ಇನ್ನೂ ವಿಕಾಸವಾಗದೆ ಇರುವ ಬಾಲಸ್ವಭಾವದವರ ಮನಸ್ಸನ್ನು ಉದ್ರೇಕಗೊಳಿಸುವುದೆ? ನೀವು ಅದಕ್ಕೆ ವ್ಯಥೆಪಡಬೇಕು. ನಿಮ್ಮ ಹೃದಯಾಂತರಾಳದಲ್ಲಿ ನೀವು ಏನಾಗಿರುವಿರಿ? ನೀವು ಮತ್ತೊಬ್ಬನಿಗೆ ಕೊಟ್ಟ ಪ್ರತಿಯೊಂದು ದುರ್ಬಲ ವಾದ ಆಲೋಚನೆಗೂ ಬಡ್ಡಿ ಸಹಿತ ಕಷ್ಟವನ್ನು ಅನುಭವಿಸಬೇಕಾಗಿದೆ. ಕರ್ಮಸಿದ್ಧಾಂತ ನಮ್ಮನ್ನು ಕಿತ್ತುತಿನ್ನದೆ ಇರುವುದಿಲ್ಲ.

ಒಂದೇ ಒಂದು ಪಾಪವಿರುವುದು. ಆದೇ ದೌರ್ಬಲ್ಯ. ನಾನು ಹುಡುಗನಾಗಿದ್ದಾಗ ಮಿಲ್ಟನ್ನಿನ “ಪ್ಯಾರಡೈಸ್​ ಲಾಸ್ಟ್​” ಓದುತ್ತಿದ್ದೆ. ಅಲ್ಲಿ ಯಾರ ವಿಷಯಕ್ಕಾದರೂ ನನಗೆ ಸ್ವಲ್ಪ ಗೌರವವಿದ್ದ ವ್ಯಕ್ತಿ ಎಂದರೆ ಸೇಟನ್​. ಯಾರು ಎಂದಿಗೂ ದುರ್ಬಲ ನಾಗುವುದಿಲ್ಲವೋ, ಎಲ್ಲವನ್ನೂ ಎದುರಿಸಬಲ್ಲನೋ, ತನ್ನ ಪ್ರಾಣವನ್ನೇ ತ್ಯಜಿಸಲು ಸಿದ್ಧನಾಗಿರುವನೋ ಆವನೇ ನಿಜವಾದ ಮಹಾತ್ಮ.

ಎದ್ದು ನಿಲ್ಲಿ. ನಿಮ್ಮ ಜೀವನವನ್ನು ತೃಣದಂತೆ ಭಾವಿಸಿ. ಹುಚ್ಚರ ಸಂಖ್ಯೆಗೆ ಮತ್ತೊಂದನ್ನು ಸೇರಿಸಬೇಡಿ. ಬರುವ ಪಾಪದೊಂದಿಗೆ ನಿಮ್ಮ ದೌರ್ಬಲ್ಯವನ್ನು ಬೆರೆಸಬೇಡಿ, ಪ್ರಪಂಚಕ್ಕೆ ನಾನು ಹೇಳಬೇಕಾಗಿರು ವುದು ಇದನ್ನೇ. ಬಲಿಷ್ಠರಾಗಿ. ನೀವೇನೊ ದೆವ್ವಗಳನ್ನು ಪಿಶಾಚಿಗಳನ್ನು ಕುರಿತು ಮಾತನಾಡುತ್ತೀರಿ. ನಾವೇ ಜೀವಂತ ಪಿಶಾಚಿಗಳು. ಶಕ್ತಿ ಮತ್ತು ಬೆಳವಣಿಗೆಯೇ ಜೀವನದ ಚಿಹ್ನೆ. ದೌರ್ಬಲ್ಯವೇ ಮರಣದ ಚಿಹ್ನೆ. ಯಾವುದು ದುರ್ಬಲವೊ ಅದನ್ನು ತ್ಯಜಿಸಿ. ಅದು ಮೃತ್ಯು ಸಮಾನ. ಶಕ್ತರಾಗಿದ್ದರೆ ನರಕ ಕೂಪಕ್ಕೂ ಹೋಗಿ ಅದನ್ನು ಸ್ವಾಧೀನಪಡಿಸಿಕೊಳ್ಳಿ. “ಧೀರರಿಗೆ ಮಾತ್ರ ಮುಕ್ತಿ. ಧೀರರು ಮಾತ್ರ ಜೀವನದಲ್ಲಿ ಶ್ರೇಷ್ಠ ವಾದುದನ್ನು ಪಡೆಯಲು ಅರ್ಹರು.” ಯಾರು ಅತಿ ಶ್ರೇಷ್ಠ ಧೀರರೊ ಅವರು ಮಾತ್ರ ಮುಕ್ತಿಗೆ ಅರ್ಹರು! ಯಾರದು ನರಕ? ಯಾರದು ಹಿಂಸೆ? ಯಾರದು ಪಾಪ? ಯಾರದು ದೌರ್ಬಲ್ಯ? ಯಾರದು ಮರಣ? ಯಾರದು ರೋಗ?

ನೀವು ದೇವರನ್ನು ನಂಬುವಿರಿ. ನೀವು ಹಾಗೆ ನಂಬಿದರೆ ನಿಜವಾದ ದೇವರನ್ನು ಮಾತ್ರ ನಂಬಿ. “ನೀನೇ ಪುರುಷ. ನೀನೇ ಸ್ತ್ರೀ, ಬಲಾಢ್ಯನಾಗಿ ನಡೆಯುತ್ತಿರುವ ಯುವಕ ಯುವತಿಯೂ ನೀನೇ. ಕೋಲನ್ನು ಹಿಡಿದುಕೊಂಡು ನಡುಗುತ್ತ ಹೋಗುತ್ತಿರುವ ವೃದ್ಧನೂ ನೀನೆ.” ನೀನೆ ದೌರ್ಬಲ್ಯ. ನೀನೆ ಅಂಜಿಕೆ. ನೀನೆ ಸ್ವರ್ಗ ನರಕಗಳು! ಕಚ್ಚುವ ಕೃಷ್ಣಸರ್ಪವೂ ನೀನೆ. ಅಂಜಿಕೆಯಂತೆ ಬಾ! ಮೃತ್ಯುವಿನಂತೆ ಬಾ! ದುಃಖದಂತೆ ಬಾ!

ಎಲ್ಲಾ ದೌರ್ಬಲ್ಯ ಮತ್ತು ಬಂಧನಗಳೂ ಭ್ರಾಂತಿ. ನೀನೊಂದು ಮಾತನ್ನು ಆಡಿದರೆ ಅವು ಪರಾರಿಯಾಗಬೇಕು. ದುರ್ಬಲನಾಗಬೇಡ.ಇದು ವಿನಃ ಬೇರೆ ದಾರಿಯೇ ಇಲ್ಲ. ಎದ್ದು ನಿಲ್ಲಿ. ಧೀರರಾಗಿ. ಅಂಜಬೇಡಿ. ಯಾವ ಮೂಢನಂಬಿಕೆಯೂ ಇಲ್ಲದಿರಲಿ. ಸತ್ಯ ಹೇಗಿದೆಯೋ ಹಾಗೆ ಎದುರಿಸಿ. ಮರಣ ಬಂದರೆ -ಅದೇ ನಮ್ಮ ದಾರುಣ ವಾದ ದುಃಖ - ಬರಲಿ! ನಾವು ಸಾಯುವುದಕ್ಕೆ ಸಿದ್ಧರಾಗಿರುವೆವು. ನನಗೆ ಗೊತ್ತಿರುವ ಧರ್ಮ ಇಷ್ಟೆ. ನಾನು ಅದನ್ನು ಪಡೆದಿಲ್ಲ. ಆದರೆ ಪಡೆದುಕೊಳ್ಳುವುದಕ್ಕೆ ಹೋರಾಡುತ್ತಿರುವೆನು. ನನಗೆ ಅದು ಸಾಧ್ಯವಾಗದೆ ಇರಬಹುದು.ಆದರೆ ನಿಮಗೆ ಸಿದ್ಧಿಸಬಹುದು. ಮುಂದುವರಿಯಿರಿ.

ಒಬ್ಬ ಎಲ್ಲಿ ಮತ್ತೊಬ್ಬನನ್ನು ನೋಡುವನೊ ಮತ್ತೊಬ್ಬನನ್ನು ಕೇಳುತ್ತಿರುವನೋ, ಎಲ್ಲಿಯವರೆಗೆ ಇಬ್ಬರು ಇರುವರೊ, ಅಲ್ಲಿಯವರೆಗೆ ಅಂಜಿಕೆ ಇರುವುದು. ಅಂಜಿಕೆಯೇ ಎಲ್ಲಾ ದುಖಃಕ್ಕೂ ಮೂಲ. ಎಲ್ಲಿ ಮತ್ತೊಬ್ಬನನ್ನು ನೋಡುವುದಿಲ್ಲವೋ, ಎಲ್ಲಿ ಎಲ್ಲಾ ಒಂದೇ ಆಗಿರು ವುದೋ, ಅಲ್ಲಿ ದುಃಖಿಸುವುದಕ್ಕೆ ಯಾರೂ ಇಲ್ಲ, ಸುಖಿಸುವುದಕ್ಕೂ ಯಾರೂ ಇಲ್ಲ. ಅಲ್ಲಿ ಏಕಮೇವ ಅದ್ವಿತೀಯ ಇರುವುದು. ಆದಕಾರಣ ಅಂಜಬೇಡಿ. ಉತ್ತಿಷ್ಠರಾಗಿ, ಜಾಗ್ರತರಾಗಿ, ಗುರಿ ಸೇರುವವರೆಗೆ ನಿಲ್ಲ ಬೇಡಿ.

