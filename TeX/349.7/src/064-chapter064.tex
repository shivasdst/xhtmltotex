
\chapter[ಜ್ಞಾನಯೋಗ ]{ಜ್ಞಾನಯೋಗ \protect\footnote{\engfoot{C.W. Vol, VI, P.91}}}

ಪ್ರಾರಂಭದಲ್ಲಿ ಧ್ಯಾನವು ನೇತಿ-ಸ್ವಭಾವದ್ದಾಗಿರಬೇಕು; ಎಲ್ಲವನ್ನೂ ಮನಸ್ಸಿನಿಂದ ಆಚೆಗೆ ತಳ್ಳಿ. ಮನಸ್ಸಿನಲ್ಲಿ ಏಳುವುದನ್ನೆಲ್ಲ ದೃಢವಾದ ಇಚ್ಛಾಶಕ್ತಿಯಿಂದ ವಿಶ್ಲೇಷಣೆ ಮಾಡಿ.

ಅನಂತರ ನಾವು ಯಾರು ಎಂಬುದನ್ನು ಅರಿಯಿರಿ. ಅದೇ ಸಚ್ಚಿದಾನಂದ ಸ್ವರೂಪ. ದೃಗ್​ ದೃಶ್ಯಗಳನ್ನು ಒಂದುಗೂಡಿಸುವುದೇ ಧ್ಯಾನ. ಹೀಗೆ ಧ್ಯಾನಿಸಿ:

“ಮೇಲೆಲ್ಲ ನಾನೇ ತುಂಬಿರುವೆ, ಕೆಳಗೆ ನಾನೇ ಇರುವೆ, ಮಧ್ಯದಲ್ಲಿ ನಾನೇ ತುಂಬಿರುವೆ, ನಾನು ಎಲ್ಲರಲ್ಲಿಯೂ ಇರುವೆನು, ಎಲ್ಲರೂ ನನ್ನಲ್ಲಿ ಇರುವರು. ‘ಓಂ ತತ್​ಸತ್​” ನಾನು ಮನಸ್ಸಿಗೆ ಅತೀತನಾಗಿರುವೆನು, ನಾನು ಪ್ರಪಂಚದ ಏಕ ಮಾತ್ರ ಸ್ವಾಮಿ, ನಾನು ಸುಖವೂ ಅಲ್ಲ, ದುಃಖವೂ ಅಲ್ಲ.

ದೇಹವು ತಿನ್ನುವುದು, ಕುಡಿಯುವುದು; ನಾನು ದೇಹವಲ್ಲ, ನಾನು ಮನಸ್ಸೂ ಅಲ್ಲ, ನಾನೇ ಅವನು.

ನಾನೇ ಸಾಕ್ಷಿ; ಪ್ರಪಂಚವನ್ನು ನೋಡುತ್ತಿರುವೆನು, ಆರೋಗ್ಯ ಬಂದಾಗ ಅದರ ಸಾಕ್ಷಿ ನಾನೇ, ರೋಗ ಬಂದಾಗ ಅದರ ಸಾಕ್ಷಿಯೂ ನಾನೇ.

ನಾನೇ ಸಚ್ಚಿದಾನಂದ, ನಾನೇ ಜ್ಞಾನಸಾರ, ಜ್ಞಾನಾಮೃತ. ಹಿಂದಿನಿಂದಲೂ ನಾನು ಬದಲಾಗಿಯೇ ಇಲ್ಲ. ನಾನು ಶಾಂತ ಸ್ವರೂಪನು. ಸ್ವಯಂ ಜ್ಯೋತಿ, ಅವಿಕಾರಿ.”

