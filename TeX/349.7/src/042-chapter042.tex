
\chapter[ಗುರಿ ]{ಗುರಿ \protect\footnote{\engfoot{C.W. Vol. VI, P. 94}}}

ದ್ವೈತವು ಈಶ್ವರ ಮತ್ತು ಜಗತ್ತು ಇವು ಶಾಶ್ವತವಾಗಿ ಬೇರೆ ಬೇರೆ ಎಂದು ಒಪ್ಪಿಕೊಳ್ಳುವುದು. ಜಗತ್ತು ಯಾವಾಗಲೂ ಈಶ್ವರನ ಅಧೀನ. ಶುದ್ಧ ಅದ್ವೈತಿಗಳು ಈ ವ್ಯತ್ಯಾಸವನ್ನು ಒಪ್ಪಿಕೊಳ್ಳುವುದಿಲ್ಲ. ಅವರು ಅಂತಿಮ ದೃಷ್ಟಿಯಿಂದ ಎಲ್ಲವೂ ಬ್ರಹ್ಮವೆಂದು ಒಪ್ಪಿಕೊಳ್ಳುವರು.ಜಗತ್ತು ಈಶ್ವರನಲ್ಲಿ ಕರಗಿಹೋಗುವುದು. ಜಗತ್ತಿ ನಲ್ಲಿ ಸನಾತನವಾಗಿರುವುದೇ ಬ್ರಹ್ಮ.

ಅದ್ವೈತಿಗಳಿಗೆ ಸಾಂತ ಮತ್ತು ಅನಂತಗಳೆರಡೂ ಬರಿಯ ಪದಗಳು. ಜಗತ್ತು ಪ್ರಕೃತಿ ಇವುಗಳೆಲ್ಲ ಇರುವುದೇ ಭೇದಭಾವದಿಂದ. ಭಿನ್ನತೆಯೇ ಪ್ರಕೃತಿ.

ದೇವರು ಏತಕ್ಕೆ ಜಗತ್ತನ್ನು ಸೃಷ್ಟಿಸಿದ? ಪೂರ್ಣನಾಗಿರುವ ದೇವರು ಅಪರಿಪೂರ್ಣವಾಗಿರುವ ಪ್ರಪಂಚವನ್ನು ಏತಕ್ಕೆ ಸೃಷ್ಟಿಸಿದ? ಇಂತಹ ಪ್ರಶ್ನೆಗಳಿಗೆ ಉತ್ತರಕೊಡುವುದಕ್ಕೆ ಆಗುವುದಿಲ್ಲ. ಏಕೆಂದರೆ ಪ್ರಶ್ನೆಗಳೇ ತರ್ಕಬದ್ಧವಲ್ಲ. ಯುಕ್ತಿ ಇರುವುದು ಪ್ರಕೃತಿಯಲ್ಲಿ. ಪ್ರಕೃತಿಯ ಆಚೆ ಯುಕ್ತಿಯಿಲ್ಲ. ದೇವರು ಸರ್ವಶಕ್ತ. ಅವನು ಏತಕ್ಕೆ ಇವನ್ನು ಮಾಡಿದ ಎಂದು ಕೇಳುವುದು ಅವನಿಗೆ ಒಂದು ಮಿತಿಯನ್ನು ಕಲ್ಪಿಸಿದಂತೆ. ಇದರ ಹಿಂದೆ ಅವನಿಗೆ ಏನೋ ಒಂದು ಉದ್ದೇಶವಿದೆ ಎಂದಂತೆ ಆಯಿತು. ಅವನಿಗೆ ಏನಾದರೂ ಉದ್ದೇಶವಿದ್ದರೆ ಅದೊಂದು ಗುರಿಗೆ ಸಾಧನವಾಯಿತು. ಎಂದರೆ ಅವನಿಗೆ ಸಾಧನ ಇಲ್ಲದೆ ಗುರಿ ದೊರಕಲಾರದು ಎಂಬಂತೆ ಆಯಿತು. ಆಶ್ರಿತ ವಸ್ತುವನ್ನು ಕುರಿತು ಮಾತ್ರ ಎಲ್ಲಿಂದ, ಏತಕ್ಕೆ ಎಂದು ಪ್ರಶ್ನಿಸ ಬಹುದು.

