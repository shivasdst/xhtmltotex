
\chapter[ಭರತಖಂಡದ ಸಂದೇಶ ]{ಭರತಖಂಡದ ಸಂದೇಶ \protect\footnote{\engfoot{C.W. Vol. V, p. 188}}}

\centerline{(ಸಂಡೆ ಟೈಮ್ಸ್, ಲಂಡನ್​, ೧೮೯೬)}

ಭರತಖಂಡದ ‘ಹವಳದ ತೀರಕ್ಕೆ’ ಪಾದ್ರಿಗಳನ್ನು ಇಂಗ್ಲಿಷಿನವರು ಕಳುಹಿಸು ವುದು ಎಲ್ಲರಿಗೂ ವೇದ್ಯವಾಗಿದೆ. “ಪ್ರಪಂಚಕ್ಕೆಲ್ಲಾ ಹೋಗಿ ಸಂದೇಶವನ್ನು ಸಾರಿ” ಎಂಬ ಕ್ರಿಸ್ತನ ಸಂದೇಶವನ್ನು ಅಕ್ಷರಶಃ ಅವರು ಪರಿಪಾಲಿಸುತ್ತಿರುವರು. ಆಂಗ್ಲೇ ಯರ ಯಾವ ಪಂಗಡವೂ ಕ್ರಿಸ್ತನ ಆಜ್ಞೆಯನ್ನು ಪರಿಪಾಲಿಸಲು ಹಿಂದೆ ಉಳಿದಿಲ್ಲ. ಭಾರತೀಯರೂ ಕೂಡ ಇಂಗ್ಲೆಂಡಿಗೆ ಮಿಷನರಿಗಳನ್ನು ಕಳುಹಿಸುತ್ತಾರೆ ಎಂಬುದು ಜನರಿಗೆ ಗೊತ್ತಿಲ್ಲವೆಂದು ಕಾಣುವುದು.

ಅಕಸ್ಮಾತ್ತಾಗಿ ಎಂದು ಬೇಕಾದರೆ ಹೇಳಬಹುದು, ಸ್ವಾಮಿ ವಿವೇಕಾನಂದರನ್ನು ಅವರು ಹಂಗಾಮಿಯಾಗಿ ತಂಗಿದ್ದ ಸೆಯಿಂಟ್​ ಜಾರ್ಜ್​ ರಸ್ತೆಯ ೬೩ನೇ ನಂಬರ್​ ಮನೆಯಲ್ಲಿ ನೋಡುವ ಸಂದರ್ಭ ಒದಗಿತು. ಅವರು ಇಂಗ್ಲೆಂಡಿಗೆ ಬಂದ ಕಾರಣ ವನ್ನು ಕುರಿತು ಮಾತನಾಡಲು ಅವರು ಒಪ್ಪಿದುದರಿಂದ ಈ ವಿಷಯವನ್ನು ಕುರಿತು ಚರ್ಚಿಸತೊಡಗಿದೆ. ನನ್ನ ಬೇಡಿಕೆಗೆ ಅವರು ಒಪ್ಪಿಗೆ ಇತ್ತುದು ಆಶ್ಚರ್ಯ ವಾಯಿತು. ಅವರು ಹೀಗೆ ಹೇಳತೊಡಗಿದರು:

ಅಮೆರಿಕ ದೇಶದಲ್ಲಿ ಬಾತ್ಮೀದಾರರನ್ನು ನೋಡಿ ನನಗೆ ಅಭ್ಯಾಸವಾಗಿದೆ. ನಮ್ಮ ದೇಶದಲ್ಲಿ ಈ ಅಭ್ಯಾಸವಿಲ್ಲ. ಆದರೆ ನನ್ನ ಉದ್ದೇಶವನ್ನು ಜನರಿಗೆ ತಿಳಿಯ ಪಡಿಸಲು ಇತರ ದೇಶಗಳಲ್ಲಿ ಇರುವ ರೀತಿಗಳನ್ನು ಬಳಸಲೂ ನಾನು ಹಿಂಜರಿಯುವು ದಿಲ್ಲ. ಅಮೆರಿಕಾ ದೇಶದಲ್ಲಿ ೧೮೯೩ರಲ್ಲಿ ಚಿಕಾಗೋ ನಗರದಲ್ಲಿ ನಡೆದ ವಿಶ್ವಧರ್ಮ ಸಮ್ಮೇಳನಕ್ಕೆ ನಾನು ಹಿಂದೂಧರ್ಮದ ಪ್ರತಿನಿಧಿಯಾಗಿ ಹೋದೆ. ಮೈಸೂರಿನ ಮಹಾರಾಜರು ಮತ್ತು ಇತರ ಸ್ನೇಹಿತರು ನನ್ನನ್ನು ಅಲ್ಲಿಗೆ ಕಳುಹಿಸಿದರು. ನನಗೆ ಅಮೆರಿಕಾ ದೇಶದಲ್ಲಿ ಸ್ವಲ್ಪ ಜಯ ಲಭಿಸಿತೆಂದೇ ಹೇಳಬಹುದು. ಚಿಕಾಗೋ ನಗರವಲ್ಲದೆ ಬೇರೆ ಬೇರೆ ನಗರಗಳಿಗೂ ನನ್ನನ್ನು ಆಹ್ವಾನಿಸಿದರು. ನಾನು ದೀರ್ಘಕಾಲ ಅಲ್ಲಿ ತಂಗಿದ್ದೆ. ಕಳೆದ ಬೇಸಿಗೆ ಕಾಲ ಮತ್ತು ಈ ಬೇಸಿಗೆ ಕಾಲದಲ್ಲಿ ಇಂಗ್ಲೆಂಡಿನಲ್ಲಿ ಕಳೆದ ದಿನಗಳನ್ನು ಬಿಟ್ಟರೆ ಉಳಿದ ಮೂರು ವರುಷಗಳನ್ನು ಅಮೆರಿಕ ದೇಶದಲ್ಲಿ ಕಳೆದಿರುವೆನು. ಅಮೆರಿಕ ಸಂಸ್ಕೃತಿ ನನ್ನ ದೃಷ್ಟಿಯಲ್ಲಿ ಬಹಳ ಮೇಲಾ ದುದು. ಯಾವ ಹೊಸಭಾವನೆ ಬಂದರೂ ಅದನ್ನು ಸ್ವೀಕರಿಸುವ ಮನೋಭಾವ ದವರು ಅಮೆರಿಕ ದೇಶದವರು. ಹೊಸದೆಂದು ಯಾವುದನ್ನೂ ಅವರು ನಿರಾಕರಿಸು ವುದಿಲ್ಲ. ಯಾವ ಹೊಸ ಭಾವನೆಯಾಗಲೀ ಅದರ ಯೋಗ್ಯತೆಯ ಮೇಲೆ ನಿಲ್ಲು ವುದು, ಇಲ್ಲವೇ ಬೀಳುವುದು.”

ಬಾತ್ಮೀದಾರ: “ಇಂಗ್ಲೆಂಡಿನಲ್ಲಿ ಹಾಗಲ್ಲವೆಂಬುದೇ ನಿಮ್ಮ ಮತ?”

ಸ್ವಾಮೀಜಿ: “ಹೌದು, ಇಂಗ್ಲೆಂಡಿನ ಸಂಸ್ಕೃತಿ ಬಹಳ ಹಳೆಯದು. ಕಾಲ ಕಳೆದಂತೆ ಅದು ಹಲವು ಅನಾವಶ್ಯಕವಾದ ವಸ್ತುಗಳನ್ನು ಸೇರಿಸಿಕೊಂಡು ಬಂದಿದೆ. ನಿಮ್ಮಲ್ಲಿ ಎಷ್ಟೋ ಮೂಢನಂಬಿಕೆಗಳಿವೆ. ಅವನ್ನು ನಿರ್ಮೂಲ ಮಾಡಬೇಕಾಗಿದೆ. ಯಾರು ನಿಮಗೆ ಹೊಸ ಭಾವನೆಯನ್ನು ಕೊಡಲು ಬರುತ್ತಾರೆಯೋ ಅವರು ನಿಮ್ಮ ಈ ಸ್ವಭಾವವನ್ನು ಎದುರಿಸಬೇಕಾಗಿದೆ.”

ಬಾತ್ಮೀದಾರ: “ಹೌದು, ಜನ ಹಾಗೆ ಹೇಳುತ್ತಾರೆ. ನೀವು ಅಮೆರಿಕದಲ್ಲಿ ಹೊಸ ಧರ್ಮ ಅಥವಾ ಚರ್ಚನ್ನು ಸ್ಥಾಪಿಸಲಿಲ್ಲ ಎಂದು ಕೇಳಿದೆ.”

ಸ್ವಾಮೀಜಿ: “ಹೌದು, ನಿಜ, ಹೊಸ ಸಂಸ್ಥೆಯನ್ನು ಸ್ಥಾಪಿಸುವುದು ನಮ್ಮ ತತ್ತ್ವಕ್ಕೆ ವಿರೋಧವಾಗಿರುವುದು. ಆಗಲೇ ಪ್ರತಿಯೊಂದು ದೇಶದಲ್ಲಿಯೂ ಬೇಕಾದಷ್ಟು ಸಂಸ್ಥೆಗಳಿವೆ. ಸಂಸ್ಥೆಯನ್ನು ಸ್ಥಾಪಿಸಿದ ಮೇಲೆ ಅದನ್ನು ನೋಡಿಕೊಳ್ಳು ವುದಕ್ಕೆ ವ್ಯಕ್ತಿಗಳು ಬೇಕಾಗುತ್ತಾರೆ. ಯಾರು ಸಂನ್ಯಾಸಿಗಳಾಗಿರುವರೋ, ಯಾರ ಗುರಿ ಆತ್ಮಸಾಕ್ಷಾತ್ಕಾರವಾಗಿರುವುದೋ ಅವರು ಈ ಕೆಲಸವನ್ನು ನೋಡಿಕೊಳ್ಳಲು ಆಗುವುದಿಲ್ಲ. ಅಷ್ಟೇ ಅಲ್ಲದೆ ಅದು ಸಂನ್ಯಾಸ ಧರ್ಮಕ್ಕೆ ವಿರುದ್ಧವೂ ಕೂಡ.

ಬಾತ್ಮೀದಾರ: “ನೀವು ಬೋಧಿಸುವುದು ತುಲನಾತ್ಮಕ ಧರ್ಮವನ್ನೆ?

ಸ್ವಾಮೀಜಿ: “ನಾನು ಬೋಧಿಸುವುದನ್ನು ಎಲ್ಲ ಧರ್ಮಗಳ ತಿರುಳು ಎಂದರೆ ಹೆಚ್ಚು ಅರ್ಥಗರ್ಭಿತವಾಗುವುದು. ಧರ್ಮದಲ್ಲಿ ಗೌಣವಾಗಿರುವ ಭಾಗಗಳನ್ನೆಲ್ಲಾ ತೆಗೆದುಹಾಕಿ ಯಾವುದು ನಿಜವಾಗಿ ಸಾರವತ್ತಾಗಿರುವುದೋ ಅದನ್ನು ಮಾತ್ರ ಹೇಳುತ್ತೇನೆ. ನಾನು ಶ‍್ರೀರಾಮಕೃಷ್ಣ ಪರಮಹಂಸರ ಶಿಷ್ಯ. ಪರಿಶುದ್ಧ ಸಂನ್ಯಾಸಿಗಳು ಅವರು. ಅವರ ಭಾವನೆಗಳಿಗೆ ಮತ್ತು ಅವರ ಪ್ರಭಾವಕ್ಕೆ ನಾನು ಒಳಗಾದೆ. ಈ ಮಹಾಪುರುಷರು ಇತರ ಧರ್ಮಗಳನ್ನು ಟೀಕಾ ದೃಷ್ಟಿಯಿಂದ ನೋಡುತ್ತಿರಲಿಲ್ಲ. ಅವುಗಳಲ್ಲಿರುವ ಒಳ್ಳೆಯದನ್ನು ಮಾತ್ರ ನೋಡುತ್ತಿದ್ದರು. ಅವುಗಳನ್ನು ಹೇಗೆ ಅನುಷ್ಠಾನಕ್ಕೆ ತರಬಹುದು ಎಂಬುದರ ಮೇಲೆಯೆ ಅವರ ಗುರಿ. ಮತ್ತೊಂದು ಧರ್ಮದೊಂದಿಗೆ ಹೋರಾಡುವುದು, ಅವರನ್ನು ಟೀಕಿಸುವುದು, ಅವರ ಬೋಧನೆಗೆ ವಿರೋಧ. ಪ್ರಪಂಚವೆಲ್ಲ ಪ್ರೇಮದ ಮೇಲೆ ನಿಂತಿದೆ ಎಂಬುದೇ ಅವರ ಭಾವನೆ. ಹಿಂದೂಗಳು ಎಂದಿಗೂ ಅನ್ಯಮತೀಯರನ್ನು ಹಿಂಸಿಸುವುದಿಲ್ಲ ಎಂಬುದು ನಿಮಗೆ ಗೊತ್ತಿದೆ. ಆ ದೇಶದಲ್ಲಿ ಎಲ್ಲಾ ಮತೀಯರೂ ಶಾಂತಿಯಿಂದ ಸೌಹಾರ್ದದಲ್ಲಿ ಬಾಳಬಹುದು. ಮಹಮ್ಮದೀಯರು ಕೊಲೆ ಮತ್ತು ಹಿಂಸೆಯನ್ನು ತಮ್ಮೊಡನೆ ತಂದರು. ಅವರು ಬರುವ ತನಕ ಭರತಖಂಡದಲ್ಲಿ ಶಾಂತಿ ನೆಲೆಸಿತ್ತು. ದೇವರನ್ನು ನಂಬದ, ಅಂತಹ ಭಾವನೆಯನ್ನು ಒಂದು ಭ್ರಾಂತಿ ಎನ್ನುವ ಜೈನರನ್ನು ಕೂಡ ಭಾರತೀಯರು ಸಹಿಷ್ಣುತಾ ದೃಷ್ಟಿಯಿಂದ ನೋಡಿದರು. ಇಂದಿಗೂ ಕೂಡ ಜೈನರು ಇರುವರು. ಭರತಖಂಡವು ನಿಜವಾದ ಶಕ್ತಿಗೆ ಮೇಲ್ಪಂಕ್ತಿಯಾಗಿರುವುದು. ನಿಜವಾದ ಶಕ್ತಿಯೇ ನಮ್ರತೆ. ರಭಸ ಹೋರಾಟ ಪೋಟಾಪೋಟಿ ಇವುಗಳೆಲ್ಲ ದೌರ್ಬಲ್ಯದ ಚಿಹ್ನೆಗಳು.”

ಬಾತ್ಮೀದಾರ: “ಇದು ಟಾಲ್​ಸ್ಟಾಯರ ಸಿದ್ಧಾಂತದಂತೆ ಕಾಣುವುದು. ಇದು ಕೇವಲ ವ್ಯಕ್ತಿಗಳಿಗೆ ಸಾಧ್ಯವಾಗಬಹುದಾದರೂ, ವೈಯಕ್ತಿಕವಾಗಿ ಅದನ್ನು ನಾನು ಅನುಮಾನಿಸುತ್ತೇನೆ. ಆದರೆ ಇಡೀ ದೇಶಕ್ಕೆ ಇದು ಹೇಗೆ ಅನ್ವಯಿಸುವುದು?

ಸ್ವಾಮೀಜಿ: “ದೇಶಕ್ಕೂ ಚೆನ್ನಾಗಿಯೇ ಇದು ಅನ್ವಯಿಸುವುದು. ಭರತಖಂಡದ ಕರ್ಮ, ಅದು ಮತ್ತೊಬ್ಬರಿಂದ ಜಯಿಸಲ್ಪಟ್ಟಿತು. ಅವರನ್ನು ಗೆದ್ದ ಮಹಮ್ಮದೀ ಯರನ್ನು ಆಗಲೇ ಹಿಂದೂಗಳು ಗೆದ್ದಿರುವರು. ವಿದ್ಯಾವಂತರಾದ ಮಹಮ್ಮದೀಯರು ಸೂಫಿಗಳು. ಅವರನ್ನು ಹಿಂದೂಗಳಿಂದ ಪ್ರತ್ಯೇಕಿಸಲು ಕಷ್ಟವಾಗುವುದು. ಹಿಂದೂ ಭಾವನೆ ಸೂಫಿಗಳ ಸಂಸ್ಕೃತಿಯನ್ನೆಲ್ಲಾ ಒಳಹೊಕ್ಕಿದೆ. ಅವರು ಶಿಷ್ಯರ ಸ್ಥಾನದಲ್ಲಿರು ವರು. ಮೊಗಲ ಚಕ್ರವರ್ತಿಯಾದ ಪ್ರಖ್ಯಾತ ಅಕ್ಬರ್​ ವ್ಯವಹಾರದಲ್ಲಿ ಹಿಂದೂವೇ ಆಗಿದ್ದನು. ಇಂಗ್ಲೆಂಡನ್ನು ಕೂಡ ಭರತಖಂಡ ಕ್ರಮೇಣ ಜಯಿಸುವುದು. ಇಂದು ಆಂಗ್ಲೇಯರಿಗೆ ಕತ್ತಿಯ ಬಲವಿದೆ. ಆದರೆ ಭಾವನಾ ಪ್ರಪಂಚದಲ್ಲಿ ಇದರಿಂದ ಏನೂ ಪ್ರಯೋಜನವಿಲ್ಲ. ಭಾರತೀಯ ಚಿಂತನೆಯ ವಿಷಯವಾಗಿ ಷೋಫನಿಯರ್​ ಏನು ಹೇಳಿದ ನಿಮಗೆ ಗೊತ್ತೆ? ಅಜ್ಞಾತ ಕಾಲವಾದ ಮೇಲೆ ಗ್ರೀಕ್​ ಮತ್ತು ರೋಮನ್​ ಸಂಸ್ಕೃತಿಯು ಯೂರೋಪಿನ ಮೇಲೆ ಯಾವ ಒಂದು ಪ್ರಭಾವವನ್ನು ಬೀರಿತೋ ಅಷ್ಟೇ ಪ್ರಮುಖವಾಗುವುದು ಭಾರತೀಯ ಭಾವನೆ, ಯೂರೋಪ್​ ಖಂಡದಲ್ಲಿ ಅದು ಪ್ರಚಾರವಾದಾಗ ಎಂದಿರುವನು.”

ಬಾತ್ಮೀದಾರ: “ದಯವಿಟ್ಟು ಕ್ಷಮಿಸಿ, ಹೀಗೆ ಆಗುವ ಹೆಚ್ಚಿನ ಚಿಹ್ನೆಗಳು ಯಾವುವೂ ಕಾಣುತ್ತಿಲ್ಲವಲ್ಲ?”

ಸ್ವಾಮೀಜಿ: “ಬಹುಶಃ ಕಾಣದಿರಬಹುದು. (ಗಂಭೀರವಾಗಿ ಹೇಳಿದರು) ಹಳೆಯ ಯೂರೋಪಿನ ಜಾಗೃತಿಯಲ್ಲೂ ಯಾವಾಗಲೂ ಬಹುಶಃ ಯಾವ ಚಿಹ್ನೆ ಯೂ ಕಂಡಿರಲಿಲ್ಲವೆಂದು ತೋರುವುದು. ಜಾಗೃತಿಯಾದಮೇಲೂ ಅದು ಹಲ ವರಿಗೆ ಕಾಣಲಿಲ್ಲ. ಕಾಲ ಬದಲಾವಣೆಯನ್ನು ಗಮನಿಸುವ ಕೆಲವರಿಗೆ ಈಗೊಂದು ದೊಡ್ಡ ಭಾವನಾತರಂಗ ಏಳುತ್ತಿದೆ ಎಂಬ ಚಿಹ್ನೆ ಗೊತ್ತಾಗುತ್ತಿದೆ. ಇತ್ತೀಚೆಗೆ ಪ್ರಾಚ್ಯ ಸಂಶೋಧನೆ ಬಹಳ ಮುಂದುವರಿದಿದೆ. ಸದ್ಯಕ್ಕೆ ಅದು ಈಗ ಕೆಲವು ಪಂಡಿತರ ಕೈಯಲ್ಲಿದೆ. ಅವರು ಮಾಡಿರುವ ಕೆಲಸ ನೀರಸವಾಗಿ ಕಾಣುವುದು; ಸಾಧಾರಣ ಜನಕ್ಕೆ ಅರ್ಥವಾಗುವಂತಿಲ್ಲ. ಆದರೆ ಕ್ರಮೇಣ ಜ್ಞಾನ ಜ್ಯೋತಿ ಮೂಡುವುದು.”

ಬಾತ್ಮೀದಾರ: “ಭವಿಷ್ಯದಲ್ಲಿ ಇಂಡಿಯಾ ಎಲ್ಲಾ ದೇಶಗಳನ್ನೂ ಜಯಿಸು ವುದು! ಆದರೂ ತನ್ನ ಸಂದೇಶವನ್ನು ಹರಡಲು ಬೋಧಕರನ್ನು ಹೊರಗೆ ಕಳಿಸುತ್ತಿಲ್ಲ. ಇಡೀ ಪ್ರಪಂಚ ತನ್ನ ಕಾಲ ಬಳಿಗೆ ಬರುವ ತನಕ ಕಾಯುವುದು ಎಂದು ತೋರುತ್ತದೆ!”

ಸ್ವಾಮೀಜಿ: “ಭರತಖಂಡ ಹಿಂದಿನ ಕಾಲದಲ್ಲಿ ದೊಡ್ಡ ಮಿಷನರಿ ಧರ್ಮ ವಾಗಿತ್ತು. ನೂರಾರು ವರುಷಗಳ ಹಿಂದೆ ಇಂಗ್ಲೆಂಡ್​ ಇನ್ನೂ ಕ್ರೈಸ್ತರಾಗುವುದಕ್ಕೆ ಮುನ್ನ ಬುದ್ಧ ಏಷ್ಯಾಖಂಡಕ್ಕೆಲ್ಲಾ ತನ್ನ ಸಂದೇಶವನ್ನು ಹರಡಲು ಬೋಧಕರನ್ನು ಕಳುಹಿಸಿದನು. ಭಾವನಾ ಜಗತ್ತು ಕ್ರಮೇಣ ಪರಿವರ್ತನವಾಗುತ್ತಿದೆ. ನಾವೀಗ ಇನ್ನೂ ಪ್ರಾರಂಭದಲ್ಲಿರುವೆವು. ಯಾವ ಒಂದು ನಿರ್ದಿಷ್ಟ ಧರ್ಮಕ್ಕೂ ಸೇರದವರ ಸಂಖ್ಯೆ ಹೆಚ್ಚುತ್ತಿದೆ. ಇದು ವಿದ್ಯಾವಂತರಲ್ಲಿ ಹೆಚ್ಚಿದೆ. ಈಚೆಗೆ ನಡೆದ ಅಮೆರಿಕ ಜನಗಣತಿಯಲ್ಲಿ ಅನೇಕರು ತಾವು ಯಾವ ಮತಕ್ಕೂ ಸೇರಿದವರಲ್ಲ ಎಂದು ಸೂಚಿಸಿ ರುವರು. ಎಲ್ಲಾ ಧರ್ಮಗಳೂ ಒಂದೇ ಸತ್ಯವನ್ನು ಸಾರುತ್ತವೆ. ಎಲ್ಲವೂ ಒಟ್ಟಿಗೆ ಪ್ರವರ್ಧಮಾನಕ್ಕೆ ಬರುವುವು, ಇಲ್ಲವೆ ಹಾಳಾಗುವುವು. ಇವೆಲ್ಲಾ ಒಂದೇ ಸತ್ಯದ ಕಿರಣಗಳು. ಜನರ ವೈವಿಧ್ಯಕ್ಕೆ ತಕ್ಕಂತೆ ಅವು ಕಾಣುತ್ತಿವೆ.”

ಬಾತ್ಮೀದಾರ: “ನಾವೀಗ ಅದನ್ನು ಸಮೀಪಿಸುತ್ತಿರುವೆವು. ಆ ಮುಖ್ಯ ಸತ್ಯ ಯಾವುದು?

ಸ್ವಾಮೀಜಿ: “ಅದೇ ಒಳಗಿರುವ ದಿವ್ಯತೆ. ಪ್ರತಿಯೊಬ್ಬನೂ ಅವನು ಎಷ್ಟೇ ಅಧೋಗತಿಗೆ ಇಳಿದಿದ್ದರೂ ಅವನಲ್ಲಿ ದಿವ್ಯತೆ ಹುದುಗಿರುವುದು. ಆ ದಿವ್ಯತೆ ಕಣ್ಣಿಗೆ ಕಾಣದಂತೆ ಮರೆಯಾಗಿರುವುದು.ಸಿಪಾಯಿ ದಂಗೆಯ ಒಂದು ಘಟನೆ ನನ್ನ ನೆನಪಿಗೆ ಬರುವುದು. ಕೆಲವು ವರುಷಗಳಿಂದ ಮೌನ ವ್ರತವನ್ನು ಆಚರಿಸುತ್ತಿದ್ದ ಒಬ್ಬ ಸ್ವಾಮಿಯನ್ನು ಮಹಮ್ಮದೀಯನೊಬ್ಬನು ಭರ್ಜಿಯಿಂದ ತಿವಿದನು. ಜನ ಅವನನ್ನು ಹಿಡಿದು ಸ್ವಾಮಿಗಳ ಹತ್ತಿರ ಕರೆದುಕೊಂಡು ಹೋಗಿ ‘ಸ್ವಾಮೀಜಿ, ಒಂದು ಮಾತು ಅಪ್ಪಣೆಮಾಡಿ, ಇವನನ್ನು ಇಲ್ಲದಂತೆ ಮಾಡುವೆವು’ ಎಂದರು. ಸ್ವಾಮೀಜಿ ಹಲವು ವರುಷಗಳ ಮೌನವ್ರತವನ್ನು ಕೊನೆಗಾಣಿಸಿ ಈ ಮಾತನ್ನು ಮಾತ್ರ ಹೇಳಿ ದರು, ‘ಮಕ್ಕಳೆ, ನೀವೆಲ್ಲಾ ಭ್ರಾಂತರು, ಅವನು ನಿಜವಾಗಿಯೂ ದೇವರೆ.’ ನಾವು ಕಲಿಯಬೇಕಾದ ದೊಡ್ಡ ನೀತಿಯೇ, ಎಲ್ಲಾ ಜೀವಿಗಳ ಹಿಂದೆ ಏಕತೆ ಇದೆ ಎಂಬುದು. ಅದನ್ನು ದೇವರು, ಪ್ರೀತಿ, ಪರಮಾತ್ಮ ಯಹೋವ ಎಂದು ಏನು ಬೇಕಾದರೂ ಕರೆಯಬಹುದು. ಆ ಏಕತೆಯೊಂದೇ ಒಂದು ಕನಿಷ್ಠ ಕೀಟದಿಂದ ಹಿಡಿದು ಮಹಾ ಪುರುಷನವರೆಗೆ ಪ್ರಪಂಚದಲ್ಲಿ ಇರುವುದು. ಒಂದು ಹೆಪ್ಪುಗಟ್ಟಿಹೋಗಿರುವ ಸಮುದ್ರವನ್ನು ಕಲ್ಪಿಸಿಕೊಳ್ಳಿ. ಅದರಲ್ಲಿ ಹಲವು ರಂಧ್ರಗಳಿವೆ. ಆ ಪ್ರತಿಯೊಂದು ರಂಧ್ರವೂ ಮುಕ್ತಾತ್ಮನಾದ ಜೀವ, ಘನೀಭೂತವಾದ ಮಂಜಿನಿಂದ ಪಾರಾಗಿ ಹೋಗಲು ಯತ್ನಿಸುತ್ತಿರುವುದು.”

ಬಾತ್ಮೀದಾರ: “ಪೌರಸ್ತ್ಯರ ಮತ್ತು ಪಾಶ್ಚಾತ್ಯರ ಸಂಸ್ಕೃತಿಗಳ ನಡುವೆ ಒಂದು ವ್ಯತ್ಯಾಸವಿದೆ ಎಂದು ನನಗೆ ತೋರುವುದು. ನೀವು ಸಂನ್ಯಾಸ, ಏಕಾಗ್ರತೆಮುಂತಾದುವುಗಳ ಮೂಲಕ ಕೆಲವು ವ್ಯಕ್ತಿಗಳನ್ನು ಶ್ರೇಷ್ಠ ಶಿಖರಕ್ಕೆ ಒಯ್ಯಲು ಪ್ರಯತ್ನಿಸುತ್ತಿರುವಿರಿ. ಪಾಶ್ಚಾತ್ಯರ ದೃಷ್ಟಿಯಾದರೋ ಇಡೀ ಸಾಮಾಜಿಕ ಪರಿಸ್ಥಿತಿ ಯನ್ನು ಉತ್ತಮಗೊಳಿಸುವುದಾಗಿದೆ. ಆದಕಾರಣವೇ ರಾಜಕೀಯ ಮತ್ತು ಸಾಮಾಜಿಕ ಕ್ಷೇತ್ರಗಳಲ್ಲಿ ನಾವು ಕೆಲಸ ಮಾಡುವುದು. ಇವುಗಳ ಭದ್ರ ಬುನಾದಿಯ ಮೇಲೆ ನಮ್ಮ ಜನರ ಮೇಲ್ಮೆ ನಿಂತಿದೆ.”

ಸ್ವಾಮೀಜಿ: “ರಾಜಕೀಯವಾಗಲೀ ಸಾಮಾಜಿಕವಾಗಲೀ ಇವುಗಳ ಮೂಲವೆಲ್ಲ ಮನುಷ್ಯನ ಒಳ್ಳೆಯತನದ ಮೇಲೆ ನಿಂತಿದೆ. ಪಾರ್ಲಿಮೆಂಟು ಯಾವುದಾದರೂ ಒಂದು ಕಾನೂನನ್ನು ಜಾರಿಗೆ ತರುವುದರಿಂದ ಯಾವ ದೇಶವೂ ಒಳ್ಳೆಯದಾಗು ವುದಿಲ್ಲ, ಪ್ರಖ್ಯಾತವಾಗುವುದಿಲ್ಲ. ಅಲ್ಲಿಯ ಜನ ಒಳ್ಳೆಯವರು ಪ್ರಖ್ಯಾತರು ಆಗಿರುವುದರಿಂದ ದೇಶ ಒಳ್ಳೆಯದಾಗುವುದು. ಬಹಳ ಅದ್ಭುತವಾದ ಸಮಾಜ ಸಂಸ್ಥೆಯಿದ್ದ ಚೈನಾ ದೇಶವನ್ನು ನಾನು ನೋಡಿರುವೆನು. ಆದರೆ ಈಗಿನ ಚೈನಾ ದೇಶ ಐಕ್ಯಮತ್ಯವಿಲ್ಲದ ದೊಂಬಿಯಂತಿದೆ. ಏಕೆಂದರೆ ಹಿಂದಿನ ಕಾಲದಲ್ಲಿ ಯಾವ ಒಂದು ಸಮಾಜವನ್ನು ರಚಿಸಿದ್ದರೋ ಅದನ್ನು ನಿರ್ವಹಿಸುವ ಶಕ್ತಿ ಈಗಿನ ಜನಾಂಗಕ್ಕೆ ಇಲ್ಲವೆಂದು ತೋರುವುದು. ಧರ್ಮ ಎಲ್ಲದರ ಮೂಲಕ್ಕೆ ಹೋಗುವುದು. ಇದು ಸರಿಯಾದರೆ ಎಲ್ಲವೂ ಸರಿ.”

ಬಾತ್ಮೀದಾರ: “ಪ್ರತಿಯೊಬ್ಬರಲ್ಲಿಯೂ ದಿವ್ಯತೆ ಇದೆ. ಆದರೆ ಅದು ಸುಪ್ತವಾಗಿದೆ ಎಂಬ ವಿಷಯ ಬಹಳ ಅಸ್ಪಷ್ಟವಾಗಿರುವುದು. ಇದು ವ್ಯವಹಾರ ಪ್ರಪಂಚಕ್ಕೆ ನಿಲುಕುವಂತೆ ಕಾಣುವುದಿಲ್ಲ.”

ಸ್ವಾಮೀಜಿ: ‘ಜನ ಅನೇಕ ವೇಳೆ ಒಂದು ಉದ್ದೇಶವನ್ನು ಇಟ್ಟುಕೊಂಡು ಕೆಲಸ ಮಾಡುವರು. ಆದರೆ ಅದು ಅವರಿಗೆ ಗೊತ್ತಿರುವುದಿಲ್ಲ. ಸರ್ಕಾರ, ನಿಯಮಗಳು, ರಾಜಕೀಯ ಇವುಗಳೆಲ್ಲ ತಾತ್ಕಾಲಿಕ ಅವಸ್ಥೆ ಎಂಬುದನ್ನು ನಾವು ಗಮನದಲ್ಲಿಡಬೇಕು. ಇವುಗಳಾಚೆ ಗುರಿ ಇರುವುದು. ಅಲ್ಲಿ ಯಾವ ಕಾನೂನು ಬೇಕಾಗಿಲ್ಲ. ಸಂನ್ಯಾಸಿಯೆಂದರೆ ಸಾಮಾನ್ಯ ನಿಯಮಕ್ಕೆ ಹೊರಗಾದವರು ಎಂದು ಅರ್ಥ. ಅವನು ಪವಿತ್ರ ಶೂನ್ಯವಾದಿ ಎಂದು ಬೇಕಾದರೆ ಹೇಳಬಹುದು. ಆದರೆ ಆ ಪದವನ್ನು ಉಪಯೋಗಿಸುವವರು ಅದನ್ನು ಸರಿಯಾಗಿ ಅರ್ಥ ಮಾಡಿಕೊಳ್ಳು ವುದಿಲ್ಲ. ಎಲ್ಲಾ ಮಹಾಪುರುಷರೂ ಬೋಧಿಸುವುದು ಒಂದನ್ನೇ. ನಿಯಮಗಳೇ ಜೀವನದ ಆಧಾರ ಸ್ಥಂಭ ಅಲ್ಲ ಎನ್ನುವುದನ್ನೂ ನೀತಿ ಮತ್ತು ಪಾವಿತ್ರ್ಯಗಳೇ ನಿಜವಾದ ಶಕ್ತಿ ಎನ್ನುವುದನ್ನೂ ಕ್ರಿಸ್ತನು ಕಂಡುಕೊಂಡನು. ಪೌರಸ್ತ್ಯ ಆತ್ಮೋನ್ನ ತಿಗೆ ಕೆಲಸ ಮಾಡುವರು. ಪಾಶ್ಚಾತ್ಯರು ಸಮಾಜವನ್ನು ಉತ್ತಮಪಡಿಸಲು ಕೆಲಸ ಮಾಡುವರು ಎಂದು ನೀವು ಹೇಳುವುದರಲ್ಲಿ ಒಂದು ತೋರಿಕೆಯ ಆತ್ಮ ಮತ್ತೊಂದು ನಿಜವಾದ ಆತ್ಮವಿದೆ ಎಂಬುದನ್ನು ನೀವು ಮರೆಯಬಾರದು.”

ಬಾತ್ಮೀದಾರ: ‘ನಾವು ತೋರಿಕೆಯ ಆತ್ಮಕ್ಕಾಗಿಯೂ, ನೀವು ನಿಜವಾದ ಆತ್ಮಕ್ಕಾಗಿಯೂ ಕೆಲಸ ಮಾಡುತ್ತಿರುವೆವು ಎಂಬುದೇ ನಿಮ್ಮ ಮತ?”

ಸ್ವಾಮೀಜಿ: “ಮನಸ್ಸು ಪೂರ್ಣತೆಯನ್ನು ಮುಟ್ಟಲು ಹಲವು ಹಂತಗಳಲ್ಲಿ ಕೆಲಸ ಮಾಡುವುದು. ಮೊದಲು ಸ್ಥೂಲವಾಗಿರುವುದನ್ನು ತೆಗೆದುಕೊಳ್ಳುವುದು, ಅನಂತರ ಸೂಕ್ಷ್ಮದ ಕಡೆಗೆ ಗಮನ ಕೊಡುವುದು. ವಿಶ್ವ ಸಹೋದರ ಭಾವನೆ ಹೇಗೆ ಬಂತು ಅದನ್ನು ನೋಡಿ. ಮೊದಲು ಇದನ್ನು ಒಂದು ಪಂಥದ ಭಾವನೆ ಎಂದು ತಿಳಿದರು. ಅದು ಆಗ ಬಹಳ ಸಂಕುಚಿತವಾಗಿತ್ತು. ಪ್ರತ್ಯೇಕವಾಗಿತ್ತು, ಬದ ಲಾಯಿಸುವುದಕ್ಕೆ ಕಷ್ಟವಾಗಿತ್ತು. ಕ್ರಮೇಣ ವಿಶಾಲವಾದ ಸಾಮಾನ್ಯ ಭಾವನೆಗಳಿಗೆ ಬರುತ್ತೇವೆ, ಸೂಕ್ಷ್ಮಭಾವನೆಗಳ ಜಗತ್ತಿಗೆ ಬರುತ್ತೇವೆ.

ಬಾತ್ಮೀದಾರ: “ಯಾವ ಪಂಥಗಳ ಮೇಲೆ ಇಂಗ್ಲೀಷಿನವರಿಗೆ ಹೆಚ್ಚು ಅಭಿಮಾನ ವಿರುವುದೋ ಅದು ಕ್ರಮೇಣ ನಿರ್ನಾಮವಾಗುವುದೆಂಬುದೇ ನಿಮ್ಮ ಅಭಿಪ್ರಾಯ? ಫ್ರೆಂಚ್​ ದೇಶವನೊಬ್ಬ ಈ ವಿಷಯದಲ್ಲಿ ಹೇಳಿರುವುದು ನಿಮಗೆ ಜ್ಞಾಪಕವಿರ ಬಹುದು: ‘ಇಂಗ್ಲೆಂಡ್​ನಲ್ಲಿ ಸಾವಿರಾರು ಪಂಥಗಳಿವೆ, ಆದರೆ ಅವರಿಗೆಲ್ಲ ಸಮಾನವಾದ ರುಚಿ ಇದೆ.”

ಸ್ವಾಮೀಜಿ: “ಅವು ಕಾಲಕ್ರಮೇಣ ಮಾಯವಾಗಿ ಹೋಗುವುದರಲ್ಲಿ ಸಂದೇಹ ವಿಲ್ಲ. ಅವುಗಳು ನಿಂತಿರುವುದೇ ಗೌಣವಸ್ತುವಿನ ಮೇಲೆ. ಅವುಗಳಲ್ಲಿ ಮುಖ್ಯವಾಗಿ ರುವಾ ಭಾಗ ಇರುವುದು. ಅದೇ ಅನಂತರ ಬೇರೊಂದು ಪಂಥವಾಗುವುದು. ಒಂದು ಚರ್ಚಿನಲ್ಲಿ ಹುಟ್ಟುವುದು ಒಳ್ಳೆಯದು, ಆದರೆ ಅಲ್ಲೇ ಸಾಯುವುದು ಕೆಟ್ಟದ್ದು ಎಂಬ ಹಳೆಯ ನಾಣ್ನುಡಿ ನಿಮಗೆ ಜ್ಞಾಪಕವಿರಬಹುದು.”

ಬಾತ್ಮೀದಾರ: “ನಿಮ್ಮ ಕೆಲಸ ಇಂಗ್ಲೆಂಡಿನಲ್ಲಿ ಹೇಗೆ ಮುಂದುವರಿಯುತ್ತಿದೆ ಎಂಬುದನ್ನು ಸ್ವಲ್ಪ ಹೇಳುವಿರಿ?”

ಸ್ವಾಮೀಜಿ: “ನಿಧಾನವಾಗಿ ನಡೆಯುತ್ತಿದೆ. ಅದಕ್ಕೆ ಕಾರಣಗಳನ್ನು ನಾನು ಆಗಲೇ ಹೇಳಿರುವೆನು. ನೀವು ಮೂಲಭೂತ ವಸ್ತುವನ್ನು ಕುರಿತು ಮಾತನಾಡುವಾಗ ಬೆಳವಣಿಗೆ ಯಾವಾಗಲೂ ನಿಧಾನ. ಈ ಭಾವನೆಗಳು ಒಂದಲ್ಲ ಒಂದು ರೀತಿ ಯಲ್ಲಿ ಹರಡುತ್ತಿವೆ ಎಂದು ನಾನು ಹೇಳಲೇಬೇಕಾಗಿಲ್ಲ. ಆ ಭಾವನೆಗಳ ಪ್ರಚಾರ ಕ್ಕೆ ಸಕಾಲ ಪ್ರಾಪ್ತವಾಗಿದೆ ಎಂದು ನಮ್ಮಲ್ಲಿ ಹಲವರಿಗೆ ತೋರುವುದು.”

ಅನಂತರ ನಾನು ಅವರ ಕೆಲಸ ಹೇಗೆ ಮುಂದುವರಿಯುತ್ತಿದೆ ಎಂಬುದನ್ನು ಕೇಳಿದೆ. ಹೇಗೆ ಹಿಂದೆ ಜ್ಞಾನದಾನ ಮಾಡುತ್ತಿದ್ದಾರೋ ಆದರಂತೆಯೇ ಯಾವ ಹಣವೂ ಇಲ್ಲದೆ, ಬೆಲೆಯೂ ಇಲ್ಲದೆ ಈ ಬೋಧನೆಯನ್ನು ಜನರಿಗೆ ಹರಡುತ್ತಿರುವರು. ಯಾರು ಇದನ್ನು ಸ್ವೀಕರಿಸುವರೋ ಅವರು ತಾವೇ ಏನಾದರೂ ಸಹಾಯ ಮಾಡಿದರೆ ಮಾತ್ರ ಅದನ್ನು ಸ್ವೀಕರಿಸುವರು. ಸ್ವಾಮಿಗಳು ಪೌರಸ್ತ್ಯ ವೇಷದಲ್ಲಿರುವ ಆಕರ್ಷಕ ವ್ಯಕ್ತಿ. ಅವರದು ಸರಳವಾದ ಸ್ನೇಹಮಯವಾದ ಸ್ವಭಾವ. ಅವರಲ್ಲಿ ಜನ ಸಾಧಾರಣವಾಗಿ ಭಾವಿಸುವ ಯಾವ ಬಾಹ್ಯ ಬೈರಾಗಿಯ ಸೂಚನೆಗಳ ಕಾಣುವುದಿಲ್ಲ. ಇಂಗ್ಲಿಷ್​ ಭಾಷೆಯಲ್ಲಿ ಅದ್ವಿತೀಯ ಪಾಂಡಿತ್ಯವಿದೆ. ಸಂಭಾಷಣಾ ವೈಖರಿ ಚೇತೋಹಾರಿಯಾಗಿದೆ. ಇವುಗಳೆಲ್ಲ ಅವರನ್ನು ಒಬ್ಬ ಆಕರ್ಷಣೀಯ ವ್ಯಕ್ತಿಯನ್ನಾಗಿ ಮಾಡಿದೆ. ಅವರ ಸಂನ್ಯಾಸಿ ವ್ರತದಲ್ಲಿ ಅಧಿಕಾರ, ಆಸ್ತಿ, ಕೀರ್ತಿ ಎಲ್ಲವನ್ನೂ ತ್ಯಾಗಮಾಡಿ ಸತತ ಆಧ್ಯಾತ್ಮಿಕ ಪುರೋಗಮನವನ್ನೇ ಚಿಂತಿಸಬೇಕಾಗಿದೆ.

