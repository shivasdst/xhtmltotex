
\chapter[ಮುಕ್ತಿ ಮಾರ್ಗ ]{ಮುಕ್ತಿ ಮಾರ್ಗ \protect\footnote{\engfoot{C.W. Vol. VIII, P. 239}}}

\centerline{(೧೯೦೦ರ ಮಾರ್ಚ್​ ೧೨ರಂದು ಓಕ್ಲ್ಯಾಂಡಿನಲ್ಲಿ ನೀಡಿದ ಉಪನ್ಯಾಸದ ವರದಿ)}

ಒಬ್ಬನು ದೇವರು ಸ್ವರ್ಗದಲ್ಲಿ ಇರುವನು ಎನ್ನುವನು. ಮತ್ತೊಬ್ಬನು ದೇವರು ಎಲ್ಲೆಲ್ಲಿಯೂ ಇರುವನು ಎನ್ನುವನು. ಪರೀಕ್ಷಾ ಸಮಯ ಬಂದಾಗ ಗುರಿ ಎಲ್ಲರಿಗೂ ಒಂದೇ ಎಂದು ಗೊತ್ತಾಗುವುದು. ನಾವೆಲ್ಲಾ ಬೇರೆ ಬೇರೆ ಕಡೆಗಳಿಂದ ಹೋಗುತ್ತೇವೆ. ಆದರೆ ಗುರಿಯು ಬೇರೆ ಬೇರೆಯಲ್ಲ.

ಪ್ರತಿಯೊಂದು ಧರ್ಮದ ಮುಖ ಪಲ್ಲವಿಯೇ ತ್ಯಾಗವಾಗಿದೆ. ಪ್ರತಿ ಯೊಬ್ಬರಿಗೂ ಸತ್ಯ ಬೇಕಾಗಿದೆ. ನಾವು ಅದನ್ನು ಇಚ್ಛಿಸಲಿ, ಬಿಡಲಿ, ಅದು ನಮಗೆ ಬಂದೇ ಬರುವುದು ಎಂದು ಗೊತ್ತಿದೆ. ನಾವೆಲ್ಲಾ ಅದಕ್ಕಾಗಿ ಪ್ರಯತ್ನಿಸುತ್ತಿರುವೆವು. ಆದರೆ ಅದನ್ನು ಪಡೆಯುವುದಕ್ಕೆ ಯಾವುದು ಅಡ್ಡಿಯಾಗಿದೆ? ನಾವೇ ಅದಕ್ಕೆ ಅಡ್ಡಿ. ನಿಮ್ಮ ಪೂರ್ವಿಕರು ಅದನ್ನು ಸೈತಾನ್​ ಎಂದು ಕರೆದರು. ಆದರೆ ಅದು ನಮ್ಮ ಭ್ರಾಂತಿಯೇ ಆಗಿದೆ.

ನಾವು ಗುಲಾಮಗಿರಿಯಲ್ಲಿರುವೆವು. ಅದರಿಂದ ಪಾರಾದರೆ ನಾವು ಸತ್ತು ಹೋಗುವೆವು ಎಂದು ಭಾವಿಸುವೆವು. ತೊಂಬತ್ತು ವರುಷ ಕತ್ತಲೆಯ ಸೆರೆಮನೆ ಯಲ್ಲಿದ್ದ ಮನುಷ್ಯನಂತೆ ನಾವು ಇರುವೆವು. ಅವನನ್ನು ಬಿಡುಗಡೆ ಮಾಡಿದಾಗ, ಅವನು ಹೊರಗೆ ಹೋದಾಗ ಆ ಬೆಳಕನ್ನು ಅವನು ಸಹಿಸಲಾರದೆಯೇ ಹೋಗುತ್ತಾನೆ. ಪುನಃ ತನ್ನನ್ನು ಜೈಲಿಗೆಯೇ ಕರೆದುಕೊಂಡು ಹೋಗಿ ಎಂದು ಅಂಗಲಾಚಿ ಬೇಡಿಕೊಳ್ಳುತ್ತಾನೆ. ಹೊಸ ಸ್ವಾತಂತ್ರ್ಯದ ಜೀವನ ನಿಮ್ಮ ಮುಂದೆ ಇದ್ದರೂ ಹಳೆಯ ಗುಲಾಮಗಿರಿಯ ಜೀವನವನ್ನು ನೀವು ತ್ಯಜಿಸಲಾರಿರಿ.

ನಾವು ವಸ್ತುವಿನ ಮೂಲಕ್ಕೆ ಹೋಗಬೇಕಾದರೆ ಬಹಳ ಕಷ್ಟ. ಧರ್ಮದ ಬಗ್ಗೆ ಮಾನವರಲ್ಲಿ ಎಂತೆಂತಹ ವಿಚಿತ್ರ ಭಾವನೆಗಳಿವೆ! ಒಂದು ದೇಶದಲ್ಲಿ ತನ್ನ ಧರ್ಮ ಹೇಳುವುದು ಎಂದು ಒಬ್ಬ ಹಲವು ಹೆಂಡರನ್ನು ಮದುವೆಯಾಗುವನು. ಮತ್ತೊಂದು ದೇಶದಲ್ಲಿ ಸ್ತ್ರೀಯೊಬ್ಬಳು ಕೇವಲ ಧರ್ಮಕ್ಕಾಗಿಯೇ ಹಲವು ಪತಿಗಳನ್ನು ವರಿಸುವಳು. ಕೆಲವರಲ್ಲಿ ಇಬ್ಬರು ದೇವರುಗಳಿರುವರು. ಕೆಲವು ಧರ್ಮಗಳಲ್ಲಿ ಒಬ್ಬನೇ ದೇವರು, ಮತ್ತೆ ಕೆಲವು ಧರ್ಮಗಳಲ್ಲಿ ದೇವರೇ ಇಲ್ಲ.

ಪ್ರೀತಿ ಮತ್ತು ಸೇವೆಯೇ ನಿಜವಾದ ಮುಕ್ತಿಮಾರ್ಗ. ನೀವು ಯಾವುದನ್ನೋ ಚೆನ್ನಾಗಿ ಕಲಿಯುತ್ತೀರಿ ಎಂದು ಭಾವಿಸಿ. ಕೆಲವು ಕಾಲದ ಮೇಲೆ ಅದನ್ನೆಲ್ಲಾ ಚೆನ್ನಾಗಿ ಜ್ಞಾಪಿಸಿಕೊಳ್ಳುವುದಕ್ಕೆ ಆಗುವುದಿಲ್ಲ. ಆದರೂ ಅದು ನಿಮ್ಮ ಮನಸ್ಸಿನ ಆಳಕ್ಕೆ ಹೋಗಿ ನಿಮ್ಮ ಜೀವನದ ಒಂದು ಭಾಗವಾಗಿದೆ. ನೀವು ಕೆಲಸಮಾಡಿದಂತೆಲ್ಲ ಅದು ಒಳ್ಳೆಯದೋ ಕೆಟ್ಟದ್ದೋ ಅದಕ್ಕೆ ತಕ್ಕಂತೆ ನಿಮ್ಮ ಭವಿಷ್ಯವನ್ನು ನಿರ್ಧರಿಸು ವುದು. ನೀವು ಒಳ್ಳೆಯ ಕೆಲಸವನ್ನು ಕೇವಲ ಕರ್ಮದ ದೃಷ್ಟಿಯಿಂದ ಕರ್ಮಕ್ಕಾಗಿ ಕರ್ಮ ಎಂದು ಮಾಡಿದರೆ ನೀವು ಇಚ್ಛಿಸುವ ಸ್ವರ್ಗಕ್ಕೆ ಹೋಗುವಿರಿ.

ಪ್ರಪಂಚದ ಇತಿಹಾಸ ಅಲ್ಲಿನ ಮಹಾಪುರುಷರ ಜೀವನವಲ್ಲ, ಅಲ್ಲಿಯ ದೇವಮಾನವರ ಜೀವನವಲ್ಲ. ಅದು ಸಮುದ್ರದ ಸಣ್ಣ ಸಣ್ಣ ದ್ವೀಪಗಳಂತೆ ಇದೆ. ಈ ಸಣ್ಣ ಸಣ್ಣ ದ್ವೀಪಸ್ತೋಮ ಸಮುದ್ರದಲ್ಲಿ ತೇಲಿಬಂದ ವಸ್ತುಗಳನ್ನೆಲ್ಲಾ ತಮ್ಮಲ್ಲಿ ಸೇರಿಸಿಕೊಂಡು ಅನಂತರ ಒಂದು ಖಂಡವಾಗುವುದು. ಪ್ರತಿಯೊಂದು ಮನೆಯಲ್ಲಿ ಮಾಡಿದ ಚಿಕ್ಕಪುಟ್ಟ ನಿಃಸ್ವಾರ್ಥ ಕರ್ಮಗಳಲ್ಲಿಯೇ ಜಗತ್ತಿನ ಇತಿಹಾಸವಿದೆ. ಮಾನವ ಧರ್ಮವನ್ನು ಸ್ವೀಕರಿಸುವುದಕ್ಕೆ ಕಾರಣ ತಾನು ತನ್ನ ಸ್ವಂತ ನಿರ್ಣಯದ ಮೇಲೆ ನಿಲ್ಲಲಾರದೆ ಇರುವುದು. ಯಾವುದೋ ಒಂದು ಆಯೋಗ್ಯವಾದ ಸ್ಥಳದಿಂದ ಪಾರಾಗುವುದಕ್ಕೆ ಇರುವ ಒಂದು ರಾಜಮಾರ್ಗವೇ ಧರ್ಮ ಎಂದು ಅದನ್ನು ಸ್ವೀಕರಿಸುವನು.

ಭಗವಂತನ ಮೇಲಿಡುವ ಭಕ್ತಿಯಮೇಲೆ ಮುಕ್ತಿ ನಿಂತಿದೆ. ನೀನಿಲ್ಲದೆ ಇದ್ದರೆ ನಾನು ಬಾಳಲಾರೆ ಎಂದು ನಿನ್ನ ಹೆಂಡತಿ ಹೇಳುವಳು. ಕೆಲವರು ತಮ್ಮ ಹಣ ಕಳೆದುಕೊಂಡರೆ ಅವರನ್ನು ಹುಚ್ಚರ ಆಸ್ಪತ್ರೆಗೆ ಕಳುಹಿಸಬೇಕಾಗುವುದು. ದೇವರಿ ಗಾಗಿ ನೀವು ಹೀಗೆ ಮರುಗುವಿರಾ? ನೀವು ದೇವರಿಗಾಗಿ ನಿಮ್ಮ ದ್ರವ್ಯ, ಸ್ನೇಹಿತರು, ತಾಯಿತಂದೆಗಳು, ಸಹೋದರ ಸಹೋದರಿಯರು ಇವರನ್ನೂ ಮತ್ತು ಪ್ರಪಂಚ ದಲ್ಲಿ ಪ್ರಿಯವಾಗಿರುವುದೆಲ್ಲವನ್ನೂ ತ್ಯಜಿಸಿ ದೇವರನ್ನು ಕೇವಲ ಭಕ್ತಿಗಾಗಿ, ಅವನ ಪ್ರೀತಿಗಾಗಿ ಮಾತ್ರ ಪ್ರಾರ್ಥಿಸಿದರೆ ನಿಮಗೆ ಮುಕ್ತಿ ದೊರಕಿದಂತೆ.

