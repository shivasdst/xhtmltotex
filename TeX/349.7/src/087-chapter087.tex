
\vspace{-0.5cm}

\chapter[ತನ್ನ ಅದೃಷ್ಟಕ್ಕೆ ತಾನೇ ಹೊಣೆ ]{ತನ್ನ ಅದೃಷ್ಟಕ್ಕೆ ತಾನೇ ಹೊಣೆ \protect\footnote{\engfoot{C.W. Vol. VIII p. 183}}}

ದಕ್ಷಿಣ ಭಾರತದಲ್ಲಿ ಒಂದು ಅತ್ಯಂತ ಬಲಶಾಲಿಯಾದ ರಾಜಮನೆತನ ಮತ್ತು ಆ\break ಮನೆತನಕ್ಕೆ ಸೇರಿದ ಎಲ್ಲ ಪ್ರಮುಖ ವ್ಯಕ್ತಿಗಳ ಜಾತಕವನ್ನು ಅವರು ಹುಟ್ಟಿದಂದಿನಿಂದಲೇ ಲೆಕ್ಕ ಹಾಕಿ ರೂಪಿಸಿಡುವ ಪದ್ಧತಿ ಅವರಲ್ಲಿತ್ತು. ಹೀಗೆ ಮುಂದಾಗುವ ಪ್ರಮುಖ ಘಟನೆಗಳ ಪಟ್ಟಿ ಅವರಲ್ಲಿತ್ತು, ಮತ್ತು ಮುಂದೆ ನಡೆಯುವ ಘಟನೆಗಳೊಂದಿಗೆ ಅದನ್ನು ಹೋಲಿಸುತ್ತಿದ್ದರು. ಹೀಗೆ ಸಾವಿರಾರು ವರ್ಷ ಮಾಡಿದ ಮೇಲೆ ಕೆಲವು ಹೊಂದಾಣಿಕೆಗಳನ್ನು ಅವರು ಕಂಡುಕೊಂಡರು. ಇದರ ಆಧಾರದ ಮೇಲೆ ಕೆಲವು ಸಿದ್ಧಾಂತಗಳನ್ನು ರೂಪಿಸಿ ಒಂದು ಬೃಹದ್​ ಗ್ರಂಥವನ್ನು ರಚಿಸಲಾಯಿತು. ಆ ರಾಜಮನೆತನವು ನಶಿಸಿಹೋಯಿತು. ಆದರೆ ಜ್ಯೋತಿಷ್ಕರ ವಂಶ ಮುಂದುವರಿದು ಈ ಗ್ರಂಥ ಅವರ ಸ್ವತ್ತಾಯಿತು. ಜ್ಯೋತಿಶ್ಶಾಸ್ತ್ರವು ಈ ರೀತಿ ಉದ್ಭವಿಸಿತೆಂದು ಕಂಡುಬರುತ್ತದೆ. ಜ್ಯೋತಿಷ್ಯದ ಪ್ರತಿಯೊಂದು ವಿವರಣೆಗೂ ಅತಿಯಾದ ಗಮನವನ್ನು ಕೊಡುವುದು ಒಂದು ಮೂಢನಂಬಿಕೆ. ಇದು ಹಿಂದೂಗಳಿಗೆ ಬಹಳ ಹಾನಿಯನ್ನುಂಟುಮಾಡಿತು.

ಮೊದಲು ಗ್ರೀಕರು ಜ್ಯೋತಿಷ್​ಶಾಸ್ತ್ರವನ್ನು ಭಾರತಕ್ಕೆ ಒಯ್ದು ಅಲ್ಲಿಂದ ಖಗೋಳಶಾಸ್ತ್ರ\-ವನ್ನು ಪಡೆದರೆಂದು ನನಗನಿಸುತ್ತದೆ. ಅನಂತರ ಈ ಖಗೋಳಶಾಸ್ತ್ರವು ಯೂರೋಪಿನಲ್ಲಿ\break ಹರಡಿತು. ಏಕೆಂದರೆ ಜ್ಯಾಮಿತಿಯ ಪ್ರಕಾರ ಮಾಡಿದ ಹಳೆಯ ವೇದಿಕೆಗಳನ್ನು ನೀವು ಭಾರತದಲ್ಲಿ ಕಾಣಬಹುದು ಮತ್ತು ನಕ್ಷತ್ರಗಳು ಒಂದು ನಿರ್ದಿಷ್ಟ ಸ್ಥಾನದಲ್ಲಿರುವಾಗ ಕೆಲವು ನಿರ್ದಿಷ್ಟ ಕರ್ಮಗಳನ್ನು ಮಾಡಬೇಕೆಂಬ ಪದ್ಧತಿ ಇತ್ತು. ಆದ್ದರಿಂದ ಗ್ರೀಕರು ಜ್ಯೋತಿಷ್​ಶಾಸ್ತ್ರವನ್ನು ಹಿಂದೂಗಳಿಗೆ ನೀಡಿದರು ಮತ್ತು ಹಿಂದೂಗಳು ಖಗೋಳಶಾಸ್ತ್ರವನ್ನು ಅವರಿಗೆ ಕೊಟ್ಟರು ಎಂಬುದು ನನ್ನ ಭಾವನೆ.

\vskip 4pt

ಕೆಲವು ಜ್ಯೋತಿಷ್ಯರು ಅದ್ಭುತವಾಗಿ ಭವಿಷ್ಯ ಹೇಳುವುದನ್ನು ನಾನು ಕಂಡಿರುವೆನು. ಆದರೆ ಅವರು ಕೇವಲ ಗ್ರಹ ನಕ್ಷತ್ರಗಳ ಸ್ಥಾನಗಳ ನೆರವಿನಿಂದ ಇದನ್ನೆಲ್ಲಾ ಹೇಳುತ್ತಾರೆ ಎಂದು ನಾನು ನಂಬಲಾರೆ. ಅನೇಕ ವೇಳೆ ಇದು ಕೇವಲ ಇತರರ ಮನಸ್ಸಿನಲ್ಲಿ ಏನಿದೆ ಎಂಬು\-ದನ್ನು ತಿಳಿಯುವುದಾಗಿದೆ. ಕೆಲವು ವೇಳೆ ಆಶ್ಚರ್ಯಕರವಾದ ಭವಿಷ್ಯಗಳನ್ನೆಲ್ಲಾ ನುಡಿಯು\-ತ್ತಾರೆ. ಆದರೆ ಅನೇಕ ಸಂದರ್ಭಗಳಲ್ಲಿ ಅವೆಲ್ಲ ಬರಿಯ ಸುಳ್ಳು ಎಂದು ಗೊತ್ತಾಗಿದೆ.

\vskip 4pt

ಲಂಡನ್ನಲ್ಲಿ ಒಬ್ಬ ಯುವಕ ನನ್ನ ಹತ್ತಿರ ಬಂದು, ಬರುವ ವರುಷ ತನಗೆ ಏನಾಗುವುದು ಎಂದು ಕೇಳಿದ. ನಾನು ಅವನನ್ನು ಏತಕ್ಕೆ ಈ ಪ್ರಶ್ನೆ ಹಾಕುವೆ ಎಂದು ಕೇಳಿದೆ. “ನಾನು ಹಣವನ್ನೆಲ್ಲಾ ಕಳೆದುಕೊಂಡು ದಟ್ಟದರಿದ್ರನಾಗಿ ಹೋಗಿರುವೆನು” ಎಂದನು. ಹಲವರಿಗೆ ಹಣವೇ ದೇವರು. ಪಾಪ, ದುರ್ಬಲರು ತಮ್ಮಲ್ಲಿರುವುದನ್ನೆಲ್ಲಾ ಕಳೆದುಕೊಂಡ ಮೇಲೆ ಇನ್ನೇನೂ ತೋಚದೆ ಇರುವಾಗ ಹಣಮಾಡುವುದಕ್ಕೆ ಯಾವುದಾದರೂ ಮಾರ್ಗವನ್ನು ಹುಡುಕಾಡುವರು. ಆಗ ಜ್ಯೋತಿಷ್ಯರು ಮುಂತಾದವರ ಹತ್ತಿರ ಬರುವರು. “ಇದು ನನ್ನ ದುರದೃಷ್ಟ ಎನ್ನುವವನು ಹೇಡಿ, ಮೂರ್ಖ” ಎನ್ನುವುದು ನಮ್ಮ ಸಂಸ್ಕೃತ ಗಾದೆಯೊಂದು. ಆದರೆ ಧೀರ “ನನ್ನ ಅದೃಷ್ಟವನ್ನು ನಾನು ನಿಶ್ಚಯಿಸಬಲ್ಲೆ” ಎನ್ನುವನು. ಮುದುಕರಾಗುತ್ತಿರು\-ವವರು ಅದೃಷ್ಟದ ಮಾತನ್ನು ಆಡುವರು. ಯುವಕರು ಸಾಧಾರಣವಾಗಿ ಜ್ಯೋತಿಷ್ಯರ\break ಹತ್ತಿರ ಬರುವುದಿಲ್ಲ. ಗ್ರಹಗಳು ನಮ್ಮ ಮೇಲೆ ಪ್ರಭಾವ ಬೀರಬಹುದು. ಆದರೆ\break ಅದರಿಂದೇನು? ಜ್ಯೋತಿಷ್ಯ ಮುಂತಾದುವನ್ನು ಹೇಳಿ ಉದರ ಪೋಷಣೆ ಮಾಡಿಕೊಳ್ಳುವವರ ಹತ್ತಿರ ಸಂಬಂಧವನ್ನು ಇಟ್ಟುಕೊಳ್ಳಕೂಡದು ಎಂದು ಬುದ್ಧ ಹೇಳಿದ. ಅವನಿಗೆ\break ಇದರ ರಹಸ್ಯ ಚೆನ್ನಾಗಿ ಗೊತ್ತಿರಬೇಕು. ಅವನು ಆರ್ಯರಲ್ಲಿ ಜನಿಸಿದ ಒಬ್ಬ ಪುರುಷ\break ಶ್ರೇಷ್ಠ. ಗ್ರಹಗಳು ತೀಕ್ಷ್ಣದೃಷ್ಟಿಯಿಂದ ನೋಡಿದರೇನಂತೆ? ಗ್ರಹವೊಂದು ನನ್ನ ಜೀವನದ ಮೇಲೆ ತನ್ನ ಪ್ರಭಾವವನ್ನು ಬೀರಿ ವ್ಯಥೆಯನ್ನು ತಂದರೆ ನನ್ನ ಜೀವನ ಕುರುಡುಕಾಸಿಗೂ ಯೋಗ್ಯವಲ್ಲ. ಜ್ಯೋತಿಷ್ಯ ಮುಂತಾದ ರಹಸ್ಯವಿದ್ಯೆಗಳನ್ನೆಲ್ಲಾ ನೆಚ್ಚುವುದು ದೌರ್ಬಲ್ಯದ ಚಿಹ್ನೆ. ಈ ಸ್ವಭಾವ ನಿಮ್ಮ ಮನಸ್ಸಿನಲ್ಲಿ ಬಲವಾಗುತ್ತಿದ್ದರೆ ನೀವು ಒಬ್ಬ ವೈದ್ಯನನ್ನು ನೋಡಿ; ಒಳ್ಳೆಯ ಆಹಾರವನ್ನು ಮತ್ತು ವಿಶ್ರಾಂತಿಯನ್ನು ತೆಗೆದುಕೊಳ್ಳಿ.

\vskip 4pt

ನಿಮಗೆ ಒಂದು ವಸ್ತುವಿನ ಗುಣದಿಂದಲೇ ಒಂದು ಘಟನೆಗೆ ಪ್ರಮಾಣ ಸಿಕ್ಕುವ ಹಾಗಿದ್ದರೆ, ಹೊರಗಿನಿಂದ ಅದಕ್ಕೆ ವಿವರಣೆಯನ್ನು ಹುಡುಕುವುದು ಮೌಢ್ಯ. ಎದುರಿಗೆ ಇರುವುದೇ ವಿವರಿಸುವುದಕ್ಕೆ ಸಾಕಾದರೆ ಅದರ ವಿವರಣೆಗಾಗಿ ಬೇರೆಲ್ಲೋ ಹೋಗುವುದು\break ಮೂರ್ಖತನ. ಯಾವನಾದರೊಬ್ಬನ ಜೀವನದಲ್ಲಿ ಅವನ ಶಕ್ತಿಯಿಂದಲೇ ನಡೆದುದೆಲ್ಲ ಎಂದು ಹೇಳುವಂಥ ಘಟನೆ ಯಾವುದಾದರೂ ನಡೆದಿದೆಯೆ? ಆದಕಾರಣ ಚಂದ್ರ ತಾರಾಬಲಗಳಿಗಾಗಿ ಏತಕ್ಕೆ ಹೋಗಬೇಕು? ನನ್ನ ಈಗಿನ ಸ್ಥಿತಿಗೆ ನನ್ನ ಕರ್ಮವೇ ಸಾಕಷ್ಟು ಕಾರಣವಾಗಿದೆ. ಇದರಂತೆಯೇ ಏಸುವಿನ ಸ್ಥಿತಿಗೂ ಕೂಡ. ಏಸುವಿನ ತಂದೆ ಬರಿಯ ಒಬ್ಬ ಬಡಗಿ ಎಂದು ನಮಗೆ ಗೊತ್ತಿದೆ. ಏಸುವಿನ ಶಕ್ತಿಗೆ ಎಲ್ಲಿಯೋ ಹೊರಗೆ ವಿವರಣೆಯನ್ನು ಹುಡುಕುವುದಕ್ಕೆ ಹೋಗುವುದು ನಿಷ್ಪ್ರಯೋಜಕ. ಏಸು ತನ್ನ ಹಿಂದಿನ ಜೀವನದ ಕರ್ಮಗಳ ಪರಿಪಾಕವಾಗಿ ಆದವನು. ಬುದ್ಧ ತನ್ನ ಹಿಂದಿನ ಜನ್ಮಗಳನ್ನು ವಿವರಿಸುತ್ತ ತಾನು ಪ್ರಾಣಿಯ ಜನ್ಮದಲ್ಲಿ ಇದ್ದುದನ್ನು ಹೇಳುವನು. ನಮ್ಮ ಅದೃಷ್ಟವನ್ನು ವಿವರಿಸಲು ತಾರೆಗಳ ತಂಟೆ ಏತಕ್ಕೆ? ಎಲ್ಲೊ ಕಿಂಚಿತ್​ ಪ್ರಭಾವ ನಮ್ಮ ಮೇಲೆ ಅವಕ್ಕೆ ಇದ್ದರೂ ಇರಬಹುದು. ಆದರೂ ಅವುಗಳಿಗೆ ಹೆದರಿ ನಾವು ಅವುಗಳ ಕಡೆಗೆ ಗಮನ ಕೊಡದೆ ಇರುವುದು ನಮ್ಮ ಕರ್ತವ್ಯ. ನನ್ನ ಬೋಧನೆಯಲ್ಲೆಲ್ಲ ನಾನು ಇದಕ್ಕೆ ಪ್ರಾಮುಖ್ಯವನ್ನು ಕೊಡುತ್ತೇನೆ: ಯಾವುದು\break ನಮ್ಮ ಆಧ್ಯಾತ್ಮಿಕ, ಮಾನಸಿಕ ಮತ್ತು ದೈಹಿಕ ದೌರ್ಬಲ್ಯಕ್ಕೆ ಕಾರಣವೋ ಅದನ್ನು ನಿಮ್ಮ\break ಕಾಲು ಬೆರಳುಗಳಿಂದಲೂ ಮುಟ್ಟಬೇಡಿ. ಮನುಷ್ಯನ ಅಂತರಂಗದಲ್ಲಿ ಸ್ವಾಭಾವಿಕವಾಗಿ ಹುದುಗಿರುವ ಶಕ್ತಿಯನ್ನು ವ್ಯಕ್ತಗೊಳಿಸುವುದೇ ಧರ್ಮ. ಅನಂತ ಶಕ್ತಿಯ ಸ್ಪ್ರಿಂಗು\break ಸುರುಳಿ ಸುತ್ತಿಕೊಂಡು ನಮ್ಮ ಪುಟ್ಟ ದೇಹದಲ್ಲಿರುವುದು. ಅದು ಹರಡುತ್ತದೆ. ಅದು\break ವಿಸ್ತಾರವಾದಂತೆಲ್ಲ ಒಂದಾದ ಮೇಲೆ ಒಂದು ದೇಹವನ್ನು ಅಸಮರ್ಪಕವೆಂದು ತ್ಯಜಿಸುತ್ತಾ\break ಉನ್ನತ ದೇಹವನ್ನು ಪಡೆಯುತ್ತ ಹೋಗುತ್ತದೆ. ಇದೇ ಮಾನವನ ಇತಿಹಾಸ, ಅವನ ಧರ್ಮ, ನಾಗರಿಕತೆ, ಪ್ರಗತಿ ಮತ್ತು ಬಂಧನದಲ್ಲಿ ನರಳುತ್ತಿರುವ ಆ ದೈತ್ಯ ವಿರಾಟ್​ ಶಕ್ತಿ ಕ್ರಮೇಣ ತನ್ನನ್ನು ಬಿಗಿದ ಸರಪಳಿಗಳನ್ನೆಲ್ಲಾ ಆಚೆಗೆ ಕಿತ್ತೊಗೆದು ಮುಕ್ತವಾಗುತ್ತಿರುವುದು. ಇದು ಯಾವಾಗಲೂ ಶಕ್ತಿಯನ್ನು ವ್ಯಕ್ತಗೊಳಿಸುತ್ತದೆ. ಜ್ಯೋತಿಷ್ಯ ಮುಂತಾದುವುಗಳಲ್ಲೆಲ್ಲ ಸ್ವಲ್ಪ ಸತ್ಯಾಂಶವಿದ್ದರೂ ಅವನ್ನು ನಾವು ನಿರ್ಲಕ್ಷ್ಯದಿಂದ ನೋಡಬೇಕು.

\vskip 4pt

ಜ್ಯೋತಿಷಿಯನ್ನು ಕುರಿತ ಒಂದು ಹಳೆಯ ಕಥೆ ಇದೆ. ಅವನು ಒಬ್ಬ ರಾಜನ ಬಳಿಗೆ\break ಬಂದು ‘ನೀನು ಇನ್ನು ಆರು ತಿಂಗಳುಗಳಲ್ಲಿ ಸಾಯುವೆ’ ಎಂದ. ರಾಜನು ಭಯದಿಂದ\break ಕಂಪಿಸತೊಡಗಿದ. ಅಂಜಿಕೆಯಿಂದ ಅವನು ಆಗಲೇ ಸಾಯುವ ಸ್ಥಿತಿಯಲ್ಲಿದ್ದ. ಆದರೆ ಮಂತ್ರಿ ಬಹಳ ಬುದ್ಧಿವಂತ. ಅವನು ಈ ಜ್ಯೋತಿಷಿಗಳನ್ನೆಲ್ಲಾ ನಂಬಕೂಡದು, ಇವರೆಲ್ಲಾ ಮೂರ್ಖರೆಂದು ರಾಜನಿಗೆ ಬುದ್ಧಿ ಹೇಳಿದ. ರಾಜ ಇದನ್ನು ನಂಬಲಿಲ್ಲ. ಮಂತ್ರಿಯು\break ಜ್ಯೋತಿಷಿಯು ಮೂರ್ಖ ಎಂಬುದನ್ನು ತೋರಿಸುವುದಕ್ಕೆ ಜ್ಯೋತಿಷಿಯನ್ನು ಪುನಃ\break ಅರಮನೆಗೆ ಬರಮಾಡಿಕೊಂಡ. ಮಂತ್ರಿಯು ನಿನ್ನ ಲೆಕ್ಕಾಚಾರವೆಲ್ಲ ಸರಿಯಾಗಿದೆಯೇ ನೋಡು ಎಂದ. ಜ್ಯೋತಿಷಿಗೆ ಲವಲೇಶವೂ ಸಂದೇಹವಿರಲಿಲ್ಲ. ಆದರೆ ಮಂತ್ರಿಯ\break ತೃಪ್ತಿಗಾಗಿ ಪುನಃ ಲೆಕ್ಕಾಚಾರ ಹಾಕಿ ಎಲ್ಲ ಸರಿಯಾಗಿದೆ ಎಂದ. ಭಯದಿಂದ ರಾಜನ ಮುಖ ಸಪ್ಪೆಯಾಯಿತು. ಮಂತ್ರಿ ಜ್ಯೊತಿಷಿಯನ್ನು “ನೀನು ಇನ್ನೆಷ್ಟು ದಿನಗಳವರೆಗೆ\break ಬದುಕಿರುವೆ?” ಎಂದ. ಜ್ಯೋತಿಷಿ ಹನ್ನೆರಡು ವರುಷ ಎಂದ. ಮಂತ್ರಿ ತಕ್ಷಣ ತನ್ನ ಸೊಂಟದಲ್ಲಿದ್ದ ಕತ್ತಿಯನ್ನು ಹಿರಿದು ಜ್ಯೋತಿಷಿಯ ತಲೆಯನ್ನು ಕತ್ತರಿಸಿದ. ಅವನು\break ರಾಜನಿಗೆ ಹೀಗೆ ಹೇಳಿದ, “ಈ ಸುಳ್ಳುಗಾರನ ಗತಿ ಏನಾಯಿತು ಗೊತ್ತಿಲ್ಲವೇ? ಹನ್ನೆರಡು ವರುಷ ಬದುಕಿರುತ್ತೇನೆ ಎಂದವನು ಈ ಕ್ಷಣ ಸತ್ತು ಹೋದನು.”

\vskip 4pt

ನಿಮ್ಮ ರಾಷ್ಟ್ರವು ಬಾಳಬೇಕಾದರೆ ಇಂತಹ ವಿಷಯಗಳಿಂದ ದೂರ ಇರಬೇಕು.\break ಏನಾದರೊಂದು ಒಳ್ಳೆಯದು ಎನ್ನಿಸಿಕೊಳ್ಳಬೇಕಾದರೆ ಅದು ನಮ್ಮನ್ನು ಧೀರರನ್ನಾಗಿ ಮಾಡಬೇಕು. ಒಳ್ಳೆಯದೇ ಜೀವನ, ಕೆಟ್ಟದ್ದೇ ಮರಣ. ಈ ಮೂಢಭಾವನೆಗಳು ನಾಯಿಕೊಡೆಗಳಂತೆ ನಿಮ್ಮ ದೇಶದಲ್ಲಿ ಹಬ್ಬುತ್ತಿವೆ. ವಿಚಾರ ಮಾಡದ ಸ್ತ್ರೀಯರು ಇವನ್ನೆಲ್ಲಾ ನಂಬುತ್ತಾ ಹೋಗುವರು. ಸ್ತ್ರೀಯರು ಇನ್ನೂ ಸ್ವಾತಂತ್ರ್ಯಕ್ಕೆ ಹೋರಾಡುತ್ತಿರುವರು. ಅವರ ಬುದ್ಧಿ ಇನ್ನೂ ಚೆನ್ನಾಗಿ ವಿಕಾಸವಾಗಿಲ್ಲ. ಯಾವುದೋ ಕಾದಂಬರಿಯ ಪ್ರಾರಂಭದಲ್ಲಿ ಬ್ರೌನಿಂಗ್​ ಕವಿಯ ಅಚ್ಚಾಗಿರುವ ಉಕ್ತಿಯನ್ನು ಕಂಠಪಾಠಮಾಡಿ ಬ್ರೌನಿಂಗ್​ ಕವಿಯ ಕವಿತೆಗಳೆಲ್ಲಾ ತನಗೆ ಗೊತ್ತಿದೆ ಎಂದು ಭಾವಿಸುವರು. ಮತ್ತೊಬ್ಬಳು ಯಾವುದೋ ವಿಷಯದ ಮೇಲೆ ಮೂರು ಉಪನ್ಯಾಸಗಳನ್ನು ಕೇಳಿಬಿಟ್ಟು ತನಗೆ ಪ್ರಪಂಚದಲ್ಲಿರುವ ವಿಷಯಗಳೆಲ್ಲಾ ಗೊತ್ತಿವೆ ಎಂದು ಭಾವಿಸುವಳು. ಸ್ವಾಭಾವಿಕವಾಗಿ ಸ್ತ್ರೀಯರಲ್ಲಿರುವ ಮೌಢ್ಯವನ್ನು ಅವರು ಕಿತ್ತೊಗೆಯಲಾರದೆ ಇರುವರು. ಇದೇ ಅವರ ತೊಂದರೆ. ಅವರ ಹತ್ತಿರ ಬೇಕಾದಷ್ಟು ಹಣ ಇದೆ; ಬುದ್ಧಿ ಸ್ವಲ್ಪ ಇದೆ. ಅವರು ಇಂತಹ ತಾತ್ಕಾಲಿಕ ಸ್ಥಿತಿಯಿಂದ ಪಾರಾದ ಮೇಲೆ ಸರಿಹೋಗುವರು. ಆದರೆ ಅನೇಕ ಮಿಥ್ಯಾಚಾರಿಗಳು ಅವರನ್ನು ಮೋಸಗೊಳಿಸುತ್ತಿರುವರು. ನೀವು ವಿಷಾದಿಸಬೇಕಾಗಿಲ್ಲ. ನಾನು ಯಾರನ್ನೂ ನೋಯಿಸಲು ಇಚ್ಛೆಪಡುವುದಿಲ್ಲ. ಆದರೆ ಸತ್ಯವನ್ನು ಹೇಳಬೇಕಾಗಿದೆ. ನೀವು ಹೇಗೆ ಇದನ್ನೆಲ್ಲಾ ಒಪ್ಪಿಕೊಳ್ಳುತ್ತಿರುವಿರಿ ಎಂಬುದು ನಿಮಗೆ ಕಾಣಿಸುವುದಿಲ್ಲವೆ? ಈ ಮಹಿಳೆಯರು ಎಷ್ಟು ನಿಷ್ಕಪಟಿಗಳು ಎಂಬುದು ನಿಮಗೆ ಗೊತ್ತು. ಎಲ್ಲರಲ್ಲಿಯೂ ಸುಪ್ತವಾಗಿರುವ ದಿವ್ಯತೆ ಎಂದಿಗೂ ನಾಶವಾಗುವುದಿಲ್ಲ ಎಂಬುದು ನಿಮಗೆ ಗೊತ್ತಾಗುವುದಿಲ್ಲವೆ? ಅವರ ದಿವ್ಯತೆಯನ್ನು ಹೇಗೆ ವ್ಯಕ್ತಗೊಳಿಸಬೇಕೆಂಬುದನ್ನು ಅವರು ಅರಿಯಬೇಕಾಗಿದೆ.

\vskip 4pt

ನಾನು ಹೆಚ್ಚು ಕಾಲ ಬಾಳಿದಷ್ಟೂ ಪ್ರತಿಯೊಬ್ಬರೂ ದಿವ್ಯಾತ್ಮರು ಎಂಬುದು\break ನಿಸ್ಸಂದೇಹವಾಗಿ ಕಾಣಿಸುತ್ತಿದೆ. ಯಾವ ಸ್ತ್ರೀಪುರುಷರಾಗಲೀ ಅವರು ಎಷ್ಟೇ ಅಧಮಾಧಮರಾಗಲಿ ದಿವ್ಯತೆ ಅವರಿಂದ ಎಂದಿಗೂ ಮಾಯವಾಗುವುದಿಲ್ಲ. ಅದನ್ನು ಹೇಗೆ ವ್ಯಕ್ತಗೊಳಿಸ\-ಬೇಕೆಂಬುದು ಅವರಿಗೆ ಗೊತ್ತಿಲ್ಲದೆ ಇರಬಹುದು. ಆದಕಾರಣ ಅವರು ಸತ್ಯಕ್ಕಾಗಿ ಕಾಯಬೇಕಾಗಿದೆ. ದುಷ್ಟರು ಇಂತಹವರನ್ನು ಕೆಲಸಕ್ಕೆ ಬಾರದ ಮೂಢನಂಬಿಕೆಗಳಿಂದ ಅಂಜಿಸುತ್ತಿರುವರು. ಒಬ್ಬ ಹಣಕ್ಕಾಗಿ ಮತ್ತೊಬ್ಬನನ್ನು ಮೋಸ ಮಾಡಿದರೆ ಅವನನ್ನು ಮೋಸಗಾರ ಎನ್ನುವಿರಿ. ಇತರರನ್ನು ಆಧ್ಯಾತ್ಮಿಕವಾಗಿ ಮೋಸಗೊಳಿಸುವವನು ಎಂತಹ ಪಾಪಿ ಇರಬೇಕು? ಇದು ಪರಮ ಪಾತಕ. ಸತ್ಯ ನಿಮ್ಮನ್ನು ಧೀರರನ್ನಾಗಿ ಮಾಡಬೇಕು; ಮೌಢ್ಯದಿಂದ ಪಾರಾಗುವಂತೆ ಮಾಡಬೇಕು. ಇದೇ ಸತ್ಯದ ಪರೀಕ್ಷೆ. ನಿಮ್ಮನ್ನು ಮೌಢ್ಯದಿಂದ ಪಾರಾಗುವಂತೆ ಮಾಡುವುದು ತತ್ತ್ವಜ್ಞಾನಿಯ ಕರ್ತವ್ಯ. ಈ ದೇಹ ಮನಸ್ಸು ಎಲ್ಲವೂ ಮೌಢ್ಯವೇ. ನೀವು ಎಂತಹ ಅನಂತಾತ್ಮರು! ಮಿನುಗುತ್ತಿರುವ ತಾರೆಗಳಿಗೆ ಅಂಜುವುದೇ! ಇದು ನಾಚಿಕೆಗೇಡು. ನೀವು ಪವಿತ್ರಾತ್ಮರು. ಮಿನುಗುವ ತಾರೆಗಳು ಇರುವುದು ನಿಮ್ಮಿಂದ.

\vskip 4pt

ನಾನು ಒಂದು ಸಲ ಹಿಮಾಲಯದಲ್ಲಿ ಹೋಗುತ್ತಿದ್ದೆ. ಮುಂದೆ ನಾವು ನಡೆದು ಹೋಗಬೇಕಾದ ದಾರಿ ಇತ್ತು. ನಾವು ಬಡ ಸಂನ್ಯಾಸಿಗಳಾದುದರಿಂದ ನಮ್ಮನ್ನು ಡೋಲಿಯಲ್ಲಿ ಹೊತ್ತುಕೊಂಡು ಹೋಗುವುದಕ್ಕೆ ಯಾರೂ ಸಿಕ್ಕಲಿಲ್ಲ. ನಾವು ಕಾಲುನಡಿಗೆಯಲ್ಲೇ ಹೋಗಬೇಕಾಯಿತು. ನಮ್ಮ ಜೊತೆಯಲ್ಲಿ ಮುದುಕನೊಬ್ಬನಿದ್ದ. ನೂರಾರು ಮೈಲುಗಳಷ್ಟು ಮೇಲೆ ಹತ್ತಿ ಕೆಳಗೆ ಇಳಿಯಬೇಕು. ಹತ್ತಬೇಕಾದ ಒಂದು ಏರನ್ನು ನೋಡಿದಾಗ ಆ ವೃದ್ಧ ಸಾಧು ‘ಅಬ್ಬ ಇದನ್ನು ಏರುವುದು ಹೇಗೆ? ನಾನಿನ್ನು ನಡೆಯಲಾರೆ, ಎದೆ ಒಡೆದುಹೋಗುವುದು’ ಎಂದ. ನಾನು ಅವನಿಗೆ ನಿನ್ನ ಹಿಂದೆ ನೋಡು ಎಂದೆ. ಅವನು ಹಿಂತಿರುಗಿ ನೋಡಿದ. ಆಗ ನಾನು ಎಂದೆ: “ಯಾವ ದಾರಿಯನ್ನು ನೀನು ನೋಡುತ್ತಿರುವೆಯೋ ಅದು ಈಗಾಗಲೇ ನೀನು ನಡೆದದ್ದು, ಅದೇ ದಾರಿ ಈಗ ನಿನ್ನ ಮುಂದೆ ಇರುವುದು. ಇನ್ನು ಸ್ವಲ್ಪ ಹೊತ್ತಿನಲ್ಲಿ\break ಅದು ನಿನ್ನ ಹಿಂದೆ ಹೋಗುವುದು.” ಶ್ರೇಷ್ಠತಮ ವಸ್ತುಗಳು ನಿಮ್ಮ ಕಾಲ ಕೆಳಗೆ ಇವೆ.\break ಏಕೆಂದರೆ ನೀವೇ ದಿವ್ಯ ತಾರೆಗಳು. ಬೇಕಾದರೆ ತಾರೆಗಳನ್ನು ನಿಮ್ಮ ಬೊಗಸೆಯಿಂದ ಎತ್ತಿ ನೀವು ನುಂಗಿ ಹಾಕಬಹುದು. ನಿಮ್ಮ ನೈಜಸ್ವಭಾವ ಅಂತಹದು. ಧೀರರಾಗಿ, ಎಲ್ಲಾ ವಿಧದ ಮೂಢನಂಬಿಕೆಗಳಿಂದ ಪಾರಾಗಿ, ಮುಕ್ತರಾಗಿ.

