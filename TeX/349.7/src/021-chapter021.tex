
\chapter[ರಾಮಾಯಣ ]{ರಾಮಾಯಣ \protect\footnote{\engfoot{C.W. Vol. IV, P. 63}}}

\begin{flushright}
\textbf{(೧೯೦೦ರ ಜನವರಿ ೩೧ರಂದು ಕ್ಯಾಲಿಫೋರ್ನಿಯಾದ ಪಸದೆನದಲ್ಲಿ ನೀಡಿದ ಉಪನ್ಯಾಸ)}
\end{flushright}

ಸಂಸ್ಕೃತ ಸಾಹಿತ್ಯದಲ್ಲಿ ಬಹಳ ಪುರಾತನವಾದ ಎರಡು ಮಹಾಕಾವ್ಯಗಳಿವೆ. ಇವಲ್ಲದೆ ಇನ್ನೂ ಬೇರೆ ನೂರಾರು ಮಹಾ ಕವಿತೆಗಳಿವೆ. ಕಳೆದ ಎರಡು ಸಾವಿರ ವರುಷಗಳಿಂದಲೂ ಸಂಸ್ಕೃತಭಾಷೆ ಮಾತನಾಡುವ ಭಾಷೆ ಅಲ್ಲದಿದ್ದರೂ ಆ ಭಾಷೆ ಮತ್ತು ಅದರ ಸಾಹಿತ್ಯ ಇಂದಿನವರೆಗೂ ಉಳಿದುಕೊಂಡು ಬಂದಿವೆ. ನಾನೀಗ ನಿಮಗೆ ಬಹಳ ಪುರಾತನವಾದ ರಾಮಾಯಣ ಮತ್ತು ಮಹಾಭಾರತಗಳೆಂಬ ಎರಡು ಕಾವ್ಯಗಳ ವಿಚಾರವಾಗಿ ಹೇಳುತ್ತೇನೆ. ಇವುಗಳಲ್ಲಿ ಪುರಾತನ ಭಾರತೀಯರ ಆಚಾರ ವ್ಯವಹಾರ, ಸಾಮಾಜಿಕ ಸ್ಥಿತಿ ಮತ್ತು ಸಂಸ್ಕೃತಿ ಮುಂತಾದ ವಿಷಯಗಳಿವೆ. ಇವೆರಡರಲ್ಲಿ ಬಹಳ ಪುರಾತನವಾದುದನ್ನು ರಾಮಾಯಣ ಅಥವಾ ರಾಮಚರಿತೆ ಎಂದು ಕರೆಯುವೆವು. ಇವಕ್ಕೆ ಮುಂಚೆ ಎಷ್ಟೋ ಕಾವ್ಯ ಸಾಹಿತ್ಯವಿತ್ತು. ಹಿಂದೂಗಳ ಪವಿತ್ರ ಗ್ರಂಥವಾದ ವೇದಗಳ ಬಹುಭಾಗವೆಲ್ಲ ಒಂದು ರೀತಿಯಲ್ಲಿ ಛಂದೋಬದ್ಧವಾಗಿವೆ. ಆದರೆ ರಾಮಾಯಣವನ್ನು ಎಲ್ಲರೂ ಒಮ್ಮತದಿಂದ ಆದಿಕಾವ್ಯ ಎನ್ನುವರು.

ಈ ಆದಿಕಾವ್ಯವನ್ನು ಬರೆದ ಕವಿಯ ಅಥವಾ ಋಷಿಯ ಹೆಸರು ವಾಲ್ಮೀಕಿ. ಕಾಲಾನಂತರ ಈ ಆದಿಕವಿಯೇ ಇನ್ನೂ ಅನೇಕ ಕಾವ್ಯಗಳನ್ನು ಬರೆದಿರಬೇಕು. ಏಕೆಂದರೆ ಹಲವು ಕಾವ್ಯಗಳು ಆತನ ಹೆಸರಿನಲ್ಲಿ ಬಳಕೆಯಲ್ಲಿವೆ. ಕಾಲಕ್ರಮೇಣ ಈತನದಲ್ಲದ ಹಲವು ಕಾವ್ಯ ಭಾಗಗಳನ್ನು ಕೂಡ ಈತನೇ ಬರೆದ ಎನ್ನುವುದು ರೂಢಿಯಾಗಿ ಹೋಯಿತು. ಈ ಪ್ರಕ್ಷಿಪ್ತ ಭಾಗಗಳನ್ನು ಗಮನಕ್ಕೆ ತಂದುಕೊಳ್ಳದೇ ಇದ್ದರೂ, ಪ್ರಪಂಚದ ಸಾಹಿತ್ಯದಲ್ಲಿ ಎಣೆಯಿಲ್ಲದಷ್ಟು ಸುಂದರವಾದ ರಾಮಾಯಣ ಮಹಾಕಾವ್ಯವು ರಚಿತವಾಗಿ ನಮ್ಮ ಭಾಗಕ್ಕೆ ದೊರಕಿದೆ.

ಒಬ್ಬ ಯುವಕನಿದ್ದ. ಯೋಗ್ಯರೀತಿಯಿಂದ ಅವನಿಗೆ ತನ್ನ ಸಂಸಾರವನ್ನು ನಿರ್ವಹಿಸುವುದಕ್ಕೆ ಆಗುತ್ತಿರಲಿಲ್ಲ. ಅವನು ಒಳ್ಳೆಯ ದೃಢಕಾಯದವನಾಗಿದ್ದ. ಕೊನೆಗೆ ಅವನು ಒಬ್ಬ ದರೋಡೆಕಾರನಾದ. ದಾರಿಯಲ್ಲಿ ಹೋಗುವವರ ಮೇಲೆ ಬಿದ್ದು ಅವರಲ್ಲಿದ್ದುದನ್ನೆಲ್ಲ\break ಅಪಹರಿಸಿ, ಹಾಗೆ ಬಂದ ಹಣದಿಂದ ತನ್ನ ತಾಯಿ ತಂದೆ ಹೆಂಡತಿ ಮಕ್ಕಳನ್ನು ಸಾಕುತ್ತಿದ್ದ. ಇದು ಹೀಗೇ ಅವಿಚ್ಛಿನ್ನವಾಗಿ ನಡೆಯುತ್ತಿದ್ದಾಗ ಒಂದು ದಿನ ನಾರದರೆಂಬ ಮಹರ್ಷಿಗಳು ಆ ರಸ್ತೆಯಲ್ಲಿ ಹೋಗುತ್ತಿದ್ದರು. ಈ ದರೋಡೆಕಾರ ಅವರ ಮೇಲೆ ಬಿದ್ದ. ನಾರದರು ಅವನನ್ನು “ನೀನು ಏತಕ್ಕೆ ನನ್ನಲ್ಲಿರುವುದನ್ನು ಅಪಹರಿಸುತ್ತೀಯೆ? ದರೋಡೆಮಾಡಿ ಜನರನ್ನು ಕೊಲ್ಲುವುದು ಮಹಾಪಾಪ. ಯಾರಿಗಾಗಿ ನೀನು ಈ ಪಾಪ ಕೃತ್ಯಗಳನ್ನೆಲ್ಲ ಮಾಡುತ್ತಿರುವೆ?” ಎಂದು ಕೇಳಿದರು. ದರೋಡೆಕಾರ “ಈ ಹಣದಿಂದ ನನ್ನ ಸಂಸಾರವನ್ನು ನಿರ್ವಹಿಸಬೇಕಾಗಿದೆ” ಎಂದ. “ಹಾಗಾದರೆ ನಿನ್ನ ಸಂಸಾರದವರು ನಿನ್ನ ಪಾಪದಲ್ಲಿ ಒಂದು ಪಾಲನ್ನು ತೆಗೆದುಕೊಳ್ಳುತ್ತಾರೆ ಎಂದು ಭಾವಿಸಿದ್ದೀಯಾ?” ಎಂದು ಕೇಳಿದರು. “ಹೌದು, ನಿಜವಾಗಿಯೂ” ಎಂದು ಅವನು ಉತ್ತರ ಕೊಟ್ಟ. ಆಗ ಋಷಿಗಳು, “ಒಳ್ಳೆಯದು, ಎಲ್ಲಿಯೂ ಹೋಗದಂತೆ ನನ್ನನ್ನು ಇಲ್ಲಿ ಕಟ್ಟಿಬಿಟ್ಟು ಹೋಗು. ಮನೆಯಲ್ಲಿ ವಿಚಾರಿಸಿ ಬಾ, ಅವರು ನಿನ್ನ ಹಣವನ್ನು ಹಂಚಿಕೊಳ್ಳುವಂತೆ ನಿನ್ನ ಪಾಪವನ್ನೂ ಹಂಚಿಕೊಳ್ಳುವರೇನೋ” ಎಂದು ಹೇಳಿದರು. ದರೋಡೆಕಾರ ಅದರಂತೆಯೇ ಮನೆಗೆ ಹೋಗಿ ತಂದೆಯನ್ನು “ಅಪ್ಪ, ನಿನಗೆ ಗೊತ್ತೆ, ನಾನು ಹೇಗೆ ನಿಮ್ಮನ್ನೆಲ್ಲಾ ಕಾಪಾಡುತ್ತಿರುವೆನು?” ಎಂದು ಕೇಳಿದ. ತಂದೆ “ನನಗೆ ಗೊತ್ತಿಲ್ಲ” ಎಂದ. ಅವನು “ನಾನೊಬ್ಬ ದರೋಡೆಕಾರ. ಜನರನ್ನು ಕೊಂದು ಅವರನ್ನು ಸುಲಿಗೆ ಮಾಡುತ್ತೇನೆ” ಎಂದ. “ಏನು! ಮಗು, ನೀನು ಅಂತಹ ಕೆಲಸ ಮಾಡುತ್ತಿರುವೆಯಾ? ತೊಲಗಾಚೆ, ಚಂಡಾಲ” ಎಂದನು. ಆನಂತರ ದರೋಡೆಕಾರ ತಾಯಿಯ ಹತ್ತಿರ ಹೋಗಿ “ಅಮ್ಮ, ನಾನು ನಿಮ್ಮನ್ನು ಹೇಗೆ ಕಾಪಾಡುತ್ತಿರುವೆನು ಎಂಬುದು ಗೊತ್ತೆ?” ಎಂದು ಕೇಳಿದ. “ಇಲ್ಲ” ಎಂದಳು ಅವಳು. ಈತ ‘ಕೊಲೆ ಮತ್ತು ಸುಲಿಗೆಯಿಂದ’ ಎಂದ. “ಎಷ್ಟು ಭಯಂಕರ ಇದು!” ಎಂದು ಆಕೆ ಕಿರುಚಿಕೊಂಡಳು. ಮಗ, “ನೀನು ನನ್ನ ಪಾಪದ ಪಾಲನ್ನು ತೆಗೆದುಕೊಳ್ಳುವೆಯಾ” ಎಂದು ಕೇಳಿದ. “ನಾನೇಕೆ ತೆಗೆದುಕೊಳ್ಳಬೇಕು? ನಾನೆಂದೂ ದರೋಡೆ ಮಾಡಿಲ್ಲ” ಎಂದಳು. ಆನಂತರ ಅವನು ಹೆಂಡತಿಯ ಬಳಿಗೆ ಹೋಗಿ “ನಾನು ನಿಮ್ಮನ್ನೆಲ್ಲ ಹೇಗೆ ಕಾಪಾಡುತ್ತಿರುವೆನೆಂದು ಗೊತ್ತೆ?” ಎಂದು ಕೇಳಿದ, ‘ಗೊತ್ತಿಲ್ಲ’ ಎಂದಳು. “ನಾನೊಬ್ಬ ದರೋಡೆಕಾರ. ಹಲವು ವರುಷಗಳಿಂದ ಜನರನ್ನು ಸುಲಿಗೆ ಮಾಡಿ ನಾನು ನಿಮ್ಮನ್ನೆಲ್ಲಾ ಕಾಪಾಡುತ್ತಿರುವೆನು. ನಾನೀಗ ತಿಳಿಯಬೇಕೆಂದು ಇರುವುದು ನೀನು ನನ್ನ ಪಾಪದಲ್ಲಿ ಭಾಗಿಯಾಗುತ್ತೀಯೊ ಎಂಬುದನ್ನು” ಎಂದ. “ಎಂದಿಗೂ ಇಲ್ಲ. ನೀನು ನನ್ನ ಗಂಡ. ನನ್ನನ್ನು ನೋಡಿಕೊಳ್ಳುವುದು ನಿನ್ನ ಕರ್ತವ್ಯ” ಎಂದಳು ಅವಳು.

ದರೋಡೆಕಾರನ ಕಣ್ಣು ತೆರೆಯಿತು: “ಇದೇ ಪ್ರಪಂಚದ ಜನರ ಸ್ವಭಾವ. ನಾನು\break ಯಾವ ಹತ್ತಿರದ ನೆಂಟರಿಗೋಸ್ಕರ ದರೋಡೆ ಮಾಡುತ್ತಿರುವೆನೊ ಅವರು ಕೂಡ ನನ್ನ ಅದೃಷ್ಟದಲ್ಲಿ ಭಾಗಿಗಳಾಗುವುದಿಲ್ಲ.” ಋಷಿಗಳನ್ನು ಕಟ್ಟಿಹಾಕಿದ ಸ್ಥಳಕ್ಕೆ ಅವನು ಹಿಂದಿರುಗಿ ಬಂದು ಅವರ ಕಟ್ಟನ್ನು ಸಡಿಲಿಸಿ ಅವರ ಕಾಲಿಗೆ ಬಿದ್ದು ಮನೆಯಲ್ಲಿ ನಡೆದ ಸಂಗತಿಯನ್ನೆಲ್ಲ ತಿಳಿಸಿ “ನೀವು ನನ್ನನ್ನು ರಕ್ಷಿಸಬೇಕು. ನಾನು ಈಗ ಏನು ಮಾಡಬೇಕು?” ಎಂದು ಬೇಡಿಕೊಂಡನು. ಆಗ ಋಷಿಗಳು ಹೇಳಿದರು: “ಈಗ ನೀನು ಮಾಡುತ್ತಿರುವ ವೃತ್ತಿಯನ್ನು ತೊರೆ. ನಿನ್ನ ಬಂಧುಗಳು ಯಾರೂ ನಿನ್ನನ್ನು ಪ್ರೀತಿಸುವುದಿಲ್ಲ ಎನ್ನುವುದು ಗೊತ್ತಾಯಿತು. ಅವರು ನಿನ್ನನ್ನು ಪ್ರೀತಿಸುವರು ಎಂಬ ಭ್ರಾಂತಿಯನ್ನು ತೊರೆ. ನೀನು ಹಣಗಾರನಾಗಿದ್ದಾಗ ಅವರು ಅದರ ಪಾಲಿಗೆ ಬರುವರು. ಆದರೆ ನಿನ್ನಲ್ಲಿ ಎಂದು ಏನೂ ಇರುವುದಿಲ್ಲವೊ ಆಗ ಅವರೆಲ್ಲ ನಿನ್ನನ್ನು ತಿರಸ್ಕರಿಸುವರು. ನಿನ್ನ ಪಾಪಕ್ಕೆ ಭಾಗಿಗಳಾಗುವವರು ಯಾರೂ ಇಲ್ಲ. ಎಲ್ಲರೂ ನಿನ್ನ ಪುಣ್ಯಕ್ಕೆ ಮಾತ್ರ ಭಾಗಿಗಳಾಗುತ್ತಾರೆ. ನಾವು ಪಾಪಮಾಡಲಿ, ಪುಣ್ಯಮಾಡಲಿ,\break ಯಾವಾಗಲೂ ನಮ್ಮೊಡನೆ ಇರುವ ಪರಮಾತ್ಮನನ್ನು ಮಾತ್ರ ಪೂಜಿಸಬೇಕು. ಅವನು ನಮ್ಮನ್ನು ಎಂದಿಗೂ ತ್ಯಜಿಸುವುದಿಲ್ಲ. ಏಕೆಂದರೆ ಪ್ರೇಮ ಎಂದಿಗೂ ಕೆಳಗ್ಗೆ ಎಳೆಯುವುದಿಲ್ಲ. ಪ್ರೇಮಕ್ಕೆ ವ್ಯಾಪಾರ-ಬುದ್ಧಿ ಇಲ್ಲ. ಅದಕ್ಕೆ ಸ್ವಾರ್ಥ ಇಲ್ಲ.”

ಭಗವಂತನನ್ನು ಹೇಗೆ ಉಪಾಸನೆ ಮಾಡಬೇಕೆಂಬುದನ್ನು ಋಷಿಗಳು ಅವನಿಗೆ\break ಬೋಧಿಸಿದರು. ಡಕಾಯಿತ ಸರ್ವಸಂಗ ಪರಿತ್ಯಾಗವನ್ನು ಮಾಡಿ ಕಾಡಿಗೆ ಹೋದನು. ಅಲ್ಲಿ ಪ್ರಾರ್ಥನೆಯಲ್ಲಿ ಮತ್ತು ಧ್ಯಾನದಲ್ಲಿ ತನ್ಮಯನಾಗತೊಡಗಿದನು. ಅದರಲ್ಲೇ ಎಷ್ಟು ಮಟ್ಟಿಗೆ ತನ್ಮಯನಾದನೆಂದರೆ ಗೆದ್ದಲು ಬಂದು ಅವನ ಸುತ್ತಲೂ ಹುತ್ತವನ್ನು ಕಟ್ಟಿದವು. ಆದರೂ ಅವನಿಗೆ ಬಾಹ್ಯಪ್ರಜ್ಞೆಯೇ ಇರಲಿಲ್ಲ. ಹಲವು ವರ್ಷಗಳಾದ ಮೇಲೆ\break ಧ್ವನಿಯೊಂದು “ಏಳು ಮಹಾಋಷಿ” ಎಂದು ಎಚ್ಚರಿಸಿತು. ಆಗ ಎಚ್ಚೆತ್ತವನು “ಮಹಾಋಷಿ! ಇಲ್ಲ. ನಾನು ಡಕಾಯಿತ” ಎಂದು ಉತ್ತರಕೊಟ್ಟ. ಆ ಧ್ವನಿ ಹೀಗೆಂದಿತು- “ನೀನಿನ್ನು ಡಕಾಯಿತನಲ್ಲ. ಪರಿಶುದ್ಧನಾದ ಮಹಾಋಷಿ. ನಿನ್ನ ಹಳೆಯ ಹೆಸರು ಹೋಯಿತು. ನಿನ್ನ ಸುತ್ತಲೂ ಬೆಳೆದ ಹುತ್ತದ ಪ್ರಜ್ಞೆಕೂಡ ನಿನಗೆ ಇಲ್ಲದಷ್ಟು ಮನಸ್ಸು ಏಕಾಗ್ರವಾಗಿದ್ದುದರಿಂದ ಇನ್ನು ಮೇಲೆ ನಿನ್ನ ಹೆಸರು ವಾಲ್ಮೀಕಿ, ಎಂದರೆ ಹುತ್ತದಲ್ಲಿ ಹುಟ್ಟಿದವನು ಎಂದು.” ಹೀಗೆ ಅವನು ಋಷಿಯಾದ.

ಅವನು ಕವಿಯಾದದ್ದು ಹೀಗೆ. ಒಂದು ದಿನ ವಾಲ್ಮೀಕಿ ಮಹರ್ಷಿಗಳು ಗಂಗಾನದಿಗೆ ಸ್ನಾನಕ್ಕೆ ಹೋಗುತ್ತಿದ್ದಾಗ ಒಂದು ಜೊತೆ ಪಾರಿವಾಳಗಳು ಸುತ್ತಲೂ ಹಾರಾಡಿ ಮುದ್ದಾಡುತ್ತಿದ್ದುವು. ಋಷಿ ತಲೆ ಎತ್ತಿ ನೋಡಿ ಸಂತುಷ್ಟನಾದ. ಆದರೆ ಮರುಕ್ಷಣದಲ್ಲಿಯೇ ಒಂದು ಬಾಣ ಅವನ ಹಿಂದುಗಡೆಯಿಂದ ಹೋಗಿ ಗಂಡು ಪಾರಿವಾಳವನ್ನು ಕೊಂದಿತು. ಅದು\break ಕೆಳಗೆ ಬೀಳುತ್ತಲೇ ಹೆಣ್ಣು ಹಕ್ಕಿ ದುಃಖದಿಂದ ತನ್ನ ಜೊತೆಗಾರನಾದ ಗಂಡುಹಕ್ಕಿಯ ಶವದ ಸುತ್ತಲೂ ಹಾರಾಡುತ್ತಿತ್ತು. ಋಷಿ ತಕ್ಷಣ ವ್ಯಥಿತನಾಗಿ ಹಿಂತಿರುಗಿ ನೋಡಿದಾಗ ಬೇಟೆಗಾರನೊಬ್ಬನಿದ್ದ. “ನೀನೊಬ್ಬ ನೀಚ. ನಿನ್ನಲ್ಲಿ ಸ್ವಲ್ಪವೂ ದಯೆಯಿಲ್ಲ. ಪ್ರೇಮದೆದುರಿಗೂ ನಿನ್ನ ಕೊಲೆಪಾತಕತನ ಕುಂಠಿತವಾಗಲಿಲ್ಲ” ಎಂದನು. ಕವಿ ತನಗೆ ತಾನೆ ತಕ್ಷಣ “ಇದೇನು ನಾನು ಹೀಗೆ ಹೇಳುತ್ತಿರುವೆ? ನಾನು ಹೀಗೆ ಎಂದೂ ಮಾತನಾಡಿರಲಿಲ್ಲ” ಎಂದುಕೊಂಡ. ಆಗೊಂದು ಧ್ವನಿ ಕೇಳಿಸಿತು: “ನೀನು ಅಂಜದಿರು. ನಿನ್ನ ಬಾಯಿಂದ ಬರುತ್ತಿರುವುದು ಕಾವ್ಯ. ಜಗತ್ತಿನ ಹಿತಕ್ಕಾಗಿ ರಾಮಚರಿತೆಯನ್ನು ಕಾವ್ಯರೂಪದಲ್ಲಿ ಬರೆ” ಎಂದಿತು. ಕಾವ್ಯದ ಆದಿ ಇದು. ಆದಿಕವಿ ವಾಲ್ಮೀಕಿಯ ಬಾಯಿಯಿಂದ ಮೊದಲ ಶ್ಲೋಕ ಶೋಕದಿಂದ ಉದಿಸಿತು. ಆನಂತರವೆ ಅವನು ಅತಿ ಸುಂದರವಾದ ರಾಮಾಯಣವನ್ನು ಬರೆದನು.

ಹಿಂದೆ ಭರತಖಂಡದಲ್ಲಿ ಅಯೋಧ್ಯೆ ಎಂಬ ಊರು ಇತ್ತು. ಅದು ಈಗಲೂ ಇದೆ. ಅದು ಇರುವ ಪ್ರಾಂತವನ್ನು ಔದ್​ ಎನ್ನುತ್ತಾರೆ. ನಿಮ್ಮಲ್ಲಿ ಹಲವರು ಅದನ್ನು ಇಂಡಿಯಾ ಭೂಪಟದ ಮೇಲೆ ನೋಡಿರಬಹುದು. ಇದೇ ಹಿಂದಿನ ಅಯೋಧ್ಯೆ. ಹಿಂದಿನ ಕಾಲದಲ್ಲಿ ದಶರಥನೆಂಬ ರಾಜ ಅದನ್ನು ಆಳುತ್ತಿದ್ದ. ಅವನಿಗೆ ಮೂವರು ರಾಣಿಯರಿದ್ದರು. ಆದರೆ ಯಾರಿಗೂ ಮಕ್ಕಳಿರಲಿಲ್ಲ. ಮಕ್ಕಳಿಗಾಗಿ ಎಲ್ಲಾ ಹಿಂದೂಗಳಂತೆ ಅವರು ತೀರ್ಥಯಾತ್ರೆ ವ್ರತ ಉಪವಾಸಾದಿಗಳನ್ನು ಮಾಡತೊಡಗಿದರು. ಕೆಲವು ಕಾಲದ ಮೇಲೆ ಅವರಿಗೆ ನಾಲ್ಕುಜನ ಮಕ್ಕಳಾದರು. ಅವರಲ್ಲಿ ಹಿರಿಯವನೆ ರಾಮ.

ನಾಲ್ಕು ಜನ ಸಹೋದರರು, ನಿಯಮದಂತೆ, ಎಲ್ಲಾ ಶಾಸ್ತ್ರಗಳಲ್ಲಿಯೂ ಪರಿಣತರಾದರು. ಮುಂದೆ ಜಗಳಬಾರದಂತೆ ನೋಡಿಕೊಳ್ಳುವುದಕ್ಕೆ ಹಿಂದಿನ ಭರತ ಖಂಡದಲ್ಲಿ ಒಂದು ರೂಢಿಯಿತ್ತು. ರಾಜ ತಾನು ಬದುಕಿರುವಾಗಲೆ ತನ್ನ ಜ್ಯೇಷ್ಠ ಪುತ್ರನನ್ನು ಯುವರಾಜನನ್ನಾಗಿ ನಾಮಕರಣ ಮಾಡುತ್ತಿದ್ದನು. ಅವನೇ ಅನಂತರ ರಾಜನಾಗುವನು.

ಜನಕನೆಂಬ ಮತ್ತೊಬ್ಬ ರಾಜನಿದ್ದ. ಅವನಿಗೆ ಸೀತೆಯೆಂಬ ಅತಿ ಚೆಲುವೆಯಾದ\break ಮಗಳಿದ್ದಳು. ಸೀತೆ ಒಂದು ಹೊಲದಲ್ಲಿ ಸಿಕ್ಕಿದವಳು. ಭೂದೇವಿಯ ಮಗಳು. ಅವಳಿಗೆ ತಂದೆ ತಾಯಿಗಳಾರೂ ಇರಲಿಲ್ಲ. ಸಂಸ್ಕೃತದಲ್ಲಿ ಸೀತಾ ಎಂದರೆ ನೇಗಿಲಿನಿಂದ ಆದ ಗೆರೆ ಎಂದು ಅರ್ಥ. ಭರತಖಂಡದ ಪುರಾಣದಲ್ಲಿ ತಂದೆಗೊ ತಾಯಿಗೊ ಒಬ್ಬರಿಗೇ ಆದ ಮಕ್ಕಳು, ತಂದೆತಾಯಿಗಳಲ್ಲಿ ಜನಿಸಿದ ಮಕ್ಕಳು, ಯಾಗದಿಂದ ಜನಿಸಿದವರು,\break ಭೂಮಿಯಲ್ಲಿ ಜನಿಸಿದವರು, ಎಲ್ಲೋ ಆಕಾಶದಿಂದ ಬಿದ್ದಂತೆ ಇದ್ದವರು - ಇಂತಹ\break ಹಲವು ಬಗೆಯ ಜನನಗಳಿವೆ. ಈ ಬಗೆಯ ವಿಚಿತ್ರ ಜನನಗಳೆಲ್ಲ ಪುರಾಣ ಪ್ರಪಂಚದಲ್ಲಿ ಸಾಧಾರಣವಾಗಿವೆ.

ಸೀತೆ ಪೃಥ್ವಿಯ ಮಗಳಾದ್ದರಿಂದ ಪರಿಶುದ್ಧಳಾಗಿದ್ದಳು. ಅವಳಿಗೆ ತಂದೆ ತಾಯಿಗಳಾರೂ ಇರಲಿಲ್ಲ. ಜನಕರಾಜ ಅವಳನ್ನು ಸಾಕಿದನು. ಅವಳಿಗೆ ಪ್ರಾಪ್ತ ವಯಸ್ಸಾದಾಗ ಮದುವೆ ಮಾಡಿಕೊಡಲು ಯೋಗ್ಯ ವರನನ್ನು ಹುಡುಕತೊಡಗಿದನು.

ಹಿಂದೆ ಭರತಖಂಡದಲ್ಲಿ ಸ್ವಯಂವರ ಎಂಬುದು ರೂಢಿಯಲ್ಲಿತ್ತು. ಅದರಲ್ಲಿ\break ರಾಜಕುಮಾರಿ ವರನನ್ನು ತಾನೇ ಆರಿಸಿಕೊಳ್ಳುತ್ತಿದ್ದಳು. ದೇಶದ ಅನೇಕ ಕಡೆಗಳ ಹಲವು\break ರಾಜಕುಮಾರರಿಗೆ ಆಮಂತ್ರಣಗಳನ್ನು ಕಳುಹಿಸಿಕೊಡುತ್ತಿದ್ದರು. ರಾಜಕುಮಾರಿ ಅಲಂಕೃತಳಾಗಿ ಕೈಯಲ್ಲಿ ಒಂದು ಹೂಮಾಲೆಯನ್ನು ಹಿಡಿದುಕೊಂಡು ರಾಜಕುಮಾರರ ನಡುವೆ ಹೋಗುತ್ತಿದ್ದಳು. ಅವಳ ಜೊತೆಯಲ್ಲಿ ಆಯಾಯ ರಾಜಕುಮಾರರ ವಿಶೇಷ ಗುಣಗಳನ್ನು ಸಾರುತ್ತಿರುವ ಭಟರು ಇರುತ್ತಿದ್ದರು. ಅವರಲ್ಲಿ ಯಾರು ತನಗೆ ಮೆಚ್ಚುಗೆಯೋ ಅವನ ಕೊರಳಿಗೆ ಮಾಲೆಯನ್ನು ಹಾಕುವಳು. ಅನಂತರ ಅತಿ ವೈಭವದಿಂದ ಅವರಿಬ್ಬರ ಮದುವೆ ನಡೆಯುತ್ತಿತ್ತು.

ಅನೇಕ ರಾಜಕುಮಾರರು ಸೀತೆಯನ್ನು ಮದುವೆ ಮಾಡಿಕೊಳ್ಳಬೇಕೆಂದು ಇದ್ದರು. ಆದರೆ ಇದಕ್ಕೆ ಒಂದು ಪರೀಕ್ಷೆ ಇತ್ತು. ಯಾರು ಹರಧನುಸ್ಸನ್ನು ಭಂಜಿಸುವರೋ ಅವರು ಮದುವೆಗೆ ಯೋಗ್ಯ ವರರಾಗುತ್ತಿದ್ದರು. ಇದಕ್ಕಾಗಿ ರಾಜಕುಮಾರರು ತಮ್ಮ ಶಕ್ತಿಯನ್ನೆಲ್ಲಾ ಪ್ರಯೋಗಿಸಿ ಯತ್ನಿಸಿದರು. ಆದರೆ ಎಲ್ಲರೂ ಸೋತು ಹೋದರು. ಕೊನೆಗೆ ರಾಮ ಆ ಅದ್ಭುತ ಧನುಸ್ಸನ್ನು ತನ್ನ ಕೈಯಲ್ಲಿ ಹಿಡಿದುಕೊಂಡು ಲೀಲಾಜಾಲವಾಗಿ ಅದನ್ನು ಎರಡು ತುಂಡು ಮಾಡಿದನು. ದಶರಥ ರಾಜಕುಮಾರನಾದ ರಾಮನನ್ನು ಸೀತೆ\break ವರಿಸಿದ್ದು ಹೀಗೆ. ಅನಂತರ ಅತಿ ವೈಭವದಿಂದ ಅವರ ಲಗ್ನವಾಯಿತು. ಬಳಿಕ ರಾಮ\break ಸೀತೆಯನ್ನು ಊರಿಗೆ ಕರೆದುಕೊಂಡು ಹೋದ. ದಶರಥ ತನಗೆ ಇನ್ನು ವಾನಪ್ರಸ್ಥಕ್ಕೆ\break ಕಾಲವಾಗಿದೆ ಎಂದು ಭಾವಿಸಿ ರಾಮನನ್ನು ಯುವರಾಜನನ್ನಾಗಿ ಮಾಡಲು ನಿಶ್ಚಯಿಸಿದ.\break ಪಟ್ಟಾಭಿಷೇಕಕ್ಕೆ ಎಲ್ಲಾ ಅಣಿಯಾಯಿತು. ಇಡೀ ದೇಶವೇ ಆನಂದೋತ್ಸಾಹದಲ್ಲಿ\break ಮುಳುಗಿತು. ಆಗ ಕಿರಿಯ ರಾಣಿ ಕೈಕೇಯಿಗೆ ಅವಳ ಪರಿಚಾರಿಕೆಯೊಬ್ಬಳು ದಶರಥ ಮೆಚ್ಚುಗೆಯನ್ನು ಸಂಪಾದಿಸಿದ್ದಳು. ಆಗ ರಾಜ “ನಿನಗೆ ಬೇಕಾದ ನನ್ನ ವಶದಲ್ಲಿರುವ\break ಎರಡು ವರಗಳನ್ನು ಕೇಳು, ಕೊಡುತ್ತೇನೆ” ಎಂದಿದ್ದನು. ಆದರೆ ಕೈಕೇಯಿ ಆಗ ಏನನ್ನೂ ಕೇಳಿಕೊಂಡಿರಲಿಲ್ಲ. ಇದನ್ನೆಲ್ಲಾ ಮರೆತುಬಿಟ್ಟಿದ್ದಳು. ಆ ದುಷ್ಟ ಸೇವಕಿ ರಾಮ ಯುವರಾಜನಾದರೆ ಕೈಕೇಯಿಗೆ ಕಷ್ಟವಾಗುವುದೆಂದು ಅವಳ ಅಸೂಯೆಯನ್ನು ಕೆರಳಿಸಿದಳು. ಅವಳ ಮಗ ಭರತನೇ ರಾಜನಾದರೆ ಎಷ್ಟು ಚೆನ್ನಾಗಿರುವುದು ಎಂಬುದನ್ನು ಸೂಚಿಸಿ ಅಸೂಯೆಯಿಂದ ಅವಳು ಉನ್ಮತ್ತಳಾಗುವಂತೆ ಹೇಳಿದಳು. ಅವುಗಳಲ್ಲಿ ಒಂದನೆಯದು ಭರತ ರಾಜನಾಗುವುದು, ಎರಡನೆಯದು ರಾಮ ಹದಿನಾಲ್ಕು ವರುಷ ವನವಾಸ ಮಾಡಬೇಕು ಎನ್ನುವುದು.

ರಾಮನೇ ಆ ವೃದ್ಧ ದೊರೆಯ ಪ್ರಾಣವಾಗಿದ್ದ, ಅವನ ಆತ್ಮನಾಗಿದ್ದ. ಈ ದುಷ್ಟ ಕೋರಿಕೆಯನ್ನು ಕೈಕೇಯಿಯು ರಾಜನ ಮುಂದೆ ಇಟ್ಟಾಗ ತಾನು ರಾಜನಾದುದರಿಂದ ಹಿಂದೆ ಆಡಿದ ಮಾತಿಗೆ ವಿರೋಧವಾಗಿ ಹೋಗಲು ಆಗಲಿಲ್ಲ. ಆದರೆ ಏನು ಮಾಡಬೇಕೊ ಅದೂ ತೋಚಲಿಲ್ಲ. ಆಗ ರಾಮನು ಅವನ ಸಹಾಯಕ್ಕೆ ಬಂದ. ತಂದೆ ಅಸತ್ಯಕ್ಕೆ ಗುರಿಯಾಗದಿರಲಿ ಎಂದು ಸ್ವಂತ ಇಚ್ಛೆಯಿಂದ ಸಿಂಹಾಸನವನ್ನು ತ್ಯಜಿಸಿ ವನವಾಸಕ್ಕೆ ಹೋಗಲು ಒಪ್ಪಿಕೊಂಡ. ತನ್ನ ಪ್ರಿಯತಮೆ ಸತಿಯಾದ ಸೀತೆ ಮತ್ತು ಎಂದಿಗೂ ಅವನಿಂದ ಅಗಲುವುದಕ್ಕೆ ಒಪ್ಪಿಕೊಳ್ಳದ ಲಕ್ಷ್ಮಣ ಇವರೊಂದಿಗೆ ಹದಿನಾಲ್ಕು ವರುಷ ವನವಾಸಕ್ಕೆ ಹೊರಟ.

ದಂಡಕಾರಣ್ಯದಲ್ಲಿ ಆಗ ಯಾರು ವಾವಸಾಗಿದ್ದರೆಂಬುದು ಆರ್ಯರಿಗೆ ಗೊತ್ತಿರಲಿಲ್ಲ.\break ಆಗಿನ ಕಾಲದಲ್ಲಿ ಅರಣ್ಯದಲ್ಲಿ ವಾಸಿಸುತ್ತಿದ್ದವರನ್ನು ವಾನರರೆಂದು ಕರೆಯುತ್ತಿದ್ದರು.\break ಅವರಲ್ಲಿ ಕೆಲವರು ಬಹಳ ಪೌರುಷವಂತರಾಗಿ ಬಲಾಢ್ಯರಾಗಿದ್ದರೆ ಅಂತಹವರನ್ನು\break ರಾಕ್ಷಸರೆಂದು ಕರೆಯುತ್ತಿದ್ದರು.

ವಾನರರು ಮತ್ತು ರಾಕ್ಷಸರು ವಾಸಿಸುತ್ತಿದ್ದ ದಂಡಕಾರಣ್ಯಕ್ಕೆ ಸೀತೆ ರಾಮಲಕ್ಷ್ಮಣರು ಹೊರಟರು. ಸೀತೆ ರಾಮನೊಂದಿಗೆ ಬರುತ್ತೇನೆ ಎಂದಾಗ ಅವನು “ರಾಜಕುಮಾರಿಯಾದ ನೀನು ಅನಿರೀಕ್ಷಿತ ಬಾಹುಳ್ಯದಿಂದ ಕೂಡಿದ ಕಾಡಿಗೆ ನನ್ನೊಡನೆ ಬಂದು ಕಷ್ಟಪಡುತ್ತೀಯೆ” ಎಂದ. ಆಗ ಸೀತೆ ಹೀಗೆಂದಳು: “ರಾಮ ಎಲ್ಲಿಗೆ ಹೋದರೆ ಸೀತೆ ಅಲ್ಲಿಗೆ ಹೋಗುತ್ತಾಳೆ. ನೀನು ನನ್ನನ್ನು ರಾಜಕುಮಾರಿ, ರಾಜವಂಶಸ್ಥರು ಎಂದು ಏಕೆ ಹೇಳುತ್ತಿರುವೆ? ನಾನು\break ನಿನ್ನ ಮುಂದೆ ಹೋಗುತ್ತೇನೆ.” ಸೀತೆ ರಾಮನನ್ನು ಅನುಸರಿಸಿದಳು. ಅವನ ತಮ್ಮನೂ\break ಅವನೊಡನೆ ಹೊರಟನು. ಗೋದಾವರಿ ನದಿ ಸಿಕ್ಕುವ ಪರ್ಯಂತ ಕಾಡಿನಲ್ಲೇ ಮುಂದೆ ಮುಂದೆ ಹೋದರು. ಆ ನದೀ ತೀರದಲ್ಲಿ ಒಂದು ಪರ್ಣಶಾಲೆಯನ್ನು ಕಟ್ಟಿದರು.\break ರಾಮ ಲಕ್ಷ್ಮಣರು ಜಿಂಕೆ ಮೊದಲಾದವನ್ನು ಬೇಟೆಯಾಡಿ ಮತ್ತು ಹಣ್ಣನ್ನು ಆರಿಸಿಕೊಂಡು ಬರುತ್ತಿದ್ದರು. ಕೆಲವು ದಿನ ಅವರು ಹೀಗೆ ಕಳೆದ ಮೇಲೆ ಅಲ್ಲಿಗೆ ಒಬ್ಬಳು ರಾಕ್ಷಸಿ\break ಬಂದಳು. ಅವಳು ಲಂಕೇಶನಾದ ರಾವಣನ ತಂಗಿ. ಕಾಡಿನಲ್ಲಿ ಸಂಚಾರ ಮಾಡುತ್ತಿದ್ದಾಗ ರಾಮ ಅವಳಿಗೆ ಗೋಚರಿಸಿದ. ಅವನ ಸುಂದರಾಕಾರವನ್ನು ನೋಡಿ ಅವಳು ಅವನನ್ನು ಮೋಹಿಸಿದಳು. ಆದರೆ ರಾಮ ಪರಿಶುದ್ಧಾತ್ಮ. ಅದೂ ಅಲ್ಲದೆ ಅವನಿಗೆ ಲಗ್ನ ಬೇರೆ ಆಗಿತ್ತು. ಆದಕಾರಣ ಅವಳನ್ನು ಒಪ್ಪಲಿಲ್ಲ. ಅವಳು ಅವನ ಮೇಲೆ ಸೇಡನ್ನು ತೀರಿಸಿಕೊಳ್ಳುವುದಕ್ಕಾಗಿ ತನ್ನ ಅಣ್ಣ ರಾವಣಾಸುರನ ಬಳಿಗೆ ಹೋಗಿ ರಾಮನ ಸತಿಯಾದ ಸೀತೆಯ ಲಾವಣ್ಯವನ್ನು ವಿವರಿಸತೊಡಗಿದಳು.

ರಾಮ ಅತ್ಯಂತ ಪರಾಕ್ರಮಶಾಲಿಯಾಗಿದ್ದ. ಯಾವ ಮನುಷ್ಯನಿಗಾಗಲಿ ರಾಕ್ಷಸನಿಗಾಗಲಿ ಅವನನ್ನು ಜಯಿಸುವುದಕ್ಕೆ ಸಾಧ್ಯವಿರಲಿಲ್ಲ. ಅದಕ್ಕಾಗಿ ರಾವಣಾಸುರ ಉಪಾಯಕ್ಕೆ ಕೈಹಾಕಿದ. ಮಂತ್ರವಾದಿಯಾದ ಮತ್ತೊಬ್ಬ ಅಸುರನ ಸಹಾಯವನ್ನು ಪಡೆದು ಅವನು ಸುಂದರವಾದ ಸ್ವರ್ಣಮೃಗದ ರೂಪವನ್ನು ತಾಳುವಂತೆ ಮಾಡಿದ. ಸೀತೆ ಇರುವ ಕಡೆ ಜಿಂಕೆ ಓಡಾಡತೊಡಗಿತು. ಸೀತೆ ಆ ಮೃಗದ ಸೌಂದರ್ಯಕ್ಕೆ ಮನಸೋತು ಅದನ್ನು ತಂದುಕೊಡಬೇಕೆಂದು ರಾಮನನ್ನು ಕೇಳಿಕೊಂಡಳು. ರಾಮ ಲಕ್ಷ್ಮಣನನ್ನು ಸೀತೆಯ ಕಾವಲಿಗಿಟ್ಟು ಆ ಮಾಯಾಮೃಗವನ್ನು ಹಿಡಿಯುವುದಕ್ಕೆ ಕಾಡಿನ ಒಳಗೆ ಹೊರಟ. ಲಕ್ಷ್ಮಣ ಪರ್ಣಶಾಲೆಯ ಸುತ್ತಲೂ ಒಂದು ಅಗ್ನಿರೇಖೆ ಎಳೆದು ಸೀತೆಗೆ “ಇಂದು ನಿನಗೆ ಏನೋ ವಿಪತ್ತು ಸಂಭವಿಸಬಹುದು, ಈ ಗೆರೆಯನ್ನು ದಾಟಿ ಹೊರಗೆ ಕಾಲಿಡಬೇಡ” ಎಂದ. ಅಷ್ಟು ಹೊತ್ತಿಗೆ ಮಾಯಾಮೃಗವನ್ನು ರಾಮ ಬಾಣದಿಂದ ಹೊಡೆದ. ತಕ್ಷಣ ಆ ಮೃಗ ತನ್ನ ಹಿಂದಿನ ರಾಕ್ಷಸ ರೂಪವನ್ನು ತಾಳಿ ಕಾಲವಾಯಿತು.

ಆಗ ಪರ್ಣಶಾಲೆಯ ಹತ್ತಿರ “ಓ ಲಕ್ಷ್ಮಣಾ, ನನ್ನ ಸಹಾಯಕ್ಕೆ ಬಾ” ಎಂಬ ರಾಮನ ಧ್ವನಿ ಕೇಳಿಸಿತು. ಸೀತೆ “ಲಕ್ಷ್ಮಣಾ, ರಾಮನ ಸಹಾಯಕ್ಕೆ ತಕ್ಷಣ ಹೋಗು” ಎಂದಳು. “ಅದು\break ರಾಮನ ದನಿಯಲ್ಲ” ಎಂದು ಲಕ್ಷ್ಮಣ ಪ್ರತಿಭಟಿಸಿದ. ಆದರೆ ಸೀತೆ ಲಕ್ಷ್ಮಣನನ್ನು\break ಕಾಡಿಬೇಡಿದುದರಿಂದ ರಾಮನ ಸಹಾಯಕ್ಕೆ ಹೋಗಬೇಕಾಯಿತು. ಲಕ್ಷ್ಮಣನು\break ಹೋದೊಡನೆಯೆ ರಾವಣಾಸುರ ಸಂನ್ಯಾಸಿಯ ರೂಪವನ್ನು ತಾಳಿ ಬಂದು ಪರ್ಣ\break ಶಾಲೆಯ ಬಾಗಿಲಲ್ಲಿ ನಿಂತು ಭಿಕ್ಷೆ ಬೇಡಿದ. “ನನ್ನ ಗಂಡ ಬರುವವರೆಗೆ ಸ್ವಲ್ಪ ತಾಳು.\break ಆಮೇಲೆ ಬೇಕಾದಷ್ಟು ಭಿಕ್ಷೆ ನೀಡುತ್ತೇನೆ” ಎಂದಳು ಸೀತೆ. ರಾವಣ “ನಾನು\break ಅಲ್ಲಿಯವರೆಗೆ ತಾಳಲಾರೆ. ತಾಯಿ, ತುಂಬಾ ಹಸಿವು. ಏನಿದೆಯೊ ಅದನ್ನು\break ಕೊಡು” ಎಂದ. ಆಗ ಸೀತೆ ಪರ್ಣಶಾಲೆಯಲ್ಲಿದ್ದ ಕೆಲವು ಹಣ್ಣುಗಳನ್ನು ತಂದಳು.\break ಸಂನ್ಯಾಸಿ “ನನಗೆ ಅದನ್ನು ಹೊರಗೆ ತಂದುಕೊಡಿ. ನಾನು ಯತಿ. ನಿಮಗೆ ಯಾವ ಅಂಜಿಕೆಯೂ ಇಲ್ಲ ನನ್ನಿಂದ” ಎಂದು ಮರುಳುಗೊಳಿಸಿದ. ಸೀತೆ ಮಂತ್ರದ ಗೆರೆಯನ್ನು ದಾಟಿ ಬಂದಳು. ತಕ್ಷಣ ಕಪಟ ಸಂನ್ಯಾಸಿ ರಾಕ್ಷಸನ ರೂಪವನ್ನು ತಾಳಿ ಸೀತೆಯನ್ನು\break ತನ್ನ ಬಾಹುಗಳಿಂದ ಸೆಳೆದು ಮಂತ್ರರಥವನ್ನು ಕರೆದು ಅದರಲ್ಲಿ ಅವಳನ್ನು ಕುಳ್ಳಿರಿಸಿ\break ಕೊಂಡು ಅಳುತ್ತಿರುವ ಸೀತೆಯೊಡನೆ ಪಲಾಯನ ಮಾಡಿದ. ರಾಕ್ಷಸ ಅವಳನ್ನು ಹೊತ್ತುಕೊಂಡು ಹೋಗುತ್ತಿರುವಾಗ ಕೆಲವು ಆಭರಣಗಳನ್ನು ಕೈಯಿಂದ ಕೆಳಗೆ ಎಸೆದಳು.

ರಾವಣಾಸುರ ಲಂಕಾದ್ವೀಪದಲ್ಲಿರುವ ತನ್ನ ರಾಜಧಾನಿಗೆ ಸೀತೆಯನ್ನು ಕರೆದುಕೊಂಡು ಹೋದ; ತನ್ನ ರಾಣಿಯಾಗಬೇಕೆಂದು ಅವಳನ್ನು ಕೇಳಿಕೊಂಡನು; ತನ್ನ ಕೋರಿಕೆಗೆ\break ಒಡಂಬಡುವಂತೆ ಹಲವು ಆಸೆಗಳನ್ನು ಅವಳಿಗೆ ತೋರಿದ. ಆದರೆ ಸೀತೆ ಪಾತಿವ್ರತ್ಯವೇ ರೂಪುವೆತ್ತಂತೆ ಇದ್ದವಳು, ರಾವಣಾಸುರನೊಡನೆ ಮಾತನ್ನು ಕೂಡ ಆಡಲಿಲ್ಲ. ಅವಳನ್ನು ಶಿಕ್ಷಿಸುವುದಕ್ಕಾಗಿ, ತನ್ನ ಹೆಂಡತಿಯಾಗುವುದಕ್ಕೆ ಒಪ್ಪಿಕೊಳ್ಳುವವರೆಗೆ ಅವಳನ್ನು ಒಂದು ವೃಕ್ಷದ ಅಡಿಯಲ್ಲಿ ಸೆರೆ ಇಟ್ಟನು.

ರಾಮಲಕ್ಷ್ಮಣರು ಪರ್ಣಶಾಲೆಗೆ ಹಿಂತಿರುಗಿ ಬಂದಾಗ ಸೀತೆ ಮಾಯವಾಗಿರುವುದನ್ನು ನೋಡಿದರು. ಆಗ ಅವರ ದುಃಖಕ್ಕೆ ಮೇರೆಯಿಲ್ಲವಾಯಿತು. ಅವಳಿಗೆ ಏನಾಗಿದೆಯೊ ಅದನ್ನು ಊಹಿಸುವುದಕ್ಕೂ ಅವರಿಗೆ ಸಾಧ್ಯವಾಗಲಿಲ್ಲ. ಇಬ್ಬರು ಸಹೋದರರೂ ಸೀತೆಯನ್ನು ಎಲ್ಲೆಲ್ಲಿ ಹುಡುಕಾಡಿದರೂ ಎಲ್ಲಿಯೂ ಅವಳ ಸುಳಿವೇ ಸಿಕ್ಕಲಿಲ್ಲ. ಕೆಲವು ದಿನ ಹುಡುಕಾಡಿದ ಮೇಲೆ ಕೆಲವು ವಾನರರ ಸಮೀಪಕ್ಕೆ ಬಂದರು. ಅಲ್ಲಿ ಅವರೊಡನೆ ಇದ್ದ ಹನುಮಂತನನ್ನು ಕಂಡರು. ವಾನರಶ್ರೇಷ್ಠನಾದ ಹನುಮಂತ ಶ‍್ರೀರಾಮನ ಶ್ರೇಷ್ಠ ಭೃತ್ಯನಾದ. ನಾವು ಮುಂದೆ ನೋಡುವಂತೆ ಅವನು ಸೀತೆಯನ್ನು ಹುಡುಕುವುದಕ್ಕೆ ರಾಮನಿಗೆ ಸಹಾಯ ಮಾಡಿದ. ರಾಮನ ಮೇಲೆ ಹನುಮಂತನಿಗೆ ಇದ್ದ ಭಕ್ತಿ ಅಪಾರವಾಗಿತ್ತು. ಈಗಲೂ ಕೂಡ ಹಿಂದೂಗಳು ಅವನನ್ನು ಭಗವಂತನ ಶ್ರೇಷ್ಠ ಭೃತ್ಯನೆಂದು ಪೂಜಿಸುವರು. ವಾನರರು, ರಾಕ್ಷಸರು ಎಂದರೆ ದಕ್ಷಿಣ ಭಾರತದ ಆದಿನಿವಾಸಿಗಳು.

ಕೊನೆಗೆ ರಾಮ ವಾನರರ ಜೊತೆ ಸಖ್ಯವನ್ನು ಬೆಳೆಸಿದನು. ಅವರು ಆಕಾಶದಲ್ಲಿ\break ಒಂದು ವಿಮಾನ ಸಂಚರಿಸುತ್ತಿದ್ದುದನ್ನು ತಾವು ನೋಡಿದೆವೆಂದೂ ಅದರಲ್ಲಿ ಒಬ್ಬ ರಾಕ್ಷಸ ಅತಿ ಸುಂದರಳಾದ ಸ್ತ್ರೀಯನ್ನು ಹೊತ್ತುಕೊಂಡು ಹೋಗುತ್ತಿದ್ದನೆಂದೂ ಆ ಸ್ತ್ರೀ ತುಂಬಾ\break ಅಳುತ್ತಿದ್ದಳೆಂದೂ ಹೇಳಿದರು. ವಿಮಾನ ತಮ್ಮ ತಲೆಯ ಮೇಲೆ ಹೋಗುತ್ತಿದ್ದಾಗ, ತಮ್ಮ\break ಲಕ್ಷ್ಯವನ್ನು ಸೆಳೆಯುವುದಕ್ಕಾಗಿ ಆಕೆ ತನ್ನ ಆಭರಣಗಳನ್ನು ಎಸೆದಳು ಎಂದರು. ಆಭರಣಗಳನ್ನು ರಾಮನಿಗೆ ತೋರಿಸಿದರು. ಲಕ್ಷ್ಮಣ ಆಭರಣಗಳನ್ನು ಕೈಗೆ ತೆಗೆದುಕೊಂಡು ಇವು ಯಾರವೋ ತನಗೆ ಗೊತ್ತಿಲ್ಲ ಎಂದನು. ರಾಮ ಅವನ್ನು ತೆಗೆದುಕೊಂಡು ನೋಡಿ\break “ಹೌದು, ಇವು ಸೀತೆಯ ಆಭರಣಗಳು” ಎಂದನು. ಲಕ್ಷ್ಮಣನಿಗೆ ಆಭರಣಗಳನ್ನು ಕಂಡು ಹಿಡಿಯಲು ಆಗಲಿಲ್ಲ. ಏಕೆಂದರೆ ಇಂಡಿಯಾ ದೇಶದಲ್ಲಿ ಅಣ್ಣನ ಹೆಂಡತಿಯನ್ನು ಬಹು ಗೌರವದಿಂದ ಕಾಣುವರು; ಅವಳ ಕೈಗಳನ್ನು ಮತ್ತು ಕತ್ತನ್ನು ತಲೆ ಎತ್ತಿಕೂಡ ನೋಡುವುದಿಲ್ಲ. ಅದಕ್ಕೇ ಲಕ್ಷ್ಮಣನಿಗೆ ಸೀತೆಯ ಕೊರಳಹಾರ ಗೊತ್ತಾಗದೆ ಇದ್ದುದು. ಈ ಘಟನೆಯಲ್ಲಿ ಪುರಾತನ ಭರತಖಂಡದ ಒಂದು ಆಚಾರವಿದೆ. ಈ ರಾಕ್ಷಸನಾರು, ಅವನೆಲ್ಲಿರುವನು ಎಂಬುದನ್ನೆಲ್ಲ ಕಪಿಗಳು ರಾಮನಿಗೆ ಹೇಳಿದುವು. ಅನಂತರ ಅವು ಅವನನ್ನು ಹುಡುಕಲು ಹೊರಟವು.

ರಾಜ್ಯಕ್ಕಾಗಿ ಕಪೀಶನಾದ ವಾಲಿ ಮತ್ತು ಅವನ ತಮ್ಮ ಸುಗ್ರೀವ ತಮ್ಮ ತಮ್ಮಲ್ಲೇ\break ಜಗಳವಾಡಿಕೊಳ್ಳುತ್ತಿದ್ದರು. ರಾಮನು ತಮ್ಮನಾದ ಸುಗ್ರೀವನಿಗೆ ಸಹಾಯ ಮಾಡಿದನು. ಇವನನ್ನು ಓಡಿಸಿದ ವಾಲಿಯಿಂದ ಆತ ರಾಜ್ಯವನ್ನು ಹಿಂತಿರುಗಿ ಪಡೆದನು. ಇದಕ್ಕೆ\break ಪ್ರತಿಯಾಗಿ ರಾಮನಿಗೆ ಸಹಾಯ ಮಾಡುವೆನೆಂದು ಸುಗ್ರೀವ ಒಪ್ಪಿಕೊಂಡ. ದೇಶವನ್ನೆಲ್ಲ ಹುಡುಕಿದರು. ಆದರೆ ಎಲ್ಲಿಯೂ ಸೀತೆ ಸಿಕ್ಕಲಿಲ್ಲ. ಕೊನೆಗೆ ಹನುಮಂತ ಇಂಡಿಯಾದೇಶ ಸಮುದ್ರತೀರದಿಂದ ಒಂದೇ ನೆಗೆತದಲ್ಲಿ ಲಂಕಾದ್ವೀಪಕ್ಕೆ ಹಾರಿದ. ಲಂಕೆಯಲ್ಲೆಲ್ಲಾ\break ಸೀತೆಯನ್ನು ಹುಡುಕಾಡಿದ. ಎಲ್ಲಿಯೂ ಸಿಕ್ಕಲಿಲ್ಲ.

ಪರಾಕ್ರಮಶಾಲಿಯಾದ ರಾವಣಾಸುರ ದೇವತೆಗಳನ್ನು ಗೆದ್ದಿದ್ದ. ಇಡೀ ಪ್ರಪಂಚವನ್ನೇ ಗೆದ್ದಿದ್ದ. ಅಲ್ಲಿರುವ ಸುಂದರ ಯುವತಿಯರನ್ನೆಲ್ಲಾ ಅಪಹರಿಸಿ ತನ್ನ ಅಂತಃಪುರಕ್ಕೆ ಸೇರಿಸಿಕೊಂಡಿದ್ದ. ಹನುಮಂತ ಆಲೋಚಿಸಿದ: “ಸೀತೆ ಉಳಿದ ಹೆಂಗಸರೊಂದಿಗೆ ಅರಮನೆಯಲ್ಲಿ ಇರಲಾರಳು. ಇಂತಹ ಸ್ಥಳದಲ್ಲಿ ಇರುವ ಬದಲು ಅವಳು ಪ್ರಾಣವನ್ನಾದರೂ ಕಳೆದುಕೊಂಡಾಳು.” ಹನುಮಂತ ಬೇರೆಕಡೆ ಅವಳನ್ನು ಹುಡುಕತೊಡಗಿದ. ಕೊನೆಗೆ ದಿಗಂತದ ಅಂಚಿನಲ್ಲಿರುವ ಶುಕ್ಲಪಕ್ಷದ ಚಂದ್ರನಂತೆ ಮ್ಲಾನಳಾಗಿ ಬಾಡಿಹೋದ ಸೀತೆಯನ್ನು ಹನುಮಂತ ಒಂದು ಮರದ ಕೆಳಗೆ ನೋಡಿದ. ಹನುಮಂತ ಸಣ್ಣ ಕಪಿಯಂತೆ ಆ ಮರದ ಮೇಲೆ ಕುಳಿತುಕೊಂಡ. ಸೀತೆಯು ತನ್ನ ಕಡೆ ಒಲಿಯುವಂತೆ ಮಾಡಲು ರಾವಣನಿಂದ ಕಳುಹಿಸಲ್ಪಟ್ಟ ರಾಕ್ಷಸಿಯರು ಸೀತೆಯನ್ನು ಅಂಜಿಸುತ್ತಿರುವುದನ್ನು ನೋಡಿದ. ಆದರೆ ಸೀತೆ ಆ ರಾಕ್ಷಸೇಶನ ಹೆಸರನ್ನು ಕೂಡ ಕೇಳಲು ಇಚ್ಛೆಪಡಲಿಲ್ಲ.

ಅನಂತರ ಹನುಮಂತ ಸೀತೆಯ ಸಮೀಪಕ್ಕೆ ಬಂದು ಸೀತೆಯನ್ನು ಹುಡುಕುವುದಕ್ಕೆ ರಾಮ ಕಳುಹಿಸಿದ ಚಾರ ತಾನು ಎಂಬುದನ್ನು ಹೇಳಿದ. ತನ್ನ ಗುರುತನ್ನು ತೋರುವುದಕ್ಕೆ ರಾಮ ಕೊಟ್ಟ ಉಂಗುರವನ್ನೇ ಸೀತೆಗೆ ಕೊಟ್ಟ. ರಾಮನಿಗೆ ಸೀತೆಯ ಸಮಾಚಾರ ತಿಳಿದೊಡನೆಯೆ ಸೈನ್ಯ ಸಮೇತ ಬಂದು ರಾವಣನನ್ನು ಸೋಲಿಸಿ ಸೀತೆಯನ್ನು ಕರೆದುಕೊಂಡು ಹೋಗುವನು ಎಂಬುದನ್ನೂ ಹೇಳಿದ. ಆದರೂ, ಸೀತೆ ತಾನು ಇಚ್ಛೆಪಟ್ಟರೆ, ತನ್ನ ಹೆಗಲ ಮೇಲೆ ಒಂದೇ ನೆಗೆತದಲ್ಲಿ ಸಾಗರವನ್ನು ಹಾರಿ ರಾಮನನ್ನು ಬೇಕಾದರೆ ಸೇರಬಹುದು ಎಂದ. ಸೀತೆ ಪತಿವ್ರತಾ ಸ್ವರೂಪಳಾದುದರಿಂದ ಗಂಡನಲ್ಲದ ಪುರುಷನನ್ನು ಸ್ಪರ್ಶ ಮಾಡುವುದಿಲ್ಲವೆಂದು ಹೇಳಿ ಹನುಮಂತನ ಸಲಹೆಯನ್ನು ತಿರಸ್ಕರಿಸಿದಳು. ಅವಳು ಅಲ್ಲಿಯೇ ಉಳಿದುಕೊಂಡಳು. ತನ್ನ ಕೇಶದಲ್ಲಿದ್ದ ಒಂದು ಆಭರಣವನ್ನು ರಾಮನಿಗೆ ಕೊಟ್ಟು ಕಳುಹಿಸಿದಳು. ಇದರೊಡನೆ ಹನುಮಂತ, ಹಿಂತಿರುಗಿದ.

ಹನುಮಂತನಿಂದ ಸೀತೆಯ ಸಮಾಚಾರವನ್ನೆಲ್ಲ ತಿಳಿದುಕೊಂಡ ರಾಮನು ವಾನರರೊಡನೆ ಭರತಖಂಡದ ದಕ್ಷಿಣ ತುದಿಗೆ ಹೊರಟನು. ಅಲ್ಲಿ ರಾಮನ ವಾನರಸೇನೆ ‘ಸೇತುಬಂಧ’ ಎಂಬ ಕಡೆ ಲಂಕಾನಗರಿಗೂ ಇಂಡಿಯಾ ದೇಶಕ್ಕೂ ಮಧ್ಯೆ ದೊಡ್ಡದೊಂದು ಸೇತುವೆಯನ್ನು ಕಟ್ಟಿತು. ಈಗಲೂ ಸಮುದ್ರದ ಇಳಿತದ ಸಮಯದಲ್ಲಿ ಆ ಮರಳ ದಿಣ್ಣೆಗಳ\break ಮೇಲಿನಿಂದಲೇ ಲಂಕಾಪಟ್ಟಣಕ್ಕೆ ಹೋಗಬಹುದು.

ರಾಮ ಒಬ್ಬ ಭಗವಂತನ ಅವತಾರ. ಇಲ್ಲದೆ ಇದ್ದರೆ ಅವನಿಗೆ ಇವನ್ನೆಲ್ಲ ಮಾಡಲು ಹೇಗೆ ಸಾಧ್ಯ? ಹಿಂದೂಗಳಿಗೆ ರಾಮ ಒಬ್ಬ ಭಗವಂತನ ಅವತಾರವಾಗಿರುವನು. ರಾಮ\break ಭಗವಂತನ ಏಳನೆಯ ಅವತಾರವೆಂದು ಭರತಖಂಡದಲ್ಲಿ ಜನರು ಅವನನ್ನು ಪರಿಗಣಿಸುತ್ತಾರೆ.

ವಾನರರು ದೊಡ್ಡ ದೊಡ್ಡ ಬೆಟ್ಟಗಳನ್ನು ಸಮುದ್ರದಲ್ಲಿಟ್ಟು, ಮರಳಿನಿಂದ ಮತ್ತು ಮರಗಳಿಂದ ಅವುಗಳನ್ನು ಮುಚ್ಚಿ ಒಂದು ದೊಡ್ಡ ಸೇತುವೆಯನ್ನು ಕಟ್ಟಿದರು. ಅಲ್ಲೊಂದು ಸಣ್ಣ ಅಳಿಲು ಮರಳ ಮೇಲೆ ಹೊರಳಾಡಿ ಸೇತುವೆಯ ಹತ್ತಿರ ದೇಹವನ್ನು ಕೊಡಹಿ ಮರಳನ್ನು ಉದುರಿಸುತ್ತಿತ್ತಂತೆ. ಅಳಿಲು ತನ್ನ ಕೈಲಾದ ಮಟ್ಟಿಗೆ ರಾಮನ ಸೇತುಬಂಧನಕ್ಕೆ ಸಹಾಯ ಮಾಡುತ್ತಿತ್ತು. ಹೊರಳಾಡಿ ಕೆಲವು ಮರಳು ಕಣಗಳನ್ನು ಕೆಡವುತ್ತಿರುವ ಅಳಿಲನ್ನು ನೋಡಿ ವಾನರರು ನಕ್ಕರು. ಏಕೆಂದರೆ ಅವರು ದೊಡ್ಡ ದೊಡ್ಡ ಪರ್ವತಗಳನ್ನೂ ಮರಳುದಿಣ್ಣೆಗಳನ್ನೂ ತಂದು ಹಾಕುತ್ತಿದ್ದರು. ಆದರೆ ರಾಮ ಇದನ್ನು ನೋಡಿ “ಧನ್ಯ ಅಳಿಲು, ತನ್ನ ಕೈಲಾದಷ್ಟು ಅದು ಮಾಡುತ್ತಿದೆ. ಆದಕಾರಣ ಅದು ನಿಮ್ಮಲ್ಲಿರುವ ಪರಮ ಪೌರುಷವಂತನಿಗೆ ಸಮ” ಎಂದು ಅಳಿಲಿನ ಬೆನ್ನ ಮೇಲೆ ಪ್ರೀತಿಯಿಂದ ತನ್ನ ಬೆರಳನ್ನು ಸವರಿದನು. ಈಗಲೂ ಅಳಿಲಿನ ಮೇಲೆ ಇರುವ ರಾಮನ ಬೆರಳುಗಳ ಗುರುತನ್ನು ನೋಡಬಹದು.

ಸೇತುಬಂಧವಾದ ಮೇಲೆ ರಾಮಲಕ್ಷ್ಮಣರ ನೇತೃತ್ವದಲ್ಲಿ ವಾನರ ಸೇನೆ ಲಂಕಾ\break ನಗರವನ್ನು ಪ್ರವೇಶಿಸಿತು. ಹಲವು ತಿಂಗಳು ಘೋರ ಯುದ್ಧ ಮತ್ತು ರಕ್ತಪಾತವಾಯಿತು. ಕೊನೆಗೆ ರಾವಣಾಸುರ ಸೋತನು. ಅವನ ಸಂಹಾರವಾಯಿತು. ಚಿನ್ನದಿಂದ ಮಾಡಿದ ಅವನ ಅರಮನೆ ಮುಂತಾದವನ್ನೆಲ್ಲ ವಶಪಡಿಸಿಕೊಂಡರು. ಇಂಡಿಯಾ ದೇಶದ ಒಳನಾಡಿನಲ್ಲಿ ಬಹಳ ದೂರದಲ್ಲಿರುವ ಹಳ್ಳಿಯ ಜನರಿಗೆ ನಾನು ಲಂಕೆಗೆ ಹೋಗಿದ್ದೆ ಎಂದು ಹೇಳಿದರೆ ಅವರು “ಪುರಾಣದಲ್ಲಿ ಅಲ್ಲಿಯ ಮನೆಗಳೆಲ್ಲ ಚಿನ್ನದಿಂದ ಆಗಿವೆ ಎಂದು ಹೇಳುವರಲ್ಲ” ಎಂದು ಪ್ರಶ್ನಿಸುವರು. ಈ ಚಿನ್ನದ ನಗರವೆಲ್ಲ ರಾಮನ ವಶವಾಯಿತು. ಅವನು ರಾವಣನ ತಮ್ಮನಾದ ವಿಭೀಷಣನಿಗೆ ಇವನ್ನೆಲ್ಲ ಕೊಟ್ಟನು. ರಾಮನಿಗೆ ಯುದ್ಧದಲ್ಲಿ ಸಹಾಯ ಮಾಡಿದುದಕ್ಕಾಗಿ ಅವನನ್ನು ಲಂಕೆಗೆ ರಾಜನನ್ನಾಗಿ ಮಾಡಿದನು.

ಅನಂತರ ರಾಮ ಸೀತೆಯೊಡಗೂಡಿ ತನ್ನವರೊಡನೆ ಲಂಕೆಯನ್ನೂ ಬಿಟ್ಟನು. ಆದರೆ ಅನುಯಾಯಿಗಳಲ್ಲಿ ಕೆಲವರು “ಪರೀಕ್ಷೆ, ಪರೀಕ್ಷೆ” ಎಂದು ಅರಚಿಕೊಂಡರು. ರಾವಣನ ಅಧೀನದಲ್ಲಿದ್ದರೂ ಸೀತೆ ಪರಿಶುದ್ಧಾತ್ಮಳಾಗಿದ್ದಳು ಎಂಬುದು ಅವರಿಗೆ ತೋರಿರಲಿಲ್ಲ.\break “ಪರಿಶುದ್ಧಳೆ! ಅವಳು ಪಾತಿವ್ರತ್ಯದ ಸಾಕಾರಮೂರ್ತಿ” ಎಂದ ರಾಮ. ಚಿಂತೆಯಿಲ್ಲ, ನಮಗೆ ಪರೀಕ್ಷೆಯಾಗಬೇಕು ಎಂದು ಅರಚಿಕೊಂಡ ಜನರು ಅಲ್ಲೊಂದು ದೊಡ್ಡ\break ಅಗ್ನಿಕುಂಡವನ್ನು ಅಣಿಗೊಳಿಸಿದರು. ಸೀತೆ ಅದನ್ನು ಪ್ರವೇಶ ಮಾಡಬೇಕಾಯಿತು.\break ರಾಮ ದುಃಖದಿಂದ ಕುದಿಯುತ್ತಿದ್ದ, ಸೀತೆ ಎಲ್ಲಿ ಸೀದು ಹೋಗುವಳೋ ಎಂದು.\break ಆದರೆ ಅಗ್ನಿಕುಂಡದಿಂದ ಅಗ್ನಿಯೆ ಆ ಕ್ಷಣ ತನ್ನ ತಲೆಯ ಮೇಲೆ ಒಂದು ಸಿಂಹಾಸನವನ್ನು ಹೊತ್ತು ತಂದನು. ಸೀತೆ ಆ ಸಿಂಹಾಸನದ ಮೇಲೆ ಇದ್ದಳು. ಆಗ ಎಲ್ಲರಿಗೂ ಸಂತೋಷವಾಯಿತು. ಎಲ್ಲರಿಗೂ ತೃಪ್ತಿಯಾಯಿತು.

ರಾಮ ವನವಾಸದಲ್ಲಿದ್ದಾಗಲೆ ಅವನ ತಮ್ಮ ಭರತ ಬಂದು, ದಶರಥನ ಮರಣದ ವಿಚಾರವನ್ನು ಹೇಳಿ, ರಾಮನನ್ನು ಸಿಂಹಾಸನವನ್ನು ಏರಬೇಕೆಂದು ಬಲಾತ್ಕರಿಸುತ್ತಿದ್ದನು. ರಾಮ ವನವಾಸದಲ್ಲಿರುವಾಗ ಭರತ ಸಿಂಹಾಸನವನ್ನು ಏರಲಿಲ್ಲ. ಅಷ್ಟೇ ಅಲ್ಲ, ರಾಮನ ಬದಲು ಅವನ ಪಾದುಕೆಗಳನ್ನು ಸಿಂಹಾಸನದ ಮೇಲೆ ಇಟ್ಟು ಅವನ ಪರವಾಗಿ\break ರಾಜ್ಯವನ್ನು ನೋಡಿಕೊಳ್ಳುತ್ತಿದ್ದನು. ರಾಮನು ಅಯೋಧ್ಯೆಗೆ ಹಿಂತಿರುಗಿ ಬಂದ ಮೇಲೆ, ಜನರೆಲ್ಲರ ಕೋರಿಕೆಯ ಮೇಲೆ ರಾಮನೇ ರಾಜನಾದನು.

ರಾಮನು ರಾಜನಾದ ಮೇಲೆ ಪ್ರಾಚೀನ ಕಾಲದಿಂದ ರಾಜರು ಜನರ ಹಿತಕ್ಕೋಸುಗ ಆಚರಿಸುತ್ತಿದ್ದ ಕೆಲವು ಆವಶ್ಯಕವಾದ ವ್ರತಗಳನ್ನು ಪಾಲಿಸತೊಡಗಿದನು. ರಾಜ ಪ್ರಜೆಗಳ ಸೇವಕ. ನಾವು ಮುಂದೆ ನೋಡುವಂತೆ ಪ್ರಜಾಭಿಪ್ರಾಯಕ್ಕೆ ಮಾನ್ಯತೆ ಕೊಡಬೇಕು. ರಾಮ ಕೆಲವು ಕಾಲ ಸೀತೆಯೊಡನೆ ಸುಖವಾಗಿದ್ದ. ಪುನಃ ಜನರು ‘ರಾವಣ ಸೀತೆಯನ್ನು ಕದ್ದುಕೊಂಡು ಸಮುದ್ರದಾಚೆ ತೆಗೆದುಕೊಂಡು ಹೋಗಿದ್ದ’ ಎಂದು ಆಕ್ಷೇಪಣೆಯನ್ನು ಎತ್ತಿದರು. ಹಿಂದಿನ ಪರೀಕ್ಷೆಯಿಂದ ಅವರು ತೃಪ್ತರಾಗಲಿಲ್ಲ. ‘ಬೇರೊಂದು ಪರೀಕ್ಷೆ ಮಾಡಬೇಕು. ಇಲ್ಲದೆ ಇದ್ದರೆ ಅವಳನ್ನು ತ್ಯಜಿಸಬೇಕು’ ಎಂದರು.

ಪ್ರಜಾಭಿಪ್ರಾಯಕ್ಕೆ ಮನ್ನಣೆ ಕೊಡುವುದಕ್ಕಾಗಿ ಸೀತೆಯನ್ನು ದೇಶದಿಂದ ಹೊರಗೆ\break ಅಟ್ಟಿ ವಾಲ್ಮೀಕಿ ಮಹರ್ಷಿಯಿದ್ದ ಅರಣ್ಯದಲ್ಲಿ ಬಿಟ್ಟರು. ರಕ್ಷಕರಿಲ್ಲದೆ ಅಳುತ್ತಿದ್ದ ಸೀತೆಯನ್ನು ವಾಲ್ಮೀಕಿ ಕಂಡು, ಅವಳ ದುಃಖದ ಕಥೆಯನ್ನು ಕೇಳಿ, ತನ್ನ ಆಶ್ರಯದಲ್ಲೆ ಅವಳನ್ನು\break ಇರಿಸಿಕೊಂಡನು. ಸೀತೆಗೆ ಹೆರಿಗೆಯ ಸಮಯವಾಗಿತ್ತು. ಅವಳು ಅವಳಿಜವಳಿ ಮಕ್ಕಳಿಗೆ\break ಜನ್ಮವಿತ್ತಳು. ಮಕ್ಕಳಿಗೆ ಅವರಾರು ಎಂಬುದನ್ನು ಕವಿ ತಿಳಿಸಲಿಲ್ಲ, ಅವರನ್ನು ಬ್ರಹ್ಮಚಾರಿಗಳಂತೆ ಸಾಕಿದನು. ಆಗ ಅವನು ರಾಮಾಯಣವನ್ನು ಬರೆದು, ಅದನ್ನು ಗಮಕವನ್ನಾಗಿ ಮಾಡಿ, ನಾಟಕೀಕರಿಸಿದನು.

ಭರತಖಂಡದಲ್ಲಿ ನಾಟಕ ಅತಿ ಪವಿತ್ರವಾದುದು. ನಾಟಕ ಮತ್ತು ಸಂಗೀತ-ಇವುಗಳನ್ನೇ ಧರ್ಮವೆಂದು ಭಾವಿಸುವರು. ಎಂತಹ ಹಾಡಾದರೂ ಆಗಲಿ, ಅದು ಪ್ರೇಮಗೀತೆಯಾಗಲಿ ಅಥವಾ ಇನ್ನಾವುದಾದರೂ ಆಗಲಿ, ಆ ಹಾಡನ್ನೇ ತದ್ಗತಪ್ರಾಣವಾಗಿ ಹಾಡಿದವನಿಗೆ ಮುಕ್ತಿ ಲಭಿಸುವುದು. ಅವನು ಇನ್ನೇನೂ ಮಾಡಬೇಕಾಗಿಲ್ಲ. ಧ್ಯಾನವು ಸಾಧಿಸುವ ಗುರಿಯನ್ನೇ ಇದೂ ಸಾಧಿಸುವುದು ಎನ್ನುವರು.

ವಾಲ್ಮೀಕಿ ರಾಮಾಯಣವನ್ನು ಒಂದು ನಾಟಕವನ್ನಾಗಿ ಮಾಡಿ ಅದನ್ನು ಹೇಗೆ ಹಾಡ\-ಬೇಕೆಂಬುದನ್ನು ರಾಮನ ಇಬ್ಬರು ಮಕ್ಕಳಿಗೆ ಕಲಿಸಿದನು.

ಹಿಂದಿನ ಕಾಲದ ಅರಸರು ಮಾಡುತ್ತಿದ್ದಂತೆ ದೊಡ್ಡದೊಂದು ಯಾಗವನ್ನು ರಾಮ ಮಾಡಬೇಕಾದ ಸಮಯ ಬಂತು. ಆದರೆ ಇಂಡಿಯಾ ದೇಶದಲ್ಲಿ ಲಗ್ನವಾದವನು\break ಯಾವುದೊಂದು ಯಾಗಯಜ್ಞವನ್ನಾಗಲಿ ಹೆಂಡತಿಯಿಲ್ಲದೆ ಮಾಡಬಾರದು. ಅವನೊಡನೆ ಸಹಧರ್ಮಿಣಿ ಇರಬೇಕು. ಸತಿಗೆ ಅವರು ಉಪಯೋಗಿಸುವ ಪದ ಇದು. ಪೂಜಾದಿಗಳನ್ನು ಮಾಡಬೇಕು, ಆದರೆ ಸಹಧರ್ಮಿಣಿ ಭಾಗಿಯಾಗದೆ ಇದ್ದರೆ ಅವನು ಏನನ್ನೂ ಶಾಸ್ತ್ರೀಯವಾಗಿ ಮಾಡಲಾರ.

ಸೀತೆಯನ್ನು ದೇಶಬಾಹಿರಳನ್ನಾಗಿ ಮಾಡಿದುದರಿಂದ ರಾಮನು ಆ ಸಮಯದಲ್ಲಿ ಸಪತ್ನೀಕನಾಗಿರಲಿಲ್ಲ. ಪುನಃ ಮದುವೆಯಾಗೆಂದು ಜನ ಅವನನ್ನು ಕೋರಿಕೊಂಡರು. ಆದರೆ ಈ ಕೋರಿಕೆಗೆ ರಾಮ ಪ್ರಥಮ ಬಾರಿ ಪ್ರಜೆಗಳಿಗೆ ವಿರೋಧವಾಗಿ ನಿಂತನು. “ಇದು ಸಾಧುವಲ್ಲ, ನನ್ನ ಪ್ರಾಣ ಸೀತೆಯಲ್ಲಿದೆ” ಎಂದನು. ಯಾಗ ಮಾಡುವುದಕ್ಕಾಗಿ ಸೀತೆಗೆ ಬದಲಾಗಿ ಸೀತೆಯ ಒಂದು ಚಿನ್ನದ ವಿಗ್ರಹವನ್ನು ಮಾಡಿಸಿದನು. ಈ ಮಹೋತ್ಸವದ ಸಮಯದಲ್ಲಿ ಧರ್ಮೋದ್ದೀಪನೆಗೊಳಿಸುವುದಕ್ಕಾಗಿ ಒಂದು ನಾಟಕವನ್ನು ಆಡಿಸಿದರು.\break ಕವಿಯಾದ ವಾಲ್ಮೀಕಿ ಮಹರ್ಷಿಗಳು ತಮ್ಮ ಶಿಷ್ಯರಾದ ರಾಮನ ಅಜ್ಞಾತ ಮಕ್ಕಳಾದ ಲವಕುಶರೊಡನೆ ಬಂದರು. ಒಂದು ರಂಗಭೂಮಿಯನ್ನು ಅನುಗೊಳಿಸಿದರು. ನಾಟಕಕ್ಕೆ ಎಲ್ಲಾ\break ಅಣಿಯಾಯಿತು. ರಾಮ, ಅವನ ಸಹೋದರರು, ಜೊತೆಗೆ ದೊಡ್ಡ ದೊಡ್ಡ ಅಧಿಕಾರ\break ವರ್ಗವೆಲ್ಲ ಅದನ್ನು ನೋಡುವುದಕ್ಕೆ ನೆರೆದರು. ಪ್ರೇಕ್ಷಕರು ಕಿಕ್ಕಿರಿದಿದ್ದರು. ವಾಲ್ಮೀಕಿಯ ನೇತೃತ್ವದಲ್ಲಿ ಲವಕುಶರು ರಾಮನ ಜೀವನವನ್ನು ಹಾಡಿದರು. ಮಕ್ಕಳ ಗಾನದಿಂಪಿಗೆ ಮತ್ತು ಅವರ ಭವ್ಯಾಕೃತಿಗೆ ನೆರೆದವರೆಲ್ಲ ಮಾರುಹೋದರು. ಆದರೆ ಪಾಪ ರಾಮನಿಗೆ ಇದನ್ನು ಸಹಿಸಲು ಆಗಲಿಲ್ಲ. ಕಥೆಯಲ್ಲಿ ಸೀತಾಪರಿತ್ಯಾಗದ ಪ್ರಸಂಗ ಬಂದಾಗಲಂತೂ ಏನು\break ಮಾಡಬೇಕೊ ಅವನಿಗೆ ಗೊತ್ತಾಗಲಿಲ್ಲ. ಆಗ ಋಷಿಗಳು “ವ್ಯಥೆಪಡಬೇಡ, ನಾನು\break ಸೀತೆಯನ್ನು ತೋರುತ್ತೇನೆ.” ಎಂದರು. ಸೀತೆಯನ್ನು ರಂಗಭೂಮಿಯ ಮೇಲೆ ತಂದರು. ರಾಮನಿಗೆ ಸೀತೆಯನ್ನು, ನೋಡಿ ಸಂತೋಷವಾಯಿತು. ತಕ್ಷಣ ‘ಪರೀಕ್ಷೆ ಮಾಡಬೇಕು, ಪರೀಕ್ಷೆ ಮಾಡಬೇಕು’ ಎಂಬ ಹಿಂದಿನ ಗೊಣಗಾಟವೆ ಕೇಳಿಸಿತು. ಸೀತೆ, ಜನ ತನ್ನ ಪಾತಿವ್ರತ್ಯವನ್ನು ಇಷ್ಟು ಅನುಮಾನಿಸುವರಲ್ಲ ಎಂದು ಅತಿ ದುಃಖಿತಳಾದಳು. ಇದನ್ನು ಸಹಿಸುವುದಕ್ಕೆ ಅವಳಿಗೆ ಆಗಲಿಲ್ಲ. ತನ್ನ ಪಾತಿವ್ರತ್ಯವನ್ನು ಪರೀಕ್ಷೆಮಾಡಿ ಸಾಧ್ವಿ ಎಂಬುದನ್ನು ತೋರಿ ಎಂದು ದೇವತೆಗಳನ್ನು ಬೇಡಿಕೊಂಡಳು. ಆಗ ಭೂಮಿ ಇಬ್ಭಾಗವಾಯಿತು. “ಇದೇ ಪರೀಕ್ಷೆ” ಎಂದು ಸೀತೆ ಅದರಲ್ಲಿ ಪ್ರವೇಶಿಸಿ ಮಾಯಾವಾದಳು. ನೆರೆದ ಜನ ಇಂತಹ ದುರಂತವನ್ನು ನೋಡಿ ಮೂಕರಾದರು. ರಾಮ ಶೋಕಾಕ್ರಾಂತನಾದನು.

ಸೀತೆ ಮಾಯವಾದ ಕೆಲವು ದಿನಗಳ ಮೇಲೆ ದೇವತೆಗಳಿಂದ ಬೇಹುಗಾರರು ಬಂದು ರಾಮನಿಗೆ ಪ್ರಪಂಚದಲ್ಲಿ ಅವನು ಮಾಡಬೇಕಾದ ಕೆಲಸ ತೀರಿತು, ಇನ್ನು ಪುನಃ ಸ್ವರ್ಗಕ್ಕೆ ಹೊರಡಬೇಕು ಎಂದು ತಿಳಿಸಿದರು. ಈ ಸುದ್ದಿ ರಾಮನಿಗೆ ತನ್ನ ನೈಜಸ್ಥಿತಿಯ ಅರಿವನ್ನು ತಂದಿತು. ಅಯೋಧ್ಯಾ ನಗರದ ಸಮೀಪದಲ್ಲಿ ಹರಿಯುತ್ತಿದ್ದ ಸರಯೂ ನದಿಗೆ ಬಿದ್ದು ಬೇರೆ ಲೋಕದಲ್ಲಿ ಸೀತೆಯನ್ನು ಸೇರಿದನು.

ಇದೇ ಭರತಖಂಡದ ಪುರಾತನ ಮಹಾಕಾವ್ಯ. ಸೀತಾರಾಮರು ಹಿಂದೂ ಜನಾಂಗದ ಆದರ್ಶ. ಎಲ್ಲಾ ಮಕ್ಕಳೂ ಅದರಲ್ಲಿಯೂ ಹುಡುಗಿಯರು ಸೀತೆಯನ್ನು ಪೂಜಿಸುವರು. ಸ್ತ್ರೀಯ ಆಕಾಂಕ್ಷೆಯ ಪರಮಾವದಿಯೇ ಪರಿಶುದ್ಧಳಾದ, ಪತಿಪರಾಯಣೆಯಾದ, ಜೀವನವೆಲ್ಲ ನೊಂದ ಸೀತೆಯಂತಾಗುವುದು. ನೀವು ಈ ಶೀಲಗಳನ್ನು ಪರಿಗಣಿಸಿದರೆ ಪೌರಸ್ತ್ಯ ಪಾಶ್ಚಾತ್ಯ ಆದರ್ಶಗಳು ಎಷ್ಟು ಭಿನ್ನವಾಗಿವೆ ಎಂಬುದು ಗೊತ್ತಾಗುವುದು. ಸಂಸಾರದಲ್ಲಿ ಹೇಗೆ ದುಃಖವನ್ನು ಸಹಿಸಬೇಕು ಎಂಬುದಕ್ಕೆ ಸೀತೆ ಇಡೀ ದೇಶಕ್ಕೆ ಆದರ್ಶವಾಗಿರುವಳು. ಪಾಶ್ಚಾತ್ಯರು “ಏನನ್ನಾದರೂ ಸಾಧಿಸಿ, ನಿಮ್ಮ ಪೌರುಷವನ್ನು ತೋರಿ” ಎನ್ನುವರು.\break ಭಾರತೀಯರು “ಸಹಿಸಿ ತೋರಿ” ಎನ್ನುವರು. ಪಾಶ್ಚಾತ್ಯರು ಜನ ಎಷ್ಟನ್ನು ಪಡೆಯಬಹುದು ಎಂಬ ಸಮಸ್ಯೆಯನ್ನು ಬಗೆಹರಿಸಿರುವರು. ಭಾರತೀಯನು ಮನುಷ್ಯನಿಗೆ ಎಷ್ಟು ಸ್ವಲ್ಪ ಸಾಕು ಎಂಬ ಸಮಸ್ಯೆಯನ್ನು ಪರಿಹರಿಸುವನು. ಎರಡು ಅತಿಯನ್ನೂ ನೋಡಿ. ಸೀತೆಯದು ಇಂಡಿಯಾದೇಶಕ್ಕೆ ಸರಿಹೊಂದಿಕೊಳ್ಳುವ ಶೀಲ, ಇಂಡಿಯಾದೇಶದ ಆದರ್ಶ ಶೀಲ. ಅವಳು ನಿಜವಾಗಿ ಪ್ರಪಂಚದಲ್ಲಿ ಇದ್ದಳೆ? ರಾಮಾಯಣ ನಿಜವಾಗಿ ಐತಿಹಾಸಿಕವೆ? ಇದಲ್ಲ ಪ್ರಶ್ನೆ. ಈ ಆದರ್ಶ ಅಲ್ಲಿದೆ. ಸೀತಾದೇವಿಯ ಶೀಲದಷ್ಟು ಮತ್ತಾವ ಪೌರಾಣಿಕ ವ್ಯಕ್ತಿಯೂ ಇಡೀ ಹಿಂದೂ ಜನಾಂಗವನ್ನು ವ್ಯಾಪಿಸಿ, ಅವರ ನಿತ್ಯ ಜೀವನಕ್ಕೆ ಪ್ರವೇಶಿಸಿ, ಸಮಾಜದ ಪ್ರತಿಯೊಂದು ಧಮನಿಯಲ್ಲಿ ಅನುರಣಿತವಾಗುತ್ತಿಲ್ಲ. ಭರತಖಂಡದಲ್ಲಿ ಎಲ್ಲಾ ಒಳ್ಳೆಯದಕ್ಕೆ, ಪರಿಶುದ್ಧವಾಗಿರುವುದಕ್ಕೆ, ಪವಿತ್ರವಾಗಿರುವುದಕ್ಕೆ, ಸೀತೆ ಎಂದು ಹೆಸರು. ಸ್ತ್ರೀಯಲ್ಲಿ ಯಾವುದನ್ನು ಸ್ತ್ರೀತ್ವ ಎನ್ನುವೆವೋ ಅದೇ ಸೀತೆ ಎಂದು ಹೆಸರು. ಸ್ತ್ರೀಯಲ್ಲಿ ಯಾವುದನ್ನು ಸ್ತ್ರೀತ್ವ ಎನ್ನುವೆವೋ ಅದೇ ಸೀತೆ. ಬ್ರಾಹ್ಮಣನೊಬ್ಬ ಸ್ತ್ರೀಯನ್ನು ಹರಸಬೇಕಾದರೆ “ಸೀತೆಯಂತಾಗು” ಎನ್ನುವರು. ಹುಡುಗಿಯನ್ನು ಹರಸುವಾಗ “ಸೀತೆಯಂತಾಗು” ಎನ್ನುವನು. ಅವರೆಲ್ಲ ಸೀತೆಯ ಮಕ್ಕಳು. ಶಾಂತಳಾದ ಸಹಿಷ್ಣುತಾ ಮೂರ್ತಿಯಂತಿರುವ, ಪತಿವ್ರತೆಯಾದ, ನಿತ್ಯ ನಿರ್ಮಲಳಾದ ಸೀತೆಯಂತಾಗುವುದಕ್ಕೆ ಯತ್ನಿಸುತ್ತಿರುವರು. ಅವಳು ಅಷ್ಟೊಂದು ಕಷ್ಟವನ್ನು ಅನುಭವಿಸಿದರೂ ರಾಮನ ಮೇಲೆ ಒಮ್ಮೆಯಾದರೂ ಕಟುವಾಗಿ ಮಾತನಾಡುವುದಿಲ್ಲ. ಇದೆಲ್ಲ ತನ್ನ ಕರ್ತವ್ಯವೆಂದು ತನ್ನ ಪಾಲಿಗೆ ಬಂದುದನ್ನು ತಾನು ನಿರ್ವಹಿಸುವಳು. ಅವಳನ್ನು ಕಾಡಿಗೆ ಅಟ್ಟಿದ ಆ ಭಯಾನಕವಾದ ಅನ್ಯಾಯವನ್ನು ನೋಡಿ! ಆದರೆ ಸೀತೆ ಯಾರನ್ನೂ ದೂರುವುದಿಲ್ಲ. ಪುನಃ ಇದು ಭಾರತೀಯರ ಆದರ್ಶ. ಬುದ್ಧ ಹೀಗೆ ಸಾರುವನು: “ಒಬ್ಬ ನಿನ್ನನ್ನು ನೋಯಿಸಿದಾಗ ನೀನು ಅವನಿಗೆ ಪ್ರತೀಕಾರ ಮಾಡುವುದಕ್ಕೆ ಹೋದರೆ, ಮಾಡಿದ ಮೊದಲ ತಪ್ಪನ್ನು ಅದು ನಿವಾರಿಸಲಾರದು. ಪ್ರಪಂಚದಲ್ಲಿ ಮತ್ತೊಂದು ದುಷ್ಟಕೃತ್ಯ ಹೆಚ್ಚಾಗುವುದು.” ಸೀತೆ ಸ್ವಭಾವತಃ ನಿಜವಾದ ಭಾರತೀಯಳು. ಹಿಂಸೆಗೆ ಪ್ರತಿಹಿಂಸೆಯನ್ನು ಅವಳು ಕೊಟ್ಟವಳಲ್ಲ.

ಯಾವುದು ನಿಜವಾದ ಆದರ್ಶ ಯಾರಿಗೆ ಗೊತ್ತು? ಪಾಶ್ಚಾತ್ಯರಲ್ಲಿರುವ ತಾತ್ಕಾಲಿಕವಾದ ಅಧಿಕಾರ ಮತ್ತು ಶಕ್ತಿಯೊ, ಅಥವಾ ಪೌರಸ್ತ್ಯನ ಸಹಿಷ್ಣುತಾ ಶಕ್ತಿಯೊ?

ಪಾಶ್ಚಾತ್ಯರು ಪಾಪವನ್ನು ಗೆದ್ದು ಅದನ್ನು ಕಡಿಮೆ ಮಾಡುವೆವು ಎನ್ನುವರು. ಪೌರಸ್ತ್ಯರು “ಪಾಪವನ್ನು ಅನುಭವಿಸಿ ಅದನ್ನು ನಿರ್ಮೂಲ ಮಾಡುತ್ತೇವೆ. ಅದನ್ನು ಗಣನೆಗೇ ತರುವುದಿಲ್ಲ. ಅದೇ ಒಂದು ಆನಂದವಾಗುವುದು” ಎನ್ನುವರು. ಎರಡೂ ದೊಡ್ಡ ಆದರ್ಶಗಳೇ. ಕೊನೆಗೆ ಯಾವುದು ಉಳಿಯುವುದೊ ಯಾರಿಗೆ ಗೊತ್ತು? ಯಾವ ದೃಷ್ಟಿಯಿಂದ ಮಾನವಕೋಟಿಗೆ ಹೆಚ್ಚು ಮೇಲಾಗುವುದೊ ಯಾರಿಗೆ ಗೊತ್ತು? ಯಾವುದು ಮಾನವನ ಮೃಗೀಯತೆಯನ್ನು ಮೆಟ್ಟಿ ನಿಲ್ಲುವುದೊ, ಗೆಲ್ಲುವುದೊ ಯಾರಿಗೆ ಗೊತ್ತು? ಅನುಭವಿಸುವುದೋ ಅಥವಾ ಏನನ್ನಾದರೂ ಸಾಧಿಸುವುದೊ?

ಏನೇ ಆಗಲಿ, ಒಬ್ಬರು ಇನ್ನೊಬ್ಬರ ಆದರ್ಶವನ್ನು ನಾಶಮಾಡದಿರೋಣ. ನಮ್ಮಿಬ್ಬರ ಗುರಿಯೂ ಒಂದೇ. ಅದೇ ಪಾಪದ ನಾಶ. ನೀವು ನಿಮ್ಮ ಆದರ್ಶವನ್ನು ತೆಗೆದುಕೊಳ್ಳಿ. ನಾವು ನಮ್ಮ ಆದರ್ಶವನ್ನು ತೆಗೆದುಕೊಳ್ಳುತ್ತೇವೆ. ಆದರ್ಶವನ್ನು ನಾಶಮಾಡದಿರೋಣ. ನಾನು ಪಾಶ್ಚಾತ್ಯನಿಗೆ ತನ್ನ ಆದರ್ಶವನ್ನು ತೆಗೆದುಕೊ ಎಂದು ಹೇಳುವುದಿಲ್ಲ. ಎಂದಿಗೂ ಹೀಗೆ ಹೇಳುವುದಿಲ್ಲ. ಗುರಿ ಒಂದೇ ಆದರೂ ಮಾರ್ಗ ಒಂದೇ ಅಲ್ಲ. ಭರತಖಂಡದ ಆದರ್ಶಗಳನ್ನು ಕೇಳಿದ ಮೇಲೆ ನೀವು ಭಾರತೀಯರಿಗೆ “ಗುರಿ ಆದರ್ಶ ಇಬ್ಬರಿಗೂ ಸರಿಯಾಗಿ ಇದೆ. ನೀವು ನಿಮ್ಮ ಆದರ್ಶವನ್ನು ಅನುಸರಿಸಿ. ನಿಮ್ಮ ರೀತಿಯಲ್ಲಿ ನೀವು ಮುಂದುವರಿಯಿರಿ. ದೇವರು ನಿಮಗೆ ಒಳ್ಳೆಯದು ಮಾಡಲಿ” ಎಂದು ಆಶೀರ್ವದಿಸುತ್ತೀರಿ ಎಂದು ಭಾವಿಸುತ್ತೇನೆ. ಜೀವನದಲ್ಲಿ ನನ್ನ ಸಂದೇಶವೇನೆಂದರೆ ಪೌರಸ್ತ್ಯರು ಪಾಶ್ಚಾತ್ಯರು ಭಿನ್ನ ಭಿನ್ನ ಆದರ್ಶಕ್ಕಾಗಿ ಹೋರಾಡಬೇಕಾಗಿಲ್ಲ. ಅದು ಎಷ್ಟೇ ವಿರೋಧವಾಗಿ ಕಂಡರೂ ಗುರಿ ಒಂದೇ ಎಂಬುದನ್ನು ತೋರುವುದು. ಈ ಜಟಿಲವಾದ ಸಂಸಾರ ಕಾನನದಲ್ಲಿ ಪ್ರಯಾಣ ಮಾಡುತ್ತಿರುವಾಗ ಒಬ್ಬರು ಇನ್ನೊಬ್ಬರಿಗೆ ಶುಭವನ್ನು ಕೋರೋಣ.

