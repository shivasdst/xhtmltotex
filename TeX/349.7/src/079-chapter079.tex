
\vspace{-0.8cm}

\chapter[ರಾಜಯೋಗದ ಗುರಿ ]{ರಾಜಯೋಗದ ಗುರಿ \protect\footnote{\engfoot{C.W. Vol. V, P. 293}}}

ಧರ್ಮದಲ್ಲಿ ಧ್ಯಾನಕ್ಕೆ ಸಂಬಂಧಪಟ್ಟ ವಿಷಯವನ್ನು ಮಾತ್ರ ರಾಜಯೋಗ ಮುಖ್ಯವಾಗಿ ಗಮನಿಸುವುದು. ನೀತಿಗೆ ಸಂಬಂಧಪಟ್ಟ ವಿಷಯವನ್ನು ಎಲ್ಲೋ ಸ್ವಲ್ಪ ಮಾತ್ರ ಆವಶ್ಯಕವಾದಾಗ ಗಮನಿಸುವುದು. ಸ್ತ್ರೀಪುರುಷರು ಶ್ರುತಿಗಿಂತ ಹೆಚ್ಚಾಗಿರುವುದನ್ನು ಆಶಿಸುವರು. ಅದು ತಮ್ಮ ಅನುಭವಕ್ಕೆ ಬರುವಂತದ್ದಾಗಿರಬೇಕಾಗಿದೆ. ಪ್ರತ್ಯಕ್ಷ ಅನುಭವ ಸಿಕ್ಕಿದಾಗ ಮಾತ್ರ ಧರ್ಮದಲ್ಲಿ ಏನಾದರೂ ಸತ್ಯವಿದೆ ಎಂಬುದನ್ನು ನಂಬಬಹುದು. ಆಧ್ಯಾತ್ಮಿಕ ಸತ್ಯಗಳನ್ನು ಮನಸ್ಸಿನ ಅತೀಂದ್ರಿಯ ಅವಸ್ಥೆಯಿಂದ ಪಡೆಯಬೇಕಾಗಿದೆ. ತಮಗೇ ವಿಶಿಷ್ಟ ಅನುಭವಗಳು ಆದವು ಎಂದು ಹೇಳುವವರ ಮನೋಸ್ಥಿತಿಯಲ್ಲಿ ನಾವು ಇರಬೇಕು. ಅನಂತರ ನಮಗೂ ಅದೇ ಅನುಭವಗಳಾದರೆ ಅವು ಪ್ರತ್ಯಕ್ಷ ಪ್ರಮಾಣವಾಗುವುವು. ಮತ್ತೊಬ್ಬರು ನೋಡಿದುದನ್ನೆಲ್ಲಾ ಅಂಥದೇ ಪರಿಸ್ಥಿತಿಯಲ್ಲಿ ನಾವೂ ನೋಡಬಹುದು. ಯಾವುದು ಹಾಗೆ ಒಮ್ಮೆ ಆಯಿತೋ ಅದು ಪುನಃ ಆಗುವುದು, ಆಗಲೇಬೇಕು. ರಾಜಯೋಗ ಅಂತಹ ಅತೀಂದ್ರಿಯ ಅವಸ್ಥೆಯನ್ನು ಹೇಗೆ ಪಡೆಯುವುದು ಎಂಬುದನ್ನು ನಮಗೆ ಬೋಧಿಸುವುದು. ಎಲ್ಲಾ ದೊಡ್ಡ ಧರ್ಮಗಳೂ ಒಂದಲ್ಲ ಒಂದು ರೀತಿಯಲ್ಲಿ ಇದನ್ನು ಒಪ್ಪಿಕೊಳ್ಳುತ್ತವೆ. ಆದರೆ ಇಂಡಿಯಾ ದೇಶದಲ್ಲಿ ಧರ್ಮದ ಈ ಭಾಗಕ್ಕೆ ಪ್ರತ್ಯೇಕವಾಗಿ ಗಮನವನ್ನು\break ಕೊಟ್ಟಿರುವರು. ಪ್ರಾರಂಭದಲ್ಲಿ ಯಾಂತ್ರಿಕವಾಗಿ ಮಾಡುವ ಕ್ರಿಯೆಗಳು ಆ ಸ್ಥಿತಿಯನ್ನು\break ಪಡೆಯಲು ಸ್ವಲ್ಪ ಸಹಾಯ ಮಾಡಬಹುದು. ಆದರೆ ಇವುಗಳಿಂದಲೇ ನಾವು ಹೆಚ್ಚನ್ನು\break ಸಾಧಿಸಲು ಸಾಧ್ಯವಿಲ್ಲ. ಕೆಲವು ರೀತಿ ಉಸಿರೆಳೆಯುವುದು, ಕೆಲವು ರೀತಿ ಕುಳಿತುಕೊಳ್ಳುವುದು ಮನಸ್ಸಿನ ಶಾಂತಿಗೆ ಮತ್ತು ಏಕಾಗ್ರತೆಗೆ ಸಹಾಯ ಮಾಡಬಹುದು. ಆದರೆ ಇದರ ಜೊತೆಗೆ ಮನಸ್ಸು ಪರಿಶುದ್ಧವಾಗಿರಬೇಕು. ಭಗವಂತನನ್ನು ಸಾಕ್ಷಾತ್ಕಾರ ಮಾಡಿಕೊಳ್ಳಬೇಕೆಂಬ ಆಸೆ ತೀವ್ರವಾಗಿರಬೇಕು. ಒಂದು ಕಡೆ ಕುಳಿತುಕೊಂಡು ಒಂದು ವಸ್ತುವಿನ ಮೇಲೆ ಮನಸ್ಸನ್ನು ಇಡಬೇಕಾದರೆ ಅದಕ್ಕೆ ಯಾವುದಾದರೂ ಸಹಾಯ ಆವಶ್ಯಕ ಎಂದು ಅನೇಕರು ಭಾವಿಸುವರು. ಮನಸ್ಸನ್ನು ಕ್ರಮೇಣ ನಿಗ್ರಹಿಸಬೇಕಾಗಿದೆ. ಇದೊಂದು ಮಕ್ಕಳ ಆಟವಲ್ಲ. ಒಂದು ದಿನ ಮಾಡಿ ಮತ್ತೊಂದು ದಿನ ಎಸೆಯುವ ವಸ್ತುವಲ್ಲ. ಜೀವನವೆಲ್ಲಾ ಈ ಕೆಲಸವನ್ನು ಮಾಡಬೇಕಾಗಿದೆ. ನಾವು ಗುರಿಗೆ ಸೇರುವ ಪ್ರಯತ್ನದಲ್ಲಿ ಅದನ್ನು ಮುಟ್ಟಿದರೆ ಸಾರ್ಥಕವಾಗುವುದು. ನಮ್ಮ ಗುರಿಯೇ ಪರಮಾತ್ಮನೊಡನೆ ನಾವು ಒಂದು ಎಂಬುದು. ಅದಕ್ಕಿಂತ ಸ್ವಲ್ಪವೂ ಕಡಮೆಯಾದದ್ದಲ್ಲ. ಇದೇ ನಮ್ಮ ಗುರಿಯಾಗಿರುವಾಗ, ನಾವು ಗುರಿಯನ್ನು ಮುಟ್ಟುವುದರಲ್ಲಿ ಸಂದೇಹವಿಲ್ಲ ಎಂಬ ಅರಿವಿರುವಾಗ, ಅದಕ್ಕಾಗಿ ನಾವು ಯಾವುದೇ\break ತ್ಯಾಗವನ್ನು ಮಾಡಿದರೂ ಅದು ಅತಿಯಾಗುವುದಿಲ್ಲ.

