
\chapter[ಭಾರತದ ಮಹಿಳೆಯರನ್ನು ಕುರಿತು – ಅವರ ಭೂತ, – ವರ್ತಮಾನ ಮತ್ತು ಭವಿಷ್ಯ ]{ಭಾರತದ ಮಹಿಳೆಯರನ್ನು ಕುರಿತು – ಅವರ ಭೂತ, – ವರ್ತಮಾನ ಮತ್ತು ಭವಿಷ್ಯ \protect\footnote{\engfoot{C.W. Vol. V, P. 228}}}

\centerline{(“ಪ್ರಬುದ್ಧ ಭಾರತ” ಡಿಸೆಂಬರ್​ ೧೮೯೯)}

ಒಂದು ಭಾನುವಾರ ಪ್ರಾತಃಕಾಲ, ಸುಂದರವಾದ ಹಿಮಾಲಯದ ಕಣಿವೆಯಲ್ಲಿ ‘ಪ್ರಬುದ್ಧ ಭಾರತ’ ಸಂಪಾದಕರ ಆಣತಿಯಂತೆ ವಿವೇಕಾನಂದರಿಂದ ಭಾರತೀಯ ನಾರಿಯರ ಆದರ್ಶದ ವಿಷಯವಾಗಿ ತಿಳಿದುಕೊಳ್ಳಲು ಹೋದೆ. ನಾನು ಅವರನ್ನು ಕಂಡಾಗ ‘ಹೊರಗೆ ಗಾಳಿ ಸಂಚಾರಕ್ಕೆ ಹೋಗೋಣ ಬಾ’ ಎಂದರು. ಆಗ ಪ್ರಪಂಚದ ಅತಿ ರಮ್ಯವಾದ ದೃಶ್ಯಗಳ ಮಧ್ಯದಲ್ಲಿ ಇಬ್ಬರೂ ಹೊರಟೆವು.

ದಾರಿಯಲ್ಲಿ ಸ್ವಲ್ಪ ಸ್ವಲ್ಪ ನೆರಳು, ಮಧ್ಯೆ ಮಧ್ಯೆ ಬಿಸಿಲು ಇತ್ತು. ದಾರಿಯಲ್ಲಿ ಮೌನವಾಗಿದ್ದ ಗ್ರಾಮಗಳಿದ್ದುವು. ಆಟದಲ್ಲಿ ನಿರತರಾದ ಮಕ್ಕಳ ಅಲ್ಲಿ ಕಂಡರು. ಹೊಂಬಣ್ಣದ ಗೋಧಿಯ ತೆನೆಗಳಿದ್ದ ಹೊಲದ ಕಡೆ ಹೊರಟೆವು. ಅಲ್ಲಿ ನೀಲಿಯಾ ಕಾಶವನ್ನು ಭೇದಿಸಿಕೊಂಡು ಹೋಗುವಂತೆ ಕೆಲವು ಮರಗಳು ಮೇಲಕ್ಕೆ ಹೋಗಿದ್ದವು. ಅಲ್ಲಿ ಹಳ್ಳಿಯ ಹುಡುಗಿಯರು ಬಾಗಿ ಕುಡುಗೋಲಿನಿಂದ ಜೋಳದ ಕಡ್ಡಿಯನ್ನು ಚಳಿಗಾಲದ ದನದ ಮೇವಿಗೆ ಕೊಯ್ಯುವುದರಲ್ಲಿ ನಿರತರಾಗಿದ್ದರು. ದಾರಿ ಒಂದು ಸೇಬಿನ ತೋಟದ ಕೆಡಗೆ ತಿರುಗಿತು. ಒಂದು ಮರದ ಕೆಳಗೆ ಹೊಂಬಣ್ಣದ ಹಣ್ಣುಗಳ ರಾಶಿ ವಿಂಗಡಿಸುವುದಕ್ಕಾಗಿ ಬಿದ್ದಿತ್ತು. ನಾವು ಇನ್ನೂ ಮುಂದುವರಿದು ಬಯಲಿಗೆ ಬಂದೆವು. ದೂರದಲ್ಲಿ ಸುಂದರ ಮನೋಹರವಾದ ಹಿಮಾವೃತ ಶಿಖರಗಳು ಮುಗಿಲ ಪಂಕ್ತಿಗಳನ್ನು ಭೇದಿಸಿಕೊಂಡು ನೀಲಿಯಾಗಸವನ್ನು ಮುಟ್ಟುತ್ತಿದ್ದವು. ಸ್ವಲ್ಪ ಹೊತ್ತಿನಲ್ಲಿ ಸ್ವಾಮೀಜಿಯವರ ಮೌನಮುದ್ರೆ ತೆರೆಯಿತು:

ಸ್ವಾಮೀಜಿ: “ಆರ್ಯ ಮತ್ತು ಸೆಮೆಟಿಕ್​ ಸ್ತ್ರೀಯರ ಆದರ್ಶ ಯಾವಾಗಲೂ ಪರಸ್ಪರ ವಿರೋಧಿಗಳು. ಸೆಮೆಟಿಕ್​ ಜನಾಂಗದಲ್ಲಿ ಸ್ತ್ರೀಯು ಭಕ್ತಿ ಸಾಧನೆಗೆ ಆತಂಕ ಎಂಬುದಾಗಿ ಭಾವಿಸಲ್ಪಟ್ಟಿದ್ದಳು. ಅವಳು ಯಾವ ಧಾರ್ಮಿಕ ಕಾರ್ಯವನ್ನೂ ಮಾಡಕೂಡದು, ಊಟಕ್ಕೆ ಒಂದು ಹಕ್ಕಿಯನ್ನು ಕೂಡ ಅವಳು ಕೊಲ್ಲಕೂಡದು. ಆದರೆ ಆರ್ಯರ ದೃಷ್ಟಿಯಲ್ಲಿ ಪುರುಷನು ಸ್ತ್ರೀಯಿಲ್ಲದೆ ಯಾವ ವ್ರತವನ್ನೂ ಮಾಡ ಕೂಡದು.”

ಕೂಡಲೇ ಅಷ್ಟೊಂದು ಅನಿರೀಕ್ಷಿತವೂ, ವಿನಾಯಿತಿರಹಿತವೂ ಆದ ಅವರ ಹೇಳಿಕೆಯಿಂದ ಚಕಿತನಾಗಿ ನಾನು ಕೇಳಿದೆ: “ಹಿಂದೂಧರ್ಮವು ಆರ್ಯರ ಧರ್ಮವಲ್ಲವೆ?”

ಸ್ವಾಮೀಜಿ: “ಆಧುನಿಕ ಹಿಂದೂಧರ್ಮವು ಹೆಚ್ಚು ಪೌರಾಣಿಕವಾದದ್ದು, ಬೌದ್ಧಧರ್ಮದ ನಂತರ ಬಂದದ್ದು, ಮನೆಯಲ್ಲಿಟ್ಟಿರುವ ಅಗ್ನಿಗೆ ಆಹುತಿ ಹಾಕುವುದು ವೈದಿಕ ಕ್ರಿಯೆ. ಈ ಕಾರ್ಯಕ್ಕೆ ಸ್ತ್ರೀ ಅತ್ಯಾವಶ್ಯಕವಾಗಿರಬೇಕಾದರೂ ಅವಳು ಸಾಲಿ ಗ್ರಾಮ ಶಿಲೆಯನ್ನಾಗಲೀ ಮನೆಯ ದೇವರನ್ನಾಗಲೀ ಮುಟ್ಟಕೂಡದು. ಏಕೆಂದರೆ ಆ ದೇವತೆಗಳ ಪೂಜಾದಿಗಳು ಈಚಿನ ಪುರಾಣಗಳ ಕಾಲಕ್ಕೆ ಸೇರಿದವು ಎಂದು ದಯಾನಂದ ಸರಸ್ವತಿಗಳು ತೋರಿಸಿದ್ದಾರೆ.

ಪ್ರಶ್ನೆ: “ಹಾಗಾದರೆ ನಮ್ಮ ಸ್ತ್ರೀಯರಿಗೆ ಸಮನಾದ ಸ್ಥಾನ ಸಿಕ್ಕದೇ ಇರುವುದಕ್ಕೆ ಬೌದ್ಧಧರ್ಮದ ಪ್ರಭಾವ ಕಾರಣ ಎಂದು ಹೇಳುತ್ತೀರಾ?”

ಸ್ವಾಮೀಜಿ: “ಹೌದು, ಎಲ್ಲಿ ಇದು ಇದೆಯೋ ಅದಕ್ಕೆ ಕಾರಣವೇ ಅದು. ಆದರೆ ಇಲ್ಲಿ ನಾವು ಜೋಪಾನವಾಗಿರಬೇಕು. ಪಾಶ್ಚಾತ್ಯರು ದೂರುವುದಕ್ಕೆಲ್ಲ ನಾವು ಒಪ್ಪಿಗೆಯನ್ನು ಸೂಚಿಸಕೂಡದು. ನಮ್ಮಲ್ಲಿ ಸ್ತ್ರೀಯರಿಗೆ ಸಮಾನತೆಯನ್ನು ನೀಡಿಲ್ಲ ಎಂಬ ಐರೋಪ್ಯರ ಟೀಕೆಗಳನ್ನು ನಾವು ಕೂಡಲೇ ಒಪ್ಪಿಕೊಳ್ಳಬಾರದು. ಹಲವು ಶತಮಾನಗಳ ಕಾಲ ಸ್ತ್ರೀಯರನ್ನು ರಕ್ಷಿಸಬೇಕಾದ ಪರಿಸ್ಥಿತಿ ನಮಗೆ ಒದಗಿ ಬಂದಿತ್ತು. ಅದೇ ಹೊರತು, ಸ್ತ್ರೀಯರನ್ನು ಪುರುಷನಿಗಿಂತ ಕಡಮೆ ಎಂದು ಭಾವಿಸಿ ನಾವು ಅವರನ್ನು ಆ ಸ್ಥಿತಿಯಲ್ಲಿ ಇಟ್ಟಿರಲಿಲ್ಲ.”

ಪ್ರಶ್ನೆ: “ಹಾಗಾದರೆ ನಮ್ಮ ಸಮಾಜದಲ್ಲಿ ಈಗ ಸ್ತ್ರೀಗೆ ಇರುವ ಸ್ಥಾನದ ಬಗ್ಗೆ ನಿಮಗೆ ತೃಪ್ತಿಯಿದೆಯೆ?”

ಸ್ವಾಮೀಜಿ: “ಎಂದಿಗೂ ಇಲ್ಲ. ಆದರೆ ನಾವು ಅವರಿಗೆ ವಿದ್ಯಾಭ್ಯಾಸವನ್ನು ಮಾತ್ರ ಕೊಡಬಲ್ಲೆವೆ ಹೊರತು ಮತ್ತೇನನ್ನೂ ಮಾಡಲಾರೆವು. ಸ್ತ್ರೀಯರು ತಮ್ಮ ರೀತಿಯಲ್ಲಿಯೇ ತಮ್ಮ ಸಮಸ್ಯೆಗಳನ್ನು ಪರಿಹರಿಸಿಕೊಳ್ಳಬಲ್ಲ ಸ್ಥಿತಿಯಲ್ಲಿ ನಾವು ಅವರನ್ನು ಇಡಬೇಕು. ಇತರರು ಯಾರೂ ಅವರಿಗೆ ಇದನ್ನು ಮಾಡಲಾರರು, ಮಾಡಕೂಡದು. ನಮ್ಮ ಭಾರತೀಯ ಸ್ತ್ರೀಯರು ಜಗತ್ತಿನ ಇತರ ಹೆಂಗಸರಷ್ಟೇ ಇದನ್ನು ಸಾಧಿಸಲು ಸಮರ್ಥರು.”

ಪ್ರಶ್ನೆ: “ಬೌದ್ಧರು ಬೀರಿರುವ ಹೀನ ಪ್ರಭಾವಕ್ಕೆ ನಿಮ್ಮ ಪ್ರಕಾರ ಕಾರಣವೇನು?”

ಸ್ವಾಮೀಜಿ: “ಧರ್ಮಶ್ರದ್ಧೆ ಕಡಮೆಯಾದಾಗ ಇದು ಬಂತು. ಪ್ರತಿಯೊಂದು ಚಳುವಳಿಯೂ ಅದರಲ್ಲಿರುವ ಯಾವುದೋ ಒಂದು ವಿಶೇಷವಾದ ಸ್ವಭಾವದಿಂದ ಮುಂದೆ ಬರುವುದು. ಅದು ಅವನತಿಗೆ ಬಂದಮೇಲೆ ಮುಂಚೆ ಇದ್ದ ವೈಶಿಷ್ಟ್ಯವೇ ಅದರ ದೌರ್ಬಲ್ಯದ ಮೂಲವಾಗುವುದು. ಮಾನವ ಶ್ರೇಷ್ಠನಾದ ಭಗವಾನ್​ ಬುದ್ಧನು ಅದ್ಭುತವಾದ ಸಂಘಟನಾ ಸಾಮರ್ಥ್ಯವುಳ್ಳವನು. ಅವನು ಇದರ ಮೂಲಕ ಜಗತ್ತಿನ ಮೇಲೆಲ್ಲಾ ತನ್ನ ಪ್ರಭಾವವನ್ನು ಬೀರಿದ. ಆದರೆ ಅವನದು ಸಂನ್ಯಾಸಿಗಳ ಧರ್ಮ ವಾಗಿತ್ತು. ಆದ್ದರಿಂದ ಗೈರಿಕವಸನ ಕಂಡರೆ ಸಾಕು, ಅದನ್ನು ಪೂಜ್ಯದೃಷ್ಟಿಯಿಂದ ಜನ ನೋಡುವಂತಹ ಕೆಟ್ಟ ಪರಿಣಾಮ ಉಂಟಾಯಿತು. ಅವನು ಪ್ರಥಮಬಾರಿಗೆ ಸಾಮೂಹಿಕ ಧಾರ್ಮಿಕ ಜೀವನವನ್ನು ಜಾರಿಗೆ ತಂದನು. ಅಲ್ಲಿ ಭಿಕ್ಷುಣಿಯರನ್ನು ಭಿಕ್ಷುಗಳಿಗಿಂತ ಕಡಮೆ ಸ್ಥಾನದಲ್ಲಿಟ್ಟನು. ಭಿಕ್ಷುಗಳ ಆಣತಿ ಇಲ್ಲದೆ ಭಿಕ್ಷುಣಿಯರು ಯಾವುದನ್ನೂ ನಿಷ್ಕರ್ಷಿಸುವ ಹಾಗಿರಲಿಲ್ಲ. ಕೆಲವು ವಿಷಯದಲ್ಲಿ ಇದರಿಂದ ತಾತ್ಕಾಲಿಕವಾಗಿ ಸಂಘದಲ್ಲಿ ಒಂದು ಐಕಮತ್ಯವೇನೋ ಬಂತು. ಆದರೆ ಇದರ ಅನಂತರದ ಪರಿಣಾಮ ಮಾತ್ರ ಶೋಚನೀಯವಾಗಿದೆ.”

ಪ್ರಶ್ನೆ: “ಆದರೆ ವೇದದಲ್ಲಿಯೂ ಸಂನ್ಯಾಸವನ್ನು ಒಪ್ಪಿಕೊಂಡಿರುವರು.”

ಸ್ವಾಮೀಜಿ: “ಹೌದು ನಿಜ. ಆದರೆ ಅಲ್ಲಿ ಸ್ತ್ರೀಪುರುಷರೆಂಬ ಭೇದವಿಲ್ಲ. ಜನಕರಾಜನ ಆಸ್ಥಾನದಲ್ಲಿ ಯಾಜ್ಞವಲ್ಕ್ಯನನ್ನು ಹೇಗೆ ಪ್ರಶ್ನಿಸಿದರು ಗೊತ್ತಿಲ್ಲವೆ? ಅವನ ಪರೀಕ್ಷಕರಲ್ಲಿ ಪ್ರಮುಖಳಾದವಳು ಬ್ರಹ್ಮವಾದಿನಿಯಾದ ವಾಕ ಬಿಲ್ಲುಗಾರನ ಕೈಯಲ್ಲಿ ಹೊಳೆಯುತ್ತಿರುವ ಎರಡು ಶರಗಳಂತಿವೆ ನನ್ನ ಪ್ರಶ್ನೆಗಳು ಎನ್ನುವಳು. ಅವಳು ಸ್ತ್ರೀ ಎಂದು ಕೂಡ ಅಲ್ಲಿ ಹೇಳುವುದಿಲ್ಲ. ನಮ್ಮ ಹಿಂದಿನ ಕಾಲದ ಅರಣ್ಯಗಳಲ್ಲಿದ್ದ ವಿದ್ಯಾಸಂಸ್ಥೆಯಲ್ಲಿ ಬಾಲಕ ಬಾಲಕಿಯರಿಗೆ ಇದ್ದ ಸಮಾನತೆಗಿಂತ ಬೇರೆ ಯಾವುದಾದರೂ ಉತ್ತಮವಾಗಿರುವುದೇನು? ಸಂಸ್ಕೃತ ನಾಟಕವನ್ನು ಓದಿ. ಶಕುಂತಲೆಯ ಕಥೆಯನ್ನು ನೋಡಿ. ಟೆನಿಸನ್ನಿನ ‘ಪ್ರಿನ್​ಸೆಸ್​’ ಏನಾದರೂ ಹೆಚ್ಚಿಗೆ ಹೇಳಬಲ್ಲದೆ ಯೋಚಿಸಿ ನೋಡಿ.”

ಪ್ರಶ್ನೆ: “ಸ್ವಾಮೀಜಿ, ನೀವು ನಮ್ಮ ಗತಕಾಲದ ವೈಭವವನ್ನು ಅಷ್ಟು ಅದ್ಭುತವಾಗಿ ವಿವರಿಸಬಲ್ಲಿರಿ?”

ಸ್ವಾಮೀಜಿ: “ಬಹುಶಃ ಪೂರ್ವ ಪಾಶ್ಚಾತ್ಯ ಸಮಾಜಗಳೆರಡರ ಅನುಭವವೂ ನನಗೆ ಇರುವುದೇ ಇದಕ್ಕೆ ಕಾರಣವಿರಬಹುದು. ಯಾವ ಜನಾಂಗ ಸೀತೆಯನ್ನು ಸೃಷ್ಟಿಸಿತೋ–ಅದೊಂದು ಕನಸಿನ ಕಲ್ಪನೆಯೇ ಆಗಿದ್ದರೂ ಸರಿಯೆ ಆ ಜನಾಂಗವು ಪ್ರಪಂಚದಲ್ಲಿ ಮತ್ತೆಲ್ಲೂ ಕಾಣಲಾಗದ ಗೌರವವನ್ನು ಸ್ತ್ರೀಯರಿಗೆ ಕೊಟ್ಟಿದೆ. ಪಾಶ್ಚಾತ್ಯ ಸ್ತ್ರೀಯರಿಗೆ ಎಷ್ಟೋ ಕಾನೂನಿನ ತೊಡಕುಗಳಿವೆ. ಆದರೆ ನಮ್ಮ ಸ್ತ್ರೀಯರಿಗೆ ಅವು ಗೊತ್ತೇ ಇಲ್ಲ. ನಮ್ಮಲ್ಲಿಯೂ ತಪ್ಪು ಇದೆ, ವಿನಾಯಿತಿ ಇದೆ. ಇದರಂತೆಯೇ ಅವರಲ್ಲಿಯೂ ಹಾಗೆಯೇ ಇರುವುದು. ಪ್ರಪಂಚದಲ್ಲೆಲ್ಲಾ ಎಲ್ಲರೂ ಪ್ರೀತಿ, ಮಾಧುರ್ಯ, ಋಜುತ್ವಗಳನ್ನು ವ್ಯಕ್ತಗೊಳಿಸಲು ಯತ್ನಿಸುತ್ತಿರುವರು. ರಾಷ್ಟ್ರೀಯ ಆಚಾರ ವಿಚಾರಗಳಲ್ಲಿ ನಾವು ಇದನ್ನು ಬಹಳ ಚೆನ್ನಾಗಿ ನೋಡಬಹುದು ಎಂಬುದನ್ನು ಮರೆಯಕೂಡದು. ಕೌಟುಂಬಿಕ ಸದ್ಗುಣಗಳ ವಿಷಯದಲ್ಲಿ ನಮ್ಮ ಭಾರತೀಯ ವಿಧಾನಗಳು ಇತರರೆಲ್ಲ ವಿಧಾನಗಳಿಗಿಂತ ಮೇಲು ಎಂದು ನಿಸ್ಸಂದೇಹ ವಾಗಿ ಹೇಳಬಲ್ಲೆ.”

ಪ್ರಶ್ನೆ: “ಹಾಗಾದರೆ ನಮ್ಮ ಹೆಂಗಸರಿಗೆ ಏನಾದರೂ ಸಮಸ್ಯೆಗಳು ಇರುವುವೆ ಸ್ವಾಮೀಜಿ?”

ಸ್ವಾಮೀಜಿ: “ಎಷ್ಟೋ ದಾರುಣ ಸಮಸ್ಯೆಗಳು ಇವೆ. ಆದರೆ ವಿದ್ಯಾಭ್ಯಾಸ ಎಂಬ ಮಂತ್ರದಿಂದ ಬಗೆಹರಿಯದ ಸಮಸ್ಯೆ ಯಾವುದೂ ಇಲ್ಲ. ಆದರೆ ನಿಜವಾದ ವಿದ್ಯೆ ಏನೆಂಬುದನ್ನು ನಮ್ಮಲ್ಲಿ ಇನ್ನೂ ಯಾರೂ ಕಲ್ಪಿಸಿಕೊಂಡಿಲ್ಲ.

ಪ್ರಶ್ನೆ: “ನೀವು ಅದನ್ನು ಹೇಗೆ ವಿವರಿಸುತ್ತೀರಿ?”

ಸ್ವಾಮೀಜಿ: “ನಾನು ಯಾವುದನ್ನೂ ವಿವರಿಸುವುದಿಲ್ಲ. ಆದರೂ ಶಿಕ್ಷಣವೆಂದರೆ ಕೇವಲ ಪದ ಸಂಗ್ರಹವಲ್ಲ, ನಮ್ಮಲ್ಲಿರುವ ಶಕ್ತಿಗಳನ್ನು ಬೆಳೆಸಿಕೊಳ್ಳುವುದು ಅಥವಾ ವ್ಯಕ್ತಿಗಳು ತಮ್ಮ ಇಚ್ಛಾಶಕ್ತಿಯನ್ನು ಸರಿಯಾದ ಮತ್ತು ಸಮರ್ಥವಾದ ರೀತಿಯಲ್ಲಿ ಬಳಸುವಂತೆ ತರಬೇತಿ ನೀಡುವುದು. ನಾವು ಭರತಖಂಡದ ಕಾರ್ಯರಂಗಕ್ಕೆ ನಿರ್ಭೀತರಾದ ಮಹಾ ಸ್ತ್ರೀಯರನ್ನು ತರಬೇಕು. ಸಂಘಮಿತ್ರ, ಲೀಲಾ, ಅಹಲ್ಯಾ ಬಾಯಿ, ಮೀರಾಬಾಯಿ, ಮುಂತಾದವರ ಪರಂಪರೆಗೆ ಸೇರಲು ಯೋಗ್ಯರಾದವರು ಬರಬೇಕು. ಮಹಾನಾಯಕರ ತಾಯಂದಿರಾಗುವುದಕ್ಕೆ ಯೋಗ್ಯರಾಗಿರಬೇಕು. ಈ ತಾಯಂದಿರು ನಿಃಸ್ವಾರ್ಥಿಗಳೂ ಪರಿಶುದ್ಧರೂ, ಭಗವಂತನ ಪಾದಕಮಲಗಳನ್ನು ಸ್ವರ್ಶಮಾಡಿ ಬರುವ ಶಕ್ತಿಯಿಂದ ಪುನೀತರಾದವರೂ ಆಗಿರಬೇಕು.

ಪ್ರಶ್ನೆ: ವಿದ್ಯಾಭ್ಯಾಸದಲ್ಲಿ ಧರ್ಮದ ಅಂಶವೂ ಇರಬೇಕು ಎನ್ನುವುದು ನಿಮ್ಮ ಭಾವನೆಯೆ?”

ಸ್ವಾಮೀಜಿ: “ವಿದ್ಯಾಭ್ಯಾಸದ ತಿರುಳೇ ಧರ್ಮ ಎಂದು ನಾನು ಭಾವಿಸುತ್ತೇನೆ. ಆದರೆ ಇದನ್ನು ಗಮನಿಸಿ. ಧರ್ಮ ಎಂದರೆ ಅದು ನನ್ನ ನಿಮ್ಮ ಅಭಿಪ್ರಾಯವಲ್ಲ. ಇಲ್ಲಿ ಗುರುವು ಶಿಷ್ಯನು ಎಲ್ಲಿ ನಿಂತಿರುವನೋ ಅಲ್ಲಿಂದ ಅವನನ್ನು ಮುಂದಕ್ಕೆ ಒಯ್ಯಬೇಕು. ತನ್ನ ಸ್ವಭಾವಕ್ಕೆ ತಕ್ಕಂತೆ ಈ ಜೀವಿಯು ಪ್ರತಿರೋಧ ಕಡಮೆ ಇರುವ ದಾರಿಯಲ್ಲಿ ವಿಕಾಸವಾಗಲು ಅವಕಾಶ ಕೊಡಬೇಕು.”

ಪ್ರಶ್ನೆ: “ಸ್ತ್ರೀಯರ ಮಾತೃ ಮತ್ತು ಸತಿಯ ಸ್ಥಾನಗಳಿಗೆ ಅಷ್ಟು ಗೌರವ ಕೊಡದೆ, ಗೃಹಸ್ಥಾಶ್ರಮದ ಕರ್ತವ್ಯಗಳಿಂದ ತಪ್ಪಿಸಿಕೊಂಡ ಬ್ರಹ್ಮಚಾರಿಣಿಯರಾದ ಸ್ತ್ರೀಯರಿಗೆ ಪ್ರಾಶಸ್ತ್ರ್ಯವನ್ನು ಕೊಟ್ಟುದು ಇಡೀ ಸ್ತ್ರೀ ಕುಲಕ್ಕೆ ಕೊಡಲಿ ಪೆಟ್ಟನ್ನು ಕೊಟ್ಟಂತೆ ಅಲ್ಲವೆ?”

ಸ್ವಾಮೀಜಿ: “ಸ್ತ್ರೀಯರಿಗೆ ಬ್ರಹ್ಮಚರ್ಯವನ್ನು ಒತ್ತಿ ಹೇಳಿದರೆ ಅದು ಪುರುಷರಿಗೂ ಅದನ್ನೇ ಒತ್ತಿ ಹೇಳುವುದು ಎಂಬುದನ್ನು ನೀವು ನೆನಪಿನಲ್ಲಿ ಇಡಬೇಕು. ನಿಮ್ಮ ಪ್ರಶ್ನೆಯು ನಿಮ್ಮ ಮನಸ್ಸಿನ ಅಭಿಪ್ರಾಯ ಸ್ಪಷ್ಟವಾಗಿಲ್ಲ ಎಂಬುದನ್ನು ತೋರಿಸುತ್ತದೆ. ಹಿಂದೂಧರ್ಮವು ಮಾನವನಿಗೆ ಒಂದು ಕರ್ತವ್ಯವನ್ನು ಮಾತ್ರ ಹೇಳುವುದು. ಅದು ಅನಿತ್ಯದಲ್ಲಿ ನಿತ್ಯವನ್ನು ಅರಸುವುದು. ಇದನ್ನು ಮಾಡುವುದಕ್ಕೆ ಇರುವುದು ಒಂದು ಮಾರ್ಗವಲ್ಲ. ಮದುವೆ ಆಗುವುದೊ, ಆಗದೆ ಇರುವುದೊ, ಪಾಪವೊ ಪುಣ್ಯವೊ, ಜ್ಞಾನವೊ ಅಜ್ಞಾನವೊ, ಇವುಗಳಲ್ಲಿ ಯಾವುದು ನಮ್ಮನ್ನು ಗುರಿಗೆ ಒಯ್ಯುತ್ತದೋ ಅದು ಒಳ್ಳೆಯದು. ಇಲ್ಲೇ ಬೌದ್ಧಧರ್ಮಕ್ಕೂ ಹಿಂದೂ ಧರ್ಮಕ್ಕೂ ವ್ಯತ್ಯಾಸ ಇರುವುದು. ಬೌದ್ಧರು ಈ ಪ್ರಪಂಚವು ಕ್ಷಣಿಕ ಎನ್ನುವುದನ್ನು ಅರಿಯಬೇಕಾಗಿದೆ. ಇದಕ್ಕೆ ಇರುವ ಮಾರ್ಗ ಒಂದೇ ಎನ್ನುತ್ತಾರೆ. ನಿಮಗೆ ಮಹಾ ಭಾರತದಲ್ಲಿ ಬರುವ, ತರುಣಯೋಗಿಯು ಕೋಪಾವಿಷ್ಟನಾಗಿ ತನ್ನ ತಪಸ್ಸಿನಿಂದ ಒಂದು ಕಾಗೆಯನ್ನು ಸುಟ್ಟನೆಂದು ಹೆಮ್ಮೆ ಪಡುವ ಕಥೆ ಗೊತ್ತಿದೆಯೆ? ಆತ ಒಬ್ಬ ಗೃಹಿಣಿಯ ಹತ್ತಿರ ಹೋದದ್ದು, ಅನಂತರ ಆತ ಧರ್ಮವ್ಯಾಧನ ಹತ್ತಿರ ಹೋದದ್ದು, ಅವರು ಕೇವಲ ತಮ್ಮ ಕರ್ತವ್ಯ ಪರಿಪಾಲನೆಯಿಂದಲೆ ಗುರಿಯನ್ನು ಸೇರಿದ್ದು; ನಿಮಗೆ ಜ್ಞಾಪಕವಿದೆಯೆ?”

ಪ್ರಶ್ನೆ: “ಈ ದೇಶದ ನಾರಿಯರಿಗೆ ನೀವು ಏನನ್ನು ಹೇಳುತ್ತೀರಿ?”

ಸ್ವಾಮೀಜಿ: “ಏತಕ್ಕೆ, ನಾನು ಗಂಡಸರಿಗೆ ಏನನ್ನು ಹೇಳುತ್ತೇನೆಯೋ ಅದನ್ನೇ ಹೆಂಗಸರಿಗೂ ಹೇಳುತ್ತೇನೆ. ಇಂಡಿಯಾ ದೇಶದಲ್ಲಿ ಮತ್ತು ಇಲ್ಲಿಯ ಜನರ ಶ್ರದ್ಧೆ ಯಲ್ಲಿ ನಂಬಿಕೆಯನ್ನು ಇಡಿ. ಶಕ್ತಿವಂತರಾಗಿರಿ, ಆಶಾವಾದಿಗಳಾಗಿರಿ, ನೀವು ನಾಚಿಕೆ ಪಟ್ಟುಕೊಳ್ಳಬೇಕಾಗಿಲ್ಲ. ಆದರೂ ಹಿಂದೂಗಳು ಬೇರೆ ಯಾವುದೇ ರಾಷ್ಟ್ರದ ವರಿಗಿಂತಲೂ ಹೆಚ್ಚಾಗಿ ಜಗತ್ತಿಗೆ ನೀಡಬೇಕಾದದ್ದು ಇದೆ ಎಂಬುದನ್ನು ನೆನಪಿನಲ್ಲಿಡಿ.

