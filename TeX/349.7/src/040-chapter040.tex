
\chapter[ಭಕ್ತಿ ]{ಭಕ್ತಿ \protect\footnote{\engfoot{C.W. Vol. V, P. 265}}}

ನಮ್ಮನ್ನು ಹೊಡೆಯುವುದಕ್ಕೆ ಒಂದು ಬೆತ್ತವನ್ನು ಕೈಯಲ್ಲಿ ಹಿಡಿದುಕೊಂಡಿ ರುವ ದೇವರಿಲ್ಲದೇ ಇದ್ದರೆ ಜನರು ನೀತಿವಂತರಾಗುವುದಿಲ್ಲ ಎಂದು ದ್ವೈತಿಗಳು ಭಾವಿಸುವರು. ಇದು ಹೇಗೆ? ಚಾವಟಿ ಪೆಟ್ಟು ಬಿದ್ದರೇನೆ ಮುಂದೆ ಹೋಗು ವಂತಹ ಗಾಡಿಗೆ ಕಟ್ಟಿದ ಕುದುರೆಯೊಂದು ನೈತಿಕತೆಯ ಮೇಲೆ ಉಪನ್ಯಾಸ ಕೊಡು ತ್ತದೆ ಎಂದಿಟ್ಟುಕೊಳ್ಳಿ. ಅದು ಮಾನವರ ಬಗ್ಗೆ ಮಾತನಾಡುತ್ತ, ಅವರು ಅನೀತಿ ವಂತರು, ನೀತಿವಂತರಾಗಿರಲಾರರು ಏತಕ್ಕೆಂದರೆ ಪ್ರತಿದಿನ ಅವರನ್ನು ಯಾರೂ ಹೊಡೆಯು ತ್ತಿಲ್ಲ ಎಂದು ಭಾವಿಸುವುದು. ಚಾವಟಿ ಏಟಿನ ಅಂಜಿಕೆಯು ಜನರನ್ನು ಹೆಚ್ಚು ಅನೀತಿವಂತರನ್ನಾಗಿಯೇ ಮಾಡುವುದು.

ನೀವೆಲ್ಲ ದೇವರೊಬ್ಬನಿರುವನು, ಅವನು ಸರ್ವಾಂತರ್ಯಾಮಿ ಎನ್ನುವಿರಿ. ನಿಮ್ಮ ಕಣ್ಣು ಮುಚ್ಚಿಕೊಂಡು ಅದನ್ನು ಸ್ವಲ್ಪ ಯೋಚಿಸಿ ನೋಡಿ. ನಿಮಗೆ ಏನು ಕಾಣುವುದು? ಆಗ ನಿಮ್ಮ ಮನಸ್ಸಿನಲ್ಲಿ ನೀವು ಸರ್ವಾಂತರ್ಯಾಮಿಯ ಭಾವನೆ ಯನ್ನು ತರಲು ಪ್ರಯತ್ನಿಸಿದರೆ ನೀವು ಆಕಾಶವನ್ನೊ, ಬಯಲನ್ನೊ, ಸಾಗರವನ್ನೊ ಚಿಂತಿಸುತ್ತಿರುವಿರೆಂದು ನಿಮಗೆ ಗೊತ್ತಾಗುವುದು. ಹೀಗಿದ್ದರೆ ದೇವರ ಸಾರ್ವಂತರ್ ಯಾಮಿತ್ವದ ಭಾವನೆ ನಿಮಗೆ ಬರುವುದಿಲ್ಲ. ನಿಮಗೆ ಅದರ ಅರ್ಥವೇ ಗೊತ್ತಾಗು ವುದಿಲ್ಲ. ಅದರಂತೆಯೇ ದೇವರಿಗೆ ಸಂಬಂಧಪಟ್ಟ ಗುಣಗಳೆಲ್ಲ, ಸರ್ವಶಕ್ತ, ಸರ್ವಜ್ಞ ಎಂಬ ಭಾವನೆಗಳನ್ನು ಕುರಿತು ನಿಮಗೆ ಏನು ತಾನೆ ಗೊತ್ತಿದೆ? ನಿಮಗೆ ಏನೂಗೊತ್ತಿಲ್ಲ. ಸಾಕ್ಷಾತ್ಕಾರವೇ ಧರ್ಮ. ನೀವು ಈ ಭಾವನೆಯನ್ನು ಸಾಕ್ಷಾತ್ಕರಿಸಿಕೊಳ್ಳಲು ಸಾಧ್ಯವಾಗದಾಗ ಮಾತ್ರ ನಿಮ್ಮನ್ನು ಆಸ್ತಿಕರು ಎಂದು ಕರೆಯುತ್ತೇನೆ. ಅದಕ್ಕಿಂತ ಮುಂಚೆ ‘ಆಸ್ತಿಕರು’ ಎಂಬುದು ಬರಿಯ ಅಕ್ಷರಗಳ ಗುಂಪು ಅಷ್ಟೇ. ಸಾಕ್ಷಾತ್ಕಾರ ಶಕ್ತಿಯೇ ಧರ್ಮ. ಅದು ನೀವು ಸುಮ್ಮನೆ ನಿಮ್ಮ ತಲೆಯೊಳಗೆ ಸಂಗ್ರಹಿಸಿಟ್ಟುಕೊಂಡ ಸಿದ್ಧಾಂತವೂ ಅಲ್ಲ, ನೀತಿಶಾಸ್ತ್ರಗಳೂ ಅಲ್ಲ. ಅವುಗಳಿಂದ ಏನೂ ಪ್ರಯೋಜನವಿಲ್ಲ. ನೀವು ಏನಾಗಿರುವಿರೋ, ಏನನ್ನು ಸಾಕ್ಷಾತ್ಕಾರ ಮಾಡಿಕೊಂಡಿರುವಿರೋ ಅದು ಮಾತ್ರ ಧರ್ಮ.

ಅವ್ಯಕ್ತವನ್ನು ಮಾಯೆಯ ತೆರೆಯ ಮೂಲಕ ನೋಡಿದಾಗ ಅದು ಸಗುಣ ದೈವವಾಗಿ ಕಾಣಿಸುವುದು. ನಾವು ಅವನನ್ನು ಪಂಚೇಂದ್ರಿಯಗಳ ಮೂಲಕ ನೋಡಿದಾಗ ಅವನನ್ನು ಸಾಕಾರದಂತೆ ನೋಡಲು ಸಾಧ್ಯ. ಆತ್ಮನನ್ನು ದೃಶ್ಯವಸ್ತುವನ್ನಾಗಿ ಮಾಡುವುದಕ್ಕೆ ಆಗುವುದಿಲ್ಲ. ಅರಿಯುವವನು ತನ್ನನ್ನು ತಾನೆ ಹೇಗೆ ಅರಿಯಬಲ್ಲ. ಆದರೆ ಅದಕ್ಕೆ ಬೇಕಾದರೆ ಒಂದು ನೆರಳು ಇರಬಹುದು. ಅದನ್ನೇ ಒಂದು ವ್ಯಕ್ತಿ ಎನ್ನಬಹುದು. ಆ ನೆರಳಿನ ಅತ್ಯಂತ ಶ್ರೇಷ್ಠವಾದ ರೂಪವೆ, ಆತ್ಮವು ತನ್ನನ್ನು ತಾನೇ ವಿಷಯೀಕರಿಸುವ ಆ ಪ್ರಯತ್ನವೇ ಸಗುಣ ದೇವರು. ಆತ್ಮವು ನಿತ್ಯಸಾಕ್ಷಿ. ನಾವು ಅದನ್ನು ಒಂದು ದೃಶ್ಯವಸ್ತುವನ್ನಾಗಿ ಮಾಡಬೇಕೆಂದು ಯತ್ನಿಸುತ್ತಿರುವೆವು. ಆ ಪ್ರಯತ್ನದಿಂದ ಈ ವಿಶ್ವ, ದ್ರವ್ಯ ಮುಂತಾದುವು ಬಂದಿವೆ. ಆದರೆ ಇವೆಲ್ಲಾ ಬಹಳ ದುರ್ಬಲವಾದ ಪ್ರಯತ್ನಗಳು. ಆತ್ಮನ ಶ್ರೇಷ್ಠ ಬಾಹ್ಯ ವ್ಯಕ್ತಿತ್ವವೇ ಸಗುಣ ದೇವರು. ಅದನ್ನು ಹೊರಗೆ ಕಾಣು ವಂತೆ ಮಾಡುವುದೇ ನಮ್ಮ ಸ್ವಭಾವವನ್ನು ವ್ಯಕ್ತಗೊಳಿಸುವುದು. ಸಾಂಖ್ಯದರ್ಶನದ ಪ್ರಕಾರ ಪ್ರಕೃತಿಯು ಪುರುಷನಿಗೆ ಈ ಅನುಭವಗಳನ್ನೆಲ್ಲಾ ಕೊಡುತ್ತಿರುವುದು. ಪುರುಷನಿಗೆ ನಿಜವಾದ ಅನುಭವ ಬಂದಾಗ ಅವನು ತನ್ನನ್ನು ತಾನು ಅರಿಯುವನು. ಅದ್ವೈತ ವೇದಾಂತದ ಪ್ರಕಾರ ಆತ್ಮವು ತನ್ನನ್ನು ಅರಿಯಲು ಯತ್ನಿಸುತ್ತಿರುವುದು. ಬೇಕಾದಷ್ಟು ಪ್ರಯತ್ನಪಟ್ಟಾದ ಮೇಲೆ ಸಾಕ್ಷಿಯು ಯಾವಾಗಲೂ ಸಾಕ್ಷಿಯಂತೆಯೇ ನಿಲ್ಲಬೇಕೆಂಬುದನ್ನು ಅರಿಯುವನು. ಆಗ ಆತ್ಮವು ಅನಾಸಕ್ತವಾಗಿ ಮುಕ್ತವಾಗುವುದು.

ಇಂತಹ ಪೂರ್ಣ ಸ್ಥಿತಿಯನ್ನು ಪಡೆದ ಮೇಲೆ ಮಾನವನು ಸಗುಣ ದೇವರಂತೆಯೇ ಆಗುವನು. ನಾನು ನನ್ನ ತಂದೆ ಇಬ್ಬರೂ ಒಂದೇ, ಬ್ರಹ್ಮವೆ ತಾನು ಎಂಬುದನ್ನು ಅವನು ಅರಿತಿರುವನು. ತಾನೇ ಸಗುಣ ದೇವರಂತೆ ರೂಪಧಾರಣ ಮಾಡುವನು. ಇದು ಕೆಲವು ವೇಳೆ ಮಹಾ ಪರಾಕ್ರಮಶಾಲಿಗಳಾದ ರಾಜರುಗಳು ಕೂಡ ಗೊಂಬೆಗಳೊಂದಿಗೆ ಆಡುವಂತೆ.

ಕೆಲವು ಕಲ್ಪನೆಗಳು ನಮ್ಮನ್ನು ಇತರ ಕಲ್ಪನೆಗಳ ಬಂಧನದಿಂದ ಪಾರು ಮಾಡು ವುವು. ಈ ವಿಶ್ವವೇ ಒಂದು ಕಲ್ಪನೆ. ಒಂದು ಬಗೆಯ ಕಲ್ಪನೆ ಮತ್ತೊಂದು ಬಗೆಯ ಕಲ್ಪನೆಯನ್ನು ಹೋಗಲಾಡಿಸುವುದು. ಪ್ರಪಂಚದಲ್ಲಿ ಪಾಪ ದುಃಖ ಮರಣ ಇವು ಇವೆ ಎನ್ನುವ ಕಲ್ಪನೆ ಭಯಾನಕ. ದೇವರೊಬ್ಬನಿರುವನು, ನೀನು ಪವಿತ್ರಾತ್ಮ, ನೋವಿಲ್ಲ ಎಂಬ ಭಾವನೆಗಳು ಒಳ್ಳೆಯವು. ಇತರ ಕಲ್ಪನೆಗಳ ಬಂಧನದಿಂದ ಇವು ನಮ್ಮನ್ನು ಪಾರುಮಾಡುವುವು. ಬಂಧನದ ಸರಪಳಿಯ ಕೊಂಡಿಗಳನ್ನೆಲ್ಲಾ ಕಳಚುವ ಅತ್ಯುನ್ನತ ಕಲ್ಪನೆಯೇ ಸಗುಣ ದೇವರು.

“ದೇವರೆ ಇದನ್ನು ನೋಡಿಕೊ, ಅದನ್ನು ಕೊಡು, ನಾನು ನಿನ್ನನ್ನು ಪ್ರಾರ್ಥಿಸು ವೆನು, ನೀನು ನನಗೆ ಪ್ರತಿದಿನ ಬೇಕಾಗಿರುವುದನ್ನೆಲ್ಲಾ ಕೊಡು, ನನ್ನ ತಲೆನೋವನ್ನು ಗುಣಮಾಡು.” ಹೀಗೆ ಪ್ರಾರ್ಥನೆ ಮಾಡುವುದು ಭಕ್ತಿಯಲ್ಲ. ಇವೆಲ್ಲ ಧರ್ಮದ ಕೆಳಗಿನ ಮೆಟ್ಟಿಲುಗಳು. ಇವೆಲ್ಲಾ ಬಹಳ ಕೆಳಮಟ್ಟದ ಕರ್ಮಗಳು. ಒಬ್ಬ ತನ್ನ ಶಕ್ತಿಯನ್ನೆಲ್ಲಾ ದೇಹದ ತೃಪ್ತಿಗೆ ಮತ್ತು ತನ್ನ ಇತರ ಬಯಕೆಗಳ ತೃಪ್ತಿಗೆ ವ್ಯಯಮಾಡಿದರೆ ಅವನಿಗೂ ಒಂದು ಪ್ರಾಣಿಗೂ ಏನು ವ್ಯತ್ಯಾಸ? ಭಕ್ತಿ ಬಹಳ ಶ್ರೇಷ್ಠವಾದುದು. ಅದು ಸ್ವರ್ಗಕ್ಕಿಂತಲೂ ಮಿಗಿಲು. ಸ್ವರ್ಗವೆಂದರೆ ಅತಿ ಸಮೃದ್ಧವಾದ ಭೋಗವಿರುವ ಸ್ಥಳವಷ್ಟೆ. ಅದೇ ಹೇಗೆ ದೇವರಾಗಬಲ್ಲದು?

ಮೂರ್ಖರು ಮಾತ್ರ ಇಂದ್ರಿಯಭೋಗವಸ್ತುಗಳನ್ನು ಹುಡುಕಲು ಕಾತರರಾಗಿ ರುವವರು. ತಿನ್ನುವುದು, ಕುಡಿಯುವುದು, ಹಿಂದಿನಂತೆಯೇ ಮಾಡುವುದು ಇವು ಸುಲಭ. ಆಧುನಿಕ ತಾತ್ತ್ವಿಕರು ಈ ಹಿತವಾದ ಭಾವನೆಗಳನ್ನು ತೆಗೆದುಕೊಂಡು ಅವಕ್ಕೆ ಧರ್ಮದ ಮುದ್ರೆಯನ್ನು ಹಾಕಿ ಎನ್ನುವರು. ಇಂತಹ ಸಿದ್ಧಾಂತ ಬಹಳ ಅಪಾಯಕರ. ವಿಷಯಾಸಕ್ತಿಯಲ್ಲಿ ಮರಣವಿದೆ. ಆಧ್ಯಾತ್ಮಿಕ ಕ್ಷೇತ್ರದಲ್ಲಿ ಮಾತ್ರ ನಿಜವಾದ ಜೀವನ ಇರುವುದು, ಇತರ ಕ್ಷೇತ್ರಗಳಲ್ಲಿ ಬಾಳುವುದು ಮೃತ್ಯು ಸಮಾನ. ಈ ಜೀವನವನ್ನೆಲ್ಲಾ ನಾವು ಒಂದು ವ್ಯಾಯಾಮಶಾಲೆ ಎಂದು ವಿವರಿಸಬಹುದು. ನಾವು ನಿಜವಾದ ಜೀವನವನ್ನು ಅನುಭವಿಸಬೇಕಾದರೆ ಇದನ್ನು ಮೀರಿಹೋಗಬೇಕು.

ಎಲ್ಲಿಯವರೆಗೂ ನನ್ನನ್ನು ಮುಟ್ಟಬೇಡ ಎಂಬುದು ನಿಮ್ಮ ಮತವಾಗಿರು ವುದೋ, ಅಡಿಗೆಯ ಪಾತ್ರೆ ನಿಮ್ಮ ದೇವರಾಗಿರುವುದೋ, ಅಲ್ಲಿಯವರೆಗೆ ನೀವು ಆಧ್ಯಾತ್ಮಿಕ ಕ್ಷೇತ್ರದಲ್ಲಿ ಮುಂದುವರಿಯಲಾರಿರಿ. ವಿವಿಧ ಧರ್ಮಗಳೊಳಗೆ ಇರುವ ಅಲ್ಪ ಭೇದಗಳೆಲ್ಲ ಬರಿಯ ಬಾಯಿಮಾತುಗಳಲ್ಲಿರುವ ಭೇದಗಳು. ಅವುಗಳಿಗೆ ಅರ್ಥವಿಲ್ಲ. ಪ್ರತಿಯೊಬ್ಬನೂ ಕೂಡ ಇದನ್ನು ನನ್ನ ಸ್ವಂತ ಅಭಿಪ್ರಾಯ ಎಂದು ಭಾವಿಸಿ ತನ್ನ ಇಚ್ಛೆಯಂತೆ ನಡೆಯಲು ಯತ್ನಿಸಿದಾಗಲೇ ಮನಸ್ತಾಪಗಳು ಉಂಟಾಗುವುವು.

ಇನ್ನೊಬ್ಬರನ್ನು ಹಳಿಯುವಾಗ ನಮ್ಮಲ್ಲಿರುವ ಯಾವುದಾದರೂ ಒಂದು ಒಳ್ಳೆಯ ಗುಣವನ್ನು ತೆಗೆದುಕೊಂಡು ಅದೇ ನನ್ನ ಜೀವನದ ಸರ್ವಸ್ವ ಎಂದು ಭಾವಿಸಿ ಇತರರಲ್ಲಿರುವ ಹೀನಗುಣಗಳೊಡನೆ ತುಲನೆ ಮಾಡಿ ನೋಡುವೆನು. ಇತರರನ್ನು ಅಳೆಯುವುದರಲ್ಲಿ ನಾವು ತಪ್ಪು ಮಾಡುವುದು ಹೀಗೆ.

ನನ್ನದೇ ಸರಿ ಎಂಬ ಮತಭ್ರಾಂತಿಯ ಮೂಲಕ ಧರ್ಮವನ್ನು ಬಹಳ ಬೇಗ ಹರಡಬಹುದು ನಿಜ. ಆದರೆ ಪ್ರತಿಯೊಬ್ಬರಿಗೂ ಅಭಿಪ್ರಾಯ ಸ್ವಾತಂತ್ರ್ಯವನ್ನು ಕೊಟ್ಟು ಅವರನ್ನು ಕ್ರಮೇಣ ಮೇಲೆ ಎತ್ತುವ ಕಾರ್ಯ ನಿಧಾನವಾದರೂ ಶಾಶ್ವತವಾದುದು.

ಮೊದಲು ಆಧ್ಯಾತ್ಮಿಕ ಭಾವನೆಗಳನ್ನು ಭರತಖಂಡದಲ್ಲೆಲ್ಲಾ ಹರಡಿ, ಅನಂತರ ಇತರ ಭಾವನೆಗಳು ಅವನ್ನು ಅನುಸರಿಸುವುವು. ಆಧ್ಯಾತ್ಮಿಕ ಮತ್ತು ಆಧ್ಯಾತ್ಮಿಕ ವಿದ್ಯೆಯ ದಾನವೇ ಸರ್ವಶ್ರೇಷ್ಠವಾದುದು. ಅವು ಜೀವಿಯನ್ನು ಹಲವು ಜನ್ಮಗಳಿಂದ ಪಾರುಮಾಡುವುವು. ಎರಡನೆಯದೆ ಲೌಕಿಕ ವಿದ್ಯೆ. ಅದು ಆಧ್ಯಾತ್ಮಿಕ ವಿದ್ಯೆಯ ಕಡೆಗೆ ನಮ್ಮನ್ನು ಒಯ್ಯುವುದು. ಅನಂತರವೇ ಪ್ರಾಣ ದಾನ; ನಾಲ್ಕನೆಯದೆ ಅನ್ನದಾನ.

ಸಾಧನೆ ಮಾಡುವಾಗ ದೇಹ ಹೋದರೂ ಚಿಂತೆಯಿಲ್ಲ. ಆದರಿಂದ ಏನು? ಸಾಧುಸಂಗದಲ್ಲಿದ್ದರೆ ಸಮಯ ಸನ್ನಿಹಿತವಾದಾಗ ಆತ್ಮಸಾಕ್ಷಾತ್ಕಾರ ಬಂದೇ ಬರುವುದು. ಒಬ್ಬನಿಗೆ ತಂಬಾಕಿನ ಚಿಲುಮೆಯನ್ನು ಸಿದ್ಧಪಡಿಸುವುದು ಕೂಡ ಧ್ಯಾನಕ್ಕಿಂತ ಲಕ್ಷಪಾಲು ಅಧಿಕ ಎಂದು ಗೊತ್ತಾಗುವ ಸಮಯವೂ ಬರುತ್ತದೆ. ಯಾರು ಸರಿಯಾಗಿ ತಂಬಾಕಿನ ಚಿಲುಮೆಯನ್ನು ಸಿದ್ಧಮಾಡಬಲ್ಲರೊ ಅವರು ಸರಿಯಾಗಿ ಧ್ಯಾನವನ್ನೂ ಮಾಡಬಲ್ಲರು. ದೇವತೆಗಳು ಮತ್ತಾರೂ ಅಲ್ಲ, ತುಂಬಾ ಹಿಂದೆಯೇ ಕಾಲವಾದ, ಮುಂದುವರಿದ, ಮಾನವರು ಅಷ್ಟೆ.

ಎಲ್ಲರೂ ಆಚಾರ್ಯರಾಗಲಾರರು. ಹಲವರು ಬೇಕಾದರೆ ಮುಕ್ತ ಜೀವಿ ಗಳಾಗಬಹುದು. ಮುಕ್ತನಿಗೆ ಇಡೀ ಪ್ರಪಂಚ ಒಂದು ಕನಸಿನಂತೆ ಕಾಣಿಸುವುದು. ಆದರೆ ಆಚಾರ್ಯರು ಜಾಗ್ರತ್​ ಸ್ವಪ್ನಗಳ ಮಧ್ಯದಲ್ಲಿ ನಿಲ್ಲಬೇಕಾಗಿದೆ. ಈ ಪ್ರಪಂಚ ನಿಜ ಎಂದು ಅವನು ತಿಳಿದಿರಬೇಕು. ಇಲ್ಲದೇ ಇದ್ದರೆ ಬೋಧಿಸುವುದು ಯಾರಿಗೆ? ಆದರೆ ಅವರು ಪ್ರಪಂಚವನ್ನು ಮಿಥ್ಯ ಎಂತಲೂ ತಿಳಿದಿರಬೇಕು. ಇಲ್ಲದೆ ಇದ್ದರೆ ಅವರು ಸಾಧಾರಣ ಮನುಷ್ಯನಿಗಿಂತ ಹೇಗೆ ಮೇಲು? ಅವರು ಬೋಧಿಸು ವುದಾದರೂ ಏನು? ಗುರುವು ಶಿಷ್ಯನ ಪಾಪದ ಹೊರೆಯನ್ನು ಹೊರ ಬೇಕಾಗಿದೆ. ಆದಕಾರಣವೆ ದೃಢಕಾಯರಾದ ಆಚಾರ್ಯರ ದೇಹಕ್ಕೂ ಕೂಡ ಹಲವು ರೋಗ ಗಳು ತಾಕುವುವು. ಅವರು ಅಪೂರ್ಣರಾದರೆ ಅವರ ಮನಸ್ಸಿನ ಮೇಲೂ ಅವು ಧಾಳಿ ಮಾಡುವುವು. ಅನಂತರ ಅವರು ಪತಿತರಾಗುವರು. ಆದಕಾರಣವೇ ಆಚಾರ್ಯನಾಗುವುದು ಬಹಳ ಕಷ್ಟ.

ಆಚಾರ್ಯನಾಗುವುದಕ್ಕಿಂತ ಜೀವನ್ಮುಕ್ತನಾಗುವುದು ಸುಲಭ. ಜೀವನ್ಮುಕ್ತನಿಗೆ ಪ್ರಪಂಚ ಒಂದು ಸ್ವಪ್ನವೆಂದು ಗೊತ್ತಿದೆ. ಅವನು ಇದರೊಂದಿಗೆ ಯಾವ ವ್ಯವಹಾರ ವನ್ನೂ ಇಟ್ಟುಕೊಳ್ಳುವುದಿಲ್ಲ. ಆಚಾರ್ಯನಿಗೆ ಇದು ಕನಸು ಎಂದು ಗೊತ್ತು. ಆದರೂ ಅವನು ಈ ಪ್ರಪಂಚದಲ್ಲಿದ್ದು ಕೆಲಸ ಮಾಡಬೇಕಾಗಿದೆ. ಎಲ್ಲರೂ ಆಚಾರ್ಯರಾಗಲು ಸಾಧ್ಯವಿಲ್ಲ. ಯಾರ ಮೂಲಕ ಭಗವಂತನ ಶಕ್ತಿ ಕೆಲಸ ಮಾಡುವುದೋ ಅವನೇ ಆಚಾರ್ಯ. ಆಚಾರ್ಯನ ದೇಹ ಇತರರ ದೇಹದಂತೆ ಅಲ್ಲ. ಆ ದೇಹವನ್ನು ಸರಿಯಾಗಿ ಇಟ್ಟಿರಬೇಕಾದರೆ ಒಂದು ಕ್ರಮ ಇದೆ. ಅವರದು ಅತಿ ಸೂಕ್ಷ್ಮವಾದ ದೇಹ, ಅತಿ ನಿಕಟ ಆನಂದವನ್ನು ಮತ್ತು ತೀವ್ರ ದುಃಖವನ್ನು ಅವರು ಅನುಭವಿಸಬಲ್ಲರು. ಅವರು ಅಸಾಧಾರಣ ಮನುಷ್ಯರು.

ಜೀವನದ ಪ್ರತಿಯೊಂದು ಕಾರ್ಯಕ್ಷೇತ್ರದಲ್ಲಿಯೂ ಆಂತರ್ಯದಲ್ಲಿರುವ ವ್ಯಕ್ತಿಯೇ ಜಯಿಸುವವನು. ಆ ವ್ಯಕ್ತಿತ್ವವೇ ಎಲ್ಲಾ ಜಯಕ್ಕೂ ಮೂಲ.

ನದಿಯಾದ ಮಹಾತ್ಮನಾದ ಶ‍್ರೀಕೃಷ್ಣಚೈತನ್ಯನಲ್ಲಿ ವ್ಯಕ್ತವಾದಷ್ಟು ಮಹಾಭಾವ ಮತ್ತೆಲ್ಲಿಯೂ ವ್ಯಕ್ತವಾಗಿಲ್ಲ.

ಶ‍್ರೀರಾಮಕೃಷ್ಣರು ಒಂದು ಶಕ್ತಿ. ಇದನ್ನೊ ಅದನ್ನೊ ಅವರ ಸಿದ್ದಾಂತ ಎಂದು ನೀವು ಭಾವಿಸಬಾರದು. ಅವರೊಂದು ಶಕ್ತಿ. ಈಗಲೂ ಅವರು ಅವರ ಶಿಷ್ಯರಲ್ಲಿ ಜೀವಿ ಸುತ್ತಿರುವರು, ಪ್ರಪಂಚದಲ್ಲಿ ಕೆಲಸಮಾಡುತ್ತಿರುವರು. ಅವರ ಭಾವನೆ ಬೆಳೆಯುತ್ತಿರುವುದನ್ನು ನಾನು ನೋಡುತ್ತಿರುವೆ.ಅವರು ಇನ್ನೂ ಬೆಳೆಯುತ್ತಿರುವರು. ಶ‍್ರೀರಾಮಕೃಷ್ಣರು ಜೀವನ್ಮುಕ್ತರು, ಜೊತೆಗೆ ಆಚಾರ್ಯರೂ ಆಗಿದ್ದರು.

