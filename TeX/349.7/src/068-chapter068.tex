
\chapter[ಪ್ರಾನಾಯಾಮ ]{ಪ್ರಾನಾಯಾಮ \protect\footnote{\engfoot{C.W. Vol. VIII, p.192}}}

ಮೊದಲು ಪ್ರಾಣಾಯಾಮವೆಂದರೆ ಏನು ಎಂಬುದನ್ನು ಸ್ವಲ್ಪ ತಿಳಿದು ಕೊಳ್ಳೋಣ. ತಾತ್ತ್ವಿಕವಾಗಿ ಪ್ರಾಣವೆಂದರೆ ಇಡಿ ವಿಶ್ವದಲ್ಲಿರುವ ಶಕ್ತಿಯ ಮೊತ್ತ. ತಾತ್ತ್ವಿಕ ದೃಷ್ಟಿಯಲ್ಲಿ ಈ ಜಗತ್ತು ಅಲೆಗಳೋಪಾದಿಯಲ್ಲಿ ಒಮ್ಮೆ ಏಳುತ್ತದೆ, ಮತ್ತೆ ಲಯವಾಗುತ್ತದೆ. ಇಡೀ ವಿಶ್ವ ದ್ರವ್ಯ ಮತ್ತು ಪ್ರಾಣಗಳಿಂದ ಆಗಿದೆ. ದ್ರವ್ಯ ಘನ ವಾಗಿರಲಿ, ಅನಿಲವಾಗಿರಲಿ, ದ್ರವವಾಗಿರಲಿ ಎಲ್ಲವೂ ಆಕಾಶ ಎನ್ನುವ ಮೂಲ ದ್ರವ್ಯದಿಂದ ಉತ್ಪನ್ನವಾಗಿದೆ ಎನ್ನುವರು, ಭಾರತೀಯ ತತ್ತ್ವವೇತ್ತರು. ನಾವು ಪ್ರಕೃತಿ ಯಲ್ಲಿ ಕಾಣುವ ಎಲ್ಲಾ ಶಕ್ತಿಗಳೂ ಯಾವ ಮೂಲಶಕ್ತಿಯ ಅಭಿವ್ಯಕ್ತಿಯೋ ಅದನ್ನು ಆ ತಾತ್ತ್ವಿಕರು ಪ್ರಾಣ ಎನ್ನುವರು. ಈ ಪ್ರಾಣವು ಆಕಾಶದ ಮೇಲೆ ತನ್ನ ಪ್ರಭಾವ ವನ್ನು ಬೀರಿದಾಗ ಸೃಷ್ಟಿಯಾಗುವುದು. ಒಂದು ಕಲ್ಪವಾದ ಮೇಲೆ ಸಷ್ಟಿಯೂ ತಾಟಸ್ಥ್ಯದ ಸ್ಥಿತಿಗೆ ಬರುವುದು. ಸೃಷ್ಟಿಯ ಒಂದು ಕಲ್ಪದ ನಂತರ ಪ್ರಳಯವು ಬರು ವುದು. ಪ್ರತಿಯೊಂದರ ಸ್ವಭಾವವೂ ಇದೇ. ಅನಂತರ ಮತ್ತೆ ವೈವಿಧ್ಯ ಮಯ ವಾದ ಈ ಕ್ರಿಯೆ ನಡೆಯುತ್ತಲೇ ಹೋಗುತ್ತದೆ. ಜಗತ್ತಿನಲ್ಲಿ ಪ್ರಳಯ ಸ್ಥಿತಿ ಬಂದಾಗ ಪೃಥ್ವಿ ಸೂರ್ಯ ಚಂದ್ರ ತಾರಾವಳಿಗಳ ಆಕಾರಗಳೆಲ್ಲ ಕರಗಿ ಹೋಗಿ ಆಕಾಶವಾಗುವುದು. ಆಕಾಶದಲ್ಲಿ ಎಲ್ಲಾ ಲೀನವಾಗಿ ಹೋಗುವುದು.ದೇಹದಲ್ಲಿ ಮತ್ತು ಮನಸ್ಸಿನಲ್ಲಿರುವ ಶಕ್ತಿಗಳೆಲ್ಲ, ಆಕರ್ಷಣ, ಚಲನೆ, ಆಲೋಚನೆ ಎಲ್ಲವೂ ಮೂಲ ಪ್ರಾಣಕ್ಕೇ ಹೋಗುವುವು. ಹೀಗೆ ಇದು ಸ್ಪಂದಿಸುತ್ತಾ ಹೋಗುತ್ತದೆ. ಇದರಿಂದ ಪ್ರಾಣಾಯಾಮದ ಪ್ರಾಮುಖ್ಯವನ್ನು ಅರ್ಥಮಾಡಿಕೊಳ್ಳಬಹುದು. ಹೇಗೆ ಆಕಾಶ ನಮ್ಮನ್ನು ಎಲ್ಲಾ ಕಡೆಗಳಲ್ಲಿಯೂ ಆವರಿಸಿಕೊಂಡಿರುವುದೋ ಅದರಿಂದ ನಾವು ಓತಪ್ರೋತವಾಗಿರುವೆವೋ, ಹಾಗೆಯೇ ಜಗತ್ತಿನಲ್ಲಿರುವ ಪ್ರತಿಯೊಂದನ್ನೂ ಆಕಾಶವು ಆವರಿಸಿಕೊಂಡು ಅದರಲ್ಲಿ ಓತಪ್ರೋತವಾಗಿರುವುದು. ಪ್ರತಿಯೊಂದೂ ಸರೋವರದ ಮೇಲೆ ತೇಲುತ್ತಿರುವ ನೀರ್ಗಲ್ಲಿನಂತೆ ಇದೆ. ನೀರ್ಗಲ್ಲು ಸರೋವರದ ನೀರಿನಿಂದ ಉಂಟಾಗಿ ನೀರಿನ ಮೇಲೆಯೇ ತೇಲುತ್ತದೆ. ಅದರಂತೆಯೇ ಪ್ರತಿಯೊಂದು ವಸ್ತುವೂ ಆಕಾಶವೆಂಬ ಮಹಾಸಾಗರದ ಮೇಲೆ ತೇಲುತ್ತದೆ. ಇದರಂತೆಯೇ ಈ ಪ್ರಾಣವೆಂಬ ಮಹಾಸಾಗರವೂ ನಮ್ಮನ್ನು ಆವರಿಸಿಕೊಂಡಿರುವುದು. ನಾವು ಉಸಿರಾಡುವುದು ಈ ಪ್ರಾಣದ ಮೂಲಕ, ನಮ್ಮ ರಕ್ತಚಲನೆಯಾಗುವುದು ಈ ಪ್ರಾಣದ ಮೂಲಕ; ಇದೇ ಶಕ್ತಿಯು ನಮ್ಮ ನರಗಳ, ಮಾಂಸಖಂಡಗಳ ಮತ್ತು ಮಿದುಳಿನ ಆಲೋಚನೆಯ ಹಿಂದೆ ಇದೆ. ಹೇಗೆ ಎಲ್ಲಾ ದ್ರವ್ಯಗಳು ಆಕಾಶದ ವಿವಿಧ ಆವಿರ್ಭಾವಗಳೋ ಹಾಗೆಯೇ ಎಲ್ಲಾ ಶಕ್ತಿಗಳೂ ಪ್ರಾಣದ ಆವಿರ್ಭಾವಗಳು. ಸ್ಥೂಲದ ಕಾರಣವು ನಮಗೆ ಯಾವಾ ಗಲೂ ಅದರ ಸೂಕ್ಷ್ಮದಲ್ಲಿ ದೊರಕುವುದು. ರಸಾಯನಶಾಸ್ತ್ರಜ್ಞ ಒಂದು ಹಿಡಿ ಅದುರನ್ನು ತೆಗೆದುಕೊಂಡು ಅದನ್ನು ವಿಶ್ಲೇಷಣೆ ಮಾಡುವನು. ಯಾವ ಸೂಕ್ಷ್ಮ ವಸ್ತುವಿನಿಂದ ಈ ಸ್ಥೂಲವು ಉಂಟಾಗಿದೆ ಎಂಬುದನ್ನು ಕಂಡುಹಿಡಿಯಲು ಯತ್ನಿಸುವನು. ಇದರಂತೆಯೇ ನಮ್ಮ ಆಲೋಚನೆ ಮತ್ತು ಜ್ಞಾನ ಕೂಡ. ಸ್ಥೂಲಕ್ಕೆ ವಿವರಣೆ ನಮಗೆ ಸೂಕ್ಷ್ಮದಲ್ಲಿ ಸಿಕ್ಕುವುದು. ಪರಿಣಾಮವೇ ಸ್ಥೂಲ, ಕಾರಣವೇ ಸೂಕ್ಷ್ಮ. ನಾವು ನೋಡುವ, ಅನುಭವಿಸುವ, ಸ್ಪರ್ಶಮಾಡುವ ಸ್ಥೂಲ ಪ್ರಪಂಚಕ್ಕೆ ಕಾರಣ ಮತ್ತು ವಿವರಣೆ ಸೂಕ್ಷ್ಮವಾದ ಆಲೋಚನೆಯಲ್ಲಿದೆ ಈ ಆಲೋಚನೆಯ ಕಾರಣ ಮತ್ತು ವಿವರಣೆ ಅದಕ್ಕೂ ಹಿಂದೆ ಇದೆ. ಇದರಂತೆಯೇ ಈ ನಮ್ಮ ದೇಹದಲ್ಲಿ ಮೊದಲು ಸ್ಥೂಲ ಚಲನೆಯನ್ನು ನೋಡುತ್ತೇವೆ. ಅದೇ ಕೈ, ಕಾಲು, ತುಟಿ ಇವುಗಳ ಚಲನೆಗಳು. ಆದರೆ ಇವಕ್ಕೆ ಕಾರಣ ಎಲ್ಲಿದೆ? ಅವೇ ಸೂಕ್ಷ್ಮವಾಗಿರುವ ನರಗಳು. ಅವುಗಳ ಚಲನೆಯನ್ನು ನಮಗೆ ಗ್ರಹಿಸುವುದಕ್ಕೆ ಆಗುವುದಿಲ್ಲ. ನಮ್ಮ ಇಂದ್ರಿಯಗಳ ಮೂಲಕ ಅವುಗಳ ಚಲನವಲನಗಳನ್ನು ನೋಡುವುದಕ್ಕೆ ಆಗುವುದಿಲ್ಲ.ಅವು ಅಷ್ಟು ಸೂಕ್ಷ್ಮವಾಗಿವೆ. ಆದರೂ ಅವೇ ಸ್ಥೂಲ ಚಲನೆಗಳಿಗೆ ಕಾರಣ ಎಂದು ನಮಗೆ ಗೊತ್ತಿದೆ. ಈ ನರಗಳ ಚಲನೆ ಅದಕ್ಕಿಂತಲೂ ಸೂಕ್ಷ್ಮವಾದ ಮತ್ತಾವುದರಿಂದಲೋ ಆಗುವುದು. ಅದನ್ನೇ ನಾವು ಆಲೋಚನೆ ಎನ್ನುವುದು. ಆ ಆಲೋಚನೆಯ ಚಲನೆಗೆ ಅದಕ್ಕಿಂತಲೂ ಸೂಕ್ಷ್ಮವಾದ ಮತ್ತೊಂದು ಕಾರಣ ಇರುವುದು. ಅದೇ ಆತ್ಮ. ನಮ್ಮನ್ನು ನಾವು ತಿಳಿದುಕೊಳ್ಳ ಬೇಕಾದರೆ ನಮ್ಮ ಇಂದ್ರಿಯಗ್ರಹಣವನ್ನು ಬಹಳ ಸೂಕ್ಷ್ಮ ಮಾಡಬೇಕು. ಯಾವ ಸೂಕ್ಷ್ಮದರ್ಶಕ ಯಂತ್ರವಾಗಲಿ ಅಥವಾ ಮತ್ತಾವ ಅತಿ ನಾಜೂಕಾದ ಉಪ ಕರಣವಾಗಲಿ ಮನಸ್ಸಿನಲ್ಲಿ ಆಗುತ್ತಿರುವುದನ್ನು ಕಂಡು ಹಿಡಿಯಲಾರದು. ಅದನ್ನು ತಿಳಿದುಕೊಳ್ಳುವಂತಹ ಯಂತ್ರವನ್ನು ಇದುವರೆಗೆ ಯಾರೂ ಕಂಡು ಹಿಡಿದಿಲ್ಲ. ತನ್ನ ಮನಸ್ಸನ್ನು ಪರೀಕ್ಷೆಮಾಡಲು ಯೋಗಿಯು ತಾನೇ ಒಂದು ಉಪಕರಣವನ್ನು ತಯಾರುಮಾಡುವನು. ಆ ಉಪಕರಣ ಅವನ ಮನಸ್ಸಿನಲ್ಲಿಯೇ ಇದೆ. ಯಾವ ಉಪಕರಣವೂ ಕಂಡುಹಿಡಿಯಲು ಸಾಧ್ಯವಿಲ್ಲದ ಅತಿ ಸೂಕ್ಷ್ಮ ವಿಷಯಗಳನ್ನು ಗ್ರಹಿಸುವುದು ಮನಸ್ಸಿಗೆ ಸಾಧ್ಯವಾಗುವುದು.

ಈ ಅತಿ ಸೂಕ್ಷ್ಮವನ್ನು ನಾವು ಗ್ರಹಿಸಬೇಕಾದರೆ ನಾವು ಸ್ಥೂಲದಿಂದ ಪ್ರಾರಂಭ ಮಾಡಬೇಕು. ನಮ್ಮ ಶಕ್ತಿ ಸೂಕ್ಷ್ಮ ಸೂಕ್ಷ್ಮವಾಗುತ್ತಾ ಬಂದಂತೆಲ್ಲಾ ನಾವು ನಮ್ಮ ಸ್ವಭಾವದ ಆಳ ಆಳಕ್ಕೆ ಹೋಗುವೆವು. ಮೊದಲು ಸ್ಥೂಲ ಚಲನವಲನಗಳೆಲ್ಲ ನಮಗೆ ಚೆನ್ನಾಗಿ ಕಾಣಿಸುವುವು. ಅನಂತರ ಆಲೋಚನೆಯ ಸೂಕ್ಷ್ಮ ಚಲನವಲನಗಳು ಗೊತ್ತಾಗುವುವು. ಆಲೋಚನೆ ಹೇಗೆ ಮೊದಲಾಯಿತು, ಅದು ಎಲ್ಲಿಗೆ ಹೋಗುತ್ತಿದೆ ಮತ್ತು ಅದು ಎಲ್ಲಿ ಪರ್ಯವಸಾನವಾಗುವುದು ಎಂಬುದನ್ನೆಲ್ಲಾ ಕಂಡುಹಿಡಿಯ ಬಹುದು. ಸಾಮಾನ್ಯ ಮನಸ್ಸಿಗೆ ಆಲೋಚನೆ ಹೇಗೆ ಪ್ರಾರಂಭವಾಯಿತು, ಹೇಗೆ ಕೊನೆಗೊಂಡಿತು ಎಂಬುದು ಗೊತ್ತಾಗುವುದಿಲ್ಲ. ಮನಸ್ಸು ಒಂದು ಸಾಗರದಂತೆ. ಅಲ್ಲಿ ಒಂದು ಅಲೆ ಏಳುವುದು. ನಾವು ಅಲೆಯನ್ನು ನೋಡಿದರೂ ಅದು ಹೇಗೆ ಬಂತು, ಎಲ್ಲಿ ಹುಟ್ಟಿತು, ಎಲ್ಲಿ ಕೊನೆಗೊಳ್ಳುವುದು–ಇದಾವುದೂ ನಮಗೆ ಗೊತ್ತಾಗು ವುದಿಲ್ಲ. ಆದರೆ ನಮ್ಮ ಗ್ರಹಣಶಕ್ತಿ ಸೂಕ್ಷ್ಮವಾದಂತೆ ಆಲೋಚನೆಯ ಅಲೆಯ ಆದಿ ಅಂತ್ಯಗಳೆಲ್ಲವೂ ಅದು ಏಳುವುದಕ್ಕೆ ಬಹಳ ಮುಂಚೆಯೇ ನಮಗೆ ಗೊತ್ತಾಗು ವುದು. ಅದು ಕಣ್ಮರೆಯಾದ ಮೇಲೆಯೂ ಅದು ಪ್ರಯಾಣ ಮಾಡಿದ ದೀರ್ಘವಾದ ಪಥವೂ ನಮಗೆ ಗೊತ್ತಾಗುತ್ತದೆ. ಅನಂತರ ನಮಗೆ ನಿಜವಾಗಿ ಮನಶ್ಶಾಸ್ತ್ರದ ಸ್ವರೂಪವು ತಿಳಿಯುತ್ತದೆ. ಈಗಿನ ಕಾಲದಲ್ಲಿ ಜನರು ಮನಶ್ಶಾಸ್ತ್ರದ ಮೇಲೆ ಏನು ಏನೋ ಪುಸ್ತಗಳನ್ನು ಬರೆಯುತ್ತಿರು ವರು, ಅವೆಲ್ಲಾ ನಮ್ಮನ್ನು ದಾರಿತಪ್ಪಿಸು ವಂಥವು. ಏಕೆಂದರೆ, ತಮ್ಮ ಮನಸ್ಸನ್ನು ವಿಶ್ಲೇಷಣೆ ಮಾಡುವುದು ಅವರಿಗೆ ಗೊತ್ತಿಲ್ಲ. ತಮಗೆ ಗೊತ್ತಿಲ್ಲದ, ಆದರೆ ಆಗಲೇ ಅದನ್ನು ಒಂದು ಸಿದ್ಧಾಂತ ಮಾಡಿದ ವಿಷಯವನ್ನು ಕುರಿತು ಅವರು ಹೇಳುತ್ತಿರುವರು. ಎಲ್ಲ ವಿಜ್ಞಾನವೂ ಸತ್ಯಾಂಶದ ಮೇಲೆ ನಿಂತಿರಬೇಕು. ಅದನ್ನು ನಾವು ಕಂಡಿರಬೇಕು; ಸರಿಯಾಗಿ ಜೋಡಿಸಿರಬೇಕು. ಸಾಮಾನ್ಯೀಕರಿಸುವುದಕ್ಕೆ ಸತ್ಯಾಂಶಗಳೆ ಇಲ್ಲದೇ ಇದ್ದರೆ ನೀವು ಮಾಡುವುದೇನು? ಹೀಗಾಗಿ ಸಾಮಾನ್ಯೀಕರಿಸುವುದರ ಕಡೆಗೆ ನಾವು ಮಾಡುವ ಪ್ರಯತ್ನಗಳೆಲ್ಲ ನಾವು ಸಾಮಾನ್ಯೀಕರಿಸುವ ವಿಷಯಗಳ ಜ್ಞಾನವನ್ನು ಅವಲಂಬಿಸಿದೆ. ಒಬ್ಬನು ಒಂದು ಸಿದ್ಧಾಂತವನ್ನು ಮಂಡಿಸುತ್ತಾನೆ. ಸಿದ್ಧಾಂತಕ್ಕೆ ಸಿದ್ಧಾಂತವನ್ನು ಸೇರಿಸುವನು. ಕೊನೆಗೆ ಇಡೀ ಗ್ರಂಥವು ತಯಾರಾಗುವುದು. ಆ ಸಿದ್ಧಾಂತಗಳಲ್ಲಿ ಯಾವುದಕ್ಕೂ ಸರಿಯಾದ ಅರ್ಥವಿರುವುದಿಲ್ಲ. ಮೊದಲು ನಿಮ್ಮ ಮನಸ್ಸಿಗೆ ಸಂಬಂಧಪಟ್ಟ ವಿಷಯಗಳನ್ನು ಆರಿಸಬೇಕು ಎಂದು ರಾಜಯೋಗವು ಹೇಳುತ್ತದೆ. ಅದನ್ನು ನಮ್ಮ ಮನಸ್ಸಿನ ವಿಶ್ಲೇಷಣೆಯಿಂದ ಮಾತ್ರ ಮಾಡಬಹುದು. ಇದಕ್ಕಾಗಿ ಮನಸ್ಸಿನ ಸೂಕ್ಷ್ಮಗ್ರಹಣ ಶಕ್ತಿಗಳನ್ನು ಅಭಿವೃದ್ಧಿಗೊಳಿಸಬೇಕು. ನಿಮ್ಮ ಮನಸ್ಸಿನಲ್ಲಿ ಏನಾಗುತ್ತಿದೆ ಎಂಬುದನ್ನು ನೀವೇ ನೋಡಬೇಕು. ಈ ವಿಷಯಗಳೆಲ್ಲಾ ಗೊತ್ತಾದಮೇಲೆ ಅವನ್ನು ಸಾಮಾ ನ್ಯೀಕರಿಸಿರಿ. ಆಗ ಮಾತ್ರ ನಿಜವಾದ ಮನಶ್ಶಾಸ್ತ್ರವು ಸಿದ್ಧವಾಗುತ್ತದೆ.

ಯಾವುದಾದರೂ ಸೂಕ್ಷ್ಮವನ್ನು ಗ್ರಹಿಸಬೇಕಾದರೆ ಅದರ ಸ್ಥೂಲ ಸ್ವರೂಪವು ನಮಗೆ ಗೊತ್ತಿರಬೇಕು. ಸೂಕ್ಷ್ಮಗ್ರಹಣಬೇಕಾದರೆ ಅದರ ಸ್ಥೂಲವು ನಮ್ಮ ವಶವಾಗಿರಬೇಕು. ಹೊರಗೆ ವ್ಯಕ್ತವಾಗುತ್ತಿರುವುದು ಸ್ಥೂಲ. ಇದನ್ನು ನಾವು ವಶ ಪಡಿಸಿಕೊಂಡು ಮುಂದುವರಿದರೆ ಸೂಕ್ಷ್ಮ, ಸೂಕ್ಷ್ಮತರ, ಸೂಕ್ಷ್ಮತಮ ಗ್ರಹಣ ಶಕ್ತಿಗಳು ನಮ್ಮ ವಶವಾಗುವುವು. ಈ ದೇಹಕ್ಕೂ ಮತ್ತು ಈ ದೇಹದಲ್ಲಿರುವ ಪ್ರತಿಯೊಂದಕ್ಕೂ ಬೇರೆ ಬೇರೆ ಅಸ್ತಿತ್ವವಿಲ್ಲ. ಸೂಕ್ಷ್ಮದಿಂದ ಸ್ಥೂಲಕ್ಕೆ ಇರುವ ಸರಪಳಿಯಲ್ಲಿ ಅವು ಹಲವು ಕೊಂಡಿಗಳು. ನೀವು ಪೂರ್ಣ, ಈ ದೇಹವು ಹೊರಗೆ ಕಾಣಿಸುತ್ತಿರುವ ಕರಟ ಇದ್ದಂತೆ, ಒಳಗೆ ಇರುವುದು ತಿರುಳು. ಹೊರಗಿನದು ಸ್ಥೂಲ, ಒಳಗಿನದು ಸೂಕ್ಷ್ಮ: ಹೀಗೆ ಆತ್ಮನವರೆಗೆ ಬರುವವರೆಗೆ ಸೂಕ್ಷ್ಮ ಸೂಕ್ಷ್ಮವಾಗುತ್ತಾ ಬರುವುದು. ಕೊನೆಗೆ ನಾವು ಆತ್ಮನ ಬಳಿಗೆ ಬಂದಾಗ ಇವುಗಳೆಲ್ಲದರಂತೆ ಕಾಣಿಸುತ್ತಿದ್ದುದು ಆತ್ಮನೇ, ಆತ್ಮನಲ್ಲದೆ ಬೇರಾವುದೂ ಇರಲೇ ಇಲ್ಲ, ಎಲ್ಲವೂ ಆತ್ಮನ ಸ್ಥೂಲ ಮತ್ತು ಸೂಕ್ಷ್ಮ ಅಭಿವ್ಯಕ್ತಿ ಎಂಬುದು ಗೊತ್ತಾಗುವುದು. ಇತರ ಎಲ್ಲ ವಸ್ತುಗಳೂ ಆ ಆತ್ಮದ ವಿವಿಧ ಪ್ರಮಾಣದ ಸ್ಥೂಲರೂಪದ ಅಭಿವ್ಯಕ್ತಿಗಳು. ಆತ್ಮವೇ ಮನಸ್ಸೂ ದೇಹವೂ ಆಗಿರುವುದು. ಈ ಉಪಮಾನದ ಮೂಲಕ ನಮಗೆ ಹೊರಗೆ ಒಂದು ಸ್ಥೂಲಪ್ರಪಂಚ ಇದೆ, ಅದರ ಹಿಂದೆ ಸೂಕ್ಷ್ಮ ಚಲನೆಯೊಂದು ಇದೆ, ಅದನ್ನೇ ಭಗವದಿಚ್ಛೆ ಎನ್ನುವುದು ಎಂಬುದು ಗೊತ್ತಾಗುತ್ತದೆ. ಆ ಇಚ್ಛೆಯ ಹಿಂದೆ ವಿಶ್ವಾತ್ಮನಿರುವನು. ವಿಶ್ವಾತ್ಮನೇ ದೇವರಾಗುತ್ತಾನೆ ಮತ್ತು ವಿಶ್ವವಾಗು ತ್ತಾನೆ ಎಂಬುದು ಆಗ ಗೊತ್ತಾಗುವುದು. ವಿಶ್ವಾತ್ಮ, ದೇವರು, ವಿಶ್ವ ಎಂಬ ಇವು ಬೇರೆ ಬೇರೆ ಅಲ್ಲ; ಇವು ಮೂರೂ ಒಂದೇ ವಸ್ತುವಿನ ಬೇರೆ ಬೇರೆ ಆವಿರ್ಭಾವದ ಸ್ಥಿತಿಗಳು ಎಂದು ಆಗ ಗೊತ್ತಾಗುವುದು.

ಇವುಗಳೆಲ್ಲಾ ಪ್ರಾಣಾಯಾಮದಿಂದ ಬರುವುವು. ದೇಹದ ಒಳಗೆ ಆಗುತ್ತಿರುವ ಸೂಕ್ಷ್ಮ ಚಲನವಲನಗಳೆಲ್ಲ ಉಸಿರಾಟಕ್ಕೆ ಸಂಬಂಧಪಟ್ಟಿವೆ. ನಾವು ಉಸಿರಾಡು ವುದನ್ನು ನಿಗ್ರಹಿಸುತ್ತಾ ಬಂದರೆ ಕ್ರಮೇಣ ಸೂಕ್ಷ್ಮ ಸೂಕ್ಷ್ಮತರವಾದ ಚಲನೆಗಳ ಮೇಲೆ ನಮಗೆ ಒಂದು ಹತೋಟಿ ಬರುವುದು. ಹೀಗೆ ಉಸಿರಿನ ನಿಯಂತ್ರಣದಿಂದ ಮನಸ್ಸಿನ ವಲಯಗಳನ್ನು ಪ್ರವೇಶಿಸಬಹುದು.

ಹಿಂದಿನ ಸಲ ನಾನು ನಿಮಗೆ ಹೇಳಿದುದು ಪ್ರಾಣಾಯಾಮದ ತಾತ್ಕಾಲಿಕವಾದ ಒಂದು ಸುಲಭವಾದ ಅಭ್ಯಾಸವನ್ನು ಅಷ್ಟೆ. ಈ ಅಭ್ಯಾಸಗಳಲ್ಲಿ ಕೆಲವು ಬಹಳ ಕಷ್ಟ. ಹಾಗೆ ಕಷ್ಟವಾಗಿರುವುದನ್ನು ನಾನು ಹೇಳುವುದಕ್ಕೆ ಪ್ರಯತ್ನಿಸುವುದಿಲ್ಲ. ಕಷ್ಟವಾಗಿರುವುದನ್ನು ಅಭ್ಯಾಸಮಾಡಬೇಕಾದರೆ ಆಹಾರದ ಮೇಲಿನ ನಿರ್ಬಂಧ ಮತ್ತು ಇನ್ನು ಕೆಲವು ನಿರ್ಬಂಧಗಳು ಬೇಕಾಗುವುವು. ಅದು ನಿಮ್ಮಲ್ಲಿ ಅನೇಕರಿಗೆ ಸಾಧ್ಯವಿಲ್ಲ. ಆದಕಾರಣ ಸುಲಭವಾದ, ನಿಧಾನವಾಗಿ ಮಾಡಬಹುದಾದ ಕೆಲವು ಸಾಧನೆಗಳನ್ನುತೆಗೆದುಕೊಳ್ಳೋಣ.

ಉಸಿರಾಡುವುದರಲ್ಲಿ ಮೂರು ಭಾಗಗಳಿವೆ. ಮೊದಲನೆಯದು ಉಸಿರನ್ನು ಸೆಳೆದುಕೊಳ್ಳುವುದು, ಇದೇ ಪೂರಕ; ಎರಡನೆಯದು ಉಸಿರನ್ನು ಒಳಗೆ ಕಟ್ಟುವುದು, ಇದೇ ಕುಂಭಕ, ಮೂರನೆಯದು ಉಸಿರನ್ನು ಹೊರಗೆ ಬಿಡುವುದು, ಇದೇ ರೇಚಕ. ನಾನು ಇಂದು ಹೇಳಿಕೊಡುವ ಮೊದಲನೆಯ ಸಾಧನೆಯೇ ಉಸಿರನ್ನು ಒಳಗೆ ಸೆಳೆದು ಕೊಂಡು ಸ್ವಲ್ಪ ಹೊತ್ತು ಒಳಗೆ ನಿಲ್ಲಿಸಿ ಅನಂತರ ನಿಧಾನವಾಗಿ ಹೊರಗೆ ಅದನ್ನು ಬಿಡುವುದು. ಉಸಿರಾಡುವುದರಲ್ಲಿ ಮತ್ತೊಂದನ್ನು ಗಮನಿಸಬೇಕಾಗಿದೆ. ಅದನ್ನು ಇಂದು ಹೇಳುವುದಿಲ್ಲ. ಇವುಗಳನ್ನು ನೀವು ಜ್ಞಾಪಕದಲ್ಲಿಡಲಾರಿರಿ; ಅದು ಬಹಳ ಕಷ್ಟವಾಗುವುದು. ಇವು ಮೂರೂ ಸೇರಿ ಒಂದು ಪ್ರಾಣಾಯಾಮವಾಗುವುದು. ಉಸಿರಾಡುವುದನ್ನು ಒಂದು ಕ್ರಮಕ್ಕೆ ತರಬೇಕು. ಇಲ್ಲದೆ ಇದ್ದರೆ ಅಪಾಯವಿದೆ. ಇವುಗಳನ್ನು ಒಂದು ಸಂಖ್ಯೆಯ ಮೂಲಕ ಕ್ರಮಕ್ಕೆ ತರಬೇಕು. ಈಗ ಕನಿಷ್ಠವಾಗಿರುವ ಸಂಖ್ಯೆಯನ್ನು ಕೊಡುತ್ತೇನೆ. ನಾಲ್ಕು ಸೆಕೆಂಡುಗಳವರೆಗೆ ಉಸಿರನ್ನು ತೆಗೆದುಕೊಳ್ಳಿ. ಎಂಡು ಸೆಕೆಂಡುಗಳವರೆಗೆ ಉಸಿರನ್ನು ಒಳಗೆ ಇಟ್ಟುಕೊಳ್ಳಿ. ಅನಂತರ ನಾಲ್ಕು ಸೆಕೆಂಡುಗಳಲ್ಲಿ ಉಸಿರನ್ನು ಹೊರಗೆ ಬಿಡಿ. ಮತ್ತೆ ಹೀಗೆ ಮಾಡಿ. ಹೀಗೆ ಬೆಳಿಗ್ಗೆ ನಾಲ್ಕು ವೇಳೆ, ಸಾಯಂಕಾಲ ನಾಲ್ಕು ವೇಳೆ ಮಾಡಿ. ಮತ್ತೊಂದು ಇದೆ. ಒಂದು ಎರಡು ಮೂರು ಎಣಿಸುವುದಕ್ಕಿಂತ ಯಾವುದಾದರೂ ಮಂತ್ರವನ್ನು ಉಚ್ಚರಿಸುವುದು ಮೇಲು. ನಮ್ಮ ದೇಶದಲ್ಲಿ “ಓಂ” ಎಂಬ ಮಂತ್ರವಿದೆ. “ಓಂ” ಎಂದರೆ ದೇವರು. ನೀವು ಅದನ್ನು ಉಚ್ಚರಿಸಿದರೆ ಒಂದು ಎರಡು ಮೂರು ಎನ್ನುವುದಕ್ಕಿಂತ ಮೇಲು. ಉಸಿರನ್ನು ಮೂಗಿನ ಎಡಹೊಳ್ಳೆಯಿಂದ ತೆಗೆದುಕೊಂಡು ಬಲಹೊಳ್ಳೆಯ ಮೂಲಕ ಬಿಡಬೇಕು. ಅನಂತರ ಬಲಹೊಳ್ಳೆಯಿಂದ ಉಸಿರನ್ನು ಸೆಳೆದುಕೊಂಡು ಎಡಹೊಳ್ಳೆ ಯಲ್ಲಿ ಬಿಡಬೇಕು. ಕೇವಲ ಇಚ್ಛಾಶಕ್ತಿಯ ಮೂಲಕ ನೀವು ಎಡ ಅಥವಾ ಬಲಹೊಳ್ಳೆ ಯಲ್ಲಿ ಉಸಿರಾಡುವುದು ಸಾಧ್ಯವಾಗಬೇಕು. ಕೆಲವು ಕಾಲದ ಮೇಲೆ ಅದು ನಿಮಗೆ ಸಾಧ್ಯವಾಗುವುದು. ಆದರೆ ಸದ್ಯಕ್ಕೆ ನಿಮಗೆ ಆ ಶಕ್ತಿ ಇಲ್ಲ ಎಂದು ಕಾಣಿಸುವುದು. ಆದಕಾರಣ ಒಂದು ಹೊಳ್ಳೆಯಲ್ಲಿ ಉಸಿರಾಡುವಾಗ ಮತ್ತೊಂದು ಹೊಳ್ಳೆಯನ್ನು ಬೆರಳಿನಿಂದ ಮುಚ್ಚಿಕೊಳ್ಳಬೇಕು. ಉಸಿರನ್ನು ಬಿಗಿಹಿಡಿಯುವಾಗ ಎರಡು ಹೊಳ್ಳೆ ಗಳನ್ನೂ ಮುಚ್ಚಿಕೊಳ್ಳಬೇಕು.

ಮೊದಲ ಎರಡು ಪಾಠಗಳನ್ನು ಮರೆಯಕೂಡದು. ಮೊದಲನೆಯದೇ ನೇರವಾಗಿ ಕುಳಿತುಕೊಳ್ಳುವುದು. ಎರಡನೆಯದು ದೇಹ ಆರೋಗ್ಯವಾಗಿ ದೃಢವಾಗಿದೆ ಎಂದು ಭಾವಿಸುವುದು. ಅನಂತರ ಸುತ್ತಲೂ ಪ್ರೇಮಪ್ರವಾಹವನ್ನು ಹೊರಹರಿಸಿ. ಇಡೀ ಜಗತ್ತು ಆನಂದದಲ್ಲಿದೆ ಎಂದು ಭಾವಿಸಿ. ಅನಂತರ ದೇವರಲ್ಲಿ ನಿಮಗೆ ನಂಬಿಕೆ ಇದ್ದರೆ, ಪ್ರಾರ್ಥಿಸಿ ಪ್ರಾಣಾಯಾಮವನ್ನು ಪ್ರಾರಂಭ ಮಾಡಿ.

ನಿಮ್ಮಲ್ಲಿ ಅನೇಕರಿಗೆ ದೈಹಿಕವಾದ ಬದಲಾವಣೆಗಳೂ ಬರುತ್ತವೆ. ದೇಹಾದ್ಯಂತ ಒಂದು ಬಗೆಯ ಸೆಳೆತ ಉಂಟಾಗುತ್ತದೆ; ಕೆಲವು ಸಲ ನರಗಳ ತುಡಿತ ಉಂಟಾಗುತ್ತದೆ. ಕೆಲವು ವೇಳೆ ಅಳುವಂತೆ ಆಗುವುದು, ಕೆಲವು ವೇಳೆ ಯಾವುದೋ ಪ್ರಚಂಡ ಚಲನೆ ಎದ್ದಂತೆ ಆಗುವುದು. ಇದಕ್ಕೆಲ್ಲ ಅಂಜಬೇಕಾಗಿಲ್ಲ. ನೀವು ಮುಂದು ವರಿದಂತೆಲ್ಲ ಇವು ಆಗಬೇಕಾಗಿರುವುವು. ಇಡಿ ದೇಹವನ್ನು ನಾವು ಪುನಃ ಬೇರೆ ರೀತಿಯಲ್ಲಿ ರಚಿಸಬೇಕಾಗಿದೆ. ಮಿದುಳಿನಲ್ಲಿ ಆಲೋಚನೆ ಹರಿಯುವುದಕ್ಕೆ ಬೇರೆ ಮಾರ್ಗಗಳನ್ನು ಮಾಡಬೇಕಾಗಿದೆ. ಇದುವರೆವಿಗೂ ನಿಮ್ಮ ಇಡೀ ಜೀವನದಲ್ಲಿ ಕೆಲಸಮಾಡದ ಕೆಲವು ನರಗಳು ಇನ್ನು ಮೇಲೆ ಕೆಲಸಮಾಡುವುದಕ್ಕೆ ಪ್ರಾರಂಭಿಸುವುವು. ದೇಹದಲ್ಲಿ ಬದಲಾವಣೆಗಳ ಸರಣಿಯೇ ಉಂಟಾಗುತ್ತದೆ.

