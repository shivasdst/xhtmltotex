
\chapter[ಭಗವತ್ಪ್ರೇಮ – ೧ ]{ಭಗವತ್ಪ್ರೇಮ – ೧ \protect\footnote{\engfoot{C.W. Vol. VIII, P. 200}}}

\centerline{(೧೮೯೩, ಸೆಪ್ಟೆಂಬರ್​ ೨೫ರ ‘ಚಿಕಾಗೊ ಹೆರಾಲ್ಡ್​’ ನಲ್ಲಿ ಪ್ರಕಟವಾದ ಉಪನ್ಯಾಸದ ವರದಿ)}

ಮೂರನೆ ಯೂನಿಟೇರಿಯನ್​ ಚರ್ಚಿನ ಸಭಾಂಗಣದಲ್ಲಿ ತುಂಬಿದ್ದ ಶ್ರೋತೃ ಗಳು ನಿನ್ನೆ ಬೆಳಗ್ಗೆ ಸ್ವಾಮಿ ವಿವೇಕಾಂದರ ಉಪನ್ಯಾಸವನ್ನು ಕೇಳಿದರು. ಉಪನ್ಯಾಸದ ವಿಷಯ ಭಗವತ್ಪ್ರೇಮ. ಅವರು ವಿಷಯವನ್ನು ಅತ್ಯಂತ ಪ್ರಭಾವಶಾಲಿಯಾಗಿ ವಿಶಿಷ್ಟ ರೀತಿಯಲ್ಲಿ ನಿರೂಪಿಸಿದರು. ಜಗತ್ತಿನ ಎಲ್ಲ ಭಾಗಗಳಲ್ಲಿಯೂ ದೇವರು ವಿವಿಧ ಹೆಸರುಗಳಲ್ಲಿ ಮತ್ತು ವಿವಿಧ ರೀತಿಯಲ್ಲಿ ಪೂಜಿಸಲ್ಪಡುತ್ತಿರುವನು ಎಂದರು ಉಪನ್ಯಾಸಕರು. ಭವ್ಯವಾದುದನ್ನು ಮತ್ತು ಸುಂದರವಾದುದನ್ನು ಪೂಜಿಸುವುದು ಮಾನವನಿಗೆ ಸ್ವಾಭಾವಿಕವಾದುದು ಮತ್ತು ಧರ್ಮವು ಜನರ ಸ್ವಭಾವದ ಒಂದು ಅಂಶವೇ ಆಗಿದೆ. ಎಲ್ಲರೂ ಭಗವಂತನ ಆವಶ್ಯಕತೆಯನ್ನು ಭಾವಿಸುತ್ತಾರೆ, ಮತ್ತು ಭಗವತ್ಪ್ರೇಮದಿಂದ ಪ್ರೇರಿತರಾಗಿಯೇ ಅವರು ದಾನಧರ್ಮಾದಿಗಳನ್ನು ಮಾಡುತ್ತಾರೆ ಮತ್ತು ನ್ಯಾಯನಿಷ್ಠರಾಗಿರುತ್ತಾರೆ. ಜನರೆಲ್ಲರೂ ಭಗವಂತನನ್ನು ಪ್ರೀತಿಸುವುದಕ್ಕೆ ಕಾರಣ ಅವನು ಪ್ರೇಮಸ್ವರೂಪನಾಗಿರುವುದು. ಚಿಕಾಗೊಗೆ ಬಂದಂದಿನಿಂದ ಮಾನವ ಸಹೋದರತ್ವದ ಬಗ್ಗೆ ತಾವು ಬಹಳ ಕೇಳಿರುವುದಾಗಿ ಉಪನ್ಯಾಸಕರು ಹೇಳಿದರು. ಎಲ್ಲರೂ ಭಗವತ್ಪ್ರೇಮದ ಸಂತಾನರೆಂದು ಭಾವಿಸಿದಾಗ ಸಹೋದರತ್ವಕ್ಕಿಂತಲೂ ಶಕ್ತಿಶಾಲಿಯಾದ ಸಂಬಂಧವು ಸಾಧ್ಯ ಎಂದರು. ಮಾನವ ಸಹೋದರತ್ವದ ಮೂಲ ಭಿತ್ತಿ, ಭಗವಂತನು ಎಲ್ಲರ ತಂದೆ ಎಂಬುದು. ಅವರು ಹೇಳಿದರು: “ನಾನು ಭರತಖಂಡದ ಕಾಡಿನಲ್ಲಿ ಅಲೆದಿರುವೆನು, ಗುಹೆಗಳಲ್ಲಿ ಮಲಗಿರುವೆನು, ಮತ್ತು ಅನುಭವವು ಬೋಧಿಸಿರುವುದೇನೆಂದರೆ–ಪ್ರಕೃತಿ ನಿಯಮವನ್ನು ಮೀರಿದ ಯಾವುದೋ ಒಂದು ಜನರನ್ನು ತಪ್ಪು ಹೆಜ್ಜೆ ಇಡದಂತೆ ಮಾಡುತ್ತಿದೆ–ಅದೇ ಭಗವತ್ಪ್ರೇಮ. ಭಗವಂತನು ಕ್ರಿಸ್ತ, ಮಹಮ್ಮದ್​ ಮತ್ತು ವೈದಿಕ ಋಷಿ ಗಳೊಡನೆ ಮಾತನಾಡಿರುವುದು ಹೌದಾದರೆ ಅವನ ಮಕ್ಕಳಲ್ಲಿ ಒಬ್ಬನಾದ ನನ್ನೊಡನೆಯೂ ಅವನೇಕೆ ಮಾತನಾಡಬಾರದು?” ನಿಜವಾಗಿಯೂ ಅವನು ಮಾತನಾಡುತ್ತಾನೆ. ಅಷ್ಟೇ ಅಲ್ಲ, ತನ್ನೆಲ್ಲ ಮಕ್ಕಳೊಡನೆಯೂ ಮಾತನಾಡುತ್ತಾನೆ. ನಮ್ಮ ಸುತ್ತಲೂ ನಾವು ಅವನನ್ನು ನೋಡುತ್ತೇವೆ, ಅವನ ಅಸೀಮ ಪ್ರೇಮ ನಿರಂತರವಾಗಿ ನಮ್ಮ ಮೇಲೆ ಪ್ರಭಾವ ಬೀರುತ್ತಿದೆ ಮತ್ತು ಆ ಪ್ರೇಮದ ಸ್ಫೂರ್ತಿಯಿಂದಲೇ ನಾವು ಉತ್ತಮ ಜೀವನವನ್ನು ನಡೆಸುತ್ತಿರುವುದು, ಉತ್ತಮ ಕಾರ್ಯವನ್ನು ಎಸಗುತ್ತಿರುವುದು.

