
\chapter[ಭಾಗವದ್ಗೀತಾ (೧) ]{ಭಗವದ್ಗೀತಾ (೧) \protect\footnote{\engfoot{C.W. Vol. I, P. 446}}}

\centerline{\textbf{(೧೯೦೦ರ ಮೇ ೨೬ರಂದು ಸ್ಯಾನ್​ಫ್ರಾನ್ಸಿಸ್ಕೋದಲ್ಲಿ ನೀಡಿದ ಉಪನ್ಯಾಸ)}}

ನಾವು ಗೀತೆಯನ್ನು ತಿಳಿದುಕೊಳ್ಳಬೇಕಾದರೆ, ಮೊದಲು ಅದರ ಚಾರಿತ್ರಿಕ ಹಿನ್ನೆಲೆಯನ್ನು ತಿಳಿದುಕೊಳ್ಳಬೇಕು. ಭಗವದ್ಗೀತೆ ಉಪನಿಷತ್ತಿಗೆ ಒಂದು ಭಾಷ್ಯ. ಉಪನಿಷತ್ತುಗಳು ಭಾರತ ದೇಶದ ಬೈಬಲ್ಲು. ನ್ಯೂಟೆಸ್ಟಮೆಂಟ್​ ಇಲ್ಲಿ (ಅಮೇರಿಕಾದಲ್ಲಿ) ಯಾವ ಸ್ಥಾನವನ್ನು ಪಡೆದಿರುವುದೊ ಅದರಂತೆಯೇ ಉಪನಿಷತ್ತುಗಳು. ಸುಮಾರು ನೂರಕ್ಕು ಹೆಚ್ಚು\break ಉಪನಿಷತ್ತುಗಳಿವೆ. ಕೆಲವು ಬಹಳ ಸಣ್ಣವು ಮತ್ತೆ ಕೆಲವು ಬಹಳ ದೊಡ್ಡವು. ಆದರೆ ಪ್ರತಿಯೊಂದೂ ಬೇರೆ ಬೇರೆ ಪ್ರಕರಣ. ಉಪನಿಷತ್ತುಗಳು ಯಾವ ಗುರುವಿನ ಜೀವನವನ್ನೂ ಕುರಿತು ಹೇಳುವುದಿಲ್ಲ. ಆದರೆ ಸುಮ್ಮನೆ ತತ್ತ್ವವನ್ನು ಮಾತ್ರ ಸಾರುತ್ತವೆ. ಅವು ಸಾಮಾನ್ಯವಾಗಿ ರಾಜಾಸ್ಥಾನಗಳಲ್ಲಿ ನಡೆಯುತ್ತಿದ್ದ ಪಂಡಿತ ಗೋಷ್ಠಿಗಳ ಚರ್ಚೆಯ ಶೀಘ್ರಲಿಪಿ ಟಿಪ್ಪಣಿಗಳಂತೆ ಇವೆ. ಉಪನಿಷತ್ತು ಎಂಬ ಪದಕ್ಕೆ ಕುಳಿತುಕೊಳ್ಳುವುದು (ಅಥವಾ ಗುರುವಿನ ಹತ್ತಿರ ಕುಳಿತುಕೊಳ್ಳುವುದು) ಎಂದು ಅರ್ಥ. ನಿಮ್ಮಲ್ಲಿ ಯಾರಾದರೂ ಕೆಲವು ಉಪನಿಷತ್ತುಗಳನ್ನು ಓದಿದರೆ ಅವು ಶೀಘ್ರಲಿಪಿಯಿಂದ ಸಂಗ್ರಹಿಸಿದ ಟಿಪ್ಪಣಿಗಳಂತೆ ಇವೆ ಎನ್ನುವುದು\break ವೇದ್ಯವಾಗುವುದು. ಬೇಕಾದಷ್ಟು ಚರ್ಚೆ ಆದಮೇಲೆ ಬಹುಶಃ ಅವುಗಳನ್ನೆಲ್ಲಾ ಜ್ಞಾಪಕದಿಂದ ಬರೆದಿಟ್ಟಿರಬಹುದು. ಇದರಲ್ಲಿರುವ ಒಂದು ತೊಂದರೆಯೆಂದರೆ, ನಮಗೆ ಅದರ ಹಿನ್ನೆಲೆ ಸ್ವಲ್ಪವೂ ಸಿಕ್ಕುವುದಿಲ್ಲ. ಬಹಳ ಮುಖ್ಯವಾದ ಭಾವನೆಗಳನ್ನು ಮಾತ್ರ ಅಲ್ಲಿ ಹೇಳುವರು. ಪುರಾತನ ಸಂಸ್ಕೃತ ಭಾಷೆಯು ಸುಮಾರು ಕ್ರಿ.ಪೂ. ಐದು ಸಾವಿರ ವರ್ಷಗಳಷ್ಟು ಹಿಂದಿನದು. ಉಪನಿಷತ್ತುಗಳು ಅದಕ್ಕೂ ಎರಡು ಸಾವಿರ ವರ್ಷಗಳು ಹಿಂದೆ ರಚಿತವಾಗಿರಬೇಕು. ಅವು ಎಷ್ಟು ಪುರಾತನವಾದವು ಎಂಬುದು ಖಚಿತವಾಗಿ ಯಾರಿಗೂ ಗೊತ್ತಿಲ್ಲ. ಗೀತೆ ಉಪನಿಷತ್ತುಗಳ ಭಾವನೆಯನ್ನು ಆಧರಿಸಿದ್ದು, ಕೆಲವು ವೇಳೆ ಅದೇ ಪದಗಳನ್ನು ಬಳಸುವುದು. ಉಪನಿಷತ್ತುಗಳು ಯಾವ ಯಾವ ವಿಷಯಗಳನ್ನು ಹೇಳುವುವೋ ಅವನ್ನೆಲ್ಲಾ ಗೀತೆ ಅಚ್ಚುಕಟ್ಟಾಗಿ ಸಂಗ್ರಹವಾಗಿ, ಕ್ರಮವಾಗಿ ಬೋಧಿಸಿರುವುದು.

ಹಿಂದೂಗಳು ತಮ್ಮ ಮೂಲಶಾಸ್ತ್ರವನ್ನು ವೇದಗಳು ಎಂದು ಕರೆಯುತ್ತಾರೆ. ಅವುಗಳು ವಿಸ್ತಾರವಾದ ಸಾಹಿತ್ಯರಾಶಿ. ಅವುಗಳ ಮೂಲ ಬರವಣಿಗೆಗಳನ್ನು ಮಾತ್ರ ತಂದರೂ ಈ ಕೋಣೆ ಅವುಗಳನ್ನು ಇಡಲು ಸಾಲುವುದಿಲ್ಲ. ಅವುಗಳಲ್ಲಿ ಎಷ್ಟೋ ನಾಶವಾಗಿ ಹೋಗಿವೆ. ಅದನ್ನು ಹಲವು ಶಾಖೆಗಳಾಗಿ ಮಾಡಿ ಪ್ರತಿಯೊಂದು ಶಾಖೆಯನ್ನು ರಕ್ಷಿಸುವುದಕ್ಕೂ ಕೆಲವು ಪುರೋಹಿತರ ಮುಖ್ಯಸ್ಥನ ವಶಕ್ಕೆ ಬಿಟ್ಟರು. ಅವರು ಅದನ್ನು ಕೇವಲ ನೆನಪಿನ ಬಲದಿಂದ ಉಳಿಸಿಕೊಂಡು ಬಂದರು. ಇಂತಹ ಜನ ಈಗಲೂ ಇರುವರು. ಅವರು ವೇದಗಳಲ್ಲಿ ಒಂದಾದ ಮೇಲೆ ಮತ್ತೊಂದನ್ನು ಕೇವಲ ನೆನಪಿನ ಸಹಾಯದಿಂದ, ಒಂದು ಸ್ವರವನ್ನೂ ಬಿಡದೆ ಹೇಳುವರು. ವೇದದ ಬಹುಭಾಗ ಈಗ ನಾಶವಾಗಿದೆ. ಆದರೆ ಕಳೆದು\break ಉಳಿದಿರುವುದೇ ಒಂದು ದೊಡ್ಡ ಪುಸ್ತಕ ಭಂಡಾರದಷ್ಟು ಇದೆ. ಅವುಗಳಲ್ಲಿ ಬಹಳ ಪುರಾತನವಾಗಿರುವುದು ಋಗ್​ ವೇದದ ಮಂತ್ರಗಳು. ಆಧುನಿಕ ವಿದ್ವಾಂಸರು ಅವುಗಳನ್ನು ಒಂದು ಕ್ರಮದಲ್ಲಿ ಜೋಡಿಸಬೇಕೆಂದು ಇರುವರು. ಆದರೆ ಹಳೆಯ ಸಂಪ್ರದಾಯವಂತರ ದೃಷ್ಟಿಯೇ ಬೇರೆ. ಬೈಬಲ್ಲಿನ ವಿಷಯದಲ್ಲಿ ಸಂಪ್ರದಾಯಸ್ಥರಿಗೂ ಆಧುನಿಕ ವಿದ್ವಾಂಸರಿಗೂ ವ್ಯತ್ಯಾಸ ಇರುವಂತೆಯೇ ಇದೂ ಕೂಡ. ವೇದಗಳನ್ನು ಎರಡು ಭಾಗ ಮಾಡಿರುವರು. ಒಂದು ಉಪನಿಷತ್ತು. ಇದು ತತ್ತ್ವಗಳ ಭಾಗವನ್ನು ಒಳಗೊಂಡಿರುವುದು. ಮತ್ತೊಂದು ಕರ್ಮಕಾಂಡ.

ಕರ್ಮಕಾಂಡದ ಕೆಲವು ಭಾವನೆಗಳನ್ನು ನಿಮಗೆ ವಿವರಿಸಲು ಯತ್ನಿಸುತ್ತೇನೆ. ಅಲ್ಲಿ ಹಲವು ಕ್ರಿಯಾವಿಧಿಗಳು ಮತ್ತು ಮಂತ್ರಗಳು ಇವೆ. ದೇವತೆಗಳನ್ನು ಉದ್ದೇಶಿಸಿದ ಹಲವು ಮಂತ್ರಗಳಿವೆ. ಕರ್ಮಕಾಂಡದಲ್ಲಿ ಬೇಕಾದಷ್ಟು ಯಾಗಯಜ್ಞಾದಿಗಳ ವಿಷಯಗಳಿವೆ. ಅದರಲ್ಲಿ ಕೆಲವು ತುಂಬಾ ವಿವರವಾಗಿವೆ. ಅವುಗಳನ್ನು ಮಾಡುವುದಕ್ಕೆ ಒಂದು ಪುರೋಹಿತರ ಸೇನೆಯೇ ಬೇಕಾಗುತ್ತದೆ. ಕರ್ಮಕಾಂಡ ಬಹಳ ವಿಸ್ತಾರವಾಗಿರುವುದರಿಂದ ಪುರೋಹಿತ ವೃತ್ತಿಯೇ ಒಂದು ಶಾಸ್ತ್ರವಾಯಿತು. ಮಂತ್ರಗಳು ಮತ್ತು ಯಾಗಯಜ್ಞಗಳನ್ನು ಜನ ಸಾಧಾರಣರು ಬಹಳ ಗೌರವದಿಂದ ನೋಡತೊಡಗಿದರು. ದೇವತೆಗಳು ಮಾಯವಾದರು. ಅವರ ಸ್ಥಾನದಲ್ಲಿ ಈ ಯಾಗಯಜ್ಞಗಳು ನಿಂತವು. ಭಾರತ ದೇಶದಲ್ಲಿ ನಡೆದ ಬಹಳ ವಿಚಿತ್ರವಾದ ಬೆಳವಣಿಗೆ ಇದು. ಮೀಮಾಂಸಕರಿಗೆ ದೇವತೆಗಳಲ್ಲಿ ನಂಬಿಕೆ ಇಲ್ಲ. ಇತರರು ದೇವರಲ್ಲಿ ನಂಬಿಕೆ ಉಳ್ಳವರು. ನೀವು ಮೀಮಾಂಸಕರನ್ನೇ ವೇದಗಳಲ್ಲಿ ಬರುವ ದೇವತೆಗಳ ಅರ್ಥವೇನು ಎಂದು ಕೇಳಿದರೆ, ಅವರು ಸಮರ್ಪಕವಾದ ಉತ್ತರವನ್ನು ಬಹುಶಃ ನಿಮಗೆ ಕೊಡಲಾರರು. ಪುರೋಹಿತರು ಈ ಮಂತ್ರಗಳನ್ನು ಹೇಳುತ್ತ ಅಗ್ನಿಕುಂಡಕ್ಕೆ ಹವಿಸ್ಸನ್ನು ಅರ್ಪಿಸು\-ವರು. ಮೀಮಾಂಸಕರನ್ನು ಈ ಕ್ರಿಯೆಗಳಿಗೆ ಅರ್ಥವೇನು ಎಂದು ಕೇಳಿದರೆ, ಶಬ್ದಗಳಿಗೆ ಕೆಲವು ಪರಿಣಾಮಗಳನ್ನು ಉಂಟುಮಾಡುವ ಶಕ್ತಿಯಿದೆ ಎಂದು ಹೇಳುತ್ತಾರೆ, ಅಷ್ಟೆ. ಶಬ್ದಗಳಿಗೆ ಪ್ರಾಕೃತಿಕ ಮತ್ತು ಅತಿ ಪ್ರಾಕೃತಿಕ ಶಕ್ತಿಗಳಿವೆ. ವೇದಗಳು ಎಂದರೆ ಶಬ್ದಗಳು. ಅವನ್ನು ಸರಿಯಾಗಿ ಉಚ್ಚರಿಸಿದರೆ ಪರಿಣಾಮವುಂಟು ಮಾಡಬಲ್ಲ ಯೌಗಿಕ ಶಕ್ತಿ ಅವುಗಳಲ್ಲಿವೆ. ಅವುಗಳನ್ನು ಸರಿಯಾಗಿ ಉಚ್ಚರಿಸದೇ ಇದ್ದರೆ ಪ್ರಯೋಜನವಿಲ್ಲ. ಪ್ರತಿಯೊಂದು ಶಬ್ದವನ್ನೂ ಸರಿಯಾಗಿ ಉಚ್ಚರಿಸಬೇಕು. ಬೇರೆ ಧರ್ಮಗಳಲ್ಲಿ ನಾವು ಯಾವುದನ್ನು ಪ್ರಾರ್ಥನೆ ಎನ್ನುತ್ತೇವೆಯೋ ಅದು ಮಾಯವಾಯಿತು. ವೇದಗಳೇ ದೇವತೆಗಳಾದವು. ಆದಕಾರಣವೇ ವೇದಗಳ ಶಬ್ದಗಳಿಗೆ ನಾವು ಅಷ್ಟು ಪ್ರಾಮುಖ್ಯ ಕೊಡುವುದು. ಈ ಶಬ್ದಗಳು ನಿತ್ಯವಾದವು. ಅವುಗಳಿಂದಲೇ ಜಗತ್ತು ಸೃಷ್ಟಿಯಾಗಿರುವುದು. ಶಬ್ದಗಳಿಲ್ಲದೆ ಯಾವ ಚಿಂತನೆಯೂ ಇರಲಾರದು. ಈ ಜಗತ್ತಿನಲ್ಲಿರುವುದೆಲ್ಲ ಆಲೋಚನೆಯ ಆವಿರ್ಭಾವ. ಶಬ್ದದ ಮೂಲಕ ಮಾತ್ರ ವ್ಯಕ್ತಪಡಿಸಬಹುದು. ಅವ್ಯಕ್ತವಾದ ಭಾವನೆಗಳು ಶಬ್ದಗಳ ಮೂಲಕ ವ್ಯಕ್ತವಾಗುತ್ತವೆ. ಅದನ್ನೇ ನಾವು ವೇದಗಳು ಎಂದು ಕರೆಯುವುದು. ಎಲ್ಲದರ ಅಸ್ತಿತ್ವವೂ ವೇದಗಳನ್ನು ಅವಲಂಬಿಸಿದೆ. ಏಕೆಂದರೆ ಶಬ್ದ ಇಲ್ಲದೆ ಇದ್ದರೆ ಚಿಂತನೆ ಇರಲಾರದು. ಕುದುರೆ ಎಂಬ ಶಬ್ದ ಇಲ್ಲದೆ ಇದ್ದರೆ ಯಾರೂ ಕುದುರೆಯನ್ನು ಕಲ್ಪಿಸಿಕೊಳ್ಳಲು ಸಾಧ್ಯವಾಗುತ್ತಿರಲಿಲ್ಲ. ಆದಕಾರಣ ಆಲೋಚನೆ, ಶಬ್ದ ಮತ್ತು ಬಾಹ್ಯವಸ್ತುಗಳ ನಡುವೆ ಒಂದು ನಿಕಟ ಸಂಬಂಧ ಇರಬೇಕಾಗುವುದು. ನಿಜವಾಗಿ ಈ ಶಬ್ದಗಳು ಏನು? ಅವೇ ವೇದಗಳು. ಇದನ್ನು ಅವರು ಸಂಸ್ಕೃತಭಾಷೆ ಎಂದು ಕರೆಯುವುದಿಲ್ಲ. ಅದನ್ನೇ ವೇದಭಾಷೆ, ದೇವಭಾಷೆ ಎನ್ನುವರು ಕೂಡ. ವೇದಭಾಷೆಗಿಂತ ಪುರಾತನವಾದುದು ಬೇರೊಂದು ಇಲ್ಲ. ವೇದಗಳನ್ನು ಯಾರು ಬರೆದರು ಎಂದು ನಾವು ಕೇಳಬಹುದು. ಅದನ್ನು ಯಾರೂ ಬರೆಯಲಿಲ್ಲ. ಶಬ್ದಗಳೇ ವೇದ. ನಾನು ಒಂದು ಶಬ್ದವನ್ನು ಸರಿಯಾಗಿ ಉಚ್ಚರಿಸಿದರೆ ಅದು ವೇದವಾಗುವುದು. ಅದರಿಂದ ತಕ್ಷಣ ಬೇಕಾದ ಪರಿಣಾಮ ದೊರಕುವುದು.

ಈ ವೇದರಾಶಿ ಯಾವಾಗಲೂ ಇರುವುದು. ಜಗತ್ತೆಲ್ಲ ಈ ಶಬ್ದರಾಶಿಯ ಆವಿರ್ಭಾವ. ಒಂದು ಯುಗವು ಕೊನೆಗೊಂಡಮೇಲೆ ಶಕ್ತಿಯ ಆವಿರ್ಭಾವಗಳು ಸೂಕ್ಷ್ಮಸೂಕ್ಷ್ಮವಾಗುವುವು. ಬರೀ ಶಬ್ದವಾಗುವುವು. ಆ ನಂತರ ಚಿಂತನೆಯಾಗುವುವು. ಮುಂದಿನ ಯುಗದಲ್ಲಿ ಮೊದಲು ಚಿಂತನೆಯು ಶಬ್ದರೂಪವನ್ನು ಧರಿಸುವುದು. ಆ ಶಬ್ದಗಳಿಂದ ಇಡೀ ಜಗತ್ತು ಸೃಷ್ಟಿಯಾಗುತ್ತದೆ. ವೇದಗಳಲ್ಲಿ ಇಲ್ಲದೇ ಇರುವುದು ಯಾವುದಾದರೂ ಇದ್ದರೆ ಅದು ನಿಮ್ಮ ಭ್ರಾಂತಿ, ಅದೆಂದಿಗೂ ಇಲ್ಲ.

ಈ ವಿಷಯದ ಮೇಲೆ ಇರುವ ಹಲವು ಗ್ರಂಥಗಳು ವೇದವನ್ನು ಸಮರ್ಥಿಸುವುವು. ಆ ಗ್ರಂಥಕರ್ತರಿಗೆ, ವೇದಗಳನ್ನು ಮೊದಲು ಜನರು ರಚಿಸಿದರು ಎಂದು ಹೇಳಿದರೆ\break ಅವರಿಗೆ ಅದು ಹಾಸ್ಯಾಸ್ಪದವಾಗಿ ತೋರುತ್ತದೆ. ಯಾವ ವ್ಯಕ್ತಿಯೂ ಅದನ್ನು ರಚಿಸಲಿಲ್ಲ. ಬುದ್ಧನ ಮಾತುಗಳನ್ನು ತೆಗೆದುಕೊಳ್ಳಿ. ಅವನು ಹಿಂದಿನ ಅನೇಕ ಜನ್ಮಗಳಲ್ಲಿ ಅದನ್ನೇ\break ಉಪದೇಶ ಮಾಡಿದ್ದ ಎಂಬ ಒಂದು ಸಂಪ್ರದಾಯವಿದೆ. ಕ್ರೈಸ್ತ ಧರ್ಮಾನುಯಾಯಿಗಳು ಎದ್ದು ನಿಂತು, ನಮ್ಮದು ಚಾರಿತ್ರಿಕ ಧರ್ಮ, ಆದಕಾರಣ ನಮ್ಮ ಧರ್ಮವೇ ಸರಿ,\break ನಿಮ್ಮದು ತಪ್ಪು ಎಂದರೆ ಮೀಮಾಂಸಕರು ಹೀಗೆ ಹೇಳುವರು: “ನಿಮ್ಮ ಧರ್ಮ\break ಚಾರಿತ್ರಿಕವಾಗಿರುವುದರಿಂದ, ಒಬ್ಬ ವ್ಯಕ್ತಿಯು ಅದನ್ನು ಸಾವಿರದ ಒಂಭೈನೂರು ವರುಷದ ಹಿಂದೆ ಕಂಡುಹಿಡಿದ ಎಂದು ನೀವೇ ಒಪ್ಪಿಕೊಂಡಂತಾಗುವುದು. ಆದರೆ ಯಾವುದು ಸತ್ಯವೋ ಅದು ಅನಂತವಾಗಿರಬೇಕು, ನಿತ್ಯವಾಗಿರಬೇಕು. ಇದೇ ಸತ್ಯದ ಏಕಮಾತ್ರ ಪ್ರಮಾಣ. ಅದೆಂದಿಗೂ ನಾಶವಾಗುವುದಿಲ್ಲ. ಅದು ಯಾವಾಗಲೂ ಒಂದೇ ಸಮನಾಗಿರುವುದು. ನಿಮ್ಮ ಧರ್ಮವನ್ನು ಯಾರೋ ಸೃಷ್ಟಿಸಿದರೆಂದು ನೀವೇ ಒಪ್ಪಿಕೊಳ್ಳುತ್ತೀರಿ.\break ವೇದಗಳನ್ನು ಯಾರೂ ಸೃಷ್ಟಿಸಲಿಲ್ಲ. ದೇವದೂತನಾದರೂ ಆಗಲಿ, ಮತ್ತಾರೇ ಆಗಲಿ ಅದನ್ನು ಸೃಷ್ಟಿಸಲಿಲ್ಲ. ಅದು ಬಂದಿರುವುದು ಅನಂತ ಶಬ್ದರಾಶಿಯಿಂದ, ಸ್ವಭಾವತಃ\break ಆದಿ ಅಂತ್ಯವಿಲ್ಲದೇ ಇದೆ. ಅದರಿಂದ ಈ ಜಗತ್ತು ಬರುವುದು ಮತ್ತು ಹೋಗುವುದು.” ಸೂಕ್ಷ್ಮದೃಷ್ಟಿಯಿಂದ ನೋಡಿದರೆ ಇದೇ ಸತ್ಯ. ಶಬ್ದವೇ ಸೃಷ್ಟಿಗೆ ಆದಿ. ಜೀವಾಣುಗಳು\break ಇರುವಂತೆಯೇ ಶಬ್ದಾಣುಗಳು ಇರಬೇಕು. ಶಬ್ದಗಳಿಲ್ಲದೆ ಯಾವ ಭಾವನೆಗಳೂ ಇರಲಾರವು. ಎಲ್ಲಿಯಾದರೂ ಇಂದ್ರಿಯ ಗ್ರಹಣವಿದ್ದರೆ, ಭಾವನೆಗಳಿದ್ದರೆ ಅಲ್ಲಿ ಶಬ್ದವಿರಬೇಕು. ಆದರೆ, ಈ ನಾಲ್ಕು ಗ್ರಂಥಗಳೇ ವೇದ, ಮತ್ತೆ ಯಾವುದೂ ಅಲ್ಲ ಎಂದಾಗ ತೊಂದರೆ ಬರುತ್ತದೆ. ಆಗ ಬೌದ್ಧರು ಎದ್ದು ನಿಂತು, ನಮ್ಮದು ಕೂಡ ವೇದವೇ; ಆನಂತರ ಅವು ನಮಗೆ ಗೋಚರಿಸಿವೆ ಎನ್ನುವರು. ಇದು ಸಾಧ್ಯವಿಲ್ಲ. ಪ್ರಕೃತಿ ಹೀಗೆ ಮಾಡುವುದಿಲ್ಲ. ಪ್ರಕೃತಿ ಸ್ವಲ್ಪಸ್ವಲ್ಪವಾಗಿ ತನ್ನ ನಿಯಮಗಳನ್ನು ತೋರುವುದಿಲ್ಲ – ಇವತ್ತು ಒಂದು ಅಂಗುಲ ಆಕರ್ಷಣ ಸಿದ್ಧಾಂತ ಆನಂತರ ಮತ್ತೊಂದು ಅಂಗುಲ ಹೀಗೆ. ಪ್ರತಿಯೊಂದು ನಿಯಮವೂ\break ಪೂರ್ಣವಾಗೇ ಇದೆ. ನಿಯಮ ಎಂದಿಗೂ ವಿಕಾಸವಾಗುವುದಿಲ್ಲ. ಅದು “ಹೊಸ ಧರ್ಮ ಉತ್ತಮವಾದ ಸ್ಛೂರ್ತಿ” ಎನ್ನುವುದೆಲ್ಲ ಕಾಡುಹರಟೆ, ಇದಕ್ಕೆ ಏನೂ ಅರ್ಥವಿಲ್ಲ. ಪ್ರಕೃತಿಯಲ್ಲಿ ಸಾವಿರಾರು ನಿಯಮಗಳು ಇರಬಹುದು. ಇಂದು ಮನುಷ್ಯನಿಗೆ ಎಲ್ಲೋ ಕೆಲವು ಮಾತ್ರ ಗೊತ್ತಿರಬಹುದು. ಕ್ರಮೇಣ ಅವುಗಳನ್ನು ನಾವು ಕಂಡು ಹಿಡಿಯುತ್ತೇವೆ ಅಷ್ಟೆ. ಶಬ್ದ ಅನಂತವಾದುದು ಎಂದು ಸಾರಿದ ಆ ಪುರಾತನ ಪುರೋಹಿತರು ದೇವತೆಗಳನ್ನು ಸಿಂಹಾಸನ ಚ್ಯುತರನ್ನಾಗಿ ಮಾಡಿ, ತಾವು ಆ ಸ್ಥಾನವನ್ನು ಆಕ್ರಮಿಸಿರುವರು. ಅವರು, “ನಿಮಗೆ ಆ ಮಂತ್ರಶಕ್ತಿ ಗೊತ್ತಿಲ್ಲ, ನಮಗೆ ಗೊತ್ತು, ನಾವೇ ಜಗತ್ತಿನ ಜೀವಂತ ದೇವತೆಗಳು. ನಮಗೆ ದಕ್ಷಿಣೆ ಕೊಡಿ. ನಾವು ಮಂತ್ರೋಚ್ಚಾರಣೆ ಮಾಡುತ್ತೇವೆ. ಇದರಿಂದ ನಿಮಗೆ ಬೇಕಾದುದು ದೊರಕುತ್ತದೆ. ನೀವೇ ಆ ಶಬ್ದಗಳನ್ನು ಉಚ್ಚಾರ ಮಾಡಬಲ್ಲಿರೆ? ನಿಮಗೆ ಅದು ಸಾಧ್ಯವಿಲ್ಲ. ಅದನ್ನು ಒಂದು ಸಲ ತಪ್ಪಾಗಿ ಹೇಳಿದರೂ ವಿರೋಧವಾಗಿರುವುದು ಆಗುತ್ತದೆ. ನೀವು ಶ‍್ರೀಮಂತರಾಗಲು, ಸುಂದರ ಪುರುಷರಾಗಲು ದೀರ್ಘಾಯುಗಳಾಗಲು, ಚೆನ್ನಾಗಿರುವ ಗಂಡನನ್ನು ಪಡೆಯಲು ಬಯಸುವಿರೇನು? ಹಾಗಾದರೆ ಪುರೋಹಿತನಿಗೆ ಸಾಕಷ್ಟು ದಕ್ಷಿಣೆ ಕೊಟ್ಟು ಸುಮ್ಮನಿರಿ, ಎಂದು ಹೇಳುವರು.”

ಆದರೂ ಇದಕ್ಕೆ ಮತ್ತೊಂದು ಮುಖವೂ ಇದೆ. ವೇದಗಳ ಮೊದಲನೆಯ ಭಾಗದ ಆದರ್ಶಕ್ಕೆ ಮತ್ತು ಎರಡನೆಯ ಭಾಗವಾದ ಉಪನಿಷತ್ತುಗಳ ಆದರ್ಶಕ್ಕೆ ಬೇಕಾದಷ್ಟು ವ್ಯತ್ಯಾಸವಿದೆ. ಮೊದಲನೆಯ ಭಾಗದ ಆದರ್ಶವು, ವೇದಾಂತವನ್ನು ಬಿಟ್ಟು, ಜಗತ್ತಿನ ಇತರ ಎಲ್ಲಾ ಧರ್ಮಗಳ ಆದರ್ಶದಂತೆಯೇ ಇದೆ. ಆ ಆದರ್ಶವೆಂದರೆ, ಇಹ ಮತ್ತು ಪರದಲ್ಲಿ ಸುಖವನ್ನು ಅನುಭವಿಸುವುದು. ಅದೇ ಗಂಡ, ಹೆಂಡತಿ, ಮಕ್ಕಳು ಮುಂತಾದವು. ನೀವು ದಕ್ಷಿಣೆ ಕೊಟ್ಟರೆ ಪುರೋಹಿತನು ನಿಮಗೆ ಸರ್ಟಿಫಿಕೇಟು ಕೊಡುವನು. ನೀವು ಆನಂತರ ಸ್ವರ್ಗದಲ್ಲಿ ಸುಖವಾಗಿರಬಹುದು. ಅಲ್ಲಿ ನಿಮ್ಮ ಜನರೆಲ್ಲರೂ ಇರುವರು. ಈ ಸುಖವು ರಂಕರಾಟಣೆಯಂತೆ \enginline{(merry–go–round)} ಎಂದೆಂದಿಗೂ ಸುತ್ತುತ್ತಿರುತ್ತದೆ. ಅದಕ್ಕೆ ಅಂತ್ಯವೆಂಬುದು ಇಲ್ಲ. ಕಣ್ಣೀರಿಲ್ಲ, ಅಳು ಇಲ್ಲ. ಸುಮ್ಮನೆ ನಗುತ್ತಿರುವುದೇ. ಹೊಟ್ಟೆನೋವು ಇಲ್ಲ. ತಿನ್ನುತ್ತಿರಬಹುದು. ತಲೆನೋವು ಇಲ್ಲ. ಎಷ್ಟು ಬೇಕಾದರೂ ಸುಖವನ್ನು\break ಅನುಭವಿಸುತ್ತಿರಬಹುದು. ಮಾನವನ ಪರಮ ಗುರಿಯೇ ಇದು ಎಂದು ಪುರೋಹಿತರು ಭಾವಿಸಿದರು.

ಈ ತತ್ತ್ವದಲ್ಲಿ ಮತ್ತೊಂದು ಭಾವನೆ ಇದೆ. ಅದು ನಿಮ್ಮ ಆಧುನಿಕ ಭಾವನೆಯನ್ನು ಹೋಲುವುದು. ಮನುಷ್ಯ ಎಂದೆಂದಿಗೂ ಪ್ರಕೃತಿಯ ಗುಲಾಮ. ಅವನು ಎಂದೆಂದಿಗೂ ಗುಲಾಮನಾಗಿಯೇ ಇರಬೇಕಾಗುವುದು. ನಾವು ಅದನ್ನು ಕರ್ಮ ಎನ್ನುತ್ತೇವೆ. ಕರ್ಮ ಎಂದರೆ ನಿಯಮ. ಅದು ಎಲ್ಲಾ ಕಡೆಯೂ ಇರಬೇಕು. ಪ್ರತಿಯೊಂದೂ ಕರ್ಮಕ್ಕೆ\break ಬದ್ಧವಾಗಿದೆ. ಇದರಿಂದ ಪಾರಾಗುವುದಕ್ಕೆ ದಾರಿ ಇಲ್ಲವೆ? “ಇಲ್ಲ! ಎಂದೆಂದಿಗೂ\break ಗುಲಾಮರಾಗಿರಿ. ಒಳ್ಳೆಯ ಗುಲಾಮರಾಗಿ. ನಾವು ಶಬ್ದವನ್ನು ಪ್ರಯೋಗಿಸಿ ನಿಮಗೆ ಸುಖ ಬರುವಂತೆ ಮಾಡುತ್ತೇವೆ. ದುಃಖ ಬರದಂತೆ ನೋಡಿಕೊಳ್ಳುತ್ತೇವೆ. ಅದಕ್ಕೆ ನೀವು ಸಂಭಾವನೆಯನ್ನು ಕೊಡಬೇಕಾಗುವುದು.” ಇದು ಮೀಮಾಂಸಕರ ಆದರ್ಶ. ಎಲ್ಲಾ ಕಾಲದಲ್ಲಿಯೂ ಸರ್ವ\-ಸಾಮಾನ್ಯವಾಗಿರುವ ಭಾವನೆಗಳು ಇವು. ಮಾನವಕೋಟಿಯ ಬಹುಸಂಖ್ಯಾತರು ಎಂದಿಗೂ ವಿಚಾರ ಮಾಡುವ ಗುಂಪಿಗೆ ಸೇರಿದವರಲ್ಲ. ಅವರು ಆಲೋಚಿಸುವುದಕ್ಕೆ ಪ್ರಯತ್ನಪಟ್ಟರೂ ಅವರ ಮೇಲೆ ಕವಿದಿರುವ ರಾಶಿರಾಶಿಯಾದ ಮೂಢನಂಬಿಕೆ ಅದಕ್ಕೆ\break ಅವಕಾಶವನ್ನು ಕೊಡುವುದಿಲ್ಲ. ಅವರು ಸ್ವಲ್ಪ ದುರ್ಬಲರಾದೊಡನೆ ಒಂದು ಪೆಟ್ಟು\break ಬೀಳುವುದು. ಅವರ ಬೆನ್ನುಮೂಳೆ ಚೂರುಚೂರಾಗುವುದು. ಆಸೆ ಅಂಜಿಕೆಗಳಿಂದ ಮಾತ್ರ ನೀವು ಅವರಿಂದ ಕೆಲಸ ಮಾಡಿಸಬಹುದು. ಅವರು ತಾವೇ ಸ್ವಂತವಾಗಿ ಏನೂ ಮಾಡುವುದಿಲ್ಲ. ಅವರನ್ನು ಚೆನ್ನಾಗಿ ಪೀಡಿಸಬೇಕು. ಅನಂತರ ಅವರು ನೀವು ಹೇಳಿದಂತೆ ಎಂದೆಂದಿಗೂ ಕೇಳುವರು. ಅವರಿಗೆ ಇನ್ನೇನೂ ಕೆಲಸವಿಲ್ಲ. ದಕ್ಷಿಣೆ ಕೊಡುವುದು, ಹೇಳಿದಂತೆ ಕೇಳುವುದು, ಇಷ್ಟೇ. ಉಳಿದವುಗಳನ್ನು ಪುರೋಹಿತರು ನೋಡಿಕೊಳ್ಳುತ್ತಾರೆ. ಧರ್ಮ ಎಷ್ಟು ಸುಲಭವಾಗುವುದು! ನೀವು ಅನಂತರ ಏನೂ ಮಾಡಬೇಕಾಗಿಲ್ಲ. ಮನೆಗೆ ಹೋಗಿ ಸುಖವಾಗಿರಿ. ಯಾರೋ ನಿಮಗೆ ಎಲ್ಲವನ್ನೂ ಕೊಡುತ್ತಿರುವರು. ಪಾಪ ಈ ಮೃಗಗಳು!

ಜೊತೆ ಜೊತೆಯಲ್ಲಿ ಬೇರೊಂದು ಸಿದ್ಧಾಂತ ಇದೆ. ಉಪನಿಷತ್ತುಗಳು ತಮ್ಮ ನಿರ್ಣಯಗಳಲ್ಲಿ ಇವುಗಳನ್ನೆಲ್ಲಾ ವಿರೋಧಿಸುತ್ತವೆ. ಮೊದಲನೆಯದಾಗಿ ಉಪನಿಷತ್ತುಗಳು\break ಸೃಷ್ಟಿಯನ್ನು ಮಾಡಿದ ಮತ್ತು ಅದನ್ನು ಆಳುತ್ತಿರುವ ದೇವರನ್ನು ನಂಬುತ್ತದೆ. ನೀವು ಅನಂತರ ದಯಾಮಯನಾದ ಭಗವಂತನ ಭಾವನೆಯನ್ನು ನೋಡುವಿರಿ. ಇದು ಸಂಪೂರ್ಣವಾಗಿ ವಿರೋಧವಾದ ಭಾವನೆ. ನಾವು ಪುರೋಹಿತರ ಮಾತನ್ನು ಕೇಳಿದರೂ ಆದರ್ಶ ಬಹಳ ಸೂಕ್ಷ್ಮವಾಗುತ್ತ ಬರುವುದು. ಹಲವು ದೇವರುಗಳ ಬದಲು ಒಂದೇ ದೇವರನ್ನು\break ಮಾಡಿದರು.

ನೀವೆಲ್ಲರೂ ಕರ್ಮ ನಿಯಮದಿಂದ ಬದ್ಧರು ಎಂಬ ಎರಡನೆಯ ಭಾವನೆಯನ್ನು ಉಪನಿಷತ್ತುಗಳು ಒಪ್ಪಿಕೊಳ್ಳುತ್ತವೆ. ಆದರೆ ಪಾರಾಗುವುದಕ್ಕೆ ಅವು ನಮಗೊಂದು ದಾರಿಯನ್ನು ತೋರುತ್ತವೆ. ಮಾನವ ಗುರಿಯೇ ನಿಯಮಾತೀತನಾಗುವುದು. ಭೋಗವೇ ಎಂದಿಗೂ ಪರಮ ಗುರಿಯಾಗಲಾರದು, ಏಕೆಂದರೆ ಭೋಗ ಇರುವುದು ಪ್ರಕೃತಿಯಲ್ಲಿ.

ಮೂರನೆಯದಾಗಿ ಉಪನಿಷತ್ತು ಎಲ್ಲಾ ಯಾಗಯಜ್ಞಗಳನ್ನು ಖಂಡಿಸುತ್ತವೆ. ಇವು ಕೆಲಸಕ್ಕೆ ಬಾರದವು ಎಂದು ಹೇಳುತ್ತವೆ. ಅದು ನಿಮಗೆ ಬೇಕಾದುದನ್ನೆಲ್ಲಾ ಕೊಡಬಹುದು. ಆದರೆ ಅದು ಯೋಗ್ಯವಲ್ಲ. ನಿಮಗೆ ಸಿಕ್ಕಿದಷ್ಟೂ ಬೇಡಿಕೆ ಜಾಸ್ತಿಯಾಗುವುದು. ನೀವು ಒಂದು ಗಾಣದ ಸುತ್ತ ಎಂದೆಂದಿಗೂ ಹೋಗುತ್ತಿರುವಿರೇ ವಿನಃ ಅದರಿಂದ ಪಾರಾಗಲಾರಿರಿ. ಭೋಗ ದುಃಖವಲ್ಲದೆ ಬೇರೆಯಲ್ಲ. ಅನಂತ ಸುಖ ಎಂಬುದು ಎಲ್ಲಿಯೂ ಸಾಧ್ಯವಿಲ್ಲ. ಇದೊಂದು ಮಕ್ಕಳ ಕನಸು. ಒಂದೇ ಶಕ್ತಿ ಸುಖವೂ ದುಃಖವೂ ಆಗುತ್ತದೆ.

ನನ್ನ ಮನಶ್ಶಾಸ್ತ್ರವನ್ನು ಇಂದು ಸ್ವಲ್ಪ ಬದಲಾಯಿಸಿದ್ದೇನೆ. ನಾನೊಂದು ವಿಚಿತ್ರವಾದ ವಿಷಯವನ್ನು ಕಂಡುಹಿಡಿದಿರುವೆನು. ನಿಮ್ಮಲ್ಲಿ ಯಾವುದೋ ಭಾವನೆ ಇದೆ. ಅದರಿಂದ ಪಾರಾಗಬೇಕೆಂದು ಇಚ್ಛಿಸುವಿರಿ. ನೀವು ಮತ್ತಾವುದೋ ಬೇರೆ ವಿಷಯವನ್ನು ಕುರಿತು\break ಚಿಂತಿಸುವಿರಿ. ನೀವು ಯಾವ ಭಾವನೆಯನ್ನು ಅದುಮಿಡಬೇಕೆಂದಿದ್ದೀರೋ ಅದು\break ಅದುಮಿಡಲ್ಪಡುತ್ತದೆ. ಅದು ಎಂತಹ ಭಾವನೆ? ಹದಿನೈದು ನಿಮಿಷಗಳಲ್ಲಿ ಅದು ಬಂದು ನನ್ನನ್ನು ಅಪ್ಪಳಿಸಿತು. ಅದು ಬಹಳ ಬಲವಾಗಿತ್ತು. ಅದು ಉಗ್ರವಾಗಿ ಪ್ರತಿಭಟಿಸುತ್ತ ಬಂದಿತು. ಆಗ ನಾನೊಬ್ಬ ಹುಚ್ಚನೆಂದು ಭಾವಿಸಿದೆ. ಅದಾದ ಮೇಲೆ ಏನಾಯಿತು?\break ಅದಕ್ಕಿಂತ ಹಿಂದೆ ಇದ್ದ ಭಾವನೆ ಸಂಪೂರ್ಣವಾಗಿ ಅಡಗಿಹೋಯಿತು. ಅದರಿಂದ ಏನು ಬಂತು? ಅದು ನಾನೇ ಅನುಭವಿಸಬೇಕಾದ ಕೆಟ್ಟ ಸಂಸ್ಕಾರವಾಗಿತ್ತು. “ಪ್ರಕೃತಿಗೆ ತನ್ನದೇ ಆದ ಒಂದು ನಿಯಮವಿದೆ. ನಿಗ್ರಹ ಅದನ್ನು ಏನು ಮಾಡಬಲ್ಲದು?” (ಗೀತೆ \enginline{III, 33}). ನಿಗ್ರಹ ಸಡಿಲವಾದೊಡನೆ ಅವುಗಳೆಲ್ಲ ಹೆಚ್ಚು ಮೇಲಕ್ಕೆ ಚಿಮ್ಮಿ ಬರುತ್ತವೆ.

ಆದರೆ ಭರವಸೆ ಇದೆ. ನೀವು ಬಲಾಢ್ಯರಾದರೆ ನಿಮ್ಮ ಪ್ರಜ್ಞೆಯನ್ನು ಏಕಕಾಲದಲ್ಲಿ ಇಪ್ಪತ್ತು ಭಾಗಗಳಾಗಿ ಮಾಡಬಹುದು. ನಾನು ನನ್ನ ಮನಶ್ಶಾಸ್ತ್ರವನ್ನು ಬದಲಾಯಿಸುತ್ತಿರುವೆನು. ಮನಸ್ಸು ಬೆಳೆಯುವುದು. ಯೋಗಿಗಳು ಹೇಳುವುದು ಇದನ್ನೇ. ಒಂದು\break ಆಸೆ ಇದೆ. ಅದು ಇನ್ನೊಂದನ್ನು ಕೆರಳಿಸುವುದು. ಹಿಂದಿನದು ಮಾಯವಾಗುವುದು.\break ನೀವು ಕೋಪಿಷ್ಠರಾಗಿದ್ದು ಅನಂತರ ಸಂತೋಷಪಟ್ಟರೆ, ತಕ್ಷಣವೇ ಕೋಪದ ಸ್ವಭಾವ\break ಮಾಯವಾಗುವುದು. ಆ ಕೋಪದಿಂದ ನೀವು ಮುಂದಿನ ಅವಸ್ಥೆಯನ್ನು ತಯಾರು\break ಮಾಡುವಿರಿ. ಈ ಸ್ಥಿತಿಗಳನ್ನು ಯಾವಾಗಲೂ ಸ್ಥಾನಾಂತರಗೊಳಿಸಬಹುದು. ಅನಂತ ಸುಖ ಮತ್ತು ಅನಂತ ದುಃಖ ಎಂಬವುಗಳು ಮಕ್ಕಳ ಕನಸು. ಆದರೆ ಮನುಷ್ಯನ ಗುರಿ ಸುಖವೂ ಅಲ್ಲ ದುಃಖವೂ ಅಲ್ಲ. ಯಾವುದರಿಂದ ಇವುಗಳು ತಯಾರಾಗುತ್ತವೆಯೋ ನಾವು ಅದಕ್ಕೆ ಒಡೆಯರಾಗಬೇಕಾಗಿದೆ ಎಂದು ಉಪನಿಷತ್ತುಗಳು ಹೇಳುತ್ತವೆ. ಪರಿಸ್ಥಿತಿಯ ಬೀಜರೂಪದಲ್ಲೇ ನಾವು ಅದನ್ನು ನಮ್ಮ ಸ್ವಾಧೀನಕ್ಕೆ ತೆಗೆದುಕೊಳ್ಳಬೇಕು.

ಮತ್ತೊಂದು ವ್ಯತ್ಯಾಸವೆಂದರೆ, ಉಪನಿಷತ್ತುಗಳು ಎಲ್ಲಾ ಯಾಗಯಜ್ಞಗಳನ್ನೂ\break ಖಂಡಿಸುತ್ತವೆ. ಅದರಲ್ಲಿಯೂ ಯಾವುವು ಪ್ರಾಣಿ ಬಲಿಯನ್ನು ಹೇಳುತ್ತವೆಯೋ ಅವನ್ನು ಒಪ್ಪಿಕೊಳ್ಳುವುದಿಲ್ಲ. ಅವುಗಳೆಲ್ಲ ಅರ್ಥವಿಲ್ಲದವು ಎಂದು ಸಾರುತ್ತವೆ. ಒಂದು ನಿರ್ದಿಷ್ಟ ಕಾಲದಲ್ಲಿ ಯಾವುದೋ ಒಂದು ಪ್ರಾಣಿಯನ್ನು ಕೊಲ್ಲಬೇಕು ಎಂದು ಹೇಳುವರು. ಅದಕ್ಕೆ ನೀವು ಪ್ರಾಣಿಹತ್ಯಾದೋಷ ಪ್ರಾಪ್ತವಾಗುವುದು ಎಂದು ಹೇಳಬಹುದು. ಅವರು ಅದಕ್ಕೆ ಅರ್ಥವಿಲ್ಲ ಎನ್ನುವರು. ನಿಮಗೆ ಏನು ಗೊತ್ತು ಯಾವುದು ಪುಣ್ಯ, ಯಾವುದು ಪಾಪ ಎಂಬುದು? ನಿಮ್ಮ ಮನಸ್ಸು ಹಾಗೆ ಹೇಳಬಹುದು. ನಿನ್ನ ಮನಸ್ಸು ಏನು ಹೇಳುವುದೋ ಲೆಕ್ಕಿಸುವವರಾರು? ನೀವು ಎಂತಹ ಅವಿವೇಕವನ್ನು ಮಾತನಾಡುತ್ತಿರುವಿರಿ. ಶಾಸ್ತ್ರಕ್ಕೆ\break ವಿರೋಧವಾಗಿ ಮಾತನಾಡುತ್ತಿರುವಿರಲ್ಲ? ನಿಮ್ಮ ಮನಸ್ಸು ಒಂದು ಹೇಳಿ, ವೇದ ಅದಕ್ಕೆ ವಿರೋಧವಾದುದನ್ನು ಹೇಳಿದರೆ, ನಿಮ್ಮ ಮನಸ್ಸನ್ನು ನಿರ್ಲಕ್ಷಿಸಿ ವೇದವನ್ನೇ ಅನುಸರಿಸಬೇಕು. ವೇದ ಒಬ್ಬನನ್ನು ಕೊಲ್ಲುವುದು ಸರಿ ಎಂದರೆ ಅದು ಧರ್ಮ, ನಿಮ್ಮ ಮನಸ್ಸು ಅದನ್ನು ಒಪ್ಪುವುದಿಲ್ಲ ಎಂದರೆ ಅದು ಸರಿಯಲ್ಲ. ನೀವು ಯಾವುದಾದರೂ ಶಾಸ್ತ್ರವನ್ನು\break ನಿತ್ಯವಾದುದು, ಪವಿತ್ರವಾದುದು, ಎಂದು ಭಾವಿಸಿದರೆ ನೀವು ಅದನ್ನು ಅನುಮಾನಿಸಲಾರಿರಿ. ಬೈಬಲ್ಲಿನ ವಿಷಯದಲ್ಲಿ “ಅವು ಎಷ್ಟು ಅದ್ಭುತವಾಗಿದೆ. ಅವೆಷ್ಟು ಧಾರ್ಮಿಕವಾಗಿವೆ. ಒಳ್ಳೆಯದಾಗಿವೆ” ಎಂದು ಹೇಳಿದರೆ, ನೀವು ಬೈಬಲ್ಲನ್ನು ಹೇಗೆ ನಂಬುತ್ತೀರೋ ಗೊತ್ತಿಲ್ಲ. ಏಕೆಂದರೆ ಬೈಬಲ್ಲು ಭಗವಂತನ ನುಡಿ ಎಂಬುದನ್ನು ನೀವು ನಂಬಿದರೆ, ಅದನ್ನು ವಿಮರ್ಶಿಸು\-ವುದಕ್ಕೆ ನಿಮಗೆ ಅಧಿಕಾರವೇ ಇಲ್ಲ. ನೀವು ಯಾವಾಗ ಅದನ್ನು ವಿಮರ್ಶಿಸುತ್ತೀರೋ ಆಗ ನೀವು ಬೈಬಲ್ಲಿಗಿಂತ ಮೇಲೆ ಕುಳಿತುಕೊಳ್ಳುತ್ತೀರಿ. ಆಗ ನಿಮ್ಮ ಬೈಬಲ್ಲಿನಿಂದ ಆಗಬೇಕಾದು\-ದೇನು? ಆದರೆ ಪುರೋಹಿತರು ಹೇಳುವರು: ನಿಮ್ಮ ಬೈಬಲ್ಲು ಅಥವಾ ಮತ್ತಾವುದರೊಂದಿಗೂ ನಾವು ಹೋಲಿಸುವುದನ್ನು ಒಪ್ಪಿಕೊಳ್ಳುವುದಿಲ್ಲ. ಸುಮ್ಮನೇ ಹೋಲಿಸಿ\break ಪ್ರಯೋಜನವಿಲ್ಲ. ಅದಕ್ಕೆ ಪ್ರಮಾಣವೇನು? ಅಲ್ಲಿಗೇ ಅದು ಕೊನೆಗಾಣುವುದು.\break ಯಾವುದೂ ಸರಿಯಾಗಿಲ್ಲ ಎಂದರೆ, ವೇದಗಳ ರೀತಿ ಯಾವುದು ಸರಿಯಾಗಿರುವುದೋ ಅದನ್ನೇ ತೆಗೆದುಕೊಳ್ಳಿ.

ಉಪನಿಷತ್ತುಗಳು ಇದನ್ನು ನಂಬುತ್ತವೆ. ಆದರೆ ಅವುಗಳಿಗೆ ಇದಕ್ಕೂ ಉತ್ತಮವಾದ ಒಂದು ಧ್ಯೇಯವಿದೆ. ಅವರು ಒಂದು ಕಡೆ ವೇದಗಳನ್ನು ಆಚೆಗೆ ಎಸೆಯಲು ಇಚ್ಛಿಸುವು\-ದಿಲ್ಲ. ಮತ್ತೊಂದು ಕಡೆ ಪ್ರಾಣಿಬಲಿಯನ್ನು ಮತ್ತು ಪುರೋಹಿತರು ಎಲ್ಲರ ದುಡ್ಡನ್ನು ಕದಿಯುವುದನ್ನು ಒಪ್ಪುವುದಿಲ್ಲ. ಆದರೆ ಮನಶ್ಶಾಸ್ತ್ರದ ವಿಷಯದಲ್ಲಿ ಅವುಗಳೆಲ್ಲ ಒಂದೇ. ಇರುವ ವ್ಯತ್ಯಾಸವೆಲ್ಲ ಜೀವದ ಸ್ವರೂಪದ ವಿಷಯದಲ್ಲಿ. ಜೀವಕ್ಕೆ ಒಂದು ದೇಹ ಮತ್ತು ಮನಸ್ಸು ಇದೆಯೇ? ಮನಸ್ಸು ಎಂದರೆ ಬರೀ ಸ್ನಾಯುಜಾಲವೇ, ಜ್ಞಾನೇಂದ್ರಿಯ ಮತ್ತು ಕರ್ಮೇಂದ್ರಿಯಕ್ಕೆ ಸಂಬಂಧಪಟ್ಟ ನರಗಳ ಸಮೂಹವೇ? ಮನಶ್ಶಾಸ್ತ್ರವು ಒಂದು ಪರಿಪೂರ್ಣ\-ವಾದ ವಿಜ್ಞಾನವೆಂದು ಇವರೆಲ್ಲರೂ ಒಪ್ಪುವರು. ಇಲ್ಲಿ ಯಾವ ವ್ಯತ್ಯಾಸವೂ ಇಲ್ಲ. ಜೀವ ಈಶ್ವರ ಮುಂತಾದವುಗಳಿಗೆ ಸಂಬಂಧಪಟ್ಟ ವಿಷಯದಲ್ಲಿ ಮಾತ್ರ ವ್ಯತ್ಯಾಸ ಇರುವುದು.

ಪುರೋಹಿತರಿಗೂ ಉಪನಿಷತ್ತುಗಳಿಗೂ ಮತ್ತೊಂದು ದೊಡ್ಡ ವ್ಯತ್ಯಾಸವಿದೆ. ಉಪ\-ನಿಷತ್ತು ತ್ಯಾಗಮಾಡು ಎನ್ನುವುದು. ಅದು ಎಲ್ಲದರ ಪರೀಕ್ಷೆ. ಎಲ್ಲವನ್ನೂ ತ್ಯಜಿಸುವುದು. ಮನಸ್ಸಿನ ಸೃಜನಶೀಲ ಶಕ್ತಿಯೇ ನಮ್ಮ ಎಲ್ಲ ಬಂಧನಗಳಿಗೂ ಕಾರಣ. ಮನಸ್ಸು ಪ್ರಶಾಂತವಾದಾಗ ತನ್ನ ಸಹಜ ಸ್ಥಿತಿಯಲ್ಲಿರುವುದು. ನೀವು ಮನಸ್ಸನ್ನು ಪ್ರಶಾಂತಗೊಳಿಸಿದೊಡನೆ ನಿಮಗೆ ಸತ್ಯದ ಅರಿವಾಗುವುದು. ಮನಸ್ಸನ್ನು ಯಾವುದು ಸುತ್ತಿಸುತ್ತಿರುವುದು? ಕಲ್ಪನೆ, ಒಂದು ವಸ್ತುವನ್ನು ಸೃಷ್ಟಿಸುವ ಸ್ವಭಾವ. ನೀವು ಸೃಷ್ಟಿಸುವುದನ್ನು ತ್ಯಜಿಸಿದೊಡನೆ ನಿಮಗೆ ಸತ್ಯದ ಅರಿವಾಗುವುದು. ಎಲ್ಲಾ ವಿಧವಾದ ಸೃಷ್ಟಿಶಕ್ತಿಯೂ ನಿಲ್ಲಬೇಕು. ಆಗ ತಕ್ಷಣವೇ ಸತ್ಯದ ಅರಿವು ನಿಮಗಾಗುವುದು.

ಆದರೆ ಪುರೋಹಿತರೆಲ್ಲ ಸೃಷ್ಟಿಯನ್ನು ಸಮರ್ಥಿಸುವವರು. ಸೃಷ್ಟಿಯ ಚಟುವಟಿಕೆ\-ಯಿಲ್ಲದ ಯಾವುದಾದರೂ ಜೀವಮಾರ್ಗವನ್ನು ಕಲ್ಪಿಸಿಕೊಳ್ಳಲು ಯತ್ನಿಸಿ. ಅದು ಅಸಾಧ್ಯ. ಒಂದು ಸುಭದ್ರವಾದ ಸಮಾಜವನ್ನು ರಚಿಸಲು ಒಂದು ಯೋಜನೆ ಆವಶ್ಯಕವಾಗಿತ್ತು. ಅದಕ್ಕಾಗಿ ಒಂದು ಕಟ್ಟುನಿಟ್ಟಾದ ಆಯ್ಕೆಯ ಪದ್ಧತಿ ಜಾರಿಗೆ ಬಂತು. ಉದಾಹರಣೆಗೆ\break ಕುರುಡರು ಕುಂಟರು ಮದುವೆಯಾಗಕೂಡದಾಗಿತ್ತು. ಇದರ ಪರಿಣಾಮವಾಗಿಯೇ\break ಅಂಗಹೀನರ ಸಂಖ್ಯೆಯು ಜಗತ್ತಿನ ಬೇರಾವ ಭಾಗಕ್ಕಿಂತ ಇಂಡಿಯಾ ದೇಶದಲ್ಲಿ ಕಡಿಮೆ. ಮೂರ್ಛೆರೋಗದವರು ಮತ್ತು ಹುಚ್ಚರ ಸಂಖ್ಯೆ ಅಲ್ಲಿ ಅಪರೂಪ. ಅವರು ಬೇಕಾದರೆ ಸಂನ್ಯಾಸಿಗಳಾಗಲಿ ಎಂದು ಪುರೋಹಿತರು ಹೇಳುತ್ತಿದ್ದರು. ಅದಕ್ಕೆ ವಿರೋಧವಾಗಿ ಉಪನಿಷತ್ತು ಹೇಳುವುದು ಹೀಗೆ: “ಹಾಗೆ ಆಗಕೂಡದು. ಜಗತ್ತಿನ ಶ್ರೇಷ್ಠತಮ ವ್ಯಕ್ತಿಗಳು, ಅಚ್ಚ ಹೊಸದಾದ ಪುಷ್ಪಗಳನ್ನು ಮಾತ್ರ ಭಗವಂತನ ಪೀಠದಡಿಯಲ್ಲಿ ಇಡಬೇಕು. ಯಾರು ಬಲಿಷ್ಠರೋ, ಯುವಕರೋ, ಮನಸ್ಸು ಮತ್ತು ದೇಹಗಳು ಯಾರಿಗೆ ಚೆನ್ನಾಗಿವೆಯೋ ಅವರು ಸತ್ಯಕ್ಕೆ ಹೋರಾಡಬೇಕು.”

ಇದರಂತೆಯೇ ಎಲ್ಲಾ ಭಿನ್ನಾಭಿಪ್ರಾಯಗಳೊಂದಿಗೂ ಪುರೋಹಿತರು ತಾವೇ ಒಂದು ಪ್ರತ್ಯೇಕವಾದ ಜಾತಿಯಾದರು ಎಂಬುದನ್ನು ಹೇಳಿರುವೆನು. ಎರಡನೆಯದೆ ರಾಜನ ಮತದ ಪ್ರಶ್ನೆ. ಉಪನಿಷತ್ತಿನ ತತ್ತ್ವಗಳೆಲ್ಲ ರಾಜರ ಮಿದುಳಿನಿಂದ ಬಂದುವು, ಬ್ರಾಹ್ಮಣನ ಮಿದುಳಿನಿಂದಲ್ಲ. ಪ್ರತಿಯೊಂದು ಧಾರ್ಮಿಕ ಹೋರಾಟದ ಹಿಂದೆಯೂ ಒಂದು ಆರ್ಥಿಕ ಹೋರಾಟವಿದೆ. ಈ ಮನುಷ್ಯ ಪ್ರಾಣಿಯ ಮೇಲೆ ಧಾರ್ಮಿಕ ಪ್ರಭಾವವಿದೆ, ಆದರೆ ಆರ್ಥಿಕ ದೃಷ್ಟಿಯು ಅವನನ್ನು ಆಳುತ್ತಿದೆ. ವ್ಯಕ್ತಿಗಳು ಮಾತ್ರ ಬೇರೆ ಯಾವುದೋ ಒಂದರಿಂದ ನಿಯಂತ್ರಿಸಲ್ಪಡುತ್ತಾರೆ. ಆದರೆ ಜನಸಮೂಹ ಆರ್ಥಿಕ ಪ್ರಯೋಜನವಿಲ್ಲದಿದ್ದರೆ ಯಾವುದನ್ನೂ ಸಾಧಾರಣವಾಗಿ ಸ್ವೀಕರಿಸುವುದಿಲ್ಲ. ನೀವು ಒಂದು ಧರ್ಮವನ್ನು ಬೋಧಿಸ\-ಬಹುದು. ಅದೇನೋ ಪ್ರತಿಯೊಂದು ದೃಷ್ಟಿಯಿಂದಲೂ ಸಮರ್ಪಕವಾಗಿಲ್ಲದೇ ಇರಬಹುದು. ಆದರೆ ಅದಕ್ಕೊಂದು ಆರ್ಥಿಕ ಹಿನ್ನೆಲೆ ಇದ್ದರೆ, ಅದನ್ನು ಬೋಧಿಸುವವನು ಚತುರನಾಗಿದ್ದರೆ, ಇಡೀ ದೇಶದ ಜನರನ್ನೇ ಒಪ್ಪಿಸಿಬಿಡಬಹುದು.

ಧರ್ಮವು ಯಾವಾಗಲಾದರೂ ಜಯಪ್ರದವಾಗಬೇಕಾದರೆ ಅದಕ್ಕೆ ಆರ್ಥಿಕ ಬೆಲೆ\break ಇರಬೇಕು. ಸಾವಿರಾರು ಮತಗಳು ಮುಂದೆ ಅಧಿಕಾರಕ್ಕಾಗಿ ಹೋರಾಡುತ್ತಿರುತ್ತವೆ.\break ಯಾವುವು ಆರ್ಥಿಕ ಸಮಸ್ಯೆಯನ್ನು ಪರಿಹರಿಸಬಲ್ಲವೊ ಅವು ಮಾತ್ರ ಜಯಪ್ರದವಾಗು\-ವುವು. ಮನುಷ್ಯನು ಹೊಟ್ಟೆ ಹೇಳಿದಂತೆ ಕೇಳುವನು. ಅವನು ನಡೆಯುವಾಗ ಮುಂದೆ ಹೊಟ್ಟೆ ಹೋಗುವುದು. ಅನಂತರ ಅವನ ತಲೆ ಹೋಗುವುದು. ನೀವು ಅದನ್ನು\break ನೋಡಿಲ್ಲವೇ? ತಲೆ ಮುಂದೆ ಹೋಗಬೇಕಾದರೆ ಅದಕ್ಕೆ ಬಹಳ ಕಾಲ ಹಿಡಿಯುವುದು.\break ಮನುಷ್ಯನಿಗೆ ಅರವತ್ತು ವರುಷಗಳು ಆಗುವ ಹೊತ್ತಿಗೆ ಅವನು ಪ್ರಪಂಚವನ್ನು\break ಬಿಡಬೇಕಾಗುವುದು. ಈ ಜೀವನವೇ ಒಂದು ಭ್ರಾಂತಿ. ವಸ್ತುವನ್ನು ನೀವು ಸರಿಯಾಗಿ ತಿಳಿದುಕೊಳ್ಳುವ ಹೊತ್ತಿಗೆ ನೀವು ಪ್ರಪಂಚವನ್ನು ಬಿಡಬೇಕಾಗಿದೆ. ಹೊಟ್ಟೆ ಎಲ್ಲಿಯವರೆಗೆ ಮುಂದೆ ಹೋಗುವುದೋ ಅಲ್ಲಿಯವರೆಗೆ ಸರಿಯಾಗಿತ್ತು. ಮಕ್ಕಳ ಕನಸುಗಳು ಮಾಯವಾಗಿ ವಸ್ತುಗಳನ್ನು ಸಹಜಸ್ಥಿತಿಯಲ್ಲಿ ತಿಳಿದುಕೊಳ್ಳುವ ಹೊತ್ತಿಗೆ ತಲೆಯು ಮುಂದಾಗುತ್ತದೆ. ನಿಮ್ಮ ತಲೆ ಮುಂದೆ ಹೋಗುವ ಹೊತ್ತಿಗೆ ನೀವೇ ಪ್ರಪಂಚವನ್ನು ಬಿಡಬೇಕಾಗುವುದು.

ಉಪನಿಷತ್ತುಗಳ ಧರ್ಮವನ್ನು ಎಲ್ಲರಿಗೂ ಹಿಡಿಯುವಂತೆ ಮಾಡುವುದು ದುಸ್ತರ. ಅಲ್ಲಿ ಅರ್ಥಲಾಭವೇನೂ ಇಲ್ಲ, ಬೇಕಾದಷ್ಟು ಲೋಕಕಲ್ಯಾಣವಿದೆ. ಉಪನಿಷತ್ತುಗಳನ್ನು ಬೋಧಿಸಿದವರು ರಾಜರಾಗಿದ್ದರೂ, ಅವರ ಕೈಯಲ್ಲಿ ಎಲ್ಲಾ ಆಸ್ತಿ ಪಾಸ್ತಿಗಳಿದ್ದರೂ, ಉಪನಿಷತ್ತುಗಳಿಗೆ ಯಾವ ಪ್ರಾಪಂಚಿಕ ಬೆಂಬಲವೂ ಇರಲಿಲ್ಲ. ಹೋರಾಟ ತುಂಬಾ ಕಠಿಣವಾಗುತ್ತ ಬಂತು. ಎರಡು ಸಾವಿರ ವರುಷಗಳಾದ ಮೇಲೆ ಅದು ಬೌದ್ಧಧರ್ಮದಲ್ಲಿ ಪರ್ಯವಸಾನವಾಯಿತು. ಬೌದ್ಧಧರ್ಮದ ಬೀಜವಿರುವುದು ರಾಜರಿಗೂ ಪುರೋಹಿತರಿಗೂ ನಡುವೆ ಇದ್ದ ಸಂಘರ್ಷದಲ್ಲಿ, ಹೋರಾಟದಲ್ಲಿ ಎಲ್ಲ ಧರ್ಮವೂ ಅವನತಿಗೆ ಬಂತು. ಒಬ್ಬರು ಧರ್ಮವನ್ನೇ ನಾಶ ಮಾಡಲು ಸಿದ್ಧರಾಗಿದ್ದರು. ಮತ್ತೊಬ್ಬರು ವೇದಗಳ ಕಾಲದ ದೇವತಾದಿಗಳಿಗೆ, ಯಜ್ಞಗಳಿಗೆ ಅಂಟಿಕೊಂಡಿರಬೇಕೆಂದಿದ್ದರು. ಬೌದ್ಧಧರ್ಮವು ಜನ ಸಾಮಾನ್ಯರ ಸಂಕೋಲೆಗಳನ್ನು ಕಿತ್ತು ಒಗೆಯಿತು. ಒಂದು ಕ್ಷಣದಲ್ಲಿ ಎಲ್ಲಾ ಜಾತಿಕೋಮುಗಳು ಒಂದೇ ಆದವು. ಆದಕಾರಣ ಶ್ರೇಷ್ಠವಾದ ಧಾರ್ಮಿಕ ಭಾವನೆಗಳು ಭಾರತದಲ್ಲಿ ಇವೆ. ಆದರೆ ಅವನ್ನು ಜನರಿಗೆ ಇನ್ನೂ ಬೋಧಿಸಬೇಕಾಗಿದೆ. ಇಲ್ಲದೆ ಇದ್ದರೆ ಅದರಿಂದ ಏನೂ ಪ್ರಯೋಜನವಿಲ್ಲ.

ಪ್ರತಿಯೊಂದು ದೇಶದಲ್ಲಿಯೂ ಪುರೋಹಿತರು ಯಾವಾಗಲೂ ಸಂಪ್ರದಾಯಸ್ಥರು. ಅದಕ್ಕೆ ಎರಡು ಕಾರಣಗಳಿವೆ. ಮೊದಲನೆಯದು ಅದೇ ಅವರ ಜೀವನೋಪಾಯ. ಎರಡನೆಯದು ಅವರು ಜನರೊಡನೆ ಮಾತ್ರ ಮುಂದೆ ಹೋಗಬಲ್ಲರು. ಪುರೋಹಿತರೆಲ್ಲ ಬಲಾಢ್ಯರಲ್ಲ. ಜನರು ಎರಡು ಸಾವಿರ ದೇವರುಗಳನ್ನು ಬೋಧಿಸಿ ಎಂದು ಪುರೋಹಿತರಿಗೆ ಹೇಳಿದರೆ, ಅವರು ಅದನ್ನು ಮಾಡಲು ಸಿದ್ಧ. ಯಾರು ಅವರಿಗೆ ದಕ್ಷಿಣೆ ಕೊಡುತ್ತಾರೆಯೋ ಅವರ ಸೇವಕರವರು. ದೇವರು ಅವರಿಗೆ ದಕ್ಷಿಣೆ ಕೊಡುವುದಿಲ್ಲ. ಆದಕಾರಣವೇ ಪುರೋಹಿತರನ್ನು ದೂರುವುದಕ್ಕೆ ಮೊದಲು ನಿಮ್ಮನ್ನು ದೂರಿಕೊಳ್ಳಬೇಕು. ನೀವು ಯಾವುದಕ್ಕೆ ಯೋಗ್ಯರೋ ಅಂತಹ ಸರ್ಕಾರ, ಧರ್ಮ ಮತ್ತು ಪುರೋಹಿತರು ನಿಮಗೆ ದೊರಕುವರು. ಅದಕ್ಕಿಂತ\break ಮೇಲಿನದು ದೊರಕುವುದಿಲ್ಲ.

ಭರತಖಂಡದಲ್ಲಿ ಹೀಗೆ ಹೋರಾಟ ಪ್ರಾರಂಭವಾಯಿತು. ಗೀತೆಯಲ್ಲಿ ಅದು ಚರಮಸೀಮೆ ಮುಟ್ಟುತ್ತದೆ. ಈ ಎರಡು ಗುಂಪುಗಳ ನಡುವಣ ಹೋರಾಟದಲ್ಲಿ ಭರತಖಂಡವು ನುಚ್ಚುನೂರಾಗುತ್ತದೆ ಎಂಬ ಭಯವು ತಲೆದೋರಿದಾಗ ಪುರೋಹಿತರ ಯಜ್ಞಯಾಗಾದಿಗಳು ಮತ್ತು ಜನರ ತಾತ್ತ್ವಿಕ ಭಾವನೆಗಳು, ಇವುಗಳ ನಡುವೆ ಶ‍್ರೀಕೃಷ್ಣನೆಂಬ ವ್ಯಕ್ತಿಯು ಉದಿಸಿದನು. ಗೀತೆಯಲ್ಲಿ ಒಂದು ಸಾಮರಸ್ಯವನ್ನು ಉಂಟುಮಾಡಲು ಯತ್ನಿಸಿದನು. ನೀವು ಕ್ರಿಸ್ತನನ್ನು ಪ್ರೀತಿಸುವಂತೆ ಪೂಜಿಸುವಂತೆ ಶ‍್ರೀಕೃಷ್ಣನನ್ನು (ಭರತಖಂಡದಲ್ಲಿ) ಪ್ರೀತಿಸು\-ವರು, ಪೂಜಿಸುವರು. ಕಾಲದಲ್ಲಿ ಮಾತ್ರ ವ್ಯತ್ಯಾಸ. ಹಿಂದೂಗಳು ಕೃಷ್ಣನ ಜನ್ಮದಿನವನ್ನು ನೀವು ಕ್ರಿಸ್ತನ ಜನ್ಮದಿನವನ್ನು ಆಚರಿಸುವಂತೆ ಆಚರಿಸುವರು. ಶ‍್ರೀಕೃಷ್ಣ ಐದು ಸಾವಿರ ವರುಷಗಳ ಹಿಂದೆ ಇದ್ದ. ಅವನ ಜೀವನದಲ್ಲಿ ಬೇಕಾದಷ್ಟು ಅದ್ಭುತ ಘಟನೆಗಳಿವೆ. ಕೆಲವು ಕ್ರಿಸ್ತನ ಜೀವನದಲ್ಲಿ ಬರುವ ಘಟನೆಗಳಂತಹವೇ. ಮಗು ಸೆರೆಯಲ್ಲಿ ಹುಟ್ಟಿತು. ತಂದೆಯು ಮಗುವನ್ನು ಕರೆದುಕೊಂಡು ಹೋಗಿ ಗೊಲ್ಲರೊಡನೆ ಬಿಟ್ಟುಬರುವನು. ಆ ವರುಷ ಹುಟ್ಟಿದ ಮಕ್ಕಳನ್ನೆಲ್ಲಾ ಕೊಲ್ಲುವಂತೆ ಆಜ್ಞೆಯಾಯಿತು. ಕೊನೆಗೆ ಅವನೂ\break (ಶ‍್ರೀ ಕೃಷ್ಣನೂ) ಕೊಲ್ಲಲ್ಪಟ್ಟನು. ಅದೇ ವಿಧಿ.

ಶ‍್ರೀಕೃಷ್ಣ ಮದುವೆಯಾದವನು. ಅವನಿಗೆ ಸಂಬಂಧಪಟ್ಟ ಸಹಸ್ರಾರು ಗ್ರಂಥಗಳಿವೆ. ನನಗೇನೂ ಅವುಗಳ ಮೇಲೆ ಅಷ್ಟು ಆಸಕ್ತಿಯಿಲ್ಲ. ನೀವೇ ನೋಡಿ, ಹಿಂದೂಗಳು ಕಥೆ ಹೇಳುವುದರಲ್ಲಿ ನಿಸ್ಸೀಮರು. ಕ್ರಿಶ್ಚಿಯನ್​ ಪಾದ್ರಿ ತನ್ನ ಬೈಬಲ್ಲಿನಿಂದ ಒಂದು ಕಥೆಯನ್ನು\break ಹೇಳಿದರೆ, ಹಿಂದೂಗಳು ಅಂತಹ ಇಪ್ಪತ್ತು ಕಥೆಗಳನ್ನು ಹೇಳುವರು. ತಿಮಿಂಗಿಲ\break ಜೋನನ ದೋಣಿಯನ್ನು ನುಂಗಿತು ಎಂದು ಹೇಳಿದರೆ, ಹಿಂದೂಗಳು ಯಾರೋ\break ಆನೆಯನ್ನು ನುಂಗಿದರು ಎಂದು ಹೇಳುವರು. ನಾನು ಬಾಲ್ಯಾರಭ್ಯ ಕೃಷ್ಣನ ಜೀವನವನ್ನು ಕೇಳಿರುವೆನು. ಶ‍್ರೀಕೃಷ್ಣನಂತಹ ವ್ಯಕ್ತಿ ಇದ್ದಿರಬೇಕು ಎಂದು ನಾನು ನಂಬುವೆನು.\break ಅವನೊಂದು ಅದ್ಭುತವಾದ ಗ್ರಂಥವನ್ನು ಜಗತ್ತಿಗೆ ಕೊಟ್ಟಿರುವನು ಎಂಬುದನ್ನೇ ಗೀತೆ ನಮಗೆ ತೋರಿಸುತ್ತದೆ. ಒಬ್ಬನಿಗೆ ಸಂಬಂಧಪಟ್ಟ ಕಥೆಗಳನ್ನು ವಿಶ್ಲೇಷಿಸಿದರೆ ಆ ವ್ಯಕ್ತಿಯ ಶೀಲ ನಮಗೆ ಅರಿವಾಗುತ್ತದೆ. ಕಥೆಗಳು ಅಲಂಕಾರಕ್ಕಾಗಿ ಇವೆ. ಯಾರನ್ನೋ ಅಲಂಕರಿಸು\-ವುದಕ್ಕಾಗಿ ಸಿದ್ಧವಾಗಿವೆ. ಉದಾಹರಣೆಗೆ ಬುದ್ಧನನ್ನು ತೆಗೆದುಕೊಳ್ಳಿ. ಅವನಿಗೆ ಸಂಬಂಧಿಸಿದ ಸಾವಿರಾರು ಜಾನಪದ ಕಥೆಗಳಿವೆ. ಆದರೆ ಪ್ರತಿಯೊಂದರಲ್ಲಿಯೂ ತ್ಯಾಗದ ಭಾವನೆ ಮುಖ್ಯ. ಲಿಂಕನ್ನಿನ ಜೀವನ ಮತ್ತು ಅವನ ಶೀಲಕ್ಕೆ ಸಂಬಂಧಪಟ್ಟ ಹಲವು ಕಥೆಗಳಿವೆ. ಅವುಗಳನ್ನೆಲ್ಲಾ ಒಟ್ಟಿಗೆ ತೆಗೆದುಕೊಂಡರೆ ಅವುಗಳಲ್ಲಿ ಯಾವುದೋ ಮುಖ್ಯವಾದ ಭಾವನೆ ಇದೆ. ಅದೇ ಮನುಷ್ಯನ ಶೀಲ. ಶ‍್ರೀಕೃಷ್ಣನಲ್ಲಿ ಅನಾಸಕ್ತಿಯೇ ಮುಖ್ಯವಾದ ಭಾವ. ಅವನಿಗೆ ಏನೂ ಬೇಕಾಗಿಲ್ಲ. ಯಾವುದರ ಆವಶ್ಯಕತೆಯೂ ಇಲ್ಲ. ಅವನು ಕರ್ಮಕ್ಕಾಗಿ ಕರ್ಮ ಮಾಡುತ್ತಾನೆ. ಕರ್ಮಕ್ಕಾಗಿ ಕರ್ಮ, ಪೂಜೆಗಾಗಿ ಪೂಜೆ. ಅದೇ ಆದರ್ಶ. ಒಳ್ಳೆಯದನ್ನು ಮಾಡಿ, ಏಕೆಂದರೆ ಒಳ್ಳೆಯದನ್ನು ಮಾಡುವುದು ಒಳ್ಳೆಯದು. ಇನ್ನೇನನ್ನೂ\break ಕೇಳಬೇಡಿ. ಇದೇ ಅವನ ಶೀಲವಾಗಿದ್ದಿರಬೇಕು. ಇಲ್ಲದಿದ್ದರೆ ಅವನಿಗೆ ಸಂಬಂಧಪಟ್ಟ\break ಕಥೆಗಳನ್ನೆಲ್ಲಾ ಅನಾಸಕ್ತಿಯೆಂಬ ಭಾವನೆಯೊಂದಿಗೆ ಜೋಡಿಸುವುದಕ್ಕೆ ಆಗುತ್ತಿರಲಿಲ್ಲ.\break ಭಗವದ್ಗೀತೆಯೊಂದೇ ಅವನ ಉಪದೇಶವಲ್ಲ.

ನನಗೆ ತಿಳಿದಿರುವ ಮಟ್ಟಿಗೆ ಇವನಷ್ಟು ಪರಿಪೂರ್ಣವಾದ ವ್ಯಕ್ತಿಯನ್ನು ನಾನು ಕಾಣೆ. ಬುದ್ಧಿ, ಭಾವ ಮತ್ತು ಕರ್ಮಗಳಲ್ಲೆಲ್ಲ ಒಂದು ಅದ್ಭುತವಾದ ಸಾಮರಸ್ಯವನ್ನು ಇವನಲ್ಲಿ ನೋಡುವೆವು. ಇವನ ಜೀವನದ ಪ್ರತಿಯೊಂದು ಕರ್ಮವೂ ಚಟುವಟಿಕೆಯಿಂದ ತುಂಬಿ\break ತುಳುಕಾಡುತ್ತಿದೆ. ಗೃಹಸ್ಥನಾಗಿ, ಯೋಧನಾಗಿ, ಮಂತ್ರಿಯಾಗಿ ಅಥವಾ ಯಾವುದಾದರೂ ಪಾತ್ರದಲ್ಲಿ ಯಾವಾಗಲೂ ಕರ್ಮಶೀಲನಾಗಿರುವನು. ಸಭ್ಯ ವ್ಯಕ್ತಿಯಾಗಿ, ವಿದ್ವಾಂಸನಾಗಿ\break ಅವನು ಉತ್ತಮೋತ್ತಮನಾಗಿರುವನು. ಸರ್ವತೋಮುಖವಾದ, ಅದ್ಭುತವಾದ ಈ\break ಚಟುವಟಿಕೆಯನ್ನು, ಹೃದಯ ಮತ್ತು ಬುದ್ಧಿಯ ಒಂದು ಸಾಮರಸ್ಯವನ್ನು ನೀವು ಗೀತೆ ಮತ್ತು ಅವನಿಗೆ ಸಂಬಂಧಿಸಿದ ಇತರ ಗ್ರಂಥಗಳಲ್ಲಿ ನೋಡುವಿರಿ. ಅದ್ಭುತವಾದ\break ಹೃದಯ ಸಂಪತ್ತು, ಸೊಗಸಾದ ಭಾಷೆ, ಇವುಗಳಿಂದ ಕೂಡಿರುವ ಭಗವದ್ಗೀತೆಗೆ\break ಸಾಟಿಯಾದುದನ್ನು ಜಗತ್ತಿನ ಬೇರೆಲ್ಲೂ ಕಾಣಲಾರಿರಿ. ಈ ವ್ಯಕ್ತಿಯ ಜೀವನ ಅತ್ಯದ್ಭುತವಾದ ಚಟುವಟಿಕೆಯ ಬಿರುಗಾಳಿ. ಇವನ ಪ್ರಭಾವವನ್ನು ಈಗಲೂ ನೋಡಬಹುದು. ಐದು ಸಾವಿರ ವರುಷಗಳಾಗಿವೆ. ಅವನು ಕೋಟ್ಯಂತರ ಜನರ ಮೇಲೆ ತನ್ನ ಪ್ರಭಾವವನ್ನು ಬೀರಿರುವನು. ನಿಮಗೆ ಗೊತ್ತಿರಲಿ, ಇಲ್ಲದೆ ಇರಲಿ, ಪ್ರಪಂಚದ ಮೇಲೆ ಇವನಿಗೆ ಇರುವ ಪ್ರಭಾವವನ್ನೇ ಕುರಿತು ಯೋಚಿಸಿ ನೋಡಿ. ನಾನು ಅವನನ್ನು ಗೌರವಿಸುವುದು ಅವನ ಬುದ್ಧಿಯ ಸಮತೋಲನಕ್ಕಾಗಿ. ಅವನ ಜ್ಞಾನದಲ್ಲಿ ಯಾವ ಸಂಶಯದ ಛಾಯೆಯಿಲ್ಲ. ಅವನಿಗೆ ನಾನು ಸಲ್ಲಿಸುವ ಗೌರವ ಇದಕ್ಕಾಗಿ. ಅವನ ಬುದ್ಧಿಯಲ್ಲಿ ಯಾವ ಅಜ್ಞಾನವೂ ಇಲ್ಲ. ಯಾವ ಮೂಢಾಚಾರವೂ ಇಲ್ಲ. ಅವನಿಗೆ ಪ್ರತಿಯೊಂದು ವಸ್ತುವಿನ ಉಪಯೋಗವೂ ಗೊತ್ತಿದೆ. ಯಾವುದಕ್ಕಾದರೂ ಒಂದು ಸ್ಥಾನವನ್ನು ಕೊಡಬೇಕಾದಾಗ ಅವನು ಇಲ್ಲಿರುವನು. ಸುಮ್ಮನೆ ಮಾತನಾಡುತ್ತಾ ಎಲ್ಲೆಲ್ಲೊ ಹೋಗುತ್ತಾ ವೇದದ ರಹಸ್ಯವನ್ನು ಪ್ರಶ್ನಿಸುವವರಿಗೆ ಸತ್ಯಗೊತ್ತಿಲ್ಲ. ಅವರು ಮೋಸಗಾರರಿಗಿಂತ ಮೇಲೇನೂ ಇಲ್ಲ. ವೇದಗಳಲ್ಲಿ ಕೂಡ ಮೂಢಾಚಾರಕ್ಕೆ ಅಜ್ಞಾನಕ್ಕೆ ಒಂದು ಸ್ಥಳವಿದೆ. ಪ್ರತಿಯೊಂದಕ್ಕೂ ಅದರದರ ಸ್ಥಳವನ್ನು ಕಂಡು ಹಿಡಿಯುವುದೇ ಮುಖ್ಯವಾದ ರಹಸ್ಯ.

ಅದ್ಭುತವಾದ ಅನುಕಂಪ ಅವನದು! ಬುದ್ಧನಿಗೆ ಮುಂಚೆ ಪ್ರತಿಯೊಂದು ಕೋಮಿನವರಿಗೂ ಮುಕ್ತಿಯ ಹೆಬ್ಬಾಗಿಲನ್ನು ತೆರೆದವನು ಶ‍್ರೀಕೃಷ್ಣನು, ಅದೊಂದು ಪ್ರಚಂಡವಾದ\break ಧೀಶಕ್ತಿ! ಅದ್ಭುತವಾದ ಕರ್ಮಪರಂಪರೆಯಿಂದ ಆವೃತವಾದ ಜೀವನ. ಬುದ್ಧನ ಚಟುವಟಿಕೆ\-ಯಾದರೊ ಜೀವನದ ಒಂದು ಕಾರ್ಯಕ್ಷೇತ್ರದಲ್ಲಿತ್ತು. ಅದೇ ಬೋಧನೆಯ ಪ್ರಪಂಚದಲ್ಲಿ. ಅವನು ತನ್ನ ಹೆಂಡತಿ ಮಕ್ಕಳೊಂದಿಗೆ ಇರಲಾಗಲಿಲ್ಲ. ಅವರನ್ನು ಬಿಟ್ಟು ಗುರುವಾಗುವನು. ಶ‍್ರೀಕೃಷ್ಣನಾದರೊ ಸಮರಾಂಗಣದ ಮಧ್ಯದಲ್ಲಿ ಬೋಧಿಸಿದನು. “ಯಾರು ಅಕರ್ಮದಲ್ಲಿ ಕರ್ಮ ನೋಡುವನೋ, ಕರ್ಮದಲ್ಲಿ ಅಕರ್ಮವನ್ನು ನೋಡುವನೋ, ಅವನೇ ಶ್ರೇಷ್ಠನಾದ ಯೋಗಿ.” ಅವನ ಸುತ್ತಲೂ ಶರವರ್ಷವೇ ಕರೆಯುತ್ತಿದ್ದರೂ ಅದನ್ನು ಆತ ಗಮನಿಸುವುದೇ ಇಲ್ಲ. ಆ ಸಮಯದಲ್ಲಿ ಪ್ರಶಾಂತಚಿತ್ತನಾಗಿ ಸುಮ್ಮನೆ ಜನನ ಮರಣಗಳ ರಹಸ್ಯವನ್ನು ಚರ್ಚಿಸುತ್ತಿರುವನು. ಪ್ರತಿಯೊಬ್ಬ ಪ್ರವಾದಿಯ ಜೀವನವೇ ಅವನ ಬೋಧನೆಗೆ ಶ್ರೇಷ್ಠವಾದ ಭಾಷ್ಯ. ನ್ಯೂ ಟೆಸ್ಟಮೆಂಟಿನಲ್ಲಿ ಏನು ಹೇಳಿದೆ ಎಂಬುದನ್ನು ತಿಳಿದುಕೊಳ್ಳಲು ನೀವು ಯಾರ ಬಳಿಗಾದರೂ ಹೋಗುತ್ತೀರಿ. ನೀವು ಅದರ ಬದಲು ನಾಲ್ಕು ಗಾಸ್ಪೆಲ್​ಗಳನ್ನೇ ಪದೇ ಪದೇ ಓದಿ; ಅಲ್ಲಿ ಬರುವ ಅದ್ಭುತವಾದ ಕ್ರಿಸ್ತನ ಜೀವನದ\break ಬೆಳಕಿನಲ್ಲಿ ಬರುವ ಸಂದೇಶವನ್ನೇ ತಿಳಿದುಕೊಳ್ಳಲು ಯತ್ನಿಸಿ. ಮಹಾಪುರುಷರು ಆಲೋಚಿಸಿದರು. ನಾವು ಕೂಡ ಆಲೋಚಿಸುತ್ತೇವೆ. ಆದರೆ ಅದು ಅನುಷ್ಠಾನದಲ್ಲಿ ವ್ಯಕ್ತವಾಗುವುದಿಲ್ಲ. ನಮ್ಮ ಆಲೋಚನೆ ಮತ್ತು ನಡತೆಗೆ ಸಂಬಂಧವೇ ಇಲ್ಲ. ವೇದವಾಗುವ ಮಾತಿನಲ್ಲಿ ಯಾವ\break ಶಕ್ತಿಯಿದೆಯೋ ಆ ಶಕ್ತಿ ನಮ್ಮ ಮಾತಿನಲ್ಲಿ ಇಲ್ಲ. ಅವರು (ಮಹಾಪುರುಷರು) ಏನು ಆಲೋಚಿಸುವರೋ, ಅದು ಆಗುವುದು. ಅವರು ಇದನ್ನು ಮಾಡುತ್ತೇವೆ ಎಂದರೆ ದೇಹ ಇದನ್ನು ಮಾಡುವುದು. ದೇಹ ಸಂಪೂರ್ಣವಾಗಿ ವಿಧೇಯವಾಗಿರುವುದು. ಇದೇ ಗುರಿ. ನೀವು ಒಂದು ಕ್ಷಣ ನೀವೇ ದೇವರು ಎಂದು ಭಾವಿಸಬಹುದು. ಆದರೆ ನೀವು\break ದೇವರಾಗಲಾರಿರಿ. ಇದೇ ಬಂದಿರುವ ಕಷ್ಟ. ಅವರು ಹೇಗೆ ಆಲೋಚಿಸುವರೋ ಹಾಗೆ\break ಆಗುತ್ತಾರೆ. ನಾವಾದರೋ ಕ್ರಮಕ್ರಮವಾಗಿ ಹಾಗೆ ಆಗುತ್ತೇವೆ.

ಇದು ಕೃಷ್ಣ ಮತ್ತು ಅವನ ಕಾಲಕ್ಕೆ ಸಂಬಂಧಪಟ್ಟ ವಿಷಯ. ಮುಂದಿನ ಉಪನ್ಯಾಸದಲ್ಲಿ ಅವನ ಗ್ರಂಥಕ್ಕೆ ಸಂಬಂಧಪಟ್ಟ ಹೆಚ್ಚು ವಿಷಯಗಳನ್ನು ತಿಳಿದುಕೊಳ್ಳೋಣ.

