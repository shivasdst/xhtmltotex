
\chapter[ರಾಜಯೋಗ ]{ರಾಜಯೋಗ \protect\footnote{\engfoot{C.W. Vol. VI, P. 89}}}

ರಾಜಯೋಗದ ಮೊದಲನೆಯ ಮೆಟ್ಟಿಲೇ ಯಮ. ಯಮ ನಮಗೆ ಸಿದ್ಧಿಸಬೇಕಾದರೆ ಐದು ವಿಷಯಗಳು ಆವಶ್ಯಕ. ಅವು ಅಹಿಂಸೆ, ಸತ್ಯ, ಅಸ್ತೇಯ, ಅಪರಿಗ್ರಹ, ಬ್ರಹ್ಮಚರ್ಯ, ಪಾವಿತ್ರ್ಯವೇ ಮಹಾಶಕ್ತಿ. ಅದರೆದುರಿಗೆ ಉಳಿದುವೆಲ್ಲಾ ಅಪ್ರಧಾನ.

ಅನಂತರ ಆಸನ ಬರುವುದು. ಆಸನ ದೃಢವಾಗಿರಬೇಕು. ಕತ್ತು ಎದೆ ಸೊಂಟ ಮೂರೂ ಒಂದೇ ಸಮನಾಗಿ ನೇರವಾಗಿರಬೇಕು. ನಾನು ಸ್ಥಿರವಾಗಿ ಕುಳಿತಿರುವೆನು, ಯಾವುದೂ ನನ್ನನ್ನು ಕದಲಿಸಲಾರದು ಎಂದು ಭಾವಿಸಿ, ಶಿರದಿಂದ ಕಾಲಿನವರೆಗೆ ದೇಹವೆಲ್ಲ ಪರಿಶುದ್ಧವಾಗಿದೆ ಎಂದುಕೊಳ್ಳಿ. ಅದು ಸ್ಪಟಿಕದಂತೆ ಶುದ್ಧವಾಗಿದೆ, ಭವಸಾಗರದ ಮೇಲೆ ತೇಲುವುದಕ್ಕೆ ಯೋಗ್ಯವಾಗಿರುವ ಒಂದು ನಾವೆಯಂತೆ ಇದೆ ಎಂದು ಭಾವಿಸಿ.

ದೇವರಲ್ಲಿ ಪ್ರಾರ್ಥಿಸಿ. ಮಹಾತ್ಮರು, ದೇವದೂತರು, ಪ್ರವಾದಿಗಳು ಇವರೆಲ್ಲರೂ ನಿಮ್ಮ ಆಧ್ಯಾತ್ಮಿಕ ಜೀವನಕ್ಕೆ ಸಹಾಯ ಮಾಡಲಿ ಎಂದು ಬೇಡಿಕೊಳ್ಳಿ.

ಅನಂತರ ಅರ್ಧಗಂಟೆ ಪ್ರಾಣಾಯಾಮ ಮಾಡಿ. ಉಚ್ಛ್ವಾಸನಿಃಶ್ವಾಸಗಳನ್ನು ಮಾಡುವಾಗ ‘ಓಂ’ ಎಂದು ಉಚ್ಚರಿಸಿ. ಚೈತನ್ಯದಿಂದ ಕೂಡಿದ ಶಬ್ದಗಳಲ್ಲಿ ಅದ್ಭುತ ಶಕ್ತಿ ಇದೆ.

ಯೋಗದ ಇತರ ಹಂತಗಳೇ ಪ್ರತ್ಯಾಹಾರ, ಧಾರಣ, ಧ್ಯಾನ, ಸಮಾಧಿ ಎಂಬುವು. ಸಮಾಧಿ ಎಂದರೆ ಪರಮಾತ್ಮನಲ್ಲಿ ಮನಸ್ಸು ತಲ್ಲೀನವಾಗಿ ಹೋಗುವುದು. ಅಲ್ಲಿ ನಾನು ಮತ್ತು ನನ್ನ ತಂದೆ ಇಬ್ಬರೂ ಒಂದೇ ಎಂಬ ಭಾವ ಬರುವುದು.

ಒಂದು ಸಾಧನೆಯನ್ನು ಒಂದು ಕಾಲದಲ್ಲಿ ಮಾಡಿ. ಆಗ ಮನಸ್ಸನ್ನೆಲ್ಲಾ ಅದರ ಮೇಲೆ ಇಡಿ; ಮತ್ತಾವುದನ್ನೂ ಚಿಂತಿಸಬೇಡಿ.

