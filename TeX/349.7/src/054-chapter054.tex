
\chapter[ಕರ್ಮವೇ ಪೂಜೆ ]{ಕರ್ಮವೇ ಪೂಜೆ \protect\footnote{\engfoot{C.W. Vol. V, p. 245}}}

ಅತಿ ಶ್ರೇಷ್ಠ ಮಾನವ ಕೆಲಸ ಮಾಡಲಾರ. ಏಕೆಂದರೆ ಅವನನ್ನು ನಿರ್ಬಂಧಗೊಳಿಸುವುದು ಯಾವುದೂ ಇಲ್ಲ. ಅವನಲ್ಲಿ ಯಾವ ಆಸಕ್ತಿಯೂ ಇಲ್ಲ. ಅಜ್ಞಾನವೂ ಇಲ್ಲ. ಒಂದು ಹಡಗು ಸಾಗರದಲ್ಲಿ ಒಂದು ಅಯಸ್ಕಾಂತ ಪರ್ವತದ ಹತ್ತಿರ ಹೋದಾಗ ಹಡಗಿನ ಮೊಳೆಗಿಳೆಗಳೆಲ್ಲ ಹೊರಗೆ ಬಂದುಬಿಟ್ಟು ಹಡಗು ಚೂರು ಚೂರಾಯಿತು. ಹೋರಾಟ ಇರುವುದು ಅಜ್ಞಾನದಲ್ಲಿ. ಏಕೆಂದರೆ ನಿಜವಾಗಿ ನಾವೆಲ್ಲ ನಾಸ್ತಿಕರು. ನಿಜವಾದ ಆಸ್ತಿಕರು ಕೆಲಸ ಮಾಡಲಾರರು. ನಾವೆಲ್ಲ ಸ್ವಲ್ಪ ಹೆಚ್ಚು ಕಡಮೆ ನಾಸ್ತಿಕರೆ. ನಾವು ದೇವರನ್ನು ನೋಡಿಯೂ ಇಲ್ಲ. ಅವನನ್ನು\break ನಂಬುವುದೂ ಇಲ್ಲ. ಅವನು ನಮಗೆ ಕೇವಲ (ಅಕ್ಷರದ) ದೇ–ವ–ರು, ಆಗಿರುವನು.\break ಇದಕ್ಕಿಂತ ಹೆಚ್ಚೇನೂ ಅಲ್ಲ. ಅವನು ನಮ್ಮ ಸಮೀಪದಲ್ಲಿರುವನು ಎಂದು ಭಾವಿಸುವ ಕೆಲವು ಕ್ಷಣಗಳಿರುವುವು. ಆದರೆ ನಾವು ಪುನಃ ಜಾರಿ ಬೀಳುವೆವು. ನೀವು ಅವನನ್ನು ನೋಡಿದ\break ಮೇಲೆ ಯಾರು ಯಾರಿಗಾಗಿ ಹೋರಾಡುವರು? ದೇವರಿಗೆ ಸಹಾಯ ಮಾಡುವುದೆ?\break ನಮ್ಮ ಭಾಷೆಯಲ್ಲಿ ಒಂದು ಗಾದೆ ಇದೆ– “ಈ ಪ್ರಪಂಚವನ್ನು ನಿರ್ಮಿಸಿದವನಿಗೆ ಮನೆ ಕಟ್ಟುವುದನ್ನು ಕಲಿಸಬೇಕೆ?” ಕೆಲಸ ಮಾಡದಿರುವ ಶ್ರೇಷ್ಠ ಪುರುಷರವರು. ಮತ್ತೊಮ್ಮೆ ನೀವು ‘ದೇವರಿಗೆ ಸಹಾಯಮಾಡಬೇಕು, ಅವನಿಗೆ ಅದು ಮಾಡಬೇಕು, ಇದು ಮಾಡಬೇಕು’ ಎಂಬ ಕಾಡುಹರಟೆ ಕೇಳಿದಾಗ ಇದನ್ನು ಜ್ಞಾಪಕದಲ್ಲಿಡಿ. ಇಂತಹ ಆಲೋಚನೆಗಳನ್ನು ಮಾಡಬೇಡಿ. ಇವು ಬರಿಯ ಸ್ವಾರ್ಥ. ನೀವು ಮಾಡುವ ಕೆಲಸವೆಲ್ಲ ನಿಮಗಾಗಿ, ನಿಮ್ಮ ಉಪಯೋಗಕ್ಕಾಗಿ. ದೇವರೇನು ಒಂದು ಮೋರಿಯಲ್ಲಿ ಬಿದ್ದಿಲ್ಲ, ಅವನಿಗೆ ಒಂದು ಆಸ್ಪತ್ರೆಯನ್ನೋ ಅಥವಾ ಮತ್ತೇನನ್ನೊ ಕಟ್ಟಿಸುವುದಕ್ಕಾಗಿ. ಅವನು ನಿಮಗೆ ಕೆಲಸ ಮಾಡುವುದಕ್ಕೆ ಅವಕಾಶ ಕೊಡುವನು. ಈ ಪ್ರಪಂಚವೆಂಬ ದೊಡ್ಡ ಗರಡಿಯ ಮನೆಯಲ್ಲಿ ನಿಮ್ಮ ಅಂಗಸಾಧನೆಗೆ ಅವಕಾಶ ಕೊಡುವನು. ಇದರಿಂದ ದೇವರಿಗೆ ಉಪಯೋಗವಾಗುವುದೆಂದಲ್ಲ; ಇದರಿಂದ ನಿಮಗೆ ಉಪಯೋಗವಾಗುವುದಷ್ಟೆ. ಒಂದು ಇರುವೆ ಕೂಡ ನಿಮ್ಮ ಸಹಾಯವಿಲ್ಲದಿದ್ದರೆ ಸಾಯುವುದು ಎಂದು ಭಾವಿಸುವಿರೇನು? ಇದೊಂದು ಭಯಂಕರ ಈಶ್ವರನಿಂದೆ. ಪ್ರಪಂಚಕ್ಕೆ ನೀವು ಬೇಕಾಗಿಯೇ ಇಲ್ಲ. ಪ್ರಪಂಚ ತನ್ನ ಪಾಡಿಗೆ ತಾನು ನಡೆಯುತ್ತಿರುವುದು. ನೀವು ಸಿಂಧುವಿನಲ್ಲಿರುವ ಬಿಂದುವಿನಂತೆ. ಅವನ ಅಣತಿಯಿಲ್ಲದೆ ಒಂದು ಎಲೆಯೂ ಕದಲಲಾರದು. ಅವನಿಗೆ ಸೇವೆ ಮಾಡುವ ಒಂದು ಅವಕಾಶ ದೊರಕಿದುದರಿಂದ ಧನ್ಯರು ನಾವು. ಅವನಿಗೆ ನಾವು ಸಹಾಯ ಮಾಡಲಾರೆವು. ಸಹಾಯ ಎಂಬ ಪದವನ್ನು ನಿಮ್ಮ ಮನಸ್ಸಿನಿಂದ ನಿರ್ಮೂಲ ಮಾಡಿ. ನೀವು ಸಹಾಯ ಮಾಡಲಾರಿರಿ. ಇದು ಈಶ್ವರನಿಂದೆ. ನೀವು ಅವನ ಇಚ್ಛೆಯಂತೆ ಇಲ್ಲಿರುವಿರಿ. ನೀವು ಅವನಿಗೆ ಸಹಾಯ ಮಾಡುವೆ ಎಂದು ಭಾವಿಸಿರುವಿರೇನು? ನೀವು ಪೂಜಿಸಬಲ್ಲಿರಿ. ನೀವು ನಾಯಿಗೊಂದು ತುತ್ತು ಅನ್ನವನ್ನು ಕೊಟ್ಟರೆ ನೀವು ನಾಯಿಯನ್ನು ದೇವರೆಂದು ಪೂಜಿಸುವಿರಿ ಎಂದೇ ಅರ್ಥ. ದೇವರು ಆ ನಾಯಿಯಲ್ಲಿರುವನು, ಅವನೇ ನಾಯಿ, ಅವನೇ ಸರ್ವವಾಗಿದ್ದಾನೆ, ಅವನೇ ಎಲ್ಲದರಲ್ಲಿಯೂ ಇರುವನು.\break ಅವನನ್ನು ಪೂಜಿಸುವುದಕ್ಕೆ ನಮಗೊಂದು ಅವಕಾಶ ದೊರಕಿದೆ. ಪ್ರಪಂಚದ ಎದುರಿಗೆ ಇಂತಹ ಗೌರವ ಭಾವದಿಂದ ನಿಲ್ಲಿ. ಆಗ ಮಾತ್ರ ಪೂರ್ಣ ಅನಾಸಕ್ತಿ ಬರುವುದು. ಇದು ನಿಮ್ಮ ಕರ್ತವ್ಯವಾಗಬೇಕು. ಕೆಲಸ ಮಾಡುವುದಕ್ಕೆ ಇದೇ ಸರಿಯಾದ ಮನೋಭಾವ.\break ಕರ್ಮಯೋಗ ಸಾರುವ ರಹಸ್ಯವೇ ಇದು.

\vskip -0.5cm

