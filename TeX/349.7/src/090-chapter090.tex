
\chapter[ವೇದಾಂತ ಮತ್ತು ಆಧುನಿಕ ಜಗತ್ತು ]{ವೇದಾಂತ ಮತ್ತು ಆಧುನಿಕ ಜಗತ್ತು \protect\footnote{\engfoot{C.W. Vol. VIII, p. 231}}}

\centerline{\textbf{(1900ರ ಫೆಬ್ರವರಿ 25ರಂದು ಓಕ್​ಲ್ಯಾಂಡ್​ನಲ್ಲಿ ನೀಡಿದ ಉಪನ್ಯಾಸದ ವರದಿ)}}

ಆಧುನಿಕ ಜಗತ್ತು ತನ್ನನ್ನು ಗಮನಕ್ಕೆ ತೆಗೆದುಕೊಳ್ಳಬೇಕು ಎಂದು ವೇದಾಂತವು ಒತ್ತಾಯಪಡಿಸುತ್ತಿದೆ. ಮಾನವಕೋಟಿಯಲ್ಲಿ ಬಹುಪಾಲು ಜನರು ಅದರ ಪ್ರಭಾವಕ್ಕೆ ಒಳಗಾಗಿರುವರು. ಅನೇಕ ಸಲ ವೇದಾಂತಿಗಳ ಮೇಲೆ ಹೊರಗಿನಿಂದ ಬಂದ ಲಕ್ಷಾಂತರ ಜನರು ಧಾಳಿಯನ್ನು ನಡೆಸಿ ಅವರನ್ನು ನಿರ್ನಾಮ ಮಾಡಲು ಯತ್ನಿಸಿರುವರು ಆದರೂ ಈ ಧರ್ಮವು ಇನ್ನೂ ಜೀವಂತವಾಗಿದೆ.

ಜಗತ್ತಿನ ಇತರ ಕಡೆ ಇಂತಹ ಒಂದು ಧರ್ಮ ಸಿಕ್ಕಬಹುದೆ? ಇತರರು ಅದನ್ನು ಗೆದ್ದರೂ ಕೊನೆಗೆ ಅದರ ಪ್ರಭಾವಕ್ಕೆ ಒಳಗಾಗಿರುವುವು. ಹಲವು ಧರ್ಮಗಳು ನಾಯಿಕೊಡೆಗಳಂತೆ ಇಂದು ಜನ್ಮತಾಳಿ ಕೆಲವು ದಿನಗಳಿದ್ದು ಅನಂತರ ನಿರ್ನಾಮವಾಗಿ ಹೋಗಿರುವುವು. ಅತ್ಯಂತ ಬಲಶಾಲಿಯಾದದ್ದು ಮಾತ್ರ ಉಳಿಯುತ್ತದೆ ಎಂಬ ಹೇಳಿಕೆಗೆ ಇದು ನಿದರ್ಶನ ಅಲ್ಲವೆ?

ಇದೊಂದು ತತ್ತ್ವವಿಕಾಸವಾಗುತ್ತಿದೆ, ಇನ್ನೂ ಪೂರ್ಣತೆಯನ್ನು ಮುಟ್ಟಿಲ್ಲ. ಇದು\break ಸಾವಿರಾರು ವರುಷಗಳಿಂದ ಬೆಳೆಯುತ್ತಿದೆ. ಒಂದು ಗಂಟೆಯಲ್ಲಿ ಸಂಕ್ಷೇಪವಾಗಿ ಅದಕ್ಕೆ ಸಂಬಂಧಪಟ್ಟ ಕೆಲವು ವಿಷಯಗಳನ್ನು ಮಾತ್ರ ಹೇಳಬಲ್ಲೆ.

ಮೊದಲು ವೇದಾಂತ ಹೇಗೆ ಹುಟ್ಟಿತು ಹೇಳುತ್ತೇನೆ. ಅದು ಜನ್ಮತಾಳಿದಾಗ ಇಂಡಿಯಾ ದೇಶದಲ್ಲಿ ಪ್ರವರ್ಧಮಾನಕ್ಕೆ ಬಂದ ಒಂದು ಧರ್ಮ ಆಗಲೇ ಇತ್ತು. ಹಲವು ವರುಷಗಳಿಂದ ಒಂದು ಸ್ಪಷ್ಟ ಆಕಾರವನ್ನು ತಾಳುತ್ತಿತ್ತು. ಆಗಲೇ ಬಹಳ ವಿಸ್ತಾರವಾದ ಆಚಾರಗಳಿದ್ದವು. ಹಲವು ವರ್ಣಗಳಿಗೆ ಅನ್ವಯಿಸುವ ಪರಿಪೂರ್ಣವಾದ ನೀತಿ ಪದ್ಧತಿಯಿತ್ತು. ಆದರೆ ಕಾಲಕ್ರಮದಲ್ಲಿ ಅನೇಕ ಧರ್ಮಗಳಿಗೆ ಪ್ರವೇಶ ಮಾಡುವ ಅರ್ಥವಿಲ್ಲದ ಗೊಡ್ಡು ಆಚರಣೆಗಳ ವಿರುದ್ಧ ಜನರು ರೊಚ್ಚಿಗೆದ್ದರು. ಬಳಿಕ ಅನೇಕ ಮಹಾಪುರುಷರು ವೇದಗಳ ಮೂಲಕ ನಿಜವಾದ ಧರ್ಮವನ್ನು ಪಡೆದರು. ವೇದಗಳು ಅನಾದಿ ಮತ್ತು ಅನಂತ ಎಂದು ಅವರಿಗೆ ಬೋಧಿಸಲಾಗಿತ್ತು. ಒಂದು ಪುಸ್ತಕ ಹೇಗೆ ಅನಾದಿ ಮತ್ತು ಅನಂತವಾಗಬಲ್ಲದು ಎಂದು ನಿಮಗೆ ಆಶ್ಚರ್ಯವಾಗಬಹುದು. ಆದರೆ ವೇದಗಳು ಎಂದರೆ ಯಾವ ಪುಸ್ತಕಗಳೂ ಅಲ್ಲ. ವೇದಗಳು ಎಂದರೆ ಬೇರೆ ಬೇರೆ ಕಾಲಗಳಲ್ಲಿ ಬೇರೆ ಬೇರೆ ವ್ಯಕ್ತಿಗಳು ಕಂಡುಕೊಂಡ ಆಧ್ಯಾತ್ಮಿಕ ನಿಯಮಗಳ ಸಂಗ್ರಹ.

ಈ ಋಷಿಗಳು ಬರುವುದಕ್ಕೆ ಮುಂಚೆಯೇ ದೇವರು ಜಗತ್ತನ್ನೆಲ್ಲಾ ಆಳುತ್ತಿರುವನು ಮತ್ತು ಮಾನವ ಅಮೃತಾತ್ಮ ಎಂಬ ಭಾವನೆ ಆಗಲೇ ಜನರಲ್ಲಿ ಹಬ್ಬಿತ್ತು. ಆದರೆ ಅದು ಅಲ್ಲೆ ನಿಂತಿತ್ತು. ಅದಕ್ಕಿಂತ ಹೆಚ್ಚಾಗಿ ಇನ್ನೂ ಏನನ್ನೂ ತಿಳಿಯುವ ಸಾಧ್ಯತೆಯಿಲ್ಲ ಎಂದು ಜನರು ಭಾವಿಸಿದ್ದರು. ಇಲ್ಲಿಯೇ ನಾವು ವೇದಾಂತಿಗಳ ಮಹಾ ಸಾಹಸವನ್ನು ನೋಡುತ್ತೇವೆ. ಮಕ್ಕಳಿಗೆ ತೃಪ್ತಿ ತರುವ ಧರ್ಮವು ವಿಚಾರವಂತರಿಗೆ ತೃಪ್ತಿ ತಾರದು. ವ್ಯಕ್ತಿಗಳಿಗೂ, ದೇವರಿಗೂ ಸಂಬಂಧಪಟ್ಟಂತೆ ಹೆಚ್ಚಾದದ್ದು ಮತ್ತೇನೋ ಇದೆ ಎಂದು ಅವರಿಗೆ ಹೊಳೆಯಿತು.

ಕೇವಲ ನೀತಿಯನ್ನು ಮಾತ್ರ ಒಪ್ಪಿಕೊಳ್ಳುವ ಅಜ್ಞೇಯತಾವಾದಿಗಳಿಗೆ ಬಾಹ್ಯದಲ್ಲಿರುವ ಜಡಪ್ರಕೃತಿ ಮಾತ್ರ ಗೊತ್ತಿರುವುದು. ಅದರಿಂದ ಅವರು ಜಗತ್ತಿನ ನಿಯಮಗಳನ್ನು ರಚಿಸುವರು. ಅವರು ನನ್ನ ಮೂಗನ್ನು ಕತ್ತರಿಸಿಹಾಕಿ ಉಳಿದುದೇ ನನ್ನ ದೇಹ ಎಂದು ಬೇಕಾದರೆ\break ವಾದಿಸಬಹುದು. ಅವರು ಆಂತರ್ಯದಲ್ಲಿ ನೋಡಬೇಕಾಗಿದೆ. ಅನಂತಾಕಾಶದಲ್ಲಿ ಚಲಿಸುವ ನಕ್ಷತ್ರಮಾಲಿಕೆ, ಇಡಿಯ ಬ್ರಹ್ಮಾಂಡವೂ ಕೂಡ ಒಂದು ಚುಕ್ಕಿಗೆ ಸಮ. ಈ ಅಜ್ಞೇಯತಾವಾದಿಗಳು ಮಹೋನ್ನತವಾದುದನ್ನು ನೋಡಲಾರರು. ವಿಶ್ವವನ್ನು ಕಂಡರೇ ಅವರಿಗೆ ಗಾಬರಿಯಾಗಿ ಹೋಗುವುದು.

ಆಧ್ಯಾತ್ಮಿಕ ಜಗತ್ತು ಎಲ್ಲಕ್ಕಿಂತಲೂ ಮಿಗಿಲಾದುದು. ಜಗತ್ತನ್ನೇ ಆಳುವ ದೇವರನ್ನು ತಾಯಿ ಮತ್ತು ತಂದೆ ಎನ್ನುವರು. ಈ ಜಗತ್ತು ಎನ್ನುವ ಅನಿಷ್ಟದಲ್ಲಿ ಏನಿದೆ? ಎಲ್ಲಿ ನೋಡಿದರೂ ದುಃಖವೇ. ಮಗು ಕಣ್ತೆರೆಯುವಾಗಲೇ ಅಳುವುದು. ಗೋಳೇ ಅದರ ಪ್ರಥಮ ಧ್ವನಿ. ಮಗು ದೊಡ್ಡದಾಗುವುದು, ದುಃಖದಲ್ಲಿ ಚೆನ್ನಾಗಿ ಪಳಗಿದ ಮೇಲೆ ಎದೆಯೊಳಗಿನ ಗೋಳನ್ನು ಮರೆಮಾಡಲು ಮೇಲೊಂದು ನಗುವಿನ ತೆರೆಯನ್ನು ಹಾಕುವುದು.

ಈ ಜಗತ್ತಿನ ಸಮಸ್ಯೆಗೆ ಪರಿಹಾರವೆಲ್ಲಿದೆ? ಯಾರು ಹೊರಗೆ ನೋಡುವರೋ ಅವರಿಗೆ ಉತ್ತರ ದೊರಕಲಾರದು. ಅವರು ಅಂತರ್ಮುಖಿಗಳಾಗಿ ಸತ್ಯವನ್ನು ಅರಿಯಬೇಕಾಗಿದೆ. ಧರ್ಮ ಒಳಗೆ ಇರುವುದು.

ಒಬ್ಬನು ಶಿರವನ್ನು ಕತ್ತರಿಸಿದರೆ ನಿನಗೆ ಮುಕ್ತಿ ದೊರೆಯುವುದು ಎಂದು ಬೋಧಿಸುತ್ತಾನೆ. ಆದರೆ ಅವನಿಗೆ ಯಾರಾದರೂ ಹಿಂಬಾಲಕರು ದೊರಕುತ್ತಾರೆಯೇ? ನಿಮ್ಮ ಕ್ರಿಸ್ತನೇ ನಿಮ್ಮಲ್ಲಿರುವುದನ್ನೆಲ್ಲಾ ದೀನರಿಗೆ ಕೊಟ್ಟು ನನ್ನನ್ನು ಅನುಸರಿಸಿ ಎನ್ನುವನು. ನಿಮ್ಮಲ್ಲಿ\break ಎಷ್ಟು ಜನ ಹೀಗೆ ಮಾಡುವಿರಿ? ನೀವು ಅವನ ಆಜ್ಞೆಯನ್ನು ಪಾಲಿಸಲಿಲ್ಲ. ಆದರೂ ಏಸು ನಿಮ್ಮ ಧರ್ಮದಲ್ಲಿ ಮಹಾಗುರುವಾಗಿರುವನು. ನೀವೆಲ್ಲಾ ವ್ಯವಹಾರಚತುರರು. ಆದರೂ\break ಏಸುವಿನ ಬೋಧನೆಯನ್ನು ಅನುಷ್ಠಾನಕ್ಕೆ ತರಲು ಸಾಧ್ಯವಿಲ್ಲ ಎನ್ನುವಿರಿ.

ಆದರೆ ವೇದಾಂತ ಅನುಷ್ಠಾನಕ್ಕೆ ತರಲು ಅಸಾಧ್ಯವಾಗಿರುವುದನ್ನು ಹೇಳುವುದಿಲ್ಲ.\break ಪ್ರತಿಯೊಂದು ವಿಜ್ಞಾನಕ್ಕೂ ಪ್ರಯೋಗಕ್ಕೆ ಒಂದು ವಸ್ತು ಬೇಕಾಗಿದೆ. ಪ್ರತಿಯೊಬ್ಬರಿಗೂ ಅವಕಾಶ, ಅಭ್ಯಾಸ, ಪಾಂಡಿತ್ಯ ಬೇಕಾಗಿದೆ. ಆದರೆ ಯಾವ ದಾರಿಹೋಕಬೇಕಾದರೂ ಧರ್ಮದ ವಿಷಯವನ್ನು ಮಾತನಾಡಬಲ್ಲ. ನೀವು ಧಾರ್ಮಿಕ ಜೀವನದಲ್ಲಿ ಮುಂದುವರಿಯಬೇಕಾದರೆ ಆ ಕ್ಷೇತ್ರದಲ್ಲಿ ಒಬ್ಬ ನಿಪುಣನನ್ನು ಅನುಸರಿಸಬೇಕಾಗಿದೆ. ದಾರಿಹೋಕ ಮಾತನಾಡುವಾಗ ಬೇಕಾದರೆ ಅದನ್ನು ನೀವು ಸ್ವಲ್ಪ ಕೇಳಬಹುದು ಅಷ್ಟೆ.

ವಿಜ್ಞಾನದ ವಿಷಯದಲ್ಲಿ ಹೇಗೋ ಧರ್ಮದ ವಿಷಯದಲ್ಲೂ ಹಾಗೇ ವ್ಯವಹರಿಸಬೇಕು. ನಿಮಗೆ ವಾಸ್ತವಾಂಶಗಳ ಪ್ರತ್ಯಕ್ಷ ಅನುಭವ ಇರಬೇಕು. ಅದರ ಆಧಾರದ ಮೇಲೆ ನೀವು ಸಿದ್ಧಾಂತವನ್ನು ಕಟ್ಟಬೇಕು.

ನಿಜವಾದ ಧರ್ಮ ನಿಮಗೆ ಬೇಕಾದರೆ ಅದಕ್ಕೆ ತಕ್ಕ ಉಪಕರಣಗಳು ನಿಮ್ಮಲ್ಲಿ ಇರಬೇಕು. ನಂಬಿಕೆಯ ಪ್ರಶ್ನೆಯೇ ಏಳುವುದಿಲ್ಲ. ನಂಬಿಕೆಯಿಂದ ಏನೂ ಆಗುವುದಿಲ್ಲ. ಏಕೆಂದರೆ ನೀವು ಏನನ್ನು ಬೇಕಾದರೂ ನಂಬಬಹುದು.

ವಿಜ್ಞಾನ ಶಾಸ್ತ್ರದಲ್ಲಿ ನಾವು ವೇಗವನ್ನು ಹೆಚ್ಚಿಸಿದಂತೆ ಘನರಾಶಿ \enginline{(Mass)} ಕಡಿಮೆಯಾಗುವುದು. ನಾವು ಘನರಾಶಿಯನ್ನು \enginline{(Mass)} ಹೆಚ್ಚಿಸಿದರೆ ವೇಗ ಕಡಮೆಯಾಗುವುದು ಎಂಬುದು ನಮಗೆ ಗೊತ್ತಿದೆ. ಹೀಗೆ ನಮ್ಮಲ್ಲಿ ದ್ರವ್ಯ ಮತ್ತು ಶಕ್ತಿ ಎರಡೂ ಇವೆ. ದ್ರವ್ಯವು ಹೇಗೋ ಶಕ್ತಿಯಲ್ಲಿ ಪರ್ಯವಸಾನವಾಗುವುದು ಶಕ್ತಿಯು ದ್ರವ್ಯವಾಗುವುದು. ದ್ರವ್ಯ ಮತ್ತು ಶಕ್ತಿ ಅಲ್ಲದೆ ಇರುವುದು ಮತ್ತಾವುದೋ ಇರಬೇಕಾಗುವುದು. ಇದನ್ನೇ ನಾವು ಮನಸ್ಸು ಎನ್ನುವುದು. ಇದೇ ವಿಶ್ವಮನಸ್ಸು.

ನೀವು ನಿಮ್ಮ ದೇಹ ಬೇರೆ, ನನ್ನ ದೇಹ ಬೇರೆ ಎನ್ನುವಿರಿ. ಮಾನವಕೋಟಿ ಎಂಬ ಮಹಾಸಾಗರದಲ್ಲಿ ನಾನು ಒಂದು ಸಣ್ಣ ಸುಳಿ. ಇದೊಂದು ಸಣ್ಣ ಸುಳಿಯಾದರೂ ಒಂದು ಅನಂತಸಾಗರದ ಅಂಶವಾಗಿದೆ.

ನೀವು ಹರಿಯುತ್ತಿರುವ ಒಂದು ನದಿಯ ಸಮೀಪದಲ್ಲಿ ನಿಂತರೆ ಅಲ್ಲಿ ಪ್ರತಿಯೊಂದು ನೀರಿನ ಕಣವೂ ಚಲಿಸುತ್ತಿರುವುದು ಕಾಣುತ್ತದೆ. ಆದರೂ ನಾವು ಅದನ್ನು ಒಂದು ನದಿ ಎನ್ನುವೆವು. ನೀರು ಬದಲಾಗುತ್ತಿದೆ, ಆದರೆ ದಡ ಸ್ಥಿರವಾಗಿರುವುದು. ಮನಸ್ಸು ಅಷ್ಟು ಬದಲಾಗುತ್ತಿಲ್ಲ. ಆದರೆ ದೇಹ ಎಷ್ಟು ಬೇಗ ಬದಲಾಗುವುದು! ನಾನೊಂದು ಮಗುವಾಗಿದ್ದೆ. ಅನಂತರ ಹುಡುಗನಾದೆ, ಯುವಕನಾದೆ. ಹಾಗೆಯೆ ಇನ್ನು ಕೆಲವು ದಿನಗಳಲ್ಲಿ ತತ್ತರಿಸುವ ವೃದ್ಧನಾಗುವೆ. ದೇಹ ಬದಲಾಗುತ್ತಿದೆ. ಆದರೆ ಮನಸ್ಸು ಬದಲಾಗುತ್ತಿಲ್ಲವೆ? ಎಂದು ನೀವು ಪ್ರಶ್ನಿಸಬಹುದು. ಹೌದು ಅದೂ ವಿಸ್ತಾರವಾಗಿದೆ. ಮಗುವಾಗಿದ್ದಾಗ ನಾನು ಆಲೋಚಿಸುತ್ತಿದ್ದೆ. ಈ ನನ್ನ ಮನಸ್ಸು ನೆನಪಿನ ಒಂದು ಮಹಾ ಸಾಗರವೇ ಆಗಿದೆ.

ಪ್ರಕೃತಿಯ ಹಿಂದೆ ಒಂದು ವಿಶ್ವಮನಸ್ಸಿದೆ. ಆತ್ಮ ಬರಿಯ ಒಂದು ವ್ಯಷ್ಟಿ, ಅದು ದ್ರವ್ಯವಲ್ಲ. ಮಾನವನು ನಿಜವಾಗಿ ಆತ್ಮನಾಗಿರುವನು. ಸತ್ತ ಮೇಲೆ ಆತ್ಮ ಎಲ್ಲಿಗೆ ಹೋಗುವುದು ಎಂಬ ಪ್ರಶ್ನೆಗೆ, ಭೂಮಿ ಏತಕ್ಕೆ ಬೀಳುವುದಿಲ್ಲ ಎಂದು ಕೇಳಿದ ಪ್ರಶ್ನೆಗೆ ಹುಡುಗ ಕೊಟ್ಟ ಉತ್ತರವನ್ನೇ ಕೊಡಬೇಕಾಗಿದೆ. ಭೂಮಿ ಬೀಳುವುದು ಎಲ್ಲಿಗೆ ಹಾಗೆಯೇ ಆತ್ಮ ಹೋಗುವುದು ಎಲ್ಲಿಗೆ? ಪ್ರಶ್ನೆಗಳೂ ಒಂದೇ, ಅವಕ್ಕೆ ಉತ್ತರಗಳೂ ಒಂದೇ.

ನಿಮ್ಮಲ್ಲಿ ಯಾರು ಕಾಲವಾದ ಅನಂತರ ಎಂದೆಂದಿಗೂ ಇರುತ್ತೇವೆ ಎಂದು ಭಾವಿಸುವರೋ ಅವರಿಗೆ ‘ಮನೆಗೆ ಹೋಗಿ ನೀವು ಸತ್ತುಹೋದಿರಿ ಎಂದು ಭಾವಿಸಲು ಯತ್ನಿಸಿ’ ಎಂದು ಹೇಳುತ್ತೇನೆ. ಹತ್ತಿರ ನಿಂತುಕೊಂಡು ನಿಮ್ಮ ಸತ್ತದೇಹವನ್ನು ಮುಟ್ಟಿನೋಡಲು ಯತ್ನಿಸಿ. ಇದು ನಿಮಗೆ ಸಾಧ್ಯವಿಲ್ಲ. ನೀವು ದೇಹದಿಂದ ಹೊರಗೆ ಹೋಗಲಾರಿರಿ. ಇಲ್ಲಿ ಅಮರತ್ವದ ಪ್ರಶ್ನೆಯಲ್ಲ ಇರುವುದು, ಆದರೆ ಸತ್ತಮೇಲೆ ಗಂಡನಿಗೆ ಅಲ್ಲಿ ಹೆಂಡತಿ ಸಿಕ್ಕುತ್ತಾಳೆಯೇ ಎಂಬುದೇ ಅವರ ಮನಸ್ಸಿನಲ್ಲಿರುವುದು.

ನೀವೊಂದು ಆತ್ಮ ಎಂಬುದನ್ನು ನೀವೇ ಪ್ರತ್ಯಕ್ಷ ತಿಳಿಯಬೇಕಾಗಿದೆ ಇದೇ. ಧರ್ಮದ ಒಂದು ದೊಡ್ಡ ರಹಸ್ಯ. ನಾನೊಂದು ಕೀಟ, ನಾನು ಯಾವುದಕ್ಕೂ ಪ್ರಯೋಜನವಿಲ್ಲ\break ಎಂದು ಗೋಳಿಡಬೇಡಿ. ನಮ್ಮ ಋಷಿಗಳು ಸಾರುವಂತೆ ನಾವು ಸಚ್ಚಿದಾನಂದ ಸ್ವರೂಪರು. ನಾನು ಪಾಪಾತ್ಮ ಎಂದು ಒರಲುತ್ತಿದ್ದರೆ ಇದರಿಂದ ಪ್ರಪಂಚಕ್ಕೆ ಯಾವ ಪ್ರಯೋಜನವೂ ಆಗುವುದಿಲ್ಲ. ನೀವು ಪೂರ್ಣಾತ್ಮರಾದಷ್ಟೂ ಅಪೂರ್ಣತೆಯನ್ನು ನೋಡುವುದು ಕಡಮೆಯಾಗುವುದು.

