
\chapter[ಅವತಾರ ]{ಅವತಾರ \protect\footnote{\engfoot{C.W. Vol. VIII. P190}}}

ಏಸುವು ದೇವರಾಗಿದ್ದನು–ಮನುಷ್ಯರೂಪವನ್ನು ಧರಿಸಿದ ಸಾಕಾರ ದೇವರಾಗಿದ್ದನು.\break ಅವನು ಅನೇಕ ವೇಳೆ ಬೇರೆ ಬೇರೆ ರೂಪಗಳಲ್ಲಿ ಪ್ರಪಂಚದಲ್ಲಿ ಅವತಾರ ಮಾಡಿರುವನು. ಈ ಅವತಾರವನ್ನು ಮಾತ್ರ ನಾವು ಪೂಜಿಸಬಲ್ಲೆವು. ಭಗವಂತನ ನಿರ್ಗುಣ ಸ್ವಭಾವವನ್ನು ನಾವು ಪೂಜಿಸಲಾರೆವು. ಇಂತಹ ದೇವರ ಪೂಜೆಗೆ ಅರ್ಥವಿಲ್ಲ. ನಾವು ಮಾನವರೂಪವನ್ನು ಧರಿಸಿರುವ ಏಸುವನ್ನೇ ದೇವರೆಂದು ಪೂಜಿಸಬೇಕು. ಅವತಾರಕ್ಕಿಂತ ಮಿಗಿಲಾದ ಯಾವುದನ್ನೂ ನೀವು ಪೂಜಿಸಲಾರಿರಿ. ನೀವು ಸೃಷ್ಟಿಸಿರುವ ಯಹೋವನನ್ನೂ ಸುಂದರವಾದ ಏಸುವನ್ನು ಹೋಲಿಸಿ ನೋಡಿ. ದೇವರು ಬೇರೆ ಇರುವನು, ಅವನನ್ನು ನಾವು ಪೂಜಿಸುವೆವು ಎಂಬ ಭಾವನೆಯನ್ನು ನೀವು ಎಷ್ಟು ಬೇಗ ಬಿಟ್ಟರೆ ಅಷ್ಟು ಒಳ್ಳೆಯದು.\break ನೀವು ಏಸುವನ್ನು ಬಿಟ್ಟು ದೇವರನ್ನು ಕಲ್ಪಿಸಿಕೊಳ್ಳುವುದಕ್ಕೆ ಮಾಡುವ ಪ್ರಯತ್ನವೆಲ್ಲ\break ನಿಷ್ಫಲವಾಗುವುದು. ದೇವರು ಮಾತ್ರ ದೇವರನ್ನು ಪೂಜಿಸಬಲ್ಲ. ಮಾನವ ಹಾಗೆ ಮಾಡಲಾರ. ಈಶ್ವರನ ಸಾಮಾನ್ಯ ಆವಿರ್ಭಾವಕ್ಕೆ ಅತೀತವಾಗಿ ನಾವು ಅವನನ್ನೇ ಉಪಾಸನೆ ಮಾಡಲು ಯತ್ನಿಸಿದರೆ ಇದರಿಂದ ಮಾನವಕೋಟಿಗೆ ಅಪಾಯವಿದೆ. ನಿಮಗೆ ಮುಕ್ತಿ ಬೇಕಾದರೆ ಏಸುವನ್ನೇ ಆರಾಧಿಸಿ. ನೀವು ಕಲ್ಪಿಸಿಕೊಳ್ಳಬಲ್ಲ ಎಲ್ಲಾ ದೇವರ ಭಾವನೆಗಳಿಗಿಂತ ಮಿಗಿಲಾಗಿರುವನು ಆತ. ಕ್ರಿಸ್ತನನ್ನು ಮಾನವ ಎಂದು ನೀವು ಭಾವಿಸಿದರೆ ಅವನನ್ನು ಪೂಜಿಸಬೇಕಾಗಿಲ್ಲ. ಅವನು ದೇವರು ಎಂದು ಭಾವಿಸಿದರೆ ಪೂಜಿಸಿ. ಯಾರು ಅವನನ್ನು ಮಾನವ ಎಂದು ಕರೆದು ಪೂಜಿಸುವರೋ ಅವರು ಈಶ್ವರ ನಿಂದೆಯನ್ನು ಮಾಡುತ್ತಿರುವರು. ಅರ್ಧಂಬರ್ಧ ಒಪ್ಪಿಕೊಳ್ಳುವುದಕ್ಕೆ ಆಗುವುದಿಲ್ಲ; ಒಪ್ಪಿಕೊಂಡರೆ ಪೂರ್ತಿ ಒಪ್ಪಿಕೊಳ್ಳಬೇಕು. “ಯಾರು ಮಗನನ್ನು ನೋಡಿರುವರೋ ಅವರು ತಂದೆಯನ್ನೂ ನೋಡಿರುವರು.” ನೀವು ಮಗನನ್ನು ನೋಡದೆ ತಂದೆಯನ್ನು ನೋಡಲಾರಿರಿ. ಅದು ಬರಿಯ ಮಾತಾಗುವುದು, ಒಣತತ್ತ್ವವಾಗುವುದು, ಭ್ರಾಂತಿಯಾಗುವುದು; ಕೇವಲ ಊಹೆ ಮಾತ್ರ ಆಗುವುದು. ಆಧ್ಯಾತ್ಮಿಕ ಜೀವನದಲ್ಲಿ ಮುಂದುವರಿಯಬೇಕೆಂಬ ಇಚ್ಛೆ ಇದ್ದರೆ ಕ್ರಿಸ್ತನ ಮೂಲಕ ಆವಿರ್ಭವಿಸಿರುವ ದೇವರನ್ನು ನೆಚ್ಚಿ.

ತಾತ್ತ್ವಿಕವಾಗಿ ಮಾತನಾಡುವುದಾದರೆ ಕ್ರಿಸ್ತ ಅಥವಾ ಬುದ್ಧರೆಂಬ ವ್ಯಕ್ತಿಗಳು ಎಂದೂ ಇರಲಿಲ್ಲ. ಅವರ ಮೂಲಕ ನಾವು ದೇವರನ್ನು ನೋಡಿದೆವು. ಕುರಾನಿನಲ್ಲಿ ಮಹಮ್ಮದ್​ ಪದೇ ಪದೇ ‘ಕ್ರಿಸ್ತನನ್ನು ಶಿಲುಬೆಗೆ ಏರಿಸಲಿಲ್ಲ, ಅದೊಂದು ತೋರಿಕೆ ಅಷ್ಟೆ’ ಎನ್ನುವನು. ಯಾರೂ ಕ್ರಿಸ್ತನನ್ನು ಶಿಲುಬೆಗೆ ಏರಿಸಲಾರರು.

\eject

ದಾರ್ಶನಿಕ ರೀತಿಯಿಂದ ಅತಿ ಕೆಳಗಿರುವುದೇ ದ್ವೈತ; ಅತ್ಯಂತ ಮೇಲಿರುವುದೇ ಅದ್ವೈತದಲ್ಲಿ ಮೂರೂ ಒಂದಾಗಿರುವುದು. ಪ್ರಕೃತಿ ಮತ್ತು ಜೀವ ಇವರಲ್ಲಿ ಈಶ್ವರನು ಹಾಸುಹೊಕ್ಕಾಗಿರುವನು. ಇದನ್ನೇ ನಾವು ಈಶ್ವರ–ಜೀವ–ಜಗತ್ತು ಇವುಗಳ ತ್ರಿಮೂರ್ತಿ ಎನ್ನುವುದು. ಅದೇ ಕಾಲದಲ್ಲಿ ಈ ಮೂರೂ ಒಂದೇ ವಸ್ತುವಿನ ಆವಿರ್ಭಾವ ಎಂಬ ಮಿಂಚುನೋಟವೂ ದೊರಕುವುದು. ಹೇಗೆ ಈ ದೇಹವು ಆತ್ಮನ ಹೊದಿಕೆಯೋ ಹಾಗೆಯೇ ಈ ಆತ್ಮವು ದೇವರ ದೇಹ. ಹೇಗೆ ನಾನು ಈ ದೇಹದ ಜೀವವೋ ಹಾಗೆಯೇ ದೇವರು ಈ ಜೀವದ ಜೀವ. ನೀವು ಒಂದು ಕೇಂದ್ರ. ಆ ಕೇಂದ್ರದ ಮೂಲಕ ನೀವು ನೀವಿರುವ ಪ್ರಕೃತಿಯನ್ನು ನೋಡುತ್ತಿರುವಿರಿ. ಜೀವ, ಜಗತ್ತು, ಈಶ್ವರ ಎಲ್ಲಾ ಸೇರಿ ಏಕವಾಗಿರುವ ವಿಶ್ವವಾಗುವುದು. ಆದಕಾರಣ ಇವು ಏಕ, ಆದರೂ ಬೇರೆ ಬೇರೆ. ಆನಂತರ ಬೇರೊಂದು ಬಗೆಯ ತ್ರಿಮೂರ್ತಿಗಳ ಭಾವನೆ ಇದೆ; ಇದು ಬಹುಮಟ್ಟಿಗೆ ಕ್ರೈಸ್ತರ ಭಾವನೆಯಂತೆ ಇದೆ. ದೇವರು ನಿರ್ಗುಣನು. ಅವನ ನಿರ್ಗುಣತ್ವವನ್ನು ನಾವು ಅರಿಯಲಾರೆವು, ಅದನ್ನು ನೇತಿ ನೇತಿ ಎಂದು ಮಾತ್ರ ಕರೆಯಬಹುದು. ಆದರೂ ಭಗವಂತನಿಗೆ ಅತ್ಯಂತ ಸಮೀಪದಲ್ಲಿರುವ ಕೆಲವು ಗುಣಗಳು ನಮಗೆ ದೊರಕುವುವು. ಅದೇ ಸಚ್ಚಿದಾನಂದ– ಕ್ರೈಸ್ತರಲ್ಲಿರುವ ತಂದೆ, ಮಗ, ಹೋಲಿಘೋಸ್ಟ್​ ಎಂಬ ಮೂರು ಭಾವನೆಗಳಂತೆ. ತಂದೆಯೇ ಸತ್ಯ, ಅದರಿಂದ ಪ್ರಪಂಚದಲ್ಲಿ ಎಲ್ಲವೂ ಬರುವುದು. ಮಗನೇ ಚಿತ್​ (ಜ್ಞಾನ), ಮಗನಲ್ಲಿ ದೇವರು ಕಾಣಿಸುವನು. ಕ್ರಿಸ್ತನಿಗೆ ಮುಂಚೆ ದೇವರು ಎಲ್ಲರಲ್ಲಿಯೂ ಎಲ್ಲಾ ಕಡೆಗಳಲ್ಲಿಯೂ ಇದ್ದನು. ಆದರೆ ಕ್ರಿಸ್ತನಲ್ಲಿ ಅವನು ನಮಗೆ ವ್ಯಕ್ತವಾಗುವನು. ಇದೇ ದೇವರು. ಮೂರನೆಯದೇ ಆನಂದ–ಹೋಲಿಘೋಸ್ಟ್​ ಎಂಬುದು. ನಿಮಗೆ ಜ್ಞಾನ ಪ್ರಾಪ್ತವಾದೊಡನೆ ಆನಂದ ಪ್ರಾಪ್ತವಾಗುವುದು. ಕ್ರಿಸ್ತ ಯಾವಾಗ ನಿಮ್ಮಲ್ಲಿರುವನು ಎಂದು ಅರಿಯುವಿರೋ ಆಗ ಆನಂದ ಬರುವುದು. ಮೂವರೂ ಅಲ್ಲಿ ಐಕ್ಯವಾಗುವರು.

\vskip -0.5cm

