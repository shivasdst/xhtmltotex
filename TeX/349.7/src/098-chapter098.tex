
\chapter[ಇಂಗ್ಲೆಂಡಿಗೆ ಭಾರತೀಯ ಪ್ರಚಾರಕರ ಸಂದೇಶ ]{ಇಂಗ್ಲೆಂಡಿಗೆ ಭಾರತೀಯ ಪ್ರಚಾರಕರ ಸಂದೇಶ \protect\footnote{\engfoot{C.W. Vol. V, p. 201}}}

\centerline{\textbf{(“ದಿ ಇಕೋ,” ಲಂಡನ್​ 1896)}}

ಸ್ವಾಮಿಗಳು ತಮ್ಮ ದೇಶದಲ್ಲಿ ಮುಂಡನ ಮಾಡಿಸಿಕೊಂಡು ಕಾವಿ ಬಟ್ಟೆ ಧರಿಸಿ ಒಂದು ಮರದ ಕೆಳಗೋ, ಹೆಚ್ಚೆಂದರೆ ದೇವಸ್ಥಾನದ ಆವರಣದಲ್ಲೋ ಇರುವರು ಎಂದು ಭಾವಿಸುತ್ತೇನೆ.\break ಆದರೆ ಲಂಡನ್ನಿನಲ್ಲಿ ಹಾಗೇನೂ ಇರಲಿಲ್ಲ. ಸಾಧಾರಣವಾಗಿ ಎಲ್ಲಾ ಜನರೂ ಇರುವಂತೆಯೇ ಇದ್ದರು. ಕಾವಿಯ ನಿಲುವಂಗಿಯಲ್ಲದೆ ಬೇರಾವ ವಿಶೇಷವೂ ಅವರ ಉಡುಗೆತೊಡಿಗೆಯಲ್ಲಿ ಕಾಣಲಿಲ್ಲ. ತಾವು ರುಮಾಲನ್ನು ಸುತ್ತಿಕೊಂಡರೆ ಲಂಡನ್ನಿನ ರಸ್ತೆಯಲ್ಲಿ ಹೋಗುವ ಜನರಿಗೆ ಹಾಸ್ಯಾಸ್ಪದವಾಗಿ ಕಾಣುವುದು; ಅವರು ಈ ವೇಷವನ್ನು ನೋಡಿ ಏನು ಮಾತಾಡಿಕೊಳ್ಳುವರೋ ಅದನ್ನು ಹೇಳುವುದು ಉಚಿತವಲ್ಲ ಎಂದು ನಗುತ್ತಾ ಹೇಳಿದರು. ಇಂಡಿಯಾ ದೇಶದ ಯೋಗಿಗಳನ್ನು ನಿಧಾನವಾಗಿ ನಿಮ್ಮ ಹೆಸರಿನ ಕಾಗುಣಿತವನ್ನು ಹೇಳಿ ಎಂದು ಕೇಳಿದೆ. ಅವರು ಬಿಡಿಬಿಡಿಯಾಗಿ ಉಚ್ಚರಿಸಿದರು.

\vskip 5pt

ಬಾತ್ಮೀದಾರ: “ಜನ ಈಗಿನ ಕಾಲದಲ್ಲಿ ಗೌಣ ವಿಷಯಗಳಿಗೆ ಹೆಚ್ಚು ಪ್ರಾಮುಖ್ಯವನ್ನು ಕೊಡುತ್ತಿರುವರು ಎಂದು ಭಾವಿಸುವಿರಾ?”

\vskip 5pt

ಸ್ವಾಮೀಜಿ: ಹೌದು, ಹಿಂದುಳಿದ ರಾಷ್ಟ್ರಗಳಲ್ಲಿ, ಹಾಗೂ ಪಶ್ಚಿಮದಲ್ಲಿ ಹೆಚ್ಚು ಸುಸಂಸ್ಕೃತರಲ್ಲದ ಜನರಲ್ಲಿ ಅದು ಇರುವುದು ಹಾಗೇ. ನಿಮ್ಮ ಪ್ರಶ್ನೆಯಿಂದ ಸುಸಂಸ್ಕೃತರ ಮತ್ತು ಶ‍್ರೀಮಂತರ ಪರಿಸ್ಥಿತಿ ಬೇರೆ ರೀತಿಯಲ್ಲಿದೆ ಎಂದು ನೀವು ಭಾವಿಸಿರುವಂತೆ ತೋರುತ್ತದೆ.\break ಅದು ಹೌದು. ಶ‍್ರೀಮಂತರು ಭೋಗದಲ್ಲಿ ಮುಳುಗಿರುವರು, ಇಲ್ಲವೆ ಹೆಚ್ಚು ಆಸ್ತಿಯನ್ನು ದೋಚುವುದರಲ್ಲಿ ಇರುವರು. ಅವರು ಮತ್ತು ಬಿಡುವಿಲ್ಲದ ಅನೇಕ ಜನ, ಧರ್ಮ ನಿಷ್ಪ್ರಯೋಜಕವಾದುದು, ಕೆಲಸಕ್ಕೆ ಬಾರದುದು, ಕೊಳೆತುಹೋಗಿದೆ ಎಂದು ನಿಜವಾಗಿ ಭಾವಿಸುವರು. ಈಗಿನ ಕಾಲದ ಶೋಕಿಯ ಒಂದು ಧರ್ಮವೇ ದೇಶಭಕ್ತಿ ಮತ್ತು ಸ್ತ್ರೀಯರ ಸಮಸ್ಯೆಗಳ ಚರ್ಚೆ. ಜನ ಮದುವೆ ಸಮಯದಲ್ಲಿ ಅಥವಾ ಯಾರಾದರೂ ಸತ್ತಾಗ ಸಮಾಧಿ ಮಾಡುವುದಕ್ಕೆ ಮಾತ್ರ ಚರ್ಚಿಗೆ ಹೋಗುತ್ತಿರುವರು.”

\vskip 3pt

ಬಾತ್ಮೀದಾರ: ‘ನಿಮ್ಮ ಸಂದೇಶ ಅವರನ್ನು ಚರ್ಚಿಗೆ ಹೆಚ್ಚಾಗಿ ಆಕರ್ಷಿಸುವುದೇ?

\vskip 2pt

ಸ್ವಾಮೀಜಿ: “ಹಾಗೆ ಮಾಡುತ್ತಿದೆ ಎಂದು ನಾನು ಭಾವಿಸುವುದಿಲ್ಲ. ನನಗೂ ಯಾವ ಆಚಾರಗಳಿಗೂ, ಮೂಢನಂಬಿಕೆಗಳಿಗೂ ಸಂಬಂಧವಿಲ್ಲ. ನನ್ನ ಉದ್ದೇಶವೆಂದರೆ ಧರ್ಮವೇ ಎಲ್ಲವೂ ಮತ್ತು ಎಲ್ಲದರಲ್ಲಿಯೂ ಧರ್ಮವಿದೆ ಎಂಬುದನ್ನು ತೋರಿಸುವುದು. ಇಂಗ್ಲೆಂಡಿನಲ್ಲಿ ಪ್ರಚಲಿತವಾಗಿರುವ ವ್ಯವಸ್ಥೆಯ ಬಗ್ಗೆ ಏನನ್ನು ತಾನೆ ನಾವು ಹೇಳಬಹುದು? ಪರಿಸ್ಥಿತಿಯನ್ನು ನೋಡಿದರೆ ಸಮಾಜವಾದ ಅಥವಾ ಪ್ರಜಾಸತ್ತಾತ್ಮಕವಾದ ಯಾವುದೋ ಒಂದು ಈ ದೇಶಕ್ಕೆ ಬರುತ್ತಿರುವಂತೆ ಕಾಣುವುದು. ಜನರ ಪ್ರಾಪಂಚಿಕ ಬಯಕೆ ಈಡೇರಬೇಕು. ಕಡಮೆ ಕೆಲಸವಿರಬೇಕು, ಯಾರೂ ತಮ್ಮನ್ನು ಆಳಕೂಡದು, ಯುದ್ಧವಿರಕೂಡದು, ಹೆಚ್ಚು ಊಟ ಸಿಕ್ಕಬೇಕು. ಧರ್ಮದ ತಳಹದಿಯ ಮೇಲೆ, ಜನರ ಒಳ್ಳೆಯತನದ ಮೇಲೆ ನಿಲ್ಲದೇ ಇದ್ದರೆ ಈ ನಾಗರಿಕತೆಯೋ ಅಥವಾ ಮತ್ತಾವ ನಾಗರಿಕತೆಯೋ ಊರ್ಜಿತವಾಗಿ ನಿಲ್ಲಬಲ್ಲದು ಎಂದು ಹೇಗೆ ದೃಢವಾಗಿ ಹೇಳುವುದು? ಇದನ್ನೆಲ್ಲಾ ನೆನಪಿನಲ್ಲಿಡಿ. ಧರ್ಮವು ಸಮಸ್ಯೆಯ ಮೂಲಕ್ಕೆ ಹೋಗುವುದು. ಅದು ಸರಿಯಾಗಿದ್ದರೆ ಎಲ್ಲವೂ ಸರಿಯಾಗಿರುವುದು.”

\vskip 2pt

ಬಾತ್ಮೀದಾರ: “ಸಾಧಾರಣ ಜನರಿಗೆ ಧರ್ಮದ ಸಾರವಾದ ತತ್ತ್ವವನ್ನು ಕೊಡುವುದು ಬಹಳ ಕಷ್ಟ. ಇದು ಅವರ ಭಾವನೆ ಮತ್ತು ಜೀವನ ವಿಧಾನಕ್ಕೆ ನಿಲುಕಲಾರದು.”

\vskip 2pt

ಸ್ವಾಮೀಜಿ: “ಎಲ್ಲಾ ಧರ್ಮಗಳಲ್ಲಿಯೂ ನಾವು ಕೆಳಗಿನ ಸತ್ಯದಿಂದ ಮೇಲಿನ ಸತ್ಯಕ್ಕೆ ಹೋಗುತ್ತೇವೆಯೇ ಹೊರತು ಎಂದಿಗೂ ಅಸತ್ಯದಿಂದ ಸತ್ಯಕ್ಕೆ ಹೋಗುವುದಿಲ್ಲ. ಸೃಷ್ಟಿಯ ಹಿಂದೆಲ್ಲ ಒಂದು ಏಕತೆ ಇದೆ. ಆದರೆ ಮನಸ್ಸುಗಳು ವಿಧವಿಧವಾಗಿರುವುವು. “ಏಕಂ ಸತ್​ ವಿಪ್ರಾಃ ಬಹುಧಾ ವದನ್ತಿ” –ಇರುವುದು ಒಂದೇ, ಜ್ಞಾನಿಗಳು ಅದನ್ನು ಬೇರೆ ಬೇರೆ ಹೆಸರುಗಳಿಂದ ಕರೆಯುತ್ತಾರೆ. ನಾನು ಹೇಳುವುದೇನೆಂದರೆ ವ್ಯಕ್ತಿಯು ಸಣ್ಣ ಸತ್ಯದಿಂದ ದೊಡ್ಡ ಸತ್ಯದೆಡೆಗೆ ಹೋಗುತ್ತಿರುವನು ಎಂದು ತುಂಬಾ ಹೀನಸ್ಥಿತಿಯಲ್ಲಿರುವ ಧರ್ಮಗಳೂ ಕೂಡ ಏಕಮಾತ್ರ ಸತ್ಯವನ್ನು ತಪ್ಪಾಗಿ ತಿಳಿದುಕೊಂಡಿರುವುವು, ಅಷ್ಟೆ. ವ್ಯಕ್ತಿಯು ಕ್ರಮೇಣ ಅರ್ಥಮಾಡಿಕೊಳ್ಳುತ್ತಾ ಹೋಗುವನು. ಭೂತಪ್ರೇತಗಳ ಆರಾಧನೆ ಕೂಡ ನಿತ್ಯವಾದ ಅವಿಕಾರಿಯಾದ ಬ್ರಹ್ಮನನ್ನು ಬೇರೆ ವಿಧವಾಗಿ ನೋಡುವುದಾಗಿದೆ. ಬೇರೆ ಧರ್ಮಗಳಲ್ಲಿಯೂ ಸ್ವಲ್ಪ ಹೆಚ್ಚು ಕಡಮೆ ಸತ್ಯವಿದ್ದೇ ಇರುವುದು. ಯಾವ ಧರ್ಮದಲ್ಲಿಯೂ ಇದು ಪೂರ್ಣವಾಗಿಲ್ಲ.”

\vskip 2pt

ಬಾತ್ಮೀದಾರ: “ನೀವು ಇಂಗ್ಲೆಂಡಿನಲ್ಲಿ ಬೋಧಿಸಲು ಬಂದಿರುವ ಧರ್ಮದ ಮೂಲಶಿಲ್ಪಿಯೆ ಎಂದು ಕೇಳಬಹುದೆ?”

\eject

ಸ್ವಾಮೀಜಿ: “ನಿಜವಾಗಿಯೂ ಇಲ್ಲ. ನಾನು ರಾಮಕೃಷ್ಣ ಪರಮಹಂಸರೆಂಬ ಮಹಾಜ್ಞಾನಿಗಳ ಶಿಷ್ಯ. ಅವರು ಇತರ ಋಷಿಗಳಂತೆ ದೊಡ್ಡ ವಿದ್ವಾಂಸರೇನೂ ಆಗಿರಲಿಲ್ಲ. ಆದರೆ ಪರಿಶುದ್ಧಾತ್ಮರು. ವೇದಾಂತದರ್ಶನ ಅವರ ಬಾಳಿನಲ್ಲಿ ಹಾಸುಹೊಕ್ಕಾಗಿತ್ತು. ನಾನು ದರ್ಶನ ಎಂದರೆ, ಧರ್ಮ ಎಂದು ಹೇಳದೆ ಇರುವುದಕ್ಕೆ ಆಗುವುದಿಲ್ಲ. ಅದು ನಿಜವಾಗಿ ದರ್ಶನ ಮತ್ತು ಧರ್ಮ. ಇತ್ತೀಚೆಗೆ ಪ್ರಕಟವಾದ “ನೈಂಟೀಂತ್​ ಸೆಂಚುರಿ” ಎಂಬ ಪತ್ರಿಕೆಯಲ್ಲಿ ಒಂದು ಲೇಖನದಲ್ಲಿ ಪ್ರೊಫೆಸರ್​ ಮ್ಯಾಕ್ಸ್​ಮುಲ್ಲರ್​ರವರು ಶ‍್ರೀ ರಾಮಕೃಷ್ಣರ ಮೇಲೆ ಬರೆದ ಒಂದು ಲೇಖನವನ್ನು ನೀವು ಓದಲೇಬೇಕು. ಶ‍್ರೀ ರಾಮಕೃಷ್ಣರು ಹೂಗ್ಲೀ ಜಿಲ್ಲೆಯಲ್ಲಿ 1836ರಲ್ಲಿ ಜನ್ಮತಾಳಿ 1886ರಲ್ಲಿ ಸಮಾಧಿಸ್ಥರಾದರು. ಅವರು ಕೇಶವ ಚಂದ್ರಸೇನ ಮತ್ತು ಇತರರ ಮೇಲೆ ದೊಡ್ಡ ಪ್ರಭಾವವನ್ನು ಬೀರಿದರು. ಸಾಧನೆಯಿಂದ ಆಧ್ಯಾತ್ಮಿಕ ಜೀವನದ ಬೆಳಕನ್ನು ಕಂಡರು. ಅವರ ಮುಖದಲ್ಲಿ ಶಿಶು ಸಹಜವಾದ ಸರಳತೆ, ಅದ್ಭುತವಾದ ನಮ್ರತೆ, ಮಾತಿನಲ್ಲಿ ಅಪೂರ್ವವಾದ ಮಾಧುರ್ಯ ಇವು ಇದ್ದುವು. ಯಾರೂ ಅವರ ಪ್ರಭಾವಕ್ಕೆ ಸಿಕ್ಕದೆ ಇರುವುದಕ್ಕೆ ಆಗುತ್ತಿರಲಿಲ್ಲ.”

ಬಾತ್ಮೀದಾರ: “ಹಾಗಾದರೆ ನಿಮ್ಮ ಬೋಧನೆ ವೇದದಿಂದ ಬಂದುದೆ.”

ಸ್ವಾಮೀಜಿ: “ಹೌದು, ವೇದಾಂತ ಎಂದರೆ ವೇದಗಳ ಕೊನೆಯ ಭಾಗ, ವೇದದ ಮೂರನೆಯ ಭಾಗ ಅಥವಾ ಉಪನಿಷತ್​. ಬೇರೆ ಕಡೆ ಇನ್ನೂ ಅಂಕುರಾವಸ್ಥೆಯಲ್ಲಿರುವ ಭಾವನೆ ಇಲ್ಲಿ ಚೆನ್ನಾಗಿ ಸ್ಪಷ್ಟವಾಗಿರುವುದು. ವೇದದ ಅತಿ ಪ್ರಾಚೀನ ಭಾಗವೇ ಸಂಹಿತೆ. ಇದು ಅತಿ ಹಳೆಯ ಸಂಸ್ಕೃತ ಭಾಷೆಯಲ್ಲಿದೆ. ಇದನ್ನು ಯಾಸ್ಕನ ನಿರುಕ್ತವೆಂಬ ನಿಘಂಟಿನ ಮೂಲಕ ಮಾತ್ರ ಅರ್ಥಮಾಡಿಕೊಳ್ಳಬಹುದು.”

ಬಾತ್ಮೀದಾರ: “ನಮ್ಮಂತಹ ಇಂಗ್ಲಿಷರಲ್ಲಿ ಇಂಡಿಯಾ ದೇಶವು ನಮ್ಮಿಂದ ಕಲಿಯುವುದು ಬಹಳ ಇದೆ ಎಂಬ ಭಾವನೆ ಇದೆ. ಸಾಧಾರಣ ಜನರಿಗೆ ಇಂಡಿಯಾ ದೇಶದಿಂದ ಏನನ್ನು ಕಲಿಯಬೇಕು ಎಂಬುದೇ ಗೊತ್ತಿಲ್ಲ.”

ಸ್ವಾಮೀಜಿ: “ಅದೇನೋ ನಿಜ. ಆದರೆ ವಿದ್ವಾಂಸರಿಗೆ ಗೊತ್ತಿದೆ, ಇಂಡಿಯಾ ದೇಶದಿಂದ\break ಕಲಿತುಕೊಳ್ಳುವುದು ಎಷ್ಟೊಂದು ಇದೆ ಎಂಬುದು. ಮ್ಯಾಕ್ಸ್​ಮುಲ್ಲರ್​, ಮೋನಿಯರ್​ ವಿಲಿಯಂಸ್​, ಸರ್​ ವಿಲಿಯಂ ಹಂಟರ್​ ಅಥವಾ ಜರ್ಮನಿಯ ಪ್ರಾಚ್ಯ ಶಾಸ್ತ್ರಜ್ಞರು ಹಿಂದೂಗಳ ದರ್ಶನವನ್ನು ಲಘುವಾಗಿ ಎಣಿಸುವುದಿಲ್ಲ.”

\delimiter

ಸ್ವಾಮಿಗಳು ವಿಕ್ಟೋರಿಯಾ ಬೀದಿಯಲ್ಲಿ 29ನೇ ಮನೆಯಲ್ಲಿ ಉಪನ್ಯಾಸ ಕೊಡುವರು. ಎಲ್ಲರಿಗೂ ಸುಸ್ವಾಗತ. ಹಿಂದಿನ ಮಹಾತ್ಮರ ಕಾಲದಲ್ಲಿದ್ದಂತೆಯೇ ಈ ಹೊಸ ಬೋಧನೆಗೆ ಯಾರೂ ಹಣ ಕೊಡಬೇಕಾಗಿಲ್ಲ. ಹಿಂದೂ ಯೋಗಿ ನೋಡುವುದಕ್ಕೆ ಅಪೂರ್ವ ಭವ್ಯ ವ್ಯಕ್ತಿ. ಇಂಗ್ಲಿಷ್​ ಭಾಷೆಯ ಮೇಲೆ ಅವರಿಗೆ ಪೂರ್ಣ ಪ್ರಭುತ್ವವಿದೆ ಎಂದು ಹೇಳಬಹುದು.

\vspace{-0.5cm}

