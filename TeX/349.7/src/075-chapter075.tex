
\chapter[ಸಂನ್ಯಾಸಿ ]{ಸಂನ್ಯಾಸಿ \protect\footnote{\engfoot{C.W. Vol. V, P. 260}}}

ಬಾಸ್ಟನ್ನಿನಲ್ಲಿ ಒಂದು ಉಪನ್ಯಾಸದ ಸಂದರ್ಭದಲ್ಲಿ ಸ್ವಾಮೀಜಿ “ಸಂನ್ಯಾಸಿ” ಎಂಬ ಪದವನ್ನು ವಿವರಿಸುತ್ತಾ ಈ ಕೆಳಕಂಡಂತೆ ಹೇಳಿದರು:

\vskip 6pt

ಒಬ್ಬ ವ್ಯಕ್ತಿಯು ಯಾವ ಆಶ್ರಯಕ್ಕೆ ಸೇರಿರುವನೊ ಅದರ ಕರ್ತವ್ಯಗಳನ್ನೆಲ್ಲ ಪರಿಪಾಲಿಸಿದ ಮೇಲೆ, ಪ್ರಪಂಚದ ಕೀರ್ತಿ ಅಧಿಕಾರ ಮುಂತಾದ ಲಾಲಸೆಗಳ ಮೇಲೆ\break ಜುಗುಪ್ಸೆಗೊಂಡು ಅವುಗಳನ್ನು ಸಂಪೂರ್ಣ ತ್ಯಜಿಸಿ ನಿವೃತ್ತಿ ಮಾರ್ಗವನ್ನು ಅನುಸರಿಸುವ\break ಅಭಿಲಾಷೆಯುಳ್ಳವನಾಗುವನು. ಪ್ರಪಂಚದ ವಸ್ತುಗಳ ಸ್ವಭಾವ ಅವನಿಗೆ ಚೆನ್ನಾಗಿ\break ಗೊತ್ತಾದಂತೆ ಪ್ರಪಂಚದ ಕ್ಷಣಿಕತೆ, ಹೋರಾಟ, ದುಃಖ ಅದರಿಂದ ಬರುವ ಅಪ್ರಯೋಜಕ\break ವಸ್ತುಗಳು, ಇವುಗಳಿಂದೆಲ್ಲ ವಿಮುಖನಾಗಿ ಪರಂಧಾಮ ಸ್ವರೂಪವಾದ ಸತ್ಯವನ್ನು, ಸನಾತನ ಪ್ರೇಮವನ್ನು ಅರಸುವನು. ಆಗ ಅವನು ಪ್ರಪಂಚಕ್ಕೆ ಸಂಬಂಧಪಟ್ಟ ಅಂತಸ್ತು ಐಶ್ವರ್ಯ ಕೀರ್ತಿ ಇವೆಲ್ಲವನ್ನೂ ತ್ಯಜಿಸಿ ಪರಿವ್ರಾಜಕನಾಗಿ ಸಂಚರಿಸುವನು. ಆಧ್ಯಾತ್ಮಿಕ ಜ್ಞಾನವನ್ನು ಪಡೆಯಲು ಪದೇ ಪದೇ ಹೋರಾಡುವನು. ಪ್ರೀತಿ ದಯೆ ಮತ್ತು ಅಂತರ್ದೃಷ್ಟಿ ಇವನ್ನು ಪಡೆಯಲು ಸಾಧನೆ ಮಾಡುವನು. ಹಲವು ವರುಷಗಳ ಧ್ಯಾನ ನಿಯಮಬದ್ಧ ಜೀವನ ಮತ್ತು ವಿಚಾರ ಇವುಗಳ ಮೂಲಕ ಜ್ಞಾನಮಾಣಿಕ್ಯಗಳನ್ನು ಪಡೆಯುವನು. ಅವನು ಅನಂತರ ಒಬ್ಬ ಗುರುವಾಗಿ ಆಧ್ಯಾತ್ಮಿಕ ಜ್ಞಾನವನ್ನು ಬಯಸುವ ಯೋಗ್ಯ ಶಿಷ್ಯನಿಗೆ (ಗೃಹಸ್ಥ ಇಲ್ಲವೆ ಸಂನ್ಯಾಸಿ) ತನ್ನ ಅನುಭವವನ್ನು ಮತ್ತು ಜ್ಞಾನವನ್ನೆಲ್ಲ ಧಾರೆ ಎರೆದುಕೊಡುವನು.

\vskip 6pt

ಸಂನ್ಯಾಸಿ ಯಾವ ಧರ್ಮಕ್ಕೂ ಸೇರಲಾರ. ಏಕೆಂದರೆ ಅವನು ಸ್ವತಂತ್ರ. ಅವನು ಒಳ್ಳೆಯದಾದುದನ್ನು ಎಲ್ಲಾ ಧರ್ಮಗಳಿಂದಲೂ ಸ್ವೀಕರಿಸುವನು. ಸಾಕ್ಷಾತ್ಕಾರಕ್ಕೆ ಧಾರೆ ಎರೆದ ಜೀವನ ಅವನದು; ಅದು ಬರಿಯ ಸಿದ್ಧಾಂತವಲ್ಲ, ಮೂಢನಂಬಿಕೆಯೂ ಅಲ್ಲ.

