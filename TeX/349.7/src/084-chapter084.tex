
\chapter[ಜ್ಞಾನ ಮತ್ತು ಕರ್ಮ ]{ಜ್ಞಾನ ಮತ್ತು ಕರ್ಮ \protect\footnote{\engfoot{C.W. Vol. VIII, P. 225}}}

\centerline{\textbf{(1895ರ ನವಂಬರ್​ 23ರಂದು ಲಂಡನ್ನಿನಲ್ಲಿ ನೀಡಿದ ಉಪನ್ಯಾಸದ ಟಿಪ್ಪಣಿ)}}

\vskip 0.2cm

ಆಲೋಚನೆಯಿಂದ ಅದ್ಭುತಶಕ್ತಿ ಹೊರಹೊಮ್ಮುತ್ತದೆ. ವಸ್ತು ಸೂಕ್ಷ್ಮವಾದಷ್ಟೂ ಹೆಚ್ಚು ಶಕ್ತಿಶಾಲಿ. ಆಲೋಚನೆಯ ಮೌನಶಕ್ತಿ ದೂರದಲ್ಲಿರುವ ಜನರ ಮೇಲೆಯೂ ತನ್ನ ಪ್ರಭಾವವನ್ನು ಬೀರುವುದು. ಏಕೆಂದರೆ ಮನಸ್ಸು ಏಕಕಾಲದಲ್ಲೇ ಏಕ ಮತ್ತು ಅನೇಕ ಆಗಿರುವುದು. ವಿಶ್ವ ಒಂದು ಜೇಡರ ಬಲೆ. ಮನಸ್ಸುಗಳು ಜೇಡರ ಹುಳುಗಳು.

ವಿಶ್ವವೇ ವಿಶ್ವೇಶ್ವರನ ಅಭಿವ್ಯಕ್ತಿ. ವಿಶ್ವೇಶ್ವರನನ್ನು ನಮ್ಮ ಇಂದ್ರಿಯಗಳ ಮೂಲಕ\break ನೋಡಿದಾಗ ಅವನು ವಿಶ್ವದಂತೆ ಕಾಣಿಸುವನು. ಇದೇ ಮಾಯೆ. ಆದಕಾರಣ ಜಗತ್ತು ಮಿಥ್ಯೆ; ಎಂದರೆ ಸತ್ಯದ ಅಪೂರ್ಣ ದೃಶ್ಯ. ಬೆಳಗಿನ ಸೂರ್ಯನು ಕೆಂಗೆಂಡದಂತೆ\break ಕಾಣಿಸುವಂತೆ ಇದು. ಎಲ್ಲಾ ಪಾಪ ಮತ್ತು ದೌರ್ಜನ್ಯ ದುರ್ಬಲತೆಯ ಪರಿಣಾಮ,\break ಒಳಿತಿನ ಅಪೂರ್ಣವಾದ ದೃಶ್ಯ.

ಒಂದು ಸರಳರೇಖೆಯನ್ನು ಅಂತ್ಯವಿಲ್ಲದೆ ಮುಂದುವರಿಸಿದರೆ ಅದೊಂದು ವೃತ್ತವಾಗುವುದು. ಭಗವಂತನನ್ನು ಹುಡುಕಿಕೊಂಡು ಹೋದರೆ ಕೊನೆಗೆ ನಾವು ಆತ್ಮನ ಬಳಿಗೇ ಬರಬೇಕಾಗಿದೆ. ನಾನೇ ದೇವರೆಂಬುದು ಮಹಾರಹಸ್ಯ. ನಾನೇ ದೇಹ, ನಾನೇ ಜೀವನಾಗಿರುವುದು. ನಾನೇ ವಿಶ್ವೇಶ್ವರ ಕೂಡ.

ಮನುಷ್ಯ ಏತಕ್ಕೆ ಪರಿಶುದ್ಧನಾಗಿರಬೇಕು; ನೀತಿವಂತನಾಗಿರಬೇಕು? ಏಕೆಂದರೆ ಇದು ಅವನ ಇಚ್ಛಾಶಕ್ತಿಯನ್ನು ಬಲಗೊಳಿಸುವುದು. ನಮ್ಮ ನೈಜಸ್ಥಿತಿಯನ್ನು ವ್ಯಕ್ತಗೊಳಿಸಿ\break ಯಾವುದು ನಮ್ಮ ಇಚ್ಛೆಯನ್ನು ಬಲಪಡಿಸುವುದೋ ಅದೆಲ್ಲಾ ನೀತಿಯುತವಾದುದು. ಯಾವುದು ಅದಕ್ಕೆ ವಿರುದ್ಧವಾಗಿರುವುದೋ ಅದೆಲ್ಲಾ ಅನೀತಿ. ನೀತಿಯ ಅಳತೆಗೋಲು ದೇಶದಿಂದ ದೇಶಕ್ಕೆ, ವ್ಯಕ್ತಿಯಿಂದ ವ್ಯಕ್ತಿಗೆ ಬದಲಾಗುವುದು. ಮಾನವ, ಪ್ರಕೃತಿಯ ನಿಯಮಗಳ ದಾಸ್ಯದಿಂದ, ಶಬ್ದ ಮುಂತಾದುವುಗಳ ದಾಸ್ಯದಿಂದ ಬಂಧನಗಳಿಂದ\break ಪಾರಾಗಬೇಕಾಗಿದೆ. ಈಗ ನಮಗೆ ಇಚ್ಛಾ ಸ್ವಾತಂತ್ರ್ಯವಿಲ್ಲ ಆದರೆ ನಾವು ಮುಕ್ತರಾದ\break ಮೇಲೆ ಸ್ವಾತಂತ್ರ್ಯ ಲಭಿಸುವುದು. ತ್ಯಾಗ ಎಂದರೆ ಈ ಪ್ರಪಂಚವನ್ನು ತ್ಯಜಿಸುವುದು ಎಂದು ಅರ್ಥ. ಇಂದ್ರಿಯಗಳ ಮೂಲಕ ಕೋಪ ಬರುವುದು, ದುಃಖ ಬರುವುದು. ತ್ಯಾಗವು\break ನಮ್ಮೊಳಗೆ ಇರದಿದ್ದರೆ, ಪ್ರತ್ಯೇಕ ಅಸ್ತಿತ್ವವನ್ನುಳ್ಳ ಜೀವ ಮತ್ತು ಅದನ್ನು ಕೆರಳಿಸುವ ಭಾವೋದ್ರೇಕಗಳು ಕೊನೆಗೆ ಒಂದಾಗಿ ಸೇರಿ ಮಾನವನು ಪ್ರಾಣಿಯಾಗಿಬಿಡುತ್ತಾನೆ.\break ಆದ್ದರಿಂದ ತ್ಯಾಗದಿಂದ ಪ್ರೇರಿತನಾಗು.

ನನಗೆ ಒಂದು ದೇಹವಿತ್ತು. ನಾನು ಹುಟ್ಟಿದೆ, ನಾನು ಹೋರಾಡುತ್ತಿರುವೆ, ನಾನು ಸತ್ತೆ ಎಂಬುದು ಎಂತಹ ಘೋರ ಭ್ರಾಂತಿ! ದೇಹದಲ್ಲಿ ಬಂದಿಯಾಗಿರುವೆನೆಂದು ತಿಳಿದು\break ಮುಕ್ತಿಗಾಗಿ ಅಳುವುದು ಎಂಥ ಭ್ರಾಂತಿ!

ತ್ಯಾಗ ಎಂದರೆ ನಾವೆಲ್ಲಾ ದೇಹವನ್ನು ದಂಡಿಸಬೇಕೆಂದೆ? ಆಗ ಯಾರು ಯಾರಿಗೆ ಸಹಾಯ ಮಾಡುವುದು? ತ್ಯಾಗವು ದೇಹದಂಡನೆಯಲ್ಲ. ಭಿಕ್ಷುಕರೆಲ್ಲ ಕ್ರಿಸ್ತರೆ? ಬಡತನ ಎಂದರೆ ಪಾವಿತ್ರ್ಯವಲ್ಲ. ಅನೇಕ ವೇಳೆ ಅದಕ್ಕೆ ವಿರುದ್ಧವಾಗಿರುವುದು. ತ್ಯಾಗ ಮನಸ್ಸಿಗೆ\break ಸಂಬಂಧಪಟ್ಟಿರುವುದು. ಇದು ಹೇಗೆ ಬರುವುದು? ನನಗೆ ಬಾಯಾರಿಕೆಯಾದಾಗ ಮರುಳು\-ಗಾಡಿನಲ್ಲಿ ಒಂದು ಸರೋವರ ಕಂಡಿತು. ಅದೊಂದು ಸುಂದರವಾದ ಸನ್ನಿವೇಶದಲ್ಲಿತ್ತು. ಸುತ್ತಲೂ ಗಿಡಮರಗಳು ಇದ್ದವು. ಅವುಗಳ ಛಾಯೆಯನ್ನು ನೀರಿನಲ್ಲಿ ತಲೆಕೆಳಗಾಗಿ ನೋಡ\-ಬಹುದಾಗಿತ್ತು. ಆದರೆ ಇವೆಲ್ಲಾ ಕೊನೆಗೆ ಒಂದು ಬಿಸಿಲುಗುದುರೆ ಎಂದು ಗೊತ್ತಾಯಿತು. ನಾನು ಒಂದು ತಿಂಗಳಿನಿಂದ ಇದನ್ನು ನೋಡುತ್ತಿದ್ದರೂ ಆ ದಿನ ಬಾಯಾರಿಕೆಯಾದಾಗ ಅದರ ಹತ್ತಿರ ಹೋದರೆ ಅದು ಮಾಯವಾದಾಗ ಅದೊಂದು ಅಸತ್ಯ ಎಂದು ಗೊತ್ತಾಯಿತು. ಮುಂದೆಯೂ ಪ್ರತಿದಿನ ಒಂದು ತಿಂಗಳು ನೋಡುತ್ತಿದ್ದೆ. ಆದರೆ ಅದು ಸತ್ಯ ಎಂದು ಭಾವಿಸಲಿಲ್ಲ. ಇದರಂತೆಯೇ ನಾವು ದೇವರನ್ನು ನೋಡಿದಾಗ ನಮ್ಮ ದೇಹ, ಪ್ರಪಂಚ ಎಂಬ ಭಾವನೆಯೆಲ್ಲ ಮಾಯವಾಗುವುದು. ಅನಂತರ ಈ ಭಾವನೆ ಪುನಃ ಬರುವುದು. ಆದರೆ ಅದು ಪುನಃ ಬಂದಾಗ ಇದೆಲ್ಲ ಒಂದು ಭ್ರಾಂತಿ ಎಂದು ನಮಗೆ ಗೊತ್ತಾಗಿರುತ್ತದೆ.

ಪ್ರಪಂಚದ ಇತಿಹಾಸವೆಂದರೆ ಏಸು, ಬುದ್ಧ ಮುಂತಾದ ಕೆಲವು ಮಹಾತ್ಮರ ಜೀವನವಷ್ಟೆ. ಅನಾಸಕ್ತರು, ಅಕಾಮಿಗಳು ಪ್ರಪಂಚಕ್ಕೆ ಹೆಚ್ಚು ಒಳ್ಳೆಯದನ್ನು ಮಾಡುವರು.\break ಏಸುವು ಬಡವರ ಜೋಪಡಿಯಲ್ಲಿದ್ದುದನ್ನು ಜ್ಞಾಪಿಸಿಕೊಳ್ಳಿ. ಅವನು ಬಡತನಕ್ಕೆ ಅತೀತವಾಗಿ\-ರುವುದನ್ನು ನೋಡುವನು. “ನನ್ನ ಸಹೋದರರೇ, ನೀವೆಲ್ಲಾ ಪವಿತ್ರಾತ್ಮರು” ಎನ್ನುವರು. ಅವರ ಕರ್ಮ ಶಾಂತವಾಗಿರುವುದು. ಅವನು ಕಾರಣಗಳಿಗೆ ಹೋಗಿ ಅದನ್ನು ನಿರ್ಮೂಲನ ಮಾಡುವನು. ನಿಮಗೆ ಕರ್ಮವು ಭ್ರಾಂತಿ ಎಂದು ಗೊತ್ತಾದಾಗ ಮಾತ್ರ ನೀವು ಪ್ರಪಂಚಕ್ಕೆ ಒಳ್ಳೆಯದನ್ನು ಮಾಡಬಲ್ಲಿರಿ. ನಾವು ಮಾಡುವ ಕೆಲಸದಲ್ಲಿ ಇದನ್ನು ನಾನು ಮಾಡುತ್ತಿರುವೆ\break ಎಂಬ ಭಾವ ಎಷ್ಟು ಕಡಮೆ ಇದ್ದರೆ ಅಷ್ಟು ಒಳ್ಳೆಯದು. ಆಗ ಮಾತ್ರ ಮನಸ್ಸು ಅತಿಪ್ರಜ್ಞಾವಸ್ಥೆಯಲ್ಲಿ ಇರಬಲ್ಲದು. ನಾವು ಒಳ್ಳೆಯದನ್ನು ಅಥವಾ ಕೆಟ್ಟದ್ದನ್ನು ಅರಸುತ್ತಿಲ್ಲ. ಆದರೆ ಸಂತೋಷ, ಪುಣ್ಯ ಇವು ಅವುಗಳಿಗೆ ವಿರುದ್ಧವಾದ ದುಃಖ ಪಾಪಗಳಿಗಿಂತ ಸತ್ಯಕ್ಕೆ ಹೆಚ್ಚು ಸಮೀಪದಲ್ಲಿರುವುವು ಒಬ್ಬನ ಕೈಗೆ ಒಂದು ಮುಳ್ಳು ಹೊಕ್ಕಿತು. ಅವನು ಮತ್ತೊಂದು ಮುಳ್ಳಿನಿಂದ ಆ ಮುಳ್ಳನ್ನು ತೆಗೆದುಹಾಕಿದನು. ಮೊದಲನೆಯ ಮುಳ್ಳು ಪಾಪ, ಎರಡನೆಯ\break ಮುಳ್ಳು ಪುಣ್ಯ. ಆದರೆ ಆತ್ಮವು ಪರಮಶಾಂತಿಯುತವಾದುದು. ಪಾಪಪುಣ್ಯಾತೀತ\break ವಾದುದು. ಅಲ್ಲಿ ಪ್ರಪಂಚ ಕರಗಿಹೋಗುತ್ತದೆ. ಮಾನವನು ದೇವರ ಸಮೀಪಕ್ಕೆ ಬರುತ್ತಿರುವನು. ಒಂದು ಕ್ಷಣ ಅವನು ನಿಜವಾಗಿ ದೇವರೇ ಆಗುವನು. ಅವನೊಬ್ಬ ಹೊಸ ಮನುಷ್ಯನಾಗುವನು, ಒಬ್ಬ ಪ್ರವಾದಿಯಾಗುವನು. ಅವನನ್ನು ಕಂಡರೆ ಆಗ ಪ್ರಪಂಚ ಅಂಜುವುದು.\break ಮೂಢನು ಮಲಗಿ ಎದ್ದ ಮೇಲೆಯೂ ಮೂಢನಾಗಿರುವನು. ಆದರೆ ದೇವಮಾನವನಾದರೋ ಸಮಾಧಿಯಿಂದ ಹಿಂತಿರುಗಿ ಬರುವಾಗ ಅನಂತಶಕ್ತಿ, ಅನಂತಪ್ರೇಮ, ಅನಂತಪಾವಿತ್ರ್ಯ ಅವನಲ್ಲಿರುವುವು. ಇದೇ ಸಮಾಧಿಯ ಪ್ರಯೋಜನ.

ಧರ್ಮವನ್ನು ಸಮರಾಂಗಣದಲ್ಲಿಯೂ ಅಭ್ಯಾಸ ಮಾಡಬಹುದು. ಗೀತೆಯನ್ನು ಬೋಧಿಸಿದ್ದು ಹೀಗೆಯೇ. ಮನಸ್ಸಿಗೆ ಮೂರು ಅವಸ್ಥೆಗಳಿವೆ. ಅವೇ ತಾಮಸಿಕ, ರಾಜಸಿಕ ಮತ್ತು ಸಾತ್ತ್ವಿಕ ಎಂಬವು. ತಾಮಸಿಕ ಅವಸ್ಥೆಯಲ್ಲಿ ಮನಸ್ಸು ಅತಿ ಮಂದವಾಗಿ ಚಲಿಸುವುದು. ರಾಜಸಿಕ ಅವಸ್ಥೆಯಲ್ಲಿ ಅತಿ ಚುರುಕಾಗಿರುವುದು. ಸಾತ್ತ್ವಿಕ ಅವಸ್ಥೆಯಲ್ಲಿ ಎಲ್ಲಕ್ಕಿಂತ ತೀವ್ರವಾಗಿ ಚಲಿಸುವುದು. ಆತ್ಮ ರಥದಲ್ಲಿ ಕುಳಿತಿದೆ ಎಂದು ತಿಳಿಯಿರಿ. ದೇಹವೇ ರಥ. ಹೊರಗಡೆಯ ಇಂದ್ರಿಯಗಳೇ ಕುದುರೆ, ಮನಸ್ಸೇ ಲಗಾಮು. ಬುದ್ಧಿಯೇ ಸಾರಥಿ.\break ಮಾನವನು ಹೀಗೆ ಮಾಯೆಯಿಂದ ಪಾರಾಗುವನು. ಅವನು ಅತೀತನಾಗಿ ಹೋಗುವನು. ದೇವರನ್ನು ಸೇರುವನು. ಮಾನವ ಇಂದ್ರಿಯ ವಶದಲ್ಲಿರುವಾಗ ಈ ಪ್ರಪಂಚಕ್ಕೆ ಸೇರಿರುವನು. ಅವನು ಇಂದ್ರಿಯವನ್ನು ನಿಗ್ರಹಿಸಿದಾಗ ಅವನು ತ್ಯಾಗಿಯಾಗುವನು.

ಕ್ಷಮೆ ಕೂಡ, ದುರ್ಬಲತೆಯಿಂದ ಪ್ರೇರೇಪಿತವಾಗಿದ್ದರೆ, ತಾಮಸಿಕವಾಗಿದ್ದರೆ, ಅದು ನಿಜವಲ್ಲ. ಅದರ ಬದಲು ಹೋರಾಡುವುದು ಮೇಲು. ಗೆಲ್ಲುವುದಕ್ಕೆ ನಿಮಗೆ ದೇವತೆಗಳ ಒಂದು ಸೈನ್ಯವೇ ಸಹಾಯಕ್ಕೆ ಇರುವಾಗ ಬೇಕಾದರೆ ಕ್ಷಮಿಸಿ. ಅರ್ಜುನ ವೈರಿಗಳನ್ನು\break ಕ್ಷಮಿಸೋಣ ಎಂದು ಹೇಳುವ ಮಾತನ್ನು ಕೃಷ್ಣ ಕೇಳುವನು. ಅವನು ಆಗ ಅರ್ಜುನನಿಗೆ\break ಹೇಳುತ್ತಾನೆ “ನೀನು ಒಬ್ಬ ಪ್ರಾಜ್ಞನಂತೆ ಮಾತನಾಡುತ್ತಿರುವೆ. ಆದರೆ ನೀನು ಪ್ರಾಜ್ಞನಲ್ಲ, ಹೇಡಿ” ಎಂದು. ಕಮಲಪತ್ರ ನೀರಿನಲ್ಲಿ ಇದ್ದರೂ ಹೇಗೆ ನೀರಿಗೆ ಅಂಟಿಕೊಂಡಿಲ್ಲವೋ ಹಾಗೆಯೆ ಜೀವಿ ಪ್ರಪಂಚದಲ್ಲಿ ಇರಬೇಕು. ಇದು ಒಂದು ರಣಕ್ಷೇತ್ರ. ಮುಂದುವರಿಯಬೇಕಾದರೆ ಹೋರಾಡಬೇಕು. ಈ ಜೀವನ ಭಗವಂತನನ್ನು ನೋಡುವುದಕ್ಕೆ ಒಂದು\break ಪ್ರಯತ್ನವಷ್ಟೆ. ತ್ಯಾಗದಿಂದ ಬಲಗೊಂಡ ಇಚ್ಛೆಯನ್ನು ಜೀವನದಲ್ಲಿ ವ್ಯಕ್ತಗೊಳಿಸಿ.

ನಾವು ಪ್ರಜ್ಞಾಪೂರ್ವಕವಾಗಿ ಎಲ್ಲಾ ಕರಣಗಳನ್ನೂ ನಿಗ್ರಹಿಸುವುದನ್ನು ಕಲಿಯಬೇಕು. ಮೊದಲ ಹಂತ ಸುಖಮಯ ಜೀವನ; ದೇಹದಂಡನೆ ಪೈಶಾಚಿಕ. ನಗುವುದು ಪ್ರಾರ್ಥನೆಗಿಂತ ಮೇಲು, ಹಾಡಿ, ದುಃಖದಿಂದ ಪಾರಾಗಿ, ದೇವರಾಣೆಗೂ ಇತರರಿಗೆ ದುಃಖವನ್ನು ಹಂಚಬೇಡಿ. ದೇವರು ಸ್ವಲ್ಪ ಸುಖವನ್ನು ಮತ್ತು ಸ್ವಲ್ಪ ದುಃಖವನ್ನು ಮಾರುತ್ತಿರುವನೆಂದು ಭಾವಿಸಬೇಡಿ. ನಿಮ್ಮ ಸುತ್ತಲೂ ಸುಂದರವಾದ ಚಿತ್ರಗಳು, ಹೂವು, ಪನ್ನೀರು, ಗಂಧ ಇರಲಿ. ಸಾಧುಗಳು ನೈಸರ್ಗಿಕ ಸೌಂದರ್ಯವನ್ನು ನೋಡುವುದಕ್ಕಾಗಿ ಬೆಟ್ಟದ ಮೇಲೆ ಹೋದರು.

ಎರಡನೆಯ ಹಂತ ಪರಿಶುದ್ಧತೆ. ಮೂರನೆಯ ಹಂತ ಮನಸ್ಸಿಗೆ ಶಿಕ್ಷಣವನ್ನು ನೀಡುವುದು. ನಿತ್ಯ ಯಾವುದು, ಅನಿತ್ಯ ಯಾವುದು ಎಂದು ವಿಚಾರ ಮಾಡಿ. ದೇವರೊಬ್ಬನೇ\break ಸತ್ಯ ಎಂಬುದನ್ನು ನೋಡಿ. ಒಂದು ಕ್ಷಣವಾದರೂ ನೀವು ದೇವರಲ್ಲ ಎಂದು ಭಾವಿಸಿದರೆ ಮಹತ್ತಾದ ಭಯ ನಿಮ್ಮನ್ನು ಮೆಟ್ಟಿಕೊಳ್ಳುವುದು. ನಾನೇ ಅವನು ಎಂದು ನೀವು ಭಾವಿಸಿದೊಡನೆ ಮಹತ್ತಾದ ಶಾಂತಿ, ಆನಂದ ಇವು ನಿಮ್ಮದಾಗುವುವು. ಇಂದ್ರಿಯಗಳನ್ನು ನಿಗ್ರಹಿಸಿ. ಒಬ್ಬ ನನ್ನನ್ನು ಶಪಿಸಿದರೂ ಸದ್ಯಕ್ಕೆ ಮನೋದೌರ್ಬಲ್ಯದಿಂದ ಅವನು ನಿಂದಿಸುತ್ತಿರುವನು ಎಂದು ಭಾವಿಸಿ ಅವನಲ್ಲಿ ದೇವರನ್ನು ನೋಡಲು ಯತ್ನಿಸುತ್ತಿರುವೆನು. ನೀವು ಸೇವೆ ಮಾಡುವುದಕ್ಕೆ ಬಡವ ನಿಮಗೆ ಒಂದು ಅವಕಾಶವನ್ನು ಕಲ್ಪಿಸುತ್ತಿರುವನು. ಭಗವಂತ ಅತಿ ಕರುಣೆಯಿಂದ ಆ ರೂಪದಲ್ಲಿ ತನ್ನನ್ನು ಆರಾಧಿಸಲು ನಿಮಗೆ ಒಂದು ಅವಕಾಶವನ್ನು ಕೊಡುತ್ತಿರುವನು.

ಜಗತ್ತಿನ ಇತಿಹಾಸವು ಆತ್ಮಶ್ರದ್ಧೆಯುಳ್ಳ ವ್ಯಕ್ತಿಗಳ ಚರಿತ್ರೆ. ಈ ಆತ್ಮಶ್ರದ್ಧೆ ಸುಪ್ತವಾಗಿರುವ\break ದಿವ್ಯತೆಯನ್ನು ವ್ಯಕ್ತಗೊಳಿಸುವುದು. ನೀವು ಏನನ್ನು ಬೇಕಾದರೂ ಸಾಧಿಸಬಹುದು. ನೀವು ಆ ಮಹಾಶಕ್ತಿಯನ್ನು ವ್ಯಕ್ತಗೊಳಿಸುವುದಕ್ಕೆ ಸಾಕಾದಷ್ಟು ಪ್ರಯತ್ನಿಸದೆ ಇದ್ದಾಗ ಮಾತ್ರ ನಿರಾಶರಾಗುವಿರಿ. ವ್ಯಕ್ತಿಯಾಗಲೀ, ದೇಶವಾಗಲೀ, ಎಂದು ಆತ್ಮಶ್ರದ್ಧೆಯನ್ನು ಕಳೆದುಕೊಳ್ಳುವುದೋ ಆಗಲೇ ನಾಶವಾದಂತೆ.

ಪ್ರತಿಯೊಬ್ಬರಲ್ಲಿಯೂ ಒಂದು ಪವಿತ್ರಾತ್ಮವು ನೆಲಸಿದೆ. ಯಾವ ಮೌಢ್ಯವಾಗಲಿ, ಮಿಥ್ಯಾಚಾರವಾಗಲಿ ಅದನ್ನು ಮರೆಮಾಡಲಾರದು. ನಾಗರಿಕತೆ ಇರುವೆಡೆಯಲ್ಲೆಲ್ಲಾ ಎಲ್ಲೋ ಕೆಲವು ಗ್ರೀಕರ ಭಾವನೆಗಳು ವ್ಯಕ್ತವಾಗುತ್ತಿರುವುವು. ಕೆಲವು ವೇಳೆ ತಪ್ಪುಗಳು ಆಗಿಯೇ ಆಗುವುವು. ಅದಕ್ಕಾಗಿ ವ್ಯಥೆಪಡಬೇಡಿ. ಅಂತರ್​ದೃಷ್ಟಿ ಇರಲಿ. ಅಯ್ಯೋ ನಾನು ಅದನ್ನು ಚೆನ್ನಾಗಿ ಮಾಡಿದರೆ ಎಂದು ಪುನಃ ಮೆಲುಕು ಹಾಕಬೇಡಿ. ಆದದ್ದು ಆಗಿಹೋಯಿತು. ಮಾನವನು ದೇವರೇ ಆಗಿಲ್ಲದೇ ಇದ್ದಿದ್ದರೆ ಇಷ್ಟು ಹೊತ್ತಿಗೆ ಮಾನವಕೋಟಿ ಧಾರ್ಮಿಕ ವಿಧಿಗಳಿಂದ ಮತ್ತು ಪಶ್ಚಾತ್ತಾಪದಿಂದ ಹುಚ್ಚಾಗಿ ಹೋಗುತ್ತಿತ್ತು.

ಯಾರೂ ಹಿಂದೆ ಉಳಿಯುವುದಿಲ್ಲ, ಯಾರೂ ನಾಶವಾಗುವುದಿಲ್ಲ. ಕೊನೆಗೆ ಎಲ್ಲರೂ ಪೂರ್ಣರಾಗುವರು; ಸಹೋದರರೆ. ಜಾಗೃತರಾಗಿ, ನೀವು ಅನಂತ ಪರಿಶುದ್ಧತೆಯ ಸಾಗರ! ದೇವರಾಗಿ! ದೇವರಂತೆ ಕಾಣಿಸಿಕೊಳ್ಳಿ!

ನಾಗರಿಕತೆ ಎಂದರೇನು? ನಮ್ಮ ಆಂತರ್ಯದಲ್ಲಿ ದಿವ್ಯಾತ್ಮನಿರುವನು ಎಂಬುದನ್ನು ಅನುಭವಿಸುವುದು. ಕಾಲಾವಕಾಶವಾದಾಗಲೆಲ್ಲ ಈ ಭಾವನೆಯನ್ನು ಮನನಮಾಡಿ, ಮುಕ್ತಿಯನ್ನು ಆಶಿಸಿ, ಇಷ್ಟು ಸಾಕು. ದೇವರಲ್ಲದ ಎಲ್ಲವನ್ನೂ ತ್ಯಜಿಸಿ, ದೇವರಾಗಿರುವುದನ್ನೆಲ್ಲಾ ಒಪ್ಪಿಕೊಳ್ಳಿ. ಹಗಲು ರಾತ್ರಿ ಇವನ್ನು ಮನಸ್ಸಿನಲ್ಲಿ ಒತ್ತಿ ಒತ್ತಿ ಹೇಳಿ. ಮಾಯೆಯ ತೆರೆಯು ಇದರಿಂದ ತೆಳ್ಳಗಾಗುತ್ತಾ ಬರುವುದು.

“ನಾನು ನರನೂ ಅಲ್ಲ, ದೇವತೆಯೂ ಅಲ್ಲ. ನನಗೆ ಲಿಂಗವಿಲ್ಲ. ಯಾವ ಮಿತಿಯೂ ಇಲ್ಲ. ನಾನೇ ಜ್ಞಾನಸ್ವರೂಪ, ಶಿವೋಽಹಂ. ನನಗೆ ಕೋಪವೂ ಇಲ್ಲ. ದ್ವೇಷವೂ ಇಲ್ಲ. ನನಗೆ ಸುಖವೂ ಇಲ್ಲ, ದುಃಖವೂ ಇಲ್ಲ. ಜನನ ಮರಣಗಳು ನನಗೆ ಎಂದಿಗೂ ಇರಲಿಲ್ಲ. ನಾನೇ ಕೇವಲ ಜ್ಞಾನ, ಕೇವಲ ಆನಂದ. ನಾನು ನನ್ನಾತ್ಮವಾದ ಅವನೇ, ನಾನೇ ಅವನು.”

ನಿರಾಕಾರಿಗಳು ನೀವು. ನಿಮಗೆ ಎಂದೂ ದೇಹವಿರಲಿಲ್ಲ. ಇದೆಲ್ಲ ಒಂದು ಭ್ರಾಂತಿ. ಎಲ್ಲಾ ದೀನರಿಗೆ, ದರಿದ್ರರಿಗೆ, ದಬ್ಬಾಳಿಕೆಗೆ ತುತ್ತಾದವರಿಗೆ ರೋಗಿಗಳಿಗೆ ದಿವ್ಯಾತ್ಮನ ಈ ಭಾವನೆಯನ್ನು ನೀಡಿ.

ಸಾಧಾರಣವಾಗಿ ಐನೂರು ವರ್ಷಗಳಿಗೆ ಒಮ್ಮೆ ಇಂತಹ ಆಲೋಚನೆಯ ಮಹದಲೆಯೊಂದು ಮೇಲೇಳುವುದು. ಹಲವು ಕಡೆ ಸಣ್ಣ ಸಣ್ಣ ಅಲೆಗಳು ಮೇಲೇಳುವುವು. ಆದರೆ ಅದರಲ್ಲಿ ಒಂದು ಉಳಿದೆಲ್ಲವನ್ನೂ ಒಳಗೊಂಡು ಇಡಿಯ ಸಮಾಜದ ಮೇಲೆ ಉರುಳುವುದು. ಯಾವ ಅಲೆಯ ಹಿಂದೆ ಎಲ್ಲಕ್ಕಿಂತ ಹೆಚ್ಚಾದ ಚಾರಿತ್ರ್ಯ ಅಡಗಿದೆಯೋ, ಅದು ಮಹದಲೆಯಾಗಿ ಏಳುವುದು.

ಕನ್ಫೂಷಿಯಸ್​, ಮೋಸೆಸ್​, ಪೈಥಾಗೊರಸ್​, ಬುದ್ಧ. ಕ್ರಿಸ್ತ, ಮಹಮ್ಮದ್​ ಲೂಥರ್​, ಕಾಲ್ವಿನ್​, ಸಿಕ್ಕರು, ಥಿಯಾಸೊಫಿ, ಸ್ಪಿರಿಚುಯಾಲಿಸಮ್​ ಮುಂತಾದುವುಗಳ ಉದ್ದೇಶ ಮಾನವನಲ್ಲಿ ಅಂತರ್ಗತವಾಗಿರುವ ದಿವ್ಯತೆಯನ್ನು ಬೋಧಿಸುವುದೇ ಆಗಿದೆ.

ಮಾನವನು ದುರ್ಬಲ ಎಂದು ಎಂದಿಗೂ ಹೇಳಬೇಡಿ. ಜ್ಞಾನಯೋಗ ಇತರ ಯೋಗಗಳಿಗಿಂತ ಮೇಲೇನೂ ಅಲ್ಲ. ಪ್ರೇಮವೇ ಆದರ್ಶ. ಅದಕ್ಕೆ ಯಾವ ಒಂದು ವ್ಯಕ್ತಿಯೂ ಬೇಕಾಗಿಲ್ಲ. ಪ್ರೇಮವೇ ದೇವರು, ಭಕ್ತಿಯ ಮೂಲಕವೂ ನಾವು ಭಗವಂತನನ್ನು ನೋಡಬಹುದು. ನಾನೇ ಅವನು. ಒಬ್ಬನು ದೇಶ, ಪ್ರಾಣಿ ಮತ್ತು ಪ್ರಪಂಚ ಇವನ್ನು ಪ್ರೀತಿಸದೆ ಹೇಗೆ ಕೆಲಸ ಮಾಡಬಲ್ಲ? ವೈಚಾರಿಕತೆಯು ವೈವಿಧ್ಯದ ಹಿಂದೆ ಇರುವ ಏಕತೆಯನ್ನು ತೋರಿಸುತ್ತದೆ. ನಾಸ್ತಿಕರು, ಅಜ್ಞೇಯತವಾದಿಗಳು ಸಾಮಾಜಿಕ ಹಿತಕ್ಕೆ ಬೇಕಾದರೆ ಕೆಲಸಮಾಡಲಿ; ಕ್ರಮೇಣ ದೇವರು ಈ ಮೂಲಕ ಬರುವನು.

ಆದರೆ ಒಂದು ವಿಷಯವನ್ನು ನೀವು ಗಮನದಲ್ಲಿಡಬೇಕು. ಯಾರ ಶ್ರದ್ಧೆಗೂ ಭಂಗವನ್ನು ತರಬೇಡಿ. ಧರ್ಮವು ಸಿದ್ಧಾಂತಗಳಲ್ಲಿ ಇಲ್ಲ ಎನ್ನುವುದು ನಿಮಗೆ ಗೊತ್ತಿರಬೇಕು. ದೇವರಂತೆ ಇರುವುದು, ದೇವರಂತೆ ಆಗುವುದು, ಇದೇ ಧರ್ಮ. ಸಾಕ್ಷಾತ್ಕಾರವೇ ಧರ್ಮ. ಮಾನವರೆಲ್ಲ ಹುಟ್ಟು ವಿಗ್ರಹಾರಾಧಕರು. ಅತಿ ಕೀಳು ಮನುಷ್ಯರು ಮೃಗ. ಅತಿ ಮೇಲೆ ಇರುವ ಮನುಷ್ಯನು ಪೂರ್ಣಾತ್ಮ. ಇವರಿಬ್ಬರ ಮಧ್ಯೆ ಇರುವವರೆಲ್ಲ ಶಬ್ದ, ಬಣ್ಣ, ಸಿದ್ಧಾಂತ, ಕ್ರಿಯಾವಿಧಿ ಮುಂತಾದುವುಗಳ ಮೂಲಕ ಆಲೋಚಿಸಬೇಕಾಗಿದೆ.

ನೀವು ವಿಗ್ರಹಾರಾಧಕರಲ್ಲ ಎನ್ನುವುದಕ್ಕೆ ಒಂದು ಪರೀಕ್ಷೆಯೆಂದರೆ “ನಾನು” ಎಂದಾಗ ನಿಮಗೆ ದೇಹದ ಭಾವನೆ ಬರುವುದೋ ಇಲ್ಲವೋ ಎಂಬುದು. ದೇಹದ ಭಾವನೆ ಬಂದರೆ ನೀವು ಇನ್ನೂ ವಿಗ್ರಹಾರಾಧಕರು. ಧರ್ಮವು ಕೇವಲ ಪಾಂಡಿತ್ಯದ ಕಸರತ್ತು ಅಲ್ಲ; ಅದು ಸಾಕ್ಷಾತ್ಕಾರ. ನೀವು ದೇವರ ವಿಷಯವನ್ನು ಇನ್ನೂ ತರ್ಕಿಸುತ್ತಿದ್ದರೆ ಮೂರ್ಖರು.\break ಅಜ್ಞಾನಿಯು ಪ್ರಾರ್ಥನೆಯಿಂದ ಮತ್ತು ಭಕ್ತಿಯಿಂದ ತತ್ತ್ವಜ್ಞಾನಿಯನ್ನೂ ಮೀರಿ ಹೋಗಬಲ್ಲನು. ದೇವರನ್ನು ಅರಿಯಬೇಕಾದರೆ ಯಾವ ತತ್ತ್ವಜ್ಞಾನವೂ ಬೇಕಾಗಿಲ್ಲ. ನಾವು ಇತರರ ಶ್ರದ್ಧೆಗೆ ಭಂಗವನ್ನು ತರಕೂಡದು. ಇದು ನಮ್ಮ ಕರ್ತವ್ಯ. ಧರ್ಮ ಎನ್ನುವುದು ಒಂದು ಅನುಭೂತಿ. ಎಲ್ಲಕ್ಕಿಂತ ಹೆಚ್ಚಾಗಿ, ಎಲ್ಲದರಲ್ಲಿಯೂ ನಿಸ್ಪೃಹರಾಗಿರಿ. ದೇಹಭಾವನೆಯೇ ದುಃಖಕ್ಕೆ ಕಾರಣ. ಏಕೆಂದರೆ ಇದು ಆಸೆಗೆ ಕಾರಣವಾಗುವುದು. ಬಡವ ಹಣವನ್ನು ನೋಡುವನು, ಹಣ ತನಗೆ ಬೇಕೆಂದು ಭಾವಿಸುವನು. ಯಾವಾಗಲೂ ಸಾಕ್ಷಿಯಾಗಿರಿ. ಎಂದಿಗೂ ಪ್ರತಿಕ್ರಿಯೆ ತೋರಿಸಬೇಡಿ.

