
\chapter[ಇಂಡಿಯಾ ಮತ್ತು ಇಂಗ್ಲೆಂಡ್ ]{ಇಂಡಿಯಾ ಮತ್ತು ಇಂಗ್ಲೆಂಡ್ \protect\footnote{\engfoot{C.W. Vol. V, p. 194}}}

\centerline{\textbf{(“ಇಂಡಿಯಾ” ಲಂಡನ್​ 1896)}}

ಲಂಡನ್ನಿನಲ್ಲಿದ್ದಾಗ ಸ್ವಾಮಿ ವಿವೇಕಾನಂದರು ತಮ್ಮ ತತ್ತ್ವಸಿದ್ಧಾಂತಗಳಲ್ಲಿ ಆಸಕ್ತಿ ತೋರುತ್ತಿದ್ದ ಅನೇಕ ಜನರಿಗೆ ಪ್ರವಚನಾದಿಗಳನ್ನೂ ಉಪನ್ಯಾಸಗಳನ್ನೂ ಕೊಡುತ್ತಿದ್ದರು. ಇಂಗ್ಲೆಂಡಿನ ಹಲವು ಜನರು ಧರ್ಮ ಬೋಧನೆಯ ಸಾಹಸವೆಲ್ಲ ಇಂಗ್ಲೆಂಡಿನ ಹಕ್ಕು ಮತ್ತು ಅದರಲ್ಲಿ ಸ್ವಲ್ಪಭಾಗ ಮಾತ್ರ ಫ್ರಾನ್ಸಿಗೆ ಸೇರಿದ್ದು ಎಂಬ ಕಲ್ಪನೆಯನ್ನು ಹೊಂದಿರುವರು. ಆದಕಾರಣ ನಾನು ಸ್ವಾಮಿಗಳನ್ನು ಅವರು ಹಂಗಾಮಿಯಾಗಿ ತಂಗಿದ್ದ ದಕ್ಷಿಣ ಬೆಲ್​ಗ್ರೇವಿಯಾದ ಮನೆಯಲ್ಲಿ ಭೇಟಿಮಾಡಲು ಹೋದೆ. ನ್ಯಾಯಾಂಗ ಮತ್ತು ಕಾರ್ಯಾಂಗಗಳ ಜವಾಬ್ದಾರಿಗಳು ಒಂದೇ ವ್ಯಕ್ತಿಯಲ್ಲಿ ಕೇಂದ್ರೀಕೃತವಾಗಿರುವುದನ್ನೂ, ಸುಡಾನ್​ ಮತ್ತು ಇತರ ಯುದ್ಧದ ಖರ್ಚುಗಳನ್ನು ಭಾರತದ ಮೇಲೆ ಹೇರಿರುವುದನ್ನೂ ಭಾರತವು ಪ್ರತಿಭಟಿಸಿದೆ. ಅದರ\break ಹೊರತಾಗಿ ಭರತಖಂಡವು ಇಂಗ್ಲೆಂಡಿಗೆ ನೀಡಬಹುದಾದ ಸಂದೇಶವು ಬೇರೇನಾದರೂ ಇದೆಯೇ ಎಂಬುದನ್ನು ಸ್ವಾಮೀಜಿಯೊಡನೆ ಚರ್ಚಿಸಲು ಅವರಲ್ಲಿಗೆ ಹೋದೆ.

ಸ್ವಾಮೀಜಿ: “ಭಾರತವು ಹೊರಗಿನ ದೇಶಗಳಿಗೆ ಪ್ರಚಾರಕರನ್ನು ಕಳುಹಿಸುವುದು\break ಏನೂ ಹೊಸದಲ್ಲ. ಬೌದ್ಧಧರ್ಮದ ಪ್ರಾರಂಭದ ಕಾಲದಲ್ಲಿ ನೆರೆಹೊರೆಯ ದೇಶಗಳಿಗೆ ಏನನ್ನಾದರೂ ಬೋಧಿಸಬೇಕಾಗಿದೆ ಎಂದು ತೋರಿದಾಗ ಅಶೋಕನ ಕಾಲದಿಂದಲೇ ಭರತಖಂಡವು ಪ್ರಚಾರಕರನ್ನು ಕಳುಹಿಸುತ್ತಿತ್ತು.”

ಬಾತ್ಮೀದಾರ: “ಹಾಗಾದರೆ ಅದನ್ನು ನಿಲ್ಲಿಸಿದ್ದು ಏತಕ್ಕೆ ಮತ್ತು ಈಗ ಅದನ್ನು ಪ್ರಾರಂಭ ಮಾಡುವುದಾದರೂ ಏತಕ್ಕೆ?”

ಸ್ವಾಮೀಜಿ: “ಭರತಖಂಡ ಅದನ್ನು ನಿಲ್ಲಿಸಿದ್ದಕ್ಕೆ ಕಾರಣ ಅದರ ಸ್ವಾರ್ಥ. ವ್ಯಕ್ತಿಗಳಂತೆ ದೇಶ ಕೂಡ ಕೊಟ್ಟು ತೆಗೆದುಕೊಳ್ಳುವುದರ ಮೇಲೆ ನಿಂತಿದೆ ಎಂಬ ನಿಯಮವನ್ನು ಮರೆಯಿತು. ಪ್ರಪಂಚಕ್ಕೆ ಭರತಖಂಡದ ಸಂದೇಶ ಯಾವಾಗಲೂ ಒಂದೇ ಆಗಿರುವುದು. ಅನಾದಿಕಾಲದಿಂದಲೂ ಅಂತರ್ಮುಖ ಜೀವನದ ಚಿಂತನೆ, ಅಧ್ಯಾತ್ಮಗಳೇ ಅದರ ಸಂದೇಶವಾಗಿದೆ. ತತ್ತ್ವ, ತರ್ಕ ಮುಂತಾದ ಗಹನ ಶಾಸ್ತ್ರಗಳೇ ಅದರ ಕ್ಷೇತ್ರ. ಇಂಗ್ಲೆಂಡ್​ ಇಂಡಿಯಾ ದೇಶಕ್ಕೆ ಬಂದ ಪರಿಣಾಮವಾಗಿ ನಾನು ಇಂಗ್ಲೆಂಡಿಗೆ ಬಂದಿರುವೆನು. ಇಂಗ್ಲೆಂಡ್​ ನಮ್ಮ ದೇಶವನ್ನು ಗೆದ್ದು ಆಳುತ್ತಿದೆ. ಅದು ಭೌತಿಕ ಶಾಸ್ತ್ರಗಳ ಜ್ಞಾನವನ್ನು ತನ್ನ ಮತ್ತು ನಮ್ಮ ಪ್ರಯೋಜನಕ್ಕೆ\break ಬಳಸುತ್ತಿದೆ. ಪ್ರಪಂಚಕ್ಕೆ ಭರತಖಂಡ ಏನನ್ನು ಕೊಡಬಲ್ಲದು ಎಂಬುದನ್ನು ಒಂದು\break ಮಾತಿನಲ್ಲಿ ಹೇಳಬೇಕಾಗಿ ಬಂದಾಗ ಸಂಸ್ಕೃತದ ಒಂದು ಗಾದೆ ಮತ್ತು ಇಂಗ್ಲಿಷಿನ ಒಂದು\break ನುಡಿಗಟ್ಟು ಜ್ಞಾಪಕಕ್ಕೆ ಬರುತ್ತದೆ. ಒಬ್ಬ ಮನುಷ್ಯ ಸತ್ತರೆ, ನೀವು ಅವನು ಜೀವ ಬಿಟ್ಟನು\break ಎನ್ನುವಿರಿ, ಆದರೆ ನಾವು ಅವನು ದೇಹ ಬಿಟ್ಟನು ಎನ್ನುವೆವು. ನಿಮ್ಮ ದೃಷ್ಟಿಯಲ್ಲಿ ದೇಹ ಒಂದು ಜೀವವನ್ನು ಹೊಂದಿದೆ ಎಂದು ಹೇಳುವಾಗ, ದೇಹವೇ ಮುಖ್ಯ, ಜೀವ ಗೌಣವಾಗಿರುವುದು. ಆದರೆ ನಾವು ಮಾನವ ಪ್ರಥಮತಃ ಜೀವ, ಅವನಿಗೊಂದು ದೇಹವಿದೆ ಎನ್ನುತ್ತೇವೆ. ಈ ನುಡಿಗಟ್ಟುಗಳು ಮೇಲೆ ತೇಲುವ ಕೆಲವು ಗುಳ್ಳೆಗಳಷ್ಟೆ. ಆದರೆ ಇವು ನಿಮ್ಮ ಜನಾಂಗದ ಭಾವನಾ ಪ್ರವಾಹವನ್ನು ವ್ಯಕ್ತಗೊಳಿಸುತ್ತವೆ. ಅಜ್ಞಾನದ ಕತ್ತಲ ಯುಗದಲ್ಲಿ ಗ್ರೀಕ್​ ಮತ್ತು ರೋಮನ್​ ಸಂಸ್ಕೃತಿ ಮಧ್ಯಯುಗದ ಯೂರೋಪಿನ ಮೇಲೆ ಯಾವ ಪ್ರಭಾವವನ್ನು ಉಂಟು ಮಾಡುತ್ತವೆ ಎಂದು ಷೋಫನಿಯರ್​ ಭವಿಷ್ಯವನ್ನು ನುಡಿದಿರುವನು. ಪ್ರಾಚ್ಯ ಅಧ್ಯಯನಗಳ ಕ್ಷೇತ್ರದಲ್ಲಿ ಸಂಶೋಧನೆಯು ಪ್ರಗತಿಯನ್ನು ಸಾಧಿಸುತ್ತಿದೆ. ಸತ್ಯಾನ್ವೇಷಕರಿಗೆ ಹೊಸದೊಂದು ಭಾವನಾ ಪ್ರಪಂಚ ಕಣ್ಣಿಗೆ ಬೀಳುತ್ತಿದೆ.”

ಬಾತ್ಮೀದಾರ: “ಕೊನೆಗೆ ಇಂಡಿಯಾ ತನ್ನನ್ನು ಜಯಿಸಿದವರನ್ನು ಜಯಿಸುವುದೆ?”

ಸ್ವಾಮೀಜಿ: “ಹೌದು, ಭಾವನಾ ಪ್ರಪಂಚದಲ್ಲಿ. ಇಂಗ್ಲೆಂಡಿನ ಕೈಯಲ್ಲಿ ಖಡ್ಗವಿದೆ, ಭೌತಿಕ ಪ್ರಪಂಚದ ಸಹಾಯವಿದೆ, ನಮ್ಮನ್ನು ಜಯಿಸಿದ ಮಹಮ್ಮದೀಯರ ಕೈಯಲ್ಲಿ\-ದ್ದಂತೆ. ಆದರೆ ಪ್ರಖ್ಯಾತನಾದ ಅಕ್ಬರ್​ ಬಾದಷಹನು ಅನುಷ್ಠಾನದಲ್ಲಿ ಹಿಂದೂವಾದನು. ವಿದ್ಯಾವಂತರಾದ ಮಹಮ್ಮದೀಯರಲ್ಲಿ ಅನೇಕ ಜನ ಸೂಫಿಗಳು ಇರುವರು. ಅವರನ್ನು ಹಿಂದೂಗಳಿಂದ ಪ್ರತ್ಯೇಕಿಸುವುದು ಬಹಳ ಕಷ್ಟ. ಅವರು ದನದ ಮಾಂಸ ತಿನ್ನುವುದಿಲ್ಲ,\break ನಮ್ಮ ಆಚಾರ ವ್ಯವಹಾರಗಳನ್ನೇ ಬಹುಪಾಲು ಅನುಸರಿಸುವರು. ಅವರ ಭಾವನೆ ನಮ್ಮ\break ಭಾವನೆಯಿಂದ ಓತಪ್ರೋತವಾಗಿದೆ.”

ಬಾತ್ಮೀದಾರ: “ಈಗ ಆಳುತ್ತಿರುವವರ ಗತಿ ಹೀಗೆಯೇ ಆಗುವುದೆಂದು ಕಾಣುವುದು, ನಿಮ್ಮ ದೃಷ್ಟಿಯಲ್ಲಿ! ಆದರೆ ಈ ಕ್ಷಣದಲ್ಲಿ ಅವರು ಹಾಗಾಗುವ ಸ್ಥಿತಿ ತುಂಬಾ ದೂರದಲ್ಲಿದೆ ಎಂದೆನಿಸುತ್ತದೆ.”

ಸ್ವಾಮೀಜಿ: “ಇಲ್ಲ, ನೀವು ಊಹಿಸುವಷ್ಟು ದೂರವೇನೂ ಇಲ್ಲ. ಧಾರ್ಮಿಕ\break ಕ್ಷೇತ್ರದಲ್ಲಿ ಹಿಂದೂಗಳಿಗೂ ಇಂಗ್ಲೀಷರಿಗೂ ಸಮಾನವಾದ ಅಂಶಗಳು ಹಲವಿವೆ. ಇದರಂತೆಯೇ ಇತರ ಧರ್ಮಗಳ ಜನರ ವಿಷಯವೂ. ಎಲ್ಲಿ ಇಂಗ್ಲಿಷ್​ ಅಧಿಕಾರಿಗಳಿಗೆ\break ಭಾರತೀಯ ಸಾಹಿತ್ಯ, ಅದರಲ್ಲಿ ಅವರ ತತ್ತ್ವಗಳ ಪರಿಚಯವಿದೆಯೋ ಅಲ್ಲಿ ಸಾಮಾನ್ಯ ಸಹೃದಯತೆ ಇದೆ. ಇದು ಕ್ರಮೇಣ ವಿಸ್ತಾರವಾಗುತ್ತಾ ಇದೆ. ಕೆಲವರು ತಾಳುವ\break ಪ್ರತ್ಯೇಕತಾ ಮನೋಭಾವಕ್ಕೆ ತಿರಸ್ಕಾರ ಮನೋಭಾವಕ್ಕೆ ಕಾರಣ ಅಜ್ಞಾನ ಎಂದು ಹೇಳುವುದು ಅತಿಶಯೋಕ್ತಿಯಾಗುವುದಿಲ್ಲ.

ಬಾತ್ಮೀದಾರ: ಹೌದು ಅದೇ ದೊಡ್ಡ ದೋಷ. ನಿಮ್ಮ ಧರ್ಮಪ್ರಸಾರ ಕಾರ್ಯಕ್ಕೆ ಇಂಗ್ಲೆಂಡಿಗೆ ಬದಲು ಅಮೆರಿಕಕ್ಕೆ ಹೋದದ್ದು ಏಕೆ ಎಂದು ಕೇಳಬಹುದೆ?

ಸ್ವಾಮೀಜಿ: ಅದು ಆಕಸ್ಮಿಕ ಎನ್ನಬಹುದು. ವಿಶ್ವಮೇಳದ (\enginline{World fair}) ಅಂಗವಾಗಿ\break ವಿಶ್ವಧರ್ಮಸಮ್ಮೇಳನವು ನ್ಯಾಯವಾಗಿ ಲಂಡನ್ನಿನಲ್ಲಿ ನಡೆಯಬೇಕಾಗಿತ್ತು. ಹಾಗಾಗದೆ ಅದು ಚಿಕಾಗೋ ನಗರದಲ್ಲಿ ನಡೆದದ್ದೇ ಇದಕ್ಕೆ ಕಾರಣ. ಮೈಸೂರು ರಾಜರು ಮತ್ತು\break ಕೆಲವು ಸ್ನೇಹಿತರು ನನ್ನನ್ನು ಹಿಂದೂಧರ್ಮದ ಪ್ರತಿನಿಧಿಯಾಗಿ ಅಮೆರಿಕಾ ದೇಶಕ್ಕೆ\break ಕಳುಹಿಸಿಕೊಟ್ಟರು. ಕಳೆದ ವರ್ಷದ ಮತ್ತು ಈ ವರ್ಷದ ಬೇಸಗೆಗಳಲ್ಲಿ ಇಲ್ಲಿ ಉಪನ್ಯಾಸ ಕೊಡುವುದಕ್ಕೆ ಬಂದ ಸಮಯವನ್ನು ಬಿಟ್ಟರೆ ಉಳಿದ ಮೂರು ವರುಷಗಳನ್ನು ಅಮೆರಿಕಾದಲ್ಲಿ ಕಳೆದಿರುವೆನು.

“ಅಮೆರಿಕದವರು ಶ್ರೇಷ್ಠ ಜನಾಂಗ. ಅವರಿಗೆ ಒಳ್ಳೆಯ ಭವಿಷ್ಯವಿದೆ. ನಾನು ಅವರನ್ನು\break ಮೆಚ್ಚುತ್ತೇನೆ. ಅವರಲ್ಲಿ ಎಷ್ಟೋ ಜನ ನನ್ನ ದಯಾವಂತರಾದ ಸ್ನೇಹಿತರು ಇರುವರು. ಅವರು ಇಂಗ್ಲಿಷರಷ್ಟು ಪೂರ್ವಾಗ್ರಹ ಪೀಡಿತರಲ್ಲ. ಭಾವನೆಗಳು ಎಷ್ಟೇ ಹೊಸದಾಗಿರಲಿ\break ಅವನ್ನು ಸರಿಯಾಗಿ ವಿಮರ್ಶಿಸಿ ಸ್ವೀಕರಿಸುವರು. ಅವರು ಅತಿಥಿ ಸತ್ಕಾರಪರರು. ಅವರ\break ಪರಿಚಯ ಮಾಡಿಕೊಳ್ಳುವುದಕ್ಕೆ ಹೆಚ್ಚು ಕಾಲ ಬೇಕಾಗುವುದಿಲ್ಲ. ನಾನು ಅಮೆರಿಕ ದೇಶದಲ್ಲಿ ಊರಿಂದೂರಿಗೆ ಸ್ನೇಹಿತರಿಗೆ ಪ್ರವಚನಗಳನ್ನು ಕೊಡುತ್ತಾ ಹೋದೆ. ಬಾಸ್ಟನ್​, ನ್ಯೂಯಾರ್ಕ್​, ಫಿಲಡೆಲ್​ಪಿಯಾ, ಬಾಲ್ಟಿಮೋರ್​, ವಾಷಿಂಗ್​ಟನ್​, ಡೆಸ್​ಮೊಯ್ನಿ,\break ಮೆಂಫಿಸ್​ ಮತ್ತು ಇತರ ಊರುಗಳನ್ನು ನೋಡಿದೆ.”

ಬಾತ್ಮೀದಾರ: “ಪ್ರತಿಯೊಂದು ಊರಿನಲ್ಲಿಯೂ ಶಿಷ್ಯರನ್ನು ಬಿಡುತ್ತಿರುವಿರೊ?

ಸ್ವಾಮೀಜಿ: “ಹೌದು, ಶಿಷ್ಯರನ್ನು ಬಿಡುತ್ತಿರುವೆನು, ಆದರೆ ಸಂಸ್ಥೆಗಳನ್ನಲ್ಲ. ಅದು\break ನನ್ನ ಕೆಲಸವಲ್ಲ. ನಿಜವಾಗಿ ನೋಡಿದರೆ ಅಂಥ ಸಂಸ್ಥೆಗಳು ಬೇಕಾದಷ್ಟಿವೆ. ಸಂಸ್ಥೆಯನ್ನು ನಡೆಸಬೇಕಾದರೆ ಜನರು ಬೇಕು. ಅಧಿಕಾರ, ಹಣ, ಪ್ರಭಾವಗಳನ್ನು ಅವರು ಸಂಪಾದಿಸಬೇಕಾಗಿದೆ. ಅನೇಕ ವೇಳೆ ಅವರು ತಮ್ಮ ಪಂಥಗಳಿಗಾಗಿ ಹೋರಾಡುವರು.”

ಬಾತ್ಮೀದಾರ: “ನಿಮ್ಮ ಧ್ಯೇಯವನ್ನು ಕೆಲವು ವಾಕ್ಯಗಳಲ್ಲಿ ವಿವರಿಸಲಾಗುವುದೆ? ನೀವು ಬೋಧಿಸಬೇಕೆಂದಿರುವುದು ಧರ್ಮಗಳ ತುಲನಾತ್ಮಕ ಅಧ್ಯಯನವನ್ನೇ?”

ಸ್ವಾಮೀಜಿ: “ಎಲ್ಲಾ ಧರ್ಮಗಳ ಸಾರವಾದ ಧರ್ಮದ ತತ್ತ್ವವನ್ನು ನಾನು ಬೋಧಿಸುವುದು. ಪ್ರತಿಯೊಂದು ಧರ್ಮದಲ್ಲಿಯೂ ಮುಖ್ಯವಾದ ಭಾಗ ಒಂದಿರುವುದು, ಗೌಣವಾದ ಭಾಗ ಒಂದಿರುವುದು. ನಾವು ಧರ್ಮಗಳಲ್ಲಿ ಗೌಣಭಾಗವನ್ನು ತೆಗೆದುಬಿಟ್ಟರೆ\break ಉಳಿಯುವ ಮುಖ್ಯಾಂಶ ಎಲ್ಲಾ ಧರ್ಮಗಳಲ್ಲಿಯೂ ಒಂದೇ. ಅವುಗಳ ಎಲ್ಲದರ ಹಿಂದೆಯೂ ಇರುವುದು ಏಕತೆ. ಇದನ್ನು ನಾವು ದೇವರು, ಅಲ್ಲಾ, ಯಹೋವ, ಆತ್ಮ, ಪ್ರೀತಿ ಎಂದು ಏನು ಬೇಕಾದರೂ ಕರೆಯಬಹುದು.ಒಂದು ಸಣ್ಣ ಕೀಟದಿಂದ ಮಹಾ\-ಪುರುಷನವರೆಗೆ ಈ ಒಂದು ಏಕತೆಯೇ ವ್ಯಕ್ತವಾಗುತ್ತಿರುವುದು. ನಾವು ಈ ಏಕತೆಯನ್ನು ಒತ್ತಿ ಹೇಳಬೇಕಾಗಿದೆ. ಆದರೆ ಪಾಶ್ಚಾತ್ಯರಲ್ಲಿ ಮತ್ತು ಇನ್ನೂ ಅನೇಕ ಕಡೆ ಧರ್ಮದ ಗೌಣಭಾಗದ ಮೇಲೆ ಹೆಚ್ಚು ಒತ್ತನ್ನು ಕೊಡುವರು. ಹೊರಗಿನ ವೇಷಕ್ಕಾಗಿ ಅದನ್ನು\break ಇತರರು ಸ್ವೀಕರಿಸಬೇಕೆಂದು ಹೋರಾಡುವರು, ಅದಕ್ಕಾಗಿ ಕೊಲೆ ಮಾಡುವರು.\break ಮುಖ್ಯವಾಗಿ ಬೇಕಾಗಿರುವುದೇ ಭಗವಂತನ ಪ್ರೀತಿ, ಮಾನವಕೋಟಿಯ ಪ್ರೀತಿ, ಅದಿಲ್ಲದೆ ಬಾಹ್ಯ ವೇಷಕ್ಕಾಗಿ ಹೋರಾಡುವುದು ಹಾಸ್ಯಾಸ್ಪದ.”

ಬಾತ್ಮೀದಾರ: “ಹಿಂದೂಗಳು ಧರ್ಮದ ಹೆಸರಿನಲ್ಲಿ ಎಂದಿಗೂ ಹಿಂಸಿಸಿಲ್ಲ ಎಂದು ಕಾಣುವುದು.

\eject

ಸ್ವಾಮೀಜಿ: “ಅವರಿನ್ನೂ ಹಾಗೆ ಮಾಡಿಲ್ಲ. ಹಿಂದೂಗಳಲ್ಲಿರುವಷ್ಟು ಧರ್ಮಸಹಿಷ್ಣುತೆ ಇತರ ಜನಾಂಗದವರಲ್ಲಿ ಅಪರೂಪ. ಅವರು ಅಷ್ಟೊಂದು ಧಾರ್ಮಿಕರಾಗಿರುವುದನ್ನು ನೋಡಿದರೆ ಅವರು ದೇವರನ್ನು ನಂಬದವರನ್ನು ಹಿಂಸಿಸುತ್ತಾರೆ ಎಂದು ಊಹಿಸಬಹುದಾಗಿತ್ತು. ಜೈನರು ದೇವರಲ್ಲಿ ನಂಬಿಕೆ ಇಡುವುದು ಬರೀ ಭ್ರಾಂತಿ ಎನ್ನುವರು. ಆದರೂ ಹಿಂದೂಗಳು ಯಾವ ಜೈನರನ್ನೂ ಹಿಂಸಿಸಿಲ್ಲ. ಇಂಡಿಯಾ ದೇಶದಲ್ಲಿ ಧರ್ಮದ ಹೆಸರಿನಲ್ಲಿ ಖಡ್ಗವನ್ನು ತೆಗೆದುಕೊಂಡವರು ಮಹಮ್ಮದೀಯರು.”

ಬಾತ್ಮೀದಾರ: “ಧರ್ಮಗಳು ಮೂಲತಃ ಒಂದೇ ಎಂಬ ಸಿದ್ಧಾಂತ ಹೇಗೆ ಇಂಗ್ಲೆಂಡಿನಲ್ಲಿ ಹರಡುವುದು? ಇಲ್ಲಾಗಲೇ ಸಾವಿರಾರು ಪಂಥಗಳಿವೆ.”

ಸ್ವಾಮೀಜಿ: “ಸ್ವಾತಂತ್ರ್ಯ ಮತ್ತು ಜ್ಞಾನ ಹರಡಿದಂತೆ ಇವು ಮಾಯವಾಗಬೇಕು.\break ಪಂಥಗಳು ಗೌಣ ವಿಷಯಗಳ ಮೇಲೆ ನಿಂತಿವೆ. ಅವು ಸ್ವಭಾವತಃ ಊರ್ಜಿತವಾಗಿ\break ನಿಲ್ಲಲಾರವು. ತಮ್ಮದೇ ಪ್ರತ್ಯೇಕ ಪಂಗಡ ಎಂದು ಆಲೋಚಿಸಿದವರ ಭಾವನೆಗೆ ತಕ್ಕಂತೆ ಈ ಪಂಥಗಳು ಕೆಲಸ ಮಾಡಿವೆ. ಈಗ ಪಂಥಗಳ ಕಾಲ ಮುಗಿಯಿತು. ಕ್ರಮೇಣ ಪಂಗಡಗಳನ್ನು ಪ್ರತ್ಯೇಕ ಮಾಡಿದ ಭಿನ್ನ ಭಿನ್ನ ಭಾವನೆಗಳೆಲ್ಲ ಮಾಯವಾಗಿ ವಿಶ್ವ ಸಹೋದರತ್ವದ ಭಾವನೆ ಬರುವುದು. ಇಂಗ್ಲೆಂಡಿನಲ್ಲಿ ಈ ಕೆಲಸ ನಿಧಾನವಾಗಿ ಆಗುವುದು. ಬಹುಶಃ ಅದಕ್ಕೆ ಸಕಾಲ ಪ್ರಾಪ್ತಿಯಾಗಿಲ್ಲ ಎಂದು ಕಾಣುವುದು. ಆದರೂ ಅದು ನಿಧಾನವಾಗಿ ಮುಂದುವರಿಯುವುದು. ಇಂಗ್ಲೆಂಡ್​ ಇಂಡಿಯಾ ದೇಶದಲ್ಲಿ ಈ ವಿಷಯವಾಗಿ ಏನು ಮಾಡುತ್ತಿದೆ ಎಂಬುದರ ಕಡೆಗೆ ನಿಮ್ಮ ಲಕ್ಷ್ಯವನ್ನು ಸೆಳೆಯುತ್ತೇನೆ. ಭರತಖಂಡದ ಅಭಿವೃದ್ಧಿಗೆ ಜಾತಿಗಳ ಪ್ರತ್ಯೇಕತೆ\break ಒಂದು ಆತಂಕ. ಜಾತಿ ಭಾವನೆ ಪ್ರತ್ಯೇಕಗೊಳಿಸುವುದು, ದೃಷ್ಟಿಯನ್ನು ಸಂಕೋಚಗೊಳಿಸುವುದು, ಇತರರೊಡನೆ ಬೆಳೆಯುವುದಕ್ಕೆ ಅವಕಾಶ ಕೊಡುವುದಿಲ್ಲ. ಭಾವನೆಗಳು ಪ್ರಚಾರ\-ವಾದಂತೆ ಜಾತಿಯ ಗೋಡೆಗಳು ಕುಸಿದು ಬೀಳುವುವು.”

ಬಾತ್ಮೀದಾರ: “ಆದರೂ ಕೆಲವು ಆಂಗ್ಲೇಯರು, ಅವರಿಗೆ ಇಂಡಿಯಾ ದೇಶದ ಮೇಲೆ ಅಭಿಮಾನವಿಲ್ಲವೆಂದಲ್ಲ ಅಥವಾ ಇಂಡಿಯಾ ದೇಶದ ಚರಿತ್ರೆ ಅವರಿಗೆ ಗೊತ್ತಿಲ್ಲ ಎಂತಲೂ ಅಲ್ಲ, ಅಂತಹವರು ಜಾತಿಯು ಭರತಖಂಡಕ್ಕೆ ಹೆಚ್ಚಿನ ಒಳ್ಳೆಯದನ್ನು ಮಾಡಿದೆ ಎನ್ನುವರು. ಇಲ್ಲದೆ ಇದ್ದರೆ ಅವರೆಲ್ಲ ಐರೋಪ್ಯರನ್ನು ಬಹಳವಾಗಿ ಅನುಕರಿಸಿಬಿಡುತ್ತಿದ್ದರು. ನಮ್ಮ ಅನೇಕ ಆದರ್ಶಗಳು ತೀರ ಲೌಕಿಕವಾದುವು ಎಂದು ನೀವೇ ಖಂಡಿಸುತ್ತಿದ್ದೀರಿ.

ಸ್ವಾಮೀಜಿ: “ನಿಜ, ಬುದ್ಧಿ ಇರುವ ಯಾರೂ ಇಂಡಿಯಾ ದೇಶವನ್ನು ಇಂಗ್ಲೆಂಡನ್ನಾಗಿ\break ಮಾಡುವುದಕ್ಕೆ ಪ್ರಯತ್ನಿಸುವುದಿಲ್ಲ. ಆಲೋಚನೆಯಿಂದ ದೇಹ ಆಗಿದೆ. ರಾಷ್ಟ್ರೀಯ ಚಿಂತನೆಯನ್ನು ರಾಷ್ಟ್ರದ ಬಾಹ್ಯ ಆಚಾರಗಳು ವ್ಯಕ್ತಗೊಳಿಸುತ್ತವೆ. ಭರತಖಂಡದಲ್ಲಿ\break ಸಾವಿರಾರು ವರುಷಗಳ ಆರ್ಯರ ಚಿಂತನೆಗಳನ್ನು ಅವು ವ್ಯಕ್ತಗೊಳಿಸುತ್ತವೆ. ಇಂಡಿಯಾ ದೇಶವನ್ನು ಐರೋಪ್ಯ ದೇಶವನ್ನಾಗಿ ಮಾಡಲೆತ್ನಿಸುವುದು ಅಸಾಧ್ಯ ಮತ್ತು ಮೌಢ್ಯ.\break ಭರತಖಂಡದಲ್ಲಿ ಅಭಿವೃದ್ಧಿಯ ಭಾವನೆಗಳು ಯಾವಾಗಲೂ ಜಾಗ್ರತವಾಗಿದ್ದವು.\break ಒಂದು ಸರಿಯಾದ ಸರ್ಕಾರ ಬಂದೊಡನೆ ಆ ಭಾವನೆಗಳು ವ್ಯಕ್ತವಾಗುವುದನ್ನು ನಾವು ನೋಡುವೆವು. ಉಪನಿಷತ್ತಿನ ಕಾಲದಿಂದ ಇಂದಿನವರೆಗೆ ನಮ್ಮ ಆಚಾರ್ಯರೆಲ್ಲಾ ಈ ಜಾತಿಯ ಗೋಡೆಯನ್ನು ಉರುಳಿಸಲು ಯತ್ನಿಸಿದರು. ಅವರು ನಿರ್ಮೂಲ ಮಾಡಲು ಯತ್ನಿಸಿದ್ದು ಅವನತಿ ಹೊಂದಿದ್ದ ಜಾತಿ ಪದ್ಧತಿಯನ್ನು, ಜಾತಿಯ ಮೂಲ ವ್ಯವಸ್ಥೆಯನ್ನಲ್ಲ. ಈಗಿನ ಜಾತಿ ಪದ್ಧತಿಯಲ್ಲಿ ಒಳ್ಳೆಯದೇನಾದರೂ ಇದ್ದರೆ ಅದು ಮೂಲ ಜಾತಿ ಪದ್ಧತಿಯಲ್ಲಿದೆ. ಬುದ್ಧನು ಜಾತಿಯನ್ನು ಅದರ ಪೂರ್ವದ ಸ್ಥಿತಿಗೆ ತರಲು ಯತ್ನಿಸಿದ. ಭರತಖಂಡದ ಜಾಗೃತಿಯ ಕಾಲದಲ್ಲೆಲ್ಲಾ ಜಾತಿಯ ಗೋಡೆಯನ್ನು ಒಡೆಯಲು ಪ್ರಯತ್ನಗಳು ನಡೆದುವು. ಆದರೆ ಯಾವಾಗಲೂ ಭವಿಷ್ಯ ಭಾರತವನ್ನು ಕಟ್ಟುವವರು ನಾವು. ಪುರಾತನ ಭಾವನೆಯ ಪರಿಣಾಮವಾಗಿ, ಆಗಿದ್ದ ಸಮಾಜ ಈಗಲೂ ಮುಂದುವರಿಯುವುದಕ್ಕೆ, ಒಳ್ಳೆಯ ಭಾವನೆಗಳು ಎಲ್ಲಿಂದ ಬಂದರೂ, ಅವನ್ನು ರಕ್ತಗತಮಾಡಿಕೊಂಡು ಬೆಳೆಸುವವರು ನಾವು. ಎಂದಿಗೂ ಹೊರಗಿನವರು ನಮ್ಮನ್ನು ಬೆಳೆಸಲಾರರು. ಬೆಳವಣಿಗೆ ಯಾವಾಗಲೂ ಒಳಗಿನಿಂದ ಆಗಬೇಕು. ಇಂಗ್ಲೆಂಡ್​ ಭರತಖಂಡಕ್ಕೆ ಮಾಡುವ ಸಹಾಯವೆಲ್ಲ ಇಂಡಿಯಾ ದೇಶ ತಾನೇ ತನ್ನ ಕಾಲ ಮೇಲೆ ನಿಂತುಕೊಳ್ಳಲು ನೆರವಾಗುವುದು ಮಾತ್ರ ಆಗಿದೆ. ಯಾರ ಕೈಗಳು ಭಾರತೀಯರ ಕುತ್ತಿಗೆಯ ಮೇಲೆ ಸದಾ ಇರುವುವೋ ಅಂತಹವರು ಹೇಳಿದಂತೆ ಪ್ರಗತಿಯನ್ನು ಸಾಧಿಸಲು ಯತ್ನಿಸುವುದು ನಿಷ್ಪ್ರಯೋಜಕ ಎಂಬುದು ನನ್ನ ಅಭಿಪ್ರಾಯ. ಯಾವ ಒಂದು ಶ್ರೇಷ್ಠವಾದ ಕಾರ್ಯವನ್ನಾಗಲೀ ಜೀತದಾಳುಗಳಿಂದ ಮಾಡಿಸಹೊರಟರೆ ಅದು ಅಧೋಗತಿಗೆ ಬರುವುದು.”

ಬಾತ್ಮೀದಾರ: “ಇಂಡಿಯನ್​ ನ್ಯಾಷನಲ್​ ಕಾಂಗ್ರೆಸ್​ನ ಚಟುವಟಿಕೆಗೆ ನಿಮ್ಮ ಗಮನವನ್ನು ಕೊಟ್ಟಿರುವಿರಾ?”

ಸ್ವಾಮೀಜಿ: “ನಾನು ಹೆಚ್ಚು ಗಮನ ಕೊಟ್ಟಿರುವೆ ಎಂದು ಹೇಳಲಾರೆ. ನನ್ನ ಕೆಲಸ\break ಬೇರೆ ಕಾರ್ಯಕ್ಷೇತ್ರದಲ್ಲಿ. ಆದರೆ ಅದರ ಕಾರ್ಯಕ್ರಮ ಬಹಳ ಮುಖ್ಯವಾದುದೆಂದು ನಾನು ಪರಿಗಣಿಸುತ್ತೇನೆ. ಅದು ತನ್ನ ಗುರಿಯನ್ನು ಸಾಧಿಸಲಿ ಎಂಬುದೇ ನನ್ನ ಅಭಿಲಾಷೆ.\break ಭರತಖಂಡದ ಭಿನ್ನ ಭಿನ್ನ ಜನಾಂಗಗಳಿಂದ ಒಂದು ರಾಷ್ಟ್ರ ನಿರ್ಮಾಣವಾಗಬೇಕಾಗಿದೆ.\break ಅದರಲ್ಲಿ ಯೂರೋಪ್​ ಖಂಡದ ಬೇರೆ ಬೇರೆ ದೇಶಗಳಲ್ಲಿರುವಷ್ಟು ವ್ಯತ್ಯಾಸಗಳಿವೆ\break ಎಂದು ನಾನು ಕೆಲವು ವೇಳೆ ಭಾವಿಸುತ್ತೇನೆ. ಹಿಂದೆ ಯೂರೋಪ್​ ಭರತಖಂಡದಲ್ಲಿ ವ್ಯಾಪಾರಕ್ಕಾಗಿ ಹೋರಾಡಿತು. ಈ ವ್ಯಾಪಾರ, ಜಗತ್ತಿನ ನಾಗರಿಕತೆಯಲ್ಲಿ ಒಂದು\break ಅದ್ಭುತವಾದ ಪರಿಣಾಮವನ್ನುಂಟುಮಾಡಿದೆ. ವ್ಯಾಪಾರ ಅವರ ಕೈಗೆ ಸಿಕ್ಕಿದ ಮೇಲೆ\break ಜಗತ್ತಿನ ಇತಿಹಾಸವೇ ಬೇರೊಂದು ರೂಪವನ್ನು ತಾಳಿದೆ ಎಂದು ಹೇಳಬಹುದು. ಈ ವ್ಯಾಪಾರಕ್ಕಾಗಿ ಡಚ್ಚರು, ಪೋರ್ಚುಗೀಸರು, ಫ್ರೆಂಚರು, ಇಂಗ್ಲಿಷರು ಒಬ್ಬರಾದ ಮೇಲೆ ಒಬ್ಬರು ಹೋರಾಡುತ್ತಿದ್ದರು. ಪೂರ್ವದಲ್ಲಿ ಆದ ನಷ್ಟಕ್ಕಾಗಿ ವೆನಿಸ್ಸಿನವರು ಪಶ್ಚಿಮಕ್ಕೆ\break ಹೋಗಿ ಅಮೆರಿಕ ದೇಶವನ್ನು ಕಂಡುಹಿಡಿದರು ಎಂದು ಬೇಕಾದರೂ ಹೇಳಬಹುದು.”

ಬಾತ್ಮೀದಾರ: “ಇದು ಹೇಗೆ ಪರ್ಯವಸಾನವಾಗುವುದು?”

\eject

ಸ್ವಾಮೀಜಿ: “ಪ್ರಜಾಪ್ರಭುತ್ವದ ಭಾವನೆಗಳನ್ನು ಪಡೆದುಕೊಂಡು ಇಂಡಿಯಾ ದೇಶವು ತನ್ನ ಐಕ್ಯತೆಯನ್ನು ಸಾಧಿಸುವುದರಲ್ಲಿ ಇದು ನಿಜವಾಗಿ ಪರ್ಯವಸಾನವಾಗುವುದು. ಬುದ್ಧಿ ಎಲ್ಲೋ ಸುಸಂಸ್ಕೃತರಾದ ಕೆಲವರ ಸ್ವತ್ತಾಗಕೂಡದು. ಇದನ್ನು ಮೇಲಿನಿಂದ ಕೆಳಗಿರುವವರಿ\-ಗೆಲ್ಲ ಹಂಚಬೇಕು. ವಿದ್ಯಾಭ್ಯಾಸ ಈಗ ಬರುತ್ತಿದೆ, ಕಡ್ಡಾಯ ಶಿಕ್ಷಣವೂ ಕ್ರಮೇಣ ಬರುವುದು. ಕೆಲಸ ಮಾಡಲು ಉತ್ಸಾಹಿತರಾದ ನಮ್ಮ ಜನರಲ್ಲಿರುವ ಅದ್ಭುತ ಶಕ್ತಿಯನ್ನು ಉಪಯೋಗಿಸಿಕೊಳ್ಳಬೇಕು. ಭರತಖಂಡದಲ್ಲಿ ಅದ್ಭುತವಾದ ಶಕ್ತಿಗಳು ಸುಪ್ತವಾಗಿವೆ. ಕಾಲಕ್ರಮೇಣ ಅವು ವ್ಯಕ್ತವಾಗುವುವು.”

ಬಾತ್ಮೀದಾರ: “ದೊಡ್ಡ ಸೇನಾ ಸಹಾಯವಿಲ್ಲದೆ ಎಂದಾದರೂ ಒಂದು ದೇಶ ಶ್ರೇಷ್ಠ ರಾಷ್ಟ್ರವಾಗಿದೆಯೆ?”

ಸ್ವಾಮೀಜಿ: “ಹೌದು, ಚೈನಾ ದೊಡ್ಡ ರಾಷ್ಟ್ರವಾಗಿತ್ತು (ಸ್ವಲ್ಪವೂ ಅನುಮಾನವಿಲ್ಲದೆ ಹೇಳಿದರು). ನಾನು ಚೈನಾ, ಜಪಾನ್​ ದೇಶಗಳನ್ನೂ ನೋಡಿರುವೆನು. ಇಂದು ಚೀನಾದೇಶ ಅವ್ಯವಸ್ಥಿತವಾದ ಒಂದು ದೊಂಬಿಯ ಜನಾಂಗದಂತೆ ಇದೆ. ಆದರೂ ಅದು ಪ್ರವರ್ಧಮಾನದಲ್ಲಿದ್ದ ಕಾಲದಲ್ಲಿ ಇತರ ದೇಶಗಳಿಗೆ ತಿಳಿಯದಿದ್ದ ಎಷ್ಟೋ ಪ್ರಶಂಸನೀಯ\-ವಾದ ಸಂಸ್ಥೆಗಳು ಅಲ್ಲಿ ಇದ್ದುವು. ನಾವು ಆಧುನಿಕ ಎಂದು ಹೇಳುವ ಹಲವು ರೀತಿಯ\break ಉಪಕರಣಗಳು, ವಿಧಾನಗಳು ಚೈನೀಯರಲ್ಲಿ ಸಾವಿರಾರು ವರುಷಗಳ ಹಿಂದೆಯೇ ಬಳಕೆಯಲ್ಲಿ ಇದ್ದವು. ಉದಾಹರಣೆಗೆ ಸ್ಪರ್ಧಾತ್ಮಕ ಪರೀಕ್ಷೆಗಳನ್ನೇ ತೆಗೆದುಕೊಳ್ಳಿ.”

ಬಾತ್ಮೀದಾರ: “ಆದರೆ ಏತಕ್ಕೆ ಈಗ ಅಲ್ಲಿ ಒಂದು ಐಕಮತ್ಯವಿಲ್ಲ?”

ಸ್ವಾಮೀಜಿ: “ಏಕೆಂದರೆ ಆ ಸಂಸ್ಥೆಯನ್ನು ನಿರ್ವಹಿಸುವಂತಹ ವ್ಯಕ್ತಿಗಳನ್ನು ಅದು ನಿರ್ಮಿಸಲಿಲ್ಲ. ಪಾರ್ಲಿಮೆಂಟಿನ ಒಂದು ಶಾಸನದಿಂದ ಜನರನ್ನು ಧಾರ್ಮಿಕರನ್ನಾಗಿ\break ಮಾಡಲಾಗುವುದಿಲ್ಲ ಎಂಬುದು ನಿಮಗೆ ಗೊತ್ತಿದೆ. ಚೈನೀಯರು ಇದನ್ನು ನಿಮಗಿಂತ ಮುಂಚೆ ಅರಿತರು. ಆದಕಾರಣವೇ ಧರ್ಮ ರಾಜಕೀಯಕ್ಕಿಂತ ಮುಖ್ಯ. ಇದು ಜನಾಂಗದ ಆಳಕ್ಕೆ ಹೋಗಿ ಅವರ ಶೀಲವನ್ನು ತಿದ್ದಲು ಯತ್ನಿಸುವುದು.”

ಬಾತ್ಮೀದಾರ: “ನೀವು ಹೇಳುವ ಜಾಗೃತಿ ಭಾರತೀಯರಿಗೆ ಅರಿವಾಗಿದೆಯೆ?”

ಸ್ವಾಮೀಜಿ: “ಪೂರ್ಣ ಅರಿವಾಗಿದೆ. ಜಗತ್ತು ಕಾಂಗ್ರೆಸ್​ ಚಳುವಳಿ ಮತ್ತು ಸಾಮಾಜಿಕ ಸುಧಾರಣಾ ಕ್ಷೇತ್ರಗಳಲ್ಲಿ ಮಾತ್ರ ಮುಖ್ಯವಾಗಿ ಅದನ್ನು ಕಾಣುತ್ತಿರಬಹುದು. ಆದರೆ\break ಈ ಜಾಗೃತಿ ಧಾರ್ಮಿಕ ಪ್ರಪಂಚದಲ್ಲಿಯೂ ಅಷ್ಟೇ ಸತ್ಯವಾಗಿ ಆಗುತ್ತಿದೆ, ಆದರೆ ಹೆಚ್ಚು\break ಮೌನವಾಗಿ.”

ಬಾತ್ಮೀದಾರ: “ಪೌರಸ್ತ್ಯ ಪಾಶ್ಚಾತ್ಯರ ಜೀವನದ ಆದರ್ಶಗಳಲ್ಲಿ ಎಷ್ಟೋ ವ್ಯತ್ಯಾಸಗಳಿವೆ. ನಾವು ಸಾಮಾಜಿಕ ಸ್ಥಿತಿಗತಿಗಳನ್ನು ಪರಿಪೂರ್ಣಗೊಳಿಸಲು ಯತ್ನಿಸುತ್ತಿರುವೆವು. ನಾವು ಈ ಕೆಲಸದಲ್ಲಿ ನಿರತರಾಗಿರುವಾಗ ಪೌರಸ್ತ್ಯರು ಸೂಕ್ಷ್ಮ ತತ್ತ್ವಗಳ ಚಿಂತನೆಯಲ್ಲಿ\break ನಿರತರಾಗಿರುವರು. ಇಲ್ಲಿ ಸುಡಾನಿನ ಭಾರತೀಯ ಸಿಪಾಯಿಗಳ ಸಂಬಳದ ಬಗ್ಗೆ\break ಪಾರ್ಲಿಮೆಂಟ್​ ಚರ್ಚಿಸುತ್ತಿರುವುದು. ಕನ್ಸರ್ವೆಟಿವ್​ ಗುಂಪಿನ ಪತ್ರಿಕೆಗಳೆಲ್ಲ ಸರ್ಕಾರದ ನಿರ್ಧಾರದ ಅನ್ಯಾಯದ ವಿಷಯವಾಗಿ ಟೀಕಿಸುತ್ತಿವೆ. ಆದರೆ ನೀವು ಇವಾವುದನ್ನೂ\break ಗಮನಕ್ಕೆ ತೆಗೆದುಕೊಳ್ಳುವಷ್ಟು ಮುಖ್ಯ ಎಂದು ಭಾವಿಸುವುದಿಲ್ಲ ಎಂದು ಕಾಣುವುದು.”

ಸ್ವಾಮೀಜಿ: (ಕನ್ಸರ್ವೆಟಿವ್​ ಪೇಪರನ್ನು ತೆಗೆದುಕೊಂಡು ನೋಡುತ್ತಾ) “ಆದರೆ\break ನೀವು ಹಾಗೆ ಭಾವಿಸಬೇಡಿ. ಇಲ್ಲಿ ನನ್ನ ಸಹಾನುಭೂತಿ ಸ್ವಾಭಾವಿಕವಾಗಿ ನನ್ನ ದೇಶದ\break ಕಡೆ ಇರುವುದು. ಆದರೂ ಒಂದು ಹಳೆಯ ಸಂಸ್ಕೃತ ಗಾದೆ ಜ್ಞಾಪಕಕ್ಕೆ ಬರುವುದು. ‘ಆನೆಯನ್ನೇ ಮಾರಿದ ಮೇಲೆ ಅದರ ಅಂಕುಶಕ್ಕೆ ಏತಕ್ಕೆ ಕಿತ್ತಾಟ?’ ಇಂಡಿಯಾ ದೇಶವೇ ಯಾವಾಗಲೂ ಹಣ ತೆರುವುದು. ರಾಜಕಾರಣಿಗಳ ಜಗಳ ಬಹಳ ವಿಚಿತ್ರ. ರಾಜಕೀಯಕ್ಕೆ ಧರ್ಮವನ್ನು ತರಬೇಕಾದರೆ ಶತಮಾನಗಳು ಬೇಕಾಗುವುದು. ಆದರೂ ಅದಕ್ಕಾಗಿ ಬೇಗನೆ ಪ್ರಯತ್ನಿಸಬೇಕು.

ಬಾತ್ಮೀದಾರ: “ಹೌದು, ಈ ಲಂಡನ್​ ಮಹಾನಗರಿಯಲ್ಲಿ ಒಂದು ಭಾವನೆಯನ್ನು\break ಬಿತ್ತುವುದು ಪ್ರಯೋಜನಕರ. ಅದು ಪ್ರಪಂಚದಲ್ಲೇ ಅತಿ ದೊಡ್ಡ ಆಡಳಿತ ಯಂತ್ರ. ನಾನು ಅನೇಕ ವೇಳೆ ಅದು ಕೆಲಸ ಮಾಡುವುದನ್ನು ನೋಡಿರುವೆನು. ಈ ಯಂತ್ರದಲ್ಲಿ ಅದ್ಭುತ\break ಶಕ್ತಿ ಇದೆ, ಪ್ರಾವೀಣ್ಯ ಇದೆ. ಇದು ಒಂದು ಅತ್ಯಂತ ಸಣ್ಣ ವಿಷಯಕ್ಕೂ ಗಮನ ಕೊಡುವುದು. ಈ ಯಂತ್ರ ಹೇಗೆ ಕೆಲಸ ಮಾಡುತ್ತಿದೆ. ಅಧಿಕಾರವನ್ನು ಹಂಚುತ್ತಿದೆ ಎಂಬುದೊಂದು ಅದ್ಭುತ. ಈ ಚಕ್ರಾಧಿಪತ್ಯ ಎಷ್ಟು ದೊಡ್ಡದು ಮತ್ತು ಇದು ಮಾಡುವ ಕೆಲಸ ಎಷ್ಟು\break ಅದ್ಭುತವಾದುದು ಎಂಬುದನ್ನು ನೋಡುವುದಕ್ಕೆ ಇದು ಸಹಾಯ ಮಾಡುವುದು. ಇತರರಿಗೆ ಇದು ಭಾವನೆಗಳನ್ನು ಹಂಚುವುದು. ಈ ಮಹಾಯಂತ್ರದಲ್ಲಿ ಒಂದು ಭಾವನೆಯನ್ನು ಇಡುವುದು ಎಷ್ಟೇ ಕಷ್ಟವಾದರೂ ಅದು ಪ್ರಯೋಜನಕಾರಿಯಾದುದೇ ಸರಿ. ಏಕೆಂದರೆ ಈ ಯಂತ್ರದ ಮೂಲಕ ಪ್ರಪಂಚದ ಯಾವ ಮೂಲೆಗಾದರೂ ಭಾವನೆಗಳನ್ನು\break ಕಳುಹಿಸಬಹುದು.”

ಸ್ವಾಮೀಜಿಯವರದು ಎದ್ದು ಕಾಣುವ ವ್ಯಕ್ತಿತ್ವ. ಎತ್ತರವಾಗಿದ್ದಾರೆ. ವಿಶಾಲವಾದ ದೇಹ, ಲಕ್ಷಣವಂತರು. ಪೌರಸ್ತ್ಯ ಪೋಷಾಕಿನಲ್ಲಿ ಅವರು ಇನ್ನೂ ಹೆಚ್ಚು ಆಕರ್ಷಣೀಯರಾಗಿರುವರು. ಇವರ ಜನ್ಮಸ್ಥಳ ಬಂಗಾಳ. ಕಲ್ಕತ್ತಾ ವಿಶ್ವವಿದ್ಯಾನಿಲಯದ ಪದವೀಧರರು. ಅದ್ಭುತವಾದ ಪ್ರೌಢಿಮೆಯುಳ್ಳ ವಾಗ್ಮಿ. ಅವರು ಒಂದೂವರೆ ಗಂಟೆವರೆಗೆ ಯಾವ ಟಿಪ್ಪಣಿಗಳನ್ನೂ ಇಟ್ಟುಕೊಳ್ಳದೆ, ಒಂದು ಮಾತಿಗೂ ತಡವರಿಸದೆ ಒಂದೇ ಸಮನಾಗಿ ಉಪನ್ಯಾಸ ಮಾಡಬಲ್ಲರು.

\begin{flushright}
ಬಾತ್ಮೀದಾರ
\end{flushright}

