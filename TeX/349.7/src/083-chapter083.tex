
\chapter[ಭಗವತ್ಪ್ರೇಮ -೨ ]{ಭಗವತ್ಪ್ರೇಮ -೨ \protect\footnote{\engfoot{C.W. Vol. VIII, P. 201}}}

\centerline{(೧೮೯೪, ಫೆಬ್ರವರಿ ೨೦ರಂದು ಡೆಟ್ರಾಯ್ಟ್​ನ ಯೂನಿಟೇರಿಯನ್​ ಚರ್ಚ್​ನಲ್ಲಿ ನೀಡಿದ ಉಪನ್ಯಾಸ. “ಡೆಟ್ರಾಯ್ಟ್​ ಫ್ರೀ ಪ್ರೆಸ್​” ನಲ್ಲಿ ಪ್ರಕಟವಾದ ವರದಿ)}

ದೇವರು ನಮಗೆ ನಿಜವಾಗಿ ಬೇಕಾಗಿದ್ದಾನೆ ಎಂದು ನಾವು ಅವನನ್ನು ಸ್ವೀಕರಿಸು ವುದಿಲ್ಲ; ಆದರೆ ನಮ್ಮ ಸ್ವಾರ್ಥಕ್ಕೆ ಅವನಿಂದ ಸಹಾಯವಾಗುವುದೆಂದು ನಾವು ಅವನನ್ನು ಸ್ವೀಕರಿಸುತ್ತೇವೆ. ಪ್ರೀತಿ ನಿಃಸ್ವಾರ್ಥವಾದುದು. ನಾವು ಪ್ರೀತಿಸುವ ವಸ್ತು ವನ್ನು ಗೌರವಿಸುವುದು, ಕೊಂಡಾಡುವುದು ಅಲ್ಲದೆ ಬೇರೆ ಯಾವ ಭಾವನೆಯೂ ಅಲ್ಲಿ ಬರುವುದಿಲ್ಲ. ಭಕ್ತನು ಸುಮ್ಮನೆ ಬಾಗಿ ಪೂಜಿಸುವನೇ ಹೊರತು ಮತ್ತಾ ವುದನ್ನೂ ಅವನು ಕೇಳುವುದಿಲ್ಲ. ಭಕ್ತನು ಕೇವಲ ಭಗವಂತನನ್ನು ಮಾತ್ರ ಪ್ರೀತಿಸ ಬೇಕೆಂದು ಅವನನ್ನು ಬೇಡುತ್ತಾನೆ.

ಹಿಂದೂ ಭಕ್ತಳೊಬ್ಬಳು ಮದುವೆಯಾದಾಗ ತನಗೆ ಆಗಲೆ ಮದುವೆಯಾಗಿದೆ ಎಂದು ತನ್ನ ಗಂಡನಾದ ದೊರೆಗೆ ಹೇಳಿದಳು. ದೊರೆ ಯಾರೊಂದಿಗೆ ಎಂದ. ದೇವರೊಂದಿಗೆ ಎಂದಳು. ಅವಳು ದೀನರ ದರಿದ್ರರ ನಡುವೆ ಹೋಗಿ ಅವರಿಗೆ ಪರಮಭಕ್ತಿಯ ವಿಷಯವನ್ನು ಬೋಧಿಸಿದಳು. ಅವಳ ಒಂದು ಪ್ರಾರ್ಥನೆ ಅತಿ ಮುಖ್ಯವಾಗಿದೆ-ಅವಳ ಹೃದಯವನ್ನು ಅದು ಬಿಚ್ಚಿ ತೋರುವಂತೆ ಇದೆ: “ನನಗೆ ಐಶ್ವರ್ಯ ಬೇಡ, ಅಧಿಕಾರ ಬೇಡ, ಮುಕ್ತಿ ಬೇಡ. ನಿನಗೆ ಇಚ್ಛೆಯಾದರೆ ನೂರು ನರಕಗಳಿಗೆ ಬೇಕಾದರೂ ನನ್ನನ್ನು ತಳ್ಳು. ಆದರೆ ಯಾವಾಗಲೂ ನಿನ್ನನ್ನು ನನ್ನ ಪ್ರಿಯತಮ ಎಂದು ತಿಳಿಯುವಂತೆ ಮಾತ್ರ ಮಾಡು.” ನಮ್ಮ ಪುರಾತನ ಭಾಷೆ ಯಲ್ಲಿ ಇವಳ ಹಲವು ಸುಂದರವಾದ ಪ್ರಾರ್ಥನೆಗಳು ಇವೆ. ಅವಳು ಕೊನೆಗಾಲ ದಲ್ಲಿ ಒಂದು ನದಿಯ ತೀರದಲ್ಲಿ ಸಮಾಧಿಸ್ಥಳಾದಳು. ಅಗ ಅವಳು ಒಂದು ಸುಂದರವಾದ ಹಾಡನ್ನು ಕಟ್ಟಿದಳು. ನಾನು ನನ್ನ ಪ್ರಿಯತಮನನ್ನು ನೋಡುವುದಕ್ಕೆ ಹೋಗುತ್ತೇನೆ ಎಂಬುದೇ ಅದು.

ಪುರುಷರು ಬೇಕಾದರೆ ತಾತ್ತ್ವಿಕ ವಿಷಯಗಳನ್ನು ಅರ್ಥಮಾಡಿಕೊಳ್ಳಬಲ್ಲರು. ಆದರೆ ಸ್ತ್ರೀ ಸ್ವಭಾವತಃ ಭಕ್ತಳು. ಅವಳು ದೇವರನ್ನು ಹೃದಯದ ಮೂಲಕ ಪ್ರೀತಿಸುವಳೆ ಹೊರತು ಬುದ್ಧಿಯ ಮೂಲಕ ಅಲ್ಲ. ಸಾಲಮನ್ನಿನ ಪ್ರಾರ್ಥನೆಗಳು ಬೈಬಲ್ಲಿನಲ್ಲಿರುವ ಅತಿ ಸುಂದರವಾದ ಭಾಗ. ಅದರ ಭಾಷೆ ಹಿಂದೂ ಭಕ್ತಳ ಭಾಷೆಯಂತೆ ಇದೆ. ಆದರೂ ಕ್ರೈಸ್ತರು ಅವುಗಳನ್ನು ಬೈಬಲ್ಲಿನಿಂದ ತೆಗೆದುಹಾಕ ಬೇಕೆಂದು ಇರುವರು ಎಂದು ಕೇಳಿದ್ದೇನೆ. ಆ ಒಂದು ಹಾಡಿನ ಈ ರೀತಿಯ ವಿವರಣೆ ಯನ್ನು ನಾನು ಕೇಳಿದೆ: ಸಾಲಮನ್ನನು ಒಬ್ಬ ಹುಡುಗಿಯನ್ನು ಪ್ರೀತಿಸುತ್ತಿದ್ದನಂತೆ. ಅವಳೂ ಕೂಡ ತನ್ನನ್ನು ಪ್ರೀತಿಸಬೇಕೆಂದು ಅವನ ಇಚ್ಛೆ ಇತ್ತಂತೆ. ಆದರೆ ಹುಡುಗಿ ಬೇರೆ ಒಬ್ಬ ಯುವಕನನ್ನು ಪ್ರೀತಿಸುತ್ತಿದ್ದುದರಿಂದ ಸಾಲಮನ್ನನನ್ನು ಪ್ರೀತಿಸಲಿಲ್ಲ. ಈ ವಿವರಣೆಯೇ ಅವರಿಗೆ ತುಂಬಾ ಚೆನ್ನಾಗಿ ಅರ್ಥವಾಗುವುದು. ಆದರೆ ಅವರಿಗೆ ಸಾಲಮನ್ನನ ಪ್ರಾರ್ಥನೆಯಲ್ಲಿರುವ ಅದ್ಭುತ ಭಗವದ್​ ಭಕ್ತಿ ಅರ್ಥವಾಗುವುದಿಲ್ಲ. ಇಂಡಿಯಾದೇಶದಲ್ಲಿನ ದೇವರ ಭಕ್ತಿ ಇತರ ಕಡೆಗಳ ದೇವರ ಭಕ್ತಿಯಂತೆ ಅಲ್ಲ. ನೀವು ಸೊನ್ನೆಯ ಕೆಳಗೆ ನಲವತ್ತು ಡಿಗ್ರಿ ಇರುವ ದೇಶಕ್ಕೆ ಬಂದಾಗ ಅಲ್ಲಿ ಜನರ ಸ್ವಭಾವ ಬದಲಾಗುವುದು. ಬೈಬಲ್ಲನ್ನು ಬರೆದ ದೇಶದ ಆದರ್ಶಕ್ಕೂ, ಹಣದಾಸೆ ಪ್ರಮುಖವೆಂದು ಭಾವಿಸುವ ದೇಶದ ಆದರ್ಶಕ್ಕೂ ಎಷ್ಟೋ ವ್ಯತ್ಯಾಸವಿದೆ. ಪಾಶ್ಚಾ ತ್ಯರು ಬೇಕಾದರೆ ದೇವರಿಗಿಂತ ಹೆಚ್ಚಾಗಿ ಡಾಲರನ್ನು ಭಯಭಕ್ತಿಯಿಂದ ಆರಾಧಿಸು ವರು. ದೇವರ ಮೇಲಿನ ಪ್ರೀತಿಯಿಂದ ನಮಗೆ ಎಷ್ಟು ಲಾಭ ಆಗುವುದು ಎಂಬ ದೃಷ್ಟಿಯಿಂದ ನೋಡುವರು. ಅವರು ಪ್ರಾರ್ಥಿಸುವಾಗ ಎಲ್ಲಾ ವಿಧದ ಸ್ವಾರ್ಥ ಬಯಕೆಗಳನ್ನೂ ಕೋರುವರು.

ಕ್ರೈಸ್ತರಿಗೆ ದೇವರು ಯಾವಾಗಲೂ ಏನನ್ನಾದರೂ ಕೊಡಬೇಕು. ಸರ್ವೇಶ್ವರ ನೆದುರಿಗೆ ಅವರು ಭಿಕ್ಷುಕರಂತೆ ಬರುವರು. ಭಿಕ್ಷುಕನೊಬ್ಬ ಚಕ್ರವರ್ತಿಯಿಂದ ಭಿಕ್ಷೆ ಯನ್ನು ಬೇಡಲು ಹೋದ ಕಥೆ ಇದು. ಭಿಕ್ಷುಕ ಕಾಯುತ್ತಿದ್ದಾಗ ಚಕ್ರವರ್ತಿಗೆ ಪ್ರಾರ್ಥನೆಯ ಸಮಯವಾಯಿತು. ಆಗ ಆತ ಹೀಗೆ ಪ್ರಾರ್ಥಿಸತೊಡಗಿದ: “ದೇವರೆ, ನನಗೆ ಇನ್ನೂ ಹೆಚ್ಚು ಹೆಚ್ಚು ಐಶ್ವರ್ಯವನ್ನು ಕೊಡು, ಹೆಚ್ಚು ಅಧಿಕಾರವನ್ನು ಕೊಡು, ಹೆಚ್ಚು ರಾಜ್ಯವನ್ನು ಕೊಡು.” ಭಿಕ್ಷುಕ ಅಲ್ಲಿಂದ ಹೊರಡಲು ಅಣಿಯಾದ. ಚಕ್ರವರ್ತಿ ಏತಕ್ಕೆ ಭಿಕ್ಷೆ ತೆಗೆದುಕೊಳ್ಳದೆ ಹೋಗುತ್ತೀಯೆ ಎಂದು ಕೇಳಿದ. ನಾನು ಭಿಕ್ಷುಕರಿಂದ ಭಿಕ್ಷೆಯನ್ನು ಬೇಡುವುದಿಲ್ಲ ಎಂದು ಭಿಕ್ಷುಕ, ಉತ್ತರ ಕೊಟ್ಟನು.

ಕೆಲವರಿಗೆ ಮಹಮ್ಮದನ ಭಕ್ತಿಯ ಹುಚ್ಚು ಅರ್ಥವಾಗುವುದಿಲ್ಲ.ಅವನು ನೆಲದ ಮೇಲೆ ಬಿದ್ದು ಹೊರಳಾಡುತ್ತಿದ್ದ. ಇಂತಹ ವ್ಯಥೆಯನ್ನು ಪಟ್ಟ ಮಹಾತ್ಮರನ್ನು ಮೂರ್ಛೆಯಿಂದ ನರಳುತ್ತಿರುವರು ಎನ್ನುವರು. ಯಾವ ವಿಧವಾದ ಸ್ವಾರ್ಥದ ಚಿಹ್ನೆಯೂ ಇಲ್ಲದಿರುವುದೇ ನಿಜವಾದ ಭಕ್ತಿಯ ಲಕ್ಷಣ. ಈಗಿನ ಕಾಲದಲ್ಲಿ ಧರ್ಮ ಬರಿಯ ಹುಡುಗಾಟವಾಗಿದೆ, ಷೋಕಿಯಾಗಿದೆ. ಜನರು ಒಂದು ಕುರಿಯ ಮಂದೆ ಯಂತೆ ಚರ್ಚಿಗೆ ಹೋಗುತ್ತಾರೆ. ದೇವರು ತಮಗೆ ಬೇಕಾಗಿದ್ದಾನೆ ಎಂದು ಅವನನ್ನು ಅಪ್ಪುವುದಿಲ್ಲ. ಅನೇಕ ಜನರು ತಮಗೆ ತಿಳಿಯದೆ ನಾಸ್ತಿಕರಾಗಿರುವರು. ಆದರೂ ತಾವು ಭಕ್ತರು ಎಂದು ಆತ್ಮ ಸಂತುಷ್ಟಿಯಿಂದ ಭಾವಿಸುವರು.

