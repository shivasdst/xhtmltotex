
\chapter[ನಿಯಮ ಮತ್ತು ಮುಕ್ತಿ ]{ನಿಯಮ ಮತ್ತು ಮುಕ್ತಿ \protect\footnote{\engfoot{C.W. Vol. V, P. 286}}}

ಮುಕ್ತನಿಗೆ ಹೋರಾಟ ಎಂಬುದು ಅರ್ಥವೇ ಆಗುವುದಿಲ್ಲ. ಆದರೆ ನಮ್ಮ ಪಾಲಿಗೆ ಅದಕ್ಕೆ ಅರ್ಥವಿದೆ. ಏಕೆಂದರೆ ನಾಮರೂಪಗಳೇ ಜಗತ್ತನ್ನು ಸೃಷ್ಟಿಸುವುದು.

ವೇದಾಂತದಲ್ಲಿ ಹೋರಾಟಕ್ಕೆ ಒಂದು ಸ್ಥಳವಿದೆ, ಆದರೆ ಅಂಜಿಕೆಗೆ ಎಡೆಯಿಲ್ಲ. ನಿಮ್ಮ ನೈಜಸ್ವಭಾವವನ್ನು ಗಟ್ಟಿಯಾಗಿ ಸ್ಥಾಪಿಸುವುದಾದರೆ ಅಂಜಿಕೆಗಳೆಲ್ಲ ಮಾಯವಾಗುವುವು. ನೀವು ಬದ್ಧರೆಂದು ಭಾವಿಸಿದರೆ ಬದ್ಧರಾಗಿಯೇ ಇರುವಿರಿ, ಮುಕ್ತರೆಂದು ಭಾವಿಸಿದರೆ ಮುಕ್ತರಾಗಿಯೇ ಇರುವಿರಿ.

ನಾವು ವ್ಯಕ್ತಪ್ರಪಂಚದಲ್ಲಿರುವಾಗ ಅನುಭವಿಸುವ ಸ್ವಾತಂತ್ರ್ಯವು ನಿತ್ಯ ಮುಕ್ತಾವಸ್ಥೆಯ ಒಂದು ಕ್ಷಣಿಕ ದರ್ಶನ ಅಷ್ಟೆ, ಆದರೂ ನಿಜವಲ್ಲ.

ಮುಕ್ತಿಯು ಪ್ರಕೃತಿ ನಿಯಮಗಳಿಗೆ ಅಧೀನವೆಂಬುದನ್ನು ನಾವು ಒಪ್ಪಿಕೊಳ್ಳುವುದಿಲ್ಲ. ನನಗೆ ಇದು ಅರ್ಥವೇ ಆಗುವುದಿಲ್ಲ. ಮಾನವ ಪ್ರಕೃತಿಯ ಅಣತಿಗೆ ವಿರೋಧವಾಗಿ ಹೋದಾಗಲೆ ಅವನು ಮುಂದುವರಿದಿರುವುದು ತೋರುವುದು. ಕೆಳಗಿನ ನಿಯಮಗಳನ್ನು ಮೇಲಿನ ನಿಯಮಗಳಿಂದ ಗೆದ್ದರು ಎನ್ನಬಹುದು. ಅಲ್ಲಿಯೂ ಕೂಡ ಮನಸ್ಸು ಸ್ವತಂತ್ರವಾಗುವುದಕ್ಕೆ ಪ್ರಯತ್ನಿಸುತ್ತಿತ್ತು. ಹೋರಾಟ ಕೂಡ ನಿಯಮದ ಮೂಲಕ ಎಂಬುದನ್ನು ಅರಿತಾಗ, ಅದನ್ನೂ ಕೂಡ ಅದು ಜಯಿಸಲು ಯತ್ನಿಸಿತು. ಆದಕಾರಣ ಸರ್ವ ಸ್ವಾತಂತ್ರ್ಯವೇ ಆದರ್ಶ. ಮರವು ಪ್ರಕೃತಿಯ ನಿಯಮವನ್ನು ಎಂದಿಗೂ ಸುಳ್ಳು ಹೇಳುವುದಿಲ್ಲ. ಆದರೂ ಅವು ಮನುಷ್ಯನಿಗಿಂತ ಮೇಲಲ್ಲ. ಸ್ವಾತಂತ್ರ್ಯವನ್ನು ಅದ್ಭುತವಾದ ರೀತಿಯಲ್ಲಿ ಸ್ಥಾಪಿಸುವುದೇ, ಅದರ ಸಮರ್ಥನೆಯೇ ಜೀವನ. ನಿಯಮಕ್ಕೆ ಬಾಗುವುದನ್ನು ನಾವು ಒಂದು ಅತಿಗೆ ತೆಗೆದುಕೊಂಡು ಹೋದರೆ, ಅದು ಸಮಾಜದಲ್ಲಾಗಲಿ, ರಾಜಕೀಯದಲ್ಲಾಗಲಿ, ಧಾರ್ಮಿಕ ಕ್ಷೇತ್ರದಲ್ಲಾಗಲಿ, ನಮ್ಮನ್ನು ಜಡತೆಗೆ ಒಯ್ಯುವುದು. ನಿಯಮಗಳ ಆಧಿಕ್ಯವೇ ಮೃತ್ಯುವಿನ ಚಿಹ್ನೆ. ಯಾವ ಸಮಾಜದಲ್ಲಿ ಹೆಚ್ಚು ನಿಯಮಗಳಿರುತ್ತವೆಯೋ ಆ ಸಮಾಜದ ನಾಶ ಸನ್ನಿಹಿತವಾಗಿದೆ. ಭರತಖಂಡದ ಸ್ವರೂಪವನ್ನು ನೀವು ಅಧ್ಯಯನ ಮಾಡಿದರೆ ಜಗತ್ತಿನ ಮತ್ತಾವ ದೇಶದಲ್ಲಿಯೂ ಇಷ್ಟೊಂದು ನಿಯಮಗಳು ಇರಲಿಲ್ಲ ಎಂಬುದು ಗೊತ್ತಾಗುತ್ತದೆ. ಆದಕಾರಣವೇ ಈ ಜನಾಂಗವು ಸತ್ತಂತೆ ಇದೆ. ಆದರೆ ಹಿಂದೂಗಳಲ್ಲಿ ಒಂದು ವೈಶಿಷ್ಟ್ಯ ಇದೆ. ಅವರು ಧಾರ್ಮಿಕ ಕ್ಷೇತ್ರದಲ್ಲಿ ಒಂದು ಮತ ತತ್ತ್ವವನ್ನಾಗಲಿ, ಸಿದ್ಧಾಂತವನ್ನಾಗಲಿ ಮಾಡಲಿಲ್ಲ. ಆದಕಾರಣ ಅದು ಚೆನ್ನಾಗಿ ಬೆಳೆಯಿತು. ನಿತ್ಯನಿಯಮವೆಂಬುದು ಯಾವಾಗಲೂ ಸ್ವಾತಂತ್ರ್ಯವಾಗಲಾರದು. ಏಕೆಂದರೆ ನಿತ್ಯವಾದುದು ನಿಯಮಬದ್ಧವೆಂದು ಹೇಳಿದರೆ ಅದನ್ನು ಮಿತಗೊಳಿಸಿದಂತೆ.

ದೇವರಿಗೆ ಯಾವ ಉದ್ದೇಶವೂ ಇರಲಾರದು. ಯಾವುದಾದರೂ ಉದ್ದೇಶವಿದ್ದರೆ ಅವನು ಮನುಷ್ಯನಿಗಿಂತ ಮೇಲಾಗಲಾರ. ಅವನಿಗೆ ಏತಕ್ಕೆ ಉದ್ದೇಶ ಇರಬೇಕು? ಯಾವುದಾದರೂ ಉದ್ದೇಶವಿದ್ದರೆ ಅವನೇ ಅದಕ್ಕೆ ಬದ್ಧನಾಗುವನು. ಅವನಿಗಿಂತ ಮಿಗಿಲಾದುದು ಮತ್ತೊಂದು ಇದ್ದಂತೆ ಆಯಿತು. ನೇಯ್ಗೆಯವನು ಒಂದು ಚೂರು ಬಟ್ಟೆಯನ್ನು ನೇಯುವನು ಎಂದು ಇಟ್ಟುಕೊಳ್ಳೋಣ. ಎಂದರೆ ಆ ಭಾವನೆ ಅವನಿಗಿಂತ ಹೊರಗಿತ್ತು, ಅದು ಅವನಿಗಿಂತ ಮೇಲಾಗಿತ್ತು. ಈಗ ದೇವರು ಯಾವ ಭಾವನೆಯೊಂದಿಗೆ ತನ್ನನ್ನು ಹೊಂದಿಸಿಕೊಳ್ಳಬೇಕಾಗಿದೆ? ಹೇಗೆ ರಾಜಾಧಿರಾಜರು ಕೆಲವು ವೇಳೆ ಆಟದಲ್ಲಿ ಮಗ್ನರಾಗುವರೋ ಹಾಗೆ ದೇವರು ಪ್ರಕೃತಿಯೊಡನೆ ಆಡುತ್ತಿರುವನು. ನಿಯಮವೆಂದರೆ ಇದೇ. ನಮ್ಮ ಕಣ್ಣೆದುರಿಗೆ ಕೆಲವು ಘಟನೆಗಳು ನಿಯಮಕ್ಕೆ ಬಾಗಿ ನಡೆಯುವುದರಿಂದ ನಿಯಮವಿದೆ ಎಂದು ಊಹಿಸುವೆವು. ಈ ನಿಯಮದ ಭಾವನೆಯೆಲ್ಲ ಈ ಒಂದು ಸಣ್ಣ ಕ್ಷೇತ್ರಕ್ಕೆ ಮಾತ್ರ ಅನ್ವಯಿಸುವುದು. ನಿಯಮ ಎಂದೆಂದಿಗೂ ಇರುವಂತಹದು, ಎಸೆದ ಕಲ್ಲು ಯಾವಾಗಲೂ ಬೀಳುವುದು ಎನ್ನುವುದು ಒಂದು ಭ್ರಾಂತಿ. ಯುಕ್ತಿಯೆಲ್ಲ ಅನುಭವದ ಮೇಲೆ ನಿಂತಿದ್ದರೆ ಐದು ಕೋಟಿ ವರುಷಗಳ ಹಿಂದೆ ಎಸೆದ ಕಲ್ಲು ಬಿದ್ದುದನ್ನು ಯಾರು ನೋಡಿದ್ದರು? ನಿಯಮ ಎಂಬುದು ಮಾನವ ಸ್ವಭಾವಕ್ಕೆ ಸೇರಿಲ್ಲ. ಮಾನವನು ಎಲ್ಲಿಂದ ಪ್ರಾರಂಭಿಸಿದ್ದಾನೊ ಅಲ್ಲಿಗೆ ಹೋಗಿ ಮುಟ್ಟುತ್ತಾನೆ ಎಂಬುದು ವೈಜ್ಞಾನಿಕ ಸಮರ್ಥನೆ. ನಿಜವಾಗಿ ನಾವು ಕ್ರಮೇಣ ನಿಯಮದಿಂದ ಪಾರಾಗುವುದನ್ನು ನೋಡುವೆವು. ಕೊನೆಗೆ ಅದರಿಂದ ಸಂಪೂರ್ಣ ಪಾರಾಗುವೆವು. ಅಲ್ಲಿ ಆಗ ಇಡೀ ಜೀವನದ ಅನುಭವ ಇರುವುದು. ನಮ್ಮ ಪ್ರಾರಂಭ ಆದದ್ದು ದೇವರಲ್ಲಿ ಮತ್ತು ಸ್ವಾತಂತ್ರ್ಯ ದಲ್ಲಿ. ದೇವರಲ್ಲಿ ಮತ್ತು ಸ್ವಾತಂತ್ರ್ಯದಲ್ಲಿಯೇ ನಮ್ಮ ಕೊನೆಯಾಗುವುದು. ಈ ನಿಯಮಗಳು ನಾವು ಸಾಗಿ ಹೋಗಬೇಕಾದ ಮಧ್ಯಸ್ಥಿತಿಯಲ್ಲಿವೆ. ನಮ್ಮ ವೇದಾಂತ ಯಾವಾಗಲೂ ಸ್ವಾತಂತ್ರ್ಯವನ್ನು ಒತ್ತಿ ಹೇಳುತ್ತದೆ. ನಿಯಮದ ಭಾವನೆಯೇ ವೇದಾಂತಿಗಳಿಗೆ ಹಿಡಿಸದು. ನಿತ್ಯನಿಯಮ ಅತಿ ಭಯಾನಕವಾದುದು. ಏಕೆಂದರೆ ನಿಯಮವು ನಿತ್ಯವಾದರೆ ಅದರಿಂದ ಪಾರಾಗುವುದಕ್ಕೆ ಸಾಧ್ಯವೇ ಇರುತ್ತಿರಲಿಲ್ಲ. ಸದಾಕಾಲದಲ್ಲಿಯೂ ವ್ಯಕ್ತಿಯನ್ನು ಒಂದು ನಿತ್ಯನಿಯಮ ಬಿಗಿದಿದ್ದರೆ ಅವನಿಗೂ ಒಂದು ಹುಲ್ಲಿನ ಎಸಳಿಗೂ ವ್ಯತ್ಯಾಸವೇನು? ಇಂತಹ ಸೂಕ್ಷ್ಮವಾದ ನಿತ್ಯನಿಯಮದ ಭಾವನೆಯನ್ನು ನಾವು ಒಪ್ಪಿಕೊಳ್ಳುವುದೇ ಇಲ್ಲ.

ನಾವು ಅರಸಬೇಕಾಗಿರುವುದೇ ಸ್ವಾತಂತ್ರ್ಯ ಎನ್ನುತ್ತೇವೆ. ದೇವರೇ ಆ ಸ್ವಾತಂತ್ರ್ಯ.\break ಎಲ್ಲದರಲ್ಲಿರುವಂತೆ ಇಲ್ಲಿಯೂ ಆನಂದವೇ ಇರುವುದು. ಆದರೆ ಮಾನವ ಅಲ್ಪ ವಸ್ತುಗಳಲ್ಲಿ ಅದನ್ನು ಅರಸಿದಾಗ ಅವನಿಗೆಲ್ಲೊ ಅಣು ಮಾತ್ರ ದೊರಕುವುದು. ಕದಿಯುವ\break ಕಳ್ಳನಿಗೆ ದೇವರನ್ನು ನೋಡಿದ ಸಾಧುವಿಗಾದಂತೆಯೇ ಸಂತೋಷವಾಗುವುದು. ಆದರೆ\break ಕಳ್ಳನಿಗೆ ಎಲ್ಲೊ ಒಂದು ಕಣದಷ್ಟು ಸಂತೋಷ ಮತ್ತು ಬೇಕಾದಷ್ಟು ದುಃಖ ದೊರಕುವುದು. ದೇವರೇ ನಿಜವಾದ ಆನಂದ. ಪ್ರೀತಿಯೇ ದೇವರು, ಸ್ವಾತಂತ್ರ್ಯವೇ ದೇವರು. ಬಂಧನದಲ್ಲಿರುವ ಯಾವುದೂ ದೇವರಲ್ಲ.

\eject

ಮಾನವನಿಗೆ ಆಗಲೇ ಸ್ವಾತಂತ್ರ್ಯವಿದೆ. ಆದರೆ ಅದನ್ನು ಅವನು ಕಂಡುಹಿಡಿಯ\break ಬೇಕಾಗಿದೆ. ಅವನಿಗೆ ಅದು ಇದೆ, ಆದರೆ ಅವನು ಪ್ರತಿಕ್ಷಣವೂ ಮರೆಯುತ್ತಿರುವನು. ಇದನ್ನು ಅರಿತೋ, ಅರಿಯದೆಯೋ, ಕಂಡುಹಿಡಿಯುವುದೇ ಪ್ರತಿಯೊಬ್ಬನ ಬಾಳಿನ ಗುರಿಯಾಗಿದೆ. ಮೂಢನಿಗೂ ಜ್ಞಾನಿಗೂ ಇರುವ ವ್ಯತ್ಯಾಸವೇ, ಮೂಢ ಅರಿಯದೇ ಪ್ರಯತ್ನಿಸುವನು, ಜ್ಞಾನಿ ಅರಿತು ಪ್ರಯತ್ನಿಸುವನು. ಪ್ರತಿಯೊಬ್ಬರೂ, ಒಂದು ಕಣದಿಂದ ನಕ್ಷತ್ರದವರೆಗೆ, ಎಲ್ಲರೂ ಸ್ವಾತಂತ್ರ್ಯಕ್ಕಾಗಿ ಹೋರಾಡುತ್ತಿರುವರು. ಮೂಢನಿಗೆ ಒಂದು ಸಣ್ಣ ಆವರಣದೊಳಗೆ ಸ್ವಾತಂತ್ರ್ಯ ಸಿಕ್ಕಿದರೆ ಸಾಕು, ಅಂದರೆ ಅವನಿಗೆ ಹಸಿವು ಬಾಯಾರಿಕೆಗಳಿಂದ ಮುಕ್ತಿ ದೊರೆತರೆ ಸಾಕು, ಅವನು ತೃಪ್ತನಾಗುವನು. ಆದರೆ ಇದಕ್ಕಿಂತ ದೊಡ್ಡದೊಂದು ಬಂಧನದಿಂದ ಪಾರಾಗಬೇಕಾಗಿದೆ ಎಂದು ಜ್ಞಾನಿ ತಿಳಿಯುವನು. ಅವನು ರೆಡ್​ ಇಂಡಿಯನ್ನರ ಸ್ವಾತಂತ್ರ್ಯವನ್ನು ಸ್ವಾತಂತ್ರ್ಯವೆಂದು ಬಗೆಯುವುದೇ ಇಲ್ಲ.

ನಮ್ಮ ದಾರ್ಶನಿಕರ ಪ್ರಕಾರ ಮುಕ್ತಿಯೇ ನಮ್ಮ ಗುರಿ, ಜ್ಞಾನವಲ್ಲ. ಏಕೆಂದರೆ, ಜ್ಞಾನ ಒಂದು ಮಿಶ್ರಣ. ಇದರಲ್ಲಿ ಶಕ್ತಿ ಮತ್ತು ಸ್ವಾತಂತ್ರ್ಯ ಎರಡೂ ಮಿಶ್ರವಾಗಿವೆ. ಆದರೆ ಅಪೇಕ್ಷಿಸಬೇಕಾದದ್ದು ಸ್ವಾತಂತ್ರ್ಯವನ್ನು ಮಾತ್ರ. ಮಾನವರು ಹೋರಾಡುವುದೇ ಇದಕ್ಕೆ. ಸುಮ್ಮನೆ ಶಕ್ತಿಯನ್ನು ಪಡೆಯುವುದು ಜ್ಞಾನವಾಗುವುದಿಲ್ಲ. ಉದಾಹರಣೆಗೆ ವಿಜ್ಞಾನಿಯು ವಿದ್ಯುತ್​ಶಕ್ತಿಯನ್ನು ಬೇಕಾದರೆ ಕೆಲವು ಮೈಲಿಗಳು ದೂರಕ್ಕೆ ಕಳುಹಿಸಬಲ್ಲ. ಆದರೆ ಪ್ರಕೃತಿ ಅದನ್ನು ಎಷ್ಟು ದೂರದವರೆಗೆ ಬೇಕಾದರೂ ಕಳುಹಿಸಬಲ್ಲದು. ಹಾಗಾದರೆ ಪ್ರಕೃತಿಗೆ ಏತಕ್ಕೆ ನಾವು ಸ್ಮಾರಕಗಳನ್ನು ಕಟ್ಟಬಾರದು? ನಮಗೆ ನಿಯಮವಲ್ಲ ಬೇಕಾಗಿರುವುದು. ನಿಯಮವನ್ನು ಮುರಿಯುವ ಶಕ್ತಿ. ಸ್ವಚ್ಛಂದರಾಗಲು ಬಯಸುವೆವು. ನೀವು ನಿಯಮಕ್ಕೆ ಅಡಿಯಾಳಾದರೆ ಒಂದು ಹಿಡಿ ಮಣ್ಣಿನ ಮುದ್ದೆಯಾಗುವಿರಿ. ನಾವು ಈಗ ನಿಯಮಗಳ ಹೊರಗೆ ಇರುವೆವೆ ಇಲ್ಲವೆ ಎಂಬುದಲ್ಲ ಪ್ರಶ್ನೆ; ನಾವು ನಿಯಮಾತೀತರು ಎಂಬ ಭಾವನೆಯ ಮೇಲೆ ಇಡೀ ಮಾನವ ಇತಿಹಾಸ ನಿಂತಿದೆ. ಒಬ್ಬನು ಕಾಡಿನಲ್ಲಿರುವನು ಎಂದು ಭಾವಿಸೋಣ. ಅವನಿಗೆ ವಿದ್ಯೆಗೆ ಬುದ್ಧಿಗೆ ಯಾವ ಅವಕಾಶವೂ ಇರುವುದಿಲ್ಲ. ಅವನು ಒಂದು ಬೀಳುವ ಕಲ್ಲನ್ನು ನೋಡುವನು. ಇದು ಪ್ರಾಕೃತಿಕ ಘಟನೆ. ಇದು ಸ್ವಾತಂತ್ರ್ಯ ಎಂದು ಅವನು ಊಹಿಸುವನು. ಅದಕ್ಕೆ ಒಂದು ಆತ್ಮವಿದೆ ಎಂದು ಅವನು ಭಾವಿಸುವನು. ಇದರಲ್ಲಿರುವ ಮುಖ್ಯ ಭಾವನೆಯೆ ಸ್ವಾತಂತ್ರ್ಯ. ಆದರೆ ಅದು ಬೀಳಲೇಬೇಕಾಗಿದೆ ಎನ್ನುವುದನ್ನು ಅರಿತಾಗ ಅದು ಪ್ರಕೃತಿ, ಜಡ, ಯಾಂತ್ರಿಕ ಚಲನೆ ಎನ್ನುವರು. ಆದರೆ ನಾನು ಈಗ ಬೀದಿಗೆ ಬೇಕಾದರೆ ಹೋಗಬಹುದು, ಇಲ್ಲದೇ ಇದ್ದರೆ ಇಲ್ಲ. ಇಲ್ಲೇ ಮಾನವನಾಗಿ ನನ್ನ ಮಹಿಮೆ ಇರುವುದು. ನಾನು ಅಲ್ಲಿಗೆ ಹೋಗಲೇಬೇಕಾಗಿದೆ ಬೇರೆ ವಿಧಿಯೇ ಇಲ್ಲ ಎಂದು ಅರಿತಾಗ ನಾನೊಂದು ಯಂತ್ರವಾಗುವೆನು. ಬೇಕಾದಷ್ಟು ಶಕ್ತಿಯನ್ನು ಪಡೆದಿರುವ ಪ್ರಕೃತಿ ಕೇವಲ ಒಂದು ಯಂತ್ರ ಮಾತ್ರ. ಸ್ವಾತಂತ್ರ್ಯವೇ ಚೇತನದ ಚಿಹ್ನೆ.

ವೇದಾಂತವು ಕಾಡಿನಲ್ಲಿರುವ ಮನುಷ್ಯನ ಭಾವನೆ ಸರಿ ಎನ್ನುವುದು. ಅವನು ನೋಡುವ ದೃಶ್ಯವೇನೋ ಸರಿ, ಕೊಟ್ಟ ವಿವರಣೆ ತಪ್ಪು, ಅಷ್ಟೆ. ಅವನು ಪ್ರಕೃತಿಯು ಸ್ವತಂತ್ರ ಎಂದು ಭಾವಿಸುವನು, ನಿಯಮಬದ್ಧವಾಗಿದೆ ಎಂದು ಭಾವಿಸುವುದಿಲ್ಲ. ನಾವು ಬೇಕಾದಷ್ಟು ಅನುಭವವನ್ನು ಪಡೆದಾದ ಮೇಲೆ ಪುನಃ ಈ ಭಾವನೆಗೇ ಬರುತ್ತೇವೆ. ಆಗ ಹೆಚ್ಚು ತಾತ್ತ್ವಿಕರಾಗಿ ಬರುತ್ತೇವೆ. ನಾನು ಈಗ ಬೀದಿಗೆ ಹೋಗಬೇಕಾಗಿದೆ ಎಂದು ಇಟ್ಟುಕೊಳ್ಳೋಣ. ಇಚ್ಛೆ ಹೀಗೆ ಮಾಡು ಎನ್ನುವುದು. ಆಗ ನಾನು ಮುನ್ನಡೆಯುತ್ತೇನೆ. ಆ ಇಚ್ಛೆಯಿಂದ ಹಿಡಿದು ಬೀದಿಗೆ ಹೋಗುವವರೆಗೆ ನಾನು ಯಾವ ವ್ಯತ್ಯಾಸವೂ ಇಲ್ಲದೆ ಅದು ಹೇಳಿದಂತೆ ಕೇಳುವೆನು. ಒಂದೇ ಸಮನಾಗಿ ನಡೆಯುವ ಘಟನೆಗಳನ್ನು ನಾವು ನಿಯಮ ಎನ್ನುವೆವು; ಆದರೆ ನಾನು ಒಂದೇ ಸಮನಾಗಿ ಮಾಡುವ ಕ್ರಿಯೆಗಳು ಖಂಡಖಂಡವಾಗಿವೆ, ಆದಕಾರಣವೇ ನನ್ನ ಕ್ರಿಯೆ ನಿಯಮಬದ್ಧವಾಗಿದೆ ಎಂದು ನಾನು ಎನ್ನುವುದಿಲ್ಲ. ನಾನು ಸ್ವತಂತ್ರವಾಗಿ ಕೆಲಸ ಮಾಡುತ್ತೇನೆ. ನಾನು ಐದು ನಿಮಿಷ ನಡೆಯುತ್ತೇನೆ. ಹಾಗೆ ನಡೆಯುವುದಕ್ಕಿಂತ ಮುಂಚೆ ಹಾಗೆ ಮಾಡುವಂತೆ ಇಚ್ಛೆ ಆಜ್ಞೆ ಮಾಡಿತು. ಆದಕಾರಣವೆ ವ್ಯಕ್ತಿಯು ತಾನು ಸ್ವತಂತ್ರ ಎಂದು ಭಾವಿಸುವನು. ಏಕೆಂದರೆ ಅವನ ಎಲ್ಲ ಕ್ರಿಯೆಗಳನ್ನು ಸಣ್ಣ ಸಣ್ಣ ಅವಧಿಗಳಾಗಿ ವಿಭಜಿಸಬಹುದು. ಆ ಸಣ್ಣ ಸಣ್ಣ ಅವಧಿಗಳಲ್ಲಿ ಏಕರೂಪ ಇದ್ದರೂ ಆ ಅವಧಿಯ ಆಚೆಗೆ ಅದೇ ರೀತಿಯ ಏಕರೂಪತೆ ಇಲ್ಲ. ಏಕರೂಪತೆ ಇಲ್ಲದಿರುವುದೇ ಸ್ವಾತಂತ್ರ್ಯದ ಭಾವನೆ. ಪ್ರಕೃತಿಯಲ್ಲಿ ನಾವು ಒಂದೇ ಸಮನಾಗಿ ನಡೆಯುವ ಘಟನೆಗಳು ದೀರ್ಘಕಾಲ ಏಕರೂಪತೆಯಿಂದ ನಡೆಯುವುದನ್ನು ನೋಡುತ್ತೇವೆ. ಆದರೆ ಅವುಗಳ ಆದಿ ಅಂತ್ಯಗಳು ಸ್ವತಂತ್ರವಾದ ಪ್ರೇರಣೆಗಳಾಗಿರಬೇಕು. ಸ್ವಾತಂತ್ರ್ಯದ ಪ್ರೇರಣೆ ಪ್ರಾರಂಭದಲ್ಲಿ ಕೊಡಲ್ಪಟ್ಟು ಅದು ಮುಂದೆ ಸಾಗುತ್ತಾ ಹೋಗುತ್ತದೆ. ಅದನ್ನು ನಮ್ಮ ಅವಧಿಗಳೊಂದಿಗೆ ಹೋಲಿಸಿದರೆ ಅದು ಬಹಳ ದೀರ್ಘವಾಗುವುದು. ದಾರ್ಶನಿಕವಾಗಿ ನಮ್ಮನ್ನು ನಾವು ವಿಶ್ಲೇಷಣೆ ಮಾಡಿಕೊಂಡರೆ ನಾವು ಸ್ವತಂತ್ರರಲ್ಲ ಎಂದು ಗೊತ್ತಾಗುವುದು. ಆದರೆ ನಾನು ಸ್ವತಂತ್ರ ಎಂಬ ಭಾವನೆಯೂ ಇರುವುದು. ಈ ಭಾವನೆ ಹೇಗೆ ಬಂದಿತು ಎಂಬುದನ್ನು ನಾವು ವಿವರಿಸಬೇಕಾಗಿದೆ. ನಮ್ಮಲ್ಲಿ ಈ ಎರಡು ಭಾವನೆಗಳು (ಮುಕ್ತ ಮತ್ತು ಬದ್ಧ) ಇವೆ. ನಮ್ಮ ಯುಕ್ತಿಯು ನಾವು ಮಾಡಿದ ಪ್ರತಿಯೊಂದು ಕೆಲಸವೂ ಕಾರ್ಯಕಾರಣ ನಿಯಮಕ್ಕೆ ಬದ್ಧವಾಗಿದೆ ಎನ್ನುವುದು. ಆದರೂ ಪ್ರತಿಯೊಂದು ಕೆಲಸವನ್ನು ಮಾಡುವಾಗಲೂ ನಾವು ಸ್ವತಂತ್ರರು ಎಂಬುದನ್ನು ಸ್ಥಾಪಿಸುತ್ತಿರುವೆವು. ವೇದಾಂತವು ಇದನ್ನು ಹೇಗೆ ಪರಿಹರಿಸುವುದು? ಸ್ವಾತಂತ್ರ್ಯ ಒಳಗೆ ಇರುವುದು. ಆತ್ಮ ಸ್ವತಂತ್ರವಾದುದು. ಆದರೆ ಆತ್ಮವು ದೇಹ ಮನಸ್ಸುಗಳ ಮೂಲಕ ಕೆಲಸ ಮಾಡುತ್ತಿದೆ. ಆ ದೇಹ ಮತ್ತು ಮನಸ್ಸು ಸ್ವತಂತ್ರವಲ್ಲ.

ನಾವು ಪ್ರತಿಕ್ರಿಯೆಗೆ ಅವಕಾಶ ಕೊಟ್ಟೊಡನೆಯೆ ದಾಸರಾಗುವೆವು. ಯಾರೊ ನನ್ನನ್ನು ನಿಂದಿಸುವರು; ತತ್​ಕ್ಷಣ ನನಗೆ ಕೋಪ ಬರುವುದು. ಅವನು ಆಡಿದ ಒಂದು ಮಾತು ನನ್ನನ್ನು ದಾಸನನ್ನಾಗಿ ಮಾಡಿತು. ಆದಕಾರಣ ನಾವು ಸ್ವಾತಂತ್ರ್ಯವನ್ನು ಪ್ರದರ್ಶಿಸಬೇಕಾಗಿದೆ. ಪರಮಜ್ಞಾನಿಯಲ್ಲಿ ಅಥವಾ ನೀಚ ಮೃಗದಲ್ಲಿ ಅಥವಾ ಅಧಮಾಧಮನಾದ\break ಮಾನವನಲ್ಲಿ, ಯಾರು ಮನುಷ್ಯನನ್ನಾಗಲಿ ಪ್ರಾಣಿಯನ್ನಾಗಲಿ ಜ್ಞಾನಿಯನ್ನಾಗಲಿ ನೋಡದೆ ಎಲ್ಲದರಲ್ಲಿಯೂ ದೇವರನ್ನು ನೋಡಬಲ್ಲರೊ ಅವರೇ ಮಹಾಋಷಿಗಳು. ಅವರು ಈ ಜೀವನದಲ್ಲಿಯೇ ಆಗಲೇ ದ್ವಂದ್ವದಿಂದ ಮುಕ್ತರಾಗಿ ಸಮತ್ವದಲ್ಲಿ ನೆಲಸಿರುವರು. ದೇವರು ಪರಿಶುದ್ಧನು, ಎಲ್ಲರಿಗೂ ಒಂದೇ; ಇದನ್ನು ಅರಿತ ಜ್ಞಾನಿಯೇ ಜೀವಂತ ದೇವರು. ಆ ಗುರಿಯೆಡೆಗೆ ನಾವೆಲ್ಲ ಹೋಗುತ್ತಿರುವೆವು. ಪ್ರತಿಯೊಂದು ಪೂಜಾವಿಧಾನವೂ, ಮಾನವನು ಮಾಡುವ ಪ್ರತಿಯೊಂದು ಕ್ರಿಯೆಯೂ ಅಂತಹ ಸ್ಥಿತಿಗೆ ಒಯ್ಯುವ ಮಾರ್ಗ. ದ್ರವ್ಯವನ್ನು ಬಯಸುವವನು ಸ್ವಾತಂತ್ರ್ಯಕ್ಕಾಗಿ ಹೋರಾಡುತ್ತಿರುವನು. ಅವನು ದಾರಿದ್ರ್ಯದ ಬಂಧನದಿಂದ ಪಾರಾಗಲು ಯತ್ನಿಸುತ್ತಿರುವನು. ಮಾನವನು ಮಾಡುವ ಪ್ರತಿಯೊಂದು ಕರ್ಮವೂ ಒಂದು ಪೂಜೆ, ಏಕೆಂದರೆ ಅವನು ಸ್ವತಂತ್ರನಾಗಲು ಯತ್ನಿಸುತ್ತಿರುವನು. ಪ್ರತಿಯೊಂದು ಕರ್ಮವೂ ಪ್ರತ್ಯಕ್ಷವಾಗಿ ಇಲ್ಲವೆ ಪರೋಕ್ಷವಾಗಿ ಸ್ವಾತಂತ್ರ್ಯದ ಕಡೆಗೆ ಒಯ್ಯುವುದು. ಯಾವುದು ನಮ್ಮನ್ನು ತಡೆಯುವುದೊ ಅದನ್ನು ನಾವು ಮಾಡಬಾರದು. ಇಡೀ ಪ್ರಪಂಚ ತಿಳಿದೋ ತಿಳಿಯದೆಯೋ ಆರಾಧಿಸುತ್ತಿದೆ. ಇದು ಪ್ರಪಂಚಕ್ಕೆ ಗೊತ್ತಿಲ್ಲ ಅಷ್ಟೆ. ಶಪಿಸುತ್ತಿರುವಾಗಲೂ ಅದೊಂದು ಬಗೆಯ ಪೂಜೆಯೇ. ಏಕೆಂದರೆ ಶಪಿಸುವವರು ಕೂಡ ಮುಕ್ತರಾಗಲು ಯತ್ನಿಸುತ್ತಿರುವರು. ತಾವು ಒಂದು ವಸ್ತುವಿನಿಂದ ಪಾರಾಗಬೇಕೆಂದು ಯತ್ನಿಸುತ್ತಿರುವಾಗ ಅವರು ಅದಕ್ಕೆ ದಾಸರಾಗುವರು ಎಂಬುದನ್ನು ಆಲೋಚಿಸುವುದಿಲ್ಲ. ಮುಳ್ಳಿನ ಬೇಲಿಯನ್ನು ಒದ್ದಂತೆ ಇದು.

ನಾವು ಬದ್ಧರು ಎಂಬ ಬಾವನೆಯಿಂದ ಪಾರಾದರೆ ಈಗಲೇ ನಾವು ಏನನ್ನು ಬೇಕಾದರೂ ಸಾಧಿಸಬಹುದು. ಕಾಲ ಪಕ್ವವಾಗಬೇಕಷ್ಟೇ. ಅದು ಹಾಗೆಂದಾದರೆ ಇನ್ನಷ್ಟು ಶಕ್ತಿಯನ್ನು ಸೇರಿಸಿದರೆ ಕಾಲವನ್ನು ಕಡಮೆ ಮಾಡಬಹುದು. ಅಮೃತಶಿಲೆಯನ್ನು ಮಾಡುವ ರಹಸ್ಯವನ್ನು ಕಂಡುಹಿಡಿದ ಪ್ರೊಫೆಸರನನ್ನು ಜ್ಞಾಪಿಸಿಕೊಳ್ಳಿ. ಪ್ರಕೃತಿ ಅಮೃತಶಿಲೆಯನ್ನು ಮಾಡಲು ಶತಶತಮಾನಗಳು ಪ್ರಯತ್ನಪಟ್ಟರೆ ಆ ಪ್ರೊಫೆಸರ್​ ಹನ್ನೆರಡೇ ವರುಷಗಳಲ್ಲಿ ಅದನ್ನು ಮಾಡಿದನು.

