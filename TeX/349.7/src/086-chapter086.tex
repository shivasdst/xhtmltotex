
\chapter[ಜನನ ಮರಣಗಳ ನಿಯಮಗಳು ]{ಜನನ ಮರಣಗಳ ನಿಯಮಗಳು \protect\footnote{\engfoot{C.W. Vol. VIII p. 235}}}

\centerline{(“ಓಕ್ಲಾಂಡ್​ ಟ್ರೈಬ್ಯುನ್​” ನಲ್ಲಿ ಪ್ರಕಟವಾದ, ೧೯೦೦, ಮಾರ್ಚ್​ ೭ರಂದು ಓಕ್ಲಾಂಡ್​ನಲ್ಲಿ ನೀಡಿದ ಉಪನ್ಯಾಸದ ವರದಿ.)}

ಜನನಮರಣಗಳಿಂದ ಪಾರಾಗುವುದು ಹೇಗೆ–ಸ್ವರ್ಗಕ್ಕೆ ಹೋಗುವುದಲ್ಲ, ಸ್ವರ್ಗಕ್ಕೆ ಹೇಗೆ ಹೋಗದೆ ಇರುವುದು, –ಇದನ್ನೇ ಪ್ರತಿಯೊಬ್ಬ ಹಿಂದೂವು ತಿಳಿದುಕೊಳ್ಳಲು ಇಚ್ಛಿಸುವುದು.

ಯಾವುದೂ ಪ್ರತ್ಯೇಕವಾಗಿ ಇರಲಾರದು. ಪ್ರತಿಯೊಂದೂ ಅಂತ್ಯವಿಲ್ಲದ ಕಾರ್ಯಕಾರಣಗಳ ಸರಪಳಿಯಲ್ಲಿ ಒಂದು ಕೊಂಡಿ. ಮಾನವನಿಗಿಂತ ಉನ್ನತ ವಾದ ಜೀವಿಗಳಿದ್ದರೆ ಅವರೂ ಈ ನಿಯಮಕ್ಕೆ ಅಧೀನರೇ. ಆಲೋಚನೆಯು ಆಲೋಚನೆಯಿಂದ ಏಳುವುದು, ಜೀವವು ಜೀವದಿಂದ ಮಾತ್ರ ಉದ್ಭವಿಸುತ್ತದೆ. ದ್ರವ್ಯವು ದ್ರವ್ಯದಿಂದ ಏಳುವುದು. ಸೃಷ್ಟಿಯನ್ನು ಹೊಸದಾಗಿ ದ್ರವ್ಯದಿಂದ ತಯಾರು ಮಾಡುವುದಕ್ಕೆ ಆಗುವುದಿಲ್ಲ. ಅದು ಎಂದೆಂದಿಗೂ ಇತ್ತು. ಹೊಸದಾಗಿ ಪ್ರಕೃತಿಯ ಕೈಗಳಿಂದ ಮನುಷ್ಯರು ಜಗತ್ತಿಗೆ ಬಂದಿದ್ದರೆ ಅವರಲ್ಲಿ ಯಾವ ಸಂಸ್ಕಾರ ಗಳೂ ಇರಬಾರದಾಗಿತ್ತು. ಆದರೆ ನಾವು ಹಾಗೆ ಬಂದಿಲ್ಲವಾದುದರಿಂದ ನಾವು ಹೊಸದಾಗಿ ಸೃಷ್ಟಿಯಾದವರಲ್ಲ. ಜೀವಗಳು ಶೂನ್ಯದಿಂದ ಬಂದಿದ್ದರೆ ಪುನಃ ಏತಕ್ಕೆ ಅವರು ಶೂನ್ಯಕ್ಕೆ ಹಿಂತಿರುಗಿ ಹೋಗಬಾರದು? ನಾವು ಭವಿಷ್ಯದಲ್ಲಿ ಎಂದೆಂದಿಗೂ ಬಾಳಬೇಕಾಗಿದ್ದರೆ ಭೂತಕಾಲದಲ್ಲಿಯೂ ಎಂದೆಂದಿಗೂ ಇದ್ದಿರಬೇಕು.

ಆತ್ಮವು ದೇಹವೂ ಅಲ್ಲ, ಮನಸ್ಸೂ ಅಲ್ಲ ಎಂಬುದು ಹಿಂದೂಗಳ ನಂಬಿಕೆ. ಸ್ಥಿರವಾಗಿ ಯಾವುದು ನಿಲ್ಲುವುದು? ನಾನು ಎಂಬುದು ಯಾವುದು? ಅದು ದೇಹವಲ್ಲ. ದೇಹ ಯಾವಾಗಲೂ ಬದಲಾಗುತ್ತಿರುವುದು. ಅದು ಮನಸ್ಸೂ ಅಲ್ಲ, ಮನಸ್ಸು ದೇಹಕ್ಕಿಂತ ವೇಗವಾಗಿ ಚಲಿಸುವುದು. ಕೆಲವು ಕ್ಷಣವಾದರೂ ಒಂದೇ ಆಲೋಚನೆ ಇರುವುದಿಲ್ಲ. ಬದಲಾಗದೆ ಇರುವಂತಹದು ಯಾವುದೋ ಇರಬೇಕು. ಚಲಿಸುತ್ತಿರುವ ನದಿಗೆ ಚಲಿಸದ ದಡವಿರುವಂತೆ ಯಾವುದೋ ಒಂದು ಇರಬೇಕು. ದೇಹದ ಹಿಂದೆ, ಮನಸ್ಸಿನ ಹಿಂದೆ ಯಾವುದೋ ಇರಬೇಕು. ಅದೇ ಆತ್ಮ. ಅದೇ ಮಾನವನನ್ನು ಒಟ್ಟು ಗೂಡಿಸುವುದು. ಆತ್ಮನೆಂಬ ಯಜಮಾನ ಮನಸ್ಸೆಂಬ ಸೂಕ್ಷ್ಮವಾದ ಉಪಕರಣದ ಮೂಲಕ ಕೆಲಸ ಮಾಡುವನು. ಇಂಡಿಯಾ ದೇಶದಲ್ಲಿ ಒಬ್ಬ ವ್ಯಕ್ತಿಯು ಸತ್ತರೆ ಆ ದೇಹವನ್ನು ಬಿಟ್ಟನು ಎನ್ನುತ್ತಾರೆ. ನೀವು ಅವನು ಜೀವವನ್ನು ಬಿಟ್ಟನು ಎನ್ನುತ್ತೀರಿ. ಹಿಂದೂಗಳು ಮಾನವನು ಆತ್ಮ, ಅವನಿಗೊಂದು ದೇಹವಿದೆ ಎನ್ನುತ್ತಾರೆ. ಪಾಶ್ಚಾತ್ಯರಾದರೋ ಮಾನವನು ದೇಹ, ಅವನಿಗೊಂದು ಜೀವ ಇದೆ ಎನ್ನುತ್ತಾರೆ.

ಜಗತ್ತಿನಲ್ಲಿ ಮೃತ್ಯು ಎಲ್ಲಾ ಮಿಶ್ರವಸ್ತುಗಳನ್ನೂ ನುಂಗಿಹಾಕುತ್ತದೆ. ಆತ್ಮವು ಯಾವುದರ ಮಿಶ್ರಣವೂ ಆಗದೆ ಇರುವುದರಿಂದ ಅದು ನಾಶವಾಗಲಾರದು.ಆತ್ಮವು ಸ್ವಭಾವತಃ ಅಮೃತವಾಗಿರಬೇಕು. ದೇಹ, ಮನಸ್ಸು, ಜೀವಗಳು ಕರ್ಮ ಚಕ್ರದಲ್ಲಿ ಸುತ್ತುತ್ತಿರುವುವು. ಯಾವುದೂ ತಪ್ಪಿಸಿಕೊಳ್ಳಲಾರವು. ಹೇಗೆ ನಕ್ಷತ್ರ ಗಳು, ಸೂರ್ಯ ಚಂದ್ರರು ನಿಯಮಕ್ಕೆ ಅತೀತರಾಗಿ ಹೋಗಲಾರರೋ ಹಾಗೆಯೇ ನಾವೂ ನಿಯಮಾತೀತರಾಗಲಾರೆವು. ಕರ್ಮಸಿದ್ಧಾಂತದ ಪ್ರಕಾರ ಪ್ರತಿಯೊಂದು ಕಾರಣಕ್ಕೆ ಪರಿಣಾಮವು ಈಗಲೋ ನಾಳೆಯೋ ಪ್ರಾಪ್ತವಾಗಬೇಕು. ಐದು ಸಾವಿರ ವರುಷಗಳ ಹಿಂದೆ ಈಜಿಪ್ಟಿನ ಪಿರಮಿಡ್​ನೊಳಗಿದ್ದ ಮಮ್ಮಿಯ ಕೈಯಿಂದ ತೆಗೆದ ಬೀಜವನ್ನು ಬಿತ್ತಿದಾಗ ಮೊಳೆತು ಸಸಿಯಾಯಿತು. ಇದರಂತೆಯೇ ಮಾನವನು ಎಂದೋ ಮಾಡಿದ ಕರ್ಮಕ್ಕೆ ಒಂದಲ್ಲ ಒಂದು ದಿನ ಫಲವನ್ನು ಉಣ್ಣಬೇಕಾಗಿದೆ. ಕರ್ಮವು ಮತ್ತೊಂದು ಕರ್ಮವನ್ನು ಉಂಟುಮಾಡಿದರೆ ಎಂದಿಗೂ ನಾಶವಾಗುವುದಿಲ್ಲ. ನಮ್ಮ ಕರ್ಮಗಳು ಅವುಗಳಿಗೆ ಅನುಗುಣವಾದ ಫಲಗಳನ್ನು ಈ ಜೀವನದಲ್ಲಿ ನೀಡುವುದು ಸತ್ಯವಾದರೆ ನಾವೆಲ್ಲ ಕಾರ್ಯಕಾರಣ ಗಳ ಚಕ್ರದಲ್ಲಿ ಸುತ್ತಲೇಬೇಕಾಗುತ್ತದೆ. ಇದೇ ಪುನರ್ಜನ್ಮ ಸಿದ್ಧಾಂತ. ನಾವು ನಿಯಮಕ್ಕೆ ದಾಸರು, ಕ್ರಿಯೆಗಳಿಗೆ ದಾಸರು, ಆಸೆಗೆ ದಾಸರು, ನೂರಾರು ವಸ್ತುಗಳಿಗೆ ದಾಸರು. ನಾವು ಪುನಃ ಹುಟ್ಟುವುದರಿಂದ ಪಾರಾದರೆ ಮಾತ್ರ ಗುಲಾಮಗಿರಿಯಿಂದ ಪಾರಾಗಬಲ್ಲೆವು. ದೇವರೊಬ್ಬನೇ ನಿತ್ಯ ಮುಕ್ತ. ದೇವರು ಮತ್ತು ಮುಕ್ತಿ ಎರಡೂ ಒಂದೇ.

