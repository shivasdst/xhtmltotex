
\chapter[ಮಾಯೆಗೆ ಕಾರಣವೇನು? ]{ಮಾಯೆಗೆ ಕಾರಣವೇನು? \protect\footnote{\engfoot{C.W. Vol. V, P 276}}}

ಮಾಯೆಗೆ ಕಾರಣವೇನು ಎಂಬ ಪ್ರಶ್ನೆಯನ್ನು ಕಳೆದ ಮೂರು ಸಾವಿರ ವರ್ಷಗಳಿಂದಲೂ ಕೇಳುತ್ತಿರುವರು. ಅದಕ್ಕೆ ಏಕಮಾತ್ರ ಉತ್ತರವೆಂದರೆ ನಿಮ್ಮ ಪ್ರಶ್ನೆಯನ್ನು ತರ್ಕಬದ್ಧವಾಗಿ ಕೇಳಿದರೆ ಉತ್ತರ ಹೇಳುವೆವು ಎನ್ನುವುದು. ಒಂದು ಪ್ರಶ್ನೆಯೇ ವಿರೋಧಾಭಾಸ. ಅಖಂಡವು ಕೇವಲ ತೋರಿಕೆಗೆ ಮಾತ್ರ ಭಿನ್ನವಾಗಿದೆ ಎಂಬುದು ನಮ್ಮ ನಿಲುವು. ಕೇವಲ ಮಾಯೆಯಲ್ಲಿ ಮಾತ್ರ ನಿರಪೇಕ್ಷವು ಸಾಪೇಕ್ಷ ವಾದಂತೆ ಇದೆ. ನಾವು ನಿರಪೇಕ್ಷವನ್ನು ಒಪ್ಪಿಕೊಳ್ಳುವುದರಿಂದಲೇ ಮತ್ತಾವುದೂ ಇದರ ಮೇಲೆ ತನ್ನ ಪರಿಣಾಮವನ್ನು ಬೀರಲಾರದು ಎಂಬುದನ್ನು ಒಪ್ಪಿ ಕೊಂಡಂತೆ ಆಗುವುದು. ಅದಕ್ಕೆ ಯಾವ ಕಾರಣವೂ ಇಲ್ಲ ಎಂದರೆ ಅದು ನಿರುಪಾಧಿಕವಾಗಿದ್ದರೆ ಯಾವುದೂ ಅದರ ಮೇಲೆ ತನ್ನ ಪರಿಣಾಮವನ್ನು ಬೀರಲಾರದು. ನಿರುಪಾಧಿಕವಾಗಿರುವುದರಲ್ಲಿ ದೇಶ ಕಾಲ ನಿಮಿತ್ತಗಳಿರುವುದಿಲ್ಲ, ಇದನ್ನು ಒಪ್ಪಿಕೊಂಡರೆ ನಿಮ್ಮ ಪ್ರಶ್ನೆ ಈ ರೂಪವನ್ನು ತಾಳುವುದು: “ಹೀಗೆ ಎಂದಿಗೂ ಆಗದೇ ಇರುವುದನ್ನು ಆಗಿರುವಂತೆ ಮಾಡುವುದು ಯಾವುದು?” ಸೋಪಾಧಿಕ ಜಗತ್ತಿನಲ್ಲಿ ಮಾತ್ರ ನಿಮ್ಮ ಪ್ರಶ್ನೆ ಸಾಧ್ಯ. ನೀವು ಇದನ್ನು ಸೋಪಾಧಿಕ ದಿಂದ ತೆಗೆದು, ನಿರುಪಾಧಿಕದಲ್ಲಿ ಇದು ಹೇಗೆ ಸಾಧ್ಯ ಎಂದು ಪ್ರಶ್ನಿಸುವಿರಿ. ನಿರುಪಾಧಿಕವಾದುದು ಯಾವಾಗ ಸೋಪಾಧಿಕವಾಗುವುದೋ ಆಗಲೆ ಕಾಲ ದೇಶ ನಿಮಿತ್ತಗಳು ಸಾಧ್ಯ. ಆಗ ಮಾತ್ರ ಈ ಪ್ರಶ್ನೆಯನ್ನು ಕೇಳಬಹುದು. ಕೇವಲ ಅಜ್ಞಾನವೇ ಈ ಮಾಯೆಗೆ ಕಾರಣ ಎಂದು ಹೇಳಬಹುದು. ಪ್ರಶ್ನೆ ಇಲ್ಲಿ ಸಾಧ್ಯವೆ. ಇಲ್ಲ. ಯಾವುದೂ ಕೇವಲವಾದುದನ್ನು ಬದಲಾಯಿಸಲಾರದು. ಕಾರಣವೇ ಇರಲಿಲ್ಲ ಎಂಬುದು ನಮಗೆ ಗೊತ್ತಿಲ್ಲ ಎಂದಲ್ಲ, ಇದು ಜ್ಞಾನಕ್ಕೆ ಅತೀತವಾಗಿರುವುದು, ಇದನ್ನು ಜ್ಞಾನದ ಮಿತಿಗೆ ತರುವುದಕ್ಕೇ ಆಗುವುದಿಲ್ಲ. ಅಜ್ಞಾನವನ್ನು ಎರಡು ಅರ್ಥಗಳಲ್ಲಿ ಉಪಯೋಗಿಸಬಹುದು. ಒಂದರಲ್ಲಿ ನಮಗೆ ಇನ್ನೂ ತಿಳುವಳಿಕೆ ಇಲ್ಲ ಎಂದು ಆಗುತ್ತದೆ. ಮತ್ತೊಂದರಲ್ಲಿ ಅದು ಅರಿವಿಗೆ ಮೀರಿದುದು ಎಂದಾಗುತ್ತದೆ. ಈಗ ನಮಗೆ ಎಕ್ಸ್​ರೇ ಗೊತ್ತಾಗಿದೆ. ಎಕ್ಸ್​ರೇಗೆ ಕಾರಣ ಇನ್ನೂ ಇತ್ಯರ್ಥವಾಗಿಲ್ಲ. ಆದರೆ ಒಂದು ದಿನ ಇದು ನಮಗೆ ಗೊತ್ತಾಗುವುದು. ಇಲ್ಲಿ ಎಕ್ಸ್​ರೇ ವಿಚಾರ ನಮಗಿನ್ನೂ ಗೊತ್ತಿಲ್ಲ ಎನ್ನಬಹುದು. ಆದರೆ ಕೇವಲವಾದುದು ನಮಗೆ ತಿಳಿಯು ವಂತೆಯೇ ಇಲ್ಲ. ಅರಿವಿನ ಮೇರೆಯೊಳಗಿದ್ದರೂ ಎಕ್ಸ್​ರೇ ವಿಷಯ ನಮಗಿನ್ನೂ ಗೊತ್ತಿಲ್ಲ, ಮುಂದೆ ಗೊತ್ತಾಗಬಹುದು. ಆದರೆ ಎರಡನೆಯದರಲ್ಲಿ ಅದು ಅರಿವಿನ ಮೇರೆಯನ್ನು ಮೀರಿರುವುದರಿಂದ ಅದು ಗೊತ್ತಾಗುವಂತೆಯೇ ಇಲ್ಲ. ಅರಿಯುವ ವನನ್ನು ಅರಿಯುವುದು ಹೇಗೆ? ನೀನು ಯಾವಾಗಲೂ ನೀನೆ ಆಗಿರುವೆ, ನಿನಗೆ ನೀನೇ ದೃಶ್ಯವಸ್ತುವಾಗಲಾರೆ. ನಮ್ಮ ದಾರ್ಶನಿಕರು ಅಮರತ್ವವನ್ನು ಪ್ರತಿಪಾದಿಸಲು ಉಪಯೋಗಿಸಿದ ವಾದಸರಣಿಗಳಲ್ಲಿ ಇದೂ ಒಂದು. ನಾನು ಸತ್ತಿರುವೆನು ಎಂದು ಆಲೋಚಿಸುವುದು ಹೇಗೆ? ನಾನು ನಿಂತು ನನ್ನದೇ ಸತ್ತ ದೇಹವನ್ನು ನೋಡಬೇಕಾಗುವುದು. ನಾನೇ ದೃಶ್ಯವಸ್ತುವಾಗಲಾರೆ.

