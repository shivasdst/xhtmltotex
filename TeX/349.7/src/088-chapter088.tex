
\chapter[ಭರತಖಂಡದಲ್ಲಿ ಕ್ರೈಸ್ತಧರ್ಮ ]{ಭರತಖಂಡದಲ್ಲಿ ಕ್ರೈಸ್ತಧರ್ಮ \protect\footnote{\engfoot{C.W. Vol. VIII, p. 214}}}

\begin{center}
\textbf{(1894ರ ಮಾರ್ಚ್​ 11ರಂದು ಡೆಟ್ರಾಯಿಟ್​ನಲ್ಲಿ ನೀಡಿದ ಉಪನ್ಯಾಸದ ಡೆಟ್ರಾಯಿಟ್​ ಫ್ರೀ ಪ್ರೆಸ್​ ವರದಿ)}
\end{center}

ಮಿಷನರಿಗಳು ಚೈನಾದಲ್ಲಿ ಮತ್ತು ಜಪಾನಿನಲ್ಲಿ ಏನು ಮಾಡುತ್ತಿರುವರು ಎಂಬುದು ನನಗೆ ಗೊತ್ತಿಲ್ಲ. ಆದರೆ ಅವರು ಇಂಡಿಯಾ ದೇಶದಲ್ಲಿ ಏನು ಮಾಡುತ್ತಿರುವರು ಎಂಬುದು ನನಗೆ ಚೆನ್ನಾಗಿ ಗೊತ್ತಿದೆ. ಇಲ್ಲಿಯ ಜನರು ಇಂಡಿಯಾ ದೇಶವೆಂದರೆ ಅದೊಂದು ಕಾಡುಜನರಿಂದ ಕೂಡಿದ ದೊಡ್ಡ ದಂಡಕಾರಣ್ಯವೆಂದು ಭಾವಿಸುವರು. ಅಲ್ಲಿ ಎಲ್ಲೋ ಅಲ್ಲೊಬ್ಬ ಇಲ್ಲೊಬ್ಬ ನಾಗರಿಕನಾದ ಬಿಳಿಯ ಮನುಷ್ಯ ಇರುವನು ಎಂದು ತಿಳಿಯುವರು! ಇಂಡಿಯಾ ದೇಶ ಅಮೆರಿಕಾದ ಅರ್ಧದಷ್ಟು ಇದೆ. ಅಲ್ಲಿ ಮೂವತ್ತು ಕೋಟಿ ಜನ ವಾಸಿಸುತ್ತಿರುವರು. ಭರತಖಂಡಕ್ಕೆ ಸಂಬಂಧಪಟ್ಟ ಎಷ್ಟೋ ಕಾಲ್ಪನಿಕ ಕಥೆಗಳನ್ನು ನಿಮ್ಮವರು ಹೇಳುತ್ತಿರುವರು. ಅವೆಲ್ಲಾ ಶುದ್ಧ ಸುಳ್ಳು ಎಂದು ಹೇಳಿ ಹೇಳಿ ನನಗೆ ಸಾಕಾಗಿದೆ. ಕ್ರೈಸ್ತರು ಬೇರೆ ದೇಶಕ್ಕೆ\break ಹೋದಾಗ ಅಲ್ಲಿನ ಆದಿವಾಸಿಗಳನ್ನು ನಾಶಮಾಡಿದಂತೆ ಭಾರತಕ್ಕೆ ಬಂದ ಆರ್ಯರು\break ಮಾಡಲಿಲ್ಲ. ಅದರ ಅನಾಗರಿಕ ಪದ್ಧತಿಗಳನ್ನು ಉತ್ತಮಪಡಿಸಲು ಯತ್ನಿಸಿದರು.\break ಸ್ಪೇನಿನವರು ಕ್ರೈಸ್ತಧರ್ಮದೊಂದಿಗೆ ಸಿಲೋನಿಗೆ ಬಂದರು. ತಮ್ಮ ಧರ್ಮಕ್ಕೆ ಸೇರದೆ ಇರುವವರನ್ನೆಲ್ಲಾ ಕೊಲ್ಲಬಹುದು, ಅವರ ಗುಡಿಗೋಪುರಗಳನ್ನೆಲ್ಲಾ ಧ್ವಂಸಮಾಡಬಹುದು\break ಎಂದು ದೇವರು ತಮಗೆ ಅಪ್ಪಣೆ ಮಾಡಿರುವನು ಎಂದು ಅವರು ಭಾವಿಸಿದರು.\break ಬೌದ್ಧರಲ್ಲಿ ಒಂದು ಅಡಿ ಉದ್ದದ ಬುದ್ಧನ ಹಲ್ಲೊಂದು ಇದ್ದಿತು. ಅದನ್ನು ಸ್ಪೇನಿನ ಕ್ರೈಸ್ತರು ಸಮುದ್ರಕ್ಕೆ ಎಸೆದು ಸಹಸ್ರಾರು ಜನರನ್ನು ಕೊಂದು ಕ್ರೈಸ್ತರನ್ನಾಗಿ ಮಾಡಿದರು. ಪೋರ್ಚುಗೀಸರು ಪಶ್ಚಿಮ ಭಾರತಕ್ಕೆ ಬಂದರು. ಹಿಂದೂಗಳು ತ್ರಿಮೂರ್ತಿಯನ್ನು ನಂಬುವರು. ಅವನಿಗಾಗಿ ಒಂದು ಗುಡಿ ಇತ್ತು. ಕ್ರೈಸ್ತರು ಈ ದೇವಸ್ಥಾನವನ್ನು ನೋಡಿ ಇದೆಲ್ಲ ಸೈತಾನನ ಕೃತಿ ಎಂದು ಅದನ್ನು ಧ್ವಂಸಮಾಡಲು ತಮ್ಮ ಫಿರಂಗಿಯನ್ನು ಅದರ ಕಡೆ ತಿರುಗಿಸಿ ದೇವಸ್ಥಾನದ ಒಂದು ಭಾಗವನ್ನು ನಾಶಮಾಡಿದರು. ಆದರೆ ಜನರು ರೋಷಾವಿಷ್ಟರಾಗಿ ಅವರನ್ನು ಓಡಿಸಿದರು. ಮೊದಲು ಬಂದ ಮಿಷನರಿಗಳಿಗೆ ಹೇಗಾದರೂ ನೆಲಸಲು ಸ್ವಲ್ಪ ಸ್ಥಳ ಬೇಕಾಗಿತ್ತು. ಬಲಾತ್ಕಾರದಿಂದ ನೆಲವನ್ನು ಪಡೆಯಲು ಹಲವರನ್ನು ಕೊಂದರು. ಕೆಲವರನ್ನು ಬಲಾತ್ಕಾರವಾಗಿ ಕ್ರೈಸ್ತಧರ್ಮಕ್ಕೆ ಸೇರಿಸಿದರು. ಹಲವರು ತಮ್ಮ ಪ್ರಾಣ ಉಳಿಸಿಕೊಳ್ಳುವುದಕ್ಕಾಗಿ ಕ್ರೈಸ್ತರಾದರು. ಪೋರ್ಚುಗೀಸರು ಕ್ರೈಸ್ತಧರ್ಮಕ್ಕೆ ಸೇರಿಸಿದ ಶೇಕಡಾ ತೊಂಭತ್ತೊಂಭತ್ತು ಜನರು ಬಲತ್ಕಾರದಿಂದ ಆ ಮತಕ್ಕೆ ಪರಿವರ್ತನೆಗೊಂಡವರು. ಅವರು, ‘ನಮಗೆ ಕ್ರೈಸ್ತಧರ್ಮದಲ್ಲಿ ನಂಬಿಕೆಯಿಲ್ಲ, ಆದರೂ ನಾವು ಕ್ರೈಸ್ತರು ಎಂದು ಹೇಳಿಕೊಳ್ಳಬೇಕಾಗಿದೆ’ ಎನ್ನುವರು. ಆದರೆ ಕ್ಯಾಥೊಲಿಕ್​ ಕ್ರೈಸ್ತಧರ್ಮ ಬಹುಬೇಗ ಮರುಕಳಿಸಿತು.

ಈಸ್ಟ್ ಇಂಡಿಯಾ ಕಂಪನಿಯು ಇಂಡಿಯಾದೇಶದ ಅನಾಯಕತೆಯನ್ನು ನೋಡಿ ಕೆಲವು ಭಾಗಗಳನ್ನು ಜಯಿಸಿತು. ಕಂಪನಿಯು ಪಾದ್ರಿಗಳನ್ನು ದೂರ ಇಟ್ಟಿತ್ತು. ಮಿಷನರಿಗಳನ್ನು ಮೊದಲು ಬರಮಾಡಿಕೊಂಡವರು ಹಿಂದೂಗಳು, ಇಂಗ್ಲಿಷರಲ್ಲ. ಅವರು ಆಗ ವ್ಯಾಪಾರದಲ್ಲಿ ನಿರತರಾಗಿದ್ದರು. ಮೊದಮೊದಲು ಬಂದ ಕ್ರೈಸ್ತ ಪಾದ್ರಿಗಳ ಮೇಲೆ ನನಗೆ ಬಹಳ ಗೌರವ ಇದೆ. ಅವರು ನಿಜವಾಗಿಯೂ ಕ್ರಿಸ್ತನ ಸೇವಕರಾಗಿದ್ದರು. ಅವರು ಜನರನ್ನು ಟೀಕಿಸಲಿಲ್ಲ, ಅವರ ವಿಷಯವಾಗಿ ಅಪಪ್ರಚಾರ ಮಾಡಲಿಲ್ಲ. ಅವರು ಒಳ್ಳೆಯ ಮನುಷ್ಯರು, ದಯಾವಂತರು. ಇಂಗ್ಲಿಷರು ಇಂಡಿಯಾದೇಶಕ್ಕೆ ಒಡೆಯರಾದ ಮೇಲೆ ಮಿಷನರಿಗಳ ಚಳುವಳಿ ಕುಂಠಿತವಾಯಿತು. ಭರತಖಂಡದಲ್ಲಿ ಅದು ಈಗಲೂ ಅದೇ ಸ್ಥಿತಿಯಲ್ಲಿದೆ. ಹಿಂದೆ ಬಂದ ಡಾಕ್ಟರ್​ ಲಾಂಗ್​ ಎಂಬ ಮಿಷನರಿ ಜನರ ಪರವಾಗಿ ನಿಂತನು. ಅವನು ನೀಲಿ ಬೆಳೆಯುವ ತೋಟದವರು ಕೂಲಿಗಳಿಗೆ ಮಾಡುತ್ತಿರುವ ಅನ್ಯಾಯವನ್ನು ಕುರಿತು ಬರೆದ ಒಂದು ಹಿಂದೂ ನಾಟಕವನ್ನು ಇಂಗ್ಲಿಷಿಗೆ ಭಾಷಾಂತರ ಮಾಡಿದನು. ಇದರ ಪರಿಣಾಮ ಏನಾಯಿತು ಗೊತ್ತೇ? ಇಂಗ್ಲಿಷರು ಅವನನ್ನು ಜೈಲಿಗೆ ಹಾಕಿದರು. ಇಂತಹ\break ಮಿಷನರಿಗಳು ಭರತಖಂಡಕ್ಕೆ ಬಹಳ ಸಹಾಯ ಮಾಡಿರುವರು. ಆದರೆ ಅವರೆಲ್ಲ ಈಗ\break ಕಾಲವಾಗಿ ಹೋಗಿರುವರು. ಸೂಯೆಜ್​ ಕಾಲುವೆ ಆದಮೇಲೆ ಹಲವು ಪಾಪಕೃತ್ಯಗಳು ಅದರ ಮೂಲಕ ಭರತಖಂಡಕ್ಕೆ ಬಂದವು.

ಈಗ ಇರುವ ಮಿಷನರಿ ಮದುವೆಯಾದವನು. ಅವನು ಮದುವೆಯಾಗಿರುವುದರಿಂದ ಸ್ವತಂತ್ರನಲ್ಲ. ಮಿಷನರಿಗೆ ಜನಗಳ ವಿಷಯ ಏನೂ ಗೊತ್ತಿಲ್ಲ. ಅವನಿಗೆ ಇಲ್ಲಿಯ ಭಾಷೆ ತಿಳಿಯದು. ಆದಕಾರಣ ಅವನು ಬಿಳಿ ಮನುಷ್ಯರು ಇರುವ ಕಡೆ ವಾಸ ಮಾಡುವನು. ಅವನು ಮದುವೆಯಾಗಿರುವುದರಿಂದ ಹಾಗೆ ಮಾಡಲೇಬೇಕಾಗಿದೆ. ಅವನಿಗೆ ಮದುವೆಯಾಗಿಲ್ಲದೇ ಇದ್ದರೆ ಜನರೊಡನೆ ಬೆರೆಯಬಹುದಾಗಿತ್ತು. ಅವಶ್ಯಕತೆ ಬಂದರೆ ನೆಲದ ಮೇಲೆಯೇ ಮಲಗಬಹುದಾಗಿತ್ತು. ಅವನು ಇಂಗ್ಲಿಷು ಮಾತನಾಡುವವರೊಡನೆ ವಾಸಮಾಡುವನು. ಮಿಷನರಿಗಳ ಪ್ರಭಾವ ಭರತಖಂಡದ ಬಹುಭಾಗಕ್ಕೆ ಇನ್ನೂ ತಾಕಿಲ್ಲ. ಅನೇಕ ಮಿಷನರಿಗಳು ಅನರ್ಹರು. ಸಂಸ್ಕೃತವನ್ನು ಅರ್ಥಮಾಡಿಕೊಳ್ಳಬಲ್ಲ ಒಬ್ಬ ಮಿಷನರಿಯನ್ನೂ ನಾನು ಇನ್ನೂ ಕಂಡಿಲ್ಲ. ಜನರ ಆಚಾರ ವ್ಯವಹಾರಗಳೇ ಗೊತ್ತಿಲ್ಲದವನು ಅವರಿಗೆ ಸಹಾನುಭೂತಿ ತೋರಿಸುವುದಾದರೂ ಹೇಗೆ? ನಾನು ದೂರುತ್ತಿಲ್ಲ. ಆದರೆ ಸಾಮರ್ಥ್ಯವಿಲ್ಲದವರನ್ನು ಮಿಷನರಿಗಳಾಗಿ ಕ್ರೈಸ್ತರು ಇಂಡಿಯಾದೇಶಕ್ಕೆ ಕಳುಹಿಸುತ್ತಿರುವರು. ಸರಿಯಾದ ಪ್ರತಿಫಲ ಬರದೆ ಇರುವಾಗ ಮತಾಂತರಗೊಳಿಸುವುದಕ್ಕೆ ಹಣವನ್ನು ಖರ್ಚುಮಾಡುತ್ತಿರುವುದೂ ವಿಷಾದಕರ.

ಕ್ರೈಸ್ತಧರ್ಮಕ್ಕೆ ಮತಾಂತರಗೊಂಡ ಕೆಲವರು ಮಿಷನರಿಗಳನ್ನು ನೆಚ್ಚಿಕೊಂಡು ಜೀವನ ಮಾಡುತ್ತಿರುವರು. ಮತಾಂತರಗೊಂಡು ಕೆಲಸ ಸಿಕ್ಕದೇ ಹೋದರೆ ಕ್ರೈಸ್ತರಾಗಿ ಉಳಿಯುವುದಿಲ್ಲ. ಇದೇ ನಿಜವಾದ ವಸ್ತುಸ್ಥಿತಿ. ಅವರನ್ನು ಕ್ರೈಸ್ತರನ್ನಾಗಿ ಮಾಡುವ ವಿಧಾನವಾದರೂ ನಾಚಿಕೆಗೇಡು, ಮಿಷನರಿಗಳು ಕೊಡುವ ದುಡ್ಡನ್ನು ತೆಗೆದುಕೊಳ್ಳುವರು. ಮಿಷನರಿಗಳು ಸ್ಥಾಪಿಸಿದ ಕಾಲೇಜುಗಳು ವಿದ್ಯಾಭ್ಯಾಸದ ದೃಷ್ಟಿಯಿಂದ ಸರಿಯಾಗಿವೆ. ಆದರೆ ಧಾರ್ಮಿಕ ಬೇರೆ. ಹಿಂದೂ ಬುದ್ಧಿವಂತ. ಗಾಳದ ಕೊನೆಗೆ ಹಾಕಿರುವ ಹುಳುವನ್ನು ಮಾತ್ರ ತೆಗೆದುಕೊಂಡು ಗಾಳದ ಬಲೆಗೆ ಬೀಳದಂತೆ ನೋಡಿಕೊಳ್ಳುವನು. ಹಿಂದೂಗಳು ಎಷ್ಟು ಸಹನಾಶೀಲರು ಎಂಬುದನ್ನು ನೋಡಿದರೆ ಆಶ್ಚರ್ಯವಾಗುವುದು. ಒಬ್ಬ ಮಿಷನರಿ ಹೀಗೆ ಹೇಳುತ್ತಾನೆ: “ಅವರ ಸಹನಾಶೀಲತೆಯೇ ದೊಡ್ಡದೊಂದು ಸಮಸ್ಯೆಯಾಗಿದೆ. ಯಾರು ತಮ್ಮಲ್ಲಿ ತಾವು ಸಂತುಷ್ಟರಾಗಿರುವರೋ ಅವರನ್ನು ಮತಾಂತರಗೊಳಿಸುವುದು ಅಸಾಧ್ಯ.”

ಇನ್ನು ಮಹಿಳಾ ಮಿಷನರಿಗಳ ವಿಚಾರ. ಅವರು ಕೆಲವರ ಮನೆಗೆ ಹೋಗಿ ಕಸೂತಿಯ\break ಕೆಲಸವನ್ನು ಹೇಳಿಕೊಟ್ಟು ಬೈಬಲನ್ನು ಓದುವುದನ್ನು ಕಲಿಸುವರು. ಅವರಿಗೆ ಇದಕ್ಕೆ\break ತಿಂಗಳಿಗೆ ಇಷ್ಟು ಎಂದು ಸಂಬಳ ಬರುವುದು. ಹಿಂದೂ ದೇಶದ ಹುಡುಗಿಯರ\break ಧರ್ಮವನ್ನು ಬದಲಾಯಿಸುವುದಕ್ಕೆ ಆಗುವುದಿಲ್ಲ. ಮಿಷನರಿಗಳ ದೇಶದಲ್ಲೇ ತಾಂಡವವಾಡುತ್ತಿರುವ ನಾಸ್ತಿಕತೆ ಮತ್ತು ಸಂಶಯ ಇವೇ ಇವರನ್ನು ತಮ್ಮ ದೇಶದಿಂದ ಹೊರಗೆ ಹೋಗುವಂತೆ ಮಾಡಿವೆ. ನಾನು ಈ ದೇಶಕ್ಕೆ ಬಂದಾಗ ಇಲ್ಲಿ ಇಷ್ಟೊಂದು ಉದಾರ ಸ್ವಭಾವದ ಸ್ತ್ರೀಪುರುಷರನ್ನು ನೋಡಿ ನನಗೆ ಆಶ್ಚರ್ಯವಾಯಿತು. ಆದರೆ ವಿಶ್ವಧರ್ಮಸಮ್ಮೇಳನವಾದ ಮೇಲೆ ಒಂದು ಪ್ರಿಸ್ಬಿಟೇರಿಯನ್​ ಪತ್ರಿಕೆ ಟೀಕಾತ್ಮಕ ಲೇಖನವನ್ನು ಪ್ರಕಟಸಿತು. ಸಂಪಾದಕ ಇದನ್ನೇ ಉತ್ಸಾಹ ಎಂದ! ಮಿಷನರಿಗಳು ತಾವು ಇಂತಹ ದೇಶಕ್ಕೆ ಸೇರಿದವರು ಎಂಬ ಭಾವವನ್ನು ಎಂದೂ ಮರೆಯಲಾರರು. ಅವರ ಮನಸ್ಸು ವಿಶಾಲವಾಗಿಲ್ಲ, ಆದಕಾರಣ ಅವರು ಸುಖವಾಗಿ ಭರತಖಂಡದಲ್ಲಿ ಕಾಲಕಳೆಯುತ್ತಿದ್ದರೂ ಇತರರನ್ನು ಧರ್ಮಕ್ಕೆ ಸೇರಿಸುವುದರಲ್ಲಿ ಅಷ್ಟು ಯಶಸ್ವಿಗಳಾಗಿಲ್ಲ. ಭರತಖಂತಕ್ಕೆ ಕ್ರಿಸ್ತನ ಸಹಾಯದ ಅಗತ್ಯವಿದೆ. ಆದರೆ ಕ್ರಿಸ್ತದ್ವೇಷಿಗಳ ಸಹಾಯ ಅವರಿಗೆ ಬೇಕಿಲ್ಲ. ಈ ಮಿಷನರಿಗಳು ಕ್ರಿಸ್ತನಂತಿಲ್ಲ, ಕ್ರಿಸ್ತನಂತೆ ನಡೆದುಕೊಳ್ಳುವುದಿಲ್ಲ. ಅವರೆಲ್ಲ ಮದುವೆಯಾದವರು, ಅವರು ಭರತಖಂಡಕ್ಕೆ ಬಂದು ಸುಖವಾಗಿ ಕಾಲಕಳೆಯುತ್ತಿರುವರು. ಅನೇಕ ಹಿಂದೂ ಸಾಧು ಸಂತರಂತೆ ಕ್ರೈಸ್ತಪಾದ್ರಿಗಳೂ ಕೂಡ ಭರತಖಂಡಕ್ಕೆ ಎಷ್ಟೋ ಉಪಕಾರ ಮಾಡಬಹುದಾಗಿತ್ತು. ಆದರೆ ಅಲ್ಲಿಗೆ ಬರುವ ಮಿಷನರಿಗಳು ಅಷ್ಟು ಪರಿಶುದ್ಧರಲ್ಲ. ಹಿಂದೂಗಳು ಸಂತೋಷದಿಂದ ನಿಜವಾದ ಕ್ರಿಸ್ತನನ್ನು ಸ್ವಾಗತಿಸುವರು. ಏಕೆಂದರೆ ಅವನ ಜೀವನ ಪವಿತ್ರವಾಗಿತ್ತು, ಸುಂದರವಾಗಿತ್ತು. ಆದರೆ ಕ್ರಿಸ್ತನ ಹೆಸರಿನಲ್ಲಿ ಬರುವ ಅಜ್ಞಾನಿಗಳಾದ ಮಿಥ್ಯಾಚಾರಿಗಳನ್ನು ಅವರು ಸ್ವಾಗತಿಸಲಾರರು.

\vskip 0.1cm

ಮನುಷ್ಯರಲ್ಲಿ ವ್ಯತ್ಯಾಸವಿದೆ. ವ್ಯತ್ಯಾಸವಿಲ್ಲದೆ ಇದ್ದಿದ್ದರೆ ಪ್ರಪಂಚ ಅಧೋಗತಿಗೆ ಬರುತ್ತಿತ್ತು. ಹಲವು ಧರ್ಮಗಳು ಇಲ್ಲದೆ ಇದ್ದರೆ ಯಾವ ಒಂದು ಧರ್ಮವೂ ಇರುತ್ತಿರಲಿಲ್ಲ. ಕ್ರೈಸ್ತರಿಗೆ ಅವರ ಧರ್ಮ ಇರಬೇಕು. ಹಿಂದೂಗಳಿಗೆ ಅವರ ಧರ್ಮ ಇರಬೇಕು. ಎಲ್ಲಾ ಧರ್ಮಗಳೂ ಅನ್ಯಧರ್ಮಗಳೊಂದಿಗೆ ಹಲವು ವರುಷಗಟ್ಟಲೆ ಹೋರಾಡಿವೆ. ಯಾವ ಧರ್ಮಕ್ಕೆ ಒಂದು ಶಾಸ್ತ್ರದ ಸಹಾಯವಿತ್ತೋ ಅದು ಇನ್ನೂ ಇದೆ. ಕ್ರೈಸ್ತರು ಯಹೂದ್ಯರನ್ನು\break ಏತಕ್ಕೆ ಮತಾಂತರಗೊಳಿಸಲಿಲ್ಲ? ಪರ್ಷಿಯಾ ದೇಶದವರನ್ನು ಏತಕ್ಕೆ ಮತಾಂತರಗೊಳಿಸಲಿಲ್ಲ? ಮಹಮ್ಮದೀಯರನ್ನು ಏತಕ್ಕೆ ಮಂತಾಂತರಗೊಳಿಸಲಿಲ್ಲ? ಚೈನಾದಲ್ಲಿ ಮತ್ತು ಜಪಾನಿನಲ್ಲಿ ಏತಕ್ಕೆ ತಮ್ಮ ಪ್ರಭಾವವನ್ನು ಬೀರಲು ಆಗಲಿಲ್ಲ? ಪ್ರಥಮ ಮಿಷನರಿ ಧರ್ಮವಾದ ಬೌದ್ಧ ಧರ್ಮದಲ್ಲಿ ಬೇರೆ ಧರ್ಮಗಳಿಗೆ ಮತಾಂತರಗೊಂಡವರಿಗಿಂತ ಎರಡರಷ್ಟು ಸಂಖ್ಯೆಯ ಜನರಿರುವರು. ಆದರೂ ಅವರು ಖಡ್ಗವನ್ನು ಬಳಸಲಿಲ್ಲ. ಮಹಮ್ಮದೀಯರು ಅತಿ ಹೆಚ್ಚಿನ ಪ್ರಮಾಣದ ಹಿಂಸೆಯನ್ನು ಮಾಡಿದರು. ಮೂರು ದೊಡ್ಡ ಮಿಷನರಿ ಧರ್ಮಗಳಲ್ಲಿ ಮಹಮ್ಮದೀಯರ ಸಂಖ್ಯೆ ಕಡಮೆ. ಮಹಮ್ಮದೀಯರ ಕಾಲ ಆಗಿಹೋಯಿತು. ಪ್ರತಿದಿನ ಕ್ರೈಸ್ತ ದೇಶಗಳು ರಕ್ತವನ್ನು ಹರಿಸಿ ಹೊಸ ಹೊಸ ದೇಶಗಳನ್ನು ಗೆಲ್ಲುತ್ತಿರುವುದನ್ನು ನೋಡುತ್ತಿರುವೆನು. ಇದಕ್ಕೆ ವಿರುದ್ಧವಾಗಿ ಮಿಷನರಿಗಳು ಮಾತನ್ನೇ ಆಡುವುದಿಲ್ಲ. ಏಕೆ? ರಕ್ತದಾಹದಿಂದ ಪ್ರೇರಿತವಾದ ದೇಶಗಳು ತಾವು ಕ್ರೈಸ್ತರೆಂದು ಏಕೆ ಹೇಳಿಕೊಳ್ಳಬೇಕು? ಇದು ನಿಜವಾದ ಕ್ರೈಸ್ತಧರ್ಮ ಅಲ್ಲ. ಯಹೂದ್ಯರು ಮತ್ತು ಅರಬ್ಬರು ಕ್ರೈಸ್ತಮತದ ಜನಕರು. ಆದರೆ ಕ್ರೈಸ್ತರು ಅವರನ್ನು ಹೇಗೆ ಹಿಂಸಿಸಿರುವರು! ಕ್ರೈಸ್ತರನ್ನು ಇಂಡಿಯಾ ದೇಶದಲ್ಲಿ ಒರಗಲ್ಲಿಗೆ ತಿಕ್ಕಿ ಆಗಿದೆ. ಅವರ ಬಂಡವಾಳ ಗೊತ್ತಾಗಿದೆ. ನಾನು ನಿರ್ದಯನಾಗಿ ಮಾತನಾಡುತ್ತಿಲ್ಲ. ಆದರೆ ಅವರು ಕ್ರೈಸ್ತೇತರರ ಕಣ್ಣಿಗೆ ಹೇಗೆ ಕಾಣಿಸುತ್ತಿರುವರು ಎಂಬುದನ್ನು ತೋರುತ್ತಿರುವೆನು. ನರಕದಲ್ಲಿ ಬೆಂಕಿಯ ಜ್ವಾಲೆಯಿಂದಾವೃತವಾದ ಬಾವಿಗಳಿವೆ ಎಂದು ಪ್ರಚಾರ ಮಾಡುವ ಮಿಷನರಿಯ ವಿಷಯದಲ್ಲಿ ಹಿಂದೂಗಳಿಗೆ ಜುಗುಪ್ಸೆ ಇದೆ. ಮಹಮ್ಮದೀಯರು ಖಡ್ಗಗಳನ್ನು ಝಳಪಿಸಿಕೊಂಡು ಒಬ್ಬರಾದ ಮೇಲೆ ಒಬ್ಬರು ಇಂಡಿಯಾ ದೇಶದ ಮೇಲೆ ಬಿದ್ದರು. ಆದರೆ ಈಗ ಅವರೆಲ್ಲ ಎಲ್ಲಿ?

ಎಲ್ಲ ಧರ್ಮಗಳ ಪರಮ ಗುರಿ ಆಧ್ಯಾತ್ಮಿಕ ಸತ್ಯವನ್ನು ಗುರುತಿಸುವುದು. ಯಾವ ಧರ್ಮವೂ ಇದನ್ನು ಮೀರಿ ಬೋಧಿಸಲಾರದು. ಪ್ರತಿಯೊಂದು ಧರ್ಮದಲ್ಲಿಯೂ\break ಧರ್ಮದ ಸಾರವಾದ ಸತ್ಯವಿದೆ ಮತ್ತು ಆ ರತ್ನವನ್ನು ರಕ್ಷಿಸುತ್ತಿರುವ ಪೆಟ್ಟಿಗೆ ಇದೆ. ಯೆಹೂದ್ಯರ ಅಥವಾ ಹಿಂದೂಗಳ ಶಾಸ್ತ್ರವನ್ನು ನಂಬುವುದು ಗೌಣ. ಸಮಯ ಸನ್ನಿವೇಶಗಳು ಬದಲಾಗುವುವು. ಪಾತ್ರೆಗಳು ಬೇರೆ ಬೇರೆಯಾಗಿ ಕಾಣಿಸುವುವು. ಆದರೆ ಮುಖ್ಯವಾದ ಸತ್ಯ ಯಾವಾಗಲೂ ಒಂದೇ ಇರುವುದು. ಸಾರವಸ್ತು ಎಲ್ಲಾ ಧರ್ಮಗಳಲ್ಲಿಯೂ ಒಂದೇ ಆಗಿರುವುದರಿಂದ ಆ ಧರ್ಮದ ವಿದ್ಯಾವಂತರು ಅದನ್ನು ಮಾತ್ರ ಲಕ್ಷ್ಯದಲ್ಲಿಡುವರು. ನೀವು ಧರ್ಮದ ಮುಖ್ಯ ವಿಷಯಗಳಾವುವು ಎಂದು ಕ್ರೈಸ್ತರನ್ನು ಕೇಳಿದರೆ ಏಸುವಿನ ಸಂದೇಶ ಎನ್ನುವರು. ಉಳಿದವುಗಳೆಲ್ಲ ಕೆಲಸಕ್ಕೆ ಬಾರದವು. ಆದರೆ ಕೆಲಸಕ್ಕೆ ಬಾರದುದು ಬೇಕು, ಏಕೆಂದರೆ ಅದರೊಳಗೇ ರತ್ನವಿರುವುದು. ಮುತ್ತಿನ ಚಿಪ್ಪು ಕೆಲಸಕ್ಕೆ ಬರುವುದಿಲ್ಲ. ಆದರೆ ಆ ಚಿಪ್ಪಿನೊಳಗೆ ಮುತ್ತು ಇದೆ. ಹಿಂದೂಗಳು ಎಂದಿಗೂ ಏಸುವಿನ ಜೀವನವನ್ನು ಟೀಕಿಸುವುದಿಲ್ಲ. ಅವನು ಮಾಡಿದ ಗುಡ್ಡದ ಮೇಲಿನ ಸಂದೇಶವನ್ನು ಗೌರವಿಸುತ್ತಾರೆ.\break ಆದರೆ ಹಿಂದೂ ಆಚಾರ್ಯರ ಸಂದೇಶಗಳನ್ನು ಎಷ್ಟು ಜನ ಕ್ರೈಸ್ತರು ಕೇಳಿರುವರು? ಅವರೊಂದು ಮೂರ್ಖರ ಸ್ವರ್ಗದಲ್ಲಿರುವರು. ಕೆಲವರನ್ನು ಮತಾಂತರಗೊಳಿಸುವುದರೊಳಗೆ\break ಕ್ರೈಸ್ತ ಧರ್ಮ ಹಲವು ಉಪಪಂಗಡಗಳಾಯಿತು. ಇದೇ ಪ್ರಕೃತಿಯ ನಿಯಮ. ಪ್ರಪಂಚದಲ್ಲಿ ಹಲವು ಧರ್ಮಗಳ ಒಂದು ವಾದ್ಯಗೋಷ್ಠಿ ಇದೆ. ಯಾವುದೋ ಒಂದನ್ನು ಮಾತ್ರ ಏತಕ್ಕೆ ತೆಗೆದುಕೊಳ್ಳುತ್ತೀರಿ? ಈ ಮಿಶ್ರಮೇಳ ಮುಂದುವರಿಯಲಿ. ಪರಿಶುದ್ಧರಾಗಿ. ಮೌಢ್ಯವನ್ನು ತೊರೆಯಿರಿ. ಪ್ರಕೃತಿಯಲ್ಲಿರುವ ಅದ್ಭುತವಾದ ಸಾಮರಸ್ಯವನ್ನು ನೋಡಿ. ಮೌಢ್ಯವೇ ಧರ್ಮದಲ್ಲಿ ಮೇಲಾಗುವುದು. ಎಲ್ಲಾ ಧರ್ಮಗಳೂ ಒಳ್ಳೆಯವೇ. ಏಕೆಂದರೆ ಅವುಗಳ ಸಾರವೆಲ್ಲಾ ಒಂದೇ. ಪ್ರತಿಯೊಬ್ಬನಿಗೂ ತನ್ನ ವೈಶಿಷ್ಟ್ಯವನ್ನು ಅಭಿವ್ಯಕ್ತಗೊಳಿಸಲು ತಕ್ಕ ಸ್ವಾತಂತ್ರ್ಯವಿರಬೇಕು. ಆದರೆ ಈ ವೈಶಿಷ್ಟ್ಯಗಳೆಲ್ಲಾ ಸೇರಿ ಒಂದು ಸಮರಸವಾದ ಪೂರ್ಣವಾಗುವುದು. ಈ ಸ್ವಭಾವ ಆಗಲೇ ಜಗತ್ತಿನಲ್ಲಿದೆ. ಪ್ರತಿಯೊಂದು ಧರ್ಮವೂ ತನ್ನ ಪಾಲಿನದನ್ನು ಸಮಷ್ಟಿಗೆ ಧಾರೆ ಎರೆಯಬೇಕಾಗಿದೆ.

ಏಸುವಿನ ಜೀವನ ಸೌಂದರ್ಯವನ್ನು ಮೆಚ್ಚಲಾರದ ಹಿಂದೂವನ್ನು ಕಂಡರೆ ನನಗೆ ವ್ಯಸನವಾಗುವುದು. ಹಾಗೆಯೇ ಹಿಂದೂಗಳಲ್ಲಿ ಕ್ರಿಸ್ತನಂತೆ ಇರುವವರಿಗೆ ಗೌರವ ತೋರದ ಕ್ರೈಸ್ತನಿಗಾಗಿಯೂ ವಿಷಾದಪಡುವೆನು. ಒಬ್ಬ ತನ್ನನ್ನು ಹೆಚ್ಚು ಪರೀಕ್ಷೆ ಮಾಡಿಕೊಂಡಷ್ಟೂ ಇತರರಲ್ಲಿ ಅವರ ಛಿದ್ರಾನ್ವೇಷಣೆ ಕಡಮೆಯಾಗುವುದು. ಯಾರು ಇತರರನ್ನು ಮತಾಂತರಗೊಳಿಸುವ ಭರದಲ್ಲಿರುವರೋ, ಇತರರ ಉದ್ಧಾರದಲ್ಲಿ ಕಾತರರಾಗಿರುವರೊ, ಅವರು ಅನೇಕವೇಳೆ ತಮ್ಮ ಆತ್ಮದ ಉದ್ಧಾರವನ್ನೇ ಮರೆಯುವರು. ಯಾರೊ ಒಬ್ಬ ಮಹಿಳೆ ಭಾರತದ ಸ್ತ್ರೀಯರನ್ನು ಏತಕ್ಕೆ ಮೇಲೆತ್ತಲಿಲ್ಲ ಎಂದು ನನ್ನನ್ನು ಪ್ರಶ್ನಿಸಿದಳು. ಅದಕ್ಕೆ ಮುಖ್ಯ ಕಾರಣ ಅನಾಗರಿಕರು ಪದೇ ಪದೇ ಭರತಖಂಡದ ಮೇಲೆ ಧಾಳಿ ಇಡುತ್ತಿದ್ದುದೇ ಆಗಿದೆ. ಜೊತೆಗೆ ಇಂಡಿಯಾ ದೇಶದವರ ತಪ್ಪೂ ಸ್ವಲ್ಪ ಇದೆ. ಆದರೆ ನಮ್ಮ ಸ್ತ್ರೀಯರು ಎಂದಾದರೂ ನಿಮ್ಮ ದೇಶದವರಿಗಿಂತ ಮೇಲು. ನಿಮ್ಮವರು ಯಾವಾಗಲೂ ಕಾದಂಬರಿ ಮತ್ತು ನಾಟ್ಯ ಇವುಗಳಿಗಾಗಿ ಹುಚ್ಚರಾಗಿರುವರು. ತಮ್ಮ ನಾಗರಿಕತೆಯ ಸಮಾನ ಇನ್ನೊಂದಿಲ್ಲ ಎಂದು ಹೆಮ್ಮೆ ಕೊಚ್ಚಿಕೊಳ್ಳುವ ದೇಶದಲ್ಲಿ ಅಧ್ಯಾತ್ಮವೆಲ್ಲಿದೆ? ನನಗೆ ಅದು ಇನ್ನೂ ಕಾಣಿಸಲಿಲ್ಲ. ಇಹ ಪರ ಎಂಬ ಮಾತು ಬರಿಯ ಮಕ್ಕಳನ್ನು ಅಂಜಿಸುವುದಕ್ಕೆ ಆಗಿದೆ. ಎಲ್ಲಾ ಇಹದಲ್ಲಿದೆ ಈಗ; ಈ ದೇಹದಲ್ಲಿರುವಾಗಲೇ ದೇವರೊಡನೆ ಇರಬೇಕು! ಸ್ವಾರ್ಥವೆಲ್ಲಾ ಮಾಯವಾಗಬೇಕು; ಮೌಢ್ಯವನ್ನೆಲ್ಲಾ ಆಚೆಗೆ ಕಿತ್ತು ಒಗೆಯಬೇಕು. ಇಂತಹ ಜನರು ಭರತಖಂಡದಲ್ಲಿ ಈಗ ಇರುವರು. ನಿಮ್ಮ ದೇಶದಲ್ಲಿ ಇಂತಹ ಜನರು ಎಲ್ಲಿ? ನಿಮ್ಮ ಮಿಷನರಿಗಳೋ ನಮ್ಮನ್ನು ಕನಸುಣಿಗಳೆಂದು ನಮ್ಮ ವಿರುದ್ಧವಾಗಿಯೇ ಬೋಧಿಸುವರು. ನಿಮ್ಮ ದೇಶದಲ್ಲಿ ನಮ್ಮಂತಹ ಕನಸುಣಿಗಳು ಹೆಚ್ಚು ಸಂಖ್ಯೆಯಲ್ಲಿ ಇದ್ದಿದ್ದರೆ ದೇಶ ಎಷ್ಟೋ ಮೇಲಾಗುತ್ತಿತ್ತು. ಈ ದೇಶದಲ್ಲಿ ಏಸುವಿನ ಸಂದೇಶವನ್ನು ಅಕ್ಷರಶಃ ಪಾಲಿಸಲು ಯತ್ನಿಸಿದರೆ ಅವನನ್ನು ಮತಾಂಧ ಎಂದು ಹಳಿಯುವರು. ಕನಸುಣಿಗಳಿಗೂ, ಹತ್ತೊಂಭತ್ತನೆಯ ಶತಮಾನದ ಹರಟೆಕೋರರಿಗೂ ಎಷ್ಟೋ ವ್ಯತ್ಯಾಸವಿದೆ. ದುಂಬಿಗಳು ಹೂವನ್ನು ಅರಸುವುವು. ಕಮಲ ಅರಳಲಿ. ಇಡೀ ಜಗತ್ತೆಲ್ಲ ದೇವರಿಂದ ತುಂಬಿತುಳುಕಾಡುತ್ತದೆ, ಪಾಪದಿಂದಲ್ಲ. ಪರಸ್ಪರ ನೆರವು ನೀಡೋಣ, ಪರಸ್ಪರ ಪ್ರೀತಿಸೋಣ. ಬೌದ್ಧರ ಒಂದು ಸೊಗಸಾದ ಪ್ರಾರ್ಥನೆ ಇದು: “ನಾನು ಎಲ್ಲಾ ಮಹಾತ್ಮರಿಗೂ ವಂದಿಸುತ್ತೇನೆ. ನಾನು ಎಲ್ಲಾ ಪ್ರವಾದಿಗಳಿಗೂ ನಮಸ್ಕರಿಸುತ್ತೇನೆ, ಜಗತ್ತಿನಲ್ಲಿರುವ ಸಂತರಾದ ಸ್ತ್ರೀಪುರುಷರಿಗೆಲ್ಲಾ ನಾನು ನಮಸ್ಕರಿಸುತ್ತೇನೆ.”

