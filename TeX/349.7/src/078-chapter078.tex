
\chapter[ಏಕಾಗ್ರತೆ ]{ಏಕಾಗ್ರತೆ \protect\footnote{\engfoot{C.W. Vol. VI, P. 123}}}

ಏಕಾಗ್ರತೆಯೇ ಎಲ್ಲಾ ಜ್ಞಾನಗಳ ಸಾರ. ಅದಿಲ್ಲದೆ ಏನೂ ಸಾಧ್ಯವಿಲ್ಲ. ಸಾಧಾರಣ ವ್ಯಕ್ತಿಯು ತನ್ನ ಮನಸ್ಸಿನ ಶೇಕಡ ತೊಂಬತ್ತರಷ್ಟು ಶಕ್ತಿಯನ್ನು ವ್ಯರ್ಥ ಮಾಡುತ್ತಿರುವನು. ಆದಕಾರಣವೇ ಅವನು ಪದೇ ಪದೇ ತಪ್ಪು ಮಾಡುತ್ತಿರು ವುದು. ಏಕಾಗ್ರತೆಯಲ್ಲಿ ಪಳಗಿದ ವ್ಯಕ್ತಿ ಅಥವಾ ಮನಸ್ಸು ತಪ್ಪುಮಾಡುವುದಿಲ್ಲ. ಮನಸ್ಸನ್ನು ಏಕಾಗ್ರಮಾಡಿ ಅದನ್ನು ಅದರ ಕಡೆಗೇ ಹಿಂದೆ ತಿರುಗಿಸಿದಾಗ ನಮ್ಮಲ್ಲಿರು ವುದೆಲ್ಲ ನಮ್ಮ ಅಧೀನವಾಗುವುದು, ನಾವು ಅದರ ಅಧೀನತೆಯಲ್ಲಿರುವುದಿಲ್ಲ. ಗ್ರೀಕರು ಏಕಾಗ್ರತೆಯನ್ನು ಬಾಹ್ಯಪ್ರಪಂಚದ ಕಡೆ ತಿರುಗಿಸಿದರು. ಆದಕಾರಣವೇ ಅವರ ಕಲೆ, ಸಾಹಿತ್ಯ ಇವುಗಳು ಪರಿಪೂರ್ಣತೆಯನ್ನು ಪಡೆದುವು. ಹಿಂದೂವು ಆಂತರಿಕ ಜಗತ್ತಿನ ಮೇಲೆ, ಅಗೋಚರವಾದ ಆತ್ಮಭೂಮಿಕೆಯ ಮೇಲೆ, ತನ್ನ ಏಕಾಗ್ರತೆಯನ್ನು ಬೀರಿದನು. ಅದರಿಂದಲೇ ಯೋಗಶಾಸ್ತ್ರ ಹುಟ್ಟಿತು. ಯೋಗ ಎಂದರೆ ಇಂದ್ರಿಯ, ಮನಸ್ಸು, ಇಚ್ಛೆ ಇವುಗಳನ್ನು ನಿಗ್ರಹಿಸುವ ಶಾಸ್ತ್ರ. ಇದರ ಪ್ರಯೋಜನವೆಂದರೆ ಮನಸ್ಸು ನಮ್ಮನ್ನು ನಿಗ್ರಹಿಸುವ ಬದಲು ನಾವು ಅದನ್ನು ನಿಗ್ರಹಿಸುತ್ತೇವೆ. ಮನಸ್ಸಿನಲ್ಲಿ ಹಲವು ಪದರಗಳಿವೆ. ನಮ್ಮ ಗುರಿಯೇ ನಮ್ಮ ಅಸ್ತಿತ್ವದ ಅನೇಕ ಪದರಗಳನ್ನು ಭೇದಿಸಿ ದೇವರನ್ನು ಕಾಣುವುದು. ಯೋಗದ ಗುರಿ ಭಗವಂತನ ಸಾಕ್ಷಾತ್ಕಾರ. ಇದನ್ನು ಮಾಡಬೇಕಾದರೆ ನಾವು ಸಾಪೇಕ್ಷ ಜ್ಞಾನವನ್ನು ಮೀರಿ ಹೋಗಬೇಕು.; ಇಂದ್ರಿಯ ಜಗತ್ತನ್ನು ಮೀರಿ ಹೋಗಬೇಕು. ಜಗತ್ತಿನ ಜನರು ಇಂದ್ರಿಯ ಜಗತ್ತಿನಲ್ಲಿ ಎಚ್ಚತ್ತಿರುವರು. ಭಗವಂತನನ್ನು ಉಪಾಸನೆ ಮಾಡುವವರು ಅಲ್ಲಿ ಮಲಗಿರುವವರು; ಈ ಜಗತ್ತಿನ ಜನರು ನಿತ್ಯಸತ್ಯದ ಕಡೆ ಅಜಾಗೃತರಾಗಿರುವರು, ಭಗವಂತನ ಉಪಾಸನೆ ಮಾಡುವವರು ಅಲ್ಲಿ ಜಾಗೃತ ರಾಗಿರುವರು. ಇಂದ್ರಿಯಗಳನ್ನು ನಿಗ್ರಹಿಸುವುದಕ್ಕೆ ಒಂದೇ ಮಾರ್ಗ ಇರುವುದು: ಈ ಜಗತ್ತಿನಲ್ಲಿ ಯಾವನು ಏಕಮಾತ್ರ ಸತ್ಯವೋ ಅವನನ್ನು ನೋಡುವುದು. ಆಗ ಮಾತ್ರ ನಾವು ನಿಜವಾಗಿ ಇಂದ್ರಿಯವನ್ನು ಜಯಿಸಬಲ್ಲೆವು.

ಮನಸ್ಸಿನ ಚಿಂತನಾ ವ್ಯಾಪ್ತಿಯನ್ನು ಸೀಮಿತಗೊಳಿಸುತ್ತಾ ಹೋಗುವುದೇ ಏಕಾಗ್ರತೆ. ಹಾಗೆ ಮಾಡುವುದಕ್ಕೆ ಎಂಟು ಮೆಟ್ಟಿಲುಗಳಿವೆ. ಮೊದಲನೆಯದೆ ಯಮ. ಬಾಹ್ಯ ವಸ್ತುಗಳನ್ನು ನಿವಾರಿಸುತ್ತಾ ಮನಸ್ಸನ್ನು ನಿಗ್ರಹಿಸುವುದೇ ಯಮ. ಎಲ್ಲಾ ನೀತಿಯೂ ಇದರಲ್ಲಿ ಅಡಕವಾಗಿದೆ. ಪಾಪವನ್ನು ಆಚರಿಸದಿರುವುದು. ಯಾರಿಗೂ ಹಿಂಸೆಯನ್ನು ಮಾಡದಿರುವುದು. ನೀವು ಹನ್ನೆರಡು ವರುಷಗಳವರೆಗೆ ಅಹಿಂಸಾ ವ್ರತವನ್ನು ಪರಿಪಾಲನೆ ಮಾಡಿದರೆ ಸಿಂಹ, ಹುಲಿಗಳು ಕೂಡ ನಿಮ್ಮ ಮುಂದೆ ಸಾಧು ಪ್ರಾಣಿಗಳಾಗುವುವು. ಸತ್ಯವನ್ನು ಹೇಳಿ, ಹನ್ನೆರಡು ವರುಷಗಳ ವರೆಗೆ ಒಬ್ಬನು ಮನೋವಾಕ್ಕಾಯವಾಗಿ ಸತ್ಯವನ್ನು ಹೇಳಿದರೆ ಅವನು ಇಚ್ಛಿಸಿದುದು ಅವನಿಗೆ ದೊರಕುವುದು. ಮನೋವಾಕ್ಕಾಯವಾಗಿ ಬ್ರಹ್ಮಚರ್ಯ ನಿಯಮವನ್ನು ಪಾಲಿಸಬೇಕು. ಬ್ರಹ್ಮಚರ್ಯವೇ ಎಲ್ಲಾ ಧರ್ಮದ ಮೂಲ. ವೈಯಕ್ತಿಕ ಪರಿಶುದ್ಧತೆ ಅತ್ಯಾವಶ್ಯಕ. ಎರಡನೆಯದೆ ನಿಯಮ. ಮನಸ್ಸು ಹೊರಗೆ ಹೋಗಲು ಅವಕಾಶ ಕೊಡದೆ ಇರುವುದು. ಅನಂತರ ಆಸನ. ಎಂಬತ್ತನಾಲ್ಕು ಆಸನಗಳಿವೆ. ಆದರೆ ಯಾವ ಆಸನ ಯಾರಿಗೆ ಸ್ವಾಭಾವಿಕವಾಗಿರುವುದೋ ಅದೇ ಒಳ್ಳೆಯದು. ಅಂದರೆ ಯಾವ ಸ್ಥಿತಿಯಲ್ಲಿ ನಾವು ದೀರ್ಘಕಾಲ ಸುಖವಾಗಿ ಕುಳಿತಿರಬಲ್ಲೆವೊ ಅದು. ಇದು ಆದ ಅನಂತರವೆ ಪ್ರಾಣಾಯಾಮ ಬರುವುದು. ಅನಂತರ ಪ್ರತ್ಯಾಹಾರ ಎಂದರೆ, ವಸ್ತುಗಳಿಂದ ಇಂದ್ರಿಯಗಳನ್ನು ಹಿಂದಕ್ಕೆ ಎಳೆದುಕೊಂಡು ಬರುವುದು. ಅನಂತರವೆ ಧಾರಣ, ಇದಾದ ಮೇಲೆ ಧ್ಯಾನ. ಕೊನೆಯದೆ ಸಮಾಧಿ. ದೇಹ ಮತ್ತು ಮನಸ್ಸು ಪರಿಶುದ್ಧವಾದಷ್ಟು ಪ್ರತಿಫಲಗಳು ಬೇಗ ಬರುತ್ತವೆ. ನೀವು ಪರಿಶುದ್ಧವಾಗಿರಬೇಕು. ಹೀನ ವಿಷಯಗಳನ್ನು ಕುರಿತು ಆಲೋಚಿಸಕೂಡದು. ಇಂತಹ ವಿಷಯಗಳು ನಿಜವಾಗಿ ನಮ್ಮನ್ನು ಕೆಳಕ್ಕೆ ಎಳೆಯುತ್ತವೆ. ನೀವು ಪರಿಶುದ್ಧ ರಾಗಿದ್ದರೆ, ಶ್ರದ್ಧೆಯಿಂದ ಅಭ್ಯಾಸಮಾಡಿದರೆ, ಮನಸ್ಸನ್ನು ಒಂದು ಅದ್ಭುತ ಶಕ್ತಿಯಿಂದ ಕೂಡಿದ ಪ್ರಕಾಶಮಾನವಾದ ಜ್ಯೋತಿಯನ್ನಾಗಿ ಮಾಡಬಹುದು. ಇದರ ಶಕ್ತಿಗೆ ಒಂದು ಮಿತಿಯಿಲ್ಲ. ಆದರೆ ಸತತ ಅಭ್ಯಾಸಮಾಡುತ್ತಿರಬೇಕು. ಅನಾಸಕ್ತಿ ಇರಬೇಕು. ವ್ಯಕ್ತಿಯು ಸಮಾಧಿ ಸ್ಥಿತಿಯನ್ನು ಸಾಧಿಸಿದಾಗ ದೇಹಭಾವನೆ ಯೆಲ್ಲ ಅಳಿಸಿ ಹೋಗುವುದು. ಆಗ ಮಾತ್ರ ಅವನು ಮುಕ್ತನಾಗುತ್ತಾನೆ. ಅಮೃತನಾಗುತ್ತಾನೆ. ತೋರಿಕೆಗೆ ಅಪ್ರಜ್ಞೆ (\enginline{Unconscious}) ಮತ್ತು ಅತಿಪ್ರಜ್ಞೆ (\enginline{Superunconscious}) ಎರಡೂ ಒಂದೇ ರೀತಿ ಇರುವುವು. ಆದರೆ ಅವುಗಳಲ್ಲಿ ಮಣ್ಣಿನ ಹೆಂಟೆಗೂ ಚಿನ್ನದ ಗಟ್ಟಿಗೂ ಇರುವಷ್ಟು ಅಂತರವಿದೆ. ಯಾರು ತಮ್ಮ ಆತ್ಮನನ್ನು ಭಗವಂತನಿಗೆ ಅರ್ಪಣೆ ಮಾಡಿರುವರೊ ಅವರಾಗಲೆ ಅತಿಪ್ರಜ್ಞೆಯ ಅವಸ್ಥೆಯನ್ನು ಮುಟ್ಟಿರುವರು.

