
\vspace{-0.8cm}

\chapter[ಕಲೆ ]{ಕಲೆ \protect\footnote{\engfoot{C.W. Vol. V, p. 258}}}

ಗ್ರೀಕ್​ ಕಲೆಯ ರಹಸ್ಯವೆ ಪ್ರಕೃತಿಯನ್ನು ಪ್ರತಿಯೊಂದು ವಿವರದಲ್ಲಿಯೂ ಅನುಕರಿಸುವುದು. ಭಾರತೀಯ ಕಲೆಯ ರಹಸ್ಯವಾದರೋ ಒಂದು ಆದರ್ಶವನ್ನು ನಿರೂಪಿಸುವುದರಲ್ಲಿದೆ. ಗ್ರೀಕ್​ ಚಿತ್ರಕಾರನು ತನ್ನ ಶಕ್ತಿಯನ್ನೆಲ್ಲ ಒಂದು ಚೂರು ಸ್ಥೂಲವಾಗಿರುವ ಮಾಂಸದ ತುಂಡನ್ನು ಚಿತ್ರಿಸುವುದರಲ್ಲಿ ವಿನಿಯೋಗಿಸುವನು. ಅದರಲ್ಲಿ ಎಷ್ಟರ ಮಟ್ಟಿಗೆ ಜಯಶೀಲ\-ನಾಗುವನು ಎಂದರೆ ನಾಯಿ ಅದನ್ನು ನಿಜವಾದ ಮಾಂಸದ ತುಂಡೆಂದು ಭ್ರಮಿಸಿ ಕಚ್ಚಲು ಹೋಗುವುದು. ಪ್ರಕೃತಿಯ ಅನುಕರಣದಲ್ಲಿ ಯಾವ ಮಹಿಮೆಯಿದೆ? ನಾಯಿಯ ಮುಂದೆ ನಿಜವಾದ ಒಂದು ಮಾಂಸದ ತುಂಡನ್ನು ಏತಕ್ಕೆ ಎಸೆಯಬಾರದು?

ಇಂದ್ರಿಯಾತೀತ ಆದರ್ಶವನ್ನು ರೂಪಿಸಲು ಹೊರಟ ಭಾರತೀಯನ ಸ್ವಭಾವವಾದರೊ ವಿಚಿತ್ರವಾದ ಚಿತ್ರಗಳನ್ನು ನಿರೂಪಿಸುವುದರಲ್ಲಿ ಅಧೋಗತಿಗಿಳಿದಿದೆ. ನಿಜವಾದ ಕಲೆಯನ್ನು ಒಂದು ಕಮಲ ಪುಷ್ಪಕ್ಕೆ ಹೋಲಿಸಬಹುದು. ಅದು ನೆಲದಿಂದ ಬರುವುದು, ನೆಲದಿಂದ ಸಾರವನ್ನು ಹೀರುವುದು, ನೆಲದೊಂದಿಗೆ ಸಂಪರ್ಕವನ್ನು ಪಡೆದಿರುವುದು, ಆದರೂ ಅದರಿಂದ ಮೇಲಿದೆ. ಇದರಂತೆಯೇ ಕಲೆಯು ಪ್ರಕೃತಿಯೊಂದಿಗೆ ಸಂಪರ್ಕವನ್ನು ಇಟ್ಟುಕೊಂಡಿರಬೇಕು. ಆ ಸಂಪರ್ಕ ಎಲ್ಲಿ ಇಲ್ಲವೊ ಅಲ್ಲಿ ಅವನತಿಗೆ ಬರುವುದು. ಆದರೂ ಅದು ಪ್ರಕೃತಿಯನ್ನು ಮೀರಿರಬೇಕು.

ಕಲೆ ಸೌಂದರ್ಯವನ್ನು ವ್ಯಕ್ತಗೊಳಿಸಬೇಕು. ಪ್ರತಿಯೊಂದರಲ್ಲಿಯೂ ಒಂದು ಕಲೆ\break ಇರಬೇಕು.

ಬರಿಯ ಕಟ್ಟಡಕ್ಕೂ ವಾಸ್ತುಶಿಲ್ಪಕ್ಕೂ ವ್ಯತ್ಯಾಸವಿದೆ. ಮೊದಲನೆಯದು ಕೇವಲ ಆರ್ಥಿಕ ದೃಷ್ಟಿಯಿಂದ ಕಟ್ಟಿದ ಕಟ್ಟಡ. ಎರಡನೆಯದು ಒಂದು ಭಾವನೆಯನ್ನು ವ್ಯಕ್ತಗೊಳಿಸುತ್ತದೆ. ಒಂದು ವಸ್ತುವಿನ ಮೌಲ್ಯವು ಅದು ಭಾವನೆಯನ್ನು ಮೂಡಿಸುವ ಸಾಮರ್ಥ್ಯದ ಮೇಲೆ ನಿಂತಿದೆ.

ನಮ್ಮ ಗುರುಗಳಾದ ಭಗವಾನ್​ ಶ‍್ರೀರಾಮಕೃಷ್ಣರಲ್ಲಿ ಕಲಾವಿದನ ಸ್ವಭಾವಪೂರ್ಣ ಪ್ರಕಾಶಕ್ಕೆ ಬಂದಿತ್ತು. ಕಲಾಕೌಶಲ್ಯ ಇಲ್ಲದೇ ಇದ್ದರೆ ಯಾರೂ ನಿಜವಾಗಿ ಆಧ್ಯಾತ್ಮಿಕ\break ಜೀವಿಗಳಾಗಲಾರರು ಎನ್ನುತ್ತಿದ್ದರು.

\vskip -0.5cm

