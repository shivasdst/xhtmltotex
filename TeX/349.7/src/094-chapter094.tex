
\chapter[ತರಗತಿಯ ಟಿಪ್ಪಣಿಗಳು – ೨ ]{ತರಗತಿಯ ಟಿಪ್ಪಣಿಗಳು – ೨ \protect\footnote{\engfoot{C.W. Vol. VIII, P.407–15}}}

\centerline{\textbf{ಕ್ರಿಸ್ತ ಮತ್ತು ಬರುವುದೆಂದು?}}

ನಾನು ವಿವರಣೆಗಳಿಗೆ ಗಮನ ಅಷ್ಟು ಕೊಡುವುದಿಲ್ಲ. ತತ್ತ್ವವನ್ನು ಮಾತ್ರ ಗಮನಿಸುವೆನು. ದೇವರು ಧರೆಗೆ ಪುನಃ ಪುನಃ ಬರುವನು ಎಂಬುದನ್ನು ನಾನು ಸಾರಬೇಕಾಗಿದೆ. ಹಿಂದೆ ಅವನು ರಾಮ ಕೃಷ್ಣ ಬುದ್ಧರಂತೆ ಬಂದ. ಅವನು ಮುಂದೆಯೂ ಬರುವನು. ಪ್ರತಿ ಐನೂರು ವರ್ಷಗಳಿಗೊಮ್ಮೆ ಪ್ರಪಂಚ ಅವನತಿಗೆ ಇಳಿಯುವುದು, ಆಗ ಒಂದು ಪ್ರಚಂಡವಾದ ಆಧ್ಯಾತ್ಮಿಕ ಅಲೆ ಮೇಲೇಳುವುದು. ಆ ಅಲೆಯ ತುತ್ತ ತುದಿಯಲ್ಲಿ ಒಬ್ಬ ಕ್ರಿಸ್ತನಿರುತ್ತಾನೆ. ಇದನ್ನು ಬೇಕಾದರೆ ನಾವು ಪ್ರತ್ಯಕ್ಷವಾಗಿ ತೋರಿಸಬಹುದು.

ಈಗ ಪ್ರಪಂಚದಲ್ಲೆಲ್ಲಾ ದೊಡ್ಡದೊಂದು ಬದಲಾವಣೆ ಆಗುತ್ತಿದೆ. ಇದೊಂದು ಕಲ್ಪ. ಮಾನವರಿಗೆ ಜೀವನದಲ್ಲಿ ನೆಚ್ಚುಗೆಡುತ್ತಿದೆ. ಈಗ ಅವರು ಮೇಲೆ ನೋಡಬೇಕೋ, ಕೆಳಗೆ ನೋಡಬೇಕೋ? ನಿಜವಾಗಿ ಮೇಲೆ ನೋಡಬೇಕು. ಕೆಳಗೆ ಹೇಗೆ ನೋಡುವುದಕ್ಕೆ ಆಗುವುದು? ಬಿರುಕು ಬಿಟ್ಟಿರುವ ಕಡೆ ನುಗ್ಗಿ ನಿಮ್ಮ ದೇಹದಿಂದ ನಿಮ್ಮ ಜೀವವನ್ನು ಧಾರೆಯಾದರೂ ಎರೆದು ಬಿರುಕನ್ನು ಮುಚ್ಚಿ. ನೀವು ಬದುಕಿರುವಾಗ ಪ್ರಪಂಚ ಹಾಳಾಗುವುದನ್ನು ಹೇಗೆ ಸಹಿಸುವಿರಿ?

\begin{center}
\textbf{ಮಾನವನಿಗೂ ಕ್ರಿಸ್ತನಿಗೂ ಇರುವ ವತ್ಯಾಸ}
\end{center}

ಆವಿರ್ಭಾವದ ದೃಷ್ಟಿಯಿಂದ ಒಂದು ವಸ್ತುವಿಗೂ ಮತ್ತೊಂದು ವಸ್ತುವಿಗೂ ವ್ಯತ್ಯಾಸವಿದೆ. ಆವಿರ್ಭಾವದ ದೃಷ್ಟಿಯಿಂದ ನೀನು ಎಂದಿಗೂ ಕ್ರಿಸ್ತನಾಗಲಾರೆ. ಜೇಡಿಮಣ್ಣಿನಿಂದ ಒಂದು ಇಲಿಯನ್ನು ಮಾಡುವರು; ಅದರಿಂದಲೇ ಮತ್ತೊಂದು ಅನೆಯನ್ನೂ ಮಾಡುವರು. ಅವೆರಡನ್ನೂ ನೀರಿಗೆ ಹಾಕಿದರೆ ಕರಗಿದ ಮೇಲೆ ಒಂದೇ ಆಗುವುದು. ಜೇಡಿಮಣ್ಣಿನ ದೃಷ್ಟಿಯಿಂದ ಆನೆ ಇಲಿ ಎರಡೂ ಒಂದೇ, ಆದರೆ ಆಕಾರದ ದೃಷ್ಟಿಯಿಂದ ಅವು ಎಂದೆಂದಿಗೂ ಬೇರೆ ಬೇರೆ. ದೇವ ಮತ್ತು ಮಾನವ ಇಬ್ಬರೂ ಆಗಿರುವುದೇ ಬ್ರಹ್ಮದಿಂದ, ಬ್ರಹ್ಮದ ದೃಷ್ಟಿಯಿಂದ ನಾವೆಲ್ಲಾ ಒಂದು. ಆದರೆ ವ್ಯಕ್ತಿಯ ದೃಷ್ಟಿಯಿಂದ, ಆವಿರ್ಭಾವದ ದೃಷ್ಟಿಯಿಂದ ದೇವರು ಎಂದೆಂದಿಗೂ ನಮಗೆ ಸ್ವಾಮಿ, ನಾವು ಎಂದೆಂದಿಗೂ ಅವನ ದಾಸರು.

ನಿಮ್ಮಲ್ಲಿ ಮೂರು ವಸ್ತುಗಳಿವೆ. ಅವೇ ದೇಹ ಮನಸ್ಸು ಮತ್ತು ಆತ್ಮ. ಆತ್ಮ ಅಗೋಚರ, ಮನಸ್ಸು ಮತ್ತು ದೇಹ ಬಂದು ಹೋಗುವುವು. ನೀವು ಆತ್ಮ. ಆದರೆ ಅನೇಕ ವೇಳೆ ದೇಹ ಎಂದು ಭಾವಿಸುವಿರಿ. ವ್ಯಕ್ತಿಯು ‘ನಾನು ಇಲ್ಲಿರುವೆನು’ ಎಂದಾಗ ತಾನು ದೇಹ ಎಂದು ಭಾವಿಸುವನು. ನೀವು ಅತ್ಯುನ್ನತ ಸ್ತರದಲ್ಲಿರುವ ಒಂದು ಸಮಯ ಬರುವುದು. ಆಗ ‘ನಾನು ಇಲ್ಲಿರುವೆನು’ ಎಂದು ನೀವು ಎನ್ನುವುದಿಲ್ಲ. ಒಬ್ಬನು ನಿಮ್ಮನ್ನು ನಿಂದಿಸಿದರೆ, ಶಾಪಕೊಟ್ಟರೆ ಆಗಲೂ ಅನುದ್ವಿಗ್ನನಾಗಿದ್ದರೆ ನೀವು ಆತ್ಮ. “ನಾನು ಮನಸ್ಸು ಎಂದು ಭಾವಿಸಿದಾಗ ನಿನ್ನಂತೆ ನಾನೂ ಒಂದು ಅನಂತಾತ್ಮನ ಕಿಡಿ. ನಾನು ಆತ್ಮ ಎಂದು ಭಾವಿಸಿದಾಗ ನಾನೂ ನೀನೂ ಇಬ್ಬರು ಒಂದೇ.” ಹೀಗೆನ್ನುತ್ತಾನೆ ದೇವರನ್ನು ಉದ್ದೇಶಿಸಿ ಒಬ್ಬ ಭಕ್ತ. ಮನಸ್ಸು ಆತ್ಮನಿಗಿಂತ ಮುಂದೆ ಹೋಗಬಲ್ಲದೆ?

ದೇವರು ತರ್ಕಿಸುವುದಿಲ್ಲ. ನಿಮಗೆ ಗೊತ್ತಿದ್ದರೆ ಏತಕ್ಕೆ ತರ್ಕ ಮಾಡಬೇಕು? ಅಲ್ಲೊಂದು ಇಲ್ಲೊಂದು ಘಟನೆಗಳನ್ನು ಆಯ್ದು ಅವನ್ನೆಲ್ಲಾ ಜೋಡಿಸಿ ಒಂದು ನಿಯಮವನ್ನು ಕಂಡುಹಿಡಿಯುವುದು ದೌರ್ಬಲ್ಯದ ಚಿಹ್ನೆ. ಅದು ಹುಳುವಿನಂತೆ ತೆವಳುತ್ತಿರುವ ಸ್ಥಿತಿ. ಆಮೇಲೆ ಎಲ್ಲವೂ ಕುಸಿದು ಬೀಳುತ್ತದೆ. ಆತ್ಮವು ಮನಸ್ಸಿನಲ್ಲಿ ಮತ್ತು ಎಲ್ಲದರಲ್ಲೂ ಪ್ರತಿಬಿಂಬಿತವಾಗಿದೆ. ಆತ್ಮಚೈತನ್ಯದಿಂದ ಮನಸ್ಸು ಚೇತನಾತ್ಮಕವಾಗುತ್ತದೆ. ಎಲ್ಲವೂ ಆತ್ಮನ ಒಂದು ಅಭಿವ್ಯಕ್ತಿ. ಮನಸ್ಸುಗಳೆಲ್ಲ ಹಲವು ಕನ್ನಡಿಗಳಂತೆ, ಪ್ರೀತಿ ದ್ವೇಷ ಅಂಜಿಕೆ ಪಾಪ ಪುಣ್ಯ ಎಂಬುವುಗಳೆಲ್ಲ ಆತ್ಮನ ಪ್ರತಿಬಿಂಬಗಳು. ಕನ್ನಡಿ ಯಾವಾಗ ಕೊಳೆಯಿಂದ ಕೂಡಿರುವುದೋ ಆಗ ಪ್ರತಿಬಿಂಬ ಸ್ಪಷ್ಟವಾಗಿರುವುದಿಲ್ಲ.

\begin{center}
\textbf{ಕ್ರಿಸ್ತ ಮತ್ತು ಬುದ್ಧರು ಒಂದೇ ಏನು?}
\end{center}

ಬುದ್ಧನೇ ಕ್ರಿಸ್ತನಂತೆ ಬಂದನು ಎಂಬುದು ನನ್ನ ಒಂದು ವಿಚಿತ್ರ ಕಲ್ಪನೆ. ಬುದ್ಧ ‘ಇನ್ನು ಐನೂರು ವರ್ಷಗಳಾದ ಮೇಲೆ ನಾನು ಬರುತ್ತೇನೆ’ ಎಂದು ಭವಿಷ್ಯ ನುಡಿದ. ಕ್ರಿಸ್ತ ಐನೂರು ವರ್ಷಗಳಾದ ಮೇಲೆ ಬಂದ. ಅವರಿಬ್ಬರೂ ಮಾನವತೆಯ ಮಹಾಜ್ಯೋತಿಗಳು. ಬುದ್ಧ ಮತ್ತು ಕ್ರಿಸ್ತ ಎಂಬ ಇಬ್ಬರು ಮಹಾಪುರುಷರಿಗೆ ಜಗತ್ತು ಜನ್ಮವನ್ನು ನೀಡಿದೆ. ಇವರಿಬ್ಬರೂ ಮಹಾವ್ಯಕ್ತಿಗಳು, ಭೀಮ ವ್ಯಕ್ತಿಗಳು, ಇಬ್ಬರೂ, ದೇವತೆಗಳು. ಇಡೀ ಪ್ರಪಂಚವನ್ನು ಇವರು ತಮ್ಮೊಳಗೆ ಪಾಲುಮಾಡಿಕೊಂಡಿರುವರು. ಪ್ರಪಂಚದಲ್ಲಿ ಎಲ್ಲಿಯಾದರೂ ಸ್ವಲ್ಪ ಜ್ಞಾನವಿದ್ದರೂ ಜನರೆಲ್ಲ ಬುದ್ಧನಿಗೋ ಕ್ರಿಸ್ತನಿಗೋ ಮಣಿಯುವರು. ಇಂತಹ ಹೆಚ್ಚು ವ್ಯಕ್ತಿಗಳನ್ನು ಜಗತ್ತಿಗೆ ತರುವುದು ಬಹಳ ಕಷ್ಟ. ಆದರೂ ಇಂತಹ ಮಹಾಮಹಿಮರು ಜಗತ್ತಿಗೆ ಹೆಚ್ಚು ಹೆಚ್ಚಾಗಿ ಬರುವರೆಂದು ನಾನು ಭಾವಿಸುತ್ತೇನೆ. ಕ್ರಿಸ್ತ ಬಂದು ಐನೂರು ವರುಷಗಳಾದ ಮೇಲೆ ಮಹಮ್ಮದನು ಬಂದನು. ಅವನು ಬಂದು ಐನೂರು ವರುಷಗಳಾದ ಮೇಲೆ ಪ್ರಾಟೆಸ್ಟೆಂಟರ ನಾಯಕನಾದ ಲೂಥರ್​ ಬಂದನು. ಅವನಾದಮೇಲೆ ಐದುನೂರು ವರ್ಷಗಳು ಕಳೆದಿವೆ.ಕೆಲವು ಸಾವಿರ ವರ್ಷಗಳಲ್ಲಿ ಬುದ್ಧ ಮತ್ತು ಏಸು ಎಂಬ ಎರಡು ವ್ಯಕ್ತಿಗಳನ್ನು ಜಗತ್ತು ಸೃಷ್ಟಿಸಿರುವುದು ಒಂದು ಅಪೂರ್ವ ಘಟನೆ. ಇಂತಹ ಇಬ್ಬರೇ ಪ್ರಪಂಚಕ್ಕೆ ಸಾಲದೆ? ಕ್ರಿಸ್ತ ಮತ್ತು ಬುದ್ಧರು ಸಾಕ್ಷಾತ್​ ದೇವರೇ ಆಗಿದ್ದರು. ಇತರರು ದೇವದೂತರು. ಇವರಿಬ್ಬರ ಜೀವನವನ್ನು ತಿಳಿದುಕೊಂಡು ಪ್ರಪಂಚದಲ್ಲಿ ಯಾವ ಒಂದು ಶಕ್ತಿ ಪ್ರವಾಹವನ್ನು ಅವರು ಹರಿಸಿರುವರು ಎಂಬುದನ್ನು ನೋಡಿ. ಶಾಂತಾತ್ಮರು, ಅಹಿಂಸಾ ವ್ರತಧಾರಿಗಳು, ಕುರುಡು ಕಾಸಿಲ್ಲದ ಭಿಕಾರಿಗಳು ಅವರು. ಜನರು ಅವರನ್ನು ಮೂಢರೆಂದು ಪಾಷಂಡಿ ಗಳೆಂದು ಕರೆಯುತ್ತಿದ್ದರು. ಆದರೆ ಇಡೀ ಮಾನವಕೋಟಿಯ ಮೇಲೆ ಅವರು ಬೀರಿದ ಪ್ರಚಂಡ ಆಧ್ಯಾತ್ಮಿಕ ಪ್ರಭಾವವನ್ನು ನೋಡಿ.

\begin{center}
\textbf{ಪಾಪವಿಮೋಚನೆ}
\end{center}

ನಾವು ಪಾಪದಿಂದ ಪಾರಾಗಬೇಕಾದರೆ ಅಜ್ಞಾನದಿಂದ ಪಾರಾಗಬೇಕು. ಪಾಪಕ್ಕೆ ಅಜ್ಞಾನವೇ ಕಾರಣ.

\begin{center}
\textbf{ಜಗನ್ಮಾತೆ}
\end{center}

ದಾದಿ ಮಗುವನ್ನು ಆಟಕ್ಕೆ ಹೊರಗೆ ಕರೆದುಕೊಂಡು ಹೋದಾಗ ತಾಯಿಯು ಯಾರೊಡನೆಯೊ ಮಗುವಿಗೆ ಬರಬೇಕೆಂದು ಹೇಳಿಕಳುಹಿಸುವಳು. ಮಗು ಆಟದಲ್ಲಿ ಮಗ್ನವಾಗಿರುವುದು. ಮಗು ಆಗ ‘ನಾನು ಈಗ ಬರುವುದಿಲ್ಲ. ನಾನು ಊಟಮಾಡುವುದಿಲ್ಲ’ ಎನ್ನುವುದು. ಸ್ವಲ್ಪ ಹೊತ್ತಾದ ಮೇಲೆ ಆಟ ಬೇಜಾರಾಗಿ ಮಗು ದಾದಿಗೆ ‘ಅಮ್ಮನ ಹತ್ತಿರ ಹೋಗಬೇಕು’ ಎನ್ನುವುದು. ಆಗ ದಾದಿ, ‘ನೋಡು, ಇಲ್ಲೊಂದು ಹೊಸ ಆಟದ ಸಾಮಾನಿದೆ’ ಎನ್ನುವಳು. ಮಗು ಆಗ ‘ನನಗೆ ಇನ್ನು ಯಾವ ಆಟದ ಸಾಮಾನೂ ಬೇಕಾಗಿಲ್ಲ, ನಾನು ಅಮ್ಮನ ಹತ್ತಿರ ಹೋಗಬೇಕು’ ಎನ್ನುವುದು. ಮಗು ಅಮ್ಮನ ಹತ್ತಿರ ಹೋಗುವವರೆಗೆ ಅಳುವುದಕ್ಕೆ ಪ್ರಾರಂಭಿಸುವುದು. ನಾವೆಲ್ಲ ಮಕ್ಕಳು. ದೇವರೇ ತಾಯಿ. ನಾವು ಹಣ ಕೀರ್ತಿ ಮುಂತಾದುವನ್ನು ಪಡೆಯುವುದರಲ್ಲಿ ಮಗ್ನರಾಗಿರುವೆವು. ಆದರೆ ನಾವು ಜಾಗೃತರಾಗುವ ಒಂದು ಕಾಲ ಬರುವುದು. ಆಗ ಪ್ರಪಂಚ ನಮಗೆ ಹೆಚ್ಚು ಆಟದ ಗೊಂಬೆಗಳನ್ನು ಕೊಡುವುದು. ನನಗೆ ಇವೆಲ್ಲ ಬೇಜಾರಾಗಿವೆ; ನಾನು ದೇವರ ಬಳಿಗೆ ಹೋಗುತ್ತೇನೆ ಎನ್ನುವೆವು.

\begin{center}
\textbf{ಭಗವಂತನ್ನುಳಿದು ವೇಕ್ತ್ತಿತ್ವವಿಲ್ಲ}
\end{center}

ನಾವು ದೇವರಿಂದ ಬೇರೆ ಆಗಿಲ್ಲದೆ ಇದ್ದರೆ, ಯಾವಾಗಲೂ ಅವನಲ್ಲಿ ಒಂದಾಗಿದ್ದರೆ ನಮಗೆ ಒಂದು ವ್ಯಕ್ತಿತ್ವವಿಲ್ಲವೆ? ಇದೆ, ಆ ವ್ಯಕ್ತಿತ್ವವೇ ದೇವರು. ಈಗಿರುವುದು ನಿಜವಾದ ವ್ಯಕ್ತಿತ್ವವಲ್ಲ. ನೀವು ನಿಮ್ಮ ನಿಜವಾದ ವ್ಯಕ್ತಿತ್ವವನ್ನು ಸಮೀಪಿಸುತ್ತಿರುವಿರಿ. ವ್ಯಕ್ತಿತ್ವ ಎಂದರೆ ಯಾವುದನ್ನು ನಾವು ವಿಭಾಗಿಸಲಾರೆವೋ ಅದು. ನಾವು ಈಗ ಇರುವ ಸ್ಥಿತಿಯನ್ನು ವ್ಯಕ್ತಿತ್ವ ಎಂದು ಕರೆಯುವುದು ಹೇಗೆ? ಒಂದು ಗಂಟೆ ಒಂದು ಬಗೆ ಯೋಚಿಸುತ್ತೀರಿ, ಇನ್ನೊಂದು ಗಂಟೆ ಇನ್ನೊಂದು ಬಗೆ, ಎರಡುಗಂಟೆಗಳ ಅನಂತರ ಮತ್ತೆ ಒಂದು ರೀತಿ. ಯಾವುದು ಬದಲಾಗುವು ದಿಲ್ಲವೋ ಅದೇ ವ್ಯಕ್ತಿತ್ವ. ನಾವು ಈಗ ಇರುವ ಸ್ಥಿತಿಯಲ್ಲಿಯೇ ಎಂದೆಂದಿಗೂ ಇರುವುದು ಬಹಳ ಅಪಾಯಕರ. ಹೀಗಾದರೆ ಕಳ್ಳ ಯಾವಾಗಲೂ ಕಳ್ಳನಾಗಿಯೇ ಉಳಿಯಬೇಕಾಗುವುದು. ಮೋಸಗಾರ ಯಾವಾಗಲೂ ಮೋಸಗಾರನೇ ಆಗಿರಬೇಕಾಗುವುದು. ಮಗು ಎಂದೆಂದಿಗೂ ಮಗುವಾಗಿಯೇ ಉಳಿದಿರ ಬೇಕಾಗುವುದು. ನಿಜವಾದ ವ್ಯಕ್ತಿತ್ವ ಎಂದಿಗೂ ಬದಲಾಗುವುದಿಲ್ಲ. ಅದೇ ನಮ್ಮೊಳಗಿನ ದೇವರು.

