
\chapter[ಮಾತೃ ಪೂಜೆ ]{ಮಾತೃ ಪೂಜೆ \protect\footnote{\engfoot{C.W. Vol. VI. P. 145}}}

ನಾವು ಎಂದಿಗೂ ಪಾರಾಗಲಾರದ ಅನುಭವಗಳು ಯಾವುವೆಂದರೆ ಸುಖ ಮತ್ತು ದುಃಖ. ಯಾವುದು ನಮಗೆ ದುಃಖವನ್ನು ತರುವುದೋ ಅದೇ ನಮಗೆ ಸುಖವನ್ನೂ ತರುವುದು. ನಮ್ಮ ಜಗತ್ತು ಇವೆರಡರಿಂದ ಆಗಿದೆ. ನಾವು ಇವುಗಳಿಂದ ಪಾರಾಗಲಾರೆವು. ಪ್ರತಿಕ್ಷಣದಲ್ಲಿಯೂ ಅವು ಇರುವುವು. ಈ ದ್ವಂದ್ವಗಳೊಳಗೆ ರಾಜಿ ಮಾಡಲು ಪ್ರಪಂಚ ಸದಾ ಯತ್ನಿಸುತ್ತಿದೆ. ಋಷಿಗಳು ಈ ದ್ವಂದ್ವವನ್ನು ಬಗೆಹರಿಸಲು ಯತ್ನಿಸುವರು. ಸಹಿಸಲಾರದ ನೋವಿನ ಮಧ್ಯೆ ಸ್ವಲ್ಪ ವಿಶ್ರಾಂತಿ, ಗಾಢಾಂಧಕಾರದ ಮಧ್ಯೆ ಮಿಂಚು–ಇವು ಬೆಳಕನ್ನು ಕೊಡುವ ಬದಲು ಗಾಢಾಂಧಕಾರವನ್ನು ಹೆಚ್ಚಿಸುವುವು.

ಮಕ್ಕಳು ಹುಟ್ಟು ಆಶಾವಾದಿಗಳು. ಆದರೆ ಅವರ ಮುಂದಿನ ಜೀವನ ನಿರಂತರವಾಗಿ ಆ ಭ್ರಾಂತಿಯನ್ನು ಹೋಗಲಾಡಿಸುವುದು. ಯಾವ ಒಂದು ಆದರ್ಶವನ್ನೂ ಪೂರ್ಣವಾಗಿ ಪಡೆಯುವುದಕ್ಕೆ ಆಗುವುದಿಲ್ಲ. ಯಾವ ಒಂದು ಆಸೆಯನ್ನೂ ತೃಪ್ತಿಪಡಿಸಲಾಗುವುದಿಲ್ಲ. ಆದುದರಿಂದ ಸಮಸ್ಯೆಯನ್ನು ಪರಿಹರಿಸಲು ಅವರು ಯತ್ನಿಸುತ್ತಲೆ ಇರುವರು. ಧರ್ಮ ಈ ಸಮಸ್ಯೆಯನ್ನು ಪರಿಹರಿಸಲು ಪ್ರಯತ್ನಿಸಿದೆ.

ದ್ವೈತಧರ್ಮಗಳಲ್ಲಿ, ಪಾರ್ಸಿಯವರಲ್ಲಿ ದೇವರು ಮತ್ತು ಸೈತಾನ್​ ಇಬ್ಬರು ಇರುವರು. ಈ ಭಾವನೆ ಯಹೂದ್ಯರ ಮೂಲಕ ಯೂರೋಪ್​ ಅಮೆರಿಕ ದೇಶಗಳಲ್ಲೆಲ್ಲ ಹರಡಿದೆ. ಸಾವಿರಾರು ವರುಷಗಳ ಹಿಂದೆ ಇದು ಅನುಷ್ಠಾನ ಸಾಧ್ಯವಾದ ಒಂದು ಊಹೆಯಾಗಿತ್ತು. ಆದರೆ ಈಗ ತರ್ಕಕ್ಕೆ ಒಪ್ಪಿಗೆಯಾಗುವಂಥದಲ್ಲ ಎಂಬುದು ಗೊತ್ತಾಗಿದೆ. ಬರಿಯ ಒಳ್ಳೆಯದ್ದಾಗಲಿ ಕೆಟ್ಟದ್ದಾಗಲಿ ಇಲ್ಲ. ಒಬ್ಬನಿಗೆ ಒಂದು ಒಳ್ಳೆಯದಾದರೆ ಅದೇ ಮತ್ತೊಬ್ಬನಿಗೆ ಕೆಟ್ಟದ್ದು. ಇಂದು ಕೆಟ್ಟದ್ದಾಗಿರುವಂತಹದು ನಾಳೆ ಒಳ್ಳೆಯದಾಗುವುದು ಹೀಗೆಯೇ.

ದೇವರು ಮೊದಲು ಒಂದು ಬುಡಕಟ್ಟಿನ ದೇವರಾಗಿದ್ದನು. ಅನಂತರ ಅವನು ದೇವ\-ದೇವನಾಗುವನು. ಪುರಾತನ ಈಜಿಪ್ಟಿನವರು ಮತ್ತು ಬ್ಯಾಬಿಲೋನಿಯದವರು ದೇವರು ಮತ್ತು ಸೈತಾನ್​ ಎಂಬ ಭಾವನೆಯನ್ನು ನಂಬುತ್ತಾ ಹೋದರು. ಅವರ ಮೋಲಾಕ್​ ದೇವರ ದೇವನಾದನು, ಸೆರೆಹಿಡಿಯಲ್ಪಟ್ಟ ಇತರ ದೇವತೆಗಳು ಮೊಲಾಕಿಗೆ ಅಡ್ಡಬೀಳಬೇಕಾಯಿತು.

ಆದರೂ ಈ ಸಮಸ್ಯೆ ಹಾಗೇ ಉಳಿಯುವುದು. ಕೆಟ್ಟದ್ದರ ಅಧಿದೇವತೆ ಯಾರು? ಒಳ್ಳೆಯದು ಒಂದೇ ಇರುವುದು, ಕೆಟ್ಟದೆಂಬುದು ಇಲ್ಲವೇ ಇಲ್ಲ, ನಮಗೆ ಮಾತ್ರ ಅದು ಅರ್ಥವಾಗುವುದಿಲ್ಲ ಎಂದು ಅನೇಕರು ಭಾವಿಸುತ್ತಾರೆ. ನಮಗೆ ಅನನುಕೂಲವಾದುದನ್ನು ನೋಡದೆ ಕೆಲಸಕ್ಕೆ ಬಾರದುದನ್ನು ಅಪ್ಪಿಕೊಂಡು, ಅದು ನಮ್ಮನ್ನು ರಕ್ಷಿಸುವುದೆಂದು ಬಗೆದಿರುವೆವು. ಆದರೂ ನಾವೆಲ್ಲಾ ನೀತಿಯನ್ನು ಅನುಸರಿಸುತ್ತೇವೆ. ನೀತಿಯಸಾರ ತ್ಯಾಗ, ನಾನಲ್ಲ ನೀನು ಎಂಬುದು. ಪ್ರಪಂಚದ ‘ಒಳ್ಳೆಯವನಾದ ದೇವರು’ ಎಂಬ ಭಾವನೆ ನೀತಿಭಾವನೆಗೆ ಎಷ್ಟು ವಿರೋಧವಾಗಿದೆ! ಆ ದೇವರು ಎಷ್ಟೊಂದು ಸ್ವಾರ್ಥಿ, ಸೇಡಿನ ಮನೋಭಾವದವನು. ಪ್ಲೇಗು, ಬರಗಾಲ, ಯುದ್ಧ ಇವನ್ನು ತನ್ನ ಕೈಗಳಲ್ಲಿಯೇ ಹಿಡಿದುಕೊಂಡಿರುವನು.

ನಮಗೆಲ್ಲ ಈ ಜೀವನದಲ್ಲಿ ಅನುಭವ ಬೇಕಾಗಿದೆ. ನಾವು ಕಹಿ ಅನುಭವಗಳಿಂದ ಸಾಧ್ಯವಾದಷ್ಟು ತಪ್ಪಿಸಿಕೊಳ್ಳಲು ಯತ್ನಿಸುವೆವು. ಆದರೆ ಅವು ಈಗಲೋ ಅನಂತರವೋ ನಮ್ಮನ್ನು ಹಿಡಿದುಕೊಳ್ಳುತ್ತವೆ. ಇಡೀ ಜೀವನವನ್ನು, ಅಂದರೆ ಸಿಹಿಕಹಿ ಅನುಭವಗಳೆರಡನ್ನೂ ಎದುರಿಸಲಾಗದವನಿಗಾಗಿ ನಾನು ಮರುಗಬೇಕಾಗಿದೆ.

ವೇದಗಳಲ್ಲಿ ಬರುವ ಮನು, ಪಾರ್ಸಿಯವರ ಧರ್ಮದಲ್ಲಿ ಅಹ್ರಿಮಾನ್​ ಆಗುವನು. ಪೌರಾಣಿಕ ವಿವರಣೆ ಏನೋ ತಣ್ಣಗಾಯಿತು; ಆದರೆ ಆ ಪ್ರಶ್ನೆ ಇನ್ನೂ ಉಳಿದಿದೆ. ಅದಕ್ಕೆ ಉತ್ತರವೇ ಇಲ್ಲ, ಸಮಾಧಾನವೇ ಇಲ್ಲ.

ವೇದಮಂತ್ರಗಳಲ್ಲಿ ದೇವತೆಗೆ ಸಂಬಂಧಪಟ್ಟ ಮತ್ತೊಂದು ಭಾವವಿದೆ. “ನಾನೇ ಜ್ಯೋತಿ, ಸೂರ್ಯಚಂದ್ರರಲ್ಲಿರುವ ಜ್ಯೋತಿ; ಎಲ್ಲಾ ಜೀವಿಗಳಿಗೂ ಪ್ರಾಣ ಕೊಡುವ ಗಾಳಿಯೇ ನಾನು.” ಈ ಭಾವನಾಂಕುರವೇ ಅನಂತರ ಮಾತೃಪೂಜೆಯ ಆದರ್ಶವಾಗುವುದು. ಮಾತೃಪೂಜೆ ಎಂದರೆ ತಂದೆತಾಯಿಗಳಲ್ಲಿ ವ್ಯತ್ಯಾಸವನ್ನು ಕಾಣುವುದು ಎಂದಲ್ಲ. ಈ ಪದವು ಸೂಚಿಸುವ ಪ್ರಥಮ ಭಾವನೆಯೇ ಶಕ್ತಿ– ಎಲ್ಲಾ ಜೀವಿಗಳಲ್ಲಿರುವ ಶಕ್ತಿಯೇ ನಾನು ಎನ್ನುವುದು.

ಮಗುವಿನಲ್ಲಿ ಕೇವಲ ಉದ್ವೇಗ ಮಾತ್ರ ಇದೆ. ಅದು ಬೆಳೆಯುತ್ತ ಹೋದಂತೆ ಅಧಿಕಾರ\-ವನ್ನು ಪಡೆದ ಮನುಷ್ಯನಾಗುವುದು. ಆದಿಯಲ್ಲಿ ಒಳ್ಳೆಯದು ಕೆಟ್ಟದ್ದು ಪ್ರತ್ಯೇಕವಾಗಿ ಬೆಳೆದು ಬರಲಿಲ್ಲ. ಅತಿ ಮುಖ್ಯವಾದ ಭಾವನೆಯೇ ಶಕ್ತಿ ಎಂಬುದನ್ನು ಮುಂದುವರಿದ\break ಮನಸ್ಸು ಅರಿಯಿತು. ಹೆಜ್ಜೆ ಹೆಜ್ಜೆಗೆ ಆತಂಕ ಮತ್ತು ಹೋರಾಟವೇ ನಿಯಮ. ನಾವು\break ಶಕ್ತಿ ಮತ್ತು ಅದರ ಪ್ರತಿರೋಧ ಇವುಗಳ ಒಟ್ಟು ಪರಿಣಾಮ. ಆಂತರಿಕ ಮತ್ತು ಬಾಹ್ಯ\break ಶಕ್ತಿಗಳೆರಡಕ್ಕೂ ಇದು ಅನ್ವಯಿಸುವುದು. ಮನಸ್ಸಿನಲ್ಲಿ ಆಗುತ್ತಿರುವ ಪ್ರತಿಯೊಂದು\break ಆಲೋಚನೆಯನ್ನೂ ದೇಹದ ಪ್ರತಿಯೊಂದು ಕಣವೂ ತಡೆಯುತ್ತಿದೆ. ನಮಗೆ ಕಾಣಿಸುವ ಮತ್ತು ತಿಳಿದಿರುವ ಪ್ರತಿಯೊಂದು ವಸ್ತುವೂ ಈ ಎರಡು ಶಕ್ತಿಗಳ ಪ್ರತಿಕ್ರಿಯೆಯೇ ಆಗಿದೆ.

ಈ ದೇವರ ಭಾವನೆ ಹೊಸದು. ವೇದಮಂತ್ರಗಳಲ್ಲಿ ವರುಣ ಮತ್ತು ಇಂದ್ರರು ತಮ್ಮ ಭಕ್ತರಿಗೆ ಒಳ್ಳೆಯ ವರಗಳನ್ನು ಕೊಡುತ್ತಿದ್ದರು; ಚೆನ್ನಾಗಿ ಆಶೀರ್ವಾದ ಮಾಡುತ್ತಿದ್ದರು. ಇದು ಮಾನವಸಹಜವಾದ ಭಾವನೆ, ಮಾನವನಿಗಿಂತ ಹೆಚ್ಚು ಮಾನವೀಯವಾದದ್ದು ಇದು.

ಇದೇ ಹೊಸ ತತ್ತ್ವ. ಈ ಸೃಷ್ಟಿಯ ಹಿಂದೆಲ್ಲ ಒಂದು ಶಕ್ತಿ ಇದೆ. ಶಕ್ತಿ ಎಲ್ಲಾ ಕಡೆಗಳ\-ಲ್ಲಿಯೂ ಶಕ್ತಿಯೇ; ಅದು ಪಾಪದಂತೆ ಇರಲಿ, ಪ್ರಪಂಚವನ್ನು ಉದ್ಧಾರ ಮಾಡುವ\break ಪುಣ್ಯದಂತೆ ಇರಲಿ. ಇದೇ ಹೊಸ ಭಾವನೆ. ಹಳೆಯದೆಂದರೆ ಮಾನವ ರೂಪಿನಲ್ಲಿ ಇರುವ ದೇವರ ಕಲ್ಪನೆ. ಇಲ್ಲೇ ಒಂದು ವಿಶ್ವಶಕ್ತಿಯ ಕಲ್ಪನೆಯ ಉಗಮವಿದೆ.

“ರುದ್ರನು ಪಾಪವನ್ನು ನಾಶಮಾಡಲು ಇಚ್ಛಿಸುವಾಗ, ಅವನ ಬಿಲ್ಲನ್ನು ನಾನು ಬಗ್ಗಿಸುವೆನು.” (ಋಗ್ವೇದ)

ಅನಂತರ ನಾವು ಗೀತೆಯಲ್ಲಿ ಇದನ್ನು ಓದುತ್ತೇವೆ: “ಓ ಅರ್ಜುನ, ನಾನೇ ಸತ್ಯ\break ಮತ್ತು ಅಸತ್ಯ, ನಾನೇ ಪಾಪ ಮತ್ತು ಪುಣ್ಯ. ಸಜ್ಜನರ ಸೌಜನ್ಯವು ನಾನೇ. ದುರ್ಜನರ ದೌರ್ಜನ್ಯವೂ ನಾನೇ.” ತಕ್ಷಣವೇ ಅವನು ಸತ್ಯವನ್ನು ಮುಚ್ಚಿಹಾಕಿ ಅದನ್ನು ಕಾಣಿಸದಂತೆ ಮಾಡುವನು. ಒಳ್ಳೆಯವರು ಎಲ್ಲಿಯವರೆಗೆ ಒಳ್ಳೆಯ ಕೆಲಸಗಳನ್ನು ಮಾಡುತ್ತಿರುವರೋ ಅಲ್ಲಿಯವರೆಗೆ ಅವರಲ್ಲಿರುವ ಶಕ್ತಿ ನಾನೇ ಎನ್ನುವುದು.

ಪಾರ್ಸಿ ಧರ್ಮದಲ್ಲಿ ಸೈತಾನನ ಭಾವನೆ ಇದೆ. ಆದರೆ ಇಂಡಿಯಾ ದೇಶದಲ್ಲಿ ಸೈತಾನನ ಭಾವನೆ ಇಲ್ಲ. ಅನಂತರ ಬಂದ ಗ್ರಂಥಗಳು ಈ ಹೊಸ ಭಾವನೆಯನ್ನು ಗ್ರಹಿಸತೊಡಗುವುವು. ಪ್ರಪಂಚದಲ್ಲಿ ಕೆಟ್ಟದ್ದು ಇದೆ. ಇದರಲ್ಲಿ ಸಂಶಯವಿಲ್ಲ. ವಿಶ್ವ ಎನ್ನುವುದು ಒಂದು ವಾಸ್ತವಾಂಶ. ಅದರಲ್ಲಿ ಒಳ್ಳೆಯದು ಕೆಟ್ಟದ್ದು ಎರಡೂ ಇವೆ. ಯಾರು ಆಳುತ್ತಿರುವರೋ ಅವರು ಒಳ್ಳೆಯದನ್ನು ಮತ್ತು ಕೆಟ್ಟದ್ದನ್ನು ಆಳಬೇಕಾಗಿದೆ. ಇದು ನಮ್ಮನ್ನು ಬದುಕುವಂತೆ ಮಾಡಿದರೆ ಸಾಯುವಂತೆ ಮಾಡುವುದು ಕೂಡ ಅದೇ. ಅಳು ನಗುಗಳೆರಡೂ ಅವಳಿ ಜವಳಿ. ಪ್ರಪಂಚದಲ್ಲಿ ನಗುವಿಗಿಂತ ಅಳುವೇ ಹೆಚ್ಚು. ಹೂವನ್ನು ಸೃಷ್ಟಿಸಿದವರು ಯಾರು? ಹಿಮಾಲಯವನ್ನು ಸೃಷ್ಟಿಸಿದವರು ಯಾರು? ಒಬ್ಬ ಬಹಳ ಒಳ್ಳೆಯ ದೇವರು. ನನ್ನ ಪಾಪಕ್ಕೆ ಮತ್ತು ದೌರ್ಬಲ್ಯಕ್ಕೆ ಕಾರಣ ಯಾರು? ಕರ್ಮ, ಸೈತಾನನೇ ಇದನ್ನೆಲ್ಲ ಮಾಡಿದ್ದು. ಹೀಗೆಂದರೆ ವಿಶ್ವ ಒಂದು ಕಾಲಿನ ಮೇಲೆ ನಿಲ್ಲಬೇಕಾಗುವುದು. ಇದರ ದೇವರೂ ಕೂಡ ಒಂದು ಕಾಲಿನ ದೇವರಾಗುವುದು.

ಒಳ್ಳೆಯದು ಕೆಟ್ಟದ್ದು ಸಂಪೂರ್ಣ ಬೇರೆ ಬೇರೆ ಎಂದು ಭಾವಿಸುವುದು ನಮ್ಮನ್ನು ಕ್ರೂರಿಗಳನ್ನಾಗಿ ಮಾಡುವುದು, ನಮ್ಮ ಹೃದಯದ ಅನುಕಂಪವೆಲ್ಲ ಬತ್ತಿ ಹೋಗುವಂತೆ ಮಾಡುವುದು. ಪತಿವ್ರತೆಯು ವೇಶ್ಯೆಯಿಂದ ದೂರ ಓಡುವಳು. ಏಕೆ? ಆ ವೇಶ್ಯೆ ಕೆಲವು ರೀತಿಯಲ್ಲಿ ಪತಿವ್ರತೆಗಿಂತ ಮೇಲಾಗಿರಬಹುದು. ಈ ಭಾವನೆ ಶಾಶ್ವತವಾದ ಅಸೂಯೆಗೆ ದ್ವೇಷಕ್ಕೆ ಕಾರಣ. ಒಬ್ಬನಿಗೂ ಮತ್ತೊಬ್ಬನಿಗೂ ದೊಡ್ಡ ಅಂತರವನ್ನು ಕಲ್ಪಿಸುವುದು. ಒಳ್ಳೆಯ ಮನುಷ್ಯನಿಗೂ ಅವನಿಗಿಂತ ಒಳ್ಳೆಯದರಲ್ಲಿ ಸ್ವಲ್ಪ ಕಡಿಮೆಯಾಗಿರುವವನಿಗೂ ಅಥವಾ ಪಾಪಿಗೂ ಒಂದು ದೊಡ್ಡ ಅಂತರವನ್ನು ಕಲ್ಪಿಸುವುದು. ಇಂತಹ ಕಠೋರವಾದ ಭಾವನೆಯು ಮಹಾಪಾತಕ; ಪಾತಕಕ್ಕಿಂತಲೂ ಪಾತಕ ಇದು. ಒಳ್ಳೆಯದು ಕೆಟ್ಟದ್ದು ಬೇರೆ ಬೇರೆ ಇಲ್ಲ. ಒಳ್ಳೆಯದರ ವಿಕಾಸ ಇದೆ. ಎಲ್ಲಿ ಒಳ್ಳೆಯದು ಕಡಮೆ ಇದೆಯೋ ಅದನ್ನು ನಾವು ಕೆಟ್ಟದ್ದು ಎನ್ನುತ್ತೇವೆ.

ಕೆಲವರು ಸಾಧುಗಳು; ಮತ್ತೆ ಕೆಲವರು ಪಾಪಿಗಳು. ಸೂರ್ಯ ಒಳ್ಳೆಯದು ಕೆಟ್ಟದ್ದು ಎರಡರ ಮೇಲೂ ಬೆಳಗುವನು. ಅವನೇನಾದರೂ ಪಕ್ಷಪಾತ ಮಾಡುವನೆ? ದೇವರನ್ನು ತಂದೆ ಎಂದು ಕರೆಯುವ ಹಳೆಯ ಭಾವನೆಯಲ್ಲಿ ಅವನು ಒಳ್ಳೆಯದಕ್ಕೆ ಮಾತ್ರ ಅಧಿಪತಿ ಎಂಬ ಭಾವನೆ ಇದೆ. ನಾವು ಸತ್ಯವನ್ನು ಮರೆಮಾಚಲು ಯತ್ನಿಸುವೆವು. ಪಾಪವೇ ಇಲ್ಲ ಎನ್ನುವೆವು. ನಾನೆಂಬುದೇ ಪಾಪ. ಆದರೆ ಈ ನಾನೆಂಬುದೇ ಹೆಚ್ಚಿಗೆ ಇರುವುದು. ಹಾಗಾದರೆ ‘ನಾನು’ ಇಲ್ಲವೆ? ಪ್ರತಿದಿನವೂ ನಾನೆಂಬುದಿಲ್ಲ ಎಂದು ಆಲೋಚಿಸಲು ಯತ್ನಿಸುವೆನು. ಆದರೂ ಸೋಲುವೆನು.

ಇವುಗಳೆಲ್ಲ ಪಾಪದಿಂದ ಪಾರಾಗುವುದಕ್ಕೆ ಮಾಡುವ ಯತ್ನ. ಆದರೆ ನಾವು ಅದನ್ನು ಎದುರಿಸಬೇಕಾಗಿದೆ. ನಾವು ಎಲ್ಲವನ್ನೂ ಎದುರಿಸಬೇಕಾಗಿದೆ. ಒಳ್ಳೆಯದು ಮತ್ತು ಸುಖ ಬಂದಾಗ ಮಾತ್ರ ನಾನು ದೇವರನ್ನು ಪ್ರೀತಿಸಬೇಕು, ದುಃಖಕಷ್ಟಗಳು ಬಂದಾಗ ಪ್ರೀತಿಸಕೂಡದು ಎಂಬ ಯಾವುದಾದರೂ ನಿಬಂಧನೆಗೆ ನಾನು ಒಳಪಟ್ಟಿರುವೆನೇನು?

ಒಂದು ದೀಪದ ಬೆಳಕಿನಲ್ಲಿ ಒಬ್ಬನು ಒಂದು ಕಾಗದಕ್ಕೆ ಕಳ್ಳರುಜು ಹಾಕುವನು. ಮತ್ತೊಬ್ಬ ಅದೇ ಬೆಳಕಿನಲ್ಲಿ ಬರಗಾರದ ಕೆಲಸಕ್ಕೆ ಒಂದು ಸಾವಿರ ಡಾಲರುಗಳಿಗೆ ಚೆಕ್ಕನ್ನು ಬರೆಯುವನು. ಆದರೆ ದೀಪದ ಬೆಳಕು ಇಬ್ಬರ ಮೇಲೂ ಒಂದೇ ಸಮನಾಗಿ ಬೀಳುತ್ತಿದೆ. ಬೆಳಕಿಗೆ ಪಾಪದ ಚಿಂತೆಯಿಲ್ಲ. ನಾವು ಮತ್ತು ನೀವು ಅದನ್ನು ಒಳ್ಳೆಯದು ಕೆಟ್ಟದ್ದು ಎಂದು ಭಾವಿಸುವುದು.

ಈ ಭಾವನೆಗೆ ಒಂದು ಹೊಸ ಹೆಸರು ಬೇಕಾಗುವುದು. ಇದನ್ನೇ ತಾಯಿ ಎನ್ನುವುದು. ಬಹಳ ಹಿಂದಿನ ಕಾಲದಲ್ಲಿ ಯಾರೋ ಒಬ್ಬಳು ಕವಯಿತ್ರಿಯನ್ನು ದೇವತೆಯ ಮಟ್ಟಕ್ಕೆ\break ಏರಿಸಿ ಅವಳಿಗೆ ಕೊಟ್ಟ ಹೆಸರು ಇದು. ಅನಂತರ ಸಾಂಖ್ಯರು ಬಂದರು. ಅವರಲ್ಲಿ\break ಶಕ್ತಿಯೆಲ್ಲ ಸ್ತ್ರೀಯೆ (ಪ್ರಕೃತಿ.) ಅಯಸ್ಕಾಂತ ಅಚಲವಾಗಿರುವುದು. ಕಬ್ಬಿಣದ ಚೂರುಗಳು ಚಲಿಸುವುವು.

ಭರತಖಂಡದಲ್ಲಿ ಸ್ತ್ರೀ ಆದರ್ಶಗಳಲ್ಲಿ ಶ್ರೇಷ್ಠತಮವಾದುದೇ ಮಾತೃತ್ವ. ಅವಳು ಹೆಂಡತಿಗಿಂತ ಮಿಗಿಲು. ಹೆಂಡತಿ ಮಕ್ಕಳು ಬೇಕಾದರೆ ಒಬ್ಬನನ್ನು ತ್ಯಜಿಸಬಹುದು. ಆದರೆ ಅವನ ತಾಯಿ ಅವನನ್ನು ಎಂದಿಗೂ ತ್ಯಜಿಸುವುದಿಲ್ಲ. ತಾಯಿ ಯಾವಾಗಲೂ ಒಂದೇ ತರಹವಾಗಿರುತ್ತಾಳೆ. ಅದೊಂದೇ ಅಲ್ಲ, ಮಗುವನ್ನು ಸ್ವಲ್ಪ ಹೆಚ್ಚಾಗಿಯೇ ಪ್ರೀತಿಸುವಳು. ತಾಯಿಯಲ್ಲಿ ಅಹೇತುಕ ಪ್ರೀತಿಯನ್ನು ನಾವು ನೋಡುತ್ತೇವೆ. ಅವಳ ಪ್ರೀತಿಯಲ್ಲಿ ಯಾವ ಪ್ರತಿಫಲಾಪೇಕ್ಷೆಯೂ ಇಲ್ಲ. ಅವಳ ಪ್ರೀತಿ ಎಂದಿಗೂ ನಾಶವಾಗುವುದಿಲ್ಲ. ಇಂತಹ\break ಪ್ರೇಮ ಯಾರಿಗೆ ಸಾಧ್ಯ? ತಾಯಿಗೆ ಮಾತ್ರ ಸಾಧ್ಯ. ಮಗನಿಗೂ ಇಲ್ಲ, ಮಗಳಿಗೂ ಇಲ್ಲ,\break ಹೆಂಡತಿಗೂ ಇಲ್ಲ.

“ಎಲ್ಲಾ ಕಡೆಗಳಲ್ಲಿಯೂ ವ್ಯಕ್ತವಾಗುತ್ತಿರುವ ಶಕ್ತಿಯೇ ನಾನು” ಎನ್ನುವಳು ತಾಯಿ. ಈ ಪ್ರಪಂಚವನ್ನು ಸೃಷ್ಟಿಸುವವಳು ಅವಳೇ, ಈ ಪ್ರಪಂಚವನ್ನು ನಾಶಮಾಡುವವಳೂ ಅವಳೇ. ವಿನಾಶವು ಸೃಷ್ಟಿಗೆ ಆದಿ ಎಂಬುದನ್ನು ಹೇಳಲೇಬೇಕಾಗಿಲ್ಲ. ಬೆಟ್ಟದ ತುದಿ ಎಂದರೆ ಅದು ಕಣಿವೆಯ ಪ್ರಾರಂಭವಾಗಿರುತ್ತದೆ.

ಧೀರರಾಗಿ, ಪ್ರಪಂಚ ಹೇಗಿದೆಯೊ ಹಾಗೆ ನೋಡಿ. ಪಾಪಕ್ಕೆ ಅಂಜಿ ಓಡಿ ಹೋಗಬೇಡಿ. ಪಾಪ ಪಾಪವೇ, ಆದರೇನಂತೆ?

ಅಂತೂ ಇದೆಲ್ಲ ತಾಯಿಯ ಲೀಲೆ. ಇದೇನೂ ಅಷ್ಟೊಂದು ಹದಗೆಟ್ಟಿಲ್ಲ. ಸರ್ವಶಕ್ತನನ್ನು ಯಾವುದು ಚಲಿಸುವಂತೆ ಮಾಡಬಲ್ಲದು? ಈ ಪ್ರಪಂಚವನ್ನು ಸೃಷ್ಟಿಸುವಂತೆ\break ಯಾರು ತಾಯಿಗೆ ಪ್ರೇರೇಪಿಸಿದ್ದು? ಅವಳಿಗೆ ಗುರಿ ಏನೂ ಇಲ್ಲ. ಯಾವುದನ್ನು ಇನ್ನೂ\break ಮುಟ್ಟಿಲ್ಲವೋ, ಅದೇ ಗುರಿ. ಈ ಸೃಷ್ಟಿ ಏತಕ್ಕೆ? ತಮಾಷೆಗೆ. ಇದನ್ನು ಮರೆತುಬಿಟ್ಟು ನಾವು ಜಗಳ ಕಾಯುತ್ತೇವೆ, ದುಃಖಪಡುತ್ತೇವೆ. ನಾವು ಜಗನ್ಮಾತೆಯೊಡನೆ ಆಡುತ್ತಿರುವವರು.

ಮಗುವನ್ನು ಬೆಳೆಸುವುದಕ್ಕೆ ತಾಯಿ ಪಡುವ ಕಷ್ಟವನ್ನು ನೋಡಿ. ಅವಳಿಗೆ ಇದರಿಂದ ಸಂತೋಷವೆ? ನಿಜವಾಗಿಯೂ ಸಂತೋಷ. ಉಪವಾಸ, ಪ್ರಾರ್ಥನೆ ಜಾಗರಣೆ\break ಮುಂತಾದುವನ್ನು ಮಗುವಿಗಾಗಿ ಮಾಡುವಳು. ಮಗುವನ್ನು ಎಲ್ಲಕ್ಕಿಂತ ಹೆಚ್ಚಾಗಿ\break ಪ್ರೀತಿಸುವಳು. ಏಕೆ? ಏಕೆಂದರೆ ಅಲ್ಲಿ ಸ್ವಾರ್ಥ ಇಲ್ಲ.

ಸುಖ ಬರುವುದು ಒಳ್ಳೆಯದೆ. ಯಾರು ಬೇಡವೆನ್ನುವರು? ದುಃಖ ಬರುವುದು ಅದನ್ನೂ ಸ್ವಾಗತಿಸಿ. ಸೊಳ್ಳೆಯೊಂದು ಎತ್ತಿನ ಕೊಂಬಿನ ಮೇಲೆ ಕುಳಿತಿತ್ತು. ಸೊಳ್ಳೆಗೆ ತಾನು ಏನೋ ದೊಡ್ಡ ಅಪರಾಧ ಮಾಡಿರುವೆ ಎಂದು ವ್ಯಥೆಯಾಗಿ: “ಎತ್ತೆ, ನಾನು ಬಹಳ ಕಾಲದಿಂದ ಇಲ್ಲಿ ಕುಳಿತುಕೊಂಡಿರುವೆ. ಬಹುಶಃ ನಿನಗೆ ತೊಂದರೆಯಾಗಿರಬಹುದು. ನಾನು ವಿಷಾದಪಡುತ್ತೇನೆ. ಹೊರಟು ಹೋಗುತ್ತೇನೆ” ಎಂದಿತು. ಆಗ ಎತ್ತು: “ಓ, ಏನೂ ಇಲ್ಲ. ನಿನ್ನ ಬಂಧುಬಳಗವನ್ನೆಲ್ಲ ಕರೆದುಕೊಂಡು ಬಂದು ಬೇಕಾದರೆ ನನ್ನ ಕೊಂಬಿನ ಮೇಲೆಯೇ ಜೀವಿಸು, ನೀನು ನನಗೆ ಏನು ಮಾಡಬಲ್ಲೆ?” ಎಂದಿತು.

ನಾವು ಹೀಗೆಯೇ ದುಃಖಕ್ಕೂ ಏತಕ್ಕೆ ಹೇಳಬಾರದು? ಧೀರನಾಗಬೇಕಾದರೆ ಜಗನ್ಮಾತೆಯಲ್ಲಿ ಶ್ರದ್ಧೆ ಇರಬೇಕು.

“ನಾನೇ ಜನನ, ನಾನೇ ಮರಣ.” ಅವಳ ಛಾಯೆಯೇ ಜನನ ಮರಣಗಳು. ಎಲ್ಲಾ ಸುಖಿಗಳಲ್ಲಿಯೂ ಸುಖ ಅವಳೇ, ದುಃಖಗಳಲ್ಲಿ ಅವಳೇ ದುಃಖ. ಜನನದ ಹಿಂದೆ ಅವಳಿರುವಳು. ಮರಣದ ಹಿಂದೆ ಅವಳಿರುವಳು. ಸ್ವರ್ಗ ಬಂದರೂ ಅದರಲ್ಲಿ ಅವಳಿರುವಳು. ನರಕ ಪ್ರಾಪ್ತಿಯಾದರೂ ಅಲ್ಲಿಯೂ ಅವಳೇ ಇರುವಳು, ಅಲ್ಲಿಗೂ ನುಗ್ಗು. ಇವನ್ನು ನೋಡುವುದಕ್ಕೆ ನಮಗೆ ಶ್ರದ್ಧೆ ಇಲ್ಲ, ತಾಳ್ಮೆ ಇಲ್ಲ. ದಾರಿಹೋಕನನ್ನಾದರೂ ನಾವು ನಂಬುವೆವು. ಆದರೆ ನಾವು ನಂಬದಿರುವ ಒಬ್ಬನು ವಿಶ್ವದಲ್ಲಿದ್ದಾನೆ, ಅವನೇ ದೇವರು. ನಾವು ಇಚ್ಛಿಸಿದಂತೆ ಅವನು ಮಾಡಿದರೆ ಆಗ ಅವನನ್ನು ಮೆಚ್ಚುವೆವು. ಪೆಟ್ಟಿನ ಮೇಲೆ ಪೆಟ್ಟು ಬೀಳುತ್ತಿದ್ದರೆ ಸ್ವಾರ್ಥ ನಾಶವಾಗುವುದು. ನಾವು ಮಾಡುವುದರಲ್ಲೆಲ್ಲಾ ಅಹಂಕಾರ ಹೆಡೆ ಎತ್ತುತ್ತಿರುವುದು. ದಾರಿಯಲ್ಲಿ ಅಷ್ಟೊಂದು ಮುಳ್ಳು ಇರುವುದು ಒಳ್ಳೆಯದು. ಅದು ಅಹಂಕಾರದ ಹಾವಿನ ಹೆಡೆಯನ್ನು ಚೆನ್ನಾಗಿ ಚುಚ್ಚುವುದು.

ಶರಣಾಗತಿಯ ಭಾವ ಕೊನೆಗೆ ಬರುವುದು. ಆಗ ನಾವು ಜಗನ್ಮಾತೆ ಹೇಳಿದಂತೆ ಕೇಳುವೆವು. ದುಃಖ ಬಂದರೆ ಸ್ವಾಗತ, ಸುಖ ಬಂದರೆ ಅದಕ್ಕೂ ಸ್ವಾಗತ. ಕೊನೆಗೆ ನಾವು ಸದಾ ಪ್ರೀತಿಯ ಮೆಟ್ಟಿಲನ್ನು ಹತ್ತಿದರೆ ಹೃದಯದ ವಕ್ರತೆಯೆಲ್ಲ ನೇರವಾಗುವುದು. ಆಗ ಬ್ರಾಹ್ಮಣ, ಪರಯ, ನಾಯಿ, ಎಲ್ಲರನ್ನೂ ಒಂದೇ ದೃಷ್ಟಿಯಿಂದ ನೋಡುತ್ತೇವೆ.\break ಸಮದೃಷ್ಟಿಯಿಂದ ಪಕ್ಷಪಾತವಿಲ್ಲದೆ ಎಂದಿಗೂ ಬತ್ತದ ಪ್ರೇಮದಿಂದ ಪ್ರಪಂಚವನ್ನು ನೋಡುವವರೆಗೆ ಪುನಃ ಪುನಃ ನಾವು ಸೋಲಬೇಕಾಗುವುದು. ಕೊನೆಗೆ ವೈವಿಧ್ಯವೆಲ್ಲ\break ಇಲ್ಲವಾಗಿ ಎಲ್ಲರಲ್ಲಿಯೂ ನೆಲಸಿರುವ ಸನಾತನಿಯಾದ ಜಗತ್ಮಾತೆಯನ್ನು ನೋಡುವೆವು.

