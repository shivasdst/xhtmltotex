
\chapter[ಮಹಮ್ಮದ್ ಗಂಬರ್ ]{ಮಹಮ್ಮದ್ ಗಂಬರ್ \protect\footnote{\engfoot{C.W. Vol. I, P. 481}}}

\centerline{\textbf{(೧೯೦೦ರ ಮಾರ್ಚ್​ ೨೫ರಂದು ಸ್ಯಾನ್​ಫ್ರಾನ್ಸಿಸ್ಕೊ ಪ್ರದೇಶದಲ್ಲಿ ನೀಡಿದ ಭಾಷಣ)}}

ಶ‍್ರೀಕೃಷ್ಣನ ಪುರಾತನ ಸಂದೇಶದಲ್ಲಿ ಬುದ್ಧ, ಕ್ರಿಸ್ತ, ಮತ್ತು ಮಹಮ್ಮದರ ಬೋಧನೆಯ ಸಾಮರಸ್ಯವಿದೆ. ಮೂವರಲ್ಲಿ ಪ್ರತಿಯೊಬ್ಬರೂ ಒಂದು ಭಾವನೆಯನ್ನು ಪ್ರಾರಂಭಿಸಿ ಅದನ್ನು ಪರಾಕಾಷ್ಠೆಗೆ ಒಯ್ದರು. ಶ‍್ರೀಕೃಷ್ಣ ಇತರ ದೇವದೂತರುಗಳೆಲ್ಲರಿಗಿಂತ ಮುಂಚೆಯೇ ಬರುವನು ಅವನದೇ ಅತ್ಯಂತ ಪ್ರಾಚೀನವಾದ ಸಂದೇಶವಾದರೂ ಅವನು ಹಳೆಯ\break ಭಾವನೆಗಳನ್ನು ತೆಗೆದುಕೊಂಡು ಅವನ್ನೆಲ್ಲ ಒಂದು ಸಾಮರಸ್ಯಕ್ಕೆ ತರುವನು. ಬುದ್ಧನ ಸಂದೇಶದ ಮಹಾಪ್ರವಾಹದಲ್ಲಿ ತಾತ್ಕಾಲಿಕವಾಗಿ ಅವನ ಸಂದೇಶವು ಅಡಗಿಹೋಯಿತು. ಇಂದು ಅದು ಇಂಡಿಯಾ ದೇಶದ ಒಂದು ವಿಶೇಷವಾದ ಸಂದೇಶವಾಗಿದೆ. ನೀವು ಒಪ್ಪಿದರೆ ನಾನು ಇಂದಿನ ಮಧ್ಯಾಹ್ನ ಮಹಮ್ಮದನ ವಿಷಯವನ್ನು ಮಾತನಾಡುತ್ತೇನೆ. ಆ ಪ್ರಖ್ಯಾತ ಅರೇಬಿಯದ ಪ್ರವಾದಿಯ ಕಾರ್ಯವನ್ನು ಕುರಿತು ಹೇಳುತ್ತೇನೆ.

ಮಹಮ್ಮದ್​ ಯುವಕನಾಗಿದ್ದಾಗ ಧರ್ಮಕ್ಕೇನೂ ಅಷ್ಟು ಮನಸ್ಸು ಕೊಟ್ಟವನಲ್ಲ. ಅವನು ಹಣ ಮಾಡುವುದರಲ್ಲಿ ನಿರತನಾಗಿದ್ದನು. ಅವನೊಬ್ಬ ಒಳ್ಳೆಯ ಯುವಕನಾಗಿದ್ದ. ನೋಡಲು ಸುಂದರವಾಗಿಯೂ ಇದ್ದನು. ಶ‍್ರೀಮಂತ ವಿಧವೆ ಇದ್ದಳು. ಅವಳು ಈ ಯುವಕನನ್ನು ಪ್ರೀತಿಸಿದಳು. ಆನಂತರ ಅವರಿಬ್ಬರೂ ಮದುವೆ ಆದರು. ಮಹಮ್ಮದ್​ ಪ್ರಪಂಚದ ದೊಡ್ಡ ಭಾಗಕ್ಕೆ ಚಕ್ರವರ್ತಿಯಾದ ಮೇಲೆ ರೋಮನ್​ ಮತ್ತು ಪರ್ಸಿಯಾದ ಚಕ್ರಾಧಿಪತ್ಯಗಳೆಲ್ಲ ಅವನ ಕಾಲ ಕೆಳಗೆ ಇದ್ದವು. ಅವನಿಗೆ ಹಲವು ಹೆಂಡರುಗಳು ಇದ್ದರು. ಒಂದು ದಿನ ಯಾರೊ, ನೀನು ಯಾವ ಹೆಂಡತಿಯನ್ನು ಹೆಚ್ಚು ಪ್ರೀತಿಸುವೆ ಎಂದು ಕೇಳಿದಾಗ ತನ್ನ ಮೊದಲನೆ ಹೆಂಡತಿಯನ್ನು ತೋರಿಸಿದನು. “ಏಕೆಂದರೆ ಅವಳು ಮೊದಲು ನನ್ನನ್ನು ನಂಬಿದಳು.” ಸ್ತ್ರೀಯರಲ್ಲಿ ನಂಬಿಕೆ ಇದೆ. ಸ್ವಾತಂತ್ರವನ್ನು ಪಡೆಯಿರಿ ಎಲ್ಲವನ್ನು ಪಡೆಯಿರಿ, ಆದರೆ ಹೆಂಗಸಿನ ಸ್ವಭಾವವಾದ ನಂಬಿಕೆಯನ್ನು ಮಾತ್ರ ಎಂದಿಗೂ ಕಳೆದುಕೊಳ್ಳಬೇಡಿ.

ಮಹಮ್ಮದನಿಗೆ ಪಾಪ, ವಿಗ್ರಹಾರಾಧನೆ, ಮಿಥ್ಯಾಚಾರ, ಮೂಢನಂಬಿಕೆ, ನರಬಲಿ ಮುಂತಾದವನ್ನು ನೋಡಿ ಬಹಳ ವ್ಯಥೆಯಾಯಿತು. ಯಹೂದ್ಯರು ಕ್ರೈಸ್ತರಿಂದ\break ಅಧೋಗತಿಗೆ ಬಂದಿದ್ದರು. ಆದರೆ ಕ್ರೈಸ್ತರಾದರೋ ಅವನ ದೇಶದ ಜನರಿಗಿಂತ ಮತ್ತೂ\break ಕೆಟ್ಟು ಹೋಗಿದ್ದರು.

ನಮಗೆ ಯಾವಾಗಲೂ ಆತುರ. ಆದರೆ ನಾವು ಏನಾದರೂ ಮಹತ್​ ಕಾರ್ಯವನ್ನು ಮಾಡಬೇಕಾದರೆ ಅದಕ್ಕೆ ಬೇಕಾದಷ್ಟು ಸಾಧನೆ ಇರಬೇಕು. ಹಗಲು ರಾತ್ರಿ ಬೇಕಾದಷ್ಟು ಪ್ರಾರ್ಥನೆ ಮಾಡಿದ ಮೇಲೆ ಮಹಮ್ಮದನಿಗೆ ಕನಸುಗಳು ಮತ್ತು ದೃಶ್ಯಗಳು ಕಾಣತೊಡಗಿದವು. ಅವನ ಕನಸಿನಲ್ಲಿ ಗೇಬ್ರಿಯಲ್​ ಕಾಣಿಸಿಕೊಂಡು, ನೀನು ಸತ್ಯದ ಸಂದೇಶವಾಹಕನೆಂದು ಹೇಳಿದನು. ಏಸು, ಮೋಸಸ್​, ಮತ್ತು ಇತರ ಮಹಾತ್ಮರ ಸಂದೇಶವೆಲ್ಲಾ ನಾಶವಾಗಿ ಹೋಗಿರುವುದೆಂದೂ, ನೀನು ಹೋಗಿ ಬೋಧಿಸಬೇಕೆಂದೂ ಹೇಳಿದನು. ಏಸುವಿನ ಹೆಸರಿನಲ್ಲಿ ಕ್ರೈಸ್ತರು ರಾಜಕೀಯವನ್ನೂ, ಪಾರ್ಸಿಗಳು ದ್ವೈತವನ್ನೂ ಬೋಧಿಸುತ್ತಿದ್ದುದನ್ನು ನೋಡಿ ಮಹಮ್ಮದನು ಹೀಗೆ ಹೇಳಿದನು: “ನಮ್ಮ ದೇವರು ಒಬ್ಬ. ಅವನೆ ಸರ್ವೇಶ್ವರ. ಅವನಿಗೂ ಇತರ ದೇವತೆಗಳಿಗೂ ಹೋಲಿಕೆಯೇ ಇಲ್ಲ.”

ದೇವರು ಎಂದರೆ ದೇವರೇ. ಅಲ್ಲಿ ಯಾವ ತೊಡಕಾದ ತತ್ತ್ವವೂ ಇಲ್ಲ, ಜಟಿಲವಾದ ನೀತಿ ನಿಯಮಾವಳಿಗಳೂ ಇಲ್ಲ. “ನಮ್ಮ ದೇವರು ಒಬ್ಬನೇ, ಅವನಿಗೆ ಎರಡನೆಯವರಿಲ್ಲ. ಮಹಮ್ಮದನೇ ಅವನ ದೂತ.” ಮಹಮ್ಮದನು ಮೆಕ್ಕಾದ ಬೀದಿಗಳಲ್ಲಿ ಬೋಧನೆಯನ್ನು ಪ್ರಾರಂಭಿಸಿದನು. ಜನ ಅವನನ್ನು ಹಿಂಸಿಸಲು ಆರಂಭಿಸಿದರು. ಅವನು ಮದೀನಕ್ಕೆ ಓಡಿ ಹೋದನು. ಅವನು ಯುದ್ಧ ಮಾಡಲು ಪ್ರಾರಂಭಿಸಿದನು. ಇಡೀ ಜನಾಂಗ ಒಂದಾಯಿತು. ದೇವರ ಹೆಸರಿನಲ್ಲಿ ಮಹಮ್ಮದೀಯ ಧರ್ಮ ಪ್ರಪಂಚವನ್ನೆಲ್ಲ ಆವರಿಸಿತು. ಅದೊಂದು ಅದ್ಭುತವಾದ ದಿಗ್ವಿಜಯ ಶಕ್ತಿಯಾಯಿತು.

ನಿಮ್ಮಂತಹ ಜನರಿಗೆ ಬದಲಾಯಿಸಲಾಗದಂತಹ ಭಾವನೆಗಳಿವೆ. ನಿಮ್ಮಲ್ಲಿ ಅಷ್ಟೊಂದು ಮೂಢನಂಬಿಕೆಗಳು ಇವೆ, ಪೂರ್ವ ನಿಶ್ಚಿತ ಅಭಿಪ್ರಾಯಗಳಿವೆ. ಈ ಸಂದೇಶವಾಹಕರು\break ದೇವರಿಂದಲೇ ಬಂದಿರಬೇಕು. ಇಲ್ಲದೇ ಇದ್ದರೆ ಅವರು ಹೇಗೆ ಇಷ್ಟು ದೊಡ್ಡವರಾಗುತ್ತಿದ್ದರು? ನೀವು ಪ್ರತಿಯೊಬ್ಬರಲ್ಲಿಯೂ ಅವಗುಣಗಳನ್ನು ನೋಡಿತ್ತೀರಿ. ಪ್ರತಿಯೊಬ್ಬನಲ್ಲಿಯೂ ಅವಗುಣಗಳಿವೆ. ಯಾರಲ್ಲಿ ಇಲ್ಲ? ಯಹೂದ್ಯರಲ್ಲಿ ನಾನು ಎಷ್ಟೋ ಲೋಪಗಳನ್ನು ತೋರಬಲ್ಲೆ. ದುರ್ಜನರು ಯಾವಾಗಲೂ ಅವಗುಣಗಳನ್ನೇ ನೋಡುತ್ತಿರುವರು. ನೊಣಗಳು ಗಾಯವನ್ನು ಹುಡುಕಿಕೊಂಡು ಬರುವುವು. ದುಂಬಿಗಳು ಹೂವಿನ ಮಕರಂದಕ್ಕೆ ಬರುವುವು. ನೊಣವನ್ನು ಅನುಸರಿಸಬೇಡಿ. ದುಂಬಿಯನ್ನು ಅನುಸರಿಸಿ.

ಅನಂತರ ಮಹಮ್ಮದನು ಹಲವು ಹೆಂಗಸರನ್ನು ಮದುವೆಯಾದನು. ಮಹಾಪುರು\-ಷರು ಪ್ರತಿಯೊಬ್ಬರೂ ಬೇಕಾದರೆ ಇನ್ನೂರು ಜನ ಸ್ತ್ರೀಯರನ್ನು ಮದುವೆಯಾಗಬಲ್ಲರು.\break ನಿಮ್ಮಂತಹ ‘ಮಹಾತ್ಮ’ರು ಒಬ್ಬರನ್ನು ಮದುವೆಯಾಗುವುದನ್ನೂ ನಾನು ಒಪ್ಪುವುದಿಲ್ಲ. ಮಹಾಪುರುಷರ ಜೀವನ ನಿಗೂಢವಾಗಿದೆ. ನಾವು ಅವರನ್ನು ತಿಳಿದುಕೊಳ್ಳಲು ಸಾಧ್ಯವಿಲ್ಲ. ನಾವು ಅವರನ್ನು ಅಳೆಯಲಾರೆವು. ಕ್ರಿಸ್ತನು ಬೇಕಾದರೆ ಮಹಮ್ಮದನನ್ನು ಅಳೆಯಬಹುದು. ನಾವು ನೀವು ಯಾರು? ಹಸುಗೂಸುಗಳು. ಅಂಥ ಮಹಾಮಹಿಮರನ್ನು ನಾವು ಏನು ಅರ್ಥಮಾಡಿಕೊಂಡೇವು?

ಮಹಮ್ಮದೀಯ ಧರ್ಮ ಜನಸಾಮಾನ್ಯರಿಗೆ ಒಂದು ಸಂದೇಶವನ್ನು ಕೊಡಲು\break ಬಂದಿತು. ಮೊದಲನೆ ಸಂದೇಶವೇ ಸಮಾನತೆಯ ಭಾವನೆ. ಪ್ರೇಮವೆಂಬ ಒಂದೇ ಧರ್ಮ ಇರುವುದು. ಅಲ್ಲಿ ಬಣ್ಣ, ಜನಾಂಗ ಮುಂತಾದ ಪ್ರಶ್ನೆಗಳೇ ಇಲ್ಲ. ಅದನ್ನು ಸೇರಿಸಿ! ಈ\break ವ್ಯವಹಾರಗುಣ ಆಗಿನ ಕಾಲದಲ್ಲಿ ಯಶಸ್ವಿಯಾಯಿತು. ಈ ಮಹಾಸಂದೇಶ ಬಹಳ ಸರಳ\-ವಾಗಿತ್ತು. ಸ್ವರ್ಗ ಮರ್ತ್ಯಗಳನ್ನು ಸೃಷ್ಟಿಸಿದ ಒಬ್ಬ ದೇವರನ್ನು ನಂಬಿ. ಅವನು ಶೂನ್ಯದಿಂದ ಎಲ್ಲವನ್ನೂ ಸೃಷ್ಟಿಸಿದನು. ಮತ್ತಾವ ಪ್ರಶ್ನೆಯನ್ನು ಕೇಳಬೇಡಿ.

ಅವರ ಮಸೀದಿಗಳು ಪ್ರಾಟೆಸ್ಟಂಟರ ಚರ್ಚುಗಳಂತೆ ಇವೆ. ಅಲ್ಲಿ ಯಾವ ಹಾಡೂ ಇಲ್ಲ. ಚಿತ್ರವೂ ಇಲ್ಲ. ಆಕಾರವೂ ಇಲ್ಲ. ಒಂದು ಮೂಲೆಯಲ್ಲಿ ಒಂದು ಪೀಠವಿದೆ. ಅದರ ಮೇಲೆ ಖುರಾನಿದೆ. ಜನರೆಲ್ಲ ಒಂದು ಸಾಲಿನಲ್ಲಿ ನಿಲ್ಲುವರು. ಅಲ್ಲಿ ಯಾವ ಪಾದ್ರಿಯೂ ಇಲ್ಲ. ಗುರುವೂ ಇಲ್ಲ. ಆಚಾರ್ಯನೂ ಇಲ್ಲ. ಯಾರು ಪ್ರಾರ್ಥಿಸುವನೊ ಅವನು ಜನರಿಂದ ಬೇರೆ ನಿಂತು ಪ್ರಾರ್ಥಿಸಬೇಕು. ಪ್ರಾರ್ಥನೆಯಲ್ಲಿ ಕೆಲವು ಭಾಗಗಳು ತುಂಬಾ\break ಸುಂದರವಾಗಿವೆ.

ಈ ಹಿಂದಿನ ಜನರೆಲ್ಲ ಭಗವಂತನ ಸಂದೇಶವಾಹಕರು. ನಾನು ಅವರಿಗೆ ಅಡ್ಡಬಿದ್ದು ಪೂಜಿಸುವೆನು. ಅವರ ಪಾದಧೂಳಿಯನ್ನು ತೆಗೆದುಕೊಳ್ಳುವೆನು. ಆದರೆ ಅವರು ಕಾಲವಾಗಿ ಹೋಗಿರುವರು. ನಾವು ಬದುಕಿರುವೆವು. ನಾವು ಮುಂದೆ ಹೋಗಬೇಕು. ಧರ್ಮ ಎಂದರೆ ಏಸು ಅಥವಾ ಮಹಮ್ಮದನನ್ನು ಅನುಸರಿಸುವುದು ಎಂದಲ್ಲ. ಒಂದು ವೇಳೆ ಅನುಕರಣ ಚೆನ್ನಾಗಿದ್ದರೂ ಅದು ಸಾಚಾ ಅಲ್ಲ. ಏಸುವಿನ ಒಂದು ಅನುಕರಣ ಆಗಬೇಡಿ. ನಿಜವಾಗಿಯೂ ಏಸುವೆ ಆಗಿ. ನೀವು ಏಸು, ಬುದ್ಧ ಅಥವಾ ಇತರರಂತೆಯೇ ದೊಡ್ಡವರು. ನಾವು ಹಾಗೆ ಇಲ್ಲದೆ ಇದ್ದರೆ ಹಾಗಾಗುವವರೆಗೆ ಪ್ರಯತ್ನಿಸಬೇಕು. ನಾನು ಏಸುವಿನ ಹಾಗೆಯೇ ಇರುವುದಕ್ಕೆ ಇಚ್ಛಿಸುವುದಿಲ್ಲ. ನಾನು ಯಹೂದ್ಯ ಆಗಿಲ್ಲದೆ ಇರುವುದರಿಂದ ಹಾಗೆ ಆಗಬೇಕಾದ ಆವಶ್ಯಕತೆಯೂ ಇಲ್ಲ. ಅತ್ಯಂತ ಉನ್ನತವಾದ ಧರ್ಮ ಎಂದರೆ ನಿಮ್ಮ ಸಹಜ ಸ್ವಭಾವಕ್ಕೆ ತಕ್ಕಂತೆ ಇರುವುದು. ಆತ್ಮಶ್ರದ್ಧೆ ಇರಲಿ. ನೀವೆ ಇಲ್ಲದೇ ಇದ್ದರೆ ದೇವರು ಅಥವಾ ಮತ್ತೆ ಯಾರಾದರೂ ಹೇಗೆ ಇರಬಲ್ಲರು? ನೀವು ಎಲ್ಲಿದ್ದರೂ, ಅನಂತವನ್ನು ಗ್ರಹಿಸುವುದು ನಿಮ್ಮ ಮನಸ್ಸೇ. ನಾನು ದೇವರನ್ನು ನೋಡುವೆ, ಆದ್ದರಿಂದಲೇ ಅವನು ಇರುವುದು. ನಾನು ದೇವರನ್ನು ಚಿಂತಿಸದೇ ಇದ್ದರೆ ನನ್ನ ಪಾಲಿಗೆ ಅವನು ಇಲ್ಲ. ಮಾನವ ಪ್ರಗತಿ ಮುಂದುವರಿಯುವುದು ಈ ರಾಜ ಮಾರ್ಗದಲ್ಲಿ.

ಈ ಮಹಾಪುರುಷರು ದಾರಿಯಲ್ಲಿರುವ ಕೈಮರಗಳಂತೆ ಅಷ್ಟೇ. ಅವರು ಮುಂದೆ ಹೋಗಿ ಸಹೋದರರೆ ಎಂದು ಅವರು ಹೇಳುವರು. ನಾವು ಅವರಿಗೆ ಅಂಟಿಕೊಳ್ಳುತ್ತೇವೆ. ಮುಂದುವರಿಯಲು ಇಚ್ಛಿಸುವುದಿಲ್ಲ. ಸ್ವಯಂ ಆಲೋಚಿಸುವುದಕ್ಕೆ ಇಚ್ಛೆಯಿಲ್ಲ. ಇತರರು ನಮಗಾಗಿ ಅದನ್ನು ಮಾಡಬೇಕು. ದೇವದೂತರು ತಮ್ಮ ಕಾರ್ಯವನ್ನು ಪ್ರಪಂಚದಲ್ಲಿ ಪೂರೈಸುವರು. ನೀವು ಮುಂದುವರಿದು ಹೋಗಿ ಎಂದು ನಮಗೆ ಹೇಳುವರು. ನೂರು ವರುಷಗಳಾದ ಮೇಲೆ ಅವರ ಸಂದೇಶವನ್ನು ಅಪ್ಪಿಕೊಂಡು ನಾವು ನಿದ್ರೆ ಹೋಗುವೆವು.

ಶ್ರದ್ಧೆ, ನಂಬಿಕೆ, ಸಿದ್ಧಾಂತ ಮುಂತಾದವುಗಳನ್ನು ಕುರಿತು ಮಾತನಾಡುವುದು ಸುಲಭ. ಆದರೆ ಚಾರಿತ್ರ್ಯವನ್ನು ರೂಢಿಸುವುದು, ಇಂದ್ರಿಯಗಳನ್ನು ಜಯಿಸುವುದು ಕಷ್ಟ. ನಾವು ಪ್ರಲೋಭನೆಗೆ ಬಲಿಯಾಗುವೆವು. ಮಿಥ್ಯಾಚಾರಿಗಳಾಗುವೆವು.

ಧರ್ಮ ಒಂದು ಸಿದ್ದಾಂತವಲ್ಲ, ಒಂದು ನಿಯಮಾವಳಿಯೂ ಅಲ್ಲ. ಅದೊಂದು\break ಬೆಳವಣಿಗೆಯ ಕ್ರಮ, ಅಷ್ಟೇ. ಸಿದ್ದಾಂತ, ನಿಯಮಾವಳಿಗಳು, ಇಲ್ಲ ನಮ್ಮವೇ ಅಂಗಸಾಧನೆಗೆ ಮಾತ್ರ. ಈ ಅಂಗಸಾಧನೆಯಿಂದ ನಾವು ಬಲಶಾಲಿಗಳಾಗುವೆವು. ಕೊನೆಗೆ\break ಬಂಧನದಿಂದ ಪಾರಾಗಿ ಮುಕ್ತರಾಗುವೆವು. ನಮ್ಮ ಅಂಗಸಾಧನೆಗಳಿಲ್ಲದೆ ಸಿದ್ಧಾಂತದಿಂದ ಏನೂ ಪ್ರಯೋಜನವಿಲ್ಲ. ಅಂಗಸಾಧನೆಯ ಮೂಲಕ ಆತ್ಮ ಪರಿಪೂರ್ಣವಾಗುವುದು. ನಾನು ನಂಬುತ್ತೇನೆ ಎಂದೊಡನೆಯೇ ಆ ಅಂಗಸಾಧನೆ ನಿಂತಂತೆಯೆ.

“ಧರ್ಮ ಯಾವಾಗ ನಾಶವಾಗುವುದೋ, ಅಧರ್ಮ ಯಾವಾಗ ಹೆಚ್ಚುವುದೊ ಆಗ ನಾನು ಅವತಾರ ಮಾಡುತ್ತೇನೆ. ಪ್ರತಿಯೊಂದು ಯುಗದಲ್ಲಿಯೂ ನಾನು ಸಾಧುಗಳನ್ನು\break ರಕ್ಷಿಸುವುದಕ್ಕೆ, ದುಷ್ಟರನ್ನು ಶಿಕ್ಷಿಸುವುದಕ್ಕೆ, ಧರ್ಮವನ್ನು ಸಂಸ್ಥಾಪನೆ ಮಾಡುವುದಕ್ಕೆ\break ಅವತಾರವೆತ್ತುವೆನು”. (ಗೀತಾ \enginline{IV, 7–8})

ಜ್ಞಾನಜ್ಯೋತಿಯ ಮಹಾಸಂದೇಶಕರು ಇವರು. ಇವರು ನಮ್ಮ ಮಹಾ ಗುರುಗಳು, ಅಣ್ಣಂದಿರು. ಆದರೆ ನಾವು ನಮ್ಮ ದಾರಿಯನ್ನು ಹಿಡಿದುಹೋಗಬೇಕಾಗಿದೆ.

