
\chapter[ಬೌದ್ಧಧರ್ಮ ಮತ್ತು ವೇದಾಂತ ]{ಬೌದ್ಧಧರ್ಮ ಮತ್ತು ವೇದಾಂತ \protect\footnote{\engfoot{C.W. Vol. V, P. 279}}}

ವೇದಾಂತ ತತ್ತ್ವವೇ ಬೌದ್ಧಧರ್ಮಕ್ಕೆ ಮತ್ತು ಇಂಡಿಯಾ ದೇಶದ ಇತರ ಧರ್ಮಗಳಿಗೆಲ್ಲಾ ತಳಹದಿ. ಆದರೆ ನಾವು ಯಾವುದನ್ನು ಆಧುನಿಕ ಅದ್ವೈತ ಸಿದ್ಧಾಂತ ಗಳೆನ್ನುವೆವೊ ಅವುಗಳಲ್ಲಿ ಹಲವು ಬೌದ್ಧ ಸಿದ್ಧಾಂತಗಳಾಗಿವೆ. ಹಿಂದೂಗಳು, ಅದರಲ್ಲೂ ಆಚಾರಶೀಲರಾದ ಹಿಂದೂಗಳು, ಇದನ್ನು ಒಪ್ಪಿಕೊಳ್ಳುವುದಿಲ್ಲ. ಏಕೆಂದರೆ ಬೌದ್ಧರು ಅವರ ಪಾಲಿಗೆ ಅವೈದಿಕರು. ಆದರೆ ಅವೈದಿಕ ಸಿದ್ಧಾಂತಗಳನ್ನು ಒಳಕೊಳ್ಳುವಂತೆ ಇಡೀ ಸಿದ್ಧಾಂತವನ್ನು ವಿಸ್ತಾರಗೊಳಿಸುವ ಪ್ರಯತ್ನಗಳು ನಡೆದಿವೆ.

ವೇದಾಂತಕ್ಕೆ ಬೌದ್ಧಧರ್ಮದೊಡನೆ ಯಾವ ಮನಸ್ತಾಪವೂ ಇಲ್ಲ. ವೇದಾಂತದ ಭಾವನೆಯೇ ಎಲ್ಲರನ್ನೂ ಒಂದು ಸೌಹಾರ್ದಕ್ಕೆ ತರುವುದಾಗಿದೆ. ಔತ್ತರೇಯ ಬೌದ್ಧರೊಡನೆ ನಮಗೆ ವ್ಯಾಜ್ಯವಿಲ್ಲ. ಬರ್ಮಾ, ಸಯಾಂ ಮತ್ತು ಇತರ ದಾಕ್ಷಿಣಾತ್ಯ ಪಂಗಡದವರೆಲ್ಲ, ನಮ್ಮೆದುರಿಗೆ ಒಂದು ವ್ಯಕ್ತ ಪ್ರಪಂಚವಿದೆ, ಅದರ ಹಿಂದೆ ಮತ್ತೊಂದು ಅವ್ಯಕ್ತವಿದೆ ಎಂದು ಊಹಿಸುವುದಕ್ಕೆ ನಿಮಗೇನು ಅಧಿಕಾರವಿದೆ ಎಂದು ಕೇಳುವರು. ವೇದಾಂತದ ಉತ್ತರವೇ ನಿಮ್ಮ ವಾದ ಅಸತ್ಯ ಎನ್ನುವುದು. ವೇದಾಂತವು ಒಂದು ವ್ಯಕ್ತ ಮತ್ತೊಂದು ಅವ್ಯಕ್ತ ಎಂಬ ಎರಡು ಇವೆ ಎಂಬುದನ್ನು ಎಂದೂ ಒಪ್ಪುವುದಿಲ್ಲ. ಇರುವುದು ಒಂದೇ. ಅದನ್ನು ಪಂಚೇಂದ್ರಿಯಗಳ ಮೂಲಕ ನೋಡಿದಾಗ ಅದು ವ್ಯಕ್ತವಾಗಿ ಕಾಣಿಸುವುದು. ಆದರೆ ಅದು ನಿಜವಾಗಿ ಯಾವಾಗಲೂ ಅವ್ಯಕ್ತವೇ. ಹಾವನ್ನು ನೋಡುವವನು ಹಗ್ಗವನ್ನು ನೋಡುವುದೇ ಇಲ್ಲ. ಅದು ಹಾವು ಇಲ್ಲವೆ ಹಗ್ಗ, ಎರಡರಲ್ಲಿ ಒಂದು, ಎಂದಿಗೂ ಎರಡಲ್ಲ. ಆದಕಾರಣ ವೇದಾಂತಿಗಳು ಎರಡು ಪ್ರಪಂಚಗಳನ್ನು ನಂಬು ತ್ತಾರೆ ಎಂಬ ಬೌದ್ಧರ ವಾದ ಸುಳ್ಳು. ಇದು ಕೇವಲ ವ್ಯಕ್ತ ಎಂದು ಹೇಳುವುದಕ್ಕೆ ಅವರಿಗೆ ಅಧಿಕಾರವಿದೆ. ಆದರೆ ಮತ್ತೊಬ್ಬರು ಅದನ್ನು ಅವ್ಯಕ್ತ ಎಂದು ಹೇಳ ಬಾರದು ಎನ್ನುವುದಕ್ಕೆ ಅವರಿಗೆ ಅಧಿಕಾರವಿಲ್ಲ.

ಬೌದ್ಧರು ಬಾಹ್ಯ ಪ್ರಪಂಚವನ್ನಲ್ಲದೆ ಬೇರೆ ಏನನ್ನೂ ಒಪ್ಪುವುದಿಲ್ಲ. ಆಸೆ ಇರುವುದು ಕಣ್ಣಿಗೆ ಕಾಣುವ ಪ್ರಪಂಚದಲ್ಲಿ ಮಾತ್ರ. ಆಸೆಯೇ ಇದೆಲ್ಲವನ್ನೂ ಸೃಷ್ಟಿಸುತ್ತಿರುವುದು. ಆಧುನಿಕ ವೇದಾಂತಿಗಳು ಇದನ್ನು ಒಪ್ಪಿಕೊಳ್ಳುವುದಿಲ್ಲ. ಯಾವುದೋ ಒಂದು ಇಚ್ಛೆಯು ರೂಪವನ್ನು ತಾಳಿದೆ ಎನ್ನುವರು. ಇಚ್ಛೆ ಒಂದು ಮಿಶ್ರಣ; ಹೊಸದಾಗಿ ತಯಾರಾದುದು. ಬಾಹ್ಯ ವಸ್ತುವಿಲ್ಲದೇ ಯಾವ ಇಚ್ಛೆಯೂ ಇರಲಾರದು. ಇಚ್ಛೆ ಪ್ರಪಂಚವನ್ನು ಸೃಷ್ಟಿಸಿತು ಎಂಬುದೇ ಯುಕ್ತಿಬದ್ಧವಲ್ಲ. ಇದು ಹೇಗೆ ಸಾಧ್ಯ? ಬಾಹ್ಯ ಪ್ರಚೋದನೆಯಿಲ್ಲದೆ ಇಚ್ಛೆ ಎಂಬುದು ಎಂದಾದರೂ ಸಾಧ್ಯವಾಗುವುದನ್ನು ನೀವು ನೋಡಿರುವಿರಾ? ಬಾಹ್ಯ ಉದ್ದೀಪನವಿಲ್ಲದೆ ಆಸೆ ಏಳಲಾರದು. ಆಧುನಿಕ ತಾತ್ತ್ವಿಕ ಭಾಷೆಯಲ್ಲಿ ಹೇಳುವುದಾದರೆ ನರಗಳ ಉದ್ದೀಪನೆ ಇಲ್ಲದೆ ಇಚ್ಛೆ ಏಳಲಾರದು. ಇಚ್ಛೆ ಎನ್ನುವುದು ಮಿದುಳಿನ ಒಂದು ಪ್ರತಿಕ್ರಿಯೆ. ಸಾಂಖ್ಯರು ಇದನ್ನೇ ಬುದ್ಧಿ ಎನ್ನುವರು. ಈ ಪ್ರತಿಕ್ರಿಯೆಯ ಹಿಂದೆ ಒಂದು ಕ್ರಿಯೆ ಇರಬೇಕಾಗುವುದು. ಕ್ರಿಯೆ ಇರಬೇಕಾದರೆ ಒಂದು ಬಾಹ್ಯ ಪ್ರಪಂಚ ಇದೆ ಎಂದು ಊಹಿಸಬೇಕು. ಬಾಹ್ಯ ಪ್ರಪಂಚವಿಲ್ಲದೇ ಇದ್ದರೆ ಎಂದಿಗೂ ಇಚ್ಛೆ ಇರುವುದಿಲ್ಲ. ಆದರೂ ನಿಮ್ಮ ಸಿದ್ಧಾಂತದ ಪ್ರಕಾರ ಇಚ್ಛೆಯೇ ವಿಶ್ವ ವನ್ನು ಸೃಷ್ಟಿಸಿತು. ಇಚ್ಛೆಗೆ ಯಾರು ಕಾರಣ? ಇಚ್ಛೆ ಮತ್ತು ಪ್ರಪಂಚ ಇರಬೇಕಾದರೆ, ಅವೆರಡೂ ಸಮಾನ ಕಾಲದಲ್ಲಿ ಮಾತ್ರ ಇರಬಲ್ಲವು. ಯಾವುದು ಪ್ರಪಂಚವನ್ನು ಸೃಷ್ಟಿಸಿತೊ ಅದೇ ಇಚ್ಛೆಯನ್ನೂ ಸೃಷ್ಟಿಸಿತು. ಆದರೆ ತತ್ತ್ವ ಅಲ್ಲಿ ನಿಲ್ಲಲಾರದು. ಇಚ್ಛೆ ಕೇವಲ ವೈಯಕ್ತಿಕವಾದುದು. ಆದಕಾರಣ ನಾವು ಷೋಫನಿಯರ್​ನ ಅಭಿಪ್ರಾಯವನ್ನು ಒಪ್ಪಲಾರೆವು. ಇಚ್ಛೆ ಒಳಗು ಹೊರಗುಗಳ ಮಿಶ್ರಣ. ಒಬ್ಬನು ಯಾವ ಇಂದ್ರಿಯಗಳೂ ಇಲ್ಲದೆ ಜನಿಸಿದರೆ ಅವನಿಗೆ ಇಚ್ಛೆ ಇರಲಾರದು. ಇಚ್ಛೆಗೆ ಹೊರಗಿನಿಂದ ಏನೋ ಬೇಕಾಗಿದೆ. ಮಿದುಳಿಗೆ ಒಳಗಿನಿಂದ ಏನೋ ಶಕ್ತಿ ದೊರಕುವುದು. ಆದಕಾರಣ ಇಚ್ಛೆ ಎಂಬುದು ಒಂದು ಮಿಶ್ರಣ. ಹೊರಗಡೆಯ ಗೋಡೆ ಅಥವಾ ಮತ್ತೊಂದು ಹೇಗೆ ಒಂದು ಮಿಶ್ರಣವೊ ಹಾಗೆಯೇ ಇದೊಂದು ಮಿಶ್ರವಸ್ತು. ನಾವು ಜರ್ಮನ್​ ದಾರ್ಶನಿಕರ ಇಚ್ಛಾ ಸಿದ್ಧಾಂತವನ್ನು ಒಪ್ಪಿಕೊಳ್ಳು ವುದಿಲ್ಲ. ಇಚ್ಛೆಯೇ ಒಂದು ಪ್ರತಿಕ್ರಿಯೆ. ಅದು ನಿರಪೇಕ್ಷವಾಗಿರಲಾರದು. ಅದು ಹಲವು ಆವಿರ್ಭಾವಗಳಲ್ಲಿ ಒಂದು. ಇಚ್ಛೆಯಲ್ಲದುದು ಯಾವುದೋ ಒಂದು ಇದೆ. ಅದು ಇಚ್ಛೆಯ ಆಕಾರದಲ್ಲಿ ಕಾಣಿಸುತ್ತಿದೆ. ಇದನ್ನು ನಾನು ಅರ್ಥಮಾಡಿ ಕೊಳ್ಳಬಲ್ಲೆ. ಆದರೆ ಇಚ್ಛೆಯೇ ಹಲವು ರೂಪಗಳನ್ನು ತಾಳಿದೆ ಎಂಬುದನ್ನು ನಾನು ಅರ್ಥಮಾಡಿಕೊಳ್ಳಲಾರೆ. ಏಕೆಂದರೆ ಪ್ರಪಂಚವನ್ನು ಬಿಟ್ಟರೆ ಇಚ್ಛೆ ಎಂಬುದು ಪ್ರತ್ಯೇಕವಾಗಿ ಇಲ್ಲ, ಸ್ವತಂತ್ರವಾಗಿರುವ ಯಾವುದೋ ಒಂದು ಇಚ್ಛೆಯ ರೂಪವನ್ನು ತಾಳಿದೆ ಎಂದಾದರೆ ಅದು ದೇಶಕಾಲನಿಮಿತ್ತಗಳಿಂದ ಆಯಿತು. ಕ್ಯಾಂಟನ ವಿಶ್ಲೇಷಣೆಯನ್ನು ತೆಗೆದುಕೊಳ್ಳಿ. ಇಚ್ಛೆ ಕಾಲದೇಶನಿಮಿತ್ತಗಳೊಳಗೆ ಇರುವುದು. ಹಾಗಾದರೆ ಇದು ಹೇಗೆ ನಿರಪೇಕ್ಷವಾಗಬಲ್ಲದು? ಒಬ್ಬನು ಕಾಲದಲ್ಲಿಲ್ಲದೆ ಇದ್ದರೆ ಅವನು ಇಚ್ಛಿಸಲಾರ.

ನಾವು ಆಲೋಚನೆಯನ್ನೆಲ್ಲಾ ನಿಲ್ಲಿಸಿದಾಗ, ನಾವು ಆಲೋಚನೆಗೆ ಅತೀತವಾಗಿರುವೆವು ಎಂಬುದು ಗೊತ್ತಿದೆ.ನೇತಿಮಾರ್ಗದಿಂದ ಇದು ನಮಗೆ ಗೊತ್ತಾಗುವುದು. ಪ್ರತಿಯೊಂದು ದೃಶ್ಯವಸ್ತುವನ್ನೂ ನಾವು ಅಲ್ಲಗಳೆದ ಮೇಲೆ ಕೊನೆಗೆ ಉಳಿಯುವುದೇ ಅದು. ಅದನ್ನು ವಿವರಿಸುವುದಕ್ಕೆ ಆಗುವುದಿಲ್ಲ. ವ್ಯಕ್ತಗೊಳಿಸಲಿಕ್ಕೆ ಆಗುವುದಿಲ್ಲ. ಏಕೆಂದರೆ ವ್ಯಕ್ತರೂಪವನ್ನು ತಾಳುವಂತಹದು ಇಚ್ಛೆಯೇ ಆಗಿದೆ.

