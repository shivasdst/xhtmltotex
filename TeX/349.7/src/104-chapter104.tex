
\chapter[ಹಿಂದೂಧರ್ಮದ ಮೇರೆ ]{ಹಿಂದೂಧರ್ಮದ ಮೇರೆ \protect\footnote{\engfoot{C.W. Vol. V, P. 233}}}

\centerline{(“ಪ್ರಬುದ್ಧ ಭಾರತ” ಏಪ್ರಿಲ್​ ೧೮೯೯)}

‘ಪ್ರಬುದ್ಧ ಭಾರತ’ ದ ಸಂಪಾದಕರ ಆಜ್ಞೆಯಂತೆ ನಾನು ಒಂದು ದಿನ ಸ್ವಾಮಿ ವಿವೇವಾನಂದರನ್ನು ಹಿಂದೂಧರ್ಮಕ್ಕೆ ಇತರರನ್ನು ಸೇರಿಸುವ ವಿಚಾರದಲ್ಲಿ ಪ್ರಶ್ನಿಸಲು ಹೋದೆ. ಗಂಗಾನದಿಯ ಮೇಲೆ ಒಂದು ದೋಣಿಯಲ್ಲಿ ಅವರನ್ನು ಕಂಡಾಗ ನನ ಗೊಂದು ಅವಕಾಶ ದೊರಕಿತು. ರಾತ್ರಿಯಾಗಿತ್ತು. ದೋಣಿ ರಾಮಕೃಷ್ಣ ಮಠದ ಘಾಟಿನ ಹತ್ತಿರ ನಿಂತಿತು. ಸ್ವಾಮೀಜಿ ಅವರು ನನ್ನೊಡನೆ ಮಾತನಾಡಲು ಕೆಳಗೆ ಇಳಿದುಬಂದರು.

ಮಾತುಕತೆಗೆ ಕಾಲದೇಶಗಳೆರಡೂ ಚೆನ್ನಾಗಿದ್ದುವು. ಮೇಲೆ ಆಕಾಶದಲ್ಲಿ ತಾರೆಗಳು ಮಿನುಗುತ್ತಿದ್ದುವು. ಸುತ್ತಲೂ ಗಂಗಾನದಿಯ ಅಲೆಗಳು ನಲಿದಾಡು ತ್ತಿದ್ದುವು.ಒಂದು ಕಡೆ ಮಂದಕಾಂತಿಯ ದೀಪಗಳಿಂದ ಕೂಡಿದ ಮಠವಿತ್ತು. ಅದರ ಹಿನ್ನೆಲೆಯಾಗಿ ವಿಶಾಲವಾದ ನೆರಳನ್ನು ಕೊಡುವ ಎತ್ತರವಾದ ಮರಗಳಿದ್ದವು.

ಪ್ರಶ್ನೆ: “ಸ್ವಾಮೀಜಿ, ಯಾರು ಹಿಂದೂಧರ್ಮವನ್ನು ಬಿಟ್ಟು ಹೋಗಿರುವರೋ ಅವರನ್ನು ಪುನಃ ಸೇರಿಸಿಕೊಳ್ಳುವ ವಿಷಯದಲ್ಲಿ ನಿಮ್ಮೊಡನೆ ಮಾತನಾಡಬೇಕಾಗಿದೆ. ಅವರನ್ನು ಹಿಂದಕ್ಕೆ ತೆಗೆದುಕೊಳ್ಳುವುದಕ್ಕೆ ನಿಮ್ಮ ಯಾವ ಅಭ್ಯತರವೂ ಇಲ್ಲವೆ?”

ಸ್ವಾಮೀಜಿ: “ಖಂಡಿತವಾಗಿಯೂ, ಅವರನ್ನು ಅವಶ್ಯಕವಾಗಿಯೂ ಸ್ವೀಕರಿಸ ಬೇಕು.

(ಸ್ವಲ್ಪ ಹೊತ್ತು ಆಲೋಚನಾಮಗ್ನರಾಗಿ ಅನಂತರ ಹೇಳಿದದು:) “ಇಲ್ಲದೇ ಇದ್ದರೆ ನಮ್ಮ ಸಂಖ್ಯೆ ಕಡಮೆಯಾಗುತ್ತಾ ಬರುವುದು ಎಂದು ಹೇಳಲಾಗಿದೆ. ಅದಕ್ಕೆ ಆಧಾರ, ಬಹುಶಃ ಅತ್ಯಂತ ಪ್ರಾಚೀನ ಮಹಮ್ಮದೀಯ ಇತಿಹಾಸಕಾರ ಫರಿಷ್ಟನ ನಿರೂಪಣೆ ಇರಬೇಕು. ಅವನ ಪ್ರಕಾರ ಆಗ ಹಿಂದೂಗಳು ಅರವತ್ತು ಕೋಟಿ ಇದ್ದರು. ಈಗ ನಾವು ಇಪ್ಪತ್ತು ಕೋಟಿ ಇರುವೆವು. ಅದಲ್ಲದೆ ಹಿಂದೂಧರ್ಮದಿಂದ ಹೊರಗೆ ಹೋದವರೆಲ್ಲ ಹಿಂದೂಧರ್ಮದಲ್ಲಿ ಒಬ್ಬ ಕಡಮೆಯಾದುದು ಮಾತ್ರವಲ್ಲ, ಹಿಂದೂಗಳ ಒಬ್ಬ ವೈರಿ ಹೆಚ್ಚಿದಂತೆ, ಇಸ್ಲಾಂ ಮತ್ತು ಕ್ರೈಸ್ತಧರ್ಮಕ್ಕೆ ಸೇರಿದವರೆಲ್ಲ ಕತ್ತಿಯ ಭಯದಿಂದ ಹಾಗೆ ಆದವರು ಅಥವಾ ಅವರ ಮನೆತನದವರು. ಹೀಗೆ ಅನ್ಯಧರ್ಮಕ್ಕೆ ಸೇರಿದವರನ್ನು ಪುನಃ ಹಿಂದೂಧರ್ಮಕ್ಕೆ ಸೇರಿಸದೆ ಇರುವುದು ನ್ಯಾಯವಲ್ಲ. ಬಹಳ ಹಿಂದಿನಿಂದಲೂ ಬೇರೆ ಧರ್ಮಕ್ಕೆ ಸೇರಿದವರ ಪ್ರಶ್ನೆಯೆ? ಹಿಂದಿನಿಂದಲೂ ಅನೇಕ ಹೊರಗಿನವರನ್ನು ಹಿಂದೂಧರ್ಮಕ್ಕೆ ಸೇರಿಸಿರುವರು. ಅದು ಈಗಲೂ ಆಗುತ್ತಿದೆ.

“ಈ ಮಾತು ಕೇವಲ ಕಾಡುಜನಾಂಗಗಳಿಗೆ, ನಮ್ಮ ನೆರೆಹೊರೆಯ ರಾಷ್ಟ್ರದವರಿಗೆ ಮತ್ತು ಮಹಮ್ಮದೀಯರು ಬರುವುದಕ್ಕೆ ಮುಂಚೆ ನಮ್ಮನ್ನು ಗೆದ್ದವರಿಗೆ ಮಾತ್ರ ವಲ್ಲದೆ ಪುರಾಣದಲ್ಲಿ ಬರುವ ಇನ್ನೂ ಹಲವು ಪಂಗಡಗಳಗೂ ಅನ್ವಯಿಸುತ್ತದೆ. ಅವರೂ ಕೂಡ ಹಿಂದೂಧರ್ಮಕ್ಕೆ ಸೇರಿದ ಹೊರಗಿನವರೆ.”

“ಹಿಂದೂಧರ್ಮದಲ್ಲಿ ಮೊದಲಿದ್ದು ಸ್ವಂತ ಇಚ್ಛೆಯಿಂದ ಅನ್ಯಮತಕ್ಕೆ ಸೇರಿ ಈಗ ಪುನಃ ಹಿಂದೂಗಳಾಗುವವರಿಗೆ ಶುದ್ಧೀಕರಣ ಮುಂತಾದವು ನಿಸ್ಸಂದೇಹವಾಗಿ ಯೋಗ್ಯವಾಗಿಯೇ ಇವೆ. ಆದರೆ ಕಾಶ್ಮೀರ ಮತ್ತು ನೇಪಾಳದಲ್ಲಿರುವಂತೆ ಅನ್ಯರ ಆಕ್ರಮಣಕ್ಕೆ ಒಳಗಾದವರು, ನಮ್ಮ ಧರ್ಮಕ್ಕೆ ಸೇರುವ ಅನ್ಯಮತೀಯರು, ಇವರಿಗೆ ಶುದ್ಧಿ ಅವಶ್ಯವಿಲ್ಲ.”

ಪ್ರಶ್ನೆ: “ಅದರೆ ಇವರನ್ನು ಯಾವ ವರ್ಣಕ್ಕೆ ಸೇರಿಸುವುದು? ಅವರಿಗೆ ಯಾವು ದಾದರೂ ಒಂದು ವರ್ಣವಿರಬೇಕು. ಇಲ್ಲದೇ ಇದ್ದರೆ ಹಿಂದೂಧರ್ಮ ಅವರನ್ನು ಹೀರಿಕೊಳ್ಳಲಾರದು. ಇವರಿಗೆ ಯಾವ ವರ್ಣದ ಅಂತಸ್ತನ್ನು ಕೊಡುವುದು?”

ಸ್ವಾಮೀಜಿ: “ಹಿಂತಿರುಗಿ ಬರುವವರಿಗೆ ಅವರು ಹಿಂದೆ ಇದ್ದ ವರ್ಣವೇ ಇದೆ.ಹೊಸದಾಗಿ ಬರುವವರಿಗೆ ಅವರು ಒಂದು ವರ್ಣವನ್ನು ಮಾಡಿಕೊಳ್ಳುವರು. ವೈಷ್ಣವರು ಇದನ್ನು ಆಗಲೇ ಮಾಡಿರುವರು ಎಂಬುದನ್ನು ಜ್ಞಾಪಕದಲ್ಲಿಡಿ. ಅನ್ಯ ಧರ್ಮದಿಂದ ಮತ್ತು ತಮ್ಮದೇ ಧರ್ಮಕ್ಕೆ ಹಿಂತಿರುಗಿ ಬಂದವರು ಇವರೆಲ್ಲ ವೈಷ್ಣವ ಪಂಥದ ಕೆಳಗೆ ಬರುವರು. ಇದೊಂದು ಗೌರವದಿಂದ ಕೂಡಿದ ಪಂಥವೇ ಆಗಿದೆ. ರಾಮಾನುಜಾಚಾರ್ಯರಿಂದ ಹಿಡಿದು ಚೈತನ್ಯ ದೇವನವರೆಗೆ ಎಲ್ಲರೂ ಇದನ್ನು ಮಾಡಿರುವರು.”

ಪ್ರಶ್ನೆ:” ಹೊಸದಾಗಿ ಸೇರುವವರು ಯಾರೊಡನೆ ವಿವಾಹ ಸಂಬಂಧವನ್ನು ಬೆಳೆಸಬೇಕು?”

ಸ್ವಾಮೀಜಿ: “ಈಗಿನಂತೆ ತಮ್ಮತಮ್ಮೊಳಗೇ ಮಾಡಿಕೊಳ್ಳಬಹುದು.”

ಪ್ರಶ್ನೆ: “ಆದರೆ ಹೊಸದಾಗಿ ಸೇರುವವರಿಗೆ ಬೇರೆ ಬೇರೆ ಹೆಸರುಗಳನ್ನು ಕೊಡಬೇಕಾಗಿದೆ. ನೀವು ಅವರಿಗೆ ಜಾತಿಸೂಚಕ ಹೆಸರನ್ನು ಕೊಡುತ್ತೀರ ಅಥವಾ ಹೇಗೆ?”

ಸ್ವಾಮೀಜಿ: (ಚೆನ್ನಾಗಿ ಯೋಚಿಸಿ) “ಹೌದು ಅವರಿಗೊಂದು ಹೊಸ ಹೆಸರು ಬೇಕು. ಅದರಲ್ಲಿ ಎಷ್ಟೋ ಮಹತ್ವವಿದೆ.” (ಈ ವಿಷಯವನ್ನು ಅವರು ಮುಂದುವರಿಸಲಿಲ್ಲ.)

ಪ್ರಶ್ನೆ: “ಅವರು ಹಿಂದೂಧರ್ಮಗಳಲ್ಲಿ ಈಗಿರುವ ಹಲವು ಆಚರಣೆಗಳಲ್ಲಿ ತಮಗೆ ತೋಚಿದನ್ನು ಆರಿಸಿಕೊಳ್ಳುವರೋ ಅಥವಾ ಹೊಸದಾಗಿ ನೀವು ಅವರಿಗೆ ಯಾವುದಾದರೂ ಧರ್ಮವನ್ನು ಸೂಚಿಸುವುರೋ?”

ಸ್ವಾಮೀಜಿ: “ಇದನ್ನು ನೀವು ಕೇಳುತ್ತೀರ? ತಮಗೆ ತೋಚಿದುದನ್ನು ಅವರು ಆರಿಸಿಕೊಳ್ಳುವರು. ಆರಿಸಿಕೊಳ್ಳುವುದಕ್ಕೆ ಅವಕಾಶವಿಲ್ಲದೇ ಇದ್ದರೆ ಹಿಂದೂಧರ್ಮದ ಭಾವನೆಗೇ ಕುಂದು ಬಂದಂತೆ. ನಮ್ಮ ಧರ್ಮದ ಸಾರವೆಂದರೆ ತಮಗೆ ತೋಚಿದ ಇಷ್ಟದೈವವನ್ನು ಅವರು ಆರಿಸಿಕೊಳ್ಳಬಹುದಾಗಿದೆ.”

ಅವರಾಡಿದ ಮಾತಿನಲ್ಲಿ ಬಹಳ ತೂಕವಿದೆ ಎನ್ನಿಸಿತು. ನನ್ನೆದುರಿಗೆ ಇರುವ ಸ್ವಾಮೀಜಿ ಅವರು ಈಗಿರುವ ನಮ್ಮಲ್ಲರಿಗಿಂತ ಹೆಚ್ಚಾಗಿ ಹಿಂದೂಧರ್ಮದ ಸಾಮಾನ್ಯ ತತ್ವವನ್ನು ವಿಮರ್ಶಾತ್ಮಕ ದೃಷ್ಟಿಯಿಂದ ಮತ್ತು ಉದಾರಭಾವನೆಯಿಂದ ಅಧ್ಯಯನ ಮಾಡಿದವರು. ಇಷ್ಟದೇವತೆಯ ತತ್ತ್ವವು ಇಡೀ ಪ್ರಪಂಚವನ್ನೇ ಅಳವಡಿಸಿಕೊಳ್ಳಬಹುದಾದಷ್ಟು ದೊಡ್ಡದಾಗಿದೆ.

ಸಂಭಾಷಣೆ ಬೇರೆ ಬೇರೆ ವಿಷಯಗಳ ಕಡೆ ತಿರುಗಿತು. ಅನಂತರ ಈ ಮಹಾ ಆಚಾರ್ಯರು ನನ್ನನ್ನು ಬೀಳ್ಕೊಟ್ಟು ಲಾಂದ್ರವನ್ನು ಹಿಡಿದುಕೊಂಡು ಮಠಕ್ಕೆ ಹೋದರು. ನಾನಾದರೋ ಗಂಗಾ ನದಿಯ ಮೇಲೆ ಹಲವು ದೋಣಿಗಳ ಮಧ್ಯೆ ಹಾಯ್ದು ಹೋಗಿ ಕಲ್ಕತ್ತೆಯನ್ನು ಸೇರಿದೆ.

