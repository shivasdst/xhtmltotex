
\chapter[ಭಕ್ತಿಯೋಗ ]{ಭಕ್ತಿಯೋಗ \protect\footnote{\engfoot{C.W. Vol. VI, P 90}}}

ಭಕ್ತಿಯೋಗ ಪರಮಾತ್ಮನೊಡನೆ ಐಕ್ಯವಾಗುವುದಕ್ಕೆ ಇರುವ ಕ್ರಮಬದ್ಧ ಭಕ್ತಿಭಾವದ ಮಾರ್ಗ. ಇದು ಸಾಕ್ಷಾತ್ಕಾರಕ್ಕೆ ಇರುವ ಬಹಳ ಸುಲಭವಾದ ಮಾರ್ಗ, ಸ್ವಾಭಾವಿಕವಾದ ಮಾರ್ಗ. ಈ ದಾರಿಯಲ್ಲಿ ಮುಂದುವರಿಯಬೇಕಾದರೆ ಭಗವತ್ಪ್ರೇಮವೊಂದೇ ಮುಖ್ಯವಾಗಿ ಬೇಕಾಗಿರುವುದು. ಈ ಭಕ್ತಿಯಲ್ಲಿ ಐದು ಮೆಟ್ಟಿಲುಗಳಿವೆ. ಮೊದಲನೆಯದರಲ್ಲಿ ಮನುಷ್ಯನಿಗೆ ಸಹಾಯ ಬೇಕಾಗಿದೆ. ಅವನಲ್ಲಿ ಸ್ವಲ್ಪ ಅಂಜಿಕೆ ಇದೆ. ಎರಡನೆಯದರಲ್ಲಿ ಅವನು ದೇವರನ್ನು ತಂದೆಯಂತೆ ಕಾಣುತ್ತಾನೆ. ಮೂರನೆಯದರಲ್ಲಿ ದೇವರನ್ನು ತಾಯಿಯಂತೆ ಕಾಣುತ್ತಾನೆ. ಆಗ ಅವನಿಗೆ ಎಲ್ಲಾ ಸ್ತ್ರೀಯರಲ್ಲಿ ಆ ಜಗನ್ಮಾತೆಯೇ ಇರುವಂತೆ ತೋರುವುದು. ಮಾತೃಭಾವನೆಯಿಂದ ನಿಜವಾದ ಪ್ರೇಮ ಪ್ರಾರಂಭವಾಗುವುದು. ನಾಲ್ಕನೆಯದೇ ಪ್ರೀತಿಗಾಗಿ ಪ್ರೀತಿ. ಅಹೇತುಕ ಪ್ರೇಮ ಎಲ್ಲಾ ಗುಣಗಳಿಗೂ ಅತೀತವಾಗಿರುವುದು. ಐದನೆಯದೇ ದಿವ್ಯ ಐಕ್ಯತೆ; ಅದು ಅತಿ ಪ್ರಜ್ಞಾಸ್ಥಿತಿಗೆ ಒಯ್ಯುತ್ತದೆ. ನಾವು ಹೇಗೆ ಸಗುಣ ಮತ್ತು ನಿರ್ಗುಣರೊ ಹಾಗೆಯೇ ದೇವರೂ ಕೂಡ ಸಗುಣ ಮತ್ತು ನಿರ್ಗುಣ.

ಪ್ರಾರ್ಥನೆ ಮತ್ತು ಸ್ತೋತ್ರಗಳು ಬೆಳವಣಿಗೆಗೆ ಮೊದಲು ಬಹಳ ಸಹಕಾರಿಗಳು.\break ಭಗವನ್ನಾಮೋಚ್ಚಾರಣೆಯಲ್ಲಿ ಅಗಾಧ ಶಕ್ತಿಯಿದೆ. ಮಂತ್ರ ಎಂದರೆ ಗುರುವು ಶಿಷ್ಯನಿಗೆ\break ಜಪವನ್ನು ಮತ್ತು ಧ್ಯಾನವನ್ನು ಮಾಡುವುದಕ್ಕಾಗಿ ಕೊಡುವ ಬೀಜಾಕ್ಷರ. ಪ್ರಾರ್ಥನೆಯ ಮತ್ತು ಧ್ಯಾನದ ಸಮಯದಲ್ಲಿ ಯಾವುದಾದರೂ ಒಂದು ಆಕಾರದ ಮೇಲೆ ಮನಸ್ಸನ್ನು ಏಕಾಗ್ರಗೊಳಿಸಬೇಕು. ಅದನ್ನೇ ಇಷ್ಟದೇವತೆ ಎನ್ನುವುದು.

ಈ ಮಂತ್ರಗಳು ಬರಿಯ ಶಬ್ದಗಳಲ್ಲ, ಸ್ವಯಂ ಭಗವಂತನೇ ಆಗಿರುವುವು. ಅವು ನಮ್ಮಲ್ಲಿಯೇ ಇರುವುವು. ಅವನನ್ನು ಕುರಿತು ಚಿಂತಿಸಿ, ಅವನನ್ನು ಕುರಿತು ಮಾತನಾಡಿ, ಯಾವ ಪ್ರಾಪಂಚಿಕ ಆಸೆಯೂ ಇಲ್ಲದಿರಲಿ. ಬುದ್ಧನ ಬೋಧನೆಯ ಪಲ್ಲವಿಯೇ ‘ನೀವು ಆಲೋಚಿಸಿದಂತೆ ಆಗುವಿರಿ’ ಎಂಬುದು.

ಸಮಾಧಿಯ ಅನಂತರ ಭಕ್ತನು ಭಗವಂತನನ್ನು ಪ್ರೀತಿಸುವುದಕ್ಕೆ ಮತ್ತು ಪೂಜಿಸುವುದಕ್ಕೆ ಕೆಳಗೆ ಇಳಿದು ಬರುವನು. ನಿಜವಾದ ಪ್ರೀತಿಯಲ್ಲಿ ಯಾವ ನಿರೀಕ್ಷಣೆಯೂ ಇರುವುದಿಲ್ಲ. ಅದು ಪಡೆಯಬೇಕಾದುದೂ ಏನೂ ಇಲ್ಲ.

ಪ್ರಾರ್ಥನೆ ಮತ್ತು ಜಪ ಇವು ಆದ ಮೇಲೆ ಧ್ಯಾನ ಬರುವುದು. ಅನಂತರ ಇಷ್ಟದೇವತೆ ಮತ್ತು ಅದರ ನಾಮ ಇವುಗಳ ಮೇಲೆ ಚಿಂತನ.

ಯಾವ ಭಗವಂತ ನಮ್ಮ ತಾಯಿಯಾಗಿರುವನೋ ತಂದೆಯಾಗಿರುವನೋ ಅವನು ನಮ್ಮ ಬಂಧನಗಳನ್ನು ಖಂಡಿಸಲಿ ಎಂದು ಪ್ರಾರ್ಥಿಸಿ.

“ತಂದೆಯಂತೆ ನನ್ನನ್ನು ಕೈ ಹಿಡಿದು ಕರೆದೊಯ್ಯಿ. ನನ್ನನ್ನು ತ್ಯಜಿಸಬೇಡ. ನನಗೆ ಐಶ್ವರ್ಯ ಬೇಡ. ಸೌಂದರ್ಯ ಬೇಡ. ಇಹಲೋಕ ಬೇಡ. ಪರಲೋಕ ಬೇಡ. ನಿನ್ನಲ್ಲಿ ಶರಣಾಗುವೆನು, ನನ್ನನ್ನು ನಿನ್ನ ಸೇವಕನನ್ನಾಗಿ ಮಾಡು. ನೀನೇ ನನ್ನ ಆಶ್ರಯವಾಗು. ನೀನೇ ನನ್ನ ತಂದೆ, ನನ್ನ ತಾಯಿ, ನನ್ನ ಪ್ರಿಯ ಸಖ. ವಿಶ್ವದ ಭಾರವನ್ನು ಹೊರುವವನು ನೀನು, ನನ್ನ ಜೀವನದ ಅಲ್ಪ ಭಾರವನ್ನು ಸಹಿಸಲು ನೆರವು ನೀಡು. ನಮ್ಮನ್ನು ಮರೆಯಬೇಡ, ನಿನ್ನಿಂದ ಎಂದಿಗೂ ಅಗಲದೆ ಇರುವೆ, ಅನುಗಾಲವೂ ನಿನ್ನಲ್ಲಿಯೇ ನೆಲಸಿರುವೆ” ಎಂದು ಪ್ರಾರ್ಥಿಸಿ.

ಭಗವತ್​ಪ್ರೇಮವು ಪ್ರಾಪ್ತವಾದ ಮೇಲೆ ಅದೊಂದೇ ತಾನೇ ತಾನಾಗಿ ನಿಂತ ಮೇಲೆ, ಈ ಪ್ರಪಂಚ ಕೇವಲ ಒಂದು ಬಿಂದುವಿನಂತೆ ಕಾಣಿಸುವುದು.

ಮೃತ್ಯುವಿನಿಂದ ಅಮೃತತ್ವಕ್ಕೆ ಹೋಗಿ; ತಮಸ್ಸಿನಿಂದ ಜ್ಯೋತಿಗೆ ಹೋಗಿ.

