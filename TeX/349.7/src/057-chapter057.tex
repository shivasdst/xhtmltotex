
\chapter[ಧರ್ಮದ ಪ್ರಮಾಣ ]{ಧರ್ಮದ ಪ್ರಮಾಣ \protect\footnote{\engfoot{C.W. Vol. VI, p 95}}}

ಯಾವುದು ಧರ್ಮವನ್ನು ಅಷ್ಟು ಅವೈಜ್ಞಾನಿಕವನ್ನಾಗಿ ಮಾಡುವುದು ಎಂಬುದೇ ಧರ್ಮಕ್ಕೆ ಸಂಬಂಧಪಟ್ಟ ದೊಡ್ಡದೊಂದು ಪ್ರಶ್ನೆ. ಧರ್ಮ ಒಂದು ವಿಜ್ಞಾನವಾದರೆ ಅದು ಇತರ ವಿಜ್ಞಾನಗಳಂತೆ ಏತಕ್ಕೆ ಖಚಿತವಾಗಿಲ್ಲ? ದೇವರು, ಸ್ವರ್ಗ ಮುಂತಾದುವುಗಳಲ್ಲಿರುವ ನಂಬಿಕೆಯೆಲ್ಲ ಬರೀ ಊಹೆಗಳು ಮಾತ್ರವಾಗಿದೆ. ಅವುಗಳ ವಿಷಯದಲ್ಲಿ ಯಾವ ಖಚಿತತೆಯೂ ಇರುವಂತೆ ಕಾಣಿಸುವುದಿಲ್ಲ. ನಮ್ಮ ಧಾರ್ಮಿಕ ಭಾವನೆಗಳು ಸದಾ ಬದಲಾಗುತ್ತಿರುವಂತೆ ಕಾಣಿಸುವುವು. ಮನಸ್ಸು ಚಿರ ಆಂದೋಳನದಲ್ಲಿರುವಂತೆ ಕಾಣಿಸುವುದು.

ಮನುಷ್ಯ ಒಂದು ಆತ್ಮವೋ ನಿರ್ವಿಕಾರಿಯೋ ಅಥವಾ ಯಾವಾಗಲೂ ಬದಲಾಗುತ್ತಿರುವ ಒಂದು ವಸ್ತುವೋ? ಇನ್ನೂ ಚೆನ್ನಾಗಿ ಅಭಿವೃದ್ಧಿಯಾಗದ ಪುರಾತನ ಬೌದ್ಧಧರ್ಮ ಒಂದನ್ನು ಬಿಟ್ಟರೆ ಇತರ ಧರ್ಮಗಳೆಲ್ಲ ಮಾನವನು ಎಂದು ನಾಶವಾಗದ, ಅಮೃತವಾಗಿರುವ ಒಂದು ಆತ್ಮ, ಒಂದು ವ್ಯಷ್ಟಿ, ಒಂದು ವ್ಯಕ್ತಿ ಎನ್ನುವುವು.

ಅತಿ ಪೂರ್ವಕಾಲದ ಬೌದ್ಧರು ಮಾನವನು ಸದಾ ಬದಲಾಗುತ್ತಿರುವ ಒಂದು ವಸ್ತು ಎನ್ನುವರು. ಅವನ ಪ್ರಜ್ಞೆಯಲ್ಲಿ ಹಲವು ಖಂಡಖಂಡವಾದ, ವೇಗವಾಗಿ ಬದಲಾಗುತ್ತಿರುವ ಘಟನೆಗಳು ಮಾತ್ರ ಇವೆ; ಒಂದು ಘಟನೆಗೂ ಮತ್ತೊಂದು ಘಟನೆಗೂ ಸಂಬಂಧವೇ ಇಲ್ಲ ಎನ್ನುವರು. ಅವರು ಕಾರ್ಯಕಾರಣ ನಿಯಮವನ್ನು ಒಪ್ಪಿಕೊಳ್ಳದೆ ಪ್ರತ್ಯೇಕವಾಗಿ ನಿಂತಿರುವರು.

ಒಂದು ವ್ಯಷ್ಟಿ ಇದ್ದರೆ ಅಲ್ಲಿ ಒಂದು ವಸ್ತುವಿರಬೇಕು. ವ್ಯಷ್ಟಿ ಯಾವಾಗಲೂ ಒಂದು ಸರಳವಾದ ಧಾತು; ಮೂಲಧಾತುವಿನಲ್ಲಿ ಮತ್ತಾವುದೂ ಮಿಶ್ರ ವಾಗಿರುವುದಿಲ್ಲ. ಅದು ಮತ್ತಾವುದರ ಆಸರೆಯ ಮೇಲೂ ನಿಂತಿಲ್ಲ. ಅದು ತಾನೇ ನಿಂತಿರುವುದು, ಅದು ಅವಿನಾಶಿ.

ಹಳೆಯ ಬೌದ್ಧಸಿದ್ಧಾಂತಿಗಳು ಯಾವುದಕ್ಕೂ ಇತರ ವಸ್ತುಗಳ ಸಂಬಂಧವಿಲ್ಲ, ಎಲ್ಲಿಯೂ ಒಂದು ವ್ಯಷ್ಟಿಯಿಲ್ಲ, ಮನುಷ್ಯ ಒಂದು ವ್ಯಷ್ಟಿ ಎನ್ನುವುದು ಕೇವಲ ಒಂದು ನಂಬಿಕೆಯೇ ಹೊರತು ಅದಕ್ಕೆ ಯಾವ ಪ್ರಮಾಣವೂ ಇಲ್ಲ ಎಂದರು.

ಹಾಗಾದರೆ ಮಾನವನು ಒಂದು ವ್ಯಷ್ಟಿಯೇ ಅಥವಾ ಅವನು ಸದಾ ಬದಲಾಗುತ್ತಿರುವನೇ ಎಂಬುದೇ ದೊಡ್ಡ ಪ್ರಶ್ನೆ. ಇದನ್ನು ಪ್ರಮಾಣೀಕರಿಸುವುದಕ್ಕೆ, ಇದಕ್ಕೆ ಉತ್ತರ ಹೇಳುವುದಕ್ಕೆ ಒಂದು ಮಾರ್ಗ ಮಾತ್ರ ಇದೆ. ಮನಸ್ಸಿನ ಚಂಚಲತೆಯನ್ನು ತಡೆಯಿರಿ; ಮನುಷ್ಯನು ಏಕ, ಅವಿಭಾಜ್ಯನು ಎಂಬುದು ಸಿದ್ಧಾಂತಗೊಳ್ಳುತ್ತದೆ. ಬದಲಾವಣೆಯೆಲ್ಲ ಇರುವುದು ಚಿತ್ತದಲ್ಲಿ. ನಾನು ಬದಲಾವಣೆಗಳಲ್ಲ. ಹಾಗಾಗಿದ್ದರೆ ಅವನ್ನು ನಿಲ್ಲಿಸುವುದಕ್ಕೆ ಆಗುತ್ತಿರಲ್ಲಿಲ್ಲ.

ಈ ಜಗತ್ತು ಚೆನ್ನಾಗಿದೆ, ನಾನು ಸುಖಿಯಾಗಿರುವೆನು, ಎಂದು ಪ್ರತಿ ಯೊಬ್ಬರೂ ಭಾವಿಸಲು ಯತ್ನಿಸುತ್ತಿರುವರು. ಆದರೆ ಜೀವನದ ಉದ್ದೇಶಗಳನ್ನೇ ಕುರಿತು ವಿಚಾರಿಸಿದರೆ ಅವನು ಅದಕ್ಕೋ ಇದಕ್ಕೋ ಅನಿವಾರ್ಯವಾಗಿ ಹೋರಾಡು ತ್ತಿರುವನು. ಅವನು ಮುಂದುವರಿದು ಹೋಗಬೇಕಾಗಿದೆ. ಅವನು ತೆಪ್ಪಗಿರಲಾರ. ಆದಕಾರಣ ಏನೋ ತನಗೆ ಬೇಕಾಗಿದೆ ಎಂದು ಅವನು ತನ್ನನ್ನು ತಾನು ನಂಬಿಕೊಳ್ಳುತ್ತಾನೆ. ಯಾರು ಆರೋಗ್ಯವಂತರೋ ಅವನು ಎಲ್ಲವೂ ಸರಿಯಾಗಿದೆ, ಚೆನ್ನಾಗಿದೆ ಎಂದು ಭಾವಿಸುವನು. ಮನುಷ್ಯನು ತನ್ನ ಮನಸ್ಸಿನಲ್ಲಿ ಆಸೆ ಅಂಕುರಿಸಿ ದೊಡನೆಯೇ ಪ್ರಶ್ನೆಯಿಲ್ಲದೆ ಅದನ್ನು ತೃಪ್ತಿಪಡಿಸಿಕೊಳ್ಳುವನು. ಯಾವುದೋ ಅವನನ್ನು ಪ್ರೇರೇಪಿಸುತ್ತದೆ; ಆದ ಕಾರಣ ಅವನು ಯೋಚನೆಯೇ ಮಾಡದೆ ಆ ಕೆಲಸಮಾಡಲು ಉದ್ಯುಕ್ತನಾಗುವನು. ಆದರೆ ತನಗೆ ಇಚ್ಛೆಯಾದುದರಿಂದ ತಾನು ಮಾಡುತ್ತೇನೆ ಎಂದು ಭಾವಿಸುತ್ತಾನೆ. ಆದರೆ ಪ್ರಗತಿಯ ಆಘಾತಕ್ಕೆ ಹಲವುವೇಳೆ ಸಿಕ್ಕಿ ಜರ್ಝರಿತನಾಗಿ ಗಾಯ ನೋವುಗಳಾದ ಮೇಲೆ ಇದನ್ನೆಲ್ಲ ಕುರಿತು ಆಲೋಚಿ ಸುವನು. ಹೆಚ್ಚು ವ್ಯಥೆಗೀಡಾದ ಮೇಲೆ, ಬೇಕಾದಷ್ಟು ಆಲೋಚನೆ ಮಾಡಿದ ಮೇಲೆ, ಇದಕ್ಕೆಲ್ಲಾ ಏನು ಅರ್ಥ ಎಂದು ಚಿಂತಿಸಲು ಪ್ರಾರಂಭಿಸುವನು. ಆಗ ತನ್ನ ಸ್ವಾಧೀನದಲ್ಲಿಲ್ಲದ ಯಾವುದೋ ಒಂದು ಆವೇಗಕ್ಕೆ ಸಿಕ್ಕಿ ಆ ಕೆಲಸ ಮಾಡ ಬೇಕಾಯಿತು, ಹಾಗೆ ಮಾಡದೇ ವಿಧಿಯೇ ಇರಲಿಲ್ಲ, ಎಂದು ಅರಿಯುವನು. ಆಗ ಅವನು ಪ್ರತಿಭಟಿಸುವನು; ಹೋರಾಟ ಪ್ರಾರಂಭವಾಗುವುದು.

ಈ ತೊಂದರೆಯಿಂದ ಪಾರಾಗುವುದಕ್ಕೆ ಒಂದು ಮಾರ್ಗವಿದ್ದರೆ ಅದು ನಮ್ಮೊಳಗೆಯೇ ಇದೆ. ನಾವು ಯಾವಾಗಲೂ ಸತ್ಯವನ್ನು ಅರಿಯುವುದಕ್ಕೆ ಪ್ರಯತ್ನಿಸು ತ್ತಿರುವೆವು. ನಾವು ಯಾವಾಗಲೂ ಅದನ್ನು ಸಹಜವಾಗಿಯೇ ಮಾಡುತ್ತಿರುವೆವು. ಮಾನವನ ಮನಸ್ಸಿನ ಸೃಷ್ಟಿಗಳೇ ದೇವರನ್ನು ಮರೆಮಾಡುವುವು. ಆದಕಾರಣವೇ ದೇವರನ್ನು ಕುರಿತಾದ ಭಾವನೆಗಳಲ್ಲಿ ಅಷ್ಟೊಂದು ವ್ಯತ್ಯಾಸ ಇರುವುದು. ಸೃಷ್ಟಿನಿಂತಾದ ಮೇಲೆ ಮಾತ್ರ ನಾವು ಆತ್ಮನನ್ನು ಕಾಣುವೆವು. ಆತ್ಮ ಹೊರಗಿನ ಸೃಷ್ಟಿಯಲ್ಲಿ ಇಲ್ಲ; ಆಂತರ್ಯದಲ್ಲಿ ಇರುವುದು. ನಾವು ಸೃಷ್ಟಿಯನ್ನೇ ನಿಲ್ಲಿಸಿದಾಗ ಆತ್ಮನನ್ನು ಅರಿಯುವೆವು. ನಾವು ನಮ್ಮನ್ನು ಕುರಿತು ಆಲೋಚಿಸುವಾಗ ನಾವು ದೇಹವೆಂದು ಬಗೆಯುವೆವು. ನಾವು ದೇವರನ್ನು ಕುರಿತು ಆಲೋಚಿಸುವಾಗಲೂ ಅವನನ್ನು ದೇಹವೆಂದೇ ಭಾವಿಸುವೆವು. ಆತ್ಮವು ಕಾಣಿಸಬೇಕಾದರೆ ಚಿತ್ತಸರೋವರ ಶಾಂತವಾಗಿರಬೇಕು. ಚಿತ್ತದ ಕ್ಷೋಭೆಯನ್ನು ನಿಲ್ಲಿಸುವುದೇ ಸಾಧನೆ. ಇದು ದೇಹದಿಂದ ಪ್ರಾರಂಭವಾಗುವುದು. ಪ್ರಾಣಾಯಾಮ ದೇಹಕ್ಕೆ ತರಬೇತನ್ನು ಕೊಡುವುದು; ಅದನ್ನು ಒಂದು ಸಮಾಧಾನ ಸ್ಥಿತಿಗೆ ತರುವುದು. ಪ್ರಾಣಾಯಾಮದ ಗುರಿ ಏಕಾಗ್ರತೆಯನ್ನು ಪಡೆಯುವುದು ಮತ್ತು ಧ್ಯಾನ. ನೀವು ಒಂದು ಕ್ಷಣ ವಾದರೂ ಮನಸ್ಸನ್ನು ಸ್ವಸ್ಥಗೊಳಿಸುವಿರಾದರೆ ನೀವು ಗುರಿಯನ್ನು ಪಡೆದಂತೆ. ಮನಸ್ಸು ಅನಂತರವೂ ಕೆಲಸಮಾಡುತ್ತಿರಬಹುದು. ಆದರೆ ಅದು ಹಿಂದಿನ ಮನಸ್ಸಾಗಿರುವುದಿಲ್ಲ. ನಿಮ್ಮ ಮನಸ್ಸನ್ನು ಒಂದು ನಿಮಿಷ ಶಾಂತಗೊಳಿಸಿದರೆ ನಿಮ್ಮ ನೈಜಸ್ವಭಾವ ಹೊಳೆಯುವುದು, ಸ್ವಾತಂತ್ರ್ಯ ನಿಮ್ಮದಾಗುವುದು. ನಿಮಗೆ ಇನ್ನು ಬಂಧನವಿಲ್ಲ. ನೀವು ಕಾಲದಲ್ಲಿ ಒಂದು ಕ್ಷಣವನ್ನು ಅರಿತರೆ ಇಡೀ ಕಾಲವನ್ನೆಲ್ಲಾ ಅರಿಯಬಹುದು ಎಂಬ ನಿಯಮದಿಂದ ಇದು ಬರುವುದು. ಏಕೆಂದರೆ ಇಡೀ ಕಾಲ ಆ ಒಂದು ಕ್ಷಣದ ಅವಿಚ್ಛಿನ್ನ ಪುನರಾವೃತ್ತಿ ಅಷ್ಟೆ. ಒಂದನ್ನು ವಶಮಾಡಿ ಕೊಂಡರೆ, ಒಂದು ಕ್ಷಣವನ್ನು ಚೆನ್ನಾಗಿ ಅರಿತರೆ ಸ್ವಾತಂತ್ರ್ಯವನ್ನು ಪಡೆದಂತೆ.

ಪ್ರಾಚೀನ ಬೌದ್ಧರ ವಿನಾ ಇತರರೆಲ್ಲ ದೇವರನ್ನು ಮತ್ತು ಆತ್ಮವನ್ನು ನಂಬುವರು. ಆಧುನಿಕ ಬೌದ್ಧರು ದೇವರನ್ನು ಮತ್ತು ಆತ್ಮವನ್ನು ನಂಬುವರು. ಬರ್ಮಾ ಸಯಾಂ ಚೈನ ಮುಂತಾದ ದೇಶದವರು ಪ್ರಾಚೀನ ಕಾಲದ ಬೌದ್ಧರ ಪಂಥಕ್ಕೆ ಸೇರಿದವರು.

ಆರ‌್ನಾ‌ಲ್ಡ್ ನ ‘ಏಷ್ಯಾದ ಜ್ಯೋತಿ’ ಪುಸ್ತಕವು ಬೌದ್ಧಧರ್ಮಕ್ಕಿಂತ ಹೆಚ್ಚಾಗಿ ವೇದಾಂತದ ಭಾವನೆಗಳನ್ನು ವಿವರಿಸುವುದು.

