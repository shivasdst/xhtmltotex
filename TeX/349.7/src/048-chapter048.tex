
\chapter[ಈಶ್ವರ ಮತ್ತು ಬ್ರಹ್ಮ ]{ಈಶ್ವರ ಮತ್ತು ಬ್ರಹ್ಮ \protect\footnote{\engfoot{C.W. Vol. V, p269}}}

ವೇದಾಂತ ದರ್ಶನದಲ್ಲಿ ಈಶ್ವರನ ಸ್ಥಾನವೇನು ಎಂಬ ಪ್ರಶ್ನೆಗೆ ಸ್ವಾಮಿ ವಿವೇಕಾನಂದರು ಯೂರೋಪಿನಲ್ಲಿರುವಾಗ ಈ ಕೆಳಗಿನ ವಿವರಣೆಯನ್ನು ನೀಡಿದರು.

ಈಶ್ವರನು ವ್ಯಕ್ತಿಗಳ ಮೊತ್ತ. ಆದರೂ ಅವನೊಂದು ವ್ಯಕ್ತಿ. ಮಾನವ ದೇಹದಲ್ಲಿ ಹಲವು ಜೀವಕಣಗಳಿವೆ. ಆದರೆ ಆ ದೇಹವೂ ಒಂದು ವ್ಯಕ್ತಿ. ಸಮಷ್ಟಿಯೇ ದೇವರು, ವ್ಯಷ್ಟಿಯೇ ಜೀವ. ಈಶ್ವರ ಇರಬೇಕಾದರೆ ಜೀವಿಗಳು ಇರಬೇಕು; ದೇಹವಿರಬೇಕಾದರೆ ಜೀವಕಣಗಳು \enginline{(cells)} ಇರಬೇಕಾದಂತೆ. ಜೀವ ಮತ್ತು ಈಶ್ವರ ಇವರಲ್ಲಿ ಒಂದು ಇದ್ದರೆ ಮತ್ತೊಂದು ಇರಬೇಕಾಗುವುದು. ಎರಡೂ ಒಟ್ಟಿಗೆಯೇ ಇರುವುವು. ನಮ್ಮ ಪ್ರಪಂಚವೊಂದನ್ನು ಬಿಟ್ಟರೆ ಇತರ ಪುಣ್ಯ ಲೋಕಗಳಲ್ಲಿ ಒಳ್ಳೆಯದು ಕೆಟ್ಟದ್ದಕ್ಕಿಂತ ಹೆಚ್ಚಾಗಿ ಇರುವುದರಿಂದ ಸಮಷ್ಟಿಯಾದ ಈಶ್ವರ ಮಂಗಳಮಯ ಎನ್ನಬಹುದು. ಸಮಷ್ಟಿಯ ದೃಷ್ಟಿಯಿಂದ ಸರ್ವಜ್ಞತೆ ಮತ್ತು ಸರ್ವಶಕ್ತಿ ಸ್ವತಸ್ಸಿದ್ಧವಾದುವು. ಇದನ್ನು ಸಮರ್ಥಿಸಲು ಚರ್ಚೆಯೇ ಬೇಕಿಲ್ಲ. ಬ್ರಹ್ಮ ಇವುಗಳಾಚೆ ಇರುವುದು. ಅದು ಅವ್ಯಕ್ತ. ಅನೇಕ ಅಂಶಗಳಿಂದ ಕೂಡಿಕೊಂಡಿದರೆ\break ಇರುವಂತಹದು. ಅದೊಂದೇ ಎಲ್ಲಾ ವಸ್ತುಗಳಲ್ಲಿಯೂ ಜೀವಾಣುವಿನಿಂದ ಈಶ್ವರನವರೆಗೆ ಓತಪ್ರೋತವಾಗಿರುವ ತತ್ತ್ವ. ಇದಿಲ್ಲದೇ ಯಾವ ವಸ್ತುವೂ ಇರಲಾರದು. ಪ್ರಪಂಚದಲ್ಲಿ ಸತ್ಯವಾಗಿರುವುದೆಲ್ಲ ತತ್ತ್ವ ಅಥವಾ ಬ್ರಹ್ಮವೊಂದೇ. ನಾನು ಬ್ರಹ್ಮವೆಂದು ಭಾವಿಸಿದರೆ ನಾನೊಬ್ಬನೇ ಇರುವೆನು. ಇದರಂತೆಯೇ ಎಲ್ಲರೂ ಕೂಡ. ಆದಕಾರಣ ಪ್ರತಿಯೊಬ್ಬರೂ ಆ ಇಡೀ ತತ್ತ್ವವೇ ಆಗಿರುವರು.

