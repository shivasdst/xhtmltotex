
\chapter[ಜಗತ್ತಿಗೆ ಬುದ್ಧನ ಸಂದೇಶ ]{ಜಗತ್ತಿಗೆ ಬುದ್ಧನ ಸಂದೇಶ \protect\footnote{\engfoot{C.W. Vol. IV. P. 92}}}

\centerline{\textbf{(೧೯೦೦ರ ಮಾರ್ಚ್​ ೧೯ರಂದು ಸ್ಯಾನ್​ಫ್ರಾನ್ಸಿಸ್ಕೊದಲ್ಲಿ ನೀಡಿದ ಉಪನ್ಯಾಸ)}}

ಚಾರಿತ್ರಿಕ ದೃಷ್ಟಿಯಿಂದ ಬೌದ್ಧಧರ್ಮ ಅತಿ ಮುಖ್ಯವಾದ ಧರ್ಮ. ಚಾರಿತ್ರಿಕ ದೃಷ್ಟಿಯಿಂದ, ತಾತ್ತ್ವಿಕ ದೃಷ್ಟಿಯಿಂದಲ್ಲ. ಏಕೆಂದರೆ ಪ್ರಪಂಚದಲ್ಲೆಲ್ಲಾ ಆದ ಅದ್ಭುತವಾದ ಧಾರ್ಮಿಕ ಜಾಗೃತಿ ಇದು. ಮಾನವ ಸಮಾಜವನ್ನೆಲ್ಲಾ ಆವರಿಸಿದ ಅದ್ಭುತ ಆಧ್ಯಾತ್ಮಿಕ ತರಂಗ ಇದು. ಒಂದಲ್ಲ ಒಂದು ರೀತಿಯಲ್ಲಿ ಇದರ ಪ್ರಭಾವಕ್ಕೆ ಒಳಗಾಗದ ನಾಗರೀಕತೆಯೇ ಇಲ್ಲ.

ಬುದ್ಧನ ಅನುಯಾಯಿಗಳು ಬಹಳ ಉತ್ಸಾಹಶಾಲಿಗಳು ಮತ್ತು ತಮ್ಮ ಧರ್ಮದ ಸಂಖ್ಯಾ ಬಲವನ್ನು ವಿಸ್ತರಿಸುವುದರಲ್ಲಿ ಕುತೂಹಲಿಗಳೂ ಆಗಿದ್ದರು. ಕೇವಲ ತಮ್ಮ ಧರ್ಮ ಇದ್ದ ದೇಶದಲ್ಲೇ ತೃಪ್ತರಾಗದವರಲ್ಲಿ ಇವರು ಮೊತ್ತ ಮೊದಲಿಗರು. ಅವರು ಬಹಳ ದೂರದೇಶಗಳಿಗೆ ಹೋದರು, ಚತುರ್ದಿಕ್ಕುಗಳಲ್ಲಿಯೂ ಹರಡಿದರು. ಅಗಮ್ಯವಾದ ಟಿಬೆಟ್ಟಿಗೆ ಹೋದರು. ಪರ್ಷಿಯಾ, ಏಷ್ಯಾಮೈನರ್​, ರಷ್ಯ ಮತ್ತು ಪಾಶ್ಚಾತ್ಯ ದೇಶಗಳಲ್ಲಿ ಪೋಲೆಂಡ್​ ಮುಂತಾದ ಕಡೆಗಳಲ್ಲಿ ಹರಡಿದರು. ಚೈನಾ, ಜಪಾನ್​, ಕೊರಿಯ, ಬರ್ಮಾ, ಸಯಾಂ, ಈಸ್ಟ್​ ಇಂಡೀಸ್​ ಮತ್ತು ಅದರಾಚೆಗೆ ಹೋದರು. ಅಲೆಕ್ಸಾಂಡರ್​ ತನ್ನ ಸೇನೆಯಿಂದ ಸಾಧಿಸಿದ ಜಯದ ಮೂಲಕ ಮೆಡಿಟರೇನಿಯನ್​ ಮತ್ತು ಭರತಖಂಡಕ್ಕೆ ಸಂಬಂಧವನ್ನು ಕಲ್ಪಿಸಿದಾಗ ಭರತಖಂಡದ ಜ್ಞಾನವು ಏಷ್ಯಾದ ಬಹುಭಾಗಗಳಿಗೂ, ಯುರೋಪಿಗೂ ಹರಡಲು ಒಂದು ಅವಕಾಶ ದೊರಕಿತು. ಬೌದ್ಧ ಭಿಕ್ಷುಗಳು ಬೇರೆ ಬೇರೆ ದೇಶಗಳಲ್ಲಿ ಬೋಧಿಸುತ್ತ ಹೋದರು. ಅವರ ಬೋಧನೆಯಿಂದ ಮೂಢನಂಬಿಕೆ ಮತ್ತು ಪೌರೋಹಿತ್ಯ ಸೂರ್ಯನೆದುರಿಗಿರುವ ಹಿಮದಂತೆ ಕರಗಿಹೋದವು.

ಈ ಚಳುವಳಿಯನ್ನು ನಾವು ಚೆನ್ನಾಗಿ ತಿಳಿದುಕೊಳ್ಳಬೇಕಾದರೆ ಆಗಿನ ಭರತ ಖಂಡದ ಸ್ಥಿತಿಯನ್ನು ಅರಿತುಕೊಳ್ಳಬೇಕು. ನಾವು ಕ್ರೈಸ್ತಧರ್ಮವನ್ನು ತಿಳಿದುಕೊಳ್ಳಬೇಕಾದರೆ ಆಗಿನ ಕಾಲದಲ್ಲಿ ಯೆಹೂದ್ಯರ ಸಮಾಜ ಹೇಗಿತ್ತು ಎಂಬುದನ್ನು ತಿಳಿದುಕೊಳ್ಳಬೇಕಾಗಿರುವಂತೆ, ಕ್ರಿಸ್ತ ಹುಟ್ಟುವುದಕ್ಕೆ ಆರುನೂರು ವರುಷಗಳ ಮುಂಚೆ ಹಿಂದೂ ಸಮಾಜ ಹೇಗಿತ್ತು ಎಂಬುದನ್ನೂ ತಿಳಿದುಕೊಳ್ಳಬೇಕು. ಆ ಸಮಯಕ್ಕೆ ಭಾರತೀಯ ನಾಗರಿಕತೆ ತನ್ನ ಬೆಳವಣಿಗೆಯ ಪೂರ್ಣತೆಯನ್ನು ಮುಟ್ಟಿತ್ತು.

ಭಾರತೀಯ ನಾಗರಿಕತೆಯನ್ನು ನೀವು ಅಧ್ಯಯನ ಮಾಡಿದರೆ ಎಷ್ಟೋ ವೇಳೆ ಅದು ಮೃತ್ಯುಮುಖವಾಗಿ, ಪುನಃ ಚೇತರಿಸಿಕೊಂಡಿರುವುದನ್ನು ನೋಡುವೆವು. ಅದೇ ಅದರ ವೈಶಿಷ್ಟ್ಯ. ಅನೇಕ ಜನಾಂಗಗಳು ಒಮ್ಮೆ ಉಚ್ಛ್ರಾಯದ ಶಿಖರವನ್ನು ಏರಿ ಅನಂತರ ಅವನತಿಗೆ ಇಳಿದು ನಾಶವಾಗಿ ಹೋಗುವುವು. ಎರಡು ಬಗೆಯ ಜನಗಳಿರುವರು. ಒಂದೇ ಸಮನಾಗಿ ಬೆಳೆಯುತ್ತಿರುವವರು ಕೆಲವರು; ಬೆಳವಣಿಗೆ ಒಂದು ಹಂತಕ್ಕೆ ಬಂದು ನಿಂತು ಹೋಗುವವರು ಕೆಲವರು. ಇಂಡಿಯಾ, ಚೈನಾ ಮುಂತಾದ ಶಾಂತಪ್ರಿಯ ದೇಶಗಳು ಕೆಳಗೆ ಬೀಳುವುವು. ಆದರೆ ಪುನಃ ಏಳುವುವು. ಆದರೆ ಮತ್ತೆ ಕೆಲವು ಪ್ರದೇಶಗಳು ಒಮ್ಮೆ ಕೆಳಗೆ ಬಿದ್ದುವೆಂದರೆ ಪುನಃ ಮೇಲೇಳುವಂತೆಯೇ ಇಲ್ಲ. ಅವು ನಾಶವಾಗುವುವು. ಪ್ರಪಂಚದಲ್ಲಿ ಶಾಂತಿ ಸಂದೇಶವನ್ನು ಹರಡುವವರೇ ಧನ್ಯರು. ಅವರೇ ಎಂದೆಂದಿಗೂ ಉಳಿಯುವರು.

ಬುದ್ಧನು ಹುಟ್ಟಿದ ಸಮಯದಲ್ಲಿ ಭರತಖಂಡಕ್ಕೆ ಒಬ್ಬ ದೊಡ್ಡ ಅಧ್ಯಾತ್ಮಿಕ ನಾಯಕನ, ಪ್ರವಾದಿಯ ಅವಶ್ಯಕತೆಯಿತ್ತು. ಆಗ ಕೆಲವು ಪ್ರಭಾವಶಾಲಿಗಳಾದ ಪುರೋಹಿತರಿದ್ದರು. ಯಹೂದ್ಯರ ಚರಿತ್ರೆಯನ್ನು ನೆನೆಸಿಕೊಂಡರೆ ಈ ಪರಿಸ್ಥಿತಿ ಚೆನ್ನಾಗಿ ಅರ್ಥವಾಗುವುದು. ಯಹೂದ್ಯರಲ್ಲಿ ಎರಡು ಬಗೆಯ ಧಾರ್ಮಿಕ ಮುಂದಾಳುಗಳು ಇದ್ದರು: ಪುರೋಹಿತರು ಮತ್ತು ಪ್ರವಾದಿಗಳು. ಪುರೋಹಿತರು ಜನರನ್ನು ಅಜ್ಞಾನದಲ್ಲಿಟ್ಟು ಮೌಢ್ಯವನ್ನು ಹರಡುತ್ತಿದ್ದರು. ಪುರೋಹಿತರು ಹೇಳುತ್ತಿದ್ದ ಪೂಜಾ ಪದ್ಧತಿ ಜನರನ್ನು ತಮ್ಮ ಹತೋಟಿಯಲ್ಲಿ ಇಟ್ಟುಕೊಳ್ಳುವುದಕ್ಕೆ ಒಂದು ಮಾರ್ಗವಾಗಿತ್ತು. ಹಳೆಯ ಟೆಸ್ಟಮೆಂಟಿನಲ್ಲೆಲ್ಲಾ ಪ್ರವಾದಿಗಳು ಪುರೋಹಿತರ ಮೂಢನಂಬಿಕೆಗಳನ್ನು ಧಿಕ್ಕರಿಸುವುದನ್ನು ನೋಡು\-ತ್ತೇವೆ. ಈ ಹೋರಾಟದ ಫಲವಾಗಿ ಪುರೋಹಿತರು ಸೋತು ಪ್ರವಾದಿಗಳು ಗೆಲ್ಲುತ್ತಾರೆ.

ಪುರೋಹಿತರು ಒಬ್ಬ ದೇವರಿರುವನೆಂದು ನಂಬುವರು. ಆದರೆ ಈ ದೇವರನ್ನು ತಮ್ಮ ಮೂಲಕ ಮಾತ್ರ ನೋಡಬಹುದು ಎನ್ನುತ್ತಾರೆ. ಜನರು ಪುರೋಹಿತರ ಅಪ್ಪಣೆ ಪಡೆದರೆ ಮಾತ್ರ ಭಗವಂತನ ಸನ್ನಿಧಿಗೆ ಹೋಗಬಲ್ಲರು. ನೀವು ಅವರಿಗೆ ಹಣ ಕೊಡಬೇಕು, ಪೂಜೆ ಸಲ್ಲಿಸಬೇಕು. ಎಲ್ಲವನ್ನೂ ಅವರಿಗೆ ಸಮರ್ಪಿಸಬೇಕು. ಇತಿಹಾಸದಲ್ಲೆಲ್ಲಾ ಇದೇ ಪುರೋಹಿತ ಪ್ರವೃತ್ತಿ ಪದೇ ಪದೇ ಮೇಲೆದ್ದಿರುವುದು ಕಾಣುತ್ತದೆ. ಅಧಿಕಾರದ ಮೇಲೆ ಅಧಿಕ ಆಸೆ ಅವರಿಗೆ. ವ್ಯಾಘ್ರದಂತಹ ಈ ರಕ್ತದಾಹ ಮಾನವನ ಒಂದು ಸ್ವಭಾವದಂತೆ ಕಾಣುವುದು. ಪುರೋಹಿತರು ನಿಮ್ಮನ್ನು ಆಳುವರು, ನೂರಾರು ಆಚಾರಗಳಿಂದ ಬಲಾತ್ಕಾರವಾಗಿ ಬಿಗಿಯುವರು. ಅತಿ ಸರಳವಾದ ಸತ್ಯವನ್ನು, ತಿಳಿಯದ ಒಂದು ಜಾಲವಾಗಿ ಮಾಡುವರು. ತಮ್ಮ ಶ್ರೇಷ್ಠತೆಗಳನ್ನು ತೋರಿಸಿಕೊಳ್ಳುವುದಕ್ಕೆ ಕಥೆಗಳನ್ನು ಹೇಳುವರು. ನೀವು ಇಹದಲ್ಲಿ ಚೆನ್ನಾಗಿರಬೇಕಾದರೆ, ಪರದಲ್ಲಿ ಸ್ವರ್ಗಕ್ಕೆ ಹೋಗಬೇಕಾದರೆ, ಅವರ ಮೂಲಕ ಸಾಗಬೇಕು. ಹಲವು ಬಗೆಯ ಕರ್ಮಗಳನ್ನು ಮಾಡಬೇಕು. ಇವೆಲ್ಲ ನಮ್ಮ ಜೀವನವನ್ನು ಒಂದು ಜಟಿಲತೆಗೆ ತಂದಿದೆ. ನಮ್ಮ ತಿಳುವಳಿಕೆಯನ್ನು ಕೆಡಿಸಿದೆ. ಇದರಿಂದ ಏನಾಗಿದೆ ಎಂದರೆ, ನಿಮಗೆ ಸರಳವಾಗಿ ಹೇಳಿದರೆ ನೀವು ಅತೃಪ್ತರಾಗಿ ಮನೆಗೆ ಹೋಗುವಿರಿ. ನಿಮ್ಮ ತಿಳುವಳಿಕೆಯೆಲ್ಲಾ ಕದಡಿಹೋಗಿದೆ. ಕಡಿಮೆ ಅರ್ಥವಾದಷ್ಟೂ ಮೇಲೆಂದು ಭಾವಿಸುವಿರಿ. ಪ್ರವಾದಿಗಳು ಪುರೋಹಿತರ ತಂತ್ರ, ಯಂತ್ರ, ಮೂಢ ನಂಬಿಕೆಗಳಿಗೆ ವಿರೋಧವಾಗಿ ಎಚ್ಚರಿಕೆ ನೀಡುವರು ಆದರೆ ಬಹುಪಾಲು ಜನ ಈ ಎಚ್ಚರಿಕೆಗೆ ಗಮನ ಕೊಡುವುದಿಲ್ಲ. ಅವರಿಗೆ ವಿದ್ಯಾಭ್ಯಾಸ ಇನ್ನೂ ಬರಬೇಕಾಗಿದೆ.

ಜನರಿಗೆ ವಿದ್ಯಾಭ್ಯಾಸಬೇಕು. ಈ ದಿನಗಳಲ್ಲಿ ‘ಡೆಮಾಕ್ರಸಿ’, ಜನರೆಲ್ಲಾ ಒಂದೇ ಸಮ ಎಂದು ಮಾತನಾಡುವರು. ಒಬ್ಬ ತಾನು ಇತರರಿಗೆ ಸಮಾನ ಎಂದು ಹೇಗೆ ತಿಳಿಯಬಲ್ಲ? ಅವನಿಗೆ ಬುದ್ಧಿ ಚುರುಕಾಗಿರಬೇಕು, ಮೂಢನಂಬಿಕೆಗಳಿಂದ ಪಾರಾದ ಸ್ಪಷ್ಟವಾದ ಮನಸ್ಸಿರಬೇಕು. ತನ್ನ ಮನಸ್ಸಿನಲ್ಲಿ ಕವಿದುಕೊಂಡಿರುವ ಮೂಢ ನಂಬಿಕೆಯ ಪದರಗಳನ್ನೆಲ್ಲಾ ತೂರಿಹೋಗಿ ತನ್ನ ಅಂತರಾಳದಲ್ಲಿರುವ ಸತ್ಯವನ್ನು ಅರಿಯಬೇಕು. ಆಗ ಎಲ್ಲಾ ಪೂರ್ಣತೆ, ಶಕ್ತಿ ತನ್ನಲ್ಲಿಯೇ ಇರುವುದು, ಯಾರೋ ಹೊರಗಿನವರು ಇವನ್ನು ತನಗೆ ಕೊಟ್ಟಿದ್ದಲ್ಲವೆಂಬುದನ್ನು ಅರಿಯುವನು. ಅವನಿಗೆ ಇದು ಗೊತ್ತಾದರೆ ತಕ್ಷಣ ಅವನು ಸ್ವತಂತ್ರನಾಗುವನು, ಆಗ ಅವನು ಸಮಾನತೆಯನ್ನು ಪಡೆಯುವನು. ಆಗ ಅವನಿಗೆ ಎಲ್ಲರೂ ತನ್ನಂತೆಯೇ ಸಮಾನರು, ಪರಿಪೂರ್ಣರು ಎಂಬುದು ಗೊತ್ತಾಗುವುದು. ಇತರ ಸಹೋದರರ ಮೇಲೆ ದೈಹಿಕವಾಗಿ ಆಗಲಿ, ಮಾನಸಿಕವಾಗಿ ಆಗಲಿ, ನೈತಿಕವಾಗಿ ಆಗಲಿ ಯಾವ ಬಲಾತ್ಕಾರವನ್ನು ಹೇರಕೂಡದು ಎಂಬುದನ್ನು ಅರಿಯುವನು. ತನಗಿಂತ ಕೀಳಾದವರು ಎಂದಾದರೂ ಯಾರಾದರೂ ಇದ್ದರು ಎಂಬ ಭಾವನೆಯನ್ನು ಅವನು ತೊರೆಯುವನು. ಆಗ ಮಾತ್ರ ಅವನು ಸಮಾನತೆಯ ವಿಷಯವನ್ನು ಮಾತನಾಡಬಹುದು. ಅದಕ್ಕೆ ಮುಂಚೆ ಅಲ್ಲ.

ನಾನು ನಿಮಗೆ ಹೇಳುತ್ತಿದ್ದಂತೆ ಯಹೂದ್ಯರಲ್ಲಿ ಪ್ರವಾದಿಗಳಿಗೂ ಪುರೋಹಿತರಿಗೂ ಯಾವಾಗಲೂ ಭಿನ್ನಾಭಿಪ್ರಾಯಗಳಿದ್ದವು. ಪುರೋಹಿತರು ಅಧಿಕಾರ ಮತ್ತು ಜ್ಞಾನವನ್ನೆಲ್ಲಾ ತಾವೇ ಇಟ್ಟುಕೊಳ್ಳಲು ಪ್ರಯತ್ನಿಸಿದರು. ಕೊನೆಗೆ ಅವರೇ ಅವನ್ನು ಕಳೆದುಕೊಂಡರು. ಇತರರಿಗೆ ತೊಡಿಸಿದ ಸರಪಳಿ ತಮಗೇ ಪ್ರಾಪ್ತವಾಯಿತು. ಯಜಮಾನ ಬೇಗ ಭೃತ್ಯನಾಗುವನು. ಈ ಹೋರಾಟದ ಕೊನೆಯಲ್ಲಿ ನಜರತ್ತಿನ ಏಸುವು ಜಯಶೀಲನಾಗಿ ನಿಂತನು. ಈ ದಿಗ್ವಿಜಯವೇ ಕ್ರೈಸ್ತಧರ್ಮದ ಇತಿಹಾಸ. ಕೊನೆಗೆ ಕ್ರಿಸ್ತನು ಮೂಢ ನಂಬಿಕೆಗಳನ್ನೆಲ್ಲಾ ಮೂಲೋತ್ಪಾಟನೆ ಮಾಡಿ ವಿಜಯಿಯಾದ. ಈ ಮಹಾತ್ಮ ಪೌರೋಹಿತ್ಯವೆಂಬ ದೌರ್ಜನ್ಯದ ಘಟಸರ್ಪವನ್ನು ನಾಶ ಮಾಡಿ, ಸತ್ಯ ಮಾಣಿಕ್ಯವನ್ನು ಅದರ ದಾಡೆಯಿಂದ ರಕ್ಷಿಸಿ ಜಗತ್ತಿಗೆ ನೀಡಿದನು. ಯಾರಿಗೆ ಅದು ಬೇಕಾಗಿರುವುದೋ ಅವರು ಇದನ್ನು ಸ್ವೇಚ್ಛೆಯಿಂದ ಇಟ್ಟುಕೊಳ್ಳಬಹುದು. ಯಾವ ಪುರೋಹಿತನಿಗೂ ಅದಕ್ಕಾಗಿ ಶರಣಾಗಬೇಕಾಗಿಲ್ಲ.

ಯೆಹೂದ್ಯರು ಎಂದಿಗೂ ತಾತ್ತ್ವಿಕ ಜನಾಂಗವಾಗಿರಲಿಲ್ಲ. ಅವರಿಗೆ ಭಾರತೀಯರಿಗಿದ್ದ ಬುದ್ಧಿಸೂಕ್ಷ್ಮತೆಯಾಗಲಿ, ಮನೋಶಕ್ತಿಯಾಗಲೀ ಇರಲಿಲ್ಲ. ಭರತಖಂಡದ ಪುರೋಹಿತರಾದ ಬ್ರಾಹ್ಮಣರಿಗೆ ಅದ್ಭುತ ಪಾಂಡಿತ್ಯವಿತ್ತು. ಮನೋಶಕ್ತಿಯಿತ್ತು. ಅವರೇ ಭರತಖಂಡದಲ್ಲಿ ಆಧ್ಯಾತ್ಮಿಕ ಪ್ರಗತಿಯನ್ನು ಪ್ರಾರಂಭಿಸಿದರು. ಅವರು ಅದ್ಭುತವಾದ ವಿಷಯಗಳನ್ನು ಸಾಧಿಸಿದರು. ಆದರೆ ಕ್ರಮೇಣ ಮೊದಲು ಬ್ರಾಹ್ಮಣರನ್ನು ಪ್ರಚೋದಿಸಿದ ವಿಕಾಸದ ಉದಾರ ಭಾವನೆ ಮಾಯವಾಯಿತು. ಅವರೇ ಅಧಿಕಾರ ಮತ್ತು ಹಕ್ಕುಗಳನ್ನು ದೋಚಲೆತ್ನಿಸಿದರು. ಬ್ರಾಹ್ಮಣ ಮತ್ತೊಬ್ಬನನ್ನು ಕೊಂದರೆ ಅದಕ್ಕೆ ಶಿಕ್ಷೆಯಿಲ್ಲ. ಬ್ರಾಹ್ಮಣ ಹುಟ್ಟಾ ಜಗದೊಡೆಯ! ಅತಿ ದುರಾಚಾರಿಯಾದ ಬ್ರಾಹ್ಮಣನನ್ನು ಕೂಡ ಪೂಜಿಸಬೇಕು!

ಪುರೋಹಿತರು ಉಚ್ಛ್ರಾಯ ಸ್ಥಿತಿಯಲ್ಲಿದ್ದಾಗ ಸಂನ್ಯಾಸಿಗಳೆಂಬ ಋಷಿ ಮಹಾತ್ಮರೂ ಇದ್ದರು. ಹಿಂದೂಗಳು ಯಾವ ವರ್ಣಕ್ಕೇ ಸೇರಿರಲಿ ಅವರು ಆಧ್ಯಾತ್ಮಿಕ ಸಂಪನ್ನರಾಗ\-ಬೇಕಾದರೆ ಕರ್ಮವನ್ನು ತ್ಯಜಿಸಿ ಮೃತ್ಯುವನ್ನು ಎದುರುಗೊಳ್ಳಲು ಸಿದ್ಧರಾಗಬೇಕು. ಅವರಿಗೆ ಪ್ರಪಂಚದ ಮೇಲೆ ಇನ್ನು ಆಸಕ್ತಿ ಇರಕೂಡದು. ಅವರು ಪ್ರಪಂಚವನ್ನು ತ್ಯಜಿಸಿ ಸಂನ್ಯಾಸಿಗಳಾಗಬೇಕು. ಪುರೋಹಿತರು ಜಾರಿಗೆ ತಂದಿರುವ ಸಾವಿರಾರು ಆಚಾರಗಳಿಗೂ ಸಂನ್ಯಾಸಿಗಳಿಗೂ ಏನೂ ಸಂಬಂಧವಿಲ್ಲ. ವಿಚಿತ್ರ ಧಾರ್ಮಿಕ ಕ್ರಿಯೆಗಳು, ಹಲವು ಅಕ್ಷರಗಳ ಮಂತ್ರವನ್ನು ಉಚ್ಚರಿಸುವುದು – ಇವೆಲ್ಲ ಸಂನ್ಯಾಸಿಗಳ ಪಾಲಿಗೆ ಅರ್ಥಹೀನ.

ಭರತಖಂಡದ ಋಷಿವರ್ಯರು ಪುರೋಹಿತರ ಆಚಾರಗಳನ್ನು ಅಲ್ಲಗಳೆದು ಸತ್ಯವನ್ನು ಬೋಧಿಸಿದರು. ಅವರು ಪುರೋಹಿತರ ದರ್ಪವನ್ನು ಅಡಗಿಸಲು ಯತ್ನಿಸಿ ಅದರಲ್ಲಿ ಸ್ವಲ್ಪ ಜಯಶೀಲರಾದರು. ಆದರೆ ಎರಡು ತಲೆಮಾರಿನ ಮೇಲೆ ಅವರ ಅನುಯಾಯಿಗಳೇ ಮೂಢಾಚಾರಕ್ಕೆ ವಶರಾಗಿ, ಪುರೋಹಿತರಿಗೆ ಶರಣಾಗಿ, ಅವರೇ ಪುರೋಹಿತರಾದರು. “ನಮ್ಮ ಮೂಲಕ ಮಾತ್ರ ನೀವು ಸತ್ಯವನ್ನು ಪಡೆಯಿರಿ” ಎಂದರು. ಪುನಃ ಸತ್ಯ ಘನೀಭೂತವಾಯಿತು. ಆಗ ಪುನಃ ದೇವದೂತರು ಅವತಾರವೆತ್ತಿ ಸುತ್ತಲೂ ಕವಿದ ಮೂಢನಂಬಿಕೆಗಳನ್ನು ಒಡೆದು ಸತ್ಯದ ಸ್ವತಂತ್ರ ಚಲನೆಗೆ ಅವಕಾಶ ಮಾಡಿದರು. ಹೀಗೇ ಮುಂದುವರಿಯಿತು. ಯಾವಾಗಲೂ ಒಬ್ಬ ಮಹಾತ್ಮ ವ್ಯಕ್ತಿ ಬೇಕಾಗಿದೆ. ಇಲ್ಲದೆ ಇದ್ದರೆ, ಮಾನವ ಕೋಟಿ ನಿರ್ನಾಮವಾಗುವುದು.

ಪುರೋಹಿತರ ಇಷ್ಟೊಂದು ಕಂದಾಚಾರಗಳೆಲ್ಲ ಏತಕ್ಕೆ ಇರಬೇಕೆಂದು ನೀವು ಆಶ್ಚರ್ಯಪಡಬಹುದು. ನೀವೇಕೆ ನೇರವಾಗಿ ಸತ್ಯದೆಡೆಗೆ ಬರಬಾರದು? ಭಗವತ್​ ಸತ್ಯವನ್ನು ಕಂಡರೆ ನಿಮಗೆ ನಾಚಿಕೆಯೆ? ಅದಕ್ಕಾಗಿ ನೀವು ಅದನ್ನು ಕಂದಾಚಾರ ಮೂಢನಂಬಿಕೆಯಲ್ಲಿ ಬಚ್ಚಿಡಲು ಯತ್ನಿಸುವಿರಾ? ಭಗವತ್​ ಸತ್ಯವನ್ನು, ಪ್ರಪಂಚದ ಎದುರು ಒಪ್ಪಿಕೊಳ್ಳದಷ್ಟು ನಾಚಿಕೆಯೇ ನಿಮಗೆ ದೇವರ ವಿಷಯದಲ್ಲಿ? ನೀವು ಅಂತಹವರನ್ನು ಧಾರ್ಮಿಕ ಮತ್ತು ಆಧ್ಯಾತ್ಮಿಕ ಜೀವಿ ಎಂದು ಕರೆಯುವಿರಾ? ಪುರೋಹಿತರು ಮಾತ್ರ ಸತ್ಯಕ್ಕೆ ಅಣಿಯಾಗಿರುವುದೆ, ಜನಸಾಧಾರಣರು ಅದಕ್ಕೆ ಅಣಿಯಾಗಿಲ್ಲವೆ? ಸತ್ಯದ ಶಕ್ತಿಯನ್ನು ಇನ್ನೂ ಕಡಿಮೆಮಾಡಬೇಕು, ಅದಕ್ಕೆ ಮತ್ತಷ್ಟು ನೀರನ್ನು ಮಿಶ್ರಮಾಡಿ!

ಸರ್ಮನ್​ ಆನ್​ ದಿ ಮೌಂಟ್​ (ಕ್ರಿಸ್ತನ ಪರ್ವತ ಪ್ರವಚನ) ಮತ್ತು ಭಗವದ್ಗೀತೆಯನ್ನು ತೆಗೆದುಕೊಳ್ಳಿ. ಅವು ಬಹಳ ಸರಳವಾಗಿವೆ. ದಾರಿಹೋಕನಿಗೂ ಅವು ಅರ್ಥವಾಗುತ್ತವೆ. ಅವು ಎಷ್ಟು ಅದ್ಭುತವಾಗಿವೆ! ಅವುಗಳಲ್ಲಿ ಸತ್ಯವು ಸರಳವಾಗಿ, ಸುಲಭವಾಗಿ ವ್ಯಕ್ತವಾಗಿದೆ. ಆದರೆ ಇಷ್ಟು ನೇರವಾಗಿ, ಸುಲಭವಾಗಿ ಸತ್ಯ ಸಿಕ್ಕುವುದೆಂಬುದನ್ನು ಪುರೋಹಿತರು ಒಪ್ಪಿಕೊಳ್ಳುವುದಿಲ್ಲ. ಎರಡು ಸಾವಿರ ಸ್ವರ್ಗನರಕಗಳನ್ನು ಅವರು ಸೃಷ್ಟಿಸಿರುವರು. ಜನರು ಇವರು ಹೇಳಿದಂತೆ ಕೇಳಿದರೆ ಸ್ವರ್ಗಕ್ಕೆ ಹೋಗುವರು. ಇಲ್ಲದೆ ಇದ್ದರೆ ಅವರು ನರಕಕ್ಕೆ ಹೋಗುವರು.

ಆದರೆ ಜನರು ಸತ್ಯವನ್ನು ತಿಳಿಯುವರು. ಎಲ್ಲರಿಗೂ ಪೂರ್ಣ ಸತ್ಯವನ್ನು ಕೊಡುವುದು ಅಪಾಯಕರ ಎಂದು ಕೆಲವರಿಗೆ ಅಂಜಿಕೆ. ಜನರಿಗೆ ಅಪ್ಪಟ ಸತ್ಯವನ್ನು ಕೊಡಕೂಡದೆಂದು ಅವರ ಮತ. ಸತ್ಯವನ್ನು ಮರೆಮಾಡಿದುದರಿಂದ ಪ್ರಪಂಚ ಏನೂ ಮೇಲಾಗಿಲ್ಲ. ಈಗ ಇರುವುದಕ್ಕಿಂತ ಇನ್ನೆಷ್ಟು ಹದಗೆಡುವುದು ಸಾಧ್ಯ? ಸತ್ಯವನ್ನು ಪ್ರಕಾಶಕ್ಕೆ ತನ್ನಿ. ಅದು ಯಥಾರ್ಥವಾದುದಾದರೆ ಅದರಿಂದ ಮೇಲಾಗುವುದು. ಜನರು ಅದನ್ನು ಒಪ್ಪದೆ ಬೇರೆ ಮಾರ್ಗವನ್ನು ಸೂಚಿಸಿದರೆ ಅವರು ಮಂತ್ರ ಮಾಟಗಳಿಗೆ ಅವಕಾಶ ಕೊಟ್ಟಂತಾಗುವುದು.

ಬುದ್ಧನ ಕಾಲದಲ್ಲಿ ಇಂತಹ ಜನರು ಭರತಖಂಡದಲ್ಲಿ ತುಂಬಿದ್ದರು. ಜನಸಾಮಾನ್ಯನಿಗೆ ಯಾವ ಜ್ಞಾನಕ್ಕೂ ಹಕ್ಕಿರಲಿಲ್ಲ. ವೇದದ ಒಂದು ಪದ ಅವನ ಕಿವಿಗೆ ಬಿದ್ದರೆ ಘೋರ ದಂಡನೆ ಅವನಿಗೆ ಕಾದಿತ್ತು. ಪುರಾತನ ಹಿಂದೂಗಳು ಕಂಡುಹಿಡಿದ ಆಧ್ಯಾತ್ಮಿಕ ಸತ್ಯಗಳನ್ನೊಳಗೊಂಡ ವೇದವನ್ನು ರಹಸ್ಯವಾಗಿಟ್ಟರು ಪುರೋಹಿತರು.

ಕೊನೆಗೆ ಇದನ್ನು ಸಹಿಸಲು ಆಗದ ಒಬ್ಬನು ಕಾಣಿಸಿಕೊಂಡ. ಅವನಿಗೆ ಬುದ್ಧಿ ಇತ್ತು. ಶಕ್ತಿ ಇತ್ತು. ವಿಶಾಲವಾದ ಆಕಾಶದಷ್ಟು ಅಸೀಮವಾಗಿತ್ತು ಅವನ ಹೃದಯ. ಜನರು ಪುರೋಹಿತರನ್ನು ಅನುಸರಿಸುತ್ತಿದ್ದುದನ್ನು ಅವನು ಕಂಡ. ಪುರೋಹಿತರು ಅಧಿಕಾರದ ಮದದಲ್ಲಿ ಮುಳುಗಿ ಹೋಗಿದ್ದರು. ಇದನ್ನು ಹೇಗಾದರೂ ನಿವಾರಿಸಬೇಕೆಂದು ಅವನು ಯತ್ನಿಸಿದ. ಯಾರ ಮೇಲೂ ಅವನಿಗೆ ಅಧಿಕಾರ ಬೇಕಿರಲಿಲ್ಲ. ಜನರ ಧಾರ್ಮಿಕ ಮತ್ತು ಆಧ್ಯಾತ್ಮಿಕ ಬಂಧನಗಳನ್ನು ಕಳಚಲು ಅವನು ಯತ್ನಿಸಿದ. ಅವನ ಹೃದಯ ವಿಶಾಲವಾಗಿತ್ತು. ನಮ್ಮ ಸುತ್ತಲಿನ ಹಲವರಿಗೆ ಹೃದಯ ಇರಬಹುದು. ನಾವೂ ಇತರರಿಗೆ ಸಹಾಯ ಮಾಡಲು ಯತ್ನಿಸಬಹುದು. ಆದರೆ ನಮಗೆ ಬುದ್ಧಿಯಿಲ್ಲ. ಇತರರಿಗೆ ಹೇಗೆ ಸಹಾಯ ನೀಡಬಹುದೋ ಅಂತಹ ಮಾರ್ಗ ಗೊತ್ತಿಲ್ಲ. ಆದರೆ ಆತ ಜೀವಿಗಳ ಬಂಧನವನ್ನು ಹೇಗೆ ನಾಶಗೊಳಿಸಬಹುದು ಎಂಬ ಮಾರ್ಗವನ್ನು ಕಂಡುಹಿಡಿದ. ಮನುಷ್ಯನು ಏತಕ್ಕೆ ವ್ಯಥೆಪಡುವನು ಎಂಬುದನ್ನು ಕಂಡುಹಿಡಿದ. ಅದರಿಂದ ಪಾರಾಗುವುದಕ್ಕೆ ಮಾರ್ಗವನ್ನು ಕಂಡುಹಿಡಿದ. ಅವನು ಮಹಾಜ್ಞಾನಿ, ಅನುಭಾವಿ. ಎಲ್ಲವನ್ನೂ ಅವನು ಅನುಷ್ಠಾನಕ್ಕೆ ತಂದ. ಭೇದಭಾವವಿಲ್ಲದೆ ಅದನ್ನು ಎಲ್ಲರಿಗೂ ಬೋಧಿಸಿ ನಿರ್ವಾಣಸುಖವನ್ನು ಹೇಗೆ ಪಡೆಯಬೇಕೆಂಬುದನ್ನು ತೋರಿದ. ಅವನೇ ಬುದ್ಧ.

ಸರ್​ ಎಡ್ವಿನ್​ ಆರ್ನಾಲ್ಡ್​ ಎಂಬುವನು ಬರೆದ ‘ಏಷ್ಯಾದ ಜ್ಯೋತಿ’ ಎಂಬ ಪುಸ್ತಕದಲ್ಲಿ ಬುದ್ಧ ರಾಜಕುಮಾರನಾಗಿ ಹುಟ್ಟಿದ್ದು, ಪ್ರಪಂಚದ ದುಃಖ ಅವನಿಗೆ ತಟ್ಟಿದ್ದು, ಅವನು ಭೋಗದ ತೊಡೆಯ ಮೇಲೆ ಬೆಳೆದಿದ್ದರೂ ಕೇವಲ ತನ್ನ ಸುಖದಿಂದ ಮತ್ತು ರಕ್ಷಣೆಯಿಂದ ಅವನಿಗೆ ಆನಂದವಾಗದೆ ಇದ್ದದ್ದು – ಇವನ್ನೆಲ್ಲ ನೀವು ಓದಿರುವಿರಿ. ಅವನು ಹೆಂಡತಿಯನ್ನು ಮತ್ತು ಆಗತಾನೆ ಹುಟ್ಟಿದ ಮಗುವನ್ನು ತ್ಯಜಿಸಿ, ತ್ಯಾಗಿಯಾಗಿ ಹೋಗಿದ್ದೂ ನಿಮಗೆ ಗೊತ್ತಿದೆ. ಒಬ್ಬ ಗುರುಗಳಿಂದ ಮತ್ತೊಬ್ಬ ಗುರುಗಳ ಬಳಿಗೆ ಹೋಗಿ ಸತ್ಯವನ್ನು ಅರಸುತ್ತ ಆತ ಅಲೆದಾಡತೊಡಗಿದ. ಕೊನೆಗೆ ಅವನು ನಿರ್ವಾಣವನ್ನು ಪಡೆದ. ಅವನು ದೀರ್ಘಕಾಲದವರೆಗೆ ಪ್ರಸಾರ ಮಾಡಿದ ಸಂದೇಶ, ಅವನ ಶಿಷ್ಯರು ಮತ್ತು ಸಂಸ್ಥೆ – ಇವೆಲ್ಲ ನಿಮಗೆ ಗೊತ್ತಿದೆ.

ಪುರೋಹಿತರಿಗೂ, ಪ್ರವಾದಿಗಳಿಗೂ ನಡೆಯುತ್ತಿದ್ದ ತಿಕ್ಕಾಟದಲ್ಲಿ ಬುದ್ಧ ಜಯಶೀಲನಾಗಿದ್ದನು. ಆದರೆ ಇಂಡಿಯಾದೇಶದ ಪುರೋಹಿತರ ಪರವಾಗಿ ನಾವು ಇದೊಂದನ್ನು ಹೇಳಬಹುದು. ಅವರಿಗೆ ಎಂದಿಗೂ ಧರ್ಮದ ವಿಷಯದಲ್ಲಿ ಅಸಹನೀಯ ಭಾವನೆ ಇರಲಿಲ್ಲ. ಅವರೆಂದಿಗೂ ಧರ್ಮಕ್ಕಾಗಿ ಜನರನ್ನು ಹಿಂಸಿಸಲಿಲ್ಲ. ಯಾರೇ ಆದರೂ ತಮ್ಮ ವಿರೋಧವಾಗಿ ಬೋಧಿಸಬಹುದಾಗಿತ್ತು. ಅವರದು ಅಂತಹ ಧರ್ಮ, ಯಾರನ್ನೂ ಅವರು, ಅವರು ಹೊಂದಿದ್ದ ಧಾರ್ಮಿಕ ಭಾವನೆಗಾಗಿ ಹಿಂಸಿಸಿದವರಲ್ಲ. ಆದರೆ ಬಹುಮಟ್ಟಿಗೆ ಎಲ್ಲಾ ಪುರೋಹಿತರಲ್ಲಿರುವ ದುರ್ಬಲತೆಯೊಂದು ಅವರಲ್ಲಿಯೂ ಇತ್ತು. ಅವರೂ ಅಧಿಕಾರವನ್ನು ಬಯಸಿದರು. ಅವರು ಹಲವು ಆಚಾರಗಳನ್ನೂ, ನಿಯಮಗಳನ್ನೂ ಬಳಕೆಗೆ ತಂದರು. ಧರ್ಮವನ್ನು ಅನಾವಶ್ಯಕವಾಗಿ ಜಟಿಲ ಸಮಸ್ಯೆಯಾಗಿ ಮಾಡಿ ತಮ್ಮ ಧರ್ಮದ ಅನುಯಾಯಿಗಳ ಶಕ್ತಿಯನ್ನು ತಗ್ಗಿಸಿದರು.

ಬುದ್ಧನು ಈ ಅನಾವಶ್ಯಕವಾದ ಕಳೆಯನ್ನೆಲ್ಲಾ ಕಿತ್ತ. ಅವನು ಅದ್ಭುತವಾದ ಸತ್ಯಗಳನ್ನು ಬೋಧಿಸಿದ. ಯಾವ ಭೇದಭಾವವೂ ಇಲ್ಲದೆ ವೇದಗಳ ತತ್ತ್ವಸಾರವನ್ನೇ ಎಲ್ಲರಿಗೂ ಬೋಧಿಸಿದ. ಅವನು ಇಡೀ ಪ್ರಪಂಚಕ್ಕೆ ಅದನ್ನು ಬೋಧಿಸಿದ. ಮಾನವರೆಲ್ಲ ಸಮಾನರು ಎಂಬುದು ಅವನು ಬೋಧಿಸಿದ ಘನ ಸಂದೇಶಗಳಲ್ಲಿ ಒಂದಾಗಿತ್ತು. ಮನುಷ್ಯರೆಲ್ಲರೂ ಸಮಾನರು. ಯಾರಿಗೂ ವಿಶೇಷ ಹಕ್ಕಿರಬೇಕಾಗಿಲ್ಲ. ಸಮಾನತೆಯ ಸಂದೇಶವನ್ನು ಸಾರಿದ ಮಹಾಗುರು ಪ್ರತಿಯೊಬ್ಬ ಸ್ತ್ರೀಪುರುಷರಿಗೂ ಆಧ್ಯಾತ್ಮಿಕ ಜೀವನದಲ್ಲಿ ಮುಂದುವರಿಯಲು ಸಮಾನವಾದ ಅಧಿಕಾರವಿದೆ ಎಂದ. ಇದೇ ಅವನ ಸಂದೇಶ. ಬ್ರಾಹ್ಮಣರಿಗೂ, ಇತರ ವರ್ಣದವರಿಗೂ ಇದ್ದ ವ್ಯತ್ಯಾಸವನ್ನು ಅವನು ತೊಡೆದುಹಾಕಿದ. ಅತಿ ನೀಚ ಜಾತಿಯವರು ಕೂಡ ಅತಿ ಶ್ರೇಷ್ಠತೆಗೆ ಅರ್ಹರಾದರು. ಎಲ್ಲರಿಗೂ ಆತ ನಿರ್ವಾಣದ ಬಾಗಿಲನ್ನು ತೆರೆದ. ಭರತಖಂಡದ ಜನರಿಗೆ ಕೂಡ ಅವನ ಸಂದೇಶ ಬಹಳ ಸಾಹಸದ್ದಾಗಿ ಕಾಣಿಸಿತು. ಎಂಥ ಧಾರ್ಮಿಕ ಬೋಧನೆಯನ್ನಾದರೂ ಭಾರತೀಯ ಅರಗಿಸಿಕೊಳ್ಳಬಲ್ಲ. ಆದರೂ ಭಾರತೀಯರಿಗೆ ಬುದ್ಧನ ಸಂದೇಶವನ್ನು ಒಪ್ಪಿಕೊಳ್ಳಲು ಕಷ್ಟವಾಯಿತು. ನಿಮಗೆ ಮತ್ತೆಷ್ಟು ಕಷ್ಟವಾಗಬಹುದು!

ಅವನ ಸಿದ್ಧಾಂತ ಇದು: ನಮ್ಮ ಜೀವನದಲ್ಲಿ ಏಕೆ ದುಃಖವಿದೆ? ಏಕೆಂದರೆ ನಾವು ಸ್ವಾರ್ಥಿಗಳು. ನಮಗಾಗಿ ನಾವು ವಸ್ತುಗಳನ್ನು ಆಶಿಸುತ್ತೇವೆ. ಅದಕ್ಕೇ ದುಃಖವಿರುವುದು. ಇದರಿಂದ ಪಾರಾಗುವುದಕ್ಕೆ ಉಪಾಯವೇನು? ಅಹಂಕಾರವನ್ನು ತ್ಯಜಿಸುವುದು. ಜೀವಾತ್ಮವೆಂಬುದು ಇಲ್ಲ. ನಮ್ಮೆದುರಿಗೆ ತೋರಿಕೆಯ ಜಗತ್ತು ಮಾತ್ರ ಇರುವುದು. ಜನನ–ಮರಣಗಳ ಹಿಂದೆ ಆತ್ಮವೆಂಬುದು ಯಾವುದೂ ಇಲ್ಲ. ಒಂದು ಆಲೋಚನಾ ಪ್ರವಾಹವಿದೆ. ಒಂದಾದ ಮೇಲೊಂದು ಆಲೋಚನೆ ಬರುತ್ತಿರುವುದು ಅಷ್ಟೇ. ಆಲೋಚಿಸುವವನು, ಅಥವಾ ಜೀವಾತ್ಮ ಇಲ್ಲ. ದೇಹ ಯಾವಾಗಲೂ ಬದಲಾಗುತ್ತಿದೆ. ಹಾಗೆಯೇ ಮನಸ್ಸು ಮತ್ತು ಪ್ರಜ್ಞೆಯೂ ಕೂಡ. ಆದಕಾರಣ ಆತ್ಮವೆಂಬುದೊಂದು ಭ್ರಾಂತಿ. ನಾವು ಈ ತೋರಿಕೆಯ ಅಹಂಕಾರಕ್ಕೆ ಅಂಟಿಕೊಂಡಿರುವುದೇ ನಮ್ಮ ಸ್ವಾರ್ಥಕ್ಕೆಲ್ಲ ಮೂಲ. ಜೀವಾತ್ಮ ಎಂಬುದಿಲ್ಲ ಎಂಬುದು ನಮಗೆ ಗೊತ್ತಾದರೆ, ಆಗ ನಾವು ಸುಖಿಗಳಾಗುವೆವು. ಇತರರಿಗೂ ಸುಖವನ್ನು ನೀಡಬಲ್ಲೆವು.

ಬುದ್ಧ ಸಾರಿದ್ದು ಇದನ್ನು. ಅವನು ಕೇವಲ ಮಾತಾಳಿಯಲ್ಲ. ಅವನು ಜಗತ್ತಿಗಾಗಿ ತನ್ನ ಪ್ರಾಣವನ್ನು ಅರ್ಪಿಸಲು ಸಿದ್ಧನಾಗಿದ್ದ. “ಪ್ರಾಣಿಬಲಿಯು ಒಳ್ಳೆಯದಾದರೆ ನರಬಲಿಯು ಮತ್ತೂ ಒಳ್ಳೆಯದು” ಎಂದು ತನ್ನ ಪ್ರಾಣವನ್ನೇ ಆಹುತಿ ಕೊಡಲು ಸಿದ್ಧನಾಗಿದ್ದ. “ಪ್ರಾಣಿಬಲಿಯೆಂಬುದೊಂದು ಮೂಢನಂಬಿಕೆ. ಜೀವ, ದೇವರು ಎಂಬುದು ಮತ್ತೊಂದು ದೊಡ್ಡ ಮೂಢನಂಬಿಕೆ. ದೇವರೆಂಬುದು ಪುರೋಹಿತರು ನಿರ್ಮಿಸಿದ ಮತ್ತೊಂದು ಮೂಢನಂಬಿಕೆ. ಈ ಬ್ರಾಹ್ಮಣರು ಹೇಳುವಂತೆ ಒಬ್ಬ ದೇವರು ಇದ್ದರೆ ಪ್ರಪಂಚದಲ್ಲಿ ಏತಕ್ಕೆ ಇಷ್ಟೊಂದು ದುಃಖವಿದೆ? ನನ್ನಂತೆ ಕಾರ್ಯಕಾರಣ ಬಂಧನಕ್ಕೆ ಆತನೂ ಒಬ್ಬ ಗುಲಾಮ. ಅವನು ಕಾರ್ಯಕಾರಣ ಬಂಧನದಿಂದ ಬಂಧಿತನಲ್ಲವಾದರೆ ಅವನೇತಕ್ಕೆ ಸೃಷ್ಟಿಸಬೇಕು? ಇಂತಹ ದೇವರ ಕಲ್ಪನೆ ಎಂದಿಗೂ ತೃಪ್ತಿಕರವಲ್ಲ. ತಮ್ಮ ಇಚ್ಛೆ ಬಂದಂತೆ ಯಾರೋ ಸ್ವರ್ಗದಲ್ಲಿ ಕುಳಿತು ಆಳುತ್ತಿರುವರು. ನಾವೆಲ್ಲ ಇಲ್ಲಿ ದುಃಖದಲ್ಲಿ ಸಿಕ್ಕಿ ಸಾಯಬೇಕು. ಅವನಿಗೆ ನಮ್ಮ ಕಡೆ ಒಂದು ನಿಮಿಷ ದೃಷ್ಟಿ ಹಾಯಿಸುವ ಸೌಜನ್ಯವಿಲ್ಲ. ನಮ್ಮ ಇಡೀ ಜೀವನ ಕಷ್ಟಮಯ. ಆದರೆ ಇಲ್ಲಿಗೇ ಇದು ಕೊನೆಗಾಣುವಂತೆ ಇಲ್ಲ. ಸತ್ತ ಮೇಲೆ ಇನ್ನೂ ಕೆಲವು ಶಿಕ್ಷೆಗಳನ್ನು ಅನುಭವಿಸುವುದಕ್ಕೆ ಬೇರೆ ಸ್ಥಳಕ್ಕೆ ಹೋಗಬೇಕಂತೆ. ಆದರೂ ಇಂತಹ ಸೃಷ್ಟಿಕರ್ತನನ್ನು ಮೆಚ್ಚಿಸಲು ಯಜ್ಞ ಯಾಗಗಳನ್ನು ಮಾಡಿದ್ದೂ ಮಾಡಿದ್ದೆ!”

ಬುದ್ಧನು ಇನ್ನೂ ಹೇಳಿದ: “ಈ ಆಚಾರಗಳೆಲ್ಲಾ ಭ್ರಾಂತಿ. ಪ್ರಪಂಚದಲ್ಲಿ ಒಂದೇ ಒಂದು ಆದರ್ಶವಿದೆ. ಭ್ರಾಂತಿಗಳನ್ನೆಲ್ಲಾ ನಿರ್ಮೂಲ ಮಾಡಿ, ಸತ್ಯವಾದುದು ಕೊನೆಗೆ ಉಳಿಯುವುದು. ಮೋಡಗಳು ಚದುರಿದೊಡನೆ ಸೂರ್ಯ ಪ್ರಕಾಶಿಸುತ್ತಾನೆ.” ಈ ಅಹಂಕಾರವನ್ನು ಕೊಲ್ಲುವುದು ಹೇಗೆ? ಸಂಪೂರ್ಣ ನಿಃಸ್ವಾರ್ಥಿಗಳಾಗಿ, ಒಂದು ಇರುವೆಗಾಗಿಯೂ ನಿಮ್ಮ ಪ್ರಾಣವನ್ನು ಅರ್ಪಿಸಲು ಸಿದ್ಧರಾಗಿ. ಯಾವ ಮೂಢ ನಂಬಿಕೆಗಾಗಿಯೂ ಕೆಲಸ ಮಾಡಬೇಡಿ. ಯಾವ ಪ್ರತಿಫಲಕ್ಕಾಗಿಯೂ ಕೆಲಸ ಮಾಡಬೇಡಿ. ನಿಮ್ಮ ಅಹಂಕಾರವನ್ನು ಅಳಿಸಿ ನಿರ್ವಾಣವನ್ನು ಪಡೆಯಲು ಮಾತ್ರ ಕೆಲಸ ಮಾಡಿ. ಪ್ರಾರ್ಥನೆ, ಪೂಜೆ ಇವೆಲ್ಲ ಕೆಲಸಕ್ಕೆ ಬಾರದವು. ನೀವೆಲ್ಲ “ದೇವರಿಗೆ ಧನ್ಯವಾದ” ಎನ್ನುವಿರಿ. ಆದರೆ ಅವನೆಲ್ಲಿರುವನು? ನಿಮಗದು ತಿಳಿಯದು. ಆದರೂ ನೀವು ಅವನ ವಿಷಯದಲ್ಲಿ ಹುಚ್ಚರಾಗಿ ಹೋಗುತ್ತಿರುವಿರಿ.

ಹಿಂದೂಗಳು ದೇವರನ್ನು ವಿನಃ ಮತ್ತೇನನ್ನೂ ಬೇಕಾದರೂ ತ್ಯಜಿಸಬಲ್ಲರು. ದೇವರನ್ನು ಅಲ್ಲಗಳೆಯುವುದು ಭಕ್ತಿಯ ಅಡಿಪಾಯವನ್ನೇ ತೆಗೆದು ಹಾಕಿದಂತೆ. ಭಕ್ತಿ ಮತ್ತು ದೇವರು ಇವೆರಡನ್ನೂ ಹಿಂದೂಗಳು ಎಂದಿಗೂ ತ್ಯಜಿಸಲಾರರು. ಆದರೆ ಬುದ್ಧನ ಬೋಧನೆಯಲ್ಲಿ ದೇವರಿಲ್ಲ. ಆತ್ಮವಿಲ್ಲ. ಬರಿಯ ಕೆಲಸವಿದೆ. ಕೆಲಸ ಏತಕ್ಕೆ? ಆತ್ಮನಿಗಾಗಿ ಅಲ್ಲ. ಆತ್ಮ ಎಂಬುದೊಂದು ಭ್ರಾಂತಿ. ಈ ಭ್ರಾಂತಿ ಹೋದರೆ ನಮ್ಮ ನಿಜಸ್ಥಿತಿ ಅರಿವಾಗುವುದು. ಇಷ್ಟು ಮೇಲೇರಿ ಕೇವಲ ಕೆಲಸಕ್ಕಾಗಿ ಕೆಲಸವನ್ನು ಮಾಡುವವರು ಪ್ರಪಂಚದಲ್ಲಿ ಬಹಳ ವಿರಳ.

ಆದರೂ ಬೌದ್ಧಧರ್ಮ ಬಹಳ ತೀವ್ರವಾಗಿ ಹಬ್ಬಿತು. ಇದು ಅದರಲ್ಲಿದ್ದ ಅದ್ಭುತ ಪ್ರೇಮದ ದೆಸೆಯಿಂದ. ಮಾನವ–ಇತಿಹಾಸದಲ್ಲಿ ಮೊದಲನೆಯ ಬಾರಿ ಮಾನವ ಹೃದಯದಿಂದ ಅದ್ಭುತ ಪ್ರೇಮವು ಉಕ್ಕಿ ಹರಿದು ಕೇವಲ ಮಾನವನ ಸೇವೆಯಲ್ಲಿ ಮಾತ್ರವಲ್ಲ, ಎಲ್ಲ ಪ್ರಾಣಿಗಳ ಸೇವೆಯಲ್ಲಿಯೂ ತಲ್ಲೀನವಾಯಿತು; ಎಲ್ಲರೂ ದುಃಖದಿಂದ ಪಾರಾಗಬೇಕೆಂಬುದಲ್ಲದೆ ಮತ್ತಾವುದನ್ನೂ ಅದು ಲೆಕ್ಕಿಸಿಲ್ಲ.

ಮಾನವ ದೇವರನ್ನು ಪ್ರೀತಿಸುತ್ತಿದ್ದ, ಆದರೆ ತನ್ನ ಸಹೋದರ ಮಾನವನನ್ನೇ ಮರೆತಿದ್ದ. ದೇವರ ಹೆಸರಿನಲ್ಲಿ ತನ್ನ ಪ್ರಾಣವನ್ನೇ ತೆರಲು ಸಿದ್ಧನಾಗಬಲ್ಲ ಮನುಷ್ಯ, ದೇವರ ಹೆಸರಿನಲ್ಲಿ ತನ್ನ ಸಹೋದರ ಮಾನವರನ್ನು ಕೊಲ್ಲಲೂ ಸಿದ್ಧನಾಗಬಲ್ಲ. ಪ್ರಪಂಚದ ಸ್ಥಿತಿ ಹೀಗಿತ್ತು. ಜನರು ದೇವರಿಗಾಗಿ ತಮ್ಮ ಮಕ್ಕಳನ್ನೇ ಬಲಿಕೊಡಲು ಸಿದ್ಧರಾಗಿದ್ದರು. ದೇವರಿಗಾಗಿ ಅನ್ಯ ರಾಷ್ಟ್ರಗಳನ್ನು ಸೂರೆ ಮಾಡುತ್ತಿದ್ದರು. ದೇವರಿಗಾಗಿ ಸಾವಿರಾರು ಜನರನ್ನು ಬಲಿಕೊಡಲು ಸಿದ್ಧರಾಗಿದ್ದರು. ಮಾನವನೆಂಬ ಇನ್ನೊಬ್ಬ ದೇವರ ಕಡೆ ತಿರುಗಿದ್ದು ಇದೇ ಮೊದಲನೆಯ ಬಾರಿ. ಪ್ರೀತಿಸಬೇಕಾಗಿರುವುದು ಮಾನವರನ್ನು. ಜೀವಿಗಳನ್ನೆಲ್ಲಾ ಪ್ರೇಮದಿಂದ ಅಪ್ಪುವ ಪ್ರಪ್ರಥಮ ಪ್ರವಾಹ ಇದು. ಸ್ವಲ್ಪವೂ ಬೆರಕೆಯಾಗದ ಈ ಶುದ್ಧ ಧರ್ಮ ಭರತ ಖಂಡದಿಂದ ಪ್ರಾರಂಭವಾಗಿ ಕ್ರಮೇಣ ಚತುರ್ದಿಕ್ಕುಗಳನ್ನೂ ವ್ಯಾಪಿಸಿತು.

ಈ ಮಹಾಗುರುವು, ಸತ್ಯವನ್ನು ಸತ್ಯವಾಗಿ ತೋರುವಂತೆ ಮಾಡಲು ಯತ್ನಿಸಿದ. ಮೃದುತ್ವ ಇಲ್ಲ, ರಾಜಿಯಿಲ್ಲ, ಪುರೋಹಿತ ವರ್ಗದ ಅಧಿಕಾರಿಗಳು ಮತ್ತು ರಾಜರು ಹೇಳಿದಂತೆ ಕೇಳುವುದಿರಲಿಲ್ಲ. ಎಷ್ಟೇ ಪ್ರಿಯವಾಗಿದ್ದರೂ ಮೂಢನಂಬಿಕೆಗೆ ಶರಣಾಗುವುದಿರಲಿಲ್ಲ. ಪುರಾತನ ಕಾಲದಿಂದ ಬಂದಿದೆ ಎಂದು ಯಾವ ಆಚಾರವನ್ನಾಗಲೀ, ಶಾಸ್ತ್ರವನ್ನಾಗಲೀ, ಗೌರವಿಸುವುದಿರಲಿಲ್ಲ. ಅವನು ಶಾಸ್ತ್ರಗಳನ್ನೆಲ್ಲಾ ಧಿಕ್ಕರಿಸಿದ. ಎಲ್ಲಾ ಧಾರ್ಮಿಕ ಆಚಾರಗಳನ್ನೂ ತಿರಸ್ಕರಿಸಿದ. ಭರತಖಂಡದಲ್ಲಿ ಹಿಂದಿನಿಂದಲೂ ಧರ್ಮವನ್ನು ಬೋಧಿಸಲು ಉಪಯೋಗಿಸಲ್ಪಡುತ್ತಿದ್ದ ಸಂಸ್ಕೃತ ಭಾಷೆಯನ್ನು ಕೂಡ ತಿರಸ್ಕರಿಸಿದ. ಏಕೆಂದರೆ ಅದರೊಂದಿಗೆ ಸಂಬಂಧ ಪಡೆದ ಮೂಢನಂಬಿಕೆಗಳನ್ನು ಅವನ ಅನುಯಾಯಿಗಳು ಪಡೆಯದಿರಲಿ ಎಂದು.

ನಾವು ವಿವರಿಸುತ್ತಿರುವ ಸತ್ಯವನ್ನು ಬೇರೊಂದು ದೃಷ್ಟಿಯಿಂದಲೂ ನೋಡಬಹುದು. ಅದೇ ಹಿಂದೂದೃಷ್ಟಿ. ಬುದ್ಧನ ನಿಃಸ್ವಾರ್ಥತೆಯ ಮಹಾ ಸಿದ್ಧಾಂತವನ್ನು ನಮ್ಮ ದೃಷ್ಟಿಯಿಂದ ನೋಡಿದರೆ ಚೆನ್ನಾಗಿ ತಿಳಿದುಕೊಳ್ಳಬಹುದು. ಉಪನಿಷತ್ತಿನಲ್ಲಿ ಆಗಲೆ ಆತ್ಮದ ಮತ್ತು ಬ್ರಹ್ಮದ ಮಹಾಸಿದ್ಧಾಂತಗಳಿವೆ. ಆತ್ಮವೇ ಬ್ರಹ್ಮ. ಇರುವುದೊಂದೆ ಆತ್ಮ. ಇದೇ ಸತ್ಯ. ಮಾಯೆಯಿಂದ ನಾವು ಅನ್ಯಥಾ ಭಾವಿಸುವೆವು. ಇರುವುದೊಂದೆ ಆತ್ಮ. ಹಲವಿಲ್ಲ. ಆ ಒಂದು ಆತ್ಮ ಹಲವು ರೂಪುಗಳಲ್ಲಿ ವ್ಯಕ್ತವಾಗುತ್ತದೆ. ಮಾನವ ಮಾನವನ ಸಹೋದರ. ಏಕೆಂದರೆ ಮಾನವರೆಲ್ಲ ಒಂದೇ. ಮಾನವನು ನನ್ನ ಸಹೋದರ ಮಾತ್ರವಲ್ಲ, ನಾನೇ ಅವನು ಎಂದು ವೇದ ಹೇಳುತ್ತದೆ. ನಾನು ಯಾರನ್ನು ವ್ಯಥೆಗೆ ಈಡುಮಾಡಿದರೂ ನನಗೇ ವ್ಯಥೆಯನ್ನು ಉಂಟು ಮಾಡಿಕೊಂಡಂತೆ. ನಾನೇ ವಿಶ್ವ. ನಾನು ಅಂತಹವನು, ಇಂತಹವನು ಎಂದು ಭಾವಿಸುವುದೆಲ್ಲ ಒಂದು ಭ್ರಮೆ.

ನೀವು ನಿಮ್ಮ ನೈಜಸ್ಥಿತಿಯನ್ನು ಹೆಚ್ಚು ಸಮೀಪಿಸಿದಂತೆಲ್ಲ ಈ ಭ್ರಾಂತಿಯು ಹೆಚ್ಚು ಹೆಚ್ಚು ಮಾಯವಾಗುತ್ತಾ ಹೋಗುವುದು. ಭೇದಗಳು, ವ್ಯತ್ಯಾಸಗಳು ಹೆಚ್ಚು ಹೆಚ್ಚು ಕಡಿಮೆಯಾದಷ್ಟೂ ಎಲ್ಲರೂ ಒಂದೇ ಪರಂಜ್ಯೋತಿಯ ಆವಿರ್ಭಾವ ಎಂಬುದು ವ್ಯಕ್ತವಾಗುವುದು. ಭಗವಂತನು ಇದ್ದಾನೆ, ಆದರೆ ಅವನೆಲ್ಲೊ ಮುಗಿಲಿನಾಚೆ ಇರುವವನಲ್ಲ. ಅವನು ಶುದ್ಧ ಚೈತನ್ಯ. ಅವನಿರುವುದೆಲ್ಲಿ? ನಿನ್ನ ಆತ್ಮನಿಗಿಂತ ಸಮೀಪದಲ್ಲಿ ಇರುವನು. ಅವನೇ ಆತ್ಮ. ದೇವರು ನಿನಗಿಂತ ಬೇರೆ ಆಗಿರುವನು ಎಂದು ಹೇಗೆ ಅರಿಯಬಲ್ಲೆ? ನಿನಗಿಂತ ದೇವರು ಬೇರೆ ಎಂದು ಭಾವಿಸಿದರೆ ನೀನವನನ್ನು ಅರಿತಿಲ್ಲ. ಅವನೇ ನೀನು. ಇದೇ ಭಾರತದ ಮಹಾತ್ಮರ ಸಂದೇಶ.

ನೀನು ಇಂತಹ ಒಂದು ವ್ಯಕ್ತಿ ಎಂದು ಭಾವಿಸುವುದು, ಪ್ರಪಂಚವೆಲ್ಲ ನಿನಗಿಂತ ಬೇರೆ ಎಂದು ಯೋಚಿಸುವುದು ಸ್ವಾರ್ಥ. ನೀನು ನನಗಿಂತ ಬೇರೆ ಎಂದು ಭಾವಿಸುವೆ. ನೀನು ನನ್ನ ಯೋಚನೆಯನ್ನೇ ಮಾಡುವುದಿಲ್ಲ. ನೀನು ಮನೆಗೆ ಹೋಗಿ ಊಟ ಮಾಡಿ ನಿದ್ರೆ ಮಾಡುತ್ತೀಯೆ. ನಾನು ಸತ್ತರೂ ನೀನು ಊಟ ಮಾಡುವೆ, ಕುಡಿಯುವೆ, ಸಂತೋಷದಿಂದಿರುವೆ. ಪ್ರಪಂಚದಲ್ಲಿ ಇತರರೆಲ್ಲರೂ ದುಃಖಪಡುತ್ತಿರುವಾಗ ನೀವು ನಿಜವಾಗಿಯೂ ಸುಖವಾಗಿರಲಾರಿರಿ. ನಾವೆಲ್ಲಾ ಒಂದು. ನಾವು ಬೇರೆ ಬೇರೆ ಎಂಬ ಭ್ರಾಂತಿಯೇ ದುಃಖಕ್ಕೆಲ್ಲ ಮೂಲ. ಆತ್ಮನಲ್ಲದಿರುವುದು ಯಾವುದೂ ಇಲ್ಲ; ಮತ್ತಾವುದೂ ಇಲ್ಲ.

ಮಾನವನಲ್ಲದೆ ಬೇರೆ ದೇವರೇ ಇಲ್ಲವೆಂಬುದೇ ಬುದ್ಧನ ಮತ. ದೇವರಿರುವನು ಎಂಬ ಊಹೆಯ ದೃಷ್ಟಿಯನ್ನು ಅವನು ತಿರಸ್ಕರಿಸಿದನು. ಇದು ಜನರನ್ನು ದುರ್ಬಲರನ್ನಾಗಿ ಮಾಡುವುದು, ಮೂಢನಂಬಿಕೆಯುಳ್ಳವರನ್ನಾಗಿ ಮಾಡುವುದು ಎಂದನು. ಎಲ್ಲವನ್ನೂ ಕೊಡು ಎಂದು ದೇವರನ್ನು ಪ್ರಾರ್ಥಿಸಿದರೆ ಕೆಲಸ ಮಾಡುವವರು ಯಾರು? ಯಾರು ಕಷ್ಟಪಟ್ಟು ಕೆಲಸ ಮಾಡುವರೋ ಅವರಿಗೆ ದೇವರು ಸಹಾಯಮಾಡುವನು. ಯಾರು ತಮಗೆ ತಾವೇ ಸಹಾಯ ಮಾಡಿಕೊಳ್ಳುವರೋ ಅವರಿಗೆ ದೇವರು ಸಹಾಯ ಮಾಡುವನು. ಇದಕ್ಕೆ ವಿರೋಧವಾದ ದೇವರ ಭಾವನೆ, ನಮ್ಮ ನರಗಳನ್ನೆಲ್ಲಾ ದುರ್ಬಲಗೊಳಿಸುವುದು, ನಮ್ಮ ಮಾಂಸಖಂಡಗಳನ್ನು ಮೃದುಗೊಳಿಸುವುದು, ನಮ್ಮನ್ನು ಆಶ್ರಿತರನ್ನಾಗಿ ಮಾಡುವುದು. ಸ್ವತಂತ್ರವಾಗಿರುವುದೆಲ್ಲ ಆನಂದ. ಪರತಂತ್ರವಾಗಿರುವುದೆಲ್ಲ ದುಃಖ. ಮನುಷ್ಯನಲ್ಲೇ ಅನಂತ ಶಕ್ತಿ ಇದೆ. ಅವನು ಅದನ್ನು ಅರಿಯಬಹುದು. ತಾನೇ ಅನಂತಾತ್ಮ ಎಂಬುದನ್ನು ಅನುಭವಿಸಬಹುದು. ಆದರೆ ನೀವು ಅದನ್ನು ನಂಬುವುದಿಲ್ಲ. ನೀವು ದೇವರನ್ನೂ ಪ್ರಾರ್ಥಿಸುತ್ತೀರಿ, ಜೊತೆಗೆ ನಿಮ್ಮ ಪ್ರಯತ್ನವನ್ನೂ ನಂಬಿರುತ್ತೀರಿ.

ಬುದ್ಧ ಇದಕ್ಕೆ ವಿರೋಧವಾದುದನ್ನು ಬೋಧಿಸಿದನು. ಜನರು ಅಳದೆ ಇರಲಿ; ಪ್ರಾರ್ಥನೆ ಮುಂತಾದುವನ್ನೆಲ್ಲ ಮಾಡದೆ ಇರಲಿ. ದೇವರು ಒಂದು ಅಂಗಡಿ ಇಟ್ಟುಕೊಂಡಿಲ್ಲ. ಪ್ರತಿಸಲ ಉಸಿರಾಡುವಾಗಲೂ ನೀವು ದೇವರನ್ನು ಪ್ರಾರ್ಥಿಸುತ್ತಿರುವಿರಿ. ನಾನು ಮಾತನಾಡುತ್ತಿರುವೆನು. ಅದೂ ಒಂದು ಪ್ರಾರ್ಥನೆಯೆ. ನೀವು ಕೇಳುತ್ತಿರುವಿರಿ. ಅದೂ ಒಂದು ಪ್ರಾರ್ಥನೆಯೇ. ಅನಂತವಾಗಿರುವ ಪವಿತ್ರ ಶಕ್ತಿಯೊಂದಿಗೆ ಭಾಗವಹಿಸದ ಯಾವುದಾದರೂ ಕಾಯಕ ಅಥವಾ ಮಾನಸಿಕ ಕ್ರಿಯೆಯು ಇದೆಯೆ? ಇವೆಲ್ಲ ಒಂದು ನಿರಂತರ ಪ್ರಾರ್ಥನೆ. ಎಲ್ಲೋ ಕೆಲವು ಪದಗಳನ್ನು ಮಾತ್ರ ನೀವು ಪ್ರಾರ್ಥನೆ ಎಂದು ಕರೆದರೆ, ಪ್ರಾರ್ಥನೆಯನ್ನು ಕೇವಲ ತೋರಿಕೆಯನ್ನಾಗಿ ಮಾಡುವಿರಿ. ಇಂತಹ ಪ್ರಾರ್ಥನೆಯಿಂದ ಅಷ್ಟು ಉಪಯೋಗವಿಲ್ಲ. ಇದರಿಂದ ನಿಜವಾದ ಪ್ರತಿಫಲ ದೊರಕುವುದು ಅಪರೂಪ.

ನೀವು ಏನೂ ಕೆಲಸ ಮಾಡದೆ, ಸುಮ್ಮನೆ ಉಚ್ಚರಿಸುವುದರಿಂದ ಅದ್ಭುತ ಪ್ರತಿಫಲವನ್ನು ಪಡೆಯುವುದಕ್ಕೆ ಪ್ರಾರ್ಥನೆಯೇನು ಒಂದು ಯಕ್ಷಿಣೀ ಮಂತ್ರವೆ? ಇಲ್ಲ. ಎಲ್ಲರೂ ಆ ಅನಂತ ಶಕ್ತಿಯ ಆಳಕ್ಕೆ ಹೋಗಬೇಕು. ಬಡವನ, ಶ‍್ರೀಮಂತರ ಹಿಂದೆ ಆ ಅನಂತ ಶಕ್ತಿ ಇದೆ. ಒಬ್ಬ ಕಷ್ಟಪಟ್ಟು ಕೆಲಸ ಮಾಡುತ್ತಾನೆ. ಮತ್ತೊಬ್ಬ ಯಾವುದೋ ಕೆಲವು ಮಂತ್ರಗಳನ್ನು ಉಚ್ಚರಿಸಿ ಪ್ರಯೋಜನ ಪಡೆಯುತ್ತಾನೆ ಎಂಬುದಿಲ್ಲ. ಈ ಸೃಷ್ಟಿಯೇ ಒಂದು ಅನವರತ ಪ್ರಾರ್ಥನೆ. ನೀವು ಈ ದೃಷ್ಟಿಯಿಂದ ಪ್ರಾರ್ಥನೆಯನ್ನು ಮಾಡಿದರೆ ನಾನು ಅದನ್ನು ಒಪ್ಪುವೆನು. ಮಾತುಗಳು ಬೇಕಾಗಿಲ್ಲ. ಮೌನವಾದ ಪ್ರಾರ್ಥನೆ ಮಾತುಗಳಿಗಿಂತ ಮೇಲು.

ಹೆಚ್ಚಿನ ಜನರಿಗೆ ಈ ಸಿದ್ಧಾಂತ ಅರ್ಥವಾಗುವುದಿಲ್ಲ. ಭರತಖಂಡದಲ್ಲಿ ಆತ್ಮನ ವಿಚಾರದಲ್ಲಿ ಏನಾದರೂ ರಾಜಿ ಮಾಡಿಕೊಳ್ಳುವುದು ಎಂದರೆ ಮಹಾಪುರುಷರ ಬೋಧನೆಯನ್ನು ಮರೆತು ಪುರೋಹಿತರಿಗೆ ನಮ್ಮ ಅಧಿಕಾರವನ್ನೆಲ್ಲ ಕೊಡುವುದು ಎಂದು ಅರ್ಥ. ಬುದ್ಧನಿಗೆ ಇದು ಗೊತ್ತಿತ್ತು. ಆದಕಾರಣವೇ ಬುದ್ಧನು ಪುರೋಹಿತರ ತತ್ತ್ವಗಳನ್ನು ಮತ್ತು ಆಚಾರಗಳನ್ನೆಲ್ಲಾ ಬದಿಗಿರಿಸಿ ಮನುಷ್ಯನನ್ನು ತನ್ನ ಸ್ವಂತ ಕಾಲ ಮೇಲೆ ನಿಲ್ಲುವಂತೆ ಮಾಡಿದನು. ಜನರ ರೂಢಿಗಳಿಗೆ ವಿರೋಧವಾಗಿ ಅವನು ಹೋಗಬೇಕಾಗಿತ್ತು. ಅವನು ಒಂದು ದೊಡ್ಡ ಕ್ರಾಂತಿಯನ್ನು ತರಬೇಕಾಗಿತ್ತು. ಇದರ ಪರಿಣಾಮವಾಗಿ ಯಾಗ ಯಜ್ಞಗಳನ್ನೊಳಗೊಂಡ ಧರ್ಮವು ಭರತ ಖಂಡದಿಂದ ಮರೆಯಾಯಿತು. ಪುನಃ ಅದು ಚೇತರಿಸಿಕೊಳ್ಳಲಿಲ್ಲ.

ತೋರಿಕೆಗೆ ಬೌದ್ಧಧರ್ಮ ಇಂಡಿಯಾದಿಂದ ಮಾಯವಾಗಿದೆ, ಆದರೆ ನಿಜವಾಗಿ ಹಾಗಾಗಿಲ್ಲ. ಬುದ್ಧನ ಬೋಧನೆಯಲ್ಲಿ ಒಂದು ಅಪಾಯಕರವಾದ ಅಂಶವಿತ್ತು – ಬೌದ್ಧಧರ್ಮವು ಒಂದು ಸುಧಾರಕ ಧರ್ಮ. ದೊಡ್ಡ ಒಂದು ಆಧ್ಯಾತ್ಮಿಕ ಬದಲಾವಣೆಯನ್ನು ತರುವುದಕ್ಕಾಗಿ ಎಷ್ಟೋ ನಿಷೇಧಾತ್ಮಕವಾದ ಬೋಧನೆಗಳನ್ನು ಕೊಡಬೇಕಾಯಿತು. ಒಂದು ಧರ್ಮ ಕೇವಲ ನಿಷೇಧದ ಭಾಗವನ್ನೇ ಒತ್ತಿ ಹೇಳುತ್ತಿದ್ದರೆ ಅದು ಕೊನೆಗೆ ನಾಶವಾಗುವ ಅಪಾಯವಿದೆ. ಒಂದು ಸುಧಾರಕ ಪಂಥವು ಕೇವಲ ಸುಧಾರಣೆಯಲ್ಲಿಯೇ ನಿರತವಾಗಿದ್ದರೆ ಅದು ಜೀವಂತವಾಗಿರಲಾರದು. ಜೀವನ ಪೋಷಕವಾದ ನಿಜವಾದ ಸ್ಫೂರ್ತಿ ಅಂದರೆ ಅದರ ಸಿದ್ಧಾಂತಗಳು, ಅವು ಮಾತ್ರ ಎಂದೆಂದಿಗೂ ಬಾಳಬಲ್ಲವು. ಒಂದು ಸುಧಾರಣೆಯನ್ನು ತಂದ ಮೇಲೆ ಅಸ್ತಿಭಾವವನ್ನು ಒತ್ತಿ ಹೇಳಬೇಕು. ಕಟ್ಟಡ ಪೂರೈಸಿದ ಮೇಲೆ ಕಾಲಾವಧಿಯನ್ನೆಲ್ಲಾ ತೆಗೆದುಹಾಕಬೇಕು.

ಕಾಲ ಕಳೆದಂತೆ ಭರತಖಂಡದಲ್ಲಿ ಬುದ್ಧನ ಅನುಯಾಯಿಗಳು ಬುದ್ಧನ ಬೋಧನೆಯ ನಿಷೇಧದ ಭಾಗವನ್ನು ಮಾತ್ರ ಒತ್ತಿ ಹೇಳತೊಡಗಿದರು. ಇದೇ ಬೌದ್ಧಧರ್ಮದ ಅವನತಿಗೆ ಕಾರಣವಾಯಿತು. ಸಕಾರಾತ್ಮಕ ಭಾವನೆಗಳು ನಿಷೇಧಾತ್ಮಕ ಭಾವನೆಗಳ ಆಧಿಕ್ಯದಿಂದ ಕುಗ್ಗಿಹೋದವು. ಬೌದ್ಧಧರ್ಮದ ಹೆಸರಿನಲ್ಲಿ ಜಾರಿಯಲ್ಲಿದ್ದ ವಿನಾಶಾತ್ಮಕ ಮನೋಭಾವಗಳನ್ನು ಭರತಖಂಡ ತಿರಸ್ಕರಿಸಿತು. ಇದೇ ಹಿಂದೂ ರಾಷ್ಟ್ರೀಯ ಚಿಂತನೆ ನೀಡಿದ ತೀರ್ಪು.

ದೇವರಿಲ್ಲ, ಆತ್ಮನಿಲ್ಲ ಎಂಬ ಅಭಾವಸೂಚಕ ಬೌದ್ಧ ಭಾವನೆಗಳು ಮಾಯವಾದವು. ಇರುವವನು ದೇವನೊಬ್ಬನೆ ಎಂದು ನಾನು ಹೇಳುತ್ತೇನೆ. ಇದು ಸಕಾರಾತ್ಮಕ ನಿರೂಪಣೆ. ಅವನೊಬ್ಬನೆ ಏಕಮಾತ್ರ ಸತ್ಯ. ಬುದ್ಧನು ಆತ್ಮನಿಲ್ಲ ಎಂದರೆ, ನಾನು “ಎಲೈ ಮಾನವ, ನೀನು ಮತ್ತು ವಿಶ್ವ ಒಂದು; ನೀನೇ ಎಲ್ಲವೂ ಆಗಿರುವಿ” ಎನ್ನುವೆನು. ಎಷ್ಟು ಸ್ಪಷ್ಟವಾಗಿದೆ! ಸುಧಾರಕಾಂಶ ನಾಶವಾಯಿತು. ಬೆಳವಣಿಗೆಯ ಅಂಶ ಎಂದೆಂದಿಗೂ ಇರುವುದು. ಬುದ್ಧ ಪ್ರಾಣಿಗಳಿಗೆ ದಯೆ ತೋರಿ ಎಂದು ಬೋಧಿಸಿದ. ಅಂದಿನಿಂದ ಎಲ್ಲರಿಗೂ ಪ್ರಾಣಿಗಳಿಗೂ ಕೂಡ, ದಯೆಯನ್ನು ತೋರಬೇಕೆಂದು ಬೋಧಿಸದ ಪಂಥವು ಯಾವುದೂ ಭರತಖಂಡದಲ್ಲಿ ಇಲ್ಲ. ಎಲ್ಲಾ ಸಿದ್ಧಾಂತಗಳಿಗಿಂತ ಮಿಗಿಲಾಗಿ ಬೌದ್ಧಧರ್ಮ ನಮಗೆ ಕೊಟ್ಟಿರುವುದು ಈ ದಯೆ, ಅನುಕಂಪ, ದಾನ ಎಂಬುವು.

ಬುದ್ಧನ ಜೀವನವೇ ಒಂದು ದೊಡ್ಡ ಆಕರ್ಷಣೆಯಾಗಿದೆ. ನನಗೆ ಬುದ್ಧನ ಜೀವನದ ಮೇಲೆ ಪ್ರೀತಿ, ಆದರೆ ಅವನ ಸಿದ್ಧಾಂತದ ಮೇಲೆ ಇಲ್ಲ. ನಾನು ಆ ಪವಿತ್ರಾತ್ಮನ ಶೀಲವನ್ನು ಎಲ್ಲರಿಗಿಂತ ಹೆಚ್ಚಾಗಿ ಗೌರವಿಸುತ್ತೇನೆ. ಅವನಲ್ಲಿದ್ದ ಧೈರ್ಯ, ನಿರ್ಭಯತೆ, ಅದ್ಭುತ ಪ್ರೇಮ! ಮಾನವ ಹಿತಕ್ಕಾಗಿ ಅವನು ಜನ್ಮವೆತ್ತಿದ. ಇತರರು ತಮಗಾಗಿ ದೇವರನ್ನು ಹುಡುಕಿರಬಹುದು, ಸತ್ಯವನ್ನು ಹುಡುಕಿರಬಹುದು. ಆದರೆ ಈತ ಕೇವಲ ತನಗಾಗಿ ಮಾತ್ರ ಸತ್ಯವನ್ನು ಹುಡುಕಲೆತ್ನಿಸಿದವನಲ್ಲ. ಜನರು ದುಃಖದಲ್ಲಿದ್ದುದರಿಂದ ಅವನು ಸತ್ಯವನ್ನು ಅರಸಿದ. ಅವರಿಗೆ ಹೇಗೆ ಸಹಾಯ ಮಾಡಬೇಕು ಎಂಬುದೊಂದೆ ಅವನ ಜೀವನದ ಏಕಮಾತ್ರ ಧ್ಯೇಯವಾಗಿತ್ತು. ಬದುಕಿರುವ ಪರ್ಯಂತರ ಒಮ್ಮೆಯಾದರೂ ತನಗಾಗಿ ಯೋಚಿಸಿದವನಲ್ಲ ಅವನು. ಈ ಮಹಿಮನ ಮಾಹಾತ್ಮ್ಯವನ್ನು ನಮ್ಮಂತಹ ಅಜ್ಞರು, ಸ್ವಾರ್ಥಿಗಳು, ಸಂಕುಚಿತ ದೃಷ್ಟಿಯವರು ಹೇಗೆ ಅರ್ಥಮಾಡಿಕೊಳ್ಳಬಲ್ಲರು.

ಅವನ ಅದ್ಭುತ ಧೀಶಕ್ತಿಯನ್ನು ನೋಡಿ; ಉದ್ವೇಗಕ್ಕೆ ಸ್ವಲ್ಪವೂ ಅವಕಾಶವಿಲ್ಲ; ಈ ಮಹಾಮೇಧಾವಿ ಮೌಢ್ಯಕ್ಕೆ ಸ್ವಲ್ಪವೂ ಶರಣಾದವನಲ್ಲ. ಯಾವುದೋ ಪುರಾತನ ಗ್ರಂಥ ಸಾರುವುದೆಂದು ನಂಬಬೇಡಿ. ನಿಮ್ಮ ಮುತ್ತಾತಂದಿರ ಕಾಲದಿಂದ ಅದು ಬಂದಿದೆ ಎಂದು ನೀವು ನಂಬಬೇಡಿ, ನಿಮ್ಮ ಸ್ನೇಹಿತರು ಹೇಳುತ್ತಾರೆ ಎಂದು ನೀವು ನಂಬಬೇಡಿ, ನೀವೇ ಯೋಚಿಸಿ ನೀವೇ ಸತ್ಯವನ್ನು ಹುಡುಕಿ, ನೀವೇ ಅದನ್ನು ಸಾಕ್ಷಾತ್ಕಾರ ಮಾಡಿಕೊಳ್ಳಿ, ಅನಂತರ ಅದರಿಂದ ಎಲ್ಲರಿಗೂ ಒಳ್ಳೆಯದಾಗುತ್ತದೆಂದು ಕಂಡುಬಂದರೆ ಅದನ್ನು ಇತರರಿಗೆ ಬೋಧಿಸಿ. ದುರ್ಬಲರು, ಅಂಜುಕುಳಿಗಳು ಸತ್ಯವನ್ನು ಪಡೆಯಲಾರರು. ಸ್ವತಂತ್ರನಾಗಿರಬೇಕು; ಆಕಾಶದಷ್ಟು ವಿಶಾಲವಾಗಿರಬೇಕು, ಸ್ಫಟಿಕದಂತೆ ಸ್ಪಷ್ಟವಾದ ಮನಸ್ಸಿರಬೇಕು. ಆಗ ಮಾತ್ರ ಸತ್ಯ ಅವನಲ್ಲಿ ಹೊಳೆಯಬಲ್ಲುದು. ನಮ್ಮಲ್ಲಿ ಬೇಕಾದಷ್ಟು ಮೂಢನಂಬಿಕೆಗಳಿವೆ. ಬಹಳ ವಿದ್ಯಾವಂತರು ಎಂದು ಜಂಭ ಕೊಚ್ಚಿಕೊಳ್ಳುವ ನಿಮ್ಮ ದೇಶದಲ್ಲಿ ಕೂಡ ಎಷ್ಟು ಸಂಕುಚಿತ ಭಾವನೆಗಳು ಮತ್ತು ಮೂಢನಂಬಿಕೆಗಳು ಇವೆ! ನಿಮಗೆ ಸಮಾನರಾದ ನಾಗರಿಕರಿಲ್ಲವೆಂದು ಹೆಮ್ಮೆ ಕೊಚ್ಚಿಕೊಳ್ಳುವ ನಿಮ್ಮ ದೇಶದಲ್ಲಿ, ನಾನು ಹಿಂದೂ ಎಂಬ ಕಾರಣದಿಂದ ಒಂದು ಸಲ ನನಗೆ ಕುಳಿತುಕೊಳ್ಳುವುದಕ್ಕೆ ಒಂದು ಕುರ್ಚಿಯನ್ನೂ ಕೊಡಲಿಲ್ಲ – ಇದನ್ನು ಯೋಚಿಸಿ ನೋಡಿ!

ಕ್ರಿಸ್ತ ಹುಟ್ಟುವುದಕ್ಕಿಂತ ಆರುನೂರು ವರ್ಷಗಳ ಮುಂಚೆ, ಬುದ್ಧನ ಕಾಲದಲ್ಲಿ\break ಭಾರತೀಯರು ಹೆಚ್ಚು ವಿದ್ಯಾಸಂಪನ್ನರಾಗಿದ್ದಿರಬಹುದು. ಬಹಳ ಸ್ವತಂತ್ರ ಬುದ್ಧಿಯುಳ್ಳವರಾಗಿದ್ದಿರಬಹುದು. ಅನೇಕ ಜನರು ಅವನ ಅನುಯಾಯಿಗಳಾದರು. ರಾಜರು ತಮ್ಮ ಸಿಂಹಾಸನವನ್ನು ತ್ಯಜಿಸಿದರು. ರಾಣಿಯರು ತಮ್ಮ ಸಿಂಹಾಸನವನ್ನು ತ್ಯಜಿಸಿದರು. ಹಿಂದಿನಿಂದಲೂ ಪುರೋಹಿತರು ಅವರಿಗೆ ಏನನ್ನು ಬೋಧಿಸಿದ್ದರೋ ಅದಕ್ಕಿಂತ ಬೇರೆಯಾಗಿರುವ, ಕ್ರಾಂತಿಕಾರಕವಾಗಿರುವ ಅವನ ಸಂದೇಶವನ್ನು ಮೆಚ್ಚುವುದಕ್ಕೆ ಮತ್ತು ಅನುಸರಿಸುವುದಕ್ಕೆ ಅವರಿಗೆ ಸಾಧ್ಯವಾಯಿತು. ಆದರೆ ಅವರ ಮನಸ್ಸು ಅಸಾಧಾರಣವಾಗಿ ಸ್ವತಂತ್ರವಾಗಿತ್ತು ಮತ್ತು ಅಸಾಧಾರಣವಾಗಿ ವಿಶಾಲವಾಗಿತ್ತು.

ಅವನ ಮರಣವನ್ನು ಕುರಿತು ಯೋಚಿಸಿ. ಅವನು ಬದುಕಿರುವಾಗ ಮಹಾತ್ಮನಾಗಿದ್ದರೆ ಸಾಯುವಾಗಲೂ ಹಾಗೆಯೇ ಇದ್ದನು. ನಿಮ್ಮ ರೆಡ್​ ಇಂಡಿಯನ್ನರ ವಂಶಕ್ಕೆ ಸಮಾನವಾದ ಒಂದು ವಂಶಕ್ಕೆ ಸೇರಿದ ಒಬ್ಬನು ಕೊಟ್ಟ ಆಹಾರವನ್ನು ಅವನು ಸ್ವೀಕರಿಸಿದ. ಹಿಂದೂಗಳು ಅಂತಹವರನ್ನು ಮುಟ್ಟುವುದು ಕೂಡ ಇಲ್ಲ. ಏಕೆಂದರೆ ಅವರು ಸಿಕ್ಕಿದುದನ್ನೆಲ್ಲ ತಿನ್ನುವರು. ಅವನು ತನ್ನ ಶಿಷ್ಯರಿಗೆ “ನೀವು ಈ ಆಹಾರವನ್ನು ತೆಗೆದುಕೊಳ್ಳಬೇಡಿ. ಆದರೆ ನನಗೆ ಇದನ್ನು ತೆಗೆದುಕೊಳ್ಳದೇ ವಿಧಿಯಿಲ್ಲ. ಆತ ನನಗೆ ಒಂದು ದೊಡ್ಡ ಉಪಕಾರವನ್ನು ಮಾಡಿರುವನು. ನನ್ನನ್ನು ದೇಹ ಬಂಧನದಿಂದ ಪಾರು ಮಾಡಿರುವನು ಎಂದು ಆತನಿಗೆ ಹೇಳಿ” ಎಂದನು. ವೃದ್ಧನೊಬ್ಬ ಅವನ ಹತ್ತಿರ ಬಂದು ಕುಳಿತ. ಅವನು ಗುರುವನ್ನು ನೋಡುವುದಕ್ಕೆ ಹಲವು ಮೈಲಿಗಳನ್ನು ನಡೆದುಕೊಂಡು ಬಂದಿದ್ದ. ಬುದ್ಧ ಅವನಿಗೆ ಬೋಧಿಸಿದ. ತನ್ನ ಶಿಷ್ಯನೊಬ್ಬ ಅಳುತ್ತಿರುವುದನ್ನು ನೋಡಿ ಅವನಿಗೆ ಬುದ್ಧಿ ಹೇಳಿದ. “ಇದೇನು? ನಾನು ಇಷ್ಟೊಂದು ಬೋಧಿಸಿದುದರ ಪರಿಣಾಮವೇ ಇದು? ಎಂದಿಗೂ ಈ ಅಸತ್ಯವಾದ ಬಂಧನವಿರದಿರಲಿ, ನನ್ನನ್ನೇ ಎಂದೂ ಆಶ್ರಯಿಸಬೇಡಿ. ಇಂದು ಇದ್ದು ನಾಳೆ ಹೋಗುವ ವ್ಯಕ್ತಿಯನ್ನು ಹೊಗಳಬೇಡಿ. ಬುದ್ಧ ಒಂದು ವ್ಯಕ್ತಿಯಲ್ಲ. ಅದೊಂದು ಸಾಕ್ಷಾತ್ಕಾರದ ಸ್ಥಿತಿ, ನಿಮ್ಮ ಮುಕ್ತಿಯನ್ನು ನೀವೇ ಸಾಧಿಸಿ.

ಬುದ್ಧನು ಸಾಯುವಾಗಲೂ ತಾನು ಪ್ರತ್ಯೇಕ ಎಂದು ಒಪ್ಪಿಕೊಳ್ಳಲಿಲ್ಲ. ನಾನು ಅವನನ್ನು ಆರಾಧಿಸುವುದು ಅದಕ್ಕಾಗಿ. ನೀವು ಕರೆಯುತ್ತಿರುವ ಕ್ರಿಸ್ತ – ಬುದ್ಧ ಎಂಬುದು ಸಾಕ್ಷಾತ್ಕಾರದ ಕೆಲವು ಅವಸ್ಥೆಗಳು. ನಾವು ನಮ್ಮ ಸ್ವಂತ ಕಾಲಿನ ಮೇಲೆ ನಿಲ್ಲಬೇಕೆಂದು ಆತ ಪ್ರಪಂಚದ ಇತರ ಗುರುಗಳೆಲ್ಲರಿಗಿಂತ ಹೆಚ್ಚಾಗಿ ಬೋಧಿಸಿದನು. ಅವನು ನಮ್ಮನ್ನು ನಮ್ಮ ತೋರಿಕೆಯ ವ್ಯಕ್ತಿತ್ವದ ಬಂಧನದಿಂದ ಪಾರುಮಾಡಿದುದು ಮಾತ್ರವಲ್ಲ, ದೇವರುಗಳೆಂಬ ಅಗೋಚರ ವ್ಯಕ್ತಿಗಳ ಹಿಡಿತದಿಂದ ನಮ್ಮನ್ನು ಸ್ವತಂತ್ರರನ್ನಾಗಿ ಮಾಡಿದ. ನಿರ್ವಾಣವೆಂಬ ಆ ಅವಸ್ಥೆಯನ್ನು ಪಡೆಯಲು ಎಲ್ಲರಿಗೂ ಆಹ್ವಾನವಿತ್ತ. ಎಲ್ಲರೂ ಒಂದು ದಿನ ಅದನ್ನು ಪಡೆಯಲೇಬೇಕು. ಅದೇ ಮಾನವನ ಪರಮಗುರಿ.

