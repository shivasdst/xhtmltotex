
\chapter[ಜೀವ ಮತ್ತು ಈಶ್ವರ ]{ಜೀವ ಮತ್ತು ಈಶ್ವರ \protect\footnote{\engfoot{C.W. Vol. VI, P. 93}}}

ದೇಶದಲ್ಲಿರುವ ಪ್ರತಿಯೊಂದಕ್ಕೂ ಆಕಾರವಿದೆ. ದೇಶಕ್ಕೇ ಆಕಾರವಿದೆ. ನೀವು ದೇಶದಲ್ಲಿರುವಿರಿ, ಇಲ್ಲವೇ ದೇಶ ನಿಮ್ಮಲ್ಲಿರುವುದು. ಆತ್ಮವು ದೇಶಾತೀತ, ದೇಶವು ಆತ್ಮನಲ್ಲಿದೆಯೇ ಹೊರತು ಆತ್ಮ ದೇಶದಲ್ಲಿಲ್ಲ.

ಆಕಾರವು ಕಾಲದೇಶಗಳ ಮಿತಿಯಲ್ಲಿದೆ, ನಿಯಮಕ್ಕೆ ಅಧೀನ, ಕಾಲ ನಮ್ಮಲ್ಲಿದೆ. ನಾವು ಕಾಲದಲ್ಲಿ ಇಲ್ಲ. ಆತ್ಮವು ಕಾರ್ಯಕಾರಣಗಳ ಕಾಲದೇಶಗಳಲ್ಲಿ ಇಲ್ಲದೆ ಇರುವುದರಿಂದ ಕಾಲ ದೇಶಗಳು ಆತ್ಮನಲ್ಲಿವೆ. ಆದಕಾರಣ ಆತ್ಮ ಸರ್ವ ವ್ಯಾಪಿ.

ಈಶ್ವರನನ್ನು ಕುರಿತು ನಮಗಿರುವ ಭಾವನೆ ನಮ್ಮ ಪ್ರತಿಬಿಂಬವೇ ಆಗಿದೆ. ಹಳೆಯ ಪಾರ್ಸಿ ಭಾಷೆಗೆ ಮತ್ತು ಸಂಸ್ಕೃತಕ್ಕೆ ಸಾಮ್ಯವಿದೆ.

ನಿಸರ್ಗವನ್ನು ಹಲವು ರೂಪಗಳಲ್ಲಿ ಪೂಜಿಸುವುದೇ ದೇವರನ್ನು ಕುರಿತಾದ ಪುರಾತನ ಭಾವನೆಯಾಗಿತ್ತು. ಎರಡನೆಯ ಹಂತವೇ ದೇವರು. ಅನಂತರವೇ ದೊರೆಯ ಪೂಜೆ. ಸ್ವರ್ಗದಲ್ಲಿ ದೇವರಿರುವನು ಎಂಬ ಭಾವನೆ ಭರತಖಂಡವನ್ನು ಬಿಟ್ಟು ಎಲ್ಲಾ ದೇಶಗಳಲ್ಲಿಯೂ ಇರುವುದು. ಈ ಭಾವನೆ ಬಹಳ ಅನಾಗರಿಕ.

ಈ ಜೀವನವು ಎಂದೆಂದಿಗೂ ಇರುವುದು ಎಂಬುದೊಂದು ಭ್ರಾಂತಿ. ಜನನದ ಭಾವನೆ ಹೋಗುವವರೆಗೆ ಮರಣದ ಭಾವನೆ ಹೋಗುವುದಿಲ್ಲ.

