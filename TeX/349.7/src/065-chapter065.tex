
\chapter[ಸತ್ಯ ಮತ್ತು ಮಿಥ್ಯೆ ]{ಸತ್ಯ ಮತ್ತು ಮಿಥ್ಯೆ \protect\footnote{\engfoot{C.W. Vol, VI, P.92}}}

ಒಂದಕ್ಕೂ ಮತ್ತೊಂದಕ್ಕೂ ವ್ಯತ್ಯಾಸವನ್ನು ಕಲ್ಪಿಸುವುದು ದೇಶ ಕಾಲ ನಿಮಿತ್ತಗಳು.

ವ್ಯತ್ಯಾಸವಿರುವುದು ಆಕಾರದಲ್ಲೇ ಹೊರತು ವಸ್ತುವಿನಲ್ಲಿ ಅಲ್ಲ.

ನೀವು ಆಕಾರವನ್ನು ನಾಶಮಾಡಿದರೆ ಅದು ಎಂದೆಂದಿಗೂ ನಾಶವಾಗುವುದು; ಆದರೆ ಅದರ ಹಿಂದೆ ಇರುವ ವಸ್ತು ಹಾಗೆಯೇ ಇರುವುದು. ನೀವು ಎಂದಿಗೂ ವಸ್ತುವನ್ನು ನಾಶಮಾಡಲಾರಿರಿ.

ವಿಕಾಸವು ಪ್ರಕೃತಿಯಲ್ಲಿರುವುದೇ ಹೊರತು ಆತ್ಮನಲ್ಲಿ ಅಲ್ಲ. ಪ್ರಕೃತಿಯ ವಿಕಾಸವು ಆತ್ಮನ ಅಭಿವ್ಯಕ್ತಿ.

ಮಾಯೆ ಎಂದರೆ ಸಾಧಾರಣವಾಗಿ ಭಾವಿಸುವಂತೆ ಒಂದು ಭ್ರಾಂತಿಯಲ್ಲ. ಮಾಯೆ ಸತ್ಯ, ಆದರೆ ಸತ್ಯವಲ್ಲ. ಯಾವ ದೃಷ್ಟಿಯಿಂದ ಅದು ಸತ್ಯ, ಎಂದರೆ ಅದರ ಹಿನ್ನೆಲೆಯಲ್ಲಿ ಸತ್ಯವಿರುವುದರಿಂದ ಅದು ನಿಜವಾಗಿರುವಂತೆ ಕಾಣಿಸುತ್ತದೆ. ಯಾವುದು ಮಾಯೆಯಲ್ಲಿ ನಿಜವಾಗಿರುವುದೋ ಅದೇ ಮಾಯೆಯ ಮೂಲಕ ವ್ಯಕ್ತವಾಗುತ್ತಿರುವ ಸತ್ಯ. ಆದರೂ ಸತ್ಯವು ಎಂದಿಗೂ ಕಾಣಿಸಿಕೊಂಡಿಲ್ಲ. ಆದಕಾರಣ ಯಾವುದು ನಮಗೆ ಕಾಣಿಸುವುದೋ ಅದು ನಿಜವಲ್ಲ. ಅದು ತನಗೆ ತಾನೇ ಸ್ವತಂತ್ರವಾಗಿರಲಾರದು. ಅದರ ಆಸ್ತಿತ್ವವು ಸತ್ಯವನ್ನು ಅವಲಂಬಿಸಿದೆ.

ಹಾಗಾದರೆ ಮಾಯೆ ಒಂದು ವಿರೋಧಾಭಾಸ ಎಂದಾಯಿತು. ನಿಜ ಆದರೂ ನಿಜವಲ್ಲ, ಭ್ರಾಂತಿ ಆದರೂ ಭ್ರಾಂತಿಯಲ್ಲ. ಯಾರಿಗೆ ಸತ್ಯಸಾಕ್ಷಾತ್ಕಾರವಾಗಿ ದೆಯೋ ಅವರು ಮಾಯೆಯಲ್ಲಿ ಭ್ರಾಂತಿಯನ್ನು ಕಾಣುವುದಿಲ್ಲ; ಸತ್ಯವನ್ನು ಕಾಣುವರು. ಯಾರಿಗೆ ಸತ್ಯಸಾಕ್ಷಾತ್ಕಾರವಾಗಿಲ್ಲವೋ ಅವರು ಮಾಯೆಯಲ್ಲಿ ಭ್ರಾಂತಿಯನ್ನು ಕಾಣುವರು; ಅದು ನಿಜವೆಂದು ಭ್ರಮಿಸುವರು.

