
\vspace{-0.5cm}

\chapter[ಧರ್ಮಾನುಷ್ಠಾನ ]{ಧರ್ಮಾನುಷ್ಠಾನ \protect\footnote{\engfoot{C.W. Vol. VI, P. 101}}}

\begin{center}
\textbf{(೧೯೦೦ರ ಮಾರ್ಚ್​ ೧೮ರಂದು ಕ್ಯಾಲಿಫೋರ್ನಿಯಾದ ಅಲಮೇಡಾದಲ್ಲಿ ನೀಡಿದ\general{\break } ಉಪನ್ಯಾಸ)}
\end{center}

ನಾವು ಹಲವು ಗ್ರಂಥಗಳನ್ನು ಓದುತ್ತೇವೆ. ಆದರೆ ಅದರಿಂದ ನಮಗೆ ಜ್ಞಾನ ಬರುವುದಿಲ್ಲ. ನಾವು ಪ್ರಪಂಚದ ಧರ್ಮಗ್ರಂಥಗಳನ್ನೆಲ್ಲ ಓದಬಹುದು. ಆದರೆ ಅವು ನಮಗೆ ಧರ್ಮವನ್ನು ಕೊಡಲಾರವು. ಧರ್ಮದ ಸಿದ್ಧಾಂತಗಳನ್ನು ತಿಳಿದುಕೊಳ್ಳುವುದು ಸುಲಭ. ಯಾರು ಬೇಕಾದರೂ ಅವನ್ನು ಪಡೆಯಬಹುದು. ನಮಗೆ ಬೇಕಾಗಿರುವುದು ಅನುಷ್ಠಾನಶೀಲವಾದ ಧರ್ಮ.

ಕ್ರೈಸ್ತರ ದೃಷ್ಟಿಯಲ್ಲಿ ಅನುಷ್ಠಾನಶೀಲವಾದ ಧರ್ಮವೆಂದರೆ ಒಳ್ಳೆಯ ಕೆಲಸಗಳನ್ನು ಮಾಡುವುದು; ಅಂದರೆ ಲೌಕಿಕ ಪ್ರಯೋಜನವನ್ನುಂಟು ಮಾಡುವಂತಹ ಕ್ರಿಯೆಗಳನ್ನು ಮಾಡುವುದು.

ಈ ಪ್ರಯೋಜನದಿಂದ ಬರುವ ಲಾಭ ಏನು? ಪ್ರಯೋಜನ ದೃಷ್ಟಿಯಿಂದ ಅಳೆದರೆ\break ಧರ್ಮ ಸೋತಂತೆ. ಪ್ರತಿಯೊಂದು ಆಸ್ಪತ್ರೆಯನ್ನು ಕಟ್ಟುವುದೂ ಹೆಚ್ಚು ರೋಗಿಗಳು\break ಅಲ್ಲಿಗೆ ಬರಲಿ ಎಂದು ಪ್ರಾರ್ಥನೆ ಮಾಡಿದಂತೆ. ದಾನ ಎಂದರೆ ಏನು ಅರ್ಥ? ದಾನವೇ ಮುಖ್ಯವಲ್ಲ. ಇದು ಪ್ರಪಂಚದ ದುಃಖವು ಮುಂದೆಯೂ ಹಾಗೆಯೇ ಇರುವುದಕ್ಕೆ\break ಸಹಕಾರಿಯೇ ಹೊರತು ಅದನ್ನು ನಿರ್ನಾಮ ಮಾಡುವುದಕ್ಕಲ್ಲ. ಒಬ್ಬನು ಕೀರ್ತಿ ಯಶಸ್ಸುಗಳಿಗಾಗಿ ಹಾತೊರೆಯುತ್ತಿರುವನು. ಅವನ್ನು ಪಡೆಯುವುದಕ್ಕೆ ದಾನದ ಗಿಲೀಟನ್ನು ತನ್ನ ಪ್ರಯತ್ನಗಳ ಮೇಲೆ ಹಚ್ಚುವನು. ಅವನು ಇತರರಿಗೆ ಸಹಾಯ ಮಾಡುವ ನೆಪದಲ್ಲಿ ತನ್ನ ಸ್ವಾರ್ಥವನ್ನು ನೆರವೇರಿಸಿಕೊಳ್ಳುವನು. ಮಾಡಿದ ಪ್ರತಿಯೊಂದು ದಾನವೂ, ಯಾವ ಪಾಪವನ್ನು ನಿರ್ಮೂಲ ಮಾಡಬೇಕೆಂದು ಆ ದಾನವು ಮಾಡಲ್ಪಟ್ಟಿದೆಯೋ, ಆ ಪಾಪಕ್ಕೇ ಪ್ರೋತ್ಸಾಹವನ್ನು ಕೊಡುತ್ತದೆ.

ಆಸ್ಪತ್ರೆಗಾಗಿಯೋ ಅಥವಾ ಇನ್ನು ಯಾವುದಾದರೂ ಸಾರ್ವಜನಿಕ ಉಪಕಾರಾರ್ಥ\-ವಾದ ಸಂಸ್ಥೆಗಾಗಿಯೋ ಸ್ತ್ರೀಪುರುಷರು ರಾತ್ರಿಯೆಲ್ಲ ನೃತ್ಯ ಮಾಡುವರು. ಅನಂತರ ಮನೆಗೆ ಹೋಗಿ ಮೃಗಗಳಂತೆ ವರ್ತಿಸಿ ಬಂದೀಖಾನೆ, ಹುಚ್ಚರ ಆಸ್ಪತ್ರೆ ಮತ್ತು ಆಸ್ಪತ್ರೆಗಳು–ಇವನ್ನು ತುಂಬಲು ಸಂತಾನೋತ್ಪತ್ತಿಯನ್ನು ಮಾಡುವರು. ಆಸ್ಪತ್ರೆ ಮುಂತಾದುವನ್ನು ಕಟ್ಟುವುದನ್ನೇ ಒಳ್ಳೆಯ ಕೆಲಸವೆನ್ನುವರು! ಒಳ್ಳೆಯ ಕೆಲಸದ ಉದ್ದೇಶ ಪ್ರಪಂಚದ ದುಃಖವನ್ನು ತಗ್ಗಿಸುವುದು ಅಥವಾ ಅದನ್ನು ನಿರ್ಮೂಲ ಮಾಡುವುದಾಗಿರಬೇಕು. ಮನಸ್ಸನ್ನು ನಿಗ್ರಹಿಸದೇ\break ಇರುವುದರಿಂದಲೇ ದುಃಖವೆಲ್ಲಾ ಬರುವುದು ಎಂದು ಯೋಗಿಗಳು ಹೇಳುವರು. ಪ್ರಕೃತಿಯ ಪಾಶದಿಂದ ಮುಕ್ತನಾಗುವುದು ಯೋಗಿಯ ಆದರ್ಶ. ಪ್ರಕೃತಿಯನ್ನು ಗೆಲ್ಲುವುದೇ ಅವನ ಕರ್ಮದ ಗುರಿ. ಎಲ್ಲಾ ಶಕ್ತಿ ಆತ್ಮನಲ್ಲಿದೆ. ದೇಹವನ್ನು ಮತ್ತು ಮನಸ್ಸನ್ನು ನಿಗ್ರಹಿಸುವುದರಿಂದ ಬರುವ ಆತ್ಮಶಕ್ತಿಯ ಬಲದಿಂದ ಅವನು ಪ್ರಕೃತಿಯನ್ನು ಗೆಲ್ಲುವನು ಎಂದು ಯೋಗಿ ಹೇಳುತ್ತಾನೆ.

ನೀವು ಮಾಡುವ ದೈಹಿಕ ಶ್ರಮಕ್ಕೆ ಅತ್ಯಗತ್ಯವಾದುದಕ್ಕಿಂತ ಕಿಂಚಿತ್​ ಹೆಚ್ಚು ಮಾಂಸಖಂಡಗಳನ್ನು ಬೆಳೆಸಿಕೊಳ್ಳುವಿರಾದರೆ ಅಷ್ಟರಮಟ್ಟಿಗೆ ನಿಮ್ಮ ಮಿದುಳಿನ ಶಕ್ತಿಯನ್ನು ಕುಗ್ಗಿಸಿಕೊಳ್ಳುವಿರಿ. ಹೆಚ್ಚು ಅಂಗಸಾಧನೆಯನ್ನು ಮಾಡಬೇಡಿ. ಅದು ಹಾನಿಕಾರಿ. ಯಾರು ಮಿತಿಮೀರಿ ಶ್ರಮಪಡುವುದಿಲ್ಲವೋ ಅವರು ದೀರ್ಘಕಾಲ ಬಾಳುವರು. ಕಡಮೆ ಊಟ ಮಾಡಿ, ಕಡಮೆ ಕೆಲಸಮಾಡಿ, ಮಿದುಳಿನ ಶಕ್ತಿಯನ್ನು ಕೂಡಿಟ್ಟುಕೊಳ್ಳಿ.

ಗೃಹಿಣಿಗೆ ಮನೆಯ ಕೆಲಸವೇ ಸಾಕು. ಸುಮ್ಮನೆ ಶಕ್ತಿಯನ್ನು ಅತಿಯಾಗಿ ಖರ್ಚುಮಾಡಬೇಡಿ. ಅದನ್ನು ನಿಧಾನವಾಗಿ ಖರ್ಚುಮಾಡಿ. ಸರಿಯಾದ ಆಹಾರವೆಂದರೆ ಸರಳವಾಗಿರುವ ಆಹಾರ. ಹೆಚ್ಚು ಮಸಾಲೆಗಳನ್ನು ಬೆರೆಸಿದುದಲ್ಲ.

