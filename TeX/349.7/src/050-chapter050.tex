
\chapter[ಜ್ಞಾನಯೋಗ ]{ಜ್ಞಾನಯೋಗ \protect\footnote{\engfoot{C.W. Vol. V, p270}}}

ಜೀವಿಗಳೆಲ್ಲ ಆಟದಲ್ಲಿ ನಿರತವಾಗಿದೆ. ಕೆಲವರಿಗೆ ಇದು ಗೊತ್ತಿದೆ ಮತ್ತೆ ಕೆಲವರಿಗೆ ಇದು ಗೊತ್ತಿಲ್ಲ. ಧರ್ಮ ಎಂದರೆ ತಿಳಿದು ಆಟವಾಡುವುದನ್ನು ಕಲಿಯುವುದು.

ಪ್ರಾಪಂಚಿಕ ಜೀವನದಲ್ಲಿ ಯಾವ ನಿಯಮ ಜಾರಿಯಲ್ಲಿದೆಯೊ, ಅದೇ ನಿಯಮ ಆಧ್ಯಾತ್ಮಿಕ ಜೀವನದಲ್ಲಿ ಮತ್ತು ವಿಶ್ವಜೀವನದಲ್ಲೆಲ್ಲಾ ಜಾರಿಯಲ್ಲಿರುವುದು. ಇದು ಒಂದು ಸರ್ವವ್ಯಾಪಿಯಾದ ನಿಯಮ. ಧರ್ಮಜೀವನದಲ್ಲಿ ಒಂದು ನಿಯಮ ಇದೆ, ಲೌಕಿಕ ಜೀವನದಲ್ಲಿ ಬೇರೊಂದು ನಿಯಮ ಇದೆ ಎಂದಲ್ಲ. ದೇವರಿಗೂ ಮಾನವನಿಗೂ ವ್ಯತ್ಯಾಸ ಇರುವುದು ತರತಮದಲ್ಲಿ ಮಾತ್ರ.

ಪಾಶ್ಚಾತ್ಯರಲ್ಲಿ ದೇವತಾಶಾಸ್ತ್ರಜ್ಞರು ದಾರ್ಶನಿಕರು ಮತ್ತು ವಿಜ್ಞಾನಿಗಳು ತಾವು ಕಾಲವಾದ ಮೇಲೆ ಜೀವಿಸುವೆವು ಎನ್ನುವುದಕ್ಕೆ ಪ್ರಮಾಣವನ್ನು ಹುಡುಕಾಡುತ್ತಿರುವರು. ಎಷ್ಟು ಕ್ಷುದ್ರವಾದ ವಿಷಯ ಅದು! ಚಿಂತಿಸುವುದಕ್ಕೆ ಇದಕ್ಕೂ ಮೇಲಾದ ವಿಷಯಗಳಿವೆ. ನೀವು ಎಂದಾದರೂ ಸಾಯುವಿರಿ ಎಂಬುದು ಎಷ್ಟು ದೊಡ್ಡ ಮೌಢ್ಯ. ನಾವು ಸಾಯುವುದಿಲ್ಲ ಎಂದು ಹೇಳುವುದಕ್ಕೆ ಯಾವ ಪುರೋಹಿತನೂ ಭೂತಪ್ರೇತವೂ ಬೇಕಾಗಿಲ್ಲ. ಇದೇ ಎಲ್ಲಕ್ಕಿಂತ ಸ್ವತಸ್ಸಿದ್ಧವಾದ ಸತ್ಯ. ಯಾರೂ ತಮ್ಮ ನಾಶವನ್ನು ತಾವೇ ಕಲ್ಪಿಸಿಕೊಳ್ಳಲಾರರು. ಅಮರತ್ವದ ಭಾವನೆ ಮಾನವನಲ್ಲಿ ಅಂತಸ್ಥವಾಗಿದೆ.

ಜೀವನ ಇರುವೆಡೆ ಮರಣವೂ ಇರುವುದು. ಜೀವನವು ಮರಣದ ಛಾಯೆ, ಮರಣವು ಜೀವನದ ಛಾಯೆ. ಇವುಗಳಿಗೆ ಇರುವ ವ್ಯತ್ಯಾಸ ಬಹಳ ಸೂಕ್ಷ್ಮ; ಗ್ರಹಿಸುವುದಕ್ಕೆ ಬಹಳ ಕಷ್ಟ. ಇದರ ಮೇಲೆ ಮನಸ್ಸನ್ನು ನಿಲ್ಲಿಸುವುದು ಮತ್ತೂ ಕಷ್ಟ.

ನಾವು ಎಂದೆಂದಿಗೂ ಒಂದು ಸರಳ ರೇಖೆಯಂತೆ ಪ್ರಗತಿ ಹೊಂದುತ್ತಿರುವೆವು ಎಂಬುದನ್ನು ನಾನು ನಂಬುವುದಿಲ್ಲ. ಇದೊಂದು ಅಪಹಾಸ್ಯ. ಇದನ್ನು ನಂಬುವುದಕ್ಕೆ ಆಗುವುದಿಲ್ಲ. ಚಲನೆ ಸದಾ ಸರಳರೇಖೆಯಲ್ಲಿ ಹೋಗಲಾರದು. ಒಂದು ಸರಳ ರೇಖೆಯನ್ನು ಮಿತಿಮೀರಿ ಮುಂದುವರಿಸಿದರೆ ಅದೊಂದು ವೃತ್ತವಾಗುವುದು. ಹೊರಕ್ಕೆ ಹೊರಟ ಶಕ್ತಿಯು ವೃತ್ತದಲ್ಲಿ ಚಲನೆಯನ್ನು ಪೂರೈಸಿ ಹೊರಟ ಸ್ಥಾನಕ್ಕೇ ಮತ್ತೆ ಬರುವುದು.

ಸರಳ ರೇಖೆಯಲ್ಲಿ ನಡೆಯುವ ಅಭಿವೃದ್ಧಿ ಎಂಬುದಿಲ್ಲ. ಪ್ರತಿಯೊಂದು ಜೀವವೂ ಒಂದು ವೃತ್ತದಲ್ಲಿ ಚಲಿಸುತ್ತಿದೆ. ಅದು ವೃತ್ತವನ್ನು ಪೂರೈಸಬೇಕು. ಯಾವ ಜೀವವೂ\break ಸದಾ ಕೆಳಗೆ ಇರಲಾರದು. ಆಗ ಅದು ಎಷ್ಟೇ ಕೆಳಗೆ ಇದ್ದರೂ ಮೇಲಕ್ಕೆ ಬರಬೇಕಾದ\break ಸಮಯ ಬಂದೇ ಬರುತ್ತದೆ. ಅದು ಮೊದಲು ಕೆಳಗೆ ಇರಬಹುದು. ಆದರೆ ವೃತ್ತವನ್ನು ಪೂರೈಸಬೇಕಾದರೆ ಅದು ಮೇಲಕ್ಕೆ ಏರಬೇಕು. ನಾವೆಲ್ಲ ದೇವರೆಂಬ ಒಂದೇ ಕೇಂದ್ರದಿಂದ ಹೊರಬಂದಿರುವೆವು. ನಮ್ಮ ಪ್ರಯಾಣವನ್ನೆಲ್ಲ ಪೂರೈಸಿದ ಮೇಲೆ ಪುನಃ ಹೊರಟ ಸ್ಥಳಕ್ಕೇ ಮರಳುತ್ತೇವೆ.

ಪ್ರತಿಯೊಂದು ಜೀವವೂ ಒಂದು ವೃತ್ತ. ದೇಹ ಇರುವ ಕಡೆ ಅದರ ಕೇಂದ್ರ ಇರುವುದು. ಚಟುವಟಿಕೆ ಅಲ್ಲಿ ವ್ಯಕ್ತವಾಗುವುದು. ನೀನು ಒಂದೇ ಕಡೆ ಇರುವೆ ಎಂದು\break ತಿಳಿದಿದ್ದರೂ ಸರ್ವವ್ಯಾಪಿಯಾಗಿರುವೆ. ನೀನಿರುವ ಕೇಂದ್ರ ಹಲವು ದ್ರವ್ಯಗಳನ್ನು ತನ್ನೆಡೆಗೆ ಸೆಳೆದುಕೊಂಡು ತನ್ನನ್ನು ವ್ಯಕ್ತಗೊಳಿಸಿಕೊಳ್ಳುವುದಕ್ಕಾಗಿ ಒಂದು ಯಂತ್ರವನ್ನು ಮಾಡಿದೆ. ಯಾವುದರ ಮೂಲಕ ಇದು ವ್ಯಕ್ತವಾಗುವುದೋ ಅದನ್ನು ದೇಹ ಎನ್ನುವರು. ನೀನು ಸರ್ವವ್ಯಾಪಿಯಾಗಿರುವೆ. ಒಂದು ದೇಹ ಅಥವಾ ಯಂತ್ರ ಕೆಲಸ ಮಾಡುವುದಕ್ಕೆ\break ಅಯೋಗ್ಯವಾದರೆ ಕೇಂದ್ರ ಬದಲಾಗುವುದು. ಇತರ ಸೂಕ್ಷ್ಮ ಅಥವಾ ಸ್ಥೂಲ ದ್ರವ್ಯಗಳನ್ನು ತೆಗೆದುಕೊಂಡು ಅವುಗಳ ಮೂಲಕ ಮತ್ತೊಂದು ದೇಹವನ್ನು ಸೃಷ್ಟಿಸಿ ಅದರ ಮೂಲಕ ಕೆಲಸ ಮಾಡುವುದು. ಅವನ ಮೂಲಕ ಮತ್ತೊಂದು ದೇಹವನ್ನು ಸೃಷ್ಟಿಸಿ ಅದರ ಮೂಲಕ ಕೆಲಸ ಮಾಡುವುದು. ಅವನೇ ಮಾನವ. ಹಾಗಾದರೆ ದೇವರೇನು? ದೇವರೂ ಒಂದು ವೃತ್ತದಂತೆ, ಅದಕ್ಕೆ ಪರಿಧಿ ಎಲ್ಲಿಯೂ ಇಲ್ಲ; ಕೇಂದ್ರ ಎಲ್ಲಾ ಕಡೆಗಳಲ್ಲಿಯೂ ಇದೆ. ಈ ವೃತ್ತದಲ್ಲಿ ಪ್ರತಿಯೊಂದು ಕೇಂದ್ರವೂ ಸಜೀವವಾಗಿರುವುದು; ಪ್ರಜ್ಞೆಯಿಂದ ಕೂಡಿರುವುದು, ಜಾಗೃತವಾಗಿರುವುದು; ಕರ್ಮದಲ್ಲಿ ನಿರತವಾಗಿರುವುದು. ಮಿತಿಯಿಂದ ಕೂಡಿದ ನಮ್ಮ ಜೀವನಿಗೆ ಒಂದು ಕೇಂದ್ರ ಮಾತ್ರ ಗೊತ್ತಿದೆ. ಆ ಕೇಂದ್ರ ಮುಂದಕ್ಕೆ ಹಿಂದಕ್ಕೆ ಚಲಿಸುತ್ತಿರುವುದು.

ಜೀವ ಒಂದು ವೃತ್ತದಂತೆ. ಅದಕ್ಕೆ ಪರಿಧಿ ಎಲ್ಲಿಯೂ ಇಲ್ಲ; ಆದರೆ ಅದರ ಕೇಂದ್ರ ಯಾವುದೋ ಒಂದು ದೇಹದಲ್ಲಿರುವುದು. ಮೃತ್ಯುವೆಂದರೆ ಕೇಂದ್ರದ ಬದಲಾವಣೆ. ದೇವರು ಒಂದು ವೃತ್ತದಂತೆ; ಅದರ ಪರಿಧಿ ಎಲ್ಲಿಯೂ ಇಲ್ಲ. ಕೇಂದ್ರದಿಂದ ಪಾರಾದರೆ ಆಗ ನಮ್ಮ ನೈಜಸ್ವರೂಪವಾದ ಆತ್ಮನನ್ನು ಅರಿಯುವೆವು.

ಒಂದು ಮಹಾಪ್ರವಾಹ ಸಮುದ್ರದ ಕಡೆ ಭೋರ್ಗರೆದು ಹರಿಯುತ್ತಿದೆ. ಆ\break ಪ್ರವಾಹದ ಮೇಲೆ ಚೂರು ಕಾಗದ, ಕಸಕಡ್ಡಿ ಮುಂತಾದುವೆಲ್ಲಾ ಅಲ್ಲಿ ಇಲ್ಲಿ ತೇಲುತ್ತಿವೆ. ಅವು ಹಿಂತಿರುಗಿ ಹೋಗುವುದಕ್ಕೆ ಯತ್ನಿಸಬಹುದು. ಆದರೆ ಕೊನೆಗೆ ಅವೆಲ್ಲಾ ಸಮುದ್ರಕ್ಕೆ ಸೇರಬೇಕಾಗಿದೆ. ನಾನು ನೀವೆಲ್ಲಾ ಆ ಹುಚ್ಚುಹೊಳೆಯಲ್ಲಿ ತೇಲುತ್ತಿರುವ ಕಸಕಡ್ಡಿಯಂತೆ. ಈ ಹುಚ್ಚುಹೊಳೆ ಜೀವನವೆಂಬ ಮಹಾಸಾಗರದ ಕಡೆಗೆ, ಪೂರ್ಣತೆಯ ಕಡೆಗೆ, ದೇವರೆಡೆಗೆ ಹರಿಯುತ್ತಿದೆ. ನೀವು ಹಿಂತಿರುಗಿ ಹೋಗಲು ಯತ್ನಿಸಬಹುದು. ಪ್ರವಾಹಕ್ಕೆ ವಿರುದ್ಧವಾಗಿ ಹೋಗಲು ಬೇಕಾದಷ್ಟು ಯತ್ನ ಮಾಡಬಹುದು. ಆದರೆ ಕೊನೆಗೆ ಈ ಸಚ್ಚಿದಾನಂದ ಸಾಗರವನ್ನು ಸೇರಲೇಬೇಕಾಗಿದೆ.

ಜ್ಞಾನ ಎಂದರೆ ಪಂಥೀಯತೆ ಇಲ್ಲದೆ ಇರುವುದು. ಅಂದರೆ ಪಂಥಗಳನ್ನು ದ್ವೇಷಿಸುವುದು ಎಂದಲ್ಲ. ಪಂಥಗಳಿಗೆ ಅತೀತವಾದ ಒಂದು ಸ್ಥಿತಿಯನ್ನು ಪಡೆಯುವುದು ಎಂದರ್ಥ. ಜ್ಞಾನಿಯು ಯಾವುದನ್ನೂ ಧ್ವಂಸಮಾಡಲು ಒಪ್ಪುವುದಿಲ್ಲ. ಎಲ್ಲರಿಗೂ ಸಹಾಯಮಾಡಲು ಯತ್ನಿಸುತ್ತಾನೆ. ನದಿಗಳೆಲ್ಲಾ ಸಾಗರಕ್ಕೆ ಸೇರಿ ಒಂದಾಗುವುವು. ಇದರಂತೆಯೇ ಎಲ್ಲಾ\break ಪಂಥಗಳೂ ಜ್ಞಾನಕ್ಕೆ ಸೇರಿ ಒಂದಾಗಬೇಕು. ಜ್ಞಾನವು ಪ್ರಪಂಚವನ್ನು ತ್ಯಾಗಮಾಡು ಎನ್ನುತ್ತದೆ. ಅಂದರೆ, ಪ್ರಪಂಚವನ್ನು ಬಿಟ್ಟುಹೋಗು ಎಂದಲ್ಲ. ಪ್ರಪಂಚದಲ್ಲಿದ್ದೂ ಇಲ್ಲದಂತೆ ಇರುವುದೇ ತ್ಯಾಗದ ನಿಜವಾದ ಪರೀಕ್ಷೆ.

ಮೊದಲಿನಿಂದಲೂ ಜ್ಞಾನವೆಲ್ಲ ನಮ್ಮಲ್ಲಿಯೇ ಸುಪ್ತವಾಗಿ ಇರಬೇಕಲ್ಲದೆ ಅದು ಬೇರೆ ಕಡೆ ಇರಲಾರದು. ನಾನು ನೀವು ಸಣ್ಣ ಸಣ್ಣ ಅಲೆಗಳಾದರೆ ಸಾಗರವೆ ಇವೆರಡಕ್ಕೂ ಹಿನ್ನೆಲೆ.

ದ್ರವ್ಯ ಮನಸ್ಸು ಆತ್ಮ ಇವುಗಳಲ್ಲಿ ನಿಜವಾಗಿಯೂ ಯಾವ ವ್ಯತ್ಯಾಸವೂ ಇಲ್ಲ. ಇವು ಒಂದೇ ಅನುಭವದ ಬೇರೆ ಬೇರೆ ಸ್ಥಿತಿಗಳು ಅಷ್ಟೆ. ಈ ಪ್ರಪಂಚವು ಪಂಚೇಂದ್ರಿಯಗಳ ಮೂಲಕ ನೋಡಿದಾಗ ದ್ರವ್ಯವಾಗಿ ಕಾಣಿಸುವುದು. ದುಷ್ಟರಿಗೆ ಇದು ನರಕವಾಗಿ ಕಾಣುವುದು, ಒಳ್ಳೆಯವರಿಗೆ ಸ್ವರ್ಗವಾಗಿ ಕಾಣುವುದು. ಪುಣ್ಯಾತ್ಮರಿಗೆ ದೇವರಂತೆ ತೋರುವುದು.

ಬ್ರಹ್ಮವೊಂದೇ ಸತ್ಯ ಎಂಬುದು ಇಂದ್ರಿಯಗೋಚರವಾಗುವಂತೆ ಮಾಡಲಾಗುವುದಿಲ್ಲ. ಆದರೆ ಇದೊಂದೇ ನಿರ್ಣಯ ನಮಗೆ ಸಾಧ್ಯ ಎಂಬುದನ್ನು ತೋರಬಹುದು.\break ಉದಾಹರಣೆಗೆ ಏಕತೆ ಎಲ್ಲಾ ಕಡೆಗಳಲ್ಲಿಯೂ ಇರಬೇಕು. ಸಾಧಾರಣ ವಸ್ತುಗಳಲ್ಲಿ ಕೂಡ ಇದು ಇರಬೇಕು. ಉದಾಹರಣೆಗೆ ಮಾನವರೆಲ್ಲ ಒಂದೇ ಎಂಬ ಸರ್ವಸಾಮಾನ್ಯತಾ ನಿಯಮವಿದೆ. ವೈವಿಧ್ಯವೆಲ್ಲ ನಾಮರೂಪಗಳಿಂದ ಆಯಿತು ಎನ್ನುತ್ತೇವೆ. ಆದರೂ ಅದನ್ನು ಗ್ರಹಿಸಲು ಯತ್ನಿಸಿದರೆ ಅದು ಎಲ್ಲಿಯೂ ಇಲ್ಲ. ಬರಿಯ ನಾಮರೂಪ, ಕಾರ್ಯಕಾರಣಗಳು ಪ್ರತ್ಯೇಕವಾಗಿ ನಿಂತಿರುವುದನ್ನು ನಾವು ಎಲ್ಲಿಯೂ ನೋಡುವುದಿಲ್ಲ. ಆದಕಾರಣ ಈ ದೃಶ್ಯ ಪ್ರಪಂಚ ಮಾಯೆ, ಇದು ಅವ್ಯಕ್ತದ ಮೇಲೆ ನಿಂತಿರುವುದು. ಅವ್ಯಕ್ತವನ್ನು ಬಿಟ್ಟರೆ ಇದು ಇಲ್ಲ. ಸಮುದ್ರದ ಒಂದು ಅಲೆಯನ್ನು ತೆಗೆದುಕೊಳ್ಳಿ. ಸಮುದ್ರದ ನೀರು ಆ ಸ್ಥಿತಿಯಲ್ಲಿರುವವರೆಗೆ ಆ ಅಲೆ ಇರುವುದು. ಆದರೆ ನೀರು ಅಲೆಯನ್ನು ಅನುಸರಿಸಿಕೊಂಡಿಲ್ಲ. ಅಲೆ ಇಲ್ಲದೇ ಇದ್ದರೂ ಸಾಗರ ಇರುವುದು.

ಇರುವುದು ಒಂದೇ ಸತ್ಯ, ಅನೇಕದಂತೆ ಕಾಣಿಸುವುದಕ್ಕೆ ಮನಸ್ಸೇ ಕಾರಣ. ನಾವು ಅನೇಕವನ್ನು ನೋಡಿದಾಗ ಏಕವು ಮಾಯವಾಗುವುದು; ನಾವು ಏಕವನ್ನು ನೋಡಿದೊಡನೆಯೆ ಅನೇಕವು ಮಾಯವಾಗುವುದು. ನೀವು ನಿತ್ಯ ಜೀವನದಲ್ಲಿ ಏಕವನ್ನು ನೋಡಿದಾಗ ಅನೇಕವನ್ನು ನೋಡಲಾರಿರಿ. ಮೊದಲು ನೀವು ಏಕದಿಂದ ಪ್ರಾರಂಭ ಮಾಡುವಿರಿ. ಚೈನಾದ ಮನುಷ್ಯನಿಗೆ ಅಮೆರಿಕನ್ನರ ಮಧ್ಯೆ ಒಬ್ಬನಿಗೂ ಮತ್ತೊಬ್ಬನಿಗೂ ವ್ಯತ್ಯಾಸ ಕಾಣಿಸುವುದಿಲ್ಲ. ಇದೊಂದು ವಿಚಿತ್ರ. ಹಾಗೆಯೆ ಇಬ್ಬರು ಚೀನೀಯರಿಗೆ ಇರುವ ವ್ಯತ್ಯಾಸ ನಿಮಗೆ ಕಾಣಿಸುವುದಿಲ್ಲ.

ನಮಗೆ ವಿಷಯಗಳು ಅರಿವಾಗುವಂತೆ ಮಾಡುವುದು ಮನಸ್ಸೇ ಎಂಬುದನ್ನು ತೋರಿಸಬಹುದು. ಕೆಲವು ವೈಶಿಷ್ಟ್ಯಗಳು ಇರುವ ವಸ್ತುಗಳು ಮಾತ್ರ ನಮಗೆ ತಿಳಿದಿರುವ ಮತ್ತು ತಿಳಿಯಬಲ್ಲಂತಹ ವಸ್ತುಗಳಾಗಬಲ್ಲವು. ಯಾವುದಕ್ಕೆ ಗುಣವಿಲ್ಲವೊ ನಾವು ಅದನ್ನು ಅರಿಯಲಾರೆವು. ಉದಾಹರಣೆಗೆ \enginline{X} ಎಂಬ ಬಾಹ್ಯ ಪ್ರಪಂಚವಿದೆ ಎನ್ನಿ. ಅದು ಅಜ್ಞಾತ ಮತ್ತು ಅಜ್ಞೇಯವಾದುದು. ನಾನು ಅದನ್ನು ನೋಡಿದಾಗ \enginline{X+} ಮನಸ್ಸು ಆಗುವುದು. ನಾನು ಪ್ರಪಂಚವನ್ನು ನೋಡಲೆತ್ನಿಸಿದಾಗ ಮನಸ್ಸೇ ಮುಕ್ಕಾಲುಪಾಲು ಅದರಲ್ಲಿ ಇರುವುದು.\break ಆಂತರಿಕ ಪ್ರಪಂಚವೇ \enginline{Y+} ಮನಸ್ಸು. ಬಾಹ್ಯಪ್ರಪಂಚ \enginline{X+} ಮನಸ್ಸು. ಬಾಹ್ಯ ಮತ್ತು\break ಆಂತರಿಕ ಪ್ರಪಂಚಗಳೊಳಗೆ ಇರುವ ವ್ಯತ್ಯಾಸವೆಲ್ಲ ಮನಸ್ಸಿನಿಂದ ಆಗಿದೆ. ನಿಜವಾಗಿ\break ಇರುವುದು ಅಜ್ಞಾತವೂ ಅಜ್ಞೇಯವೂ ಆಗಿರುವುದು. ಇದು ಜ್ಞಾನಕ್ಕೆ ಅತೀತವಾಗಿರುವುದು. ಯಾವುದು ಜ್ಞಾನಕ್ಕೆ ಅತೀತವಾಗಿರುವುದೋ ಅದರಲ್ಲಿ ಭಿನ್ನತೆಗಳಿಲ್ಲ. ಆದಕಾರಣ,\break ಹೊರಗಿನ \enginline{X} ಮತ್ತು \enginline{Y} ಒಳಗಿರುವ ಒಂದೇ ಆದಕಾರಣ ಸತ್ಯ ಒಂದೇ ಆಗಿರುವುದು.

ದೇವರು ವಿಚಾರ ಮಾಡುವುದಿಲ್ಲ. ನಿನಗೆ ಗೊತ್ತಿದ್ದ ಮೇಲೆ ನೀನು ಏತಕ್ಕೆ ವಿಚಾರ ಮಾಡುವೆ? ಕೆಲವು ವಿಷಯಗಳನ್ನು ತಿಳಿದುಕೊಳ್ಳುವುದಕ್ಕೆ ಕ್ರಿಮಿಗಳಂತೆ ನಾವು ತೆವಳಿಕೊಂಡು ಹೋಗುವುದು ದುರ್ಬಲತೆಯ ಚಿಹ್ನೆ. ಸ್ವಲ್ಪ ತಿಳಿದೊಡನೆ ಎಲ್ಲಾ ಪುನಃ\break ಕುಸಿದು ಬೀಳುವುದು. ಆತ್ಮವು ಮನಸ್ಸಿನಲ್ಲಿಯೂ ಮತ್ತು ಪ್ರತಿಯೊಂದು ವಸ್ತುವಿನಲ್ಲಿಯೂ ಪ್ರತಿಬಿಂಬಿಸಲ್ಪಡುತ್ತಿದೆ. ಆತ್ಮಜ್ಯೋತಿಯೇ ಮನಸ್ಸನ್ನು ಚೇತನಾತ್ಮಕವನ್ನಾಗಿ ಮಾಡುವುದು. ಎಲ್ಲವೂ ಆತ್ಮದ ಆವಿರ್ಭಾವ. ನಮ್ಮ ಮನಸ್ಸುಗಳು ಹಲವು ಕನ್ನಡಿಗಳಂತೆ. ಪ್ರೀತಿ, ಅಂಜಿಕೆ, ದ್ವೇಷ, ಪಾಪ ಪುಣ್ಯ ಇವೆಲ್ಲ ಆತ್ಮನ ಪ್ರತಿಬಿಂಬಗಳು. ಪ್ರತಿಬಿಂಬಕವು ಸ್ವಚ್ಛವಾಗಿಲ್ಲದೇ ಇದ್ದರೆ ಪ್ರತಿಬಿಂಬ ಚೆನ್ನಾಗಿ ಇರುವುದಿಲ್ಲ.

ನಿಜವಾದ ಅಸ್ತಿತ್ವಕ್ಕೆ ಯಾವ ಬಾಹ್ಯ ಅಭಿವ್ಯಕ್ತಿಯೂ ಇಲ್ಲ. ನಾವು ಅದನ್ನು ಗ್ರಹಿಸಲಾರೆವು. ಏತಕ್ಕೆ ಎಂದರೆ ನಾವು ಅದನ್ನು ಒಂದು ಮನಸ್ಸಿನ ಮೂಲಕ ಗ್ರಹಿಸಬೇಕಾಗುವುದು. ಆದರೆ ಮನಸ್ಸೇ ಒಂದು ಅಭಿವ್ಯಕ್ತಿಯಾಗಿದೆ. ಆತ್ಮವು ಅಗ್ರಾಹ್ಯ. ಅದೇ ಅದರ ಮಹಿಮೆ. ಬೆಳಕಿನ ಅತಿ ತೀವ್ರ ಅಥವಾ ಅತಿ ಮಂದ ಸ್ಪಂದನಗಳೆರಡೂ ನಮಗೆ ಕಾಣಿಸುವುದಿಲ್ಲ ಎಂಬುದನ್ನು ನೆನಪಿನಲ್ಲಿಡಬೇಕು. ಆದರೆ ಇವುಗಳಲ್ಲಿ ಧ್ರುವಗಳಷ್ಟು ಅಂತರವಿದೆ. ಈಗ ಗೊತ್ತಾಗದಂತಹ, ಮುಂದೆ ಗೊತ್ತಾಗುವಂತಹ ಕೆಲವು ವಿಷಯಗಳು ಇವೆ. ನಮ್ಮ ಅಜ್ಞಾನದಿಂದಾಗಿ ಅವನ್ನು ನಾವು ಈಗ ತಿಳಿಯುತ್ತಿಲ್ಲ. ಆದರೆ ಮತ್ತು ಕೆಲವು ಇವೆ, ನಾವು ಅವನ್ನು ಎಂದೆಂದಿಗೂ ತಿಳಿಯುವಂತೆಯೇ ಇಲ್ಲ. ಏಕೆಂದರೆ ಅವು ಪರಮಜ್ಞಾನದ ಕಂಪನಗಳಿಗೂ ಅತೀತವಾಗಿರುವುವು. ನಮಗೆ ಗೊತ್ತಾಗದೇ ಇದ್ದರೂ ನಾವು ಸದಾಕಾಲದಲ್ಲಿಯೂ ನಿತ್ಯರಾಗಿರುವೆವು. ಅಲ್ಲಿ ಜ್ಞಾನವೂ ಅಸಾಧ್ಯ. ಅದನ್ನು ತಿಳಿಯಲಾರದೆ ಇರುವುದೇ ಅದರ\break ಅಸ್ತಿತ್ವಕ್ಕೆ ಮೂಲಾಧಾರವಾಗಿದೆ. ಉದಾಹರಣೆಗೆ ನನ್ನಲ್ಲಿ ನನ್ನಷ್ಟು ನಿಜವಾಗಿರುವುದು\break ಮತ್ತಾವುದೂ ಇಲ್ಲ. ಆದರೂ ನಾನು ಅದನ್ನು ದೇಹದಂತೆಯೊ, ಮನಸ್ಸಿನಂತೆಯೊ,\break ಗಂಡಸಿನಂತೆಯೊ, ಹೆಂಗಸಿನಂತೆಯೊ, ಸುಖಿಯಾಗಿಯೊ, ದುಃಖಿಯಾಗಿಯೊ ಇದ್ದೆ ಎಂದು ಮಾತ್ರ ಭಾವಿಸಬಲ್ಲೆ. ಆದರೂ ಅದರ ನೈಜಸ್ಥಿತಿಯನ್ನು ಅರಿಯಲು ನಾನು ಯತ್ನಿಸುವೆನು. ಹಾಗೆ ಅರಿಯಲು ಅದನ್ನು ಅಧೋಗತಿಗೆ ಎಳೆಯುವುದಲ್ಲದೆ ಬೇರೆ ದಾರಿಯೇ ಇಲ್ಲ. ಆದರೂ ನನಗೆ ಆ ಸತ್ಯವು ಖಚಿತವಾಗಿ ಗೊತ್ತಿದೆ. “ಓ ಪ್ರಿಯೆ, ಯಾರೂ ಪತಿಯನ್ನು ಪತಿಗಾಗಿ ಪ್ರೀತಿಸುವುದಿಲ್ಲ. ಅವನಲ್ಲಿರುವ ಆತ್ಮನಿಗಾಗಿ ಪ್ರೀತಿಸುವರು. ಆತ್ಮನ ಮೂಲಕ ಮಾತ್ರವೇ ಅವಳು ಪತಿಯನ್ನು ಪ್ರೀತಿಸುವಳು; ಓ ಪ್ರಿಯೆ, ಯಾರೂ ಸತಿಯನ್ನು ಸತಿಗಾಗಿ ಪ್ರೀತಿಸುವುದಿಲ್ಲ, ಆತ್ಮನ ಮೂಲಕ ಮಾತ್ರವೇ ಅವನು ಸತಿಯನ್ನು ಪ್ರೀತಿಸುವನು.” ನಮಗೆ ಗೊತ್ತಿರುವ ಸತ್ಯ ಅದೊಂದೇ. ಏಕೆಂದರೆ ಅದರಿಂದ ಮತ್ತು ಅದರ ಮೂಲಕ ಮಾತ್ರ ನಾವು ಎಲ್ಲವನ್ನೂ ಅರಿಯುವೆವು. ಆದರೂ ನಾವು ಅದನ್ನು ಗ್ರಹಿಸಲಾರೆವು. ಅರಿಯುವವನನ್ನು\break ಅರಿಯುವುದು ಹೇಗೆ? ನಮಗೆ ಗೊತ್ತಾಗುವುದಾದರೆ ಅದು ಜ್ಞೇಯವಾಗುವುದು,\break ಜ್ಞಾತೃವಾಗುವುದಿಲ್ಲ, ಅದೊಂದು ದೃಶ್ಯವಸ್ತುವಾಗುವುದು.

ಪರಮ ಸಾಕ್ಷಾತ್ಕಾರವನ್ನು ಪಡೆದವರು ಹೀಗೆ ಹೇಳುವನು–“ನಾನೇ ರಾಜಾಧಿರಾಜ. ನನ\-ಗಿಂತ ಅಧಿಕರು ಯಾರೂ ಇಲ್ಲ. ನಾನೇ ದೇವಾಧಿದೇವ, ನನಗಿಂತ ಮಿಗಿಲಾದ ದೇವತೆ\-ಗಳಿಲ್ಲ. ಎರಡಿಲ್ಲದ ಏಕವಾದ ನಾನೊಬ್ಬನೇ ಇರುವುದು.” ವೇದಾಂತದ ಅದ್ವೈತ\break ಭಾವನೆ ಅನೇಕರಿಗೆ ಭಯಾನಕವಾಗಿ ಕಾಣಿಸುವುದು. ಹಾಗಾಗುವುದು ಮೂಢನಂಬಿಕೆಯ ದೆಸೆಯಿಂದ.

ನಾವು ಶಾಂತವಾದ ನಿತ್ಯಶಾಂತಿಯಲ್ಲಿರುವ ಆತ್ಮ. ನಾವು ಅಳಕೂಡದು, ಆತ್ಮಕ್ಕೆ\break ಅಳುವಿಲ್ಲ. ನಾವು ಭ್ರಾಂತರಾಗಿ ದೇವರು ತನ್ನ ಸಿಂಹಾಸನದ ಮೇಲೆ ಕುಳಿತು\break ಕನಿಕರದಿಂದ ಅಳುತ್ತಿರುವನೆಂದು ಭಾವಿಸುವೆವು. ಅಂತಹ ದೇವರನ್ನು ಪಡೆಯುವುದು ಯೋಗ್ಯವಲ್ಲ. ದೇವರು ಏತಕ್ಕೆ ಅಳಬೇಕು? ಅಳುವುದು ದೌರ್ಬಲ್ಯದ ಕುರುಹು,\break ಬಂಧನದ ಕುರುಹು.

ಶ್ರೇಷ್ಠವಾದುದನ್ನು ಅರಸಿ. ಯಾವಾಗಲೂ ಶ್ರೇಷ್ಠವಾದುದನ್ನೇ ಅರಸಿ. ಅನಂತಾನಂದ\-ವಿರುವುದು ಶ್ರೇಷ್ಠವಾದ ವಸ್ತುವಿನಲ್ಲೇ. ನಾನು ಬೇಟೆಯಾಡಬೇಕಾದರೆ ಸಿಂಹವನ್ನು ಬೇಟೆಯಾಡುವೆನು. ದರೋಡೆ ಮಾಡಬೇಕಾದರೆ ರಾಜನ ಬೊಕ್ಕಸವನ್ನು ದರೋಡೆ\break ಮಾಡುತ್ತೇನೆ, ಯಾವಾಗಲೂ ಶ್ರೇಷ್ಠವಾದುದನ್ನೇ ಅರಸಿ.

ಓ! ಆ ಒಂದು ವಸ್ತುವಿಗೆ ಮೇರೆಯನ್ನು ಕಲ್ಪಿಸಲಾಗುವುದಿಲ್ಲ. ಅದನ್ನು ವಿವರಿಸುವುದಕ್ಕೆ ಆಗುವುದಿಲ್ಲ. ಅದನ್ನು ನಮ್ಮ ಹೃದಯಾಂತರಾಳದಲ್ಲೇ ಅನುಭವಿಸಬಹುದು;\break ಅದು ಉಪಮಾತೀತ. ಅನಂತ ಆಕಾಶದಂತೆ ಅಸೀಮ, ಅವಿಕಾರಿ. ಹೇ ಪವಿತ್ರಾತ್ಮನೇ, ಆ ಪೂರ್ಣವನ್ನು ಅರಿ. ಮತ್ತೇನನ್ನೂ ಅರಸಬೇಡ.

ಯಾವುದು ಪ್ರಕೃತಿಯ ಬದಲಾವಣೆಗಳಿಗೆ ಅತೀತವಾಗಿರುವುದೋ, ಆಲೋಚನೆಗೆ ಮೀರಿರುವುದೋ, ಕೂಟಸ್ಥವೋ, ಅಚಲವೋ, ಶಾಸ್ತ್ರಗಳೆಲ್ಲ ಯಾವುದನ್ನು ಸ್ತುತಿಸುವುವೋ, ಋಷಿಗಳು ಯಾವುದನ್ನು ಪೂಜಿಸುವರೊ, ಅದನ್ನು ಮಾತ್ರ ಅರಸು ಪವಿತ್ರಾತ್ಮನೆ, ಉಳಿದುದನ್ನು ತ್ಯಜಿಸು.

ಇದು ಹೋಲಿಕೆಗೆ ನಿಲುಕದುದು, ಅನಂತವಾದುದು, ಏಕವಾದುದು. ಇದನ್ನು ಹೋಲಿಸುವುದಕ್ಕೇ ಆಗುವುದಿಲ್ಲ. ಮೇಲೆ ನೀರು, ಕೆಳಗೆ ನೀರು, ಎಡಗಡೆ ನೀರು; ನೀರಿನ ಮೇಲೆ ಅಲೆಯಿಲ್ಲ, ಯಾವ ಸುಳಿಯೂ ಇಲ್ಲ, ಎಲ್ಲಾ ಮೌನ, ನಿತ್ಯಾನಂದ. ಇದು ನಿನ್ನ ಹೃದಯಕ್ಕೆ ಗೊತ್ತಾಗುವುದು. ಮತ್ತಾವುದನ್ನೂ ಅರಸಬೇಡ.

\eject

ಅಳುವುದೇಕೆ ಸಹೋದರನೆ? ನಿನಗೆ ಮರಣವಿಲ್ಲ, ರೋಗವಿಲ್ಲ. ಅಳುವುದೇಕೆ\break ಸಹೋದರನೆ? ನಿನಗೆ ದುಃಖವಿಲ್ಲ, ದಾರಿದ್ರ್ಯವಿಲ್ಲ, ಅಳುವುದೇಕೆ ಸಹೋದರನೆ? ನಿನಗೆ ಬದಲಾವಣೆಯಿಲ್ಲ, ಸಾವಿಲ್ಲ, ಅಖಂಡ ಅಸ್ತಿತ್ವವೇ ನೀನಾಗಿರುವೆ.

ದೇವರೇನೆಂಬುದು ನನಗೆ ಗೊತ್ತು. ಅದನ್ನು ನಿನಗೆ ಹೇಳಲಾರೆ. ದೇವರೇನೆಂದು ನನಗೆ ಗೊತ್ತಿಲ್ಲ. ಅದನ್ನು ನಿನಗೆ ಹೇಳುವುದು ಹೇಗೆ? ನನ್ನ ಸಹೋದರನೆ, ನಿನಗೆ ಗೊತ್ತಿಲ್ಲ. ನೀನೆ ಅವನು, ನೀನೆ ಅವನು. ಅಲ್ಲಿ ಇಲ್ಲಿ ದೇವರನ್ನು ಹುಡುಕಿಕೊಂಡು ಏಕೆ ಅಲೆಯುವೆ? ಅರಸಬೇಡ, ಅದೇ ದೇವರು. ಆತ್ಮನಲ್ಲಿ ಪ್ರತಿಷ್ಠಿತನಾಗು.

ನೀನೇ ನಮ್ಮ ತಂದೆ ತಾಯಿ ಪ್ರಿಯಸಖ. ಪ್ರಪಂಚದ ಭಾರವನ್ನು ಹೊರುವವನು ನೀನು. ನಮ್ಮ ಜೀವನದ ಭಾರವನ್ನು ಹೊರಲು ನೆರವು ನೀಡು. ನೀನೇ ನಮ್ಮ ಸಖ, ಆಶ್ರಯ, ಪತಿ; ನೀನೇ ನಾವು.

