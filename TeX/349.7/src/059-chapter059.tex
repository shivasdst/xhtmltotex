
\chapter[ಧರ್ಮ ಮತ್ತು ವಿಜ್ಞಾನ ]{ಧರ್ಮ ಮತ್ತು ವಿಜ್ಞಾನ \protect\footnote{\engfoot{C.W. Vol. VI, p81}}}

ಅನುಭವವೇ ಜ್ಞಾನಕ್ಕೆ ಮೂಲ. ಜಗತ್ತಿನಲ್ಲಿ ಖಚಿತತೆ ಇಲ್ಲದ ಶಾಸ್ತ್ರವೆಂದರೆ ಧರ್ಮ ಒಂದೇ. ಏಕೆಂದರೆ ಇದನ್ನು ಒಂದು ಅನುಭವದ ಶಾಸ್ತ್ರದಂತೆ ಬೋಧಿಸಿಲ್ಲ. ಆದರೆ ಇದು ಹಾಗಿರಕೂಡದು. ಆದರೂ ತಮ್ಮ ಅನುಭವದಿಂದ ಧರ್ಮವನ್ನು ಬೋಧಿಸುವ ಕೆಲವರು ಇದ್ದಾರೆ. ಇವರನ್ನೇ ಅನುಭಾವಿಗಳು ಎನ್ನುವರು. ಪ್ರತಿಯೊಂದು ಧರ್ಮದಲ್ಲಿಯೂ ಈ ಅನುಭಾವಿಗಳು ಒಂದೇ ಸತ್ಯವನ್ನು ಬೋಧಿಸುವರು. ಇದೇ ನಿಜವಾದ ಧರ್ಮವಿಜ್ಞಾನ. ಹೇಗೆ ಗಣಿತಶಾಸ್ತ್ರ ಬೇರೆ ಬೇರೆ ದೇಶಗಳಲ್ಲಿ ಬೇರೆ ಬೇರೆಯಾಗಿಲ್ಲವೊ ಹಾಗೆಯೇ ಅನುಭಾವಿಗಳಲ್ಲಿ ಭಿನ್ನಾಭಿಪ್ರಾಯಗಳಿಲ್ಲ. ಇವರೆಲ್ಲ ಒಂದೇ ಸ್ಥಿತಿಯಲ್ಲಿರುವರು. ಇವರ ಅನುಭವ ಒಂದೇ. ಅದೇ ಒಂದು ನಿಯಮವಾಗುವುದು.

ಚರ್ಚಿಗೆ ಹೋಗುವವರು ಮೊದಲು ಅಲ್ಲಿ ಒಂದು ಧರ್ಮವನ್ನು ಕಲಿಯುತ್ತಾರೆ; ಅನಂತರ ಅಭ್ಯಾಸ ಮಾಡುತ್ತಾರೆ. ಅವರು ಅನುಭವವನ್ನು ತಮ್ಮ ನಂಬಿಕೆಗೆ ಆಧಾರವಾಗಿ ತೆಗೆದುಕೊಳ್ಳುವುದಿಲ್ಲ. ಆದರೆ ಅನುಭಾವಿಗಳು ಮೊದಲು ಸತ್ಯವನ್ನು ಅರಸುವರು. ಮೊದಲು ಅದನ್ನು ಅನುಭವಿಸಿ ಅನಂತರ ಅದನ್ನು ಒಂದು ಸಿದ್ಧಾಂತ ಮಾಡುವರು. ಚರ್ಚು ಮತ್ತೊಬ್ಬರ ಅನುಭವವನ್ನು ತೆಗೆದುಕೊಳ್ಳುವುದು. ಆದರೆ ಅನುಭಾವಿಗೆ ತನ್ನ ಅನುಭವವೇ ಇರುವುದು. ಚರ್ಚು ಸಿದ್ಧಾಂತದಿಂದ ಸತ್ಯಕ್ಕೆ ಹೋಗುವುದು. ಅನುಭಾವಿಗಳು ಸತ್ಯದಿಂದ ಸಿದ್ಧಾಂತಕ್ಕೆ ಬರುವರು.

ರಸಾಯನಶಾಸ್ತ್ರ ಮುಂತಾದ ಭೌತವಿಜ್ಞಾನಗಳು ಹೇಗೆ ಭೌತಿಕ ವಿಷಯಗಳನ್ನು\break ಹೇಳುವುವೋ ಹಾಗೆಯೇ ಧರ್ಮವು ತಾತ್ತ್ವಿಕ ವಿಷಯಗಳನ್ನು ಹೇಳುವುದು. ರಸಾಯನಶಾಸ್ತ್ರವನ್ನು ಕಲಿಯಬೇಕಾದರೆ ವ್ಯಕ್ತಿಯು ಪ್ರಕೃತಿಯೆಂಬ ಪುಸ್ತಕಕ್ಕೆ ಹೋಗಬೇಕು. ಆದರೆ ಧರ್ಮವನ್ನು ನಿಮ್ಮ ಮನಸ್ಸಿನಿಂದ ಮತ್ತು ಹೃದಯದಿಂದಲೇ ಕಲಿಯಬೇಕಾಗಿದೆ.\break ಸಂತರು ಭೌತವಿಜ್ಞಾನಗಳ ವಿಷಯದಲ್ಲಿ ತಜ್ಞರಲ್ಲ. ಏಕೆಂದರೆ ಅವರು ಅಧ್ಯಯನ\break ಮಾಡುವುದು ತಮ್ಮ ಅಂತರಂಗವನ್ನು ಮಾತ್ರ. ವಿಜ್ಞಾನಿಗೆ ಅನೇಕ ವೇಳೆ ಧಾರ್ಮಿಕ\break ವಿಷಯಗಳು ಗೊತ್ತಿರುವುದಿಲ್ಲ. ಏಕೆಂದರೆ ಅವನು ಬಾಹ್ಯ ಪ್ರಕೃತಿಯನ್ನು ಮಾತ್ರ ಅಧ್ಯಯನ ಮಾಡುವುದು.

ಪ್ರತಿಯೊಂದು ವಿಜ್ಞಾನವನ್ನು ಕಲಿಯಬೇಕಾದರೂ ಒಂದು ಕ್ರಮವಿದೆ. ಅದರಂತೆಯೇ ಧರ್ಮವೂ ಕೂಡ. ಇಲ್ಲಿ ಹಲವು ಮಾರ್ಗಗಳಿವೆ. ಏಕೆಂದರೆ ಇದಕ್ಕೆ ಸಂಬಂಧಪಟ್ಟ ಸಾಮಗ್ರಿ ಹೆಚ್ಚು. ಬಾಹ್ಯ ಪ್ರಕೃತಿಯಂತೆ ಮಾನವರ ಮನಸ್ಸು ಒಂದೇ ಸಮನಾಗಿಲ್ಲ. ಸ್ವಭಾವಕ್ಕೆ ತಕ್ಕಂತೆ ಬೇರೆ ಬೇರೆ ಮಾರ್ಗಗಳು ಇರಬೇಕಾಗುವುದು. ಕೆಲವರಿಗೆ ಯಾವುದಾದರೂ ಒಂದು ಇಂದ್ರಿಯ ಇತರ ಇಂದ್ರಿಯಗಳಿಗಿಂತ ಚೆನ್ನಾಗಿ ಕೆಲಸ ಮಾಡುವುದು. ಕೆಲವರು ಚೆನ್ನಾಗಿ ಕೇಳುವರು; ಮತ್ತೆ ಕೆಲವರು ಚೆನ್ನಾಗಿ ನೋಡುವರು, ಇತ್ಯಾದಿ. ಇದರಂತೆಯೇ ಒಬ್ಬೊಬ್ಬರಲ್ಲಿ ಮನಸ್ಸಿನ ಒಂದೊಂದು ಭಾಗ ತೀವ್ರವಾಗಿರುವುದು. ಇವುಗಳ ಮೂಲಕವೇ ಅವರು ತಮ್ಮ ಮನಸ್ಸನ್ನು ಅರಿಯಲು ಮುಂದುವರಿದು ಹೋಗಬೇಕಾಗಿದೆ. ಮನಸ್ಸುಗಳಲ್ಲಿ ವ್ಯತ್ಯಾಸವಿದ್ದರೂ ಇವುಗಳಲ್ಲೆಲ್ಲಾ ಒಂದು ಸಾಮಾನ್ಯ ಗುಣವೂ ಇದೆ. ಎಲ್ಲಾ ಮನಸ್ಸುಗಳಿಗೂ ಅನ್ವಯಿಸುವ ಒಂದು ಶಾಸ್ತ್ರವಿದೆ. ಧರ್ಮವಿಜ್ಞಾನವು ಮಾನವನ ಆತ್ಮ ವಿಶ್ಲೇಷಣೆಯನ್ನು ಅವಲಂಬಿಸಿದೆ. ಇದಕ್ಕೆ ಯಾವ ಪಂಥವೂ ಇಲ್ಲ.

ಎಲ್ಲರಿಗೂ ಒಂದೇ ಧರ್ಮ ಸರಿಹೋಗುವುದಿಲ್ಲ. ಪ್ರತಿಯೊಂದು ಧರ್ಮವೂ ಸರದಲ್ಲಿ\break ಪೋಣಿಸಿರುವ ಒಂದು ಮುತ್ತಿನಂತೆ. ಪ್ರತಿಯೊಂದರಲ್ಲಿಯೂ ನಾವು ಒಂದೊಂದು ವೈಶಿಷ್ಟ್ಯವನ್ನು ನೋಡಬೇಕಾಗಿದೆ. ಯಾರೂ ಒಂದು ಧರ್ಮದಲ್ಲಿ ಹುಟ್ಟಲಾರರು. ಅವರ ಹೃದಯದಲ್ಲೆ ಅವರಿಗೊಂದು ಧರ್ಮವಿರುವುದು. ಒಬ್ಬನ ವೈಶಿಷ್ಟ್ಯವನ್ನು ನಾಶಮಾಡಲೆತ್ನಿಸುವ\break ಧರ್ಮವು ಕಟ್ಟಕಡೆಗೆ ಹಾನಿಕಾರಿಯಾಗುತ್ತದೆ. ಪ್ರತಿಯೊಂದು ಜೀವನದ ಅಂತರಾಳ\break ದಲ್ಲಿಯೂ ಒಂದು ಪ್ರವಾಹ ಹರಿಯುತ್ತದೆ. ಈ ಪ್ರವಾಹವೇ ಕೊನೆಗೆ ಅವನನ್ನು\break ದೇವರೆಡೆಗೆ ಒಯ್ಯುವುದು. ಎಲ್ಲ ಧರ್ಮಗಳ ಗುರಿಯೂ ಭಗವಂತನ ಸಾಕ್ಷಾತ್ಕಾರ. ಭಗವಂತನನ್ನು ಮಾತ್ರ ಆರಾಧಿಸುವುದೇ ಪರಮಸಾಧನೆ. ಪ್ರತಿಯೊಬ್ಬರೂ ತಮ್ಮ ತಮ್ಮ\break ಇಷ್ಟವನ್ನು ಆರಿಸಿಕೊಂಡು ಅದರಲ್ಲೇ ನಿರತರಾದರೆ ಧಾರ್ಮಿಕ ಭಿನ್ನಾಭಿಪ್ರಾಯಗಳೆಲ್ಲ\break ಕೊನೆಗಾಣುವುವು.

