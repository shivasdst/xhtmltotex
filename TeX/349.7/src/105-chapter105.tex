
\chapter[ಇತಿಹಾಸದ ಸೇಡು ]{ಇತಿಹಾಸದ ಸೇಡು \protect\footnote{\engfoot{C.W. Vol. VII, p. 278}}}

\centerline{\textbf{(ಶ‍್ರೀಮತಿ ರೈಟ್​ ಅವರಿಂದ)}}

\vskip 5pt

೧೮೯೩ನೇ ಆಗಸ್ಟ್​ ಕೊನೆಯಲ್ಲಿ ಸ್ವಾಮಿ ವಿವೇಕಾನಂದರು ಆನಿಸ್ಕ್ವಾಂನಲ್ಲಿ ಪ್ರೊಫೆಸರ್​ ಜೆ.ಎಚ್​. ರೈಟ್​ ಅವರ ಮನೆಯಲ್ಲಿದ್ದರು. ಪ್ರಶಾಂತವಾದ ನ್ಯೂ ಇಂಗ್ಲೆಂಡ್​ ಪ್ರದೇಶದ ಈ ಹಳ್ಳಿಯಲ್ಲಿ ಜನರ ಮನಸ್ಸಿನಲ್ಲಿ ಈ ಭವ್ಯ ವ್ಯಕ್ತಿ ಯಾರಿರಬಹುದು, ಅವರು ಎಲ್ಲಿಂದ ಬಂದಿರಬಹುದು ಎಂಬ ಕುತೂಹಲವಿತ್ತು. ಮೊದಲು ಜನ, ಸ್ವಾಮೀಜಿ ಅವರು ಭರತಖಂಡದಿಂದ ಬಂದ ಬ್ರಾಹ್ಮಣರು ಇರಬೇಕು ಎಂದು ಆಲೋಚಿಸಿದರು. ಆದರೆ ಅವರ ನಡತೆ ಇದಕ್ಕೆ ಸರಿಯಾಗಿರಲಿಲ್ಲ. ಇದಕ್ಕೊಂದು ವಿವರಣೆ ಬೇಕಾಗಿತ್ತು. ವೃತ್ತಪತ್ರಿಕಾ ಪ್ರತಿನಿಧಿಗಳು ಸ್ವಾಮೀಜಿ ಅವರಿದ್ದ ಮನೆಗೆ ಈ ಹೊಸಬರು ಮಾತನಾಡುವುದನ್ನು ಕೇಳಲು ಹೋಗಿದ್ದರು.

\vskip 7pt

ಸ್ವಾಮೀಜಿ ಅವರು ತಮ್ಮ ಸುಮಧುರವಾದ ಧ್ವನಿಯಲ್ಲಿ “ಈಗೆಲ್ಲೊ ಕೆಲವು ದಿನಗಳ ಹಿಂದೆ ಸ್ವಲ್ಪ ದಿನಗಳ ಹಿಂದೆ-ನಾನ್ನೂರು ವರುಷಗಳು ಕೂಡ ಆಗಿಲ್ಲ” ಎಂದು ಪ್ರಾರಂಭಮಾಡಿದರು. ಸಹಿಷ್ಣುಗಳಾದ, ಯಾವಾಗಲೂ ಸಂಕಟವನ್ನು ಅನುಭವಿಸುತ್ತಿರುವ ಜನಾಂಗದ ಮೇಲೇ ದಬ್ಬಾಳಿಕೆ ನಡೆಸುವುದು ಮತ್ತು ಕ್ರೌರ್ಯದಿಂದ ವರ್ತಿಸಿರುವುದು ಮುಂತಾದ ಘಟನೆಗಳು ಪ್ರಾರಂಭವಾದವು. “ಕೊನೆಗೆ ದೇವರು ತನ್ನ ತೀರ್ಪನ್ನು ಕೊಡುವನು” ಎಂದರು. “ಓ ಇಂಗ್ಲಿಷಿನವರು - ಅವರೆಲ್ಲೊ ಕೆಲವು ವರ್ಷಗಳ ಹಿಂದೆ ಕಾಡುಮನುಷ್ಯರಾಗಿದ್ದರು. ಹುಳುಗಳು ಆ ಹೆಂಗಸರ ದೇಹದ ಮೇಲೆ ಹೊರಳಾಡುತ್ತಿದ್ದವು. ಅವರು ತಮ್ಮ ದೇಹದ ದುರ್ಗಂಧವನ್ನು ಮರೆಸಲು ಸುಗಂಧದ್ರವ್ಯಗಳನ್ನು ದೇಹಕ್ಕೆ ಲೇಪಿಸುತ್ತಿದ್ದರು... ಇದು ಜುಗುಪ್ಸಾಕರ! ಈಗಲೂ ಕೂಡ ಅವರು ಆ ಅವಸ್ಥೆಯಿಂದ ಈಗತಾನೆ ಪಾರಾಗುತ್ತಿರುವರು” ಎಂದರು.

\vskip 7pt

ಅವರು ಮಾತನ್ನು ಕೇಳುತ್ತಿದ್ದವರೊಬ್ಬರಿಗೆ ರೇಗಿ “ನೀವು ಹೇಳುವುದಕ್ಕೆ ಅರ್ಥವಿಲ್ಲ. ಆ ಸ್ಥಿತಿಯಾಗಿ ಸುಮಾರು ಐದುನೂರು ವರ್ಷಗಳಾದರೂ ಆಗಿರಬೇಕು” ಎಂದರು.

\vskip 7pt

“ನಾನು ಈಗತಾನೆ ಎಂದು ಹೇಳಲಿಲ್ಲವೆ, ಸ್ವಲ್ಪ ವರುಷಗಳ ಹಿಂದೆ ಎಂದು. ಮಾನವನ ಆತ್ಮನ ಕಾಲದೊಂದಿಗೆ ಹೋಲಿಸಿದರೆ ಕೆಲವು ನೂರು ವರ್ಷಗಳೇನು?” ಎಂದರು. ಅನಂತರ ಅವರ ಧ್ವನಿ ಬದಲಾಯಿಸಿತು, ಮಧುರವಾಯಿತು, ವಿಚಾರಪರವಾಯಿತು. “ಅವರೆಲ್ಲ ನಿಜವಾಗಿ ಕಾಡುಜನ- ಸಹಿಸಲಸಾಧ್ಯವಾದ ಛಳಿ, ಮತ್ತು ಆ ವಾತಾವರಣದಲ್ಲಿ ಇರುವುದಕ್ಕೆ ಅವಕಾಶವೇ ಇಲ್ಲ” ಎಂದು ಹೇಳಿ, ಅನಂತರ ಸ್ವಲ್ಪ ಬಿರುಸಿನಿಂದ ವೇಗವಾಗಿ ಹೀಗೆಂದರು: “ಆ ವಾತಾವರಣ ಅವರನ್ನು ಅನಾಗರಿಕರನ್ನಾಗಿ ಮಾಡುವುದು. ಅವರು ಸುಮ್ಮನೆ ಕೊಲ್ಲುವುದನ್ನೇ ಆಲೋಚಿಸುತ್ತಾರೆ. ಅವರ ಧರ್ಮ ಎಲ್ಲಿ ಹೋಯಿತು? ಆ ಪವಿತ್ರಾತ್ಮನ ಹೆಸರನ್ನು (ಕ್ರಿಸ್ತನನ್ನು) ತೆಗೆದುಕೊಳ್ಳುವರು. ತಾವು ಮಾನವಕೋಟಿಯನ್ನು ಪ್ರೀತಿಸುತ್ತೇವೆ. ಎನ್ನುವರು. ಕ್ರೈಸ್ತಧರ್ಮದಿಂದ ಇತರರನ್ನು ನಾಗರಿಕರನ್ನಾಗಿ ಮಾಡುತ್ತೇವೆ ಎನ್ನುವರು. ಅವರನ್ನು ನಾಗರೀಕರನ್ನಾಗಿ ಮಾಡಿರುವುದು ಅವರ ಹಸಿವು, ಅವರ ದೇವರಲ್ಲ. ಮಾನವಪ್ರೀತಿ ಎಂಬುದು ಬರೀ ಅವರ ಮಾತಿನಲ್ಲಿ. ಆಚರಣೆಯಲ್ಲಿ ಹಿಂಸೆ ಮತ್ತು ಪಾಪವಲ್ಲದೆ ಬೇರಿಲ್ಲ. ‘ಸಹೋದರನೇ ನಾನು ನಿನ್ನನ್ನೇ ಪ್ರೀತಿಸುತ್ತೇನೆ, ಪ್ರೀತಿಸುತ್ತೇನೆ’ ಎಂದು ಹೇಳುತ್ತಿರುವಾಗಲೇ, ಅವನ ಕೊರಳನ್ನು ಕತ್ತರಿಸುವರು. ಅವರ ಕೈಗಳು ರಕ್ತದಿಂದ ನೆನೆದುಹೋಗಿವೆ.” ಅನಂತರ ಅವರ ಮಾತು ನಿಧಾನವಾಗುತ್ತಾ ಬಂತು. ಅತಿಮಧುರವಾದ ಧ್ವನಿ ಒಂದು ಘಂಟೆಯ ಧ್ವನಿಯಂತೆ ಸ್ಪಷ್ಟವಾಯಿತು. “ಆದರೆ ದೇವರ ತೀರ್ಪು ಅವರ ಮೇಲೆ ಬರುವುದು. ‘ಸೇಡು ನನ್ನದು. ಅದನ್ನು ನಾನು ತೀರಿಸಿಕೊಳ್ಳುತ್ತೇನೆ’ ಎನ್ನುವನು ದೇವರು. ಅವರ ನಾಶ ಸಮೀಪಿಸುತ್ತಿದೆ. ನಿಮ್ಮ ಕ್ರೈಸ್ತ ಜನಾಂಗದಲ್ಲಿ ಏನಿದೆ? ಈ ಪ್ರಪಂಚದಲ್ಲಿ ಅವರು ಮೂರನೆ ಒಂದು ಭಾಗ ಕೂಡ ಇಲ್ಲ. ಕೋಟ್ಯಂತರ ಜನರಿರುವ ಚೀನಿಯರ ಕಡೆ ನೋಡಿ. ಅವರೇ ಭಗವಂತನ ಸೇಡನ್ನು ತೀರಿಸಿಕೊಳ್ಳುವುದಕ್ಕೆ ನಿಮ್ಮ ಮೇಲೆ ಬೀಳುವರು. ಮತ್ತೊಂದು ಹೂಣರ ದಂಡಯಾತ್ರೆ ಆಗುವುದು.” ಅವರು ಯೂರೋಪಿನ ಮೇಲೆ ಧಾಳಿ ನಡೆಸುವರು. ಅವರು ಮೇಲೆದ್ದಿರುವ ಒಂದು ಕಲ್ಲನ್ನೂ ಬಿಡುವುದಿಲ್ಲ. ಎಲ್ಲವನ್ನೂ ಧ್ವಂಸ ಮಾಡುವರು. ಗಂಡಸರು, ಹೆಂಗಸರು, ಮಕ್ಕಳು ಎಲ್ಲಾ ನಾಶವಾಗುವರು. ಪುನಃ ಅನಾಗರಿಕ ಯುಗ ಪ್ರಾರಂಭವಾಗುವುದು.” ಅವರ ಧ್ವನಿಯನ್ನು ವಿವರಿಸಲು ಅಸದಳವಾಯಿತು. ದುಃಖದಿಂದ ಮತ್ತು ಕರುಣೆಯಿಂದ ಹೀಗೆ ಹೇಳಿದರು: “ನಾನೆ-ನಾನು ಯಾವುದನ್ನು ಲೆಕ್ಕಿಸುವುದಿಲ್ಲ. ಪ್ರಪಂಚ ಅದರಿಂದ ಚೇತರಿಸಿಕೊಂಡಾಗ ಮತ್ತೂ ಚೆನ್ನಾಗುವುದು. ಆದರೆ ಅದು ಬರುತ್ತಿದೆ. ದೇವರ ಸೇಡು ಬೇಗ ಬರುತ್ತಿದೆ!”

\vskip 7pt

“ಏನು, ಬೇಗ ಬರುತ್ತಿದೆಯೆ?” ಎಂದು ಎಲ್ಲರೂ ಕೇಳಿದರು.

\vskip 7pt

“ಅದಾಗುವುದಕ್ಕೆ ಒಂದು ಸಾವಿರ ವರ್ಷಗಳು ಹಿಡಿಯಬಹುದು” ಎಂದರು. ಕೇಳುತ್ತಿದ್ದವರು ಸದ್ಯಕ್ಕೆ ಈಗಲೇ ಆಗುವಂತಿಲ್ಲವಲ್ಲ ಎಂದು ಸಮಾಧಾನಪಟ್ಟರು.

\vskip 7pt

ಸ್ವಾಮೀಜಿ ಅವರು ಇನ್ನೂ ಮುಂದುವರಿಸಿದರು: “ದೇವರು ಸೇಡನ್ನು ತೀರಿಸಿಕೊಳ್ಳುವನು. ನೀವು ಅದನ್ನು ಧರ್ಮದಲ್ಲಿ ಕಾಣದೆ ಇರಬಹುದು, ನೀವು ಅದನ್ನು ರಾಜಕೀಯದಲ್ಲಿ ಕಾಣದೇ ಇರಬಹುದು. ನೀವು ಅದನ್ನು ಇತಿಹಾಸದಲ್ಲಿ ನೋಡುವಿರಿ. ಎಂದಿನಂತೆ ಇದೂ ಆಗುವುದು. ನೀವು ಜನರನ್ನು ಹಿಂಸಿಸಿದರೆ ನೀವು ಕೂಡ ಅದೇ ಹಿಂಸೆಗೆ ಒಳಗಾಗಬೇಕಾಗುವುದು. ಭರತಖಂಡದಲ್ಲಿರುವ ನಾವು ದೇವರ ಕೋಪಕ್ಕೆ ತುತ್ತಾಗಿರುವೆವು. ಈ ವಿಷಯಗಳನ್ನು ನೋಡಿ, ತಮ್ಮ ಐಶ್ವರ್ಯವನ್ನು ವೃದ್ಧಿಮಾಡಿಕೊಳ್ಳುವುದಕ್ಕಾಗಿ ಬಡಜನರ ಕೈಯಿಂದ ಚೆನ್ನಾಗಿ ದುಡಿಸಿಕೊಂಡರು. ಅವರ ಗೋಳಿಗೆ ಕಿವಿಕೊಡಲಿಲ್ಲ. ಬಡವರು ಹೊಟ್ಟೆಗೆ ಹಿಟ್ಟು\break ಇಲ್ಲದೆ ಅಳುತ್ತಿದ್ದಾಗ ಇವರು ಬೆಳ್ಳಿ ಚಿನ್ನದ ತಟ್ಟೆಯಲ್ಲಿ ಊಟ ಮಾಡುತ್ತಿದ್ದರು. ಮಹಮ್ಮದೀ\-ಯರು ಅವರ ಮೇಲೆ ಧಾಳಿಯಿಟ್ಟು ಅವರನ್ನು ಕೊಂದರು. ಅವರನ್ನು ಕೊಲ್ಲುತ್ತ ಎಲ್ಲರನ್ನು ಗೆದ್ದರು. ಹಲವು ವರ್ಷಗಳಿಂದ ಭರತಖಂಡವನ್ನು ಇತರರು ಗೆದ್ದರು. ಕಟ್ಟಕಡೆಗೆ ಬಂದ ಅನಿಷ್ಟವೇ ಆಂಗ್ಲೇಯರ ವಶವಾದದ್ದು. ಭರತಖಂಡದಲ್ಲಿ ನೋಡಿ. ಹಿಂದೂಗಳು ಏನನ್ನು ಬಿಟ್ಟಿರುವರು? ಅತಿ ಅದ್ಭುತವಾದ ದೇವಾಲಯಗಳನ್ನು ಎಲ್ಲಾ ಕಡೆಗಳಲ್ಲಿಯೂ ಬಿಟ್ಟಿರುವರು. ಮಹಮ್ಮದೀಯರು ಏನನ್ನು ಬಿಟ್ಟಿರುವರು? ಸುಂದರವಾದ ಅರಮನೆಗಳನ್ನು. ಇಂಗ್ಲಿಷಿನವರು ಏನನ್ನು ಬಿಟ್ಟರು? ಒಡೆದ ಬ್ರಾಂಡಿ ಬಾಟಲುಗಳ ರಾಶಿಯನ್ನಲ್ಲದೆ\break ಮತ್ತೇನನ್ನೂ ಇಲ್ಲ. ನಮ್ಮ ಜನರ ಮೇಲೆ ದೇವರಿಗೆ ಕರುಣೆಯಿರಲಿಲ್ಲ. ಏಕೆಂದರೆ ನಮ್ಮ\break ಜನರಲ್ಲೇ ಕರುಣೆ ಇರಲಿಲ್ಲ. ತಮ್ಮ ಕ್ರೂರತನದಿಂದ ಅವರು ಜನಸಾಧಾರಣರನ್ನು\break ಅಧೋಗತಿಗೆ ತಂದರು. ತಮಗೆ ಸಹಾಯ ಬೇಕಾದಾಗ ಜನಸಾಧಾರಣರಿಗೆ ಸಹಾಯ ಮಾಡುವುದಕ್ಕೆ ಸಾಧ್ಯವಿರಲಿಲ್ಲ. ಮನುಷ್ಯನಿಗೆ ದೇವರ ಪ್ರತೀಕಾರದಲ್ಲಿ ನಂಬಿಕೆ ಇಲ್ಲದೆ ಇದ್ದರೆ ಚರಿತ್ರೆಯ ಪ್ರತೀಕಾರದಲ್ಲಾದರೂ ಅವನು ನಿಜವಾಗಿ ನಂಬದೆ ಇರಲಾರ. ಇದು ಇಂಗ್ಲಿಷಿನವರ ಮೇಲೆ ಬರುವುದು. ಅವರು ನಮ್ಮ ಕತ್ತಿನ ಮೇಲೆ ತಮ್ಮ ಕಾಲನ್ನು ಇಟ್ಟಿರುವರು. ತಮ್ಮ ಸುಖಕ್ಕಾಗಿ ನಮ್ಮ ರಕ್ತದ ಕೊನೆಯ ಬಿಂದುವನ್ನೂ ಹೀರುವರು. ನಮ್ಮ ಹಳ್ಳಿಗಳು, ಪ್ರಾಂತ್ಯಗಳು ಉಪವಾಸಕ್ಕೆ ತುತ್ತಾದಾಗ ನಮ್ಮ ಕೋಟ್ಯಂತರ ಐಶ್ವರ್ಯವನ್ನು\break ತಮ್ಮ ದೇಶಕ್ಕೆ ಸಾಗಿಸಿರುವರು. ಈಗ ಚೈನಾ ದೇಶೀಯರು ಅವರ ಮೇಲೆ ಪ್ರತಿಕಾರವನ್ನು ತೀರಿಸಿಕೊಳ್ಳುವುದಕ್ಕೆ ಬೀಳುವರು. ಇಂದು ಚೀನೀಯರೆಲ್ಲ ಜಾಗ್ರತರಾಗಿ ಇಂಗ್ಲಿಷರನ್ನು ಸಮುದ್ರದ ಪಾಲು ಮಾಡಿದರೆ, ಅದು ಅವರಿಗೆ ಯೋಗ್ಯವಾದ ಶಿಕ್ಷೆಯೇ ಆಗುವುದು. ಇದು ಅವರ ಪಾಲಿಗೆ ಬರಬೇಕಾಗುವ ಶಿಕ್ಷೆಗಿಂತ ಹೆಚ್ಚಾದುದಲ್ಲ.”

\vskip 7pt

ಇದನ್ನು ಹೇಳಿದ ಮೇಲೆ ಸ್ವಾಮೀಜಿ ಮೌನವಾಗಿದ್ದರು. ಅದನ್ನು ಬೇಕು ಬೇಡದಂತೆ ಕೇಳುತ್ತಿದ್ದರು. ಮಧ್ಯೆ ಮಧ್ಯೆ ಅವರು ಛಾವಣಿಯ ಮೇಲೆ ತಮ್ಮ ದೃಷ್ಟಿಯನ್ನು ಪಸರಿಸುತ್ತ, ‘ಶಿವ, ಶಿವ’ ಎಂದು ಹೇಳುತ್ತಿದ್ದರು. ಇವರ ಸುತ್ತಲೂ ಇದ್ದ ಈ ಸಣ್ಣ ತಂಡದವರು ಮಾತ್ರ ಈ ವಿಚಿತ್ರ ಮನುಷ್ಯನ ಅಂತರಾಳದಿಂದ ಜ್ವಾಲಾಮುಖಿಯಿಂದ ಏಳುವ ಶಿಲಾಪ್ರವಾಹದಂತೆ ಇರುವ ಪ್ರಚಂಡವಾದ ಭಾವನೆ ಮತ್ತು ಮತ್ತೊಬ್ಬರನ್ನು ಸದೆಬಡಿಯುವಂತಹ ಉದ್ವಿಗ್ನತೆಗೆ ಅಸ್ತವ್ಯಸ್ತರಾದರು.

\vskip 7pt

ಅವರು ಕೆಲವು ದಿನಗಳು ಇದ್ದರು (ಅವರು ನಿಜವಾಗಿ ಒಂದು ವಾರದ ಅಂತ್ಯ ದಿನವನ್ನು ಮಾತ್ರ ಅಲ್ಲಿ ಕಳೆದರು). ಅವರ ಮಾತಿನಲ್ಲೆಲ್ಲ ಸುಂದರವಾದ ದೃಶ್ಯಗಳು ಮತ್ತು ಚಮತ್ಕಾರವಾದ ಕಥೆಗಳು ಉದಾಹೃತವಾಗಿದ್ದುವು.

\vskip 7pt

ಅವರು ಹೇಳಿದ ಒಂದು ಸುಂದರವಾದ ಕಥೆಯೇ ಇದು: ಒಬ್ಬನ ಹೆಂಡತಿ ಗಂಡನನ್ನು ಅವನಿಗೆ ಪ್ರಾಪ್ತವಾದ ಕಷ್ಟಗಳಿಗಾಗಿ ಜರಿಯತೊಡಗಿದಳು. ಹೇಗೆ ಇತರರು ಎಲ್ಲವನ್ನೂ\break ಗೆಲ್ಲುತ್ತಿರುವರು, ಮತ್ತು ಆತನು ಹೇಗೆ ಎಲ್ಲವನ್ನೂ ಕಳೆದುಕೊಳ್ಳುತ್ತಿರುವನು ಎಂಬುದನ್ನು ವಿವರಿಸಿದಳು. “ಇಷ್ಟೊಂದು ವರ್ಷಗಳು ದೇವರ ಸೇವೆಯನ್ನು ಮಾಡಿದ ಪ್ರತಿಫಲವಾಗಿ ಇದನ್ನೇ ಏನು ಅವನು ಮಾಡಿರುವುದು?” ಎಂದು ಅವಳು ಗಂಡನನ್ನು ಪ್ರಶ್ನಿಸಿದಳು. ಅದಕ್ಕೆ ಗಂಡ ಹೀಗೆ ಉತ್ತರಿಸಿದನು: “ನಾನೇನು ವ್ಯಾಪಾರಿ ದೃಷ್ಟಿಯಿಂದ ದೇವರನ್ನು ಪ್ರೀತಿಸುವೆನೇನು? ಆ ಬೆಟ್ಟಗಳನ್ನು ನೋಡು. ಅವು ನನಗೆ ಏನು ಮಾಡಿವೆ, ಅಥವಾ ನಾನು ಅವಕ್ಕೆ ಏನು ಮಾಡಬಲ್ಲೆ? ಆದರೂ ನಾನು ಅವನ್ನು ಪ್ರೀತಿಸುತ್ತೇನೆ. ಏಕೆಂದರೆ ಯಾವುದು ಸುಂದರವಾಗಿದೆಯೋ ಅದನ್ನು ಪ್ರೀತಿಸುವುದು ನನ್ನ ಸ್ವಭಾವವಾಗಿದೆ. ಅದರಂತೆಯೇ ನಾನು ದೇವರನ್ನು ಪ್ರೀತಿಸುತ್ತೇನೆ.” ರಾಜನೊಬ್ಬ, ಋಷಿಗೆ ಒಂದು ಬಹುಮಾನವನ್ನು ಕೊಡಬೇಕೆಂದು ಇಚ್ಛಿಸಿದ. ಋಷಿ ಇದನ್ನು ತಿರಸ್ಕರಿಸಿದನು. ಆದರೆ ರಾಜ ಬಲವಂತದಿಂದ\break ಋಷಿಯನ್ನು ಬೇಡಿಕೊಂಡು ಒಪ್ಪಿಕೊಳ್ಳುವಂತೆ ಮಾಡಿ, ಅವನನ್ನು ತನ್ನ ಜೊತೆಗೆ ಬರುವಂತೆ ಕೇಳಿಕೊಂಡನು. ಅವರಿಬ್ಬರೂ ಅರಮನೆಗೆ ಬಂದಾದ ಮೇಲೆ ರಾಜ ಪ್ರಾರ್ಥಿಸುತ್ತಿರುವುದನ್ನು ಋಷಿ ಕೇಳಿದನು. ರಾಜ ಐಶ್ವರ್ಯ, ಅಧಿಕಾರ, ದೀರ್ಘಾಯಸ್ಸು ಇವುಗಳನ್ನು ಬೇಡುತ್ತಿದ್ದ. ಋಷಿ ಆಶ್ಚರ್ಯದಿಂದ ಇದನ್ನು ಕೇಳುತ್ತ ಇದ್ದಂತೆ ತನ್ನ ಆಸನವನ್ನು ಕೈಯಲ್ಲಿ ಹಿಡಿದುಕೊಂಡು ಹೊರಟನು. ಧ್ಯಾನದಲ್ಲಿದ್ದ ರಾಜ ಕಣ್ಣುಗಳನ್ನು ತೆರೆದು ಹೋಗುತ್ತಿರುವ ಋಷಿಗಳನ್ನು ನೋಡಿ, “ಏನು ಹೋಗುತ್ತಿರುವಿರಲ್ಲ? ನನ್ನಿಂದ ನೀವು ಏನನ್ನೂ ತೆಗೆದುಕೊಳ್ಳಲಿಲ್ಲವಲ್ಲ?” ಎಂದು ಪ್ರಶ್ನಿಸಿದನು. ಅದಕ್ಕೆ ಋಷಿ “ನಾನು ಒಬ್ಬ ಭಿಕ್ಷುಕನಿಂದ ಬೇಡಲೇ” ಎಂದ.

\vskip 7pt

ಸಭಿಕರಲ್ಲಿ ಯಾರೋ ಒಬ್ಬರು ಕ್ರೈಸ್ತಧರ್ಮವು ಮಾನವಕೋಟಿಯನ್ನು ಉದ್ಧರಿಸಿದೆ ಎಂದರು. ಸ್ವಾಮೀಜಿ ಅವರು ತಮ್ಮ ವಿಶಾಲವಾದ ಕಪ್ಪುಕಣ್ಣುಗಳನ್ನು ಅರಳಿಸಿ ಅವನ ಕಡೆ ನೋಡಿ “ಕ್ರೈಸ್ತಧರ್ಮ ನಿಜವಾಗಿ ಇನ್ನೊಬ್ಬರನ್ನು ಉದ್ದರಿಸುವ ಶಕ್ತಿಯನ್ನು ಹೊಂದಿದ್ದರೆ ಇಥಿಯೋಪಿಯಾ ಮತ್ತು ಅಬಿಸೀನಿಯಾದವರನ್ನು ಏತಕ್ಕೆ ಉದ್ಧರಿಸಿಲ್ಲ?” ಎಂದು ಪ್ರಶ್ನಿಸಿದರು.

\vskip 7pt

“ಇದನ್ನು ಸಂನ್ಯಾಸಿಗೆ ಮಾಡುವ ಧೈರ್ಯ ಅವರಲ್ಲಿ ಇಲ್ಲ” ಎಂಬ ಮಾತು ಅವರ ಬಾಯಲ್ಲಿ ಹಲವು ವೇಳೆ ಬರುತ್ತಿತ್ತು. ಇಂಗ್ಲೀಷ್​ ಸರ್ಕಾರ ತನ್ನನ್ನು ಸೆರೆಹಿಡಿದು ಗುಂಡಿನಿಂದ ಕೊಂದರೆ ಒಳ್ಳೆಯದಾದೀತು ಎಂಬ ಮಾತುಗಳು ಕೂಡ ಕೆಲವು ವೇಳೆ ಅವರ ಬಾಯಿಯಿಂದ ಹೊರಟವು. ಅದು ಇಂಗ್ಲೀಷರ ಶವಪೆಟ್ಟಿಗೆಗೆ ಹೊಡೆಯುವ ಮೊದಲನೆಯ ಮೊಳೆಯಾಗುವುದೆಂದರು. ತಮ್ಮ ಶುಭ್ರವಾದ ಹಲ್ಲುಗಳನ್ನು ಸ್ವಲ್ಪ ವ್ಯಕ್ತಪಡಿಸಿ ಮಂದಹಾಸದಿಂದ “ನನ್ನ ಸಾವು ದೇಶದಾದ್ಯಂತವೂ ಕಾಡ್ಗಿಚ್ಚಿನಂತೆ ಹರಡುವುದು” ಎಂದರು.

\vskip 7pt

ಅವರ ಮೆಚ್ಚಿನ ಮಹಾನಾಯಕಿಯೇ ಸಿಪಾಯಿದಂಗೆಯಲ್ಲಿ ಭಾಗವಹಿಸಿದ ಆ ಅದ್ಭುತಳಾದ (?) ರಾಣಿ. ಅವಳು ಯುದ್ಧದಲ್ಲಿ ತಾನೇ ಸ್ವತಃ ಸೇನೆಯನ್ನು ನಡೆಸುತ್ತಿದ್ದಳು. ಹಲವು ಹಳೆಯ ದಂಗೆಕಾರರು ತಮ್ಮನ್ನು ಮರೆಮಾಚಿಕೊಳ್ಳುವುದಕ್ಕಾಗಿ ಸಂನ್ಯಾಸಿಗಳಂತೆ ವೇಷ ಹಾಕಿಕೊಂಡಿದ್ದರು. ಆದಕಾರಣವೇ ಸಂನ್ಯಾಸಿಗಳನ್ನು ಬಹಳ ಅಪಾಯಕರವಾದ ವ್ಯಕ್ತಿಗಳೆಂದೇ ನಂಬುತ್ತಿದ್ದರು. ಅವರಲ್ಲಿ ಒಬ್ಬ ತನ್ನ ನಾಲ್ಕು ಜನ ಮಕ್ಕಳನ್ನು ಕಳೆದುಕೊಂಡಿ\-ದ್ದನು. ಅವರ ವಿಷಯವಾಗಿ ಮಾತನಾಡುವಾಗ ಶಾಂತ ರೀತಿಯಿಂದ ಧೈರ್ಯವಾಗಿ ಮಾತನಾಡುತ್ತಿದ್ದ. ಆದರೆ ರಾಣಿಯ ಹೆಸರನ್ನು ಎತ್ತಿದಾಗ ಆತನ ಕಣ್ಣುಗಳಲ್ಲಿ ನೀರು ಸುರಿಯುತ್ತಿತ್ತು. “ಆ ಹೆಂಗಸು ದೇವಿ, ಸೋತಾಗ ಅವಳು ಪುರುಷನಂತೆ ತನ್ನ ಕತ್ತಿಯ ಮೇಲೆ ಬಿದ್ದು ಸತ್ತಳು!” ಸಿಪಾಯಿದಂಗೆಯ ಮತ್ತೊಂದು ದೃಷ್ಟಿಕೋನ ಸೋಜಿಗವಾಗಿದೆ. ಅದಕ್ಕೆ ಬೇರೊಂದು ದೃಷ್ಟಿಯಿರುವುದೆಂದು ನೀವು ನಂಬುವುದಿಲ್ಲ. ಹಿಂದೂಗಳು ಎಂದಿಗೂ ಸ್ತ್ರೀಯನ್ನು ಕೊಲ್ಲುವುದಿಲ್ಲ. ಖಂಡಿತವಾಗಿ ಇಲ್ಲ.

\vspace{-0.5cm}

