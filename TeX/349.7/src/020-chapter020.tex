
\chapter[ಮಹಾಭಾರತ ]{ಮಹಾಭಾರತ \protect\footnote{\engfoot{C.W. VOl. IV, P. 78}}}

\centerline{\textbf{(೧೯೦೦ರ ಫೆಬ್ರವರಿ ೧ರಂದು ಕ್ಯಾಲಿಫೋರ್ನಿಯಾದ ಪ್ರದೇಶದಲ್ಲಿ ನೀಡಿದ ಉಪನ್ಯಾಸ)}}

\vskip 0.3cm

ನಾನು ಇಂದಿನ ಸಂಜೆ ಮಾತಾಡಬೇಕೆಂದಿರುವ ಮತ್ತೊಂದು ಮಹಾಕಾವ್ಯವೆಂದರೆ\break ಮಹಾಭಾರತ. ದುಷ್ಯಂತ ಶಕುಂತಲೆಯರ ಮಗನಾದ ಭರತನ ವಂಶಜರ ಕಥೆ ಇದರಲ್ಲಿದೆ. ಮಹಾ ಎಂದರೆ ದೊಡ್ಡದು, ಭಾರತ ಎಂದರೆ ಭರತ ವಂಶಜರು. ಅವನಿಂದಲೇ ಭಾರತ ಎಂಬ ಹೆಸರು ಇಂಡಿಯಾ ದೇಶಕ್ಕೆ ಬಂದಿದೆ. ಮಹಾಭಾರತ ಎಂದರೆ ಬೃಹತ್​ ಭರತಖಂಡ, ಅಥವಾ ಪ್ರಖ್ಯಾತ ಭರತ ವಂಶಜರ ಕಥೆ. ಪ್ರಾಚೀನ ಕುರುದೇಶವೇ ಮಹಾಕಾವ್ಯದ ರಂಗಭೂಮಿ. ಕುರು ಮತ್ತು ಪಾಂಚಾಲರಿಗೆ ಆದ ಮಹಾಯುದ್ಧದ ಮೇಲೆ ಈ ಕಥೆ ಬೆಳೆದಿದೆ. ಕಥಾ ಪ್ರಸಂಗಕ್ಕೆ ಸಂಬಂಧಿಸಿದ ಭೂಭಾಗ ಅಷ್ಟೇನೂ ವಿಸ್ತಾರವಾದುದಲ್ಲ. ಭರತಖಂಡದಲ್ಲೆಲ್ಲಾ ಹೆಚ್ಚು ಜನಪ್ರಿಯವಾಗಿರುವುದು ಈ ಕಾವ್ಯ. ಹೋಮರನ ಕಾವ್ಯ ಗ್ರೀಕರ ಮೇಲೆ ಯಾವ ಪ್ರಭಾವವನ್ನು ಬೀರಿತೊ ಅದೇ ಪ್ರಭಾವವಿದೆ ಇದಕ್ಕೆ. ಕಾಲ ಕಳೆದಂತೆ ಇದಕ್ಕೆ ಹೆಚ್ಚು ಹೆಚ್ಚು ಕಥಾ ಸಾಮಗ್ರಿ ಸೇರಿ ಸುಮಾರು ಒಂದು ಲಕ್ಷ ಶ್ಲೋಕಗಳಿರುವ ಒಂದು ಬೃಹತ್​ ಗ್ರಂಥವಾಗಿದೆ. ಎಲ್ಲಾ ಬಗೆಯ ಕಥೆ, ದಂತಕಥೆ, ಪುರಾಣ, ತತ್ತ್ವ, ಇತಿಹಾಸ ಮುಂತಾದ ವಿಷಯಗಳೆಲ್ಲ ಕಾಲಕಾಲಕ್ಕೆ ಸೇರಿ ಅದೊಂದು ವಿಶಾಲವಾದ ಮಹಾಕಾವ್ಯವಾಗಿದೆ. ಅದರಲ್ಲೆಲ್ಲ ಹಳೆಯ ಆದಿಕಥೆಯೊಂದು ಹರಿಯುತ್ತಿದೆ. ಮಹಾಭಾರತದ ಮುಖ್ಯ ಕಥೆಯೇ ಇಂಡಿಯಾ ದೇಶದ ಚಕ್ರಾಧಿಪತ್ಯಕ್ಕಾಗಿ ಕೌರವ ಪಾಂಡವರೆಂಬ ದಾಯಾದಿಗಳಿಗೆ ನಡೆದ ಯುದ್ಧ.

\vskip 0.1cm

ಆರ್ಯರು ಸಣ್ಣ ಸಣ್ಣ ಪಂಗಡಗಳಾಗಿ ಭರತಖಂಡಕ್ಕೆ ಬಂದರು; ಕ್ರಮೇಣ ಅವರು ವಿಸ್ತಿರಿಸುತ್ತಾ ಹೋದರು. ಕೊನೆಗೆ ಅವರು ಭರತಖಂಡದ ನಿರ್ವಿವಾದ ಪ್ರಭುಗಳಾದರು. ಅನಂತರವೇ ದಾಯಾದಿಗಳಲ್ಲಿ ತಾವು ಚಕ್ರವರ್ತಿಗಳಾಗಬೇಕೆಂಬ ಕಲಹ ಪ್ರಾರಂಭವಾಯಿತು. ಯಾರು ಭಗವದ್ಗೀತೆಯನ್ನು ಓದಿರುವರೊ ಅವರಿಗೆ ಕಾದಲನುವಾಗಿ ಎರಡು ಸೇನೆಗಳು ನೆರೆದ ಕುರುಕ್ಷೇತ್ರ ಸಮರಾಂಗಣದ ವಿವರಣೆಯೊಂದಿಗೆ ಗೀತೆ ಪ್ರಾರಂಭವಾಗುವುದು ಎಂಬುದು ಗೊತ್ತಿದೆ. ಅದೇ ಮಹಾಭಾರತ ಯುದ್ಧ.

\vskip 0.1cm

ಚಕ್ರವರ್ತಿಯ ಮಕ್ಕಳಾದ ಇಬ್ಬರು ಸಹೋದರರಿದ್ದರು. ಹಿರಿಯನೆ ಧೃತರಾಷ್ಟ್ರ ಕಿರಿಯನೆ ಪಾಂಡು. ಹಿರಿಯನಾದ ಧೃತರಾಷ್ಟ್ರ ಹುಟ್ಟು ಕುರುಡ. ಇಂಡಿಯಾ ದೇಶದ ಕಾನೂನಿಗೆ ಅನುಸಾರವಾಗಿ ಯಾವ ಕುರುಡನೇ ಆಗಲಿ, ತೊದಲು ಮಾತನಾಡುವವನು ಅಥವಾ ಯಾವ ಬಗೆಯ ಹೀನಾಂಗನೇ ಆಗಲಿ ರಾಜನಾಗಲಾರ. ಅವನಿಗೆ ಜೀವನಾಂಶ ಮಾತ್ರ ಸಿಕ್ಕುವುದು. ಆದಕಾರಣ ಧೃತರಾಷ್ಟ್ರ ಹಿರಿಯನಾದರೂ ರಾಜನಾಗಲಿಲ್ಲ. ಪಾಂಡು ರಾಜನಾದ.

\vskip 0.1cm

ಧೃತರಾಷ್ಟ್ರನಿಗೆ ನೂರು ಜನ ಮಕ್ಕಳಿದ್ದರು; ಪಾಂಡುವಿಗೆ ಐದು ಜನ ಮಕ್ಕಳಿದ್ದರು. ಪಾಂಡು ಬಾಲ್ಯದಲ್ಲೆ ತೀರಿಹೋಗಲು, ಧೃತರಾಷ್ಟ್ರ ರಾಜನಾಗಿ ಪಾಂಡು ರಾಜನ ಮಕ್ಕಳನ್ನೂ ತನ್ನ ಮಕ್ಕಳೊಡನೆ ನೋಡಿಕೊಳ್ಳುತ್ತಿದ್ದನು. ಅವರು ಪ್ರಾಪ್ತವಯಸ್ಕರಾದ ಮೇಲೆ ಪ್ರಖ್ಯಾತ ಬ್ರಾಹ್ಮಣಯೋಧ ದ್ರೋಣಾಚಾರ್ಯರ ಮಾರ್ಗದರ್ಶನದಲ್ಲಿ ಅವರನ್ನು ಬಿಟ್ಟನು. ರಾಜರಿಗೆ ಯೋಗ್ಯವಾದ ಹಲವು ಶಸ್ತ್ರಾಸ್ತ್ರಗಳ ವಿದ್ಯೆಯಲ್ಲಿ ದ್ರೋಣ ಅವರನ್ನು ಪರಿಣತರನ್ನಾಗಿ ಮಾಡಿದನು. ವಿದ್ಯೆ ಪೂರೈಸಿದ ಮೇಲೆ ಪಾಂಡುರಾಜನ ಹಿರಿಯಮಗನಾದ ಯುಧಿಷ್ಠಿರನನ್ನು ತನ್ನ ತಂದೆಯ ಸಿಂಹಾಸನದ ಮೇಲೆ ಕುಳ್ಳಿರಿಸಿದನು. ಯುಧಿಷ್ಠಿರನ ಶೀಲಸಂಪತ್ತು ಮತ್ತು ಅವನ ತಮ್ಮಂದಿರ ಪೌರುಷ ಮತ್ತು ಶ್ರದ್ಧಾಭಕ್ತಿಗಳು ಧೃತರಾಷ್ಟ್ರನ ಮಕ್ಕಳಲ್ಲಿ ಅಸೂಯೆಯನ್ನು ಕೆರಳಿಸಿದುವು. ಅವರಲ್ಲಿ ಹಿರಿಯನಾದ ದುರ್ಯೋಧನನ ಪ್ರೇರೇಪಣೆಯಂತೆ ವಾರಣಾವತದಲ್ಲಿ ಆಗುತ್ತಿದ್ದ ಯಾವುದೋ ಒಂದು ಮೇಳವನ್ನು ನೋಡಲು ಹೋಗುವಂತೆ ಪಾಂಡವರನ್ನು ಕಳುಹಿಸಿದರು. ಅಲ್ಲಿ ಅವರು ದುರ್ಯೋಧನನ ಆಣತಿಯಂತೆ ಕಟ್ಟಲ್ಪಟ್ಟಿದ್ದ ಅರಗು ಕರ್ಪೂರ ಮುಂತಾದ ಹತ್ತಿಕೊಂಡು ಉರಿಯುವ ವಸ್ತುಗಳಿಂದ ಮಾಡಿದ ಮನೆಯಲ್ಲಿ ತಂಗುವಂತೆ ಮಾಡಿದರು. ಅನಂತರ ಅದಕ್ಕೆ ರಹಸ್ಯವಾಗಿ ಬೆಂಕಿ ಇಟ್ಟರು. ಆದರೆ ಇದಕ್ಕೆ ಮುಂಚೆ ಧೃತರಾಷ್ಟ್ರನ ಮಲ ತಮ್ಮನಾದ ಸಾಧು ವಿದುರನಿಗೆ ದುರ್ಯೋಧನ ಮತ್ತು ಅವನ ಅನುಯಾಯಿಗಳ ದುಷ್ಟಪ್ರೇರಣೆ ಗೊತ್ತಾಗಿ ಪಾಂಡವರಿಗೆ ಎಚ್ಚರಿಕೆ ಕೊಟ್ಟಿದ್ದನು. ಇದರಿಂದ ಯಾರಿಗೂ ತಿಳಿಯದೆ ಪಾಂಡವರು ತಪ್ಪಿಸಿಕೊಂಡು ಹೋಗಲು ಸಾಧ್ಯವಾಯಿತು. ಕೌರವರು ಮನೆ ಭಸ್ಮೀಭೂತವಾಗಿ ಹೋದುದನ್ನು ನೋಡಿ ತಮ್ಮ ಮಾರ್ಗದ ವಿಘ್ನಗಳೆಲ್ಲ ನಿವಾರಣೆಯಾಯಿತೆಂದು ಸಮಾಧಾನಪಟ್ಟರು. ಆಗ ಧೃತರಾಷ್ಟ್ರನ ಮಕ್ಕಳು ರಾಜರಾದರು. ಪಾಂಡವ ಸಹೋದರರು ತಾಯಿಯಾದ ಕುಂತೀ ಸಮೇತ ಕಾಡಿಗೆ ಓಡಿಹೋಗಬೇಕಾಯಿತು. ಅಲ್ಲಿ ಬ್ರಾಹ್ಮಣ ವಟುಗಳಂತೆ ವೇಷ ಮರೆಸಿಕೊಂಡು ಭಿಕ್ಷೆಯಲ್ಲಿ ಕಾಲಯಾಪನೆ ಮಾಡಬೇಕಾಯಿತು. ಗೊಂಡಾರಣ್ಯದಲ್ಲಿ ಅವರು ಎಷ್ಟೋ\break ಕಷ್ಟಪಡಬೇಕಾಯಿತು. ಎಷ್ಟೋ ಸಾಹಸಗಳನ್ನು ಮಾಡಬೇಕಾಯಿತು. ಆದರೆ ಅವರ ಧೈರ್ಯ ಪೌರುಷ ಪರಾಕ್ರಮಗಳಿಂದ ಎಲ್ಲಾ ಅಪಾಯಗಳಿಂದಲೂ ಪಾರಾಗಿ ಬಂದರು. ಹೀಗೆ ಕಾಲಕಳೆಯುತ್ತಿದ್ದಾಗ ಪಕ್ಕದ ರಾಷ್ಟ್ರದ ರಾಜಕುಮಾರಿಗೆ ಸ್ವಯಂವರ ಎಂಬ ಸುದ್ದಿ ಕೇಳಿತು.

\vskip 0.1cm

ನಾನು ಕಳೆದ ರಾತ್ರಿ ಪುರಾತನ ಭರತಖಂಡದಲ್ಲಿ ಬಳಕೆಯಲ್ಲಿದ್ದ ಮದುವೆಯಾಗುವ ಒಂದು ವಿಚಿತ್ರ ಅಭ್ಯಾಸವನ್ನು ನಿಮಗೆ ಹೇಳಿದ್ದೆ. ಇದೇ ಸ್ವಯಂವರ; ಎಂದರೆ ರಾಜಕುಮಾರಿ ತಾನೇ ವರನನ್ನು ಆರಿಸಿಕೊಳ್ಳುವುದು; ಅನೇಕ ಜನ ರಾಜ ಕುಮಾರರು ಮತ್ತು ಶ‍್ರೀಮಂತರು ನೆರೆಯುತ್ತಿದ್ದರು. ಇವರಿಂದ ರಾಜಕುಮಾರಿ ತನಗೆ ಬೇಕಾದ ವರನನ್ನು ಆರಿಸಿಕೊಳ್ಳುತ್ತಿದ್ದಳು. ತನ್ನ ಹೊಗಳುಭಟ್ಟರು ಮತ್ತು ಮಾಗಧಿಗಳೊಂದಿಗೆ ಹಾರ ಹಿಡಿದುಕೊಂಡು ಹೋಗುತ್ತಿದ್ದಳು. ಇವಳನ್ನು ವರಿಸಲು ನೆರೆದ, ಸಿಂಹಾಸನದ ಮೇಲೆ ಮಂಡಿಸಿದ ಪ್ರತಿಯೊಬ್ಬ ರಾಜಕುಮಾರನ ಸಮೀಪಕ್ಕೆ ಬಂದಾಗಲೂ ಆ ರಾಜಕುಮಾರನನ್ನು ಮಾಗಧಿಗಳು ಹೊಗಳುವರು. ಅವನು ಯುದ್ಧದಲ್ಲಿ ತೋರಿದ ಪ್ರತಾಪಗಳನ್ನು ಬಣ್ಣಿಸುವರು. ರಾಜಕುಮಾರಿ ಯಾರನ್ನು ತನ್ನ ವರನೆಂದು ನಿಶ್ಚಯಿಸುವಳೊ ಅವನ ಕೊರಳಿಗೆ ಒಂದು ಸ್ವಯಂವರದ ಮಾಲೆಯನ್ನು ಹಾಕುತ್ತಿದ್ದಳು. ಅನಂತರ ಲಗ್ನ ನಿಶ್ಚಯವಾಗುತ್ತಿತ್ತು. ಪಾಂಚಾಲ ರಾಜನಾದ ದ್ರುಪದ ಪ್ರಖ್ಯಾತನಾಗಿದ್ದ. ಅವನ ಮಗಳು ದ್ರೌಪದಿ\break ಸೌಂದರ್ಯ ಮತ್ತು ಕುಶಲತೆಗೆ ಹೆಸರಾಂತವಳು. ಒಬ್ಬ ಶೂರನನ್ನು ಪತಿಯಾಗಿ ಆರಿಸಿಕೊಳ್ಳುವಳಾಗಿದ್ದಳು.

\vskip 0.1cm

ಸ್ವಯಂವರದಲ್ಲಿ ಯಾವಾಗಲೂ ಶಸ್ತ್ರಾಸ್ತ್ರಗಳ ಪ್ರೌಢಿಮೆ ಮತ್ತು ಇತರ ದೃಶ್ಯಗಳನ್ನು ಪ್ರದರ್ಶಿಸುತ್ತಿದ್ದರು. ಈ ಸಮಯದಲ್ಲಿ ಒಂದು ಮತ್ಸ್ಯವನ್ನು ಲಕ್ಷ್ಯವಾಗಿ ಮೇಲೆ ಆಕಾಶದಲ್ಲಿಟ್ಟರು. ಅದರ ಕೆಳಗೆ ಒಂದು ಸುತ್ತುತ್ತಿರುವ ಚಕ್ರ, ಆ ಚಕ್ರದಲ್ಲಿ ಒಂದು ರಂಧ್ರವಿತ್ತು. ಕೆಳಗೆ ನೆಲದ ಮೇಲೆ ಒಂದು ಪಾತ್ರೆಯಲ್ಲಿ ನೀರಿಟ್ಟರು. ಕೆಳಗೆ ಇರುವ ಪಾತ್ರೆಯಲ್ಲಿ ಬೀಳುವ ಪ್ರತಿಬಿಂಬವನ್ನು ನೋಡಿ ಮೇಲಿರುವ ಚಕ್ರದ ಮೂಲಕ ಮೀನಿನ ಕಣ್ಣನ್ನು ಭೇದಿಸಬೇಕು. ಯಾರು ಇದರಲ್ಲಿ ಜಯಶೀಲರಾಗುವರೋ ಅವರನ್ನು ರಾಜಕುಮಾರಿ ಮದುವೆಯಾಗುವಳು. ರಾಜಕುಮಾರಿಯನ್ನು ಪಡೆಯಬೇಕೆಂಬ ಉತ್ಸಾಹದಿಂದ ಭರತಖಂಡದ ನಾನಾ ಕಡೆಗಳಿಂದ ರಾಜರು ರಾಜಕುಮಾರರು ನೆರೆದರು. ಒಬ್ಬರಾದ ಮೇಲೆ ಒಬ್ಬರು ಮತ್ಸ್ಯಯಂತ್ರವನ್ನು ಭೇದಿಸಲು ಪ್ರಯತ್ನಪಟ್ಟು ನಿರಾಶರಾಗಿ ಹೋದರು.

\vskip 0.1cm

ಇಂಡಿಯಾ ದೇಶದಲ್ಲಿ ಚತುರ್ವರ್ಣಗಳಿವೆ ಎಂಬುದು ನಿಮಗೆ ಗೊತ್ತಿದೆ. ಮೊದಲನೆಯದು ಅನುವಂಶಿಕವಾಗಿ ಬಂದ ಬ್ರಾಹ್ಮಣ ಪುರೋಹಿತ, ಎರಡನೆಯದು ರಾಜರು, ಯೋಧರನ್ನೊಳಗೊಂಡ ಕ್ಷತ್ರಿಯರು, ಮೂರನೆಯವರೆ ವ್ಯಾಪಾರಸ್ಥರಾದ ವೈಶ್ಯರು.\break ಕೊನೆಯವರೆ ಊಳಿಗದವರಾದ ಶೂದ್ರರು. ರಾಜಕುಮಾರಿ ಎರಡನೆ ಗುಂಪಿಗೆ ಸೇರಿದ\break ಕ್ಷತ್ರಿಯಳು.

\vskip 0.1cm

ನೆರೆದ ಕ್ಷತ್ರಿಯರು ಮತ್ಸ್ಯಯಂತ್ರವನ್ನು ಭೇದಿಸದೆ ಹೋದಾಗ ದ್ರುಪದನ ಮಗ ಸಭೆಯಲ್ಲಿ ಎದ್ದು “ಈಗ ಕ್ಷತ್ರಿಯರು ಸೋತರು, ಇತರ ವರ್ಗದವರು ಬೇಕಾದರೆ ಪ್ರಯತ್ನಿಸಬಹುದು. ಬ್ರಾಹ್ಮಣನಾಗಲಿ, ಶೂದ್ರನಾದರೂ ಚಿಂತೆಯಿಲ್ಲ ಪ್ರಯತ್ನಿಸಬಹುದು. ಯಾರು ಯಂತ್ರವನ್ನು ಭೇದಿಸುವರೊ ಅವರನ್ನು ದ್ರೌಪದಿ ವರಿಸುವಳು” ಎಂದನು.

\vskip 0.1cm

ಬ್ರಾಹ್ಮಣ ಸಭೆಯಲ್ಲಿ ಪಂಚಪಾಂಡವ ಸಹೋದರರು ಕುಳಿತಿದ್ದರು. ಮೂರನೆಯವನಾದ ಅರ್ಜುನನು ಧನುರ್ವಿದ್ಯೆಗೆ ಹೆಸರಾಂತವನು. ಅವನು ಎದ್ದು ಮುಂದೆ ಬಂದ. ಒಟ್ಟಿನಲ್ಲಿ ಬ್ರಾಹ್ಮಣರು ಶಾಂತಪ್ರಿಯರು, ಸ್ವಲ್ಪ ಅಂಜುಕುಳಿಗಳು. ಧರ್ಮಶಾಸ್ತ್ರದ ಪ್ರಕಾರ ಅವರು ಶಸ್ತ್ರಗಳನ್ನು ಮುಟ್ಟಕೂಡದು. ಕತ್ತಿಯನ್ನು ಧರಿಸಬಾರದು. ಅಪಾಯಕರವಾದ ಯಾವ ಪ್ರಸಂಗದಲ್ಲಿಯೂ ಭಾಗಿಗಳಾಗಕೂಡದು. ಅವರ ಜೀವನವು ಧ್ಯಾನ, ಅಧ್ಯಯನಗಳಿಂದ ಕೂಡಿ ಸಂಯಮಪೂರಿತವಾದುದು. ಅವರು ಎಂತಹ ಸಮಾಧಾನ ಚಿತ್ತರು ಮತ್ತು ಶಾಂತಿಪ್ರಿಯರು ಎಂಬುದನ್ನು ನೀವೇ ಊಹಿಸಬಹುದು. ಅರ್ಜುನ ಬ್ರಾಹ್ಮಣರ ಗುಂಪಿನಿಂದ ಎದ್ದು ಬಂದಾಗ ಅವರು ತಾವೆಲ್ಲಾ ಕ್ಷತ್ರಿಯರ ಕೋಪಕ್ಕೆ ತುತ್ತಾಗಿ ನಾಶವಾಗಿ\break ಹೋಗುವೆವು ಎಂದು ಭಾವಿಸಿದರು. ಆದಕಾರಣ ಅವರೆಲ್ಲ ಅರ್ಜುನನಿಗೆ ಸುಮ್ಮನೆ ಕುಳಿತುಕೊಳ್ಳುವಂತೆ ಹೇಳಿದರು. ಆದರೆ ಅರ್ಜುನ ಯೋಧನಾದುದರಿಂದ ಅವರ ಮಾತನ್ನು ಕೇಳಲಿಲ್ಲ. ಅವನು ಬಿಲ್ಲನ್ನು ಕೈಯಲ್ಲಿ ತೆಗೆದುಕೊಂಡು ಲೀಲಾಜಾಲವಾಗಿ ಹೆದೆಯೇರಿಸಿ ಚಕ್ರದ ಮೂಲಕ ಮತ್ಸ್ಯಯಂತ್ರವನ್ನು ಭೇದಿಸಿದನು.

ಅನಂತರ ದೊಡ್ಡ ಸಂಭ್ರಮವಾಯಿತು. ರಾಜಕುಮಾರಿ ದ್ರೌಪದಿ ಸುಂದರವಾದ ಪುಷ್ಪಮಾಲೆಯನ್ನು ಅರ್ಜುನನ ಕೊರಳಿಗೆ ಹಾಕಿದಳು. ಆದರೆ ಕ್ಷತ್ರಿಯರು ಇದನ್ನು\break ಸಹಿಸಲಿಲ್ಲ. ಇಷ್ಟೊಂದು ಜನ ಕ್ಷತ್ರಿಯರು ನೆರೆದಿದ್ದರೂ ಕ್ಷತ್ರಿಯ ಕುಲಕ್ಕೆ ಸೇರಿದ ದ್ರೌಪದಿ ಅವರಾರಿಗೂ ದಕ್ಕದೆ ಒಬ್ಬ ದರಿದ್ರ ಬ್ರಾಹ್ಮಣನನ್ನು ಮದುವೆಯಾಗಬೇಕಾಯಿತಲ್ಲ ಎಂದು ಅವರಿಗೆ ಅಸಮಾಧಾನ. ಆದಕಾರಣ ಅರ್ಜುನನೊಡನೆ ಯುದ್ಧಮಾಡಿ ಬಲಾತ್ಕಾರದಿಂದ ಅವಳನ್ನು ಅಪಹರಿಸಬೇಕೆಂದು ಯತ್ನಿಸಿದರು. ಪಾಂಡವರು ಕ್ಷತ್ರಿಯರೊಡನೆ ಭೀಷಣವಾಗಿ ಕಾದಾಡಬೇಕಾಯಿತು. ಆದರೂ ಸೋಲದೆ ಜಯಶಾಲಿಗಳಾಗಿ ದ್ರೌಪದಿಯನ್ನು ಕರೆದುಕೊಂಡು ಹೋದರು.

ಪಂಚಪಾಂಡವರು ಕುಂತಿಯ ಬಳಿಗೆ ಬಂದರು. ಬ್ರಾಹ್ಮಣರು ಭಿಕ್ಷೆಯಿಂದ ಜೀವಿಸಬೇಕಾಗಿತ್ತು. ಆದಕಾರಣ ಬ್ರಾಹ್ಮಣರಾದ ಇವರು ಭಿಕ್ಷೆಮಾಡಿ ಸಿಕ್ಕಿದುದನ್ನು ಮನೆಗೆ\break ತರುತ್ತಿದ್ದರು. ತಾಯಿ ಅದನ್ನು ಎಲ್ಲರಿಗೂ ಹಂಚುತ್ತಿದ್ದಳು. ತಾಯಿ ಇದ್ದ ಗುಡಿಸಿಲಿಗೆ ದ್ರೌಪದೀ ಸಮೇತ ಪಾಂಡವರು ಬಂದರು. ಅವರು ಸಂತೋಷದಿಂದ ತಾಯಿಗೆ “ಅಮ್ಮ, ಇವತ್ತು ಒಂದು ವಿಚಿತ್ರವಾದ ಭಿಕ್ಷೆಯನ್ನು ತಂದಿರುವೆವು” ಎಂದು ಕೂಗಿಕೊಂಡರು. ತಾಯಿ “ನೀವೆಲ್ಲರೂ ಸಮನಾಗಿ ಹಂಚಿಕೊಳ್ಳಿ” ಎಂದಳು. ಕುಂತಿ ನೋಡಿದಾಗ ದ್ರೌಪದಿ ಆ ಭಿಕ್ಷೆಯಾಗಿದ್ದಳು. “ಅಯ್ಯೊ! ನಾನೇನೆಂದೆ; ಇದೊಂದು ಹುಡುಗಿ!” ಎಂದಳು. ಆದರೆ ಮಾಡುವುದೇನು? ಆಗಲೇ ತಾಯಿ ಮಾತನಾಡಿಬಿಟ್ಟಿದ್ದಳು. ಅದನ್ನು ಪಾಲಿಸದೆ ವಿಧಿಯಿಲ್ಲ. ಅದನ್ನು ಸನ್ಮಾನಿಸಲೇಬೇಕು. ತಾಯಿಯ ಮಾತು ಸುಳ್ಳಾಗಕೂಡದು. ಹಾಗೆ ಹಿಂದೆ ಎಂದೂ ಆಗಿರಲಿಲ್ಲ, ಆದಕಾರಣ ದ್ರೌಪದಿ ಪಂಚಪಾಂಡವರಿಗೆಲ್ಲ ಸತಿಯಾದಳು.

ಪ್ರತಿಯೊಂದು ಸಮಾಜದಲ್ಲಿಯೂ ಅದು ವಿಕಾಸವಾಗುತ್ತಿರುವ ಹಂತಗಳಿವೆ. ಈ ಕಾವ್ಯದಲ್ಲಿ ಪುರಾತನ ಇತಿಹಾಸ ಕಾಲದಲ್ಲಿ ರೂಢಿಯಲ್ಲಿದ್ದ ಒಂದು ವಿಚಿತ್ರವಾದ ಆಚಾರವನ್ನು ನೋಡಬಹುದು. ಕವಿ ಐದುಜನ ಸಹೋದರರು ಒಬ್ಬಳೇ ಹೆಂಗಸನ್ನು ಮದುವೆಯಾಗುವುದನ್ನು ಹೇಳುವನು. ಅವನು ಹೀಗೆ ಸಮಾಧಾನ ಮಾಡಲೆತ್ನಿಸುವನು: ಇದು ತಾಯಾಜ್ಞೆ, ಅವಳೇ ಈ ಪವಿತ್ರ ವಿವಾಹಕ್ಕೆ ಸಮ್ಮತಿಕೊಟ್ಟಳು ಎಂದು ಹೇಳುವನು. ಪ್ರತಿಯೊಂದು ದೇಶದ ಸಮಾಜದ ಒಂದು ಹಂತದಲ್ಲಿ ಬಹುಪತಿತ್ವ ಜಾರಿಯಲ್ಲಿತ್ತು. ಒಂದು ಮನೆಯ ಸಹೋದರರೆಲ್ಲ ಒಬ್ಬ ಹೆಂಗಸನ್ನು ಮದುವೆಯಾಗುತ್ತಿದ್ದರು. ಆದ್ದರಿಂದ ಈ ಪ್ರಸಂಗ ಹಳೆಯ ಬಹುಪತಿತ್ವದ ಹಂತವನ್ನು ಸೂಚಿಸುತ್ತದೆ.

ಅಷ್ಟುಹೊತ್ತಿಗೆ ರಾಜಕುಮಾರಿಯ ಅಣ್ಣನಿಗೆ ಆಶ್ಚರ್ಯವಾಯಿತು. “ಈ ಜನ\break ಯಾರು? ನನ್ನ ತಂಗಿ ಮದುವೆಯಾಗುವ ಈತ ಯಾರು? ಅವರಲ್ಲಿ ರಥ, ಕುದುರೆ\break ಏನೂ ಇಲ್ಲ. ಪಾದಚಾರಿಗಳಾಗಿರುವರು!” ಎಂದು ತರ್ಕಿಸತೊಡಗಿದ. ಸ್ವಲ್ಪ ದೂರದಿಂದ\break ಅವರ ಹಿಂದೆಯೆ ಹೋಗಿ ರಾತ್ರಿ ಅವರ ಸಂಭಾಷಣೆಯನ್ನು ಕೇಳಿ ಅವರು ನಿಜವಾಗಿ\break ಕ್ಷತ್ರಿಯರು ಎಂದು ತನ್ನ ಸಂದೇಹವನ್ನು ದೂರಮಾಡಿ ಕೊಂಡನು. ಅನಂತರ ದ್ರುಪದ\-ರಾಜನಿಗೆ ಅವರಾರೆಂಬುದು ಗೊತ್ತಾಗಿ ಅವನಿಗೆ ತುಂಬಾ ಸಂತೋಷವಾಯಿತು.

ಮೊದಲು ಅವರ ಮದುವೆಗೆ ಬೇಕಾದಷ್ಟು ಆಕ್ಷೇಪಣೆ ತಂದೊಡ್ಡಿ ದರೂ ವ್ಯಾಸರೂ\break ಈ ರಾಜಕುಮಾರಿಗೆ ಇಂತಹ ಮದುವೆಯಾಗಲು ಯಾವ ಬಾಧಕವೂ ಇಲ್ಲ ಎಂದು\break ಹೇಳಿದ ಮೇಲೆ ದ್ರುಪದ ಇದಕ್ಕೆ ಒಪ್ಪಬೇಕಾಯಿತು. ರಾಜಕುಮಾರಿ ಪಂಚಪಾಂಡವರನ್ನು ಮದುವೆಯಾದಳು.

ಅನಂತರ ಪಾಂಡವರು ನೆಮ್ಮದಿಯಿಂದ ಸುಖವಾಗಿ ಬಾಳತೊಡಗಿದರು. ದಿನಕಳೆದಂತೆ ಅವರು ಹೆಚ್ಚು ಪ್ರಖ್ಯಾತರಾಗುತ್ತ ಬಂದರು. ದುರ್ಯೋಧನ ಮತ್ತು ಅವನ\break ಅನುಯಾಯಿಗಳು ಪಾಂಡವರನ್ನು ನಾಶಮಾಡುವುದಕ್ಕೆ ಬೇರೆ ಬೇರೆ ಉಪಾಯಗಳನ್ನು ಕುರಿತು ಯೋಚಿಸಿದರೂ, ಹಿರಿಯರು ಧೃತರಾಷ್ಟ್ರನಿಗೆ ಬುದ್ಧಿ ಹೇಳಿ ಪಾಂಡವರೊಡನೆ ರಾಜಿಮಾಡಿಕೊ ಎಂದರು. ಅವರನ್ನು ಹಿಂದಕ್ಕೆ ಕರೆಸಿದನು. ಜನರೆಲ್ಲ ಪಾಂಡವರ ಬರವನ್ನು ಕಂಡು ಸಂತೋಷಪಟ್ಟರು. ಧೃತರಾಷ್ಟ್ರ ಅವರಿಗೆ ಅರ್ಧ ರಾಜ್ಯವನ್ನು ಕೊಟ್ಟನು. ಪಂಚಪಾಂಡವರು ಇಂದ್ರಪ್ರಸ್ಥ ಎಂಬ ರಾಜಧಾನಿಯನ್ನು ಕಟ್ಟಿ ತಮ್ಮ ಆಡಳಿತವನ್ನು ವಿಸ್ತರಿಸಿದರು. ಜನರೆಲ್ಲ ಅವರಿಗೆ ಕಪ್ಪಕಾಣಿಕೆಯನ್ನು ಸಲ್ಲಿಸಬೇಕಾಯಿತು. ಅವರಲ್ಲಿ ಹಿರಿಯವನಾದ ಯುಧಿಷ್ಠಿರನನ್ನು ಚಕ್ರವರ್ತಿಯೆಂದು ಒಪ್ಪಿಕೊಂಡು, ಯಾಗಕ್ಕೆ ತಮ್ಮ ಸಹಾಯವನ್ನು ನೀಡಬೇಕಾಯಿತು. ಶ‍್ರೀಕೃಷ್ಣ ಅವರ ನೆಂಟನಾಗಿದ್ದನು. ಈಗ ಅವರ ಸ್ನೇಹಿತನಾಗಿ ಅವರ ಬಳಿಗೆ ಬಂದು ರಾಜಸೂಯ ಯಾಗಕ್ಕೆ ತನ್ನ ಅನುಮತಿಯನ್ನು ಇತ್ತನು. ಆದರೆ ಅದನ್ನು ಸಾಂಗವಾಗಿ ನೆರವೇರಿಸುವುದಕ್ಕೆ ಒಂದು ವಿಘ್ನವಿತ್ತು. ಜರಾಸಂಧನೆಂಬ ರಾಜನು ನೂರುಜನ ರಾಜರನ್ನು ದೇವಿಗೆ ಬಲಿ ಕೊಡುತ್ತೇನೆಂದು ಅದಕ್ಕಾಗಿ ಎಂಭತ್ತಾರು ರಾಜರನ್ನು ಸೆರೆಯಲ್ಲಿಟ್ಟದ್ದನು. ಅವನ ಮೇಲೆ ಯುದ್ಧ ಮಾಡುವಂತೆ ಶ‍್ರೀಕೃಷ್ಣ ಬೋಧಿಸಿದನು. ಶ‍್ರೀಕೃಷ್ಣ ಭೀಮ ಅರ್ಜುನರು ಜರಾಸಂಧನಲ್ಲಿಗೆ ಹೋಗಿ ಅವನನ್ನು ಯುದ್ಧಕ್ಕೆ ಕರೆದರು. ಅವನು ಯುದ್ಧ ಮಾಡುವುದಕ್ಕೆ ಒಪ್ಪಿಕೊಂಡನು. ಭೀಮ ಅವನೊಡನೆ ಹದಿನಾಲ್ಕು ದಿನ ಯುದ್ಧ ಮಾಡಿ ಅವನನ್ನು ಸೋಲಿಸಿದನು. ಅನಂತರ ಅವನು ಸೆರೆಯಲ್ಲಿಟ್ಟ ರಾಜರನ್ನೆಲ್ಲಾ\break ಬಿಡಿಸಿದನು.

ನಾಲ್ಕು ಜನ ಸಹೋದರರು ಅನಂತರ ಸೇನಾ ಸಮೇತ ಜೈತ್ರಯಾತ್ರೆಗೆ ಹೊರಟರು. ಪ್ರತಿಯೊಬ್ಬರೂ ಒಂದೊಂದು ದಿಕ್ಕಿಗೆ ಹೋಗಿ ಆಯಾ ರಾಜರನ್ನು ಯುಧಿಷ್ಠಿರನ\break ಆಡಳಿತದ ಅಡಿಗೆ ತಂದರು. ಹಿಂತಿರುಗಿ ಬಂದಾಗ ಮಹಾಯಾಗಕ್ಕಾಗಿ ತಾವು ಶೇಖರಿಸಿದ ನಿಧಿರಾಶಿಯನ್ನೆಲ್ಲಾ ಅಣ್ಣನ ಮುಂದೆ ಇಟ್ಟರು.

\vskip 0.1cm

ಬಿಡುಗಡೆಯಾದ ರಾಜಕುಮಾರರ ಜೊತೆಗೆ ಪಾಂಡವರು ಯಾರನ್ನು ಗೆದ್ದರೋ\break ಅವರೆಲ್ಲರೂ ರಾಜಸೂಯ ಯಾಗಕ್ಕೆ ಬಂದು ಯುಧಿಷ್ಠಿರನಿಗೆ ಕಪ್ಪ ಕಾಣಿಕೆಗಳನ್ನು\break ಒಪ್ಪಿಸಿದರು. ಧೃತರಾಷ್ಟ್ರನಿಗೂ ಕೂಡ ತನ್ನ ಮಕ್ಕಳೊಡನೆ ಬಂದು ಯಾಗದಲ್ಲಿ\break ಭಾಗಿಯಾಗಬೇಕೆಂದು ಆಮಂತ್ರಣ ಕಳುಹಿಸಿದರು. ಯಾಗದ ಕೊನೆಯಲ್ಲಿ ಯುಧಿಷ್ಠಿರನಿಗೆ ಚಕ್ರವರ್ತಿ ಬಿರುದನ್ನು ಕೊಟ್ಟು ಅವನ ಸಾರ್ವಭೌಮತ್ವವನ್ನು ಎಲ್ಲರೂ ಒಪ್ಪಿಕೊಂಡರು. ಇದೇ ಮುಂದಿನ ಕಲಹಕ್ಕೆ ಬೀಜ ನೆಟ್ಟಂತಾಯಿತು. ಯುಧಿಷ್ಠಿರನ ಮೇಲೆ ಅಸೂಯೆಪಡುತ್ತ ದುರ್ಯೋಧನ ಹಿಂತಿರುಗಿದನು. ಅವರ ಸಾರ್ವಭೌಮತ್ವ, ಅಧಿಕಾರ, ಐಶ್ವರ್ಯವನ್ನು ಇವನು ಸಹಿಸಲಾರದೆ ಹೋದ. ಶಕ್ತಿಯ ಮೂಲಕ ಅವರನ್ನು ಗೆಲ್ಲಲು ಸಾಧ್ಯವಿಲ್ಲವೆಂದು ಅರಿತು ಯುಕ್ತಿಯಿಂದ ಅವರ ಪತನಕ್ಕೆ ಯೋಜನೆಗಳನ್ನು ಚಿಂತಿಸತೊಡಗಿದನು.\break ಯುಧಿಷ್ಠಿರ ರಾಜನಿಗೆ ಪಗಡೆಯಾಟದ ಮೇಲೆ ಆಸಕ್ತಿ. ದುರ್ಯೋಧನನ ಕುಯುಕ್ತಿಯಿಂದ ಪ್ರೇರಿತನಾಗಿ, ಚಮತ್ಕಾರವಾಗಿ ಪಗಡೆಯಾಟವಾಡಬಲ್ಲ ಶಕುನಿ ಯುಧಿಷ್ಠಿರನನ್ನು ವಿಷಗಳಿಗೆಯಲ್ಲಿ ಆಟಕ್ಕೆ ಕರೆದನು. ಹಿಂದಿನ ಕಾಲದಲ್ಲಿ ಕ್ಷತ್ರಿಯರನ್ನು ಯಾರಾದರೂ ಹೋರಾಡಲು ಕರೆದರೆ ಏನಾದರೂ ಆಗಲಿ ತನ್ನ ಗೌರವಕ್ಕೆ ಕಳಂಕ ಬರದಂತೆ ಹೋರಾಡಲು\break ಅನುವಾಗಬೇಕು. ಪಗಡೆಯಾಟಕ್ಕೆ ಅವನನ್ನು ಕರೆದರೆ ತನ್ನ ಗೌರವ ಕಾಪಾಡಿಕೊಳ್ಳುವುದಕ್ಕೆ ಆಡಲೇಬೇಕು. ಇಲ್ಲದಿದ್ದರೆ ಅದು ಅಗೌರವ. ಯುಧಿಷ್ಠಿರನು ಎಲ್ಲಾ ಶ್ರೇಷ್ಠ ಗುಣಗಳಿಗೂ ತೌರುಮನೆಯಾಗಿದ್ದನು. ಇಂತಹ ರಾಜಋಷಿ ಕೂಡ ಆಟವನ್ನು ಆಡಲು ಒಪ್ಪಿಕೊಳ್ಳಲೇಬೇಕಾಯಿತು. ಶಕುನಿ ಮತ್ತು ಅವನ ಕಡೆಯವರ ಕಪಟದಿಂದ ಯುಧಿಷ್ಠಿರ ಪ್ರತಿಸಲವೂ ಸೋಲುತ್ತಾ ಬಂದ. ಸೋಲಿನಿಂದ ಉನ್ಮತ್ತನಾಗಿ ತನ್ನಲ್ಲಿರುವುದನ್ನೆಲ್ಲಾ ಪಣ ಒಡ್ಡಿ ಆಡುತ್ತಾ ಹೋದ. ಕೊನೆಗೆ ಸರ್ವಸ್ವವನ್ನೂ ತನ್ನ ರಾಜ್ಯವನ್ನೂ ಸೋತ. ಕಟ್ಟಕಡೆಗೆ ಪಣ ಒಡ್ಡುವುದಕ್ಕೆ ಅವನಲ್ಲಿ ಏನೂ ಇರಲಿಲ್ಲ. ತನ್ನ ಸಹೋದರರು, ತಾನು, ಕೊನೆಗೆ ಸುಂದರಾಂಗನೆ ದ್ರೌಪದಿ ಎಲ್ಲರನ್ನು ಸೋತನು. ಈಗ ಸಂಪೂರ್ಣವಾಗಿ ಕೌರವರು ಹೇಳಿದಂತೆ ಕೇಳಬೇಕು. ಕೌರವರು ಪಾಂಡವರನ್ನು ಎಲ್ಲಾ ಬಗೆಯ ಅಪಹಾಸ್ಯಕ್ಕೆ ಗುರಿ ಮಾಡಿದರು. ದ್ರೌಪದಿಯನ್ನು ಅತಿ ನೀಚವಾಗಿ ಕಂಡರು. ಕೊನೆಗೆ ಧೃತರಾಷ್ಟ್ರ ಮಧ್ಯೆ ಬಂದು ಪಾಂಡವರು ಮುಕ್ತರಾಗಿ ಊರಿಗೆ ಹಿಂತಿರುಗಿ ರಾಜ್ಯಭಾರ ಮಾಡುವಂತೆ ಹೇಳಿದನು. ಆದರೆ ದುರ್ಯೋಧನನಿಗೆ ಇದರಿಂದ ಮುಂದೆ ಬರುವ ಅಪಾಯದ ಸುಳಿವು ಗೊತ್ತಾಯಿತು. ಮತ್ತೊಮ್ಮೆ ಪಗಡೆ ಆಟ ಆಡುವುದಕ್ಕೆ ಅವಕಾಶ ಕೊಡೆಂದು ತಂದೆಯ ಅನುಮತಿಯನ್ನು ಬಲತ್ಕಾರದಿಂದ ಪಡೆದನು. ಇದರಲ್ಲಿ ಸೋತವರು ಹನ್ನೆರಡು ವರುಷ ವನವಾಸ, ಒಂದು ವರುಷ ಅಜ್ಞಾತವಾಸ ಮಾಡಬೇಕು. ಈ ಸಮಯದಲ್ಲಿ ಅವರನ್ನು ಕಂಡುಹಿಡಿದರೆ ಪುನಃ ಹನ್ನೆರಡು ವರುಷ ವನವಾಸ ಮಾಡಬೇಕು. ಆನಂತರ ಮಾತ್ರ ಅವರಿಗೆ ರಾಜ್ಯ ದೊರಕುವುದು ಎಂದನು. ಈ ಕೊನೆಯ ಆಟದಲ್ಲಿಯೂ ಯುಧಿಷ್ಠಿರನು ಸೋತು ತನ್ನ ಸಹೋದರರು ಮತ್ತು ದ್ರೌಪದಿಯೊಡನೆ\break ವನವಾಸಕ್ಕೆ ಹೋದನು. ಹನ್ನೆರಡು ವರುಷ ಅವರು ಗಿರಿ ವನಗಳಲ್ಲಿ ವಾಸಿಸಿದರು. ಅವರು ಅಲ್ಲಿ ಎಷ್ಟೋ ಧರ್ಮ ಕಾರ್ಯಗಳನ್ನು ಮಾಡಿದರು. ಸಾಹಸ ಕೃತ್ಯಗಳನ್ನು ಮಾಡಿದರು. ಅನೇಕ ವೇಳೆ ದೂರದೂರದ ಪುಣ್ಯ ಸ್ಥಳಗಳಿಗೆ ಯಾತ್ರೆ ಹೋಗುತ್ತಿದ್ದರು. ಈ ಭಾಗದ ಕಥೆ ಬಹಳ ರಸಭರಿತವಾಗಿದೆ. ಈ ಭಾಗದಲ್ಲಿ ಎಷ್ಟೋ ಘಟನೆಗಳು, ಕಥೆ ಉಪ ಕಥೆಗಳು ಇವೆ. ಪುರಾತನ ಭರತಖಂಡದ ಧರ್ಮ ಮತ್ತು ತತ್ತ್ವಕ್ಕೆ ಸಂಬಂಧಪಟ್ಟ ಹಲವು ಭವ್ಯ ಕಥೆಗಳಿವೆ. ಅವರು ವನವಾಸದಲ್ಲಿದ್ದಾಗ ದೊಡ್ಡ ದೊಡ್ಡ ಋಷಿಗಳು ಬಂದು ವನವಾಸದ ಕಷ್ಟವನ್ನು ಹಗುರ ಮಾಡಲು ಬೋಧಪ್ರದವಾದ ಹಲವು ಹಿಂದಿನ ಕಾಲದ ಕಥೆಗಳನ್ನು ಹೇಳುತ್ತಿದ್ದರು. ಅದರಲ್ಲಿ ಒಂದನ್ನು ಮಾತ್ರ ನಿಮಗೆ ಹೇಳುತ್ತೇನೆ.

\vskip 0.1cm

ಅಶ್ವಪತಿಯೆಂಬ ರಾಜನಿದ್ದ. ಅವನಿಗೊಬ್ಬಳು ಮಗಳಿದ್ದಳು. ಅವಳು ಬಹಳ\break ಒಳ್ಳೆಯವಳು ಮತ್ತು ಅತಿ ಸುಂದರವಾಗಿದ್ದಳು. ಅವಳಿಗೆ ಸಾವಿತ್ರಿ ಎಂದು ಹೆಸರು\break ಕೊಟ್ಟರು. ಸಾವಿತ್ರಿ ಎಂಬುದು ಹಿಂದೂಗಳ ಒಂದು ಪವಿತ್ರ ಪ್ರಾರ್ಥನೆ. ಸಾವಿತ್ರಿ ಪ್ರಾಪ್ತ ವಯಸ್ಕಳಾದಾಗ ತಂದೆಯು ಮಗಳನ್ನು ಕುರಿತು ನೀನೇ ಒಬ್ಬ ವರನನ್ನು ಆರಿಸಿಕೋ ಎಂದನು. ನೋಡಿ, ಪುರಾತನ ಕಾಲದ ಭಾರತದ ರಾಜಕುಮಾರಿಯರು ಬಹಳ ಸ್ವಾತಂತ್ರ್ಯ ಪ್ರಿಯರು, ಅವರು ತಾವೇ ಅರಸು ಮಕ್ಕಳನ್ನು ಗಂಡನಾಗಿ ಆರಿಸಿಕೊಳ್ಳುತ್ತಿದ್ದರು.

\vskip 0.1cm

ಸಾವಿತ್ರಿ ಒಪ್ಪಿ ಚಿನ್ನದ ರಥವೇರಿ ದೂರ ದೇಶಕ್ಕೆ ಅವಳ ತಂದೆ ನೇಮಿಸಿದ ಸೇವಕರು ಮತ್ತು ವೃದ್ಧ ಊಳಿಗದವರೊಡನೆ ಹೊರಟಳು. ಎಷ್ಟೋ ರಾಜಧಾನಿಗಳಲ್ಲಿ ತಂಗಿ,\break ಎಷ್ಟೋ ರಾಜಕುಮಾರರನ್ನು ನೋಡಿದಳು. ಆದರೆ ಯಾರೂ ಸಾವಿತ್ರಿಯ ಒಪ್ಪಿಗೆಗೆ\break ಅರ್ಹರಾಗಿರಲಿಲ್ಲ. ಕೊನೆಗೆ ಅವರೊಂದು ಪರ್ಣಶಾಲೆಗೆ ಬಂದರು. ಅದು ಕೇವಲ\break ಮೃಗಗಳಿಗೆ ಮಾತ್ರ ಮೀಸಲಾಗಿದ್ದ ಸ್ಥಳ. ಅಲ್ಲಿ ಯಾವ ಪ್ರಾಣಿಗಳನ್ನೂ ಕೊಲ್ಲಬಾರದಾಗಿತ್ತು. ಮೃಗಗಳಿಗೆ ಮನುಷ್ಯರ ಭಯವಿರಲಿಲ್ಲ. ಸರೋವರದಲ್ಲಿರುವ ಮೀನುಗಳು\break ಕೂಡ ಹತ್ತಿರ ಬಂದು ಕೈಯಿಂದ ಆಹಾರವನ್ನು ತೆಗೆದುಕೊಳ್ಳುತ್ತಿದ್ದವು. ಸಾವಿರಾರು\break ವರ್ಷಗಳಿಂದ ಅಲ್ಲಿ ಯಾರೂ ಏನನ್ನೂ ಕೊಂದಿರಲಿಲ್ಲ. ಋಷಿಗಳು ಮತ್ತು ವೃದ್ಧರು, ಪಕ್ಷಿಗಳು ಮತ್ತು ಜಿಂಕೆಗಳೊಡನೆ ಅಲ್ಲಿ ವಾಸಿಸಲು ಹೋಗುತ್ತಿದ್ದರು. ತಪ್ಪಿತಸ್ಥರಿಗೂ ಕೂಡ ಅಲ್ಲಿ ಯಾವ ಅಪಾಯವೂ ಇರಲಿಲ್ಲ. ಜನರು, ಜೀವನವು ಜಿಗುಪ್ಸೆಯಾದರೆ ಕಾಡಿಗೆ ಹೋಗಿ ಋಷಿಗಳೊಡನೆ ಧಾರ್ಮಿಕ ವಿಚಾರಗಳನ್ನು ಮಾತನಾಡುತ್ತಾ ತಮ್ಮ ಕೊನೆಗಾಲವನ್ನು ಕಳೆಯುತ್ತಿದ್ದರು.

\vskip 0.1cm

ದ್ಯುಮತ್ಸೇನನೆಂಬ ರಾಜನಿದ್ದ. ಅವನು ವೈರಿಗಳ ಕೈಯಲ್ಲಿ ಸೋತು, ರಾಜ್ಯವನ್ನು ಕಳೆದುಕೊಂಡು ವೃದ್ಧನಾಗಿ ಕಣ್ಣು ಕಳೆದುಕೊಂಡ. ಈ ವೃದ್ಧನಾದ ದರಿದ್ರ ಅಂಧದೊರೆ\break ತನ್ನ ರಾಣಿ ಮತ್ತು ಮಗನೊಂದಿಗೆ ಕಾಡಿನಲ್ಲಿ ಆಶ್ರಯ ಪಡೆದು ತಪಸ್ಸಿನಲ್ಲಿ ನಿರತನಾಗಿದ್ದನು. ಆ ಮಗನ ಹೆಸರು ಸತ್ಯವಾನ್​.

\vskip 0.1cm

ಹಲವು ಅರಸರ ಆಸ್ಥಾನಗಳನ್ನೆಲ್ಲಾ ನೋಡಿದ ಮೇಲೆ ಸಾವಿತ್ರಿ ಈ ಪರ್ಣಶಾಲೆಗೆ ಬಂದಳು. ಎಂತಹ ದೊಡ್ಡ ರಾಜನಾಗಲಿ ಪರ್ಣಶಾಲೆಯ ಮುಂದೆ ಹೋಗುವಾಗ\break ಅಲ್ಲಿರುವ ಋಷಿಗಳಿಗೆ ಗೌರವ ತೋರದೆ ಇರುತ್ತಿರಲಿಲ್ಲ. ಏಕೆಂದರೆ ಋಷಿಗಳನ್ನು ಅಷ್ಟು ಪೂಜ್ಯ ದೃಷ್ಟಿಯಿಂದ ಗೌರವದಿಂದ ನೋಡುತ್ತಿದ್ದರು. ಭರತಖಂಡದ ಸಾರ್ವಭೌಮನು ಕೂಡ ತನ್ನ ಮೂಲ ಪುರುಷರು ಅರಣ್ಯದಲ್ಲಿ ಕಂದಮೂಲ ತಿಂದು ನಾರುಡುಗೆ ಉಟ್ಟ ಋಷಿ ಎಂದು ಅರಿಯಲು ಸಂತೋಷಪಡುವನು. ನಾವೆಲ್ಲ ಋಷಿಪುತ್ರರು. ಧರ್ಮಕ್ಕೆ\break ಕೊಡುವ ಗೌರವ ಇದು. ರಾಜರು ಪರ್ಣಶಾಲೆಯ ಸಮೀಪದಲ್ಲಿ ಹೋಗುತ್ತಿದ್ದರೆ\break ಒಳಗೆ ಹೋಗಿ ಋಷಿಗಳಿಗೆ ಗೌರವ ತೋರುವುದು ಒಂದು ಭಾಗ್ಯವೆಂದು ಭಾವಿಸುವರು.\break ಕುದುರೆಯ ಮೇಲೆ ಬಂದರೆ ಇಳಿದು ಸ್ವಲ್ಪ ದೂರ ನಡೆದುಕೊಂಡು ಹೋಗುವರು.\break ರಥ ಪದಾತಿಗಳೊಡನೆ ಬಂದರೆ ಕಾಡಿನ ಹೊರಗೆ ಅದನ್ನು ಬಿಟ್ಟು ಅನಂತರ ಆಶ್ರಮ\break ಪ್ರದೇಶವನ್ನು ಪ್ರವೇಶಿಸಬೇಕು.

\vskip 0.1cm

ಸಾವಿತ್ರಿ ಪರ್ಣಶಾಲೆಗೆ ಬಂದು ಋಷಿಯ ಮಗ ಸತ್ಯವಾನನನ್ನು ಕಂಡಳು. ಅವನು ಅವಳ ಹೃದಯವನ್ನು ತಕ್ಷಣ ಒಲಿಸಿಕೊಂಡನು. ಎಲ್ಲಾ ಆಸ್ಥಾನಗಳ, ಅರಮನೆಗಳಲ್ಲಿರುವ ಅರಸು ಮಕ್ಕಳಿಂದ ಪಾರಾಗಿ ಬಂದವಳ ಹೃದಯವನ್ನು ದ್ಯುಮತ್ಸೇನನ ಪರ್ಣಶಾಲೆಯಲ್ಲಿದ್ದ ಸತ್ಯವಾನ್​ ಗೆದ್ದನು.

\vskip 0.1cm

ಸಾವಿತ್ರಿ ತಂದೆಯ ಮನೆಗೆ ಹಿಂತಿರುಗಿ ಬಂದ ಮೇಲೆ, ತಂದೆ “ಮಗು ಸಾವಿತ್ರಿ ಮಾತಾಡು, ನೀನು ಮದುವೆಯಾಗಬಲ್ಲ ಯಾರನ್ನಾದರೂ ನೋಡಿದೆಯಾ?” ಎಂದು ಕೇಳಿದ. ಸಾವಿತ್ರಿ ಮೆಲುಧ್ವನಿಯಲ್ಲಿ ಲಜ್ಜೆಯಿಂದ “ಹೌದು ಅಪ್ಪ” ಎಂದಳು. ತಂದೆ ಆ ರಾಜಕುಮಾರನ ಹೆಸರೇನು ಎಂದನು. “ಅವನು ರಾಜಕುಮಾರನಲ್ಲ, ರಾಜ್ಯವನ್ನು ಕಳೆದುಕೊಂಡ ದ್ಯುಮತ್ಸೇನನ ಮಗ. ರಾಜ್ಯವಿಲ್ಲದ ರಾಜಕುಮಾರ, ಆಶ್ರಮವಾಸಿ ಅವನು, ಸಂನ್ಯಾಸಿಯಂತೆ ಇರುವನು. ಒಂದು ಪರ್ಣಶಾಲೆಯಲ್ಲಿರುವ ವೃದ್ಧ ತಂದೆ ತಾಯಿಗಳನ್ನು ಕಂದಮೂಲಗಳನ್ನು ಆಯ್ದು ತಂದು ನೋಡಿಕೊಳ್ಳುತ್ತಿರುವನು” ಎಂದಳು.

\vskip 0.1cm

ಇದನ್ನು ಕೇಳಿದ ಮೇಲೆ ರಾಜನು ಅಲ್ಲಿಗೆ ಬಂದಿದ್ದ ನಾರದರ ಸಲಹೆಯನ್ನು ಕೇಳಿದನು. “ಇದರಷ್ಟು ಅಮಂಗಳವಾದ ಆಯ್ಕೆಯೇ ಇಲ್ಲ” ಎಂದರು ಅವರು. ರಾಜ ಇದು ಹೇಗೆ ಎಂದು ಕೇಳಿದಾಗ “ಮದುವೆಯಾದ ಹನ್ನೆರಡು ತಿಂಗಳಿಗೆ ಆ ಯುವಕ ಸಾಯುವನು” ಎಂದನು. ರಾಜ ಭಯದಿಂದ ಕಂಪಿತನಾಗಿ “ಸಾವಿತ್ರಿ, ಹನ್ನೆರಡು ತಿಂಗಳಲ್ಲಿ ಆ ಹುಡುಗ ಸಾಯುವನು, ನೀನು ವಿಧವೆಯಾಗುವೆ; ಇದನ್ನು ಆಲೋಚಿಸಿ ನೋಡು. ನಿನ್ನ ಆಯ್ಕೆಯನ್ನು ಹಿಂತೆಗೆದುಕೊ. ಆ ಅಲ್ಪಾಯುವಿಗೆ ನಿನ್ನನ್ನು ಎಂದಿಗೂ ಲಗ್ನ ಮಾಡಿಕೊಡುವುದಿಲ್ಲ” ಎಂದನು. “ತಂದೆ, ಚಿಂತೆಯಿಲ್ಲ, ಇನ್ನೊಬ್ಬನನ್ನು ಮದುವೆಯಾಗಿ ನನ್ನ ಮನದ ಪಾತಿವ್ರತ್ಯವನ್ನು ಬಲಿಕೊಡು ಎನ್ನಬೇಡ. ನಾನು ಅವನನ್ನು ಪ್ರೀತಿಸುತ್ತೇನೆ. ಆ ಶೀಲಸಂಪನ್ನ, ಧೀರ ಸತ್ಯವಾನನನ್ನು ಮಾತ್ರ ನಾನು ಪತಿಯೆಂದು ಮನದಲ್ಲಿ ಸ್ವೀಕರಿಸಿರುವೆನು. ಕನ್ಯೆ ಆರಿಸಿಕೊಳ್ಳುವುದು ಒಂದೇ ಬಾರಿ, ಅವಳು ಅದಕ್ಕೆ ವಿಮುಖಳಾಗುವುದಿಲ್ಲ” ಎಂದಳು. ಸಾವಿತ್ರಿ ದೃಢ ಮನಸ್ಕಳಾಗಿರುವುದನ್ನು ನೋಡಿ ತಂದೆಯು ಮದುವೆಗೆ ಒಪ್ಪಬೇಕಾಯಿತು. ಸಾವಿತ್ರಿ ಸತ್ಯವಾನನನ್ನು ಮದುವೆಯಾಗಿ ತನ್ನ ತಂದೆಯ ಅರಮನೆಯಿಂದ ವನವಾಸಕ್ಕೆ ಯಾವ ಉದ್ವೇಗವೂ ಇಲ್ಲದೆ ತಾನೊಲಿದ ಪತಿಯೊಂದಿಗೆ ಬಾಳಲು, ಅವನ ಮಾತಾಪಿತೃಗಳಿಗೆ ಸೇವೆ ಸಲ್ಲಿಸಲು ಹೊರಟಳು. ಸತ್ಯವಾನ್​ ಎಂದು ಸಾಯುವನು ಎಂಬುದು ಸರಿಯಾಗಿ ಅವಳಿಗೆ ಗೊತ್ತಿದ್ದರೆ ಅದನ್ನು ಪತಿಗೆ ತಿಳಿಸಲಿಲ್ಲ. ಪ್ರತಿದಿನ ಸತ್ಯವಾನ್​ ಕಾಡಿಗೆ ಹೋಗಿ ಹಣ್ಣು ಹೂಗಳನ್ನು ಆಯ್ದು ಕಟ್ಟಿಗೆಯನ್ನು ಹೊತ್ತು ಪರ್ಣಶಾಲೆಗೆ ತರುತ್ತಿದ್ದನು. ಸಾವಿತ್ರಿ ಅಡಿಗೆ ಮಾಡಿ ವೃದ್ಧರ ಶುಶ್ರೂಷೆ ಮಾತುತ್ತಿದ್ದಳು. ಹೀಗೆ ಅವರ ಜೀವನ ಸಾಗಿತು. ದುರಂತದ ದಿನ ಬಳಿ ಸಾರಿತು. ಅದಕ್ಕೆ ಕೇವಲ ಮೂರು ದಿನಗಳು ಮಾತ್ರ ಉಳಿದಿತ್ತು. ಸಾವಿತ್ರಿಯು ಮೂರು ದಿನ ಉಪವಾಸ ಪ್ರಾರ್ಥನೆಗಳ ಕಠಿಣ ವ್ರತವನ್ನು ಕೈಕೊಂಡಳು. ದುಃಖದಿಂದ ನಿದ್ರೆ ಮಾಡದೆ ರಾತ್ರೆಯನ್ನೆಲ್ಲಾ ಪ್ರಾರ್ಥನೆಯಲ್ಲಿ ಕಳೆದಳು. ಇತರರಿಗೆ ಕಾಣದೆ ಕಣ್ಣೀರು ಸುರಿಸಿದಳು. ದುರಂತದ ದಿನ ಕೊನೆಗೆ ಬಂತು. ಅಂದು ಸತ್ಯವಾನನಿಂದ ಕ್ಷಣವಾದರೂ ಸಾವಿತ್ರಿ ಅಗಲಿರಲು ಬಯಸಲಿಲ್ಲ. ಗಂಡ ಹಣ್ಣು ಹಂಪಲುಗಳನ್ನು ತರಲು ಕಾಡಿಗೆ ಹೋದಾಗ ತನಗೂ ಅವನೊಡನೆ ಹೋಗಲು ಅಪ್ಪಣೆ ಕೊಡಿ ಎಂದು ಅತ್ತೆ ಮಾವಂದಿರನ್ನು ಕೇಳಿ ಗಂಡನನ್ನು ಹಿಂಬಾಲಿಸಿದಳು. ಸತ್ಯವಾನ್​ ಏಕೋ ಸಾಕಾಗುತ್ತಿದೆ ಎಂದು ತಡೆದು ತಡೆದು ಹೇಳಿದನು. “ಪ್ರಿಯೆ ಸಾವಿತ್ರಿ, ನನ್ನ ತಲೆ ಏಕೋ ಸುತ್ತುತ್ತಿದೆ. ಕೈಕಾಲುಗಳು ಸೋಲುತ್ತಿವೆ. ಕತ್ತಲು ನನ್ನನ್ನು ಕವಿಯುವಂತಿದೆ. ಸ್ವಲ್ಪ ನಿನ್ನ ಸಮೀಪದಲ್ಲಿ ಮಲಗುತ್ತೇನೆ” ಎಂದ. “ಪ್ರಿಯತಮ, ನನ್ನ ತೊಡೆಯ ಮೇಲೆ ಮಲಗಿ” ಎಂದಳು ಸಾವಿತ್ರಿ, ಸುಡುತ್ತಿರುವ ತನ್ನ ತಲೆಯನ್ನು ಹೆಂಡತಿಯ ತೊಡೆಯ ಮೇಲೆ ಇಟ್ಟೊಡನೆಯೆ ನಿಟ್ಟುಸಿರುಬಿಟ್ಟು ಪ್ರಾಣವನ್ನು ತೊರೆದನು. ಅವನನ್ನು ಅಪ್ಪಿಕೊಂಡು ಕಂಬನಿ ಕರೆಯುತ್ತಾ ಆ ನಿರ್ಜನಾರಣ್ಯದಲ್ಲಿ ಒಬ್ಬಳೇ ಕುಳಿತಳು. ಆಗ ಯಮದೂತರು ಸತ್ಯವಾನನ ಪ್ರಾಣವನ್ನು ತೆಗೆದುಕೊಂಡು ಹೋಗಲು ಬಂದರು. ಆದರೆ ಅವರಿಗೆ ಆಕೆಯ ಸಮೀಪಕ್ಕೆ ಹೋಗಲಾಗಲಿಲ್ಲ. ಅವಳ ಸುತ್ತ ಅಗ್ನಿಯ ವಲಯವಿತ್ತು. ಅವರು ಹಿಂತಿರುಗಿ ಹೋಗಿ ಏಕೆ ಸತ್ಯವಾನನ ಜೀವವನ್ನು ತೆಗೆದುಕೊಂಡು ಬರಲು ಆಗಲಿಲ್ಲ ಎಂಬುದನ್ನು ಯಮಧರ್ಮನಿಗೆ ವಿವರಿಸಿದರು.

\vskip 0.1cm

ಆಗ ಮೃತರಿಗೆ ತೀರ್ಪನ್ನು ನೀಡುವ ಯಮಧರ್ಮನೇ ಬಂದನು. ಅವನೇ ಪೃಥ್ವಿಯಲ್ಲಿ ಕಾಲವಾದವರಲ್ಲಿ ಮೊದಲನೆಯವನು. ಅವನು ಅನಂತರ ಕಾಲವಾಗುವವರಿಗೆಲ್ಲ ರಾಜನಾದನು. ಮನುಷ್ಯ ಕಾಲವಾದ ಮೇಲೆ ಅವನನ್ನು ಶಿಕ್ಷಿಸಬೇಕೇ ರಕ್ಷಿಸಬೇಕೇ ಎಂಬುದನ್ನು ನಿರ್ಧರಿಸುವನು. ಈಗ ಅವನೇ ಪ್ರತ್ಯಕ್ಷವಾಗಿ ಬಂದನು. ಅವನು ದೇವನಾದುದ\break ರಿಂದ ಅಗ್ನಿಚಕ್ರದೊಳಗೆ ಹೋಗಬಲ್ಲವನಾಗಿದ್ದನು ಸಾವಿತ್ರಿಯನ್ನು ಕುರಿತು “ಮಗಳೆ, ಈ ಮೃತದೇಹವನ್ನು ತೊರೆ. ಮರ್ತ್ಯರೆಲ್ಲ ಮೃತರಾಗಲೇಬೇಕು. ಮೃತನಾದ ಮೊದಲ ಮರ್ತ್ಯ ನಾನೆ. ಅಂದಿನಿಂದ ಪ್ರತಿಯೊಬ್ಬರೂ ಸಾಯಲೇಬೇಕಾಗಿದೆ. ಮನುಷ್ಯರೆಲ್ಲ ಸಾಯಲೇಬೇಕಾಗಿದೆ” ಎಂದನು ಯಮರಾಜ. ಹೀಗೆ ಹೇಳಿದ ಮೇಲೆ ಸಾವಿತ್ರಿ ಹಿಂದೆ ಸರಿದಳು. ಯಮ ಸತ್ಯವಾನನ ಪ್ರಾಣವನ್ನು ಸೆಳೆದುಕೊಂಡನು. ಯಮ ಅವನ ಪ್ರಾಣವನ್ನು ತೆಗೆದುಕೊಂಡು ಮುಂದೆ ನಡೆದ. ಅವನು ಸ್ವಲ್ಪ ದೂರ ಹೋಗುವುದರೊಳಗೆ ಹಿಂದೆ ತರಗೆಲೆಯ ಮೇಲೆ ಕಾಲ ಸಪ್ಪಳ ಕೇಳಿಸಿತು. “ಮಗಳೆ ಸಾವಿತ್ರಿ, ನೀನು ಏತಕ್ಕೆ ನನ್ನನ್ನು ಹಿಂಬಾಲಿಸುತ್ತಿರುವಿ? ಎಲ್ಲ ಮಾನವರ ಅಂತ್ಯವೇ ಇದು” ಎಂದನು. ಸಾವಿತ್ರಿ “ತಂದೆ ನಾನು ನಿನ್ನನ್ನು ಅನುಸರಿಸುತ್ತಿಲ್ಲ. ಆದರೆ ಇದು ಹೆಂಗಸರ ಗತಿ. ಅವಳು ಪತಿ ಹೋದೆಡೆ ಅವನನ್ನು ಹಿಂಬಾಲಿಸಬೇಕು. ಸನಾತನ ಧರ್ಮವು ಪ್ರೀತಿಸುತ್ತಿದ್ದ ಪತಿಯಿಂದ ಪತಿವ್ರತಳಾದ ಸತಿಯನ್ನು ಎಂದೂ ಅಗಲಿಸಬಾರದು” ಎಂದಳು. ಆಗ ಯಮಧರ್ಮನು “ನಿನ್ನ ಗಂಡನ ಪ್ರಾಣವನ್ನು ಹೊರತು ಮತ್ತಾವುದಾದರೊಂದು ವರವನ್ನು ಕೇಳು” ಎಂದನು. “ನೀನು ನನಗೆ ವರವನ್ನು ಕರುಣಿಸುವ ಹಾಗಿದ್ದರೆ ನನ್ನ ಮಾವ ಅಂಧತ್ವದಿಂದ ಪಾರಾಗಿ ಸುಖವಾಗಿರಲಿ” ಎಂದು ಬೇಡಿಕೊಂಡಳು. “ಕರ್ತವ್ಯಪರಾಯಣಳಾದ ಮಗಳೆ, ನಿನ್ನ ಶುಭೇಚ್ಛೆ ಫಲಿಸಲಿ” ಎಂದನು. ಅನಂತರ ಯಮಧರ್ಮ ಸತ್ಯವಾನನ ಪ್ರಾಣದೊಂದಿಗೆ ಮುಂದುವರಿದನು. ಪುನಃ ಹಿಂದಿನಿಂದ ಅದೇ ಕಾಲ ಸಪ್ಪಳ ಕೇಳಿಸಿತು, ಹಿಂತಿರುಗಿ ನೋಡಿದ. “ಮಗಳೆ ಸಾವಿತ್ರಿ, ನೀನು ಪುನಃ ನನ್ನನ್ನೇ ಅನುಸರಿಸುತ್ತಿರುವೆಯಲ್ಲ?” ಎಂದು ಕೇಳಿದ. “ಹೌದು ತಂದೆ, ಹಾಗೆ ಮಾಡದೆ ವಿಧಿಯಿಲ್ಲ. ನಾನು ಹಿಂತಿರುಗಿ ಹೋಗಬೇಕೆಂದು ಎಷ್ಟೋ ಪ್ರಯತ್ನಿಸುವೆನು. ಆದರೆ ಮನಸ್ಸು ಸತ್ಯವಾನನನ್ನೇ ಅನುಸರಿಸುವುದು. ದೇಹವೂ ಅದನ್ನೇ ಹಿಂಬಾಲಿಸುವುದು. ನನ್ನ ಜೀವ ಎಂದೋ ಹೋಯಿತು. ಅವನ ಜೀವದಲ್ಲಿ ನನ್ನದೂ ಇದೆ. ನೀನು ಜೀವವನ್ನು ಸೆಳೆದರೆ ದೇಹವೂ ಹಿಂಬಾಲಿಸುವುದಿಲ್ಲವೆ?” ಎಂದಳು. “ನಿನ್ನ ಮಾತಿನಿಂದ ನನಗೆ ತೃಪ್ತಿಯಾಯಿತು. ಸಾವಿತ್ರಿ ನೀನು ಮತ್ತೊಂದು ವರವನ್ನು ಕೇಳು. ಆದರೆ ಅದು ಸತ್ಯವಾನನ ಪ್ರಾಣವಾಗಿರಬಾರದು” ಎಂದನು. “ತಂದೆ, ನೀನು ಇನ್ನೊಂದು ವರವನ್ನು ಕರುಣಿಸಬೇಕೆಂದು ಕೃಪೆಮಾಡಿದರೆ ನನ್ನ ಮಾವನ ಕಳೆದುಹೋದ ರಾಜ್ಯ ಮತ್ತು ಐಶ್ವರ್ಯ ಹಿಂತಿರುಗಲಿ” ಎಂದಳು. “ಪ್ರಿಯ ಮಗಳೆ, ಈ ವರವನ್ನು ಕರುಣಿಸಿದೆ. ಆದರೆ ಈಗ ಹಿಂತಿರುಗು. ಬದುಕಿರುವವರು ಯಮನನ್ನು ಹಿಂಬಾಲಿಸಲಾರರು” ಎಂದ. ಯಮನು ಮುಂದೆ ಹೊರಟ. ಆದರೆ ಪತಿವ್ರತಳಾದ ಸಾಧ್ವಿ ಸತಿಸಾವಿತ್ರಿ ತನ್ನ ಪತಿಯನ್ನು ಅನುಸರಿಸಿದಳು. ಯಮ ಹಿಂತಿರುಗಿ ನೋಡಿ, “ಸಾಧ್ವಿ ಸಾವಿತ್ರಿ, ವೃಥಾ ನನ್ನನ್ನು ಹಿಂಬಾಲಿಸಬೇಡ, ಎಂದನು. “ನೀನು ನನ್ನ ಪ್ರಿಯತಮನನ್ನು ಒಯ್ಯುವೆಡೆ ನಾನು ಬರದೇ ವಿಧಿಯಿಲ್ಲ” ಎಂದಳು. “ಸಾವಿತ್ರಿ, ಹಾಗಾದರೆ ನಿನ್ನ ಗಂಡ ದುರಾತ್ಮನಾದರೆ ನರಕಕ್ಕೆ ಹೋಗಬೇಕಾಗುವುದು. ಆಗ ಸಾವಿತ್ರಿ ತಾನು ಪ್ರೀತಿಸುವವನನ್ನೇ ಹಿಂಬಾಲಿಸುವಳೇನು?” ಎಂದು ಯಮ ಕೇಳಿದ. “ಅವನು ಹೋದೆಡೆ ಸಂತೋಷದಿಂದ ಹೋಗುವೆನು, ಜನನವಾಗಲೀ, ಮರಣವಾಗಲೀ, ಸ್ವರ್ಗವಾಗಲೀ, ನರಕವಾಗಲೀ” ಎಂದಳು ಆ ಪ್ರಿಯ ಸತಿ. “ಮಗು, ಧನ್ಯವಾದ ಮಾತುಗಳಿವು, ನಿನ್ನನ್ನು ಕಂಡು ನನಗೆ ತುಂಬಾ ಸಂತೋಷವಾಯಿತು. ಮತ್ತೊಂದು ವರವನ್ನು ಕೇಳು. ಆದರೆ ಮೃತರು ಬದುಕಲಾರರು” ಎಂದ ಯಮ. “ನೀನು ಹಾಗೆ ಮತ್ತೊಂದು ವರವನ್ನು ಕರುಣಿಸುವುದಕ್ಕೆ ಒಪ್ಪಿಕೊಂಡಿರುವುದರಿಂದ ನನ್ನ ಮಾವನ ವಂಶ ನಾಶವಾಗದಿರಲಿ. ಅವನ ರಾಜ್ಯ ಸತ್ಯವಾನನ ಮಕ್ಕಳಿಗೆ ಬರಲಿ” ಎಂದಳು. ಆಗ ಯಮ ಮಂದಹಾಸದಿಂದ “ಮಗಳೆ, ನಿನ್ನ ಇಚ್ಛೆ ಪೂರ್ಣವಾಗಲಿ, ನಿನ್ನ ಪತಿಯ ಪ್ರಾಣವಿಲ್ಲಿದೆ\break ತೆಗೆದುಕೋ, ಅವನು ಪುನಃ ಜೀವಿಸುವನು. ಅವನೊಬ್ಬ ತಂದೆಯಾಗುವನು. ಕಾಲಕ್ರಮೇಣ ನಿಮ್ಮ ಮಕ್ಕಳು ರಾಜ್ಯವಾಳುವರು. ಈಗ ಮನೆಗೆ ಮರಳು. ಪ್ರೇಮ ಮೃತ್ಯುವನ್ನು ಜಯಿಸಿತು. ಯಾವ ಹೆಂಗಸೂ ಪ್ರೀತಿಸಿಲ್ಲ. ನಿಜವಾದ ಪ್ರೇಮಶಕ್ತಿಯ ಎದುರಿಗೆ ನಾನು ಮೃತ್ಯು ರಾಜನಾದವನೂ ಕೂಡ ಸೋತೆನೆಂಬುದಕ್ಕೆ ನೀನೊಂದು ಉದಾಹರಣೆ!” ಎಂದನು.

\vskip 0.1cm

ಇದೇ ಸಾವಿತ್ರಿಯ ಕಥೆ, ಭರತಖಂಡದಲ್ಲಿ ಪ್ರತಿಯೊಬ್ಬ ಹುಡುಗಿಯೂ ಸಾವಿತ್ರಿಯಂತಾಗಬೇಕು. ಮೃತ್ಯು ಅವಳ ಪ್ರೇಮವನ್ನು ಜಯಿಸಲಾರದೆ ಹೋಯಿತು, ಅವಳು ಪ್ರೇಮದ ಅದ್ಭುತ ಶಕ್ತಿಯಿಂದ ಯಮನ ಕೈಗಳಿಂದಲೂ ತನ್ನ ಪತಿಯ ಪ್ರಾಣವನ್ನು\break ತೆಗೆದುಕೊಂಡು ಬಂದಳು.

\vskip 0.1cm

ಕಾವ್ಯದಲ್ಲಿ ಇಂತಹ ನೂರಾರು ಸುಂದರವಾದ ಕಥೆಗಳಿವೆ. ಪ್ರಪಂಚದಲ್ಲಿ ಮಹಾಭಾರತ ಒಂದು ಅತಿದೊಡ್ಡ ಗ್ರಂಥ, ಹದಿನೆಂಟು ಪರ್ವಗಳನ್ನೊಳಗೊಂಡ ಸುಮಾರು ಒಂದು ಲಕ್ಷ ಶ್ಲೋಕಗಳಿವೆ ಅದರಲ್ಲಿ.

\vskip 0.1cm

ನಮ್ಮ ಮೂಲಕಥೆಗೆ ಈಗ ಹಿಂತಿರುಗೋಣ. ಪಾಂಡವರನ್ನು ವನವಾಸದಲ್ಲಿ ಬಿಟ್ಟಿದ್ದೆವು. ಅಲ್ಲಿ ಕೂಡ ದುರ್ಯೋಧನನ ಕುತಂತ್ರಗಳ ದೆಸೆಯಿಂದ ನೆಮ್ಮದಿ ಇರಲಿಲ್ಲ. ಆದರೆ ಕುತಂತ್ರಗಳೆಲ್ಲ ವ್ಯರ್ಥವಾದವು.

\vskip 0.1cm

ಅವರ ವನವಾಸದ ಒಂದು ಕಥೆಯನ್ನು ನಿಮಗೆ ಹೇಳುತ್ತೇನೆ. ಒಂದು ದಿನ ಅವರಿಗೆ ಕಾಡಿನಲ್ಲಿ ಬಾಯಾರಿಕೆಯಾಯಿತು. ಯುಧಿಷ್ಠಿರ ನಕುಲನಿಗೆ ನೀರು ತರುವಂತೆ ಹೇಳಿಕಳುಹಿಸಿದನು. ನಕುಲ ತಕ್ಷಣ ಹೊರಟು ಒಂದು ಸ್ವಚ್ಛವಾದ ಸರೋವರ ಇರುವ ಕಡೆ ಬಂದು ಆ ನೀರನ್ನು ಕುಡಿಯುವುದರಲ್ಲಿದ್ದನು. ಆಗ ಒಂದು ಧ್ವನಿಯು “ಮಗು ಸ್ವಲ್ಪ ತಾಳು, ನಾನು ಕೇಳುವ ಪ್ರಶ್ನೆಗೆ ಉತ್ತರವಿತ್ತು ನೀರು ಕುಡಿ” ಎಂದಿತು. ಆದರೆ ನಕುಲನಿಗೆ ಬಹಳ ಬಾಯಾರಿಕೆಯಾಗಿದ್ದರಿಂದ ಅದನ್ನು ಲಕ್ಷ್ಯ ಮಾಡದೆ ನೀರು ಕುಡಿದು ಸತ್ತು ಬಿದ್ದ. ನಕುಲ ಬಾರದೆ ಇರುವುದನ್ನು ನೋಡಿ ಯುಧಿಷ್ಠಿರ ನಕುಲನನ್ನು ಹುಡುಕಿಕೊಂಡು ನೀರನ್ನು ತರುವಂತೆ ಸಹದೇವನಿಗೆ ಹೇಳಿದನು. ಸಹದೇವ ಸರೋವರದ ಸಮೀಪಕ್ಕೆ ಹೋದಾಗ ನಕುಲ ಸತ್ತು ಬಿದ್ದಿರುವುದನ್ನು ಕಂಡನು. ತಮ್ಮನ ಮರಣದಿಂದ ದುಃಖಿತನಾಗಿ ಬಾಯಾರಿದ್ದುದರಿಂದ ನೀರು ಕುಡಿಯಲು ಹೋದನು. ಆಗ ಪುನಃ ಅದೇ ಧ್ವನಿ ಕೇಳಿಸಿತು. “ಮಗು ಮೊದಲು ನನ್ನ ಪ್ರಶ್ನೆಗೆ ಉತ್ತರಕೊಟ್ಟು ಅನಂತರ ಜಲಪಾನ ಮಾಡು” ಎಂದಿತು. ಅವನು ಇದನ್ನು ಲಕ್ಷ್ಯ ಮಾಡದೆ ನೀರನ್ನು ಕುಡಿದುದರಿಂದ ಸತ್ತು ಬಿದ್ದನು. ಕ್ರಮೇಣ ಅರ್ಜುನ ಭೀಮರನ್ನು ಅದೇ ಕೆಲಸಕ್ಕೆ ಕಳುಹಿಸಿದ, ಅವರೂ ನೀರು ಕುಡಿದು ಸತ್ತು ಬಿದ್ದರು. ಯಾರೂ ಹಿಂತಿರುಗಿ ಬರಲಿಲ್ಲ. ಕೊನೆಗೆ ಆ ರಮ್ಯ ತಟಾಕದ ಸಮೀಪಕ್ಕೆ ಯುಧಿಷ್ಠಿರನೇ ಬಂದಾಗ ತನ್ನ ತಮ್ಮಂದಿರೆಲ್ಲ ಅಲ್ಲಿ ಸತ್ತುಬಿದ್ದಿರುವುದನ್ನು ಕಂಡನು. ಇದನ್ನು ನೋಡಿ ತುಂಬಾ ವ್ಯಥೆಪಟ್ಟನು. ರೋದಿಸಲು ಆರಂಭಿಸಿದನು. ಆಗ ಅದೇ ಧ್ವನಿ ಕೇಳಿಸಿತು. “ಮಗು, ಅವಸರದಿಂದ ವರ್ತಿಸಬೇಡ. ನಾನು ಸಣ್ಣ ಮೀನನ್ನು ತಿನ್ನುತ್ತಿರುವ ಬಕನಂತೆ ಇರುವ ಯಕ್ಷ. ನಿನ್ನ ತಮ್ಮಂದಿರೆಲ್ಲ ಸತ್ತಿದ್ದು ನನ್ನಿಂದ. ನೀನು ಕೂಡ ನಾನು ಕೇಳುವ ಪ್ರಶ್ನೆಗೆ ಉತ್ತರ ಕೊಟ್ಟು ಅನಂತರ ನೀನು ಎಷ್ಟು ಬೇಕಾದರೂ ನೀರನ್ನು ತೆಗೆದುಕೊಂಡು ಹೋಗಬಹುದು” ಎಂದನು “ನಿನ್ನ ಪ್ರಶ್ನೆಗೆ ನನ್ನ ಬುದ್ಧಿಗೆ ತೋರಿದಂತೆ ಉತ್ತರ ಕೊಡುತ್ತೇನೆ. ಪ್ರಶ್ನೆ ಕೇಳು” ಎಂದನು. ಧರ್ಮರಾಜನಿಗೆ ಯಕ್ಷನು ಹಾಕಿದ ಪ್ರಶ್ನೆಗಳಲ್ಲಿ ಒಂದು: “ಪ್ರಪಂಚದಲ್ಲಿ ಯಾವುದು ಅತಿ ವಿಚಿತ್ರವಾಗಿರುವುದು?” ಎಂಬುದು. “ಸುತ್ತಲೂ ನಮ್ಮ ಜನ ಸಾಯುವುದನ್ನು ಪ್ರತಿದಿನ ನೋಡುತ್ತಿರುವೆವು. ಆದರೆ ಉಳಿದವರು ತಾವು ಮಾತ್ರ ಸಾಯುವುದಿಲ್ಲ ಎಂದು ಬಗೆಯುವರು. ಇದೇ ಒಂದು ದೊಡ್ಡ ವಿಚಿತ್ರ. ಸುತ್ತಲೂ ಸಾವನ್ನು ನೋಡುತ್ತಿದ್ದರೂ ತಾವು ಸಾಯುತ್ತೇವೆ ಎಂದು ಭಾವಿಸುವುದೇ ಇಲ್ಲ,” ಎಂದನು ಧರ್ಮರಾಯ. ಹಾಕಿದ ಮತ್ತೊಂದು ಪ್ರಶ್ನೆಯೇ ಧರ್ಮರಹಸ್ಯವನ್ನು ಅರಿಯುವುದು ಹೇಗೆ ಎಂಬುದು. “ವಾದದಿಂದ ಯಾವುದನ್ನೂ ನಿರ್ಧರಿಸಲಾಗುವುದಿಲ್ಲ. ಎಷ್ಟೋ ಸಿದ್ಧಾಂತಗಳಿವೆ, ಎಷ್ಟೋ ಶಾಸ್ತ್ರಗಳಿವೆ. ಒಂದನ್ನೊಂದು ವಿರೋಧಿಸುತ್ತಿರುತ್ತವೆ. ಏಕಾಭಿಪ್ರಾಯವಿರುವ ಇಬ್ಬರು ಕೂಡ ಇಲ್ಲ. ಧರ್ಮದ ರಹಸ್ಯ ಎಲ್ಲೋ ಗಾಢಾಂಧಕಾರದ ಗುಹೆಯಲ್ಲಿ ಹುದುಗಿದಂತೆ ಇದೆ. ಆದಕಾರಣ ಹಿಂದಿನ ಮಹಾಪುರುಷರು ಯಾವ\break ಮಾರ್ಗದಲ್ಲಿ ಹೋದರೋ ಅದನ್ನು ಅನುಸರಿಸುವುದೇ ಸರಿಯಾದ ಮಾರ್ಗ” ಎಂದನು ಧರ್ಮರಾಯ. ಆಗ ಯಕ್ಷ ಹೇಳಿದ “ನಾನು ಸಂತುಷ್ಟನಾದೆ. ಬಕನಂತೆ ಇರುವ ಧರ್ಮವೇ ನಾನು. ನಿನ್ನನ್ನು ಪರೀಕ್ಷಿಸುವುದಕ್ಕೆ ಬಂದೆ. ಈಗ ನೋಡು ನಿನ್ನ ತಮ್ಮಂದಿರು ಯಾರೂ ಸತ್ತಿಲ್ಲ. ಇದೆಲ್ಲ ನನ್ನ ಮಾಯೆ. ಸುಖಕ್ಕಿಂತ, ಲಾಭಕ್ಕಿಂತ, ಅಹಿಂಸೆಯೇ ಪರಮಧರ್ಮವೆಂದು ಸಾರಿದುದರಿಂದ, ಹೇ ಭರತರ್ಷಭಾ ನಿನ್ನ ತಮ್ಮಂದಿರೆಲ್ಲ ಬದುಕಲಿ.” ಎಂದಿತು ಧರ್ಮ, ಯಕ್ಷ ಹೀಗೆ ಹೇಳಿದ ಮೇಲೆ ಪಾಂಡವರೆಲ್ಲ ಎಚ್ಚೆತ್ತರು.

\vskip 0.1cm

ಯುಧಿಷ್ಠಿರನ ಸ್ವಭಾವದ ಒಂದು ಉದಾಹರಣೆ ಇದು. ಅವನ ಉತ್ತರದಿಂದ\break ಅವನು ರಾಜನಿಗಿಂತ ಹೆಚ್ಚಾಗಿ ಒಬ್ಬ ಜ್ಞಾನಿಯಾಗಿದ್ದ, ಯೋಗಿಯಾಗಿದ್ದ ಎಂಬುದು ಗೊತ್ತಾಗುವುದು. ಅವರ ವನವಾಸದ ಹನ್ನೆರಡನೆಯ ವರುಷ ಪೂರೈಸುತ್ತ ಬರಲು\break ಯಕ್ಷ ವಿರಾಟನ ದೇಶದಲ್ಲಿ ತಮಗೆ ತೋರಿದ ರೀತಿ ಅಜ್ಞಾತವಾಸವನ್ನು ಕಳೆಯಿರಿ ಎಂದು ಹೇಳಿದನು.

\vskip 0.1cm

ಹನ್ನೆರಡು ವರುಷ ವನವಾಸವಾದ ಮೇಲೆ ಉಳಿದ ಒಂದು ವರುಷ ಅಜ್ಞಾತ\-ವಾಸವನ್ನು ಕಳೆಯಲು ವಿರಾಟನಗರಿಗೆ ಬೇರೆ ಬೇರೆ ವೇಷದಲ್ಲಿ ಹೋದರು. ರಾಜನ ಅರಮನೆಯಲ್ಲಿ ಸೇವಕರಾಗಿ ನಿಂತರು. ಯುಧಿಷ್ಠಿರನು ಪಗಡೆಯಾಟದಲ್ಲಿ ನಿಪುಣನಾದ ವಿರಾಟನ ಬಳಿ ಬ್ರಾಹ್ಮಣ ಆಸ್ಥಾನಿಕನಾದನು. ಭೀಮ ಅಡುಗೆಯವನಾದ. ಅರ್ಜುನ ಬೃಹನ್ನಳೆಯಾಗಿ ರಾಜಕುಮಾರಿ ಉತ್ತರೆಗೆ ನೃತ್ಯ ಮತ್ತು ಸಂಗೀತ ಶಾಸ್ತ್ರವನ್ನು ಕಲಿಸುತ್ತ ಅಂತಃಪುರದಲ್ಲಿದ್ದನು. ನಕುಲನು ರಾಜನ ಕುದುರೆಯನ್ನು ನೋಡಿಕೊಳ್ಳುವವನಾದ. ಸಹದೇವ ಅವನ ಗೋವುಗಳನ್ನು ಕಾಯತೊಡಗಿದ. ದ್ರೌಪದಿ ಒಬ್ಬ ಪರಿಚಾರಿಕೆಯಂತೆ ರಾಣಿಯ ಊಳಿಗಕ್ಕೆ ಸೇರಿದಳು. ಹೀಗೆ ತಮ್ಮ ವ್ಯಕ್ತಿತ್ವವನ್ನು ಮರೆಮಾಚಿಕೊಂಡು ಒಂದು ವರುಷ ಅಜ್ಞಾತವಾಸವನ್ನು ಜಯಪ್ರದವಾಗಿ ಕಳೆದರು. ಅವರನ್ನು ಕಂಡುಹಿಡಿಯಬೇಕೆಂಬ ದುರ್ಯೋಧನನ ಪ್ರಯತ್ನ ವಿಫಲವಾಯಿತು. ಸರಿಯಾಗಿ ಒಂದು ವರುಷ ಕಳೆದ ಮೇಲೆ ಮಾತ್ರ ಅವರನ್ನು ಕಂಡು\-ಹಿಡಿದರು.

\vskip 0.1cm

ಅನಂತರ ಯುಧಿಷ್ಠಿರನು ಒಬ್ಬ ರಾಯಭಾರಿಯನ್ನು ದುರ್ಯೋಧನನ ಬಳಿಗೆ\break ಕಳುಹಿಸಿ ತಮಗೆ ಸಲ್ಲಬೇಕಾದ ಅರ್ಧ ರಾಜ್ಯವನ್ನು ಕೊಡಬೇಕೆಂದು ಕೇಳಿದನು. ಆದರೆ ದುರ್ಯೋಧನನಿಗೆ ತನ್ನ ದಾಯಾದಿಗಳನ್ನು ಕಂಡರೆ ಆಗುತ್ತಿರಲಿಲ್ಲ. ಅವರ ನ್ಯಾಯವಾದ ಬೇಡಿಕೆಯನ್ನು ಆಲಿಸಲಿಲ್ಲ. ಪಾಂಡವರು ತಮಗೆ ಒಂದು ನಾಡನ್ನೋ ಅಥವಾ ಐದು ಹಳ್ಳಿಗಳನ್ನಾದರೂ ಕೊಟ್ಟಿದ್ದರೆ ಒಪ್ಪಿಕೊಳ್ಳುತ್ತಿದ್ದರು. ಆದರೆ ಭಂಡನಾದ ದುರ್ಯೋಧನನು ಒಂದು ಸೂಜಿಮೊನೆಯಷ್ಟು ಜಾಗವನ್ನಾದರೂ ಯುದ್ಧವಿಲ್ಲದೆ ಬಿಟ್ಟುಕೊಡುವುದಿಲ್ಲ\break ಎಂದನು. ಧೃತರಾಷ್ಟ್ರ ಶಾಂತಿಗಾಗಿ ಮತ್ತೆ ಮತ್ತೆ ಪ್ರಯತ್ನಿಸಿದನು. ಆದರೂ ಕೌರವರು ಒಪ್ಪಲಿಲ್ಲ. ಕೃಷ್ಣನೂ ಕೂಡ ಹೋಗಿ ಮುಂದೊದಗುವ ಯುದ್ಧವನ್ನು ನಿಲ್ಲಿಸುವುದಕ್ಕೆ, ಬಂಧುಹತ್ಯೆಯನ್ನು ತಡೆಗಟ್ಟುವುದಕ್ಕೆ ಪ್ರಯತ್ನಿಸಿದ. ಆಸ್ಥಾನದ ಧರ್ಮಜ್ಞರಾದ ಹಿರಿಯರೆಲ್ಲ ಯತ್ನಿಸಿದರು. ಶಾಂತಿಯಿಂದ ರಾಜ್ಯವನ್ನು ಭಾಗ ಮಾಡಿಕೊಡುವ ಪ್ರಯತ್ನಗಳೆಲ್ಲ ನಿಷ್ಫಲವಾದುವು. ಕೊನೆಗೆ ಇಬ್ಬರೂ ಯುದ್ಧಕ್ಕೆ ಸಿದ್ಧರಾದರು, ಎಲ್ಲಾ ಕ್ಷತ್ರಿಯ ರಾಜರೂ ಯುದ್ಧದಲ್ಲಿ ಭಾಗವಹಿಸಿದರು.

\vskip 0.1cm

ಆಗಿನ ಕಾಲದಲ್ಲಿ ಕ್ಷತ್ರಿಯರಲ್ಲಿ ಬಳಕೆಯಲ್ಲಿದ್ದ ರೂಢಿಯನ್ನೇ ಅನುಸರಿಸಿದರು. ದುರ್ಯೋಧನ ಒಂದು ಪಕ್ಷ, ಯುಧಿಷ್ಠಿರ ಒಂದು ಪಕ್ಷ ವಹಿಸಿಕೊಂಡರು. ನೆರೆಯ\break ರಾಷ್ಟ್ರಗಳಿಗೆ ಯುಧಿಷ್ಠಿರ ಅವರ ಸಹಾಯವನ್ನು ಕೋರಿ ದೂತರನ್ನು ಕಳುಹಿಸಿದ.\break ಗೌರವಸ್ಥರು ತಮ್ಮನ್ನು ಯಾರು ಮೊದಲು ಕೋರುವರೋ ಅವರಿಗೆ ಸಹಾಯ\break ನೀಡುವರು. ಅವರವರ ಕೋರಿಕೆಗೆ ಅನುಸಾರವಾಗಿ ಕೆಲವರು ಕೌರವರ ಮತ್ತೆ\break ಪಾಂಡವರ ಪಕ್ಷವನ್ನು ವಹಿಸಿಕೊಂಡು ದೇಶದ ನಾನಾ ಕಡೆಗಳಿಂದ ಯೋಧರು\break ನೆರೆದರು. ಇದರಲ್ಲಿ ತಮ್ಮ ಒಂದು ಪಕ್ಷಕ್ಕೆ ಸೇರಿದ್ದರು. ಅಣ್ಣ ಒಂದು ಪಕ್ಷಕ್ಕೆ\break ಸೇರಿದ್ದರು. ತಂದೆ ಒಂದು ಪಕ್ಷಕ್ಕೆ, ಮಗ ಒಂದು ಪಕ್ಷಕ್ಕೆ ಸೇರಿದ್ದರು. ಆಗಿನ ಕಾಲದ ಯುದ್ಧನೀತಿ ವಿಚಿತ್ರವಾದುದು. ದಿನದ ಯುದ್ಧ ಕಳೆದು ಸಂಜೆಯಾದೊಡನೆ ವಿರೋಧ\break ಪಕ್ಷದವರು ಸ್ನೇಹಿತರಾಗುತ್ತಿದ್ದರು, ಪರಸ್ಪರರ ಗುಡಾರಗಳಿಗೆ ಹೋಗಿ ಬಂದು\break ಮಾಡುತ್ತಿದ್ದರು. ಬೆಳಗಾದ ಮೇಲೆ ಮತ್ತೆ ಪರಸ್ಪರ ಯುದ್ಧದಲ್ಲಿ ತೊಡಗುತ್ತಿದ್ದರು.\break ಮಹಮ್ಮದೀಯರ ದಾಳಿಯವರೆವಿಗೂ ಹಿಂದುಗಳಲ್ಲಿದ್ದ ಒಂದು ವಿಚಿತ್ರ ಯುದ್ಧ\break ಪದ್ಧತಿ ಅದು. ಅಶ್ವಾರೂಢನಾದವನು ಪದಾತಿಯನ್ನು ಕೊಲ್ಲಕೂಡದು, ಅಸ್ತ್ರಗಳಿಗೆ ವಿಷ ಮಿಶ್ರ ಮಾಡಕೂಡದು. ಅಸಮಾನಸ್ಥರೊಂದಿಗೆ ಯುದ್ಧ ಮಾಡಿ ಅಥವಾ ಅನ್ಯಾಯದಿಂದ\break ವೈರಿಗಳನ್ನು ಸೋಲಿಸಕೂಡದು. ಒಬ್ಬನು ಮತ್ತೊಬ್ಬನು ಅಸಹಾಯ ಸ್ಥಿತಿಯಲ್ಲಿದ್ದಾಗ\break ಅವನೊಂದಿಗೆ ಯುದ್ಧ ಮಾಡಕೂಡದು. ಹೀಗೆ ಎಷ್ಟೋ ವಿಧಿನಿಯಮಗಳಿದ್ದವು.\break ಯಾರಾದರೂ ಈ ವಿಧಿನಿಯಮಗಳನ್ನು ಮೀರಿದರೆ ಅವನನ್ನು ತುಚ್ಛವಾಗಿ ಕಾಣುತ್ತಿದ್ದರು. ಧಿಕ್ಕರಿಸುತ್ತಿದರು. ಕ್ಷಾತ್ರಿಯರನ್ನು ಈ ರೀತಿಯಲ್ಲಿ ತರಪೇತು ಮಾಡಿದ್ದರು. ಮಧ್ಯ ಏಷ್ಯಾದಿಂದ ಹೊರಗಿನವರು ಧಾಳಿ ಇಟ್ಟಾಗ ಹಿಂದುಗಳು ಇದೇ ರೀತಿ ಅವರನ್ನು ಕಂಡರು.\break ಅವರನ್ನು ಎಷ್ಟೋ ವೇಳೆ ಸೋಲಿಸಿದರು. ಆದರೆ ಪ್ರತಿಸಲವೂ ಅವರಿಗೆ ಬಹುಮಾನಗಳನ್ನು ಕೊಟ್ಟು ಮನೆಗೆ ಕಳುಹಿಸಿದರು. ಮತ್ತೊಬ್ಬರ ದೇಶವನ್ನು ಅಪಹರಿಸಬಾರದು ಎಂಬುದೇ ಪದ್ಧತಿಯಲ್ಲಿದ್ದ ನೀತಿ. ಒಬ್ಬನನ್ನು ಸೋಲಿಸಿದರೆ ಅವನ ಸ್ಥಾನಕ್ಕೆ ಸಲ್ಲುವ ಗೌರವ ತೋರಿ\break ಅವನನ್ನು ಹಿಂದಕ್ಕೆ ಕಳುಹಿಸಬೇಕಾಗಿತ್ತು. ಆದರೆ ಮಹಮ್ಮದೀಯ ಆಕ್ರಮಣಕಾರರು ಬೇರೆ ವಿಧದಿಂದ ಹಿಂದುಗಳನ್ನು ಕಂಡರು. ಒಂದು ಸಲ ಅವರನ್ನು ಸೋಲಿಸಿದರೆ ನಿಷ್ಕರುಣೆಯಿಂದ ವೈರಿಗಳನ್ನು ನಾಶ ಮಾಡುತ್ತಿದ್ದರು.

\vskip 0.1cm

ಇದನ್ನು ಲಕ್ಷ್ಯದಲ್ಲಿಡಿ. ಈ ನಮ್ಮ ಕಥೆ ನಡೆಯುತ್ತಿದ್ದ ಕಾಲದಲ್ಲಿ ಶಸ್ತ್ರವಿದ್ಯೆ ಎಂದರೆ ಬರೀ ಬಿಲ್ಲುಬಾಣಗಳ ಉಪಯೋಗವಲ್ಲ. ಅದು ಮಂತ್ರ ವಿದ್ಯೆಯೂ ಆಗಿತ್ತು. ಅಲ್ಲಿ ಮಂತ್ರ ಮತ್ತು ಏಕಾಗ್ರತೆ ಬಹುಮುಖ್ಯ ಎಂಬುದನ್ನು ಕಾವ್ಯ ಹೇಳುವುದು. ಒಬ್ಬನು ಲಕ್ಷಾಂತರ ಜನರೊಂದಿಗೆ ಯುದ್ಧ ಮಾಡಿ ಅವರನ್ನು ದಹಿಸಬಹುದಾಗಿತ್ತು. ಅವನು ಒಂದು ಬಾಣವನ್ನು ಕಳುಹಿಸಿದರೆ ಅದೊಂದು ಬಾಣದ ಮಳೆಯಾಗಿ ಸಿಡಿಲು ಗುಡುಗುಗಳನ್ನು ಕಾರಿ ಬೀಳುತ್ತಿತ್ತು. ಅವನು ಯಾವುದನ್ನು ಬೇಕಾದರೂ ದಹಿಸಬಹುದಾಗಿತ್ತು. ಇವೆಲ್ಲ ದೈವದತ್ತ ಮಂತ್ರವಿದ್ಯೆಯಾಗಿತ್ತು. ರಾಮಾಯಣ ಮಹಾಭಾರತಗಳೆರಡರಲ್ಲಿಯೂ ಮಂತ್ರಾಸ್ತ್ರಗಳ ಜೊತೆಗೆ ಫಿರಂಗಿಗಳು ಬಳಕೆಯಲ್ಲಿದ್ದವು ಎಂಬುದನ್ನು ಕಾವ್ಯದ ಮೂಲಕ ತಿಳಿಯುವೆವು. ಈ ಫಿರಂಗಿ ಬಹಳ ಪುರಾತನವಾದುದು. ಚೀನೀಯರು ಮತ್ತು ಹಿಂದುಗಳು ಇದನ್ನು\break ಬಳಕೆಗೆ ತಂದಿದ್ದರು. ನಗರದ ಕೋಟೆ ಗೋಡೆಗಳ ಮೇಲೆ ಟೊಳ್ಳು ಕಬ್ಬಿಣದ ಕೊಳವಿಯಿಂದ ಮಾಡಿರುವ ನೂರಾರು ವಿಚಿತ್ರ ಅಸ್ತ್ರಗಳಿದ್ದುವು. ಅದರಲ್ಲಿ ಮದ್ದು ಮತ್ತು ಗುಂಡು ತುಂಬಿ ಅನೇಕ ಜನರನ್ನು ಕೊಲ್ಲುತ್ತಿದ್ದರು. ಜನರು, ಚೀನೀಯರು ಮಂತ್ರದಿಂದ ಕೊಳವಿಯಲ್ಲಿ ಒಂದು ಭೂತವನ್ನು ಕೂಡಿಡುತ್ತಿದ್ದರು. ಅದಕ್ಕೆ ಒಂದು ಕಿಡಿ ತಾಕಿದೊಡನೆಯೆ ಅದು\break ಗರ್ಜನೆ ಮಾಡುತ್ತ ಕೊಳವಿಯಿಂದ ಹೊರಗೆ ಬಂದು ನೂರಾರು ಜನರನ್ನು ಕೊಲ್ಲುತ್ತಿತ್ತು ಎಂದು ನಂಬಿದ್ದರು.

\vskip 0.1cm

ಅಂದಿನ ಕಾಲದಲ್ಲಿ ಯೋಧರು ಮಂತ್ರಾಸ್ತ್ರಗಳಿಂದ ಯುದ್ಧ ಮಾಡುತ್ತಿದ್ದರು.\break ಒಬ್ಬ ಲಕ್ಷಾಂತರ ಜನರೊಂದಿಗೆ ಬೇಕಾದರೆ ಯುದ್ಧ ಮಾಡಬಹುದಾಗಿತ್ತು. ಅವರಲ್ಲಿಯೂ ಸಮರ ವ್ಯೂಹ ಮತ್ತು ತಂತ್ರಗಳಿದ್ದವು. ಅವರಲ್ಲಿಯೂ ಪದಾತಿಗಳು, ಅಶ್ವಾರೂಢರು ಇದ್ದರು. ಇನ್ನೆರಡು ಭಾಗಗಳನ್ನು ಆಧುನಿಕರು ಮರೆತು ತ್ಯಜಿಸಿದ್ದಾರೆ. ಅವೇ ಗಜ\break ಸಮೂಹ. ನೂರಾರು ಗಜಗಳಿದ್ದುವು. ಅವುಗಳ ಮೇಲೆ ಉಕ್ಕಿನ ಕವಚಧಾರಿಗಳಾದ ಯೋಧರು ಇರುತ್ತಿದ್ದರು. ಈ ಆನೆಗಳು ವೈರಿ ಸಮೂಹದ ಮೇಲೆ ನುಗ್ಗುತ್ತಿದ್ದುವು.\break ಅನಂತರ ರಥಗಳಿದ್ದುವು. (ಇಂತಹ ಹಳೆಯ ರಥಗಳ ಚಿತ್ರಗಳನ್ನು ನೀವೆಲ್ಲರೂ\break ನೋಡಿರುವಿರಿ. ಎಲ್ಲಾ ದೇಶಗಳ್ಲಿಯೂ ಅವು ಬಳಕೆಯಲ್ಲಿದ್ದವು.) ಆಗಿನ ಕಾಲದ ಸೇನೆಯ ನಾಲ್ಕು ವಿಭಾಗಗಳು ಇವು.

\vskip 0.1cm

ಉಭಯ ಪಕ್ಷದವರೂ ಕೃಷ್ಣನ ಸಹಾಯವನ್ನು ಪಡೆಯಲು ಇಚ್ಛಿಸಿದರು. ಆದರೆ ಕೃಷ್ಣನು ಯುದ್ಧದಲ್ಲಿ ಭಾಗಿಯಾಗಲು ಇಚ್ಛಿಸಲಿಲ್ಲ. ಪಾಂಡವರ ಸಲಹೆಗಾರ ಮತ್ತು\break ಸ್ನೇಹಿತನಾಗಿ ಅರ್ಜುನನ ಸಾರಥಿಯಾಗಲು ಒಪ್ಪಿಕೊಂಡನು. ದುರ್ಯೋಧನನಿಗೆ ತನ್ನ ಶ್ರೇಷ್ಠ ಯೋಧರಿಂದ ಒಡಗೂಡಿದ ಸೈನ್ಯವನ್ನು ಕೊಟ್ಟನು.

\vskip 0.1cm

ಅನಂತರ ಕುರುಕ್ಷೇತ್ರದ ವಿಶಾಲ ಸಮರಾಂಗಣದಲ್ಲಿ ಯುದ್ಧ ನಡೆಯಿತು. ಭೀಷ್ಮ, ದ್ರೋಣ, ಕರ್ಣ, ಸುಯೋಧನ ಮತ್ತು ಅವನ ಸಹೋದರರು, ಉಭಯ ಪಕ್ಷದ\break ಬಂಧುಗಳು ಮತ್ತು ಹಲವು ಯೋಧರು ಮಡಿದರು. ಯುದ್ಧ ಹದಿನೆಂಟು ದಿನ\break ಜರುಗಿತ್ತು. ನೆರೆದ ಹದಿನೆಂಟು ಅಕ್ಷೋಹಿಣಿ ಸೈನ್ಯದಲ್ಲಿ ಎಲ್ಲೋ ಕೆಲವರು ಮಾತ್ರ\break ಉಳಿದರು. ದುರ್ಯೋಧನನ ಮರಣದಿಂದ ಪಾಂಡವರ ಪರವಾಗಿ ಯುದ್ಧ ಕೊನೆಗೊಂಡಿತು. ಅನಂತರ ಗಾಂಧಾರಿ, ರಾಣಿ ಮತ್ತು ಇತರ ವಿಧವೆಯರ ಶೋಕ\break ಮೊದಲಾಯಿತು. ಮಡಿದ ಯೋಧರ ಅಂತ್ಯಕ್ರಿಯೆಯಾಯಿತು.

\vskip 0.1cm

ಯುದ್ಧದಲ್ಲಿ ನಡೆದ ದೊಡ್ಡ ಘಟನೆಯೆ ಅದ್ಭುತವಾದ ಅಮರ ಕಾವ್ಯ ಭಗವದ್ಗೀತೆಯ\break ಉದಯ. ಇದು ಭಾರತದ ಜನಾದರಣೀಯವಾದ ಗ್ರಂಥ, ಶ್ರೇಷ್ಠತಮವಾದ ಬೋಧನೆ, ಕುರುಕ್ಷೇತ್ರದಲ್ಲಿ ಯುದ್ಧಾರಂಭಕ್ಕೆ ಮುಂಚೆ ಕೃಷ್ಣಾರ್ಜುನರಿಗೆ ನಡೆದ ಸಂಭಾಷಣೆ\break ಇದರಲ್ಲಿದೆ. ಯಾರು ಅದನ್ನು ಓದಿಲ್ಲವೋ ಅಮರರಿಗೆ ಅದನ್ನು ಓದುವಂತೆ ಹೇಳುತ್ತೇನೆ. ನಿಮ್ಮ ದೇಶದ ಜನರ ಮೇಲೆ ಅದು ಎಷ್ಟೊಂದು ಪ್ರಭಾವವನ್ನುಂಟುಮಾಡಿದೆ ಎಂಬುದು ನಿಮಗೆ ಗೊತ್ತಾಗಬೇಕಾಗಿದ್ದರೆ ಅದನ್ನು ಓದಬೇಕು. ಎಮರ್​ಸನ್​ ಸ್ಫೂರ್ತಿಯ ಮೂಲ ಈ ಭಗವದ್ಗೀತೆ. ಎಮರ್ಸನ್​ ಕಾರ್ಲೈಲ್ಸನ್ನು ನೋಡಲು ಹೋದಾಗ ಕಾರ್ಲೈಲ್​ ಗೀತೆಯನ್ನು ಬಹುಮಾನವಾಗಿ ಕೊಟ್ಟನು. ಆ ಸಣ್ಣ ಪುಸ್ತಕವೇ ‘ಕನ್​ಕಾರ್ಡಿ’ ನ ಚಳುವಳಿಗೆ ಕಾರಣ.\break ಅಮೆರಿಕಾ ದೇಶದ ಉದಾರ ಸ್ವಭಾವದ ಚಳುವಳಿಗಳು ಒಂದಲ್ಲ ಒಂದು ರೀತಿಯಲ್ಲಿ ಕನ್​ಕಾರ್ಡ್​ ಪಾರ್ಟಿಗೆ ಋಣಿ.

\vskip 0.1cm

ಗೀತೆಯ ಮುಖ್ಯ ವ್ಯಕ್ತಿ ಕೃಷ್ಣ. ನೀವು ನಜರೇತಿನ ಏಸುವನ್ನು ಹೇಗೆ ಭಗವಂತನ\break ಅವತಾರವೆಂದು ಪೂಜಿಸುತ್ತೀರೋ ಹಾಗೆಯೇ ಹಿಂದೂಗಳು ಹಲವು ಭಗವಂತನ\break ಅವತಾರಗಳನ್ನು ಪೂಜಿಸುತ್ತಾರೆ. ಹಿಂದೂಗಳು ಎಲ್ಲೋ ದೇವರ ಒಂದೆರಡು ಅವತಾರದಲ್ಲಿ ಮಾತ್ರ ನಂಬುವುದಿಲ್ಲ. ಪ್ರಪಂಚದ ಅವಶ್ಯಕತೆಗೆ ತಕ್ಕಂತೆ ಧರ್ಮಸಂಸ್ಥಾಪನೆಗೆ,\break ಅಧರ್ಮನಾಶಕ್ಕೆ ಜಗತ್ತಿಗೆ ಬಂದ ಹಲವು ಅವತಾರಗಳನ್ನು ನಂಬುವರು. ಪ್ರತಿಯೊಂದು ಪಂಗಡದವರಿಗೂ ಒಂದೊಂದು ಅವತಾರವಿದೆ. ಕೃಷ್ಣ ಅವರಲ್ಲಿ ಒಬ್ಬ. ಭಾರತದಲ್ಲಿ\break ಕೃಷ್ಣನಿಗೆ ಉಳಿದೆಲ್ಲ ಅವತಾರಗಳಿಗಿಂತ ಹೆಚ್ಚಿನ ಸಂಖ್ಯೆಯ ಅನುಯಾಯಿಗಳಿದ್ದಾರೆ.\break ಭರತಖಂಡದಲ್ಲಿ ಶ‍್ರೀಕೃಷ್ಣನ ಅನುಯಾಯಿಗಳು ಭಗವಂತನ ಪೂರ್ಣಾವತಾರ\break ಕೃಷ್ಣ ಎಂದು ಭಾವಿಸುವರು. ಏತಕ್ಕೆ? ಎಂದರೆ ಬುದ್ಧ ಮುಂತಾದವರನ್ನು ನೋಡಿ,\break ಅವರೆಲ್ಲಾ ಸಂನ್ಯಾಸಿಗಳಾಗಿದ್ದರು. ಅವರಿಗೆ ಗೃಹಸ್ಥರ ವಿಷಯದಲ್ಲಿ ಕರುಣೆಯಿರಲಿಲ್ಲ.\break ಅವರಿಗೆ ಕರುಣೆ ಹೇಗೆ ಇರಬಲ್ಲದು? ಆದರೆ ಕೃಷ್ಣನನ್ನು ನೋಡಿ, ಅವನು ಆದರ್ಶ\break ಮಗನಾಗಿದ್ದ, ರಾಜನಾಗಿದ್ದ, ತಂದೆಯಾಗಿದ್ದ, ಅವನು ಇದೇ ಜೀವನದಲ್ಲಿ ಏನನ್ನು ಬೋಧಿಸಿದನೊ “ಯಾರು ಕರ್ಮದಲ್ಲಿ ಅಕರ್ಮವನ್ನು ನೋಡುವರೋ, ಅಕರ್ಮದಲ್ಲಿ ಕರ್ಮವನ್ನು ನೋಡುವರೋ ಅವರಿಗೆ ಜೀವನ ರಹಸ್ಯ ಗೊತ್ತಿದೆ,” ಎಂಬುದಕ್ಕೆ ಜಾಜ್ವಲ್ಯಮಾನವಾದ ಉದಾಹರಣೆಯಾಗಿರುವನು. ಕೃಷ್ಣ ಅನಾಸಕ್ತನಾಗಿ ಇದನ್ನು ಹೇಗೆ ಮಾಡುವುದು ಎಂಬುದಕ್ಕೆ ಮಾರ್ಗದರ್ಶಕನಾಗಿರುವನು. ಎಲ್ಲವನ್ನೂ ಮಾಡು. ಆದರೆ ಯಾವುದರಲ್ಲೂ ಆಸಕ್ತನಾಗಬೇಡ. ನೀನು ಆತ್ಮ, ಪರಿಶುದ್ಧ, ನಿತ್ಯಮುಕ್ತ, ಸಾಕ್ಷಿ, ಕರ್ಮದಿಂದಲ್ಲ ನಮಗೆ ದುಃಖಪ್ರಾಪ್ತಿ, ಅದರಲ್ಲಿ ಆಸಕ್ತರಾಗಿರುವುದರಿಂದ, ಉದಾಹರಣೆಗೆ ಹಣವನ್ನು ತೆಗೆದುಕೊಳ್ಳಿ, ಹಣವಿರುವುದು ಒಳ್ಳೆಯದು. ಅದನ್ನು ಸಂಪಾದಿಸಿ, ಅದಕ್ಕಾಗಿ ಕಷ್ಟಪಡಿ\break ಎನ್ನುವನು ಕೃಷ್ಣ; ಆದರೆ ಆಸಕ್ತರಾಗಬೇಡಿ ಎನ್ನುವನು. ಇದರಂತೆಯೇ ಮಕ್ಕಳು, ಗಂಡ ಹೆಂಡತಿ, ಬಂಧುಬಳಗ, ಕೀರ್ತಿ ಮುಂತಾದುವು ಕೂಡ. ಅವನ್ನು ನೀವು ತ್ಯಜಿಸಬೇಕಾಗಿಲ್ಲ. ಆದರೆ ಅದರಲ್ಲಿ ಆಸಕ್ತರಾಗಬೇಡಿ. ಒಬ್ಬರಲ್ಲಿ ಮಾತ್ರ ಆಸಕ್ತರಾಗಿರಬೇಕು. ಅದು ದೇವರಲ್ಲಿ. ಅನ್ಯರಲ್ಲಲ್ಲ. ಅವರಿಗಾಗಿ ಕೆಲಸಮಾಡಿ, ಅವರಿಗೆ ಒಳ್ಳೆಯದನ್ನು ಮಾಡಿ, ಅವರನ್ನು ಪ್ರೀತಿಸಿ, ಅವರಿಗಾಗಿ ಸಾಧ್ಯವಾದರೆ ನೂರು ಜನ್ಮಗಳನ್ನಾದರೂ ಅರ್ಪಿಸಿ, ಆದರೆ ಎಂದಿಗೂ\break ಆಸಕ್ತರಾಗಬೇಡಿ. ಅವನ ಜೀವನವೇ ಅದಕ್ಕೆ ಉತ್ತಮ ಉದಾಹರಣೆಯಾಗಿತ್ತು.

\vskip 0.2cm

ಕೃಷ್ಣನ ಜೀವನವನ್ನು ವಿವರಿಸುವ ಗ್ರಂಥ ಹಲವು ಸಾವಿರ ವರುಷಗಳಷ್ಟು ಪುರಾತನ ಎಂಬುದನ್ನು ನೆನಪಿನಲ್ಲಿಡಿ. ಅದರಲ್ಲಿ ಕೆಲವು ಭಾಗ ನಜರೇತಿನ ಏಸುವಿನ ಜೀವನದಂತೆಯೆ ಇದೆ. ಕೃಷ್ಣ ರಾಜವಂಶಿ. ಕಂಸನೆಂಬ ಕ್ರೂರ ದೊರೆಯಿದ್ದ. ಇಂಥದೊಂದು ವಂಶದಲ್ಲಿ\break ಹುಟ್ಟುವವನು ಮುಂದೆ ರಾಜನಾಗುವನೆಂಬ ಭವಿಷ್ಯವಿತ್ತು. ಆದಕಾರಣ ಎಲ್ಲಾ ಗಂಡು ಮಕ್ಕಳನ್ನು ಕೊಲ್ಲುವಂತೆ ಕಂಸ ಆಜ್ಞೆ ಮಾಡಿದನು. ಕೃಷ್ಣನ ತಾಯಿತಂದೆಗಳನ್ನು ಕಂಸ\break ಸೆರೆಮನೆಗೆ ನೂಕಿದ. ಅವರಿಗೊಂದು ಮಗು ಜನಿಸಿತು. ಸೆರೆಮನೆಯಲ್ಲಿ ಒಂದು\break ಜ್ಯೋತಿ ಪ್ರಕಾಶಿಸಿತು. ಮಗು ಹೇಳಿತು “ನಾನು ಜಗದ ಬೆಳಕು. ಜಗದ ಉದ್ಧಾರಕ್ಕಾಗಿ ಜನಿಸಿರುವೆನು” ಎಂದು. ಕೃಷ್ಣನನ್ನು ಸಾಂಕೇತಿಕವಾಗಿ ಗೋಪಾಲ ಎಂದು ಚಿತ್ರಿಸಿರುವರು. ದೇವರೇ ಜನಿಸಿದ ಎಂದು ಋಷಿಗಳು ಹೇಳಿ ಅವನನ್ನು ಗೌರವಿಸುವುದಕ್ಕೆ ಹೋದರು. ಇತರ ಭಾಗಗಳಲ್ಲಿ ಕ್ರಿಸ್ತ ಮತ್ತು ಕೃಷ್ಣನ ಜೀವನಗಳಲ್ಲಿ ಹೋಲಿಕೆಯಿಲ್ಲ.

\vskip 0.2cm

ಕೃಷ್ಣ ಕ್ರೂರಿ ಕಂಸನನ್ನು ಗೆದ್ದನು. ಆದರೆ ಸಿಂಹಾಸನವನ್ನು ತಾನು ಪಡೆಯುವ ಯೋಚನೆ ಮಾಡಲಿಲ್ಲ. ಅದನ್ನು ಸ್ವೀಕರಿಸಲೂ ಇಲ್ಲ. ಅದಕ್ಕೂ ಇವನಿಗೂ ಏನೂ ಸಂಬಂಧ\-ವಿರಲಿಲ್ಲ. ಅವನು ತನ್ನ ಕರ್ತವ್ಯವನ್ನು ಮಾಡಿದ. ಅಲ್ಲಿಗೇ ಅದು ಕೊನೆಯಾಯಿತು.

\vskip 0.2cm

ಕುರುಕ್ಷೇತ್ರ ಯುದ್ಧವಾದ ಮೇಲೆ ಹದಿನೆಂಟು ದಿನಗಳಲ್ಲಿ ಹತ್ತು ದಿನ ಯುದ್ಧ ಮಾಡಿದ ಪ್ರಖ್ಯಾತ ಯೋಧ ಪೂಜ್ಯ ಪಿತಾಮಹ ಭೀಷ್ಮಾಚಾರ್ಯರು ಶರಪಂಜರದಲ್ಲಿ ಮಲಗಿರುವಾಗ ಯುದಿಷ್ಠಿರನಿಗೆ ಕರ್ತವ್ಯ, ನಾಲ್ಕು ವರ್ಣಗಳು ಮತ್ತು ಆಶ್ರಮಗಳು ಮದುವೆ ಮತ್ತು ದಾನಕ್ಕೆ ಸಂಬಂಧಪಟ್ಟ ವಿಷಯಗಳನ್ನು ಹಿಂದಿನ ಋಷಿಗಳ ಬೋಧನೆಗೆ ತಕ್ಕಂತೆ ಬೋಧಿಸಿದರು. ಅವನಿಗೆ ಸಾಂಖ್ಯಯೋಗ, ತತ್ತ್ವಗಳನ್ನು ವಿವರಿಸಿದನು. ಎಷ್ಟೋ ರಾಜರು ಋಷಿಗಳು, ದೇವತೆಗಳಿಗೆ ಸಂಬಂಧಪಟ್ಟ ಐತಿಹ್ಯಗಳನ್ನು ಹೇಳುವರು, ಕಥೆಗಳನ್ನು ಹೇಳುವರು. ಈ ಭಾಗವೇ ಒಟ್ಟು ಪುಸ್ತಕದಲ್ಲಿ ನಾಲ್ಕನೇ ಒಂದು ಭಾಗವಾಗುವುದು. ಇದು ಹಿಂದೂಗಳ ನೀತಿ ಮತ್ತು ಧಾರ್ಮಿಕ ಶಾಸ್ತ್ರಗಳ ಮಹಾನಿಧಿಯಾಗಿದೆ. ಯುಧಿಷ್ಠಿರನಿಗೆ ಚಕ್ರವರ್ತಿ ಪಟ್ಟಾಭಿಷೇಕವಾಯಿತು. ಆದರೆ ಅಷ್ಟೊಂದು ರಕ್ತಪಾತ ಅಷ್ಟೊಂದು ಜನ ಗುರುಹಿರಿಯ ಬಂಧುಬಳಗದವರ ಕೊಲೆಯಿಂದ ತುಂಬಾ ವ್ಯಥಿತನಾದನು. ಅನಂತರ ವ್ಯಾಸರ ಸಲಹೆ ಮೇಲೆ ಅಶ್ವಮೇಧ ಯಜ್ಞವನ್ನು ಮಾಡಿದನು.

\vskip 0.1cm

ಯುದ್ಧವಾದ ಮೇಲೆ ಹದಿನೈದು ವರುಷಗಳು ಧೃತರಾಷ್ಟ್ರ ಗೌರವದಿಂದ ಯುಧಿಷ್ಠಿರ ಮತ್ತು ಅವನ ಸಹೋದರರ ಮನ್ನಣೆಗೆ ಪಾತ್ರನಾಗಿ ನೆಮ್ಮದಿಯಾಗಿದ್ದ. ಅನಂತರ\break ವಯಸ್ಸಾದ ಧೃತರಾಷ್ಟ್ರ ಯುಧಿಷ್ಠಿರನನ್ನು ಸಿಂಹಾಸನದ ಮೇಲೆ ಬಿಟ್ಟು ಪತಿಪರಾಯಣೆಯಾದ ತನ್ನ ಸತಿ ಗಾಂಧಾರಿ ಮತ್ತು ಪಾಂಡವರ ತಾಯಿಯಾದ ಕುಂತಿಯೊಂದಿಗೆ ವಾನಪ್ರಸ್ಥಾಶ್ರಮದಲ್ಲಿ ಕಾಲ ಕಳೆಯಲು ಕಾಡಿಗೆ ಹೋದರು.

\vskip 0.1cm

ಯುಧಿಷ್ಠಿರ ಸಿಂಹಾಸನಕ್ಕೆ ಬಂದು ಮೂವತ್ತಾರು ವರುಷಗಳಾದವು. ಆಗ ಕೃಷ್ಣ\break ಕಾಲವಾದ ಎಂಬ ಸಮಾಚಾರ ಬಂತು. ತನ್ನ ಪಾಲಿಗೆ ಋಷಿ, ಸಖ, ಆಪ್ತ ಗುರುವಾಗಿದ್ದ\break ಕೃಷ್ಣ ಪ್ರಪಂಚವನ್ನು ತೊರೆದಿದ್ದನು. ಅರ್ಜುನ ಅವಸರದಿಂದ ದ್ವಾರಕೆಗೆ ಹೋದನು.\break ಕೃಷ್ಣ ಮತ್ತು ಯಾದವರೆಲ್ಲ ನಾಶವಾದುದು ನಿಜವೆಂಬ ಸುದ್ದಿಯನ್ನು ತಂದನು. ಆಗ ರಾಜ ಮತ್ತು ಅವನ ಸಹೋದರರು ದುಃಖಾಕ್ರಾಂತರಾದರು. ಪ್ರಪಂಚವನ್ನು ತೊರೆಯುವುದಕ್ಕೆ ತಮಗೂ ಕಾಲ ಬಂದಿದೆ ಎಂದು ಹೇಳಿದರು. ಚಕ್ರವರ್ತಿಯ ಜವಾಬ್ದಾರಿಯನ್ನು ಕಿತ್ತೊಗೆದು, ಅರ್ಜುನನ ಮೊಮ್ಮಗನಾದ ಪರೀಕ್ಷಿತ ರಾಜನನ್ನು ಸಿಂಹಾಸನದ ಮೇಲೆ ಕುಳ್ಳಿರಿಸಿ, ಮಹಾಪ್ರಸ್ಥಾನಕ್ಕಾಗಿ ಹಿಮಾಲಯಕ್ಕೆ ಹೋದರು. ಇದೊಂದು ವಿಧದ ವಿಚಿತ್ರ ಸಂನ್ಯಾಸಪದ್ಧತಿ. ವಯಸ್ಸಾದ ರಾಜರು ವಾನಪ್ರಸ್ಥಿಗಳಾಗುವುದು ರೂಢಿ. ಹಿಂದಿನ ಕಾಲದಲ್ಲಿ ವಯಸ್ಸಾದ ಮೇಲೆ ಅವರು ಎಲ್ಲವನ್ನೂ ತ್ಯಜಿಸುತ್ತಿದ್ದರು. ಅದರಂತೆಯೇ ಈ ರಾಜರೂ ಕೂಡ ಮಾಡಿದರು. ಮನುಷ್ಯನಿಗೆ ಇನ್ನು ಹೆಚ್ಚು ಬದುಕಿರಬೇಕೆಂಬ ಇಚ್ಛೆ ಇಲ್ಲದೇ ಇದ್ದರೆ ಹಿಮಾಲಯದಲ್ಲಿ ಏನನ್ನೂ ತಿನ್ನದೆ, ಕುಡಿಯದೆ ಅವರ ದೇಹ ಸೋಲುವವರೆಗೆ ನಡೆದುಕೊಂಡು ಹೋಗುತ್ತಿದ್ದರು. ಸದಾ ದೇವರನ್ನೇ ಕುರಿತು ಚಿಂತಿಸುತ್ತಿದ್ದರು. ಪ್ರಾಣವಿರುವ\break ಪರಿಯಂತರ ಹೀಗೆ ನಡೆದುಕೊಂಡು ಹೋಗುತ್ತಿದ್ದರು.

\vskip 0.1cm

ಆಗ ಋಷಿಗಳು ಮತ್ತು ದೇವತೆಗಳು ಬಂದು ಯುಧಿಷ್ಠಿರನಿಗೆ ನೀನು ಸ್ವರ್ಗಕ್ಕೆ ಹೋಗಬೇಕು ಎಂದರು. ಒಬ್ಬ ಸ್ವರ್ಗಕ್ಕೆ ಹೋಗಬೇಕಾದರೆ ಹಿಮಾಲಯದ ದೊಡ್ಡ ಶಿಖರಗಳನ್ನು ದಾಟಬೇಕು. ಹಿಮಾಲಯದ ಆಚೆ ಮೇರುಪರ್ವತವನ್ನು ದಾಟಬೇಕು. ಮೇರುಪರ್ವತದ ಮೇಲೆ ಸ್ವರ್ಗವಿರುವುದು. ಈ ದೇಹದೊಂದಿಗೆ ಯಾರೂ ಅಲ್ಲಿಗೆ ಹೋಗಿಲ್ಲ. ದೇವತೆಗಳು ಅಲ್ಲಿ ವಾಸ ಮಾಡುವರು. ದೇವತೆಗಳು ಯುಧಿಷ್ಠಿರನಿಗೆ ಸ್ವರ್ಗಕ್ಕೆ ಹೋಗಬೇಕೆಂದು\break ಹೇಳಿದರು.

ಪಂಚಪಾಂಡವರು ಮತ್ತು ಅವರ ಸತಿ ದ್ರೌಪದಿ ನಾರುಡುಗೆಯುಟ್ಟು ಪಯಣಕ್ಕೆ ಹೊರಟರು. ದಾರಿಯಲ್ಲಿ ಒಂದು ನಾಯಿ ಅವರನ್ನು ಹಿಂಬಾಲಿಸಿತು. ಮುಂದೆ ಮುಂದೆ ಹೋದರು. ಸೋತ ಕಾಲಿನಲ್ಲಿ ಹಿಮಾಲಯದಲ್ಲಿ ಭವ್ಯಶಿಖರಗಳಿರುವ ಉತ್ತರದ ಕಡೆ ನಡೆದುಕೊಂಡು ಹೋದರು ಕೊನೆಗೆ ಅಗಾಧವಾದ ಮೇರುಪರ್ವತವನ್ನು ಮುಂದೆ\break ನೋಡಿದರು. ಮೌನವಾಗಿ ಆ ಹಿಮಾಲಯದ ಮೇಲೆ ನಡೆದುಕೊಂಡು ಹೋದರು. ಆಗ ರಾಣಿ ದ್ರೌಪದಿ ನೆಲಕ್ಕೆ ಬಿದ್ದಳು. ಪುನಃ ಏಳಲಿಲ್ಲ. ಮುಂದೆ ಹೋಗುತ್ತಿದ್ದ ಯುಧಿಷ್ಠಿರನಿಗೆ ಭೀಮ “ನೋಡು ರಾಜ, ರಾಣಿ ಬಿದ್ದಳು” ಎಂದ. ಯುಧಿಷ್ಠಿರ ಕಂಬನಿದುಂಬಿದನು.\break ಆದರೆ ಹಿಂತಿರುಗಿ ನೋಡಲಿಲ್ಲ. “ನಾವು ಕೃಷ್ಣನನ್ನು ನೋಡುವುದಕ್ಕೆ ಹೋಗುತ್ತಿರುವೆವು. ಹಿಂದಿರುಗಿ ನೋಡುವುದಕ್ಕೆ ಸಮಯವಿಲ್ಲ. ಮುಂದೆ ಸಾಗಿ” ಎಂದನು. ಕೆಲವು ಕಾಲದ ಮೇಲೆ ಭೀಮ “ನೋಡು, ನಮ್ಮ ತಮ್ಮ ಸಹದೇವ ಬಿದ್ದನು” ಎಂದನು. ಅರಸ ಕಂಬನಿದುಂಬಿದ. ಆದರೆ ಹಿಂತಿರುಗಿ ನೋಡಲಿಲ್ಲ. “ಮುಂದೆ ಸಾಗಿ” ಎಂದ. ಒಬ್ಬೊಬ್ಬರಾಗಿ ಆ ಹಿಮದಲ್ಲಿ, ಚಳಿಯಲ್ಲಿ, ನಾಲ್ಕು ಜನ ಸಹೋದರರೂ ಸತ್ತು ಬಿದ್ದರು. ವಿಚಲಿತನಾಗದೆ ಯುಧಿಷ್ಠಿರ ಮುಂದೆ ಸಾಗಿದ. ಹಿಂತಿರುಗಿ ನೋಡಿದಾಗ ವಿಧೇಯವಾದ\break ನಾಯಿಯೊಂದು ಹಿಂಬಾಲಿಸುತ್ತಿದ್ದುದನ್ನು ನೋಡಿದನು. ರಾಜ ಮತ್ತು ನಾಯಿ ಇಬ್ಬರೂ ಹಿಮದಲ್ಲಿ ಮಂಜಿನ ಕಣಿವೆ ಏರನೇರಿ ಮೇರುಪರ್ವತ ಸಿಕ್ಕುವವರೆಗೆ ಮುನ್ನಡೆದರು.\break ಅಲ್ಲಿ ಮಂಗಳಧ್ವನಿ ಕೇಳಿಸಿತು. ಧರ್ಮಾತ್ಮನಾದ ದೊರೆಯ ಮೇಲೆ ದೇವತೆಗಳು\break ಪುಷ್ಪವೃಷ್ಟಿಯನ್ನು ಕರೆದರು. ಆಗ ದೇವತೆಗಳ ವಿಮಾನ ಕೆಳಗಿಳಿಯಿತು. ಆಗ ಇಂದ್ರ “ವಿಮಾನವನ್ನೇರು ನರಶ್ರೇಷ್ಠನೆ. ಮಾನವ ದೇಹವನ್ನು ಬದಲಾಯಿಸದೆ ಸ್ವರ್ಗಕ್ಕೆ\break ಬರುವುದಕ್ಕೆ ನಿನಗೊಬ್ಬನಿಗೆ ಮಾತ್ರ ಸಾಧ್ಯವಾಯಿತು” ಎಂದನು. ಆದರೆ ಅವನ ನೆಚ್ಚಿನ ಸಹೋದರರಿಲ್ಲದೆ, ದ್ರೌಪದಿ ಇಲ್ಲದೆ, ಧರ್ಮರಾಯ ವಿಮಾನವನ್ನು ಏರಲಿಲ್ಲ. ಆಗ ಇಂದ್ರ, ಸಹೋದರರೆಲ್ಲ ಆಗಲೆ ಸ್ವರ್ಗದಲ್ಲಿ ಇರುವರೆಂದನು.

ಯುಧಿಷ್ಠಿರ ಸುತ್ತಲೂ ನೋಡಿ ನಾಯಿಗೆ “ಮಗು ವಿಮಾನ ಏರು” ಎಂದ. ಇಂದ್ರನಿಗೆ ಆಶ್ಚರ್ಯವಾಯಿತು. “ಏನು ನಾಯಿಯೆ? ನಾಯಿಯನ್ನು ಆಚೆಗೆ ಓಡಿಸು. ನಾಯಿ ಸ್ವರ್ಗಕ್ಕೆ ಹೋಗಲಾರದು. ಏನು ಮಹಾರಾಜ ಇದು? ನೀನೇನು ಹುಚ್ಚನೆ? ಮಾನವ ಕೋಟಿಯಲ್ಲಿ ಧರ್ಮಶ್ರೇಷ್ಠನಾದ ನೀನೊಬ್ಬನೆ ನಿನ್ನ ದೇಹದೊಡನೆ ಸ್ವರ್ಗವನ್ನೇರುವವನು” ಎಂದನು.

\vskip 0.1cm

ಯುಧಿಷ್ಠಿರ “ಈ ನಾಯಿಯು ನಾನು ಮಂಜಿನ ಮೇಲೆ ನಡೆಯುವಾಗಲೆಲ್ಲ ನನ್ನ ಆಪ್ತಮಿತ್ರನಾಗಿರುವುದು. ನನ್ನ ಸಹೋದರರು ಮಡಿದ ಮೇಲೆ, ನನ್ನ ರಾಣಿ ಮಡಿದ ಮೇಲೆ, ಇದೊಂದೆ ನನ್ನನ್ನು ಅಗಲಲಿಲ್ಲ. ಈಗ ಅದನ್ನು ಹೇಗೆ ತ್ಯಜಿಸಲಿ?”

\vskip 0.1cm

ಇಂದ್ರ: “ನಾಯಿಯೊಂದಿಗೆ ಬರುವವರಿಗೆ ಸ್ವರ್ಗದಲ್ಲಿ ಸ್ಥಳವಿಲ್ಲ. ಅದನ್ನು ಹಿಂದೆಯೇ ಬಿಡಬೇಕು. ಅದರಲ್ಲಿ ಅಧರ್ಮವೇನೂ ಇಲ್ಲ.”

\vskip 0.1cm

ಯುಧಿಷ್ಠಿರ; “ನಾನು ಸ್ವರ್ಗಕ್ಕೆ ಹೋಗುವುದಿಲ್ಲ ನಾಯಿ ಇಲ್ಲದೆ. ಯಾರು ನನ್ನಲ್ಲಿ ಆಶ್ರಯ ಪಡೆದಿರುವರೋ ಅವರನ್ನು ನನ್ನ ಪ್ರಾಣವಿರುವ ಪರಿಯಂತರ ತ್ಯಜಿಸಲಾರೆ. ನಾನೆಂದಿಗೂ ಧರ್ಮದಿಂದ ವಿಮುಖನಾಗಲಾರೆ. ಸ್ವರ್ಗಸುಖವಾಗಲಿ ದೇವತೆಗಳ\break ಆಜ್ಞೆಯಾಗಲಿ ಚಿಂತೆಯಿಲ್ಲ.”

\vskip 0.1cm

ಇಂದ್ರ: “ಹಾಗಾದರೆ ಒಂದು ಷರತ್ತಿನ ಮೇಲೆ ನಾಯಿ ಸ್ವರ್ಗಕ್ಕೆ ಹೋಗುವುದು.\break ನೀನು ಮಾನವರಲ್ಲಿ ಧರ್ಮಶ್ರೇಷ್ಠನಾಗಿರುವೆ. ಆದರೆ ಅದು ನಾಯಿ, ಮೃಗಗಳನ್ನು ಕೊಂದು ತಿನ್ನುತ್ತಿತ್ತು. ಅದು ಪಾಪಿ, ಬೇಟೆಯಾಗಿ ಪ್ರಾಣ ತೆಗೆಯುತ್ತಿತ್ತು. ನೀನು ಬೇಕಾದರೆ ನಿನ್ನ ಬದಲು ಅದಕ್ಕೆ ಸ್ವರ್ಗವನ್ನು ಕೊಡಬಹುದು.”

\vskip 0.1cm

ಯುಧಿಷ್ಠಿರ: “ಆಗಲಿ, ನಾಯಿಯೆ ಸ್ವರ್ಗಕ್ಕೇರಲಿ”

\vskip 0.1cm

ತಕ್ಷಣ ದೃಶ್ಯ ಬದಲಾಯಿಸಿತು. ಯುಧಿಷ್ಠಿರನ ಧರ್ಮವಾಕ್ಯವನ್ನು ಕೇಳಿ, ನಾಯಿ ಧರ್ಮದ ಆಕಾರವನ್ನು ತಾಳಿತು. ಅದೇ ಮೃತ್ಯುವಿನ ಮತ್ತು ನ್ಯಾಯದ ಅಧಿಪತಿಯಾದ ಯಮಧರ್ಮರಾಯ. ಮುಂದೆ ಅವನು ಇಂತೆಂದನು: “ನೋಡು ರಾಜ, ನಿನ್ನಂಥ ನಿಃಸ್ವಾರ್ಥಿಗಳು ಇನ್ನಿಲ್ಲ. ಒಂದು ನಿಕೃಷ್ಟ ನಾಯಿಗಾಗಿ ಸ್ವರ್ಗಸುಖವನ್ನೇ ಬಲಿಯಿತ್ತೆ.\break ನಿನ್ನಲ್ಲಿರುವ ಸದ್ಗುಣಗಳನ್ನು ಕಡೆಗಣಿಸಿ, ನಾಯಿಗಾಗಿ ನರಕಕ್ಕೆ ಹೋಗಲು ಸಿದ್ಧ\-ನಾಗಿರುವೆ. ಹೇ ರಾಜನೇ, ಧನ್ಯ ನಿನ್ನ ಬಾಳು. ಎಲ್ಲಾ ಪ್ರಾಣಿಗಳ ಮೇಲೂ ನಿನಗೆ ಕರುಣೆಯಿದೆ. ಇದೊಂದು ಅದಕ್ಕೆ ಉದಾಹರಣೆ. ಅಮರ ಲೋಕ ನಿನ್ನದು ರಾಜ, ನೀನದನ್ನು\break ಸಂಪಾದಿಸಿರುವೆ. ಅಮೃತ ಸ್ವರೂಪವಾದ ದಿವ್ಯವಾದ ಗುರಿ ನಿನ್ನದು.”

\vskip 0.1cm

ಅನಂತರ ಯುಧಿಷ್ಠಿರ ಧರ್ಮ ಮತ್ತು ಇತರ ದೇವತೆಗಳೊಡನೆ ಪುಷ್ಪಕ ವಿಮಾನದಲ್ಲಿ ಸ್ವರ್ಗವನ್ನು ಸೇರಿದರು. ದಾರಿಯಲ್ಲಿ ಸ್ವಲ್ಪ ಕಷ್ಟ ಅನುಭವಿಸಿ ಅಮರ ನದಿಯಲ್ಲಿ ಮಿಂದು ಹೊಸ ದೇಹವನ್ನು ಧರಿಸಿದನು. ಈಗ ಅಮರರಾದ ತನ್ನ ಸಹೋದರರನ್ನು ಸಂಧಿಸಿದನು. ಕೊನೆಗೆ ಎಲ್ಲ ಮಂಗಳವಾಗುವುದು. ಅಧರ್ಮನಾಶ, ಧರ್ಮವಿಜಯವನ್ನು ವಿವರಿಸುವ ದಿವ್ಯಕಾವ್ಯವಾದ ಮಹಾಭಾರತದ ಕಥೆ ಕೊನೆಗಾಣುವುದು ಹೀಗೆ.

ಮಹಾಭಾರತದ ವಿಷಯವಾಗಿ ನಿಮಗೆ ತಿಳಿಸುವಾಗ ಅಸಾಧಾರಣ ಪ್ರತಿಭಾನ್ವಿತ ಮಹಾಮೇಧಾವಿ ವ್ಯಾಸರು ಚಿತ್ರಿಸುವ ಕೊನೆಮೊದಲಿಲ್ಲದ ನಾಯಕವರೇಣ್ಯರ ಭವ್ಯಗಂಭೀರ ವ್ಯಕ್ತಿಗಳ ಶೀಲವನ್ನು ನಿಮ್ಮ ಮುಂದಿಡಲು ನನಗೆ ಸಾಧ್ಯವಿಲ್ಲ. ದೈವಭೀರು\break ಆದರೂ ದುರ್ಬಲನಾದ, ತನ್ನ ಮಕ್ಕಳ ಮೇಲಿನ ಮಮತೆ ಮತ್ತು ಧರ್ಮ ಇವುಗಳ\break ಘರ್ಷಣೆಯಲ್ಲಿ ಸಿಕ್ಕಿರುವ ಅಂಧನಾದ ವೃದ್ಧ ಧೃತರಾಷ್ಟ್ರ, ಹೈಮಾಚಲ ಸದೃಶ\break ಗಾಂಭೀರ್ಯದಿಂದ ಕೂಡಿದವನಾದ ಭೀಷ್ಮ ಪಿತಾಮಹ, ಧರ್ಮಾತ್ಮನಾದ ಉದಾತ್ತ\break ಯುಧಿಷ್ಠಿರ ಮತ್ತು ಶೌರ್ಯದಲ್ಲಿ ಹೇಗೆ ಅಪ್ರತಿಮರೋ ಹಾಗೆಯೇ ಶ್ರದ್ಧೆ ಭಕ್ತಿಗಳಲ್ಲಿ\break ಅಗ್ರಗಣ್ಯರಾದ ಅವನ ನಾಲ್ಕು ಜನ ಸಹೋದರರು, ಮಾನವನ ಜ್ಞಾನದ ಪರಾಕಾಷ್ಠತೆಯನ್ನು ಮೀರಿದ ಆ ಅನುಪಮ ಶ‍್ರೀಕೃಷ್ಣನ ಚಾರಿತ್ರ್ಯ, ಅದಕ್ಕಿಂತ ಕಡಿಮೆ ಏನೂ ಅಲ್ಲದ ಸ್ತ್ರೀಪಾತ್ರಗಳು, ರಾಜಮಹಿಷಿ ಗಾಂಧಾರಿ, ಒಲುಮೆಯ ತಾಯಿ ಕುಂತಿ, ಪತಿಪರಾಯಣೆ\-ಯಾಗಿ ಕಷ್ಟಗಳನ್ನೆಲ್ಲ ಸಹಿಸಿದ ದ್ರೌಪದಿ, ಹೀಗೆ ರಾಮಾಯಣ ಮತ್ತು ಮಹಾಭಾರತದಲ್ಲಿ ಬರುವ ನೂರಾರು ಪಾತ್ರಗಳು ಸಾವಿರಾರು ವರ್ಷಗಳಿಂದ ಹಿಂದೂಗಳಿಗೆ ಸಂಸ್ಕೃತಿಯ ಪುಣ್ಯನಿಧಿಯಾಗಿವೆ. ಇವೇ ಅವರ ಭಾವನೆ, ನೀತಿ, ಧರ್ಮಗಳಿಗೆ ಮೂಲವಾಗಿವೆ. ರಾಮಾಯಣ ಮಹಾಭಾರತಗಳೆರಡೂ ಪುರಾತನ ಆರ್ಯರ ಜೀವನ ಮತ್ತು ಧರ್ಮವನ್ನು ವಿವರಿಸುವ ವಿಶ್ವಕೋಶಗಳಾಗಿವೆ. ಮಾನವಕೋಟಿ ಇನ್ನೂ ಸಾಧಿಸಬೇಕಾಗಿರುವ\break ಆದರ್ಶವನ್ನು ಇವು ಚಿತ್ರಿಸುತ್ತವೆ.

