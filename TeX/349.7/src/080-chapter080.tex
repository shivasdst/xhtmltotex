
\chapter[ಯೋಗಶಾಸ್ತ್ರ ]{ಯೋಗಶಾಸ್ತ್ರ \protect\footnote{\engfoot{C.W. Vol. VII, P. 430}}}

\begin{center}
\textbf{(1900ರ ಏಪ್ರಿಲ್​ 13ರಂದು ಕ್ಯಾಲಿಫೋರ್ನಿಯಾದ ಆಲಮೇಡದ ಟ್ಯೂಕರ್​ ಹಾಲಿನಲ್ಲಿ ಮಾಡಿದ ಭಾಷಣ)}
\end{center}

ಸಂಸ್ಕೃತದಲ್ಲಿ ಯೋಗ ಎಂದರೆ ಚಿತ್ತವೃತ್ತಿ ನಿರೋಧ–ಎಂದರೆ ಚಂಚಲವಾಗಿರುವ\break ಮನಸ್ಸನ್ನು ಏಕಾಗ್ರತೆಗೆ ಹೇಗೆ ತರುವುದು ಎಂಬುದನ್ನು ಹೇಳುವುದು. ಮನಸ್ಸು ಚಿತ್ತದಿಂದ ಆಗಿದೆ. ಇದು ಬಾಹ್ಯ ಮತ್ತು ಆಂತರಿಕ ಪರಿಣಾಮಗಳಿಗೆ ಒಳಗಾಗಿ ಯಾವಾಗಲೂ\break ಅಲೆಗಳಂತೆ ಬದಲಾವಣೆಗೊಳ್ಳುತ್ತಿದೆ. ಆ ಮನಸ್ಸನ್ನು ಹೇಗೆ ನಿಗ್ರಹಿಸುವುದು ಮತ್ತು ಅದು ಅಲೆಗಳಾಗಿ ಸಮತೋಲನ ಕಳೆದುಕೊಳ್ಳದಂತೆ ಹೇಗೆ ನೋಡಿಕೊಳ್ಳುವುದು ಎಂಬುದನ್ನು ಯೋಗವು ಬೋಧಿಸುವುದು.

ಹೀಗೆಂದರೆ ಏನು? ಧರ್ಮವನ್ನು ಅಧ್ಯಯನ ಮಾಡುವವನಿಗೆ, ಧರ್ಮಕ್ಕೆ ಸಂಬಂಧಪಟ್ಟ ಶೇಕಡ ತೊಂಭತ್ತೊಂಭತ್ತು ಪಾಲು ಭಾವನೆಗಳು ಮತ್ತು ಗ್ರಂಥಗಳು ಕೇವಲ\break ಊಹೆಯಾಗಿವೆ. ಒಬ್ಬನು ಧರ್ಮ ಎಂದರೆ ಇದು ಎನ್ನುವನು, ಮತ್ತೊಬ್ಬ ಅದು ಎನ್ನುವನು.\break ಒಬ್ಬನು ಮತ್ತೊಬ್ಬನಿಗಿಂತ ಜಾಣನಾಗಿದ್ದರೆ ಆತನು ಇನ್ನೊಬ್ಬನ ಊಹೆಗಳನ್ನೆಲ್ಲ ಖಂಡಿಸಿ ಬೇರೆ ಊಹೆಗಳನ್ನು ತರುವನು. ಕಳೆದ ಎರಡು ಸಾವಿರ ವರುಷಗಳೊ, ನಾಲ್ಕು ಸಾವಿರ ವರುಷಗಳೊ–ಎಷ್ಟು ಸಾವಿರ ವರುಷಗಳೊ ಸರಿಯಾಗಿ ಯಾರಿಗೂ ಗೊತ್ತಿಲ್ಲದಷ್ಟು.\break ಹಿಂದಿನಿಂದ ಮಾನವರು ಧಾರ್ಮಿಕ ಭಾವನೆಗಳನ್ನು ಅಧ್ಯಯನ ಮಾಡುತ್ತಿರುವರು. ಯಾವಾಗ ಅದನ್ನು ತಾರ್ಕಿಕವಾಗಿ ಸ್ಥಾಪಿಸಲು ಸಾಧ್ಯವಾಗುತ್ತಿರಲಿಲ್ಲವೋ ಆಗ ಇದನ್ನು ಸುಮ್ಮನೆ ನಂಬಿ ಎಂದು ಹೇಳುತ್ತಿದ್ದರು. ಅವರಿಗೆ ಶಕ್ತಿ ಇದ್ದರೆ ತಮ್ಮ ನಂಬಿಕೆಗಳನ್ನು\break ಬಲಾತ್ಕಾರದಿಂದ ಹೊರಿಸುತ್ತಿದ್ದರು. ಇದು ಯಾವಾಗಲೂ ಹೀಗೆಯೇ ಆಗುತ್ತಾ ಬಂದಿದೆ.

ಆದರೆ ಅಂಥದರಿಂದ ತೃಪ್ತಿ ಪಡೆಯದ ಒಂದು ವರ್ಗದ ಜನರಿರುವರು. ಇದರಿಂದ\break ಪಾರಾಗಲು ಸಾಧ್ಯವಿಲ್ಲವೆ ಎಂದು ಅವರು ಕೇಳುವರು. ನಿಮ್ಮ ಭೌತಶಾಸ್ತ್ರದಲ್ಲಿ, ರಸಾಯನ\break ಶಾಸ್ತ್ರದಲ್ಲಿ, ಗಣಿತಶಾಸ್ತ್ರದಲ್ಲಿ ಬರಿಯ ಊಹೆಗೆ ಅವಕಾಶವಿಲ್ಲ. ಧರ್ಮಶಾಸ್ತ್ರವು ಕೂಡ ಉಳಿದ ಶಾಸ್ತ್ರಗಳಂತೆ ಏತಕ್ಕೆ ಇರಬಾರದು? ಅವರು ಹೀಗೆ ಒಂದು ಸಲಹೆಯನ್ನು\break ಕೊಟ್ಟರು: ಮಾನವನ ಆತ್ಮ ಎಂಬುದು ಇದ್ದರೆ, ಅದು ಅಮರವಾಗಿದ್ದರೆ, ದೇವರು ಈ ವಿಶ್ವಕ್ಕೆ ಈಶ್ವರನಾಗಿದ್ದರೆ, ಅವನನ್ನು ಇಲ್ಲೇ ಅರಿಯಬೇಕು; ಇದನ್ನೆಲ್ಲ ಮನುಷ್ಯ ತನ್ನ ಪ್ರಜ್ಞೆಯ ಮೂಲಕ ಗ್ರಹಿಸಲು ಸಾಧ್ಯವಾಗಬೇಕು.

ಮನಸ್ಸನ್ನು ಯಾವ ಬಾಹ್ಯ ಸಲಕರಣೆಯ ಮೂಲಕವೂ ವಿಶ್ಲೇಷಣೆ ಮಾಡಲಾಗುವುದಿಲ್ಲ. ನಾನು ಆಲೋಚನೆ ಮಾಡುತ್ತಿರುವಾಗ ನೀವು ನನ್ನ ಮೆದುಳನ್ನು ನೋಡುತ್ತೀರಿ\break ಎಂದು ಭಾವಿಸೋಣ. ಅಲ್ಲಿ ನಿಮಗೆ ಕೆಲವು ಕಣಗಳು ಬದಲಾಯಿಸುತ್ತಿರುವುದು ಮಾತ್ರ\break ಕಾಣಿಸುವುದು. ನಿಮಗೆ ಅಲ್ಲಿ ಆಲೋಚನೆಗಳಾಗಲೀ, ಪ್ರಜ್ಞೆಯಾಗಲೀ, ಭಾವನೆಯಾಗಲೀ,\break ಕಲ್ಪನೆಗಳಾಗಲೀ ಕಾಣುವುದಿಲ್ಲ. ಕೆಲವು ಕಂಪನಗಳ ರಾಶಿಯನ್ನು ಮಾತ್ರ ನೋಡುತ್ತೀರಿ. ಅವೆಲ್ಲ ಬರೀ ರಾಸಾಯನಿಕ ಮತ್ತು ಭೌತಿಕ ಬದಲಾವಣೆಗಳು. ಈ ಉದಾಹರಣೆಯಿಂದ ಇಂತಹ ವಿಶ್ಲೇಷಣೆಯಿಂದ ಏನೂ ಪ್ರಯೋಜನವಿಲ್ಲ ಎಂದು ಗೊತ್ತಾಗುವುದು.

ಮನಸ್ಸನ್ನು ಮನಸ್ಸಿನಂತೆಯೇ ವಿಶ್ಲೇಷಿಸಲು ಸಾಧ್ಯವಾಗುವ ಮತ್ತಾವುದಾದರೂ\break ಪ್ರಯೋಜನವಿದೆಯೇ? ಹಾಗೇನಾದರೂ ಇದ್ದರೆ ಆಗ ಮಾತ್ರ ಧರ್ಮದ ವಿಜ್ಞಾನ\break ಸಾಧ್ಯ. ರಾಜಯೋಗವು ಇದು ಸಾಧ್ಯ ಎನ್ನುವುದು. ನಾವುಗಳೆಲ್ಲ ಇದಕ್ಕೆ ಪ್ರಯತ್ನಿಸಿ ಆ ದಾರಿಯಲ್ಲಿ ಸ್ವಲ್ಪ ಮುಂದುವರಿಯಬಹುದು. ಆದರೆ ಈ ಮಾರ್ಗದಲ್ಲಿ ಒಂದು\break ತೊಂದರೆಯಿದೆ; ಬಾಹ್ಯ ವಿಜ್ಞಾನಗಳಲ್ಲಿ ವಸ್ತುವನ್ನು ಸುಲಭವಾಗಿ ನಾವು ನೋಡಬಹುದು. ವಿಶ್ಲೇಷಣೆಗೆ ಬಳಸುವ ಯಂತ್ರಗಳು ಕೂಡ ಸ್ಥಿರವಾಗಿ ಇರುವುವು. ಉಪಕರಣಗಳು\break ಮತ್ತು ಯಾವ ವಸ್ತುವನ್ನು ನೋಡಬೇಕೆಂದಿರುವೆವೋ ಅವೆರಡೂ ಹೊರಗೆ ಇವೆ. ಆದರೆ\break ಮನಸ್ಸಿನ ವಿಷಯದಲ್ಲಿಯಾದರೋ, ನಾವು ಯಾವುದರ ಮೂಲಕ ಮತ್ತು ಯಾವುದನ್ನು ನೋಡಬೇಕೆಂದಿರುವೆವೋ ಅವುಗಳೆರಡೂ ಒಂದೇ. ದೃಗ್​ ಮತ್ತು ದೃಶ್ಯ ಎರಡೂ ಒಂದೇ ಆಗುವುವು.

ಬಾಹ್ಯವಿಭಜನೆಯ ಮೆದುಳಿನವರೆಗೂ ಹೋಗಿ ಅಲ್ಲಿ ಆಗುತ್ತಿರುವ ಭೌತಿಕ ಮತ್ತು ರಾಸಾಯನಿಕ ಬದಲಾವಣೆಗಳನ್ನು ನೋಡುವುದು. ಈ ಪ್ರಜ್ಞೆ ಎಂದರೆ ಏನು? ಈ ಕಲ್ಪನೆ ಎಂದರೆ ಏನು ಎಂಬುದನ್ನು ಅದು ಎಂದಿಗೂ ಜಯಪ್ರದವಾಗಿ ಉತ್ತರಿಸಲಾರದು.\break ನಿಮ್ಮಲ್ಲಿರುವ ಇಷ್ಟೊಂದು ಭಾವನೆಗಳ ಸಮೂಹವು ಎಲ್ಲಿಂದ ಬಂದಿತು ಮತ್ತು ಎಲ್ಲಿಗೆ ಹೋಗುತ್ತದೆ ಎಂಬುದೂ ಗೊತ್ತಾಗುವುದಿಲ್ಲ. ನಾವು ಅವುಗಳನ್ನು ಅಲ್ಲಗಳೆಯುವುದಕ್ಕೆ ಆಗುವುದಿಲ್ಲ. ಅವುಗಳೆಲ್ಲ ವಾಸ್ತವಿಕ ಅಂಶಗಳು. ನಾನು ನನ್ನ ಮೆದುಳನ್ನು ಎಂದೂ ನೋಡಿಲ್ಲ. ನನಗೆ ಒಂದು ಮೆದುಳಿದೆ, ಎಂದು ನಂಬಬೇಕಾಗಿದೆ ಅಷ್ಟೆ. ಆದರೆ ಮಾನವ ತನ್ನ ಪ್ರಜ್ಞಾಪೂರ್ವಕವಾಗಿ ಕಲ್ಪಿಸಿಕೊಳ್ಳುವುದನ್ನು ಅಲ್ಲಗಳೆಯಲಾಗುವುದಿಲ್ಲ.

ನಾವೇ ದೊಡ್ಡದೊಂದು ಸಮಸ್ಯೆ. ಒಂದಾದ ಮೇಲೆ ಒಂದು ಕ್ಷಿಪ್ರ ಕ್ರಮದಲ್ಲಿ\break ಬರುತ್ತಿದ್ದರೂ ಯಾವ ಸಂಬಂಧವೂ ಇಲ್ಲದ ಒಂದು ದೊಡ್ಡ ಸರಪಳಿಯೇ ನಾನು?\break ಯಾವಾಗಲೂ ಬದಲಾಗುತ್ತಿರುವ ಪ್ರಜ್ಞೆಯ ಸ್ಥಿತಿಯೆ ನಾನು? ಅಥವಾ ಅವುಗಳಿಗಿಂತ ಹೆಚ್ಚೇನಾದರೂ ಆಗಿರುವೆನೋ? ಒಂದು ವಸ್ತು ಅಥವಾ ಒಂದು ವ್ಯಕ್ತಿ ಎನ್ನುವ ಆತ್ಮನಾಗಿರು\-ವೆನೋ? ಯಾವ ಸಂಬಂಧವೂ ಇಲ್ಲದ ಪ್ರಜ್ಞೆಗಳ ಒಂದು ಕಂತೆಯ ಸ್ಥಿತಿಯೇ ಅಥವಾ ನಾನು ಒಂದು ಅಭಿನ್ನವಾದ ಒಂದು ವ್ಯಕ್ತಿಯೆ? ಇದೇ ದೊಡ್ಡ ಪ್ರಶ್ನೆ, ನಾವು ಕೇವಲ ಪ್ರಜ್ಞೆಯ ಒಂದು ಕಂತೆಯಾಗಿದ್ದರೆ, ಅಮರತ್ವ ಎಂಬ ಭಾವನೆ ಗಾಳಿಗೋಪುರದಂತೆ\break ಆಗುವುದು. ಅದರ ಬದಲು ಯಾವುದಾದರೂ ಅಖಂಡವಾದುದು ನನ್ನಲ್ಲಿ ಇದ್ದರೆ, ಒಂದು ವ್ಯಕ್ತಿತ್ವವಿದ್ದರೆ ಆಗ ನಾನು ಅಮರನಾಗುವೆನು. ಆ ಐಕ್ಯತೆಯನ್ನು ಚೂರು ಚೂರಾಗಿ ಒಡೆದು ಹಾಕುವುದಕ್ಕೆ ಆಗುವುದಿಲ್ಲ, ಧ್ವಂಸ ಮಾಡುವುದಕ್ಕೆ ಆಗುವುದಿಲ್ಲ. ಸಂಯುಕ್ತಗಳನ್ನು ಮಾತ್ರ ಒಡೆಯಬಹುದು.

ಬೌದ್ಧಧರ್ಮ ವಿನಃ ಉಳಿದ ಧರ್ಮಗಳೆಲ್ಲ ಒಂದಲ್ಲ ಒಂದು ರೀತಿಯಲ್ಲಿ ಅಂತಹ ವ್ಯಕ್ತಿತ್ವವನ್ನು ನಂಬುವುವು. ಅದನ್ನು ಪಡೆಯುವುದಕ್ಕೆ ಹೇಗೋ ಪ್ರಯತ್ನಿಸುವುವು. ಬೌದ್ಧಧರ್ಮ ಇದನ್ನು ಇಲ್ಲ ಎನ್ನುವುದು. ಅದಕ್ಕೆ ಇದರಿಂದಲೇ ತೃಪ್ತಿ. ಈಶ್ವರ, ಜೀವ, ಅಮರತ್ವ ಮುಂತಾದುವುಗಳ ವಿಷಯದಲ್ಲಿ ಸುಮ್ಮನೆ ನಿಮ್ಮ ತಲೆಯನ್ನು ಕೆಡಿಸಿಕೊಳ್ಳಬೇಡಿ ಎನ್ನುತ್ತದೆ ಅದು. ಆದರೆ ಪ್ರಪಂಚದ ಇತರ ಧರ್ಮಗಳೆಲ್ಲ ಒಂದು ಆತ್ಮನಲ್ಲಿ ನಂಬುವುವು. ಅವುಗಳು\break ಎಷ್ಟೇ ಬದಲಾವಣೆಗಳು ಆಗುತ್ತಿದ್ದರೂ ಆತ್ಮನು ಈ ದೇಹದಲ್ಲಿರುವನು, ದೇವರು ಈ\break ವಿಶ್ವದಲ್ಲಿ ಇರುವನು ಎನ್ನುವುವು.

ಅವರೆಲ್ಲ ಆತ್ಮನ ಅಮರತ್ವವನ್ನು ನಂಬುವರು. ಇವುಗಳೆಲ್ಲ ಊಹೆಗಳು. ಕ್ರೈಸ್ತರಿಗೂ ಬೌದ್ಧರಿಗೂ ಇರುವ ವಿವಾದವನ್ನು ಯಾರು ಬಗೆಹರಿಸಬೇಕು? ಎಂದೆಂದಿಗೂ ಇರುವ\break ಆತ್ಮವಿದೆ ಎಂದು ಕ್ರೈಸ್ತರು ಹೇಳುತ್ತಾರೆ. ನನ್ನ ಬೈಬಲ್​ ಹಾಗೆ ಹೇಳುತ್ತದೆ ಎನ್ನುತ್ತಾರೆ. ಬೌದ್ಧರು ನಾವು ನಿಮ್ಮ ಗ್ರಂಥವನ್ನು ನಂಬುವುದಿಲ್ಲ ಎನ್ನುತ್ತಾರೆ.

ಪ್ರಶ್ನೆಯೇ ಇದು: ನಾವು ಆತ್ಮವೆ, ಅಥವಾ ಸೂಕ್ಷ್ಮವಾಗಿ ಸದಾ ಕಾಲದಲ್ಲಿ ಬದಲಾಯಿಸು\-ತ್ತಿರುವ ಮನಸ್ಸೆ? ನಮ್ಮ ಮನಸ್ಸು ಸದಾ ಬದಲಾಯಿಸುತ್ತಿರುವುದು. ಅಲ್ಲಿ ಇರುವುದಾವುದೂ ನಮಗೆ ಕಾಣುವುದಿಲ್ಲ. ನಾನು ಈಗ ಇದಾಗಿರುವೆನು, ಆಮೇಲೆ ಆದಾಗಿರುವೆನು. ಈ ಬದಲಾವಣೆಗಳನ್ನು ಕ್ಷಣ ಕಾಲವಾದರೂ ನಿಲ್ಲಿಸುವುದಕ್ಕೆ ಸಾಧ್ಯವಾದರೆ ನಾನು ಆತ್ಮನ\break ಅಸ್ತಿತ್ವವನ್ನು ನಂಬುತ್ತೇನೆ.

ದೇವರು ಸ್ವರ್ಗ ಮುಂತಾದ ಭಾವನೆಗಳೆಲ್ಲ ವ್ಯವಸ್ಥಿತವಾದ ಧರ್ಮಗಳ ಸಣ್ಣ ಸಣ್ಣ ನಂಬಿಕೆಗಳು. ಯಾವ ವೈಜ್ಞಾನಿಕವಾದ ಧರ್ಮವೂ ಕೂಡ ಇಂತಹ ನಂಬಿಕೆಗಳನ್ನು ಸಾರಲಾರದು.

ಚಿತ್ರದಲ್ಲಿ ವೃತ್ತಿಗಳು ಏಳದಂತೆ ತಡೆಗಟ್ಟುವುದೇ ಯೋಗಶಾಸ್ತ್ರ. ನೀವು ಮನಸ್ಸನ್ನು ಯೋಗದ ಪರಿಪೂರ್ಣಸ್ಥಿತಿಗೆ ತರುವುದರಲ್ಲಿ ಜಯಶೀಲರಾದರೆ, ನೀವು ಆ ಕ್ಷಣ ಈ\break ಸಮಸ್ಯೆಯನ್ನು ಬಗೆಹರಿಸಬಲ್ಲಿರಿ. ನಿಮ್ಮ ನೈಜ ಸ್ವಭಾವ ಆಗ ನಿಮಗೆ ಅರ್ಥವಾಗಿದೆ\break ಎಂದಾಯಿತು. ಬದಲಾವಣೆಗಳನ್ನೆಲ್ಲ ನೀವು ಆಗ ನಿಗ್ರಹಿಸಿರುವಿರಿ. ಅನಂತರ ನೀವು\break ಮನಸ್ಸನ್ನು ಅಲೆದಾಡಲು ಬಿಡಬಹುದು. ಆದರೆ ಅದು ಹಿಂದಿನ ಮನಸ್ಸಲ್ಲ. ಅದು\break ಪೂರ್ಣವಾಗಿ ನಿಮ್ಮ ಸ್ವಾಧೀನಕ್ಕೆ ಒಳಪಟ್ಟಿದೆ. ಅದು ಕುದುರೆಯಂತೆ ನಿಮ್ಮನ್ನು ಎಲ್ಲಿ\break ಬೇಕಾದರೂ ಅಪ್ಪಳಿಸಲಾರದು. ನೀವು ಆಗ ದೇವರನ್ನು ನೋಡಿರುವಿರಿ. ಇನ್ನು ಮೇಲೆ ಅದೊಂದು ಊಹಾವಸ್ತುವಲ್ಲ. ಇನ್ನು ಮೇಲೆ ಅವನು ಯಾವುದೋ ವ್ಯಕ್ತಿಯಲ್ಲ.\break ಅವನಿಗೆ ಇನ್ನು ಯಾವ ಶಾಸ್ತ್ರಗಳೂ ಬೇಕಾಗಿಲ್ಲ, ವೇದಗಳು ಬೇಕಾಗಿಲ್ಲ. ಬೋಧಕರಲ್ಲಿರುವ ಚರ್ಚಾಸ್ಪದ ವಸ್ತುವಿನ ಕಡೆಗೂ ಅವನು ಗಮನ ಕೊಡುವುದಿಲ್ಲ. ನೀವು ನೀವೇ ಆಗಿದ್ದೀರಿ. ಎಲ್ಲ ಬದಲಾವಣೆಗಳನ್ನು ಮೀರಿದ ವಸ್ತು ನಾನಾಗಿದ್ದೇನೆ. ನಾನು ಕೇವಲ ಬದಲಾವಣೆ ಆಗಿದ್ದರೆ ಅದನ್ನು ನಿಲ್ಲಿಸಲು ನನ್ನಿಂದ ಸಾಧ್ಯವಾಗುತ್ತಿರಲಿಲ್ಲ. ನಾನು ಬದಲಾವಣೆಯನ್ನು ನಿಲ್ಲಿಸಬಲ್ಲೆ. ಆದಕಾರಣ ನಾನು ಬದಲಾವಣೆಗಳು ಆಗಲಾರೆ. ಇದೇ ಯೋಗಶಾಸ್ತ್ರದ ಸಲಹೆ.

ಈ ಬದಲಾವಣೆಗಳನ್ನು ಕಂಡರೆ ನಮಗೆ ಆಗದು. ನಮಗೆ ಬದಲಾವಣೆಗಳೇ ಇಷ್ಟವಿಲ್ಲ. ಪ್ರತಿಯೊಂದು ಬದಲಾವಣೆಯೂ ನಮ್ಮ ಮೇಲೆ ಬಲಾತ್ಕಾರವಾಗಿ ಹೇರಲ್ಪಡುವುದು. ನಮ್ಮ ದೇಶದಲ್ಲಿ ಗಾಣಕ್ಕೆ ಕಟ್ಟಿರುವ ಎತ್ತಿನ ಮುಂದೆ ಸ್ವಲ್ಪ ಹುಲ್ಲನ್ನು ಕಟ್ಟುತ್ತಾರೆ. ಅದು ಎತ್ತಿಗೆ ಸುಮ್ಮನೇ ಆಸೆ ತೋರಿಸುವುದಕ್ಕಾಗಿ. ಎತ್ತಿಗೆ ಮಾತ್ರ ಅದು ಎಂದಿಗೂ ಸಿಕ್ಕುವಂತಿಲ್ಲ. ಎತ್ತು ಅದನ್ನು ಹಿಡಿಯುವುದಕ್ಕಾಗಿ ಮುಂದೆ ಹೋಗುವುದರಿಂದ ಗಾಣ ಆಡುವುದು. ನಾವುಗಳು ಕೂಡ ಆ ಎತ್ತಿನಂತೆ. ಮುಂದಿರುವ ಹುಲ್ಲನ್ನು ತಿನ್ನಲು ಯಾವಾಗಲೂ ಅದನ್ನು ಅನುಸರಿಸುತ್ತಿರುವೆವು. ಗಾಣವನ್ನು ಸುತ್ತುವ ಎತ್ತಿನಂತೆ ನಾವು ಸುತ್ತುತ್ತಿರುವೆವು. ಯಾರಿಗೂ ಬದಲಾವಣೆ ಬೇಕಾಗಿಲ್ಲ. ನಿಜವಾಗಿಯೂ ಬೇಕಾಗಿಲ್ಲ. ಈ ಬದಲಾವಣೆಗಳನ್ನೆಲ್ಲ ಯಾರೋ ಬಲಾತ್ಕಾರವಾಗಿ ನಮ್ಮ ಮೇಲೆ ಹೇರಿರುವರು. ನಾವು ಅದರಿಂದ ತಪ್ಪಿಸಿಕೊಳ್ಳಲಾರೆವು. ಒಮ್ಮೆ ನಾವು ನೊಗಕ್ಕೆ ತಲೆಯೊಡ್ಡಿದರೆ ಯಾವಾಗಲೂ ಸುತ್ತುತ್ತಿರಬೇಕಾಗುತ್ತದೆ. ನಾವು ನಿಂತೊಡನೆಯೇ ಯಾವುದೊ ಚಾವಟಿ ಏಟು ಕಾದಿದೆ. ಮುಂದೆ ಹೋಗುವುದಕ್ಕಿಂತ ಅಪಾಯಕರ ಅದು.

ನಮಗೆ ದುಃಖ ಉಂಟಾಗುತ್ತದೆ. ಇದೆಲ್ಲ ದುಃಖಮಯ. ಏಕೆಂದರೆ ಇದೆಲ್ಲ ನಮಗೆ ಇಷ್ಟವಾದುದಲ್ಲ. ಇವೆಲ್ಲ ಬಲಾತ್ಕಾರದಿಂದ ಹೇರಿದುದು. ಪ್ರಕೃತಿ ನಮಗೆ ಆಜ್ಞಾಪಿಸುವುದು. ನಾವು ಅದನ್ನು ಪಾಲಿಸುವೆವು. ನಮಗೆ ಪ್ರಕೃತಿಯ ಮೇಲೆ ಯಾವ ಪ್ರೀತಿಯೂ ಇಲ್ಲ. ಎಲ್ಲಾ ಕೆಲಸಗಳನ್ನೂ ನಾವು ಮಾಡುತ್ತಿರುವುದು ಪ್ರಕೃತಿಯಿಂದ ಪಾರಾಗುವುದಕ್ಕಾಗಿ. ನಾವು ಪ್ರಕೃತಿಯನ್ನು ಆನಂದದಿಂದ ಸ್ವೀಕರಿಸುತ್ತೇವೆ ಎಂದು ಹೇಳುತ್ತೇವೆ. ಆದರೆ ನಮ್ಮ ಮನಸ್ಸನ್ನು ನಾವು ಪರೀಕ್ಷಿಸಿಕೊಂಡರೆ, ನಾವು ಪ್ರಕೃತಿಯಿಂದ ಪಾರಾಗಿ ಯಾವುದನ್ನಾದರೂ ಪಡೆದು ಸುಖವಾಗಿರಲು ಪ್ರಯತ್ನಿಸುತ್ತಿರುವೆವು ಎಂದು ಗೊತ್ತಾಗುವುದು. ಆಂಗ್ಲೇಯನನ್ನು ತನ್ನ ಮನೆಗೆ ಒಬ್ಬ ಫ್ರೆಂಚ್​ ವ್ಯಕ್ತಿ ಕರೆದು, ತನ್ನ ಉಗ್ರಾಣದಲ್ಲಿ ಸೊಗಸಾದ ದ್ರಾಕ್ಷಾರಸವಿದೆ ಎಂದಂತೆ ಪ್ರಕೃತಿ. ಅವನು ಒಂದು ಬುಡ್ಡಿ ಹಳೆಯ ದ್ರಾಕ್ಷಾರಸವನ್ನು ತರಿಸಿದ. ಅದು ನೋಡುವುದಕ್ಕೆ ಅಷ್ಟು ಚೆನ್ನಾಗಿತ್ತು. ಚಿನ್ನದಂತೆ ಒಳಗೆ ಹೊಳೆಯುತ್ತಿತ್ತು. ಬಟ್ಲರ್​ ಒಂದು ಬುಡ್ಡಿಯನ್ನು ಬಿಚ್ಚಿ ಗ್ಲಾಸಿನಲ್ಲಿ ಹಾಕಿ ಆಂಗ್ಲೇಯನಿಗೆ ಕೊಟ್ಟನು. ಅವನು ಅದನ್ನು ಮರುಮಾತಿಲ್ಲದೆ ಕುಡಿದನು. ಬಟ್ಲರ್​ ಒಂದು ಬುಡ್ಡಿ ಹರಳೆಣ್ಣೆಯನ್ನು ತಂದಿದ್ದನು! ನಾವು ಯಾವಾಗಲೂ ದ್ರಾಕ್ಷಾರಸವೆಂದು ಕುಡಿಯುತ್ತಿರುವುದು ಹರಳೆಣ್ಣೆಯನ್ನೇ. ನಾವು ಅದರಿಂದ ಪಾರಾಗಲಾರೆವು.

ಸಾಧಾರಣವಾಗಿ ಜನರು ಒಂದು ಯಂತ್ರದಂತೆ ಆಗುವರು. ಅವರು ಯಾವುದನ್ನೂ ಆಲೋಚಿಸುವುದಿಲ್ಲ. ನಾಯಿ, ಬೆಕ್ಕು ಮತ್ತು ಇತರ ಪ್ರಾಣಿಗಳಂತೆ ಪ್ರಕೃತಿ ಮನುಷ್ಯನನ್ನು ಒಂದು ಚಾವಟಿಯಿಂದ ನಡೆಸುತ್ತದೆ. ಅವರು ಎಂದಿಗೂ ಆಜ್ಞೆಯನ್ನು ವಿರೋಧಿಸುವುದಿಲ್ಲ. ಎಂದಿಗೂ ಆಲೋಚಿಸುವುದಿಲ್ಲ. ಆದರೆ ಅಂಥವರಿಗೂ ಪ್ರಪಂಚದಲ್ಲಿ ಸ್ವಲ್ಪ ಅನುಭವ\break ಬರುವುದು.

ಎಲ್ಲೋ ಆದರೆ ಕೆಲವರು ಪ್ರಶ್ನಿಸುವರು: ಇದೇನು? ಈ ಅನುಭವಗಳೆಲ್ಲ ಏತಕ್ಕೆ?\break ಆತ್ಮವೆಂದರೆ ಏನು? ಪಾರಾಗಲು ಯಾವುದಾದರೂ ಮಾರ್ಗವಿದೆಯೆ? ಜೀವನಕ್ಕೆ ಏನಾದರೂ ಅರ್ಥವಿದೆಯೆ?

ಒಳ್ಳೆಯವರೂ ಸಾಯುವರು, ಕೆಟ್ಟವರೂ ಸಾಯುವರು. ರಾಜರೂ ಸಾಯುವರು. ಭಿಕ್ಷುಕರೂ ಸಾಯುವರು. ದೊಡ್ಡ ದುಃಖವೆ ಮೃತ್ಯು. ನಾವು ಯಾವಾಗಲೂ ಅದರಿಂದ\break ಪಾರಾಗಲು ಯತ್ನಿಸುತ್ತಿರುವೆವು. ನಾವು ಒಂದು ಅನುಕೂಲವಾದ ಧರ್ಮದಲ್ಲಿದ್ದು\break ಕಾಲವಾದರೆ, ನಾವು ಅನಂತರ ಬೇರೆ ಲೋಕದಲ್ಲಿ ಮೃತಪಟ್ಟ ನಮ್ಮ ಸಂಬಂಧಿಕರೊಡನೆ ಚೆನ್ನಾಗಿ ಕಾಲ ಕಳೆಯಬಹುದು ಎಂದು ಭಾವಿಸುತ್ತೇವೆ.

ನಿಮ್ಮ ದೇಶದಲ್ಲಿ ಸಿಯಾನ್ನೆಸ್ಗಳಲ್ಲಿ (ಪಿತೃಗಳನ್ನು ದರ್ಶನ ಮಾಡಲು ಇರುವ ಆಲಯಗಳು) ಗತಿಸಿದ ನಿಮ್ಮ ಕಡೆಯವರನ್ನು ಕೆಳಗೆ ಕರೆದುಕೊಂಡು ಬರುತ್ತೀರಿ. ನಾನು ಅನೇಕ ವೇಳೆ ಅಂತಹವರನ್ನು ಕಂಡೆ. ಅವರೊಡನೆ ಕೈಗಳನ್ನು ಕುಲುಕಿರುವೆನು. ನಿಮ್ಮಲ್ಲಿ ಹಲವರು ಅವರನ್ನು ನೋಡಿರಬಹುದು. ಅವರು ಪಿಯಾನೊಗಳನ್ನು ಬಡಿದು “ಬ್ಯುಲಾಲ್ಯಾಂಡ್​” ಎಂದು ಹಾಡುವರು. ಅಮೆರಿಕಾ ದೊಡ್ಡ ದೇಶ. ನನ್ನ ದೇಶವಾದರೊ ಭೂಗೋಳದ ಅತ್ತ ಕಡೆ. ನನಗೆ ಬ್ಯುಲಾಲ್ಯಾಂಡ್​ ಎಲ್ಲಿರುವುದೊ ಗೊತ್ತಿಲ್ಲ. ನಿಮಗೆ ಅದು ಯಾವ ಭೂಗೋಳದ ಪುಸ್ತಕದಲ್ಲಿಯೂ ದೊರಕುವುದಿಲ್ಲ. ನೋಡಿ, ನಮ್ಮ ಒಳ್ಳೆಯ ಕ್ಷೇಮವಾದ ಧರ್ಮ ಎಂದರೆ ಏನು? ಇವೆಲ್ಲ ಅತ್ಯಂತ ಪುರಾತನವಾದ ನುಸಿ ಹಿಡಿದ ನಂಬಿಕೆಗಳು!

ಇಂತಹ ಜನ ಆಲೋಚನೆ ಮಾಡಲಾರರು. ಅವರಿಗಾಗಿ ಏನು ಮಾಡಬೇಕು?\break ಜಗತ್ತು ಅವರನ್ನು ತಿಂದು ಹಾಕಿದೆ. ಆಲೋಚಿಸುವುದಕ್ಕೆ ಅವರಲ್ಲಿ ಏನೂ ಇಲ್ಲ. ಅವರ ಮೂಳೆಗಳು ಟೊಳ್ಳಾಗಿವೆ. ಅವರ ಮೆದುಳಿನಲ್ಲಿ ಯಾವ ಸತ್ತ್ವವೂ ಇಲ್ಲ. ಮಿದುಳು ಬೆಣ್ಣೆಯಂತೆ ಮೆತ್ತಗಾಗಿದೆ. ನಾನು ಅವರಿಗೆ ಸಹಾನುಭೂತಿಯನ್ನು ತೋರುತ್ತೇನೆ. ಪಾಪ ತಮ್ಮ ಸೌಖ್ಯವನ್ನು ಅವರು ಇಟ್ಟುಕೊಂಡಿರಲಿ. ಬ್ಯೂಲಾಲ್ಯಾಂಡಿನಿಂದ ಬಂದು ತಮ್ಮ ಪಿತೃಗಳನ್ನು ನೋಡಿ ಎಷ್ಟೋ ಜನಕ್ಕೆ ಸಮಾಧಾನವಾಗುವುದು.

ಪ್ರೇತ ಆವಾಹಕರಲ್ಲಿ ಒಬ್ಬನು ನನ್ನ ಗತಿಸಿದವರನ್ನು ಕರೆದುಕೊಂಡು ಬಂದು ತೋರಿಸುತ್ತೇನೆ ಎಂದ. ನಾನು ಹೇಳಿದೆ: “ಸದ್ಯಕ್ಕೆ ನಿಲ್ಲಿಸಿ, ನೀವು ಏನು ಬೇಕಾದರೂ ಮಾಡಿ. ಆದರೆ ನೀವೇನಾದರೂ ನನ್ನ ಗತಿಸಿದವರನ್ನು ಕರೆದುಕೊಂಡು ಬಂದರೆ ನಾನು ಸಹನೆಯಿಂದ\break ಇರುತ್ತೇನೋ ಇಲ್ಲವೊ” ಎಂದು ಅವರಿಗೆ ಎಚ್ಚರಿಕೆ ಕೊಟ್ಟೆ, ಆ ಪ್ರೇತ ಆವಾಹಕ ದಯೆಯಿಂದ ಸದ್ಯಕ್ಕೆ ಅದನ್ನು ನಿಲ್ಲಿಸಿದನು.

ನಮ್ಮ ದೇಶದಲ್ಲಿ ಯಾವುದಾದರೂ ಒಂದು ಕಡೆಯಿಂದ ನಮಗೆ ತೊಂದರೆಯಾದರೆ, ನಾವು ಪೂಜಾರಿಗಳಿಗೆ ಏನನ್ನಾದರೂ ಕೊಟ್ಟು ದೇವರ ಹತ್ತಿರ ರಾಜಿ ಮಾಡಿಕೊಳ್ಳುತ್ತೇವೆ. ಸದ್ಯಕ್ಕೆ ನಮಗೆ ಸ್ವಲ್ಪ ಸಮಾಧಾನವಾಗುವುದು. ಆದರೆ ಅನಂತರ ಪೀಡೆಯು ಪ್ರಾರಂಭವಾಗುವುದು. ಪುನಃ ದುಃಖ ಬರುವುದು. ಇಲ್ಲಿ ಯಾವಾಗಲೂ ದುಃಖ ತಪ್ಪಿದ್ದಲ್ಲ. ನಿಮ್ಮ ಜನರು ನಮ್ಮ ದೇಶಕ್ಕೆ ಬಂದು “ನೀವು ನಮ್ಮ ಧರ್ಮವನ್ನು ನಂಬಿದರೆ ಒಳ್ಳೆಯದಾಗುವುದು” ಎಂದು ಹೇಳುವರು. ನಮ್ಮ ಜನರಲ್ಲಿ ತುಂಬಾ ಕೆಳಗಡೆ ಇರುವವರು ನಿಮ್ಮ ಸಿದ್ಧಾಂತವನ್ನು ನಂಬುವರು. ಒಂದೇ ಬದಲಾವಣೆ ಎಂದರೆ ಅವರು ಭಿಕ್ಷುಕರಾಗುವರು. ಆದರೆ ಅದು ಧರ್ಮವೆ? ಅದು ರಾಜಕೀಯವೆ ಹೊರತು ಧರ್ಮವಲ್ಲ. ನೀವು ಬೇಕಾದರೆ ಧರ್ಮವನ್ನು ಆ ಮಟ್ಟಕ್ಕೆ ಎಳೆದು ಆ ಹೆಸರಿನಿಂದ ಕರೆಯಬಹುದು. ಆದರೆ ಇದು ಆಧ್ಯಾತ್ಮಿಕತೆಯಲ್ಲ.

ಸಹಸ್ರಾರು ಜನರಲ್ಲಿ ಎಲ್ಲೋ ಕೆಲವು ಮಂದಿ ಮಾತ್ರ ಈ ಪ್ರಪಂಚದಲ್ಲಿರುವುದಕ್ಕಿಂತ ಹೆಚ್ಚನ್ನು ಆಶಿಸುವರು. ಉಳಿದವರು ಕುರಿಮಂದೆಯಂತೆ. ಸಹಸ್ರಾರು ಜನರಲ್ಲಿ ಎಲ್ಲೋ ಕೆಲವರು ಇದನ್ನು ತಿಳಿದುಕೊಳ್ಳುವುದಕ್ಕೆ, ಇದರಿಂದ ಪಾರಾಗುವುದಕ್ಕೆ ಪ್ರಯತ್ನಿಸುವರು. ಪಾರಾಗುವುದಕ್ಕೆ ಒಂದು ದಾರಿ ಇದೆಯೇ ಎಂಬುದೇ ಪ್ರಶ್ನೆ. ಮಾರ್ಗವಿದ್ದರೆ ಅದು ಆತ್ಮನ ಮೂಲಕ ಮಾತ್ರ ಇದೆ. ಬೇರೆ ಎಲ್ಲೂ ಇಲ್ಲ. ಬೇರೆ ಕಡೆ ತಪ್ಪಿಸಿಕೊಳ್ಳುವುದಕ್ಕೆ ಬೇಕಾದಷ್ಟು\break ಪ್ರಯತ್ನಿಸಿ ವ್ಯರ್ಥವಾಗಿದೆ. ಜನರಿಗೆ ತೃಪ್ತಿ ಸಿಕ್ಕುವುದಿಲ್ಲ. ಇಷ್ಟೊಂದು ಸಿದ್ಧಾಂತಗಳು\break ಮತ್ತು ಆಚಾರಗಳು ಇರುವುದನ್ನು ನೋಡಿದರೆ ಅದರಿಂದ ತೃಪ್ತಿ ಸಿಕ್ಕುವುದಿಲ್ಲ ಎಂಬುದನ್ನು\break ತೋರುವುದು.

ನಾವು ಪಾರಾಗಬೇಕಾದರೆ ಆತ್ಮನ ಮೂಲಕ ಮಾತ್ರ ಸಾಧ್ಯ ಎಂದು ಯೋಗ ಹೇಳುವುದು. ನಾವು ಪ್ರತ್ಯೇಕ ವ್ಯಕ್ತಿತ್ವವುಳ್ಳವರಾಗಬೇಕು. ಸತ್ಯವೇನಾದರೂ ಇದ್ದರೆ ಅದನ್ನು ನಮ್ಮ ಆತ್ಮನಂತೆ ನಾವು ಕಂಡುಕೊಳ್ಳಲು ಸಾಧ್ಯ. ಒಂದು ಕಡೆಯಿಂದ ಮತ್ತೊಂದು ಕಡೆಗೆ\break ಪ್ರಕೃತಿಯ ಚಾವಟಿಯ ಪೆಟ್ಟಿಗೆ ಅಂಜಿ ಓಡಿಹೋಗುವುದನ್ನು ಆಗ ಬಿಡುತ್ತೇವೆ.

ಬಾಹ್ಯ ಜಗತ್ತು ಯಾವಾಗಲೂ ಬದಲಾಯಿಸುತ್ತಿದೆ. ಯಾವುದು ಬದಲಾಯಿಸುವುದಿಲ್ಲವೋ ಅದನ್ನು ಸೇರುವುದು ನಮ್ಮ ಗುರಿ. ನಾವು ಆದಾಗಬೇಕು. ಆ ಅನಂತವಾಗಬೇಕು. ಬದಲಾಯಿಸದೆ ಇರುವ ಸತ್ಯ ಅದೇ. ಆ ಸತ್ಯವನ್ನು ಪಡೆಯದಂತೆ ಯಾವುದು ನಮ್ಮನ್ನು ತಡೆಯುತ್ತಿರುವುದು? ಅದೇ ಸೃಷ್ಟಿ. ಸೃಷ್ಟಿಸುವ ಮನಸ್ಸು ಸದಾಕಾಲದಲ್ಲಿಯೂ ವಸ್ತುಗಳನ್ನು ಸೃಷ್ಟಿಸಿ, ಅದರಲ್ಲಿ ಮುಳುಗಿಹೋಗುತ್ತಿದೆ. ಆದರೆ ಆ ಶಕ್ತಿಯೇ ದೇವರನ್ನು ಕಂಡುಹಿಡಿದದ್ದು ಎಂಬುದನ್ನು ನಾವು ಜ್ಞಾಪಕದಲ್ಲಿಡಬೇಕು. ಪ್ರತಿಯೊಂದು ಆತ್ಮನಲ್ಲಿಯೂ ಇರುವ ಅನಂತವನ್ನು ಕಂಡುಹಿಡಿದುದೇ ಸೃಷ್ಟಿ.

ನಮ್ಮ ವಿವರಣೆಗೆ ಹಿಂತಿರುಗೋಣ. ಯೋಗ ಎಂದರೆ ವೃತ್ತಿಗಳಾಗದಂತೆ ಮನಸ್ಸನ್ನು ನಿಗ್ರಹಿಸುವುದು. ಈ ಸೃಷ್ಟಿಯನ್ನೆಲ್ಲ ನಿಲ್ಲಿಸಿದ ಮೇಲೆ, ಇದನ್ನು ನಿಲ್ಲಿಸಲು ಸಾಧ್ಯವಾದರೆ ಆಗ ನಿಜವಾಗಿ ನಾವು ಏನಾಗಿರುವೆವೊ ಅದನ್ನು ಅರಿಯುವೆವು. ನಿಜವಾದದ್ದು ಎಂದರೆ ಸೃಷ್ಟಿಯಾಗದುದು, ಆದರೆ ಎಲ್ಲವನ್ನು ಸೃಷ್ಟಿಸುವಂಥದು. ಆಗ ಅದು ವ್ಯಕ್ತವಾಗುವುದು.

ಹಲವು ಯೋಗ ಮಾರ್ಗಗಳಿವೆ. ಅವುಗಳಲ್ಲಿ ಕೆಲವು ಬಹಳ ಕಷ್ಟ. ಅವುಗಳಲ್ಲಿ ಮುಂದುವರಿಯಬೇಕಾದರೆ ಬಹಳ ಅಭ್ಯಾಸ ಆವಶ್ಯಕ. ಕೆಲವು ಸುಲಭ. ಯಾರಿಗೆ ಶಕ್ತಿ ಇದೆಯೋ ಮತ್ತು ಬಿಡದೆ ಮುಂದುವರಿಯುವ ಸಾಹಸವಿದೆಯೊ ಅವರಿಗೆ ಬಹಳ ಉತ್ತಮವಾದ ಪ್ರತಿಫಲಗಳು ದೊರಕುತ್ತವೆ. ಯಾರಿಗೆ ಇದು ಸಾಧ್ಯವಿಲ್ಲವೋ ಅವರು ಸುಲಭವಾದ ಹಾದಿಯನ್ನು ಹಿಡಿದು ಅದರ ಮೂಲಕವೂ ಸ್ವಲ್ಪ ಪ್ರಯೋಜನವನ್ನು ಪಡೆಯಬಹುದು.

ಮನಸ್ಸನ್ನೇ ನಾವು ಸರಿಯಾಗಿ ವಿಶ್ಲೇಷಣೆ ಮಾಡಿದಾಗ, ಅದರೊಂದಿಗೆ ಹೋರಾಡುವುದು ಎಷ್ಟು ಕಷ್ಟ ಎಂಬುದು ನಮಗೆ ಗೊತ್ತಾಗುವುದು. ನಾವೆಲ್ಲ ದೇಹಗಳಾಗಿ ಹೋಗಿರುವೆವು. ನಾವು ಆತ್ಮ ಎಂಬುದನ್ನು ಸಂಪೂರ್ಣ ಮರೆತು ಬಿಟ್ಟಿರುವೆವು. ನಾವು ನಮ್ಮನ್ನು ಕುರಿತು ಚಿಂತಿಸಿದರೆ ನಮಗೆ ಬರುವುದೇ ದೇಹದ ಭಾವನೆ. ನಾವು ದೇಹವೆಂಬಂತೆ\break ವರ್ತಿಸುತ್ತೇವೆ. ನಾವೇ ದೇಹವೆಂಬಂತೆ ಮಾತಾಡುತ್ತೇವೆ. ನಾವೆಲ್ಲ ದೇಹಗಳು ನಿಜ. ನಾವು ಈ ದೇಹದಿಂದ ಆತ್ಮನನ್ನು ಪ್ರತ್ಯೇಕಿಸಬೇಕಾಗಿದೆ. ಆದಕಾರಣ ಸಾಧನೆ ದೇಹದಿಂದಲೇ ಪ್ರಾರಂಭವಾಗುವುದು. ಕೊನೆಗೆ ಆತ್ಮವೊಂದೇ ವ್ಯಕ್ತವಾಗುವುದು. ಈ ಸಾಧನೆಗಳಲ್ಲೆಲ್ಲ ಇರುವ ಮುಖ್ಯ ವಿಷಯವೆಂದರೆ ಆ ಏಕಾಗ್ರತೆಯನ್ನು, ಧ್ಯಾನಸ್ಥಿತಿಯನ್ನು ಪಡೆಯುವುದಾಗಿದೆ.

