
\chapter[ಪಾಶ್ಚ್ಯಾತ್ಯರಲ್ಲಿ ಪ್ರಥಮ ಹಿಂದೂ ಪ್ರಚಾರಕರು ಮತ್ತು ಭಾರತ ಪುನರುದ್ಧಾರಕ ಅವರು ಯೋಜನೆ ]{ಪಾಶ್ಚ್ಯಾತ್ಯರಲ್ಲಿ ಪ್ರಥಮ ಹಿಂದೂ ಪ್ರಚಾರಕರು ಮತ್ತು ಭಾರತ ಪುನರುದ್ಧಾರಕ ಅವರು ಯೋಜನೆ \protect\footnote{\engfoot{C.W. Vol. V, P. 218}}}

(ಮದ್ರಾಸ್​ ಟೈಮ್ಸ್-ಫೆಬ್ರವರಿ ೧೮೯೭)

ಕಳೆದ ಕೆಲವು ವಾರಗಳಿಂದ ಮದ್ರಾಸಿನ ಹಿಂದೂ ಪುರಜನರು ಜಗದ್ವಿಖ್ಯಾತ ಸ್ವಾಮಿ ವಿವೇಕಾನಂದರ ಬರವನ್ನು ಎದುರುನೋಡುತ್ತಿದ್ದಾರೆ. ಈಗ ಎಲ್ಲರ ಬಾಯಲ್ಲೂ ಅವರ ಮಾತೆ. ಶಾಲೆಯಲ್ಲಿ, ಕಾಲೇಜಿನಲ್ಲಿ, ಹೈಕೋರ್ಟಿನಲ್ಲಿ, ಸಮುದ್ರ ತೀರದಲ್ಲಿ, ಅಂಗಡಿಗಳಲ್ಲಿ, ಬೀದಿಬೀದಿಗಳಲ್ಲಿ ನೂರಾರು ಜನರು ಕುತೂಹಲದಿಂದ ಸ್ವಾಮಿಗಳು ಎಂದು ಬರುವರು ಎಂದು ಕೇಳುತ್ತಿರುವರು. ವಿಶ್ವವಿದ್ಯಾನಿಲಯ ಪರೀಕ್ಷೆಗೆ ದೂರದೂರದಿಂದ ಮದ್ರಾಸಿಗೆ ಬಂದ ವಿದ್ಯಾರ್ಥಿಗಳು ಪರೀಕ್ಷೆ ಮುಗಿದು ಹೋದರೂ, ಅವರ ತಂದೆ ತಾಯಿಗಳು ಬರುವಂತೆ ಒತ್ತಾಯದಿಂದ ಕಾಗದ ಬರೆದರೂ, ಅವರ ಹಾಸ್ಟಲಿನ ವೆಚ್ಚ ದಿನದಿನಕ್ಕೆ ಹೆಚ್ಚುತ್ತಿದ್ದರೂ, ಸ್ವಾಮಿಗಳ ಆಗಮನಕ್ಕಾಗಿ ಕಾಯುತ್ತಿರುವರು. ಇನ್ನು ಕೆಲವು ದಿನಗಳಲ್ಲಿ ಸ್ವಾಮಿಗಳು ಇಲ್ಲಿಗೆ ಬರುತ್ತಾರೆ. ಈ ಪ್ರಾಂತ್ಯದ ಬೇರೆ ಬೇರೆ ಊರಿನಲ್ಲಿ ಅವರಿಗೆ ಸಿಕ್ಕಿದ ಸ್ವಾಗತಕ್ಕೆ ತಕ್ಕಂತೆ ಇಲ್ಲಿ ಅದಕ್ಕಾಗಿ ಸಿದ್ಧತೆಗಳು ನಡೆಯುತ್ತಿರುವುವು. ಹಿಂದೂ ಸಾರ್ವಜನಿಕರಪರವಾಗಿ ಸ್ವಾಮಿಗಳು ತಂಗುವ ಕ್ಯಾಸಲ್​ ಕರ್ನನ್​ ಎಂಬ ಕಟ್ಟಡದ ಹತ್ತಿರ ಕಟ್ಟಿರುವ ಜಯಸೂಚಕ ಚಪ್ಪರಗಳನ್ನೂ ಊರಿನ ಪ್ರಮುಖರಾದ ಹೈಕೋರ್ಟಿನ ನ್ಯಾಯಾಧಿಪತಿ ಗಳಾದ ಜಸ್ಟಿಸ್​ ಸುಬ್ರಮಣ್ಯ ಅಯ್ಯರ್​ ಮುಂತಾದವರು ಸಮಾರಂಭದಲ್ಲಿ ವಹಿಸುತ್ತಿರುವ ಪ್ರಮುಖ ಪಾತ್ರವನ್ನೂ ನೋಡಿದರೆ, ಸ್ವಾಮೀಜಿ ಅವರಿಗೆ ಅಭೂತ ಪೂರ್ವ ಸ್ವಾಗತ ಸಿಕ್ಕುವುದರಲ್ಲಿ ಸಂದೇಹವಿಲ್ಲ ಎನ್ನಬಹುದು. ಸ್ವಾಮೀಜಿಯವರ ಮಹಾಪ್ರತಿಭೆಯನ್ನು ಮೊದಲು ಕಂಡುಹಿಡಿದು ಅವರನ್ನು ಚಿಕಾಗೋ ನಗರಕ್ಕೆ ಕಳುಹಿಸಿದ್ದು ಮದ್ರಾಸು. ತಮ್ಮ ಮಾತೃಭೂಮಿಯ ಗೌರವ ಧ್ವಜವನ್ನು ಮೇಲೆತ್ತು ವುದಕ್ಕೆ ಅಷ್ಟೊಂದು ಪ್ರಯತ್ನಪಟ್ಟ ಆ ಮಹಾಪುರುಷನನ್ನು ಗೌರವಿಸುವ ಮಹಾ ಭಾಗ್ಯ ಮದ್ರಾಸಿಗೆ ದೊರಕುವುದು. ನಾಲ್ಕು ವರುಷಗಳ ಹಿಂದೆ ಸ್ವಾಮೀಜಿ ಅವರು ಮದ್ರಾಸಿಗೆ ಬಂದಾಗ ಅವರೊಬ್ಬ ಅನಾಮಧೇಯ ವ್ಯಕ್ತಿಗಳಾಗಿದ್ದರು. ಸೆಂಟ್​ ಥೋಮಿನಲ್ಲಿ ಯಾರಿಗೂ ಗೊತ್ತಿಲ್ಲದಿದ್ದ ಒಂದು ನಿವಾಸದಲ್ಲಿ ಎರಡು ತಿಂಗಳು ತಂಗಿದ್ದು, ತಮ್ಮ ಬಳಿಗೆ ಬಂದವರಿಗೆಲ್ಲ ಧಾರ್ಮಿಕ ವಿಷಯಗಳನ್ನು ಬೋಧಿಸುತ್ತಿ ದ್ದರು. ಆಗಲೇ ವಿದ್ಯಾವಂತರಾದ, ಸೂಕ್ಷ್ಮದೃಷ್ಟಿಯುಳ್ಳ ಕೆಲವು ಯುವಕರಿಗೆ ಈ ಸ್ವಾಮೀಜಿಗಳಲ್ಲಿ ಯಾವುದೋ ಒಂದು ಅದ್ಭುತವಾದ, ಅನನ್ಯಸಾಧಾರಣವಾದ ತೇಜಸ್ಸಿದೆ, ನಿಸ್ಸಂದೇಹವಾಗಿ ಇವರು ನಾಯಕರಾಗಬಲ್ಲರು ಎಂಬುದು ಹೊಳೆದಿತ್ತು. “ಮತಿಗೆಟ್ಟ ಉತ್ಸಾಹಿಗಳು, ಭ್ರಾಂತ ಸುಧಾರಕರು” ಎನ್ನಿಸಿಕೊಂಡಿದ್ದ ಆ ಯುವಕರಿಗೆ ತಮ್ಮ ಸ್ವಾಮಿಗಳು ಯೂರೋಪ್​, ಅಮೆರಿಕಾ ದೇಶಗಳಿಂದ ಪ್ರಖ್ಯಾತಿಗಳಿಸಿ ಬಂದುದನ್ನು ನೋಡಿ ಪಾರವಿಲ್ಲದ ಸಂತೋಷವಾಗಿದೆ. ಸ್ವಾಮೀಜಿ ಅವರ ಸಂದೇಶ ಮುಖ್ಯವಾಗಿ ಆಧ್ಯಾತ್ಮಿಕವಾದುದು. ಆಧ್ಯಾತ್ಮಿಕ ಶಕ್ತಿಗೆ ತೌರೂರಾದ ಭರತಖಂಡಕ್ಕೆ ಭವ್ಯ ಭವಿಷ್ಯವಿದೆ ಎಂದು ಅವರು ದೃಢವಾಗಿ ನಂಬುವರು. ವೇದಾಂತದ ಗಹನ ಸಂದೇಶಗಳನ್ನು ಪಾಶ್ಚಾತ್ಯರು ಕ್ರಮೇಣ ಹೆಚ್ಚು ಹೆಚ್ಚು ಮೆಚ್ಚುವರು ಎಂದು ಅವರು ನಿಸ್ಸಂದೇಹವಾಗಿ ಭಾವಿಸುವರು. ಅವರ ಉಪದೇಶದ ಪಲ್ಲವಿಯೆ ‘ಸಹಾಯಮಾಡಿ, ಹೋರಾಡಬೇಡಿ. ಜೀರ್ಣಿಸಿಕೊಳ್ಳಿ ಧ್ವಂಸಮಾಡಬೇಡಿ, ಶಾಂತಿ ಮತ್ತು ಔದಾರ್ಯ, ವೈಮನಸ್ಯವಲ್ಲ’ ಎಂಬುದು. ಇತರ ಬೇರೆ ಬೇರೆ ಪಂಗಡದವರಲ್ಲಿ ಎಷ್ಟೇ ಭಿನ್ನಾ ಭಿಪ್ರಾಯವಿದ್ದರೂ ಸ್ವಾಮಿಗಳು ಹಿಂದೂವಿನಲ್ಲಿರುವ ಒಳ್ಳೆಯದನ್ನು ಪಾಶ್ಚಾತ್ಯ ರಿಗೆ ತೋರಿ ತಲೆತೂಗುವಂತೆ ಮಾಡಿರುವರು ಎಂದು ಎಲ್ಲರೂ ನಿರ್ವಿವಾದವಾಗಿ ಒಪ್ಪಿಕೊಳ್ಳಬೇಕಾಗಿದೆ. ಯಾವುದನ್ನು ತಾವು ಧಾರ್ಮಿಕ ಶಾಂತಿ ಎಂದು ನಂಬಿರುವರೋ ಅದನ್ನು ಧೈರ್ಯದಿಂದ ಸಮುದ್ರವನ್ನು ದಾಟಿ ಪಾಶ್ಚಾತ್ಯರಿಗೆ ಬೋಧಿಸಿದ ಪ್ರಥಮ ಹಿಂದೂ ಸಂನ್ಯಾಸಿ ಎಂದು ಎಲ್ಲರೂ ಅವರನ್ನು ಯಾವಾಗಲೂ ಸ್ಮರಿಸುವರು.

ನಮ್ಮ ಪತ್ರಿಕೆಯ ಬಾತ್ಮೀದಾರರೊಬ್ಬರು ಪಾಶ್ಚಾತ್ಯ ದೇಶಗಳಲ್ಲಿ ಸ್ವಾಮಿ ವಿವೇಕಾನಂದರ ಉದ್ದೇಶ ಹೇಗೆ ಜಯಪ್ರದವಾಯಿತು ಎಂದು ತಿಳಿಯುವುದ ಕ್ಕೋಸ್ಕರ ಅವರನ್ನು ಭೇಟಿಮಾಡಲು ಹೋದರು. ಸ್ವಾಮೀಜಿ ಅವರು ನಮ್ಮ ಬಾತ್ಮೀದಾರರನ್ನು ಗೌರವದಿಂದ ಬರಮಾಡಿಕೊಂಡು ತಮ್ಮ ಪಕ್ಕದಲ್ಲಿ ಒಂದು ಕುರ್ಚಿಯ ಮೇಲೆ ಕುಳಿತುಕೊಳ್ಳುವಂತೆ ಹೇಳಿದರು. ಸ್ವಾಮೀಜಿ ಅವರು ಗೈರಿಕ ವಸನಧಾರಿಗಳಾಗಿದ್ದರು. ಅವರ ವದನ ಪ್ರಶಾಂತವಾಗಿದ್ದು ಗಂಭೀರ ಮುದ್ರೆಯನ್ನು ಸೂಚಿಸುತ್ತಿತ್ತು. ಯಾವ ಪ್ರಶ್ನೆಯನ್ನು ಕೇಳಿದರೂ ಉತ್ತರ ಹೇಳುವುದಕ್ಕೆ ಸಿದ್ಧರಾಗಿ ದ್ದಂತೆ ತೋರಿತು. ನಮ್ಮ ಬಾತ್ಮೀದಾರರು ಶೀಘ್ರಲಿಪಿಯಲ್ಲಿ ಬರೆದುಕೊಂಡ ಸಂಭಾಷಣೆಯನ್ನು ಕೆಳಗೆ ಕೊಡುವೆವು:

ಪ್ರಶ್ನೆ: “ನಿಮ್ಮ ಬಾಲ್ಯಜೀವನದ ವಿಷಯವಾಗಿ ಏನನ್ನಾದರೂ ಹೇಳ ಬಲ್ಲಿರಾ?”

ಸ್ವಾಮೀಜಿ: “ನಾನು ಕಲ್ಕತ್ತೆಯಲ್ಲಿ ವಿದ್ಯಾರ್ಥಿಯಾಗಿದ್ದಾಗಲೇ ಆಧ್ಯಾತ್ಮಿಕ ಪ್ರವೃತ್ತಿಯವನಾಗಿದ್ದೆ. ಆಗಲೂ ಕೂಡ ನಾನು ನನ್ನ ಜೀವನವನ್ನೇ ವಿಮರ್ಶೆಮಾಡಿ ಕೊಳ್ಳುವ ಸ್ವಭಾವದವನಾಗಿದ್ದೆ. ಬರೀ ಮಾತು ನನಗೆ ತೃಪ್ತಿಯನ್ನು ಕೊಡುತ್ತಿರಲಿಲ್ಲ. ಕ್ರಮೇಣ ನಾನು ಶ‍್ರೀ ರಾಮಕೃಷ್ಣ ಪರಮಹಂಸರನ್ನು ಸಂದರ್ಶಿಸಿದೆ, ಅವರೊಡನೆ ಬಹಳ ಕಾಲ ಕಳೆದೆ, ಅವರಿಂದ ಹಲವು ವಿಷಯಗಳನ್ನು ಕಲಿತೆ. ನಮ್ಮ ತಂದೆಯ ಕಾಲಾನಂತರ ಭರತಖಂಡವನ್ನು ಸಂಚರಿಸಲು ಉಪಕ್ರಮಿಸಿದೆ. ಅನಂತರ ಕಲ್ಕತ್ತೆ ಯಲ್ಲಿ ಒಂದು ಮಠವನ್ನು ಸ್ಥಾಪಿಸಿದೆ. ನಾನು ಪರಿವ್ರಾಜಕನಾಗಿ ಅಲೆಯುತ್ತಿದ್ದಾಗ ಮದ್ರಾಸ್​ ಪ್ರಾಂತಕ್ಕೆ ಬಂದೆ. ಮೈಸೂರು ಮತ್ತು ರಾಮನಾಡಿನ ಮಹಾರಾಜರಿಂದ ಸಹಾಯ ದೊರಕಿತು.”

ಪ್ರಶ್ನೆ: “ಹಿಂದೂಧರ್ಮದ ಸಂದೇಶವನ್ನು ಪಾಶ್ಚಾತ್ಯ ದೇಶಗಳಿಗೆ ಒಯ್ಯುವಂತೆ ನಿಮ್ಮನ್ನು ಯಾವುದು ಪ್ರೇರೇಪಿಸಿತು?”

ಸ್ವಾಮೀಜಿ: “ನನಗೆ ಅನುಭವ ಬೇಕಾಗಿತ್ತು. ನಮ್ಮ ದೇಶದ ಅವನತಿಗೆ ಮುಖ್ಯ ಕಾರಣವೇ ನಾವು ಇತರ ದೇಶಗಳೊಡನೆ ಬೆರೆಯದಿರುವುದು. ಇದೇ ಏಕಮಾತ್ರ ಮುಖ್ಯ ಕಾರಣ. ನಾವು ಕೂಪಮಂಡೂಕಗಳಾಗಿರುವೆವು.”

ಪ್ರಶ್ನೆ: “ನೀವು ಪಾಶ್ಚಾತ್ಯ ದೇಶಗಳಲ್ಲಿ ಬೇಕಾದಷ್ಟು ಸಂಚರಿಸಿರಬಹುದಲ್ಲವೆ?

ಸ್ವಾಮೀಜಿ: “ನಾನು ಜರ್ಮನಿ, ಫ್ರಾನ್ಸ್​ ಸಹಿತ ಯೂರೋಪಿನ ಬಹುಭಾಗ ವನ್ನು ನೋಡಿರುವೆನು. ಆದರೆ ನನ್ನ ಕಾರ್ಯಗಳ ಮುಖ್ಯ ಕೇಂದ್ರ ಅಮೆರಿಕ ಮತ್ತು ಇಂಗ್ಲೆಂಡ್​. ಮೊದಲು ನನ್ನ ಸ್ಥಿತಿ ಚಿಂತಾಜನಕವೇ ಆಗಿತ್ತು. ಅದಕ್ಕೆ ಕಾರಣ ಭರತಖಂಡದಿಂದ ಅಲ್ಲಿಗೆ ಹೋದವರು ಇಲ್ಲಿಯ ಜನರನ್ನು ವಿರೋಧಿಸತೊಡ ಗಿದ್ದುದು. ಪ್ರಪಂಚದಲ್ಲಿ ಹಿಂದೂಗಳಷ್ಟು ಧಾರ್ಮಿಕರು ಮತ್ತು ನೀತಿವಂತರು ಮತ್ತಾರೂ ಇಲ್ಲ. ಹಿಂದೂ ಜನಾಂಗವನ್ನು ಇತರರೊಡನೆ ಹೋಲಿಸುವುದೇ ಈಶ್ವರನಿಂದೆ. ಮೊದ ಮೊದಲು ಜನ ನನ್ನ ಮೇಲೆ ಸಿಕ್ಕಾಪಟ್ಟೆ ಅಪವಾದಗಳನ್ನು ಹೇರಿದರು. ನನ್ನನ್ನು ಒಬ್ಬ ಮೋಸಗಾರನೆಂದು ಕರೆದರು. ನನಗೆ ಒಂದು ಜನಾನಾದ ತುಂಬ ಹೆಂಡರಿರುವರೆಂದೂ, ದೊಡ್ಡ ಒಂದು ಮಕ್ಕಳ ಸೇನೆಯೇ ಇದೆ ಎಂದೂ ಪ್ರಚಾರ ಮಾಡಿದರು. ಧರ್ಮದ ಹೆಸರಿನಲ್ಲಿ ಆ ಪಾದ್ರಿಗಳು ಏನನ್ನು ಮಾಡಬಲ್ಲರು ಎಂದು ನನಗೆ ಆಗ ಗೊತ್ತಾಯಿತು. ಇಂಗ್ಲೆಂಡಿನಲ್ಲಿ ಮಿಷನರಿಗಳನ್ನು ಕೇಳುವವರೇ ಇಲ್ಲ. ನನ್ನ ಹತ್ತಿರ ವ್ಯಾಜ್ಯವಾಡಲು ಯಾರೂ ಬರಲಿಲ್ಲ. ಮಿಸ್ಟರ್​ ಲಂಡ್​ ನನ್ನನ್ನು ನಿಂದಿಸಲು, ನಾನಿಲ್ಲಿದ್ದಾಗಲೇ ಅಮೆರಿಕ ದೇಶಕ್ಕೆ ಹೋದ. ಜನ ಯಾರೂ ಅವನ ಮಾತನ್ನು ಕೇಳಲಿಲ್ಲ. ಅಲ್ಲಿಯವರೆಗೆ ನಾನು ತುಂಬಾ ಬೇಕಾದವನಾಗಿದ್ದೆ. ನಾನು ಇಂಗ್ಲೆಂಡಿಗೆ ಬಂದಾಗ ಈ ಮಿಷನರಿಗಳು ಅಪಪ್ರಚಾರ ಮಾಡಲು ಇಲ್ಲಿಗೂ ಬರುತ್ತಾರೆ ಎಂದು ತಿಳಿದಿದ್ದೆ. ಆದರೆ ಸತ್ಯ ಅವರ ಬಾಯಿ ಮುಚ್ಚುವಂತೆ ಮಾಡಿತು. ಇಂಗ್ಲೆಂಡಿನಲ್ಲಿ ಸಾಮಾಜಿಕ ಅಂತಸ್ತು ಪ್ರಜ್ಞೆ ನಮ್ಮ ಜಾತಿವಿಚಾರಕ್ಕಿಂತ ಹೆಚ್ಚಾಗಿದೆ. ಇಂಗ್ಲಿಷ್​ನ ಚರ್ಚಿಗೆ ಸೇರಿದವರೆಲ್ಲ ಗೌರವಸ್ಥರು. ಆದರೆ ಮಿಷನರಿಗಳು ಹಾಗಿಲ್ಲ. ಇಂಗ್ಲಿಷಿನವರು ನನಗೆ ಹೆಚ್ಚು ಸಹಾನುಭೂತಿಯನ್ನು ತೋರುವರು. ಇಂಗ್ಲಿಷ್​ ಚರ್ಚಿನ ಸುಮಾರು ಮೂವತ್ತು ಜನ ಪಾದ್ರಿಗಳು ನನ್ನ ಧಾರ್ಮಿಕ ಅಭಿಪ್ರಾಯ ಗಳನ್ನೆಲ್ಲಾ ಅನುಮೋದಿಸುವರು. ಇಂಗ್ಲಿಷ್​ನ ಪಾದ್ರಿಗಳು ನನ್ನನ್ನು ಒಪ್ಪದೇ ಇದ್ದರೂ ನನ್ನ ಬೆನ್ನ ಹಿಂದೆ ಅವರು ನನ್ನನ್ನು ನಿಂದಿಸಲಿಲ್ಲ, ಅಥವಾ ಕತ್ತಲೆಯಲ್ಲಿ ಚೂರಿ ಹಾಕಲಿಲ್ಲ. ಇದು ನನಗೆ ಹಿತಕರವಾದ ರೀತಿಯಲ್ಲಿ ಆಶ್ಚರ್ಯವನ್ನುಂಟು ಮಾಡಿತು. ಅವರಲ್ಲಿ ಜಾತಿಯ ಮತ್ತು ವಂಶಾನುಗತವಾಗಿ ಬಂದ ಸಂಸ್ಕೃತಿಯ ಪ್ರಭಾವವಿದೆ.”

ಪ್ರಶ್ನೆ: “ಪಾಶ್ಚಾತ್ಯ ದೇಶಗಳಲ್ಲಿ ನಿಮ್ಮ ಕೆಲಸ ಎಷ್ಟರ ಮಟ್ಟಿಗೆ ಯಶಸ್ವಿ ಯಾಯಿತು?”

ಸ್ವಾಮೀಜಿ: “ಇಂಗ್ಲೆಂಡಿಗಿಂತ ಹೆಚ್ಚಾಗಿ ಅಮೆರಿಕದಲ್ಲಿ ಜನರು ನನಗೆ ಸಹಾನುಭೂತಿ ತೋರಿದರು. ಕೀಳು ದರ್ಜೆಗೆ ಸೇರಿದ ಮಿಷನರಿಗಳ ಅಪಪ್ರಚಾರವು ನನ್ನ ಉದ್ದೇಶವು ಜಯಪ್ರದವಾಗಲು ಸಹಾಯಕವಾಯಿತೆಂದೇ ಹೇಳಬೇಕು. ನನ್ನಲ್ಲಿ ದುಡ್ಡಿರಲಿಲ್ಲ. ಇಂಡಿಯಾ ದೇಶದ ಜನರು ಬರೀ ನನ್ನ ಪ್ರಯಾಣಕ್ಕೆ ಕೊಟ್ಟಿದ್ದ ಹಣ ಬೇಗ ಖರ್ಚಾಗಿ ಹೋಯಿತು. ನಾನು ಇಲ್ಲಿ ಜೀವಿಸುವಂತೆಯೇ ಕೆಲವು ಸ್ನೇಹಿತರ ಔದಾರ್ಯವನ್ನು ಅವಲಂಬಿಸಿ ಬದುಕಬೇಕಾಯಿತು. ಅಮೆರಿಕ ಜನರು ಅತಿಥಿ ಸತ್ಕಾರಪರರು. ಅಮೆರಿಕ ದೇಶದಲ್ಲಿ ಮೂರನೆ ಒಂದು ಭಾಗ ಕ್ರೈಸ್ತರು. ಉಳಿದವರಿಗೆ ಯಾವ ಧರ್ಮವೂ ಇಲ್ಲ. ಅಂದರೆ ಅವರು ಯಾವ ಚರ್ಚಿಗೂ ಹೋಗುವುದಿಲ್ಲ. ಆದರೆ ಇಂತಹವರ ಪೈಕಿಯಲ್ಲಿ ದೊಡ್ಡ ದೊಡ್ಡ ಆಧ್ಯಾತ್ಮಿಕ ವ್ಯಕ್ತಿಗಳು ಇರುವರು. ನಾನು ಇಂಗ್ಲೆಂಡಿನಲ್ಲಿ ಮಾಡಿದ ಕೆಲಸ ಭದ್ರವಾಗಿದೆ ಎಂದು ಭಾವಿಸುತ್ತೇನೆ. ನಾನು ನಾಳೆಯೇ ಕಾಲವಾಗಿ ಹೋದರೂ ಯಾವ ಪ್ರಚಾರಕರನ್ನೂ ಅಲ್ಲಿಗೆ ಕಳುಹಿಸದೆ ಇದ್ದರೂ ನಾನು ಇಂಗ್ಲೆಂಡಿನಲ್ಲಿ ಮಾಡಿರುವ ಕೆಲಸ ವ್ಯರ್ಥ ವಾಗುವುದಿಲ್ಲ. ಆಂಗ್ಲೇಯರು ಬಹಳ ಒಳ್ಳೆಯವರು. ಬಾಲ್ಯದಿಂದಲೇ ಅವರು ತಮ್ಮ ಉದ್ವೇಗವನ್ನು ತಡೆದುಕೊಳ್ಳುವಂತೆ ಅವರಿಗೆ ಶಿಕ್ಷಣ ನೀಡಲಾಗುವುದು. ಅವರು ವಿಷಯವನ್ನು ತಿಳಿದುಕೊಳ್ಳುವುದು ಸ್ವಲ್ಪ ನಿಧಾನ. ಫ್ರೆಂಚ್​ ಮತ್ತು ಅಮೆರಿಕ ದವರಂತೆ ತಕ್ಷಣ ವಿಷಯವನ್ನು ಗ್ರಹಿಸುವುದಿಲ್ಲ. ಆಂಗ್ಲೇಯರು ವ್ಯವಹಾರ ಚತುರರು. ಅಮೆರಿಕ ಜನ ತ್ಯಾಗವನ್ನು ಅರ್ಥಮಾಡಿಕೊಳ್ಳಲಾರರು. ಅವರಿನ್ನೂ ಬೆಳೆಯಬೇಕಾಗಿದೆ. ಇಂಗ್ಲೆಂಡ್​ ಐಶ್ವರ್ಯ, ಭೋಗಗಳನ್ನು ಹಲವು ಶತಮಾನಗಳಕಾಲ ಅನುಭವಿಸಿದೆ. ಅಲ್ಲಿ ಅನೇಕ ಜನ ತ್ಯಾಗಕ್ಕೆ ಸಿದ್ಧರಾಗಿರುವರು. ನಾನು ಇಂಗ್ಲೆಂಡಿನಲ್ಲಿ ಮೊದಲು ಮಾತನಾಡಿದಾಗ ಕೇಳುವುದಕ್ಕೆ ಇಪ್ಪತ್ತು ಮೂವತ್ತು ಜನ ಬರುತ್ತಿದ್ದರು. ನಾನು ಇಂಗ್ಲೆಂಡನ್ನು ಬಿಟ್ಟಾಗ ಆ ಕ್ಲಾಸನ್ನು ಮುಂದುವರಿಸಲು ಬೇರೊಬ್ಬ ಸ್ವಾಮಿಗಳನ್ನು ಬಿಟ್ಟೆ. ನಾನು ಪುನಃ ಅಮೆರಿಕದಿಂದ ಬಂದ ಮೇಲೆ ನನ್ನ ಕ್ಲಾಸಿಗೆ ಒಂದು ಸಾವಿರ ಜನ ಬರುತ್ತಿದ್ದರು. ಅಮೆರಿಕ ದೇಶದಲ್ಲಿ ನನ್ನ ಉಪನ್ಯಾಸಕ್ಕೆ ಇನ್ನೂ ಹೆಚ್ಚಿನ ಜನ ನೆರೆಯುತ್ತಿದ್ದರು. ಏಕೆಂದರೆ ನಾನು ಅಲ್ಲಿ ಮೂರು ವರುಷಗಳಿದ್ದೆ. ಇಂಗ್ಲೆಂಡಿನಲ್ಲಾದರೊ ಇದ್ದದ್ದು ಒಂದೇ ವರುಷ. ಈಗ ಒಬ್ಬ ಸಂನ್ಯಾಸಿಗಳು ಅಮೆರಿಕದಲ್ಲಿಯೂ ಮತ್ತೊಬ್ಬರು ಇಂಗ್ಲೆಂಡಿನಲ್ಲಿಯೂ ಇರುವರು. ಇತರ ದೇಶಗಳಿಗೂ ನಾನು ಪ್ರಚಾರಕರನ್ನು ಕಳುಹಿಸಬೇಕೆಂದಿರುವೆನು.

“ಆಂಗ್ಲೇಯರು ಅದ್ಭುತ ಕರ್ಮಪಟುಗಳು. ನೀವು ಅವರಿಗೆ ಏನಾದರೂ ಹೇಳಿದರೆ, ಅವರು ಅದನ್ನು ಚೆನ್ನಾಗಿ ತಿಳಿದುಕೊಂಡಿದ್ದರೆ ಆ ಭಾವನೆ ಎಂದಿಗೂ ವ್ಯರ್ಥವಾಗುವುದಿಲ್ಲ. ಇಲ್ಲಿಯ ಜನ ವೇದಗಳನ್ನೆಲ್ಲ ಬಿಟ್ಟಿರುವರು. ನಿಮ್ಮ ವೇದಾಂತ ವೆಲ್ಲ ಅಡಿಗೆ ಮನೆಯಲ್ಲಿದೆ. ಸದ್ಯಕ್ಕೆ ಇಂಡಿಯಾ ದೇಶದಲ್ಲಿರುವ ಧರ್ಮವೇ ‘ನನ್ನನ್ನು ಮುಟ್ಟಬೇಡಿ’ ಎನ್ನುವುದು. ಆಂಗ್ಲೇಯರು ಎಂದಿಗೂ ಇಂತಹ ಧರ್ಮವನ್ನು ಸ್ವೀಕರಿಸ ಲಾರರು. ನಮ್ಮ ಪೂರ್ವಿಕರ ಮಹದಾಲೋಚನೆಗಳು, ಅವರು ಕಂಡುಹಿಡಿದ ಜೀವಪೋಷಕ ಸಂದೇಶಗಳನ್ನು ಪ್ರತಿಯೊಂದು ದೇಶವೂ ಸ್ವೀಕರಿಸುವುದು. ಇಂಗ್ಲಿಷ್​ ಚರ್ಚಿನ ಅನೇಕ ಪ್ರಮುಖರು ನಾನಾ ವೇದಾಂತವನ್ನು ಬೈಬಲ್ಲಿಗೆ ಸೇರಿಸುತ್ತಿರುವೆ ಎಂದರು. ಈಗ ಹಿಂದೂಧರ್ಮ ಅವನತಿಯಲ್ಲಿರುವುದು. ಸ್ವಲ್ಪವಾದರೂ ವೇದಾಂತ ಭಾವನೆಯನ್ನು ಒಳಗೊಳ್ಳದ ಯಾವುದೇ ದಾರ್ಶನಿಕ ಗ್ರಂಥವೂ ಇಂದು ರಚಿತ ವಾಗಿಲ್ಲ. ಹರ್​ಬರ್ಟ್​ ಸ್ಟೆನ್ಸರ್​ನ ಗ್ರಂಥಗಳಲ್ಲಿಯೂ ಅದು ಇದೆ. ಈ ಯುಗದ ದರ್ಶನ ಅದ್ವೈತ ತತ್ತ್ವ.ಪ್ರತಿಯೊಬ್ಬರೂ ಇದನ್ನು ಕುರಿತು ಮಾತನಾಡುತ್ತಿರುವರು. ಯೂರೋಪಿನಲ್ಲಿ ಮಾತ್ರ ಸ್ವಂತಿಕೆಯನ್ನು ವ್ಯಕ್ತಪಡಿಸಲು ಯತ್ನಿಸುತ್ತಿರುವರು. ಅವರು ಹಿಂದೂಗಳನ್ನು ನಿಕೃಷ್ಟದೃಷ್ಟಿಯಿಂದ ನೋಡಿದರೂ ಹಿಂದೂಗಳ ಭಾವನೆಗಳನ್ನೆಲ್ಲಾ ಸ್ವೀಕರಿಸುತ್ತಿರುವರು. ಪ್ರೊಫೆಸರ್​ ಮ್ಯಾಕ್ಸಮುಲ್ಲರು ಒಳ್ಳೆಯ ವೇದಾಂತಿಗಳು. ಅವರು ವೇದಾಂತದ ಮೇಲೆ ಶ್ಲಾಘನೀಯ ಕೆಲಸವನ್ನು ಮಾಡಿರುವರು. ಅವರಿಗೆ ಜನ್ಮಾಂತರದಲ್ಲಿ ನಂಬಿಕೆಯುಂಟು.”

ಪ್ರಶ್ನೆ: “ಇಂಡಿಯಾ ದೇಶದ ಪುನರುದ್ಧಾರಕ್ಕೆ ನೀವು ಏನು ಮಾಡಬೇಕೆಂದಿರು ವಿರಿ?”

ಸ್ವಾಮೀಜಿ: “ನಮ್ಮ ಜನಾಂಗದ ಮಹಾಪಾತಕವೇ ಜನಸಾಮಾನ್ಯರನ್ನು ನಿರ್ಲಕ್ಷಿಸಿದ್ದು. ಇದೇ ನಮ್ಮ ಅವನತಿಗೆ ಮುಖ್ಯ ಕಾರಣ. ಜನಸಾಮಾನ್ಯರು ವಿದ್ಯಾ ವಂತರಾಗಿ ಅವರಿಗೆ ಊಟಕ್ಕೆ ಬಟ್ಟೆಗೆ ಸಾಕಷ್ಟು ಸಿಕ್ಕುವವರೆಗೆ ಯಾವ ರಾಜಕೀಯ ದಿಂದಲೂ ಏನೂ ಪ್ರಯೋಜನವಿಲ್ಲ. ಅವರು ನಮ್ಮ ವಿದ್ಯಾಭ್ಯಾಸಕ್ಕೆ ಕಂದಾಯ ಕೊಡುವರು, ನಮ್ಮ ದೇವಸ್ಥಾನ ಕಟ್ಟುವರು. ಅದಕ್ಕೆ ಬದಲು ನಮ್ಮಿಂದ ಅವರಿಗೆ ದೊರಕುವುದು ನಿಂದೆ; ಅವರು ನಿಜವಾಗಿ ನಮ್ಮ ಗುಲಾಮರಾಗಿರುವರು. ನಾವುಭರತಖಂಡವನ್ನು ಉದ್ಧಾರ ಮಾಡಬೇಕಾದರೆ ಅವರಿಗಾಗಿ ಕೆಲಸ ಮಾಡಬೇಕು. ಯುವಕರನ್ನು ಪ್ರಚಾರಕರನ್ನಾಗಿ ತಯಾರು ಮಾಡಲು ಕಲ್ಕತದಲ್ಲಿ ಒಂದು, ಮದ್ರಾಸಿ ನಲ್ಲಿ ಒಂದು, ಹೀಗೆ ಎರಡು ಕೇಂದ್ರಗಳನ್ನು ತೆರೆಯಬೇಕೆಂದಿದ್ದೇನೆ. ಕಲ್ಕತ್ತಾ ಕೇಂದ್ರವನ್ನು ಸ್ಥಾಪಿಸಲು ಬೇಕಾದ ಹಣ ನನ್ನಲ್ಲಿದೆ. ಆಂಗ್ಲೇಯರು ನನಗೆ ಅದಕ್ಕೆ ದುಡ್ಡನ್ನು ಕೊಡುವರು.”

“ನನ್ನ ಭರವಸೆಯೆಲ್ಲ ಈಗಿನ ಯವಕರ ಮೇಲೆ ನಿಂತಿದೆ. ಇದರಿಂದ ಕೆಲಸ ಮಾಡುವ ಯುವಕರು ನಮಗೆ ಬರಬೇಕಾಗಿದೆ. ಅನಂತರ ಅವರು ಕೆಚ್ಚೆದೆಯ ಸಿಂಹ ದಂತೆ ಕೆಲಸಮಾಡುವರು. ನಾನು ಹೇಗೆ ಮಾಡಬೇಕೆಂಬುದನ್ನು ಹೇಳಿರುವೆನು. ನಾನು ಅದಕ್ಕೆ ನನ್ನ ಜನ್ಮವನ್ನೇ ಕೊಟ್ಟಿರುವೆನು ನಾನು ಜಯಶೀಲನಾಗದೆ ಹೋದರೆ, ಅದನ್ನು ಮಾಡಲು ಮತ್ತೊಬ್ಬರು ಬರುವರು. ಹೋರಾಟದಲ್ಲೇ ನನಗೆ ತೃಪ್ತಿ. ಜನಸಾಮಾನ್ಯರಿಗೆ ಅವರ ಹಕ್ಕುಗಳನ್ನು ನೀಡಬೇಕು, ಅದೇ ಸಮಸ್ಯೆ. ಜಗತ್ತಿನಲ್ಲೇ ಶ್ರೇಷ್ಠವಾದ ಧರ್ಮ ನಿಮ್ಮಲ್ಲಿದೆ. ಆದರೆ ನೀವು ಜನರಿಗೆ ಕೆಲಸಕ್ಕೆ ಬಾರದ ಮೂಢ ನಂಬಿಕೆಯನ್ನು ಕೊಡುವಿರಿ. ಅಮೃತಪ್ರವಾಹ ಹತ್ತಿರವೇ ಹರಿಯುತ್ತಿದೆ. ನೀವು ಅವರಿಗೆ ಚರಂಡಿಯ ನೀರನ್ನು ಕೊಡುತ್ತಿರುವಿರಿ. ನಿಮ್ಮ ಮದ್ರಾಸಿನ ಪದವೀಧರರು ಪರೆಯನನ್ನು ಮುಟ್ಟಲಾರರು. ಆದರೆ ತಮ್ಮ ವಿದ್ಯಾವ್ಯಾಸಂಗಕ್ಕೆ ಹಣವನ್ನು ಅವ ನಿಂದ ಸ್ವೀಕರಿಸಲು ಸಿದ್ಧವಾಗಿರುವರು. ನಾನು ಪ್ರಚಾರಕರನ್ನು ತರಬೇತುಮಾಡಲು ಎರಡು ಕೇಂದ್ರಗಳನ್ನು ತೆರೆಯುವೆನು. ಅವರು ನಮ್ಮ ಜನರಿಗೆ ಲೌಕಿಕ ಮತ್ತು ಆಧ್ಯಾತ್ಮಿಕ ಶಿಕ್ಷಣಗಳೆರಡನ್ನೂ ನೀಡುವರು. ಅವರು ಭರತಖಂಡದ ಎಲ್ಲ ಕೇಂದ್ರ ಗಳಲ್ಲೂ ಕೆಲಸವನ್ನು ಪೂರೈಸುವವರೆಗೆ ಕೇಂದ್ರದಿಂದ ಕೇಂದ್ರಕ್ಕೆ ಹೋಗುತ್ತಿರುವರು. ದೇವರಲ್ಲಿಡುವ ಶ್ರದ್ಧೆಗಿಂತ ಮೊದಲು ಆತ್ಮಶ್ರದ್ಧೆ ಬೇಕಾಗಿದೆ, ಅದು ಮುಖ್ಯ. ಆದರೆ ದಿನಕಳೆದಂತೆ ನಾವು ಆತ್ಮಗೌರವವನ್ನು ಕಳೆದುಕೊಳ್ಳುತ್ತಿರುವೆವು. ನಾನು ಸುಧಾರಕರನ್ನು ಇದಕ್ಕೇ ಆಕ್ಷೇಪಿಸುವುದು. ಆಚಾರಶೀಲರಲ್ಲಿ ಸ್ವಲ್ಪ ಮೂಢನಂಬಿಕೆ ಇದ್ದರೂ ಅವರಲ್ಲಿ ಶ್ರದ್ಧೆ ಇದೆ, ಶಕ್ತಿ ಇದೆ. ಆದರೆ ಸುಧಾರಕರಾದರೋ ಸುಮ್ಮನೆ ಪಾಶ್ಚಾತ್ಯರನ್ನು ಅನುಸರಿಸಿ ಅವರು ಹೇಳಿದಂತೆ ಕೇಳುತ್ತಾರೆ. ಇತರ ದೇಶದ ಜನ ಸಾಮಾನ್ಯರೊಡನೆ ನಮ್ಮ ಜನಸಾಮಾನ್ಯರನ್ನು ಹೋಲಿಸಿದರೆ ನಮ್ಮ ಜನ ದೇವತೆ ಗಳಂತೆ ಇರುವರು. ಬಡತನವು ಒಂದು ಅಪರಾಧವಲ್ಲವೆಂದು ತಿಳಿದಿರುವುದು ಈ ದೇಶದಲ್ಲಿ ಮಾತ್ರ. ಅವರು ದೈಹಿಕವಾಗಿ, ಮಾನಸಿಕವಾಗಿ ಚೆನ್ನಾಗಿರುವರು. ಆದರೆ ನಾವಾದರೋ ಅವರನ್ನು ನಿಕೃಷ್ಟ ದೃಷ್ಟಿಯಿಂದ ನೋಡುತ್ತಾ ಹೋದೆವು. ಈಗ ಅವರು ತಮ್ಮ ಶ್ರದ್ಧೆಯನ್ನೆಲ್ಲಾ ಕಳೆದುಕೊಂಡಿರುವರು. ತಾವು ಹುಟ್ಟು ಗುಲಾಮರು ಎಂದು ಅವರು ಭಾವಿಸುವರು. ಅವರ ಹಕ್ಕನ್ನು ಅವರಿಗೆ ಕೊಡಿ. ಸ್ವತಂತ್ರವಾಗಿ ಅವರು ಅದರ ಮೇಲೆ ನಿಂತುಕೊಳ್ಳಲಿ. ಅಮೆರಿಕದ ನಾಗರಿಕತೆಯ ಮಹಿಮೆಯೇ ಇದು. ಮಂಡಿ ಬಾಗಿ, ಹೊಟ್ಟೆಗಿಲ್ಲದೆ, ಒಂದು ಕೋಲು ಮತ್ತು ಒಂದು ಚಿಂದಿ ಗಂಟನ್ನು ತೆಗೆದುಕೊಂಡು ಅಮೆರಿಕಾ ದೇಶಕ್ಕೆ ಬರುವ ಐರಿಷ್​ ಜನ ಆರು ತಿಂಗಳಮೇಲೆ ಅಮೆರಿಕಾ ದೇಶದಲ್ಲಿ ಹೇಗೆ ಇರುವರು ಎಂಬುದನ್ನು ನೋಡಿ. ಧೈರ್ಯವಾಗಿ ಅವರು ನಡೆಯುತ್ತಾರೆ. ಗುಲಾಮನಾಗಿದ್ದ ದೇಶದಿಂದ ಬಂದವನೊಬ್ಬ, ಆ ದೇಶದಲ್ಲಿ ಸಹೋದರನಂತೆ ಇದ್ದಾನೆ.

“ಆತ್ಮವು ಸರ್ವಶಕ್ತವಾದುದು, ಅನಂತವಾದುದು, ಅಮರವಾದುದು ಎಂದು ದೃಢವಾಗಿ ನಂಬಿ. ನನ್ನ ವಿದ್ಯಾಭ್ಯಾಸದ ಆದರ್ಶವೆಂದರೆ ಗುರುಗೃಹವಾಸ. ಗುರುವಿನ ಪ್ರತ್ಯಕ್ಷ ಜೀವನದ ಮೇಲ್ಪಂಕ್ತಿ ಇಲ್ಲದೆ ನಿಜವಾದ ವಿದ್ಯಾಭ್ಯಾಸವಿಲ್ಲ. ನಮ್ಮ ವಿಶ್ವವಿದ್ಯಾನಿಲಯಗಳನ್ನು ತೆಗೆದುಕೊಳ್ಳಿ. ಐವತ್ತು ವರುಷಗಳಿಂದ ಅವರೇನು ಮಾಡಿರುವರು? ಸ್ವತಂತ್ರವಾಗಿ ಆಲೋಚಿಸಬಲ್ಲ ಒಬ್ಬನೇ ಒಬ್ಬನನ್ನೂ ಅದು ಸೃಷ್ಟಿಸಿಲ್ಲ. ವಿಶ್ವವಿದ್ಯಾನಿಲಯವು ಪರೀಕ್ಷೆಗಳನ್ನು ನಡೆಸುವ ಸಂಸ್ಥೆಗಳಾಗಿವೆ. ಎಲ್ಲರ ಹಿತಕ್ಕಾಗಿ ತ್ಯಾಗಮಾಡಬೇಕೆಂಬ ಭಾವನೆ ನಮ್ಮ ದೇಶದಲ್ಲಿ ಇನ್ನೂ ಬೆಳೆದಿಲ್ಲ.

ಪ್ರಶ್ನೆ: “ಶ‍್ರೀಮತಿ ಬೆಸೆಂಟ್​ ಮತ್ತು ಥಿಯಾಸಫಿಯ ವಿಷಯದಲ್ಲಿ ನಿಮ್ಮ ಅಭಿಪ್ರಾಯವೇನು?”

ಸ್ವಾಮೀಜಿ: “ಶ‍್ರೀಮತಿ ಬೆಸೆಂಟ್​ ಬಹಳ ಒಳ್ಳೆಯ ಮಹಿಳೆ. ನಾನು ಲಂಡನ್ನಿ ನಲ್ಲಿ ಆಕೆಯ ಮನೆಯಲ್ಲಿ ಉಪನ್ಯಾಸ ಮಾಡಿದೆ. ನನಗೆ ಆಕೆಯ ವಿಷಯ ಹೆಚ್ಚು ಗೊತ್ತಿಲ್ಲ. ಆಕೆಗೆ ನಮ್ಮ ಧರ್ಮ ಅಷ್ಟು ತಿಳಿಯದು. ಆಕೆ ಎಲ್ಲೋ ಅಲ್ಲೊಂದು ಚೂರು ಇಲ್ಲೊಂದು ಚೂರನ್ನು ಆಯ್ದುಕೊಂಡಿರುವಳು. ಆಕೆಗೆ ಸರಿಯಾಗಿ ಹಿಂದೂ ಶಾಸ್ತ್ರವನ್ನು ಓದುವುದಕ್ಕೆ ಸಮಯವೇ ಇರಲಿಲ್ಲ. ಅವಳ ಪರಮ ಶತ್ರುಗಳು ಕೂಡ, ಆಕೆ ಪರಮ ಪ್ರಾಮಾಣಿಕಳು ಎಂಬುದನ್ನು ಒಪ್ಪಿಕೊಳ್ಳುವರು. ಇಂಗ್ಲೆಂಡಿ ನಲ್ಲೂ ಆಕೆ ಶ್ರೇಷ್ಠ ವಾಗ್ಮಿ ಎಂದು ಪರಿಗಣಿಸಲ್ಪಟ್ಟಿರುವಳು. ಆಕೆ ಸಂನ್ಯಾಸಿನಿ, ಆದರೆ ನಾನು ಮಹಾತ್ಮ ಕುತುಮಿ ಮುಂತಾದುವುಗಳಲ್ಲಿ ನಂಬುವುದಿಲ್ಲ. ಆಕೆ ಥಿಯಾಸಫಿಕಲ್​ ಸೊಸೈಟಿಯೊಂದಿಗೆ ತನ್ನ ಸಂಬಂಧವನ್ನು ತ್ಯಜಿಸಿ ಸ್ವತಂತ್ರವಾಗಿ ಯಾವುದನ್ನು ಸರಿ ಎಂದು ತಿಳಿದುಕೊಂಡಿರುವಳೊ ಅದನ್ನು ಬೋಧಿಸಲಿ.”

ಸಮಾಜ ಸುಧಾರಣೆಯ ವಿಷಯ ಮಾತನಾಡುವಾಗ ವಿಧವಾ ವಿವಾಹದ ವಿಷಯದಲ್ಲಿ ಅವರು ಹೀಗೆ ಹೇಳಿದರು: “ಒಂದು ದೇಶದ ಅದೃಷ್ಟವು ಅಲ್ಲಿಯ ವಿಧವೆಯರಿಗೆ ಸಿಕ್ಕುವ ಗಂಡಂದಿರ ಸಂಖ್ಯೆಯ ಮೇಲೆ ನಿಂತಿರುವುದನ್ನು ನಾನಿನ್ನೂ ನೋಡಬೇಕಾಗಿದೆ.”

ಇನ್ನೂ ಹಲವರು ಸ್ವಾಮೀಜಿಯವರನ್ನು ನೋಡಲು ಕೆಳಗೆ ಕಾತರರಾಗಿರು ವುದನ್ನು ಕಂಡು ನಮ್ಮ ಬಾತ್ಮೀದಾರರು ಸ್ವಾಮೀಜಿಗೆ ತಾವು ಕೊಟ್ಟ ಪತ್ರಿಕೋದ್ಯಮ ಕಿರುಕುಳವನ್ನು ಅವರು ಎಷ್ಟು ಸಹನೆಯಿಂದ ಸಹಿಸಿದರು ಎಂಬುದಕ್ಕಾಗಿ ವಂದನಾರ್ಪಣೆಯನ್ನು ಸಲ್ಲಿಸಿ ಹಿಂತಿರುಗಿದರು.

ಸ್ವಾಮೀಜಿ ಅವರ ಜೊತೆಯಲ್ಲಿ ಜೆ.ಎಚ್​. ಸೇವಿಯರ್​ ದಂಪತಿಗಳು, ಮಿಸ್ಟರ್​ ಟಿ.ಜಿ. ಹ್ಯಾರಿಸನ್​, ಕೊಲಂಬೊ ನಗರದ ಒಬ್ಬ ಬೌದ್ಧರು ಮತ್ತು ಜೆ.ಜೆ. ಗುಡ್​ವಿನ್​ ಇವರು ಬಂದಿರುವರು. ಸೇವಿಯರ್​ ದಂಪತಿಗಳು ಹಿಮಾಲಯದಲ್ಲಿ ನೆಲೆಸಲು ಸ್ವಾಮೀಜಿ ಅವರೊಂದಿಗೆ ಹೋಗುತ್ತಿರುವರಂತೆ. ಅಲ್ಲಿ ಸ್ವಾಮೀಜಿಯ ಪಾಶ್ಚಾತ್ಯ ಶಿಷ್ಯರಿಗೆ ಒಂದು ಮಠವನ್ನು ಸ್ಥಾಪಿಸುವರಂತೆ. ಕಳೆದ ಇಪ್ಪತ್ತು ವರುಷಗಳಿಂದಸೇವಿಯರ್​ ದಂಪತಿಗಳು ಯಾವ ಧರ್ಮವನ್ನೂ ಅನುಸರಿಸುತ್ತಿರಲಿಲ್ಲ. ಯಾವುದೂ ಅವರಿಗೆ ತೃಪ್ತಿಯನ್ನು ಕೊಟ್ಟಿರಲಿಲ್ಲ. ಆದರೆ ಸ್ವಾಮೀಜಿ ಅವರ ಭಾಷಣಗಳನ್ನು ಕೇಳಿದ ಮೇಲೆ ತಮ್ಮ ಹೃದಯ ಮತ್ತು ಬುದ್ಧಿಗೆ ತೃಪ್ತಿ ತರಬಲ್ಲ ಒಂದು ಧರ್ಮ ತಮಗೆ ಸಿಕ್ಕಿತು ಎಂದು ಅವರು ಒಪ್ಪಿಕೊಂಡರು. ಅಂದಿನಿಂದ ಅವರು ಸ್ವಾಮೀಜಿ ಜೊತೆಯಲ್ಲಿ ಸ್ವಿಡ್ಜರ್ಲೆಂಡ್​, ಜರ್ಮನಿ, ಇಟಲಿ ಮತ್ತು ಈಗ ಇಂಡಿಯಾ ದೇಶದಲ್ಲಿ ಸಂಚರಿಸುತ್ತಿರುವರು. ಇಂಗ್ಲೆಂಡಿನಲ್ಲಿ ಪತ್ರಿಕಾ ಬಾತ್ಮೀದಾರನಾದ ಗುಡ್​ವಿನ್​ ಎಂಬುವನು ಪ್ರಥಮ ಬಾರಿ ನ್ಯೂಯಾರ್ಕಿನಲ್ಲಿ ಸ್ವಾಮೀಜಿ ಅವರನ್ನು ಹದಿನಾಲ್ಕು ತಿಂಗಳ ಹಿಂದೆ ಕಂಡಮೇಲೆ ಅವರ ಶಿಷ್ಯನಾದನು. ಅವನು ಅಂದಿನಿಂದ ಪತ್ರಿಕಾ ವ್ಯವಸಾಯವನ್ನು ಬಿಟ್ಟು ಸ್ವಾಮೀಜಿಯವರ ಕೆಲಸದಲ್ಲಿ ನಿರತನಾಗಿರುವನು. ಸ್ವಾಮೀಜಿಯವರು ಮಾಡುವ ಉಪನ್ಯಾಸಗಳನ್ನು ಶೀಘ್ರಲಿಪಿಯಲ್ಲಿ ತೆಗೆದು ಕೊಳ್ಳುತ್ತಿರುವನು. ಅವನು ತಾನು ತನ್ನ ಕೊನೆಗಾಲದವರೆಗೂ ಸ್ವಾಮೀಜಿಯವರ ಜೊತೆಯಲ್ಲೇ ಇರುತ್ತೇನೆ ಎನ್ನುವನು. ಅವನು ಎಲ್ಲ ವಿಷಯಗಳಲ್ಲಿಯೂ ಸ್ವಾಮೀಜಿ ಯವರ ನಿಜವಾದ ಶಿಷ್ಯನಾಗಿರುವನು.

