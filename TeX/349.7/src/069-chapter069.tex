
\vskip -0.5cm

\chapter[ಮಾತಾಂಧತೆ ]{ಮತಾಂಧತೆ \protect\footnote{\engfoot{C.W. Vol. p.242}}}

ಹಲವು ಬಗೆಯ ತೀವ್ರ ಮತಾಂಧರು ಇರುವರು. ಕೆಲವರು ಮದ್ಯದ ತೀವ್ರ ಮತಾಂಧರು, ಮತ್ತೆ ಕೆಲವರು ಸಿಗರೇಟಿನ ವಿಷಯದಲ್ಲಿ ತೀವ್ರ ಮತಾಂಧರು. ಕೆಲವು ಜನರು ತಂಬಾಕನ್ನು\break ಸೇದುವುದನ್ನು ಬಿಟ್ಟುಬಿಟ್ಟರೆ ಜಗತ್ತು ರಾಮರಾಜ್ಯವಾಗುವುದೆಂದು ಭಾವಿಸುವರು. ಹಲವು ಸ್ತ್ರೀಯರು ಈ ಗುಂಪಿಗೆ ಸೇರಿರುವರು. ಒಂದು ದಿನ ಇಲ್ಲಿ ಉಪನ್ಯಾಸ ಕೇಳುವುದಕ್ಕೆ ಒಬ್ಬಳು ಹೆಂಗಸು ಬಂದಿದ್ದಳು. ಇಂತಹ ಹೆಂಗಸರು ಕೆಲವರು ಸೇರಿ ಚಿಕಾಗೋದಲ್ಲಿ ಒಂದು ಮನೆಯನ್ನು ಕಟ್ಟಿರುವರು. ಇಲ್ಲಿ ಶ್ರಮಜೀವಿಗಳಿಗೆ ಸಂಗೀತವನ್ನು ಮತ್ತು ಅಂಗಸಾಧನೆಯನ್ನು ಹೇಳಿಕೊಡುವರು. ಒಂದು ದಿನ ಆ ತರುಣಿ ಜಗತ್ತಿನಲ್ಲಿರುವ ಪಾಪದ ವಿಷಯವಾಗಿ ಮಾತನಾಡುತ್ತಿದ್ದಳು. ಇದಕ್ಕೆ ಪರಿಹಾರ ತನಗೆ ಗೊತ್ತಿದೆ ಎಂದಳು. ನಿನಗೆ ಹೇಗೆ ಗೊತ್ತು ಎಂದು ನಾನು ಕೇಳಿದೆ. ನಿಮಗೆ ಹಲ್​ ಹೌಸ್​ ಗೊತ್ತೆ ಎಂದು ಕೇಳಿದಳು. ಆಕೆಯ ದೃಷ್ಟಿಯಲ್ಲಿ ಮನುಷ್ಯನ ಪಾತಕಗಳಿಗೆಲ್ಲಾ ಒಂದು ರಾಮಬಾಣವಿದೆ. ಅದೇ ಹಲ್​ ಹೌಸ್​. ಇದೇ ಸರಿ ಎಂಬ ಅಭಿಪ್ರಾಯ ಅವಳಲ್ಲಿ ಭದ್ರವಾಗಿ ಬೇರೂರಿರುವುದು. ನಾನು ಅವಳಿಗಾಗಿ ವಿಷಾದಪಡಬೇಕಾಗಿದೆ. ಭರತಖಂಡದಲ್ಲಿ ಕೆಲವರು ಭ್ರಾಂತರಿರುವರು. ಅವರ ದೃಷ್ಟಿಯಲ್ಲಿ ಹೆಂಗಸಿನ ಗಂಡ ಕಾಲವಾದ ಮೇಲೆ ಅವಳಿಗೆ ಪುನಃ ಮದುವೆ ಮಾಡಿಕೊಳ್ಳಲು ಅವಕಾಶ ಸಿಕ್ಕಿದರೆ ಅದರಿಂದ ಭರತಖಂಡದ ಕಷ್ಟಗಳೆಲ್ಲ ಪರಿಹಾರವಾಗುವುದೆಂದು ಭಾವಿಸುವರು. ಇದೊಂದು ಬಗೆಯ ಭ್ರಾಂತಿ.

ನಾನು ಹುಡುಗನಾಗಿದ್ದಾಗ ಮತಾಂಧತೆ ಕೆಲಸಕ್ಕೆ ಅತ್ಯಾವಶ್ಯಕ ಎಂದು ಭಾವಿಸಿದ್ದೆ. ಆದರೆ ನನಗೆ ವಯಸ್ಸಾದಂತೆಲ್ಲ, ಇದು ನಿಜವಲ್ಲ ಎಂದು ಗೊತ್ತಾಗುತ್ತಿದೆ.

ಹೆಂಗಸೊಬ್ಬಳು ಕದಿಯಬಹುದು. ಮತ್ತೊಬ್ಬರ ಚೀಲವನ್ನು ತನ್ನ ಮನೆಗೆ ತೆಗೆದುಕೊಂಡು ಹೋಗುವುದಕ್ಕೆ ಅವಳಿಗೆ ಮನಸ್ಸು ಸ್ವಲ್ಪವೂ ಅಳುಕುವುದಿಲ್ಲ. ಆದರೆ ಬಹುಶಃ ಆ ಹೆಂಗಸು ಸಿಗರೇಟನ್ನು ಸೇದುವುದಿಲ್ಲ ಎಂದು ಕಾಣಿಸುವುದು, ಅವಳು ಸಿಗರೇಟಿನ ವಿರುದ್ಧ ತೀವ್ರವಾಗಿ ಹೋರಾಡುತ್ತಾಳೆ. ಸಿಗರೇಟು ಸೇದುವವರನ್ನು ನೋಡಿದರೆ ಕಟುವಾಗಿ\break ನಿಂದಿಸುವಳು. ಒಬ್ಬ ಮನುಷ್ಯ ಎಲ್ಲರಿಗೂ ಮೋಸ ಮಾಡುತ್ತಿರಬಹುದು. ಅವನನ್ನು ನೆಚ್ಚು\-ವುದಕ್ಕೆ ಯಾರಿಗೂ ಆಗುವುದಿಲ್ಲ. ಯಾವ ಹೆಂಗಸೂ ಅವನ ಹತ್ತಿರ ಸುರಕ್ಷಿತಳಾಗಿ ಇರಲಾ\-ರಳು. ಆದರೆ ಈ ನೀಚನು ಮದ್ಯ ಕುಡಿಯುವುದಿಲ್ಲ. ಅವನು ಮದ್ಯವನ್ನು ಕುಡಿಯುವ ಇತರರನ್ನು ನೋಡಿದರೆ ಅವರಲ್ಲಿ ಏನೂ ಒಳ್ಳೆಯ ಗುಣವಿರಲಾರದು ಎಂದು ಭಾವಿಸುವನು. ತಾನು ಮಾಡುವ ಪಾತಕಗಳನ್ನೆಲ್ಲ ಗಣನೆಗೇ ತರುವುದಿಲ್ಲ. ವ್ಯಕ್ತಿಯ ವಿಷಯದಲ್ಲಿ ಸಂಕುಚಿತವಾದ ದೃಷ್ಟಿಯನ್ನು ಹೊಂದಿರುವುದು ಮಾನವ ಸಹಜವಾದ ಸ್ವಾರ್ಥ.

ದೇವರು ಜಗತ್ತನ್ನು ಆಳುತ್ತಿರುವನು. ಅದನ್ನು ಮಾನವನ ದಯೆಗೆ ಬಿಟ್ಟಿಲ್ಲ\break ಎಂಬುದನ್ನು ನಾವು ಗಮನಿಸಬೇಕು. ಭಗವಂತನೇ ಈ ಪ್ರಪಂಚವನ್ನೆಲ್ಲಾ ಆಳುತ್ತಿರುವನು, ಪಾಲಿಸುತ್ತಿರುವನು. ಈ ಮದ್ಯ ಭ್ರಾಂತರು, ದುರಭಿಮಾನಿಗಳು, ಧೂಮಪಾನ ಭ್ರಾಂತರು ವಿಧವಾ ವಿವಾಹ ಭ್ರಾಂತರು ಎಷ್ಟು ಗಲಾಟೆ ಮಾಡುತ್ತಿದ್ದರೂ ಜಗತ್ತು ತನ್ನ ಪಾಡಿಗೆ ತಾನು ಸಾಗುತ್ತಿರುವುದು. ಈ ಜನರೆಲ್ಲ ಇಂದೇ ಕಾಲವಾದರೂ ಜಗತ್ತು ಎಂದಿನಂತೆಯೇ ಮುಂದೆ ಸಾಗುತ್ತಿರುವುದು.

ನಿಮ್ಮ ಚರಿತ್ರೆಯಲ್ಲಿ ಹೇಗೆ ‘ಮೇ ಫ್ಲವರ್​’ ಎಂಬ ಹೆಸರಿನ ಹಡಗಿನಲ್ಲಿ ಜನರು ಬಂದು ತಾವು ಪ್ಯೂರಿಟನ್ಸ್​ ಎಂದು ಕರೆದುಕೊಂಡರು ಎಂಬುದು ನಿಮಗೆ ಜ್ಞಾಪಕವಿಲ್ಲವೆ? ಅವರು ತಮ್ಮಷ್ಟಕ್ಕೆ ಪರಿಶುದ್ಧರಾಗಿದ್ದರು, ಒಳ್ಳೆಯವರಾಗಿದ್ದರು. ಆದರೆ ತಮ್ಮ ಪಂಥಕ್ಕೆ ಸೇರದ ಇತರರ ಸಂಪರ್ಕವಾದಾಗ ಅವರನ್ನು ಹಿಂಸಿಸಲು ಉಪಕ್ರಮಿಸಿದರು. ವಿಶ್ವದ ಇತಿಹಾಸದಲ್ಲೆಲ್ಲಾ ಹೀಗೆಯೇ ಆಗಿರುವುದು. ಹಿಂಸೆಯನ್ನು ಸಹಿಸಲಾರದೆ ಓಡಿಹೋದವರು ಕೂಡ ತಮಗೆ ಅವಕಾಶ ಸಿಕ್ಕಿದಲ್ಲಿ ಇತರರನ್ನು ಹಿಂಸಿಸಲು ಉಪಕ್ರಮಿಸುವರು.

ಮತಾಂಧರಲ್ಲಿ ನೂರಕ್ಕೆ ತೊಂಭತ್ತು ಮಂದಿಗೆ ಯಕೃತ್​ ಚೆನ್ನಾಗಿರಲಿಲ್ಲ. ಇಲ್ಲವೇ ಅವರು ಅಗ್ನಿಮಾಂದ್ಯಕ್ಕೆ ಅಥವಾ ಇನ್ನು ಯಾವುದಾದರೂ ರೋಗಕ್ಕೆ ತುತ್ತಾಗಿದ್ದರು. ವೈದ್ಯರು ಮತಾಂಧತೆ ಎಂಬುದು ಒಂದು ಬಗೆಯ ಜಾಡ್ಯ ಎಂದು ಕಾಲಕ್ರಮೇಣ\break ಕಂಡುಹಿಡಿದರು. ನಾನು ಇಂತಹ ಹಲವಾರು ಜನರನ್ನು ಕಂಡಿರುವೆನು. ದೇವರು ನನ್ನನ್ನು ಅವರಿಂದ ದೂರವಿರಿಸಲಿ.

ಎಲ್ಲಾ ಬಗೆಯ ಮತಾಂಧತೆಯ ಸುಧಾರಣೆಗಳಿಂದ ಪಾರಾಗುವುದು ಮೇಲೆಂದು ನನ್ನ ಅನುಭವ ಸಾರುವುದು. ಈ ಪ್ರಪಂಚ ನಿಧಾನವಾಗಿ ಸಾಗುತ್ತಿದೆ. ನಿಧಾನವಾಗಿಯೇ ಸಾಗಲಿ, ನಿಮಗೆ ಏತಕ್ಕೆ ಇಷ್ಟು ಅವಸರ? ಚೆನ್ನಾಗಿ ನಿದ್ರೆ ಮಾಡಿ. ಯಾವ ಉದ್ವೇಗವೂ ಇಲ್ಲದಿರಲಿ. ಒಳ್ಳೆಯ ಆಹಾರವನ್ನು ಸೇವಿಸಿ. ನಿಮ್ಮ ನರಗಳು ಸುಸ್ಥಿತಿಯಲ್ಲಿರಲಿ. ಪ್ರಪಂಚಕ್ಕೆ ಅನುಕಂಪವನ್ನು ತೋರಿ. ದುರಭಿಮಾನಿಗಳು ದ್ವೇಷವನ್ನು ಸೃಷ್ಟಿಸುವರು. ಮದ್ಯಪಾನವನ್ನು ನಿಷೇಧಿಸುವವರು ಕುಡುಕರನ್ನು ಪ್ರೀತಿಸುತ್ತಾರೆ ಎಂದು ಭಾವಿಸಿದಿರಾ? ಅವನಿಗೆ ಏನೋ ಪ್ರತಿಫಲಬೇಕಾಗಿದೆ. ಅದಕ್ಕೆ ಅವನೊಬ್ಬ ದುರಭಿಮಾನಿಯಾಗಿರುವನು. ಹೋರಾಟ ನಿಂತೊಡನೆಯೆ ದರೋಡೆಗೆ ಧಾವಿಸುವನು. ಇಂತಹ ದುರಭಿಮಾನಿಗಳ ಸಂಗದಿಂದ ನೀವು ಪಾರಾದಮೇಲೆ ನಿಜವಾದ ಪ್ರೀತಿ ಮತ್ತು ಕರುಣೆ ಏನೆಂಬುದು ಗೊತ್ತಾಗುವುದು. ನಿಮ್ಮಲ್ಲಿ ಪ್ರೀತಿ ಮತ್ತು ಅನುಕಂಪ ಹೆಚ್ಚಿದಂತೆಲ್ಲ ಈ ದೀನರನ್ನು ನಿಂದಿಸುವುದು ಕಡಮೆಯಾಗುವುದು. ಬದಲಿಗೆ ಅವರ ಲೋಪದೋಷಗಳನ್ನು ಕಂಡು ನಿಮಗೆ ಅವರ ಮೇಲೆ ಸಹಾನುಭೂತಿ ಹುಟ್ಟುವುದು. ಆಗ ಕುಡುಕನೊಂದಿಗೆ ನೀವು ಸಹಾನುಭೂತಿ ತೋರಲು ಸಾಧ್ಯವಾಗುವುದು; ಅವನೂ ಕೂಡ ನಿಮ್ಮಂತೆ ಮನುಷ್ಯ ಎಂಬುದು ಗೊತ್ತಾಗುವುದು. ಆಗ ಅವನನ್ನು ಯಾವ ಯಾವ ಸನ್ನಿವೇಶಗಳು ಕುಡುಕನಂತೆ ಮಾಡಿದವು ಎಂಬುದನ್ನು ನೀವು ಅರಿತುಕೊಳ್ಳುವಿರಿ ಮತ್ತು ನೀವು ಆ ಸ್ಥಿತಿಯಲ್ಲಿದ್ದರೆ ಆತ್ಮಹತ್ಯೆಯನ್ನು ಮಾಡಿಕೊಳ್ಳುತ್ತಿದ್ದಿರಿ ಎಂದು ನಿಮಗೆ ಅನ್ನಿಸುವುದು. ಒಬ್ಬಳು ಸ್ತ್ರೀಯಳ ಗಂಡ ತುಂಬಾ ಕುಡುಕನಾಗಿದ್ದ ನನಗೆ ಗೊತ್ತಿದೆ. ಒಂದು ದಿನ ಆ ಹೆಂಗಸು ಗಂಡ ಕುಡುಕನಾಗಿ ಹೋಗಿರುವನೆಂದು ದೂರುತ್ತಿದ್ದಳು. “ನೋಡಿ, ನಿಮ್ಮಂತಹ ಇಪ್ಪತ್ತು ಲಕ್ಷ ಹೆಂಡತಿಯರಿದ್ದರೆ ಅವರ ಗಂಡಂದಿರೆಲ್ಲ ಕುಡುಕರಾಗುತ್ತಿದ್ದರು” ಎಂದೆ. ಪತ್ನಿಯರೇ ಬಹುತೇಕ ಕುಡುಕ ಗಂಡಂದಿರನ್ನು ತಯಾರಿಸುವ ಕಾರ್ಖಾನೆಗಳು ಎಂಬುದು ನನ್ನ ಮನಸ್ಸಿಗೆ ದೃಢವಾಗಿದೆ. ಸತ್ಯವನ್ನು ಹೇಳುವುದು ನನ್ನ ಕರ್ತವ್ಯ. ಸುಮ್ಮನೆ ಇತರರನ್ನು ಹೊಗಳುವುದಲ್ಲ. ಸಹನೆ ತಾಳ್ಮೆ ಎಂಬ ಭಾವನೆಗಳೆಲ್ಲ ಮಾಯವಾದ ಇಂತಹ ಸ್ವಚ್ಚಂದ ಹೆಂಗಸರು, ಸ್ವಾತಂತ್ರ್ಯ ಎಂಬ ತಪ್ಪು ಭಾವನೆಯಿಂದ ಪ್ರೇರಿತರಾಗಿ, ಗಂಡಸರು ಯಾವಾಗಲೂ ತಾವು ಹೇಳಿದಂತೆ ಕೇಳಬೇಕೆಂದು ಕೂಗುತ್ತಾರೆ. ಅಲ್ಲದೆ ತಮಗೆ ಆಗದೆ ಇರುವುದರ ಒಂದು ಚೂರನ್ನು ಅವರು ಕೇಳಿದಾಗ ದೊಡ್ಡ ರಂಪವನ್ನು ಎಬ್ಬಿಸುವರು. ಇಂತಹ ಸ್ತ್ರೀಯರೇ ಜಗತ್ತಿನ ದೊಡ್ಡ ರೋಗ. ಪ್ರಪಂಚದ ಅರ್ಧಪಾಲು ಗಂಡಸರು ಆತ್ಮಹತ್ಯೆ ಮಾಡಿಕೊಳ್ಳದಿರುವುದೇ ಒಂದು ಆಶ್ಚರ್ಯ. ಹೀಗೆ ಇರುವುದಕ್ಕೆ ಸಾಧ್ಯವಿಲ್ಲ. ಬದುಕು ಅವರು ನಂಬಿರುವಷ್ಟು ಸುಖದಿಂದ ಕೂಡಿಲ್ಲ. ಅದು ತುಂಬ ಗಂಭೀರವಾದದ್ದು.

ಒಬ್ಬನಿಗೆ ಶ್ರದ್ಧೆ ಮಾತ್ರ ಇದ್ದರೆ ಸಾಲದು, ಜೊತೆಗೆ ಬೌದ್ಧಿಕ ಶ್ರದ್ಧೆಯೂ ಇರಬೇಕು. ಒಬ್ಬನು ಪ್ರತಿಯೊಂದನ್ನೂ ಸ್ವೀಕರಿಸಿ ನಂಬುವಂತೆ ಮಾಡುವುದು ಅವನನ್ನು ಹುಚ್ಚನನ್ನಾಗಿ ಮಾಡಿದಂತೆ. ನನಗೆ ಯಾರೊ ಒಬ್ಬರು ಒಂದು ಪುಸ್ತಕ ಕಳುಹಿಸಿದರು. ಅದರಲ್ಲಿರುವುದ\-ನ್ನೆಲ್ಲಾ ನಂಬಬೇಕೆಂದಿತ್ತು. ಅಲ್ಲಿ ಆತ್ಮ ಇಲ್ಲ ಎಂದು ಇತ್ತು. ಆದರೆ ಸ್ವರ್ಗದಲ್ಲಿ ದೇವದೇವತೆಗಳು ಇದ್ದರು. ನಮ್ಮ ತಲೆಯ ಮೂಲಕ ಜ್ಯೋತಿತಂತು ಒಂದು ಸ್ವರ್ಗಕ್ಕೆ ಹೋಗುವುದಂತೆ! ಪುಸ್ತಕ ಬರೆದವಳಿಗೆ ಇದೆಲ್ಲಾ ಹೇಗೆ ಗೊತ್ತಾಯಿತು? ಆಕೆಗೆಲ್ಲೊ ಒಂದು ಆವೇಶ ಬಂದಿರಬಹುದು. ನಾನು ಕೂಡ ಇದನ್ನು ನಂಬಬೇಕು ಎಂದು ಬರೆದಿದ್ದಳು. ನಾನು ಅದನ್ನು ಒಪ್ಪಿಕೊಳ್ಳದೆ ಹೋದುದರಿಂದ “ನೀನೊಬ್ಬ ಪಾಪಿ. ನಿನಗೆ ಭರವಸೆಯೇ ಇಲ್ಲ” ಎಂದಳು. ಇದೊಂದು ಬಗೆಯ ಭ್ರಾಂತಿ.

\vskip -0.5cm

