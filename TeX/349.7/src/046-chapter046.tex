
\chapter[ವಿಕಾಸ ವಾದ ]{ವಿಕಾಸ ವಾದ \protect\footnote{\engfoot{C.W. Vol. V, P. 277}}}

ಆಕಾಶ ಮತ್ತು ಪ್ರಾಣ ಇವು ಈ ವ್ಯಕ್ತ ಪ್ರಪಂಚವಾಗಿ, ಪುನಃ ಅವ್ಯಕ್ತವಾಗುವ ವಿಚಾರದಲ್ಲಿ ಭಾರತೀಯ ದರ್ಶನಕ್ಕೂ ಆಧುನಿಕ ವಿಜ್ಞಾನಕ್ಕೂ ಬಹಳ ಸಾಮ್ಯವಿದೆ. ಆಧುನಿಕರಿಗೆ ವಿಕಾಸವಾದವಿದೆ; ಹಾಗೆಯೇ ಯೋಗಿಗಳಿಗೂ ಉಂಟು. ಆದರೆ ಯೋಗಿಗಳ ವಿಕಾಸವಾದದ ವಿವರಣೆಯು ಉತ್ತಮವಾದದ್ದು ಎಂದು ತೋರುವುದು. “ಜಾತ್ಯಂತರ ಪರಿಣಾಮವು ಪ್ರಕೃತಿಯ ಪೂರೈಕೆಯಿಂದ ಆಗುವುದು.” ನಾವು ಒಂದು ಜೀವಜಾತಿಯಿಂದ ಮತ್ತೊಂದು ಜೀವ ಜಾತಿಗೆ ಬದಲಾಗು ತ್ತಿರುವೆವು ಮತ್ತು ಮನುಷ್ಯನೇ ಈಗ ಇರುವ ಶ್ರೇಷ್ಠ ಜೀವಿ ಎಂಬುದೇ ಮೂಲ ಭಾವನೆ. ಪತಂಜಲಿಯು ಪ್ರಕೃತಿಯ ಪೂರೈಕೆ ಎಂಬುದು ಕೃಷಿಕನು ನೀರು ಹಾಯಿಸುವಂತೆ ಎನ್ನುವನು. ನಮ್ಮ ವಿದ್ಯಾಭ್ಯಾಸ, ನಮ್ಮ ಅಭಿವೃದ್ಧಿ ಎಂದರೆ ಬೆಳವಣಿಗೆಗೆ ಇರುವ ಆತಂಕವನ್ನು ತೆಗೆದುಹಾಕುವುದು. ಆಗ ನಮ್ಮ ದಿವ್ಯ ಸ್ವರೂಪವು ತಾನಾಗಿಯೇ ವ್ಯಕ್ತವಾಗುವುದು. ಅಸ್ತಿತ್ವಕ್ಕಾಗಿ ನಡೆಯುವ ಹೋರಾಟವನ್ನೆಲ್ಲಾ ಇದು ನಿವಾರಿ ಸುತ್ತದೆ. ಜೀವನದ ದುಃಖವೆಲ್ಲ ಒಂದು ಆತಂಕ. ಅದನ್ನು ಸಂಪೂರ್ಣ ತೊಡೆದು ಹಾಕಬಹುದು. ವಿಕಾಸಕ್ಕೆ ಅದು ಅವಶ್ಯಕವಲ್ಲ. ಅದಿಲ್ಲದೆ ಇದ್ದರೂ ನಾವು ಮುಂದು ವರಿಯುತ್ತಿದ್ದೆವು. ಸ್ವಭಾವತಃ ವಸ್ತುಗಳು ವ್ಯಕ್ತವಾಗುವಂತಹವುಗಳು. ಪ್ರಚೋದನೆ ಹೊರಗಿನಿಂದ ಬರುವುದಿಲ್ಲ, ಒಳಗಿನಿಂದಲೇ ಉಂಟಾಗುವುದು. ಸುಪ್ತವಾಗಿರುವ ವಿಶ್ವಾನುಭವ ಆಗಲೆ ಪ್ರತಿಯೊಂದು ಜೀವಿಯ ಅಂತರಾಳದಲ್ಲಿಯೂ ಹುದುಗಿದೆ. ಈ ಅನುಭವಗಳಲ್ಲಿ ಯಾವುದಕ್ಕೆ ಸೂಕ್ತ ವಾತಾವರಣ ದೊರಕುವುದೊ ಅದು ಮಾತ್ರ ವ್ಯಕ್ತವಾಗುವುದು.

ಬಾಹ್ಯ ವಸ್ತುಗಳು ಕೇವಲ ವಾತಾವರಣವನ್ನು ಮಾತ್ರ ಕಲ್ಪಿಸಬಲ್ಲವು. ಸ್ಪರ್ಧೆ, ಹೋರಾಟ, ಪಾಪ ಇವು ವಿಕಾಸದ ಪರಿಣಾಮವಾಗಿ ಆದವುಗಳಲ್ಲ. ಇವು ವಿಕಾಸದ ದಾರಿಯಲ್ಲಿರುವ ಆತಂಕಗಳು. ಇವುಗಳಿಲ್ಲದೇ ಇದ್ದರೂ ಮಾನವ ವಿಕಾಸಗೊಂಡು ದೇವರಾಗುತ್ತಿದ್ದ. ಏಕೆಂದರೆ ಹೊರಗೆ ವ್ಯಕ್ತವಾಗುವುದೇ ದೇವರ ಸ್ವಭಾವವಾಗಿದೆ. ನನ್ನ ಮನಸ್ಸಿಗೆ ಆ ಭಯಾನಕ ಸ್ಪರ್ಧೆಯ ಭಾವನೆಗಿಂತ ಇದು ಹೆಚ್ಚು ಆಶಾದಾಯಕವಾದುದು. ಇತಿಹಾಸವನ್ನು ಹೆಚ್ಚು ಹೆಚ್ಚು ಓದಿದಂತೆಲ್ಲ ಸ್ಪರ್ಧೆಯ ಭಾವನೆ ತಪ್ಪು ಎಂದು ಹೆಚ್ಚು ಸ್ಪಷ್ಟವಾಗಿ ಗೊತ್ತಾಗುತ್ತಿದೆ. ಮಾನವನು ಮಾನವನೊಂದಿಗೆ ಹೋರಾಡದಿದ್ದರೆ ಅವನು ಪ್ರಗತಿ ಹೊಂದಲಾರ ಎಂದು ಕೆಲವರು ಹೇಳುವರು. ನಾನೂ ಹಾಗೆಯೇ ಹಿಂದೆ ಆಲೋಚಿಸುತ್ತಿದ್ದೆ. ಆದರೆ ಪ್ರತಿಯೊಂದು ಯುದ್ಧವೂ ಮಾನವ ಪ್ರಗತಿಯನ್ನು ಐವತ್ತು ವರುಷಗಳಷ್ಟು ಹಿಂದಕ್ಕೆ ತಳ್ಳಿದೆ. ಜನರು ಬೇರೊಂದು ದೃಷ್ಟಿಯಿಂದ ಚರಿತ್ರೆಯನ್ನು ಓದುವ ಕಾಲ ಬರುವುದು. ಆಗ ಸ್ಪರ್ಧೆ ಕಾರ್ಯವೂ ಅಲ್ಲ, ಕಾರಣವೂ ಅದು ಅವಶ್ಯಕವೇ ಅಲ್ಲ ಎಂದು ಗೊತ್ತಾಗುವುದು.

ವಿಚಾರಶೀಲರು ಒಪ್ಪಿಕೊಳ್ಳಬಹುದಾದಂತಹ ಏಕಮಾತ್ರ ಸಿದ್ಧಾಂತ ಪತಂಜಲಿ ಯದು ಎಂದು ನಾನು ಭಾವಿಸುತ್ತೇನೆ. ಆಧುನಿಕ ಸಿದ್ಧಾಂತಗಳು ಎಷ್ಟೊಂದು ದೌರ್ಜನ್ಯವನ್ನು ಹರಡುವುವು! ಪ್ರತಿಯೊಬ್ಬ ದುರಾತ್ಮನೂ ತಾನು ದುರಾತ್ಮ ನಾಗಲು ಒಂದು ಅಪ್ಪಣೆಯ ಚೀಟಿ ಇದೆ ಎಂದು ಭಾವಿಸುವನು. ಈ ದೇಶದಲ್ಲಿ (ಅಮೆರಿಕಾದಲ್ಲಿ) ಕಳ್ಳತನ ಖೂನಿ ಇವನ್ನು ಮಾಡುವವರನ್ನೆಲ್ಲಾ ನಿರ್ಮೂಲ ಮಾಡಬೇಕು. ಇದೊಂದೇ ಮಾರ್ಗ ಸಮಾಜ ಅವರ ಉಪಟಳದಿಂದ ಪಾರಾಗ ಬೇಕಾದರೆ, ಎಂದು ಭೌತಶಾಸ್ತ್ರಜ್ಞರು ಹೇಳುವುದನ್ನು ನಾನು ಕೇಳಿರುವೆನು. ವಾತಾವರಣವು ಪ್ರಗತಿಗೆ ಆತಂಕವಾಗಬಹುದೇ ಹೊರತು ಅದು ಆವಶ್ಯಕವಲ್ಲ. ಸ್ಪರ್ಧೆಯ ವಿಷಯದಲ್ಲಿರುವ ಒಂದು ಭಯಾನಕ ವಿಷಯವೆಂದರೆ ಒಬ್ಬನು ವಾತಾವರಣವನ್ನು ಗೆಲ್ಲಬಹುದು, ಆದರೆ ಒಬ್ಬನು ಎಲ್ಲಿ ಗೆಲ್ಲುವನೊ ಅಲ್ಲಿಸಾವಿರಾರು ಜನರನ್ನು ಅವನು ಆಚೆಗೆ ದಬ್ಬುವನು ಎಂಬುದು. ಎಂದರೆ ಇದೊಂದು ಮಹಾ ಪಾತಕವಾಯಿತು. ಯಾವುದು ಕೇವಲ ಒಬ್ಬನಿಗೆ ಮಾತ್ರ ಸಹಾಯಮಾಡಿ ಹಲವರನ್ನು ಆಚೆಗೆ ತಳ್ಳುವುದೊ ಅದನ್ನು ಒಳ್ಳೆಯದು ಎನ್ನಲಾಗು ವುದಿಲ್ಲ. ಈ ಸ್ಪರ್ಧೆ ನಮ್ಮ ಅಜ್ಞಾನದಿಂದ ಮಾತ್ರ ಇರುವುದು, ಇದು ಆವಶ್ಯಕವಲ್ಲ, ಇದು ಮಾನವನ ವಿಕಾಸದ ಒಂದು ಅಂಶವಲ್ಲ ಎನ್ನುವನು ಪತಂಜಲಿ. ನಮ್ಮ ಅಸಹನೆಯೇ ಸ್ಪರ್ಧೆಗೆ ಕಾರಣ. ನಮ್ಮ ಮುಕ್ತಿಯನ್ನು ನಾವು ಕಂಡುಹಿಡಿದುಕೊಳ್ಳು ವುದಕ್ಕೆ ಬೇಕಾದ ತಾಳ್ಮೆ ನಮಗೆ ಇಲ್ಲ. ಉದಾಹರಣೆಗೆ ಒಂದು ನಾಟಕಶಾಲೆ ಯೊಳಗೆ ಬೆಂಕಿ ಬಿದ್ದಿದೆ ಎನ್ನಿ, ಎಲ್ಲೊ ಕೆಲವರು ಮಾತ್ರ ತಪ್ಪಿಸಿಕೊಂಡು ಹೋಗುವರು. ಇತರರು ತಪ್ಪಿಸಿಕೊಂಡು ಹೋಗುವಾಗ ಒಬ್ಬರ ಮೇಲೆ ಒಬ್ಬರು ಬಿದ್ದು ಸಾಯುವರು. ಅದು ಕಟ್ಟಡವನ್ನು ಉಳಿಸುವುದಕ್ಕಾಗಲಿ, ಇಬ್ಬರು ಮೂವರು ತಪ್ಪಿಸಿಕೊಂಡು ಬರುವುದಕ್ಕಾಗಲಿ ಅವಶ್ಯಕವಾಗಿರಲಿಲ್ಲ. ಎಲ್ಲರೂ ನಿಧಾನವಾಗಿ ಹೋಗಿದ್ದರೆ ಒಬ್ಬರಿಗೂ ಗಾಯವಾಗುತ್ತಿರಲಿಲ್ಲ. ಜೀವನದಲ್ಲಿಯೂ ಹಾಗೆಯೆ. ನಮಗೆ ಬಾಗಿಲು ತೆರೆದಿದೆ. ಸ್ಪರ್ಧೆ ಹೋರಾಟಗಳಿಲ್ಲದೆ ನಾವೆಲ್ಲ ಹೊರಗೆ ಹೋಗಬಹುದು. ಆದರೂ ನಾವು ಹೋರಾಡುವುವೆವು. ನಮ್ಮಲ್ಲಿ ಜ್ಞಾನ ಮತ್ತು ತಾಳ್ಮೆ ಇಲ್ಲವಾಗಿರುವುದೇ ಈ ಹೋರಾಟಕ್ಕೆಲ್ಲ ಕಾರಣ. ನಮಗೆ ತುಂಬಾ ಆತುರ. ಶಕ್ತಿಯ ಅತ್ಯುನ್ನತ ಅಭಿವ್ಯಕ್ತಿಯೇ ಶಾಂತಿ ಮತ್ತು ಸ್ವಾವಲಂಬನೆಗೆ ಕಾರಣ.

