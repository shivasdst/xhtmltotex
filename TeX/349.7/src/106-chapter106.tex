
\chapter[ಪವಾಡಗಳು ]{ಪವಾಡಗಳು \protect\footnote{\engfoot{C.W. Vol. V, p. 183}}}

\centerline{\textbf{(ಮೆಂಫಿಸ್​ ಕಮರ್ಶಿಯಲ್​ 15ನೇ ಜನವರಿ 1894)}}

ವೃತ್ತಪತ್ರಿಕೆಯ ಬಾತ್ಮಿದಾರನೊಬ್ಬ ಸ್ವಾಮೀಜಿ ಅವರನ್ನು ಅಮೆರಿಕ ದೇಶದ ವಿಷಯವಾಗಿ ತಮ್ಮ ಅಭಿಪ್ರಾಯವೇನು ಎಂದು ಪ್ರಶ್ನಿಸಿದನು. ಅದಕ್ಕೆ ಸ್ವಾಮೀಜಿ ಹೀಗೆಂದರು:

“ಈ ದೇಶದ ವಿಷಯವಾಗಿ ಅದರಲ್ಲಿಯೂ ಅಮೆರಿಕ ದೇಶದ ಮಹಿಳೆಯರ ವಿಷಯವಾಗಿ ನನಗೆ ಒಳ್ಳೆಯ ಅಭಿಪ್ರಾಯವಿದೆ. ಅಮೇರಿಕ ದೇಶದಲ್ಲಿ ಬಡತನ ಇಲ್ಲದೆ ಇರುವ ವಿಷಯವನ್ನು ನಾನು ಆಗಲೇ ವ್ಯಕ್ತಪಡಿಸಿರುವೆನು.”

ಸಂಭಾಷಣೆ ಅನಂತರ ಧರ್ಮದ ಕಡೆಗೆ ತಿರುಗಿತು. ವಿಶ್ವಧರ್ಮ ಸಮ್ಮೇಳನ ಬಹಳ ಪ್ರಯೋಜನಕಾರಿಯಾಗಿತ್ತು, ಭಾವನೆಗಳನ್ನು ವಿಶಾಲಗೊಳಿಸಲು ಇದು ಬಹಳ ಸಹಾಯ ಮಾಡಿದೆ, ಎಂದರು. ಬಾತ್ಮಿದಾರ: “ಕ್ರೈಸ್ತಧರ್ಮಕ್ಕೆ ಸೇರಿದವರು ಮರಣಾನಂತರ ಏನಾಗುತ್ತಾರೆ ಎಂಬುದರಲ್ಲಿ ಸಾಮಾನ್ಯವಾಗಿ ಹಿಂದೂಗಳ ಅಭಿಪ್ರಾಯವೇನು?” ಸ್ವಾಮೀಜಿ: “ಒಳ್ಳೆಯವರಾಗಿದ್ದರೆ ಅವರು ಉದ್ಧಾರವಾಗುತ್ತಾರೆ. ಒಬ್ಬ ನಾಸ್ತಿಕವಾದಿಯಾದರೂ ಅವನು ಒಳ್ಳೆಯ ಮನುಷ್ಯನಾಗಿದ್ದರೆ ಅವನು ಉದ್ಧಾರವಾಗಲೇಬೇಕು ಎಂದು ನಾವು ನಂಬುತ್ತೇವೆ. ಇದು ನಮ್ಮ ಧರ್ಮ. ಎಲ್ಲಾ ಧರ್ಮಗಳೂ ಕೂಡ ಒಳ್ಳೆಯವೆ. ಆದರೆ ಒಬ್ಬರು ಮತ್ತೊಬ್ಬರೊಡನೆ ಕಲಹವಾಡಕೂಡದು.

\eject

ವಿವೇಕಾನಂದರನ್ನು ಕುರಿತು ಮಾಯಮಂತ್ರಗಳು, ಆಕಾಶದಲ್ಲಿ ಮೇಲಕ್ಕೆ ಏಳುವುದು, ಮತ್ತು ಜೀವಸಹಿತ ನೆಲದೊಳಗೆ ಇರುವುದು ಮುಂತಾದ ವಿಷಯಗಳ ಮೇಲೆ ಪ್ರಶ್ನೆ ಮಾಡಿದರು. ಅದಕ್ಕೆ ವಿವೇಕಾನಂದರು ಹೀಗೆಂದರು: “ನಾವು ಮಾಯತಂತ್ರಗಳನ್ನು ನಂಬುವುದೇ ಇಲ್ಲ. ಆದರೆ ಯಾವುದನ್ನು ನಾವು ವಿಚಿತ್ರ ಎಂದು ಭಾವಿಸುವೆವೋ ಅದನ್ನು ಪ್ರಕೃತಿನಿಯಮಗಳಿಗೆ ಅನುಸಾರ ಮಾಡಬಹುದು. ಈ ವಿಷಯಗಳ ಮೇಲೆ ಭರತಖಂಡದಲ್ಲಿ ಬೇಕಾದಷ್ಟು ಸಾಹಿತ್ಯವಿದೆ. ಅಲ್ಲಿ ಜನ ಈ ವಿಷಯಗಳನ್ನು ಚೆನ್ನಾಗಿ ಅಧ್ಯಯನ ಮಾಡಿರುವರು.”

“ಹಠಯೋಗಿಗಳು ಮತ್ತೊಬ್ಬರ ಮನಸ್ಸಿನ ಆಲೋಚನೆಯನ್ನು ಓದುವುದು, ಮತ್ತು ಭವಿಷ್ಯದಲ್ಲಿ ಏನಾಗುವುದು ಎಂಬ ವಿಷಯಗಳನ್ನು ತಿಳಿಯುವುದು, ಇವನ್ನು ಜಯಪ್ರದವಾಗಿ ಅಭ್ಯಾಸ ಮಾಡಿರುವರು.”

“ಮೇಲಕ್ಕೆ ಏಳುವ ವಿಷಯದಲ್ಲಿ –ಯಾರೂ ಭೂಮಿಯ ಆಕರ್ಷಣೆಯಿಂದ ಪಾರಾಗಿ\break ಮೇಲಕ್ಕೆ ಎದ್ದುದನ್ನು ನಾನು ಕಂಡಿಲ್ಲ. ಆದರೆ ಹೀಗೆ ಮಾಡುವುದಕ್ಕೆ ಪ್ರಯತ್ನಿಸುತ್ತಿದ್ದ\break ಹಲವರನ್ನು ನಾನು ನೋಡಿರುವೆನು. ಅವರು ಈ ವಿಷಯದ ಮೇಲೆ ಇರುವ ಪುಸ್ತಕಗಳನ್ನು ಓದಿ, ಅದನ್ನು ಅಭ್ಯಾಸ ಮಾಡಲು ಅನೇಕ ವರುಷಗಳ ಪ್ರಯತ್ನ ಮಾಡುವರು.\break ಈ ಪ್ರಯತ್ನದಲ್ಲಿ ಕೆಲವರು ಉಪವಾಸ ಮಾಡುವರು. ಇದರಿಂದ ಅವರು ಬಹಳ ತೆಳ್ಳಗೆ\break ಆಗುವರು. ಯಾರಾದರೂ ಅವರ ಹೊಟ್ಟೆಯನ್ನು ಬೆರಲಿನಿಂದ ಒತ್ತಿದರೆ ಅದಕ್ಕೆ ಅವರ ಬೆನ್ನೆಲುಬಿನ ಸ್ಪರ್ಶವಾಗುವುದು.”

“ಕೆಲವು ಯೋಗಿಗಳು ಬಹಳ ಕಾಲದವರೆಗೆ ಜೀವಿಸುತ್ತಾರೆ.”

ಜೀವಸಹಿತ ನೆಲದ ಒಳಗೆ ಇರುವ ವಿಷಯವನ್ನು ಚರ್ಚಿಸಲಾಯಿತು. ಅದಕ್ಕೆ ಸ್ವಾಮೀಜಿ ಕಮರ್ಶಿಯಲ್​ ಪತ್ರಿಕೆಯ ಬಾತ್ಮಿದಾರನಿಗೆ ‘ಒಬ್ಬನು ಒಳಗೆ ಹೋಗಿ ಬಾಗಿಲುಗಳನ್ನೆಲ್ಲ ಹಾಕಿಕೊಂಡಿರುವುದನ್ನು ನೋಡಿರುವೆನು’ ಎಂದರು. ‘ಅವನು ಒಳಗೆ ಪ್ರವೇಶ ಮಾಡಿದ ಕೂಡಲೆ ಆ ಬಾಗಿಲು ಹಾಕಿಕೊಂಡಿತು. ಅವನು ಅದರ ಒಳಗೆ ಹಲವು ವರುಷಗಳು ಊಟವಿಲ್ಲದೆ ಇದ್ದನು’ ಈ ವಿಷಯವನ್ನು ಕೇಳಿದ ಜನರಲ್ಲಿ ಕುತೂಹಲ ಕೆರಳಿತು.\break ವಿವೇಕಾನಂದರು ಈ ವಿಷಯದ ಬಗ್ಗೆ ಸ್ವಲ್ಪವೂ ಅನುಮಾನಪಡಲಿಲ್ಲ. ನೆಲದೊಳಗೆ\break ಇರುವ ಸ್ಥಿತಿಯಲ್ಲಿ ಅವರ ಬೆಳವಣಿಗೆ ತಾತ್ಕಾಲಿಕವಾಗಿ ನಿಲ್ಲುವುದು ಎಂದರು. ಒಬ್ಬನನ್ನು ನೆಲದೊಳಗೆ ಹೂಳಿ ಅವನ ಗೋರಿಯ ಮೇಲೆ ಗೋಧಿಯ ಬೆಳೆಯನ್ನು ತೆಗೆದರು. ಅನಂತರ ನೆಲ ಅಗೆದು ಮನುಷ್ಯನನ್ನು ಹೊರಗೆ ತೆಗೆದಾಗ ಅವನಿಗೆ ಇನ್ನೂ ಜೀವವಿತ್ತು. ಇದನ್ನು ನೋಡಿರುವವರು ಎಷ್ಟೋ ಜನ ಇರುವರು. ಇದನ್ನು ಮಾಡುವುದಕ್ಕೂ ಪ್ರಯತ್ನಪಟ್ಟರು. ಚಳಿಗಾಲದಲ್ಲಿ ಪ್ರಾಣಿಗಳು ಸುಪ್ತಾವಸ್ಥೆಗೆ ಹೋಗುವುದನ್ನು ನೋಡಿ ಅದರಿಂದ ಕಲಿತುಕೊಂಡಿರಬಹುದೆಂದರು.

ಒಬ್ಬ ಮೇಲಕ್ಕೆ ಒಂದು ಹಗ್ಗವನ್ನು ಎಸೆದು, ನೆಲದಿಂದ ಅದನ್ನು ಹತ್ತಿಕೊಂಡು ಹೋಗಿ ಕೊನೆಗೆ ಅಂತರ್ಧಾನನಾಗುವುದನ್ನು ತಾವು ಇಂಡಿಯಾ ದೇಶದಲ್ಲಿ ಎಂದಿಗೂ ನೋಡಿಲ್ಲ ಎಂದರು; ಈ ಘಟನೆ ಇಂಡಿಯಾ ದೇಶದಲ್ಲಿ ನೋಡಿದ್ದೇವೆ ಎಂದು ಕೆಲವು ಬರಹಗಾರರು ಬರೆದಿರುವರು.

ಬಾತ್ಮಿದಾರ ಪ್ರಶ್ನೆ ಹಾಕುತ್ತಿದ್ದಾಗ ಅಲ್ಲಿದ್ದ ಒಬ್ಬ ಮಹಿಳೆಯು ಸ್ವಾಮೀಜಿಯವರನ್ನು, “ನನ್ನನ್ನು ಯಾರೋ ಕೆಲವರು ಸ್ವಾಮೀಜಿಯವರು ಮಾಯಮಂತ್ರಗಳನ್ನು ಮಾಡಬಲ್ಲರೇ? ಸಂಘಕ್ಕೆ ಸೇರುವ ಸಮಯದಲ್ಲಿ ನೆಲದಲ್ಲಿ ಅವರು ಜೀವಸಹಿತ ಹೂಳಲ್ಪಟ್ಟಿದ್ದರೇ?” ಎಂದು ಪ್ರಶ್ನೆ ಮಾಡಿದರು ಎಂದು ಹೇಳಿದಳು. ಸ್ವಾಮೀಜಿಯವರು ಎರಡು ಪ್ರಶ್ನೆಗಳಿಗೂ ಇಲ್ಲ ಎಂದು ಉತ್ತರ ಕೊಟ್ಟರು. ಇದಕ್ಕೂ ಆಧ್ಯಾತ್ಮಿಕ ಜೀವನಕ್ಕೂ ಏನು ಸಂಬಂಧ ಎಂದು ಸ್ವಾಮೀಜಿ ಕೇಳಿದರು. ಇದರಿಂದ ಒಬ್ಬನ ಮನಸ್ಸು ಪರಿಶುದ್ಧವಾಗುವುದೇ? ನಿಮ್ಮ ಬೈಬಲ್ಲಿನ ಸೈತಾನ್​ ಬಹಳ ಪರಾಕ್ರಮಶಾಲಿ. ಆದರೆ ದೇವರೊಂದಿಗೆ ಹೋಲಿಸಿದರೆ ಅವನು ದೇವರಷ್ಟು ಪರಿಶುದ್ಧನಲ್ಲ ಎಂದರು ಸ್ವಾಮೀಜಿ.

ಹಠಯೋಗದ ಪಂಗಡದ ವಿಷಯವಾಗಿ ಸ್ವಾಮೀಜಿ ಅವರು ಒಂದು ಮಾತನ್ನು ಹೇಳಿದರು: ಶಿಷ್ಯನಿಗೆ ದೀಕ್ಷೆ ಕೊಡುವ ಸಂದರ್ಭದಲ್ಲಿ ಬೈಬಲ್ಲಿನಲ್ಲಿ ಬರುವ ಒಂದು ಮಾತನ್ನು ಅದು ನೆನಪಿಗೆ ತರುವುದು. ಅದೇ ಅವರು ಶಿಷ್ಯನನ್ನು ನಲವತ್ತು ದಿನಗಳು ಒಬ್ಬನೇ ಇರುವಂತೆ ಮಾಡುತ್ತಾರೆ ಎಂಬುದು. ಇದು ಕಾಕತಾಳೀಯ ನ್ಯಾಯ ಇರಬಹುದು.

