
\vspace{-1cm}

\chapter[ವಿಶ್ವೈಕ್ಯತೆ]{ವಿಶ್ವೈಕ್ಯತೆ \protect\footnote{\engfoot{C.W. Vol. V, P. 293}}}

“ಓಕ್​ಬೀಚ್​ ಕ್ರಿಶ್ಚಿಯನ್​ ಯೂನಿಟಿ” ಎಂಬ ಸಂಸ್ಥೆಯಲ್ಲಿ ‘ವಿಶ್ವೈಕ್ಯತೆ’ ಎಂಬ ವಿಷಯವಾಗಿ ಮಾತನಾಡುತ್ತಾ ಸ್ವಾಮೀಜಿ ಹೀಗೆ ಹೇಳಿದರು. ಧರ್ಮಗಳೆಲ್ಲ ಆಂತರ್ಯದಲ್ಲಿ ಒಂದೇ. ಬೈಬಲ್ಲಿನ ದೃಷ್ಟಾಂತ ಕಥೆಯಲ್ಲಿ ಬರುವ ಫ್ಯಾರಸಿಗಳಂತೆ ಕ್ರೈಸ್ತ ಚರ್ಚಿನವರು ತಾವೊಬ್ಬರೇ ಸತ್ಯ, ಮಿಕ್ಕವರೆಲ್ಲ ಮಿಥ್ಯ ಎಂದು ಭಾವಿಸುವರು. ಇತರರಿಗೆಲ್ಲ ಕ್ರೈಸ್ತಬೋಧನೆಯ ಆವಶ್ಯಕತೆ ಇದೆ ಎಂದು ಅವರು ಭಾವಿಸಿ ಅದಕ್ಕಾಗಿ ದೇವರನ್ನು ವಂದಿಸುವರು. ದಯೆಯಿಂದ ಪ್ರೇರಿತವಾಗಿ ಜಗತ್ತು ಕ್ರೈಸ್ತ ಚರ್ಚಿನೊಡನೆ ಸಂಗಮಗೊಳ್ಳಲು\break ಒಪ್ಪಬೇಕಾದರೆ ಕ್ರೈಸ್ತರು ಮೊದಲು ಸಹನಶೀಲರಾಗಬೇಕಾಗಿದೆ. ಉದಾರಿಗಳಾಗಬೇಕಾಗಿದೆ. ದೇವರು ಯಾರ ಹೃದಯದಲ್ಲೂ ಸಾಕ್ಷಿಯನ್ನು ಇಟ್ಟಿಲ್ಲದೆ ಇಲ್ಲ. ಜನರು, ಅದರಲ್ಲೂ ಕ್ರಿಸ್ತನ ಅನುಯಾಯಿಗಳು ಇದನ್ನು ಒಪ್ಪಬೇಕಾಗಿದೆ. ಪ್ರತಿಯೊಬ್ಬ ಒಳ್ಳೆಯವನೂ ದೇವರ ಕುಲಕ್ಕೆ ಸೇರಿದವನೆಂದು ಏಸುಕ್ರಿಸ್ತನು ಭಾವಿಸಿ ಅವನನ್ನು ಅಲ್ಲಿಗೆ ಸೇರಿಸಿಕೊಳ್ಳುವುದಕ್ಕೆ ಸಿದ್ಧನಾಗಿದ್ದನು. ಯಾವುದೊ ಒಂದನ್ನು ನಂಬುವವನಲ್ಲ. ಆದರೆ ದೇವರ ಅಣತಿಯಂತೆ ನಡೆಯುವವನು ಋಜುಮಾರ್ಗದಲ್ಲಿರುವನು. ಸರಿಯಾಗಿರುವುದು ಮತ್ತು ಸರಿಯಾದುದನ್ನು ಮಾಡುವುದು ಇವುಗಳ ತಳಹದಿಯ ಮೇಲೆ ಇಡೀ ಜಗತ್ತು ಒಂದುಗೂಡುವುದು.

