
\chapter[ರಾಮಾಯಣ ]{ರಾಮಾಯಣ \protect\footnote{\engfoot{C.W. Vol. VI, p 102}}}

\centerline{(ಟಿಪ್ಪಣಿಗಳ ತುಣುಕುಗಳು)}

ನಾವು ಪಾಪವನ್ನು ಮಾಡಲಿ, ಪುಣ್ಯವನ್ನು ಮಾಡಲಿ, ಯಾವನು ನಮ್ಮನ್ನು ತ್ಯಜಿಸುವುದಿಲ್ಲವೋ, ಯಾವನು ನಮ್ಮೊಡನೆ ಅನುಗಾಲವೂ ಇರುವನೋ ಅವನನ್ನು ಪೂಜಿಸಿ. ಪ್ರೇಮ ಎಂದಿಗೂ ನಮ್ಮನ್ನು ಕೆಳಗೆ ಎಳೆಯುವುದಿಲ್ಲ. ಪ್ರೇಮಕ್ಕೆ ವ್ಯಾಪಾರ ದೃಷ್ಟಿಯಿಲ್ಲ, ಸ್ವಾರ್ಥವಿಲ್ಲ.

ರಾಮನೇ ದಶರಥನ ಜೀವವಾಗಿದ್ದ. ಅವನು ರಾಜನಾಗಿದ್ದುದರಿಂದ ಆಡಿದ ಮಾತಿಗೆ ತಪ್ಪಲು ಸಾಧ್ಯವಾಗಲಿಲ್ಲ.

“ರಾಮ ಹೋದೆಡೆ ನಾನೂ ಹೋಗುವೆನು” ಎನ್ನುವನು ತಮ್ಮ ಲಕ್ಷಣ.

ಅಣ್ಣನ ಹೆಂಡತಿಯು ಹಿಂದೂ ಸಂಸ್ಕೃತಿಯಲ್ಲಿ ತಾಯಿಗೆ ಸಮಾನ.

ಕೊನೆಗೆ ಸೀತೆಯನ್ನು ಹನುಮಂತ ಕಂಡನು. ಕಾಂತಿಹೀನಳಾಗಿ ತೆಳ್ಳಗಾಗಿ ದಿಗಂತದಲ್ಲಿರುವ ಅರ್ಧಚಂದ್ರನಂತೆ ಇದ್ದಳು ಸೀತೆ. ಪಾತಿವ್ರತ್ಯವೇ ರೂಪ ವೆತ್ತಂತೆ ಇದ್ದಳು ಸೀತೆ. ಅವಳು ತನ್ನ ಗಂಡನ ದೇಹವನ್ನಲ್ಲದೆ ಯಾರ ದೇಹವನ್ನೂ ಮುಟ್ಟುತ್ತಿರಲಿಲ್ಲ.

“ಪವಿತ್ರಳೇ? ಅವಳು ಸಾಕ್ಷಾತ್​ ಪಾತಿವ್ರತ್ಯವೇ ಆಗಿದ್ದಳು” ಎನ್ನುವನು ರಾಮ.

ಸಂಗೀತ ಮತ್ತು ನಾಟಕಗಳು ಇವು ಧರ್ಮವೇ ಆಗಿವೆ. ಅದು ಎಂತಹ ಹಾಡಾದರೂ ಆಗಲಿ, ಪ್ರೀತಿಯದ್ದಾಗಲಿ ಅಥವಾ ಇನ್ನು ಎಂತಹ ಹಾಡೇ ಆಗಲಿ, ಚಿಂತೆಯಿಲ್ಲ, ಹಾಡುವವನ ಮನಸ್ಸು ಆ ಹಾಡಿನಲ್ಲಿದ್ದರೆ ಅವನಿಗೆ ಅದರಿಂದಲೇ ಮುಕ್ತಿ ದೊರಕುವುದು. ಅವನು ಇನ್ನೇನನ್ನೂ ಮಾಡಬೇಕಾಗಿಲ್ಲ. ಅವನ ಮನಸ್ಸೆಲ್ಲ ಅದರಲ್ಲಿ ದ್ದರೆ ಅವನಿಗೆ ಮುಕ್ತಿ ದೊರಕುವುದು. ಗಾನವು ಅದೇ ಗುರಿಯೆಡೆಗೆ ಒಯ್ಯುವುದು ಎನ್ನುವರು.

ಸತಿಯು ಸಹಧರ್ಮಚಾರಿಣಿ, ಹಿಂದೂಗಳು ನೂರಾರು ವ್ರತಗಳನ್ನು ಮಾಡಬೇಕಾಗಿದೆ. ಆದರೆ ಸತಿ ಇಲ್ಲದೇ ಇದ್ದರೆ ಯಾವುದನ್ನೂ ಮಾಡುವಂತೆ ಇಲ್ಲ. ಪುರೋಹಿತರು ಅವರಿಬ್ಬರಿಗೆ ಗಂಟುಹಾಕುವರು, ಅವರು ಆ ಸ್ಥಿತಿಯಲ್ಲಿಯೇ ಹಲವು ಯಾತ್ರೆಗಳನ್ನು ಮಾಡುವರು.

ರಾಮ ತನ್ನ ದೇಹವನ್ನು ತ್ಯಜಿಸಿ ಪರಲೋಕದಲ್ಲಿ ಸೀತೆಯನ್ನು ಸೇರಿದನು. ಸೀತೆ ಪರಿಶುದ್ಧಳು. ಪವಿತ್ರಾತ್ಮಳು. ಆಜನ್ಮ ದುಃಖಿನಿ.

ಭರತಖಂಡದಲ್ಲಿ ಯಾವುದು ಶುಭವೋ, ಒಳ್ಳೆಯದೋ, ಪವಿತ್ರವೋ ಅದಕ್ಕೆಲ್ಲ ಸೀತೆ ಎಂದು ಹೆಸರು. ಸ್ತ್ರೀಯರಲ್ಲಿ ಯಾವುದನ್ನು ಸ್ತ್ರೀತ್ವ ಎನ್ನುವೆವೋ ಅದೇ ಸೀತೆ.

ಸೀತೆ ಸಹಿಷ್ಣುತಾ ಮೂರ್ತಿ. ಅವಳು ಚಿರದುಃಖಿನಿ, ನಿತ್ಯ ಪತಿವ್ರತೆ, ನಿತ್ಯ ಪರಿಶುದ್ಧಳಾದ ಸತಿ. ಅವಳು ಅಷ್ಟೊಂದು ಕಷ್ಟವನ್ನು ಅನುಭವಿಸಿದರೂ ರಾಮನ ವಿರುದ್ಧವಾಗಿ ಒಂದಾದರೂ ಕಟುನುಡಿಯನ್ನು ಆಡಿಲ್ಲ.

ಸೀತೆ ಹಿಂಸೆಗೆ ಪ್ರತಿಹಿಂಸೆಯನ್ನು ಕೊಟ್ಟವಳಲ್ಲ.

“ಸೀತೆಯಂತಾಗಿರಿ.”

