
\chapter[ಮಧುರೆಯಲ್ಲಿ ಸ್ವಾಮಿ ವಿವೇಕಾನಂದರೊಡನೆ ಒಂದು ಗಂಟೆ ]{ಮಧುರೆಯಲ್ಲಿ ಸ್ವಾಮಿ ವಿವೇಕಾನಂದರೊಡನೆ ಒಂದು ಗಂಟೆ \protect\footnote{\engfoot{C.W. Vol. V, P. 204}}}

\centerline{(“ದಿ ಹಿಂದು,” ಮದರಾಸು–ಫೆಬ್ರವರಿ ೧೮೯೭)}

ಪ್ರಶ್ನೆ: “ಜಗತ್ತು ಮಿಥ್ಯ ಎನ್ನುವುದನ್ನು ಕೆಳಗಿನ ಕೆಲವು ದೃಷ್ಟಿಯಲ್ಲಿ ಹೇಳು ವರು. (ಎ) ಅನಂತತೆಯೊಡನೆ ಹೋಲಿಸಿ ನೋಡಿದರೆ ನಾಮರೂಪದ ಜಗತ್ತು ಬಹಳ ಕ್ಷಣಿಕ; (ಬಿ) ಎರಡು ಪ್ರಳಯಗಳಿಗೂ ಮಧ್ಯೆ ಇರುವ ಅಂತರ ಅನಂತತೆ ಯೊಡನೆ ಹೋಲಿಸಿದರೆ ಅತ್ಯಲ್ಪ; (ಸಿ) ಕಪ್ಪೆ ಚಿಪ್ಪಿನಲ್ಲಿ ಬೆಳ್ಳಿ ಮತ್ತು ಹಗ್ಗದಲ್ಲಿ ಹಾವು ಕಾಣುವಂತೆ ಈ ಜಗತ್ತು ಸದ್ಯಕ್ಕೆ ಸತ್ಯವಾಗಿ ಕಾಣುವುದು, ಇದಕ್ಕೆ ಕಾರಣ ಮನಸ್ಸಿನ ಆಗಿನ ಸ್ಥಿತಿ. ಆದರೆ ಅದು ಎಂದೆಂದಿಗೂ ಸತ್ಯವಲ್ಲ; (ಡಿ) ಮೊದಲ ಕೊಂಬಿನಂತೆ ಅಥವಾ ಬಂಜೆಯ ಮಗನಂತೆ ಜಗತ್ತು ಒಂದು ಭ್ರಾಂತಿ. ಮೇಲಿನ ದೃಷ್ಟಿಗಳಲ್ಲಿ ಯಾವುದರ ದೃಷ್ಟಿಯಿಂದ ಅದ್ವೈತಿಗಳು ಪ್ರಪಂಚವನ್ನು ಮಿಥ್ಯ ಎನ್ನುವರು?”

ಉತ್ತರ: “ಹಲವು ಬಗೆಯ ಅದ್ವೈತ ಸಿದ್ಧಾಂತಗಳಿವೆ. ಒಬ್ಬೊಬ್ಬರು ಒಂದೊಂದು ದೃಷ್ಟಿಯಿಂದ ಮಿಥ್ಯೆ ಎನ್ನುವರು. ಶಂಕರಾಚಾರ್ಯರು (ಸಿ) ದೃಷ್ಟಿಯಿಂದ ಮಿಥ್ಯೆ ಎನ್ನುವರು. ಈಗಿರುವ ಮನಸ್ಸಿನ ದೃಷ್ಟಿಯಿಂದ ಈ ಜಗತ್ತು ಎಲ್ಲರಿಗೂ ಎಲ್ಲಾ ವಿಷಯಗಳಿಗೂ ಸತ್ಯವಾಗಿ ಕಾಣುತ್ತಿದೆ. ಆದರೆ ಮನಸ್ಸು ಇನ್ನೂ ಮೇಲಿನ ಮಟ್ಟಕ್ಕೆ ಹೋದರೆ ಅದು ಮಾಯವಾಗುವುದು. ಎದುರಿಗಿರುವ ಒಂದು ಮರದ ಕಾಂಡವನ್ನು ನೋಡಿ ದೆವ್ವ ಎಂದು ಭ್ರಾಂತಿ ಪಡುವೆ. ತತ್ಕಾಲಕ್ಕೆ ದೆವ್ವದ ಭಾವನೆ ಸತ್ಯ. ಅದು ನಿಜವಾದ ದೆವ್ವ ಮನಸ್ಸಿನ ಮೇಲೆ ಯಾವ ಪ್ರತಿಕ್ರಿಯೆಯನ್ನು ಉಂಟು ಮಾಡುವುದೋ ಅದನ್ನೆಲ್ಲಾ ಮಾಡುವುದು. ಆದರೆ ಅದೊಂದು ಮರದ ದಿಮ್ಮಿ ಎಂದು ಅರಿತೊಡನೆ ದೆವ್ವದ ಭಾವನೆ ಮಾಯವಾಗುವುದು. ಮರದ ದಿಮ್ಮಿ ದೆವ್ವ ಎರಡೂ ಒಟ್ಟಿಗೆ ಇರಲಾರವು. ಒಂದು ಇದ್ದರೆ ಮತ್ತೊಂದು ಇರುವುದಿಲ್ಲ.”

ಪ್ರಶ್ನೆ: “ನಾಲ್ಕನೇ ದೃಷ್ಟಿಯನ್ನು ಕೂಡ ಕೆಲವು ಕಡೆ ಶಂಕರಾಚಾರ್ಯರು ಹೇಳಿಲ್ಲವೆ?”

ಉತ್ತರ: “ಇಲ್ಲ, ಇತರರು ಯಾರೊ ಶಂಕರಾಚಾರ್ಯರನ್ನು ಚೆನ್ನಾಗಿ ತಿಳಿದು ಕೊಳ್ಳದವರು ಅವರ ತತ್ತ್ವವನ್ನು ಅತಿರೇಕಕ್ಕೆ ಒಯ್ಯುವ ಅವಸರದಲ್ಲಿ ನಾಲ್ಕನೇ ದೃಷ್ಟಿಯನ್ನು ಅವರ ಮೇಲೆ ಆರೋಪಿಸಿರುವರು. ಮೊದಲನೆ ಮತ್ತು ಎರಡನೆ ದೃಷ್ಟಿಯನ್ನು ಬೇರೆ ಅದ್ವೈತ ಸಿದ್ಧಾಂತಿಗಳು ಬಳಸಿರುವರೇ ಹೊರತು ಶಂಕರಾಚಾರ್ಯರು ಅದಕ್ಕೆ ತಮ್ಮ ಒಪ್ಪಿಗೆಯನ್ನು ಕೊಟ್ಟಿಲ್ಲ.”

ಪ್ರಶ್ನೆ: “ತೋರಿಕೆಯ ಸತ್ಯಕ್ಕೆ ಕಾರಣವೇನು?

ಉತ್ತರ: “ನೀವು ಮರದ ದಿಮ್ಮಿಯನ್ನು ದೆವ್ವವೆಂದು ಭ್ರಮಿಸುವುದಕ್ಕೆ ಕಾರಣವೇನು? ಜಗತ್ತು ಯಾವಾಗಲೂ ಒಂದೇ ಸಮನಾಗಿರುವುದು. ಆದರೆ ನಿಮ್ಮ ಮನಸ್ಸು ಈ ವ್ಯತ್ಯಾಸಕ್ಕೆಲ್ಲಾ ಕಾರಣ.”

ಪ್ರಶ್ನೆ: “ವೇದ ಅನಾದಿ ನಿತ್ಯವಾದುದು ಎನ್ನುವುದಕ್ಕೆ ನಿಜವಾದ ಅರ್ಥವೇನು? ಇದು ವೇದಗಳಲ್ಲಿರುವ ಸತ್ಯಕ್ಕೆ ಅನ್ವಯಿಸುತ್ತದೆಯೆ ಅಥವಾ ವೇದೋಕ್ತಿಗೆ ಅನ್ವಯಿಸುವುದೆ? ಇದು ವೇದಗಳಲ್ಲಿ ಬರುವ ಸತ್ಯಕ್ಕೆ ಅನ್ವಯಿಸುವ ಹಾಗಿದ್ದರೆ ತರ್ಕ, ರೇಖಾಗಣಿತ, ರಸಾಯನಶಾಸ್ತ್ರ ಮುಂತಾದುವುಗಳು ಕೂಡ ಆದಿ ಅಂತ್ಯ ವಿಲ್ಲದವೇ ಆಯಿತಲ್ಲ? ಏಕೆಂದರೆ ಅವುಗಳಲ್ಲೂ ನಿತ್ಯ ಸತ್ಯಗಳೇ ಇವೆ.”

ಉತ್ತರ: “ವೇದಗಳಲ್ಲಿರುವ ಸತ್ಯ ಸ್ಥಿರವಾದುದು ಮತ್ತು ಅವಿಕಾರಿಯಾದುದು ಎಂಬ ದೃಷ್ಟಿಯಿಂದ ವೇದವನ್ನು ಅನಾದಿ, ಅನಂತ ಎಂದು ನೋಡುತ್ತಿದ್ದ ಕಾಲವಿತ್ತು. ಅನಂತರ ವೇದಗಳ ಉಚ್ಚಾರಣೆ, ಅದರ ಅರ್ಥದೊಂದಿಗೆ ಮುಖ್ಯವಾಯಿತು. ಈ ಮಂತ್ರಗಳು ಭಗವಂತನಿಂದ ಬಂದವು ಎಂದು ಭಾವಿಸಿದರು. ಇನ್ನೂ ಕೆಲವು ಕಾಲದ ಮೇಲೆ, ಇವುಗಳ ಅರ್ಥದ ಮೂಲಕ ನೋಡಿದರೆ ಎಲ್ಲಾ ಭಗವಂತನ ಮೂಲದಿಂದ ಬಂದವುಗಳಲ್ಲ ಎಂದು ತಿಳಿದುಬಂತು. ಏಕೆಂದರೆ ಅವುಗಳಲ್ಲಿ ಎಷ್ಟೋ ಪಾಪಕರವಾದ ಕೆಲಸಗಳನ್ನು ಮನುಷ್ಯ ಮಾಡಬೇಕೆಂದು ಹೇಳಿದೆ, ಉದಾಹರಣೆಗೆ ಪ್ರಾಣಿಹಿಂಸೆ. ವೇದಗಳಲ್ಲಿ ಎಷ್ಟೋ ಕೆಲಸಕ್ಕೆ ಬಾರದ ಕಥೆಗಳೂ ಇರುತ್ತವೆ. ವೇದ ಅನಾದಿ ಅನಂತ ಎಂಬುದಕ್ಕೆ ನಿಜವಾದ ಅರ್ಥ, ಅದರಲ್ಲಿರುವ ಸತ್ಯ ನಿತ್ಯವಾದುದು, ಎಂದಿಗೂ ಬದಲಾಯಿಸುವುದಿಲ್ಲ ಎಂಬುದು. ತರ್ಕ, ರೇಖಾಗಣಿತ, ರಸಾಯನಶಾಸ್ತ್ರ ಇವುಗಳೂ ಕೂಡ ಸತ್ಯವಾಗಿರುವ, ಬದಲಾಯಿಸದೆ ಇರುವ ಸತ್ಯವನ್ನು ವಿವರಿಸುತ್ತವೆ. ಆ ದೃಷ್ಟಿಯಲ್ಲಿ ಇವು ಕೂಡ ಅನಾದಿ ಮತ್ತು ಅನಂತವೆ. ಆದರೆ ವೇದದಲ್ಲಿ ಇಲ್ಲದೆ ಇರುವ ಸತ್ಯವೇ ಇಲ್ಲ. ವೇದದಲ್ಲಿ ಇಲ್ಲದೆ ಇರುವ ಯಾವುದಾದರೂ ಸತ್ಯವನ್ನು ಹೇಳಿ ಎಂದು ಯಾರನ್ನಾದರೂ ಪ್ರಶ್ನಿಸುತ್ತೇನೆ.”

ಪ್ರಶ್ನೆ: “ಅದ್ವೈತ ಸಿದ್ಧಾಂತದ ಪ್ರಕಾರ ಮುಕ್ತಿ ಎಂದರೇನು? ಈ ಸ್ಥಿತಿಯಲ್ಲಿ ಪ್ರಜ್ಞೆ ಇರುವುದೆ? ಅದ್ವೈತದ ಮುಕ್ತಿಗೂ ಬೌದ್ಧರ ನಿರ್ವಾಣಕ್ಕೂ ಏನಾದರೂ ವ್ಯತ್ಯಾಸವಿದೆಯೆ?”

ಉತ್ತರ: “ಮುಕ್ತಿಯಲ್ಲಿ ಪ್ರಜ್ಞೆಯಿದೆ. ನಾವು ಇದನ್ನು ಅತಿ ಪ್ರಜ್ಞೆ ಎನ್ನುವೆವು. ಅದು ನಿಮ್ಮ ಈಗಿನ ಪ್ರಜ್ಞೆಗಿಂತ ಬೇರೆಯಾಗಿರುವುದು. ಮುಕ್ತಿಯಲ್ಲಿ ಪ್ರಜ್ಞೆಯಿಲ್ಲ ಎನ್ನುವುದು ತರ್ಕಬದ್ಧವಲ್ಲ. ಪ್ರಜ್ಞೆಯಲ್ಲಿ ಬೆಳಕಿನಂತೆ ಮೂರು ವಿಧಗಳಿವೆ. ಮಂದ, ಮಧ್ಯಮ ಮತ್ತು ಉತ್ತಮ. ಬೆಳಕು ತುಂಬಾ ಕೋರೈಸುತ್ತಿರುವಾಗ ಕಾಣುವುದಿಲ್ಲ. ಹೇಗೆ ಒಂದು ಮಂದ ದೀಪದಲ್ಲಿ ಕಣ್ಣು ಕಾಣಿಸುವುದಿಲ್ಲವೋ ಹಾಗೆ ಅತಿ ತೀವ್ರ ಬೆಳಕಿನಲ್ಲಿಯೂ ಕಣ್ಣು ಕಾಣಿಸುವುದಿಲ್ಲ. ಬೌದ್ಧರು ಏನಾದರೂ ಹೇಳಲಿ, ಅವರ ನಿರ್ವಾಣದಲ್ಲಿ ಪ್ರಜ್ಞೆ ಇರಲೇಬೇಕು. ನಾವು ಕೊಡುವ ಮುಕ್ತಿಯ ವಿವರಣೆ ಇತ್ಯಾತ್ಮಕವಾದುದು; ಬೌದ್ಧರು ಕೊಡುವ ವಿವರಣೆ ನೇತ್ಯಾತ್ಮಕವಾದುದು.”

ಪ್ರಶ್ನೆ: “ನಿರ್ವಿಶೇಷ ಬ್ರಹ್ಮವು ವಿಶ್ವವನ್ನು ನಿರ್ಮಿಸುವುದಕ್ಕಾಗಿ ಏತಕ್ಕೆ ಸವಿಶೇಷ ರೂಪವನ್ನು ತಾಳಬೇಕು?”

ಉತ್ತರ: “ನಿಮ್ಮ ಪ್ರಶ್ನೆಯೇ ತಾರ್ಕಿಕವಾಗಿಲ್ಲ. ಬ್ರಹ್ಮ ಅವಾಙ್ಮಾನಸ ಗೋಚರ ಎಂದರೆ ಮಾತು, ಮನಸ್ಸುಗಳಿಗೆ ಮೀರಿದುದು. ದೇಶ ಕಾಲ ನಿಮಿತ್ತದ ಆಚೆ ಇರುವುದನ್ನು ಮನಸ್ಸು ಗ್ರಹಿಸಲಾರದು. ವಿಚಾರ ತರ್ಕ ಇರುವುದೆಲ್ಲ ದೇಶ ಕಾಲ ನಿಮಿತ್ತದ ಒಳಗೆ. ಸ್ಥಿತಿ ಹೀಗಿರುವಾಗ ಮನಸ್ಸಿಗೆ ಮೀರಿರುವುದನ್ನು ಕುರಿತು ಪ್ರಶ್ನಿಸುವುದು ವ್ಯರ್ಥ.”

ಪ್ರಶ್ನೆ: “ಗೂಢವಾದ ಭಾವನೆಗಳನ್ನು ಪುರಾಣಗಳು ರೂಪಕಗಳಾಗಿ ಚಿತ್ರಿಸುತ್ತವೆ ಎಂದು ಹೇಳಲಾಗಿದೆ. ಪುರಾಣಗಳಲ್ಲಿ ಯಾವ ಚಾರಿತ್ರಿಕ ಸತ್ಯವೂ ಇಲ್ಲದೆ ಇರಬಹುದು, ಆದರೆ ಅತ್ಯುನ್ನತವಾದ ಆದರ್ಶವನ್ನು ಕಾಲ್ಪನಿಕ ಪಾತ್ರಗಳ ಮೂಲಕ ಅಲ್ಲಿ ಚಿತ್ರಿಸಲಾಗಿದೆ ಎಂದು ಕೆಲವೊಮ್ಮೆ ಹೇಳಲಾಗಿದೆ. ವಿಷ್ಣುಪುರಾಣ, ರಾಮಾ ಯಣ, ಭಾರತಗಳನ್ನು ತೆಗೆದುಕೊಳ್ಳಿ. ಅಲ್ಲಿ ಚಾರಿತ್ರಿಕ ಘಟನೆಗಳಿವೆಯೋ? ಇಲ್ಲವೆ ಇವುಗಳೆಲ್ಲ ಆಧ್ಯಾತ್ಮಿಕ ತತ್ತ್ವಗಳನ್ನು ವಿವರಿಸುವ ಕಾಲ್ಪನಿಕ ಘಟನೆಗಳೊ? ಅಥವಾ ಮಾನವರು ಹೇಗಿರಬೇಕೆಂದು ತೋರಿಸಲು ಅವರ ಮುಂದೆ ಇಟ್ಟಿರುವ ಪರಮಾದರ್ಶ ಗಳೊ? ಅಥವಾ ಹೋಮರನ ಕಾವ್ಯದಂತೆ ಇವುಗಳೆಲ್ಲ ಬರೀ ಕಾವ್ಯಗಳೋ?

ಉತ್ತರ: “ಎಲ್ಲಾ ಪುರಾಣಗಳಿಗೂ ಕೆಲವು ಚಾರಿತ್ರಿಕ ಘಟನೆಗಳೇ ತಳಹದಿ. ಪುರಾಣಗಳ ಗುರಿ ಪರಮಸತ್ಯವನ್ನು ಜನರಿಗೆ ಬೇರೆ ಬೇರೆ ವಿಧದಲ್ಲಿ ಹೇಳುವುದಾಗಿದೆ. ಅವುಗಳಲ್ಲಿ ಚಾರಿತ್ರಿಕ ಸತ್ಯವಿಲ್ಲದೇ ಇದ್ದರೂ ಪಾರಮಾರ್ಥಿಕ ದೃಷ್ಟಿಯಿಂದ ಇವು ಒಂದು ಅಧಿಕಾರವಾಣಿಯಿಂದ ಮಾತನಾಡುತ್ತವೆ. ಉದಾಹರಣೆಗೆ ರಾಮಾಯಣ ವನ್ನು ತೆಗೆದುಕೊಳ್ಳಿ. ಆದರ್ಶ ಶೀಲ ದೃಷ್ಟಿಯಿಂದ ನೋಡಿದರೆ ರಾಮನಂತಹ ವ್ಯಕ್ತಿ ಇರಬೇಕೆಂಬುದು ಆವಶ್ಯಕವೇನೂ ಅಲ್ಲ. ರಾಮಾಯಣ, ಮಹಾಭಾರತಗಳಲ್ಲಿ ಬರುವ ಧರ್ಮದ ಮಹಿಮೆ ರಾಮ ಅಥವಾ ಕೃಷ್ಣನ ಇತಿಹಾಸದ ಮೇಲೆ ನಿಂತಿಲ್ಲ. ಅವರಿರಲಿಲ್ಲವೆಂದು ನಂಬಿದರೂ ಯಾವ ಪರಮ ಆದರ್ಶಗಳನ್ನು ಅವರು ಮಾನವನ ಮುಂದೆ ಇಡುವರೋ ಅವಕ್ಕೆ ದೊಡ್ಡ ಪ್ರಮಾಣದಂತೆ ಇವೆ ಈ ಗ್ರಂಥಗಳು.

ನಮ್ಮ ದರ್ಶನಗಳ ಸತ್ಯಗಳು ಯಾವ ವ್ಯಕ್ತಿಗಳನ್ನೂ ಅವಲಂಬಿಸಿಲ್ಲ. ಕೃಷ್ಣ ತನ್ನದೇ ಆದ ಯಾವ ಹೊಸ ಸಿದ್ಧಾಂತವನ್ನೂ ಪ್ರಪಂಚಕ್ಕೆ ಬೋಧಿಸಲಿಲ್ಲ. ಹಿಂದಿನ ಶಾಸ್ತ್ರಗಳಲ್ಲಿ ಇಲ್ಲದೆ ಇರುವುದನ್ನು ರಾಮಾಯಣ ಏನೂ ಹೇಳುವುದಿಲ್ಲ. ಕ್ರಿಸ್ತನಿಲ್ಲದೆ ಕ್ರೈಸ್ತಧರ್ಮ, ಮಹಮ್ಮದನಿಲ್ಲದೆ ಮಹಮ್ಮದೀಯ ಧರ್ಮ, ಬುದ್ಧ ನಿಲ್ಲದೆ ಬೌದ್ಧಧರ್ಮ ನಿಲ್ಲಲಾರದು ಎಂದು ಬೇಕಾದರೆ ಹೇಳಬಹುದು. ಆದರೆ ಹಿಂದೂಧರ್ಮ ಯಾವ ವ್ಯಕ್ತಿಯ ಮೇಲೂ ನಿಂತಿಲ್ಲ. ಪುರಾಣಗಳಲ್ಲಿರುವ ತಾತ್ತ್ವಿಕ ಸತ್ಯಗಳನ್ನು ಪ್ರಮಾಣೀಕರಿಸಲು ಅಲ್ಲಿ ಬರುವ ವ್ಯಕ್ತಿಗಳು ಚಾರಿತ್ರಿಕವೆ ಅಥವಾ ಕಾಲ್ಪನಿಕವೆ ಎಂಬುದನ್ನು ಗಮನಿಸಲೇಬೇಕಾಗಿಲ್ಲ. ಪುರಾಣದ ಉದ್ದೇಶ ಜನರನ್ನು ಶಿಕ್ಷಿತರನ್ನಾಗಿ ಮಾಡುವುದು. ಅದನ್ನು ಬರೆದ ಋಷಿಗಳು ಆಗಿನ ಕಾಲದ ಜನರ ನೆನಪಿನಲ್ಲಿದ್ದ ಕೆಲವು ಚಾರಿತ್ರಿಕ ವ್ಯಕ್ತಿಗಳನ್ನು ತೆಗೆದುಕೊಂಡು, ಅವರಲ್ಲಿ ಎಲ್ಲಾ ಒಳ್ಳೆಯದನ್ನೊ ಕೆಟ್ಟದ್ದನ್ನೊ ತಮಗೆ ಇಷ್ಟಬಂದಂತೆ ಆರೋಪಮಾಡಿ, ಮಾನವನೀತಿ ನಡವಳಿಕೆಗೆ ಅವರನ್ನು ಆದರ್ಶವನ್ನಾಗಿ ಮಾಡಿದರು. ರಾಮಾಯಣದಲ್ಲಿ ಬರುವ ದಶಕಂಠನು ನಿಜವಾಗಿದ್ದಿರಬೇಕೆ? ನಾವು ತಿಳಿದುಕೊಳ್ಳಬೇಕಾದ ಒಂದು ಆದರ್ಶದ ಪ್ರತಿನಿಧಿ ಅವನು. ಅವನು ನಿಜವಾಗಿ ಇದ್ದನೆ ಇಲ್ಲವೆ ಎಂಬುದಲ್ಲ ಮುಖ್ಯ. ನೀವು ಬೇಕಾದರೆ ಕೃಷ್ಣನನ್ನು ಇನ್ನೂ ಆಕರ್ಷಕ ರೀತಿಯಲ್ಲಿ ಚಿತ್ರಿಸಬಹುದು. ನಿಮ್ಮ ಭಾವನೆಯ ಭವ್ಯತೆಗೆ ತಕ್ಕಂತೆ ವಿವರಣೆಗಳೂ ವ್ಯತ್ಯಾಸವಾಗುವುವು. ಆದರೆ ಅದರ ಹಿಂದೆ ಪುರಾಣದಲ್ಲಿರುವ ಗಹನ ಸತ್ಯವಿದೆ.”

ಪ್ರಶ್ನೆ: “ಒಬ್ಬನು ಸಿದ್ಧನಾದರೆ ತನ್ನ ಪೂರ್ವಜನ್ಮಗಳನ್ನು ಜ್ಞಾಪಿಸಿಕೊಳ್ಳು ವುದಕ್ಕೆ ಸಾಧ್ಯವೆ? ಹಿಂದಿನ ಜನ್ಮದ ಮಿದುಳಿನಲ್ಲಿ ಯಾವ ಅನುಭವಗಳನ್ನು ಅವನು ಶೇಖರಿಸಿದ್ದನೋ ಆ ಮಿದುಳು ಈಗಿರುವುದಿಲ್ಲ. ಈ ಜನ್ಮದಲ್ಲಿ ಅವನಿಗೆ ಬೇರೊಂದು ಮಿದುಳಿದೆ. ಹೀಗಿದ್ದರೆ ಈ ಮಿದುಳಿಗೆ ಹಿಂದಿನ ಜನ್ಮದ ಮತ್ತೊಂದು ಮಿದುಳಿನ ಅನುಭವವನ್ನು ಜ್ಞಾಪಿಸಿಕೊಳ್ಳುವುದು ಹೇಗೆ ಸಾಧ್ಯ?

ಸ್ವಾಮೀಜಿ: “ಸಿದ್ಧ ಎಂದರೇನು?”

ಪ್ರಶ್ನೆ: “ತನ್ನ ಸ್ವಭಾವದ ಸುಪ್ತಶಕ್ತಿಗಳನ್ನು ಜಾಗೃತಗೊಳಿಸಿಕೊಂಡಿರುವವನು.”

ಉತ್ತರ: “ಸುಪ್ತವಾಗಿರುವುದು ಹೇಗೆ ಜಾಗೃತವಾಗುವುದೋ ನನಗೆ ಗೊತ್ತಿಲ್ಲ. ನೀವು ಏನನ್ನು ಉದ್ದೇಶಿಸುವಿರೋ ಅದು ನನಗೆ ಗೊತ್ತಿದೆ. ಆದರೆ ಉಪಯೋಗಿಸುವ ಪದ ನಿರ್ದಿಷ್ಟವಾಗಿರಬೇಕು, ಸ್ಪಷ್ಟವಾಗಿರಬೇಕು. ಯಾವ ಶಕ್ತಿಯ ಮೇಲೆ ಆವರಣವಿತ್ತೊ ಅದು ಆಚೆಗೆ ಸರಿಯಿತು ಎನ್ನಬಹುದು. ಯಾರು ತಮ್ಮ ನಿಜವಾದ ಶಕ್ತಿಯನ್ನು ಅರಿತಿರುವರೋ ಅವರಿಗೆ ತಮ್ಮ ಹಿಂದಿನ ಜನ್ಮದ ಸ್ಮರಣೆ ಸಾಧ್ಯ. ಏಕೆಂದರೆ ಈಗಿರುವ ಅವರ ಮಿದುಳು ಹಿಂದಿನ ಜನ್ಮದ ಸೂಕ್ಷ್ಮ ಶರೀರದ ಬೀಜವಾಗಿದೆ.”

ಪ್ರಶ್ನೆ: “ಹಿಂದೂಗಳಲ್ಲದವರು ಹಿಂದೂಗಳಾಗುವುದಕ್ಕೆ ಈ ಧರ್ಮವು ಅವಕಾಶ ಕೊಡುವುದೆ? ಬ್ರಾಹ್ಮಣನು ಚಂಡಾಲನ ಉಪದೇಶವನ್ನು ಕೇಳಬಹುದೆ?”

ಉತ್ತರ: ಹಿಂದೂಧರ್ಮವು ಮತಾಂತರವನ್ನು ಒಪ್ಪಿಕೊಳ್ಳುತ್ತದೆ. ಯಾರಾದರೂ ಆಗಿರಲಿ, ಶೂದ್ರನಾಗಲಿ, ಚಂಡಾಲನಾಗಲಿ, ಅವನು ತತ್ತ್ವವನ್ನು ಬ್ರಾಹ್ಮಣನಿಗೂ ಬೋಧಿಸಬಹುದು. ಸತ್ಯವನ್ನು ಎಂತಹ ಪಾಮರನಿಂದಲೂ ಕಲಿತು ಕೊಳ್ಳಬಹುದು. ಅವನು ಯಾವ ಜಾತಿ ಅಥವಾ ಕುಲಕ್ಕೆ ಸೇರಿದ್ದರೂ ಚಿಂತೆಯಿಲ್ಲ.”

ಇಲ್ಲಿ ಸ್ವಾಮಿಗಳು ಇದನ್ನು ವಿವರಿಸುವುದಕ್ಕೆ ಶಾಸ್ತ್ರದಿಂದ ಒಂದು ಶ್ಲೋಕವನ್ನು ಉದಹರಿಸಿದರು.

ಸಂದರ್ಶನ ಮುಗಿಯಿತು. ತಮ್ಮ ಕಾರ್ಯಕ್ರಮದ ಪ್ರಕಾರ ಅವರು ದೇವ ಸ್ಥಾನಕ್ಕೆ ಹೋಗುವ ಸಮಯವಾಯಿತು. ಅವರು ಬಾತ್ಮೀದಾರರನ್ನು ಬೀಳ್ಕೊಂಡು ದೇವಸ್ಥಾನಕ್ಕೆ ಹೋದರು.

