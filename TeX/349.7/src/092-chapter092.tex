
\chapter[ಹಿಂದೂಗಳು ಮತ್ತು ಕ್ರೈಸ್ತರು ]{ಹಿಂದೂಗಳು ಮತ್ತು ಕ್ರೈಸ್ತರು \protect\footnote{\engfoot{C.W. Vol, VIII, P. 209}}}

\centerline{(೧೮೯೪ರ ಫೆಬ್ರವರಿ ೨೧ರಂದು ಡೆಟ್ರಾಯಿಟ್​ನಲ್ಲಿ ನೀಡಿದ ಉಪನ್ಯಾಸ-ಡೆಟ್ರಾಯಿಟ್​ ಫ್ರೀ ಪ್ರೆಸ್​ ಪತ್ರಿಕೆಯಲ್ಲಿ ವರದಿಯಾದಂತೆ)}

ವಿವಿಧ ತತ್ತ್ವಗಳಲ್ಲಿ ಒಂದಾದ ಹಿಂದೂಧರ್ಮದ ಸಿದ್ಧಾಂತ ಯಾವಾಗಲೂ ಧ್ವಂಸದ ಕಡೆ ಗಮನಕೊಡವುದಿಲ್ಲ; ಅದು ಇತರ ತತ್ತ್ವಗಳೊಂದಿಗೆ ಹೊಂದಿ ಕೊಂಡು ಹೋಗಬೇಕೆಂದು ಯತ್ನಿಸುವದು. ಭರತಖಂಡಕ್ಕೆ ಯಾವುದಾದರೂ ಒಂದು ಹೊಸ ಭಾವನೆ ಬಂದರೆ ನಾವು ಅದನ್ನು ವಿರೋಧಿಸುವುದಿಲ್ಲ; ಅದನ್ನು ಸ್ವೀಕರಿಸಿ ಅದರೊಂದಿಗೆ ಹೊಂದಿಕೊಂಡು ಹೋಗಲು ಯತ್ನಿಸುವೆವು. ಏಕೆಂದರೆ ಭಗವಂತನ ಅವತಾರವಾದ ಶ‍್ರೀಕೃಷ್ಣ ನಮಗೆ ಬಹಳ ಹಿಂದೆಯೇ ಈ ಮಾರ್ಗವನ್ನು ತೋರಿಸಿರುವನು. ಅವನು “ನಾನು ಭಗವಂತನ ಅವತಾರ, ನಾನೇ ಎಲ್ಲಾ ವಿದ್ಯೆಗಳ ಪ್ರೇರಕ, ಎಲ್ಲಾ ಧರ್ಮಗಳ ಸ್ಫೂರ್ತಿ” ಎಂದಿರುವನು. ಆದಕಾರಣ ನಾವು ಯಾವುದನ್ನೂ ತಿರಸ್ಕರಿಸುವಂತಿಲ್ಲ.

ನಮಗೂ ಕ್ರೈಸ್ತರಿಗೂ ಮತ್ತೊಂದು ವ್ಯತ್ಯಾಸವಿದೆ. ಅದನ್ನು ಹಿಂದೂಗಳು ಎಂದಿಗೂ ಯಾರಿಗೂ ಬೋಧಿಸಿಲ್ಲ. ಅದು ಯಾವುದೆಂದರೆ ಜೀಸಸ್ಸನ ರಕ್ತದ ಮೂಲಕ ಅಥವಾ ಮತ್ತಾರದೋ ರಕ್ತದಿಂದ ನಾವು ಮುಕ್ತರಾಗುತ್ತೇವೆ ಮತ್ತು ಪರಿಶುದ್ಧರಾಗುತ್ತೇವೆ ಎಂಬ ಬಾವನೆ. ಯೆಹೂದ್ಯರಲ್ಲಿರುವಂತೆ ನಮ್ಮಲ್ಲಿ ಯಾಗ ಯಜ್ಞಗಳಿದ್ದವು. ಯಾಗ ಎಂದರೆ ಇದೇ ಅರ್ಥ. ನನಗೆ ಊಟ ಮಾಡುವುದಕ್ಕೆ ಸ್ವಲ್ಪ ಆಹಾರವಿದೆ. ಅದರಲ್ಲಿ ಸ್ವಲ್ಪವನ್ನು ದೇವರಿಗೆ ಕೊಡದೆ ಇದ್ದರೆ ತಪ್ಪಾಗುತ್ತದೆ. ಆದಕಾರಣ ದೇವರಿಗೆ ಕೊಡುತ್ತೇಲೆ. ಇದು ಶುದ್ಧ ಮತ್ತು ಸರಳ ಭಾವನೆ. ಆದರೆ ಯಹೂದ್ಯರು ತಾವು ಬಲಿಕೊಡುವ ಕುರಿಗೆ ತಮ್ಮ ಪಾಪಗಳನ್ನೆಲ್ಲಾ ಆರೋಪಿಸುವರು. ಹಾಗೆ ಮಾಡಿ ಕುರಿಯನ್ನು ಬಲಿಕೊಟ್ಟು ತಾವು ಪಾಪದಿಂದ ತಪ್ಪಿಸಿಕೊಳ್ಳುವರು. ಭರತಖಂಡದಲ್ಲಿ ಇಂತಹ ವಿಚಿತ್ರ ಭಾವನೆ ಬೆಳೆಯಲಿಲ್ಲ. ಇದರಿಂದ ನನಗೇನೋ ಸಂತೋಷ. ನನಗೆ ಸಿದ್ದಾಂತದ ಮೂಲಕ ಮುಕ್ತನಾಗಲು ಇಚ್ಛೆಯಿಲ್ಲ. ಯಾರಾದರೂ ಬಂದು ನನ್ನ ರಕ್ತದಿಂದ ಉದ್ದಾರ ವಾಗು ಎಂದರೆ ನಾನು ಅವನಿಗೆ “ಬೇಕಾಗಿಲ್ಲ, ಬೇಕಾದರೆ ನಾನು ನರಕಕ್ಕೆ ಹೋಗುತ್ತೇನೆ. ಇನ್ನೊಬ್ಬ ನಿರಪರಾಧಿಯ ರಕ್ತವನ್ನು ತೆಗೆದುಕೊಂಡು ಸ್ವರ್ಗಕ್ಕೆ ಹೋಗಲು ನಾನೇನು ಒಬ್ಬ ಹೇಡಿಯಲ್ಲ. ನಾನು ನರಕಕ್ಕೆ ಹೋಗಲು ಅಣಿಯಾಗಿ ರುವೆನು” ಎನ್ನುವೆನು. ಆದಕಾರಣ ಈ ಸಿದ್ಧಾಂತ ನಮ್ಮಲ್ಲಿ ತಲೆಯೆತ್ತಲಿಲ್ಲ. ನಮ್ಮ ಅವತಾರಪುರುಷ, ‘ಎಂದು ಪ್ರಪಂಚದಲ್ಲಿ ಅಧರ್ಮ ಮತ್ತು ಪಾಪ ಹೆಚ್ಚಾಗುವುದೋ ಆಗ ನಾನು ಅವತಾರ ಮಾಡುತ್ತೇನೆ,’ ಎನ್ನುವನು. ಅವನು ಪ್ರಪಂಚಕ್ಕೆ ಬಂದು ತನ್ನ ಮಕ್ಕಳನ್ನು ರಕ್ಷಿಸುವನು. ಇದನ್ನು ಅನೇಕ ಕಡೆಗಳಲ್ಲಿ, ಅನೇಕ ವೇಳೆಗಳಲ್ಲಿ ಹಿಂದಿನಿಂದ ಮಾಡುತ್ತ ಬಂದಿರುವನು. ಪ್ರಪಂಚದ ಯಾವ ಮೂಲೆಯಲ್ಲಾದರೂ ಆಗಲಿ ಒಬ್ಬ ಅಸಾಧಾರಣ ಮಹಾತ್ಮ ಪ್ರಪಂಚವನ್ನು ಉದ್ಧರಿಸಲು ಯತ್ನಿಸುತ್ತಿದ್ದರೆ ಅವನಲ್ಲಿ ದೇವರು ಇರುವನು.

ಆದಕಾರಣವೇ ನಾವು ಯಾವ ಧರ್ಮದವರೊಂದಿಗೂ ಹೋರಾಡುವುದಿಲ್ಲ. ನಮ್ಮದೊಂದೇ ಮುಕ್ತಿಗೆ ಮಾರ್ಗ ಎಂದು ನಾವು ಹೇಳುವುದಿಲ್ಲ. ಪ್ರತಿಯೊಬ್ಬರೂ ಪರಿಪೂರ್ಣರಾಗಬಹುದು. ಇದಕ್ಕೆ ಪ್ರಮಾಣವೇನು? ಏಕೆಂದರೆ ಪ್ರಪಂಚದಲ್ಲೆಲ್ಲಾ ಅತ್ಯಂತ ಪವಿತ್ರರಾದವರು ಇರುವುದನ್ನು ನಾವು ನೋಡುತ್ತೇವೆ. ನಮ್ಮ ಧರ್ಮಕ್ಕೆ ಸೇರದ ಎಷ್ಟೋ ಮಂದಿ ಒಳ್ಳೆಯ ಸ್ತ್ರೀಪುರುಷರು ಇರುವರು. ಆದಕಾರಣ ನಮ್ಮದೊಂದೇ ಮುಕ್ತಿಗೆ ಮಾರ್ಗ ಎನ್ನುವುದಿಲ್ಲ ನಾವು. “ಹಲವು ನದಿಗಳು ಬೇರೆ ಬೇರೆ ಬೆಟ್ಟಗಳಲ್ಲಿ ಹುಟ್ಟಿ ಕೊನೆಗೆ ಅವೆಲ್ಲಾ ಸಾಗರಕ್ಕೆ ಸೇರುವಂತೆ ಭಿನ್ನ ಭಿನ್ನ ದೃಷ್ಟಿಯಿಂದ ಉತ್ಪನ್ನವಾದ ಧರ್ಮಗಳೆಲ್ಲ ಕೊನೆಗೆ ನನ್ನ ಸಮೀಪಕ್ಕೆ ಬರುವುವು.” ಪ್ರತಿಯೊಬ್ಬ ಬಾಲಕನೂ ಪ್ರತಿದಿನ ಭರತಖಂಡದಲ್ಲಿ ಮಾಡುವ ಪ್ರಾರ್ಥನೆಯ ಒಂದು ಭಾಗ ಇದು. ಇಂತಹ ಪ್ರಾರ್ಥನೆ ಹಿಂದಿನಿಂದಲೂ ಇರುವಾಗ ಧರ್ಮದ ಹೆಸರಿನಲ್ಲಿ ವ್ಯಾಜ್ಯವಾಡಲು ಅವಕಾಶವೇ ಇರುವುದಿಲ್ಲ. ಇದು ಭರತಖಂಡದ ದಾರ್ಶನಿಕ ದೃಷ್ಟಿ. ನಾವು ಎಲ್ಲಾ ಮಹಾತ್ಮರನ್ನೂ ಗೌರವಿಸುತ್ತೇವೆ, ಅದರಲ್ಲಿಯೂ ಹಿಂದಿನ ಧರ್ಮಗಳನ್ನೆಲ್ಲಾ ಉದಾರ ದೃಷ್ಟಿಯಿಂದ ನೋಡಿದ ಶ‍್ರೀಕೃಷ್ಣನ ಹೃದಯ ವೈಶಾಲ್ಯಕ್ಕೆ ಮಣಿಯುತ್ತೇವೆ.

ವಿಗ್ರಹದ ಎದುರು ನಮಸ್ಕರಿಸುತ್ತಿರುವುದನ್ನು ತೆಗೆದುಕೊಳ್ಳಿ. ಇದು ಬ್ಯಾಬಿಲೋನ್​ನ ಮತ್ತು ರೋಮನ್ನರ ವಿಗ್ರಹರಾಧನೆಯ ರೀತಿಯದಲ್ಲ. ಇದು ಹಿಂದೂಗಳಿಗೆ ವಿಶಿಷ್ಟವಾದುದು. ಭಕ್ತನು ವಿಗ್ರಹದ ಎದುರು ಇರುವಾಗ ಕಣ್ಣು ಮುಚ್ಚಿಕೊಂಡು ಹೀಗೆ ಆಲೋಚಿಸುವನು: “ನಾನೇ ಅವನು. ನನಗೆ ಜನನ ಮರಣಗಳಿಲ್ಲ. ತಂದೆತಾಯಿಗಳಿಲ್ಲ, ನಾನು ಕಾಲದೇಶಗಳಿಂದ ಬದ್ಧನಲ್ಲ. ನಾನೇ ಸಚ್ಚಿದಾನಂದ, ಶಿವೋಠ್ಹಂ; ನಾನು ಶಾಸ್ತ್ರಗಳಿಂದ ಬದ್ಧನಲ್ಲ; ತೀರ್ಥಕ್ಕೆ ಬದ್ಧನಲ್ಲ; ಯಾತ್ರೆಗೆ ಬದ್ಧನಲ್ಲ. ನಾನೇ ಅವನು. ನಾನೇ ಅವನು.” ಇವುಗಳ ನ್ನೆಲ್ಲಾ ಹೇಳಿಯಾದ ಮೇಲೆ “ಹೇ ದೇವ, ನಾನು ನಿನ್ನ ಯಥಾರ್ಥಸ್ಥಿತಿಯನ್ನುಕಲ್ಪಿಸಿಕೊಳ್ಳಲಾರೆ. ನಾನು ದುರ್ಬಲ” ಎನ್ನುವನು. ಧರ್ಮವು ಪಾಂಡಿತ್ಯದ ಮೇಲೆ ನಿಂತಿಲ್ಲ. ಅದು ಆತ್ಮಕ್ಕೆ ಮತ್ತು ದೇವರಿಗೆ ಸಂಬಂಧಪಟ್ಟ ವಿಷಯ. ಶಕ್ತಿಯುತ ವಾದ ಮಾತಿನಿಂದ ಅಥವಾ ಪಾಂಡಿತ್ಯದಿಂದ ಪಡೆಯುವಂತಹದಲ್ಲ. ನೀವು ಮಹಾ ವಿದ್ವಾಂಸರನ್ನು ಕುರಿತು ದೇವರನ್ನು ಅವನ ಅಧ್ಯಾತ್ಮ ಸ್ವರೂಪದಲ್ಲಿ ಭಾವಿಸಿ ಕೊಳ್ಳಿ ಎನ್ನಿ. ಅವರಿಗೆ ಅದು ಸಾಧ್ಯವಿಲ್ಲ. ನೀವು ಅಧ್ಯಾತ್ಮವನ್ನು ಕಲ್ಪಿಸಿಕೊಳ್ಳ ಬಹುದು, ಅವರೂ ಅಧ್ಯಾತ್ಮವನ್ನು ಕಲ್ಪಿಸಿಕೊಳ್ಳಬಹುದು. ಆದರೆ ಅಭ್ಯಾಸವಿಲ್ಲದೆ ಅದನ್ನು ಭಾವಿಸಲಾಗುವುದಿಲ್ಲ. ಆದಕಾರಣ ನೀವು ಎಷ್ಟೇ ಶಾಸ್ತ್ರಗಳನ್ನು ಓದಿರಬಹುದು, ನೀವು ದೊಡ್ಡ ವೇದಾಂತಿಗಳಾಗಿರಬಹುದು, ದೊಡ್ಡ ವಿದ್ವಾಂಸರಾಗಿರಬಹುದು; ಆದರೆ ಹಿಂದೂ ಹುಡುಗನು ಇದಕ್ಕೂ ಆಧ್ಯಾತ್ಮಿಕ ಜೀವನಕ್ಕೂ ಏನೂ ಸಂಬಂಧವಿಲ್ಲ ಎನ್ನುವನು. ನೀವು ಅಧ್ಯಾತ್ಮವನ್ನು ಅಧ್ಯಾತ್ಮ ದಂತೆ ಆಲೋಚಿಸಬಲ್ಲಿರಾ? ಆಗ ಮಾತ್ರ ಸಂಶಯಗಳೆಲ್ಲ ಪರಿಹಾರವಾಗುವುವು ಹೃದಯದ ವಕ್ರತೆಗಳೆಲ್ಲ ನೇರವಾಗುವುವು. ಮಾನವ ಭಗವಂತನನ್ನು ಪ್ರತ್ಯಕ್ಷ ನೋಡಿದಾಗ ಮಾತ್ರವೇ ಅವನ ಅಂಜಿಕೆಗಳು ಮತ್ತು ಸಂದೇಹಗಳು ಪರಿಹಾರವಾಗುವುವು.

ಒಬ್ಬನು ಪಾಶ್ಚಾತ್ಯರ ದೃಷ್ಟಿಯಲ್ಲಿ ಘನವಿದ್ವಾಂಸನಾಗಿರಬಹುದು, ಆದರೂ ಅವನಿಗೆ ಆಧ್ಯಾತ್ಮಿಕ ವಿಷಯಗಳ ಪರಿಚಯವೇ ಇಲ್ಲದಿರಬಹುದು. ನಾನು ಅವನೊಡನೆ ಇದನ್ನು ಕೇಳುತ್ತೇನೆ: ನೀನು ಆತ್ಮನನ್ನು ಆತ್ಮದಂತೆ ಆಲೋಚಿಸ ಬಲ್ಲೆಯಾ? ಅಧ್ಯಾತ್ಮವಿದ್ಯೆಯಲ್ಲಿ ನೀನು ಮುಂದುವರಿದಿರುವೆಯಾ? ನೀನು ದೇಹಾತೀತವಾದ ಆತ್ಮನನ್ನು ವ್ಯಕ್ತಗೊಳಿಸಿರುವೆಯಾ? ಇಲ್ಲ ಎಂದರೆ ಧರ್ಮ ನಿನಗೆ ಇನ್ನೂ ತಿಳಿದಿಲ್ಲ; ಅದು ಬರಿಯ ಮಾತು; ಪುಸ್ತಕವಿದ್ಯೆ ಮತ್ತು ಅಹಂಕಾರ ವಲ್ಲದೆ ಬೇರೆಯಲ್ಲ ಎನ್ನುತ್ತೇನೆ. ಆದರೆ ಈ ಹಿಂದುವು ವಿಗ್ರಹದ ಎದುರು ಕುಳಿತುಕೊಂಡು ತಾನೇ ಅವನು ಎಂದು ಚಿಂತಿಸಲು ಯತ್ನಿಸಿ “ದೇವರೇ, ನಾನು ನಿನ್ನ ನೈಜಸ್ಥಿತಿಯನ್ನು ಕುರಿತು ಆಲೋಚಿಸಲಾರೆ. ಆದಕಾರಣ ನಿನ್ನನ್ನು ಈ ವಿಗ್ರಹದ ಮೂಲಕ ನೋಡುತ್ತೇನೆ” ಎನ್ನುವನು. ಅನಂತರ ಅವನು ಕಣ್ಣುಗಳನ್ನು ತೆರೆದು ವಿಗ್ರಹಕ್ಕೆ ನಮಸ್ಕಾರ ಮಾಡಿ ಪ್ರಾರ್ಥಿಸುವನು. ಇದೇ ಅವನ ಪ್ರಾರ್ಥನೆ, “ಹೇ ಭಗವಾನ್​, ನಾನು ನಿನ್ನನ್ನು ಹೀಗೆ ಅಪೂರ್ಣವಾದ ರೀತಿಯಲ್ಲಿ ಪೂಜಿಸು ವುದಕ್ಕೆ ದಯವಿಟ್ಟು ಕ್ಷಮಿಸು.”

ಹಿಂದೂಗಳು ಕಲ್ಲನ್ನು, ಮಣ್ಣನ್ನು ಪೂಜೆ ಮಾಡುವರು ಎಂದು ಯಾವಾಗಲೂ ನಿಮಗೆ ಹೇಳುತ್ತಿರುವರು. ಹಿಂದೂಗಳ ಭಕ್ತಿಭಾವವನ್ನು ನೀವು ಏನೆಂದು ಭಾವಿಸುವಿರಿ? ಪಾಶ್ಚಾತ್ಯದೇಶಗಳಿಗೆ ಬಂದ ಪ್ರಥಮ ಸಂನ್ಯಾಸಿ ನಾನು. ಜಗತ್ತಿನ ಇತಿಹಾಸದಲ್ಲಿ ಹಿಂದೂ ಸಂನ್ಯಾಸಿಯೊಬ್ಬನು ಸಮುದ್ರವನ್ನು ದಾಟಿದ್ದು ಇದೇ ಮೊದಲು. ನಾವು ನಿಮ್ಮ ಟೀಕೆಯನ್ನೆಲ್ಲಾ ಕೇಳುತ್ತಿರುವೆವು. ನಮ್ಮ ರಾಷ್ಟ್ರವು ಯಾವ ದೃಷ್ಟಿಯಿಂದ ನಿಮ್ಮನ್ನು ನೋಡುತ್ತದೆ ಗೊತ್ತೆ? ಅಲ್ಲಿ ಜನರು ಸುಮ್ಮನೆ ನಕ್ಕು; “ಅಯ್ಯೋ, ಅವರು ಬರಿಯ ಮಕ್ಕಳು. ವಿಜ್ಞಾನದಲ್ಲಿ ಅವರು ಪ್ರಖ್ಯಾತ ರಾಗಿರಬಹುದು, ಅವರು ದೊಡ್ಡ ದೊಡ್ಡ ಯಂತ್ರಗಳನ್ನು ರಚಿಸಿರಬಹುದು.ಆದರೆ ಧರ್ಮದಲ್ಲಿ ಅವರಿನ್ನೂ ಮಕ್ಕಳು.” ಇದೇ ನಮ್ಮ ದೇಶದವರ ಮನೋಭಾವ, ನಿಮ್ಮನ್ನು ಕುರಿತು.

ನಾನು ನಿಮಗೆ ಒಂದು ವಿಷಯವನ್ನು ಹೇಳುತ್ತೇನೆ. ನಾನು ನಿಮ್ಮನ್ನು ದೂರುತ್ತೇನೆ ಎಂದು ಭಾವಿಸಬೇಡಿ. ನೀವು ಮಿಷನರಿಗಳಿಗೆ ವಿದ್ಯೆ ಬಟ್ಟೆಬರೆ ಮುಂತಾದುವನ್ನು ಕೊಟ್ಟು ತರಬೇತು ಮಾಡಿ ನಮ್ಮ ದೇಶಕ್ಕೆ ಕಳುಹಿಸುವಿರಿ? ಏನನ್ನು ಮಾಡುವುದಕ್ಕೆ ಅವರನ್ನು ಕಳುಹಿಸುತ್ತೀರಿ? ನಮ್ಮ ದೇಶಕ್ಕೆ ಬಂದು ನಮ್ಮ ಪೂರ್ವಿಕರನ್ನೆಲ್ಲಾ ದೂರಿ, ನಮ್ಮ ಧರ್ಮ, ಆಚಾರ ವ್ಯವಹಾರಗಳನ್ನೆಲ್ಲಾ ಅವರು ಆಕ್ಷೇಪಿಸುವರು. ದೇವಸ್ಥಾನದ ಹತ್ತಿರ ಹೋಗಿ “ವಿಗ್ರಹಾರಾಧಕರೇ, ನೀವು ನರಕಕ್ಕೆ ಹೋಗುವಿರಿ” ಎನ್ನುವರು. ಆದರೆ ಭಾರತದ ಮಹಮ್ಮದೀಯರೊಡನೆ ಅವರು ಹೀಗೆ ವರ್ತಿಸಲಾರರು. ಅವರ ಕತ್ತಿ ತಕ್ಷಣ ಹೊರಗೆ ಬರುವುದು. ಆದರೆ ಹಿಂದೂ ಸಾಧು ಸ್ವಭಾವದವನು. ಅವನು ಸುಮ್ಮನೆ ನಗುತ್ತಾ ಮೂರ್ಖರು ಹರಟೆ ಹೊಡೆಯಲಿ ಎಂದು ಸುಮ್ಮನೆ ಹೋಗುವನು. ನೀವು ನಮ್ಮನ್ನು ದೂರುವುದಕ್ಕೆ ಅವರನ್ನು ತರಬೇತು ಮಾಡುವಿರಿ. ಆದರೆ ನಾನು ಒಳ್ಳೆಯ ದೃಷ್ಟಿಯಿಂದ ನಿಮ್ಮಲ್ಲಿರುವ ಏನಾದರೂ ಸ್ವಲ್ಪ ಲೋಪವನ್ನು ತೋರಿದರೂ ನೀವು ಕೂಗಾಡುವಿರಿ. “ನಮ್ಮನ್ನು ಮುಟ್ಟಬೇಡಿ, ನಾವು ಅಮೆರಿಕಾ ದೇಶದವರು, ಪ್ರಪಂಚದಲ್ಲಿ ಎಲ್ಲರನ್ನೂ ಟೀಕಿಸುತ್ತೇವೆ, ಎಲ್ಲರನ್ನೂ ನಿಂದಿಸುತ್ತೇವೆ, ನಮಗೆ ತೋರಿದುದನ್ನು ಹೇಳುತ್ತೇವೆ. ಆದರೆ ನೀವು ಮಾತ್ರ ನಮ್ಮ ವಿಚಾರ ಎತ್ತಬೇಡಿ. ನಮಗೆ ಸಹಿಸುವುದಕ್ಕೆ ಆಗುವುದಿಲ್ಲ” ಎನ್ನುವಿರಿ. ನಿಮ್ಮ ಮನಸ್ಸಿಗೆ ತೋರಿದುದನ್ನು ನೀವು ಮಾಡಬಹುದು. ಆದರೆ ನಾವು ಈಗ ಇರುವ ಸ್ಥಿತಿಯಲ್ಲೇ ತೃಪ್ತರಾಗಿರುವೆವು ಎಂದು ನಿಮಗೆ ಹೇಳುತ್ತೇನೆ. ಆದರೆ ಒಂದರಲ್ಲಿ ನಿಮಗಿಂತ ನಾವು ಮೇಲು ಅಂದರೆ ನಮ್ಮ ಮಕ್ಕಳಿಗೆ “ಎಲ್ಲ ದೃಶ್ಯಗಳೂ ಸುಂದರವಾಗಿವೆ, ಆದರೆ ಮಾನವನು ಮಾತ್ರ ಪಾಪಿ” ಎಂಬ ಘೋರಭಾವನೆಯನ್ನು ಕಲಿಸುವುದಿಲ್ಲ. ನಿಮ್ಮ ಪಾದ್ರಿಗಳು ನಮ್ಮನ್ನು ಟೀಕಿಸಿದಾಗ ಅವರು ಇದನ್ನು ನೆನಪಿನಲ್ಲಿಡಲಿ: ಇಂಡಿಯಾ ದೇಶದವರೆಲ್ಲ ಒಟ್ಟಿಗೆ ಕಲೆತು ಹಿಂದೂಮಹಾಸಾಗರದ ಅಡಿಯಲ್ಲಿ ಇರುವ ಕೆಸರನ್ನೆಲ್ಲಾ ತೆಗೆದು ಪಾಶ್ಚಾತ್ಯರ ಮುಖಕ್ಕೆ ಬಳಿದರೂ ನೀವು ನಮಗೆ ಏನು ಮಾಡುತ್ತಿರುವಿರೋ ಅದರಲ್ಲಿ ಸಾವಿರದ ಒಂದು ಪಾಲಿನಷ್ಟನ್ನಾದರೂ ನಾವು ನಿಮಗೆ ಮಾಡಿದಂತೆ ಆಗುವುದಿಲ್ಲ. ನೀವು ಈ ಟೀಕೆಯನ್ನೆಲ್ಲಾ ಏತಕ್ಕೆ ಮಾಡು ವುದು? ಜಗತ್ತಿನಲ್ಲಿ ಯಾರನ್ನಾದರೂ ಮತಾಂತರಗೊಳಿಸಲು ನಾವು ಒಬ್ಬ ಪ್ರಚಾರಕರನ್ನಾದರೂ ಕಳುಹಿಸಿರುವೆವೆ? ನಾವು, ನಿಮ್ಮ ಧರ್ಮಕ್ಕೆ ಸ್ವಾಗತ, ಆದರೆ ನಮ್ಮ ಧರ್ಮ ನಮಗಿರಲಿ ಎನ್ನುವೆವು. ನೀವು ನಿಮ್ಮ ಧರ್ಮವನ್ನು ಧರ್ಮ ಎನ್ನುವಿರಿ. ನೀವೇನೋ ಆಕ್ರಮಣಶೀಲರು. ಆದರೆ ಎಷ್ಟು ಜನರನ್ನು ನಿಮ್ಮ ಧರ್ಮಕ್ಕೆ ಸೇರಿಸಿ ಕೊಂಡಿರುವಿರಿ? ಪ್ರಪಂಚದ ಆರನೆಯ ಒಂದು ಪಾಲು ಚೀಣೀಯರು. ಅಲ್ಲಿರುವವರೆಲ್ಲ ಬೌದ್ಧರು, ಅನಂತರ ಜಪಾನು, ಟಿಬೆಟ್​, ರಷ್ಯ, ಸೈಬೀರಿಯಾ, ಬರ್ಮ, ಸಯಾಂ ಬೇರೆ ಇವೆ. ಕ್ರೈಸ್ತನೀತಿ, ಕ್ಯಾಥೋಲಿಕ್​ ಚರ್ಚು ಇವು ಎಲ್ಲಾ ಅವುಗಳಿಂದ ಬಂದವು ಎಂದು ಹೇಳಿದರೆ ನಿಮಗೆ ಅಷ್ಟು ರುಚಿಸದೆ ಇರಬಹುದು. ಮೇಲಿನ ದೇಶಗಳಲ್ಲೆಲ್ಲಾ ಜನ ಹೇಗೆ ಬೌದ್ಧರಾದರು, ಅದೂ, ಒಂದು ತೊಟ್ಟು ರಕ್ತವನ್ನೂ ಚೆಲ್ಲದೆ? ನೀವು ಎಷ್ಟೇ ಜಂಭ ಕೊಚ್ಚಿಕೊಂಡರೂ ಕ್ರಿಸ್ತನ ಧರ್ಮ ಕತ್ತಿಯಿಲ್ಲದೆ ಎಲ್ಲಿ ಹಬ್ಬಿದೆ? ಜಗತ್ತಿನಲ್ಲಿ ಒಂದು ಸ್ಥಳವನ್ನಾದರೂ ತೋರಿಸಿ. ಕ್ರೈಸ್ತಧರ್ಮದ ಇತಿಹಾಸದಲ್ಲಿ ಒಂದು ಉದಾಹರಣೆಯನ್ನಾದರೂ ಕೊಡಿ ನನಗೆ ಎರಡು ಬೇಕಾಗಿಲ್ಲ. ನಿಮ್ಮ ಮುತ್ತಾತಂದಿರು ಹೇಗೆ ಕ್ರೈಸ್ತ ರಾದರೆಂಬುದು ನನಗೆ ಗೊತ್ತಿದೆ. ಅವರು ಕ್ರೈಸ್ತ ಧರ್ಮವನ್ನು ಸ್ವೀಕರಿಸಬೇಕಾಗಿತ್ತು ಇಲ್ಲವೆ ಕತ್ತಿಯ ಬಾಯಿಗೆ ಬೀಳಬೇಕಾಗಿತ್ತು. ಅಷ್ಟೆ. ನೀವು ಎಷ್ಟೇ ಜಂಭ ಕೊಚ್ಚಿಕೊಂಡರೂ ಮಹಮ್ಮದೀಯರಿಗಿಂತ ನೀವು ಹೇಗೆ ಮೇಲು? “ನಾವು ಮಾತ್ರವೇ ಭಗವಂತನ ಪ್ರೀತಿಗೆ ಪಾತ್ರರು.” ಏತಕ್ಕೆ ಎಂದರೆ ನಾವು ಮತ್ತೊಬ್ಬರನ್ನು ಕೊಲ್ಲಬಹುದು ಅದಕ್ಕೆ. ಅರಬ್ಬರೂ ಹಾಗೆಯೇ ಹೇಳಿದರು. ಅವರು ಜಂಭಕೊಚ್ಚಿಕೊಳ್ಳುತ್ತಿದ್ದರು. ಈಗ ಅರಬ್ಬನೆಲ್ಲಿ? ಈಗ ಅವನೊಬ್ಬ ಮರುಳುಕಾಡಿನ ದರೋಡೆಗಾರ. ರೋಮನ್ನರೂ ಹಾಗೇ ಹೇಳುತ್ತಿದ್ದರು. ಈಗ ಅವರು ಎಲ್ಲಿ? ಶಾಂತಿದೂತರೇ ಧನ್ಯರು; ಅವರೇ ಜಗತ್ತನ್ನು ಅನುಭವಿಸುವರು. ಮರಳಿನ ಮೇಲೆ ಕಟ್ಟಿರುವುದೆಲ್ಲ ನಾಶವಾಗುವುದು. ಬಹುಕಾಲ ನಿಲ್ಲುವುದಿಲ್ಲ.

ಯಾವುದು ಸ್ವಾರ್ಥದ ಮೇಲೆ ನಿಂತಿದೆಯೋ, ಯಾವುದಕ್ಕೆ ಸ್ಪರ್ಧೆಯೇ ಮೂಲಮಂತ್ರವಾಗಿದೆಯೋ, ಭೋಗವೇ ಪರಮಗುರಿಯಾಗಿರುವುದೋ ಅದೆಲ್ಲ ಇಂದಲ್ಲ ನಾಳೆ ನಾಶವಾಗಲೇಬೇಕು. ನನ್ನ ಸಹೋದರರೇ, ನಿಮಗೆ ಹೇಳುತ್ತೇನೆ ಕೇಳಿ. ನಿಮಗೆ ನಿಜವಾಗಿ ಬದುಕಬೇಕೆಂದು ಇಚ್ಛೆ ಇದ್ದರೆ, ನಿಮ್ಮ ಜನಾಂಗ ಬದುಕಿರಬೇಕೆಂದು ಇಚ್ಛೆ ಇದ್ದರೆ, ಕ್ರಿಸ್ತನ ಬಳಿಗೆ ಹೋಗಿ. ನೀವು ಇನ್ನೂ ನಿಜವಾದ ಕ್ರೈಸ್ತರಾಗಿಲ್ಲ. ಇಲ್ಲ, ಇಡೀ ರಾಷ್ಟ್ರ ಇನ್ನೂ ಕ್ರೈಸ್ತರಾಗಿಲ್ಲ. ಕ್ರಿಸ್ತನ ಬಳಿಗೆ ಹೋಗಿ, ಯಾರಿಗೆ ತಂಗುವುದಕ್ಕೆ ಎಲ್ಲಿಯೂ ಸ್ಥಳವಿರಲಿಲ್ಲವೋ ಅವನ ಬಳಿಗೆ ಹೋಗಿ. “ಹಕ್ಕಿಗಳಿಗೆ ಗೂಡುಗಳಿವೆ, ಪ್ರಾಣಿಗಳಿಗೆ ಪೊದೆಗಳಿವೆ; ಆದರೆ ಮಾನವ ಪುತ್ರನಿಗೆ ತಂಗುವುದಕ್ಕೆ ಸ್ಥಳವಿಲ್ಲ.” ಭೋಗದ ಹೆಸರಿನಲ್ಲಿ ಪ್ರಚಾರಮಾಡಿದ ಧರ್ಮ ನಿಮ್ಮದು. ಎಂತಹ ವಿಧಿವಿಲಾಸ ಇದು! ನೀವು ನಿಜವಾಗಿ ಬಾಳಬೇಕಾದರೆ ಇದನ್ನು ತಲೆಕೆಳಗು ಮಾಡಬೇಕು. ನಾನು ಈ ದೇಶದಲ್ಲಿ ಕೇಳಿರುವುದೆಲ್ಲ ಕಪಟ. ನಿಮ್ಮ ದೇಶ ಇನ್ನೂ ಬದುಕಬೇಕಾದರೆ ಅವನ ಬಳಿಗೆ ಹೋಗಿ. ಏಕಕಾಲದಲ್ಲಿ ದೇವರನ್ನೂ ಮತ್ತು ಧನಪಿಶಾಚಿಯನ್ನೂ ಆರಾಧಿಸಲಾರಿರಿ. ಏನು, ಈಗಿರುವ ವಿಲಾಸವೆಲ್ಲ ಕ್ರಿಸ್ತನಿಂದ ಬಂದುದೆ? ಕ್ರಿಸ್ತ ಬದುಕಿದ್ದರೆ ಇದನ್ನೆಲ್ಲಾ ಈಶ್ವರನಿಂದೆ ಎಂದು ತಿರಸ್ಕರಿಸುತ್ತಿದ್ದನು. ಐಶ್ವರ್ಯೋಪಾಸನೆಯ ಮೂಲಕ ಬರುವ ಜಯವೆಲ್ಲ ಕ್ಷಣಿಕ. ನಿತ್ಯವಾದುದು ಅವನಲ್ಲಿ ಮಾತ್ರ ಇರುವುದು. ನಿಮಗೆ ಈ ಅದ್ಭುತವಾದ ಪ್ರಾಪಂಚಿಕ ವೈಭವವನ್ನು ಕ್ರಿಸ್ತನ ಆದರ್ಶದೊಂದಿಗೆ ಹೊಂದಿಸಲು ಸಾಧ್ಯವಾಗುವುದಾದರೆ ಒಳ್ಳೆಯದು. ಅದು ಸಾಧ್ಯವಿಲ್ಲದೇ ಇದ್ದರೆ ಇದನ್ನು ತ್ಯಜಿಸಿ ಕ್ರಿಸ್ತನ ಬಳಿಗೆ ಹೋಗಿ, ಚಿಂದಿಯನ್ನು ಧರಿಸಿ ಕ್ರಿಸ್ತನೊಂದಿಗೆ ಇರುವುದು ಅವನಿಲ್ಲದ ಅರಮನೆಯಲ್ಲಿ ಬಾಳುವುದಕ್ಕಿಂತ ಮೇಲು.

