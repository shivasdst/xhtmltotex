
\chapter[ಪ್ರೇಮಯೋಗ]{ಪ್ರೇಮಯೋಗ \protect\footnote{\engfoot{C.W. Vol. VIII, P. 220}}}

\centerline{\textbf{(1895ರ ನವೆಂಬರ್​ 16ರಂದು ಲಂಡನ್ನಿನಲ್ಲಿ ನೀಡಿದ ಉಪನ್ಯಾಸದ ಟಿಪ್ಪಣಿ)}}

\vskip 0.1cm

ಆತ್ಮಸಾಕ್ಷಾತ್ಕಾರವನ್ನು ಪಡೆಯುವುದಕ್ಕೆ ಮುಂಚೆ ವ್ಯಕ್ತಿಯು ಪ್ರತೀಕಗಳು ಮತ್ತು ಕ್ರಿಯಾ\break ವಿಧಿಗಳು ಇವುಗಳ ಮೂಲಕ ಸಾಗಿಹೋಗಬೇಕು. ಆದುದರಿಂದಲೇ ನಾವು ಇಂಡಿಯಾ\break ದೇಶದಲ್ಲಿ ಹೇಳುತ್ತೇವೆ, ಒಂದು ಚರ್ಚಿನಲ್ಲಿ ಹುಟ್ಟುವುದು ಒಳ್ಳೆಯದು, ಆದರೆ ಅಲ್ಲೇ ಸಾಯಬಾರದು ಎಂದು. ಒಂದು ಸಣ್ಣ ಸಸಿಯನ್ನು ಸುತ್ತಲೂ ಬೇಲಿ ಹಾಕಿ ರಕ್ಷಿಸಬೇಕು. ಅದು ಬೆಳೆದು ದೊಡ್ಡ ಮರವಾದ ಮೇಲೆ ಬೇಲಿಯು ಅದಕ್ಕೆ ಅಡಚಣೆಯಾಗುತ್ತದೆ.\break ಹಿಂದಿನ ಧರ್ಮಗಳನ್ನು, ಆಚಾರಗಳನ್ನು ದೂರಬೇಕಾಗಿಲ್ಲ. ಧರ್ಮದ ವಿಕಾಸವು ಅಗತ್ಯ ಎಂಬುದನ್ನು ನಾವು ಮರೆಯುತ್ತೇವೆ.

ಮೊದಲು ನಾವು ಸಗುಣ ದೇವರನ್ನು ಕುರಿತು ಯೋಚಿಸುತ್ತೇವೆ. ಅವನನ್ನು ಸೃಷ್ಟಿಕರ್ತ,\break ಸರ್ವಶಕ್ತ, ಸರ್ವವ್ಯಾಪಿ ಎನ್ನುತ್ತೇವೆ. ಆದರೆ ಪ್ರೇಮೋದಯವಾದ ಮೇಲೆ ದೇವರು ಪ್ರೇಮಮಯನಾಗುವನು. ಭಕ್ತನಿಗೆ ದೇವರು ಏನೇನು ಆಗಿರುವನೋ ಅದಾವುದೂ ಬೇಕಾಗಿಲ್ಲ. ಏಕೆಂದರೆ ಅವನಿಂದ ಭಕ್ತನು ಏನನ್ನೂ ಇಚ್ಛಿಸುವುದಿಲ್ಲ. ಭಾರತದ ಸಂತನೊಬ್ಬನು ‘ದೇವರ ಹತ್ತಿರ ಬೇಡುವುದಕ್ಕೆ ನಾನೇನು ಭಿಕ್ಷುಕನಲ್ಲ’ ಎನ್ನುವನು. ಅವನು ದೇವರನ್ನು ಕಂಡರೆ ಅಂಜುವುದೂ ಇಲ್ಲ. ದೇವರನ್ನು ಮಾನವನಂತೆ ಪ್ರೀತಿಸುವನು.

ಪ್ರೇಮವನ್ನು ಅವಲಂಬಿಸಿದ ಕೆಲವು ಮಾರ್ಗಗಳಿವೆ: 1) ಶಾಂತ, ದೇವರನ್ನು ತಂದೆ, ರಕ್ಷಕ ಎಂದು ಕರೆಯುವುದು 2) ದಾಸ್ಯ. ಇಲ್ಲಿ ಸೇವೆಯೇ ಪರಮಾದರ್ಶ. ದೇವರನ್ನು ಇಲ್ಲಿ ಒಡೆಯ, ಸೇನಾನಾಯಕ, ರಾಜ ಮುಂತಾಗಿ ಭಾವಿಸುವುದು. ಅವನು ಶಿಕ್ಷೆ, ರಕ್ಷೆಗಳನ್ನು ನೀಡುತ್ತಾನೆ ಎಂದು ಭಾವಿಸುವುದು. 3) ವಾತ್ಸಲ್ಯ. ದೇವರನ್ನು ತಾಯಿ ಅಥವಾ ಮಗು ಎಂದು ಭಾವಿಸುವುದು. ಭಾರತದಲ್ಲಿ ತಾಯಿಯು ಎಂದೂ ಮಗುವನ್ನು ಶಿಕ್ಷಿಸುವುದಿಲ್ಲ. ಈ ಎಲ್ಲ ಹಂತಗಳಲ್ಲಿಯೂ ಭಕ್ತನು ದೇವರನ್ನು ಒಂದು ಆದರ್ಶ ಎಂದು ಭಾವಿಸಿ ಪೂಜಿಸುತ್ತಾನೆ. 4) ಸಖ್ಯ. ದೇವರನ್ನು ಸ್ನೇಹಿತನಂತೆ ಕಾಣುವುದು. ಅಲ್ಲಿ ಅಂಜಿಕೆ ಇಲ್ಲ, ಸಲಿಗೆ ಇದೆ, ಸರಿಸಮಾನತೆ ಇದೆ. 5) ಕೊನೆಯದೇ ಮಧುರ, ಸತಿಪತಿಯರೊಳಗೆ ಇರುವ ಪ್ರೇಮ. ಸೈಂಟ್​ಥೆರೀಸ ಮುಂತಾದವರು ಇದಕ್ಕೆ ಉದಾಹರಣೆ. ಪರ್ಷಿಯಾದ ಜನರು ದೇವರನ್ನು ಸತಿಯಂತೆ ನೋಡುವರು. ಹಿಂದೂಗಳು ದೇವರನ್ನು ಪತಿಯಂತೆ ನೋಡುವರು. ಭಗವಂತನೊಬ್ಬನೇ ತನ್ನ ಪತಿ ಎಂದ ಭಕ್ತೆ ಮೀರಾಬಾಯಿಯನ್ನು ನೀವು ಜ್ಞಾಪಿಸಿಕೊಳ್ಳಬಹುದು. ಕೆಲವರು ಮತ್ತೂ ಅತಿರೇಕಕ್ಕೆ ಹೋಗುವರು. ದೇವರನ್ನು ಸರ್ವಶಕ್ತ, ತಂದೆ ಎನ್ನುವುದು ಕೂಡ ಈಶ್ವರನಿಂದೆ ಎನ್ನುವರು. ಈ ವಿಧದ ಉಪಾಸನೆಯಲ್ಲಿ ಭಕ್ತರು ಕಾಮಸಂಬಂಧವಾದ ಭಾಷೆಯನ್ನು ಬಳಸುವರು, ನ್ಯಾಯವಿರುದ್ಧವಾದ ಪ್ರೇಮಭಾಷೆಯನ್ನೂ ಬಳಸುವರು. ಕೃಷ್ಣ ಮತ್ತು ಗೋಪಿಯರ ಪ್ರೇಮ ಈ ಗುಂಪಿಗೆ ಸೇರುವುದು. ಇದರಿಂದೆಲ್ಲಾ ಭಕ್ತನು ತುಂಬ ಅಧೋಗತಿಗೆ ಬಂದಂತೆ ನಿಮಗೆ ಕಾಣಿಸಬಹುದು. ಕೆಲವರು ಇದರಿಂದ ಅಧೋಗತಿಗೂ ಬಂದಿರುವರು. ಆದರೂ ಈ ಮಾರ್ಗದಿಂದಲೂ ಅನೇಕರು ಮಹಾನ್​ ಸಂತರಾಗಿರುವರು. ದೋಷವಿಲ್ಲದ ಯಾವ ಮಾನವ ವ್ಯವಸ್ಥೆಯೂ ಇಲ್ಲ. ಅನೇಕ ಭಿಕ್ಷುಕರಿರುವರು ಎಂದು ನಾವು ಅಡಿಗೆ ಮಾಡದೆಯೇ ಇರುವುದಕ್ಕೆ ಆಗುವುದೇ? ಕಳ್ಳರು ಇರುವರು ಎಂದು ನೀವು ಏನನ್ನೂ ಇಟ್ಟುಕೊಳ್ಳದೇ ಇರುವುದಕ್ಕೆ ಆಗುವುದೇ? “ಪ್ರಿಯತಮ, ನಿನ್ನ ಚೆಂದುಟಿಯ ಚುಂಬನವೊಂದು ಸಾಕು, ಒಮ್ಮೆ ಅದರ ಸವಿಯನ್ನು ಅರಿತರೆ ಅದು ನನ್ನನ್ನು ಜೀವಾವಧಿ ಉನ್ಮತ್ತಳ ಹಾಗೆ ಮಾಡುವುದು.”

ಈ ಭಾವದಲ್ಲಿ ಪರಿಣತನಾದ ಮೇಲೆ ಭಕ್ತನಿಗೆ ಯಾವ ಪಂಥಕ್ಕೂ ಸೇರಲಾಗುವುದಿಲ್ಲ, ಯಾವ ಬಾಹ್ಯ ಆಚಾರಗಳನ್ನೂ ಸಹಿಸುವುದಕ್ಕೆ ಆಗುವುದಿಲ್ಲ. ಭರತಖಂಡದಲ್ಲಿ ಧರ್ಮವು ಮುಕ್ತಿಯಲ್ಲಿ ಪರ್ಯವಸಾನವಾಗುವುದು. ಈ ಮುಕ್ತಿಯನ್ನು ಕೂಡ ಕೊನೆಗೆ ತ್ಯಜಿಸುವರು; ಕೊನೆಗೆ ಉಳಿಯುವುದೊಂದೇ, ಅದೇ ಪ್ರೇಮ. ಎಲ್ಲಾ ಪ್ರೇಮಕ್ಕಾಗಿ.

ಪ್ರೇಮದ ಕೊನೆಯ ಘಟ್ಟದಲ್ಲಿ ಪ್ರಿಯತಮ–ಪ್ರೇಯಸಿಯರಲ್ಲಿ ಯಾವ ವ್ಯತ್ಯಾಸವೂ ಇರುವುದಿಲ್ಲ; ಉಳಿಯುವುದು ಆತ್ಮವೊಂದೇ. ಒಂದು ಹಳೆಯ ಪರ್ಷಿಯನ್​ ಕವನವಿದೆ. ಪ್ರಿಯತಮನೊಬ್ಬ ತನ್ನ ಪ್ರೇಯಸಿಯ ಕೋಣೆಯ ಕದವನ್ನು ತಟ್ಟಿದ. “ನೀನಾರು?” ಎಂದು ಅವಳು ಕೇಳಿದಳು. ಆತ “ಪ್ರಿಯತಮೆ, ನಾನು ಇಂತಹವನು” ಎಂದಾಗ, “ಹೋಗು, ಅಂತಹವರು ಯಾರೂ ನನಗೆ ಗೊತ್ತಿಲ್ಲ” ಎಂದಳು. ಆತ ನಾಲ್ಕನೆಯ ಸಲ ಬಂದಾಗ ಅವಳು ಪುನಃ “ನೀನಾರು?” ಎಂದು ಕೇಳಿದಾಗ “ನಾನೇ ನೀನು: ಪ್ರಿಯತಮೆ, ಬಾಗಿಲು ತೆಗೆ” ಎಂದಾಗ ಬಾಗಿಲು ತೆರೆಯಿತು.

ಒಬ್ಬ ಮಹಾಭಕ್ತನು ತರುಣಿಯೊಬ್ಬಳ ಭಾಷೆಯಲ್ಲಿ ಆ ಪ್ರೇಮವೇನೆಂಬುದನ್ನು ವಿವರಿಸಿರುವನು. “ನಾಲ್ಕು ಕಣ್ಣುಗಳು ಸಂಧಿಸಿದುವು, ಎರಡು ಆತ್ಮಗಳಲ್ಲಿ ಬದಲಾವಣೆಗಳು ಆದುವು. ಈಗ ನಾನು, ಅವನು ಗಂಡು, ನಾನು ಹೆಂಗಸು ಎಂತಲೂ ಹೇಳಲಾರೆ, ನಾನು ಗಂಡಸು ಅವನು ಹೆಂಗಸು ಎಂತಲೂ ಹೇಳಲಾರೆ. ಈಗ ಇಷ್ಟು ಮಾತ್ರ ನೆನಪಿದೆ: ಎರಡು ಆತ್ಮಗಳಿದ್ದುವು. ಪ್ರೇಮ ಬಂತು. ಈಗಿರುವುದೊಂದೇ ಆಗಿದೆ.”

ಪರಾಪ್ರೇಮದಲ್ಲಿ ಆತ್ಮದ ಐಕ್ಯತೆ ಮಾತ್ರ ಇರುವುದು. ಬೇರೆ ಎಲ್ಲ ವಿಧದ ಪ್ರೇಮಗಳೂ ನಶ್ವರ. ಅಧ್ಯಾತ್ಮಪ್ರೇಮ ಮಾತ್ರ ಉಳಿಯುವುದು. ಇದೊಂದೇ ಬೆಳೆಯುತ್ತಾ ಬರುವುದು.

ಪ್ರೇಮವು ಆದರ್ಶವನ್ನು ನೋಡುವುದು. ಇದೇ ಪ್ರೇಮ–ತ್ರಿಭುಜದ ಮೂರನೆಯ ಕೋನ. ದೇವರೇ ಕಾರಣ, ಸೃಷ್ಟಿಕರ್ತ, ತಂದೆ. ಪ್ರೇಮವೇ ಪರಾಕಾಷ್ಠೆ. ಮಗುವಿಗೆ ಗೂನು ಬೆನ್ನಿದೆ ಎಂದು ತಾಯಿ ಮೊದಲು ವ್ಯಥೆಪಡುವಳು. ಆದರೆ ಕೆಲವು ದಿನ ಅದನ್ನು ಸಾಕಿದ ಮೇಲೆ ಆ ಮಗುವಿನ ಸಮಾನ ಯಾರೂ ಇಲ್ಲ ಎಂದು ಭಾವಿಸುವಳು. ಪ್ರೇಮಿಯು ಎಂತಹ ಕುರೂಪಿಯಲ್ಲಿಯೂ ರತಿಯನ್ನೇ ಕಾಣುತ್ತಾನೆ. ಸಾಧಾರಣವಾಗಿ ನಮಗೆ ಏನಾಗುವುದು ಎಂಬುದು ಗೊತ್ತಾಗುವುದಿಲ್ಲ. ಕುರೂಪವು ಕೇವಲ ಒಂದು ಸೂಚನೆ ಮಾತ್ರ. ಪ್ರೇಮಿಯು ಅಲ್ಲಿ ಒಬ್ಬಳು ಮಹಾರತಿಯನ್ನು ನೋಡುವನು. ಇವನು ತನ್ನ ಆದರ್ಶವನ್ನು ಆ ಸೂಚನೆಯ ಮೇಲೆ ಆರೋಪಿಸಿ ತನ್ನ ಮನಸ್ಸಿನಲ್ಲಿರುವುದನ್ನು ಹೊರಗೆ ನೋಡುವನು; ಮುತ್ತಿನ ಹುಳು ಒಂದು ಮರಳು ಕಣವನ್ನು ಮುತ್ತನ್ನಾಗಿ ಮಾಡುವಂತೆ. ಮಾನವನು ದೇವರೆಂಬ ಆದರ್ಶದ ಮೂಲಕ ಎಲ್ಲರನ್ನೂ ನೋಡುತ್ತಾನೆ.

ಕೊನೆಗೆ ನಾವು ಪ್ರೀತಿಯನ್ನೇ ಪ್ರೀತಿಸುವ ಸ್ಥಿತಿಗೆ ಬರುತ್ತೇವೆ. ಈ ಪ್ರೀತಿ ಅನಿರ್ವಚನೀಯ ಸ್ವರೂಪ. ಇದರ ವಿಷಯದಲ್ಲಿ ನಾವು ಮೂಕರು. ಇದನ್ನು ವಿವರಿಸುವುದಕ್ಕೆ\break ಆಗುವುದಿಲ್ಲ.

ಪ್ರೀತಿಯಲ್ಲಿ ನಮ್ಮ ಇಂದ್ರಿಯಗಳೆಲ್ಲ ತೀಕ್ಷ್ಣವಾಗುವುವು. ಮಾನವ ಪ್ರೇಮದಲ್ಲಿ ಹಲವು ಗುಣಗಳು ಸೇರಿವೆ ಎಂಬುದನ್ನು ನೆನಪಿನಲ್ಲಿಡಬೇಕು. ಇದು ನಾವು ಯಾರನ್ನು ಪ್ರೀತಿಸುವೆವೋ ಅವರ ಮನೋಭಾವದ ಮೇಲೂ ನಿಂತಿರುವುದು. ಈ ಪ್ರೇಮದ ಅನ್ಯೋನ್ಯ ಆಶ್ರಯವನ್ನು ವಿವರಿಸುವುದಕ್ಕೆ ಭಾರತೀಯ ಭಾಷೆಗಳಲ್ಲಿ ಅನೇಕ ಶಬ್ದಗಳಿವೆ. ಅತಿ ಕೆಳಗಿನ\break ಪ್ರೇಮ ಸ್ವಾರ್ಥದಿಂದ ಕೂಡಿದೆ. ಇಲ್ಲಿ ಸಂತೋಷವು ಇನ್ನೊಬ್ಬರು ನನ್ನನ್ನು ಪ್ರೀತಿಸುವರು\break ಎಂಬುದನ್ನು ಅವಲಂಬಿಸಿದೆ. “ಒಬ್ಬಳು ತನ್ನ ಕೆನ್ನೆಯನ್ನು ಕೊಡುತ್ತಾಳೆ. ಮತ್ತೊಬ್ಬನು ಮುತ್ತುಗಳನ್ನು ಕೊಡುವನು” ಇದಕ್ಕಿಂತ ಮೇಲೆ ಇರುವುದೇ ಅನ್ಯೋನ್ಯ ಪ್ರೀತಿ. ಇದೂ ಕೂಡ ನಿಂತುಹೋಗುವುದು. ನಿಜವಾದ ಪ್ರೀತಿ ಯಾವಾಗಲೂ ಕೊಡುವಂಥದು. ನಾವು ಪ್ರೀತಿಸುವವನನ್ನು ನೋಡುವುದಕ್ಕೂ ಇಚ್ಛಿಸುವುದಿಲ್ಲ. ಪ್ರೇಮದ ಭಾವನೆಯನ್ನು ವ್ಯಕ್ತ\-ಪಡಿಸುವುದೂ ಇಲ್ಲ. ನೀಡುವುದರಲ್ಲಿಯೇ ತೃಪ್ತಿ. ಮತ್ತೊಬ್ಬ ಮಾನವನನ್ನು ಈ ರೀತಿ ಪ್ರೀತಿಸುವುದಕ್ಕೆ ಸಾಧ್ಯವೇ ಇಲ್ಲ ಎನ್ನಬಹುದು. ಆದರೆ ದೇವರನ್ನು ಮಾತ್ರ ಈ ರೀತಿ ಪ್ರೀತಿಸಬಹುದು.

ಹುಡುಗರು ಬೀದಿಯಲ್ಲಿ ಜಗಳ ಕಾಯುವಾಗ ದೇವರ ಆಣೆಗೂ ಎಂದರೆ ಇಂಡಿಯಾ ದೇಶದಲ್ಲಿ ಅದನ್ನು ಪಾಪ ಎಂದು ಭಾವಿಸುವುದಿಲ್ಲ. ಬೆಂಕಿಯಲ್ಲಿ ನೀವು ಕೈಯಿಟ್ಟರೆ ನಿಮಗೆ ನೋವು ಆಗಲಿ, ಬಿಡಲಿ, ಕೈ ಅಂತೂ ಸುಡುವುದು. ಅದರಂತೆಯೇ ದೇವರ ಹೆಸರನ್ನು ಉಚ್ಚಾರ ಮಾಡಿದರೆ ಒಳ್ಳೆಯದಲ್ಲದೆ ಮತ್ತೇನೂ ಆಗುವುದಿಲ್ಲ.

ಈಶ್ವರನಿಂದೆ \enginline{(Blasphemy)} ಎಂಬ ಭಾವನೆ ಬಂದಿರುವುದಕ್ಕೆ ಕಾರಣ ಪರ್ಷಿಯಾ ದೇಶದವರ ಸ್ವಾಮಿಭಕ್ತಿಯಿಂದ ಯಹೂದ್ಯರು ಪ್ರಭಾವಿತರಾದುದು. ದೇವರು ನ್ಯಾಯಾಧಿಪತಿ, ಶಿಕ್ಷಕ ಎನ್ನುವುದು ತಮ್ಮಷ್ಟಕ್ಕೇ ಕೆಟ್ಟದ್ದೇನೂ ಅಲ್ಲ. ಆದರೆ ಅದು ಬಹಳ ಕೆಳಗಿನ ಭಾವನೆ, ಕೀಳುಭಾವನೆ, ಪ್ರೀತಿಯ ತ್ರಿಭುಜಗಳೆಂದರೆ ಪ್ರೀತಿ ಭಿಕ್ಷೆ ಬೇಡುವುದಿಲ್ಲ. ಪ್ರೀತಿಗೆ ಅಂಜಿಕೆ ಇಲ್ಲ, ಪ್ರೀತಿಗಾಗಿ ಪ್ರೀತಿಸುವುದು ಎಂಬುವು.

“ಈ ಪ್ರಪಂಚವನ್ನೆಲ್ಲಾ ನಮ್ಮ ಪ್ರೇಮೇಶ್ವರ ತುಂಬಿತುಳುಕಾಡದೆ ಇದ್ದರೆ ಬದುಕಲು ಯಾರಿಗೆ ಸಾಧ್ಯವಾಗುತ್ತಿತ್ತು? ಯಾರಿಗೆ ಒಂದು ಕ್ಷಣವಾದರೂ ಉಸಿರಾಡುವುದಕ್ಕೆ ಆಗುತ್ತಿತ್ತು?”

ನಾವು ಕೇವಲ ಸೇವೆಗಾಗಿ ಜನ್ಮ ತಾಳಿರುವೆವೆಂದು ನಮ್ಮಲ್ಲಿ ಅನೇಕರಿಗೆ ಗೊತ್ತಾಗುವುದು. ಫಲವನ್ನು ದೇವರಿಗೆ ಬಿಡಬೇಕು. ದೇವರ ಪ್ರೀತಿಗಾಗಿ ಕರ್ಮವನ್ನು ಮಾಡಿಯಾಯಿತು. ಸೋಲಾದರೆ ಅದಕ್ಕೆ ದುಃಖಪಡಬೇಕಾಗಿಲ್ಲ. ದೇವರ ಪ್ರೀತಿಗಾಗಿ ಅದನ್ನು ಮಾಡಿದ್ದು.

ಸ್ತ್ರೀಯರಲ್ಲಿ ಮಾತೃ ಸ್ವಭಾವ ಚೆನ್ನಾಗಿ ಅಭಿವೃದ್ಧಿಯಾಗಿದೆ. ಅವರು ದೇವರನ್ನು ತಮ್ಮ ಮಗುವೆಂದು ಪ್ರೀತಿಸುವರು. ಅವರು ಮಗುವಿನಿಂದ ಏನನ್ನೂ ಆಶಿಸುವುದಿಲ್ಲ. ಅದಕ್ಕೆ ಏನನ್ನು ಬೇಕಾದರೂ ಮಾಡಲು ಸಿದ್ಧರಾಗಿರುವರು.

ಕ್ಯಾಥೊಲಿಕ್​ ಪಂಥ ಇಂತಹ ಅನೇಕ ಆಧ್ಯಾತ್ಮಿಕ ವಿಷಯಗಳನ್ನು ಬೋಧಿಸುವುದು. ಈ ಧರ್ಮ ಅಷ್ಟು ವಿಶಾಲವಲ್ಲವಾಗಿದ್ದರೂ ಅದು ಅದರ ಉತ್ತಮಾರ್ಥದಲ್ಲಿ ಧಾರ್ಮಿಕವಾದ\-ದ್ದಾಗಿದೆ. ಈಗಿನ ಕಾಲದಲ್ಲಿ ಪ್ರಾಟೆಸ್ಟೆಂಟ್​ ಧರ್ಮವು ಉದಾರವಾದದ್ದಾಗಿದೆ. ಆದರೆ ಅದರಲ್ಲಿ ಆಳವಿಲ್ಲ. ಜಗತ್ತಿಗೆ ಅದರಿಂದ ಏನು ಪ್ರಯೋಜನ ಎಂಬ ಅಳತೆಕೋಲಿನಿಂದ ಸತ್ಯವನ್ನು ಅಳೆಯುವುದು, ಮಗು ವೈಜ್ಞಾನಿಕವಾದ ನವನವಾನ್ವೇಷಣೆಗಳಿಂದ ಏನು ಪ್ರಯೋಜನ ಎಂದು ಕೇಳಿದಂತೆ.

\eject

ನಾವು ಸಮಾಜವನ್ನು ಮೀರಿ ಬೆಳೆಯಬೇಕು. ಕಾನೂನುಗಳನ್ನು ಅಲ್ಲಗಳೆದು ನಾವು ನಿಯಮಾತೀತರಾಗಬೇಕು. ಪ್ರಕೃತಿಗೆ ಅವಕಾಶ ಕೊಡುವುದು ನಾವು ಅದನ್ನು ಗೆಲ್ಲುವುದಕ್ಕೆ ಮಾತ್ರ. ಭಗವಂತ ಮತ್ತು ಜಗತ್ತು ಎರಡಕ್ಕೂ ಏಕಕಾಲದಲ್ಲಿ ಸೇವೆ ಸಲ್ಲಿಸಲು ಆಗುವುದಿಲ್ಲ; ಆಗುವುದಿಲ್ಲ ಎಂಬುದೇ ತ್ಯಾಗ ಎಂಬುದರ ಅರ್ಥ.

ನಿಮ್ಮ ಬುದ್ಧಿಶಕ್ತಿಯನ್ನು ಮತ್ತು ಪ್ರೀತಿಯನ್ನು ಬಲಗೊಳಿಸಿ. ನಿಮ್ಮ ಹೃದಯಪುಷ್ಪ ವಿಕಾಸಗೊಳ್ಳಲಿ. ದುಂಬಿಗಳು ತಾವಾಗಿಯೇ ಬರುವುವು. ಮೊದಲು ನಿಮ್ಮನ್ನೇ ನಂಬಿ, ಅನಂತರ ದೇವರನ್ನು ನಂಬಿ. ಸ್ವಲ್ಪವೇ ಸ್ವಲ್ಪ ಸಂಖ್ಯೆಯ ಧೀರರು ಜಗತ್ತನ್ನೇ ಅಳ್ಳಾಡಿಸುವರು.\break ನಾವು ಅನುಭವಿಸುವುದಕ್ಕೆ ನಮಗೊಂದು ಹೃದಯ ಬೇಕು, ಆಲೋಚಿಸುವುದಕ್ಕೆ ಒಂದು ಮಿದುಳು ಬೇಕು, ಕೆಲಸ ಮಾಡುವುದಕ್ಕೆ ಬಲವಾದ ಬಾಹುಗಳು ಬೇಕು. ಬುದ್ಧನು ಪ್ರಾಣಿಗಳಿಗೆ ತನ್ನ ಬಾಳನ್ನೇ ಕೊಟ್ಟ. ಕೆಲಸ ಮಾಡುವುದಕ್ಕೆ ನೀವು ಒಂದು ಯೋಗ್ಯ\break ಉಪಕರಣವಾಗಿ. ಆದರೆ ನಿಜವಾಗಿ ಕೆಲಸ ಮಾಡುವವನು ದೇವರು, ನೀವಲ್ಲ. ಒಬ್ಬನಲ್ಲಿ ಇಡೀ ವಿಶ್ವವಿದೆ. ಕಣವೊಂದರಲ್ಲಿ ಇಡೀ ವಿಶ್ವಶಕ್ತಿ ಹುದುಗಿದೆ. ಹೃದಯಕ್ಕೂ ಬುದ್ಧಿಗೂ ಘರ್ಷಣೆಯಾದಾಗ ಯಾವಾಗಲೂ ಹೃದಯವನ್ನು ಅನುಸರಿಸಿ.

ಸ್ಪರ್ಧೆ ಎಂಬುದು ನಿನ್ನೆಯ ನಿಯಮ. ಇಂದು ಸಹಕಾರವೇ ಜೀವನದ ನಿಯಮ\-ವಾಗಿದೆ. ಮುಂದೆ ಯಾವ ನಿಯಮವೂ ಇಲ್ಲದೆ ಇರಬಹುದು. ಸಜ್ಜನರು ನಿನ್ನನ್ನು ಹೊಗಳಿದರೇನು ಅಥವಾ ಪ್ರಾಪಂಚಿಕರು ನಿನ್ನನ್ನು ತೆಗಳಿದರೇನು? ಲಕ್ಷ್ಮಿ ಒಲಿದು ಬರಲಿ, ಇಲ್ಲವೆ ದರಿದ್ರ ಲಕ್ಷ್ಮಿ ನಿನ್ನನ್ನು ಅಣಕಿಸಲಿ. ಒಂದು ದಿನ ಕಾಡಿನ ಕಂದ ಮೂಲಗಳನ್ನು ತಿನ್ನು, ಮತ್ತೊಂದು ದಿನ ಐವತ್ತು ಬಗೆಯ ಅಡಿಗೆ ಮಾಡಿ ಊಟ ಮಾಡು. ಅತ್ತ ಇತ್ತ ನೋಡದೆ ನೀನು ಸುಮ್ಮನೆ ಮುಂದುವರಿ.

ಸ್ವಾಮೀಜಿಯವರು ಯಾರೋ ಕೇಳಿದ ಪ್ರಶ್ನೆಗೆ ಉತ್ತರವಾಗಿ ಪವಹಾರಿ ಬಾಬಾರ ಜೀವನದಿಂದ ಕೆಲವು ಘಟನೆಗಳನ್ನು ಉದಾಹರಿಸಿದರು. ಒಮ್ಮೆ ಪವಹಾರಿ ಬಾಬಾರು ತಮ್ಮಿಂದ ಏನೋ ಕದ್ದುಕೊಂಡು ಹೋದ ಕಳ್ಳನಿಗೆ ಕೊಡುವುದಕ್ಕಾಗಿ ಮಿಕ್ಕಿರುವ ಕೆಲವು\break ಪಾತ್ರೆಗಳನ್ನು ತೆಗೆದುಕೊಂಡು ಅವನ ಹಿಂದೆ ಓಡಿದರು. ಕಳ್ಳನನ್ನು ಕಂಡಾದ ಮೇಲೆ ಅವನ ಕಾಲಿಗೆ ಬಿದ್ದು “ಸ್ವಾಮಿ, ನೀವು ಅಲ್ಲಿದ್ದುದು ನನಗೆ ಗೊತ್ತಾಗಲಿಲ್ಲ. ಇವುಗಳನ್ನೂ ತೆಗೆದುಕೊಳ್ಳಿ. ಇವೂ ನಿಮ್ಮವು. ನಿಮ್ಮ ಮಗು ನಾನು. ದಯವಿಟ್ಟು ಕ್ಷಮಿಸಿ” ಎಂದು ಬೇಡಿಕೊಂಡರು.

ಮತ್ತೊಮ್ಮೆ ನಾಗರಹಾವು ಇವರನ್ನು ಕಚ್ಚಿ ಪ್ರಜ್ಞೆ ತಪ್ಪಿ ಬಿದ್ದರು. ಸಂಜೆ ಪ್ರಜ್ಞೆ ಬಂದ ಮೇಲೆ “ಪ್ರೇಮಿಯಿಂದ ಒಬ್ಬ ದೂತ ಬಂದಿದ್ದ” ಎಂದರು.

