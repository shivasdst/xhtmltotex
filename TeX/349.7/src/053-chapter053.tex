
\vspace{-0.8cm}

\chapter[ಸ್ವಾರ್ಥತ್ಯಾಗವೇ ಧರ್ಮ ]{ಸ್ವಾರ್ಥತ್ಯಾಗವೇ ಧರ್ಮ \protect\footnote{\engfoot{C.W. Vol. VI. p 83}}}

ಜಗತ್ತಿನ ಹಕ್ಕುಗಳನ್ನು ಹಂಚುವುದಕ್ಕೆ ಆಗುವುದಿಲ್ಲ. ಹಕ್ಕು ಎಂದು ಹೇಳುವುದೇ ಮಿತಿಯನ್ನು ಕಲ್ಪಿಸುವುದು. ಇದು ಹಕ್ಕಲ್ಲ, ಒಂದು ‘ಜವಾಬ್ದಾರಿ.’ ಜಗತ್ತಿನಲ್ಲಿ ಎಲ್ಲೇ ಕೆಟ್ಟದಿದ್ದರೂ ಪ್ರತಿಯೊಬ್ಬನೂ ಅದಕ್ಕೆ ಜವಾಬ್ದಾರ. ಯಾರೂ ತಮ್ಮ ಸಹೋದರರಿಗಿಂತ ಬೇರೆ ಆಗಲಾರರು. ಯಾವುದು ನಮ್ಮನ್ನು ವಿಶ್ವದೊಂದಿಗೆ ಒಂದುಗೂಡಿಸುವುದೊ ಅದೆಲ್ಲ ಪುಣ್ಯ. ಬೇರ್ಪಡಿಸುವುದೆಲ್ಲ ಪಾಪ. ನೀನು ಅಖಂಡದ ಒಂದು ಅಂಶ, ಇದೇ ನಿನ್ನ ಸ್ವಭಾವ, ಆದಕಾರಣ ನೀನು ನಿನ್ನ ಸಹೋದರನ ರಕ್ಷಕ.

ಜೀವನದ ಒಂದು ತುದಿ ಜ್ಞಾನ, ಮತ್ತೊಂದು ತುದಿ ಆನಂದ. ಜ್ಞಾನ ಆನಂದಗಳೆರಡೂ ಮನುಷ್ಯನನ್ನು ಸ್ವಾತಂತ್ರ್ಯಕ್ಕೆ ಒಯ್ಯುವುವು. ಪ್ರತಿಯೊಬ್ಬರಿಗೂ, ಒಂದು ನಾಯಿಗೂ ಕೂಡ ಮುಕ್ತಿ ಪ್ರಾಪ್ತವಾಗುವವರೆಗೆ ಯಾರಿಗೂ ಮುಕ್ತಿ ದೊರಕದು. ಎಲ್ಲರೂ ಸುಖಿಯಾಗುವ ತನಕ ಯಾರೂ ಸುಖಿಯಾಗಲಾರರು. ನೀನು ಯಾರನ್ನಾದರೂ ನೋಯಿಸಿದರೆ ಅದು ನಿನ್ನನ್ನೇ ನೋಯಿಸಿಕೊಂಡಂತೆ. ಏಕೆಂದರೆ ನೀನು ನಿನ್ನ ಸಹೋದರರಿಬ್ಬರೂ ಒಂದೇ. ಯಾರು ತನ್ನನ್ನೆಲ್ಲರಲ್ಲಿಯೂ ಕಾಣಬಲ್ಲನೋ ಮತ್ತು ಎಲ್ಲರಲ್ಲಿಯೂ ತನ್ನನ್ನು ಕಾಣಬಲ್ಲನೋ ಅವನೇ ದೊಡ್ಡ ಯೋಗಿ. ಆತ್ಮತ್ಯಾಗವೇ ಜಗತ್ತಿನ ಶ್ರೇಷ್ಠ ನಿಯಮ. ಆತ್ಮ ಪೋಷಣವಲ್ಲ, ಜಗತ್ತಿನಲ್ಲಿ ಇಷ್ಟೊಂದು ಪಾಪವಿದೆ. ಅದಕ್ಕೆ ಕಾರಣ, ಪಾಪವನ್ನು ತಡೆಯಬೇಡಿ ಎಂಬ ಏಸುವಿನ ಸಂದೇಶವನ್ನು ಪ್ರಯೋಗಿಸದೆ ಇರುವುದು. ಆತ್ಮತ್ಯಾಗ ಒಂದೇ ಸಮಸ್ಯೆಯನ್ನು ಬಗೆಹರಿಸಬಲ್ಲದು. ಆತ್ಮತ್ಯಾಗ ತೀವ್ರವಾದಾಗ ಅಧ್ಯಾತ್ಮ ಬರುವುದು. ಏನನ್ನೂ ನಿಮಗಾಗಿ ಆಶಿಸಬೇಡಿ. ಎಲ್ಲವನ್ನೂ ಇತರರಿಗಾಗಿ ಮಾಡಿ. ಭಗವಂತನಲ್ಲಿರುವುದು, ಅವನಲ್ಲಿ ಬಾಳುವುದು, ಅವನಲ್ಲಿ ವ್ಯವಹರಿಸು ವುದು ಎಂದರೆ ಇದೇ.

