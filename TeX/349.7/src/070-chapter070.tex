
\chapter[ನಿಜವಾದ ಗುರು ಯಾರು? ]{ನಿಜವಾದ ಗುರು ಯಾರು? \protect\footnote{\engfoot{C.W. Vol. V, P. 257}}}

ನಿಜವಾದ ಗುರು ಕಾಲಕಾಲಕ್ಕೆ ಅವತಾರ ಮಾಡುವ ಮಹಾ ಆಧ್ಯಾತ್ಮಿಕ ಶಕ್ತಿ. ಅವನು ಆ ಆಧ್ಯಾತ್ಮಿಕ ಶಕ್ತಿಯನ್ನು ಗುರುಶಿಷ್ಯರ ಪರಂಪರೆಯ ಮೂಲಕ ಮುಂದಿನವರಿಗೆ ನೀಡುವನು. ಈ ಆಧ್ಯಾತ್ಮಿಕ ಪ್ರವಾಹ ಕಾಲಕಾಲಕ್ಕೆ ಗತಿಯನ್ನು ಬದಲಾಯಿಸುವುದು. ನದಿ ಹೇಗೆ ಕೆಲವು ವೇಳೆ ಹಳೆಯ ಪಾತ್ರವನ್ನು ಬಿಟ್ಟು ಹೊಸ ಪಾತ್ರವನ್ನು ಮಾಡಿಕೊಳ್ಳುವುದೊ ಅದರಂತೆ ಅದು. ಆದಕಾರಣ ಹಳೆಯ ಧಾರ್ಮಿಕ ಪಂಗಡಗಳು ಕೆಲವು ಕಾಲದ ಮೇಲೆ ನಿರ್ಜೀವವಾಗಿ ಹೋಗುವುವು. ಚೇತನವಿರುವ ಹೊಸ ಹೊಸ ಪಂಥಗಳು ಹುಟ್ಟುವುವು. ನಿಜವಾಗಿಯೂ ಬುದ್ಧಿವಂತರಾಗಿರುವವರು ಹೊಸ ಜೀವಕಳೆ ತುಂಬಿರುವ ಪಂಥದ ಆಶ್ರಯದಲ್ಲಿ ಬೆಳೆಯುವರು. ಹಳೆಯ ಪಂಥಗಳು ವಸ್ತು ಪ್ರದರ್ಶನಾಲಯದಲ್ಲಿರುವ ದೊಡ್ಡ ದೊಡ್ಡ ಮೃಗಗಳ ಅಸ್ಥಿಪಂಜರದಂತೆ. ಅವುಗಳನ್ನು ನಾವು ಗೌರವದಿಂದ ಕಾಣಬೇಕು. ಆದರೆ ಅವು ಮಾನವನ ತೀವ್ರ ಆಧ್ಯಾತ್ಮಿಕ ಹಂಬಲವನ್ನು ತೃಪ್ತಿಪಡಿಸಲಾರವು. ಒಣಗಿದ ಮಾವಿನ ಮರಕ್ಕೆ ಅದರ ಹಣ್ಣನ್ನು ಇಚ್ಛಿಸುವವನ ಬೇಡಿಕೆಯನ್ನು ತೃಪ್ತಿಪಡಿಸಲು ಅಸಾಧ್ಯವಾದಂತೆ.

ನಾವು ಅಹಂಕಾರದಿಂದ ಪಾರಾಗಬೇಕಾದರೆ, ನಮ್ಮಲ್ಲಿ ಏನೋ ಒಂದು ಆಧ್ಯಾತ್ಮಿಕ ಮಹಾ ಸಂಪತ್ತು ಇದೆ ಎಂದು ಭಾವಿಸುವೆವು. ಇದನ್ನು ತೊರೆದು ಗುರುವಿನ ಅಡಿದಾವರೆಯಲ್ಲಿ ಶರಣಾಗತರಾಗಬೇಕು. ಪೂರ್ಣತೆಯೆಡೆಗೆ ಯಾವುದು ಒಯ್ಯುವುದು ಎಂಬುದು ಗುರುವಿಗೆ ಮಾತ್ರ ಗೊತ್ತು; ನಮಗೆ ಗೊತ್ತಿಲ್ಲ. ನಮಗೆ ಏನೂ ಗೊತ್ತಿಲ್ಲ. ಇಂತಹ ದೈನ್ಯ\break ಇದ್ದರೆ ಆಧ್ಯಾತ್ಮಿಕ ಸತ್ಯವನ್ನು ಸ್ವೀಕರಿಸಲು ನಾವು ಯೋಗ್ಯರಾಗುವೆವು. ನಮ್ಮಲ್ಲಿ ಅಹಂಕಾರದ ಅವಶೇಷ ಸ್ವಲ್ಪವಿದ್ದರೂ ಸತ್ಯ ಮನಸ್ಸಿಗೆ ಹೊಳೆಯುವುದಿಲ್ಲ. ನೀವೆಲ್ಲರೂ ಅಹಂಕಾರದ ಭೂತವನ್ನು ನಿರ್ಮೂಲ ಮಾಡಲು ಯತ್ನಿಸಬೇಕು. ಪರಮಾರ್ಥದ ಸಾಕ್ಷಾತ್ಕಾರಕ್ಕೆ\break ಸಂಪೂರ್ಣ ಶರಣಾಗತಿಯೊಂದೇ ಮಾರ್ಗ.

