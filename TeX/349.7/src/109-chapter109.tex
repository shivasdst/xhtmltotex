
\chapter[ಪತಿತಳಾದ ಸ್ತ್ರೀ ]{ಪತಿತಳಾದ ಸ್ತ್ರೀ \protect\footnote{\engfoot{C.W. Vol. VII, p. 291}}}

\centerline{(‘ಡೆಟ್ರಾಯಿಟ್​ ಟ್ರಿಬ್ಯೂನ್​’ – ಮಾರ್ಚ್​ ೧೭, ೧೮೯೪)}

“ಲಾಲುನ್​ ಎಂಬುವಳು ಪ್ರಪಂಚದ ಒಂದು ಅತ್ಯಂತ ಪುರಾತನವಾದ ಉದ್ಯೋಗದ ಸದಸ್ಯಳು. ಲಿಲಿತ್​ ಎಂಬವಳು ಅವಳ ಮುತ್ತಜ್ಜಿ. ಅದು ನಿಮಗೆಲ್ಲ ಗೊತ್ತಿರುವಂತೆ ಈಗಿನ ಕಾಲಕ್ಕಿಂತ ಮುಂಚಿನ ಪ್ರಸಂಗ. ಪಾಶ್ಚಾತ್ಯ ದೇಶದಲ್ಲಿ ಲಾಲುನ್​ ಉದ್ಯೋಗದ ವಿಷಯದಲ್ಲಿ ಬಹಳ ಹೀನವಾದ ವಿಷಯಗಳನ್ನು ಹೇಳುವರು. ಅದರ ಮೇಲೆ ಉಪನ್ಯಾಸಗಳನ್ನು ಬರೆಯುವರು, ಅದನ್ನು ಯುವಕರಿಗೆ ಹಂಚುವರು. ಇದರಿಂದ ಅವರು ತಮ್ಮ ನೈತಿಕ ಜೀವನಕ್ಕೆ ಚ್ಯುತಿಬರದಂತೆ ರಕ್ಷಿಸಿಕೊಳ್ಳಲಿ ಎಂದು. ಪ್ರಾಚ್ಯ ದೇಶದಲ್ಲಿ ಈ ಉದ್ಯೋಗ ಅನುವಂಶಿಕವಾಗಿ ಬಂದದದು. ತಾಯಿಯಿಂದ ಮಗಳಿಗೆ ಬರುವುದು. ಇವರ ವಿಷಯದಲ್ಲಿ ಯಾರು ಉಪನ್ಯಾಸಗಳನ್ನು ಮಾಡುವುದಿಲ್ಲ, ಮತ್ತು ಯಾರೂ ಇವುಗಳನ್ನು ಗಮನಕ್ಕೂ ತೆಗೆದುಕೊಳ್ಳುವುದಿಲ್ಲ.” (ರುಡ್​ಯಾರ್ಡ್​ ಕಿಪ್​ಲಿಂಗ್​).

ಈ ವಾಕ್ಯಗಳು ಪ್ರಾರಂಭವಾಗುವುದಕ್ಕೆ ಮುಂಚೆ ಬರುವ ಕಥೆಯನ್ನು ಭರತಖಂಡದಲ್ಲಿ ಬರೆದದ್ದು. ಇದನ್ನು ರುಡ್​ಯಾರ್ಡ್​ ಕಿಪ್​ಲಿಂಗ್​ ಬರೆದನು. ಇವನಿಂದಲೇ ಇಂಡಿಯಾದೇಶದಲ್ಲಿ ಅವರು ನಮ್ಮ ರೈತರ ಮೇಲೆ ಪೋಟಾಪೋಟಿ ಮಾಡುವಷ್ಟು ಗೋಧಿಯನ್ನು ಬೆಳೆಯುವರು, ಅಲ್ಲಿಯ ಜನ ದಿನಕ್ಕೆ ಎರಡು ಸೆಂಟು ಕೂಲಿಗೆ ದುಡಿಯುವರು, ಅಲ್ಲಿಯ ಹೆಂಗಸರು ತಮ್ಮ ಮಕ್ಕಳನ್ನು ಆ ದೇಶದ ಪವಿತ್ರನದಿಯಾದ ಗಂಗಾನದಿಗೂ ಹಾಕುವರು, ಎಂಬ ವಿಷಯಗಳುಗೊತ್ತಿವೆ

ಆದರೆ ವಿವೇಕಾನಂದರು ಈ ದೇಶಕ್ಕೆ ಬಂದಾಗಿನಿಂದಲೂ ಅಲ್ಲಿಯ ಹೆಂಗಸರು ತಮ್ಮ ಮಕ್ಕಳನ್ನು ಮೊಸಳೆಯ ಬಾಯಿಗೆ ಎಸೆಯುವರು ಎಂಬ ಕಥೆಯನ್ನು ಅಲ್ಲಗಳೆದಿರುವರು. ಈಗ ಅವರು ತಾವು ಅಮೆರಿಕಾ ದೇಶಕ್ಕೆ ಬರುವುದಕ್ಕೆ ಮುಂಚೆ ರುಡ್​ಯಾರ್ಡ್​ ಕಿಪ್​ಲಿಂಗ್​ ಅವರ ಹೆಸರನ್ನೇ ಕೇಳಿರಲಿಲ್ಲ ವೆಂದು ಹೇಳುವರು. ಲಾಲುನ್​ ಉದ್ಯೋಗವನ್ನು (ವೈಶ್ಯಾವೃತ್ತಿ) ಕುರಿತು ಭರತಖಂಡದಲ್ಲಿ ಬಹಿರಂಗವಾಗಿ ಮಾತನಾಡುವುದು ಯೋಗ್ಯವಲ್ಲವೆಂದು ಭಾವಿಸುವರು. ಕಿಪ್​ಲಿಂಗ್​ ಆ ವಿಷಯವನ್ನೇ ತೆಗೆದುಕೊಂಡು ಒಂದು ರಸವತ್ತಾದ ಮತ್ತು ನೀತಿಬೋಧಕ ಕಥೆಯನ್ನು ಬರೆದಿರುವನು.

ವಿವೇಕಾನಂದರು ಹೇಳಿದರು: “ಇಂಡಿಯಾ ದೇಶದಲ್ಲಿ ಜನ ಇವುಗಳನ್ನು ಚರ್ಚಿಸುವುದಿಲ್ಲ. ಯಾರೂ ಅಂತಹ ದುರದೃಷ್ಟ ಹೆಂಗಸಿನ ವಿಷಯವನ್ನು ಕುರಿತು ಮಾತನಾಡುವುದಿಲ್ಲ. ಹೆಂಗಸು ಭ್ರಷ್ಟಳು ಎಂದು ತಿಳಿದ ತಕ್ಷಣವೇ ಅವಳನ್ನು ಅವಳ ಜಾತಿಯಿಂದ ಬಹಿಷ್ಕರಿಸುವರು. ಅನಂತರ ಯಾರೂ ಅವಳನ್ನು ಮುಟ್ಟಲಾರರು, ಮತ್ತು ಅವಳೊಂದಿಗೆ ಮಾತನಾಡಲಾರರು. ಅವಳೇನಾದರೂ ಯಾರ ಮನೆಗಾದರೂ ಹೋದರೆ ಅವಳು ಸೋಂಕಿದ ಚಾಪೆಯನ್ನು ತೊಳೆಯುವರು. ಅವಳ ಉಸಿರು ತಾಕಿದ ಗೋಡೆಯನ್ನು ಶುದ್ಧ ಮಾಡುವರು. ಯಾರೂ ಅವಳೊಂದಿಗೆ ಯಾವ ವ್ಯವಹಾರವನ್ನೂ ಇಟ್ಟುಕೊಳ್ಳುವುದಿಲ್ಲ. ಹಿಂದೂ ಸಮಾಜ ದಲ್ಲಿ ಪತಿವ್ರತೆಯರಲ್ಲದ ಸ್ತ್ರೀಯರಿಲ್ಲ. ಅದು ಈ ದೇಶದಲ್ಲಿರುವಂತೆ ಇಲ್ಲ. ಈ ದೇಶದ ಸಮಾಜದಲ್ಲಿ ಒಳ್ಳೆಯ ಹೆಂಗಸರ ಜೊತೆಜೊತೆಯಲ್ಲಿಯೇ ಕೆಟ್ಟ ಹೆಂಗಸರೂ ಇರುತ್ತಾರೆ. ಅಮೆರಿಕಾ ದೇಶದಲ್ಲಿ ಯಾರು ಒಳ್ಳೆಯವರು ಯಾರು ಕೆಟ್ಟವರು ಎಂದು ಹೇಳುವುದಕ್ಕೆ ಆಗುವುದಿಲ್ಲ. ಆದರೆ ಇಂಡಿಯಾ ದೇಶದಲ್ಲಿ ಸ್ತ್ರೀ ಒಮ್ಮೆ ಪತಿತ ಳಾದಳು ಎಂದರೆ, ಅವಳು ಎಂದೆಂದಿಗೂ ಹೊರಗೆ ಹೋದಂತೆ. ಅವಳು, ಅವಳ ಹೆಣ್ಣು ಮಕ್ಕಳು, ಗಂಡು ಮಕ್ಕಳು ಎಲ್ಲರೂ ಬಹಿಷ್ಕೃತರು. ಇದು ತುಂಬಾ ಘೋರವಾದದು ಎಂದು ನಾನು ಒಪ್ಪಿಕೊಳ್ಳುತ್ತೇನೆ. ಆದರೆ ಇದು ಸಮಾಜವನ್ನು ಶುದ್ಧವಾಗಿ ಮಾಡುವುದು.”

“ಪುರುಷರ ವಿಷಯ ಹೇಗೆ? ಪುರುಷನಿಗೂ ಇದೇ ನಿಯಮ ಒಪ್ಪುವುದೇ? ಅವರು ಪತಿತರಾಗಿರುವರು ಎಂಬುದು ಗೊತ್ತಾದ ಮೇಲೆ ಅವರನ್ನು ಬಹಿಷ್ಕರಿಸು ವರೇ?” ಎಂದು ಕೇಳಿದರು.

“ಇಲ್ಲ. ಅವರ ವಿಷಯ ಬೇರೆ. ಪುರುಷರನ್ನು ಕಂಡುಹಿಡಿದರೆ ಬಹುಶಃ ಅವರಿಗೂ ಅದೇ ಪಾಡಾಗಬಹುದು. ಆದರೆ ಪುರುಷರು ತಪ್ಪಿಸಿಕೊಳ್ಳುವರು. ಊರಿಂದೂರಿಗೆ ಹೋಗುವರು. ಅವರನ್ನು ಕಂಡುಹಿಡಿಯುವುದು ಕಷ್ಟ. ಹೆಂಗಸರು ಮನೆಯಲ್ಲೇ ಇರಬೇಕಾಗುವುದು. ಅವರು ಏನನ್ನಾದರೂ ಮಾಡಿದರೆ ಸಿಕ್ಕಿಹಾಕಿಕೊಳ್ಳುವುದು ಖಂಡಿತ. ಅವರನ್ನು ಕಂಡುಹಿಡಿದ ಮೇಲೆ ಅವರನ್ನು ಸಮಾಜದಿಂದ ಹೊರಗೆ ಹಾಕುವರು. ಯಾವುದೂ ಅವರನ್ನು ರಕ್ಷಿಸಲಾರದು. ಕೆಲವು ವೇಳೆ ತಂದೆ ಮಗಳನ್ನು ತ್ಯಜಿಸುವಾಗ ಸಹಿಸಲು ಬಹಳ ಕಷ್ಟವಾಗ ಬಹುದು. ಅವರೇನಾದರೂ ಹಾಗೆ ಅವರನ್ನು ತ್ಯಜಿಸದೆ ಇದ್ದರೆ, ಅವರನ್ನೂ ಅವರೊಡನೆ ಬಹಿಷ್ಕರಿಸುವರು. ಆದರೆ ಈ ದೇಶದಲ್ಲಿ ಮಾತ್ರ ಬಹಳ ವ್ಯತ್ಯಾಸ. ಇಲ್ಲಿ ಹೆಂಗಸರು ಹೊರಗೆ ಹೋಗಿ ಇತರರೊಡನೆ ಮಾತುಕತೆಯಾಡುವಂತೆ ಅಲ್ಲಿ ಮಾಡಲು ಸಾಧ್ಯವಿಲ್ಲ. ಇದೇನೋ ಬಹಳ ಘೋರವಾದುದು. ಆದರೆ ಇದು ಸಮಾಜವನ್ನು ಶುದ್ಧವಾಗಿಡುವುದು.”

“ನಿಮ್ಮ ದೇಶದಲ್ಲಿ ಪಾತಿವ್ರತ್ಯ ಗುಣವಿಲ್ಲದಿರುವುದೊಂದು ಮಹಾಪಾಪ. ಅದು ಹೀಗೆಯೇ ಇರಬೇಕಾಗಿದೆ. ಇಲ್ಲಿ ಅಷ್ಟೊಂದು ಭೋಗವಸ್ತುಗಳಿವೆ. ಪಾಪ ಹುಡುಗಿ ಒಂದು ಹೊಸ ಟೋಪಿಗೆ ತನ್ನ ದೇಹವನ್ನು ಬೇಕಾದರೂ ಮಾರಲು ಸಿದ್ಧ ಳಾಗಿರುವಳು. ಎಲ್ಲಿ ಇಷ್ಟೊಂದು ಸುಖಭೋಗಗಳಿವೆಯೋ ಅಲ್ಲಿ ಪರಿಸ್ಥಿತಿ ಹೀಗೆಯೇ ಇರುತ್ತದೆ.”

ಮಿಸ್ಟರ್​ ಕಿಪ್​ಲಿಂಗ್​ ಎಂಬಾತನು ಲಾಲುನ್​ ಮತ್ತು ಅವಳ ಉದ್ಯೋಗದ ವಿಷಯದಲ್ಲಿ ಹೀಗೆ ಹೇಳುವನು: “ಲಾಲುನ್​ಳ ನಿಜವಾದ ಗಂಡ (ಪೌರಸ್ತ್ಯ ದೇಶಗಳಲ್ಲಿ ಲಾಲುನ್​ಳ ಸಹೋದ್ಯೋಗಿಗಳಿಗೂ ಗಂಡಂದಿರುವರು) ಒಂದು ದೊಡ್ಡ ಬೋರೆಯ ಮರ. ಅವಳ ತಾಯಿಯು ಒಂದು ಅತ್ತಿಯ ಮರವನ್ನು ಮದುವೆ ಮಾಡಿಕೊಂಡಿದ್ದಳು. ಅವಳು ಲಾಲುನ್​ ಮದುವೆಗೆ ಹತ್ತು ಸಾವಿರ ರೂ. ಖರ್ಚು ಮಾಡಿದ್ದಳು. ತಾಯಿಯ ಪಂಗಡಕ್ಕೆ ಸೇರಿದ ನಲವತ್ತೇಳು ಜನ ಪುರೋಹಿತರು ಬಂದು ಅವಳನ್ನು ಆಶೀರ್ವದಿಸಿದರು. ಐದು ಸಾವಿರ ರೂಪಾಯಿ ಗಳನ್ನು ಬಡವರಿಗೆ ದಾನ ಮಾಡಿದರು. ಇದು ಆ ದೇಶದ ಒಂದು ಆಚಾರ.”

ಇಂಡಿಯಾ ದೇಶದಲ್ಲಿ ಯಾವಾಗ ಗಂಡನಿಗೆ ಹೆಂಡತಿ ನಿಷ್ಠೆಯಿಂದ ಇರುವು ದಿಲ್ಲವೊ ಆಗ ಅವಳು ತನ್ನ ಜಾತಿಯನ್ನು ಕಳೆದುಕೊಳ್ಳುವಳು. ಆದರೆ ಅವಳು ತನ್ನ ಯಾವ ಪೌರ ಮತ್ತು ಧಾರ್ಮಿಕ ಹಕ್ಕುಬಾಧ್ಯತೆಗಳನ್ನೂ ಕಳೆದುಕೊಳ್ಳುವುದಿಲ್ಲ. ಅವಳು ಆಸ್ತಿಯನ್ನು ಇಟ್ಟುಕೊಂಡಿರಬಹುದು. ದೇವಸ್ಥಾನಕ್ಕೆ ಹೋಗಬಹುದು.

ವಿವೇಕಾನಂದರು ಹೀಗೆಂದರು: “ಕೆಟ್ಟ ಹೆಂಗಸಿಗೆ ಮದುವೆ ಮಾಡಿಕೊಳ್ಳಲು ಸಮಾಜ ಅವಕಾಶ ಕೋಡುವುದಿಲ್ಲ. ಅವಳನ್ನು ಯಾರಾದರೂ ಮದುವೆಯಾದರೆ ಅವರೂ ಭ್ರಷ್ಟರಾಗಬೇಕಾಗುವುದು. ಅದಕ್ಕಾಗಿಯೇ ಅವಳೊಂದು ಮರವನ್ನೊ ಅಥವಾ ಕತ್ತಿಯನ್ನೊ ಮದುವೆ ಮಾಡಿಕೊಳ್ಳುವಳು. ಇದು ಅಲ್ಲಿಯ ಆಚಾರ. ಕೆಲವು ವೇಳೆ ಇಂತಹ ಸ್ತ್ರೀಯರು ಬಹಳ ಧನಿಕರಾಗುವರು, ಮತ್ತು ದಾನಿಗಳೂ ಆಗುವರು. ಆದರೆ ಪುನಃ ಅವರು ಹಿಂದಿನ ಜಾತಿಗೆ ಬರಲಾರರು. ದೇಶದ ಒಳ ಭಾಗದಲ್ಲಿ ಜನರು ಇನ್ನೂ ಹಳೆಯ ಸಂಪ್ರದಾಯಕ್ಕೆ ಬದ್ಧರಾಗಿರುವರು. ಆ ಹೆಂಗಸು ಎಷ್ಟೇ ಶ‍್ರೀಮಂತಳಾಗಿರಲಿ ಅವಳು ಗಾಡಿಯಲ್ಲಿ ಹೋಗಲಾರಳು. ಅವಳಿಗೆ ಹೆಚ್ಚು ಎಂದರೆ ಒಂದು ಜೊತೆ ಎತ್ತನ್ನು ಮಾತ್ರ ಉಪಯೋಗಿಸುವುದಕ್ಕೆ ಅಪ್ಪಣೆ ಕೊಡುವರು. ಇಂಡಿಯಾ ದೇಶದಲ್ಲಿಯಾದರೊ ಅವಳು ತನ್ನದೇ ರೀತಿಯ ವಸ್ತ್ರ ವನ್ನು ಉಡಬೇಕು. ಏಕೆಂದರೆ ಜನರಿಗೆ ಇವಳು ಇಂಥವಳು ಎಂದು ಗೊತ್ತಾಗ ಬೇಕು. ನೀವು ಇಂತಹ ಜನರು ಹೋಗುತ್ತಿರುವುದನ್ನು ನೋಡಬಹುದು. ಆದರೆ ಯಾರೂ ಅವಳೊಡನೆ ಮಾತನ್ನು ಆಡುವುದಿಲ್ಲ. ಇಂತಹ ಜನರು ಹೆಚ್ಚಾಗಿ ಇರುವುದು ದೊಡ್ಡ ದೊಡ್ಡ ಊರುಗಳಲ್ಲಿ. ಇವರಲ್ಲಿ ಹಲವರು ಯಹೂದ್ಯರು ಕೂಡ ಇರುವರು. ಆದರೆ ಊರಿನಲ್ಲಿ ಅವರಿಗೆಲ್ಲ ಬೇರೆ ಸ್ಥಳವಿದೆ. ಅವರೆಲ್ಲ ಯಾವಾಗಲೂ ಬೇರೆಯೇ ಇರುವರು. ಅವರು ಎಷ್ಟೇ ಕೆಟ್ಟವರಾಗಿದ್ದರೂ, ಎಷ್ಟೇ ದರಿದ್ರರಾಗಿದ್ದರೂ (ಅವರಲ್ಲಿ ಎಷ್ಟೋ ಜನ ಕಡುಬಡವರು) ಎಂದಿಗೂ ಕ್ರೈಸ್ತ ಮತೀಯನನ್ನು ಪ್ರೇಮಿಸುವುದಿಲ್ಲ. ಅವರೊಡನೆ ಊಟ ಮಾಡುವುದಿಲ್ಲ. ಅವರನ್ನು ಅವರು ಮುಟ್ಟುವುದಿಲ್ಲ. ಸರ್ವಭಕ್ಷಕ ಮ್ಲೇಚ್ಛರು ಎಂದು ಅವರನ್ನು ಕರೆಯುವರು. ಅವರನ್ನು ಹಾಗೆ ಕರೆಯುವುದಕ್ಕೆ ಕಾರಣ ಅವರು ಏನನ್ನು ಬೇಕಾದರೂ ತಿಂದುಬಿಡುವರು. ನೀವು ಆ ಖಾಯಿಲೆಯನ್ನು, ಮಾತಿನ ಮೂಲಕ ವಿವರಿಸಬಾರದ ಆ ರೋಗವನ್ನು, ಇಂಡಿಯಾ ದೇಶದಲ್ಲಿ ಏನೆಂದು ಕರೆಯುತ್ತಾರೆ ನಿಮಗೆ ಗೊತ್ತೆ? ಕೆಟ್ಟ ಫರಂಗಿ, ಅಂದರೆ ಕ್ರೈಸ್ತರ ರೋಗ. ಆ ರೋಗವನ್ನು ಇಂಡಿಯಾ ದೇಶಕ್ಕೆ ತಂದವರು ಕ್ರೈಸ್ತರು.”

ಪ್ರಶ್ನೆ: “ಈ ಸಮಸ್ಯೆಯನ್ನು ಬಗೆಹರಿಸಲು ಏನಾದರೂ ಪ್ರಯತ್ನ ನಡೆದಿದೆಯೆ? ಅಮೆರಿಕ ದೇಶದಲ್ಲಿ ಇರುವಂತೆ ಇದೊಂದು ಸಾರ್ವಜನಿಕ ಸಮಸ್ಯೆ ಆಗಿದೆಯೆ?”

ಉತ್ತರ: “ಇಲ್ಲ. ಇಂಡಿಯಾ ದೇಶದಲ್ಲಿ ಈ ವಿಷಯದಲ್ಲಿ ಬಹಳ ಸ್ವಲ್ಪ ಮಾಡಿರುವರು. ಭರತಖಂಡದ ವೇಶ್ಯೆಯರನ್ನು ಪರಿಶುದ್ಧ ಮಾಡಲು ಸ್ತ್ರೀ ಮಿಷನರಿಗಳಿಗೆ ಒಂದು ದೊಡ್ಡ ಕ್ಷೇತ್ರವಿದೆ. ಅವರು ಇಂಡಿಯಾ ದೇಶದಲ್ಲಿ ಏನನ್ನೂ ಮಾಡುವುದಿಲ್ಲ. ವೈಷ್ಣವರೆಂಬ ಒಂದು ಪಂಗಡವಿದೆ. ಅವರು ಈ ಸ್ತ್ರೀಯರನ್ನು ಉದ್ಧರಿಸಲು ಯತ್ನಿಸುವರು. ಇದೊಂದು ಧಾರ್ಮಿಕ ಪಂಗಡ. ವೇಶ್ಯೆಯಲ್ಲಿ ಶೇಕಡ ತೊಂಭತ್ತು ಜನ ಈ ಪಂಗಡಕ್ಕೆ ಸೇರಿದವರು ಎಂದು ನಾನು ಭಾವಿಸುತ್ತೇನೆ. ಅವರು ಜಾತಿಯನ್ನು ಒಪ್ಪುವುದಿಲ್ಲ. ಅವರು ಎಲ್ಲಿಗೆ ಬೇಕಾದರೂ ಹೋಗುವರು. ಜಗನ್ನಾಥ ದೇವಾಲಯದಂತಹ ಕೆಲವು ದೇವಾಲಯಗಳು ಇವೆ. ಅಲ್ಲಿ ಯಾವ ಜಾತಿಬೇಧವೂ ಇಲ್ಲ. ಯಾರು ಆ ಊರಿಗೆ ಹೋಗಲಿ, ಅವನು ಅಲ್ಲಿರುವವರೆಗೆ ತನ್ನ ಜಾತಿಯನ್ನು ತೆಗೆದು ಹಾಕುವನು. ಏಕೆಂದರೆ ಅದು ಪವಿತ್ರ ಸ್ಥಳ. ಅಲ್ಲಿ ಪ್ರತಿಯೊಂದೂ ಪವಿತ್ರವಾಗಿದೆ ಎಂದು ಭಾವಿಸುವರು. ಅವನು ಆ ಊರುನಿಂದ ಹೊರಗೆ ಹೊರಟರೆ ಪುನಃ ಜಾತಿಯನ್ನು ಒಪ್ಪಿಕೊಳ್ಳುವನು. ಏಕೆಂದರೆ ಜಾತಿ ಎಂಬುದು ಕೇವಲ ಲೌಕಿಕವಾದುದು. ಕೆಲವು ಜಾತಿಗಳಿವೆ, ಅದರಲ್ಲಿರುವವರು ತಾವು ಮಾಡುದುದನ್ನಲ್ಲದೆ ಮತ್ತಾರೊ ಮಾಡಿದ ಅಡಿಗೆ ಯನ್ನು ಊಟ ಮಾಡುವುದಿಲ್ಲ. ಅವರು ತಮ್ಮ ಜಾತಿಯಿಂದ ಹೊರಗೆ ಇರುವ ಯಾರನ್ನೂ ಮುಟ್ಟುವುದಿಲ್ಲ. ಆದರೆ ದೊಡ್ಡ ದೊಡ್ಡ ಊರುಗಳಲ್ಲಿ ಅವರೆಲ್ಲ ಒಟ್ಟಿಗೆ ಇರುವರು. ಈ ಪಂಗಡದವರು ಮಾತ್ರ ಇತರರನ್ನು ತಮ್ಮ ಪಂಗಡಕ್ಕೆ ಸೇರಿಸಿಕೊಳ್ಳುವರು. ಇದು ಎಲ್ಲರನ್ನೂ ತಮ್ಮ ಪಂಗಡಕ್ಕೆ ಸೇರಿಸಿಕೊಳ್ಳುವುದು. ಇವರು ಹಿಮಾಲಯಗಳಿಗೆ ಹೋಗಿ ಅಲ್ಲಿರುವ ಕಾಡುಜನರನ್ನು ಮತಾಂತರ ಗೊಳಿಸುವರು. ಭರತಖಂಡದಲ್ಲಿ ಕಾಡುಜನರು ಇದ್ದರು ಎಂಬುದು ಬಹುಶಃ ನಿಮಗೆ ಗೊತ್ತಿಲ್ಲ ಎಂದು ಕಾಣುವುದು. ಅಲ್ಲಿ ಕಾಡುಜನರು ಇರುವುದು ನಿಜ. ಅವರು ಹಿಮಾಲಯದ ತಪ್ಪಲಿನ ಪ್ರಾಂತ್ಯದಲ್ಲಿರುವರು” ವಿವೇಕಾನಂದರನ್ನು ಮತ್ತೊಂದು ಪ್ರಶ್ನೆ ಕೇಳಿದರು.

“ಸ್ತ್ರೀಯನ್ನು ಪತಿತಳೆಂದು ಸಾರುವ ಯಾವುದಾದರೂ ಪೌರ ಪದ್ಧತಿ ಇದೆಯೇ?”

“ಇಲ್ಲ. ಇದಕ್ಕೆ ಒಂದು ಪೌರಪದ್ಧತಿಯಿಲ್ಲ. ಇದು ಬರೀ ಒಂದು ಆಚಾರ ಅಷ್ಟೆ. ಕೆಲವು ವೇಳೆ ಯಾವುದಾದರೂ ಒಂದು ಬಾಹ್ಯ ಆಚಾರವಿರುವುದು, ಮತ್ತೆ ಕೆಲವು ವೇಳೆ ಇರುವುದಿಲ್ಲ. ಅವರನ್ನು ಪರೆಯರು ಎಂದು ಪರಿಗಣಿಸುವರು. ಕೆಲವು ವೇಳೆ ಯಾರಾದರೂ ಸ್ತ್ರೀಯನ್ನು ಅನುಮಾನಿಸಿದರೆ, ಕೆಲವರು ಒಟ್ಟು ಕಲೆತು ಅದನ್ನು ವಿಚಾರಿಸುವರು. ಅವಳು ತಪ್ಪಿತಸ್ಥಳು ಎಂಬುದು ಖಚಿತವಾದರೆ, ಅನಂತರ ಅವಳ ವಿಷಯದಲ್ಲಿ ಆ ಮತದಲ್ಲಿರುವ ಜನರಿಗೆಲ್ಲ ಇದನ್ನು ತಿಳಿಸಿ ಅವಳನ್ನು ಆ ಜಾತಿಯಿಂದ ಬಹಿಷ್ಕರಿಸುವರು.

ಅವರು ಅನಂತರ ಹೀಗೆ ಹೇಳಿದರು: “ಆದರೆ ಇದನ್ನು ಗಮನದಲ್ಲಿಡಿ. ಇದು ಸಮಸ್ಯೆಯ ಪರಿಹಾರ ಎಂದು ನಾನು ಭಾವಿಸುವುದಿಲ್ಲ. ಅಲ್ಲಿ ಆಚಾರ ತುಂಬಾ ಸಂಕುಚಿತವಾದುದು. ನೀವೂ ಕೂಡ ಅದಕ್ಕೆ ಯಾವ ಪರಿಹಾರೋಪಾಯವನ್ನು ಕಂಡುಹಿಡಿದಿಲ್ಲ. ಇದೊಂದು ಘೋರವಾದ ಪರಿಸ್ಥಿತಿ. ಪಾಶ್ಚಾತ್ಯ ಜನಾಂಗದ ಒಂದು ಮಹಾಪರಾಧ ಇದು.”

