
\chapter[ಜಡಭರತ]{ಜಡಭರತ \protect\footnote{\engfoot{C.W. Vol. IV. P. 111}}}

\centerline{\textbf{(ಕ್ಯಾಲಿಫೋರ್ನಿಯಾದಲ್ಲಿ ನೀಡಿದ ಉಪನ್ಯಾಸ)}}

ಹಿಂದೆ ಭರತನೆಂಬ ಚಕ್ರವರ್ತಿ ಇದ್ದ. ಹೊರಗಿನವರು ಯಾವ ದೇಶವನ್ನು ಇಂಡಿಯಾ ಎಂದು ಕರೆಯುವರೋ ಅದು ಆ ದೇಶದವರಿಗೆ ಭಾರತವರ್ಷ. ಪ್ರತಿಯೊಬ್ಬ ಹಿಂದೂ ಕೂಡ ವಯಸ್ಸಾದ ಮೇಲೆ ಲೌಕಿಕ ವ್ಯವಹಾರವನ್ನೆಲ್ಲಾ ಮಗನಿಗೆ ಕೊಟ್ಟು, ಐಶ್ವರ್ಯ ಸುಖಭೋಗಗಳನ್ನೆಲ್ಲಾ ತ್ಯಜಿಸಿ, ವಾನಪ್ರಸ್ಥನಾಗಿ ಹೋಗಬೇಕು. ಅಲ್ಲಿ ತನ್ನ ಏಕಮಾತ್ರ ಸತ್ಯವಾದ ಆತ್ಮನನ್ನು ಕುರಿತು ಚಿಂತಿಸುತ್ತಾ ಪ್ರಪಂಚದ ಬಂಧನಗಳನ್ನೆಲ್ಲಾ ಕತ್ತರಿಸಿ ಹಾಕಬೇಕು. ಕ್ಷತ್ರಿಯನಾಗಲೀ, ಬ್ರಾಹ್ಮಣನಾಗಲೀ, ವೈಶ್ಯ ನಾಗಲೀ, ಶೂದ್ರನಾಗಲೀ, ಗಂಡಸಾಗಲೀ, ಹೆಂಗಸಾಗಲೀ, ಯಾರೂ ತಮ್ಮ ಈ ಕರ್ತವ್ಯದಿಂದ ವಿಮುಖರಾಗಲಾರರು. ಸಹೋದರ, ಪತಿ, ಪಿತ, ಸುತ, ಸತಿ, ತಾಯಿ ಮುಂತಾದ ಎಲ್ಲರ ಕರ್ತವ್ಯಗಳೂ, ಆತ್ಮನನ್ನು ಬಂಧಿಸಿರುವ ಬಂಧನಗಳನ್ನೆಲ್ಲಾ ಎಂದೆಂದಿಗೂ ಕತ್ತರಿಸಿಹಾಕುವ ಸ್ಥಿತಿಗೆ ನಮ್ಮನ್ನು ಒಯ್ಯುವುದಕ್ಕೆ ಸಹಕಾರಿ.

ಭರತ ಮಹಾರಾಜ ವೃದ್ಧಾಪ್ಯದಲ್ಲಿ ತನ್ನ ರಾಜ್ಯವನ್ನು ಮಗನಿಗೊಪ್ಪಿಸಿ ಕಾಡಿಗೆ\break ಹೋದನು. ಸಹಸ್ರಾರು ಜನರನ್ನು ಆಳುತ್ತಿದ್ದವನು, ಚಿನ್ನಬೆಳ್ಳಿಯಿಂದ ಕೆತ್ತನೆ ಮಾಡಿದ ಅಮೃತಶಿಲಾಸೌಧಗಳಲ್ಲಿ ವಿಹರಿಸುತ್ತಿದ್ದವನು, ವಜ್ರ ವೈಢೂರ್ಯ ಕೆತ್ತನೆಯ ಬಟ್ಟಲಲ್ಲಿ ಕುಡಿಯುತ್ತಿದ್ದ ರಾಜ, ತನ್ನ ಕೈಗಳಿಂದಲೇ ಹುಲ್ಲಿನಿಂದ ನದೀತೀರದಲ್ಲಿ, ಹಿಮಾಲಯ ದಲ್ಲಿ, ಒಂದು ಪರ್ಣಶಾಲೆಯನ್ನು ಕಟ್ಟಿದನು. ಅಲ್ಲಿ ತಾನೇ ಆಯ್ದು ತಂದ ಕಂದ ಮೂಲಗಳ ಮೇಲೆ ಜೀವಿಸುತ್ತ ಅಂತರ್ಯಾಮಿಯಾದ ಪರಮಾತ್ಮನನ್ನು ಕುರಿತು ಚಿಂತಿಸುತ್ತಿದ್ದನು. ಹೀಗೆ ದಿನ, ಮಾಸ, ವರುಷಗಳು ಕಳೆದವು. ಒಂದು ದಿನ ಜಿಂಕೆಯೊಂದು ಆ ರಾಜಋಷಿಯು ತಪಸ್ಸು ಮಾಡುತ್ತಿದ್ದ ಸ್ಥಳಕ್ಕೆ ನೀರು ಕುಡಿಯಲು ಬಂದಿತು. ಅದೇ ಸಮಯದಲ್ಲಿ ಒಂದು ಸಿಂಹ ಅನತಿದೂರದಲ್ಲಿ ಘರ್ಜಿಸಿತು. ಜಿಂಕೆಗೆ ಬಹಳ ಭಯವಾಗಿ ನೀರನ್ನು ಕುಡಿಯದೆ ನದಿಯನ್ನು ದಾಟಲು ನೆಗೆಯಿತು. ಜಿಂಕೆ ಗರ್ಭಿಣಿಯಾಗಿತ್ತು. ಭಯಗ್ರಸ್ತವಾಗಿ ಕಷ್ಟದಿಂದ ಹಾರುವಾಗ ಮರಿಹಾಕಿ ಸತ್ತು ಹೋಯಿತು. ಮರಿ ನೀರಿಗೆ ಬಿದ್ದು ವೇಗವಾಗಿ ಹರಿಯುವ ನದಿಯಲ್ಲಿ ಕೊಚ್ಚಿ ಕೊಂಡು ಹೋಗುತ್ತಿತ್ತು. ಆ ದೃಶ್ಯವು ದೊರೆಯ ಗಮನವನ್ನು ಸೆಳೆಯಿತು. ರಾಜನು ಧ್ಯಾನದಿಂದೆದ್ದು ಆ ಮರಿಯನ್ನು ಸಂರಕ್ಷಿಸಿ ಪರ್ಣಶಾಲೆಗೆ ತೆಗೆದುಕೊಂಡುಹೋಗಿ ಬೆಂಕಿ ಹತ್ತಿಸಿ ಮರಿಯನ್ನು ಸುಧಾರಿಸಿ, ಮುದ್ದಿಸಿ, ಚೇತರಿಸಿಕೊಳ್ಳುವಂತೆ ಮಾಡಿದನು. ಆ ದಯಾಪೂರಿತನಾದ ರಾಜ ಅದನ್ನು ಸಂರಕ್ಷಿಸತೊಡಗಿದನು. ಅದಕ್ಕೆ ಗರಿಕೆ ಹುಲ್ಲು ಮತ್ತು ಹಣ್ಣನ್ನು ತಿನ್ನಿಸತೊಡಗಿದನು. ನಿವೃತ್ತ ಚಕ್ರವರ್ತಿಯ ಪಿತೃಸಹಜ ಪಾಲನೆಯಲ್ಲಿ ಆ ಮರಿ ಬೆಳೆದು ಚೆನ್ನಾದ ಜಿಂಕೆಯಾಯಿತು. ಆ ರಾಜನು ಅಧಿಕಾರ, ಅಂತಸ್ತು, ಬಂಧುಬಳಗಗಳ ಆಕರ್ಷಣೆಯಿಂದ ಪಾರಾಗಿ ಬಂದವನು ಇಲ್ಲಿ ತಾನು ರಕ್ಷಿಸಿದ ಜಿಂಕೆಮರಿಯಲ್ಲಿ ಆಸಕ್ತನಾದನು. ಜಿಂಕೆಯ ಮೇಲೆ ಪ್ರೇಮ ಹೆಚ್ಚಿದಂತೆಲ್ಲಾ ಪರಮಾತ್ಮನನ್ನು ಕುರಿತು ಚಿಂತಿಸುವುದು ಕಡಿಮೆಯಾಯಿತು. ಜಿಂಕೆ ಮೇಯುವು ದಕ್ಕೆ ಕಾಡಿಗೆ ಹೋಗಿ ಕಾಲಕ್ಕೆ ಸರಿಯಾಗಿ ಬರದೇ ಇದ್ದರೆ ರಾಜಋಷಿ ವ್ಯಾಕುಲ ನಾಗಿ ಕಾತುರಪಡುತ್ತಿದ್ದನು. ಅಯ್ಯೋ ಪಾಪ, ಆ ಮರಿಯನ್ನು ಯಾವುದೋ ಹುಲಿ ಹಿಡಿದಿರಬಹುದು, ಅಥವಾ ಇನ್ನೇನೋ ವಿಪತ್ತು ಸಂಭವಿಸಿರಬಹುದು, ಇಲ್ಲದೆ ಇದ್ದರೆ ಏತಕ್ಕೆ ಇನ್ನೂ ಬರಲಿಲ್ಲ ಎಂದು ಚಿಂತಾಕ್ರಾಂತನಾಗುವನು.

ಹೀಗೆ ಕೆಲವು ವರುಷಗಳು ಕಳೆದವು. ಒಂದು ದಿನ ಮೃತ್ಯು ಬಂತು. ಅದರ ಹಿಂದೆ ಇವನು ಹೋಗಬೇಕಾಯಿತು. ಆದರೆ ಅವನ ಮನಸ್ಸು ಪರಮಾತ್ಮನನ್ನು ಕುರಿತು ಚಿಂತಿಸುವ ಬದಲು ಜಿಂಕೆಯನ್ನು ಕುರಿತು ಚಿಂತಿಸುತ್ತಿತ್ತು. ತನ್ನ ಪ್ರೀತಿಗೆ ಪಾತ್ರವಾದ ಆ ಜಿಂಕೆಯ ದುಃಖದ ಮುಖವನ್ನು ನೋಡುತ್ತಾ ಅವನು ಪ್ರಾಣ ಬಿಟ್ಟನು. ಇದರಿಂದ ಅವನು ಮುಂದಿನ ಜನ್ಮದಲ್ಲಿ ಜಿಂಕೆಯಾಗಿಯೇ ಹುಟ್ಟಿದನು. ಆದರೆ ಯಾವ ಕರ್ಮವೂ ನಾಶವಾಗಲಾರದು. ಅವನು ರಾಜನಾಗಿದ್ದಾಗ, ಋಷಿ ಯಾದ ಮೇಲೆ ಮಾಡಿದ ಪುಣ್ಯ ಕರ್ಮಗಳೆಲ್ಲಾ ಫಲಿಸಿದುವು. ಈಗ ಜಿಂಕೆ ಜಾತಿ ಸ್ಮರವಾಗಿತ್ತು. ಜಿಂಕೆಯು ದೇಹದಲ್ಲಿದ್ದರೂ, ಮೂಕವಾಗಿದ್ದರೂ, ಅದಕ್ಕೆ ತನ್ನ ಹಿಂದಿನ ಜನ್ಮದ ಜ್ಞಾಪಕವಿತ್ತು. ತನ್ನ ಬಳಗದ ಜಿಂಕೆಯಿಂದ ಯಾವಾಗಲೂ ದೂರಹೋಗಿ ಉಪನಿಷತ್ತನ್ನು ಅಧ್ಯಯನ ಮಾಡುತ್ತಿದ್ದ. ಯಾಗಯಜ್ಞಗಳಿಗೆ ಹವಿಸ್ಸನ್ನು ಅರ್ಪಿಸುತ್ತಿದ್ದ ಪರ್ಣಶಾಲೆಯ ಸಮೀಪದಲ್ಲಿ ಮೇಯುತ್ತಿತ್ತು.

ಜಿಂಕೆ ಜನ್ಮವನ್ನು ಬಾಳಿ ಆದಮೇಲೆ, ಅವನು ಕಾಲವಾಗಿ ಒಬ್ಬ ಶ‍್ರೀಮಂತ ಬ್ರಾಹ್ಮಣನ ಕಿರಿಯ ಮಗನಾಗಿ ಹುಟ್ಟಿದನು. ಈ ಜನ್ಮದಲ್ಲಿಯೂ ಅವನಿಗೆ ತನ್ನ ಹಿಂದಿನ ಜನ್ಮಗಳೆಲ್ಲ ಜ್ಞಾಪಕವಿತ್ತು. ಮಗುವಾಗಿದ್ದಾಗಲೇ ಮತ್ತೊಮ್ಮೆ ಪಾಪಪುಣ್ಯ ಗಳ ಬಲೆಗೆ ಬೀಳುವುದಿಲ್ಲವೆಂದು ಸಂಕಲ್ಪ ಮಾಡಿಕೊಂಡನು. ಮಗು ಬೆಳೆದಂತೆ ಬಲವಾಗಿ ಧೃಢವಾಗುತ್ತಾ ಬಂತು. ಆದರೆ ಒಂದು ಮಾತನ್ನೂ ಆಡುತ್ತಿರಲಿಲ್ಲ. ಪ್ರಪಂಚದಲ್ಲಿ ಮತ್ತೊಮ್ಮೆ ಎಲ್ಲಿ ಬೀಳುವೆನೋ ಎಂದು ಜಡನಂತೆ, ಹುಚ್ಚನಂತೆ ಇದ್ದನು. ಅವನು ಯಾವಾಗಲೂ ಅಂತರಾತ್ಮನನ್ನು ಕುರಿತು ಚಿಂತಿಸುತ್ತ ತನ್ನ ಪ್ರಾರಬ್ಧಕರ್ಮವನ್ನು ಸವೆಸುವುದಕ್ಕೆ ಮಾತ್ರ ಇದ್ದನು. ಕಾಲಕ್ರಮೇಣ ತಂದೆ ಕಾಲವಾದ ಮೇಲೆ ಮಕ್ಕಳು ಆಸ್ತಿಯನ್ನು ತಮ್ಮಲ್ಲೇ ಹಂಚಿಕೊಂಡರು. ಕೊನೆಯವನು ಮೂಗ, ಯಾವುದಕ್ಕೂ ಪ್ರಯೋಜನವಿಲ್ಲವೆಂದು ಅವನ ಪಾಲನ್ನು ಇವರೇ ಕಸಿದು ಕೊಂಡರು. ಅವನು ಪ್ರಾಣದಿಂದಿರುವುದಕ್ಕೆ ಸ್ವಲ್ಪ ಆಹಾರವನ್ನು ಮಾತ್ರ ಕೊಡುತ್ತಿದ್ದರು. ಇಷ್ಟೇ ಅವರಿಗೆ ಕೊನೆಯ ತಮ್ಮನ ಮೇಲಿದ್ದ ದಯೆ. ಅಣ್ಣನ ಹೆಂಡತಿಯರು ಕೆಲವು ವೇಳೆ ಬಹಳ ಕ್ರೂರವಾಗಿ ಇವನ ಹತ್ತಿರ ವರ್ತಿಸುತ್ತಿದ್ದರು. ಎಲ್ಲಾ ಕಷ್ಟದ ಕೆಲಸವನ್ನೂ ಅವನಿಗೆ ಕೊಡುತ್ತಿದ್ದರು. ಅವರು ಹೇಳುವುದನ್ನೆಲ್ಲಾ ಇವನಿಗೆ ಮಾಡಲು ಅಸಾಧ್ಯವಾದರೆ ಅವನನ್ನು ಬಹಳ ನಿರ್ದಯದಿಂದ ಕಾಣು ತ್ತಿದ್ದರು. ಆದರೆ ಇವನು ಕೋಪವನ್ನಾಗಲೀ, ಅಂಜಿಕೆಯನ್ನಾಗಲೀ ತೋರಿಸಿಕೊಳ್ಳು ತ್ತಿರಲಿಲ್ಲ. ಒಂದು ಮಾತ್ನು ಕೂಡಾ ಆಡುತ್ತಿರಲಿಲ್ಲ. ಇವನನ್ನು ವಿಪರೀತ ಹಿಂಸಿಸಿದರೆ ಮನೆಯಿಂದ ಹೊರಗೆ ಹೋಗಿ ಕೆಲವು ಗಂಟೆಗಳು ಒಂದು ಮರದ ಕೆಳಗೆ ಕುಳಿತು ಅವರ ಕೋಪವೆಲ್ಲ ಶಮನವಾದ ಮೇಲೆ ಮನೆಗೆ ಬರುತ್ತಿದ್ದನು.

ಒಂದು ದಿನ ಭರತನ ಅಣ್ಣಂದಿರ ಹೆಂಡತಿಯರು ಎಂದಿಗಿಂತ ಹೆಚ್ಚಾಗಿ ನಿರ್ದಯರಾಗಿ ಅವನನ್ನು ಕಾಡಿದಾಗ, ಅವನು ಮನೆಯಿಂದ ಹೊರಗೆ ಹೋಗಿ ಒಂದು ಮರದ ನೆರಳ ಕೆಳಗೆ ಕುಳಿತುಕೊಂಡನು. ಆ ಸಮಯದಲ್ಲಿ ಆ ದೇಶದ ರಾಜನು ಬೋಯಿಗಳು ಹೊತ್ತುಕೊಂಡು ಹೋಗುತ್ತಿದ್ದ ಪಲ್ಲಕ್ಕಿಯೊಳಗೆ ಹೋಗುತ್ತಿ ದ್ದನು. ಅವರಲ್ಲಿ ಒಬ್ಬ ಇದ್ದಕ್ಕಿದಂತೆಯೇ ಖಾಯಿಲೆ ಬಿದ್ದನು. ಅವನ ಬದಲು ಮತ್ತೊಬ್ಬನನ್ನು ಅವರ ಕಡೆಯವರು ಹುಡುಕುತ್ತಿದ್ದರು. ಮರದಕೆಳಗೆ ಇದ್ದ ಜಡಭರತನನ್ನುಕಂಡರು. ಅವನು ಧೃಡಕಾಯನಾದ ಯುವಕನಾಗಿದ್ದನು. ಖಾಯಿಲೆ ಯಾದವನ ಬದಲು ಪಲ್ಲಕ್ಕಿಯನ್ನು ಹೊರುತ್ತೀಯಾ? ಎಂದು ಕೇಳಿದರು. ಭರತ ಸುಮ್ಮನೆ ಇದ್ದನು. ಅವನು ಬಲಶಾಲಿಯಾಗಿದ್ದುದನ್ನು ನೋಡಿ ರಾಜನ ಕಡೆಯವರು ಅವನಿಗೆ ಪಲ್ಲಕ್ಕಿಯನ್ನು ಹೊರಿಸಿದರು. ಒಂದು ಮಾತನ್ನೂ ಆಡದೆ ಭರತ ನಡೆದು ಕೊಂಡು ಹೋದ. ಇದಾದ ಮೇಲೆ ಪಲ್ಲಕ್ಕಿಯನ್ನು ಜನ ಸರಿಯಾಗಿ ಹೊರುತ್ತಿಲ್ಲ ವೆಂದು ದೊರೆಗೆ ಗೊತ್ತಾಯಿತು. ರಾಜ ಹೊಸಬನಿಗೆ “ಮೂರ್ಖ, ಸ್ವಲ್ಪ ಸುಧಾರಿಸಿಕೊ, ನಿನ್ನ ಹೆಗಲು ನೋಯುತ್ತಿದ್ದರೆ” ಎಂದ. ಭರತ ಆಗ ಪಲ್ಲಕ್ಕಿಯ ಹಿಡಿಯನ್ನು ಬದಿಗಿರಿಸಿ ಜೀವನದಲ್ಲಿ ಪ್ರಥಮಬಾರಿ ಮಾತನಾಡಿದನು. “ಹೇ ರಾಜ ನೀನು ಯಾರನ್ನು ಮೂರ್ಖನೆನ್ನುವೆ? ಯಾರಿಗೆ ನೀನು ಪಲ್ಲಕ್ಕಿಯನ್ನು ಇಳಿಸು ಎಂದು ಹೇಳುತ್ತಿರುವೆ? ಯಾರಿಗೆ ನೋವು ಎನ್ನುವೆ? ನೀನು ಯಾರನ್ನು ‘ನೀನು’ ಎಂದು ಸಂಬೋಧಿಸುತ್ತಿರುವೆ? ಹೇ ರಾಜ, ‘ನೀನು’ ಎಂದರೆ ಈ ಮಾಂಸದ ಮುದ್ದೆಯಾದರೆ ಇದು ನಿನ್ನ ದೇಹದಂತೆಯೇ ಆಗಿದೆ. ಇದು ಜಡ, ಇದಕ್ಕೆ ದೇಹಾಲಸ್ಯ ನೋವು ಎಂಬುದು ಇಲ್ಲ. ಇದು ಮನಸ್ಸಾದರೆ ನಿನ್ನ ಮನಸ್ಸಿನಂತೆಯೇ ಸರ್ವವ್ಯಾಪಿ ಯಾಗಿದೆ. ಆದರೆ ‘ನೀನು’ ಎನ್ನುವುದನ್ನು ಇದಕ್ಕೆ ಮೀರಿರುವುದಕ್ಕೆ ಅನ್ವಯಿಸಿದರೆ ಅದೇ ಆತ್ಮ. ಅದು ನನ್ನಲ್ಲಿದೆ, ಅದೇ ನಿನ್ನಲ್ಲಿದೆ. ಅದೊಂದೇ ವಿಶ್ವದಲ್ಲೆಲ್ಲಾ ಇರು ವುದು. ರಾಜ, ಆತ್ಮಕ್ಕೆ ಎಂದಾದರೂ ದಣಿವಾಗುವುದೆಂದು ಭಾವಿಸಿದೆಯಾ? ಅದಕ್ಕೆ ಎಂದಾದರೂ ಆಲಸ್ಯ ಆಗುವುದೇ? ಅದಕ್ಕೆ ಏನಾದರೂ ನೋವಾಗುವುದೇ? ಓ ರಾಜ, ಈ ದೇಹವು ನೆಲದ ಮೇಲೆ ಹಾರಾಡುವ ಕ್ರಿಮಿಗಳನ್ನು ತುಳಿಯುವುದು ನನಗೆ ಇಚ್ಛೆಯಿರಲಿಲ್ಲ. ಅದನ್ನು ತಪ್ಪಿಸುವುದಕ್ಕಾಗಿ ಪ್ರಯತ್ನಿಸುತ್ತಿದ್ದಾಗ ಪಲ್ಲಕ್ಕಿ ಸ್ವಲ್ಪ ಅಡ್ಡಾದಿಡ್ಡಿಯಾಯಿತು. ಆದರೆ ಆತ್ಮನಿಗೆ ಸಾಕಾಗಿರಲಿಲ್ಲ, ಅದು ದುರ್ಬಲ ವಾಗಿರಲಿಲ್ಲ. ಅದು ಪಲ್ಲಕ್ಕಿಯ ಹಿಡಿಯನ್ನು ಎಂದೂ ಹೊರಲಿಲ್ಲ. ಏಕೆಂದರೆ ಆತ್ಮ ಸರ್ವವ್ಯಾಪಿ, ವಿಭು.” ಹೀಗೆ ಜಡಭರತ ಆತ್ಮವಿದ್ಯೆ ಮತ್ತು ಶ್ರೇಷ್ಠ ಜ್ಞಾನ ಗಳನ್ನು ಕುರಿತು ಅದ್ಭುತವಾಗಿ ಬೋಧನೆ ಮಾಡಿದನು. ಪಾಂಡಿತ್ಯ, ಬುದ್ಧಿ, ತತ್ತ್ವ ಜ್ಞಾನದಲ್ಲಿ ತನಗೆ ಸಮಾನರಿಲ್ಲ ಎಂದು ಮೆರೆಯುತ್ತಿದ್ದ ರಾಜ ಪಲ್ಲಕ್ಕಿಯಿಂದ ಇಳಿದು ಜಡಭರತನಿಗೆ ಹೀಗೆ ಹೇಳಿನು: “ಹೇ ಮಹಾನುಭಾವ, ನನ್ನನ್ನು ಕ್ಷಮಿಸ ಬೇಕು. ಪಲ್ಲಕ್ಕಿಯನ್ನು ಹೊರು ಎಂದಾಗ ತಾವು ಮಹಾಜ್ಞಾನಿಗಳು ಎಂದು ನನಗೆ ಗೊತ್ತಿರಲಿಲ್ಲ” ಎಂದ. ಭರತ ಅವನನ್ನು ಆಶೀರ್ವದಿಸಿ ಹೊರಟ. ಅವನು ತನ್ನ ಹಿಂದಿನ ಜೀವನದ ಗತಿಯನ್ನು ಮುಂದುವರಿಸಿದ. ಭರತನು ದೇಹವನ್ನು ತ್ಯಜಿಸಿದಾಗ ಎಂದೆಂದಿಗೂ ಸಂಸಾರ ಬಂಧನದಿಂದ ಮುಕ್ತನಾದನು.

