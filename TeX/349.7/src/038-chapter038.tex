
\vspace{-1cm}

\chapter[ಸಾಧನೆ ]{ಸಾಧನೆ \protect\footnote{\engfoot{*C.W. Vol. V, P 249}}}

ಅಟಾವಿಸಂ (ತಂದೆ ತಾಯಿಗಳನ್ನು ಹೋಲದೆ ಹಿಂದಿನ ಪಿತೃಗಳನ್ನು ಹೋಲುವುದು) ಮೇಲುಗೈಯಾದರೆ ನೀನು ಅವನತಿ ಹೊಂದುವೆ. ವಿಕಾಸ ಮೇಲುಗೈಯಾದರೆ ನೀನು\break ಮುಂದುವರಿಯುವೆ. ಆದಕಾರಣ ಅಟಾವಿಸಂಗೆ ನಾವು ಅವಕಾಶ ಕೊಡಬಾರದು.\break ನನ್ನ ದೇಹದಲ್ಲೆ ನಾನು ಮೊದಲು ಕಲಿತುಕೊಳ್ಳಬೇಕಾಗಿರುವುದು ಬೇಕಾದಷ್ಟು ಇದೆ. ಆದರೆ ನಾವು ನೆರೆಯವರನ್ನು ಸರಿಮಾಡುವುದರಲ್ಲಿ ಇದ್ದೇವೆ. ಅದೇ ನಮ್ಮ ತೊಂದರೆ, ಮೊದಲು ನಮ್ಮ ದೇಹದಿಂದ ನಾವು ಪ್ರಾರಂಭ ಮಾಡಬೇಕು. ಹೃದಯ, ಯಕೃತ್​\break ಇವುಗಳೆಲ್ಲ ಅಟಾವಿಸಂ ಸ್ವರೂಪದವು. ಅವುಗಳನ್ನು ನಿಮ್ಮ ಅರಿವಿಗೆ ತನ್ನಿ, ಅವನ್ನು\break ನಿಗ್ರಹಿಸಿ. ಅನಂತರ ಅವು ನಿಮ್ಮ ಇಚ್ಛಾನುಸಾರ ವರ್ತಿಸುವುವು. ಯಕೃತ್​ ನಮ್ಮ\break ಸ್ವಾಧೀನದಲ್ಲಿದ್ದ ಕಾಲ ಒಂದಿತ್ತು. ಹಸುವಿನಂತೆ ನಮ್ಮ ಚರ್ಮವನ್ನೆಲ್ಲಾ ಅಲ್ಲಾಡಿಸಬಹುದಾಗಿತ್ತು. ಕೆಲವರು ಕಠಿಣವಾದ ಅಭ್ಯಾಸ ಬಲದಿಂದ ಇವುಗಳನ್ನು ಮಾಡುವುದನ್ನು\break ನಾನು ನೋಡಿರುವೆನು. ಒಂದು ಸಲ ಸಂಸ್ಕಾರವಾದರೆ ಅದು ಅಲ್ಲಿ ಇರುವುದು.\break ಸುಪ್ತವಾಗಿರುವ ಚಟುವಟಿಕೆಗಳನ್ನೆಲ್ಲ ತೆಗೆದುಕೊಂಡು ಬನ್ನಿ, ಕರ್ಮಸಾಗರವನ್ನೆ ಹೊರಗೆ ತನ್ನಿ. ಇದೇ ನಾವು ಕಲಿಯಬೇಕಾದ ಪ್ರಥಮ ಪಾಠ. ಇದು ನಮ್ಮ ಸಮಾಜದ ಹಿತಕ್ಕೆ\break ಅತ್ಯಾವಶ್ಯಕ. ಯಾವುದು ನಮ್ಮ ಅರಿವಿನ ಮೇರೆಯೊಳಗೆ ಇರುವುದೊ ಅದನ್ನು ಸದಾ ಕಲಿಯಬೇಕಾಗಿಲ್ಲ.

ಅನಂತರ ಅಧ್ಯಯನದ ಬೇರೊಂದು ಭಾಗವಿದೆ. ಇದು ನಮ್ಮ ಸಾಮಾಜಿಕ ಜೀವನಕ್ಕೆ ಅಷ್ಟು ಆವಶ್ಯಕವಲ್ಲ. ಇದು ನಮ್ಮ ಮುಕ್ತಿಗೆ ಕಾರಣವಾಗುವುದು. ಅದರ ಪ್ರತ್ಯಕ್ಷ ಗುರಿಯೇ ನಮ್ಮ ಆತ್ಮನ ಬಂಧನವನ್ನು ತುಂಡರಿಸುವುದು; ಅಜ್ಞಾನದ ಕಣಿವೆಗೆ ಜ್ಞಾನಜ್ಯೋತಿಯನ್ನು ತರುವುದು, ಹಿಂದಿರುವುದನ್ನು ಶುದ್ಧ ಮಾಡುವುದು. ಅವುಗಳನ್ನು ಕದಲಿಸಿ ಸಾಧ್ಯವಾದರೆ ಅವುಗಳ ಪಾಶದಿಂದ ತಪ್ಪಿಸಿಕೊಂಡು ಹೋಗಿ, ಅಜ್ಞಾನದಿಂದ ಪಾರಾಗಿ ಮುಂದುವರಿಯುವಂತೆ ಪ್ರೇರೇಪಿಸುವುದಾಗಿದೆ. ಇದೇ ನಮ್ಮ ಗುರಿ, ಇದೇ ಸಮಾಧಿಯ ಅವಸ್ಥೆ. ಈ\break ಸ್ಥಿತಿಯನ್ನು ಪಡೆದರೆ ಈಗಿರುವ ಮನುಷ್ಯರೇ ದಿವ್ಯಾತ್ಮನಾಗುವನು, ಮುಕ್ತಾತ್ಮನಾಗುವನು. ಎಲ್ಲವನ್ನೂ ಅತಿಕ್ರಮಿಸಿ ಹೋಗುವುದಕ್ಕೆ ಅಣಿಯಾದ ಮನಸ್ಸಿಗೆ ಪ್ರಪಂಚ ಕ್ರಮೇಣ\break ತನ್ನ ರಹಸ್ಯಗಳನ್ನೆಲ್ಲಾ ನೀಡುವುದು. ಪ್ರಕೃತಿಯೆಂಬ ಪುಸ್ತಕದಲ್ಲಿ ಅಧ್ಯಾಯಗಳಾದ\break ಮೇಲೆ ಅಧ್ಯಾಯಗಳನ್ನು ಓದುತ್ತೇವೆ. ಜನನಮರಣಗಳಿಂದ ಪಾರಾಗಿ, ಜನನ ಮರಣಗಳಿಲ್ಲದ ಸ್ಥಳಕ್ಕೆ ಹೋಗುವೆವು. ಅಲ್ಲಿ ನಮಗೆ ಸತ್ಯದ ಅರಿವು ಉಂಟಾಗುವುದು. ನಾವೇ\break ಸತ್ಯವಾಗುವೆವು.

ನಮಗೆ ಮೊದಲು ಬೇಕಾಗಿರುವುದೇ ಶಾಂತವಾದ ಉದ್ವಿಗ್ನತೆಯಿಲ್ಲದ ಜೀವನ. ಜೀವನೋಪಾಯಕ್ಕಾಗಿ ಇಡೀ ದಿನ ದುಡಿಯುತ್ತಿದ್ದರೆ ನಾನು ಈ ಜೀವನದಲ್ಲಿ\break ಉತ್ತಮವಾದುದಾವುದನ್ನೂ ಸಾಧಿಸಲಾಗುವುದಿಲ್ಲ. ಬಹುಶಃ ಮತ್ತೊಂದು ಜೀವನದಲ್ಲಿ ಉತ್ತಮ ವಾತಾವರಣದಲ್ಲಿ ಹುಟ್ಟಬಹುದು. ಆದರೆ ನನ್ನಲ್ಲಿ ತೀವ್ರ ಆಸಕ್ತಿ ಇದ್ದರೆ ಈ\break ಜೀವನದಲ್ಲೇ ವಾತಾವರಣ ಬದಲಾಗುವುದು. ನಿನಗೆ ನಿಜವಾಗಿ ಬೇಕಾದುದು\break ಯಾವುದಾದರೂ ಸಿಕ್ಕದೆ ಇದೆಯೆ? ಅದು ಸಾಧ್ಯವಿಲ್ಲ. ಏಕೆಂದರೆ ಆಸೆಯೇ ದೇಹವನ್ನು ಸೃಷ್ಟಿಸುವುದು. ಹೊರಗಿನ ಬೆಳಕು ನಿನ್ನ ತಲೆಯಲ್ಲಿ ಒಂದು ರಂಧ್ರವನ್ನು ನಿರ್ಮಿಸಿದೆಯೋ ಎಂಬಂತೆ ಕಣ್ಣನ್ನು ಸೃಷ್ಟಿಸಿರುವುದು. ಬೆಳಕು ಎಂಬುದೇ ಇಲ್ಲದೇ ಇದ್ದರೆ ನಿನಗೆ ಕಣ್ಣುಗಳೇ ಇರುತ್ತಿರಲಿಲ್ಲ. ಶಬ್ದ ಕಿವಿಯನ್ನು ಸೃಷ್ಟಿಸಿದೆ. ಇಂದ್ರಿಯಗಳಿಗಿಂತ ಮುಂಚೆ ಅವಕ್ಕೆ ಸಂಬಂಧಪಟ್ಟ ವಸ್ತುಗಳಿದ್ದುವು. ಇನ್ನು ಕೆಲವು ಸಹಸ್ರಾರು ವರುಷಗಳಲ್ಲಿಯೋ ಅಥವಾ ಅದಕ್ಕಿಂತ\break ಮುಂಚೆಯೊ ವಿದ್ಯುತ್ತನ್ನು ಮತ್ತು ಇತರ ಶಕ್ತಿಗಳನ್ನು ಗ್ರಹಿಸಬಲ್ಲ ಇಂದ್ರಿಯ ನಮಗೆ\break ಬರಬಹುದು. ಶಾಂತವಾದ ಮನಸ್ಸಿಗೆ ಯಾವ ಆಸೆಯೂ ಇಲ್ಲ. ಹೊರಗೆ ನಾವು ಬಯಸುವ ವಸ್ತುಗಳು ಇಲ್ಲದಿದ್ದಾಗ ನಮಗೆ ಆಸೆ ಬರುವುದಿಲ್ಲ. ಹೊರಗಿನಿಂದ ಏನೋ ಒಂದು\break ನಮ್ಮ ದೇಹದಲ್ಲಿ ರಂಧ್ರಮಾಡಿ ಒಳಗೆ ಪ್ರವೇಶಿಸಿ ಮನಸ್ಸನ್ನು ಮುಟ್ಟುತ್ತದೆ. ಶಾಂತವಾದ ಉದ್ವಿಗ್ನತೆ ಇಲ್ಲದ ಜೀವನವನ್ನು ನೀವು ನಡೆಸಬೇಕೆಂಬ ಆಸೆ ಇದ್ದರೆ ಅದು ಸಿದ್ಧಿಸುವುದು. ಇದು ನನ್ನ ಅನುಭವ ಎಂದು ತಿಳಿಯಿರಿ. ಅದು ಸಹಸ್ರಾರು ಜನ್ಮಗಳಾದ ಮೇಲೆ ಬರಬಹುದು. ಆದರೆ ಅದು ಬರುವುದರಲ್ಲಿ ಯಾವ ಸಂದೇಹವೂ ಇಲ್ಲ. ಈ ಆಸೆಯನ್ನು ವ್ಯತ್ಯಾಸವಿದೆ ಎಂಬುದನ್ನು ನೀವು ಗಮನಿಸಬೇಕು. ಗುರು ಶಿಷ್ಯನಿಗೆ “ಮಗೂ, ನಿನಗೆ ದೇವರು ಬೇಕಾದರೆ ದೇವರು ದೊರಕುವನು” ಎಂದನು. ಶಿಷ್ಯನಿಗೆ ಇದು ಸರಿಯಾಗಿ\break ಅರ್ಥವಾಗಲಿಲ್ಲ. ಒಂದು ದಿನ ಇಬ್ಬರೂ ನದಿಗೆ ಸ್ನಾನಕ್ಕೆ ಹೋದರು. ಗುರು ಶಿಷ್ಯನಿಗೆ ಮುಳುಗು ಎಂದನು. ಶಿಷ್ಯ ಮುಳುಗಿದ. ತಕ್ಷಣ ಗುರು ತನ್ನ ಕೈಗಳನ್ನು ನೀರಿನಲ್ಲಿರುವ ಶಿಷ್ಯನ ತಲೆಯ ಮೇಲಿಟ್ಟು ಒತ್ತಿಹಿಡಿದನು. ಶಿಷ್ಯನು ಉಸಿರಿಗಾಗಿ ಹೋರಾಡಿ ಬಳಲಿದ ಮೇಲೆ ಕೈಯನ್ನು ಬಿಟ್ಟನು. ಆತ ಹೊರಗೆ ಬಂದಮೇಲೆ “ಮಗು, ನೀರಿನಲ್ಲಿ ನಿನ್ನ ಅನುಭವ ಹೇಗಿತ್ತು?” ಎಂದನು. “ಓ! ಒಂದು ಸಲ ಉಸಿರಾಡುವುದಕ್ಕೆ ಅವಕಾಶ ಸಿಕ್ಕಿದ್ದರೆ! ಎಂದು ಅನಿಸಿತು” ಎಂದ. ಆಗ ಗುರು “ನಿನಗೆ ದೇವರ ಮೇಲೆ ಇಂತಹ ಆಸೆ ಇದೆಯೇ?” ಎಂದು ಕೇಳಿದನು. ಶಿಷ್ಯ “ಇಲ್ಲ” ಎಂದ. “ದೇವರನ್ನು ಪಡೆಯಬೇಕಾದರೆ ಇಂತಹ ಆಸೆ ಇರಬೇಕು. ಆಗ ದೇವರು ದೊರಕುವನು” ಎಂದನು.

ಯಾವುದಿಲ್ಲದೆ ನಾವು ಬದುಕಲಾರೆವೊ ಅದು ನಮಗೆ ದೊರಕಲೇಬೇಕು. ಅದು ದೊರಕದಿದ್ದರೆ ಜೀವನ ಸಾಗುವುದಿಲ್ಲ.

ನೀವು ಯೋಗಿಯಾಗಬೇಕೆಂದು ಬಯಸಿದರೆ ನೀವು ಸ್ವತಂತ್ರವಾಗಿ ಉದ್ವೇಗದಿಂದ ಪಾರಾಗಿ ಏಕಾಂತವಾದ ಸ್ಥಳದಲ್ಲಿರಬೇಕು. ಯಾರು ಸುಖವಾದ ಜೀವನ ನಡೆಸಿಕೊಂಡು ಜೊತೆಗೆ ಆತ್ಮ ಸಾಕ್ಷಾತ್ಕಾರವನ್ನೂ ಪಡೆಯಬೇಕೆಂದು ಇಚ್ಛಿಸುವರೊ ಅವರು ಒಂದು ಮೊಸಳೆಯನ್ನು ಮರದ ತುಂಡೆಂದು ಭ್ರಮಿಸಿ ಅದನ್ನು ಹಿಡಿದುಕೊಂಡು ನದಿಯನ್ನು ದಾಟಲು ಯತ್ನಿಸುವ ಮೂರ್ಖನಂತೆ, “ಮೊದಲು, ದೇವರನ್ನು ಮತ್ತು ಧರ್ಮವನ್ನು ಅರಸಿ. ಅನಂತರ ಇವುಗಳೆಲ್ಲ ನಿಮಗೆ ಬರುವುವು. “ಯಾವನು ಎಲ್ಲವನ್ನೂ ನಿರ್ಲಕ್ಷ್ಯದಿಂದ ನೋಡುವನೊ ಅವನಿಗೆ ಎಲ್ಲವೂ ಬರುವುದು. ಅದೃಷ್ಟವು ಚಂಚಲ ಸ್ತ್ರೀಯಂತೆ. ಯಾರು ಬೇಕೆಂದು ಬಯಸುತ್ತಿರುವರೊ ಅವರನ್ನು ಅವಳು ಲೆಕ್ಕಿಸುವುದಿಲ್ಲ. ಆದರೆ ಯಾರು ಅವಳನ್ನು ಲೆಕ್ಕಿಸುವುದಿಲ್ಲವೊ ಅವರ ಪದತಳದಲ್ಲಿ ಅವಳು ಓಲೈಸುತ್ತಿರುವಳು. ಯಾರಿಗೆ ದ್ರವ್ಯ ಬೇಡವೊ ಅವರಿಗೆ ಬೇಕಾದಷ್ಟು ಹಣ ದೊರಕುವುದು. ಇದರಂತೆಯೇ ಕೀರ್ತಿ ಮತ್ತು ಗೌರವಗಳೂ ಕೂಡ ಸಾಕಾಗುವಷ್ಟು ಬರುತ್ತಿರುತ್ತವೆ. ಇವು ಯಾವಾಗಲೂ ಯಜಮಾನನಿಗೆ ಬರುವುವು, ಆಳಿಗೆ ಏನೂ ಬರುವುದಿಲ್ಲ. ಯಜಮಾನ ಇವುಗಳಿಲ್ಲದೆ ಜೀವಿಸಬಲ್ಲ. ಪ್ರಪಂಚದ ಕೆಲಸಕ್ಕೆ ಬಾರದ ಕೀರ್ತಿ ಗೌರವಗಳನ್ನು ಅವನು ಲೆಕ್ಕಿಸುವುದಿಲ್ಲ. ಒಂದು ಆದರ್ಶಕ್ಕಾಗಿ ಬಾಳಿ; ಅದೊಂದೇ ಆದರ್ಶವಿರಲಿ; ನಿಮ್ಮ ಜೀವನದಲ್ಲಿ ಆದರ್ಶದ ಮೇಲಿನ ಆಸೆ ಉತ್ಕಟವಾಗಿರಲಿ, ತೀವ್ರವಾಗಿರಲಿ. ಮನಸ್ಸಿನಲ್ಲಿ ಮತ್ತಾವುದಕ್ಕೂ ಎಡೆ ಇಲ್ಲದಿರಲಿ. ಮತ್ತಾವುದನ್ನೂ ಚಿಂತಿಸುವುದಕ್ಕೆ ಕಾಲವಿಲ್ಲದಿರಲಿ.

ಕೆಲವು ಜನರು ಹೇಗೆ ತಮ್ಮ ಶಕ್ತಿಯನ್ನೆಲ್ಲಾ, ಕಾಲವನ್ನೆಲ್ಲಾ, ಬುದ್ಧಿಯನ್ನೆಲ್ಲಾ,\break ದೇಹವನ್ನೆಲ್ಲ ಐಶ್ವರ್ಯ ಸಂಪಾದನೆಗೆ ನೀಡುತ್ತಾರೆ! ಅವರಿಗೆ ಉಪಾಹಾರಕ್ಕೂ ಸಮಯವಿರುವುದಿಲ್ಲ. ಬೆಳಿಗ್ಗೆ ಅಷ್ಟು ಹೊತ್ತಿಗೇ ಮನೆ ಬಿಡುವರು; ಕೆಲಸದಲ್ಲಿ ನಿರತರಾಗುವರು. ನೂರರಲ್ಲಿ ತೊಂಭತ್ತು ಜನರು ಪ್ರಯತ್ನದಲ್ಲೇ ಸಾಯುವರು. ಹಣ ಮಾಡಿದ ಉಳಿದ ಹತ್ತು ಜನರಿಗೆ ಅದನ್ನು ಅನುಭವಿಸುವುದಕ್ಕೆ ಸಮಯವಿರುವುದಿಲ್ಲ. ಇದೆಷ್ಟು ವಿಚಿತ್ರ! ಶ‍್ರೀಮಂತರಾಗುವುದಕ್ಕೆ ಪ್ರಯತ್ನ ಮಾಡುವುದು ತಪ್ಪು ಎನ್ನುವುದಿಲ್ಲ. ಇದು ಅದ್ಭುತವಾದುದು, ವಿಚಿತ್ರವಾದುದು. ಏತಕ್ಕೆ? ಇದು ಏನನ್ನು ತೋರುತ್ತದೆ? ಹಣಕ್ಕಾಗಿ ಯಾವ\break ಶಕ್ತಿಯನ್ನು ಉಪಯೋಗಿಸುವನೊ, ಎಷ್ಟು ಹೋರಾಟ ನಡೆಸುವನೊ ಅದನ್ನೇ ಮುಕ್ತನಾಗು\break ವುದಕ್ಕೂ ಉಪಯೋಗಿಸಬಹುದು ಎಂಬುದನ್ನು ತೋರುತ್ತದೆ. ನಾವು ಕಾಲವಾದಾಗ ಹಣ ಮುಂತಾದುವನ್ನೆಲ್ಲಾ ಇಲ್ಲೇ ಬಿಡಬೇಕಾಗುವುದು. ಆದರೂ ಅದನ್ನು ಪಡೆಯಲು ನಾವು ಎಷ್ಟೊಂದು ಶಕ್ತಿಯನ್ನು ವ್ಯಯಮಾಡುತ್ತೇವೆ! ಆದರೆ ಯಾವುದು ಎಂದಿಗೂ ನಾಶವಾಗುವುದಿಲ್ಲವೊ, ಎಂದೆಂದಿಗೂ ನಮ್ಮೊಡನೆ ಇರುವುದೊ, ಅದನ್ನು ಪಡೆಯುವುದಕ್ಕೆ ಸಾವಿರಪಾಲಿನಷ್ಟು ಹೆಚ್ಚು ಶಕ್ತಿಯನ್ನಾದರೂ ಉಪಯೋಗಿಸಬೇಡವೆ? ನಾವು ಕಾಲವಾದ ಮೇಲೆ ನಮ್ಮ ಹಿಂದೆ ಬರುವುದು ಇವು ಮಾತ್ರ; ಅವೇ ನಮ್ಮ ಆಧ್ಯಾತ್ಮಿಕತೆ ಮತ್ತು ಪುಣ್ಯ ಕರ್ಮಗಳು. ಅವು ಮಾತ್ರ ನಮ್ಮನ್ನು ಅಗಲದ ಪರಮ ಸ್ನೇಹಿತರು. ಉಳಿದವುಗಳನ್ನೆಲ್ಲ ಇಲ್ಲೇ ದೇಹದೊಡನೆ ಬಿಟ್ಟುಹೋಗಬೇಕಾಗಿದೆ.

ಆದರ್ಶದ ಮೇಲೆ ಅಭಿಮಾನ ಇರಬೇಕು. ಇದೇ ಮೊದಲನೆಯ ಮುಖ್ಯವಾದ ಮೆಟ್ಟಿಲು. ಅನಂತರ ಸುಲಭವಾಗುವುದು. ಇದನ್ನೊಬ್ಬ ಭಾರತೀಯ ಕಂಡುಹಿಡಿದನು. ಅಲ್ಲಿ, ಭರತಖಂಡದಲ್ಲಿ ಜನರು ಸತ್ಯಸಾಕ್ಷಾತ್ಕಾರಕ್ಕೆ ಏನನ್ನು ಬೇಕಾದರೂ ಮಾಡಲು ಸಿದ್ಧರಾಗಿರುವರು. ಆದರೆ ಇಲ್ಲಿ ಪಾಶ್ಚಾತ್ಯರುಗಳಲ್ಲಿರುವ ತೊಡಕೆಂದರೆ ಎಲ್ಲವನ್ನೂ ಸುಲಭ ಮಾಡಿಬಿಡುವುದು. ಸತ್ಯವಲ್ಲ, ಬೆಳವಣಿಗೆಯೇ ಅವರ ಪರಮಗುರಿ. ಹೋರಾಟವೇ ನಾವು ಕಲಿಯಬೇಕಾದ ದೊಡ್ಡ ಪಾಠ. ಇವನ್ನು ಗಮನಿಸಿ. ಜಗತ್ತಿನಲ್ಲಿ ಹೋರಾಟದಿಂದ ದೊಡ್ಡ ಲಾಭವಿದೆ; ನಾವು ಹೋರಾಟದ ಮೂಲಕ ಮುಂದುವರಿಯಬೇಕಾಗಿದೆ. ಸ್ವರ್ಗಕ್ಕೆ ರಸ್ತೆ ಇದ್ದರೆ ಅದು ನರಕದ ಮೂಲಕವೇ. ನರಕದ ಮೂಲಕವೇ ಯಾವಾಗಲೂ ಸ್ವರ್ಗಕ್ಕೆ ರಸ್ತೆ\break ಇರುವುದು. ಆತ್ಮ ಹೊರಗಿನ ಸನ್ನಿವೇಶಗಳೊಡನೆ ಹೋರಾಡಿ ಮೃತ್ಯುವಶವಾದಾಗ, ಸಾವಿರಾರು ವೇಳೆ ಮೃತ್ಯುವಶವಾದ ಮೇಲೆಯೂ ನೆಚ್ಚುಗೆಡದೆ ಪುನಃ ಹೋರಾಡುವುದು. ಕೊನೆಗೆ ಹೋರಾಟದಿಂದ ಅದ್ಭುತ ಶಕ್ತಿಯನ್ನು ಗಳಿಸಿ ತಾನು ಯಾವ ಆದರ್ಶಕ್ಕಾಗಿ ಹೋರಾಡುತ್ತಿತ್ತೋ ಅದನ್ನು ನೋಡಿ ನಗುವುದು. ಆಗ ಅದು ತಾನು ಯಾವ ಆದರ್ಶಕ್ಕಾಗಿ ಹೋರಾಡುತ್ತಿತ್ತು ಅದಕ್ಕಿಂತ ತಾನು ಎಷ್ಟುಪಾಲು ಮೇಲು ಎಂಬುದನ್ನು ಅರಿಯುವುದು. ತನ್ನ ಆತ್ಮವೇ ಪರಮಾವಧಿ, ಮತ್ತಾವುದೂ ಅಲ್ಲ. ನನ್ನ ಆತ್ಮದೊಡನೆ ಹೋಲಿಸುವುದಕ್ಕೆ\break ಮತ್ತೇನಿದೆ? ಚಿನ್ನದ ಕೊಪ್ಪರಿಗೆ ನನ್ನ ಜೀವನದ ಆದರ್ಶವಾಗಬಲ್ಲುದೆ? ನಿಜವಾಗಿ ಎಂದಿಗೂ ಇಲ್ಲ. ನನ್ನ ಆತ್ಮವೇ ನಾನು ಭಾವಿಸಬಲ್ಲ ಪರಮ ಆದರ್ಶ. ನನ್ನ ನೈಜಸ್ವಭಾವವನ್ನು\break ಅರಿಯುವುದೇ ನನ್ನ ಜೀವನದ ಏಕಮಾತ್ರ ಗುರಿ.

\vskip 0.1cm

ಈ ಪ್ರಪಂಚದಲ್ಲಿ ಬರಿಯ ಪಾಪವೆಂಬುದು ಯಾವುದೂ ಇಲ್ಲ. ದೇವರಿಗೆ\break ಹೇಗೆ ಒಂದು ಸ್ಥಳವಿದೆಯೋ ಹಾಗೆಯೇ ಸೈತಾನನಿಗೂ ಒಂದು ಸ್ಥಳವಿದೆ. ಇಲ್ಲದೇ ಇದ್ದರೆ ಅವನು ಇಲ್ಲಿ ಇರುತ್ತಿರಲಿಲ್ಲ. ನಾನು ಈಗ ತಾನೆ ಹೇಳಿದಂತೆ ನರಕದ ಮೂಲಕ ಸ್ವರ್ಗಕ್ಕೆ ಹೋಗಬೇಕಾಗಿದೆ. ನಮ್ಮ ತಪ್ಪಿಗೂ ಒಂದು ಸ್ಥಳವಿದೆ. ನುಗ್ಗಿ ನಡೆಯಿರಿ ಮುಂದೆ. ಯಾವುದೋ ತಪ್ಪನ್ನು ಮಾಡಿರುವೆ ಎಂದು ಹಿಂತಿರುಗಿ ನೋಡದಿರಿ. ನೀವು ಆ ತಪ್ಪುಗಳನ್ನೆಲ್ಲಾ ಮಾಡದೇ ಇದ್ದಿದ್ದರೆ ಈಗಿನಂತೆ ಇರುತ್ತಿದ್ದಿರಿ ಎಂದು ಭಾವಿಸಿದಿರೇನು? ಹಾಗಾದರೆ ನಿಮ್ಮ ತಪ್ಪುಗಳಿಗೆ ಕೃತಜ್ಞರಾಗಿರಿ. ಅವು ನಿಮಗೆ ಅರಿಯದ ದೇವದೂತರಂತೆ ಇದ್ದವು. ಹಿಂಸೆಗೆ ಧನ್ಯವಾದ! ಸಂತೋಷಕ್ಕೂ ಧನ್ಯವಾದ! ನಿಮ್ಮ ಪಾಲಿಗೆ ಏನು ಬರುವುದೋ ಅದನ್ನು ಕುರಿತು ಚಿಂತಿಸಬೇಡಿ. ಆದರ್ಶವನ್ನು ಬಿಡಬೇಡಿ. ಹೋರಾಟ ಮಾಡಿದ ಸಣ್ಣ ಸಣ್ಣ ತಪ್ಪುಗಳ ಕಡೆಗೇ ಪುನಃ ಪುನಃ ನೋಡುತ್ತಿರಬೇಡಿ. ಜೀವನದ ಸಮರಾಂಗಣದಲ್ಲಿ ತಪ್ಪೆಂಬ ಧೂಳು ಏಳುವುದು ಸಹಜ. ಯಾರ ದೇಹವು ಈ ಧೂಳನ್ನು ಸಹಿಸದಷ್ಟು ದುರ್ಬಲವೊ ಅವರು ಹೋರಾಟವನ್ನು ಬಿಟ್ಟು ಹೊರಡಲಿ.

\vskip 0.1cm

ಹೋರಾಡಬೇಕೆಂಬ ಈ ಒಂದು ವಜ್ರಸಂಕಲ್ಪ ಈ ಜಗತ್ತಿನ ವಸ್ತುಗಳನ್ನು ಪಡೆಯಬೇಕೆಂಬ ಸಂಕಲ್ಪಕ್ಕಿಂತ ನೂರುಪಾಲು ಹೆಚ್ಚು ಅಧಿಕವಾಗಿರಬೇಕು. ಇದೇ ಪ್ರಥಮ\break ಸಿದ್ಧತೆ.

\vskip 0.1cm

ಇದರೊಡನೆ ಧ್ಯಾನವೂ ಇರಬೇಕು. ಧ್ಯಾನ ಅತ್ಯಾವಶ್ಯಕ. ಧ್ಯಾನಮಾಡಿ. ಧ್ಯಾನವೇ ಅತಿ ಶ್ರೇಷ್ಠವಾದ ಸಾಧನೆ. ಧ್ಯಾನಾವಸ್ಥೆ ಆಧ್ಯಾತ್ಮಿಕ ಜೀವನಕ್ಕೆ ಅತಿ ಸಮೀಪದ ಮಾರ್ಗ. ಈ ಧ್ಯಾನದ ಕಾಲದಲ್ಲಿ ಮಾತ್ರವೇ ನಮ್ಮ ಮನಸ್ಸು ಪ್ರಾಪಂಚಿಕವಾಗಿರುವುದಿಲ್ಲ. ಆಗ ಆತ್ಮವು ಪ್ರಪಂಚದಿಂದ ಪಾರಾಗಿ ತನ್ನ ನೈಜಸ್ಥಿತಿಯನ್ನು ಕುರಿತು ಯೋಚಿಸುತ್ತಿರುವುದು. ಧ್ಯಾನಾವಸ್ಥೆಯಲ್ಲಿ ಆತ್ಮನ ಒಂದು ಅಪೂರ್ವ ಸ್ಪರ್ಶವಾಗುತ್ತದೆ.

\eject

ದೇಹವೇ ನಮ್ಮ ಶತ್ರು. ಆದರೂ ಅದೇ ನಮ್ಮ ಮಿತ್ರ. ನಿಮ್ಮಲ್ಲಿ ಯಾರು ದುಃಖವನ್ನು ಸಹಿಸಬಲ್ಲಿರಿ? ಆದರೆ ಅದನ್ನು ಚಿತ್ರದಲ್ಲಿ ನೋಡಿದಾಗ ಯಾರು ಅದನ್ನು ಸಹಿಸಲಾರರು? ಅದು ಅಸತ್ಯ. ಆದಕಾರಣ ನಾವು ಅದರೊಂದಿಗೆ ತಾದಾತ್ಮ್ಯ ಭಾವವನ್ನು ಹೊಂದುವುದಿಲ್ಲ. ಅದನ್ನು ಕೇವಲ ಒಂದು ಚಿತ್ರವೆಂದು ನೋಡುತ್ತೇವೆ. ಅದು ನಮ್ಮನ್ನು ಆಶೀರ್ವದಿಸಲಾರದು. ಅದು ನಮಗೆ ವ್ಯಥೆಯನ್ನು ಕೊಡಲಾರದು. ಕ್ಯಾನ್​ವಾಸಿನ ಮೇಲೆ ಬರೆದ ಅತಿ ರೌದ್ರವಾದ ಚಿತ್ರವನ್ನು ಕೂಡ ನೋಡಿ ಆನಂದಿಸುವೆವು. ಆ ಚಿತ್ರಕಾರನ ಕುಶಲತೆಯನ್ನು ಹೊಗಳುವೆವು. ಅವನು ಬರೆದಿರುವುದು ದುಃಖಮಯವಾದ ಚಿತ್ರವಾದರೂ ಅವನ ಅದ್ಭುತ ಪ್ರತಿಭೆಗೆ ಆಶ್ಚರ್ಯಪಡುವೆವು. ಅನಾಸಕ್ತಿಯೇ ಇದರ ರಹಸ್ಯ. ಆದ್ದರಿಂದ ಕೇವಲ ಸಾಕ್ಷಿಯಾಗಿರಿ.

\vskip 0.1cm

ನಾನು ಕೇವಲ ಸಾಕ್ಷಿ ಎಂದು ಅರಿಯುವವರೆಗೆ ಯಾವ ಪ್ರಾಣಾಯಾಮದಿಂದಲೂ ಆಸನದಿಂದಲೂ ಪ್ರಯೋಜನವಿಲ್ಲ. ಹಿಂಸಕನು ನಿನ್ನ ಕೊರಳನ್ನು ಹಿಸುಕುತ್ತಿರುವಾಗ “ನಾನು ಕೇವಲ ಸಾಕ್ಷಿ, ನಾನು ಆತ್ಮ; ಯಾವ ಬಾಹ್ಯ ವಸ್ತುವೂ ನನ್ನನ್ನು ಸ್ಪರ್ಶಿಸಲಾರದು” ಎನ್ನಿ. ಹೀಗೆ ಆಲೋಚನೆಗಳು ಮನಸ್ಸಿನಲ್ಲಿ ಎದ್ದಾಗ ಚಾವಟಿಯ ಪೆಟ್ಟನ್ನು ಕೊಟ್ಟು ಓಡಿಸಿ. “ನಾನು ಆತ್ಮ, ನಿತ್ಯ ಧನ್ಯನಾದ ಸಾಕ್ಷಿ. ನಾನು ಇದನ್ನು ಮಾಡುವುದಕ್ಕೆ ಕಾರಣವಿಲ್ಲ; ಅನುಭವಿಸುವು\break ದಕ್ಕೆ ಕಾರಣವಿಲ್ಲ. ನಾನು ಇವುಗಳನ್ನೆಲ್ಲಾ ಮುಗಿಸಿರುವೆನು. ನಾನು ನಿತ್ಯ ಸಾಕ್ಷಿ. ನಾನು ನನ್ನ\break ಚಿತ್ರಶಾಲೆಯಲ್ಲಿರುವೆನು. ಈ ಪ್ರಪಂಚ ನನ್ನ ವಸ್ತುಪ್ರದರ್ಶನ ಶಾಲೆ. ನಾನು ಈ ಚಿತ್ರಗಳ\break ನ್ನೆಲ್ಲಾ ನೋಡುತ್ತಿರುವೆನು. ಅವೆಲ್ಲ ಒಳ್ಳೆಯವಾಗಲಿ, ಕೆಟ್ಟವಾಗಲಿ ಚೆನ್ನಾಗಿವೆ. ನಾನು ಚಿತ್ರದ\break ನೈಪುಣ್ಯಕ್ಕೆ ತಲೆದೂಗುತ್ತಿರುವೆನು. ಆದರೆ ಇವೆಲ್ಲಾ ಒಂದು. ಆ ಸನಾತನ ಕಲಾಕೋವಿದನ ಅನಂತಸ್ಫೂರ್ತಿಯಿಂದ ಹೊರಹೊಮ್ಮಿದ ಸ್ತೋತ್ರಗಳು!” ನಿಜವಾಗಿ ಹೇಳುವುದಾದರೆ ಯಾವುದೂ ಇಲ್ಲ. ಇಚ್ಛೆಯೂ ಇಲ್ಲ. ಆಸೆಯೂ ಇಲ್ಲ; ಅವನೇ ಸರ್ವವೂ ಆಗಿರುವನು. ಆ ಜಗನ್ಮಾತೆ ಆಡುತ್ತಿರುವಳು. ನಾವೆಲ್ಲ ಆಕೆಯ ಆಟದ ಸಾಮಾನುಗಳು. ಆಕೆಯ ಆಟದಲ್ಲಿ ಭಾಗಿಗಳು. ಇಲ್ಲಿ ಒಬ್ಬನನ್ನು ಭಿಕ್ಷುಕನ ಪಾತ್ರದಲ್ಲಿಡುವಳು. ಮತ್ತೊಬ್ಬನನ್ನು ರಾಜನ ಪೋಷಾಕಿನಲ್ಲಿಡುವಳು. ಒಂದು ಸಲ ಸಾಧು ವೇಷವನ್ನು ಹಾಕುವಳು. ಮತ್ತೊಂದು ಸಲ ದುರಾತ್ಮನ ವೇಷ ಹಾಕುವಳು. ಜಗನ್ಮಯಿಯ ಆಟದಲ್ಲಿ ನೆರವಾಗಲು ನಾವು ಹಲವು ವೇಷಗಳನ್ನು ಧರಿಸುತ್ತಿರುವೆವು.

\vskip 0.1cm

ಮಗು ಆಡುತ್ತಿರುವಾಗ ಕರೆದರೂ ಬರುವುದಿಲ್ಲ. ಆಟ ಮಗಿದ ಮೇಲೆ ತಾಯಿಯ ಹತ್ತಿರ ಓಡುವುದು. ಆಗ ಅವಳು ಬರಬೇಡ ಎಂದರೂ ಕೇಳುವುದಿಲ್ಲ. ನಾವು ಜೀವನದಲ್ಲಿ\break ನಮ್ಮ ಆಟ ಮುಗಿಯಿತು ಎನ್ನುವ ಸಮಯ ಬರುವುದು. ಆಗ ನಾವು ತಾಯಿಯೆಡೆಗೆ\break ಹೋಗಲು ಕಾತರರಾಗುವೆವು. ಆಗ ನಾವು ಪಟ್ಟ ಪ್ರಯತ್ನ-ಗಂಡ ಹೆಂಡತಿ ಮಕ್ಕಳು ಐಶ್ವರ್ಯ ಕೀರ್ತಿ ಯಶಸ್ಸು ಈ ಜೀವನದ ಸುಖ ಮತ್ತು ವೈಭವಗಳು ಶಿಕ್ಷೆ ಮತ್ತು ಜಯ\break ಗಳು-ಇವುಗಳಾವುವೂ ಇರುವುದಿಲ್ಲ. ಜೀವನವೇ ಒಂದು ನಾಟಕದಂತೆ ಕಾಣಿಸುವುದು. ತುದಿಮೊದಲಿಲ್ಲದ, ಗೊತ್ತುಗುರಿಯಿಲ್ಲದ ಒಂದು ಅನಂತ ನೃತ್ಯವಾಗುತ್ತಿರುವಂತೆ ಕಾಣಿಸುವುದು. ನಮ್ಮ ಆಟ ಮುಗಿಯಿತು ಎಂದು ಆಗ ಹೇಳುವೆವು.

