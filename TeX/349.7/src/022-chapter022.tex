
\chapter[ಹಿಂದೂಧರ್ಮ ]{ಹಿಂದೂಧರ್ಮ \protect\footnote{\engfoot{C.W. Vol. I, P. 329}}}

\centerline{(೧೮೯೪ರ ಡಿಸಂಬರ್​ ೩೦ರಂದು ಬೂಕಲಿನ್​ನಲ್ಲಿ ನೀಡಿದ ಉಪನ್ಯಾಸದ ಸಾರಾಂಶ)}

ಕಲಿಯುವುದೇ ನನ್ನ ಧರ್ಮ. ನಿಮ್ಮ ಬೈಬಲ್ಲಿನ್ ಬೆಳಕಿನಲ್ಲಿ ನಮ್ಮ ಶಾಸ್ತ್ರಗಳನ್ನು ಚೆನ್ನಾಗಿ ಅರ್ಥಮಾಡಿಕೊಳ್ಳಬಲ್ಲೆ. ನಮ್ಮ ಧರ್ಮದ ಅಸ್ಪಷ್ಟವಾದ ಭವಿಷ್ಯ ವಾಣಿಗಳನ್ನು ನಿಮ್ಮ ಧರ್ಮದ ವಾಣಿಗಳೊಂದಿಗೆ ಹೋಲಿಸಿದಾಗ ಅವು ಹೆಚ್ಚು ಸ್ಪಷ್ಟವಾಗುವುವು. ಸತ್ಯವು ಯಾವಾಗಲೂ ಸಾರ್ವತ್ರಿಕವಾದುದು. ನಿಮ್ಮ ಕೈಗಳಲ್ಲಿ ಐದು ಬೆರಳುಗಳಿದ್ದು ನನ್ನ ಕೈಯಲ್ಲಿ ಮಾತ್ರ ಆರು ಬೆರಳುಗಳಿದ್ದರೆ ಪ್ರಕೃತಿಯ ಉದ್ದೇಶ ಆರು ಬೆರಳೆಂದು ನೀವು ಭಾವಿಸುವುದಿಲ್ಲ, ಅದು ಅಸ್ವಾಭಾವಿಕವಾದುದು, ಅದೊಂದು ರೋಗ ಎನ್ನುವಿರಿ. ಇದರಂತೆಯೇ ಧರ್ಮ ಕೂಡ. ಒಂದು ಧರ್ಮ ಮಾತ್ರ ಸತ್ಯವಾಗಿದ್ದು ಉಳಿದ ಧರ್ಮಗಳೆಲ್ಲ ಅಸತ್ಯವಾಗಿದ್ದರೆ ಆ ಧರ್ಮ ಅನಾರೋಗ್ಯಕರವಾದದ್ದು ಎಂದು ಹೇಳಲು ನಿಮಗೆ ಅಧಿಕಾರವಿದೆ. ಒಂದು ಧರ್ಮ ಸತ್ಯವಾಗಿದ್ದರೆ ಎಲ್ಲಾ ಧರ್ಮಗಳೂ ಸತ್ಯವಾಗಿರಬೇಕು. ಆ ದೃಷ್ಟಿಯಿಂದ ಹಿಂದೂ ಧರ್ಮ ನನಗೆ ಸೇರಿದಷ್ಟೇ ನಿಮಗೆ ಸೇರಿದೆ. ಭರತಖಂಡದಲ್ಲಿರುವ ಇಪ್ಪತ್ತೊಂಬತ್ತು ಕೋಟಿ ಜನರಲ್ಲಿ ಎರಡು ಕೋಟಿ ಜನರು ಮಾತ್ರ ಕ್ರೈಸ್ತರು, ಆರು ಕೋಟಿ ಜನರು ಮಹಮ್ಮದೀಯರು, ಉಳಿದವರೆಲ್ಲ ಹಿಂದೂಗಳು.

ಹಿಂದೂಗಳು ಪುರಾತನ ವೇದಗಳ ಆಧಾರದ ಮೇಲೆ ತಮ್ಮ ಧರ್ಮವನ್ನು ಕಟ್ಟಿರುವರು. ವಿದ್​ ಅಂದರೆ ತಿಳಿ ಎಂಬ ಪದದಿಂದ ವೇದ ಎಂಬುದು ಬಂದಿದೆ. ಇವುಗಳಲ್ಲಿ ಹಲವು ಗ್ರಂಥಗಳಿವೆ. ನಮ್ಮ ದೃಷ್ಟಿಯಲ್ಲಿ ಇವುಗಳಲ್ಲಿ ಧರ್ಮದ ಸಾರವೆಲ್ಲ ಇದೆ. ಆದರೆ ಸತ್ಯ ಇವುಗಳಲ್ಲಿ ಮಾತ್ರ ಇದೆ ಎಂದು ನಾವು ಭಾವಿಸುವುದಿಲ್ಲ. ಅವು ಆತ್ಮನ ಅಮರತ್ವವನ್ನು ಬೋಧಿಸುತ್ತವೆ. ಪ್ರತಿಯೊಂದು ದೇಶದಲ್ಲಿಯೂ ಪ್ರತಿಯೊಂದು ವ್ಯಕ್ತಿಯಲ್ಲಿಯೂ ಬದಲಾಗದ ನೆಲೆಯನ್ನು ಕಂಡುಹಿಡಿಯಬೇಕೆಂಬ ಆಸೆ ಸ್ವಾಭಾವಿಕವಾಗಿ ಇದೆ. ನಮಗೆ ಇದು ಪ್ರಕೃತಿಯಲ್ಲಿ ದೊರಕುವುದಿಲ್ಲ. ಪ್ರಕೃತಿಯಲ್ಲಿ ತುದಿ ಮೊದಲಿಲ್ಲದ ಬದಲಾವಣೆಗಳಲ್ಲದೆ ಮತ್ತೇನೂ ಇಲ್ಲ. ಆದರೆ ಇದರಿಂದ ಬದಲಾಗದ ಇರುವುದು ಯಾವುದೂ ಇಲ್ಲ ಎಂದು ನಿರ್ಧರಿಸಿದರೆ, ಬೌದ್ಧ ಥೇರಾವಾದಿಗಳು ಮತ್ತು ಚಾರ್ವಾಕರು ಮಾಡಿದ ತಪ್ಪನ್ನೇ ನಾವೂ ಮಾಡಿದಂತೆ ಆಗುತ್ತದೆ. ಚಾರ್ವಾಕರು ಇರುವುದೆಲ್ಲ ದ್ರವ್ಯ \enginline{(matter)}, ಮನಸ್ಸೆಂಬುವುದಿಲ್ಲ; ಧರ್ಮವೆಲ್ಲ ಒಂದು ಮೋಸ, ನೀತಿ ಒಳ್ಳೆಯತನ ಇವೆಲ್ಲ ಕೇವಲ ಕೆಲಸಕ್ಕೆ ಬಾರದ ಮೌಢ್ಯ ಎನ್ನುವರು. ವೇದಾಂತವು ಮಾನವನು ತನ್ನ ಪಂಚೇಂದ್ರಿಯಗಳಲ್ಲಿ ಬಂಧಿಯಲ್ಲ ಎಂಬುದನ್ನು ಬೋಧಿಸುವುದು. ಚಾರ್ವಾಕ ರಿಗೆ ಕೇವಲ ಈಗಿರುವುದು ಮಾತ್ರಗೊತ್ತಿದೆ. ಅವರಿಗೆ ಹಿಂದಿನದೂ ಗೊತ್ತಿಲ್ಲ, ಮುಂದಿನದೂ ಗೊತ್ತಿಲ್ಲ. ಆದರೆ ವರ್ತಮಾನವಿದ್ದರೆ ಭೂತ ಭವಿಷ್ಯತ್ತುಗಳೂ ಇರಬೇಕು. ಇವು ಮೂರೂ ಕಾಲಗಣತಿಗೆ ಆವಶ್ಯಕ. ಇಂದ್ರಿಯಗಳನ್ನು ಮೀರಿದ, ಕಾಲವನ್ನು ಅವಲಂಬಿಸದ ಮತ್ತು ಭೂತ ಭವಿಷ್ಯತ್ತುಗಳನ್ನು ವರ್ತಮಾನದಲ್ಲಿ ಒಂದುಗೂಡಿಸದ ಯಾವುದೋ ಒಂದು ಇಲ್ಲದಿದ್ದರೆ ವರ್ತಮಾನ ಎಂಬುದನ್ನು ತಿಳಿಯುವುದಕ್ಕೆ ಆಗುತ್ತಿರಲಿಲ್ಲ.

ಹಾಗಾದರೆ ಸ್ವತಂತ್ರವಾಗಿರುವುದು ಯಾವುದು? ಅದು ನಮ್ಮ ದೇಹವಲ್ಲ, ಏಕೆಂದರೆ ಅದು ಬಾಹ್ಯ ಸ್ಥಿತಿಗತಿಗಳನ್ನು ಅವಲಂಬಿಸಿದೆ. ಅದು ನಮ್ಮ ಮನಸ್ಸು ಕೂಡ ಅಲ್ಲ, ಏಕೆಂದರೆ ಅದರ ಆಲೋಚನೆಗಳೆಲ್ಲ ಬೇರೊಂದರ ಕಾರ್ಯ ವಾಗಿದೆ. ಅದೇ ನಮ್ಮ ಆತ್ಮ. ಈ ವಿಶ್ವವೆಲ್ಲ ಸ್ವತಂತ್ರವಾದುದರ ಮತ್ತು ಪರ ತಂತ್ರವಾದುದರ ಒಂದು ಮಿಶ್ರಣ, ಸ್ವೇಚ್ಛೆಯ ಮತ್ತು ದಾಸ್ಯದ ಒಂದು ಮಿಶ್ರಣ. ಆದರೆ ಅವುಗಳ ಮೂಲಕ ಸ್ವತಂತ್ರವಾದ ಅಮೃತವಾದ ಪರಿಶುದ್ಧವಾದ ಪರಿ ಪೂರ್ಣವಾದ ಪವಿತ್ರವಾದ ಆತ್ಮ ಕಂಗೊಳಿಸುತ್ತಿರುವುದು ಎಂದು ವೇದಗಳು ಸಾರುತ್ತವೆ. ಅದು ಸ್ವತಂತ್ರವಾಗಿದ್ದರೆ ನಾಶವಾಗಲಾರದು. ಏಕೆಂದರೆ ಸಾವು ಒಂದು ಬದಲಾವಣೆ; ಅದು ಬಾಹ್ಯ ಸನ್ನಿವೇಶಗಳಿಗೆ ಅಧೀನ. ಅದು ಸ್ವತಂತ್ರ ವಾಗಿದ್ದರೆ ಪರಿಪೂರ್ಣವಾಗಿರಬೇಕು. ಏಕೆಂದರೆ ಅಪೂರ್ಣವು ನಿಯಮಬದ್ಧ, ಆದುದರಿಂದ ಅದು ಪರಾಧೀನ, ಈ ಅಮೃತವಾದ ಪರಿಪೂರ್ಣವಾದ ಆತ್ಮವು ಪವಿತ್ರಾತ್ಮನಲ್ಲಿ ಮತ್ತು ಅತಿ ಸಾಮಾನ್ಯನಲ್ಲಿಯೂ ಒಂದೇ ಆಗಿರಬೇಕು. ವ್ಯತ್ಯಾಸ ಇರುವುದು ಆತ್ಮನ ಆವಿರ್ಭಾವದ ತರತಮದಲ್ಲಿ ಮಾತ್ರ.

ಆದರೆ ಆತ್ಮ ಏತಕ್ಕೆ ಒಂದು ದೇಹವನ್ನು ಸ್ವೀಕರಿಸಬೇಕು? ನಾನು ನನ್ನ ಮುಖ ನೋಡಿಕೊಳ್ಳುವುದಕ್ಕೆ ಒಂದು ಕನ್ನಡಿಯನ್ನು ತೆಗೆದುಕೊಳ್ಳುವಂತೆಯೇ ಅದು ಒಂದು ದೇಹವನ್ನು ತೆಗೆದುಕೊಳ್ಳುವುದು. ದೇಹದಲ್ಲಿ ಆತ್ಮವು ಪ್ರತಿ ಬಿಂಬಿತವಾಗುವುದು. ಆತ್ಮನೇ ದೇವರು. ಪ್ರತಿ ಮಾನವನಲ್ಲಿ ಕೂಡ ದಿವ್ಯತೆ ಸುಪ್ತವಾಗಿದೆ. ಪ್ರತಿಯೊಬ್ಬರೂ ತಮ್ಮ ದಿವ್ಯತೆಯನ್ನು ಈಗಲೊ ಅನಂತರವೊ ಪ್ರಕಟಗೊಳಿಸಬೇಕಾಗಿದೆ. ನಾನು ಒಂದು ಕತ್ತಲೆ ಕೋಣೆಯಲ್ಲಿದ್ದರೆ ಎಷ್ಟು ಕೂಗಾಡಿದರೂ ಬೆಳಕು ಬರುವುದಿಲ್ಲ. ನಾನು ಒಂದು ದೀಪದ ಕಡ್ಡಿಯನ್ನು ಗೀರ ಬೇಕು. ಇದರಂತೆಯೇ ತಮ್ಮ ಅಪೂರ್ಣವಾದ ದೇಹವನ್ನು ಬರಿಯ ಗದ್ದಲ ಮಾಡುವುದರಿಂದ, ಗೊಣಗಾಡುವುದರಿಂದ ಪೂರ್ಣಮಾಡಲಾಗುವುದಿಲ್ಲ. ಆದರೆ ವೇದಾಂತವು: ನಿಮ್ಮ ಆತ್ಮವನ್ನು ಪ್ರಕಟಗೊಳಿಸಿ, ನಿಮ್ಮ ಪವಿತ್ರತೆಯನ್ನು ವ್ಯಕ್ತಗೊಳಿಸಿ ಎಂದು ಬೋಧಿಸುತ್ತದೆ. ನಿಮ್ಮ ಮಕ್ಕಳಿಗೆ ಅವರು ಪವಿತ್ರಾತ್ಮರು ಎಂದು ಬೋಧಿಸಿ. ಧರ್ಮ, ಸತ್ಯವಾಗಿರುವ ಯಾವುದೋ ಒಂದನ್ನು ಬೋಧಿಸುವುದು, ನಿಷೇಧಾತ್ಮಕವಾದುದನ್ನು ಬೋಧಿಸುವುದಿಲ್ಲ. ಧರ್ಮ ಎಂದರೆ ದಬ್ಬಾಳಿಕೆಗೆ ತುತ್ತಾಗಿ ನರಳಾಡುವುದಲ್ಲ. ಆದರೆ ಸ್ವಭಾವ ವಿಸ್ತಾರವಾಗುವುದು, ಮತ್ತು ಆತ್ಮ ಶಕ್ತಿಯನ್ನು ವ್ಯಕ್ತಗೊಳಿಸುವುದು.

ಪ್ರತಿಯೊಂದು ಧರ್ಮವೂ, ಮನುಷ್ಯನ ವರ್ತಮಾನದ ಮತ್ತು ಭವಿಷ್ಯದ ಸ್ಥಿತಿಗಳು ಅವನ ಭೂತಕಾಲದಿಂದ ನಿರ್ಧರಿಸಲ್ಪಡುತ್ತವೆ, ವರ್ತಮಾನವು ಭೂತಕಾಲದ ಬರಿಯ ಒಂದು ಪರಿಣಾಮ ಎಂಬುದನ್ನು ಒಪ್ಪುತ್ತದೆ. ಹಾಗಾದರೆ ಪ್ರತಿಯೊಂದು ಮಗುವೂ ಆನುವಂಶಿಕತೆಯಿಂದ ವಿವರಿಸಲಾಗದ ಸಂಸ್ಕಾರಗಳನ್ನು ಹೊಂದಿ ಜನಿಸಿರುವುದು ಹೇಗೆ? ಒಂದು ಮಗು ಒಳ್ಳೆಯ ತಾಯಿತಂದೆಗಳಿಗೆ ಹುಟ್ಟುತ್ತದೆ, ಒಳ್ಳೆಯ ವಿದ್ಯಾಭ್ಯಾಸ ಪಡೆದು ಯೋಗ್ಯವಾಗುತ್ತದೆ. ಇದು ಏತಕ್ಕೆ? ಮತ್ತೊಂದು ಮಗು ಯೋಗ್ಯ ಮಾತಾಪಿತೃಗಳಿಗೆ ಜನಿಸಿಯೂ ಕೊಲೆಪಾತಕಿ ಯಂತೆ ಗಲ್ಲಿಗೆ ಏರಬೇಕಾಗುವುದು ಏತಕ್ಕೆ? ದೇವರನ್ನು ಇದಕ್ಕೆ ಹೊಣೆಮಾಡದೆ ಇದೆಲ್ಲವನ್ನು ಹೇಗೆ ವಿವರಿಸುತ್ತೀರಿ? ಆ ದಯಾಮಯನಾದ ತಂದೆ ಮಕ್ಕಳನ್ನು ಇಂತಹ ದುಃಖಕಾರಕ ಸ್ಥಿತಿಯಲ್ಲಿ ಏತಕ್ಕೆ ಬಿಡಬೇಕು? ಅನಂತರ ದೇವರು ಇದಕ್ಕೆಲ್ಲ ಪರಿಹಾರ ನೀಡುವನು ಎಂದರೆ ಅದು ವಿವರಿಸಿದಂತೆ ಆಗಲಿಲ್ಲ. ದೇವರ ಬಳಿ ‘ರಕ್ತದಂಡ’ (ಎಂದರೆ ಹತ್ಯಾ ದೋಷ ಪರಿಹಾರಕ್ಕೆಂದು ಹತನಾದ ವನ ಸಂಬಂಧಿಗೆ ತೆರುವ ಹಣ) ಇಲ್ಲ. ಇದೇ ನನ್ನ ಪ್ರಥಮ ಜನ್ಮವಾದರೆ ನನ್ನ ಸ್ವಾತಂತ್ರ್ಯವೇನಾಗಬೇಕು? ಹಿಂದಿನ ಜನ್ಮದ ಅನುಭವವಿಲ್ಲದೆ ಈ ಪ್ರಪಂಚಕ್ಕೆ ಬರುವುದಾದರೆ ಮತ್ತು ಇತರರ ಅನುಭವದ ಮೇಲೆ ನನ್ನ ಮಾರ್ಗವನ್ನು ನಿಷ್ಕರ್ಷಿಸಿದರೆ ನನಗೆ ಸ್ವಾತಂತ್ರ್ಯವೇ ಇರುವುದಿಲ್ಲ. ನನ್ನ ಅದೃಷ್ಟಕ್ಕೆ ನಾನೇ ಹೊಣೆಗಾರನಲ್ಲದೇ ಇದ್ದರೆ ನಾನು ಸ್ವತಂತ್ರನಲ್ಲ. ಈಗಿನ ಪಾಪಕ್ಕೆ ನಾನೇ ಕಾರಣ ಎಂದು ಅದರ ಜವಾಬ್ದಾರಿಯನ್ನು ತೆಗೆದುಕೊಂಡು ಮುಂದೆ ಅದಾಗದಂತೆ ನಾನು ನೋಡಿಕೊಳ್ಳುತ್ತೇನೆ. ಇದೇ ನಮ್ಮ ಪುನರ್ಜನ್ಮದ ತತ್ತ್ವ. ನಾವು ಈ ಜನ್ಮಕ್ಕೆ ಹಿಂದಿನ ಅನುಭವದೊಂದಿಗೆ ಬರುತ್ತೇವೆ. ಈಗಿನ ಒಳ್ಳೆಯದಕ್ಕೆ ಮತ್ತು ಕೆಟ್ಟದ್ದಕ್ಕೆಲ್ಲಾ ನಾನು ಹಿಂದೆ ಮಾಡಿದ್ದೇ ಕಾರಣ. ನಾನು ಬರುಬರುತ್ತಾ ಮೇಲಾಗುತ್ತಾ, ಪೂರ್ಣನಾಗುವವರೆಗೆ ಹೀಗೆಯೇ ಮುಂದುವರಿಯುತ್ತಿರುವೆನು.

ಈ ವಿಶ್ವದ ಪಿತನಾದ, ಸರ್ವಜ್ಞನಾದ ಅನಂತವೂ ಸರ್ವಶಕ್ತನೂ ಆದ ದೇವರನ್ನು ನಾವು ನಂಬುತ್ತೇವೆ. ನಮ್ಮ ಆತ್ಮವು ಪೂರ್ಣವೆಂದಾದರೆ ಅದು ಅನಂತವೂ ಆಗಿರಬೇಕು. ಆದರೆ ಎರಡು ನಿರಪೇಕ್ಷವಾದ ಅನಂತಗಳಿಗೆ ಎಡೆಯಿಲ್ಲ. ಆದಕಾರಣ ನಾವೇ ಆ ದೇವರೆಂದು ನಂಬುತ್ತೇವೆ. ಇವೇ ಪ್ರತಿಯೊಂದು ಧರ್ಮವೂ ಏರಿ ಬಂದ ಮೂರು ಹಂತಗಳು. ಮೊದಲು ದೇವರನ್ನು ದೂರದಲ್ಲಿ ಹೊರಗೆ ನೋಡುತ್ತೇವೆ. ಅನಂತರ ನಾವು ಅವನನ್ನು ಸಮೀಪಿಸಿ, ಅವನು ಎಲ್ಲೆಲ್ಲಿಯೂ ಇರುವನು ಎನ್ನುವೆವು, ಅವನಲ್ಲಿ ಬಾಳುವೆವು. ಕೊನೆಗೆ ನಾವೇ ಅವನು ಎಂಬುದನ್ನು ಅರಿಯುತ್ತೇವೆ. ಹೊರಗಿರುವ ದೇವರು ಅಸತ್ಯವಲ್ಲ. ಪ್ರತಿಯೊಂದು ಧರ್ಮವೂ, ಪ್ರತಿಯೊಂದು ದೇವರ ಭಾವನೆಯೂ ಸತ್ಯ. ಪ್ರತಿಯೊಂದೂ ಈ ಮಹಾಯಾತ್ರೆಯಲ್ಲಿ ಒಂದು ಹಂತ. ಹೀಗೆಂಬುದೇ ವೇದದ ಪೂರ್ಣದೃಷ್ಟಿ. ಆದಕಾರಣ ಹಿಂದೂಗಳಾದ ನಾವು ಅನ್ಯ ಧರ್ಮ ಸಹಿಷ್ಣುತೆಯನ್ನು ಮಾತ್ರ ತೋರುವುದಲ್ಲ. ನಾವು ಪ್ರತಿಯೊಂದು ಧರ್ಮವನ್ನೂ ಸತ್ಯವೆಂದು ಒಪ್ಪಿಕೊಳ್ಳುತ್ತೇವೆ. ಮಹಮ್ಮದೀಯರು ಮಸೀದಿಯಲ್ಲಿ ಪ್ರಾರ್ಥಿಸು ವುದು, ಪಾರ್ಸಿಯರಂತೆ ಅಗ್ನಿಪೂಜೆ ಮಾಡುವುದು, ಕ್ರೈಸ್ತರಂತೆ ಶಿಲುಬೆಯೆದುರಿಗೆ ಬಾಗುವುದು, ಇವುಗಳನ್ನೆಲ್ಲ ಒಪ್ಪಿಕೊಳ್ಳುವೆವು. ಅನಾಗರಿಕವಾದ ಕಲ್ಲು ಮಣ್ಣುಗಳ ಆರಾಧನೆಯಿಂದ ಹಿಡಿದು ಪರಮಾದ್ವೈತದವರೆಗೆ ಎಲ್ಲವು ಮಾನವನು ಆ ಪೂರ್ಣಬ್ರಹ್ಮನನ್ನು ಗ್ರಹಿಸಲು ಮಾಡಿದ ಹಲವು ಪ್ರಯತ್ನಗಳು. ಇದರಲ್ಲಿ ಪ್ರತಿಯೊಂದೂ ಕಾಲ ದೇಶ ವಾತಾವರಣಗಳಿಗೆ ತಕ್ಕಂತೆ ಬೆಳೆದುಕೊಂಡು ಹೋಗಿದೆ; ಪ್ರತಿಯೊಂದೂ ಪ್ರಗತಿಯ ಒಂದೊಂದು ಮೆಟ್ಟಿಲು. ನಾವು ಈ ಪುಷ್ಪಗಳನ್ನೆಲ್ಲ ಆರಿಸಿ ಪ್ರೇಮವೆಂಬ ದಾರದಿಂದ ಕಟ್ಟಿ ಪೂಜಿಸುವುದಕ್ಕೆ ಒಂದು ಸುಂದರವಾದ ಹಾರವನ್ನು ಮಾಡುವೆವು.

ನಾವು ದೇವರಾದರೆ, ನನ್ನ ಆತ್ಮವೇ ಆ ಪರಮಾತ್ಮನ ದೇವಾಲಯ. ನನ್ನ ಪ್ರತಿಯೊಂದು ಚಲನವಲನವೂ ಅವನ ಪೂಜೆಯಾಗಬೇಕು. ಯಾವ ಬಹು ಮಾನದ ಆಕಾಂಕ್ಷೆಯಿಲ್ಲದೆ, ಶಿಕ್ಷೆಯ ಭಯವಿಲ್ಲದೆ ಪ್ರೀತಿಗಾಗಿ ಪ್ರೀತಿ ಮಾಡ ಬೇಕು, ಕರ್ತವ್ಯಕ್ಕಾಗಿ ಕರ್ತವ್ಯ ಮಾಡಬೇಕು. ಆದಕಾರಣ ನನಗೆ ಧರ್ಮ ಎಂದರೆ ವಿಸ್ತಾರವಾಗುವುದು ಎಂದು ಅರ್ಥ. ವಿಸ್ತಾರವಾಗುವುದು ಎಂದರೆ ಪರಮಸತ್ಯವನ್ನು ಸಾಕ್ಷಾತ್ಕಾರ ಮಾಡಿಕೊಳ್ಳುವುದು. ಸುಮ್ಮನೆ ಮಂತ್ರೋಚ್ಚಾರಣೆ ಯಲ್ಲ, ಮಂಡಿಯೂರುವುದಲ್ಲ. ಮಾನವ ಪವಿತ್ರನಾಗಬೇಕು. ದಿನ ಸಾಗಿದಂತೆ ಅವನು ಹೆಚ್ಚು ಹೆಚ್ಚು ಆ ಪವಿತ್ರತೆಯನ್ನು ಅನುಭವಿಸಬೇಕು.

