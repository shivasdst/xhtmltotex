
\vspace{-0.6cm}

\chapter[ಅಧಿಕಾರಿವಾದದ ದೋಷಗಳು ]{ಅಧಿಕಾರಿವಾದದ ದೋಷಗಳು \protect\footnote{\engfoot{C.W. Vol. V, p262}}}

ಪ್ರಶ್ನೋತ್ತರ ತರಗತಿಯೊಂದರಲ್ಲಿ ವಿಷಯವು ಅಧಿಕಾರಿವಾದದ ಕಡೆ ತಿರುಗಿತು. ಸ್ವಾಮೀಜಿಯವರು ಈ ವಾದದ ದೋಷಗಳನ್ನು ತೀವ್ರವಾಗಿ ಖಂಡಿಸುತ್ತ ಈ ಕೆಳಗಿನಂತೆ ನುಡಿದರು:

ನಾನು ಪೂರ್ವಕಾಲದ ಋಷಿಗಳಿಗೆ ಎಷ್ಟೇ ಗೌರವ ತೋರಿದರೂ ಅವರು ಜನರಿಗೆ\break ಬೋಧಿಸಿದ ರೀತಿಯನ್ನು ಖಂಡಿಸದೆ ಇರಲಾರೆ. ಯಾವಾಗಲೂ ಅವರು ಜನರಿಗೆ\break ಇದನ್ನು ಮಾಡಿ ಎಂದು ಹೇಳುತ್ತಿದ್ದರೆ ಹೊರತು ಏತಕ್ಕೆ ಮಾಡಬೇಕು ಎಂಬುದನ್ನು\break ವಿವರಿಸುತ್ತಿರಲಿಲ್ಲ. ಈ ಮಾರ್ಗವೆಲ್ಲಾ ಹಾನಿಕರವಾಯಿತು. ಇದರಿಂದ ಜನರಿಗೆ ಗುರಿ\break ಸಾಧಿಸಲು ನೆರವಾಗುವ ಬದಲು ಅವರು ಕೆಲಸಕ್ಕೆ ಬಾರದ ಬೇಕಾದಷ್ಟು ಮೂಢನಂಬಿಕೆಗಳನ್ನು ಹೊರುವಂತೆ ಮಾಡಲಾಯಿತು. ಅವರು ಅದನ್ನು ವಿವರಿಸದೆ ಇದ್ದುದಕ್ಕೆ ಕಾರಣವೆಂದರೆ ಕೇಳುವವರು ಯೋಗ್ಯರಲ್ಲದೆ ಇದ್ದುದರಿಂದ ಅವರಿಗೆ ಅರ್ಥವನ್ನು ವಿವರಿಸಿದರೂ ಅವರಿಗೆ ತಿಳಿದುಕೊಳ್ಳಲು ಆಗುತ್ತಿರಲಿಲ್ಲ ಎಂಬುದು. ಈ ಅಧಿಕಾರಿವಾದಕ್ಕೆ ಕಾರಣ ಬರಿಯ ಸ್ವಾರ್ಥ. ತಮಗೆ ತಿಳಿದುದನ್ನೆಲ್ಲಾ ಅವರಿಗೆ ತಿಳಿಸಿದರೆ ಬೋಧಕರ ಗೌರವಸ್ಥಾನಕ್ಕೆ ಎಲ್ಲಿ ಚ್ಯುತಿ ಬರುವುದೋ ಎಂದು ಅಂಜಿದ್ದರು. ಆದಕಾರಣವೆ ಅವರು ಅಧಿಕಾರಾವಾದಕ್ಕೆ\break ಬೆಂಬಲ ಕೊಡುತ್ತಿದ್ದರು. ಒಬ್ಬನು ಈ ವಿಷಯಗಳನ್ನು ತಿಳಿದುಕೊಳ್ಳಲು ಇನ್ನೂ ಸಮರ್ಥನಲ್ಲ ಎಂದು ನಾವು ಭಾವಿಸಿದರೆ, ಅವನಿಗೆ ತಿಳಿವನ್ನು ಕೊಡಲು ಹೆಚ್ಚು ಶ್ರಮ ತೆಗೆದುಕೊಂಡು, ಅವನ ಬುದ್ಧಿಯು ಅತಿ ಸೂಕ್ಷ್ಮವಾಗಿರುವುದನ್ನೂ ಅರ್ಥಮಾಡಿಕೊಳ್ಳುವಂತೆ ಮಾಡಬೇಕು. ಅಧಿಕಾರಿ\-ವಾದದ ಬೆಂಬಲಿಗರು ಮಾನವನಲ್ಲಿ ಅನಂತ ಸಾಧ್ಯತೆಗಳು\break ಸುಪ್ತವಾಗಿದೆ ಎಂಬುದನ್ನು ಗಮನಿಸಬೇಕು. ಪ್ರತಿಯೊಬ್ಬರಿಗೂ ಅವರವರ ಭಾಷೆಯಲ್ಲಿ ವಿವರಿಸಿದರೆ ಅವರು ಅರ್ಥಮಾಡಿಕೊಳ್ಳಬಲ್ಲರು. ಮತ್ತೊಬ್ಬರಿಗೆ ಗೊತ್ತಾಗುವಂತೆ ಪಾಠ\break ಹೇಳಲು ಆಗದ ಗುರು, ಶಿಷ್ಯರಲ್ಲಿ ಬುದ್ಧಿ ಇಲ್ಲ, ಉನ್ನತ ಜ್ಞಾನ ಅವರಿಗಿಲ್ಲ ಎಂಬ\break ನೆಪಗಳಿಂದ ಎಂದೆಂದಿಗೂ ಅಜ್ಞಾನದಲ್ಲಿ ಮತ್ತು ಮೂಢನಂಬಿಕೆಗಳಲ್ಲಿ ಇರು ಎಂದು\break ಅವರನ್ನು ನಿಂದಿಸಿ ಶಪಿಸುವ ಬದಲು, ತಾನೆ ತನ್ನ ಅಯೋಗ್ಯತೆಗೆ ಅಳಬೇಕಾಗಿದೆ.\break ಧೈರ್ಯವಾಗಿ ಸತ್ಯವನ್ನು ಹೇಳಿ. ದುರ್ಬಲರಿಗೆ ಇದು ಭ್ರಾಂತಿಯನ್ನು ಉಂಟುಮಾಡುವುದೆಂದು ಅಂಜಬೇಡಿ. ಜನರು ಸ್ವಾರ್ಥಿಗಳು. ಇತರರೂ ತಮ್ಮ ಸ್ಥಿತಿಗೆ ಬಂದರೆ ಎಲ್ಲಿ ತಮ್ಮ ಹಕ್ಕಿಗೆ ಮತ್ತು ಗೌರವಕ್ಕೆ ಚ್ಯುತಿ ಬರುವುದೊ ಎಂದು ಅಂಜುವರು. ದುರ್ಬಲರಿಗೆ ಶ್ರೇಷ್ಠ ಆಧ್ಯಾತ್ಮಿಕ ಸತ್ಯಗಳನ್ನು ಹೇಳಿದರೆ ಅವರಿಗೆ ಭ್ರಾಂತಿಯುಂಟಾಗುವುದೆಂದು ಅವರ ಮತ. ಆದಕಾರಣವೆ-

\begin{verse}
 (ನ ಬುದ್ಧಿಭೇದಂ ಜನಯೇದಜ್ಞಾನಾಂ ಕರ್ಮಸಂಗಿನಾಮ್​~।\\
 ಯೋಜಯೇತ್ಸರ್ವಕರ್ಮಾಣಿ ವಿದ್ವಾನ್​ ಯುಕ್ತಃ ಸಮಾಚರನ್​~॥)
\end{verse}

“ಕರ್ಮದಲ್ಲಿ ಆಸಕ್ತರಾದ ಅಜ್ಞಾನಿಗಳಲ್ಲಿ ಬುದ್ಧಿಭೇದವನ್ನು ಉಂಟುಮಾಡಬಾರದು. ಜ್ಞಾನಿಯು ಯುಕ್ತನಾಗಿ ಕರ್ಮಗಳನ್ನು ಆಚರಿಸುತ್ತಾ (ಅವರಿಂದಲೂ) ಸರ್ವಕರ್ಮಗಳನ್ನು ಮಾಡಿಸಬೇಕು” -ಭಗವದ್ಗೀತೆ.

ಬೆಳಕು ಹೆಚ್ಚಿನ ಕತ್ತಲೆಯನ್ನು ತರುವುದೆಂಬ ವಿರೋಧಾಭಾಸವನ್ನು ನಾನು ಒಪ್ಪಲಾರೆ. ಇದು ಸಚ್ಚಿದಾನಂದ ಸಾಗರದಲ್ಲಿ ಪ್ರಾಣ ಹೋಗುವುದೆಂದು ಅಂಜಿದಂತೆ. ಇದು ಎಂತಹ ತಿಳಿಗೇಡಿತನ! ಜ್ಞಾನ ಎಂದರೆ ಅಜ್ಞಾನದ ತಪ್ಪುಗಳಿಂದ ಪಾರಾಗುವುದು ಎಂದು ಅರ್ಥ. ಜ್ಞಾನ ಹೇಗೆ ಅಜ್ಞಾನಕ್ಕೆ ಕಾರಣವಾಗುವುದು? ತಿಳಿವಳಿಕೆ ಭ್ರಾಂತಿಗೆ ಕಾರಣವೆ? ಇದು ಸಾಧ್ಯವೆ? ಜನರು ವಿಶಾಲವಾದ ಸತ್ಯವನ್ನು ಇತರರಿಗೆ ಬೋಧಿಸಲು ಅಂಜುವರು. ಏಕೆಂದರೆ ಎಲ್ಲಿ ಜನರಿಂದ ಬರುವ ಗೌರವ ಹೋಗುವುದೊ ಎಂಬ ಅಂಜಿಕೆ. ನಿತ್ಯವಾದ ಸನಾತನ ಸತ್ಯಕ್ಕೂ ಜನರ ಕೆಲಸಕ್ಕೆ ಬಾರದ ಮೂಢನಂಬಿಕೆಗಳಿಗೂ ರಾಜಿಮಾಡಿಕೊಂಡು ಲೋಕಾಚಾರ ದೇಶಾಚಾರಗಳು ಇರಬೇಕು ಎನ್ನುವರು. ರಾಜಿ ಬೇಡ. ಮರೆಮಾಡುವುದು ಬೇಡ. ಹೆಣವನ್ನು ಹೂವಿನ ರಾಶಿಯಿಂದ ಮುಚ್ಚಬೇಕಾಗಿಲ್ಲ. “ತಥಾಪಿ ಲೋಕಾಚಾರ:” ‘ಆದರೂ ಲೋಕಾಚಾರವನ್ನು ಅನುಸರಿಸಬೇಕು’ ಎಂಬಂಥ ಶಾಸ್ತ್ರವಾಕ್ಯಗಳನ್ನು ಆಚೆಗೆ ಎಸೆಯಿರಿ. ಇವೆಲ್ಲಾ ಕೆಲಸಕ್ಕೆ ಬಾರದವು. ಇಂತಹ ರಾಜಿಯ ಪರಿಣಾಮವಾಗಿ ಅನರ್ಘ್ಯ ಸತ್ಯಗಳು ಕೆಲಸಕ್ಕೆ ಬಾರದ ಕಸದಿಂದ ಮುಚ್ಚಿಹೋಗುವುವು. ಈ ಕಸವನ್ನೇ ಜನರು ಪರಮಸತ್ಯ ಎಂದು ಭಾವಿಸುವರು. ಶ‍್ರೀಕೃಷ್ಣ ಧೈರ್ಯವಾಗಿ ಸಾರಿದ ಉದಾತ್ತ ಗೀತಾಸಂದೇಶ ಕೂಡ ಅನಂತರ ಬಂದು ಶಿಷ್ಯರಿಂದ ರಾಜಿಗೆ ಒಳಗಾಯಿತು. ಇದರ ಪರಿಣಾಮವಾಗಿಯೇ ಪ್ರಪಂಚದ ಅತ್ಯಮೋಘವಾದ ಶಾಸ್ತ್ರದಲ್ಲಿ ಈಗ ಜನರನ್ನು ಭ್ರಾಂತಿಗೆ ವಶಮಾಡುವ ಹಲವು ವಿಷಯಗಳಿವೆ.

ಕುಲಗೆಟ್ಟ ನೀಚವಾದ ಹೇಡಿತನವೇ ರಾಜಿಗೆ ಕಾರಣ. ಧೈರ್ಯವಾಗಿರಿ. ನನ್ನ ಪುತ್ರರು ಎಲ್ಲಕ್ಕಿಂತ ಹೆಚ್ಚಾಗಿ ಧೀರರಾಗಿರಬೇಕು. ಯಾವ ಕಾರಣದಿಂದಲೂ ಸ್ವಲ್ಪವೂ ರಾಜಿಮಾಡಿಕೊಳ್ಳಕೂಡದು. ಪರಮಸತ್ಯಗಳನ್ನು ಬೋಧಿಸಿ ಜಗತ್ತಿನಲ್ಲಿ ಸಾರಿ. ಎಲ್ಲಿ ಗೌರವಕ್ಕೆ ಕುಂದು ಬರುವುದೋ ಇತರರಿಗೆ ವ್ಯಥೆಯಾಗುವುದೊ ಎಂದು ಅಂಜಬೇಡಿ. ಪ್ರಲೋಭನೆಗಳನ್ನು ತ್ಯಜಿಸಿ ನೀವು ಸತ್ಯೋಪಾಸಕರಾದರೆ ಅದ್ಭುತ ಶಕ್ತಿ ನಿಮ್ಮ ವಶವಾಗುವುದು. ಆಗ ಅಸತ್ಯವಾಗಿರುವ ಏನನ್ನೂ ಜನರು ನಿಮ್ಮ ಎದುರಿಗೆ ಹೇಳಲು ಅಂಜುವರು. ನೀವು ಹದಿನಾಲ್ಕು ವರುಷಗಳವರೆಗೆ ಸ್ವಲ್ಪವೂ ಬಿಡದೆ ಸತ್ಯವನ್ನು ಅನುಸರಿಸಿದರೆ ಜನರು ನೀವು ಹೇಳುವುದನ್ನು ಸಂಪೂರ್ಣ ನಂಬುವರು. ಇದರಿಂದ ನೀವು ಜನರಿಗೆ ಪರಮ ಮಂಗಳವನ್ನು ನೀಡುವಿರಿ. ಅವರನ್ನು ಬಂಧನಗಳಿಂದ ವಿಮೋಚನೆ ಮಾಡುವಿರಿ; ಇಡೀ ದೇಶವನ್ನು ಮೇಲೆತ್ತುವಿರಿ.

