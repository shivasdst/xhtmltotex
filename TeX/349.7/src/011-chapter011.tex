
\chapter[ಗೀತಾ ವಿಚಾರ ]{ಗೀತಾ ವಿಚಾರ \protect\footnote{\engfoot{C.W. Vol. IV, P. 102}}}

ಕ್ರಿ.ಶ. ೧೮೯೭ರಲ್ಲಿ ಸ್ವಾಮಿ ವಿವೇಕಾನಂದರು ಕಲ್ಕತ್ತೆಯಲ್ಲಿದ್ದಾಗ ಬಹು ಕಾಲವನ್ನೆಲ್ಲ ಶ‍್ರೀರಾಮಕೃಷ್ಣ ಸಂಘದ ಕೇಂದ್ರವಾದ ಆಲಂಬಜಾರಿನ ಮಠದಲ್ಲಿ ಕಳೆಯುತ್ತಿದ್ದರು. ಇದೇ ಕಾಲದಲ್ಲಿ ಕೆಲವು ಕಾಲದಿಂದ ಸಾಧುಜೀವನಕ್ಕೆ ಸಿದ್ಧರಾಗು ತ್ತಿದ್ದ ಕೆಲವು ಯುವಕರು ಅವರ ಸುತ್ತ ನೆರೆದರು. ಸ್ವಾಮೀಜಿಯವರಿಂದ ಬ್ರಹ್ಮಚರ್ಯ ಮತ್ತು ಸಂನ್ಯಾಸದೀಕ್ಷೆ ತೆಗೆದು ಕೊಂಡರು. ಸ್ವಾಮೀಜಿಯವರು ಅವರನ್ನು ಭವಿಷ್ಯ ಕಾರ್ಯಕ್ಕೆ ತರಬೇತು ಮಾಡತೊಡಗಿದರು. ಗೀತೆ ಮತ್ತು ವೇದಾಂತದ ಮೇಲೆ ಪ್ರವಚನ ಗಳನ್ನು ನೀಡುತ್ತಿದ್ದರು. ಜೊತೆಗೆ ಅವರನ್ನು ಧ್ಯಾನಾಭ್ಯಾಸದಲ್ಲಿ ತೊಡಗಿಸುತ್ತಿದ್ದರು. ಹೀಗೆ ಪ್ರವಚನ ಮಾಡುವಾಗ ಒಂದು ಸಲ ಅವರು ಅನುವಾದದ ಬಂಗಾಲಿಯಲ್ಲಿ ಗೀತೆಯ ಮೇಲೆ ಅತ್ಯಮೋಘವಾದ ಭಾಷಣ ಮಾಡಿದರು. ಮಠದ ದಿನಚರಿಯಲ್ಲಿ ದಾಖಲಾಗಿರುವ ಅದರ ಅನುವಾದದ ಸಾರಾಂಶ ಇದು:

ಗೀತೆ ಎಂಬ ಗ್ರಂಥ ಮಹಾಭಾರತದ ಒಂದು ಭಾಗ. ಗೀತೆಯನ್ನು ಚೆನ್ನಾಗಿ ಅರಿಯಬೇಕಾದರೆ ಹಲವು ವಿಷಯಗಳನ್ನು ತಿಳಿದುಕೊಳ್ಳುವುದು ಅವಶ್ಯಕ. ಮೊದ ಲನೆಯದು ಅದು ಮಹಾಭಾರತದ ಒಂದು ಭಾಗವಾಗಿತ್ತೆಂಬುದು. ಅಂದರೆ ಅದನ್ನು ವೇದವ್ಯಾಸರು ಬರೆದರೆ ಅಥವಾ ಒಂದು ಪ್ರಕ್ಷಿಪ್ತ ಭಾಗವೇ? ಎರಡನೆಯದಾಗಿ ಶ‍್ರೀಕೃಷ್ಣ ಎಂಬ ಚಾರಿತ್ರಿಕವ್ಯಕ್ತಿ ಇದ್ದನೆ? ಮೂರನೆಯದಾಗಿ ಗೀತೆಯಲ್ಲಿ ಉಕ್ತ ವಾಗಿ ರುವ ಮಹಾಭಾರತ ಯುದ್ಧ ನಿಜವಾಗಿ ಜರುಗಿತೆ? ನಾಲ್ಕನೆಯದು ಅರ್ಜುನ ಮುಂತಾದವರು ಚಾರಿತ್ರಿಕ ವ್ಯಕ್ತಿಗಳೆ?

ಈಗ ಮೊದಲನೆಯದಾಗಿ ಇಂತಹ ವಿಮರ್ಶೆಗೆ ಆಸ್ಪದವಿದೆಯೆ ಎಂಬುದನ್ನು ನೋಡೋಣ. ವೇದವ್ಯಾಸರೆಂದು ಕರೆಯಿಸಿಕೊಂಡ ಹಲವರಿದ್ದರು ಎಂಬುದು ನಮಗೆ ಗೊತ್ತಿದೆ. ಅವರಲ್ಲಿ ನಿಜವಾಗಿ ಗೀತೆಯನ್ನು ಬರೆದವರು ಯಾರು? ಬಾದರಾಯಣ ವ್ಯಾಸನೆ? ದ್ವೈಪಾಯನ ವ್ಯಾಸನೆ? “ವ್ಯಾಸ” ಎನ್ನುವುದು ಕೇವಲ ಒಂದು ಬಿರುದು, ಯಾರು ಹೊಸದಾಗಿ ಒಂದು ಪುರಾಣವನ್ನು ಬರೆದರೂ ಅವರ ನ್ನೆಲ್ಲ ವ್ಯಾಸ ಎಂದು ಕರೆಯುತ್ತಿದ್ದರು. ವಿಕ್ರಮಾದಿತ್ಯ ಎಂದು ರೂಢಿಯಲ್ಲಿರುವ ಸಾಮಾನ್ಯ ಪದದಂತೆ ವ್ಯಾಸ ಎಂಬುದು ಕೂಡ. ಮತ್ತೊಂದು ವಿಷಯವೆಂದರೆ ಶಂಕರಾಚಾರ್ಯರು ಭಾಷ್ಯವನ್ನು ಬರೆದು ಗೀತೆಯನ್ನು ಪ್ರಸಿದ್ಧ ಮಾಡುವುದಕ್ಕೆ ಮುಂಚೆ ಸಾಧಾರಣ ಜನರಿಗೆ ಗೀತೆಯ ವಿಷಯ ಹೆಚ್ಚಾಗಿ ಗೊತ್ತಿರಲಿಲ್ಲ. ಅದಕ್ಕಿಂತ ಲೂ ಬಹಳ ಮುಂಚೆ ಅನೇಕರ ಅಭಿಪ್ರಾಯದಲ್ಲಿ ಗೀತೆಯ ಮೇಲೆ ಬೋಧಾ ಯನ ವೃತ್ತಿ ಎಂಬುದೊಂದು ಇತ್ತು. ಇದನ್ನು ನಾವು ಸಪ್ರಮಾಣವಾಗಿ ದೃಢ ಪಡಿಸಿದರೆ ಗೀತೆಯು ಪುರಾತನವಾದದು ಮತ್ತು ವೇದವ್ಯಾಸರೆ ಅದರ ಕರ್ತೃಗಳು ಎಂಬುದನ್ನು ತೋರಲು ಸಾಧ್ಯವಾಗುವುದು ಎಂಬುದರಲ್ಲಿ ಸಂದೇಹವೇನೂ ಇಲ್ಲ. ಆದರೆ ಬೋಧಾಯನರು ವೇದಾಂತ ಸೂತ್ರಗಳ ಮೇಲೆ ಬರೆದ ಭಾಷ್ಯ (ಇದರಿಂದಲೇ ರಾಮಾನುಜರು ಶ‍್ರೀಭಾಷ್ಯವನ್ನು ಕ್ರೋಢೀ ಕರಿಸಿದರು. ಶಂಕರಾಚಾರ್ಯರು ಈ ಹೆಸರನ್ನು ಹೇಳುವರು ಮತ್ತು ಕೆಲವು ವೇಳೆ ಅದರಿಂದ ತಮ್ಮ ಭಾಷ್ಯದಲ್ಲಿ ಉದಾಹರಿಸುವರು. ಸ್ವಾಮಿ ದಯಾನಂದರು ಇದನ್ನೇ ಬಹಳ ಚರ್ಚಿಸುವರು) ಭರತಖಂಡವನ್ನೆಲ್ಲಾ ನಾನು ಸಂಚಾರ ಮಾಡುವಾಗ ನನಗೆ ಎಲ್ಲೂ ಸಿಕ್ಕಲಿಲ್ಲ. ರಾಮಾನುಜರು ಕೂಡ ತಮ್ಮ ಭಾಷ್ಯವನ್ನು ಬರೆಯುವಾಗ ತಮಗೆ ಸಿಕ್ಕಿದ ಯಾವುದೋ ಬಹಳ ಪುರಾತನವಾದ ಹುಳು ತಿಂದುಹೋದ ಓಲೆಗರಿಯಿಂದ ಅದನ್ನು ಕ್ರೋಢೀಕರಿಸಿದರಂತೆ. ವೇದಾಂತ ಸೂತ್ರಗಳ ಮೇಲೆ ಬರೆದ ಪ್ರಖ್ಯಾತವಾದ ಬೋಧಾಯನ ಭಾಷ್ಯವೆ ಅನುಮಾನಾಸ್ಪದವಾದ ಕಾರ್ಗತ್ತಲಲ್ಲಿ ಹುದುಗಿದ್ದರೆ ಗೀತೆಯ ಮೇಲೆ ಬೋಧಾಯನ ಭಾಷ್ಯವಿತ್ತು ಎಂಬುದನ್ನು ದೃಢಪಡಿಸಲು ಸಾಧ್ಯವೇ ಇಲ್ಲ. ಕೆಲವರು ಶಂಕರಾಚಾರ್ಯರೇ ಗೀತೆಯ ಕರ್ತೃವೆಂದೂ ಅದನ್ನು ಮಹಾಭಾರತದಲ್ಲಿ ಸೇರಿಸಿದರೆಂದೂ ಹೇಳುವರು.

ಚರ್ಚೆಯ ಎರಡನೆಯ ವಿಷಯವಾದ ಕೃಷ್ಣನ ವ್ಯಕ್ತಿತ್ವದ ವಿಷಯದಲ್ಲಿ ಎಷ್ಟೋ ಸಂದೇಹವಿದೆ. ಛಾಂದೋಗ್ಯ ಉಪನಿಷತ್ತಿನಲ್ಲಿ ಒಂದು ಕಡೆ ದೇವಕೀಪುತ್ರನಾದ ಕೃಷ್ಣನು ಘೋರಾ ಎಂಬ ಯೋಗಿಯಿಂದ ಉಪದೇಶವನ್ನು ಪಡೆದನೆಂದು ಹೇಳಿದೆ. ಮಹಾಭಾರತದಲ್ಲಿ ಕೃಷ್ಣ ದ್ವಾರಕೆಯ ರಾಜ. ವಿಷ್ಣುಪುರಾಣದಲ್ಲಿ ಅವನ ಗೋಪೀ ಲೀಲೆಯ ವಿವರಣೆಯನ್ನು ನೋಡುತ್ತೇವೆ. ಪುನಃ ಭಾಗವತದಲ್ಲಿ ಅವನ ರಾಸಲೀಲೆ ಯ ವರ್ಣನೆಯನ್ನು ವಿವರವಾಗಿ ಕೊಡುವರು. ಬಹಳ ಹಿಂದೆ ನಮ್ಮ ದೇಶದಲ್ಲಿ ಮದನೋತ್ಸವ ಎಂಬುದು ರೂಢಿಯಲ್ಲಿತ್ತು. ಅದನ್ನೇ ಡೋಲಜಾತ್ರೆಯನ್ನಾಗಿ ಮಾಡಿ ಕೃಷ್ಣನ ಮೇಲೆ ಆರೋಪಮಾಡಿದರು. ಅವನಿಗೆ ಸಂಬಂಧಿಸಿದ ರಾಸಲೀಲೆ ಮುಂತಾ ದುವನ್ನು ಹೀಗೆಯೇ ಆರೋಪ ಮಾಡಿಲ್ಲ ಎಂದು ಹೇಳುವುದಕ್ಕೆ ಯಾರಿಗೆ ಧೈರ್ಯವಿದೆ? ಸಂಶೋಧನಾತ್ಮಕ ದೃಷ್ಟಿಯಿಂದ ಚಾರಿತ್ರಿಕ ಸತ್ಯಗಳನ್ನು ತಿಳಿದು ಕೊಳ್ಳಬೇಕೆಂಬ ಸ್ವಭಾವವೇ ಹಿಂದಿನ ಕಾಲದಲ್ಲಿ ನಮ್ಮಲ್ಲಿ ಇರಲಿಲ್ಲ. ಪ್ರತಿಯೊಬ್ಬರೂ ತಮಗೆ ತೋಚಿದುದನ್ನೇ ಸಾಕಷ್ಟು ಪ್ರಮಾಣಗಳಿಲ್ಲದೆ ಇದೇ ಸರಿ ಎಂದು ವಾದಿ ಸುತ್ತಿದ್ದರು. ಮತ್ತೊಂದು ವಿಷಯ: ಆಗಿನ ಕಾಲದಲ್ಲಿ ಕೀರ್ತಿಯ ಮತ್ತು ಹೆಸರಿನ ಆಸೆ ಜನರಲ್ಲಿ ಇರಲಿಲ್ಲ. ಒಬ್ಬ ಒಂದು ಗ್ರಂಥವನ್ನು ಬರೆದು ಅದನ್ನು ತನ್ನ ಗುರುವಿನ ಹೆಸರಿನಲ್ಲೊ ಅಥವಾ ಇನ್ನೊಬ್ಬನ ಹೆಸರಿನಲ್ಲೊ ಬಳಕೆಗೆ ತರುತ್ತಿದ್ದ. ಅನೇಕ ವೇಳೆ ಹೀಗೆ ಆಗುತ್ತಿತ್ತು. ಇಂತಹ ಸ್ಥಿತಿಯಲ್ಲಿ ಚಾರಿತ್ರಿಕ ಸತ್ಯಗಳನ್ನು ಕಂಡುಹಿಡಿಯ ಬೇಕೆನ್ನುವವನಿಗೆ ಅವಕಾಶವೇ ಇರುವುದಿಲ್ಲ. ಆಗಿನ ಕಾಲದಲ್ಲಿ ಭೂಗೋಳ ಜ್ಞಾನವೇ ಇರಲಿಲ್ಲ. ಕಲ್ಪನೆ ಲಂಗು ಲಗಾಮಿಲ್ಲದೆ ಓಡುತ್ತಿತ್ತು. ಆದಕಾರಣವೇ ಎಷ್ಟೋ ಮನೋಕಲ್ಪಿತ ದಧಿಸಾಗರ, ಮಧುಸಾಗರ, ತುಪ್ಪದಸಾಗರ ಮುಂತಾದವು ಇವೆ. ಪುರಾಣಗಳಲ್ಲಿ ಒಬ್ಬ ಹತ್ತುಸಾವಿರ ವರುಷ ಬದುಕುವನು. ಮತ್ತೊಬ್ಬ ಒಂದು ಲಕ್ಷ ವರುಷ ಬದುಕುವನು. ಆದರೆ ವೇದ “ಶತಾಯುರ್ವೈ ಪುರುಷಃ” ಎಂದು ಹೇಳು ವುದು. ಆದಕಾರಣ ಕೃಷ್ಣನ ವಿಷಯದಲ್ಲಿ ಒಂದು ಸರಿಯಾದ ನಿರ್ಧಾರಕ್ಕೆ ಬರುವುದೇ ದುಸ್ಸಾಧ್ಯ.

ಒಬ್ಬ ಮಹಾಪುರುಷನ ಜೀವನದ ಸುತ್ತಲೂ ಎಷ್ಟೋ ಅಲೌಕಿಕ ವಿಶೇಷಗುಣಗಳನ್ನು ಆರೋಪಮಾಡುವುದು ಮಾನವರ ಸ್ವಭಾವ. ಕೃಷ್ಣನ ವಿಷಯದಲ್ಲಿಯೂ ಹೀಗೆಯೇ ಆಗಿರಬಹುದು. ಅವನು ಬಹುಶಃ ರಾಜನಾಗಿದ್ದಿರ ಬಹುದು. ನಾನು “ಬಹುಶಃ” ಎಂದು ಹೇಳುತ್ತೇನೆ. ಏಕೆಂದರೆ ನಮ್ಮ ದೇಶದಲ್ಲಿ ಹಿಂದಿ ನ ಕಾಲದಲ್ಲಿ ಬ್ರಹ್ಮಜ್ಞಾನವನ್ನು ಬೋಧಿಸಲು ತೊಂದರೆ ತೆಗೆದುಕೊಳ್ಳುತ್ತಿದ್ದವರಲ್ಲಿ ಬಹುಮಂದಿ ರಾಜರಾಗಿದ್ದರು. ನಾವು ವಿಶೇಷವಾಗಿ ಗಮನಿಸಬೇಕಾದ ಮತ್ತೊಂದು ವಿಷಯವೆಂದರೆ ಗೀತೆಯ ಕರ್ತೃಗಳು ಯಾರಾದರೂ ಆಗಿರಲಿ, ಗೀತೆಯ ಬೋಧನೆ ಮಹಾಭಾರತದ ಬೋಧನೆಯಂತೆಯೇ ಇದೆ. ಇದರಿಂದ ಮಹಾಭಾರತದ ಕಾಲದಲ್ಲಿ ಒಬ್ಬ ಮಹಾಪುರುಷನು ಹೊಸ ರೀತಿಯಲ್ಲಿ ಆಗಿನ ಸಮಾಜಕ್ಕೆ ಬ್ರಹ್ಮಜ್ಞಾನವನ್ನು ಬೋಧಿಸಿದನು ಎಂದು ಧೈರ್ಯವಾಗಿ ಹೇಳಬಹುದು. ಮತ್ತೊಂದು ಸಂಗತಿ ಇಲ್ಲಿ ವ್ಯಕ್ತವಾಗುವುದು. ಅದೇನೆಂದರೆ ಒಂದು ಪಂಥವಾದ ಮೇಲೆ ಮತ್ತೊಂದು ಪಂಥ ಬಂದಾಗ ಯಾವುದಾದರೂ ಹೊಸದೊಂದು ಶಾಸ್ತ್ರ ಬಳಕೆಗೆ ಬರುತಿತ್ತು. ಕಾಲ ಕ್ರಮೇಣ ಹೊಸ ಪಂಥ ಮತ್ತು ಜಾರಿಗೆ ಬಂದ ಶಾಸ್ತ್ರ ಎರಡು ಕೂಡ ಹೆಸರಿಲ್ಲದಂತೆ ನಾಶವಾಗುತ್ತಿತ್ತು. ಅಥವಾ ಪಂಥ ಮಾತ್ರವೇ ನಾಶವಾಗಿ ಶಾಸ್ತ್ರವು ಹಿಂದೆ ಉಳಿಯುವ ಸಂದರ್ಭಗಳೂ ಇದ್ದವು. ಹೀಗೆ ಗೀತೆಯು ಅಂತಹ ಒಂದು ಪಂಥದ ಉಚ್ಚ ಧ್ಯೇಯಗಳನ್ನೊಳಗೊಂಡ ಶಾಸ್ತ್ರವಾಗಿದ್ದಿರಬಹುದು.

ಮೂರನೆಯ ವಿಷಯ. ಕುರುಕ್ಷೇತ್ರ ಯುದ್ಧದ ಸಂಬಂಧವಾಗಿ ಅದು ನಡೆಯಿತೆಂದು ಸಾಧಿಸುವ ವಿಶೇಷ ಪ್ರಮಾಣಗಳಾವುವೂ ಇಲ್ಲ. ಕುರುವಂಶಜರಿಗೆ ಮತ್ತು ಪಾಂಚಾಲರಿಗೆ ಯುದ್ಧ ನಡೆಯಿತು ಎನ್ನುವುದರಲ್ಲಿ ಯಾವ ಸಂದೇಹವೂ ಇಲ್ಲ. ಆದರೆ ಕಾದಲನುವಾಗಿ ಕೊನೆಯ ಸಂಜ್ಞೆಗೆ ಕಾದು ನಿಂತಿದ್ದ ಆ ಸೇನಾ ಸಮೂಹದಿಂದ ತುಂಬಿದ ಸಮರಾಂಗಣದಲ್ಲಿ ಜ್ಞಾನ, ಭಕ್ತಿ ಮತ್ತು ಯೋಗ – ಇವುಗಳ ವಿಷಯದಲ್ಲಿ ಇಷ್ಟೊಂದು ದೀರ್ಘ ಚರ್ಚೆಗೆ ಅವಕಾಶವಿದೆಯೆ? ಆ ರಣಕೋಲಾಹಲದ ಮಧ್ಯದಲ್ಲಿ ಕೃಷ್ಣಾರ್ಜುನರಿಗೆ ನಡೆದ ಸಂಭಾಷಣೆಯನ್ನು ಬರೆದು ತೆಗೆದುಕೊಳ್ಳುವುದಕ್ಕೆ ಯಾರಾ ದರೂ ಒಬ್ಬ ಶೀಘ್ರಲಿಪಿಕಾರನಿದ್ದನೆ? ಕೆಲವು ದೃಷ್ಟಿಯಲ್ಲಿ ಕುರುಕ್ಷೇತ್ರವೆಂಬುದು ಕೇವಲ ಒಂದು ರೂಪಕಕಥೆ. ಅದರ ಅಂತರಾರ್ಥವನ್ನು ಸಂಗ್ರಹಿಸಿ ಹೇಳುವುದಾದರೆ ಅನವರತವಾಗಿ ಮನುಷ್ಯನಲ್ಲಿ ಅವನ ದುಷ್ಟ ಸ್ವಭಾವಕ್ಕೂ ಸತ್​ ಸ್ವಭಾವಕ್ಕೂ ನಡೆಯುತ್ತಿರುವ ಘರ್ಷಣೆಯೇ ಯುದ್ಧ. ಇದು ಕೂಡ ವಿಚಾರಸರಣಿಗೆ ವಿರೋಧವಲ್ಲ.

ಅರ್ಜುನ ಮುಂತಾದವರು ಚಾರಿತ್ರಿಕ ವ್ಯಕ್ತಿಗಳೆ ಅಥವಾ ಅಲ್ಲವೆ ಎಂಬ ನಾಲ್ಕನೆಯ ವಿಷಯದಲ್ಲಿ ಎಷ್ಟೋ ಸಂದೇಹಗಳಿವೆ. ಶತಪಥ ಬ್ರಾಹ್ಮಣ ಬಹಳ ಪುರಾತನ ಗ್ರಂಥ. ಅಲ್ಲಿ ಅಶ್ವಮೇಧ ಯಜ್ಞ ಮಾಡಿದವರ ಹೆಸರುಗಳೆಲ್ಲ ಇವೆ. ಆದರೆ ಅಲ್ಲಿ ಅರ್ಜುನನ ಮತ್ತು ಅವನ ಸಹೋದರರ ಹೆಸರುಗಳಾಗಲಿ, ಅಂತಹವರಿದ್ದರು ಎಂಬ ಸೂಚನೆ ಯಾಗಲೀ ಇಲ್ಲ. ಅಲ್ಲಿ ಅರ್ಜುನನ ಮೊಮ್ಮಗನಾದ ಪರೀಕ್ಷಿತನ ಮಗನಾದ ಜನ ಮೇಜಯನ ಹೆಸರಿದೆ. ಆದರೂ ಮಹಾಭಾರತದಲ್ಲಿ ಯುಧಿಷ್ಠಿರ, ಅರ್ಜುನ ಮುಂತಾ ದ ವರು ಅಶ್ವಮೇಧಯಾಗವನ್ನು ಮಾಡಿದರು ಎಂದು ಬರೆದಿದೆ.

ನಾವು ಮತ್ತೊಂದು ವಿಷಯವನ್ನು ಇಲ್ಲಿ ವಿಶೇಷವಾಗಿ ಗಮನಿಸಬೇಕು.ಐತಿಹಾಸಿಕ ಸಂಶೋಧನೆಗಳಿಗೂ ನಿಜವಾದ ನಮ್ಮ ಉದ್ಧೇಶಕ್ಕೂ ಯಾವ ಸಂಬಂಧ ವೂ ಇಲ್ಲ ಎಂಬುದು. ಯಾವ ಜ್ಞಾನವನ್ನು ಅರಿತರೆ ನಾವು ಧರ್ಮಾತ್ಮರಾಗು ತ್ತೇ ವೋ ಆ ಜ್ಞಾನವೇ ನಮ್ಮ ಗುರಿ. ಚಾರಿತ್ರಿಕ ದೃಷ್ಟಿಯಿಂದ ಇವೆಲ್ಲ ಶುದ್ಧ ಸುಳ್ಳು ಎಂದು ನಿರ್ವಿವಾದವಾಗಿ ಸಾಧಿಸಿದರೂ ಅದರಿಂದ ನಮಗೆ ಲವಲೇಶವೂ ನಷ್ಟವಿಲ್ಲ. ಹಾಗಾದರೆ ಇಷ್ಟೊಂದು ಚಾರಿತ್ರಿಕ ಸಂಶೋಧನೆ ಏಕೆ ಎನ್ನಬಹುದು. ಇದರಿಂದಲೂ ಪ್ರಯೋಜನ ವಿದೆ. ನಾವು ಸತ್ಯವನ್ನು ಅರಿಯಬೇಕಾಗಿದೆ. ಅಜ್ಞಾನದಿಂದ ತಪ್ಪು ಭಾವನೆಗಳನ್ನು ನೆಚ್ಚಿಕೊಂಡಿರುವುದು ಯೋಗ್ಯವಲ್ಲ. ಈ ದೇಶದಲ್ಲಿ ಇಂತಹ ಸಂಶೋಧನೆಗೆ ಪ್ರಾಧಾನ್ಯವನ್ನೇ ಕೊಡುವುದಿಲ್ಲ. ಅನೇಕ ಪಂಥದವರು ಹಲವರಿಗೆ ಉಪಯೋಗ ಕರವಾದ ಒಳ್ಳೆಯ ವಿಷಯಗಳನ್ನು ಹೇಳಬೇಕಾದರೆ ಒಂದು ಸುಳ್ಳನ್ನು ಹೇಳಿದರೂ ಚಿಂತೆಯಿಲ್ಲವೆಂದು ಭಾವಿಸುವರು. ಆದರೆ ಗುರಿ ಮುಖ್ಯವೇ ಹೊರತು ದಾರಿಯಲ್ಲ. ಆದಕಾರಣ ನಮ್ಮ ಹಲವು ತಂತ್ರ ಶಾಸ್ತ್ರಗಳು ಮಹಾದೇವನು ಪಾರ್ವತಿಗೆ ಹೇಳಿದ ಎಂದು ಪ್ರಾರಂಭವಾಗುವುದು. ಆದರೆ ನಾವು ನಂಬುವುದು ಸತ್ಯವೇ ಎಂಬುದನ್ನು ವಿಮರ್ಶಿಸುವುದು ನಮ್ಮ ಕರ್ತವ್ಯ. ಸತ್ಯವನ್ನು ಮಾತ್ರ ನಂಬುವುದು ನಮ್ಮ ಕರ್ತವ್ಯ. ಮೂಢನಂಬಿಕೆ ಅಥವಾ ಹಿಂದಿನಿಂದ ಬಂದ ಕಂದಾ ಚಾರಗಳನ್ನು ವಿವೇಚನೆ ಇಲ್ಲದೆ ನಂಬುವುದು ಎಷ್ಟು ಬಲವಾಗಿದೆ ಎಂದರೆ ಜನರು ಮುಂದೆ ಹೋಗ ದಂತೆ ಅವರ ಕೈಕಾಲುಗಳನ್ನು ಕಟ್ಟಿದಂತೆ ಇದೆ. ಏಸುಕ್ರಿಸ್ತ, ಮಹಮದ್​ ಮುಂತಾದ ಮಹಾಪುರುಷರೆ ಇಂತಹ ಹಲವು ಮೂಢ ನಂಬಿಕೆಗಳನ್ನು ನಂಬಿದ್ದರು, ಅದರಿಂದ ಪಾರಾಗಿರಲಿಲ್ಲ. ನಿಮ್ಮ ದೃಷ್ಟಿ ಯಾವಾಗಲೂ ಸತ್ಯದ ಕಡೆಗೆ ಇರಬೇಕು. ಎಲ್ಲ ಬಗೆಯ ಮೂಢನಂಬಿಕೆಗಳನ್ನೂ ನಿರ್ಲಕ್ಷಿಸಬೇಕು.

ಗೀತೆಯಲ್ಲಿ ಏನಿದೆ ಎಂಬುದನ್ನು ನಾವೀಗ ನೋಡಬೇಕಾಗಿದೆ. ನಾವು ಉಪನಿಷತ್ತನ್ನು ಓದುವಾಗ, ಎಷ್ಟೋ ಅಸಂಗತವಾದ ವಿಷಯಗಳ ಮೂಲಕ ಹೋಗುತ್ತಿರುವಾಗ ಇದ್ದಕ್ಕಿದ್ದಂತೆ ಒಂದು ದೊಡ್ಡ ತಾತ್ತ್ವಿಕ ಜಿಜ್ಞಾಸೆ ಪ್ರಾರಂಭ ವಾಗುವುದು – ಒಂದು ದೊಡ್ಡ ಕಾಡಿನಲ್ಲಿ ಸಂಚಾರ ಮಾಡುತ್ತಿರುವಾಗ ಅನಿರೀಕ್ಷಿತವಾಗಿ ಅನನ್ಯ ಸಾಧಾರಣವಾಗಿ ಸುಂದರವಾಗಿರುವ ಒಂದು ಗುಲಾಬಿ ಹೂವು ಗಿಡದ ಮುಳ್ಳು ಬಳ್ಳಿ, ಬೇರುಗಳಲ್ಲಿ ಸಿಕ್ಕಿಕೊಂಡಿರುವುದನ್ನು ನೋಡಿದಂತೆ. ಅದರೊಡನೆ ಹೋಲಿಸಿದರೆ ಗೀತೆ ಚೆನ್ನಾಗಿ ಅಣಿ ಮಾಡಿದ ಅಥವಾ ಅಪೂರ್ವ ಪುಷ್ಪಗಳನ್ನೊಳ ಗೊಂಡ ತುರಾಯಿಯಂತೆ ಇದೆ. ಉಪನಿಷತ್ತುಗಳು ಅನೇಕ ಕಡೆ ಶ್ರದ್ಧೆಯ ವಿಷಯ ವಾಗಿ ಹೇಳುತ್ತವೆ. ಅದರೆ ಅಲ್ಲೆಲ್ಲೂ ಭಕ್ತಿಯ ಮಾತೇ ಇಲ್ಲ. ಆದರೆ ಗೀತೆಯಲ್ಲಿ ಭಕ್ತಿಯ ವಿಷಯವನ್ನು ಮತ್ತೆ ಮತ್ತೆ ಹೇಳಿರುವುದು ಮಾತ್ರವಲ್ಲ, ಅಲ್ಲಿ ಭಕ್ತಿಭಾವ ತನ್ನ ಪರಾಕಾಷ್ಠೆಯನ್ನು ಮುಟ್ಟಿದೆ.

ಈಗ ಗೀತೆಯಲ್ಲಿ ಚರ್ಚಿಸಿರುವ ಕೆಲವು ಮುಖ್ಯ ವಿಷಯಗಳನ್ನು ತೆಗೆದು ಕೊಳ್ಳೋಣ. ಗೀತೆ ಅದಕ್ಕೆ ಹಿಂದೆ ಇದ್ದ ಶಾಸ್ತ್ರಗಳಿಗಿಂತ ಯಾವ ವಿಧದಲ್ಲಿ ವಿಶೇಷವಾಗಿರುವುದು? ಅದೇ ಇದು: ಗೀತೆಗೆ ಮುಂಚೆ ಜ್ಞಾನ, ಭಕ್ತಿಯೋಗಗಳ ಕಟ್ಟಾ ಅನುಯಾಯಿಗಳಿದ್ದರೂ, ಎಲ್ಲರೂ ತಮ್ಮ ತಮ್ಮೊಳಗೆ ಕಾದಾಡುತ್ತಿದ್ದರು.ಪ್ರತಿಯೊಬ್ಬರೂ ತಮ್ಮದೇ ಶ್ರೇಷ್ಠಮಾರ್ಗ ಎಂದು ಹೇಳುತ್ತಿದ್ದರು. ಯಾರೂ ಭಿನ್ನ ಭಿನ್ನ ಮಾರ್ಗಗಳನ್ನು ಸೌಹಾರ್ದಭಾವದಿಂದ ನೋಡಲು ಯತ್ನಿಸಲಿಲ್ಲ. ಗೀತೆಯ ಕರ್ತೃ ಪ್ರಥಮಬಾರಿಗೆ ಇವುಗಳನ್ನು ಒಂದು ಸೌಹಾರ್ದದೃಷ್ಟಿಯಿಂದ ನೋಡಲು ಯತ್ನಿಸಿದನು. ಆಗಿನ ಕಾಲದಲ್ಲಿ ಪ್ರತಿಯೊಂದು ಪಂಥದವರೂ ಕೊಟ್ಟ ಶ್ರೇಷ್ಠ ಭಾವನೆಗಳನ್ನೆಲ್ಲಾ ತೆಗೆದುಕೊಂಡು ಅದನ್ನೆಲ್ಲಾ ಗೀತೆಯಲ್ಲಿ ಪೋಣಿಸಿರುವನು. ಎಲ್ಲಿ ಶ‍್ರೀಕೃಷ್ಣನು ಕೂಡಾ ಸಮನ್ವಯವನ್ನು ತರಲು ಸಾಧ್ಯವಾಗಲಿಲ್ಲವೋ ಅಲ್ಲಿ ಶ‍್ರೀರಾಮಕೃಷ್ಣರು ಹತ್ತೊಂಬತ್ತನೇ ಶತಮಾನದಲ್ಲಿ ಅದನ್ನು ಸಾಧಿಸಿದರು.

ಅನಂತರವೇ ನಿಷ್ಕಾಮ ಕರ್ಮ. ಈಗ ಜನರು ಅದನ್ನು ಬಹು ವಿಧಗಳಲ್ಲಿ ಅರ್ಥ ಮಾಡಿಕೊಳ್ಳುವರು. ಅನಾಸಕ್ತಿ ಎಂದರೆ ಗೊತ್ತುಗುರಿ ಇಲ್ಲದೆ ಕೆಲಸ ಮಾಡುವುದು ಎಂದು ಕೆಲವರು ಭಾವಿಸುವರು. ಇದೇ ನಿಜವಾದ ಅರ್ಥ ಎಂದಾದರೆ ನಿರ್ದಯರಾದ ಮೂಢರು ಮತ್ತು ಗೋಡೆಗಳು ನಿಷ್ಕಾಮಕರ್ಮಕ್ಕೆ ಉದಾಹರಣೆಯಾಗುವುವು. ಮತ್ತೆ ಕೆಲವರು ಜನಕನ ಉದಾಹರಣೆ ಕೊಡುವರು. ತಾವೇ ನಿಷ್ಕಾಮಕರ್ಮದಲ್ಲಿ ನಿಪುಣರು ಎಂದು ಜನ ಭಾವಿಸಲಿ ಎಂಬ ಕುತೂಹಲ ಅವರಿಗೆ. ಆದರೆ ಜನಕ ಸಂತಾನೋತ್ಪತ್ತಿ ಯಿಂದ ನಿಷ್ಕಾಮಕರ್ಮಕ್ಕೆ ಉದಾಹರಣೆಯಾಗಲಿಲ್ಲ. ಆದರೆ ಇವರೆಲ್ಲ ಒಂದು ಮಕ್ಕಳ ಪಡೆಗೆ ತಂದೆಯಾಗಿರುವುದೊಂದೆ ಇವರ ವಿಶೇಷ ಅರ್ಹತೆ! ಇದರಿಂದಲೇ ಇವರು ಜನಕರಾಗಬಯಸುವರು. ಹಾಗಲ್ಲ, ನಿಜವಾದ ನಿಷ್ಕಾಮಕರ್ಮಿ ಮೂಢನೂ ಇಲ್ಲ, ಜಡನೂ ಅಲ್ಲ, ನಿರ್ದಯನೂ ಅಲ್ಲ. ಅವನು ತಾಮಸಿಕನಲ್ಲ. ಶುದ್ಧ ಸಾತ್ತ್ವಿಕ ಸ್ವಭಾವ ದವನು. ಅವನ ಹೃದಯ ಪ್ರೀತಿ ಸಹಾನುಭೂತಿಗಳಿಂದ ತುಂಬಿ ತುಳುಕಾಡುತ್ತಿರುವುದು. ಅವನು ಇಡೀ ವಿಶ್ವವನ್ನು ಪ್ರೀತಿಯಿಂದ ಆಲಿಂಗಿಸ ಬಲ್ಲ. ಜಗತ್ತು ಅವನ ಪ್ರೀತಿಯನ್ನು, ವಿಶ್ವಕಾರುಣ್ಯವನ್ನು ಸಾಧಾರಣವಾಗಿ ಅರಿಯಲಾರದು.

ಭಿನ್ನ ಭಿನ್ನ ಮಾರ್ಗಗಳ ಸಮನ್ವಯ ಮತ್ತು ನಿಷ್ಕಾಮಕರ್ಮ ಇವೇ ಗೀತೆಯ ಎರಡು ಮುಖ್ಯ ವೈಶಿಷ್ಟ್ಯಗಳು.

ಈಗ ಎರಡನೇ ಅಧ್ಯಾಯದಿಂದ ಸ್ವಲ್ಪ ಓದೋಣ.

\begin{verse}
॥ ಸಂಜಯ ಉವಾಚ~॥
\end{verse}

\begin{verse}
ತಂ ತಥಾ ಕೃಪಯಾವಿಷ್ಟಮಶ್ರುಪೂರ್ಣಾಕುಲೇಕ್ಷಣಂ\\ವಿಷೀದಂತಮಿದಂ ವಾಕ್ಯಮುವಾಚ ಮಧುಸೂದನಃ~॥ 1~॥
\end{verse}

\begin{verse}
॥ ಶ‍್ರೀ ಭಗವಾನುವಾಚ~॥
\end{verse}

\begin{verse}
ಕುತಸ್ತ್ವಾ ಕಶ್ಮಲಮಿದಂ ವಿಷಮೇ ಸಮುಪಸ್ಥಿತಮ್​\\ಅನಾರ್ಯಜುಷ್ಟ ಮಸ್ವರ್ಗಮಕೀರ್ತಿಕರಮರ್ಜುನ~॥ 2~॥\\ಕ್ಲೈಬ್ಯಂ ಮಾಸ್ಮಗಮಃ ಪಾರ್ಥ ನೈತತ್ತ್ವಯ್ಯುಪಪದ್ಯತೇ\\ಕ್ಷುದ್ರಂ ಹೃದಯದೌರ್ಬಲ್ಯಂ ತ್ಯಕ್ತ್ವೋತ್ತಿಷ್ಠ ಪರಂತಪ~॥ 3~॥
\end{verse}

ಸಂಜಯ ಹೇಳಿದ: ಮರುಕ ಮತ್ತು ದುಃಖಗಳಿಂದ ಪರವಶನಾಗಿ ದುಃಖಾಶ್ರು ಗಳಿಂದ ಕೂಡಿದ ಅರ್ಜುನನಿಗೆ ಮಧುಸೂದನ ಹೀಗೆ ಹೇಳಿದನು.

ಶ‍್ರೀಕೃಷ್ಣನು ಹೀಗೆ ಹೇಳಿದನು: ಓ ಅರ್ಜುನ, ಇಂತಹ ವಿಷಮಸ್ಥಿತಿಯಲ್ಲಿ ನೀನು ಹೇಗೆ ಖಿನ್ನನಾಗಬಲ್ಲೆ? ಇದು ಅನಾರ್ಯರ ರೀತಿ, ಇದರಿಂದ ಸ್ವರ್ಗಪ್ರಾಪ್ತಿಯಿಲ್ಲ. ಕ್ಲೈಬ್ಯಕ್ಕೆ ವಶನಾಗದಿರು. ಕುಂತೀಕುಮಾರನೆ, ನಿನಗೆ ಇದು ಯೋಗ್ಯವಲ್ಲ. ವೈರಿ ಭಯಂಕರನೇ, ಹೃದಯ ದೌರ್ಬಲ್ಯದಿಂದ ಪಾರಾಗಿ ಮೇಲೇಳು.

“ತಂ ತಥಾ ಕೃಪಯಾವಿಷ್ಟಂ” ಎಂದು ಪ್ರಾರಂಭವಾಗುವ ಶ್ಲೋಕದಲ್ಲಿ ಎಷ್ಟು ಸುಂದರವಾಗಿ ಕಾವ್ಯರೀತಿಯಲ್ಲಿ ಅರ್ಜುನನ ನಿಜಸ್ಥಿತಿ ಚಿತ್ರಿತವಾಗಿರುವುದು! ಅನಂತರ ಕೃಷ್ಣನು ಅರ್ಜುನನಿಗೆ ಬೋಧನೆ ಮಾಡಿದನು. “ಕ್ಲೈಬ್ಯಂ ಮಾ ಸ್ಮ ಗಮಃ ಪಾರ್ಥ” ಮುಂತಾದ ಪದಗಳಿಂದ ಅರ್ಜುನನನ್ನು ಯುದ್ಧಕ್ಕೆ ಏತಕ್ಕೆ ಪ್ರೇರೇಪಿಸುತ್ತಿರುವನು? ಏಕೆಂದರೆ ಅರ್ಜುನ ಯುದ್ಧ ಮಾಡದೇ ಇರುವುದಕ್ಕೆ ಕಾರಣ ಅವನು ಸತ್ತ್ವಗುಣಾಧಿಕ್ಯ ನಾಗಿದ್ದನು ಎಂಬುದಲ್ಲ, ಅದೆಲ್ಲ ತಮಸ್ಸಿನಿಂದ ಪ್ರೇರಿತವಾದುದು. ಸತ್ತ್ವ ಗುಣಿಯ ಸ್ವಭಾವ ಜಯಾಪಜಯಗಳಲ್ಲಿ ಅನುದ್ವಿಗ್ನನಾಗಿರುವುದು. ಆದರೆ ಅರ್ಜುನನಿಗೆ ಅಂಜಿಕೆಯಾಗಿತ್ತು. ಅವನು ದೈನ್ಯದಿಂದ ಪ್ರೇರಿತನಾಗಿದ್ದನು. ಅವನು ಯುದ್ಧ ಮಾಡು ವುದಕ್ಕೆ ಬಂದಿದ್ದುದರಿಂದ ಯುದ್ಧಮಾಡುವುದು ಅವನ ಸ್ವಭಾವವಾಗಿತ್ತು ಮತ್ತು ಅದರಲ್ಲಿ ಆಸಕ್ತಿ ಇತ್ತು ಎನ್ನುವುದು ಗೊತ್ತಾಗುವುದು. ನಮ್ಮ ಜೀವನದಲ್ಲಿ ಕೂಡ ಕೆಲವು ವೇಳೆ ಇಂತಹ ಘಟನೆಗಳಾಗುವುವು. ಹಲವರು ತಾವು ಸಾತ್ತ್ವಿಕರೆಂದು ಭಾವಿಸುವರು. ಆದರೆ ಅದು ತಾಮಸವಲ್ಲದೇ ಬೇರೆ ಅಲ್ಲ. ಹಲವರು ಅಶುಚಿಯಾಗಿ ಜೀವಿಸಿ ತಾವು ಪರಮಹಂಸರೆಂದು ಭಾವಿಸುವರು. ಏಕೆಂದರೆ ಶಾಸ್ತ್ರ ಪರಮಹಂಸರು ಉನ್ಮತ್ತನಂತೆ, ಜಡನಂತೆ, ಪಿಶಾಚಿಯಂತೆ ಇರುವರೆಂದು ಹೇಳುವುದು. ಪರಮಹಂಸರನ್ನು ಶಿಶುಗಳಿಗೆ ಹೋಲಿಸುವರು, ಆದರೆ ಯಾವುದೊ ಒಂದು ಸ್ವಭಾವವನ್ನು ಮಾತ್ರ ಅವರು ಇಲ್ಲಿ ಹೋಲಿಸುವರೆಂದು ತಿಳಿಯಬೇಕು. ಪರಮಹಂಸರಿಗೂ ಮಕ್ಕಳಿಗೂ ವ್ಯತ್ಯಾಸವಿಲ್ಲದೆ ಇಲ್ಲ. ಇಬ್ಬರೂ ಒಂದು ಅತಿರೇಕದ ಕೊನೆಯಾಗಿರುವುದರಿಂದ ಸಮಾನವಾಗಿ ಕಾಣುವರು. ಒಬ್ಬ ಜ್ಞಾನವನ್ನು ಮೀರಿ ಹೋಗಿರುವನು. ಮತ್ತೊಬ್ಬನಿಗೆ ಎಳ್ಳಷ್ಟೂ ಜ್ಞಾನ ಬಂದಿಲ್ಲ. ಅತಿ ತೀವ್ರವಾದ ಮತ್ತು ಅತಿಮಂದವಾದ ಬೆಳಕಿನ ಸ್ಪಂದನಗಳೆರಡೂ ನಮ್ಮ ಕಣ್ಣಿಗೆ ಕಾಣುವುದಿಲ್ಲ. ಆದರೆ ಒಂದರಲ್ಲಿ ಶಾಖದ ಅತಿ ಇದೆ. ಮತ್ತೊಂದರಲ್ಲಿ ಶಾಖವೇ ಇಲ್ಲ. ಅದರಂತೆಯೇ ಪರಸ್ಪರ ವಿರೋಧವಾಗಿರುವ ಸತ್ತ್ವ ಮತ್ತು ತಮೋಗುಣಗಳು ಕೆಲವು ವಿಧಗಳಲ್ಲಿ ಒಂದೇ ಬಗೆಯಾಗಿ ತೋರಬಹುದು. ಆದರೆ ಅವುಗಳಲ್ಲಿ ಧ್ರುವಗಳಷ್ಟು ಅಂತರ ವಿದೆ. ತಮೋಗುಣ ಸತ್ತ್ವದ ಸೋಗಿನಲ್ಲಿ ಹೋಗಲಿಚ್ಛಿಸುವುದು. ವೀರಯೋಧ ಅರ್ಜುನನಲ್ಲಿ ಇದು ದಯೆಯ ರೂಪದಲ್ಲಿ ಬಂದಿರುವುದು.

ಅರ್ಜುನನ ಈ ಭ್ರಾಂತಿಯನ್ನು ಹೋಗಲಾಡಿಸಲು ಭಗವಂತನು ಏನು ಹೇಳಿದ? ಒಬ್ಬ ಮನುಷ್ಯನನ್ನು ಪಾಪಿ ಎಂದು ಎಂದಿಗೂ ತೆಗಳಕೂಡದು. ಅವನಲ್ಲಿರುವ ಅನಂತ ಶಕ್ತಿಯ ಕಡೆ ಅವನ ಮನಸ್ಸನ್ನು ಸೆಳೆಯಬೇಕೆಂದು ನಾನು ಯಾವಾಗಲೂ ಹೇಳು ವಂತೆ, ಭಗವಂತ ಅರ್ಜುನನಿಗೆ “ನಿನಗೆ ಇದು ತರವಲ್ಲ” ಎನ್ನುವನು. ನೀನು ಎಲ್ಲಾ ಪಾಪಗಳ ಆಚೆ ಇರುವ ಅವಿನಾಶಿಯಾದ ಆತ್ಮ. ನಿನ್ನ ನಿಜವಾದ ಸ್ವಭಾವವನ್ನು ಮರೆತು ಪಾಪಿ ಎಂದು ಭಾವಿಸಿ, ದೇಹ ಮನಸ್ಸಿನ ಉಪಾಧಿಗಳಿಗೆ ವಶನಾಗಿ ನೀನು ಹಾಗೇ ಆಗಿರುವೆ. ಆದಕಾರಣ ನಿನಗೆ ಇದು ಯೋಗ್ಯವಲ್ಲ ಎನ್ನುವನು. “ಎಂದಿಗೂ ಕ್ಲೈಬ್ಯಕ್ಕೆ ವಶನಾಗಬೇಡ, ಕುಂತೀಸುತ” ಪ್ರಪಂಚದಲ್ಲಿ ವ್ಯಸನವೂ ಇಲ್ಲ, ಪಾಪವೂ ಇಲ್ಲ, ರೋಗವೂ ಇಲ್ಲ, ಶೋಕವೂ ಇಲ್ಲ, ಪ್ರಪಂಚದಲ್ಲಿ ಪಾಪವೆನ್ನುವುದು ಏನಾದರೂ ಇದ್ದರೆ ಅದೇ ಅಂಜಿಕೆ, ಸುಪ್ತವಾದ ನಿನ್ನ ಶಕ್ತಿಯನ್ನು ಯಾವುದು ವ್ಯಕ್ತಗೊಳಿಸುವಂತೆ ಮಾಡುವುದೋ ಅದೇ ಪುಣ್ಯ. ಯಾವುದು ನಿನ್ನ ದೇಹ ಮತ್ತು ಮನಸ್ಸುಗಳನ್ನು ದುರ್ಬಲಗೊಳಿಸುವುದೊ ಅದೇ ನಿಜವಾದ ಪಾಪ. ಈ ದೌರ್ಬಲ್ಯ ದಿಂದ ಪಾರಾಗು, ಹೇಡಿತನದಿಂದ ಪಾರಾಗು. “ನೀನು ವೀರ, ನಿನಗೆ ಇದು ಯೋಗ್ಯವಲ್ಲ.”

ನನ್ನ ಮಕ್ಕಳಾದ ನೀವು “ಕ್ಲೈಬ್ಯಂ ಮಾ ಸ್ಮ ಗಮಃ ಪಾರ್ಥ ನೈತತ್ತ್ವ ಯ್ಯು ಪಪದ್ಯತೇ” ಎಂಬ ಸಂದೇಶವನ್ನು ಜಗತ್ತಿಗೆ ಸಾರಿದರೆ ಮೂರು ದಿನಗಳಲ್ಲಿ ರೋಗ, ಶೋಕ, ಪಾಪ, ವ್ಯಸನಗಳೆಲ್ಲ ಪ್ರಪಂಚದಿಂದ ಕಣ್ಮರೆಯಾಗುವುದು. ಈ ದೌರ್ಬಲ್ಯದ ಚಿಹ್ನೆಗಳೇ ಇರುವುದಿಲ್ಲ. ಈ ಅಂಜಿಕೆಯ ಭಾವನಾತರಂಗ ಈಗ ಎಲ್ಲೆಲ್ಲಿಯೂ ಇರುವುದು. ಇದಕ್ಕೆ ವಿರೋಧವಾದ ಭಾವನೆಗಳನ್ನು ತನ್ನಿ, ಅನಂತರ ನೋಡಿ, ಪ್ರಪಂಚದಲ್ಲಿ ಪವಾಡದಂತೆ ಆಗುವ ಬದಲಾವಣೆಯನ್ನು. ನೀನು ಸರ್ವಶಕ್ತ, ಹೋಗು, ಹೋಗು ಫಿರಂಗಿಯ ಬಾಯಿಗೆ ಬೇಕಾದರೆ, ಅಂಜಬೇಡ.

ಎಂತಹ ಅಧಮಾಧಮನನ್ನೂ ದ್ವೇಷಿಸಬೇಡ. ಅವನ ಬಾಹ್ಯವನ್ನು ನೋಡಬೇಡ. ಪರಮಾತ್ಮನಿರುವ ಅಂತರಂಗದ ಕಡೆ ನೋಡು. ಇಡೀ ಪ್ರಪಂಚಕ್ಕೇ ಶೌರ್ಯವಾಣಿಯಲ್ಲಿ ಸಾರಿ ಹೇಳಿ: “ನಿನ್ನಲ್ಲಿ ಪಾಪವಿಲ್ಲ, ನಿನ್ನಲ್ಲಿ ವ್ಯಸನವಿಲ್ಲ, ಅನಂತ ಶಕ್ತಿಯ ಅನಂತ ನೀನು.” ಏಳಿ, ಜಾಗ್ರತರಾಗಿ, ನಮ್ಮ ಅಂತರಂಗದ ಪವಿತ್ರತೆಯನ್ನು ವ್ಯಕ್ತಗೊಳಿಸಿ.

\begin{verse}
“ಕ್ಲೈಬ್ಯಂ ಮಾ ಸ್ಮ ಗಮಃ ಪಾರ್ಥ ನೈತತ್ತ್ವಯ್ಯುಪಪದ್ಯತೇ~।\\ಕ್ಷುದ್ರಂ ಹೃದಯದೌರ್ಬಲ್ಯಂ ತ್ಯಕ್ತ್ವೋತ್ತಿಷ್ಠ ಪರಂತಪ~॥
\end{verse}

ಒಬ್ಬ ಮೇಲಿನ ಶ್ಲೋಕವೊಂದನ್ನು ಓದಿದರೆ ಇಡೀ ಗೀತಾಪಾರಾಯಣದ ಫಲ ಬರುವುದು. ಈ ಒಂದು ಶ್ಲೋಕದಲ್ಲಿ ಗೀತಾಬೋಧನೆಯ ಸಾರವೆಲ್ಲ ಅಂತರ್ಗತವಾಗಿದೆ.

