
\chapter[ವೇದಾಂತ ]{ವೇದಾಂತ \protect\footnote{\engfoot{C.W. Vol. VI, P. 85}}}

ಹಿಂದೂ ದರ್ಶನದ ಮೂಲ ಸಿದ್ಧಾಂತಗಳು ವೇದಗಳಲ್ಲಿ ಬರುವ ಉಪಾಸನೆ ತತ್ತ್ವ ಮತ್ತು ನೀತಿ ಇವುಗಳ ತಳಹದಿಯ ಮೇಲೆ ನಿಂತಿವೆ. ವಿಶ್ವವು ಕಾಲ ದೇಶಗಳಲ್ಲಿ ಅನಂತವಾಗಿದೆ ಎಂದು ವೇದಗಳು ಸಾರುವುವು. ಅದಕ್ಕೆ ಒಂದು ಆದಿಯೂ ಇರಲಿಲ್ಲ. ಅದಕ್ಕೆ ಒಂದು ಅಂತ್ಯವೂ ಇರಲಾರದು. ಭೌತಿಕ ಪ್ರಪಂಚದಲ್ಲಿ ಆತ್ಮಶಕ್ತಿ ಹಲವು ವಿಧಗಳಲ್ಲಿ ಆವಿರ್ಭವಿಸಿದೆ. ಅನಂತವು ಸಾಂತದಲ್ಲಿ ಹಲವು ವಿಧಗಳಲ್ಲಿ ತೋರುತ್ತಿದೆ. ಆದರೆ ಅನಂತ ಮಾತ್ರ ಸ್ವಯಂಭು, ಸನಾತನ, ಅವಿಕಾರಿಯಾಗಿದೆ. ಕಾಲವು ಅನಂತದ ಮೇಲೆ ಯಾವ ಪರಿಣಾಮವನ್ನೂ ಉಂಟುಮಾಡಲಾರದು. ಮಾನವನ ತಿಳಿವಳಿಕೆಗೆ ಅತೀತವಾದುದು ಅದು, ಇಂದ್ರಿಯಾತೀತವಾದುದು ಅದು, ಅಲ್ಲಿ ಭೂತವೂ ಇಲ್ಲ, ಭವಿಷ್ಯವೂ ಇಲ್ಲ.

ಮಾನವನ ಆತ್ಮವು ಜನನ ಮರಣಾತೀತ ಎಂದು ವೇದಗಳು ಬೋಧಿಸುತ್ತವೆ. ದೇಹವು ವೃದ್ಧಿ ಕ್ಷಯಗಳ ನಿಯಮಕ್ಕೆ ಒಳಪಟ್ಟಿರುವುದು; ವೃದ್ಧಿಯಾಗುವುದು ಕ್ಷಯಿಸಲೇಬೇಕಾಗಿದೆ. ಆದರೆ ಅದರಲ್ಲಿರುವ ಆತ್ಮ ಅಭಂತವಾದುದು, ಸನಾತನವಾದುದು, ಅದಕ್ಕೊಂದು ಆದಿ ಇರಲಿಲ್ಲ, ಒಂದು ಅಂತ್ಯವೂ ಇರಲಾರದು. ವೈದಿಕ ಧರ್ಮಕ್ಕೂ ಕ್ರೈಸ್ತಧರ್ಮಕ್ಕೂ ಇರುವ ಒಂದು ವ್ಯತ್ಯಾಸವೇ, ಕ್ರೈಸ್ತ ಧರ್ಮವು ಮಾನವನು ಈ ಜಗತ್ತಿಗೆ ಬಂದಾಗ ಆತ್ಮ ಹುಟ್ಟಿತು ಎನ್ನುವುದು, ಆದರೆ ವೇದಾಂತಿಗಳಾದರೋ, ಅದು ಭಗವಂತನ ಒಂದು ಕಿರಣದಂತೆ,\break ದೇವರಿಗೆ ಹೇಗೆ ಒಂದು ಆದಿ ಇಲ್ಲವೊ, ಹಾಗೆಯೇ ಜೀವರಿಗೆ ಆದಿಯಿಲ್ಲ ಎನ್ನುವರು. ಕರ್ಮತತ್ತ್ವಕ್ಕೆ ಅನುಸಾರವಾಗಿ ಅದು ಹಲವು ಜನ್ಮಗಳನ್ನು ಅನುಭವಿಸಿಕೊಂಡು\break ಹೋಗುತ್ತಿದೆ. ಪೂರ್ಣತೆಯನ್ನು ಪಡೆದ ಮೇಲೆ ಅದರಲ್ಲಿ ಯಾವ ಬದಲಾವಣೆಯೂ ಇರುವುದಿಲ್ಲ.

