
\chapter[ಮಾನಸಿಕ ಅಥವಾ ಆಧ್ಯಾತ್ಮಿಕ ಸಂಶೋಧನೆಯ ಅಸ್ತಿಭಾರ ]{ಮಾನಸಿಕ ಅಥವಾ ಆಧ್ಯಾತ್ಮಿಕ ಸಂಶೋಧನೆಯ ಅಸ್ತಿಭಾರ \protect\footnote{\engfoot{C.W. Vol. IV, 192}}}

ಸ್ವಾಮಿ ವಿವೇಕಾನಂದರು ಪಾಶ್ಚಾತ್ಯ ದೇಶಗಳಲ್ಲಿದ್ದಾಗ ಚರ್ಚೆಗಳಲ್ಲಿ ಭಾಗ ವಹಿಸುತ್ತಿದ್ದುದು ಬಹಳ ಅಪರೂಪ. ಅವರು ಒಮ್ಮೆ ಲಂಡನ್ನಿನಲ್ಲಿದ್ದಾಗ ಅಪರೂಪ ವಾಗಿ ಒಂದು ಚರ್ಚೆಯಲ್ಲಿ ಭಾಗವಹಿಸಿದರು. ಆ ಚರ್ಚೆ ಮಾನಸಿಕ ಘಟನೆಗಳನ್ನು ವೈಜ್ಞಾನಿಕ ಶಾಸ್ತ್ರರೀತಿಯಲ್ಲಿ ನಿರ್ಣಯಿಸಬಹುದೇ ಎಂಬುದನ್ನು ಕುರಿತಾಗಿತ್ತು. ಅದರಲ್ಲಿ ಯಾವುದೊ ಒಂದು ವಿಷಯವನ್ನು ಕೇಳಿದರು. ಆ ವಿಷಯವನ್ನು ಕೇಳಿದ್ದು ಪಾಶ್ಚಾತ್ಯ ದೇಶಗಳಲ್ಲಿ ಅದೇ ಮೊದಲನೆ ಸಲವಲ್ಲ. ಆ ವಿಷಯವನ್ನು ಕುರಿತು ಅವರು ಈ ಕೆಳಕಂಡಂತೆ ಮಾತನಾಡಿದರು;

ನಾನು ಒಂದು ವಿಷಯವನ್ನು ನಿಮಗೆ ತಿಳಿಸಬೇಕೆಂದು ಇರುವೆನು. ಸ್ತ್ರೀಯರಿಗೆ ಆತ್ಮವಿಲ್ಲವೆಂದು ಮಹಮ್ಮದೀಯರು ನಂಬುತ್ತಾರೆ ಎಂದು ಭಾವಿಸುವುದು ತಪ್ಪು. ಇದು ಕ್ರೈಸ್ತರಲ್ಲಿರುವ ಒಂದು ಹಳೆಯ ದೋಷ. ಅವರು ಆ ದೋಷವನ್ನೇ ಪ್ರೀತಿಸು ತ್ತಾರೆ ಎಂದು ತೋರುವುದು. ಇದು ಮಾನವ ಸ್ವಭಾವದ ಒಂದು ದೋಷ. ಅವರಿಗೆ ಮತ್ತೊಬ್ಬರನ್ನು ಕಂಡರೆ ಆಗದೇ ಇದ್ದರೆ ಅವರ ವಿಷಯವಾಗಿ ಏನಾದರೂ ದೋಷಾರೋಪಣೆ ಮಾಡುವರು. ಅದೂ ಅಲ್ಲದೆ ನಿಮಗೆ ಗೊತ್ತಿದೆ ನಾನು ಮಹ ಮ್ಮದೀಯನಲ್ಲ ಎಂಬುದು. ಆದರೂ ನನಗೆ ಮಹಮ್ಮದೀಯರ ಧರ್ಮವನ್ನು ಚೆನ್ನಾಗಿ ತಿಳಿದುಕೊಳ್ಳುವ ಅವಕಾಶವಿತ್ತು.ಖೊರಾನಿನಲ್ಲಿ ಹೆಂಗಸರಿಗೆ ಆತ್ಮವಿಲ್ಲ ಎಂಬ ಒಂದು ನುಡಿಯೂ ಇಲ್ಲ. ಅದರ ಬದಲು ಅದು ಸ್ತ್ರೀಯರಿಗೆ ಆತ್ಮವಿದೆ ಎಂದು ಹೇಳುವುದು.

ನೀವು ಚರ್ಚಿಸುತ್ತಿರುವ ಅತೀಂದ್ರಿಯ ವ್ಯಾಪಾರಗಳ ವಿಷಯದಲ್ಲಾದರೋ ನನಗೆ ಹೇಳುವುದಕ್ಕೆ ಬಹಳ ಕಡಿಮೆ ವಿಷಯವಿದೆ. ಮೊದಲನೆಯದಾಗಿ ಅತೀಂದ್ರಿಯ ಘಟನೆಗಳನ್ನು ವೈಜ್ಞಾನಿಕವಾಗಿ ತೋರಿಸುವುದಕ್ಕೆ ಸಾಧ್ಯವೇ ಎಂಬುದು. ಹೀಗೆ ತೋರಿಸುವುದು ಎಂದರೆ ಅರ್ಥವೇನು? ಮೊದಲನೆಯದಾಗಿ ಹಾಗೆ ಮಾಡುವುದಕ್ಕೆ ಜ್ಞಾತೃ ಮತ್ತು ಜ್ಞೇಯ ಎಂಬ ಎರಡೂ ಆವಶ್ಯಕ. ನಿಮಗೇ ಅತ್ಯಂತ ಪರಿಚಿತವಾಗಿರುವ ರಸಾಯನಶಾಸ್ತ್ರ ಅಥವಾ ಭೌತಶಾಸ್ತ್ರ ವನ್ನೇ ತೆಗೆದುಕೊಂಡರೆ, ಅವುಗಳಲ್ಲಿ ಬರುವ ಸಾಮಾನ್ಯ ವಿಷಯಗಳನ್ನಾದರೂ ಪ್ರಪಂಚದಲ್ಲಿ ಇರುವ ಎಲ್ಲರೂ ಅರ್ಥ ಮಾಡಿಕೊಳ್ಳಲು ಸಾಧ್ಯವೇ? ಯಾರಾದರೂ ಒಬ್ಬ ದಡ್ಡನಿಗೆ ನಿಮ್ಮ ಕರಪ್ರಯೋಗವನ್ನು ಮಾಡಿತೋರಿಸಿ. ಅವನಿಗೆ ಏನು ಗೊತ್ತಾಗುವುದು? ಏನೂ ಇಲ್ಲ. ಒಂದು ಪ್ರಯೋಗವನ್ನಾದರೂ ಅರ್ಥಮಾಡಿ ಕೊಳ್ಳಬೇಕಾದರೆ ಬೇಕಾದಷ್ಟು ತರಬೇತಿಯನ್ನು ಒಬ್ಬ ಪಡೆದಿರಬೇಕು. ಇಲ್ಲದೇ ಇದ್ದರೆ ಅವನು ಅದನ್ನು ಅರ್ಥಮಾಡಿಕೊಳ್ಳಲಾರ. ಇದೇ ನಮ್ಮ ದಾರಿಯಲ್ಲಿರುವ ದೊಡ್ಡ ಕಷ್ಟ. ವೈಜ್ಞಾನಿಕ ಪ್ರದರ್ಶನ ಎಂದರೆ, ಎಲ್ಲರಿಗೂ ಅರ್ಥವಾಗುವ ರೀತಿಯಲ್ಲಿ ವೈಜ್ಞಾನಿಕ ಪ್ರದರ್ಶನಗಳನ್ನು ತೋರುವುದು. ಎಂದರೆ, ಈ ಪ್ರಪಂಚ ದಲ್ಲಿ ಯಾವ ವಿಷಯವನ್ನೂ ಹಾಗೆ ಪ್ರದರ್ಶಿಸಲು ಸಾಧ್ಯವಿಲ್ಲ ಎನ್ನುತ್ತೇನೆ ನಾನು. ಅದು ಹಾಗೆ ಇದ್ದರೆ ನಮ್ಮ ವಿಶ್ವವಿದ್ಯಾನಿಲಯಗಳು ಮತ್ತು ವಿದ್ಯಾಭ್ಯಾಸ ಎಲ್ಲವೂ ನಿಷ್ಪ್ರಯೋಜನವಾದಂತೆ ಆಯಿತು. ಈ ಪ್ರಪಂಚದಲ್ಲಿ ವೈಜ್ಞಾನಿಕವಾದುದನ್ನೆಲ್ಲ ನಮ್ಮ ಜನ್ಮಾರಭ್ಯದಿಂದಲೇ ನಾವು ಅರ್ಥಮಾಡಿಕೊಳ್ಳುವ ಹಾಗೆ ಇದ್ದರೆ, ನಾವು ವಿದ್ಯಾವಂತರಾಗುವ ಅವಶ್ಯಕತೆಯೇನು? ಇಷ್ಟೊಂದು ಓದುವುದು ಏತಕ್ಕೆ? ಇದರಿಂದ ಏನೂ ಪ್ರಯೋಜನವಿಲ್ಲ. ಬಹಳ ಸೂಕ್ಷ್ಮವಾದ ವಿಷಯಗಳನ್ನು ಈಗ ನಾವು ಇರುವ ಸ್ಥಿತಿಯಲ್ಲಿ ತಿಳಿದುಕೊಳ್ಳುವಂತೆ ಮಾಡುವುದೇ ವೈಜ್ಞಾನಿಕ ಪ್ರದರ್ಶನ ಎಂದಾದರೆ ಅದು ಅರ್ಥವಿಲ್ಲದಂತೆ ತೋರುವುದು. ಇನ್ನೊಂದು ಅರ್ಥ ಸರಿ ಇರಬೇಕು. ಅದಾವುದೆಂದರೆ, ಕೆಲವು ಸೂಕ್ಷ್ಮ ವಿಷಯಗಳನ್ನು ಸಮರ್ಥಿಸಲು ಕೆಲವು ಸ್ಥೂಲ ವಿಷಯಗಳನ್ನು ದೃಷ್ಟಾಂತವಾಗಿ ಕೊಡುವುದು. ಕೆಲವು ತುಂಬಾ ಸೂಕ್ಷ್ಮವಾದ ಜಟಿಲವಾದ ವಿಷಯಗಳಿವೆ. ಅವುಗಳನ್ನು ನಾವು ಸುಲಭ ಗ್ರಾಹ್ಯ ವಾದ ವಿಷಯಗಳ ಮೂಲಕ ವಿವರಿಸಲು ಯತ್ನಿಸುತ್ತೇವೆ. ಇದರಿಂದ ನಾವು ಆ ಸತ್ಯಕ್ಕೆ ಹತ್ತಿರ ಬರಬಹುದು. ಇದರಿಂದ ನಮ್ಮ ಈಗಿನ ಮನೋಭೂಮಿಕೆಯ ಸ್ಥಿತಿಗೆ ವಿಷಯಗಳನ್ನು ಕ್ರಮೇಣ ತರಬಹುದು. ಆದರೆ ಇದೂ ಕೂಡ ಬಹಳ ಜಟಿಲವಾದ, ಕಷ್ಟವಾದ ಸಮಸ್ಯೆ. ಇದಕ್ಕೂ ತರಬೇತಿಬೇಕಾಗುವುದು. ಆದ ಕಾರಣ ಅತೀಂದ್ರಿಯ ಘಟನೆಗಳಿಗೆ ವೈಜ್ಞಾನಿಕ ವಿವರಣೆ ನಮಗೆ ಬೇಕಾದರೆ, ಆ ಘಟನೆಗಳಿಗೆ ಸಂಬಂಧ ಪಟ್ಟಂತೆ ಸೂಕ್ತವಾದ ಪ್ರಮಾಣಗಳು ಬೇಕಾದುದು ಮಾತ್ರವಲ್ಲದೆ, ಅವನ್ನು ಯಾರು ನೋಡಬೇಕೆಂದು ಬಯಸುವರೋ ಅವರಿಗೆ ಕೂಡ ಸಾಕಾದಷ್ಟು ತರಬೇತು ಇದೆಲ್ಲಾ ಇದ್ದರೆ ಅದು ಸಾಧ್ಯ. ಇಲ್ಲದೆ ಇದ್ದರೆ ಇಲ್ಲ ಎಂದು ಹೇಳಬಹುದು. ಅದಕ್ಕೆ ಮುಂಚೆ ಬಹಳ ಅದ್ಭುತವಾದುದನ್ನು ಆಗಲಿ ಅಥವಾ ಮತ್ತೆ ಮತ್ತೆ ವರದಿಯಾಗುತ್ತಿರುವ ಘಟನೆಗಳನ್ನು ಆಗಲಿ ಸುಮ್ಮನೆ ಸಮರ್ಥಿಸುವುದಕ್ಕೆ, ನನ್ನ ದೃಷ್ಟಿಯಲ್ಲಿ ಸಾಧ್ಯವೇ ಇಲ್ಲ.

ಧರ್ಮಗಳು ಸ್ವಪ್ನದಿಂದ ಜನಿಸಿದವು ಎನ್ನುವಂತಹುದೆಲ್ಲ ದುಡುಕಿನ ವಿವರಣೆ ಗಳು. ಯಾರು ಈ ವಿಷಯಗಳನ್ನು ಕೂಲಂಕಷವಾಗಿ ವಿಚಾರ ಮಾಡಿರುವರೊ ಅವರು ಇಂತಹ ವಿವರಣೆಗಳೆಲ್ಲ ಕೇವಲ ಊಹೆಗಳು ಎಂದು ಹೇಳುವರು ಬಹಳ ಸುಲಭವಾಗಿ ವಿವರಿಸಿ ಬಿಟ್ಟಿರುವಂತೆ, ಧರ್ಮವು ಸ್ವಪ್ನದ ಫಲ ಎಂದು ನಂಬು ವುದಕ್ಕೆ ಸಾಧ್ಯವಿಲ್ಲ.ಆಗ ನಾವು ಬೇಕಾದರೆ ಅಜ್ಞೇಯವಾದಿಯ ಸ್ಥಾನದಲ್ಲಿ ಸುಲಭ ವಾಗಿ ನಿಲ್ಲಬಹುದು. ಆದರೆ ದುರದೃಷ್ಟವಶಾತ್​ ವಿಷಯವನ್ನು ನಾವು ಅಷ್ಟು ಸುಲಭವಾಗಿ ವಿವರಿಸಲು ಆಗುವುದಿಲ್ಲ. ಈಗಲೂ ಕೂಡ ಬಹಳ ಅದ್ಭುತವಾದ ಘಟನೆಗಳು ನಡೆಯುತ್ತಾ ಇವೆ. ನಾವು ಇವುಗಳನ್ನೆಲ್ಲ ಚೆನ್ನಾಗಿ ಪರೀಕ್ಷಿಸ ಬೇಕಾಗಿದೆ. ಈಗಲೂ ಕೂಡ ಅವು ಪರೀಕ್ಷಿಸಲ್ಪಡುತ್ತಿವೆ. ಕುರುಡ ಸೂರ್ಯನಿಲ್ಲ ಎಂದು ಹೇಳುತ್ತಾನೆ. ಆದರೆ ಇದರಿಂದ ಸೂರ್ಯನಿಲ್ಲ ಎನ್ನುವುದು ಸತ್ಯ ವಾಗಿ ಕಂಡುಬರುವುದಿಲ್ಲ. ಈ ಘಟನೆಗಳನ್ನೆಲ್ಲಾ ಹಲವು ವರುಷಗಳಹಿಂದೆಯೇ ಪರೀಕ್ಷಿಸಿ ನೋಡಿದ್ದಾರೆ. ನಮ್ಮ ನರಗಳು ಹೇಗೆ ಕೆಲಸ ಮಾಡುತ್ತವೆ ಎಂಬುದನ್ನು ತಿಳಿಯಲು ಮಾನವ ವರ್ಗವು ಹಲವು ಶತಮಾನಗಳಿಂದ ಅದಕ್ಕಾಗಿ ತರಬೇತನ್ನು ಪಡೆದುಕೊಂಡಿದೆ. ಬಹಳ ಹಿಂದೆಯೇ ಅದರ ದಾಖಲೆಗಳೆಲ್ಲ ಪ್ರಕಟವಾಗಿವೆ. ಈ ವಿಷಯಗಳನ್ನು ಅಭ್ಯಾಸ ಮಾಡಲು ಕಾಲೇಜುಗಳನ್ನು ಸ್ಥಾಪಿಸಿರುವರು. ಈ ವಿಷಯಗಳನ್ನು ಸತ್ಯವೆಂದು ತೋರುವಂತಹ ಸ್ತ್ರೀಪುರುಷರು ಈಗಲೂ ಇರುವರು. ಇವುಗಳಲ್ಲಿ ಬೇಕಾದಷ್ಟು ಕಪಟವಿರಬಹುದೆಂಬುದನ್ನು ನಾನು ಒಪ್ಪಿಕೊಳ್ಳುತ್ತೇನೆ. ಆದರೆ ಇದು ಎಲ್ಲಿ ತಾನೇ ಇಲ್ಲ? ಯಾವುದಾದರೂ ಸಾಮಾನ್ಯ ವೈಜ್ಞಾನಿಕ ಘಟನೆಯನ್ನೇ ತೆಗೆದುಕೊಳ್ಳಿ.ವಿಜ್ಞಾನಿಗಳು ಮತ್ತು ಜನ ಸಾಧಾರಣರು ಇದು ಸಂಪೂರ್ಣ ನಿಜ ಎನ್ನುವ ಎರಡು ಮೂರು ವಿಷಯಗಳು ಮಾತ್ರ ಇವೆ. ಉಳಿದವುಗಳೆಲ್ಲ ಬರಿ ತಿರುಳಿಲ್ಲದ ಊಹೆ. ಅಜ್ಞೇಯತಾವಾದಿಯು ಯಾವುದನ್ನು ನಂಬುವುದಕ್ಕೆ ಮನಸ್ಸಿಲ್ಲವೊ ಅವುಗಳನ್ನು ಪರೀಕ್ಷಿಸಲು ಯಾವ ವಿಧಾನಗಳನ್ನು ಅನ್ವಯಿಸುವನೋ, ತಾನು ನಂಬುವ ವಿಜ್ಞಾನವನ್ನು ಪರೀಕ್ಷಿಸಲು ಅದೇ ವಿಧಾನಗಳನ್ನು ಬಳಸಲಿ. ತಕ್ಷಣವೇ ಅರ್ಧದಷ್ಟು ಬುಡಸಹಿತ ಕಿತ್ತು ಹೋಗುವುದು. ನಾವು ಮೂಢನಂಬಿಕೆಗಳ ಮೇಲೆ ಜೀವಿಸಬೇಕಾಗಿದೆ. ನಾವು ಈಗ ಎಲ್ಲಿರುವೆವೋ ಅಲ್ಲಿಯೇ ತೃಪ್ತರಾಗಿರುವುದಕ್ಕೆ ಸಾಧ್ಯವಿಲ್ಲ. ಸ್ವಾಭಾವಿಕವಾಗಿ ಬೆಳೆಯುವ ರೀತಿಯೇ ಇದು. ನಾವು ಒಂದು ಕಡೆ ಅಜ್ಞೇಯತಾವಾದಿಗಳಾಗಿ, ಮತ್ತೊಂದು ಕಡೆ ಮನಸ್ಸಿಗೆ ಬಂದುದನ್ನೆಲ್ಲ ನಂಬುವುದಕ್ಕೆ ಆಗುವುದಿಲ್ಲ. ಆದಕಾರಣವೇ ನಾವು ನಮ್ಮ ಮಿತಿಗೆ ಅತೀತರಾಗಿ ಹೋಗಬೇಕಾಗಿದೆ. ಯಾವುದು ಅಸಾಧ್ಯ ಎಂದು ಭಾವಿಸುವೆವೊ ಅದನ್ನು ತಿಳಿಯಲು ಹೋರಾಡ ಬೇಕಾಗಿದೆ. ಈ ಹೋರಾಟ ಮುಂದುವರಿಯಬೇಕಾಗಿದೆ.

ನನ್ನ ದೃಷ್ಟಿಯಲ್ಲಿ ನಾನು ಉಪನ್ಯಾಸಕನಿಗಿಂತ ಒಂದು ಹೆಜ್ಜೆ ಮುಂದೆ ಹೋಗುತ್ತೇನೆ. ಮುಕ್ಕಾಲು ಪಾಲು ಅತೀಂದ್ರಿಯ ಘಟನೆಗಳು-ಪ್ರೇತಗಳು ಶಬ್ದ ಮಾಡುವುದು, ಟೇಬಲನ್ನು ಬಡಿಯುವುದು, ಮುಂತಾದವುಗಳೆಲ್ಲ ಮಕ್ಕಳಾಟ. ಟೆಲಿಪತಿಯನ್ನು ಮಕ್ಕಳು ಮಾಡುವುದನ್ನು ನಾನು ನೋಡಿದ್ದೇನೆ. ಕೊನೆಯ ಉಪನ್ಯಾಸಕರು ದೂರದಲ್ಲಿ ಆಗುವುದನ್ನು ಕೇಳುವುದು ಮುಂತಾದುವನ್ನು ಮೇಲಿನ ಮಟ್ಟದ ಅನುಭವ ಎಂದರು; ಆದರೆ ಅವುಗಳೆಲ್ಲ ಅತೀಂದ್ರಿಯ ಅನುಭವದ ಸೋಪಾನ ಪಂಕ್ತಿಗಳೆಂದು ನಾನು ಹೇಳುತ್ತೇನೆ. ಇವುಗಳನ್ನೆಲ್ಲ ಪರೀಕ್ಷೆಮಾಡ ಬೇಕಾಗಿದೆ. ಮೊದಲನೆಯದಾಗಿ, ಮನಸ್ಸು ಆ ಸ್ಥಿತಿಗೆ ಬರಲು ಸಾಧ್ಯವೇ ಇಲ್ಲವೇ ಎನ್ನುವುದನ್ನು ನೋಡಬೇಕಾಗಿದೆ. ನನ್ನ ವಿವರಣೆ ಉಪನ್ಯಾಸಕರ ವಿವರಣೆಗಿಂತ ಬೇರೆಯಾಗಿದೆ. ಆದರೆ ನಾವು ಬಳಸುವ ಪದಗಳನ್ನು ಸರಿಯಾಗಿ ವಿವರಿಸಿದ ಮೇಲೆ ಒಪ್ಪಿಕೊಳ್ಳಬಹುದು. ಸತ್ತ ಮೇಲೆ ಈಗಿನ ಪ್ರಜ್ಞೆ ಇರುವುದೇ ಇಲ್ಲವೇ ಎಂಬ ಪ್ರಶ್ನೆ ಮುಖ್ಯವಲ್ಲ. ಏಕೆಂದರೆ ಈಗ ನಮಗೆ ಕಾಣುವ ಪ್ರಪಂಚಕ್ಕೂ ಆ ಪ್ರಜ್ಞೆಗೂ ಏನೂ ಸಂಬಂಧವಿಲ್ಲ. ಪ್ರಜ್ಞೆಯು ಆಸ್ತಿತ್ವದೊಡನೆಯೇ ಇರುವುದಿಲ್ಲ. ನನ್ನ ಸ್ವಂತ ದೇಹದಲ್ಲಿ ಮತ್ತು ನಮ್ಮಗಳೆಲ್ಲರ ದೇಹಗಳಲ್ಲಿ ಕೂಡ, ನಮ್ಮ ದೇಹದ ವಿಷಯದಲ್ಲಿ ಇರುವ ಪ್ರಜ್ಞೆ ಅತ್ಯಲ್ಪ ಎಂಬುದನ್ನು ನಾವೆಲ್ಲರೂ ಒಪ್ಪಿಕೊಳ್ಳ ಬೇಕಾಗಿದೆ. ಅದರಲ್ಲಿ ಬಹುಪಾಲು ನಮ್ಮ ಅರಿವಿನಲ್ಲಿಲ್ಲ. ಆದರೂ ಅದು ಇದೆ. ಉದಾಹರಣೆಗೆ ಯಾರಿಗೂ ತಮ್ಮ ಮಿದುಳಿನ ಪ್ರಜ್ಞೆಯೆ ಇಲ್ಲ. ನಾನು ನನ್ನ ಮಿದುಳನ್ನು ಎಂದೂ ನೋಡಿಲ್ಲ. ನನಗೆ ಅದರ ಪ್ರಜ್ಞೆಯೂ ಇಲ್ಲ. ಆದರೂ ಅದು ಇದೆ ಎಂಬುದು ನನಗೆ ಗೊತ್ತಿದೆ. ಆದಕಾರಣ ನಮಗೆ ಬೇಕಾಗಿರುವುದು ಪ್ರಜ್ಞೆಯಲ್ಲ, ಯಾವುದೋ ಒಂದು ಸ್ಥೂಲವಲ್ಲದುದರ ಅಸ್ತಿತ್ವ. ಆ ಜ್ಞಾನವನ್ನು ಈ ಜನ್ಮದಲ್ಲೇ ಪಡೆಯಬಹುದು. ಇತರ ವಿಜ್ಞಾನದ ವಿಷಯಗಳನ್ನು ಹೇಗೆ ಪಡೆದುಕೊಂಡು ಅದನ್ನು ಪ್ರಯೋಗಿಸುವರೋ ಅದರಂತೆಯೇ ಇದನ್ನು ಈ ಜೀವನದಲ್ಲಿಯೇ ಪಡೆದುಕೊಂಡು ಪ್ರಯೋಗಿಸಲಾಗಿದೆ. ನಾವು ಇವುಗಳನ್ನು ಗಮನಿಸಬೇಕಾಗಿದೆ. ಇಲ್ಲಿ ಇರುವವರು ಒಂದು ವಿಷಯವನ್ನು ಜ್ಞಾಪಕದಲ್ಲಿಡ ಬೇಕೆಂದು ಹೇಳುತ್ತೇನೆ. ಈ ವಿಷಯದಲ್ಲಿ ಅನೇಕ ವೇಳೆ ನಾವು ಭ್ರಮೆಗೆ ಒಳಗಾಗಿ ರುವೆವು. ಕೆಲವರು ಆಧ್ಯಾತ್ಮಿಕ ಸ್ವಭಾವಕ್ಕೆ ಸಂಬಂಧಿಸಿದ ಕೆಲವು ವಿಷಯಗಳನ್ನು ತೋರಿಸಬಹುದು. ಅವುಗಳನ್ನು ನಾವು ತಿರಸ್ಕರಿಸುತ್ತೇವೆ. ಏಕೆಂದರೆ ಅವು ಸತ್ಯವೆಂದು ನಮಗೆ ತೋರುವುದಿಲ್ಲ. ಅನೇಕ ವೇಳೆ ವಸ್ತುಸ್ಥಿತಿ ಸರಿಯಲ್ಲದೇ ಇರಬಹುದು. ಮತ್ತೆ ಅನೇಕ ಸಂದರ್ಭಗಳಲ್ಲಿ ನಾವು ಅವುಗಳನ್ನು ಪಡೆಯಲು ಯೋಗ್ಯರಾಗಿರುವೆವೆ ಎಂಬುದನ್ನು ಕುರಿತು ಆಲೋಚಿಸುವುದೇ ಇಲ್ಲ. ನಮ್ಮ ದೇಹ ಮತ್ತು ಮನಸ್ಸನ್ನು ಅಂತಹ ಅನುಭವಗಳನ್ನು ಸ್ವೀಕರಿಸಲು ಅಣಿಮಾಡಿರು ವೆವೆ ಎಂಬುದನ್ನೂ ಗಮನಿಸಿರುವುದಿಲ್ಲ.

