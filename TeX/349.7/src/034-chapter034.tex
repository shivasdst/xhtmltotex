
\chapter[ಸಾಕ್ಷಾತ್ಕಾರವೇ ಧರ್ಮ ]{ಸಾಕ್ಷಾತ್ಕಾರವೇ ಧರ್ಮ \protect\footnote{\engfoot{C.W. Vol. VI, P. 82}}}

ಮಾನವನು ದೇವರಿಗೆ ಕೊಡಬಹುದಾದ ಮಹಾ ನಾಮವೇ ಸತ್ಯ ಎಂಬುದು. ಸತ್ಯವು ಸಾಕ್ಷಾತ್ಕಾರದ ಫಲ. ಆದಕಾರಣ ಅದನ್ನು ನಿಮ್ಮೊಳಗೇ ಅರಸಿ. ಎಲ್ಲಾ ಶಾಸ್ತ್ರಗಳಿಂದ, ಆಕಾರಗಳಿಂದ ಪಾರಾಗಿ ನಿಮ್ಮ ಆತ್ಮವನ್ನು ಮಾತ್ರ ನೋಡಿ. ನಾವು ಗ್ರಂಥಗಳಿಂದ ಭ್ರಾಂತರಾಗುವೆವು, ಹುಚ್ಚರಾಗುವೆವು ಎನ್ನುವನು ಶ‍್ರೀಕೃಷ್ಣ. ಪ್ರಕೃತಿಯ ದ್ವಂದ್ವದಿಂದ ಪಾರಾಗಿ. ನೀವು, ಬಾಹ್ಯ ಆಚಾರ ತಂತ್ರಗಳು ಇವೇ ಧರ್ಮ ಎಂದು ಭಾವಿಸಿದರೆ ಬಂಧನದಲ್ಲಿರುವಿರಿ. ಇತರರಿಗೆ ಸಹಾಯ ಮಾಡಲು ಅವುಗಳಲ್ಲಿ ಭಾಗಿಯಾಗಿ. ಆದರೆ ಜೋಪಾನ, ಅವು ಬಂಧನವಾಗದಂತೆ ನೋಡಿಕೊಳ್ಳಿ. ಧರ್ಮ ಒಂದು, ಆದರೆ ಅದು ಹಲವು ಬಗೆಯಾಗಿ ವ್ಯಾಪಿಸಿದೆ. ಪ್ರತಿಯೊಬ್ಬರೂ ತಮ್ಮ ಬೋಧನೆಯನ್ನು ನೀಡಲಿ, ಆದರೆ ಅನ್ಯಧರ್ಮಗಳಲ್ಲಿ ತಪ್ಪು ಕಂಡುಹಿಡಿಯಬೇಡಿ. ನಿಮಗೆ ಸತ್ಯ ಬೇಕಾದರೆ ನೀವು ಎಲ್ಲಾ ಆಕಾರಗಳಿಂದಲೂ ಪಾರಾಗಬೇಕು. ಭಗವಂತನ ಜ್ಞಾನಾಮೃತವನ್ನು ಉನ್ಮತ್ತರಾಗುವವರೆಗೆ ಪಾನಮಾಡಿ. “ಸೋಹಂ-ನಾನೇ ಅವನು” ಎಂಬುದನ್ನು ಅರಿತವನು ಚಿಂದಿ ಬಟ್ಟೆಯನ್ನು ಹೊದ್ದುಕೊಂಡಿದ್ದರೂ ಸುಖಿ. ಅನಂತಶಕ್ತಿಯ ಮೂಲಕ್ಕೆ ಹೋಗಿ ಅನಂತಶಕ್ತಿಯನ್ನು ಹೊಂದಿಬನ್ನಿ. ಗುಲಾಮನಾದವನು ಸತ್ಯಾನ್ವೇಷಣೆಗೆ ಹೊರಡುವನು; ಮುಕ್ತನಾಗಿ ಹಿಂದಿರುಗುವನು.

