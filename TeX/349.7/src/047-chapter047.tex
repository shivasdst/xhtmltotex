
\chapter[ಯೋಜನಾ ಸಿದ್ಧಾಂತ ]{ಯೋಜನಾ ಸಿದ್ಧಾಂತ \protect\footnote{\engfoot{C.W. Vol. VI, P. 97}}}

ಪ್ರಕೃತಿಯಲ್ಲಿ ನಿಯಮಬದ್ಧವಾದ ಘಟನೆಗಳಿರುವುದನ್ನು ನೋಡಿ ದೇವರ ಮನಸ್ಸಿನಲ್ಲಿಯೂ ಅಂತಹ ಒಂದು ಯೋಜನೆ ಇದೆ ಎಂದು ಭಾವಿಸುವುದು, ದೇವರ ಸೌಂದರ್ಯ–ಶಕ್ತಿ–ಮಹಿಮೆಗಳನ್ನು ತೋರುವ ಒಂದು ಬಾಲಪಾಠದಂತೆ ಇದೆ. ಅದರಿಂದ ಮಕ್ಕಳನ್ನು ಧರ್ಮಮಾರ್ಗದ ಮತ್ತು ದೇವರ ಕಡೆಗೆ ತಿರುಗಿಸುವುದಕ್ಕೆ ಸಹಾಯವಾಗುವುದು. ಇದಲ್ಲದೆ ಅದರಿಂದ ಮತ್ತಾವ ಪ್ರಯೋಜನವೂ ಇಲ್ಲ. ಇದು ತರ್ಕಬದ್ಧವೂ ಅಲ್ಲ. ದಾರ್ಶನಿಕ ದೃಷ್ಟಿಯಿಂದ, ದೇವರನ್ನು ಸರ್ವಶಕ್ತ ಎನ್ನುವುದಕ್ಕೆ ಯಾವ ಆಧಾರವೂ ಇಲ್ಲ.

ಪ್ರಕೃತಿಯು ಸೃಷ್ಟಿನಿರ್ಮಾಣದಲ್ಲಿ ಭಗವಂತನಿಗೆ ಇರುವ ಶಕ್ತಿಯನ್ನು ತೋರಿದರೂ, ಸೃಷ್ಟಿನಿರ್ಮಾಣದಲ್ಲಿ ಅವನು ಒಂದು ಯೋಜನೆಯನ್ನು ಇಟ್ಟುಕೊಂಡಿರುವನೆಂಬುದು ಅವನ ದೌರ್ಬಲ್ಯವನ್ನೂ ತೋರುವುದು. ದೇವರು ಸರ್ವಶಕ್ತನಾದರೆ, ಒಂದು ಕಾರ್ಯವನ್ನು ಸಾಧಿಸಬೇಕೆಂದರೆ ಅವನಿಗೆ ಯಾವ ಯೋಜನೆಯಾಗಲೀ, ರಚನಾಕ್ರಮವಾಗಲಿ ಬೇಕಾಗಿಲ್ಲ; ಅವನು ಇಚ್ಛಿಸಿದರೆ ಸಾಕು, ಅದು ಸಿದ್ಧಿಸುವುದು. ಪ್ರಕೃತಿಯಲ್ಲಿ ದೇವರ ಯಾವ ಯೋಜನೆಯಾಗಲೀ ರಚನಾಕ್ರಮವಾಗಲಿ ಇಲ್ಲ.

ಭೌತಿಕ ಪ್ರಪಂಚವು ಮಾನವನ ಮಿತವಾದ ಪ್ರಜ್ಞೆಯ ಪರಿಣಾಮ. ಮಾನವನು ತನ್ನ ದಿವ್ಯತೆಯನ್ನು ಅರಿತಾಗ, ನಮಗೆ ಈಗ ಕಾಣಿಸುವಂತಿರುವ ಪ್ರಕೃತಿ, ದ್ರವ್ಯ ಇವೆಲ್ಲವೂ ಮಾಯವಾಗುವುವು.

ಯಾವುದಾದರೊಂದು ಗುರಿಯನ್ನು ಸಾಧಿಸಬೇಕಾದರೆ ಒಂದು ಭೌತಿಕ ಜಗತ್ತು ಇರಬೇಕು ಎಂಬುದು ಸರ್ವಾಂತರ್ಯಾಮಿಗೆ ಅವಶ್ಯಕವಲ್ಲ. ಒಂದು ವೇಳೆ ಹಾಗೆ ಇದ್ದರೆ ವಿಶ್ವವು ದೇವರಿಗೆ ಒಂದು ಮಿತಿಯನ್ನು ಕಲ್ಪಿಸಿದಂತೆ ಆಗುತ್ತಿತ್ತು. ಅವನ ಅಪ್ಪಣೆಯಂತೆ ಪ್ರಕೃತಿ ಇರುವುದು ಎಂದರೆ, ಅದು ಮಾನವನ್ನು ಪೂರ್ಣ ಮಾಡು ವುದಕ್ಕಾಗಲಿ ಅಥವಾ ಮತ್ತಾವುದಾದರೂ ಕಾರಣಕ್ಕಾಗಲಿ ದೇವರಿಗೆ ಅದು ಆವಶ್ಯಕ ಎಂದಂತೆ ಆಗುವುದಿಲ್ಲ.

ಈ ಸೃಷ್ಟಿ ಮಾನವನ ಆವಶ್ಯಕತೆಯೆ ಹೊರತು ದೇವರ ಆವಶ್ಯಕತೆಯಲ್ಲ. ಸೃಷ್ಟಿಯಲ್ಲಿ ಭಗವಂತನ ಯಾವ ಯೋಜನೆಯೂ ಇಲ್ಲ. ಅವನು ಸರ್ವಶಕ್ತನಾದರೆ ಅವನಿಗೆ ಹೇಗೆ ಒಂದು ಯೋಜನೆ ಇರಬಲ್ಲದು? ಅವನು ಏನನ್ನಾದರೂ ಸಾಧಿಸ ಬೇಕಾದರೆ ಅವನಿಗೆ ಒಂದು ಯೋಜನೆಯ ಅವಶ್ಯಕತೆಯಿದೆ ಎಂದು ಹೇಳಿದರೆ ಅವನಿಗೆ ಪರಿಮಿತಿಯನ್ನು ಕಲ್ಪಿಸಿದಂತೆ ಆಯಿತು; ಅವನ ಸರ್ವಶಕ್ತಿತ್ವಕ್ಕೆ ಭಂಗ ತಂದಂತೆ ಆಯಿತು.

ಉದಾಹರಣೆಗೆ ನೀವು ಒಂದು ದೊಡ್ಡ ನದಿಯ ತೀರಕ್ಕೆ ಬಂದರೆ ಅದನ್ನು ಒಂದು ಸೇತುವೆಯಿಲ್ಲದೆ ದಾಟುವುದಕ್ಕೆ ಆಗುವುದಿಲ್ಲ. ನೀವು ಒಂದು ಸೇತುವೆಯನ್ನು ಕಟ್ಟಬೇಕಾಗಿರುವುದೇ ನಿಮ್ಮ ಮಿತಿಯನ್ನು ತೋರುವುದು. ಅದು ನಿಮ್ಮಲ್ಲಿರುವ ಸೇತುವೆಯನ್ನು ಕಟ್ಟುವ ಶೌರ್ಯವನ್ನು ತೋರಿದರೂ, ಸೇತುವೆ ಇಲ್ಲದೆ ನೀವು ದಾಟಲಾರಿರಿ ಎಂಬ ದೌರ್ಬಲ್ಯವನ್ನು ತೋರುವುದು. ನಿಮಗೆ ಆ ಮಿತಿ ಇಲ್ಲದೇ ಇದ್ದರೆ ನೀವು ಆ ನದಿಯ ಮೇಲೆ ಸುಮ್ಮನೆ ನಡೆದುಕೊಂಡು ಹೋಗುತ್ತಿದ್ದಿರಿ. ಸೇತುವೆಯನ್ನು ಕಟ್ಟುವ ಆವಶ್ಯಕತೆಯೇ ಇರುತ್ತಿರಲಿಲ್ಲ. ಸೇತುವೆಯನ್ನು ಕಟ್ಟುವುದು ನಿಮ್ಮ ಶೌರ್ಯವನ್ನು ತೋರುವುದಕ್ಕಿಂತ ಹೆಚ್ಚಾಗಿ ಸೇತುವೆಯ ಸಹಾಯ ಇಲ್ಲದೆ ನೀವು ನದಿಯನ್ನು ದಾಟಲಾರಿರಿ ಎಂಬ ದೌರ್ಬಲ್ಯವನ್ನು ತೋರುವುದು. ಸೇತುವೆ ನಿಮ್ಮ ಪೌರುಷದ ಚಿಹ್ನೆಯಲ್ಲ. ದೌರ್ಬಲ್ಯದ ಚಿಹ್ನೆ.

ದ್ವೈತ ಅದ್ವೈತಗಳೆರಡೂ ಮುಖ್ಯವಾಗಿ ಒಂದೇ. ಅವುಗಳಲ್ಲಿರುವ ವ್ಯತ್ಯಾಸ ಬಾಹ್ಯ ವ್ಯಕ್ತತೆಯಲ್ಲಿ ಮಾತ್ರ. ದ್ವೈತಿಗಳು ತಂದೆ ಮಗ ಬೇರೆ ಬೇರೆ ಎಂದರೆ ಅದ್ವೈತಿಗಳು ಅವರು ಒಂದೇ ಎನ್ನುವರು. ಆವಿರ್ಭಾವದಲ್ಲಿ, ಪ್ರಕೃತಿಯಲ್ಲಿ, ದ್ವೈತ ವಿದೆ. ಅದ್ವೈತವೇ ನಿಜವಾಗಿ ಅಧ್ಯಾತ್ಮದ ಸಾರ.

ಭಗವಂತನನ್ನು ಸಾಕ್ಷಾತ್ಕಾರ ಮಾಡಿಕೊಳ್ಳಲು ತ್ಯಾಗಭಾವನೆಯು ಒಂದು ಸಾಧನವಾಗಿ ಎಲ್ಲಾ ಧರ್ಮಗಳಲ್ಲಿಯೂ ಇದೆ.

