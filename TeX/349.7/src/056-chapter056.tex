
\vspace{-0.6cm}

\chapter[ಕರ್ಮಯೋಗ ]{ಕರ್ಮಯೋಗ \protect\footnote{\engfoot{C.W. Vol. V, p. 239}}}

ದೈಹಿಕ ಮತ್ತು ಮಾನಸಿಕ ವಸ್ತುಗಳೆಲ್ಲದರಿಂದಲೂ ಆತ್ಮವನ್ನು ಬೇರ್ಪಡಿಸುವುದೇ ನಮ್ಮ ಗುರಿ. ಇದು ಸಿದ್ಧಿಸಿದ ಮೇಲೆ ಆತ್ಮ ತಾನು ಯಾವಾಗಲೂ ಏಕಾಂಗಿಯಾಗಿ ಇದ್ದೆ ಎಂಬುದನ್ನು ಅರಿಯುವುದು. ತನ್ನನ್ನು ಸುಖಿಯನ್ನಾಗಿ ಮಾಡುವುದಕ್ಕೆ ಮತ್ತಾವುದೂ ಬೇಕಾಗಿಲ್ಲ ಎಂದೂ ಗೊತ್ತಾಗುವುದು. ಎಲ್ಲಿಯವರೆವಿಗೂ ನಮ್ಮನ್ನು ಸುಖಿಗಳನ್ನಾಗಿ ಮಾಡುವುದಕ್ಕೆ ಮತ್ತೊಬ್ಬರ ಅಗತ್ಯವಿದೆಯೋ ಅಲ್ಲಿಯವರೆಗೆ ನಾವು ಗುಲಾಮರು. ಪುರುಷನು ತಾನು ಮುಕ್ತ, ತನ್ನನ್ನು ಪೂರ್ಣಗೊಳಿಸುವುದಕ್ಕೆ ಮತ್ತಾವುದೂ ಬೇಕಾಗಿಲ್ಲ, ಈ ಪ್ರಕೃತಿ ಅನಾವಶ್ಯಕ ಎಂದು ಅರಿತಾಗ ಕೈವಲ್ಯವನ್ನು ಪಡೆಯುವನು.

\vskip 3pt

ಮನುಷ್ಯರು ಹಣದಲ್ಲಿ ಆಸಕ್ತರಾಗಿರುವರು. ಅದನ್ನು ಪಡೆಯುವುದಕ್ಕೆ ಮತ್ತೊಬ್ಬರನ್ನು ಮೋಸಪಡಿಸಲೂ ಹಿಂಜರಿಯುವುದಿಲ್ಲ. ಆದರೆ ಮನುಷ್ಯರು ಈ ಸ್ವಭಾವವನ್ನು ತಡೆದರೆ ಕೆಲವೇ ವರ್ಷಗಳಲ್ಲಿ ತಾವು ಇಚ್ಛಿಸಿದರೆ ಕೋಟ್ಯಂತರ ರೂಪಾಯಿಗಳನ್ನು ಪಡೆಯಬಲ್ಲ ಅದ್ಭುತ ಶೀಲವನ್ನು ರೂಢಿಸಿಕೊಳ್ಳುವರು. ಆಗ ಅವರು ಇಡೀ ಜಗತ್ತನ್ನೆ ಆಳುವರು. ಆದರೆ ನಾವೆಲ್ಲಾ ಇದನ್ನರಿಯದ ಮೂಢರು!

\vskip 3pt

ನಮ್ಮ ತಪ್ಪನ್ನು ಸಾರಿಕೊಳ್ಳುವುದರಿಂದ ಆಗುವ ಪ್ರಯೋಜನವೇನು? ಹೀಗೆ ಮಾಡುವುದರಿಂದ ಆ ತಪ್ಪನ್ನು ಹೋಗಲಾಡಿಸಲು ಆಗುವುದಿಲ್ಲ. ಒಬ್ಬ ತಾನು ಮಾಡಿರುವುದನ್ನು ಅನುಭವಿಸಬೇಕಾಗಿದೆ. ಮುಂದೆ ಸಾಧ್ಯವಾದಷ್ಟು ಒಳ್ಳೆಯದನ್ನು ಮಾಡಲು ಪ್ರಯತ್ನಿಸಬೇಕು. ಜಗತ್ತು ತನ್ನ ಸಹಾನುಭೂತಿಯನ್ನು ತೋರುವುದು ಬಲಾಢ್ಯರಿಗೆ ಮತ್ತು ಶಕ್ತಿವಂತರಿಗೆ ಮಾತ್ರ.

\vskip 3pt

ಮಾನವಕೋಟಿಗೆ ಮತ್ತು ಜಗತ್ತಿಗೆ ಹೃತ್ಪೂರ್ವಕವಾಗಿ ಸ್ವತಂತ್ರವಾಗಿ ಅರ್ಪಿಸಿದ ಕರ್ಮ ಮಾತ್ರ ಯಾವ ವಿಧವಾದ ಬಂಧನವನ್ನೂ ತರಲಾರದು.

\vskip 3pt

ಯಾವ ವಿಧವಾದ ಕರ್ತವ್ಯವನ್ನೂ ನಾವು ನಿಕೃಷ್ಟ ದೃಷ್ಟಿಯಿಂದ ನೋಡಬಾರದು. ಕೀಳು ಕೆಲಸ ಮಾಡುವವನು ಉತ್ತಮವಾದ ಕೆಲಸ ಮಾಡುವವನಿಗಿಂತ ಕೀಳಲ್ಲ. ಮಾಡುವ ಕೆಲಸವನ್ನು ನೋಡಿ ನಾವು ಒಬ್ಬನನ್ನು ಅಳೆಯಬಾರದು. ಅವನು ಯಾವ ದೃಷ್ಟಿಯಿಂದ ಆ ಕರ್ಮವನ್ನು ಮಾಡುತ್ತಾನೆ ಎಂಬುದನ್ನು ನೋಡಬೇಕು. ಅವನು ಮಾಡುವ ದೃಷ್ಟಿ ಮತ್ತು ಅದನ್ನು ಮಾಡುವ ಶಕ್ತಿ ಇವೇ ಮನುಷ್ಯನ ಪರೀಕ್ಷೆಗೆ ಒರೆಗಲ್ಲು. ಬಹಳ ಅಲ್ಪಕಾಲದಲ್ಲಿ ಅತಿ ಸುಂದರವಾದ ಒಂದು ಜೊತೆ ಬೂಟ್ಸನ್ನು ಮಾಡುವ ಮೋಚಿ ಅವನ ಕರ್ತವ್ಯದ ದೃಷ್ಟಿಯಿಂದ ಮೇಲು. ವರುಷವೆಲ್ಲ ಕೆಲಸಕ್ಕೆ ಬಾರದ ಹರಟೆ ಹೊಡೆಯುವ ಪ್ರಾಧ್ಯಾಪಕನಿಗಿಂತ ಮೇಲು ಅವನು.

\vskip 3pt

ಪ್ರತಿಯೊಂದು ಕರ್ತವ್ಯವೂ ಪವಿತ್ರ; ಕರ್ತವ್ಯಪರಾಯಣತೆಯೆ ಶ್ರೇಷ್ಠ ಭಗವದುಪಾಸನೆ. ಬದ್ಧಜೀವಿಗಳನ್ನು, ಅಜ್ಞಾನದಲ್ಲಿ ನರಳುತ್ತಿರುವವರನ್ನು ಉದ್ಧರಿಸಲೆತ್ನಿಸುವುದು ಒಂದು ಮಹದುಪಕಾರವೇ ಸರಿ.

\vskip 3pt

ನಮಗೆ ಸಮೀಪವಾಗಿರುವ ನಮ್ಮ ಕರ್ತವ್ಯವನ್ನು ನಾವು ಚೆನ್ನಾಗಿ ಮಾಡುವುದರಿಂದ ಬಲಶಾಲಿಗಳಾಗುವೆವು. ಹೀಗೆ ಕೆಲಸಮಾಡಿ ಕ್ರಮೇಣ ನಮ್ಮ ಶಕ್ತಿಯನ್ನು ವೃದ್ಧಿಪಡಿಸಿಕೊಂಡ ಮೇಲೆ ಜೀವನದಲ್ಲಿ ಮತ್ತು ಸಮಾಜದಲ್ಲಿ ಯಾವುದನ್ನು ಅತಿ ಮುಖ್ಯ ಎನ್ನುವರೋ ಆ ಕೆಲಸಗಳನ್ನು ಮಾಡುವ ಸದವಕಾಶವೂ ನಮ್ಮ ಪಾಲಿಗೆ ಬರುವುದು.

\eject

ಪ್ರಕೃತಿಯ ನ್ಯಾಯ ಎಲ್ಲರಿಗೂ ಒಂದೇ ಸಮನಾಗಿ ಕಠೋರವಾಗಿರುವುದು,\break ನಿರ್ದಾಕ್ಷಿಣ್ಯವಾಗಿರುವುದು.

\vskip 3pt

ವ್ಯವಹಾರ ಚತುರರು ಜೀವನವನ್ನು ಒಳ್ಳೆಯದೆಂದಾಗಲಿ ಕೆಟ್ಟದ್ದೆಂದಾಗಲಿ ಹೇಳು\break ವುದಿಲ್ಲ.

\vskip 3pt

ಜೀವನದಲ್ಲಿ ಜಯಶೀಲನಾದ ಪ್ರತಿಯೊಬ್ಬನ ಹಿಂದೆಯೂ ಅದ್ಭುತವಾದ ಚಾರಿತ್ರ ಮತ್ತು ಸತ್ಯಸಂಧತೆ ಇವೆ. ಇವೇ ಅವನ ಅನುಪಮ ಜಯಕ್ಕೆ ಕಾರಣ. ಅವನು ಸಂಪೂರ್ಣ ನಿಃಸ್ವಾರ್ಥಿಯಲ್ಲದೆ ಇರಬಹುದು. ಆದರೂ ಅವನು ಅದರ ಕಡೆಗೆ ಸಾಗುತ್ತಿರುತ್ತಾನೆ. ಅವನು ಸಂಪೂರ್ಣವಾಗಿ ನಿಃಸ್ವಾರ್ಥಿಯಾಗಿದ್ದರೆ ಅವನು ಕ್ರಿಸ್ತ ಮತ್ತು ಬುದ್ಧ ಇವರಷ್ಟೇ ಅಪಾರ ಜಯವನ್ನು ಗಳಿಸುತ್ತಿದ್ದನು. ಅವನು ಎಷ್ಟುಮಟ್ಟಿಗೆ ನಿಃಸ್ವಾರ್ಥಪರನೋ ಅಷ್ಟೇ ಜಯವನ್ನು ಪಡೆಯುವುದನ್ನು ಎಲ್ಲಾ ಕಡೆಗಳಲ್ಲಿಯೂ ನಾವು ನೋಡುವೆವು.

\vskip 3pt

ಪ್ರಪಂಚದ ಮಹಾನಾಯಕರು ಮಾತಾಳಿಗಳ ಕ್ಷೇತ್ರಗಳಿಗಿಂತ ಉತ್ತಮವಾದ ಕಾರ್ಯಕ್ಷೇತ್ರದಲ್ಲಿ ನುರಿತವರು.

\vskip 3pt

ಈ ಜೀವನದಲ್ಲಿ ಯಾವ ಒಂದು ಕ್ರಿಯೆಯನ್ನೂ ಸಂಪೂರ್ಣವಾಗಿ ಶುದ್ಧವಾದದ್ದು, ಅಥವಾ ಸಂಪೂರ್ಣವಾಗಿ ಅಶುದ್ಧವಾದದ್ದು ಎಂದು ಹೇಳುವಂತಿಲ್ಲ. ಶುದ್ಧ, ಅಶುದ್ಧ ಎಂಬುದನ್ನು ಅಹಿಂಸಾತ್ಮಕವಾಗಲಾರದು. ನಾವು ಮತ್ತೊಬ್ಬರಿಗೆ ವ್ಯಥೆಕೊಡದೆ ಉಸಿರಾಡಲಾರೆವು. ಬಾಳಲಾರೆವು. ನಾವು ಊಟಮಾಡುವ ಪ್ರತಿಯೊಂದು ತುತ್ತು ಅನ್ನವೂ\break ಮತ್ತೊಂದು ಬಾಯಿಂದ ಕಸಿದುಕೊಂಡದ್ದು. ನಾವು ಇಲ್ಲಿರುವುದಕ್ಕೆ ಕಾರಣವೆ ಇತರ\-ರನ್ನು ಆಚೆಗೆ ನೂಕಿರುವುದಾಗಿರುವುದು. ಅವು ಮನುಷ್ಯರೋ ಪ್ರಾಣಿಗಳೋ ಸಸ್ಯಗಳೋ\break ಆಗಿರಬಹುದು. ಆದರೆ ಎಲ್ಲೋ ಯಾರನ್ನೋ ನಾವು ಆಚೆಗೆ ದಬ್ಬಬೇಕಾಗಿದೆ. ನಿಜಸ್ಥಿತಿ ಹೀಗಿರುವಾಗ ಕರ್ಮದ ಮೂಲಕ ಪರಿಪೂರ್ಣತೆಯನ್ನು ನಾವು ಪಡೆಯಲಿಕ್ಕಾಗುವುದಿಲ್ಲ ಎಂಬುದು ಸ್ವಾಭಾವಿಕವಾಯಿತು. ಪ್ರಪಂಚ ಇರುವತನಕ ನಾವು ಕೆಲಸ ಮಾಡಬಹುದು. ಆದರೂ ಈ ಕರ್ಮಜಾಲದಿಂದ ತಪ್ಪಿಸಿಕೊಳ್ಳಲು ಮಾರ್ಗವೇ ಇರುವುದಿಲ್ಲ. ನಾವು\break ಸುಮ್ಮನೆ ಕೆಲಸಮಾಡುತ್ತಾ ಹೋಗಬಹುದು. ಆದರೆ ಅದಕ್ಕೊಂದು ಕೊನೆಯೇ ಇರುವುದಿಲ್ಲ.

\vskip 3pt

ಯಾರು ಸ್ವತಂತ್ರವಾಗಿ ಪ್ರೀತಿಯಿಂದ ಕೆಲಸಮಾಡುವರೊ ಅವರು ಫಲವನ್ನು ಲೆಕ್ಕಿಸುವುದಿಲ್ಲ. ಆದರೆ ಗುಲಾಮನಿಗೆ ಚಾವಟಿಪೆಟ್ಟು ಬೀಳಬೇಕಾಗಿದೆ; ಆಳಿಗೆ ಸಂಬಳ ಬೇಕಾಗಿದೆ. ಇದರಂತೆಯೇ ಜೀವನವೆಲ್ಲ. ಉದಾಹರಣೆಗೆ ಸಾರ್ವಜನಿಕ ಜೀವನವನ್ನು ತೆಗೆದುಕೊಳ್ಳಿ. ಉಪನ್ಯಾಸ ಮಾಡುವವನಿಗೆ ಮಂದಿಯಿಂದ ಕರತಾಡನ ಇಲ್ಲವೆ, ಟೀಕೆಗಳು ಬೇಕಾಗಿವೆ. ನೀವು ಅವನಿಗೆ ಇವುಗಳಾವುದನ್ನೂ ಕೊಡದೆ ಒಂದು ಮೂಲೆಯಲ್ಲಿ ಕೂಡಿಸಿದರೆ ನೀವು ಅವನನ್ನು ಕೊಲ್ಲುತ್ತೀರಿ. ಏಕೆಂದರೆ ಅವನು ಇವುಗಳಿಲ್ಲದೇ ಇದ್ದರೆ ಜೀವಿಸಲಾರ. ಗುಲಾಮನಂತೆ ಕೆಲಸ ಮಾಡುವುದು ಎಂದರೆ ಇದೇ. ಇಂತಹ ಸ್ಥಿತಿಯಲ್ಲಿ ಏನನ್ನಾದರೂ ನಿರೀಕ್ಷಿಸುವುದು\break ಅವನ ಒಂದು ಸ್ವಭಾವವಾಗಿ ಹೋಗುವುದು. ಅನಂತರವೇ ಆಳಿನ ಕೆಲಸ ಬರುವುದು.\break ಅವನಿಗೆ ಸ್ವಲ್ಪ ಸಂಬಳ ಬೇಕಾಗಿದೆ. ನಾನು ಅವನಿಗೆ ಸಂಬಳ ಕೊಟ್ಟರೆ ಅವನು ಸ್ವಲ್ಪ ಕೆಲಸವನ್ನು ಮಾಡುವನು. ನಾನು ಕರ್ಮಕ್ಕೋಸುಗ ಕರ್ಮ ಮಾಡುತ್ತೇನೆ ಎನ್ನುವುದು ಬಹಳ ಸುಲಭ. ಆದರೆ ಇದರಷ್ಟು ಕಷ್ಟ ಮತ್ತೊಂದು ಇಲ್ಲ. ಕೇವಲ ಕರ್ಮಕ್ಕಾಗಿ ಕರ್ಮ ಮಾಡುವವನನ್ನು ನೋಡುವುದಕ್ಕಾಗಿ ನಾನು ಕೈಕಾಲುಗಳ ಮೇಲೆ ತೆವಳಿಕೊಂಡು ಇಪ್ಪತ್ತು ಮೈಲಿಗಳಾದರೂ ಹೋಗಲು ಸಿದ್ಧನಾಗಿರುವೆನು. ಎಲ್ಲೋ ಒಂದು ಆಸೆ ಇರುತ್ತದೆ. ಅದು ಹಣಕ್ಕೋಸ್ಕರವಲ್ಲದೆ ಇದ್ದರೆ ಅಧಿಕಾರದ ಲಾಲಸೆ. ಅದು ಅಧಿಕಾರಕ್ಕಾಗಿಯಲ್ಲದೆ ಇದ್ದರೆ ಮತ್ತಾವುದೊ ಒಂದು ಲಾಭಕ್ಕಾಗಿ. ಎಲ್ಲೋ ಹೇಗೊ ಒಂದು ಆಸೆ ಹುದುಗಿಕೊಂಡಿದೆ. ನೀನು ನನ್ನ ಸ್ನೇಹಿತ, ಆದಕಾರಣ ನಿನ್ನೊಡನೆ ನಿನಗಾಗಿ ನಾನು ಕೆಲಸಮಾಡಲು ಇಚ್ಛಿಸುತ್ತೇನೆ. ಇದೆಲ್ಲ ಚೆನ್ನಾಗಿ ಏನೋ ಇರುವುದು. ಪ್ರತಿಕ್ಷಣವೂ ನನ್ನ ಪ್ರಾಮಾಣಿಕತೆಯನ್ನು ಕುರಿತು ಆಣೆ ಇಡುತ್ತೇನೆ. ಆದರೆ ಜೋಪಾನವಾಗಿರು. ನಾನು ಹೇಳುವುದನ್ನೆಲ್ಲ ನೀನು ಕೇಳಬೇಕು, ನೀನು ಹಾಗೆ ಕೇಳದೆ ಇದ್ದರೆ ನಾನು ನಿನ್ನನ್ನು ನೋಡಿಕೊಳ್ಳುವುದಿಲ್ಲ, ನಿನಗಾಗಿ ಕೆಲಸ ಮಾಡುವುದಿಲ್ಲ ಎನ್ನುವೆವು. ಉದ್ದೇಶದಿಂದ ಪ್ರೇರಿತವಾದ ಇಂತಹ ಕೆಲಸ ದುಃಖವನ್ನು ತರುವುದು. ನಾವು ನಮ್ಮ ಮನಸ್ಸಿನ ಯಜಮಾನನಂತೆ ಕೆಲಸ ಮಾಡಿದಾಗ ಮಾತ್ರ ಕರ್ಮವು ಅನಾಸಕ್ತಿಯನ್ನು ಮತ್ತು ಆನಂದವನ್ನು ತರುವುದು.

ಪ್ರಪಂಚದಲ್ಲಿ ಇದು ಸರಿಯೆ ತಪ್ಪೆ ಎಂದು ನಿರ್ಧರಿಸುವುದಕ್ಕಿರುವ ಒಂದು ಪ್ರಮಾ\-ಣವು ನಾನಲ್ಲ ಎಂಬುದೇ ನಾವು ಕಲಿಯಬೇಕಾದ ದೊಡ್ಡ ಪಾಠ. ಪ್ರತಿಯೊಬ್ಬರನ್ನೂ\break ಅವರವರ ಆದರ್ಶದ ದೃಷ್ಟಿಯಿಂದ ಅಳೆಯಬೇಕು. ಪ್ರತಿಯೊಂದು ಜನಾಂಗವನ್ನೂ\break ಅದರದರ ಆದರ್ಶದ ದೃಷ್ಟಿಯಿಂದ ಅಳೆಯಬೇಕು. ಆಯಾಯಾ ದೇಶದ ಲೋಕಾಚಾರಗಳನ್ನು ಅವರವರ ವಾತಾವರಣ, ಅವರವರ ವಿವೇಚನೆ ಇವುಗಳ ದೃಷ್ಟಿಯಿಂದಲೇ\break ಅಳೆಯಬೇಕು. ಅಮೆರಿಕಾ ದೇಶದ ಆಚಾರ ವ್ಯವಹಾರಗಳಿಗೆ ಕಾರಣ ಅಲ್ಲಿಯ ಜನರು ಜೀವಿಸುವ ವಾತಾವರಣ. ಇದರಂತೆಯೇ ಚೈನಾ, ಜಪಾನ್​, ಇಂಗ್ಲೆಂಡ್​ ಮತ್ತು ಇತರ\break ದೇಶಗಳೂ.

\vskip 3pt

ನಮಗೆ ಯೋಗ್ಯವಾದ ಸ್ಥಳಗಳಲ್ಲಿ ನಾವು ಇರುವೆವು. ಪ್ರತಿಯೊಬ್ಬನೂ ತನಗೆ ತಕ್ಕ\break ಸ್ಥಾನವನ್ನು ಕಂಡುಕೊಳ್ಳುತ್ತಾನೆ. ಒಬ್ಬನಿಗೆ ಮತ್ತೊಬ್ಬನಲ್ಲಿ ಇಲ್ಲದ ಕೆಲವು ಶಕ್ತಿಗಳಿದ್ದರೆ ಪ್ರಪಂಚ ಅದನ್ನು ಕಂಡು ಸಿಡಿಯುವುದು. ಈ ಪ್ರಪಂಚದ ಹೊಂದಾಣಿಕೆಯಲ್ಲಿ ಇದೆಲ್ಲ ಆಗುತ್ತಿರುವುದು. ಆದಕಾರಣ ಸುಮ್ಮನೆ ಗೊಣಗಾಡಿದರೆ ಪ್ರಯೋಜನವಿಲ್ಲ. ಒಬ್ಬ ದುರಾತ್ಮನಾದ ಶ‍್ರೀಮಂತನಿದ್ದರೆ ಅವನನ್ನು ಶ‍್ರೀಮಂತನನ್ನಾಗಿ ಮಾಡಿದ ಯಾವುದೊ ಒಂದು ಗುಣವಿರಬೇಕು. ಇತರರಲ್ಲಿಯೂ ಆ ಗುಣವಿದ್ದರೆ ಅವನೂ ಶ‍್ರೀಮಂತನಾಗುವನು. ಸುಮ್ಮನೆ ಜಗಳವಾಡಿ, ದೂರಿ ಪ್ರಯೋಜನವೇನು? ಇದರಿಂದ ಪರಿಸ್ಥಿತಿ ಉತ್ತಮವಾಗುವುದಿಲ್ಲ. ಯಾರು ತಮ್ಮ ಭಾಗಕ್ಕೆ ಬಂದ ಸಣ್ಣ ಕೆಲಸಗಳಿಗೆ ಗೊಣಗುತ್ತಾರೆಯೊ ಅವರು\break ಎಲ್ಲದಕ್ಕೂ ಗೊಣಗಾಡುವರು; ಯಾವಾಗಲೂ ಗೊಣಗಾಡುತ್ತಾ ದುಃಖದಲ್ಲೇ ಜೀವಿಸುವರು. ಅವರು ಮುಟ್ಟಿದ್ದೆಲ್ಲಾ ಹಾಳು. ಆದರೆ ಒಬ್ಬನು ತನಗೆ ಬಂದ ಕರ್ತವ್ಯವನ್ನು ಹೆಗಲುಕೊಟ್ಟು ಮಾಡುತ್ತಾ ಹೋದರೆ ಅವನಿಗೆ ಜ್ಞಾನ ಹೊಳೆಯುವುದು. ಉತ್ತಮೋತ್ತಮ ಕರ್ತವ್ಯಗಳು ಅವನ ಪಾಲಿಗೆ ಬರುವುವು.

\vspace{-0.6cm}

