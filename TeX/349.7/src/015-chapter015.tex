
\chapter[ನನ್ನ ಗುರುದೇವರು ]{ನನ್ನ ಗುರುದೇವ\protect\footnote{೧೮೯೬ರಲ್ಲಿ ನ್ಯೂಯರ್ಕ್​ ಮತ್ತು ಇಗ್ಲೆಂಡ್​ಗಳಲ್ಲಿ ನೀಡಿದ ಎರಡು ಉಪನ್ಯಾಸಗಳನ್ನು ಒಟ್ಟಿಗೆ ಸೇರಿಸಿ ಪ್ರಕಟಿಸಲಾಗಿದೆ. \engfoot{C.W. Vol. IV, P. 154}}}

“ಧರ್ಮಕ್ಕೆ ಎಂದು ಗ್ಲಾನಿ ಬರುವುದೊ, ಅಧರ್ಮ ಎಂದು ಮೇಲೇಳುವುದೊ ಆಗ ಮಾನವಕೋಟಿಯ ಉದ್ಧಾರಕ್ಕಾಗಿ ನಾನು ಜನ್ಮವೆತ್ತುವೆನು” ಎಂದು ಶ‍್ರೀಕೃಷ್ಣ ಭಗವದ್ಗೀತೆಯಲ್ಲಿ ಸಾರಿರುವನು. ನಮ್ಮ ಜಗತ್ತು ವಿಕಾಸವಾಗುತ್ತಿರುವಾಗ, ಹಲವಾರು ಸಮಸ್ಯೆಗಳನ್ನು ನಾವು ಎದುರಿಸುತ್ತಿರುವಾಗ, ದೇಶದಲ್ಲಿ ಹೊಸ ಸುವ್ಯವಸ್ಥೆ ಬೇಕಾದಾಗ ಒಂದು ಶಕ್ತಿತರಂಗ ಏಳುವುದು. ಮಾನವನು ಆಧ್ಯಾತ್ಮಿಕ ಮತ್ತು ಭೌತಿಕ ಎಂಬ ಎರಡು ಕ್ಷೇತ್ರಗಳಲ್ಲಿ ಕೆಲಸ ಮಾಡುತ್ತಿರುವುದರಿಂದ ಸುವ್ಯವಸ್ಥೆಗೊಳಿಸುವ ತರಂಗ ಎರಡು ಕ್ಷೇತ್ರಗಳಿಂದಲೂ ಬರುವುದು. ಒಂದೆಡೆ ಭೌತಕ್ಷೇತ್ರದ ವ್ಯವಸ್ಥೆಗೆ ಆಧುನಿಕ ಕಾಲದಲ್ಲಿ ಯೂರೋಪ್​ ದೇಶ ಮುಖ್ಯ ಆಧಾರವಾಗಿದೆ. ಆಧ್ಯಾತ್ಮಿಕ ಕ್ಷೇತ್ರದಲ್ಲಿ ಮತೊಂದು ಸುವ್ಯವಸ್ಥೆ ಬೇಕಾಗಿದೆ. ಅದನ್ನು ಜಗತ್ತಿನ ಇತಿಹಾಸದುದ್ದಕ್ಕೂ ಏಷ್ಯಖಂಡವು ನೀಡುತ್ತಾ ಬಂದಿದೆ. ಇಂದು ಪ್ರಾಪಂಚಿಕ ಭಾವನೆಗಳ ಪ್ರಭಾವ ಮತ್ತು ಶಕ್ತಿಗಳು ಕೀರ್ತಿಯ ಅತ್ಯುನ್ನತ ಶಿಖರದಲ್ಲಿರುವಾಗ, ಮಾನವನ ಬಾಹ್ಯ ಜಗತ್ತಿನ ಮೇಲಿರುವ ಅವಲಂಬನೆ ಹೆಚ್ಚಾಗಿ ಆತನು ಕೇವಲ ಹಣವನ್ನು ತಯಾರು ಮಾಡುವ ಒಂದು ಯಂತ್ರವಾಗುವ ಅಧೋಗತಿಗೆ ಇಳಿದಿದ್ದಾನೆ, ಮತ್ತು ತನ್ನ ದೈವಿಕ ಸ್ವಭಾವವನ್ನು ಮರೆಯುವ ಸ್ಥಿತಿಯಲ್ಲಿ ಇದ್ದಾನೆ. ಈಗ ಆಧ್ಯಾತ್ಮಿಕ ಕ್ಷೇತ್ರದಲ್ಲಿ ಒಂದು ವ್ಯವಸ್ಥೆ ಆವಶ್ಯಕವಾಗಿದೆ. ಈಗ ಧ್ವನಿಯೊಂದು ಕೇಳಿಬರುತ್ತಿದೆ. ಮುತ್ತುತ್ತಿರುವ ಜಡವಾದವೆಂಬ ಮುಗಿಲ ತಂಡವನ್ನು ಚೆದುರಿಸುವ ಶಕ್ತಿಯೊಂದು ಉದಯಿಸಿದೆ. ಅದು ಚಲಿಸುತ್ತಿದೆ. ಅದು ಪುನಃ ಮಾನವಕೋಟಿಗೆ ತನ್ನ ನೈಜಸ್ವಭಾವವನ್ನು ತೋರುವ ಸಮಯ ಸನ್ನಿಹಿತವಾಗಿದೆ. ಪುನಃ ಈ ಶಕ್ತಿಯ ಕೇಂದ್ರ ಏಷ್ಯಾಖಂಡವಾಗುವುದು.

ನಮ್ಮ ಜಗತ್ತು ನಿಂತಿರುವುದು ಕರ್ಮ ವಿಭಜನೆಯ ಯೋಜನೆಯ ಮೇಲೆ. ಒಬ್ಬನೇ ಎಲ್ಲವನ್ನೂ ಪಡೆಯುತ್ತೇನೆ ಎಂದು ಭಾವಿಸುವುದು ಒಂದು ಭ್ರಾಂತಿ. ಆದರೂ ಎಷ್ಟು ಹುಡುಗಾಟ ನಮ್ಮದು! ಮಗ ತನ್ನ ಅಜ್ಞಾನದಲ್ಲಿ ಜಗತ್ತಿನಲ್ಲೆಲ್ಲ ಅತಿ ಜಾಗರೂಕತೆಯಿಂದ ಪಡೆದುಕೊಳ್ಳಬೇಕಾದ ವಸ್ತು ತನ್ನ ಆಟದ ಗೊಂಬೆ ಎಂದು ಭಾವಿಸುವುದು. ಅದರಂತೆಯೇ ಪ್ರಾಪಂಚಿಕ ಶಕ್ತಿ ಸಂಪತ್ತಿನಲ್ಲಿ ಅಧಿಕವಾದ ಒಂದು ಜನಾಂಗವೂ ಕೂಡ, ಬಹಳ\break ಜಾಗರೂಕತೆಯಿಂದ ಪಡೆಯಬೇಕಾದದ್ದು ಅದೊಂದೇ ಎಂದೂ, ಅದೇ ಪ್ರಗತಿಯ ಮತ್ತು ನಾಗರಿಕತೆಯ ಗುರುತೆಂದೂ ಬಾಹ್ಯ ಸಂಪತ್ತನ್ನು ಅಪೇಕ್ಷಿಸದೆ ಇರುವ ಇತರ ಜನಾಂಗಗಳಿದ್ದರೆ, ಅವು ಶಕ್ತಿಸಂಪತ್ತನ್ನು ಹೊಂದದೆ ಇದ್ದರೆ, ಅವು ಬದುಕುವುದಕ್ಕೆ ಯೋಗ್ಯವಲ್ಲ, ಅವು ಬಾಳಿ ಪ್ರಯೋಜನವಿಲ್ಲ ಎಂದೂ ಭಾವಿಸುತ್ತದೆ. ಇದರಂತೆಯೇ ಮತೊಂದು ಜನಾಂಗವು ಕೇವಲ ಲೌಕಿಕ ಸಂಪತ್ತಿನ ನಾಗರಿಕತೆಯಲ್ಲಿ ಎಳ್ಳಷ್ಟೂ ಪ್ರಯೋಜನವಿಲ್ಲವೆಂದು ತಿಳಿಯುವುದು. ಸೂರ್ಯನ ಕೆಳಗೆ ಇರುವ ಪ್ರತಿಯೊಂದು ವಸ್ತುವನ್ನು ಹೊಂದಿದ್ದರೂ, ಅಧ್ಯಾತ್ಮವಿಲ್ಲದೇ ಇದ್ದರೆ ಪ್ರಯೋಜನವೇನು ಎಂದು ಪ್ರಾಚ್ಯ ದೇಶದಿಂದ ಬಂದ ಧ್ವನಿಯೊಂದು ಹಿಂದೆ ಜಗತ್ತನ್ನು ಕೇಳಿತು. ಇದು ಪ್ರಾಚ್ಯ ರೀತಿ, ಮತ್ತೊಂದು ಪಾಶ್ಯಾತ್ಯ ರೀತಿ.

ಎರಡು ಆದರ್ಶಗಳಲ್ಲಿಯೂ ಒಂದು ವೈಭವವಿದೆ, ಮಹಿಮೆಯಿದೆ. ಆಧುನಿಕ ವ್ಯವಸ್ಥೆ ಈ ಎರಡು ಆದರ್ಶಗಳನ್ನು ಸಂಗಮಗೊಳಿಸಿ ಸಮನ್ವಯಗೊಳಿಸುವುದು. ಪಾಶ್ಚಾತ್ಯರಿಗೆ ಇಂದ್ರಿಯ ಪ್ರಪಂಚ ಎಷ್ಟು ಸತ್ಯವೊ ಅಷ್ಟೇ ಸತ್ಯ ಪ್ರಾಚ್ಯರಿಗೆ ಆಧ್ಯಾತ್ಮಿಕ ಪ್ರಪಂಚ. ಪ್ರಾಚ್ಯನಿಗೆ ಏನು ಬೇಕೊ, ಯಾವುದನ್ನು ಕುರಿತು ಅವನು ಹಂಬಲಿಸುತ್ತಿರುವನೊ, ಯಾವುದು ಜೀವನವನ್ನು ಸತ್ಯಮಯವಾಗಿ ಮಾಡಬಲ್ಲದೊ, ಅವೆಲ್ಲವನ್ನೂ ಅವನು ಅಧ್ಯಾತ್ಮದಲ್ಲಿ ಕಂಡುಕೊಳ್ಳುತ್ತಾನೆ. ಪಾಶ್ಯಾತ್ಯ ದೃಷ್ಟಿಯಲ್ಲಿ ಪ್ರಾಚ್ಯನು ಒಬ್ಬ ಕನಸುಣಿ. ವಯಸ್ಸಾದ ಸ್ತ್ರೀಪುರುಷರು ಎಂದಾದರೊಂದು ದಿನ ವೇಗವಾಗಿಯೊ ನಿಧಾನವಾಗಿಯೊ ಬಿಟ್ಟುಹೋಗಬೇಕಾದ ಒಂದು ಹಿಡಿ ಮಣ್ಣಿಗೆ ಇಷ್ಟು ಪ್ರಾಧಾನ್ಯ ಕೊಡುವುದನ್ನು ನೋಡಿ ಪ್ರಾಚ್ಯನು ನಗುವನು. ಒಬ್ಬನು ಮತ್ತೊಬ್ಬನನ್ನು ಕನಸುಣಿ ಎಂದು ಕರೆಯುವನು. ಮಾನವ ಜನಾಂಗದ ಪ್ರಗತಿಗೆ ಪಾಶ್ಚಾತ್ಯ ಆದರ್ಶ ಎಷ್ಟು ಮುಖ್ಯವೊ ಪ್ರಾಚ್ಯ ಆದರ್ಶವೂ ಅಷ್ಟೇ ಮುಖ್ಯ- ಅದು ಹೆಚ್ಚು ಆವಶ್ಯಕವೆಂದು ಭಾವಿಸುತ್ತೇನೆ. ಯಂತ್ರಗಳು ಮಾನವ ಜನಾಂಗವನ್ನು ಹಿಂದೆ ಎಂದೂ ಸುಖಿಯನ್ನಾಗಿ ಮಾಡಿಲ್ಲ, ಮುಂದೆ ಮಾಡುವಂತೆಯೂ ಇಲ್ಲ. ಯಾರು ನಾವು ಇದನ್ನು ನಂಬುವಂತೆ ಮಾಡುತ್ತಿರುವರೊ, ಅವರು ಸುಖವು ಯಂತ್ರದಲ್ಲಿದೆ ಎಂದು ಭಾವಿಸುವರು. ಆದರೆ ಸುಖವು ಯಾವಾಗಲೂ ಇರುವುದು ಮನಸ್ಸಿನಲ್ಲಿ. ಯಾವನು ತನ್ನ ಮನಸ್ಸಿನ ಒಡೆಯನಾಗಿರುವನೋ ಆ ವ್ಯಕ್ತಿಯೊಬ್ಬನೇ ಸುಖಿಯಾಗಬಲ್ಲ, ಉಳಿದವರಲ್ಲ. ಯಂತ್ರಶಕ್ತಿಯಲ್ಲೇನಿದೆ? ತಂತಿಯ ಮೂಲಕ ವಿದ್ಯುತ್​ ಪ್ರವಾಹವನ್ನು ಹರಿಸಬಲ್ಲ ಒಬ್ಬ ಮನುಷ್ಯನನ್ನು ದೊಡ್ಡ ವ್ಯಕ್ತಿಯೆಂದು, ಅತಿ ಬುದ್ಧಿವಂತನೆಂದು ಏತಕ್ಕೆ ನಾವು ಕರೆಯಬೇಕು? ಪ್ರತಿ ಕ್ಷಣದಲ್ಲಿಯೂ ಪ್ರಕೃತಿ ಇದಕ್ಕಿಂತಲೂ ಹತ್ತುಲಕ್ಷ ಹೆಚ್ಚು ಮಾಡುತ್ತಿಲ್ಲವೇನು? ಹೀಗಿರುವಾಗ ಪ್ರಕೃತಿಗೆ ಬಾಗಿ ಏತಕ್ಕೆ ಪೂಜಿಸಬಾರದು? ಜಗತ್ತೆಲ್ಲ ನಿಮ್ಮ ಅಧಿಕಾರಕ್ಕೆ ಒಳಪಟ್ಟಿದ್ದರೆ, ಅದರ ಪ್ರತಿಯೊಂದು ಕಣವೂ ನಿಮ್ಮ ಸ್ವಾಧೀನದಲ್ಲಿ ಇದ್ದರೆ ಬಂದ ಪ್ರಯೋಜನವೇನು? ನಿಮ್ಮನ್ನು ನೀವು ಗೆದ್ದು ಅಂತರಂಗ ಸುಖವನ್ನು ಪಡೆದುಕೊಳ್ಳುವ ಶಕ್ತಿಯನ್ನು ಪಡೆಯುವವರೆಗೆ ಅದು ನಿಮ್ಮನ್ನು ಸುಖಿಗಳನ್ನಾಗಿ ಮಾಡಲಾರದು. ಮಾನವನು ಪ್ರಕೃತಿಯನ್ನು ಗೆಲ್ಲುವುದಕ್ಕಾಗಿ ಹುಟ್ಟಿರುವುದು ಸಹಜ. ಆದರೆ ಪಾಶ್ಚಾತ್ಯನು ಪ್ರಕೃತಿ ಎಂದರೆ ಕೇವಲ ಭೌತಿಕ ಮತ್ತು ಬಾಹ್ಯ ಜಗತ್ತು ಎಂದು ತಿಳಿದಿದ್ದಾನೆ. ಬಾಹ್ಯ ಪ್ರಕೃತಿಯು ಪರ್ವತ ಸಾಗರ ನದಿ ಇವುಗಳ ಅನಂತ ಶಕ್ತಿಯಿಂದ ಮತ್ತು ವೈವಿಧ್ಯಗಳಿಂದ ಬಹು ವೈಭವಯುತವಾಗಿರುವುದು ನಿಜ. ಆದರೆ ಇದಕ್ಕಿಂತಲೂ ವೈಭವಯುತವಾಗಿ, ಸೂರ್ಯ ಚಂದ್ರ ತಾರೆಗಳನ್ನು ಮೀರಿ, ನಮ್ಮ ಕಿರಿಯ ಜೀವನಗಳಿಗೆ ಅತೀತವಾಗಿರುವ ಮಾನವನ ಆಂತರಿಕ ಸ್ವಭಾವವಿರುವುದು. ಇದು ಬೇರೊಂದು ಬಗೆಯ ಅಧ್ಯಯನವನ್ನು ನಮ್ಮ ಮುಂದೆ ತೋರುವುದು. ಅಲ್ಲಿ ಹೇಗೆ ಪಾಶ್ಚಾತ್ಯರು ತಮ್ಮ ಆದರ್ಶದಲ್ಲಿ ನಮ್ಮನ್ನು ಸೋಲಿಸುವರೋ ಹಾಗೆ ಪ್ರಾಚ್ಯರು ಇಲ್ಲಿ ಅವರನ್ನು ಮೀರಿಸುವರು.\break ಆದಕಾರಣ ಆಧ್ಯಾತ್ಮಿಕ ವ್ಯವಸ್ಥೆ ಆಗುವಾಗಲೆಲ್ಲಾ ಅದು ಪ್ರಾಚ್ಯ ದೇಶಗಳಿಂದ\break ಬರಬೇಕಾಗಿರುವುದು ನ್ಯಾಯವಾಗಿಯೇ ಇರುವುದು. ಪ್ರಾಚ್ಯರು ಯಂತ್ರ ಕುಶಲತೆಯನ್ನು ಕಲಿಯಬೇಕಾದರೆ ಪಾಶ್ಚಾತ್ಯರ ಪದತಳದಲ್ಲಿ ಕುಳಿತು ಅದನ್ನು ಕಲಿಯಬೇಕಾ\-ಗಿರುವುದು ಸರಿಯಾಗಿಯೇ ಇರುವುದು. ಪಾಶ್ಚಾತ್ಯರು ಅಧ್ಯಾತ್ಮ, ದೇವರು, ಆತ್ಮ, ಈ\break ಜಗತ್ತಿನ ರಹಸ್ಯ ಮತ್ತು ಅದರ ವಿವರಣೆ ಇವುಗಳನ್ನು ಪ್ರಾಚ್ಯರ ಪದತಳದಲ್ಲಿ ಕುಳಿತು\break ಕಲಿಯಬೇಕಾಗಿರುವುದು

ಭರತಖಂಡದಲ್ಲಿ ಅಂತಹ ಆಧ್ಯಾತ್ಮಿಕ ತರಂಗವನ್ನು ಚಾಲನೆಗೊಳಿಸಿದ ಒಂದು\break ವ್ಯಕ್ತಿಯ ಜೀವನವನ್ನು ನಿಮಗೆ ನಿರೂಪಿಸುತ್ತೇನೆ. ಅದಕ್ಕೆ ಮುಂಚೆ ಭರತಖಂಡವೆಂದರೇನು ಎಂಬ ರಹಸ್ಯವನ್ನು ನಿಮಗೆ ತಿಳಿಸಲು ಯತ್ನಿಸುತ್ತೇನೆ. ಯಾರ ಕಣ್ಣುಗಳು ಕಂಗೊಳಿಸುತ್ತಿರುವ ಪ್ರಾಪಂಚಿಕ ವಸ್ತುಗಳ ಬೆಡಗಿನಿಂದ ಕುರುಡಾಗಿದೆಯೊ, ಯಾರು ತಮ್ಮ ಜೀವನವನ್ನು ತಿಂದು ಕುಡಿದು ನಲಿಯುವುದಕ್ಕಾಗಿ ಮೀಸಲಾಗಿ ಇಟ್ಟಿರುವರೊ, ಯಾರ ಆಸ್ತಿ ಹೊನ್ನು\break ಮಣ್ಣುಗಳೊ, ಯಾರ ಸಂತೋಷ ಇಂದ್ರಿಯ ಸುಖವೊ, ಯಾರು ದೇವರು ದ್ರವ್ಯವೊ, ಯಾರ ದೃಷ್ಟಿ ಎಂದೂ ಭವಿಷ್ಯದ ಕಡೆಗೆ ಇಲ್ಲವೊ, ಯಾರ ಗುರಿ ಇಹಜಗತ್ತಿನ ಸುಖ ಮಾತ್ರವೊ, ಯಾರು ತಾವು ಜೀವಿಸುವ ವಿಷಯ ವಸ್ತುಗಳ ಜಗತ್ತನ್ನು ಮೀರಿದ ವಸ್ತುವನ್ನು ಕುರಿತು ಆಲೋಚಿಸುವುದು ಅಪರೂಪವೊ, ಇಂತಹ ಜನರು ಭರತಖಂಡಕ್ಕೆ ಹೋದರೆ ಅವರು ಅಲ್ಲಿ ನೋಡುವುದೇನು? ದಾರಿದ್ರ್ಯ, ಕಶ್ಮಲ, ಮೂಢನಂಬಿಕೆ, ಅಜ್ಞಾನ ಮತ್ತು ಎಲ್ಲೆಡೆಯಲ್ಲೂ ಅಸಹ್ಯವಾದ ಪರಿಸ್ಥಿತಿ. ಇದು ಏತಕ್ಕೆ? ಏತಕ್ಕೆಂದರೆ ಅವರ ಮನಸ್ಸಿನಲ್ಲಿ\break ಜ್ಞಾನವೆಂದರೆ ಉಡುಪು, ವಿದ್ಯೆ, ಸಾಮಾಜಿಕ ವಿನಯ ಎಂದು ಭಾವಿಸುವರು. ಪಾಶ್ಚಾತ್ಯ ದೇಶಗಳು ತಮ್ಮ ಪ್ರಾಪಂಚಿಕ ಸ್ಥಿತಿಯನ್ನು ಉತ್ತಮಪಡಿಸಲು ಹಲವು ರೀತಿಯಿಂದ ಪ್ರಯತ್ನಿಸಿದರೆ ಭರತಖಂಡ ಬೇರೊಂದು ಬಗೆಯಲ್ಲಿ ವ್ಯವಹರಿಸುವುದು. ಮಾನವ\break ಜನಾಂಗದ ಇತಿಹಾಸದಲ್ಲೆಲ್ಲ ನೋಡಿದರೆ, ಭಾರತೀಯರು ಪರರನ್ನು ಗೆಲ್ಲುವುದಕ್ಕೆ ತಮ್ಮ ಗಡಿಯನ್ನು ಮೀರಿ ಹೋಗಲಿಲ್ಲ, ಅವರು ಪರರ ಆಸ್ತಿಗೆ ಆಸೆಪಡಲಿಲ್ಲ, ತಮ್ಮ ದೇಶ\break ಫಲವತ್ತಾಗಿದ್ದುದರಿಂದ ಸ್ವಂತ ದುಡಿತದಿಂದ ಸಂಪತ್ತನ್ನು ಗಳಿಸಿದರು. ಅನ್ಯರು ಬಂದು ತಮ್ಮ ಆಸ್ತಿಯನ್ನು ದೋಚಿಕೊಂಡು ಹೋಗುವಂತೆ ಇದು ಪ್ರೇರೇಪಿಸಿತು. ಇದೇ ಅವರ ಒಂದು ದೊಡ್ಡ ತಪ್ಪು. ಅವರು ತಮ್ಮ ಆಸ್ತಿಯನ್ನು ಕಳೆದುಕೊಂಡರೂ, ಅನ್ಯರಿಂದ ಅನಾಗರಿಕರೆಂದು ಕರೆಸಿಕೊಂಡರೂ ಶಾಂತಚಿತ್ತರಾಗಿದ್ದು ಅದಕ್ಕೆ ಪ್ರತಿಫಲವಾಗಿ ಬ್ರಹ್ಮದರ್ಶನವನ್ನು ಜಗತ್ತಿಗೆ ಕೊಟ್ಟರು. ನಿಜವಾದ ಆತ್ಮವನ್ನು ಆವರಿಸಿರುವ ತೆರೆಯನ್ನು ಸೀಳುವುದಕ್ಕಾಗಿ ಮಾನವನ ಸ್ವಭಾವದ ರಹಸ್ಯಗಳನ್ನು ನಿಸ್ಸಂಕೋಚವಾಗಿ ಬಹಿರಂಗಪಡಿಸಿದರು. ಏಕೆಂದರೆ ಅವರಿಗೆ ಈ ಕನಸು ಗೊತ್ತು. ಈ ಭೌತತತ್ತ್ವದ ಹಿಂದೆ ಯಾವ ಪಾಪವೂ ಮಲಿನ ಮಾಡದ, ಯಾವ ದುಷ್ಕೃತ್ಯವೂ ನಾಶಮಾಡದ ಯಾವ ಕಾಮವೂ ಕಾಂತಿಯನ್ನು ತಗ್ಗಿಸದ, ಬೆಂಕಿಯು ದಹಿಸದ, ನೀರು ತೋಯಿಸದ, ಶಾಖ ಒಣಗಿಸದ, ಸಾವು ನಾಶಮಾಡಲಾಗದ, ನಿಜವಾದ ಮಾನವ ದೈವೀಸ್ವಭಾವವಿರುವುದು ಅವರಿಗೆ ಗೊತ್ತು. ಅವರಿಗೆ ಮಾನವನ ಈ ನೈಜಸ್ವಭಾವ, ಪಾಶ್ಚಾತ್ಯರ ಇಂದ್ರಿಯಗಳಿಗೆ ಸ್ಥೂಲವಸ್ತು ಎಷ್ಟು ಸತ್ಯವೋ ಅಷ್ಟೇ ಸತ್ಯವಾಗಿತ್ತು.

ಹೇಗೆ ನೀವು ಜಯಧ್ವನಿಯನ್ನು ಮಾಡಿ ಧೈರ್ಯದಿಂದ ತೋಪಿನ ಬಾಯಿಗೆ\break ನೆಗೆಯಬಲ್ಲಿರೊ, ಹೇಗೆ ನೀವು ದೇಶಭಕ್ತಿಯ ಹೆಸರಿನಲ್ಲಿ ಧೈರ್ಯವಾಗಿ ನಿಮ್ಮ\break ನಾಡಿಗೋಸುಗ ಪ್ರಾಣವನ್ನು ಕೊಡಬಲ್ಲಿರೊ, ಹಾಗೆಯೆ ಅವರು ದೇವರ ಹೆಸರಿನಲ್ಲಿ ಧೈರ್ಯವಾಗಿರುವರು. ಜಗತ್ತೆಲ್ಲಾ ನಮ್ಮದು ಭಾವನಾಜಾಲವೆಂದು ಸಾರಿದರೆ, ಎಲ್ಲವೂ ಕನಸು ಎಂದು ಸಾರಿದರೆ, ತಾನು ಆಲೋಚಿಸಿರುವುದು, ನಂಬಿರುವುದು ಸತ್ಯವೆಂದು ತೋರುವುದಕ್ಕೆ ತನ್ನ ಆಸ್ತಿ ಬಟ್ಟೆಬರೆ ಎಲ್ಲವನ್ನೂ ತೊರೆಯಲು ಸಿದ್ಧನಾಗಿರುವನು. ಜೀವನ ಅನಂತವಾದುದು ಎಂದು ತಿಳಿದ ಮೇಲೆ, ದೇಹವನ್ನು ತೃಣಸಮಾನ ಎಂದು ಭಾವಿಸಿ ಅದನ್ನು ಎಸೆಯಲು ಸಿದ್ಧನಾಗಿ ವ್ಯಕ್ತಿಯು ನದಿಯ ತೀರದಲ್ಲಿ ಕುಳಿತುಕೊಳ್ಳುತ್ತಾನೆ.\break ಅವರ ಧೈರ್ಯ ಇಲ್ಲಿದೆ. ಸಾವನ್ನು ನಮ್ಮ ಸಹೋದರನಂತೆ ಎದುರುಗೊಳ್ಳಲು ಅವರು ಸಿದ್ಧರಾಗಿರುವರು. ಏಕೆಂದರೆ ಮಾನವನಿಗೆ ನಿಜವಾಗಿಯೂ ಸಾವೆಂಬುದು ಇಲ್ಲವೆನ್ನುವುದು ಅವರಿಗೆ ನಿಸ್ಸಂದೇಹವಾಗಿ ಗೊತ್ತಿದೆ. ನೂರಾರು ವರ್ಷಗಳಿಂದಲೂ ಹೊರಗಿನವರ ದಾಳಿ ದಬ್ಬಾಳಿಕೆ ಕ್ರೂರವರ್ತನೆ - ಇವುಗಳಿಗೆ ತುತ್ತಾದರೂ ಅಜೇಯರಾಗುವಂತೆ ಮಾಡಿದ ಶಕ್ತಿ ಅವರಲ್ಲಿದೆ. ಭರತಖಂಡ ಇನ್ನೂ ಬದುಕಿರುವುದು. ಆ ದೇಶದಲ್ಲಿ ಅತಿ ಶೋಚನೀಯ ದುಃಸ್ಥಿತಿಯಲ್ಲಿಯೂ ಆಧ್ಯಾತ್ಮಿಕ ವೀರರ ಜನನಕ್ಕೆ ಬರಗಾಲವಿಲ್ಲ. ಪಾಶ್ಚಾತ್ಯ ದೇಶಗಳು\break ಹೇಗೆ ರಾಜಕೀಯ ಕ್ಷೇತ್ರದಲ್ಲಿ, ವಿಜ್ಞಾನ ಕ್ಷೇತ್ರದಲ್ಲಿ, ವೀರರಿಗೆ ಜನ್ಮ ನೀಡುವುವೋ ಹಾಗೆಯೇ ಏಷ್ಯಾಖಂಡ ಆಧ್ಯಾತ್ಮಿಕ ವೀರರಿಗೆ ಜನ್ಮ ನೀಡುವುದು. ಈ ಶತಮಾನದ\break ಆದಿಭಾಗದಲ್ಲಿ ಪಾಶ್ಚಾತ್ಯ ಸಂಸ್ಕೃತಿ ತನ್ನ ಪ್ರಭಾವವನ್ನು ಭರತಖಂಡದ ಮೇಲೆ ಬೀರಲು ಮೊದಲು ಮಾಡಿತು. ಪಾಶ್ಚಾತ್ಯ ವಿಜಯಿಗಳು ಖಡ್ಗ ಸನ್ನದ್ಧರಾಗಿ ಬಂದು ಋಷಿಸಂಜಾತ ಶಿಶುಗಳಿಗೆ, ಅವರು ಕೇವಲ ಅನಾಗರಿಕರೆಂದು ಕನಸುಣಿಗಳೆಂದು, ಅವರ ಧರ್ಮ ಕೇವಲ ಕಂತೆ ಪುರಾಣವೆಂದು, ಅವರು ಸಾಧಿಸುವುದಕ್ಕೆ ಪ್ರಯತ್ನಪಡುತ್ತಿದ್ದ ದೇವರು ಆತ್ಮ\break ಮುಂತಾದ ಭಾವನೆಗಳೆಲ್ಲ ಅರ್ಥಹೀನ ಪದಗಳೆಂದು, ಸಾವಿರಾರು ವರ್ಷಗಳ ಸಾಧನೆ, ಅಂತ್ಯವರಿಯದ ತ್ಯಾಗ ಎಲ್ಲ ನಿರರ್ಥಕವೆಂದು ಸಾರಿದಾಗ, ವಿಶ್ವವಿದ್ಯಾನಿಲಯಗಳಲ್ಲಿ\break ಓದುತ್ತಿದ್ದ ತರುಣರ ಮನಸ್ಸಿನಲ್ಲಿ, ಅದುವರೆವಿಗೂ ಬಾಳಿದ ನಮ್ಮ ಜನಾಂಗದ ಜೀವನ ವ್ಯರ್ಥವೆ, ಪಾಶ್ಚಾತ್ಯರ ಮೇಲ್ಪಂಕ್ತಿಯನ್ನು ಈಗ ಹೊಸದಾಗಿ ಅನುಕರಿಸಬೇಕೆ, ತಮ್ಮ ಹಳೆಯ ಶಾಸ್ತ್ರಗಳನ್ನು ಹರಿದು ಹಾಕಿ, ತತ್ತ್ವಶಾಸ್ತ್ರಗಳನ್ನು ಸುಟ್ಟು, ಬೋಧಕರನ್ನು ಓಡಿಸಿ, ತಮ್ಮ ದೇವಾಲಯಗಳನ್ನು ಒಡೆಯಬೇಕೆ ಎಂದು ಅನ್ನಿಸಿತು. ಪಾಶ್ಚಾತ್ಯರು ವಿಜಯಿಗಳಲ್ಲವೆ? ಕತ್ತಿಯ ಮತ್ತು ಕೋವಿಯ ಮೂಲಕ ತಮ್ಮ ಧರ್ಮವನ್ನು ಪ್ರದರ್ಶಿಸುತ್ತಿದ್ದ ಅವರು ಹಿಂದಿನ ಆಚಾರವೆಲ್ಲ ಬರಿಯ ಮೂಢನಂಬಿಕೆ ಮತ್ತು ವಿಗ್ರಹಾರಾಧನೆ ಎಂದು ಸಾರಿಲ್ಲವೆ? ಹೊಸ ಶಾಲೆಯಲ್ಲಿ ಕಲಿತ ಮಕ್ಕಳು ಪಾಶ್ಚಾತ್ಯರ ಆದರ್ಶವನ್ನು ಅನುಸರಿಸಿ, ಬಾಲ್ಯಾರಭ್ಯದಿಂದಲೂ\break ಈ ಭಾವಗಳನ್ನು ಮನನ ಮಾಡಿದಾಗ ಈ ಸಂದೇಹಗಳು ಅವರ ಮನಸ್ಸಿನಲ್ಲಿ\break ಎದ್ದುದರಲ್ಲಿ ಆಶ್ಚರ್ಯವಿಲ್ಲ. ಮೂಢನಂಬಿಕೆಯನ್ನು ಆಚೆಗೆಸೆದು, ಸತ್ಯಾನ್ವೇಷಣೆಯನ್ನು ಮಾಡುವ ಬದಲು, ಸತ್ಯಪ್ರಮಾಣವೆಂದರೆ “ಪಾಶ್ಚಾತ್ಯರು ಏನು ಹೇಳುತ್ತಾರೆ?”\break ಎಂಬುದಾಯಿತು. ಪುರೋಹಿತರು ಹೋಗಬೇಕು, ವೇದಗಳನ್ನು ಸುಡಬೇಕು, ಏಕೆಂದರೆ ಪಾಶ್ಚಾತ್ಯರು ಹೀಗೆ ಹೇಳುತ್ತಾರೆ. ಹೀಗೆ ಉಂಟಾದ ಅಶಾಂತಿ ಭಾವನೆಯಿಂದ ಭರತ\-ಖಂಡದಲ್ಲಿ ಸುಧಾರಣೆ ಎಂಬ ಅಲೆಯೊಂದೆದ್ದಿತು.

ನೀವು ನಿಜವಾಗಿ ಸುಧಾರಕರಾಗಬೇಕಾದರೆ ಮೂರು ವಿಷಯಗಳು ಆವಶ್ಯಕ.\break ಪ್ರಥಮವಾಗಿ ಸಹಾನುಭೂತಿ. ನೀವು ನಿಮ್ಮ ಸಹೋದರರಿಗೋಸುಗ ನಿಜವಾಗಿಯೂ\break ಅನುತಾಪಪಡುವಿರೇನು? ಜಗತ್ತಿನಲ್ಲಿ ಇಷ್ಟೊಂದು ದುಃಖವಿದೆ, ಮೂಢನಂಬಿಕೆ ಇದೆ ಎಂದು ನಿಮಗೆ ಅನ್ನಿಸುವುದೇನು? ನೀವು ನಿಜವಾಗಿಯೂ ಮನುಷ್ಯರೆಲ್ಲಾ ನಿಮ್ಮ\break ಸಹೋದರರೆಂದು ತಿಳಿಯುವಿರಾ? ಈ ಭಾವನೆ ನಿಮ್ಮ ಮನಸ್ಸನ್ನೆಲ್ಲಾ ವ್ಯಾಪಿಸುವುದೆ? ಇದು ನಿಮ್ಮ ರಕ್ತದಲ್ಲಿ ಸಂಚರಿಸುತ್ತಿದೆಯೆ? ಇದು ನಿಮ್ಮ ನಾಡಿಗಳಲ್ಲಿ ಸ್ಪಂದಿಸುತ್ತಿದೆಯೆ?\break ಇದು ನಿಮ್ಮ ದೇಹದ ಅಂಗೋಪಾಂಗಗಳಲ್ಲಿ, ನರದ ಪ್ರತಿಯೊಂದು ಶಾಖೆಯಲ್ಲಿ\break ಸಂಚರಿಸುತ್ತಿದೆಯೆ? ನೀವು ಸಹಾನುಭೂತಿಯಿಂದ ತುಂಬಿ ತುಳುಕಾಡುತ್ತಿರುವಿರಾ?\break ಹೌದು ಎಂದಾದರೆ ಅದು ಮೊದಲಿನ ಮೆಟ್ಟಲು ಮಾತ್ರ. ಅನಂತರ ಇದಕ್ಕೆ ಏನಾದರೂ ಪರಿಹಾರ ಸಿಕ್ಕಿದೆಯೆ ಎಂಬುದನ್ನು ಆಲೋಚಿಸಬೇಕು. ಪುರಾತನ ಭಾವನೆಗಳೆಲ್ಲ\break ಮೂಢನಂಬಿಕೆಯಾಗಿರಬಹುದು. ಆದರೆ ಈ ಮೂಢ ನಂಬಿಕೆಯ ರಾಶಿಯ ಒಳಗೆ\break ಮತ್ತು ಸುತ್ತಲೂ ಸತ್ಯದ ಸ್ವರ್ಣ ಘಟ್ಟಿ ಇರುವುದು. ಮಲಿನತೆಯನ್ನು ಕಳೆದು\break ಚಿನ್ನವನ್ನು ಮಾತ್ರ ಉಳಿಸಿಕೊಳ್ಳುವುದಕ್ಕೆ ಯಾವುದಾದರೂ ಮಾರ್ಗವನ್ನು ಕಂಡುಹಿಡಿ\-ದಿರುವಿರಾ? ನೀವು ಇದನ್ನು ಮಾಡಿದ್ದರೆ ಇದು ಎರಡನೆಯ ಮೆಟ್ಟಲು ಮಾತ್ರ.\break ಮತ್ತೊಂದು ಆವಶ್ಯಕವಿದೆ. ನಿಮ್ಮ ಉದ್ದೇಶವೇನು? ಹಣ, ಕೀರ್ತಿ, ಅಧಿಕಾರದ\break ಲಾಲಸೆ ಇವು ನಿಮ್ಮನ್ನು ಪ್ರೇರೇಪಿಸಿಲ್ಲವೆಂದು ನಿಮಗೆ ಸತ್ಯವಾಗಿಯೂ ಗೊತ್ತೆ?\break ಇಡೀ ಜಗತ್ತೇ ನಿಮ್ಮ ನಾಶಕ್ಕೆ ಸಿದ್ಧವಾಗಿದ್ದರೂ, ನಿಮ್ಮ ಆದರ್ಶಗಳನ್ನು ಬಿಡದೆ\break ಸಾಧಿಸುವುದಕ್ಕೆ ಸಿದ್ಧರಾಗಿರುವಿರಾ? ನಿಮಗೆ ಏನು ಬೇಕೆಂಬುದು ಚೆನ್ನಾಗಿ ಗೊತ್ತೆ?\break ನಿಮ್ಮ ಪ್ರಾಣಕ್ಕೆ ಅಪಾಯ ಬಂದರೂ, ನಿಮ್ಮ ಕರ್ತವ್ಯವನ್ನಲ್ಲದೆ ಬೇರೆ ಯಾವುದನ್ನೂ ಮಾಡುವುದಿಲ್ಲವೆ? ಪ್ರಾಣವಿರುವವರೆಗೆ, ಹೃದಯದಲ್ಲಿ ಚಲನೆ ಇರುವವರೆಗೆ, ಹಿಡಿದ ಕೆಲಸವನ್ನು ಬಿಡದೆ ಹೋರಾಡಬಲ್ಲಿರಾ? ಆಗ ನೀವು ನಿಜವಾದ ಸುಧಾರಕರು,\break ಗುರುಗಳು, ಮಾನವ ಕೋಟಿಗೆ ಮಂಗಳಪ್ರದಾಯಕರು. ಆದರೆ ಮಾನವನಿಗೆ\break ತಾಳ್ಮೆಯಿಲ್ಲ, ದೂರದೃಷ್ಟಿಯಿಲ್ಲ, ಕಾಯುವುದಕ್ಕೆ ತಾಳ್ಮೆಯಿಲ್ಲ, ನೋಡುವುದಕ್ಕೆ\break ಶಕ್ತಿಯಿಲ್ಲ. ಆಳಬೇಕೆಂದು ಅವನು ಇಚ್ಛಿಸುವನು, ಪ್ರತಿಫಲ ತಕ್ಷಣ ಬೇಕು. ಇದು\break ಏತಕ್ಕೆ? ಫಲಗಳನ್ನು ತಾನೇ ಪಡೆಯಬೇಕೆಂದು ಆಶಿಸುವನು. ನಿಜವಾಗಿಯೂ\break ಅವನು ಮತ್ತೊಬ್ಬನನ್ನು ಲಕ್ಷಿಸುವುದಿಲ್ಲ. ಕರ್ತವ್ಯಕ್ಕೋಸುಗ ಕರ್ತವ್ಯವಲ್ಲ ಅವನಿಗೆ\break ಬೇಕಾಗಿರುವುದು. “ಕರ್ಮಕ್ಕೆ ಮಾತ್ರ ನಿನಗೆ ಅಧಿಕಾರವಿದೆ. ಅದರ ಫಲಕ್ಕಲ್ಲ” ಎಂದು ಶ‍್ರೀಕೃಷ್ಣ ಹೇಳುವನು. ಫಲ ತನ್ನನ್ನು ತಾನೇ ನೋಡಿಕೊಳ್ಳಲಿ. ಆದರೆ ಮನುಷ್ಯನಿಗೆ\break ತಾಳ್ಮೆಯಿಲ್ಲ. ಯಾವುದಾದರೊಂದು ಯೋಜನೆಯನ್ನು ಅವನು ತೆಗೆದುಕೊಳ್ಳುವನು. ಪ್ರಪಂಚದಲ್ಲಿರುವ ಬಹುಮಂದಿ ಸುಧಾರಕರನ್ನು ನಾವು ಈ ಪಂಗಡಕ್ಕೆ ಸೇರಿಸಬಹುದು.

ನಾನು ಹೇಳಿದಂತೆ ಭರತಖಂಡದ ಮೇಲೆ ದಾಳಿಯಿಟ್ಟ ಜಡವಾದವೆಂಬ\break ಅಲೆಯು ಆರ್ಯಮಹರ್ಷಿಗಳ ಸಂದೇಶವನ್ನು ಕೊಚ್ಚಿಕೊಂಡು ಹೋಗಿಬಿಡುವುದೋ ಎನ್ನುವ ಕಾಲದಲ್ಲಿ ಸುಧಾರಣಾ ಭಾವನೆ ಮೂಡಿತು. ಆದರೆ ಇಂತಹ ಬದಲಾವಣೆಯ ಸಹಸ್ರಾರು ಅಲೆಗಳ ತಾಡನವನ್ನು ಭರತಖಂಡ ಹಿಂದೆ ಸಹಿಸಿತ್ತು. ಅದರೊಂದಿಗೆ ಹೋಲಿಸಿ ನೋಡಿದರೆ ಇದು ಸೌಮ್ಯ. ಒಂದು ಅಲೆಯಾದ ಮೇಲೆ ಮತ್ತೊಂದು ನೂರಾರು\break ವರ್ಷಗಳಿಂದಲೂ ಎಲ್ಲವನ್ನೂ ಮುರಿದು ನಾಶಮಾಡುತ್ತ ಭರತಖಂಡವನ್ನು ಆವರಿಸಿತ್ತು. ಖಡ್ಗಕಾಂತಿ ಚಿಮ್ಮಿ “ಅಲ್ಲಾಹೋ ಅಕ್ಬರ್​” ಎಂಬ ಧ್ವನಿ ಭಾರತ ದೇಶದ ದಿಕ್​ತಟಗಳನ್ನು ಒಂದು ಕಾಲದಲ್ಲಿ ತುಂಬಿಬಿಟ್ಟಿತ್ತು. ಆದರೆ ಈ ಮಹಾಪ್ರವಾಹ ಜನಾಂಗದ ಆದರ್ಶವನ್ನು ಬದಲಾಯಿಸದೆ ಬತ್ತಿಹೋಯಿತು.

ಭಾರತ ರಾಷ್ಟ್ರವನ್ನು ನಾಶಮಾಡಲಾಗುವುದಿಲ್ಲ. ಎಲ್ಲಿಯವರೆವಿಗೂ ಅಧ್ಯಾತ್ಮ ಅದರ ಹಿನ್ನೆಲೆಯಲ್ಲಿ ಇರುವುದೊ, ಎಲ್ಲಿಯವರೆಗೆ ಜನರು ಆತ್ಮಶಕ್ತಿಯನ್ನು ತೊರೆಯುವುದಿಲ್ಲವೊ, ಅಲ್ಲಿಯವರೆಗೆ ಭಾರತಾವನಿ ಅಮರವಾಗಿರುವುದು. ಅವರು ಭಿಕ್ಷುಕರಾಗಿರಬಹುದು, ದೀನರಾಗಿರಬಹುದು, ಸದಾಕಾಲದಲ್ಲಿಯೂ ಹೊಲಸು ಕೊಳೆ ಅವರನ್ನು ಆವರಿಸಿರಬಹುದು. ಆದರೆ ಅವರು ತಮ್ಮ ದೇವರನ್ನು ಬಿಡದೆ ಇರಲಿ. ಆರ್ಯಮಹರ್ಷಿಗಳ\break ಪುತ್ರರು ತಾವು ಎಂಬುದನ್ನು ಮರೆಯದಿರಲಿ. ಪಾಶ್ಚಾತ್ಯ ದೇಶದಲ್ಲಿ ದಾರಿಹೋಕನು ಕೂಡ ಮಧ್ಯಕಾಲದ ಡಕಾಯಿತ ಪಾಳೇಗಾರನ ವಂಶಕ್ಕೆ ತಾನು ಸೇರಿದವನೆಂಬುದನ್ನು ಹೇಗೆ ಸಮರ್ಥಿಸಲು ಆಶಿಸುವನೊ, ಹಾಗೆಯೆ ಭಾರತದಲ್ಲಿ ಸಿಂಹಾಸನಾರೂಢನಾದ ಚಕ್ರವರ್ತಿ ಕೂಡ ಕಾಡಿನಲ್ಲಿ ಮರದ ತೊಗಟೆಯನ್ನು ಹೊದ್ದು, ಅರಣ್ಯದ ಫಲಮೂಲಗಳನ್ನು ತಿಂದು, ಭಗವಂತನೊಂದಿಗೆ ಸಂಬಂಧವನ್ನು ಪಡೆದ ಒಬ್ಬ ಋಷಿಯ ಕುಲಕ್ಕೆ ತಾನು ಸೇರಿದವನೆಂದು ಹೇಳಲು ಬಯಸುವನು. ನಮಗೆ ಬೇಕಾಗಿರುವುದು ಅಂತಹ ವಂಶ. ಎಲ್ಲಿಯವರೆಗೂ ಪವಿತ್ರತೆಯನ್ನು ಶ್ರೇಷ್ಠ ಗೌರವದಿಂದ ಕಾಣುತ್ತೇವೆಯೊ ಅಲ್ಲಿಯವರೆಗೂ ಭರತಖಂಡ\break ನಾಶವಾಗಲಾರದು.

“ನೈನ್​ಟೀನ್ತ್​ ಸೆಂಚುರಿ” ಎಂಬ ಪತ್ರಿಕೆಯ ಈಚಿನ ಸಂಚಿಕೆಯೊಂದರಲ್ಲಿ ಮ್ಯಾಕ್ಸ್​ ಮುಲ್ಲರ್​ ಬರೆದ “ಒಬ್ಬ ನಿಜವಾದ ಮಹಾತ್ಮ” ಎಂಬ ಲೇಖನವನ್ನು ನಿಮ್ಮಲ್ಲಿ ಹಲವರು ಬಹುಶಃ ಓದಿರಬಹುದು. ಶ‍್ರೀರಾಮಕೃಷ್ಣರು ತಾವು ಬೋಧಿಸಿದ ಸಂದೇಶದ ಜೀವಂತ ಉದಾಹರಣೆಯಾಗಿರುವುದರಿಂದ ಅವರ ಜೀವನ ಸ್ವಾರಸ್ಯಕರವಾಗಿದೆ. ಬಹುಶಃ ಭರತಖಂಡಕ್ಕಿಂತ ಸಂಪೂರ್ಣ ಬೇರೆಯಾದ ವಾತಾವರಣದಲ್ಲಿರುವ ಪಾಶ್ಚಾತ್ಯರಿಗೆ ಇದು ಸ್ವಲ್ಪ ವಿಚಿತ್ರವಾಗಿ ಕಾಣಬಹುದು. ಭರತಖಂಡಕ್ಕೂ ಪಾಶ್ಚಾತ್ಯರ ಸಡಗರ ಜೀವನಕ್ಕೂ ಬಹಳ ವ್ಯತ್ಯಾಸವಿರುವುದು. ಆದರೂ ಹೀಗಿರುವುದರಿಂದಲೇ ಅದು ಹೆಚ್ಚು ರಸಭರಿತವಾಗಿ ಕಾಣಬಹುದು. ಅನೇಕರು ಆಗಲೇ ಕೇಳಿರುವ ಒಂದು ವಿಷಯವನ್ನು ನೂತನ ದೃಷ್ಟಿಯಿಂದ ನೋಡಲು ಅದು ಅವಕಾಶ ಕೊಡುವುದು.

ಹಲವು ಬಗೆಯ ಸುಧಾರಣೆಗಳು ಭರತಖಂಡದಲ್ಲಿ ತಲೆಯೆತ್ತುತ್ತಿದ್ದಾಗ, ಬಂಗಾಳದ ಕುಗ್ರಾಮವೊಂದರ ಬಡ ಬ್ರಾಹ್ಮಣರ ಕುಟುಂಬದಲ್ಲಿ ಕ್ರಿ.ಶ. ೧೮೩೬ರ ಫೆಬ್ರವರಿ ೧೮ನೆಯ ದಿನ ಒಂದು ಮಗು ಜನಿಸಿತು. ತಂದೆತಾಯಿಗಳು ಬಹಳ ಆಚಾರವಂತರು. ನಿಜವಾಗಿ ಆಚಾರವಂತನಾದ ಬ್ರಾಹ್ಮಣನ ಜೀವನ ಒಂದು ನಿರಂತರವಾದ ತ್ಯಾಗ ಪರಂಪರೆ. ಅವನು ಕೆಲವು ವೃತ್ತಿಗಳನ್ನು ಮಾತ್ರ ಮಾಡಬಹುದು. ಇದನ್ನು ಮೀರಿ ಯಾವ ಪ್ರಾಪಂಚಿಕ ವ್ಯವಹಾರದಲ್ಲಿಯೂ ಕೈಹಾಕಕೂಡದು. ಸಿಕ್ಕಿಸಿಕ್ಕಿದವರ ಕೈಯಿಂದ ಅವನು ದಾನವನ್ನು ಸ್ವೀಕರಿಸಬಾರದು. ಇಂತಹ ಜೀವನ ಎಷ್ಟು ಕಠಿಣವಾಗುವುದೆಂಬುದನ್ನು ನೀವು ಯೋಚಿಸಬಹುದು. ಬ್ರಾಹ್ಮಣರ ಮತ್ತು ಪುರೋಹಿತರ ವಿಷಯಗಳನ್ನು ನೀವು ಹಲವು ವೇಳೆ ಕೇಳಿರುವಿರಿ. ಆದರೆ ನಿಮ್ಮಲ್ಲಿ ಕೆಲವರು, ಇಂತಹ ವಿಚಿತ್ರ ವ್ಯಕ್ತಿಗಳು ತಮ್ಮ ಜನರ ನಾಯಕರು ಹೇಗೆ ಆಗಬಲ್ಲರು ಎಂದು ಕೇಳಬಹುದು. ದೇಶದಲ್ಲಿರುವ ಹಲವು ಪಂಗಡಗಳಲ್ಲಿ ಅವರು ಅತಿ ದರಿದ್ರರು. ಅವರ ಶಕ್ತಿಯ ಮೂಲ ತ್ಯಾಗ. ಎಂದಿಗೂ ಅವರು ದ್ರವ್ಯಕ್ಕೆ ಆಶಿಸುವವರಲ್ಲ. ಪ್ರಪಂಚದಲ್ಲೇ ಅತಿ ದೀನವಾದ ಪುರೋಹಿತವರ್ಗ ಅವರದು. ಆದಕಾರಣವೆ ಅವರು ಅತ್ಯಂತ ಪ್ರಭಾವಶಾಲಿಗಳು. ಇಂತಹ ದಾರಿದ್ರ್ಯದ ಸ್ಥಿತಿಯಲ್ಲಿಯೂ ಬ್ರಾಹ್ಮಣನ ಗೃಹಿಣಿ ಯಾವ ದೀನನಾದರೂ ಹಳ್ಳಿಗೆ ಹೋದರೆ ತಿನ್ನಲು ಏನನ್ನಾದರೂ ಕೊಡದೆ ಕಳುಹಿಸುವುದಿಲ್ಲ. ಭರತಖಂಡದಲ್ಲಿ ಇದನ್ನು ತಾಯಿಯಾದವಳ ಪರಮಕರ್ತವ್ಯವೆಂದು ಭಾವಿಸಲಾಗಿದೆ. ಆಕೆ ತಾಯಿ, ಆದಕಾರಣವೆ ಊಟಕ್ಕೆ ಇಕ್ಕಬೇಕು. ತಾನು ಊಟ ಮಾಡುವುದಕ್ಕೆ ಮುಂಚೆ ಎಲ್ಲರಿಗೂ ಊಟವಾಗಿದೆಯೆ ಎಂದು ನೋಡಬೇಕು. ಭರತಖಂಡದಲ್ಲಿ ತಾಯಿಯನ್ನು ದೇವರೆಂದು ಭಾವಿಸಲು ಇದೇ ಕಾರಣ. ನಮ್ಮ ಕಥಾನಾಯಕನ ತಾಯಿ, ಈ ಗೃಹಿಣಿ, ಹಿಂದೂ ಸಂಸ್ಕೃತಿಯ ಆದರ್ಶವಾಗಿದ್ದಳು. ಜಾತಿ ಶ್ರೇಷ್ಠವಾಗಿದ್ದಷ್ಟೂ ಅಡಚಣೆಗಳು ಹೆಚ್ಚು. ತುಂಬಾ ಕೆಳಗಿನ ಜಾತಿಯವರು ತಮಗೆ ತೋರಿದುದನ್ನು ಕುಡಿಯಬಹುದು, ತಿನ್ನಬಹುದು. ಆದರೆ ಸಮಾಜದಲ್ಲಿ ಮೇಲುಮೇಲಿನ ಹಂತಗಳಿಗೆ ಹೋದಂತೆಲ್ಲ ಹೆಚ್ಚು ಹೆಚ್ಚು ಆತಂಕಗಳು ಬರುವುವು. ವಂಶಾನುಗತವಾಗಿ ಬಂದ ಪೌರೋಹಿತ್ಯವನ್ನು ಮಾಡುವ ಬ್ರಾಹ್ಮಣ ಜಾತಿಯವರ ಜೀವನ ಕ್ಷೇತ್ರವನ್ನು ನೋಡಿದರೆ ಅದೊಂದು ನಿರಂತರ ತ್ಯಾಗ ಜೀವನವೆನ್ನಬಹುದು. ಬಹುಶಃ ಜಗತ್ತಿನಲ್ಲೆಲ್ಲ ಮತ್ತೊಬ್ಬರನ್ನು ತಮ್ಮ ಪಂಗಡಕ್ಕೆ ಸೇರಿಸಿಕೊಳ್ಳದ ಜನಾಂಗವೆಂದರೆ ಹಿಂದೂಗಳು. ಅವರಲ್ಲಿ ಆಂಗ್ಲೇಯರಂತೆ ಹೆಚ್ಚು ಸ್ಥಿರತೆ ಕಾಣುವುದು. ಆದರೆ ಅವರಲ್ಲಿ ಅದು ಹೆಚ್ಚು ವಿಶದವಾಗಿ ಪ್ರಕಾಶಿತವಾಗಿರುವುದು. ಅವರಿಗೆ ಯಾವುದಾದರೊಂದು ಆಲೋಚನೆ ಹೊಳೆದರೆ ಕೊನೆಯ ತನಕ ಅದನ್ನು ಸಾಧಿಸುವರು. ನೀವು ಒಮ್ಮೆ ಅವರಿಗೆ ಯಾವುದಾದರೂ ಭಾವನೆಯನ್ನು ಕೊಟ್ಟರೆ ಪುನಃ ಅದನ್ನು ಹಿಂದಕ್ಕೆ ಪಡೆಯಲು ಆಗುವುದಿಲ್ಲ. ಆದರೆ ಹೊಸ ಭಾವನೆಯನ್ನು ಗ್ರಹಿಸುವಂತೆ ಅವರನ್ನು ಮಾಡುವುದು ಕಷ್ಟ.

ಆಚಾರಶೀಲರಾದ ಹಿಂದೂಗಳು ಮತ್ತೊಬ್ಬರೊಂದಿಗೆ ಬೆರೆಯುವುದಿಲ್ಲ. ಅವರು\break ತಮ್ಮ ಆಲೋಚನೆಯ ಮತ್ತು ಭಾವನೆಯ ಪ್ರಪಂಚದಲ್ಲಿ ಮಾತ್ರ ವಿಹರಿಸುವರು. ಅವರ ಜೀವನ ಹೇಗಿರಬೇಕೆಂಬುದರ ಸಣ್ಣ ಪುಟ್ಟ ವಿವರಗಳೂ ಕೂಡ ನಮ್ಮ ಶಾಸ್ತ್ರದಲ್ಲಿ\break ಉಕ್ತವಾಗಿದೆ. ಇಂತಹ ಸಣ್ಣ ಪುಟ್ಟ ನಿಯಮಗಳನ್ನು ಅವರು ಅತ್ಯಂತ ಕಟ್ಟುನಿಟ್ಟಾಗಿ\break ಅನುಸರಿಸುವರು. ತಮ್ಮ ಜಾತಿಯ ಒಳಪಂಗಡಕ್ಕೆ ಸೇರದವನು ಮಾಡಿದ ಅಡಿಗೆಯನ್ನು ಊಟ ಮಾಡುವ ಬದಲು ಅವರು ಉಪವಾಸವಿರುವರು. ಅವರಲ್ಲಿ ಪ್ರಚಂಡ ಶ್ರದ್ಧೆ\break ಇದೆ. ದೃಢಪ್ರಯತ್ನ, ತೀವ್ರ ಆಸಕ್ತಿ ಮತ್ತು ಧರ್ಮಾಕಾಂಕ್ಷೆ-ಇವು ಆಚಾರಶೀಲರಾದ ಹಿಂದೂಗಳಲ್ಲಿ ಹೆಚ್ಚು. ಅವರ ಆಚಾರಶೀಲತೆ, ಅದು ಸತ್ಯವೆಂಬ ದೃಢವಿಶ್ವಾಸದಿಂದ ಹೊರಹೊಮ್ಮುವುದು. ಅವರು ಛಲದಿಂದ ಯಾವುದನ್ನು ನಂಬಿಕೊಂಡಿರುವರೋ\break ಅದು ಸತ್ಯವೆಂದು ನಾವೆಲ್ಲ ಭಾವಿಸದೆ ಇರಬಹುದು. ಆದರೆ ಅವರಿಗೆ ಅದು ಸತ್ಯ.\break ಮನುಷ್ಯನು ವಿಪರೀತವಾಗಿ ದಾನಶೀಲನಾಗಿರಬೇಕು ಎಂದು ನಮ್ಮ ಶಾಸ್ತ್ರದಲ್ಲಿ ಹೇಳಿದೆ. ಮತ್ತೊಬ್ಬನಿಗೆ ಸಹಾಯ ಮಾಡುವುದಕ್ಕೆ, ಅವನ ಪ್ರಾಣವನ್ನು ಉಳಿಸುವುದಕ್ಕೆ ಒಬ್ಬನು ಉಪವಾಸವಿದ್ದು ಸತ್ತರೂ ಚಿಂತೆಯಿಲ್ಲ. ಹಾಗೆಯೇ ಮಾಡಬೇಕು ಎನ್ನುವರು. ಬ್ರಾಹ್ಮಣ ಈ ಭಾವನೆಯನ್ನು ಅಕ್ಷರಶಃ ಪರಿಪಾಲಿಸಬೇಕೆಂದು ನಿರೀಕ್ಷಿಸಿದೆ. ಭಾರತದ ಸಾಹಿತ್ಯದ ಪರಿಚಯವಿರುವವರಿಗೆ ಮಹಾಭಾರತದಲ್ಲಿ ಬರುವ, ಮೇರೆಮೀರಿ ದಾನಮಾಡಿದ,\break ಒಂದು ಇಡೀ ಸಂಸಾರ ಉಪವಾಸವಿದ್ದು ಭಿಕ್ಷುಕನಿಗೆ ತಮ್ಮ ಪಾಲನ್ನು ಕೊಟ್ಟ ಸುಂದರ ಕಥೆಯ ನೆನಪಿರಬಹುದು. ಇದು ಉತ್ಪ್ರೇಕ್ಷೆಯಲ್ಲ. ಈಗಲೂ ನಡೆಯುತ್ತವೆ. ನನ್ನ\break ಗುರುದೇವರ ತಾಯಿ ತಂದೆ ಅಂಥವರನ್ನು ಹೋಲುತ್ತಿದ್ದರು. ಅವರು ಬಹಳ\break ಬಡವರು. ಆದರೂ ತಾಯಿ ಮತ್ತೊಬ್ಬರಿಗೆ ಸಹಾಯ ಮಾಡುವುದಕ್ಕಾಗಿ ಹಲವು ವೇಳೆ\break ಉಪವಾಸವಿರುತ್ತಿದ್ದಳು. ಅಂತಹ ಮಾತಾಪಿತೃಗಳಿಗೆ ಈ ಶಿಶು ಜನಿಸಿತು. ಬಾಲ್ಯದಿಂದಲೇ ಆ ಮಗು ವಿಚಿತ್ರವಾಗಿತ್ತು. ಜನನಾರಭ್ಯದಿಂದಲೇ ಅವರಿಗೆ ತಮ್ಮ ಹಿಂದಿನ ಜನ್ಮದ\break ನೆನಪಿತ್ತು. ಈ ಜಗತ್ತಿಗೆ ತಾವು ಬಂದ ಉದ್ದೇಶವನ್ನು ತಿಳಿದಿದ್ದರು. ಈ ಉದ್ದೇಶ\break ಸಾಧನೆಗಾಗಿ ತಮ್ಮ ಸರ್ವಶಕ್ತಿಯನ್ನೂ ಉಪಯೋಗಿಸಿದರು.

ಬಾಲ್ಯದಲ್ಲಿ ಅವರ ತಂದೆ ತೀರಿಹೋದರು. ಆಮೇಲೆ ಹುಡುಗನನ್ನು ಶಾಲೆಗೆ\break ಹಾಕಿದರು. ಬ್ರಾಹ್ಮಣರ ಹುಡುಗ ಶಾಲೆಗೆ ಹೋಗಲೇಬೇಕು. ವಿದ್ಯಾಪ್ರಧಾನವಾದ\break ಕಸುಬನ್ನೇ ಕಲಿಯಬೇಕೆಂಬುದು ಆ ಜಾತಿಯ ವಿಧಿ. ಅದು ಈಗಲೂ ಹಲವಾರು\break ಕಡೆ ಇರುವ ಹಿಂದಿನ ಕಾಲದ ಭರತಖಂಡದ ವಿದ್ಯಾಭ್ಯಾಸದ ರೀತಿ, ಅದರಲ್ಲಿಯೂ ಸಂನ್ಯಾಸಿಗಳಿಗೆ ಸಂಬಂಧಿಸಿದುದು. ಆಧುನಿಕ ವಿದ್ಯಾಭ್ಯಾಸದ ರೀತಿಗಿಂತ ಎಷ್ಟೋ\break ವ್ಯತ್ಯಾಸವಾಗಿರುವುದು. ವಿದ್ಯಾರ್ಥಿಗಳು ಏನನ್ನೂ ಕೊಡಬೇಕಾಗಿರಲಿಲ್ಲ. ಜ್ಞಾನ ಅತ್ಯಂತ ಪವಿತ್ರ; ಅದನ್ನು ಯಾರೂ ಮಾರಕೂಡದು ಎಂದು ಜನ ಭಾವಿಸಿದ್ದರು. ಜ್ಞಾನವನ್ನು\break ಉಚಿತವಾಗಿ, ಯಾವ ಬೆಲೆಯೂ ಇಲ್ಲದೆ ಕೊಡಬೇಕು. ಗುರುಗಳು ವಿದ್ಯಾರ್ಥಿಗಳನ್ನು ಉಚಿತವಾಗಿ ಸ್ವೀಕರಿಸುತ್ತಿದ್ದರು. ಇದು ಮಾತ್ರವಲ್ಲ. ಅನೇಕ ಗುರುಗಳು ಅವರಿಗೆ\break ಆಹಾರವನ್ನೂ ಬಟ್ಟೆಯನ್ನೂ ಒದಗಿಸುತ್ತಿದ್ದರು. ಈ ಗುರುಗಳ ಸಂರಕ್ಷಣಾರ್ಥವಾಗಿ ಶ‍್ರೀಮಂತರು ಮದುವೆ ಶ್ರಾದ್ಧ ಮುಂತಾದ ಸಮಯಗಳಲ್ಲಿ ಅವರಿಗೆ ದಾನ\break ನೀಡುತ್ತಿದ್ದರು. ಕೆಲವು ದಾನಗಳಿಗೆ ಗುರುಗಳೇ ಪ್ರಥಮ ಮತ್ತು ಪ್ರಧಾನ\break ಹಕ್ಕುದಾರರು ಎಂದು ಅವರು ಭಾವಿಸಿದ್ದರು. ಗುರುಗಳು ಇದರಿಂದ ವಿದ್ಯಾರ್ಥಿ\break ಗಳಿಗೆ ಊಟ ವಸತಿ ಸೌಕರ್ಯಗಳನ್ನು ಏರ್ಪಡಿಸುತ್ತಿದ್ದರು. ಶ‍್ರೀಮಂತರ ಮನೆಯಲ್ಲಿ\break ಯಾವಾಗಲಾದರೂ ಮದುವೆಯಾದರೆ ಪಂಡಿತರನ್ನು ಆಹ್ವಾನಿಸುತ್ತಿದ್ದರು. ಇವರು\break ಅಲ್ಲಿಗೆ ಬಂದು ಹಲವಾರು ವಿಷಯಗಳನ್ನು ಚರ್ಚಿಸುತ್ತಿದ್ದರು. ಈ ಹುಡುಗ ಇಂತಹ ಪಂಡಿತರ ಒಂದು ಗೋಷ್ಠಿಗೆ ಹೋದ. ಪಂಡಿತರು ಈ ಹುಡುಗನ ವಯಸ್ಸಿಗೆ\break ಮೀರಿದ ನ್ಯಾಯ ಜ್ಯೋತಿಷ್ಯ ಮುಂತಾದ ಹಲವಾರು ವಿಷಯಗಳನ್ನು ಚರ್ಚಿಸುತ್ತಿದ್ದರು. ನಾನು ನಿಮಗೆ ಹೇಳಿದಂತೆ ಹುಡುಗ ವಿಚಿತ್ರ ಸ್ವಭಾವದವನು. ಅವರ ವಾದದಿಂದ\break ಆತ ಈ ನೀತಿಯನ್ನು ಕಲಿತ: “ಇಷ್ಟೇ ಅವರ ಜ್ಞಾನದ ಫಲವೆಲ್ಲ. ಏತಕ್ಕೆ ಅವರು\break ಇಷ್ಟು ವಾದ ಮಾಡುತ್ತಿರುವರು? ಕೇವಲ ಹಣಕ್ಕಾಗಿ ಮಾತ್ರ. ಯಾರು ಇಲ್ಲಿ ಹೆಚ್ಚು ಪಾಂಡಿತ್ಯವನ್ನು ಪ್ರದರ್ಶಿಸುವರೊ ಅವರಿಗೆ ಒಂದು ಜೊತೆ ಒಳ್ಳೆಯ ಪಂಚೆ ಸಿಕ್ಕುವುದು. ಇವರು ವಾದ ಮಾಡುವುದೆಲ್ಲ ಇದಕ್ಕಾಗಿಯೆ. ನಾನು ಇನ್ನು ಶಾಲೆಗೆ ಹೋಗುವುದಿಲ್ಲ.”\break ಆತ ಶಾಲೆಗೆ ಹೋಗಲಿಲ್ಲ. ಶಾಲೆಗೆ ಹೋಗುವುದು ಕೊನೆಗೊಂಡಿತಾದರೂ, ವಿದ್ಯಾವಂತನಾದ ಈ ಹುಡುಗನ ಅಣ್ಣ ತಮ್ಮನಿಗೆ ವಿದ್ಯಾಭ್ಯಾಸ ಮಾಡಿಸಲು ಆತನನ್ನು ಕಲ್ಕತ್ತೆಗೆ\break ಕರೆದುಕೊಂಡು ಹೋದ. ಕೆಲವು ದಿನಗಳಾದ ನಂತರ ಎಲ್ಲಾ ಪ್ರಾಪಂಚಿಕ ಜ್ಞಾನಾರ್ಜನೆಯ\break ಗುರಿ ಜೀವನೋಪಾಯಕ್ಕಾಗಿ ಅಲ್ಲದೆ ಬೇರೆ ಅಲ್ಲ ಎಂದು ನಿಸ್ಸಂದೇಹವಾಗಿ ಆ ಹುಡುಗ ಮನಗಂಡ. ಆಗ ವ್ಯಾಸಂಗವನ್ನು ತೊರೆದು ಆಧ್ಯಾತ್ಮಿಕ ಚಿಂತನೆಯಲ್ಲಿ ಕಾಲ ಕಳೆಯಬೇಕೆಂದು ಆತ ಸಂಕಲ್ಪಿಸಿದ. ಆಗ ಆತನ ತಂದೆ ಕಾಲವಾಗಿದ್ದರು. ಬಡ ಸಂಸಾರ; ಈ ಹುಡುಗ ತನ್ನ ಜೀವನೋಪಾಯವನ್ನು ಹುಡುಕಬೇಕಾಗಿತ್ತು. ಕಲ್ಕತ್ತೆಯ ಸಮೀಪದ\break ಒಂದು ಊರಿಗೆ ಹೋಗಿ ಅಲ್ಲಿ ದೇವಸ್ಥಾನದಲ್ಲಿ ಪೂಜಾರಿ ಆದ. ಪೂಜಾರಿ ಆಗುವುದು\break ಅತಿ ಹೀನವೆಂದು ಬ್ರಾಹ್ಮಣರು ಭಾವಿಸಿದ್ದರು. ನಮ್ಮ ದೇವಸ್ಥಾನಗಳು ನಿಮ್ಮ ಚರ್ಚಿನಂತೆ ಅಲ್ಲ. ಅವು ಬಹಿರಂಗ ಆರಾಧನೆಯ ಸ್ಥಳಗಳಲ್ಲ. ಭರತಖಂಡದಲ್ಲಿ ಬಹಿರಂಗಪೂಜೆ ಎಂಬುದು ನಿಜವಾಗಿ ಇಲ್ಲ. ಹೆಚ್ಚಾಗಿ ಶ‍್ರೀಮಂತರ ಪುಣ್ಯಕರ್ಮವೆಂದು ಭಾವಿಸಿ ದೇವಸ್ಥಾನವನ್ನು ಕಟ್ಟಿಸುವರು.

ಒಬ್ಬನಿಗೆ ಹೆಚ್ಚು ಆಸ್ತಿ ಇದ್ದರೆ ಗುಡಿ ಕಟ್ಟಿಸಲು ಆಶಿಸುವನು. ಅಲ್ಲಿ ದೇವರ ಯಾವುದಾದರೊಂದು ಅವತಾರದ ವಿಗ್ರಹವನ್ನು ಪ್ರತಿಷ್ಠೆಮಾಡಿ, ದೇವರ ಹೆಸರಿನಲ್ಲಿ ಪೂಜೆಮಾಡಲು ಆಸ್ತಿಯನ್ನು ಸಮರ್ಪಿಸುವನು. ಅಲ್ಲಿಯ ಪೂಜೆ ರೋಮನ್​ಕ್ಯಾಥೊಲಿಕ್​ ಚರ್ಚಿನ\break ಆರಾಧನೆಯಂತೆ ಇರುವುದು. ಶಾಸ್ತ್ರಗಳಿಂದ ಕೆಲವು ಭಾಗವನ್ನು ಓದುವರು. ವಿಗ್ರಹಕ್ಕೆ ಮಂಗಳಾರತಿ ಮಾಡುವರು. ನಾವು ಒಬ್ಬ ಮಹಾಪುರುಷನನ್ನು ಸತ್ಕರಿಸುವಂತೆ ವಿಗ್ರಹವನ್ನು ಸತ್ಕರಿಸುವರು. ದೇವಸ್ಥಾನದಲ್ಲಿ ಮಾಡುವುದೆಲ್ಲ ಇಷ್ಟೆ. ದೇವಸ್ಥಾನಕ್ಕೆ ಹೋಗುವವರು,\break ದೇವಸ್ಥಾನಕ್ಕೆ ಹೋಗದವರಿಗಿಂತ ಉತ್ತಮರೆಂದು ಪರಿಗಣಿಸಲ್ಪಡುವುದಿಲ್ಲ. ನಿಜವಾಗಿ\break ದೇವಸ್ಥಾನಕ್ಕೆ ಹೋಗದವರು ಹೆಚ್ಚು ಧಾರ್ಮಿಕರೆಂದು ಪರಿಗಣಿಸಲ್ಪಡುವರು. ಏಕೆಂದರೆ ಭರತಖಂಡದಲ್ಲಿ ಧರ್ಮ ಪ್ರತಿಯೊಬ್ಬನ ಆಪ್ತವಿಷಯ. ಪ್ರತಿಯೊಬ್ಬನ ಮನೆಯಲ್ಲಿಯೂ ಒಂದು ಸಣ್ಣ ದೇವಾಲಯವಿರುವುದು ಅಥವಾ ಅದಕ್ಕಾಗಿ ಒಂದು ಪ್ರತ್ಯೇಕವಾದ\break ಕೊಠಡಿ ಇರುವುದು. ಪ್ರಾತಃಕಾಲ, ಸಂಜೆ ಅಲ್ಲಿಗೆ ಹೋಗಿ ಒಂದುಕಡೆ ಕುಳಿತು ಪೂಜೆ ಮಾಡುವನು. ಈ ಪೂಜೆ ಮಾನಸಿಕವಾದುದು. ಈತ ಏನು ಮಾಡುತ್ತಿರುವನು ಅಥವಾ ಏನು ಹೇಳುತ್ತಿರುವನು ಎಂಬುದೂ ಕೂಡ ಮತ್ತೊಬ್ಬನಿಗೆ ಗೊತ್ತಾಗುವುದಿಲ್ಲ. ಅಲ್ಲಿ\break ಅವನು ಕುಳಿತುಕೊಂಡಿರುವುದು, ಬೆರಳುಗಳನ್ನು ವಿಚಿತ್ರವಾಗಿ ಅಲ್ಲಾಡಿಸುತ್ತಿರುವುದು ಅಥವಾ ಮೂಗಿನ ಹೊಳ್ಳೆಯನ್ನು ಮುಚ್ಚಿಕೊಂಡು ಒಂದು ರೀತಿ ಉಸಿರಾಡುತ್ತಿರುವುದು ಮಾತ್ರ ಕಾಣುವುದು. ಇದಕ್ಕಿಂತ ಹೆಚ್ಚು ಅವನು ಏನು ಮಾಡುತ್ತಿರುವನು ಎಂಬುದು ಬಹುಶಃ ಅವನ ಸಹೋದರನಿಗೆ ಅಥವಾ ಹೆಂಡತಿಗೆ ಕೂಡ ಗೊತ್ತಾಗುವುದಿಲ್ಲ. ಎಲ್ಲಾ\break ಅವನ ಮನೆಯೊಳಗೆ ರಹಸ್ಯವಾಗಿ ನಡೆಯುವುದು. ಯಾರಿಗೆ ಮನೆಯಲ್ಲಿ ಒಂದು\break ದೇವರ ಕೋಣೆ ಇಲ್ಲವೋ ಅವರು ನದಿ ಸರೋವರ ಅಥವಾ ಸಮುದ್ರ ಹತ್ತಿರ\break ಇದ್ದರೆ ಅದರ ತೀರಕ್ಕೆ ಹೋಗುವರು; ಕೆಲವು ವೇಳೆ ಜನರು ವಿಗ್ರಹಕ್ಕೆ ನಮಸ್ಕಾರ ಮಾಡುವುದಕ್ಕಾಗಿ ದೇವಸ್ಥಾನಕ್ಕೆ ಹೋಗುವರು. ಅಲ್ಲಿಗೆ ಅವರ ದೇವಸ್ಥಾನದ\break ಕರ್ತವ್ಯ ಮುಗಿಯುವುದು. ಬಹಳ ಹಿಂದಿನ ಕಾಲದಿಂದಲೂ ನಮ್ಮ ದೇಶದಲ್ಲಿ\break ಪೂಜಾರಿಯಾಗುವುದು ಅವಮಾನಕರವೆಂದು ಪರಿಗಣಿಸಲ್ಪಟ್ಟಿದೆ. ಮನುವು ಕೂಡ\break ಹಾಗೆಯೇ ಹೇಳಿರುವನು. ಕೆಲವು ಗ್ರಂಥಗಳು ಇದು ಬ್ರಾಹ್ಮಣನನ್ನು ನಿಂದಾಸ್ಪದನನ್ನಾಗಿ ಮಾಡುವುದು ಎಂದು ಸಾರುವುವು. ಇದರ ಹಿಂದೆ ಇನ್ನೊಂದು ಭಾವನೆ ಇದೆ.\break ವಿದ್ಯಾಭ್ಯಾಸದಲ್ಲಿ ಹೇಗೋ, ಹಾಗೆಯೇ, ಅದಕ್ಕಿಂತ ಹೆಚ್ಚಾಗಿ, ಧಾರ್ಮಿಕ ಜೀವನದಲ್ಲಿ ಪೂಜಾರಿಗಳು ತಾವು ಮಾಡುವ ಕೆಲಸಕ್ಕೆ ದ್ರವ್ಯವನ್ನು ಸ್ವೀಕರಿಸುವುದರಿಂದ ಪವಿತ್ರ\break ಕಾರ್ಯ ಒಂದು ವ್ಯಾಪಾರವಾಗಿ ಹೋಗುವುದು. ದಾರಿದ್ರ್ಯದಿಂದ ಪ್ರೇರೇಪಿತನಾಗಿ ತನಗೆ ಅವಕಾಶವಿರುವ ಒಂದೇ ಉದ್ಯೋಗವಾದ ದೇವಸ್ಥಾನದ ಪೂಜಾರಿಯ ಕೆಲಸವನ್ನು ಸ್ವೀಕರಿಸಿದಾಗ ಆ ಹುಡುಗನ ಮನೋಭಾವನೆ ಹೇಗಿದ್ದಿರಬಹುದೆಂಬುದನ್ನು ನೀವೇ ಊಹಿಸಬಹುದು.

ವಂಗದೇಶದಲ್ಲಿ ಹಲವು ಕವಿಗಳಿರುವರು. ಅವರ ಕಾವ್ಯಗಳು ಜನಸಾಮಾನ್ಯರಿಗೆ ಪರಿಚಿತವಾಗಿವೆ. ಕಲ್ಕತ್ತ ನಗರದ ಬೀದಿ ಬೀದಿಗಳಲ್ಲೂ ಮತ್ತು ಹಳ್ಳಿಹಳ್ಳಿಗಳಲ್ಲೂ ಅವನ್ನು ಹಾಡುವರು. ಇವುಗಳಲ್ಲಿ ಹೆಚ್ಚಿನವು ಧಾರ್ಮಿಕ ಭಾವನೆಯುಳ್ಳವುಗಳು. ಭರತಖಂಡದ ಮತಗಳ ವೈಶಿಷ್ಟ್ಯವಾದ ಆತ್ಮಸಾಕ್ಷಾತ್ಕಾರವೆ ಅವುಗಳ ಮುಖ್ಯಭಾವನೆ. ಈ ಭಾವನೆಯನ್ನು ಬೀರದಿರುವ ಧರ್ಮ ಗ್ರಂಥಗಳು ಭರತಖಂಡದಲ್ಲಿ ಇಲ್ಲವೇ ಇಲ್ಲ. ಮನುಷ್ಯ\break ದೇವರನ್ನು ಸಾಕ್ಷಾತ್ಕಾರ ಮಾಡಿಕೊಳ್ಳಬೇಕು. ಆತನನ್ನು ತನ್ನಲ್ಲಿ ಅನುಭವಿಸಬೇಕು, ನೋಡಬೇಕು, ಮಾತನಾಡಬೇಕು. ಇದೇ ಧರ್ಮ. ಭರತಖಂಡದ ವಾತಾವರಣವು ಭಗವಂತನ ದರ್ಶನ ಪಡೆದ ಮಹಾಪುರುಷರ ಕಥೆಗಳಿಂದ ತುಂಬಿ ತುಳುಕಾಡುತ್ತಿದೆ. ಈ ಸಿದ್ಧಾಂತವೇ ಧರ್ಮದ ತಳಹದಿ. ಹಿಂದಿನ ಕಾಲದ ಗ್ರಂಥಗಳು, ಶಾಸ್ತ್ರಗಳು ಆಧ್ಯಾತ್ಮಿಕ ಸತ್ಯಾಂಶವನ್ನು ಪ್ರತ್ಯಕ್ಷವಾಗಿ ಮನಗಂಡ ಮಹನೀಯರ ಬರಹಗಳಾಗಿವೆ. ಕೇವಲ ಯುಕ್ತಿಯ\break ದೃಷ್ಟಿಯಿಂದ ಇವುಗಳನ್ನು ಬರೆಯಲಿಲ್ಲ ಅಥವಾ ಯಾವ ಬಗೆಯ ಯುಕ್ತಿಗೂ ಇದನ್ನು\break ತಿಳಿಯಲು ಆಗದು. ಏಕೆಂದರೆ ತಾವು ಬರೆದ ವಿಷಯಗಳನ್ನು ಪ್ರತ್ಯಕ್ಷ ನೋಡಿದವರು ಇದನ್ನು ಬರೆದರು. ಯಾರು ಇದರ ಅನುಭವದ ಶಿಖರಕ್ಕೆ ಏರಬಲ್ಲರೊ ಅವರು\break ಮಾತ್ರ ಇದನ್ನು ತಿಳಿದುಕೊಳ್ಳಬಹುದು. ಈ ಜನ್ಮದಲ್ಲಿ ಸಾಕ್ಷಾತ್ಕಾರ ಎಂಬುದು ಇದೆ,\break ಇದು ಎಲ್ಲರಿಗೂ ಸಾಧ್ಯ ಎನ್ನುವರು. ಧರ್ಮವೆಂದರೆ ಈ ಭಾವನೆಯನ್ನು\break ಜಾಗೃತಗೊಳಿಸುವುದಾಗಿದೆ. ಎಲ್ಲಾ ಧರ್ಮಗಳ ಮುಖ್ಯ ಭಾವನೆಯೇ ಅದು.\break ಆದಕಾರಣವೇ ಒಬ್ಬನಲ್ಲಿ ವಾಗ್ಮಿತೆ ತನ್ನ ಪರಾಕಾಷ್ಠೆಯನ್ನು ಮುಟ್ಟಿರಬಹುದು. ಅವನು ತರ್ಕಬದ್ಧವಾಗಿ ಮಾತನಾಡಬಹುದು. ಅತ್ಯುತ್ತಮ ಸಿದ್ಧಾಂತಗಳನ್ನು ಹೇಳುತ್ತಿರಬಹುದು. ಆದರೂ ಅವನ ಸಂದೇಶವನ್ನು ಕೇಳುವುದಕ್ಕೆ ಜನರು ಸಿಕ್ಕುವುದಿಲ್ಲ. ಆದರೆ ಮತ್ತೊಬ್ಬ\break ದೀನನಾಗಿರಬಹುದು. ತನ್ನ ಸ್ವಂತ ಭಾಷೆಯಲ್ಲಿ ಕೂಡ ಸರಿಯಾಗಿ ಮಾತನಾಡುವುದಕ್ಕೆ ಬರದೇ ಇರಬಹುದು. ಆದರೂ ಆತನನ್ನು ಅರ್ಧ ಜನಾಂಗವೇ ಆತನ ಜೀವಮಾನದಲ್ಲೇ ದೇವರೆಂದು ಆರಾಧಿಸುವರು. ಭರತಖಂಡದಲ್ಲಿ ಯಾರಾದರೂ ಒಬ್ಬ ಆತ್ಮಸಾಕ್ಷಾತ್ಕಾರದ ಶಿಖರಕ್ಕೆ ಏರಿರುವನು, ಧರ್ಮ ಆತನಿಗೆ ಕೇವಲ ಊಹೆಯಲ್ಲ, ಆತ್ಮದ ಅಮರತ್ವ, ದೇವರು ಮುಂತಾದ ಮುಖ್ಯ ವಿಷಯಗಳಲ್ಲಿ ಆತನು ಅಜ್ಞಾನದಲ್ಲಿ ತೊಳಲುತ್ತಿಲ್ಲ ಎಂಬ ಭಾವನೆ ಹೇಗಾದರೂ ಹರಡಿದರೆ, ಎಲ್ಲಾ ಕಡೆಯಿಂದಲೂ ಜನರು ಅವನನ್ನು ನೋಡಲು ಬರುವರು. ಕ್ರಮೇಣ ಅವನನ್ನು ಪೂಜಿಸುವರು.

ದೇವಸ್ಥಾನದಲ್ಲಿ ಆನಂದಮಯಿಯಾದ ಜಗನ್ಮಾತೆಯ ವಿಗ್ರಹವಿದ್ದಿತು. ಬೆಳಗ್ಗೆ\break ಸಾಯಂಕಾಲ ಈ ಹುಡುಗ ಅದಕ್ಕೆ ಪೂಜೆಯನ್ನು ಮಾಡಬೇಕಾಗಿತ್ತು. ಕ್ರಮೇಣ ಈ\break ಭಾವನೆ ಈತನ ಮನಸ್ಸನ್ನು ಆವರಿಸಿತು: “ಈ ವಿಗ್ರಹದ ಹಿಂದೆ ಏನಾದರೂ ಸತ್ಯ\break ಇದೆಯೆ? ಜಗತ್ತಿನಲ್ಲಿ ಆನಂದಮಯಿಯಾದ ಜಗನ್ಮಾತೆ ಇರುವುದು ಸತ್ಯವೆ? ಅವಳಿದ್ದು ಜಗತ್ತನ್ನೆಲ್ಲ ನಡೆಸುತ್ತಿರುವುದು ಸತ್ಯವೆ ಅಥವಾ ಇದೆಲ್ಲ ಕನಸೆ? ಧರ್ಮದಲ್ಲಿ ಏನಾದರೂ ಸತ್ಯವಿದೆಯೆ?”

ಈ ಅನುಮಾನ ಹಿಂದೂ ಹುಡುಗನಿಗೆ ಬರುವುದು ಸಹಜ. ನಾವು ಮಾಡುತ್ತಿರುವುದು ಸತ್ಯವೆ-ಎನ್ನುವುದು ನಮ್ಮ ಇಡೀ ಜನಾಂಗದ ಅನುಮಾನ. ಆತ್ಮ, ದೇವರು\break ಇವರಿಗೆ ಸಂಬಂಧಪಟ್ಟ ಎಲ್ಲಾ ತತ್ತ್ವಗಳೂ ನಮಗೆ ಸಮೀಪದಲ್ಲಿದ್ದರೂ, ಕೇವಲ\break ಸಿದ್ಧಾಂತಗಳು ನಮಗೆ ತೃಪ್ತಿಯನ್ನು ಕೊಡಲಾರವು. ಗ್ರಂಥವಾಗಲಿ ಸಿದ್ಧಾಂತವಾಗಲಿ\break ನಮಗೆ ತೃಪ್ತಿಯನ್ನು ತರಲಾರವು. ಸಾಕ್ಷಾತ್ಕಾರವೆಂಬ ಒಂದು ಭಾವನೆ ನಮ್ಮ\break ಸಾವಿರಾರು ಜನರನ್ನು ಮೆಟ್ಟಿಕೊಳ್ಳುವುದು. ದೇವರಿರುವುದು ಸತ್ಯವೆ? ಇದು ಸತ್ಯವಾದರೆ ನಾನು ಅವನನ್ನು ನೋಡಬಲ್ಲೆನೆ? ನಾನು ಸತ್ಯವನ್ನು ಸಾಕ್ಷಾತ್ಕಾರ ಮಾಡಿಕೊಳ್ಳಬಲ್ಲೆನೆ? ಪಾಶ್ಚಾತ್ಯರು ಇದನ್ನು ಕಾರ್ಯತಃ ಅಸಾಧ್ಯವೆಂದು ಭಾವಿಸಬಹುದು. ಆದರೆ ನಮಗೆ\break ಇದು ಅಕ್ಷರಶಃ ಅನುಷ್ಠಾನ ಸಾಧ್ಯ. ಈ ಭಾವನೆಗಾಗಿ ಜನರು ಪ್ರಾಣವನ್ನು ಬಲಿಕೊಡುವರು. ಕೇವಲ ಸಾಧಾರಣ ಬುದ್ಧಿಯ ದೃಷ್ಟಿಯಿಂದ ತಿಳಿದುಕೊಳ್ಳುವುದಲ್ಲ, ಯುಕ್ತಿ\break ಭೂಮಿಕೆಯ ಮೇಲೆ ನಿಂತು ಸತ್ಯವನ್ನು ಗ್ರಹಿಸುವುದಲ್ಲ, ಕೇವಲ ಅಜ್ಞಾನದಲ್ಲಿ\break ತೊಳಲುವುದಲ್ಲ, ಈ ಪ್ರಪಂಚ ನಮ್ಮ ಇಂದ್ರಿಯಗಳಿಗೆ ಎಷ್ಟು ಸತ್ಯವೋ ಅದಕ್ಕಿಂತ\break ದೇವರು ಹೆಚ್ಚು ತೀವ್ರವಾಗಿ ಸತ್ಯ ಎಂಬುದನ್ನು ಅರಿಯುವುದಕ್ಕೆ ಪುರಾತನ ಕಾಲದಿಂದಲೂ ಗುಹೆಯಲ್ಲಿ ವಾಸಮಾಡಲು ತಮ್ಮ ಎಲ್ಲಾ ಸುಖವನ್ನು ತೊರೆದವರು ಹಲವರಿದ್ದರು.\break ಪವಿತ್ರ ನದಿಯ ತೀರದಲ್ಲಿ ಕುಳಿತು ದೇವರಿಗಾಗಿ ದುಃಖಭಾರದಿಂದ ಅಳುವುದಕ್ಕಾಗಿ ನೂರಾರು ಜನ ತಮ್ಮ ಮನೆ ಮಠಗಳನ್ನು ಬಿಟ್ಟುಹೋಗಿದ್ದರು. ಇದನ್ನು ನೀವು\break ಈಗತಾನೆ ಕೇಳಿರುವಿರಿ. ಇದು ಆದರ್ಶ. ಅದನ್ನು ಕುರಿತು ನಾನು ಈಗ ಏನನ್ನೂ\break ಹೇಳುವುದಿಲ್ಲ. ಆದರೆ ಈ ಒಂದು ಭಾವನೆ ಅವರ ಇಡೀ ಜೀವನವನ್ನೇ\break ವ್ಯಾಪಿಸುವುದು. ಸಹಸ್ರಾರು ಜನರು ಸಾಯಬಹುದು. ಸಾಯುವುದಕ್ಕೆ ಸಹಸ್ರಾರು\break ಜನರು ಸಿದ್ಧರಾಗಿರುವರು. ಈ ಒಂದು ಆದರ್ಶಕ್ಕಾಗಿ ಸಹಸ್ರಾರು ವರ್ಷಗಳಿಂದಲೂ\break ಇಡೀ ಜನಾಂಗ ತ್ಯಾಗ ಮಾಡುತ್ತಿರುವುದು, ಪ್ರಾಣವನ್ನು ಅರ್ಪಿಸುತ್ತಿರುವುದು. ಈ\break ಆದರ್ಶಕ್ಕಾಗಿ ಪ್ರತಿವರ್ಷವೂ ಸಹಸ್ರಾರು ಜನ ಹಿಂದೂಗಳು ತಮ್ಮ ಮನೆಗಳನ್ನು\break ತೊರೆಯುವರು, ತಾವು ಅನುಭವಿಸಬೇಕಾದ ಕಷ್ಟದಲ್ಲಿ ಹಲವು ಜನ ಮಡಿಯುವರು. ಪಾಶ್ಚಾತ್ಯರಿಗೆ ಇದೊಂದು ಭ್ರಾಂತಿಯಾಗಿ ತೋರಬಹುದು. ಅವರ ಈ ದೃಷ್ಟಿಗೆ ಕಾರಣ ನನಗೆ ಗೊತ್ತಿದೆ. ನಾನು ಪಾಶ್ಚಾತ್ಯ ದೇಶದಲ್ಲಿ ಇದ್ದರೂ ಈ ಒಂದು ಭಾವನೆ ಜೀವನದಲ್ಲಿ ಅನುಷ್ಠಾನ ಸಾಧ್ಯವೆಂದು ಭಾವಿಸುತ್ತೇನೆ.

ಬೇರೆ ಆಲೋಚನೆಯಲ್ಲಿ ಕಳೆದ ಪ್ರತಿಯೊಂದು ಕ್ಷಣವೂ ನನಗೆ ನಷ್ಟದಾಯಕ.\break ಭೌತಶಾಸ್ತ್ರದ ಅದ್ಭುತಗಳು ಕೂಡ ನಷ್ಟದಾಯಕ. ಆ ಭಾವನೆಯಿಂದ ನನ್ನನ್ನು ಚಲಿಸುವಂತೆ ಮಾಡುವ ಪ್ರತಿಯೊಂದೂ ವ್ಯರ್ಥ. ದೇವತೆಯ ಜ್ಞಾನವಿರಲಿ, ಮೃಗದ ಅಜ್ಞಾನವಿರಲಿ, ಜೀವನ ಕ್ಷಣಿಕ. ಚಿಂದಿ ಬಟ್ಟೆಗಳನ್ನು ತೊಟ್ಟ ಕಡುಬಡವನ ದಾರಿದ್ರ್ಯವಿರಲಿ\break ಅಥವಾ ಜಗತ್ತಿನಲ್ಲೆಲ್ಲ ಶ‍್ರೀಮಂತನ ಆಸ್ತಿ ಇರಲಿ, ಜೀವನ ಕ್ಷಣಿಕ. ಪಾಶ್ಚಾತ್ಯ ದೇಶದ ಪ್ರಖ್ಯಾತ ಬೀದಿಯೊಂದರಲ್ಲಿ ಗುಲಾಮಗಿರಿಯಲ್ಲಿ ನರಳುತ್ತಿರಲಿ ಅಥವಾ ಲಕ್ಷಾಂತರ\break ಜನರನ್ನು ಆಳುವ ಚಕ್ರವರ್ತಿಯಾಗಲಿ, ಜೀವನ ಕ್ಷಣಿಕ. ಮಾನವ, ಭಾವಜೀವಿ\-ಯಾಗಿರಲಿ ಅಥವಾ ಕ್ರೂರನಾಗಿರಲಿ, ಜೀವನ ಕ್ಷಣಿಕ. ಜೀವನ ಸಮಸ್ಯೆಗೆ ಒಂದೇ ಒಂದು ಪರಿಹಾರವಿದೆ. ಅದನ್ನೇ ದೇವರೆಂದು ಮತ್ತು ಧರ್ಮವೆಂದು ಹಿಂದೂ ಹೇಳುವುದು. ಇದು\break ಸತ್ಯವಾದರೆ ಆಗ ಜೀವನವನ್ನೂ ವಿವರಿಸಬಹುದು, ಸಹಿಸಬಹುದು, ಆನಂದಿಸಬಹುದು. ಇಲ್ಲದೇ ಹೋದರೆ ಜೀವನ ವ್ಯರ್ಥ. ಇದೇ ನಮ್ಮ ಆದರ್ಶ. ಆದರೆ ನಾವು ಎಷ್ಟು\break ತರ್ಕಿಸಿದರೂ ಇದನ್ನು ಸಮರ್ಥಿಸುವುದಕ್ಕೆ ಆಗುವುದಿಲ್ಲ. ತರ್ಕ ಇದನ್ನು ಸಂಭವನೀಯವೆಂದು ಸಾಧಿಸಬಹುದು. ಅಲ್ಲೇ ಅದು ನಿಲ್ಲುವುದು. ನಮಗೆ ತಿಳಿದಿರುವ ಎಲ್ಲಾ\break ಕ್ಷೇತ್ರಗಳ ಜ್ಞಾನದ ಪರಮಾವಧಿ ಒಂದು ವಸ್ತು ಸಂಭವನೀಯ ಎಂದು ಸಾರುವುದಷ್ಟೇ. ಅದನ್ನು ಮೀರಲಾರದು. ಭೌತಜ್ಞಾನದ ಅತಿ ಪ್ರದರ್ಶನಾಯೋಗ್ಯ ವಿಷಯಗಳು ಕೇವಲ ಸಂಭವನೀಯ ವಿಷಯಗಳಲ್ಲದೆ ಅವು ಇನ್ನೂ ವಾಸ್ತವಿಕ ವಿಷಯಗಳಾಗಿಲ್ಲ. ವಾಸ್ತವಿಕ ವಿಷಯಗಳು ಇಂದ್ರಿಯಗಳ ಅನುಭವದಲ್ಲಿವೆ. ವಾಸ್ತವಿಕ ಸ್ಥಿತಿಯನ್ನು ನಾವು\break ಮನಗಾಣಬೇಕು. ನಾವು ಧರ್ಮವನ್ನು ಅರಿಯಬೇಕಾದರೆ ಅದನ್ನು ಅನುಷ್ಠಾನಕ್ಕೆ\break ತರಬೇಕು. ದೇವರು ಇರುವನು ಎಂದು ನಾವು ನಿಸ್ಸಂದೇಹವಾಗಿ ನಂಬಬೇಕಾದರೆ\break ದೇವರನ್ನು ನೋಡಬೇಕು. ನಾವು ಧರ್ಮಕ್ಕೆ ಸಂಬಂಧಪಟ್ಟ ವಿಷಯಗಳು ಸತ್ಯವೆಂದು ತಿಳಿಯಬೇಕಾದರೆ ಅವನ್ನು ಅನುಭವಿಸಬೇಕು. ಅವು ನಮ್ಮ ಸ್ವಂತ ಅನುಭವವಾಗಬೇಕು. ಸ್ವಂತ ಅನುಭವವನ್ನು ಬಿಟ್ಟು ಯಾವುದೂ, ಯಾವ ಯುಕ್ತಿಯೂ, ಈ ವಿಷಯಗಳನ್ನು ಸತ್ಯವನ್ನಾಗಿ ಮಾಡಲಾರದು. ನಮ್ಮ ನಂಬಿಕೆ ಬಂಡೆಯಂತೆ ಸ್ಥಿರವಾಗಲಾರದು. ಇದು ನನ್ನ ಭಾವನೆ. ಇದು ಭರತಖಂಡದ ಭಾವನೆ.

ಈ ಒಂದು ಭಾವನೆ ಆ ಹುಡುಗನ ಮನಸ್ಸನ್ನು ಆಕ್ರಮಿಸಿಕೊಂಡಿತು. ಆತನ ಇಡೀ ಜೀವನ ಇದರ ಮೇಲೆ ಏಕಾಗ್ರವಾಯಿತು. ಪ್ರತಿದಿನವೂ ಅಳುತ್ತ, “ತಾಯಿ, ನೀನು\break ಇರುವುದು ನಿಜವೆ ಅಥವಾ ಇದೆಲ್ಲ ಕೇವಲ ಕಲ್ಪನೆಯೆ? ಆನಂದಮಯಿಯಾದ ಮಾತೆ ಕವಿಗಳ ಮತ್ತು ಇತರ ಭ್ರಾಂತ ಜನರ ಒಂದು ಕಲ್ಪನೆಯೆ? ಅಥವಾ ಇಂತಹ ಒಂದು\break ಸತ್ಯವಿದೆಯೆ? ನಮ್ಮ ಪ್ರಾಪಂಚಿಕವಾದ ಅರ್ಥದಲ್ಲಿ, ವಿದ್ಯಾಭ್ಯಾಸ ಮತ್ತು ಪುಸ್ತಕ\break ಪರಿಚಯ ಯಾವುದೂ ಅವರಿಗೆ ಇರಲಿಲ್ಲವೆಂಬುದನ್ನು ನೋಡಿದೆವು. ಆದಕಾರಣ\break ಅವರ ಮನಸ್ಸು ಹೆಚ್ಚು ನೈಸರ್ಗಿಕವಾಗಿ, ಆರೋಗ್ಯವಾಗಿ, ಪರಿಶುದ್ಧವಾಗಿತ್ತು.\break ಮತ್ತೊಬ್ಬರ ಭಾವನೆಯೊಂದಿಗೆ ಬೆರೆತು ಅದು ನಿಸ್ಸಾರವಾಗಿರಲಿಲ್ಲ. ಅವರು\break ವಿಶ್ವವಿದ್ಯಾನಿಲಯಕ್ಕೆ ಹೋಗದೆ ಇದ್ದ ಕಾರಣ ತಾವೇ ಸ್ವಂತವಾಗಿ ಆಲೋಚನೆ\break ಮಾಡಿದರು. ನಾವು ಅರ್ಧ ಜೀವಮಾನವನ್ನು ವಿಶ್ವವಿದ್ಯಾನಿಲಯದಲ್ಲಿ ಕಳೆಯುವುದ\-ರಿಂದ ನಮ್ಮ ಮನಸ್ಸು ಅನ್ಯರ ಆಲೋಚನೆಗಳ ಸಂಗ್ರಹವಾಗಿದೆ. ನಾನು ನಿಮಗೆ ಈಗ ತಾನೆ\break ಹೇಳಿದ ಪ್ರೊಫೆಸರ್​ ಮ್ಯಾಕ್ಸ್ ಮುಲ್ಲರ್​ ತಮ್ಮ ಲೇಖನದಲ್ಲಿ ಹೀಗೆಂದಿರುವರು:\break “ಇವರೊಬ್ಬರು ಸ್ವಂತಿಕೆಯುಳ್ಳ ಪರಿಶುದ್ಧ ವ್ಯಕ್ತಿಗಳು. ಅವರ ಸ್ವಂತಿಕೆಯ ರಹಸ್ಯಕ್ಕೆ\break ಕಾರಣ ಅವರು ಯಾವ ವಿಶ್ವವಿದ್ಯಾನಿಲಯದ ವಾತಾವರಣದಲ್ಲಿಯೂ ಬೆಳೆಯದೆ\break ಇದ್ದುದು.” ಅಂತೂ ಅವರ ಮನಸ್ಸಿನಲ್ಲಿ ಬಹುಮುಖ್ಯವಾಗಿದ್ದ “ದೇವರನ್ನು ನೋಡಬಹುದೆ?” ಎಂಬ ಆಲೋಚನೆ ಬರಬರುತ್ತ ತೀವ್ರವಾಗಿ ಅವರು ಮತ್ತಾವುದನ್ನೂ\break ಆಲೋಚಿಸದಂತೆ ಆಯಿತು. ಅವರಿಗೆ ಪೂಜೆಯನ್ನು ಶಾಸ್ತ್ರೀಯವಾಗಿ ಮಾಡಲಾಗಲಿಲ್ಲ. ವಿವರವಾಗಿ ಪ್ರತಿಯೊಂದಕ್ಕೂ ಗಮನ ಕೊಡಲು ಆಗಲಿಲ್ಲ. ಅವರು ಅನೇಕ ವೇಳೆ\break ದೇವರಿಗೆ ನೈವೇದ್ಯ ಮಾಡುವುದನ್ನು ಮರೆಯುತ್ತಿದ್ದರು; ಕೆಲವು ವೇಳೆ ಮಂಗಳಾರತಿ ಎತ್ತುವುದನ್ನು ಮರೆಯುತ್ತಿದ್ದರು ಅಥವಾ ಹಲವು ಗಂಟೆಗಳ ಕಾಲ ಅದನ್ನೇ ಮಾಡುತ್ತಾ\break ಎಲ್ಲವನ್ನೂ ಮರೆಯುತ್ತಿದ್ದರು.

ಪ್ರತಿದಿನವೂ ಈ ಒಂದು ಭಾವನೆ ಅವರ ಮನಸ್ಸಿನಲ್ಲಿ ಕವಿಯುತ್ತಿತ್ತು: “ತಾಯಿ, ನೀನಿರುವುದು ಸತ್ಯವೆ? ನೀನು ಏತಕ್ಕೆ ಮಾತನಾಡೆ? ಸತ್ತಿರುವೆಯಾ?” ಬಹುಶಃ ನಮ್ಮಲ್ಲಿ\break ಕೆಲವರು ಇದನ್ನು ಅನುಭವಿಸಿರಬಹುದು. ಜಡವಾದ ನಿಸ್ಸಾರವಾದ ತರ್ಕರೀತಿಯಲ್ಲಿ ಬೇಕಾದಷ್ಟು ಆಲೋಚಿಸಿ ಬೇಸತ್ತ ಮೇಲೆ, ಹಲವಾರು ಗ್ರಂಥಗಳನ್ನು ಓದಿ ಸಾಕಾದಾಗ (ಅವೇನೂ ನಮಗೆ ಕಲಿಸುವುದಿಲ್ಲ), ಕೇವಲ ಯುಕ್ತಿಪೂರಿತ ಸಿದ್ಧಾಂತದ ಅಮಲನ್ನು\break ಸೇವಿಸಿ ಅದರಿಂದ ಬೇಸತ್ತಾಗ ನಮ್ಮ ಹೃದಯಾಂತರಾಳವು ಕೆಲವು ವೇಳೆ ಈ ಶೋಚ\-ನೀಯ ರೋದನೆಯನ್ನು ಹೊರಸೂಸುವುದು: “ನನಗೆ ಬೆಳಕನ್ನು ತೋರಿಸುವವರು ಜಗತ್ತಿನಲ್ಲಿ ಯಾರೂ ಇಲ್ಲವೆ? ನೀನು ಇದ್ದರೆ ಬೆಳಕನ್ನು ತೋರು. ನೀನೇಕೆ ಮಾತನಾಡೆ?\break ನೀನೇಕೆ ಅಷ್ಟು ಅಪರೂಪ? ಹಲವು ಚಾರರನ್ನು ಕಳುಹಿಸುವೆ ಏಕೆ? ನೀನೇ ನನ್ನೆಡೆಗೆ ಬಾರೆಯೇತಕೆ? ಭಿನ್ನಾಭಿಪ್ರಾಯಗಳು ಮತ್ತು ಮನಸ್ತಾಪಗಳಿಂದ ತುಂಬಿ ತುಳುಕಾಡುತ್ತಿರುವ ಜಗತ್ತಿನಲ್ಲಿ ನಾನು ಯಾರನ್ನು ಅನುಸರಿಸಲಿ? ಯಾರನ್ನು ನಂಬಲಿ? ನೀನು ಪ್ರತಿಯೊಬ್ಬ ನರನಾರಿಯರಿಗೂ ದೇವರಾಗಿದ್ದರೂ ನೀನೇಕೆ ನಿನ್ನ ಮಗುವಿನ ಹತ್ತಿರ ಮಾತನಾಡುವುದಕ್ಕೆ ಬರಬಾರದು?” ಅತಿ ದಾರುಣವಾದ ಮನೋವ್ಯಾಕುಲತೆಯಲ್ಲಿ\break ಸಿಕ್ಕಿರುವಾಗ ನಮ್ಮಲ್ಲಿ ಎಲ್ಲರಲ್ಲಿಯೂ ಅಂತಹ ಭಾವನಾ ತರಂಗಗಳೇಳುವುವು.\break ಆದರೂ ನಮ್ಮನ್ನು ಆವರಿಸಿರುವ ಪ್ರಲೋಭನೆಗಳಿಂದ ಮರುಕ್ಷಣದಲ್ಲೇ ಅದನ್ನು\break ಮರೆಯುವೆವು. ಕ್ಷಣದಲ್ಲಿ ಮುಕ್ತಿದ್ವಾರ ತೆರೆಯುವಂತೆ ಇತ್ತು, ಕ್ಷಣದಲ್ಲಿ ಸಚ್ಚಿದಾನಂದದ ಕಡಲಿನಲ್ಲಿ ಧುಮುಕುವಂತೆ ಇದ್ದೆವು. ಆದರೆ ಈ ಮಾನವನ ಮೃಗೀಯತೆ ದೈವಿಕ\break ಭಾವನೆಯನ್ನು ಹೊರದೂಡುವುದು. ಪುನಃ ನಾವು ಮೃಗೀಯ ಮಾನವರಾಗಿ ಉಳಿದುಕೊಳ್ಳುವೆವು. ತಿಂದು ಕುಡಿದು ಸಾಯುವೆವು. ಪುನಃ ಪುನಃ ತಿನ್ನುವುದರಲ್ಲಿ ಕುಡಿಯುವುದರಲ್ಲಿ ಮಗ್ನರಾಗುವೆವು. ಆದರೆ ಒಮ್ಮೆ ಅದರ ಆಕರ್ಷಣೆಗೆ ಸಿಲುಕಿದರೆ ಮತ್ತೆ ಯಾವ\break ಪ್ರಲೋಭನೆ ಬಂದರೂ ಹಿಂತಿರುಗದ, ಜೀವನ ಹೇಗೂ ಒಂದು ದಿನ ಕೊನೆಗಾಣಬೇಕು ಎಂದು ತಿಳಿದು ಸತ್ಯವನ್ನೇ ಬಯಸುವ, ಬಹಳ ಸುಲಭವಾಗಿ ಹಿಮ್ಮುಖರಾಗದ ಕೆಲವು ಪ್ರಚಂಡ ಅಸಾಧಾರಣ ವ್ಯಕ್ತಿಗಳಿರುವರು. ಅವರು ಜೀವನವನ್ನು ಒಂದು ಪವಿತ್ರ ವಿಜಯ ಸಾಧನೆಗೆ ಅರ್ಪಿಸುವರು. ಮೃಗೀಯ ಮಾನವನನ್ನು ಗೆಲ್ಲುವುದಕ್ಕಿಂತ, ಜನನ ಮರಣ, ಪಾಪ ಪುಣ್ಯಗಳ ರಹಸ್ಯವನ್ನು ಭೇದಿಸುವುದಕ್ಕಿಂತ ಮಿಗಿಲಾದ ಜಯವಾವುದಿದೆ?\break ಎನ್ನುವರು ಅವರು.

ಕೊನೆಗೆ ದೇವಸ್ಥಾನದಲ್ಲಿ ಪೂಜೆ ಮಾಡಲು ಅವರಿಗೆ ಅಸಾಧ್ಯವಾಯಿತು. ಅದನ್ನು ಬಿಟ್ಟು ಹತ್ತಿರ ಒಂದು ವನಕ್ಕೆ ಹೋಗಿ ವಾಸ ಮಾಡಿದರು. ಅನೇಕ ವೇಳೆ ಜೀವನದ ಈ\break ಕಾಲವನ್ನು ಕುರಿತು ಅವರು, ತಮಗೆ ಸೂರ್ಯೋದಯ ಸೂರ್ಯಾಸ್ತಗಳು ಅರಿವಾಗು\-ತ್ತಿರಲಿಲ್ಲ ಅಥವಾ ತಾವು ಹೇಗೆ ಜೀವಿಸುತ್ತಿದ್ದರು ಎನ್ನುವುದು ಕೂಡ ಗೊತ್ತಿರಲಿಲ್ಲ\break ವೆಂದು ಹೇಳಿರುವರು. ತಮ್ಮನ್ನು ತಾವು ಮರೆತರು; ಊಟಮಾಡುವುದನ್ನು ಮರೆತರು;\break ಈ ಸಮಯದಲ್ಲಿ ಅವರ ಹತ್ತಿರದ ಬಂಧು ಒಬ್ಬನು ಅವರನ್ನು ಪ್ರೀತಿಯಿಂದ\break ನೊಡಿಕೊಳ್ಳುತ್ತಿದ್ದನು. ಅವರ ಬಾಯಿಗೆ ಆಹಾರವನ್ನು ತುರುಕುತ್ತಿದ್ದನು. ಅದನ್ನು ಅವರು ಯಾವ ಪ್ರಜ್ಞೆಯೂ ಇಲ್ಲದೆ ನುಂಗುತ್ತಿದ್ದರು.

ಹೀಗೆ ಹಗಲು ರಾತ್ರಿಗಳು ಕಳೆದವು. ದಿನ ಕಳೆದಾಗ ಸಂಜೆ, ದೇವಸ್ಥಾನದ ಘಂಟೆ\break ಮತ್ತು ಪ್ರಾರ್ಥನಾ ನಿನಾದ ಕಾನನವನ್ನು ಮುಟ್ಟಿದಾಗ ಹುಡುಗನು ತುಂಬಾ\break ವ್ಯಾಕುಲನಾಗಿ ಅಳುತ್ತಿದ್ದನು ಮತ್ತು ಹೇಳುತ್ತಿದ್ದನು: “ಹೇ ತಾಯಿ, ಮತ್ತೊಂದು ದಿನ ವ್ಯರ್ಥವಾಯಿತು. ನೀನು ಇನ್ನೂ ಬರಲಿಲ್ಲ. ಈ ಜೀವನದ ಅಲ್ಪ ಆಯುಷ್ಯದಲ್ಲಿ\break ಆಗಲೇ ಒಂದು ದಿನ ಕಳೆಯಿತು. ನನಗಿನ್ನೂ ಸತ್ಯದ ಅರಿವಾಗಲಿಲ್ಲ.” ದಾರುಣವಾದ ಹೃದಯದ ಯಾತನೆಯಿಂದ ಅವರು ಕೆಲವು ವೇಳೆ ತಮ್ಮ ಮುಖವನ್ನು ನೆಲಕ್ಕೆ ತೀಡಿ\break ಅಳುವರು. ಆಗ ಈ ಒಂದು ಪ್ರಾರ್ಥನೆ ಅವರ ಅಂತರಾಳದಿಂದ ಹೊರಚಿಮ್ಮುವುದು;\break “ಹೇ ಜಗನ್ಮಯಿ, ನಿನ್ನ ಇರವನ್ನು ಇಲ್ಲಿ ತೋರು. ನನಗೆ ನೀನು ಮಾತ್ರ ಬೇಕು;\break ಮತ್ತಾವುದೂ ಬೇಡವಾಗಲಿ!” ಅವರು ತಮ್ಮ ಆದರ್ಶಕ್ಕೆ ತಕ್ಕಂತೆ ನಿಷ್ಠರಾಗಲು ಬಯಸಿದರು. ತಾಯಿ, ಸರ್ವಸ್ವವನ್ನು ತೆತ್ತಲ್ಲದೆ ಬರುವುದಿಲ್ಲವೆಂಬುದನ್ನು ಅವರು ಕೇಳಿದ್ದರು.\break ಜಗನ್ಮಾತೆ ಸರ್ವರಿಗೂ ದರ್ಶನವನ್ನೀಯಲು ಇಚ್ಛಿಸುವಳು, ಆದರೆ ಜನರಿಗೆ ಇಚ್ಛೆಯಿಲ್ಲ. ಜನರಿಗೆ ಬೇಡುವುದಕ್ಕೆ ಹಲವು ಬಗೆಯ ಕೆಲಸಕ್ಕೆ ಬಾರದ ವಿಗ್ರಹಗಳು ಬೇಕು, ಅವರ\break ಸುಖ ಒಂದೇ ಅವರಿಗೆ ಬೇಕು; ಜಗನ್ಮಾತೆ ಬೇಡ. ಹೃತ್ಪೂರ್ವಕವಾಗಿ ಎಲ್ಲವನ್ನೂ\break ತೊರೆದು ಜಗನ್ಮಾತೆಯೊಬ್ಬಳನ್ನೆ ಅವರು ಇಚ್ಛಿಸಿದ ಕ್ಷಣವೆ ಆಕೆ ಬರುವಳು ಎಂದು\break ಕೇಳಿದ್ದರು. ಈ ಒಂದು ಕಾರಣಕ್ಕಾಗಿ ತಮ್ಮನ್ನು ದಂಡಿಸಿಕೊಳ್ಳಲು ಮೊದಲು ಮಾಡಿದರು. ವ್ಯವಹಾರ ಪ್ರಪಂಚದಲ್ಲಿಯೂ ಅಕ್ಷರಶಃ ಸತ್ಯವಂತರಾಗಿರಲು ಯತ್ನಿಸಿದರು. ಅವರ\break ಹತ್ತಿರ ಇದ್ದ ಅಲ್ಪ ವಸ್ತುಗಳನ್ನೆಲ್ಲ ಆಚೆಗೆ ಎಸೆದರು. ಎಂದಿಗೂ ಹಣವನ್ನು ಮುಟ್ಟುವುದಿಲ್ಲ\-ವೆಂದು ಶಪಥ ಮಾಡಿದರು. “ನಾನು ದ್ರವ್ಯವನ್ನು ಮುಟ್ಟುವುದಿಲ್ಲ” ಎಂಬ ಒಂದು\break ಭಾವನೆ ಅವರ ಜೀವನದಲ್ಲಿ ಪಕ್ವವಾಗಿ ಹೋಗಿಬಿಟ್ಟಿತ್ತು. ಇದೊಂದು ವಿಚಿತ್ರವಾಗಿ\break ಕಾಣಬಹುದು. ಅವರ ಅನಂತರದ ಜೀವನದಲ್ಲಿಯೂ ಅವರು ನಿದ್ರಿಸುತ್ತಿದ್ದಾಗ\break ಹಣವನ್ನು ದೇಹಕ್ಕೆ ತಾಕಿಸಿದರೆ ಅವರ ಕೈ ಬಾಗುತ್ತಿತ್ತು. ದೇಹಕ್ಕೆ ಪಾರ್ಶ್ವವಾಯು\break ಬಂದಂತೆ ಆಗುತ್ತಿತ್ತು. ಕಾಮವೇ ಮಾನವನ ಶತ್ರು ಎಂಬುದು ಅವರ ಮನಸ್ಸಿನಲ್ಲಿ\break ಇದ್ದ ಇನ್ನೊಂದು ಭಾವನೆ. ಮನುಷ್ಯನ ಆತ್ಮ ಲಿಂಗಾತೀತ, ಆತ್ಮವು ಗಂಡಸೂ ಅಲ್ಲ\break ಹೆಂಗಸೂ ಅಲ್ಲ, ಜಗನ್ಮಾತೆಯ ದರ್ಶನಕ್ಕೆ ಆತಂಕಕಾರಿಯಾದ ಎರಡು ವಿಷಯಗಳೆಂದರೆ ಕಾಮ ಕಾಂಚನಗಳೆಂದು ಅವರು ಯೋಚಿಸಿದರು. ಸೃಷ್ಟಿಯೆಲ್ಲ ಜಗನ್ಮಾತೆಯ ಆವಿರ್ಭಾವ. ಆಕೆ ಹೆಂಗಸಿನ ದೇಹದಲ್ಲಿಯೂ ಇರುವಳು. ಪ್ರತಿಯೊಬ್ಬ ನಾರಿಯೂ ಜಗನ್ಮಾತೆಯ ಪ್ರತಿನಿಧಿಯಾಗಿರುವಾಗ ನಾನು ಹೆಂಗಸನ್ನು ಹೇಗೆ ಕಾಮ ದೃಷ್ಟಿಯಿಂದ ನೋಡಲಿ?\break ಪ್ರತಿಯೊಬ್ಬ ನಾರಿಯೂ ಅವರ ತಾಯಿ. ಎಲ್ಲ ನಾರಿಯರಲ್ಲಿಯೂ ಮಾತೆಯನ್ನಲ್ಲದೆ\break ಬೇರಾರನ್ನೂ ಕಾಣದ ಸ್ಥಿತಿಗೆ ಬರಬೇಕು. ಇದನ್ನು ತಮ್ಮ ಜೀವನದಲ್ಲಿ ಅನುಷ್ಠಾನಕ್ಕೆ ತಂದರು.

ಮಾನವ ಹೃದಯವನ್ನು ಆವರಿಸುವ ಅತಿ ಉತ್ಕಟ ಆಕಾಂಕ್ಷೆ ಇದು. ಅನಂತರ\break ಇವರೇ ನನಗೆ ಹೇಳಿದರು: “ಮಗು, ಒಂದು ಕೋಣೆಯಲ್ಲಿ ಚಿನ್ನದ ಚೀಲವಿದೆ.\break ಅದರ ಪಕ್ಕದ ಕೋಣೆಯಲ್ಲಿಯೇ ಕಳ್ಳನಿರುವನು. ಆಗ ಕಳ್ಳ ನಿದ್ದೆ ಮಾಡುವನೆಂದು\break ಭಾವಿಸುವೆಯ? ಇಲ್ಲ. ಅವನ ಮನಸ್ಸು ಆ ಕೋಣೆಗೆ ಹೋಗಿ ಚಿನ್ನದ ಚೀಲವನ್ನು ಯಾವಾಗ ಅಪಹರಿಸಿಯೇನು ಎಂದು ಆಲೋಚಿಸುತ್ತಿರುವುದು. ಈ ತೋರಿಕೆಯ\break ಜಗತ್ತಿನ ಹಿಂದೆ ಸತ್ಯವಿದೆ. ಒಬ್ಬ ದೇವರಿರುವನು. ನಾಶವಿಲ್ಲದವನೊಬ್ಬನಿರುವನು,\break ಅನಂತನೂ ಆನಂದಮಯನೂ ಆಗಿರುವವನೊಬ್ಬನಿರುವನು. ಆ ಆನಂದದೊಂದಿಗೆ\break ಹೋಲಿಸಿ ನೋಡಿದರೆ ನಮ್ಮ ಇಂದ್ರಿಯಸುಖಕ್ಕೆ ಬೆಲೆಯಿಲ್ಲ ಎಂದು ದೃಢವಾಗಿ\break ನಂಬಿದವನು ಅದನ್ನು ಪಡೆಯುವುದಕ್ಕೆ ಪ್ರಯತ್ನಪಡದೆ ಸುಮ್ಮನೆ ಇರುವನೆಂದು\break ತಿಳಿದೆಯ? ಒಂದು ಕ್ಷಣವಾದರೂ ಅವನು ಪ್ರಯತ್ನವನ್ನು ಬಿಡುವನೆ? ಇಲ್ಲ. ಆಸೆಯಿಂದ\break ಹುಚ್ಚನಾಗಿ ಹೋಗುವನು.” ದೈವೋನ್ಮಾದ ಆ ಹುಡುಗನನ್ನು ಮೆಟ್ಟಿಕೊಂಡಿತು. ಆ ಸಮಯದಲ್ಲಿ ಅವರಿಗೆ ಯಾವ ಗುರುಗಳೂ ಇರಲಿಲ್ಲ. ಬುದ್ಧಿವಾದ ಹೇಳುವುದಕ್ಕೆ\break ಯಾರೂ ಇರಲಿಲ್ಲ. ಎಲ್ಲರೂ ಈ ಹುಡುಗ ಹುಚ್ಚನೆಂದು ಭಾವಿಸಿದ್ದರು. ಇದು ಪ್ರಪಂಚದ ಸಾಮಾನ್ಯ ಸ್ಥಿತಿ. ಯಾರಾದರೂ ಪ್ರಪಂಚದ ಸುಖಭೋಗಗಳನ್ನು ತೊರೆದರೆ ಅವರನ್ನು ಹುಚ್ಚರೆಂದು ಕರೆಯುತ್ತಾರೆ. ಆದರೆ ಅಂಥವರೇ ಪ್ರಪಂಚದ ಸಾರ. ಅಂತಹ ಉನ್ಮಾದದಿಂದಲೇ ಭವಿಷ್ಯದಲ್ಲಿ ಜಗತ್ತನ್ನು ಚಲಿಸುವಂತಹ ಶಕ್ತಿಯು ಬರಬೇಕಾಗಿದೆ.

ಹೀಗೆ ಸಾಕ್ಷಾತ್ಕಾರಕ್ಕಾಗಿ ನಿರಂತರ ಸಾಧನೆಯಲ್ಲಿ ಹಲವು ದಿನಗಳು ವಾರಗಳು\break ತಿಂಗಳುಗಳು ಉರುಳಿದವು. ಹಲವು ನೋಟಗಳನ್ನು, ಅದ್ಭುತ ದೃಶ್ಯಗಳನ್ನು ಅವರು ನೋಡಲಾರಂಭಿಸಿದರು. ಅವರ ರಹಸ್ಯ ಸ್ವಭಾವ ಕ್ರಮೇಣ ಅನಾವರಣವಾದಂತಾಯಿತು. ಅವರಿಗೆ ಜಗನ್ಮಾತೆಯೇ ಗುರುವಾದಳು. ಯಾವ ಸತ್ಯಾನ್ವೇಷಣಕ್ಕಾಗಿ ಅವರು ಯತ್ನಿಸುತ್ತಿದ್ದರೊ ಅದನ್ನು ಅವಳೇ ಕಲಿಸಿದಳು. ಆ ಸಮಯದಲ್ಲಿ ಅಲ್ಲಿಗೆ ಅತಿ ಸುಂದರಳಾದ ಅಸದೃಶ ಪಂಡಿತಳೊಬ್ಬಳು ಬಂದಳು. ಅನಂತರ ಆ ಸ್ತ್ರೀಯನ್ನು ಕುರಿತು ಅವರು\break “ಆಕೆ ಜ್ಞಾನಿಯಲ್ಲ, ಜ್ಞಾನಘನಳು, ಸಾಕಾರ ಜ್ಞಾನ ಸ್ವರೂಪಳು” ಎನ್ನುತ್ತಿದ್ದರು. ಇಲ್ಲಿಯೂ ಹಿಂದೂ ಜನಾಂಗದ ಒಂದು ವೈಶಿಷ್ಟ್ಯವನ್ನು ನೋಡುವಿರಿ. ಪಾಶ್ಚಾತ್ಯರ ದೃಷ್ಟಿಯಿಂದ ಅಸ್ವತಂತ್ರರಾದ, ಅಜ್ಞಾನಾಂಧಕಾರದಲ್ಲಿ ಬಾಳುವ ಸಾಧಾರಣ ಹಿಂದೂ ಸ್ತ್ರೀಯರಲ್ಲಿ\break ಅದ್ವಿತೀಯ ಆಧ್ಯಾತ್ಮಿಕ ವ್ಯಕ್ತಿ ಉದಯಿಸಲು ಸಾಧ್ಯವಾಯಿತು. ಆಕೆ ಸಂನ್ಯಾಸಿನಿ,\break ಹೆಂಗಸರೂ ಪ್ರಪಂಚವನ್ನು ತ್ಯಾಗಮಾಡುವರು; ಆಸ್ತಿಯನ್ನು ತೊರೆದು ಲಗ್ನವಾಗದೆ\break ಭಗವದುಪಾಸನೆಯಲ್ಲಿ ಕಾಲಕಳೆಯುವರು. ಆಕೆ ಬಂದಳು. ವನದಲ್ಲಿರುವ ಹುಡುಗನ ವಿಷಯವನ್ನು ಕೇಳಿ ಅವನ ಬಳಿಗೆ ಹೊರಟಳು. ಅವನು ಸ್ವೀಕರಿಸಿದ ಮೊದಲನೆಯ\break ಸಹಾಯವೇ ಈ ಹೆಂಗಸಿನದು. ತಕ್ಷಣವೇ ಈ ಹುಡುಗನ ತೊಂದರೆ ಆಕೆಗೆ\break ತಿಳಿಯಿತು. ಆಕೆ ಹೀಗೆಂದಳು: “ಮಗು, ಯಾರಿಗೆ ಇಂತಹ ಹುಚ್ಚು ಹಿಡಿಯುವುದೋ\break ಆತ ಪುಣ್ಯಾತ್ಮ. ಪ್ರಪಂಚದ ಜನರೆಲ್ಲ ಹುಚ್ಚರು. ಕೆಲವರು ದ್ರವ್ಯಕ್ಕೆ, ಕೆಲವರು ಸುಖಕ್ಕೆ, ಕೆಲವರು ಕೀರ್ತಿಗೆ, ಮತ್ತೆ ಕೆಲವರು ನೂರಾರು ಪ್ರಾಪಂಚಿಕ ವಸ್ತುಗಳಿಗೆ. ಅವರು\break ಹೊನ್ನಿಗೆ ಹುಚ್ಚರು, ಗಂಡಿಗೆ ಹುಚ್ಚರು, ಹೆಣ್ಣಿಗೆ ಹುಚ್ಚರು, ನಿಷ್ಟ್ರಯೋಜಕ ವಸ್ತುಗಳಿಗೆ ಹುಚ್ಚರು, ಮತ್ತಾರನ್ನೊ ತುಳಿಯಬೇಕೆಂಬ ಹುಚ್ಚು ಕೆಲವರಿಗೆ, ಐಶ್ವರ್ಯವಂತರಾಗಬೇಕೆಂಬ ಹುಚ್ಚು ಕೆಲವರಿಗೆ, ಇನ್ನೂ ಎಷ್ಟೋ ಭ್ರಾಂತಿಗಳಿಗೆ ಹುಚ್ಚರು, ದೇವರಿಗೆ\break ಮಾತ್ರ ಇಲ್ಲ. ಅವರ ಹುಚ್ಚು ಅವರಿಗೆ ಮಾತ್ರ ಗೊತ್ತಾಗುವುದು. ಮತ್ತೊಬ್ಬ ಹೊನ್ನಿಗಾಗಿ ಹುಚ್ಚನಾದರೆ ಅವನಿಗೆ ಸಹಾನುಭೂತಿ ತೋರುವರು, ಮರುಗುವರು, ಅವನೇ\break ಸರಿ ಎನ್ನುವರು. ಏಕೆಂದರೆ ಹುಚ್ಚರು ತಾವೇ ಪ್ರಾಜ್ಞರೆಂದು ಭಾವಿಸುವರು. ದೇವರ ಮೇಲಿನ ಪ್ರೇಮದಿಂದ ಹುಚ್ಚರಾದರೆ ಉಳಿದ ಹುಚ್ಚರಿಗೆ ಅದು ಹೇಗೆ ಗೊತ್ತಾಗಬೇಕು? ‘ಅವನು ಹುಚ್ಚನಾಗಿರುವನು, ಅವನೊಂದಿಗೆ ನಮ್ಮ ವ್ಯವಹಾರವೇನೂ ಇಲ್ಲ’ ಎನ್ನುವರು.\break ಆದಕಾರಣವೆ ಅವರು ನಿನ್ನನ್ನು ಹುಚ್ಚನೆನ್ನುವರು. ನಿನ್ನ ಹುಚ್ಚು ನ್ಯಾಯವಾದುದು.\break ದೇವರಿಗಾಗಿ ಹುಚ್ಚನಾದವನೆ ಧನ್ಯ. ಅಂತಹವರು ಅತಿ ವಿರಳ ಪ್ರಪಂಚದಲ್ಲಿ.” ಈ\break ಹೆಂಗಸು ಹುಡುಗನ ಹತ್ತಿರ ಹಲವು ವರ್ಷಗಳವರೆಗೆ ಇದ್ದಳು. ಆಕೆ ಭರತಖಂಡದ\break ಹಲವಾರು ಸಾಧನೆಗಳನ್ನು ಅವರಿಗೆ ಕಲಿಸಿದಳು. ಹಲವು ಬಗೆಯ ಯೋಗಭ್ಯಾಸಗಳನ್ನು ಉಪದೇಶ ಮಾಡಿದಳು. ಅವರ ಆಧ್ಯಾತ್ಮಿಕ ಪ್ರಚಂಡ ಪ್ರವಾಹದಲ್ಲಿ ಒಂದು ಸಾಮರಸ್ಯವನ್ನು ತಂದಳು ಎನ್ನಬಹುದು.

ಇದಾದ ಅನಂತರ ಅದೇ ವನಕ್ಕೆ ಘನವಿದ್ವಾಂಸನೂ ತತ್ತ್ವಜ್ಞಾನಿಯೂ ಆದ\break ಪರಿವ್ರಾಜಕ ಸಂನ್ಯಾಸಿಯೊಬ್ಬನು ಬಂದನು. ಅವನೊಬ್ಬ ವಿಚಿತ್ರ ವ್ಯಕ್ತಿ, ಜ್ಞಾನಿ, ಸತ್ಯವಾಗಿಯೂ ಪ್ರಪಂಚವಿದೆ ಎಂಬುದನ್ನು ಅವನು ನಂಬಿಯೇ ಇರಲಿಲ್ಲ. ಇದನ್ನು\break ಪ್ರಕಟಿಸುವುದಕ್ಕಾಗಿ ಅವನು ಯಾವ ಚಾವಣಿಯ ಕೆಳಗೂ ಹೋಗುತ್ತಿರಲಿಲ್ಲ. ಯಾವಾಗಲೂ ಮಳೆ ಬಿಸಿಲಿನಲ್ಲಿ ಹೊರಗೇ ಇರುತ್ತಿದ್ದನು. ಈತನು ವೇದಾಂತ ತತ್ತ್ವವನ್ನು\break ಹುಡುಗನಿಗೆ ಬೋಧಿಸಲು ಉಪಕ್ರಮಿಸಿದನು. ಅವನು ತನ್ನ ಶಿಷ್ಯನು ಬಹುಬೇಗ ಹಲವು ರೀತಿಗಳಲ್ಲಿ ತನಗಿಂತ ಮೇಲಾಗಿರುವನೆಂದು ತಿಳಿದುಕೊಂಡು ಆಶ್ಚರ್ಯಪಟ್ಟನು. ಹಲವು ತಿಂಗಳುಗಳವರೆಗೆ ಶಿಷ್ಯನ ಹತ್ತಿರವಿದ್ದು, ಸಂನ್ಯಾಸ ದೀಕ್ಷೆಯನ್ನು ಕೊಟ್ಟು ಅವರಿಂದ ಬೀಳ್ಕೊಂಡನು.

ನನ್ನ ಗುರು ದೇವಸ್ಥಾನದಲ್ಲಿ ಪೂಜಾರಿಯಾಗಿದ್ದಾಗ ಅವರ ಅದ್ವಿತೀಯ ಪೂಜಾ\-ವಿಧಾನವನ್ನು ನೋಡಿ ಜನರು ಅವರನ್ನು ಹುಚ್ಚರೆಂದು ಭಾವಿಸಿದ್ದರು. ಅವರನ್ನು ಮನೆಗೆ\break ಕರೆದುಕೊಂಡು ಹೋಗಿ ಮದುವೆ ಮಾಡಿದರೆ ಅವರ ಮನಸ್ಸನ್ನು ಪುನಃ ಸ್ಥಿಮಿತಕ್ಕೆ\break ತರಬಹುದೆಂದು ಅವರ ನೆಂಟರು ಭಾವಿಸಿ ಒಂದು ಹುಡುಗಿಯೊಂದಿಗೆ ಲಗ್ನ ಮಾಡಿ\-ದರು. ಆದರೆ ಅವರು ಪುನಃ ಹಿಂತಿರುಗಿ ಬಂದು ಮತ್ತೂ ಹೆಚ್ಚು ದೈವೋನ್ಮಾದಗ್ರಸ್ತರಾದರು. ಕೆಲವು ಸಲ ನಮ್ಮ ದೇಶದಲ್ಲಿ ಸಣ್ಣ ವಯಸ್ಸಿನಲ್ಲೇ ಹುಡುಗರಿಗೆ ಮದುವೆ\break ಮಾಡುವರು. ಮದುವೆಯ ವಿಷಯದಲ್ಲಿ ವಧೂವರರ ಅಭಿಪ್ರಾಯಕ್ಕೆ ಗಮನ ಕೊಡದೆ ತಾಯಿ ತಂದೆಗಳೇ ಮದುವೆ ಮಾಡುವರು. ಆದರೆ ಅದು ಕೇವಲ ನಿಶ್ಚಿತಾರ್ಥಕ್ಕಿಂತ\break ಸ್ವಲ್ಪ ಮೇಲು ಅಷ್ಟೆ. ಮದುವೆಯ ಅನಂತರ ವಧೂವರರು ಅವರ ತಂದೆತಾಯಿಗಳ\break ಮನೆಯಲ್ಲಿ ವಾಸ ಮಾಡುವರು. ಹುಡುಗ ದೊಡ್ಡವನಾದ ಮೇಲೆ ಮಾತ್ರ ನಿಜವಾದ ಮದುವೆ. ಆಗ ಗಂಡ ಹೆಂಡತಿಯನ್ನು ತನ್ನ ಮನೆಗೆ ಕರೆದುಕೊಂಡು ಬರುವುದು\break ರೂಢಿ. ಆದರೆ ಈ ಗಂಡ, ತನಗೆ ಒಬ್ಬಳು ಹೆಂಡತಿ ಇರುವಳು ಎಂಬುದನ್ನೇ ಸಂಪೂರ್ಣ ಮರೆತುಬಿಟ್ಟಿದ್ದ. ದೂರದ ಹಳ್ಳಿಯಲ್ಲಿದ್ದ ಸತಿ ತನ್ನ ಗಂಡ ದೊಡ್ಡ ದೈವಭಕ್ತನಾಗಿರುವನೆಂದೂ ಕೆಲವರು ಅವನು ಹುಚ್ಚನಾಗಿರುವನೆಂದು ಹೇಳುವುದನ್ನೂ ಕೇಳಿದಳು. ತಾನೆ ಪ್ರತ್ಯಕ್ಷವಾಗಿ ಸತ್ಯಾಂಶವನ್ನು ತಿಳಿಯಬೇಕೆಂದು ಸಂಕಲ್ಪಿಸಿ ಆಕೆ ತನ್ನ ಗಂಡನಿರುವ ಊರಿಗೆ ನಡೆದು ಹೋದಳು. ಕೊನೆಗೆ ಆಕೆ ತನ್ನ ಗಂಡನ ಬಳಿಗೆ ಬಂದು ನಿಂತಳು! ಭರತಖಂಡದಲ್ಲಿ\break ಗಂಡಸಾಗಲಿ ಹೆಂಗಸಾಗಲಿ, ತ್ಯಾಗಜೀವನವನ್ನು ಕೈಗೊಂಡರೆ ಅವರು ಪ್ರಪಂಚದ ಎಲ್ಲಾ ಜವಾಬ್ದಾರಿಯಿಂದಲೂ ಪಾರಾದವರು ಎಂಬ ಭಾವನೆ ರೂಢಿಯಲ್ಲಿದೆ. ಆದರೂ ಸತಿಗೆ\break ತನ್ನ ಜೀವನದ ಮೇಲೆ ಅಧಿಕಾರವಿದೆ ಎಂದು ಆ ಪತಿ ಒಪ್ಪಿಕೊಂಡ. ಆ ಯುವಕನು\break ಸತಿಯ ಪಾದಗಳಿಗೆ ಎರಗಿ ಹೀಗೆಂದನು: “ನನಗೆ ಜಗನ್ಮಾತೆಯು ತಾನೇ ಎಲ್ಲಾ\break ಸ್ತ್ರೀಯರಲ್ಲಿಯೂ ಇರುವಳೆಂಬುದನ್ನು ತೋರಿರುವಳು. ಆದಕಾರಣ ನಾನು ಎಲ್ಲ\break ಸ್ತ್ರೀಯರನ್ನೂ ತಾಯಿಯಂತೆ ಕಾಣುವೆನು. ನಿನ್ನ ವಿಷಯದಲ್ಲಿ ನನಗೆ ಇರಬಹುದಾದ ಭಾವನೆ ಇದೊಂದೆ. ಆದರೆ ನಾನು ನಿನ್ನನ್ನು ಮದುವೆಯಾಗಿರುವುದರಿಂದ ನೀನು ನನ್ನನ್ನು ಈ ಪ್ರಪಂಚಕ್ಕೆ ಎಳೆಯಲು ಸಿದ್ಧಳಾಗಿದ್ದರೆ ನಾನು ನಿನ್ನ ಅಭಿಲಾಷೆಯನ್ನು ಪೂರ್ಣ ಮಾಡಲು ಸಿದ್ಧನಾಗಿರುವೆನು.”

ಪವಿತ್ರಾತ್ಮಳೂ ಉದಾರ ಹೃದಯಳೂ ಆಗಿದ್ದಳು ಆ ತರುಣಿ. ತನ್ನ ಪತಿಯ ಆಕಾಂಕ್ಷೆಯನ್ನು ತಿಳಿದು ಆಕೆ ಸಹಾನುಭೂತಿಯನ್ನು ತೋರಿದಳು. ಪ್ರಾಪಂಚಿಕ ವಿಷಯಕ್ಕೆ\break ಪತಿಯನ್ನು ಎಳೆಯಲು ಇಚ್ಛೆ ತನಗಿಲ್ಲವೆಂದು ತಕ್ಷಣವೆ ತಿಳಿಸಿದಳು. ಪತಿಗೆ ಸೇವೆ ಸಲ್ಲಿಸಿ\break ಅವರನ್ನು ತಿಳಿದುಕೊಳ್ಳುವುದೆ ತನ್ನ ಇಚ್ಛೆಯೆಂದಳು. ಯಾವಾಗಲೂ ಅವರನ್ನು\break ದೇವರೆಂದು ಗೌರವಿಸುತ್ತಿದ್ದು, ಆಕೆ ಅತಿ ಶ್ರೇಷ್ಠ ರೀತಿಯ ಶಿಷ್ಯೆಯಾದಳು. ತಮ್ಮ ಸತಿಯ ಅನುಮತಿಯಿಂದ ಅವರ ಕೊನೆಯ ಆತಂಕ ಕಳಚಿಬಿತ್ತು. ತಮಗೆ ಸೂಕ್ತ ತೋರಿದ\break ಜೀವನವನ್ನು ಮುಂದುವರಿಸಲು ಅವರು ಸ್ವತಂತ್ರರಾದರು.

ಅನಂತರ ಈ ವ್ಯಕ್ತಿಯ ಹೃದಯವನ್ನು ಆವರಿಸಿದ ಅಭಿಲಾಷೆಯೆಂದರೆ ಇತರ ಮತಗಳ ಸತ್ಯವನ್ನು ತಿಳಿಯಬೇಕೆಂಬುದು. ಅದುವರೆಗೂ ತಮ್ಮ ಧರ್ಮವನ್ನು ಬಿಟ್ಟು\break ಅನ್ಯಧರ್ಮಗಳ ಪರಿಚಯ ಅವರಿಗೆ ಇರಲಿಲ್ಲ. ಉಳಿದ ಧರ್ಮಗಳು ಹೇಗಿವೆ\break ಎಂಬುದನ್ನು ತಿಳಿಯಲು ಅವರು ಆಶಿಸಿದರು. ಅನ್ಯಧರ್ಮಗಳ ಗುರುಗಳನ್ನು\break ಹುಡುಕಿದರು. ಭರತಖಂಡದಲ್ಲಿ ಗುರುವೆಂದರೆ ಏನೆಂಬುದನ್ನು ನೀವು ಯಾವಾಗಲೂ ಜ್ಞಾಪಕದಲ್ಲಿಡಬೇಕು. ಗುರುವೆಂದರೆ ಪುಸ್ತಕ ಪಿಶಾಚಿಯಲ್ಲ, ಸಾಕ್ಷಾತ್ಕಾರವನ್ನು ಪಡೆ\break ದವನು, ಸತ್ಯವನ್ನು ಪ್ರತ್ಯಕ್ಷವಾಗಿ ಮನಗಂಡವನು; ಆ ಜ್ಞಾನ ಅನ್ಯರಿಂದ ಕೇಳಿ\break ತಿಳಿದುದಲ್ಲ. ಅವರಿಗೆ ಒಬ್ಬ ಮಹಮ್ಮದೀಯ ಫಕೀರ ಸಿಕ್ಕಿದ. ಅವನೊಂದಿಗೆ ಅವರು\break ವಾಸಮಾಡುವುದಕ್ಕೆ ಹೋದರು. ಆತ ಹೇಳಿದ ಸಾಧನೆಯನ್ನು ಶ್ರದ್ಧೆಯಿಂದ ಮಾಡಿದ ಮೇಲೆ, ಈ ಭಕ್ತಿಮಾರ್ಗವು ತಾವು ಆಗಲೇ ಸೇರಿದ ಗುರಿಯೆಡೆಗೆ ತಮ್ಮನ್ನು ತಂದುದನ್ನು ನೋಡಿ ಅವರಿಗೆ ಆಶ್ಚರ್ಯವಾಯಿತು. ಇತರ ಧರ್ಮಗಳನ್ನು ಅನುಸರಿಸಿಯೂ ಅವರು ಇದೇ ಅನುಭವವನ್ನು ಪಡೆದರು. ತಮಗೆ ತೋರಿದ ಎಲ್ಲಾ ಪಂಥಗಳಿಗೂ ಹೋದರು. ತಾವು ಯಾವುದನ್ನು ಅನುಸರಿಸಿದರೂ ತಮ್ಮ ಇಡೀ ಮನಸ್ಸನ್ನು ಅದರಲ್ಲಿಟ್ಟರು. ತಮ್ಮ\break ಸ್ವಂತ ಅನುಭವದಿಂದ ಈ ನಿರ್ಧಾರಕ್ಕೆ ಬಂದರು; ಎಲ್ಲಾ ಧರ್ಮಗಳ ಗುರಿಯೂ\break ಒಂದೇ, ಪ್ರತಿಯೊಬ್ಬರೂ ಬೋಧಿಸುವುದು ಒಂದನ್ನೇ, ಮಾರ್ಗದಲ್ಲಿ ಮಾತ್ರ ವ್ಯತ್ಯಾಸ, ಭಾಷೆಯಲ್ಲಿ ಅದಕ್ಕಿಂತ ಹೆಚ್ಚು ವ್ಯತ್ಯಾಸ, ಅಷ್ಟೆ. ಅಂತರಾಳದಲ್ಲಿ ಎಲ್ಲಾ ಧರ್ಮಗಳ\break ಮತ್ತು ಪಂಗಡಗಳ ಗುರಿ ಒಂದೇ. ಅವರು ಹೊಡೆದಾಡುವುದು ಕೇವಲ ಸ್ವಾರ್ಥಕ್ಕಾಗಿ. ಅವರ ಆಕಾಂಕ್ಷೆ ಸತ್ಯವಲ್ಲ; “ನನ್ನ ಹೆಸರು”, “ನಿನ್ನ ಹೆಸರು” ಎಂಬುದೇ ಅವರಿಗೆ\break ಮುಖ್ಯ. ಇಬ್ಬರೂ ಒಂದೇ ಸತ್ಯವನ್ನು ಬೋಧಿಸಿದರು. ಅದರಲ್ಲಿ ಒಬ್ಬ, “ಅದು\break ಸತ್ಯವಾಗಲಾರದು. ಅದರ ಮೇಲೆ ನನ್ನ ಹೆಸರಿನ ಚೀಟಿಯನ್ನು ಅಂಟಿಸಿಲ್ಲ, ಆದಕಾರಣ ನೀವು ಅದನ್ನು ಕೇಳಬೇಡಿ” ಎಂದ. ಮತ್ತೊಬ್ಬನು, “ಅವನನ್ನು ಕೇಳಬೇಡಿ, ನಾನು\break ಹೇಳುವುದನ್ನೆ ಅವನು ಬೋಧಿಸುತ್ತಿದ್ದರೂ ಅದು ಸತ್ಯವಲ್ಲ. ಕಾರಣ ನನ್ನ ಹೆಸರಿನಲ್ಲಿ ಬೋಧಿಸುತ್ತಿಲ್ಲ” ಎಂದ.

ನನ್ನ ಗುರುದೇವರು ಅದನ್ನು ಕಂಡುಹಿಡಿದರು. ಅನಂತರ ವಿನಯವನ್ನು ಅಭ್ಯಾಸ ಮಾಡತೊಡಗಿದರು. ಏಕೆಂದರೆ ಎಲ್ಲಾ ಧರ್ಮಗಳ ಒಂದು ಭಾವನೆ “ನಾನಲ್ಲ,\break ನೀನು” ಎಂಬುದು. ಯಾರು ನಾನಲ್ಲ ಎನ್ನುವರೋ ಅಲ್ಲಿ ದೇವರು ನೆಲಸುವನು.\break ಎಷ್ಟು ಕಡಿಮೆ ಅಹಂಕಾರವಿರುವುದೋ ಅಷ್ಟು ಹೆಚ್ಚು ದೇವರು ಇರುವನು. ಎಲ್ಲಾ ಧರ್ಮಗಳ ಸತ್ಯವಿದು ಎಂಬುದನ್ನು ತಿಳಿದು ಅದನ್ನು ಅನುಷ್ಠಾನಕ್ಕೆ ತರಲು ಅವರು\break ಯತ್ನಿಸಿದರು. ನಾನು ನಿಮಗೆ ಆಗಲೇ ಹೇಳಿದಂತೆ ಅವರು ಏನನ್ನಾದರೂ\break ಮಾಡಬೇಕೆಂದು ಬಯಸಿದಾಗ ಕೇವಲ ಸೂಕ್ಷ್ಮಸಿದ್ಧಾಂತಗಳ ಕಡೆಗೆ ಗಮನವನ್ನು\break ಕೊಡುತ್ತಿರಲಿಲ್ಲ. ತಕ್ಷಣ ಅದನ್ನು ಅಭ್ಯಾಸ ಮಾಡಲು ಯತ್ನಿಸುತ್ತಿದ್ದರು. ದಾನ,\break ಸರ್ವ ಸಮಾನತೆ, ಅನ್ಯರ ಹಕ್ಕು ಮುಂತಾದ ಹಲವು ವಿಷಯಗಳ ಮೇಲೆ ಜನರು\break ಅತಿ ಸುಂದರವಾಗಿ ಮಾತನಾಡುವುದನ್ನು ನಾವು ಕೇಳಿರುವೆವು. ಆದರೆ ಇದು ಕೇವಲ ಬಾಯಿಮಾತು. ಸಿದ್ಧಾಂತವನ್ನು ತಮ್ಮ ಜೀವನದಲ್ಲಿಯೇ ಅನುಷ್ಠಾನಕ್ಕೆ ತಂದ\break ಮಹಾನುಭಾವರನ್ನು ನೋಡಲು ನಾನು ತುಂಬ ಅದೃಷ್ಟಶಾಲಿಯಾಗಿದ್ದೆ. ತಮಗೆ\break ಸರಿಯೆಂದು ತೋರಿದ ಎಲ್ಲಾ ವಿಷಯಗಳನ್ನೂ ಅನುಷ್ಠಾನಕ್ಕೆ ತರುವ ಅದ್ವಿತೀಯ ಶಕ್ತಿ\break ಅವರಲ್ಲಿ ಇತ್ತು.

ಹೊಲೆಯರ ಒಂದು ಕುಟುಂಬ ಹತ್ತಿರದಲ್ಲಿ ವಾಸಮಾಡುತ್ತಿತ್ತು. ಭರತಖಂಡದಲ್ಲೆಲ್ಲ ಲಕ್ಷಾಂತರ ಮಂದಿ ಹೊಲೆಯರು ಇರುವರು. ಅವರು ಸಮಾಜದಲ್ಲಿ ಬಹಳ ಕೆಳ ಮಟ್ಟದಲ್ಲಿ ಇರುವರು. ಬ್ರಾಹ್ಮಣನೇನಾದರೂ ಮನೆಯಿಂದ ಬರುತ್ತಾ ಹೊಲೆಯನ ಮುಖವನ್ನು ನೋಡಿದರೆ, ಅಂದಿನ ದಿನ ಉಪವಾಸವಿದ್ದು ಮಂತ್ರೋಚ್ಚಾರಣೆಯಿಂದ ಪ್ರಾಯಶ್ಚಿತ್ತ ಮಾಡಿಕೊಳ್ಳಬೇಕೆಂದು ನಮ್ಮ ಶಾಸ್ತ್ರ ಹೇಳುತ್ತದೆ. ಕೆಲವು ವೇಳೆ ಹಿಂದೂಗಳ ಊರಿಗೆ\break ಹೊಲೆಯನು ಪ್ರವೇಶಿಸುವಾಗ ತಾನು ಹೊಲೆಯನೆಂದು ತೋರಿಸಿಕೊಳ್ಳುವುದಕ್ಕೆ ಕಾಗೆಯ ಒಂದು ಪುಕ್ಕವನ್ನು ಧರಿಸಿ ಗಟ್ಟಿಯಾಗಿ, “ಪಾರಾಗಿ, ಹೊಲೆಯ ದಾರಿಯಲ್ಲಿ ಬರುತ್ತಿರುವನು”\break ಎಂದು ಅರಚಿಕೊಳ್ಳಬೇಕು ಜನರು ಅವನೆಡೆಯಿಂದ ಇದ್ದಕ್ಕಿದ್ದಂತೆ ಓಡಿಹೋಗುವುದನ್ನು ನೋಡುವಿರಿ. ಏಕೆಂದರೆ ಅಕಸ್ಮಾತ್​ ಅವನನ್ನು ಮುಟ್ಟಿದರೆ ಬಟ್ಟೆ ಬದಲಾಯಿಸಿ, ಸ್ನಾನಮಾಡಿ, ಇನ್ನೂ ಹಲವಾರು ಪ್ರಾಯಶ್ಚಿತ್ತ ಮಾಡಿಕೊಳ್ಳಬೇಕು. ಸಹಸ್ರಾರು ವರುಷಗಳಿಂದಲೂ ಹೊಲೆಯನು, ತನ್ನ ಸ್ವರ್ಶ ಎಲ್ಲರನ್ನೂ ಅಪವಿತ್ರ ಮಾಡುವುದು, ಇದು ಸತ್ಯ ಎಂದು ನಂಬಿರುವನು. ನನ್ನ ಗುರುಗಳು ಹೊಲೆಯನ ಮನೆಗೆ ಹೋಗಿ, ತಮಗೆ ಅವನ ಮನೆಯನ್ನು ಶುಚಿಮಾಡಲು ಅನುಮತಿಯನ್ನು ಕೊಡಬೇಕೆಂದು ಕೇಳಿದರು. ಹೊಲೆಯನ ಕೆಲಸ ಊರುಕೇರಿಯನ್ನು ಶುದ್ಧಿ ಮಾಡುವುದು, ಮನೆಗಳನ್ನು ಚೊಕ್ಕಟವಾಗಿ ಇಟ್ಟಿರುವುದು. ಅವನು ಮನೆಯ ಮುಂಭಾಗದಿಂದ ಪ್ರವೇಶಿಸಲಾರ; ಮನೆಯ ಹಿಂಭಾಗದಿಂದ ಬರುವನು. ಅವನು ನಡೆದ ಸ್ಥಳಗಳನ್ನೆಲ್ಲ ಗಂಗಾ ನೀರನ್ನು ಚೆಲ್ಲಿ ಪವಿತ್ರಮಾಡುವರು. ಬ್ರಾಹ್ಮಣ ಜನ್ಮತಃ ಪವಿತ್ರತೆಯ ಚಿಹ್ನೆಯಾಗಿರುವನು. ಹೊಲೆಯ ಅದಕ್ಕೆ ವಿರೋಧವಾಗಿರುವನು. ಈ ಬ್ರಾಹ್ಮಣ ಗುರುಗಳು ಹೊಲೆಯನ ಮನೆಯಲ್ಲಿ ನೀಚ ಕೆಲಸವನ್ನು ಮಾಡಲು\break ಅನುಮತಿಯನ್ನು ಬೇಡಿದರು. ಹೊಲೆಯ ಇದಕ್ಕೆ ಅನುಮತಿ ಕೊಡಲಿಲ್ಲ. ಇಂತಹ ಹೀನ ಕೆಲಸವನ್ನು ಬ್ರಾಹ್ಮಣ ಮಾಡಿದರೆ ಇದೊಂದು ಮಹಾಪಾಪವೆಂದೂ ಹೊಲೆಯರೆಲ್ಲ ನಾಶವಾಗಿ ಬಿಡುವರೆಂದೂ ಅವನು ತಿಳಿದಿದ್ದ. ಆದ್ದರಿಂದಲೇ ಅವನು ಅನುಮತಿ ಕೊಡಲಿಲ್ಲ. ಅರ್ಧರಾತ್ರಿಯಲ್ಲಿ, ಎಲ್ಲರೂ ಮಲಗಿದ್ದಾಗ ರಾಮಕೃಷ್ಣರು ಮನೆಯನ್ನು ಪ್ರವೇಶಿಸುತ್ತಿದ್ದರು. ಅವರಿಗೆ ದೀರ್ಘ ಕೇಶವಿತ್ತು. “ಹೇ ಜಗನ್ಮಾತೆ, ನನ್ನನ್ನು ಹೊಲೆಯನ ದಾಸನನ್ನಾಗಿ ಮಾಡು. ಹೊಲೆಯನಿಗಿಂತ ಕೀಳೆಂದು ಭಾವಿಸುವಂತೆ ಮಾಡು” ಎನ್ನುತ್ತಾ ತಮ್ಮ ಕೇಶದಿಂದ ನೆಲವನ್ನು ಒರಸುತ್ತಿದ್ದರು. “ಯಾರು ನನ್ನ ಭಕ್ತರನ್ನು ಪೂಜಿಸುವರೋ ಅವರೇ ನನ್ನನ್ನು ವಿಶೇಷವಾಗಿ ಆರಾಧಿಸುವವರು. ಅವರೆಲ್ಲ ನನ್ನ ಮಕ್ಕಳು. ಅವರಿಗೆ ಸೇವೆ ಸಲ್ಲಿಸುವುದು ನಮಗೊಂದು ಸದವಕಾಶ” ಎಂಬುದೇ ಹಿಂದೂ ಶಾಸ್ತ್ರದ ಬೋಧನೆ.

ಅವರು ಇನ್ನೂ ಹಲವು ಸಾಧನೆಗಳನ್ನು ಮಾಡಿದರು. ಅದನ್ನು ಹೇಳಬೇಕಾದರೆ ಬಹಳ ದೀರ್ಘಕಾಲ ಹಿಡಿಯುವುದು. ಅವರ ಜೀವನದ ಒಂದು ಸೂಕ್ಷ್ಮ ಪರಿಚಯವನ್ನು ಮಾತ್ರ ನಾನು ಕೊಡುತ್ತೇನೆ. ಅನೇಕ ವರ್ಷಗಳ ಕಾಲ ಅವರು ಹೀಗೆ ತಮಗೆ ತಾವು ಶಿಕ್ಷಣವನ್ನು ನೀಡಿಕೊಂಡರು. ಪರಿಪೂರ್ಣನಾಗಬೇಕಾದರೆ ಕಾಮಭಾವನೆ ಹೋಗಬೇಕು. ಆತ್ಮನಿಗೆ ಲಿಂಗವಿಲ್ಲ. ಅದು ಗಂಡೂ ಅಲ್ಲ ಹೆಣ್ಣೂ ಅಲ್ಲ. ಲಿಂಗವಿರುವುದು\break ದೇಹದಲ್ಲಿ. ಆತ್ಮನನ್ನು ಬಯಸುವವರು ಲಿಂಗ ವೈವಿಧ್ಯಕ್ಕೆ ಗಮನಕೊಡಲಾರರು. ಪುರುಷನಾಗಿ ಹುಟ್ಟಿದ ನನ್ನ ಗುರುಗಳು ಈಗ ಎಲ್ಲದರಲ್ಲಿಯೂ ಸ್ತ್ರೀರೂಪವನ್ನು ತರಲು ಯತ್ನಿಸಿದರು. ತಾವು ಸ್ತ್ರೀಯೆಂದು ಭಾವಿಸತೊಡಗಿದರು; ಸ್ತ್ರೀಯಂತೆ ಉಟ್ಟರು, ತೊಟ್ಟರು. ಸ್ತ್ರೀಯಂತೆ ಮಾತನಾಡಿದರು. ಸದ್ಗೃಹಸ್ಥರ ಒಂದು ಕುಟುಂಬದಲ್ಲಿ ಮಹಿಳೆಯಂತೆ\break ಮಹಿಳೆಯರೊಡನೆ ವಾಸಿಸಿದರು. ಈ ಸಾಧನೆಯಲ್ಲಿ ಹಲವು ವರ್ಷಗಳು ಕಳೆದಮೇಲೆ ಅವರ ಮನಸ್ಸು ಬದಲಾವಣೆಯಾಯಿತು. ಲಿಂಗಭಾವನೆಯನ್ನು ಪೂರ್ಣ ಮರೆತರು. ಅವರ ಇಡೀ ಜೀವನ ದೃಷ್ಟಿಯೇ ಬದಲಾಯಿತು. ಪಾಶ್ಚಾತ್ಯ ದೇಶಗಳಲ್ಲಿ ಸ್ತ್ರೀಯರ\break ಆರಾಧನೆಯ ವಿಷಯವನ್ನು ಕೇಳಿರುವೆವು. ಆದರೆ ಇದು ಆಕೆಯ ಯೌವನ ಮತ್ತು\break ಸೌಂದರ್ಯಕ್ಕೆ ಕೊಡುವ ಗೌರವ. ಶ‍್ರೀರಾಮಕೃಷ್ಣರಿಗಾದರೂ ಸ್ತ್ರೀಯ ಆರಾಧನೆಯೆಂದರೆ ಪ್ರತಿಯೊಬ್ಬ ನಾರೀವದನವೂ ಜಗನ್ಮಾತೆಯ ಪ್ರತಿಬಿಂಬವಲ್ಲದೆ ಬೇರೆಯಲ್ಲ, ಸ್ಪರ್ಶ\break ಮಾಡಲೂ ಯೋಗ್ಯರಲ್ಲದ ಸ್ತ್ರೀಯರ ಮುಂದೆ ಬಾಗಿ ಕಂಬನಿದುಂಬಿ “ಹೇ! ಜಗಜ್ಜನನಿ, ಒಂದು ರೂಪದಲ್ಲಿ ನೀನು ಬೀದಿಯಲ್ಲಿರುವೆ. ಮತ್ತೊಂದು ರೂಪದಲ್ಲಿ ನೀನೇ ಜಗಜ್ಜನನಿ. ತಾಯಿ, ನಿನಗೆ ನಮಸ್ಕಾರ” ಎಂದು ಹೇಳುವುದನ್ನು ನಾನೇ ಪ್ರತ್ಯಕ್ಷ ನೋಡಿರುವೆನು. ಕಾಮವೆಲ್ಲಾ ನಾಶವಾಗಿ, ಸ್ತ್ರೀಕುಲವನ್ನೇ ಪ್ರೀತಿ ಗೌರವಗಳಿಂದ ನೋಡುವಾಗ; ಎಲ್ಲಾ ಸ್ತ್ರೀಮುಖ ಭಾವವೂ ರೂಪಾಂತರ ಹೊಂದಿ, ಅಲ್ಲಿ ಮಾನವಕೋಟಿಯನ್ನು ಸಂರಕ್ಷಿಸುತ್ತಿರುವ ಆನಂದಮಯಿ ಆದಿಶಕ್ತಿಯ ಮುಖವೇ ಕಾಣುವಾಗ; ಆ ದರ್ಶನದಿಂದ ಬರುವ ದಿವ್ಯಾನಂದವನ್ನು ಊಹಿಸಿ ನೋಡಿ! ನಮಗೆ ಬೇಕಾಗಿರುವುದು ಅದು. ನಾವು ಸ್ತ್ರೀಯ ಹಿಂದಿರುವ ದೈವತ್ವವನ್ನು ಎಂದಾದರೂ ಅಲ್ಲಗಳೆಯುವುದಕ್ಕೆ ಸಾಧ್ಯವೆ? ಹಿಂದೆ ಅದು ಎಂದಿಗೂ ಸಾಧ್ಯವಾಗಿಲ್ಲ. ಮುಂದೆಯೂ ಅದು ಸಾಧ್ಯವಾಗಲಾರದು. ಅದು ಯಾವಾಗಲೂ ಸ್ವತಃ ವ್ಯಕ್ತವಾಗುವುದು. ತಪ್ಪದೆ ಅದು ಮೋಸವನ್ನು ಕಂಡುಹಿಡಿಯುವುದು, ಕಾಪಟ್ಯವನ್ನು ಕಂಡುಹಿಡಿಯುವುದು. ನಿಜವಾಗಿಯೂ ಅದು ಸತ್ಯದ ಶಾಂತಿಯನ್ನೂ, ಆಧ್ಯಾತ್ಮಿಕ ಕಾಂತಿಯನ್ನೂ, ಪವಿತ್ರತೆಯ ಭಾವವನ್ನೂ ಗ್ರಹಿಸುವುದು. ನಿಜವಾದ ಆಧ್ಯಾತ್ಮಿಕ ಸಿದ್ಧಿಯನ್ನು ಪಡೆಯಬೇಕಾದರೆ ಅಂತಹ ಪವಿತ್ರತೆ ಅತ್ಯಾವಶ್ಯಕ.

ಕಠಿಣವಾದ ಅಕಳಂಕ ಪವಿತ್ರತೆ ಇವರ ಜೀವನದಲ್ಲಿ ಉದಿಸಿತು. ನಮ್ಮ ಜೀವನದ ಮುಂದಿರುವ ಹೋರಾಟವನ್ನೆಲ್ಲ ಅವರು ಮೀರಿ ಹೋಗಿದ್ದರು. ತಮ್ಮ ಆಯುಷ್ಯದ ಮುಕ್ಕಾಲು ಪಾಲನ್ನು ಸವೆಯಿಸಿ ಪಡೆದ ಆಧ್ಯಾತ್ಮಿಕ ಅನರ್ಘ್ಯ ರತ್ನಗಳನ್ನು ಮಾನವಕೋಟಿಗೆ ದಾನಮಾಡುವುದಕ್ಕಾಗಿ ಅವರು ಸಿದ್ಧರಾಗಿದ್ದರು. ಅವರ ಬೋಧನೆ ಮತ್ತು ಉಪದೇಶ ವಿಚಿತ್ರ ರೀತಿಯದು. ನಮ್ಮ ದೇಶದಲ್ಲಿ ಗುರುವನ್ನು ಅತಿ ಪೂಜ್ಯದೃಷ್ಟಿಯಿಂದ ನೋಡುತ್ತೇವೆ. ಅವರನ್ನು ದೇವರಂತೆ ಭಾವಿಸುತ್ತೇವೆ. ನಮ್ಮ ತಾಯಿತಂದೆಗಳಿಗೂ ಅಷ್ಟು ಗೌರವವನ್ನು ತೋರುವುದಿಲ್ಲ. ತಾಯಿ ತಂದೆಗಳು ನಮಗೆ ದೇಹವನ್ನು ನೀಡುವರು. ಗುರು\break ಮುಕ್ತಿ ಮಾರ್ಗವನ್ನು ತೋರುವನು. ನಾವು ಗುರುವಿನ ಮಕ್ಕಳು; ಗುರುವಿನ ಆಧ್ಯಾತ್ಮಿಕ ಸಂತತಿಯಲ್ಲಿ ನಾವು ಜನಿಸುವುದು. ಗುರುವಿಗೆ ಗೌರವ ತೋರಲು ಎಲ್ಲ ಹಿಂದೂಗಳೂ ಮುಂದೆ ಬರುವರು, ಅವನನ್ನು ಆದರಿಸುವರು. ಆದರೆ ಈ ಗುರುವಿಗೆ ಜನರು ತಮ್ಮನ್ನು ಗೌರವಿಸಬೇಕೆ ಬೇಡವೆ ಎಂಬ ಭಾವನೆಯು ಇರಲಿಲ್ಲ. ತಾನೊಬ್ಬ ದೊಡ್ಡ ಗುರುವೆಂಬ\break ಭಾವನೆಯೂ ಅವರಲ್ಲಿ ಇರಲಿಲ್ಲ. ಎಲ್ಲವನ್ನೂ ಮಾಡುತ್ತಿರುವುದು ಜಗನ್ಮಾತೆಯಲ್ಲದೆ ತಾನಲ್ಲವೆಂದು ಅವರು ಭಾವಿಸಿದ್ದರು. “ನನ್ನ ಬಾಯಿಂದ ಏನಾದರೂ ಒಳ್ಳೆಯದು\break ಬಂದರೆ ಅದನ್ನು ಹೊರಡಿಸುತ್ತಿರುವವಳು ತಾಯಿ. ಅದರೊಂದಿಗೆ ನನಗೇನು ಸಂಬಂಧವಿದೆ?” ಎಂದು ಅನವರತವೂ ಹೇಳುತ್ತಿದ್ದರು. ಅವರು ಮಾಡಿದ ಪ್ರತಿಯೊಂದು ಕೆಲಸದ ಹಿಂದೆ ಇದ್ದ ಭಾವ ಇದು. ಸಾಯುವವರೆಗೆ ಅವರು ಇದನ್ನು ತೊರೆಯಲಿಲ್ಲ. ಅವರು ಯಾರನ್ನೂ ಅರಸಿಕೊಂಡು ಹೋಗಲಿಲ್ಲ. ‘ಮೊದಲು ಶೀಲವನ್ನು ರೂಢಿಸಿ, ಆಧ್ಯಾತ್ಮಿಕ\break ಶಕ್ತಿಯನ್ನು ಗಳಿಸಿ; ಫಲ ತಾನಾಗಿ ಬರುವುದು.’ ಇದೇ ಅವರ ಸಿದ್ಧಾಂತ. ಅವರಿಗೆ ಅತ್ಯಂತ ಪ್ರಿಯವಾಗಿದ್ದ ಉಪಮೆ ಇದು: “ಕಮಲ ಅರಳಿದರೆ ಮಧುವನ್ನು ಹುಡುಕಿಕೊಂಡು\break ದುಂಬಿಗಳು ತಾವಾಗಿ ಬರುವುವು. ಅದರಂತೆಯೇ ನಿಮ್ಮ ಶೀಲದ ಕಮಲ ಸಂಪೂರ್ಣ ಅರಳಲಿ. ಫಲ ತಾನಾಗಿ ಬರುವುದು.” ನಾವು ಕಲಿಯಬೇಕಾದ ದೊಡ್ಡ ನೀತಿ ಇದು. ನೂರಾರು ವೇಳೆ ನನ್ನ ಗುರು ಇದನ್ನು ಬೋಧಿಸಿದರು. ಆದರೂ ಅನೇಕ ವೇಳೆ ನಾನು ಇದನ್ನು ಮರೆಯುವೆನು. ಆಲೋಚನಾ ಶಕ್ತಿಯನ್ನು ಎಲ್ಲೋ ಕೆಲವರು ತಿಳಿದುಕೊಳ್ಳಬಲ್ಲರು. ಒಬ್ಬ ಗುಹೆಗೆ ಹೋಗಿ ಯಾರಿಗೂ ಕಾಣದೆ ಒಂದು ಮಹದಾಲೋಚನೆಯನ್ನು ಮಾಡಿ ಸತ್ತರೆ, ಆ ಆಲೋಚನೆ ಗುಹೆಯ ಗೋಡೆಯನ್ನು ತೂರಿ, ಆಕಾಶದಲ್ಲಿ ಸ್ಪಂದಿಸಿ, ಎಲ್ಲಾ ಜನಾಂಗಗಳ ಅಂತರಾಳದಲ್ಲಿಯೂ ಪ್ರವಹಿಸುವುದು. ಇದು ಆಲೋಚನಾ ಶಕ್ತಿ.\break ಮತ್ತೊಬ್ಬರಿಗೆ ನಿಮ್ಮ ಭಾವನೆಯನ್ನು ಕೊಡಲು ಅವಸರಪಡಬೇಡಿ. ಮೊದಲು\break ಕೊಡುವುದಕ್ಕೆ ನಿಮ್ಮಲ್ಲಿ ಏನಾದರೂ ಇರಲಿ. ಯಾರಿಗೆ ಕೊಡುವುದಕ್ಕೆ ಏನಾದರೂ ಇದೆಯೊ ಅವನೇ ಬೋಧಿಸಬಲ್ಲ. ಬೋಧನೆ ಎಂದರೆ ಬರಿಯ ಮಾತಲ್ಲ, ಸಿದ್ಧಾಂತವನ್ನು ಸಾರುವುದಲ್ಲ; ಅದೊಂದು ಸಂವಹನ. ನಾನು ನಿಮಗೆ ಒಂದು ಹೂವನ್ನು ಕೊಡುವಂತೆ ಆಧ್ಯಾತ್ಮಿಕತೆ\-ಯನ್ನು ಕೊಡಬಹುದು. ಇದು ಅಕ್ಷರಶಃ ಸತ್ಯ. ಭಾರತದಲ್ಲಿ ಈ ಭಾವನೆ ಹಳೆ\-ಯದು. ಪಾಶ್ಚಾತ್ಯ ಧರ್ಮ ಪ್ರವರ್ತಕರ ಪರಂಪರೆಯಲ್ಲಿ ಇಟ್ಟಿರುವ ನಂಬಿಕೆ ಇದಕ್ಕೆ ತಕ್ಕ\break ಉದಾಹರಣೆ. ಮೊದಲು ಚಾರಿತ್ರ್ಯವನ್ನು ರೂಢಿಸಿ. ನೀವು ಮಾಡಬಹುದಾದ ಅತಿ ಪವಿತ್ರ ಕರ್ತವ್ಯ ಅದು. ಮೊದಲು ಸತ್ಯವನ್ನು ನೀವು ಗ್ರಹಿಸಿ. ಅನಂತರ ನಿಮ್ಮಿಂದ ಅದನ್ನು ಪಡೆಯುವುದಕ್ಕೆ ಹಲವಾರು ಜನರಿರುವರು. ಅವರೆಲ್ಲ ಬರುವರು. ಇದು ನನ್ನ ಗುರುವಿನ ದೃಷ್ಟಿ. ಯಾರನ್ನೂ ಅವರು ದೂರಲಿಲ್ಲ. ಹಲವಾರು ವರುಷಗಳು ಅವರ ಸಮೀಪದಲ್ಲಿದ್ದೆ. ಯಾವ ಪಂಥವನ್ನೂ ಒಮ್ಮೆಯಾದರೂ ಅವರು ದೂರಿದುದನ್ನು ನಾನು ಕೇಳಲಿಲ್ಲ.\break ಎಲ್ಲಾ ಪಂಥಗಳ ಮೇಲೂ ಅವರಿಗೆ ಒಂದೇ ಸಮನಾದ ಸಹಾನುಭೂತಿ ಇತ್ತು. ಎಲ್ಲದರಲ್ಲಿಯೂ ಒಂದು ಸಾಮರಸ್ಯವನ್ನು ಅವರು ಕಂಡಿದ್ದರು. ಒಬ್ಬ ಜ್ಞಾನಿಯಾಗಿರಬಹುದು, ಭಕ್ತನಾಗಿರಬಹುದು, ಯೋಗಿಯಾಗಿರಬಹುದು. ಅದ್ವಿತೀಯ ಕರ್ಮಿಯಾಗಿರಬಹುದು. ಎಲ್ಲಾ ಧರ್ಮಗಳೂ ಇವುಗಳಲ್ಲಿ ಒಂದಲ್ಲ ಒಂದು ಭಾವವನ್ನು ಪ್ರತಿನಿಧಿಸುತ್ತವೆ. ‘ಇವೆಲ್ಲವನ್ನೂ ಒಬ್ಬನೇ ರೂಢಿಸುವುದು ಸಾಧ್ಯ. ಭವಿಷ್ಯದ ಜನಾಂಗ ಮಾಡುವುದು ಇದನ್ನೇ.’ ಇದೇ ಅವರ ಭಾವನೆ. ಅವರು ಯಾರನ್ನೂ ದೂರಲಿಲ್ಲ, ಎಲ್ಲರಲ್ಲಿಯೂ ಒಳ್ಳೆಯದನ್ನೇ ನೋಡಿದರು.

ಈ ಅದ್ಭುತ ವ್ಯಕ್ತಿಯ ಸಂದೇಶವನ್ನು ಕೇಳುವುದಕ್ಕೆ, ಅವರ ದರ್ಶನ ಪಡೆಯುವು\-ದಕ್ಕೆ ಸಹಸ್ರಾರು ಮಂದಿ ಬಂದರು. ಅವರು ಮಾತನಾಡುತ್ತಿದ್ದುದು ಒಂದು ಬಗೆಯ ಗ್ರಾಮ್ಯಭಾಷೆ. ಆಡಿದ ಪ್ರತಿ ನುಡಿಯಲ್ಲಿಯೂ ಕಿಡಿಯಿತ್ತು. ಅದು ನೇರವಾಗಿ ಕೇಳುವವರ\break ಎದೆಯನ್ನು ಸೇರುತ್ತಿತ್ತು. ನಾವು ಮಾತನಾಡುವ ವಿಷಯವಲ್ಲ ಮುಖ್ಯ, ರೀತಿಯಲ್ಲ ಮುಖ್ಯ. ಹೇಳುವ ಮನುಷ್ಯನ ವ್ಯಕ್ತಿತ್ವ ಅವನಾಡುವ ಪ್ರತಿಯೊಂದು ಮಾತಿಗೂ ಶಕ್ತಿಯನ್ನು\break ನೀಡುವುದು. ಕೆಲವು ವೇಳೆ ನಮಗೆಲ್ಲ ಇದು ಭಾಸವಾಗುವುದು. ಅತಿ ಸುಂದರ ಭಾಷಣವನ್ನು ಕೇಳುವೆವು. ಯುಕ್ತಿಪೂರಿತ ಸಂಭಾಷಣೆಯನ್ನು ಕೇಳುವೆವು. ಮನೆಗೆ ಹೋಗಿ\break ಇದನ್ನೆಲ್ಲ ಮರೆಯುವೆವು. ಮತ್ತೆ ಕೆಲವು ವೇಳೆ ಅತಿ ಸರಳವಾದ ಕೆಲವು ಮಾತುಗಳನ್ನು ಕೇಳುವೆವು. ಅವು ನಮ್ಮ ಜೀವನಕ್ಕೆ ತಾಕಿ, ನಮ್ಮಲ್ಲಿ ಐಕ್ಯವಾಗಿ, ಶಾಶ್ವತವಾದ ಪರಿಣಾಮವನ್ನು ಉಂಟುಮಾಡುವುವು. ಯಾರು ತಮ್ಮ ಮಾತಿನಲ್ಲಿ ವ್ಯಕ್ತಿತ್ವವನ್ನು ತುಂಬಬಲ್ಲರೊ ಆ ನುಡಿಯೆ ಪರಿಣಾಮಕಾರಿಯಾಗುವುದು. ಆದರೆ ಅಲ್ಲಿ ಅದ್ವಿತೀಯ ಮಹಿಮಾತಿಶಯವಿರಬೇಕು. ಬೋಧನೆಯೆಲ್ಲ ದಾನದ ಮತ್ತು ಸ್ವೀಕಾರದ ಮೇಲೆ ನಿಂತಿದೆ. ಗುರು ವಿದ್ಯೆಯನ್ನು ದಾನ ಮಾಡುವನು. ಶಿಷ್ಯ ಅದನ್ನು ಸ್ವೀಕರಿಸುವನು. ಆದರೆ ಒಬ್ಬನಿಗೆ ಕೊಡುವುದಕ್ಕೆ\break ಏನಾದರೂ ಇರಬೇಕು; ಮತ್ತೊಬ್ಬ ಸ್ವೀಕರಿಸುವುದಕ್ಕೆ ಯೋಗ್ಯನಾಗಿರಬೇಕು.

ಈ ಮಹಾನುಭಾವರು ನಮ್ಮ ದೇಶದಲ್ಲಿ ಅತಿ ಮುಖ್ಯ ವಿಶ್ವವಿದ್ಯಾನಿಲಯವಿರುವ ಭರತಖಂಡದ ರಾಜಧಾನಿಯಾದ ಕಲ್ಕತ್ತೆಯ ಸಮೀಪದಲ್ಲಿ ವಾಸಿಸುತ್ತಿದ್ದರು. ಈ\break ವಿಶ್ವವಿದ್ಯಾನಿಲಯ ವರ್ಷಂಪ್ರತಿ ನೂರಾರು ಜನ ಸಂಶಯವಾದಿಗಳು, ಜಡವಾದಿಗಳನ್ನು ಹೊರಗೆ ಕಳುಹಿಸುತ್ತಿತ್ತು. ವಿಶ್ವವಿದ್ಯಾನಿಲಯದ ಹಲವು ಸಂಶಯವಾದಿಗಳೂ, ಜಡವಾದಿಗಳೂ, ಇವರ ಮಾತನ್ನು ಕೇಳಲು ಬರುತ್ತಿದ್ದರು. ನಾನೂ ಇವರ ವಿಷಯವನ್ನು ತಿಳಿದು ಇವರ ಮಾತನ್ನು ಕೇಳಲು ಹೋದೆ. ಯಾವ ಒಂದು ವಿಶೇಷವೂ ಇಲ್ಲದೆ ಸಾಧಾರಣ ಮಾನವರಂತೆ ಇವರು ಕಂಡರು, ಅವರ ಭಾಷೆ ಅತಿ ಸರಳವಾಗಿತ್ತು. “ಇವರು ಮಹಾಗುರುಗಳೇ!” ಎಂದು ನಾನು ಯೋಚಿಸಿ, ಅವರ ಸಮೀಪಕ್ಕೆ ಹೋಗಿ ನಾನು ಜೀವನದಲ್ಲಿ ಎಲ್ಲರನ್ನೂ ಕೇಳುತ್ತಿದ್ದ ಪ್ರಶ್ನೆಯನ್ನೇ ಹಾಕಿದೆ: “ಮಹಾಶಯರೆ, ನೀವು ದೇವರನ್ನು ನಂಬುವಿರಾ?” ಎಂದೆ. “ಹೌದು” ಎಂದರು ಅವರು. “ನೀವು ಅದನ್ನು ಸಮರ್ಥಿಸಬಲ್ಲಿರಾ?” ಎಂದೆ. “ಹೌದು” ಎಂದರು. “ಹೇಗೆ?” ಎಂದೆ. “ಏಕೆಂದರೆ, ನಾನು ನಿನ್ನನ್ನು ಇಲ್ಲಿ ಹೇಗೆ ನೋಡುತ್ತಿರುವೆನೊ ಹಾಗೇ ದೇವರನ್ನೂ ನೋಡುವೆ. ಆದರೆ ಇದಕ್ಕಿಂತ ಹೆಚ್ಚು ಸತ್ಯವಾಗಿ” ಎಂದರು. ಇದು ನನ್ನನ್ನು ತಕ್ಷಣವೇ ಆಕರ್ಷಿಸಿತು. “ನಾನು ದೇವರನ್ನು ನೋಡಿರುವೆನು; ನಾವು ಪ್ರಪಂಚವನ್ನು ನೋಡುವುದಕ್ಕಿಂತ ಅತಿಶಯವಾಗಿ ಧರ್ಮದ ಸತ್ಯಗಳನ್ನು ನೋಡಬಹುದು. ಅನುಭವಿಸಬಹುದು” ಎಂದು ಧೈರ್ಯವಾಗಿ ಹೇಳಬಲ್ಲ ವ್ಯಕ್ತಿಯನ್ನು ನಾನು ಜೀವನದಲ್ಲಿ ಪ್ರಥಮಬಾರಿ ಕಂಡೆ. ಪ್ರತಿದಿನವೂ ಅವರ ಬಳಿಗೆ ಹೋಗಲು ಉಪಕ್ರಮಿಸಿದೆ. ಅಧ್ಯಾತ್ಮವನ್ನು ವಾಸ್ತವವಾಗಿ ಕೊಡಬಹುದು ಎಂಬುದನ್ನು ಸತ್ಯವಾಗಿ ಕಂಡೆ. ಅವರ ಒಂದು ಸ್ಪರ್ಶ, ಒಂದು ಕುಡಿನೋಟ, ಇಡೀ ಜೀವನವನ್ನೇ ಬದಲಾಯಿಸಬಲ್ಲದು. ಹಿಂದಿನ ಕಾಲದ ಆಧ್ಯಾತ್ಮಿಕ ಮಹಾನುಭಾವರಾದ ಕ್ರಿಸ್ತ ಬುದ್ಧ ಮಹಮ್ಮದರು ಮಾನವನಿಗೆ,\break “ನೀನು ಪೂರ್ಣನಾಗು” ಎಂದು ಹೇಳಿದೊಡನೆ ಮನುಷ್ಯನು ಪೂರ್ಣನಾಗುವನು\break ಎಂಬುದನ್ನು ಕೇಳಿದ್ದೆ. ಅದನ್ನು ಈಗ ಸತ್ಯವೆಂದು ತಿಳಿದೆ. ಈ ವ್ಯಕ್ತಿಯನ್ನು ನೋಡಿದ\break ಮೇಲೆ ಅನುಮಾನವೆಲ್ಲ ಬಗೆಹರಿಯಿತು. ಇದನ್ನು ಮಾಡಬಹುದು ಎಂದು ಅರಿತೆ. ನನ್ನ ಗುರುದೇವ ಎಲ್ಲಾ ಪದಾರ್ಥಗಳಿಗಿಂತ ಹೆಚ್ಚಾಗಿ, ಆಧ್ಯಾತ್ಮಿಕವನ್ನು ಕೊಡಬಹುದು. ಸ್ವೀಕರಿಸಬಹುದು” ಎನ್ನುತ್ತಿದ್ದರು. ಆದಕಾರಣ ಅಧ್ಯಾತ್ಮ ಶಕ್ತಿಯನ್ನು ಮೊದಲು ಪಡೆಯಿರಿ; ಕೊಡುವುದಕ್ಕೆ ಏನಾದರೂ ಇರಲಿ; ಅನಂತರ ಪ್ರಪಂಚದೆದುರಿಗೆ ನಿಂತು ಅದನ್ನು ಕೊಡಿ. ಧರ್ಮ ಬಾಯಿಮಾತಲ್ಲ; ನಂಬಿಕೆಯಲ್ಲ; ಸಿದ್ಧಾಂತವಲ್ಲ; ಅಥವಾ ಅದೊಂದು\break ಕೋಮುವಾರು ಭಾವನೆಯೂ ಅಲ್ಲ. ಧರ್ಮವು ಕೋಮುಗಳಲ್ಲೂ, ಸಂಘಗಳಲ್ಲೂ\break ಜೀವಿಸಲಾರದು. ಆತ್ಮನಿಗೂ ದೇವರಿಗೂ ಇರುವ ಸಂಬಂಧ ಅದು. ಇದನ್ನು ಒಂದು ಸಂಘವಾಗಿ ಹೇಗೆ ಮಾಡಬಹುದು? ಅನಂತರ ಇದೊಂದು ವ್ಯಾಪಾರವಾಗುವುದು.\break ಎಲ್ಲಿ ವ್ಯಾಪಾರದೃಷ್ಟಿ ಇದೆಯೋ, ವ್ಯಾಪಾರ ನಿಯಮಗಳಿವೆಯೋ ಅಲ್ಲಿ ಆಧ್ಯಾತ್ಮಿಕತೆ\break ಕೊನೆಗಾಣುವುದು. ಧರ್ಮ ದೇವಸ್ಥಾನ ಕಟ್ಟುವುದರಲ್ಲಿ ಇಲ್ಲ. ಅಥವಾ ಸಾಮೂಹಿಕ ಪೂಜೆಗೆ ಹೋಗುವುದರಲ್ಲಿಯೂ ಇಲ್ಲ. ಅದು ಗ್ರಂಥದಲ್ಲಿಲ್ಲ, ಮಾತಿನಲ್ಲಿಯೂ ಇಲ್ಲ, ಉಪನ್ಯಾಸದಲ್ಲಿಯೂ ಇಲ್ಲ, ಅಥವಾ ಸಂಸ್ಥೆಯಲ್ಲಿಯೂ ಇಲ್ಲ. ಧರ್ಮವಿರುವುದು\break ಸಾಕ್ಷಾತ್ಕಾರದಲ್ಲಿ. ನಾವೇ ಪ್ರತ್ಯಕ್ಷ ಸತ್ಯವನ್ನು ಅನುಭವಿಸುವವರೆಗೂ ಯಾವುದೂ ನಮಗೆ ತೃಪ್ತಿಯನ್ನು ಕೊಡಲಾರದು ಎಂಬುದು ನಮಗೆಲ್ಲ ಗೊತ್ತಿದೆ. ನಾವು ಎಷ್ಟು ವಾದಿಸಿದರೂ, ಕೇಳಿದರೂ ಒಂದು ಮಾತ್ರ ನಮಗೆ ತೃಪ್ತಿಯನ್ನು ಕೊಡಬಲ್ಲದು. ಅದೇ ನಮ್ಮ ಸಾಕ್ಷಾತ್ಕಾರ. ಪ್ರಯತ್ನಪಟ್ಟರೆ ನಮ್ಮೆಲ್ಲರಿಗೂ ಅದು ಸಾಧ್ಯ. ಇದನ್ನು ನಾವು ಸಾಧಿಸಬೇಕಾದರೆ ಮೊದಲು ತ್ಯಾಗ ಮಾಡಬೇಕು. ಸಾಧ್ಯವಾದಮಟ್ಟಿಗೂ ನಾವು ತ್ಯಜಿಸಬೇಕು. ಕತ್ತಲೆ ಮತ್ತು ಬೆಳಕು, ಪ್ರಾಪಂಚಿಕ ಭೋಗ ಮತ್ತು ದೇವರ ಸುಖ-ಇವು ಒಟ್ಟಿಗೆ ಇರಲಾರವು. ರಾಮ ಮತ್ತು ಕಾಮ ಇವೆರಡನ್ನೂ ನಾವು ತೃಪ್ತಿಪಡಿಸಲಾರೆವು. ಸಾಧ್ಯವಾದರೆ ಜನರು ಪ್ರಯತ್ನಪಡಲಿ. ಇದನ್ನು ಸಾಧಿಸುವುದಕ್ಕೆ ಪ್ರಯತ್ನಪಟ್ಟ ಲಕ್ಷಾಂತರ ಜನರನ್ನು ನಾನು ಎಲ್ಲಾ\break ದೇಶಗಳಲ್ಲಿಯೂ ನೋಡಿರುವೆನು. ಆದರೆ ಇದು ವ್ಯರ್ಥವಾಗುವುದು. ಮೇಲಿನ\break ಸಂದೇಶದಲ್ಲಿ ಒಂದು ಸತ್ಯಾಂಶವಿದ್ದರೆ ಅದೇ ದೇವರಿಗಾಗಿ ಸರ್ವವನ್ನೂ ತ್ಯಜಿಸುವುದು. ಇದು ಬಹಳ ಕಷ್ಟ. ನಿಧಾನವಾದ ಕೆಲಸ. ಆದರೆ ನೀವು ಇಲ್ಲೇ ಈಗಲೇ ಪ್ರಾರಂಭಿಸಬಹುದು. ಕ್ರಮೇಣ ನಾವು ಗುರಿಯೆಡೆಗೆ ಸಾಗುವೆವು.

ನಾನು ಗುರುದೇವನಿಂದ ಕಲಿತ ಬಹುಶಃ ಅತ್ಯಂತ ಮುಖ್ಯವಾದ ಎರಡನೆಯ\break ಭಾವನೆಯೆ ಇದು: ಜಗತ್ತಿನ ಧರ್ಮಗಳು ಪರಸ್ಪರ ವಿರೋಧವಲ್ಲ. ಅವೆಲ್ಲ ಒಂದು ಸನಾತನ ಧರ್ಮದ ಹಲವಾರು ಪ್ರತಿಬಿಂಬಗಳು. ಆ ಒಂದು ಸತ್ಯವನ್ನೇ ಕಾರ್ಯಕ್ಷೇತ್ರದ ಹಲವು ಭೂಮಿಕೆಗಳಲ್ಲಿ ಬೇರೆ ಬೇರೆ ಜನಾಂಗದವರಿಗೆ ಅವರವರ ಮನಸ್ಸಿಗೆ ರುಚಿಸುವಂತೆ\break ಉಪಯೋಗಿಸಿದೆ. ಎಂದಿಗೂ ನನ್ನ ಧರ್ಮ ಅಥವಾ ನಿಮ್ಮ ಧರ್ಮ ಎಂಬುದು ಇರಲಿಲ್ಲ.\break ನನ್ನ ರಾಷ್ಟ್ರೀಯ ಧರ್ಮ ಅಥವಾ ನಿಮ್ಮ ರಾಷ್ಟ್ರೀಯ ಧರ್ಮ ಎಂಬುದೂ ಇರಲಿಲ್ಲ.\break ‘ಹಲವು’ ಧರ್ಮಗಳು ಎಂದಿಗೂ ಇರಲಿಲ್ಲ. ಯಾವಾಗಲೂ ಇದ್ದುದು ಒಂದೇ ಧರ್ಮ. ಒಂದು ಸನಾತನಧರ್ಮ ಅನಾದಿಕಾಲದಿಂದಲೂ ಬಂದಿದೆ. ಅದೇ ಎಂದೆಂದಿಗೂ ಇರುವುದು. ಈ ಧರ್ಮವೇ ಹಲವು ದೇಶಗಳಲ್ಲಿ ಹಲವು ರೀತಿಗಳಲ್ಲಿ ವ್ಯಕ್ತವಾಗುತ್ತ\break ಹೋಗುವುದು. ಆದಕಾರಣ ನಾವು ಎಲ್ಲಾ ಧರ್ಮಗಳಿಗೂ ಗೌರವವನ್ನು ತೋರಬೇಕು. ಸಾಧ್ಯವಾದ ಮಟ್ಟಿಗೆ ನಾವು ಎಲ್ಲವನ್ನೂ ಸ್ವೀಕರಿಸಬೇಕು. ಧರ್ಮಗಳು ಜನಾಂಗ ಅಥವಾ ಭೌಗೋಳಿಕ ಸ್ಥಿತಿಗತಿಗಳಿಗೆ ಅನುಗುಣ ಮಾತ್ರವಲ್ಲದೆ, ವ್ಯಕ್ತಿಯ ಶಕ್ತಿಗೆ ಅನುಗುಣವಾಗಿಯೂ ಅಭಿವ್ಯಕ್ತಿಗೊಳ್ಳುತ್ತವೆ. ಒಬ್ಬನಲ್ಲಿ ಧರ್ಮ ಪ್ರಚಂಡ ಕ್ರಿಯಾರೂಪದಲ್ಲಿ ವ್ಯಕ್ತವಾಗುತ್ತದೆ. ಮತ್ತೊಬ್ಬನಲ್ಲಿ ಭಕ್ತಿಯಂತೆ, ಮಗದೊಬ್ಬನಲ್ಲಿ ಯೋಗದಂತೆ ಇನ್ನೂ ಒಬ್ಬನಲ್ಲಿ ಜ್ಞಾನದಂತೆ ವ್ಯಕ್ತವಾಗುತ್ತದೆ. ನಾವು ಮತ್ತೊಬ್ಬರಿಗೆ ನಿಮ್ಮ ಮಾರ್ಗ ಸರಿಯಲ್ಲವೆಂದು ಹೇಳುವುದು ತಪ್ಪು. ಬಹುಶಃ ಭಾವಜೀವಿಯಾದವನೊಬ್ಬನು, ಇತರರಿಗೆ ಒಳ್ಳೆಯದನ್ನು ಮಾಡುವವನದು ಧರ್ಮಕ್ಕೆ ಸೂಕ್ತವಾದ ಮಾರ್ಗವಲ್ಲ ಎಂದು ಭಾವಿಸಬಹುದು. ಏಕೆಂದರೆ ಅದು ತನ್ನ ರೀತಿಯಲ್ಲ, ಆದಕಾರಣ ತಪ್ಪು. “ಈಶ್ವರ ಪ್ರೇಮಸ್ವರೂಪ, ಅವನನ್ನು ಪ್ರೀತಿಸುವುದು ಎಂದರೆ ಈ ಮೂಢರಿಗೆ ಏನು ಗೊತ್ತಿದೆ” ಎಂದು ತತ್ತ್ವಜ್ಞಾನಿಯು ಭಾವಿಸಿದರೆ ಅದು ತಪ್ಪು.\break ಏಕೆಂದರೆ ಅವರ ಮಾರ್ಗವೂ ಸರಿಯಿರಬಹುದು, ಹಾಗೆಯೇ ಇವನದೂ ಕೂಡ.

ಸತ್ಯ ಏಕವಾಗಿದ್ದರೂ ಅನಂತ ರೂಪಗಳನ್ನು ತಾಳಬಲ್ಲದು, ಬೇರೆ ಬೇರೆ ನಿಲುವುಗಳಿಂದ ಒಂದೇ ಸತ್ಯದ ಬೇರೆ ಬೇರೆ ದೃಶ್ಯಗಳನ್ನು ಪಡೆಯಬಹುದು ಎಂಬ ಮುಖ್ಯ ರಹಸ್ಯವನ್ನು ನಾವು ತಿಳಿಯಬೇಕಾಗಿದೆ. ಆಗ ಮತ್ತೊಬ್ಬರನ್ನು ದ್ವೇಷಿಸುವ ಬದಲು ಎಲ್ಲರಿಗೂ ಅನಂತ ಸಹಾನುಭೂತಿಯನ್ನು ತೋರಬಹುದು. ಎಲ್ಲಿಯವರೆಗೂ ಹಲವು ಸ್ವಭಾವಗಳು ಇರುವುವೋ ಅಲ್ಲಿಯವರೆಗೂ ಒಂದೇ ಸತ್ಯ ಹಲವು ರೀತಿಗಳನ್ನು ತಾಳಬೇಕಾಗುವುದೆಂಬುದನ್ನು ತಿಳಿದಾದ ಮೇಲೆ, ಪರಸ್ಪರ ಸಹಾನುಭೂತಿ ಅತ್ಯಾವಶ್ಯಕವೆಂದು ಗೊತ್ತಾಗುವುದು. ಹೇಗೆ ಪ್ರಕೃತಿಯ ವೈವಿಧ್ಯದಲ್ಲಿ ಐಕ್ಯತೆ ಇದೆಯೋ, ಕಣ್ಣೆದುರಿಗೆ ಬಹಳ ವೈವಿಧ್ಯ ಇದ್ದರೂ ಅದರೊಳಗೆ ಅನಂತವಾದ ಅವಿಕಾರಿಯಾದ ಏಕತ್ವವಿದೆಯೋ ಇದರಂತೆಯೆ ಪ್ರತಿಯೊಬ್ಬ ಮಾನವನಲ್ಲಿಯೂ ಕೂಡ. ಪಿಂಡಾಂಡವು ಬ್ರಹ್ಮಾಂಡದ ಬರಿಯ ಒಂದು ಸಂಕ್ಷಿಪ್ತ ಪುನರಾವೃತ್ತಿ. ಈ ವೈವಿಧ್ಯಗಳಿದ್ದರೂ ಇವುಗಳ ಅಂತರಾಳದಲ್ಲೆಲ್ಲಾ ಸನಾತನ ಸ್ವರಮೇಳವಿರುವುದು. ನಾವು ಇದನ್ನು ಒಪ್ಪಬೇಕಾಗಿದೆ. ಉಳಿದ ಭಾವನೆಗಳಿಗಿಂತ ಈ ಒಂದು ಭಾವನೆ ನಮಗೆ ಇಂದು ಅತ್ಯಾವಶ್ಯಕವಾಗಿ ಬೇಕಾಗಿರುವುದು. ನನ್ನ ದೇಶದಲ್ಲಿ ಹಲವು ಮತ ಪಂಥಗಳು ತುಂಬಿ ತುಳುಕಾಡುತ್ತಿವೆ. ಅದೃಷ್ಟವಶಾತ್​ ಅಥವಾ ದುರದೃಷ್ಟವಶಾತ್​ ಧರ್ಮಭಾವನೆ ಇರುವ ಪ್ರತಿಯೊಬ್ಬನೂ ತನ್ನ ಧರ್ಮವನ್ನು ಪ್ರಚಾರ ಮಾಡಲು ಪ್ರಚಾರಕರ ಒಂದು ಗುಂಪನ್ನು ಅಲ್ಲಿಗೆ ಕಳುಹಿಸಬೇಕೆಂದು ಭಾವಿಸುವನು. ಬಾಲ್ಯಾರಭ್ಯದಿಂದಲೂ ಪ್ರಪಂಚದ ಹಲವು ಪಂಗಡಗಳ ಪರಿಚಯ ನನಗಿದೆ. ಮರ್​ಮನ್ನರೂ ತಮ್ಮ ಸಂದೇಶವನ್ನು ಹರಡುವುದಕ್ಕೆ ಭರತಖಂಡಕ್ಕೆ ಬಂದರು. ಎಲ್ಲರೂ ಬರಲಿ! ಧರ್ಮವನ್ನು ಬೋಧಿಸುವುದಕ್ಕೆ ಅದೇ ಸ್ಥಳ. ಉಳಿದೆಲ್ಲ ದೇಶಗಳಿಗಿಂತಲೂ ಅಲ್ಲಿ ಚೆನ್ನಾಗಿ ಅದು ಬೇರನ್ನು ಬಿಡುವುದು. ನೀವು ಹಿಂದೂಗಳಿಗೆ ರಾಜಕೀಯವನ್ನು ಬೋಧಿಸಿದರೆ ಅವರಿಗೆ ಅರ್ಥವಾಗುವುದಿಲ್ಲ. ಆದರೆ ನೀವು ಧರ್ಮವನ್ನು ಬೋಧಿಸುವುದಕ್ಕೆ ಬಂದರೆ, ಅದು ಎಷ್ಟು ವಿಚಿತ್ರವಾಗಿದ್ದರೂ ಬಹುಬೇಗ ನೂರಾರು ಮಂದಿ ಅನುಯಾಯಿಗಳು ದೊರಕುವರು. ನಿಮ್ಮ ಜೀವಮಾನ ಕಾಲದಲ್ಲೆ ನೀವೊಬ್ಬ ಜೀವಂತ ದೇವರಾಗುವ ಸಂಭವವಿದೆ. ಅದು ಹೀಗಿರುವುದು ನನಗೆ ಸಂತೋಷ. ನಮಗೆ ಭರತಖಂಡದಲ್ಲಿ ಬೇಕಾಗಿರುವುದು ಇದೊಂದೇ.

ಹಿಂದೂಗಳಲ್ಲಿ ಎಷ್ಟೋ ಪಂಥಗಳಿವೆ. ಅವುಗಳಲ್ಲಿ ಕೆಲವು ತೋರಿಕೆಗೆ ಪರಸ್ಪರ ವಿರೋಧವಾಗಿ ಕಾಣುವುವು. ಆದರೆ ಅವೆಲ್ಲ ಒಂದೇ ಧರ್ಮದ ಹಲವು ಅಭಿವ್ಯಕ್ತಿಗಳೆಂದು ಅವರು ಹೇಳುವರು. “ಹಲವು ನದಿಗಳು ಬೇರೆ ಬೇರೆ ಪರ್ವತಗಳಲ್ಲಿ ಜನಿಸಿ, ನೇರವಾಗಿಯೋ ವಕ್ರವಾಗಿಯೋ ಹರಿದು, ಕೊನೆಗೆ ಎಲ್ಲಾ ಸಾಗರದಲ್ಲಿ ಸಂಗಮವಾಗುವಂತೆ, ಬೇರೆ ಬೇರೆ ದೃಷ್ಟಿಯುಳ್ಳ ಹಲವು ಪಂಗಡಗಳು ಕೊನೆಗೆ ನಿನ್ನೆಡೆಗೆ ಬರುವುವು.” ಇದೊಂದು ಸಿದ್ಧಾಂತವಾದರೆ ಸಾಲದು. ನಾವು ಇದನ್ನು ಸ್ವೀಕರಿಸಬೇಕು. ಆದರೆ ಕೆಲವರು ಅನುಗ್ರಹ ತೋರುವಂತೆ ವರ್ತಿಸುತ್ತ “ನಿಜ, ಆ ಧರ್ಮದಲ್ಲಿ ಹಲವು ಒಳ್ಳೆಯ ವಿಷಯಗಳಿವೆ. ಇದನ್ನೇ ನಾವು ಜನಾಂಗೀಯ ಧರ್ಮಗಳು ಎಂದು ಕರೆಯುವೆವು. ಇವುಗಳಲ್ಲಿಯೂ ಕೆಲವು ಒಳ್ಳೆಯ ವಿಷಯಗಳಿವೆ” ಎಂದು ಹೇಳುವುದು ಸರಿಯಲ್ಲ. ಮತ್ತೆ ಕೆಲವರಲ್ಲಿ ಅತಿ ವಿಚಿತ್ರ ಔದಾರ್ಯವಿದೆ; ಉಳಿದ ಧರ್ಮಗಳೆಲ್ಲ ಇತಿಹಾಸ ಪೂರ್ವದ ಅವಶೇಷಗಳು, ಆದರೆ ನಮ್ಮದು ಧರ್ಮದ ಪರಾಕಾಷ್ಠೆ ಎಂದು ಭಾವಿಸುವರು ಅವರು. ಆದರೆ ಒಬ್ಬನು ತನ್ನದು ಅತಿ ಪುರಾತನ ಧರ್ಮ, ಆದಕಾರಣ ಅದು ಅತ್ಯುತ್ತಮ ಎನ್ನುವನು. ಮತ್ತೊಬ್ಬನು ತನ್ನದು ಅತ್ಯಾಧುನಿಕ, ಆದಕಾರಣ ಇದಕ್ಕೆ ಅತ್ಯುತ್ತಮ ಎಂದು ಅನಿಸಿಕೊಳ್ಳಲಿಕ್ಕೆ ಅಧಿಕಾರವಿದೆಯೆನ್ನುವನು. ಪ್ರತಿಯೊಂದು ಧರ್ಮದಲ್ಲಿಯೂ ಮತ್ತೊಂದರಲ್ಲಿರುವಷ್ಟೇ ಮಾನವನನ್ನು ಉದ್ಧಾರ ಮಾಡುವ ಶಕ್ತಿ ಇದೆ ಎನ್ನುವುದನ್ನು ನಾವು ತಿಳಿದುಕೊಳ್ಳಬೇಕು. ಇದಕ್ಕೆ ವಿರೋಧವಾಗಿ ದೇವಸ್ಥಾನದಲ್ಲಿ ಅಥವಾ ಚರ್ಚಿನಲ್ಲಿ ಕೇಳುವ ಅಭಿಪ್ರಾಯಗಳೆಲ್ಲ ಮೂಢನಂಬಿಕೆಯ ರಾಶಿ. ದೇವರೊಬ್ಬನೇ ಎಲ್ಲರಿಗೂ ಜವಾಬ್ದಾರನು. ಮಾನವನ ಆತ್ಮದ ಸ್ವಲ್ಪಮಟ್ಟಿನ ಉದ್ಧಾರಕ್ಕಾದರೂ ನೀವು ನಾನು ಅಥವಾ ಮತ್ತಾರೂ ಹೊಣೆಗಾರರಲ್ಲ, ಸರ್ವಶಕ್ತನಾದ ಈಶ್ವರನೊಬ್ಬನೆ ಇದಕ್ಕೆ ಹೊಣೆಗಾರ, ದೇವರನ್ನು ನಂಬುವೆವು ಎನ್ನುವರು. ಆದರೆ ಜೊತೆಗೆ, ‘ಎಲ್ಲೊ ಕೆಲವರಿಗೆ ಅವನು ಸತ್ಯವನ್ನೆಲ್ಲಾ ಕೊಟ್ಟುಬಿಟ್ಟಿರುವನು, ಅವರು ಉಳಿದ ಮಾನವ ಕೋಟಿಯ ರಕ್ಷಕರು’ ಎನ್ನುವುದು ನನಗೆ ಅರ್ಥವಾಗುವುದಿಲ್ಲ. ಇದನ್ನು ನೀವು ಧರ್ಮ ಎಂದು ಹೇಗೆ ಹೇಳುತ್ತೀರಿ? ಸಾಕ್ಷಾತ್ಕಾರವೆ ಧರ್ಮ. ಬರಿಯ ಮಾತು, ಯಾವುದನ್ನೊ ನಂಬುವುದು, ಅಂಧಕಾರದಲ್ಲಿ ಅರಸುವುದು, ಪೂರ್ವಿಕರಿಂದ ಬಂದ ಕೆಲವು ಮಂತ್ರಗಳನ್ನು ಅರಗಿಳಿಯಂತೆ ಉಚ್ಚರಿಸುವುದು, ಧಾರ್ಮಿಕ ಸತ್ಯಕ್ಕೆ ರಾಜಕೀಯ ಮೆರಗು ಕೊಡುವುದು, ಇವು ಎಂದಿಗೂ ಧರ್ಮವಲ್ಲ, ಎಲ್ಲಾ ಕೋಮುಗಳಲ್ಲಿಯೂ, ಅತಿ ಪ್ರತ್ಯೇಕರೆಂದು ನಾವು ಭಾವಿಸುವ ಮಹಮ್ಮದೀಯರಲ್ಲಿಯೂ, ಧರ್ಮದ ಸಾಕ್ಷಾತ್ಕಾರಕ್ಕೆ ಪ್ರಯತ್ನಿಸುತ್ತಿರುವಾಗ ಅವರ ಬಾಯಿಂದ ಈ ಉಜ್ಜ್ವಲ ಪದಗಳು ಹೊರಟಿವೆ: “ನೀನೇ ಎಲ್ಲರಿಗೂ ಈಶ್ವರ; ನೀನೇ ಲೋಕ ಗುರು, ನಮಗಿಂತ ಹೆಚ್ಚು ನೀನು ನಿನ್ನ ಮಕ್ಕಳ ದೇಶವನ್ನು ಪ್ರೀತಿಸುವೆ.” ಯಾರ ಶ್ರದ್ಧೆಯನ್ನೂ ವಿಚಲಿತಗೊಳಿಸಬೇಡಿ. ಸಾಧ್ಯವಾದರೆ ಇರುವುದಕ್ಕಿಂತ ಉತ್ತಮವಾದುದನ್ನು ಅವರಿಗೆ ಕೊಡಿ. ಸಾಧ್ಯವಾದರೆ ಅವನಿರುವ ಸ್ಥಳವನ್ನು ತಿಳಿದು ಮುಂದಕ್ಕೆ ಹೋಗುವುದಕ್ಕೆ ಸಹಾಯಮಾಡಿ. ಆದರೆ ಅವನಲ್ಲಿರುವುದನ್ನು ನಾಶಮಾಡಬೇಡಿ. ಕ್ಷಣದಲ್ಲಿ ಸಹಸ್ರಾರು\break ಮಂದಿಯಂತೆ ಯಾವನು ಆಗಬಲ್ಲನೊ ಆತನೇ ನಿಜವಾದ ಗುರು ತಕ್ಷಣವೇ ಶಿಷ್ಯನ\break ಮಟ್ಟಕ್ಕೆ ಇಳಿದುಬಂದು, ತನ್ನ ಶಕ್ತಿಯನ್ನು ಅವನಿಗೆ ಕೊಟ್ಟು, ಅವನ ಕಣ್ಣಿನ ಮೂಲಕ\break ನೋಡಬಲ್ಲ, ಕಿವಿಯ ಮೂಲಕ ಕೇಳಬಲ್ಲ, ಅವನ ಮನಸ್ಸಿನ ಮೂಲಕ ಅರಿಯಬಲ್ಲವನೇ ನಿಜವಾದ ಗುರು. ಅಂತಹ ಗುರುವೇ ನಿಜವಾಗಿ ಉಪದೇಶ ಮಾಡಬಲ್ಲ. ಇತರರಲ್ಲ, ಜಗತ್ತಿನಲ್ಲಿ ನಿಷೇಧಾತ್ಮಕವಾದ ಮಾರ್ಗವನ್ನು ಅನುಸರಿಸುವ ಯಾರೂ ಜಗತ್ತಿಗೆ ಹಿತವನ್ನು ಮಾಡಲಾರರು.

ಮಾನವನು ಈ ದೇಹದಲ್ಲಿರುವಾಗಲೇ ಪೂರ್ಣನಾಗಬಲ್ಲ ಎಂಬುದನ್ನು ನಾನು ನನ್ನ ಗುರು ಸಾನ್ನಿಧ್ಯದಲ್ಲಿ ಕಲಿತೆನು. ಅವರ ಬಾಯಿ ಯಾರನ್ನೂ ಶಪಿಸಲಿಲ್ಲ. ಟೀಕಿಸಲೂ\break ಇಲ್ಲ. ಅವರ ಕಣ್ಣು ಪಾಪವನ್ನು ಲವಲೇಶವಾದರೂ ನೋಡುವ ಸ್ಥಿತಿಯಲ್ಲಿರಲಿಲ್ಲ.\break ಹೀನ ಆಲೋಚನೆ ಮಾಡುವುದಕ್ಕೆ ಅವರಿಂದ ಸಾಧ್ಯವೇ ಇರಲಿಲ್ಲ. ಅವರ ಮನಸ್ಸು ಶುಭವನ್ನಲ್ಲದೆ ಬೇರೆ ಏನನ್ನೂ ನೋಡುತ್ತಿರಲಿಲ್ಲ. ಆ ಮಹಾ ಪವಿತ್ರತೆ ಮತ್ತು ಅಸಾಧಾರಣ ತ್ಯಾಗ ಇವೇ ಅಧ್ಯಾತ್ಮ ಜೀವನದ ರಹಸ್ಯ. “ಐಶ್ವರ್ಯದ ಮೂಲಕವಾಗಿ ಅಲ್ಲ, ಸಂತತಿಯ ಮೂಲಕವಾಗಿ ಅಲ್ಲ, ತ್ಯಾಗದ ಮೂಲಕ ಮಾತ್ರ ಮೋಕ್ಷವನ್ನು ಪಡೆಯಬಹುದು”\break ಎಂದು ವೇದಗಳು ಸಾರುತ್ತವೆ. “ನಿನ್ನಲ್ಲಿರುವುದನ್ನೆಲ್ಲ ಮಾರಿ ದೀನರಿಗೆ ಕೊಟ್ಟು\break ನನ್ನನ್ನು ಅನುಸರಿಸು” ಎನ್ನುವನು ಕ್ರಿಸ್ತ. ಎಲ್ಲಾ ಮಹಾತ್ಮರೂ ದೇವದೂತರೂ ಇದನ್ನೇ ಹೇಳಿರುವರು, ಇದನ್ನೇ ಅನುಷ್ಠಾನ ಮಾಡಿರುವರು. ಅಂತಹ ಪ್ರಚಂಡ ತ್ಯಾಗವಿಲ್ಲದೆ ಮಹಾ ಆಧ್ಯಾತ್ಮಿಕತೆ ಬರುವುದು ಹೇಗೆ? ಆಧ್ಯಾತ್ಮಿಕ ಭಾವನೆಗಳು ಎಲ್ಲಿದ್ದರೂ ತ್ಯಾಗವೇ\break ಅವುಗಳ ಹಿನ್ನೆಲೆಯಾಗಿರುವುದು. ಈ ತ್ಯಾಗಭಾವನೆ ಎಷ್ಟು ಕುಗ್ಗುತ್ತ ಹೋದರೆ\break ಧಾರ್ಮಿಕ ಭಾವನೆ ಅಷ್ಟು ಕಡಿಮೆಯಾಗಿ ಪ್ರಾಪಂಚಿಕ ವಿಷಯಗಳು ಬಂದು ನೆಲಸುವುದು ಕಾಣುತ್ತದೆ.

ತ್ಯಾಗಮೂರ್ತಿಯಾಗಿದ್ದರು ಆ ಮಹಾನುಭಾವರು. ಸಂನ್ಯಾಸಿಯಾದರೆ ಪ್ರಾಪಂಚಿಕ ದ್ರವ್ಯವನ್ನು ಮತ್ತು ಪದವಿಗಳನ್ನು ತ್ಯಜಿಸುವುದು ನಮ್ಮ ದೇಶದಲ್ಲಿ ರೂಢಿ. ನನ್ನ ಗುರುದೇವ ಅಕ್ಷರಶಃ ಇದನ್ನು ಪಾಲಿಸಿದರು. ತಮ್ಮಿಂದ ಯಾವುದಾದರೊಂದು ಕಾಣಿಕೆಯನ್ನು ಅವರು ಸ್ವೀಕರಿಸಿದರೆ ತಾವು ಧನ್ಯರೆಂದು ಎಷ್ಟೋ ಜನ ಭಾವಿಸಿದ್ದರು. ಅವರು ಸ್ವೀಕರಿಸುವುದಾದರೆ ಸಹಸ್ರಾರು ರೂಪಾಯಿಗಳನ್ನು ಸಂತೋಷದಿಂದ ಕೊಡಲು ಸಿದ್ಧರಾಗಿದ್ದರು. ಆದರೆ ಮೊದಲು ಇಂತಹವರಿಂದ ದೂರವಿರಲು ಅವರು ಯತ್ನಿಸಿದರು. ಕಾಮಕಾಂಚನಗಳ\break ತ್ಯಾಗದಲ್ಲಿ ಸಂಪೂರ್ಣ ವಿಜಯಿಯಾದ ಬದುಕಿನ ಉದಾಹರಣೆ ಅವರ ಜೀವನ. ಈ ಭಾವನೆಗಳನ್ನೆಲ್ಲ ಅವರು ಮೀರಿ ಹೋಗಿದ್ದರು. ಈ ಶತಮಾನಕ್ಕೆ ಅಂತಹ ವ್ಯಕ್ತಿಗಳು\break ಅತ್ಯಾವಶ್ಯಕ. ಜೀವನದ ಅತ್ಯಾವಶ್ಯಕ ವಸ್ತುಗಳು ಇಲ್ಲದೆ (ಈಗ ಅವು ಪ್ರತಿದಿನವೂ ಎಲ್ಲ ಪ್ರಮಾಣವನ್ನೂ ಮೀರಿ ಹೆಚ್ಚುತ್ತಿವೆ) ತಾವು ಒಂದು ತಿಂಗಳೂ ಇರಲಾರವೆಂದು\break ಭಾವಿಸುವಾಗ, ಈ ಮಹಾತ್ಯಾಗ ಅತಿಮುಖ್ಯ. ಇಂತಹ ಕಾಲದಲ್ಲಿ ಸಂದೇಹವಾದಿಗಳಾದ ಜನರೆದುರಿಗೆ, ಪ್ರಪಂಚದ ಐಶ್ವರ್ಯ ಕೀರ್ತಿಗಳಾವುದನ್ನೂ ಗಮನಕ್ಕೆ ತಾರದೆ ಇರುವುದು ಸಾಧ್ಯವೆಂದು ಎದ್ದು ತೋರುವುದು ಅವಶ್ಯಕ. ಇಂತಹ ವ್ಯಕ್ತಿಗಳು ಜಗತ್ತಿನಲ್ಲಿ ಇನ್ನೂ\break ಇರುವರು.

ನನ್ನ ಗುರುದೇವರ ಜೀವನದ ಮತ್ತೊಂದು ಮುಖ, ಅವರಿಗೆ ಮಾನವ ಕೋಟಿಯಲ್ಲಿ ಇದ್ದ ಅಪಾರ ಪ್ರೇಮ. ನನ್ನ ಗುರುದೇವರ ಜೀವನದ ಅರ್ಧಭಾಗ ಅಧ್ಯಾತ್ಮ ಶಕ್ತಿಯನ್ನು ಶೇಖರಿಸುವುದರಲ್ಲಿ ಕಳೆಯಿತು. ಉಳಿದ ಆಯುಷ್ಯವನ್ನು ಅದನ್ನು ಹಂಚುವುದರಲ್ಲಿ\break ಕಳೆದರು. ನಮ್ಮ ದೇಶದಲ್ಲಿ ಆಧ್ಯಾತ್ಮಿಕ ಗುರು ಅಥವಾ ಸಂನ್ಯಾಸಿಯ ದರ್ಶನ ಪಡೆಯಬೇಕಾದರೆ ನಿಮ್ಮ ದೇಶದಲ್ಲಿರುವ ರೂಢಿ ಇಲ್ಲ. ಯಾವುದೋ ಒಂದು ವಿಷಯವನ್ನು ಕೇಳುವುದಕ್ಕೆ ಒಬ್ಬರು ಬರುವರು. ಒಂದು ಪ್ರಶ್ನೆಯನ್ನು ಅವರಿಗೆ ಹಾಕಲು, ಅವರಿಂದ ಒಂದು ನುಡಿಯನ್ನು ಕೇಳಲು, ನೂರಾರು ಮೈಲಿಗಳಿಂದ ನಡೆದು ಬರುವರು. “ನನ್ನ ಮುಕ್ತಿಗೆ ಒಂದು ಮಾತನ್ನು ಹೇಳಿ” ಎನ್ನುವರು. ಅವರು ಬರುತ್ತಿದ್ದ ಬಗೆ ಹೀಗೆ. ಜನರು ಸಾಮಾನ್ಯವಾಗಿ ಇವರ ಸ್ಥಳಕ್ಕೆ ಗುಂಪು ಗುಂಪಾಗಿ ನಿಸ್ಸಂಕೋಚವಾಗಿ ಬರುತ್ತಿದ್ದರು. ಅವರನ್ನು ಒಂದು ಮರದ ಕೆಳಗೆ ಕಾಣಬಹುದು; ಅಲ್ಲೆ ಪ್ರಶ್ನೆ ಹಾಕುವರು. ಒಂದು ಗುಂಪಿನ ಜನರು\break ಹೋಗುವುದೇ ತಡ, ಇನ್ನೊಂದು ಗುಂಪು ಬರುವುದು, ಒಬ್ಬ ಅತಿ ಹೆಚ್ಚಿನ ಗೌರವಕ್ಕೆ ಪಾತ್ರನಾದರೆ ಅಂಥವನಿಗೆ ಕೆಲವು ವೇಳೆ ಹಗಲು ರಾತ್ರಿ ವಿಶ್ರಾಂತಿಯೇ ಸಿಕ್ಕುವುದಿಲ್ಲ. ಒಂದೇ ಸಮನಾಗಿ ಮಾತನಾಡಬೇಕು. ಹಲವು ಗಂಟೆಗಳವರೆಗೆ ಜನಸಂದಣಿ ಬರುವುದು, ಇವನು ಅವರಿಗೆ ಉಪದೇಶ ಮಾಡುವನು.

ಇವರ ವಾಣಿಯನ್ನು ಕೇಳುವುದಕ್ಕೆ ಗುಂಪುಗುಂಪಾಗಿ ಜನರು ಬರಲು ಮೊದಲಾಯಿತು. ದಿನದ ಇಪ್ಪತ್ತುನಾಲ್ಕು ಗಂಟೆಗಳಲ್ಲಿ ಇಪ್ಪತ್ತುಗಂಟೆ ಅವರು ಮಾತನಾಡುತ್ತಿದ್ದರು. ಅದೂ ಒಂದು ದಿನವಲ್ಲ, ಹಲವಾರು ತಿಂಗಳುಗಳವರೆಗೆ. ಪ್ರಚಂಡ ದುಡಿತಕ್ಕೆ ಈಡಾಗಿ ಕೊನೆಗೆ ಅವರ ದೇಹ ಜರ್ಝರಿತವಾಯಿತು. ಅವರಿಗೆ ಮಾನವಕೋಟಿಯ ಮೇಲೆ\break ಅಭೂತಪೂರ್ವ ಅನುಕಂಪವಿತ್ತು. ಇವರ ಸಹಾಯವನ್ನು ಅಪೇಕ್ಷಿಸಿದ ಸಹಸ್ರಾರು ಜನರಲ್ಲಿ ಒಬ್ಬ ದೀನನನ್ನೂ ನಿರಾಕರಿಸಲಿಲ್ಲ. ಕ್ರಮೇಣ ಗಂಟಲಿನ ರೋಗ ಪ್ರಾಪ್ತವಾಯಿತು.\break ಆದರೂ ‘ತೊಂದರೆಗೆ ಒಳಗಾಗಬೇಡಿ’ ಎಂದು ಹೇಳಿ ಅವರನ್ನು ಸುಮ್ಮನಿರಿಸಲು\break ಆಗಲಿಲ್ಲ. ಜನರು ತಮ್ಮ ದರ್ಶನಲಾಭವನ್ನು ಇಚ್ಛಿಸುತ್ತಿರುವರು ಎಂದು ಕೇಳಿದೊಡನೆಯೆ ಅವರನ್ನು ಒಳಗೆ ಬಿಡಬೇಕೆಂದು ಬಲಾತ್ಕಾರಮಾಡಿ ಅವರ ಸಂದೇಹವನ್ನೆಲ್ಲಾ ನಿವಾರಿಸುತ್ತಿದ್ದರು. ಅವರನ್ನು ಎಚ್ಚರಿಸಿದಾಗ ಹೀಗೆ ಹೇಳುವರು: “ನಾನು ಇದನ್ನು ಲೆಕ್ಕಿಸುವುದಿಲ್ಲ. ಇಂಥ ಇಪ್ಪತ್ತು ಸಹಸ್ರ ದೇಹಗಳನ್ನು ಬೇಕಾದರೂ, ಒಬ್ಬನಿಗೆ ಸಹಾಯ ಮಾಡಲು,\break ನಾನು ಬಿಡಲು ಸಿದ್ಧನಾಗಿದ್ದೇನೆ. ಒಬ್ಬ ಮಾನವನಿಗೆ ಸಹಾಯ ಮಾಡುವುದೇ ಮಹಾಪುಣ್ಯ.” ಅವರಿಗೆ ವಿಶ್ರಾಂತಿಯೇ ಇರಲಿಲ್ಲ. ಒಂದು ದಿನ ಒಬ್ಬರು, “ಸ್ವಾಮಿ, ನೀವು ಮಹಾಯೋಗಿಗಳು, ನಿಮ್ಮ ದೇಹದ ಮೇಲೆ ಮನಸ್ಸನ್ನು ಸ್ವಲ್ಪ ತಿರುಗಿಸಿ ರೋಗವನ್ನು ಏತಕ್ಕೆ ಗುಣಮಾಡಿಕೊಳ್ಳಬಾರದು” ಎಂದು ಕೇಳಿದರು. ಮೊದಲು ಅವರು ಉತ್ತರಕೊಡಲಿಲ್ಲ. ಪುನಃ ಪ್ರಶ್ನೆಯನ್ನು ಹಾಕಿದಮೇಲೆ, “ಸಖನೆ, ನೀನೊಬ್ಬ ಜ್ಞಾನಿಯೆಂದು ಭಾವಿಸಿದ್ದೆ. ಆದರೆ ಇತರ ಪ್ರಾಪಂಚಿಕರಂತೆಯೇ ಮಾತನಾಡುವೆ. ಭಗವಂತನಿಗೆ ಈ ಮನಸ್ಸನ್ನು ಸಮರ್ಪಿಸಿದ್ದೇನೆ. ನಾನು ಇದನ್ನು ಪುನಃ ಹಿಂತಿರುಗಿಸಿ ಕೇವಲ ಆತ್ಮದ ಒಂದು ಗೂಡಿನಂತೆ ಇರುವ ದೇಹದ ಮೇಲೆ ಇಡು ಎನ್ನುವೆಯಾ?” ಎಂದರು.

ಅವರು ಜನರಿಗೆ ಉಪದೇಶ ನೀಡುತ್ತಾ ಹೋದರು. ಅವರ ದೇಹ ಬಹಳ ಬೇಗ ಬಿದ್ದು ಹೋಗಬಹುದೆಂಬ ಸುದ್ದಿ ಹಬ್ಬತೊಡಗಿತು. ಎಂದಿಗಿಂತಲೂ ಹೆಚ್ಚಾಗಿ ಜನರು\break ಬರಲು ಮೊದಲಾಯಿತು. ಭರತಖಂಡದಲ್ಲಿ ಮಹಾ ದೈವಭಕ್ತರನ್ನು ನೋಡಲು ಜನರು ಹೇಗೆ ಬರುತ್ತಾರೆ. ಅವರ ಸುತ್ತಲೂ ಮುತ್ತಿ, ಅವರು ಬದುಕಿರುವಾಗಲೇ ಅವರನ್ನು ಹೇಗೆ ದೇವರಂತೆ ಭಾವಿಸುತ್ತಾರೆ ಎಂಬುದನ್ನು ನೀವು ಊಹಿಸಲಾರಿರಿ. ಅವರ ವಸ್ತ್ರದ ತುದಿಯನ್ನು ಮುಟ್ಟಲು ಸಹಸ್ರಾರು ಜನರು ಕಾದು ನಿಂತಿರುತ್ತಾರೆ. ಇತರರಲ್ಲಿ ಅಧ್ಯಾತ್ಮಿಕತೆಯನ್ನು ಗೌರವಿಸುವುದರಿಂದಲೇ ನಮ್ಮಲ್ಲಿ ಆಧ್ಯಾತ್ಮಿಕತೆ ಉದ್ಭವಿಸುವುದು. ಯಾವುದು ಮನುಷ್ಯನಿಗೆ ಬೇಕೋ, ಯಾವುದನ್ನು ಅವನು ಗೌರವಿಸುತ್ತಾನೆಯೋ ಅದನ್ನು ಪಡೆಯುವನು. ರಾಷ್ಟ್ರವೂ ಹೀಗೆಯೆ. ನೀವು ಭರತಖಂಡಕ್ಕೆ ಹೋಗಿ ರಾಜಕೀಯ ಉಪನ್ಯಾಸವನ್ನು ಕೊಟ್ಟರೆ, ಅದು ಎಷ್ಟೇ ಮನೋರಂಜಕವಾಗಿದ್ದರೂ, ಅದನ್ನು ಕೇಳುವುದಕ್ಕೆ ಸಭಿಕರೇ ಇರುವುದಿಲ್ಲ. ಹೋಗಿ ಧರ್ಮವನ್ನು ಬೋಧಿಸಿ; ಮಾತನಾಡುವುದು ಮಾತ್ರವಲ್ಲ, ಅದರಂತೆ ಬಾಳಿ,\break ನಿಮ್ಮ ದರ್ಶನ ಮಾಡಲು, ಪಾದಗಳನ್ನು ಮುಟ್ಟಲು, ನೂರಾರು ಜನರು ನೆರೆಯುವರು. ಜನರು ಈ ಮಹಾಪುರುಷ ಬೇಗ ತಮ್ಮಿಂದ ಕಣ್ಮರೆಯಾಗುವನೆಂಬುದನ್ನು ಕೇಳಿ\break ಎಂದಿಗಿಂತಲೂ ಹೆಚ್ಚಾಗಿ ಬರತೊಡಗಿದರು. ನನ್ನ ಗುರು ಆರೋಗ್ಯವನ್ನು ಸ್ವಲ್ಪವೂ\break ಗಮನಿಸದೆ ಉಪದೇಶ ಮಾಡುತ್ತ ಹೋದರು. ನಮಗೆ ಇದನ್ನು ನಿಲ್ಲಿಸಲು ಸಾಧ್ಯವಾಗಲಿಲ್ಲ. ಹಲವು ಜನರು ಬಹಳ ದೂರದಿಂದ ಬರುತ್ತಿದ್ದರು. ಅವರ ಪ್ರಶ್ನೆಗಳನ್ನು ಬಗೆಹರಿಸು\-ವವರೆಗೂ ಇವರು ಶಾಂತರಾಗುತ್ತಿರಲಿಲ್ಲ. “ಎಲ್ಲಿಯವರೆಗೆ ನನಗೆ ಮಾತನಾಡುವುದು ಸಾಧ್ಯವೋ ಅಲ್ಲಿಯವರೆವಿಗೂ ಉಪದೇಶ ಮಾಡುವೆನು” ಎನ್ನುತ್ತಿದ್ದರು. ಮಾತಿನಷ್ಟೇ\break ಸತ್ಯವಾಗಿತ್ತು ಅವರ ನಡವಳಿಕೆ. ಒಂದು ದಿನ ಅವರು, ಇಂದು ತಮ್ಮ ದೇಹವನ್ನು\break ತೊರೆಯುತ್ತೇನೆಂದರು. ವೇದಗಳ ಪವಿತ್ರತಮ ಮಂತ್ರಗಳನ್ನು ಉಚ್ಚರಿಸುತ್ತ ಸಮಾಧಿಸ್ಥರಾಗಿ ದೇಹವನ್ನು ತ್ಯಜಿಸಿದರು.

ಅವರ ಭಾವನೆ ಮತ್ತು ಸಂದೇಶ, ಇವನ್ನು ವಿವರಿಸುವುದಕ್ಕೆ ಅರ್ಹರಾದ ಎಲ್ಲೋ\break ಕೆಲವರಿಗೆ ಮಾತ್ರ ತಿಳಿದಿದ್ದಿತು. ಪ್ರಪಂಚವನ್ನು ತೊರೆದು ಅವರ ಕಾರ್ಯವನ್ನು\break ಮಾಡಲು ಬದ್ಧಕಂಕಣರಾದ ಕೆಲವು ಬಾಲಕರನ್ನು ಅವರು ಹಿಂದೆ ಬಿಟ್ಟಿದ್ದರು; ಅವರನ್ನು ಅಡಗಿಸಲು ಜನರು ಪ್ರಯತ್ನಪಟ್ಟರು. ಆದರೆ ಆ ಮಹಾನುಭಾವನ ಜೀವನ ಸ್ಫೂರ್ತಿಯಿಂದ ಅವರು ಅಚಲರಾಗಿ ನಿಂತರು. ಆ ಪುಣ್ಯಾತ್ಮನ ಹಲವು ವರ್ಷಗಳ ಸಂಗವಶದಿಂದ ತಮ್ಮ ಆದರ್ಶವನ್ನು ಅವರು ತೊರೆಯಲಿಲ್ಲ. ಸಂನ್ಯಾಸಿಗಳಾಗಿದ್ದ ಈ ತರುಣರು,\break ಕೆಲವರು ಶ‍್ರೀಮಂತರ ಕುಲದಲ್ಲಿ ಹುಟ್ಟಿದ್ದರೂ, ತಾವು ಹುಟ್ಟಿದ ಊರಿನಲ್ಲಿ ಭಿಕ್ಷೆ\break ಬೇಡಿದರು. ಮೊದಲು ಅತಿ ವಿರೋಧವನ್ನು ಎದುರಿಸಬೇಕಾಗಿತ್ತು. ಆದರೂ ಅವರು\break ಛಲದಿಂದ ಆದರ್ಶವನ್ನು ತೊರೆಯದೆ ದಿನ ಕಳೆದಂತೆ ಆ ಮಹಾತ್ಮನ ಸಂದೇಶದಿಂದ ಇಡೀ ದೇಶವೆಲ್ಲ ಅನುರಣಿತವಾಗುವಂತೆ ಭರತಖಂಡದಲ್ಲೆಲ್ಲ ಉಪದೇಶವನ್ನು ಹರಡುತ್ತ\break ಹೋದರು. ಎಲ್ಲೋ ಬಂಗಾಳದ ದೂರದ ಹಳ್ಳಿಯಲ್ಲಿದ್ದು, ಯಾವ ವಿದ್ಯಾಭ್ಯಾಸವೂ ಇಲ್ಲದೆ, ತಮ್ಮ ಸ್ವಂತ ಇಚ್ಛಾಶಕ್ತಿಯ ಛಲದಿಂದ ಸತ್ಯಸಾಕ್ಷಾತ್ಕಾರವನ್ನು ಮಾಡಿಕೊಂಡು, ಅದನ್ನು ಸಜೀವವಾಗಿಟ್ಟಿರುವುದಕ್ಕೆ ಕೆಲವು ಶಿಷ್ಯರನ್ನು ಬಿಟ್ಟು, ಎಲ್ಲರಿಗೂ ಅದನ್ನು\break ಹರಡಿದರು ಆ ಮಹಾನುಭಾವರು.

ಇಂದು ಭರತಖಂಡದಲ್ಲಿರುವ ಲಕ್ಷಾಂತರ ಜನರು ಶ‍್ರೀರಾಮಕೃಷ್ಣ ಪರಮಹಂಸರ ಹೆಸರನ್ನು ಕೇಳಿರುವರು. ಅದು ಮಾತ್ರವಲ್ಲ ಅವರ ಶಕ್ತಿ ಪ್ರಭಾವ ಭರತಖಂಡವನ್ನು\break ಮೀರಿಹೋಗಿರುವುದು. ಪ್ರಪಂಚದಲ್ಲಿ ಎಲ್ಲಿಯಾದರೂ ನಾನು ಒಂದು ಸತ್ಯವನ್ನು\break ಹೇಳಿದ್ದರೆ, ಆಧ್ಯಾತ್ಮಿಕ ಭಾವವನ್ನು ಬೀರಿದ್ದರೆ, ಅದು ನನ್ನ ಗುರುವಿನ ಪ್ರಭಾವ; ತಪ್ಪುಗಳು ಮಾತ್ರ ನನ್ನವು.

ಆಧುನಿಕ ಪ್ರಪಂಚಕ್ಕೆ ಶ‍್ರೀರಾಮಕೃಷ್ಣರ ಸಂದೇಶವಿದು: “ಸಿದ್ಧಾಂತಗಳನ್ನು ಲೆಕ್ಕಿಸಬೇಡಿ, ಮೂಢನಂಬಿಕೆಗಳನ್ನು ಲೆಕ್ಕಿಸಬೇಡಿ, ಕೋಮು, ಚರ್ಚು ದೇವಸ್ಥಾನ ಇವುಗಳನ್ನು ಲೆಕ್ಕಿಸಬೇಡಿ. ಪ್ರತಿಯೊಬ್ಬ ಮಾನವನ ಜೀವನದ ಸಾರವಾದ ಅಧ್ಯಾತ್ಮದೊಂದಿಗೆ ಇವನ್ನು ಹೋಲಿಸಿದರೆ ಇವಕ್ಕೆ ಬೆಲೆಯಿಲ್ಲ. ಮನುಷ್ಯನಲ್ಲಿ ಇದು ಹೆಚ್ಚು ವಿಶದವಾಗಿ ಸ್ಪಷ್ಟವಾಗಿ ತೋರಿದಷ್ಟು ಸತ್ಕಾರ್ಯ ಸಾಧನೆಗೆ ಅವನು ಮಹಾಶಕ್ತಿಯಾಗುವನು. ಮೊದಲು ಅದನ್ನು ಸಂಪಾದಿಸಿ, ಸಾಧಿಸಿ, ಯಾರನ್ನೂ ದೂರಬೇಡಿ. ಎಲ್ಲಾ ಸಿದ್ಧಾಂತಗಳಲ್ಲೂ ಪಂಥಗಳಲ್ಲೂ ಸ್ವಲ್ಪ ಸತ್ಯವಿದೆ. ಧರ್ಮವೆಂದರೆ ಮಾತಲ್ಲ. ಹೆಸರಲ್ಲ, ಪಂಥವಲ್ಲ; ಅದು ಆಧ್ಯಾತ್ಮಿಕ ಸಾಕ್ಷಾತ್ಕಾರವೆಂಬುದನ್ನು ನಿಮ್ಮ ಜೀವನದಲ್ಲಿ ತೋರಿ. ಯಾರು ಇದನ್ನು ಅನುಭವಿಸುವರೊ, ಅವರು ಮಾತ್ರ ಇದನ್ನು ತಿಳಿದುಕೊಳ್ಳಬಲ್ಲರು. ಯಾರಲ್ಲಿ ಆಧ್ಯಾತ್ಮಿಕತೆ ಇದೆಯೊ ಅವರು ಮಾತ್ರ ಮತ್ತೊಬ್ಬರಿಗೆ ಇದನ್ನು ಕೊಡಬಲ್ಲರು, ಮಾನವಕೋಟಿಯ ಮಹಾಗುರುಗಳಾಗಬಲ್ಲರು, ಅವರೇ ಜ್ಞಾನಶಕ್ತಿ.”

ಇಂದು ದೇಶದಲ್ಲಿ ಇಂತಹ ಮಹಾತ್ಮರು ಹೆಚ್ಚಿದಷ್ಟೂ ಆ ದೇಶದ ಗೌರವ ಹೆಚ್ಚುವುದು. ಯಾವ ದೇಶದಲ್ಲಿ ಇಂತಹ ಮಹಾತ್ಮರು ಇಲ್ಲವೇ ಇಲ್ಲವೋ ಆ ದೇಶ ನಾಶವಾದಂತೆ. ಯಾವ ಶಕ್ತಿಯೂ ಅದನ್ನು ಸಂರಕ್ಷಿಸಲಾರದು. ಆದಕಾರಣ ಮಾನವಕೋಟಿಗೆ ನನ್ನ ಗುರುದೇವನ ಸಂದೇಶ “ಆಧ್ಯಾತ್ಮಿಕ ಸತ್ಯವನ್ನು ನೀವೇ ಸಾಕ್ಷಾತ್ಕಾರ ಮಾಡಿಕೊಳ್ಳಿ” ಎಂಬುದು. ನಿಮ್ಮ ಸಹೋದರರಿಗಾಗಿ ತ್ಯಾಗ ಮಾಡಿ ಎಂದು ಅವರು ನಮಗೆ ಹೇಳುತ್ತಿದ್ದರು. “ನಿಮ್ಮ ಸಹೋದರರಾಗಿ ಪ್ರೀತಿಯನ್ನು ತೋರುವೆವು ಎಂದು ಮಾತನಾಡುವುದನ್ನು ನಿಲ್ಲಿಸಿ. ನಿಮ್ಮ ಮಾತನ್ನು ಸಮರ್ಥಿಸಲು ಕೆಲಸದಲ್ಲಿ ನಿರತರಾಗಿ” ಎನ್ನುತ್ತಿದ್ದರು. ತ್ಯಾಗಕ್ಕೆ ಸಮಯ ಬಂದಿದೆ. ಸಾಕ್ಷಾತ್ಕಾರಕ್ಕೆ ಸಮಯ ಬಂದಿದೆ. ಸಾಕ್ಷಾತ್ಕಾರವಾದಾಗ ವಿಶ್ವದ ಧರ್ಮಗಳಲ್ಲೆಲ್ಲಾ ಒಂದು ಸಾಮರಸ್ಯವನ್ನು ನೋಡುವಿರಿ. ವೈಮನಸ್ಯಕ್ಕೆ ಕಾರಣವಿಲ್ಲವೆಂದು ಆಗ ಗೊತ್ತಾಗುವುದು. ಆಗ ಮಾತ್ರ ನೀವು ಮಾನವಕೋಟಿಗೆ ಸಹಾಯಮಾಡಲು ಸಿದ್ಧರಾಗುವಿರಿ. ಎಲ್ಲ ಧರ್ಮಗಳ ಅಂತರಾಳದಲ್ಲಿರುವ ಅತಿ ಮುಖ್ಯವಾದ ಐಕ್ಯತೆಯನ್ನು ವಿವರಿಸುವುದೆ, ಅದನ್ನು ಜಗತ್ತಿಗೆ ಸಾರುವುದೆ, ನನ್ನ ಗುರುದೇವನ ಜೀವನ ಧ್ಯೇಯವಾಗಿತ್ತು. ಉಳಿದ ಗುರುಗಳು ತಮ್ಮ ಹೆಸರಿನಲ್ಲಿ ಬೇರೆ ಬೇರೆ ಧರ್ಮಗಳನ್ನು ಬೋಧಿಸಿದರು. ಆದರೆ ಇಪ್ಪತ್ತನೇ ಶತಮಾನದ ಈ ಮಹಾಪುರುಷ ಯಾವುದನ್ನೂ ತನ್ನದೆಂದು ಸಮರ್ಥಿಸಲಿಲ್ಲ. ಅವರು ಯಾವ ಧರ್ಮವನ್ನೂ ಅಲುಗಿಸಲು ಹೋಗಲಿಲ್ಲ. ಏಕೆಂದರೆ ವಾಸ್ತವವಾಗಿ ಎಲ್ಲಾ ಧರ್ಮಗಳೂ ಒಂದೇ ಸನಾತನ ತತ್ತ್ವದ ಹಲವು ಅವಿಭಾಜ್ಯ ಅಂಗಗಳು.

