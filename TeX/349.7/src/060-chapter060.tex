
\vspace{-0.8cm}

\chapter[ಧರ್ಮದ ಸಾರ ]{ಧರ್ಮದ ಸಾರ \protect\footnote{\engfoot{C.W. Vol. VIII, P. 254}}}

\centerline{\textbf{(ಅಮೆರಿಕದಲ್ಲಿ ನೀಡಿದ ಉಪನ್ಯಾಸದ ವರದಿ)}}

‘ಮಾನವನ ಹಕ್ಕುಗಳು’ ಎಂಬ ಫ್ರೆಂಚ್​ ಜನಾಂಗದ ಪಲ್ಲವಿಯು ಬಹಳ ಕಾಲದವರೆಗೆ ಇತ್ತು. ಅಮೆರಿಕದಲ್ಲಿ ಸ್ತ್ರೀಯರ ಹಕ್ಕು ಎಂಬುದು ಈಗಲೂ ಕೇಳಿಸುತ್ತದೆ. ಆದರೆ ಇಂಡಿಯಾದಲ್ಲಿ ದೇವರ ಹಕ್ಕು ನಮಗೆ ಮುಖ್ಯವಾಗಿದೆ.

ವೇದಾಂತವು ಎಲ್ಲಾ ಪಂಥಗಳಲ್ಲೂ ಒಳಗೊಳ್ಳುವುದು ಇಂಡಿಯಾ ದೇಶದಲ್ಲಿ ಒಂದು ವಿಚಿತ್ರ ಭಾವನೆ ಇದೆ. ನನಗೆ ಒಂದು ಮಗು ಇದ್ದರೆ ಅದಕ್ಕೆ ನಾನು ಯಾವ ಧರ್ಮವನ್ನೂ ಕಲಿಸುವುದಿಲ್ಲ. ಮನಸ್ಸನ್ನು ಹೇಗೆ ಏಕಾಗ್ರಮಾಡುವುದು ಎಂಬುವುದನ್ನು ಹೇಳಿಕೊಟ್ಟು, “ತ್ರಿಭುವನಗಳಿಗೆ ಒಡೆಯನಾದ ಪರಂಜ್ಯೋತಿಯನ್ನು ನಾನು ಧ್ಯಾನಿಸುತ್ತೇನೆ, ಅವನು ನನ್ನ ಬುದ್ಧಿಯನ್ನು ಪ್ರಚೋದಿಸಲಿ” ಎಂಬ ಪ್ರಾರ್ಥನೆಯನ್ನು- ಪ್ರಾರ್ಥನೆ ಎಂಬುದು ನಿಮ್ಮ ಅರ್ಥದಲ್ಲಿ ಅಲ್ಲ-ಮಾತ್ರ ಹೇಳಿ ಕೊಡುತ್ತೇನೆ.

ವಯಸ್ಸಿಗೆ ಬಂದ ಮೇಲೆ ಅವನು ಹಲವು ತತ್ತ್ವಗಳನ್ನು ಮತ್ತು ಧರ್ಮಗಳನ್ನು\break ಕೇಳುತ್ತಾ ಹೋಗುವುದು. ಅವುಗಳಲ್ಲಿ ಯಾವುದನ್ನು ಬೇಕಾದರೂ ಅವನು ಆಯ್ದುಕೊಳ್ಳ\-ಬಹುದು ತನಗೆ ತೋರಿದ ಇಷ್ಟದೇವತೆಯನ್ನು ಉಪಾಸನೆ ಮಾಡುವುದಕ್ಕೆ ಅವನಿಗೆ ಸ್ವಾತಂತ್ರ್ಯವಿದೆ. ಆದಕಾರಣ ಏಕಕಾಲದಲ್ಲೇ ನನ್ನ ಮಗ ಬೌದ್ಧನಾಗಿರಬಹುದು, ಆಗ\break ಅವನ ಆ ಧರ್ಮದ ಸತ್ಯವನ್ನು ಬೋಧಿಸುವ ಗುರುವಿನ ಶಿಷ್ಯನಾಗಬಹುದು. ಅಥವಾ ಕ್ರಿಸ್ತನನ್ನೋ, ಬುದ್ಧನನ್ನೋ ಮಹಮ್ಮದನನ್ನೋ ಪೂಜಿಸಬಹುದು. ನನ್ನ ಹೆಂಡತಿ ಕ್ರೈಸ್ತಳಾಗಿರ\break ಬಹುದು, ನಾನು ಮಹಮ್ಮದೀಯನಾಗಿರಬಹುದು. ಪ್ರತಿಯೊಬ್ಬನಿಗೂ ತನ್ನದೇ ಆದ ಸ್ವಾತಂತ್ರ್ಯವಿದೆ.

ಎಲ್ಲಾ ಪಥಗಳೂ ಒಬ್ಬನೇ ದೇವರೆಡೆಗೆ ಒಯ್ಯುವುವು ಎಂಬುದನ್ನು ನೆನಸಿಕೊಂಡರೆ ನಮಗೆ ಸಂತೋಷವಾಗುವುದು. ಎಲ್ಲರೂ ನನ್ನ ಮೂಲಕವೇ ದೇವರನ್ನು ನೋಡಿದಾಗ ಮಾತ್ರ ಪ್ರಪಂಚದ ಉದ್ಧಾರವಾಗುವುದು ಎಂದು ನಾವು ಭಾವಿಸುವುದಿಲ್ಲ. ನಿನ್ನ ತತ್ತ್ವ ನನ್ನ ತತ್ತ್ವವಾಗಲಾರದು, ನನ್ನದು ನಿನ್ನದಾಗಲಾರದು ಎಂಬುದು. ನಮ್ಮ ಮೂಲಭೂತ ಸಿದ್ಧಾಂತ ನಾನು ನನ್ನದೇ ಆದ ಪಂಥ. ಪ್ರಪಂಚದಲ್ಲಿರುವ ಏಕಮಾತ್ರ ವಿಚಾರಪರ ಧರ್ಮವನ್ನು ನಾವು ಭರತಖಂಡದಲ್ಲಿ ನಾವು ಸೃಷ್ಟಿಸಿರುವೆವು ಎಂಬುದು ಸತ್ಯ. ಆದರೆ ದೇವರನ್ನು ನಾವು ಹಲವು ಬಗೆಗಳಲ್ಲಿ ಹುಡುಕುವವರೆಲ್ಲ ಸೇರುವರು; ಎಲ್ಲಾ ಉಪಾಸನೆಗಳನ್ನು ಸೌಹಾರ್ದದಿಂದ ನೋಡುವರು, ಪ್ರಪಂಚದಲ್ಲಿ ದೇವರೆಡೆಗೆ ಹೋಗುವವರನ್ನೆಲ್ಲಾ ಇದು ಎಂದೆಂದಿಗೂ ಸ್ವಾಗತಿಸುವುದು.

ನಮ್ಮ ಸಿದ್ಧಾಂತದಲ್ಲಿ ನ್ಯೂನತೆ ಇದೆ ಎಂಬುದನ್ನು ನಾವು ಒಪ್ಪಿಕೊಳ್ಳುತ್ತೇವೆ. ಏಕೆಂದರೆ ಸತ್ಯ ಎಲ್ಲಾ ಸಿದ್ಧಾಂತಗಳಿಗೂ ಅತೀತವಾಗಿರಬೇಕು. ನಾವು ಇದನ್ನು ಒಪ್ಪಿಕೊಳ್ಳುವುದರಿಂದ\break ಅನಂತ ವಿಕಾಸಕ್ಕೆ ಮತ್ತು ಬೆಳವಣಿಗೆಗೆ ಇಲ್ಲಿ ಅವಕಾಶವಿದೆ. ಜಾತಿ, ಆಚಾರ, ಶಾಸ್ತ್ರ,\break ಇವುಗಳೆಲ್ಲ ಮಾನವನು ತನ್ನ ದಿವ್ಯತೆಯನ್ನು ತಿಳಿದುಕೊಳ್ಳುವುದಕ್ಕೆ ಸಹಾಯ ಮಾಡಿದರೆ\break ಒಳ್ಳೆಯದು. ನಿಜವಾದ ಸತ್ಯವನ್ನು ಅರಿತ ಮೇಲೆ ಇದನ್ನೆಲ್ಲಾ ಅವನೇ ತ್ಯಜಿಸುವನು.

ನಾನೂ ವೇದವನ್ನು ತ್ಯಜಿಸುತ್ತೇನೆ ಎಂಬುದು ವೇದಾಂತದರ್ಶನದ ಚರಮ\break ಬೋಧನೆ. ಸತ್ಯದ ಸಮೀಪಕ್ಕೆ ಹೋದರೆ ಆಚಾರ, ಪ್ರಾರ್ಥನೆ, ಶಾಸ್ತ್ರ ಮುಂತಾದುವೆಲ್ಲ\break ಅವನಿಗೆ ಮಾಯವಾಗುವುವು. ‘ಸೋಽಹಂ’ - ನಾನೇ ಅವನು ಎಂದು ಕೊನೆಗೆ\break ಹೇಳುವನು. ದೇವರು ಹಾಗೂ ನಾನು ಬೇರೆ ಬೇರೆ ಎಂದು ಹೇಳುವುದು ಈಶ್ವರನಿಂದೆ. ಏಕೆಂದರೆ ಅವನು ದೇವರೊಂದಿಗೆ ಏಕವಾಗಿರುವನು.

ವೇದದಲ್ಲಿ ಎಷ್ಟು ಅಂಶ ನನ್ನ ಯುಕ್ತಿಗೆ ಸಮಂಜಸವಾಗಿ ಕಾಣಿಸುವುದೋ ಅಷ್ಟನ್ನು ಮಾತ್ರ ನಾನು ಸ್ವೀಕರಿಸುತ್ತೇನೆ. ವೇದಗಳಲ್ಲಿ ಕೆಲವು ಭಾಗಗಳು ವಿರೋಧಾಭಾಸಗಳಿಂದ ತುಂಬಿವೆ. ಪಾಶ್ಚಾತ್ಯರ ದೃಷ್ಟಿಯಲ್ಲಿ ವೇದಗಳು ಸ್ಫೂರ್ತಿಯಿಂದ ಹುಟ್ಟಿದವುಗಳಲ್ಲ; ಆದರೆ ಅವು ನಮಗೆ ಭಗವಂತನ ವಿಷಯವಾಗಿ ಇರುವ ಜ್ಞಾನದ ಮೊತ್ತ. ಆದರೆ ವೇದಗಳಲ್ಲಿ ಮಾತ್ರ ಆ ಸತ್ಯವಿದೆ, ಉಳಿದ ಕಡೆಗಳಲ್ಲಿ ಇಲ್ಲ ಎಂಬುದು ಕುತರ್ಕ. ಎಲ್ಲಾ ಧರ್ಮಗಳಲ್ಲಿಯೂ ಈ ಸತ್ಯ ಸಾಮಾನ್ಯವಾಗಿದೆ ಎಂಬುದು ನನಗೆ ಗೊತ್ತಿದೆ. ಮನುವು ವೇದದಲ್ಲಿ ಯಾವ ಭಾಗ ನಮ್ಮ ವಿಚಾರದ ಪರೀಕ್ಷೆಗೆ ನಿಲ್ಲುವುದೋ ಅದು ಮಾತ್ರ ನಿಜ ಎನ್ನುವನು. ನಮ್ಮಲ್ಲಿ ಅನೇಕ ತತ್ತ್ವಜ್ಞರು ಇದೇ ದೃಷ್ಟಿಯಿಂದ ನೋಡುವರು. ಪ್ರಪಂಚದ ಧರ್ಮಗ್ರಂಥಗಳಲ್ಲೆಲ್ಲಾ ಈ ವೇದ ಒಂದೇ, ‘ವೇದಾಧ್ಯಯನವು ಗೌಣ’ ಎಂದು ಸಾರುವುದು.

ಯಾವುದರಿಂದ ಎಂದಿಗೂ ಬದಲಾಗದ ಸತ್ಯವನ್ನು ನಾವು ಅರಿಯುವೆವೋ ಅದೇ ನಿಜವಾದ ವಿದ್ಯೆ. ಇದು ಓದಿನಿಂದ, ಯುಕ್ತಿಯಿಂದ, ನಂಬಿಕೆಯಿಂದ ಬರುವುದಿಲ್ಲ. ಇದಕ್ಕೆ ಪ್ರತ್ಯಕ್ಷ ಅನುಭವ ಮತ್ತು ಸಮಾಧಿ ಬೇಕು. ಒಬ್ಬ ಈ ಸ್ಥಿತಿಯನ್ನು ಪಡೆದಾಗ ಅವನ ಸ್ವಭಾವ ಕೂಡ ಸಗುಣ ಬ್ರಹ್ಮನ ಸ್ವಭಾವದಂತೆಯೇ ಆಗುತ್ತದೆ. “ನಾನು ಮತ್ತು ನನ್ನ ತಂದೆ ಇಬ್ಪರೂ ಒಂದೇ.” ತಾನು ನಿರ್ಗುಣ ಬ್ರಹ್ಮ ಎಂದು ಗೊತ್ತಿದೆ. ಆದರೆ ಸಗುಣ ಬ್ರಹ್ಮನಂತೆ (ಈಶ್ವರನಂತೆ) ವ್ಯಕ್ತವಾಗುತ್ತಿರುವನು. ಮಾಯೆಯ ತೆರೆಯ ಮೂಲಕ ಕಾಣಿಸುವ ನಿರ್ಗುಣ ಬ್ರಹ್ಮವೇ ಈಶ್ವರ (ಸಗುಣ ಬ್ರಹ್ಮ).

ನಾನು ಪಂಚೇಂದ್ರಿಯಗಳ ಮೂಲಕ ಅವನನ್ನು ನೋಡಿದಾಗ ಅವನನ್ನು ಸಾಕಾರನಂತೆ\break ಮಾತ್ರ ನೋಡಬಹುದು. ಆತ್ಮವನ್ನು ನಾವು ಬಾಹ್ಯದಲ್ಲಿ ಕಾಣುವಂತೆ ಮಾಡುವುದಕ್ಕೆ\break ಆಗುವುದಿಲ್ಲ. ತಿಳಿಯುವವನು ತನ್ನನ್ನು ತಾನೇ ಹೇಗೆ ತಿಳಿಯಬಲ್ಲ? ಆದರೆ ಅವನು\break ಬೇಕಾದರೆ ತನ್ನ ಛಾಯೆ ಬೀಳುವಂತೆ ಮಾಡಬಹುದು. ಆ ಛಾಯೆಯ ಅತ್ಯುನ್ನತ ರೂಪವೇ, ಎಂದರೆ ತನ್ನನ್ನು ತಾನು ವಿಷಯೀಕರಿಸಿಕೊಳ್ಳುವ ಪ್ರಯತ್ನವೇ ಸಗುಣ ಬ್ರಹ್ಮ. ಆತ್ಮ ನಿತ್ಯ ಸಾಕ್ಷಿ. ನಾವು ನಿತ್ಯವೂ ಅದನ್ನು ಹೊರಗೆ ವ್ಯಕ್ತಗೊಳಿಸಲು ಯತ್ನಿಸುತ್ತಿರುವೆವು. ಈ ಹೋರಾಟದಿಂದ ವಿಶ್ವವು ಸೃಷ್ಟಿಯಾಗಿದೆ. ಅದೇ ದ್ರವ್ಯ. ಆದರೆ ಇವೆಲ್ಲ ದುರ್ಬಲ ಪ್ರಯತ್ನಗಳು. ಆತ್ಮನ ವಿಷಯೀಕರಣದ ಅತ್ಯುತ್ತಮ ರೂಪವೇ ಈಶ್ವರ.

“ಒಬ್ಬ ಸತ್ಯಸಂಧನಾದ ದೇವರು ಮಾನವ ಶ್ರೇಷ್ಠತಮ ಕೃತಿ” ಎಂದು ಒಬ್ಬ ಪಾಶ್ಚಾತ್ಯ ತತ್ತ್ವಜ್ಞಾನಿ ಹೇಳಿರುವನು. ಮನುಷ್ಯನಂತೆ ದೇವರು. ಯಾರೂ ಮಾನವನ ಮೂಲಕ\break ಅಲ್ಲದೆ ದೇವರನ್ನು ಕಾಣಲಾರರು. ದೇವರನ್ನು ಮಾನವನ ಮೂಲಕ ಅಲ್ಲದೆ ನಾವು ನೋಡಲಾರೆವು. ನಮ್ಮಂತೆಯೇ ಅವನು ಇರುವನು. ಏನೂ ತಿಳಿಯದ ಒಬ್ಬನಿಗೆ ಶಿವನನ್ನು\break ಮಾಡು ಎಂದು ಯಾರೋ ಹೇಳಿದರು. ಹಲವು ದಿನಗಳವರೆಗೆ ಪ್ರಯತ್ನಪಟ್ಟಾದ\break ಮೇಲೆ ಅವನೊಂದು ಕೋತಿಯ ವಿಗ್ರಹ ಮಾಡಿದನು! ದೇವರು ಹೇಗಿರುವನೋ ಹಾಗೆ ನಾವು ಭಾವಿಸಬೇಕೆಂದು ಯತ್ನಿಸಿದಾಗ ನಮ್ಮ ಪ್ರಯತ್ನವೆಲ್ಲ ನಿಷ್ಪ್ರಯೋಜಕವಾಗುವುದು. ಏಕೆಂದರೆ ಈಗ ನಾವಿರುವ ಸ್ಥಿತಿಯಲ್ಲಿ ದೇವರನ್ನು ಮಾನವರಂತೆ ಮಾತ್ರ ನೋಡಲು ಸಾಧ್ಯ.

ಎಮ್ಮೆಗಳು ದೇವರನ್ನು ಪೂಜಿಸಬೇಕೆಂದು ಯತ್ನಿಸಿದರೆ ಅವು ತಮ್ಮ ಸ್ವಭಾವಕ್ಕೆ\break ಅನುಗುಣವಾಗಿ ದೇವರನ್ನು ದೊಡ್ಡ ಎಮ್ಮೆಯಂತೆ ಮಾತ್ರ ನೋಡಬಲ್ಲವು. ಮೀನು\break ದೇವರನ್ನು ಉಪಾಸನೆ ಮಾಡಬೇಕಾದರೆ ಒಂದು ದೊಡ್ಡ ಮೀನಿನಂತೆ ಮಾತ್ರ ದೇವರನ್ನು ಕಲ್ಪಿಸಿಕೊಳ್ಳಬಲ್ಲುದು. ಮಾನವನು ಮಾನವನಂತೆ ಮಾತ್ರ ದೇವರನ್ನು ಕಲ್ಪಿಸಿಕೊಳ್ಳಬಲ್ಲ. ಮನುಷ್ಯ, ಎಮ್ಮೆ, ಮೀನು ಇವೆಲ್ಲ ಬೇರೆ ಬೇರೆ ಪಾತ್ರೆಗಳು ಎಂದು ಊಹಿಸಿ. ಇವೆಲ್ಲಾ ದೇವರು ಎಂಬ ಸಮುದ್ರಕ್ಕೆ ನೀರು ತರಲು ಹೋದರೆ ಈ ನೀರು ಆಯಾಯ ಪಾತ್ರೆಯ ಯೋಗ್ಯತೆಗೆ ತಕ್ಕಂತೆ ಹಿಡಿಯುವುದು; ಮನುಷ್ಯನ ಪಾತ್ರೆಯಲ್ಲಿ ಮನುಷ್ಯನಂತೆ ಇರುವುದು. ಎಮ್ಮೆಯ ಪಾತ್ರೆಯಲ್ಲಿ ಎಮ್ಮೆಯಂತೆ ಇರುವುದು, ಮೀನಿನ ಪಾತ್ರೆಯಲ್ಲಿ ಮೀನಿನಂತೆ ಇರುವುದು. ಆದರೆ ಎಲ್ಲಾ ಪಾತ್ರೆಗಳಲ್ಲಿಯೂ ದೇವರೆಂಬ ನೀರು ಮಾತ್ರ ಇರುವುದು.

ಎರಡು ಬಗೆಯ ಜನರು ದೇವರನ್ನು ಮಾನವರಂತೆ ಪೂಜಿಸುವುದಿಲ್ಲ; ಒಬ್ಬ ತನ್ನ ಮಾನವತೆಯನ್ನು ಮೀರಿಹೋಗಿರುವ ಪರಮಹಂಸ - ಅವನಿಗೆ ಪ್ರಪಂಚ ತನ್ನ ಆತ್ಮಮಯನಾಗಿರುವನು. ಅವನು ಮಾತ್ರ ದೇವರನ್ನು ಅರಿಯಬಲ್ಲ. ಇನ್ನೊಬ್ಬ ಪಶುಸದೃಶನಾದ ಮಾನವ. ಅವನು ದೇವರನ್ನು ಮಾನವರಂತೆ ಪೂಜಿಸುವುದಿಲ್ಲ. ಏಕೆಂದರೆ ದೇವರ ಆವಶ್ಯಕತೆ ಅವನಿಗಿನ್ನೂ ಬಂದಿಲ್ಲ. ಜೀವನ್ಮುಕ್ತರು ಮಾನವರಂತೆ ದೇವರನ್ನು ಪೂಜಿಸುವುದಿಲ್ಲ. ಏಕೆಂದರೆ ತಮ್ಮಲ್ಲಿಯೇ ಅವರು ದೇವರನ್ನು ಅರಿತುಕೊಂಡಿರುವರು. ಅವರು “ಸೋಽಹಂ’ ‘ನಾನೇ ಅವನು’ ಎನ್ನುವರು. ಹೀಗಿರುವಾಗ ಅವರು ತಮ್ಮನ್ನು ತಾವೇ ಹೇಗೆ ಪೂಜೆ ಮಾಡಿಕೊಳ್ಳಬಲ್ಲರು?

ನಾನು ನಿಮಗೆ ಒಂದು ಸಣ್ಣ ಕತೆಯನ್ನು ಹೇಳುತ್ತೇನೆ. ಒಂದು ಸಲ ಸಿಂಹವೊಂದು\break ಕುರಿಮಂದೆಯಲ್ಲಿ ಮರಿಹಾಕಿ ಹೊರಟುಹೋಯಿತು. ಕುರಿಗಳು ಅದನ್ನು ನೋಡಿಕೊಳ್ಳುತ್ತಿದ್ದವು. ಕುರಿಗಳು ಅರಚುತ್ತಿದ್ದಂತೆ ಸಿಂಹದ ಮರಿಯೂ ಬ್ಯಾ ಎಂದು ಅರಚುತ್ತಿತ್ತು. ಒಂದು ದಿನ ಮತ್ತೊಂದು ಸಿಂಹ ಬಂದು ಇದನ್ನು ನೋಡಿತು. ನೀನು ಇಲ್ಲಿ ಏನು ಮಾಡುತ್ತಿರುವೆ ಎಂದು ಕೇಳಿತು. ಆ ಸಿಂಹದ ಮರಿ ನಾನೊಂದು ಕುರಿಮರಿ ನನಗೆ ನಿನ್ನನ್ನು ಕಂಡರೆ ಅಂಜಿಕೆಯಾಗುವುದು. ಎಂದು ಹೇಳಿತು. ಸಿಂಹ ಎಂತಹ ಮೂಢ ನೀನು, ನನ್ನೊಡನೆ ಬಾ, ನೀನು ಯಾರು ಎನ್ನುವುದನ್ನು ತೋರಿಸುತ್ತೇನೆ ಎಂದು ನೀರಿನ ಸಮೀಪಕ್ಕೆ ಕರೆದುಕೊಂಡು ಹೋಗಿ ತನ್ನ ಪ್ರತಿಬಿಂಬ, ಮತ್ತು ಅದರ ಪ್ರತಿಬಿಂಬ ತೋರಿ ನನ್ನಂತೆ ನೀನು ಸಹ ಎಂದಿತು. ಮರಿ ಆಗ ತನ್ನ ಪ್ರತಿಬಿಂಬವನ್ನು ನೋಡಿಕೊಂಡು, “ಹೌದು, ನಾನು ಕುರಿಯಂತೆ ಇಲ್ಲ, ನಾನೂ ಒಂದು ಸಿಂಹ” ಎಂದು ಗರ್ಜಿಸಿತು. ಇದರ ಗರ್ಜನೆಗೆ ಬೆಟ್ಟಗುಡ್ಡಗಳೆಲ್ಲ ಕಂಪಿಸಿದವು.

ನಾವು ಕುರಿಯ ಹೊದಿಕೆಯಲ್ಲಿರುವ ಸಿಂಹಗಳು. ಸುತ್ತಲೂ ಇರುವ ದುರ್ಬಲರಂತೆ ನಾವೂ ಕೂಡ ದುರ್ಬಲರು ಎಂದು ಭ್ರಾಂತರಾಗಿರುವೆವು. ವೇದಾಂತ ‘ಈ ಭ್ರಾಂತಿಯಿಂದ ಪಾರಾಗಿ’ ಎನ್ನುವುದು. ನಮ್ಮ ಗುರಿ ಮುಕ್ತಿ. ಮುಕ್ತಿ ಎಂದರೆ ಪ್ರಕೃತಿಯ ನಿಯಮಗಳಿಗೆ ತಲೆಬಾಗುವುದು ಎಂದು ನಾನು ಒಪ್ಪಿಕೊಳ್ಳುವುದಿಲ್ಲ. ನನಗೆ ಇದರ ಅರ್ಥ ಗೊತ್ತಾಗುವುದಿಲ್ಲ. ಮಾನವ ಪ್ರಗತಿಯ ಇತಿಹಾಸವನ್ನು ನೋಡಿದರೆ ಪ್ರಕೃತಿಯನ್ನು ಅವನು ವಿರೋಧಿಸಿದಾಗ ಮಾತ್ರ ಮಾನವನು ಪ್ರಗತಿಪರನಾದ ಎಂಬುದು ಗೊತ್ತಾಗುತ್ತದೆ. ವಿಕಾಸ ಎಂದರೆ ಪ್ರಕೃತಿಯ ಕೆಳಗಿನ ನಿಯಮವನ್ನು ಮೇಲಿನ ನಿಯಮದಿಂದ ಗೆಲ್ಲುವುದು ಎಂದು ಬೇಕಾದರೆ ಹೇಳಬಹುದು. ಅಲ್ಲಿಯೂ ಜೈತ್ರಯಾತ್ರೆಗೆ ಸಿದ್ಧವಾದ ಮನಸ್ಸು ಮೇಲಿನ ನಿಯಮವನ್ನು ಅಲ್ಲಗಳೆಯಲು ಹೊಂಚುಹಾಕುತ್ತಿರುವುದು ಕಾಣಿಸುವುದು. ನಿಯಮದ ಮೂಲಕ ಹೋರಾಟ ಮಾಡುತ್ತಿರುವೆ ಎಂದು ಗೊತ್ತಾದೊಡನೆ ಆ ನಿಯಮಕ್ಕೂ ಅತೀತವಾಗಿ ಹೋಗುವುದಕ್ಕೆ ಮನಸ್ಸು ಯತ್ನಿಸಿತು. ಆದಕಾರಣ ಆದರ್ಶ ಯಾವಾಗಲೂ ಸ್ವಾತಂತ್ರ್ಯ. ಮರ ಎಂದಿಗೂ ನಿಯಮಕ್ಕೆ ವಿರೋಧವಾಗಿ ಹೋಗುವುದಿಲ್ಲ. ಹಸು ಕದಿಯುವುದನ್ನು ನಾನೆಂದೂ ನೋಡಿಲ್ಲ. ಬಾತು ಎಂದಿಗೂ ಸುಳ್ಳನ್ನು ಹೇಳುವುದಿಲ್ಲ. ಆದರೂ ಇವು ಮಾನವನಿಗಿಂತ ಮೇಲಲ್ಲ.

ನಾವು ನಿಯಮಕ್ಕೆ ಯಾವಾಗಲೂ ಅಧೀನರಾಗಿದ್ದರೆ ಸಮಾಜದಲ್ಲಿ, ರಾಜಕೀಯದಲ್ಲಿ, ಧರ್ಮದಲ್ಲಿ, ಎಲ್ಲಾ ಕಾರ್ಯಕ್ಷೇತ್ರಗಳಲ್ಲಿಯೂ ನಾವು ಭೌತ ದ್ರವ್ಯವಾಗುವೆವು. ಜೀವನವು\break ಸ್ವಾತಂತ್ರ್ಯವನ್ನು ಮತ್ತೆ ಮತ್ತೆ ಅದ್ಭುತವಾಗಿ ವ್ಯಕ್ತಗೊಳಿಸುತ್ತಿದೆ. ನಿಯಮಬಾಹುಳ್ಯ\break ಎಂದರೆ ಮೃತ್ಯು. ಹಿಂದೂಗಳ ಸಮಾಜದಲ್ಲಿ ಇರುವಷ್ಟು ನಿಯಮಾವಳಿಗಳು ಯಾವ\break ರಾಷ್ಟ್ರದಲ್ಲಿಯೂ ಇಲ್ಲ. ಆದಕಾರಣವೇ ಇಡೀ ಸಮಾಜ ಮೃತ್ಯುಮುಖವಾಗಿದೆ. ಆದರೆ ಹಿಂದೂಗಳಲ್ಲಿ ಮತ್ತೊಂದು ವಿಶೇಷವಿತ್ತು. ಅವರು ಧಾರ್ಮಿಕ ಪ್ರಪಂಚದಲ್ಲಿ ಸಿದ್ಧಾಂತಗಳನ್ನು ರಚಿಸಲಿಲ್ಲ. ಆದಕಾರಣವೇ ಧರ್ಮವು ಅದ್ಭುತವಾಗಿ ಬೆಳೆಯಿತು. ಅಲ್ಲಿ ನಾವು\break ವ್ಯವಹಾರ ಚತುರರು. ಆದರೆ ನೀವು ಅಲ್ಲಿ ಕಾರ್ಯಪಟುಗಳಲ್ಲ.

ಅಮೆರಿಕಾದಲ್ಲಿ ಕೆಲವು ಜನರು ಒಟ್ಟಿಗೆ ಸೇರಿ ಒಂದು ಜಾಯಿಂಟ್​ ಸ್ಟಾಕ್​ ಕಂಪನಿ ಮಾಡೋಣ ಎಂದು ನಿರ್ಧರಿಸುವರು. ಐದು ನಿಮಿಷದಲ್ಲಿ ಇದು ನೆರವೇರುವುದು. ಆದರೆ ಇಂಡಿಯಾ ದೇಶದಲ್ಲಿ ದಿನಗಟ್ಟಲೆ ಇಪ್ಪತ್ತು ಜನರು ಚರ್ಚಿಸುತ್ತಾ ಹೋಗಬಹುದು. ಆದರೆ ಕಂಪನಿ ಕಾರ್ಯಗತವಾಗುವುದಿಲ್ಲ. ಆದರೆ ಕೈಯನ್ನು ಮೇಲಕ್ಕೆ ಎತ್ತಿಕೊಂಡು ನಲವತ್ತು ವರುಷಗಳವರೆಗೆ ನಿಂತರೆ ಮುಕ್ತಿ ಸಿಗುವುದು ಎಂದು ಯಾರಾದರೂ ಸಾರಿದರೆ ಅದನ್ನು ಮಾಡಲು ಹಲವರು ಸಿದ್ಧರಾಗಿರುವರು. ಹೀಗೆ ನಾವು ನಮ್ಮ ಕ್ಷೇತ್ರದಲ್ಲಿ ಕಾರ್ಯಪಟುಗಳು, ನೀವು ನಿಮ್ಮ ಕ್ಷೇತ್ರದಲ್ಲಿ ಕಾರ್ಯಪಟುಗಳು.

ಸಾಕ್ಷಾತ್ಕಾರಕ್ಕೆ ಇರುವ ಏಕಮಾತ್ರ ಪಥವೇ ಪ್ರೇಮ. ಒಬ್ಬನು ದೇವರನ್ನು ಪ್ರೀತಿಸಿದರೆ ಇಡೀ ಪ್ರಪಂಚ ಅವನಿಗೆ ಪ್ರಿಯವಾಗುವುದು. ಏಕೆಂದರೆ ಇದೆಲ್ಲ ಭಗವಂತನಿಗೆ ಸೇರಿರುವುದು. ಎಲ್ಲವೂ ಅವನದು. ಅವನೇ ನನ್ನ ಪ್ರಿಯತಮ, ನಾನು ಅವನನ್ನು ಪ್ರೀತಿಸುತ್ತೇನೆ ಎನ್ನುವನು ಭಕ್ತ. ಎಲ್ಲವೂ ಭಗವಂತನಿಗೆ ಸೇರಿರುವುದರಿಂದ ಭಕ್ತನಿಗೆ ಪ್ರಪಂಚದಲ್ಲಿರುವು\-ದೆಲ್ಲ ಪವಿತ್ರವಾಗುವುದು. ಅವನು ಆಗ ಮತ್ತೊಬ್ಬನನ್ನು ಹೇಗೆ ಹಿಂಸಿಸಬಲ್ಲ? ಅವನು\break ಆಗ ಮತ್ತೊಬ್ಬನನ್ನು ಹೇಗೆ ಪ್ರೀತಿಸದೆ ಇರಬಲ್ಲ? ಭಗವಂತನ ಪ್ರೇಮವೊಂದು ನಮ್ಮ ಹೃದಯದಲ್ಲಿ ಉದಯಿಸಿದರೆ ಕ್ರಮೇಣ ಅದರ ಪರಿಣಾಮವಾಗಿ ಪ್ರಪಂಚದಲ್ಲಿ ಎಲ್ಲರನ್ನೂ ಪ್ರೀತಿಸುತ್ತೇವೆ. ನಾವು ಭಗವಂತನನ್ನು ಸಮೀಪಿಸಿದಷ್ಟೂ ಎಲ್ಲವೂ ಅವನಲ್ಲಿದೆ ಎಂಬುದನ್ನು\break ಅನುಭವಿಸುವೆವು. ನಮ್ಮ ಹೃದಯವು ಪ್ರೇಮದ ನಿತ್ಯಚಿಲುಮೆಯಾಗುವುದು. ಈ\break ಪ್ರೇಮಜ್ಯೋತಿಯ ಎದುರಿಗೆ ಮಾನವ ರೂಪಾಂತರ ಹೊಂದುವನು. ಕೊನೆಗೆ ಪ್ರೀತಿ, ಪ್ರಿಯತಮ, ಪ್ರಿಯತಮೆ ಎಲ್ಲವೂ ಒಂದೇ ಎಂಬ ಸುಂದರವಾದ ಸ್ಫೂರ್ತಿದಾಯಕ ಸತ್ಯವನ್ನು ಅರಿಯುವನು.

