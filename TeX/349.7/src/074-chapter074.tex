
\chapter[ಭಾಷೆ ]{ಭಾಷೆ \protect\footnote{\engfoot{C.W. Vol. V, p.259}}}

ಸರಳ ಶೈಲಿಯೇ ರಹಸ್ಯ. ನನ್ನ ಗುರುದೇವನ ಭಾಷೆಯೇ ನನ್ನ ಆದರ್ಶ ಭಾಷೆ. ಅದು ಬಹಳ ಗ್ರಾಮ್ಯವಾದ, ಆದರೂ ಅತ್ಯಂತ ಭಾವಪ್ರದರ್ಶಕವಾದ ಭಾಷೆಯಾಗಿತ್ತು. ನಾವು ಯಾವ ಭಾವನೆಯನ್ನು ವ್ಯಕ್ತಗೊಳಿಸಬೇಕೆಂದು ಇರುವೆವೊ ಭಾಷೆಯು ಅದನ್ನು ಸೂಚಿಸುವಂತಹದಿರಬೇಕು.

ಇಷ್ಟು ಬೇಗ ಬಂಗಾಳಿ ಭಾಷೆಯನ್ನು ಪರಿಪೂರ್ಣಮಾಡಲು ಪ್ರಯತ್ನ ಪಟ್ಟರೆ ಅದು ನೀರಸವಾಗುವುದು. ಅದರಲ್ಲಿ ಕ್ರಿಯಾಪದಗಳು ಇಲ್ಲ ಎನ್ನಬಹುದು; ಮೈಕೇಲ್​ ಮಧುಸೂದನ ದತ್ತನು ಕಾವ್ಯದಲ್ಲಿ ಇದನ್ನು ಸರಿಪಡಿಸಲು ಯತ್ನಿಸಿದನು. ಬಂಗಾಳ ದೇಶದ ಶ್ರೇಷ್ಠ ಕವಿಯೆಂದರೆ ಕವಿ ಕಂಕಣ. ಸಂಸ್ಕೃತದಲ್ಲಿರುವ ಶ್ರೇಷ್ಠಗದ್ಯವೇ ಪತಂಜಲಿಯ ಮಹಾಭಾಷ್ಯ. ಅಲ್ಲಿ ಭಾಷೆ ಓಜಸ್ಸಿನಿಂದ ಕೂಡಿದೆ. ಹಿತೋಪದೇಶದ ಭಾಷೆಯೂ ಅಷ್ಟೇನೂ ಕೆಟ್ಟಿಲ್ಲ. ಆದರೆ ಕಾದಂಬರಿಯ ಭಾಷೆ ಅನವತಿಯ ಚಿಹ್ನೆ.

ಬಂಗಾಳಿ ಭಾಷೆಯನ್ನು ಸಂಸ್ಕೃತದ ಮೇಲ್ಪಂಕ್ತಿಯನ್ನು ಅನುಸರಿಸಿ, ರೂಢಿಸ ಬಾರದು. ಪಾಳೀಭಾಷೆಯನ್ನು ಅನುಸರಿಸಬೇಕು. ಇವೆರಡಕ್ಕೂ ಸಾಮ್ಯ ಹೆಚ್ಚು. ಬಂಗಾಳಿಯಲ್ಲಿ ತಾಂತ್ರಿಕ ಶಬ್ದಗಳನ್ನು ಭಾಷಾಂತರಿಸುವಾಗ ಆಗಲಿ ಅಥವಾ ಸೃಷ್ಟಿಸುವಾಗ ಆಗಲಿ ಸಂಸ್ಕೃತ ಪದಗಳನ್ನು ತೆಗೆದುಕೊಳ್ಳಬೇಕು. ಹೊಸ ಶಬ್ದಗಳನ್ನು ಸೃಷ್ಟಿಸಲು ಪ್ರಯತ್ನಿಸಬೇಕು. ಇದಕ್ಕಾಗಿ ಸಂಸ್ಕೃತ ಶಬ್ದಕೋಶದಿಂದ ತಾಂತ್ರಿಕ ಪದಗಳ ಒಂದು ಸಂಗ್ರಹ ಮಾಡಿದರೆ ಬಂಗಾಳಿ ಭಾಷೆಗೆ ಇದರಿಂದ ಬಹಳ ಪ್ರಯೋಜನವಾಗುವುದು.

