
\chapter[ನಿಷ್ಕಾಮಕರ್ಮ ]{ನಿಷ್ಕಾಮಕರ್ಮ \protect\footnote{\engfoot{C.W. Vol. V. P 246}}}

ಕಲ್ಕತ್ತಾದಲ್ಲಿ ೧೮೯೮, ಮಾರ್ಚ್​ ೨೦ರಂದು ನಡೆದ ರಾಮಕೃಷ್ಣ ಮಿಶನ್ನಿನ ಸಭೆಯಲ್ಲಿ ಸ್ವಾಮಿ ವಿವೇಕಾನಂದರು ನಿಷ್ಕಾಮಕರ್ಮದ ಬಗ್ಗೆ ಈ ಕೆಳಗಿನಂತೆ ಮಾತನಾಡಿದರು:

ಭಗವದ್ಗೀತೆಯನ್ನು ಮೊದಲು ಬೋಧಿಸಿದಾಗ ಎರಡು ಪಂಥಗಳ ನಡುವೆ ದೊಡ್ಡದೊಂದು ಚರ್ಚೆ ನಡೆಯುತ್ತಿತ್ತು. ಕೆಲವರು ಯಾಗ-ಯಜ್ಞ-ಬಲಿ ಮುಂತಾದುವುಗಳನ್ನೆ ಧರ್ಮ ಎಂದು ಭಾವಿಸಿದ್ದರು. ಮತ್ತೆ ಕೆಲವರು ಲೆಕ್ಕ ವಿಲ್ಲದಷ್ಟು ಸಂಖ್ಯೆಯ ಕುದುರೆಗಳನ್ನು ಮತ್ತು ದನಗಳನ್ನು ಬಲಿಕೊಡುವುದು ಧರ್ಮವಲ್ಲ ಎಂದು ಹೇಳುತ್ತಿದ್ದರು. ಈ ಗುಂಪಿಗೆ ಸೇರಿದವರು ಬಹುಪಾಲು ಮಂದಿ ಸಂನ್ಯಾಸಿಗಳು, ಜ್ಞಾನ ಮಾರ್ಗಾನುಯಾಯಿಗಳು. ಕರ್ಮವನ್ನೆಲ್ಲಾ ತ್ಯಜಿಸಿ ಆತ್ಮ ಸಾಕ್ಷಾತ್ಕಾರವನ್ನು ಪಡೆಯುವುದೇ ಮೋಕ್ಷಕ್ಕೆ ಮಾರ್ಗ ಎಂದು ಅವರು ನಂಬಿದ್ದರು. ಫಲಾಪೇಕ್ಷೆಯಿಲ್ಲದ ಕರ್ಮವನ್ನು ಬೋಧಿಸಿ ಗೀತಾಚಾರ್ಯನು ಎರಡು ವಿರೋಧಿ ಪಂಗಡಗಳ ಚರ್ಚೆಯನ್ನು ಕೊನೆಗಾಣಿಸಿದನು.

ಗೀತೆಯನ್ನು ಮಹಾಭಾರತದ ಕಾಲದಲ್ಲಿ ಬರೆಯಲಿಲ್ಲ, ಅನಂತರ ಅದನ್ನು ಸೇರಿಸಿದರು ಎಂದು ಹಲವರು ಅಭಿಪ್ರಾಯ ಪಡುವರು. ಇದು ಸರಿಯಲ್ಲ. ಗೀತೆಯ ಸಂದೇಶದ ವೈಶಿಷ್ಟ್ಯವನ್ನು ಮಹಾಭಾರತದ ಮೂಲೆಮೂಲೆಯಲ್ಲೂ ನೋಡುತ್ತೇವೆ. ನಾವು ಗೀತೆಯನ್ನು ಮಹಾಭಾರತದಿಂದ ತೆಗೆದುಹಾಕಿದರೆ, ಆ ಸಂದೇಶಕ್ಕೆ ಸಂಬಂಧಪಟ್ಟ ಇತರ ಭಾಗಗಳನ್ನೆಲ್ಲಾ ತೆಗೆದುಹಾಕಬೇಕಾಗುವುದು.

ಫಲಾಪೇಕ್ಷೆಯಿಲ್ಲದೆ ಕೆಲಸ ಮಾಡುವುದೆಂದರೇನು? ಸುಖದುಃಖಗಳಾವುವೂ ಮನಸ್ಸಿಗೆ ಸೋಂಕದ ರೀತಿಯಲ್ಲಿ ಕೆಲಸ ಮಾಡುವುದು ಎಂದು ಹಲವರು ಭಾವಿಸುವರು. ಇದೇ ಅದರ ನಿಜವಾದ ಅರ್ಥವಾದರೆ, ಪ್ರಾಣಿಗಳು ಫಲಾಪೇಕ್ಷೆ ಯಿಲ್ಲದೆ ಕೆಲಸ ಮಾಡುತ್ತವೆ ಎಂದು ಹೇಳಬಹುದು. ಕೆಲವು ಪ್ರಾಣಿಗಳು ತಮ್ಮ ಮರಿಯನ್ನೇ ತಿನ್ನುವುವು. ಇದರಿಂದ ಅವಕ್ಕೆ ಸ್ವಲ್ಪವೂ ವ್ಯಥೆಯಾಗುವುದಿಲ್ಲ. ಕಳ್ಳರು ಇತರರ ಆಸ್ತಿಯನ್ನು ಅಪಹರಿಸಿ ಅವರನ್ನು ನಾಶಮಾಡುವರು. ಅವರು ಹಾಗೆ ಮಾಡುವಾಗ ಅವರಿಗೆ ಯಾವ ದಯಾದಾಕ್ಷಿಣ್ಯಗಳೂ ಇಲ್ಲದೇ ಇದ್ದರೆ ಅವರೂ ಫಲಾಪೇಕ್ಷೆಯಿಲ್ಲದೆ ಕೆಲಸ ಮಾಡಿದಂತೆ ಆಗುವುದು. ಇದೇ ಅದಕ್ಕೆ ಅರ್ಥವಾದರೆ ಅತಿ ನಿರ್ದಯನಾದ ಘೋರ ಕೊಲೆಪಾತಕಿಯು ಫಲಾಪೇಕ್ಷೆ ಯಿಲ್ಲದೆ ಕೆಲಸ ಮಾಡುತ್ತಿರುವನೆಂದು ಭಾವಿಸಬಹುದು. ಗೋಡೆಗೆ ಸುಖದುಃಖಗಳ ಅರಿವಿಲ್ಲ. ಹಾಗೆಯೇ ಒಂದು ಕಲ್ಲಿಗೂ ಅಷ್ಟೆ. ಅವುಗಳು ಫಲಾಪೇಕ್ಷೆಯಿಲ್ಲದೆ ಕೆಲಸ ಮಾಡುತ್ತಿವೆ ಎನ್ನಲಾಗುವುದಿಲ್ಲ. ಈ ದೃಷ್ಟಿಯಲ್ಲಿ ದುರ್ಜನರಿಗೆ ಇದೊಂದು ದೊಡ್ಡ ಅಸ್ತ್ರವಾಗುವುದು. ಅವರು ದೌರ್ಜನ್ಯಕೃತ್ಯಗಳನ್ನು ಮಾಡುತ್ತಾ ನಮಗೆ ಯಾವ ಒಂದು ಆಸಕ್ತಿಯೂ ಇಲ್ಲ ಎನ್ನಬಹುದು. ಅನಾಸಕ್ತಿಗೆ ಇದು ಅರ್ಥವಾದರೆ ಗೀತೆಯನ್ನು ಬೋಧಿಸಿ ದೊಡ್ಡದೊಂದು ಅನರ್ಥಕಾರಿಯಾದ ಭಾವನೆಯನ್ನು ಬಿತ್ತಿದಂತಾಗುವುದು. ನಿಜವಾಗಿ ಇದಲ್ಲ ಅರ್ಥ. ಇದೂ ಅಲ್ಲದೆ ಗೀತೆಯ ಬೋಧನೆಗೆ ಸಂಬಂಧಿಸಿದಂತೆ ಜೀವನ ನಡೆಸಿದ ಮಹಾತ್ಮರನ್ನು ನೋಡಿದರೆ ಅವರು ಬೇರೆ ವಿಧವಾಗಿ ಜೀವನ ನಡೆಸುತ್ತಿದ್ದುದು ಕಾಣಿಸುವುದು. ಅರ್ಜುನನು ದ್ರೋಣನನ್ನು ಮತ್ತು ಭೀಷ್ಮಾಚಾರ್ಯರನ್ನು ಕೊಂದನು. ಆದರೆ ಇದೆಲ್ಲಕ್ಕಿಂತ ಹೆಚ್ಚಾಗಿ ತನ್ನ ಆಸೆ ಆಕಾಂಕ್ಷೆ, ತನ್ನ ಸ್ವಾರ್ಥ, ತನ್ನ ಕೀಳು ಸ್ವಭಾವ ಇವನ್ನು ಕೋಟಿ ಪಾಲು ಹೆಚ್ಚಾಗಿ ಬಲಿಕೊಟ್ಟನು.

ಗೀತೆ ಕರ್ಮಯೋಗವನ್ನು ಬೋಧಿಸುವುದು. ನಾವು ಯೋಗದ ಮೂಲಕ (ಏಕಾಗ್ರತೆಯಿಂದ) ಕೆಲಸ ಮಾಡಬೇಕು. ಕರ್ಮ ಮಾಡುವಾಗ ಇರುವ ಇಂತಹ ಏಕಾಗ್ರತೆಯಲ್ಲಿ ಕೀಳು ಅಹಂಕಾರದ ಭಾವ ಇರುವುದಿಲ್ಲ. ಯೋಗದ ದೃಷ್ಟಿ ಯಿಂದ ಕೆಲಸ ಮಾಡುವಾಗ, ನಾನು ಈ ಕೆಲಸ ಮಾಡುತ್ತಿರುವೆ ಎಂಬ ಭಾವನೆಯೇ ಇರುವುದಿಲ್ಲ. ಪಾಶ್ಚಾತ್ಯರಿಗೆ ಇದು ಅರ್ಥವಾಗುವುದಿಲ್ಲ. ಅಹಂಕಾರ ವಿಲ್ಲದೇ ಇದ್ದರೆ, ನಾನೆಂಬುದೇ ಇಲ್ಲದೆ ಇದ್ದರೆ, ಒಬ್ಬ ಹೇಗೆ ಕೆಲಸ ಮಾಡಬಲ್ಲ ಎನ್ನುವರು ಅವರು. ಆದರೆ ಒಬ್ಬ ಏಕಾಗ್ರತೆಯಿಂದ ಅಹಂಕಾರವನ್ನು ಮರೆತು ಕೆಲಸ ಮಾಡಿದಾಗ ಅದು ಅತ್ಯುತ್ತಮವಾಗುವುದು. ಇದನ್ನು ಪ್ರತಿಯೊಬ್ಬರೂ ತಮ್ಮ ಜೀವನದಲ್ಲಿ ಅನುಭವಿಸಿರಬಹುದು. ನಾವು ಅರಿವಿಲ್ಲದೆ ಹಲವು ಕೆಲಸ ಗಳನ್ನು ಮಾಡುತ್ತೇವೆ. ಜೀರ್ಣಿಸಿಕೊಳ್ಳುವುದು ಇತ್ಯಾದಿಗಳು ಅಂಥವು. ಕೆಲವನ್ನು ಅರಿತು ಮಾಡುವೆವು. ಮತ್ತೆ ಕೆಲವನ್ನು ಸಮಾಧಿಯಲ್ಲಿ ಮಗ್ನರಾಗಿರುವಾಗ ಮಾಡುವೆವು. ಅಲ್ಲಿ ಅಲ್ಪಾತ್ಮನ ಭಾವನೆ ಇರುವುದಿಲ್ಲ. ಒಬ್ಬ ಚಿತ್ರಕಾರ ತನ್ನನ್ನು ತಾನೆ ಮರೆತು ತಾನು ಬರೆಯುತ್ತಿರುವ ಚಿತ್ರದಲ್ಲಿ ತನ್ಮಯನಾದರೆ ಅವನು ಅದ್ಭುತವಾದ ಚಿತ್ರಗಳನ್ನು ಬರೆಯಬಲ್ಲ. ಒಳ್ಳೆಯ ಅಡಿಗೆ ಮಾಡುವವನು, ಅಡಿಗೆಯ ಸಾಮಗ್ರಿಗಳ ಮೇಲೆ ತನ್ನ ಮನಸ್ಸನ್ನೆಲ್ಲಾ ಇಡುವನು. ಸದ್ಯಕ್ಕೆ ಅವನು ಮಿಕ್ಕಿರುವುದನ್ನೆಲ್ಲಾ ಮರೆಯುವನು. ಆದರೆ ಜನರು ತಮಗೆ ಯಾವ ಕೆಲಸವನ್ನು ಮಾಡಿ ಅಭ್ಯಾಸವಿದೆಯೊ ಅಂತಹ ಒಂದು ಕೆಲಸವನ್ನು ಮಾತ್ರ ಅವರು ಚೆನ್ನಾಗಿ ಮಾಡಬಲ್ಲರು. ಆದರೆ ಎಲ್ಲಾ ಕೆಲಸವನ್ನೂ ಹೀಗೆ ಮಾಡಬೇಕೆಂದು ಗೀತೆ ಬೋಧಿಸುವುದು. ಯಾರು ಯೋಗದ ಮೂಲಕ ಭಗವಂತನಲ್ಲಿ ಏಕವಾಗಿರು ವರೊ ಅವರು ಪ್ರತಿಯೊಂದು ಕೆಲಸವನ್ನೂ ಏಕಾಗ್ರತೆಯಿಂದ ಮಾಡುವರು; ಅವರು ಮತ್ತಾವ ವೈಯಕ್ತಿಕ ಫಲವನ್ನೂ ಅಪೇಕ್ಷಿಸುವುದಿಲ್ಲ. ಹೀಗೆ ಕರ್ಮವನ್ನು ಮಾಡಿದರೆ ಇದರಿಂದ ಪ್ರಪಂಚಕ್ಕೆ ಶುಭ ಮಾತ್ರ ಆಗುವುದು; ಇದರಿಂದ ಎಂದಿಗೂ ಅಶುಭವಾಗುವುದಿಲ್ಲ. ಯಾರು ಹೀಗೆ ಕೆಲಸ ಮಾಡುವರೊ ಅವರು ಎಂದಿಗೂ ಸ್ವಾರ್ಥಕ್ಕಾಗಿ ಏನನ್ನೂ ಮಾಡಿಕೊಳ್ಳುವುದಿಲ್ಲ.

ಪ್ರತಿಯೊಂದು ಕರ್ಮದಲ್ಲಿಯೂ ಒಳ್ಳೆಯದು ಕೆಟ್ಟದ್ದು ಮಿಶ್ರವಾಗಿವೆ. ಸ್ವಲ್ಪವೂ ಕೆಟ್ಟದ್ದು ಇಲ್ಲದ ಒಳ್ಳೆಯ ಕರ್ಮವೇ ಇಲ್ಲ. ಬೆಂಕಿಯ ಸುತ್ತಲೂ ಇರುವ ಹೊಗೆಯಂತೆ. ಸ್ವಲ್ಪ ಮಾತ್ರ ಕೆಟ್ಟದ್ದಾವುದೋ ಅಂತಹ ಕೆಲಸದಲ್ಲಿ ನಾವು ನಿರತ ರಾಗಬೇಕು. ಅರ್ಜುನನು ಬೀಷ್ಮರನ್ನು ಮತ್ತು ದ್ರೋಣರನ್ನು ಕೊಂದನು. ಹೀಗೆ ಮಾಡುವುದಕ್ಕೆ ಆಗದೆ ಇದ್ದಿದ್ದರೆ ದುರ್ಯೋಧನನ್ನು ಜಯಿಸುವುದಕ್ಕೆ ಆಗುತ್ತಿರಲಿಲ್ಲ. ಆಗ ದೇಶದಲ್ಲಿ ಒಂದು ದೊಡ್ಡ ಅನಾಯಕತೆ ಇರುತ್ತಿತ್ತು. ದೇಶದ ಆಡಳಿತವನ್ನು ಮದಾಂಧರಾದ ಅಧರ್ಮಿಗಳು ವಶಪಡಿಸಿಕೊಂಡು ಪ್ರಜೆಗಳನ್ನು ಪೀಡಿಸುತ್ತಿದ್ದರು. ಇದರಂತೆಯೇ ಶ‍್ರೀಕೃಷ್ಣನು ದುರಾತ್ಮರಾದ ಕಂಸ, ಜರಾಸಂಧ ಮುಂತಾದವರನ್ನು ಕೊಂದನು. ಆದರೆ ಅವನು ಯಾವುದನ್ನೂ ತನ್ನ ಸ್ವಾರ್ಥಕ್ಕಾಗಿ ಮಾಡಲಿಲ್ಲ. ಮಾಡಿದ ಪ್ರತಿಯೊಂದು ಕೆಲಸವೂ ಇತರರ ಹಿತಕ್ಕಾಗಿ. ನಾವು ಗೀತೆಯನ್ನು ಮೋಂಬತ್ತಿಯ ದೀಪದಲ್ಲಿ ಓದುತ್ತಿರುವೆವು. ಆದರೆ ಹಲವು ಕ್ರಿಮಿಗಳು ಆ ಬೆಳಕಿನಲ್ಲಿ ಬಿದ್ದು ಸಾಯುತ್ತಿವೆ. ಆದಕಾರಣ ಕರ್ಮಕ್ಕೆ ಸ್ವಲ್ಪ ದೋಷ ಯಾವಾಗಲೂ ಅಂಟಿಕೊಂಡಿರುವುದು. ತಮ್ಮ ಕೀಳು ಅಹಂಕಾರದಿಂದ ಪ್ರೇರಿತ ರಾಗದೆ ಯಾರು ಕರ್ಮ ಮಾಡುವರೊ ಅವರು ದೋಷಕ್ಕೆ ಪಾತ್ರರಾಗುವುದಿಲ್ಲ. ಏಕೆಂದರೆ ಅವರು ಪ್ರಪಂಚದ ಹಿತಕ್ಕಾಗಿ ಕೆಲಸ ಮಾಡುವರು. ಫಲಾಪೇಕ್ಷೆ ಯಿಲ್ಲದೆ ಕೆಲಸ ಮಾಡುವುದರಿಂದ ಅನಾಸಕ್ತಿಯಿಂದ ಕೆಲಸ ಮಾಡುವುದರಿಂದ ಶ್ರೇಷ್ಠ ಆನಂದ ಲಭಿಸುವುದು, ಮುಕ್ತಿ ಲಭಿಸುವುದು. ಕರ್ಮಯೋಗದ ರಹಸ್ಯವನ್ನು ಭಗವಾನ್​ ಶ‍್ರೀಕೃಷ್ಣನು ಈ ಗೀತೆಯಲ್ಲಿ ಬೋಧಿಸಿರುವನು.

