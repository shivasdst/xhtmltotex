

\part{ವಾರ್ತ ಪ್ರತಿನಿಧಿಗಳೊಡನೆ ಭೇಟಿಗಳು}

\chapter[ಲಂಡನ್ನಿನಲ್ಲಿ ಇಂಡಿಯಾ ದೇಶದ ಯೋಗಿ ]{ಲಂಡನ್ನಿನಲ್ಲಿ ಇಂಡಿಯಾ ದೇಶದ ಯೋಗಿ \protect\footnote{\engfoot{C.W. Vol. V, P. 185}}}

\centerline{\textbf{ವೆಸ್ಟ್ ಮಿನ್ಸ್ಟಿರ್​ ಗೆಜೆಟ್​ ೨೩ ಅಕ್ಟೋಬರ್​ ೧೮೯೫}}

ಇತ್ತೀಚೆಗೆ ಇಂಡಿಯಾ ದೇಶದ ತತ್ತ್ವಶಾಸ್ತ್ರ ಅನೇಕರ ಮೇಲೆ ಬಹಳ ಪರಿಣಾಮ ಕಾರಿಯಾದ ಪ್ರಭಾವವನ್ನು ಬೀರುತ್ತಿದೆ. ಇದುವರೆಗೆ ಭಾರತೀಯ ತತ್ತ್ವ ಶಾಸ್ತ್ರವನ್ನು ಇಲ್ಲಿ ಸಾರಿದವರೆಲ್ಲಾ ತಮ್ಮ ಭಾವನೆಯಲ್ಲಿ ಮತ್ತು ತರಬೇತಿಯಲ್ಲಿ ಪಾಶ್ಚಾತ್ಯರೇ ಆಗಿದ್ದರು. ಆದಕಾರಣವೇ ವೇದಾಂತ ತತ್ತ್ವದ ಗಹನ ಸತ್ಯಗಳಲ್ಲಿ ಎಲ್ಲೋ ಸ್ವಲ್ಪ ಭಾಗ ಮಾತ್ರ ಜನಗಳಿಗೆ ಗೊತ್ತಿತ್ತು. ಆ ಸ್ವಲ್ಪವೂ ಎಲ್ಲೋ ಕೆಲವು ಮಂದಿಗಳಿಗೆ ಮಾತ್ರ. ಕೇವಲ ಭಾಷಾಶಾಸ್ತ್ರದ ದೃಷ್ಟಿಯಿಂದ ಮಾಡಿರುವ ದೊಡ್ಡ ದೊಡ್ಡ ಭಾಷಾಂತರಗಳನ್ನು ತಿಳಿದುಕೊಳ್ಳಲು ಧೈರ್ಯವಾಗಲೀ ಉತ್ಸಾಹವಾಗಲಿ ಅನೇಕರಿಗೆ ಇಲ್ಲ. ಪೌರಸ್ತ್ಯ ಸಂಸ್ಕೃತಿಯ ಪರಂಪರೆಯಲ್ಲಿ ಬೆಳೆದ ಯೋಗ್ಯನಾದ ಜ್ಞಾನಿಗೆ ಮಾತ್ರ ವೇದಾಂತದಲ್ಲಿರುವ ಪರಮ ಸತ್ಯಗಳು ಕಾಣುವುವು.

ಆಸಕ್ತಿಯಿಂದಲೂ ಮತ್ತು ಸ್ವಲ್ಪ ಕುತೂಹಲದಿಂದಲೂ ಪ್ರೇರಿತನಾಗಿ, ಪಾಶ್ಚಾತ್ಯರಿಗೆ ಅಪರಿಚಿತನಾದ ಒಬ್ಬ ಜ್ಞಾನಿಯನ್ನು ನೋಡಲು ಹೋದೆ ಎಂದು ಒಬ್ಬ ಪತ್ರಿಕೆಯ ಬಾತ್ಮೀದಾರರು ಬರೆಯುತ್ತಾರೆ. ಆ ಜ್ಞಾನಿಗಳೇ ಇಂಡಿಯಾ ದೇಶದ ನಿಜವಾದ ಯೋಗಿಗಳಾದ ಸ್ವಾಮಿ ವಿವೇಕಾನಂದರು. ಪುರಾತನಕಾದ ಮಹರ್ಷಿಗಳಿಂದ ವಂಶಾನುಗತವಾಗಿ ಬಂದ ಈ ಜ್ಞಾನವನ್ನು ಪಾಶ್ಚಾತ್ಯರಿಗೆ ಕೊಡುವುದಕ್ಕಾಗಿ ಧೈರ್ಯದಿಂದ ಪಾಶ್ಚಾತ್ಯ ದೇಶಗಳಿಗೆ ಅವರು ಬಂದಿರುವರು. ಈ ಉದ್ದೇಶ ಸಾಧನೆಗಾಗಿಯೇ ಕಳೆದ ರಾತ್ರಿ ಪ್ರಿನ್ಸೆಸ್​ ಹಾಲಿನಲ್ಲಿ ಒಂದು ಉಪನ್ಯಾಸವನ್ನು ಕೊಟ್ಟರು.

ಸ್ವಾಮಿ ವಿವೇಕಾನಂದರು ಆಕರ್ಷಣೀಯ ಗಂಭೀರ ವ್ಯಕ್ತಿ. ತಲೆಗೆ ಒಂದು ರುಮಾಲನ್ನು ಸುತ್ತಿರುವರು, ಸುಪ್ರಸನ್ನರು, ದಯಾರ್ದ್ರ ಹೃದಯರು.

“ನಿಮ್ಮ ಹೆಸರಿಗೆ ಏನಾದರೂ ಅರ್ಥವಿದೆಯೆ?” ಎಂದು ಕೇಳಿದ್ದಕ್ಕೆ ಸ್ವಾಮೀಜಿ ಹೀಗೆ ಹೇಳಿದರು: “ಈಗ ನನ್ನನ್ನು ಜನ ಕರೆಯುವುದು ಸ್ವಾಮಿ ವಿವೇಕಾನಂದ ಎಂದು. ಇದರಲ್ಲಿ ಸ್ವಾಮಿ ಎಂದರೆ ಪ್ರಪಂಚವನ್ನು ತ್ಯಜಿಸಿದ ತ್ಯಾಗಿ ಎಂದು ಅರ್ಥ. ಎರಡನೆಯದು ನಾನಾಗಿ ಇಟ್ಟುಕೊಂಡ ಹೆಸರು. ಮನೆಯನ್ನು ಬಿಟ್ಟು ಮೇಲೆ ಸಂನ್ಯಾಸಿಗಳು ಬೇರೊಂದು ಹೆಸರನ್ನು ಇಟ್ಟುಕೊಳ್ಳುವರು. ಇದರ ಅರ್ಥ; ವಿವೇಕದ ಆನಂದ.”

“ಸ್ವಾಮಿ, ನೀವು ಸಾಮಾನ್ಯ ಲೌಕಿಕ ಮಾರ್ಗವನ್ನು ತ್ಯಜಿಸುವುದಕ್ಕೆ ಯಾವುದು ಪ್ರೇರಣೆ?” ಎಂದು ನಾನು ಕೇಳಿದೆ.

ಸ್ವಾಮೀಜಿ: “ಬಾಲ್ಯದಿಂದಲೂ ನನಗೆ ಧರ್ಮ, ತತ್ತ್ವ ಎಂದರೆ ಬಹಳ ಇಷ್ಟ. ಶಾಸ್ತ್ರಗಳು ತ್ಯಾಗವೇ ಪರಮಶ್ರೇಷ್ಠ ಎಂದು ಸಾರುತ್ತವೆ. ತ್ಯಾಗಜೀವನವನ್ನು ಅನುಸರಿಸುವಂತೆ ಪ್ರೇರೇಪಿಸುವುದಕ್ಕೆ ಒಂದು ಕಿಡಿ ಮಾತ್ರ ಬೇಕಾಗಿತ್ತು. ಅದು ಸ್ವಯಂ ತ್ಯಾಗಜೀವನವನ್ನು ಆರಿಸಿಕೊಂಡಿದ್ದ ಶ‍್ರೀರಾಮಕೃಷ್ಣ ಪರಮಹಂಸರಿಂದ ಬಂದಿತು. ಅವರೂ ಕೂಡ ತ್ಯಾಗಿಗಳಾಗಿದ್ದರು. ಅವರಲ್ಲಿ ನನ್ನ ಜೀವನದ ಪರಮ ಧ್ಯೇಯ ರೂಪುಗೊಂಡಿರುವುದನ್ನು ನೋಡಿದೆ.”

ಬಾತ್ಮೀದಾರ: “ಹಾಗಾದರೆ ನೀವು ಈಗ ಯಾವುದರ ಪ್ರತಿನಿಧಿಯಾಗಿ ಬಂದಿರುವಿರೋ ಆ ಪಂಥವನ್ನು ಅವರು ಸ್ಥಾಪಿಸಿದರೆ?”

ಸ್ವಾಮೀಜಿ: “ಇಲ್ಲ. ಅವರ ಇಡೀ ಬಾಳು ಮತ ಮತ್ತು ಕೋಮುಗಳಿಗೆ ಹಾಕಿರುವ ಬೇಲಿಯನ್ನು ತೆಗೆಯುವುದಕ್ಕೆ ಮೀಸಲಾಗಿತ್ತು. ಅವರು ಯಾವ ಪಂಥ ವನ್ನೂ ಸ್ಥಾಪನೆ ಮಾಡಲಿಲ್ಲ. ಅದಕ್ಕೆ ಬದಲಾಗಿ ಎಲ್ಲರಿಗೂ ಭಾವನಾ ಸ್ವಾತಂತ್ರ್ಯ ವನ್ನು ಕೊಡಲು ಹೋರಾಡಿದರು. ಅವರು ಮಹಾಯೋಗಿಗಳು.”

ಬಾತ್ಮೀದಾರ: “ಹಾಗಾದರೆ ನೀವು ಈ ದೇಶದಲ್ಲಿ ಥಿಯಸಾಫಿಕಲ್​ ಸೊಸೈಟಿ, ಕ್ರಿಶ್ಚಿಯನ್​ ಸೈನ್​ಟಿಸ್ಟ್​ ಮುಂತಾದ ಯಾವುದಕ್ಕೂ ಸೇರಿಲ್ಲವೆ?”

ಸ್ವಾಮೀಜಿ: “ಇಲ್ಲ (ಮಗುವಿನಂತೆ ಅವರ ಮೊಗ ಬೆಳಗಿತು. ಸರಳವಾಗಿ, ನೇರವಾಗಿ, ಸತ್ಯವಾಗಿತ್ತು ಅವರಿತ್ತ ಉತ್ತರ). ನಾನು ಹೇಳುವುದು ನಮ್ಮ ಶಾಸ್ತ್ರ ಗಳಿಗೆ ಕೊಡುವ ನನ್ನ ಸ್ವಂತ ವ್ಯಾಖ್ಯಾನವಷ್ಟೆ. ನನ್ನ ಗುರುದೇವರು ತೋರಿದ ಬೆಳಕಿನಲ್ಲಿ ಅದನ್ನೇ ವಿವರಿಸುತ್ತಿರುವೆನು. ನನಗೆ ಯಾವ ಅಲೌಕಿಕ ಅಧಿಕಾರವೂ ಇದೆ ಎಂದು ಹೇಳಿಕೊಳ್ಳುವುದಿಲ್ಲ. ಮೇಧಾವಿಗಳು ನನ್ನ ಅಭಿಪ್ರಾಯವನ್ನು ಒಪ್ಪಿ ಕೊಂಡರೆ, ತಮ್ಮ ಜೀವನದಲ್ಲಿ ಅದನ್ನು ಬಳಸಿಕೊಂಡರೆ ಸಾಕು, ನಾನು ಪಟ್ಟ ಪ್ರಯತ್ನ ಸಾರ್ಥಕವಾಯಿತೆಂದು ಭಾವಿಸುವೆನು. ಎಲ್ಲಾ ಧರ್ಮಗಳಲ್ಲಿಯೂ ಜ್ಞಾನ, ಭಕ್ತಿಯೋಗಗಳು ಅನುಷ್ಠಾನ ಮಾಡಬಹುದಾದ ರೀತಿಯಲ್ಲಿ ಇದ್ದೇ ಇರುತ್ತವೆ. ವೇದಾಂತ ತತ್ತ್ವವು ಗಹನವಾದ ವಿಜ್ಞಾನ ಶಾಸ್ತ್ರ. ಅದು ಮೇಲಿನ ಮಾರ್ಗಗಳ ನ್ನೆಲ್ಲಾ ಒಳಗೊಂಡಿರುವುದು. ನಾನು ಬೋಧಿಸುವುದು ಇದನ್ನೇ. ಪ್ರತಿಯೊಬ್ಬನೂ ತನ್ನ ತನ್ನ ಧರ್ಮಕ್ಕೆ ಇದನ್ನು ಅನ್ವಯಿಸಿಕೊಳ್ಳಬಹುದು. ನಾನು ಪ್ರತಿಯೊಬ್ಬನ ಅನು ಭವಕ್ಕೂ ನಿಲುಕುವಂತಹ ವಿಷಯಗಳನ್ನು ಹೇಳುತ್ತೇನೆ. ಎಲ್ಲಿ ಪುಸ್ತಕದಿಂದ ತೆಗೆದು ವಿಷಯಗಳನ್ನು ಹೇಳುತ್ತೇನೆಯೋ ಆ ಪುಸ್ತಕಗಳಲ್ಲಿ ನೀವು ನೋಡಿದರೆ ಆ ವಿಷಯ ಗಳು ನಿಮಗೆ ದೊರಕುತ್ತವೆ. ಯಾರೋ ಕಣ್ಣಿಗೆ ಕಾಣದ ಮಹಾತ್ಮರು ನನ್ನ ಮೂಲಕ ಏನನ್ನೊ ಹೇಳುತ್ತಾರೆಂದಾಗಲೀ, ಯಾರ ಕಣ್ಣಿಗೂ ಬೀಳದ ಶಾಸ್ತ್ರಗಳಿಂದ ವಿಷಯ ಗಳನ್ನು ತೆಗೆದು ನಾನು ಹೇಳುತ್ತೇನೆ ಎಂದಾಗಲೀ ಭಾವಿಸಬಾರದು. ನಾನು ಯಾವ ರಹಸ್ಯಗಳನ್ನೂ ಜನರಿಗೆ ಬೋಧಿಸುವುದಿಲ್ಲ. ಅಂಥವುಗಳಿಂದ ಯಾವ ಒಳ್ಳೆಯದೂ ಆಗುತ್ತದೆ ಎಂದು ನಾನು ನಂಬುವುದಿಲ್ಲ. ಸತ್ಯವು ತನ್ನದೇ ಸ್ವಂತ ಆಧಾರದ ಮೇಲೆ ನಿಂತಿರುವುದು. ಸತ್ಯವು ಯಾವ ಟೀಕೆಯನ್ನು ಬೇಕಾದರೂ ಎದುರಿಸಬಲ್ಲದು.

ಬಾತ್ಮೀದಾರ: “ಸ್ವಾಮಿ ನೀವು ಯಾವುದಾದರೊಂದು ಸಂಘವನ್ನು ಸ್ಥಾಪನೆ ಮಾಡಬೇಕೆಂದು ಇರುವಿರಾ?”

ಸ್ವಾಮೀಜಿ: “ಇಲ್ಲ, ಯಾವ ಸಮಾಜವನ್ನೂ ನಾನು ಸ್ಥಾಪಿಸಬೇಕೆಂದಿಲ್ಲ. ಎಲ್ಲರಲ್ಲಿಯೂ ಇರುವ, ಎಲ್ಲರಿಗೂ ಅನ್ವಯಿಸುವ ಆತ್ಮವೊಂದನ್ನೇ ನಾನು ಬೋಧಿ ಸುವುದು. ಆತ್ಮನನ್ನು ಅರಿತ, ಅದರ ಬೆಳಕಿನಲ್ಲಿ ಜೀವಿಸುತ್ತಿರುವ ಕೆಲವು ಜನರು ಈಗಲೂ ಕೂಡ ಪ್ರಪಂಚವನ್ನೇ ಜಾಗ್ರತಗೊಳಿಸಬಹುದು. ಹಿಂದೆ ಹೇಗೆ ಇದು ಅಂತಹ ಮಹಾವ್ಯಕ್ತಿಗಳಿಗೆ ಸಾಧ್ಯವಾಗಿತ್ತೋ ಹಾಗೆಯೇ ಇಂದಿಗೂ ಸಾಧ್ಯ.”

ಬಾತ್ಮೀದಾರ: “ನೀವು ಈಗತಾನೆ ಭರತಖಂಡದಿಂದ ಬಂದಿರಾ?” (ಸ್ವಾಮೀಜಿಯವರು ಭರತಖಂಡದ ಬಿಸಿಲಿನಲ್ಲಿ ಬಾಡಿದ್ದಂತೆ ಕಂಡರು.)

ಸ್ವಾಮೀಜಿ: “ಇಲ್ಲ. ನಾನು ಚಿಕಾಗೋ ನಗರದಲ್ಲಿ ೧೮೯೩ರಲ್ಲಿ ನಡೆದ ವಿಶ್ವಧರ್ಮ ಸಮ್ಮೇಳನಕ್ಕೆ ಹಿಂದೂಧರ್ಮದ ಪ್ರತಿನಿಧಿಯಾಗಿ ಹೋಗಿದ್ದೆ. ಅಂದಿನಿಂದ ನಾನು ಅಮೆರಿಕಾ ದೇಶದಲ್ಲಿ ಸಂಚಾರ ಮಾಡುತ್ತಾ ಉಪನ್ಯಾಸಗಳನ್ನು ಕೊಡುತ್ತಿರುವೆ. ಅಮೆರಿಕಾ ದೇಶದ ಜನರು ಬಹಳ ಉತ್ಸಾಹದಿಂದ ಕೇಳುವರು, ಮತ್ತು ಅವರು ಸಹಾನುಭೂತಿಯುಳ್ಳ ಮಿತ್ರರು. ನಾನು ಅಲ್ಲಿ ಮಾಡಿರುವ ಕೆಲಸ ಚೆನ್ನಾಗಿ ಬೇರುಬಿಟ್ಟಿದೆ. ಸದ್ಯದಲ್ಲೇ ಅಲ್ಲಿಗೆ ಹೋಗಬೇಕಾಗಿದೆ.”

ಬಾತ್ಮೀದಾರ: “ಪಾಶ್ಚಾತ್ಯ ಧರ್ಮವನ್ನು ನೀವು ಯಾವ ದೃಷ್ಟಿಯಿಂದ ನೋಡುತ್ತೀರಿ?”

ಸ್ವಾಮೀಜಿ: “ನಾನು ಬೋಧಿಸುವ ತತ್ತ್ವವು ಪ್ರಪಂಚದ ಧರ್ಮಗಳಿಗೆಲ್ಲಾ ತಳಹದಿಯಾಗಬಲ್ಲದು. ನಾನು ಎಲ್ಲಾ ಧರ್ಮಗಳನ್ನೂ ಆತ್ಯಂತ ಸಹಾನುಭೂತಿ ಯಿಂದ ನೋಡುತ್ತೇನೆ. ನಾನು ಯಾವ ಧರ್ಮವನ್ನೂ ವಿರೋಧಿಸುವುದಿಲ್ಲ. ನಾನು ವ್ಯಕ್ತಿಗೆ ಪ್ರತ್ಯೇಕವಾಗಿ ಗಮನ ಕೊಡುತ್ತೇನೆ. ಅವನನ್ನು ಬಲಾಢ್ಯನನ್ನಾಗಿ ಮಾಡಲು ಯತ್ನಿಸುತ್ತೇನೆ, ಅವನನ್ನು ಪವಿತ್ರಾತ್ಮ ಎಂದು ಸಂಬೋಧಿಸುತ್ತೇನೆ. ಎಲ್ಲರೂ ತಮ್ಮಲ್ಲಿರುವ ದಿವ್ಯತೆಯನ್ನು ವ್ಯಕ್ತಗೊಳಿಸಬೇಕೆಂಬುದೇ ನನ್ನ ಆಶಯ. ಇದೇ ಪ್ರತಿಯೊಂದು ಧರ್ಮದ (ಪ್ರತ್ಯಕ್ಷ ಅಥವಾ ಪರೋಕ್ಷವಾದ) ಆದರ್ಶವಾಗಿದೆ.”

ಬಾತ್ಮೀದಾರ: “ಈ ದೇಶದಲ್ಲಿ ನಿಮ್ಮ ಕಾರ್ಯ ಯಾವ ರೀತಿ ಆಗುವುದು?”

ಸ್ವಾಮೀಜಿ: “ನಾನು ಮೇಲೆ ಹೇಳಿದ ಭಾವನೆಯ ಮೇಲೆ ಕೆಲವು ವ್ಯಕ್ತಿಗಳಿಗೆ ಉತ್ಸಾಹ ಹುಟ್ಟುವಂತೆ ಮಾಡಬೇಕೆಂದಿರುವೆನು. ಅನಂತರ ಅವರು ಈ ಭಾವನೆ ಯನ್ನು ತಮ್ಮದೇ ಆದ ರೀತಿಯಲ್ಲಿ ಇತರರಿಗೂ ಹೇಳಬಹುದು. ತಮಗೆ ತೋರಿದ ರೀತಿಯಲ್ಲಿ ಇದನ್ನೇ ಹೇಳಬಹುದು. ನಾನು ಅದನ್ನು ಒಂದು ಮತದಂತೆ ಹೇಳು ವುದಿಲ್ಲ. ಕೊನೆಗೆ ಸತ್ಯವೇ ಜಯಿಸಬೇಕು.”

“ನಾನು ಹೇಗೆ ಕೆಲಸಮಾಡಬೇಕೊ ಅದು ನನ್ನ ಕೆಲವು ಸ್ನೇಹಿತರ ಕೈಯಲ್ಲಿದೆ. ಅಕ್ಟೋಬರ್​ ೨೨ನೆಯ ತಾರೀಕು ಇಂಗ್ಲೀಷ್​ ಸಭಿಕರಿಗೆ ಪಿಕಾಡಿಲಿಯಲ್ಲಿರುವಪ್ರಿನ್ಸೆಸ್​ ಹಾಲಿನಲ್ಲಿ ೮:೨೦ ಗಂಟೆಗೆ ನನ್ನ ಉಪನ್ಯಾಸವನ್ನು ಏರ್ಪಡಿಸಿರುವರು. ಈ ಕಾರ್ಯಕ್ರಮವನ್ನು ಪತ್ರಿಕೆಗಳಲ್ಲಿ ಜಾಹೀರಾತು ಮಾಡಿರುವರು. ವಿಷಯ “ನನ್ನ ತತ್ತ್ವದ ತಿರುಳಾದ ಆತ್ಮಜ್ಞಾನ” ಎನ್ನುವುದು. ಅನಂತರ ಕೆಲಸ ಯಾವ ರೀತಿ ನಡೆಯುವುದೋ ಅದನ್ನೇ ಅನುಸರಿಸುವೆನು. ಜನರನ್ನು ಭೇಟಿ ಮಾಡುವುದು, ಪತ್ರಗಳಿಗೆ ಉತ್ತರ ಕೊಡುವುದು ಅಥವಾ ಪ್ರತ್ಯಕ್ಷವಾಗಿ ಅವರೊಡನೆ ಚರ್ಚಿಸು ವುದು. ಈ ಹಣದ ಯುಗದಲ್ಲಿ ನಾನು ಮಾಡುವ ಯಾವುದೂ ಹಣವನ್ನು ಗಳಿಸು ವುದಕ್ಕಾಗಿ ಅಲ್ಲ.”

ನನ್ನ ಜೀವನದಲ್ಲೇ ಅಪೂರ್ವ ವ್ಯಕ್ತಿಯೊಬ್ಬರನ್ನು ನೋಡುವ ಸುಯೋಗ ನನಗೆ ಒದಗಿತ್ತು. ಅನಂತರ ಅವರಿಂದ ನಾನು ಬೀಳ್ಕೊಂಡೆ.

