
\chapter[ಶ‍್ರೀಕೃಷ್ಣ ]{ಶ‍್ರೀಕೃಷ್ಣ \protect\footnote{\engfoot{C.W. Vol. 1, P. 437}}}

\centerline{(೧೯೦೦ರ ಏಪ್ರಿಲ್​ ೧ರಂದು ಕ್ಯಾಲಿಫೋರ್ನಿಯಾದಲ್ಲಿ ನೀಡಿದ ಉಪನ್ಯಾಸ)}

ಭರತಖಂಡದಲ್ಲಿ ಬುದ್ಧನ ಜನನ ಕಾಲದಲ್ಲಿದ್ದ ಪರಿಸ್ಥಿತಿಯು ಶ‍್ರೀಕೃಷ್ಣನ ಜನನದ ಕಾಲದಲ್ಲಿಯೂ ಇತ್ತು. ಇಷ್ಟು ಮಾತ್ರವಲ್ಲ. ನಮ್ಮ ಕಾಲದಲ್ಲೂ ಕೂಡ ಅದೇ ಪರಿಸ್ಥಿತಿ ಇದೆ. ಎಲ್ಲೋ ಒಂದು ಆದರ್ಶ ಇದೆ. ಆದರೆ ಮಾನವ ಕೋಟಿಯ ಬಹುಪಾಲು ಜನ ಬೌದ್ಧಿಕವಾಗಿ ಕೂಡ ಆ ಆದರ್ಶದ ಮಟ್ಟಕ್ಕೆ ಏರಲಾರರು. ಬಲಿಷ್ಠರು ಆದರ್ಶವನ್ನು ಅನುಷ್ಠಾನಕ್ಕೆ ತರುವರು. ಅವರಿಗೆ ಅನೇಕವೇಳೆ ದುರ್ಬಲರ ವಿಷಯದಲ್ಲಿ ಸಹಾನುಭೂತಿ ಇರುವುದಿಲ್ಲ. ಬಲಾಢ್ಯನಿಗೆ ಬಲಹೀನನು ಭಿಕ್ಷುಕನಂತೆ ಕಾಣುವನು. ಬಲಾಢ್ಯರೇನೋ ಮುಂದುವರಿದು ಹೋಗುವರು. ಆದರೆ ಯಾರು ದುರ್ಬಲರಾಗಿ ರುವರೊ ಅವರಿಗೆ ಸಹಾನುಭೂತಿಯನ್ನು ತೋರುವುದು ಮತ್ತು ಅವರಿಗೆ ನೆರ ವಾಗುವುದು – ಇದೇ ನಾವು ತಾಳ ಬೇಕಾದ ಉನ್ನತ ದೃಷ್ಟಿ. ಆದರೆ ಅನೇಕ ವೇಳೆ ತತ್ತ್ವಜ್ಞಾನಿಯು ಇತರರಿಗೆ ನಾವು ಸಹಾನುಭೂತಿಯನ್ನು ತೋರುವುದಕ್ಕೆ ಆತಂಕ ವಾಗಿ ಪರಿಣಮಿಸುವನು. ಅನಂತ ಜೀವನವನ್ನು ನಮ್ಮ ಈ ಜೀವನದ ಕೆಲವು ವರುಷ ಗಳ ಆಯಸ್ಸಿನಲ್ಲಿ ನಿರ್ಧರಿಸಬೇಕಾಗಿದೆ ಎಂಬ ಸಿದ್ಧಾಂತವನ್ನು ಅನುಸರಿಸಿದರೆ ನಾವು ಹತಾಶರಾಗುತ್ತೇವೆ. ಯಾರು ದುರ್ಬಲರೋ ಅವರನ್ನು ಗಮನಿಸುವುದಕ್ಕೆ ನಮಗೆ ಕಾಲವೇ ಇರುವುದಿಲ್ಲ. ಆದರೆ ಪರಿಸ್ಥಿತಿ ಹೀಗಾಗದೆ ಈ ಜನ್ಮವು ನಾವು ಸಾಗಿಹೋಗ ಬೇಕಾದ ಹಲವು ಅನುಭವಗಳಲ್ಲಿ ಒಂದಾಗಿದ್ದರೆ, ನಮ್ಮ ಅನಂತ ಜೀವನವನ್ನು ಅನಂತ ವಾದ ನಿಯಮಗಳ ಎರಕದಲ್ಲಿ ಹಾಕಿ ರೂಪಿಸಿ ನಿಯಂತ್ರಿಸಬೇಕಾಗಿರುವುದು ಸತ್ಯವಾದರೆ, ಪ್ರತಿಯೊಂದು ಜೀವಿಗೂ ಬೇಕಾದಷ್ಟು ಅವಕಾಶ ಗಳು ಇರುವುವು. ಆಗ ನಾವು ಅವಸರ ಪಡಬೇಕಾಗಿಲ್ಲ. ನಮಗೆ ಸುತ್ತಲೂ ನೋಡುವುದಕ್ಕೆ, ಸಹಾನುಭೂತಿ ತೋರುವುದಕ್ಕೆ, ದುರ್ಬಲರಿಗೆ ಕೈ ನೀಡಿ ಅವರನ್ನು ಮೇಲೆ ಎತ್ತುವುದಕ್ಕೆ ಬೇಕಾದಷ್ಟು ಕಾಲಾವಕಾಶವಿರುತ್ತದೆ.

ಸಂಸ್ಕೃತದಲ್ಲಿ ಬೌದ್ಧಧರ್ಮದಲ್ಲಿರುವಂತೆ ಎರಡು ಪದಗಳು ಇವೆ. ಇಂಗ್ಲೀಷಿನಲ್ಲಿ ಒಂದನ್ನು \enginline{religion} (ಧರ್ಮವೆಂದು) ಮತ್ತೊಂದನ್ನು \enginline{sect} (ಪಂಥ) ಎಂದು ಅನುವಾದಿಸುವರು. ಶ‍್ರೀಕೃಷ್ಣನ ಶಿಷ್ಯರು, ಅವನ ಅನುಯಾಯಿಗಳು ಮತ್ತು ಧರ್ಮಕ್ಕೆ ಒಂದು ಹೆಸರು ಇಲ್ಲದಿರುವುದೇ ಒಂದು ಸೋಜಿಗವಾಗಿದೆ. ಹೊರಗಿನವರು ಅವರ ಧರ್ಮವನ್ನು ಹಿಂದೂ ಧರ್ಮ ಅಥವಾ ಬ್ರಾಹ್ಮಣ ಧರ್ಮ ಎಂದು ಕರೆಯುತ್ತಾರೆ. ಒಂದೇ ಧರ್ಮ ಇರುವುದು, ಆದರೆ ಪಂಥಗಳು ಹಲವು ಇವೆ. ನೀವು ಅದಕ್ಕೆ ಒಂದು ಹೆಸರನ್ನು ಕೊಟ್ಟು ಉಳಿದವುಗಳಿಂದ ಅದನ್ನು ಪ್ರತ್ಯೇಕ ಮಾಡಿದೊಡನೆ ಅದೊಂದು ಪಂಥವಾಗುವುದು, ಎಂದಿಗೂ ಧರ್ಮವಾಗುವುದಿಲ್ಲ. ಪಂಥವು ತನ್ನಲ್ಲಿರುವ ಸತ್ಯವನ್ನೇ ಸಾರುತ್ತದೆ, ಬೇರೆ ಯಾವ ಪಂಥದಲ್ಲಿಯೂ ಸತ್ಯವಿಲ್ಲ ಎಂದು ಹೇಳುತ್ತದೆ. ಆದರೆ ಧರ್ಮವಾದರೋ, ಜಗತ್ತಿನಲ್ಲೆಲ್ಲಾ ಹಿಂದೆ ಒಂದೇ ಧರ್ಮ ಇತ್ತು ಮತ್ತು ಈಗ ಇರುವುದೂ ಒಂದೇ ಧರ್ಮವೇ ಎನ್ನುವುದು. ಎಂದಿಗೂ ಎರಡು ಧರ್ಮಗಳು ಇರಲಿಲ್ಲ. ಒಂದೇ ಧರ್ಮ ಬೇರೆ ಬೇರೆ ದೇಶಗಳಲ್ಲಿ ಮತ್ತು ಕಾಲದಲ್ಲಿ ಬೇರೆ ಬೇರೆ ಹೆಸರುಗಳನ್ನು ತಾಳಿದೆ. ಮಾನವ ಕೋಟಿಯ ಗುರಿ ಮತ್ತು ಅದರ ವ್ಯಾಪ್ತಿಯನ್ನು ಕಂಡುಹಿಡಿಯುವುದೇ ನಮ್ಮ ಕೆಲಸವಾಗಿದೆ.

ಶ‍್ರೀಕೃಷ್ಣ ಮಾಡಿದ ಮಹಾನ್​ ಕಾರ್ಯವೇ ಇದು. ನಮ್ಮ ಕಣ್ಣು ಗಳನ್ನು ಶುದ್ಧಿ ಮಾಡಿ, ಮಾನವ ಕೋಟಿಯು ಮುಂದೆ ಸಾಗುತ್ತಿರುವಾಗ ಅದನ್ನು ಒಂದು ಉದಾರವಾದ ದೃಷ್ಟಿಯಿಂದ ನೋಡುವಂತೆ ಮಾಡಿದನು. ಎಲ್ಲದರಲ್ಲೂ ಒಂದೇ ಸತ್ಯವನ್ನು ನೋಡಿದ ಪ್ರಥಮ ವ್ಯಕ್ತಿ ಅವನು. ಪ್ರತಿಯೊಬ್ಬರಿಗೂ ತೃಪ್ತಿಯನ್ನೇ ತರುವಂತಹ ಸುಂದರವಾದ ನುಡಿಗಳು ಪ್ರಪಂಚಕ್ಕೆ ಪ್ರಥಮ ಬಾರಿ ಬಂದದ್ದು ಈ ವ್ಯಕ್ತಿಯ ವದನದ ಮೂಲಕ.

ಶ‍್ರೀಕೃಷ್ಣ ನು ಬುದ್ಧನಿಗಿಂತ ಸುಮಾರು ಒಂದು ಸಾವಿರ ವರುಷಗಳ ಮುಂಚೆ ಇದ್ದನು. ಅನೇಕ ಜನ ಅವನು ಎಂದಾದರೂ ಇದ್ದ ಎಂಬುದನ್ನು ನಂಬುವುದಿಲ್ಲ. ಹಿಂದೆ ಬಳಕೆಯಲ್ಲಿದ್ದ ಸೂರ್ಯೋಪಾಸನೆಯಿಂದ ಶ‍್ರೀಕೃಷ್ಣ ಉಪಾಸನೆ ಪ್ರಾರಂಭವಾಗಿರಬೇಕು ಎಂದು ಕೆಲವರು ಭಾವಿಸುವರು. ಹಲವು ಕೃಷ್ಣರು ಇರುವಂತೆ ಕಾಣುವುದು. ಒಬ್ಬನು ಉಪನಿಷತ್ತಿನಲ್ಲಿ ಬರುವನು. ಮತ್ತೊಬ್ಬ ರಾಜನಾಗಿದ್ದನು. ಇನ್ನೊಬ್ಬನು ದಂಡನಾಯಕನಾಗಿದ್ದ. ಇವರೆಲ್ಲ ಸೇರಿ ಒಂದು ಕೃಷ್ಣನೆಂಬ ವ್ಯಕ್ತಿ ಯಾಗಿರಬಹುದು. ಆದರೆ ಇದರಿಂದ ಏನೂ ಬಾಧಕ ಇಲ್ಲ. ಸತ್ಯಾಂಶವೇನೆಂದರೆ, ಆಧ್ಯಾತ್ಮಿಕ ಜೀವನದಲ್ಲಿ ಅಪೂರ್ವವಾದ ವ್ಯಕ್ತಿಯೊಬ್ಬ ಉದಿಸುವುದು. ಅನಂತರ ಬೇಕಾದಷ್ಟು ಕಥೆಗಳು ಅವನ ಸುತ್ತಲೂ ಬೆಳೆದುಕೊಳ್ಳುವುದು. ಕೃಷ್ಣನೆಂಬ ವ್ಯಕ್ತಿಗೆ ಸಂಬಂಧಪಟ್ಟ ಪುರಾಣಕಥೆಗಳೆಲ್ಲಾ ಒಂದು ವ್ಯಕ್ತಿಯ ಜೀವನದೊಂದಿಗೆ ಹೊಂದಿ ಕೊಳ್ಳಬೇಕಾಗಿದೆ. ನ್ಯೂ ಟೆಸ್ಟಮೆಂಟಿನಲ್ಲಿ ಬರುವ ಕಥೆಗಳನ್ನೆಲ್ಲಾ ನಾವು ಒಪ್ಪಿ \-ಕೊಂಡಿರುವ ಕ್ರಿಸ್ತನ ಜೀವನ ಮತ್ತು ಶೀಲದೊಂದಿಗೆ ಹೊಂದಿಕೊಳ್ಳಬೇಕಾಗಿದೆ. ಇಂಡಿಯಾ ದೇಶದಲ್ಲಿ ರೂಢಿಯಲ್ಲಿರುವ ಬುದ್ಧನಿಗೆ ಸಂಬಂಧಪಟ್ಟ ಕಥೆಗಳಲ್ಲೆಲ್ಲಾ ಒಂದು ಮುಖ್ಯವಾದ ಕೇಂದ್ರ ಭಾವನೆಯೇ ಇತರರಿಗಾಗಿ ತ್ಯಾಗ ಮಾಡುವುದು. ಇದರೊಂದಿಗೆ ಅವನ ಜೀವನವನ್ನು ಹೊಂದಿಸಿಕೊಳ್ಳಬೇಕಾಗಿದೆ.

ಶ‍್ರೀಕೃಷ್ಣನ ಸಂದೇಶದಲ್ಲಿ ಎರಡು ಮುಖ್ಯ ಭಾವನೆಗಳನ್ನು ನಾವು ನೋಡತ್ತೇವೆ. ಮೊದಲನೆಯದು ಭಿನ್ನಭಿನ್ನ ಭಾವನೆಗಳ ಮಧ್ಯದಲ್ಲಿ ತೋರುವ ಸಾಮರಸ್ಯ. ಎರಡನೆಯದು ಅನಾಸಕ್ತಿ. ಒಬ್ಬ ಮನುಷ್ಯ ಸಿಂಹಾಸನದ ಮೇಲೆ ಕುಳಿತು ರಾಜ್ಯ ಭಾರ ಮಾಡುತ್ತಿದ್ದರು, ದೊಡ್ಡ ಸೈನ್ಯದ ದಂಡನಾಯಕನಾಗಿದ್ದರು, ದೇಶದ ಪ್ರಗತಿಗೆ ದೊಡ್ಡ ಯೋಜನೆಗಳನ್ನು ಕಾರ್ಯಗತ ಮಾಡುವುದರಲ್ಲಿ ನಿರತನಾಗಿದ್ದರೂ, ಅವನು ಪೂರ್ಣಾತ್ಮನಾಗಬಹುದು. ಜೀವನದ ಪರಮಗುರಿಯನ್ನೇ ಮುಟ್ಟಬಹುದು. ವಾಸ್ತವವಾಗಿ ಕೃಷ್ಣನ ಮಹಾನ್​ ಉಪದೇಶವು ಬೋಧಿಸಲ್ಪಟ್ಟಿದ್ದು ಯುದ್ಧ ಭೂಮಿಯಲ್ಲಿ.

ಶ‍್ರೀಕೃಷ್ಣ, ಹಿಂದಿನ ಕಾಲದ ಪುರೋಹಿತರ ಆಚಾರಗಳಲ್ಲಿದ್ದ ನಟನೆ ಮತ್ತು ಭಾವಹೀನತೆ ಇವುಗಳನ್ನೆಲ್ಲಾ ಚೆನ್ನಾಗಿ ಮನಗಂಡನು. ಅವುಗಳಲ್ಲಿ ಸ್ವಲ್ಪ ಒಳ್ಳೆಯ ಅಂಶವನ್ನೂ ಕಂಡನು.

ನೀನು ತುಂಬಾ ಬಲಶಾಲಿಯಾದರೆ ಒಳ್ಳೆಯದು. ಆದರೆ ಯಾರು ನಿನ್ನಷ್ಟು ಬಲಶಾಲಿಗಳಲ್ಲವೋ ಅವರನ್ನು ಹಳಿಯಬೇಡ. ಪ್ರತಿಯೊಬ್ಬನೂ “ಅಯ್ಯೊ ಈ ಜನರು ಅಜ್ಞರು” ಎಂದು ಶಪಿಸುವವರೆ. ಆದರೆ ಯಾರು “ಅಯ್ಯೊ ನಿಮಗೆ ಸಹಾಯ ಮಾಡಲು ಸಾಧ್ಯವಿಲ್ಲವಲ್ಲ! ನಾನು ಎಂತಹ ಸ್ಥಿತಿಯಲ್ಲಿರುವೆನು” ಎಂದು ಕೊಳ್ಳುವ ವರು ಅಪರೂಪ. ಜನರು ತಮ್ಮ ಯೋಗ್ಯತೆ, ತಿಳುವಳಿಕೆ ಮತ್ತು ಸಾಮರ್ಥ್ಯಗಳಿಗೆ ತಕ್ಕಂತೆ ಕೆಲಸ ಮಾಡುತ್ತಿರುವರು. ನಾವಿರುವ ಸ್ಥಳಕ್ಕೆ ಅವರನ್ನು ತರಲು ಸಾಧ್ಯವಿಲ್ಲವಲ್ಲ, ಎಷ್ಟು ಶೋಚನೀಯ ಸ್ಥಿತಿ!

ಆದಕಾರಣವೇ ಹಲವು ಆಚಾರಗಳು, ದೇವದೇವತೆಗಳ ಪೂಜೆ, ಪುರಾಣಗಳು ಇವುಗಳೆಲ್ಲ ಸರಿ ಎನ್ನುವನು ಶ‍್ರೀಕೃಷ್ಣ. ಏತಕ್ಕೆ ಎಂದರೆ ಇವುಗಳೆಲ್ಲ ಒಂದೇ ಗುರಿ ಯೆಡೆಗೆ ನಮ್ಮನ್ನು ಒಯ್ಯುತ್ತವೆ. ಆಚಾರ, ಶಾಸ್ತ್ರ ವಿಚಾರ ಇವುಗಳೆಲ್ಲ ಒಂದು ಉದ್ದ ನೆಯ ಸರಪಳಿಯಲ್ಲಿರುವ ಕೊಂಡಿಗಳು. ಅವುಗಳಲ್ಲಿ ಯಾವುದಾದರೂ ಒಂದನ್ನು ಹಿಡಿದುಕೊಳ್ಳಿ, ಅದು ಮುಖ್ಯವಾದುದು. ನಿನ್ನಲ್ಲಿ ಶ್ರದ್ಧೆಯಿದ್ದರೆ, ನೀನು ಯಾವು ದಾದರೂ ಒಂದು ಕೊಂಡಿಯನ್ನಾದರೂ ಹಿಡಿದುಕೊಂಡಿದ್ದರೆ, ಅದನ್ನು ಬಿಡದೇ ಇದ್ದರೆ ಉಳಿದವುಗಳೆಲಾ ತಾವೇ ತಾವಾಗಿ ಬರುತ್ತವೆ. ಆದರೆ ಜನ ಯಾವುದನ್ನೂ ಹಿಡಿದು ಕೊಳ್ಳುವುದಿಲ್ಲ. ತಾವು ಯಾವುದನ್ನು ಹಿಡಿದುಕೊಳ್ಳಬೇಕೆಂಬುದನ್ನು ನಿರ್ಧರಿಸು ವುದಕ್ಕಾಗಿಯೇ ಜನ ಸುಮ್ಮನೆ ಹೋರಾಡುವರು. ಆದರೆ ಯಾವುದನ್ನೂ ಹಿಡಿದು ಕೊಳ್ಳುವುದಿಲ್ಲ. ನಾವು ಯಾವಾಗಲೂ ಸತ್ಯವನ್ನು ಅರಸುತ್ತಿರುವೆವು. ಆದರೆ ಅದನ್ನು ಪಡೆಯಲು ಇಚ್ಚಿಸುವುದಿಲ್ಲ. ಸುಮ್ಮನೆ ಹೋಗಿ ಕೇಳುವ ಸಂತೋಷ ಮಾತ್ರ ನಮಗೆ ಬೇಕು. ನಮ್ಮಲ್ಲಿ ಬೇಕಾದಷ್ಟು ಶಕ್ತಿ ಇದೆ. ನಾವು ಅದನ್ನು ಹೀಗೆ ವ್ಯಯಮಾಡುತ್ತೇವೆ. ಆದಕಾರಣವೇ ಮೂಲಕೇಂದ್ರದಿಂದ ಹಬ್ಬಿರುವ ಯಾವುದಾದರೂ ಒಂದು ಸರಪಳಿಯನ್ನು ಹಿಡಿದುಕೊಳ್ಳಿ ಎಂದು ಶ‍್ರೀಕೃಷ್ಣ ಹೇಳುವನು. ಯಾವುದೋ ಒಂದು ಹಂತವು ಮತ್ತೊಂದಕ್ಕಿಂತ ಮೇಲಲ್ಲ. ಎಲ್ಲಿಯವರೆಗೆ ಯಾರ ಧರ್ಮ ನಿಷ್ಕಪಟವಾಗಿದೆಯೊ ಅಲ್ಲಿಯವರೆಗೆ ಅಲ್ಲಿರುವುದಾವುದನ್ನೂ ಹಳಿಯ ಬೇಡಿ. ನೀವು ಇವುಗಳಲ್ಲಿ ಯಾವುದಾದರು ಒಂದು ಕೊಂಡಿಯನ್ನು ಹಿಡಿದು ಕೊಂಡರೂ ಸಾಕು. ಅದು ನಿಮ್ಮನ್ನು ಕೇಂದ್ರದ ಕಡೆಗೆ ಸೆಳೆಯುವುದು. ನಿಮ್ಮ ಹೃದಯವೆ ಉಳಿದವುಗಳನ್ನೆಲ್ಲಾ ಬೋಧಿಸುವುದು. ಅಂತರ್ಯಾಮಿಯಾದ ಗುರುವೇ ಎಲ್ಲಾ ಮತ ಗಳನ್ನೂ ತತ್ತ್ವಗಳನ್ನೂ ಬೋಧಿಸುವನು.

ಕ್ರಿಸ್ತನಂತೆ ಶ‍್ರೀಕೃಷ್ಣ ತಾನೇ ದೇವರು ಎಂದು ಹೇಳಿಕೊಳ್ಳುವನು. ಅವನು ದೇವರನ್ನು ತನ್ನಲ್ಲಿಯೇ ನೋಡುತ್ತಿರುವನು. “ಯಾರೂ ನನ್ನನ್ನು ಬಿಟ್ಟು ಒಂದು ದಿನ ವೂ ಇರಲಾರರು, ಎಲ್ಲರೂ ನನ್ನ ಬಳಿಗೇ ಬರಬೇಕಾಗಿದೆ. ಯಾರಿಗೆ ಯಾವ ಆಕಾರದ ಮೂಲಕ ನನ್ನ ಆರಾಧನೆ ಮಾಡಬೇಕೆಂಬ ಇಚ್ಛೆ ಇದೆಯೋ ನಾನು ಅವನಿಗೆ ಅದರಲ್ಲಿಯೇ ಶ್ರದ್ಧೆಯನ್ನು ಹುಟ್ಟಿಸುವೆನು. ಅದರ ಮೂಲಕವೇ ನಾನು ಶ‍್ರೀಕೃಷ್ಣ ೬೯ ಅವರನ್ನು ನೋಡುತ್ತೇನೆ” ಎನ್ನುವನು ಶ‍್ರೀಕೃಷ್ಣ. ಅವನ ಹೃದಯ ಜನ ಸಾಮಾನ್ಯರಿಗಾಗಿ ಮಿಡಿಯುತ್ತಿರುವುದು.

ಶ‍್ರೀಕೃಷ್ಣನು ಸರ್ವತಂತ್ರ ಸ್ವತಂತ್ರ. ಅವನ ಧೈರ್ಯವೇ ನಮ್ಮನ್ನು ಅಂಜಿಸುವುದು. ನಾವು ಎಲ್ಲಕ್ಕೂ ಯಾರನ್ನಾದರೂ ಆಶ್ರಯಿಸುತ್ತಿರುವೆವು. ಯಾವುದಾದರೂ ಸನ್ನಿವೇಶ ನಮ್ಮ ಸಹಾಯಕ್ಕೆ ಬೇಕು. ಯಾರಾದರೂ ಹೇಳುವ ಒಂದೆರಡು ಒಳ್ಳೆಯ ಮಾತುಗಳು ನಮಗೆ ಬೆಂಬಲವಾಗಿರಬೇಕು. ಆತ್ಮವು ಯಾವಾಗ ಯಾವುದನ್ನೂ ಅವಲಂಬಿಸಿಲ್ಲವೋ, ಜೀವನವನ್ನೂ ಅವಲಂಬಿಸಿಲ್ಲವೋ, ಅದೇ ತತ್ತ್ವದ ಪರಾ ಕಾಷ್ಠೆ, ಪುರುಷಾಗ್ರಣಿಯ ಪರಾಕಾಷ್ಠೆ. ಪೂಜೆಯು ಕೂಡ ನಮ್ಮನ್ನು ಅದೇ ಗುರಿಯ ಕಡೆಗೆ ಒಯ್ಯುವುದು. ಶ‍್ರೀಕೃಷ್ಣ ಪೂಜೆಗೆ ಹೆಚ್ಚು ಪ್ರಾಶಸ್ತ್ಯವನ್ನು ಕೊಡುವನು. ದೇವರನ್ನು ಪೂಜಿಸಿ.

ಈ ಜಗತ್ತಿನಲ್ಲಿ ಹಲವು ಬಗೆಯ ಪೂಜೆಗಳನ್ನು ನಾವು ನೋಡುತ್ತೇವೆ. ತನ್ನ ಸಂಪತ್ತನ್ನು ಕಳೆದುಕೊಂಡ ಮನುಷ್ಯನಿರುವನು. ಅವನು ದ್ರವ್ಯವನ್ನು ಪಡೆಯಲು ಬೇಕಾದಷ್ಟು ಪ್ರಾರ್ಥಿಸುವನು. ಆದರೆ ಪರಾಪೂಜೆ ಪ್ರೀತಿಗಾಗಿಯೇ ದೇವರನ್ನು ಪ್ರೀತಿಸುವುದು. ದೇವನೊಬ್ಬನಿದ್ದರೆ ಇಷ್ಟೊಂದು ದುಃಖವೇಕೆ ಇರಬೇಕು? ಎಂಬ ಪ್ರಶ್ನೆಯು ಬರಬಹುದು. ಆದರೆ ಭಕ್ತನಾದರೋ ಹೀಗೆ ಹೇಳುವನು “ಈ ಪ್ರಪಂಚ ದಲ್ಲಿ ದುಃಖವಿದೆ, ಅದುದರಿಂದ ನಾನು ದೇವರನ್ನು ಪ್ರೀತಿಸುವುದನ್ನು ಬಿಡಲಾರೆ. ಆದರೆ ನನ್ನ ದುಃಖವನ್ನೆಲ್ಲಾ ತೊಡೆದು ಹಾಕೆಂದು ನಾನು ದೇವರಲ್ಲಿ ಪ್ರಾರ್ಥಿಸುವುದಿಲ್ಲ. ನಾನು ಅವನನ್ನು ಪ್ರೀತಿಸುತ್ತೇನೆ. ಏಕೆಂದರೆ ಅವನು ಪ್ರೇಮಸ್ವರೂಪಿ.” ಇತರ ಪೂಜಾದಿಗಳು ಕೆಳಮಟ್ಟದ್ದು. ಆದರೆ ಶ‍್ರೀಕೃಷ್ಣ ಯಾವುದನ್ನೂ ದೂರುವುದಿಲ್ಲ. ಏನನ್ನೂ ಮಾಡದೇ ಇರುವುದಕ್ಕಿಂತ ಏನಾದರೂ ಮಾಡುವುದು ಒಳ್ಳೆಯದು. ದೇವರನ್ನು ಪೂಜಿಸಲು ಪ್ರಾರಂಭಿಸಿದ ಮನುಷ್ಯ ಕ್ರಮೇಣ ಬೆಳೆಯುತ್ತಾ ಬರುವನು. ಆನಂತರ ಅವನು ದೇವರನ್ನು ಪ್ರೀತಿಗಾಗಿ ಪ್ರೀತಿಸುವನು.

ಈ ಜನ್ಮವನ್ನು ತ್ಯಜಿಸಿ ನಾವು ಪವಿತ್ರತೆಯನ್ನು ಹೇಗೆ ಪಡೆಯಬಲ್ಲೆವು? ನಾವೆಲ್ಲ ಕಾಡಿನ ಗುಹೆಗಳಿಗೆ ಹೋಗಲು ಸಾಧ್ಯವೇ? ಇದರಿಂದ ಏನು ಪ್ರಯೋಜನ? ಮನಸ್ಸನ್ನು ನಿಗ್ರಹಿಸದೇ ಇದ್ದರೆ ಗುಹೆಯಲ್ಲಿದ್ದ್ದೂ ಪ್ರಯೋಜನವಿಲ್ಲ. ಏಕೆಂದರೆ ಇದೇ ಮನಸ್ಸು ಅಲ್ಲಿ ಎಲ್ಲಾ ತರಹದ ಹಾವಳಿಯನ್ನೂ ಉಂಟುಮಾಡುವುದು. ಆ ಗುಹೆಯಲ್ಲಿಯೇ ಇಪ್ಪತ್ತು ದೆವ್ವಗಳು ಇರುವುವು. ಏಕೆಂದರೆ ಅವುಗಳೆಲ್ಲ ನಮ್ಮ ಮನಸ್ಸಿನಲ್ಲೆ ಇವೆ. ಮನಸ್ಸು ಸ್ವಾಧೀನದಲ್ಲಿದ್ದರೆ ಗುಹೆಯನ್ನೇ ಎಲ್ಲಿ ಬೇಕಾದರೂ ಮಾಡಿ ಕೊಳ್ಳಬಹುದು, ನಾವಿರುವ ಕಡೆಯೇ ಇರಬಹುದು.

ಜಗತ್ತು ಈಗ ಇರುವ ಸ್ಥಿತಿಗೆ ಕಾರಣ ನಮ್ಮ ಮನಸ್ಸು. ನಮ್ಮ ಆಲೋಚನೆಯೇ ಒಂದು ವಸ್ತುವನ್ನು ಸುಂದರವಾಗಿ ಮಾಡುವುದು. ನಮ್ಮ ಆಲೋಚನೆಯೇ ಅದನ್ನು ವಿಕಾರವಾಗುವಂತೆ ಮಾಡುವುದು. ಈ ಪ್ರಪಂಚವೆಲ್ಲ ನಮ್ಮ ಮನಸ್ಸಿನಲ್ಲಿದೆ.ವಸ್ತುಗಳನ್ನು ಸರಿಯಾದ ರೀತಿಯಲ್ಲಿ ನೋಡುವುದನ್ನು ಅಭ್ಯಾಸಮಾಡಿ. ಮೊದಲನೆಯದಾಗಿ ಈ ಪ್ರಪಂಚವನ್ನು ನಂಬಿ. ಪ್ರತಿಯೊಂದಕ್ಕೂ ಅರ್ಥವಿದೆ. ಈ ಪ್ರಪಂಚದಲ್ಲಿ ಇರುವ ಪ್ರತಿಯೊಂದೂ ಒಳ್ಳೆಯದು, ಪವಿತ್ರವಾದುದು, ಸುಂದರವಾದುದು. ನೀವು ಯಾವುದಾದರೂ ಕೆಟ್ಟದ್ದನ್ನು\break ನೋಡುತ್ತಿರುವಿರಾದರೆ ಅದನ್ನು ನೀವು ಸರಿಯಾಗಿ ಅರ್ಥಮಾಡಿಕೊಳ್ಳುತ್ತಿಲ್ಲ ಎಂದು ಭಾವಿಸಿ. ಜವಾಬ್ದಾರಿಯನ್ನು ನಿಮ್ಮ ಮೇಲೆ ಹಾಕಿಕೊಳ್ಳಿ. ಈ ಪ್ರಪಂಚ ಹಾಳಾಗುತ್ತಿದೆ ಎಂದು ನಾವು ಯಾವಾಗಲಾದರೂ ಹೇಳಿದರೆ, ನಮ್ಮ ಮನಸ್ಸನ್ನು ನಾವೇ ಶೋಧಿಸಿಕೊಳ್ಳಬೇಕಾಗುತ್ತದೆ. ಆಗ ಗೊತ್ತಾಗು ವುದು, ವಸ್ತುವಿನ ಸಹಜಸ್ಥಿತಿಯನ್ನು ನೋಡುವ ಅಭ್ಯಾಸವನ್ನು ನಾವು ಕಳೆದು ಕೊಂಡಿರುವೆವು ಎಂಬುದು.

ಹಗಲೂ ರಾತ್ರಿ ಕೆಲಸಮಾಡಿ. “ನೋಡು, ನಾನು ವಿಶ್ವೇಶ್ವರ. ನನಗೆ ಯಾವ ಕರ್ತವ್ಯವೂ ಇಲ್ಲ. ಪ್ರತಿಯೊಂದು ಕರ್ತವ್ಯವೂ ಬಂಧನ. ಆದರೆ ನಾನು ಕರ್ಮಕ್ಕಾಗಿ ಕರ್ಮವನ್ನು ಮಾಡುತ್ತೇನೆ. ನಾನೇನಾದರೂ ಒಂದು ಕ್ಷಣ ಕೆಲಸ ಮಾಡದೇ ಇದ್ದರೂ ಈ ಪ್ರಪಂಚ ಅನಾಯಕ\-ವಾಗುವುದು” (ಗೀತಾ, \enginline{II, 22–23}). ಆದಕಾರಣವೇ ಇದು ನನ್ನ ಕರ್ತವ್ಯ ಎಂಬ ಭಾವನೆಯನ್ನು ತ್ಯಜಿಸಿ ಕರ್ಮ ಮಾಡಿ.

ಈ ಜಗತ್ತು ಒಂದು ಆಟ, ನೀವು ಅವನ ಆಟದ ಸಂಗಾತಿಗಳು. ಯಾವ ದುಃಖವೂ ಇಲ್ಲದೆ, ಯಾವ ಕಷ್ಟವೂ ಇಲ್ಲದೆ ಕೆಲಸ ಮಾಡಿ. ಅವನು ಬಡವನ ಗುಡಿಸಿಲಿ ನಲ್ಲಿ, ಭಾಗ್ಯವಂತನ ಮನೆಯಲ್ಲಿ ಹೇಗೆ ಆಟವಾಡು ತ್ತಿರುವನು ಎಂಬುದನ್ನು ಗಮನಿಸಿ. ಜನರನ್ನು ಮೇಲೆತ್ತುವುದಕ್ಕಾಗಿ ಕೆಲಸ ಮಾಡಿ. ಅವರು ಅಧೋಗತಿಗೆ ಇಳಿದವರು ಅಥವಾ ಪಾಪಿಗಳು ಎಂದಲ್ಲ. ಶ‍್ರೀಕೃಷ್ಣ ಇದನ್ನು ಹೇಳುವುದಿಲ್ಲ.

ಈ ಜಗತ್ತಿನಲ್ಲಿ ಒಳ್ಳೆಯ ಕೆಲಸ ಅಷ್ಟು ಕಡಿಮೆ ಏತಕ್ಕೆ ಎಂಬುವುದು ನಿಮಗೆ ಗೊತ್ತೆ? ನಮ್ಮ ಮಹಿಳೆಯೊಬ್ಬಳು ಬಡಜನರಿರುವ ಬಿಡಾರಕ್ಕೆ ಹೋಗಿ “ಅಯ್ಯೋ ದರಿದ್ರನೇ, ಇದನ್ನು ತೆಗೆದುಕೊಂಡು ಸುಖವಾಗಿರು” ಎಂದು ಕಾಸನ್ನು ಎಸೆಯುತ್ತಾಳೆ. ಅಥವಾ ದಾರಿಯಲ್ಲಿ ಹೋಗುತ್ತಿರುವ ಕುಲೀನ ಮಹಿಳೆಯೊಬ್ಬಳು ಒಬ್ಬ ಬಡವನನ್ನು ಅಲ್ಲಿ ನೋಡುವಳು. ಆಗ ಅವನಿಗೊಂದು ಐದು ಕಾಸಿನ ಬಿಲ್ಲೆಯನ್ನು ಎಸೆಯುವಳು. ಈ ಕೆಲಸದ ಹಿಂದೆ ಇರುವ ಈಶ್ವರನ ನಿಂದೆಯನ್ನು ನೋಡಿ! ನಿಮ್ಮ ಟೆಸ್ಟಮೆಂಟಿ ನಲ್ಲಿಯೇ ನಮಗೆ ದೇವರು ತನ್ನ ಸಂದೇಶವನ್ನು ಹೀಗೆ ಸಾರಿರುವುದರಿಂದ ನಾವು ಧನ್ಯರು– ಜೀಸಸ್​ ಹೀಗೆ ಹೇಳುವನು: “ಈ ಕನಿಷ್ಟ ಪಕ್ಷದ ಸಹೋದರರಿಗೆ ಯಾರು ಎಳ್ಳಷ್ಟು ಉಪಕಾರ ಮಾಡುವರೋ ಅವರು ಅದನ್ನು ನನಗೇ ಮಾಡಿರುವರು.” ನಾವು ಇತರರಿಗೆ ಸಹಾಯ ಮಾಡುವೆವು ಎಂದು ತಿಳಿಯುವುದು ಪಾಷಂಡತನ. ಮತ್ತೊಬ್ಬನಿಗೆ ಸಹಾಯ ಮಾಡುವೆವು ಎಂಬ ಭಾವನೆಯನ್ನು ಮೊದಲು ಅಮೂಲಾಗ್ರವಾಗಿ ಕಿತ್ತುಹಾಕಿ, ನಂತರ ಪೂಜೆಗೆ ಹೋಗಿ ಭಗವಂತನ ಮಕ್ಕಳೇ ನಿಮ್ಮ ಪ್ರಭುವಿನ ಮಕ್ಕಳು. ಮಕ್ಕಳು ಅಂದರೆ ತಂದೆಯ ಬೇರೆ ಬೇರೆ ರೂಪಗಳು ಮಾತ್ರ. ನೀವು ಅವನ ಸೇವಕರು. ಜೀವಂತ ದೇವರಿಗೆ ಸೇವೆ ಸಲ್ಲಿಸಿ. ದೇವರು ಕುರುಡನಂತೆ, ಕುಂಟನಂತೆ, ದೀನನಂತೆ, ಪಟಿಂಗನಂತೆ ನಿಮ್ಮ ಬಳಿಗೆ ಬರುವನು. ಅವನನ್ನು ಪೂಜಿಸುವುದಕ್ಕೆ ನಿಮಗೆ ಎಂತಹ ಅದ್ಭುತವಾದ ಅವಕಾಶ ಸಿಕ್ಕುವುದು. ನೀವು ಯಾವಾಗ ಅವನಿಗೆ ಸಹಾಯ ಮಾಡುತ್ತಿರುವೆ ಎಂದು ಭಾವಿಸುವಿರೋ ಆಗ ನೀವು ಮಾಡಿರುವು ದನ್ನೆಲ್ಲಾ ಹಾಳುಮಾಡಿಕೊಳ್ಳುವಿರಿ, ಅಧೋಗತಿಗೆ ಬರುವಿರಿ. ಇದನ್ನು ಚೆನ್ನಾಗಿ ತಿಳಿದುಕೊಂಡು ಕೆಲಸ ಮಾಡಿ. ಇದರಿಂದ ಏನಾಗುವುದು ಎಂದು ನೀವು ಕೇಳಬಹುದು. ಕೆಲಸದ ಆನಂತರ ನಿಮಗೆ ಯಾವ ನಿರಾಶೆಯೂ ಆಗುವುದಿಲ್ಲ, ಯಾವ ವ್ಯಸನವೂ ಆಗುವುದಿಲ್ಲ. ಕೆಲಸ ಒಂದು ಗುಲಾಮಗಿರಿಯಾಗುವುದಿಲ್ಲ. ಅದೊಂದು ಬರೇ ಆಟವಾಗುವುದು. ಅದೇ ಸಂತೋಷದಾಯಕವಾಗುವುದು. ಕೆಲಸಮಾಡಿ, ಆದರೆ ಅನಾಸಕ್ತ ರಾಗಿರಿ. ಇದೇ ನಿಜ ರಹಸ್ಯ. ನೀವು ಆಸಕ್ತರಾದರೆ ದುಃಖಪಡಬೇಕಾಗುವುದು.

ನಾವು ಜೀವನದಲ್ಲಿ ಯಾವುದೇ ಕೆಲಸವನ್ನು ಮಾಡಲಿ, ಅದರೊಂದಿಗೆ ಆಸಕ್ತರಾಗು ರುವೆವು. ಇಲ್ಲೊಬ್ಬ ಮನುಷ್ಯನು ನನಗೆ ಕಟುವಾಗಿ ಮಾತನಾಡುವನು. ಆಗ ಕೋಪ ನನ್ನ ಮನಸ್ಸಿನಲ್ಲಿ ಏಳುವಂತೆ ಕಾಣುವುದು. ಕೆಲವೇ ಕ್ಷಣಗಳಲ್ಲಿ ಕೋಪ ಮತ್ತು ನಾನು ಬೆರೆತು ಹೋಗುವೆವು. ಆಗ ದುಃಖವಾಗುವುದು. ದೇವರೊಬ್ಬನಲ್ಲಿ ಮಾತ್ರ ಆಸಕ್ತರಾಗಿ. ಮತ್ತಾವುದಕ್ಕೂ ಬೇಡ. ಏಕೆಂದರೆ ಉಳಿದವುಗಳೆಲ್ಲ ಅಸತ್ಯ. ಅಸತ್ಯಕ್ಕೆ ನಾವು ಆಸಕ್ತ ರಾದರೆ ದುಃಖ ನಮಗೆ ತಪ್ಪಿದ್ದಲ್ಲ. ಸತ್ಯ ವಾಗಿರುವುದೇ ಉಳಿಯುವುದು. ಒಂದೇ ಅಸ್ತಿತ್ವ ಇರುವುದು. ಅಲ್ಲಿ ದೃಗ್​ ಮತ್ತು ದೃಶ್ಯದ ಭೇದವಿಲ್ಲ.

ಅನಾಸಕ್ತವಾದ ಪ್ರೀತಿ ನಮ್ಮನ್ನು ದುಃಖಕ್ಕೆ ಗುರಿಮಾಡಲಾರದು. ನೀವು ಯಾವ ಕೆಲಸವನ್ನಾದರೂ ಮಾಡಿ. ಬೇಕಾದರೆ ಮದುವೆಯಾಗಿ, ಮಕ್ಕಳನ್ನು ಪಡೆಯಿರಿ. ನಿಮಗೆ ಏನು ಬೇಕೋ ಅದನ್ನೇ ಮಾಡಿ. ಯಾವುದೂ ನಿಮಗೆ ವ್ಯಥೆಯನ್ನು ತರಲಾರದು. ನನ್ನದು ಎಂಬ ಭಾವನೆಯನ್ನು ಕಡೆಗಾಣಿಸಿ ಕರ್ತವ್ಯಕ್ಕಾಗಿ ಕರ್ತವ್ಯವನ್ನು ಮಾಡಿ, ಕರ್ಮಕ್ಕಾಗಿ ಕರ್ಮವನ್ನು ಮಾಡಿ. ಅದರಿಂದ ನಿಮಗೇನಂತೆ? ನೀವು ದೂರ ಸರಿದು ನಿಲ್ಲಿ.

ನಾವು ಅನಾಸಕ್ತಿಯ ಸ್ಥಿತಿಯನ್ನು ಪಡೆದಾಗ ಮಾತ್ರ ಈ ಪ್ರಪಂಚದ ಅದ್ಭುತವಾದ ರಹಸ್ಯ ನಮಗೆ ಅರಿವಾಗುವುದು; ಅದು ಹೇಗೆ ತೀವ್ರವಾದ ಕರ್ಮ ಮತ್ತು ಸ್ಪಂದನಗಳಿಂದ ಕೂಡಿದ್ದರೂ, ಅದೇ ಸಂದರ್ಭದಲ್ಲಿಯೇ ಪ್ರಶಾಂತವೂ ಸ್ತಬ್ಧವೂ ಆಗಿದೆ; ಅದು ಹೇಗೆ ಪ್ರತಿಕ್ಷಣವೂ ಕ್ರಿಯಾಶೀಲವಾಗಿದೆ, ಮತ್ತು ಪ್ರತಿಕ್ಷಣವೂ ವಿರಾಮದಿಂದಿದೆ ಎಂಬುದು ಗೊತ್ತಾಗುವುದು. ಇದೇ ವಿಶ್ವದ ರಹಸ್ಯ. ಏಕಕಾಲದಲ್ಲೇ ಅದು ವ್ಯಕ್ತಿಗತವಾಗಿದೆ ಮತ್ತು ವ್ಯಕ್ತಿಗೆ ಅತೀತವಾಗಿದೆ, ಅದು ಶಾಂತ ಮತ್ತು ಅನಂತವಾಗಿದೆ. ಆಗಲೇ ನಮಗೆ ಆ ರಹಸ್ಯ ಅರಿವಾಗುವುದು. “ಯಾವನು ಕರ್ಮದಲ್ಲಿ ಅಕರ್ಮವನ್ನು ನೋಡುತ್ತಾನೆಯೋ, ಅಕರ್ಮದಲ್ಲಿ ಕರ್ಮವನ್ನು ನೋಡುತ್ತಾನೆಯೊ ಅವನೇ ಯೋಗಿ.” (ಗೀತಾ). ಅವನೇ ನಿಜವಾಗಿ ಕರ್ಮವನ್ನು ತಿಳಿದವನು, ಇತರರಲ್ಲ. ನಾವು ಸ್ವಲ್ಪ ಕೆಲಸ ಮಾಡಿ ಕುಗ್ಗಿ ಹೋಗುವೆವು. ಏತಕ್ಕೆ? ನಾವು ಕೆಲಸದಲ್ಲಿ ಆಸಕ್ತರಾಗುವೆವು. ನಾವು ಆಸಕ್ತರಾಗದೆ ಇದ್ದರೆ ಕೆಲಸದಿಂದ ಅನಂತ ಶಾಂತಿ ಪ್ರಾಪ್ತವಾಗುವುದು.

ಇಂತಹ ಅನಾಸಕ್ತಿಯನ್ನು ಪಡೆಯಬೇಕಾದರೆ ಎಷ್ಟು ಕಷ್ಟ! ಆದಕಾರಣವೇ ಶ‍್ರೀಕೃಷ್ಣನು ನಮಗೆ ಅದಕ್ಕಿಂತ ಕೆಳಗಿರುವ ಮಾರ್ಗಗಳನ್ನು ತೋರಿಸುವನು.ಪ್ರತಿಯೊಬ್ಬರೂ ತಮಗೆ ಬಂದ ಕೆಲಸವನ್ನು ಮಾಡಿ ಫಲಕ್ಕೆ ಆಸಕ್ತರಾಗದೆ ಇರು ವುದು ಅತ್ಯಂತ ಸುಲಭವಾದ ಮಾರ್ಗ. ನಮ್ಮನ್ನು ಬಂಧನಕ್ಕೆ ಗುರಿಮಾಡುವುದು ನಮ್ಮ ಆಸೆ. ನಾವು ಕರ್ಮದ ಫಲಕ್ಕೆ ಕೈಯೊಡ್ಡಿದರೆ ಅದು ಒಳ್ಳೆಯದಾಗಲಿ, ಕೆಟ್ಟ ದ್ದಾಗಲಿ ಅದನ್ನೇ ಸಹಿಸಬೇಕಾಗುತ್ತದೆ. ಆದರೆ ನಾವು ನಮಗೋಸುಗ ಕೆಲಸ ಮಾಡದೆ ಕೇವಲ ಭಗವಂತನನ್ನು ಕೊಂಡಾಡುವುದಕ್ಕೆ ಮಾತ್ರ ಮಾಡಿದರೆ, ಫಲಗಳು ತಮ್ಮನ್ನು ತಾವು ನೋಡಿಕೊಳ್ಳುವುವು. “ಕೆಲಸ ಮಾಡುವುದಕ್ಕೆ ಮಾತ್ರ ನಿನಗೆ ಅಧಿಕಾರ. ಅದ ರಿಂದ ಬರುವ ಫಲಗಳಿಗಲ್ಲ.” (ಗೀತಾ \enginline{II, 47}). ಸಿಪಾಯಿ ಯಾವ ಪ್ರತಿಫಲವನ್ನೂ ಎದುರುನೋಡದೇ ಕೆಲಸ ಮಾಡುವನು; ಅವನು ತನ್ನ ಕರ್ತವ್ಯವನ್ನು ಮಾಡುವನು. ಸೋಲು ಬಂದರೆ ಅದು ದಂಡನಾಯಕನಿಗೆ ಸೇರಿದ್ದು, ತನಗಲ್ಲ. ನಾವು ಪ್ರೀತಿಗಾಗಿ ನಮ್ಮ ಕರ್ತವ್ಯವನ್ನು ಪರಿಪಾಲಿಸುವೆವು. ಎಲ್ಲರ ಪ್ರೇಮಕ್ಕೆ ಮತ್ತು ಭಗವಂತನ ಪ್ರೇಮ ಕ್ಕೆ ಮಾತ್ರ ಕರ್ತವ್ಯ.

ನೀವು ಬಲಶಾಲಿಗಳಾದರೆ ವೇದಾಂತತತ್ತ್ವವನ್ನು ಸ್ವೀಕರಿಸಿ ಸ್ವತಂತ್ರರಾಗಿ. ಅದು ಸಾಧ್ಯವಿಲ್ಲದೇ ಇದ್ದರೆ ದೇವರನ್ನು ಪೂಜಿಸಿ. ಅದೂ ಸಾಧ್ಯವಿಲ್ಲದೆ ಇದ್ದರೆ ಯಾವುದಾದರೂ\break ವಿಗ್ರಹವನ್ನು ಪೂಜಿಸಿ. ಇದನ್ನು ಮಾಡುವುದಕ್ಕೂ ನಿಮಗೆ ಶಕ್ತಿ ಇಲ್ಲದೆ ಇದ್ದರೆ ಲಾಭದಾಸೆಯಿಲ್ಲದೆ ಯಾವುದಾದರೂ ಒಳ್ಳೆಯ ಕೆಲಸವನ್ನು ಮಾಡಿ. ನಿಮ್ಮಲ್ಲಿ ಇರುವುದನ್ನೆಲ್ಲಾ ಭಗವಂತನ ಸೇವೆಗಾಗಿ ಅರ್ಪಿಸಿ, ಹೋರಾಡಿ. “ಪತ್ರ ವಾಗಲಿ, ಪುಷ್ಪ ವಾಗಲಿ ಜಲವಾಗಲಿ ನೀನು ಯಾವುದನ್ನು ನನಗೆ ಅರ್ಪಿಸುವಿಯೋ ಅದನ್ನು ನಾನು ಸ್ವೀಕರಿಸುವೆನು.” (ಗೀತಾ \enginline{IX, 26}). ನಿನಗೆ ಮಾಡಲು ಸಾಧ್ಯ ವಿಲ್ಲದೆ ಇದ್ದರೆ ಒಂದು ಒಳ್ಳೆಯ ಕೆಲಸವನ್ನೂ ಮಾಡಲು ಇದನ್ನೂ ಆಗದೆ ಇದ್ದರೆ ಭಗವಂತನಲ್ಲಿ ಶರಣಾಗು. “ದೇವರು ಎಲ್ಲರ ಹೃದಯದಲ್ಲಿಯೂ ನೆಲೆಸಿ ಅವರನ್ನು ಯಂತ್ರದಂತೆ ಚಲಿಸುತ್ತಿರುವನು. ನೀನು ನಿನ್ನ ಪ್ರಾಣ, ಮನಸ್ಸುಗಳೆರಡನ್ನು ಅವನಿಗೆ ಅರ್ಪಿಸಿ ಅವನಲ್ಲಿ ಶರಣಾಗು” (ಗೀತಾ \enginline{XVIII, 61–62}).

ಶ‍್ರೀಕೃಷ್ಣ ಗೀತೆಯಲ್ಲಿ ಪ್ರೀತಿಯ ವಿಷಯವಾಗಿ ಬೋಧಿಸಿದ ಕೆಲವು ಭಾವನೆಗಳು ಇವು. ಇತರ ಪವಿತ್ರ ಗ್ರಂಥಗಳಲ್ಲಿಯೂ ಪ್ರೇಮದ ಮೇಲೆ ಬೋಧನೆಗಳಿವೆ. ಬುದ್ಧ ಮತ್ತು ಏಸು ಕೂಡ ಅದನ್ನು ಬೋಧಿಸಿದವರು.

ಶ‍್ರೀಕೃಷ್ಣನ ಜೀವನದ ವಿಷಯವಾಗಿ ಕೆಲವು ಮಾತುಗಳು. ಕೃಷ್ಣನ ಜೀವನಕ್ಕೂ ಜೀಸಸ್ಸನ ಜೀವನಕ್ಕೂ ಹಲವು ಸಾಮ್ಯಗಳಿವೆ. ಯಾರು ಯಾರಿಂದ ತೆಗೆದುಕೊಂಡರು ಎಂಬ ವಿಷಯದಲ್ಲಿ ಚರ್ಚೆ ನಡೆಯುತ್ತಿದೆ. ಎರಡು ಕಡೆಯೂ ದುರಾತ್ಮರಾದ ರಾಜರಿದ್ದರು. ಇಬ್ಬರೂ ಗೊಂತಿನಲ್ಲಿ ಹುಟ್ಟಿದರು. ಎರಡು ಸಂದರ್ಭಗಳಲ್ಲಿಯೂ ಮಾತಾಪಿತರು ಬದ್ಧ ರಾಗಿದ್ದರು. ಇಬ್ಬರನ್ನೂ ದೇವದೂತರು ಬಿಡಿಸಿದರು. ಎರಡು ಸಂದರ್ಭಗಳಲ್ಲಿಯೂ ಆ ವರುಷ ಹುಟ್ಟಿದ ಮಕ್ಕಳನ್ನೆಲ್ಲಾ ಕೊಂದರು. ಬಾಲ್ಯ ಒಂದೇ. ಕೊನೆಗೆ ಇಬ್ಬರೂ ಹತರಾದರು. ಶ‍್ರೀಕೃಷ್ಣನು ಆಕಸ್ಮಿಕವಾಗಿ ಸತ್ತನು. ತನ್ನನ್ನು ಕೊಂದವನನ್ನೇ ವೈಕುಂಠಕ್ಕೆ ಒಯ್ದನು. ಕ್ರಿಸ್ತನೂ ಹತನಾದನು ಮತ್ತು ಡಕಾಯಿತನನ್ನೇ ಕ್ಷಮಿಸಿ ಅವನನ್ನು ಸ್ವರ್ಗಕ್ಕೆ ಒಯ್ದನು.

ನ್ಯೂ ಟೆಸ್ಟಮೆಂಟ್​ ಮತ್ತು ಗೀತೆ ಬೋಧನೆಯಲ್ಲಿ ಹಲವು ಸಮಾನ ಭಾವನೆಗಳು ಬರುತ್ತವೆ. ಮಾನವ ಭಾವನೆ ಒಂದೇ ಜಾಡನ್ನು ಹಿಡಿಯುವುದು. ಶ‍್ರೀಕೃಷ್ಣನ ಮಾತಿನಲ್ಲೇ ನಿಮಗೆ ಉತ್ತರವನ್ನು ಹೇಳುತ್ತೇನೆ. “ಎಂದು ಧರ್ಮ ಕ್ಷೀಣವಾಗಿ ಅಧರ್ಮ ಹೆಚ್ಚುವುದೋ ಆಗ ನಾನು ಪುನಃ ಬರುವೆನು” (ಗೀತಾ \enginline{IV, 8}). “ಎಂದು ಒಬ್ಬ ಮಹಾವ್ಯಕ್ತಿ ಮಾನವನನ್ನು ಉದ್ಧರಿಸಲು ಯತ್ನಿಸುತ್ತಿರುವನೋ ನಾನೇ ಅವನು ಎಂದು ತಿಳಿದುಕೋ. ನನ್ನನ್ನೇ ಪೂಜಿಸು” (ಗೀತಾ \enginline{X, 41}).

ಅವನು ಏಸುವಿನಂತೆ ಅಥವಾ ಬುದ್ಧನಂತೆ ಬಂದಿದ್ದರೆ ಏಕೆ ಇಷ್ಟೊಂದು ಪಂಥ\-ಗಳಿರುವುದು? ಆದರೆ ಅವನ ಬೋಧನೆಯನ್ನು ಮಾತ್ರ ಅನುಸರಿಸಬೇಕಾಗಿದೆ. ಹಿಂದೂ ಭಕ್ತನು ದೇವರೊಬ್ಬನೇ ಕ್ರಿಸ್ತನಂತೆ, ಕೃಷ್ಣನಂತೆ, ಬುದ್ಧನಂತೆ ಮತ್ತು ಎಲ್ಲಾ ಗುರುಗಳಂತೆ ಬರುವುದು ಎನ್ನುವನು. ಹಾಗೆಯೇ ಹಿಂದೂ ತಾತ್ತ್ವಿಕನೂ ಹೇಳುವನು. ಅವರು ಮಹಾವ್ಯಕ್ತಿಗಳು, ಆಗಲೇ ಮುಕ್ತಾತ್ಮರು. ಅವರು ಮುಕ್ತರಾದರೂ, ಜಗತ್ತಿನ ಜನ ಬಂಧನದಲ್ಲಿ ನರಳು\-ತ್ತಿರುವಾಗ, ಅವರು ತಮ್ಮ ಮುಕ್ತಿಯನ್ನು ಲೆಕ್ಕಿಸಲಿಲ್ಲ. ಅವರು ಪುನಃ ಪುನಃ ದೇಹವನ್ನು ಧರಿಸಿ ಬಂದು ಮಾನವನಿಗೆ ಸಹಾಯ ಮಾಡಲು ಯತ್ನಿಸುವರು. ಅವರಿಗೆ ಬಾಲ್ಯದಿಂದಲೂ ತಾವಾರು ಮತ್ತು ತಾವು ಏತಕ್ಕೆ ಬಂದಿರುವೆವು ಎಂಬುದು ಗೊತ್ತಿರುತ್ತದೆ. ಅವರು ನಮ್ಮಂತೆ ಬದ್ಧರಾಗಿ ಬರುವುದಿಲ್ಲ. ಅವರು ತಾವೇ ಇಚ್ಛಿಸಿ ಬರುವರು. ಅವರಲ್ಲಿ ಅಷ್ಟೊಂದು ಆಧ್ಯಾತ್ಮಿಕ ಶಕ್ತಿ ಇರುವುದರಿಂದ ಹಾಗೆ ಬರದೆ ಇರಲು ಸಾಧ್ಯವೇ ಇಲ್ಲ. ನಾವು ಅದನ್ನು ತಡೆಯಲಾರೆವು. ಮಾನವ ಕೋಟಿಯನ್ನೇ ಆಧ್ಯಾತ್ಮಿಕ ಪ್ರವಾಹದ ಕೇಂದ್ರಕ್ಕೆ ಅವರು ಸೆಳೆಯುವರು. ಈ ಸ್ಪಂದನ ಮುಂದೆ ಸಾಗುತ್ತ ಹೋಗುವುದು. ಏಕೆಂದರೆ ಇದು ಒಬ್ಬ ಮಹಾವ್ಯಕ್ತಿಯಿಂದ ಪ್ರಾರಂಭವಾಯಿತು. ಮಾನವರೆಲ್ಲಾ ಮುಕ್ತರಾಗಿ ಈ ನಮ್ಮ ಪೃಥ್ವಿಯ ಮೇಲಿನ ಆಟವೆಲ್ಲ ನಿಲ್ಲುವವರೆಗೆ ಇದು ಸಾಗುತ್ತಿರುವುದು.

ನಾವು ಅಧ್ಯಯನ ಮಾಡುತ್ತಿರುವ ಮಹಾಗುರುಗಳಿಗೆ ಜಯವಾಗಲಿ, ಪ್ರಪಂಚದ ಜೀವಂತ ದೇವರುಗಳು ಅವರು. ನಾವು ಪೂಜಿಸುವುದಕ್ಕೆ ಯೋಗ್ಯವಾದ ವ್ಯಕ್ತಿಗಳು ಅವರು. ದೇವರು ನನ್ನ ಬಳಿಗೆ ಬಂದರೆ, ಅವನು ಒಂದು ಮಾನವಾಕಾರವನ್ನು ಧರಿಸಿದ್ದರೆ ಮಾತ್ರ ಅವನನ್ನು ಗುರುತು ಹಿಡಿಯಬಹುದು. ಅವನು ಎಲ್ಲೆಲ್ಲಿ ಯೂ ಇರುವನು. ಆದರೆ ನಾವು ಅವನನ್ನು ನೋಡಿರುವೆವೆ? ಅವನನ್ನು ಮಾನವ ನಂತೆ ಮಾತ್ರ ನಾವು ಅರ್ಥಮಾಡಿಕೊಳ್ಳಬಲ್ಲೆವು. ಮಾನವ ಮತ್ತು ಪ್ರಾಣಿಗಳಲ್ಲಿ ದೇವರ ಆವಿರ್ಭಾವಗಳಿವೆ. ಆದರೆ ಈ ಮಾನವ ಕೋಟಿಯ ಬೋಧಕರಾದರೊ ನಾಯಕರು, ಗುರುಗಳು. ಆದಕಾರಣ ಯಾರ ಪಾದಕಮಲಗಳನ್ನು ದೇವತೆಗಳೂ ಪೂಜಿಸುತ್ತಿರುವರೋ ಅಂತಹ ನಿನಗೆ ನಮಸ್ಕಾರ, ಮಾನವ ಕೋಟಿಯ ನಾಯಕರೇ ನಿಮಗೆ ನಮಸ್ಕಾರ, ಮಹಾಗುರುಗಳೇ ನಿಮಗೆ ನಮಸ್ಕಾರ, ನಾಯಕರೇ ಎಂದೆಂದಿಗೂ ನಿಮಗೆ ನಮ್ಮ ನಮಸ್ಕಾರಗಳು.

