
\chapter[ವಿಶ್ವ ಮತ್ತು ಜೀವ ]{ವಿಶ್ವ ಮತ್ತು ಜೀವ \protect\footnote{\engfoot{C.W. Vol. V, P. 255}}}

ಪ್ರಕೃತಿಯಲ್ಲಿರುವ ಸರ್ವವೂ ಸೂಕ್ಷ್ಮಬೀಜಾವಸ್ಥೆಯಿಂದ ಬಂದು ಸ್ಥೂಲವಾಗಿ ಕೆಲವು\break ಕಾಲವಿದ್ದು ಅನಂತರ ಸೂಕ್ಷ್ಮಸ್ಥಿತಿಗೆ ಹೋಗುವುವು. ಉದಾಹರಣೆಗೆ ನಮ್ಮ ಪೃಥ್ವಿ ಒಂದು ಅಸ್ಪಷ್ಟವಾದ ನೀಹಾರಿಕಾ ಸ್ಥಿತಿಯಿಂದ ಬಂದು ತಣ್ಣಗಾಗಿ ತಣ್ಣಗಾಗಿ ಈಗ ನಾವು\break ವಾಸಿಸುತ್ತಿರುವ ಗ್ರಹವಾಗಿದೆ. ಕ್ರಮೇಣ ಅದು ಭವಿಷ್ಯದಲ್ಲಿ ಚೂರು ಚೂರಾಗಿ ತನ್ನ ಹಿಂದಿನ ನೀಹಾರಿಕಾ ಸ್ಥಿತಿಗೆ ಹೋಗುವುದು. ಇದು ವಿಶ್ವದಲ್ಲಿ ಈಗ ಆಗುತ್ತಿದೆ. ಅನಾದಿಕಾಲದಿಂದಲೂ\break ಹೀಗೆಯೇ ಆಗುತ್ತಿರುವುದು. ಇದೇ ಮಾನವನ ಪೂರ್ಣ ಇತಿಹಾಸ; ವಿಶ್ವದ ಪೂರ್ಣ\break ಇತಿಹಾಸ; ಜೀವನದ ಪೂರ್ಣ ಇತಿಹಾಸ.

ಪ್ರತಿಯೊಂದು ವಿಕಾಸವೂ ಒಂದು ಸಂಕೋಚನ \enginline{(Involution)} ದಿಂದ ಬಂದಿದೆ. ಮರವೆಲ್ಲ ಅದಕ್ಕೆ ಕಾರಣವಾದ ಬೀಜದಲ್ಲಿ ಅಂತರ್ಗತವಾಗಿದೆ. ಇಡೀ ಮನುಷ್ಯನು\break ಒಂದು ಪ್ರೋಟೋಪ್ಲಾಸಮ್ಮದಲ್ಲಿರುವನು. ವಿಶ್ವವೆಲ್ಲ ಸೂಕ್ಷ್ಮವಾದ ಅವ್ಯಕ್ತದಲ್ಲಿದೆ. ಪ್ರತಿಯೊಂದೂ ಕಾರಣ ಸ್ಥಿತಿಯಲ್ಲಿ ಸೂಕ್ಷ್ಮವಾಗಿರುವುದು. ವಿಕಾಸ ಎಂದರೆ ಸೂಕ್ಷ್ಮದಿಂದ ಕ್ರಮವಾಗಿ ಸ್ಥೂಲವಾದ ಆಕಾರಗಳು ವ್ಯಕ್ತವಾಗುವುದು. ಆದರೆ ಪ್ರತಿಯೊಂದು ವಿಕಾಸದ ಹಿಂದೆಯೂ ಒಂದು ಸಂಕೋಚನ ಇದ್ದೇ ಇರುವುದು. ಇಡೀ ವಿಶ್ವವು ವ್ಯಕ್ತವಾಗುವುದಕ್ಕೆ ಮೊದಲು ಒಂದು ಅವ್ಯಕ್ತ ಸ್ಥಿತಿಯಲ್ಲಿದ್ದಿರಬೇಕು. ಕಡೆಗೆ ಪುನಃ ಅದು ಅವ್ಯಕ್ತ ಸ್ಥಿತಿಗೆ ಹೋಗುವುದು. ಉದಾಹರಣೆಗೆ ಒಂದು ಸಣ್ಣ ಸಸಿಯನ್ನು ತೆಗೆದುಕೊಳ್ಳಿ. ಅದರ ಜೀವನ ಪರಿಪೂರ್ಣವಾಗಬೇಕಾದರೆ ಎರಡು ಅಂಶಗಳು ಬೇಕು. ಅದು ಜನಿಸಬೇಕು ಮತ್ತು ಬೆಳೆಯಬೇಕು, ಕ್ಷಯಿಸಬೇಕು ಮತ್ತು ನಾಶವಾಗಬೇಕು. ಇವೆಲ್ಲ ಒಟ್ಟಿಗೆ ಸೇರಿದರೆ ಸಸ್ಯಜೀವನವಾಗುವುದು. ಈ ಸಸ್ಯಜೀವನವೆಂಬುದನ್ನು ಜೀವನದ ಅನಂತ ಸರಪಳಿಯ ಒಂದು ಕೊಂಡಿ ಎಂದು ಭಾವಿಸಬಹುದು. ಆಗ ಆದಿಯಲ್ಲಿ ಪ್ರೋಟೋಪ್ಲಾಸಮ್​, ಅಂತ್ಯದಲ್ಲಿ ಪೂರ್ಣತೆಯನ್ನು ಪಡೆದ ಮಾನವನಿರುವನು. ಮನುಷ್ಯನು ಆ ಸರಪಳಿಯಲ್ಲಿ ಒಂದು ಕೊಂಡಿ. ಹಲವು ಪ್ರಾಣಿಗಳು, ಸಸ್ಯಗಳು, ಈ ಸರಪಳಿಯ ಬೇರೆ ಬೇರೆ ಕೊಂಡಿಗಳು. ಈಗ ಜೀವನವು ಪ್ರಾರಂಭವಾದ ಸೂಕ್ಷ್ಮವಸ್ತುಗಳ ಮೂಲಕ್ಕೆ ಹೋಗಿ ನೋಡಿ. ಇಡೀ ಜೀವನದ ವಿಕಾಸವನ್ನು ಸಮಗ್ರ ದೃಷ್ಟಿಯಿಂದ ನೋಡಿದಾಗ ಪ್ರತಿಯೊಂದು ವಿಕಾಸವೂ ಅದರ ಹಿಂದೆ ಮೊದಲೇ ಇದ್ದ ಯಾವುದೋ ಒಂದರ ವಿಕಾಸ ಎಂದು ಗೊತ್ತಾಗುವುದು. ವಿಕಾಸವು ಎಲ್ಲಿ ಪ್ರಾರಂಭವಾಗುತ್ತದೆಯೋ ಅಲ್ಲಿಯೇ ಅದು ಕೊನೆಗೊಳ್ಳುತ್ತದೆ. ಈ ಪ್ರಪಂಚದ ಕೊನೆಯಲ್ಲಿ ಬರುವುದು ಚೈತನ್ಯ ಅಲ್ಲವೆ? ವಿಕಾಸವಾದಿಗಳ ದೃಷ್ಟಿಯಲ್ಲಿ ಕೊನೆಗೆ ಬರುವುದು ಚೈತನ್ಯ. ಹಾಗಾದರೆ ಅದೇ ಕಾರಣವೂ ಆಗಬೇಕಾಯಿತು. ಸೃಷ್ಟಿಯ ಆದಿಯಲ್ಲೂ ಅದು ಇರಬೇಕಾಯಿತು. ಆದಿಯಲ್ಲಿ ಅದು ಸಂಕೋಚನಗೊಂಡಿರುವುದು. ಕೊನೆಗೆ ಅದು ವಿಕಾಸವಾಗುವುದು. ಜಗತ್ತಿನಲ್ಲಿ ವ್ಯಕ್ತವಾಗಿರುವ ಚೈತನ್ಯದ ಮೊತ್ತವೆಲ್ಲ ಮೊದಲು ಸಂಕೋಚನಗೊಂಡಿದ್ದು ಈಗ ಅದು ವಿಕಾಸಗೊಳ್ಳುತ್ತಿರುವ ವಿಶ್ವಚೈತನ್ಯವಾಗಿರಬೇಕು. ಈ ವಿಶ್ವಚೈತನ್ಯವನ್ನೇ ನಾವು ದೇವರು ಎನ್ನುವುದು. ನಮ್ಮ ಶಾಸ್ತ್ರಗಳು ಸಾರುವಂತೆ ನಾವು ಬಂದುದು ಅಲ್ಲಿಂದಲೇ, ಮತ್ತು ಕೊನೆಗೆ ಹೋಗುತ್ತಿರುವುದು ಅಲ್ಲಿಗೇ, ನೀವು ಅದನ್ನು ಯಾವ ಹೆಸರಿನಿಂದ ಬೇಕಾದರೂ ಕರೆಯಬಹುದು; ಆದರೆ ಆದಿಯಲ್ಲಿ ಈ ಅನಂತ ವಿಶ್ವಚೈತನ್ಯವು ಇರಲಿಲ್ಲವೆಂದು ನೀವು ಹೇಳಲಾರಿರಿ.

ಸಂಯುಕ್ತ ವಸ್ತು ಹೇಗೆ ಉಂಟಾಯಿತು? ಸಂಯುಕ್ತ ವಸ್ತು ಎಂದರೆ ಹಲವು\break ಕಾರಣಗಳು ಸೇರಿ ಒಂದು ಕಾರ್ಯವಾಗಿರುವುದು. ಈ ಸಂಯುಕ್ತ ವಸ್ತುಗಳು ಕಾರ್ಯಕಾರಣ ನಿಯಮದ ವಲಯದೊಳಗೆ ಮಾತ್ರ ಇರಬಲ್ಲವು. ಕಾರ್ಯಕಾರಣ ನಿಯಮಗಳಿ\-ಗನುಗುಣವಾಗಿ ಸಂಯೋಗಗಳು ಮತ್ತು ಮಿಶ್ರಣಗಳು ಇರುತ್ತವೆ. ಇವುಗಳಾಚೆ ಸಂಯೋಗ ಅಸಾಧ್ಯ. ಏಕೆಂದರೆ ಅಲ್ಲಿ ಯಾವ ನಿಯಮವೂ ಜಾರಿಯಲ್ಲಿ ಇಲ್ಲ. ನಿಯಮವು\break ಜಾರಿಯಲ್ಲಿರುವುದು ನಮ್ಮ ಇಂದ್ರಿಯಗಳಿಗೆ ಮತ್ತು ಮನಸ್ಸಿಗೆ ಸಂಬಂಧಪಟ್ಟ ಜಗತ್ತಿನಲ್ಲಿ ಮಾತ್ರ; ಇವುಗಳಾಚೆ ಯಾವ ನಿಯಮವೂ ಇಲ್ಲ. ನಾವು ಯಾವುದನ್ನು ಗ್ರಹಿಸುತ್ತೇವೊ ಮತ್ತು ಕಲ್ಪಿಸುತ್ತೇವೊ ಅದೇ ನಮ್ಮ ಜಗತ್ತು. ಪ್ರತ್ಯಕ್ಷವಾಗಿ ನಮ್ಮ ಇಂದ್ರಿಯಕ್ಕೆ ನಿಲುಕುವುದನ್ನು ಮಾತ್ರ ನಾವು ಗ್ರಹಿಸುತ್ತೇವೆ. ನಮ್ಮ ಮನಸ್ಸಿನಲ್ಲಿರುವುದನ್ನು ಮಾತ್ರ ನಾವು ಕಲ್ಪಿಸಿಕೊಳ್ಳುತ್ತೇವೆ. ನಮ್ಮ ದೇಹಕ್ಕೆ ಅತೀತವಾಗಿರುವುದು, ಇಂದ್ರಿಯಗಳಿಗೂ ಅತೀತವಾಗಿದೆ. ಮನಸ್ಸಿಗೆ ಅತೀತವಾಗಿರುವುದು ಕಲ್ಪನೆಗೂ ಅತೀತವಾಗಿದೆ. ಆದಕಾರಣ ಅವು ಕಾರ್ಯಕಾರಣ ನಿಯಮಗಳಿಗೂ ಅತೀತವಾಗಿರಬೇಕು. ಮಾನವನ ಆತ್ಮವು ಕಾರ್ಯಕಾರಣ ನಿಯಮಕ್ಕೆ ಅತೀತವಾಗಿರುವುದರಿಂದ ಅದು ಒಂದು ಸಂಯುಕ್ತ ವಸ್ತುವಲ್ಲ. ಅದು ಯಾವ ಒಂದು ಕಾರ್ಯದ ಪರಿಣಾಮವೂ ಅಲ್ಲ; ಆದಕಾರಣ ಅದು ನಿತ್ಯಮುಕ್ತವಾಗಿರಬೇಕು. ನಿಯಮದಲ್ಲಿರುವುದನ್ನೆಲ್ಲ ಅದು ಆಳುವ ಸ್ಥಿತಿಯಲ್ಲಿರಬೇಕು. ಅದೊಂದು ಸಂಯುಕ್ತ\break ವಸ್ತುವಾಗದೆ ಇರುವುದರಿಂದ ಅದು ಸಾಯಲಾರದು. ಏಕೆಂದರೆ ಸಾವು ಎಂದರೆ ಕಾರಣಕ್ಕೆ ಮರಳುವುದು ಎಂದರ್ಥ. ಇದಕ್ಕೆ ನಾಶವಾಗಲು ಸಾಧ್ಯವಾಗದೆ ಇರುವುದರಿಂದ ಇದು ಹುಟ್ಟಲೂ ಆಗದು. ಏಕೆಂದರೆ ಜನನಮರಣಗಳೆರಡೂ ಒಂದೇ ವಸ್ತುವಿನ ಆವಿರ್ಭಾವಗಳು. ಆದಕಾರಣ ಆತ್ಮವು ಜನನಮರಣಾತೀತ. ನೀವು ಎಂದೂ ಹುಟ್ಟಲಿಲ್ಲ. ನೀವು\break ಎಂದೂ ಸಾಯುವುದೂ ಇಲ್ಲ. ಜನನಮರಣಗಳು ದೇಹಕ್ಕೆ ಮಾತ್ರ ಅನ್ವಯಿಸುವುವು.

ಅದ್ವೈತವು ಇರುವುದೊಂದೇ ಜಗತ್ತು ಎನ್ನುವುದು. ಸ್ಥೂಲವೋ ಸೂಕ್ಷ್ಮವೋ ಅದೆಲ್ಲ ಇಲ್ಲೇ ಇರುವುದು. ಕಾರ್ಯಕಾರಣಗಳೆರಡೂ ಇಲ್ಲೇ ಇರುವುವು. ಇದಕ್ಕೆ ವಿವರಣೆಯೂ ಇಲ್ಲೇ ಇರುವುದು. ಸಮಷ್ಟಿಯಲ್ಲಿ ಏನಿದೆಯೊ ಅದೇ ವ್ಯಷ್ಟಿಯಲ್ಲಿಯೂ ಇರುವುದು ನಾವು ನಮ್ಮ ಸ್ವರೂಪ ಜ್ಞಾನದಿಂದ ವಿಶ್ವದ ಸ್ವರೂಪ ಜ್ಞಾನವನ್ನು ಪಡೆಯುವೆವು. ಯಾವುದು ನಮಗೆ ಸತ್ಯವೊ ಅದೇ ವಿಶ್ವಕ್ಕೂ ಅನ್ವಯಿಸುವುದು. ಸ್ವರ್ಗ ಮುಂತಾದ ಲೋಕಗಳೆಲ್ಲ, ಅವು ಸತ್ಯವಾಗಿದ್ದರೂ, ಈ ವಿಶ್ವದಲ್ಲಿಯೇ ಇರುವುವು. ಇವುಗಳೆಲ್ಲ ಸೇರಿ ಒಂದು ಸಮಷ್ಟಿಯಾಗುವುದು. ಆದಕಾರಣ ಸಮಷ್ಟಿಯ ಪ್ರಥಮ ಭಾವನೆಯೆ ಜಗತ್ತಿನಲ್ಲಿರುವ ಪ್ರತಿಯೊಂದು ಸೂಕ್ಷ್ಮ ವಸ್ತುವಿನ ಮೊತ್ತ. ನಮ್ಮಲ್ಲಿ ಪ್ರತಿಯೊಬ್ಬರೂ ಇದರಲ್ಲಿ ಒಂದು ಅಂಶವಾಗಿರುವೆವು. ಆವಿರ್ಭಾವದ ದೃಷ್ಟಿಯಿಂದ ನಾವೆಲ್ಲಾ ಬೇರೆಯಾಗಿ ಕಾಣುವೆವು. ಆದರೆ ನಿಜವಾಗಿ\break ನಾವೆಲ್ಲ ಒಂದೇ. ನಾವು ಸಮಷ್ಟಿಯಿಂದ ಬೇರೆ ಎಂದು ಭಾವಿಸಿದಷ್ಟೂ ದುಃಖಿಗಳಾಗುವೆವು. ಆದಕಾರಣ ಅದ್ವೈತವೇ ನೀತಿಗೆ ತಳಹದಿ.

