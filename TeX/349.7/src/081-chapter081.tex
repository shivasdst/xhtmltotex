
\chapter[ಪ್ರಾಚ್ಯ ಮಹಿಳೆ ]{ಪ್ರಾಚ್ಯ ಮಹಿಳೆ \protect\footnote{\engfoot{C.W. Vol. VIII, P.198}}}

\centerline{\textbf{(1893ರ ಸೆಪ್ಟೆಂಬರ್​ 22ರಂದು ‘ಚಿಕಾಗೋ ಡೈಲಿ’ಯಲ್ಲಿ ವರದಿಯಾದ ಉಪನ್ಯಾಸ)}\footnote{ ನಿನ್ನೆ ಮಧ್ಯಾಹ್ನ ಸಭಾಂಗಣ–7ರಲ್ಲಿ ಎಷ್ಟು ಜನ ಹಿಡಿಯಬಹುದೊ ಅಷ್ಟು ಜನ ಮಹಿಳೆಯರು, ಪ್ರಾಚ್ಯ ದೇಶದ ಸಹೋದರಿಯ ಜೀವನ ವಿಧಾನದ ಬಗ್ಗೆ ತಿಳಿಯಲು, ಕಿಕ್ಕಿರಿದು ನೆರೆದಿದ್ದರು. ಶ‍್ರೀಮತಿ ಪಾಟರ್​ ಪಾಮರ್​ ಮತ್ತು ಶ‍್ರೀಮತಿ ಚಾರ್ಲ್ಸ್​ ಹೆನ್ರಾಟಿನ್​ ವೇದಿಕೆಯ ಮೇಲೆ ಕುಳಿತಿದ್ದರು. ಸ್ವಾಮಿ ವಿನೇಕಾನಂದರು ಪೇಟಧಾರಿಯಾಗಿ ಕುಳಿತಿದ್ದರು.}}

ಒಂದು ಜನಾಂಗದವರು ಎಷ್ಟರ ಮಟ್ಟಿಗೆ ಮುಂದುವರಿದಿದ್ದಾರೆ ಎಂಬುದಕ್ಕೆ ಅವರು\break ತಮ್ಮ ಸ್ತ್ರೀಯರನ್ನು ಹೇಗೆ ನೋಡಿಕೊಂಡರು ಎಂಬುದೇ ಒರೆಗಲ್ಲು. ಪುರಾತನ ಗ್ರೀಸಿನಲ್ಲಿ\break ಸ್ತ್ರೀ ಪುರುಷರ ಅಂತಸ್ತಿನಲ್ಲಿ ಯಾವ ವ್ಯತ್ಯಾಸವೂ ಇರಲಿಲ್ಲ. ಅವರಲ್ಲಿ ಸಮಾನತೆ\break ಪೂರ್ಣವಾಗಿತ್ತು. ಯಾವ ಹಿಂದುವೂ ವಿವಾಹವಾಗದೆ ಇದ್ದರೆ ಪುರೋಹಿತನಾಗಲಾರ.\break ಏಕೆಂದರೆ ಗಂಡಸು ಕೇವಲ ಅರ್ಧಭಾಗ ಮಾತ್ರ, ಅವನು ಪೂರ್ಣವಾಗಬೇಕಾದರೆ\break ಮದುವೆಯಾಗಬೇಕು. ಸ್ತ್ರೀತ್ವದ ಪೂರ್ಣ ಭಾವನೆಯೆಂದರೆ ಪೂರ್ಣ ಸ್ವಾತಂತ್ರ್ಯ. ಆಧುನಿಕ\break ಹಿಂದೂ ಸ್ತ್ರೀಯ ಜೀವನದ ಕೇಂದ್ರ ಭಾವನೆ ಎಂದರೆ ಅವಳ ಪಾತಿವ್ರತ್ಯ. ಪತ್ನಿಯೇ ವೃತ್ತದ ಕೇಂದ್ರ. ಅದರ ಸ್ಥಿರತೆಯೇ ಅವಳ ಪಾತಿವ್ರತ್ಯದ ಮೇಲೆ ನಿಂತಿದೆ. ಈ ಆದರ್ಶದ ಅತಿರೇಕವೇ ಕೆಲವು ವೇಳೆ ಹಿಂದೂ ವಿಧವೆಯರ ಸಹಗಮನಕ್ಕೆ ಕಾರಣವಾಗಿದೆ. ಹಿಂದೂ ನಾರಿಯರು ಧಾರ್ಮಿಕ ಮತ್ತು ಆಧ್ಯಾತ್ಮಿಕ ಪ್ರವೃತ್ತಿಯವರು. ಬಹುಶಃ ಪ್ರಪಂಚದ ಇತರ ನಾರಿಯರಿಗಿಂತ ಹೆಚ್ಚು ಎಂದು ಬೇಕಾದರೆ ಹೇಳಬಹುದು. ನಾವು ಈ ಪವಿತ್ರ ಭಾವನೆಯನ್ನು ರಕ್ಷಿಸಿಕೊಂಡು ನಮ್ಮ ಸ್ತ್ರೀಯರ ಬುದ್ಧಿಯ ಬೆಳವಣಿಗೆಗೆ ಅವಕಾಶ ಕೊಟ್ಟರೆ ಭಾರತದ ಭವಿಷ್ಯ ನಾರಿಯು ಇಡೀ ವಿಶ್ವಕ್ಕೇ ಆದರ್ಶವಾಗುವಳು.

\vskip 6pt

