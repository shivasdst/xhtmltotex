
\chapter[ಬಿಲ್ವಮಂಗಳ ]{ಬಿಲ್ವಮಂಗಳ \protect\footnote{\engfoot{C.W. Vol. I P. 485}}}

ಬಿಲ್ವಮಂಗಳ ಎನ್ನುವುದು “ಸಂತರ ಜೀವನ” ಎಂದು ಭರತಖಂಡದಲ್ಲಿ ಪ್ರಚಾರದಲ್ಲಿರುವ ಪುಸ್ತಕ ಒಂದರಲ್ಲಿ ಬರುವ ಕಥೆ. ಒಂದು ಗ್ರಾಮದಲ್ಲಿ ಒಬ್ಬ ಬ್ರಾಹ್ಮಣ ಯುವಕನು ಇದ್ದನು. ಮತ್ತೊಂದು ಹಳ್ಳಿಯ ಒಬ್ಬ ವೇಶ್ಯಾ ಸ್ತ್ರೀಯಲ್ಲಿ ಇವನು ಅನುರಕ್ತನಾದನು. ಈ ಎರಡು ಗ್ರಾಮಗಳ ಮಧ್ಯೆ ಒಂದು ದೊಡ್ಡ ನದಿ ಹರಿಯುತ್ತಿತ್ತು. ಪ್ರತಿದಿನವೂ ನದಿಯನ್ನು ದೋಣಿಯಲ್ಲಿ ದಾಟಿಕೊಂಡು ಆ ಹುಡುಗಿಯ ಮನೆಗೆ ಹೋಗುತ್ತಿದ್ದ. ಒಂದು ದಿನ ಅವನು ಅವನ ತಂದೆಯ ಶ್ರಾದ್ಧವನ್ನು ಮಾಡಬೇಕಾಯಿತು. ಆದಕಾರಣ ಅವನು ಆ ಹುಡುಗಿಯ ಮನೆಗೆ ಹೋಗಬೇಕೆಂದು ಎಷ್ಟೋ ಕಾತರನಾಗಿದ್ದರೂ, ಅದಕ್ಕಾಗಿ ಸಾಯುತ್ತಿದ್ದರೂ, ಹೋಗುವುದಕ್ಕೆ ಆಗಲಿಲ್ಲ. ಶ್ರಾದ್ಧವನ್ನು ಮಾಡಲೇಬಾಕಾಗಿತ್ತು. ಅದಕ್ಕೆ ಸಂಬಂಧಪಟ್ಟದ್ದನ್ನೆಲ್ಲಾ ಆಚರಿಸಿದ್ದಾಯಿತು. ಇದು ಹಿಂದೂಗಳ ಮನೆಯಲ್ಲಿ ಅತ್ಯಾವಶ್ಯವಾಗಿ ಆಚರಿಸಬೇಕಾದ ಕರ್ಮ. ಅದನ್ನು ಮಾಡುವಾಗ ಇವನು ಬೇಕಾದಷ್ಟು ಗೊಣಗಾಡುತ್ತಿದ್ದ. ಆದರೂ ವಿಧಿ ಯಿಲ್ಲದೇ ಮಾಡಬೇಕಾಯಿತು. ಕೊನೆಗೆ ಶ್ರಾದ್ಧ ತೀರಿತು. ಅಷ್ಟು ಹೊತ್ತಿಗೆ ಕತ್ತಲಾಯಿತು. ರಾತ್ರಿ ದೊಡ್ಡದೊಂದು ಬಿರುಗಾಳಿ ಪ್ರಾರಂಭವಾಯಿತು. ಆಗ ಮಳೆ ಯು ಧಾರಾಕಾರವಾಗಿ ಸುರಿಯತೊಡಗಿತು. ನದಿಯಲ್ಲಿ ದೊಡ್ಡ ದೊಡ್ಡ ಅಲೆಗಳು ಎದ್ದವು. ಅದನ್ನು ದಾಟುವುದು ಬಹಳ ಅಪಾಯಕರವಾಗಿತ್ತು. ಆದರೂ ಅವನು ನದಿಯ ತೀರಕ್ಕೆ ಹೊರಟನು. ಅಲ್ಲಿ ಯಾವ ದೋಣಿಯೂ ಇರಲಿಲ್ಲ. ಇದ್ದ ಒಂದು ದೋಣಿಯವರು ನದಿಯನ್ನು ದಾಟಲು ಅಂಜಿದರು. ಆದರೆ ಅವನು ಹೋಗಲೇ ಬೇಕಾಗಿತ್ತು. ಅವನ ಹೃದಯವಾದರೊ ಆ ಹುಡುಗಿಯ ಮೇಲಿನ ಪ್ರೇಮದಿಂದ ಉನ್ಮತ್ತವಾಗಿತ್ತು. ಆದಕಾರಣ ಅವನು ಹೋಗಲು ಹಟ ಮಾಡಿದನು.ನೀರಿನ ಮೇಲೆ ಒಂದು ತೊಲೆ ತೇಲಿಕೊಂಡು ಹೋಗುತ್ತಿತ್ತು. ಅದನ್ನೇ ಹಿಡಿದು ಕೊಂಡು ಅದರ ಸಹಾಯದಿಂದ ನದಿಯನ್ನು ದಾಟಿದನು. ಆಚೆಯ ತೀರವನ್ನು ಸೇರಿದ ಮೇಲೆ ತೊಲೆಯನ್ನು ಸೆಳೆದು ತೀರದಲ್ಲಿಟ್ಟನು. ಅನಂತರ ಮನೆಯೊಳಗೆ ಹೋದನು.\break ಬಾಗಿಲನ್ನು ತಟ್ಟಿದನು. ಆದರೆ ಗಾಳಿ ಭೋರ್ಗರೆದು ಬೀಸುತ್ತಿತ್ತು. ಯಾರಿಗೂ ಅವನು ಮಾಡುವ ಶಬ್ದ ಕೇಳಿಸಲಿಲ್ಲ. ಅವನು ಗೋಡೆಯ ಸುತ್ತಲೂ ಹೋಗಿ ನೋಡಿದನು. ಕೊನೆಗೆ ಗೋಡೆಯ ಮೇಲಿನಿಂದ ಒಂದು ಹಗ್ಗ ಕೆಳಗೆ ಬಿದ್ದಿರುವುದನ್ನು ಕಂಡನು. ಅವನು ಆ ಹಗ್ಗವನ್ನು ಹಿಡಿದುಕೊಂಡು “ಓ, ನನ್ನ ಪ್ರಿಯಳು ಗೋಡೆಯನ್ನು ಹತ್ತುವುದಕ್ಕೆ ಒಂದು ಹಗ್ಗವನ್ನು ಇಳಿಬಿಟ್ಟಿರುವಳು” ಎಂದು ಅದನ್ನು ಹಿಡಿದುಕೊಂಡು ಮೇಲಕ್ಕೆ ಹತ್ತಿದನು. ಹತ್ತಿ ಕೆಳಗೆ ಇಳಿಯುವಾಗ ಜಾರಿ ಕೆಳಗೆ ಬಿದ್ದನು. ಹುಡುಗಿಯು ಹೊರಗೆ ಬಂದು ನೋಡಿದಾಗ ಯುವಕನು ಪ್ರಜ್ಞೆ ಇಲ್ಲದೆ ಬಿದ್ದಿದ್ದನು. ಅವನನ್ನು ಉಪಚರಿಸಿ ಪ್ರಜ್ಞೆ ಬರುವಂತೆ ಮಾಡಿದಳು. ಅವನಿಂದ ತುಂಬಾ ದುರ್ವಾಸನೆ ಬರುತ್ತಿರುವುದನ್ನು ನೋಡಿ, “ಏನು ಸಮಾಚಾರ? ಏತಕ್ಕೆ ಇಂತಹ ದುರ್ಗಂಧ ನಿನ್ನಲ್ಲಿ? ನೀನು ಹೇಗೆ ಮನೆ ಒಳಗೆ ಬಂದೆ?” ಎಂದು ಕೇಳಿದಳು. ಅವನು ಅದಕ್ಕೆ “ನೀನು ನಿನ್ನ ಪ್ರಿಯತಮ ಹತ್ತಿ ಬರುವುದಕ್ಕೆ ಹಗ್ಗವನ್ನು ಹಾಕಿರಲಿಲ್ಲವೆ?” ಎಂದು ಕೇಳಿದನು. ಅವಳು ನಕ್ಕು “ಎಂತಹ ಪ್ರೀತಿ, ನಾವಿರುವುದು ಹಣಕ್ಕಾಗಿ! ಮೂಢ, ನೀನು ಹತ್ತಿ ಬರುವುದಕ್ಕೆ ನಿನಗೆ ಹಗ್ಗವನ್ನು ನಾನು ಇಳಿಬಿಟ್ಟಿದ್ದೆ ಎಂದು ಭಾವಿಸಿದೆಯಾ? ನೀನು ನದಿಯನ್ನು ಹೇಗೆ ದಾಟಿದೆ?” ಎಂದು ಕೇಳಿದಳು. “ನಾನೊಂದು ತೊಲೆಯನ್ನು ಹಿಡಿದುಕೊಂಡು ತೇಲಿಬಂದೆ” ಎಂದನು. ಅದನ್ನು ಹೋಗಿ ನೋಡೋಣ ಎಂದಳು ಹುಡುಗಿ. ಆ ಹಗ್ಗ ನಾಗರಹಾವಾಗಿತ್ತು. ಘೋರ ವಿಷಜಂತು. ಸ್ವಲ್ಪ ಕಚ್ಚಿದರೆ ಸಾಕು, ಮನುಷ್ಯ ಸತ್ತುಹೋಗುತ್ತಿದ್ದ. ಅದು ಒಂದು ಬಿಲದೊಳಗೆ ತನ್ನ ತಲೆಯನ್ನು ಇಟ್ಟಿತ್ತು. ಒಳಗೆ ಹೋಗಲು ಪ್ರಯತ್ನಿಸುತ್ತಿತ್ತು. ಯುವಕ ಅದರ ಬಾಲವನ್ನು ಹಿಡಿದುಕೊಂಡು ಅದನ್ನು ಒಂದು ಹಗ್ಗವೆಂದು ಭಾವಿಸಿ ದನು. ಪ್ರೇಮದ ಹುಚ್ಚು ಹಾಗೆ ಮಾಡಿಸಿತು. ಹಾವಿನ ತಲೆ ಬಿಲದೊಳಗೆ ಇದ್ದಾಗ ದೇಹ ಹೊರಗೆ ಇತ್ತು. ಅದರ ಬಾಲವನ್ನು ಹಿಡಿದು ಎಳೆದರೆ ಅದು ಬಿಲದಿಂದ ಹೊರಗೆ ಬರುವುದಿಲ್ಲ. ಈ ಯುವಕ ಹಾಗೆ ಮೇಲೆ ಹತ್ತಿ ಬಂದಿದ್ದ. ಆದರೆ ಈ ಸೆಳೆತದಿಂದ ಹಾವು ಸತ್ತಿತ್ತು. “ನಿನಗೆ ತೊಲೆ ಎಲ್ಲಿ ಸಿಕ್ಕಿತು?” ಎಂದು ಹುಡುಗಿ ಕೇಳಿದಳು. “ಅದು ನದಿಯಲ್ಲಿ ತೇಲಿಕೊಂಡು ಹೋಗುತ್ತಿತ್ತು” ಎಂದನು. ಅದೋ ಒಂದು ಕೊಳೆತು ನಾರುವ ಶವ. ನದಿ ಅದನ್ನು ಹೊಡೆದುಕೊಂಡು ಹೋಗು ತ್ತಿತ್ತು. ಈ ಯುವಕನು ಅದನ್ನು ಒಂದು ತೊಲೆ ಎಂದು ಭಾವಿಸಿದ್ದನು. ಇದರಿಂದ ಅವನು ಅಷ್ಟು ದುರ್ವಾಸನೆಯಿಂದ ಕೂಡಿದ್ದನು. ಆ ಹೆಂಗಸು ಅವನನ್ನು ನೋಡಿ ಹೇಳಿದಳು: “ನಾನೆಂದೂ ಹಿಂದೆ ಪ್ರೀತಿಯಲ್ಲಿ ನಂಬಿಕೆ ಇಟ್ಟಿರಲಿಲ್ಲ. ಈಗಲೂ ನಾನೆಂದಿಗೂ ನಂಬುವುದಿಲ್ಲ. ಆದರೆ ಇದು ಪ್ರೀತಿ ಅಲ್ಲದೇ ಇದ್ದರೆ ದೇವರು ನನ್ನನ್ನು ರಕ್ಷಿಸಬೇಕು. ಪ್ರೀತಿ ಎಂದರೇನು ನಮಗೇನೂ ಗೊತ್ತಿಲ್ಲ. ಆದರೆ ಸ್ನೇಹಿತನೆ ಆ ಪ್ರೀತಿಯನ್ನು ನನ್ನಂತಹ ಹೆಂಗಸಿಗೆ ಏಕೆ ನೀಡುವೆ? ದೇವರಿಗೆ ಏತಕ್ಕೆ ನೀನು ಅದನ್ನು ಸಮರ್ಪಿಸಬಾರದು? ನೀನು ಆಗ ಪೂರ್ಣಾತ್ಮನಾಗುವೆ.” ಯುವಕನಿಗೆ ಇದು ವಜ್ರಾಘಾತದಂತೆ ಬಿತ್ತು. ತಾತ್ಕಾಲಿಕವಾಗಿ ಇಂದ್ರಿಯಾತೀತವಾದ ಅನು ಭವವನ್ನು ಅವನು ಕ್ಷಣಕಾಲ ಅನುಭವಿಸಿದನು. ದೇವರಿರುವನೆ ಎಂದು ಅವನು ಕೇಳಿದನು. “ಹೌದು, ಹೌದು, ಸಖನೆ ದೇವರಿರುವನು” ಎಂದಳು ಅವಳು. ಆತ ಅಲ್ಲಿಂದ ಹೊರಟ. ಒಂದು ಕಾಡನ್ನು ಪ್ರವೇಶಿಸಿ ಅಳುತ್ತ ಪ್ರಾರ್ಥನೆ ಮಾಡತೊಡಗಿದ: “ದೇವರೇ ನನಗೆ ನೀನು ಬೇಕು. ನನ್ನ ಪ್ರೇಮ ಪ್ರವಾಹವನ್ನು ಕ್ಷುದ್ರ ಮಾನವ ಪಾತ್ರೆಯಲ್ಲಿ ಹುದುಗಿರಲು ಸಾಧ್ಯವಿಲ್ಲ. ನನ್ನ ಪ್ರೇಮದ ಮಹಾನದಿ ಪ್ರವೇಶಿಸ\-ಬಲ್ಲ ಮಹಾಸಾಗರವನ್ನು ನಾನು ಪ್ರೀತಿಸಬೇಕು. ನನ್ನ ಪ್ರೇಮದ ಉಗ್ರ ಪ್ರವಾಹ ಹಳ್ಳ ಕೊಳ್ಳಗಳನ್ನು ಪ್ರವೇಶಿಸಬಾರದು. ಅದಕ್ಕೆ ಅನಂತ ಸಾಗರ ಬೇಕು. ನೀನೇ ಅದು. ನೀನು ನನ್ನೆಡೆಗೆ ಬಾ.” ಹೀಗೆ ಹಲವು ವರುಷಗಳು ಅವನು ತಪಸ್ಸಿನಲ್ಲಿ ನಿರತ\-ನಾಗಿದ್ದ. ಹಲವು ವರುಷಗಳು ಹೀಗೆ ಇದ್ದು, ಆಮೇಲೆ ಜಯಶೀಲನಾಗಿರಬಲ್ಲೆ ಎಂದು ಭಾವಿಸಿ ಸಂನ್ಯಾಸಿಯಾಗಿ ಊರುಗಳನ್ನು ಪ್ರವೇಶ ಮಾಡಿದನು. ಒಂದು ದಿನ ಅವನು ಒಂದು ನದಿತೀರದಲ್ಲಿ ಸೋಪಾನ ಪಂಕ್ತಿಯ ಮೇಲೆ ಕುಳಿತಿದ್ದನು. ಆ ಊರಿನ ವರ್ತಕನ ಸುಂದರಳಾದ ಹೆಂಡತಿ ಅಲ್ಲಿಗೆ ತನ್ನ ಆಳಿನೊಂದಿಗೆ ಬಂದು ಹೋದಳು. ಅವಳು ಕಣ್ಣಿಗೆ ಬಿದ್ದೊಡನೆಯೇ ಇವನ\-ಲ್ಲಿದ್ದ ಹಳೆಯ ವಾಸನೆಗಳು ಮೇಲೆದ್ದವು. ಆ ಮನೋಹರವಾದ ಆಕಾರ ಪುನಃ ಇವನನ್ನು ಆಕರ್ಷಿಸಿತು. ಸಂನ್ಯಾಸಿ ಅವಳನ್ನು ದುರುಗುಟ್ಟಿಕೊಂಡು ನೋಡಿ ಎದ್ದು ನಿಂತು ಅವಳನ್ನೆ ಅನುಸರಿಸಿ ಮನೆಗೆ ಹೋದನು. ಸ್ವಲ್ಪ ಹೊತ್ತಿನಲ್ಲೆ ಗಂಡ ಬಂದ. ಕಾವಿಯನ್ನು ಧರಿಸಿದ ಸಂನ್ಯಾಸಿಯನ್ನು ನೋಡಿ “ದಯವಿಟ್ಟು ಒಳಗೆ ಬನ್ನಿ ಸ್ವಾಮಿ, ನಾನು ನಿಮಗೆ ಏನು ಮಾಡಬೇಕಾಗಿದೆ?” ಎಂದು ಕೇಳಿಕೊಂಡನು. ಸಂನ್ಯಾಸಿ “ನಾನೊಂದು ಭೀಕರವಾದ ವಸ್ತುವನ್ನು ಕೇಳುತ್ತೇನೆ” ಎಂದನು. ಆಗ ಗೃಹಸ್ಥನು, “ಏನನ್ನು ಬೇಕಾದರೂ ಕೇಳಿ ಸ್ವಾಮಿ, ನಾನು ಗೃಹಸ್ಥ. ತಾವು ಏನನ್ನು ಕೇಳಿದರೂ ಕೊಡುತ್ತೇನೆ” ಎಂದನು. “ನಾನು ನಿನ್ನ ಹೆಂಡತಿಯನ್ನು ನೋಡ ಬೇಕು” ಎಂದು ಸಂನ್ಯಾಸಿ ಹೇಳಿದನು. ಆಗ ಆತ ದೇವರೆ ಏನಿದು! ನಾನು ಶುದ್ಧ ವಾಗಿರುವೆನು, ನನ್ನ ಹೆಂಡತಿ ಶುದ್ಧವಾಗಿರುವಳು. ದೇವರು ಎಲ್ಲರನ್ನೂ ರಕ್ಷಿಸುತ್ತಿರುವನು. ಒಳ್ಳೆಯದು, “ಬನ್ನಿ ಸ್ವಾಮಿ ಒಳಗೆ” ಎಂದನು. ಆತ ಒಳಗೆ ಬಂದ ಮೇಲೆ ಅವನು ತನ್ನ ಹೆಂಡತಿಯನ್ನು ಇವರಿಗೆ ಪರಿಚಯ ಮಾಡಿಸಿದನು. ಆಕೆ ನಾನು ನಿಮಗೆ ಏನನ್ನು ಮಾಡಬೇಕು, ಎಂದು ಕೇಳಿಕೊಂಡಳು. ಅವಳನ್ನೆ ಚೆನ್ನಾಗಿ ನೋಡಿ, “ತಾಯಿ, ನಿನ್ನ ತಲೆಯಲ್ಲಿರುವ ಎರಡು ಸೂಜಿಗಳನ್ನು ಕೊಡುವೆಯ?” ಎಂದು ಕೇಳಿದನು. ತೆಗೆದುಕೊಳ್ಳಿ ಎಂದು ಅವಳು ಕೊಟ್ಟಳು. ಅದನ್ನು ಸಂನ್ಯಾಸಿ ತೆಗೆದುಕೊಂಡು ತಕ್ಷಣವೇ ತನ್ನ ಕಣ್ಣಿನೊಳಗೆ ಅದನ್ನು ಚುಚ್ಚಿಕೊಂಡನು. “ಪಾಪಿ ಕಣ್ಣುಗಳೇ ನಾಶವಾಗಿ. ಇನ್ನು ಮೇಲೆ ನೀವು ಯಾವತ್ತೂ ಮಾಂಸದಿಂದ ಮಾಡಿದ ಆಕಾರವನ್ನು ನೋಡಲಾರಿರಿ. ನೀವೇನಾದರೂ ನೋಡಬೇಕಾದರೆ ಬೃಂದಾವನದ ಗೋಪಾಲನನ್ನು ಅಂತರ್​ ಚಕ್ಷುಗಳಿಂದ ಮಾತ್ರ ನೋಡಬೇಕು. ಕಣ್ಣುಗಳೇ ನಿಮಗೆ ಇನ್ನು ಇಷ್ಟೇ ಇರುವುದು” ಎಂದನು. ಅವನು ಪುನಃ ಕಾಡಿಗೆ ಹಿಂತಿರುಗಿ ಹೋದನು. ಅವನು ಪುನಃ ಬೇಕಾದಷ್ಟು ಅತ್ತನು. ಅವನಲ್ಲಿದ್ದ ಈ ಪ್ರೇಮ ಪ್ರವಾಹ ಸತ್ಯಸಾಗರವನ್ನು ಸೇರಲು ಹಾತೊರೆಯುತ್ತಿತ್ತು. ಕೊನೆಗೆ ಅವನು ಜಯ ಪ್ರದನಾದನು. ಅವನು ತನ್ನ ಪ್ರೇಮದ ನದಿಗೆ, ತನ್ನ ಜೀವನಕ್ಕೆ, ಸರಿಯಾದ ಮಾರ್ಗವನ್ನು ತೋರಿದನು. ಅದು ಗೋಪಾಲನ ಬಳಿಗೆ ಬಂದಿತು. ಅವನು ಶ‍್ರೀಕೃಷ್ಣನಂತೆ ದೇವರನ್ನು ನೋಡಿದನು ಎಂದು ಕಥೆ ಹೇಳುವುದು. ಆಗ ಕಣ್ಣನ್ನು ಕಳೆದುಕೊಂಡದಕ್ಕೆ ಪುನಃ ವ್ಯಥೆಪಟ್ಟನು. ಅವನನ್ನು ಆಂತರಿಕವಾಗಿ ಮಾತ್ರ ನೋಡಲು ಸಾಧ್ಯವಾಯಿತು. ಅವನು ಭಕ್ತಿಯ ಮೇಲೆ ಅತ್ಯಂತ ಸುಂದರವಾದ ಶ್ಲೋಕಗಳನ್ನು ರಚಿಸಿದನು. ಸಂಸ್ಕೃತ ಗ್ರಂಥಗಳಲ್ಲೆಲ್ಲ ಅದನ್ನು ಬರೆದವರು ಮೊದಲು ಗುರುಗಳನ್ನು ಸ್ತುತಿಸುತ್ತಾರೆ. ಹಾಗೆಯೆ ಈತ ಹಿಂದೆ ತಾನು ಪ್ರೀತಿಸುತ್ತಿದ್ದ ಹೆಂಗಸನ್ನು ತನ್ನ ಗುರುವಿನಂತೆ ಅಲ್ಲಿ ಸ್ತುತಿಸುತ್ತಾನೆ.

