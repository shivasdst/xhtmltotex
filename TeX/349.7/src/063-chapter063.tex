
\chapter[ಸತ್ಯ ಮತ್ತು ಛಾಯೆ ]{ಸತ್ಯ ಮತ್ತು ಛಾಯೆ \protect\footnote{\engfoot{C.W. Vol. VIII, P. 237}}}

\centerline{(೧೯೦೦ರ ಮಾರ್ಚ್​ ೮ರಂದು ಓಕ್ಲ್ಯಾಂಡಿನಲ್ಲಿ ನೀಡಿದ ಉಪನ್ಯಾಸದ ವರದಿ)}

ಈ ಪ್ರಪಂಚದಲ್ಲಿ ಎಂದಿಗೂ ಬದಲಾಗುವ ಯಾವುದಾದರೂ ಒಂದು ವಸ್ತು ಇದೆಯೇ ಎಂದು ಮಾನವನ ಆತ್ಮವು ಅರಸುತ್ತಿದೆ; ಯಾವಾಗಲೂ ಅದು ಅರಸುತ್ತಲೇ ಇದೆ. ಅದಕ್ಕೆ ಎಂದಿಗೂ ತೃಪ್ತಿಯಿಲ್ಲ. ಐಶ್ವರ್ಯ, ಆಕಾಂಕ್ಷೆ, ಹಸಿವು, ಇವನ್ನು ಎಷ್ಟು ತೃಪ್ತಿಪಡಿಸಿದರೂ ಅವು ಇನ್ನೂ ವೃದ್ಧಿಯಾಗುತ್ತಿರುವುವು. ಇವನ್ನು ಪಡೆದ ಮೇಲೂ ಮನುಷ್ಯನಿಗೆ ತೃಪ್ತಿಯಿಲ್ಲ. ಅವಿಕಾರಿಯಾದುದನ್ನು ಪಡೆಯಬೇಕೆಂಬ ಆಸೆಯನ್ನು ತೃಪ್ತಿಪಡಿಸುವುದೇ ಧರ್ಮವಿಜ್ಞಾನ. ಭಿನ್ನ ಭಿನ್ನ ಭಾಷೆಗಳು ಮತ್ತು ಆಚಾರಗಳು ಒಂದೇ ವಿಷಯವನ್ನು ಬೋಧಿಸುವುವು-ಅದೇ ‘ಮಾನವನ ಆತ್ಮ ಮಾತ್ರ ಸತ್ಯ’ ಎನ್ನುವುದು.

ಎರಡು ಜಗತ್ತುಗಳಿವೆ ಎಂದು ವೇದಾಂತತತ್ತ್ವ ಹೇಳುವುದು. ಒಂದು ಬಾಹ್ಯ ಜಗತ್ತು, ಮತ್ತೊಂದು ಆಂತರಿಕ ಜಗತ್ತು ಅಥವಾ ಆಲೋಚನಾ ಜಗತ್ತು.

ಅದು ಮೂರು ವಸ್ತುಗಳನ್ನೂ ಒಪ್ಪಿಕೊಳ್ಳುತ್ತದೆ. ಅದೇ ದೇಶ ಕಾಲ ನಿಮಿತ್ತ. ಇವುಗಳೆಲ್ಲ ಸೇರಿ ಮಾಯೆ ಆಗಿದೆ. ಅವು ಮಾನವನ ಆಲೋಚನೆಗೆ ತಳಪಾಯ.ಆಲೋಚನೆಯಿಂದ ಆದವುಗಳಲ್ಲ, ಅವು ಆಲೋಚನೆಗೆ ಆಸರೆಯಾದುವುಗಳು. ಬಹಳ ಕಾಲದಲಮೇಲೆ ಕ್ಯಾಂಟ್​ ಎಂಬ ಜರ್ಮನ್​ ತತ್ತ್ವಜ್ಞಾನಿ ಇದೇ ನಿರ್ಣಯಕ್ಕೆ ಬಂದನು.

ನನ್ನ, ಜಗತ್ತಿನ ಮತ್ತು ದೇವರ ಸತ್ಯವೆಲ್ಲವೂ ಒಂದೇ. ವ್ಯತ್ಯಾಸವಿರುವುದು ಅಭಿವ್ಯಕ್ತಿಯಲ್ಲಿ ಮಾತ್ರ. ವೈವಿಧ್ಯವೆಲ್ಲ ಮಾಯೆಯಿಂದ ಆಗಿದೆ; ಸಮುದ್ರದ ತೀರವು ಅದನ್ನು ನೇರವಾಗಿಯೋ ವಕ್ರವಾಗಿಯೋ ಮಾಡುವಂತೆಯೆ ಇದು. ಆದರೆ ಸಮುದ್ರದ ನೀರಿಗೆ ಒಂದು ಆಕಾರವನ್ನು ಕೊಡುವ ತೀರವು ಮಾಯವಾದರೆ ಒಂದು ಅವಿಚ್ಛಿನ್ನವಾದ ನೀರು ಉಳಿಯುವುದು. ಇದರಂತೆಯೇ ಮಾಯೆ ಹೋದ ಮೇಲೆ ವೈವಿಧ್ಯವೆಲ್ಲಾ ಅಳಿಸಿಹೋಗುವುದು.

ಮಾನವನು ಆದಿಯಲ್ಲಿ ಪವಿತ್ರನಾಗಿದ್ದ, ಈಗ ಪತಿತನಾಗಿರುವನು, ಪುನಃ ಪವಿತ್ರನಾಗುವನು ಎಂದು ಎಲ್ಲಾ ಆಧುನಿಕ ಧರ್ಮಗಳೂ ಸಾರುವುವು. ಅವರಿಗೆ ಎಲ್ಲಿಂದ ಈ ಭಾವನೆ ಬಂದಿತೋ ನನಗೆ ಗೊತ್ತಿಲ್ಲ. ಆತ್ಮವೇ ಜ್ಞಾನಭಂಡಾರ; ಬಾಹ್ಯಸನ್ನಿವೇಶ ಆತ್ಮನನ್ನು ಪ್ರಚೋದಿಸುವುದು. ಜ್ಞಾನವೇ ಆತ್ಮನ ಶಕ್ತಿ. ಶತ ಶತಮಾನಗಳಿಂದ ಆತ್ಮವು ದೇಹಗಳನ್ನು ಸೃಷ್ಟಿಸುತ್ತಾ ಬಂದಿದೆ. ಹಲವು ಜನ್ಮಗಳು ಆತ್ಮನ ಜೀವನದ ಕಥೆಯ ಹಲವು ಅಧ್ಯಾಯಗಳಾಗಿವೆ. ನಾವು ನಿರಂತರವಾಗಿ ದೇಹಗಳನ್ನು ರಚಿಸುತ್ತಾ ಬಂದಿದ್ದೇವೆ. ಇಡೀ ವಿಶ್ವವು ಸಂಕೋಚವಾಗುತ್ತ ವಿಕಾಸವಾಗುತ್ತ ಬದಲಾಗುತ್ತಿದೆ. ಆತ್ಮನ ಸ್ವಭಾವ ಮತ್ತು ಶಕ್ತಿ ಬದಲಾಗುವುದಿಲ್ಲ. ಮಾಯೆಯಿಂದ ತೋರಿಕೆಗೆ ಬದಲಾದಂತೆ ತೋರುವುದು ಎನ್ನುತ್ತದೆ ವೇದಾಂತ. ಮನಸ್ಸಿನ ಮಿತಿಗೆ ಒಳಪಟ್ಟ ದೇವರೇ ಪ್ರಕೃತಿ. ಪ್ರಕೃತಿಯ ವಿಕಾಸವೇ ಜೀವಿಯ ವಿಕಾಸ. ಆತ್ಮವು ಎಲ್ಲಾ ಜೀವಿಗಳಲ್ಲಿಯೂ ಒಂದೇ ಸಮನಾಗಿರುವುದು. ಅವುಗಳು ದೇಹದಿಂದ ವ್ಯತ್ಯಾಸಗೊಳ್ಳುವುದು. ಮಾನವಕೋಟಿಯ ಸರ್ವಸಾಮಾನ್ಯ ವಸ್ತು ವಾದ ಆತ್ಮನೇ ನೀತಿಗೆ ಮತ್ತು ಧರ್ಮಕ್ಕೆ ತಳಹದಿ. ಈ ದೃಷ್ಟಿಯಲ್ಲಿ ಎಲ್ಲರೂ ಒಂದೇ; ಮತ್ತೊಬ್ಬರಿಗೆ ತೊಂದರೆ ಕೊಟ್ಟರೆ ನನಗೆ ತೊಂದರೆ ಉಂಟುಮಾಡಿಕೊಂಡಂತೆ.

ಪ್ರೇಮವು ಅನಂತ ಏಕತೆಯನ್ನು ವ್ಯಕ್ತಗೊಳಿಸುವ ಒಂದು ಸಾಧನವಾಗಿದೆ. ಯಾವ ದ್ವೈತಸಿದ್ಧಾಂತದ ಆಧಾರದ ಮೇಲೆ ನೀವು ಪ್ರೀತಿ ಎಂಬುದನ್ನು ವಿವರಿಸ ಬಲ್ಲಿರಿ? ಒಬ್ಬ ಐರೋಪ್ಯ ತತ್ತ್ವಶಾಸ್ತ್ರಜ್ಞನು ಚುಂಬಿಸುವುದು ನರಮಾಂಸ ಭಕ್ಷಣೆಯ ಒಂದು ಅವಶೇಷ ಎನ್ನುವನು. ಇದರ ಹಿಂದೆ ನೀನು ಎಷ್ಟು ರುಚಿಯಾಗಿರುವೆಯೊ ನೋಡುತ್ತೇನೆ ಎಂಬ ಭಾವ ಇದೆಯಂತೆ! ಆದರೆ ನಾನು ಇದನ್ನು ನಂಬುವುದಿಲ್ಲ. ನಾವೆಲ್ಲ ಹುಡುಕುವುದು ಏನು? ಸ್ವಾತಂತ್ರ್ಯವನ್ನು. ನಮ್ಮ ಹೋರಾಟ ಪ್ರಯತ್ನವೆಲ್ಲ ಸ್ವಾತಂತ್ರ್ಯವನ್ನು ಪಡೆಯುವುದಕ್ಕೆ. ಇಡೀ ಜನಾಂಗ, ಪ್ರಪಂಚ ಮತ್ತು ವ್ಯೂಹಗಳು ಸ್ವಾತಂತ್ರ್ಯದೆಡೆಗೆ ಧಾವಿಸುತ್ತಿವೆ.

ನಾವು ಬದ್ಧರಾದರೆ ಯಾರು ನಮ್ಮನ್ನು ಬಂಧಿಸಿರುವುದು? ಅನಂತಾತ್ಮವೂ ತನ್ನನ್ನು ತಾನೇ ಬಂಧಿಸಬೇಕಲ್ಲದೆ ಅದನ್ನು ಯಾವುದೂ ಬಂಧಿಸಲಾರದು.

