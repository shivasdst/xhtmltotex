
\chapter[ಭಾರತೀಯ ತತ್ತ್ವ ವಿಚಾರದ ಸೋಪಾನಗಳು ]{ಭಾರತೀಯ ತತ್ತ್ವ ವಿಚಾರದ ಸೋಪಾನಗಳು \protect\footnote{\engfoot{C.W. Vol. I, P. 393}}}

ಧಾರ್ಮಿಕ ಭಾವನೆಗಳ ಮೊದಲ ಆವಿಷ್ಕಾರವು ಸ್ಫೂರ್ತಿ ಮತ್ತು ಶ್ರುತಿ ಇವನ್ನು ಒಳಗೊಳ್ಳುತ್ತವೆ. ನಾನು ಪ್ರಸ್ತಾಪಿಸುತ್ತಿರುವುದು ಪ್ರಧಾನವಾದ ಧರ್ಮಗಳ ಭಾವನೆಗಳನ್ನೇ ಹೊರತು ಧರ್ಮ ಎಂಬ ಹೆಸರಿಗೇ ಅರ್ಹವಲ್ಲದ ಕ್ಷುದ್ರ ಭಾವನೆಗಳನ್ನಲ್ಲ. ಧಾರ್ಮಿಕ ಭಾವನೆಗಳ ಮೊದಲ ಗುಚ್ಛವು ಈಶ್ವರನ ಭಾವನೆಯಿಂದ ಮೊದಲಾಗುತ್ತದೆ. ಇಲ್ಲೊಂದು ಜಗತ್ತು ಇದೆ. ಅದನ್ನು ಯಾರೊ ಸೃಷ್ಟಿಸಿರುವರು. ಈ ವಿಶ್ವದಲ್ಲಿರುವುದನ್ನೆಲ್ಲ ಅವನು ಸೃಷ್ಟಿಸಿರುವನು. ಇದಾದ ನಂತರ ಜೀವಾತ್ಮದ ಭಾವನೆ ಮುಂದೆ ಬರುವುದು. ಇಲ್ಲೊಂದು ದೇಹ ಇದೆ, ಅದರಲ್ಲಿ ದೇಹವಲ್ಲದ ಮತ್ತೊಂದು ಇದೆ. ನಮಗೆ ತಿಳಿದ ಧರ್ಮದ ಅತ್ಯಂತ ಪ್ರಾಚೀನ ಭಾವನೆ ಇದು. ಭರತಖಂಡದಲ್ಲಿ ಇದರ ಅನುಯಾಯಿಗಳು ಕೆಲವರು ಇರುವರು. ಆದರೆ ಬಹಳ ಹಿಂದೆಯೇ ಇದನ್ನು ತ್ಯಜಿಸಿರುವರು. ಭಾರತೀಯ ಧರ್ಮಗಳು ಒಂದು ವಿಚಿತ್ರ ರೀತಿಯಲ್ಲಿ ಪ್ರಾರಂಭವಾಗುತ್ತವೆ. ಬೇಕಾದಷ್ಟು ಊಹೆ ಮಾಡಿದ ಮೇಲೆ, ವಿಶ್ಲೇಷಣೆ ಮಾಡಿದ ಮೇಲೆ, ಇಂತಹ ಸ್ಥಿತಿ ಇತ್ತು ಎಂಬುದನ್ನು ನಾವು ಕಲ್ಪಿಸಿಕೊಳ್ಳಬೇಕಾಗಿದೆ. ಅದು ನಮಗೆ ಸ್ಪಷ್ಟವಾಗಿ ಕಾಣುವುದೇ ಅನಂತರ ಬಂದ ಸ್ಥಿತಿಯ ಪ್ರಥಮ ಮೆಟ್ಟಿಲಲ್ಲಿ. ಪ್ರಥಮದಲ್ಲಿ ಸೃಷ್ಟಿಯ ಭಾವನೆ ಒಂದು ವಿಚಿತ್ರವಾಗಿದೆ. ಈ ವಿಶ್ವವೆಲ್ಲ ಶೂನ್ಯದಿಂದಾಯಿತು, ದೇವರ ಇಚ್ಛೆಯಿಂದಾಯಿತು ಎಂಬುದು. ಈ ವಿಶ್ವವಾವುದೂ ಆದಿಯಲ್ಲಿರಲಿಲ್ಲ, ಶೂನ್ಯದಿಂದ ಇದೆಲ್ಲ ಬಂದಿತು ಎನ್ನುವರು. ಇನ್ನೊಂದು ಮೆಟ್ಟಿಲು ಮುಂದೆ ಹೋದರೆ ಈ ನಿರ್ಣಯವನ್ನೇ ಅವರು ಪ್ರಶ್ನಿಸುವರು. ಶೂನ್ಯದಿಂದ ಸೃಷ್ಟಿ ಹೇಗೆ ಸಾಧ್ಯ? ವೇದಾಂತದಲ್ಲಿ ಮೊದಲೇ ಈ ಪ್ರಶ್ನೆಯನ್ನು ಹಾಕುವರು. ಈ ವಿಶ್ವವು ಇದ್ದರೆ ಅದು ಮತ್ತಾವುದರಿಂದಲೋ ಬಂದಿರಬೇಕು. ಏಕೆಂದರೆ ಎಲ್ಲಿಯೂ ಶೂನ್ಯದಿಂದ ಏನೂ ಬರಲಾರದು, ಮನುಷ್ಯನು ಮಾಡುವ ಕೆಲಸಗಳಿಗೆ ವಸ್ತುಗಳು ಬೇಕಾಗುತ್ತವೆ. ಒಂದು ಮನೆಯನ್ನು ಕಟ್ಟಲು ಅದಕ್ಕೆ ಸಾಮಾನು ಮುಂಚೆ ಇರಬೇಕು ದೋಣಿಯನ್ನು ಕಟ್ಟಲು ಅದಕ್ಕೆ ಸಾಮಾನು ಮುಂಚೆ ಇರಬೇಕು. ಯಾವುದಾದರೂ ಆಯುಧಗಳನ್ನು ಮಾಡಿದರೆ ಅದಕ್ಕೆ ಮುಂಚೆ ವಸ್ತುಗಳಿದ್ದುವು. ಕಾರಣದಿಂದ ಕಾರ್ಯ ಆಗುವುದು. ಆದಕಾರಣ ಜಗತ್ತು ಶೂನ್ಯದಿಂದ ಸೃಷ್ಟಿಯಾಯಿತು ಎಂಬುದನ್ನು ತಿರಸ್ಕರಿಸಿದರು. ಈ ಪ್ರಪಂಚ ಯಾವುದರಿಂದ ಆಯಿತೋ ಆ ಉಪಾದಾನ ಬೇಕಾಗಿತ್ತು. ಧಾರ್ಮಿಕ ಚರಿತ್ರೆಯೆಲ್ಲಾ ಈ ಉಪಾದಾನ ವಸ್ತುವನ್ನು ಅರಸುವುದಾಗಿದೆ.

ಯಾವುದರಿಂದ ಇದೆಲ್ಲ ಆಯಿತು? ನಿಮಿತ್ತಕಾರಣ ಯಾವುದು? ವಿಶ್ವವನ್ನು ದೇವರು ಸೃಷ್ಟಿಸಿದನೆ? ಮುಂತಾದವುಗಳೆಲ್ಲಕ್ಕಿಂತ ಮುಖ್ಯವಾದ ಪ್ರಶ್ನೆಯೇ ಯಾವುದರಿಂದ ಅವನು ಈ ಪ್ರಪಂಚವನ್ನು ಸೃಷ್ಟಿಸಿದ ಎಂಬುದು. ತಾತ್ತ್ವಿಕರ ಗಮನವನ್ನೆಲ್ಲ ಈ ಪ್ರಶ್ನೆ ಸೆಳೆದಿದೆ. ಒಂದು ಪರಿಹಾರ ಹೀಗಿದೆ: ಜೀವ-ಈಶ್ವರ ಜಗತ್ತುಗಳು ಮೂರೂ, ಬೇರೆ ಬೇರೆ ಸಮಾನಾಂತರದಲ್ಲಿ ನಿರಂತರವಾಗಿ ಹೋಗುತ್ತಿರುವ ರೇಖೆಗಳಂತೆ. ಜೀವ ಮತ್ತು ಜಗತ್ತು ಅನ್ಯಾಶ್ರಿತ, ಈಶ್ವರ ಸ್ವತಂತ್ರ ಸತ್ಯ. ಪ್ರತಿಯೊಂದು ಜೀವವೂ ದ್ರವ್ಯದ ಪ್ರತಿಯೊಂದು ಕಣದಂತೆ ಈಶ್ವರೇಚ್ಛೆಗೆ ಸಂಪೂರ್ಣವಾಗಿ ಅಧೀನವಾಗಿದೆ. ಬೇರೆ ವಿಷಯಗಳನ್ನು ತೆಗೆದುಕೊಳ್ಳುವುದಕ್ಕೆ ಮುಂಚೆ ಜೀವದ ವಿಷಯವನ್ನು ತೆಗೆದುಕೊಳ್ಳೋಣ. ವೇದಾಂತಿಗಳು ಈ ವಿಷಯದಲ್ಲಿ ಪಾಶ್ಚಾತ್ಯ ತತ್ತ್ವಜ್ಞಾನಿಗಳಿಗಿಂತ ಸಂಪೂರ್ಣವಾಗಿ ಭಿನ್ನವಾದ ಅಭಿಪ್ರಾಯವನ್ನು ಹೊಂದಿದ್ದಾರೆ. ವೇದಾಂತಿಗಳಿಗೆಲ್ಲ ಸಮಾನವಾದ ಒಂದು ಮನಃಶಾಸ್ತ್ರವಿದೆ. ಅವರ ತತ್ತ್ವ ಏನಾದರೂ ಆಗಲಿ ಅವರ ಮನಃಶಾಸ್ತ್ರವೆಲ್ಲ ಭಾರತದಲ್ಲಿ ಒಂದೇ ಆಗಿದೆ. ಅದು ಸಾಂಖ್ಯ. ಇದರ ಪ್ರಕಾರ ನಮಗೆ ಒಂದು ವಸ್ತು ಇದೆ ಎಂಬ ಅರಿವು ಉಂಟಾಗಬೇಕಾದರೆ ಮುಂದೆ ಹೇಳಿದಂತೆ ಆಗುವುದು: ಬಾಹ್ಯಪ್ರಪಂಚದ ಸ್ಪಂದನ ಮೊದಲು ಬಾಹ್ಯ ಇಂದ್ರಿಯಗಳಿಗೆ ತಾಕುವುದು, ಅಲ್ಲಿಂದ ಅಂತರಿಂದ್ರಿಯಗಳಿಗೆ, ಅಲ್ಲಿಂದ ಮನಸ್ಸಿಗೆ, ಮನಸ್ಸಿನಿಂದ ಬುದ್ಧಿಗೆ, ಬುದ್ಧಿಯಿಂದ ಆತ್ಮನಿಗೆ ಹೋಗುವುದು. ಈಗಿನ ಕಾಲದ ಶರೀರಶಾಸ್ತ್ರಕ್ಕೆ ಬಂದರೆ, ಅದು ಪ್ರತಿಯೊಂದು ಸಂವೇದನೆಗೂ ಮಿದುಳಿನಲ್ಲಿ ಅದಕ್ಕೆ ಸಂಬಂಧಪಟ್ಟ ಒಂದು ಕೇಂದ್ರವನ್ನು ಕಂಡುಹಿಡಿದಿರುವುದು ಕಾಣುತ್ತದೆ. ಮೊದಲು ಕೆಳಗಿನ ಕೇಂದ್ರವನ್ನು ಅದು ಕಂಡುಹಿಡಿಯುವುದು, ಅನಂತರ ಅದಕ್ಕಿಂತ ಸೂಕ್ಷ್ಮವಾದ ಕೇಂದ್ರಗಳನ್ನು ಕಂಡುಹಿಡಿಯುವುದು. ಇವೆರಡು ಕೇಂದ್ರಗಳೂ ಅಂತರಿಂದ್ರಿಯಗಳು ಮತ್ತು ಮನಸ್ಸನ್ನು ಹೋಲುವುವು. ಆದರೆ ಉಳಿದ ಕೇಂದ್ರಗಳನ್ನೆಲ್ಲ ತನ್ನ ಸ್ವಾಧೀನದಲ್ಲಿಟ್ಟುಕೊಂಡಿರುವ ಒಂದೇ ಒಂದು ಕೇಂದ್ರವನ್ನು ಇವರು ಕಂಡುಹಿಡಿದಿಲ್ಲ. ಆದಕಾರಣ ಶರೀರಶಾಸ್ತ್ರವು ಬೇರೆ ಬೇರೆ ಕೇಂದ್ರಗಳನ್ನೆಲ್ಲ ಯಾವುದು ಒಟ್ಟುಗೂಡಿಸುವುದು ಎಂಬುದನ್ನು ಹೇಳಲಾರದು. ಈ ಕೇಂದ್ರಗಳೆಲ್ಲ ಎಲ್ಲಿ ಒಟ್ಟುಗೂಡುತ್ತವೆ? ಮಿದುಳಿನಲ್ಲಿರುವ ಕೇಂದ್ರಗಳೆಲ್ಲ ಬೇರೆ ಬೇರೆ ಇವೆ. ಇವನ್ನೆಲ್ಲ ಸ್ವಾಧೀನದಲ್ಲಿಟ್ಟುಕೊಂಡಿರುವ ಯಾವ ಒಂದು ಕೇಂದ್ರವೂ ಇಲ್ಲ. ಆದಕಾರಣ ಈ ವಿಷಯದಲ್ಲಿ ಹಿಂದೂಗಳ ಮನಃಶಾಸ್ತ್ರವನ್ನು ಯಾರೂ ಪ್ರಶ್ನಿಸಲಾರರು. ಪೂರ್ಣಚಿತ್ರವು ದೊರೆಯಬೇಕಾದರೆ ಎಲ್ಲಿ ಸಂವೇದನೆಗಳೆಲ್ಲ ಪ್ರತಿಫಲಿಸಿ ಒಂದುಗೂಡುತ್ತವೆಯೋ ಅಂಥದೊಂದು ಇರಬೇಕಾಗುತ್ತದೆ. ಹಾಗೆ ಒಂದು ಇರುವ ತನಕ ನೀನಾರು ಎಂಬುದಾಗಲಿ, ಯಾವುದೇ ಚಿತ್ರವಾಗಲಿ, ಅಥವಾ ಇನ್ನೇನೇ ಆಗಲಿ ನನಗೆ ತಿಳಿದುಬರುವುದಿಲ್ಲ. ಹಾಗೆ ಎಲ್ಲವನ್ನೂ ಒಟ್ಟುಗೂಡಿಸುವುದೊಂದು ಇಲ್ಲದೇ ಇದ್ದಿದ್ದರೆ ನಾವು ಮೊದಲು ಕೇವಲ ನೋಡುವೆವು, ಕೆಲವು ಕಾಲದ ಮೇಲೆ ಉಸಿರಾಡುತ್ತೇವೆ, ಅನಂತರ ಕೇಳುತ್ತೇವೆ. ಹೀಗೆ ಆಗುತ್ತಿತ್ತು. ನಾನು ಒಬ್ಬನ ಮಾತನ್ನು ಕೇಳುತ್ತಿದ್ದರೆ ಅವನನ್ನು ನೋಡುವುದಕ್ಕೆ ಆಗುತ್ತಿರಲಿಲ್ಲ. ಏಕೆಂದರೆ\break ಕೇಂದ್ರಗಳೆಲ್ಲ ಬೇರೆ ಬೇರೆ.

ಈ ದೇಹ, ದ್ರವ್ಯ ಎಂದು ನಾವು ಕರೆಯುವ ಕಣಗಳಿಂದ ಆಗಿದೆ. ಇದು ಜಡ, ಅಚೇತನ. ವೇದಾಂತಿಗಳು ಸೂಕ್ಷ್ಮದೇಹ ಎನ್ನುವುದು ಕೂಡ ಇದೇ ರೀತಿಯಲ್ಲಿದೆ. ಅವರ ದೃಷ್ಟಿಯಲ್ಲಿ ಸೂಕ್ಷ್ಮದೇಹ ದ್ರವ್ಯದಿಂದ ಆಗಿದೆ. ಆದರೆ ಅತಿ ಸೂಕ್ಷ್ಮ ಕಣಗಳಿಂದ ಆದ ಪಾರದರ್ಶಕ ದೇಹ. ಅವೆಷ್ಟು ಸೂಕ್ಷ್ಮವಾಗಿವೆ ಎಂದರೆ ಯಾವ ಸೂಕ್ಷ್ಮದರ್ಶಕ ಯಂತ್ರವೂ ಅವನ್ನು ನೋಡಲಾರದು. ಇದರಿಂದ ಏನು ಪ್ರಯೋಜನ? ಇದು ಸೂಕ್ಷ್ಮಶಕ್ತಿಯ ಆಶ್ರಯಸ್ಥಾನ. ಈ ಸ್ಥೂಲ ದೇಹ ಹೇಗೆ ಸ್ಥೂಲಶಕ್ತಿಗೆ ಆಶ್ರಯಸ್ಥಾನವೋ ಹಾಗೆಯೇ ಸೂಕ್ಷ್ಮದೇಹವು ಸೂಕ್ಷ್ಮಶಕ್ತಿಗಳ ಎಂದರೆ ವಿವಿಧ ಆಲೋಚನೆಗಳ ಆಶ್ರಯಸ್ಥಾನ. ಮೊದಲು ದೇಹ; ಸ್ಥೂಲ ದ್ರವ್ಯದಿಂದಾಗಿದೆ, ಇಲ್ಲಿ ಸ್ಥೂಲಶಕ್ತಿಗಳಿವೆ. ಶಕ್ತಿಯು ದ್ರವ್ಯವಿಲ್ಲದೆ ಇರಲಾರದು. ಅದು ನೆಲಸಬೇಕಾದರೆ, ಯಾವುದಾದರೂ ಒಂದು ದ್ರವ್ಯ ಬೇಕಾಗಿದೆ. ಆದಕಾರಣವೇ ಸ್ಥೂಲಶಕ್ತಿಗಳು ಸ್ಥೂಲದೇಹದಲ್ಲಿ ಕೆಲಸ ಮಾಡುವುವು. ಈ ಶಕ್ತಿಗಳೇ ಸೂಕ್ಷ್ಮವಾಗುವುವು. ಯಾವ ಶಕ್ತಿ ಸ್ಥೂಲದ ಮೂಲಕವೂ ಕೆಲಸ ಮಾಡುವುದೊ ಅದೇ ಸೂಕ್ಷ್ಮದ ಮೂಲಕ ಕೆಲಸ ಮಾಡುವದು, ಅಲ್ಲಿ ಆಲೋಚನೆಯಾಗುವುದು. ಇವುಗಳಲ್ಲಿ ವ್ಯತ್ಯಾಸವೇನೂ ಇಲ್ಲ. ಇವೆರಡೂ ಒಂದೇ ಶಕ್ತಿಯ ಸ್ಥೂಲ ಮತ್ತು ಸೂಕ್ಷ್ಮ ಆವಿರ್ಭಾವಗಳು, ಸೂಕ್ಷ್ಮದೇಹ ಮತ್ತು ಸ್ಥೂಲದೇಹಗಳ ನಡುವೆ ಯಾವುದೇ ವ್ಯತ್ಯಾಸವೂ ಇಲ್ಲ. ಸೂಕ್ಷ್ಮದೇಹ ಕೂಡ ದ್ರವ್ಯದಿಂದಲೇ ಆಗಿದೆ ಆದರೆ ಅದು ಬಹು ಸೂಕ್ಷ್ಮದ್ರವ್ಯ ಅಷ್ಟೆ. ಹೇಗೆ ಈ ಸ್ಥೂಲದೇಹ ಸ್ಥೂಲಶಕ್ತಿಯು ಕೆಲಸ ಮಾಡಲು ಬೇಕಾದ ಒಂದು ಉಪಕರಣದಂತೆ ಇದೆಯೊ ಹಾಗೆಯೇ ಸೂಕ್ಷ್ಮದೇಹವು ಸೂಕ್ಷ್ಮಶಕ್ತಿಯ ಆವಿರ್ಭಾವಕ್ಕೆ ಅಗತ್ಯವಾದ ಒಂದು ಉಪಕರಣದಂತೆ ಇದೆ. ಈ ಶಕ್ತಿಗಳೆಲ್ಲ ಬರುವುದೆಲ್ಲಿಂದ? ವೇದಾಂತ ದರ್ಶನದ ಪ್ರಕಾರ ಪ್ರಕೃತಿಯಲ್ಲಿ ಎರಡು ವಸ್ತುಗಳಿವೆ. ಮೊದಲನೆಯದೆ ಆಕಾಶ; ಅದು ದ್ರವ್ಯದ ಅತ್ಯಂತ ಸೂಕ್ಷ್ಮರೂಪ. ಮತ್ತೊಂದು ಪ್ರಾಣ, ಅಂದರೆ ಶಕ್ತಿ. ನೀರು ವಾಯು ಪೃಥ್ವಿ ಮುಂತಾದ ಯಾವುದನ್ನು ಕೇಳಲಿ, ಅನುಭವಿಸಲಿ, ನೋಡಲಿ, ಇವೆಲ್ಲ ದ್ರವ್ಯ; ಇವು ಆಕಾಶದಿಂದ ಆದುವು. ಇವು ಪ್ರಾಣದ ಪ್ರಭಾವಕ್ಕೆ ಸಿಕ್ಕಿ ಹೆಚ್ಚು ಹೆಚ್ಚು ಸ್ಥೂಲ ಅಥವಾ ಸೂಕ್ಷ್ಮ ಆಗುತ್ತ ಬರುವುವು. ಆಕಾಶದಂತೆ ಪ್ರಾಣವೂ ಸರ್ವವ್ಯಾಪಿಯಾಗಿದೆ, ಎಲ್ಲದರಲ್ಲಿಯೂ ಹಾಸುಹೊಕ್ಕಾಗಿದೆ. ಆಕಾಶ ನೀರಿನಂತೆ ಇದೆ. ಪ್ರಪಂಚದಲ್ಲಿರುವ ಇತರ ವಸ್ತುಗಳೆಲ್ಲ ಅದೇ ನೀರಿನಿಂದ ಆದ ತೇಲುತ್ತಿರುವ ಮಂಜಿನ ಗಡ್ಡೆಗಳಂತೆ ಇವೆ. ಪ್ರಾಣವೇ ಆಕಾಶದ ವಿವಿಧ ಆಕಾರಗಳಿಗೆಲ್ಲ ಕಾರಣವಾದ ಶಕ್ತಿ. ಸ್ಥೂಲದೇಹವು ಆಕಾಶದಿಂದಾದ\break ಉಪಕರಣ. ನಡೆಯುವುದು, ಕುಳಿತುಕೊಳ್ಳುವುದು ಇವೇ ಮುಂತಾದ ಪ್ರಾಣದ ಸ್ಥೂಲ ಅಭಿವ್ಯಕ್ತಿಗಳಿಗೆ ಈ ಸ್ಥೂಲದೇಹವು ಉಪಕರಣವಾಗಿದೆ. ಸ್ಥೂಲ ದೇಹವು ಅತ್ಯಂತ ಸೂಕ್ಷ್ಮವಾದ ಆಕಾಶದಿಂದ ನಿರ್ಮಿತವಾಗಿದೆ. ಅದು ಪ್ರಾಣದ ಸೂಕ್ಷ್ಮರೂಪವಾದ ಆಲೋಚನೆಯ ಅಭಿವ್ಯಕ್ತಿಗೆ ಉಪಕರಣವಾಗಿದೆ. ಮೊದಲು ಈ ಸ್ಥೂಲ ದೇಹವಿದೆ, ಅದರಾಚೆ ಸೂಕ್ಷ್ಮದೇಹವಿದೆ. ಇದರಾಚೆಯೇ ಜೀವ ಅಥವಾ ನಿಜವಾದ ಮನುಷ್ಯ ಇರುವನು. ಹೇಗೆ ನಾವು ಉಗುರುಗಳನ್ನು ಎಷ್ಟು ವೇಳೆ ಕತ್ತರಿಸಿ ಹಾಕಿದರೂ ಅವು ಪುನಃ ಹಿಂದಿನಂತೆಯೇ ಬೆಳೆಯುವುವೊ, ಹೇಗೆ ಅವು ನಮ್ಮ ಒಂದು ಭಾಗವಾಗಿರುವುದೊ, ಅದರಂತೆಯೇ ಸ್ಥೂಲದೇಹಕ್ಕೂ ನಮ್ಮ ಸೂಕ್ಷ್ಮದೇಹಕ್ಕೂ ಇರುವ ಸಂಬಂಧ. ಒಬ್ಬನಿಗೆ ಸ್ಥೂಲ ಮತ್ತು ಸೂಕ್ಷ್ಮ ದೇಹಗಳೆರಡೂ\break ಇವೆ ಎಂದು ಅಲ್ಲ. ಇರುವುದು ಒಂದೇ. ಯಾವುದು ಹೆಚ್ಚು ಕಾಲ ಇರುವುದೋ ಅದು ಸೂಕ್ಷ್ಮ. ಯಾವುದು ಬೇಗ ನಾಶವಾಗುವುದೊ ಅದು ಸ್ಥೂಲ. ನಾನು ನನ್ನ ಬೆರಳಿನ\break ಉಗುರನ್ನು ಹೇಗೆ ಎಷ್ಟು ಸಲ ಬೇಕಾದರೂ ಕತ್ತರಿಸಿಕೊಳ್ಳಬಲ್ಲೆನೊ ಹಾಗೆಯೆ, ನನ್ನ\break ಸ್ಥೂಲ ದೇಹವನ್ನು ಎಷ್ಟು ಲಕ್ಷ ವೇಳೆಯಾದರೂ ತ್ಯಜಿಸಬಹುದು. ಆದರೆ ಸೂಕ್ಷ್ಮದೇಹ ಇದ್ದೇ ಇರುವುದು. ದ್ವೈತಿಗಳ ದೃಷ್ಟಿಯಲ್ಲಿ ಜೀವ ಅಥವಾ ನಿಜವಾದ ವ್ಯಕ್ತಿ, ಅತಿ ಸೂಕ್ಷ್ಮ ಅಣುವಿನಂತೆ ಇರುವುದು.

ಇಲ್ಲಿಯ ತನಕ, ಮನುಷ್ಯನಿಗೆ ಒಂದು ಸ್ಥೂಲದೇಹವಿದೆ, ಅದು ಬೇಗ ನಾಶವಾಗುವುದು, ಅನಂತರ ಒಂದು ಸೂಕ್ಷ್ಮದೇಹವಿದೆ, ಅದು ಕಲ್ಪಾಂತರಗಳವರೆಗೆ ಇರುವುದು, ಅನಂತರ ಅದರ ಹಿಂದೆ ಒಂದು ಜೀವ ಇದೆ ಎಂಬುದನ್ನು ನೋಡಿದೆವು. ವೇದಾಂತದ\break ದೃಷ್ಟಿಯಲ್ಲಿ ದೇವರಿಗೆ ಹೇಗೆ ಆದಿ ಅಂತ್ಯಗಳಿಲ್ಲವೋ ಹಾಗೆಯೇ ಈ ಜೀವಕ್ಕೂ ಆದಿ\break ಅಂತ್ಯಗಳಿಲ್ಲ. ಪ್ರಕೃತಿ ಕೂಡ ಆದಿ ಅಂತ್ಯಗಳಿಲ್ಲದ್ದು, ಆದರೆ ಅದು ವಿಕಾರಗೊಳ್ಳುತ್ತದೆ.\break ಪ್ರಕೃತಿಯ ದ್ರವ್ಯಗಳಾದ ಆಕಾಶ ಮತ್ತು ಪ್ರಾಣಗಳು ಆದಿ ಅಂತ್ಯಗಳಿಲ್ಲದುವು. ಆದರೆ ಅವು ಯಾವಾಗಲೂ ಹಲವು ಆಕಾರಗಳನ್ನು ತಾಳುತ್ತಿರುತ್ತವೆ. ಜೀವವು ಆಕಾಶದಿಂದ ಅಥವಾ ಪ್ರಾಣದಿಂದ ತಯಾರಾಗಿಲ್ಲ. ಅದು ಅದ್ರವ್ಯ. ಆದಕಾರಣ ಅದು ಎಂದೆಂದಿಗೂ ಇರುವುದು. ಯಾವುದು ಸಂಯೋಗದಿಂದ ಆಗಿಲ್ಲವೋ ಅದೆಂದಿಗೂ ನಾಶವಾಗುವುದಿಲ್ಲ.\break ಏಕೆಂದರೆ ನಾಶ ಎಂದರೆ ಕಾರಣ ಸ್ಥಿತಿಗೆ ಹಿಂದಿರುಗಿ ಹೋಗುವುದು ಎಂದು ಅರ್ಥ. ಸ್ಥೂಲ ದೇಹವು ಆಕಾಶ ಮತ್ತು ಇವುಗಳ ಸಂಯೋಗದಿಂದ ಆಗಿದೆ. ಆದಕಾರಣ ಅದು\break ನಾಶವಾಗುವುದು. ಸೂಕ್ಷ್ಮದೇಹ ಕೂಡ ಬಹಳ ಕಾಲದ ಮೇಲೆ ನಾಶವಾಗುವುದು.\break ಆದರೆ ಜೀವ ಯಾವ ಮಿಶ್ರಣದಿಂದಲೂ ಆಗಿಲ್ಲ. ಅದೆಂದಿಗೂ ನಾಶವಾಗುವುದಿಲ್ಲ.\break ಆದಕಾರಣವೇ ಅದು ಎಂದಿಗೂ ಹುಟ್ಟಿಯೂ ಇಲ್ಲ. ಯಾವುದು ಅಮಿಶ್ರವಾದುದೋ ಅದು ಯಾವಾಗಲೂ ಜನ್ಮರಹಿತವಾದುದು. ಮಿಶ್ರವಾದುದಕ್ಕೆ ಮಾತ್ರ ಜನ್ಮವಿರುತ್ತದೆ.\break ಇದೇ ವಾದ ಮುಂದೆಯೂ ಅನ್ವಯಿಸುವುದು. ಯಾವುದು ಮಿಶ್ರಣದಿಂದ ಆಗಿದೆಯೊ ಅದು ಮಾತ್ರ ಜನಿಸಬಲ್ಲದು. ಕೋಟ್ಯಂತರ ಜೀವಿಗಳಿರುವ ಸೃಷ್ಟಿಯು ಭಗವಂತನ\break ಅಧೀನದಲ್ಲಿದೆ. ಈಶ್ವರ ಸರ್ವವ್ಯಾಪಿ, ಸರ್ವಜ್ಞ, ನಿರಾಕಾರ. ಅವನು ಪ್ರಕೃತಿಯ ಮೂಲಕ ಹಗಲು ರಾತ್ರಿ ಕೆಲಸ ಮಾಡುತ್ತಿರುವನು. ಈ ಸೃಷ್ಟಿಯೆಲ್ಲ ಅವನ ಅಧೀನದಲ್ಲಿದೆ. ಅವನೇ ಸನಾತನ ಈಶ್ವರ. ದ್ವೈತಿಗಳು ಹೇಳುವುದು ಹೀಗೆ. ಆಗ ಮತ್ತೊಂದು ಪ್ರಶ್ನೆ ಉದ್ಭವಿಸುವುದು. ದೇವರು ಪ್ರಪಂಚದ ನಿಯಾಮಕನಾದರೆ, ಇಂತಹ ಕೆಟ್ಟ ಪ್ರಪಂಚವನ್ನು ಅವನು ಏತಕ್ಕೆ ಸೃಷ್ಟಿಸಬೇಕಾಗಿತ್ತು? ನಾವೇತಕ್ಕೆ ಇಷ್ಟೊಂದು ಕಷ್ಟವನ್ನು ಅನುಭವಿಸಬೇಕು? ಇದು ದೇವರ ತಪ್ಪಲ್ಲ ಎನ್ನುವರು ಅವರು. ನಾವು ನಮ್ಮ ತಪ್ಪಿನಿಂದ ವ್ಯಥೆಪಡುತ್ತೇವೆ. ಮಾಡಿದ್ದುಣ್ಣೊ ಮಹಾರಾಯ. ನಮ್ಮನ್ನು ಶಿಕ್ಷಿಸಲು ಅವನು ಏನನ್ನೂ ಮಾಡಿಲ್ಲ. ಮನುಷ್ಯ ದರಿದ್ರನಾಗಿಯೋ ಅಂಧನಾಗಿಯೋ ಹುಟ್ಟುತ್ತಾನೆ. ಇದಕ್ಕೆ ಕಾರಣವೇನು? ಅವನು ಹಿಂದೆ ಏನನ್ನೋ ಮಾಡಿರಬೇಕು. ಆದ್ದರಿಂದ ಹಾಗೆ ಹುಟ್ಟಿದ್ದಾನೆ. ಜೀವ ಎಲ್ಲಾ ಕಾಲದಲ್ಲಿಯೂ\break ಇರುವುದು. ಅದನ್ನು ಹೊಸದಾಗಿ ಯಾರೂ ಸೃಷ್ಟಿಸಲಿಲ್ಲ. ಅದು ಏನೇನನ್ನೊ ಸದಾ\break ಕಾಲವೂ ಮಾಡುತ್ತಿತ್ತು. ನಾವು ಏನನ್ನು ಮಾಡಿರುವೆವೊ ಅದೆಲ್ಲ ನಮಗೆ ಹಿಂತಿರುಗಿ\break ಬರುವುದು. ನಾವು ಒಳ್ಳೆಯದನ್ನು ಮಾಡಿದರೆ ಸುಖ, ಕೆಟ್ಟದ್ದನ್ನು ಮಾಡಿದರೆ ದುಃಖ ಪ್ರಾಪ್ತವಾಗುತ್ತದೆ. ಆದಕಾರಣವೆ ಜೀವ, ಸುಖ ದುಃಖಗಳಿಗೆ ಸಿಕ್ಕಿ ಏನೇನನ್ನೊ ಮಾಡಿಕೊಂಡು ಹೋಗುವುದು.

ಸತ್ತ ಮೇಲೆ ಏನಾಗುವುದು? ಜೀವ ಸ್ವಭಾವತಃ ಪರಿಶುದ್ಧವಾದುದು, ಆದರೆ ಅಜ್ಞಾನ ಅದನ್ನು ಮುತ್ತಿದೆ ಎಂದು ಮಾತ್ರ ವೇದಾಂತಿಗಳು ಹೇಳುವರು. ಪಾಪಕರ್ಮದಿಂದ ಅದು ಅಜ್ಞಾನದ ಆವರಣದೊಳಗೆ ಸಿಕ್ಕಿದರೆ ಪುನಃ ಪುಣ್ಯಕರ್ಮದಿಂದ ಅದು ತನ್ನ ನಿಜಸ್ವರೂಪವನ್ನು ಅರಿಯುವುದು. ಜೀವಕ್ಕೆ ಆದಿ ಅಂತ್ಯಗಳಿಲ್ಲ ಮಾತ್ರವಲ್ಲ, ಅದು ಪರಿಶುದ್ಧವೂ ಆಗಿರುವುದು. ಪ್ರತಿಯೊಂದು ಜೀವಿಯ ಸ್ವಭಾವವೂ ಪರಿಶುದ್ಧವಾದುದು.

ಪುಣ್ಯಕರ್ಮಗಳಿಂದ ಜೀವಿಯ ಪಾಪಗಳ ಮತ್ತು ದುಷ್ಕೃತ್ಯಗಳ ಪರಿಣಾಮಗಳೆಲ್ಲ ತೊಳೆದುಹೋದ ಮೇಲೆ ಅದು ಪುನಃ ಪರಿಶುದ್ಧವಾಗಿ ದೇವಯಾನಕ್ಕೆ ಹೋಗುವುದು. ವಾಗಿಂದ್ರಿಯವು ಮನಸ್ಸಿನಲ್ಲಿ ಐಕ್ಯವಾಗುವುದು. ನೀವು ಶಬ್ದಗಳಿಲ್ಲದೆ ಆಲೋಚಿಸಲಾರಿರಿ. ಆಲೋಚನೆ ಇರುವೆಡೆ ಶಬ್ದಗಳೂ ಇರಬೇಕು. ಅವು ಮನಸ್ಸಿನಲ್ಲಿ ಐಕ್ಯವಾದ ಮೇಲೆ ಮನಸ್ಸು ಪ್ರಾಣದಲ್ಲಿ ಲೀನವಾಗುವುದು, ಪ್ರಾಣ ಜೀವದಲ್ಲಿ ಲೀನವಾಗುವುದು. ಆಗ ಜೀವ ದೇಹದಿಂದ ಅಗಲಿ ಸೂರ್ಯಲೋಕಕ್ಕೆ ಹೋಗುವುದು. ಈ ವಿಶ್ವದಲ್ಲಿ ಎಷ್ಟೋ ಲೋಕಗಳಿವೆ. ಈ ಪೃಥ್ವಿ ಭೂಲೋಕ, ಇಲ್ಲಿ ಸೂರ್ಯ ಚಂದ್ರ ನಕ್ಷತ್ರಗಳು ಇರುವುವು. ಇದರಾಚೆ ಸೂರ್ಯ ಲೋಕವಿದೆ, ಅದರಾಚೆ ಚಂದ್ರಲೋಕವಿದೆ, ಅದರಾಚೆ ವಿದ್ಯುತ್​ ಲೋಕವಿದೆ. ಜೀವ ಅಲ್ಲಿಗೆ ಹೋದಾಗ, ಸಿದ್ಧಿಪಡೆದ ಮತ್ತೊಂದು ಜೀವ ಅದನ್ನು ಸ್ವಾಗತಿಸಲು ಬರುವುದು. ಅಲ್ಲಿಂದ ಅದು ಆ ಜೀವಿಯನ್ನು ಬ್ರಹ್ಮಲೋಕವೆಂಬ ಶ್ರೇಷ್ಠಲೋಕಕ್ಕೆ ಒಯ್ಯುವುದು. ಜೀವ ಅಲ್ಲಿ ಎಂದೆಂದಿಗೂ ಇರುವುದು, ಇನ್ನು ಮೇಲೆ ಹುಟ್ಟುವುದೂ ಇಲ್ಲ, ಸಾಯುವುದೂ ಇಲ್ಲ. ಅನಂತಕಾಲದವರೆಗೆ ಅದು ಆನಂದದಲ್ಲಿರುವುದು, ಲೋಕವನ್ನು ಸೃಷ್ಟಿಸುವ ಶಕ್ತಿಯೊಂದು ವಿನಃ ಉಳಿದೆಲ್ಲವೂ ಅದಕ್ಕೆ ಪ್ರಾಪ್ತವಾಗುವುದು. ಈ ಪ್ರಪಂಚಕ್ಕೆ ಒಬ್ಬನೇ ಈಶ್ವರ ಇರುವುದು, ಅವನೇ ದೇವರು. ಇತರರು ಯಾರೂ ದೇವರಾಗಲಾರರು. ನೀವು ದೇವರೆಂದರೆ ಅದು ಈಶ್ವರನಿಂದೆ ಎನ್ನುವರು ದ್ವೈತಿಗಳು. ಸೃಷ್ಟಿಶಕ್ತಿ ವಿನಃ ಉಳಿದೆಲ್ಲ ಶಕ್ತಿಗಳೂ ಜೀವಿಗೆ ಪ್ರಾಪ್ತವಾಗುವುವು. ಬೇರೆ ಬೇರೆ ಲೋಕಗಳಲ್ಲಿ ದೇಹವನ್ನು ತಾಳಿ ಅದು ಕೆಲಸ ಮಾಡಬೇಕೆಂದು ಇಚ್ಛಿಸಿದರೆ ಹಾಗೆ ಮಾಡಬಹದು. ಅದರ ಆಜ್ಞೆಯಾದರೆ ದೇವತೆಗಳೆಲ್ಲ ಮತ್ತು ಪಿತೃಗಳೆಲ್ಲ ಅದರ ಮುಂದೆ ಬರುವರು. ಅದಕ್ಕೆ ಇನ್ನು ಮೇಲೆ ವ್ಯಥೆ ಎಂಬುದಿಲ್ಲ; ಅದು ಎಷ್ಟುಕಾಲ ಬೇಕಾದರೂ ಬ್ರಹ್ಮಲೋಕದಲ್ಲಿ ಇರಬಹುದು. ಅದೇ ಮಾನವ ಶ್ರೇಷ್ಠ. ಭಗವತ್ಪ್ರೇಮ ಅದಕ್ಕೆ ಲಭಿಸಿದೆ, ಅದು ಸಂಪೂರ್ಣ ನಿಃಸ್ವಾರ್ಥಿಯಾಗಿರುವುದು, ಪರಿಶುದ್ಧವಾಗಿರುವುದು. ಆಸೆಗಳನ್ನೆಲ್ಲ ತ್ಯಜಿಸಿರುವುದು. ಅದು ಭಗವಂತನ ಪೂಜೆಯನ್ನು ಮತ್ತು ಅವನನ್ನು ಪ್ರೀತಿಸುವುದನ್ನಲ್ಲದೆ ಮತ್ತೇನನ್ನೂ ಬಯಸುವುದಿಲ್ಲ.

ಮತ್ತೆ ಕೆಲವರು ಇರುವರು, ಅವರು ಅಷ್ಟು ಶ್ರೇಷ್ಠರಲ್ಲ, ಅವರು ಒಳ್ಳೆಯ ಕೆಲಸಗಳನ್ನು ಮಾಡುವರು, ಆದರೆ ಅವರು ಅದಕ್ಕೆ ಪ್ರತಿಫಲವನ್ನು ಆಶಿಸುವರು. ಅವರು ಬಡವರಿಗೆ ದಾನಧರ್ಮಗಳನ್ನು ಮಾಡುತ್ತಾರೆ. ಅದಕ್ಕಾಗಿ ಸ್ವರ್ಗ ಸುಖವನ್ನು ಆಶಿಸುವರು. ಅವರು ಕಾಲವಾದರೆ ಏನಾಗುವರು? ಮಾತು ಮನಸ್ಸಿನಲ್ಲಿ ಐಕ್ಯವಾಗುವುದು, ಮನಸ್ಸು ಪ್ರಾಣದಲ್ಲಿ ಐಕ್ಯವಾಗುವುದು, ಪ್ರಾಣ ಜೀವನಲ್ಲಿ ಐಕ್ಯವಾಗುವುದು. ಜೀವ ದೇಹದಿಂದ ಹೊರಟು ಚಂದ್ರಲೋಕಕ್ಕೆ ಹೋಗುವುದು. ಅಲ್ಲಿ ಬಹಳ ಕಾಲ ಸುಖವನ್ನು ಅನುಭವಿಸುವುದು. ಪುಣ್ಯಕರ್ಮವು ತೀರಿದ ಮೇಲೆ ತನ್ನ ಆಸೆಗಳಿಗೆ ಅನುಸಾರವಾಗಿ ಧರೆಗೆ ಇಳಿದು ದೇಹ ಧರಿಸುವುದು. ಚಂದ್ರಲೋಕದಲ್ಲಿ ಜೀವನು ದೇವತೆಯಾಗಿರುವನು. ಕ್ರೈಸ್ತರ ಮತ್ತು ಮಹಮ್ಮದೀಯರ ದೃಷ್ಟಿಯಲ್ಲಿ ಅವನು ದೇವದೂತನಾಗಿರುವನು. ದೇವತೆ ಎಂಬುದು ಕೆಲವು ಪದವಿಗಳ ಹೆಸರು. ಉದಾಹರಣೆಗೆ ಇಂದ್ರನನ್ನು ತೆಗೆದುಕೊಳ್ಳಿ. ಅವನು ದೇವತೆಗಳಿಗೆ ಒಡೆಯ. ಇಂದ್ರ ಎನ್ನುವುದು ಒಂದು ಪದವಿಯ ಹೆಸರು. ಸಾವಿರಾರು ಜನ ಈ ಪದವಿಗೆ ಬರುತ್ತಾರೆ. ಧರ್ಮಾತ್ಮನೊಬ್ಬ ಮಹಾಯಾಗವನ್ನು ಮಾಡಿದರೆ ಕಾಲವಾದ ಮೇಲೆ ದೇವೆಂದ್ರನಾಗುವನು. ಆಗ ಇದುವರೆಗೆ ದೇವೇಂದ್ರನಾಗಿದ್ದವನು ಪುನಃ ಪೃಥ್ವಿಗೆ ಬಂದು ಮಾನವನಾಗುವನು. ಇಲ್ಲಿ ರಾಜರುಗಳು ಹೇಗೆ ಬದಲಾಗುವರೋ ಹಾಗೆಯೇ ಇಂದ್ರರೂ ಬದಲಾಗಬೇಕು. ಸ್ವರ್ಗದಲ್ಲಿಯೂ ಜನರು ಸಾಯುವರು. ಬ್ರಹ್ಮಲೋಕ ಒಂದರಲ್ಲೆ\break ಮೃತ್ಯುವಿಲ್ಲ; ಅಲ್ಲಿ ಜನನವೂ ಇಲ್ಲ, ಮರಣವೂ ಇಲ್ಲ.

ಜೀವರು ಹೀಗೆ ಸ್ವರ್ಗಕ್ಕೆ ಹೋಗಿದ್ದು, ಕೆಲವು ವೇಳೆ ರಾಕ್ಷಸರು ಬಂದು ಅವರಿಗೆ ತೊಂದರೆ ಕೊಡುವಾಗ ವಿನಃ ಉಳಿದ ಕಾಲದಲ್ಲಿ ಸುಖವಾಗಿರುವರು. ನಮ್ಮ ಪುರಾಣದಲ್ಲಿ ರಾಕ್ಷಸರು ಇರುವರು. ಕೆಲವು ವೇಳೆ ಅವರು ದೇವತೆಗಳಿಗೆ ತೊಂದರೆ ಕೊಡುವರು. ಎಲ್ಲ ದೇಶಗಳ ಪುರಾಣಗಳಲ್ಲೂ ದೇವಾಸುರರ ಯುದ್ಧವನ್ನು ಕುರಿತು ಓದುತ್ತೇವೆ. ರಾಕ್ಷಸರು ಕೆಲವು ವೇಳೆ ದೇವತೆಗಳನ್ನು ಗೆಲ್ಲುತ್ತಿದ್ದರು. ಆದರೂ ಅನೇಕ ವೇಳೆ ರಾಕ್ಷಸರು ದೇವತೆಗಳಷ್ಟು ಹೀನ ಕೆಲಸ ಮಾಡಲಿಲ್ಲ. ಪುರಾಣಗಳಲ್ಲಿ ದೇವತೆಗಳಿಗೆಲ್ಲ ಹೆಂಗಸರ ಮೇಲೆ ಬಹಳ ಆಸೆ ಎಂಬುದನ್ನು ಓದುತ್ತೇವೆ. ಅವರು ತಮ್ಮ ಕರ್ಮಗಳ ಪುಣ್ಯಫಲವನ್ನು ಅನುಭವಿಸಿದ\break ಮೇಲೆ ಪುನಃ ಧರೆಗೆ ಬೀಳುವರು. ಮೋಡಗಳ ಮೂಲಕ ಮಳೆಯಲ್ಲಿ ಧರೆಗೆ ಬಿದ್ದು ಯಾವುದಾದರೂ ಸಸ್ಯದ ಮೂಲಕ ಮನುಷ್ಯನ ದೇಹವನ್ನು ಸೇರುವರು. ತಂದೆಯು ಜೀವಿಗೆ ಒಂದು ದೇಹವನ್ನು ನೀಡುವನು. ಈ ದೇಹ ಅದಕ್ಕೆ ಸರಿಯಾಗದಿದ್ದರೆ ಜೀವಿ\break ಗಳು ಬೇರೆ ದೇಹವನ್ನು ಸೃಷ್ಟಿಸಿಕೊಳ್ಳಬೇಕಾಗಿದೆ. ಮಾಡಬಾರದ ಕಾರ್ಯಗಳನ್ನೆಲ್ಲ\break ಮಾಡಿದ ಹಲವು ದುರ್ಜನರು ಇರುವರು. ಅವರು ಪ್ರಾಣಿಗಳಾಗಿ ಹುಟ್ಟುವರು. ಇನ್ನೂ ಕೆಟ್ಟವರಾಗಿದ್ದರೆ ಅತಿ ಕೆಳಮಟ್ಟದ ಮೃಗ ಗಿಡ ಮರ ಕಲ್ಲು ಆಗುವರು.

ದೇವತೆಯ ದೇಹದಲ್ಲಿದ್ದಾಗ ಜೀವಿಗಳು ಯಾವ ಕರ್ಮವನ್ನೂ ಮಾಡುವುದಿಲ್ಲ. ಮಾನವ ರೂಪದಲ್ಲಿದ್ದಾಗ ಮಾತ್ರ ಕರ್ಮವನ್ನು ಮಾಡುತ್ತವೆ. ಕರ್ಮ ಎಂದರೆ ಫಲಕ್ಕೆ\break ಕಾರಣವಾದ ಕೆಲಸ. ಮಾನವ ಮರಣದ ನಂತರ ದೇವನಾದರೆ, ಆ ಕಾಲದಲ್ಲಿ ಕೇವಲ ಸುಖವನ್ನು ಮಾತ್ರ ಅನುಭವಿಸುವನು. ಹೊಸದಾಗಿ ಯಾವ ಕರ್ಮವನ್ನೂ ಮಾಡುವುದಿಲ್ಲ. ಇದು ಅವನ ಹಿಂದಿನ ಪುಣ್ಯಕರ್ಮಕ್ಕೆ ಬಹುಮಾನವಷ್ಟೆ. ಪುಣ್ಯಕರ್ಮ ಕ್ಷೀಣವಾದ ಮೇಲೆ ಅವನು ಪ್ರಪಂಚಕ್ಕೆ ಬರುವನು. ಅವನು ಪುನಃ ಮನುಷ್ಯನಾಗುವನು, ಆಗ ಪುಣ್ಯಕರ್ಮಗಳನ್ನು ಮಾತ್ರ ಮಾಡಿ ಪರಿಶುದ್ಧನಾದರೆ ಬ್ರಹ್ಮಲೋಕಕ್ಕೆ ಹೋಗುವನು, ಹಿಂತಿರುಗಿ\break ಬರುವುದಿಲ್ಲ.

ಕೆಳಗಿನ ಅವಸ್ಥೆಯಿಂದ ಜೀವ ವಿಕಾಸವಾಗುತ್ತ ಬರುವಾಗ ಪ್ರಾಣಿರೂಪದಲ್ಲಿ ಅದು ಕೆಲವು ಕಾಲ ಇರುವುದು. ಕೆಲವು ಕಾಲದ ಮೇಲೆ ಈ ಪ್ರಾಣಿ ಮನುಷ್ಯನಾಗುವುದು. ಮಾನವರ ಸಂಖ್ಯೆ ಹೆಚ್ಚಿದಂತೆಲ್ಲ ಪ್ರಾಣಿಗಳ ಸಂಖ್ಯೆ ಕಡಮೆಯಾಗುತ್ತಿರುವುದು ಗಮನಾರ್ಹವಾದ ಸಂಗತಿ. ಪ್ರಾಣಿ ರೂಪದಲ್ಲಿದ್ದ ಜೀವಗಳೆಲ್ಲ ಮಾನವರಾಗುತ್ತಿವೆ. ಎಷ್ಟೋ ಜಾತಿಯ ಪ್ರಾಣಿಗಳು ಮನುಷ್ಯರಾಗಿವೆ. ಇಲ್ಲದೆ ಇದ್ದರೆ ಅವೆಲ್ಲ ಎಲ್ಲಿ ಹೋದವು?

ವೇದಗಳಲ್ಲಿ ನರಕದ ಮಾತೆ ಇಲ್ಲ. ಅನಂತರ ಬಂದ ಪುರಾಣಗಳು, ನರಕವಿಲ್ಲದೇ ಇದ್ದರೆ ಯಾವ ಧರ್ಮವೂ ಪೂರ್ಣವಾಗುವುದಿಲ್ಲವೆಂದು, ಹಲವು ರೀತಿಯ ನರಕ ಲೋಕಗಳನ್ನು ಸೃಷ್ಟಿಸಿದವು. ಇಂತಹ ನರಕಗಳಲ್ಲಿ ಕೆಲವು ಕಡೆ ಜೀವರನ್ನು ಗರಗಸದಲ್ಲಿ ಕೊಯ್ಯುವರು, ಹಲವು ಭಯಂಕರ ಶಿಕ್ಷೆಗಳನ್ನು ಕೊಡುವರು. ಆದರೂ ಅವರು ಸಾಯುವುದಿಲ್ಲ. ಅವರು ಆ ಸ್ಥಿತಿಯಲ್ಲಿ ದಾರುಣ ವ್ಯಥೆಯನ್ನು ಅನುಭವಿಸುವರು. ಆದರೆ ಪುರಾಣದಲ್ಲಿ ಜೀವಿಗಳ ಮೇಲೆ ಸ್ವಲ್ಪ ದಯೆ ಇದೆ. ಈ ಯಾತನೆ ಕೆಲವು ಕಾಲ ಮಾತ್ರ ಇರುವುದು ಎಂದು ಹೇಳುವರು. ಆ ಸ್ಥಿತಿಯಲ್ಲಿ ತಮ್ಮ ಪಾಪಕರ್ಮ ಕ್ಷಯವಾಗಿ ಪುನಃ ಪ್ರಪಂಚಕ್ಕೆ ಬರುತ್ತಾರೆ, ಒಳ್ಳೆಯವರಾಗುವುದಕ್ಕೆ ಮತ್ತೊಂದು ಅವಕಾಶ ಅವರಿಗೆ ದೊರಕುವುದು. ಆದಕಾರಣವೇ ಮಾನವ ಜನ್ಮದಲ್ಲಿ ದೊಡ್ಡದೊಂದು ಅವಕಾಶವಿದೆ. ಮಾನವ ದೇಹವನ್ನು ಕರ್ಮದೇಹ ಎಂದು ಕರೆಯುತ್ತಾರೆ. ಇಲ್ಲಿ ನಮ್ಮ ಅದೃಷ್ಟವನ್ನು ನಾವು ನಿಶ್ಚಯಿಸುವೆವು. ನಾವೊಂದು ದೊಡ್ಡ ವೃತ್ತದಲ್ಲಿ ಸುತ್ತುತ್ತಿರುವೆವು. ಭವಿಷ್ಯವನ್ನು ನಿರ್ಧರಿಸಬಲ್ಲ ಬಿಂದುವೇ ಈ ದೇಹ.\break ಆದಕಾರಣವೆ ದೇಹದಲ್ಲೆಲ್ಲಾ ಮಾನವದೇಹ ಅತ್ಯಂತ ಶ್ರೇಷ್ಠ ಎಂದು ಪರಿಗಣಿಸುವರು. ಮಾನವರು ದೇವತೆಗಳಿಗಿಂತಲೂ ಶ್ರೇಷ್ಠರು.

ಇದೆಲ್ಲ ಸುಲಭವಾದ ಶುದ್ಧವಾದ ದ್ವೈತ ವಿಷಯವಾಯಿತು. ಅನಂತರವೇ ಇದಕ್ಕಿಂತ\break ಮೇಲಿರುವ ವೇದಾಂತ ತತ್ತ್ವ ಬಂದು, ಇದು ಹೀಗಿರಲು ಸಾಧ್ಯವಿಲ್ಲ ಎನ್ನುವುದು. ದೇವರು ಜಗತ್ತಿನ ಉಪಾದಾನಕಾರಣ ಮತ್ತು ನಿಮಿತ್ತಕಾರಣ. ಅನಂತನಾದ ದೇವರೊಬ್ಬ\-ನಿರುವನು, ಅನಂತನಾದ ಜೀವನೊಬ್ಬನಿರುವನು, ಅನಂತವಾದ ಪ್ರಕೃತಿಯೊಂದು ಇರುವುದು ಎಂದು ಬೇಕಾದಷ್ಟು ಅನಂತಗಳನ್ನು ಹೇಳುತ್ತಾ ಹೋದರೆ, ಅವು ಯುಕ್ತಿಗೆ ವಿರೋಧವಾಗುವುವು. ಅವಕ್ಕೆ ಅರ್ಥವಿಲ್ಲ. ಆದಕಾರಣ ದೇವರು ಜಗತ್ತಿನ ಉಪಾದಾನಕಾರಣ ಮತ್ತು ನಿಮಿತ್ತಕಾರಣನಾಗಿರುವನು. ದೇವರು ತನ್ನಿಂದಲೇ ಈ ಸೃಷ್ಟಿಯನ್ನು ವ್ಯಕ್ತಗೊಳಿಸುವನು. ಹಾಗಾದರೆ ದೇವರು ಈ ಗೋಡೆ, ಮೇಜು, ಹಂದಿ, ಕೊಲೆಪಾತಕಿ, ಪಾಪಿ ಎಲ್ಲವೂ ಹೇಗೆ ಆಗಿರುವನು? ನಾವು ದೇವರನ್ನು ಪರಿಶುದ್ಧ ಎನ್ನುತ್ತೇವೆ. ಅವನು ಇವೆಲ್ಲ ಹೇಗೆ ಆಗುತ್ತಾನೆ? ಇದಕ್ಕೆ ನಮ್ಮ ಉತ್ತರ, ಹೇಗೆ ನಾನೊಂದು ಜೀವಿಯಾದರೂ ನನಗೆ ಒಂದು ದೇಹವಿದೆಯೋ,\break ಒಂದು ದೃಷ್ಟಿಯಿಂದ ಈ ದೇಹ ನನ್ನಿಂದ ಬೇರೆ ಅಲ್ಲ, ಆದರೂ ನಿಜವಾದ ಜೀವ\break ಈ ದೇಹವಲ್ಲ. ನಾನು ಮಗು, ಯುವಕ, ವೃದ್ಧ ಎಂದು ಹೇಳಿಕೊಂಡರೂ ನನ್ನ ಆತ್ಮ\break ಬದಲಾಗಿಲ್ಲ. ಇದು ಅದೇ ಆತ್ಮನಾಗಿರುವುದು. ಇದರಂತೆಯೇ ಇಡೀ ಬ್ರಹ್ಮಾಂಡ, ಪ್ರಕೃತಿ ಮತ್ತು ಕೋಟ್ಯಂತರ ಜೀವಗಳು, ಇವುಗಳನ್ನೊಳಗೊಂಡ ಈ ವಿಶ್ವವೇ ಭಗವಂತನ ದೇಹದಂತೆ. ದೇವರು ಇದರಲ್ಲೆಲ್ಲಾ ಹಾಸುಹೊಕ್ಕಾಗಿರುವನು, ಅವನೊಬ್ಬನೇ ಅವಿಕಾರಿ. ಆದರೆ ಪ್ರಕೃತಿ ಮತ್ತು ಜೀವಗಳು ಬದಲಾಗುತ್ತವೆ. ಪ್ರಕೃತಿಯಲ್ಲಿ ಮತ್ತು ಜೀವರಲ್ಲಿ ಬದಲಾವಣೆಯಾದರೆ ಅದರಿಂದ ಅವನಿಗೆ ಏನೂ ಬಾಧಕವಿಲ್ಲ. ಪ್ರಕೃತಿ ಹೇಗೆ ಬದಲಾಗುವುದು? ತನ್ನ ಆಕಾರದಲ್ಲಿ ಅದು ಹೊಸ ಹೊಸ ರೂಪಗಳನ್ನು ಧರಿಸುವುದು. ಆದರೆ ಆತ್ಮ ಹೀಗೆ ಬದಲಾಗುವುದಿಲ್ಲ. ಆತ್ಮನ ಜ್ಞಾನದಲ್ಲಿ ಸಂಕೋಚ-ವಿಕಾಸಗಳಾಗುವುವು. ಅದು ಪಾಪಕರ್ಮದಿಂದ ಸಂಕುಚಿತವಾಗುವುದು. ಆತ್ಮನ ಸ್ವಾಭಾವಿಕ ಜ್ಞಾನವನ್ನು ಮತ್ತು ಪೂರ್ಣತೆಯನ್ನು ಸಂಕುಚಿತ ಮಾಡುವುದೆಲ್ಲ ಪಾಪಕರ್ಮ. ಆತ್ಮನ ಸ್ವಾಭಾವಿಕವಾದ ಪರಿಪೂರ್ಣತೆಯನ್ನು ಯಾವುದು ಹೊರಗೆ ತರುವುದೊ ಅದೆಲ್ಲ ಪುಣ್ಯಕರ್ಮ. ಆತ್ಮಗಳೆಲ್ಲ ಪರಿಶುದ್ಧವಾಗಿದ್ದವು. ಆದರೆ ಈಗ ಅವು ಸಂಕುಚಿತವಾಗಿವೆ. ಭಗವಂತನ ದಯೆಯಿಂದ ಮತ್ತು ಪುಣ್ಯಕರ್ಮದಿಂದ ವಿಕಾಸಗೊಂಡು ತಮ್ಮ ಸ್ವಾಭಾವಿಕವಾದ ಪಾವಿತ್ರ್ಯವನ್ನು ವ್ಯಕ್ತಗೊಳಿಸುವುದಕ್ಕೆ ಪ್ರತಿಯೊಬ್ಬರಿಗೂ ಅವಕಾಶವಿದೆ. ಕೊನೆಗೆ ಎಲ್ಲರೂ ಇದರಿಂದ ಪಾರಾಗಬೇಕಾಗಿದೆ. ಆದರೆ ಈ ಸೃಷ್ಟಿ ಇಲ್ಲಿಗೇ ನಿಲ್ಲುವುದಿಲ್ಲ. ಏಕೆಂದರೆ ಇದು ಅನಂತ. ಇದೇ ಎರಡನೆಯ ಸಿದ್ಧಾಂತ. ಮೊದಲನೆಯದು ದ್ವೈತ. ಎರಡನೆಯದು ವಿಶಿಷ್ಟಾದ್ವೈತ, ಈ ತತ್ತ್ವದ ಪ್ರಕಾರ: ಜೀವ, ಜಗತ್ತು ಮತ್ತು ದೇವರು ಎಂಬ-ಈ ಮೂರು ಪರಸ್ಪರ ಬೇರ್ಪಡಿಸಲಾಗದ ಒಂದೇ ಸ್ಥಿತಿ ಅಥವಾ ವಿಶಿಷ್ಟ ರೀತಿಯ ಅದ್ವೈತ ಸ್ಥಿತಿ ಎನ್ನಬಹುದು. ಇದು ಮೊದಲನೆಯದಕ್ಕಿಂತ ಒಂದು ಮೆಟ್ಟಿಲು ಮೇಲೆ ಇರುವುದು. ಇದನ್ನೇ ವಿಶಿಷ್ಟಾದ್ವೈತ ಎನ್ನುವರು. ದ್ವೈತದಲ್ಲಿ ಬ್ರಹ್ಮಾಂಡವೇ ಒಂದು ಯಂತ್ರವೆಂದೂ, ಅದನ್ನು ನಡೆಸುತ್ತಿರುವವನು ದೇವರೆಂದೂ ಹೇಳುವರು. ಎರಡನೆಯದರಲ್ಲಿ ಈ ಬ್ರಹ್ಮಾಂಡವನ್ನು ಒಂದು ದೇಹವೆಂದೂ, ಇದರ ಒಳಗೆ ಮತ್ತು ಹೊರಗೆಲ್ಲ ಭಗವಂತ ಓತಪ್ರೋತನಾಗಿರುವನೆಂದೂ ಹೇಳುವರು.

ಕೊನೆಯದೆ ಅದ್ವೈತ. ಅದ್ವೈತಿಗಳು ದೇವರು ಸೃಷ್ಟಿಯ ಉಪಾದಾನಕಾರಣ ಮತ್ತು ನಿಮಿತ್ತ ಕಾರಣವಾಗಿರಬೇಕೆಂಬ ಪ್ರಶ್ನೆಯನ್ನು ಎತ್ತುವರು. ಆದಕಾರಣ ದೇವರೇ ಈ ವಿಶ್ವವಾಗಿರುವನು. ಇದಕ್ಕೆ ವಿರುದ್ಧವಾಗಿ ಏನೂ ಹೇಳುವಂತಿಲ್ಲ. ಆದರೆ ಇತರರು, ದೇವರೆ ಆತ್ಮ, ಈ ಪ್ರಪಂಚವೇ ಅವನ ದೇಹ, ಇದು ಬದಲಾಗುತ್ತಿದೆ ಎಂದರೆ, ಅದ್ವೈತಿಗಳು ಇದಕ್ಕೆ ಅರ್ಥವಿಲ್ಲ ಎನ್ನುವರು. ಹಾಗಾದರೆ ದೇವರು ಈ ಪ್ರಪಂಚದ ಉಪಾದಾನಕಾರಣ ಎಂದು ಹೇಳಿದುದರ ಅರ್ಥವೇನು ಎಂದು ಕೇಳುವನು. ಉಪಾದಾನಕಾರಣ ಎಂದರೆ ಕಾರಣವೇ ಕಾರ್ಯವಾಗಿದೆ. ಕಾರ್ಯವು ಕಾರಣದ ಬೇರೊಂದು ಅವಸ್ಥೆ ಅಷ್ಟೆ. ನೀವು ಎಲ್ಲಿ ಒಂದು ಕಾರ್ಯವನ್ನು ನೋಡುತ್ತೀರೋ ಅದೆಲ್ಲ ಕಾರಣದ ಬೇರೊದು ಅವಸ್ಥೆ. ಈ ವಿಶ್ವ ಒಂದು ಕಾರ್ಯವೇ ಆಗಿರಬೇಕು. ಭಗವಂತ ಕಾರಣರೂಪ. ಇದು ಭಗವಂತನ ಅಭಿವ್ಯಕ್ತಿ ಈ ವಿಶ್ವವು ಭಗವಂತನ ದೇಹ, ಈ ದೇಹ ಸಂಕುಚಿತವಾಗಿ, ಸೂಕ್ಷ್ಮವಾಗಿ, ಕಾರಣವಾಗುವುದು, ಅದರಿಂದ, ಈ ಜಗತ್ತು ಬಂದಿದೆ ಎಂದು ನೀವು ಹೇಳಿದರೆ, ಅದ್ವೈತಿಯು ದೇವರೇ ಈ ಪ್ರಪಂಚವಾಗಿರುವನು ಎನ್ನುವನು. ಈಗ ಒಂದು ಸೂಕ್ಷ್ಮ ಪ್ರಶ್ನೆ ಏಳುವುದು. ದೇವರೇ ಈ ಪ್ರಪಂಚವಾಗಿದ್ದರೆ, ನೀವು ಮತ್ತು ಇತರ ವಸ್ತುಗಳೆಲ್ಲವೂ ದೇವರೇ ಆದಿರಿ. ನಿಜವಾಗಿ ಈ ಗ್ರಂಥವೂ ದೇವರಾಗಬೇಕಾಯಿತು, ಪ್ರತಿಯೊಂದೂ ದೇವರಾಗಬೇಕಾಯಿತು. ನನ್ನ ದೇಹವು ಭಗವಂತ, ನನ್ನ ಮನಸ್ಸು ಭಗವಂತ, ನನ್ನ ಆತ್ಮ ಭಗವಂತ. ಹಾಗಾದರೆ ಇಷ್ಟೊಂದು ಜೀವಗಳು ಏಕೆ ಇವೆ? ದೇವರು ಕೋಟ್ಯಂತರ ಜೀವಗಳಾಗಿ ಭಾಗವಾದನೆ? ಆದರೆ, ಹೇಗೆ ಆದ? ಆ ಅಖಂಡವಾದ, ಅನಂತವಾದ ಏಕಮಾತ್ರ ಶಕ್ತಿ ಹೇಗೆ ಕವಲೊಡೆದು ಭಿನ್ನವಾಯಿತು? ನಾವು ಅನಂತವನ್ನು ವಿಭಾಗಿಸಲಾರೆವು. ಆ ಪವಿತ್ರಾತ್ಮ ಈ ಪ್ರಪಂಚ ಹೇಗೆ ಆಗಬಲ್ಲ? ಅವನು ಈ ಪ್ರಪಂಚವಾದರೆ ಅವನೂ ವಿಕಾರಿಯಾಗುತ್ತಾನೆ, ಅವನು ಬದಲಾದರೆ ಪ್ರಕೃತಿಯ ಭಾಗವಾಗುವನು. ಯಾವುದು ಪ್ರಕೃತಿಯಲ್ಲಿದೆಯೋ, ಯಾವುದು ಬದಲಾಗುವುದೋ, ಅದು ಜನನಮರಣಗಳಿಗೆ ತುತ್ತಾಗುವುದು. ನಮ್ಮ ದೇವರು ವಿಕಾರಿಯಾದರೆ ಅವನೂ ಒಂದು ದಿನ ಸಾಯಬೇಕು. ಇದನ್ನು ಗಮನಿಸಿ. ಪುನಃ ದೇವರ ಎಷ್ಟು ಭಾಗ ಈ ಪ್ರಪಂಚವಾಗಿದೆ? ಯಾವುದೋ ಒಂದು \enginline{X} ಭಾಗ ಈಗ ಪ್ರಪಂಚವಾಗಿದೆ ಎಂದರೆ, ದೇವರು \enginline{\supskpt{-}x} ಆಗುವನು. ಸೃಷ್ಟಿಗೆ ಮುಂಚೆ ಇದ್ದ ದೇವರಲ್ಲ ಈಗ ಇರುವ ದೇವರು. ಏಕೆಂದರೆ ಅವನ ಸ್ವಲ್ಪ ಭಾಗ ಆಗಲೆ ವಿಶ್ವವಾಗಿದೆ.

ಆದಕಾರಣ ಅದ್ವೈತ ಹೀಗೆ ಹೇಳುವುದು: ಈ ಪ್ರಪಂಚ ಇಲ್ಲವೇ ಇಲ್ಲ. ಇದೊಂದು ಭ್ರಾಂತಿ. ಈ ಬ್ರಹ್ಮಾಂಡವೆಲ್ಲ-ದೇವತೆಗಳು, ದೇವದೂತರು, ಹುಟ್ಟಿಸಾಯುವ ಎಲ್ಲರೂ, ಈ ಜಗತ್ತಿಗೆ ಬಂದು ಹೋಗುತ್ತಿರುವ ಕೋಟ್ಯಂತರ ಜೀವಿಗಳೆಲ್ಲ ಒಂದು ಕನಸಿನಂತೆ. ನಿಜವಾಗಿ ಜೀವವೇ ಇಲ್ಲದಿರುವಾಗ ಇಷ್ಟೊಂದು ಜೀವಿಗಳು ಹೇಗೆ ಇರಬಲ್ಲವು? ಇರುವುದೆಲ್ಲ ಒಂದು ಅನಂತ. ಒಬ್ಬ ಸೂರ್ಯ ಹಲವು ಹಿಮಮಣಿಗಳ ಮೇಲೆ ಬಿದ್ದಾಗ ಹೇಗೆ ಹಲವು ಸೂರ್ಯರಂತೆ ಕಾಣುವನೋ, ಅದರ ಪ್ರತಿಯೊಂದು ಪ್ರತಿಬಿಂಬದಲ್ಲಿಯೂ ಒಬ್ಬೊಬ್ಬ ಪರಿಪೂರ್ಣ ಸೂರ್ಯನಿರುವಂತೆ ಕಂಡರೂ, ಸೂರ್ಯ ಮಾತ್ರ ಒಬ್ಬನೇ ಇರುವಂತೆ, ಈ ಜೀವಿಗಳೆಲ್ಲ ಹಲವು ಪ್ರತಿಬಿಂಬಗಳು. ಹಲವು ಮನಸ್ಸುಗಳು ಒಬ್ಬನೇ ಪರಮಾತ್ಮನನ್ನು ಪ್ರತಿಬಿಂಬಿಸುವ ಹಲವು ಹಿಮಮಣಿಗಳಂತೆ, ದೇವರು ಎಲ್ಲಾ ಜೀವಿಗಳಲ್ಲಿಯೂ ಪ್ರತಿಬಿಂಬಿತನಾಗುತ್ತಿರುವನು. ಕನಸಿಗೆ ಯಾವುದಾದರೂ ಒಂದು ಸತ್ಯದ ಆಧಾರ ಬೇಕಾಗಿದೆ. ಆ ಸತ್ಯವೇ ಅಖಂಡ ಬ್ರಹ್ಮ. ದೇಹ ಮನಸ್ಸು ಜೀವದಂತೆ ಇರುವ ನೀವೊಂದು ಭ್ರಾಂತಿ. ಆದರೆ ನಿಜವಾಗಿ ಸಚ್ಚಿದಾನಂದ ಸ್ವರೂಪ ನೀವೇ, ಜಗತ್ತಿನ ದೇವರು ನೀವೇ. ಈ ವಿಶ್ವವನ್ನು ಸೃಷ್ಟಿಸುತ್ತಿರುವವನು, ತಿರೋಧಾನ ಮಾಡುತ್ತಿರುವವನು ನೀವೇ ಎಂದು ಅದ್ವೈತ ಹೇಳುವುದು. ಆದಕಾರಣ ಈ ಜನನ ಮರಣಗಳೆಲ್ಲ. ಬರುವುದು ಹೋಗುವುದೆಲ್ಲ ಒಂದು ಮಾಯೆ. ನೀನು ಅನಂತ, ನೀನು ಹೋಗುವುದೆಲ್ಲಿಗೆ? ಸೂರ್ಯಚಂದ್ರರೆಲ್ಲ, ಇಡೀ ಬ್ರಹ್ಮಾಂಡ ನಿನ್ನ ನೈಜ ಅತೀಂದ್ರಿಯ ಸ್ವಭಾವದಲ್ಲಿ ಒಂದು ಬಿಂದುವಿನಂತೆ. ನೀನು ಹೇಗೆ ಹುಟ್ಟಬಲ್ಲೆ ಅಥವಾ ಸಾಯಬಲ್ಲೆ? ನಾನು ಎಂದೂ ಹುಟ್ಟಿಲ್ಲ; ಹುಟ್ಟುವುದೂ ಇಲ್ಲ; ನನಗೆ ಎಂದಿಗೂ ತಾಯಿತಂದೆಗಳು ಇರಲಿಲ್ಲ. ಮಿತ್ರ ಶತ್ರುಗಳೂ ಇರಲಿಲ್ಲ. ಏಕೆಂದರೆ ನಾನೇ ಸಚ್ಚಿದಾನಂದ, ಶಿವೋಽಹಂ. ಈ ಸಿದ್ಧಾಂತದ ಪ್ರಕಾರ ಗುರಿ ಏನು? ಯಾರು ಈ ಜ್ಞಾನವನ್ನು ಸ್ವೀಕರಿಸುವರೊ ಅವರು ವಿಶ್ವದಲ್ಲಿ ಒಂದಾಗಿರುವರು. ಅವರಿಗೆ ಸ್ವರ್ಗಲೋಕ, ಬ್ರಹ್ಮಲೋಕ ಎಂಬುದೇ ಇಲ್ಲ. ಅವರಿಗೆ ಈ ಭ್ರಾಂತಿಯೆಲ್ಲ ನಾಶವಾಗಿದೆ. ಅವರೇ ಈ ವಿಶ್ವದ ಸನಾತನ ದೇವರೆಂದು ಅವರಿಗೆ ಅರಿವಾಗಿದೆ. ಅವರು ತಮ್ಮ ನಿಜವಾದ ವ್ಯಕ್ತಿತ್ವವನ್ನು ಪಡೆಯುವರು. ಅನಂತ ಜ್ಞಾನ ಮತ್ತು ಆನಂದ ಅವರದಾಗಿ ಅವರು ಮುಕ್ತರಾಗುವರು. ಅಲ್ಪದರಲ್ಲಿ ಅವರಿಗೆ ಸುಖವಿಲ್ಲ. ನಾವು ಈ ಅಲ್ಪದೇಹದಲ್ಲಿ, ಅಲ್ಪವ್ಯಕ್ತಿತ್ವದಲ್ಲಿ ಸಂತೋಷವನ್ನು ಪಡೆಯುತ್ತಿರುವೆವು. ಈ ಬ್ರಹ್ಮಾಂಡವೇ ನನ್ನ ದೇಹವಾದರೆ ಆಗ ಆನಂದ ಎಷ್ಟು ಅಧಿಕವಾಗುವುದು! ಆಗಲೇ ಸ್ವಾತಂತ್ರ್ಯ ಪ್ರಾಪ್ತಿ, ಇದನ್ನೆ ಅದ್ವೈತವೆನ್ನುವುದು.

ವೇದಾಂತದ ಮೂರು ಮೆಟ್ಟಿಲುಗಳು ಇವು. ನಾವು ಇದಕ್ಕಿಂತ ಮುಂದೆ ಹೋಗಲಾರೆವು. ಏಕೆಂದರೆ ನಾವು ಏಕತೆಯನ್ನು ಮೀರಿ ಹೋಗಲಾರೆವು. ವಿಜ್ಞಾನ ಒಂದು\break ಏಕತೆಯನ್ನು ಪಡೆದ ಮೇಲೆ ಮುಂದೆ ಹೋಗಲಾರದು. ನೀವು ಈ ನಿರಪೇಕ್ಷವನ್ನು ಮೀರಿ ಹೋಗಲಾರಿರಿ.

ಎಲ್ಲರಿಗೂ ಅದ್ವೈತ ತತ್ತ್ವವನ್ನು ಸ್ವೀಕರಿಸಲು ಆಗುವುದಿಲ್ಲ. ಅದು ಬಹಳ ಕಷ್ಟ. ಮೊದಲನೆಯದಾಗಿ ಅದನ್ನು ತಿಳಿದುಕೊಳ್ಳುವುದೇ ಕಷ್ಟ. ಬುದ್ಧಿ ತುಂಬಾ ಸೂಕ್ಷ್ಮವಾಗಿರಬೇಕು, ತುಂಬಾ ಧೈರ್ಯ ಇರಬೇಕು. ಎರಡನೆಯದಾಗಿ ಮುಕ್ಕಾಲು ಪಾಲು ಜನರಿಗೆ ಅದು ಹಿಡಿಸುವುದಿಲ್ಲ. ಆದಕಾರಣವೆ ಮೂರು ಮೆಟ್ಟಿಲುಗಳು ಇರುವುದು. ಮೊದಲನೆಯ ಮೆಟ್ಟಿಲಿನಿಂದ ಪ್ರಾರಂಭಿಸಿ, ಮೊದಲನೆಯದನ್ನು ಅರ್ಥಮಾಡಿಕೊಂಡ ಮೇಲೆ\break ಎರಡನೆಯದು ತಾನಾಗಿಯೆ ಮುಂದೆ ಬರುವುದು. ಜನಾಂಗ ಮುಂದುವರಿದಂತೆ ವ್ಯಕ್ತಿಯೂ ಮುಂದುವರಿಯಬೇಕಾಗಿದೆ. ಮಾನವ ಜನಾಂಗ ಯಾವ ದಾರಿಯ ಮೂಲಕ ಹೋಗಿದೆಯೊ ಅದೇ ದಾರಿಯಲ್ಲಿ ಪ್ರತಿಯೊಬ್ಬ ವ್ಯಕ್ತಿಯೂ ಹೋಗಬೇಕಾಗಿದೆ. ಇಡೀ ಮಾನವಕೋಟಿ ಒಂದು ಮೆಟ್ಟಿಲಿನಿಂದ ಮತ್ತೊಂದು ಮೆಟ್ಟಿಲಿಗೆ ಹೋಗುವುದಕ್ಕೆ ಕೋಟ್ಯಂತರ ವರ್ಷಗಳು ತೆಗೆದುಕೊಂಡಿರಬಹುದು. ಆದರೆ ವ್ಯಕ್ತಿಗಳು ಅದನ್ನು ಬಹಳ ಕಡಿಮೆ ಕಾಲದಲ್ಲಿ ಮುಗಿಸಬಹುದು. ಆದರೆ ನಮ್ಮಲ್ಲಿ ಪ್ರತಿಯೊಬ್ಬರೂ ಈ ಮೆಟ್ಟಿಲುಗಳ ಮೂಲಕವಾಗಿಯೇ ಹೋಗಬೇಕಾಗಿದೆ. ನಿಮ್ಮಲ್ಲಿ ಅದ್ವೈತಿಗಳಾದವರು ಉಗ್ರದ್ವೈತಿಗಳಾಗಿದ್ದ ಕಾಲವನ್ನು ಯೋಚಿಸಿ ನೋಡಿ. ನೀವು ದೇಹ ಮತ್ತು ಮನಸ್ಸು ಎಂದೊಡನೆಯೇ ಇಡೀ ಕನಸನ್ನು ಸ್ವೀಕರಿಸಬೇಕಾಗುತ್ತದೆ. ನೀವು ಇದರಲ್ಲಿ ಒಂದು ಭಾಗವನ್ನು ತೆಗೆದುಕೊಂಡರೆ ಉಳಿದಿರುವುದನ್ನೆಲ್ಲ ತೆಗೆದುಕೊಳ್ಳಬೇಕಾಗಿದೆ. ಯಾರು ‘ಪ್ರಪಂಚ ಇಲ್ಲಿದೆ ನಿಜ, ಆದರೆ ದೇವರಿಲ್ಲ’ - ಎನ್ನುವರೊ ಅವರು ಮೂರ್ಖರು. ಏಕೆಂದರೆ ಒಂದು ಪ್ರಪಂಚ ಇದ್ದರೆ, ಅದಕ್ಕೊಂದು ಕಾರಣವಿರಬೇಕು. ಅದೇ ದೇವರು ಎಂಬುದು. ಕಾರಣವಿಲ್ಲದೆ ಕಾರ್ಯವಾಗಲಾರದು. ಪ್ರಪಂಚ ಮಾಯವಾದಾಗ ದೇವರೂ ಮಾಯವಾಗುವನು. ಆಗ ನೀನು ಬ್ರಹ್ಮನಾಗುವೆ.\break ಆಗ ನಿನಗೆ ಈ ಪ್ರಪಂಚವೇ ಇರುವುದಿಲ್ಲ. ನೀನೊಂದು ದೇಹ ಎಂಬ ಕನಸು\break ಇರುವವರೆಗೆ ನೀನು ಹುಟ್ಟುಸಾವುಗಳನ್ನು ನೋಡಲೇಬೇಕಾಗಿದೆ. ಆದರೆ ಕನಸು\break ಮಾಯವಾದೊಡನೆ ನೀನು ಹುಟ್ಟುತ್ತಿರುವೆ ಸಾಯುತ್ತಿರುವೆ ಎಂಬ ಕನಸು ಕೂಡ\break ಮಾಯವಾಗುವುದು. ಯಾವುದು ಈಗ ಪ್ರಪಂಚದಂತೆ ಕಾಣುತ್ತಿದೆಯೋ ಅದೇ ಅನಂತ ಬ್ರಹ್ಮನಂತೆ ಕಾಣುವುದು. ಯಾವ ದೇವರು ಇದುವರೆಗೆ ಬಾಹ್ಯದಲ್ಲಿದ್ದನೊ ಅವನೇ ಅಂತರ್ಯಾಮಿಯಾಗುವನು, ನಮ್ಮ ಆತ್ಮನಾಗುವನು.

