
\chapter[ಹಿಂದೂಗಳು ಮತ್ತು ಗ್ರೀಕರು ]{ಹಿಂದೂಗಳು ಮತ್ತು ಗ್ರೀಕರು \protect\footnote{\engfoot{C.W. Vol. VI, p85}}}

ಮೂರು ಪರ್ವತಗಳು ಪ್ರಗತಿಯ ಚಿಹ್ನೆಗಳಂತೆ ನಿಂತಿವೆ: ಇಂಡೋ ಆರ್ಯನರ\break ಹಿಮಾಲಯ, ಹಿಬ್ರೂಗಳ ಸಿನೈ, ಗ್ರೀಕರ ಒಲಿಂಪಸ್​. ಆರ್ಯರು ಭರತಖಂಡಕ್ಕೆ ಬಂದಮೇಲೆ ಇಲ್ಲಿಯ ಹವಾಗುಣ ಬಹಳ ಉಷ್ಣವಾಗಿದ್ದುದರಿಂದ ಹೆಚ್ಚು ಶ್ರಮಪಟ್ಟು ಕೆಲಸ ಮಾಡಲು ಆಗಲಿಲ್ಲ. ಆದಕಾರಣ ಆಲೋಚನೆ ಮಾಡಲು ಯತ್ನಿಸಿದರು. ಅಂತರ್ಮುಖಿಗಳಾಗಿ ಧಾರ್ಮಿಕ ಜೀವನದಲ್ಲಿ ಮುಂದುವರಿದರು. ಶಕ್ತಿಗೆ ಒಂದು ಮಿತಿಯಿಲ್ಲ ಎಂಬುದನ್ನು ಅರಿತರು. ಅದನ್ನು ನಿಗ್ರಹಿಸಲು ಯತ್ನಿಸಿದರು. ಇದರ ಮೂಲಕ ದೇಹದಲ್ಲಿ ಯಾವುದೋ ಸುಪ್ತವಾಗಿರುವ ಶಕ್ತಿ ವ್ಯಕ್ತವಾಗಲು ಯತ್ನಿಸುತ್ತಿದೆ ಎಂಬುದನ್ನು ಅರಿತರು. ಇದನ್ನು ವ್ಯಕ್ತಗೊಳಿಸುವುದೇ ಅವರ ಮುಖ್ಯ ಗುರಿಯಾಯಿತು. ಆರ್ಯರ ಮತ್ತೊಂದು ಶಾಖೆಯು ಸಣ್ಣದಾದರೂ ಅತಿ ಸುಂದರವಾದ ಗ್ರೀಸ್​ಗೆ ಹೋಯಿತು. ಅಲ್ಲಿಯ\break ವಾತಾವರಣ ಅನುಕೂಲವಾಗಿದ್ದುದರಿಂದ ಬಾಹ್ಯ ಪ್ರವೃತ್ತಿಯವರಾಗಿ ಹಲವು ಕಲೆ\break ಮತ್ತು ಬಾಹ್ಯ ಸ್ವಾತಂತ್ರ್ಯಗಳ ಬೆಳವಣಿಗೆಗಳಿಗೆ ಕಾರಣರಾದರು. ಗ್ರೀಕರು ರಾಜಕೀಯ ಸ್ವಾತಂತ್ರ್ಯವನ್ನು ಅರಸಿದರು. ಹಿಂದೂಗಳು ಆಧ್ಯಾತ್ಮಿಕ ಸ್ವಾತಂತ್ರ್ಯವನ್ನು ಅರಸಿದರು.\break ಇಬ್ಬರ ದೃಷ್ಟಿಯೂ ಪಕ್ಷಪಾತದಿಂದ ಕೂಡಿರುವುದು. ಭಾರತೀಯನು ದೇಶಭಕ್ತಿ, ದೇಶರಕ್ಷಣೆ ಇವುಗಳನ್ನು ಗಣನೆಗೆ ತೆಗೆದುಕೊಳ್ಳುವುದಿಲ್ಲ; ಕೇವಲ ತನ್ನ ಧರ್ಮವನ್ನು ರಕ್ಷಿಸುವನು. ಗ್ರೀಸಿನಲ್ಲಿ ಮತ್ತು ಗ್ರೀಕ್​ ಸಂಸ್ಕೃತಿ ಯಾವ ದೇಶದಲ್ಲಿ ಮುಂದುವರಿಯಿತೋ\break ಅಂತಹ ಯೂರೋಪಿನಲ್ಲಿ ದೇಶ ಮೊದಲು ಬರುವುದು. ಕೇವಲ ಧಾರ್ಮಿಕ ಸ್ವಾತಂತ್ರ್ಯವನ್ನು ಲಕ್ಷಿಸುವುದು, ದೇಶದ ಸ್ವಾತಂತ್ರ್ಯವನ್ನು ಅಲಕ್ಷಿಸುವುದು ಒಂದು ಕುಂದು. ಆದರೆ ಪ್ರತಿಪಕ್ಷದವರಲ್ಲಿ ಇದಕ್ಕಿಂತಲೂ ಹೆಚ್ಚು ಕುಂದು ಇರುವುದು. ದೇಶ ಸ್ವಾತಂತ್ರ್ಯ ಮತ್ತು ಆತ್ಮ ಸ್ವಾತಂತ್ರ್ಯ ಇವೆರಡನ್ನೂ ಆನವು ಆಶಿಸಬೇಕಾಗಿದೆ.

