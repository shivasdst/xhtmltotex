
\chapter[ಸೋಽಹಂ ]{ಸೋಽಹಂ \protect\footnote{\engfoot{C.W. Vol. VIII, p. 244}}}

\centerline{\textbf{(1900 ಮಾರ್ಚಿ 20ರಂದು ಸ್ಯಾನ್​ಫ್ರಾನ್ಸಿಸ್ಕೋದಲ್ಲಿ ನೀಡಿದ ಉಪನ್ಯಾಸದ ಟಿಪ್ಪಣಿಗಳು)}}

\vskip 0.3cm

ನಾನಿಂದು ಮಾತನಾಡಬೇಕೆಂದು ಇರುವುದು ಮಾನವನ ವಿಷಯವಾಗಿ, ಪ್ರಕೃತಿಗೆ ವಿರುದ್ಧವಾಗಿ ಮಾನವನ ವಿಷಯವಾಗಿ. ಬಹಳ ಕಾಲದವರೆಗೆ ಪ್ರಕೃತಿ ಎಂಬ ಶಬ್ದವನ್ನು ಕೇವಲ ಬಾಹ್ಯ ಘಟನೆಗಳನ್ನು ಮಾತ್ರ ನಿರ್ದೇಶಿಸುವುದಕ್ಕೆ ಉಪಯೋಗಿಸುತ್ತಿದ್ದರು. ಇವು ಒಂದು ಕ್ರಮದಲ್ಲಿ ಆಗುತ್ತಿರುವಂತೆ ಕಾಣಿಸಿದವು. ಕೆಲವು ವೇಳೆ ಪುನಃ ಪುನಃ ಆಗುತ್ತಿದ್ದುವು. ಹಿಂದೆ ಆದುದು ಈಗಲೂ ಆಗುವುದು. ಯಾವುದೂ ಒಂದು ವೇಳೆ ಮಾತ್ರ ಆಗಿ ಅನಂತರ ಆಗದೆ ಇರುವುದಿಲ್ಲ. ಈ ಕಾರಣಗಳಿಂದ ಪ್ರಕೃತಿಯಲ್ಲಿ ಏಕರೂಪತೆ ಇದೆ ಎಂಬ ಭಾವನೆ ನಮ್ಮ ಮನಸ್ಸಿನಲ್ಲಿ ಬಂದು ಹೋಗಿದೆ. ಏಕರೂಪತೆ ಇಲ್ಲದೆ ಇದ್ದರೆ ಬಾಹ್ಯಘಟನೆಗಳನ್ನು ನಾವು ತಿಳಿದುಕೊಳ್ಳುವುದಕ್ಕೆ ಆಗುವುದಿಲ್ಲ. ನಾವು ಯಾವುದನ್ನು ನಿಯಮ ಎನ್ನುವೆವೋ ಅದಕ್ಕೆ\break ಏಕರೂಪತೆಯೇ ಆಧಾರ.

ಕ್ರಮೇಣ ‘ಪ್ರಕೃತಿ’ ಮತ್ತು ‘ಏಕರೂಪತೆ’ಯ ಭಾವನೆಗಳನ್ನು ಆಂತರಿಕ ಘಟನೆಗಳಿಗೂ, ಜೀವನ ಮತ್ತು ಮನಸ್ಸುಗಳ ವಿಷಯಕ್ಕೂ ಅನ್ವಯಿಸಲಾಯಿತು. ಭಿನ್ನತೆಯೇ ಪ್ರಕೃತಿ.\break ಪ್ರಕೃತಿಯೇ ಗಿಡದ ಗುಣ ಪ್ರಾಣಿಯ ಗುಣ ಮತ್ತು ಮಾನವನ ಗುಣ. ಮಾನವನ\break ಜೀವನವು ಒಂದು ನಿರ್ದಿಷ್ಟವಾದ ಕ್ರಮವನ್ನು ಅನುಸರಿಸುತ್ತದೆ. ಇದರಂತೆಯೇ ಅವನ ಮನಸ್ಸು ಕೂಡ. ಆಲೋಚನೆಗಳು ಇದ್ದಕ್ಕೆ ಇದ್ದಂತೆ ಮನಸ್ಸಿನಲ್ಲಿ ಏಳುವುದಿಲ್ಲ. ಆಲೋಚನೆ ಏಳುಬೀಳುಗಳ ಹಿಂದೆ ಒಂದು ಕ್ರಮವಿದೆ. ಹೇಗೆ ಬಾಹ್ಯ ಘಟನೆಗಳು ಒಂದು ನಿಯಮಕ್ಕೆ\break ಅಧೀನವೋ ಹಾಗೆಯೇ ಆಂತರಿಕ ಘಟನೆಗಳು ಎಂದರೆ ಮಾನವನ ಬದುಕು ಮತ್ತು\break ಚಿಂತನೆಗಳು ಒಂದು ನಿಯಮಕ್ಕೆ ಬದ್ಧವಾಗಿವೆ.

ಮಾನವ ಜೀವನ ಮತ್ತು ಮನಸ್ಸು ಒಂದು ನಿಯಮಕ್ಕೆ ಅಧೀನ ಎಂದು ಹೇಳುವುದಾದರೆ ಇಚ್ಛಾಸ್ವಾತಂತ್ರ್ಯ ಮತ್ತು ಸ್ವತಂತ್ರ ಅಸ್ತಿತ್ವ ಎಂಬುವು ಇಲ್ಲವೇ ಇಲ್ಲ ಎಂದಾಯಿತು. ಪ್ರಾಣಿಗಳ ಸ್ವಭಾವ ಒಂದು ನಿಯಮಕ್ಕೆ ಬದ್ಧವಾಗಿದೆ ಎಂದು ನಮಗೆ ಗೊತ್ತಿದೆ. ಪ್ರಾಣಿಗಳು ಇಚ್ಛಾಸ್ವಾತಂತ್ರ್ಯವನ್ನು ವ್ಯಕ್ತಪಡಿಸುವಂತೆ ಕಾಣಿಸುವುದಿಲ್ಲ. ಇದೇ ಮಾನವರಿಗೂ\break ಅನ್ವಯಿಸುತ್ತದೆ. ಮಾನವ ಸ್ವಭಾವವೂ ನಿಯಮಕ್ಕೆ ಬದ್ಧವಾಗಿದೆ. ಮಾನವನ ಮನಸ್ಸನ್ನು ಆಳುತ್ತಿರುವ ನಿಯಮಕ್ಕೆ ಕರ್ಮ ನಿಯಮ ಎಂದು ಹೆಸರು.

ಯಾರೂ ಶೂನ್ಯದಿಂದ ಏನಾದರೂ ತಯಾರಾಗುವುದನ್ನು ನೋಡಿಲ್ಲ. ಮನಸ್ಸಿನಲ್ಲಿ ಯಾವ ಆಲೋಚನೆಯಾದರೂ ಎದ್ದರೆ ಅದು ಮತ್ತಾವುದರಿಂದಲೋ ಎದ್ದಿರಬೇಕು. ನಾವು ಇಚ್ಛಾಸ್ವಾತಂತ್ರ್ಯ ಎಂದಾಗ ಇಚ್ಛೆ ಯಾವುದರಿಂದಲೂ ತಯಾರಾಗಿಲ್ಲ ಎಂದಾಗುತ್ತದೆ. ಆದರೆ ಇದು ನಿಜವಲ್ಲ. ಇಚ್ಛೆ ಮತ್ತಾವುದರಿಂದಲೋ ಆಗಿದೆ. ಆದುದರಿಂದ ಅದು ಸ್ವತಂತ್ರವಲ್ಲ, ನಿಯಮಕ್ಕೆ ಬದ್ಧ. ನಾನು ನಿಮ್ಮೊಡನೆ ಮಾತನಾಡಲು ಇಚ್ಛಿಸುವುದು, ನೀವು ಕೇಳುವುದಕ್ಕೆ ಬರುವುದು ಇವೆರಡೂ ಒಂದು ನಿಯಮವೇ. ನಾನು ಮಾಡುವುದು, ಆಲೋಚಿಸುವುದು, ಭಾವಿಸುವುದು ಮತ್ತು ನನ್ನ ಚಲನೆಯ ಪ್ರತಿಯೊಂದು ಭಾಗವೂ ಯಾವುದರಿಂದಲೋ ಆಗಿದೆ. ಆದಕಾರಣ ಸ್ವತಂತ್ರವಲ್ಲ. ನಮ್ಮ ಜೀವನ ಮತ್ತು ಮನಸ್ಸು ಯಾವುದೋ ಒಂದು ನಿಯಮವನ್ನು ಅನುಸರಿಸಿ ಹೋಗುತ್ತಿವೆ. ಅದೇ ಕರ್ಮ ನಿಯಮ.

ಹಿಂದಿನ ಕಾಲದಲ್ಲಿ ಇಂತಹ ಸಿದ್ಧಾಂತವನ್ನು ಪಾಶ್ಚಾತ್ಯ ದೇಶಗಳಿಗೆ ಹೇಳಿಕೊಟ್ಟಿದ್ದರೆ\break ಪ್ರಚಂಡ ಕೋಲಾಹಲವೇ ಉಂಟಾಗುತ್ತಿತ್ತು. ಪಾಶ್ಚಾತ್ಯರು ತಮ್ಮ ಮನಸ್ಸು ಒಂದು\break ನಿಯಮಕ್ಕೆ ಬದ್ಧವಾಗಿದೆ ಎಂದು ಆಲೋಚಿಸಲು ಇಚ್ಚಿಸುವುದಿಲ್ಲ. ಇಂಡಿಯಾ ದೇಶದಲ್ಲಿ\-ಯಾದರೋ ಅತಿ ಪುರಾತನಕಾಲದ ದಾರ್ಶನಿಕರು ಇದನ್ನು ಹೇಳಿದಾಗ ಎಲ್ಲರೂ ಕೂಡಲೇ ಒಪ್ಪಿಕೊಂಡರು. ಮಾನಸಿಕ ಸ್ವಾತಂತ್ರ್ಯ ಎಂಬುದು ಇಲ್ಲ. ಅದು ಇರಲಾರದು. ಇದರಿಂದ ಭಾರತೀಯ ಮನೋಲೋಕದಲ್ಲಿ ಏಕೆ ಗೊಂದಲ ಉಂಟಾಗಲಿಲ್ಲ? ಭಾರತೀಯರು ಶಾಂತವಾಗಿ ಇದನ್ನು ಸ್ವೀಕರಿಸಿದರು. ಭಾರತೀಯ ಚಿಂತನೆಯ ಒಂದು ವೈಶಿಷ್ಟ್ಯ ಇದು. ಈ ವಿಷಯದಲ್ಲಿ ಅದು ಜಗತ್ತಿನ ಇತರ ಜನಾಂಗಗಳ ಆಲೋಚನಾ ವಿಧಾನಕ್ಕಿಂತ ಭಿನ್ನವಾದದ್ದು.

ಬಾಹ್ಯ ಮತ್ತು ಆಂತರಿಕ ಪ್ರಕೃತಿಗಳೆರಡೂ ಬೇರೆ ಬೇರೆ ಅಲ್ಲ; ಎರಡೂ ವಾಸ್ತವವಾಗಿ ಒಂದೇ. ಪ್ರಕೃತಿ ಎಂದರೆ ಎಲ್ಲ ಘಟನೆಗಳ ಮೊತ್ತ. ಪ್ರಕೃತಿ ಎಂದರೆ ಚರಾಚರ ವಸ್ತುಗಳೆಲ್ಲ ಸೇರಿ ಆದದ್ದು. ಮನಸ್ಸು ಮತ್ತು ಭೌತವಸ್ತು ಇವುಗಳಲ್ಲಿ ನಾವು ಒಂದು ದೊಡ್ಡ ಭೇದವನ್ನು ಕಲ್ಪಿಸುತ್ತೇವೆ. ಮನಸ್ಸು ಭೌತವಸ್ತುವಿಗಿಂತ ಸಂಪೂರ್ಣ ಬೇರೆ ಎಂದು ಭಾವಿಸುತ್ತೇವೆ. ನಿಜವಾಗಿ ಇವೆರಡೂ ಪ್ರಕೃತಿಯ ಎರಡು ಭಾಗಗಳು. ಒಂದು ಯಾವಾಗಲೂ ಮತ್ತೊಂದರ ಮೇಲೆ ತನ್ನ ಪ್ರಭಾವವನ್ನು ಬೀರುತ್ತಿದೆ. ಭೌತವಸ್ತುವು ವಿವಿಧ ಸಂವೇದನೆಗಳ ರೂಪದಲ್ಲಿ ಮನಸ್ಸಿನ ಮೇಲೆ ತನ್ನ ಪ್ರಭಾವವನ್ನು ಬೀರುತ್ತಿದೆ. ಈ ಸಂವೇದನೆಗಳೆಲ್ಲ \enginline{(Sensation)} ಶಕ್ತಿಯಲ್ಲದೆ ಬೇರೇನೂ ಅಲ್ಲ. ಹೊರಗಿನ ಶಕ್ತಿ ಒಳಗಿನ ಶಕ್ತಿಯನ್ನು ಕೆರಳಿಸುತ್ತದೆ. ಬಾಹ್ಯ ಸಂವೇದನೆಯನ್ನು ಸ್ವಾಗತಿಸಬೇಕು ಅಥವಾ ತಿರಸ್ಕರಿಸಬೇಕು ಎಂಬ ಈ ಸಂಕಲ್ಪದಿಂದ ಆಲೋಚನೆ ಎನ್ನುವುದು ಉಂಟಾಗುವುದು.

\eject

ಭೌತವಸ್ತು ಮತ್ತು ಮನಸ್ಸು ಎರಡೂ ಶಕ್ತಿತರಂಗಗಳಲ್ಲದೆ ಬೇರೇನೂ ಅಲ್ಲ. ಚೆನ್ನಾಗಿ ವಿಶ್ಲೇಷಣೆ ಮಾಡಿದರೆ ಮೂಲದಲ್ಲಿ ಎರಡೂ ಒಂದೇ ಎಂಬುದು ಗೊತ್ತಾಗುವುದು. ಬಾಹ್ಯಶಕ್ತಿಯು ಒಳಗಿನ ಶಕ್ತಿಯನ್ನು ಹೇಗಾದರೂ ಕೆರಳಿಸುವುದು ಎಂದೊಡನೆ ಅವು ಎಲ್ಲೋ ಸಂಧಿಸುವುವು ಎಂಬುದನ್ನು ಒಪ್ಪಿಕೊಳ್ಳಬೇಕಾಗುವುದು. ಇವೆರಡೂ ಒಂದೇ ಆಗಿರಬೇಕು. ನೀವು ಮೂಲಕ್ಕೆ ಹೋದರೆ ಇವು ಸಾಮಾನ್ಯವಾಗಿ ಸರಳವಾಗುವುವು. ಒಂದೇ ಶಕ್ತಿ ಒಂದು ಕಡೆ ಭೌತವಸ್ತುವಿನಂತೆ ಕಂಡು, ಮತ್ತೊಂದು ಕಡೆ ಮನಸ್ಸಿನಂತೆ ಕಾಣಿಸುವುದರಿಂದ ಭೌತವಸ್ತು ಮತ್ತು ಮನಸ್ಸು ಬೇರೆ ಬೇರೆ ಎಂದು ಹೇಳಲಾಗುವುದಿಲ್ಲ.\break ಮನಸ್ಸು ಭೌತವಸ್ತುವಾಗುವುದು, ಭೌತವಸ್ತುವು ಮನಸ್ಸಾಗುವುದು. ಆಲೋಚನಾಶಕ್ತಿ ನರಗಳ ಶಕ್ತಿಯಾಗುವುದು, ಮಾಂಸದ ಶಕ್ತಿಯಾಗುವುದು. ಮಾಂಸಖಂಡದ ನರಗಳ ಶಕ್ತಿ ಮನಸ್ಸಿನ ಶಕ್ತಿ ಆಗುವುದು. ಪ್ರಕೃತಿ ಎಂದರೆ ಇದೇ ಶಕ್ತಿ. ಅದು ಭೌತವಸ್ತುವಾಗಿ ಅಥವಾ ಮನಸ್ಸಾಗಿ ಅಭಿವ್ಯಕ್ತಿ ಪಡೆಯಬಹುದು.

ಅತಿ ಸೂಕ್ಷ್ಮ ಮನಸ್ಸಿಗೂ ಅತಿ ಸ್ಥೂಲ ಭೌತವಸ್ತುವಿಗೂ ಇರುವ ವ್ಯತ್ಯಾಸ ತರತಮದಲ್ಲಿ ಮಾತ್ರ. ಆದಕಾರಣ ಇಡಿಯ ವಿಶ್ವವನ್ನು ಮನಸ್ಸು ಅಥವಾ ಭೌತವಸ್ತು ಎಂದು ಬೇಕಾದರೆ\break ಕರೆಯಬಹುದು. ಯಾವುದಾದರೂ ಚಿಂತೆಯಿಲ್ಲ. ಮನಸ್ಸನ್ನು ಬೇಕಾದರೆ ಸೂಕ್ಷ್ಮಗೊಂಡ\break ಭೌತವಸ್ತು ಎಂದು ಕರೆಯಬಹುದು; ಅಥವಾ ಭೌತವಸ್ತುವನ್ನು ಘನೀಭೂತವಾದ ಮನಸ್ಸು ಎನ್ನಬಹುದು; ನೀವು ಯಾವುದನ್ನು ಯಾವ ಹೆಸರಿನಿಂದ ಕರೆದರೂ ಏನೂ ವ್ಯತ್ಯಾಸವಾಗು\-ವುದಿಲ್ಲ. ಜಡವಾದಕ್ಕೆ ಮತ್ತು ಆಧ್ಯಾತ್ಮಿಕವಾದಕ್ಕೆ ಇರುವ ಭಿನ್ನಾಭಿಪ್ರಾಯಗಳೆಲ್ಲ ಸರಿಯಾಗಿ ಆಲೋಚಿಸದೆ ಇರುವುದರಿಂದ ಆಗಿವೆ. ನಿಜವಾಗಿ ಇವೆರಡಕ್ಕೂ ಯಾವ ವ್ಯತ್ಯಾಸವೂ ಇಲ್ಲ. ನನಗೂ ಮತ್ತು ಅತಿ ಕೀಳು ಹಂದಿಗೂ ಇರುವ ವ್ಯತ್ಯಾಸ ತರತಮದಲ್ಲಿ ಮಾತ್ರ. ಅದು ಕಡಮೆ ವ್ಯಕ್ತಗೊಳಿಸಿದೆ. ನಾನು ಹೆಚ್ಚು ವ್ಯಕ್ತಗೊಳಿಸುವೆನು. ಕೆಲವು ವೇಳೆ ನನಗಿಂತ ಹಂದಿಯೇ ಮೇಲು.

ಇವೆರಡರಲ್ಲಿ ಯಾವುದು ಮುಂಚೆ ಬಂದಿತು ಎಂಬುದನ್ನು ಚರ್ಚಿಸುವುದರಿಂದಲೂ ಏನೂ ಪ್ರಯೋಜನವಿಲ್ಲ. ಮೊದಲು ಮನಸ್ಸು ಬಂದು ಅನಂತರ ಭೌತವಸ್ತು ಬಂತೆ? ಅಥವಾ ಭೌತವಸ್ತು ಮೊದಲಿದ್ದು ಅನಂತರ ಮನಸ್ಸು ಬಂತೆ? ಹಲವು ತಾತ್ತ್ವಿಕ ವಾದಗಳು ಇಂತಹ ಕೆಲಸಕ್ಕೆ ಬಾರದ ಪ್ರಶ್ನೆಗಳಿಂದ ಹುಟ್ಟುವುವು. ಇದು ಮೊಟ್ಟೆ ಮೊದಲು ಬಂತೆ, ಕೋಳಿ ಮೊದಲು ಬಂತೆ ಎಂದು ಪ್ರಶ್ನೆ ಹಾಕುವಂತೆ ಇದೆ. ಎರಡೂ ಮೊದಲು ಎರಡೂ ಕೊನೆ. ಮೊದಲು ಭೌತವಸ್ತುವಿತ್ತು, ಅದು ಸೂಕ್ಷ್ಮ ಸೂಕ್ಷ್ಮವಾಗುತ್ತಾ ಮನಸ್ಸಾಯಿತು ಎಂದರೆ, ಭೌತವಸ್ತುವಿಗೆ ಮುಂಚೆ ಮನಸ್ಸು ಇದ್ದಿರಬೇಕು ಎಂಬುದನ್ನು ನಾವು ಒಪ್ಪಿಕೊಳ್ಳಬೇಕಾಗುವುದು. ಇಲ್ಲದೇ ಇದ್ದರೆ ಭೌತವಸ್ತು ಎಲ್ಲಿಂದ ಬಂತು? ಭೌತವಸ್ತು ಮನಸ್ಸಿನ ಹಿಂದೆ ಇದೆ, ಮನಸ್ಸು ಭೌತವಸ್ತುವಿನ ಹಿಂದೆ ಇದೆ. ಇದೆಲ್ಲ ಬೀಜವೃಕ್ಷ ನ್ಯಾಯದಂತೆ.

ಪ್ರಕೃತಿಯೆಲ್ಲ ಕಾರ್ಯಕಾರಣ ನಿಯಮಕ್ಕೆ ಬದ್ಧವಾಗಿದೆ, ಅದು ಕಾಲದೇಶಗಳಲ್ಲಿದೆ. ದೇಶದಲ್ಲಿ ಇಲ್ಲದ ಯಾವ ವಸ್ತುವನ್ನೂ ನಾವು ನೋಡಲಾರೆವು. ಆದರೂ ನಾವು ದೇಶವನ್ನು ನೋಡಲಾರೆವು. ನಾವು ಕಾಲಕ್ಕೆ ಅತೀತವಾಗಿ ಯಾವ ವಸ್ತುವನ್ನೂ ನೋಡಲಾರೆವು. ಆದರೂ ಕಾಲವನ್ನು ನೋಡಲಾರೆವು. ಕಾರ್ಯಕಾರಣವೆಂದರೇನು ಎಂಬುದು ನಮಗೆ ಗೊತ್ತಿಲ್ಲ. ಕಾಲ ದೇಶ–ಕಾರ್ಯಕಾರಣಗಳು ಪ್ರತಿಯೊಂದು ಘಟನೆಯಲ್ಲಿಯೂ ಇರುವುವು. ಆದರೆ ಅವೇ ಘಟನೆಗಳಲ್ಲ. ನಾವು ಗ್ರಹಿಸುವುದಕ್ಕೆ ಮುಂಚೆ ಪ್ರತಿಯೊಂದು ವಸ್ತುವೂ ಈ ಅಚ್ಚಿನಲ್ಲಿ ಬೀಳಬೇಕು. ಇಲ್ಲದೆ ಇದ್ದರೆ ಅದನ್ನು ನಮಗೆ ಗ್ರಹಿಸುವುದಕ್ಕೆ ಆಗುವುದಿಲ್ಲ. ಭೌತವಸ್ತು ಎಂದರೆ ವಸ್ತು+ಕಾಲದೇಶನಿಮಿತ್ತ. ಮನಸ್ಸು ಎಂದರೆ ವಸ್ತು+ಕಾಲದೇಶನಿಮಿತ್ತ.

ಇದನ್ನು ನಾವು ಬೇರೊಂದು ವಿಧದಲ್ಲಿ ವಿವರಿಸಬಹುದು. ಪ್ರತಿಯೊಂದೂ ನಾಮರೂಪಗಳಿಂದ ಕೂಡಿದೆ. ನಾಮರೂಪಗಳು ಬಂದು ಹೋಗುವುವು. ಆದರೆ ವಸ್ತು\break ಯಾವಾಗಲೂ ಇರುವುದು. ವಸ್ತು ಮತ್ತು ನಾಮರೂಪ ಇವು ಸೇರಿ ಮಡಕೆಯಾಗುವುದು. ಅದು ಒಡೆದು ಹೋದರೆ ಆಮೇಲೆ ಅದನ್ನು ಮಡಕೆ ಎಂದು ಕರೆಯುವುದಿಲ್ಲ. ಅದರ ನಾಮರೂಪಗಳು ಹೋಗುವುವು. ಆದರೆ ವಸ್ತು ಇರುವುದು. ವಸ್ತುವಿನ ವೈವಿಧ್ಯಕ್ಕೆಲ್ಲಾ ಕಾರಣ ನಾಮರೂಪ. ಇವು ನಿಜವಲ್ಲ. ಏಕೆಂದರೆ ಅವು ಮಾಯವಾಗುವುವು. ನಾವು ಯಾವುದನ್ನು ಪ್ರಕೃತಿ ಎನ್ನುವೆವೋ ಅದು ಅವಿಕಾರಿಯಾದ ಮತ್ತು ಅವಿನಾಶಿಯಾದ ವಸ್ತುವಲ್ಲ. ಕಾರಣದೇಶನಿಮಿತ್ತಗಳೇ ಪ್ರಕೃತಿ. ನಾಮರೂಪಗಳೇ ಪ್ರಕೃತಿ. ಪ್ರಕೃತಿ ನಾಮ ಮತ್ತು ರೂಪ.\break ಪ್ರಕೃತಿ ಮಾಯೆ. ಮಾಯೆ ಎಂದರೆ ನಾಮರೂಪಗಳು. ಪ್ರತಿಯೊಂದೂ ಈ ಎರಕಕ್ಕೆ\break ಬೀಳುವುದು. ಮಾಯೆ ಸತ್ಯವಲ್ಲ. ಅದು ನಿಜವಾಗಿದ್ದರೆ ಅದನ್ನು ನಾವು ನಾಶಮಾಡುವುದಕ್ಕೆ ಆಗುತ್ತಿರಲಿಲ್ಲ, ಬದಲಾಯಿಸುವುದಕ್ಕೆ ಆಗುತ್ತಿರಲಿಲ್ಲ. ದ್ರವ್ಯ ಮೂಲವಸ್ತು; ಮಾಯೆ ಬರಿಯ ದೃಶ್ಯವಸ್ತು. ಯಾವುದರಿಂದಲೂ ನಾಶವಾಗದ ನಿಜವಾದ ನಾನು ಎಂಬುದು ಒಂದು ಇದೆ. ಪ್ರತಿಕ್ಷಣವೂ ಕಂಡು ಮಾಯವಾಗುತ್ತಿರುವುದು ತೋರಿಕೆಯ ನಾನು.

ಅಸ್ತಿತ್ವದಲ್ಲಿರುವ ಪ್ರತಿಯೊಂದಕ್ಕೂ ಎರಡು ಮುಖಗಳಿವೆ. ಒಂದು ಅವಿಕಾರಿಯಾದ ಅವಿನಾಶಿಯಾದ ಮೂಲವಸ್ತು, ಮತ್ತೊಂದು ಬದಲಾಗುತ್ತಿರುವ ತೋರಿಕೆಯದು. ಮಾನ\-ವನ ನೈಜಸ್ಥಿತಿಯೇ ಆ ಮೂಲವಸ್ತು. ಅದೇ ಆತ್ಮ. ಈ ಆತ್ಮವು ಎಂದಿಗೂ ಬದಲಾಗುವುದಿಲ್ಲ; ಎಂದಿಗೂ ನಾಶವಾಗುವುದಿಲ್ಲ. ಅದಕ್ಕೆ ಒಂದು ನಾಮರೂಪದ ಹೊದಿಕೆ ಇರುವಂತೆ ಕಾಣಿಸುವುದು. ಈ ನಾಮರೂಪಗಳು ಅವಿಕಾರಿಗಳಲ್ಲ, ಅವಿನಾಶಿಗಳಲ್ಲ, ಅವು ಯಾವಾಗಲೂ ಬದಲಾಗುತ್ತಿರುವುವು, ನಾಶವಾಗುವುವು. ಆದರೂ ಮಾನವರು ಭ್ರಾಂತರಾಗಿ ಈ ದೇಹದಲ್ಲಿ ಮತ್ತು ಮನಸ್ಸಿನಲ್ಲಿ ಅಮೃತತ್ತ್ವವನ್ನು ಪಡೆಯಲು ಇಚ್ಛಿಸುವರು. ಅವರೊಂದು ನಿತ್ಯ ದೇಹವನ್ನು ಬಯಸುವರು. ನನಗೆ ಇಂತಹ ಅಮೃತತ್ವ ಬೇಕಿಲ್ಲ.

ನನಗೂ ಪ್ರಕೃತಿಗೂ ಏನು ಸಂಬಂಧ? ನಾಮರೂಪ ಅಥವಾ ಕಾಲ ದೇಶ ನಿಮಿತ್ತ–ಇವುಗಳನ್ನೇ ಪ್ರಕೃತಿ ಎಂದು ಪರಿಗಣಿಸಿದಾಗ ನಾನು ಪ್ರಕೃತಿಯ ಭಾಗವಲ್ಲ. ಏಕೆಂದರೆ\break ನಾನು ಮುಕ್ತ, ಅಮೃತ, ಅವಿಕಾರಿ, ಅನಂತ. ನನಗೆ ಇಚ್ಛಾಸ್ವಾತಂತ್ರ್ಯ ಇದೆಯೇ ಇಲ್ಲವೆ ಎನ್ನುವ ಪ್ರಶ್ನೆ ಏಳುವುದಿಲ್ಲ ನಾನು ಇಚ್ಛೆಗೆ ಅತೀತನಾಗಿರುವೆನು. ಇಚ್ಛೆ ಎಲ್ಲಿ ಇದೆಯೋ ಅದೆಂದಿಗೂ ಸ್ವತಂತ್ರವಾಗಿರಲಾರದು. ಇಚ್ಛಾಸ್ವಾತಂತ್ರ್ಯ ಎಂಬುದು ಎಂದಿಗೂ ಇಲ್ಲ. ಯಾವುದು ನಾಮರೂಪಗಳ ಬಲೆಗೆ ಬಿದ್ದು ಇಚ್ಛೆಯ ರೂಪವನ್ನು ತಾಳಿ, ನಾಮರೂಪಗಳಿಗೆ ಅಧೀನವಾಗಿದೆಯೋ ಆ ಆತ್ಮ ಮಾತ್ರ ಸ್ವತಂತ್ರ. ಆತ್ಮ ಮುಂಚೆ ಸ್ವತಂತ್ರವಾಗಿದ್ದು ನಾಮರೂಪದ ಅಚ್ಚಿನಲ್ಲಿ ಬಿದ್ದು ಅನಂತರ ಬದ್ಧವಾಗುವುದು. ಆದರೂ ಅದರ ಮೂಲಸ್ವಭಾವ ಅಲ್ಲಿರುವುದು. ಆದಕಾರಣವೇ ಬಂಧನದಲ್ಲಿದ್ದರೂ ನಾನು ಮುಕ್ತ ಎನ್ನುವುದು ಅದು. ಅದು ಈ ಸ್ವಾತಂತ್ರ್ಯವನ್ನು ಎಂದಿಗೂ ಮರೆಯುವುದಿಲ್ಲ.

ಆತ್ಮವು ಯಾವಾಗ ಇಚ್ಛೆಯಾಗುವುದೋ ಆಗ ಅದು ತನ್ನ ಸ್ವಾತಂತ್ರ್ಯವನ್ನು ಕಳೆದುಕೊಳ್ಳುವುದು. ಪ್ರಕೃತಿಯು ಸೂತ್ರವನ್ನು ಎಳೆದಂತೆ ಅದು ಆಡಬೇಕಾಗುವುದು. ಆದಕಾರಣವೇ ಅನಾದಿಕಾಲದಿಂದಲೂ ಅದು ಎಳೆದಂತೆ ನಾನು ಮತ್ತು ನೀವೆಲ್ಲ ಕುಣಿದಾಡಿರುವೆವು. ನಾವು ನೋಡುವ, ಅನುಭವಿಸುವ, ತಿಳಿಯುವ ವಿಷಯಗಳೆಲ್ಲ, ನಮ್ಮ ಆಲೋಚನೆ ಮತ್ತು ಕಾರ್ಯಗಳೆಲ್ಲ, ಪ್ರಕೃತಿಯ ಆಣತಿಯಂತೆ ನಡೆಯುವುದಾಗಿದೆ. ಇವುಗಳಲ್ಲಿ ಎಲ್ಲಿಯೂ ಸ್ವಾತಂತ್ರ್ಯವಿಲ್ಲ. ಒಂದು ಅಣುವಿನಿಂದ ಮಹತ್ತಿನವರೆಗೆ ಎಲ್ಲ ಚಿಂತನೆಗಳೂ ಕ್ರಿಯೆಗಳೂ ನಿಯಮಕ್ಕೆ ಅಧೀನ. ಇವಾವುವೂ ನಮ್ಮ ನಿಜವಾದ ಆತ್ಮವಲ್ಲ.

ನನ್ನ ನಿಜವಾದ ಆತ್ಮವು ಎಲ್ಲಾ ನಿಯಮಗಳಿಗೂ ಅತೀತವಾಗಿದೆ. ಪ್ರಕೃತಿಯ\break ಬಂಧನದಲ್ಲೇ ಸುಖವಾಗಿರುವಿರಿ. ಆದರೆ ಪ್ರಕೃತಿ ಹೇಳಿದಂತೆ ಕೇಳಿದಂತೆಲ್ಲ ನೀವು\break ಹೆಚ್ಚು ಹೆಚ್ಚು ಬದ್ಧರಾಗುವಿರಿ. ನೀವು ಅಜ್ಞಾನಕ್ಕೆ ಹೊಂದಿಕೊಂಡಿದ್ದಷ್ಟೂ ಪ್ರಕೃತಿಯ\break ಪ್ರತಿಯೊಂದು ಆಣತಿಯನ್ನೂ ಪರಿಪಾಲಿಸಬೇಕಾಗಿದೆ. ನಿಯಮಕ್ಕೆ ಅಧೀನವಾಗಿರುವುದು, ಪ್ರಕೃತಿಯೊಂದಿಗೆ ಸೌಹಾರ್ದದಿಂದ ಇರುವುದು, ಮಾನವನ ನೈಜಸ್ವಭಾವಕ್ಕೆ ತಕ್ಕುದೆ? ಲೋಹವು ಎಂದು ನಿಯಮದೊಂದಿಗೆ ಜಗಳವಾಡಿತು? ಅದು ಎಂದು ನಿಯಮವನ್ನು ವಿರೋಧಿಸಿ ಹೋಯಿತು? ಈ ಮೇಜು ಪ್ರಕೃತಿಯೊಂದಿಗೆ ಸೌಹಾರ್ದದಿಂದಿದೆ; ನಿಯಮವನ್ನು ಪರಿಪಾಲಿಸುತ್ತಿದೆ. ಆದರೆ ಮೇಜು ಮೇಜಾಗಿಯೇ ಉಳಿಯುವುದು, ಅದೆಂದಿಗೂ ಅದಕ್ಕಿಂತ ಮೇಲಾಗುವುದಿಲ್ಲ. ಮಾನವ ಹೋರಾಡುವನು, ಪ್ರಕೃತಿಗೆ ವಿರೋಧವಾಗಿ ಹೋರಾಡುವನು. ಅವನು ಎಷ್ಟೋ ತಪ್ಪುಗಳನ್ನು ಮಾಡಿ ವ್ಯಥೆಪಡುವನು, ಆದರೆ ಕೊನೆಗೆ ಪ್ರಕೃತಿಯನ್ನು ಗೆದ್ದು ತನ್ನ ಸ್ವತಂತ್ರ ಸ್ವಭಾವವನ್ನು ಅರಿಯುವನು. ಅವನು ಮುಕ್ತನಾದ ಮೇಲೆ ಪ್ರಕೃತಿ ಅವನ ಗುಲಾಮನಾಗುವುದು.

ತಾನು ಬಂಧನದಲ್ಲಿರುವೆ ಎಂಬ ಅರಿವು ಉಂಟಾಗಿ ಅದಕ್ಕೆ ವಿರುದ್ಧವಾಗಿ ಹೋರಾಡುವುದನ್ನೇ ಜೀವನ ಎನ್ನುವುದು. ಈ ಹೋರಾಟದ ಜಯವನ್ನೇ ವಿಕಾಸ \enginline{(Evolution)} ಎನ್ನುವುದು. ಕೊನೆಗೆ ಬಂಧನಗಳೆಲ್ಲ ಮಾಯವಾದ ಮೇಲೆ ಇದನ್ನೇ ಮುಕ್ತಿ, ನಿರ್ವಾಣ,\break ಸ್ವಾತಂತ್ರ್ಯ ಎನ್ನುವುದು. ಜಗತ್ತಿನಲ್ಲಿ ಎಲ್ಲವೂ ಸ್ವಾತಂತ್ರ್ಯಕ್ಕಾಗಿ ಹೋರಾಡುತ್ತಿರುವುದು.\break ನಾನು ಪ್ರಕೃತಿಗೆ ದಾಸನಾದಾಗ, ನಾಮರೂಪಗಳ ಬಲೆಯಲ್ಲಿ ಇರುವಾಗ, ಕಾಲದೇಶನಿಮಿತ್ತಗಳ ಜಾಲದಲ್ಲಿ ಸಿಕ್ಕಿರುವಾಗ, ನನ್ನ ನೈಜಸ್ಥಿತಿ ನನಗೆ ಅರಿವಾಗದು. ಆದರೆ ಈ ಬಂಧನದ ಸ್ಥಿತಿಯಲ್ಲಿಯೂ ನನ್ನ ನೈಜಸ್ವಭಾವ ಸಂಪೂರ್ಣವಾಗಿ ನಾಶವಾಗುವುದಿಲ್ಲ. ನಾನು ಬಂಧನಗಳನ್ನು ಹರಿದೊಗೆಯಲು ಹೋರಾಡುವೆನು. ಒಂದೊಂದಾಗಿ ಅವು\break ಕತ್ತರಿಸಲ್ಪಟ್ಟು ಬೀಳುವುವು. ಆಗ ನನ್ನ ನೈಜಸ್ಥಿತಿ ಗೊತ್ತಾಗುತ್ತಾ ಬರುವುದು. ಕೊನೆಗೆ ಪೂರ್ಣ ಮುಕ್ತಿ ಬರುವುದು. ಆಗ ನನ್ನ ನೈಜಸ್ಥಿತಿ ಸ್ಪಷ್ಟವಾಗಿ ಪೂರ್ಣವಾಗಿ ವ್ಯಕ್ತವಾಗುವುದು. ನಾನು ಅನಂತಾತ್ಮ, ಪ್ರಕೃತಿಯ ಸ್ವಾಮಿ, ಅದರ ಗುಲಾಮನಲ್ಲ ಎಂದು ಆಗ ನನಗೆ ಗೊತ್ತಾಗುವುದು. ಎಲ್ಲಾ ಬಗೆಯ ವೈವಿಧ್ಯದ ಆಚೆ, ಮಿಶ್ರಣದ ಆಚೆ, ಕಾಲ ದೇಶ ನಿಮಿತ್ತಾತೀತನಾಗಿ ನಾನು ನನ್ನ ನೈಜಸ್ಥಿತಿಯಲ್ಲಿ ನೆಲೆಸಿರುವೆನು.

