
\chapter[ಧರ್ಮ ನಾಗರೀಕತೆ ಮತ್ತು ಅದ್ಭುತಗಳು ]{ಧರ್ಮ, ನಾಗರೀಕತೆ ಮತ್ತು ಅದ್ಭುತಗಳು \protect\footnote{\engfoot{C.W. Vol. VII, p. 282}}}

\centerline{\textbf{(‘ದಿ ಅಪೀಲ್​ ಅವಲಾಂಚ್​’ ಜನವರಿ 21, 1894)}}

\vskip 3pt

ಸ್ವಾಮೀಜಿ ಅವರು ಲಾ ಸೆಲಿಟಿ ಅಕಾಡೆಮಿಯ ನಡುಮನೆಯಲ್ಲಿ ಕುಳಿತ್ತಿದ್ದರು. ಅವರು ಮೆಂಫಿಸ್​ನಲ್ಲಿದ್ದಾಗ ಸಾಧಾರಣವಾಗಿ ಇಲ್ಲಿ ತಂಗುತ್ತಿದ್ದರು. ಆ ಸಮಯದಲ್ಲಿ ಮಾತನಾಡುತ್ತಿದ್ದಾಗ ಹೀಗೆಂದರು: “ನಾನು ಸಂನ್ಯಾಸಿ, ಪುರೋಹಿತನಲ್ಲ. ನಾನು ನನ್ನ ದೇಶದಲ್ಲಿರುವಾಗ ಹಳ್ಳಿಯಿಂದ ಹಳ್ಳಿಗೆ ಪಟ್ಟಣದಿಂದ ಪಟ್ಟಣಕ್ಕೆ ಬೋಧಿಸುತ್ತ ಹೋಗುವೆ. ನಾನು ಹಣವನ್ನು ಸಂಪಾದನೆ ಮಾಡಕೂಡದು. ಆದಕಾರಣ ನಾನು ನನ್ನ ಜೀವನೋಪಾಯಕ್ಕೆ ಜನರನ್ನು\break ನೆಚ್ಚಿಕೊಂಡಿರುವೆನು.”

\vskip 3pt

ಕೇಳಿದ ಒಂದು ಪ್ರಶ್ನೆಗೆ ಉತ್ತರವಾಗಿ ಅವರು ಹೇಳಿದರು: “ನಾನು ಬಂಗಾಳ ದೇಶದಲ್ಲಿ ಹುಟ್ಟಿದೆ. ನಾನೇ ಮನಸ್ಸು ಮಾಡಿ ಸಂನ್ಯಾಸಿಯಾದೆ. ಬ್ರಹ್ಮಚರ್ಯ ವ್ರತವನ್ನು ಸ್ವೀಕರಿಸಿದೆ. ನಾನು ಹುಟ್ಟಿದ ಸಮಯದಲ್ಲಿ ನನ್ನ ತಂದೆ ಒಂದು ಜಾತಕವನ್ನು ಬರೆಸಿದರು. ಆದರೆ ನನಗೆ ಅದನ್ನು ತೋರಿಸಿರಲಿಲ್ಲ. ಅನಂತರ ಕೆಲವು ವರುಷಗಳಾದ ಮೇಲೆ ನಾನು ಮನೆಗೆ ಹಿಂತಿರುಗಿದಾಗ ನನ್ನ ತಂದೆ ತೀರಿಹೋಗಿದ್ದರು. ನಮ್ಮ ತಾಯಿಯ ಹತ್ತಿರ ಇದ್ದ ಕಾಗದ ಪತ್ರಗಳ ಜೊತೆಯಲ್ಲಿ ನನ್ನ ಜಾತಕವೂ ಕೂಡ ಇದ್ದುದನ್ನು ನೋಡಿದೆ. ಅಲ್ಲಿ ನಾನು ಪರಿವ್ರಾಜಕ ಸಂನ್ಯಾಸಿಯಾಗುವೆನು ಎಂದು ಇದ್ದುದ್ದನ್ನು ಕಂಡೆನು.”

\vskip 3pt

ಮಾತನಾಡುವಾಗ ಅವರ ಧ್ವನಿ ಕರುಣಾಮಯವಾಗಿತ್ತು. ಅದನ್ನು ಕೇಳುತ್ತಿದ್ದವರ ಹೃದಯದಿಂದ ಒಂದು ಸಹಾನುಭೂತಿಯ ತರಂಗ ಹೊರಹೊಮ್ಮಿತು. ವಿವೇಕಾನಂದರು ತಮ್ಮ ಚುಟ್ಟದಿಂದ ಹೊಗೆಸೊಪ್ಪಿನ ಬೂದಿಯನ್ನು ಚೆಲ್ಲಿ ಸ್ವಲ್ಪ ಕಾಲ ಮೌನವಾಗಿದ್ದರು.

\vskip 3pt

“ನಿಮ್ಮ ಧರ್ಮ ನೀವು ಹೇಳುವಷ್ಟೇ ಆಗಿದ್ದರೆ, ಅದೊಂದೆ ಸತ್ಯವಾದ ಧರ್ಮ ಆಗಿದ್ದರೆ, ನಿಮ್ಮ ಜನರೇಕೆ ನಾಗರಿಕತೆಯಲ್ಲಿ ಈಗಿರುವ ಸ್ಥಿತಿಗಿಂತ ಹೆಚ್ಚು ಮುಂದುವರಿದಿಲ್ಲ, ಪ್ರಪಂಚದ ರಾಷ್ಟ್ರಗಳ ಮಧ್ಯದಲ್ಲಿ ಅದು ಒಂದು ಗೌರವ ಸ್ಥಾನವನ್ನು ಏತಕ್ಕೆ ಪಡೆದಿಲ್ಲ?

\vskip 3pt

ಹಿಂದೂ ಸಂನ್ಯಾಸಿ ಗಂಭೀರವಾಗಿ ಉತ್ತರ ಕೊಟ್ಟರು: “ಏಕೆಂದರೆ ಅದು ಧರ್ಮದ ಕ್ಷೇತ್ರವಲ್ಲ. ನಮ್ಮ ಜನ ಪ್ರಪಂಚದಲ್ಲೆಲ್ಲ ಶ್ರೇಷ್ಠ ನೀತಿಬದ್ಧರು. ಅಥವಾ ಇದರಲ್ಲಿ ಅವರು ಅತಿ ಮುಂದುವರಿದ ರಾಷ್ಟ್ರಕ್ಕೇ ಸರಿಸಮನಾಗಿ ನಿಲ್ಲುವರು. ಅವರು ತಮ್ಮ ಮಾನವ ಸಹೋದರರ ಹಕ್ಕುಬಾಧ್ಯತೆಗಳ ವಿಚಾರದಲ್ಲಿ ತುಂಬಾ ದಾಕ್ಷಿಣ್ಯಪರರು. ಮೂಕಪ್ರಾಣಿಗಳ ವಿಷಯದಲ್ಲಿ ಕೂಡ ಅವರು ಮರುಕವನ್ನು ತೋರುವರು. ಆದರೆ ಅವರು ಜಡವಾದಿಗಳಲ್ಲ. ಪ್ರಪಂಚದಲ್ಲಿ ಯಾವ ಧರ್ಮವೂ ಜನರ ಅಥವಾ ರಾಷ್ಟ್ರದ ಚಿಂತನೆಯನ್ನಾಗಲಿ, ಸ್ಫೂರ್ತಿಯನ್ನಾಗಲಿ ಪ್ರಗತಿಗೊಳಿಸಲಿಲ್ಲ. ನಿಜವಾಗಿ ಪ್ರಪಂಚದಲ್ಲಿ ಯಾವ ದೊಡ್ಡದನ್ನು\break ಸಾಧಿಸಿದ್ದರೂ ಧರ್ಮ ಅದಕ್ಕೆಲ್ಲ ಆತಂಕಪ್ರಾಯವಾಗಿ ಪರಿಣಮಿಸಿದೆ. ನೀವು ಹೆಮ್ಮೆ\break ಕೊಚ್ಚಿಕೊಳ್ಳುವ ಕ್ರೈಸ್ತಧರ್ಮ ಕೂಡ ಇದಕ್ಕೆ ಅಪವಾದ ಅಲ್ಲ. ನಿಮ್ಮ ಡಾರ್ವಿನ್ನರು, ಮಿಲ್ಲರು, ಹ್ಯೂಮ್​ ಮುಂತಾದವರಿಗೆ ನಿಮ್ಮ ಪಾದ್ರಿಗಳು ತಮ್ಮ ಸಮ್ಮತಿಯನ್ನು ತೋರಿಸಿಲ್ಲ. ಇದಕ್ಕಾಗಿ ನೀವು ನಮ್ಮ ಧರ್ಮದಲ್ಲಿ ತಪ್ಪನ್ನು ಏತಕ್ಕೆ ಕಂಡುಹಿಡಿಯುವುದು?

\vskip 3pt

ಈ ಪಂಗಡದಲ್ಲಿ ಒಬ್ಬರು ಹೀಗೆಂದರು: “ಯಾವುದು ಮಾನವನ ಲೌಕಿಕ ಸ್ಥಿತಿಯನ್ನು ಉತ್ತಮಪಡಿಸಲಾರದೊ ಅಂತಹ ಧರ್ಮವನ್ನು ನಾವು ನಿಕೃಷ್ಟವಾಗಿ ಭಾವಿಸುವೆವು. ಅದೂ ಅಲ್ಲದೆ ನೀವು ಈಗತಾನೆ ಹೇಳಿದ ಮಾತನ್ನು ನಾವು ಸರಿ ಎಂದು ಒಪ್ಪುವುದಿಲ್ಲ. ಕ್ರೈಸ್ತಧರ್ಮ ಕಾಲೇಜುಗಳನ್ನು ಸ್ಥಾಪಿಸಿದೆ, ಔಷಧಾಲಯಗಳನ್ನು ತೆರೆದಿದೆ, ಅಧೋಗತಿಯಲ್ಲಿದ್ದವರನ್ನು ಮೇಲಕ್ಕೆ ಎತ್ತಿದೆ. ತುಂಬಾ ಕೀಳು ವರ್ಗದಲ್ಲಿದ್ದವರನ್ನು ಮೇಲೆತ್ತಿ, ಅವರ ಜೀವನೋಪಾ\-ಯಕ್ಕೆ ಸಹಾಯಮಾಡಿದೆ.”

\vskip 3pt

ಸಂನ್ಯಾಸಿಗಳು ಪ್ರಶಾಂತ ರೀತಿಯಿಂದ ಉತ್ತರ ಕೊಟ್ಟರು: “ನೀವು ಹೇಳುವ ಮಾತಿನಲ್ಲಿ ಸ್ವಲ್ಪ ಸತ್ಯವಿದೆ. ನೀವು ಹೇಳುವ ಉಪಕಾರವೆಲ್ಲ ಪ್ರತ್ಯಕ್ಷವಾಗಿ ಕ್ರೈಸ್ತ ಧರ್ಮದಿಂದ ಆದದ್ದು ಎಂಬುದನ್ನು ತೋರಿಲ್ಲ. ಈ ಪರಿಣಾಮವನ್ನು ಉಂಟು ಮಾಡಲು ಪಾಶ್ಚಾತ್ಯ ದೇಶಗಳಲ್ಲಿ ಹಲವು ಕಾರಣಗಳು ಕೆಲಸ ಮಾಡುತ್ತಿವೆ.”

\vskip 3pt

“ಧಾರ್ಮಿಕ ಭಾವನೆ ಮನುಷ್ಯನ ಆಧ್ಯಾತ್ಮಿಕ ಜೀವನಕ್ಕೆ ಸಹಾಯ ಮಾಡಬೇಕು. ವಿಜ್ಞಾನ, ಕಲೆ, ಪಾಂಡಿತ್ಯ, ತಾತ್ತ್ವಿಕ ಸಂಶೋಧನೆಗಳು ಇವುಗಳಿಗೆಲ್ಲ ಜೀವನದಲ್ಲಿ\break ನಿರ್ದಿಷ್ಟವಾದ ಸ್ಥಾನವಿದೆ. ಆದರೆ ನೀವು ಅವುಗಳನ್ನೆಲ್ಲ ಒಟ್ಟಾಗಿ ಬೆರೆಸಲು ಯತ್ನಿಸಿದರೆ, ಅವುಗಳ ವೈಶಿಷ್ಟ್ಯಕ್ಕೆ ಭಂಗತರುವಿರಿ. ಕೊನೆಗೆ ಧರ್ಮದಿಂದ ನೀವು ಆಧ್ಯಾತ್ಮಿಕತೆಯನ್ನೇ ಬಹಿಷ್ಕರಿಸುವಿರಿ. ಅಮೆರಿಕ ದೇಶೀಯರಾದ ನೀವು ಏನನ್ನು ಪೂಜಿಸುತ್ತೀರಿ? ಡಾಲರನ್ನು. ಅದನ್ನು ಸಂಗ್ರಹಿಸುವ ಭರದಲ್ಲಿ ನೀವು ನಿಮ್ಮ ಆಧ್ಯಾತ್ಮಿಕತೆಯನ್ನೇ ಮರೆಯುತ್ತೀರಿ. ನಿಮ್ಮ ದೇಶ ಒಂದು ಜಡವಾದಿಗಳ ದೇಶವಾಗುತ್ತದೆ. ನಿಮ್ಮ ಬೋಧಕರು ಮತ್ತು ನಿಮ್ಮ ಚರ್ಚುಗಳು ಕೂಡ ಪ್ರಪಂಚದ ಆಸೆಗಳಿಂದ ಕಲುಷಿತವಾಗಿವೆ. ನಮ್ಮ ದೇಶದಲ್ಲಿ ತೋರಿ. ಮೃತ್ಯು ಸಮಯದಲ್ಲಿಯೂ “ಸಹೋದರ ಮೃತ್ಯುವೇ, ನಾನು ನಿನ್ನನ್ನು ಬರಮಾಡಿಕೊಳ್ಳುವೆನು” ಎಂದು ಹೇಳುವಂತಹವರು ಎಲ್ಲಿರುವರು? ನಿಮ್ಮ ಧರ್ಮ ಈಫೆಲ್​ ಗೋಪುರಗಳು, ಮತ್ತು ಪೆರ್ರಿ ಚಕ್ರಗಳನ್ನು ಕಟ್ಟುವುದಕ್ಕೆ ಪ್ರೋತ್ಸಾಹಿಸುವುದು. ಆದರೆ ಅದು ನಿಮ್ಮ ಆಂತರಿಕ ಜೀವನವನ್ನು ವ್ಯವಸ್ಥೆಗೊಳಿಸಲು ಸಹಾಯ ಮಾಡುವುದೆ?”

\vskip 3pt

ಸ್ವಾಮೀಜಿ ಅವರು ಬಹಳ ಉತ್ಸಾಹದಿಂದ ಮಾತನಾಡಿದರು. ಅವರ ಧ್ವನಿ ಸುಂದರವಾಗಿತ್ತು, ಸ್ಪಷ್ಟವಾಗಿತ್ತು. ನಡುಮನೆಯ ಮೇಲೆ ಕವಿದಿದ್ದ ಸಂಜೆಯ ವಾತಾವರಣದಲ್ಲಿ, ದುಃಖಪೂರಿತವೂ ಭರ್ತ್ಸನಯುಕ್ತವೂ ಆದ ಅವರ ಮಾತು ಅರ್ಥಗರ್ಭಿತವಾಗಿತ್ತು.\break ಅಮೇರಿಕಾ ದೇಶದ ಹತ್ತೊಂಭತ್ತನೇ ಶತಮಾನದ ನಾಗರೀಕತೆ ಆರು ಸಾವಿರ ವರುಷಗಳ ಮುಂಚೆಯೇ ಇದ್ದ ದೇಶದಿಂದ ಬಂದ ಆ ಮನುಷ್ಯನ ಟೀಕೆಗಳಲ್ಲಿ ಒಂದು ಅಲೌಕಿಕತೆ ಇತ್ತು.

\vskip 3pt

ಯಾರೋ ಒಬ್ಬರು ಹೀಗೆ ಕೇಳಿದರು: “ನೀವು ಆಧ್ಯಾತ್ಮಿಕತೆಯನ್ನು ಅನುಸರಿಸಿಕೊಂಡು ಹೋದರೆ ತಾತ್ಕಾಲಿಕ ಸ್ಥಿತಿಯನ್ನು ಗಮನಿಸುವುದೇ ಇಲ್ಲ. ನಿಮ್ಮ ಸಿದ್ಧಾಂತ ಜನರಿಗೆ ಸಹಾಯ ಮಾಡುವುದಿಲ್ಲ.”

\vskip 3pt

“ಅದು ಅವರನ್ನು ಸಾಯುವುದಕ್ಕೆ ಸಿದ್ಧರನ್ನಾಗಿ ಮಾಡುವುದು” ಇದೇ ಸ್ವಾಮೀಜಿಯ ಉತ್ತರವಾಗಿತ್ತು.

\vskip 3pt

ಪ್ರಶ್ನೆ: “ನಮಗೆ ಈಗಿನ ಸ್ಥಿತಿಯಲ್ಲಿ ದೃಢ ನಿಶ್ಚಯವಿದೆ.”

\vskip 3pt

ಉತ್ತರ: “ನಿಮಗೆ ಯಾವುದರಲ್ಲಿಯೂ ದೃಢ ನಿಶ್ಚಯವಿಲ್ಲ.”

\vskip 3pt

ಪ್ರಶ್ನೆ: “ಆದರ್ಶ ಧರ್ಮವು ನಾವು ಈ ಪ್ರಪಂಚದಲ್ಲಿ ಹೇಗೆ ಬಾಳಬೇಕು ಎಂಬುದಕ್ಕೆ ಸಹಾಯ ಮಾಡಬೇಕು ಮತ್ತು ಏಕಕಾಲದಲ್ಲಿಯೇ ಮೃತ್ಯುವನ್ನು ಎದುರಿಸುವುದಕ್ಕೂ\break ಅವನನ್ನು ಅಣಿ ಮಾಡಬೇಕು.”

\vskip 3pt

ಹಿಂದೂ ಸಂನ್ಯಾಸಿ ತಕ್ಷಣವೇ ಹೀಗೆ ಹೇಳಿದರು: “ನೀವು ಹೇಳುವುದು ಸರಿ. ನಾವು ಪಡೆಯಲು ಪ್ರಯತ್ನಿಸುತ್ತಿರುವುದು ಅದನ್ನೇ. ಹಿಂದೂವಿನಲ್ಲಿರುವ ಶ್ರದ್ಧೆ ಲೌಕಿಕವನ್ನು ಕಡೆಗಣಿಸಿ ಆಧ್ಯಾತ್ಮಿಕತೆಯ ಕಡೆ ಹೆಚ್ಚು ಗಮನ ಕೊಟ್ಟಿದೆ ಎಂದು ಭಾವಿಸುತ್ತೇನೆ. ಪಾಶ್ಚಾತ್ಯ ದೇಶಗಳಲ್ಲಿಯಾದರೂ ಅದಕ್ಕೆ ವಿರೋಧ. ಪಾಶ್ಚಾತ್ಯರ ಭೌತಿಕತೆ ಜೊತೆಗೆ ಪೌರಸ್ತ್ಯರ ಆಧ್ಯಾತ್ಮಿಕತೆಯನ್ನು ಬೆರೆಸಿದರೆ ನಾವು ಇಂದಿಗಿಂತ ಹೆಚ್ಚನ್ನು ಸಾಧಿಸಬಹುದು. ಈ ಪ್ರಯತ್ನದಲ್ಲಿ ಹಿಂದೂವಿನ ವೈಶಿಷ್ಟ್ಯಕ್ಕೆ ಬಹಳ ಲೋಪ ಬರಬಹುದು.”

\vskip 3pt

“ನೀವು ಏನನ್ನು ಸಾಧಿಸಬೇಕೆಂದಿರುವಿರೋ ಅದನ್ನು ಮಾಡಬೇಕಾದರೆ ಸಮಾಜದ\break ಸ್ಥಿತಿಯನ್ನೇ ಬದಲಾಯಿಸಬೇಕಲ್ಲವೆ?”

\vskip 3pt

“ಬಹುಶಃ ಅದು ನಿಜವಿರಬಹುದು. ಆದರೂ ಧರ್ಮಕ್ಕೆ ಯಾವ ಚ್ಯುತಿಯೂ ಬರುವುದಿಲ್ಲ.”

\vskip 3pt

ಇಲ್ಲಿ ಸಂಭಾಷಣೆ ಹಿಂದೂಗಳ ಉಪಾಸನೆಯ ಕಡೆಗೆ ತಿರುಗಿತು. ವಿವೇಕಾನಂದರು ಇಲ್ಲಿ ಈ ವಿಷಯದ ಮೇಲೆ ಸ್ವಾರಸ್ಯವಾದ ಭಾವಗಳನ್ನು ವ್ಯಕ್ತಪಡಿಸಿದರು. ಭರತಖಂಡದಲ್ಲಿ\break ಎಲ್ಲಾ ಕಡೆಗಳಲ್ಲಿ ಇರುವಂತೆ ನಾಸ್ತಿಕರು ಮತ್ತು ಅಜ್ಞೇಯತಾವಾದಿಗಳು ಇರುವರು.\break ಬ್ರಹ್ಮತತ್ತ್ವದ ಉಪಾಸಕರಿಗೆ ಮುಖ್ಯವಾಗಿರುವುದು ಸಾಕ್ಷಾತ್ಕಾರ. ನಂಬಿಕೆ ಮುಖ್ಯವಲ್ಲ. ವಿವೇಕಾನಂದರಿಗೆ ಥಿಯಾಸಫಿ ಗೊತ್ತಿಲ್ಲ. ಇದು ಅವರ ಧರ್ಮಕ್ಕೆ ಸೇರಿದ್ದೂ ಅಲ್ಲ. ಅವರು ಇಚ್ಛಿಸಿದರೆ ಬೇಕಾದರೆ ಇದಕ್ಕೆ ಸೇರಬಹುದು. ಅದು ಬೇರೊಂದು ಬಗೆಯ ಅಧ್ಯಯನ. ವಿವೇಕಾನಂದರು ಮ್ಯಾಡಮ್​ ಬ್ಲಾವೆಟ್​ಸ್ಕಿ ಅವರನ್ನು ಕಂಡಿರಲಿಲ್ಲ. ಅಮೆರಿಕನ್​ ಥಿಯಸಾಫಿಕಲ್​ ಸೊಸೈಟಿಯ ಕರ್ನಲ್​ ಆಲ್​ಕಾಟ್​ರನ್ನು ಕಂಡಿದ್ದರು. ಅವರಿಗೆ ಅನಿಬೆಸೆಂಟರ ಪರಿಚಯ ಕೂಡ ಇದೆ. ಪ್ರಪಂಚದಲ್ಲೆಲ್ಲ ಪ್ರಖ್ಯಾತರಾದ ಫಕೀರರು, ಮಾಯಾ ಮಂತ್ರವಾದಿಗಳು ಇವರನ್ನು ಕುರಿತು ತಾವೇ ನೋಡಿದ ಕೆಲವು ನಿದರ್ಶನಗಳನ್ನು ಕೊಟ್ಟರು. ಇದು ನಂಬುವುದಕ್ಕೂ ಆಗದಷ್ಟು ಸೋಜಿಗವಾಗಿರುವುದು.

\vskip 3pt

ಈ ವಿಷಯದ ಮೇಲೆ ಪ್ರಶ್ನಿಸಿದಾಗ ಅವರು ಹೀಗೆ ಹೇಳಿದರು: “ಐದು ತಿಂಗಳ ಹಿಂದೆ, ಅಥವಾ ನಾನು ಈ ದೇಶಕ್ಕೆ ಬರುವುದಕ್ಕೆ ಒಂದು ತಿಂಗಳು ಮುಂಚಿತವಾಗಿ, ಇಪ್ಪತ್ತೈದು ಜನರ ಒಂದು ಗುಂಪಿನಲ್ಲಿದ್ದೆ. ದೇಶದ ಒಳಭಾಗದಲ್ಲಿರುವ ಒಂದು ಊರನ್ನು ನೋಡಬೇಕೆಂದು ಹೋಗುತ್ತಿದ್ದೆ. ಅಲ್ಲಿ ಒಬ್ಬ ಪರಿವ್ರಾಜಕನಾಗಿ ಅಲೆಯುತ್ತಿದ್ದ ಮಂತ್ರವಾದಿಯ ವಿಷಯ ತಿಳಿಯಿತು. ಅವನು ಚಮತ್ಕಾರವಾದ ಕೆಲಸಗಳನ್ನು ಮಾಡಬಲ್ಲನು ಎಂಬುದು ಗೊತ್ತಾಯಿತು. ಅವನನ್ನು ನಮ್ಮ ಬಳಿಗೆ ಕರೆತಂದೆವು. ನೀವು ಕೇಳಿದ ಸಾಮಾನುಗಳನ್ನು ತಾನು ತಯಾರು ಮಾಡಬಲ್ಲೆ ಎಂದು ಅವನು ಹೇಳಿದನು. ಅವನ ಇಷ್ಟದಂತೆ ಅವನ ಮೇಲೆ ಇರುವ ಬಟ್ಟೆಗಳನ್ನೆಲ್ಲ ತೆಗೆದುಹಾಕಿದೆವು. ಅವನ ಮೈ ಮೇಲೆ ಒಂದು ತುಂಡು ಬಟ್ಟೆಯನ್ನು ಮಾತ್ರ ಬಿಟ್ಟೆವು. ಅವನನ್ನು ಒಂದು ಕೋಣೆಯ ಮೂಲೆಯ ಸಮೀಪದಲ್ಲಿ ಬಿಟ್ಟೆವು. ನಾನು ಅವನ ಮೇಲೆ ನನ್ನ ಹತ್ತಿರವಿದ್ದ ಪ್ರಯಾಣದ ಕಂಬಳಿಯನ್ನು ಎಸೆದೆನು. ನಾವು ಹೇಳಿದಂತೆ ಮಾಡಿ ತೋರಿಸು ಎಂದು ಕೇಳಿದೆವು. ನಿಮಗೆ ಏನು ಬೇಕು ಎಂದು ಅವನು ನಮ್ಮನ್ನು ಕೇಳಿದನು. ನಾನು ಅವನನ್ನು ಕ್ಯಾಲಿಫೋರ್ನಿಯಾ ದ್ರಾಕ್ಷಿಯ ಹಣ್ಣಿನ ಗೊಂಚಲನ್ನು ಕೇಳಿದೆನು. ಅವನು ತಕ್ಷಣವೇ ನಾನು ಕೊಟ್ಟ ಕಂಬಳಿಯ ಒಳಗಿನಿಂದ ಅದನ್ನು ಹೊರಗೆ ತೆಗೆದನು. ಕಿತ್ತಳೆ ಮುಂತಾದ ಹಣ್ಣುಗಳನ್ನು ತೆಗೆದು ತೋರಿಸಿದನು. ಕೊನೆಗೆ ಬಿಸಿಬಿಸಿಯಾಗಿ ಹೊಗೆ ಬರುತ್ತಿದ್ದ ಅನ್ನದ ತಟ್ಟೆಯನ್ನು ತೆಗೆದನು”

\vskip 3pt

ಸ್ವಾಮಿಗಳು ಮಾತನ್ನು ಮುಂದುವರಿಸುತ್ತ ತಾವು ಆರನೆ ಇಂದ್ರಿಯದ ಬಗ್ಗೆ ನಂಬುತ್ತೇನೆ ಎಂದರು. ಟೆಲಿಪತಿ (ದೂರದಲ್ಲಿ ಆಗುವುದನ್ನು ನೋಡುವುದು) ಇದನ್ನೂ ನಂಬುತ್ತೇನೆ ಎಂದರು. ಫಕೀರರು ಮಾಡುವ ಮಾಯಾಮಂತ್ರಗಳ ವಿಷಯದಲ್ಲಿ ಅವರು ಯಾವ ವಿವರಣೆಯನ್ನೂ ಕೊಡಲಿಲ್ಲ. ಅವು ಅತ್ಯಂತ ಸೋಜಿಗ ಎಂದು ಮಾತ್ರ ಹೇಳಿದರು. ಅನಂತರ ವಿಗ್ರಹದ ಮಾತು ಬಂತು. ವಿಗ್ರಹಗಳು ತಮ್ಮ ಧರ್ಮದಲ್ಲಿ ಒಂದು ಭಾಗ. ಆದರೆ ಇವು ಕೇವಲ ಚಿಹ್ನೆಗಳು ಎಂದರು.

\vskip 3pt

“ನೀವು ಏನನ್ನು ಪೂಜಿಸುತ್ತೀರಿ? ದೇವರೆಂದರೆ ನಿಮ್ಮ ಭಾವನೆ ಏನು?” ಎಂದು ಸ್ವಾಮೀಜಿ ಕೇಳಿದರು.

\vskip 3pt

“ಸ್ಪಿರಿಟ್​” ಎಂದಳು ಒಬ್ಬ ಹೆಂಗಸು.

\eject

“ಸ್ಪಿರಿಟ್​ ಎಂದರೆ ಏನು? ಪ್ರಾಟೆಸ್​ಟೆಂಟರಾದ ನೀವು ಬೈಬಲ್ಲಿನ ಅಕ್ಷರಗಳನ್ನು ಪೂಜಿಸುತ್ತೀರೋ, ಅಥವಾ ಅದಕ್ಕಿಂತ ಅತೀತವಾಗಿರುವುದನ್ನೇನಾದರೂ ಪೂಜಿಸುತ್ತಿರೋ? ನಾವು ವಿಗ್ರಹಗಳ ಮೂಲಕ ದೇವರನ್ನು ಪೂಜಿಸುತ್ತೇವೆ”

\vskip 3pt

ಅಪರಿಚಿತನ ಮಾತುಗಳನ್ನು ಗಮನದಿಂದ ಕೇಳುತ್ತಿದ್ದ ಮನುಷ್ಯನೊಬ್ಬನು, “ಎಂದರೆ ನೀವು ಒಂದು ವಸ್ತುವಿನ ಮೂಲಕ ಅದರ ಹಿಂದೆ ಇರುವ ಭಾವವನ್ನು ಪೂಜಿಸುತ್ತೀರಿ.” ಎಂದರು.

\vskip 3pt

ಸ್ವಾಮೀಜಿ ಅವರು ಕೃತಜ್ಞತೆಯಿಂದ “ಹೌದು, ನೀವು ಹೇಳುವುದು ಸರಿ” ಎಂದರು.

\vskip 3pt

ಇದೇ ರೀತಿ ವಿವೇಕಾನಂದರು ಸಂಭಾಷಣೆಯನ್ನು ಮಾಡುತ್ತಿದ್ದರು. ಆದರೆ ಅವರ ಉಪನ್ಯಾಸಕ್ಕೆ ಸಮಯವಾದ್ದರಿಂದ ಇಷ್ಟಕ್ಕೆ ಮಾತು ಮುಕ್ತಾಯವಾಯಿತು.

