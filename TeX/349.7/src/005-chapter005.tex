
\chapter[ಭಗವಾನ್ ಬುದ್ಧ ]{ಭಗವಾನ್ ಬುದ್ಧ \protect\footnote{\engfoot{C.W. Vol. IV. P. 135}}}

\centerline{\textbf{(ಡೆಟ್ರಾಯಿಟ್​ನಲ್ಲಿ ನೀಡಿದ ಉಪನ್ಯಾಸ)}}

ಪ್ರತಿಯೊಂದು ಧರ್ಮದಲ್ಲಿಯೂ ಒಂದೊಂದು ಬಗೆಯ ನೀತಿಯು ಹೆಚ್ಚು ಪ್ರವೃದ್ಧಮಾನವಾಗಿರುವುದು ನಮಗೆ ಕಂಡುಬರುತ್ತದೆ. ಅನಾಸಕ್ತಿ ಯಿಂದ ಕೆಲಸ ಮಾಡುವುದನ್ನು ಬೌದ್ಧಧರ್ಮದಲ್ಲಿ ಹೆಚ್ಚು ರೂಢಿಸಿರುವರು. ಬೌದ್ಧಧರ್ಮ ಮತ್ತು ಬ್ರಾಹ್ಮಣಧರ್ಮ ಒಂದೇ ಎಂದು ಭಾವಿಸಬೇಡಿ. ನೀವು ಈ ದೇಶದಲ್ಲಿ (ಅಮೇರಿಕಾ) ಹಾಗೆ ಮಾಡುವ ಸಂಭವ ಇದೆ. ಬೌದ್ಧಧರ್ಮ ನಮ್ಮಲ್ಲಿ ಒಂದು ಶಾಖೆ ಅಷ್ಟೇ. ಇದು ಗೌತಮನೆಂಬ ಮಹಾತ್ಮನಿಂದ ಸ್ಥಾಪಿತವಾಯಿತು. ಆಗಿನ ಕಾಲದಲ್ಲಿ ನಡೆಯು ತ್ತಿದ್ದ ವೃಥಾ ಚರ್ಚೆ, ಆಚಾರಗಳು ಮತ್ತು ಎಲ್ಲಕ್ಕಿಂತ ಹೆಚ್ಚಾಗಿ ಜಾತಿ ಪದ್ಧತಿ – ಇವುಗಳನ್ನು ನೋಡಿ ರೋಸಿ ಬೌದ್ಧಧರ್ಮವನ್ನು ಅವನು ಸ್ಥಾಪಿಸಿದನು. ಕೆಲವರು ತಾವೊಂದು ಜಾತಿಯಲ್ಲಿ ಜನ್ಮತಾಳಿರುವೆವು, ಆದಕಾರಣ ಹಾಗೆ ಜನ್ಮ ತಾಳದ ಇತರರಿಗಿಂತ ತಾವು ಮೇಲು ಎಂದು ಭಾವಿಸುವರು. ಗೌತಮನು ಪೌರೋಹಿತ್ಯವನ್ನು ಕಟುವಾಗಿ ವಿರೋಧಿಸುತ್ತಿದ್ದ. ಯಾವ ಸ್ವಾರ್ಥೋದ್ದೇಶವೂ ಇಲ್ಲದ ಒಂದು ಧರ್ಮವನ್ನು ಅವನು ಬೋಧಿಸಿದನು. ಈಶ್ವರನಿಗೆ ಸಂಬಂಧಪಟ್ಟ ಸಿದ್ಧಾಂತಗಳ ವಿಷಯದಲ್ಲಿ ಅವನು ಅಜ್ಞೇಯತಾವಾದಿಯಾಗಿದ್ದ. ದೇವರಿರುವನೆ ಎಂದು ಅನೇಕ ವೇಳೆ ಅವನಿಗೆ ಪ್ರಶ್ನೆ ಹಾಕಿದರು. ಅವನು ನನಗೆ ಗೊತ್ತಿಲ್ಲ ಎಂದನು. ನೀತಿಯ ವಿಷಯವಾಗಿ ಕೇಳಿದಾಗ ಒಳ್ಳೆಯವರಾಗಿ, ಒಳ್ಳೆಯದನ್ನು ಮಾಡಿ ಎಂದು ಹೇಳುತ್ತಿದ್ದನು. ಐದು ಜನ ಬ್ರಾಹ್ಮಣರು ಅವನ ಬಳಿ ಬಂದು ತಮ್ಮ ವ್ಯಾಜ್ಯವನ್ನು ಪರಿಹರಿಸ ಬೇಕೆಂದು ಹೇಳಿದರು. ಒಬ್ಬ “ಸ್ವಾಮಿ, ನಮ್ಮ ಶಾಸ್ತ್ರ, ದೇವರು ಹೀಗಿರುವನು, ಅವನೆಡೆಗೆ ಹೋಗುವುದಕ್ಕೆ ಇದೇ ದಾರಿ, ಎಂದು ಹೇಳುವುದು” ಎಂದನು. ಮತ್ತೊಬ್ಬ “ಅದು ತಪ್ಪು, ನನ್ನ ಶಾಸ್ತ್ರ ಬೇರೆ ಬಗೆಯಾಗಿ ಹೇಳುವುದು, ದೇವರೆಡೆಗೆ ಬರುವುದಕ್ಕೆ ಇದೇ ದಾರಿ” ಎಂದನು. ಹೀಗೆ ಎಲ್ಲರೂ ವಾದಿಸಿದರು. ಇವರು ಹೇಳುವುದನ್ನೆಲ್ಲಾ ಶಾಂತವಾಗಿ ಕೇಳಿ ಆದಮೇಲೆ ಬುದ್ಧ ಇವರನ್ನು ಕುರಿತು “ನಿಮ್ಮ ಯಾವುದಾದರೂ ಗ್ರಂಥ, ದೇವರಿಗೆ ಕೋಪ ಬರುವುದು, ಅವನು ಇತರರನ್ನು ನೋಯಿಸುವನು, ಅವನು ಅಶುದ್ಧ ಎಂದು ಹೇಳುವುದೆ?” ಎಂದು ಪ್ರಶ್ನಿಸಿದ. “ಇಲ್ಲ ಸ್ವಾಮಿ, ಅವನು ಪರಿಶುದ್ಧ, ಅವನು ಒಳ್ಳೆಯವನು ಎಂದು ಎಲ್ಲಾ ಶಾಸ್ತ್ರಗಳೂ ಸಾರುವುವು” ಎಂದರು. “ಹಾಗಾದರೆ ನನ್ನ ಸ್ನೇಹಿತರೆ, ಮೊದಲು ನೀವೆಲ್ಲ ಒಳ್ಳೆಯವರಾಗಿ, ಪವಿತ್ರರಾಗಿ, ಆಗ ದೇವರನ್ನು ನೀವು ಅರಿಯಬಹುದು” ಎಂದನು.

ಅವನು ಹೇಳುವ ತತ್ತ್ವವನ್ನೆಲ್ಲಾ ನಾನು ಒಪ್ಪಿಕೊಳ್ಳುವುದಿಲ್ಲ. ನನ್ನ ಪಾಲಿಗೆ ಬೇಕಾದಷ್ಟು ತತ್ತ್ವಚಿಂತನೆ ಬೇಕು. ಅನೇಕ ಅಂಶಗಳಲ್ಲಿ ನನಗೆ ಅಭಿಪ್ರಾಯ ಭೇದ ವಿದೆ, ಆದರೆ ಅದರಿಂದಾಗಿ ಆ ಮನುಷ್ಯನ ಮಾಹಾತ್ಮ್ಯವನ್ನು ಮೆಚ್ಚಕೂಡದೇ? ಯಾವ ಆಸೆಯೂ ಇಲ್ಲದವನು ಅವನೊಬ್ಬನೆ. ಇನ್ನೂ ಬೇರೆ ಮಹಾತ್ಮರಿದ್ದರು. ಅವರೆಲ್ಲ ತಾವು ಅವತಾರವೆಂದೂ ತಮ್ಮನ್ನು ನಂಬಿದವರಿಗೆ ಮೋಕ್ಷ ದೊರಕು ವುದೆಂದೂ ಸಾರಿದರು. ಆದರೆ ಬುದ್ಧ ಮೃತ್ಯುಶಯ್ಯೆಯಲ್ಲಿಯೂ ಏನನ್ನು ಹೇಳಿದ? “ಯಾರೂ ನಿಮಗೆ ಸಹಾಯ ಮಾಡಲಾರರು. ನಿಮಗೆ ನೀವೇ ಸಹಾಯ ಮಾಡಿ ಕೊಳ್ಳಬೇಕು. ನಿಮ್ಮ ಮುಕ್ತಿಯನ್ನು ನೀವೇ ಹುಡುಕಿಕೊಳ್ಳಬೇಕು.” ಎಂದು ಹೇಳಿದ. ಬುದ್ದ ತನ್ನನ್ನು ಕುರಿತು ಹೀಗೆ ಹೇಳುವನು: “ಬುದ್ಧ ಎಂಬುದು ಆಕಾಶದಂತೆ ವಿಶಾಲವಾದ ಅನಂತಜ್ಞಾನದ ಸ್ಥಿತಿ. ಗೌತಮನಾದ ನಾನು ಅದನ್ನು ಪಡೆದಿರುವೆನು. ನೀವೂ ಅದಕ್ಕೆ ಪ್ರಯತ್ನಪಟ್ಟರೆ ಅದನ್ನು ಪಡೆಯುತ್ತೀರಿ.” ಅವನಲ್ಲಿ ಯಾವ ಆಸೆಯೂ ಇಲ್ಲದೆ ಇದ್ದುದರಿಂದ ಅವನು ಸ್ವರ್ಗಕ್ಕೆ ಹೋಗಲು ಇಚ್ಛಿಸಲಿಲ್ಲ. ಅವನಿಗೆ ಹಣ ಬೇಕಾಗಿರಲಿಲ್ಲ. ಅವನು ರಾಜ್ಯ ಮುಂತಾದು ವನ್ನೆಲ್ಲಾ ತ್ಯಜಿಸಿದ್ದನು. ಸಾಗರದಷ್ಟು ವಿಶಾಲವಾದ ಹೃದಯದಿಂದ ಪ್ರಾಣಿ ಮತ್ತು ಮನುಷ್ಯರ ಹಿತವನ್ನು ಬೋಧಿಸುತ್ತ ದಾರಿಯಲ್ಲಿ ಭಿಕ್ಷೆ ಬೇಡುತ್ತಾ ಹೋದನು.

ಒಂದು ಪ್ರಾಣಿಬಲಿಯನ್ನು ತಪ್ಪಿಸಲು ತನ್ನ ಪ್ರಾಣವನ್ನೇ ಬಲಿಕೊಡಲು ಸಿದ್ಧನಾಗಿದ್ದ\-ವನು ಅವನೊಬ್ಬನೆ. ಅವನು ಒಮ್ಮೆ ದೊರೆಗೆ “ಒಂದು ಕುರಿಮರಿಯನ್ನು ಬಲಿಕೊಡುವುದರಿಂದ ಸ್ವರ್ಗಕ್ಕೆ ಹೋಗುವ ಹಾಗಿದ್ದರೆ, ನರಬಲಿಯಿಂದ ಇನ್ನೂ ಹೆಚ್ಚಿನ ಪ್ರಯೋಜನ\break ವಾಗುವುದು. ನನ್ನನ್ನು ಬಲಿಕೊಡು” ಎಂದನು. ದೊರೆಗೆ ಆಶ್ಚರ್ಯ ವಾಯಿತು. ಆದರೂ ಈ ಮನುಷ್ಯನಲ್ಲಿ ಎಳ್ಳಷ್ಟೂ ಆಸೆ ಇರಲಿಲ್ಲ. ಕರ್ಮ ಯೋಗದ ಪರಾಕಾಷ್ಠಕ್ಕೆ ಅವನು ಸಂಕೇತನಾಗಿರುವನು. ಅವನು ಏರಿದ ಎತ್ತರವನ್ನೇ ನಾವೂ ಕೂಡ ಕರ್ಮದ ಸಹಾಯದಿಂದ ಏರಬಹುದೆಂಬುದನ್ನು ಅವನು ತೋರಿಸಿರುವನು.

ಅನೇಕರಿಗೆ ಒಬ್ಬ ಸಾಕಾರದೇವನನ್ನು ಅವರು ನಂಬಿದರೆ ಪ್ರಯಾಣ ಸುಲಭವಾಗು ವುದು. ದೇವರನ್ನು ನಂಬದಿದ್ದರೂ, ಯಾವ ಸಿದ್ದಾಂತವು ಇಲ್ಲದೆ ಇದ್ದರೂ, ಯಾವ ಪಂಗಡಕ್ಕೆ ಸೇರದೆ ಇದ್ದರೂ, ಯಾವ ದೇವಸ್ಥಾನ ಅಥವಾ ಮಠಕ್ಕೆ ಹೋಗದೆ ಇದ್ದರೂ, ನಾಸ್ತಿಕನಾಗಿದ್ದರೂ, ಅವನು ಕೂಡ ಪರಮಪದವನ್ನು ಮುಟ್ಟಬಹುದೆಂಬು ದನ್ನು ಬುದ್ಧನ ಜೀವನ ತೋರುವುದು. ಅವನ ಮೇಲೆ ತೀರ್ಪು ಕೊಡುವುದಕ್ಕೆ ನಮಗೆ ಅಧಿಕಾರವಿಲ್ಲ. ಆ ಬುದ್ಧನ ಅಸೀಮ ಅನುಕಂಪದ ಲವಲೇಶವಾದರೂ ನನ್ನಲ್ಲಿ ಇದ್ದಿದ್ದರೆ ಎಂದು ಬಯಸುವೆನು. ಬುದ್ಧನು ದೇವರನ್ನು ನಂಬಿದ್ದನೋ ಇಲ್ಲವೋ ಚಿಂತೆಯಿಲ್ಲ. ನನಗೆ ಇದು ಬೇಕಾಗಿಲ್ಲ. ಇತರರು ಭಕ್ತಿ, ಯೋಗ, ಜ್ಞಾನದ ಮೂಲಕ ಯಾವ ಪರಮಪದವನ್ನು ಮುಟ್ಟುವರೋ ಅದನ್ನೇ ಅವನೂ ಸೇರಿದನು. ಪರಿ ಪೂರ್ಣತೆ ಊಹೆ ಅಥವಾ ನಂಬಿಕೆಯಿಂದ ಬರುವುದಿಲ್ಲ. ಮಾತಿನಿಂದ ಏನೂ ಪ್ರಯೋಜನವಿಲ್ಲ. ಅರಗಿಳಿಗಳು ಇವನ್ನು ಮಾಡಬಲ್ಲವು.\break ಅನಾಸಕ್ತಿಯಿಂದ ಕರ್ಮ ಮಾಡುವುದರಿಂದ ಪರಿಪೂರ್ಣತೆ ಸಿದ್ಧಿಸುವುದು.

