
\chapter[ದೇವದೂತ ಏಸುಕ್ರಿಸ್ತ ]{ದೇವದೂತ ಏಸುಕ್ರಿಸ್ತ \protect\footnote{\engfoot{C.W. Vol. IV. P. 138}}}

\centerline{\textbf{(೧೯೦೦ರಲ್ಲಿ ಅಮೆರಿಕದ ಲಾಸ್​ ಏಂಜಲೀಸ್​ನಲ್ಲಿ ನೀಡಿದ ಉಪನ್ಯಾಸ)}}

ಅಲೆಯೊಂದು ಸಾಗರದಲ್ಲಿ ಏಳುವುದು, ಪುನಃ ಬೀಳುವುದು. ಮತ್ತೊಂದು ಅಲೆ ಹಿಂದಿನ ಅಲೆಗಿಂತ ದೊಡ್ಡದಾಗಿರಬಹುದು, ಅದು ಏಳುವುದು, ಪುನಃ ಬೀಳುವುದು, ಪುನಃ ಏಳುವುದು. ಹೀಗೆ ಮುಂದೆ ಸಾಗುವುದು. ಜಗತ್ತಿನ ಜೀವನಗತಿಯಲ್ಲಿಯೂ ಹೀಗೆ ಏಳುಬೀಳುಗಳನ್ನು ನೋಡುವೆವು. ನಮ್ಮ ಗಮನವೆಲ್ಲ ಹೆಚ್ಚಾಗಿ ಏಳಿನ ಕಡೆ. ಬೀಳಿನ ಕಡೆ ದೃಷ್ಟಿಯೇ ಬೀಳುವು\-ದಿಲ್ಲ. ಆದರೆ ಎರಡೂ ಅವಶ್ಯಕ, ಎರಡೂ ಮುಖ್ಯವಾದುವೆ. ಇದೇ ಪ್ರಪಂಚದ ಸ್ವಭಾವ. ಆಲೋಚನಾ ಕ್ಷೇತ್ರದಲ್ಲಾಗಲಿ, ಸಮಾಜದ ವ್ಯವಹಾರದಲ್ಲಾಗಲಿ, ಆಧ್ಯಾತ್ಮಿಕ ಜೀವನದಲ್ಲಾಗಲಿ, ಎಲ್ಲೆಡೆಗಳಲ್ಲಿಯೂ ಈ ಏಳುಬೀಳುಗಳು ಸಾಗುತ್ತಿವೆ. ಘಟನೆಗಳ ಸರಣಿಯಲ್ಲಿ ಪ್ರಮುಖವಾದ ಉದಾತ್ತ ಆದರ್ಶಗಳು, ಅನಾವಶ್ಯಕವಾದ ಹಿಂದಿನದನೆಲ್ಲ ಜೀರ್ಣಿಸಿಕೊಂಡು ಗತಕಾಲದ ನ್ಯೂನಾತಿರೇಕಗಳನ್ನು ಅರಿತುಕೊಂಡು, ಹಿಂದಿನದಕ್ಕಿಂತ ಹೆಚ್ಚಿನ ಉಚ್ಛ್ರಾಯ ಸ್ಥಿತಿಗೆ ಏರುವುದಕ್ಕೆ ಶಕ್ತಿಯನ್ನು ಸಂಗ್ರಹಿಸಿಕೊಂಡು ಅಣಿಯಾಗು\-ತ್ತಿರುವುವು.

ದೇಶಗಳ ಇತಿಹಾಸ ಕೂಡ ಎಂದಿಗೂ ಹೀಗೆಯೆ. ಇಂದು ನಾವು ಯಾರ ವಿಷಯವನ್ನು ತಿಳಿಯಬೇಕೆಂದು ಇರುವೆವೊ ಅಂತಹ ಮಹಾತ್ಮ ದೇವದೂತನು ತನ್ನ ದೇಶವು ತುಂಬಾ ಅವನತಿಯ ಸ್ಥಿತಿಯಲ್ಲಿದ್ದಾಗ ಬಂದನು. ಅವನ ಉಪದೇಶದ ಮತ್ತು ಕಾರ್ಯಗಳ ಅಲ್ಪ ಸ್ವಲ್ಪ ವಿವರ ನಮಗೆ ಸಂಕ್ಷಿಪ್ತವಾಗಿ ದೊರಕುವುದು ಇಂತಹ ಮಹಾವ್ಯಕ್ತಿಯ ಉಪದೇಶ ಮತ್ತು ಘಟನೆಗಳೆಲ್ಲ ದೊರೆತು, ಅವನ್ನು ಬರೆದಿಟ್ಟಿದ್ದರೆ ಅವು ಬೃಹದಾಕಾರ ತಾಳುತ್ತಿದ್ದುವೆಂದು ಹೇಳಬಹುದು. ಮೂರು ವರುಷಗಳ ಅವನ ಕಾರ್ಯ ಒಂದು ಯುಗವನ್ನೇ ಭಟ್ಟಿಯಿಳಿಸಿದಂತಿದೆ. ಅದು ವಿವರವಾಗಿ ಪ್ರಕಾಶಕ್ಕೆ ಬರಲು ಹತ್ತೊಂಭತ್ತು ಶತಮಾನಗಳು ಹಿಡಿದಿವೆ. ಅದು ಪೂರ್ಣವಾಗಲೂ ಇನ್ನೂ ಎಷ್ಟು ಕಾಲ ಬೇಕೋ ಯಾರಿಗೆ ಗೊತ್ತು! ನಮ್ಮ ನಿಮ್ಮಂತಹ ಅಲ್ಪ ವ್ಯಕ್ತಿಗಳು ಎಲ್ಲೋ ಅಲ್ಪ ಸ್ವಲ್ಪ ಶಕ್ತಿಗೆ ಮಾತ್ರ ಹಕ್ಕುದಾರರು ಅದನ್ನೆಲ್ಲಾ ಪೂರ್ತಿ ಬಳಸಬೇಕಾದರೆ ಕೆಲವು ಕ್ಷಣಗಳು, ಗಂಟೆಗಳು ಹೆಚ್ಚು ಎಂದು ಕೆಲವು ವರುಷಗಳು ಸಾಕು. ಅನಂತರ ನಾವು ಸಂಪೂರ್ಣ ಕಣ್ಮರೆಯಾಗುವೆವು. ಆದರೆ ಈ ಮಹಾಮಹಿಮನನ್ನು ಗಮನಿಸಿ. ಶತಮಾನಗಳ ಮೇಲೆ ಶತಮಾನಗಳು ಉರುಳಿವೆ. ಆದರೂ ಅವನು ಜಗತ್ತಿನ ಮೇಲೆ ಬೀರಿದ ಶಕ್ತಿ ಇನ್ನೂ ಪೂರ್ಣ ವಿಸ್ತಾರವಾಗಿಲ್ಲ, ಇನ್ನೂ ಶಿಥಿಲವಾಗಿಲ್ಲ, ಕಾಲ ಕಳೆದಂತೆ ಅದು ಇನ್ನೂ ಹೆಚ್ಚು ಪ್ರಬಲವಾಗುತ್ತಾ ಬರುವುದು.

ಏಸುಕ್ರಿಸ್ತನ ಜೀವನದಲ್ಲಿ ಅವನ ಹಿಂದಿನ ಕಾಲವೆಲ್ಲವೂ ಅವನಲ್ಲಿ ಅಂತರ್ಗತವಾಗಿರುವುದು ಕಾಣುವುದು. ಒಂದು ರೀತಿಯಲ್ಲಿ ಪ್ರತಿಯೊಬ್ಬನ ಜೀವನದಲ್ಲಿಯೂ ಅವನ ಭೂತಕಾಲವೆಲ್ಲ ಅಡಗಿದೆ. ಅನುವಂಶೀಯತೆ, ವಾತಾವರಣ, ವಿದ್ಯಾಭ್ಯಾಸ ಮತ್ತು ತನ್ನ ಹಿಂದಿನ ಜನ್ಮಗಳು ಇವುಗಳ ಮೂಲಕ ಒಬ್ಬನಲ್ಲಿ ಅವನ ಜನಾಂಗದ ಭೂತಕಾಲವೆಲ್ಲ ಬಂದು ನೆಲಸುವುದು. ಒಂದು ರೀತಿ ಜಗತ್ತಿನ ಭೂತಕಾಲ, ಇಡೀ ಬ್ರಹ್ಮಾಂಡದ ಭೂತಕಾಲ ಪ್ರತಿಯೊಂದು ಜೀವಿಯಲ್ಲಿಯೂ ಅಂತರ್ಗತವಾಗಿರುವುದು. ನಮ್ಮ ವರ್ತಮಾನಕಾಲದಲ್ಲಿ ಆಗಿರುವುದು ಅನಂತ ಭೂತಕಾಲದ ಪರಿಣಾಮವಲ್ಲದೆ ಮತ್ತೇನು? ಘಟನೆಗಳ ಅನಂತ ಪ್ರವಾಹದ ಮೇಲೆ ಅನೈಚ್ಛಿಕವಾಗಿ ಮುಂದೆ ಮುಂದೆ ನುಗ್ಗುತ್ತಿರುವ, ವಿಶ್ರಾಂತಿಗೆ ಸಾಧ್ಯವಿಲ್ಲದ ಕಿರಿ ಅಲೆಗಳಲ್ಲದೆ ಮತ್ತೇನು? ನಾವು ನೀವುಗಳೆಲ್ಲ ಸಣ್ಣಗುಳ್ಳೆಗಳು. ವಿಶ್ವಘಟನೆಗಳ ಮಹಾಸಾಗರದ ಮೇಲೆ ಪರ್ವತೋಪಮ ಅಲೆಗಳಿವೆ. ನಮ್ಮ ನಿಮ್ಮಲ್ಲಿ ಜನಾಂಗದ ಭೂತಕಾಲ ಎಲ್ಲೋ ಅಲ್ಪಸಲ್ಪ ಅಡಗಿದೆ. ಆದರೆ ಕೆಲವು ಮಹಾವ್ಯಕ್ತಿಗಳಿರುವರು. ಅವರು ಇಡೀ ಭೂತಕಾಲವನ್ನೇ ತಮ್ಮಲ್ಲಿ ಅಳವಡಿಸಿಕೊಂಡು ಅನಂತ ಭವಿಷ್ಯದ ಸ್ವೀಕಾರಕ್ಕೂ ಕರಗಳನ್ನು ಚಾಚಿರುವರು. ಅಲ್ಲಿ ಇಲ್ಲಿ ಇರುವ ಕೈಮರಗಳಂತೆ ಇವರು ಮಾನವಪ್ರಗತಿ ಎತ್ತಕಡೆ ಸಾಗುತ್ತಿದೆ ಎಂಬುದನ್ನು ತೋರುವರು. ಮೇರುಪ್ರಾಯ ವ್ಯಕ್ತಿಗಳಿವರು. ಅವರ ಭೀಮಛಾಯೆ ಪೃಥ್ವಿಯನ್ನೆಲ್ಲಾ ವ್ಯಾಪಿಸುವುದು. ಅಮರವಾಗಿ ಅಮೃತರಾಗಿ ನಿಂತಿರುವರು. ಇದೇ ದೇವದೂತ “ಯಾರಿಗೂ ಯಾವ ಕಾಲದಲ್ಲಿಯೂ ದೇವರ ಮಗನ ಮೂಲಕವಲ್ಲದೆ ಪ್ರತ್ಯಕ್ಷ ದೇವರ ದರ್ಶನವಾಗಿಲ್ಲ” ಎಂದು ಹೇಳುವನು. ಇದು ಸತ್ಯ. ಅವನ ಮಗನಲ್ಲಿ ಅಲ್ಲದೆ ನಾವು ಬೇರೆಲ್ಲಿ ದೇವರನ್ನು ಕಾಣುವುದು? ನಮ್ಮಲ್ಲಿ, ನಿಮ್ಮಲ್ಲಿ, ಅತಿದೀನಲ್ಲಿ, ನೀಚರಲ್ಲಿ ಕೂಡ ದೇವರಿರುವನು, ನಾವು ಅವನನ್ನು ಪ್ರತಿಬಿಂಬಿಸುತ್ತಿರುವೆವು ಎಂಬುದೇನೋ ಸತ್ಯ. ಜ್ಯೋತಿಸ್ಪಂದನ ಸರ್ವವ್ಯಾಪಿಯಾಗಿ ಇರುವುದು. ಆದರೆ ಅದರ ದರ್ಶನವಾಗಬೇಕಾದರೆ ಒಂದು ದೀವಿಗೆಯನ್ನು ಹಚ್ಚಬೇಕಾಗಿದೆ. ಸರ್ವವ್ಯಾಪಿಯಾದ ಈಶ್ವರನದರ್ಶನ, ದೇವಮಾನವರು ದೇವದೂತರು ಅವತಾರಗಳಂತಹ ವಿರಾಟ್​ ಜ್ಯೋತಿ ಸ್ತಂಭಗಳಲ್ಲಿ ಪ್ರತಿಬಿಂಬಿತವಾಗದೆ ನಮಗೆ ಅದರ ಅರಿವು ಉಂಟಾಗಲಾರದು.

ದೇವರಿರುವನೆಂದು ನಮಗೆಲ್ಲಾ ಗೊತ್ತಿದೆ. ಆದರೂ ನಮಗೆ ಅವನು ಅಗೋಚರನು, ಅವನನ್ನು ತಿಳಿಯಲಾರೆವು. ಇಂತಹ ಒಬ್ಬ ದೇವದೂತರನ್ನು ತೆಗೆದುಕೊಂಡು, ನೀವು ಕಲ್ಪಿಸಿಕೊಂಡ ಅತಿಶ್ರೇಷ್ಠ ಭಗವದ್ಭಾವನೆಯೊಂದಿಗೆ ಆ ಮಹಾಮಹಿಮರ ಶೀಲವನ್ನು ಹೋಲಿಸಿ ನೋಡಿ. ಆಗ ನಿಮ್ಮ ಭಗವಂತನ ಭಾವನೆ ಕನಿಷ್ಠವಾದುದೆಂದು ಗೊತ್ತಾಗುವುದು;\break ಮಹಾವ್ಯಕ್ತಿಯ ಶೀಲ ಅದನ್ನೂ ಮೀರಿ ನಿಲ್ಲುವುದು. ಜನ್ಮವೆತ್ತ ಮಹಾಮಹಿಮರು ಭಗ\break ವಂತನ ಯಾವ ಆದರ್ಶವನ್ನು ತಮ್ಮ ಜೀವನದಲ್ಲಿ ಅಳವಡಿಸಿಕೊಂಡು ನಮಗೆ ಮೇಲ್ಪಂಕ್ತಿಯಂತೆ ಇರುವರೊ ಅದಕ್ಕಿಂತ ಉತ್ತಮವಾದ ಭಗವಂತನ ಭಾವನೆಯನ್ನು ನೀವು ಊಹಿಸಲೂ ಆರಿರಿ. ಆದಕಾರಣ ಇಂತಹ ಮಹಾಮಹಿಮರ ಚರಣಗಳಿಗೆರಗಿ, ಪ್ರಪಂಚದಲ್ಲಿರುವ ದೇವಾಂಶ ಪುರುಷರು ಇವರು ಮಾತ್ರ ಎಂದು ಆರಾಧಿಸುವುದರಲ್ಲಿ ದೋಷವೇನಿರುವುದು?\- ನಮ್ಮ ಭಗವದ್ಭಾವನೆಗಳೆಲ್ಲಕ್ಕಿಂತಲೂ ಇವರು ನಿಜವಾಗಿಯೂ ಮೇಲಾಗಿದ್ದರೆ ಇವರನ್ನು ಪೂಜಿಸುವುದರಲ್ಲಿ ಯಾವ ದೋಷವಿದೆ? ದೋಷವಿಲ್ಲ ಮಾತ್ರವಲ್ಲ, ಪ್ರತ್ಯಕ್ಷ ಆರಾಧನೆಗೆ ಇದೊಂದೇ ಮಾರ್ಗ. ಎಷ್ಟೇ ಸಾಧನೆ ಮಾಡಿ, ಹೋರಾಡಿ, ಸ್ಥೂಲವಸ್ತುಗಳನ್ನು ತ್ಯಜಿಸಿ ಸೂಕ್ಷ್ಮಾಂಶವನ್ನು ಸ್ವೀಕರಿಸಿ ಅಥವಾ ನಿಮಗೆ ತಿಳಿದ ಯಾವುದೇ ಮಾರ್ಗವನ್ನು ಅನುಸರಿಸಿ; ಆದರೆ ಎಲ್ಲಿಯವರೆಗೆ ನೀವು ಮಾನವರೊಂದಿಗೆ ಒಬ್ಬ ಮಾನವರಾಗಿರುವಿರೋ ಅಲ್ಲಿಯ\-ವರೆಗೆ ನಿಮ್ಮ ಜಗತ್ತು ಮಾನವಸಹಜವಾದುದು, ನಿಮ್ಮ ದೇವರು ಮಾನವ ಸಹಜವಾದವನು. ಇದು ಅನ್ಯಥಾ ಇರಲಾರದು. ಒಂದು ಸ್ಥೂಲ ಮಧ್ಯವರ್ತಿ ಇಲ್ಲದೆ ಗ್ರಹಿಸಲಸಾಧ್ಯವಾದ ಕೇವಲ ಸೂಕ್ಷ್ಮ ಭಾವನೆಯನ್ನು ತ್ಯಜಿಸಿ ಎದುರಿಗಿರುವ ಅನುಷ್ಠಾನ ಯೋಗ್ಯವಾದ ವಾಸ್ತವಿಕ ವ್ಯಕ್ತಿಯನ್ನು ಯಾರು ತಾನೆ ಸ್ವೀಕರಿಸುವುದಿಲ್ಲ? ಆದಕಾರಣವೆ ಎಲ್ಲಾ ದೇಶಗಳಲ್ಲಿಯೂ ಎಲ್ಲ ಕಾಲಗಳಲ್ಲಿಯೂ ಭಗವಂತನ ಅವತಾರಗಳನ್ನು ಜನರು ಆರಾಧಿಸುವರು.

ಯೊಹೂದ್ಯರಲ್ಲಿ ಅವತರಿಸಿದ ಏಸುಕ್ರಿಸ್ತನ ಜೀವನದ ಸ್ವಲ್ಪ ಭಾಗವನ್ನೀಗ ಅಧ್ಯಯನ ಮಾಡೋಣ. ಎರಡು ಏಳುಗಳ ಮಧ್ಯೆ ಇದ್ದ ಒಂದು ಬೀಳಿನಲ್ಲಿ ಯೆಹೂದ್ಯ ಜನಾಂಗವಿತ್ತು.\break ಆ ಸ್ಥಿತಿಯಲ್ಲಿ ಜನಾಂಗ ಮುಂದೆ ಹೋಗುವುದರಲ್ಲಿ ತಾತ್ಕಾಲಿಕವಾಗಿ ಬೇಸತ್ತು, ತನ್ನಲ್ಲಿರುವುದನ್ನು ಮಾತ್ರ ಸಂರಕ್ಷಿಸಿಕೊಳ್ಳುವುದರಲ್ಲಿ ಆಸಕ್ತವಾಗಿತ್ತು. ಏಸುವು ಅವತರಿಸಿದ ಕಾಲದಲ್ಲಿ\break ಜೀವನದ ಮುಖ್ಯ ಸಮಸ್ಯೆಗಳನ್ನು ಗಣನೆಗೆ ತಾರದೆ ಅನಾವಶ್ಯಕವಾದ ವಿವರಗಳ ಮೇಲೆ ಮಾತ್ರ ಜನರ ಲಕ್ಷ್ಯವಿತ್ತು. ಜನಾಂಗವು ಮುಂದೆ ಹೋಗುವುದಕ್ಕಿಂತ ಇದ್ದ ಸ್ಥಳದಲ್ಲಿಯೇ ಇರಲು ಯತ್ನಿಸುತ್ತಿತ್ತು. ಹೊಸ ಉದ್ಯಮದಲ್ಲಿ ನಿರತವಾಗದೆ ಹಿಂದಿನದನ್ನೇ ಅನುಭವಿಸುವ\break ಸ್ಥಿತಿಯಲ್ಲಿತ್ತು. ಇಂತಹ ಪರಿಸ್ಥಿತಿಯನ್ನು ನಾನು ದೂರುವುದಿಲ್ಲ. ಇದನ್ನು ನೀವು ಗಮನದಲ್ಲಿಡಬೇಕು. ಇದನ್ನು ದೂರುವುದಕ್ಕೆ ನಮಗೆ ಅಧಿಕಾರವಿಲ್ಲ. ಈ ಅವನತಿ ಇಲ್ಲದಿದ್ದಲ್ಲಿ ನಜರೆತ್ತಿನ ಜೀಸಸ್ಸಿನ ಅವತಾರಕ್ಕೆ ಅವಕಾಶವೇ ಇರುತ್ತಿರಲಿಲ್ಲ. ಫ್ಯಾರಿಸಿಗಳು ಮತ್ತು ಸ್ಯಾಡ್ಯೂಸಿಗಳು (ಯೆಹೂದ್ಯರಲ್ಲಿ ಎರಡು ಪಂಗಡಗಳು) ಪ್ರಾಮಾಣಿಕರಲ್ಲದೆ ಇರಬಹುದು. ಅವರು ಯಾವುದನ್ನು ಮಾಡಬಾರದೊ ಅದನ್ನು ಮಾಡುತ್ತಿದ್ದಿರಬಹುದು. ಅವರು ಆಷಾಢಭೂತಿಗಳಾಗಿಯೂ ಇದ್ದಿರಬಹುದು. ಅವರು ಏನಾದರೂ ಆಗಿರಲಿ, ಈ ಸ್ಥಿತಿಗಳೇ ದೇವದೂತನ ಅವತಾರಕ್ಕೆ ಕಾರಣವಾಗಿದೆ. ಒಂದು ಕೊನೆಯಲ್ಲಿರುವ ಆ ದುರ್ಜನರು ಮತ್ತು ಕಪಟಿಗಳು ಮತ್ತೊಂದು ಕಡೆ ಮಹಾಮಹಿಮನಾದಂಥ ಏಸುಕ್ರಿಸ್ತ ಬರುವುದಕ್ಕೆ ಕಾರಣರಾದರು.

ಧರ್ಮದ ದೈನಂದಿನ ವ್ಯವಹಾರ, ಪೂಜೆ, ಪುರಸ್ಕಾರ ಇವುಗಳನ್ನು ನೋಡಿ ಕೆಲವು ವೇಳೆ\break ನಾವು ನಗಬಹುದು. ಆದರೂ ಅವುಗಳ ಅಂತರಾಳದಲ್ಲಿ ಶಕ್ತಿ ಇದೆ. ಹಲವು ವೇಳೆ ನಾವು ಮುಂದೆ ಧಾವಿಸುವುದರಿಂದ ಶಕ್ತಿಯನ್ನು ಕಳೆದುಕೊಳ್ಳುತ್ತೇವೆ. ವಾಸ್ತವಿಕವಾಗಿ ಮತಗಳ\break ವಿಷಯದಲ್ಲಿ ಉದಾರವಾದ ಭಾವನೆ ಉಳ್ಳವನಿಗಿಂತ ಮತಭ್ರಾಂತನಲ್ಲಿ ಹೆಚ್ಚು ಶಕ್ತಿ ಇರುತ್ತದೆ. ಮತಭ್ರಾಂತನಲ್ಲಿಯೂ ಒಂದು ಒಳ್ಳೆಯಗುಣವಿದೆ. ಅವನು ಅದ್ಭುತ ಶಕ್ತಿಯನ್ನು ಸಂಗ್ರಹಿಸುವನು. ವ್ಯಕ್ತಿಯಂತೆ ಜನಾಂಗವು ಕೂಡ ಮುಂದಿನ ಉಪಯೋಗಕ್ಕೆ ಶಕ್ತಿಯನ್ನು ಸಂಚಯಿಸುವುದು. ಯೊಹೂದ್ಯರು ಸುತ್ತಲೂ ಬಾಹ್ಯವೈರಿಗಳಿಂದ ಸುತ್ತಲ್ಪಟ್ಟು, ರೋಮನರಿಂದ ಒಂದು ಕೇಂದ್ರದ ಕಡೆಗೆ ಒಯ್ಯಲ್ಪಟ್ಟು, ಆಲೋಚನಾಕ್ಷೇತ್ರಗಳಲ್ಲಿ ಗ್ರೀಕರ ಒತ್ತಡಕ್ಕೆ ಒಳಗಾಗಿ; ಪರ್ಷಿಯ, ಇಂಡಿಯ, ಅಲೆಗ್ಸಾಂಡ್ರಿಯ–ಈ ಕಡೆಗಳಿಂದ ಬರುವ ಅಲೆಗಳ ಹೊಡತಕ್ಕೆ ಒಳಗಾಗಬೇಕಾಯಿತು. ಹೀಗೆ ಯಹೂದ್ಯ ಜನಾಂಗವು ಸುತ್ತಲೂ ಭೌತಿಕವಾಗಿ, ಮಾನಸಿಕವಾಗಿ ಮತ್ತು ನೈತಿಕವಾಗಿ ಬಾಹ್ಯ ಒತ್ತಡಗಳಿಂದ ಆವರಿಸಲ್ಪಟ್ಟಿತ್ತು. ಅನುವಂಶಿಕವಾಗಿ ಬಂದ ಪೂರ್ವಾಚಾರದ ಅದ್ಭುತ ಶಕ್ತಿಯೊಂದಿಗೆ ಇದೆಲ್ಲವನ್ನೂ ಎದುರಿಸಿ ನಿಂತಿತು. ಇಂದಿಗೂ ಈ ಶಕ್ತಿ ಅವರ ಅನುಯಾಯಿಗಳಲ್ಲಿ ನಾಶವಾಗಿಲ್ಲ. ಆ ಜನಾಂಗ ತನ್ನ ಶಕ್ತಿಯನ್ನೆಲ್ಲಾ ಯಹೂದ್ಯ ಮತದ ಮೇಲೆ ಮತ್ತು ಜೆರುಸಲೆಮ್​ ನಗರದ ಮೇಲೆ ಬಲಾತ್ಕಾರವಾಗಿ ಕೇಂದ್ರೀಕರಿಸಬೇಕಾಯಿತು. ಶಕ್ತಿಯನ್ನೆಲ್ಲಾ ಒಂದು ಕಡೆ ಸಂಗ್ರಹಿಸಿದರೆ ಅದು ಒಂದೇ ಕಡೆ ಬಹಳಕಾಲ ಇರಲಾರದು. ಅದು ಖರ್ಚಾಗಬೇಕು, ವಿಸ್ತಾರವಾಗಬೇಕು. ಯಾವ ಶಕ್ತಿಯನ್ನೂ ನಾವೊಂದು ಅಲ್ಪಕ್ಷೇತ್ರದಲ್ಲಿ ಬಹಳಕಾಲ ಕೂಡಿಡುವುದಕ್ಕೆ ಆಗುವುದಿಲ್ಲ.

ಯಹೂದ್ಯರಲ್ಲಿ ಕೇಂದ್ರಿಕೃತವಾದ ಈ ಶಕ್ತಿ ಅನಂತರ ಕ್ರೈಸ್ತ–ಧರ್ಮೋದಯದಲ್ಲಿ ವ್ಯಕ್ತವಾಯಿತು. ಹಲವು ಸಣ್ಣ ಗಿರಿಝರಿಗಳು ಸಂಗಮವಾದವು. ಉಳಿದ ಪ್ರವಾಹಗಳೆಲ್ಲಾ ಅನಂತರ ಒಂದುಗೂಡಿ ಒಂದು ಮಹಾಪ್ರವಾಹವಾಯಿತು. ಈ ಪ್ರವಾಹದ ಅಗ್ರಭಾಗದಲ್ಲಿ ನಜರೆತ್ತಿನ ಕ್ರಿಸ್ತನ ಶೀಲ ಮಂಡಿಸಿತ್ತು. ಹೀಗೆ ಪ್ರತಿಯೊಬ್ಬ ಮಹಾಪುರುಷನೂ ಅವನ ಕಾಲದ ಸೃಷ್ಟಿ; ಅವನ ಜನಾಂಗದ ಭೂತಕಾಲದ ಪರಿಣಾಮ; ಜೊತೆಗೆ ಅವನೇ ಭವಿಷ್ಯತ್ತನ್ನು ನಿರ್ಮಿಸುವವನು. ಹಿಂದಿನ ಕಾರಣ ಇಂದಿನ ಪರಿಣಾಮವಾಗಿದೆ. ಮುಂದಿನ ಭವಿಷ್ಯತ್ತಿಗೆ ಇಂದಿನದೇ ಕಾರಣ. ದೇವದೂತರನು ಈ ಸ್ಥಿತಿಯಲ್ಲಿರುವನು; ಅವನ ಜನಾಂಗದಲ್ಲಿ ಶ್ರೇಷ್ಠತಮವಾಗಿರುವುದೆಲ್ಲ, ಭವ್ಯವಾಗಿರುವುದೆಲ್ಲ, ಈ ವ್ಯಕ್ತಿಯಲ್ಲಿ ವ್ಯಕ್ತವಾಗಿರುವುದು. ಹಲವು ಶತಮಾನಗಳಿಂದ ಯಾವ ಒಂದು ಉದ್ದೇಶಕ್ಕೋಸುಗ ಆ ಜನಾಂಗ ಹೋರಾಡಿದೆಯೋ ಆ ಆದರ್ಶವೆಲ್ಲಾ ಈ ವ್ಯಕ್ತಿಯಲ್ಲಿ ಅಂತರ್ಗತವಾಗಿದೆ. ಈ ವ್ಯಕ್ತಿಯೇ ಭವಿಷ್ಯ ನಿರ್ಮಾಪಕ. ತನ್ನ ಜನಾಂಗ ಒಂದರ ಭವಿಷ್ಯ ಮಾತ್ರವಲ್ಲ, ಪ್ರಪಂಚದ ಹಲವು ಜನಾಂಗಗಳ ಭವಿಷ್ಯಕ್ಕೂ ಅವನು ಕಾರಣನಾಗುವನು.

ನಾವು ಮತ್ತೊಂದು ವಿಷಯವನ್ನೂ ಗಮನಿಸಬೇಕಾಗಿದೆ. ನಾನು ನಜರೆತ್ತಿನ ಮಹಾ ದೇವದೂತನನ್ನು ನೋಡುವುದು ಪ್ರಾಚ್ಯದೃಷ್ಟಿಯಿಂದ. ಅನೇಕ ವೇಳೆ ಕ್ರಿಸ್ತ ಪ್ರಾಚ್ಯರಲ್ಲಿ ಪ್ರಾಚ್ಯ ಎಂಬುದನ್ನು ಮರೆಯುವಿರಿ. ನೀವು ಅವನನ್ನು ನೀಲಿಕಣ್ಣು, ಹಳದಿಕೂದಲು–\break ಇವುಗಳಿಂದ ಕೂಡಿದ್ದಾಗಿ ಕಲ್ಪಿಸಿಕೊಳ್ಳುವುದಕ್ಕೆ ಎಷ್ಟು ಪ್ರಯತ್ನಪಟ್ಟರೂ ಕ್ರಿಸ್ತ ಶುದ್ಧ ಪ್ರಾಚ್ಯನಾಗಿದ್ದ. ಬೈಬಲ್ಲಿನಲ್ಲಿ ಬರುವ ಉಪಮಾನ–ಕಲ್ಪನೆಗಳೆಲ್ಲಾ ಅಲ್ಲಿಯ ದೃಶ್ಯಗಳು, ಆ ಸ್ಥಳ, ಅವರ ನಡೆನುಡಿ, ಜನಸಮೂಹ, ಕಾವ್ಯ, ಅವರ ಪ್ರತೀಕಗಳು ಎಲ್ಲಾ ಪ್ರಾಚ್ಯವನ್ನು ಕುರಿತವುಗಳು. ಅಲ್ಲಿಯ ತಿಳಿ ಆಗಸ, ಉರಿ ಬೇಸಿಗೆ, ಸೂರ್ಯ, ಮರುಭೂಮಿ, ಅಲ್ಲಿ ಬರುವ ಬಾಯಾರಿದ ಮನುಷ್ಯರು, ಪ್ರಾಣಿಗಳು, ನೀರಿಗಾಗಿ ಕೊಡಗಳನ್ನು ತಲೆಯ ಮೇಲೆ ಹೊತ್ತುಕೊಂಡು ಬರುವ ನರನಾರಿಯರು, ಕುರಿಯಮಂದೆ, ರೈತರು, ಸುತ್ತಲೂ ನಡೆಯುತ್ತಿರುವ ಬೇಸಾಯ, ನೀರಿನ ಬಾಯಿ, ನೀರನ್ನು ಎತ್ತುವ ಯಂತ್ರಚಕ್ರ, ಬೀಸುವ ಕಲ್ಲು ಅವನ್ನೆಲ್ಲಾ ಏಷ್ಯಾಖಂಡದಲ್ಲಿ ಇಂದಿಗೂ ನೋಡಬಹುದು.

ಏಷ್ಯಾಖಂಡದ ವಾಣಿ ಧರ್ಮವಾಣಿ, ಯೂರೋಪಿನ ವಾಣಿ ರಾಜಕೀಯ ವಾಣಿ. ಪ್ರತಿಯೊಂದೂ ಅದರದರ ಸ್ಥಾನದಲ್ಲಿ ದೊಡ್ಡದೆ. ಯೂರೋಪಿನ ವಾಣಿ ಪುರಾತನ ಗ್ರೀಕರ ವಾಣಿ. ಗ್ರೀಕರ ಮನಸ್ಸಿಗೆ ತಮ್ಮ ಮುಂದಿರುವ ಸಮಾಜವೇ ಸರ್ವಸ್ವವಾಗಿತ್ತು. ಅದರಾಚೆ ಉಳಿದುವೆಲ್ಲವೂ ಅನಾಗರಿಕ.ಗ್ರೀಕರಲ್ಲದ ಮತ್ತಾರಿಗೂ ಬಾಳುವುದಕ್ಕೆ ಅಧಿಕಾರವಿಲ್ಲ. ಯಾವುದನ್ನು ಗ್ರೀಕರು ಮಾಡುತ್ತಾರೆಯೊ ಅದೇ ಸರಿ, ಅದೇ ಧರ್ಮ. ಪ್ರಪಂಚದಲ್ಲಿ\break ಇರುವ ಉಳಿದುವೆಲ್ಲ ಧರ್ಮವೂ ಅಲ್ಲ, ಸರಿಯೂ ಅಲ್ಲ, ಮತ್ತು ಅವುಗಳ ಉಳಿಗಾಲಕ್ಕೆ\break ಅವಕಾಶವನ್ನೂ ಕೊಡಕೂಡದು. ಆದಕಾರಣವೆ ಗ್ರೀಕರ ಸಹಾನುಭೂತಿ ಮಾನವ ಸಹಜ\-ವಾಗಿದೆ, ಸ್ವಾಭಾವಿಕವಾಗಿ ಕಲಾಪೂರ್ಣವಾಗಿದೆ. ಗ್ರೀಕರು ಈ ಜಗತ್ತಿನಲ್ಲಿ ಮಾತ್ರ ವಿಹರಿಸುವರು, ಕನಸುಣಿಯಾಗಲಿಚ್ಛಿಸುವುದಿಲ್ಲ. ಅವರ ಕಾವ್ಯ ಕೂಡ ವ್ಯವಹಾರ ಪ್ರಧಾನವಾದುದು. ಅವರ ದೇವದೇವತೆಗಳು ಮಾನವರು ಮಾತ್ರವಲ್ಲ, ತೀವ್ರ ಮಾನವ ಭಾವನೆ ಅವರಲ್ಲಿದೆ:\break ಅವರು ನಮ್ಮಲ್ಲಿರುವಂತೆಯೆ ಕೋಪತಾಪಗಳ ಭಾವನೆಗಳಿಂದ ತುಂಬಿ ತುಳುಕಾಡುತ್ತಿರುವರು. ಸುಂದರವಾಗಿರುವುದನ್ನು ಅವರು ಪ್ರೀತಿಸುವರು. ಆದರೆ ಇದನ್ನು ನೆನಪಿನಲ್ಲಿಡಿ: ಬಾಹ್ಯಸೌಂದರ್ಯ, ಪರ್ವತಹಿಮ, ಹೂವು, ಆಕಾರ, ಮತ್ತು ರೂಪು ರಚನೆಯ\break ಸೌಂದರ್ಯ, ಮಾನವವರ್ಗದ ಸೌಂದರ್ಯ– ಅನೇಕ ವೇಳೆ ಅವರ ಶರೀರ ಸೌಂದರ್ಯ, ಇವನ್ನೇ ಗ್ರೀಕರು ಪ್ರೀತಿಸುತ್ತಿದ್ದುದು. ಗ್ರೀಕರು ಅನಂತರ ಬಂದ ಐರೋಪ್ಯರಿಗೆಲ್ಲಾ ಗುರುಗಳಾಗಿದ್ದುದರಿಂದ ಯೂರೋಪಿನ ವಾಣಿ ಗ್ರೀಕ್​ ವಾಣಿಯಾಗಿದೆ.

ಏಷ್ಯಾಖಂಡದಲ್ಲಿ ಮತ್ತೊಂದು ಮಾದರಿ ಇದೆ. ಆ ವಿಶಾಲವಾದ ಏಷ್ಯಾಖಂಡವನ್ನು ಆಲೋಚಿಸಿ ನೋಡಿ. ಅವುಗಳ ಗಿರಿಶಿಖರಗಳು ಮುಗಿಲನ್ನು ತೂರಿ ಹೋಗಿ ನೀಲ ಗಗನ\break ತಳವನ್ನು ತಾಕುವಂತಿದೆ. ಕುಡಿಯುವುದಕ್ಕೆ ಒಂದು ತೊಟ್ಟು ನೀರು ದೊರಕದ, ಒಂದು ಹುಲ್ಲಿನೆಸಳೂ ಕೂಡ ಬೆಳೆಯಲಾರದ ಮೈಲುಮೈಲುಗಳವರೆಗೆ ವಿಸ್ತಾರವಾಗಿ ಹಬ್ಬಿರುವ ಮರಳುಕಾಡುಗಳು, ಅಭೇದ್ಯವಾದ ದಟ್ಟ ಅರಣ್ಯಗಳು, ಸಮುದ್ರದೆಡೆ ಧಾವಿಸುತ್ತಿರುವ ಭೀಮನದಿಗಳು ಅಲ್ಲಿವೆ. ಇಂತಹ ವಾತಾವರಣದಲ್ಲಿ ಪ್ರಾಚ್ಯದ ಭವ್ಯಸೌಂದರ್ಯ ಪಿಪಾಸೆ ಮತ್ತೊಂದು ದಿಕ್ಕಿನಲ್ಲಿ ವಿಕಾಸವಾಯಿತು. ಅದು ಅಂತರ್ಮುಖವಾಯಿತು, ಬಾಹ್ಯಮುಖವಾಗಲಿಲ್ಲ. ಅಲ್ಲಿಯೂ ಬಾಹ್ಯವಸ್ತುವನ್ನು ತಿಳಿದುಕೊಳ್ಳಬೇಕು, ಶಕ್ತಿಯನ್ನು ಪಡೆಯಬೇಕು,\break ಸುಂದರವಾದುದಕ್ಕೆ ಮನ ಸೋಲಬೇಕು ಮುಂತಾದ ಗ್ರೀಕ್​ ಮತ್ತು ಅನಾರ್ಯ ಭಾವನೆಗಳಿವೆ. ಆದರೆ ಇದು ವಿಶಾಲವಾದ ಕ್ಷೇತ್ರಗಳನ್ನು ಆವರಿಸಿತು. ಏಷ್ಯಾಖಂಡದಲ್ಲಿ ಈಗಲೂ ಕೂಡ ಜನ್ಮ, ಬಣ್ಣ, ಭಾಷೆಗಳೇ ಒಂದು ಜನಾಂಗವನ್ನು ಸೃಷ್ಟಿಸುವುದಿಲ್ಲ, ಧರ್ಮವೇ ಜನಾಂ\break ಗವನ್ನು ಸೃಷ್ಟಿಸುವುದು. ನಾವೆಲ್ಲ ಕ್ರೈಸ್ತರು, ಮಹಮ್ಮದೀಯರು, ಹಿಂದುಗಳು ಇಲ್ಲವೇ ಬೌದ್ಧರು. ಬೌದ್ಧರು ಚೈನಾದೇಶಕ್ಕೆ ಸೇರಿರಲಿ ಅಥವಾ ಪರ್ಷಿಯಾ ದೇಶಕ್ಕೆ ಸೇರಿರಲಿ, ಅವರೆಲ್ಲಾ ತಾವು ಸಹೋದರರೆಂದು ಭಾವಿಸುವರು. ಏಕೆಂದರೆ ಎಲ್ಲರೂ ಒಂದೇ ಧರ್ಮಕ್ಕೆ ಸೇರಿದವರು. ಧರ್ಮವೇ ಮಾನವಕೋಟಿಯನ್ನು ಒಂದುಗೂಡಿಸುವುದು. ಈ ಕಾರಣದಿಂದ\break ಪ್ರಾಚ್ಯನು ಕಲ್ಪನಾಜೀವಿ, ಹುಟ್ಟು ಕನಸುಣಿ, ಜಲಪಾತ, ಪಕ್ಷಿಗಾನ, ಸೂರ್ಯ, ಚಂದ್ರ, ತಾರೆ, ಇಡೀ ಪೃಥ್ವಿಯೆಲ್ಲ ಸುಂದರವಾಗಿ ಏನೋ ಇವೆ. ಆದರೆ ಪ್ರಾಚ್ಯನಿಗೆ ಇಷ್ಟೇ ಸಾಲದು.\break ಅವನು ಇದನ್ನು ಮೀರಿರುವುದರ ಕನಸುಕಟ್ಟಲು ಹವಣಿಸುವನು. ವರ್ತಮಾನ ಕಾಲವನ್ನು\break ಅತಿಕ್ರಮಿಸಲು ಯತ್ನಿಸುವನು. ವರ್ತಮಾನಕಾಲಕ್ಕೆ ಅವನು ಬೆಲೆಯನ್ನೇ ಕೊಡುವುದಿಲ್ಲ. ಯುಗ ಯುಗಾಂತರಗಳಿಂದಲೂ ಮಾನವ ಜನಾಂಗದ ತೊಟ್ಟಿಲಾಗಿರುವುದು ಪ್ರಾಚ್ಯ. ಅದು ಎಷ್ಟೋ ಸುಖದುಃಖಗಳನ್ನು ಅನುಭವಿಸಿರುವುದು. ಸಾಮ್ರಾಜ್ಯಗಳಾದ ಮೇಲೆ ಸಾಮ್ರಾಜ್ಯಗಳು, ಚಕ್ರಾಧಿಪತ್ಯಗಳಾದ ಮೇಲೆ ಚಕ್ರಾಧಿಪತ್ಯಗಳ ವೈಭವ ಅಲ್ಲಿ ಬಂದುಹೋದವು; ಮಾನವ ಶಕ್ತಿ, ಐಶ್ವರ್ಯಗಳೆಲ್ಲಾ ಒಂದಾದ ಮೇಲೆ ಒಂದು ಉರುಳಿಹೋದವು. ಐಶ್ವರ್ಯದ ಮತ್ತು ಅಧಿಕಾರದ ಮಹಾ ಸ್ಮಶಾನ ಅದು. ರಾಜ್ಯ, ಅಧಿಕಾರ,\break ಪಾಂಡಿತ್ಯ ಇವುಗಳ ಅದ್ಭುತ ಮಹಾ ಸ್ಮಶಾನವೇ ಪ್ರಾಚ್ಯ. ಪ್ರಾಚ್ಯನ ದೃಷ್ಟಿ ಪ್ರಪಂಚದ ಕ್ಷಣಿಕ ವಸ್ತುಗಳನ್ನು ನಿಕೃಷ್ಟ ದೃಷ್ಟಿಯಿಂದ ನೋಡಿ, ಯಾವುದು ಬದಲಾಗುವುದಿಲ್ಲವೋ, ಯಾವುದು ಸಾಯುವುದಿಲ್ಲವೋ, ಮೃತ್ಯಮಯ ಮತ್ತು ದುಃಖಮಯ ಜಗತ್ತಿನಲ್ಲಿ\break ಯಾವುದು ಅನಂತವಾಗಿರುವುದೋ, ಆನಂದಮಯವಾಗಿರುವುದೊ, ಅಮೃತಮಯ\-ವಾಗಿರುವುದೊ ಅದನ್ನು ಇಚ್ಛಿಸುವುದರಲ್ಲಿ ಆಶ್ಚರ್ಯವೇನಿಲ್ಲ. ಈ ಆದರ್ಶಗಳನ್ನು ಪುನಃ ಪುನಃ ಒತ್ತಿ ಹೇಳುವುದಕ್ಕೆ ಯಾವ ಪ್ರಾಚ್ಯ ಮಹಾತ್ಮನಿಗೂ ಬೇಜಾರಿಲ್ಲ. ವಿನಾಯಿತಿ ಇಲ್ಲದೆ ಎಲ್ಲಾ ಮಹಾತ್ಮರೂ ದೇವದೂತರೂ ಪ್ರಾಚ್ಯದೇಶದಿಂದ ಬಂದವರು.

ಈ ಮಹಾ ದೇವದೂತನ ಸಂದೇಶದಲ್ಲಿ ನಾವು ಮೊದಲು “ಈ ಪ್ರಪಂಚವಲ್ಲ ನಮ್ಮ\break ಗುರಿ, ನಮ್ಮ ಗುರಿ ಇದನ್ನು ಮೀರಿರುವುದು” ಎಂಬ ಉಪದೇಶದ ಪಲ್ಲವಿಯನ್ನು ನೋಡು\break ವೆವು. ನಿಜವಾದ ಪ್ರಾಚ್ಯನ ಸಂತತಿಗೆ ಸೇರಿದವನಂತೆ ಇವನು ಈ ಕಾರ್ಯಕ್ಷೇತ್ರದಲ್ಲಿ ಚತುರ. ಪಾಶ್ಚಾತ್ಯರಾದ ನೀವೂ ಕೂಡ ಸೈನ್ಯ ಕಟ್ಟುವುದರಲ್ಲಿ ರಾಜಕೀಯ ಕ್ಷೇತ್ರ ಮುಂತಾ\-ದುವುಗಳಲ್ಲಿ ವ್ಯವಹಾರ ನಿಪುಣರು. ಪ್ರಾಚ್ಯನು ಇಂತಹ ಕಾರ್ಯಕ್ಷೇತ್ರದಲ್ಲಿ ಚತುರನಲ್ಲ\-ದಿರಬಹುದು, ಆದರೆ ಅವನು ತನ್ನ ಕಾರ್ಯಕ್ಷೇತ್ರದಲ್ಲಿ, ಧರ್ಮದಲ್ಲಿ, ವ್ಯವಹಾರ ಚತುರ. ಒಬ್ಬ ಇಂದು ಯಾವುದಾದರೂ ತತ್ತ್ವವನ್ನು ಉಪದೇಶಿಸಿದರೆ ಅದನ್ನು ತಮ್ಮ ಜೀವನದಲ್ಲಿ ಅನುಷ್ಠಾನಕ್ಕೆ ತರಲು ಯತ್ನಿಸುವ ನೂರಾರು ಜನರು ದೊರಕುವರು. ಒಂದು ಕಾಲಿನ ಮೇಲೆ ನಿಂತರೆ ಮುಕ್ತಿ ಸಿಕ್ಕುವುದೆಂದು ಸಾರಿದರೆ ಹಾಗೆ ಮಾಡುವುದಕ್ಕೆ ಐದುನೂರು ಜನರು ತಕ್ಷಣ ದೊರಕುವರು. ನೀವು ಇದನ್ನು ವಿಲಕ್ಷಣ ಎನ್ನಬಹುದು. ಆದರೆ ಅದರ ಹಿಂದೆ ಅವರ ತತ್ತ್ವ, ತೀವ್ರವಾದ ವ್ಯವಹಾರ ಚತುರತೆಗಳು ಇವೆ. ಪಾಶ್ಚಾತ್ಯರಲ್ಲಿ ಮುಕ್ತಿಯ ಯೋಚನೆಗಳೆಂದರೆ ಕೆಲವು ಬುದ್ಧಿವಂತಿಕೆಯ ಕಸರತ್ತು ಮಾತ್ರ. ಕಾರ್ಯಕಾರಿಯಾಗದ, ಎಂದಿಗೂ ಅನುಷ್ಠಾನಕ್ಕೆ ಬರದ, ಕೆಲವು ಯೋಜನೆಗಳು ಅಷ್ಟೆ. ಪಾಶ್ಚಾತ್ಯರಲ್ಲಿ ಯಾರು ಅತ್ಯುತ್ತಮ ವಾಗ್ಮಿಗಳೋ ಅವರೇ ಶ್ರೇಷ್ಠ ಉಪದೇಶಕರು.

ಏಸು ಪ್ರಥಮತಃ ನಿಜವಾದ ಪ್ರಾಚ್ಯ ಸಂತಾನವಾಗಿದ್ದ. ಅವನು ಧರ್ಮ ಪ್ರಪಂಚದಲ್ಲಿ ಅನುಷ್ಠಾನಪರನಾಗಿದ್ದ ಎಂಬುದನ್ನು ನೋಡುವೆವು. ಈ ಕ್ಷಣಿಕ ಪ್ರಪಂಚದ ಮೇಲೆ ಮತ್ತು ಅದಕ್ಕೆ ಸೇರಿದ ವಸ್ತುಗಳ ಮೇಲೆ ಅವನಿಗೆ ನಂಬಿಕೆ ಇರಲಿಲ್ಲ. ಆಧುನಿಕ ಕಾಲದಲ್ಲಿ ಪಾಶ್ಚಾತ್ಯ ದೇಶದಲ್ಲಿರುವಂತೆ ಶಾಸ್ತ್ರಪೀಡನೆಯ ಆವಶ್ಯಕತೆ ಇರಲಿಲ್ಲ. ಸುಮ್ಮನೆ ಶಾಸ್ತ್ರಗಳನ್ನು ಎಲ್ಲಿಯವರೆಗೆ ಸಾಧ್ಯವೊ ಅಲ್ಲಿಯವರೆಗೆ ಎಳೆದಾಡಬೇಕಾಗಿಲ್ಲ. ಶಾಸ್ತ್ರಗಳು ಇಂಡಿಯಾ ರಬ್ಬರಿನಂತೆ ಅಲ್ಲ. ಆ ರಬ್ಬರಿಗೂ ಒಂದು ಮಿತಿ ಇದೆ. ಆಧುನಿಕ ಕಾಲದ ಇಂದ್ರಿಯತೃಪ್ತಿಗೆ ಧರ್ಮಸಾ\break ಧಕವಾಗಬೇಕಾಗಿಲ್ಲ! ಇದನ್ನು ಗಮನಿಸಿ ನಾವೆಲ್ಲ ಪ್ರಾಮಾಣಿಕರಾಗಿರೋಣ. ಒಂದು ಆದ\break ರ್ಶವನ್ನು ಅನುಸರಿಸಲು ನಮಗೆ ಸಾಧ್ಯವಿಲ್ಲದೆ ಇದ್ದರೆ ನಮ್ಮ ದುರ್ಬಲತೆಯನ್ನು ಒಪ್ಪಿಕೊಳ್ಳೋಣ. ಆದರ್ಶವನ್ನು ಅಧೋಗತಿಗೆ ಎಳೆಯಲು ಯತ್ನಿಸದೆ ಇರೋಣ, ಪಾಶ್ಚಾತ್ಯರು ಕ್ರಿಸ್ತನ ಜೀವನವನ್ನು ನಿರೂಪಿಸುವ ರೀತಿಯನ್ನು ನೋಡಿದರೆ ಜುಗುಪ್ಸೆಯಾಗುವುದು. ಅವನು ಏನಾಗಿದ್ದನೊ ಏನಾಗಿರಲಿಲ್ಲವೊ ನನಗೆ ತಿಳಿಯದು! ಒಬ್ಬ ಅವನನ್ನು ರಾಜಕಾರಣ ಪಟುವನ್ನಾಗಿ ಮಾಡುವನು; ಮತ್ತೊಬ್ಬ ಅವನನ್ನು ವೀರ ಯೋಧನನ್ನಾಗಿ ಮಾಡುವನು; ಮಗದೊಬ್ಬ ಅವನನ್ನು ದೇಶಭಕ್ತ ಯಹೂದಿಯನ್ನಾಗಿ ಮಾಡುವನು ಇತ್ಯಾದಿ. ಅಂಥಹ ಊಹೆಗಳಿಗೆಲ್ಲ ಶಾಸ್ತ್ರಗಳಲ್ಲಿ ಏನಾದರೂ ಆಧಾರವಿದೆಯೆ? ಒಬ್ಬ ಮಹಾವ್ಯಕ್ತಿಯ ಜೀವನವೇ ಅವನ ಬಗೆಗಿನ ಶ್ರೇಷ್ಠ ಭಾಷ್ಯ. “ನರಿಗಳಿಗೆ ಬಿಲಗಳಿವೆ, ಆಗಸದ ಹಕ್ಕಿಗಳಿಗೆ ಗೂಡುಗಳಿವೆ, ಆದರೆ ಮಾನವ ಪುತ್ರನಿಗೆ ತಲೆಯನ್ನು ಇರಿಸಲು ಸ್ಥಳವಿಲ್ಲ.” ಮುಕ್ತಿಗೆ ಇದೊಂದೇ ಮಾರ್ಗ ಎಂದು ಏಸು ಸಾರುವನು. ಮತ್ತಾವ ಮಾರ್ಗವನ್ನೂ ತೋರುವುದಿಲ್ಲ. ಇದು ನಮಗೆ ಸಾಧ್ಯವಿಲ್ಲ ಎಂದು ನಿರ್ವಂಚನೆಯಿಂದ ಒಪ್ಪಿಕೊಳ್ಳೋಣ. ನಮಗೆಲ್ಲ ನಾನು ಎಂಬ ಅಭಿಮಾನ ಇನ್ನೂ ಇದೆ. ನಮಗೆಲ್ಲಾ ಆಸ್ತಿ, ಐಶ್ವರ್ಯ, ದ್ರವ್ಯ ಬೇಕು. ಧಿಃಕ್ಕಾರ ನಮಗೆ! ಇದನ್ನು ನಾವು ಒಪ್ಪಿಕೊಳ್ಳೋಣ. ಮಾನವಕೋಟಿಯ ಶ್ರೇಷ್ಠ ಗುರುವಿಗೆ ಅಪವಾದವನ್ನು ತಾರದಿರೋಣ. ಅವನಿಗೆ ಯಾವ ಸಂಸಾರ ಬಂಧನವೂ ಇರಲಿಲ್ಲ. ಅವನಲ್ಲಿ ಯಾವುದಾದರೂ ದೈಹಿಕ ಭಾವನೆ ಉಳಿದಿತ್ತು ಎಂದು ಭಾವಿಸುವಿರೇನು? ಈ ಜ್ಞಾನಿ, ಮಾನವನಲ್ಲ, ದೇವನು ಮೃಗಗಳೊಂದಿಗೆ ಇರುವುದಕ್ಕೆ ಬಂದನೆಂದು ಊಹಿಸುವಿರೇನು! ಆದರೂ ಜನರು ಎಷ್ಟೋ ವಿಷಯಗಳನ್ನು ಅವನ ಬಾಯಿಯಿಂದ ಬಂದವುಗಳೆಂದು ಹೇಳುವರು. ಅವನಿಗೆ ಲಿಂಗಭಾವನೆ ಇರಲಿಲ್ಲ, ಆತ್ಮವಾಗಿದ್ದನು. ಅದಲ್ಲದೆ ಬೇರೆ ಅಲ್ಲ. ಮಾನವ ಕಲ್ಯಾಣಕ್ಕಾಗಿ ದೇಹದ ಮೂಲಕ ಕೆಲಸಮಾಡುತ್ತಿದ್ದನು ಅಷ್ಟೆ. ಆತ್ಮನಲ್ಲಿ ಲಿಂಗಭೇದವಿಲ್ಲ. ಮುಕ್ತಾತ್ಮನಿಗೆ\break ದೇಹದೊಂದಿಗೆ ಸಂಬಂಧವಿಲ್ಲ, ಮೃಗದೊಂದಿಗೆ ಸಂಬಂಧವಿಲ್ಲ. ಆದರ್ಶ ಬಹಳ ದೂರ\-ದಲ್ಲಿರಬಹುದು, ನಮ್ಮನ್ನು ಮೀರಿರಬಹುದು, ಚಿಂತೆಯಿಲ್ಲ. ಆದರ್ಶವನ್ನು ಯಾವಾಗಲೂ ಮರೆಯಬೇಡಿ. ಅದು ನಮ್ಮ ಆದರ್ಶ, ಆದರೆ ನಾವು ಅದನ್ನು ಇನ್ನೂ ಸೇರಲಾರೆವು ಎಂದು ಒಪ್ಪಿಕೊಳ್ಳೋಣ.

ತಾನೊಬ್ಬ ಆತ್ಮನೆಂಬ ಭಾವನೆಯಲ್ಲದೆ ಬೇರಾವ ಆಲೋಚನೆಯೂ ಅವನಲ್ಲಿ ಇರಲಿಲ್ಲ.\break ಅವನು ದೇಹಾತೀತನಾದ, ಬಂಧನರಹಿತನಾದ ಮುಕ್ತಾತ್ಮ. ಇಷ್ಟು ಮಾತ್ರ ಅಲ್ಲ, ತನ್ನ ಅಪೂರ್ವ ಅನುಭವದಿಂದ ಪ್ರತಿಯೊಬ್ಬ ಸ್ತ್ರೀಪುರುಷರೂ– ಯೆಹೂದ್ಯನಾಗಲಿ ಅಲ್ಲದಿರಲಿ,\break ಶ‍್ರೀಮಂತನಾಗಲಿ, ದೀನನಾಗಲಿ, ಪಾಪಿಯಾಗಲಿ, ಪುಣ್ಯಾತ್ಮನಾಗಲಿ–ಎಲ್ಲರೂ ತನ್ನಂತೆಯೆ ಚ್ಯುತಿ ಇಲ್ಲದ ಆತ್ಮದ ಆವಿರ್ಭಾವ ಎಂಬುದನ್ನು ಅರಿತಿದ್ದನು. ಅವನ ಜೀವನದ ಏಕಮಾತ್ರ ಉದ್ದೇಶವೆ ಇತರರೂ ತಮ್ಮ ಆಧ್ಯಾತ್ಮಿಕ ಸ್ವಭಾವವನ್ನು ತಿಳಿದುಕೊಳ್ಳುವಂತೆ ಮಾಡುವುದು. ನೀವು ದೀನರು ದರಿದ್ರರು ಎಂಬ ಕೆಲಸಕ್ಕೆ ಬಾರದ ಸ್ವಪ್ನವನ್ನು ತ್ಯಜಿಸಿ ಎನ್ನುವನು. ಇತರರು ನಿಮ್ಮನ್ನು ಗುಲಾಮರೆಂದು ತುಳಿಯುತ್ತಿರುವರು, ನಿಮ್ಮ ಅಸುವನ್ನು ಹೀರುತ್ತಿರುವರು ಎಂದು ಭಾವಿಸಬೇಡಿ. ನಿಮ್ಮಲ್ಲಿರುವುದನ್ನು ಯಾವುದೂ ತುಳಿಯಲಾರದು, ಯಾವುದೂ ನಾಶಮಾಡಲಾರದು, ಯಾವುದೂ ಅದಕ್ಕೆ ತೊಂದರೆ ಕೊಡಲಾರದು. ನೀವೆಲ್ಲ ಭಗವಂತನ ಮಕ್ಕಳು, ಅಮೃತಾತ್ಮರು. “ಪರಂಧಾಮ ನಿಮ್ಮಲ್ಲಿಯೆ ಇದೆ” ಎನ್ನುವನು, ಧೈರ್ಯವಾಗಿ ಎದ್ದು ನಿಂತು “ನಾನು ಭಗವಂತನ ಸಂತಾನ ಮಾತ್ರವಲ್ಲ; ನಾನು ಮತ್ತು ನನ್ನ ತಂದೆ ಇಬ್ಬರೂ ಒಂದೆ” ಎಂದು ಹೇಳಿ. ನಜರೆತ್ತಿನ ಜೀಸಸ್​ ಸಾರಿದ್ದು ಇದನ್ನು. ಈ ಜಗತ್ತನ್ನು ಮತ್ತು ಈ ಜೀವನವನ್ನು ಕುರಿತು ಮಾತನಾಡುವುದೆ ಇಲ್ಲ. ಜಗತ್ತಿನಿಂದ ಏನನ್ನೂ ನಿರೀಕ್ಷಿಸುವುದಿಲ್ಲ. ಅದನ್ನು ಈಗಿರುವ ಸ್ಥಿತಿಯಿಂದ ತೆಗೆದುಕೊಂಡು ಮುಂದೆ ತಳ್ಳಿ. ಪೃಥ್ವಿಯೆಲ್ಲ ಆ ಭಗವಂತನ ಜ್ಯೋತಿಯಿಂದ ಆಚ್ಛಾದಿತವಾಗಿ, ಪ್ರತಿಯೊಬ್ಬರೂ ತಮ್ಮ ಆಧ್ಯಾತ್ಮಿಕ ಸ್ವಭಾವವನ್ನು ಅರಿತು, ಮರಣ ಶೋಕಗಳು ಸಂಪೂರ್ಣ ಪ್ರಪಂಚದಿಂದ ಕಣ್ಮರೆಯಾಗುವವರೆಗೆ ಅವನು ಅದನ್ನು ಮುಂದುಮುಂದಕ್ಕೆ ನೂಕುವನು.

ಅವನ ವಿಷಯವಾಗಿ ಬಂದಿರುವ ಹಲವು ಕಥೆಗಳನ್ನು ನಾವು ಓದಿರುವೆವು. ಪಂಡಿತರು, ಅವರ ಬರಹ, ಅವರ ಉತ್ತಮ ವಿಮರ್ಶೆಗಳು ಮತ್ತು ಅಧ್ಯಯನ ಇವುಗಳಿಂದ ಆಗುವುದೆಲ್ಲವನ್ನೂ ನಾವು ತಿಳಿದುಕೊಂಡಿರುವೆವು. ನ್ಯೂ ಟೆಸ್ಟಮೆಂಟಿನಲ್ಲಿ ಇರುವುದು ಎಷ್ಟು ಸತ್ಯ. ಅವನ ಜೀವನದಲ್ಲಿ ಚಾರಿತ್ರಿಕ ದೃಷ್ಟಿಯಿಂದ ಎಷ್ಟು ಸರಿ ಎಂಬುದನ್ನು ನಾವೀಗ ವಿಮರ್ಶಿಸುವುದಿಲ್ಲ. ಕ್ರಿಸ್ತ ಹುಟ್ಟಿದ ಐದುನೂರು ವರುಷಗಳ ಅನಂತರ ನ್ಯೂಟೆಸ್ಟಮೆಂಟನ್ನು ಬರೆದರೆ, ಅಥವಾ ಅವನ ಜೀವನದ ಘಟನೆಗಳಲ್ಲಿ ಎಷ್ಟು ಸತ್ಯ ಎಂಬ ಈ ಅಂಶಗಳಾವುವೂ ಬೇಕಿಲ್ಲ. ಆದರೆ ಇವುಗಳ ಹಿಂದೆಲ್ಲ ನಾವು ಅನುಸರಿಸಬಹುದಾದುದೊಂದು ಇದೆ. ನಾವು ಸುಳ್ಳನ್ನು ಹೇಳಬೇಕಾದರೆ ಸತ್ಯವನ್ನು ಒಂದನ್ನು ಅನುಕರಿಸಬೇಕು. ವಸ್ತು ಸತ್ಯ. ಎಂದಿಗೂ ಇಲ್ಲದ ವಸ್ತುವನ್ನು ಯಾರೂ ಅನುಕರಿಸಲಾರರು. ಪ್ರಪಂಚದಲ್ಲಿ ಅವತರಿಸಿದ ಒಂದು ಅದ್ಭುತ ಆಧ್ಯಾತ್ಮಿಕ ಶಕ್ತಿಯ ಆವಿರ್ಭಾವದ ಒಂದು ಕೇಂದ್ರ ಇದ್ದೇ ಇರಬೇಕಾಗಿತ್ತು. ನಾವೀಗ ಆ ವಿಷಯವನ್ನು ಕುರಿತು ಮಾತನಾಡುತ್ತಿರುವೆವು. ಅದು ಅಲ್ಲಿರುವುದು. ಆದುದರಿಂದ ನಾವು ಪಂಡಿತರ ವಿಮರ್ಶೆಗೆ ಅಂಜಬೇಕಾಗಿಲ್ಲ. ನಾನು ಪ್ರಾಚ್ಯನಂತೆ ಏಸುಕ್ರಿಸ್ತನನ್ನು ಆರಾಧಿಸಬೇಕಾದರೆ ನನಗೆ ಇರುವುದೊಂದೆ ಮಾರ್ಗ. ಅದೇ ಅವನನ್ನು ದೇವರಂತೆ ಮಾತ್ರ ಪೂಜಿಸುವುದು. ಆ ರೀತಿ ಅವನನ್ನು ಪೂಜಿಸಲು ಅಧಿಕಾರವಿಲ್ಲ ಎಂದು ಹೇಳುತ್ತೀರೇನು? ನಮ್ಮ ಮಟ್ಟಕ್ಕೆ ಅವನನ್ನು ಎಳೆತಂದು ಅವನೊಬ್ಬ ಮಹಾಪುರುಷನೆಂದು ಸ್ವಲ್ಪ ಗೌರವ ತೋರಿದರೆ ಸಾಲದೆ? ಏತಕ್ಕೆ ಪೂಜಿಸಬೇಕು? ನಮ್ಮ ಶಾಸ್ತ್ರ ಹೀಗೆ ಸಾರುವುದು: “ಆ ಪರಂಜ್ಯೋತಿಯ ಸಂತಾನರು ತಾವೇ ಆ ಜ್ಯೋತಿಯನ್ನು ವ್ಯಕ್ತಗೊಳಿಸುತ್ತಿರುವರು, ತಾವೇ ಆ ಜ್ಯೋತಿಯಾಗಿರುವರು. ನಾವು ಅವರನ್ನು ಆರಾಧಿಸಿದರೆ ಅವರು ನಮ್ಮೊಡನೆ ಐಕ್ಯವಾಗುವರು. ನಾವು ಅವರೊಡನೆ ಐಕ್ಯರಾಗುವೆವು.”

ಮಾನವನು ದೇವರನ್ನು ಮೂರು ರೀತಿ ಭಾವಿಸುವನೆಂಬುದನ್ನು ನೀವು ನೋಡುವಿರಿ. ಮೊದಲು ಅವಿದ್ಯಾವಂತನಾದ ಅಪ್ರಬುದ್ಧ ಬುದ್ಧಿಯವನು ದೇವರು ಎಲ್ಲೋ ದೂರದ ಸ್ವರ್ಗದಲ್ಲಿ ದೊಡ್ಡ ನ್ಯಾಯಾಧಿಪತಿಯಂತೆ ಸಿಂಹಾಸನದ ಮೇಲೆ ಕುಳಿತಿರುವನೆಂದು ಭಾವಿಸುವನು. ಅವನನ್ನು ಬೆಂಕಿಯೆಂದು ಭಾವಿಸುವನು; ಭಯಾನಕ ವ್ಯಕ್ತಿ ಅವನು ಎಂದು ಭಾವಿಸುವನು. ಅದೇನೊ ಒಳ್ಳೆಯದೆ, ಅದರಲ್ಲಿ ಏನೂ ಕೆಟ್ಟುದಿಲ್ಲ. ಮಾನವಕೋಟಿ ದೋಷದಿಂದ ಸತ್ಯದೆಡೆಗೆ ಹೋಗುತ್ತಿಲ್ಲ, ಅದು ಸತ್ಯದಿಂದ ಮತ್ತೊಂದು ಸತ್ಯದೆಡೆಗೆ ಹೋಗುತ್ತಿದೆ, ಅಥವಾ ನೀವು ಬೇಕಾದರೆ ಸಣ್ಣ ಸತ್ಯದಿಂದ ದೊಡ್ಡ ಸತ್ಯದೆಡೆಗೆ ಹೋಗುತ್ತಿರುವುದು ಎಂದು ವಿಶದವಾಗಿ ಹೇಳಬಹುದು. ಆದರೆ ಎಂದಿಗೂ ದೋಷದಿಂದ ಸತ್ಯಕ್ಕಲ್ಲ. ನೀವು ಇಲ್ಲಿಂದ ನೇರವಾಗಿ ಸೂರ್ಯನೆಡೆಗೆ ಹೊರಡುತ್ತೀರಿ ಎಂದು ಭಾವಿಸೋಣ. ಇಲ್ಲಿಗೆ ಸೂರ್ಯ ಸಣ್ಣದಾಗಿ ಕಾಣುವನು. ನೀವು ಸುಮಾರು ಕೋಟಿ ಮೈಲುಗಳಷ್ಟು ದೂರ ಹೋದರೆ ಅವನು ಬಹು ದೊಡ್ಡವನಾಗಿ ಕಾಣುವನು. ಹಂತಹಂತಕ್ಕೂ ಅವನು ದೊಡ್ಡದಾಗುತ್ತಾ ಬರುವನು. ಸೂರ್ಯನ ಇಪ್ಪತ್ತು ಸಾವಿರ ಛಾಯಾಚಿತ್ರಗಳನ್ನು ಬೇರೆ ಬೇರೆ ಕಡೆಗಳಿಂದ ತೆಗೆದು\-ಕೊಂಡರು ಎಂದು ಭಾವಿಸೋಣ. ಈ ಛಾಯಾಚಿತ್ರಗಳಲ್ಲಿ ವ್ಯತ್ಯಾಸಗಳಿರುವುದು ಸತ್ಯ. ಆದರೆ ಪ್ರತಿಯೊಂದು ಒಬ್ಬನೇ ಸೂರ್ಯನ ಛಾಯಾಚಿತ್ರ ಎಂಬುದನ್ನು ನೀವು ಅಲ್ಲಗಳೆಯು\-ವಿರಾ? ಇದರಂತೆಯೇ ಎಲ್ಲಾ ಸಣ್ಣ ದೊಡ್ಡ ಧರ್ಮಗಳೂ ಕೂಡ ಸ್ವಯಂಪ್ರಕಾಶಮಾನನಾದ ದೇವರ ಬೇರೆ ಬೇರೆ ಕಡೆಗಳಿಂದ ತೆಗೆದ ಚಿತ್ರಗಳಷ್ಟೆ. ಆದಕಾರಣ ಜಗತ್ತಿನಲ್ಲೆಲ್ಲಾ ಅಪ್ರಬುದ್ಧ ಜನಸಾಮಾನ್ಯರ ಧರ್ಮದಲ್ಲೆಲ್ಲಾ, ದೇವರು ಪ್ರಕೃತಿಯ ಹೊರಗೆ ಇರುವನು, ಸ್ವರ್ಗದಲ್ಲಿ ವಿಹರಿಸುವನು, ಅವನು ಅಲ್ಲಿಂದ ಆಳುವನು, ಆತ ದುಷ್ಟ ಶಿಕ್ಷಕ ಮತ್ತು ಶಿಷ್ಟ ರಕ್ಷಕ ಇಂತಹ ಭಾವನೆ ಯಾವಾಗಲೂ ಎಲ್ಲಾ ಕಡೆಗಳಲ್ಲಿಯೂ ಇರಲೇಬೇಕಾಗುವುದು.\break ಮಾನವನು ಆಧ್ಯಾತ್ಮಿಕವಾಗಿ ಮುಂದುವರಿದಂತೆ ದೇವರು ಸರ್ವವ್ಯಾಪಿ, ಅವನು ತನ್ನ\-ಲ್ಲಿಯೂ ಇರಬೇಕು, ಅವನೇ ಸರ್ವಾಂತರಾತ್ಮನು ಎಂದು ಭಾವಿಸುವನು. ನನ್ನ ಜೀವ ಹೇಗೆ ನನ್ನ ದೇಹವನ್ನು ಚಲಿಸುವಂತೆ ಮಾಡುವುದೊ, ಹಾಗೆಯೇ ದೇವರು ನನ್ನ ಜೀವವನ್ನು ಆಡಿಸುತ್ತಿರುವನು. ಅವನು ಅಂತರಾತ್ಮ. ತುಂಬಾ ಮುಂದುವರಿದ ಕೆಲವರು, ಪರಿಶುದ್ಧಾತ್ಮರಾದ ಕೆಲವರು, ಇನ್ನೂ ಮುಂದೆ ಹೋಗಿ ದೇವರನ್ನು ಕಂಡರು. ನ್ಯೂ ಟೆಸ್ಟಮೆಂಟ್​ ಹೇಳುವಂತೆ “ಪರಿಶುದ್ಧಾತ್ಮರೆ ಧನ್ಯರು. ಅವರೇ ದೇವರನ್ನು ನೋಡುವರು.” ಕೊನೆಗೆ ಅವರಿಗೆ ತಾವು ಮತ್ತು ತಂದೆ ಇಬ್ಬರೂ ಒಂದೇ ಎಂದು ಗೊತ್ತಾಯಿತು.

ಈ ಮೂರು ಸ್ಥಿತಿಗಳನ್ನೂ ಆ ಮಹಾಬೋಧಕ ನ್ಯೂ ಟೆಸ್ಟಮೆಂಟಿನಲ್ಲಿ ಸಾರಿರುವನು. “ಸ್ವರ್ಗದಲ್ಲಿರುವ ತಂದೆಯೆ, ಧನ್ಯವಾಗಲಿ ನಿನ್ನ ನಾಮ” ಎಂಬ ಸಾಮಾನ್ಯ ಪ್ರಾರ್ಥನೆಯನ್ನು ಗಮನಿಸಿ. ಇದು ಸಾಮಾನ್ಯ ಪ್ರಾರ್ಥನೆ. ಎಳೆ ಹಸುಳೆಯ ಪ್ರಾರ್ಥನೆ. ಇದು ಸಾಮಾನ್ಯ ಪ್ರಾರ್ಥನೆ, ಏಕೆಂದರೆ ಅವಿದ್ಯಾವಂತರಿಗೆ ಬೋಧಿಸಿದ್ದು ಎಂಬುದನ್ನು ಗಮನಿಸಿ. ಮುಂದುವರಿದ ಸ್ವಲ್ಪ ವಿದ್ಯಾವಂತರಿಗೆ “ನಾನು ತಂದೆಯಲ್ಲಿರುವೆನು. ನೀವು ನನ್ನಲ್ಲಿರುವಿರಿ, ನಾನು ನಿಮ್ಮಲ್ಲಿರುವೆನು” ಎಂಬ ಸ್ವಲ್ಪ ಉತ್ತಮ ಬೋಧನೆಯನ್ನು ಕೊಡುವನು. ಇದು ನಿಮಗೆ ಜ್ಞಾಪಕವಿದೆಯೆ? ನೀನು ಯಾರೆಂದು ಯಹೂದ್ಯರು ಅವನನ್ನು ಕೇಳಿದಾಗ, ನಾನು ನನ್ನ ತಂದೆ ಇಬ್ಬರೂ ಒಂದೇ ಎಂದನು. ಯಹೂದ್ಯರು ಇವನೊಬ್ಬ ಈಶ್ವರನಿಂದಕನೆಂದು ಭಾವಿಸಿದರು. ಕ್ರಿಸ್ತ ಹೀಗೆಂದುದರ ಅರ್ಥವೇನು? ನಿಮ್ಮ ಪುರಾತನ ದೇವದೂತರು ಕೂಡ “ನೀವೆಲ್ಲ ದೇವತೆಗಳು, ಆ ಪವಿತ್ರಾತ್ಮನ ಮಕ್ಕಳು” ಎಂದು ಹೇಳಿರುವರು. ಈ ಮೂರು ಮೆಟ್ಟಿಲುಗಳನ್ನೂ ನೀವು ಗಮನಿಸಿ, ಮೊದಲಿನ ಮೆಟ್ಟಲಿನಿಂದ ಪ್ರಾರಂಭಿಸಿ ಕೊನೆ ಮೆಟ್ಟಿಲನ್ನು ಮುಟ್ಟುವುದು ಸುಲಭವೆಂದು ನಿಮಗೆ ಅರಿವಾಗುವುದು.

ದೇವದೂತನು ಮಾರ್ಗದರ್ಶಿಯಾಗಲು ಬಂದನು. ಅಧ್ಯಾತ್ಮವು ಬಾಹ್ಯ ಆಚರಣೆಯಲ್ಲಿ ಇಲ್ಲ. ತತ್ತ್ವಶಾಸ್ತ್ರದ ಜಟಿಲ ಸಮಸ್ಯೆಗಳಿಂದ ನೀವು ಅಧ್ಯಾತ್ಮವನ್ನು ಅರಿಯಲಾರಿರಿ. ನಿಮಗೆ ಪಾಂಡಿತ್ಯವಿಲ್ಲದೆ ಇದ್ದರೆ ಮೇಲು. ನೀವು ಒಂದು ಗ್ರಂಥವನ್ನೂ ಓದದೆ ಇದ್ದರೆ ಮೇಲು, ಐಶ್ವರ್ಯ, ಅಂತಸ್ತು, ಅಧಿಕಾರ ಇವಾವುವೂ ಮುಕ್ತಿಗೆ ಆವಶ್ಯಕವಲ್ಲ. ಆವಶ್ಯಕ\-ವಾಗಿರುವುದು ಹೃದಯ ನೈರ್ಮಲ್ಯ, “ಪರಿಶುದ್ಧಾ ತ್ಮರೆ ಧನ್ಯರು.” ಆತ್ಮ ಸ್ವಭಾವತಃ ಪರಿ\-ಶುದ್ಧವಾದುದು. ಅದು ಮತ್ತೇನು ಆಗಿರಬೇಕು! ಅದು ಭಗವಂತನದು, ಭಗವಂತನಿಂದ ಬಂದಿದೆ. ಬೈಬಲ್ಲಿನ ಭಾಷೆಯಲ್ಲಿ ಇದೇ ಭಗವಂತನ ಉಸಿರು. ಖುರಾನಿನ ಭಾಷೆಯಲ್ಲಿ ಇದೇ ದೇವರ ಆತ್ಮ. ಭಗವಂತನ ಆತ್ಮ ಎಂದಾದರೂ ಅಶುದ್ಧವಾಗುವುದೆಂದು ಭಾವಿಸುವಿ\-ರೇನು? ಆದರೆ ಅದು ನಮ್ಮ ಒಳ್ಳೆಯ ಮತ್ತು ಕೆಟ್ಟ ಕರ್ಮಗಳೆಂಬ ಧೂಳು, ಕೊಳೆಗಳಿಂದ ಆವೃತ\-ವಾದಂತೆ ಇರುವುದು. ಹಲವು ಅಸತ್ಯವಾದ ಅಧರ್ಮ ಕರ್ಮಗಳ ಧೂಳು ಜೀವಿಯನ್ನು ಆವರಿಸಿದಂತೆ ಇರುವುದು. ಧೂಳು ಕೊಳೆಗಳನ್ನು ಶುದ್ಧಮಾಡಿದರೆ ಆತ್ಮ ತಕ್ಷಣವೇ ಹೊಳೆಯುವುದು. “ಪರಿಶುದ್ಧ ಹೃದಯರೆ ಧನ್ಯರು. ಅವರೇ ದೇವರನ್ನು ನೋಡುವವರು.” “ಭಗವಂತನ ಸಾಮ್ರಾಜ್ಯವು ನಿಮ್ಮೊಳಗೇ ಇದೆ.” ಭಗವಂತನ ಇರವು ನಿಮ್ಮಲ್ಲಿಯೇ ಇರುವಾಗ ಅವನನ್ನು ಮತ್ತೆಲ್ಲಿ ಅರಸುತ್ತೀರಿ? ಎಂದು ನಜರೆತ್ತಿನ ಜೀಸಸ್ಸನು ಕೇಳುತ್ತಾನೆ. ಹೃದಯವನ್ನು ಶುದ್ಧಿ ಮಾಡಿ. ಅದಲ್ಲೇ ಇರುವುದು, ಆದಾಗಲೇ ನಿಮ್ಮದು. ನಿಮ್ಮದಲ್ಲದುದನ್ನು ನೀವು ಪಡೆಯುವುದಾದರೂ ಹೇಗೆ? ನೀವು ಅಮೃತತ್ವಕ್ಕೆ ಹಕ್ಕು\-ದಾರರು, ಅನಂತಾತ್ಮನ ಮಕ್ಕಳು.

ದೇವದೂತನ ಮಹಾಬೋಧನೆ ಇದು. ಮತ್ತೊಂದು, ಎಲ್ಲಾ ಧರ್ಮಗಳ ತಳಹದಿಯಂತಿರುವುದೇ ತ್ಯಾಗ. ನೀವು ಆತ್ಮನನ್ನು ಹೇಗೆ ಶುದ್ಧಮಾಡುವಿರಿ? ತ್ಯಾಗದಿಂದ. ಒಬ್ಬ ಶ‍್ರೀಮಂತ ತರುಣ ಕ್ರಿಸ್ತನನ್ನು “ಗುರುದೇವ, ನಾನು ಭಗವಂತನನ್ನು ಸೇರಬೇಕಾದರೆ ಏನು ಮಾಡಬೇಕು?” ಎಂದು ಕೇಳಿದನು. ಅದಕ್ಕೆ ಕ್ರಿಸ್ತನು “ನೀನೊಂದನ್ನು ಮಾಡಿಲ್ಲ. ಹೋಗಿ, ನಿನ್ನಲ್ಲಿರುವುದನ್ನೆಲ್ಲ ಮಾರು. ಅದನ್ನು ಬಡವರಿಗೆ ಮಾರು. ಅದನ್ನು ಬಡವರಿಗೆ ಕೊಡು. ನಿನಗೆ ಸ್ವರ್ಗದಲ್ಲಿ ಐಶ್ವರ್ಯ ದೊರಕುವುದು. ಒಂದು ಶಿಲುಬೆಯನ್ನು ಹೊತ್ತು ನನ್ನನ್ನು ಅನುಸರಿಸು” ಎಂದನು. ಇದನ್ನು ಕೇಳಿ ಆ ತರುಣನಿಗೆ ವ್ಯಥೆ\-ಯಾಯಿತು. ಅವನು ಹೊರಟುಹೋದನು. ಆತನಿಗೆ ಬಹಳ ಸಂಪತ್ತು ಇತ್ತು. ನಾವೆಲ್ಲ ಹೆಚ್ಚು ಕಡಿಮೆ ಹಾಗೆಯೆ. ಹಗಲು ರಾತ್ರಿ ಆ ಧ್ವನಿ ನಮ್ಮ ಕಿವಿಯಲ್ಲಿ ಅನುರಣಿತವಾಗುತ್ತಿರುತ್ತದೆ. ನಮ್ಮ ಸುಖ ಭೋಗಗಳ ಮಧ್ಯೆ ನಾವು ಪ್ರಪಂಚದಲ್ಲಿ ನಿರತರಾಗಿದ್ದಾಗ ನಾವೆಲ್ಲವನ್ನೂ ಮರೆತಂತೆ ತೋರುವುದು. ಸ್ವಲ್ಪ ಹೊತ್ತು ನಿಂತು ಪುನಃ ಆ ಧ್ವನಿ ಕೇಳಿಸುವುದು. “ನಿನ್ನಲ್ಲಿರುವುದನ್ನೆಲ್ಲಾ ತ್ಯಜಿಸಿ ನನ್ನನ್ನು ಅನುಸರಿಸು.” “ಯಾರು ತಮ್ಮ ಪ್ರಾಣವನ್ನು ಸಂರಕ್ಷಿಸಿಕೊಳ್ಳುವರೊ ಅವರು ಅದನ್ನು ಕಳೆದುಕೊಳ್ಳುವರು. ಯಾರು ನನಗಾಗಿ ತಮ್ಮ ಪ್ರಾಣವನ್ನು ಕಳೆದುಕೊಳ್ಳುವರೊ ಅವರು ಅದನ್ನು ಪಡೆಯುವರು.” ಭಗವಂತನಿಗಾಗಿ ಯಾರು ತಮ್ಮ ಪ್ರಾಣವನ್ನು ಕಳೆದು\-ಕೊಳ್ಳುವರೊ ಅವರು ಅನಂತಾತ್ಮನನ್ನು ಪಡೆಯುವರು. ನಮ್ಮ ದುರ್ಬಲತೆಯ ಮಧ್ಯೆ ಒಂದು ಕ್ಷಣ ಸ್ತಬ್ಧತೆ ಬರುವುದು. ಆಗ “ನಿಮ್ಮಲ್ಲಿರುವುದನ್ನೆಲ್ಲ ತ್ಯಜಿಸಿ ದೀನರಿಗೆ ದಾನ ಮಾಡಿ ನನ್ನನ್ನು ಅನುಸರಿಸಿ” ಎಂಬ ವಾಣಿಯನ್ನು ಕೇಳುವೆವು. ಅವನು ಬೋಧಿಸುವ ಆದರ್ಶ ಇದೊಂದೆ. ಜಗತ್ತಿನ ಎಲ್ಲಾ ಮಹಾತ್ಮರೂ ಸಾರಿದ ಬೋಧನೆಯೆ ಇದು–ತ್ಯಾಗ. ತ್ಯಾಗವೆಂದರೇನು? ನೀತಿಯಲ್ಲಿ ಏಕಮಾತ್ರ ಆದರ್ಶವಿರುವುದು–ಅದೇ ನಿಃಸ್ವಾರ್ಥನಾಗು. ಪೂರ್ಣ ನಿಃಸ್ವಾರ್ಥವೆ ಆದರ್ಶ. ಬಲಗೆನ್ನೆಗೆ ಹೊಡೆದರೆ ಎಡಗೆನ್ನೆಯನ್ನು ತೋರುವನು. ಅವನ ಕೋಟನ್ನು ಕಸಿದುಕೊಂಡರೆ ಮೇಲಂಗಿಯನ್ನು ಕಳಚಿ ಕೊಡುವನು.

ನಾವು ಆದರ್ಶವನ್ನು ಕೆಳಕ್ಕೆ ಎಳೆಯದೆ ಸಾಧ್ಯವಾದಷ್ಟು ಅನುಷ್ಠಾನಕ್ಕೆ ತರಲು ಯತ್ನಿಸಬೇಕು. ಯಾವ ಮನುಷ್ಯನಲ್ಲಿ ಸ್ವಾರ್ಥ ಇಲ್ಲವೊ, ಆಸ್ತಿಪಾಸ್ತಿಗಳಿಲ್ಲವೊ, ನಾನು ನನ್ನದು ಎಂಬುದಿಲ್ಲವೊ, ಯಾರು ಸಂಪೂರ್ಣವಾಗಿ ತನ್ನನ್ನು ತೆತ್ತಿರುವನೊ, ತನ್ನ ಅಹಂಕಾರವನ್ನು ಸಂಪೂರ್ಣ ನಿರ್ನಾಮ ಮಾಡಿಕೊಂಡಿರುವನೊ, ಅಂತಹವನಲ್ಲಿ ದೇವರೇ ಇರುವನು. ಏಕೆಂದರೆ ಅವನಲ್ಲಿ ಸ್ವಾರ್ಥ, ಆಸಕ್ತಿ ಇಲ್ಲ. ಅವು ಪೂರ್ಣ ನಾಶವಾಗಿವೆ, ನಿರ್ನಾಮ\-ವಾಗಿವೆ. ಅವನು ಆದರ್ಶ ವ್ಯಕ್ತಿ. ಆ ಸ್ಥಿತಿಯನ್ನು ನಾವಿನ್ನೂ ಪಡೆದಿಲ್ಲ. ಆದರೂ ಆದರ್ಶವನ್ನು ಆರಾಧಿಸೋಣ. ನಿಧಾನವಾಗಿ, ತಪ್ಪು ಹೆಜ್ಜೆಗಳನ್ನಿಟ್ಟಾದರೂ ಅದನ್ನು ನೋಡೋಣ. ಅದು ನಾಳೆಯೆ ಬರಬಹುದು ಅಥವಾ ಸಾವಿರ ವರುಷ ಆದ ಮೇಲೆ ಬರಬಹುದು. ಆದರೆ ನಾವು ಗುರಿಯನ್ನು ಒಂದಲ್ಲ ಒಂದು ದಿನ ಸೇರಲೇಬೇಕು. ಅದು ಗುರಿ ಮಾತ್ರವಲ್ಲ, ದಾರಿಯೂ ಹೌದು. ನಿಃಸ್ವಾರ್ಥನಾಗುವುದು, ಸಂಪೂರ್ಣ ನಿಃಸ್ವಾರ್ಥನಾಗುವುದೆಂದರೆ ಅದೇ ಮುಕ್ತಿ. ಒಳಗಿರುವ ಮಾನವ ಅಳಿದು ದೇವನೊಬ್ಬನೇ ಉಳಿಯುವನು.

ಮತ್ತೊಂದು ವಿಷಯ. ಜಗತ್ತಿನ ಗುರುಗಳೆಲ್ಲ ನಿಃಸ್ವಾರ್ಥಿಗಳು. ನಜರೆತ್ತಿನ ಜೀಸಸ್​ ಬೋಧಿಸುತ್ತಿದ್ದ ಎಂದು ಭಾವಿಸಿ. ಒಬ್ಬ ಮನುಷ್ಯ ಅವನ ಬಳಿಗೆ ಬಂದು “ನೀನು ಹೇಳುವುದು ಸುಂದರವಾಗಿದೆ. ಪೂರ್ಣತೆಗೆ ಇದೇ ದಾರಿ ಎಂದು ನಂಬುತ್ತೇನೆ. ನಾನು ಇದನ್ನು ಅನುಸರಿಸಲು ಸಿದ್ಧನಾಗಿರುವೆನು. ಆದರೆ ನಾನು ನೀನೊಬ್ಬನೇ ದೇವರ ಮಗ ಎಂದು ಪೂಜಿಸಲಾರೆ” ಎನ್ನುವನು ಎಂದು ಭಾವಿಸೋಣ. ಇದಕ್ಕೆ ಕ್ರಿಸ್ತನ ಉತ್ತರ ಏನಿರಬಹುದೆಂದು ನೀವು ಊಹಿಸುವಿರಿ? “ಒಳ್ಳೆಯದು ಸಹೋದರನೇ, ಆದರ್ಶವನ್ನು ಅನುಸರಿಸು,\break ನಿನ್ನ ಮಾರ್ಗದಲ್ಲಿಯೆ ಮುಂದುವರಿ. ನೀನು ನನ್ನ ಬೋಧನೆಯನ್ನು ಗೌರವಿಸುವೆಯೊ ಇಲ್ಲವೊ ಎಂಬುದನ್ನು ಲೆಕ್ಕಿಸುವುದಿಲ್ಲ. ನಾನು ವರ್ತಕನಲ್ಲ, ಧರ್ಮವನ್ನು ವಿಕ್ರಯ ಮಾಡುವುದಿಲ್ಲ. ನಾನು ಕೇವಲ ಸತ್ಯವನ್ನು ಮಾತ್ರ ಬೋಧಿಸುತ್ತೇನೆ. ಅದು ಯಾರ ಆಸ್ತಿಯೂ ಅಲ್ಲ. ಇದು ನಮ್ಮದು ಎಂದು ಯಾರೂ ಹೇಳಿಕೊಳ್ಳಲಾರರು. ಸತ್ಯವೇ ದೇವರು, ಮುಂದುವರಿ.” ಆದರೆ ಈಗ ಅವನ ಶಿಷ್ಯರು ಏನು ಹೇಳುತ್ತಾರೆ ಎಂದರೆ: “ನೀನು ಬೋಧನೆಯನ್ನು ಅನುಷ್ಠಾನಕ್ಕೆ ತರುವೆಯೊ ಇಲ್ಲವೊ ಚಿಂತೆಯಿಲ್ಲ. ನೀನು ಕ್ರಿಸ್ತನಿಗೆ ಗೌರವ ತೋರಿಸುವೆಯಾ? ಅವನಿಗೆ ಗೌರವ ತೋರಿದರೆ ನೀನು ಉದ್ಧಾರವಾಗುವೆ. ಇಲ್ಲದೆ ಇದ್ದರೆ ಉದ್ಧಾರವಾಗಲಾರೆ.” ಗುರುದೇವನ ಸಂದೇಶ ಹೀಗೆ ಅಧೋಗತಿಗೆ ಬಂದಿದೆ. ಈಗಿರುವ ಕಚ್ಚಾಟವೆಲ್ಲ ಆ ವ್ಯಕ್ತಿಯ ವ್ಯಕ್ತಿತ್ವಕ್ಕಾಗಿ ಮಾತ್ರ. ಇಂತಹ ವ್ಯತ್ಯಾಸವನ್ನು ತರುವುದರಿಂದ ಯಾರನ್ನು ಗೌರವಿಸಬೇಕೆಂದಿರುವರೋ ಅವರಿಗೆ ಅಗೌರವವನ್ನು ತರುತ್ತಿರುವರು. ಇಂತಹ ಭಾವನೆಯಿಂದ ಕ್ರಿಸ್ತನೆ ನಾಚಿಕೆಯಿಂದ ಹಿಂದೆ ಸರಿಯುತ್ತಿದ್ದ ಎಂಬುದನ್ನು ಅವರು ಅರಿಯರು. ಜಗತ್ತಿನಲ್ಲಿ ತನ್ನನ್ನು ಒಂದು ವ್ಯಕ್ತಿಯಾದರೂ ಜ್ಞಾಪಕದಲ್ಲಿಟ್ಟಿರಲಿ ಅಥವಾ ಬಿಡಲಿ, ಅವನು ಅದನ್ನು ಲೆಕ್ಕಿಸುತ್ತಿದ್ದನೇನು? ಅವನು ಆ ಸಂದೇಶವನ್ನು ಸಾರಬೇಕಾಗಿತ್ತು, ಸಾರಿದನು. ಅವನಿಗೆ ಇಪ್ಪತ್ತು ಸಾವಿರ ಜನ್ಮಗಳಿದ್ದರೂ ಜಗತ್ತಿನ ಅತಿ ದರಿದ್ರನಿಗಾಗಿ ಅದನ್ನು ಧಾರೆಯೆರೆಯುತ್ತಿದ್ದನು. ಕೋಟ್ಯಂತರ ಅವಹೇಳನಗಳಿಗೆ ಗುರಿಯಾದ ಜನರಿಗಾಗಿ ಕೋಟ್ಯಂತರ ವೇಳೆ ಅವನು ಹಿಂಸೆಗೆ ಈಡಾಗಬೇಕಾದರೂ, ಅವರಲ್ಲಿ ಪ್ರತಿಯೊಬ್ಬರ ಮೋಕ್ಷಕ್ಕೂ ತಾನು ಪ್ರಾಯಶ್ಚಿತ್ತಾರ್ಥವಾಗಿ ಹಿಂಸೆಯನ್ನು ಅನುಭವಿಸಬೇಕಾದರೂ ತನ್ನ ಪ್ರಾಣವನ್ನು ಕೊಡುತ್ತಿದ್ದನು. ಮತ್ತೊಂದು ವ್ಯಕ್ತಿಗೆ ತನ್ನ ಹೆಸರು ತಿಳಿಯಲಿ ಎಂಬ ಭಾವವಿಲ್ಲದೆ ಈ ಮಹಾತ್ಯಾಗವನ್ನು ಮಾಡುತ್ತಿದ್ದನು. ಮೌನವಾಗಿ ಮತ್ತೊಬ್ಬರಿಗೆ ತಿಳಿಯದೆ ಶಾಂತಿಯಿಂದ ದೇವರು ಕೆಲಸ ಮಾಡುವಂತೆ ಇದನ್ನು ಮಾಡುತ್ತಿದ್ದನು. ಈಗ ಅವನ ಶಿಷ್ಯರು ಏನು ಹೇಳುತ್ತಾರೆ– “ನೀನು ಪರಿಪೂರ್ಣನಾಗಿರಬಹುದು, ಶುದ್ಧ ನಿಃಸ್ವಾರ್ಥಿಯಾಗಿರಬಹುದು. ಆದರೂ ನಮ್ಮ ಬೋಧಕನಿಗೆ, ಮಹಾತ್ಮನಿಗೆ ಗೌರವ ತೋರದೆ ಹೋದರೆ ಪ್ರಯೋಜನವಿಲ್ಲ” ಎನ್ನುವರು. ಏತಕ್ಕೆ? ಈ ಅಜ್ಞಾನಕ್ಕೆ, ಮೂಢನಂಬಿಕೆಗೆ ಕಾರಣವೇನು? ಶಿಷ್ಯರು, ದೇವರು ಒಮ್ಮೆ ಮಾತ್ರ ವ್ಯಕ್ತನಾಗುವನು ಎಂದು ತಿಳಿಯುವರು. ಅವರ ದೋಷವೆಲ್ಲ ಇಲ್ಲಿ ಇರುವುದು. ದೇವರು ನಿಮಗೆ ಮಾನವನಂತೆ ವ್ಯಕ್ತನಾಗುವನು. ಆದರೆ ಪ್ರಕೃತಿಯಲ್ಲಿ ಯಾವುದು ಒಮ್ಮೆ ಆಗಿದೆಯೊ ಅದು ಹಿಂದೆ ಆಗಿರಬೇಕು, ಮುಂದೆಯೂ ಆಗಬೇಕು. ನಿಯಮಬದ್ಧವಾಗದೆ ಇರುವುದು ಪ್ರಕೃತಿಯಲ್ಲಿ ಇಲ್ಲ. ಅಂದರೆ ಯಾವುದು ಒಮ್ಮೆ ಆಗಿದೆಯೊ ಅದು ಹಿಂದೆಯೂ ಆಗಿದ್ದಿರಬೇಕು, ಮುಂದೆಯೂ ಆಗಬೇಕು.

ಭರತಖಂಡದಲ್ಲಿ ದೇವರ ಅವತಾರವನ್ನು ಕುರಿತು ಇದೇ ಭಾವನೆಯನ್ನು ಹೊಂದಿರುವರು. ಅವರ ಅವತಾರಶ್ರೇಷ್ಠನೊಬ್ಬ ಶ‍್ರೀಕೃಷ್ಣ. ಅವರ ಭಗವದ್ಗೀತೆಯನ್ನು ನಿಮ್ಮಲ್ಲಿ ಹಲವರು ಓದಿರಬಹುದು. ಅದರಲ್ಲಿ “ನಾನು ಅವ್ಯಕ್ತನಾದರೂ ಅಜನಾದರೂ ಸರ್ವಭೂತಗಳಿಗೆ ಒಡೆಯನಾದರೂ ನನ್ನ ಪ್ರಕೃತಿಯನ್ನು ಸ್ವಾಧೀನ ಮಾಡಿಕೊಂಡು ನನ್ನ ಮಾಯೆಯಿಂದ ಪ್ರಪಂಚಕ್ಕೆ ಬರುವೆನು. ಎಂದು ಅಧರ್ಮ ಹೆಚ್ಚುವುದೊ, ಧರ್ಮ ಕುಗ್ಗುವುದೊ ಆಗ ನಾನು ಜನ್ಮವೆತ್ತುವೆನು. ಶಿಷ್ಟರನ್ನು ಪರಿಪಾಲಿಸುವುದಕ್ಕೆ ದುಷ್ಟರನ್ನು ನಿಗ್ರಹಿಸುವುದಕ್ಕೆ ಧರ್ಮ ಸಂಸ್ಥಾಪನೆಗೆ ನಾನು ಪುನಃ ಪುನಃ ಅವತಾರವೆತ್ತುತ್ತೇನೆ” ಎಂದಿರುವನು. ಎಂದು ಪ್ರಪಂಚ ಅವನತಿಗೆ ಇಳಿಯುವುದೊ ಅಂದು ಅದನ್ನು ಉದ್ಧರಿಸಲು ಬರುವನು. ಇದನ್ನು ಕಾಲಕಾಲಕ್ಕೆ ಬೇರೆ ಬೇರೆ ದೇಶಗಳಲ್ಲಿ ಮಾಡುವನು. ಮತ್ತೊಂದು ಕಡೆ ಅವನು ಹೀಗೆ ಹೇಳುವನು; “ಎಲ್ಲಿ ಒಬ್ಬನು ಅಸಾಧ್ಯ ಶಕ್ತಿಯಿಂದ ಮತ್ತು ಪವಿತ್ರತೆಯಿಂದ ಜಗದುದ್ಧಾರಕ್ಕೆ ಪ್ರಯತ್ನಿಸುತ್ತಿರುವನೊ ಅವನು ನನ್ನ ಅಂಶದಿಂದ ಜನಿಸಿರುವನೆಂದು ತಿಳಿ. ನಾನು ಅವನ ಮೂಲಕ ಕೆಲಸ ಮಾಡುತ್ತಿರುವೆನು.”

ಆದಕಾರಣ ನಾವು ದೇವರನ್ನು ಏಸುಕ್ರಿಸ್ತನೊಬ್ಬನಲ್ಲಿ ಮಾತ್ರ ಕಾಣದೆ ಅವನಿಗಿಂತ ಮುಂಚೆ ಬಂದ ಎಲ್ಲಾ ಮಹಾತ್ಮರಲ್ಲಿಯೂ ಅವನ ನಂತರ ಬಂದವರಲ್ಲಿಯೂ ಮುಂದೆ ಬರುವವರಲ್ಲಿಯೂ ಕಾಣೋಣ. ನಮ್ಮ ಆರಾಧನೆ ಅನಿರ್ಬಂಧವಾದುದು, ಸ್ವಾಭಾವಿಕವಾ\-ದುದು. ಅವರೆಲ್ಲರೂ ಅನಂತ ಪರಮಾತ್ಮನ ಆವಿರ್ಭಾವಗಳು. ಅವರೆಲ್ಲರೂ ಪರಿಶುದ್ಧರು, ನಿಃಸ್ವಾರ್ಥಿಗಳು, ನಮ್ಮಂತಹ ದೀನ ಮಾನವರ ಉದ್ಧಾರಕ್ಕೆ ಹೋರಾಡಿ ಪ್ರಾಣವನ್ನು ಅರ್ಪಿಸಿರುವರು. ಅವರಲ್ಲಿ ಪ್ರತಿಯೊಬ್ಬರೂ, ಎಲ್ಲರೂ, ನಮ್ಮಲ್ಲಿ ಪ್ರತಿಯೊಬ್ಬರ ದೋಷಕ್ಕೂ ಮುಂದೆ ಬರುವವರ ದೋಷಕ್ಕೂ ತಾವೆ ಪ್ರಾಯಶ್ಚಿತ್ತ ಮಾಡಿಕೊಳ್ಳುವರು. ಒಂದು ದೃಷ್ಟಿಯಲ್ಲಿ ನೀವೆಲ್ಲಾ ಜಗತ್ತಿನ ಭಾರವನ್ನು ನಿಮ್ಮ ಹೆಗಲ ಮೇಲೆ ಹೊರುವ ದೇವದೂತರೆ. ತಮ್ಮ ಜೀವನದ ಭಾರವನ್ನು ತಾವೆ ಮೌನವಾಗಿ, ಸಾವಧಾನವಾಗಿ ಹೊರದ ಯಾವ ಸ್ತ್ರೀ\-ಪುರುಷರನ್ನಾಗಲೀ ನೀವು ನೋಡಿರುವಿರಾ? ಮಹಾ ದೇವದೂತರು, ಭೀಮ ವ್ಯಕ್ತಿಗಳು ಜಗ\break ತ್ತಿನ ಭಾರವನ್ನು ಹೊತ್ತರು. ಅವರೊಂದಿಗೆ ಹೋಲಿಸಿದರೆ ನಾವು ಕ್ಷುದ್ರ ವ್ಯಕ್ತಿಗಳೆಂಬುದೇನೊ ನಿಜ. ಆದರೂ ನಾವೂ ಇದೇ ಕೆಲಸವನ್ನು ನಮ್ಮ ಕಿರುವಲಯಗಳಲ್ಲಿ, ಮನೆಗಳಲ್ಲಿ ಮಾಡುತ್ತಿ\break ರುವೆವು; ನಮ್ಮ ತೊಂದರೆಗಳನ್ನು ಹೊರುತ್ತಿರುವೆವು. ಎಂತಹ ಪಾಪಿಯಾಗಲಿ ಮೂರ್ಖನಾ\break ಗಲಿ ಎಲ್ಲರೂ ತಮ್ಮ ವ್ಯಾಕುಲತೆಯನ್ನು ತಾವೇ ಹೊರಬೇಕಾಗಿದೆ. ನಾವು ಎಷ್ಟೇ ತಪ್ಪು ಮಾಡಿರಲಿ, ಹೀನ ಕಾರ್ಯಗಳನ್ನು ಮತ್ತು ಹೀನ ಆಲೋಚನೆಗಳನ್ನು ಮಾಡಿರಲಿ, ಎಲ್ಲೋ ಒಂದು ಕಡೆ ಪರಿಶುದ್ಧ ಸ್ಥಳವಿದೆ; ಅನವರತ ನಮ್ಮನ್ನು ಭಗವಂತ\-ನೊಂದಿಗೆ ಸಂಬಂಧಿಸಿರುವ ಪೂರ್ಣತಂತು ಒಂದಿದೆ. ಭಗವಂತನೊಂದಿಗೆ ಸಂಬಂಧ ಕಡಿದುಹೋದ ತಕ್ಷಣವೇ ಸರ್ವನಾಶ ಕಾದಿದೆ ಎಂದು ನಿಸ್ಸಂಶಯವಾಗಿ ತಿಳಿಯಿರಿ. ಯಾರೂ ಸರ್ವನಾಶವಾಗಲಾರದೆ ಇರುವುದರಿಂದ ನಮ್ಮ ಹೃದಯಾಂತರಾಳದಲ್ಲಿ ಎಷ್ಟೇ ಅಸ್ಪಷ್ಟವಾದರೂ ಭಗವಂತನೊಂದಿಗೆ ಸಂಬಂಧ ಕಲ್ಪಿಸಿಕೊಂಡಿರುವ ಜ್ಯೋತಿವೃತ್ತವೊಂದಿದೆ. ಪೂರ್ವಕಾಲದ ಮಹಾತ್ಮರು ಯಾವ ಕುಲ–ಗೋತ್ರ–ದೇಶಗಳಿಗೆ ಸೇರಿರಲಿ, ಅವರ ಜೀವನವನ್ನು ಮತ್ತು ಸಂದೇಶವನ್ನು ನಾವಿಂದು ಪಡೆದಿರುವೆವು. ಅವರಿಗೆಲ್ಲಾ ನಮ್ಮ ನಮನಗಳು. ಈಗ ಮಾನವನ ಉದ್ಧಾರಕ್ಕೆ ಕೆಲಸ ಮಾಡುತ್ತಿರುವ ಈಶ್ವರಸಮಾನರಾದ ಸ್ತ್ರೀ ಪುರುಷರಿಗೆ, ಅವರು ಯಾವ ಜಾತಿ–ಜನಾಂಗ–ಬಣ್ಣಗಳಿಗೆ ಸೇರಿರಲಿ, ಅವರಿಗೆಲ್ಲಾ ನಮ್ಮ ನಮನಗಳು. ನಮ್ಮ ಅನಂತರ ಬರುವ, ಜನರ ಉದ್ಧಾರಕ್ಕೆ ನಿಃಸ್ವಾರ್ಥದಿಂದ ಕೆಲಸ ಮಾಡುವ ಜಂಗಮ ಈಶ್ವರರಿಗೆ ನಮ್ಮ ನಮನಗಳು.

