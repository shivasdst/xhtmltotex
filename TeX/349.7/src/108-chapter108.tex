
\chapter[ಧರ್ಮಗಳಲ್ಲಿ ಸೌಹಾರ್ದ ಭಾವನೆ ]{ಧರ್ಮಗಳಲ್ಲಿ ಸೌಹಾರ್ದ ಭಾವನೆ \protect\footnote{\engfoot{C.W. Vol. VII, p. 286}}}

\centerline{(‘ಡೆಟ್ರಾಯಿಟ್​ ಫ್ರೀಪ್ರೆಸ್​’ ಪೆಬ್ರುವರಿ ೧೪, ೧೮೯೪)}

ಸ್ವಾಮೀಜಿ ಅವರು ಮಧ್ಯಮ ತರಗತಿಗೆ ಸೇರಿದ ಎತ್ತರವುಳ್ಳವರು. ತಮ್ಮ ಜನಾಂಗದಲ್ಲಿ ಸಾಮಾನ್ಯವಾಗಿರುವ ಮಸುಕು ಬಣ್ಣವುಳ್ಳವರು. ನಡತೆಯಲ್ಲಿ ಅಭಿಜಾತ ಕುಲಕ್ಕೆ ಸೇರಿದವರು, ಗಂಭೀರ ಗಮನವುಳ್ಳವರು, ಪ್ರತಿಯೊಂದು ಮಾತು ಕತೆ, ಚಲನವಲನಗಳಲ್ಲಿ ಅತಿ ವಿನಯವನ್ನು ವ್ಯಕ್ತಪಡಿಸುವರು. ಆದರೆ ಅವರ ವ್ಯಕ್ತಿತ್ವದಲ್ಲೆಲ್ಲ ನಮ್ಮನ್ನು ಹೆಚ್ಚು ಆಕರ್ಷಿಸುವುದೇ ಅವರ ಕಣ್ಣುಗಳು. ಅವು ಜ್ವಾಜಲ್ಯಮಾನವಾಗಿವೆ. ಸಂಭಾಷಣೆ ಸ್ವಾಭಾವಿಕವಾಗಿ ಧರ್ಮದ ಕಡೆ ತಿರುಗಿತು. ಸ್ವಾಮೀಜಿ ಅವರು ಹೇಳದ ಹಲವು ಮುಖ್ಯ ನಿಷಯಗಳಲ್ಲಿ ಕೆಳಗೆ ಬರುವುವು ಕೆಲವು:

“ನಾನು ಧರ್ಮಕ್ಕೂ ಪಂಥಕ್ಕೂ ಒಂದು ವ್ಯತ್ಯಾಸವನ್ನು ಕಲ್ಪಿಸುವೆನು. ಧರ್ಮ ಎಂದರೆ ಎಲ್ಲಾ ಮತಗಳನ್ನು ಸ್ವೀಕರಿಸುವುದನ್ನು ನೋಡುವುದು. ಪಂಥಗಳಾದರೋ, ಒಂದು ಮತ್ತೊಂದರೊಡನೆ ಕಾದಾಡುವ ಸ್ವಭಾವವುಳ್ಳವು, ಪರಸ್ಪರ ವಿರೋಧ ಅಭಿಪ್ರಾಯಗಳುಳ್ಳವು. ಹಲವು ಬಗೆಯ ಜನರಿರುವುದರಿಂದ ಹಲವು ಬಗೆಯ ಮತಗಳು ಇವೆ. ಜನರಿಗೆ ಏನು ಬೇಕೋ ಅದನ್ನು ಮತಗಳು ಒದಗಿಸುತ್ತವೆ. ಇದು ಜನಸಮುದಾಯಕ್ಕೆ ಹೊಂದಿಕೊಳ್ಳುವುದು. ಪ್ರಪಂಚದಲ್ಲಿ ಜನರು ಬೌದ್ಧಿಕ, ಆಧ್ಯಾತ್ಮಿಕ, ಮತ್ತು ಆರ್ಥಿಕ ಕ್ಷೇತ್ರದಲ್ಲಿ ಬೇರೆ ಬೇರೆ ಸ್ವಭಾವದವರಾಗಿರುವುದರಿಂದ, ಯಾವುದು ನಮಗೆ ಹೊಂದಿಕೊಳ್ಳುವುದೋ ಅಂತಹ ಉತ್ತಮವಾದ ನೀತಿಯಿಂದ ಕೂಡಿದ ಮತವನ್ನು ಸ್ವೀಕರಿಸುವರು. ಧರ್ಮ ಇದನ್ನು ಒಪ್ಪಿಕೊಳ್ಳುವುದು. ಹಲವು ಬಗೆಯ ಮತಗಳಿರುವುದಕ್ಕೆ ಅದು ತನ್ನ ಸಂತೋಷವನ್ನು ವ್ಯಕ್ತಪಡಿಸುವುದು. ಏಕೆಂದರೆ ಅವುಗಳಲ್ಲೆಲ್ಲ ಒಂದು ಸುಂದರವಾದ ನಿಯಮ ಹಿನ್ನೆಲೆ ಇದೆ. ಬೇರೆ ಬೇರೆ ದಾರಿಗಳಿಂದ ಒಂದೇ ಗುರಿಯನ್ನು ಮುಟ್ಟುವರು. ನನ್ನ ದಾರಿ ಪಾಶ್ಚಾತ್ಯ ಜನಾಂಗದ ಸ್ವಭಾವದವರಿಗೆ ಹಿಡಿಸದೆ ಇರಬಹುದು. ಅದರಂತೆಯೇ ಅವರ ದಾರಿ ನನ್ನ ಸ್ವಭಾವ ಮತ್ತು ನನ್ನ ತಾತ್ತ್ವಿಕವಿವೇಚನೆಗೆ ಒಪ್ಪದೆ ಇರಬಹುದು. ನಾನು ಹಿಂದೂಧರ್ಮಕ್ಕೆ ಸೇರಿದವನು. ಅದು ಬೌದ್ಧರ ಮತವಲ್ಲ. ಅದು ಹಿಂದೂಗಳ ಒಂದು ಒಳಪಂಗಡ. ನಾವು ಮಿಷನರಿ ಕೆಲಸಗಳಿಗೆ ಕೈಹಾಕುವುದಿಲ್ಲ. ನಮ್ಮ ಧರ್ಮದ ಮುಖ್ಯ ತತ್ತ್ವವೇ ಅದಕ್ಕೆ ವಿರೋಧವಾಗಿದೆ. ಅಥವಾ ನೀವು ಈ ದೇಶದಿಂದ ಹೊರಗೆ ಕಳುಹಿಸುವ ಮಿಷನರಿಗಳ ವಿಷಯದಲ್ಲಿ ನಾವು ಏನನ್ನೂ ಹೇಳುವುದಿಲ್ಲ. ಅವರು ಪ್ರಪಂಚದ ಯಾವ ಮೂಲೆಗೆ ಬೇಕಾದರೂ ಹೋಗಲು ನಮ್ಮ ಒಪ್ಪಿಗೆ ಇದೆ. ಹಲವರು ನಮ್ಮ ಬಳಿಗೆ ಬರುವರು. ಆದರೆ ಅದಕ್ಕಾಗಿ ನಾವು ಪ್ರಯತ್ನಪಡುವುದಿಲ್ಲ. ನಮ್ಮಂತೆ ಇತರರು ಆಲೋಚಿಸಲಿ ಎಂದು ಬೋಧಿಸಲು ನಮ್ಮಲ್ಲಿ ಯಾವ ಮಿಷನರಿ ತಂಡಗಳೂ ಇಲ್ಲ. ನಮ್ಮ ಕಡೆಯಿಂದ ಯಾವ ಪ್ರಯತ್ನವನ್ನು ಮಾಡದೆ ಇದ್ದರೂ, ಹಿಂದೂಧರ್ಮದ ಹಲವು ರೀತಿಗಳು ಪ್ರಪಂಚದಲ್ಲಿ ಹರಡುತ್ತವೆ. ಇವು ಕ್ರಿಶ್ಚಿಯನ್​ ವಿಜ್ಞಾನ, ಥಿಯಾಸಫಿ, ಎಡ್​ವಿನ್​ ಆರ್​ನಾಲ್ಡ್​ ಬರೆದ “ಏಷ್ಯದಜೋತಿ” ಮುಂತಾದ ರೂಪಗಳನ್ನು ಧರಿಸುತ್ತಿವೆ. ನಮ್ಮ ಧರ್ಮವಾದರೋ ಇತರ ಧರ್ಮಗಳಿಗಿಂತ ಮತ್ತು ಕ್ರೈಸ್ತ ಮತಕ್ಕಿಂತ ಎಷ್ಟೋ ಪುರಾತನವಾದುದು. ನಾನು ನಿಮ್ಮದನ್ನು ಧರ್ಮ ಎಂದು ಕರೆಯು ವುದಿಲ್ಲ. ಏಕೆಂದರೆ ಅದರಲ್ಲಿರುವ ಪರಸ್ಪರ ವಿರೋಧ ಅಭಿಪ್ರಾಯಗಳು ನೇರವಾಗಿ ಹಿಂದೂಧರ್ಮದಿಂದ ಬಂದಿವೆ. ಇದೊಂದು ದೊಡ್ಡ ಶಾಖೆ. ಕ್ಯಾಥೋಲಿಕ್​ ಮತ ಕೂಡ ನಮ್ಮಿಂದ ಹಲವು ವಸ್ತುಗಳನ್ನು ತೆಗೆದುಕೊಂಡಿದೆ. ತಪ್ಪನ್ನು ಒಪ್ಪಿಕೊಳ್ಳುವುದು, ಸಾಧುಗಳಲ್ಲಿ ನಂಬಿಕೆಯಿಡುವುದು, ಮುಂತಾವುದುಗಳೇ ಅವು. ಕ್ರೈಸ್ತರು ಮತ್ತು ಹಿಂದೂಗಳಲ್ಲಿರುವ ಸಮಾನವಾದ ಆಚಾರಗಳನ್ನು ನೋಡಿ, ಇವು ಹಿಂದೂ ಧರ್ಮದಿಂದಲೇ ಬಂದಿರಬೇಕೆಂದು ಒಪ್ಪಿಕೊಂಡ ಒಬ್ಬ ಕ್ಯಾಥೋಲಿಕ್​ ಪಾದ್ರಿಯನ್ನು ತನ್ನ ಪದವಿಯಿಂದ ತೆಗೆದುಹಾಕಿದರು. ಏಕೆಂದರೆ ಅವನು ತಾನು ಏನನ್ನು ಕಂಡನೊ ಅದನ್ನೆಲ್ಲ ಧೈರ್ಯವಾಗಿ ಬರೆದ. ಕ್ರೈಸ್ತಧರ್ಮದ ಆಚಾರಗಳು ಹಿಂದೂಧರ್ಮದಿಂದ ಬಂದವೆಂದು ಆತ ಒಪ್ಪಿಕೊಂಡನು.”

“ನೀವು ನಿಮ್ಮ ಧರ್ಮದಲ್ಲಿ ಅಜ್ಞೇಯತಾವಾದಿಗಳನ್ನು ಒಪ್ಪಿಕೊಳ್ಳುತ್ತೀ ರಾ?” ಎಂದು ಪ್ರಶ್ನೆಯನ್ನು ಕೇಳಿದರು.

“ಓ, ಹೌದು, ನೀವು ಯಾರನ್ನು ನಾಸ್ತಿಕರೆನ್ನುವಿರೋ ಅಂತಹ ತಾತ್ತ್ವಿಕ ಅಜ್ಞೇಯತಾವಾದಿಗಳನ್ನು ನಾವು ಒಪ್ಪಿಕೊಳ್ಳುತ್ತೇವೆ. ನಾವು ಬುದ್ಧನನ್ನು ಒಬ್ಬ ಮಹಾತ್ಮನೆಂದು ಪರಿಗಣಿಸುತ್ತೇವೆ. ಒಮ್ಮೆ ಅವರ ಅನುಯಾಯಿಯೊಬ್ಬನು “ದೇವರು ಇರುವನೇ” ಎಂದು ಕೇಳಿದನು. ಅದಕ್ಕೆ ಬುದ್ಧನು “ದೇವರೇ!ನಾನು ಎಂದು ನಿನಗೆ ದೇವರ ವಿಷಯವಾಗಿ ಹೇಳಿದೆ? ನಾನು ನಿನಗೆ ಹೇಳು ವುದು ಇದು: ಒಳ್ಳೆಯವನಾಗಿರು, ಒಳ್ಳೆಯದನ್ನು ಮಾಡು” ಎಂದನು. ನಮ್ಮಲ್ಲಿ ಹಲವರು ತಾತ್ತ್ವಿಕವಾಗಿ ಅಜ್ಞೇಯತಾವಾದಿಗಳು. ಪ್ರಕೃತಿಯಲ್ಲಿರುವ ನೀತಿ ನಿಯಮವನ್ನು, ಕೊನೆಗೆ ಎಲ್ಲರೂ ಪರಿಪೂರ್ಣರಾಗುವುದನ್ನು ನಂಬುವೆವು. ಎಲ್ಲ ಜನಗಳು ಸ್ವೀಕರಿಸುವ ಮತಗಳೆಲ್ಲವು ಮಾನವ ಜನಾಂಗವು ಮುಂದೊಮ್ಮೆ ಆತ್ಮನ ಅನಂತತೆಯನ್ನು ಸಾಕ್ಷಾತ್ಕರಿಸಿಕೊಳ್ಳಲು ಮಾಡಿರುವ ಪ್ರಯತ್ನಗಳೇ ಆಗಿವೆ.

“ಮಿಷನರಿ ಕೆಲಸಕ್ಕೆ ಕೈಹಾಕುವುದು ನಿಮ್ಮ ಧರ್ಮದ ಗೌರವಕ್ಕೆ ಕುಂದು ತರುವಂತಹುದೆ?”

ಇದಕ್ಕೆ ಉತ್ತರವನ್ನು ಕೊಡುವುದಕ್ಕಾಗಿ ಪೌರಸ್ತ್ಯ ಸಮೀಕ್ಷಕರು ಒಂದು ಸಣ್ಣ ಪುಸ್ತಕವನ್ನು ಕೈಗೆ ತೆಗೆದುಕೊಂಡರು. ಅಲ್ಲಿ ಬರುವ ಮುಖ್ಯವಾದ ಹಲವು ಶಿಲಾಶಾಸನಗಳಲ್ಲಿ ಒಂದನ್ನು ಅವರು ಸೂಚಿಸಿದರು.

“ಈ ಶಾಸನಗಳನ್ನು ಕ್ರಿಸ್ತಪೂರ್ವ ೨೦೦ ವರುಷಗಳ ಹಿಂದೆ ಬರೆದದ್ದು. ನೀವು ಕೇಳುವ ಪ್ರಶ್ನೆಗೆ ಕೊಡಬಲ್ಲ ಶ್ರೇಷ್ಠ ಉತ್ತರವೇ ಇದು.”

ಸಂತೋಷದಾಯಕವಾದ ಸ್ಪಷ್ಟವಾಗಿ ಕೇಳುವಂತಹ ಧ್ವನಿಯಲ್ಲಿ ಅವರು ಕೆಳಗಿದನ್ನು ಓದಿದರು:

“ದೇವತೆಗಳಿಗೆ ಪ್ರಿಯವಾದ ಪ್ರಿಯದರ್ಶಿ ರಾಜನು, ಎಲ್ಲಾ ಮತಗಳನ್ನು ಗೌರವಿಸುವನು. ಅವು ಸಂನ್ಯಾಸಿಗಳದಾಗಿರಬಹುದು, ಅಥವಾ ಗೃಹಸ್ಥರದಾಗಿರ ಬಹುದು. ಅವರಿಗೆ ಭಿಕ್ಷೆ ಮತ್ತು ಬಹುಮಾನಗಳನ್ನು ಕೊಡುವನು. ಆದರೆ ಅವರಿಗೆ ಕೊಡುವ ಗೌರವ ಮತ್ತು ಬಹುಮಾನಗಳಿಗಿಂತ ಹೆಚ್ಚಾಗಿ ಮುಖ್ಯವಾದ ಸದಾಚಾರವನ್ನು ಪ್ರೋತ್ಸಾಹಿಸಲು ಯತ್ನಿಸುವನು. ಬೇರೆ ಬೇರೆ ಮತಗಳಲ್ಲಿರುವ ಬೇರೆ ಬೇರೆ ಮುಖ್ಯವಾದ ನೀತಿಗಳ ವಿಷಯದಲ್ಲಿ ವ್ಯತ್ಯಾಸಗಳವೆ ಎಂಬುದು ನಿಜ. ಆದರೆ ಅವುಗಳಿಗೆಲ್ಲ ಒಂದು ಸಾಮಾನ್ಯ ಹಿನ್ನೆಲೆ ಇದೆ. ಅದೇ ದಯೆ; ಮಾತು ಮತ್ತು ನಡತೆಯಲ್ಲಿ ಸಂಯಮ. ತನ್ನ ಮತವನ್ನು ಹೊಗಳಿ, ಇತರರದನ್ನು ತೆಗಳಕೂಡದು. ಪ್ರತಿಯೊಬ್ಬರಿಗೂ ಅವರವರಿಗೆ ಸಲ್ಲುವ ಗೌರವವನ್ನು ಕೊಡಬೇಕು. ಇತರ ಮತಗಳಗೆ ಸಹಾಯ ಮಾಡುವುದರಿಂದ ತನ್ನ ಮತಕ್ಕೇ ಸಹಾಯ ಮಾಡಿದಂತೆ ಆಗುವುದು. ಅದಕ್ಕೆ ವಿರೋಧವಾಗಿ ಪ್ರಯತ್ನ ಮಾಡಿದರೆ, ತನ್ನ ಮತಕ್ಕೆ ಅವನು ಕೆಡಕುಮಾಡಿಕೊಳ್ಳುತ್ತಾನೆ. ತನ್ನ ಮತದ ಮೇಲಿರುವ ಆಸಕ್ತಿಯಿಂದ ಅದನ್ನು ಮುಂದೆ ತರಬೇಕೆಂದು ಅನ್ಯ ಮತಗಳನ್ನು ತೆಗಳಿದರೆ ಅವನು ತನ್ನ ಮತಕ್ಕೇ ಕಳಂಕ ತರುವನು. ಆದಕಾರಣ ಸೌಹಾರ್ದ ಭಾವನೆಯೇ ಶ್ರೇಷ್ಠವಾದುದು. ಪ್ರತಿಯೊಬ್ಬರು ಇನ್ನೊಬ್ಬರ ಅಭಿಪ್ರಾಯಗಳಿಗೆ ಗೌರವವನ್ನು ಕೊಡಬೇಕು. ಈ ದೃಷ್ಟಿಯಿಂದ ಈ ಶಾಸನವನ್ನು ಕೆತ್ತಿಸಿದೆ. ಜನ ಯಾವ ಮತಕ್ಕೇ ಸೇರಿರಲಿ, ಪ್ರತಿಯೊಂದು ಮತದಲ್ಲಿಯೂ ಇರುವ ಮೂಲನೀತಿಗಳನ್ನು ಅನುಷ್ಠಾನಕ್ಕೆ ತಂದು, ಇತರರ ಮತಗಳಿಗೆ ಗೌರವವನ್ನುಸಲ್ಲಿಸಬೇಕು. ಇದನ್ನು ಸಾಧಿಸುವುದಕ್ಕಾಗಿಯೇ, ಧರ್ಮಬೋಧಕರು, ಮತ್ತು ಅದಕ್ಕೆ ಸಂಬಂಧಪಟ್ಟ ಇತರ ಅಧಿಕಾರಿಗಳು ಕೆಲಸ ಮಾಡಬೇಕಾಗುವುದು.”

ಪರಿಣಾಮಕಾರಕವಾದ ಈ ಪಂಕ್ತಿಯನ್ನು ಓದಿದ ಮೇಲೆ ಸ್ವಾಮಿ ವಿವೇಕಾನಂದರು, ಈ ಶಾಸನವನ್ನು ಕೆತ್ತುವುದಕ್ಕೆ ಆಜ್ಞೆ ಮಾಡಿದ ಒಳ್ಳೆಯ ರಾಜನು, ಯುದ್ಧವು ವಿಶ್ವವ್ಯಾಪಿಯಾದ ಎಲ್ಲಾ ನೀತಿನಿಯಮಗಳಿಗೂ ವಿರೋಧವಾಗಿರುವುದರಿಂದ ಅದನ್ನು ನಿಲ್ಲಿಸಿದನು ಎಂದರು. “ಈ ಕಾರಣದಿಂದ ಭರತಖಂಡ ಭೌತಿಕ ಪ್ರಪಂಚದಲ್ಲಿ ಹಿಂದೆ ಉಳಿದಿದೆ. ಎಲ್ಲಿ ಮೃಗೀಯ ಶಕ್ತಿ ಮತ್ತು ರಕ್ತಪಾತ ಇತರ ರಾಷ್ಟ್ರಗಳನ್ನು ಮುಂದೆ ತಂದಿದೆಯೊ, ಅಲ್ಲಿ ಭರತಖಂಡ ಇದನ್ನು ಖಂಡಿಸಿದೆ. ವ್ಯಕ್ತಿಗೆ ಅನ್ವಯಿಸುವಂತೆ ಜನಾಂಗಗಳಿಗೂ ಅನ್ವಯಿಸುವ ಬಲಿಷ್ಠವಾದುದು ಮಾತ್ರ ಬದುಕುವುದು ಎಂಬ ನಿಯಮಾನುಸಾರ ಭರತಖಂಡವು ಭೌತಿಕ ಪ್ರಪಂಚದ ದೃಷ್ಟಿಯಲ್ಲಿ ಮಾತ್ರ ಹಿಂದೆ ಬಿದ್ದಿದೆ.”

“ಸಮರಕ್ಕನುಗುಣವಾಗಿ ನಿಂತಿರುವ ಶ್ರೇಷ್ಠ ಪಾಶ್ಚಾತ್ಯ ದೇಶಗಳಲ್ಲಿ, ಹತ್ತೊಂಬತ್ತನೇ ಶತಮಾನಕ್ಕೆ ಅತಿ ಅವಶ್ಯಕವಾಗಿ ವ್ಯವಹಾರ ಪ್ರಪಂಚದ ಅಗತ್ಯ ವಸ್ತುಗಳನ್ನು ಉತ್ಪತ್ತಿಮಾಡಿಕೊಳ್ಳಲು ಅದ್ಭುತವಾದ ಶಕ್ತಿ ಇರಬೇಕು. ಇಂತಹ ದೇಶಗಳಲ್ಲಿ ಪ್ರಶಾಂತವಾದ ಭರತಖಂಡದಲ್ಲಿರುವಂತಹ ಗುಣವನ್ನು ಇಚ್ಚಿಸು ವುದು ಅನುಚಿತವಲ್ಲವೆ?” ಎಂದು ಪ್ರಶ್ನಿಸಿದರು.

ಅವರ ತೇಜಃಪೂರ್ಣವಾದ ಕಣ್ಣುಗಳು ಥಳಥಳಿಸಿದುವು. ಆ ಪೌರಸ್ತ್ಯ ಸಹೋದರನ ಮುಖದ ಮೇಲೆ ಒಂದು ಮಂದಹಾಸ ಮಿಂಚಿನಂತೆ ಕಾಣಿಸಿಕೊಂಡಿತು.

“ಸಿಂಹದ ಪರಾಕ್ರಮದೊಂದಿಗೆ ಕುರಿಮರಿಯ ಸಾಧುತ್ವವನ್ನು ಸೇರಿಸುವುದಕ್ಕೆ ಆಗುವುದಿಲ್ಲವೆ?” ಎಂದು ಕೇಳಿದರು.ಸಂಭಾಷಣೆಯನ್ನು ಮುಂದು ವರಿಸುತ್ತ ಅವರು ಭಾಗಶಃ ಭವಿಷ್ಯದಲ್ಲಿ ಇಂತಹ ಮಿಲನವಾಗಬಹುದು, ಇಂತಹ ಮಿಲನದಿಂದ ಅದ್ಭುತವಾದ ಪರಿಣಾಮಗಳು ಉಂಟಾಗುವುವು ಎಂದರು. ಪಾಶ್ಚಾತ್ಯ ಸ್ವಭಾವದಲ್ಲಿರುವ ಒಂದು ಒಳ್ಳೆಯ ಗುಣವೇ ಸ್ತ್ರೀಯರನ್ನು ಗೌರವದಿಂದ ನೋಡು ವುದು; ಮತ್ತು ಅವರನ್ನು ನಯವಿನಯದಿಂದ ಕಾಣುವುದು; ಎಂಬುದಾಗಿ ನುಡಿದರು.

ಮರಣೋನ್ಮುಖನಾದ ಬುದ್ಧ ಹೇಳಿದಂತೆ “ನಿಮ್ಮ ಮುಕ್ತಿಯನ್ನು ನೀವೇ ಸಂಪಾದನೆ ಮಾಡಿಕೊಳ್ಳಿ. ನಾನು ನಿಮಗೆ ಸಹಾಯ ಮಾಡಲಾರೆ. ಮತ್ತಾರೂ ನಿಮಗೆ ಸಹಾಯ ಮಾಡಲಾರರು. ನಿಮಗೆ ನೀವೇ ಸಹಾಯ ಮಾಡಿಕೊಳ್ಳಬೇಕು” ಎಂದು ಸ್ವಾಮೀಜಿ ಅವರು ಹೇಳುವರು. ಸೌಹಾರ್ದ ಮತ್ತು ಶಾಂತಿ; ವೈಮನಸ್ಯ ವಲ್ಲ ಎಂಬುದು ಅವರ ಮೂಲ ಮಂತ್ರ.

ಒಂದು ಮತದವರು ಮತ್ತೊಬ್ಬರಲ್ಲಿ ತಪ್ಪು ಕಂಡುಹಿಡಿಯುವ ವಿಷಯದಲ್ಲಿ ಅವರು ಕೆಳಗೆ ಬರುವ ಒಂದು ಕಥೆಯನ್ನು ಹೇಳಿದರು:

ಒಂದು ಕಪ್ಪೆಯು ಬಾವಿಯಲ್ಲಿ ವಾಸ ಮಾಡುತ್ತಿತ್ತು. ಅದು ಅಲ್ಲಿ ಬಹಳ ಕಾಲವಿತ್ತು. ಅದು ಅಲ್ಲೇ ಹುಟ್ಟಿ ಬೆಳೆದ ಸಣ್ಣ ಕಪ್ಪೆಯಾಗಿತ್ತು. ಆ ಕಪ್ಪೆಗೆ ಕಣ್ಣುಗಳು ಇದ್ದವೆ ಇಲ್ಲವೆ ಎಂದು ಹೇಳುವುದಕ್ಕೆ ವಿಕಾಸವಾದಿಗಳು ಇರಲಿಲ್ಲ. ಆದರೆ ನಮ್ಮ ಕಥೆಯ ದೃಷ್ಟಿಯಿಂದ ಅದಕ್ಕೆ ಕಣ್ಣುಗಳು ಇದ್ದವು ಎಂದು ಬಾವಿಸೋಣ. ಅದು ಪ್ರತಿದಿನವೂ ನೀರಿನಲ್ಲಿದ್ದ ಸಣ್ಣ ಸಣ್ಣ ಕ್ರಿಮಿಕೀಟಗಳನ್ನೆಲ್ಲ ತಿಂದು ಶುಭ್ರವಾಗಿಡುತ್ತಿತ್ತು. ಹೀಗೆ ಅದು ಬೆಳೆಯುತ್ತಾ ಹೋಗಿ ನುಣುಪಾದ ದಪ್ಪನಾದ ಕಪ್ಪೆಯಾಯಿತು. ಬಹುಶಃ ನನ್ನಷ್ಟೆ ದೊಡ್ಡದಾಯಿತು ಎಂದು ಹೇಳಬಹುದು. ಹೀಗೆ ಇದ್ದಾಗ ಒಂದು ದಿನ ಸಮುದ್ರದಲ್ಲಿದ್ದ ಒಂದು ಕಪ್ಪೆ ಆ ಬಾವಿಗೆ ಬಂದು ಬಿತ್ತು.

“ನೀನೆಲ್ಲಿಂದ ಬಂದೆ” ಎಂದು ಬಾವಿ ಕಪ್ಪೆ ಕೇಳಿತು.

“ನಾನು ಸಮುದ್ರದಿಂದ ಬಂದೆ” ಎಂದಿತು.

“ಸಮುದ್ರವೇ? ಅದೆಷ್ಟು ದೊಡ್ಡದು. ಅದೇನು ನನ್ನ ಬಾವಿಯಷ್ಟು ದೊಡ್ಡದೊ” ಎಂದು ಹೇಳಿ ಬಾವಿಯ ಒಂದು ಕಡೆಯಿಂದ ಮತ್ತೊಂದು ಕಡೆಗೆ ನೆಗೆಯಿತು.

ಸಮುದ್ರದ ಕಪ್ಪೆ: “ನನ್ನ ಸಹೋದರ! ನಿನ್ನ ಸಣ್ಣ ಬಾವಿಯೊಂದಿಗೆ ಸಮುದ್ರವನ್ನು ಹೇಗೆ ಹೋಲಿಸಬಲ್ಲೆ?” ಎಂದು ಕೇಳಿತು.

ಬಾವಿಯ ಕಪ್ಪೆ ಮತ್ತೊಂದು ಸಾರಿ ನೆಗೆದು “ನಿನ್ನ ಸಮುದ್ರ ಇದಕ್ಕಿಂತಲೂ ದೊಡ್ಡದೊ?” ಎಂದು ಕೇಳಿತು.

“ಸಮುದ್ರವನ್ನು ನೀನು ಹೇಗೆ ನಿನ್ನ ಕಿರಿಯ ಬಾವಿಯೊಂದಿಗೆ ಹೋಲಿಸಬಲ್ಲೆ? ಇದೊಂದು ಮೌಢ್ಯ” ಎಂದಿತು ಸಮುದ್ರದ ಕಪ್ಪೆ.

ಆಗ ಬಾವಿಯ ಕಪ್ಪೆ ಕೇಳಿತು “ಆದರೆ ನನ್ನ ಬಾವಿಗಿಂತ ಯಾವುದು ದೊಡ್ಡದಾಗಿರಲಾರದು. ಇದಕ್ಕಿಂತಲೂ ಯಾವುದೂ ದೊಡ್ಡಲಾಗಿರಲಾರದು. ಇವನೊಬ್ಬ ಸುಳ್ಳುಗಾರ. ಇವನನ್ನು ಹೊರಕ್ಕೆ ತಳ್ಳಿ.”

“ಇದೇ ಯಾವಾಗಲೂ ನಮಗೆ ತೊಂದರೆ.”

“ನಾನು ಹಿಂದೂವಾಗಿರುವೆನು. ನಾನೊಂದು ಸಣ್ಣ ಬಾವಿಯಲ್ಲಿ ಕುಳಿತುಕೊಂಡು ಈ ಪ್ರಪಂಚವೇ ಹೀಗೆ ಎಂದು ಭಾವಿಸುವೆನು. ಕ್ರೈಸ್ತನು ತನ್ನ ಬಾವಿಯಲ್ಲಿ ಕುಳಿತುಕೊಂಡು ಇದೇ ಪ್ರಪಂಚ ಎಂದು ಭಾವಿಸುವನು. ನಮ್ಮ ನಮ್ಮ ಕಿರಿಯ ಪ್ರಪಂಚದ ಗೋಡೆಗಳನ್ನು ಒಡೆಯಲು ನೀವು ಮಾಡುತ್ತಿರುವ ಪ್ರಯತ್ನಕ್ಕಾಗಿ ನಾನು ಅಮೆರಿಕ ದೇಶದವರನ್ನು ಅಭಿನಂದಿಸುವೆನು. ನೀವು ಮುಂದೆ ಈ ಆದರ್ಶವನ್ನು ಸಾಧಿಸುವಂತೆ ದೇವರು ನಿಮ್ಮನ್ನು ಆಶೀರ್ವದಿಸಲಿ ಎಂದು ಬೇಡುತ್ತೇನೆ.”

