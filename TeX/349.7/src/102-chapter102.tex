
\chapter[ರಾಷ್ಟ್ರಮಟ್ಟದಲ್ಲಿ ಹಿಂದೂಧರ್ಮದ ಜಾಗೃತಿ ]{ರಾಷ್ಟ್ರಮಟ್ಟದಲ್ಲಿ ಹಿಂದೂಧರ್ಮದ ಜಾಗೃತಿ \protect\footnote{\engfoot{C.W. Vol. V, P. 225}}}

\centerline{\textbf{(“ಪ್ರಬುದ್ಧ ಭಾರತ”, ಸೆಪ್ಟೆಂಬರ್​, 1898)}}

ಇತ್ತೀಚೆಗೆ ಪ್ರಬುದ್ಧಭಾರತದ ಪ್ರತಿನಿಧಿಯೊಬ್ಬರು ಪ್ರಖ್ಯಾತ ಸ್ವಾಮಿ ವಿವೇಕಾನಂದರನ್ನು ಕಂಡಾಗ ನಿಮ್ಮ ಚಳುವಳಿಯ ವೈಶಿಷ್ಟ್ಯವೇನು ಎಂದು ಪ್ರಶ್ನಿಸಿದರು. ಅದಕ್ಕೆ ಸ್ವಾಮೀಜಿ ಅವರು ಹೀಗೆ ಉತ್ತರವಿತ್ತರು:

ಸ್ವಾಮೀಜಿ: “ಆಕ್ರಮಣ, ಧಾರ್ಮಿಕ ದೃಷ್ಟಿಯಿಂದ ಮಾತ್ರ, ಎಷ್ಟೋ ಪಂಗಡಗಳು ಇಂಡಿಯಾ ದೇಶದಲ್ಲೆಲ್ಲಾ ಆಧ್ಯಾತ್ಮಿಕ ವಿಷಯಗಳನ್ನು ಹರಡಿವೆ. ಆದರೆ ಬೌದ್ಧರ ಕಾಲದಿಂದ ಈಚೆಗೆ ಭರತಖಂಡ ಮೇರೆಯನ್ನು ಮೀರಿ, ಜಗತ್ತಿನಲ್ಲಿ ಆಧ್ಯಾತ್ಮಿಕ ಪ್ರವಾಹವನ್ನು\break ಹರಿಸುತ್ತಿರುವುದು ಇದೇ ಮೊದಲ ಬಾರಿ.

ಪ್ರಶ್ನೆ: “ಭರತಖಂಡದಲ್ಲಿ ನೀವು ಮಾಡುವ ಕೆಲಸದ ರೀತಿ ಹೇಗೆ?”

ಸ್ವಾಮೀಜಿ: “ಹಿಂದೂಧರ್ಮದ ಸಾಮಾನ್ಯ ತತ್ತ್ವಗಳನ್ನು ಕಂಡುಹಿಡಿದು ಅದರ ಮೂಲಕ ರಾಷ್ಟ್ರೀಯ ಭಾವನೆಗಳನ್ನು ಜಾಗ್ರತಗೊಳಿಸುವುದು. ಸದ್ಯಕ್ಕೆ ಹಿಂದೂ ಎಂಬ\break ಹೆಸರಿನ ಕೆಳಗೆ ಮೂರು ಪಂಥಗಳಿವೆ. ಪೂರ್ವಾಚಾರದ ಗುಂಪಿಗೆ ಸೇರಿದವರು,\break ಮಹಮ್ಮದೀಯ ಕಾಲದ ಸುಧಾರಕರ ಗುಂಪು, ಈಗಿನ ಕಾಲದ ಸುಧಾರಕರ ಗುಂಪು. ಉತ್ತರದಿಂದ ದಕ್ಷಿಣದವರೆಗೆ ಹಿಂದೂಗಳೆಲ್ಲ ಒಂದನ್ನು ಮಾತ್ರ ಒಪ್ಪಿಕೊಳ್ಳುವರು. ಅದೇ ಗೋಮಾಂಸವನ್ನು ತಿನ್ನದೇ ಇರುವುದು.”

ಪ್ರಶ್ನೆ: “ವೇದಗಳ ಮೇಲೆ ಇರುವ ಸಾಮಾನ್ಯ ಪ್ರೀತಿ ಅಲ್ಲವೆ?”

ಸ್ವಾಮೀಜಿ: “ನಿಜವಾಗಿಯೂ ಅಲ್ಲ. ನಾವು ಜಾಗ್ರತಗೊಳಿಸಬೇಕೆಂದಿರುವುದೇ ಅದನ್ನು; ಭರತಖಂಡ ಇನ್ನೂ ಬುದ್ಧನು ಮಾಡಿದ ಕಾರ್ಯವನ್ನು ಅರಗಿಸಿಕೊಂಡಿಲ್ಲ. ಇದು ಅವನ ಸಂದೇಶವನ್ನು ಮೋಹಿಸಿದೆಯೇ ಹೊರತು ಅದನ್ನು ಅನುಷ್ಠಾನಕ್ಕೆ ತಂದಿಲ್ಲ.”

\vskip 5pt

ಪ್ರಶ್ನೆ: “ಭರತಖಂಡದಲ್ಲಿ ಬೌದ್ಧಧರ್ಮದ ಪ್ರಾಬಲ್ಯವನ್ನು ನೀವು ಈಗ ಎಲ್ಲಿ ನೋಡುತ್ತೀರಿ?”

\vskip 5pt

ಸ್ವಾಮೀಜಿ: “ಇದು ಎಲ್ಲರಿಗೂ ತಿಳಿದೇ ಇದೆ. ಇದಕ್ಕೆ ಬೇರೆ ಪ್ರಮಾಣವೇ ಬೇಕಾಗಿಲ್ಲ. ನೋಡಿ ಭರತಖಂಡ ಎಂದಿಗೂ ತನ್ನಲ್ಲಿರುವುದನ್ನು ಕಳೆದುಕೊಳ್ಳುವುದಿಲ್ಲ. ಆದರೆ ಅದನ್ನು ಜೀರ್ಣಿಸಿಕೊಂಡು ರಕ್ತಗತ ಮಾಡಿಕೊಳ್ಳುವುದಕ್ಕೆ ಸ್ವಲ್ಪ ಕಾಲಾವಕಾಶ ಬೇಕಾಗುವುದು. ಬುದ್ಧನು ಪ್ರಾಣಿವಧೆಯನ್ನು ಸಂಪೂರ್ಣವಾಗಿ ನಿಲ್ಲಿಸಿದ. ಭರತಖಂಡವು ಈ ಪೆಟ್ಟಿನಿಂದ ಇನ್ನೂ ಚೇತರಿಸಿಕೊಂಡಿಲ್ಲ. ಬುದ್ಧನು ಗೋವುಗಳನ್ನು ಕೊಲ್ಲಬೇಡಿ ಎಂದ. ಗೋಹತ್ಯೆಯು ಈಗ ನಮಗೆ ಸಾಧ್ಯವೇ ಇಲ್ಲ.”

\vskip 5pt

ಪ್ರಶ್ನೆ: “ನೀವು ಹೇಳಿದ ಮೂರು ಪಂಗಡಗಳಲ್ಲಿ ನೀವು ಯಾವುದಕ್ಕೆ ಸೇರುತ್ತೀರಿ?”

\vskip 5pt

ಸ್ವಾಮೀಜಿ: “ಎಲ್ಲಕ್ಕೂ ಸೇರುತ್ತೇನೆ. ನಾವು ಆಚಾರಶೀಲ ಹಿಂದೂಗಳು,” (ಇಷ್ಟು ಹೇಳಿ ಸ್ವಾಮೀಜಿ ತಕ್ಷಣವೇ ತುಂಬ ಕಳಕಳಿಯಿಂದ ಇದನ್ನು ಒತ್ತಿ ಹೇಳಿದರು:) ಆದರೆ ‘ನಮ್ಮನ್ನು ಮುಟ್ಟಬೇಡಿ’ ಎನ್ನುವವರ ಗುಂಪಿಗೆ ಸೇರಿಲ್ಲ. ಅದು ಹಿಂದೂಧರ್ಮವಲ್ಲ. ಅದು ನಮ್ಮ ಯಾವ ಶಾಸ್ತ್ರದಲ್ಲಿಯೂ ಇಲ್ಲ. ಇದೊಂದು ಪೂರ್ವಾಚಾರ ಪರಾಯಣರ ಮೂಢನಂಬಿಕೆ. ಇದು ಯಾವಾಗಲೂ ರಾಷ್ಟ್ರದ ಪ್ರಗತಿಗೆ ಅಡ್ಡವಾಗಿ ನಿಂತಿರುವದು.”

\vskip 5pt

ಪ್ರಶ್ನೆ: “ಹಾಗಾದರೆ ನಿಮಗೆ ಬೇಕಾಗಿರುವುದು ರಾಷ್ಟ್ರಾಭಿವೃದ್ಧಿಗೆ ಬೇಕಾದ ಸಾಮ\-ರ್ಥ್ಯವೆ?”

\vskip 5pt

ಸ್ವಾಮೀಜಿ: “ಖಂಡಿತವಾಗಿ. ಆರ್ಯ ಜನಾಂಗದಲ್ಲಿ ಭರತಖಂಡ ಅವನತಿಯಲ್ಲಿರು\-ವುದಕ್ಕೆ ಮತ್ತೇನು ಕಾರಣವನ್ನು ಕೊಡಬಲ್ಲಿರಿ, ನೀವು? ಇವರಿಗೆ ಬುದ್ಧಿಗೆ ಬರಗಾಲವೆ, ಕೌಶಲ್ಯಕ್ಕೆ ಬರಗಾಲವೆ? ಅವರ ಕಲೆ, ಗಣಿತಶಾಸ್ತ್ರ, ತತ್ತ್ವ ಇವನ್ನು ನೋಡಿ ಹೌದು\break ಎನ್ನಬಲ್ಲಿರಾ? ಈಗ ಪ್ರಗತಿಪರ ರಾಷ್ಟ್ರಗಳೊಡನೆ ಅಗ್ರಭಾಗದಲ್ಲಿ ನಿಲ್ಲಬೇಕಾದರೆ\break ಬೇಕಾಗಿರುವುದು ಶತಮಾನಗಳಿಂದ ಇದ್ದ ತಾಮಸಿಕ ಸ್ಥಿತಿಯಿಂದ ಪಾರಾಗಿ ಜಾಗ್ರತರಾಗುವುದು.”

\vskip 5pt

ಪ್ರಶ್ನೆ: “ಆದರೆ ಭರತಖಂಡದಲ್ಲಿ ಯಾವಾಗಲೂ ಆಧ್ಯಾತ್ಮ ಪ್ರವಾಹ ಇಂಗಿರಲಿಲ್ಲ. ನೀವು ಭರತಖಂಡವನ್ನು ಕರ್ಮಪ್ರಪಂಚದಲ್ಲಿ ಮುಂದುವರಿಯುವಂತೆ ಪ್ರಚೋದಿಸುವುದರಿಂದ ಅದರ ಏಕಮಾತ್ರ ಆಸ್ತಿಯಾದ ಅಧ್ಯಾತ್ಮಕ್ಕೆ ಧಕ್ಕೆ ಬರುವುದಿಲ್ಲವೆ?”

\eject

ಸ್ವಾಮೀಜಿ: “ಎಂದಿಗೂ ಧಕ್ಕೆ ಬರುವುದಿಲ್ಲ. ವಿಶ್ವದ ಇತಿಹಾಸವನ್ನು ನೋಡಿದರೆ ಭಾರತ ಅಂತರ್ಮುಖ ಜೀವನದಲ್ಲಿಯೂ, ಯೂರೋಪ್​ ಬಹಿರ್ಮುಖ ಜೀವನದಲ್ಲಿಯೂ\break ಮುಂದುವರಿದಿರುವುದು ಕಾಣುವುದು. ಇಲ್ಲಿಯವರೆಗೆ ಇವು ಬೇರೆಬೇರೆಯಾಗಿದ್ದವು. ಈಗ ಎರಡು ಒಟ್ಟಿಗೆ ಸೇರುವ ಸಮಯ ಬಂದಿದೆ. ಶ‍್ರೀರಾಮಕೃಷ್ಣ ಪರಮಹಂಸರು ಸದಾ ಅಂತರ್ಮುಖ ಜೀವನದ ಆಳದಲ್ಲಿ ಮುಳುಗಿದ್ದರು. ಆದರೂ ಧರ್ಮಜೀವನದಲ್ಲಿ\break ಯಾರು ಅವರಿಗಿಂತ ಹೆಚ್ಚು ಜಾಗ್ರತರಾಗಿದ್ದರು? ಇದೇ ರಹಸ್ಯ. ನಿಮ್ಮ ಜೀವನವು\break ಸಾಗರದಷ್ಟು ಆಳವಾಗಿರಲಿ, ಜೊತೆಗೆ ಆಗಸದಷ್ಟು ವಿಶಾಲವಾಗಿಯೂ ಇರಲಿ.

ಬಾಹ್ಯ ಜೀವನದಲ್ಲಿ ಆತಂಕಗಳಿದ್ದಷ್ಟೂ ಅಂತರ್ಮುಖ ಜೀವನ ಹೆಚ್ಚು ವಿಕಾಸವಾಗುತ್ತಾ ಹೋಗುವುದು ಒಂದು ಆಶ್ಚರ್ಯ. ಆದರೆ ಇದು ಒಂದು ಆಕಸ್ಮಿಕ. ಇದೊಂದು ನಿಯಮವಲ್ಲ. ನಾವು ಇಲ್ಲಿ ಭರತಖಂಡದಲ್ಲಿ ಸರಿಯಾದರೆ ಇಡೀ ಪ್ರಪಂಚ ಸರಿಯಾಗುವುದು.\break ಏಕೆಂದರೆ ನಾವೆಲ್ಲ ಒಂದೇ ಅಲ್ಲವೆ?”

ಪ್ರಶ್ನೆ: “ನೀವು ಈಗತಾನೆ ಹೇಳಿದುದರಿಂದ ಮತ್ತೊಂದು ಪ್ರಶ್ನೆ ಏಳುವುದು. ಯಾವ ಅರ್ಥದಲ್ಲಿ ಶ‍್ರೀರಾಮಕೃಷ್ಣರು ಹಿಂದೂಧರ್ಮದ ಪುನರುತ್ಥಾನದ ಒಂದು ಭಾಗವಾಗಿದ್ದರು?”

ಸ್ವಾಮೀಜಿ: “ಇದನ್ನು ನಿರ್ಣಯಿಸುವುದು ನನ್ನ ಕೆಲಸವಲ್ಲ. ನಾನೆಂದಿಗೂ ವ್ಯಕ್ತಿಗಳನ್ನು ಕುರಿತು ಬೋಧಿಸಿಲ್ಲ. ನನ್ನ ಜೀವನವೇ ಈ ಮಹಾವ್ಯಕ್ತಿಯ ಸ್ಫೂರ್ತಿಯಿಂದ ರೂಪಿಸಲ್ಪಟ್ಟಿದೆ. ಆದರೆ ಎಷ್ಟರಮಟ್ಟಿಗೆ ಇತರರು ಇವರಿಗೆ ಋಣಿ ಎಂಬುದನ್ನು ಅವರ ಜೀವನ ನಿಷ್ಕರ್ಷಿಸಬೇಕಾಗಿದೆ. ಸ್ಫೂರ್ತಿಯು ಕೇವಲ ಒಂದೇ ವ್ಯಕ್ತಿಯ ಮೂಲಕ (ಅವನು ಎಷ್ಟೇ ಶ್ರೇಷ್ಠನಾಗಿರಲಿ) ಪ್ರಪಂಚಕ್ಕೆ ಬರುವುದಿಲ್ಲ. ಪ್ರತಿಯೊಂದು ಕಾಲದಲ್ಲಿಯೂ ಹೊಸಹೊಸ ವ್ಯಕ್ತಿಗಳಿಂದ ಸ್ಫೂರ್ತಿ ಬರಬೇಕು. ನಾವೆಲ್ಲ ದೇವರಲ್ಲವೆ?”

ಪ್ರಶ್ನೆ: “ನಿಮಗೆ ಧನ್ಯವಾದ. ನಿಮ್ಮನ್ನು ಕೇಳಬೇಕಾದ ಇನ್ನೊಂದು ಪ್ರಶ್ನೆ ಮಾತ್ರ ಇರುವುದು. ನಿಮ್ಮ ಪಂಥಕ್ಕೆ ಸೇರಿದವರ ದೃಷ್ಟಿಯಿಂದ ನೀವು ಯಾವ ರೀತಿ ಕೆಲಸ ಮಾಡುತ್ತೀರಿ ಎಂಬುದನ್ನು ವಿವರಿಸಬಲ್ಲಿರಾ?”

ಸ್ವಾಮೀಜಿ: “ನಮ್ಮ ಮಾರ್ಗವನ್ನು ಬಹಳ ಸುಲಭವಾಗಿ ವಿವರಿಸಬಹುದು. ರಾಷ್ಟ್ರದ ಜೀವನವನ್ನು ಪುನಃ ಜಾಗ್ರತಗೊಳಿಸಬೇಕಾಗಿದೆ. ಬುದ್ಧನು ತ್ಯಾಗವನ್ನು ಬೋಧಿಸಿದ. ಭರತಖಂಡ ಅದನ್ನು ಕೇಳಿತು. ಆದರೂ ಆರು ಶತಮಾನಗಳಲ್ಲಿ ಉಚ್ಛ್ರಾಯದ ಶಿಖರವನ್ನು ಮುಟ್ಟಿತು. ರಹಸ್ಯ ಇಲ್ಲಿರುವುದು. ಭರತಖಂಡದ ಜನಾಂಗದ ಆದರ್ಶವೇ ತ್ಯಾಗ ಮತ್ತು ಸೇವೆ. ಇವೆರಡನ್ನು ಚೆನ್ನಾಗಿ ರೂಢಿಸಿ, ಉಳಿದುವೆಲ್ಲ ತಮಗೆ ತಾವೇ ಹೊಂದಿಕೊಳ್ಳುವುವು. ಬರೀ ತ್ಯಾಗದ ಆದರ್ಶವನ್ನು ಮಾತ್ರ ರೂಢಿಸಿದರೆ ಸಾಲದು. ಇವೆರಡನ್ನೂ ಸೇರಿಸಬೇಕು. ಅದರಲ್ಲೇ ಇರುವುದು ನಮ್ಮ ಮುಕ್ತಿ.

\vspace{-0.5cm}

