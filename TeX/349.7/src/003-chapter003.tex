
\chapter[ಪ್ರಹ್ಲಾದ]{ಪ್ರಹ್ಲಾದ \protect\footnote{\engfoot{C.W. Vol. IV. P. 115}}}

\centerline{\textbf{(ಕ್ಯಾಲಿಫೋರ್ನಿಯಾದಲ್ಲಿ ನೀಡಿದ ಉಪನ್ಯಾಸ)}}

ಹಿರಣ್ಯಕಶಿಪು ದೈತ್ಯರ ರಾಜನಾಗಿದ್ದ. ದೇವತೆ ಮತ್ತು ದೈತ್ಯರು ಒಂದೇ ತಂದೆ ತಾಯಿಗಳಿಂದ ಜನಿಸಿದವರಾದರೂ ಯಾವಾಗಲೂ ಪರಸ್ಪರ ಯುದ್ಧ ಮಾಡುತ್ತಿದ್ದರು. ಮಾನವರು ಕೊಡುವ ಆಹುತಿಗಳು ದೈತ್ಯರಿಗೆ ಸಿಕ್ಕುತ್ತಿರಲಿಲ್ಲ. ಪ್ರಪಂಚವನ್ನು ಆಳುವುದಾಗಲೀ ಅದರ ರಕ್ಷಣೆಯಾಗಲೀ, ಯಾವುದರೊಂದಿಗೂ ಇವರಿಗೆ ಸಂಬಂಧವಿರಲಿಲ್ಲ. ಕೆಲವು ವೇಳೆ ದೈತ್ಯರು ಬಲಶಾಲಿಗಳಾಗಿ ದೇವತೆಗಳನ್ನೆಲ್ಲಾ ಸ್ವರ್ಗದಿಂದ ಓಡಿಸಿ ಅದನ್ನು ಆಕ್ರಮಿಸಿ ಕೆಲವು ಕಾಲ ಆಳುತ್ತಿದ್ದರು. ಆಗ ದೇವತೆಗಳು ಸರ್ವಾಂತರ್ಯಾಮಿಯಾದ ವಿಷ್ಣುವನ್ನು ಪ್ರಾರ್ಥಿಸುತ್ತಿದ್ದರು. ಅವನು ದೇವತೆಗಳನ್ನು ಕಷ್ಟದಿಂದ ಪಾರು ಮಾಡುತ್ತಿದ್ದನು. ದೈತ್ಯರನ್ನು ಓಡಿಸಿ ಪುನಃ ದೇವತೆಗಳು ಆಳುತ್ತಿದ್ದರು. ದೈತ್ಯೇಶನಾದ ಹಿರಣ್ಯಕಶಿಪು ತನ್ನ ಸರದಿ ಬಂದಾಗ ತನ್ನ ದಾಯಾದಿಗಳಾದ ದೇವತೆಗಳನ್ನು ಗೆದ್ದು ಸ್ವರ್ಗವನ್ನು ಆಕ್ರಮಿಸಿ, ಮನುಷ್ಯ ಮತ್ತು ಪ್ರಾಣಿಗಳಿಂದ ತುಂಬಿದ ಈ ಪೃಥ್ವಿ, ದೇವತೆಗಳಿಂದ ತುಂಬಿದ ಸ್ವರ್ಗ, ಅಸುರರಿಂದ ತುಂಬಿದ ಪಾತಾಳ ಲೋಕಗಳನ್ನು ಆಳತೊಡಗಿದ. ಹಿರಣ್ಯಕಶಿಪು ತಾನೇ ಮೂರು ಲೋಕಗಳಿಗೂ ಒಡೆಯನೆಂದೂ, ತಾನಿಲ್ಲದೆ ಬೇರೆ ದೇವರೇ ಇಲ್ಲವೆಂದೂ, ಎಲ್ಲಿಯೂ ಸರ್ವಾಂತರ್ಯಾಮಿಯಾದ ವಿಷ್ಣುವನ್ನು ಪೂಜಿಸಕೂಡದೆಂದೂ ಇಂದಿನಿಂದ ಪೂಜೆಯೆಲ್ಲಾ ತನಗೇ ಆಗಬೇಕೆಂದೂ ಕಟ್ಟಪ್ಪಣೆ ಮಾಡಿದನು.

ಹಿರಣ್ಯಕಶಿಪುವಿಗೆ ಪ್ರಹ್ಲಾದನೆಂಬ ಮಗನಿದ್ದ. ಪ್ರಹ್ಲಾದ ಬಾಲ್ಯಾರಭ್ಯದಿಂದಲೂ ಭಗವಂತನ ಭಕ್ತನಾಗಿದ್ದ. ಅವನು ಸಣ್ಣ ಮಗುವಾಗಿರುವಾಗಲೇ ಅವನಲ್ಲಿ ಭಕ್ತನ ಚಿಹ್ನೆಗಳು ಕಾಣುತ್ತಿದ್ದುವು. ತಾನು ನಿವಾರಿಸಬೇಕೆಂದುಕೊಂಡಿದ್ದ ಅನಿಷ್ಟ ತನ್ನ ಮನೆಯಲ್ಲಿಯೇ ಎಲ್ಲಿ ಹುಟ್ಟಿಕೊಳ್ಳುವುದೋ ಎಂಬ ಅಂಜಿಕೆಯಿಂದ, ವಿಧೇಯತೆಯನ್ನು ಬಲಾತ್ಕರಿಸುವುದರಲ್ಲಿ ಪ್ರಖ್ಯಾತರಾದ ಚಂಡ ಮತ್ತು ಅಮರಕ ಎಂಬ ಇಬ್ಬರು ಗುರುಗಳ ಅಧೀನಕ್ಕೆ ಮಗನನ್ನು ಒಪ್ಪಿಸಿದನು. ಪ್ರಹ್ಲಾದನು ವಿಷ್ಣುವಿನ ಹೆಸರನ್ನು ಕೂಡ ಕೇಳದಂತೆ ನೋಡಿಕೊಳ್ಳಬೇಕೆಂದು ಕಟ್ಟಪ್ಪಣೆ ಇತ್ತನು. ಗುರುಗಳು ಹುಡುಗನನ್ನು ತಮ್ಮ ಮನೆಗೆ ಕರೆದುಕೊಂಡು ಹೋಗಿ ಆತನ ಓರಗೆಯ ಹುಡುಗರೊಂದಿಗೆ ಓದಲು ಹಾಕಿದರು. ಆದರೆ ಆ ಹುಡುಗ ಪ್ರಹ್ಲಾದ ಪುಸ್ತಕದಿಂದ ಓದನ್ನು ಕಲಿಯುವ ಬದಲು ಕಾಲವನ್ನೆಲ್ಲಾ ವಿಷ್ಣುವನ್ನು ಹೇಗೆ ಪೂಜಿಸಬೇಕೆಂದು ಇತರ ಹುಡುಗರಿಗೆ ಬೋಧಿಸುವುದರಲ್ಲಿ ಕಳೆಯುತ್ತಿದ್ದನು. ಗುರುಗಳಿಗೆ ಇದು ಗೊತ್ತಾದಾಗ ಅವರು ಅಂಜಿದರು, ಏಕೆಂದರೆ ಅವರು ಪ್ರಚಂಡ ಹಿರಣ್ಯಕಶಿಪುವಿನ ಕೋಪಕ್ಕೆ ಪಾತ್ರರಾಗುವುದರಲ್ಲಿದ್ದರು. ಅವರು ಸಾಧ್ಯವಾದಷ್ಟು ಅಂತಹ ಬೋಧನೆಗಳಲ್ಲಿ ಪ್ರಹ್ಲಾದನು\break ತೊಡಗದಿರುವಂತೆ ನೋಡಿಕೊಂಡರು. ಆದರೆ ಪ್ರಹ್ಲಾದ ಹೇಗೆ ಉಸಿರಾಡುವುದನ್ನು ಬಿಡಲಾರನೋ ಹಾಗೆಯೇ ಭಗವಂತನನ್ನು ಪೂಜಿಸುವುದನ್ನಾಗಲೀ, ಅವನನ್ನು ಇತರರಿಗೆ ಬೋಧಿಸುವುದನ್ನಾಗಲೀ ಬಿಡಲಿಲ್ಲ. ತಾವು ತಪ್ಪಿತಸ್ಥರಲ್ಲವೆಂಬುದನ್ನು ತೋರುವುದಕ್ಕಾಗಿ, ಪ್ರಹ್ಲಾದನು ತಾನು ಮಾತ್ರ ವಿಷ್ಣುವನ್ನು ಪೂಜಿಸುವುದಲ್ಲ, ಇತರರಿಗೂ ಅದನ್ನು ಬೋಧಿಸಿ ಅವರನ್ನೆಲ್ಲಾ ಹಾಳುಮಾಡುತ್ತಿರುವನು ಎಂಬ ಭಯಂಕರವಾದ ಸತ್ಯವನ್ನು ರಾಜನಿಗೆ ಗುರುಗಳು ಹೇಳಿದರು.

\vskip 0.3cm

ರಾಜ ಇದನ್ನು ಕೇಳಿದಾಗ ರೇಗಿಹೋಗಿ ಮಗನನ್ನು ತನ್ನ ಬಳಿಗೆ ಕರೆದನು. ಮಗುವಿಗೆ ಉಪಾಯದಿಂದ ನಯವಾದ ಮಾತಿನಲ್ಲಿ ಎಲ್ಲಾ ಪೂಜೆಯನ್ನು ಬಿಡು, ರಾಜನಾದ ತಾನೊಬ್ಬನು ಮಾತ್ರ ಪೂಜೆಗೆ ಯೋಗ್ಯನಾದವನು ಎಂದನು. ಆದರೆ ಇದರಿಂದ ಏನೂ ಪ್ರಯೋಜನವಾಗಲಿಲ್ಲ. ಪುನಃ ಪುನಃ ಮಗು ಸರ್ವವ್ಯಾಪಿಯಾದ ವಿಷ್ಣು ಒಬ್ಬನೇ ಪೂಜೆಗೆ ಯೋಗ್ಯನಾದ ದೇವರೆಂದೂ, ರಾಜನು ಕೂಡ ವಿಷ್ಣುವಿನ ಇಚ್ಛಾನುಸಾರ ರಾಜನಾಗಿರುವನೆಂದೂ ಹೇಳಿದನು. ಆಗ ರಾಜನು ಕೋಪಕ್ಕೆ ಮಿತಿ ಇರಲಿಲ್ಲ. ತಕ್ಷಣ ಹುಡುಗನನ್ನು ಕೊಲ್ಲಿ ಎಂದು ಆಜ್ಞಾಪಿಸಿದನು. ದೈತ್ಯರು ಹುಡುಗನನ್ನು ಮೊನಚಾದ ಅಸ್ತ್ರಗಳಿಂದ ಹೊಡೆದರು. ಆದರೆ ಪ್ರಹ್ಲಾದನ ಮನಸ್ಸು ಭಗವಂತನಲ್ಲಿ ತಲ್ಲೀನವಾಗಿದ್ದುದರಿಂದ ಅವನಿಗೆ ಬಾಧೆಯೇ ಆಗಲಿಲ್ಲ.

\vskip 0.3cm

ರಾಜನಾದ ತಂದೆ ಇದನ್ನು ನೋಡಿದಾಗ ಅಂಜಿದನು. ಆದರೆ ರಾಕ್ಷಸನ ಕೋಪ\break ಉಕ್ಕೇರಲು ಮಗುವನ್ನು ಕೊಲ್ಲಲು ಹಲವು ಚಿತ್ರ ವಿಚಿತ್ರ ಹಿಂಸೆಗಳನ್ನು ಆಲೋಚಿಸತೊಡಗಿದನು. ಒಂದು ಆನೆಯಿಂದ ಮಗುವನ್ನು ತುಳಿಸಿ ಎಂದನು. ಆ ಮದಕರಿ ಒಂದು ಕಬ್ಬಿಣದ ಕಲ್ಲನ್ನು ಹೇಗೆ ನಾಶಮಾಡಲಾರದೋ ಹಾಗೆ ಪ್ರಹ್ಲಾದನನ್ನು ಕೊಲ್ಲದೇ ಹೋಯಿತು. ಇದರಿಂದಲೂ ಏನೂ ಪ್ರಯೋಜನವಾಗಲಿಲ್ಲ. ಅನಂತರ ರಾಜ ಒಂದು ಕಡಿದಾದ ಸ್ಥಳದಿಂದ ಮಗುವನ್ನು ಕೆಳಕ್ಕೆ ತಳ್ಳಿ ಎಂದು ಆಜ್ಞಾಪಿಸಿದನು. ಇದನ್ನೂ ಮಾಡಿದರು. ವಿಷ್ಣು ಪ್ರಹ್ಲಾದನ ಹೃದಯದಲ್ಲಿ ನೆಲೆಸಿದ್ದುದರಿಂದ ಹೂವಿನೆಸಳು ಹುಲ್ಲಿನ ಮೇಲೆ ಬೀಳುವಂತೆ ಧರೆಯಮೇಲೆ ನಿಧಾನವಾಗಿ ಬಿದ್ದನು. ವಿಷ, ಬೆಂಕಿ, ಉಪವಾಸ, ಬಾವಿಗೆ ನೂಕುವುದು, ಮಾಟ ಮಂತ್ರ ಮುಂತಾದುವನ್ನೆಲ್ಲಾ ಒಂದಾದ ಮೇಲೊಂದನ್ನು ಪ್ರಯತ್ನಿಸಿದರು. ಆದರೆ ಯಾವುದರಿಂದಲೂ ಪ್ರಯೋಜನವಾಗಲಿಲ್ಲ.

\vskip 0.3cm

ಕೊನೆಗೆ ರಾಜ ಪಾತಾಳದಿಂದ ಘಟಸರ್ಪವನ್ನು ತಂದು ಅದರಿಂದ ಪ್ರಹ್ಲಾದನನ್ನು ಕಟ್ಟುವಂತೆ ಆಜ್ಞಾಪಿಸಿದನು. ಅನಂತರ ಅವನನ್ನು ಸಮುದ್ರಕ್ಕೆ ತಳ್ಳಿ, ಅವನ ಮೇಲೆ ಬೆಟ್ಟಗುಡ್ಡಗಳನ್ನು ಹೇರುವಂತೆ ಆಜ್ಞಾಪಿಸಿದನು. ಹೀಗೇ ಬಿಟ್ಟರೆ ತಕ್ಷಣ ಅಲ್ಲದೆ ಇದ್ದರೆ ನಿಧಾನವಾಗಿಯಾದರೂ ಮಗು ಸಾಯುವುದೆಂದು ಆಶಿಸಿದನು. ಇಂತಹ ಹಿಂಸೆಗೆ ಗುರಿಯಾದರೂ ಮಗು “ವಿಶ್ವೇಶನಾದ ವಿಷ್ಣುವೇ, ನಿನಗೆ ನಮಸ್ಕಾರ, ಸುಂದರ ಕಮಲದವನನೆ ನಿನಗೆ ನಮಸ್ಕಾರ” ಎಂದು ಪ್ರಾರ್ಥಿಸತೊಡಗಿದನು. ವಿಷ್ಣುವಿನ ಚಿಂತನೆ ಮತ್ತು ಅವನ ಧ್ಯಾನದಲ್ಲಿ ಮಗ್ನನಾದಾಗ ವಿಷ್ಣು ತನ್ನ ಸಮೀಪದಲ್ಲಿರುವನು, ತನ್ನ ಆಂತರ್ಯದಲ್ಲಿರುವನು, ತಾನೇ ವಿಷ್ಣು, ತಾನೇ ಸರ್ವಸ್ವ, ಮತ್ತು ಸರ್ವವ್ಯಾಪಿ ಎಂದು ಭಾವಿಸತೊಗಿದನು.

ಮೇಲಿನ ಅನುಭವವಾದೊಡನೆ ಸರ್ಪಬಂಧನ ಕಳಚಿತು. ಮೇಲಿದ್ದ ಬೆಟ್ಟಗಳು ಪುಡಿಯಾದುವು, ಸಮುದ್ರ ಉಕ್ಕಿ ಪ್ರಹ್ಲಾದನನ್ನು ಅಲೆಯ ಮೇಲೆ ತೇಲಿಸಿಕೊಂಡು ಸುರಕ್ಷಿತವಾಗಿ ಕರೆಗೆ ತಂದುಬಿಟ್ಟಿತು. ಪ್ರಹ್ಲಾದ ಅಲ್ಲಿ ನಿಂತಾಗ, ತಾನೊಬ್ಬ ದೈತ್ಯ, ತನಗೊಂದು ದೇಹವಿದೆ ಎಂಬುದನ್ನೇ ಮರೆತನು. ತಾನೇ ವಿಶ್ವ, ವಿಶ್ವಶಕ್ತಿಯೆಲ್ಲಾ ತನ್ನಿಂದ ಆವಿರ್ಭವಿಸುತ್ತಿದೆ, ಪ್ರಕೃತಿಯಲ್ಲಿ ಯಾವುದೂ ತನ್ನನ್ನು ನಾಶ ಮಾಡಲಾರದು, ತಾನೇ ಪ್ರಕೃತಿಗೆ ಒಡೆಯ ಎಂದು ಭಾವಿಸಿದನು. ಅವಿಚ್ಛಿನ್ನವಾದ ಆನಂದದಲ್ಲಿ ಕೆಲವು ಕಾಲ ಹೀಗೆ ಕಳೆದ ಮೇಲೆ ಕ್ರಮೇಣ ಪ್ರಹ್ಲಾದನಿಗೆ ದೇಹ ಜ್ಞಾನ ಬಂದು ತಾನು ಪ್ರಹ್ಲಾದ ಎಂದು ಅರಿತನು. ಅವನಿಗೆ ದೇಹಭಾವ ಬಂದೊಡನೆಯೆ ದೇವರು ಅಂತರ ಮತ್ತು ಬಾಹ್ಯದಲ್ಲಿರುವನು, ‘ವಿಷ್ಣುಮಯಂ ಸರ್ವಂ ಇದಂ’ ಎಂದು ಭಾವಿಸಿದನು.

ಹಿರಣ್ಯಕಶಿಪುಗೆ ತನ್ನ ದ್ವೇಷಿಯಾದ ವಿಷ್ಣುವಿನ ತದೇಕಭಕ್ತನಾದ ಪ್ರಹ್ಲಾದನನ್ನು ಕೊಲ್ಲಲು ಮಾಡಿದ ಮಾನವ ಪ್ರಯತ್ನವೆಲ್ಲಾ ವಿಫಲವಾದ ಮೇಲೆ ಭಯಗ್ರಸ್ತನಾದನು. ಮುಂದೇನು ಮಾಡಬೇಕೋ ಗೊತ್ತಾಗಲಿಲ್ಲ. ರಾಜ ಪುನಃ ಮಗುವನ್ನು ತನ್ನ ಬಳಿಗೆ ಕರೆಯಿಸಿ ಮೃದುವಾಗಿ ತನ್ನ ಬುದ್ಧಿವಾದವನ್ನು ಕೇಳುವಂತೆ ಪ್ರಯತ್ನಿಸಿದನು. ಆದರೆ ಪ್ರಹ್ಲಾದ ಅದೇ ಉತ್ತರವನ್ನು ಕೊಟ್ಟ. ಮಗುವಿಗೆ ವಯಸ್ಸಾದಂತೆ ತಿಳುವಳಿಕೆ ಬಂದ ಮೇಲೆ ಬಾಲ ಭಾವನೆಗಳನ್ನೆಲ್ಲಾ ತೊರೆಯುವುದೆಂದು ಭಾವಿಸಿ, ರಾಜನ ಕರ್ತವ್ಯವನ್ನು ಅವನಿಗೆ ಬೋಧಿಸುವಂತೆ ಚಂಡ ಮತ್ತು ಅಮರಕ ಎಂಬ ಗುರುಗಳಿಗೆ ಅಜ್ಞಾಪಿಸಿದನು. ಆದರೆ ಈ ಬೋಧನೆಗಳಾವುವೂ ಪ್ರಹ್ಲಾದನಿಗೆ ಹಿಡಿಸಲಿಲ್ಲ. ಅವನು ತನ್ನ ಸಹಪಾಠಿಗಳಿಗೆ ವಿಷ್ಣು ಭಕ್ತಿಯನ್ನು ಬೋಧಿಸುವುದರಲ್ಲೇ ತನ್ನ ಕಾಲವನ್ನೆಲ್ಲಾ ಕಳೆಯುತ್ತಿದ್ದ.

ತಂದೆಗೆ ಇದು ಗೊತ್ತಾದಾಗ ಅವನು ಕೋಪದಿಂದ ಉನ್ಮತ್ತನಾಗಿ ಮಗನನ್ನು ಬಳಿಗೆ ಕರೆದು ಅವನನ್ನು ಕೊಂದುಬಿಡುತ್ತೇನೆ ಎಂದು ಹೆದರಿಸಿ, ವಿಷ್ಣುವನ್ನು ಮನಸ್ಸಿಗೆ ಬಂದಂತೆ ನಿಂದಿಸಿದನು. ಆದರೆ ಪ್ರಹ್ಲಾದ, ವಿಷ್ಣು ಒಬ್ಬನೇ ಆದಿ ಅಂತ್ಯರಹಿತನು, ವಿಭು ಸರ್ವಾಂತ\break ರ್ಯಾಮಿ, ಅವನೊಬ್ಬನೇ ಪೂಜೆಗೆ ಯೋಗ್ಯನೆಂದು ಹೇಳಿದನು. ರಾಜ ಕೋಪದಿಂದ ಗರ್ಜಿಸಿ “ಎಲೈ ಪಾಪಿ, ವಿಷ್ಣು ಸರ್ವಾಂತರ್ಯಾಮಿಯಾದರೆ ಎದುರಿಗಿರುವ ಕಂಬದಲ್ಲಿ ಏತಕ್ಕೆ ಇರಬಾರದು?” ಎಂದನು. ಪ್ರಹ್ಲಾದ, “ಹೌದು, ಅಲ್ಲಿ ನಿಜವಾಗಿಯೂ ಇರುವನು” ಎಂದು ನಮ್ರನಾಗಿ ಹೇಳಿದನು. “ಹಾಗಾದರೆ ಅವನು ನಿನ್ನನ್ನು ರಕ್ಷಿಸಲಿ, ನಾನು ನಿನ್ನನ್ನು ಕೊಲ್ಲುತ್ತೇನೆ” ಎಂದು ಹಿರಿದ ಕತ್ತಿಯಿಂದ ಎದುರಿಗಿರುವ ಕಂಬವನ್ನು ಒಡೆದನು. ತಕ್ಷಣ ಒಂದು ದೊಡ್ಡ ಗರ್ಜನೆ ಕೇಳಿಸಿತು. ಆ ಕಂಬದಿಂದ ವಿಷ್ಣು ಉಗ್ರನರಸಿಂಹನಂತೆ ಬಂದುದನ್ನು ನೋಡಿದನು. ಅಂಜಿ ದೈತ್ಯರೆಲ್ಲಾ ಚೆಲ್ಲಾಪಿಲ್ಲಿಯಾಗಿ ಓಡಿಹೋದರು. ಆದರೆ ಹಿರಣ್ಯಕಶಿಪು ಅವನೊಡನೆ ದೀರ್ಘಕಾಲ ಕಷ್ಟಪಟ್ಟು ಹೋರಾಡಿ ಕೊನೆಗೆ ಸತ್ತು ಹೋದನು.

ದೇವತೆಗಳು ಆಗ ಸ್ವರ್ಗದಿಂದ ಬಂದು ವಿಷ್ಣುವನ್ನು ಸ್ತುತಿಸತೊಡಗಿದರು. ಪ್ರಹ್ಲಾದನು ದೇವರಿಗೆ ನಮಿಸಿ ಭಗವಂತನನ್ನು ಭಕ್ತಿಯಿಂದ ಕೊಂಡಾಡತೊಡಗಿದನು. ವಿಷ್ಣು ಇವನನ್ನು “ಮಗು ಪ್ರಹ್ಲಾದ ನಿನಗೆ ಏನು ಬೇಕೋ ಅದನ್ನು ಕೇಳು. ನೀನು ನನ್ನ ಮಗ, ಮೆಚ್ಚುಗೆಗೆ ಪಾತ್ರನು. ನಿನಗೆ ಬೇಕಾದುದನ್ನು ಕೇಳು” ಎಂದನು. ಪ್ರಹ್ಲಾದ ಭಕ್ತಿಪರವಶನಾಗಿ “ಹರಿ, ನಾನು ನಿನ್ನ ದರ್ಶನ ಪಡೆದೆ. ನನಗೆ ಇನ್ನೇನು ಬೇಕು. ನನಗೆ ಜಗತ್ತು ಅಥವಾ ಸ್ವರ್ಗದ ಆಸೆಯನ್ನು ತೋರಬೇಡ” ಎಂದು ಬೇಡಿಕೊಂಡ. “ಆದರೂ ಮಗು ಏನನ್ನಾದರೂ ಕೇಳು” ಎಂದ ವಿಷ್ಣು. ಆಗ ಪ್ರಹ್ಲಾದ “ಅವಿವೇಕಿಗಳು ಪ್ರಾಪಂಚಿಕ ವಸ್ತುಗಳನ್ನು ಎಷ್ಟು ಆಸಕ್ತಿಯಿಂದ ನೋಡುವರೋ, ಅದೇ ಆಸಕ್ತಿ ನನಗೆ ನಿನ್ನ ಮೇಲಿರಲಿ. ನಿನ್ನ ಮೇಲೆ ಅದೇ ಉತ್ಕಟ ಪ್ರೀತಿ ಇರಲಿ, ಆದರೆ ಅದು ಕೇವಲ ಪ್ರೇಮಕ್ಕೋಸುಗವಾಗಲಿ” ಎಂದು ಪ್ರಾರ್ಥಿಸಿದನು.

ಆಗ ವಿಷ್ಣು ಇಂತೆಂದನು: “ನನ್ನ ಪರಮಭಕ್ತನೆ, ನೀನು ಎಂದಿಗೂ ಏನನ್ನೂ ಆಶಿಸದೆ ಇದ್ದರೂ ಈ ಕಲ್ಪದ ಕೊನೆಯವರೆಗೆ ಪ್ರಪಂಚದ ಸುಖವನ್ನು ಅನುಭವಿಸಿ, ಮನಸ್ಸನ್ನೆಲ್ಲಾ ನನ್ನ ಮೇಲಿಟ್ಟು ಹಲವು ಧರ್ಮಕಾರ್ಯಗಳನ್ನು ಮಾಡು. ಕಾಲಾ ನಂತರ ನೀನು ಕಾಲವಾದ ಮೇಲೆ ನನ್ನನ್ನು ಸೇರುವೆ.” ಪ್ರಹ್ಲಾದನನ್ನು ಹೀಗೆ ಆಶೀರ್ವದಿಸಿ ವಿಷ್ಣು ಮಾಯವಾದನು. ಆಗ ದೇವತೆಗಳು ಬ್ರಹ್ಮನ ನಾಯಕತ್ವದಲ್ಲಿ ಪ್ರಹ್ಲಾದನನ್ನು ದೈತ್ಯರ ರಾಜನನ್ನಾಗಿ ಮಾಡಿ ತಮ್ಮ ಲೋಕಗಳಿಗೆ ತೆರಳಿದರು.

