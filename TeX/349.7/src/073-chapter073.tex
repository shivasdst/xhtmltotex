
\chapter[ಮನಃಶಕ್ತಿ ]{ಮನಃಶಕ್ತಿ \protect\footnote{\engfoot{C.W. Vol. VI, p 125}}}

ಕಾರಣವೇ ಕಾರ್ಯವಾಗುವುದು. ಕಾರ್ಯಕಾರಣಗಳು ಬೇರೆ ಬೇರೆ ಅಲ್ಲ. ಕಾರಣವು ಪೂರ್ಣಗೊಂಡ ಅವಸ್ಥೆಯೇ ಕಾರ್ಯ. ಯಾವಾಗಲೂ ಕಾರಣವೇ ಕಾರ್ಯವಾಗುವುದು. ಸಾಧಾರಣವಾಗಿ ಕಾರಣ ಬೇರೆ, ಕಾರ್ಯ ಬೇರೆ ಎಂದು ಜನರು ಭಾವಿಸುವರು. ಆದರೆ ಅದು ಹಾಗಲ್ಲ. ಪರಿಣಾಮ ಬೇರೊಂದು ರೂಪ ದಲ್ಲಿರುವ ಕಾರಣ, ಅಷ್ಟೆ.

ಜಗತ್ತು ಯಾವಾಗಲೂ ಏಕರೂಪವಾಗಿರುವುದು. ವ್ಯತ್ಯಾಸ ಕೇವಲ ತೋರಿಕೆಗೆ ಮಾತ್ರ, ಪ್ರಕೃತಿಯಲ್ಲೆಲ್ಲ ಬೇರೆ ಬೇರೆ ದ್ರವ್ಯಗಳು ಮತ್ತು ಬೇರೆ ಬೇರೆ ಶಕ್ತಿಗಳು ಇರುವಂತೆ ಕಾಣುವುದು. ಉದಾಹರಣೆಗೆ ಒಂದು ಚೂರು ಮರ ಒಂದು ಚೂರು ಗಾಜು ಇವೆರಡನ್ನೂ ತೆಗೆದುಕೊಂಡು ಚೆನ್ನಾಗಿ ನೀವು ಪುಡಿ ಪುಡಿ ಮಾಡಿ. ಅವೆರಡನ್ನೂ ಇನ್ನೂ ಹಲವು ಅವಸ್ಥೆಗಳಾದ ಮೇಲೆ ಅದರ ಮೂಲಕ್ಕೆ ತೆಗೆದು ಕೊಂಡು ಹೋದರೆ ಅಲ್ಲಿ ಯಾವ ವ್ಯತ್ಯಾಸವೂ ಕಾಣುವುದಿಲ್ಲ. ಎಲ್ಲಾ ದ್ರವ್ಯಗಳುಕೊನೆಯ ವಿಶ್ಲೇಷಣೆಯಲ್ಲಿ ಒಂದೇ ಆಗಿರುವುವು. ಒಂದೇ ಸಮನಾಗಿರುವುದೇ ಸತ್ಯ. ಅಸಮಾನತೆ ಕೇವಲ ತೋರಿಕೆಗೆ ಮಾತ್ರ. ಅದ್ವೈತವೇ ಏಕರೂಪತೆ, ಅದು ಅನೇಕದಂತೆ ತೋರುತ್ತಿರುವುದೇ ಭಿನ್ನತೆ.

ಕೇಳುವುದು, ನೋಡವುದು, ರುಚಿ ನೋಡುವುದು ಇವುಗಳನ್ನೆಲ್ಲಾ ಮನಸ್ಸೇ ಬೇರೆ ಬೇರೆ ಸ್ಥಿತಿಯಲ್ಲಿ ಮಾಡುತ್ತದೆ.

ಒಂದು ಕೋಣೆಯ ವಾತಾವರಣವನ್ನೆ ಬೇಕಾದರೆ ವಶ್ಯಸುಪ್ತಿಯ ಅವಸ್ಥೆಗೆ ತರಬಹುದು. ಅಲ್ಲಿಗೆ ಬಂದವರೆಲ್ಲ ಏನೇನೋ ನೋಡುವಂತೆ ಮಾಡಬಹುದು; ಹಲವು ವಸ್ತುಗಳು ಮತ್ತು ಮನುಷ್ಯರು ಅಲ್ಲಿ ಹಾರಾಡುವಂತೆ ತೋರಬಹುದು.

ಎಲ್ಲರೂ ಆಗಲೇ ಸಮ್ಮೋಹಿತರಾಗಿರುವರು. ನಾವು ಮುಕ್ತರಾಗುವುದು, ನಮ್ಮ ಆತ್ಮ ಸ್ವರೂಪವನ್ನು ತಿಳಿಯುವುದು ಎಂದರೆ ನಮ್ಮ ಭ್ರಾಂತಿಯಿಂದ ಪಾರಾಗುವುದು ಎಂದು ಅರ್ಥ.

ನಾವು ಹೊಸದಾಗಿ ಶಕ್ತಿಯನ್ನು ಪಡೆಯುತ್ತಿಲ್ಲ ಎಂಬುದನ್ನು ನಾವು ಜ್ಞಾಪಕ ದಲ್ಲಿಡಬೇಕು. ಅದಾಗಲೇ ನಮ್ಮಲ್ಲಿದೆ. ಭ್ರಾಂತಿಯಿಂದ ಪಾರಾಗುವುದೇ ಬೆಳವಣಿಗೆ. ಮನಸ್ಸು ಪರಿಶುದ್ಧವಾದಷ್ಟೂ ಅದನ್ನು ನಿಗ್ರಹಿಸುವುದು ಸುಲಭ. ನೀವು ಮನಸ್ಸನ್ನು ನಿಗ್ರಹಿಸಬೇಕಾದರೆ ಅದು ಪರಿಶುದ್ಧವಾಗಿರಬೇಕು. ಅದ್ಭುತ ಶಕ್ತಿಯ ಪಡೆಯ ಬೇಕೆಂದು ಮಾತ್ರ ಯತ್ನಿಸಬೇಡಿ ಅವನ್ನು ತ್ಯಜಿಸಿ. ಅದ್ಭುತ ಶಕ್ತಿಯನ್ನು ಪಡೆದವನು ಅದರ ವ್ಯಾಮೋಹಕ್ಕೆ ಒಳಗಾಗುವನು, ಅದ್ಭುತ ಶಕ್ತಿಯನ್ನು ಇಚ್ಛಿಸಿದವರೆಲ್ಲ ಅದರ ಬಲೆಗೆ ಬೀಳುವರು.

ಮನಸ್ಸನ್ನು ಪೂರ್ಣವಾಗಿ ನಿಗ್ರಹಿಸಬೇಕಾದರೆ ಒಬ್ಬನು ಪೂರ್ಣವಾಗಿ ನೀತಿವಂತನಾಗಿರಬೇಕು. ನೈತಿಕ ಜೀವನದಲ್ಲಿ ಪರಿಶುದ್ಧನಾದವನು ಮತ್ತೇನನ್ನೂ ಮಾಡಬೇಕಾಗಿಲ್ಲ. ಅವನು ಆಗಲೇ ಮುಕ್ತ. ಯಾವನು ಪೂರ್ಣವಾಗಿ ನೀತಿವಂತ ನಾಗಿರುವನೋ ಅವನು ಯಾರಿಗೂ ತೊಂದರೆಕೊಡಲಾರ, ಹಿಂಸೆಮಾಡಲಾರ. ಮುಕ್ತನಾಗಲೆತ್ನಿಸುವವನು ಅಹಿಂಸೆಯನ್ನು ಅಭ್ಯಾಸ ಮಾಡಬೇಕು. ಪೂರ್ಣ ಅಹಿಂಸಾವ್ರತಿಯಷ್ಟು ಮತ್ತಾರೂ ಶಕ್ತರಲ್ಲ. ಯಾರೂ ಅವನೆದುರಿಗೆ ಹೋರಾಡು ವುದಿಲ್ಲ, ಜಗಳ ಕಾಯುವುದಿಲ್ಲ. ಅವನಿದ್ದೆಡೆ ಶಾಂತಿ ಮತ್ತು ಪ್ರೀತಿ ತಾಂಡವ ವಾಡುವುವು. ಅವನು ಮತ್ತೇನನ್ನೂ ಮಾಡಬೇಕಾಗಿಲ್ಲ. ದುಷ್ಟ ಪ್ರಾಣಿಗಳು ಕೂಡ ಅವನ ಎದುರಿಗೆ ಶಾಂತವಾಗಿರುವುವು.

ಹಿಂದೆ ನನಗೆ ಒಬ್ಬ ಯೋಗಿಗಳ ಪರಿಚಯವಿತ್ತು. ಅವರು ಬಹಳ ವೃದ್ಧರು. ಅವರು ನೆಲದೊಳಗೆ ಒಂದು ಕೋಣೆಯಲ್ಲಿ ಒಬ್ಬರೇ ವಾಸಿಸುತ್ತಿದ್ದರು. ಅವರ ಹತ್ತಿರ ಅಡಿಗೆ ಮಾಡಿಕೊಳ್ಳುವುದಕ್ಕೆ ಪಾತ್ರೆಗಳು ಮಾತ್ರ ಇದ್ದವು. ಅವರು ಊಟ ಮಾಡುತ್ತಿದ್ದುದು ಬಹಳ ಸ್ವಲ್ಪ. ಹಾಕಿಕೊಳ್ಳುತ್ತಿದ್ದ ಬಟ್ಟೆಯೂ ಬಹಳ ಕಡಮೆ. ಯಾವಾಗಲೂ ಧ್ಯಾನದಲ್ಲಿರುತ್ತಿದ್ದರು.

ಅವರು ಎಲ್ಲರನ್ನೂ ಒಂದೇ ಸಮನಾಗಿ ಕಾಣುತ್ತಿದ್ದರು. ಯಾರಿಗೂ ಹಿಂಸೆಯನ್ನು ಮಾಡುತ್ತಿರಲಿಲ್ಲ. ಅವರು ಪ್ರತಿಯೊಂದು ಪ್ರಾಣಿಯಲ್ಲಿ, ವಸ್ತುವಿನಲ್ಲಿ, ಮನುಷ್ಯನಲ್ಲಿ ಭಗವಂತನನ್ನೇ ನೋಡುತ್ತಿದ್ದರು. ಅವರಿಗೆ ಪ್ರತಿಯೊಂದು ಪ್ರಾಣಿಯೂ, ಪ್ರತಿಯೊಬ್ಬ ಮನುಷ್ಯನೂ ಭಗವಂತನೇ ಆಗಿದ್ದನು. ಬೇರೆ ವಿಧದಲ್ಲಿ ಅವರು ಇತರರನ್ನು ಸಂಬೋಧಿಸುತ್ತಲೇ ಇರಲಿಲ್ಲ. ಒಂದು ದಿನ ಕಳ್ಳನೊಬ್ಬ ಅವರ ಪಾತ್ರೆಯನ್ನು ಕದ್ದನು. ಯೋಗಿಗಳು ಅವನನ್ನು ನೋಡಿ ಅವನ ಹಿಂದೆ ಓಡಿ ಹೋದರು. ಬಹಳ ದೂರ ಓಡಿದ ಮೇಲೆ ಕಳ್ಳ ಸಾಕು ಸಾಕಾಗಿ ನಿಂತನು. ಯೋಗಿ ಅವನ ಹತ್ತಿರ ಹೋಗಿ ಅವನ ಕಾಲಿಗೆ ಬಿದ್ದು “ಹೇ ಭಗವಾನ್​, ನೀನು ನನ್ನ ಬಳಿಗೆ ಬಂದದ್ದು ಮಹಾಭಾಗ್ಯ ನನಗೆ. ಮತ್ತೊಂದು ಪಾತ್ರೆಯನ್ನೂ ದಯವಿಟ್ಟು ಸ್ವೀಕರಿಸಿ ನನ್ನನ್ನು ಕೃತಾರ್ಥನನ್ನಾಗಿ ಮಾಡು. ಇದೂ ನಿನ್ನದೇ” ಎಂದರು. ಆ ವೃದ್ಧರು ಈಗ ಕಾಲವಾಗಿರುವರು. ಅವರಿಗೆ ಪ್ರಪಂಚದಲ್ಲಿ ಪ್ರತಿಯೊಬ್ಬರ ಮೇಲೂ ಅಷ್ಟೊಂದು ವಿಶ್ವಾಸ. ಅವರು ಇರುವೆಗಾಗಿ ಬೇಕಾದರೂ ಪ್ರಾಣವನ್ನು ಕೊಡಲು ಸಿದ್ಧರಿದ್ದರು. ಕಾಡಿನ ಪ್ರಾಣಿಗಳು ಸ್ವಭಾವತಃ ಅವರನ್ನು ತಮ್ಮ ಸ್ನೇಹಿತ ಎಂದು ಭಾವಿಸಿದ್ದವು. ಅವೆಲ್ಲ ಇವರನ್ನು ಪ್ರೀತಿಸುತ್ತಿದ್ದವು. ಅವರೆದುರಿಗೆ ಎಂದೂ ಜಗಳ ಕಾಯುತ್ತಿರಲಿಲ್ಲ.

ಇತರರ ಲೋಪದೋಷಗಳನ್ನು ಕುರಿತು ಮಾತಾಡಬೇಡಿ, ಅದು ಎಂತಹ ದೋಷವಾದರೂ ಚಿಂತೆಯಿಲ್ಲ. ಅದರಿಂದ ಏನೂ ಪ್ರಯೋಜನವಾಗುವುದಿಲ್ಲ. ಅವನ ತಪ್ಪನ್ನು ಕುರಿತು ಮಾತನಾಡಿದರೆ ಅವನಿಗೆ ಎಳ್ಳಷ್ಟೂ ಸಹಾಯವಾಗು ವುದಿಲ್ಲ. ಇದರಿಂದ ನೀವು ಅವನನ್ನು ಹಿಂಸಿಸುವಿರಿ. ನಿಮಗೂ ಕೂಡ ಅದು ಕೇಡು.

ಊಟ ಮತ್ತು ಇತರ ಅಭ್ಯಾಸಗಳೆಲ್ಲ ಎಲ್ಲಿಯವರೆಗೆ ಆಧ್ಯಾತ್ಮಿಕ ಆದರ್ಶಕ್ಕೆ ಸಹಕಾರಿಗಳಾಗಿರುವುವೋ ಅಲ್ಲಿಯವರೆಗೆ ಒಳ್ಳೆಯವು. ಅವೇ ಗುರಿಯಲ್ಲ. ಅವು ಗುರಿಯ ಎಡೆಗೆ ಹೋಗುವುದಕ್ಕೆ ಸಹಾಯಕವಾಗುತ್ತವೆ ಅಷ್ಟೆ.

ಧರ್ಮದ ವಿಷಯದಲ್ಲಿ ಜಗಳ ಕಾಯಬೇಡಿ. ವಾದ ಚರ್ಚೆ ಇವುಗಳೆಲ್ಲ ಆಧ್ಯಾತ್ಮಿಕತೆಯ ಅಭಾವವನ್ನು ತೋರುತ್ತವೆ. ಜಗಳ ಕಾಯುವುದು ಯಾವಾಗಲೂ ಕರಟಕ್ಕಾಗಿ; ಪಾವಿತ್ರ್ಯ ಮತ್ತು ಅಧ್ಯಾತ್ಮ ಯಾವಾಗ ಇಲ್ಲವೋ ಆಗ ಆತ್ಮ ಶುಷ್ಕವಾಗುವುದು. ಆಗಲೇ ಜಗಳ ಪ್ರಾರಂಭ, ಅದಕ್ಕೆ ಮುಂಚೆ ಅಲ್ಲ.

