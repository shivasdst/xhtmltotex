
\chapter{ಭಗವದ್ಗೀತಾ (೨)\protect\footnote{\engfoot{}}}

\centerline{\textbf{(೧೯೦೦ರ ಮೇ ೨೮ರಂದು ಸ್ಯಾನ್​ಫ್ರಾನ್ಸಿಸ್ಕೋದಲ್ಲಿ ನೀಡಿದ ಉಪನ್ಯಾಸ)}}

ಗೀತೆಯ ಪ್ರಾರಂಭದ ಪರಿಚಯ ಸ್ವಲ್ಪ ಅವಶ್ಯಕ. ಈ ದೃಶ್ಯ ಕುರುಕ್ಷೇತ್ರದ ಸಮರಾಂಗಣದಲ್ಲಿ ಇರುವುದು. ಸುಮಾರು ಐದು ಸಾವಿರ ವರುಷಗಳ ಹಿಂದೆ ಒಂದೇ ವಂಶದ ಎರಡು\break ಪಕ್ಷದವರು ಒಂದು ಚಕ್ರಾಧಿಪತ್ಯಕ್ಕಾಗಿ ಹೋರಾಡುತ್ತಿದ್ದರು. ಧರ್ಮವು ಪಾಂಡ\-ವರ ಪಕ್ಷದಲ್ಲಿತ್ತು. ಆದರೆ ಸೈನ್ಯ ಬಲವು ಕೌರವರ ಪಕ್ಷದಲ್ಲಿತ್ತು. ಪಾಂಡವರು ಐದು ಜನ\break ಸಹೋದರರು. ಅವರು ಕಾಡಿನಲ್ಲಿ ವಾಸಿಸುತ್ತಿದ್ದರು. ಶ‍್ರೀಕೃಷ್ಣ ಪಾಂಡವರ ಸ್ನೇಹಿತನಾಗಿದ್ದ. ಕೌರವರು ಪಾಂಡವರಿಗೆ ಒಂದು ಮುಳ್ಳಿನ ಮೊನೆಯಷ್ಟು ಕೂಡ ರಾಜ್ಯವನ್ನು ಕೊಡಲು ಒಪ್ಪಲಿಲ್ಲ.

ಯುದ್ಧರಂಗವೇ ಪ್ರಾರಂಭದ ದೃಶ್ಯ. ಎರಡು ಕಡೆಯವರೂ ತಮ್ಮ ಸ್ನೇಹಿತರು ಮತ್ತು ಬಂಧುಗಳನ್ನೇ ನೋಡುತ್ತಾರೆ. ಒಬ್ಬ ಸಹೋದರ ಒಂದು ಕಡೆ ಇರುವನು. ಮತ್ತೊಬ್ಬ ಸಹೋದರ ಇನ್ನೊಂದು ಕಡೆ ಇರುವನು. ತಾತ ಒಂದು ಕಡೆ ಇರುವನು. ಮೊಮ್ಮಗ ಅವನಿಗೆ ವಿರೋಧವಾಗಿರುವನು. ಅರ್ಜುನ ತನ್ನ ಸ್ನೇಹಿತರು ಮತ್ತು ಬಂಧುಗಳನ್ನೇ ವಿರೋಧ ಪಕ್ಷದಲ್ಲಿ ನೋಡುವನು. ಅವರನ್ನು ಯುದ್ಧದಲ್ಲಿ ಕೊಲ್ಲಬೇಕಾಗಿದೆ ಎಂದು ಅರಿಯುವನು. ಆಗ ಅವನ ಮನಸ್ಸು ದುರ್ಬಲವಾಗುವುದು. ತಾನು ಯುದ್ಧ ಮಾಡುವುದಿಲ್ಲವೆಂದು\break ಹೇಳುವನು. ಗೀತೆ ಪ್ರಾರಂಭವಾಗುವುದು ಹೀಗೆ.

ನಮಗೆಲ್ಲ ಈ ಪ್ರಪಂಚ ಒಂದು ನಿರಂತರವಾದ ಹೋರಾಟದಂತೆ ಇದೆ. ಜೀವನದಲ್ಲಿ ಅನೇಕ ವೇಳೆ ನಾವು ನಮ್ಮ ದೌರ್ಬಲ್ಯ ಮತ್ತು ಹೇಡಿತನವನ್ನು ಕ್ಷಮೆ ಮತ್ತು ತ್ಯಾಗವೆಂದು ಕರೆಯಲು ಯತ್ನಿಸುತ್ತೇವೆ. ಭಿಕ್ಷುಕನ ತ್ಯಾಗದಲ್ಲಿ ಪ್ರಯೋಜನವಿಲ್ಲ. ಪೆಟ್ಟಿಗೆ ಪೆಟ್ಟನ್ನು ಕೊಡುವ ಶಕ್ತಿಯಿದ್ದರೂ ಯಾರು ಕ್ಷಮಿಸುವರೋ ಅದರಲ್ಲಿ ಪ್ರಯೋಜನವಿದೆ. ಅನೇಕ ವೇಳೆ ನಾವು ನಮ್ಮ ಜೀವನದಲ್ಲಿ ಸೋಮಾರಿತನ ಮತ್ತು ಅಧೈರ್ಯದಿಂದ ಹೋರಾಟವನ್ನೇ ತ್ಯಜಿಸಿ ತುಂಬಾ ಧೀರರು ಎಂದು ಭಾವಿಸುತ್ತ ಭ್ರಮೆಗೆ ಬೀಳುವೆವು.

ಗೀತೆಯು ಮುಂದಿನ ಪ್ರಮುಖವಾದ ಶ್ಲೋಕದಿಂದ ಪ್ರಾರಂಭವಾಗುವುದು. “ಹೇ ರಾಜಕುಮಾರನೇ ಏಳು, ಜಾಗೃತನಾಗು, ಈ ಹೃದಯ ದೌರ್ಬಲ್ಯವನ್ನು ತ್ಯಜಿಸು. ಎದ್ದುನಿಂತು ಹೋರಾಡು.” ಆಗ ಅರ್ಜುನನು ಕೃಷ್ಣನೊಂದಿಗೆ ಚರ್ಚೆ ಮಾಡಲು ಯತ್ನಿಸಿದಾಗ ಹಿಂಸೆಗಿಂತ ಅಹಿಂಸೆ ಹೇಗೆ ಮೇಲು ಎಂಬುದನ್ನು ಸಾಧಿಸಲು ಯತ್ನಿಸುವನು. ಅರ್ಜುನ ತಾನು ಮಾಡುವುದೇ ಸರಿ ಎಂದು ಸಾಧಿಸಲು ಯತ್ನಿಸುವನು. ಆದರೆ ಶ‍್ರೀಕೃಷ್ಣನನ್ನು ಅವನು ಮರುಳು ಮಾಡಲಾರ. ಶ‍್ರೀಕೃಷ್ಣನೇ ಪುರುಷೋತ್ತಮ ಅಥವಾ ದೇವರು. ಅರ್ಜುನನ ವಾದದ ಹಿಂದೆ ಇರುವ ಮಾನಸಿಕ ದೌರ್ಬಲ್ಯವನ್ನು ಅವನು ತಕ್ಷಣ\break ಅರಿಯುವನು. ಅರ್ಜುನ ತನ್ನ ಬಂಧು ಬಾಂಧವರನ್ನು ನೋಡಿದಾಗ ಅವರನ್ನು ಕೊಲ್ಲಲು ಇಚ್ಛಿಸುವುದಿಲ್ಲ.

ಅರ್ಜುನನ ಮನಸ್ಸಿನಲ್ಲಿ ತನ್ನ ಕರ್ತವ್ಯ ಮತ್ತು ಮಮತೆಗಳ ಮಧ್ಯೆ ದೊಡ್ಡದೊಂದು ಹೋರಾಟವಾಗುವುದು. (ವಿಕಾಸ ಪಥದಲ್ಲಿ) ನಾವು ಪಶುಪಕ್ಷಿಗಳ ಸ್ಥಿತಿಗೆ ಸಮೀಪದಲ್ಲಿರುವಾಗ ನಮ್ಮ ಭಾವೋದ್ವೇಗಗಳು ನರಕವಾಗಿ ನಮ್ಮನ್ನು ಕಾಡುತ್ತವೆ. ನಾವು ಅದನ್ನು ಪ್ರೀತಿ ಎಂದು ಕರೆಯುವೆವು. ಆದರೆ ಇದೊಂದು ಭ್ರಮೆ. ನಾವು ಪ್ರಾಣಿಗಳಂತೆ ನಮ್ಮ ಮಮತಾ ವ್ಯಾಮೋಹಗಳಲ್ಲಿ ಸಿಕ್ಕಿ ನರಳುತ್ತಿರುವೆವು. ಒಂದು ಹಸು ತನ್ನ ಕರುವಿಗೆ ಬೇಕಾದರೆ ತನ್ನ ಪ್ರಾಣವನ್ನೇ ಬಲಿ ಕೊಡಬಲ್ಲದು. ಪ್ರತಿಯೊಂದು ಪ್ರಾಣಿಯೂ ಇದನ್ನು ಮಾಡಬಲ್ಲದು. ಇದರಿಂದ ಏನು ಪ್ರಯೋಜನ? ಪಕ್ಷಿ ಸಹಜವಾದ ವ್ಯಾಮೋಹ ನಮ್ಮನ್ನು ಪೂರ್ಣತೆಗೆ ಒಯ್ಯಲಾರದು. ಅಖಂಡ ಚೈತನ್ಯವನ್ನು ಸೇರುವುದೇ ಮಾನವನ ಗುರಿ. ಉದ್ವೇಗಕ್ಕೆ ಮತ್ತು ವ್ಯಾಮೋಹಕ್ಕೆ ಅಲ್ಲಿ ಸ್ಥಳವಿಲ್ಲ. ಪಂಚೇಂದ್ರಿಯಗಳಿಗೆ ಸಂಬಂಧಪಟ್ಟ ಯಾವುದಕ್ಕೂ ಅಲ್ಲಿ\break ಸ್ಥಳವಿಲ್ಲ. ಅಲ್ಲಿ ಪರಿಶುದ್ಧವಾದ ಯುಕ್ತಿಯ ಪರಂಜ್ಯೋತಿಗೆ ಮಾತ್ರ ಸ್ಥಳ. ಅಲ್ಲಿ ಮಾನವನು ಆತ್ಮನಂತೆ ನಿಲ್ಲುವನು.

ಅರ್ಜುನನು ಈಗ ವ್ಯಾಮೋಹದಲ್ಲಿ ಸಿಕ್ಕಿ ನರಳುತ್ತಿರುವನು. ಅವನು ತನ್ನ ನೈಜ ಸ್ಥಿತಿಯಲ್ಲಿ ಇಲ್ಲ. ಮನಸ್ಸನ್ನು ನಿಗ್ರಹಿಸಿದ, ಯುಕ್ತಿಯ ಪರಂಜ್ಯೋತಿಯ ಬೆಳಕಿನಲ್ಲಿ ಕೆಲಸ ಮಾಡುತ್ತಿರುವ ಜ್ಞಾನಿಯಂತೆ ಇಲ್ಲ. ಅವನೊಂದು ಮೃಗದಂತೆ ಆಗಿರುವನು, ಒಂದು ಮಗುವಿನಂತೆ ಆಗಿರುವನು. ವಿಚಾರವನ್ನೆಲ್ಲಾ ಬದಿಗೆ ಒಡ್ಡಿ ಉದ್ವೇಗಕ್ಕೆ ಬಲಿಯಾಗಿರುವನು. ಒಬ್ಬ ಉನ್ಮತ್ತನಂತೆ ಆಗಿರುವನು. ಪ್ರೀತಿ ಮುಂತಾದ ಆಪ್ಯಾಯಮಾನವಾದ ಪದಪುಂಜಗಳಿಂದ ತನ್ನ ಹೃದಯ ದೌರ್ಬಲ್ಯವನ್ನು ಮರೆಮಾಡಲು ಯತ್ನಿಸುತ್ತಿರುವನು. ಶ‍್ರೀಕೃಷ್ಣ ಇವುಗಳನ್ನು ಭೇದಿಸಿ ನೋಡುವನು. ಅರ್ಜುನ ಒಬ್ಬ ಪಂಡಿತನಂತೆ ಮಾತನಾಡುತ್ತ ಅನೇಕ ಕಾರಣಗಳನ್ನು ಒಡ್ಡುವನು. ಆದರೂ ಅವನು ತಿಳಿಗೇಡಿಯಂತೆ ಮಾತನಾಡುತ್ತಿರುವನು.

ಆಗ ಶ‍್ರೀಕೃಷ್ಣ ಹೇಳುವನು: “ಜ್ಞಾನಿ ಬದುಕಿರುವವರಿಗೂ ವ್ಯಥೆ ಪಡುವುದಿಲ್ಲ.\break ಸತ್ತವರಿಗೂ ವ್ಯಥೆ ಪಡುವುದಿಲ್ಲ. \enginline{(II, 11)} ನೀನು ಸಾಯಲಾರೆ. ನಾನೂ ಸಾಯಲಾರೆ. ನಾವಿಬ್ಬರೂ ಇಲ್ಲದ ಕಾಲ ಎಂದಿಗೂ ಇರಲಿಲ್ಲ. ನಾವಿಬ್ಬರೂ ಇರದ ಕಾಲ ಎಂದಿಗೂ ಬರುವುದಿಲ್ಲ. ಈ ಜನ್ಮದಲ್ಲಿ ಒಬ್ಬ ಬಾಲ್ಯದಿಂದ ಪ್ರಾರಂಭವಾಗಿ ಯೌವನ, ವೃದ್ಧಾಪ್ಯಗಳ ಮೂಲಕವಾಗಿ ಸಾಗಿ ಮರಣದಲ್ಲಿ ಬೇರೊಂದು ದೇಹವನ್ನು ಪಡೆಯುವನು. ಜ್ಞಾನಿಗಳು ಏತಕ್ಕೆ ವ್ಯಥೆ ಪಡುವರು? \enginline{(II, 12–13)} ನಿನ್ನನ್ನು ಮೆಟ್ಟಿರುವ ವ್ಯಾಮೋಹ ಎಲ್ಲಿಂದ ಪ್ರಾರಂಭವಾಯಿತು? ಅದು ಇಂದ್ರಿಯಗಳಲ್ಲಿ ಇದೆ. ಈ ಇಂದ್ರಿಯದ ಬಂಧನದಿಂದಲೇ ಜಗತ್ತು ಬರುವುದು. ಶೀತ ಉಷ್ಣ ಸುಖ ದುಃಖ ಇವುಗಳೆಲ್ಲ ಬಂದು ಹೋಗುವುವು. \enginline{(II, 14)} ವ್ಯಕ್ತಿಯು ಈ ಕ್ಷಣ ದುಃಖದಲ್ಲಿರುವನು, ಮರುಕ್ಷಣ ಸುಖವಾಗಿರುವನು. ಆದಕಾರಣ ಅವನು ಆತ್ಮನ ಸ್ಥಿತಿಯನ್ನು ಅರಿಯಲಾರ.”

“ಸದ್​ವಸ್ತು ಎಂದಿಗೂ ಇಲ್ಲದೇ ಇಲ್ಲ. ಅಸದ್​ವಸ್ತು ಎಂದಿಗೂ ಇರಲಿಲ್ಲ. ಈ\break ವಿಶ್ವವನ್ನೆಲ್ಲಾ ವ್ಯಾಪಿಸಿರುವ ಆದಿ ಸತ್ಯಕ್ಕೆ ಅಂತ್ಯವಿಲ್ಲ. ಇದು ನಿರ್ವಿಕಾರಿ. ಇದನ್ನು\break ಬದಲಾಯಿಸುವಂತಹುದು ಜಗತ್ತಿನಲ್ಲಿ ಯಾವುದೂ ಇಲ್ಲ. ಈ ದೇಹಕ್ಕೆ ಒಂದು ಆದಿ\break ಅಂತ್ಯವಿದ್ದರೂ ಈ ದೇಹದಲ್ಲಿರುವವನು ಅನಂತ, ಅನಾದಿ.” \enginline{II, 16–18.}

ಇದನ್ನೇ ಅರಿತು ಎದ್ದುನಿಂತು ಹೋರಾಡು, ಎಂದಿಗೂ ಒಂದು ಹೆಜ್ಜೆಯನ್ನೂ ಹಿಂದೆ ಇಡಬೇಡ. ಅದೇ ಆದರ್ಶ. ಏನಾದರೂ ಆಗಲಿ ಹೋರಾಡಬೇಕು. ನಕ್ಷತ್ರಗಳೇ ತಮ್ಮ\break ಗತಿಯನ್ನು ಬದಲಾಯಿಸಲಿ, ಇಡೀ ಜಗತ್ತೇ ನಮಗೆ ವಿರೋಧವಾಗಿ ನಿಲ್ಲಲಿ, ಚಿಂತೆಯಿಲ್ಲ. ಮರಣವೆಂದರೆ ಬಟ್ಟೆಯ ಬದಲಾವಣೆಯಂತಷ್ಟೆ. ಅದರಿಂದೇನು? ಹೋರಾಡಿ.\break ಹೇಡಿಗಳಾದರೆ ನೀವು ಏನನ್ನೂ ಸಾಧಿಸಲಾರಿರಿ. ಈ ಜಗತ್ತಿನ ದೇವರುಗಳನ್ನೆಲ್ಲ ನೀವು ಪ್ರಾರ್ಥಿಸುವಿರಿ. ನಿಮ್ಮ ದುಃಖ ಕೊನೆಗೊಂಡಿದೆಯೆ? ಭಾರತ ದೇಶದಲ್ಲಿ ಜನರು\break ಆರುಕೋಟಿ ದೇವರಿಗೆ ಗೋಳಿಡುವರು. ಆದರೂ ನಾಯಿಗಳಂತೆ ಸಾಯುವರು. ಈ ದೇವತೆಗಳು ಎಲ್ಲಿರುವರು? ನೀವು ಗೆದ್ದ ಮೇಲೆ ದೇವರು ನಿಮ್ಮ ಸಹಾಯಕ್ಕೆ ಬರುತ್ತಾನೆ. ಆದಕಾರಣ ಇದರಿಂದ ಏನು ಪ್ರಯೋಜನ? ಮೌಢ್ಯತೆಗೆ ಬಾಗುವುದು, ನಿಮ್ಮ ದುರ್ಬಲ ಮನಸ್ಸು ಹೇಳಿದಂತೆ ಕೇಳುವುದು, ಆತ್ಮನಿಗೆ ಯೋಗ್ಯವಲ್ಲ. ಇದೊಂದು ಸಾವಿನ ಆಟ...ನೀವು ಅನಂತಾತ್ಮರು, ಜನನ ಮರಣಾತೀತರು. ನೀವು ಅನಂತಾತ್ಮರಾಗಿರುವುದರಿಂದ ಗುಲಾಮರಾಗುವುದು ಯೋಗ್ಯವಲ್ಲ. ಉತ್ತಿಷ್ಠರಾಗಿ, ಜಾಗೃತರಾಗಿ, ಎದ್ದುನಿಂತು ಹೋರಾಡಿ. ಸಾಯಲೇಬೇಕಾದರೆ ಸಾಯಿರಿ. ಅದನ್ನು ಯಾರೂ ತಪ್ಪಿಸುವುದಕ್ಕೆ ಆಗುವುದಿಲ್ಲ. ನಿಮ್ಮ ಸಹಾಯಕ್ಕೆ ಇನ್ನು ಯಾರೂ ಇಲ್ಲ. ನೀವೆ ವಿಶ್ವವೆಲ್ಲ ಆಗಿರುವಿರಿ. ನಿಮಗೆ ಇನ್ನು ಯಾರು ಸಹಾಯ ಮಾಡಬಲ್ಲರು?

“ಜೀವಿಗಳು ಜನನಕ್ಕೆ ಮುಂಚೆ ಮತ್ತು ಮರಣಾ ನಂತರ ನಮಗೆ ಕಾಣುವುದಿಲ್ಲ. ಇವುಗಳ ಮಧ್ಯದಲ್ಲಿ ಮಾತ್ರ ಅವರು ಗೋಚರಿಸುವರು. ಇದಕ್ಕಾಗಿ ಏತಕ್ಕೆ ವ್ಯಥೆ ಪಡಬೇಕು.” \enginline{(II, 28)}

“ಇದನ್ನೇ ಆಶ್ಚರ್ಯವೆಂದು ಕೆಲವರು ಪರಿಗಣಿಸುವರು. ಕೆಲವರು ಆಶ್ಚರ್ಯವೆಂದು ಇದನ್ನು ಕುರಿತು ಮಾತನಾಡುವರು. ಮತ್ತೆ ಕೆಲವರು ಆಶ್ಚರ್ಯವೆಂದು ಇದನ್ನು\break ಕೇಳುವರು. ಇತರರು ಇದನ್ನು ಕೇಳಿಯೂ ತಿಳಿದುಕೊಳ್ಳಲಾರರು.” \enginline{(II, 29)}

ಇದನ್ನೆಲ್ಲ ಕೊಲ್ಲುವುದು ಪಾಪವೆಂದು ಭಾವಿಸಿದರೂ ನಿನ್ನ ವರ್ಣದ ಕರ್ತವ್ಯದ\break ದೃಷ್ಟಿಯಿಂದ ಇದನ್ನು ನೋಡಬೇಕಾಗಿದೆ. “ಸುಖದುಃಖಗಳನ್ನು ಒಂದೇ ಸಮನಾಗಿ ನೋಡಿ ಜಯಾಪಜಯಗಳನ್ನು ಸಮನಾಗಿ ಎದುರಿಸಿ ಎದ್ದು ನಿಂತು ಹೋರಾಡು.” \enginline{(II, 38)}

ಗೀತೆಯಲ್ಲಿ ಬರುವ ಮತ್ತೊಂದು ಸಿದ್ಧಾಂತದ ಪ್ರಾರಂಭ ಇದು. ಇದೇ ಅನಾಸಕ್ತಿಯೋಗ, ನಾವು ಫಲಗಳಲ್ಲಿ ಆಸಕ್ತರಾಗಿರುವುದರಿಂದ ವ್ಯಥೆ ಪಡುತ್ತೇವೆ ಎಂಬುದು.\break “ಕರ್ತವ್ಯಕ್ಕೆ ಕರ್ತವ್ಯವೆಂದು ನಾವು ಯಾವುದನ್ನು ಮಾಡಿರುವೆವೊ ಅದು ಮಾತ್ರ ನಮ್ಮ ಕರ್ಮಬಂಧನವನ್ನು ಬಿಡಿಸಬಲ್ಲದು” (\enginline{II, 39}), ಅದನ್ನು ಹೆಚ್ಚು ವೆಚ್ಚ ಮಾಡಿದರೆ ಯಾವ ಅಪಾಯವೂ ಇಲ್ಲ. “ನಾವು ಅದನ್ನು ಎಷ್ಟು ಸ್ವಲ್ಪ ಮಾಡಿದರೂ ನಮ್ಮನ್ನು ಮಹಾ ಭಯದಿಂದ (ಜನನ ಮರಣಗಳಿಂದ) ಪಾರು ಮಾಡುವುದು.” (\enginline{II, 40})

“ಅರ್ಜುನ, ಇದನ್ನು ತಿಳಿದುಕೊ ಯಾವ ಮನಸ್ಸನ್ನು ಏಕಾಗ್ರಗೊಳಿಸಿರುವೆವೊ ಅದೇ ಗೆಲ್ಲುವುದು. ಸುಮ್ಮನೆ ಎರಡು ಸಾವಿರ ವಸ್ತುಗಳನ್ನು ಮನಸ್ಸಿಗೆ ಹಚ್ಚಿಕೊಂಡರೆ ಮನಸ್ಸು ಛಿದ್ರಛಿದ್ರವಾಗಿ ಹೋಗುವುದು. ಕೆಲವರು ಆಲಂಕಾರಿಕವಾದ ಮಾತನ್ನಾಡಿ ವೇದಗಳಿಗೆ ಮೀರಿದುದು ಇನ್ನು ಯಾವುದೂ ಇಲ್ಲ ಎಂದು ಹೇಳುವರು. ಅವರು ಸ್ವರ್ಗಕ್ಕೆ ಹೋಗಲು ಇಚ್ಛಿಸುವರು. ವೇದಗಳ ಮೂಲಕ ಅವರಿಗೆ ಚೆನ್ನಾದ ವಸ್ತುಗಳು ಬೇಕು. ಅದಕ್ಕಾಗಿ ಯಜ್ಞಗಳನ್ನು ಮಾಡುವರು. ಅಂಥವರು ಇಂತಹ ಪ್ರಾಪಂಚಿಕ ಬಯಕೆಗಳನ್ನೆಲ್ಲ ತ್ಯಜಿಸುವವರೆಗೆ ಆಧ್ಯಾತ್ಮಿಕ ಜೀವನದಲ್ಲಿ ಎಂದಿಗೂ ಮುಂದುವರಿಯುವುದಿಲ್ಲ.” (\enginline{II, 41–44})

ಇದೊಂದು ಅತಿ ಉತ್ತಮವಾದ ಬೋಧನೆ. ಪ್ರಾಪಂಚಿಕ ಬಯಕೆಗಳನ್ನೆಲ್ಲ ತ್ಯಜಿಸುವ\-ವರೆಗೆ ಆಧ್ಯಾತ್ಮಿಕ ಸಂಪತ್ತು ನಮಗೆ ಸಿಕ್ಕಲಾರದು. ಇಂದ್ರಿಯಗಳಲ್ಲಿ ಏನಿದೆ? ಇಂದ್ರಿಯಗಳೆಲ್ಲ ಒಂದು ಭ್ರಮೆ. ತಾವು ಕಾಲವಾದ ಮೇಲೂ ಸ್ವರ್ಗದಲ್ಲಿ ಇಲ್ಲಿಯ ಕಣ್ಣುಗಳು, ಮೂಗುಗಳು ಇವನ್ನು ಇಟ್ಟುಕೊಂಡಿರಲು ಆಶಿಸುವ ಕೆಲವರು, ತಮಗೆ, ಈಗ ಇರುವುದಕ್ಕಿಂತ ಹೆಚ್ಚು ಇಂದ್ರಿಯಗಳು ಸ್ವರ್ಗದಲ್ಲಿ ಇರುತ್ತವೆ ಎಂದು ಕಲ್ಪಿಸಿಕೊಳ್ಳುವರು. ದೇವರು ತನ್ನ ಸ್ಥೂಲ ದೇಹದಲ್ಲಿ ಸಿಂಹಾಸನದ ಮೇಲೆ ಕುಳಿತುಕೊಂಡು ಇರುವುದನ್ನೇ ಶಾಶ್ವತವಾಗಿ ನೋಡಲು ಆಶಿಸುವರು. ಅಂತಹ ಮನುಷ್ಯನ ಬಯಕೆಗಳೆಲ್ಲ ದೇಹ, ಆಹಾರ, ಕುಡಿತ ಮತ್ತು ಇಂದ್ರಿಯಸುಖ ಇವೇ. ಇದು ಮುಂದುವರಿಸಿದ ಭೌತಿಕ ಜೀವನ.\break ಇದರಾಚೆ ಮನುಷ್ಯ ಏನನ್ನೂ ಆಲೋಚಿಸಲಾರ. ಈ ಜೀವನವೆಲ್ಲ ಇರುವುದು ಈ\break ದೇಹಕ್ಕಾಗಿಯೇ. “ಯಾವುದು ನಮ್ಮನ್ನು ಮುಕ್ತರನ್ನಾಗಿ ಮಾಡುವುದೋ ಅಂತಹ ಏಕಾಗ್ರತೆ ಇವನಿಗೆ ದೊರಕಲಾರದು.” (\enginline{II, 44})

“ವೇದಗಳು ಸತ್ತ್ವ, ರಜಸ್ಸು, ತಮೋಗುಣಗಳೆಂಬ ಮೂರನ್ನು ಕುರಿತ ವಿಷಯಗಳನ್ನು ಮಾತ್ರ ಹೇಳುತ್ತವೆ.” (\enginline{II, 45}) ವೇದಗಳು ಪ್ರಕೃತಿಗೆ ಸಂಬಂಧಪಟ್ಟದ್ದನ್ನೇ\break ಹೇಳುತ್ತವೆ. ಜನರು ಈ ಜಗತ್ತಿನಲ್ಲಿ ಏನನ್ನು ನೋಡುವರೋ ಅದನ್ನಲ್ಲದೇ ಬೇರೆ\break ವಸ್ತುವನ್ನು ಆಲೋಚಿಸಲಾರರು. ಅವರೇನಾದರೂ ಸ್ವರ್ಗದ ವಿಷಯವನ್ನು ಮಾತನಾಡಿದರೆ ಒಬ್ಬ ರಾಜ ಸಿಂಹಾಸನದ ಮೇಲೆ ಕುಳಿತಿರುವುದು, ಅವನ ಮುಂದೆ ಜನ ಧೂಪ ಹಾಕುತ್ತಿರುವುದು ಇಷ್ಟು ಮಾತ್ರ ಯೋಚಿಸುತ್ತಾರೆ. ಇವೆಲ್ಲ ಪ್ರಕೃತಿ. ಇದಲ್ಲದೆ ಬೇರೆ ಅಲ್ಲ. ವೇದಗಳು ಪ್ರಕೃತಿಯ ವಿನಃ ಬೇರೆ ಯಾವುದನ್ನೂ ಬೋಧಿಸುವುದಿಲ್ಲ. ನೀನು ಪ್ರಕೃತಿಗೆ ಅತೀತನಾಗಬೇಕು. ಜೀವನದ ದ್ವಂದ್ವಗಳಿಂದ ಪಾರಾಗಬೇಕು. ನಿನ್ನ ಪ್ರಜ್ಞೆಗೂ ಅತೀತನಾಗಬೇಕು. ಪಾಪಪುಣ್ಯಗಳಾವುದನ್ನೂ ಲೆಕ್ಕಿಸದೇ ಮುಂದುವರಿಯಬೇಕು.

ನಾವು ದೇಹವೇ ನಾವೆಂಬ ಭಾವನೆಗೆ ದಾಸರಾಗಿರುವೆವು. ನಾವು ಬರೀ ದೇಹ, ಅಥವಾ ದೇಹದಿಂದ ಮೆಟ್ಟಿಕೊಂಡಿರುವವರು. ನಮ್ಮನ್ನು ಯಾರಾದರೂ ಜಿಗುಟಿದರೆ ಸಾಕು, ಅರಚುವೆವು. ಇದೆಲ್ಲ ಅವಿವೇಕ. ಏಕೆಂದರೆ ನಾನೊಂದು ಆತ್ಮ. ನಾವು ದೇಹ ಭಾವನೆಗೆ ದಾಸರಾಗಿರುವುದರಿಂದಲೇ ಈ ಸಂಕಟದ ಪರಂಪರೆ. ಪ್ರಾಣಿಗಳು, ದೇವರು ದೈತ್ಯರು ಎಂಬ ಕಲ್ಪನೆಯ ಪ್ರಪಂಚವೆಲ್ಲ ದೇಹಭಾವನೆಗೆ ದಾಸರಾಗಿರುವುದರಿಂದ ಬರುವುದು. ನಾನು ಆತ್ಮ, ನೀವು ನನ್ನನ್ನು ಜಿಗುಟಿದಾಗ ನಾನೇಕೆ ನೆಗೆದಾಡಬೇಕು? ಈ ಗುಲಾಮಗಿರಿಯನ್ನು ನೋಡಿ ನಿಮಗೆ ನಾಚಿಕೆಯಾಗುವುದಿಲ್ಲವೇ? ನಾವೇನೋ ಧಾರ್ಮಿಕರೇ ಸರಿ! ನಾವೇನೋ ದೊಡ್ಡ ತತ್ತ್ವಜ್ಞಾನಿಗಳೇ ಸರಿ! ನಾವೇನೊ ದೊಡ್ಡ ಋಷಿಗಳು! ದೇವರು ನಮ್ಮನ್ನು ರಕ್ಷಿಸಬೇಕು. ನಾವಾರು? ನಾವೇ ಜೀವಂತ ನರಕ. ಇಷ್ಟೇ ನಾವು. ಹುಚ್ಚರು ಇಷ್ಟೇ ನಾವು!

ನಾವು ದೇಹಭಾವನೆಯನ್ನು ತ್ಯಜಿಸಲಾರೆವು. ಪ್ರಪಂಚದ ದಾಸರು ನಾವು.\break ಸ್ಮಶಾನವೇ ನಮ್ಮ ಭಾವನೆ. ನಾವು ದೇಹವನ್ನು ತ್ಯಜಿಸಿದ ಮೇಲೂ ಸಾವಿರಾರು\break ಆಸೆಗಳಿಂದ ಬದ್ಧರಾಗಿರುವೆವು.

ಆಸಕ್ತಿಯಿಲ್ಲದೇ ಯಾರು ಕರ್ಮವನ್ನು ಮಾಡಬಲ್ಲರು? ಅದೇ ನಿಜವಾದ ಪ್ರಶ್ನೆ.\break ಆಸಕ್ತಿಯಿಲ್ಲದೆ ಕರ್ಮ ಮಾಡಬಲ್ಲವನು ತಾನು ಇಡೀ ಜೀವನದಲ್ಲಿ ಮಾಡಿದ ಕೆಲಸವೆಲ್ಲ ಕ್ಷಣದಲ್ಲಿ ಭಸ್ಮೀಭೂತವಾದರೂ ಸ್ವಲ್ಪವೂ ಗಮನಿಸುವುದಿಲ್ಲ. ತನ್ನ ಕೆಲಸವು ಯಶಸ್ಸನ್ನು\break ಪಡೆಯಲಿ ಬಿಡಲಿ ಅವನು ಗಮನಿಸುವುದಿಲ್ಲ. “ಫಲಾಪೇಕ್ಷೆಯಿಲ್ಲದೆ ಕೆಲಸ ಮಾಡುವ ಮಹಾತ್ಮನೇ ಇವನು. ಇವನು ಜನನ ಮರಣಗಳ ಯಾತನೆಯಿಂದ ಹೀಗೆ ಪಾರಾಗುವನು. ಇವನು ಹೀಗೆ ಮುಕ್ತನಾಗುವನು” (\enginline{II, 51}). ಆಗ ಆಸಕ್ತಿಯೆಲ್ಲ ಒಂದು ಭ್ರಮೆ ಎನ್ನುವುದನ್ನು ನೋಡುವನು. ಆತ್ಮ ಎಂದಿಗೂ ಆಸಕ್ತವಾಗಿರಲಾರದು. ಅವನು ಆಗ ಎಲ್ಲಾ ಶಾಸ್ತ್ರಗಳಿಗೆ, ತತ್ತ್ವಗಳಿಗೆ ಅತೀತನಾಗಿ ಹೋಗುವನು. ಮನಸ್ಸು ಭ್ರಾಂತಿಗೊಳಗಾಗಿ ಗ್ರಂಥಗಳು ಮತ್ತು ಶಾಸ್ತ್ರಗಳಿಗೆ ಸಿಕ್ಕಿ ನರಳಿದರೆ, ಇಂತಹ ಶಾಸ್ತ್ರಗಳಿಂದ ಏನು ಪ್ರಯೋಜನ? ಒಂದು ಶಾಸ್ತ್ರ ಹೀಗೆ ಹೇಳುವುದು. ಮತ್ತೊಂದು ಶಾಸ್ತ್ರ ಹಾಗೆ ಹೇಳುವುದು. ನೀನು ಯಾವ ಶಾಸ್ತ್ರವನ್ನು ಅನುಸರಿಸುವೆ? ನೀನು ಏಕಾಂಗಿಯಾಗಿ ನಿಲ್ಲು. ಆತ್ಮನ ಮಹಿಮೆಯನ್ನು ನೋಡು. ಆಗ ಮಾತ್ರ ನೀನು ಸ್ಥಿತಪ್ರಜ್ಞನಾಗುವೆ.

ಅರ್ಜುನನು ಸ್ಥಿತಪ್ರಜ್ಞನಾರು ಎಂದು ಪ್ರಶ್ನಿಸುವನು. ಶ‍್ರೀಕೃಷ್ಣ ಅದಕ್ಕೆ ಈ ಉತ್ತರವನ್ನು ಹೇಳುವನು: “ಯಾರು ಎಲ್ಲ ಬಯಕೆಗಳನ್ನು ತ್ಯಜಿಸಿದ್ದಾನೆಯೋ, ಯಾರು ಏನನ್ನೂ, ಈ ಜೀವನವನ್ನೂ ಬಯಸುವುದಿಲ್ಲವೋ, ಸ್ವಾತಂತ್ರ್ಯವನ್ನಾಗಲೀ, ದೇವತೆಗಳನ್ನಾಗಲಿ,\break ಕೆಲಸವನ್ನಾಗಲಿ ಅಥವಾ ಮತ್ತೇನನ್ನೇ ಆಗಲಿ ಬಯಸುವುದಿಲ್ಲವೋ ಅವನೇ ಸ್ಥಿತಪ್ರಜ್ಞ. ಅವನು ಆತ್ಮತೃಪ್ತನಾದ ಮೇಲೆ ಅವನಿಗೆ ಇನ್ನು ಮೇಲೆ ಯಾವ ಬಯಕೆಗಳೂ ಇಲ್ಲ”\break (\enginline{II, 55}). ಅವನು ಆತ್ಮನ ಮಹಿಮೆಯನ್ನು ಅರಿತಿರುವನು. ಪ್ರಪಂಚ, ದೇವರುಗಳು, ಸ್ವರ್ಗ ಇವೆಲ್ಲ ತನ್ನಾತ್ಮನಲ್ಲಿದೆ ಎಂಬುದನ್ನು ಅರಿತಿರುವನು. ಆಗ ದೇವತೆಗಳು\break ದೇವತೆಗಳೇ ಅಲ್ಲ. ಸಾವುಗಳು ಸಾವೇ ಅಲ್ಲ. ಜೀವನವು ಜೀವನವೇ ಅಲ್ಲ.\break ಪ್ರತಿಯೊಂದೂ ರೂಪಾಂತರ ಹೊಂದುವುದು. “ಯಾರ ಇಚ್ಛೆ ದೃಢವಾಗಿರುವುದೋ, ಯಾರ ಮನಸ್ಸು ದುಃಖದಿಂದ ವಿಚಲಿತವಾಗುವುದಿಲ್ಲವೋ, ಯಾರು ಯಾವ ಸುಖವನ್ನೂ ಆಶಿಸುವುದಿಲ್ಲವೋ, ಯಾರು ಎಲ್ಲಾ ಆಸಕ್ತಿ, ಭಯ, ಕ್ರೋಧಗಳಿಂದ ಪಾರಾಗಿರುವನೋ ಅವನು ಜ್ಞಾನಿ ಎನಿಸುವನು” (\enginline{II, 56}),.

“ಆಮೆ ತನ್ನ ಅಂಗಾಂಗಗಳನ್ನು ಒಳಗೆ ಸೆಳೆದುಕೊಳ್ಳುವುದು. ಅದನ್ನು ಹೊಡೆದರೆ ಅದರ ಒಂದು ಕಾಲೂ ಹೊರಗೆ ಬರುವುದಿಲ್ಲ. ಇದರಂತೆ ಜ್ಞಾನಿ ಕೂಡ ಇಂದ್ರಿಯಗಳನ್ನೆಲ್ಲ ಒಳಗೆ ಸೆಳೆದುಕೊಳ್ಳುವನು” (\enginline{II, 58}). ಅವುಗಳನ್ನು ಯಾವುದೂ ಹೊರಗೆ ಬರುವಂತೆ ಬಲಾತ್ಕರಿಸಲಾರವು. ಯಾವ ಪ್ರಲೋಭನೆಯಾಗಲಿ ಅವನನ್ನು ವಿಚಲಿತನನ್ನಾಗಿ ಮಾಡಲಾರದು. ಈ ಪ್ರಪಂಚವೇ ಅವನ ಮುಂದೆ ಪುಡಿ ಪುಡಿಯಾದರೂ ಅವನು ವಿಚಲಿತಗೊಳ್ಳುವುದಿಲ್ಲ.

ಅನಂತರ ಬಹಳ ಮುಖ್ಯವಾದ ಪ್ರಶ್ನೆ ಬರುವುದು. ಕೆಲವು ವೇಳೆ ಜನರು ಹಲವು\break ದಿವಸಗಳು ಉಪವಾಸ ಮಾಡುವರು. ಅವನು ತುಂಬಾ ದುರಾತ್ಮನಾದರೂ ಇಪ್ಪತ್ತು ದಿವಸ ಉಪವಾಸ ಮಾಡಿದರೆ ಅವನು ತುಂಬಾ ಮೃದುವಾಗುತ್ತಾನೆ. ಉಪವಾಸ ಮತ್ತು ದೇಹದಂಡನೆ ಇವು ಪ್ರಪಂಚದಲ್ಲೆಲ್ಲ ಜನರು ಅಭ್ಯಾಸ ಮಾಡಿರುವರು. ಕೃಷ್ಣನ ಭಾವನೆ\break ಇವೆಲ್ಲ ಅವಿವೇಕ ಎಂಬುದು. ಯಾರು ತನ್ನ ದೇಹವನ್ನು ದಂಡಿಸಿಕೊಳ್ಳುವನೋ\break ಅವನಿಂದ ಇಂದ್ರಿಯಗಳು ಕೆಲವು ಕಾಲ ಹಿಂದೆ ಸರಿಯುತ್ತವೆ. ಆದರೆ ಅನಂತರ ಅವು\break ಹಿಂದಿಗಿಂತ ಇಪ್ಪತ್ತರಷ್ಟು ಬಲವಾಗಿ ಮೇಲೇಳುತ್ತವೆ. ಆಗ ನೀವು ಏನು ಮಾಡುವಿರಿ? ಯಾವ ಉಗ್ರ ದೇಹದಂಡನೆಯೂ ಇಲ್ಲದೆ ಸ್ವಾಭಾವಿಕವಾಗಿರುವುದನ್ನು ಅಭ್ಯಾಸ ಮಾಡಿ. ಕೆಲಸ ಮಾಡುತ್ತ ಹೋಗಿ. ಆಸಕ್ತರಾಗದಂತೆ ಮಾತ್ರ ನೋಡಿಕೊಳ್ಳಿ. ಅನಾಸಕ್ತಿಯ ರಹಸ್ಯವನ್ನು ಯಾರು ಅರಿತುಕೊಂಡು ಅದನ್ನು ಅಭ್ಯಾಸ ಮಾಡಿಲ್ಲವೊ ಅವನು ತನ್ನ ಇಚ್ಛೆಯನ್ನು\break ಬಲವಾಗಿ ನಿಗ್ರಹಿಸಲಾರ.

ನಾನು ಹೊರಗೆ ಹೋಗಿ ಕಣ್ಣನ್ನು ತೆರೆಯುತ್ತೇನೆ. ಅಲ್ಲಿ ಯಾರಾದರೂ ಇದ್ದರೆ ನಾನು ಅವರನ್ನು ನೋಡಬೇಕು. ನೋಡದೇ ವಿಧಿಯಿಲ್ಲ. ಮನಸ್ಸು ಪಂಚೇಂದ್ರಿಯಗಳ ಕಡೆ ಧಾವಿಸುವುದು. ಇನ್ನು ಮೇಲೆ ಇಂದ್ರಿಯಗಳು ಪ್ರಕೃತಿಗೆ ಪ್ರತಿಕ್ರಿಯಿಸುವುದನ್ನು ತ್ಯಜಿಸಬೇಕು.

“ಯಾವುದು ಅಜ್ಞಾನಿಗೆ ರಾತ್ರಿಯೋ ಆಗ ಜ್ಞಾನಿಯು ಜಾಗ್ರತನಾಗಿರುವನು. ಅವನಿಗೆ ಆಗ ಹಗಲು. ಎಲ್ಲಿ ಪ್ರಪಂಚ ಎಚ್ಚೆತ್ತಿರುವುದೋ ಅಲ್ಲಿ ಜ್ಞಾನಿ ನಿದ್ರಿಸುತ್ತಿರುವನು” (\enginline{II, 69}). ಪ್ರಪಂಚ ಎಲ್ಲಿ ಜಾಗ್ರತವಾಗಿರುವುದು? ಇಂದ್ರಿಯಗಳಲ್ಲಿ. ಜನ ತಿನ್ನುವುದು ಕುಡಿಯುವುದು, ಮಕ್ಕಳನ್ನು ಪಡೆಯುವುದು ಇದರಲ್ಲಿ ಮಗ್ನರಾಗಿದ್ದಾರೆ. ಬಳಿಕ ಅವರು ಯಾವಾಗಲೂ ಇಂದ್ರಿಯ ಸುಖಕ್ಕೆ ಜಾಗ್ರತರಾಗಿರುವರು. ಅವರಿಗೆ ಧರ್ಮ ಕೂಡ ಇಷ್ಟೇ. ಅವರಿಗೆ ಸಹಾಯ ಮಾಡುವ, ಹೆಚ್ಚು ಜನ ಸ್ತ್ರೀಯರು, ದ್ರವ್ಯ ಮತ್ತು ಮಕ್ಕಳನ್ನು ಕೊಡುವ ದೇವರನ್ನು ಅವರು ಕಲ್ಪಿಸಿಕೊಳ್ಳುವರು. ದೇವರಂತೆ ಆಗುವುದಕ್ಕೆ ಸಹಾಯ ಮಾಡುವ ದೇವರನ್ನು ಕಲ್ಪಿಸಿಕೊಳ್ಳುವುದಿಲ್ಲ. “ಪ್ರಪಂಚವೇ ಜಾಗ್ರತವಾಗಿರುವಾಗ ಜ್ಞಾನಿ ನಿದ್ರಿಸುವನು. ಆದರೆ ಜ್ಞಾನಿಗಳು ಎಲ್ಲಿ ಮಲಗಿರುವರೋ ಅಲ್ಲಿ ಅಜ್ಞಾನಿಗಳು ಎಚ್ಚೆತ್ತಿರುವರು. ಬೆಳಕಿನ ಪ್ರಪಂಚದಲ್ಲಿ ಮಾನವ ತಾನು ಪಶು, ಪಕ್ಷಿ, ದೇಹವೆಂದು ನೋಡದೆ, ಜನನ ಮರಣಾತೀತ ಅನಂತಾತ್ಮನಂತೆ ಭಾವಿಸುವನು. ಅಜ್ಞಾನಿಗಳು ಎಲ್ಲಿ ನಿದ್ರಿಸುವರೋ ಎಲ್ಲಿ ಅವರಿಗೆ ವಿಚಾರ ಮಾಡಲು, ಆಲೋಚನೆ ಮಾಡಲು ಸಮಯವಿಲ್ಲವೋ ಅಲ್ಲಿ ಜ್ಞಾನಿ ಜಾಗ್ರತನಾಗಿರುವನು. ಅದೇ ಹಗಲು ಅವನಿಗೆ.

“ಜಗತ್ತಿನ ನದಿಗಳೆಲ್ಲ ಅನವರತ ಸಾಗರಕ್ಕೆ ತಮ್ಮ ನೀರನ್ನೆಲ್ಲ ಧಾರೆಯೆರೆಯುತ್ತಿದ್ದರೂ ಹೇಗೆ ಸಾಗರದ ಮಹಾಗಂಭೀರ ಸ್ವಭಾವ ಅಲ್ಲೋಲಕಲ್ಲೋಲವಾಗುವುದಿಲ್ಲವೋ,\break ಸ್ವಲ್ಪವೂ ಬದಲಾಯಿಸುವುದಿಲ್ಲವೋ, ಅದರಂತೆ ಇಂದ್ರಿಯಗಳೆಲ್ಲ ಪ್ರಕೃತಿಯಿಂದ\break ಸುದ್ಧಿ ಸಮಾಚಾರವನ್ನು ಸಾಗರದಂತೆ ಇರುವ ಜ್ಞಾನಿಯ ಹೃದಯಕ್ಕೆ ತರುತ್ತಿದ್ದರೂ ಅಲ್ಲಿ ಯಾವ ಚಾಂಚಲ್ಯವೂ ಇಲ್ಲ, ಯಾವ ಅಂಜಿಕೆಯೂ ಇಲ್ಲ” (\enginline{II, 70}). ಕೋಟ್ಯಂತರ\break ನದಿಗಳ ಮೂಲಕ ದುಃಖ ಬರಲಿ, ನೂರಾರು ನದಿಗಳ ಮೂಲಕ ಸುಖ ಬರಲಿ! ನಾನು ದುಃಖಕ್ಕೆ ಗುಲಾಮನಲ್ಲ, ಸುಖಕ್ಕೆ ಗುಲಾಮನಲ್ಲ.

