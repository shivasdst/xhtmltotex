
\vspace{-0.6cm}

\chapter[ಮುಕ್ತರಾಗುವುದು ಹೇಗೆ? ]{ಮುಕ್ತರಾಗುವುದು ಹೇಗೆ? \protect\footnote{\engfoot{C.W. Vol. VI, P. 92}}}

ಪ್ರಕೃತಿಯಲ್ಲಿರುವ ಪ್ರತಿಯೊಂದೂ ನಿಯಮಕ್ಕೆ ಅನುಸಾರವಾಗಿ ನಡೆಯುವುದು. ಯಾವುದಕ್ಕೂ ಅದರಿಂದ ವಿನಾಯಿತಿಯಿಲ್ಲ. ಮನಸ್ಸು ಮತ್ತು ಬಾಹ್ಯ ಪ್ರಕೃತಿಯಲ್ಲಿರುವ ಪ್ರತಿಯೊಂದು ವಸ್ತುವೂ ನಿಯಮಕ್ಕೆ ಅಧೀನವಾಗಿರುವುದು.

\eject

ಬಾಹ್ಯ ಮತ್ತು ಆಂತರಿಕ ಪ್ರಕೃತಿ, ಮನಸ್ಸು ಮತ್ತು ದ್ರವ್ಯ, ಇವೆಲ್ಲವೂ ದೇಶ\break ಕಾಲಗಳಲ್ಲಿವೆ, ನಿಯಮಕ್ಕೆ ಬದ್ಧವಾಗಿವೆ. ಮಾನಸಿಕ ಸ್ವಾತಂತ್ರ್ಯವೆಂಬುದೊಂದು ಭ್ರಾಂತಿ. ನಿಯಮಕ್ಕೆ ಅಧೀನವಾಗಿರುವಾಗ, ಅದಕ್ಕೆ ಬದ್ಧವಾಗಿರುವಾಗ, ಮನಸ್ಸು ಹೇಗೆ ಸ್ವತಂತ್ರವಾಗಿರಬಲ್ಲದು? ಕರ್ಮನಿಯಮವೇ ಒಂದು ನಿಯಮ ಜಾಲ. ನಾವು ಮುಕ್ತರಾಗಬೇಕು. ನಾವು ಆಗಲೇ ಮುಕ್ತರು. ನಾವು ಅದನ್ನು ಅರಿಯಬೇಕಾಗಿದೆ ಅಷ್ಟೆ. ಎಲ್ಲಾ ಬಗೆಯ ದಾಸ್ಯವನ್ನೂ, ಬಂಧನವನ್ನೂ ನಾವು ತ್ಯಜಿಸಬೇಕು. ಪ್ರಪಂಚದ ವಸ್ತುಗಳ ಮತ್ತು ವ್ಯಕ್ತಿಗಳ ಬಂಧನವನ್ನು ತೊರೆಯುವುದು ಮಾತ್ರವಲ್ಲ, ಸುಖಗಳ ಬಂಧನವನ್ನೂ ತೊರೆಯಬೇಕು. ನಾವು ಆಸೆಯಿಂದ ಪ್ರಪಂಚಕ್ಕೆ ದಾಸರು; ಅದರಂತೆಯೇ ದೇವರು, ದೇವತೆಗಳು, ಸ್ವರ್ಗ ಮುಂತಾದವುಗಳಿಗೂ ದಾಸರು. ದೇವರಿಗಾಗಲಿ, ದೇವತೆಗಳಿಗಾಗಲಿ, ಮನುಷ್ಯನಿಗಾಗಲಿ ದಾಸನಾದವನು ದಾಸನೇ.

\vskip 0.1cm

ಸ್ವರ್ಗದ ಭಾವನೆ ತೊಲಗಬೇಕು. ಕಾಲವಾದ ಮೇಲೆ ಪುಣ್ಯಾತ್ಮರು ಎಂದೆಂದಿಗೂ ಸುಖಿಗಳಾಗಿರಬಹುದಾದ ಸ್ವರ್ಗಲೋಕವೆಂಬುದು ಒಂದು ಇದೆ ಎಂಬುದು ಭ್ರಮೆ. ಇದರಲ್ಲಿ ಎಳ್ಳಷ್ಟೂ ಸತ್ಯವಿಲ್ಲ. ಇದಕ್ಕೆ ಅರ್ಥವೇ ಇಲ್ಲ. ಎಲ್ಲಿ ಸುಖವಿದೆಯೋ ಅಲ್ಲಿ ದುಃಖ ಒಂದಲ್ಲ ಒಂದು ವೇಳೆ ಬರಲೇಬೇಕು. ಸಂತೋಷ ಇರುವ ಕಡೆ ವ್ಯಸನವೂ ಇರಬೇಕು. ಇದು ನಿಸ್ಸಂದೇಹವಾಗಿ ಸತ್ಯ. ಪ್ರತಿಯೊಂದು ಕ್ರಿಯೆಗೂ ಹೇಗೋ ಒಂದು ಪ್ರತಿಕ್ರಿಯೆ ಇದ್ದೇ ಇದೆ.

\vskip 0.1cm

ಸ್ವಾತಂತ್ರ್ಯವೇ ಮುಕ್ತಿಯ ನಿಜವಾದ ಭಾವನೆ. ಎಲ್ಲದರಿಂದಲೂ ಸ್ವತಂತ್ರರಾಗಬೇಕು. ಇಂದ್ರಿಯಗಳ ಪಾಶದಿಂದ ಪಾರಾಗಬೇಕು. ಅದು ಸುಖವಾಗಲಿ, ದುಃಖವಾಗಲಿ, ಒಳ್ಳೆಯದಾಗಲಿ, ಕೆಟ್ಟದ್ದಾಗಲಿ ಎಲ್ಲದರಿಂದಲೂ ಪಾರಾಗಬೇಕು.

\vskip 0.1cm

ಇದಕ್ಕಿಂತಲೂ ಹೆಚ್ಚಾಗಿ ನಾವು ಮರಣದಿಂದ ಮುಕ್ತರಾಗಬೇಕಾದರೆ ಜೀವನದಿಂದಲೂ ಮುಕ್ತರಾಗಬೇಕು. ಜೀವನವೆಂಬುದು ಮರಣದ ಒಂದು ಕನಸು. ಜೀವನ ಇರುವ ಕಡೆ ಮರಣ ಇರಲೇಬೇಕು. ನೀವು ಜೀವನದಿಂದ ಪಾರಾದಾಗ ಮಾತ್ರ ಮರಣದಿಂದ\break ಪಾರಾಗುವಿರಿ.

\vskip 0.1cm

ನಮಗೆ ಸಾಕಷ್ಟು ಶ್ರದ್ಧೆ ಇದ್ದರೆ, ಸಾಕಷ್ಟು ನಂಬಿಕೆ ಇದ್ದರೆ, ನಾವು ನಿತ್ಯ ಮುಕ್ತರು. ನೀವು ಆತ್ಮ, ಸ್ವತಂತ್ರರು, ಜನನಮರಣಾತೀತರು, ನಿತ್ಯಮುಕ್ತರು, ನಿತ್ಯಧನ್ಯರು. ನಿಮ್ಮಲ್ಲಿ ಸಾಕಷ್ಟು ಶ್ರದ್ಧೆ ಇರಲಿ, ನೀವು ಒಂದು ಕ್ಷಣದಲ್ಲಿ ಮುಕ್ತರಾಗುವಿರಿ.

\vskip 0.1cm

ದೇಶಕಾಲನಿಮಿತ್ತದಲ್ಲಿರುವ ಪ್ರತಿಯೊಂದೂ ಬದ್ಧ. ಆತ್ಮ ದೇಶ-ಕಾಲ ನಿಮಿತ್ತಾತೀತ. ಬದ್ಧವಾಗಿರುವುದು ಪ್ರಕೃತಿಯೇ ಹೊರತು ಆತ್ಮನಲ್ಲ. ಆದಕಾರಣ ನಿಮ್ಮ ಸ್ವಾತಂತ್ರ್ಯವನ್ನು ಘೋಷಿಸಿ; ನಿತ್ಯಮುಕ್ತರಾಗಿ, ನಿತ್ಯಧನ್ಯರಾಗಿ. ದೇಶಕಾಲನಿಮಿತ್ತವನ್ನೇ ನಾವು ಮಾಯೆ ಎನ್ನುವುದು.

