

\part{ಉಪನ್ಯಾಸಗಳು}

\mainmatter

\chapter[ಈಶ್ವರಾನುರಾಗ]{ಈಶ್ವರಾನುರಾಗ \protect\footnote{\engfoot{C.W. Vol. II. P. 38}}}

ಕೆಲವು ಧರ್ಮಗಳಲ್ಲಿ ವಿನಹ ಉಳಿದ ಎಲ್ಲಾ ಧರ್ಮಗಳಲ್ಲಿಯೂ ಸಗುಣ ದೇವರಿಗೆ ಸ್ಥಾನವಿದೆ. ಜೈನ ಮತ್ತು ಬೌದ್ಧ ಧರ್ಮಗಳನ್ನು ಬಿಟ್ಟರೆ ಬಹುಶಃ ಪ್ರಪಂಚದಲ್ಲಿರುವ ಉಳಿದ ಧರ್ಮಗಳಲ್ಲೆಲ್ಲಾ ಸಗುಣ ದೇವರ ಭಾವನೆ ಇರುವುದು. ಈ ಭಾವನೆಯಿಂದ ಭಕ್ತಿ, ಪೂಜೆ\break ಮುಂತಾದವು ಜನಿಸುವುವು. ಬೌದ್ಧರಲ್ಲಿ ಮತ್ತು ಜೈನರಲ್ಲಿ ಸಗುಣ ದೇವರು ಇಲ್ಲದೇ ಇದ್ದರೂ, ತಮ್ಮ ಧರ್ಮಸ್ಥಾಪಕರನ್ನು, ಇತರರು ಯಾವ ಭಾವನೆಯಿಂದ ದೇವರನ್ನು ಆರಾಧಿಸುವರೋ, ಅದರಂತೆಯೇ ಆರಾಧಿಸುವರು. ನಮ್ಮ ಪ್ರೇಮವನ್ನು ಪ್ರತಿಬಿಂಬಿಸುವ, ನಮಗಿಂತ ಮಿಗಿಲಾದ ವ್ಯಕ್ತಿಗೆ ಭಕ್ತಿ ಗೌರವಗಳನ್ನು ತೋರಿಸುವುದು ಸಾರ್ವತ್ರಿಕ\-ವಾದುದು. ಯಾವುದಾದರೊಂದು ವ್ಯಕ್ತಿಗೆ ನಾವು ತೋರುವ ಭಕ್ತಿ, ಪೂಜೆ ಎಲ್ಲರಲ್ಲಿಯೂ ಇವೆ. ಈ ಪ್ರೀತಿ ಮತ್ತು ಭಕ್ತಿ ಬೇರೆ ಬೇರೆ ಧರ್ಮಗಳಲ್ಲಿ, ಬೇರೆ ಬೇರೆ ಪ್ರಮಾಣಗಳಲ್ಲಿ ಬೇರೆ ಬೇರೆ ಹಂತಗಳಲ್ಲಿ ಅಭಿವ್ಯಕ್ತವಾಗಿವೆ. ಅತ್ಯಂತ ಕೆಳಗಿನ ಹಂತವೇ ಬಾಹ್ಯಾಚಾರಗಳು. ಅಲ್ಲಿ ಅಮೂರ್ತವಾದ ಭಾವನೆಗಳು ಅಸಾಧ್ಯ. ಅಲ್ಲಿ ಅಮೂರ್ತವನ್ನು ತುಂಬಾ ಕೆಳಕ್ಕೆ ಎಳೆದು ಅತಿ ಸ್ಥೂಲವಾಗಿ ಮಾಡಿರುವರು. ಹಲವು ಆಕಾರಗಳು ರೂಢಿಗೆ ಬರುವುವು. ಜತೆಗೆ ಹಲವು ಸಂಕೇತಗಳು ಬರುವುವು. ಜಗತ್ತಿನ ಇತಿಹಾಸದಲ್ಲೆಲ್ಲಾ ಮಾನವನು ಸಂಕೇತಗಳ ಮೂಲಕ ಅಮೂರ್ತವಾದುದನ್ನು ಗ್ರಹಿಸಲು ಪ್ರಯತ್ನಿಸುವುದನ್ನು ಕಾಣಬಹುದು. ಧರ್ಮದ ಬಾಹ್ಯ ಅಭಿವ್ಯಕ್ತಿಗಳಾದ ಗಂಟೆ, ಸಂಗೀತ, ಪೂಜೆ, ಶಾಸ್ತ್ರ, ವಿಗ್ರಹ, ಇವುಗಳೆಲ್ಲಾ ಅದರ ಕೆಳಗೆ ಬರುವುವು. ಯಾವುದು ನಮ್ಮ ಇಂದ್ರಿಯಗಳಿಗೆ ಹಿತವಾಗಿರುತ್ತದೋ, ಸೂಕ್ಷ್ಮವನ್ನು ಸ್ಥೂಲವಾಗಿ ಗ್ರಹಿಸುವುದಕ್ಕೆ ಯಾವುದು ಸಹಾಯ ಮಾಡುವುದೋ, ಅದನ್ನು ಸ್ವೀಕರಿಸಿ ಪೂಜಿಸುವರು.

ಕಾಲಕಾಲಕ್ಕೆ ಎಲ್ಲಾ ಬಾಹ್ಯಾಚಾರಗಳನ್ನು, ಎಲ್ಲ ಸಂಕೇತಗಳನ್ನು ವಿರೋಧಿಸಿದ ಹಲವು ಸುಧಾರಕರು ಎಲ್ಲಾ ಧರ್ಮಗಳಲ್ಲಿಯೂ ಇರುವರು. ಆದರೆ ಅವರ ವಿರೋಧವೆಲ್ಲಾ ವ್ಯರ್ಥವಾಯಿತು. ಏಕೆಂದರೆ ಎಲ್ಲಿಯವರೆಗೆ ಮಾನವನು ಈಗಿನ ಸ್ಥಿತಿಯಲ್ಲಿರುವನೋ, ಅಲ್ಲಿಯ\-ವರೆವಿಗೆ ಬಹುಜನರಿಗೆ ಗ್ರಹಿಸುವುದಕ್ಕೆ ಯಾವುದಾದರೂ ಸ್ಥೂಲವಾಗಿರುವುದೊಂದು ಬೇಕು; ತಮ್ಮ ಭಾವನೆಗಳನ್ನು ಕೇಂದ್ರಿಕರಿಸುವುದಕ್ಕೆ ಯಾವುದಾದರೊಂದು ನೆಲೆ ಬೇಕು; ತಮ್ಮ ಮನಸ್ಸಿನಲ್ಲಿರುವ ಎಲ್ಲಾ ಆಲೋಚನಾ ರೂಪಗಳಿಗೂ ಒಂದು ಕೇಂದ್ರ ಬೇಕು. ಎಲ್ಲಾ ಪೂಜಾ ವಿಧಿಗಳನ್ನು ನಿರ್ಮೂಲಗೊಳಿಸಬೇಕೆಂದು ಮಹಮ್ಮದೀಯರು ಮತ್ತು ಪ್ರಾಟೆಸ್ಟಂಟ್​\break ಜನಾಂಗಗಳು ಬಹಳವಾಗಿ ಪ್ರಯತ್ನಿಸಿದವು. ಆದರೂ ಅವರಲ್ಲಿಯೂ ಪೂಜಾ ವಿಧಿಗಳು ತೂರಿ ಬಂದಿರುವುದು ಕಾಣುವುದು. ಅವನ್ನು ಹೊರದೂಡುವುದಕ್ಕೆ ಆಗುವುದಿಲ್ಲ. ಬಹಳ ಹೋರಾಟವಾದ ಮೇಲೆ ಜನರು ಒಂದು ಪ್ರತೀಕವನ್ನು ಬಿಟ್ಟು ಮತ್ತೊಂದನ್ನು ಸ್ವೀಕರಿಸುವರು. ಅನ್ಯಧರ್ಮಕ್ಕೆ ಸೇರಿದ ಎಲ್ಲಾ ಬಗೆಯ ಆಚಾರ, ಆಕಾರ, ವಿಗ್ರಹ, ಪೂಜೆ, ಇವನ್ನು ಪಾಪವೆಂದು ಭಾವಿಸುವ ಮಹಮ್ಮದೀಯನು, ಕಾಬಾದಲ್ಲಿರುವ ತನ್ನ ಮಸೀದಿಗೆ ಬಂದಾಗ, ಅದನ್ನು ಪಾಪವೆಂದು ಪರಿಗಣಿಸುವುದಿಲ್ಲ. ಪ್ರತಿಯೊಬ್ಬ ಆಚಾರಶೀಲ ಮಹಮ್ಮದೀಯನು ತಾನು ಎಲ್ಲಿ ಪ್ರಾರ್ಥಿಸಿದರೂ ಕಾಬಾದಲ್ಲಿರುವ ಮಸೀದಿಯ ಎದುರು ನಿಂತಿರುವೆನೆಂದು ಭಾವಿಸಬೇಕು. ಅಲ್ಲಿಗೆ ಅವನು ಯಾತ್ರೆಗೆ ಹೋದಾಗ ಮಸೀದಿಯಲ್ಲಿರುವ ಕಪ್ಪು ಕಲ್ಲನ್ನು ಚುಂಬಿಸಬೇಕು. ಕೋಟ್ಯಂತರ ಭಕ್ತರು ಆ ಕಲ್ಲಿಗೆ ಕೊಟ್ಟ ಚುಂಬನಗಳೆಲ್ಲ ತೀರ್ಪನ್ನು ಕೊಡುವ ಕೊನೆಯ ದಿನದಂದು ಅವರ ಉದ್ಧಾರಕ್ಕೆ ಸಾಕ್ಷಿಯಾಗಿ ಎದ್ದು ನಿಲ್ಲುವುವು. ಆನಂತರ, ಅಲ್ಲಿ ಜಮ್​ ಜಮ್​ ಬಾವಿ ಇದೆ. ಯಾರು ಆ ಬಾವಿಯಿಂದ ಸ್ವಲ್ಪ ನೀರನ್ನು ಸೇದುವರೋ ಅವರ ಪಾಪ ಪರಿಹಾರವಾಗಿ, ಅಂತಿಮ ತೀರ್ಪಿನ ಅನಂತರ ಹೊಸದೊಂದು ದೇಹದಲ್ಲಿ ಶಾಶ್ವತವಾಗಿ ಬಾಳುವೆವು ಎಂದು ಮಹಮ್ಮದೀಯರು ನಂಬುತ್ತಾರೆ. ಮತ್ತೆ ಕೆಲವರಲ್ಲಿ ಈ ಸಾಂಕೇತಿಕತೆಯು ಕಟ್ಟಡರೂಪದಲ್ಲಿ ಇರುವುದು. ಪ್ರಾಟೆಸ್ಟೆಂಟರು ಉಳಿದ ಎಲ್ಲಾ ಸ್ಥಳಗಳಿಗಿಂತ ಚರ್ಚು ಅತಿ ಪವಿತ್ರವೆಂದು ಭಾವಿಸುವರು. ಚರ್ಚು ಅಥವಾ ಒಂದು ಗ್ರಂಥ ಸಂಕೇತವಾಗುತ್ತದೆ. ಅವರಿಗೆ ಉಳಿದುದೆಲ್ಲಕ್ಕಿಂತ ಗ್ರಂಥ ಪವಿತ್ರ.

ಸಂಕೇತಗಳ ಬಳಕೆಯ ವಿರೋಧವಾಗಿ ಬೋಧಿಸುವುದು ವ್ಯರ್ಥ. ಅದನ್ನು ನಾವು ಏಕೆ ಅಲ್ಲಗಳೆಯಬೇಕು? ಮಾನವರು ಸಂಕೇತಗಳನ್ನು ಬಳಸಬಾರದು ಎಂದು ಹೇಳುವುದಕ್ಕೆ ಯಾವುದೇ ಕಾರಣವೂ ಇಲ್ಲ. ತಮ್ಮ ಭಾವನೆಗಳನ್ನು ಪ್ರತಿನಿಧಿಸುವುದಕ್ಕೆ ಜನರು ಸಂಕೇತಗಳನ್ನು ಬಳಸುತ್ತಾರೆ. ಈ ವಿಶ್ವವೇ ಒಂದು ಸಂಕೇತ. ಈ ಸಂಕೇತದ ಮೂಲಕ ಇದರ ಹಿಂದೆ ಇರುವ ಮತ್ತು ಇದನ್ನು ಮೀರಿರುವ ಸತ್ಯವನ್ನು ಗ್ರಹಿಸಲು ಪ್ರಯತ್ನಿಸುತ್ತೇವೆ. ನಮ್ಮ ಗುರಿಯು. ಜಡವಸ್ತುವಲ್ಲ, ಆತ್ಮ. ರೂಪ, ವಿಗ್ರಹ, ಗಂಟೆ, ದೀಪ, ಶಾಸ್ತ್ರ, ಚರ್ಚು, ದೇವಸ್ಥಾನ ಮತ್ತು ಇತರ ಎಲ್ಲಾ ಪವಿತ್ರ ಸಂಕೇತಗಳು ಬೆಳೆಯುತ್ತಿರುವ ಆಧ್ಯಾತ್ಮಿಕ ಸಸಿಗೆ ಸಹಾಯಕವಾಗಿವೆ. ಆದರೆ ಇಷ್ಟೇ, ಅದರಾಚೆಗೆ ಇಲ್ಲ. ಹೆಚ್ಚಿನ ಜನ ಆಧ್ಯಾತ್ಮಿಕ ವಿಷಯದಲ್ಲಿ ಕೇವಲ ಸಸಿಯಾಗಿಯೇ ಉಳಿಯುತ್ತಾರೆ. ಒಂದು ಚರ್ಚಿನಲ್ಲಿ ಹುಟ್ಟುವುದು ಒಳ್ಳೆಯದು, ಆದರೆ ಅಲ್ಲೇ ಸಾಯುವುದು ಕೆಟ್ಟದು. ಸಣ್ಣ ಆಧ್ಯಾತ್ಮಿಕ ಸಸಿಯ ಬೆಳವಣಿಗೆಗೆ ಸಹಕಾರಿಯಾದ ಕೆಲವು ಆಚಾರ ವ್ಯವಹಾರಗಳ ವಾತಾವರಣದಲ್ಲಿ ಹುಟ್ಟುವುದು ಒಳ್ಳೆಯದು. ಆದರೆ ಜೀವಿಯು ಆ ವಲಯದಲ್ಲೇ ಸತ್ತರೆ, ಅವನ ಜೀವನ ಬೆಳೆದಿಲ್ಲ, ಅವನ ಆತ್ಮ\break ಅಭಿವೃದ್ಧಿಯಾಗಿಲ್ಲ ಎಂದು ಅರ್ಥ.

ಯಾರಾದರೂ ಸಂಕೇತ, ಪೂಜೆ, ಆಕಾರ ಇವನ್ನು ಎಂದೆಂದಿಗೂ ಇಟ್ಟುಕೊಂಡಿರಲೇ\-ಬೇಕೆಂದರೆ ಅದು ತಪ್ಪು. ಆದರೆ ಜೀವಿಯ ಶೈಶವಾವಸ್ಥೆಯಲ್ಲಿ ಸಂಕೇತ ಮತ್ತು ಇತರ ಬಾಹ್ಯ ಸಲಕರಣೆಗಳು ಆಧ್ಯಾತ್ಮಿಕ ಪುರೋಗಮನಕ್ಕೆ ಸಹಾಯಕವಾಗುವುದಾದರೆ ಅದು ಸರಿ. ಆದರೆ ಜೀವಿಯ ಬೆಳವಣಿಗೆ ಎಂದರೆ ಬುದ್ಧಿವಂತಿಕೆಯ ಬೆಳವಣಿಗೆ ಎಂದು ತಿಳಿಯ\-ಬಾರದು. ಒಬ್ಬನು ಪ್ರಚಂಡ ಮೇಧಾವಿಯಾಗಿರಬಹುದು, ಆದರೂ ಆಧ್ಯಾತ್ಮಿಕತೆಯಲ್ಲಿ ಏನೂ ಅರಿಯದ ಹಸುಳೆಯಂತೆ ಇರಬಹುದು. ಈ ಕ್ಷಣವೇ ಇದನ್ನು ಬೇಕಾದರೆ ಪರೀಕ್ಷಿಸಬಹುದು. ನೀವೆಲ್ಲರೂ ಸರ್ವವ್ಯಾಪಿಯಾದ ಭಗವಂತನನ್ನು ನಂಬಬೇಕೆಂದು ಬೋಧಿಸಲಾಗಿದೆ; ಆದರೆ ಅದನ್ನು ಕುರಿತು ಆಲೋಚಿಸಲು ಯತ್ನಿಸಿರಿ. ಸರ್ವವ್ಯಾಪಿತ್ವದ ಅರ್ಥ ಎಷ್ಟು ಸ್ವಲ್ಪ ಜನಕ್ಕೆ ಅರ್ಥವಾಗಬಲ್ಲುದು! ನೀವು ತುಂಬಾ ಪ್ರಯತ್ನಪಟ್ಟರೆ, ಸಾಗರ, ಆಕಾಶ, ವಿಶಾಲವಾದ ಹಸಿರು ಬಯಲು ಅಥವಾ ಒಂದು ಮರಳು ಕಾಡಿನ ಭಾವನೆ ಬರುವುದು. ಇವೆಲ್ಲ ಜಡ ಚಿತ್ರಗಳು. ಎಲ್ಲಿಯವರೆವಿಗೂ ನಾವು ಆದರ್ಶವನ್ನು ಆದರ್ಶದಂತೆ, ಸೂಕ್ಷ್ಮವನ್ನು ಸೂಕ್ಷ್ಮದಂತೆ ಗ್ರಹಿಸಲಾರೆವೋ, ಅಲ್ಲಿಯವರೆವಿಗೂ ಈ ಆಕಾರಗಳನ್ನೂ, ಜಡ ವಿಗ್ರಹಗಳನ್ನೂ. ನಾವು ಅನುಸರಿಸಬೇಕಾಗಿದೆ. ವಿಗ್ರಹವು ಮನಸ್ಸಿನ ಒಳಗೆ ಇದೆಯೇ ಅಥವಾ ಹೊರಗೆ ಇದೆಯೇ ಎಂಬುದರಿಂದ ಅಷ್ಟು ವ್ಯತ್ಯಾಸವೇನೂ ಆಗುವುದಿಲ್ಲ. ನಾವೆಲ್ಲ ಹುಟ್ಟು ವಿಗ್ರಹಾರಾಧಕರು. ವಿಗ್ರಹಾರಾಧನೆ ಒಳ್ಳೆಯದು, ಏಕೆಂದರೆ ಇದು ಮಾನವನ ಸ್ವಭಾವದಲ್ಲೇ ಇದೆ. ಯಾರು ಇದನ್ನು ಮೀರಿ ಹೋಗಬಲ್ಲರು? ಪೂರ್ಣಾತ್ಮನು ಮಾತ್ರ, ದೇವಮಾನವನು ಮಾತ್ರ ಹಾಗೆ ಮೀರಿ ಹೋಗಬಲ್ಲನು. ಉಳಿದವರೆಲ್ಲ ವಿಗ್ರಹಾರಾಧಕರು. ಎಲ್ಲಿಯರೆವಿಗೂ ನಮ್ಮೆದುರು ನಾಮರೂಪಗಳ ವಿಶ್ವವನ್ನು ನೋಡುತ್ತಿರುವೆವೋ, ಅಲ್ಲಿಯವರೆಗೆ ನಾವು ವಿಗ್ರಹಾರಾಧಕರು, ನಾವು ಆರಾಧಿಸುತ್ತಿರುವ ಭೂಮವಾದ ಸಂಕೇತವೇ ಈ ವಿಶ್ವ. ಯಾರು ತಾನು ದೇಹವೆಂದು ಹೇಳುವನೊ ಅವನೂ ಹುಟ್ಟು ವಿಗ್ರಹಾರಾಧಕ. ನಾವು ಆತ್ಮ. ಆತ್ಮಕ್ಕೆ ನಾಮ ರೂಪಗಳಿಲ್ಲ. ಆತ್ಮ ಅನಂತವಾದುದು, ಜಡವಸ್ತುವಲ್ಲ. ಯಾರು ತಮ್ಮ ನೈಜ ಸ್ಥಿತಿಯನ್ನು ಅರಿಯಲಾರರೋ, ಸೂಕ್ಷ್ಮವನ್ನು ತಿಳಿಯಲಾರರೋ, ಪಂಚಭೂತಾತ್ಮಕವಾದ ದೇಹ ಭಾವನೆಯಲ್ಲಿಯೆ ಇರುವರೊ ಅವರು ವಿಗ್ರಹಾರಾಧಕರು. ಆದರೂ ಜನರು ಇತರರನ್ನು ವಿಗ್ರಹಾರಾಧಕರೆಂದು ಹಳಿಯುತ್ತಾ ಎಷ್ಟೊಂದು ಹೊಡೆದಾಡುವರು! ಇದನ್ನು ಬೇರೆ ಮಾತಿನಲ್ಲಿ ಹೇಳುವುದಾದರೆ ಪ್ರತಿಯೊಬ್ಬನೂ ತನ್ನ ವಿಗ್ರಹವೇ ಸರಿಯಾದದ್ದು ಉಳಿದವು ಅಲ್ಲ ಎಂದು ಹೇಳುತ್ತಾನೆ.

ಆದುದರಿಂದ ನಾವು ಇಂತಹ ಬಾಲಿಶ ಭಾವನೆಗಳಿಂದ ಪಾರಾಗಬೇಕು. ಧರ್ಮವೆಂದರೆ ಕೇವಲ ಸುಂದರ ಪದ ಸಂಯೋಜನೆಯ ಮತ ಅಥವಾ ಸಿದ್ದಾಂತ ಎಂದು ತಿಳಿದಿರುವ ಮಾತಿನ ಮಲ್ಲರಿಂದ ಬಹುದೂರ ಇರಬೇಕು. ಯಾರಿಗೆ ಧರ್ಮವೆಂದರೆ ಯುಕ್ತಿಯ ಆಧಾರದ ಮೇಲೆ ಒಂದು ಒಪ್ಪಿಗೆ ಅಥವಾ ವಿರೋಧವನ್ನು ಸೂಚಿಸುವುದಾಗಿದೆಯೋ, ಯಾರಿಗೆ ತಮ್ಮ ಪುರೋಹಿತರು ಹೇಳುವ ಕೆಲವು ವಿಷಯಗಳನ್ನು ನಂಬುವುದಾಗಿದೆಯೋ, ಯಾರಿಗೆ ಧರ್ಮವೆಂದರೆ ತಮ್ಮ ಪೂರ್ವಿಕರ ನಂಬಿಕೆಯಾಗಿದೆಯೋ, ಯಾರಿಗೆ ಧರ್ಮವೆಂದರೆ ತಮ್ಮ ಜನಾಂಗವೇ ನಂಬಿಕೊಂಡಿರುವ ಕೆಲವು ಭಾವನೆ ಮತ್ತು ಮೂಢ ನಂಬಿಕೆಗೆ ಅಂಟಿಕೊಂಡಿರುವುದಾಗಿದೆಯೋ ಅಂತಹವರಿಂದ ನಾವು ದೂರವಿರಬೇಕು. ನಾನು ಅವುಗಳನ್ನೆಲ್ಲ ಮೀರಿ ಹೋಗಬೇಕು. ಮಾನವಕೋಟಿ, ಬೆಳಕಿನೆಡೆಗೆ ಮುಂದುವರಿಯು\-ತ್ತಿರುವ ಒಂದು ಬೃಹತ್​ ಜೀವಿಗಳ ವ್ಯೂಹ ಎಂದು ಭಾವಿಸಬೇಕು. ಭಗವಂತನೆಂಬ ಅತ್ಯದ್ಭುತ ಸತ್ಯವನ್ನು ವ್ಯಕ್ತಗೊಳಿಸುತ್ತಿರುವ ಅತಿ ಅದ್ಭುತವಾದ ವೃಕ್ಷ ಈ ಮಾನವಕೋಟಿ. ಭಗವಂತನೆಡೆಗೆ ಹೋಗುವ ಪ್ರಾಥಮಿಕ ಬಾಹ್ಯ ಚಲನೆಯನ್ನು ಸ್ಥೂಲ ವಸ್ತುಗಳ ಮೂಲಕ, ಆಚಾರಗಳ ಮೂಲಕ ಮಾತ್ರ ಆಲೋಚಿಸಲು ಸಾಧ್ಯ.

ಮತಾಚರಣೆಗಳ ಅಂತರಾಳದಲ್ಲೆಲ್ಲಾ ಒಂದು ಮಾತ್ರ ಬಹು ಪ್ರಧಾನವಾಗಿ ಎದ್ದು ನಿಲ್ಲುವುದು. ಅದೇ ನಾಮೋಪಾಸನೆ. ಯಾರು ಪುರಾತನ ಕ್ರೈಸ್ತಪಂಗಡಗಳನ್ನು ಕುರಿತು ಅಧ್ಯಯನ ಮಾಡಿರುವರೋ ಅಥವಾ ಜಗತ್ತಿನ ಇತರ ಧರ್ಮಗಳನ್ನು ಅಧ್ಯಯನ ಮಾಡಿರುವರೋ, ಅವರಿಗೆ ಅಲ್ಲೆಲ್ಲ ನಾಮೋಪಾಸನೆ ಬಳಕೆಯಲ್ಲಿರುವುದು ಮನವರಿಕೆಯಾಗಿರಬಹುದು. ನಾಮ ಅತಿ ಪವಿತ್ರ. ಭಗವಂತನ ಪವಿತ್ರ ನಾಮಕ್ಕೆ ಹೋಲಿಸುವಂತಹುದು ಯಾವುದೂ ಇಲ್ಲ, ಅದು ಎಲ್ಲಕ್ಕಿಂತಲೂ ಪವಿತ್ರವಾದದ್ದು ಎಂದು ಬೈಬಲ್ಲಿನಲ್ಲಿ ಓದಿದ್ದೇವೆ. ಎಲ್ಲಾ ನಾಮಗಳಿಗಿಂತ ಇದು ಪವಿತ್ರೋತ್ತಮ. ಈ ನಾಮವೇ ದೇವರು ಎಂದು ಕೂಡ ನಂಬಿದ್ದರು. ಇದು ಸತ್ಯ. ಈ ಪ್ರಪಂಚ ನಾಮರೂಪಗಳಲ್ಲದೆ ಮತ್ತೇನೂ ಪದಗಳಿಲ್ಲದೆ ನೀವು ಆಲೋಚಿಸಬಲ್ಲಿರಾ? ಪದವನ್ನು ಮತ್ತು ಆಲೋಚನೆಯನ್ನು ಪ್ರತ್ಯೇಕಿಸಲು ಅಸಾಧ್ಯ. ನಿಮ್ಮಲ್ಲಿ ಯಾರಿಗಾದರೂ ಅದನ್ನು ಪ್ರತ್ಯೇಕಿಸಲು ಸಾಧ್ಯವೇ ಪ್ರಯತ್ನಿಸಿ ನೋಡಿ. ನೀವು ಆಲೋಚಿಸುವಾಗಲೆಲ್ಲ ಶಬ್ದರೂಪಗಳ ಮೂಲಕ ಮಾತ್ರ ಅದನ್ನು ಮಾಡುತ್ತಿರುವಿರಿ. ಒಂದು ಮತ್ತೊಂದನ್ನು ಜ್ಞಾಪಕಕ್ಕೆ ತರುವುದು. ಆಲೋಚನೆ ಶಬ್ದವನ್ನು ತರುವುದು. ಶಬ್ದ ಆಲೋಚನೆಯನ್ನು ತರುವುದು. ಈ ವಿಶ್ವವೆಲ್ಲ ಭಗವಂತನ ಬಾಹ್ಯ ಪ್ರತೀಕದಂತಿದೆ. ಅದರ ಹಿಂದೆ ಅವನ ಪವಿತ್ರ ನಾಮ ಹುದುಗಿದೆ. ಪ್ರತಿಯೊಂದು ದೇಹವೂ ಒಂದು ಆಕಾರ. ಆ ಪ್ರತ್ಯೇಕ ಆಕಾರದ ಹಿಂದೆ ಹೆಸರಿದೆ. ನಮ್ಮ ಸ್ನೇಹಿತ ಹಾಗೆ ಇರುವನು, ಹೀಗೆ ಇರುವನು, ಎಂದು ಭಾವಿಸಿದೊಡನೆ ಅವನ ಆಕಾರ ನಮಗೆ ಹೊಳೆಯುವುದು. ಆ ಸ್ನೇಹಿತನ ಆಕಾರವನ್ನು ಆಲೋಚಿಸಿದೊಡನೆಯೇ ಅವನ ಹೆಸರು ಬರುವುದು. ಮನುಷ್ಯನ ಸ್ವಭಾವವೇ ಇದು. ಮನಶ್ಶಾಸ್ತ್ರದ ಪ್ರಕಾರ ಮನುಷ್ಯನ ಮನಸ್ಸಿನಲ್ಲಿ ರೂಪ ಇಲ್ಲದೆ ನಾಮ ಬರಲಾರದು. ನಾಮ ಇಲ್ಲದೆ ರೂಪ ಬರಲಾರದು. ಇವನ್ನು ಪ್ರತ್ಯೇಕಿಸುವುದು ಅಸಾಧ್ಯ. ಇವು ಒಂದೇ ಅಲೆಯ ಬಾಹ್ಯ ಮತ್ತು ಆಂತರಿಕ ಭಾಗಗಳು. ಆದಕಾರಣವೆ ನಾಮವನ್ನು ಜಗತ್ತಿನಲ್ಲೆಲ್ಲಾ ಕೊಂಡಾಡಿರುವರು ಮತ್ತು ಪೂಜಿಸುವರು. ತಿಳಿದೊ ತಿಳಿಯದೆಯೊ ಮಾನವನು ನಾಮದ ಮಹಿಮೆ ಅರಿತುಕೊಂಡಿದ್ದಾನೆ.

ಧರ್ಮಗಳಲ್ಲಿ ಮಹಾಪುರುಷರ ಆರಾಧನೆಯು ಇರುವುದನ್ನು ನೋಡುತ್ತೇವೆ. ಅವರು ಕೃಷ್ಣ ಬುದ್ಧ ಏಸು ಮುಂತಾದವರನ್ನು ಪೂಜಿಸುತ್ತಾರೆ. ಅನಂತರ ಸಂತರ ಆರಾಧನೆ ಇರುವುದು. ಜಗತ್ತಿನಲ್ಲೆಲ್ಲಾ ಇಂತಹ ನೂರಾರು ಜನರನ್ನು ಪೂಜಿಸುವರು. ಇದನ್ನು ಏತಕ್ಕೆ ಮಾಡಬಾರದು? ಬೆಳಕಿನ ಸ್ಪಂದನ ಎಲ್ಲೆಡೆಯಲ್ಲಿಯೂ ಇರುವುದು. ಗೂಬೆ ಆ ಸ್ಪಂದನವನ್ನು ಕತ್ತಲೆಯಲ್ಲಿ ನೋಡುವುದು. ಮನುಷ್ಯನ ಕಣ್ಣಿಗೆ ಇದು ಕಾಣದೆ ಹೋದರೂ, ಅಲ್ಲಿದೆ ಎನ್ನುವುದನ್ನು ತೋರುವುದು. ಮನುಷ್ಯರಿಗೆ ಆ ಸ್ಪಂದನ ದೀಪದಲ್ಲಿ, ಸೂರ್ಯಚಂದ್ರರಲ್ಲಿ ಮಾತ್ರ ಕಾಣುವುದು. ಭಗವಂತನು ಸರ್ವವ್ಯಾಪಿಯಾಗಿರುವನು. ಎಲ್ಲದರಲ್ಲೂ ಆತ ವ್ಯಕ್ತವಾಗಿರುವನು. ಆದರೆ ಮನುಷ್ಯನಿಗೆ ದೇವರು ಮನುಷ್ಯನಲ್ಲಿ ಮಾತ್ರ ಗೋಚರಿಸುವನು. ದೇವರಜ್ಯೋತಿ, ಶಕ್ತಿ, ವ್ಯಕ್ತಿತ್ವ, ಮಾನವನಲ್ಲಿ ಪ್ರಕಾಶಿಸಿದಾಗ ಮಾತ್ರ ಮಾನವನು ಅದನ್ನು ಗ್ರಹಿಸಬಲ್ಲ. ಆದಕಾರಣವೇ ಎಲ್ಲಾ ಕಾಲದಲ್ಲಿಯೂ ಮಾನವನು ದೇವರನ್ನು ಮಾನವನ ಮೂಲಕ ಪೂಜಿಸುತ್ತಿರುವನು. ಮಾನವನು ಎಲ್ಲಿಯವರೆಗೆ ಮಾನವನಾಗಿರುವನೋ ಅಲ್ಲಿಯವರೆಗೂ ಹೀಗೆ ಮಾಡದೆ ವಿಧಿಯಿಲ್ಲ. ಇದನ್ನು ವಿರೋಧಿಸಬಹುದು, ಇದರಿಂದ ಪಾರಾಗಲು ಯತ್ನಿಸಬಹುದು. ಆದರೆ ಎಂದು ಅವನನ್ನು ಪಡೆಯಲು ಯತ್ನಿಸುವನೋ, ಆಗ ದೇವರನ್ನು ಪಡೆಯಲು ಅವನನ್ನು ಮನುಷ್ಯನಂತೆ ಭಾವಿಸುವುದು ಅನಿವಾರ್ಯವೆಂದು ಗೊತ್ತಾಗುವುದು. ಆದಕಾರಣವೇ ಪ್ರಪಂಚದಲ್ಲಿರುವ ಪ್ರತಿಯೊಂದು ಧರ್ಮದಲ್ಲಿಯೂ ಭಗವಂತನ ಪೂಜೆಗೆ ಸಂಬಂಧಪಟ್ಟ ಈ ಮೂರು ಭಾವನೆಗಳನ್ನು–ರೂಪ, ನಾಮ, ದೇವಮಾನವರ ಆರಾಧನೆ ಇವನ್ನು ನೋಡುತ್ತೇವೆ. ಎಲ್ಲಾ ಧರ್ಮಗಳಲ್ಲಿಯೂ ಇವು ಇವೆ. ಆದರೆ ಒಬ್ಬರು ಮತ್ತೊಬ್ಬರೊಂದಿಗೆ ಹೋರಾಡಲಿಚ್ಛಿಸುವರು. ಒಬ್ಬನು, “ನಮ್ಮ ನಾಮವೇ ನಿಜವಾದ ನಾಮ, ನಾವು ಆಲೋಚಿಸುವ ರೂಪವೇ ನಿಜವಾದ ರೂಪ. ನಮ್ಮ ಮಹಾತ್ಮರೇ ಪ್ರಪಂಚದಲ್ಲಿ ಉತ್ತಮ ಮಹಾತ್ಮರು, ನಿಮ್ಮದೆಲ್ಲ ಕಂತೆಪುರಾಣ” ಎನ್ನುವನು. ಈಗಿನ ಕಾಲದಲ್ಲಿ ಕ್ರೈಸ್ತ ಪಾದ್ರಿಗಳು ಸ್ವಲ್ಪ ದಯೆ ತಾಳಿ, ಹಳೆಯ ಧರ್ಮಗಳಲ್ಲಿದ್ದ ಹಲವು ಬಗೆಯ ಆರಾಧನೆಗಳೆಲ್ಲ ಮುಂದೆ ಬರಲಿರುವ ಕ್ರೈಸ್ತೋಪಾಸನೆಯ ಮುನ್​ಸೂಚನೆಗಳು ಎನ್ನುವರು. ಅವರು ಇದೊಂದೇ ಸತ್ಯವೆಂದು ಭಾವಿಸುವರು. ಹಿಂದಿನ ಕಾಲದಲ್ಲಿ ದೇವರು ಇದನ್ನೆಲ್ಲ ಸೃಷ್ಟಿಸಿ, ಇಂತಹ ಆಕಾರ ಬರುವುದಕ್ಕೆ ಪ್ರಯತ್ನಿಸಿ, ತನ್ನ ಶಕ್ತಿಯನ್ನು ಪರೀಕ್ಷಿಸಿದನು. ಆ ಪರೀಕ್ಷೆಯ ಚರಮಾವಸ್ಥೆಯೇ ಕ್ರೈಸ್ತಧರ್ಮ. ಇದೆಷ್ಟೋ ಮುಂದುವರಿದಂತೆ! ಐವತ್ತು ವರುಷ\-ಗಳ ಹಿಂದೆ ಇದನ್ನು ಕೂಡ ಅವರು ಹೇಳುತ್ತಿರಲಿಲ್ಲ. ತಮ್ಮ ಧರ್ಮವಲ್ಲದೆ ಮತ್ತಾವುದೂ ಅವರಿಗೆ ನಿಜವಲ್ಲ. ಈ ಭಾವನೆ ಯಾವ ಒಂದು ಧರ್ಮ, ಜನಾಂಗ, ಪಂಗಡಕ್ಕೂ ಮೀಸಲಾಗಿಲ್ಲ; ಎಲ್ಲರೂ ಭಾವಿಸುವುದು ತಾವು ಏನು ಮಾಡುತ್ತಿರುವರೋ ಅದೇ ಸತ್ಯ. ಉಳಿದವರು ಇದನ್ನು ಅನುಸರಿಸಬೇಕು ಎಂದು. ಇಲ್ಲೇ ಭಿನ್ನ ಬಿನ್ನ ಧರ್ಮಗಳ ಅಧ್ಯಯನ ನಮಗೆ ಸಹಾಯ ಮಾಡುವುದು. ಯಾವ ಭಾವನೆಯನ್ನು ಇದು ನಮ್ಮದು, ನಮಗೆ ಮಾತ್ರ ಮೀಸಲಾಗಿರುವುದು, ಎಂದು ಭಾವಿಸಿದ್ದೆವೋ, ಅದು ನೂರಾರು ವರ್ಷಗಳ ಹಿಂದೆ\break ಅನ್ಯರಲ್ಲಿಯೂ ಇತ್ತು, ಅನೇಕ ವೇಳೆ ನಮ್ಮಲ್ಲಿ ಈಗ ಇರುವುದಕ್ಕಿಂತ ಅಲ್ಲಿ ಅದು ಹೆಚ್ಚು\break ಸ್ಪಷ್ಟವಾಗಿತ್ತು.

ಇವೇ ಭಕ್ತಿಯ ಬಾಹ್ಯ ರೂಪಗಳು. ಇವುಗಳ ಮೂಲಕ ಮಾನವನು ಸಾಗಿ ಹೋಗಬೇಕಾಗಿದೆ. ಅವನು ನಿಜವಾಗಿಯೂ ಪ್ರಾಮಾಣಿಕನಾಗಿದ್ದರೆ, ಸತ್ಯವನ್ನು ಪಡೆಯಬೇಕೆಂದು ಇಚ್ಛಿಸಿದರೆ ಇವುಗಳನ್ನು ಮೀರಿಹೋಗುವನು. ಅಲ್ಲಿ ರೂಪಗಳಿಗೆ ಬೆಲೆಯೇ ಇಲ್ಲ. ದೇವಸ್ಥಾನ, ಚರ್ಚು, ಶಾಸ್ತ್ರ, ಆಚಾರ ಇವು ಧಾರ್ಮಿಕ ಶಿಶು ವಿಹಾರಗಳು; ಇವು ಮುಂದೆ ಹೆಜ್ಜೆ ಇಡಲು ಆಧ್ಯಾತ್ಮಿಕ ಶಿಶುಗಳಿಗೆ ಶಕ್ತಿಕೊಡುವುದಕ್ಕೆ ಮಾತ್ರ. ಧರ್ಮಬೇಕಾದರೆ ಇಂತಹ ಮೊದಲಿನ ಹೆಜ್ಜೆ ಆವಶ್ಯಕ. ಭಗವಂತನಿಗಾಗಿ ಪಡುವ ವ್ಯಾಕುಲತೆಯೊಂದಿಗೆ ನಿಜವಾದ ಭಕ್ತಿ ಬರುವುದು. ಈ ಹಂಬಲಿಕೆ ಯಾರಲ್ಲಿರುವುದು? ಅದೇ ಪ್ರಶ್ನೆ. ಧರ್ಮವು ಸಿದ್ಧಾಂತದಲ್ಲಿಲ್ಲ, ಮೂಢನಂಬಿಕೆಯಲ್ಲಿಲ್ಲ, ಬೌದ್ಧಿಕವಾದದಲ್ಲಿಲ್ಲ. ಅದರಂತೆ ಇರುವುದೆ,\break ಅದರಂತೆ ಆಗುವುದೇ ಧರ್ಮ. ಇದೇ ಸಾಕ್ಷಾತ್ಕಾರ. ಅನೇಕರು ದೇವರು, ಆತ್ಮ, ಜಗದದ್ಭುತ ರಹಸ್ಯ ಮುಂತಾದ ವಿಷಯಗಳನ್ನು ಕುರಿತು ಮಾತನಾಡುವರು. ಅವರಲ್ಲಿ ಪ್ರತಿಯೊಬ್ಬರನ್ನು ನೀವು “ದೇವರನ್ನು ಕಂಡಿರುವಿರಾ! ಆತ್ಮನನ್ನು ಕಂಡಿರುವಿರಾ?” ಎಂದು ಪ್ರಶ್ನೆ ಮಾಡಿದರೆ, ಎಷ್ಟು ಜನರು ತಮಗೆ ಅದರ ಅನುಭವ ಇದೆ ಎಂದು ಹೇಳಬಲ್ಲರು? ಆದರೂ ಒಬ್ಬರು ಮತ್ತೊಬ್ಬರೊಂದಿಗೆ ಹೋರಾಡುತ್ತಿರುವರು. ಒಮ್ಮೆ ಭರತಖಂಡದಲ್ಲಿ ಭಿನ್ನ ಭಿನ್ನ ಮತಾವಲಂಬಿಗಳ ಪ್ರತಿನಿಧಿಗಳೆಲ್ಲಾ ಸೇರಿ ವಾದ ಮಾಡಲು ಮೊದಲು ಮಾಡಿದರು. ಒಬ್ಬ ಶಿವನೇ ದೇವರೆಂದನು. ಮತ್ತೊಬ್ಬ ವಿಷ್ಣುವೇ ಆ ದೇವರು ಎಂದನು. ಹೀಗೆ ಅವರ ವಾದಕ್ಕೆ ತುದಿ ಮೊದಲೇ ಇರಲಿಲ್ಲ. ಒಬ್ಬ ಋಷಿ ಆ ದಾರಿಯಲ್ಲಿ ಹೋಗುತ್ತಿದ್ದನು. ವಾದಿಸುತ್ತಿದ್ದವರು ಆ ಋಷಿಯನ್ನು ಕರೆದು ತೀರ್ಪುಕೊಡುವಂತೆ ಕೇಳಿಕೊಂಡರು. ಋಷಿಯು ಶಿವನೇ ದೊಡ್ಡ ದೇವರು ಎಂದು ಸಾರುತ್ತಿದ್ದವನನ್ನು ಕರೆದು “ನೀನು ಶಿವನನ್ನು ಕಂಡಿರುವೆಯಾ? ನಿನಗೆ ಅವನ ಪರಿಚಯವಿದೆಯೆ? ಇಲ್ಲದೆ ಇದ್ದರೆ ಅವನೇ ದೊಡ್ಡ ದೇವರೆಂಬುದು ನಿನಗೆ ಹೇಗೆ ಗೊತ್ತು?” ಎಂದು ಪ್ರಶ್ನಿಸಿದನು. ವಿಷ್ಣು ಭಕ್ತನನ್ನು ಕುರಿತು “ನೀನು ವಿಷ್ಣುವನ್ನು ಕಂಡಿರುವೆಯಾ?” ಎಂದು ಕೇಳಿದನು. ಎಲ್ಲರಿಗೂ ಈ ಪ್ರಶ್ನೆಯನ್ನು ಹಾಕಿದ ಮೇಲೆ ಯಾರಿಗೂ ದೇವರ ವಿಷಯ ಗೊತ್ತಿಲ್ಲವೆಂಬುದು ಚೆನ್ನಾಗಿ ಮನದಟ್ಟಾಯಿತು. ಆದ\-ಕಾರಣವೇ ಅವರು ಅಷ್ಟು ವಾದಮಾಡುತ್ತಿದ್ದರು. ಅವರಿಗೆ ಏನಾದರೂ ದೇವರ ಪರಿಚಯವಿದ್ದಿದ್ದರೆ ವಾದ ಮಾಡುತ್ತಿರಲಿಲ್ಲ. ಒಂದು ಕೊಡಕ್ಕೆ ನೀರು ಬೀಳುತ್ತಿರುವಾಗ ಕೊಡ ಶಬ್ದ ಮಾಡುವುದು ಆದರೆ ಅದು ತುಂಬಿದ ಮೇಲೆ ಶಬ್ದವೇ ಇರುವುದಿಲ್ಲ. ಪಂಗಡಗಳಲ್ಲಿ ಇಷ್ಟೊಂದು ವಾದ ಹೋರಾಟ ಇರುವುದರಿಂದಲೇ ಧರ್ಮದ ವಿಷಯದಲ್ಲಿ ಅವರಿಗೆ ಏನೂ ಗೊತ್ತಿಲ್ಲವೆಂಬುದು ವ್ಯಕ್ತವಾಗುವುದು. ಅವರಿಗೆ ಧರ್ಮವೆಂದರೆ ಕೇವಲ ಮಾತಿನ ನೊರೆ, ಗ್ರಂಥಗಳಿಂದ ಸಂಗ್ರಹಿಸುವ ವಸ್ತು. ಪ್ರತಿಯೊಬ್ಬನೂ ಸಾಧ್ಯವಾದಷ್ಟು ದೊಡ್ಡ ಗ್ರಂಥವನ್ನು ಬರೆಯಲು ತವಕ ಪಡುವನು. ಹಲವಾರು ಗ್ರಂಥಗಳಿಂದ ಅದಕ್ಕೆ ವಿಷಯ ಸಂಗ್ರಹಿಸಿದರೂ ಗ್ರಂಥಋಣವನ್ನು ವ್ಯಕ್ತಪಡಿಸನು. ಆಗಲೇ ಗೊಂದಲದಿಂದ ತುಂಬಿ ತುಳುಕಾಡುವ ಜಗತ್ತಿನಲ್ಲಿ ಮತ್ತಷ್ಟು ಗದ್ದಲವೆಬ್ಬಿಸಲು ತನ್ನ ಗ್ರಂಥವನ್ನು ಬೆಳಕಿಗೆ ತರುವನು!

ಮಾನವರಲ್ಲಿ ಬಹುಮಂದಿ ನಾಸ್ತಿಕರು. ಆಧುನಿಕ ಕಾಲದಲ್ಲಿ ಪಾಶ್ಚಾತ್ಯ ದೇಶದಲ್ಲಿ ಮತ್ತೊಂದು ಬಗೆಯ ನಾಸ್ತಿಕರು ಎಂದರೆ ಜಡವಾದಿಗಳು ಬಂದಿರುವುದರಿಂದ ನನಗೆ ಸಂತೋಷವಾಗಿದೆ. ಅವರು ಪ್ರಾಮಾಣಿಕ ನಾಸ್ತಿಕರು, ಧರ್ಮದ ಸೋಗಿನಲ್ಲಿರುವ ಠಕ್ಕರಾದ ನಾಸ್ತಿಕರಿಗಿಂತ ಅವರು ಮೇಲು. ಕಪಟಿಗಳು ಧರ್ಮದ ಹೆಸರಿನಲ್ಲಿ ವಾದಿಸುವರು, ಹೋರಾಡುವರು. ಆದರೂ ಅವರಿಗೆ ಧರ್ಮ ಬೇಕಾಗಿಲ್ಲ. ಅದನ್ನು ಸಾಕ್ಷಾತ್ಕಾರ ಮಾಡಿಕೊಳ್ಳಲು ಆಸೆ ಇಲ್ಲ, ಅದನ್ನು ನಿಜವಾಗಿ ತಿಳಿದುಕೊಳ್ಳಲು ಯತ್ನಿಸುವುದಿಲ್ಲ. ಕ್ರಿಸ್ತನ ಬೋಧನೆಯನ್ನು ನೆನಪಿನಲ್ಲಿಡಿ. “ಕೇಳಿ ನಿಮಗೆ ಅದು ದೊರಕುವುದು; ಅರಸಿ ಅದು ಸಿಗುವುದು; ಬಾಗಿಲನ್ನು ತಟ್ಟಿ, ತೆರೆಯುವುದು.” ಈ ಪದಗಳು ಅಕ್ಷರಶಃ ಸತ್ಯ, ಅಲಂಕಾರವಲ್ಲ,\break ಕಟ್ಟುಕಥೆಯಲ್ಲ. ಈ ಸಂದೇಶವು ನಮ್ಮ ಜಗತ್ತಿಗೆ ಬಹಳ ಅಪರೂಪವಾಗಿ ಬಂದ ಭಗವಂತನ ಅತ್ಯುತ್ತಮ ಮಕ್ಕಳಲ್ಲಿ ಒಬ್ಬನಾದ, ಮಹಾಭಕ್ತಾಗ್ರಣಿಯ ಹೃದಯಾಂತರಾಳದಿಂದ ಹೊರಹೊಮ್ಮಿದುದು. ಸಾಕ್ಷಾತ್ಕಾರದ ಪರಿಣಾಮವಾಗಿ ಬಂದ ಸಂದೇಶವಿದು. ನಾವು ಕಣ್ಣೆದುರಿಗೆ ಇರುವ ಕಟ್ಟಡವನ್ನು ನೋಡುವುದಕ್ಕಿಂತ ನೂರು ಮಡಿ ಹೆಚ್ಚಾಗಿ ದೇವರನ್ನು ಅನುಭವಿಸಿ, ಸಾಕ್ಷಾತ್ಕಾರ ಮಾಡಿಕೊಂಡು, ಆತನೊಡನೆ ಮಾತನಾಡಿ ಬಾಳಿದ ಮಹಾವ್ಯಕ್ತಿಯ ಜೀವನದ ಉಸಿರಿದು. ದೇವರು ಯಾರಿಗೆ ಬೇಕು? ಅದೇ ಪ್ರಶ್ನೆ. ಪ್ರಪಂಚದಲ್ಲಿರುವ ಇಷ್ಟೊಂದು ಜನಸಂದಣಿಗೆ ದೇವರು ಬೇಕು, ಆದರೂ ಸಿಕ್ಕುವುದಿಲ್ಲವೆಂದು ತಿಳಿದಿರುವಿರೇನು? ಅದು ಅಸಾಧ್ಯ. ಬಾಹ್ಯದಲ್ಲಿ ಅದಕ್ಕೆ ಅನುರೂಪವಾದ ವಸ್ತುವಿಲ್ಲದ ಬಯಕೆ ಯಾವುದಿದೆ? ಮಾನವ ಉಸಿರಾಡಬಯಸುವನು, ಅದಕ್ಕೆ ಗಾಳಿ ಹೊರಗೆ ಇದೆ. ಮಾನವನಿಗೆ ಊಟ ಮಾಡಲು ಆಸೆ. ಅದಕ್ಕೇ ಹೊರಗೆ ಆಹಾರವಿದೆ. ಈ ಬಯಕೆಯನ್ನು ಹುಟ್ಟಿಸುವುದು ಯಾವುದು? ಬಾಹ್ಯ ಜಗತ್ತಿನಲ್ಲಿ ವಸ್ತುಗಳಿರುವುದು. ಬೆಳಕು ಕಣ್ಣನ್ನು ಮಾಡಿತು, ಶಬ್ದ ಕಿವಿಯನ್ನು ಮಾಡಿತು. ಮನುಷ್ಯನಲ್ಲಿರುವ ಪ್ರತಿಯೊಂದು ಅಸೆಯೂ ಹೊರಗೆ ಇರುವ ಯಾವುದೋ ವಸ್ತುವಿನಿಂದ ಆಯಿತು. ಪೂರ್ಣತೆಯನ್ನು ಪಡೆಯಬೇಕು, ಗುರಿಯನ್ನು ಸೇರಬೇಕು, ಪ್ರಕೃತಿಯನ್ನು ಮೀರಿ ಹೋಗಬೇಕೆಂಬ ಆಸೆಯನ್ನು ಮತ್ತಾವುದೋ ಹುಟ್ಟಿಸದೇ ಇದ್ದರೆ, ಮನುಷ್ಯನ ಅಂತರಾಳಕ್ಕೆ ಪ್ರವೇಶಿಸಿ ಅಲ್ಲಿ ನೆಲ್ಸುವಂತೆ ಮಾಡದೇ ಇದ್ದರೆ, ಅದು ಇರಲು ಹೇಗೆ ಸಾಧ್ಯ? ಯಾರಲ್ಲಿ ಈ ಹಂಬಲ ಜಾಗೃತವಾಗಿದೆಯೊ ಅವರು\break ಗುರಿಯನ್ನು ಸೇರುವರು. ನಮಗೆ ದೇವರು ವಿನಃ ಎಲ್ಲಾ ಬೇಕು. ನಿಮ್ಮ ಸುತ್ತಲೂ\break ಕಾಣುವುದು ಧರ್ಮವಲ್ಲ. ದಿವಾನ್​ ಖಾನೆಯಲ್ಲಿ ಪ್ರಪಂಚದ ನಾನಾ ಭಾಗಗಳಿಂದ ತಂದ ವಸ್ತುಗಳಿವೆ. ಈಗ ಜಪಾನ್​ ದೇಶಕ್ಕೆ ಸಂಬಂಧಿಸಿದ ಕೆಲವು ವಸ್ತುಗಳನ್ನು ಅಲ್ಲಿ ಸಂಗ್ರಹಿಸ\-ಬೇಕಾಗಿದೆ. ಅದಕ್ಕೆ ಜಪಾನೀಯ ಪುಷ್ಪಕರಂಡ ಒಂದನ್ನು ಕೊಂಡು ಯಜಮಾನಿ ಅಲ್ಲಿ ಇಡುವಳು. ಬಹುಪಾಲು ಜನರಿಗೆ ಧರ್ಮವೆಂದರೆ ಇದೇ. ಅವರಿಗೆ ಎಲ್ಲಾ ವಸ್ತುಗಳಿವೆ. ಅದಕ್ಕೆ ಸ್ವಲ್ಪ ಧರ್ಮದ ಒಗ್ಗರಣೆಯನ್ನು ಹಾಕಿದಲ್ಲದೆ ಪೂರ್ತಿಯಾಗುವುದಿಲ್ಲ. ಏಕೆಂದರೆ, ಅದಿಲ್ಲದೆ ಇದ್ದರೆ ಸಮಾಜ ಟೀಕಿಸುವುದು. ಸಮಾಜ ನಿರೀಕ್ಷಿಸುವುದರಿಂದ ಅವರಿಗೆ ಸ್ವಲ್ಪ ಧರ್ಮ ಬೇಕಾಗಿದೆ. ಪ್ರಪಂಚದ ಈಗಿನ ಧಾರ್ಮಿಕ ಸ್ಥಿತಿ ಇದು.

ಶಿಷ್ಯನೊಬ್ಬ ತನ್ನ ಗುರುವಿನ ಹತ್ತಿರ ಹೋಗಿ, “ಸ್ವಾಮಿ ನನಗೆ ಧರ್ಮ ಬೇಕು” ಎಂದನು. ಗುರು ಯುವಕನನ್ನು ನೋಡಿ ಮೌನ ಮಂದಹಾಸವನ್ನು ಬೀರಿದನು. ಪ್ರತಿದಿನವೂ ಯುವಕನು ಬಂದು ಧರ್ಮಬೇಕೆಂದು ಕೇಳುತ್ತಿದ್ದನು. ಆದರೆ ವಯೋವೃದ್ಧನಾದ ಗುರುವಿಗೆ ಜನರ ಅನುಭವ ಚೆನ್ನಾಗಿತ್ತು. ಒಂದು ದಿನ ತುಂಬಾ ಬಿಸಿಲಾಗಿರುವಾಗ ನದಿಗೆ ಸ್ನಾನಕ್ಕಾಗಿ ತನ್ನೊಂದಿಗೆ ಯುವಕನನ್ನು ಕರೆದುಕೊಂಡು ಹೋದನು. ಯುವಕನು ನೀರಿನಲ್ಲಿ ಮುಳುಗಿದನು. ಗುರು ನೀರಿನಲ್ಲಿಳಿದು ಬಲಾತ್ಕಾರವಾಗಿ ಯುವಕನನ್ನು ನೀರಿನಲ್ಲಿ ಅದುಮಿ ಹಿಡಿದುಕೊಂಡನು. ಯುವಕನು ನೀರಿನಲ್ಲಿ ಕೆಲವು ಕ್ಷಣ ಒದ್ದಾಡಿದ ಮೇಲೆ,\break ಅವನನ್ನು ಹೊರಗೆ ಬಿಟ್ಟು, ನೀರಿನಲ್ಲಿರುವಾಗ ಅವನಿಗೆ ಏನು ಅತ್ಯಾವಶ್ಯಕವಾಗಿತ್ತು ಎಂದು ಪ್ರಶ್ನಿಸಿದನು. “ಸ್ವಲ್ಪ ಗಾಳಿ ಅವಶ್ಯಕವಾಗಿತ್ತು” ಎಂದು ಉತ್ತರವಿತ್ತನು, ಯುವಕ. ಈ ರೀತಿಯಲ್ಲಿ ದೇವರು ನಿಮಗೆ ಬೇಕೆ? ಹಾಗಿದ್ದರೆ ನಿಮಗೆ ಕ್ಷಣದಲ್ಲಿ ದೊರಕುವನು. ಇಂತಹ ದಾರುಣ ವ್ಯಾಕುಲತೆ ಬರುವ ತನಕ, ಆಸೆ ಹುಟ್ಟುವತನಕ, ನೀವು ಎಷ್ಟು ಯುಕ್ತಿಶಾಸ್ತ್ರ, ರೂಪಗಳೊಂದಿಗೆ ಹೋರಾಡಿದರೂ ನಿಮಗೆ ಧರ್ಮ ದೊರಕಲಾರದು. ಇಂತಹ ವ್ಯಾಕುಲತೆ ನಿಮ್ಮಲ್ಲಿ ಹುಟ್ಟುವ ಪರಿಯಂತರ ಯಾವ ನಾಸ್ತಿಕನಿಗಿಂತಲೂ ನೀವು ಮೇಲಲ್ಲ. ನಾಸ್ತಿಕ ಪ್ರಾಮಾಣಿಕ, ನೀವು ಅಲ್ಲ, ಅಷ್ಟೆ.

\vskip 0.5cm

ಒಬ್ಬ ಮಹಾಪುರುಷ ಹೀಗೆ ಹೇಳುತ್ತಿದ್ದನು: “ಒಂದು ಕೋಣೆಯಲ್ಲಿ ಒಬ್ಬ ಕಳ್ಳನಿರುವನೆಂದು ಭಾವಿಸೋಣ. ಪಕ್ಕದ ಕೋಣೆಯಲ್ಲಿ ಬೇಕಾದಷ್ಟು ದ್ರವ್ಯವಿದೆ ಎಂದು ಭಾವಿಸೋಣ. ಹೇಗೋ ಇದು ಕಳ್ಳನಿಗೆ ಗೊತ್ತಾಯಿತು. ಅದಕ್ಕೂ ಅವನಿಗೂ ಮಧ್ಯೆ ಒಂದು ಸಣ್ಣ ಗೋಡೆ ಮಾತ್ರ ಇದೆ ಎಂದು ಇಟ್ಟುಕೊಳ್ಳೋಣ. ಆ ಕಳ್ಳನ ಅವಸ್ಥೆ ಆಗ ಹೇಗೆ ಇರುವುದು ಗೊತ್ತೆ? ಆಗ ಅವನು ನಿದ್ರಿಸಲಾರ, ಊಟ ಮಾಡಲಾರ, ಯಾವ ಕೆಲಸವನ್ನೂ ಮಾಡಲಾರ, ಅವನ ಮನಸ್ಸೆಲ್ಲಾ ಆ ದ್ರವ್ಯವನ್ನು ಹೇಗೆ ಪಡೆಯಬೇಕೆಂಬುದರ ಮೇಲೆ ಇರುತ್ತದೆ. ಇಲ್ಲಿರುವ ಜನರಿಗೆಲ್ಲಾ, ಸುಖದ ಆನಂದಾತಿಶಯದ ಮಹಾಗಣಿ ಇಲ್ಲೇ ಇರುವುದೆಂಬ ನಂಬಿಕೆ ಇದ್ದರೆ, ದೇವರನ್ನು ಕಾಣುವ ಬದಲು ಇವರು ತಮ್ಮ ಕೆಲಸದಲ್ಲಿ ನಿರತರಾಗುತ್ತಿದ್ದರು, ಎಂದು ಭಾವಿಸುವಿರೇನು?” ದೇವರೊಬ್ಬನಿರುವನು ಎಂದು ಒಬ್ಬ ತಿಳಿದೊಡನೆಯೇ ಅವನನ್ನು ಪಡೆಯಬೇಕೆಂದು ಹುಚ್ಚನಾಗಿ ಹೋಗುವನು. ಇತರರು ತಮ್ಮ ಪಾಡಿಗೆ ತಾವು ಇರಬಹುದು. ಆದರೆ ಯಾರಿಗೆ, ಇಲ್ಲಿ ತಾವು ಬಾಳುವ ಸ್ಥಿತಿಗಿಂತ ಉತ್ತಮ ಜೀವನ ಒಂದಿದೆ, ಇಂದ್ರಿಯಗಳೇ ಪರಮಾವಧಿಯಲ್ಲ, ಅದನ್ನು ಮೀರಿದ ಒಂದು ಸ್ಥಿತಿ ಇದೆ, ಅಮೃತನೂ ಅನಂತನೂ ಆನಂದಮಯನೂ ಆದ ಆತ್ಮನೊಂದಿಗೆ ಜಡದೇಹವನ್ನು ಹೋಲಿಸಿ ನೋಡಿದಾಗ, ಇದಕ್ಕೆ ಬೆಲೆ ಇಲ್ಲ ಎಂದು ನಿಸ್ಸಂದೇಹವಾಗಿ ಗೊತ್ತಾಗುವುದೊ, ಆ ಆನಂದವನ್ನು ಪಡೆಯುವ ಪರಿಯಂತರವೂ ಅವನು ಹುಚ್ಚನಾಗಿ ಹೋಗುವನು. ಈ ಹುಚ್ಚನ್ನೇ, ಈ ದಾಹವನ್ನೇ, ಈ ಸ್ವಭಾವವನ್ನೇ “ಧಾರ್ಮಿಕ ಜಾಗೃತಿ” ಎಂದು ಹೇಳುವರು. ಈ ವ್ಯಾಕುಲತೆ ಬಂದಾಗ ಮಾನವನು ಧಾರ್ಮಿಕನಾಗುತ್ತಿರುವನು. ಆದರೆ ಇದಕ್ಕೆ ಬಹಳ ಕಾಲಬೇಕು. ಬಾಹ್ಯ ವಿಗ್ರಹ, ಆಚಾರ, ಪ್ರಾರ್ಥನೆ, ಯಾತ್ರೆ, ಶಾಸ್ತ್ರ, ಗಂಟೆ, ದೀಪ, ಪುರೋಹಿತ ಇವೆಲ್ಲ ಸಿದ್ಧತೆಗಳು ಮಾತ್ರ. ಅವು ಜೀವನದ ಕಿಲುಬನ್ನು ತೊಳೆಯುವುವು. ಜೀವವು ಪವಿತ್ರವಾದ ಮೇಲೆ ಸ್ವಾಭಾವಿಕವಾಗಿ ಅದು ಎಲ್ಲಾ ಪವಿತ್ರತೆಯ ಮೂಲವಾದ ಭಗವಂತನಲ್ಲಿಗೆ ಹೋಗಲೆತ್ನಿಸುವುದು. ಹಲವು ಶತಮಾನಗಳಿಂದಲೂ ಧೂಳಿನಿಂದ ಮುಚ್ಚಲ್ಪಟ್ಟ ಕಬ್ಬಿಣದ ಚೂರು, ಅಯಸ್ಕಾಂತ ಶಿಲೆಯ ಹತ್ತಿರವಿದ್ದರೂ ಆಕರ್ಷಿತವಾಗದೆ, ಈ ಧೂಳನ್ನು ತೊಳೆದ ತಕ್ಷಣವೇ, ಆ ಅಯಸ್ಕಾಂತ ಶಿಲೆಯಿಂದ ಆಕರ್ಷಿಸಲ್ಪಡುವಂತೆಯೇ, ಮಾನವ ಜೀವ ಹಿಂದಿನಿಂದಲೂ ಕೊಳೆ, ಕ್ರೌರ್ಯ, ಪಾಪಗಳಿಂದ ಕೂಡಿದ್ದರೂ ವಿಗ್ರಹ, ಆಚಾರ, ಪರೋಪಕಾರ, ಪ್ರೀತಿ ಮುಂತಾದುವುಗಳಿಂದ ಪರಿಶುದ್ಧವಾಗಿ, ಅದರಲ್ಲಿ ಸ್ವಭಾವತಃ ಇರುವ ಆಧ್ಯಾತ್ಮಿಕ ಆಕರ್ಷಣೆ ಜಾಗೃತವಾಗಿ ಭಗವಂತನೆಡೆಗೆ ಹೋಗಲು ಯತ್ನಿಸುವುದು.

ಆದರೂ ವಿಗ್ರಹ, ಪ್ರತೀಕ, ಇವು ಪ್ರಾರಂಭ ಮಾತ್ರ, ಇವೇ ನಿಜವಾದ ಭಗವತ್​ ಪ್ರೀತಿಯಲ್ಲ. ಎಲ್ಲೆಲ್ಲೂ ಪ್ರೀತಿಯ ವಿಷಯ ಮಾತನಾಡುವುದನ್ನು ಕೇಳುವೆವು. ಪ್ರತಿಯೊಬ್ಬನೂ ದೇವರನ್ನು ಪ್ರೀತಿಸಿ ಎನ್ನುವನು. ಪ್ರೀತಿ ಎಂದರೆ ಏನೆಂದು ಮನುಷ್ಯನಿಗೆ ಗೊತ್ತಿಲ್ಲ. ಅದು ಅವನಿಗೆ ಗೊತ್ತಿದ್ದರೆ ಇಷ್ಟು ಅಜಾಗರೂಕತೆಯಿಂದ ಮಾತನಾಡುತ್ತಿರಲಿಲ್ಲ. ಪ್ರತಿಯೊಬ್ಬ ಪುರುಷನೂ ತಾನು ಪ್ರೀತಿಸಬಲ್ಲೆ ಎನ್ನುವನು, ಆದರೆ ತನ್ನಲ್ಲಿ ಈ ಪ್ರೀತಿ ಇಲ್ಲವೆಂದು ಬಹುಬೇಗ ತಿಳಿದುಕೊಳ್ಳುವನು. ಪ್ರತಿಯೊಬ್ಬ ಸ್ತ್ರೀಯೂ ತಾನು ಪ್ರೀತಿಸಬಲ್ಲೆ ಎಂದು ಬಗೆಯುವಳು. ಆದರೆ ಬಹುಬೇಗ ಅದು ತನಗೆ ಸಾಧ್ಯವಿಲ್ಲವೆಂದು ಅರಿಯುವಳು. ಪ್ರಪಂಚವೆಲ್ಲಾ ಪ್ರೀತಿಯ ಮಾತಿನಿಂದ ತುಂಬಿದೆ. ಆದರೆ ಪ್ರೀತಿಸುವುದು ಬಹಳ ಕಷ್ಟ, ಪ್ರೀತಿ ಎಲ್ಲಿದೆ? ಅಲ್ಲಿ ಪ್ರೀತಿ ಇದೆ ಎಂದು ನಿಮಗೆ ಹೇಗೆ ಗೊತ್ತು? ಪ್ರೀತಿಯ ಪ್ರಥಮ ಲಕ್ಷಣ ಅದಕ್ಕೆ ವ್ಯಾಪಾರ ತಿಳಿಯದು. ಎಲ್ಲಿಯವರೆವಿಗೂ ಒಬ್ಬನು ಮತ್ತೊಬ್ಬನಿಂದ ಏನನ್ನಾದರೂ ಪಡೆಯಬೇಕೆಂದು ಪ್ರೀತಿಸುತ್ತಿರುವನೋ, ಅಲ್ಲಿ ಪ್ರೀತಿ ಇಲ್ಲವೆಂದು ತಿಳಿಯಿರಿ. ಅದೊಂದು ವ್ಯಾಪಾರ. ಎಲ್ಲಿ, ಕೊಂಡು ಮಾರುವ ಪ್ರಶ್ನೆ ಇದೆಯೋ, ಅದು ಪ್ರೀತಿಯಲ್ಲ. ಒಬ್ಬನು ದೇವರನ್ನು, “ನನಗೆ ಇದನ್ನು ಕೊಡು, ಅದನ್ನು ಕೊಡು” ಎಂದು ಪ್ರಾರ್ಥಿಸಿದಾಗ ಅದು ಪ್ರೀತಿಯಲ್ಲ. ಅದು ಹೇಗೆ ಪ್ರೀತಿಯಾಗಬಲ್ಲದು? ನಾನು ನಿಮ್ಮನ್ನು ಹೊಗಳುತ್ತೇನೆ. ನೀವು ನನಗೆ ಅದರ ಬದಲು ಏನನ್ನಾದರೂ ಕೊಡುತ್ತೀರಿ, ಇದೊಂದು ವ್ಯಾಪಾರವಷ್ಟೆ.

ಒಬ್ಬ ದೊಡ್ಡ ರಾಜ ಬೇಟೆಗಾಗಿ ಕಾಡಿಗೆ ಹೋದನು. ಅಲ್ಲಿ ಒಬ್ಬ ಋಷಿಯನ್ನು ಕಂಡನು. ಆತನ ಹತ್ತಿರ ಸ್ವಲ್ಪ ಸಂಭಾಷಣೆ ಮಾಡಿದ ಮೇಲೆ, ಆತನನ್ನು ತುಂಬಾ ಮೆಚ್ಚಿ, ತನ್ನಿಂದ ಯಾವುದಾದರೊಂದು ಕಾಣಿಕೆಯನ್ನು ಸ್ವೀಕರಿಸಬೇಕೆಂದು ಕೇಳಿಕೊಂಡನು. ಋಷಿ ಹೀಗೆಂದನು: “ಬೇಕಾಗಿಲ್ಲ, ನನ್ನ ಸ್ಥಿತಿಯಲ್ಲಿಯೇ ನಾನು ತೃಪ್ತನಾಗಿರುವೆನು. ಈ ಮರಗಳು ನನಗೆ ತಿನ್ನುವಷ್ಟು ಹಣ್ಣನ್ನು ಕೊಡುತ್ತವೆ. ಈ ಸುಂದರವಾದ ಪರಿಶುದ್ಧ ಹೊಳೆ ನನಗೆ ಬೇಕಾದ ನೀರನ್ನೆಲ್ಲಾ ಒದಗಿಸುವುದು. ನಾನು ಈ ಗುಹೆಯಲ್ಲಿ ವಾಸಿಸುವೆನು. ನೀನು ಚಕ್ರವರ್ತಿಯಾಗಿದ್ದರೂ ನೀನು ಕೊಡುವ ಕಾಣಿಕೆಯಿಂದ ನನಗೆ ಏನು ಪ್ರಯೋಜನ?” ಚಕ್ರವರ್ತಿ “ನನ್ನ ಆತ್ಮ ಶುದ್ಧಿಗಾಗಿ, ನನ್ನ ತೃಪ್ತಿಗಾಗಿ ನೀವು ಕಾಣಿಕೆಯನ್ನು ಸ್ವೀಕರಿಸಿ, ರಾಜಧಾನಿಗೆ ಬನ್ನಿ” ಎಂದನು. ಕೊನೆಗೆ ಋಷಿ ಚಕ್ರವರ್ತಿಯೊಂದಿಗೆ ಹೋಗಲು ಒಪ್ಪಿದನು. ಆತನನ್ನು\break ಅರಮನೆಗೆ ಅವನು ಕರೆದುಕೊಂಡು ಹೋದನು. ಅಲ್ಲಿ ದ್ರವ್ಯ, ಹಲವು ಬೆಲೆಬಾಳುವ ನಗನಾಣ್ಯ, ಅಮೃತಶಿಲಾ ಖಚಿತ ನೆಲ ಇದ್ದವು. ಎಲ್ಲೆಲ್ಲೂ ಐಶ್ವರ್ಯ ಮತ್ತು ರಾಜವೈಭವ ಕಾಣುತ್ತಿತ್ತು. ಚಕ್ರವರ್ತಿ ಋಷಿಗೆ, “ನಾನು ಪ್ರಾರ್ಥನೆ ಮಾಡಿಕೊಂಡು ಬರುವವರೆಗೆ ಸ್ವಲ್ಪ ತಾಳಿ” ಎಂದು ಹೇಳಿ, ಒಂದು ಮೂಲೆಗೆ ಹೋಗಿ ಹೀಗೆ ಪ್ರಾರ್ಥಿಸಿದನು: “ದೇವರೆ, ನನಗೆ ಹೆಚ್ಚು ದ್ರವ್ಯವನ್ನು ಕೊಡು, ಹೆಚ್ಚು ಮಕ್ಕಳನ್ನು ಕೊಡು, ಹೆಚ್ಚು ರಾಜ್ಯವನ್ನು ಕೊಡು” ಎಂದು ಆ ಋಷಿ ಎದ್ದು ಹೋಗಲುಪಕ್ರಮಿಸಿದನು. ಅವನು ಹೋಗುತ್ತಿರುವುದನ್ನು ಚಕ್ರವರ್ತಿ ನೋಡಿ, ಹಿಂದೆಯೇ ಹೋಗಿ, “ನಿಲ್ಲಿ ಸ್ವಾಮಿ ನೀವು ನನ್ನಿಂದ ಇನ್ನೂ ಏನನ್ನೂ\break ಸ್ವೀಕರಿಸಿಲ್ಲ. ಆಗಲೇ ಹೋಗುತ್ತಿರುವಿರಲ್ಲ?” ಎಂದನು. ಋಷಿ ಅವನ ಕಡೆಗೆ ತಿರುಗಿ, “ನೀನೇ ಭಿಕ್ಷುಕ! ನಾನು ಭಿಕ್ಷುಕನನ್ನು ಬೇಡುವುದಿಲ್ಲ. ನೀನು ಏನು ಕೊಡಬಲ್ಲೆ, ನೀನೆ ಇದುವರೆವಿಗೂ ಬೇಡುತ್ತಿದ್ದೆ!” ಎಂದನು. ಅದು ಪ್ರೀತಿಯ ಭಾಷೆ ಅಲ್ಲ. ನೀವು ದೇವರನ್ನು ಇದು ಕೊಡು, ಅದು ಕೊಡು ಎಂದು ಬೇಡಿದರೆ, ಪ್ರೀತಿಗೂ ವ್ಯಾಪಾರಕ್ಕೂ ಏನು ವ್ಯತ್ಯಾಸವಿದೆ? ಪ್ರೀತಿಯ ಮೊದಲಿನ ಪರೀಕ್ಷೆ ಎಂದರೆ ಅಲ್ಲಿ ಲಾಭದ ದೃಷ್ಟಿ ಇಲ್ಲದಿರುವುದು. ಪ್ರೀತಿ ಯಾವಾಗಲೂ ಕೊಡುವುದು, ಎಂದಿಗೂ ಸ್ವೀಕರಿಸುವುದಿಲ್ಲ. ಭಗವಂತನ ಮಗು ಹೀಗೆ ಹೇಳುವುದು: “ದೇವರಿಗೆ ಬೇಕಾದರೆ ನನ್ನ ಸರ್ವಸ್ವವನ್ನೂ ಕೊಡುತ್ತೇನೆ. ಆದರೆ ನನಗೆ ಅವನಿಂದ ಏನೂ ಬೇಡ. ನನಗೆ ಈ ಪ್ರಪಂಚದಲ್ಲಿ ಏನೂ ಬೇಕಾಗಿಲ್ಲ. ನಾನು ಅವನನ್ನು ಪ್ರೀತಿಸಬೇಕೆಂದು ಇರುವುದರಿಂದ ಪ್ರೀತಿಸುತ್ತೇನೆ. ನನಗೆ ಅವನಿಂದ ಯಾವ ಉಪಕಾರವೂ ಬೇಕಾಗಿಲ್ಲ. ದೇವರು ಸರ್ವಶಕ್ತನೋ ಅಲ್ಲವೋ ಯಾರಿಗೆ ಬೇಕು? ನನಗೆ ಅವನಿಂದ ಯಾವ ಶಕ್ತಿಯೂ ಬೇಕಾಗಿಲ್ಲ. ಶಕ್ತಿಯ ಆವಿರ್ಭಾವವೂ ಬೇಕಾಗಿಲ್ಲ. ಅವನು ಪ್ರೇಮಮಯ ಭಗವಂತನೆಂಬುದೇ ನನಗೆ ಸಾಕು. ನಾನು ಮತ್ತಾವ ಪ್ರಶ್ನೆಯನ್ನೂ ಹಾಕುವುದಿಲ್ಲ.”

ಎರಡನೆಯ ಪರೀಕ್ಷೆ ಪ್ರೀತಿಗೆ ಅಂಜಿಕೆ ಇಲ್ಲ ಎಂಬುದು. ದೇವರೆಂದರೆ ಎಲ್ಲಿಯೋ ಮೋಡಗಳಾಚೆ ಕುಳಿತುಕೊಂಡಿರುವ ವ್ಯಕ್ತಿ; ಒಂದು ಕೈಯಿಂದ ವರಪ್ರದಾನ ಮಾಡುತ್ತ ಮತ್ತೊಂದು ಕೈಯಿಂದ ದಂಡಿಸುತ್ತ ಇರುವನು ಎಂಬ ಭಾವನೆ ಇರುವ ಪರಿಯಂತವೂ ಪ್ರೀತಿ ಇರುವುದಿಲ್ಲ. ಅಂಜಿಸಿ ಪ್ರೀತಿಸುವಂತೆ ಮಾಡಲು ಸಾಧ್ಯವೇ? ಕುರಿಮರಿ ಸಿಂಹವನ್ನು ಪ್ರೀತಿಸಬಲ್ಲದೆ? ಇಲಿ ಬೆಕ್ಕನ್ನು ಪ್ರೀತಿಸಬಲ್ಲದೆ? ಸೇವಕ ಸ್ವಾಮಿಯನ್ನು ಪ್ರೀತಿಸಬಲ್ಲನೆ? ಸೇವಕರು ಕೆಲವು ವೇಳೆ ಪ್ರೀತಿಸುವಂತೆ ನಟಿಸುವರು. ಆದರೆ ಅದು ಪ್ರೀತಿಯೇ? ಅಂಜಿಕೆಯಲ್ಲಿ ಪ್ರೇಮವಿರುವುದನ್ನು ನೀವೆಲ್ಲಿ ನೋಡಿರುವಿರಿ? ಅದು ಯಾವಾಗಲೂ ನಟನೆ. ಪ್ರೀತಿಯೊಂದಿಗೆ ಅಂಜಿಕೆಯ ಭಾವನೆ ಬರುವುದೇ ಇಲ್ಲ. ದಾರಿಯಲ್ಲಿರುವ ಕಿರಿಯ ತಾಯಿಯೊಬ್ಬಳನ್ನು ನೋಡಿ; ಒಂದು ನಾಯಿ ಬೊಗಳಿದರೆ ಹತ್ತಿರವಿರುವ ಮನೆಗೆ ಓಡುವಳು. ಮಾರನೇ ದಿನ ಅದೇ ರಸ್ತೆಯಲ್ಲಿ ತನ್ನ ಮಗುವಿನೊಂದಿಗೆ ಇರುವಾಗ ಒಂದು ಸಿಂಹ ಮಗುವಿನ ಕಡೆ ನುಗ್ಗಿತು ಎಂದು ಭಾವಿಸೋಣ. ಆಗ ಆ ಹೆಂಗಸು ಎಲ್ಲಿರುವಳು ಎಂದು ತಿಳಿಯುವಿರಿ. ತನ್ನ ಮಗುವನ್ನು ರಕ್ಷಿಸುವುದಕ್ಕಾಗಿ ಸಿಂಹದ ದವಡೆಯಲ್ಲಿರುವಳು. ಪ್ರೀತಿ ಅವಳ ಅಂಜಿಕೆಯನ್ನೆಲ್ಲ ಓಡಿಸಿತು. ಇದರಂತೆಯೇ ಭಗವಂತನ ಪ್ರೀತಿ ಕೂಡ. ದೇವರು ದುಷ್ಟ ನಿಗ್ರಹನೋ ಶಿಷ್ಟರಕ್ಷಕನೋ ಎಂಬುದು ಯಾರಿಗೆ ಬೇಕು? ಅದು ಪ್ರೀತಿಸುವವನ ಭಾವನೆ ಅಲ್ಲ. ನ್ಯಾಯಾಧಿಪತಿ ಮನೆಗೆ ಬಂದಾಗ ಅವನ ಹೆಂಡತಿ ಅವನಲ್ಲಿ ಏನನ್ನು ಕಾಣುವಳು? ನ್ಯಾಯಾಧಿಪತಿಯನ್ನು ಅಲ್ಲ, ಶಿಕ್ಷಿಸುವವನನ್ನು ಅಲ್ಲ, ಬಹುಮಾನವನ್ನು ಕೊಡುವವನನ್ನೂ ಅಲ್ಲ; ತನ್ನ ಪತಿಯನ್ನು ಕಾಣುವಳು, ಪ್ರಿಯತಮನನ್ನು ನೋಡುವಳು. ಅವನ ಮಕ್ಕಳು ಆತನಲ್ಲಿ ಏನನ್ನು ನೋಡುವರು? ಪ್ರೀತಿಸುವ ತಂದೆಯನ್ನು, ಶಿಕ್ಷಿಸುವವನನ್ನೂ ಅಲ್ಲ, ಬಹುಮಾನ ಕೊಡುವವನನ್ನೂ ಅಲ್ಲ. ಇದರಂತೆಯೇ ಭಗವಂತನ ಮಕ್ಕಳು ಅವನಲ್ಲಿ ಶಿಕ್ಷಕನನ್ನು ಕಾಣುವುದಿಲ್ಲ, ಪ್ರತಿಫಲ ಕೊಡುವವನನ್ನೂ ಕಾಣುವುದಿಲ್ಲ. ಯಾರು ಪ್ರೀತಿಯ ಸವಿಯನ್ನು ಅನುಭವಿಸಿಲ್ಲವೋ ಅವರು ಅಂಜುವರು, ನಡುಗುವರು. ಭಗವಂತ ಶಿಕ್ಷಕ, ವರದಾಯಕ ಎಂಬ ಭಯಂಕರ ಭಾವನೆಯು ಕಾಡುಮನುಷ್ಯರಿಗೆ ಸ್ವಲ್ಪ ಪ್ರಯೋಜನಕಾರಿಯಾಗಿರಬಹುದು. ಕೆಲವರು ಅತ್ಯಂತ ಬುದ್ಧಿವಂತರಾದರೂ, ಆಧ್ಯಾತ್ಮಿಕತೆಯಲ್ಲಿ ಅನಾಗರಿಕರು. ಅವರಿಗೆ ಈ ಭಾವನೆ ಸಹಕಾರಿಯಾಗಬಹುದು. ಆದರೆ ಯಾರು ಭಕ್ತರಾಗಿರುವರೋ, ಯಾರಲ್ಲಿ ಆಧ್ಯಾತ್ಮಿಕ ದೃಷ್ಟಿ ಜಾಗೃತವಾಗಿದೆಯೋ, ಅವರಿಗೆ ಅಂತಹವೆಲ್ಲ ಬಾಲಿಶ ಹಾಗೂ ಮೂರ್ಖಭಾವನೆಗಳು ಎಂಬುದು ಗೊತ್ತಾಗುತ್ತದೆ. ಅವರು ಅಂಜಿಕೆಯ ಭಾವನೆಯನ್ನೆಲ್ಲಾ ತ್ಯಜಿಸುತ್ತಾರೆ.

ಮೂರನೆಯದು ಇದಕ್ಕಿಂತಲೂ ಹೆಚ್ಚಿನ ಪರೀಕ್ಷೆ. ಪ್ರೇಮವೇ ಅತ್ಯುತ್ತಮ ಆದರ್ಶ. ಮೊದಲನೇ ಎರಡು ಅಂತಸ್ತುಗಳನ್ನು ದಾಟಿಹೋದ ಮೇಲೆ, ವ್ಯಾಪಾರ ದೃಷ್ಟಿಯನ್ನು ತ್ಯಜಿಸಿ, ಅಂಜಿಕೆಯನ್ನು ಕಿತ್ತೊಗೆದ ಮೇಲೆ, ಅನವರತವೂ ಪ್ರೇಮವೇ ಪರಮೋತ್ಕೃಷ್ಟ ಆದರ್ಶವೆಂದು ತಿಳಿಯುತ್ತಾನೆ. ಎಷ್ಟೋ ವೇಳೆ ಪ್ರಪಂಚದಲ್ಲಿ ಅತಿ ರೂಪವತಿಯಾದ ಸ್ತ್ರೀಯು ಕುರೂಪಿಯಾದ ಪುರುಷನಲ್ಲಿ ಅನುರಕ್ತಳಾಗಿರುವುದನ್ನು ನಾವು ನೋಡುತ್ತೇವೆ! ಎಷ್ಟೋ ವೇಳೆ ಪ್ರಪಂಚದಲ್ಲಿ ಸುಂದರ ಯುವಕ ಕುರೂಪಿ ಸ್ತ್ರೀಯಲ್ಲಿ ಅನುರಕ್ತನಾಗಿರುವುದನ್ನು ನೋಡುತ್ತೇವೆ! ಇದರ ಹಿಂದೆ ಇರುವ ಆಕರ್ಷಣೆ ಯಾವುದು? ಹೊರಗಿನವರು ಸುಂದರ ಯುವಕನನ್ನೋ ಅಥವಾ ಯುವತಿಯನ್ನೋ ನೋಡುವರು. ಆದರೆ ಪ್ರೇಮಿ ಹಾಗಲ್ಲ. ಪ್ರೇಮಿಗೆ ತನ್ನ ಪ್ರಿಯತಮೆ ಪ್ರಪಂಚದಲ್ಲಿ ಎಲ್ಲೂ ಇಲ್ಲದ ಪರಮೋತ್ಕೃಷ್ಟ ಲಾವಣ್ಯದ ಖನಿ. ಇದು ಹೇಗೆ? ಕುರೂಪಿಯನ್ನು ಪ್ರೀತಿಸುವ ನಾರಿ, ತಾನು ಪ್ರೀತಿಸುವ ಪುರುಷನಲ್ಲಿ ತನ್ನ ಪರಮೋತ್ಕೃಷ್ಟ ಲಾವಣ್ಯದ ಆದರ್ಶವನ್ನು ಆರೋಪಿಸುವಳು. ಅವಳು ಪ್ರೀತಿಸುವುದು ಪೂಜಿಸುವುದು ಕುರೂಪಿಯನ್ನಲ್ಲ, ತನ್ನ ಆದರ್ಶವನ್ನು. ಆ ಮನುಷ್ಯ ಒಂದು ಸೂಚನೆ ಮಾತ್ರ. ಆ ಸೂಚನೆಯ ಮೇಲೆ ತನ್ನ ಆದರ್ಶವನ್ನೆಲ್ಲಾ ಆರೋಪಿಸಿ ಮುಚ್ಚುವಳು. ಇದು ಅವಳ ಆರಾಧನೆಯ ಇಷ್ಟಮೂರ್ತಿಯಾಗುವುದು. ನಾವು ಪ್ರೀತಿಸುವ ಕಡೆಗಳಲ್ಲೆಲ್ಲಾ, ಇದೇ ಅನ್ವಯಿಸುವುದು. ನಮ್ಮಲ್ಲಿ ಹಲವರಿಗೆ ಸಾಧಾರಣರಾಗಿರುವ ಸಹೋದರ ಸಹೋದರಿಯರು ಇರುವರು. ಆದರೂ ಅವರು ನಮ್ಮ ಸಹೋದರ ಸಹೋದರಿಯ\-ರಾದುದರಿಂದ ಸುಂದರವಾಗಿ ಕಾಣುವರು.

ಇದರ ಹಿಂದೆ ಇರುವ ತತ್ತ್ವವೇ, ಪ್ರತಿಯೊಬ್ಬನೂ ತನ್ನ ಆದರ್ಶವನ್ನು ಆರೋಪಿಸಿ ಪೂಜಿಸುವನು ಎಂಬುದು. ಈ ಬಾಹ್ಯ ಜಗತ್ತೆಲ್ಲಾ ಒಂದು ಸೂಚನೆಯ ಜಗತ್ತು ಮಾತ್ರ. ನಾವು ಕಾಣುವ ಎಲ್ಲವನ್ನೂ ಮನಸ್ಸಿನಿಂದ ಆರೋಪಿಸುವೆವು. ಮುತ್ತಿನ ಹುಳುವಿನ ಚಿಪ್ಪಿನಲ್ಲಿ ಮರಳುಕಣವೊಂದು ಸೇರಿ ವೇದನೆಯನ್ನುಂಟು ಮಾಡುವುದು. ಈ ವೇದನೆ ಹುಳುವಿನಲ್ಲಿ ಒಂದು ವಿಧವಾದ ದ್ರವವನ್ನು ಸ್ರವಿಸಿ ಮರಳುಕಣವನ್ನು ಮುಚ್ಚುವುದು; ಇದರಿಂದಲೇ ಸುಂದರವಾದ ಮುತ್ತಾಗುವುದು. ಇದರಂತೆಯೇ ಬಾಹ್ಯ ಜಗತ್ತು ನಮಗೆ ಸೂಚನೆಯನ್ನು ಒದಗಿಸುವುದು. ಅವುಗಳ ಮೇಲೆ ನಮ್ಮ ಆದರ್ಶವನ್ನು ಆರೋಪಿಸಿ ನಮಗೆ ಬೇಕಾದ ವಿಷಯ\-ಗಳನ್ನಾಗಿ ಮಾಡಿಕೊಳ್ಳುವೆವು. ದುರ್ಜನರು ಈ ಜಗತ್ತನ್ನು ನರಕಮಯವಾಗಿ ಕಾಣುವರು. ಸಜ್ಜನರು ಇದನ್ನು ಸ್ವರ್ಗಮಯವೆಂದು ಭಾವಿಸುವರು. ಪ್ರಣಯಿಗಳು ಪ್ರಪಂಚವನ್ನು ಪ್ರೇಮಮಯವಾಗಿ ಕಾಣುವರು. ದ್ವೇಷಿ, ದ್ವೇಷದಿಂದ ತುಂಬಿ ತುಳುಕಾಡುವಂತೆ ಕಾಣುವನು. ಜಗಳ ಕಾಯುವವರಿಗೆ ಇಲ್ಲಿ ಹೋರಾಟವಲ್ಲದೆ ಮತ್ತೇನೂ ಕಾಣುವುದಿಲ್ಲ. ಶಾಂತಾತ್ಮನಿಗೆ ಶಾಂತಿಯಲ್ಲದೆ ಮತ್ತೇನೂ ಕಾಣುವುದಿಲ್ಲ. ಪೂರ್ಣಾತ್ಮನಿಗೆ ದೇವರಲ್ಲದೆ ಮತ್ತೇನೂ ಕಾಣುವುದಿಲ್ಲ. ಹೀಗೆ ನಾವು ಯಾವಾಗಲೂ ಪೂಜಿಸುವುದು ನಮ್ಮ ಆದರ್ಶದ ಪರಾಕಾಷ್ಠೆಯನ್ನು. ಅದನ್ನು ನಾವು ಮುಟ್ಟಿದ ಮೇಲೆ, ಆದರ್ಶವನ್ನು ಆದರ್ಶದಂತೆ ಪ್ರೀತಿಸಿದ ಮೇಲೆ, ವಾದ ಸಂಶಯಗಳ ತುಮುಲದಿಂದ ಒಂದೇ ಸಲ ಪಾರಾಗುವೆವು. ದೇವರನ್ನು ಹೊರಗೆ ತೋರಿಸುವುದಕ್ಕೆ ಸಾಧ್ಯವೇ ಅಥವಾ ಇಲ್ಲವೆ ಎಂಬುದು ಯಾರಿಗೆ ಬೇಕು? ಆದರ್ಶವೆಂದಿಗೂ ಕಣ್ಮರೆಯಾಗದು. ಅದು ನನ್ನ ಸ್ವಭಾವದ ಅಂಶವಾಗಿದೆ. ನನ್ನ ಇರವನ್ನೇ ನಾನು ಅನುಮಾನಿಸಿದಾಗ, ಆದರ್ಶವನ್ನು ಅನುಮಾನಿಸುತ್ತೇನೆ. ನಾನು ಮೊದಲ\-ನೆಯದನ್ನು ಹೇಗೆ ಪ್ರಶ್ನಿಸಲಾರೆನೋ ಅದರಂತೆಯೆ ಆದರ್ಶವನ್ನೂ ಪ್ರಶ್ನಿಸಲಾರೆ. ಏಕಕಾಲದಲ್ಲಿ ದೇವರು ಸರ್ವಶಕ್ತನಾಗಿ ದಯಾಮಯನಾಗಿರಲು ಸಾಧ್ಯವೆ? ಇವೆಲ್ಲಾ ಯಾರಿಗೆ ಬೇಕು? ಅವನು ಮಾನವ ಕೋಟಿಗೆ ಮಂಗಳಪ್ರದಾಯಕನೋ, ಪ್ರಜಾಪೀಡಕನಂತೆ ಅವನು ನಮ್ಮನ್ನು ನೋಡುತ್ತಿರುವನೋ ಅದು ಬೇಕಾಗಿಲ್ಲ.

ಭಕ್ತನು ಶಿಕ್ಷೆ, ಬಹುಮಾನಗಳನ್ನು ಮೀರಿಹೋಗಿರುವನು; ಭಯ, ಸಂದೇಹಗಳನ್ನು ಮೀರಿ ಹೋಗಿರುವನು. ವೈಜ್ಞಾನಿಕ ಅಥವಾ ಮತ್ತಾವುದಾದರೂ ರೀತಿಯಲ್ಲಿ ದೇವರನ್ನು ಪ್ರಮಾಣಿಕರಿಸಲು ಸಾಧ್ಯವೆ ಎಂಬುದನ್ನು ಮೀರಿಹೋಗಿರುವನು. ಅವನಿಗೆ ಪ್ರೇಮಾದರ್ಶವೊಂದು ಸಾಕಾಗಿದೆ. ಈ ವಿಶ್ವ, ಪ್ರೇಮದ ಒಂದು ಆವಿರ್ಭಾವ ಎಂಬುದೊಂದು ಸ್ವತಸ್ಸಿದ್ಧ ಪ್ರಮಾಣವಲ್ಲವೇ? ಈ ಒಂದು ಕಣ ಮತ್ತೊಂದು ಕಣದೊಂದಿಗೆ, ಅಣು ಅಣುವಿನೊಂದಿಗೆ ಸೇರುವಂತೆ ಮಾಡುವುದು, ಗ್ರಹಗಳು ಒಂದು ಮತ್ತೊಂದನ್ನು ಅನುಸರಿಸಿಕೊಂಡು ಹೋಗುವಂತೆ ಮಾಡುವುದು ಯಾವುದು? ಮನುಷ್ಯ ಮನುಷ್ಯನನ್ನು, ಸ್ತ್ರೀ ಪುರುಷನನ್ನು, ಪುರುಷ ಸ್ತ್ರೀಯನ್ನೂ, ಪ್ರಾಣಿ ಪ್ರಾಣಿಗಳನ್ನು ಆಕರ್ಷಿಸಿ, ಇಡೀ ವಿಶ್ವವನ್ನೇ ಒಂದು ಕೇಂದ್ರದ ಕಡೆ ಎಳೆಯುವಂತೆ ಮಾಡಿರುವುದು ಯಾವುದು? ಇದೇ ಪ್ರೇಮವೆನ್ನುವುದು. ಅತಿ ಕ್ಷುದ್ರಕಣದಿಂದ ಹಿಡಿದು ಪರಮೇಶ್ವರನವರೆಗೂ ಇರುವುದೆಲ್ಲಾ ಇದರ ಆವಿರ್ಭಾವವೆ. ಸರ್ವಶಕ್ತಿಮಾನ್​, ಸರ್ವವ್ಯಾಪಿ ಈ ಪ್ರೇಮ. ಯಾವುದು ಚೇತನದಲ್ಲಿ ಅಚೇತನದಲ್ಲಿ, ವ್ಯಷ್ಟಿಯಲ್ಲಿ ಸಮಷ್ಟಿಯಲ್ಲಿ, ಪರಸ್ಪರ ಆಕರ್ಷಣೆಯಂತೆ ಇದೆಯೊ, ಅದೇ ಪ್ರೀತಿ. ವಿಶ್ವದಲ್ಲಿರುವ ಏಕಮಾತ್ರ ಪ್ರಚೋದಕ ಶಕ್ತಿ ಇದು. ಆ ಪ್ರೇಮಾಕರ್ಷಣೆಯ ವೇಗದಿಂದ ಮಾನವಕೋಟಿಗೆ ಕ್ರೈಸ್ತ ತನ್ನ ಪ್ರಾಣವನ್ನು ಕೊಡುವನು; ಬುದ್ಧ ಒಂದು ಪ್ರಾಣಿಗೆ ತನ್ನ ಜೀವದಾನವನ್ನು ಮಾಡುವನು; ತಾಯಿ ಮಗುವಿಗೆ, ಗಂಡ ಹೆಂಡತಿಗೆ ಪ್ರಾಣ ಕೊಡುವರು. ಅದೇ ಪ್ರೇಮಾಕರ್ಷಣದ ಉದ್ವೇಗದಿಂದಲೇ, ದೇಶಕ್ಕಾಗಿ ತಮ್ಮ ಪ್ರಾಣವನ್ನು ಈಡಾಡಲು ಜನರು ಸಿದ್ಧರಾಗಿರುವರು. ಆಶ್ಚರ್ಯವೆಂದರೆ ಅದೇ ಪ್ರೇಮಾಕರ್ಷಣೆಗೆ ಸಿಲುಕಿಯೇ ಕಳ್ಳ\break ಕದಿಯುವುದು, ಕೊಲೆಪಾತಕಿ ಕೊಲ್ಲುವುದು. ಆದರೂ ಇಂತಹ ಸನ್ನಿವೇಶಗಳಲ್ಲಿಯೂ ಸ್ಫೂರ್ತಿಯೊಂದಿದೆ; ಆದರೆ ಅದರ ಆವಿರ್ಭಾವದಲ್ಲಿ ವ್ಯತ್ಯಾಸ ಕಾಣುವುದು. ಇದೊಂದೇ ಪ್ರಪಂಚದಲ್ಲಿರುವ ಏಕಮಾತ್ರ ಪ್ರಚೋದಕ ಶಕ್ತಿ. ಕಳ್ಳನಿಗೆ ಚಿನ್ನದ ಮೇಲೆ ಆಸೆ, ಅಲ್ಲಿ ಪ್ರೀತಿ ಇದೆ, ಆದರೆ ಅದು ತಪ್ಪುದಾರಿಗೆ ತಿರುಗಿರುವುದು. ಹಾಗೆ ಎಲ್ಲಾ ಪುಣ್ಯ ಮತ್ತು ಪಾಪ ಕೆಲಸಗಳ ಹಿಂದೆ ಅನಂತ ಪ್ರೀತಿ ಇರುವುದು. ನ್ಯೂಯಾರ್ಕಿನ ನಿರಾಶ್ರಿತರಿಗಾಗಿ ಒಬ್ಬ ತನ್ನ ಕೋಣೆಯಲ್ಲಿ ಒಂದು ಸಾವಿರ ಡಾಲರುಗಳಿಗೆ ಚೆಕ್ಕನ್ನು ಬರೆದರೆ, ಅದೇ ಕಾಲದಲ್ಲಿ ಅದೇ ಕೋಣೆಯಲ್ಲಿ ಮತ್ತೊಬ್ಬ ಕಳ್ಳ ರುಜು ಹಾಕುವನು ಎಂದು ಇಟ್ಟುಕೊಳ್ಳೋಣ. ಇಬ್ಬರೂ ಒಂದೇ ಬೆಳಕಿನಲ್ಲಿ ಬರೆದರು. ಆದರೂ ಪ್ರತಿಯೊಬ್ಬನೂ ಹೇಗೆ ಅದನ್ನು ಉಪಯೋಗಿಸಿಕೊಂಡನೊ ಅದಕ್ಕೆ ಅವನೇ ಜವಾಬ್ದಾರ, ಬೆಳಕಲ್ಲ, ಅದಕ್ಕಾಗಿ ನಾವು ಬೆಳಕನ್ನು ಹೊಗಳುವುದು ಅಥವಾ ತೆಗಳುವುದು ಸರಿಯಲ್ಲ. ಎಲ್ಲದರಲ್ಲಿಯೂ ನಿಸ್ಸಂಗಿಯಾಗಿ ಪ್ರಕಾಶಿ\-ಸುತ್ತಿರುವುದು ಪ್ರೇಮ. ಇದೇ ವಿಶ್ವಚಾಲಕ ಶಕ್ತಿ. ಇದಿಲ್ಲದೇ ಇದ್ದರೆ ವಿಶ್ವವು ಕ್ಷಣಾರ್ಧದಲ್ಲಿ ನುಚ್ಚುನೂರಾಗಿ ಹೋಗುವುದು. ಈ ಪ್ರೇಮವೇ ಈಶ್ವರ.

“ಪ್ರೀಯತಮೆ! ಯಾರೂ ಪತಿಯನ್ನು ಪತಿಗೋಸುಗ ಪ್ರೀತಿಸುವುದಿಲ್ಲ, ಪತಿಯಲ್ಲಿರುವ ಆತ್ಮನಿಗಾಗಿ; ಯಾರೂ ಸತಿಯನ್ನು ಸತಿಗೋಸುಗ ಪ್ರೀತಿಸುವುದಿಲ್ಲ, ಸತಿಯಲ್ಲಿರುವ ಆತ್ಮಕ್ಕಾಗಿ; ಯಾರೂ ಎಂದಿಗೂ ಏನನ್ನೂ ಆತ್ಮನಿಗಾಗಿ ಅಲ್ಲದೆ ಅನ್ಯಥಾ ಪ್ರೀತಿಸುವುದಿಲ್ಲ.” ನಿಂದಾಸ್ಪದವಾದ ಸ್ವಾರ್ಥ ಕೂಡ ಪ್ರೇಮದ ಒಂದು ಅಂಶ. ಈ ಆಟದಿಂದ ಹೊರಗೆ ನಿಲ್ಲಿ, ಅದರಲ್ಲಿ ಬೆರೆಯಬೇಡಿ. ದಿವ್ಯ ದೃಶ್ಯಗಳನ್ನು ನೋಡಿ. ಅಂಕವಾದ ಮೇಲೆ ಅಂಕದಂತೆ ಸಾಗುತ್ತಿರುವ ಅದ್ಭುತ ಸೃಷ್ಟಿನಾಟಕವನ್ನು ನೋಡಿ. ಅದರಲ್ಲಿರುವ ಸಾಮರಸ್ಯ–ಮೇಳಗಳನ್ನು ಕೇಳಿ. ಎಲ್ಲಾ ಒಂದೇ ಪ್ರೇಮದ ಅನಂತ ದೃಶ್ಯಗಳು. ಸ್ವಾರ್ಥದಲ್ಲಿಯೂ ಆ ಜೀವ ವಿಕಾಸವಾಗುವುದು. ಮಕ್ಕಳಾದ ಮೇಲೆ ಹಲವಾಗುವುದು; ಹೀಗೆ ಮಾನವ, ಪೃಥ್ವಿಯೆಲ್ಲ ತನ್ನದಾಗುವವರೆಗೆ, ವಿಶ್ವವೇ ತನ್ನದಾಗುವವರೆಗೆ ವಿಶಾಲವಾಗುತ್ತಾ ಹೋಗುವನು. ವಿಶ್ವ ಪ್ರೇಮದ ಖನಿಯಾಗುವನು, ಅನಂತ ಪ್ರೇಮವಾಗುವನು. ಆ ಪ್ರೇಮವೇ ಈಶ್ವರ.

ಹೀಗೆ ಪರಾಭಕ್ತಿಯನ್ನು ಪ್ರವೇಶಿಸುವೆವು. ಇಲ್ಲಿ ಆಕಾರ ಪ್ರತೀಕಗಳು ಮಾಯವಾಗುವುವು. ಯಾರು ಅದನ್ನು ಪಡೆದಿರುವರೋ ಅವರು ಯಾವ ಪಂಥಕ್ಕೂ ಸೇರಲಾರರು. ಏಕೆಂದರೆ ಎಲ್ಲಾ ಪಂಥಗಳೂ ಅವರಲ್ಲಿರುವುವು. ಅವನು ಸೇರುವುದು ಯಾವುದಕ್ಕೆ? ಎಲ್ಲಾ ಚರ್ಚು–ದೇವಸ್ಥಾನಗಳೂ ಅವನಲ್ಲಿರುವುವು. ಅವನನ್ನು ಅಳವಡಿಸಿಕೊಳ್ಳುವ ದೊಡ್ಡ ಚರ್ಚು ಎಲ್ಲಿರುವುದು? ಅಂತಹ ವ್ಯಕ್ತಿ ಕೆಲವು ಆಚಾರಗಳ ಬೇಲಿಯೊಳಗೆ ಇರಲಾರ. ಯಾವ ಅನಂತ ಪ್ರೇಮದೊಂದಿಗೆ ಅವನು ಐಕ್ಯನಾಗಿರುವುವನೋ ಅದಕ್ಕೆ ಪರಿಮಿತಿ ಎಲ್ಲಿರುವುದು? ಈ ಪ್ರೇಮಾದರ್ಶನವನ್ನು ಅನುಸರಿಸಿರುವ ಎಲ್ಲಾ ಧರ್ಮಗಳೂ ಅದನ್ನು ವ್ಯಕ್ತಪಡಿಸಲು ಪ್ರಯತ್ನಿಸುತ್ತವೆ. ನಮಗೆ ಈ ಪ್ರೇಮದ ಪರಿಚಯವಿದ್ದರೂ, ಪ್ರಪಂಚದಲ್ಲಿರುವ ಪರಸ್ಪರ ಅನುರಾಗ ಆಕರ್ಷಣೆಗಳೆಲ್ಲ ಇದರ ಆವಿರ್ಭಾವಗಳು ಎಂದು ಗೊತ್ತಾದರೂ, ಎಲ್ಲಾ ದೇಶದ ಭಕ್ತರು ಮಹಾತ್ಮರು ಇದನ್ನು ವಿವರಿಸುವುದಕ್ಕೆ ಭಾಷಾಶಕ್ತಿಯನ್ನೆಲ್ಲಾ ಉಪಯೋಗಿಸಿ, ಸಾಲದೆ, ಶೃಂಗಾರಭಾವನೆಗಳನ್ನು ರೂಪಾಂತರಗೊಳಿಸಿ ಅದನ್ನು ದೈವೀ ಪ್ರೇಮದ ಭಾವದಲ್ಲಿ ಉಪಯೋಗಿಸಿರುವರು.

ಯೆಹೂದ್ಯ ರಾಜರ್ಷಿ ಹೀಗೆ ಹಾಡಿದನು, ಭರತಖಂಡದ ಭಕ್ತರೂ ಹೀಗೆ ಹಾಡಿದರು; “ಪ್ರಿಯತಮೆಯೆ! ನಿನ್ನ ಮುದ್ದು ತುಟಿಯ ಒಂದು ಚುಂಬನವಾದ ಮೇಲೆ ಎಂದಿಗೂ ನಿನ್ನ ಮೇಲಿನ ಹಂಬಲ ಹೆಚ್ಚುವುದು! ದುಃಖ ಶಮನವಾಗುವುದು. ಭೂತ ಭವಿಷ್ಯತ್​ ವರ್ತಮಾನಗಳನ್ನೆಲ್ಲ ಮರೆಯುವೆನು. ನಿನ್ನನ್ನು ಮಾತ್ರ ಯೋಚಿಸುವೆನು.” ಉಳಿದೆಲ್ಲ ಆಸೆಗಳು ಮಾಯವಾದ ಮೇಲೆ ಇರುವುದೇ ಪ್ರೇಮೋನ್ಮಾದ. “ಮುಕ್ತಿ ಯಾರಿಗೆ ಬೇಕು? ಉದ್ಧಾರ\-ವಾಗಲೂ ಯಾರು ಯತ್ನಿಸುವರು? ಪೂರ್ಣವಾಗುವ ಇಚ್ಛೆ ತಾನೇ ಯಾರಿಗೆ ಇದೆ? ಸ್ವಾತಂತ್ರ್ಯ ಯಾರಿಗೆ ಬೇಕು?” ಎನ್ನುವನು ಈಶ್ವರಾನುರಾಗಿ.

“ನನಗೆ ದ್ರವ್ಯ ಬೇಡ, ಆರೋಗ್ಯ ಬೇಡ, ನನಗೆ ಸೌಂದರ್ಯ ಬೇಡ, ಬುದ್ಧಿ ಬೇಡ; ಪ್ರಪಂಚದಲ್ಲಿರುವ ಹಲವು ಪಾಪಗಳ ಮಧ್ಯೆ ನಾನು ಪದೇ ಪದೇ ಜನಿಸಿದರೂ ಗೊಣಗಾಡುವುದಿಲ್ಲ. ನಾನು ನಿನ್ನನ್ನು ಪ್ರೀತಿಗೋಸುಗವಾಗಿ ಪ್ರೀತಿಸುತ್ತೇನೆ.” ಈ ಹಾಡುಗಳಲ್ಲಿ ಹೊರದೋರುವ ಪ್ರೇಮಾಧಿಕ್ಯವಿದು; ಇದು ಅತಿ ತೀವ್ರವಾದುದು. ಅತ್ಯುತ್ತಮವಾದ ಅತಿ ಪ್ರಬಲವಾಗಿರುವ ಆಕರ್ಷಣೀಯವಾಗಿರುವ ಮಾನವೀಯ ಪ್ರೇಮವೇ ಸ್ತ್ರೀಪುರುಷರ ಪರಸ್ಪರಾನುರಾಗ; ಆದಕಾರಣವೇ ಇಂತಹ ಭವ್ಯವಾದ ಭಕ್ತಿಯನ್ನು ವಿವರಿಸುವುದಕ್ಕೆ ಈ ಬಗೆಯ ಭಾಷೆಯನ್ನು ಉಪಯೋಗಿಸುವರು. ಈ ಮಾನವೀಯ ಪ್ರೀತಿಯ ಉನ್ಮಾದ, ಭಕ್ತರ ದೈವೋನ್ಮಾದದ ಅತಿ ಕ್ಷೀಣ ಮರುಧ್ವನಿ ಮಾತ್ರ. ನಿಜವಾದ ಈಶ್ವರಾನುರಾಗಿಗಳು ಭಗವದ್ಭಕ್ತಿ ಎಂಬ ಅಮೃತಪಾನದಿಂದ ಉನ್ಮತ್ತರಾಗಬಯಸುವರು, ಈಶ್ವರಾನುರಾಗಿಗಳಾದ ಭಕ್ತರಾಗಬಯಸುವರು. ಪ್ರತಿ ಯೊಂದು ಧರ್ಮದ ಸಂತಮಹಾಪುರುಷರೂ ತಮ್ಮ ಇಡೀ ಸಾಧನೆಯ ಅಮೃತವನ್ನು ಧಾರೆ ಎರೆದು ಯಾವ ಫಲಾಪೇಕ್ಷೆಯನ್ನೂ ಇಚ್ಛಿಸದೆ, ಕೇವಲ ಈಶ್ವರಾನುರಾಗ ಒಂದನ್ನೇ ತಮ್ಮ ಜೀವನದ ಚರಮ ಗುರಿಯಾಗಿ ಮಾಡಿಕೊಂಡು, ಜೀವನದ ಆಕಾಂಕ್ಷೆಯ ಸಾರವನ್ನು ಸುರಿದು, ಅಣಿಮಾಡಿರುವ ಪ್ರೇಮಾಮೃತ ಕಲಶವನ್ನು ಹೀರಿ ಮೈಮರೆಯಲೆತ್ನಿಸುವರು. ಪ್ರೀತಿಯ ಫಲ ಪ್ರೀತಿಯೊಂದೆ. ಇದು ಎಂತಹ ಪ್ರತಿಫಲ! ನಮ್ಮ ಜೀವನದ ದುಃಖವನ್ನೆಲ್ಲ ಮೈಮರೆಸುವುದು ಇದೊಂದೇ: ಭವರೋಗವನ್ನು ನಾಶಮಾಡುವ ಪಾನೀಯ ಇದೊಂದೇ. ಇದರಿಂದ ಮಾನವ ದೈವೋನ್ಮತ್ತನಾಗುವನು, ತಾನು ಮನುಷ್ಯನೆಂಬುದನ್ನು ಮರೆಯುವನು.

ಕೊನೆಗೆ, ಎಲ್ಲಾ ಬಗೆಯ ಸಿದ್ಧಾಂತಗಳು ಪರಿಪೂರ್ಣ ಏಕತೆಯಲ್ಲಿ ಒಂದಾಗುವುವು. ನಾವು ಯಾವಾಗಲೂ ದ್ವೈತಿಗಳಂತೆ ಪ್ರಾರಂಭಿಸುವೆವು. ದೇವರು ಬೇರೆ, ಮಾನವ ಬೇರೆ. ಅನುರಾಗ ಅವರಿಬ್ಬರ ನಡುವೆ ಬರುವುದು; ಮಾನವ ದೇವರೆಡೆಗೆ ನಡೆಯುವನು, ದೇವರು ಮಾನವನೆಡೆಗೆ ಬಂದಂತೆ ತೋರುವುದು. ಮಾನವನು ತಂದೆಯಂತೆ ತಾಯಿಯಂತೆ ಸ್ನೇಹಿತನಂತೆ ಪ್ರಣಯಿಯಂತೆ ಹಲವು ಮಾನವೀಯ ಸಂಬಂಧಗಳ ಮೂಲಕ ದೇವರನ್ನು ಪ್ರೀತಿಸಿ, ಕೊನೆಗೆ ತನ್ನ ಆರಾಧನೆಯ ಇಷ್ಟ ದೈವದಲ್ಲಿ ತನ್ಮಯನಾದಾಗ ಪರಾಕಾಷ್ಠೆಯನ್ನು ಮುಟ್ಟುವನು. “ನಾನೇ ನೀನು, ನೀನೇ ನಾನು, ನಿನ್ನ ಪೂಜೆಯೇ ನನ್ನ ಪೂಜೆ, ನನ್ನ ಪೂಜೆಯೇ ನಿನ್ನ ಪೂಜೆ.” ಇಲ್ಲಿ ಮಾನವನು ತಾನು ಪ್ರಾರಂಭಿಸಿದುದರ ಅಂತಿಮಗುರಿ ಸಿಕ್ಕುವುದು. ಮೊದಲು ತನಗಾಗಿ ಪ್ರೀತಿ. ಕ್ಷುದ್ರ ಜೀವನದ ಕಾದಾಟ ಪ್ರೀತಿಯನ್ನು ಸ್ವಾರ್ಥವನ್ನಾಗಿ ಮಾಡಿತು. ಕೊನೆಗೆ ಪ್ರೇಮದ ಪೂರ್ಣಪ್ರಕಾಶ ಬೆಳಗಿತು; ಅಂದು ಕ್ಷುದ್ರಜೀವವು ಪರಮಾತ್ಮನಾಯಿತು. ಮೊದಲು ಎಲ್ಲಿಯೋ ಹೊರಗೆ ಇರುವ ದೇವರು, ಅನಂತರ ವಿಶ್ವಪ್ರೇಮವಾಗಿ ಪರಿಣಮಿಸಿದನು. ಮಾನವನೂ ರೂಪಾಂತರ ಹೊಂದಿದನು. ಅವನೂ ದೇವರನ್ನು ಸಮೀಪಿಸುತ್ತಿದ್ದನು. ಮೊದಲು ತನ್ನಲ್ಲಿ ತುಂಬಿದ್ದ ಕೆಲಸಕ್ಕೆ ಬಾರದ ಆಸೆಗಳನ್ನೆಲ್ಲ ಹೊರಗೆ ಎಸೆಯುತ್ತ ಬಂದನು. ಆಸೆಯೊಂದಿಗೆ ಸ್ವಾರ್ಥ ಮಾಯವಾಯಿತು. ತುತ್ತತುದಿಯಲ್ಲಿ ಪ್ರೇಮಿ ಪ್ರೇಯಸಿ ಎರಡೂ ಒಂದೇ ಎಂದು ಕಂಡವು.

