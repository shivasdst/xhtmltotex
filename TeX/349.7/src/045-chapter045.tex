
\chapter[ಆತ್ಮ ಮತ್ತು ಪ್ರಕೃತಿ ]{ಆತ್ಮ ಮತ್ತು ಪ್ರಕೃತಿ \protect\footnote{\engfoot{C.W. Vol. VI, P. 98}}}

ಆತ್ಮನನ್ನು ಆತ್ಮವಾಗಿ ತಿಳಿಯುವುದೇ ಧರ್ಮ; ಅದನ್ನು ದ್ರವ್ಯವೆಂದು ಭಾವಿಸುವುದಲ್ಲ. ಧರ್ಮ ಒಂದು ಬೆಳವಣಿಗೆ. ಪ್ರತಿಯೊಬ್ಬನೂ ಅದನ್ನು ತಾನೇ ಅನುಭವಕ್ಕೆ ತಂದುಕೊಳ್ಳಬೇಕು. ಕ್ರೈಸ್ತರು ಏಸುವು ಮಾನವರ ಉದ್ಧಾರಕ್ಕಾಗಿ ಸತ್ತನೆಂದು ನಂಬುವರು. ನಿಮ್ಮಲ್ಲಿ (ಪಾಶ್ಚಾತ್ಯರಲ್ಲಿ) ಧರ್ಮವೆಂದರೆ ಒಂದು ಸಿದ್ಧಾಂತದಲ್ಲಿ ನಂಬಿಕೆ. ನಿಮ್ಮ ಮುಕ್ತಿಗೆ ಇದೇ ದಾರಿ ಎಂದು ನಂಬುವಿರಿ. ಆದರೆ ನಮ್ಮಲ್ಲಿ ಸಿದ್ಧಾಂತಕ್ಕೂ ಮುಕ್ತಿಗೂ ಏನೂ ಸಂಬಂಧವಿಲ್ಲ. ಪ್ರತಿಯೊಬ್ಬನೂ ತನಗೆ ತೋಚಿದ ಸಿದ್ಧಾಂತವನ್ನು ನಂಬಬಹುದು ಅಥವಾ ಯಾವ ಸಿದ್ಧಾಂತವನ್ನೂ ನಂಬದೇ ಇರಬಹುದು. ಕ್ರಿಸ್ತ ಯಾವುದೋ ಒಂದು ಕಾಲದಲ್ಲಿ ಇದ್ದ ಎಂದು ನಂಬುವುದರಿಂದ ನಿಮ್ಮಲ್ಲಿ ಯಾವ ವ್ಯತ್ಯಾಸ ಆಗುವುದು? ಮೋಸೆಸ್​ ಉರಿಯು ತ್ತಿರುವ ಹೊಗೆಯಲ್ಲಿ ದೇವರನ್ನು ಕಂಡ ಎಂದು ನಂಬುವುದರಿಂದ ನಿಮ್ಮಲ್ಲಿ ಯಾವ ಬದಲಾವಣೆ ಆಗುವುದು? ಹೀಗೆ ನಂಬುವುದರಿಂದ ಮೋಸೆಸ್​ಗಳನ್ನು ನೋಡಿದಂತೆ ಆಗಲಿಲ್ಲ, ಅಲ್ಲವೆ? ಹಾಗೇನಾದರೂ ಆದರೆ ಅವನು ಊಟಮಾಡಿದ ಎಂದು ನಂಬಿದರೆ ಸಾಕು, ನೀವು ಊಟ ಮಾಡಬೇಕಾಗಿಲ್ಲ. ಒಂದು ವಾದವು ಮತ್ತೊಂದರಷ್ಟೇ ಯುಕ್ತಿಯುಕ್ತವಾಗಿದೆ. ವಾದ ಸರಣಿಯೆಲ್ಲಾ ಒಂದೇ. ಹಿಂದಿ ನವರು ಏನು ಮಾಡಿದರು ಎಂಬುದರಿಂದ ಏನೂ ಪ್ರಯೋಜನವಿಲ್ಲ. ಹಾಗೆ ಮಾಡುವಂತೆ ನಮ್ಮನ್ನು ಪ್ರೇರೇಪಿಸಿದರೆ ಮಾತ್ರ ಅವುಗಳಿಂದ ಸ್ವಲ್ಪ ಪ್ರಯೋಜನ. ಕ್ರಿಸ್ತ, ಮೋಸೆಸ್​ ಮುಂತಾದವರು ಏನು ಮಾಡಿದರೋ ಅದರಂತೆ ಮಾಡಲು ನಮ್ಮನ್ನೂ ಪ್ರೋತ್ಸಾಹಿಸದೆ ಇದ್ದರೆ, ಅದರಿಂದ ನಮಗೆ ಎಳ್ಳಷ್ಟೂ ಪ್ರಯೋಜನ ವಿಲ್ಲ.

ಪ್ರತಿಯೊಬ್ಬನಿಗೂ ತನ್ನದೇ ಆದ ಒಂದು ವಿಶಿಷ್ಟ ಸ್ವಭಾವವಿದೆ. ಅವನು ಆ ಸ್ವಭಾವವನ್ನು ಅನುಸರಿಸಬೇಕು. ಅವನಿಗೆ ಅದರ ಮೂಲಕ ಮುಕ್ತಿ ದೊರಕು ವುದು. ನಿಮ್ಮ ಸ್ವಭಾವಕ್ಕೆ ತಕ್ಕ ಮಾರ್ಗ ಯಾವುದು ಎಂಬುದನ್ನು ನಿಮ್ಮ ಗುರು ಹೇಳಬಲ್ಲವನಾಗಿರಬೇಕು ಮತ್ತು ನಿಮ್ಮನ್ನು ಆ ಮಾರ್ಗದಲ್ಲಿ ನಡೆಸಬೇಕು. ಅವನು ನಿಮ್ಮ ಮುಖವನ್ನು ನೋಡಿದೊಡನೆಯೇ ನೀವು ಯಾವ ಮಾರ್ಗಕ್ಕೆ ಯೋಗ್ಯರು ಎಂಬುದನ್ನು ಅರಿತು ನಿಮಗೆ ಅದನ್ನು ತೋರಬಲ್ಲವನಾಗಿರಬೇಕು. ನೀವು ಮತ್ತೊಬ್ಬನ ಮಾರ್ಗವನ್ನು ಅನುಸರಿಸಕೂಡದು. ಏಕೆಂದರೆ ಅದು ಅವನ ಮಾರ್ಗ, ನಿಮ್ಮದಲ್ಲ. ನಿಮ್ಮ ಮಾರ್ಗವನ್ನು ಅವನು ಅನುಸರಿಸಕೂಡದು. ಏಕೆಂದರೆ ಅದು ನಿಮ್ಮ ಮಾರ್ಗ, ಅವನದಲ್ಲ. ನಿಮ್ಮ ಮಾರ್ಗವನ್ನು ನೀವು ಅರಿತ ಮೇಲೆ ಆ ಮಾರ್ಗದ ಮೇಲೆ ತೇಲಿಕೊಂಡುಹೋದರೆ ಸಾಕು, ಅದು ನಿಮ್ಮನ್ನು ಮುಕ್ತಿಗೆ ಒಯ್ಯುವುದು. ನೀವು ಅದನ್ನು ಅರಿತ ಮೇಲೆ ಅದರಿಂದ ಕದಲಬೇಡಿ. ನಿಮ್ಮ ಮಾರ್ಗವೇ ನಿಮಗೆ ಒಳ್ಳೆಯದು. ಆದರೆ ಅದು ಎಲ್ಲರಿಗೂ ಒಳ್ಳೆಯದಲ್ಲ.

ನಿಜವಾದ ಆಧ್ಯಾತ್ಮಿಕ ಜೀವಿಗಳು ಎಲ್ಲವನ್ನೂ ಆಧ್ಯಾತ್ಮಿಕ ದೃಷ್ಟಿಯಿಂದ ನೋಡುವರು, ಭೌತಿಕದೃಷ್ಟಿಯಿಂದ ನೋಡುವುದಿಲ್ಲ. ಪ್ರಕೃತಿಯನ್ನು ಚಲಿಸು ವಂತೆ ಮಾಡುವುದೇ ಅಧ್ಯಾತ್ಮ. ಅದೇ ಪ್ರಕೃತಿಯಲ್ಲಿ ಇರುವ ಸತ್ಯ. ಕ್ರಿಯೆಯು ಪ್ರಕೃತಿಯಲ್ಲಿರುವುದು; ಆತ್ಮನಲ್ಲಿ ಅಲ್ಲ. ಆತ್ಮ ಯಾವಾಗಲೂ ಒಂದೇ; ನಿರ್ವಿಕಾರಿ, ಸನಾತನ. ಆತ್ಮ ಮತ್ತು ದ್ರವ್ಯ ಎರಡೂ ಒಂದೇ. ಆದರೆ ಆತ್ಮವು ದ್ರವ್ಯವಾಗ ಲಾರದು; ದ್ರವ್ಯವು ಆತ್ಮವಾಗಲಾರದು.

ಆತ್ಮವು ಎಂದಿಗೂ ಕೆಲಸ ಮಾಡುವುದಿಲ್ಲ. ಅದು ಏತಕ್ಕೆ ಮಾಡಬೇಕು? ಅದು ಸುಮ್ಮನೆ ಇರುವುದು. ಅಷ್ಟೇ ಸಾಕು. ಅದು ಸುಮ್ಮನೆಯೇ ಇರುವುದು; ನಿರಪೇಕ್ಷವಾಗಿರುವುದು. ಅದು ಯಾವ ಕೆಲಸವನ್ನೂ ಮಾಡಬೇಕಾಗಿಲ್ಲ.

ನೀವು ನಿಯಮಕ್ಕೆ ಬದ್ಧರಾಗಿಲ್ಲ. ಬದ್ಧವಾಗಿರುವುದು ನಿಮ್ಮ ಪ್ರಕೃತಿ. ಮನಸ್ಸು ಪ್ರಕೃತಿ. ಮನಸ್ಸು ಪ್ರಕೃತಿಯಲ್ಲಿರುವುದು, ಅದು ನಿಯಮಕ್ಕೆ ಬದ್ಧವಾಗಿರುವುದು. ಪ್ರಕೃತಿಯೆಲ್ಲ ನಿಯಮಕ್ಕೆ ಅಧೀನ. ತನ್ನ ಕರ್ಮದಿಂದಲೇ ಅದು ಬದ್ಧವಾಗಿರುವುದು. ಈ ನಿಯಮವನ್ನು ಎಂದಿಗೂ ತೊಡೆದು ಹಾಕುವುದಕ್ಕೆ ಆಗುವುದಿಲ್ಲ. ನೀವು ಪ್ರಕೃತಿಯ ನಿಯಮವನ್ನು ಮೀರುವ ಹಾಗಿದ್ದರೆ ಕ್ಷಣದಲ್ಲಿ ಪ್ರಕೃತಿಯೆಲ್ಲ ನಾಶವಾಗುವುದು. ಅನಂತರ ಪ್ರಕೃತಿಯೇ ಇರುವುದಿಲ್ಲ. ಯಾರು ಮುಕ್ತಿಯನ್ನು ಪಡೆಯುವರೋ ಅವರು ಪ್ರಕೃತಿಯ ನಿಯಮಕ್ಕೆ ಅತೀತರಾಗಿ ಹೋಗುವರು. ಪ್ರಕೃತಿ ಇನ್ನು ಅವರ ಪಾಲಿಗೆ ಇರುವುದಿಲ್ಲ. ಅದಕ್ಕೆ ಅವರ ಮೇಲೆ ಯಾವ ಅಧಿಕಾರವೂ ಇರುವುದಿಲ್ಲ. ಪ್ರತಿಯೊಬ್ಬರೂ ಕೊನೆಗೆ ಪ್ರಕೃತಿಯ ನಿಯಮದಿಂದ ಪಾರಾಗುವರು. ಅನಂತರ ಅವರು ಪ್ರಕೃತಿಯ ಯಾವ ತೊಂದರೆಗೂ ಒಳಗಾಗುವುದಿಲ್ಲ.

ಸರಕಾರ ಸಮಾಜ ಇವುಗಳೆಲ್ಲ ಒಂದು ದೃಷ್ಟಿಯಲ್ಲಿ ಲೋಪದೋಷಗಳಿಂದಕೂಡಿವೆ. ಸಮಾಜಗಳೆಲ್ಲ ಲೋಪದೋಷಗಳಿಂದ ಕೂಡಿದ ಸಾಮಾನ್ಯೀಕರಣದ ಮೇಲೆ ನಿಂತಿವೆ. ನೀವು ಒಂದು ಸಂಘವನ್ನು ಸ್ಥಾಪಿಸಿ ಅದಕ್ಕೆ ಸೇರಿದೊಡನೆಯೇ ಅದಕ್ಕೆ ಸೇರದವರನ್ನೆಲ್ಲಾ ದ್ವೇಷಿಸುವಿರಿ. ನೀವು ಒಂದು ಸಂಸ್ಥೆಗೆ ಸೇರಿದರೆ ಒಂದು ಮಿತಿಯನ್ನು ಕಲ್ಪಿಸಿಕೊಳ್ಳುವಿರಿ. ನಿಮ್ಮ ಸ್ವಾತಂತ್ರ್ಯಕ್ಕೆ ಧಕ್ಕೆ ಬರುವುದು. ಅತ್ಯಂತ ಶ್ರೇಷ್ಠವಾದ ಒಳಿತು ಎಂದರೆ ಪರಮ ಸ್ವಾತಂತ್ರ್ಯ. ಪ್ರತಿಯೊಬ್ಬರೂ ಈ ಸ್ವಾತಂತ್ರ್ಯವನ್ನು ಪಡೆಯುವಂತೆ ಇರಬೇಕು. ಇದೇ ನಮ್ಮ ಗುರಿ. ಹೆಚ್ಚು ಒಳ್ಳೆಯ ತನ ಇರಬೇಕು; ಕೃತಕ ಕಾನೂನುಗಳು ಕಡಿಮೆ ಇರಬೇಕು. ಅಂಥ ಕಾನೂನುಗಳು ಕಾನೂನುಗಳೇ ಅಲ್ಲ. ಅದೊಂದು ನಿಯಮವಾಗಿದ್ದರೆ ನಾವು ಅದನ್ನು ಮೀರು ವುದಕ್ಕೆ ಆಗುತ್ತಿರಲಿಲ್ಲ. ನಿಯಮಗಳು ಎಂದು ತೋರುವಂಥವನ್ನು ನಾವು ಮುರಿಯು ತ್ತಿದ್ದೇವೆ ಎಂಬುದೇ ಅವು ನಿಯಮಗಳೇ ಅಲ್ಲ ಎಂಬುದನ್ನು ಸ್ಪಷ್ಟವಾಗಿ ತೋರಿಸು ತ್ತದೆ. ಯಾವುದು ನಿಯಮವೋ ಅದನ್ನು ಮುರಿಯುವುದಕ್ಕೆ ಆಗುವುದೇ ಇಲ್ಲ.

ನೀವು ಯಾವಾಗಲಾದರೂ ಒಂದು ಆಲೋಚನೆಯನ್ನು ಏಳದಂತೆ ಅದುಮಿ ದರೆ ಅದು ಒಂದು ಸ್ಪ್ರಿಂಗನ್ನು ಅದುಮಿದಂತೆ ಒಳಗೆ ಹೋಗಿ ಕುಳಿತುಕೊಳ್ಳುವುದು. ಅವಕಾಶ ದೊರೆತೊಡನೆ ತತ್​ಕ್ಷಣ ವೇಗವಾಗಿ ಹೊರಚಿಮ್ಮುವುದು. ಆಗ ಕ್ಷಣವೊಂದರಲ್ಲಿ ಬಹಳ ಕಾಲದಲ್ಲಿ ಆಗುವುದನ್ನು ಮಾಡಿಬಿಡುವುದು.

ತೊಲದಷ್ಟು ಸುಖವು ಅನಂತರ ಸೇರಿನಷ್ಟು ದುಃಖವನ್ನು ತರುವುದು. ಒಂದೇ ಶಕ್ತಿ ಮೊದಲು ಸುಖದಂತೆ ತೋರುವುದು; ಅನಂತರ ದುಃಖದಂತೆ ತೋರುವುದು. ಒಂದು ಸಂವೇದನೆ ನಿಂತೊಡನೆ ಬೇರೆ ಸಂವೇದನೆ ಮೊದಲಾಗುವುದು. ಕೆಲವು ವೇಳೆ ಬಹಳ ಮುಂದುವರಿದ ವ್ಯಕ್ತಿಗಳಲ್ಲಿ ಒಂದೊಂದು ಸಲ ಎರಡು ಅಥವಾ ನೂರಾರು ಆಲೋಚನೆಗಳು ಏಕಕಾಲದಲ್ಲಿ ತಮ್ಮ ಪ್ರಭಾವವನ್ನು ಬೀರುತ್ತಿರಬಲ್ಲವು.

ಮನಸ್ಸು ತನ್ನ ಸ್ವಭಾವದಿಂದಲೇ ಚಲಿಸುವುದು. ಮಾನಸಿಕ ಕ್ರಿಯೆ ಎಂದರೆ ಸೃಷ್ಟಿ. ಮೊದಲ ಆಲೋಚನೆ ಶಬ್ದ. ಶಬ್ದದ ನಂತರ ಆಕಾರ. ಮನಸ್ಸು ಆತ್ಮನನ್ನು ಪ್ರತಿಬಿಂಬಿಸಬೇಕಾದರೆ ಮೊದಲು ಈ ಭೌತಿಕ ಮತ್ತು ಮಾನಸಿಕ ಸೃಷ್ಟಿಕ್ರಿಯೆಯೆಲ್ಲ ನಿಲ್ಲಬೇಕು.

