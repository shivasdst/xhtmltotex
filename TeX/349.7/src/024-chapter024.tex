
\chapter[ಭಾರತೀಯ ಧಾರ್ಮಿಕ ಭಾವನೆ ]{ಭಾರತೀಯ ಧಾರ್ಮಿಕ ಭಾವನೆ \protect\footnote{\engfoot{C.W. Vol. IV, P 188}}}

\centerline{\textbf{(ಬ್ರೂಕ್ಲಿನಿನ್ನನ ಬ್ರೂಕ್ಲಿನ್​ ಎಥಿಕಲ್​ ಸೊಸೈಟಿಯ ಆಶ್ರಯದಲ್ಲಿ ನೀಡಿದ ಉಪನ್ಯಾಸ)}}

ಇಂಡಿಯಾ ದೇಶವು ಸಂಯುಕ್ತ ಸಂಸ್ಥಾನಕ್ಕೆ ಅರ್ಧದಷ್ಟು ವಿಸ್ತಾರವಾಗಿದ್ದರೂ ಅಲ್ಲಿ\break ಇಪ್ಪತ್ತೊಂಭತ್ತು ಕೋಟಿ ಜನರು ಇರುವರು. ಅವರಲ್ಲಿ ಮುಖ್ಯವಾಗಿ ಮೂರು\break ಧರ್ಮಗಳು ಪ್ರಚಾರದಲ್ಲಿವೆ. ಅವು ಇಸ್ಲಾಂ, ಬೌದ್ಧ, ಜೈನರೂ ಸೇರಿ ಮತ್ತು ಹಿಂದೂಧರ್ಮಗಳು. ಮೊದಲನೆಯದಕ್ಕೆ ಆರು ಕೋಟಿ ಜನರು ಸೇರಿರುವರು. ಎರಡನೆಯದಕ್ಕೆ ಸುಮಾರು ತೊಂಭತ್ತು ಲಕ್ಷ ಜನರು, ಕೊನೆಯದು ಇಪ್ಪತ್ತು ಕೋಟಿ ಅರವತ್ತು ಲಕ್ಷಗಳು. ಹಿಂದೂಧರ್ಮದ ಮುಖ್ಯ ವಿಷಯಗಳು ಧ್ಯಾನ, ತತ್ತ್ವ ಮತ್ತು ವೇದಗಳಲ್ಲಿ ಬರುವ ನೀತಿಶಾಸ್ತ್ರಗಳು, ಇವುಗಳ ಮೇಲೆ ನಿಂತಿವೆ. ವೇದಗಳು ಪ್ರಪಂಚವನ್ನು ಕಾಲದೇಶಗಳ ಮೂಲಕ ಅನಂತವೆಂದು ಸಾರುತ್ತವೆ. ಅದಕ್ಕೆ ಒಂದು ಆದಿ ಇಲ್ಲ, ಒಂದು ಅಂತ್ಯವಿಲ್ಲ. ಭೌತಿಕ ಪ್ರಪಂಚದಲ್ಲಿ ಆತ್ಮ ಅನೇಕ ರೀತಿ ವ್ಯಕ್ತವಾಗಿದೆ. ಅನಂತವು ಸಾಂತದಲ್ಲಿ ಹಲವು ರೀತಿ ವ್ಯಕ್ತವಾಗಿದೆ. ಆದರೆ ಅನಂತಾತ್ಮನಾದರೊ ಸ್ವಯಂಭೂ, ನಿರ್ವಿಕಾರಿ, ಅನಾದ್ಯನಂತನು. ಅನಂತವೆಂಬ ಗಡಿಯಾರದ ಮುಖಫಲಕದ ಮೇಲೆ ಕಾಲದ ಬದಲಾವಣೆಯು ಯಾವ ಒಂದು ಪ್ರಮುಖವಾದ ಗುರುತನ್ನೂ ಬಿಡುವುದಿಲ್ಲ. ಮಾನವನು ತನ್ನ ಸ್ವಾಭಾವಿಕವಾದ ಸ್ಥಿತಿಯಿಂದ ಅತೀಂದ್ರಿಯವಾದ ಆತ್ಮನನ್ನು ಗ್ರಹಿಸಲಾರ. ಅಲ್ಲಿ ಯಾವ ಭೂತವೂ ಇಲ್ಲ, ಭವಿಷ್ಯವೂ ಇಲ್ಲ. ಮಾನವನ ಆತ್ಮವು ಅಮೃತವಾದುದು ಎಂದು ವೇದಗಳು ಸಾರುವುವು. ದೇಹವು ಷಡ್ವಿಕಾರಾತ್ಮಕವಾದುದು. ಯಾವುದು ಬೆಳೆಯುವುದೋ ಅದು ನಾಶವಾಗಲೇಬೇಕು. ಆದರೆ ಒಳಗೆ ಇರುವ ಆತ್ಮನಾದರೊ ಅನಂತವಾದ ಮತ್ತು ನಿತ್ಯವಾದ ಜೀವನಕ್ಕೆ ಸಂಬಂಧಿಸಿರುವುದು. ಅದಕ್ಕೆ ಒಂದು ಆದಿಯೇ ಇರಲಿಲ್ಲ. ಅಂತ್ಯವೂ ಇರುವುದಿಲ್ಲ. ಹಿಂದುಧರ್ಮಕ್ಕೂ ಕ್ರೈಸ್ತಧರ್ಮಕ್ಕೂ ಇರುವ ವ್ಯತ್ಯಾಸವೇ ಇದು. ಕ್ರೈಸ್ತರು ಪ್ರತಿಯೊಂದು ಜೀವವೂ ಹೊಸದಾಗಿ ಜನನ ಕಾಲದಲ್ಲಿ ಹುಟ್ಟಿತು ಎಂದು ನಂಬುವರು. ಆದರೆ ಹಿಂದೂಗಳಾದರೊ, ಜೀವವು ಪರಮೇಶ್ವರನಿಂದ ಬಂದುದು ಅವನಿಗೆ ಹೇಗೆ ಆದಿ ಅಂತ್ಯಗಳಿಲ್ಲವೊ ಹಾಗೆಯೇ ಜೀವಿಗೂ ಆದಿ ಅಂತ್ಯಗಳಿಲ್ಲ, ಎಂದು ನಂಬುವರು. ಈ ಜೀವ, ಹಿಂದಿನ ಜನ್ಮಗಳಲ್ಲಿ ಹಲವು ದೇಹಗಳನ್ನು ಧರಿಸಿತ್ತು ಮತ್ತು ಮುಂದೆಯೂ ಧರಿಸುವುದು. ಕೊನೆಗೆ ಮುಕ್ತಿಯನ್ನು ಪಡೆಯುವ ತನಕ ಕರ್ಮಾನುಸಾರ ಬೇರೆ ಬೇರೆ ದೇಹಗಳನ್ನು ಧರಿಸುತ್ತಾ ಹೋಗುವುದು. ಪೂರ್ಣತೆಯನ್ನು ಪಡೆದ ಮೇಲೆ ಅದು ಇನ್ನು ಯಾವ ದೇಹವನ್ನೂ ಧರಿಸುವುದಿಲ್ಲ.

ಇದು ನಿಜವಾಗಿದ್ದರೆ ನಮಗೇಕೆ ಹಿಂದಿನ ಜನ್ಮಗಳು ನೆನಪಿನಲ್ಲಿ ಇಲ್ಲ ಎಂದು ಕೇಳುವರು. ಅದಕ್ಕೆ ಇದೇ ನಮ್ಮ ವಿವರಣೆ: ಮನಸ್ಸೆಂಬ ಸಾಗರದ ಮೇಲಿನ ಪದರಕ್ಕೆ ಮಾತ್ರ ಪ್ರಜ್ಞೆ ಎಂದು ಹೆಸರು. ಅದರ ಒಳಗೆ ಎಲ್ಲಾ ನೆನಪುಗಳು, ದುಃಖಕರವಾದುವು ಮತ್ತು ಸುಖಕರವಾದುವು ಎಲ್ಲ ಅಲ್ಲಿ ಸಂಗ್ರಹವಾಗಿವೆ. ಯಾವುದಾದರೂ ಒಂದು ಸ್ಥಿರವಾದ ವಸ್ತುವನ್ನು ಕಂಡುಹಿಡಿಯಬೇಕೆಂಬುದು ಮಾನವನ ಇಚ್ಛೆ. ನಮ್ಮ ಮನಸ್ಸು, ದೇಹ ಮತ್ತು ಪ್ರಕೃತಿಯ ಎಲ್ಲಾ ವಸ್ತುಗಳೂ ಯಾವಾಗಲೂ ಬದಲಾಯಿಸುತ್ತ ಇರುತ್ತವೆ. ಆದರೆ ನಮ್ಮ ಚೇತನದ ಶ್ರೇಷ್ಠ ಗುರಿಯಾದರೋ, ಬಲಾಯಿಸದೆ ಇರುವುದನ್ನು ಕಂಡುಹಿಡಿಯುವುದಾಗಿದೆ, ನಿತ್ಯ ಪರಿಪೂರ್ಣವಾಗಿರುವ ವಸ್ತುವನ್ನು ಕಂಡುಹಿಡಿಯುವುದಾಗಿದೆ. ಮಾನವನು ಅನಂತತೆಗಾಗಿ ಪರಿತಪಿಸುತ್ತಿರುವನು. ನಮ್ಮ ನೈತಿಕ ಮತ್ತು ಭೌತಿಕ ಬೆಳವಣಿಗೆಯು ಸೂಕ್ಷ್ಮವಾದಂತೆಲ್ಲ ನಿರ್ವಿಕಾರಿಯಾದ ಪರಮಾತ್ಮ ವಸ್ತುವನ್ನು ಪಡೆಯುವ ಇಚ್ಛೆಯು ಉತ್ಕಟವಾಗುವುದು.

ಆಧುನಿಕ ಬೌದ್ಧರು ನಾವು ಪಂಚೇಂದ್ರಿಯಗಳ ಮೂಲಕ ಗ್ರಹಿಸಲಾಗದ ಯಾವ ವಸ್ತುವೂ ಅಸ್ತಿತ್ವದಲ್ಲಿಲ್ಲ, ಮಾನವನು ಒಂದು ಸ್ವತಂತ್ರ ವ್ಯಕ್ತಿ, ಎಂದು ಭಾವಿಸುವುದೊಂದು ಭ್ರಮೆ ಎಂದು ಸಾರುವರು. ಭಾವಸತ್ತಾವಾದಿಗಳಾದರೊ \enginline{(idealists)} ಅದಕ್ಕೆ ಪ್ರತಿಯಾಗಿ ಪ್ರತಿಯೊಂದು ವ್ಯಕ್ತಿಯೂ ಸ್ವತಂತ್ರ ಮತ್ತು ಬಾಹ್ಯ ಜಗತ್ತು ಇವನ ಮನೋಗ್ರಹಿಕೆಯ ಆಚೆ ಇಲ್ಲ ಎಂದು ಸಾರುವರು. ಆದರೆ ಆ ಸಮಸ್ಯೆಯ ನಿಜವಾದ ಪರಿಹಾರೋಪಾಯವೇ ಪ್ರಕೃತಿಯಲ್ಲಿ ಸ್ವಾತಂತ್ರ್ಯ ಮತ್ತು ಪಾರತಂತ್ರ್ಯ, ವಸ್ತುಸತ್ತಾವಾದ ಭಾವಸತ್ತಾವಾದ ಇವು ಮಿಶ್ರವಾಗಿದೆ ಎಂದು ಭಾವಿಸುವುದು. ನಮ್ಮ ದೇಹ ಮತ್ತು ಮನಸ್ಸು ಬಾಹ್ಯ ಪ್ರಪಂಚವನ್ನು ಅವಲಂಬಿಸಿವೆ. ಈ ಅವಲಂಬನವು ದೇಹಮನಸ್ಸುಗಳಿಗೆ ಬಾಹ್ಯಜಗತ್ತಿನ ವಿಷಯದಲ್ಲಿ ಇರುವ ಸಂಬಂಧಕ್ಕೆ ಅನುಗುಣವಾಗಿ ವ್ಯತ್ಯಾಸವಾಗುತ್ತದೆ. ಆಸರೆಯು ವ್ಯಕ್ತಿಯಿಂದ ವ್ಯಕ್ತಿಗೆ ಬದಲಾಯಿಸುವುದು. ಆದರೆ ಒಳಗಿರುವ ಆತ್ಮನಾದರೋ ದೇವರಂತೆಯೇ ಸ್ವತಂತ್ರನು. ಅವನು ತನ್ನ ದೇಹ ಮತ್ತು ಮನಸ್ಸನ್ನು ತನ್ನ ಬೆಳವಣಿಗೆಗೆ ಅನುಗುಣವಾಗಿ ನಿರ್ವಹಿಸಬಲ್ಲ.

ಮೃತ್ಯು ಎಂದರೆ ಇರುವ ಸ್ಥಿತಿಯ ಬದಲಾವಣೆ. ನಾವು ಹಿಂದಿನ ವಿಶ್ವದಲ್ಲಿಯೇ ಇರುವೆವು. ಹಿಂದಿನಂತೆಯೇ ಪ್ರಕೃತಿಯ ನಿಯಮಾವಳಿಗೆ ಬದ್ಧರಾಗಿರುವೆವು. ಯಾರು ಇದಕ್ಕೆ ಅತೀತರಾಗಿ ಹೋಗಿರುವರೊ, ಸೌಂದರ್ಯೋಪಾಸನೆ, ಆಧ್ಯಾತ್ಮಿಕ ಅನುಭವ ಮುಂತಾದುವುಗಳಲ್ಲಿ ಬಹಳ ಉನ್ನತ ಮಟ್ಟವನ್ನು ಪಡೆದಿರುವರೊ, ಅವರು ವಿಶ್ವ\break ಸೈನ್ಯದ ಮುಂದಳದವರು. ಉಳಿದವರು ಅವರನ್ನು ಅನುಸರಿಸುವರು. ಶ್ರೇಷ್ಠತಮ\break ಆತ್ಮಕ್ಕೂ ತುಂಬ ಕನಿಷ್ಟನಾದವನಲ್ಲಿರುವ ಆತ್ಮಕ್ಕೂ ಸಂಬಂಧವಿದೆ. ಎಲ್ಲರಲ್ಲಿಯೂ\break ಅನಂತ ಪರಿಪೂರ್ಣತೆಯ ಸ್ಥಿತಿಯು ಭ್ರೂಣಸ್ಥಿತಿಯಲ್ಲಿದೆ. ಪ್ರಪಂಚವು ಕೊನೆಗೆ\break ಶುಭದಲ್ಲಿ ಪರ್ಯವಸಾನವಾಗುವುದು ಎಂಬ ಮನೋಭಾವವನ್ನೇ ರೂಢಿಸಬೇಕು.\break ಪ್ರತಿಯೊಂದರಲ್ಲಿರುವ ಒಳ್ಳೆಯದನ್ನು ನೋಡುವ ಅಭ್ಯಾಸವನ್ನು ರೂಢಿಸಬೇಕಾಗಿದೆ. ನಾವು ನಮ್ಮ ದೇಹ ಮತ್ತು ಮನಸ್ಸಿನ ಲೋಪದೋಷಗಳನ್ನು ಕುರಿತು ಸುಮ್ಮನೆ ವ್ಯಾಕುಲಪಡುತ್ತಿದ್ದರೆ ಏನೂ ಪ್ರಯೋಜನವಿಲ್ಲ. ಭಾರತೀಯ ಧಾರ್ಮಿಕ ಭಾವನೆ
 ನಮ್ಮನ್ನು\break ಸೋಲಿಸಲು ಬರುವ ಸನ್ನಿವೇಶಗಳೊಡನೆ ಧೈರ್ಯದಿಂದ ಹೋರಾಡಿ ಅವುಗಳನ್ನು ಮೆಟ್ಟಿ ನಿಲ್ಲುವುದೇ ನಮ್ಮ ಆತ್ಮನನ್ನು ಮುಂದು ಮುಂದಕ್ಕೆ ಒಯ್ಯುವುದು. ಆಧ್ಯಾತ್ಮಿಕ ಜೀವನ ಹೇಗೆ ವಿಕಾಸವಾಗುತ್ತದೆ ಎಂಬ ನಿಯಮಗಳನ್ನು ಅರಿಯುವುದೇ ಜೀವನದಗುರಿ. ಕ್ರೈಸ್ತರು ಹಿಂದೂಗಳಿಂದ ಕಲಿಯಬಹುದು, ಹಿಂದೂಗಳು ಕ್ರೈಸ್ತರಿಂದ ಕಲಿಯಬಹುದು. ಇಬ್ಬರೂ ಜಗತ್ತಿನ ಆಧ್ಯಾತ್ಮಿಕ ಭಂಡಾರಕ್ಕೆ ತಮ್ಮ ತಮ್ಮ ಮೌಲ್ಯಗಳ ಕಾಣಿಕೆಗಳನ್ನು ಅರ್ಪಿಸಿರುವರು.

ನಿಮ್ಮ ಮಕ್ಕಳಿಗೆ ಧರ್ಮವೆಂದರೆ, ಸ್ಪಷ್ಟವಾದ ಭಾವಗಳನ್ನು ರೂಢಿಸುವುದು,\break ನಿಷೇಧಾತ್ಮಕವಾದ ಭಾವಗಳನ್ನು ರೂಢಿಸುವುದಲ್ಲ ಎಂಬುದನ್ನು ಒತ್ತಿಹೇಳಿ. ಸುಮ್ಮನೆ\break ಪಾಪವನ್ನು ಮಾಡದೆ ಇದ್ದರೆ ಮಾತ್ರ ಸಾಲದು. ಏನನ್ನಾದರೂ ಪುಣ್ಯ ಕೆಲಸಗಳನ್ನು ಮಾಡಬೇಕು. ಪುಸ್ತಕಗಳನ್ನು ಓದಿದರೇ ಆಗಲಿ, ಉಪನ್ಯಾಸಗಳನ್ನು ಕೇಳಿದರೇ ಆಗಲಿ, ನಿಜವಾದ ಧರ್ಮ ಬರುವುದಿಲ್ಲ. ಪರಿಶುದ್ಧವಾದ ಸಾಹಸಕರವಾದ ಪುಣ್ಯಕೆಲಸಗಳನ್ನು ಮಾಡಿದುದರ ಪರಿಣಾಮವಾಗಿ ನಮ್ಮ ಆತ್ಮ ಜಾಗ್ರತವಾಗುವುದು. ಜಗತ್ತಿನಲ್ಲಿ ಹುಟ್ಟುವ ಪ್ರತಿಯೊಂದು ಹಸುಳೆಯೂ ತನ್ನ ಹಿಂದಿನ ಜನ್ಮಗಳಿಂದ ಸಂಗ್ರಹಿಸಿಕೊಂಡಿರುವ ಸಂಸ್ಕಾರಗಳನ್ನು ತರುವುದು. ಇವು ನಮ್ಮ ದೇಹ ಮತ್ತು ಮನಸ್ಸುಗಳ ಮೇಲೆ ತಮ್ಮ ಪ್ರಭಾವವನ್ನು ಬಿಟ್ಟಿರುವುದನ್ನು ನಾವು ನೋಡುತ್ತೇವೆ. ನಮ್ಮಲ್ಲಿ ಬಲಯುತವಾಗಿರುವ ಸ್ವಾತಂತ್ರ್ಯದ ಭಾವನೆಯು ದೇಹ ಮನಸ್ಸುಗಳಿಂದ ಬೇರೆಯಾದ ಏನೊ ಬಂದು ನಮ್ಮಲ್ಲಿ ಇದೆ ಎಂಬುದನ್ನು ತೋರಿಸುತ್ತದೆ. ನಮ್ಮೊಳಗೆ ಆಳುತ್ತಿರುವ ಆತ್ಮವು ಸ್ವತಂತ್ರವಾದುದು, ಅದೇ ಮುಕ್ತರಾಗಬೇಕೆಂಬ ಅಭಿಲಾಷೆಯನ್ನು ನಮ್ಮಲ್ಲಿ ಹುಟ್ಟಿಸುವುದು. ನಾವೇ ಮುಕ್ತರಾಗದೆ ಇದ್ದರೆ ಈ ಪ್ರಪಂಚವನ್ನು ಹೇಗೆ ಉತ್ತಮ ಸ್ಥಿತಿಗೆ ತರಲು ಸಾಧ್ಯ? ಆತ್ಮನ ಕ್ರಿಯೆಯ ಪರಿಣಾಮವಾಗಿ ಮಾನವಕೋಟಿಯು ಮುಂದುವರಿಯುವುದು ಎಂದು ನಾವು ನಂಬುತ್ತೇವೆ. ಈ ಜಗತ್ತು ಈಗ ಏನಾಗಿರುವುದೊ, ನಾವು ಈಗ ಏನಾಗಿರುವೆವೊ ಇವೆರಡೂ ಆತ್ಮನ ಸ್ವಾತಂತ್ರ್ಯದ ಫಲ.

ನಾವೆಲ್ಲ ಒಂದು ದೇವರನ್ನು ನಂಬುತ್ತೇವೆ. ಅವನು ನಮಗೆಲ್ಲರಿಗೂ ತಂದೆ;\break ಸರ್ವವ್ಯಾಪಿ, ಸರ್ವಶಕ್ತ. ಅವನು ತನ್ನ ಮಕ್ಕಳನ್ನು ಅನಂತ ಪ್ರೇಮದಿಂದ ರಕ್ಷಿಸಿ\break ಅವರಿಗೆ ಮಾರ್ಗವನ್ನು ತೋರುವನು. ನಾವು ಕ್ರೈಸ್ತರಂತೆ ಸಗುಣ ಬ್ರಹ್ಮನಲ್ಲಿ\break ನಂಬುತ್ತೇವೆ. ಆದರೆ ಅವರಿಗಿಂತ ಒಂದು ಹೆಜ್ಜೆ ಮುಂದೆ ಹೋಗುತ್ತೇವೆ. ನಾವೇ\break ಅವನು ಎಂದು ನಂಬುತ್ತೇವೆ. ಅವನು ನಮ್ಮಲ್ಲಿ ಆವಿರ್ಭೂತನಾಗಿರುವನು. ದೇವರು ನಮ್ಮಲ್ಲಿ ಇರುವನು. ನಾವು ದೇವರಲ್ಲಿ ಇರುವೆವು. ಎಲ್ಲಾ ಧರ್ಮಗಳಲ್ಲಿಯೂ ಸತ್ಯಾಂಶಗಳಿವೆ ಎಂದು ನಂಬುತ್ತೇವೆ. ಹಿಂದೂಗಳು ಎಲ್ಲಾ ಧರ್ಮಗಳನ್ನೂ ಗೌರವಿಸುತ್ತಾರೆ.\break ಏಕೆಂದರೆ ಜಗತ್ತಿನಲ್ಲಿ ಸತ್ಯವನ್ನು ಕಂಡುಹಿಡಿಯಬೇಕಾದುದು ಕೂಡುವುದರಿಂದಲೇ\break ಹೊರತು ಕಳೆಯುವುದರಿಂದ ಅಲ್ಲ. ಭಿನ್ನ ಭಿನ್ನ ಧರ್ಮಗಳ ಪುಷ್ಪಗಳನ್ನೆಲ್ಲಾ ಸೇರಿಸಿ\break ದೇವರಿಗೆ ಒಂದು ತುರಾಯಿಯನ್ನು ಮಾಡಿ ಅರ್ಪಿಸುತ್ತೇವೆ. ನಾವು ದೇವರನ್ನು\break ಪ್ರೀತಿಗಾಗಿ ಪ್ರೀತಿಸಬೇಕು. ಮತ್ತಾವುದೊ ವಸ್ತುವನ್ನು ಪಡೆಯಬೇಕೆಂಬ ಆಸೆಯಿಂದಲ್ಲ.\break ನಾವು ಕರ್ತವ್ಯಕ್ಕಾಗಿ ಕರ್ತವ್ಯವನ್ನು ಮಾಡಬೇಕು, ಫಲಾಕಾಂಕ್ಷೆಗಲ್ಲ. ಸೌಂದರ್ಯಕ್ಕಾಗಿ ಆರಾಧಿಸಬೇಕು, ಫಲಾಪೇಕ್ಷೆಯಿಂದಲ್ಲ. ಹೀಗೆ ನಮ್ಮ ಹೃದಯದ ಪವಿತ್ರತೆಯಲ್ಲಿ\break ನಾವು ದೇವರನ್ನು ನೋಡುತ್ತೇವೆ. ಬಲಿ, ಮಂತ್ರೋಚ್ಚಾರಣೆ, ಮುದ್ರೆ, ಆಚಾರಗಳು ಇವು ಧರ್ಮವಲ್ಲ. ಸುಂದರವಾದ ಸಾಹಸಪೂರ್ಣವಾದ ಕೆಲಸಕ್ಕೆ ನಮ್ಮನ್ನು ಪ್ರಚೋದಿಸಿ ನಮ್ಮ ಮನಸ್ಸನ್ನು ಪವಿತ್ರವಾದ ಪರಿಪೂರ್ಣತೆಯ ದರ್ಶನಕ್ಕೆ ಒಯ್ದರೆ ಮಾತ್ರ ಅವು ಒಳ್ಳೆಯವು.

ನಾವು ಪ್ರಾರ್ಥನೆ ಮಾಡುವಾಗ, ದೇವರೇ ಎಲ್ಲರಿಗೂ ತಂದೆ ಎಂದು ಒಪ್ಪಿಕೊಂಡು, ನಮ್ಮ ನಿತ್ಯ ಜೀವನದಲ್ಲಿ ಮಾನವರನ್ನು ನಮ್ಮ ಸಹೋದರರಂತೆ ನೋಡದೆ ಇದ್ದರೆ ಏನು ಪ್ರಯೋಜನ? ಶಾಸ್ತ್ರವಿರುವುದು ನಮಗೆ ಸನ್ಮಾರ್ಗಕ್ಕೆ ದಾರಿ ತೋರುವುದಕ್ಕಾಗಿ.\break ನಾವು ಆ ಮಾರ್ಗದಲ್ಲಿ ಹಿಂಜರಿಯದೆ ಮುಂದುವರಿದರೆ ಮಾತ್ರ ಪ್ರಯೋಜನ.\break ಪ್ರತಿಯೊಬ್ಬರ ವ್ಯಕ್ತಿತ್ವವನ್ನು ಗಾಜಿನ ಗೋಳಕ್ಕೆ ಹೋಲಿಸಬಹುದು. ಅವುಗಳ ಮಧ್ಯದಲ್ಲಿ ಭಗವಂತನಿಂದ ಬಂದ ಪರಿಶುದ್ಧವಾದ ಶ್ವೇತ ಜ್ಯೋತಿಯು ಹೊಳೆಯುತ್ತಿದೆ. ಆದರೆ ಸುತ್ತಲೂ ಇರುವ ಗಾಜು, ಗಾತ್ರದಲ್ಲಿ ಹೆಚ್ಚು ಕಡಿಮೆ ಇರುವುದರಿಂದ, ಅದರ ಬಣ್ಣದಲ್ಲಿ ವ್ಯತ್ಯಾಸವಿರುವುದರಿಂದ, ಕಾಂತಿಯ ಆವಿರ್ಭಾವದಲ್ಲಿ ವ್ಯತ್ಯಾಸವಿದೆ. ಕೇಂದ್ರದಲ್ಲಿರುವ ಮೂಲ ಜ್ಯೋತಿ ಸೌಂದರ್ಯದಲ್ಲಿ ಮತ್ತು ಪರಿಪೂರ್ಣತೆಯಲ್ಲಿ ಒಂದೇ ಆಗಿದೆ. ಅವುಗಳು ಹೊರಗೆಡಹುತ್ತಿರುವ ಕಾಂತಿಯ ವ್ಯತ್ಯಾಸಕ್ಕೆ ಕಾರಣ ಮಧ್ಯವರ್ತಿಯ ಅಪೂರ್ಣತೆ. ನಾವು ಆಧ್ಯಾತ್ಮಿಕ ಜೀವನದಲ್ಲಿ ಮೇಲುಮೇಲಕ್ಕೆ ಏರುತ್ತ ಹೋದಂತೆ ಮಧ್ಯವರ್ತಿ ಹೆಚ್ಚು ಹೆಚ್ಚು ಪಾರದರ್ಶಕವಾಗುತ್ತ ಬರುವುದು.

