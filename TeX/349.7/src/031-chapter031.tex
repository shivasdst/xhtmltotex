
\chapter[ವೇದಾಂತ ಅದರ್ಶನ ]{ವೇದಾಂತ ಅದರ್ಶನ \protect\footnote{\engfoot{C.W. Vol. V, P 281}}}

ಮನುಷ್ಯ ಹುಟ್ಟುವುದೂ ಇಲ್ಲ, ಸಾಯುವುದೂ ಇಲ್ಲ, ಸ್ವರ್ಗಕ್ಕೆ ಹೋಗು ವುದೂ ಇಲ್ಲ ಎನ್ನುತ್ತದೆ ವೇದಾಂತ. ಆತ್ಮನ ದೃಷ್ಟಿಯಿಂದ ಪುನರ್ಜನ್ಮವೆಂಬುದು ಒಂದು ಭ್ರಾಂತಿ. ವೇದಾಂತಿಗಳು ಹಾಳೆಗಳನ್ನು ತಿರುವಿ ಹಾಕುತ್ತಿರುವ ಪುಸ್ತಕದ ಒಂದು ಉದಾಹರಣೆಯನ್ನು ಕೊಡುವರು. ಪುಸ್ತಕ ಬದಲಾಗುತ್ತಿರುವುದು, ಅದನ್ನು ತಿರುವಿ ಹಾಕುತ್ತಿರುವ ಮನುಷ್ಯನಲ್ಲ ಬದಲಾಗುತ್ತಿರುವುದು. ಪ್ರತಿಯೊಂದು ಆತ್ಮವೂ ಸರ್ವವ್ಯಾಪಿ. ಅದು ಬರುವುದೆಲ್ಲಿಗೆ, ಹೋಗುವುದೆಲ್ಲಿಗೆ? ಜನನ ಮರಣಗಳು ಪ್ರಕೃತಿಯಲ್ಲಿ ಆಗುತ್ತಿರುವ ಬದಲಾವಣೆಗಳು. ಅವು ನಮ್ಮಲ್ಲಿ ಆಗುತ್ತಿವೆ ಎಂದು ನಾವು ಭ್ರಾಂತಿ ಪಡುವೆವು.

ಪುನರ್ಜನ್ಮ ಎಂಬುದು ಪ್ರಕೃತಿಯ ಒಂದು ಪರಿಣಾಮ, ಅದರಲ್ಲಿ ಸುಪ್ತ ವಾಗಿರುವ ಆತ್ಮನ ಅಭಿವ್ಯಕ್ತಿ.

ವೇದಾಂತವು ಪ್ರತಿಯೊಂದು ಜನ್ಮವೂ ಕಳೆದ ಜನ್ಮಗಳ ಮೇಲೆ ನಿಂತಿದೆ ಎನ್ನುತ್ತದೆ. ನಮ್ಮ ನಮ್ಮ ಹಿಂದನ್ನೆಲ್ಲಾ ನೋಡಿದಾಗ ಮುಕ್ತರಾಗುವೆವು. ಮುಕ್ತನಾಗಬೇಕೆಂಬ ಆಸೆ ವ್ಯಕ್ತಿಯನ್ನು ಆಧ್ಯಾತ್ಮಿಕ ಪ್ರವೃತ್ತಿಯುಳ್ಳವನನ್ನಾಗಿ ಮಾಡುತ್ತದೆ. ಕೆಲವು ವರ್ಷಗಳಾದ ಮೇಲೆ ಈ ಸತ್ಯವೆಲ್ಲ ಸ್ಪಷ್ಟವಾಗುವುದು. ಒಬ್ಬನು ಈ ಜನ್ಮವನ್ನು ತ್ಯಜಿಸಿ ಮತ್ತೊಂದು ಜನ್ಮಕ್ಕೆ ಕಾಯುತ್ತಿರುವಾಗಲೂ ಈ ಜಗತ್ತಿನಲ್ಲೇ ಇರುವನು.

ಆತ್ನನನ್ನು ನಾವು ಹೀಗೆ ವಿವರಿಸಬಹುದು: ಆತ್ಮನನ್ನು ಖಡ್ಗ ಛೇದಿಸದು; ಭರ್ಜಿ ಇರಿಯಲಾರದು; ಬೆಂಕಿ ಸುಡಲಾರದು; ನೀರು ಕರಗಿಸಲಾರದು. ಆತ್ಮ ಅವಿನಾಶಿ, ಸರ್ವಗತ. ಆದಕಾರಣ ಅದಕ್ಕಾಗಿ ಅಳಬೇಡ.

ಇದುವರೆಗೂ ನಾವು ತುಂಬಾ ಕಷ್ಟಪಟ್ಟಿದ್ದರೆ ಮುಂದೆ ಮೇಲಾಗುವುದೆಂದು ನಂಬುತ್ತೇವೆ. ಪ್ರತಿಯೊಬ್ಬರಿಗೂ ನಿತ್ಯಸ್ವಾತಂತ್ರ್ಯವಿದೆ ಎಂಬುದೇ ಮೂಲ ಸಿದ್ಧಾಂತ. ಪ್ರತಿಯೊಬ್ಬರೂ ಇಲ್ಲಿಗೆ ಬಂದೇ ತೀರಬೇಕು. ಮುಕ್ತರಾಗಬೇಕೆಂಬ ನಮ್ಮ ಆಸೆಯಿಂದ ಪ್ರಚೋದಿತರಾಗಿಯೇ ನಾವು ಹೋರಾಡಬೇಕಾಗಿದೆ. ಮುಕ್ತ ರಾಗಬೇಕೆಂಬ ಆಸೆಯ ವಿನಾ ಉಳಿದವುಗಳೆಲ್ಲ ಭ್ರಾಂತಿ. ಪ್ರತಿಯೊಂದು ಪುಣ್ಯ ಕರ್ಮವೂ ಆ ಸ್ವಾತಂತ್ರ್ಯದ ಒಂದು ಅಭಿವ್ಯಕ್ತಿ ಎನ್ನುವರು ವೇದಾಂತಿಗಳು.

ಪ್ರಪಂಚದ ಪಾಪಗಳೆಲ್ಲ ಮಾಯವಾಗುವ ಒಂದು ಸಮಯ ಬರುವುದು ಎಂದು ನಾನು ನಂಬುವುದಿಲ್ಲ. ಇದು ಹೇಗೆ ಸಾಧ್ಯ? ಈ ಪ್ರವಾಹ ಸಾಗುತ್ತಲೇ ಇರುವುದು. ನದಿಯಲ್ಲಿ ಒಂದು ಕಡೆ ನೀರು ಸಮುದ್ರಕ್ಕೆ ಹರಿದು ಹೋಗುತ್ತಿದ್ದರೆ ಮತ್ತೊಂದು ಕಡೆಯಿಂದ ಹೊಸದಾಗಿ ನೀರು ನದಿಗೆ ಹರಿಯುತ್ತಿರುತ್ತದೆ.

ನೀನು ಶುದ್ಧನು, ಪೂರ್ಣನು; ಪಾಪಪುಣ್ಯಗಳಾಚೆ ಒಂದು ಸ್ಥಿತಿ ಇದೆ, ಅದೇ ನಿನ್ನ ಸ್ವಭಾವ ಎನ್ನುವುದು ವೇದಾಂತ. ಇದು ಪುಣ್ಯಕ್ಕಿಂತ ಮಿಗಿಲಾದ ಸ್ಥಿತಿ. ಒಳ್ಳೆಯದು ಕೆಟ್ಟದ್ದಕ್ಕಿಂತ ಸ್ವಲ್ಪ ಮೇಲೆ ಅಷ್ಟೆ. ನಮ್ಮಲ್ಲಿ ಪಾಪವೆಂಬುದಿಲ್ಲ, ನಾವು ಅದನ್ನು ಅಜ್ಞಾನ ಎಂದು ಕರೆಯುತ್ತೇವೆ. ಇತರರರೊಡನೆ ನಮಗೆ ಇರುವ ವ್ಯವಹಾರ, ನಮ್ಮ ನೀತಿ ಇವುಗಳೆಲ್ಲ ಈ ಪ್ರಪಂಚಕ್ಕೆ ಅನ್ವಯಿಸುವವುಗಳು. ನಾವು ಸತ್ಯದ ಪೂರ್ಣವಾದ ಒಂದು ವಿವರಣೆಯನ್ನು ಕೊಡುವಾಗ ದೇವರ ಮೇಲೆ ಅಜ್ಞಾನವನ್ನು ಆರೋಪಿಸುವುದಿಲ್ಲ. ಅವನನ್ನು ಸಚ್ಚಿದಾನಂದ ಎಂದು ಕರೆಯುತ್ತೇವೆ. ನಾವು ದೇವರನ್ನು ವಿವರಿಸುವುದಕ್ಕೆ ಉಪಯೋಗಿಸುವ ಭಾಷೆ ಮತ್ತು ಆಲೋಚನೆ ಇವು ದೇವರನ್ನು ಸಾಂತಗೊಳಿಸುವುವು, ದೇವರ ಸ್ವಭಾವ ವನ್ನೇ ಅಲ್ಲಗಳೆಯುವುವು.

ನಾವು ಇದೊಂದನ್ನು ಗಮನಿಸಬೇಕಾಗಿದೆ. ನಾನೇ ದೇವರು ಎಂದು ಇಂದ್ರಿಯ ಜಗತ್ತಿನ ಸಂಬಂಧದಲ್ಲಿ ಹೇಳಲು ಆಗುವುದಿಲ್ಲ. ಇಂದ್ರಿಯ ಜಗತ್ತಿನ ಸಂದರ್ಭದಲ್ಲಿ ನೀನೇ ದೇವರು ಎಂದರೆ ನೀನು ಮಾಡುವ ತಪ್ಪುಗಳನ್ನು ತಡೆಯುವವರು ಯಾರು? ಆದಕಾರಣ ಪಾವಿತ್ರ್ಯ ಎಂಬುದು ಆತ್ಮನ ನಿರಪೇಕ್ಷ ಸ್ಥಿತಿಗೆ ಮಾತ್ರ ಅನ್ವಯಿಸುತ್ತದೆ. ನಾನು ದೇವನಾದರೆ, ನಾನು ಇಂದ್ರಿಯಗಳ ಹೀನ ಸ್ವಭಾವಗಳಿಗೆ ಅತೀತನಾಗಿರುವೆನು; ನಾನು ಯಾವ ಪಾಪವನ್ನೂ ಮಾಡಲಾರೆನು. ನೀತಿಯೇ ಮಾನವನ ಗುರಿಯಲ್ಲ. ಮುಕ್ತಿಗೆ ಅದೊಂದು ದಾರಿ ಅಷ್ಟೆ. ದಿವ್ಯತೆಯ ಸಾಕ್ಷಾತ್ಕಾರಕ್ಕೆ ಯೋಗ ಒಂದು ಮಾರ್ಗ ಎನ್ನುವುದು ವೇದಾಂತ. ಆತ್ಮ ಸ್ವಾರಾಜ್ಯವನ್ನು ಅರಿತ ಮೇಲೆ ಎಲ್ಲವೂ ಅದಕ್ಕೆ ಬಾಗುವುದು. ಧರ್ಮ ನೀತಿ ಇವುಗಳೆಲ್ಲ ಅದಕ್ಕೆ ದಾರಿ ಮಾಡಿಕೊಡುವುವು.

ಅದ್ವೈತದ ಮೇಲೆ ಆರೋಪಿಸುವ ಟೀಕೆಗಳನ್ನೆಲ್ಲ ಈ ಮಾತುಗಳಲ್ಲಿ ಹೇಳ ಬಹುದು: ಇದು ಇಂದ್ರಿಯಭೋಗಕ್ಕೆ ಅವಕಾಶ ಕೊಡುವುದಿಲ್ಲ. ನಾವು ಇದನ್ನು ಸಂತೋಷವಾಗಿ ಒಪ್ಪಿಕೊಳ್ಳುತ್ತೇವೆ.

ವೇದಾಂತದರ್ಶನವು ನಿರಾಶಾವಾದದಿಂದ ಪ್ರಾರಂಭವಾಗಿ ನಿಜವಾದ ಆಶಾ ವಾದದಲ್ಲಿ ಪರ್ಯವಸಾನವಾಗುವುದು. ನಾವು ಇಂದ್ರಿಯ ಸುಖಕ್ಕೆ ಸಂಬಂಧಿಸಿ ದಂತೆ ಆಶಾವಾದವನ್ನು ಅಲ್ಲಗಳೆಯುತ್ತೇವೆ. ಆದರೆ ಇಂದ್ರಿಯಗಳಿಗೆ ಅತೀತವಾದ ನಿಜವಾದ ಆಶಾವಾದವನ್ನು ಒಪ್ಪುತ್ತೇವೆ. ಆ ನಿಜವಾದ ಆನಂದ ಇಂದ್ರಿಯಗಳಲ್ಲಿ ಇಲ್ಲ. ಅದು ಅವುಗಳಿಗೆ ಅತೀತವಾಗಿರುವುದು. ಅದು ಪ್ರತಿಯೊಬ್ಬರಲ್ಲಿಯೂ ಇರುವುದು. ಪ್ರಪಂಚದಲ್ಲಿಯ ಆಶಾಭಾವನೆ ಇಂದ್ರಿಯಗಳ ಮೂಲಕ ನಮ್ಮನ್ನು ದುರಂತಕ್ಕೆ ಒಯ್ಯುವುದು.

ನಮ್ಮ ದರ್ಶನದಲ್ಲಿ ತ್ಯಾಗಕ್ಕೆ ಬಹಳ ಮುಖ್ಯವಾದ ಸ್ಥಾನವಿದೆ. ಪ್ರಪಂಚವನ್ನು ನಿರಾಕರಿಸುವುದು ಎಂದರೆ ನಿಜವಾದ ಆತ್ಮನನ್ನು ಒಪ್ಪಿಕೊಳ್ಳುವುದು ಎಂದು ಅರ್ಥ. ವೇದಾಂತವು ಇಂದ್ರಿಯ ಜಗತ್ತನ್ನು ಅಲ್ಲಗಳೆಯುವುದರಿಂದ ಅದು ಜಗತ್ತಿನಲ್ಲಿ ನಿರಾಶೆಯನ್ನೇ ನೋಡುವ ದೃಷ್ಟಿಯುಳ್ಳದ್ದು ಎಂಬುದು ನಿಜ. ಆದರೆ ಅದು ನಿಜವಾದ ಜಗತ್ತನ್ನು ಎಂದರೆ ಸುಖಸ್ವರೂಪವಾದ ಆತ್ಮನನ್ನು ಒಪ್ಪಿ ಕೊಳ್ಳುವುದರಿಂದ, ಅದು ಆಶಾವಾದವನ್ನು ಒಪ್ಪಿಕೊಳ್ಳುವ ದರ್ಶನ ಎಂದಾ ಗುತ್ತದೆ.

ವೇದಾಂತವು ವಿಚಾರ ಶಕ್ತಿಗಿಂತ ಅತೀತವಾದುದೊಂದು ಇದೆ ಎಂದು ಹೇಳುತ್ತಿದ್ದರೂ, ಅದು ವಿಚಾರ ಶಕ್ತಿಗೆ ಹೆಚ್ಚು ಪ್ರಾಧಾನ್ಯವನ್ನು ನೀಡುತ್ತದೆ. ಯುಕ್ತ್ಯಾತೀತ ಅವಸ್ಥೆಗೆ ನಾವು ಬುದ್ಧಿಶಕ್ತಿಯ ಮೂಲಕವೇ ಹೋಗಬೇಕಾಗಿದೆ.

ಹಳೆಯ ಮೂಢನಂಬಿಕೆಗಳನ್ನೆಲ್ಲಾ ಗುಡಿಸಿ ಹಾಕಬೇಕಾದರೆ ನಮಗೆ ವಿಚಾರ ಶಕ್ತಿ ಆವಶ್ಯಕ. ಅನಂತರ ಉಳಿಯುವುದೇ ವೇದಾಂತ. ಒಂದು ಸುಂದರವಾದ ಸಂಸ್ಕೃತ ಪದ್ಯವಿದೆ. ಅದರಲ್ಲಿ “ನನ್ನ ಸಖನೆ, ನೀನು ಅಳುವುದೇಕೆ? ನಿನಗೆ ಭಯವೂ ಇಲ್ಲ, ಮೃತ್ಯುವೂ ಇಲ್ಲ. ನೀನು ಅಳುವುದೇಕೆ? ನಿನಗೆ ದುಃಖವೂ ಇಲ್ಲ. ಅವಿಕಾರಿಯಾದ ಅನಂತ ನೀಲಾಕಾಶದಂತೆ ನೀನು. ಅಲ್ಲಿ ಹಲವು ಬಗೆಯ ಮೋಡಗಳು ಬಂದು ಕೆಲವು ಕಾಲ ಆಡಿ ಹೋಗುವುವು. ಅದು ಎಂದಿನಂತೆಯೇ ಅದೇ ಆಕಾಶ, ನೀನು ಮೋಡಗಳನ್ನು ಮಾತ್ರ ಚದುರಿಸಬೇಕಾಗಿದೆ.”

ನಾವು ಕೆರೆಯ ತೂಬನ್ನು ತೆರೆದು ಪಾತ್ರವನ್ನು ಸರಿಮಾಡಬೇಕಾಗಿದೆ, ಅಷ್ಟೆ. ನೀರು ಸ್ವಾಭಾವಿಕವಾಗಿ ಹರಿದುಬರುವುದು. ಏಕೆಂದರೆ ನೀರು ಆಗಲೆ ಕೆರೆಯಲ್ಲಿ ಇರುವುದು.

ಮನುಷ್ಯನು ಬಹುಮಟ್ಟಿಗೆ ಪ್ರಜ್ಞಾಸ್ಥಿತಿಯಲ್ಲಿರುವನು \enginline{(Conscious)}. ಅವನ ಒಂದು ಭಾಗ ಅಪ್ರಜ್ಞೆಯ ಸ್ಥಿತಿಯಲ್ಲಿರುವುದು \enginline{(unconsious)}. ಅವನು ಬೇಕಾದರೆ ಪ್ರಜ್ಞಾತೀತ ಅವಸ್ಥೆಗೂ \enginline{(Superconscious)} ಏರಬಹುದು. ನಾವು ನಿಜವಾದ ಮಾನವರಾದಾಗ ಮಾತ್ರ ವಿಚಾರವನ್ನು ಮೀರಿ ಹೋಗಬಹುದು. ಉನ್ನತವಾದುದು, ಕೆಳಗಿನದು ಎಂಬ ಪದಗಳನ್ನು ಬಾಹ್ಯ ಪ್ರಪಂಚದಲ್ಲಿ ಮಾತ್ರ ಉಪಯೋಗಿಸಬಹುದು. ಅವನ್ನು ಅವ್ಯಕ್ತ ಪ್ರಪಂಚಕ್ಕೆ ಉಪಯೋಗಿಸಿದರೆ ಅವಕ್ಕೆ ಅರ್ಥವೇ ಇಲ್ಲ. ಏಕೆಂದರೆ ಅಲ್ಲಿ ವ್ಯತ್ಯಾಸವೇ ಇಲ್ಲ. ಪ್ರಪಂಚದಲ್ಲಿ ಮಾನವ ಅವಿರ್ಭಾವವೇ ಶ್ರೇಷ್ಠವಾದುದು. ಮಾನವ ದೇವನಿಗಿಂತಲೂ ಮಿಗಿಲು ಎಂದು ವೇದಾಂತ ಹೇಳುವುದು. ದೇವತೆಗಳೆಲ್ಲ ಕಾಲವಾಗಿ ಪುನಃ ಮಾನವರಾಗಿ ಹುಟ್ಟಬೇಕು. ಮಾನವ ಜನ್ಮದಲ್ಲಿ ಮಾತ್ರ ಅವರು ಮುಕ್ತರಾಗುವರು.

ನಾವು ಒಂದು ಸಿದ್ಧಾಂತವನ್ನು ರೂಪಿಸಬಹುದು ನಿಜ. ಆದರೆ ಅದು ಪೂರ್ಣವಾಗಿಲ್ಲ ಎಂಬುದನ್ನು ಒಪ್ಪಿಕೊಳ್ಳಬೇಕಾಗುವುದು. ಏಕೆಂದರೆ ಸತ್ಯವು ಸಿದ್ಧಾಂತಗಳನ್ನೆಲ್ಲ ಮೀರಿರಬೇಕು. ಒಂದು ಸಿದ್ಧಾಂತವನ್ನು ಬೇಕಾದರೆ ಇತರ ಸಿದ್ಧಾಂತಗಳೊಂದಿಗೆ ಹೋಲಿಸಿ, ಅದೇ ಎಲ್ಲಕ್ಕಿಂತ ಯುಕ್ತಿಬದ್ಧವಾಗಿರುವುದು ಎಂದು ಬೇಕಾದರೆ ತೋರಬಹುದು. ಆದರೆ ಅದೂ ಪೂರ್ಣವಾಗಿಲ್ಲ. ಏಕೆಂದರೆ ಯಾವ ಯುಕ್ತಿಯೂ ಪೂರ್ಣವಾಗಲಾರದು. ಆದರೆ ಮಾನವನಿಗೆ ಸಾಧ್ಯವಾದ ಯುಕ್ತಿಬದ್ಧವಾದ ಸಿದ್ಧಾಂತ ಎಂದರೆ ಇದೊಂದೇ ಎಂದು ಹೇಳಬಹುದು.

ಒಂದು ಸಿದ್ಧಾಂತ ಬಲವಾಗಿರಬೇಕಾದರೆ ಅದು ಪ್ರಚಾರವಾಗಬೇಕು ಎಂಬುದೇನೊ ಸ್ವಲ್ಪಮಟ್ಟಿಗೆ ನಿಜ. ಮತ್ತಾವ ಸಿದ್ಧಾಂತವೂ ವೇದಾಂತದಷ್ಟು ಪ್ರಚಾರವಾಗಿಲ್ಲ. ಈಗಲೂ ಕೂಡ ವೈಯಕ್ತಿಕ ಸಂಪರ್ಕವೇ ಬೋಧಿಸುವುದು. ಸುಮ್ಮನೆ ಓದಿದರೆ ಒಬ್ಬ ಮಹಾಪುರುಷನಾಗುವುದಿಲ್ಲ. ಯಾರು ನಿಜವಾಗಿ ಮಹಾತ್ಮರಾಗಿರುವರೊ ಅವರು ಮತ್ತೊಬ್ಬರ ಸಂಪರ್ಕದಿಂದ ಹಾಗೆ ಆದರು. ಇಂತಹ ನಿಜವಾದ ಮಹಾತ್ಮರು ಬಹಳ ಕಡಿಮೆ ಎಂಬುದೇನೊ ನಿಜ. ಆದರೆ ಅವರ ಸಂಖ್ಯೆ ಕ್ರಮೇಣ ಹೆಚ್ಚಾಗುವುದು. ಆದರೂ ಒಂದು ದಿನ ಬರುವುದು, ಆಗ ಇರುವವರೆಲ್ಲ ತತ್ತ್ವಜ್ಞಾನಿಗಳಾಗುವರು ಎಂಬುದು ನಿಜವಲ್ಲ. ಎಲ್ಲಾ ಸುಖಮಯವಾದ, ದುಃಖವೇ ಇಲ್ಲದ ಕಾಲ ಒಂದು ಬರುವುದು ಎಂಬುದನ್ನು ನಾವು ನಂಬುವುದಿಲ್ಲ.

ಯಾವಾಗಲಾದರೊಮ್ಮೆ ಅತ್ಯಾನಂದದ ಕ್ಷಣವೊಂದು ನಮಗೆ ಲಭಿಸುವುದು. ಆಗ ನಾವು ಏನನ್ನೂ ಕೇಳುವುದಿಲ್ಲ; ಏನನ್ನೂ ನೋಡುವುದಿಲ್ಲ. ಆನಂದವಲ್ಲದೆ ಮತ್ತೇನೂ ನಮಗೆ ಗೊತ್ತಿರುವುದಿಲ್ಲ. ಅನಂತರ ಅದು ಮಾಯವಾಗುವುದು. ಪ್ರಪಂಚದ ದೃಶ್ಯ ಕಣ್ಣ ಮುಂದೆ ಸುಳಿಯುವುದು. ಇದೊಂದು ದೇವರು ಎಂಬ ಭಿತ್ತಿಯ ಮೇಲೆ ಬಿಡಿಸಿದ ಚಿತ್ರ; ಅವನೇ ಇದರ ಹಿನ್ನೆಲೆಯಲ್ಲಿರುವನು ಎಂಬುದು ನಮಗೆ ಗೊತ್ತಿದೆ.

ಇಲಿಯೆ, ಈ ಕ್ಷಣವೆ, ಬೇಕಾದರೆ ನಿರ್ವಾಣವನ್ನು ಪಡೆಯಬಹುದು; ಸಾಯುವ ತನಕ ಅದಕ್ಕಾಗಿ ಕಾಯಬೇಕಾಗಿಲ್ಲ ಎನ್ನುತ್ತದೆ ವೇದಾಂತ. ಆತ್ಮ ಸಾಕ್ಷಾತ್ಕಾರವೇ ನಿರ್ವಾಣ. ಅದನ್ನು ಒಂದು ಕ್ಷಣವಾದರೂ ಅರಿತ ಮೇಲೆ ಪುನಃ ವ್ಯಕ್ತಿಯು ಈ ವ್ಯಕ್ತಿತ್ವವೆಂಬ ಮರೀಚಿಕೆಯ ಮೋಹಕ್ಕೆ ಒಳಗಾಗುವುದಿಲ್ಲ. ನಮಗೆ ಕಣ್ಣಿರುವುದ ರಿಂದ ನಾವು ಈ ತೋರಿಕೆಯ ಜಗತ್ತನ್ನು ನೋಡಬೇಕಾಗಿದೆ. ಆದರೆ ಇದು ಏನು ಎಂಬುದು ನಮಗೆ ಸದಾ ತಿಳಿದಿರುತ್ತದೆ. ಅದರ ನೈಜ ಸ್ವಭಾವ ನಮಗೆ ಗೊತ್ತಾಗಿದೆ. ಬದಲಾಗದೆಯೇ ಇರುವ ಆತ್ಮವನ್ನು ಜಗತ್ತೆಂಬ ತೆರೆಯು ಮುಚ್ಚಿರುವುದು. ತೆರೆಯನ್ನು ಸರಿಸಿದರೆ ಅದರ ಹಿಂದೆ ಇರುವ ಆತ್ಮವು ಗೋಚರಿ ಸುತ್ತದೆ. ಬದಲಾವಣೆ ಇರುವುದೆಲ್ಲ ತೆರೆಯಲ್ಲಿಯೇ. ಸಂತನಲ್ಲಿ ಈ ತೆರೆ ಬಹಳ ತೆಳುವಾಗಿರುತ್ತದೆ. ಸತ್ಯ ಇದರ ಮೂಲಕ ಹೊಳೆಯುತ್ತಿರುವುದು. ಪಾಪಿಯಲ್ಲಿ ತೆರೆ ಬಹಳ ದಪ್ಪವಾಗಿರುವುದರಿಂದ ಅದರ ಹಿಂದೆ ಇರುವ ಆತ್ಮಜ್ಯೋತಿ ಜ್ಞಾನಿಯ ಹಿಂದೆ ಇರುವಂತೆಯೇ ಇದ್ದರೂ ವ್ಯಕ್ತವಾಗುವುದಿಲ್ಲ. ಆ ತೆರೆ ಸಂಪೂರ್ಣವಾಗಿ ಹೋದ ಮೇಲೆ ಅದೆಂದಿಗೂ ಇರಲೇ ಇಲ್ಲ. ನಾವು ಆತ್ಮವೇ ಅಲ್ಲದೆ ಬೇರೆಯಲ್ಲ, ಆವರಣವೇ ಇರಲಿಲ್ಲ ಎಂದು ಎನ್ನಿಸುವುದು.

ಜ್ಞಾನಿಯ ಎರಡು ಲಕ್ಷಣಗಳು ಇವು: ಮೊದಲೆಯದಾಗಿ ಜ್ಞಾನಿಯು ಯಾವುದರಿಂದಲೂ ಬಾಧಿತನಾಗುವುದಿಲ್ಲ, ಎರಡನೆಯದಾಗಿ ಅವನು ಮಾತ್ರ ಜಗತ್ತಿಗೆ ಒಳ್ಳೆಯದು ಮಾಡಬಲ್ಲ. ಆತನಿಗೆ ಮಾತ್ರ ಮತ್ತೊಬ್ಬನಿಗೆ ಉಪಕಾರ ಮಾಡುವುದು ಏನೆಂಬುದು ಚೆನ್ನಾಗಿ ಗೊತ್ತಿದೆ. ಏಕೆಂದರೆ ಹಾಗೆ ಹೇಳಿದರೆ ಹಲವನ್ನು ಒಪ್ಪಿಕೊಂಡಂತೆ ಆಗುವುದು. ಇದೇ ನಿಜವಾದ ನಿಃಸ್ವಾರ್ಥ. ಇದುಸಮಷ್ಟಿಯನ್ನು ನೋಡುವುದು, ವ್ಯಷ್ಟಿಯನ್ನು ಅಲ್ಲ. ಪ್ರೀತಿ ಸಹಾನುಭೂತಿ ಇವನ್ನು ತೋರಿದಾಗಲೆಲ್ಲ ನಾವು ಸಮಷ್ಟಿಯನ್ನು ಒಪ್ಪಿಕೊಳ್ಳುತ್ತಿರುವೆವು. “ನಾನಲ್ಲ ನೀನು”. ದಾರ್ಶನಿಕವಾಗಿ ಹೇಳುವುದಾದರೆ, ಇನ್ನೊಬ್ಬನಿಗೆ ನೀನು ಸಹಾಯ ಮಾಡಿದರೆ ನೀನು ಅವನಲ್ಲಿರುವೆ; ಅದಕ್ಕೆಂದೇ ಸಹಾಯ ಮಾಡುವೆ. ನಿಜವಾದ ವೇದಾಂತಿ ಮಾತ್ರ ಯಾವ ಅಳುಕೂ ಇಲ್ಲದೆ ತನ್ನ ಪ್ರಾಣವನ್ನು ಇತರರಿಗೆ ಕೊಡಬಲ್ಲ. ಏಕೆಂದರೆ ಅವನಿಗೆ ತಾನು ಸಾಯುವುದಿಲ್ಲ ಎಂದು ಚೆನ್ನಾಗಿ ಗೊತ್ತಿದೆ. ಒಂದೇ ಒಂದು ಕೀಟವು ಪ್ರಪಂಚದಲ್ಲಿರುವವರೆಗೂ ಅವನು ಬದುಕಿರುವನು. ಒಂದೇ ಒಂದು ಬಾಯಿ ಊಟ ಮಾಡುತ್ತಿರುವವರೆಗೂ ಅವನು ಊಟ ಮಾಡು ತ್ತಿರುವನು. ಹೀಗೆ ಅವನು ಇತರರಿಗೆ ಒಳ್ಳೆಯದನ್ನು ಮಾಡುತ್ತಾ ಹೋಗುವನು. ತನ್ನ ದೇಹವನ್ನು ರಕ್ಷಣೆ ಮಾಡಿಕೊಳ್ಳಬೇಕು, ಎಂಬ ಈಗಿನ ಕಾಲದ ಭಾವನೆಗೆ ಅವನು ದಾಸನಾಗುವುದಿಲ್ಲ. ಒಬ್ಬನು ತ್ಯಾಗದ ಇಂತಹ ಸ್ಥಿತಿಯನ್ನು ಮಟ್ಟಿದರೆ ಅವನು ನೀತಿಯ ಹೋರಾಟದಿಂದ ಪಾರಾಗುವನು, ಎಲ್ಲಾ ಘರ್ಷಣೆಗಳಿಂದಲೂ ಪಾರಾಗುವನು, ಪಂಡಿತನಾಗಲಿ, ದನವಾಗಲಿ, ನಾಯಿಯಾಗಲಿ, ಅತಿ ಹೀನವಾದ ಸ್ಥಳವಾಗಲಿ-ಇವನು ಪಂಡಿತ, ಇದು ದನ, ಇದು ನಾಯಿ, ಇದು ಅಸಹ್ಯ ವಾದುದು ಎಂದು ಅವನು ಯಾವುದನ್ನೂ ನೋಡುವುದಿಲ್ಲ; ಒಂದೇ ಸತ್ಯ ಇವುಗಳಲ್ಲೆಲ್ಲಾ ವ್ಯಕ್ತವಾಗುತ್ತಿರುವುದನ್ನು ನೋಡುವನು. ಅವನೇ ಸುಖಿ, ಅವನೇ ಸಮದರ್ಶಿ. ಬದುಕಿರುವಾಗಲೇ ಅವನು ದ್ವಂದ್ವಾತೀತನಾಗಿರುವನು. ದೇವರು ಪರಿಶುದ್ಧ. ಆದಕಾರಣ ಇಂತಹವನು ದೇವರಲ್ಲಿರುವನು. ಏಸುವು ಅಬ್ರಹಾಮನಿ ಗಿಂತ ಮುಂಚೆ ನಾನಿದ್ದೆ ಎನ್ನುವನು. ಆದರೆ ಏಸು ಮತ್ತು ಹಾಗೆ ಇರುವವರೆಲ್ಲ ಮುಕ್ತಾತ್ಮರು ಎಂದು ಅರ್ಥ. ಏಸು ಮಾನವಾಕೃತಿಯನ್ನು ಧಾರಣ ಮಾಡಿದುದು ತನ್ನ ಕರ್ಮದಿಂದ ಪ್ರೇರಿತನಾಗಿ ಅಲ್ಲ. ಮಾನವ ಕೋಟಿಗೆ ಮಂಗಳವನ್ನು ಮಾಡುವುದಕ್ಕೆ ಮಾತ್ರ. ಒಬ್ಬನು ಮುಕ್ತನಾದರೆ ಅವನು ಸುಮ್ಮನೆ ನಿಂತು ಒಂದು ಮಣ್ಣಿನ ಮುದ್ದೆಯಂತೆ ಆಗುವನು ಎಂದಲ್ಲ, ಅವನು ಇತರರಿಗಿಂತ ಹೆಚ್ಚು ಕರ್ಮ ದಲ್ಲಿ ನಿರತನಾಗಿರುವನು. ಇತರರು ಕರ್ಮಕ್ಕೆ ಸಿಕ್ಕಿ ನಿರ್ವಾಹವಿಲ್ಲದೆ ಅದನ್ನು ಮಾಡಬೇಕಾಗಿದೆ. ಆದರೆ ಮುಕ್ತನು ಮಾತ್ರ ಸ್ವತಂತ್ರನಾಗಿ ಕರ್ಮ ಮಾಡುವನು.

ನಾವು ದೇವರಿಂದ ಬೇರೆಯಲ್ಲವಾದರೆ ನಮಗೆ ವ್ಯಕ್ತಿತ್ವವಿಲ್ಲವೆ? ಇದೆ. ಆ ವ್ಯಕ್ತಿತ್ವವೇ ದೇವರು. ನಮ್ಮ ವ್ಯಕ್ತಿತ್ವವೇ ದೇವರು. ಅದು ನಿಮಗೆ ಈಗ ಇರುವ ವ್ಯಕ್ತಿತ್ವವಲ್ಲ. ನೀವು ಅದರ ಸಮೀಪಕ್ಕೆ ಬರುತ್ತಿರುವಿರಿ. ವ್ಯಕ್ತಿತ್ವ ಎಂದರೆ ಯಾವುದನ್ನು ವಿಭಾಗಿಸಲಾರೆವೊ ಅದು. ಈಗಿರುವುದನ್ನು ಹೇಗೆ ವ್ಯಕ್ತಿತ್ವ ಎನ್ನುವಿರಿ? ಈ ಕ್ಷಣ ಒಂದು ರೀತಿಯಾಗಿ ಆಲೋಚಿಸುವಿರಿ. ಮರುಕ್ಷಣವೆ ಬೇರೊಂದು ರೀತಿಯಾಗಿ ಆಲೋಚಿಸುವಿರಿ. ಇನ್ನೆರಡು ಗಂಟೆಗಳ ತರುವಾಯ ಮತ್ತೊಂದು ಬಗೆಯಾಗಿ ಆಲೋಚಿಸುವಿರಿ. ವ್ಯಕ್ತಿತ್ವ ಎಂದರೆ ಬದಲಾಗದೆ ಇರುವಂತಹದು. ಅದು ಎಲ್ಲಕ್ಕೂ ಅತೀತವಾಗಿರುವುದು, ಅವಿಕಾರಿಯಾದುದು. ಈಗಿರುವ ಸ್ಥಿತಿಯೇ ಯಾವಾಗಲೂ ಇದ್ದರೆ ಬಹಳ ಅಪಾಯಕರವಾಗುವುದು. ಆಗ ಕಳ್ಳ ಎಂದಿಗೂ ಕಳ್ಳನಾಗಿಯೇ ಇರುವನು. ಮೋಸಗಾರ ಎಂದೆಂದಿಗೂ ಮೋಸಗಾರನಾಗಿಯೇ ಇರಬೇಕಾಗುವುದು. ನಿಜವಾದ ವ್ಯಕ್ತಿತ್ವವಾದರೊ ಎಂದಿಗೂ ಬದಲಾಗುವುದಿಲ್ಲ. ಅದೇ ನಮ್ಮ ಅಂತರಂಗದಲ್ಲಿರುವ ದೇವರು.

ವೇದಾಂತವು ಅಸೀಮ ಸಾಗರದಂತೆ. ಅಲ್ಲಿ ದೊಡ್ಡದೊಂದು ಹಡಗು ಕೂಡ ಒಂದು ಸಣ್ಣ ದೋಣಿಯ ಹತ್ತಿರ ಇರಬಲ್ಲದು. ಇದರಂತೆಯೇ ವೇದಾಂತಸಾಗರ ದಲ್ಲಿ ಒಬ್ಬ ನಿಜವಾದ ಯೋಗಿ ವಿಗ್ರಹಾರಾಧಕರ ಸಮೀಪದಲ್ಲಿರಬಹುದು, ನಾಸ್ತಿಕನ ಸಮೀಪದಲ್ಲಿರಬಹುದು. ಇದಕ್ಕಿಂತ ಹೆಚ್ಚಾಗಿ ವೇದಾಂತ ಸಾಗರದಲ್ಲಿ ಹಿಂದೂ, ಮಹಮ್ಮದೀಯ, ಕ್ರಿಸ್ತ, ಪಾರ್ಸಿ ಎಲ್ಲರೂ ಒಂದು, ಒಬ್ಬ ಪರಮೇಶ್ವರನ ಮಕ್ಕಳು.

