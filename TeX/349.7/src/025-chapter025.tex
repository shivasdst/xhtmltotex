

\part{ಪ್ರವಚನ ಸಾರಾಂಶ}

\chapter[ರಾಜಯೋಗವನ್ನು ಕುರಿತು ]{ರಾಜಯೋಗವನ್ನು ಕುರಿತು \protect\footnote{\engfoot{C.W. Vol. VI, P 128}}}

\centerline{(ಇದು ಮತ್ತು ಭಕ್ತಿಯೋಗವು ಇಂಗ್ಲೆಂಡಿನಲ್ಲಿ ನೀಡಿದ ತರಗತಿಗಳ ಟಿಪ್ಪಣಿಗಳನ್ನು ಆಧರಿಸಿದೆ)}

\begin{center}
\textbf{ಪ್ರಾಣ}
\end{center}

ದ್ರವ್ಯವು ಪಂಚಭೂತಗಳ ಅವಸ್ಥೆಯಲ್ಲಿರುವುದು ಎಂದು ಸೃಷ್ಟಿ ಸಿದ್ಧಾಂತವು ಸಾರುತ್ತದೆ. ಆಕಾಶ, ಬೆಳಕು, ಅನಿಲ, ದ್ರವ್ಯ ಮತ್ತು ಘನ ಇವುಗಳೆಲ್ಲ ಅತಿ ಸೂಕ್ಷ್ಮವಾದ ಒಂದು ಮೂಲಾಕಾಶದಿಂದ ಬಂದಿವೆ.

ವಿಶ್ವಶಕ್ತಿಯನ್ನೇ ಪ್ರಾಣ ಎಂದು ಕರೆಯುವುದು. ಈ ಶಕ್ತಿಯು ಮೇಲೆ ಹೇಳಿದ ಎಲ್ಲಾ ಭೂತಗಳಲ್ಲಿಯೂ ಇದೆ. ಮನಸ್ಸೇ ಪ್ರಾಣವನ್ನು ಉಪಯೋಗಿಸುವ ದೊಡ್ಡ ಉಪಕರಣ. ಮನಸ್ಸು ಭೌತಿಕವಾದುದು. ಇದರ ಹಿಂದೆಯೇ ಆತ್ಮನಿರುವುದು. ಈ ಆತ್ಮವೇ ಪ್ರಾಣವನ್ನು ವಶಕ್ಕೆ ತೆಗೆದುಕೊಳ್ಳುವುದು. ಪ್ರಾಣವೇ ಜಗತ್ತಿನ ಕ್ರಿಯಾಶಕ್ತಿ. ಇದನ್ನು ಜೀವನದ ಎಲ್ಲಾ ಕಾರ್ಯಕ್ಷೇತ್ರಗಳಲ್ಲಿಯೂ ನೋಡುತ್ತೇವೆ. ದೇಹ ಮತ್ತು ಮನಸ್ಸು ಇವೆರಡೂ ಮಿಶ್ರವಾಗಿರುವುದರಿಂದ ನಾಶವಾಗಲೇಬೇಕು. ಎಲ್ಲದರ ಹಿಂದೆ ಎಂದಿಗೂ ನಾಶವಾಗದ ಆತ್ಮ ಇರುವುದು. ಆತ್ಮನೇ ಶುದ್ಧಚೈತನ್ಯ. ಅದೇ ಪ್ರಾಣವನ್ನು ನಡೆಸುತ್ತಿರುವುದು. ಆದರೆ ನಮ್ಮ ಸುತ್ತಲೂ ಕಾಣುವ ಚೈತನ್ಯ ಅಪೂರ್ಣ. ಅದು ಎಲ್ಲಿ ಪೂರ್ಣವಾಗುತ್ತದೋ ಅಲ್ಲಿ ಅವತಾರವನ್ನು ಕಾಣುತ್ತೇವೆ, ಕ್ರಿಸ್ತನನ್ನು ಕಾಣುತ್ತೇವೆ. ಚೈತನ್ಯವು ಯಾವಾಗಲೂ ವ್ಯಕ್ತವಾಗಲು ಯತ್ನಿಸುತ್ತಿದೆ. ಅದಕ್ಕಾಗಿ ಅದು ಬೆಳವಣಿಗೆಯ ಬೇರೆ ಬೇರೆ ಹಂತಗಳಲ್ಲಿರುವ ದೇಹಗಳನ್ನು ಸೃಷ್ಟಿಸುತ್ತದೆ. ವಾಸ್ತವವಾಗಿ ಎಲ್ಲಾ ಜೀವಿಗಳೂ ಸಮಾನ.

ಮನಸ್ಸು ಅತಿ ಸೂಕ್ಷ್ಮವಾದ ದ್ರವ್ಯ. ಪ್ರಾಣವು ವ್ಯಕ್ತವಾಗುವುದಕ್ಕೆ ಅದೇ ಉಪಕರಣ. ಶಕ್ತಿ ವ್ಯಕ್ತವಾಗಬೇಕಾದರೆ ಅದಕ್ಕೆ ದ್ರವ್ಯ ಬೇಕು.

ಮುಂದಿನ ಪ್ರಶ್ನೆ ನಾವು ಈ ಪ್ರಾಣವನ್ನು ಹೇಗೆ ಉಪಯೋಗಿಸಬೇಕು ಎಂಬುದು. ನಾವೆಲ್ಲ ಅದನ್ನು ಉಪಯೋಗಿಸುತ್ತೇವೆ; ಆದರೆ ಎಷ್ಟೊಂದನ್ನು ಹಾಳುಮಾಡುತ್ತೇವೆ! ಸಾಧನೆಯ ಪ್ರಾರಂಭದ ಹಂತದಲ್ಲಿ ಜ್ಞಾನವೆಲ್ಲ ನಮ್ಮ ಅನುಭವಜನ್ಯ ಎಂಬುದೇ ಮೊದಲ ಸಿದ್ಧಾಂತ. ಯಾವುದು ನಮ್ಮ ಪಂಚೇಂದ್ರಿಯ ಗಳಿಗೆ ಮೀರಿದೆಯೋ ಅದೂ ಕೂಡ ನಮ್ಮ ಮಟ್ಟಿಗೆ ನಿಜವಾಗಬೇಕಾದರೆ ನಮ್ಮ ಅನುಭವಕ್ಕೆ ಬರಲೇಬೇಕು.

ನಮ್ಮ ಮನಸ್ಸು ಮೂರು ಭೂಮಿಕೆಗಳಲ್ಲಿ ಕೆಲಸ ಮಾಡುತ್ತಿರುವುದು; ಅವೇ ಅಪ್ರಜ್ಞೆ, ಪ್ರಜ್ಞೆ ಮತ್ತು ಅತಿಪ್ರಜ್ಞೆ. ಜನರಲ್ಲಿ ಯೋಗಿಯು ಮಾತ್ರ ಅತಿಪ್ರಜ್ಞೆ ಯನ್ನು ಅರಿಯಬಲ್ಲ. ಮನಸ್ಸನ್ನು ಅತಿಕ್ರಮಿಸಿ ಹೋಗುವುದೇ ಯೋಗದ ಗುರಿ. ಈ ಮೂರು ಭೂಮಿಕೆಗಳನ್ನು ಧ್ವನಿ ಅಥವಾ ಬೆಳಕಿನ ಸ್ಪಂದನದ ಉದಾಹರಣೆಯ ಮೂಲಕ ತಿಳಿದುಕೊಳ್ಳಬಹುದು. ಕೆಲವು ಬೆಳಕಿನ ಸ್ಪಂದನಗಳು ಬಹಳ ಕಡಿಮೆ ಆಗಿರುವುದರಿಂದ ನಮಗೆ ಅದನ್ನು ನೋಡಲು ಸಾಧ್ಯವಿಲ್ಲ. ಸ್ಪಂದನ ಸ್ವಲ್ಪ ಹೆಚ್ಚಾದಾಗ ಬೆಳಕಿನಂತೆ ನಾವು ಅದನ್ನು ನೋಡಬಹುದು. ಸ್ಪಂದನದ ವೇಗ ಇನ್ನೂ ಹೆಚ್ಚಾದಾಗ ಅದು ಕಣ್ಣಿಗೆ ಕಾಣುವುದಿಲ್ಲ. ಇದರಂತೆಯೇ ಧ್ವನಿ ಕೂಡ.

ನಮ್ಮ ಆರೋಗ್ಯಕ್ಕೆ ಹಾನಿಯನ್ನು ತರದೆ ಇಂದ್ರಿಯಗಳನ್ನು ಹೇಗೆ ಅತಿಕ್ರಮಿಸು ವುದು ಎಂಬುದನ್ನೇ ಕಲಿಯಬೇಕಾಗಿರುವುದು. ಪಾಶ್ಯಾತ್ಯರಲ್ಲಿ ಕೆಲವರು ಹೇಗೋ ಕೆಲವು ಮಾನಸಿಕ ಶಕ್ತಿಗಳನ್ನು ಪಡೆದಿರುವರು. ಅವರಿಗೆ ಅದು ಅಸ್ವಾಭಾವಿಕ; ಅದನ್ನು ಒಂದು ರೋಗವೆಂದು ಪರಿಗಣಿಸುವರು. ಹಿಂದೂಗಳು ಈ ಶಾಸ್ತ್ರವನ್ನು ಚೆನ್ನಾಗಿ ಪರೀಕ್ಷಿಸಿ ಅದನ್ನು ಪೂರ್ಣತೆಗೆ ಕೊಂಡೊಯ್ದಿರುವರು. ಈಗ ಯಾರು ಬೇಕಾದರೂ ಅದನ್ನು ಯಾವ ಅಪಾಯ, ಅಂಜಿಕೆ ಇಲ್ಲದೆ ತಿಳಿದುಕೊಳ್ಳಬಹುದು.

ಮನಸ್ಸಿನ ಮೂಲಕ ಒಂದು ಕಾಯಿಲೆಯನ್ನು ಗುಣಮಾಡುವುದು ಅತಿ ಪ್ರಜ್ಞೆಗೆ ಒಂದು ಒಳ್ಳೆಯ ಉದಾಹರಣೆ. ಗುಣಮಾಡುವ ಆಲೋಚನೆ ಪ್ರಾಣದ ಒಂದು ಸ್ಪಂದನ, ಅದು ಆಲೋಚನೆಯಂತೆ ಹೋಗುವುದಿಲ್ಲ. ಅದಕ್ಕೂ ಮೇಲುತರದ ಮತ್ತಾವುದೋ ಶಕ್ತಿಯಂತೆ ಹೋಗುವುದು. ಅದನ್ನು ಹೆಸರಿಸಲಾರೆವು.

ಪ್ರತಿಯೊಂದು ಆಲೋಚನೆಗೂ ಮೂರು ಅವಸ್ಥೆಗಳಿವೆ. ಮೊದಲನೆಯದೇ ಅದರ ಆದಿ. ನಮಗೆ ಅದರ ಅರಿವು ಇರುವುದಿಲ್ಲ. ಎರಡನೆಯದು ಅದು ನಮ್ಮ ಪ್ರಜ್ಞೆಯ ವಲಯಕ್ಕೆ ಬರುವುದು; ಮೂರನೆಯದು ನಮ್ಮಿಂದ ಹೊರಗೆ ಹೋಗುವುದು. ಆಲೋಚನೆ ನೀರಿನ ಮೇಲೆ ಏಳುವ ಗುಳ್ಳೆಯಂತೆ. ಆ ಆಲೋಚನೆ ಯನ್ನು ಇಚ್ಛೆಗೆ ಸೇರಿಸಿದಾಗ ಅದನ್ನು ಶಕ್ತಿ ಎನ್ನುವೆವು. ನೀವು ಸಹಾಯಮಾಡಲು ಯತ್ನಿಸುವ ರೋಗಿಯನ್ನು ಗುಣಮಾಡುವುದು ಆಲೋಚನೆಯಲ್ಲ. ಅದರ ಶಕ್ತಿ. ಇವುಗಳ ಹಿಂದೆಲ್ಲ ಇರುವವನೇ ಸೂತ್ರಾತ್ಮ.

ಪ್ರಾಣದ ಕೊನೆಯ ಮತ್ತು ಅತ್ಯುನ್ನತವಾದ ಆವಿರ್ಭಾವವೇ ಪ್ರೀತಿ. ಪ್ರಾಣ ದಿಂದ ಪ್ರೇಮವನ್ನು ಸೃಷ್ಟಿಮಾಡಲು ನಿಮಗೆ ಸಾಧ್ಯವಾದಾಗ ಮುಕ್ತರಾಗುವಿರಿ. ಪಡೆಯಲು ಅತ್ಯಂತ ಕಠಿಣವಾದ, ಶ್ರೇಷ್ಠವಾದ ವಸ್ತು, ಪ್ರೇಮ. ನೀವು ಇತರ ರನ್ನು ಟೀಕಿಸಕೂಡದು, ನಿಮ್ಮನ್ನು ನೀವೇ ಟೀಕಿಸಿಕೊಳ್ಳಬೇಕು. ಕುಡುಕನನ್ನು ನೋಡಿದರೆ ಅವನನ್ನು ದೂರಬೇಡಿ; ನೀವೇ ಬೇರೊಂದು ರೂಪದಲ್ಲಿ ಅವನು ಎಂದು ತಿಳಿಯಿರಿ. ಯಾರಲ್ಲಿ ದೋಷವಿಲ್ಲವೋ ಅವರು ಇತರರಲ್ಲಿ ದೋಷವನ್ನು ನೋಡಲಾರರು. ನಿಮ್ಮಲ್ಲಿ ಏನಿದೆಯೋ ಅದನ್ನೇ ನೀವು ಇತರರಲ್ಲಿ ನೋಡುತ್ತೀರಿ. ಸುಧಾರಣೆಗೆ ಇದೇ ಅತ್ಯುತ್ತಮ ಮಾರ್ಗ. ಛಿದ್ರಾನ್ವೇಷಣೆ ಮಾಡುತ್ತಿರುವ ಸುಧಾರಕರು ತಾವು ಮಾಡುತ್ತಿರುವ ತಪ್ಪನ್ನು ನಿಲ್ಲಿಸಿದರೆ ಜಗತ್ತು ಉತ್ತಮ ಗೊಳ್ಳುತ್ತದೆ. ಇದನ್ನು ನಿಮ್ಮ ಮನಸ್ಸಿಗೆ ಒತ್ತಿಹೇಳಿ.

\begin{center}
\textbf{ಯೋಗಾಭ್ಯಾಸ}
\end{center}

ದೇಹವನ್ನು ಸರಿಯಾಗಿ ಕಾಪಾಡಿಕೊಳ್ಳಬೇಕು. ಅದನ್ನು ವೃಥಾ ದಂಡಿಸುವುದು ಆಸುರೀ ಪ್ರವೃತ್ತಿ. ಯಾವಾಗಲೂ ಸಂತೋಷಚಿತ್ತರಾಗಿರಿ. ವ್ಯಸನಕರವಾದ ಭಾವನೆ ಬಂದರೆ ಅದನ್ನು ಆಚೆಗೆ ತಳ್ಳಿ. ಯೋಗಿಯು ಹೆಚ್ಚು ಊಟ ಮಾಡಬಾರದು.ಆದರೆ ಅವನು ಉಪವಾಸವನ್ನೂ ಮಾಡಬಾರದು. ಅವನು ಹೆಚ್ಚು ನಿದ್ರೆ ಮಾಡಬಾರದು, ಆದರೆ ನಿದ್ರೆಯಿಲ್ಲದೆ ಇರಲೂ ಕೂಡದು. ಯಾರು ಸುವರ್ಣ ಮಧ್ಯಮಮಾರ್ಗವನ್ನು ಆಹಾರ, ವಿಹಾರ, ನಿದ್ರೆಗಳಲ್ಲಿ ಅವಲಂಬಿಸುವರೋ ಅವರೇ ಯೋಗಿಗಳಾಗಬಲ್ಲರು.

ಯೋಗಭ್ಯಾಸಕ್ಕೆ ಯಾವುದು ಸರಿಯಾದ ಕಾಲ? ಉಷಃಕಾಲ ಮತ್ತು ಸಂಧ್ಯಾ ಸಮಯ. ಆಗ ಪ್ರಕೃತಿಯೆಲ್ಲ ಶಾಂತವಾಗಿರುವುದು. ಪ್ರಕೃತಿಯ ಸಹಾಯವನ್ನು ತೆಗೆದುಕೊಳ್ಳಿ. ಸುಖಾಸನದಲ್ಲಿ ಕುಳಿತುಕೊಳ್ಳಿ. ತಲೆ, ಭುಜ ಮತ್ತು ಪಕ್ಕೆಲುಬುಗಳು ನೇರವಾಗಿರಲಿ. ಬೆನ್ನೆಲುಬು ಬಾಗದೆ ಇರಲಿ. ಮುಂದಕ್ಕೋ ಅಥವಾ ಹಿಂದಕ್ಕೋ ಬಾಗಬೇಡಿ. ದೇಹದ ಪ್ರತಿಯೊಂದು ಭಾಗವೂ ಪರಿಪೂರ್ಣವಾಗಿದೆ ಎಂದು ಭಾವಿಸಿ. ಜಗತ್ತಿಗೆಲ್ಲ ಪ್ರೀತಿಯ ತರಂಗಗಳನ್ನು ಕಳುಹಿಸಿ. ಅನಂತರ ಜ್ಞಾನವನ್ನು ನೀಡೆಂದು ಭಗವಂತನನ್ನು ಪ್ರಾರ್ಥಿಸಿ. ಕೊನೆಗೆ ನಿಮ್ಮ ಮನಸ್ಸನ್ನು ಉಸಿರಿನೊಡನೆ ಸೇರಿಸಿ ಅದರ ಚಲನವಲನಗಳ ಮೇಲೆ ನಿಮ್ಮ ಮನಸ್ಸನ್ನು ಏಕಾಗ್ರಮಾಡಿ. ಕ್ರಮೇಣ ನಿಮಗೆ ಇದರ ಪ್ರಯೋಜನ ಗೊತ್ತಾಗುವುದು.

\begin{center}
\textbf{ಓಜಸ್ಸು}
\end{center}

ಒಬ್ಬನಿಗೂ ಮತ್ತೊಬ್ಬನಿಗೂ ಇರುವ ವ್ಯತ್ಯಾಸವು ಓಜಸ್ಸಿನಲ್ಲಿದೆ. ಯಾರಲ್ಲಿ ಹೆಚ್ಚು ಓಜಸ್ಸು ಇದೆಯೋ ಅವನೇ ಜನರ ನಾಯಕನಾಗುವನು. ಇದು ಜನರನ್ನು ಆಕರ್ಷಿಸುವ ಅದ್ಭುತ ಶಕ್ತಿಯನ್ನು ಕೊಡುವುದು. ಓಜಸ್ಸು ನರಗಳ ಶಕ್ತಿಯ ಪ್ರವಾಹದಿಂದ \enginline{(Nerve Current)} ಉತ್ಪನ್ನವಾಗುತ್ತದೆ. ಯಾವುದು ಲೈಂಗಿಕ ಶಕ್ತಿಯಾಗಿ ವ್ಯಕ್ತವಾಗುವುದೋ ಅದರ ಮೂಲಕ ಓಜಸ್ಸು ಬಹಳ ಸುಲಭವಾಗಿ ತಯಾರಾಗುವುದು. ಲೈಂಗಿಕ ಕ್ರಿಯೆಗಳ ಕೇಂದ್ರಗಳ ಶಕ್ತಿಯನ್ನು ವ್ಯರ್ಥಮಾಡದೇ ಇದ್ದರೆ ಅದನ್ನು ಓಜಸ್ಸಾಗಿ ಪರಿವರ್ತನಗೊಳಿಸಬಹುದು. ದೇಹದ ಎರಡು ನರಗಳ ಪ್ರವಾಹವು ಮಿದುಳಿನಿಂದ ಹೊರಟು ಬೆನ್ನೆಲುಬಿನ ಎರಡು ಕಡೆಗಳಲ್ಲಿ ಹರಿದು ಹೋಗುತ್ತವೆ. ಆದರೆ ತಲೆಯ ಹಿಂದೆ “೮” ರ ಆಕಾರದಲ್ಲಿ ಒಂದನೊಂದು ದಾಟುವುವು. ಆದಕಾರಣವೇ ದೇಹದ ಎಡಭಾಗವು ತಲೆಯ ಬಲಭಾಗದ ಅಧೀನದಲ್ಲಿದೆ. ಬೆನ್ನುಮೂಳೆಯ ಕೆಳಭಾಗದಲ್ಲಿ ಜನನೇಂದ್ರಿಯದ ಕೇಂದ್ರವಿದೆ \enginline{(sacral Plexus)}. ಎರಡು ನರಗಳ ಮೂಲಕ ಬರುವ ನರಗಳ ಶಕ್ತಿಯ ಬಹುಭಾಗ ಈ ಕೇಂದ್ರದಲ್ಲಿ ಸಂಗ್ರಹವಾಗಿರುವುದು. ಬೆನ್ನೆಲುಬಿನ ಕೊನೆಯ ಮೂಳೆ ಜನನೇಂದ್ರಿಯ ಕೇಂದ್ರದ ಮೇಲುಭಾಗದಲ್ಲಿದೆ. ಇದನ್ನು ಸಾಂಕೇತಿಕವಾಗಿ ತ್ರಿಕೋಣಾಕಾರದಲ್ಲಿದೆ ಎನ್ನುವರು. ಇಲ್ಲಿ ಶಕ್ತಿ ಸಂಗ್ರಹ ವಾಗಿರುವುದರಿಂದ ಇದನ್ನು ಒಂದು ಸರ್ಪವೆಂದು ಹೇಳುವರು. ಪ್ರಜ್ಞೆ ಮತ್ತು ಅಪ್ರಜ್ಞೆ ಈ ನರಗಳ ಮೂಲಕ ಕೆಲಸ ಮಾಡುತ್ತಿರುವುವು. ಆದರೆ ಅತಿಪ್ರಜ್ಞೆಯು, ನರಗಳ ಪ್ರವಾಹವು ಕೆಳಗಿನ ಕೇಂದ್ರವನ್ನು ಮುಟ್ಟಿದಾಗ ಅದನ್ನು ಅಲ್ಲೆ ತಡೆದು, ಪುನಃ ಎಲ್ಲಿಂದ ಬಂದಿತೋ ಅಲ್ಲಿಗೆ ಹೋಗದಂತೆ ಮಾಡಿ, ಅದನ್ನು ಜನನೇಂದ್ರಿಯ ಕೇಂದ್ರದಿಂದ ಓಜಸ್ಸಿನ ರೂಪದಲ್ಲಿ ಮಾಡಿ ಬೆನ್ನುಮೂಳೆಯ ಮೇಲಿನ ಭಾಗಕ್ಕೆ ಹೋಗುವಂತೆ ಮಾಡುವುದು. ಬೆನ್ನಿನ ನರವು ಸಾಧಾರಣವಾಗಿಪ್ರಾರಂಭವಾಗುವುದು. ಧರ್ಮ ಎಂದಿಗೂ ಬುದ್ಧಿಯ ಕ್ಷೇತ್ರದಲ್ಲಿ ಇರಲಿಲ್ಲ. ಇಂದ್ರಿಯಗಳಿಗೆ ಕಾಣಿಸುವ ವಸ್ತುಗಳನ್ನು ಮಾತ್ರ ಬುದ್ಧಿಯ ಕ್ಷೇತ್ರವು ಗ್ರಹಿಸಬಲ್ಲುದು. ಧರ್ಮಕ್ಕೂ ಇಂದ್ರಿಯಗಳಿಗೂ ಯಾವ ಸಂಬಂಧವೂ ಇಲ್ಲ. ಅಜ್ಞೇಯತಾವಾದಿಗಳು ದೇವರನ್ನು ಅರಿಯುವುದಕ್ಕೆ ಆಗುವುದಿಲ್ಲ ಎನ್ನುವರು. ಅವರು ಹೇಳುವುದು ನಿಜವೇ. ಏಕೆಂದರೆ ಅವರು ಇಂದ್ರಿಯಗಳ ಪರಿಧಿಯಲ್ಲೆಲ್ಲಾ ಸುತ್ತಾಡಿ ಮುಗಿಸಿರುವರು. ಆದರೂ ದೇವರಿಗೆ ಸಂಬಂಧಪಟ್ಟ ವಿಷಯ ತಿಳಿಯುತ್ತಿಲ್ಲ. ಆದುದರಿಂದ ಧರ್ಮ, ದೇವರು, ಅಮರತ್ವ ಮುಂತಾದುವನ್ನು ಪ್ರಮಾಣಪಡಿಸಬೇಕಾದರೆ ಒಬ್ಬನು ಇಂದ್ರಿಯಾತೀತನಾಗಿ ಹೋಗಬೇಕು. ಎಲ್ಲಾ ಮಹಾತ್ಮರೂ ಋಷಿಗಳೂ ದೇವರನ್ನು ಕಂಡಿರುವೆವು ಎಂದು ಹೇಳುತ್ತಾರೆ. ಎಂದರೆ ಅವರಿಗೆ ಪ್ರತ್ಯಕ್ಷಾನುಭವ ದೊರೆತಿತ್ತು. ಅನುಭವವಿಲ್ಲದೆ ಯಾವ ಜ್ಞಾನವೂ ಇಲ್ಲ. ಮಾನವನು ತನ್ನಾತ್ಮನಲ್ಲಿಯೇ ದೇವರನ್ನು ಕಾಣಬೇಕಾಗಿದೆ. ಜಗತ್ತಿನಲ್ಲೆಲ್ಲಾ ಸತ್ಯಸ್ಯ ಸತ್ಯವಾಗಿರುವ ಭಗವಂತನನ್ನು ಕಂಡಾಗ ಮಾತ್ರ ಅವನ ಸಂಶಯಗಳು ನಾಶವಾಗುವುವು; ಹೃದಯದ ವಕ್ರತೆ ನೇರವಾಗುವುದು. ದೇವರನ್ನು ನೋಡುವುದು ಎಂದರೆ ಇದೇ. ನಾವು ಇದನ್ನು ಪರೀಕ್ಷೆ ಮಾಡಿ ನೋಡಬೇಕು; ಸುಮ್ಮನೆ ಹೇಳಿದ್ದನ್ನೆಲ್ಲಾ ಕೇಳುವುದಲ್ಲ. ಧರ್ಮವು ಇತರ ವಿಜ್ಞಾನಗಳಂತೆ ವಿಷಯಗಳನ್ನು ಸಂಗ್ರಹಿಸಿ ಸತ್ಯವನ್ನು ನೀವೇ ಕಾಣುವಂತೆ ಮಾಡಬೇಕು. ಪಂಚೇಂದ್ರಿಯಗಳಿಗೆ ಅತೀತವಾಗಿ ಹೋದಾಗ ಮಾತ್ರ ಅದು ಸಾಧ್ಯ. ಧಾರ್ಮಿಕ ಸತ್ಯಗಳನ್ನು ಪ್ರತಿಯೊಬ್ಬರೂ ಒರೆಗೆ ಹಚ್ಚಿ ನೋಡಬೇಕು. ಭಗವಂತನನ್ನು ನೋಡುವುದೇ ಏಕಮಾತ್ರ ಗುರಿ. ಶಕ್ತಿಯನ್ನು ಪಡೆಯುವುದು ಗುರಿಯಲ್ಲ. ಸಚ್ಚಿದಾನಂದವೇ ದೇವರು, ದೇವರೇ ಪರಮಪ್ರೇಮ. ಮುಚ್ಚಿ ಹೋಗಿರುವುದು. ಆದರೆ ಓಜಸ್ಸು ಹರಿದು ಹೋಗುವುದಕ್ಕೆ ಇಲ್ಲಿ ಒಂದು ದಾರಿಯನ್ನು ಮಾಡಬಹುದು. ಬೆನ್ನಿನ ನರದಲ್ಲಿ ಶಕ್ತಿಯು ಒಂದು ಕೇಂದ್ರದಿಂದ ಮತ್ತೊಂದು ಕೇಂದ್ರಕ್ಕೆ ಹರಿದು ಹೋಗುವಾಗ ನೀವು ಒಂದು ಮನೋ ಭೂಮಿಕೆಯಿಂದ ಮತ್ತೊಂದು ಮನೋಭೂಮಿಕೆಗೆ ಹೋಗುವಿರಿ. ಆದ ಕಾರಣವೇ ಮನುಷ್ಯನು ಇತರ ಎಲ್ಲಾ ಪ್ರಾಣಿಗಳಿಗಿಂತಲೂ ಶ್ರೇಷ್ಠ. ಏಕೆಂದರೆ ಈ ದೇಹದಲ್ಲಿರುವ ಆತ್ಮನಿಗೆ ಎಲ್ಲಾ ಅನುಭವಗಳೂ, ಎಲ್ಲಾ ಭೂಮಿಕೆಗಳೂ ಸಾಧ್ಯ. ನಮಗೆ ಮತ್ತೊಂದು ದೇಹವೇ ಬೇಕಾಗಿಲ್ಲ. ಮಾನವನು ಬೇಕಾದರೆ ಈ ದೇಹದಲ್ಲೇ ತನ್ನ ಅನುಭವವನ್ನೆಲ್ಲಾ ಮುಗಿಸಿ ಪರಿಶುದ್ಧಾತ್ಮ ನಾಗಬಹುದು. ಓಜಸ್ಸು ಮೇಲೆ ಮೇಲಕ್ಕೆ ಕೇಂದ್ರದಿಂದ ಕೇಂದ್ರಕ್ಕೆ ಹೋಗಿ ಕೊನೆಗೆ ಸಹಸ್ರಾರನ್ನು ಸೇರುವುದು. ಆಗ ಮನುಷ್ಯನು ದೇಹಕ್ಕೆ ಮತ್ತು ಮನಸ್ಸಿಗೆ ಅತೀತನಾಗಿ ಹೋಗುವನು. ಅವನು ಎಲ್ಲಾ ಬಂಧನಗಳಿಂದ ಮುಕ್ತನಾಗುವನು.

ಮನುಷ್ಯನಿಗೆ ಅದ್ಭುತ ಶಕ್ತಿಗಳು ದೊರಕುವುದು ಒಂದು ಅಪಾಯವೇ; ಮನುಷ್ಯ ಅವನ್ನು ಹೇಗೆ ಸರಿಯಾಗಿ ಉಪಯೋಗಿಸಬೇಕೋ, ಅವನಿಗೆ ತಿಳಿಯದು. ಅವನಿಗೆ ಏನು ಬಂದಿದೆಯೋ ಅದು ಹೇಗೆ ಬಂತು ಎಂಬುದು ಗೊತ್ತಿರುವುದಿಲ್ಲ. ಅದಕ್ಕೆ ಅವನು ಸಿದ್ಧನಾಗಿರುವುದಿಲ್ಲ. ಈ ಅದ್ಭುತಶಕ್ತಿಗಳನ್ನು ಉಪಯೋಗಿಸುವಾಗ ಒಬ್ಬನ ಕಾಮ ಅತಿಯಾಗಿ ಉದ್ರೇಕವಾಗುವುದು. ಏಕೆಂದರೆ ಈ ಶಕ್ತಿಗಳು ಜನನೇಂದ್ರಿಯದ ಕೇಂದ್ರದಿಂದ ಉತ್ಪತ್ತಿಯಾಗಿವೆ. ಈ ಅದ್ಭುತ ಶಕ್ತಿಗಳನ್ನು ತ್ಯಜಿಸುವುದೇ ಶ್ರೇಯಸ್ಕರವಾದ ಮತ್ತು ಸರಿಯಾದ ಮಾರ್ಗ. ಸರಿಯಾಗಿ ಸಾಧನೆಮಾಡದೆ ಅವನ್ನು ಪಡೆದ ಅಜ್ಞರಿಗೆ ಅವು ಹೇರಳವಾಗಿ ತೊಂದರೆಗಳನ್ನು ತಂದೊಡ್ಡುತ್ತವೆ.

ಪುನಃ ಸಂಕೇತಗಳಿಗೆ ಹೋಗೋಣ. ಓಜಸ್ಸು ಬೆನ್ನುಮೂಳೆಯಲ್ಲಿ ಸುರುಳಿ ಸುತ್ತಿಕೊಂಡು ಮೇಲೆ ಹೋಗುವುದರಿಂದ ಅದನ್ನು ಸರ್ಪ ಎನ್ನುವರು. ಈ ಸರ್ಪವು ತ್ರಿಕೋಣ ಅಥವಾ ಮೂಳೆಯ ಮೇಲೆ ನಿಂತಿರುವುದು. ಅದು ಜಾಗೃತವಾಗಿ ಚಕ್ರದಿಂದ ಚಕ್ರಕ್ಕೆ ಮೇಲೆ ಹೋಗುವಾಗ ನಮಗೆ ಹೊಸ ಸಹಜ ಪ್ರಪಂಚವೊಂದು ಒಳಗೆ ತೆರೆದಂತೆ ಆಗುವುದು. ಅದನ್ನು ಕುಂಡಲಿನಿಯ ಜಾಗೃತಿ ಎನ್ನುವರು.

\begin{center}
\textbf{ಆದ್ಯಾತ್ಮ}
\end{center}

ಪಶ್ಚಾತ್ತಾಪ ಪಡು, ಏಕೆಂದರೆ ಸ್ವರ್ಗ ನಿನ್ನ ಸಮೀಪದಲ್ಲಿಯೇ ಇರುವುದು. \enginline{Repent} ಎಂಬ ಪದವು ಗ್ರೀಸಿನ \enginline{Metanoeite} ಎಂಬ ಪದದಿಂದ ಬಂದಿದೆ (\enginline{Metanoeite} ಎಂದರೆ ಅತೀತವಾಗಿ ಹೋಗು ಎಂದು). ನಿಜವಾದ ಅರ್ಥ ಜ್ಞಾನಕ್ಕೆ ಅತೀತವಾಗಿ ಹೋಗು ಅಥವಾ–ಪಂಚೇಂದ್ರಿಯಗಳ ಜ್ಞಾನಕ್ಕೆ ಅತೀತವಾಗಿ ಹೋಗು ಎಂದು. ಆಂತರ್ಯದಲ್ಲಿ ನೋಡಿ, ನಿಮ್ಮ ಪರಂಧಾಮ ಅಲ್ಲಿಯೇ ಇರುವುದು.

ಸರ್​ ವಿಲಿಯಮ್​ ಹ್ಯಾಮಿಲ್​ಟನ್​ ಎಂಬುವನು ಒಂದು ತತ್ತ್ವಗ್ರಂಥದ ಕೊನೆಯಲ್ಲಿ ಹೀಗೆ ಹೇಳುತ್ತಾನೆ: ಅಲ್ಲಿ ತತ್ತ್ವವು ಕೊನೆಗಾಣುವುದು, ಅಲ್ಲಿ ಧರ್ಮ ಪ್ರಾರಂಭವಾಗುವುದು. ಧರ್ಮ ಎಂದಿಗೂ ಬುದ್ಧಿಯ ಕ್ಷೇತ್ರದಲ್ಲಿ ಇರಲಿಲ್ಲ. ಇಂದ್ರಿಯಗಳಿಗೆ ಕಾಣಿಸುವ ವಸ್ತುಗಳನ್ನು ಮಾತ್ರ ಬುದ್ಧಿಯ ಕ್ಷೇತ್ರವು ಗ್ರಹಿಸಬಲ್ಲುದು. ಧರ್ಮಕ್ಕೂ ಇಂದ್ರಿಯಗಳಿಗೂ ಯಾವ ಸಂಬಂಧವೂ ಇಲ್ಲ. ಅಜ್ಞೇಯತಾವಾದಿಗಳು ದೇವರನ್ನು ಅರಿಯುವುದಕ್ಕೆ ಆಗುವುದಿಲ್ಲ ಎನ್ನುವರು. ಅವರು ಹೇಳುವುದು ನಿಜವೇ. ಏಕೆಂದರೆ ಅವರು ಇಂದ್ರಿಯಗಳ ಪರಿಧಿಯಲ್ಲೆಲ್ಲಾ ಸುತ್ತಾಡಿ ಮುಗಿಸಿರುವರು. ಆದರೂ ದೇವರಿಗೆ ಸಂಬಂಧಪಟ್ಟ ವಿಷಯ ತಿಳಿಯುತ್ತಿಲ್ಲ. ಆದುದರಿಂದ ಧರ್ಮ, ದೇವರು, ಅಮರತ್ವ ಮುಂತಾದುವನ್ನು ಪ್ರಮಾಣಪಡಿಸಬೇಕಾದರೆ ಒಬ್ಬನು ಇಂದ್ರಿಯಾತೀತನಾಗಿ ಹೋಗಬೇಕು. ಎಲ್ಲಾ ಮಹಾತ್ಮರೂ ಋಷಿಗಳೂ ದೇವರನ್ನು ಕಂಡಿರುವೆವು ಎಂದು ಹೇಳುತ್ತಾರೆ. ಎಂದರೆ ಅವರಿಗೆ ಪ್ರತ್ಯಕ್ಷಾನುಭವ ದೊರೆತಿತ್ತು. ಅನುಭವವಿಲ್ಲದೆ ಯಾವ ಜ್ಞಾನವೂ ಇಲ್ಲ. ಮಾನವನು ತನ್ನಾತ್ಮನಲ್ಲಿಯೇ ದೇವರನ್ನು ಕಾಣಬೇಕಾಗಿದೆ. ಜಗತ್ತಿನಲ್ಲೆಲ್ಲಾ ಸತ್ಯಸ್ಯ ಸತ್ಯವಾಗಿರುವ ಭಗವಂತನನ್ನು ಕಂಡಾಗ ಮಾತ್ರ ಅವನ ಸಂಶಯಗಳು ನಾಶವಾಗುವುವು; ಹೃದಯದ ವಕ್ರತೆ ನೇರವಾಗುವುದು. ದೇವರನ್ನು ನೋಡುವುದು ಎಂದರೆ ಇದೇ. ನಾವು ಇದನ್ನು ಪರೀಕ್ಷೆ ಮಾಡಿ ನೋಡಬೇಕು; ಸುಮ್ಮನೆ ಹೇಳಿದ್ದನ್ನೆಲ್ಲಾ ಕೇಳುವುದಲ್ಲ. ಧರ್ಮವು ಇತರ ವಿಜ್ಞಾನಗಳಂತೆ ವಿಷಯಗಳನ್ನು ಸಂಗ್ರಹಿಸಿ ಸತ್ಯವನ್ನು ನೀವೇ ಕಾಣುವಂತೆ ಮಾಡಬೇಕು. ಪಂಚೇಂದ್ರಿಯಗಳಿಗೆ ಅತೀತವಾಗಿ ಹೋದಾಗ ಮಾತ್ರ ಅದು ಸಾಧ್ಯ. ಧಾರ್ಮಿಕ ಸತ್ಯಗಳನ್ನು ಪ್ರತಿಯೊಬ್ಬರೂ ಒರೆಗೆ ಹಚ್ಚಿ ನೋಡಬೇಕು. ಭಗವಂತನನ್ನು ನೋಡುವುದೇ ಏಕಮಾತ್ರ ಗುರಿ. ಶಕ್ತಿಯನ್ನು ಪಡೆಯುವುದು ಗುರಿಯಲ್ಲ. ಸಚ್ಚಿದಾನಂದವೇ ದೇವರು, ದೇವರೇ ಪರಮಪ್ರೇಮ.

\begin{center}
\textbf{ಚಿಂತನೆ, ಕಲ್ಪನೆ ಮತ್ತು ಧ್ಯಾನ}
\end{center}

ಕನಸಿನಲ್ಲಿ ಮತ್ತು ಆಲೋಚನೆಯಲ್ಲಿ ಉಪಯೋಗಿಸುವ ಕಲ್ಪನಾಶಕ್ತಿಯೇ ಸತ್ಯಸಾಕ್ಷಾತ್ಕಾರಕ್ಕೂ ಸಹಾಯಕವಾಗುವುದು. ಕಲ್ಪನೆ ತೀವ್ರವಾದಾಗ ವಸ್ತು ಕಣ್ಣಿಗೆ ಕಾಣಿಸುವುದು. ಇದರ ಮೂಲಕ ನಾವು ದೇಹವನ್ನು ಆರೋಗ್ಯವುಳ್ಳದ್ದಾಗಿ ಅಥವಾ ಅನಾರೋಗ್ಯವುಳ್ಳದ್ದಾಗಿ ಬೇಕಾದರೂ ಮಾಡಬಹುದು. ನಾವು ಒಂದು ವಸ್ತುವನ್ನು ನೋಡಿದಾಗ ಮಿದುಳಿನ ಕಣಗಳು ಕಲೈಡೋಸ್ಕೋಪಿನಲ್ಲಿ ಕಾಣುವ ಚಿತ್ರದಂತೆ ಒಂದು ರೂಪವನ್ನು ಧರಿಸುವುವು; ಈ ಕಣಗಳ ರೂಪವನ್ನು ಪುನಃ ಮನಸ್ಸಿನಲ್ಲಿ ತರುವುದೇ ಸ್ಮೃತಿ. ನಮ್ಮ ಇಚ್ಛಾಶಕ್ತಿ ಬಲವಾಗಿದ್ದಷ್ಟೂ ಆ ರಚನೆ ಹೆಚ್ಚು ಸ್ಪಷ್ಟವಾಗಿ ಕಾಣಿಸುವಂತೆ ಮಾಡಬಹುದು. ದೇಹವನ್ನು ಗುಣಪಡಿಸುವ ಒಂದು ಶಕ್ತಿ ಪ್ರತಿಯೊಬ್ಬನಲ್ಲಿಯೂ ಇರುವುದು. ಔಷಧಿ ಈ ಗುಣವನ್ನು ಜಾಗೃತ ಗೊಳಿಸುವುದು, ಅಷ್ಟೆ. ದೇಹಕ್ಕೆ ಧಾಳಿಯಿಟ್ಟಿರುವ ವಿಷಯವನ್ನು ಹೊರಗೆ ಕಳುಹಿಸುವ ಪ್ರಯತ್ನವೇ ಅನಾರೋಗ್ಯ. ಔಷಧಿಗಳಿಗೆ ದೇಹದಲ್ಲಿರುವ ವಿಷಯವನ್ನು ಹೊರಕ್ಕೆ ಹಾಕುವ ಶಕ್ತಿಯನ್ನು ಜಾಗೃತಗೊಳಿಸುವ ಬಲ ಇರುವುದಾದರೂ, ಆ ಕೆಲಸವನ್ನು ಚಿಂತನೆಯ ಶಕ್ತಿಯಿಂದ ಶಾಶ್ವತ ಗೊಳಿಸಬಹುದು. ಕಾಯಿಲೆ ಬಂದಾಗ ಆರೋಗ್ಯದ ಅತ್ಯುನ್ನತ ಸ್ಥಿತಿಯ ಸ್ಮೃತಿಯನ್ನು ಜಾಗೃತಗೊಳಿಸಬೇಕು, ಮತ್ತು ಆರೋಗ್ಯವಾಗಿದ್ದಾಗ ಕಣಗಳು ಯಾವ ಸ್ಥಿತಿಯಲ್ಲಿದ್ದುವೋ ಆ ಸ್ಥಿತಿಗೆ ಅವು ಮರುಳುವಂತೆ ಮಾಡುವುದಕ್ಕಾಗಿ ಆರೋಗ್ಯ ಮತ್ತು ಶಕ್ತಿಯ ಭಾವನೆಗಳನ್ನು ಕಲ್ಪನೆಯು ಬಲವಾಗಿ ಹಿಡಿದಿರಬೇಕು. ಆಗ ದೇಹದ ಪ್ರವೃತ್ತಿಯು ಮಿದುಳಿನ ಪ್ರವೃತ್ತಿಯನ್ನು ಅನುಸರಿಸುತ್ತದೆ.

ಎರಡನೆಯ ಮೆಟ್ಟಿಲು ಬೇರೆಯವರ ಮನಸ್ಸು ನಮ್ಮ ಮೇಲೆ ಕೆಲಸ ಮಾಡು ತ್ತಿರುವುದರ ಮೂಲಕ ಮೇಲಿನ ಪರಿಣಾಮವನ್ನು ಪಡೆಯುವುದು. ಪ್ರತಿದಿನ ಇದಕ್ಕೆ ಉದಾಹರಣೆಗಳನ್ನು ನೋಡುತ್ತಿರಬಹುದು. ಮಾತುಗಳು ಕೇವಲ ಒಂದು ಮನಸ್ಸು ಮತ್ತೊಂದು ಮನಸ್ಸಿನ ಮೇಲೆ ಕೆಲಸಮಾಡುತ್ತಿರುವ ಒಂದು ವಿಧಾನ ಅಷ್ಟೆ. ಒಳ್ಳೆಯ ಮತ್ತು ಕೆಟ್ಟ ಆಲೋಚನೆಗಳೆರಡರಲ್ಲಿಯೂ ಶಕ್ತಿ ಸುಪ್ತವಾಗಿದೆ, ಇವು ಪ್ರಪಂಚವನ್ನೆಲ್ಲ ತುಂಬುವವು. ಚಿಂತನೆಗಳು ಕ್ರಿಯಾರೂಪಕ್ಕೆ ಬರುವ ವರೆಗೂ ಚಿಂತನೆಗಳಾಗಿಯೇ ಉಳಿದಿರುತ್ತವೆ. ಉದಾಹರಣೆಗೆ ಒಬ್ಬನು ಏಟನ್ನು ಕೊಡುವವರೆಗೆ ಎಂದರೆ, ಶಕ್ತಿಯನ್ನು ಕ್ರಿಯೆಯಾಗಿ ಮಾರ್ಪಡಿಸುವವರೆಗೆ ಅವನ ಬಾಹುಗಳಲ್ಲಿ ಶಕ್ತಿಯು ಸುಪ್ತವಾಗಿರುತ್ತದೆ. ನಾವು ಒಳ್ಳೆಯ ಮತ್ತು ಕೆಟ್ಟ ಆಲೋಚನೆಗಳಿಗೆ ಹಕ್ಕುದಾರರು. ನಾವು ಪರಿಶುದ್ಧರಾಗಿ ಒಳ್ಳೆಯ ಆಲೋಚನೆ ಗಳನ್ನು ಸ್ವೀಕರಿಸುವ ಉಪಕರಣಗಳಾದರೆ ಆಗ ಒಳ್ಳೆಯ ಆಲೋಚನೆಗಳು ನಮ್ಮೊಳಗೆ ಪ್ರವೇಶಿಸುವುವು. ಒಳ್ಳೆಯ ವ್ಯಕ್ತಿಯು ಕೆಟ್ಟ ಆಲೋಚನೆಯನ್ನು ಸ್ವೀಕರಿಸುವುದಿಲ್ಲ. ಕೆಟ್ಟವರಲ್ಲಿ ಕೆಟ್ಟ ಆಲೋಚನೆಗಳು ಒಳ್ಳೆಯ ಕ್ಷೇತ್ರವನ್ನೆ ಪಡೆಯುತ್ತವೆ. ಅವು ವಿಷದ ಕ್ರಿಮಿಗಳಂತೆ ತಕ್ಕ ವಾತಾವರಣ ಸಿಕ್ಕಿದರೆ ಅಭಿವೃದ್ಧಿಯಾಗುವುವು. ಬರಿಯ ಆಲೋಚನೆಗಳು ಕಿರಿಯ ಅಲೆಗಳಂತೆ. ಅವುಗಳು ಚಲಿಸುವಂತೆ ಮಾಡಲು ಹೊಸ ಪ್ರೇರಣೆಗಳು ಏಕಕಾಲದಲ್ಲಿ ಬರುತ್ತವೆ. ಕೊನೆಗೆ ದೊಡ್ಡದೊಂದು ಅಲೆಯೆದ್ದು ಕಿರಿಯ ಅಲೆಗಳನ್ನೆಲ್ಲ ಆಪೋಶನ ತೆಗೆದುಕೊಳ್ಳುವುದು. ಈ ವಿಶ್ವದ ಭಾವತರಂಗಗಳು ಐನೂರು ವರುಷಕ್ಕೆ ಒಮ್ಮೆ ಏಳುವಂತೆ ತೋರುವುವು. ಆಗ ಆ ಮಹದಲೆ ಉಳಿದ ಕಿರಿ ಅಲೆಗಳನ್ನೆಲ್ಲ ತನ್ನಲ್ಲಿ ಅಡಗಿಸಿಕೊಳ್ಳುವುದು. ದೊಡ್ಡ ಅಲೆಯನ್ನೇ ದೇವದೂತ ಎನ್ನುವುದು. ಅವನು ತನ್ನ ಮನಸ್ಸಿನಲ್ಲಿ ಆ ಯುಗದ ಆಲೋಚನೆಯನ್ನು ಕೇಂದ್ರೀಕರಿಸಿ ಅದನ್ನು ಸ್ಥೂಲರೂಪದಲ್ಲಿ ಮಾನವ ಕೋಟಿಗೆ ಕೊಡುವನು. ಕೃಷ್ಣ, ಬುದ್ಧ, ಕ್ರಿಸ್ತ, ಮಹಮ್ಮದ್​, ಲೂಥರ್​ ಮುಂತಾದವರನ್ನು ತಮ್ಮ ಸಮಕಾಲೀನ ಅಲೆಗಳ ನಡುವೆ ಮೇಲೆದ್ದ ಮಹದಲೆಗಳಂತೆ ಪರಿಗಣಿಸಬಹುದು. (ಅವರ ಮಧ್ಯದಲ್ಲಿ ಸುಮಾರು ಐನೂರು ವರುಷಗಳ ಅಂತರವಿರುವುದು ಕಂಟುಬರುತ್ತದೆ.) ಯಾವ ಅಲೆಯ ಹಿಂದೆ ಅತ್ಯಂತ ಪವಿತ್ರತೆ ಇದೆಯೋ, ಶುದ್ಧ ಚಾರಿತ್ರ್ಯವಿದೆಯೋ, ಅದೇ ಒಂದು ಮಹಾಸುಧಾರಣೆಯಂತೆ ಸಮಾಜದ ಮೇಲೆ ಬೀಳುವುದು. ಪುನಃ ನಮ್ಮ ಕಾಲದಲ್ಲೇ ಒಂದು ಭಾವತರಂಗದ ಸ್ಪಂದನವಿದೆ. ಅದರ ಮುಖ್ಯ ಭಾವನೆಯೇ ಪ್ರಪಂಚದಲ್ಲಿ ಓತಪ್ರೋತನಾದ ದೇವರು. ಅದು ಎಲ್ಲಾ ಕಡೆಗಳಲ್ಲಿಯೂ ಹಲವು ರೂಪಗಳಲ್ಲಿ ಹಲವು ಪಂಗಡಗಳಲ್ಲಿ ಕಾಣಿಸುತ್ತಿದೆ. ಈ ಅಲೆಗಳಲ್ಲಿ ಒಮ್ಮೆ ಸೃಷ್ಟಿಪರವಾದ ಭಾವನೆಗಳು ಮತ್ತೊಮ್ಮೆ ವಿನಾಶಕರ ಭಾವನೆಗಳು ಬರುತ್ತಿವೆ. ಆದರೂ ಧ್ವಂಸಕಾರ್ಯವನ್ನು ಸೃಷ್ಟಿಕಾರ್ಯವು ಕೊನೆಗಾಣಿಸುತ್ತಿದೆ. ಮಾನವ ತನ್ನ ಆಧ್ಯಾತ್ಮಿಕ ಸ್ವಭಾವವನ್ನು ಅರಿಯಲು ಅಂತರಾಳಕ್ಕೆ ಹೋದಂತೆಲ್ಲ ಮೂಢನಂಬಿಕೆಗಳ ಬಂಧನದಿಂದ ಬಿಡುಗಡೆ ಹೊಂದುತ್ತಾನೆ. ಬಹುಪಾಲು ಪಂಗಡಗಳೆಲ್ಲ ಕ್ಷಣಿಕವಾಗಿರುತ್ತವೆ, ನೀರಿನ ಗುಳ್ಳೆಗಳಂತೆ ಎದ್ದು ಮಾಯವಾಗುತ್ತವೆ. ಏಕೆಂದರೆ ಅವುಗಳ ಮುಂದಾಳುಗಳಲ್ಲಿ ಚಾರಿತ್ರ್ಯಶುದ್ಧಿಯುಳ್ಳವರು ಅಪರೂಪ. ಎಂದಿಗೂ ಮತ್ತೊಬ್ಬರಿಗೆ ತೊಂದರೆಯನ್ನು ಬಯಸದ ಪೂರ್ಣ ಪ್ರೇಮವೇ ಚಾರಿತ್ರ್ಯ ಶುದ್ಧಿಯನ್ನು ರೂಢಿಸಬಲ್ಲದು. ನಾಯಕನಲ್ಲಿ ಚಾರಿತ್ರ್ಯಶುದ್ಧಿ ಇಲ್ಲದೇ ಇದ್ದರೆ ಯಾರೂ ಅವನಿಗೆ ವಿಧೇಯರಾಗಿರುವುದಿಲ್ಲ. ಎಲ್ಲಿ ಚಾರಿತ್ರ್ಯಶುದ್ಧಿ ಇದೆಯೋ ಅಲ್ಲಿ ಜನರು ಎಂದೆಂದಿಗೂ ವಿಧೇಯರಾಗಿರುವರು; ಜನರು ಅವನಲ್ಲಿ ನಂಬಿಕೆಯನ್ನು ಇಡುವರು.

ಒಂದು ಆದರ್ಶವನ್ನು ತೆಗೆದುಕೊಂಡು ಸಾಧನೆಮಾಡಿ, ಸಹನೆಯಿಂದ ಹೋರಾಡಿ, ಕೊನೆಗೆ ನಿಮಗೆ ಬೆಳಕು ಬರುವುದು.

\delimiter

ನಾವು ಪುನಃ ಕಲ್ಪನೆಗೆ ಹಿಂತಿರುಗೋಣ. ನಾವು ಕುಂಡಲಿನಿಯನ್ನು ಕಲ್ಪಿಸಿ ಕೊಳ್ಳಬೇಕಾಗಿದೆ. ಮೂಲಾಧಾರದ ತ್ರಿಕೋಣದಲ್ಲಿ ಸರ್ಪ ಮುದುರಿಕೊಂಡಿದೆ ಎಂದು ಕುಂಡಲಿನಿಯನ್ನು ಊಹಿಸಬೇಕು.

ಅನಂತರ ಹಿಂದೆ ಹೇಳಿದಂತೆ ಶ್ವಾಸೋಚ್ಫ್ವಾಸಗಳನ್ನು ಮಾಡಿ. ಉಸಿರನ್ನು ದೇಹದ ಒಳಗೆ ನಿಲ್ಲಿಸಿರುವಾಗ, ಅದು ಬೆನ್ನುಮೂಳೆಯ ಕೆಳಗೆ ಹೋಗುತ್ತಿದೆ ಎಂದೂ, ಅದು ತ್ರಿಕೋಣವನ್ನು ಮುಟ್ಟಿದಾಗ ಅಲ್ಲಿ ಸುಪ್ತವಾಗಿರುವ ಸರ್ಪವನ್ನು ಜಾಗೃತಗೊಳಿಸುತ್ತಿದೆ ಎಂದೂ, ಆ ಸರ್ಪ ಮೇಲುಮೇಲಕ್ಕೆ ಹೋಗುತ್ತಿದೆ ಎಂದೂ ಭಾವಿಸಿ. ಉಸಿರನ್ನು ಆಲೋಚನೆಯ ಮೂಲಕ ಮೂಲಾಧಾರಕ್ಕೆ ಕಳುಹಿಸಿ.

ಇಲ್ಲಿಯವರೆಗೆ ದೇಹಕ್ಕೆ ಸಂಬಂಧಪಟ್ಟದ್ದು ಆಯಿತು. ಇಲ್ಲಿಂದ ಮುಂದೆ ಅದು ಮಾನಸಿಕವಾಗುವುದು.

ಮೊದಲನೆಯದೇ ಪ್ರತ್ಯಾಹಾರ. ಚಂಚಲವಾಗಿರುವ ಮನಸ್ಸನ್ನು ಒಂದು ಕಡೆ ಕೇಂದ್ರೀಕರಿಸಬೇಕು.

ದೇಹಕ್ಕೆ ಸಂಬಂಧಪಟ್ಟದ್ದನ್ನು ಮಾಡಿದಮೇಲೆ ಮನಸ್ಸನ್ನು ಸ್ವಚ್ಛಂದವಾಗಿ ಹರಿಯಲು ಬಿಡಿ, ಅದನ್ನು ನಿಗ್ರಹಿಸಬೇಡಿ. ಆದರೆ ಮನಸ್ಸು ಮಾಡುವುದನ್ನೆಲ್ಲ ಪರೀಕ್ಷಿಸುತ್ತಿರಿ. ಸಾಕ್ಷಿಯಂತೆ ಇರಿ. ಇಲ್ಲಿ ಮನಸ್ಸು ಇಬ್ಭಾಗವಾಗುವುದು; ಒಂದು ಭಾಗ ಆಡುತ್ತಿರುವುದು, ಮತ್ತೊಂದು ಭಾಗ ಆಡುವ ಭಾಗವನ್ನು ನೋಡುತ್ತಿರು ವುದು. ಸಾಕ್ಷಿಯಾಗಿರುವುದನ್ನು ಬಲಗೊಳಿಸಿ.ಚಂಚಲವಾಗಿ ಓಡುತ್ತಿರುವ ಮನಸ್ಸಿನ ಕಡೆಗೆ ಅಷ್ಟು ಗಮನಕೊಡಬೇಡಿ. ಮನಸ್ಸು ಆಲೋಚಿಸಬೇಕಾಗಿದೆ. ನಿಧಾನವಾಗಿ ಕ್ರಮೇಣ ಸಾಕ್ಷಿ ತನ್ನ ಪಾತ್ರವನ್ನು ನಿರ್ವಹಿಸಿದಂತೆ, ಆಡುವವನು ಹೆಚ್ಚು ಹೆಚ್ಚು ಸ್ವಾಧೀನತೆಗೆ ಒಳಗಾಗುತ್ತಾ ಬರುವನು. ಕೊನೆಗೆ ಮನಸ್ಸಿನ ಸಂಚಾರವೆಲ್ಲ, ಆಟವೆಲ್ಲ ನಿಲ್ಲುವುದು.

ಎರಡನೆಯದೇ ಧ್ಯಾನ. ಇದನ್ನು ಎರಡು ಭಾಗಗಳಾಗಿ ಮಾಡಬಹುದು. ನಾವೆಲ್ಲ ದೇಹಧಾರಿಗಳು. ಮನಸ್ಸು ಆಕಾರದ ಮೂಲಕ ಮಾತ್ರ ಆಲೋಚಿಸ ಬೇಕಾಗುವುದು. ಧರ್ಮವು ಅದರ ಆವಶ್ಯಕತೆಯನ್ನು ಒಪ್ಪಿಕೊಂಡು ಬಾಹ್ಯಾಚಾರ ಮುಂತಾದವುಗಳಿಗೆ ಅವಕಾಶವನ್ನು ಕಲ್ಪಿಸಿಕೊಡುವುದು. ಆಕಾರವಿಲ್ಲದೆ ನೀವು ದೇವರನ್ನು ಕುರಿತು ಧ್ಯಾನಿಸಲಾರಿರಿ. ಯಾವುದಾದರೊಂದು ಆಕಾರ ನಿಮಗೆ ಹೊಳೆಯುವುದು; ಏಕೆಂದರೆ ಆಲೋಚನೆಯನ್ನು ಮತ್ತು ಆಕಾರವನ್ನು ಬೇರ್ಪಡಿಸಲಾಗುವುದಿಲ್ಲ. ಆಕಾರದ ಮೇಲೆ ಮನಸ್ಸನ್ನು ಕೇಂದ್ರೀಕರಿಸಿ.

ಧ್ಯಾನಾಭ್ಯಾಸದಿಂದ ಇದು ನಿಜವಾಗಿ ಸಾಧ್ಯ. ಇದೇ ಮೂರನೆಯದು. ಆಗ ಮನಸ್ಸು ಏಕಾಗ್ರವಾಗುವುದು. ಮನಸ್ಸು ಸಾಧಾರಣವಾಗಿ ಒಂದು ವೃತ್ತದಲ್ಲಿ ಕೆಲಸ ಮಾಡುವುದು. ಅದನ್ನು ಒಂದು ಕೇಂದ್ರದಲ್ಲಿ ನಿಲ್ಲುವಂತೆ ಮಾಡಿ.

ಕೊನೆಯದೇ ಪ್ರತಿಫಲ. ಮನಸ್ಸು ಈ ಸ್ಥಿತಿಗೆ ಬಂದಾಗ ಗುಣಮಾಡುವ ಶಕ್ತಿ, ದೂರದಲ್ಲಿವುದನ್ನು ನೋಡುವ ಶಕ್ತಿ ಮುಂತಾದುವೆಲ್ಲ ಬರುವುವು. ಏಸು ಹೇಗೆ ತನ್ನ ಮನಸ್ಸನ್ನು ಇತರರ ಮೇಲೆ ಬೀರಿ ಅವರನ್ನು ಗುಣಪಡಿಸುತ್ತಿದ್ದನೋ ಹಾಗೆಯೇ ನಿಮ್ಮ ಮನಸ್ಸನ್ನು ಯಾರ ಮೇಲೆ ಬೇಕಾದರೂ ಬೀರಬಹುದು.

ಜನರು ಸಾಧಾರಣವಾಗಿ ಯಾವ ಪೂರ್ವಸಿದ್ಧತೆಯೂ ಇಲ್ಲದೆ ಈ ಸಿದ್ಧಿಗಳನ್ನು ಅಕಸ್ಮಾತ್ತಾಗಿ ಪಡೆದಿರುವರು. ಆದರೆ ನಿಧಾನವಾಗಿ ಮೇಲಿನ ಮಾರ್ಗಗಳನ್ನೆಲ್ಲಾ ಅಭ್ಯಾಸ ಮಾಡಿ. ಅನಂತರ ಎಲ್ಲವೂ ನಿಮ್ಮ ಸ್ವಾಧೀನವಾಗುವುದು. ಕೇವಲ ಪ್ರೀತಿಯೇ ನಿಮ್ಮ ಉದ್ದೇಶವಾಗಿದ್ದರೆ ಇತರರನ್ನು ಗುಣಪಡಿಸುವುದನ್ನು ಬೇಕಾದರೆ ಸ್ವಲ್ಪ ಯತ್ನಿಸಬಹುದು. ಇದರಿಂದ ಅಷ್ಟು ಅಪಾಯವಿಲ್ಲ. ಆದರೆ ಮನುಷ್ಯನಿಗೆ ದೂರದೃಷ್ಟಿಯಿಲ್ಲ, ತಾಳ್ಮೆಯಿಲ್ಲ, ಎಲ್ಲರಿಗೂ ಸಿದ್ಧಿಗಳು ಬೇಕೆಂದು ಆಸೆ. ಯಾರೂ ಅವನ್ನು ಸರಿಯಾಗಿ ಸಂಗ್ರಹಿಸುವುದಿಲ್ಲ. ಅವನ್ನು ಸಂಗ್ರಹಿಸುವುದಕ್ಕೆ ಮುಂಚೆಯೇ ಹಂಚುವನು. ಅವನ್ನು ಸಂಗ್ರಹಿಸುವುದಕ್ಕೆ ಬಹಳಕಾಲ ಹಿಡಿಯು ವುದು. ಆದರೆ ಅವನ್ನು ಖರ್ಚು ಮಾಡಲು ಒಂದು ಕ್ಷಣಕಾಲ ಸಾಕು. ಆದ ಕಾರಣ ಈ ಶಕ್ತಿಗಳು ಬಂದಂತೆ ಇವುಗಳನ್ನೆಲ್ಲಾ ಸಂಗ್ರಹಿಸಿಡಿ. ಅವನ್ನು ವ್ಯರ್ಥ ಮಾಡಬೇಡಿ.

ಮನಸ್ಸಿನಲ್ಲಿ ಏಳುವ ಕಾಮಕ್ರೋಧಾದಿಗಳನ್ನು ನಿಗ್ರಹಿಸಿದಷ್ಟೂ ನಿಮಗೆ ಪ್ರಯೋಜನವಾಗುವುದು. ಆದಕಾರಣ ಕೋಪಕ್ಕೆ ಪ್ರತಿಯಾಗಿ ಕೋಪವನ್ನು ಕೊಡಬೇಡಿ. ನೀತಿಯುಕ್ತವಾದ ನಡವಳಿಕೆಯೇ ಇದು. ಕ್ರಿಸ್ತ “ನಮಗೆ ಕೇಡು ಮಾಡುವವನಿಗೆ ಕೇಡು ಮಾಡಬೇಡಿ” ಎಂದ. ಇದು ನೀತಿ ಮಾತ್ರವಲ್ಲ, ಒಳ್ಳೆಯ ಮಾರ್ಗವೂ ಹೌದು ಎಂಬುದನ್ನು ಅರಿಯುವವರೆಗೆ ಅದು ನಮಗೆ ಅರ್ಥವಾಗುವುದಿಲ್ಲ. ಏಕೆಂದರೆ ಯಾವನು ಕೋಪಗೊಳ್ಳುವನೋ ಅವನು ಶಕ್ತಿಯನ್ನು ವ್ಯಯಮಾಡುವನು. ಮನಸ್ಸು ದ್ವೇಷಾಸೂಯೆಗಳನ್ನು ತಾಳದಿರಲಿ.

ಮೂಲಧಾತುವನ್ನು ಕಂಡುಹಿಡಿದ ಮೇಲೆ ರಸಾಯನಶಾಸ್ತ್ರಜ್ಞನ ಕೆಲಸ ಕೊನೆಗೊಂಡಂತೆ. ಏಕತೆಯನ್ನು ಸೇರಿದಾಗ ಧರ್ಮಶಾಸ್ತ್ರದಲ್ಲಿ ಪೂರ್ಣತೆಯನ್ನು ಪಡೆದಂತೆ. ಅದನ್ನು ಸಾವಿರಾರು ವರುಷಗಳ ಹಿಂದೆಯೇ ಸೇರಿದರು. “ನಾನು ಮತ್ತು ನನ್ನ ತಂದೆ ಇಬ್ಬರೂ ಒಂದೇ” ಎಂದಾಗ ಪೂರ್ಣತೆಯನ್ನು ಮುಟ್ಟಿದರು.

