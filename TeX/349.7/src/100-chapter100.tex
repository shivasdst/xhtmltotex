
\chapter[ವಿದೇಶ ಮತ್ತು ಭಾರತದ ಆಂತರಿಕ ಸಮಸ್ಯೆಗಳು ]{ವಿದೇಶ ಮತ್ತು ಭಾರತದ ಆಂತರಿಕ ಸಮಸ್ಯೆಗಳು \protect\footnote{\engfoot{C.W. Vol. V, p. 209}}}

\centerline{\textbf{(“ದಿ ಹಿಂದು,” ಮದರಾಸು–ಫೆಬ್ರವರಿ, 1897, ರೈಲಿನಲ್ಲಿ)}}

\vskip 0.4cm

ನಮ್ಮ ಪ್ರತಿನಿಧಿ ಸ್ವಾಮಿ ವಿವೇಕಾನಂದರನ್ನು ಚಂಗಲ್​ಪೇಟೆ ಸ್ಟೇಷನ್ನಿನಲ್ಲಿ ರೈಲಿನಲ್ಲಿ ಭೇಟಿಮಾಡಿ ಅವರೊಡನೆ ಮದ್ರಾಸಿಗೆ ಬಂದರು. ಕೆಳಗಿನದು ಭೇಟಿಯ ವಿವರ:

\vskip 2pt

ಪ್ರಶ್ನೆ: ಸ್ವಾಮೀಜಿ, ನಿಮ್ಮನ್ನು ಯಾವುದು ಅಮೆರಿಕ ದೇಶಕ್ಕೆ ಹೋಗುವಂತೆ ಮಾಡಿತು?

\vskip 2pt

ಸ್ವಾಮೀಜಿ: “ಇದಕ್ಕೆ ಸಂಕ್ಷೇಪವಾಗಿ ಉತ್ತರ ಕೊಡುವುದು ಬಹಳ ಕಷ್ಟ. ಅದಕ್ಕೆ ಈಗ ಭಾಗಶಃ ಉತ್ತರ ಕೊಡಬಲ್ಲೆ. ನಾನು ಇಂಡಿಯಾ ದೇಶವನ್ನೆಲ್ಲಾ ಸಂಚಾರ ಮಾಡಿದ ಮೇಲೆ ಬೇರೆ ದೇಶಗಳನ್ನು ನೋಡಬೇಕೆನಿಸಿತು. ಆದಕಾರಣವೇ ದೂರ ಪ್ರಾಚ್ಯಗಳ ಮೂಲಕ\break ಅಮೆರಿಕಾ ದೇಶಕ್ಕೆ ಹೋದೆ.”

\vskip 2pt

ಸ್ವಾಮೀಜಿ: “ನೀವು ಜಪಾನಿನಲ್ಲಿ ಏನನ್ನು ನೋಡಿದಿರಿ? ಭರತಖಂಡವೂ ಜಪಾನಿನಂತೆ ಪ್ರಗತಿಪರವಾದ ಹಾದಿಯಲ್ಲಿ ಹೋಗುವ ಸಂಭವ ಉಂಟೆ?”

\vskip 2pt

ಸ್ವಾಮೀಜಿ: “ಇಂಡಿಯಾ ದೇಶದ ಮೂವತ್ತು ಕೋಟಿ ಜನರೆಲ್ಲ ಒಟ್ಟಿಗೆ ಕಲೆತು ಒಂದು ರಾಷ್ಟ್ರವಾದರೆ ಮಾತ್ರ ಅದು ಸಾಧ್ಯ, ಇಲ್ಲದೇ ಇದ್ದರೆ ಇಲ್ಲ. ಜಪಾನೀಯರಷ್ಟು ದೇಶಭಕ್ತರು ಮತ್ತು ಕಲಾಪ್ರೇಮಿಗಳಾದವರನ್ನು ಜಗತ್ತು ಕಂಡೇ ಇಲ್ಲ. ಅವರಲ್ಲಿರುವ\break ಒಂದು ವೈಶಿಷ್ಟ್ಯ ಇದು. ಯೂರೋಪು ಮತ್ತು ಇತರ ದೇಶಗಳಲ್ಲಿ ಕಲೆ ಮತ್ತು ಕೊಳೆ\break ಒಟ್ಟಿಗೆ ಇರುತ್ತದೆ. ಆದರೆ ಜಪಾನಿನಲ್ಲಿ ಕಲೆ ಮತ್ತು ಶುಭ್ರತೆ ಒಟ್ಟಿಗೆ ಇರುವುವು. ನಮ್ಮಲ್ಲಿ ಪ್ರತಿಯೊಬ್ಬರೂ ತಮ್ಮ ಜೀವಿತ ಕಾಲದಲ್ಲಿ ಒಮ್ಮೆಯಾದರೂ ಜಪಾನನ್ನು ನೋಡಿ ಬರಲಿ ಎಂಬುದು ನನ್ನ ಆಸೆ. ಅಲ್ಲಿಗೆ ಹೋಗುವುದು ಬಹಳ ಸುಲಭ. ಜಪಾನೀಯರು ಹಿಂದುಗಳಿಗೆ ಸಂಬಂಧಪಟ್ಟುದೆಲ್ಲ ಮಹತ್ತರವಾದುದು ಎಂದು ಭಾವಿಸಿರುವರು. ಭರತಖಂಡ ಪುಣ್ಯಭೂಮಿ ಎಂದು ಅವರು ತಿಳಿಯುವರು. ಜಪಾನಿನ ಬೌದ್ಧಧರ್ಮವು ಸಿಲೋನಿನ ಬೌದ್ಧಧರ್ಮದಂತೆ ಅಲ್ಲ. ಅದು ವೇದಾಂತದಂತೆಯೇ ಇದೆ. ಅದು ಈಶ್ವರನನ್ನು ಒಪ್ಪಿಕೊಳ್ಳುವ, ಮತ್ತು ಇತ್ಯಾತ್ಮಕವಾದ ಸಿದ್ಧಾಂತ. ಸಿಲೋನಿನದರಂತೆ ನಿರೀಶ್ವರವಾದಿಯೂ ನೇತ್ಯಾತ್ಮಕವೂ ಆದದ್ದಲ್ಲ.

\vskip 2pt

ಪ್ರಶ್ನೆ: “ಜಪಾನಿನ ಹಠಾತ್​ ಅಭಿವೃದ್ಧಿಗೆ ಕಾರಣವೇನು?”

\vskip 2pt

ಸ್ವಾಮೀಜಿ: “ಜಪಾನೀಯರಲ್ಲಿ ಇರುವ ಆತ್ಮಶ್ರದ್ಧೆ ಮತ್ತು ಅವರ ದೇಶಪ್ರೀತಿ.\break ಭರತಖಂಡದಲ್ಲಿ ಸಂಪೂರ್ಣ ನಿಃಸ್ವಾರ್ಥಪರರಾಗಿ, ದೇಶಕ್ಕಾಗಿ ತಮ್ಮ ಸರ್ವಸ್ವವನ್ನೂ\break ಧಾರೆಯೆರೆಯಬಲ್ಲ ವ್ಯಕ್ತಿಗಳು ಜನಿಸಿದರೆ, ಆಗ ಭರತಖಂಡವು ಪ್ರತಿಯೊಂದು ಕಾರ್ಯಕ್ಷೇತ್ರದಲ್ಲಿಯೂ ಮಹತ್ತಾದುದಾಗುವುದು. ಜನರೇ ಒಂದು ದೇಶವನ್ನು ನಿರ್ಮಿಸುವವರು. ಬರೀ ದೇಶದಲ್ಲಿ ಏನಿದೆ? ಸಾಮಾಜಿಕವಾಗಿ ಮತ್ತು ರಾಜಕೀಯವಾಗಿ ಅವರ ನೀತಿಯನ್ನು ನೀವು ಅನುಸರಿಸಿದರೆ ನೀವು ಅವರಷ್ಟೇ ಮಹತ್ತಿಗೆ ಏರುವಿರಿ. ಜಪಾನೀಯರು ತಮ್ಮ ದೇಶಕ್ಕಾಗಿ ಯಾವುದೇ ತ್ಯಾಗಕ್ಕೂ ಸಿದ್ಧರಾಗಿರುವರು. ಆದ್ದರಿಂದ ಅವರು ಒಂದು ಮಹಾನ್​ ಜನಾಂಗವಾಗಿದ್ದಾರೆ. ಆದರೆ ನೀವು ಹಾಗಿಲ್ಲ, ಹಾಗೆ ಆಗಲಾರಿರಿ. ನೀವು ನಿಮ್ಮ ಸಂಸಾರಕ್ಕೆ ಮತ್ತು ಆಸ್ತಿಗೆ ಮಾತ್ರ ಸರ್ವಸ್ವವನ್ನೂ ಧಾರೆಯೆರೆಯಬಲ್ಲಿರಿ.”

\vskip 2pt

ಪ್ರಶ್ನೆ: “ಭರತಖಂಡವು ಜಪಾನಿನಂತೆ ಆಗಬೇಕೆಂಬುದು ನಿಮ್ಮ ಇಚ್ಚೆಯೆ?”

ಸ್ವಾಮೀಜಿ: “ಎಂದಿಗೂ ಇಲ್ಲ. ಭರತಖಂಡವು ಈಗ ಏನಾಗಿದೆಯೋ ಹಾಗೆಯೇ ಇರಬೇಕು. ಇಂಡಿಯಾ ದೇಶವು ಜಪಾನಿನಂತೆ ಹೇಗೆ ಆಗಬಲ್ಲದು? ಅಥವಾ ಒಂದು ದೇಶವು ಮತ್ತೊಂದು ದೇಶದಂತೆ ಹೇಗೆ ಆಗಬಲ್ಲದು? ಪ್ರತಿಯೊಂದು ದೇಶದಲ್ಲಿಯೂ ಸಂಗೀತದಲ್ಲಿರುವಂತೆ ಒಂದು ಮುಖ್ಯ ಸ್ವರವಿದೆ. ಪ್ರತಿಯೊಂದು ಅದಕ್ಕೆ ಅಧೀನ ಭರತಖಂಡದ ಆದರ್ಶ ಧರ್ಮ. ಸಾರ್ವಜನಿಕ ಮತ್ತು ಇತರ ಸುದಾರಣೆಗಳೆಲ್ಲ ಗೌಣ. ಆದಕಾರಣ ಇಂಡಿಯಾ ದೇಶವು ಜಪಾನಿನಂತೆ ಆಗಲಾರದು. ಹೃದಯ ಭಗ್ನವಾದಾಗ ಚಿಂತನಾಲಹರಿಗಳು ಹೊರಹೊಮ್ಮುವುವು ಎಂದು ಹೇಳುವರು. ಭರತಖಂಡದ ಹೃದಯ ಒಡೆಯಬೇಕು, ಆಧ್ಯಾತ್ಮಿಕ ಪ್ರವಾಹ ಅಲ್ಲಿಂದ ಹರಿಯುವುದು. ಭರತಖಂಡ ಭರತಖಂಡವೆ, ನಾವು ಜಪಾನಿಯರಲ್ಲ, ಹಿಂದೂಗಳು. ಭರತಖಂಡ ವಾತಾವರಣವೆ ಬೆಂದ ಜೀವಕ್ಕೆ ತಂಪನ್ನು ನೀಡುವುದು. ನಾನು ಇಲ್ಲಿ ಬಿಡುವಿಲ್ಲದೆ ಕೆಲಸದಲ್ಲಿ ನಿರತನಾಗಿರುವೆನು. ಈ ಕೆಲಸದ ಮಧ್ಯೆ ನನಗೆ ವಿರಾಮ ದೊರಕುವುದು. ಭಾರತದಲ್ಲಿ ಆಧ್ಯಾತ್ಮಿಕ ಕೆಲಸ ಕಾರ್ಯಗಳೇ ನಿನಗೆ ವಿಶ್ರಾಂತಿಯನ್ನು ನೀಡುವುವು. ನೀವು ಪ್ರಾಪಂಚಿಕ ಕೆಲಸದಲ್ಲಿ ನಿರತರಾದರೆ ಮಧುಮೇಹದಿಂದ (\enginline{Diabetes}) ಸಾಯುವಿರಿ.”

\vskip 3pt

ಪ್ರಶ್ನೆ: “ಇಷ್ಟು ಜಪಾನಿನ ಕಥೆಯಾಯಿತು. ಅಮೆರಿಕ ದೇಶದಲ್ಲಿ ನಿಮ್ಮ ಪ್ರಥಮ ಅನುಭವ ಹೇಗಿತ್ತು ಸ್ವಾಮೀಜಿ?”

\vskip 4pt

ಸ್ವಾಮೀಜಿ: “ಮೊದಲಿನಿಂದ ಕೊನೆಯವರೆಗೂ ಅದು ತುಂಬಾ ಚೆನ್ನಾಗಿತ್ತು. ಮಿಷನರಿಗಳು ಮತ್ತು ಕೆಲವು ‘ಚರ್ಚಿನ ಹೆಂಗಸರು’ ಇವರನ್ನು ಬಿಟ್ಟರೆ ಉಳಿದವರೆಲ್ಲ ಒಳ್ಳೆಯ ಅತಿಥಿ ಸತ್ಕಾರಪರರು, ದಯಾಳುಗಳು ಮತ್ತು ಒಳ್ಳೆಯ ಸ್ವಭಾವದವರು.

\vskip 4pt

ಪ್ರಶ್ನೆ: “ನೀವು ಹೇಳಿದ ‘ಚರ್ಚಿನ ಹೆಂಗಸರು’ ಯಾರು ಸ್ವಾಮೀಜಿ?”

\vskip 4pt

ಸ್ವಾಮೀಜಿ: “ಹೆಂಗಸರು ಒಬ್ಬ ಗಂಡನನ್ನು ಹುಡುಕುವುದಕ್ಕಾಗಿ ಬೇಕಾದಷ್ಟು ಪ್ರಯತ್ನ ಮಾಡುವರು. ಅವಳು ಸಮುದ್ರ ತೀರದಲ್ಲಿರುವ ಎಲ್ಲಾ ಫ್ಯಾಷನಬಲ್​ ಹೋಟೆಲುಗಳಿಗೆ ಹೋಗಿ ತನ್ನ ಬುದ್ಧಿವಂತಿಕೆ, ಉಪಾಯಗಳನ್ನೆಲ್ಲ ಒಬ್ಬ ವರನನ್ನು ಹುಡುಕಲು ಬಳಸುವಳು.\break ಅವಳು ತನ್ನ ಪ್ರಯತ್ನಗಳು ವ್ಯರ್ಥವಾದಾಗ ಅಮೆರಿಕದಲ್ಲಿ ‘ಓಲ್ಡ್​ ಮೆಯ್ಡ್​’ ಎಂದು ಕರೆಯುವ ಎಂದರೆ ಪ್ರಾಯವಿಳಿದ ಹೆಂಗಸಾಗಿ ಚರ್ಚಿಗೆ ಸೇರುವಳು. ಅವರಲ್ಲಿ ಕೆಲವರು ‘ಚರ್ಚಿ’ಗಳಾಗುವರು. ಈ ಚರ್ಚಿನ ಹೆಂಗಸರ ಮತಭ್ರಾಂತಿ ಹೇಳತೀರದು. ಇವರು ಅಲ್ಲಿ ಪಾದ್ರಿಯ ಕೈಕೆಳಗೆ ಇರುವರು. ಇವರು ಮತ್ತು ಪಾದ್ರಿಗಳು ಒಟ್ಟಿಗೆ ಸೇರಿ ಜಗತ್ತನ್ನು ನರಕಸದೃಶವನ್ನಾಗಿ ಮಾಡುವರು, ಧರ್ಮವನ್ನೆಲ್ಲಾ ಹಾಳುಮಾಡುವರು. ಇದನ್ನು ಬಿಟ್ಟರೆ ಅಮೆರಿಕದವರು ಬಹಳ ಒಳ್ಳೆಯವರು. ಅವರು ನನ್ನನ್ನು ಪ್ರೀತಿಸುತ್ತಿದ್ದರು, ನಾನೂ ಅವರನ್ನು ವಿಶೇಷವಾಗಿ ಪ್ರೀತಿಸುವೆನು. ನಾನು ಅವರಲ್ಲಿ ಒಬ್ಬನು ಎಂದು ಭಾವಿಸಿದ್ದೆ.”

\vskip 4pt

ಪ್ರಶ್ನೆ: “ವಿಶ್ವಧರ್ಮ ಸಮ್ಮೇಳನದ ಪರಿಣಾಮಗಳನ್ನು ಕುರಿತಂತೆ ನಿಮ್ಮ ಅಭಿಪ್ರಾಯವೇನು?”

\vskip 3pt

ಸ್ವಾಮೀಜಿ: “ನನಗೆ ಕಂಡಂತೆ ಅದು ಪ್ರಪಂಚಕ್ಕೆ ಕ್ರೈಸ್ತೇತರ ಧರ್ಮಗಳ ಹುಳುಕನ್ನು ತೋರುವುದಕ್ಕೆ ಕಟ್ಟಿದ ನಾಟಕ. ಆದರೆ ಕ್ರೈಸ್ತೇತರ ಧರ್ಮಗಳೇ ಪ್ರಬಲವಾಗಿ ಅದೊಂದು ಕ್ರೈಸ್ತರ ಮರ್ಮವನ್ನು ತೆರೆದು ತೋರಿಸಿದ ನಾಟಕವೇ ಆಯಿತು. ಕ್ರೈಸ್ತರ ದೃಷ್ಟಿಯಿಂದ ವಿಶ್ವಧರ್ಮ ಸಮ್ಮೇಳನ ನಿಷ್ಪ್ರಯೋಜಕವಾಯಿತು. ಮತ್ತೊಂದು ವಿಶ್ವಧರ್ಮ ಸಮ್ಮೇಳನವನ್ನು ಪ್ಯಾರಿಸ್ಸಿನಲ್ಲಿ ನಡೆಸಬೇಕೆಂದಾಗ ರೋಮನ್​ ಕ್ಯಾಥೋಲಿಕ್ಕರು ಅದನ್ನು ವಿರೋಧಿಸುತ್ತಿರುವರು. ಆದರೆ ಚಿಕಾಗೊ ವಿಶ್ವಧರ್ಮ ಸಮ್ಮೇಳನ ಭರತಖಂಡಕ್ಕೆ ಮತ್ತು ಭಾರತೀಯ ಚಿಂತನೆಯ ದೃಷ್ಟಿಯಿಂದ ಯಶಸ್ವಿಯಾಯಿತು. ಇದು ವೇದಾಂತದ ಭಾವನೆಗಳು ಜಗತ್ತಿನಲ್ಲೆಲ್ಲ ಹರಡುವಂತೆ ಮಾಡಿತು. ಕ್ರೈಸ್ತಪಾದ್ರಿಗಳು ಮತ್ತು ಚರ್ಚಿನ ಹೆಂಗಸರು ಬಿಟ್ಟರೆ ಉಳಿದ ಅಮೆರಿಕಾದವರೆಲ್ಲ ವಿಶ್ವಧರ್ಮ ಸಮ್ಮೇಳನದ ಪರಿಣಾಮದ ವಿಷಯದಲ್ಲಿ ಸಂತೋಷವನ್ನು ವ್ಯಕ್ತಪಡಿಸುತ್ತಿದ್ದಾರೆ.”

\vskip 3pt

ಪ್ರಶ್ನೆ: “ಇಂಗ್ಲೆಂಡಿನಲ್ಲಿ ನಿಮ್ಮ ಭಾವನೆಗಳು ಪ್ರಚಾರವಾಗುವ ಸಂಭವ ಹೇಗಿದೆ, ಸ್ವಾಮೀಜಿ?”

\vskip 3pt

ಸ್ವಾಮೀಜಿ: ಬೇಕಾದಷ್ಟು ಅವಕಾಶವಿದೆ. ಕೆಲವು ವರುಷಗಳಲ್ಲಿ ಅನೇಕ ಜನ\break ಇಂಗ್ಲಿಷರು ವೇದಾಂತಿಗಳಾಗುವರು. ಅಮೆರಿಕಾ ದೇಶಕ್ಕಿಂತ ಇಲ್ಲಿ ಹೆಚ್ಚು ಅವಕಾಶವಿದೆ. ಅಮೆರಿಕಾ ದೇಶದಲ್ಲಿ ಬೇಕಾದಷ್ಟು ಆಡಂಬರವಿದೆ ಅದು ಇಂಗ್ಲಿಷಿನವರಲ್ಲಿ ಇಲ್ಲ.\break ಕ್ರೈಸ್ತರೂ ಕೂಡ ವೇದಾಂತವನ್ನು ಓದದೆ ನ್ಯೂಟೆಸ್ಟಮೆಂಟನ್ನು (ಹೊಸ ಒಡಂಬಡಿಕೆಯನ್ನು) ಅರ್ಥಮಾಡಿಕೊಳ್ಳಲಾರರು. ವೇದಾಂತವೇ ಎಲ್ಲಾ ಧರ್ಮಗಳ ಯುಕ್ತಿಪೂರ್ವಕ ವಿವರಣೆ. ವೇದಾಂತ ತತ್ತ್ವ ಇಲ್ಲದೇ ಇದ್ದರೆ ಧರ್ಮಗಳೆಲ್ಲ ಮೂಢನಂಬಿಕೆ. ವೇದಾಂತವಿದ್ದರೆ ಅವುಗಳೆಲ್ಲ ಧರ್ಮವಾಗುವುವು.”

\vskip 3pt

ಪ್ರಶ್ನೆ: “ಆಂಗ್ಲೇಯರ ಶೀಲದಲ್ಲಿ ನೀವು ಯಾವ ವಿಶೇಷವನ್ನು ಕಂಡಿರಿ?”

\vskip 3pt

ಸ್ವಾಮೀಜಿ: “ಆಂಗ್ಲೇಯನು. ಏನನ್ನಾದರೂ ನಂಬುವುದೇ ತಡ, ಅದನ್ನು ಕಾರ್ಯಗತಮಾಡಲು ಯತ್ನಿಸುವನು. ಕೆಲಸ ಮಾಡುವುದಕ್ಕೆ ಅವನಲ್ಲಿ ಪ್ರಚಂಡ ಶಕ್ತಿಯಿದೆ. ಆಂಗ್ಲೇಯ ಸ್ತ್ರೀ ಅಥವಾ ಪುರುಷರನ್ನು ಪ್ರಪಂಚದಲ್ಲಿ ಕೆಲಸದಲ್ಲಿ ಯಾರೂ ಮೀರಿಸಲಾರರು. ಆದಕಾರಣವೇ ನಾನು ಅವರಲ್ಲಿ ಶ್ರದ್ಧೆಯನ್ನು ಇಟ್ಟಿರುವುದು. ಜಾನ್​ ಬುಲ್​ (ಇಂಗ್ಲೀಷಿನವನು) ವಿಷಯವನ್ನು ಗ್ರಹಿಸುವುದು ಸ್ವಲ್ಪ ನಿಧಾನ. ಅವನಿಗೆ ಒಂದು ವಿಷಯವನ್ನು ಪದೇ ಪದೇ ಒತ್ತಿ ಹೇಳುತ್ತಿರಬೇಕು. ಆದರೆ ಒಮ್ಮೆ ಅದನ್ನು ತಿಳಿದುಕೊಂಡರೆ ಅವನು ಬಡಪಟ್ಟಿಗೆ ಬಿಡುವುದಿಲ್ಲ. ಇಂಗ್ಲೆಂಡಿನಲ್ಲಿ ಯಾವ ಮಿಷನರಿಯಾಗಲೀ, ಇತರರಾಗಲೀ ನನಗೆ ವಿರೋಧವಾಗಿ ಏನನ್ನೂ ಹೇಳಲಿಲ್ಲ, ನನ್ನ ಬಗ್ಗೆ ಅಪಪ್ರಚಾರ ಮಾಡಲಿಲ್ಲ. ಆಶ್ಚರ್ಯಕರವಾದುದು ಏನೆಂದರೆ ನನ್ನ ಅನೇಕ ಸ್ನೇಹಿತರು ಇಂಗ್ಲೆಂಡಿನ ಚರ್ಚಿಗೆ ಸೇರಿದವರು. ಈ ಮಿಷನರಿಗಳು ಇಂಗ್ಲೆಂಡಿನ ಮೇಲಿನ ವರ್ಗದವರಿಂದ ಬಂದವರಲ್ಲ ಎಂದು ಕೇಳಿದ್ದೇನೆ. ಜಾತಿ ಇಲ್ಲಿರುವಷ್ಟೇ ಅಲ್ಲಿಯೂ ಬಹಳ ಕಟ್ಟುನಿಟ್ಟು. ಆಂಗ್ಲೇಯ ಪಾದ್ರಿಗಳು ಭದ್ರಮನುಷ್ಯರ ಗುಂಪಿಗೆ ಸೇರಿದವರು. ಅವರಿಗೂ ನಿಮಗೂ ಅಭಿಪ್ರಾಯಭೇದ ಇರಬಹುದು. ಆದರೆ ಅವರು ನಿಮ್ಮ ಸ್ನೇಹಿತರಾಗುವುದಕ್ಕೆ ಅದರಿಂದ ಯಾವ ಅಭ್ಯಂತರವೂ ಇರುವುದಿಲ್ಲ. ಆದಕಾರಣ ನಮ್ಮ ದೇಶದವರಿಗೆ ನಾನು ನೀಡುವ ಹಿತವಚನವಿದು: ನಿಂದಿಸುತ್ತಿರುವ ಪಾದ್ರಿಗಳನ್ನು ಗಮನಿಸಬೇಡಿ. ಅವರೇನು ಎಂಬುದು ನನಗೆ ಗೊತ್ತಿದೆ. ಅಮೆರಿಕದವರು ಹೇಳುವಂತೆ ಅವರ ಸ್ಥಾನವನ್ನು ಅವರಿಗೆ ತೋರಿಸಿದ್ದಾಗಿದೆ. ಅವರನ್ನು ಲಕ್ಷ್ಯಕ್ಕೆ ತೆಗೆದುಕೊಳ್ಳದಿರುವುದೇ ಸರಿಯಾದ ಮನೋಭಾವ.”

\vskip 3pt

ಪ್ರಶ್ನೆ: “ಅಮೆರಿಕ ಮತ್ತು ಇಂಗ್ಲೆಂಡಿನ ಸಮಾಜ ಸುಧಾರಣಾ ಚಳುವಳಿಗಳ ವಿಷಯದಲ್ಲಿ ನಿಮಗೆ ತಿಳಿದಿರುವುದನ್ನು ಹೇಳುತ್ತೀರ, ಸ್ವಾಮೀಜಿ?”

\vskip 4pt

ಸ್ವಾಮೀಜಿ: “ಆಗಲಿ. ಈಗ ಆಗುತ್ತಿರುವ ಸಮಾಜದ ಕ್ರಾಂತಿಯಲ್ಲಿ, ಅದರಲ್ಲೂ ಅದರ ಮುಂದಾಳುಗಳು ತಮ್ಮ ಸಮತಾವಾದ ಸಿದ್ಧಾಂತಕ್ಕೆಲ್ಲ ಒಂದು ಆಧ್ಯಾತ್ಮಿಕ ತಳಹದಿ ಬೇಕೆಂದು ಮನಗಂಡಿರುವರು. ಆ ತಳಹದಿ ವೇದಾಂತದಲ್ಲಿ ಮಾತ್ರ ದೊರಕುವುದು. ನನ್ನ ಉಪನ್ಯಾಸಗಳನ್ನು ಕೇಳುವುದಕ್ಕೆ ಬರುತ್ತಿದ್ದ ಅನೇಕ ಮುಂದಾಳುಗಳು ಹೊಸ ಜಾಗೃತಿಗೆ ತಳಹದಿಯಾಗಿ ವೇದಾಂತ ಬೇಕಾಗಿದೆ ಎಂದು ಹೇಳುತ್ತಿದ್ದರು ಎಂಬುದಾಗಿ ಕೇಳಿದ್ದೇನೆ.”

\vskip 4pt

ಪ್ರಶ್ನೆ: “ಭರತಖಂಡ ಜನಸಾಮಾನ್ಯರ ವಿಷಯವಾಗಿ ನಿಮ್ಮ ಅಭಿಪ್ರಾಯವೇನು?”

\vskip 4pt

ಸ್ವಾಮೀಜಿ: “ಓ ನಮ್ಮ ದಾರಿದ್ರ್ಯ ಹೇಳತೀರದು. ನಮ್ಮ ಜನರಿಗೆ ಲೌಕಿಕ ವಿಷಯವೇ\break ತಿಳಿಯದು. ನಮ್ಮ ಜನಸಾಮಾನ್ಯರು ಒಳ್ಳೆಯ ಸ್ವಭಾವದವರು. ಏಕೆಂದರೆ ಬಡತನ ಎಂಬುದು ಇಲ್ಲಿ ಒಂದು ಅಪರಾಧವಲ್ಲ. ನಮ್ಮ ಜನ ಹಿಂಸಾಪ್ರವೃತ್ತಿಯವರಲ್ಲ. ಅನೇಕವೇಳೆ ಅಮೆರಿಕಾ ಮತ್ತು ಇಂಗ್ಲೆಂಡಿನಲ್ಲಿ ಜನ ನನ್ನ ಉಡಿಗೆ ತೊಡಿಗೆಗಾಗಿ ನನ್ನನ್ನು ಸುತ್ತುಗಟ್ಟುತ್ತಿದ್ದರು. ಭಾರತದಲ್ಲಿ ಒಬ್ಬನು ವಿಚಿತ್ರವಾಗಿ ವೇಷಭೂಷಣಗಳನ್ನು ಹಾಕಿಕೊಂಡಿದ್ದಾನೆ ಎಂಬ ಕಾರಣಕ್ಕಾಗಿ ಜನ ಅವನನ್ನು ಮುತ್ತಿಕೊಳ್ಳುವುದು ಕಂಡುಬರುವುದಿಲ್ಲ. ಉಳಿದ ಎಲ್ಲ ವಿಷಯಗಳಲ್ಲೂ ನಮ್ಮ ಸಾಮಾನ್ಯ ಜನರು ಹೆಚ್ಚು ಸುಸಂಸ್ಕೃತರು.”

\vskip 4pt

ಪ್ರಶ್ನೆ: “ನಮ್ಮ ಜನಸಾಮಾನ್ಯರನ್ನು ಮೇಲೆತ್ತಲು ನೀವು ಏನು ಸಲಹೆಯನ್ನು ಕೊಡುತ್ತೀರಿ?”

\vskip 4pt

ಸ್ವಾಮೀಜಿ: “ಅವರಿಗೆ ಲೌಕಿಕ ಶಿಕ್ಷಣ ಕೊಡಬೇಕು. ನಮ್ಮ ಪೂರ್ವಿಕರ ಮಾರ್ಗವನ್ನು ಅನುಸರಿಸಬೇಕು. ಅಂದರೆ ಆದರ್ಶವು ಕ್ರಮೇಣ ಎಲ್ಲರಿಗೂ ತಿಳಿಯುವಂತೆ ಮಾಡಬೇಕು. ಲೌಕಿಕ ವಿದ್ಯೆಯನ್ನು ಕೂಡ ಧರ್ಮದ ಮೂಲಕ ಬೋಧಿಸಿ, ಅವರನ್ನು ಕ್ರಮೇಣ ಮೇಲೆತ್ತಿ, ಸಮಾನತೆಯ ಮಟ್ಟಕ್ಕೆ ಏರಿಸಿ.”

\vskip 4pt

ಪ್ರಶ್ನೆ: “ಸ್ವಾಮೀಜಿ, ಇದೇನು ಸುಲಭವಾಗಿ ಸಾಧಿಸುವ ಕೆಲಸ ಎಂದು ಭಾವಿಸಿದಿರಾ?”

\vskip 4pt

ಸ್ವಾಮೀಜಿ: “ಇದನ್ನು ನಾವು ಕ್ರಮೇಣ ಮಾಡಬೇಕಾಗಿದೆ. ಆದರೆ ನನ್ನೊಡನೆ ಕೆಲಸ ಮಾಡಬಲ್ಲ. ಸ್ವಾರ್ಥತ್ಯಾಗ ಮಾಡಬಲ್ಲ ತರುಣರಿದ್ದರೆ ನಾಳೆಯೇ ಇದನ್ನು ಮಾಡಬಹುದು. ಈ ಕಾರ್ಯಸಾಧನೆ ಉತ್ಸಾಹ ಮತ್ತು ತ್ಯಾಗವನ್ನು ಅವಲಂಬಿಸಿದೆ.”

\vskip 4pt

ಪ್ರಶ್ನೆ: “ಆದರೆ ಈಗಿನ ಸ್ಥಿತಿ ಅವರ ಹಿಂದಿನ ಕರ್ಮದ ಫಲವಾದರೆ ಅವರು ಅಷ್ಟು ಸುಲಭವಾಗಿ ಹೇಗೆ ಇದರಿಂದ ಪಾರಾದಾರು? ನೀವು ಅವರಿಗೆ ಹೇಗೆ ಸಹಾಯ ಮಾಡಬಲ್ಲಿರಿ?”

\vskip 4pt

ಸ್ವಾಮೀಜಿ: (ತಕ್ಷಣ ಉತ್ತರ ಕೊಟ್ಟರು) “ಕರ್ಮ ಎಂಬುದು ಮಾನವ ಸ್ವಾತಂತ್ರ್ಯದ\break ನಿತ್ಯಪ್ರತಿಪಾದನೆ. ನಾವು ನಮ್ಮ ಕರ್ಮದಿಂದ ಅಧೋಗತಿಗೆ ಬಂದಿದ್ದರೆ, ಪುನಃ ಕರ್ಮದಿಂದಲೇ ಉತ್ತಮಸ್ಥಿತಿಗೆ ಬರುವುದೂ ನಮ್ಮ ಕೈಯಲ್ಲೇ ಇದೆ ಎಂಬುದು ಖಂಡಿತ.\break ಜನಸಾಮಾನ್ಯರೇ ತಮ್ಮ ಕರ್ಮದಿಂದ ಈ ಸ್ಥಿತಿಗೆ ಬರಲಿಲ್ಲ. ಆದಕಾರಣ ಅವರು ಕೆಲಸ ಮಾಡುವುದಕ್ಕೆ ಒಳ್ಳೆಯ ವಾತಾವರಣವನ್ನು ಕಲ್ಪಿಸಿಕೊಡಬೇಕು. ಜಾತಿಗಳನ್ನೆಲ್ಲ ಒಂದೇ ಮಟ್ಟಕ್ಕೆ ತರಬೇಕೆಂದು ನಾನು ಹೇಳುವುದಿಲ್ಲ. ಜಾತಿ ಒಳ್ಳೆಯದೇ, ಆದ್ದರಿಂದ ಅದರ ಯೋಜನೆಯನ್ನು ಮುಂದುವರಿಸೋಣ. ನಿಜವಾಗಿ ಜಾತಿ ಎಂದರೆ ಏನೆಂದು ಹತ್ತು ಲಕ್ಷಕ್ಕೆ\break ಒಬ್ಬರಿಗೂ ತಿಳಿಯದು. ಪ್ರಪಂಚದಲ್ಲಿ ಜಾತಿ ಇಲ್ಲದ ದೇಶವೇ ಇಲ್ಲ. ಇಂಡಿಯಾ ದೇಶದಲ್ಲಿ ಜಾತಿಯಿಂದ ಜಾತ್ಯತೀತ ಸ್ಥಿತಿಗೆ ನಾವು ಹೋಗುತ್ತೇವೆ. ಜಾತಿ ಎಲ್ಲಾ ಕಡೆಯಲ್ಲಿಯೂ ಈ ಸಿದ್ಧಾಂತದ ಮೇಲೆ ನಿಂತಿರುವುದು. ಇಂಡಿಯಾ ದೇಶದಲ್ಲಿ ಪ್ರತಿಯೊಬ್ಬರನ್ನೂ ಬ್ರಾಹ್ಮಣ\-ರನ್ನಾಗಿ ಮಾಡಬೇಕು. ಇದೇ ನಮ್ಮ ಗುರಿ. ಬ್ರಾಹ್ಮಣನೇ ಮಾನವನ ಆದರ್ಶ. ನೀವು ಇಂಡಿಯಾ ದೇಶದ ಚರಿತ್ರೆಯನ್ನು ಓದಿದರೆ ಅನೇಕ ವೇಳೆ ಜನ ಸಾಮಾನ್ಯರನ್ನು ಮೇಲಕ್ಕೆ ಎತ್ತುವ ಪ್ರಯತ್ನ ಮಾಡಿರುವುದು ಕಾಣುವುದು. ಎಷ್ಟೋ ಜಾತಿಗಳನ್ನು ಮೇಲಕ್ಕೆ ಎತ್ತಿರುವರು. ಎಲ್ಲರೂ ಬ್ರಾಹ್ಮಣರಾಗುವವರೆಗೆ ಇನ್ನೂ ಹಲವು ಪ್ರಯತ್ನಗಳು ನಡೆಯುವುವು. ಇದೇ ನಮ್ಮ ಯೋಜನೆ. ನಾವು ಯಾರನ್ನೂ ಕೆಳಗೆ ಎಳೆಯದೆ ಅವರನ್ನು ಮೇಲಕ್ಕೆ ಎತ್ತಬೇಕಾಗಿದೆ. ಮುಕ್ಕಾಲುಪಾಲು ಇದನ್ನು ಬ್ರಾಹ್ಮಣರು ಮಾಡಬೇಕಾಗಿದೆ. ಏಕೆಂದರೆ ತನ್ನ ಸಮಾಧಿಯನ್ನು ತಾನೇ ತೋಡಿಕೊಳ್ಳುವುದು ಪ್ರತಿಯೊಂದು ಶ‍್ರೀಮಂತ ವರ್ಗದ ಕರ್ತವ್ಯ. ಅವರು ಎಷ್ಟು ಬೇಗ ಅದನ್ನು ಮಾಡಿಕೊಂಡರೆ ಅಷ್ಟು ಎಲ್ಲರಿಗೂ ಒಳ್ಳೆಯದು. ಕಾಲ ವಿಳಂಬ ಮಾಡಕೂಡದು. ಯೂರೋಪ್​ ಅಮೆರಿಕ ದೇಶಗಳಲ್ಲಿರುವ ಜಾತಿಪದ್ಧತಿಗಿಂತ ಇಂಡಿಯಾ ದೇಶದಲ್ಲಿರುವ ಜಾತಿಪದ್ಧತಿ ಮೇಲು. ಇದು ಸಂಪೂರ್ಣವಾಗಿ ಒಳ್ಳೆಯದು ಎಂದು ನಾನು ಹೇಳುವುದಿಲ್ಲ. ಜಾತಿಪದ್ಧತಿ ಇಲ್ಲದೇ ಇದ್ದಿದ್ದರೆ ನೀವು ಎಲ್ಲಿ ಇರುತ್ತಿದ್ದಿರಿ? ಜಾತಿಪದ್ಧತಿ ಇಲ್ಲದೆ ಇದ್ದಿದ್ದರೆ ನಿಮ್ಮ ವಿದ್ಯೆ ಮುಂತಾದುವೆಲ್ಲ ಎಲ್ಲಿ ಇರುತ್ತಿತ್ತು? ಜಾತಿಪದ್ಧತಿ ಇಲ್ಲದೇ\break ಇದ್ದಿದ್ದರೆ ಐರೋಪ್ಯರು ಇಂಡಿಯಾ ದೇಶದಲ್ಲಿ ತಿಳಿದುಕೊಳ್ಳುವಂಥದು ಏನೂ ಇರುತ್ತಿರಲಿಲ್ಲ. ಮಹಮ್ಮದೀಯರು ಎಲ್ಲವನ್ನೂ ಧ್ವಂಸಮಾಡಿಬಿಡುತ್ತಿದ್ದರು. ಹಿಂದೂ ಸಮಾಜ ಬದಲಾಯಿಸದೆ ಎಂದು ಇತ್ತು? ಅದು ಯಾವಾಗಲೂ ಬದಲಾಯಿಸುತ್ತಿದೆ. ಕೆಲವು ವೇಳೆ ಪರದೇಶದವರು ದಾಳಿ ಇಟ್ಟಾಗ ಚಲನೆ ಬಹಳ ಮಂದವಾಗಿತ್ತು. ಇತರ ಕಾಲದಲ್ಲಿ ಅದು ಚುರುಕಾಗಿತ್ತು. ನಮ್ಮ ದೇಶದವರಿಗೆ ಇದನ್ನೇ ನಾನು ಹೇಳುವುದು. ನಾನು ಜಾತಿಪದ್ಧತಿಯನ್ನು ದ್ವೇಷಿಸುವುದಿಲ್ಲ. ಅವು ಹಿಂದೆ ಏನನ್ನು ಸಾಧಿಸಿವೆ ಎಂಬುದನ್ನು ನೋಡುವೆನು. ಆ ಸ್ಥಿತಿಯಲ್ಲಿ ಮತ್ತಾವ ಜನಾಂಗವೂ ಅದಕ್ಕಿಂತ ಚೆನ್ನಾಗಿ ಏನನ್ನೂ ಮಾಡಿರಲಾರದು. ಅವರು ಚೆನ್ನಾಗಿಯೆ ಮಾಡಿರುವರು ಎಂದು ನಾನು ಹೇಳುತ್ತೇನೆ. ಮತ್ತೂ ಚೆನ್ನಾಗಿ ಮಾಡಿ ಎಂದು ಮಾತ್ರ ಹೇಳುತ್ತೇನೆ.”

\vskip 4pt

ಪ್ರಶ್ನೆ: ಜಾತಿಪದ್ಧತಿ ಮತ್ತು ಕ್ರಿಯಾವಿಧಿಗಳು ಇವುಗಳ ಸಂಬಂಧದ ವಿಷಯದಲ್ಲಿ ನಿಮ್ಮ ಅಭಿಪ್ರಾಯವೇನು ಸ್ವಾಮೀಜಿ?”

\vskip 4pt

ಸ್ವಾಮೀಜಿ: “ಜಾತಿಗಳು ಯಾವಾಗಲೂ ಬದಲಾಯಿಸುತ್ತಿವೆ, ಅದರಂತೆಯೇ ಕ್ರಿಯಾವಿಧಿಗಳೂ ಬದಲಾಯಿಸುತ್ತಿವೆ. ಅದರಂತೆಯೇ ಅವುಗಳನ್ನು ಮಾಡುವ ರೀತಿ ಕೂಡ. ಅವುಗಳ ಹಿಂದಿರುವ ತತ್ತ್ವ ಮತ್ತು ಸತ್ಯ ಮಾತ್ರ ಬದಲಾಯಿಸುವುದಿಲ್ಲ. ನಾವು ನಮ್ಮ ಧರ್ಮವನ್ನು ವೇದಗಳ ಅಧ್ಯಯನದಿಂದ ತಿಳಿದುಕೊಳ್ಳಬೇಕಾಗಿದೆ. ವೇದಗಳನ್ನು ಹೊರತು ಉಳಿದೆಲ್ಲ ಶಾಸ್ತ್ರಗಳೂ ಬದಲಾಗಬೇಕು. ವೇದಗಳು ಎಲ್ಲಾ ಕಾಲಕ್ಕೂ ಪ್ರಮಾಣ. ಇತರ ಶಾಸ್ತ್ರಗಳ ಪ್ರಮಾಣವಾದರೊ ಆಯಾ ಕಾಲಕ್ಕೆ ಮಾತ್ರ. ಉದಾಹರಣೆಗೆ ಒಂದು ಸ್ಮೃತಿ ಒಂದು ಕಾಲದಲ್ಲಿ ಹೆಚ್ಚು ರೂಢಿಯಲ್ಲಿರುವುದು, ಮತ್ತೊಂದು ಕಾಲದಲ್ಲಿ ಮತ್ತೊಂದು ಸ್ಮೃತಿ ಬಳಕೆಯಲ್ಲಿರುವುದು. ಮಹಾತ್ಮರು ಅನೇಕ ವೇಳೆ ಬಂದು ಯಾವ ಮಾರ್ಗದಲ್ಲಿ ನಾವು ಹೋಗಬೇಕೆಂಬುದನ್ನು ತೋರುತ್ತಿರುವರು. ಕೆಲವು ಮಹಾತ್ಮರು ಅಂತ್ಯಜರಿಗೆ ಅನುಕೂಲ ಮಾಡಿಕೊಟ್ಟರು. ಮಧ್ವರಂತಹ ಕೆಲವರು ಸ್ತ್ರೀಯರಿಗೂ ವೇದವನ್ನು ಓದಲು ಅಧಿಕಾರವಿದೆ\break ಎಂದು ಹೇಳಿದರು. ಜಾತಿಪದ್ಧತಿ ಹೋಗಬೇಕಾಗಿಲ್ಲ. ಆದರೆ ಅದು ಕಾಲಕ್ಕೆ ತಕ್ಕಂತೆ ಹೊಂದಿಕೊಂಡು ಹೋಗಬೇಕು ಅಷ್ಟೆ. ಆ ಹಳೆಯ ಪದ್ಧತಿಯಲ್ಲಿ ಬೇಕಾದಷ್ಟು ಹೊಸದನ್ನು ಮಾಡುವ ಪ್ರಾಣಶಕ್ತಿ ಇದೆ. ವರ್ಣವಿಚಾರವನ್ನು ಧ್ವಂಸಮಾಡುವುದು ತಿಳಿಗೇಡಿತನ.\break ಹಳೆಯದು ರೂಪಾಂತರ ಹೊಂದಬೇಕು. ಅದೇ ಹೊಸ ಮಾರ್ಗ.”

\eject

ಪ್ರಶ್ನೆ: “ಹಿಂದೂಗಳಿಗೆ ಸಮಾಜ ಸುಧಾರಣೆ ಬೇಡವೆ?”

ಸ್ವಾಮೀಜಿ: “ಸಮಾಜ ಸುಧಾರಣೆಯ ಆವಶ್ಯಕತೆಯೇನೋ ಇದೆ. ಹಿಂದೆ ಮಹಾತ್ಮರು\break ಯಾವ ರೀತಿ ಸಮಾಜ ಮುಂದುವರಿಯಬೇಕೆಂಬುದನ್ನು ಹೇಳುತ್ತಿದ್ದರು. ರಾಜರು ಅದಕ್ಕೆ ಕಾನೂನಿನ ಮುದ್ರೆಯನ್ನು ಒತ್ತುತ್ತಿದ್ದರು. ಹಿಂದೆ ಭಾರತದಲ್ಲಿ ಸಮಾಜ ಸುಧಾರಣೆಯನ್ನು ಜಾರಿಗೆ ತರಬೇಕಾದರೆ ಹೀಗೆ ಮಾಡುತ್ತಿದ್ದರು. ಈಗ ಆಧುನಿಕ ಕಾಲದಲ್ಲಿ ಅಂತಹ ಪ್ರಗತಿಪರ ಸುಧಾರಣೆಯನ್ನು ತರಬೇಕಾದರೆ ಹಾಗೆ ಮಾಡಬಲ್ಲ ಅಧಿಕಾರ ವರ್ಗವನ್ನು ಸೃಷ್ಟಿಸಬೇಕಾಗಿದೆ. ರಾಜರು ಈಗ ಇಲ್ಲವಾಗಿ ಅಧಿಕಾರವೆಲ್ಲ ಪ್ರಜೆಗಳಲ್ಲಿದೆ. ಆದಕಾರಣ ಜನರೆಲ್ಲ ವಿದ್ಯಾವಂತರಾಗುವವರೆಗೆ, ತಮ್ಮ ಆವಶ್ಯಕತೆಗಳನ್ನು ತಾವು ಮನಗಂಡು ಅವುಗಳನ್ನು ಪರಿಹರಿಸುವ ಸ್ಥಿತಿಗೆ ಅವರು ಬರುವ ತನಕ ನಾವು ಕಾಯಬೇಕು. ಅಲ್ಪಸಂಖ್ಯಾತರ ದಬ್ಬಾಳಿಕೆ, ಪ್ರಪಂಚದಲ್ಲೆಲ್ಲಾ ನಡೆಯುವ ದೊಡ್ಡ ದೌರ್ಜನ್ಯ. ಆದಕಾರಣ ಆದರ್ಶಪ್ರಾಯವಾದ ಸುಧಾರಣೆಯ ವಿಷಯದಲ್ಲಿ ನಮ್ಮ ಶಕ್ತಿಯನ್ನು ವ್ಯಯಮಾಡುವುದಕ್ಕಿಂತ, (ಅದೆಂದಿಗೂ\break ಕಾರ್ಯಗತವಾಗುವುದಿಲ್ಲ) ಸಮಸ್ಯೆಯ ಮೂಲಕ್ಕೇ ಹೋಗೋಣ. ಶಾಸನ ಮಾಡತಕ್ಕ\break ವ್ಯಕ್ತಿಗಳನ್ನು ತರಬೇತು ಮಾಡೋಣ. ಎಂದರೆ ತಮ್ಮ ಸಮಸ್ಯೆಯನ್ನು ತಾವೇ ಪರಿಹರಿಸಿಕೊಳ್ಳುವುದಕ್ಕೆ ಜನರನ್ನು ವಿದ್ಯಾವಂತರನ್ನಾಗಿ ಮಾಡೊಣ. ಇದಾಗುವವರೆಗೆ ಆದರ್ಶ ಸುಧಾರಣೆಗಳೆಲ್ಲ ಆದರ್ಶಗಳಾಗಿಯೇ ಉಳಿಯುವುವು. ಜನ ತಮ್ಮ ವಿಮೋಚನೆಯನ್ನು ತಾವೇ ಮಾಡಿಕೊಳ್ಳಬೇಕಾಗಿದೆ. ಇದೇ ನಮ್ಮ ವಿಧಾನ. ಹಿಂದೆ ಯಾವಾಗಲೂ ಅರಸರು ಜನರನ್ನು ಆಳುತ್ತಿದ್ದುದರಿಂದ ಈಗ ಇದನ್ನು ಭರತಖಂಡದಲ್ಲಿ ಜಾರಿಗೆ ತರಬೇಕಾದರೆ\break ಕಾಲ ಹಿಡಿಯುವುದು.”

\vskip 4pt

ಪ್ರಶ್ನೆ: “ಹಿಂದೂ ಸಮಾಜವು ಐರೋಪ್ಯ ಸಮಾಜದ ರೀತಿನೀತಿಗಳನ್ನು ಯಶಸ್ವಿಯಾಗಿ ಅಳವಡಿಸಿಕೊಳ್ಳುವುದು ಸಾಧ್ಯವೆಂದು ನೀವು ಭಾವಿಸುವಿರಾ?”

\vskip 4pt

ಸ್ವಾಮೀಜಿ: “ಇಲ್ಲ, ನಾವು ಅದನ್ನು ಪೂರ್ತಿ ಅನುಸರಿಸಬೇಕಾಗಿಲ್ಲ. ಗ್ರೀಕಿನ ಮನಸ್ಸು (ಈಗಿನ ಐರೋಪ್ಯರ ಶಕ್ತಿ) ಹಿಂದೂಗಳ ಅಧ್ಯಾತ್ಮದೊಂದಿಗೆ ಸಂಗಮವಾದರೆ ಭರತಖಂಡದಲ್ಲಿ ಒಂದು ಆದರ್ಶ ಸೃಷ್ಟಿಯಾಗುವುದು, ಎಂಬುದು ನನ್ನ ಅಭಿಪ್ರಾಯ ಉದಾಹರಣೆಗೆ,\break ನಮ್ಮ ಶಕ್ತಿಯನ್ನು ಕೆಲಸಕ್ಕೆ ಬಾರದ ಆದರ್ಶಗಳನ್ನು ಕುರಿತು ಚರ್ಚಿಸಿ ವೃಥಾ ವ್ಯರ್ಥಮಾಡಿ\-ಕೊಳ್ಳುವುದಕ್ಕಿಂತ ಇಂಗ್ಲಿಷಿನವರು ತಮ್ಮ ನಾಯಕನಿಗೆ ತೋರುವ ಅಚಂಚಲವಾದ\break ವಿಧೇಯತೆ, ಹಿಡಿದ ಕೆಲಸವನ್ನು ಬಿಡದೆ ಮಾಡುವ ಅದ್ಭುತ ಛಲ ಮತ್ತು ಅವರಲ್ಲಿರುವ\break ಅಚಲವಾದ ಆತ್ಮಶ್ರದ್ಧೆ ಇವುಗಳನ್ನು ಕಲಿಯುವುದು ಮೇಲು. ಅವರು ಯಾವುದಾದರೂ\break ಒಂದು ಕೆಲಸಕ್ಕೆ ಒಬ್ಬ ಮುಂದಾಳುವನ್ನು ಗೊತ್ತುಮಾಡಿದರೆ ಜಯಾಪಜಯಗಳಲ್ಲಿ\break ಯಾವಾಗಲೂ ಅವನನ್ನು ಅನುಸರಿಸುವರು. ಆದರೆ ಇಂಡಿಯಾ ದೇಶದಲ್ಲಿ ಪ್ರತಿಯೊಬ್ಬರೂ\break ನಾಯಕರಾಗಲು ಬಯಸುವರು. ಅಪ್ಪಣೆಯನ್ನು ಪಾಲಿಸುವುದಕ್ಕೆ ಯಾರೂ ಇಲ್ಲ.\break ಪ್ರತಿಯೊಬ್ಬನೂ ಅಪ್ಪಣೆ ಕೊಡುವುದಕ್ಕೆ ಮುಂಚೆ ಆಣತಿಯನ್ನು ಪಾಲಿಸುವುದನ್ನು ಕಲಿಯಬೇಕು. ನಮ್ಮ ಅಸೂಯೆಗೆ ಕೊನೆ ಮೊದಲಿಲ್ಲ. ಹಿಂದೂ ಪ್ರಮುಖನಾದಷ್ಟೂ ಅಸೂಯೆ ಹೆಚ್ಚು. ಅಸೂಯೆ ಇಲ್ಲದಿರುವುದು, ನಾಯಕನ ಅಪ್ಪಣೆಯನ್ನು ಪಾಲಿಸುವುದು ಇವೆರಡನ್ನು ಹಿಂದೂಗಳು ಕಲಿಯುವವರೆಗೆ ನಮ್ಮಲ್ಲಿ ಯಾವ ಸಂಘಟನಾ ಶಕ್ತಿಯೂ ಇರುವುದಿಲ್ಲ.\break ಈಗಿರುವಂತೆ ಯಾವಾಗಲೂ ಅನೈಕಮತ್ಯದಿಂದ, ಗೊಂದಲದಲ್ಲಿರುವ ದೊಂಬಿಯಂತೆ\break ಇರಬೇಕಾಗುವುದು. ಯಾವಾಗಲೂ ಕನಸು ಕಾಣುವುದು; ಕಾರ್ಯತಃ ಮಾತ್ರ ಏನೂ ಇಲ್ಲ. ಭಾರತೀಯರು ಐರೋಪ್ಯರಿಂದ ಬಾಹ್ಯ ಪ್ರಕೃತಿಯನ್ನು ಹೇಗೆ ನಿಗ್ರಹಿಸಬೇಕು ಎಂಬುದನ್ನು ಅರಿಯಬೇಕು. ಐರೋಪ್ಯರು ಭಾರತೀಯರಿಂದ ಆಂತರಿಕ ಪ್ರಕೃತಿಯನ್ನು ಹೇಗೆ ನಿಗ್ರಹಿಸಬೇಕು ಎಂಬುದನ್ನು ಕಲಿತುಕೊಳ್ಳಬೇಕು. ಆಗ ಹಿಂದುವೂ ಇರುವುದಿಲ್ಲ. ಐರೋಪ್ಯರೂ ಇರುವುದಿಲ್ಲ. ಬಾಹ್ಯ ಮತ್ತು ಆಂತರಿಕ ಪ್ರಕೃತಿಯನ್ನು ಗೆದ್ದ ಆದರ್ಶ ಜನಾಂಗವೊಂದು ಇರುವುದು. ನಾವು ಮಾನವಕೋಟಿಯ ಒಂದು ಭಾಗದಲ್ಲಿ ಮುಂದುವರಿದಿರುವೆವು. ಅವರು ಬೇರೊಂದು ಭಾಗದಲ್ಲಿ ಮುಂದುವರಿದಿರುವರು. ಇವೆರಡರ ಸಂಗಮ ನಮಗೆ ಬೇಕಾಗಿರುವುದು. ನಮ್ಮ ಧರ್ಮದ ಪಲ್ಲವಿಯಾದ ಸ್ವಾತಂತ್ರ್ಯ ಎಂಬುದು ದೈಹಿಕ,\break ಮಾನಸಿಕ, ಆಧ್ಯಾತ್ಮಿಕ ಸ್ವಾತಂತ್ರ್ಯ.”

\vskip 0.1cm

ಪ್ರಶ್ನೆ: “ಧರ್ಮಕ್ಕೂ ಕ್ರಿಯಾವಿಧಿಗಳಿಗೂ ಇರುವ ಸಂಬಂಧ ಎಂಥದು?”

\vskip 0.1cm

ಸ್ವಾಮೀಜಿ: “ಕ್ರಿಯಾವಿಧಿಗಳು ಮಕ್ಕಳಿಗೆ ಶಿಶುವಿಹಾರವಿದ್ದಂತೆ. ಈಗಿನ ಸ್ಥಿತಿಯಲ್ಲಿರುವ ಪ್ರಪಂಚಕ್ಕೆ ಇವು ಅತ್ಯಾವಶ್ಯಕ. ನಾವು ಜನರಿಗೆ ಹೊಸ ಹೊಸ ಕ್ರಿಯಾ ವಿಧಿಗಳನ್ನು\break ಕೊಡಬೇಕಾಗಿದೆ. ಕೆಲವು ಮೇಧಾವಿಗಳು ಇದನ್ನು ಕುರಿತು ಆಲೋಚಿಸಬೇಕು. ಹಳೆಯ ವಿಧಿಗಳನ್ನು ಬಿಟ್ಟು ಹೊಸ ವಿಧಿಗಳನ್ನು ತೆಗೆದುಕೊಳ್ಳಬೇಕಾಗಿದೆ.”

\vskip 0.1cm

ಪ್ರಶ್ನೆ: “ಹಾಗಾದರೆ ವಿಧಿಗಳನ್ನು ಬಿಡಬೇಕು ಎಂಬುದನ್ನು ನೀವು ಅನುಮೋದಿಸು\-ತ್ತೀರಾ?”

\vskip 0.1cm

ಸ್ವಾಮೀಜಿ: “ಇಲ್ಲ, ನನ್ನ ಸಂದೇಶ ಧ್ವಂಸವಲ್ಲ, ನಿರ್ಮಾಣ, ಈಗಿರುವ ವಿಧಿಗಳಿಂದ ಹೊಸ ವಿಧಿಗಳನ್ನು ಸೃಷ್ಟಿಸಬೇಕು. ಪ್ರತಿಯೊಬ್ಬರೂ ವಿಕಾಸವಾಗುವುದಕ್ಕೆ ಅನಂತಶಕ್ತಿಯಿದೆ.\break ಇದೇ ನನ್ನ ನಂಬಿಕೆ. ಒಂದು ಪರಮಾಣುವಿನ ಹಿನ್ನೆಲೆಯಲ್ಲಿ ಇಡೀ ವಿಶ್ವಶಕ್ತಿ ನಿಂತಿದೆ.\break ಹಿಂದೂ ಜನಾಂಗದ ಇತಿಹಾಸದಲ್ಲಿ ಹಿಂದಿನಿಂದಲೂ ಎಂದಿಗೂ ಧ್ವಂಸಕಾರ್ಯ ಇರಲಿಲ್ಲ, ಯಾವಾಗಲೂ ಹೊಸದನ್ನು ನಿರ್ಮಿಸುತ್ತಿದ್ದರು. ಒಂದು ಪಂಗಡ ಧ್ವಂಸಕ್ಕೆ ಯತ್ನಿಸಿತು.\break ಭರತಖಂಡ ಅದನ್ನು ಹೊರದೂಡಿತು. ಅದೇ ಬೌದ್ಧಧರ್ಮ. ಶಂಕರ, ರಾಮಾನುಜ, ಚೈತನ್ಯರೆಂಬ ಹಲವು ಸುಧಾರಕರು ಬಂದರು. ಇವರೆಲ್ಲ ದೊಡ್ಡ ದೊಡ್ಡ ಸುಧಾರಕರು. ಅವರು ಯಾವಾಗಲೂ ಸೃಷ್ಟಿ ಮಾರ್ಗವನ್ನು ಅನುಸರಿಸಿದರು; ಅವರು ತಮ್ಮ ಕಾಲಕ್ಕೆ ತಕ್ಕಂತೆ ಸಮಾಜವನ್ನು ರಚಿಸಿದರು. ನಮ್ಮ ಕೆಲಸದ ವಿಶಿಷ್ಟವಾದ ರೀತಿಯೇ ಇದು.\break ಆಧುನಿಕ ಸುಧಾರಕರೆಲ್ಲ ಧ್ವಂಸಕಾರಕ ಐರೋಪ್ಯಕ್ರಾಂತಿಯ ರೀತಿಯನ್ನು ಅನುಸರಿಸುವರು. ಅದು ಹಿಂದೆ ಯಾರಿಗೂ ಒಳ್ಳೆಯದನ್ನು ಮಾಡಿಲ್ಲ ಮತ್ತು ಮುಂದೆ ಮಾಡುವಂತೆಯೂ ಇಲ್ಲ. ಆಧುನಿಕರಲ್ಲಿ ಹೆಚ್ಚು ನಿರ್ಮಾಣಪಂಥದ ಸುಧಾರಕರೆಂದರೆ\break ರಾಜಾರಾಮ ಮೋಹನರಾಯ್​ ಒಬ್ಬರೇ. ಹಿಂದೂ ಜನಾಂಗವು ವೇದಾಂತದ ಆದರ್ಶದ ಕಡೆಗೆ ಮುಂದುವರಿಯುತ್ತಿದೆ. ಭಾರತೀಯರ ಇತಿಹಾಸವೆಲ್ಲ ಸುಖ ಮತ್ತು ದುಃಖದ ಮಾರ್ಗಗಳ ಮೂಲಕ ವೇದಾಂತದ ಆದರ್ಶವನ್ನು ಸಾಕ್ಷಾತ್ಕಾರ ಮಾಡಿಕೊಳ್ಳುವುದಾಗಿದೆ. ಯಾವ ಸುಧಾರಕ ಪಂಥವು ವೇದಾಂತದ ಆದರ್ಶವನ್ನು ಕಿತ್ತೊಗೆಯಿತೊ ಅದು\break ನಿರ್ನಾಮವಾಯಿತು.”

\vskip 0.1cm

ಪ್ರಶ್ನೆ: “ಇಲ್ಲಿ ನೀವು ಹಾಕಿಕೊಂಡಿರುವ ಕಾರ್ಯಯೋಜನೆ ಏನು?”

\vskip 0.1cm

ಸ್ವಾಮೀಜಿ: “ನನ್ನ ಯೋಜನೆಯನ್ನು ಕಾರ್ಯಗತಮಾಡಲು ಒಂದು ಕೇಂದ್ರವನ್ನು ಮದ್ರಾಸಿನಲ್ಲಿ ಮತ್ತೊಂದು ಕೇಂದ್ರವನ್ನು ಕಲ್ಕತ್ತೆಯಲ್ಲಿ ತೆರೆಯಲು ಯತ್ನಿಸುವೆನು.\break ವೇದಾಂತದ ಆದರ್ಶವನ್ನು ಪಾಪಿ ಮತ್ತು ಪುಣ್ಯವಂತ, ಪಂಡಿತ ಮತ್ತು ಪಾಮರ, ಬ್ರಾಹ್ಮಣ ಅಥವಾ ಚಂಡಾಲ ಎಲ್ಲರ ಜೀವನದಲ್ಲಿಯೂ ಕಾರ್ಯಗತವಾಗುವಂತೆ ಮಾಡುವುದೇ ನನ್ನ ಉದ್ದೇಶ.”

ನಮ್ಮ ಪ್ರತಿನಿಧಿಗಳು ಇಲ್ಲಿ ಇಂಡಿಯಾದ ರಾಜಕೀಯಕ್ಕೆ ಸಂಬಂಧಿಸಿದ ಕೆಲವು ಪ್ರಶ್ನೆಗಳನ್ನು ಹಾಕಲು ಯತ್ನಿಸಿದರು. ಸ್ವಾಮೀಜಿಯವರು ಉತ್ತರ ಕೊಡುವುದರೊಳಗೆ ರೈಲು ಎಗ್​ಮೋರ್​ ನಿಲ್ದಾಣವನ್ನು ತಲುಪಿತು. ಅವಸರದಲ್ಲಿ ಅವರು ಇಂಡಿಯಾ ಮತ್ತು\break ಯೂರೋಪುಗಳ ರಾಜಕೀಯ ಜಟಿಲತೆಯಲ್ಲಿ ತಲೆಹಾಕುವುದು ನನಗೆ ಸ್ವಲ್ಪವೂ ಇಷ್ಟವಿಲ್ಲ ಎಂದು ಹೇಳಿದರು. ಅಷ್ಟಕ್ಕೆ ಭೇಟಿ ಕೊನೆಗೊಂಡಿತು.

