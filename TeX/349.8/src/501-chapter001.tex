
\textbf{ಸ್ವಾಮಿ ವಿವೇಕಾನಂದರ ಕವನಗಳ ಕವನರೂಪದ ಕನ್ನಡ ಅನುವಾದ:}\\ಶ‍್ರೀ ಕುವೆಂಪು (ಸಂನ್ಯಾಸಿಗೀತೆ, ನಿರ್ವಾಣಷಟ್ಕಂ, ಶ‍್ರೀರಾಮಕೃಷ್ಣ ಆರಾತ್ರಿಕ)\\ಡಾ.ಜಿ.ಎಸ್.ಶಿವರುದ್ರಪ್ಪ (ಶಿವನರ್ತನ)\\ಡಾ. ಎಚ್.ಎನ್.ಮುರಳೀಧರ (ಇನ್ನುಳಿದ ಕವನಗಳು)

\selecteng

\chapter[IN SEARCH OF GOD]{\enginline{IN SEARCH OF GOD}\protect\footnote{\engfoot{C.W. (Eng) Vol. VII, P. 450}}}

\enginline{O'er hill and dale and mountain range,\\In temple, church, and mosque,\\In Vedas, Bible, Al Koran\\I had searched for Thee in vain.\\Like a child in the wildest forest lost\\I have cried and cried alone,\\'Where art Thou gone, my God, my love?'\\The echo answered, ‘gone'.}

\enginline{And days and nights and years then passed\\A fire was in the brain;\\I knew not when day changed in night,\\The heart seemed rent in twain.\\I laid me down on Ganga's shore,\\Exposed to sun and rain;\\With burning tears I laid the dust\\And wailed with waters' roar.}

\enginline{I called on all the holy names\\Of every clime and creed,\\'Show me the way, in mercy, ye\\Great ones who have reached the goal.'}

\enginline{Years then passed in bitter cry,\\Each moment seemed an age,\\Till one day 'midst my cries and groans\\Some one seemed calling me.}

\enginline{A gentle soft and soothing voice\\That said 'my son',\\'my son', That seemed to thrill in unison\\With all the chords of my soul.}

\enginline{I stood on my feet and tried to find\\The place the voice came from;\\I searched and searched and turned to see\\Round me, before, behind.\\Again, again it seemed to speak–\\The voice divine to me.\\In rapture all my soul was hushed,\\Entrance, enthralled in bliss.}

\enginline{A flash illumined all my soul;\\The heart of my heart opened wide\\O joy, O bliss, what do I find!\\My love, my love, you are here,\\And you are here, my love, my all!}

\enginline{And I was searching Thee!\\From all eternity you were there\\Enthroned in majesty!}

\enginline{From that day forth, where ere I roam,\\I feel Him standing by\\O'er hill and dale, high mount and vale,\\Far far away and high.}

\enginline{The moon's soft light, the stars so bright,\\The glorious orb of day,\\He shines in them; His beauty–might\\Reflected lights are they.\\The majestic morn, the melting eve,\\The boundless billowy sea,\\In nature's beauty, songs of birds,\\I see through them–it is He.}

\enginline{When dire calamity seizes me,\\The heart seems weak and faint,\\All nature seems to crush me down,\\With laws that never bend.}

\enginline{Me seems I hear Thee whispering sweet,\\‘My love, I am near' 'I am near',\\My heart gets strong. With thee, my love,\\A thousand deaths no fear.\\Thou speakest in the mother's lay\\That shuts the babies' eye;\\When innocent children laugh and play\\I see Thee standing by.}

\enginline{When holy friendship shakes the hand,\\He stands between them too;\\He pours the nectar in mother's kiss\\And the babies' sweet 'mama'.\\Thou wert my God with prophets old;\\All creeds do come from Thee;\\The Vedas, Bible, and Koran bold\\Sing Thee in harmony.}

\enginline{'Thou art', 'Thou art' the Soul of souls\\In the rushing stream of life.\\'Om Tat Sat Om.'! Thou art my God,\\My love, I am Thine, I am Thine.}

\selectkan

\begin{center}
\textbf{ಈಶಾನ್ವೇಷಣೆ}
\end{center}

\enginline{'In Search of God'} ಎಂಬ ಈ ಕವನ ಸ್ವಾಮಿಜಿಯವರು ಪ್ರೊ. ಜಾನ್ ಹೆನ್ರಿ ರೈಟ್ ಅವರಿಗೆ ೧೮೯೩ರ ಸೆಪ್ಟೆಂಬರ್ ೪ ರಂದು ಬರೆದ ಪತ್ರದ ಭಾಗವಾಗಿದೆ. ಸ್ವಾಮಿಜಿ ಶಿಕಾಗೋದ ವಿಶ್ವಧರ್ಮ ಸಮ್ಮೇಳನದಲ್ಲಿ ಪಾಲ್ಗೊಳ್ಳಲು ಅನುವಾಗುವಂತೆ ಅದರ ಅಧ್ಯಕ್ಷರಿಗೊಂದು ಪರಿಚಯ ಪತ್ರವನ್ನು ನೀಡಿದವರು ಪ್ರೊ. ರೈಟ್. ಈ ಕವನದ ಬಗ್ಗೆ ಸ್ವಾಮಿಜಿಯವರು ರೈಟ್ ಅವರಿಗೆ ಹೀಗೆ ಬರೆಯುತ್ತಾರೆ: "ಕವನವನ್ನು ರಚಿಸಬೇಕೆಂದು ಪ್ರಯತ್ನಿಸಿ ಬರೆದ ಕೆಲವು ಸಾಲುಗಳು ಇಲ್ಲಿವೆ. ನಿಮ್ಮ ಪ್ರೀತಿ ಈ 'ತೊಂದರೆ'ಯನ್ನು ಮನ್ನಿಸುವುದಾಗಿ ಆಶಿಸುತ್ತೇನೆ."

ಸುದೀರ್ಘವಾದ ಈ ಕವನವನ್ನು ಇಲ್ಲಿ ಎರಡು ಭಾಗಗಳಾಗಿ ವಿಂಗಡಿಸಿಕೊಳ್ಳಲಾಗಿದೆ. ಸತ್ಯಾನ್ವೇಷಣೆಯಲ್ಲಿನ ತಮ್ಮ ತೀವ್ರ ತೊಳಲಾಟವನ್ನೂ ಶ‍್ರೀರಾಮಕೃಷ್ಣರನ್ನು ಕಾಣುವುದರೊಂದಿಗೆ ಆ ಅನ್ವೇಷಣೆ ಸಾರ್ಥಕವಾದದ್ದನ್ನೂ ಸ್ವಾಮಿಜಿ ಮೊದಲ ಭಾಗದಲ್ಲಿ ನಿರೂಪಿಸಿದ್ದಾರೆ. ಶ‍್ರೀರಾಮಕೃಷ್ಣರ ಸಂಪರ್ಕದಿಂದಾಗಿ ಸ್ವಾಮಿಜಿಯವರ ಜೀವನಕ್ಕೆ ಒದಗಿದ ದರ್ಶನ ಹಾಗೂ ಅದರ ವಿಸ್ತಾರಗಳ ವರ್ಣನೆ ಎರಡನೆಯ ಭಾಗದಲ್ಲಿದೆ. ಹೀಗಾಗಿ ಈ ಕವನ ಒಂದು ಬಗೆಯ ಆತ್ಮಚರಿತ್ರಾತ್ಮಕ ಮಹತ್ತ್ವವನ್ನು ಹೊಂದಿದೆ.

\begin{center}
—೧—
\end{center}

\begin{verse}
ಬೆಟ್ಟಗುಡ್ಡಗಳಲ್ಲಿ, ಪರ್ವತದ ತಪ್ಪಲಲಿ,\\ಗುಡಿ ಮಸೀದಿಗಳಲ್ಲಿ, ಚರ್ಚುಗಳಲಿ,\\ವೇದ ಶಾಸ್ತ್ರಗಳಲ್ಲಿ, ಬೈಬಲು ಕುರಾನಿನಲಿ\\ನಿನ್ನನರಸುತ ನೊಂದೆ ವಿಫಲತೆಯಲಿ.
\end{verse}

\begin{verse}
ಘೋರತಮ ವಿಪಿನದಲಿ ದಾರಿತಪ್ಪಿದ ಶಿಶುವಿನಂತೆ\\ನಾನೇಕಾಂಗಿ ಮೊರೆಯುತಿರಲು;\\“ನೀನೆತ್ತ ಹೋಗಿರುವೆ ಎನ್ನ ಪ್ರಭುವೇ?” ಎನಲು\\ಮರುದನಿಯ ಉತ್ತರವು ಬರಿಯ ಬಯಲು.
\end{verse}

\begin{verse}
ಹಗಲು ಇರುಳೂ ಮತ್ತೆ ವರುಷಗಳೆ ಉರುಳಿದುವು–\\ತಲೆಯು ಸಿಡಿವಂತಿತ್ತು ತಾಪದಲ್ಲಿ; \versenum{೧೦}\\ಹಗಲು ಇರುಳಿನ ಮನೆಯ ಹೊಕ್ಕುದರ ಪರಿವಿಲ್ಲ,\\ಎದೆಯು ಚೂರಾಗಿತ್ತು ನೋವಿನಲ್ಲಿ.
\end{verse}

\begin{verse}
ಬಿಸಿಲು ಮಳೆಗಳಿಗೆಲ್ಲ ಮೈಯನೊಡ್ಡುತ ನಾನು\\ಗಂಗೆಯ ದಡದಲ್ಲಿ ಹೊರಳುತಿದ್ದೆ;\\ಸುಡುವ ಕಂಬನಿಯೊಡನೆ ಧೂಳು ಮುಸುಕಿತ್ತೆನಗೆ,\\ನದಿಯ ಮೊರೆತದ ಕೂಡೆ ಅಳುತಲಿದ್ದೆ.
\end{verse}

\begin{verse}
ಮತಪಥಗಳೆಲ್ಲದರ ಪುಣ್ಯನಾಮವ ನಾನು\\ಕರೆಕರೆದು ಕೂಗುತಲಿ ತೊಳಲುತಿದ್ದೆ:\\"ನಿಮ್ಮ ಕೃಪೆಯನು ಬೀರಿ ದಿಟಕೆ ದಾರಿಯ ತೋರಿ,\\ಪರಮ ಸಿದ್ಧರೆ" ಎಂದು ಬೇಡುತಿದ್ದೆ. \versenum{೨೦}
\end{verse}

\begin{verse}
ಕ್ಷಣವೊಂದು ಯುಗವಾಗಿ ಬರಿಯ ಕಹಿಗೋಳಿನಲಿ\\ವರುಷವೆನಿತೋ ಉರುಳಿ ಸಾಗುತಿತ್ತು;\\ಕಡೆಗೊಂದು ದಿನದಲ್ಲಿ ಅಳಲುಗಳ ಮಧ್ಯದಲಿ\\ಯಾರೊ ಕರೆವಂತೆನ್ನ ಕೇಳಿಸಿತ್ತು:
\end{verse}

\begin{verse}
"ನನ್ನ ಮಗು, ಓ ಕಂದ" ಎಂಬ ಮೃದುನುಡಿಯೊಂದು\\ಬೆಂದ ಹೃದಯಕೆ ತಂಪನೆರೆಯುತಿರಲು;\\ಎನ್ನ ಜೀವದ ಶ್ರುತಿಗೆ ಸಮಶ್ರುತಿಯ ನೀಡುವೊಲು\\ಸ್ವರಕೆ ಸ್ವರವೀಯುತದು ಸ್ಪಂದಿಸಿರಲು,
\end{verse}

\begin{verse}
ಮೇಲೆದ್ದು ನಿಂತು ನಾ ಕಣ್ದೆರೆದು ನೋಡಿದೆನು\\ದನಿಯು ನನ್ನೆಡೆ ಬಂದ ತಾಣದೆಡೆಗೆ; \versenum{೩೦}\\ಮೊಗವ ತಿರುಗಿಸಿ ನಿಂತು ಹತ್ತುಕಡೆ ಹುಡುಕಿದೆನು\\ಸುತ್ತಮುತ್ತಲು, ಹಿಂದೆ, ಮೇಲೆ, ಕೆಳಗೆ.
\end{verse}

\begin{verse}
ಬಾರಿಬಾರಿಗು ತಾನು ಮಾತಾಡುವಂತಿತ್ತು\\ಪರಮಧ್ವನಿಯದು ಎನಗೆ ದಿವ್ಯವಾಗಿ;\\ಭಾವದುನ್ಮಾದದಲಿ ಮೂಕವಾದುದು ಜೀವ\\ಮೈಯ ಮರೆತಾನಂದಜಲದಿ ಮುಳುಗಿ.
\end{verse}

\begin{verse}
ಜೀವವೆಲ್ಲವ ಬೆಳಗಿ ಹೊಳೆವ ಕೋಲ್ಮಿಂಚಿನೊಲು\\ಅನುಭಾವದನುಭವವು ಆತ್ಮದಲ್ಲಿ\\ಪ್ರವಹಿಸುತ ಹೃದಯಾಂತರಾಳವನೆ ಹೊಕ್ಕಿತ್ತು\\ಎದೆಯ ಬಾಗಿಲ ತೆರೆದು ಬೆಳಕ ಚೆಲ್ಲಿ! \versenum{೪೦}
\end{verse}

\begin{verse}
ಏನು ನಿರುಪಮ ನಲಿವು, ಎಂಥ ಆನಂದ, ಓ,\\ಇಂದಿಲ್ಲಿ ನಾನೇನ ನೋಡುತಿರುವೆ!\\ಓ ನನ್ನ ಪ್ರಿಯಸಖನೆ, ನನ್ನ ಜೀವದ ಜೀವ,\\ನನ್ನೊಲವೆ ನೀನಿಲ್ಲೆ ವಾಸಿಸಿರುವೆ!
\end{verse}

\begin{verse}
ನನ್ನ ಸರ್ವಸ್ವ ನೀನಿಲ್ಲಿಯೇ ಇರುತಿರಲು\\ಬರಿದೆ ಬಳಲುತಲಿದ್ದೆ ನಿನ್ನನರಸಿ;\\ಯುಗಯುಗಗಳಿಂದಲೂ ಇಲ್ಲೆ ನೀ ನೆಲಸಿರುವ\\ರಾಜಪೀಠದಿ ಹಿರಿಮೆ ಮುಕುಟ ಧರಿಸಿ!~।
\end{verse}

\begin{center}
—೨—
\end{center}

\begin{verse}
ಅಂದಿನಿಂದಲು ನಾನು ಎಲ್ಲಲೆಯುತಿದ್ದರೂ\\ನನ್ನ ಬಳಿಯೆಂದಿಗೂ ಅವನ ನಿಲುವು; \versenum{೫೦}\\ಬೆಟ್ಟಗುಡ್ಡಗಳಲ್ಲಿ, ಪರ್ವತದ ತಪ್ಪಲಲಿ,\\ದೂರ ಸನಿಹಗಳಲ್ಲು ಅವನ ಇರವು!~।
\end{verse}

\begin{verse}
ಚಂದಿರನ ತಂಬೆಳಕು, ನಕ್ಷತ್ರಗಳ ಹೊಳಪು,\\ವೈಭವದ ಹಗಲ ಕಣ್ಗೊಂಬೆಯಲ್ಲಿ\\ಅವನ ತಾ ಹೊಳೆಯುವನು; ಹೊಳೆವುದರ ಬೆಳಕೆಲ್ಲ\\- ಅವನ ಸೌಂದರ್ಯ–ಪ್ರತಿಬಿಂಬದಲ್ಲಿ.
\end{verse}

\begin{verse}
ಶೋಭಿಸುವ ಬೆಳಗಿನಲಿ, ಕರಗುತಿಹ ಸಂಜೆಯಲಿ,\\ಪಾರವಿಲ್ಲದ ಕಡಲಿನಲೆಗಳಲ್ಲಿ,\\ಪ್ರಕೃತಿಸೌಂದರ್ಯದಲಿ, ಹಕ್ಕಿಗಳ ಹಾಡಿನಲಿ,\\ಅವನ ನಾ ಕಾಣುತಿಹೆ – ಎಲ್ಲದರಲಿ. \versenum{೬೦}
\end{verse}

\begin{verse}
ಭೀಕರದ ದುರ್ಭಾಗ್ಯವೆನ್ನ ಮುಸುಕಿರುವಾಗ,\\ದೌರ್ಬಲ್ಯದಿಂದೆದೆಯು ಕುಸಿಯುವಾಗ,\\ಪ್ರಕೃತಿಯೆಲ್ಲವು ತನ್ನ ಕಠಿಣ ನಿಯಮಗಳಿಂದ\\ಎನ್ನ ತಳಕೊತ್ತುತ್ತ ತುಳಿಯುವಾಗ;
\end{verse}

\begin{verse}
'ನಾ ನಿನ್ನ ಬಳಿಯಲ್ಲಿ, ನಿತ್ಯವೂ ಇರುತಿರುವೆ'\\ಎಂಬ ಸವಿನುಡಿಯ ನೀನುಸಿರುತಿರುವೆ;\\ನೀನೆನ್ನ ಜೊತೆಯಿರಲು ಎದೆಗೆ ಬಲ ತುಂಬುವುದು,\\ನೂರು ಸಾವಿಗು ನಾನು ಅಂಜದಿರುವೆ.
\end{verse}

\begin{verse}
ಕಂದನನು ಕಣ್ಮುಚ್ಚಿ ತೊಟ್ಟಿಲಲಿ ಮಲಗಿಸುವ\\ತಾಯ ಜೋಗುಳದಲ್ಲು ನೀನೆ ನುಡಿವೆ; \versenum{೭೦}\\ಮುಗ್ಧ ಮಕ್ಕಳು ಆಟವನ್ನಾಡಿ ನಲಿವಾಗ\\ನೀನು ಬಳಿ ನಿಂದಿಹುದ ನಾ ನೋಡುವೆ.
\end{verse}

\begin{verse}
ಸುಪವಿತ್ರ ಗೆಳೆತನವು ಕೈಕುಲುಕುತಿರುವಾಗ\\ನೀ ನಿಲುವೆ ಆ ಮೈತ್ರಿ ಮಧ್ಯದಲ್ಲಿ;\\ನೀನೆ ಮಧುವನು ಸುರಿವ ತಾಯ ಸಿಹಿಮುತ್ತಿನಲಿ,\\ಮುದ್ದು ಮಗುವಿನ ಸವಿಯ ಮಾತಿನಲ್ಲಿ.
\end{verse}

\begin{verse}
ಅವತರಗಳೊಡನಿರುವ ನೀನೆ ನನ್ನಯ ಪ್ರಭುವು,\\ಜಗದ ಮತಗಳಿಗೆಲ್ಲ ಮೂಲ ನೀನು;\\ಸಾಮಾದಿ ವೇದಗಳು ಬೈಬಲು ಕುರಾನುಗಳು\\ನುತಿಸುತಿರುವುವು ನಿನ್ನ ಮಹಿಮೆಗಳನು! \versenum{೮೦}
\end{verse}

\begin{verse}
ರಭಸದಲಿ ಹರಿಯುತಿಹ ಈ ಬಾಳ ಹೊಳೆಯಲ್ಲಿ\\ಸಕಲ ಜೀವದ ಜೀವ ನೀನೆ ವಿಭುವೆ;\\ಏಕೈಕ ದಿಟದಿರವು ನೀನೆ, ನನ್ನಯ ಗುರುವೆ,\\ನಾನೆಂದು ನಿನ್ನವನು, ನನ್ನ ಒಲವೇ!\\ಓಂ! ತತ್ ಸತ್, ಓಂ!
\end{verse}

\selecteng

\chapter[THE SONG OF THE SANNYASIN]{\enginline{THE SONG OF THE SANNYASIN}\protect\footnote{\engfoot{C.W. Vol. IV, P. 392}}}

\begin{verse}
\enginline{Wake up the note! the song that had its birth\\Far off, where worldly taint could never reach,\\In mountain caves and glades of forest deep,\\Whose calm no sigh for lust or wealth or fame\\Could ever dare to break; where rolled the stream\\Of knowledge, truth, and bliss that follows both.\\Sing high that note, Sannyasin bold! Say – \general{\versenum{"Om Tat Sat, Om!"}} }
\end{verse}

\begin{verse}
\enginline{Strike off thy fetters! Bonds that bind thee down,\\Of shining gold, or darker, baser ore;\\Love–hate, good–bad, and all the dual throng,\\Know, slave is slave caressed or whipped, not free;\\For fetters, though of gold, are not less strong to bind;\\Then off with them, Sannyasin bold! Say – \general{\versenum{"Om Tat Sat, Om!"}} }
\end{verse}

\begin{verse}
\enginline{Let darkness go; the will–o'–the–wisp that leads\\With blinking light to pile more gloom on gloom.\\This thirst for life, for ever quench; it drags\\From birth to death, and death to birth, the soul.\\He conquers all who conquers self. Know this\\And never yield, Sannyasin bold! Say – \general{\versenum{"Om Tat Sat, Om!"}} }
\end{verse}

\begin{verse}
\enginline{“Who sows must reap”, they say, “and cause must bring\\The sure effect; good, good; bad, bad; and none\\Escape the law. But who so wears a form\\Must wear the chain". Too true; but far beyond\\Both name and form is Atman, ever free,\\Know thou art That, Sannyasin bold' Say – \general{\versenum{"Om Tat Sat, Om!”}} }
\end{verse}

\begin{verse}
\enginline{They know not truth, who dream such vacant dreams\\As father, mother, children, wife, and friend.\\The sexless Self! whose father He? whose child?\\Whose friend, whose foe is He who is but One?\\The Self is all in all, none else exists;\\And thou art That, Sannyasin bold! Say – \general{\versenum{"Om Tat Sat, Om!"}} }
\end{verse}

\begin{verse}
\enginline{There is but One–The Free–The Knower–Self!\\Without a name, without a form or stain.\\In Him is Maya, dreaming all this dream.\\The Witness, He appears as nature, soul.\\Know thou art That, Sannyasin bold! Say –\general{\versenum{ “Om Tat Sat, Om!”}} }
\end{verse}

\begin{verse}
\enginline{Where seekest thou? That freedom, friend, this world\\Nor that can give. In books and temples vain\\Thy search. Thine only is the hand that holds\\The rope that drags thee on. Then cease lament,\\Let go thy hold, Sannyasin bold! Say – \general{\versenum{ “Om Tat Sat, Om!”}}\\Say, “Peace to all: From me no danger be\\To aught that lives. In those that dwell on high\\In those that lowly creep, I am the Self in all!\\All life both here and there, do I renounce,\\All heavens and earths and hells, all hopes and fears.”\\Thus cut thy bonds, Sannyasin bold! Say – \general{\versenum{"Om Tat Sat, Om!"}}\\Heed then no more how body lives or goes,\\Its task is done. Let Karma float it down;\\Let one put garlands on, another kick\\This fame; say naught. No praise or blame can be\\Where praiser praised, and blamer blamed are one.\\Thus be thou calm, Sannyasin bold! Say – \general{\versenum{“Om Tat Sat, Om!”}}\\Truth never comes where lust and fame and greed\\Of gain reside. No man who thinks of woman\\As his wife can ever perfect be;\\Nor he who owns the least of things, nor he\\Whom anger chains, can ever pass thro' Maya's gates,\\So, give these up, Sannyasin bold! Say – \general{\versenum{"Om Tat Sat, Om!”}}\\Have thou no home. What home can hold thee, friend?\\The sky thy roof, the grass thy bed, and food\\What chance may bring, well cooked or ill, judge not.\\No food or drink can taint that noble Self\\Which knows Itself. Like rolling river free\\Thou ever be, Sannyasin bold! Say–\\\general{\versenum{“Om Tat Sat, Om!"}}\\Few only know the truth. The rest will hate\\And laugh at thee, great one; but pay no heed.\\Go thou, the free, from place to place, and help\\Them out of darkness, Maya's veil. Without\\The fear of pain or search for pleasure, go\\Beyond them both, Sannyasin bold! Say– \general{\versenum{"Om Tat Sat, Om!”}}\\Thus, day by day, till Karma's powers spent\\Release the soul for ever. No more is birth,\\Nor I, nor thou, nor God, nor man. The “I”\\Has All become, the All is “I” and Bliss.\\Know thou art That, Sannyasin bold! Say– \general{\versenum{"Om Tat Sat, Om!”}} }
\end{verse}

\selectkan

\begin{center}
\textbf{ಸಂನ್ಯಾಸಿಗೀತೆ}\\(ಅನುವಾದ: ಕುವೆಂಪು)
\end{center}

ಇಂಗ್ಲಿಷ್ ಮೂಲದಲ್ಲಿ \enginline{'The Song of the Sannyasin'} ಎಂದು ಪ್ರಸಿದ್ಧವಾಗಿರುವ ಈ ಕವನವನ್ನು ಸ್ವಾಮಿಜಿಯವರು ರಚಿಸಿದ್ದು ೧೮೯೫ರ ಜುಲೈ ತಿಂಗಳಿನಲ್ಲಿ; ನ್ಯೂಯಾರ್ಕಿನ 'ಥೌಸೆಂಡ್ ಐಲೆಂಡ್ ಪಾರ್ಕ್' ಎಂಬಲ್ಲಿ. ಇಲ್ಲಿ ಸ್ವಾಮಿಜಿಯವರು ಏಳು ವಾರಗಳ ಕಾಲ ವಾಸವಾಗಿದ್ದು, ತಮ್ಮ ಕೆಲವು ಪಾಶ್ಚಾತ್ಯ ಶಿಷ್ಯರಿಗೆ ವಿಶೇಷ ಆಧ್ಯಾತ್ಮಿಕ ತರಬೇತಿಯನ್ನು ನೀಡುತ್ತಿದ್ದರು. ಒಂದು ದಿನ ಅವರು ತಮ್ಮ ಶಿಷ್ಯರ ಮುಂದೆ ತ್ಯಾಗಜೀವನದ ವೈಭವವನ್ನು ಬಣ್ಣಿಸುತ್ತಿದ್ದವರು ಇದ್ದಕ್ಕಿದ್ದಂತೆ ಎದ್ದು ಹೊರಟುಹೋದರು; ಸ್ವಲ್ಪ ಸಮಯದಲ್ಲೇ ಅವರು ಈ ಕವನವನ್ನು ರಚಿಸಿದ್ದರು.

\begin{center}
೧
\end{center}

\begin{myquote}
ಏಳು, ಮೇಲೇಳೇಳು ಸಾಧುವೆ, ಹಾಡು ಚಾಗಿಯ ಹಾಡನು;\\ಹಾಡಿನಿಂದೆಚ್ಚರಿಸು ಮಲಗಿಹ ನಮ್ಮ ಈ ತಾಯ್ನಾಡನು!\\ದೂರದಡವಿಯೊಳೆಲ್ಲಿ ಲೌಕಿಕವಿಷಯವಾಸನೆ ಮುಟ್ಟದೊ,\\ಎಲ್ಲಿ ಗಿರಿಗುಹೆಕಂದರದ ಬಳಿ ಜಗದ ಗಲಿಬಿಲಿ ತಟ್ಟದೊ,\\ಎಲ್ಲಿ ಕಾಮವು ಸುಳಿಯದೊ, – ಮೇಣ್\\ಎಲ್ಲಿ ಜೀವವು ತಿಳಿಯದೊ\\ಕೀರ್ತಿ ಕಾಂಚನವೆಂಬುವಾಸೆಗಳಿಂದ ಜನಿಸುವ ಭ್ರಾಂತಿಯ,\\ಎಲ್ಲಿ ಆತ್ಮವು ಪಡೆದು ನಲಿವುದೊ ನಿಚ್ಚವಾಗಿಹ ಶಾಂತಿಯ,\\ನನ್ನಿಯರಿವಾನಂದವಾಹಿನಿಯಲ್ಲಿ ಸಂತತ ಹರಿವುದೊ,\\ಎಲ್ಲಿ ಎಡೆಬಿಡದಿರದ ತೃಪ್ತಿಯ ಝರಿ ನಿರಂತರ ಸುರಿವುದೊ,\\ಅಲ್ಲಿ ಮೂಡಿದ ಹಾಡನುಲಿಯೈ, ಮೀರಿ ಸಂನ್ಯಾಸಿ\\ಓಂ!ತತ್ ಸತ್ ಓಂ!
\end{myquote}

\begin{center}
೨
\end{center}

\begin{myquote}
ಕುಟ್ಟಿ ಪುಡಿಪುಡಿಮಾಡು ಮಾಯೆಯು ಕಟ್ಟಿಬಿಗಿದಿಹ ಹಗ್ಗವ;\\ಕಿತ್ತು ಬಿಸುಡೈ ಹೊಳೆವ ಹೊನ್ನಿನ ಹೆಣ್ಣು ಮಣ್ಣಿನ ಕಗ್ಗವ!\\ಮುದ್ದಿಸಲಿ ಪೀಡಿಸಲಿ ದಾಸನು ದಾಸನೆಂಬುದೆ ಸತ್ಯವು;\\ಕಬ್ಬಿಣವೊ? ಕಾಂಚನವೊ? ಕಟ್ಟಿದ ಕಣ್ಣಿ ಕಣ್ಣಿಯೆ ನಿತ್ಯವು.\\ಪಾಪ ಪುಣ್ಯಗಳೆಂಬವು, – ಮಾ\\ತ್ಸರ್ಯ ಪ್ರೇಮಗಳೆಂಬವು\\ದ್ವಂದ್ವರಾಜ್ಯದ ಧೂರ್ತಚೋರರು; ಬಿಟ್ಟು ಕಳೆ, ಕಳೆಯವರನು!\\ಮೋಹಗೊಳಿಪರು, ಬಿಗಿವರಿರಿವರು; ಎಚ್ಚರಿಕೆಯಿಂದವರನು\\ತಳ್ಳು ದೂರಕೆ, ಓ ವಿರಕ್ತನೆ! ಹಾಡು ಚಾಗಿಯ ಹಾಡನು!\\ಹಾಡಿನಿಂದೆಚ್ಚರಿಸು ಮಲಗಿಹ ನಮ್ಮ ಈ ತಾಯ್ನಾಡನು!\\ಹಾಡು ಮುಕ್ತಿಯ ಗಾನವನು, ಓ ವೀರ ಸಂನ್ಯಾಸಿ\\ಓಂ!ತತ್ ಸತ್ ಓಂ!
\end{myquote}

\begin{center}
೩
\end{center}

\begin{myquote}
ಕತ್ತಲಳಿಯಲಿ; ಮಬ್ಬುಕವಿಸುವ ಭವದ ತೃಷ್ಣೆಯು ಬತ್ತಲಿ;\\ಬಾಳಮೋಹವು ಮರುಮರೀಚಿಕೆ; ಮಾಯೆ ಕೆತ್ತಿದ ಪುತ್ತಳಿ!\\ಜನನದೆಡೆಯಿಂ ಮರಣದೆಡೆಗಾಗೆಳೆವುದೆಮ್ಮನು ದೇಹವು!\\ಜನ್ಮ ಜನ್ಮದಿ ಮರಳಿ ಮರಳುವುದೆಮ್ಮ ಬಿಗಿಯಲು ಮೋಹವು!\\ತನ್ನ ಜಯಿಸಿದ ಶಕ್ತನು – ಅವ\\ನೆಲ್ಲ ಜಯಿಸಿದ ಮುಕ್ತನು!\\ಎಂಬುದನ್ನು ತಿಳಿ; ಹಿಂಜರಿಯದಿರು, ಸಂನ್ಯಾಸಿಯೇ, ನಡೆ ಮುಂದಕೆ.\\ಗುರಿಯು ದೊರಕುವವರೆಗೆ ನಡೆ ನಡೆ; ನೋಡದಿರು ನೀ ಹಿಂದಕೆ.\\ಏಳು ಮೇಲೇಳೇಳು ಸಾಧುವೆ, ಹಾಡು ಚಾಗಿಯ ಹಾಡನು;\\ಹಾಡಿನಿಂದೆಚ್ಚರಿಸು ಮಲಗಿಹ ನಮ್ಮ ಈ ತಾಯ್ನಾಡನು!\\ಹಾಡು ಸಿದ್ಧನೆ, ಓ ಪ್ರಬುದ್ಧನೆ, ಹಾಡು ಸಂನ್ಯಾಸಿ\\ಓಂ ತತ್ ಸತ್!ಓಂ!
\end{myquote}

\begin{center}
೪
\end{center}

\begin{myquote}
“ಬೆಳಯ ಕೊಯ್ವನು ಬಿತ್ತಿದಾತನು; ಪಾಪ ಪಾಪಕೆ ಕಾರಣ;\\ವೃಕ್ಷಕಾರ್ಯಕೆ ಬೀಜ ಕಾರಣ; ಪುಣ್ಯ ಪುಣ್ಯಕೆ ಕಾರಣ;\\ಹುಟ್ಟಿ ಮೈವಡೆದಾತ್ಮ ಬಾಳಿನ ಬಲೆಯ ತಪ್ಪದೆ ಹೊರುವುದು;\\ಕಟ್ಟು ಮೀರಿಹನಾವನಿರುವನು? ಕಟ್ಟು ಕಟ್ಟನೆ ಹೆರುವುದು!''\\ಎಂದು ಪಂಡಿತರೆಂಬರು – ಮೇಣ್\\ತತ್ತ್ವದರ್ಶಿಗಳೆಂಬರು!\\ಆದೊಡೇನಂತಾತ್ಮವೆಂಬುದು ನಾಮರೂಪಾತೀತವು;.\\ಮುಕ್ತಿಬಂಧಗಳಿಲ್ಲದಾತ್ಮವು ಸರ್ವನಿಯಮಾತೀತವು!\\ತತ್ತ್ವಮಸಿ ಎಂದರಿತು, ಸಾಧುವೆ, ಹಾಡು ಚಾಗಿಯ ಹಾಡನು!\\ಹಾಡಿನಿಂದೆಚ್ಚರಿಸು ಮಲಗಿಹ ನಮ್ಮ ಈ ತಾಯ್ನಾಡನು!\\ಸಾರು ಸಿದ್ಧನೆ, ವಿಶ್ವವರಿಯಲಿ!ಹಾಡು ಸಂನ್ಯಾಸಿ\\ಓಂ ತತ್ ಸತ್!ಓಂ!
\end{myquote}

\begin{center}
೫
\end{center}

\begin{myquote}
ತಂದೆ ತಾಯಿಯು ಸತಿಯು ಮಕ್ಕಳು ಗೆಳೆಯರೆಂಬುವರರಿಯರು;\\ಕನಸು ಕಾಣುತಲವರು ಸೊನ್ನೆಯೆ ಸರ್ವವೆನ್ನುತ ಮೆರೆವರು.\\ಲಿಂಗವರಿಯದ ಆತ್ಮವಾರಿಗೆ ಮಗುವು? ಆರಿಗೆ ತಾತನು?\\ಆರ ಮಿತ್ರನು? ಆರ ಶತ್ರುವು? ಒಂದೆಯಾಗಿರುವಾತನು?\\ಆತ್ಮವೆಲ್ಲಿಯು ಇರುವುದು; – ಮೇಣ್\\ಆತ್ಮವೊಂದಾಗಿರುವುದು.\\ಭೇದವೆಂಬುವ ತೋರಿಕೆಯು ನಮ್ಮಾತ್ಮನಾಶಕೆ ಹೇತುವು.\\ಭೇದವನು ತೊರೆದೊಂದೆಯೆಂಬುದನರಿಯೆ ಮುಕ್ತಿಗೆ ಸೇತುವು.\\ಧೈರ್ಯದಿಂದಿದನೆಲ್ಲರಾಲಿಸೆ ಹಾಡು ಚಾಗಿಯ ಹಾಡನು!\\ಹಾಡಿನಿಂದೆಚ್ಚರಿಸು ಮಲಗಿಹ ನಮ್ಮ ಈ ತಾಯ್ನಾಡನು!\\ಸಾರು, ಜೀವನ್ಮುಕ್ತ! ಸಾರೈ ಧೀರ ಸಂನ್ಯಾಸಿ\\ಓಂ ತತ್ ಸತ್!ಓಂ!
\end{myquote}

\begin{center}
೬
\end{center}

\begin{myquote}
ಇರುವುದೊಂದೇ! ನಿತ್ಯಮುಕ್ತನು, ಸರ್ವಜ್ಞಾನಿಯು ಆತ್ಮನು!\\ನಾಮರೂಪಾತೀತನಾತನು; ಪಾಪಪುಣ್ಯಾತೀತನು!\\ವಿಶ್ವಮಾಯಾಧೀಶನಾತನು; ಕನಸು ಕಾಣುವನಾತನು!\\ಸಾಕ್ಷಿಯಾತನು; ಪ್ರಕೃತಿಜೀವರ ತೆರದಿ ತೋರುವನಾತನು!\\ಎಲ್ಲಿ ಮುಕ್ತಿಯ ಹುಡುಕುವೆ? –ಏ–\\ಕಿಂತು ಸುಮ್ಮನೆ ದುಡುಕುವೆ?\\ಇಹವು ತೋರದು, ಪರವು ತೋರದು; ಗುಡಿಯೊಳದು ಮೈದೋರದು.\\ವೇದ ತೋರದು, ಶಾಸ್ತ್ರ ತೋರದು; ಮತವು ಮುಕ್ತಿಯ ತೋರದು!\\ನಿನ್ನ ಕೈಲಿದೆ ನಿನ್ನ ಬಿಗಿದಿಹ ಕಬ್ಬಿಣದ ಯಮಪಾಶವು;\\ಬರಿದೆ ಶೋಕಿಪುದೇಕೆ? ಬಿಡು, ಬಿಡು! ನಿನಗೆ ನೀನೇ ಮೋಸವು!\\ಬೇಡ, ಪಾಶವ ಕಡಿದು ಕೈಬಿಡು! ಹಾಡು ಸಂನ್ಯಾಸಿ\\ಓಂ!ತತ್ ಸತ್!ಓಂ!
\end{myquote}

\begin{center}
೭
\end{center}

\begin{myquote}
"ಶಾಂತಿ ಸರ್ವರಿಗಿರಲಿ" ಉಲಿಯೈ, "ಜೀವಜಂತುಗಳಾಳಿಗೆ\\ಹಿಂಸೆಯಾಗದೆ ಇರಲಿ ಎನ್ನಿಂದೆಲ್ಲ ಸೊಗದಲಿ ಬಾಳುಗೆ!\\ಬಾನೊಳಾಡುವ, ನೆಲದೊಳೋಡುವ ಸರ್ವರಾತ್ಮನು ನಾನಹೆ;\\ನಾಕ ನರಕಗಳಾಸೆಭಯಗಳನೆಲ್ಲ ಮನದಿಂ ದೂಡುವೆ!"\\ದೇಹ ಬಾಳಲಿ, ಬೀಳಲಿ; – ಅದು\\ಕರ್ಮನದಿಯಲಿ ತೇಲಲಿ!\\ಕೆಲರು ಹಾರಗಳಿಂದ ಸಿಂಗರಿಸದನು ಪೂಜಿಸಿ ಬಾಗಲಿ!\\ಕೆಲರು ಕಾಲಿಂದೊದೆದುನೂಕಲಿ! ಹುಡಿಯು ಹುಡಿಯೊಳ ಹೋಗಲಿ!\\ಎಲ್ಲ ಒಂದಿರಲಾರು ಹೊಗಳುವರಾರು ಹೊಗಳಿಸಿಕೊಂಬರು?\\ನಿಂದ ನಿಂದಿಪರೆಲ್ಲ ಕೂಡಲು ಯಾರು ನಿಂದೆಯನುಂಬರು?\\ಪಾಶಗಳ ಕಡಿ! ಬಿಸುಡು, ಕಿತ್ತಡಿ! ಹಾಡು ಸಂನ್ಯಾಸಿ\\ಓಂ ತತ್ ಸತ್ ಓಂ!
\end{myquote}

\begin{center}
೮
\end{center}

\begin{myquote}
ಎಲ್ಲಿ ಕಾಮಿನಿಯಲ್ಲಿ ಕಾಂಚನದಾಸೆ ನೆಲೆಯಾಗಿರುವುದೊ,\\ಸತ್ಯವೆಂಬುವುದಲ್ಲಿ ಸುಳಿಯದು! ಎಲ್ಲಿ ಕಾಮವು ಇರುವುದೊ\\ಅಲ್ಲಿ ಮುಕ್ತಿಯು ನಾಚಿ ತೋರದು! ಎಲ್ಲಿ ಸುಳಿವುದೊ ಭೋಗವು\\ಅಲ್ಲಿ ತೆರೆಯದು ಮಾಯೆ ಬಾಗಿಲನಲ್ಲಿಹುದು ಭವರೋಗವು;\\ಎಲ್ಲಿ ನೆಲಸದೊ ಚಾಗವು, –ದಿಟ–\\ವಲ್ಲಿ ಸೇರದು ಯೋಗವು!\\ಗಗನವೇ ಮನೆ! ಹಸುರೆ ಹಾಸಿಗೆ! ಮನೆಯು ಸಾಲ್ವುದೆ ಚಾಗಿಗೆ?\\ಹಸಿಯೊ, ಬಿಸಿಯೋ? ಬಿದಿಯು ಕೊಟ್ಟಾಹಾರವನ್ನವು ಯೋಗಿಗೆ!\\ಏನು ತಿಂದರೆ, ಏನು ಕುಡಿದರೆ, ಏನು? ಆತ್ಮಗೆ ಕೊರತೆಯೆ?\\ಸರ್ವಪಾಪವ ತಿಂದುತೇಗುವ ಗಂಗೆಗೇಂ ಕೊಳೆ ಕೊರತೆಯೆ?\\ನೀನು ಮಿಂಚೆ! ನೀನು ಸಿಡಿಲೈ! ಮೊಳಗು ಸಂನ್ಯಾಸಿ\\ಓಂ ತತ್ ಸತ್!ಓಂ!
\end{myquote}

\begin{center}
೯
\end{center}

\begin{myquote}
ನಿಜವನರಿತವರೆಲ್ಲೊ ಕೆಲವರು; ನಗುವರುಳಿದವರೆಲ್ಲರೂ\\ನಿನ್ನ ಕಂಡರೆ, ಹೇ ಮಹಾತ್ಮನೆ! ಕುರುಡರೇನನು ಬಲ್ಲರು?\\ಗಣಿಸದವರನು ಹೋಗು, ಮುಕ್ತನೆ, ನೀನು ಊರಿಂದೂರಿಗೆ\\ಸೊಗವ ಬಯಸದೆ, ಅಳಲಿಗಳುಕದೆ! ಕತ್ತಲಲಿ ಸಂಚಾರಿಗೆ\\ನಿನ್ನ ಬೆಳಕನ್ನು ನೀಡಿ; –ಸಂ–\\ಸಾರ ಮಾಯೆಯ ದೂಡಿ!\\ಇಂತು ದಿನದಿನ ಕರ್ಮಶಕ್ತಿಯು ಮುಗಿವವರೆಗೂ ಸಾಗೆಲೈ!\\ನಾನು ನೀನುಗಳಳಿದು ಆತ್ಮದೊಳಿಳಿದು ಕಡೆಯೊಳು ಹೋಗೆಲೈ!\\ಏಳು, ಮೇಲೇಳೇಳು, ಸಾಧುವೆ, ಹಾಡು ಚಾಗಿಯ ಹಾಡನು!\\ಹಾಡಿನಿಂದೆಚ್ಚರಿಸು ಮಲಗಿಹ ನಮ್ಮ ಈ ತಾಯ್ನಾಡನು!\\ತತ್ತ್ವಮಸಿ ಎಂದರಿತು ಹಾಡೈ, ಧೀರ ಸಂನ್ಯಾಸಿ\\ಓಂ ತತ್!ಸತ್!ಓಂ!
\end{myquote}

\selecteng

\chapter[MANY HAPPY RETURNS]{\enginline{MANY HAPPY RETURNS}\protect\footnote{\engfoot{C.W, Vol. VII, 526}}}

\begin{myquote}
\enginline{The mother's heart, the hero's will,\\The softest flower's sweetest feel;\\The charm and force that ever sway\\The altar fire's flaming play;\\The strength that leads, in love obeys;\\Far reaching dreams, and patient ways,\\Eternal faith in self, in all\\The light Divine is great in small;\\All these, and more than I could see\\Today may Mother grant to thee.}
\end{myquote}

\begin{flushright}
\enginline{Ever yours with love and blessings,\\\textbf{VIVEKANANDA}}
\end{flushright}

\selectkan

\begin{center}
\textbf{ಶುಭ ಹಾರೈಕೆ}
\end{center}

ಸ್ವಾಮಿಜಿಯವರು ಪ್ಯಾರಿಸಿನಿಂದ ಮುನ್ನೂರು ಮೈಲಿ ದೂರದಲ್ಲಿ ಇಂಗ್ಲಿಷ್ ಕಾಲುವೆಯ ತೀರದ ಬ್ರಿಟಾನಿ ಎಂಬಲ್ಲಿದ್ದಾಗ ೧೯೦೦ರ ಸಪ್ಟೆಂಬರ್ ೨೨ರಂದು \enginline{'A Benediction'} ಮತ್ತು \enginline{'Many Happy Returns'} ಎಂಬ ಎರಡು ಇಂಗ್ಲಿಷ್ ಕವನಗಳನ್ನು ಬರೆದರು. ಈ ಎರಡೂ ಕವನಗಳ ನಡುವೆ ಬಹಳಷ್ಟು ಹೋಲಿಕೆಯಿರುವುದರಿಂದ ಇವನ್ನು ಒಂದೇ ಕವನದ ಎರಡು ರೂಪಗಳೆಂದರೂ ಸರಿಯೆ. ಒಂದನ್ನು ಸ್ವಾಮಿಜಿ ಸೋದರಿ ನಿವೇದಿತಗಾಗಿ ಬರೆದರು. ಇನ್ನೊಂದು ಕುಮಾರಿ ಅಲ್ಬರ್ಟಾ ಸ್ಟರ್ಜೆಸ್ ಎಂಬಾಕೆಗೆ ಹುಟ್ಟುಹಬ್ಬಕ್ಕೆಂದು ನೀಡಿದ್ದು.

ಆಲ್ಬರ್ಟಾಗೆ ನೀಡಿದ ಕವನದ ಕೊನೆಯಲ್ಲಿ ಸ್ವಾಮಿಜಿ ಬರೆದಿದ್ದರು: "ನಿನ್ನ ಹುಟ್ಟುಹಬ್ಬಕ್ಕಾಗಿ ಈ ಚಿಕ್ಕ ಕವನ. ಇದು ಅಷ್ಟೇನೂ ಚೆನ್ನಾಗಿಲ್ಲದಿದ್ದರೂ ನನ್ನ ಪ್ರೀತಿಯಿಂದ ತುಂಬಿದೆ. ಆದ್ದರಿಂದ ನೀನು ಅದನ್ನು ಖಂಡಿತವಾಗಿಯೂ ಮೆಚ್ಚುವೆಯೆಂಬ ಭರವಸೆ ನನಗಿದೆ."

\begin{myquote}
ಮಾತೆಯೆದೆ, ಧೀರಾತ್ಮನಲ್ಲಿರುವ ಸಂಕಲ್ಪ,\\ಮೃದು ಮಧುರ ಕುಸುಮಗಳ ಮಧುರತರ ಸ್ಪರ್ಶ,\\ಯಜ್ಞಕುಂಡದಿ ನೆಲಸಿ ಜ್ವಾಲೆಗಳನಾಡಿಸುತ\\ಸೆಳೆವ ಮಾಂತ್ರಿಕ ಶಕ್ತಿ,\\ಮುಂದೆ ನಡೆಸುವ ಬಲವು, ಮಣಿವ ಒಲವು,\\ಮೇರೆ ಮೀರಿದ ಕನಸು,\\ತಾಳ್ಮೆಯಿಂದಿಡುವ ಅಡಿ,\\ಅಣುವಿನಲಿ ಮಹತಿನಲಿ\\ತಾನಾಗಿ ಬೆಳಗುತಿಹ ದಿವ್ಯಾತ್ಮದಲಿ ಶ್ರದ್ಧೆ\\-ಇನಿತೆಲ್ಲ, ಇನಿತನೂ ಮೀರ್ದ ಒಳಿತುಗಳೆಲ್ಲ\\ನಿನ್ನದಾಗಲಿ ತಾಯ ಕೃಪೆಯಿಂದಲಿ!
\end{myquote}

\selecteng

\chapter[AN INTERESTING CORRESPONDENCE]{\enginline{AN INTERESTING CORRESPONDENCE}\protect\footnote{\engfoot{C.W, Vol. VIII, P.162}}}

\begin{myquote}
\enginline{Now Sister Mary,\\You need not be sorry\\For the hard raps I gave you,\\You know full well,\\Though you like me tell,\\With my whole heart I love you.}
\end{myquote}

\begin{myquote}
\enginline{The babies I bet,\\The best friends I met,\\Will stand by me in weal and woe.\\And so will I do,\\You know it too.}
\end{myquote}

\begin{myquote}
\enginline{Life, name, or fame, even heaven forgo\\For the sweet sisters four\\Sans reproche et sans peur,\\The truest, noblest, steadfast, best.}
\end{myquote}

\begin{myquote}
\enginline{The wounded snake its hood unfurls,\\The flame stirred up doth blaze,\\The desert air resounds the calls\\Of heart–struck lion's rage.}
\end{myquote}

\begin{myquote}
\enginline{The cloud puts forth its deluge strength\\When lightning cleaves its breast,\\When the soul is stirred to its inmost depth\\Great ones unfold their best.}
\end{myquote}

\begin{myquote}
\enginline{Let eyes grow dim and heart grow faint,\\And friendship fail and love betray,\\Let Fate its hundred horrors send,\\And clotted darkness block the way.}
\end{myquote}

\begin{myquote}
\enginline{All nature wear one angry frown,\\To crush you out–still know, my soul,\\You are Divine. March on and on,\\Nor right nor left but to the goal.}
\end{myquote}

\begin{myquote}
\enginline{Nor angel I, nor man, nor brute,\\Nor body, mind, nor he or she,\\The books do stop in wonder mute\\To tell my nature; I am He.}
\end{myquote}

\begin{myquote}
\enginline{Before the sun, the moon, the earth,\\Before the stars or comets free,\\Before e'en time has had its birth,\\I was, I am, and I will be.}
\end{myquote}

\begin{myquote}
\enginline{The beauteous earth, the glorious sun,\\The calm sweet moon, the spangled sky,\\Causation's laws do make them run;\\They live in bonds, in bonds they die.}
\end{myquote}

\begin{myquote}
\enginline{And mind its mantle dreamy net\\Cast o'er them all and holds them fast,\\In warp and woof of thought are set,\\Earth, hells, and heavens, or worst or best.}
\end{myquote}

\begin{myquote}
\enginline{Know these are but the outer crust\\All space and time, all effect, cause.\\I am beyond all sense, all thoughts,\\The witness of the universe.}
\end{myquote}

\begin{myquote}
\enginline{Not two or many, 'tis but one,\\And thus in me all me's I have;\\I cannot hate, I cannot shun\\Myself from me, I can but love.}
\end{myquote}

\begin{myquote}
\enginline{From dreams awake, from bonds be free.\\Be not afraid. This mystery,\\My Shadow, cannot frighten me,\\Know once for all that I am He.}
\end{myquote}

\begin{myquote}
\enginline{Well, so far my poetry. Hope you are all right. Give my love to mother and Father Pope. I am busy unto death and have almost no time to write even a line. So excuse me if later on I am rather late in writing.}
\end{myquote}

\begin{flushright}
\enginline{Yours eternally,\\\textbf{VIVEKANANDA}}
\end{flushright}

\begin{center}
\enginline{Miss M.B.H. sent Swami the following doggerel in reply:}
\end{center}

\begin{myquote}
\enginline{The monk he would a poet be\\And wooed the muse right earnestly;\\In thought and word he could well beat her,\\What bothered him though was the metre.}
\end{myquote}

\begin{myquote}
\enginline{His feet were all too short too long.\\The form not suited to his song;\\He tried the sonnet, lyric, epic,\\And worked so hard, he waxed dyspeptic.}
\end{myquote}

\begin{myquote}
\enginline{While the poetic mania lasted\\He e'en from vegetables fasted,\\Which Leon\supskpt{\footnote{\enginline{Leon Landsberg, a disciple of the Swami who lived with him for some time.}}} had with tender care\\Prepared for Swami's dainty fare.}
\end{myquote}

\begin{myquote}
\enginline{One day he sat and mused along–\\Sudden a light around him shone,\\The “still small voice” his thoughts inspire\\And his words glow like coals of fire.}
\end{myquote}

\begin{myquote}
\enginline{And coals of fire they proved to be\\Heaped on the head of contrite me\\My scolding letter I deplore\\And beg forgiveness o'er and o'er.}
\end{myquote}

\begin{myquote}
\enginline{The lines you sent to your sisters four\\Be sure they'll cherish evermore\\For you have made them clearly see\\The one main truth that “all is He”.}
\end{myquote}

\enginline{Then Swami:}

\begin{myquote}
\enginline{In days of yore,\\On Ganga's shore preaching,\\A hoary priest was teaching\\How Gods they come\\As sita Ram\\And gentle Sita pining, weeping,\\The sermons end,\\They homeward wend their way–\\The hearers musing, thinking.}
\end{myquote}

\begin{myquote}
\enginline{When from the crowd\\A Voice aloud\\This question asked beseeching, seeking–\\“Sir, tell me, pray,\\who were but they\\These Sita Ram you were teaching, speaking!”}
\end{myquote}

\begin{myquote}
\enginline{So Mary Hale,\\Allow me tell,\\You mar my doctrines wronging, baulking.\\I never taught\\Such queer thought\\That all was God–unmeaning talking!}
\end{myquote}

\begin{myquote}
\enginline{But this I say,\\Remember pray,\\That God is true, all else is nothing,\\This world's a dream Though true it seem,\\And only truth is He the living!\\The real me is none but He,\\And never, never matter changing!\\With undying love and gratitude to you all...}
\end{myquote}

\begin{flushright}
\enginline{VIVEKANANDA}
\end{flushright}

\enginline{And then Miss M.B.H.:}

\begin{myquote}
\enginline{The difference I clearly see\\'Twixt tweedledum and tweedledee–\\That is a proposition sane,\\But truly 'tis beyond my vein\\To make your Eastern logic plain.}
\end{myquote}

\begin{myquote}
\enginline{If “God is truth, all else is naught, "\\This "world a dream”, delusion up wrought,\\What can exist which God is not?}
\end{myquote}

\begin{myquote}
\enginline{All those who “many" see have much to fear,\\He only lives to whom the “One” is clear.\\So again I say\\In my poor way,\\I cannot see but that all's He,\\If I'm in Him and He in me.}
\end{myquote}

\enginline{Then the Swami replied:}

\begin{myquote}
\enginline{Of temper quick, a girl unique,\\A freak of nature she,\\A lady fair, no question there,\\Rare soul is Miss Mary.\\Her feelings deep she cannot keep,\\But creep they out at last,\\A spirit free, I can foresee,\\Must be of fiery cast.}
\end{myquote}

\begin{myquote}
\enginline{Tho' many a lay her muse can bray,\\And play piano too,\\Her heart so cool, chills as a rule\\The fool who comes to woo.\\Though, Sister Mary, I hear they say\\The sway your beauty gains,\\Be cautious now and do not bow,\\However Sweet, to chains.}
\end{myquote}

\begin{myquote}
\enginline{For 'twill be soon, another tune\\The moon–struck mate will hear\\If his will but clash, your words will hash\\And smash his life I fear.\\These lines to thee, Sister Mary,\\Free will I offer, take\\"Tit for tat" –a monkey chat,\\For monk alone can make.}
\end{myquote}

\selectkan

\begin{center}
\textbf{ಒಂದು ಸ್ವಾರಸ್ಯಕರ ಪತ್ರವ್ಯವಹಾರ}
\end{center}

ಈ ಪತ್ರ ವ್ಯವಹಾರಕ್ಕೆ ಒಂದು ಹಿನ್ನೆಲೆಯಿದೆ. ಸ್ವಾಮಿಜಿಯವರು ತಮ್ಮ ವೇದಾಂತದ ದೃಷ್ಟಿಕೋನವನ್ನು ತುಂಬ ಬಲವಾಗಿ ಪ್ರತಿಪಾದಿಸಿದಲ್ಲಿ ಅದರಿಂದ ಅಮೆರಿಕನ್ ಸಮಾಜಕ್ಕೆ ಆಘಾತವಾಗಬಹುದೆಂದೂ, ಆದ್ದರಿಂದ ಅವರು ತಮ್ಮ ಪ್ರತಿಪಾದನೆಯನ್ನು ಸ್ವಲ್ಪ ನಯಗೊಳಿಸಿಕೊಳ್ಳುವುದು ಲೇಸೆಂದೂ ಕುಮಾರಿ ಹೇಲ್ ಸ್ವಾಮಿಜಿಯವರಿಗೆ ಸಲಹೆ ನೀಡಿದ್ದಳು. ಸಮಾಜ ಸತ್ಯಕ್ಕೆ ಋಣಿಯಾಗಿರಬೇಕೇ ಹೊರತು ಸತ್ಯವು ಸಮಾಜಕ್ಕೆ ಋಣಿಯಾಗಿರಬೇಕಿಲ್ಲ ಎನ್ನುವ ಸ್ವಾಮಿಜಿಯವರು ೧೮೯೫ರ ಫೆಬ್ರವರಿ ೧ರಂದು ಕುಮಾರಿ ಹೇಲ್‌ಗೆ ಬರೆದ ಪತ್ರದಲ್ಲಿ ತಮ್ಮ ಸತ್ಯನಿಷ್ಠೆಯನ್ನು ಓರ್ವ ನಿರ್ಭೀತ ಸಂನ್ಯಾಸಿಯ ದೃಷ್ಟಿಯಿಂದ ತೀವ್ರವಾಗಿಯೇ ಸಮರ್ಥಿಸಿಕೊಂಡಿದ್ದರು. ಇದರಿಂದ ಕುಮಾರಿ ಹೇಲ್ ಸ್ವಲ್ಪ ಪೆಚ್ಚಾದಾಗ, ಸ್ವಾಮಿಜಿ ತಮ್ಮ ನಿಲುವನ್ನು ಸಡಿಲಿಸದಂತೆ, ಆದರೆ ಆಕೆಗೆ ಒಂದಿಷ್ಟು ಸಾಂತ್ವನವಾಗಲೆಂದು ಕವನರೂಪದ ಈ ಪತ್ರವನ್ನು ಬರೆದರು. ಎಲ್ಲ ಬಗೆಯ ದ್ವಂದ್ವಗಳನ್ನು ದಾಟಿ ಸತ್ಯದಲ್ಲಿ ನೆಲೆಗೊಂಡ ಮುಕ್ತನ ಸ್ಥಿತಿಯನ್ನು ಈ ಕವನದಲ್ಲಿ ಅತ್ಯಂತ ಉಜ್ವಲವಾಗಿ ಚಿತ್ರಿಸಲಾಗಿದೆ.

\begin{myquote}
ಇನ್ನು, ಸಹೋದರಿ, ಮೇರಿ,\\ನಾನು ಕೊಟ್ಟ ಆ ಮಾತಿನ ಪೆಟ್ಟಿಗೆ\\ಮನವು ಮುದುಡದಿರಲಿ!\\ನೀನು ಬಲ್ಲೆ ಬಲು ಚೆನ್ನಾಗಿ–
\end{myquote}

\begin{myquote}
ಆದರು ನಾನಿದ ನುಡಿಯಲೆಂದು ನೀ\\ಮನದಿ ತವಕಿಸಿರುವೆ\\ಹೃದಯದಾಳದಿಂ ನನ್ನ ಪ್ರೀತಿಯದು\\ಹರಿದಿದೆ ನಿನ್ನಡೆಗೆ!\\ನನಗೆ ದೊರೆತ ನಿಜ ಸ್ನೇಹದ ಖನಿಗಳು\\ಸೋದರಿಯರು ನೀವು; ಆಣೆಯಿಡುವೆ ನಾನು!\\ನೋವು – ನಲಿವಿನಲಿ ನನ್ನೊಡನಿರುವಿರಿ,\\ಅಂತೆಯೆ ನಾ ನಿಮಗೆ;\\ಇದು ತಿಳಿಯದೆ ನಿಮಗೆ?
\end{myquote}

\begin{myquote}
ನನ್ನೀ ಜೀವನ, ಹೆಸರು – ಕೀರ್ತಿಯನು,\\ಸಗ್ಗವನೂ ತೊರೆವೆ;\\ದಿಟರು, ಉನ್ನತರು, ದೃಢರು, ಉತ್ತಮರು\\ಸೋದರಿಯರು ಈ ನಾಲ್ವರ ನಾ ತೊರೆಯ!
\end{myquote}

\begin{myquote}
ಕೆರಳಿದ ನಾಗರ ಹೆಡೆಯೆತ್ತುವುದು,\\ಕೆದಕಿದ ಬೆಂಕಿಯು ಪ್ರಜ್ವಲಿಸುವುದು,\\ಘಾತದಿ ನೊಂದಿಹ ಕೇಸರಿ ಗರ್ಜಿಸೆ\\ಮರುಧರೆ ಮಾರುತ ಮಾರ್ದನಿಸುವುದು.
\end{myquote}

\begin{myquote}
ಮೋಡಗಳೊಡಲನು ಮಿಂಚದು ಭೇದಿಸೆ\\ಸುರಿಮಳೆ ವಿಪ್ಲವ ಧರೆಗಿಳಿಯುವುದು.\\ಜೀವದಾಳವನು ಕಲಕಲು ಮಹಿಮರ\\ಗುಪ್ತಶಕ್ತಿ ಬಾನೆತ್ತರ ಪುಟಿವುದು.
\end{myquote}

\begin{myquote}
ಕಣ್ ಮಂಜಾದರು, ಹೃದಯವೆ ಕುಸಿದರು,\\ಸ್ನೇಹ ಸೋತು ಕಳೆಯಗಲಿದರು,\\ಕ್ರೌರ್ಯಸಾಸಿರವ ವಿಧಿಯಟ್ಟಿದರೂ,\\ಹೆಪ್ಪುಗಟ್ಟಿ ನಿಶೆಯಡ್ಡನಿಂದರೂ,
\end{myquote}

\begin{myquote}
ಪ್ರಕೃತಿಯದೆಲ್ಲವು ಬದ್ಧಭ್ರುಕುಟಿಯಲಿ\\ತುಳಿಯುತ ಪುಡಿಪುಡಿ ಮಾಡಿದರೂ–\\ದಿವ್ಯಾತ್ಮನು ನೀನೆಂಬುದನರಿಯುತ\\ಮುಂದೆ ಮುಂದೆ ನಡೆ ಗುರಿಯೆಡೆಗೆ!
\end{myquote}

\begin{myquote}
ದೇವನು–ಮನುಜನು–ಪ್ರಾಣಿಯಲ್ಲಿ ನಾ\\ದೇಹ–ಮನಸು–ಹೆಣ್–ಗಂಡಲ್ಲ;\\ನನ್ನ ಬಣ್ಣಿಸಲು ಬಾಯ್ದೆರೆದಾ ಶ್ರುತಿ\\ಮೂಕವಾಗಿಹುದು – ನಾ ಶಿವನು!
\end{myquote}

\begin{myquote}
ರವಿ–ಶಶಿ–ತಾರಕ–ಗ್ರಹ–ನೀಹಾರಿಕೆ–\\ಗಳಿಗಿಂತಲು ನಾ ಮೊದಲಿಗನು;\\ಕಾಲನ ಜನ್ಮಕು ಮೊದಲಿದ್ದೆನು, ನಾ–\\ನಿರುವೆನೀಗ, ಮುಂದಿರುತಿರುವೆ!
\end{myquote}

\begin{myquote}
ಸುಂದರ ಧರೆ, ಕೋರೈಸುವ ಸೂರ್ಯನು,\\ತಂಬೆಳಕಿನ ಶಶಿ, ಮಿಂಚುವ ಬಾನ್\\ಕಾರ್ಯಕಾರಣದ ಸೆರೆಯೊಳು ಸಿಲುಕಿವೆ,\\ಹುಟ್ಟುತ ಸಾವವು ಸೆರೆಯೊಳಗೆ!
\end{myquote}

\begin{myquote}
ಮನದ ಹೊದಿಕೆಯಲಿ ಕನಸಿನ ಬಲೆಯಲಿ\\ಚಿಂತನೆಯೆಳೆಗಳ ಹಾಸುಹೊಕ್ಕಿನಲಿ\\ನಾಕನರಕಗಳು, ಧರೆಯಾಗಸಗಳು\\ಒಳಿತು ಕೆಡುಕುಗಳು ಸಿಲುಕಿಹವು!
\end{myquote}

\begin{myquote}
ಹೊರಗಿನ ಪೊರೆಯಿವು – ಕಾಲದೇಶಗಳು,\\ಕಾರ್ಯಕಾರಣದ ಚಕ್ರಗಳು;\\ಇಂದ್ರಿಯಕೆಟುಕದೆ, ಮನಸಿಗೆ ನಿಲುಕದೆ\\ವಿಶ್ವಸಾಕ್ಷಿ ನಾನಿರುತಿಹೆನು!
\end{myquote}

\begin{myquote}
ಎರಡಲ್ಲವು, ಹಲವಲ್ಲ, ಒಂದೆ ನಾ,\\ಎಲ್ಲ ನಾನುಗಳು ನನ್ನಲ್ಲೇ;\\ಹಗೆಯೆನಗಿಲ್ಲವು, ನಾನೇ ಎಲ್ಲವು,\\ಪ್ರೀತಿ ಮಾತ್ರವನೆ ಕೊಡಬಲ್ಲೆ!
\end{myquote}

\begin{myquote}
ಕನಸ ಹರಿದು, ಪಾಶಗಳ ಕಡಿದು ನಡೆ,\\ಗುಟ್ಟುಗಳಿವು ನನ್ನಯ ನೆರಳು;\\ನನ್ನ ನೆರಳುಗಳಿಗಂಜುವೆನೇ? ನಾ\\ನೆಂದೆಂದಿಗು ಶಿವ; ಇದೆ ತಿರುಳು!
\end{myquote}

ಒಳ್ಳೆಯದು; ಇದಿಷ್ಟು ನನ್ನ ಕವನವಾಯಿತು. ನೀವೆಲ್ಲರೂ ಕುಶಲವೆಂದು ಭಾವಿಸುವೆ. ಮದರ್ ಚರ್ಚ್ ಮತ್ತು ಫಾದರ್ ಪೋಪ್ ಗೆ ನನ್ನ ವಿಶ್ವಾಸಗಳನ್ನು ತಿಳಿಸು. ಸಾಯುವ ತನಕವೂ ನಾನೆಷ್ಟು ಕಾರ್ಯನಿರತನೆಂದರೆ ಒಂದೇ ಒಂದು ಸಾಲು ಬರೆಯುವುದಕ್ಕೂ ನನಗೆ ಬಿಡುವಿಲ್ಲ. ಆದ್ದರಿಂದ, ಮುಂದೆ ನಾನು ಪತ್ರ ಬರೆಯುವುದು ತಡವಾದರೆ ಕ್ಷಮಿಸು.

\begin{flushright}
ಎಂದೆಂದಿಗೂ ನಿಮ್ಮ\\ವಿವೇಕಾನಂದ
\end{flushright}

ಇದಕ್ಕೆ ಉತ್ತರವಾಗಿ ಮೇರಿ ಹೇಲ್ ಸ್ವಾಮಿಜಿಯವರಿಗೆ ಈ 'ಕುಹಕಕಾವ್ಯ'ವನ್ನು ಕಳಿಸುತ್ತಾಳೆ.

\begin{myquote}
ಸಂನ್ಯಾಸಿಯಾತ ತಾ ಕವಿಯಾಗಲೆಳಸುತ್ತ\\ಕಾವ್ಯದೇವತೆಯನ್ನು ಓಲೈಸಿದ;\\ಭಾಷೆ–ಚಿಂತನೆಗಳಲಿ ಆಕೆಗವ ಸರಿಮಿಗಿಲು,\\ಛಂದಸ್ಸಿಗಾಗಿ ತುಸು ಪರದಾಡಿದ!
\end{myquote}

\begin{myquote}
ಸಾಲುಗಳು ಬಲು ಉದ್ದ, ಇಲ್ಲವೋ, ಅತಿ ಗಿಡ್ಡ\\ಹಾಡಿಗೊಪ್ಪುವ ಛಂದ ದೊರಕಲಿಲ್ಲ;\\ಅಷ್ಟಷಟ್ಟದಿ–ಭಾವಗೀತೆಗಳ–ಕಬ್ಬಗಳ\\ಮರಮರಳಿ ಯತ್ನಿಸುತ ಪಡುತಿದ್ದ ಪಾಡಿನಲಿ\\ಜೀರ್ಣಾಗ್ನಿಯಾತನದು ಮಂದವಾಯ್ತು!
\end{myquote}

\begin{myquote}
ಕಾವ್ಯರಚನೆಯ ಗೀಳು ತಾನೊಂದೆ ಉಳಿ\\ರಸಗವಳವೂ ಕೂಡ ರುಚಿಸಲಿಲ್ಲ;\\ನಿರಶನದ ಹಾದಿಯನು ಹಿಡಿದನಲ್ಲ!
\end{myquote}

\begin{myquote}
ಒಂದು ದಿನ, ಏಕಾಂಗಿ, ಮುಳುಗಿರಲು ಚಿಂತನದಿ\\ಇದ್ದಕಿದ್ದೊಲೆ ಬೆಳಕು ಹೊಳೆದು ಹರಡಿ\\ಉರಿವ ಕೆಂಡಗಳಂತೆ ಬೆಳಗುತಿಹ ನುಡಿ ಸಿಡಿದು\\ನನ್ನೆದೆಯೊಳಡಗಿದ್ದ 'ಪಿಸುದನಿ'ಯನೆಚ್ಚರಿಸಿ\\ಸ್ಫೂರ್ತಿಗೊಳಿಸಿದವು!\\ಕಾದ ಕೆಂಡದ ನುಡಿಗಳವೆ ತಾನು–ಮತ್ತೇನು?
\end{myquote}

\begin{myquote}
ಪರಿತಾಪಿಯಾದೆನ್ನ ಶಿರದ ಮೇಲೆ\\ಸುರಿಯುತಿರೆ ನನ್ನ ಮುನಿಸಿನ ಓಲೆಗಾಗಿ ನಾ\\ಕ್ಷಮೆಯ ಕೋರುವೆನಿಂದು ಮೇಲೆ ಮೇಲೆ!\\ಈ ನಾಲ್ವರಿಗೆ – ನಿಮ್ಮ ಸೋದರಿಯರಿಗೆ ನೀವು\\ಕಳಿಸಿರುವ ಸಾಲುಗಳು ಚಿರದ ನಿಧಿಯ;\\'ಎಲ್ಲವೂ ಅವನೆಂ'ಬ ಮುಖ್ಯ ಸತ್ಯವ ತೋರಿ\\ಕಣ್ತೆರೆಸಿಹಿರಿ ನೀವು, ಇದುವೆ ಸಿರಿಯು!
\end{myquote}

\begin{myquote}
ಬಳಿಕ ಸ್ವಾಮಿಜಿ:\\ಒಂದಾನೊಂದು ಕಾಲದಲ್ಲಿ, ಗಂಗೆಯ ದಡದಲ್ಲಿ,\\ಮುದಿ ಪೂಜಾರಿಯು ಬೋಧಿಸುತ್ತಿದ್ದನು–\\ದೇವದೇವಿಯರು ರಾಮಸೀತೆಯರು\\ಬಂದ ಪರಿಯದೆಂತು,\\ಸಾದ್ವಿಶಿರೋಮಣಿ ಸೀತಾಮಾತೆಯು\\ಅತ್ತ ಪರಿಯದೆಂತು.
\end{myquote}

\begin{myquote}
ಪ್ರವಚನ ಮುಗಿಯಿತು; ಗಹನಾಲೋಚನೆ–\\ಗೈಯುತ ಜನರೆಲ್ಲ\\ಮನೆಕಡೆ ನಡೆದಿರೆ\\ಗುಂಪಿನ ಮಧ್ಯದಿ\\ಕೂಗು ಬಂದಿತೊಂದು – ಪ್ರಶ್ನೆಗೆ\\ಉತ್ತರ ಬೇಡಿತದು:\\"ಸ್ವಾಮೀ, ಹೇಳಿ, ಬೇಡುವೆ ನಿಮ್ಮ,
\end{myquote}

\begin{myquote}
ನೀವು ಪೇಳ್ದ ಆ ಸೀತಾರಾಮರು\\ಯಾರವರು ಸ್ವಾಮೀ? ದಯ\\ಮಾಡಿ ತಿಳಿಸಿ, ಸ್ವಾಮೀ!"\\ಅಂತೆಯೆ, ಮೇರೀ,\\ಪೇಳುವೆನಿದ ತಿಳಿ\\ನನ್ನ ತತ್ತ್ವಗಳನ್ನು\\ತಪ್ಪುತಿಳಿವೆ ನೀನು;\\ಎಲ್ಲವು ದೇವರು – ದಿಂಡರು ಎಂದು\\ಚಿತ್ರವಿಚಿತ್ರದ ಸಿದ್ಧಾಂತಗಳನ್ನು\\ಬೋಧಿಸಿಲ್ಲ ನಾನು!
\end{myquote}

\begin{myquote}
ಪೇಳುವೆನಿದ ನಾ,\\ಮರೆಯದಿರೆಂದಿಗು–\\ದೇವನೆ ತಾ ಸತ್ಯ;\\ಉಳಿದುದೆಲ್ಲ ಮಿಥ್ಯ.\\ನನಸು ತೋರಿದರು\\ಕನಸಿದು ಲೋಕವು,\\ಒಂದೆ ಸತ್ಯವಿದು–\\ಅವನು ಮಾತ್ರ ಇಹನು; ನ–\\ನ್ನೊಳಗಿನ ನಿಜದ ನಾನು ಅವನು!\\- ಎಂದೆಂದಿಗು ನೀ\\ಲೆಕ್ಕಿಸದಲೆ ಇರು\\ಬದಲಾವಣೆಗಳನು!
\end{myquote}

ನಿಮ್ಮೆಲ್ಲರಿಗೂ ಅಳಿಯದ ಪ್ರೀತಿ ಮತ್ತು ಕೃತಜ್ಞತೆಗಳೊಂದಿಗೆ...

\begin{flushright}
ವಿವೇಕಾನಂದ
\end{flushright}

\begin{myquote}
ಬಳಿಕ ಮೇರಿ ಹೇಲ್:\\ಹೆಸರಿಗಷ್ಟೇ ಎರಡು– ಅಂತರವು ಅತ್ಯಲ್ಪ!–\\ವ್ಯತ್ಯಾಸವೆನಗೀಗ ಕಾಣುತಿಹುದು;\\ಇಷ್ಟಾದರೂ ನಿಮ್ಮ ಪೌರ್ವಾತ್ಯ ತರ್ಕವದು\\ನನ್ನ ಬುದ್ಧಿಗದೇಕೊ ಎಟುಕದಿಹುದು!\\'ಭಗವಂತನೇ ಸತ್ಯ ಉಳಿದುದೆಲ್ಲವು ಮಿಥ್ಯ'\\'ಲೋಕವೆಲ್ಲವು ಕನಸು' – ಹೀಗಾದರೆ,\\ಭಗವಂತನನ್ನುಳಿದು ಉಳಿವುದೇನು?
\end{myquote}

\begin{myquote}
'ಹಲವನ್ನು ಕಾಣುವರು ಎಲ್ಲದಕು ಹೆದರುವರು\\'ಏಕ'ವನು ಕಾಂಬಾತ ನಿಜದಿ ಜೀವಿಪನು;\\ಹೇಳುವೆನು ಮತೊಮ್ಮೆ – ನನಗೆ ತೋಚುವುದಿಷ್ಟೆ:\\ಅವನಲ್ಲದೇನನೂ ನಾ ಕಾಣೆನು;\\ನಾನವನ ಒಳಗಿರುವೆ, ನನ್ನೊಳವನು!
\end{myquote}

\begin{myquote}
ಬಳಿಕ ಸ್ವಾಮಿಜಿ ಉತ್ತರಿಸುತ್ತಾರೆ:\\ಚುರುಕಿನ ಬುದ್ಧಿ – ಹೋಲಿಸಲಸದಳ\\ಮೇರಿ ವಿಲಕ್ಷಣ ಹುಡುಗಿ;\\ಅನುಪಮ ಸುಂದರಿ, ಮರುಮಾತಿಲ್ಲವು,\\ಅಪರೂಪಳು ಈ ಬೆಡಗಿ!
\end{myquote}

\begin{myquote}
ಭಾವದಾಳನು ಹಿಡಿದಿಡಲಾರಳು\\ಉಕ್ಕಿ ಹರಿಯುವುದು ಕೊನೆಗೆ;\\ಮುಕ್ತಚೇತವದು, ಕಾಣುತಿರುವೆ ನಾ,\\ಬೆಂಕಿಯಾಗಿರದೆ ಹೀಗೆ?
\end{myquote}

\begin{myquote}
ಹಲವು ಹಾಡುಗಳ ಸ್ಫೂರ್ತಗಾಯಕಿಯು,\\ವಾದ್ಯಗಾರ್ತಿಯೂ ನಿಜದಿ;\\ತಂಪೆದೆಯವಳು, ತಂಪಾಗಿಸುವಳು\\ಬಯಸುವ ದಡ್ಡನ ಕ್ಷಣದಿ!
\end{myquote}

\begin{myquote}
ಸೋದರಿ ಮೇರಿ, ಸೌಂದರ್ಯದಿ ನೀ\\ಸೆಳೆಯುವೆ ಎಲ್ಲರ ನಿನ್ನೆಡೆಗೆ;\\ಜೋಕೆ ಜೋಕೆ, ನೀ, ಮಧುರವೆನಿಸಿದರು\\ಗೋಣನೊಡ್ಡದಿರು ಶೃಂಖಲೆಗೆ!\\ಶೀಘ್ರದಲ್ಲೇ ಆ ಮರುಳ ಸಂಗಾತಿ\\ಕೇಳುವ ರಾಗವೆ ಬೇರೆ!\\ಮನಸಿಗು ಮನಸಿಗು ಕದನವು; ಆತನ\\ಬದುಕು ನುಚ್ಚುನೂರೆ!
\end{myquote}

\begin{myquote}
ಈ ಸಾಲುಗಳಿವು, ಸೋದರಿ ಮೇರಿ,\\ಏಟಿಗೆ ಏಟು, ಮಂಗಾಲಾಪ!\\ಮಾಡಬಲ್ಲನಿದ ಸಾಧುವೊಬ್ಬನೇ!
\end{myquote}

\selecteng

\chapter[TO MY OWN SOUL]{\enginline{TO MY OWN SOUL}\protect\footnote{\engfoot{C.W, Vol. VIII, P.170}}}

\begin{center}
(Composed at Ridgely Manor, New York, in 1899.)
\end{center}

\begin{myquote}
\enginline{Hold yet a while, Strong Heart,\\Not part a lifelong yoke\\Though blighted looks the present, future gloom.}
\end{myquote}

\begin{myquote}
\enginline{And age it seems since you and I began our\\March up hill or down. Sailing smooth o’ er\\Seas that are so rare\\Thou nearer unto me, than oft–times I myself\\Proclaiming mental moves before they were!}
\end{myquote}

\begin{myquote}
\enginline{Reflector true–Thy pulse so timed to mine,\\Thou perfect note of thoughts, however fine–\\Shall we now part, Recorder, say?}
\end{myquote}

\begin{myquote}
\enginline{In thee is friendship, faith,\\For thou didst warn when evil thoughts were brewing–\\And though, alas, thy warning thrown away,\\Went on the same as ever–good and true.}
\end{myquote}

\selectkan

\begin{center}
\textbf{ನನ್ನಂತರಾತ್ಮನಿಗೆ.}
\end{center}

\enginline{'To My Own Soul'} ಎಂಬ ಈ ಕವನವನ್ನು ಸ್ವಾಮಿಜಿಯವರು ಬರೆದದ್ದು ೧೮೯೯ರ ನವೆಂಬರ್‌ನಲ್ಲಿ, ನ್ಯೂಯಾರ್ಕ್ ರಾಜ್ಯದ 'ರಿಜ್ಲಿ ಮೇನರ್' ಎಂಬಲ್ಲಿದ್ದಾಗ.

'ಅಂತರಾತ್ಮ'ನೊಂದಿಗಿನ ಸಂಭಾಷಣೆ ಈ ಕವನದ ವಸ್ತು. ಎಂತಹ ಪ್ರತಿಕೂಲ ಪರಿಸ್ಥಿತಿಯಲ್ಲಿಯೂ ಸೋಲೊಪ್ಪದ ಅಧ್ಯಾತ್ಮದ ಪ್ರಯಾಣವನ್ನು ಮುಂದುವರಿಸಬೇಕೆಂಬ ಕರೆ ಇದರಲ್ಲಿದೆ.

\begin{myquote}
ತಡೆದುಕೋ ಧೀರಾ,\\ವರ್ತಮಾನಕೆ ಮಂಜು ಕವಿದಿದ್ದರೂ,\\ಭವಿತವ್ಯ ಕುರುಡುಗತ್ತಲೆಯಾದರೂ,\\ಬಾಳಿನುದ್ದಕು ಹೊತ್ತ ನೊಗವಿದನು ಇಂದು ನೀ\\ಕೆಳಗಿಡದಿರು.\\ನನ್ನ–ನಿನ್ನಯ ಪಯಣದಾದಿ ಯಾವಾಗಲೋ–\\ಬೆಟ್ಟಗಳ ಹತ್ತಿದೆವು, ಕಣಿವೆಗಳನಿಳಿದೆವು,\\ಅಬ್ಬರದ ಕಡಲುಗಳ ಸುಲಭದಲಿ ದಾಟಿದೆವು–\\ನನ್ನೆದೆಗೆ ನನಗಿಂತ ಸನಿಹ ನೀನು;\\ಮನದ ಕಂಪನಗಳವು ಮೂಡುವುದಕ್ಕೂ ಮೊದಲೆ
\end{myquote}

\begin{myquote}
ನನ್ನರಿವಿನೆಚ್ಚರಕೆ ತರುವ ನೀನು!\\ದಿಟದ ಪ್ರತಿಬಿಂಬ ನೀನಾಗಿರುವೆ ನನಗೆ–\\ನಿನ್ನ ನಾಡಿಯ ಮಿಡಿತ ನನ್ನದಾಗಿರಲು,\\ನನ್ನ ಚಿಂತನೆಸ್ವರಕೆ ಎಳೆಎಳೆಗಳಲ್ಲಿಯೂ\\ನೀ ಸ್ಪಂದಿಸಿರಲು,\\ಈಗ ನಾವಗಲುವುದೆ, ಸಖನೆ, ಹೇಳು?
\end{myquote}

\begin{myquote}
ಕೆಳೆಯು ನಿನ್ನೊಳಗಿಹುದು,\\ನಂಬಿಹೆನು ನಿನ್ನನ್ನು;\\ದುಶ್ಚಿಂತೆಗಳು ಮನವ ಮುಸುಕಿದಾಗ\\ನಿನ್ನೆಚ್ಚರಿಕೆ ದನಿಗೆ ನಾ ಕಿವುಡನಾದರೂ\\ಎಚ್ಚರಿಸುವನು ಸತತ ನೀನಲ್ಲವೇ?
\end{myquote}

\selecteng

\chapter[NO ONE TO BLAME]{\enginline{NO ONE TO BLAME}\protect\footnote{\engfoot{C.W, Vol. VIII, P.175}}}

\begin{center}
(Written from New York, 16th May, 1895.)
\end{center}

\begin{myquote}
\enginline{The sun goes down, its crimson rays\\Light up the dying day;\\A startled glance I throw behind\\And count my triumph shame;\\No one but me to blame.}
\end{myquote}

\begin{myquote}
\enginline{Each day my life I make or mar,\\Each deed begets its kind,\\Good good, bad bad, the tide once set\\No one can stop or stem;\\No one but me to blame.}
\end{myquote}

\begin{myquote}
\enginline{I am my own embodied past;\\Therein the plan was made;\\The Will, the thought, to that conform,\\To that the outer frame;\\No one but me to blame.}
\end{myquote}

\begin{myquote}
\enginline{Love comes reflected back as love,\\Hate breeds more fierce hate,\\They mete their measures, lay on me\\Through life and death their claim;\\No one but me to blame.}
\end{myquote}

\begin{myquote}
\enginline{I cast off fear and vain remorse,\\I feel my Karma's sway\\I face the ghosts my deeds have raised\\Joy, Sorrow, censure, fame;\\No one but me to blame.}
\end{myquote}

\begin{myquote}
\enginline{Good, bad, love, hate, and pleasure, pain\\Forever linked go,\\I dream of pleasure without pain,\\It never, never came;\\No one but me to blame.}
\end{myquote}

\begin{myquote}
\enginline{I give up hate, I give up love,\\My thirst for life is gone;\\Eternal death is what I want,\\Nirvanam goes life's flame;\\No one is left to blame.}
\end{myquote}

\begin{myquote}
\enginline{One only man, one only God, one ever perfect soul,\\One only sage who ever scorned the dark and dubious ways,\\One only man who dared think and dared show the goal\\That death is curse, and so is life, and best when stops to be.}
\end{myquote}

\begin{myquote}
\enginline{Om Namo Bhagavate Sambuddhaya\\Om, I salute the Lord, the awakened.}
\end{myquote}

\selectkan

\begin{center}
\textbf{ಯಾರ ದೂರಲಿ ನಾನು?}
\end{center}

ಇದರ ಮೂಲ ಇಂಗ್ಲಿಷ್ ಕವನ \enginline{'No One to Blame'} ಎಂಬ ಶೀರ್ಷಿಕೆಯದು. ಇದನ್ನು ಸ್ವಾಮಿಜಿಯವರು ರಚಿಸಿದ್ದು ೧೮೯೫ರ ಮೇ ೧೬ರಂದು, ನ್ಯೂಯಾರ್ಕಿನಲ್ಲಿ, ಈ ಕವನ ರಚನೆಯ ಸಂದರ್ಭದ ಬಗೆಗೆ ಹೆಚ್ಚಿನ ಮಾಹಿತಿಯೇನೂ ದೊರೆತಿಲ್ಲ.

ಪ್ರತಿಯೊಬ್ಬ ಮನುಷ್ಯನೂ ತನ್ನ ಭವಿಷ್ಯವನ್ನು ರೂಪಿಸಿಕೊಳ್ಳುವುದಕ್ಕೆ ತಾನೇ ರೂವಾರಿಯಾಗಿದ್ದಾನೆ ಎಂಬ ಸ್ವಾಮಿಜಿಯವರ ಮಾತನ್ನು ಈ ಕವನದ ಹಿನ್ನೆಲೆಯಲ್ಲಿ ನೆನೆಯಬಹುದು. ಅಲ್ಲದೆ ಅವರು ಹೇಳುತ್ತಾರೆ: “ನಮ್ಮ ಪ್ರತಿಯೊಂದು ಆಲೋಚನೆ, ಮಾಡುವ ಪ್ರತಿಯೊಂದು ಕ್ರಿಯೆ ಕೆಲವು ನಿರ್ದಿಷ್ಟ ಕಾಲದ ಬಳಿಕ ಸೂಕ್ಷ್ಮವಾಗುತ್ತದೆ, ಬೀಜರೂಪಕ್ಕೆ ಹೋಗುತ್ತದೆ ಮತ್ತು ಸೂಕ್ಷ್ಮಶರೀರದಲ್ಲಿ ಸುಪ್ತರೂಪದಿಂದ ಜೀವಂತವಾಗಿರುತ್ತದೆ. ಕಾಲಾನಂತರದಲ್ಲಿ ಅದು ಮತ್ತೆ ಹೊರಹೊಮ್ಮಿ ತನ್ನ ಪರಿಣಾಮವನ್ನು ಬೀರುತ್ತದೆ. ಈ ಪರಿಣಾಮಗಳೇ ಮಾನವನ ಜೀವನ ಪರಿಸ್ಥಿತಿಯನ್ನು ನಿಯಮಿಸುತ್ತವೆ. ಹೀಗೆ ಮಾನವ ತನ್ನ ಜೀವನವನ್ನು ತಾನೇ ರೂಪಿಸಿಕೊಳ್ಳುತ್ತಾನೆ. ಮಾನವ ತನಗೆ ತಾನೇ ಮಾಡಿಕೊಳ್ಳುವ ನಿಯಮಗಳ ಹೊರತಾಗಿ ಇನ್ನಾವ ನಿಯಮಗಳಿಂದಲೂ ಬದ್ಧನಲ್ಲ.” ಆದ್ದರಿಂದ ಮನುಷ್ಯ ತನ್ನ ಪರಿಸ್ಥಿತಿಯ ಜವಾಬ್ದಾರಿಯನ್ನು ತಾನೇ ಹೊರಬೇಕಲ್ಲದೆ ಇನ್ನಾರನ್ನೂ ದೂರುವಂತಿಲ್ಲ.

\begin{myquote}
ಪಡುವಣದಿ ರವಿಯಿಳಿಯೆ ಕೆಂಬಣ್ಣ ಕಿರಣಗಳು\\ನಶಿಸುತಿಹ ದಿನವನ್ನು ಬೆಳಗುತಿರಲು\\ನನ್ನ ಭೂತವ ಕಂಡು ನಾನೆ ಗಾಬರಿಗೊಂಡು\\ನನ್ನಿದಿರು ನಾನೆ ತಲೆತಗ್ಗಿಸಿರಲು
\end{myquote}

\begin{flushright}
–ಯಾರ ದೂರಲಿ ನಾನು ನನ್ನನುಳಿದು?
\end{flushright}

\begin{myquote}
ದಿನದಿನವು ಕಟ್ಟುವೆನು, ಕೆಡಹುವೆನು ಬದುಕನ್ನು,\\ಬೀಜದಂತೆಯೆ ಫಲವು ಬರುವುದಲ್ಲ;\\ಒಳಿತು ಒಳಿತನೆ ಹೊಮ್ಮಿ, ಕೆಡುಕಿನಲಿ ವಿಷ ಚಿಮ್ಮಿ,\\ಮೂಡಿ ಬಹ ಅಲೆಯೆಂದು ನಿಲ್ಲದಲ್ಲ!
\end{myquote}

\begin{flushright}
–ಯಾರ ದೂರಲಿ ನಾನು ನನ್ನನುಳಿದು?
\end{flushright}

\begin{myquote}
ಭೂತವದು ನನ್ನಲ್ಲಿ ಸಾಕಾರಗೊಂಡಿಹುದು,\\ಇಂದಿನಾಕಾರಗಳು ಉದಿಸಿತಲ್ಲಿ;\\ಬುದ್ಧಿ – ಚಿತ್ರಗಳೆಲ್ಲ ಅದಕ್ಕೆ ಹೊಂದುತಲಿರಲು\\ಒಳಗು ಹೊರಗಿನ ನಿಲುವ ಪಡೆಯಿತಲ್ಲಿ,
\end{myquote}

\begin{flushright}
–ಯಾರ ದೂರಲಿ ನಾನು ನನ್ನನುಳಿದು?
\end{flushright}

\begin{myquote}
ಒಲುಮೆ ಒಲುಮೆಯ ತರಲು, ಹಗೆಯು ಕಡು ಹಗೆಯಾಗಿ\\ಮರಿಗೆ ಮರಿ ಮರಿಯಾಗಿ ಬೆಳೆಯುತಿರಲು,\\ಒಲುಮೆ – ಹಗೆಗಳ ಜೋಡಿ ಹೆಗಲ ಹತ್ತಿಹುದೆನ್ನ,\\ಕಾಡುತಿರುವುದು ಜನುಮಜನುಮದಲ್ಲು!
\end{myquote}

\begin{flushright}
–ಯಾರ ದೂರಲಿ ನಾನು ನನ್ನನುಳಿದು?
\end{flushright}

\begin{myquote}
ಭಯವ ನಾ ಹರಿದೆಸೆದೆ, ದುಃಖವನು ಕಿತ್ತೊಗೆದೆ,\\ಕರ್ಮದಲೆಗಳ ಮೇಲೆ ತೇಲುತಿರುವೆ;\\ಸಂತಸಕೆ, ಸಂಕಟಕೆ, ನಿಂದನೆಗೆ, ಹಿರಿ ಯಶಕೆ\\ತಂದ ಭೂತದ ಮುಖದ ಕಾಣುತಿರುವೆ!
\end{myquote}

\begin{flushright}
–ಯಾರ ದೂರಲಿ ನಾನು ನನ್ನನುಳಿದು?
\end{flushright}

\begin{myquote}
ಒಳಿತುಗಳು–ಕೆಡುಕುಗಳು, ಸುಖ – ದುಃಖ, ಪ್ರೀತಿ – ಹಗೆ.\\ಒಂದರೊಡನೊಂದೆಂದು ಬೆಸೆದಿರುವುದು;\\ನೋವಿರದ ಸಂತಸದ ಕನಸ ಕಾಣುವೆ ನಾನು,\\ಆದರದು ಎಂದೆಂದು ಬಾರದಿಹುದು!
\end{myquote}

\begin{flushright}
–ಯಾರ ದೂರಲಿ ನಾನು ನನ್ನನುಳಿದು?
\end{flushright}

\begin{myquote}
ಒಲುಮೆ – ಹಗೆಗಳನೆಲ್ಲ ನೀಗಿರುವೆ, ಜೀವನದ\\ತೃಷ್ಣೆಯಿಂದಿಲ್ಲೀಗ ಬತ್ತಿರುವುದು;\\ಶಾಶ್ವತದ ಮೃತ್ಯುವಿನ ದಾರಿ ಕಾಯುತಲಿರುವೆ,\\ನಿರ್ವಾಣ ಬಾಳಿನುರಿ ನುಂಗಿರುವುದು!
\end{myquote}

\begin{flushright}
–ದೂರಲಾರಿಹರಿಲ್ಲಿ, ಎಲ್ಲರಳಿದು!
\end{flushright}

\begin{myquote}
ಅಂಧಕಾರದ ಅಡ್ಡದಾರಿಗಳ ಕಿತ್ತೊಗೆದ\\ಒಬ್ಬ ಮಾನವ, ಒಬ್ಬ ದೈವನವ, ಎಂದೆಂದು\\ಒಬ್ಬನೇ ಪರಿಪೂರ್ಣನಾದಾತನು;\\ಆ ಒಬ್ಬನೇ ಧೀರಚಿಂತಕನು
\end{myquote}

\begin{myquote}
ಧೀರತನದಲಿ ತೋರ್ದ ದಾರಿಯಿದುವೆ:\\ಮೃತ್ಯುವೆಂಬುದು ಶಾಪ, ಬದುಕು ಬೇರೇನಲ್ಲ,\\ಹುಟ್ಟು ಸಾವುಗಳೆರಡು ನಿಲ್ಲುವುದೆ ಪರಮಗುರಿ!\\ಓಂ! ಆ ಭಗವಂತನನ್ನು, ಎಚ್ಚತ್ತವನನ್ನು, ನಮಿಸುವೆನು ನಾನು!
\end{myquote}

\selecteng

\chapter[KALI THE MOTHER]{\enginline{KALI THE MOTHER}\protect\footnote{\engfoot{C.W. Vol. IV, P. 384}}}

\begin{myquote}
\enginline{The stars are blotted out,\\The clouds are covering clouds,\\It is darkness vibrant, sonant.\\In the roaring, whirling wind\\Are the souls of a million lunatics,\\Just loose from the prision– house, \\Wrenching trees by the roots, \\Sweeping all from the path.\\The sea has joined the fray,\\And swirls up mountain–waves,\\To reach the pitchy sky.\\The flash of lurid light\\Reveals on every side\\A thousand, thousand shades\\Of Death begrimed and black\\Scattering plagues and sorrows,\\Dancing mad with joy.\\Come, Mother, come!\\For Terror is Thy name,\\Death is in Thy breath,\\And every shaking step\\Destroys a world for e'er.\\Thou “Time”, the all–Destroyer!\\Come, O Mother, come!\\Who dares misery love,\\And hug the form of Death,\\Dance in Destruction's dance,\\To him the Mother comes.}
\end{myquote}

\selectkan

\begin{center}
\textbf{ತಾಯಿ ಕಾಳಿ}
\end{center}

\enginline{'Kali the Mother'} ಎಂಬ ಈ ಪ್ರಸಿದ್ಧವಾದ ಕವನವನ್ನು ಸ್ವಾಮಿಜಿಯವರು ೧೮೯೮ರಲ್ಲಿ ಕಾಶ್ಮೀರದಲ್ಲಿ ರಚಿಸಿದರು. ಅಮರನಾಥ ಯಾತ್ರೆಗೆ ಹೋಗಿಬಂದ ಬಳಿಕ ಆ ದಿನಗಳಲ್ಲಿ ಸ್ವಾಮಿಜಿಯವರು ಸದಾ ಜಗನ್ಮಾತೆಯ ಭಾವದಲ್ಲೇ ಮುಳುಗಿರುತ್ತಿದ್ದರು. ಹೀಗಿರುವಾಗ ಒಂದು ಸಂಜೆ ಅವರಿಗಾದ ದಿವ್ಯಾನುಭವ ಈ ಕವನದಲ್ಲಿ ಮೈದಾಳಿದೆ. ಅಂದಿನ ಅವರ ಅನುಭವ ಎಷ್ಟು ತೀವ್ರತರವಾಗಿತ್ತೆಂದರೆ, ಈ ಕವನದ ಕೊನೆಯ ಪದವನ್ನು ಬರೆಯುವಷ್ಟರಲ್ಲಿ ಅವರ ಕೈಯಲ್ಲಿನ ಲೇಖನಿ ಜಾರಿಬಿದ್ದು ಅವರೂ ನೆಲಕ್ಕುರುಳಿದರು. ಬಾಹ್ಯಪ್ರಜ್ಞೆ ತಪ್ಪಿ ಅವರು ಭಾವಸಮಾಧಿಯ ಅತ್ಯುನ್ನತ ಸ್ತರದಲ್ಲಿದ್ದರು.

\begin{myquote}
ತಾರೆಗಳವು ಕಾಲ್ಕಿತ್ತಿವೆ,\\ಮುಗಿಲು ಮುಗಿಲ ನುಂಗುತಲಿದೆ,\\ಕಾರಿರುಳಿನ ಸ್ಪಂದನ,\\ಕತ್ತಲ ಕಂಪನನ!
\end{myquote}

\begin{myquote}
ಗರ್ಜರಿಸುತ ಗರ್ಗರಿಸುತ\\ಮುಗ್ಗರಿಸುತ ಮಾರುತ\\ಮರಮರಗಳ ಮುರಿಮುರಿಯುತ\\ಕೊಚ್ಚುತ್ತಿದೆ ಪಥದಿ\\ನುಗ್ಗುತಿಹುದು ಕಟ್ಟು ಕಡಿದು\\ಸೆರೆಮನೆಯಿಂ ಮುಕ್ತಿ ಪಡೆದ\\ಮರುಳರ ಪಡೆ ತೆರದಿ!
\end{myquote}

\begin{myquote}
ಆಗಸವನು ತಬ್ಬಲೆಂದು\\ಗಿರಿತರಂಗವೆಬ್ಬಿಸುತ್ತ\\ಸಾಗರಭೈರವನು\\ಕದನಕೆ ನಿಂತಿಹನು!\\ಮೃತ್ಯುವರ್ಣ ಮಿಂಚುತಿರಲು\\ಎಡೆಎಡೆಯೊಳು ತೋರುತಿಹುದು\\ಲಕ್ಷಲಕ್ಷ ಕರಿಕರಾಳ\\ಮೃತ್ಯುಛಾಯೆಯ!
\end{myquote}

\begin{myquote}
ಹುಚ್ಚುವರಿದು ಕುಣಿಕುಣಿಯುತ\\ಸಂಕಟಗಳನೀಡಾಡುತ\\ಬಾ, ತಾಯಿ, ಬಾ!\\ನಲಿನಲಿಯುತ ಬಾ!\\ಕರಾಳಿ ಕಾಳಿ ನಿನ್ನ ಹೆಸರು,\\ಮೃತ್ಯುವೆ ನಿನ್ನುಸಿರು!\\ನಿನ್ನಡಿಗಳ ಕಂಪನದಿಂ\\ನಿರ್ನಾಮವು ಜಗವು!\\ಕಾಲರೂಪಿ, ತಾಯೆ, ನೀನು!\\ನುಂಗುವೆ ಎಲ್ಲವನು!\\ಬಾ, ತಾಯಿ, ಬಾ!
\end{myquote}

\delimiter

\begin{myquote}
ಅಳಲುಗಳನು ಪ್ರೀತಿಸಿ,\\ಮೃತ್ಯುವನಾಲಿಂಗಿಸಿ,\\ಪ್ರಳಯನಾಟ್ಯಗೈವ ಧೀರ–\\ನೆಡೆಗೆ ತಾಯಿ ಬರುವಳು!
\end{myquote}

\selecteng

\chapter[WHO KNOWS HOW MOTHER PLAYS!]{\enginline{WHO KNOWS HOW MOTHER PLAYS!}\protect\footnote{\engfoot{CW, Vol. V, P.439}}}

\begin{myquote}
\enginline{Perchance a prophet thou\\Who knows? Who dares touch\\The depths where Mother hides\\Her silent failless bolts!}
\end{myquote}

\begin{myquote}
\enginline{Perchance the child had glimpse\\Of shades, behind the scenes,\\With eager eyes and strained,\\Quivering forms–ready\\To jump in front and be\\Events, resistless, strong.\\Who knows but Mother, how,\\And where, and when, they come?}
\end{myquote}

\begin{myquote}
\enginline{Perchance the shining sage\\Saw more than he could tell;\\Who knows, what soul, and when,\\The Mother makes Her throne?}
\end{myquote}

\begin{myquote}
\enginline{What law would freedom bind?\\What merit guide Her will,\\Whose freak is gretest order,\\Whose will resistless law?}
\end{myquote}

\begin{myquote}
\enginline{To child may glories ope\\Which father never dreamt;\\May thousand fold in daughter\\Her powers Mother store.}
\end{myquote}

\selectkan

\begin{center}
\textbf{ಯಾರು ಬಲ್ಲರು ತಾಯ ಲೀಲೆಯನ್ನು!}
\end{center}

\enginline{'Who Knows How Mother Plays!'} ಎಂಬ ಈ ಕವನ ಸ್ವಾಮಿಜಿಯವರು ದಕ್ಷಿಣ ಕ್ಯಾಲಿಫೋರ್ನಿಯಾದಲ್ಲಿದ್ದಾಗ ಬರೆದದ್ದು. ಇದನ್ನು ಅವರು ಸೋದರಿ ನಿವೇದಿತೆಗೆ ಕಳಿಸಿಕೊಡುತ್ತಾರೆ. ಜಗನ್ಮಾತೆಯ ಲೀಲೆಗಳ ಸಾಧ್ಯತೆಗಳನ್ನು ಇಂತಿಷ್ಟೇ ಎಂದು ನಿರ್ಣಯ ಮಾಡಿಡಲು ಸಾಧ್ಯವಿಲ್ಲ; ಆಕೆ ಯಾವ ನಿಯಮಗಳಿಂದಲೂ ಬದ್ಧಳಲ್ಲ; ಏನನ್ನು ಹೇಗೆ ಬೇಕಾದರೂ ಮಾಡಬಲ್ಲ ಶಕ್ತಿಶಾಲಿನಿ ಆಕೆ ಎಂಬುದನ್ನು ಈ ಕವನದಲ್ಲಿ ಬೇರೆ ಬೇರೆ ನಿದರ್ಶನಗಳ ಮೂಲಕ ನಿರೂಪಿಸಲಾಗಿದೆ.

ಯಾರು ಬಲ್ಲರು?–\\ನೀನು ದ್ರಷ್ಟಾರನಿರಬಹುದು!

\begin{myquote}
ಸದ್ದು ಮಾಡದ, ಗುರಿಯನೆಂದೆಂದು ತಪ್ಪದಿಹ\\ಮಿಂಚುಗಳ ಮಹತಾಯಿ ಅವಿತಿಟ್ಟ ಆಳವನು\\ಯಾರು ನಿಲುಕಲುಬಹುದು?
\end{myquote}

\begin{myquote}
ಇದ್ದಕ್ಕಿದ್ದೊಲೆ ಧುಮುಕಿ ಚಣ ಮಾತ್ರ ಮಿಂಚಿ\\ಕರಗುತಿಹ ರೂಪಗಳ–\\ಕಂಗಳಿಗೆ ಕಾಣಿಸುವ ನೋಟದಾಚೆಯ ನೆರಳ–\\ತುಣುಕೊಂದು ಶಿಶುವಿನ ಕಣ್ಣ ಕಾಣಲುಬಹುದು.\\ಅವು ಬಂದುದಾವಾಗ, ಎಲ್ಲಿಂದಲೋ,\\ಅದನು\\ತಾಯ ಹೊರತಿನ್ನಾರು ತಿಳಿಯಲಹುದು?
\end{myquote}

\begin{myquote}
ನುಡಿಗೆ ನಿಲುಕದ ನಿಜವ\\ಋಷಿಯು ಕಂಡಿರಬಹುದು; –\\ಯಾವ ಸಮಯದಿ ಯಾವ ಜೀವದಲಿ ಮಹತ್\\ಎದೆಯ ಸಿಂಹಾಸನದಿ ರಾಜಿಸುವಳೋ,\\ಅದನಾರು ತಾನೆ ತಾವರಿಯಬಹುದು?
\end{myquote}

\begin{myquote}
ಮುಕ್ತಿಯನ್ನು ಬಂಧಿಸುವ\\ಹಗ್ಗ ವಿನ್ನಾವುದಿದೆ?\\ಅವಳ ಇಚ್ಛೆಗೆ ಯಾವ ನೀತಿಗಡಿಯು?\\ಮರುಳುಲೀಲೆಯೆ ಅವಳ\\ಉನ್ನತೋನ್ನತ ತಂತ್ರ,\\ನುಗ್ಗಿ ಬಹ ಸಂಕಲ್ಪ ಹಿರಿಯ ಮಂತ್ರ!\\ಪಿತನ ಕನಸಿಗೆ ಬರದ\\ಸಂಪದದ ಸಿರಿ ಸುತನ
\end{myquote}

\begin{myquote}
ನಪ್ಪಬಹುದು;\\ತಾಯ ಬಲಕಿಂ ಮಿಗಿಲು\\ಬಲದ ಗಣಿ ಮಗಳಲ್ಲಿ\\ಹುದುಗಲಹುದು!\\ಯಾರು ಬಲ್ಲರು ತಾಯ ಲೀಲೆಯನ್ನು!
\end{myquote}

\selecteng

\chapter[ANGELSUNAWARES]{\enginline{ANGELSUNAWARES}\protect\footnote{\engfoot{C.W. Vol. IV, P. 385}}}

\begin{center}
I
\end{center}

\begin{myquote}
\enginline{One bending low with load of life–\\That meant no joy, but suffering harsh and hard–\\And wending on his way through dark and dismal paths,\\Without a flash of light from brain or heart\\To give a moment's cheer, till the line\\That marks out pain from pleasure, death from life,\\And good from what is evil was well–nigh wiped from sight,\\Saw, one blessed night, a faint but beautiful ray of light\\Descend to him. He knew not what or wherefrom,\\But called it God and worshipped\\Hope, an utter stranger, come to him and spread\\Through all his parts, and life to him meant more\\Than he could ever dream and covered all he knew,\\Nay, peeped beyond his world. The Sages\\Winked, and smiled, and called it “superstition”.\\But he did feel its power and peace\\And gently answered back–}
\end{myquote}

\begin{flushright}
“O Blessed Superstition”
\end{flushright}

\begin{center}
II
\end{center}

\begin{myquote}
\enginline{One drunk with wine of wealth and power\\And health to enjoy them both, whirled on\\His maddening course till the earth, he thought,\\Was made for him, his pleasure–garden, and man,\\The crawling worm, was made to find him sport,\\Till the thousand lights of joy, with pleasure fed,\\That flickered day and night before his eyes,\\With constant change of colours, began to blur\\His sight, and cloy his senses; till selfishness,\\Like a horny growth, had spread all o'er his heart;\\Bereft of feeling; and life in the sense,\\So joyful, precious once, a rotting corpse between his arms,\\Which he forsooth would shun, but more he tried, the more\\It clung to him; and wished, with frenzied brain,\\A thousand forms of death, but quailed before the charm,\\Then sorrow came–and Wealth and Power went–\\And made him kinship find with all the human race\\In groans and tears, and though his friends would laugh,\\His lips would speak in grateful accents–}
\end{myquote}

\begin{flushright}
“O Blessed Misery!"
\end{flushright}

\begin{center}
III
\end{center}

\begin{myquote}
\enginline{One born with healthy frame–but not of will\\That can resist emotions deep and strong,\\Nor impulse throw, surcharged with potent strength,\\And just the sort that pass as good and kind,\\Beheld that he was safe, whilst others long\\And vain did struggle 'gainst the surging waves.\\Till, morbid grown, his mind could see, like flies\\That seek the putrid part, but what was bad.\\Then Fortune smiled on him, and his foot slipped.\\That ope'd his eyes for e’ er, and made him find\\\textit{That stones and trees ne'er break the law,\\But stones and trees remain; that man alone}\\Is blest with power to fight and conquer Fate,\\Transcending bounds and laws.\\From him his passive nature fell, and life appeared\\As broad and new, and broader, newer grew,\\Till light ahead began to break, and glimpse of That\\Where Peace Eternal dwells–yet one can only reach\\By wading through the sea of struggles–courage–giving came.\\Then looking back on all that made him kin\\To stocks and stones, and on to what the world\\Had shunned him for, his fall,\\he blessed the fall,\\And with a joyful heart, declared it–}
\end{myquote}

\begin{flushright}
"Blessed Sin!"
\end{flushright}

\enginline{}

\selectkan

\begin{center}
\textbf{ಸುಪ್ತ ದೈವರು}
\end{center}

ಸ್ವಾಮಿಜಿಯವರು ೧೮೯೮ರ ಸೆಪ್ಟೆಂಬರ್ ೧ ರಂದು ಕಾಶ್ಮೀರದಲ್ಲಿದ್ದಾಗ ರಚಿಸಿದ \enginline{'Angels Unawares'} ಎಂಬ ಇಂಗ್ಲಿಷ್ ಕವನ ಇದರ ಮೂಲ.

ಪ್ರತಿಯೊಬ್ಬ ವ್ಯಕ್ತಿಯ ಪ್ರಗತಿಗೂ ಅವನಿಗೇ ತೀರ ವಿಶಿಷ್ಟವಾದ ನಿಯಮಗಳಿರುತ್ತವೆ. ಪ್ರತಿಯೊಬ್ಬನು ತನ್ನದೇ ಆದ ಈ ನಿಯಮಗಳಿಗೆ ಅನುಗುಣವಾಗಿಯೇ ಬೆಳೆಯಬೇಕಾಗುತ್ತದೆ. ಹೀಗಾಗಿ ಈ ನಿಯಮಗಳನ್ನು ಸಾಧಾರಣೀಕರಿಸಿ ಸರ್ವರಿಗೂ ಅನ್ವಯಿಸಲು ಪ್ರಯತ್ನಿಸುವುದು ಅವೈಜ್ಞಾನಿಕವೆನ್ನುತ್ತಾರೆ ಸ್ವಾಮಿಜಿ, ಒಂದೇ ಸಂಗತಿ ಒಬ್ಬನ ಬೆಳವಣಿಗೆಗೆ ಪೂರಕವಾದರೆ ಇನ್ನೊಬ್ಬನ ಬೆಳವಣಿಗೆಗೆ ಮಾರಕವಾಗುತ್ತದೆ......."ಒಬ್ಬ ಮನುಷ್ಯನ ವಿಧಾನ ತಪ್ಪೆನ್ನಲು ನಿಮಗೇನು ಹಕ್ಕಿದೆ? ನಿಮ್ಮ ವಿಷಯದಲ್ಲಿ ಅದು ತಪ್ಪಿರಬಹುದು. ಎಂದರೆ, ಆ ವಿಧಾನವನ್ನು ಅನುಸರಿಸುವುದರಿಂದ ನೀವು ಅವನತಿ ಹೊಂದಬಹುದು. ಆದರೆ ಇದರ ಅರ್ಥ ಅದನ್ನು ಅನುಸರಿಸುವ ಆ ಇನ್ನೊಬ್ಬನೂ ಅವನತಿ ಹೊಂದುವನೆಂದಲ್ಲ. ಆದ್ದರಿಂದ....ನಿಮಗೆ ತಿಳಿವಳಿಕೆ ಇದ್ದು, ಒಬ್ಬ ವ್ಯಕ್ತಿ ದುರ್ಬಲನಾಗಿರುವುದನ್ನು ಕಂಡರೆ, ಅವನನ್ನು ಹೀಯಾಳಿಸಬೇಡಿ...." ಇದು ಸ್ವಾಮಿಜಿಯವರ ಸಿದ್ಧಾಂತ \enginline{(Complete Works, Vol. I, P.470),} ಹೀಗಾಗಿ ವ್ಯವಹಾರದ ಸ್ವರದಲ್ಲಿ ಒಳಿತಾಗಲಿ ಕೆಡುಕಾಗಲಿ, ಪ್ರಯೋಜಕವಾಗಲಿ ನಿಷ್ಪ್ರಯೋಜಕವಾಗಲಿ ನಿರಪೇಕ್ಷವಾದುದಲ್ಲ; ಆಯಾ ವ್ಯಕ್ತಿಗೆ, ಆಯಾ ಸಂದರ್ಭಕ್ಕೆ ತಕ್ಕಂತೆ ಸಾಪೇಕ್ಷವಾದುದು.

ಪ್ರಸ್ತುತ ಕವನದಲ್ಲಿಯೂ ಈ ತತ್ತ್ವ ರೂಪುದಾಳಿರುವುದನ್ನು ಕಾಣಬಹುದು. ಈ ಕವನದ ಮೂರು ಭಾಗಗಳಲ್ಲಿ ಸ್ವಾಮಿಜಿ ಮೂರು ಬಗೆಯ ವಿಭಿನ್ನ ವ್ಯಕ್ತಿತ್ವಗಳ ಜೀವನಚಿತ್ರಣಗಳನ್ನು ನೀಡುತ್ತಾರೆ. ದುಃಖ–ಸಂತಾಪಗಳಲ್ಲಿ ಮುಳುಗಿರುವ ಮೊದಲನೆಯ ವ್ಯಕ್ತಿಯ ಜೀವನದಲ್ಲಿ 'ಕಿರಣವೊಂದು' ಬಂದು ಪರಿವರ್ತನೆಯನ್ನು ತಂದಾಗ ಅವನು ಅದನ್ನು ತಾರ್ಕಿಕವಾಗಿ ವಿವರಿಸಲಾಗದ ಸ್ಥಿತಿಯಲ್ಲಿರುತ್ತಾನೆ. ಹೀಗಾಗಿ, ತಿಳಿದವರೆನ್ನಿಸಿಕೊಂಡವರು ಅದನ್ನೊಂದು ಮೌಢ್ಯವೆಂದರೂ ಆ ಮೌಢ್ಯವೇ ಅವನಿಗೆ ಧನ್ಯತಾಸ್ವರೂಪದ್ದಾಗಿರುತ್ತದೆ.

ಭೋಗದುನ್ಮಾದದಲ್ಲಿ ಮುಳುಗಿದ್ದ ಎರಡನೆಯ ವ್ಯಕ್ತಿಗೆ ದೈವವು ನೋವು–ಸಂಕಟಗಳ ರೂಪದಲ್ಲಿ ಬರುತ್ತದೆ. ಆ ಮೂಲಕವೇ ಆತ ಉದ್ಧಾರದ ಹಾದಿಯನ್ನು ಹಿಡಿಯುತ್ತಾನೆ. ಇದನ್ನು ಅರಿಯಲಾಗದವರು ಅವನ ನೋವಿಗಾಗಿ ಕನಿಕರಿಸಿದರೂ, ಅವನು ಮಾತ್ರ ಸಂಕಟದಲ್ಲಿಯೇ ಧನ್ಯತೆಯನ್ನು ಕಂಡುಕೊಳ್ಳುತ್ತಾನೆ.

ಮೂರನೆಯವನು ನಿಷ್ಕ್ರಿಯತೆಯಲ್ಲಿ ಸಜ್ಜನಿಕೆಯನ್ನು ಪಡೆದಿರುವ ವ್ಯಕ್ತಿ. ಆದರೆ ತನ್ನ ಈ ಸ್ಥಿತಿಯನ್ನು ಒಂದು ಸಿದ್ಧಿಯೆಂದು ಭ್ರಮಿಸುವ ಈತ ಎಲ್ಲದರಲ್ಲೂ ದೋಷವನ್ನು ಕಾಣುವುದರಲ್ಲೇ ನಿರತನಾಗಿರುತ್ತಾನೆ. ಹೀಗಾಗಿ ದೈವ ಈತನಿಗೆ ಅಧಃಪತನದ ರೂಪದಲ್ಲಿ ಬರುತ್ತದೆ. ತಾನೇ ಜಾರಿಬಿದ್ದಾಗ ಈತ ಕ್ರಿಯಾತ್ಮಕನಾಗುತ್ತಾನೆ; ಪ್ರಗತಿಯ ಹಾದಿಯಲ್ಲಿ ಸೆಣಸಲಾರಂಭಿಸುತ್ತಾನೆ. ಇದನ್ನು ತಿಳಿಯದವರು ಅವನನ್ನು ಅಧಃಪತನಕ್ಕಾಗಿ ಮೂದಲಿಸಿದರೂ, ಅವನಿಗೆ ಮಾತ್ರ ಆ ಅಧಃಪತನಕ್ಕಾಗಿ ಕ್ಷಣವೇ ಧನ್ಯತೆಯ ಮೂಲವಾಗುತ್ತದೆ.

ಆದ್ದರಿಂದಲೇ ಯಾವುದೂ ತನ್ನಷ್ಟಕ್ಕೇ ವರವೂ ಅಲ್ಲ, ಶಾಪವೂ ಅಲ್ಲ; ಅದು ವ್ಯಕ್ತಿಯ ಪ್ರಗತಿಯ ಮೇಲೆ ಉಂಟುಮಾಡುವ ಪರಿಣಾಮದ ದೃಷ್ಟಿಯಿಂದ ಮಾತ್ರ ಅದರ ನಿಜವಾದ ಸ್ವರೂಪವನ್ನು ಅರಿಯಲು ಸಾಧ್ಯ ಎನ್ನುವುದೇ ಈ ಕವನದಲ್ಲಿ ಮೈದಾಳಿರುವ 'ದರ್ಶನ'.

\begin{center}
–೧–
\end{center}

\begin{myquote}
ಜೀವನದ ಭಾರದಲಿ ಕುಗ್ಗಿ ನಡೆದಿಹನೊಬ್ಬ – ಅದರಲ್ಲಿ\\ನಲಿವಿಲ್ಲ, ಬರಿಯ ಕಡು ಕಷ್ಟ.\\ಕತ್ತಲ–ನಿರಾಸೆಗಳ ದಾರಿ ಸವೆಸುತ್ತಿಹನು –\\ಚಣಕಾಲವಾದರೂ ಸಂತಸವನೀವುದಕೆ\\ಬುದ್ಧಿಯಿಂದಲೊ ಭಾವದಿಂದಲೊ ಬೆಳಕು ಒಂದಿನಿತಿಲ್ಲ.\\ಕಟ್ಟ ಕಡೆಗೆ\\ನೋವು–ನಲಿವಿನ ನಡುವೆ, ಸಾವು–ಬದುಕಿನ ನಡುವೆ,\\ಒಳಿತು–ಕೆಡುಕಿನ ನಡುವೆ\\ಇರುವ ಗೆರೆ ಅಳಿಸಿಯೇ ಹೋಗಿತ್ತು.\\ಇಂತಿರಲು ಒಂದಿರುಳು\\ಮಸಕು ಮಸಕಾದರೂ ಬೆಳಗುತಿಹ ಕಿರಣವೊಂದು\\ತನ್ನೆಡೆಗೆ ಇಳಿವುದನು ಕಂಡನವನು;\\ಯಾವುದದು, ಎಲ್ಲಿಂದ ಬಂದಿತದು\\ಎಂಬುದನ್ನು ಅರಿಯನವನು.\\ದೇವರೆನ್ನುತ ಅದನು ಕರೆದು ಪೂಜಿಸಿದನು.
\end{myquote}

\begin{myquote}
ಕಲ್ಪನೆಗು ನಿಲುಕದಿಹ ಭರವಸೆಯು ಬಂತು –\\ಅವನಿರವಿನಂಗಾಂಗವೆಲ್ಲವನು ವ್ಯಾಪಿಸಿತು.\\ಕನಸಿನಲು ಊಹಿಸದ ತೆರದಿ ಅವನಿಗೆ ಬದುಕು\\ಸಾರ್ಥಕತೆಯನು ತೆಳೆದು ಅವನ ತಿಳಿವೆಲ್ಲವನು\\ಒಳಗೊಂಡಿತು; ಮತ್ತೆ\\ಜಗದಾಚೆಗೂ ಇಣುಕಿತು!
\end{myquote}

\begin{myquote}
ಅದ ಕಂಡು ಪಂಡಿತರು ಕಣ್ಣ ಮಿಟುಕಿಸಿ ನಕ್ಕು, ಅದನೊಂದು\\'ಮೂಢತನ'ವೆಂದೂರೆದರು.\\ಇಂತಾದರೂ ಅದರ ಶಕ್ತಿಯನ್ನು ಶಾಂತಿಯನು\\ಅನುಭವಿಸುತಿದ್ದನವನು;\\ಮೆಲುನುಡಿಯೊಳೇ ತಾನು ಮಾರ್ನುಡಿದನು –\\'ಓ, ಮೌಡ್ಯವೇ, ಧನ್ಯ ಧನ್ಯ ನೀನು!'
\end{myquote}

\begin{center}
–೨–
\end{center}

\begin{myquote}
ಐಶ್ವರ್ಯ–ಅಧಿಕಾರಗಳ ಮದಿರೆಯನು ಕುಡಿದು\\ಉನ್ಮತ್ತನಾಗಿದ್ದನವನೊಬ್ಬನು. ಭೋಗಿಸಲು\\ಆರೋಗ್ಯವೂ ಇತ್ತು ಆ ವ್ಯಕ್ತಿಗೆ.\\ಗಿರಗಿರನೆ ಬುಗುರಿಯೊಲು ಸುತ್ತುತ್ತ ತೇಲುತ್ತ\\ಸಾಗುತಿದ್ದನು ತನ್ನ ಹುಚ್ಚುದಾರಿಯಲಿ:\\'ಈ ಭೂಮಿ, ನನ್ನದಿದೊ, ಸಂತಸದ ಉದ್ಯಾನ,\\ನನಗಾಗಿ ಮಾಡಿರುವ ಸುಖದಾಗರ;\\ಇಲ್ಲಿ ತೆವಳುತ್ತಿರುವ ಈ ಮನುಜಕ್ರಿಮಿ ನನ್ನ\\ಆಟದಾ ಬೊಂಬೆ – ನಾ ಸೂತ್ರಧಾರ!'\\ಭೋಗತೈಲದಿ ಉರಿವ ಸಂತಸದ ಸಾಸಿರದ ದೀಪಗಳನು,\\ಬಣ್ಣಗಳ ಬದಲಿಸುತ ಹಗಲಿರುಳು ಕಣ್ಮುಂದೆ\\ಕುಣಿಕುಣಿವ ಆಸೆಗಳ ಕುಡಿಗಳನ್ನು\\ಕಂಡು ಕಂಡೂ ಕಂಡು\\ಮಂಜುಕವಿಯಿತು ದೃಷ್ಟಿ; ಇಂದ್ರಿಯಕೊ\\ಎಲ್ಲವೂ ಸಪ್ಪೆ ಸಪ್ಪೆ.\\ಸ್ವಾರ್ಥವೆಲ್ಲವು ಮುಳ್ಳುಕಂಟಿಯೊಲು ಬೆಳೆಬೆಳೆದು\\ಹರಡಿಕೊಂಡಿತು ಎದೆಯ ಬಯಲ ತುಂಬ.\\ನಲಿವೆಲ್ಲ ನೋವಾಗೆ,\\ಭಾವನೆಯ ಶೂನ್ಯತೆಯು; ಇಂದ್ರಿಯದ ಬದುಕಂದು\\ಸಂತಸದ ಕಡಲಿನಲೆಯನ್ನೇಳಿಸುತ್ತಿತ್ತು,\\ಇಂದು ತೋಳಪ್ಪುಗೆಯ ಬಲೆಯೊಳಗೆ ಸಿಲುಕಿರುವ\\ಕೊಳೆತ ಹೆಣವು!\\ಕೊಡಹುವೆನು ಇದನೆಂದು ಮರಮರಳಿ ಯತ್ನಿಸಿದ\\ಬಿಡಿಸಿದಷ್ಟೂ ಅಂಟು ಬೆಳೆಯುತಿತ್ತು!\\ಹುಚ್ಚು ಕೆರಳಿತು, 'ಸಾವು ಸಾಸಿರದ ರೂಪದಲ್ಲಿ\\ಮುತ್ತಬಾರದೆ ಎನ್ನ?'\\ಜೀವದಾಸೆಯ ಮುಂದೆ ಸಾವು ನಿಲುವುದೆ?–\\ಎಲ್ಲೊ ಮುದುಡಿ ಮಲಗಿತ್ತು.\\ಆಗ ಬಂದಿತು ನೋವು–ಸಂಕಟದ ಸಾಗರವು –\\ಐಶ್ವರ್ಯ–ಅಧಿಕಾರ ಮಂಗಮಾಯ!\\ಕೆಳಗೆ ಉರುಳಿದನಿವನು – ಕಂಡುಕೊಂಡನು ತನ್ನ\\ಸೋದರತೆಯನು ಸಕಲ ಮನುಜಕುಲದಿ.\\ನರಳಾಟ–ಕಂಬನಿಯ ಕಂಡು ನೇಹಿಗರೆಲ್ಲ ನಗುತಿದ್ದರೂ\\ಇವನ ತುಟಿಗಳು ಮಾತ್ರ\\ಎದೆ ತುಂಬಿ ಬಹ ಭಾವ–\\ಸಾಗರವ ನುಡಿದಿಹವು–\\'ಸಂಕಟವೆ ನೀನಿಂದು ಧನ್ಯ, ಧನ್ಯ!'
\end{myquote}

\begin{center}
೩–
\end{center}

\begin{myquote}
ಮತ್ತೊಬ್ಬನಿದ್ದನವ ದೃಢಕಾಯನು –\\ಹೃದಯದಾಳದ ಪ್ರಬಲ ಭಾವದಲೆಗಳ ತಡೆವ\\ಸಂಕಲ್ಪಶಕ್ತಿಯನ್ನು ಪಡೆಯದವನು;\\ಅವ್ಯಕ್ತಶಕ್ತಿಯಲಿ ನುಗ್ಗಿ ಬರುತಿಹ ಚಿತ್ರ–\\ದೊತ್ತಡದ ರಭಸವನು ತಡೆಯದವನು.\\ಜಗದ ಕಣ್ಣಲ್ಲಿವನು ಸಜ್ಜನನು, ಕರುಣಾಳು;\\ನುಗ್ಗಿ ಬಹ ಅಲೆಯೆದುರು ಗುದ್ದಾಡುತಿರೆ ಪರರು\\ಸುಕ್ಷೇಮ ಇವನಿಲ್ಲಿ, ಆಪತ್ತುಗಳಿಗೆಂದು\\ಸಿಲುಕದವನು!
\end{myquote}

\begin{myquote}
ಕೊಳೆತು ನಾರುವ ಹುಣ್ಣನರಸಿ ಹೋಗುವ ನೊಣದ\\ತೆರದಲೀತನ ಮನಸು ಹೀನವಾಯ್ತು;\\ಎಲ್ಲೆಲ್ಲು ದೋಷವನೆ ಹುಡುಕುತಿತ್ತು.\\ಕಡೆಗೊಮ್ಮೆ ವಿಧಿಯೊಲುಮೆ ಇವನಿಗಾಯ್ತು–\\ಜಾರಿಬಿದ್ದನು ಕೂಪದಾಳದಲ್ಲಿ!\\ಆಗ ತೆರೆಯಿತು ಇವನ ಬಗೆಗಣ್ಣು\\ಮತ್ತೊಂದು ಮುಚ್ಚದಂತೆ:\\ಕಲ್ಲು –ಮರ–ಮಣ್ಣುಗಳು ನಿಯಮಗಳ ಮುರಿಯುವವೆ?\\ಆದರೆಂದೆಂದಿಗೂ\\ಅವು ಮರವು–ಮಣ್ಣು!\\ಮನುಜನೊಬ್ಬನೆ ವಿಧಿಯ ಸೆಣಸಿ ಗುದ್ದಾಡುವನು,\\ನಿಯಮ–ಮಿತಿಗಳನೆಲ್ಲ ಮೀರಿ ಗೆಲ್ಲುವನು –\\ಈ ನಿತ್ಯಸತ್ಯವನ್ನು ಕಂಡನವನು.
\end{myquote}

\begin{myquote}
ಸಪ್ಪೆತನ ಅವನಿಂದ ದೂರವಾಯ್ತು;\\ಹೊಸತು ದೃಷ್ಟಿಯಲೀಗ ಬದುಕು ವಿಸ್ತರಿಸಿರಲು\\ಬೆಳೆಬೆಳೆದು ವಿಸ್ತರಿಸಿ ನಿಂದನವನು.\\ಬೆಳಕು ಮೂಡಿತು – ನಿತ್ಯ ಶಾಂತಿಧಾಮದ ಇಣುಕು–\\ನೋಟ ಕಂಡಿತು – ಅದನು\\ಸೇರಲೊಂದೇ ದಾರಿ ಲೋಕದಲ್ಲಿ,\\ಸಂಘರ್ಷಸಾಗರದ ಮಧ್ಯದಲ್ಲಿ.\\ಈಸುತ್ತ ಸಾಗಿದನು, ಎದೆಗಾರಿಕೆಯ ಹೊನಲು\\ಹರಿಯಿತಲ್ಲಿ!
\end{myquote}

\begin{myquote}
ಬಳಿಕ ಹಿಂದಕೆ ತಿರುಗಿ ನೋಡಿದಾಗ ಅವ ಕಂಡುದೇನನ್ನು?–\\ಕಲ್ಲು–ಮರ–ಮಣ್ಣುಗಳ ಇರವಿನೊಳಗೊಂದಾಗಿ\\ತನ್ನ ತಾ ಮರೆತಿದ್ದ ದಿನಗಳನ್ನು;\\ಯಾವ ಕಾರಣಕೆ ಜಗ ಇವನಾಚೆಗಟ್ಟಿತ್ತೊ\\ಆ ಅಧಃಪತನದ ಕ್ಷಣಗಳನ್ನು!\\ಎದೆ ತುಂಬಿ ಬಂದಿರಲು ಉದ್ಘೋಷಗೈದನವ–\\'ಓ, ಅಧಃಪತನವೇ, ಧನ್ಯ ನೀನು!'
\end{myquote}

\selecteng

\chapter[HOLD ON YET A WHILE, BRAVE HEART]{\enginline{HOLD ON YET A WHILE, BRAVE HEART}\protect\footnote{\engfoot{C.W, Vol IV, P. 389}}}

\begin{myquote}
\enginline{If the sun by the cloud is hidden a bit,\\
If the welkin shows but gloom,\\
Still hold on yet a while, brave heart,\\
The victory is sure to come.}
\end{myquote}

\begin{myquote}
\enginline{No winter was but summer came behind,\\
Each hollow crests the wave,\\
They push each other in light and shade;\\
Be steady then and brave.}
\end{myquote}

\begin{myquote}
\enginline{The duties of life are sore indeed,\\
And its pleasures fleeting, vain,\\
The goal so shadowy seems and dim,\\
Yet plod on through the dark, brave heart,\\
With all thy might and main.}
\end{myquote}

\begin{myquote}
\enginline{Not a work will be lost, no struggle vain,\\
Though hopes be blighted, powers gone;\\
Of thy loins shall come the heirs to all,\\
Then hold on yet a while, brave soul,\\
No good is e’er undone.}
\end{myquote}

\begin{myquote}
\enginline{Though the good and the wise in life are few,\\
Yet theirs are the reins to lead,\\
The masses know but late the worth;\\
Heed none and gently guide.}
\end{myquote}

\begin{myquote}
\enginline{With thee are those who see afar,\\
With thee is the Lord of might,\\
All blessings pour on thee, great soul,\\
To thee may all come right!}
\end{myquote}

\selectkan

\begin{center}
\textbf{ಧೀರಾತ್ಮನಿಗೆ}
\end{center}

\enginline{'Hold on Yet a while Brave Heart'} ಎಂಬ ಈ ಇಂಗ್ಲಿಷ್ ಕವನವನ್ನು ಸ್ವಾಮಿಜಿ ತಮ್ಮ ಶ್ರದ್ಧಾವಂತ ಶಿಷ್ಯನಾದ ಖೇತ್ರಿಯ ಮಹಾರಾಜನಿಗೆ ಬರೆದು ಕಳಿಸಿದ್ದರು. ಎಂತಹ ಪ್ರತಿಕೂಲ ಪರಿಸ್ಥಿತಿಯಲ್ಲೂ ಎದೆಗೆಡದೆ ಮುಂದುವರಿಯಬೇಕೆಂಬ ಸ್ಫೂರ್ತಿದಾಯಕ ಕರೆ ಈ ಕವನದಲ್ಲಿದೆ.

\begin{myquote}
ಸೂರ್ಯನು ಮುಗಿಲಲಿ ಮರೆಯಾಗಿದ್ದರು,\\ಕಾಳಿಮೆಯಲಿ ಬಾನ್ ಮುಳುಗಿದರು,\\ಹತಾಶೆಗೊಳದಿರು, ಓ ಧೀರಾತ್ಮನೆ,
\end{myquote}

\begin{flushright}
ಜಯವದು ನಿಶ್ಚಿತ, ನಿಶ್ಚಿತವು!
\end{flushright}

\begin{myquote}
ಶಿಶಿರದ ಬೆನ್ನೆಡೆ ವಸಂತ ಬರುವುದು,\\ಇಳಿದರು ಅಲೆ ಮೇಲೇಳುವುದು;\\ನೆರಳ ನಡುವಿನಿಂ ಬೆಳಕು ನುಗ್ಗುವುದು
\end{myquote}

\begin{flushright}
ಸ್ಥಿರವಾಗಿರು, ಓ ಧೀರಾತ್ಮ!
\end{flushright}

\begin{myquote}
ಬಾಳ ಹೊಣೆಗಳವು ಬೇವಾಗುವವು,\\ಸಂತಸ ಕ್ಷಣದಲ್ಲಿ ಮುಗಿಯುವುದು;\\ಗುರಿಯನು ನೆರಳದು ನುಂಗುವುದು
\end{myquote}

\begin{flushright}
– ಕತ್ತಲ ಸೀಳುತ ನಡೆ, ಧೀರಾತ್ಮನೆ,\\ಸಂತತ ಸಾವಿರ ಯತ್ನದಲಿ!
\end{flushright}

\begin{myquote}
ಇಟ್ಟ ಅಡಿಗಳು, ಪಟ್ಟ ಪಾಡುಗಳು\\ವ್ಯರ್ಥವಲ್ಲವೆಂದೆಂದಿಗೂ;\\ಕನಸು ಕರಗಿದರು, ಶಕ್ತಿಗುಂದಿದರು\\ನಿನ್ನುದ್ಧಾರಕನುದಿಸುವನು!
\end{myquote}

\begin{flushright}
– ತಡೆ, ತಡದಿರು ನೀ, ಓ ಧೀರಾತ್ಮನೆ,\\ಒಳಿತೆಂದಿಗು ಅಳಿಯುವುದಿಲ್ಲ!
\end{flushright}

\begin{myquote}
ತಿಳಿದು ಬಾಳುವರು ಕೆಲವರಾದರೂ\\ಅವರೆ ಬಾಳ್ಗೆ ಬೆಳಕೀಯುವರು;\\ಮಂದಿ ನಿಜದ ಬೆಲೆಯರಿವುದೆಂದಿಗೋ,\\ಕಿವುಡನೆನಿಸವರ ನುಡಿಗಳಿಗೆ!
\end{myquote}

\begin{myquote}
ದೂರದರ್ಶಿಗಳು ನೆರವಿಗೆ ನಿಲುವರು,\\ಸರ್ವಶಕ್ತ ನಿನ್ನೊಡನಿಹನು;\\ಹರಕೆಯದೆಲ್ಲವು ಹರಿದು ಬರಲಿ, ನೀ\\ನಡೆ, ಧೀರಾತ್ಮನೆ, ಗುರಿಯೆಡೆಗೆ!
\end{myquote}

\selecteng

\chapter[TO THE AWAKENED INDIA]{\enginline{TO THE AWAKENED INDIA}\protect\footnote{\engfoot{C.W, Vol. V, P. 387}}}

Once more awake!

\begin{myquote}
\enginline{For sleep it was, not death, to bring thee life\\
Anew, and rest to lotus–eyes, for visions\\
Daring yet. The world in need awaits, O Truth!\\
No death for thee!}
\end{myquote}

Resume thy march,

\begin{myquote}
\enginline{With gentle feet that would not break the\\
peaceful rest even of the roadside dust\\
That lies so low. Yet strong and stedy,\\
Blissful, bold, and free. Awakener, ever\\
Forward! Speak thy stirring words.}
\end{myquote}

Thy home is gone,

\begin{myquote}
\enginline{Where loving hearts had brought thee up, and\\
Watched with joy thy growth. But Fate is strong—\\
This is the law—all things come back to the source\\
They sprung, their strength to renew.}
\end{myquote}

Then start afresh

\begin{myquote}
\enginline{From the land of thy birth, where vast cloud–belted\\
Snows do bless and put their strength in thee,\\
For working wonders new. The heavenly\\
River tune thy voice to her own immortal. song;\\
Deodar shades give thee eternal peace.}
\end{myquote}

And all above,

\begin{myquote}
\enginline{Himala’s daughter Uma, gentle, pure,\\
The Mother that resides in all as Power\\
And Life, who works all works and\\
Makes of One the world, whose mercy\\
Opens the gate to Truth and shows\\
The One in All, give thee untiring\\
Strength, which is Infinite Love.}
\end{myquote}

They bless thee all,

\begin{myquote}
\enginline{The seers great, whom age nor clime\\
Can claim their own, the fathers of the\\
Race, who felt the heart of Truth the same,\\
And bravely taught to man ill–voiced or\\
well. Their servant, thou hast got\\
The secret–’ tis but one.}
\end{myquote}

Then speak, O Love

\begin{myquote}
\enginline{Before thy gentle voice serene, behold how\\
Visions melt, and fold on fold of dreams\\
Departs to void, till Truth and Truth alone.\\
In all its glory shines—}
\end{myquote}

And tell the world

\begin{myquote}
\enginline{Awake, arise, and dream no more!\\
This is the land of dreams, where Karma\\
Weaves unthreaded garlands with our thoughts,\\
Of flowers sweet or noxious, and none\\
Has root or stem, being born in naught, which\\
The softest breath of Truth drives back to\\
Primal nothingness. Be bold, and face\\
The Truth! Be one with it! Let visions cease,\\
Or, if you cannot, dream but truer dreams,\\
Which are Eternal Love and Service Free.}
\end{myquote}

\selectkan

\begin{center}
\textbf{'ಪ್ರಬುದ್ಧ ಭಾರತ'ಕ್ಕೆ}
\end{center}

ಶ‍್ರೀನಗರದಲ್ಲಿ ೧೮೯೮ರ ಜೂನ್ ತಿಂಗಳಿನಲ್ಲಿ 'ಪ್ರಬುದ್ಧ ಭಾರತ' ಪತ್ರಿಕೆಗೆ ರಚಿಸಿದ ಕವನ; ಮೊದಲು ಮದ್ರಾಸಿನಿಂದ ಪ್ರಕಟವಾಗುತ್ತಿದ್ದ ಪತ್ರಿಕೆ ಆಲ್ಮೋರಾಕ್ಕೆ ವರ್ಗಾವಣೆಯಾದ ಸಂದರ್ಭ. ಸತ್ಯದ ಮುಖವಾಣಿಯಾಗಬೇಕೆಂದು ಸ್ವಾಮಿಜಿಯವರು ಪತ್ರಿಕೆಗೆ ನೀಡಿದ ಕರೆ ಈ ಕವನದಲ್ಲಿದೆ.

ಏಳು ನೀ ಮತ್ತೊಮ್ಮೆ!

\begin{myquote}
ನಿನ್ನದಿದು ಬರಿ ನಿದ್ರೆ, ಮರಣವಲ್ಲವಿದು;\\ಬಳಲಿದ್ದ ನಯನಕುಸುಮಗಳ ಭಾರವ ಕಳೆದು\\ಭವ್ಯತಮ ದರ್ಶನಕ್ಕೆ ಸಜ್ಜುಗೊಳಿಸಿಹುದು;\\ಜಗವೆಲ್ಲ ನಿನಗಾಗಿ ಕಾಯುತಿಹುದು,\\ಪರಮಸತ್ಯವೆ, ನಿನಗೆ ಅಳಿವಿಲ್ಲವು!
\end{myquote}

ಮತ್ತೆ ಮುನ್ನಡೆ ನೀನು!

\begin{myquote}
ಶಾಂತಿ–ವಿಶ್ರಾಂತಿಗಳ ಕೆಡಿಸದಂತೆ,\\ಬೀದಿಧೂಳೂ ಕೂಡ ಅಲುಗದಂತೆ\\ಮೆಲ್ಲಡಿಯನಿಟ್ಟರೂ ದಿಟ್ಟ ನಡೆ ನಿನದಿರಲಿ;\\ಬಿಡುಗಡೆಯ ಆನಂದ ಹೊರಹೊಮ್ಮಲಿ;\\ಸ್ಫೂರ್ತಿವಾಣಿಯು ಮತ್ತೆ ಚಿಮ್ಮುತಿರಲಿ!
\end{myquote}

ಇನ್ನಿಲ್ಲ ನಿನಗೆ ಮನೆ–

\begin{myquote}
ಒಲವಿನೆದೆಗಳು ನಿನ್ನ ಬೆಳಸಿದ್ದುವಲ್ಲಿ.\\ಇಂತಾದರೂ ವಿಧಿಯೆ ಬಲು ಪ್ರಬಲವಿಲ್ಲಿ.\\ನಿಯಮವಿದು – ವ್ಯಕ್ತ ವಸ್ತುಗಳೆಲ್ಲ ಮತ್ತೊಮ್ಮೆ\\ತಮ್ಮ ಮೂಲಕ ತಾವು ಬರಲೆಬೇಕು;\\ನವಚೇತನವ ಮರಳಿ ಪಡೆಯಬೇಕು.
\end{myquote}

ಹೊಸ ಹೆಜ್ಜೆಯಿಟ್ಟು ನಡೆ

\begin{myquote}
ನಿನ್ನ ತಾಯ್ನೆಲದಿಂದ; ಮೋಡದೊಡ್ಯಾಣವನು\\ತೊಟ್ಟು ಹಿಮಶಿಖರವದು ಹೊಸ ವಿಸ್ಮಯಕ ಹರಸಿ\\ಬಲ ತುಂಬಿದೆ;\\ದೇವನದಿಯಮರಗಾನದ ಮಧುರ ಶ್ರುತಿಯೊಡನೆ\\ನಿನ್ನ ದನಿ ಸಮಶ್ರುತಿಯ ಮಿಡಿಯುತ್ತಿದೆ;\\ದೇವತರು ನೆರಳು ಶಾಂತಿಯ ಚೆಲ್ಲಿದೆ.
\end{myquote}

ಎಲ್ಲಕ್ಕೂ ಮಿಗಿಲಾಗಿ

\begin{myquote}
ಉಮೆ ಹೈಮವತಿಯವಳು, ಜೀವಿಗಳ ಎದೆಯಲ್ಲಿ\\ತಾಯಾಗಿ ನಿಂತುಸಿರ ತುಂಬಿದವಳು,\\ಎಲ್ಲವನು ಮಾಡುವಳು, ಜಗವನೊಂದಾಗಿಪಳು,\\ಕರುಣೆಯಲಿ ದಿಟದ ಬಾಗಿಲ ತೆರೆವಳು,\\ಆ ಅನೇಕದ ಒಡಲಿನೇಕವನು ತೋರುವಳು,\\ದಣಿವರಿಯದಿಹ ಶಕ್ತಿ, ಒಲವೀವಳು.
\end{myquote}

ನಿನ್ನ ಹರಸಲಿ ಅವರು

\begin{myquote}
ಆ ಮಹಾ ಋಷಿಗಳು, ಕಾಲದೇಶಗಳನ್ನು\\ಮೀರಿದವರು;\\ಕುಲಕೆ ಮೊದಲಿಗರವರು, ದಿಟದಾಳಕಿಳಿದವರು,\\ಮಸಕುನುಡಿಯೊಳೊ ಸ್ಪಷ್ಟವಾಣಿಯೊಳೊ\\ಸಾರಿದರು ಪರಮ ಸತ್ಯವನವರು, ಧೀರರವರು;\\ಅವರ ಕಿಂಕರನು ನೀ, ಗುಟ್ಟನರಿತಿರುವೆ\\ಹಲವಿಲ್ಲವಿಲ್ಲಿ ತಾನೆಲ್ಲವೊಂದೆ.
\end{myquote}

ನನ್ನೊಲವೆ ನುಡಿ ನೀನು–

\begin{myquote}
ಮೃದುಮಧುರಗಂಭೀರ ವಾಣಿಯನ್ನಾಲಿಸುತ\\ಮೃಗಜಲವು ಬತ್ತುವುದು, ಕನಸು ಕರಗುವುದು;\\ಸತ್ಯವೊಂದೇ ಕೊನೆಗೆ ತನ್ನೆಲ್ಲ ವೈಭವದಿ ಬೆಳಕೀವುದು.
\end{myquote}

ಸಾರು ನೀ ಲೋಕಕ್ಕೆ:

\begin{myquote}
ಏಳು, ಮೇಲೇಳೇಳು, ಕನಸ ಕಾಣದಿರು!\\ಈ ಸ್ವರಾಜ್ಯದೊಳು ಮನವೆಂಬ ಕಟು–ಮಧುರ\\ಪುಷ್ಪಗಳ ಮಾಲೆಯನು ಕರ್ಮವದು ಕಟ್ಟುವುದು\\ಸೂತ್ರವಿಲ್ಲದಲೆ!\\ಅಡಿಮುಡಿಗಳದಕಿಲ್ಲ – ಸೊನ್ನೆಯಿಂ ಬಂದುದದು\\ಸೊನ್ನೆಯಲೆ ಕರಗುವುದು ಸತ್ಯದುಸಿರದು ಇನಿತೆ\\ಸೋಕಿ ಸುಳಿದಿರಲು,\\ಧೈರ್ಯದಲಿ ಸತ್ಯಕ್ಕೆ ಮುಖವಾಗು, ಒಂದಾಗು\\ಸತ್ಯದಲಿ, ಕನಸೆಲ್ಲ ತಾನಳಿಯಲಿ.\\ಇಲ್ಲದಿರೆ ನಿಜದ ಕನಸುಗಳ ನೀ ಕಾಣದುವೆ\\ಶಾಶ್ವತದ ಪ್ರೇಮ ಮೇಣ್ ನಿಃಸ್ವಾರ್ಥ ಸೇವೆ.
\end{myquote}

\selecteng

\chapter[TO THE FOURTH OF JULY]{\enginline{TO THE FOURTH OF JULY}\protect\footnote{\engfoot{C.W. Vol. V, P.439}}}

\begin{myquote}
\enginline{Behold, the dark clouds melt away,\\
That gathered thick at night, and hung\\
So like a gloomy pall above the earth!\\
Before thy magic touch, the world\\
Awakes.The birds in chorus sing.\\
The flowers raise their star—like crowns—\\
Dew–set, and wave thee welcome fair.\\
The lakes are opening wide in love\\
Their hundred thousand lotus–eyes\\
To welcome thee, with all their depth.\\
All hail to thee, thou Lord of Light!\\
A welcome new to thee, today,\\
O Sun! Today thou sheddest Liberty!}
\end{myquote}

\begin{myquote}
\enginline{Bethink thee how the world did wait,\\
And search for thee, through time and clime.\\
Some gave up home and love of friends,\\
And went in quest of thee, self–banished,\\
Through dreary oceans, through primeval forests,\\
Each step a struggle for their life or death;\\
Then came the day when work bore fruit;\\
And worship, love, and sacrifice,\\
Fulfilled, accepted, and complete.\\
Then thou, propitious, rose to shed\\
The light of Freedom on mankind.}
\end{myquote}

\begin{myquote}
\enginline{Move on, O Lord, in the resistless path!\\
Till thy high noon o’erspreads the world.\\
Till every land reflects thy light,\\
Till men and women, with uplifted head,\\
Behold their shackles broken, and\\
Know, in springing joy, their life renewed!}
\end{myquote}

\selectkan

\begin{center}
\textbf{ಜುಲೈ ನಾಲ್ಕನೇ ದಿನಕ್ಕೆ}
\end{center}

೧೮೯೮ರ ಜುಲೈ ೪ರಂದು ಸ್ವಾಮಿಜಿಯವರು ಕಾಶ್ಮೀರದಲ್ಲಿದ್ದಾಗ \enginline{The Fourth of July'} ಎಂಬ ಈ ಇಂಗ್ಲೀಷ್ ಕವನವನ್ನು ರಚಿಸಿದರು. ಜುಲೈ ನಾಲ್ಕು ಅಮೆರಿಕದ ಸ್ವಾತಂತ್ರ್ಯ ದಿನ. ಅಂದು ಸ್ವಾಮಿಜಿ ತಮ್ಮ ಅಮೆರಿಕನ್ ಶಿಷ್ಯರೊಂದಿಗೆ ಈ ಸ್ವಾತಂತ್ರ್ಯ ದಿನವನ್ನು ಆಚರಿಸುವಾಗ ಈ ಕವನವನ್ನು ಓದಿದರು.

ಒಂದೆಡೆ ಈ ಕವನ ಅಮೆರಿಕದ ರಾಜಕೀಯ ಸ್ವಾತಂತ್ರ್ಯವನ್ನು ಕುರಿತದ್ದಾದರೂ ಇನ್ನೊಂದೆಡೆ ಇದು ತ್ಯಾಗ–ಸಂಘರ್ಷಗಳ ಮೂಲಕ ಮುಕ್ತಿಯೆಡೆಗೆ ನಡೆಯುವ ಆತ್ಮನ ಸ್ವಾತಂತ್ರ್ಯವನ್ನೂ ಕುರಿತದ್ದು. ನಾಲ್ಕು ವರ್ಷಗಳ ಬಳಿಕ, ಎಂದರೆ ೧೯೦೨ರ ಜುಲೈ ೪ರಂದು, ಸ್ವಾಮಿಜಿ ಶರೀರತ್ಯಾಗ ಮಾಡಿ 'ಅಂತಿಮ ಸ್ವಾತಂತ್ರ್ಯ'ವನ್ನು ಪಡೆದದ್ದರಿಂದ, ಇನ್ನೊಂದು ವಿಶೇಷ ಮಹತ್ವವೂ ಈ ಕವನಕ್ಕಿದೆ.

ವಿವರಗಳಿಗೆ ನೋಡಿ: ವಿಶ್ವಮಾನವ ವಿವೇಕಾನಂದ 'ಝೇಲಮಿನ ಜಲದ ಮೇಲೆ' ಎಂಬ ಅಧ್ಯಾಯ.

\begin{myquote}
ಕತ್ತಲಲಿ ದಟ್ಟಿಸಿ ಇಳೆಗೆ ಮಬ್ಬನು ಕವಿದ\\ಕಾಳಮೇಘಗಳೆಲ್ಲ ಕರಗುತಿವೆ ನೋಡು!\\ನಿನ್ನ ಮಂತ್ರಸ್ಪರ್ಶ ಜಗವನೆಚ್ಚರಿಸುತಿದೆ,\\ಹಕ್ಕಿಗಳ ಇಂಪುದನಿ ಗುಂಪಾಗಿದೆ;\\ಹಿಮಮಣಿಗಳಿಂದಿಡಿದ ನಕ್ಷತ್ರಮುಕುಟಗಳ\\ಹೂವುಗಳು ಮೇಲೆತ್ತಿ ತೂಗುತ್ತಿವೆ;\\ನಿನಗೆ ಸುಸ್ವಾಗತವ ಹಾರೈಸಿವೆ!\\ನೂರುಸಾಸಿರದಳದ ಕಮಲನಯನಗಳಿಂದ\\ಒಲುಮೆಯಾಳವ ಸರಸಿ ತೆರೆಯುತ್ತಿವೆ;\\ನಿನಗೆ ಸುಸ್ವಾಗತವ ಹಾರೈಸಿವೆ!\\ಎಲ್ಲವೂ ನಿನಗೆ ಜಯಜಯವೆಂದಿವೆ!\\ಓವೊ, ಬೆಳಕಿನ ಪ್ರಭುವೆ,\\ಸ್ವಾಗತವು ನಿನಗೆ;\\ನೀನಿಂದು ಮುಕ್ತಿಯನ್ನು ಪಸರಿಸಿರುವೆ!
\end{myquote}

\begin{myquote}
ಎನಿತೊ ನಾಡುಗಳಲ್ಲಿ ಎನಿತೊ ಕಾಲಗಳಲ್ಲಿ\\ನಿನಗಾಗಿ ಜಗವೆನಿತು ಕಾಯುತಿತ್ತು!\\ನಿನ್ನನ್ನೆ ಹಗಲಿರುಳು ಹುಡುಕುತಿತ್ತು.\\ಮನೆಯ ತೊರೆದರು ಕೆಲರು, ನೇಹಿಗರನುಳಿದವರು,\\ನಿನ್ನನರಸುತ ಜೀವ ತೇದರವರು;\\ಕಡಲಿನಲೆಗಳ ದಾಟಿ, ಕಾನನವ ಬೆನ್ನಟ್ಟಿ,\\ಅಡಿಯ ಕಿತ್ತಡಿಯಿಡುತ ನಡೆದರವರು\\ಸಾವು–ಬದುಕಿನ ನಡುವೆ ಹೆಣಗಿದವರು!\\ಕೊನೆಗೊಂದು ದಿನ ಬಂತು, ಕೆಲಸ ಹಣ್ಣಾಯಿತು,\\ತ್ಯಾಗ – ಪೂಜೆಗಳೆಲ್ಲ ಪೂರ್ಣವಾಯ್ತು.\\
ಅಂದು, ಕರುಣಾನಿಧಿಯೆ, ಮೇಲೆದ್ದು ನಿಂತು ನೀ\\ಮುಕ್ತಿಪ್ರಭೆಯನು ಸುರಿದೆ ಜಗದ ಮೇಲೆ!
\end{myquote}

\begin{myquote}
ನಡೆ ಮುಂದೆ, ಓ ಪ್ರಭುವೆ, ತಡೆಯದಲೆ ನಡೆ ಮುಂದೆ,\\ನಾಡದೆಲ್ಲವು ಬೆಳಕ ಪ್ರತಿಫಲಿಸುವರೆಗೂ,\\ನಿನ್ನ ಬೆಳಕದು ಜಗವ ತಬ್ಬುವರೆಗೂ,\\ಜನವೆಲ್ಲ ತಲೆಯೆತ್ತಿ, ಸಂಕೋಲೆಗಳ ಕಡಿದು\\ಸಂತಸದಿ ಮರುಜೀವ ಪಡೆವವರೆಗೂ!
\end{myquote}

\selecteng

\chapter[THOU BLESSED DREAM]{\enginline{THOU BLESSED DREAM}\protect\footnote{\engfoot{C.W, Vol. VIII, P.168}}}

\begin{center}
(Written in Paris on August 17, 1900.)
\end{center}

\begin{myquote}
\enginline{If things go ill or well—\\
If joy rebounding spreads the face,\\
Or sea of sorrow swells–—\\
A play—all each have part,\\
Each one to weep or laugh as may;\\
Each one his dress to don—\\
Its scenes, alternative shine and rain.}
\end{myquote}

\begin{myquote}
\enginline{Thou dream, O blessed dream!\\
Spread far and near thy veil of haze,\\
Tone down the lines so sharp,\\
Make smooth what roughness seems.}
\end{myquote}

\begin{myquote}
\enginline{No magic but in thee!\\Thy touch makes deserts bloom to life.\\Harsh thunder, sweetest song,\\Fell death, the sweet release.}
\end{myquote}

\selectkan

\begin{center}
\textbf{ನೀ ಧನ್ಯ, ಕನಸೆ!}
\end{center}

೧೯೦೦ರ ಆಗಸ್ಟ್ ೧೭ರಂದು ಪ್ಯಾರಿಸಿನಲ್ಲಿದ್ದಾಗ ಬರೆದ \enginline{'Thou Blessed Dream'} ಎಂಬ ಈ ಕವನವನ್ನು ಸ್ವಾಮಿಜಿ ಸೋದರಿ ಕ್ರಿಸ್ಟೀನೆಗೆ ಕಳಿಸಿದ್ದರು. ಆದರ್ಶದ ಕನಸಿಗೆ ಬಲವಾಗಿ ಅಂಟಿಕೊಂಡಿರುವ ವ್ಯಕ್ತಿ, ವಾಸ್ತವದಲ್ಲಿ ಅನಿವಾರ್ಯವಾಗುವ ಕಠಿಣ ಪರಿಸ್ಥಿತಿಗಳನ್ನು ಎದುರಿಸುವ ಮನೋಧರ್ಮವನ್ನು ಹೇಗೆ ಪಡೆದುಕೊಳ್ಳುವನೆಂಬುದನ್ನು ಈ ಕವನದಲ್ಲಿ ನಿರೂಪಿಸಲಾಗಿದೆ. ಈ ದೃಷ್ಟಿಯಿಂದ, ಕನಸು ಇಲ್ಲಿ ಓರ್ವ 'ಮಂತ್ರವಾದಿ'ಯಾಗುತ್ತದೆ; ವ್ಯಾವಹಾರಿಕದ ಕಲ್ಲುಗಳನ್ನು ಕುಸುಮಗಳನ್ನಾಗಿಸುತ್ತದೆ.

\begin{myquote}
ಸುಖವೆ ಸುರಿದರು, ದುಃಖ ಹರಿದರು,\\ಪುಟಿಪುಟಿವ ಸಂತೋಷ ಮುಖದಿ ನಲಿದಾಡಿದರು,\\ಸಂಕಟದ ಸಾಗರವೆ ಭೋರ್ಗರೆದರೂ\\ಅವರವರ ಪಾತ್ರಗಳನಭಿನಯಿಸಬೇಕು;\\ನಗಬೇಕು, ಅಳಬೇಕು,\\ತನ್ನ ವೇಷದ ತಾನೆ ಧರಿಸಬೇಕು;\\ಒಮ್ಮೆ ಬೆಳಗುವ ರವಿಯು,\\ಮತ್ತೊಮ್ಮೆ ಸುರಿಮಳೆಯು,\\- ಬದಲಾಗಿ ದೃಶ್ಯಗಳು ಸಾಗಲೇಬೇಕು!
\end{myquote}

\begin{myquote}
ಓ ಕನಸೆ, ಸವಿಗನಸೆ,\\ದೂರಸನಿಹಗಳಲ್ಲಿ ಹಿಮದ ತೆರೆಯನ್ನಿಳಿಸು,\\ಮುಳ್ಳುಗಳ ಮೃದುಗೊಳಿಸು,\\ಕಲ್ಲುಗಳ ನಯಗೊಳಿಸು;\\ಆವ ಮಾಂತ್ರಿಕನಿಹನು ನಿನ್ನ ಹೊರತು!
\end{myquote}

\begin{myquote}
ನೀ ಸೋಕೆ ಬೆಂಗಾಡು\\ಹಸಿರುಟ್ಟು ನಿಲ್ಲುವುದು,\\ಸಿಡಿಲ ಗರ್ಜನೆ ಮಧುರಗಾನವಹುದು,\\ಮೃತ್ಯು ಚಿರಮಾಧುರ್ಯ ಮುಕ್ತಿಯಹುದು!
\end{myquote}

\selecteng

\chapter[LIGHT]{\enginline{LIGHT}\protect\footnote{\engfoot{C.W, Vol. VIII, P.168}}}

\begin{myquote}
\enginline{I look behind and after\\
And find that all is right,\\
In my deepest sorrows\\
There is a soul of light.}
\end{myquote}

\selectkan

\begin{center}
\textbf{ಬೆಳಕು}
\end{center}

\enginline{'Light'} ಎಂಬ ಈ ಕಿರುಗವನ ಸ್ವಾಮಿಜಿಯರು ೧೯೦೦ರ ಡಿಸೆಂಬರ್ ೨೬ರಂದು ಬೇಲೂರುಮಠದಿಂದ ಕುಮಾರಿ ಜೋಸೆಫೈನ್ ಮ್ಯಾಕ್ಲಾಡಳಿಗೆ ಬರೆದ ಪತ್ರದಲ್ಲಿದೆ. ಜೀವನದಲ್ಲಿ ನೋವು–ಸಂಕಟಗಳನ್ನು ಅನುಭವಿಸುವ ಅಂದಂದಿನ ಕ್ಷಣಗಳಲ್ಲಿ ಅವು ಅಸಹನೀಯವೆಂದು ತೋರಿದರೂ, ಬದುಕಿನ ಸಮಗ್ರ ವ್ಯವಸ್ಥೆಯಲ್ಲಿ ಪ್ರತಿಯೊಂದಕ್ಕೂ ಒಂದೊಂದು ಸ್ಥಾನವಿದ್ದು ಸಾರ್ಥಕತೆಗೆ ಕಾರಣವಾಗುತ್ತದೆ ಎಂಬ ತತ್ತ್ವ ಇಲ್ಲಿದೆ.

\begin{myquote}
ಹಿಂದುಮುಂದೆಲ್ಲವನು ಪರಿಕಿಸಲು ಕಾಣುವೆನು\\ಎಲ್ಲ ಸರಿಯಿರುವುದೆಂದು;\\ಆಳದಾಳದ ಕಹಿಯ ಅಳಲಿನಲು ಕಾಣುತಿದೆ\\ಬೆಳಗುತಿಹ ಕಿರಣವೊಂದು!
\end{myquote}

\selecteng

\chapter[THE LIVING GOD]{\enginline{THE LIVING GOD}\protect\footnote{\engfoot{C.W, Vol. VIII, P. 169}}}

\begin{myquote}
\enginline{He who is in you and outside you,\\
Who works through all hands,\\
Who walks on all feet,\\
Whose body are all ye,\\
Him worship, and break all other idols!}
\end{myquote}

\begin{myquote}
\enginline{He who is at once the high and low,\\
The sinner and the saint,\\
Both God and worm,\\
Him worship—visible, knowable, real, omnipresent,\\
Break all other idols!}
\end{myquote}

\begin{myquote}
\enginline{In whom is neither past life\\
Nor future birth nor death,\\
In whom we always have been\\
And always shall be one,\\
Him worship. Break all other idols!}
\end{myquote}

\begin{myquote}
\enginline{Ye fools! Who neglect the living God,\\
And His infinite reflections with which the world is full.\\
While ye run after imaginary shadows,\\
That lead alone to fights and quarrels,\\
Him worship, the only visible!\\
Break all other idols!}
\end{myquote}

\selectkan

\begin{center}
\textbf{ಜೀವಂತ ಭಗವಂತ}
\end{center}

\enginline{"The Living God"} ಎಂಬ ಈ ಕವನವನ್ನು ಸ್ವಾಮಿಜಿಯವರು ೧೮೯೭ರ ಜುಲೈ ೯ ರಂದು ತಮ್ಮ ಅಮೆರಿಕದ ಸ್ನೇಹಿತನೊಬ್ಬನಿಗೆ ಬರೆದುಕಳಿಸಿದ್ದರು. ಇದೇ ಭಾವವನ್ನು ಸ್ವಾಮಿಜಿ ಬೇರೆ ಬೇರೆ ಸಂದರ್ಭಗಳಲ್ಲಿಯೂ ವ್ಯಕ್ತಪಡಿಸಿದ್ದಾರೆ. ನಿದರ್ಶನಕ್ಕೆ ನೋಡುವುದಾದರೆ–

"... ಪೂಜೆಗಳಲ್ಲೆಲ್ಲ ಮೊದಲ ಪೂಜೆಯೆಂದರೆ 'ವಿರಾಟ್ ಪೂಜೆ', ನಮ್ಮ ಸುತ್ತಲೂ ಇರುವವರ ಪೂಜೆ, ಅವರನ್ನು ಪೂಜಿಸಿ....ಮನುಷ್ಯರು, ಪ್ರಾಣಿಗಳು ಇವರೇ ನಮ್ಮ ದೇವರು.....ಜಾಗೃತವಾಗಿರುವ ಏಕೈಕ ದೇವರೆಂದರೆ ಇದೊಂದೇ – ನಮ್ಮ ಜನಾಂಗ: 'ಎಲ್ಲೆಡೆಗಳಲೂ ಅವನ ಕೈಗಳು, ಎಲ್ಲೆಡೆಗಳಲೂ ಅವನ ಕಾಲ್ಗಳು, ಎಲ್ಲೆಡೆಗಳಲೂ ಅವನ ಕಿವಿಗಳು, ಎಲ್ಲವ ವ್ಯಾಪಿಸಿ ನಿಂತವನವನು!' ಇನ್ನೆಲ್ಲ ದೇವರುಗಳೂ ನಿದ್ರಿಸುತ್ತಿದ್ದಾರೆ. ನಮ್ಮ ಸುತ್ತಲೂ ಎಲ್ಲೆಲ್ಲಿಯೂ ಕಾಣುವ ಈ ದೇವರನ್ನು, ಈ ವಿರಾಟ್ ಸ್ವರೂಪವನ್ನು ನಾವು ಪೂಜಿಸಲಾರೆವಾದರೆ, ಇನ್ನಾವ ಕೆಲಸಕ್ಕೆ ಬಾರದ ದೇವರುಗಳನ್ನು ಹುಡುಕಿಕೊಂಡುಹೋಗೋಣ? ಈ ವಿರಾಟ್ ಭಗವಂತನನ್ನು ನಾವು ಪೂಜಿಸುವಂತಾದಾಗ, ಇನ್ನೆಲ್ಲ ದೇವರುಗಳನ್ನೂ ಪೂಜಿಸಬಲ್ಲೆವು."

\begin{myquote}
ಯಾರು ನಿನ್ನೊಳಹೊರಗ ತುಂಬುತ\\ದುಡಿವೆಲ್ಲ ಕೈಯಾಗಿರುವನೊ\\ನಡೆವೆಲ್ಲ ಕಾಲಾಗಿರುವನೊ\\ಎಲ್ಲರೊಡಲಾಗಿರುವನೊ
\end{myquote}

\begin{myquote}
ಅವನನಾರಾದಿಸುತ ಮುರಿದೆಸೆ\\ಮಿಕ್ಕ ಮೂರ್ತಿಗಳೆಲ್ಲವ!
\end{myquote}

\begin{myquote}
ಮೇಲು ಕೆಳಗುಗಳೆಲ್ಲ ವ್ಯಾಪಿಸಿ\\ಪಾಪಿಯಲ್ಲಿಯು ಸಂತನಲ್ಲಿಯು\\ದೇವನಲ್ಲಿಯು ಕೀಟದಲ್ಲಿಯು\\ದಿಟದ ತಿಳಿವಾಗಿರುವನೊ
\end{myquote}

\begin{myquote}
ಅವನನಾರಾಧಿಸುತ ಮುರಿದೆಸೆ\\ಮಿಕ್ಕ ಮೂರ್ತಿಗಳೆಲ್ಲವ!
\end{myquote}

\begin{myquote}
ಹಿಂದು ಮುಂದುಗಳಾರಲಿಲ್ಲವೊ\\ಜನನಮರಣಗಳಿಲ್ಲವೊ,\\ನಮ್ಮ ನಿತ್ಯದ ನೆಲೆಯದಾವನೊ,\\ನಮ್ಮ ಇರುವಿಕೆಯೆಲ್ಲ ಸಂತತ\\ಯಾರಲೊಂದಾಗಿರುವುದೊ
\end{myquote}

\begin{myquote}
ಅವನನಾರಾಧಿಸುತ ಮುರಿದೆಸೆ\\ಮಿಕ್ಕ ಮೂರ್ತಿಗಳೆಲ್ಲವ!
\end{myquote}

\begin{myquote}
ಯಾರ ಪ್ರತಿಬಿಂಬಗಳು ಜಗದಲಿ\\ಎಡೆಬಿಡದೆ ತುಂಬಿರುವುವೋ\\ಅವನ ಬೆಳಕಿಗೆ ಕುರುಡರಾಗುತ\\ನೆರಳುಗಳ ಹಿಂದೋಡುತ\\ಭ್ರಾಂತಿ – ಕದನದ ಕದವ ತೆರೆವೀ\\ಮೂರ್ಖತನವಿನ್ನೇತಕೆ?\\ಯಾರು ಚೇತನದೇವನೆನಿಪನೊ
\end{myquote}

\begin{myquote}
ಅವನನಾರಾಧಿಸುತ ಮುರಿದೆಸೆ\\ಮಿಕ್ಕ ಮೂರ್ತಿಗಳೆಲ್ಲವ!
\end{myquote}

\selecteng

\chapter[TO AN EARLY VIOLET]{\enginline{TO AN EARLY VIOLET}\protect\footnote{\engfoot{C.W, Vol. VIII, P. 169}}}

\begin{myquote}
\enginline{What though thy bed be frozen earth,\\
Thy cloak the chilling blast;\\
What though no mate to cheer thy path,\\
Thy sky with gloom o’ercast;\\
What though if love itself doth fail,\\
Thy fragrance stewed in vain;\\
What though if bad o’er good prevail,\\
And vice o’er virtue reign;\\
Change not thy nature, gentle bloom,\\
Thou violet, sweet and pure,\\
But ever pour thy sweet perfume\\
Unasked, unstinted, sure!}
\end{myquote}

\selectkan

\begin{center}
\textbf{ಮುನ್ನವೆ ಅರಳಿದ ನೀಲಕುಸುಮಕೆ}
\end{center}

\enginline{'To an Early Violet'} ಎಂಬ ಈ ಕವನ ಸ್ವಾಮಿಜಿ ೧೮೯೬ರ ಜನವರಿ ೬ರಂದು ನ್ಯೂಯಾರ್ಕಿನಿಂದ ಸೋದರಿ ಕ್ರಿಸ್ಟೀನೆಗೆ ಬರೆದು ಕಳಿಸಿದ್ದು, ಪಾಶ್ಚಾತ್ಯ ದೇಶಗಳಲ್ಲಿ ಈ ವಯಲೆಟ್ ಎಂಬುದು ವಸಂತದಲ್ಲಿ ಅರಳುವ ಪುಷ್ಪ. ಆದರೆ ಅದೇನಾದರೂ ವಸಂತಕ್ಕೆ ಮೊದಲೇ, ಎಂದರೆ ಚಳಿಗಾಲದಲ್ಲೇ ಅರಳಿಬಿಟ್ಟರೆ, ಆಗ ಅದು ಶೀತಲಮಾರುತನ ವಿರುದ್ಧ ಸೆಣಸಾಡಬೇಕಾಗುತ್ತದೆ. ಹಾಗೆಯೇ ಪ್ರತಿಕೂಲ ಪರಿಸ್ಥಿತಿಗಳಲ್ಲೂ ಎದೆಗುಂದದೆ ಎದ್ದು ನಿಲ್ಲಬೇಕೆಂಬ ಕರೆ ಈ ಕವನದಲ್ಲಿದೆ.

\begin{myquote}
ನೀ ಹಾಸಿರುವ ನೆಲದಿ ಹಿಮ ಹೆಪ್ಪುಗಟ್ಟಿದರು,\\ಕೊರೆವ ಚಳಿಯಂಗಿಯನೆ ನೀ ತೊಟ್ಟರೂ,\\ಸಂತಸವನೀವ ಸಖ ನಿನಗೆ ಸಿಗದಿದ್ದರೂ,\\ಆಗಸದಿ ಕಾರ್ಮೋಡ ಕವಿದಿದ್ದರೂ–
\end{myquote}

\begin{myquote}
ಪ್ರೇಮವೇ ಸೋಲುಂಡು ನೆಲವ ಕಚ್ಚಿದರೂ,\\ನಿನ್ನ ಪರಿಮಳ ಬಯಲ ಪಾಲಾದರೂ,\\ಕೆಡುಕು ಒಳಿತಿನ ತಲೆಯ ಮೆಟ್ಟಿದ್ದರೂ,\\ಪಾಪಭಾರದಿ ಪುಣ್ಯ ನಲುಗಿದ್ದರೂ–
\end{myquote}

\begin{myquote}
ಸವಿಗಂಪ ಪಸರಿಸುವ ನೀಲಕೋಮಲಸುಮವೆ,\\ನಿನ್ನ ತನವನ್ನೆಂದು ನೀ ಬಿಡದಿರು;\\ಕೇಳದಿದ್ದರು ಸರಿಯೇ, ನಿನ್ನ ಪರಿಮಳದೊಲವ\\ಸವಿಧಾರೆಯನು ಸತತ ಹರಿಸುತ್ತಿರು!
\end{myquote}

\selecteng

\chapter[THE CUP]{\enginline{THE CUP}\protect\footnote{\engfoot{C.W, Vol. VI, P.177}}}

\begin{myquote}
\enginline{This is your cup—the cup assigned\\
to you from the beginning.\\
Nay, My child, I know how much\\
of that dark drink is your own brew\\
Of fault and passion, ages long ago,\\
In the deep years of yesterday, I know.}
\end{myquote}

\begin{myquote}
\enginline{This is your road–a painful road and drear.\\
I made the stones that never give you rest.\\
I set your friend in pleasant ways and clear.\\
And he shall come like you, unto My breast.\\
But you, My child, must travel here.}
\end{myquote}

\begin{myquote}
\enginline{This is your task. It has no joy nor grace,\\
But it is not meant for any other hand,\\
And in my universe hath measured place,\\
Take it. I do not bid you understand.\\
I bid you close your eyes to see My face.}
\end{myquote}

\selectkan

\begin{center}
\textbf{ಬಟ್ಟಲು}
\end{center}

ಇಂಗ್ಲಿಷ್ ಮೂಲದಲ್ಲಿ \enginline{'The Cup'} ಎಂಬ ಶೀರ್ಷಿಕೆಯಿಂದಿರುವ ಈ ಕವನವನ್ನು ಸ್ವಾಮಿಜಿಯವರು ಬರೆದ ಕಾಲ ಮತ್ತು ಸಂದರ್ಭ ತಿಳಿದಿಲ್ಲ. ತನ್ನೆಲ್ಲ ಕರ್ಮಗಳ ಜವಾಬ್ದಾರಿಯನ್ನು ವ್ಯಕ್ತಿ ತಾನೇ ಹೊರಬೇಕೆಂಬ ತತ್ತ್ವ ಈ ಕವನದ ಪ್ರಧಾನ ನೆಲೆಯಾಗಿದೆ. ಇಲ್ಲಿ ಬಟ್ಟಲು ಎನ್ನುವುದು ಜೀವನದ ಸಂಕೇತವಾಗುತ್ತದೆ. ನೋವು–ಸಂಕಟಗಳಿಂದ ತುಂಬಿ ಈ ಬಟ್ಟಲು ಕಹಿಯಾದಾಗಲೂ ಅದನ್ನು ದೈವದತ್ತವೆಂದು ಎದೆಗಾರಿಕೆಯಿಂದ ಸ್ವೀಕರಿಸಬೇಕೆಂಬ ಕರೆ ಇಲ್ಲಿದೆ.

ಭಗವಂತನೇ ಮನುಷ್ಯನನ್ನು ಉದ್ದೇಶಿಸಿ ಹೇಳಿರುವ ರೀತಿಯಲ್ಲಿ ಈ ಕವನದ ನಿರೂಪಣೆಯಿದೆ.

\begin{myquote}
ಮಗು, ನಿನ್ನ ಬಟ್ಟಲಿದೊ,\\ಆದಿಯಿಂದಲು ಇದುವೆ ನಿನಗಾಗಿದೆ.\\ಜನ್ಮಜನ್ಮಾಂತರದಿ\\ಭೂತದಾಳಗಳಲ್ಲಿ\\ಎಡವುತಲಿ, ಕೆರಳುತಲಿ,\\ನಿನ್ನ ಕಹಿ ಬಟ್ಟಲನು ನೀನೆ ತುಂಬಿರುವೆ!
\end{myquote}

\begin{myquote}
ಈ ದಾರಿ ನಿನ್ನದಿದೊ–\\ನೋವು – ಸಂಕಟವಿಲ್ಲಿ ಹಾಸಿ ಹಬ್ಬಿಹುದು;\\ಬೆಂಡಾದ ನಿನಗಿಲ್ಲಿ ವಿಶ್ರಾಂತಿ ಸಿಗದವೊಲು\\ಹಾದಿಯುದ್ದಕು ಕಲ್ಲ ನಾನೆ ಹರಡಿರುವೆ;\\ನಿನ್ನ ನೇಹಿಗನ ಪಥ ಸುಗಮಗೊಳಿಸಿಹೆ ನಾನೆ,\\ನಿನ್ನಂತೆಯೇ ಬರುವನವನು ನನ್ನೆದೆಗೆ!
\end{myquote}

\begin{myquote}
ಇಂತಾದರೂ, ಮಗುವೆ, ನೀ ನಡೆವ ದಾರಿಯಿದು:\\ನಿನ್ನ ಹೊಣೆಯಿದು, ನಿನಗೆ ಸೊಗವ ತರದಿರಬಹುದು;\\
– ಆದರಿದು ನಿನಗಾಗಿ ಮೀಸಲಿಹುದು.\\ಈ ನನ್ನ ವಿಶ್ವದಲ್ಲಿ ಎಲ್ಲದಕು ಸ್ಥಾನವನ್ನು\\ಅಳೆದಳೆದು ಕೊಟ್ಟಿಹನು,\\ಸ್ವೀಕರಿಸು ನೀನು.\\ಇದನರಿಯಲಾರೆ ನೀ,\\ನಿನ್ನ ಕಂಗಳ ಮುಚ್ಚಿ ಕಾಣು ನನ್ನನ್ನು!
\end{myquote}

\selecteng

\chapter[A BENEDICTION]{\enginline{A BENEDICTION}\protect\footnote{\engfoot{C.W, Vol. VI, P.178}}}

\begin{myquote}
\enginline{The mother's heat, the hero's will,\\The sweetness of the southern breeze,\\The sacred charm and strength that dwell\\
On Aryan altars, flaming, free;\\
All these be yours, and many more\\
No ancient soul could dream before—\\
Be thou to India’ s future son\\
The mistress, servant, friend in one.}
\end{myquote}

\selectkan

\begin{center}
\textbf{ಹಾರೈಕೆ}
\end{center}

\begin{center}
ಇದು ಸ್ವಾಮಿಜಿ ಸೋದರಿ ನಿವೇದಿತೆಗೆ ಬರೆದ ಕವನ ರೂಪದ 'ಹಾರೈಕೆ'.
\end{center}

\begin{myquote}
ಮಾತೆಯೆದೆ, ಧೀರಾತ್ಮನಲ್ಲಿರುವ ಸಂಕಲ್ಪ,\\ಮೃದು ಮಧುರ ಕುಸುಮಗಳ ಮಧುರತರ ಸ್ಪರ್ಶ,\\ಯಜ್ಞಕುಂಡದಿ ನೆಲಸಿ ಜ್ವಾಲೆಗಳನಾಡಿಸುತ\\ಸೆಳೆವ ಮಾಂತ್ರಿಕ ಶಕ್ತಿ,\\ಮುಂದೆ ನಡೆಸುವ ಬಲವು, ಮಣಿವ ಒಲವು,\\ಮೇರೆ ಮೀರಿದ ಕನಸು,\\ತಾಳ್ಮೆಯಿಂದಿಡುವ ಅಡಿ,\\ಅಣುವಿನಲಿ ಮಹತಿನಲಿ\\ತಾನಾಗಿ ಬೆಳಗುತಿಹ ದಿವ್ಯಾತ್ಮದಲಿ ಶ್ರದ್ಧೆ
\end{myquote}

\begin{myquote}
–ಇನಿತೆಲ್ಲ, ಇನಿತನೂ ಮೀರ್ದ ಒಳಿತುಗಳೆಲ್ಲ\\ನಿನ್ನದಾಗಲಿ ತಾಯ ಕೃಪೆಯಿಂದಲಿ!
\end{myquote}

\selecteng

\chapter[REQUIESCAT IN PACE]{\enginline{REQUIESCAT IN PACE}\protect\footnote{\engfoot{C.W, Vol. IV, P. 389}}}

\begin{myquote}
\enginline{Speed forth, O soul! upon thy star–strewn path;\\
Speed, blissful one! where thought is ever free,\\
Where time and space no longer mist the view,\\
Eternal peace and blessings be with thee!}
\end{myquote}

\begin{myquote}
\enginline{Thy service true, complete thy sacrifice,\\
Thy home the heart of love transcendent find;\\
Remembrance sweet, that kills all space and time,\\
Like altar roses fill thy place behind!}
\end{myquote}

\begin{myquote}
\enginline{Thy bonds are broke, thy quest in bliss is found,\\
And one with That which comes as Death and Life; \\
Thou helpful one! unselfish e’er on earth,\\
Ahead! still help with love this world of strife!}
\end{myquote}

\selectkan

\begin{center}
\textbf{ಶಾಂತಿರಸ್ತು}
\end{center}

೧೮೯೮ರ ಜೂನ್ ತಿಂಗಳಿನಲ್ಲಿ ಆಲ್ಮೋರದಲ್ಲಿದ್ದಾಗ ಬರೆದ \enginline{'Requiescat in pace'} ಎಂಬ ಕವನ ಇದು; ಅದೇ ಜೂನ್ ಎರಡರಂದು ಊಟಿಯಲ್ಲಿ ನಿಧನನಾದ ಅವರ ಆಪ್ತ ಶಿಷ್ಯ ಜೆ.ಜೆ. ಗುಡ್ವಿನ್ನನ ನೆನಪಿಗಾಗಿ ರಚಿಸಿದ್ದು. ಅವರ ಬಹಳಷ್ಟು ಉಪನ್ಯಾಸಗಳನ್ನು ಶೀಘ್ರಲಿಪಿಯಲ್ಲಿ ಬರೆದುಕೊಂಡವನು ಆತ. ಈ ಯುವಶಿಷ್ಯನ ಮರಣದಿಂದ ಸ್ವಾಮಿಜಿಯವರಿಗೆ ತುಂಬ ಆಘಾತವಾಯಿತು. ತಮ್ಮ ಸಾರ್ವಜನಿಕ ಉಪನ್ಯಾಸದ ದಿನಗಳು ಮುಗಿದುವಂದೇ ಅವರು ಭಾವಿಸಿದರು. ಈ ಕವನವನ್ನು ಸ್ವಾಮಿಜಿ ಗುಡ್ವಿನ್ನನ ವಿಧವೆ ತಾಯಿಗೆ ಕಳಿಸಿಕೊಡುತ್ತಾರೆ.

\begin{myquote}
ನುಗ್ಗಿ ನಡೆಯೆಲೆ ಆತ್ಮ! ಮುನ್ನುಗ್ಗಿ ನಡೆ ಜವದಿ\\ಚುಕ್ಕೆ ಚೆಲ್ಲಿದ ನಿನ್ನ ಹಾದಿಯಲ್ಲಿ;\\ನುಗ್ಗಿ ನಡೆ, ಅಮೃತಾತ್ಮ, ಕಾಲ – ದೇಶಗಳೆಲ್ಲ\\ನಿನ್ನ ಕಾಣ್ಕೆಯನೆಂದು ಮಸುಳದಲ್ಲಿ –\\ಶಾಂತಿಶುಭವೆಂದೆಂದು ನಿನಗಾಗಲಿ!
\end{myquote}

\begin{myquote}
ನಿನ್ನ ಸೇವೆಯು ಸತ್ಯ, ನಿನ್ನ ತ್ಯಾಗವು ಪೂರ್ಣ,\\ನಿನ್ನೊಲವಿನೆದೆ ತನ್ನ ಮನೆಯ ಕಾಣುವುದು;\\ಕಾಲದೇಶವ ಕೊಲ್ವ ಚಿರ ಮಧುರ ಸ್ಮೃತಿಗಳವು\\ದೇವಗರ್ಪಿತವಾದ ಕುಸುಮರಾಶಿಗಳಂತೆ\\ನೀ ನಡೆದ ಹಾದಿಯನು ಹಿಂದೆ ತುಂಬುವುದು!\\ಬಂಧಗಳ ಹರಿದೆಸೆದೆ, ಆನಂದವನ್ನು ಸವಿದೆ;\\ಜನನ–ಮರಣದ ರೂಪ ಧರಿಸಿ ಬಹ ಸತ್ಯದಲಿ\\ಎಂದೆಂದು ಒಂದಾಗಿ ಸ್ಥಿರದಿ ನಿಂದೆ;\\ನೆರವ ನೀಡುವ ನೀನು, ನಿಸ್ಸೀಮ ನಿಃಸ್ವಾರ್ಥಿ\\ಬವಣೆ ತುಂಬಿಹ ಜಗಕೆ ನಿನ್ನೊಲವ ಹರಿಸುತ್ತ\\ಎಂದೆಂದು ಮುನ್ನಡೆದು ನೆರವ ನೀಡು!
\end{myquote}

\selecteng

\chapter[MY PLAY IS DONE]{\enginline{MY PLAY IS DONE}\protect\footnote{\engfoot{C.W. Vol. VI. P. 175}}}

\begin{myquote}
\enginline{Ever rising, ever falling with the waves of time,\\
still rolling on I go\\
From fleeting scene to scene ephemeral,\\
with life’s currents’ ebb and flow.\\
Oh! I am sick this unending force;\\
these shows they please no more.\\
This ever running, never reaching,\\
nor e’en a distant glimpse of shore!\\
From life to life I’m waiting at the gates,\\
alas, they open not.\\
Dim are my eyes with vain attempt\\
to catch one ray long sought.\\
On little life’s high, narrow bridge\\
I stand and see below\\
The struggling, crying, laughing throng.\\
For what? No one can know.\\
In front yon gates stand frowning dark,\\
and say: “No farther way,\\
This is the limit; tempt not Fate,\\
bear it as best you may;\\
Go, mix with them and drink this cup\\
and be as and as they.\\
Who dares to know but comes to grief;\\
stop then, and with them stay.”\\
Alas for me, I cannot rest.\\
This floating bubble, earth—\\
Its hollow form, its hollow name,\\
its hollow death and birth—\\
For me is nothing. How I long\\
to get beyond the crust\\
Of name and form! Ah! open the gates;\\
to me they open must.\\
Open the gates of light, O Mother, to me Thy tired son.\\
I long, oh, long to return home!\\
Mother, my play is done.\\
You sent me out in the dark to play,\\
and wore a frightful mask;\\
Then hope departed, terror came,\\
and play became a task.\\
Tossed to and fro, from wave to wave\\
in this seething, surging sea\\
Of passions strong and sorrows deep,\\
grief is, and joy to be,\\
Where life is living death, alas! and death—\\
who knows but’ tis\\
Another start, another round of this old wheel\\
of grief and bliss?\\
Where children dream bright, golden dreams,\\
too soon to find them dust,\\
And aye look back to hope long lost\\
and life a mass of rust!\\
Too late, the knowledge age doth gain;\\
scarce from the wheel we’re gone\\
When fresh, young lives put their strength\\
to the wheel, which thus goes on\\
From day to day and year to year.\\
’Tis but delusions’ s toy,\\
False hope its motor; desire, nave;\\
its spokes are grief and joy.\\
I go adrift and know not whither.\\
Save me from this fire!\\
Rescue me, merciful Mother, from floating with desire!\\
Turn not to me Thy awful face,\\
’tis more than I can bear.\\
Be merciful and kind to me,\\
to chide my faults forbear.\\
Take me. O Mother, to those shores\\
where strifes for ever cease;\\
Beyond all sorrows, beyond tears,\\
beyond e’en earthly bliss;\\
Whose glory neither sun, nor moon,\\
nor stars that twinkle bright,\\
Nor flash of lightning can express.\\
They but reflect its light.\\
Let never more delusive dreams\\
veil of Thy face from me.\\
My play is done, O Mother,\\
break my chains and make me free!}
\end{myquote}

\selectkan

\begin{center}
\textbf{ನನ್ನ ಆಟವು ಮುಗಿಯಿತು}
\end{center}

\enginline{'My Play is Done'} ಎಂಬ ಈ ಇಂಗ್ಲಿಷ್ ಕವನವನ್ನು ಸ್ವಾಮಿಜಿಯವರು ೧೮೯೫ರ ಮಾರ್ಚ್ ೧೬ರಂದು ನ್ಯೂಯಾರ್ಕಿನಲ್ಲಿದ್ದಾಗ ಬರೆದರು. ಅಪ್ಪಟ ಸಂನ್ಯಾಸಿಯೋರ್ವನ ನಿರ್ಲಿಪ್ತ ಮನೋಭಾವವನ್ನು, ಎಲ್ಲ ಜಂಜಡಗಳನ್ನು ಕೊಡಹಿಕೊಳ್ಳಬೇಕೆಂದು ಸ್ವಾಮಿಜಿ ಉತ್ಸುಕರಾಗುತ್ತಿದ್ದ ಬಗೆಯನ್ನು ಈ ಕವನದಲ್ಲಿ ಕಾಣಬಹುದು. ಅವರ 'ಮುಕ್ತಿಗೀತೆ' ಮತ್ತು 'ಸಂನ್ಯಾಸಿಗೀತ' ಎಂಬ ಕವನಗಳೂ ಇದೇ ಸರಿಸುಮಾರಿನಲ್ಲೇ ರಚಿತವಾಗಿದ್ದು, ಇದನ್ನು ಆ ಕವನಗಳೊಂದಿಗೂ ಇಟ್ಟುನೋಡಬಹುದಾಗಿದೆ.

\begin{myquote}
ಮೇಲಕೇಳುತ, ಕೆಳಗೆ ಬೀಳುತ, ಕಾಲದಲೆಗಳ ಕೂಡುತ,\\ಬೀಳುತೇಳುತ, ಜೀವನದ ನದಿಯೊಡನೆ ಸುಮ್ಮನೆ ತೇಲುತ,\\ಉರುಳುತುರುಳುತ ಬರಿದೆ ಸಾಗಿಹೆ ಕ್ಷಣಿಕ ದೃಶ್ಯವ ನೋಡುತ–
\end{myquote}

\begin{myquote}
ಮುಗಿಯದಾಟವ ಕಂಡು ಮನವಿದು ಬರಿದೆ ಬೇಸರಗೊಂಡಿದೆ;\\ತೋರಿಕೆಯು ತಾನಿನ್ನು ರುಚಿಸದು; ಜೀವನದ ಬರಿಯೋಟವು\\ನೆಲೆಯನೆಂದಿಗು ತಲುಪದಿರುವುದು; ದಡದ ಸುಳಿವೇ ತೋರದು!
\end{myquote}

\begin{myquote}
ಜನುಮಜನುಮದಿ ಕಾಯುತಿದ್ದರು ತೆರೆಯದಿರುವುದು ಬಾಗಿಲು;\\ಬೆಳಕ ಕಾಣುವ ವಿಫಲಯತ್ನದಿ ಕಣ್ಗೆ ಕತ್ತಲೆ ಮುಸುಕಲು\\ಬರಿಯ ನಿಷ್ಟುರ ನೋಟ ಬೀರುತ ಮುಚ್ಚಿಹುದು ಹೆಬ್ಬಾಗಿಲು!
\end{myquote}

\begin{myquote}
ಅಲ್ಪ ಜೀವನವಿದರ ಮೇಲಿಹ ಸಣ್ಣ ಸೇತುವೆ ಮೇಗಡೆ\\ನಿಂತು ಸುಮ್ಮನೆ ನೋಡುತಿರುವೆನು ದೃಶ್ಯವೆಲ್ಲವ ಕೆಳಗಡೆ\\ಅಳುವು–ನಗೆಯಲಿ ಹೆಣಗುತಿರುವುದು ನೂಕುನುಗ್ಗುತ ಮಂದಿಯು;\\ಬವಣೆಯರ್ಥವದೇನು ಎಂಬುದನೊಬ್ಬರೂ ತಾವರಿಯರು.
\end{myquote}

\begin{myquote}
ಸಿಡುಕುಕತ್ತಲು ನಿಂತು ಬಾಗಿಲ ಬಳಿಯೊಳೀ ತೆರ ನುಡಿದಿದೆ:\\"ದಾರಿಯಿಲ್ಲವು ಇನ್ನು ಮುಂದಕೆ, ಇಲ್ಲಿಗೇ ಕಡೆ; ನಿಲುಗಡೆ.\\ವಿಧಿಯ ಕೆಣಕಲು ಬೇಡ, ಸಹಿಸಿಕೊ, ನಿನ್ನ ಅಳವಿನ ಮಿತಿಯಲಿ\\ಜನದ ಜಂಗುಳಿ ಸೇರಿ ಬಟ್ಟಲ ಹೀರು ಮರುಳನ ತೆರದಲಿ;\\ತಿಳಿವಿಗೆಳಸುವ ಧೀರನವನಿಗೆ ದುಃಖವದು ಅತಿ ನಿಶ್ಚಯ,\\ಮುಗಿಸು ಪಯಣವನಿಲ್ಲೆ, ಈಗಲೆ, ಕೂಡು ಲೋಕದ ಮಂದಿಯ."
\end{myquote}

\begin{myquote}
ಇನ್ನು ವಿರಮಿಸಲಾರೆ ಜಗದೊಳು–ಪೊಳ್ಳು ಗುಳ್ಳೆಗಳೆಲ್ಲವು!\\ಪೊಳ್ಳು ನಾಮವು, ಪೊಳ್ಳು ರೂಪವು, ಪೊಳ್ಳು ಜನನವು–ಮರಣವು!\\ನಾಮರೂಪವ ಮೀರಿ ನಡೆಯಲು ಜೀವ ಕಾತರಗೊಂಡಿದೆ;\\ಬಾಗಿಲುಗಳೇ, ತೆರೆಯಿರೆನಗೆ – ತೆರೆಯಲೇಬೇಕಿಂದಿಗೆ!
\end{myquote}

\begin{myquote}
ಬೆಳಕು ಬಾಗಿಲ ತಾಯೆ ತೆರೆಯೌ, ನಿನ್ನ ಶಿಶು ನಾ ಬಳಲಿಹೆ;\\ಮನೆಗೆ ಮರಳಲು ನಿನ್ನ ನೋಡಲು ತುಂಬುಹಂಬಲಗೊಂಡಿಹೆ;\\ಎನ್ನ ಆಟವು ಮುಗಿಯಿತಿಂದಿಗೆ –ನಿನ್ನ ಸೇರಲು ಕಾದಿಹೆ!
\end{myquote}

\begin{myquote}
ತಾಯೆ ನನ್ನನು ಪಾರುಮಾಡಿಸು, ಜಗದ ಜಂಜಡ ತಪ್ಪಿಸು;\\ಭ್ರಮೆಯ ತೆರೆಗಳ ತೆರೆಯ ಸರಿಸುತ ನಿನ್ನ ಮೊಗವನು ತೋರಿಸು;\\ಎನ್ನ ಆಟವು ಮುಗಿಯಿತಿಂದಿಗೆ, ನಿನ್ನ ಮಡಿಲಿಗೆ ಸೇರಿಸು!
\end{myquote}

\begin{myquote}
ಕಾಳಕತ್ತಲಿನಲ್ಲಿ ಆಟಕೆ, ತಾಯೆ, ಎನ್ನನು ಕಳುಹಿದೆ.\\ಭಯದ ಮುಖವಾಡವನು ಧರಿಸುತ ನೀನಿದೇನನು ಮಾಡಿದೆ?\\ಆಸೆಯಗಲಿತು, ಅಳಲು ಹೊಮ್ಮಿತು, ಕಾಟವಾಯಿತು ಆಟವು!\\ಕೆರಳಿದಾಸೆಯ, ದುಃಖದಾಳದ, ಕುದಿವ ಸಾಗರವಿದರಲಿ\\ಹಿಂದೆಮುಂದಕೆ ಬರಿದೆ ಸಾಗಿಹೆ ಮೊರೆಯುವಲೆಗಳ ಜೊತೆಯಲಿ;\\ಇಂದು ಸುತ್ತಿದ ಬರಿಯ ಸಂಕಟ, ಸುಖವದೆಂದಿಗು ಮುಂದಕೆ;\\ಉಸಿರನಾಡುವ ಮೃತ್ಯುವಾಗಿದೆ, ಬಾಳು ಸುಮ್ಮನೆ ಹೆಸರಿಗೆ!\\ಸುಖದ ದುಃಖದ ಹಳೆಯ ಚಕ್ರವ ತಳ್ಳುವಾಸೆಯೆ ಸಾವಿಗೆ?\\ಸ್ವರ್ಣಸ್ವಪ್ನವ ಕಾಣುವೆಳೆಯರ ಕನಸ ಗೋಪುರ ಕುಸಿವುದು;\\ಬೆನ್ನ ಹಿಂದಿನ ಬಾಳ ಭರವಸೆಯಳಿದು ಹತ್ತಿದ ತುಕ್ಕಿದು.\\ಕಡೆಗೆ, ಬಾಳಿನ ಸಂಜೆಯಾಗಲು ಅರಿವು ಕಣ್ಣನು ತೆರೆವುದು.
\end{myquote}

\begin{myquote}
ಬಿಸಿದು ನೆತ್ತರ ಹೊಸತು ಜವ್ವನ ಬಲವು ಚಕ್ರವ ಚಲಿಸುತ\\ಮುಂದುಮುಂದಕೆ ಉರುಳಿಸುತ್ತಿರೆ ವರುಷವೆನಿತನೊ ತಳ್ಳುತ–\\ಮರುಳರಾಟಿಗೆಯಿದನು ಭ್ರಾಂತಿಯ ಯಂತ್ರ ತಿರುಗಿಸುತ್ತಿರುವುದು;\\ಸುಖದ–ದುಃಖದ ಅರೆಯು ಕೂಡಿರೆ ಬಯಕೆಗಾಲಿಯು ಹರಿವುದು.
\end{myquote}

\begin{myquote}
ಬರಿದೆ ತೇಲುತಲಿರುವೆ ಬಯಕೆಯ ಅಲೆಗಳೊಂದಿಗೆ ಅರಿಯದೆ;\\ಕರುಣೆ ತೋರುವ ತಾಯೆ, ರಕ್ಷಿಸು, ಭವದ ಬೆಂಕಿಯು ಸುಡುತಿದೆ!\\ಬದ್ಧಭ್ರುಕುಟಿಯ ರುದ್ರಮುಖವನು ತಾಳಲಾರೆನು, ತಾಯಿಯೆ;\\ಸಹನೆಯಿಂದಲೆ ತಿದ್ದು ತಪ್ಪನು, ಕರುಣಿ, ನಿನ್ನನು ಬೇಡುವೆ.
\end{myquote}

\begin{myquote}
ಹೆಣಗು–ಬವಣೆಗಳಿರದ ತೀರಕೆ, ತಾಯೆ, ನನ್ನನು ಸೇರಿಸು–\\ಸುಖವು–ದುಃಖವು–ಹರುಷ–ಕಂಬನಿಯಾಚೆಗೆನ್ನನು ತಲುಪಿಸು.\\ಸೂರ್ಯ–ಚಂದ್ರರು, ಹೊಳೆವ ತಾರಕೆ, ಮಿಂಚುಗಳಿಗದು ಕಾಣದು;\\ದಿವದ ಬೆಳಕಿಗೆ ಹಿಡಿದ ಕನ್ನಡಿಗಳಿವು, ಬೆಳಗುತಲಿರುವುವು!
\end{myquote}

\begin{myquote}
ಕನಸುಗಳ ತೆರೆಮರೆಯೊಳಡಗದೆ ನಿನ್ನ ಮುಖವನ್ನು ತೋರಿಸು.\\ಪಾಶಗಳ ಕಡಿಕಡಿದು ಬಿಡುಗಡೆ ದಾರಿಗೆನ್ನನು ಕೂಡಿಸು;\\ನನ್ನ ಆಟವು ಮುಗಿಯಿತಿಂದಿಗೆ, ತಾಯೆ, ಮಡಿಲಿಗೆ ಸೇರಿಸು!
\end{myquote}

\selecteng

\chapter[PEACE]{\enginline{PEACE}\protect\footnote{\engfoot{C.W. Vol. IV, P. 395}}}

\begin{myquote}
\enginline{Behold, it comes in might,\\
The power that is not power,\\
The light that is in darkness,\\
The shade in the dazzling light.}
\end{myquote}

\begin{myquote}
\enginline{It is joy that never spoke,\\
And grief unfelt, profound,\\
Immortal life unlived,\\
Eternal death unmourned..}
\end{myquote}

\begin{myquote}
\enginline{It is not joy nor sorrow,\\
But that which is between,\\
It is not night nor morrow,\\
But that which joins them in..}
\end{myquote}

\begin{myquote}
\enginline{It is sweet rest in music;\\
And pause in sacred art;\\
The silence between speaking;\\
Between two fits of passion–\\It is the calm of heart..}
\end{myquote}

\begin{myquote}
\enginline{It is beauty never seen,\\
And love that stands alone,\\
Is is song that lives un—sung,\\
And knowledge never known.}
\end{myquote}

\begin{myquote}
\enginline{It is death between two lives.\\
And lull between two storms,\\
The void whence rose creation,\\
And that where it returns..}
\end{myquote}

\begin{myquote}
\enginline{To it the tear–drop goes,\\
To spread the smiling form\\
It is the Goal of Life.\\
And Peace–its only home!.}
\end{myquote}

\selectkan

\begin{center}
\textbf{ಶಾಂತಿ}
\end{center}

ಇದಕ್ಕೆ ಮೂಲ ಇಂಗ್ಲಿಷಿನಲ್ಲಿರುವ \enginline{'Peace'} ಎಂಬ ಕವನ. ಇದನ್ನು ಸ್ವಾಮಿಜಿಯವರು ಬರೆದದ್ದು ೧೮೯೯ರ ಸೆಪ್ಟೆಂಬರ್೨೧ರಂದು; ನ್ಯೂಯಾರ್ಕ್ ರಾಜ್ಯದಲ್ಲಿ ಹಡ್ಸನ್ ನದಿಯ ತೀರದಲ್ಲಿ ಲೆಗೆಟ್ ದಂಪತಿಗಳ 'ರಿಜ್ಲಿ ಮೇನರ್' ಎಂಬ ಹಳ್ಳಿಯ ಮನೆಯಲ್ಲಿ. ಸೋದರಿ ನಿವೇದಿತಾ ತ್ಯಾಗಜೀವನದ ಸಂಕಲ್ಪವನ್ನು ಮಾಡಿದ ಆ ದಿನ ಸ್ವಾಮಿಜಿ ಅವಳಿಗೆ ಈ ಕವನವನ್ನು ಕೊಡುತ್ತಾರೆ.

ವ್ಯಾವಹಾರಿಕ ಜಗತ್ತು ಸಾಪೇಕ್ಷವಾದ ಅತಿಗಳಿಂದ ಕೂಡಿದ್ದು. ಆದರೆ ಆತ್ಮಜ್ಞಾನದಿಂದ ಬರುವ 'ಶಾಂತಿ' ಎಲ್ಲ ಬಗೆಯ ದ್ವಂದ್ವಗಳನ್ನು ಮೀರಿದ್ದು; ನಿರಪೇಕ್ಷವಾದದ್ದು. ವ್ಯಾವಹಾರಿಕ ಸ್ತರದ ಸಾಪೇಕ್ಷವಾದ ಸಂಗತಿಗಳನ್ನು ಬಳಸಿಕೊಂಡೇ ಈ ಕವನದಲ್ಲಿ ಶಾಂತಿಯ ನಿರಪೇಕ್ಷ ಸ್ಥಿತಿಯನ್ನು ವರ್ಣಿಸಲಾಗಿದೆ. ಸ್ವಾಮಿಜಿಯವರ ಕಾವ್ಯಪ್ರತಿಭೆಯ ಅತ್ಯುತ್ತಮ ನಿದರ್ಶನ ಈ ಕವನ.

\begin{myquote}
ಅದೊ ನೋಡು! ಪೌರುಷದಿ\\ಧಾವಿಸುತ್ತಿದೆ ತಾನು–\\ಶಕ್ತಿಯಲ್ಲದ ಶಕ್ತಿ.\\ಕತ್ತಲೊಳಗಿನ ಬೆಳಕು,\\ಕೋರೈಸಿ ಹೊಳೆಹೊಳೆವ\\ಬೆಳಕಿನಾ ನೆರಳು!
\end{myquote}

\begin{myquote}
ನುಡಿಯಲಾಗದ ಹರ್ಷ,\\ಅನುಭವಿಸದಿಹ ದುಃಖ,\\ಬಾಳದಿಹ ಚಿರ ಬದುಕು,\\ಅಳದ ಚಿರಮರಣ!\\ಸುಖ – ದುಃಖ ತಾನಲ್ಲ\\-ಅದರ ನಡುವಿನದು;\\ದಿವಸ – ರಾತ್ರಿಗಳಲ್ಲ\\- ಅದ ಸೇರಿಸುವುದು!
\end{myquote}

\begin{myquote}
ಗಾನದಲ್ಲಿ ವಿಶ್ರಾಂತಿ,\\ಕಲೆಯ ನಡುವಣ ಬಿಡುವು,\\ಶಬ್ದ ಮಧ್ಯದ ಮೌನ,\\ರಾಗದಾವೇಶದಲಿ\\ಎದೆಯ ಶಾಂತತೆಯು!\\ಕಂಡಿರದ ಸೌಂದರ್ಯ,\\ಏಕಾಕಿ ಒಲುಮೆ;\\ಹಾಡದಿಹ ಗಾನವದು,\\ತಿಳಿಯದಿಹ ತಿಳಿವು!
\end{myquote}

\begin{myquote}
ಬದುಕು – ಬದುಕಿನ ನಡುವೆ\\ಮೃತ್ಯುವದು, ಬೀಸಿ ಬಹ\\ಚಂಡಮಾರುತಗಳಲಿ\\ನಿಶ್ಚಲತೆಯು;\\ಸೃಷ್ಟಿಯನ್ನು ಹೊರಚೆಲ್ಲಿ\\ಮರಳಿ ತನ್ನೆಡೆ ಸೆಳೆವ\\ಪರಮ ಶೂನ್ಯತೆಯು!
\end{myquote}

\begin{myquote}
ಅಲ್ಲಿ ಕಂಬನಿ ಸುರಿದು\\ನಗೆಯ ಹರಿಸುವುದು;\\ಶಾಂತಿಯದು – ಜೀವನದ ಚರಮಗುರಿ\\ಮತ್ತದುವೆ\\ಒಂದೆ ಮನೆಯು!
\end{myquote}

\selecteng

\chapter[ONE MORE CIRCLE]{\enginline{ONE MORE CIRCLE}\protect\footnote{\engfoot{C.W, Vol. IX, P 302}}}

\begin{myquote}
\enginline{One circle more the spiral path of life ascends,\\
And Time's restless shuttle–running back and fro\\
Through maze of warp and woof of shining\\
Threads of life–spins out a stronger piece.}
\end{myquote}

\begin{myquote}
\enginline{Hand in hand they stand–and try\\
To fathom depths whence springs eternal love,\\
Each in other's eyes, }
\end{myquote}

\begin{myquote}
\enginline{And find no power holds o'er that age\\
But brings the youth anew to them,\\
And time–the good, the pure, the true.}
\end{myquote}

\selectkan

\begin{center}
\textbf{ಮತ್ತೊಂದು ವೃತ್ತ}
\end{center}

ಸ್ವಾಮಿಜಿಯವರು \enginline{'One More Circle'} ಎಂಬ ಈ ಕವನವನ್ನು ಬರೆದದ್ದು ೧೮೯೯ರಲ್ಲಿ ರಿಜ್ಲಿ ಮೇನರ್‌ನಲ್ಲಿ ಲೆಗೆಟ್ ದಂಪತಿಗಳ ಅತಿಥಿಯಾಗಿದ್ದಾಗ. ವಯಸ್ಸಾಗಿದ್ದರೂ ಹದಿಹರೆಯದ ಚಟುವಟಿಕೆ–ಸಂಭ್ರಮಗಳಿಂದ ಕೂಡಿದ್ದ ಲೆಗೆಟ್ ದಂಪತಿಗಳ ಜೀವನೋತ್ಸಾಹದ ಪ್ರಶಂಸೆ ಈ ಕವನದಲ್ಲಿ ಮೂಡಿದೆ.

\begin{myquote}
ಮತ್ತೆ ಮೇಲೇರುತಿದೆ ಜೀವನದ ವೃತ್ತಪಥ,\\ಕಾಲದವಿರತ ಲಾಳಿಯಾಡುತಿದೆ–ವಸನವನು\\ನೇಯುತಿದೆ, ಜೀವನದ\\ಹೊಳೆಹೊಳೆವ ನೂಲುಗಳ\\ಹಾಸುಗಳ ಹೊಕ್ಕುಗಳ ಜಾಲದಲ್ಲಿ!
\end{myquote}

\begin{myquote}
ಜೊತೆಜೊತೆಗೆ ನಿಂತಿವರು\\ಒರ್ವರೊರ್ವರ ಕಣ್ಣ ನೋಟದಲಿ ಚಿಮ್ಮುತಿಹ\\ನಿತ್ಯದೊಲುಮೆಯ ಬುಗ್ಗೆಯಾಳವನ್ನಳೆಯುವರು!\\ಇವರ ವಯಸಿಗೆ ಅಂಕುಶವನಿಡುವ\\ಶಕ್ತಿಯದು ತಾನೆಲ್ಲಿಯೂ ಇಲ್ಲ;\\ಒಳ್ಳಿತಹ, ಶುದ್ಧವಹ, ದಿಟದ ಕಾಲವು ತಾನು\\ಹೊಸತು ಹರೆಯುವ ತರುವುದಿವರಿಗೆಂದೂ!
\end{myquote}

\selecteng

\chapter[NIRVANASHATKAM, OR SIX STANZAS ON NIRVANA]{\enginline{NIRVANASHATKAM, OR SIX STANZAS ON NIRVANA}\protect\footnote{\engfoot{C.W. Vol. IV, P. 391}}}

\begin{myquote}
\enginline{I am neither the mind, nor the intellect, nor the ego, nor the mind—stuff;\\
I am neither the body, nor the changes of the body;\\
I am neither the senses of hearing, taste, smell or sight,\\
Nor am I the ether, the earth, the fire, the air;\\
I am Existance Absolute, Knowledge Absolute, Bliss Absolute.\\
I am He, I am He. (Shivoham, Shivoham), }
\end{myquote}

\begin{myquote}
\enginline{I am neither the Prana, nor the five vital airs;\\
I am neither the materials of the body, nor the five sheaths;\\
Neither am I the organs of action, nor objects of the senses;\\
I am Existence Absolute, Knowledge Absolute, Bliss Absolute.\\
I am He, I am He. (Shivoham, Shivoham).}
\end{myquote}

\begin{myquote}
\enginline{I have neither aversion nor attachment, neither greed nor delusion;\\
Neither egotism nor envy, neither Dharma nor Moksha;\\
I am neither desire nor objects of desire;\\
I am Existence Absoulte, Knowledge Absoulute, Bliss Absolute\\
I am He, I am He. (Shivoham, Shivoham).}
\end{myquote}

\begin{myquote}
\enginline{I am neither sin nor virtue, neither pleasure nor pain,\\
Nor temple nor worship, nor pilgrimage nor scriptures,\\
Neither the act of enjoying, the enjoyable nor the enjoyer;\\
I am Existence Absolute, Knowledge Absolute, Bliss Absolute\\
I am He, I am He. (Shivoham, Shivoham).}
\end{myquote}

\begin{myquote}
\enginline{I have neither death nor fear of death, nor caste;\\
Nor was I ever born, nor had I parents, friends and relations;\\
I have neither Guru nor disciple;\\
I am Existence Absolute, Knowledge Absolute, Bliss Absolute\\
I am He. I am He. (Shivoham, Shivoham).}
\end{myquote}

\begin{myquote}
\enginline{I am untouched by the senses, I am neither Mukti nor knowable;\\
lam without form, without limit, beyond space, beyond time;\\
I am in everything; I am the basis of the universe; everywhere am I.\\
I am ExiStence Absolute, Knowledge Absolute, Bliss Absolute;\\
I am He, I am He. (Shivoham, Shivoham).}
\end{myquote}

\selectkan

\begin{center}
\textbf{ನಿರ್ವಾಣಷಟ್ಕಂ}
\end{center}

\begin{center}
(ಶಂಕರಾಚಾರ್ಯರ ಸಂಸ್ಕೃತ ಮೂಲವನ್ನು ಸ್ವಾಮೀಜಿ ಇಂಗ್ಲಿಷಿನಲ್ಲಿ ಭಾಷಾಂತರಿಸಿದ್ದಾರೆ.)\\(ಅನುವಾದ: ಕುವೆಂಪು)
\end{center}

ಬುದ್ಧಿ ಮನ ಚಿತ್ರಹಂಕಾರ ನಾನಲ್ಲ, ಕಣ್ಣು ಕಿವಿ ಮೂಗಲ್ಲ ನಾಲಗೆಯುಮಲ್ಲ ನಭವಲ್ಲ, ಧರೆಯಲ್ಲ, ಶಿಖಿವಾಯುವಲ್ಲ; ಸಚ್ಚಿದಾನಂದಾತ್ಮ ಶಿವ ನಾನು, ನಾನು! ಪಂಚವಾಯುಗಳಲ್ಲ, ಪ್ರಾಣ ನಾನಲ್ಲ; ಸಪ್ತಧಾತುಗಳಲ್ಲ, ಕೋಶವೈದಲ್ಲ ಕರ್ಮಾಂಗ, ವಿಷಯೇಂದ್ರಿಯಂಗಳಾನಲ್ಲ; ಸಚ್ಚಿದಾನಂದಾತ್ಮ ಶಿವ ನಾನು, ನಾನು. ದ್ವೇಷರಾಗಗಳಿಲ್ಲ, ಲೋಭಮೆನಗಿಲ್ಲ, ಮೋಹ ಮದ ಮಾತ್ಸರ್ಯ ಭಾವಮೆನಗಿಲ್ಲ ಧರ್ಮಾರ್ಥ ಕಾಮಾದಿ ಮೋಕ್ಷನಾನಲ್ಲ; ಸಚ್ಚಿದಾನಂದಾತ್ಮ ಶಿವ ನಾನು, ನಾನು. ಪಾಪ ಪುಣ್ಯಗಳಲ್ಲ, ಸುಖ ದುಃಖವಲ್ಲ, ಮಂತ್ರ ತೀರ್ಥಗಳಲ್ಲ, ವೇದಮಖವಲ್ಲ ಭೋಜ್ಯ ಭೋಜನವಲ್ಲ, ಭೋಕ್ತ ನಾನಲ್ಲ; ಸಚ್ಚಿದಾನಂದಾತ್ಮ ಶಿವ ನಾನು, ನಾನು. ಜಾತಿ ಭೇದಗಳಿಲ್ಲ, ಮೃತ್ಯುಭಯವಿಲ್ಲ, ಜನ್ಮ ಮೃತ್ಯುಗಳಿಲ್ಲ, ತಂದೆ ತಾಯಿಲ್ಲ ಬಂಧು ಸಖರಿಲ್ಲ, ಗುರುಶಿಷ್ಯರಾರಿಲ್ಲ; ಸಚ್ಚಿದಾನಂದಾತ್ಮ ಶಿವ ನಾನು, ನಾನು.

ನಿರ್ವಿಕಲ್ಪನು ನಾ ನಿರಾಕಾರ ನಾನು, ಸರ್ವೇಂದ್ರಿಯಾತೀತ ಸರ್ವತ್ರ ನಾನು

ಕಾಲ ದೇಶಾತೀತ ವಿಶ್ವವಿಭು ನಾನು; ಸಚ್ಚಿದಾನಂದಾತ್ಮ ಶಿವ ನಾನು, ನಾನು.

\selecteng

\chapter[THE HYMN OF CREATION]{\enginline{THE HYMN OF CREATION}\protect\footnote{\engfoot{C.W. Vol. VI. P. 178}}}

\begin{center}
(Rig–Veda: 10th Mandala, 129)
\end{center}

\begin{myquote}
\enginline{Existence was not then, nor non–existence,\\
The world was not, the sky beyond was neither,\\
What covered the mist? Of whom was that?\\
What was in the depths of darkness thick?}
\end{myquote}

\begin{myquote}
\enginline{Death was not then, nor immortality,\\
The night was neither separate from day,\\
But motionless did That vibrate\\
Alone, with Its own glory one—\\
Beyond That nothing did exist.}
\end{myquote}

\begin{myquote}
\enginline{At first in darkness hidden darkness lay,\\
Undistinguished as one mass of water,\\
Then That which lay in void thus covered\\
A glory did put forth by Tapas!}
\end{myquote}

\begin{myquote}
\enginline{First Desire rose, the primal seed of mind,\\
(The sages have seen all this in their hearts\\
Sifting existence from non–existence.)\\
lts rays above, below and sideways spread.}
\end{myquote}

\begin{myquote}
\enginline{Creative then became the glory,\\
With self–sustaining principle below,\\
And Creative Energy above.}
\end{myquote}

\begin{myquote}
\enginline{Who knew the way? Who there declared\\
Whence this arose? Projection whence?\\
For after this projection came the gods,\\
Who therefore knew indeed, came out this whence?}
\end{myquote}

\begin{myquote}
\enginline{This projection whence arose,\\
Whether held or whether not,\\
He the ruler in the supreme sky, of this\\
He, O Sharman! knows, or knows not He perchance!\\}
\end{myquote}

\selectkan

\begin{center}
\textbf{ನಾಸದೀಯಸೂಕ್ತ, ಸೃಷ್ಟಿ –ಗಾನ}\\(ಮೂಲ ಋಗ್ವೇದದಲ್ಲಿ ಬರುವುದು. ಸ್ವಾಮೀಜಿ ಅದನ್ನು ಇಂಗ್ಲಿಷಿನಲ್ಲಿ ಭಾಷಾಂತರಿಸಿದ್ದಾರೆ)
\end{center}

ಆಗ ಅಸ್ತಿಯೂ ಇರಲಿಲ್ಲ ನಾಸ್ತಿಯೂ ಇರಲಿಲ್ಲ. ಪೃಥ್ವಿಯೂ ಇರಲಿಲ್ಲ, ದೂರದ ಆಕಾಶವೂ ಇರಲಿಲ್ಲ. ಈ ಮಹಾಶೂನ್ಯವನ್ನು ಯಾವುದು ಆವರಿಸಿತ್ತು? ಇದು ಯಾರದು? ಈ ಗಾಢಾಂಧಕಾರದ ಒಳಗೆ ಏನಿತ್ತು?

ಆಗ ಮೃತ್ಯುವೂ ಇರಲಿಲ್ಲ, ಅಮರತ್ವವೂ ಇರಲಿಲ್ಲ. ಆಗ ಹಗಲಿರುಳುಗಳು ಇರಲಿಲ್ಲ. ಆಗ ಅದೊಂದೇ ಸ್ಪಂದಿಸುತ್ತಿತ್ತು. ಅದು ತನ್ನ ಮಹಿಮೆಯಲ್ಲಿ ತಾನು ನೆಲಸಿತ್ತು. ಅದರಾಚೆ ಯಾವುದೂ ಇರಲಿಲ್ಲ.

ಮೊದಲು ಕತ್ತಲೆ ಕತ್ತಲೆಯ ಗರ್ಭದಲ್ಲಿ ಹುದುಗಿತ್ತು. ಅಖಂಡ ಜಲ ರಾಶಿಯಲ್ಲಿ ಬೇರ್ಪಡಿಸಲು ಆಗದಂತೆ ಇತ್ತು. ಯಾವುದು ಮಹಾಶೂನ್ಯದಲ್ಲಿತ್ತೊ ಅದು ತಪಸ್ಸಿನ ಪ್ರಭಾವದಿಂದ ವ್ಯಕ್ತವಾಯಿತು.

ಮನಸ್ಸಿನ ಮೂಲಬೀಜವಾದ ಕಾಮ ಮೊದಲು ಅಂಕುರಿಸಿತು. ಆಸ್ತಿ ನಾಸ್ತಿಗಳನ್ನು ವಿಭಜನೆ ಮಾಡಿದ ಋಷಿಗಳು ಇದನ್ನು ತಮ್ಮ ಹೃದಯದಲ್ಲಿ ಅನುಭವಿಸಿರುವರು. ಇದು ತನ್ನ ಜ್ಯೋತಿಯನ್ನು ಮೇಲೆ ಕೆಳಗೆ ಸುತ್ತಲೂ ಹರಡಿತು. ಕೆಳಗೆ ಮೂಲ ಭೂತಗಳು, ಮೇಲೆ ಪ್ರಾಣ ಇವುಗಳ ಕ್ರಿಯೆ ಪ್ರತಿಕ್ರಿಯೆಗಳಿಂದ ಸೃಷ್ಟಿ ಪ್ರಾರಂಭವಾಯಿತು.

ಇದು ಹೇಗೆ ಆಯಿತು ಎನ್ನುವುದು ಯಾರಿಗೆ ಗೊತ್ತು? ಇದು ಎಲ್ಲಿಂದ ಬಂತು ಎಂಬುದನ್ನು ಯಾರು ಕೆಚ್ಚೆದೆಯಿಂದ ಹೇಳಬಲ್ಲರು? ಎಲ್ಲಿಂದ ಇದು ಪ್ರಕ್ಷೇಪವಾಯಿತು? ಈ ಪ್ರಕ್ಷೇಪವಾದ ಮೇಲೆಯೇ ದೇವರುಗಳೆಲ್ಲರೂ ಬಂದದ್ದು. ಆದಕಾರಣ ಇವರಿಗೆ ಮುಂಚೆ ಏನಿತ್ತು, ಅದು ಹೇಗಿತ್ತು ಎಂಬುದು ಯಾರಿಗೆ ಗೊತ್ತು?

ಈ ಪ್ರಕ್ಷೇಪ ಎಲ್ಲಿಂದ ಬಂತು? ಅವನಿಂದಲೇ ಅಲ್ಲವೇ? ಆ ಪರಂಧಾಮದಲ್ಲಿ ಪರಮೇಶ್ವರನಾಗಿರುವವನೊಬ್ಬನಿಗೇ ಗೊತ್ತು ಅಥವಾ ಅವನಿಗೂ ಅದು ಗೊತ್ತಿಲ್ಲವೋ ಏನೊ!

\selecteng

\chapter[THY LOVE I FEAR]{\enginline{THY LOVE I FEAR}\protect\footnote{\engfoot{C.W, Vol. VI, P.170}}}

\begin{center}
(Rig–Veda: 10th Mandala, 129)
\end{center}

\begin{myquote}
\enginline{Thy knowledge, man! I value not,\\
It is thy lover I fear;\\
It is thy love that shakes My throne,\\
Brings God to human tear.}
\end{myquote}

\begin{myquote}
\enginline{For love, behold the Lord of all,\\
The formless, ever free,\\
Is made to take the human form\\
To play and live with thee.\\
What learning, they of Vrinda's groves,\\
The herdsmen, ever got?\\
What science, girls that milked the kine?\\
They loved, and Me they bought.}
\end{myquote}

\selectkan

\begin{center}
\textbf{ನಡುಗಿಪುದು ನಿನ್ನೊಲುಮೆ}
\end{center}

ಇದು ಮೂಲತಃ ಒಂದು ಬಂಗಾಳಿ ಗೀತೆಯಾಗಿದ್ದು, ಇದನ್ನು ಸ್ವಾಮಿಜಿ \enginline{'The Story of the Boy Gopala'} ಎಂಬ ತಮ್ಮ ಒಂದು ಬರಹದಲ್ಲಿ ಇಂಗ್ಲಿಷಿಗೆ ಅನುವಾದಿಸಿದ್ದಾರೆ. ಅಹೇತುಕವಾದ ಭಕ್ತಿಯ ಮಹಿಮೆಯನ್ನು ಇಲ್ಲಿ ಸ್ವಯಂ ಭಗವಂತನೇ ಕೊಂಡಾಡುತ್ತಿದ್ದಾನೆ. ಶ‍್ರೀರಾಮಕೃಷ್ಣರಿಗೆ ಬಹಳವಾಗಿ ಮೆಚ್ಚುಗೆಯಾಗಿದ್ದ ಗೀತೆ ಇದು.

\begin{myquote}
ನಿನ್ನ ಜ್ಞಾನವ ನಾನು ಲೆಕ್ಕಿಸೆನು, ಎಲೆ ಮನುಜ,\\ನಿನ್ನ ಪ್ರೀತಿಗೆ ನಾನು ಹೆದರುತಿಹೆನು;\\ನಿನ್ನೊಲುಮೆ ನಡುಗಿಪುದು ನನ್ನ ಹಿರಿ ಪೀಠವನು,\\ನಿನ್ನ ಕಂಬನಿ ಸೆಳೆಯುತಿಹುದೆನ್ನನು!
\end{myquote}

\begin{myquote}
ನಾಮರೂಪವ ಮೀರ್ದ ನಿತ್ಯಮುಕ್ತನು ನಾನು\\ಒಲುಮೆಸೆರೆಯಲ್ಲೀಗ ಸಿಲುಕಿಕೊಂಡು\\ಮನುಜರೂಪವ ಧರಿಸಿ ಇಳೆಯೊಳಾಡುತಲಿಹೆನು,\\ನಿನ್ನ ಭಕ್ತಿಗೆ ನನ್ನ ತೆತ್ತುಕೊಂಡು!
\end{myquote}

\begin{myquote}
ಗೋಕುಲದ ಗೋವಳರ ವಿದ್ಯೆಯಿನ್ನಾವುದದು?\\ಪಾಲ್ಗರೆವ ಗೋಪಿಯರ ಶಾಸ್ತ್ರವೇನು?\\ತಮ್ಮೊಲುಮೆ ಬಲದಿಂದ ಕೊಂಡರೆನ್ನನು ಅವರು,\\ಮಾರಿಕೊಂಡೆನು ನಿಜಕು ನಾನೆ ನನ್ನನ್ನು!
\end{myquote}

\selectkan

\insertlangkanintotoc

\chapter[ಸಂಸ್ಕೃತ ಸ್ತೋತ್ರಗಳು]{ಸಂಸ್ಕೃತ ಸ್ತೋತ್ರಗಳು\protect\footnote{\engfoot{C.W, Vol. VIII, P.172}} (೧)}

\begin{center}
\textbf{ಶ‍್ರೀರಾಮಕೃಷ್ಣ ಸ್ತೋತ್ರಾಣಿ}
\end{center}

\begin{myquote}
ಓಂ ಹ್ರೀಂ ಋತಂ ತ್ವಮಚಲೋ ಗುಣಜಿದ್ಗುಣೇಡ್ಯೋ\\ನಕ್ತಂ ದಿವಂ ಸಕರುಣಂ ತವಪಾದಪದ್ಮಂ~॥\\ಮೋಹಂಕಷಂ ಬಹುಕೃತಂ ನ ಭಜೇ ಯತೋಽಹಂ\\ತಸ್ಮಾತ್ ತ್ವಮೇವ ಶರಣಂ ಮಮ ದೀನಬಂಧೋ
\end{myquote}

\versenum{॥ ೧~॥}

\begin{myquote}
ಭಕ್ತಿರ್ಭಗಶ್ಚ ಭಜನಂ ಭವಭೇದಕಾರಿ\\ಗಚ್ಛಂತ್ಯಲಂ ಸುವಿಪುಲಂ ಗಮನಾಯ ತತ್ತ್ವಮ್~।\\ವಕ್ತ್ರೋದ್ಧೃತೋಪಿ ತು ಹೃದಯೇ ನ ಚ ಭಾತಿ ಕಿಂಚಿತ್\\ತಸ್ಮಾತ್ ತ್ವಮೇವ ಶರಣಂ ಮಮ ದೀನಬಂಧೋ
\end{myquote}

\versenum{॥ ೨~॥}

\begin{myquote}
ತೇಜಸ್ತರಂತಿ ತರಸಾ ತ್ವಯಿ ತೃಪ್ತತೃಷ್ಣಾಃ\\ರಾಗೇಕೃತೇ ಋತಪಥೇ ತ್ವಯಿ ರಾಮಕೃಷ್ಣೇ!\\ಮರ್ತ್ಯಾಮೃತಂ ತವಪದಂ ಮರಣೋರ್ಮಿನಾಶಂ\\ತಸ್ಮಾತ್ ತ್ವಮೇವ ಶರಣಂ ಮಮ ದೀನಬಂಧೋ
\end{myquote}

\versenum{॥ ೩~॥}

\begin{myquote}
ಕೃತ್ಯಂ ಕರೋತಿ ಕಲುಷಂ ಕುಹಕಾಂತಕಾರಿ\\ಷ್ಣಾಂತಂ ಶಿವಂ ಸುವಿಮಲಂ ತವನಾಮ ನಾಥ~।\\ಯಸ್ಮಾದಹಂ ತ್ವಶರಣೋ ಜಗದೇಕಗಮ್ಯ\\ತಸ್ಮಾತ್ ತ್ವಮೇವ ಶರಣಂ ಮಮ ದೀನಬಂಧೋ
\end{myquote}

\versenum{॥ ೪~॥}

\delimiter

\begin{myquote}
ಓಂ ಸ್ಥಾಪಕಾಯ ಚ ಧರ್ಮಸ್ಯ ಸರ್ವಧರ್ಮಸ್ವರೂಪಿಣೇ\\ಅವತಾರವರಿಷ್ಠಾಯ ರಾಮಕೃಷ್ಣಾಯ ತೇ ನಮಃ
\end{myquote}

\begin{center}
\textbf{ಶ‍್ರೀರಾಮಕೃಷ್ಣ ಸ್ತೋತ್ರ (೧)}
\end{center}

ಈ ಸ್ತೋತ್ರವನ್ನೂ ಮುಂದಿನ ಸ್ತೋತ್ರವನ್ನೂ ಸ್ವಾಮಿಜಿಯವರು ರಚಿಸಿದ್ದು ೧೮೯೮ರ ನವೆಂಬರ್ ತಿಂಗಳಿನಲ್ಲಿ; ಬೇಲೂರಿನ ಬಾಡಿಗೆ ಕಟ್ಟಡವೊಂದರಲ್ಲಿ ಮಠ ಮಾಡಿದ್ದಾಗ. ಈ ಸ್ತೋತ್ರಗಳ ರಚನೆಯ ಸಂದರ್ಭವನ್ನು ಸ್ವಾಮಿಜಿಯವರ ಗೃಹೀಶಿಷ್ಯನಾದ ಶರತ್‌ಚಂದ್ರ ಚಕ್ರವರ್ತಿ ತನ್ನ ದಿನಚರಿಯಲ್ಲಿ ದಾಖಲಿಸಿದ್ದಾನೆ. "ಆ ದಿನ ಸ್ವಾಮಿಜಿಯವರ ನಾಲಗೆಯ ಮೇಲೆ ಸಾಕ್ಷಾತ್ ಸರಸ್ವತಿಯೇ ನೆಲಸಿದಂತಿತ್ತು" ಎಂದಿದ್ದಾನೆ ಶರತ್‌ಚಂದ್ರ. ಈ ಸ್ತೋತ್ರಗಳನ್ನು ಶಿಷ್ಯನ ಕೈಗೆ ಕೊಡುತ್ತ ಸ್ವಾಮಿಜಿ, "ಇವುಗಳಲ್ಲಿ ಛಂದಸ್ಸಿನ ದೋಷಗಳೇನಾದರೂ ಇವೆಯೋ ನೋಡು" ಎಂದರಲ್ಲದೆ ಮತ್ತೆ ಹೇಳುತ್ತಾರೆ: “ನೋಡು, ನಾನು ಆಲೋಚನೆಯಲ್ಲೇ ಮುಳುಗಿ ಬರೆಯುತ್ತಿರುವಾಗ ಕೆಲವೊಮ್ಮ ವ್ಯಾಕರಣ ದೋಷಗಳು ತಲೆದೋರುವುದುಂಟು. ಆದ್ದರಿಂದಲೇ ಅವುಗಳನ್ನೊಂದಿಷ್ಟು ಪರಿಶೀಲಿಸು ಎಂದು ನಿನಗೆ ಹೇಳುವುದು.” ಇದಕ್ಕೆ ಶಿಷ್ಯನೆನ್ನುತ್ತಾನೆ: “ಮಹಾಶಯರೇ, ಇವು ದೋಷಗಳಲ್ಲ, ಮಹಾಕವಿಪ್ರಯೋಗ!”

೧. ಓಂ ಹ್ರೀಂ, ನೀನು ಸತ್ಯಸ್ವರೂಪನು, ಅಚಲನು, ತ್ರಿಗುಣಗಳನ್ನು ಜಯಿಸಿದವನು ಮತ್ತು ಕಲ್ಯಾಣಗುಣಗಳ ಮೂಲಕ ಪ್ರಶಂಸನೀಯನು, ಮೋಹ ನಾಶಕವೂ ಪೂಜನೀಯವೂ ಆದ ನಿನ್ನ ಅಡಿದಾವರೆಗಳನ್ನು ನಾನು ಹಗಲೂ ರಾತ್ರಿಯೂ ವ್ಯಾಕುಲದಿಂದ ಭಜಿಸಲಿಲ್ಲ: ಆದುದರಿಂದ ಹೇ ದೀನಬಂಧು ನೀನೇ ನನಗೆ ಶರಣು.

೨. ಸಂಸಾರನಾಶಕವಾದ ಭಕ್ತಿ ಜ್ಞಾನೈಶ್ವರ್ಯಾದಿಗಳು ಮತ್ತು ಭಜನೆ ಇವು ಮಹಾತತ್ತ್ವವನ್ನು ಹೊಂದಲು ಸಾಕು. ಆದರೆ ಇದು ನನ್ನ ಬರಿಯ ಮಾತಾಗಿ ಹೃದಯದಲ್ಲಿ ಸ್ವಲ್ಪವೂ ಹೊಳೆಯದೆ ಇರುವುದರಿಂದ ಹೇ ದೀನ ಬಂಧು, ನೀನೇ ನನಗೆ ಶರಣು.

೩. ಹೇ ರಾಮಕೃಷ್ಣ, ಸತ್ಯವೇ ಪಥನಾದ ನಿನ್ನಲ್ಲಿ ಅನುರಾಗವು ಉಂಟಾದರೆ ಮನುಷ್ಯರು ನಿನ್ನನ್ನು ಹೊಂದಿ, ಪೂರ್ಣಕಾಮರಾಗಿ ರಜೋಗುಣವನ್ನು ದಾಟುವರು. ಮರಣವೆಂಬ ಅಲೆಗಳನ್ನು ನಾಶಮಾಡುವ ನಿನ್ನ ಚರಣಗಳು ಮರ್ತ್ಯಲೋಕದಲ್ಲಿ ಅಮೃತವಾಗಿರುವುವು. ಆದುದರಿಂದ ಹೇ ದೀನಬಂಧು ನೀನೇ ನನಗೆ ಶರಣು.

೪, ಹೇ ನಾಥ, ಮಾಯೆಯನ್ನು ನಾಶಮಾಡುವುದೂ ಮಂಗಳವೂ ವಿಮಲವೂ 'ಷ್ಣ' ಎಂಬ ಅಕ್ಷರದಿಂದ ಅಂತ್ಯವಾಗುವುದೂ ಆದ ನಿನ್ನ ಹೆಸರು ಪಾಪವನ್ನು ಕೂಡ ಪುಣ್ಯವನ್ನಾಗಿ ಮಾಡುವುದು. ಜಗತ್ತಿಗೆ ಏಕಮಾತ್ರ ಗುರಿಯಾದ ದೀನಬಂಧು, ನನಗೆ ಯಾವ ಆಶ್ರಯವೂ ಇಲ್ಲದಿರುವುದರಿಂದ ನೀನೇ ನನಗೆ ಶರಣು.

ಧರ್ಮಸಂಸ್ಥಾಪಕನೂ ಸರ್ವಧರ್ಮಸ್ವರೂಪಿಯೂ ಅವತಾರಶ್ರೇಷ್ಠನೂ ಆದ ಶ‍್ರೀರಾಮಕೃಷ್ಣನಿಗೆ ನಮಸ್ಕಾರಗಳು.

\begin{center}
\textbf{ಶ‍್ರೀರಾಮಕೃಷ್ಣ ಸ್ತೋತ್ರಂ}\footnote{\engfoot{C.W, Vol. VIII, P.173}} (೨)
\end{center}

\begin{myquote}
ಆಚಂಡಾಲಾಪ್ರತಿಹತರಯೋ ಯಸ್ಯ ಪ್ರೇಮಪ್ರವಾಹಃ\\ಲೋಕಾತೀತೋಪ್ಯಹಹ ನ ಜಹೌ ಲೋಕಕಲ್ಯಾಣಮಾರ್ಗಮ್~॥\\
ತ್ರೈಲೋಕ್ಯೇಽಪ್ಯಪ್ರತಿಮಮಹಿಮಾ ಜಾನಕೀಪ್ರಾಣಬಂಧೋ\\ಭಕ್ತ್ಯಾ ಜ್ಞಾನಂ ವೃತವರವಪುಃ ಸೀತಯಾ ಯೋ ಹಿ ರಾಮಃ
\end{myquote}

\versenum{॥ ೧~॥}

\begin{myquote}
ಸ್ತಬ್ಧೀಕೃತ್ಯ ಪ್ರಲಯಕಲಿತಂ ವಾಹವೋತ್ಥಂ ಮಹಾನ್ತಂ\\ಹಿತ್ವಾ ರಾತ್ರಿಂ ಪ್ರಕೃತಿಸಹಜಾಮನ್ಧತಾಮಿಸ್ರಮಿಶ್ರಾಮ್~।\\
ಗೀತಂ ಶಾನ್ತಂ ಮಧುರಮಪಿ ಯಃ ಸಿಂಹನಾದಂ ಜಗರ್ಜ\\ಸೋಽಯಂ ಜಾತಃ ಪ್ರಥಿತಪುರುಷೋ ರಾಮಕೃಷ್ಣದಾನೀಮ್
\end{myquote}

\versenum{॥ ೨~॥}

\begin{myquote}
ಶಕ್ತಿಸಮುದ್ರಸಮುತ್ಥತರಂಗಂ\\ದರ್ಶಿತಪ್ರೇಮವಿಜೃಂಭಿತರಂಗಂ\\ಸಂಶಯರಾಕ್ಷಸನಾಶಮಹಾಸ್ತ್ರಂ\\ಯಾಮಿ ಗುರುಂ ಶರಣಂ ಭವವೈದ್ಯಂ
\end{myquote}

\versenum{॥ ೩~॥}

\begin{myquote}
ಅದ್ವಯತತ್ತ್ವಮಾಹಿತಚಿತ್ತಂ\\ಪ್ರೋಜ್ವಲಭಕ್ತಿಪಟಾವೃತವೃತ್ತಂ\\ಕರ್ಮಕಲೇವರಮದ್ಭುತ ಚೇಷ್ಟಂ\\ಯಾಮಿ ಗುರುಂ ಶರಣಂ ಭವವೈದ್ಯಂ
\end{myquote}

\versenum{॥ ೪~॥}

\begin{myquote}
ನರದೇವ ದೇವ\\ಜಯ ಜಯ ನರದೇವ~॥
\end{myquote}

\begin{center}
\textbf{ಶ‍್ರೀರಾಮಕೃಷ್ಣ ಸ್ತೋತ್ರ}
\end{center}

\begin{myquote}
ಯಾರ ಪ್ರೇಮದ ಪರಮಪೂರವು\\ನೀಚರೆದೆಗೂ ಹರಿಯಿತೊ,\\ಯಾವ ಲೋಕಾತೀತಮಹಿಮನ\\ಕರುಣೆ ಲೋಕಕೆ ದುಡಿಯಿತೊ,\\ಯಾವನಪ್ರತಿ–ಮಹಿಮನೊ ಮೇಣ್\\ಮಾತೆ ಸೀತೆಯ ನಾಥನೊ,\\ಯಾರು ಸೀತೆಯ ಭಕುತಿಯಿಂದಲಿ\\ಜ್ಞಾನದೇಹದಿ ವ್ಯಾಪ್ತನೊ;
\end{myquote}

\begin{myquote}
ಯಾರು ಮಧುತರ ಶಾಂತಗೀತೆಯ\\ಯುದ್ಧರಂಗದಿ ಮೊಳಗುತ\\ಪ್ರಳಯಶಬ್ದವ ಸ್ತಬ್ಧಗೊಳಿಸುತ\\ಸಿಂಹಗರ್ಜನೆ ಮಾಡುತ\\ಮೋಹತಿಮಿರವನಿಲ್ಲ ಗೈಯುತ\\ಕೃಷ್ಣರೂಪದಿ ನಿಂದನೊ\\ಅವನೆ ಇಂದಿಗೆ ರಾಮಕೃಷ್ಣನ\\ಹೆಸರೊಳೆಸೆಯುತಲಿರುವನು!
\end{myquote}

\begin{myquote}
ಶಕ್ತಿ ಸಮುದ್ರದೊಳೆದ್ದ ತರಂಗವು\\ದಿವ್ಯ ಪ್ರೇಮಮಯ ಲೀಲಾರಂಗವು\\ಸಂಶಯ ರಾಕ್ಷಸ ನಾಶಮಹಾಸ್ತ್ರವು\\ಆ ಭವವೈದ್ಯ ಶ‍್ರೀಗುರುವಿಗೆ ಶರಣು!
\end{myquote}

\begin{myquote}
ಅದ್ವಯ ತತ್ತ್ವದಿ ನೆಲಸಿದ ಚಿತ್ತನು\\ಉಜ್ಜ್ವಲ ಭಕ್ತಿಯಿಂದಾವೃತ ಗಾತ್ರನು\\ಅದ್ಭುತ ಕರ್ಮನಿರತ ಪರಿಪೂರ್ಣನು\\ಆ ಭವವೈದ್ಯ ಶ‍್ರೀಗುರುವಿಗೆ ಶರಣು!
\end{myquote}

\begin{center}
\textbf{ಶ‍್ರೀರಾಮಕೃಷ್ಣ ಸ್ತೋತ್ರಂ (೩)}\footnote{\engfoot{C.W Vol. IX, P.304}}
\end{center}

\begin{myquote}
ಸಾಮಾಖ್ಯಾದ್ಯೈರ್ಗೀತಿ ಸುಮಧುರೈರ್ಮೇಘ ಗಂಭೀರ ಘೋಷೈ?\\ಯಜ್ಞಧ್ವಾನಧ್ವನಿತಗಗನೈರ್ಬ್ರಾಹ್ಮಣೈರ್ಜ್ಞಾತ ವೇದೈಃ~।\\ವೇದಾನ್ತಾಖ್ಯೈಃ ಸುವಿಹಿತ ಮಖೋದ್ಭಿನ್ನಮೋಹಾಂಧಕಾರೈಃ\\ಸ್ತುತೋ ಗೀತೋ ಯ ಇಹ ಸತತಂ ತಂ ಭಜೇ ರಾಮಕೃಷ್ಣಮ್~॥
\end{myquote}

\begin{center}
\textbf{ಶ‍್ರೀರಾಮಕೃಷ್ಣ ಸ್ತೋತ್ರ (೩)}
\end{center}

ವೇದತತ್ತ್ವಜ್ಞರೂ, ಬ್ರಾಹ್ಮಣರೂ ಯಜ್ಞಸ್ಥಳದಲ್ಲಿ ಮಂತ್ರೋಚ್ಛಾರಣೆಯಿಂದ ಗಗನವನ್ನು ಪ್ರತಿಧ್ವನಿಗೊಳಿಸಿದರು; ವಿಧಿಪೂರ್ವಕ ಯಜ್ಞವನ್ನು ಅರ್ಜಿಸಿ ಅವರ ಹೃದಯ ಪರಿಶುದ್ಧವಾಗಿ, ವೇದಾಂತವಾಕ್ಯಗಳ ಮೂಲಕ ಭ್ರಮೆ ಮತ್ತು ಅಂಧಕಾರವು ದೂರವಾಗಿದ್ದಿತು; ಅವರು ಆ ಮೇಘಗಳಂತೆ ಗಂಭೀರ ಸುಮಧುರ ರಾಗದಿಂದ ಸಾಮಗಾನವೇ ಮುಂತಾದುವುಗಳಿಂದ ಯಾರ ಸ್ತುತಿಯನ್ನು ಮಾಡಿರುವರೋ, ಯಾರ ಮಹಿಮೆಯನ್ನು ಹಾಡಿರುವರೋ – ಅಂತಹ ಶ‍್ರೀರಾಮಕೃಷ್ಣರ ಭಜನೆಯನ್ನು ನಾನು ಸರ್ವದಾ ಮಾಡುತ್ತೇನೆ.

\begin{center}
\textbf{ಭಯವೇತಕೆ?}\footnote{\engfoot{C.W, Vol. VI, P. 275}}
\end{center}

\begin{myquote}
ಕುರ್ಮಸ್ತಾರಕಚರ್ವಣಂ ತ್ರಿಭುವನಮುತ್ಪಾಟಯಾಮೋ ಬಲಾತ್\\ಕಿಂ ಭೋನ ವಿಜಾನಾಸ್ಯಸ್ಮಯಾನ್, ರಾಮಕೃಷ್ಣದಾಸಾ ವಯಮ್~॥\\ಕ್ಷೀಣಾಃ ಸ್ಮ ದೀನಾಃ ಸಕರುಣಾ ಜಲ್ಪಂತಿ ಮೂಢಾ ಜನಾಃ\\ನಾಸ್ತಿಕ್ಯಂತ್ವಿದಂತು ಅಹಹ ದೇಹಾತ್ಮ ವಾದಾತುರಾಃ~।
\end{myquote}

\begin{myquote}
ಪ್ರಾಪ್ತಾ ಸ್ಮ ವೀರಾ ಗತಭಯಾ ಅಭಯಂ ಪ್ರತಿಷ್ಠಾ ಯದಾ\\ಆಸ್ತಿಕ್ಯಂತ್ವಿದಂತು ಚಿನುಮಃ ರಾಮಕೃಷ್ಣದಾಸಾ ವಯಮ್~॥
\end{myquote}

\begin{myquote}
ಪೀತ್ವಾ ಪೀತ್ವಾ ಪರಮಮಮೃತಂ ವೀತಸಂಸಾರರಾಗಾಃ\\ಹಿತ್ವಾ ಹಿತ್ವಾ ಸಕಲಕಲಹಪ್ರಾಪಿಣೀಂ ಸ್ವಾರ್ಥಸಿದ್ಧಮ್~।\\ಧ್ಯಾತ್ವಾ ಧ್ಯಾತ್ವಾ ಗುರುವರಪದಂ ಸರ್ವಕಲ್ಯಾಣರೂಪಂ\\ನತ್ವಾ ನತ್ವಾ ಸಕಲಭುವನಂ ಪಾತುಮಾಮಂತ್ರಯಾಮಃ~॥
\end{myquote}

\begin{myquote}
ಪ್ರಾಪ್ತಂ ಯದ್ವೈ ತ್ವನಾದಿನಿಧನಂ ವೇದೋದಧಿಂ ಮಢಿತ್ವಾ\\ದತ್ತಂ ಯಸ್ಯ ಪ್ರಕರಣೇ ಹರಿಹರಬ್ರಹ್ಮಾದಿ ದೇವೈರ್ಬಲಮ್~।\\
ಪೂರ್ಣೇ ಯತ್ತು ಪ್ರಾಣಸಾರೈರ್ಭೌಮನಾರಾಯಣಾನಾಂ\\ರಾಮಕೃಷ್ಣಸ್ತನುಂ ಧತ್ತೇ ತತ್ಪೂರ್ಣಪಾತ್ರಮಿದಂ ಭೋಃ~॥
\end{myquote}

\begin{center}
\textbf{ಭಯವೇತಕೆ?}
\end{center}

೧೮೯೪ರ ಸೆಪ್ಟೆಂಬರ್ ೨೫ರಂದು ಸ್ವಾಮಿಜಿಯವರು ಬಾರಾನಗರ ಮಠದ ತಮ್ಮ ಸೋದರಸಂನ್ಯಾಸಿಗಳಿಗೆ ಬರೆದ ಪತ್ರದಲ್ಲಿ ಈ ಸಂಸ್ಕೃತ ಕವನವನ್ನು ಬರೆದಿದ್ದಾರೆ. ಶ‍್ರೀರಾಮಕೃಷ್ಣರಲ್ಲಿನ ಅವಿಚಲಶ್ರದ್ಧೆ ಎಂತಹ ನಿರ್ಭಯತೆಯನ್ನು ತಂದುಕೊಡಲ್ಲುದೆಂಬುದರ ಶಕ್ತಿಯುತವಾದ ನಿರೂಪಣೆ ಇದರಲ್ಲಿದೆ.

\begin{myquote}
ತಾರೆಗಳ ಕುಟ್ಟಿ ಪುಡಿ ಮಾಡುವೆವು ನಾವು,\\ಲೋಕವನೆ ತಲೆಕೆಳಗು ಮಾಡುವೆವು ನಾವು;\\ನಾವು ಯಾರೆಂಬುದನು ಬಲ್ಲಿರಾ ನೀವು?\\ಶ‍್ರೀರಾಮಕೃಷ್ಣರ ಕಿಂಕರರೊ ನಾವು!\\ದೇಹಾತ್ಮಭಾವದಲಿ ದೀನತೆಯ ತಳೆದು\\ಗೋಳಿಡುವುದದೆ ನಾಸ್ತಿಕತೆಯೆಂದು ತಿಳಿದು\\ಅಭಯದಲಿ ನೆಲೆಗೊಂಡ ವೀರರಾಗಿಹೆವು;\\ತತ್ತ್ವವಿದು ಶ‍್ರೀರಾಮಕೃಷ್ಣದಾಸರದು!
\end{myquote}

\begin{myquote}
ಮೋಹವನು ತೊರೆದು, ಪರಮಾಮೃತವ ಸವಿದು,\\ಎಲ್ಲ ಕಲಹದ ಮೂಲ ಸ್ವಾರ್ಥವನ್ನುಳಿದು,\\ಸಕಲ ಸಂಪದವೀವ ಗುರುಪದಕೆ ಮಣಿದು,\\ಜಗವೆಲ್ಲವನು ಶುಭಕೆ ಕರೆಯುತಿಹೆವಿಂದು!
\end{myquote}

\begin{myquote}
ವೇದವಾರಿಧಿಮಥನದಿಂ ಬಂದ ಸುಧೆಯು\\ಹರಿಹರಬ್ರಹ್ಮಾದಿ ಸುರರ ಬಲನಿಧಿಯು;\\ತುಂಬಿ ತುಳುಕಿಹುದದುವೆ ಪೂರ್ಣರೂಪದಲ್ಲಿ\\ಅವತಾರಸಾರ ಶ‍್ರೀರಾಮಕೃಷ್ಣರಲಿ!
\end{myquote}

\begin{center}
\textbf{ಶಿವಸ್ತೋತ್ರಂ}\footnote{\engfoot{C.W, Vol. IV, P. 501}}
\end{center}

\begin{myquote}
ನಿಖಿಲಭುವನ ಜನ್ಮಸ್ಥೇಮಭಂಗ ಪ್ರರೋಹಾಃ\\ಅಕಲಿತಮಹಿಮಾನಃ ಕಲ್ಪಿತಾ ಯತ್ರ ತಸ್ಮಿನ್~।\\ಸುವಿಮಲ ಗಗನಾಭೇ ಈಶಸಂಸ್ಥೇಪ್ಯನೀಶೇ\\ಮಮ ಭವತು ಭವೇಸ್ಮಿನ್ ಭಾಸುರೋ ಭಾವಬಂಧಃ
\end{myquote}

\versenum{॥ ೧~॥}

\begin{myquote}
ನಿಹತನಿಖಿಲಮೋಹೇಽಧೀಶತಾ ಯತ್ರ ರೂಢಾ\\ಪ್ರಕಟಿತಪರಪ್ರೇಮ್ಣಾ ಯೋ ಮಹಾದೇವಸಂಜ್ಞಃ~।\\ಅಶಿಥಿಲಪರಿರಂಭಃ ಪ್ರೇಮರೂಪಸ್ಯ\\ಯಸ್ಯ ಹೃದಿ ಪ್ರಣಯತಿ ವಿಶ್ವಂ ವ್ಯಾಜಮಾತ್ರಂ ವಿಭುತ್ವಮ್
\end{myquote}

\versenum{॥ ೨~॥}

\begin{myquote}
ವಹತಿ ವಿಪುಲವಾತಃ ಪೂರ್ವಸಂಸ್ಕಾರ ರೂಪಃ\\ಪ್ರಮಥತಿ ಬಲವೃಂದಂ ಪೂರ್ಣಿತೇವೋರ್ಮಿಮಾಲಾ~।\\ಪ್ರಚಲತಿ ಖಲು ಯುನ್ಮಂ ಯುಷ್ಮದಸ್ಮತ್ ಪ್ರತೀತಂ\\ಅತಿವಿಕಲಿತರೂಪಂ ನೌಮಿ ಚಿತ್ತಂ ಶಿವಸ್ಥಮ್
\end{myquote}

\versenum{॥ ೩~॥}

\begin{myquote}
ಜನಕಜನಿತಭಾವೋ ವೃತ್ತಯಃ ಸಂಸ್ಕೃತಾಶ್ಚ\\ಅಗಣನಬಹುರೂಪೋ ಯತ್ರ ಏಕೋ ಯಥಾರ್ಥಃ~।\\ಶಮಿತ ವಿಕೃತಿವಾತೇ ಯತ್ರ ನಾಂತರ್ಬಹಿಶ್ಚ\\ತಮಹಹ ಹರಿಮೀಡೇ ಚಿತ್ತವೃತ್ತೇರ್ನಿರೋಧಮ್
\end{myquote}

\versenum{॥ ೪~॥}

\begin{myquote}
ಗಲಿತತಿಮಿರಮಾಲಃ ಶುಭ್ರತೇಜಃಪ್ರಕಾಶಃ\\ಧವಲಕಮಲಶೋಭಃ ಜ್ಞಾನಪುಂಜಾಟ್ಟಹಾಸಃ~।\\ಯಮಿಜನ ಹೃದಿಗಮ್ಯಃ ನಿಷ್ಕಲಂ ಧ್ಯಾಯಮಾನಃ\\ಪ್ರಣತಮವತು ಮಾಂ ಸಃ ಮಾನಸೋ ರಾಜಹಂಸಃ
\end{myquote}

\versenum{॥ ೫~॥}

\begin{myquote}
ದುರಿತದಲನದಕ್ಷಂ ದಕ್ಷಜಾದತ್ತದೋಷಂ~।\\ಕಲಿತಕಲಿಕಲಂಕಂ ಕಮ್ರಕಲ್ಹಾರ ಕಾಂತಮ್~।\\ಪರಹಿತಕರಣಾಯ ಪ್ರಾಣವಿಚ್ಛೇದಸೂತ್ಕಂ\\ನತನಯನನಿಯುಕ್ತಂ ನೀಲಕಂಠಂ ನಮಾಮಃ
\end{myquote}

\versenum{॥ ೬~॥}

\begin{center}
\textbf{ಶಿವಸ್ತೋತ್ರ}
\end{center}

೧. ಅವನ (ಶಿವನ) ಮಹಿಮೆಯನ್ನು ಬಣ್ಣಿಸಲು ಅಸದಳ. ಅವನು ಶುಭ್ರತೆಯಲ್ಲಿ ಆಕಾಶವನ್ನು ಹೋಲುವನು. ಅವನಲ್ಲಿ ನಿಖಿಲ ಪ್ರಪಂಚದ ಸೃಷ್ಟಿ ಸ್ಥಿತಿ ಪ್ರಳಯಗಳ ಕಾರ್ಯವು ಕಲ್ಪಿತವಾಗಿದೆ. ಪ್ರಜ್ವಲಿಸುವ ನನ್ನ ಈ ಜೀವನದ ಭಕ್ತಿ ವಿಶ್ವಕ್ಕೆಲ್ಲ ಒಡೆಯನಾದರೂ ತನಗೆ ಯಾರೂ ಒಡೆಯನಿಲ್ಲದ ಶಿವನ ಪಾದಪದ್ಮಗಳಲ್ಲಿ ನೆಲಸಲಿ.

೨. ಮೋಹವನ್ನು ನಾಶಮಾಡುವ ಶಿವನಲ್ಲಿ ಈಶತ್ವ ಸದಾ ನೆಲೆಸಿರುವುದು. ಪ್ರಕಟವಾದ ಮಹಾಪ್ರೇಮದಿಂದ ಮಹಾದೇವನೆಂಬ ಹೆಸರು ಅವನಿಗೆ ಸಲ್ಲುವುದು. ಆ ಶಿವನ ಪರಮಪ್ರೇಮದಿಂದ ಕೂಡಿದ ಆಲಿಂಗನವು ವಿಶ್ವದ ಪ್ರಭುತ್ವವೂ ಕೂಡ ಒಂದು ವ್ಯಾಜ ಎಂಬ ಭಾವವನ್ನು ಹೃದಯದಲ್ಲಿ ಉಂಟುಮಾಡುವುದು.

೩. ಬಿರುಗಾಳಿ ಬೀಸಿ ಅಲೆಯನ್ನು ಎಬ್ಬಿಸುವಂತೆ ಪೂರ್ವಸಂಸ್ಕಾರವೆಂಬ ಮಹಾ ಬಿರುಗಾಳಿ ಬೀಸಿ, 'ನಾನು' 'ನೀನು' ಎಂಬ ಅಲೆಗಳನ್ನು ಉಂಟುಮಾಡಿದೆ. ಶಿವನಲ್ಲಿ ನೆಲಸಿರುವ ಅಂತಹ ವಿಕಲಿತ ರೂಪವಾಗಿದ್ದ ಮನಸ್ಸನ್ನು ನಮಿಸುವೆನು.

೪. ಎಲ್ಲಿ ಜನಕ ಜನಿತಭಾವ, ಪರಿಶುದ್ಧವಾದ ವೃತ್ತಿಗಳು ಮತ್ತು ಅಗಣಿತವಾದ ರೂಪಗಳು ಒಂದೇ ಸತ್ಯದಲ್ಲಿ ಏಕವಾಗುವುವೋ, ಎಲ್ಲಿ ಒಳಗೆ ಹೊರಗೆ ಎಂಬ ಭಾವ ಕೊನೆಗಾಣುವುದೋ, ಪ್ರವೃತ್ತಿಗಳೆಂಬ ಅಲೆಗಳು ಎಲ್ಲಿ ಶಾಂತವಾಗುವುವೋ ಅಂತಹ ಚಿತ್ತವೃತ್ತಿಯನ್ನು ವಿರೋಧಿಸುವ ಹರನನ್ನು ನಾನು ಸ್ತುತಿಸುವೆನು.

೫. ತಿಮಿರಮಾಲೆ ಯಾರಲ್ಲಿ ಜಾರಿಹೋಗಿರುವುದೋ, ಶುಭ್ರತೇಜಃ ಪ್ರಕಾಶನಾಗಿ ಧವಳ ಕಮಲದಂತೆ ಯಾರು ಶೋಭಿಸುತ್ತಿರುವನೋ, ಯಾರ ಅಟ್ಟಹಾಸ ಜ್ಞಾನವನ್ನು ಪ್ರಚೋದಿಸುವುದೋ, ಯಾರನ್ನು ನಿರಂತರಧ್ಯಾನದಿಂದ ತಮ್ಮ ಹೃದಯಾಂತರಾಳದಲ್ಲಿ ಜ್ಞಾನಿಗಳು ತಿಳಿದುಕೊಂಡಿರುವರೋ ಅಂತಹ ಮಾನಸ ಸರೋವರದಲ್ಲಿ ವಿಹರಿಸುತ್ತಿರುವ ಪರಮಶಿವನೆಂಬ ರಾಜಹಂಸವು ಪ್ರಣತನಾದ ನನ್ನನ್ನು ರಕ್ಷಿಸಲಿ.

೬. ಯಾರು ನಮ್ಮ ದುರಿತಗಳನ್ನು ಪರಿಹರಿಸುವುದರಲ್ಲಿ ದಕ್ಷನೋ, ಯಾರಿಗೆ ದಕ್ಷ ತನ್ನ ಮಗಳನ್ನು ಕೊಟ್ಟನೋ, ಯಾರು ಕಲಿ ಕಲಂಕವನ್ನು ಹೋಗಲಾಡಿಸಿದನೋ, ಯಾರು ಕಲ್ಲಾರಪುಷ್ಪದಂತೆ ಸುಂದರನಾಗಿರುವನೋ, ಯಾರು ಮತ್ತೊಬ್ಬರ ಹಿತಕ್ಕಾಗಿ ಸದಾ ತನ್ನ ಪ್ರಾಣವನ್ನು ಕೊಡಲು ಸಿದ್ಧನಾಗಿದ್ದನೋ, ಯಾರ ಕೃಪಾದೃಷ್ಟಿ ದೀನರ ಮೇಲೆ ಇರುವುದೋ ಅಂತಹ ನೀಲಕಂಠನಿಗೆ ನಮಸ್ಕರಿಸುತ್ತೇನೆ.

\begin{center}
\textbf{ಅಂಬಾ ಸ್ತೋತ್ರಂ}\footnote{\engfoot{C.W, Vol. IV, P. 498}}
\end{center}

\begin{myquote}
ಕಾ ತ್ವಂ ಶುಭೇ ಶಿವಕರೇ ಸುಖದುಃಖಹಸ್ತೇ\\ಆಘೂರ್ಣಿತಂ ಭವಜಲಂ ಪ್ರಬಲೋರ್ಮಿಭಂಗೈಃ~।\\ಶಾಂತಿಂ ವಿಧಾತುಮಿಹ ಕಿಂ ಬಹುಧಾ ವಿಭಗ್ನಾಂ\\
ಮಾತಃ ಪ್ರಯತ್ನ ಪರಮಾಸಿ ಸದೈವ ವಿಶ್ವೇ
\end{myquote}

\versenum{॥ ೧~॥}

\begin{myquote}
ಸಂಪಾದಯತ್ಯವಿರತಂ ತ್ವವಿರಾಮವೃತ್ತಾ\\ಯಾ ವೈ ಸ್ಥಿತಾ ಕೃತಫಲಂ ತ್ವಕೃತಸ್ಯ ನೇತ್ರೀ~।\\
ಸಾ ಮೇ ಭವತ್ವನುದಿನಂ ವರದಾ ಭವಾನೀ\\ಜಾನಾಮ್ಯಹಂ ಧ್ರುವಮಿದಂ ಧೃತಕರ್ಮಪಾಶಾ
\end{myquote}

\versenum{॥ ೨~॥}

\begin{myquote}
ಕೋ ವಾ ಧರ್ಮಃ ಕಿಮಕೃತಂ ಕಃ ಕಪಾಲಲೇಖಃ\\ಕಿಂವಾದೃಷ್ಟಂ ಫಲಮಹಾಸ್ತಿ ಹಿ ಯಾಂ ವಿನಾ ಭೋಃ~।\\
ಇಚ್ಛಾಪಾಶೈರ್ನಿಯಮಿತಾ ನಿಯಮಾಃ ಸ್ವತಂತ್ರೈಃ\\ಯಸ್ಯಾ ನೇತ್ರೀ ಭವತು ಸಾ ಶರಣಂ ಮಮಾದ್ಯಾ
\end{myquote}

\versenum{॥ ೩~॥}

\begin{myquote}
ಸಂತಾನಯಂತಿ ಜಲಧಿಂ ಜನಿಮೃತ್ಯುಜಾಲಂ\\ಸಂಭಾವಯಂತ್ಯವಿಕೃತಂ ವಿಕೃತಂ ವಿಭಗ್ನಮ್~।\\
ಯಸ್ಯಾ ವಿಭೂತಯ ಇಹಾಮಿತಶಕ್ತಿಪಾಲಾಃ\\ನಾಶ್ರಿತ್ಯ ತಾಂ ವದ ಕುತಃ ಶರಣಂ ವ್ರಜಾಮಃ
\end{myquote}

\versenum{॥ ೪~॥}

\begin{myquote}
ಮಿತ್ರೇ ಶತ್ರೌ ತ್ವವಿಷಮಂ ತವ ಪದ್ಮನೇತ್ರಂ\\ಸ್ವಸ್ಥೇ ದುಸ್ಥೇ ತ್ವವಿತಥಂ ತವ ಹಸ್ತಪಾತಃ~॥\\ಮೃತ್ಯುಚ್ಛಾಯಾ ತವದಯಾ ತ್ವಮೃತಂಚ ಮಾತಃ\\ಮಾ ಮಾಂ ಮುಂಚಂತು ಪರಮೇ ಶುಭದೃಷ್ಟಯಸ್ತೇ
\end{myquote}

\versenum{॥ ೫~॥}

\begin{myquote}
ಕ್ವಾಂಬಾ ಸರ್ವಾ ಕ್ವ ಗಣನಂ ಮಮ ಹೀನಬುದ್ಧೇಃ\\ಧರ್ತ್ಥುಂ ದೋರ್ಭ್ಯಾಮಿವ ಮತಿರ್ಜಗದೇಕಧಾತ್ರೀಂ\\ಶ‍್ರೀಸಂಚಿಂತ್ಯಂ ಸುಚರಣಂ ಅಭಯ ಪ್ರತಿಷ್ಠಂ\\ಸೇವಾಸಾರೈರಭಿನುತಂ ಶರಣಂ ಪ್ರಪದ್ಯೇ
\end{myquote}

\versenum{॥ ೬~॥}

\begin{myquote}
ಯಾ ಮಾಮಾಜನ್ಮ ವಿನಯತ್ಯತಿದುಃಖಮಾರ್ಗೈಃ~।\\
ಆಸಂಸಿದ್ಧೇಃ ಸ್ವಕಲಿತೈರ್ಲಲಿತೈರ್ವಿಲಾಸೈಃ\\ಯಾ ಮೇ ಬುದ್ಧಿಂ ಸುವಿದಧೇ ಸತತಂ ಧರಣ್ಯಾಂ\\ಸಾಂಬಾ ಸರ್ವಾ ಮಮ ಗತಿಃ ಸಫಲೇಫಲೇ ವಾ
\end{myquote}

\versenum{॥ ೭~॥}

\begin{center}
\textbf{ಅಂಬಾ ಸ್ತೋತ್ರ}
\end{center}

೧. ಸುಖದುಃಖಗಳನ್ನು ತನ್ನ ಕೈಗಳೊಳಗೆ ಧರಿಸಿರುವವಳೇ, (ದರ್ಶನ ಮಾತ್ರದಿಂದ) ಶಾಂತಿಸುಖವನ್ನು ಬೀರುತ್ತಿರುವ ಮಂಗಳ ಸ್ವರೂಪಿಣಿಯೇ ನೀನಾವಳು? ಈ ವಿಶ್ವದಲ್ಲಿ ಸಂಸಾರಸಾಗರದ ನೀರು ಪ್ರಬಲವಾದ ಅಲೆಗಳ ಏರಿಳಿತಗಳಿಂದ ಕ್ಷೋಭವನ್ನು ಹೊಂದಿ ಸುಳಿಗಳಿಂದ ಕೂಡಿದೆ. ಎಲೈ ತಾಯೆ, ಈ ಭವಸಾಗರದಲ್ಲಿ ಅನೇಕ ವಿಧದಿಂದ ನಷ್ಟವಾದ ಶಾಂತಿಯನ್ನು ಸ್ಥಾಪಿಸುವುದಕ್ಕಾಗಿ ನೀನು ಅನವರತವೂ ಅತ್ಯಂತವಾಗಿ ಪ್ರಯತ್ನಿಸುತ್ತಿರುವೆಯಾ?

೨. ಯಾವಳು ವಿರಾಮವಿಲ್ಲದೆ ದುಡಿಯುತ್ತ ಅನವರತವೂ ಜೀವಿಗಳಿಗೆ ಪ್ರಾರಬ್ಧ ಕರ್ಮಗಳ ಫಲವನ್ನು ನೀಡುತ್ತಲೂ, ವರ್ತಮಾನ ಭವಿಷ್ಯತ್ಕಾಲಗಳಲ್ಲಿ ಮಾಡತಕ್ಕ ಕರ್ಮಗಳಿಗೆ ಜೀವಿಗಳನ್ನು ಒಯ್ಯುತ್ತಲೂ ಇರುವಳೋ ಆ ಭವಾನಿಯು ನನಗೆ ಅನುದಿನವೂ ಅಭೀಷ್ಟಗಳನ್ನು ಅನುಗ್ರಹಿಸುವವಳಾಗಲಿ. ಅವಳೇ ಕರ್ಮ ಪಾಶವನ್ನು ಹಿಡಿದಿರುವವಳೆಂದು ನಾನು ತಿಳಿದಿದ್ದೇನೆ; ಇದು ಖಂಡಿತವಾದ ಮಾತು.

೩. ಎಲೆ ಜೀವಿಗಳೇ, ಯಾವ ಧರ್ಮವು, ಯಾವ (ಜೀವಿಗಳ) ವರ್ತಮಾನ ಭವಿಷ್ಯತ್ಕಾಲಗಳ ಕರ್ಮವು, ಯಾವ ಹಣೇಬರಹವು ಮತ್ತು ಯಾವ ಅದೃಷ್ಟ ಫಲ ತಾನೇ ಈ ಭವಾನಿಯನ್ನು ಬಿಟ್ಟು ಇದೆ? (ಅಂದರೆ ಸರ್ವವೂ ಅವಳ ಇಚ್ಛಾಧೀನ). ಎಲ್ಲ ನಿಯಮಗಳೂ ಅವಳ ಸ್ವತಂತ್ರವಾದ ಇಚ್ಛಾಪಾಶಗಳಿಂದ ಕಟ್ಟಲ್ಪಟ್ಟಿವೆ. ಜಗತ್ತಿನ ಪ್ರಥಮ ಮಾರ್ಗದರ್ಶಿಯಾದ ಆ ಭವಾನಿಯೇ ನನಗೆ ರಕ್ಷಕಳಾಗಲಿ.

೪, ಯಾವ ನಿನ್ನ ಅಮಿತವಾದ ಶಕ್ತಿಯುಳ್ಳ ವಿಭೂತಿಗಳು (ಪ್ರಭಾವಗಳು) ಈ ವಿಶ್ವದಲ್ಲಿ ಜನನ ಮರಣ ಪರಂಪರೆಗಳುಳ್ಳ ಭವಸಾಗರವನ್ನು ಉಕ್ಕಿಸುತ್ತವೆಯೋ ಮತ್ತು ವಿರಾಮವಿಲ್ಲದ ಅಖಂಡ ಬ್ರಹ್ಮವಸ್ತುವನ್ನು ನಾಮರೂಪಗಳಿಂದ ವಿಕಾರವುಳ್ಳದ್ದಾಗಿಯೂ, ಅನೇಕ ಜೀವಾತ್ಮಗಳ ರೂಪದಿಂದ ಖಂಡ ಖಂಡಗಳನ್ನಾಗಿಯೂ ಮಾಡಿ, ಕೊನೆಯಲ್ಲಿ ಆ ಜೀವಾತ್ಮಗಳನ್ನು ಪರಮಾತ್ಮನಲ್ಲಿ ಲೀನಗಳನ್ನಾಗಿ ಮಾಡುತ್ತವೆಯೋ ಅಂತಹ ನಿನ್ನನ್ನು ಆಶ್ರಯಿಸದೆ ಮತ್ತಾರಿಂದ ನಾವು ರಕ್ಷಣೆಯನ್ನು ಹೊಂದಬಲ್ಲೆವು? ಹೇಳು.

೫. ನಿನ್ನ ಕಮಲನೇತ್ರಗಳು ಮಿತ್ರರಲ್ಲಿಯೂ ಶತ್ರುಗಳಲ್ಲಿಯೂ ವೈಷಮ್ಯವಿಲ್ಲದೆ ಸಮಾನದೃಷ್ಟಿಯನ್ನು ಬೀರುತ್ತವೆ. ಸ್ವಸ್ಥರಾಗಿದ್ದವರಲ್ಲಿಯೂ ದುರವಸ್ಥೆಯನ್ನು ಅನುಭವಿಸುತ್ತಿರುವವರಲ್ಲಿಯೂ ನಿನ್ನ (ಅಭಯ) ಹಸ್ತದ ಸ್ಪರ್ಶವು ನಿಜವಾಗಿಯೂ ಸಮಾನವಾಗಿಯೇ ಇದೆ. ಎಲೆ ತಾಯೆ, ಮೃತ್ಯುವಿನ ಛಾಯೆ (ಈ ಭೂಲೋಕದಲ್ಲಿ ಸಂಸಾರಿಯಾಗಿರುವಿಕೆ) ಮತ್ತು ಮೋಕ್ಷ ಇವೆರಡೂ ನಿನ್ನ ಕೃಪಾಪ್ರಸಾದಗಳೇ ಆಗಿವೆ. ಎಲೆ ಪರಮಾತ್ಮ ಸ್ವರೂಪಳೆ, ನಿನ್ನ ಮಂಗಳಕರವಾದ ದೃಷ್ಟಿಯು ನನ್ನಿಂದ ವಿಮುಖವಾಗದಿರಲಿ.

೬. ವಿಶ್ವಸ್ವರೂಪಳೂ ಜಗನ್ಮಾತೆಯೂ ಆದ ನೀನೆಲ್ಲಿ? ಅತ್ಯಲ್ಪಮತಿಯಾದ ನನ್ನ ಈ ಸ್ತೋತ್ರವೆಲ್ಲಿ? ಲೋಕವನ್ನು ತಾನೊಬ್ಬಳೆ ಧರಿಸಿಕೊಂಡಿರುವವಳಾದ ನಿನ್ನನ್ನು ನನ್ನ ಕೈಗಳಿಂದ ಧರಿಸಿಕೊಳ್ಳಲು ನನ್ನ ಮನಸ್ಸು ಉದ್ಯುಕ್ತವಾದಂತಿದೆ. ಶ‍್ರೀದೇವಿಯಿಂದ ಸತತವೂ ಧ್ಯಾನಿಸಲ್ಪಡುತ್ತಿರುವ ಅಭಯಕ್ಕೆ ಆಧಾರವಾದ ಮತ್ತು ನಿನ್ನ ಸೇವೆಯೇ ಪ್ರಧಾನವಾಗುಳ್ಳ (ಭಕ್ತಾಗ್ರಣಿಗಳಿಂದ) ಸ್ತುತಿಸಲ್ಪಡುತ್ತಿರುವ ನಿನ್ನ ಶೋಭನವಾದ ಚರಣಗಳನ್ನು ನಾನು ಶರಣು ಹೊಂದಿದ್ದೇನೆ.

೭. ಯಾವ ಮಾತೆಯು ನನ್ನ ಜನನಾರಭ್ಯ ತನ್ನದೇ ಆದ ಲಲಿತವಿಲಾಸಗಳಿಂದ ಮೋಕ್ಷಪ್ರಾಪ್ತಿಯವರೆಗೂ ಅತ್ಯಂತ ದುಃಖಕರವಾದ ಮಾರ್ಗಗಳ ಮೂಲಕ ನನ್ನನ್ನು ಕೊಂಡೊಯ್ಯುತ್ತಿರುವಳೋ, ಯಾರು ನನ್ನ ಬುದ್ಧಿಯನ್ನು ಯಾವಾಗಲೂ ಪ್ರಚೋದಿಸುತ್ತಿರುವಳೋ, ಆ ಮಾತೆಯೇ ನಾನು ಫಲಸಹಿತನಾಗಲಿ ಅಥವಾ ಫಲರಹಿತನಾಗಲಿ ನನಗೆ ಗತಿ ಮತ್ತು ಸರ್ವಸ್ವ.

\chapter[ಮೂಲ ಬಂಗಾಳಿಯ ಕವನಗಳು]{ಮೂಲ ಬಂಗಾಳಿಯ ಕವನಗಳು}

\begin{center}
\textbf{ಶ‍್ರೀ ರಾಮಕೃಷ್ಣ ಆರಾತ್ರಿಕ\supskpt{\footnote{\engfoot{C.W, Vol. IV, P. 504}}}}
\end{center}

\begin{myquote}
ಖಂಡನ ಭವಬಂಧನ ಜಗವಂದನ ವಂದಿತೋಮಾಯ!\\ನಿರಂಜನ ನರರೂಪಧರ ನಿರ್ಗುಣಗುಣಮಯ
\end{myquote}

\versenum{॥ ೧~॥}

\begin{myquote}
ಮೋಚನ ಅಘದೂಷಣ ಜಗಭೂಷಣ ಚಿದ್ಘನಕಾಯ~।\\ಜ್ಞಾನಾಂಜನ ವಿಮಲ ನಯನ ವೀಕ್ಷಣೆ ಮೋಹಜಾಯ.
\end{myquote}

\versenum{॥ ೨~॥}

\begin{myquote}
ಭಾಸ್ವರ ಭಾವಸಾಗರ ಚಿರ ಉನ್ಮದ ಪ್ರೇಮ ಪಾಥಾರ~।\\ಭಕ್ತಾರ್ಜನ ಯುಗಲಚರಣ, ತಾರಣ ಭವಪಾರ
\end{myquote}

\versenum{॥ ೩~॥}

\begin{myquote}
ಝೃಂಬಿತ ಯುಗ ಈಶ್ವರ ಜಗದೀಶ್ವರ ಯೋಗಸಹಾಯ~।\\ನಿರೋಧನ ಸಮಾಹಿತ ಮನ ನಿರಖಿ ತವ ಕೃಪಾಯ
\end{myquote}

\versenum{॥ ೪~॥}

\begin{myquote}
ಭಂಜನ ದುಃಖಗಂಜನ, ಕರುಣಾಘನ ಕರ್ಮಕಠೋರ~।\\ಪ್ರಾಣಾರ್ಪಣ–ಜಗತ ತಾರಣ, ಕೃಂತನ ಕಲಿಡೋರ
\end{myquote}

\versenum{॥ ೫~॥}

\begin{myquote}
ವಂಚನ ಕಾಮಕಾಂಚನ ಅತಿನಿಂದಿತ ಇಂದ್ರಿಯ ರಾಗ~।\\ತ್ಯಾಗೀಶ್ವರ ಹೇ ನರವರ ದೇಹಪದೆ ಅನುರಾಗ
\end{myquote}

\versenum{॥ ೬~॥}

\begin{myquote}
ನಿರ್ಭಯ ಗತಸಂಶಯ ದೃಢನಿಶ್ಚಯ ಮಾನಸವಾನ್~।\\ನಿಷ್ಕಾರಣ ಭಕತಶರಣ ತ್ಯಜಿ ಜಾತಿ ಕುಲ ಮಾನ
\end{myquote}

\versenum{॥ ೭~॥}

\begin{myquote}
ಸಂಪದ ತವ ಶ‍್ರೀಪದ ಭವಗೋಷ್ಪದ ವಾರಿ ಯಥಾಯ~।\\ಪ್ರೇಮಾರ್ಪಣ ಸಮದರಶನ ಜಗಜನ ದುಃಖ ಜಾಯ
\end{myquote}

\versenum{॥ ೮~॥}

\begin{myquote}
ನಮೋ ನಮೋ ಪ್ರಭು ವಾಕ್ಯಮನಾತೀತ ಮನೋವಚನೈಕಾಧಾರ~।\\ಜ್ಯೋತಿರ ಜ್ಯೋತಿ ಉಜ್ವಲ ಹೃದಿಕಂದರ ತುಮಿತಮ ಭಂಜನಹಾರ
\end{myquote}

\versenum{॥ ೯~॥}

\begin{myquote}
ಧೇ ಧೇ ಧೇ ಲಂಗರಂಗಭಂಗ ಬಾಜೇ ಅಂಗಸಂಗ ಮೃದಂಗ\\ಗಾಯಿಛೇ ಛಂದ ಭಕತವೃಂದ, ಆರತಿ ತೊಮಾರ~।
\end{myquote}

\begin{myquote}
ಜಯಜಯ ಆರತಿ ತೊಮಾರ\\ಹರಹರ ಆರತಿ ತೊಮಾರ\\ಶಿವ ಶಿವ ಆರತಿ ತೊಮಾರ~।\\ಖಂಡನ ಭವ ಬಂಧನ ಜಗವಂದನ ವಂದಿತೊಮಾಯ
\end{myquote}

\versenum{॥ ೧೦~॥}

\begin{center}
\textbf{ಶ‍್ರೀರಾಮಕೃಷ್ಣ ಆರಾತ್ರಿಕ}
\end{center}

ಇದು ಸ್ವಾಮಿಜಿಯವರು ಬಂಗಾಳಿಯಲ್ಲಿ ರಚಿಸಿರುವ ಪ್ರಸಿದ್ಧವಾದ ಆರತಿಗೀತೆ. ಇದಕ್ಕೆ ರಾಗಸಂಯೋಜನೆ ಮಾಡಿದವರೂ ಸ್ವತಃ ಸ್ವಾಮಿಜಿಯವರೇ. ಇದನ್ನು ರಾಮಕೃಷ್ಣ ಮಹಾಸಂಘದ ಎಲ್ಲ ಶಾಖೆಗಳಲ್ಲಿ ಆರತಿಯ ಸಂದರ್ಭದಲ್ಲಿ ಹಾಡಲಾಗುತ್ತದೆ. ಶ‍್ರೀರಾಮಕೃಷ್ಣರ ವ್ಯಕ್ತಿತ್ವವನ್ನು ಸ್ವಾಮಿಜಿಯವರು ಕಂಡ ಬಗೆಯನ್ನು ಈ ಗೀತೆಯಲ್ಲಿ ಮನಗಾಣಬಹುದು.

ನೋಡಿ: ಗುರುವಿನೊಡನೆ ದೇವರಡಿಗೆ (ಅನು: ಕುವೆಂಪು) ಗ್ರಂಥದಲ್ಲಿ 'ಭಗವಂತನೆ ಶ‍್ರೀರಾಮಕೃಷ್ಣನ ರೂಪದಲ್ಲಿ' ಎಂಬ ಹದಿನೇಳನೆಯ ಅಧ್ಯಾಯ.

\begin{center}
(ಅನುವಾದ: ಕುವೆಂಪು)
\end{center}

\begin{myquote}
ಭವಬಂಧನವನು ಖಂಡಿಸುವಾತನೆ,\\ಲೋಕವೆ ವಂದಿಸುವಾತನೆ,\\ವಂದಿಸುವೆವು ನಿನಗೆ!\\ನರರೂಪಧರ ನಿರಂಜನ ನಿರ್ಗುಣ ಗುಣಮಯನೆ,\\ವಂದಿಸುವೆವು ನಿನಗೆ!
\end{myquote}

\begin{myquote}
ಭಕ್ತಜನ ಭವಸಾಗರ ತಾರಣ ಚರಣಯುಗಲ ಭಾಸ್ವರನೆ,\\ಹೇ ಯುಗಈಶ್ವರ, ಜಗದೀಶ್ವರ, ಯೋಗಸಹಾಯನೆ,\\ಪದತಲದಲಿ ಚಿತ್ತವನಿಡೆ ಕೃಪೆದೋರೈ!
\end{myquote}

\begin{myquote}
ದುಃಖಾಟವಿ ದವರೂಪನೆ, ಕರ್ಮಕಠೋರನೆ, ಕರುಣಾ ಘನಮೂರ್ತಿ,\\ಜಗದುದ್ಧಾರಣ ಪ್ರಾಣಾರ್ಪಣ ಕಾರಣ ಹೇ ಕಲಿವಿಧ್ವಂಸನ ಕೀರ್ತಿ;\\ಕಾಮಿನಿಕಾಂಚನ ಅತಿನಿಂದಿತ ಇಂದ್ರಿಯರಾಗವಿದೂರ,\\ನಿರ್ಭಯ ಗತಸಂಶಯ ದೃಢನಿಶ್ಚಯ ಮಾನಸ ಸಾರ,\\ಜಯ ನರವರ, ಹೇ ತ್ಯಾಗೀಶ್ವರ, ವಂದನೆ ಶತವಂದನೆ ಕೃಪೆದೋರ!
\end{myquote}

\begin{myquote}
ತವ ಪದ ಸಂಪದಕೀ ಭವಸಾಗರವೂ ಗೋಷ್ಪದ ವಾರಿ ಸಮಾನ,\\ಭಕ್ತ ಶರಣ ಹೇ ಮುಕ್ತಿ ನಿಷ್ಕಾರಣ ತ್ಯಕ್ತ ಜಾತಿ ಮಾನ!\\ಉಜ್ವಲ ಸ್ವರ್ ಜ್ಯೋತಿರ್‌ಜ್ಯೋತಿಯೆ, ಹೃತ್ಕಂದರ ಘನತಮಹಾರಿ,
\end{myquote}

\begin{myquote}
ನಮೋ ನಮೋ ಹೇ ಗುರುದೇವವರೇಣ್ಯ, ಜಗತ್ ಸರ್ವಚಿತ್ಸಂಚಾರಿ!\\ಆರತಿಯಿದೊ ಆರತಿಯಿದೊ ಆರತಿಯಿದೊ ದೇವ!\\ಜಯ ಜಯ ಆರತಿ ಇದೊ ದೇವ!\\ಹರ ಹರ ಆರತಿ ಇದೋ ದೇವ!\\ಶಿವ ಶಿವ ಆರತಿ ಇದೊ ದೇವ!\\ಆರತಿಯಿದೊ ಆರತಿಯಿದೊ ಆರತಿಯಿದೊ ಶ‍್ರೀಗುರುದೇವ!
\end{myquote}

\begin{center}
\textbf{ಶಿವಸಂಗೀತ}\footnote{\engfoot{C.W, Vol. VIII, P.171}}
\end{center}

\begin{myquote}
ತಾಥೈಯಾ ತಾಥೈಯಾ ನಾಚೇ ಭೋಲಾ\\ಬಂ ಬಬ ಬಾಜೇ ಗಾಲ~॥
\end{myquote}

\begin{myquote}
ಡಿಮಿ ಡಿಮಿ ಡಿಮಿ ಡಮರು ಬಾಜೆ\\ದುಲಿಛೆ ಕಪಾಲ ಮಾಲ~॥
\end{myquote}

\begin{myquote}
ಗರಜೇ ಗಂಗಾ ಜಟಾ ಮಾಝೇ\\ಉಗರೆ ಅನಲ ತ್ರಿಶೂಲ ರಾಜೇ\\ಧಕ್ ಧಕ್ ಧಕ್ ಮೌಲಿ ಬಂಧ\\ಜ್ವಲೇ ಶಶಾಂಕ ಭಾಲ~॥
\end{myquote}

\begin{center}
\textbf{ಶಿವನರ್ತನ\\೧}
\end{center}

\begin{center}
(ಅನುವಾದ: ಡಾ~॥ ಜಿ. ಎಸ್. ಶಿವರುದ್ರಪ್ಪ)
\end{center}

ಈ ಬಂಗಾಳಿಗೀತೆಯನ್ನು ಸ್ವಾಮೀಜಿಯವರು ರಚಿಸಿದ್ದು ೧೮೮೭ರಲ್ಲಿ, ಬಾರಾನಗರ ಮಠದಲ್ಲಿ ಶಿವರಾತ್ರಿಯ ಸಂದರ್ಭದಲ್ಲಿ.

ನೋಡಿ: ಶ‍್ರೀರಾಮಕೃಷ್ಣ ವಚನವೇದ (ಉತ್ತರಾರ್ಧ), 'ಮಹಾ ಸಮಾಧಿಯ ಅನಂತರ' ಎಂಬ ಐವತ್ತೆರಡನೆಯ ಅಧ್ಯಾಯ.

\begin{myquote}
ಅದೋ ಅಲ್ಲಿ ಪರಮಶಿವನು\\ನರ್ತಿಸುವನು ಮೋದದಿ,\\ಡಿಮಿ ಡಿಮಿ ಡಿಮಿ ಡಮರುನಾದ\\ಮೊಳಗಿದೆ ಬ್ರಹ್ಮಾಂಡದಿ!
\end{myquote}

\begin{myquote}
ಕೊರಳ ಸುತ್ತ ರುಂಡಮಾಲೆ\\ದಿಕ್ಕು ದೆಸೆಗೆ ತೂಗಿದೆ,\\ಜಡೆಯ ಗಂಗೆ ಬುಸುಗುಟ್ಟಿದೆ,\\ಉಗ್ರ ಅನಲ ಶೂಲಕಾಂತಿ\\ಕಣ್ಣುಗಳನೆ ಕುಕ್ಕಿದೆ!
\end{myquote}

\begin{myquote}
ಕಟಿಯ ಸುತ್ತ ಸರ್ಪರಾಜಿ\\ಹೆಡೆಯಾಡಿಸಿ ಮಿರುಗಿದೆ,\\ಅದೊ, ಲಲಾಟದಲ್ಲಿ ನೋಡು\\ಶಶಿಕಳೆಯೂ ಜ್ವಲಿಸಿದೆ!
\end{myquote}

\begin{center}
(೨)
\end{center}

\begin{myquote}
ಹರ ಹರ ಹರ ಭೂತನಾಥ ಪಶುಪತಿ\\ಯೋಗೀಶ್ವರ ಮಹಾದೇವ ಶಿವ ಪಿನಾಕಪಾಣಿ~॥
\end{myquote}

\begin{myquote}
ಉರ್ಧ್ವ ಜ್ವಲತ ಜಟಾಜಾಲ\\ನಾಚತ ವ್ಯೋಮಕೇಶ ಭಾಲ\\ಸಪ್ತಭುವನ ಧರತ ತಾಲ\\ಟಲಮಲ ಅವನೀ~॥
\end{myquote}

\begin{center}
(೩)
\end{center}

\begin{myquote}
ಹರ ಹರ ಹರ ಭೂತನಾಥ\\ಕುಣಿಯುತಿಹನು ಪಶುಪತಿ;\\ಯೋಗೀಶ್ವರ ಮಹಾದೇವ\\ಶಿವ ಪಿನಾಕಪಾಣಿಯು!
\end{myquote}

\begin{myquote}
ಜ್ವಲಿಸುತ್ತಿರುವ ಜಾಟಾಜೂಟ\\ಆಗಸದಲಿ ಹಬ್ಬಿದೆ;\\ಸಪ್ತಭುವನ ತಾಳಗೊಳಲು\\ಭುವಿಯೆ ಕಂಪಿಸುತ್ತಿದೆ!
\end{myquote}

\begin{center}
\textbf{ಸೃಷ್ಟಿ}\footnote{\engfoot{C.W, Vol. IV, P.497}}
\end{center}

\begin{myquote}
ಏಕರೂಪ ಅರೂಪ ನಾಮವರಣ\\ಅತೀತ ಅಗಾಮಿ ಕಾಲಹೀನ~।\\ದೇಶಹೀನ ಸರ್ವಹೀನ\\ನೇತಿನೇತಿ ವಿರಾಮ ಯಥಾಯ್
\end{myquote}

\versenum{॥ ೧~॥}

\begin{myquote}
ಸೇಥಾ ಹತೇ ವಹೇ ಕಾರಣಧಾರ\\ಧರಿಯೆ ವಾಸನಾವೇಶ ಉಜಲಾ\\ಗರಜಿ ಗರಜಿ ಉಠೇ ತಾರ ವಾಣಿ\\ಅಹಮಹಮಿತಿ ಸರ್ವಕ್ಷಣ
\end{myquote}

\versenum{॥ ೨~॥}

\begin{myquote}
ಸೇ ಅಪಾರ ಇಚ್ಚಾ ಸಾಗರ ಮಾಝೇ\\ಅಯುತ ಅನಂತ ತರಂಗ ರಾಜೇ\\ಕೊತೊಯಿರೂಪ ಕೊತೊಯಿ ಶಕತಿ\\ಕೊತೂ ಗತಿಸ್ಥಿತಿ ಕೇತೊರೆ ಗಣನ
\end{myquote}

\versenum{॥ ೩~॥}

\begin{myquote}
ಕೋಟಿ ಚಂದ್ರ ಕೋಟಿ ತಪನ\\ಲೋಭಿಯೆ ಸೇಯಿ ಸಾಗರೇ ಜನಮ\\ಮಹಾಘೋರ ರೊವೆ ಛಾಯಿಲ ಗಗನ\\ಕೊರಿ ದಶದಿಕ್ ಜೋತಿಃ ಮಗನ
\end{myquote}

\versenum{॥ ೪~॥}

\begin{myquote}
ತಾಹೇ ವಸೇ ಕೊತೊಜಡಜೀವ ಪ್ರಾಣೀ\\ಸುಖ ದುಃಖ ಜರಾ ಜನಮ ಮರಣ\\ಜೇಯಿ ಸೂರ್ಯ ತಾರಿ ಕಿರಣ\\ಜೇಯಿ ಸೂರ್ಯ ಸೇಯಿ ಕಿರಣ
\end{myquote}

\versenum{॥ ೫~॥}

\begin{center}
\textbf{ಸೃಷ್ಟಿ}
\end{center}

ಈ ಕವನದಲ್ಲಿ ಅವ್ಯಕ್ತದಿಂದ ವ್ಯಕ್ತತೆಯ ಕಡೆಗೆ ಸೃಷ್ಟಿಯ ವಿವಿಧ ಹಂತಗಳನ್ನು ಅತ್ಯಂತ ಮೂರ್ತವಾಗಿ ಚಿತ್ರಿಸಲಾಗಿದೆ. 'ಇಲ್ಲ' ಎಂದು ಪ್ರಾರಂಭವಾಗುವ ಕವನ 'ಇಹುದು' ಎಂಬುದರಲ್ಲಿ ಕೊನೆಗೊಳ್ಳುವುದನ್ನು ಗಮನಿಸಬೇಕು.

\begin{myquote}
ಭೇದಗಳು ಅಲ್ಲಿಲ್ಲ,\\ನಾಮರೂಪಗಳಿಲ್ಲ,\\ಕಾಲದೇಶಗಳಿಲ್ಲ,\\ಮೇರೆಯಿಲ್ಲ;\\'ನೇತಿ'ಯೆಂಬುದು ಕೂಡ\\ವಿರಮಿಸಿಹುದಲ್ಲಿ!
\end{myquote}

\begin{myquote}
ಕಾರಣದ ಧಾರೆಯದು\\ಅಲ್ಲಿಂದ ಹರಿಯಿತದೊ\\ಬಯಕೆವೇಷವ ತಾನೆ\\ಧರಿಸಿಕೊಂಡು;\\ಅದರ ಹೊನಲಿನ ಅಲೆಯು\\ಎತ್ತರದಿ ಗರ್ಜಿಸಿದೆ\\'ನಾನು, ನಾನೆ'ನ್ನುತ್ತ\\ಎಡೆಬಿಡದಲೆ!
\end{myquote}

\begin{myquote}
ಪಾರವನು ಮೀರಿರುವ\\ಸಂಕಲ್ಪಸಾಗರದ ಮಧ್ಯದಲ್ಲಿ\\ಲೆಕ್ಕವಿಲ್ಲದ ಅಲೆಗಳೇಳುತಿಹವು;\\ಅದರ ರೂಪಗಳೆನಿತೊ,\\ಶಕ್ತಿಯೆನಿತೋ ಮತ್ತೆ\\ಸ್ಥಿತಿಗತಿಗಳೆನಿತೆಂದು ಗಣಿಸಿದವರಾರು?
\end{myquote}

\begin{myquote}
ಕೋಟಿ ಭಾಸ್ಕರ, ಕೋಟಿ\\ಚಂದ್ರರೆದ್ದಿಹರಲ್ಲಿ\\ಆ ಕಡಲಿನಲ್ಲಿ;\\ಆ ಮಹಾಗರ್ಜನೆಯು\\ದಿಗ್ದಿಸೆಯ ತುಂಬುತಿದೆ\\ಬೆಳಕಿನಲ್ಲಿ!
\end{myquote}

\begin{myquote}
ಅಲ್ಲಿ ಜಡ–ಜೀವವಿದೆ,\\ಸುಖ–ದುಃಖವಿದೆ, ಜನನ–\\ಮರಣವಿಹುದು;\\ಅಲ್ಲಿ ಆತನೆ ಸೂರ್ಯ,\\ಅವನದೇ ಕಿರಣ;\\ಅಲ್ಲ; ಆತನೆ ಸೂರ್ಯ,\\ಅವನ ತಾ ಕಿರಣ!
\end{myquote}

\begin{center}
\textbf{ಪ್ರಲಯ್ ವಾ ಗಭೀರ್ ಸಮಾಧಿ}\footnote{\engfoot{C.W, Vol. IV, P 498}}
\end{center}

\begin{myquote}
ನಹಿ ಸೂರ್ಯ ನಹಿ ಜ್ಯೋತಿಃ ಶಶಾಂಕಸುಂದರ\\ಭಾಸೆ ವ್ಯೋಮ ಛಾಯಸಮ ಛವಿ ವಿಶ್ವ ಚರಾಚರ
\end{myquote}

\versenum{॥ ೧~॥}

\begin{myquote}
ಅಸ್ಪುಟ ಮನ ಆಕಾಶ ಜಗತ ಸಂಸಾರ ಭಾಸೆ\\ಊಠಭಾಸೆ ಡೋಬೆ ಪುನಃ ಅಹಂಸ್ರೋತೇ ನಿರಂತರ
\end{myquote}

\versenum{॥ ೨~॥}

\begin{myquote}
ಧೀರೆ ಧೀರೆ ಛಾಯಾದಲ ಮಹಾಲಯ ಪ್ರವೇಶಿಲ\\ವಹಮಾತ್ರ 'ಅಮಿ, ಅಮಿ' ಏಯಿ ಧಾರಾ ಅನುಕ್ಷಣ
\end{myquote}

\versenum{॥ ೩~॥}

\begin{myquote}
ಸೇ ಧಾರಾಒ ಬದ್ಧ ಹೊಲ ಶೂನ್ಯ ಮಿಲಾಯಿಲ\\ಅವಾಂಗ್ ಮನಸ ಗೋಚರಂ ಬೊಝೆ ಪ್ರಾಣ ಬೊಝೆ ಜಾಯ
\end{myquote}

\versenum{॥ ೪~॥}

\begin{center}
\textbf{ಸಮಾಧಿ}
\end{center}

ಸಮಾಧಿಯಲ್ಲಿ ಲೀನವಾಗುವ ಪ್ರಜ್ಞೆಯ ವಿವಿಧ ಹಂತಗಳನ್ನು ಇದರಲ್ಲಿ ಚಿತ್ರಿಸಲಾಗಿದೆ

\begin{myquote}
ಸೂರ್ಯ ತಾನಲ್ಲಿಲ್ಲ, ಚಾರು ಚಂದ್ರಮನಿಲ್ಲ,\\ಜ್ಯೋತಿಯಿಲ್ಲ;\\ಆ ಮಹಾಶೂನ್ಯದೊಳು\\ನೆರಳಿನೊಲು ತೇಲುತಿದೆ\\ವಿಶ್ವಾಕೃತಿ!
\end{myquote}

\begin{myquote}
ಅಸ್ಪುಟದ ಚಿತ್ತದಾಕಾಶದಲಿ ನಾನೆಂಬ\\ಸ್ರೋತದಲಿ ಜಗವೆಲ್ಲ\\ಎದ್ದು ಬೀಳುತಿದೆ!\\ಮೆಲ್ಲ ಮೆಲ್ಲನೆ ನೆರಳ ದಳವೆಲ್ಲ ಹೊಗುತಿಹುದು\\ಮಹಾಲಯವನು;\\'ನಾನಿರುವೆ', 'ನಾನಿರುವೆ'\\ಎಂಬ ಧಾರೆಯದೊಂದೆ\\ಇದ್ದಿತಲ್ಲಿ!
\end{myquote}

\begin{myquote}
ನಿಂತಿತದುವೂ, ಈಗ\\ಶೂನ್ಯ ಶೂನ್ಯವ ಸೇರಿ ಒಂದಾಯಿತು;\\ಮಾತು ಮನಗಳ ಮೀರ್ದ\\ಆ ಇರವಿನಾಳವನು\\ಬಲ್ಲವನೆ ಬಲ್ಲ!
\end{myquote}

\begin{center}
\textbf{ಸಖಾರ ಪ್ರತಿ}\footnote{\engfoot{C.W Vol. IV. P. 493}}
\end{center}

\begin{myquote}
ಅಂಧಾರೆ ಆಲೋಕ ಅನುಭವ ದುಃಖ ಸುಖ ರೋಗ ಸ್ವಸ್ಥ ಬಾನ್\\ಪ್ರಾಣಸಾಕ್ಷಿ ಶಿಶುರಕ್ರಂದನ ಹೇಥಾ ಸುಖ ಇಚ್ಛಾಮತಿಮಾನ್~।
\end{myquote}

\versenum{॥ ೧~॥}

\begin{myquote}
ದ್ವಂದ್ವಯುದ್ಧ ಚಲೆ ಅನಿಬಾರ ಪಿತಾಪುತ್ರೆ ನಹಿದೇಯಿ ಸ್ಥಾನ\\'ಸ್ವಾರ್ಥ' 'ಸ್ವಾರ್ಥ' ಸದಾ ಎಯಿರಬ ಹೇಥಾ ಕೊಥಾ ಶಾಂತೀರ್ ಆಕಾರ
\end{myquote}

\versenum{॥ ೨~॥}

\begin{myquote}
ಸಾಕ್ಷಾತ್ ನರಕ ಸ್ವರ್ಗಮಯ–ಕೇನಾ ಪಾರೆ ಛಾಡಿತೇ ಸಂಸಾರ\\ಕರ್ಮ–ಪಾಶ ಗಲೆ ಬಂಧಾಜಾರ್–ಕೃತದಾಸ ಬೊಲೊ ಕುಥಾಜಾಯ
\end{myquote}

\versenum{॥ ೩~॥}

\begin{myquote}
ಯೋಗ–ಭೋಗ, ಗಾರಹಸ್ಥ–ಸಂನ್ಯಾಸ, ಜಪ ತಪ ಧನ–ಉಪಾರ್ಜನ\\ವ್ರತತ್ಯಾಗ ತಪಸ್ಯಕಠೋರ, ಸಬ್ ಮರ್ಮ ದೇಖೇಚಿ ಏಬಾರ
\end{myquote}

\versenum{॥ ೪~॥}

\begin{myquote}
ಜೆನೆಛಿ ಸುಖೇರನಹಿಲೇಶ ಶರೀರಧಾರಣ ವಿಡಂಬನ\\ಜತ ಉಚ್ಛ ತುಮಾರ್‌ ಹೃದಯ ತಥದುಃಖ ಜಾಹ ನಿಶ್ಚಯ
\end{myquote}

\versenum{॥ ೫~॥}

\begin{myquote}
ಹೃದಿವಾನ್ ನಿಸ್ವಾರ್ಥಪ್ರೇಮಿಕ್~। ಏಯಿಜಗತ ನಹಿ ತವಸ್ಥಾನ\\ಲೋಹಪಿಂಡ ಸಹಜೆ ಆಘತ ಮರ್ಮರ ಮೂರತಿ ತಾಕಿಸಯ
\end{myquote}

\versenum{॥ ೬~॥}

\begin{myquote}
ಹವೋ ಜಡಪ್ರಾಯ, ಅತಿನೀಚ, ಮುಖೆ ಮಧು ಅಂತರೆ ಗರಳ\\ಸತ್ಯಹೀನ ಸ್ವಾರ್ಥಪರಾಯಣ ತವಪಾವೆ ಏ ಸಂಸಾರೇ ಸ್ಥಾನ
\end{myquote}

\versenum{॥ ೭~॥}

\begin{myquote}
ವಿದ್ಯಾಹೇತು ಕರಿ ಪ್ರಾಣಪನ ಅರ್ಧೇಕ್ ಕರೇಛಿ ಆಯುಕ್ಷಯ\\ಪ್ರೇಮಹೇತು ಉನ್ಮಾದೇರ ಮತ, ಪ್ರಾಣಹೀನ ದರೇಚಿ ಛಾಯಾಯ
\end{myquote}

\versenum{॥ ೮~॥}

\begin{myquote}
ಧರ್ಮತರ ಕರಿ ಕಥ ಮತ್, ಗಂಗಾತೀರ ಸ್ಮಶಾನ ಆಲಾಯ\\ನದೀ ತೀರ ಪರ್ವತಗಹ್ವರ ಭಿಕ್ಷಾಸನೆ ಕತಕಾಲ್ ಜಾಯ
\end{myquote}

\versenum{॥ ೯~॥}

\begin{myquote}
ಅಸಹಾಯ ಭಿನ್ನವಾಸ ಧರೆ ದ್ವಾರೆ ದ್ವಾರೆ ಉದರಪೂರನ್\\ಭಗ್ನದೇಹ ತದಸ್ಯಾರ್‌ ಭಾರೆ ಕಿದನ ಕರಿನು ಉಪಾರ್ಜುನ
\end{myquote}

\versenum{॥ ೧೦~॥}

\begin{myquote}
ಶೊನೋ ಇಲಿ ಮರಮೇರ್ ಕಥಾ ಜೇನೆಚಿ ಜೀವನ ಸತ್ಯಸಾರ\\ತರಂಗ ಅಕುಲ ಭವಘೋರ್, ಏಕ್ ತೂರಿ ಕರೆ ಷಾರಾಪಾರ
\end{myquote}

\versenum{॥ ೧೧~॥}

\begin{myquote}
ಮಂತ್ರ–ಮಂತ್ರ ಪ್ರಾಣ ನಿಯಮನ ಮತಾಮತ್ ದರ್ಶನ್–ವಿಜ್ಞಾನ\\ತ್ಯಾಗ–ಭೋಗ–ಬುದ್ಧಿ‌ರ್ ವಿಭ್ರಮ, ಪ್ರೇಮ ಪ್ರೇಮ ಪ್ರೇಮ–ಏಯಿಮಾತ್ರ ಧನ
\end{myquote}

\versenum{॥ ೧೨~॥}

\begin{myquote}
ಜೀವ ಬ್ರಹ್ಮ, ಮಾನವ, ಈಶ್ವರ, ಭೂತಪ್ರೇತ, ಆದಿ ದೇವಗಣ\\ಪಶುಪಕ್ಷಿ ಕೀಟ ಅನುಕೀಟ ಏಯಿ ಪ್ರೇಮ ಹೃದಯೆ ಸಬಾರ
\end{myquote}

\versenum{॥ ೧೩~॥}

\begin{myquote}
ದೇವ ದೇವ ಬೊಲೊ ಆರ್ ಕೇವಾ? ಕೇವಾ ಬೊಲೊ ಸಬರೇ ಚಾಲಾಯ್\\ಪುತ್ರತರೆ ಮಾಯ ದೇಯ ಪ್ರಾಣ ದುಸ್ಸುಹರೆ ಪ್ರೇಮೇರ್‌ ಪ್ರೇರಣ
\end{myquote}

\versenum{॥ ೧೪~॥}

\begin{myquote}
ಹೊಯ ವಾಕ್ಯ ಮನ ಅಗೋಚರಕ ಸುಖದುಃಖೆ ತಿನಿ ಅಧಿಷ್ಠಾನ\\ಮಹಾಶಕ್ತಿ ಕಾಲಿ ಮೃತ್ಯುರೂಪಾ ಮಾತೃಭಾವ ತಾರೀ ಆಗಮನ
\end{myquote}

\versenum{॥ ೧೫~॥}

\begin{myquote}
ರೋಗ ಶೋಕ ದಾರಿದ್ರ್ಯ ಯಾತನ ಧರ್ಮಾಧರ್ಮ ಶಭಾಶುಭ ಫಲ\\ಸಬ ಭಾವೆ ತಾರೀ ಉಪಾಸನ ಜೀವೆ ಬಲ ಕೇವಾಕೀವಾ ಕರೆ
\end{myquote}

\versenum{॥ ೧೬~॥}

\begin{myquote}
ಭ್ರಾನ್ತ ಸೇಯಿ ಯೇವಾ ಸುಖ ಚಾಯಿ ದುಃಖಚಾಯಿ ಉನ್ಮಾದ್ ಸೇಜನ\\ಮೃತ್ಯುಮಾಂಗೆ ಸೇವೋ ಜೊ ಪಾಗಲ್ ಅಮೃತತ್ವ ವೃಥಾ ಅಕಿಂಚಿನ
\end{myquote}

\versenum{॥ ೧೭~॥}

\begin{myquote}
ಯತ ದೂರ ಯತ ದೂರ ಜಾವೂ ಬುದ್ಧಿ ರಥೆ ಕರಿ ಆರೋಹಣ\\ಏಯಿ ಸೇಯಿ ಸಂಸಾರ ಜಲಧಿ, ದುಃಖ ಸುಖ ಕರೆ ಆವರ್ತನ
\end{myquote}

\versenum{॥ ೧೮~॥}

\begin{myquote}
ಪಕ್ಷಹೀನ ಶೊನೊ ವಿಹಂಗಮ ಏ ಜೆ ನಹೆ ಪಥ ಪಾಲಾಬಾರ\\ಬಾರಂಬಾರ ಪಾಯಿಚೊ ಆಘಾತ್ –ಕೆನೊ ಕರೊ ವೃಥಾಯ ಉದ್ಯಮ
\end{myquote}

\versenum{॥ ೧೯~॥}

\begin{myquote}
ಛಡೊ ವಿದ್ಯಾ ಜಪಯಜ್ಞ ಬಲ ಸ್ವಾರ್ಥಹೀನ ಪ್ರೇಮಯಿ ಸಂಬಲ\\ದೇಖ್ ಶಿಕ್ಷಾ ದೇಯ ಪತಂಗಮ ಅಗ್ನಿಶಿಖಾ ಕೊರಿ ಆಲಿಂಗನ
\end{myquote}

\versenum{॥ ೨೦~॥}

\begin{myquote}
ರೂಪಮುಗ್ಧ ಅಂಧ ಕೀಟಾಧಮ ಪ್ರೇಮಮತ್ತ ತುಮಾರ ಹೃದಯ\\ಹೇ ಪ್ರೇಮಿಕ, ಸ್ವಾರ್ಥ–ಮಲಿನತ ಅಗ್ನಿ ಕುಂಡೆ ಕೊರೊ ವಿಸರ್ಜನ
\end{myquote}

\versenum{॥ ೨೧~॥}

\begin{myquote}
ಭಿಕ್ಷುಕೇರ ಕಬೆ ಬೋಲೊ ಸುಖ? ಕೃಪಾ ಪಾತ್ರ ಹೊಯ ಕಿವಾ ಫಲ\\ದಾವ್ ಆರ್ ಫಿರೆ ನಹಿ ಚಾವೊ ಥಾಕೆ ಯದಿ ಹೃದಯ ಸಂಬಲ
\end{myquote}

\versenum{॥ ೨೨~॥}

\begin{myquote}
ಅನಂತೇರ್ ತುಮಿ ಅಧಿಕಾರಿ ಪ್ರೇಮಸಿಂಧು ಹೃದಯ ವಿದ್ಯಮಾನ\\'ದಾವ್ ದಾವ್''–ಯೇವಾ ಫಿರೆ ಚಾಯ, ತಾರ ಸಿಂಧು ಬಿಂದು ಹೊಯೆಜಾಯ್
\end{myquote}

\versenum{॥ ೨೩~॥}

\begin{myquote}
ಬ್ರಹ್ಮ ಹೊತೆ ಕೀಟ ಪರಮಾಣು ಸರ್ವಭೂತೆ ಸೇಯಿ ಪ್ರೇಮಮಯ\\ಮನ ಪ್ರಾಣ ಶರೀರ ಅರ್ಪಣ ಕೊರೊ ಸಖೆ ಯೇ ಸಬಾರ ಪಾಯ
\end{myquote}

\versenum{॥ ೨೪~॥}

\begin{myquote}
ಬಹುರೂಪೆ ಸಂಮುಖೆ, ತೊಮಾರ, ಛಾಡಿ ಕೂಥಾ ಖುಂಜಿಛೊ ಈಶ್ವರ\\ಜೀವೆ ಪ್ರೇಮ ಕೊರೊ ಯೇಯಿ ಜನ ಸೇಯಿ ಜನ ಸೇವಿಛೆ ಈಶ್ವರ
\end{myquote}

\versenum{॥ ೨೫~॥}

\begin{center}
\textbf{ಓ, ನನ್ನ ಸಖನೆ!}
\end{center}

'ಸಖಾರ್ ಪ್ರತಿ' ಎಂಬ ಬಂಗಾಳಿ ಮೂಲದ ಈ ಕವನವನ್ನು ಸ್ವಾಮಿಜಿಯವರು ಬರೆದದ್ದು ೧೮೯೮ರ ಡಿಸೆಂಬರ್ ಅಥವಾ ೧೮೯೯ರ ಜನವರಿಯಲ್ಲಿ, ಬಹುಶಃ ದೇವಘಡದಲ್ಲಿದ್ದಾಗ. ಇದು ಮೊದಲು "ಉದ್ಭೋಧನ" ಪತ್ರಿಕೆಯಲ್ಲಿ ಪ್ರಕಟವಾಯಿತು (ಸಂಪುಟ: ೧, ಸಂಚಿಕೆ: ೨).

ತಮ್ಮ ವೈಯಕ್ತಿಕ ಜೀವನಾನುಭವದ ಹಿನ್ನೆಲೆಯಲ್ಲಿ ಸ್ವಾಮಿಜಿಯವರು ಜಗತ್ತಿನ ಕಠೋರ ಸ್ವರೂಪವನ್ನು ಈ ಕವನದಲ್ಲಿ ಹಿಡಿದಿಟ್ಟಿದ್ದಾರೆ. ನಿಃಸ್ವಾರ್ಥ ಪ್ರೇಮದ ಮೇಲಿನ ಸುಂದರ ಭಾಷ್ಯವೆಂದೂ ಈ ಕವನವನ್ನು ಗುರುತಿಸಬಹುದು.

\begin{myquote}
ಕತ್ತಲೆಯ ಬೆಳಕೆಂದು, ಸಂಕಟವ ಸವಿಯೆಂದು,\\ರೋಗವನ್ನು ಆರೋಗ್ಯಭಾಗ್ಯವೆಂದು,\\ನವಜಾತ ಶಿಶು ತಾನು ರೋದಿಸಿರೆ ಕಂಡದನು\\ಜೀವಂತಿಕೆಯ ಸಾಕ್ಷಿ ಮಾತ್ರವೆಂದು\\ಕಾಣುವೆಡೆಯೊಳು ನೀನು, ಓ ಜಾಣನೇ ಹೇಳು,\\ಎಂತು ಸುಖವನು ತಾನೆ ಪಡೆಯಲಹುದು?
\end{myquote}

\begin{myquote}
ಯುದ್ಧ –ಸ್ಪರ್ಧೆಗಳಿಲ್ಲಿ ಸಂತತವು ಸಾಗಿರಲು,\\ತಂದೆಯೇ ಮಗನಿಗೆದುರಾಗಿ ನಿಂದಿರಲು,\\ನಾನೆಂಬ ಸ್ವಾರ್ಥಶ್ರುತಿ ಎಡೆಬಿಡದೆ ಮಿಡಿದಿರಲು,\\ಪರಮಶಾಂತಿಯಲ್ಲಿ ಅರಸಬಹುದೆ? \versenum{೧೦}
\end{myquote}

\begin{myquote}
ನಾಕನರಕಗಳೆಲ್ಲ ಬೆಸೆದುಕೊಂಡಿಹುದಿಲ್ಲಿ,\\ಸಂಸಾರಮಾಯೆಯನು ಮೀರ್ವರಾರು?\\ಕರ್ಮಗಳ ಪಾಶವದು ಕುತ್ತಿಗೆಯ ಬಿಗಿಯುತಿರೆ\\ಜೀತದಾಳಿವನೆಲ್ಲಿ ಓಡಬಹುದು?
\end{myquote}

\begin{myquote}
ಯೋಗಗಳು, ಭೋಗಗಳು, ಗಾರ್ಹಸ್ಥ್ಯ, ಸಂನ್ಯಾಸ,\\ಭಕ್ತಿ–ಪೂಜೆಯು ಮತ್ತೆ ಐಶ್ವರ್ಯವು,\\ತ್ಯಾಗ–ವ್ರತ, ಕಠಿಣತಪಗಳನೆಲ್ಲ ನೋಡಿರುವೆ,\\ಇದರಿಂದ ಕಡೆಗೆ ನಾ ಕಂಡುದೇನು?–
\end{myquote}

\begin{myquote}
ಎಳ್ಳನಿತು ಸುಖವಿಲ್ಲ ಜಗದೊಡಲಿನಾಳದಲಿ,\\ದೇಹಧಾರಣೆಯೊಂದು ಬರಿಯ ಅಣಕ; \versenum{೩೦}\\ಎನಿತೆನಿತು ಹೃದಯವಂತಿಕೆಯು ನಿನ್ನೊಳಗಿಹುದೊ\\ಅನಿತನಿತು ದುಃಖಸಂತಪ್ತ ನೀನು!
\end{myquote}

\begin{myquote}
ನಿಃಸ್ವಾರ್ಥಪ್ರೇಮಿಗೀ ಜಗದೊಳಗೆ ಎಡೆಯಿಲ್ಲ\\ಕಬ್ಬಿಣವು ತಡೆಯುವತಿ ಭೀಕರದ ಪೆಟ್ಟುಗಳ\\ಹಾಲುಗಲ್ಲಿನ ಮೂರ್ತಿ ತಾಳಲಹುದೆ?\\ಕಲ್ಲೆದೆಯು, ನೀಚತನ,\\ಮಾತಿನಲಿ ಜೇನಹನಿ, ಬಗೆಯೊಳಗೆ ನಂಜು,\\ಸತ್ಯಕ್ಕೆ ಸುವಿದಾಯ, ಸ್ವಾರ್ಥದಾರಾಧನೆಯು–\\ಅಂಥವನ ಕೈಬೀಸಿ ಕರೆವುದೀ ಜಗವು!
\end{myquote}

\begin{myquote}
ಜ್ಞಾನಕ್ಷುಧೆಯನು ತಣಿಸೆ ಪಣತೊಟ್ಟು ಜೀವನದ \versenum{೩೦}\\ಆಯುಷ್ಯವರ್ಧವನೆ ತೇದೆ ನಾನು;\\ಪ್ರೀತಿಜಲವನ್ನರಸಿ, ಬಾಯಾರಿ, ಮರುಳನೊಲು\\ಸತ್ತ ನೆರಳುಗಳನ್ನೆ ಹಿಡಿದೆ ನಾನು.
\end{myquote}

\begin{myquote}
ಧರ್ಮವನ್ನರಸುತಲಿ ಎನಿತೊ ಮತಗಳ ಸಾರ್ದೆ,\\ಗುಹೆಗಳಲಿ, ಮಸಣದಲಿ ತೊಳಲುತಿದ್ದೆ;\\ಪುಣ್ಯನದಿಗಳ ಮಿಂದೆ, ಭಿಕ್ಷಾರ್ಥಿಯಾದೆ ನಾ,\\ಕಳೆಕಳೆದು ಹೋಯಿತಹ, ದಿನಗಳೆನಿತೊ!
\end{myquote}

\begin{myquote}
ಅಸಹಾಯಕನು ನಾನು, ಹರಕು ಚಿಂದಿಯನುಟ್ಟು,\\ವಿಧಿ ತರುವುದನೆ ನೆಮ್ಮಿ,\\ಮನೆಮನೆಯ ಬಾಗಿಲಿಗೆ ಅಲೆಯುತಿದ್ದೆ;\\ತಪದ ಭಾರಕೆ ದೇಹ ಕುಸಿದು ನೆಲ ಕಚ್ಚಿತ್ತು\\'ಭಾಗ್ಯವೇನಿದರಿಂದೆ?' ಎನುವೆ ನೀನು.
\end{myquote}

\begin{myquote}
ಕೇಳು, ಗೆಳೆಯನೆ, ನನ್ನ ಬಗೆಯ ಬಿಚ್ಚಿಡುತಿರುವೆ,\\ನಾ ಕಂಡ ಪರಮಸತ್ಯದ ಬೆಳಕಿದು\\ಅಲೆಗಳಲಿ ಬೆಂಡಾಗಿ, ಸುಳಿಗೆ ಕಂಗಾಲಾಗಿ\\ಕುಸಿದ ಜೀವನಕೊಂದು ಮುಕ್ತಿಯಿಹುದು:\\ಭವದ ಪಾರಕೆ ಒಯ್ವ ನಾವೆಯಿಹುದು.
\end{myquote}

\begin{myquote}
ಪೂಜೆಗಳೊ ಮಂತ್ರಗಳೊ ಯಮ–ನಿಯಮ ತಂತ್ರಗಳೊ\\ವಿಜ್ಞಾನ–ವೇದಾಂತ–ಷಟ್‌ಶಾಸ್ತ್ರವೊ\\ತ್ಯಾಗವೋ ಭೋಗವೋ ಬರಿಯ ವಿಭ್ರಮೆಗಳವು,\\ಪ್ರೇಮವೊಂದೇ ಪರಮಸತ್ಯ ನಿಧಿಯು!
\end{myquote}

\begin{myquote}
ಜೀವನಲಿ ಬ್ರಹ್ಮದಲಿ ಮನುಜನಲಿ ದೇವನಲಿ\\ಭೂತಪ್ರೇತಗಳಲ್ಲಿ ದೇವಗಣದಿ\\ಪಶುಪಕ್ಷಿ ಕೀಟದಲಿ ಹೃದಯಾಂತರಾಳದಲಿ\\
 ಪ್ರೇಮವೊಂದೇ ನೆಲಸಿ ಬೆಳಗುತಿಹುದು.
\end{myquote}

\begin{myquote}
ದೇವದೇವರಿಗೆಲ್ಲ ದೇವನಾವನು, ಹೇಳು,\\ಸಕಲ ಸಚರಾಚವ ಚಲಿಪನಾರು?\\ಶಿಶುವಿಗೋಸುಗ ತಾಯಿ ಪ್ರಾಣವನೆ ತೊರೆಯುವಳು,\\ಕಳ್ಳ ತಾನೆಳಸುವನು ಕಳ್ಳತನಕೆ\\ಇರ್ವರೆದೆಯೊಳು ತುಡಿವ ಪ್ರೇಮವೊಂದೇ!~।
\end{myquote}

\begin{myquote}
ಮಾತುಮನಗಳನೆಲ್ಲ ಮೀರಿರುವ ತಾನದುವೆ\\ದುಃಖಸಂಕಟಗಳಲು ನೆಲಸಿರುವುದು;\\ಮೃತ್ಯುರೂಪವ ತಾಳಿ ಕಾಳಿಯೆಂದೆನಿಸುವುದು,\\ತಾಯೊಲುಮೆ ತಾನಾಗಿ ಮಧು ಸುರಿವುದು.
\end{myquote}

\begin{myquote}
ಶೋಕ–ದಾರಿದ್ರ್ಯಗಳು, ದುಃಖ–ಧರ್ಮಾಧರ್ಮ,\\ಒಳಿತು–ಕೆಡುಕಿನ ಫಲದ ರೂಪದಲ್ಲಿ\\ಪ್ರೇಮದಾರಾಧನೆಯೆ ಬಗೆಬಗೆಯೊಳೆಸೆದಿಹುದು\\ತನ್ನ ಬಲವೊಂದನ್ನೆ ನೆಮ್ಮಿ ಜೀವನು ತಾನೆ\\ಏನನೆಸಗುವನಿಲ್ಲಿ ಲೋಕದಲ್ಲಿ?
\end{myquote}

\begin{myquote}
ಸುಖವನೆಳಸುವನವನು ವಿಭ್ರಾಂತನಾಗಿಹನು,\\ದುಃಖವನ್ನರಸುವನು ಅತಿ ಮೂಢನು;\\ಮೃತ್ಯುವಿಗೆ ಕಾತರನು ಅತಿಮರುಳನಾಗಿರಲು,\\ಅಮೃತತ್ವದಾಕಾಂಕ್ಷೆ ಬಯಲಗಾಳಿ!
\end{myquote}

\begin{myquote}
ಬುದ್ಧಿ ರಥವನ್ನೇರಿ ದೂರದೂರಕೆ ಸಾರಿ\\ಅಲೆದಲೆದು ಬಂದರೂ ಇರುವುದೇನು?\\ಸಂಸಾರಸಾಗರವೆ ಮುತ್ತಿಹುದು ಸುತ್ತಲೂ\\ಸುಳಿವ ಸುಖದುಃಖಸುಳಿ ಕಾಣದೇನು?
\end{myquote}

\begin{myquote}
ರೆಕ್ಕೆಗಳ ಕಳಕೊಂಡ ಬಾನಾಡಿಯೇ, ಕೇಳು,\\ನಿನ್ನ ಬಿಡುಗಡೆಗಿದು ದಾರಿಯಲ್ಲ;\\ಬಾರಿಬಾರಿಗು ಪೆಟ್ಟು ತಿಂದು ಕುಸಿಯುವೆ ನೀನು,\\ವ್ಯರ್ಥಯತ್ನವನೇಕೆ ಮಾಡುತಿರುವೆ?
\end{myquote}

\begin{myquote}
ಜ್ಞಾನವನ್ನು ವಿದ್ಯೆಯನು ಯಜ್ಞ–ಜಪ–ತಪಗಳನು\\ತೊರೆ ನೀನು; ನಿಃಸ್ವಾರ್ಥ ಪ್ರೇಮವೊಂದೇ ನಿನಗೆ\\ಪ್ರಬಲ ಆಧಾರ.\\ಅಗ್ನಿಶಿಖೆಯನು ತಬ್ಬುತಿಹ ಪತಂಗವ ನೋಡು,\\ಅದುವೆ ಕಲಿಸುವುದಯ್ಯ ಸತ್ಯಸಾರ!
\end{myquote}

\begin{myquote}
ರೂಪಕ್ಕೆ ಮರುಳಾದ ಕುರುಡುಕೀಟವು ನೀನು,\\ಎದೆಯು ತುಳುಕಿದ ಪ್ರೇಮಪಾನದಲ್ಲಿ;\\ನಿಜದೊಲುಮೆಗೊಲಿದವನೆ, ನಿನ್ನ ಹಮ್ಮಿನ ಕೊಳಕ\\ಸುಟ್ಟುಬಿಡು ನಿಸ್ವಾರ್ಥದಗ್ನಿಯಲ್ಲಿ!
\end{myquote}

\begin{myquote}
ಭಿಕ್ಷುಕನಿಗುಂಟೆ ಸುಖ? ದೈನ್ಯವನ್ನು ತೊರೆಯೊ ನೀ,\\ಪ್ರತಿಫಲವ ಬಯಸದಲೆ ನೀಡು, ನೀಡು;\\ಆಗುವುದು ನಿನ್ನೆದೆಯು ನಿಧಿಯ ಗೂಡು!
\end{myquote}

\begin{myquote}
ನಿಧಿಯನಂತಕೆ ನೀನು ಅಧಿಕಾರಿಯಾಗಿರುವೆ,\\ಪ್ರೇಮವಾರಿಧಿಯಿಹುದು ನಿನ್ನೆದೆಯಲಿ;\\ತಡೆಯದಲೆ ನೀ ನೀಡು, ಮರಳಿ ನೀ ಬಯಸಿದೊಡೆ\\ಸಿಂಧುವೂ ಕರಗುವುದು ಬಿಂದುವಿನಲಿ!
\end{myquote}

\begin{myquote}
ಬಿತ್ತರದ ಬ್ರಹ್ಮದಲಿ, ಹೊರಳುತಿಹ ಕೀಟದಲಿ,\\ಕಣಕಣದಿ ತುಂಬಿರುವ ಅಣುವಿನಲ್ಲಿ\\ಪ್ರೇಮರೂಪನು ತಾನೆ ಹರಡಿ ಹಬ್ಬಿರುವನು–\\ಅರ್ಪಿಸೆಲ್ಲವನವನ ಪದತಲದಲಿ!
\end{myquote}

\begin{myquote}
ಬಗೆಬಗೆಯ ರೂಪದಲಿ ಜಗವ ತುಂಬಿಹನವನು,\\ಬರಿದೆ ಹುಡುಕುವೆಯವನ ಅಲ್ಲಿ–ಇಲ್ಲಿ;\\ಬಗೆಯೊಳೊಲುಮೆಯ ತುಂಬಿ ಭೇದವಿಲ್ಲದೆ ಕೊಟ್ಟ–\\ರದುವೆ ಆತನ ಪೂಜೆ ಸತ್ಯದಲ್ಲಿ!
\end{myquote}

\begin{center}
\textbf{ನಾಚುಕ್ ತಾಹಾತೆ ಶ್ಯಾಮಾ}\footnote{\engfoot{C.W, Vol. IV, P. 506}}
\end{center}

\begin{myquote}
ಪುಲ್ಲ ಪೂಲ್ ಸೌರಭೆ ಆಕುಲ ಮತ್ತ ಅಲಿಕುಲ ಗುಲಿಜರಿಛೆ ಆಶೆ ಪಾಶೆ\\ಶುಭ್ರ ಶಶಿ ಯೆನ ಹುಸಿರಾಶಿ ಯತ ಸ್ವರ್ಗವಾಸಿ ವಿತರಿಛೆ ಧರವಾಸೆ
\end{myquote}

\versenum{॥ ೧~॥}

\begin{myquote}
ಮೃದ ಮಂದ ಮಲಯ ಪವನ ಯಾರ್ ಪರಶನ ಸ್ಮೃತಿಪಠೆ ದೇಯ್ ಖುಲೆ\\ನದಿ ನದ ಸರಸಿ ಹಿಲ್ಲೋಲ ಭ್ರಮರ ಚಂಚಲ ಕತವಾ ಕಮಲ ದೊಲೆ
\end{myquote}

\versenum{॥ ೨~॥}

\begin{myquote}
ಫೇನಮಯಿ ಝರೆ ನಿರ್ಜರಿಣಿ ತಾನತರಂಗಿಣಿ ಗುಹಾ ದೇಯ ಪ್ರತಿಧ್ವನಿ\\ಸ್ವರಮಯ ಪತತ್ರಿನಿಚಯ ಲುಕಾಯೆ ಪಾತಾಯ್ ಶುನಾಯ್ ಸೊಹಾಗ್ ವಾಣಿ
\end{myquote}

\versenum{॥ ೩~॥}

\begin{myquote}
ಚಿತ್ರಕರ ತರುಣಭಾಸ್ಕರ ಸ್ವರ್ಣತುಲಿಕರ ಛೋಯಾ ಮಾತ್ರ ಧರಾಪಟೆ\\ವರ್ಣಖೇಲಾ ಧರಾತಲ ಛಾಯ ರಾಗಪರಿಚಯ ಭಾವರಾಶಿ ಜೇಗೆ ಉಠೆ
\end{myquote}

\versenum{॥ ೪~॥}

\begin{myquote}
ಮೇಘಮಂದ್ರ ಕುಲಿಶ ನಿಸ್ವನ ಮಹಾರಣ ಭೂಲೋಕ ಧ್ಯೂಲೋಕ ವ್ಯಾಪಿ\\ಅಂಧಕಾರ್ ಉಗರೆ ಆಧಾರ ಹುಹುಂಕುರ ಸ್ವಶಿ ಛ ಪ್ರಲಯವಾಯು
\end{myquote}

\versenum{॥ ೫~॥}

\begin{myquote}
ಝಲಕಿ ಝಲಕಿ ತಹೆ ಭಾಯ ರಕ್ತಕಾಯ ಕರಾಲ ಬಿಜಲಿಜ್ವಾಲಾ\\ಫೇನಮಯ ಗರ್ಜಿ ಮಹಾಕಾಯ ಊರ್ಮಿದಾಯ ಲಂಘಿತೆ ಪರ್ವತಚೂಡಾ
\end{myquote}

\versenum{॥ ೬~॥}

\begin{myquote}
ಘೋಶ ಭೀಮ ಗಂಭೀರ ಭೂತಲ ಟಲಮಲ ರಸಾತಲ ಜಾಯ ಧರಾ\\ಪೃಥ್ವಿಛ್ಛೇದಿ ಉಟೆಚೆ ಅನಲ ಮಹಾಚಲ ಚೂರ್ಣ ಹಯೆ ಜಾಯ ವೇಗೆ
\end{myquote}

\versenum{॥ ೭~॥}

\begin{myquote}
ಶೋಭಾಮಯ ಮಂದಿರ ಆಲಯ ಹ್ರದನೀಲ ಪಯತಹೆ ಕುವಲಯ ಶ್ರೇಣಿ\\ದ್ರಾಕ್ಷಾಫಲ ಹೃದಯರುಧಿರ, ಫೇನಶುಭ್ರಶಿರ ಬಲೆ ಮೃದುಮೃದುವಾಣಿ
\end{myquote}

\versenum{॥ ೮~॥}

\begin{myquote}
ಶ್ರುತಿ ಪಥೆ ವೀಣಾರವ ಝಂಕಾರ ವಾಸನಾ ವಿಸ್ತಾರ ರಾಗ ಕಾಲಮಾನ ಲಯೆ\\ಕತಮತ ಬ್ರಜೇರ್‌ ಉಚ್ಛ್ವಾಸ, ಗೋಪಿ ತಪ್ತಶ್ವಾಸ, ಅಶ್ರುರಾಷಿ ಪಡೆ ಬಯೆ
\end{myquote}

\versenum{॥ ೯~॥}

\begin{myquote}
ಬಿಂಬ ಪಲ ಯುವತಿ ಅಧರ ಭಾವೇರಸಾಗರ ನೀಲೋತ್ಪಲ ದುಟಿ ಅಂಕಿ\\ದುಟಿಕರ–ಬಂಚಾ ಅಗ್ರಸರ ಪ್ರೇಮೇರ ಪಿಂಜರ ತಾಹೇ ಬಾದಾ ಪ್ರಾಣಪಾಖೀ
\end{myquote}

\versenum{॥ ೧೦~॥}

\begin{myquote}
ಡಾಕೇ ಭೇರೆ ಬಾಜೆ ಝರ್ ಝರ್, ದಮಾನು ನಕ್ಕಾಡ್ ವೀರದಾಪೆ ಕಾಪೇಧರಾ\\ಘೋಶೆತೋಪ ಬಬ–ಬಬ–ಬಮ್, ಬಬ–ಬಬ–ಬಮ್ ಬಂದೂಕೇರ್ ಕಡಕಡ
\end{myquote}

\versenum{॥ ೧೧~॥}

\begin{myquote}
ಧೂಮೆ ಧೂಮೆ ಭೀಮ ರಣಸ್ಥಲ, ಗರಜಿ ಅನಲ ವಾಮಿ ಶತ ಜ್ವಾಲಾಮುಖಿ\\ಫಾಟಿಗೋಲಾ ಲಾಗೆ ವುಕಿಗಾಯ ಕೊಥಾ ಉಡೇ ಜಾಯ್ ಆಸೋಯಾರ ಗೋಢಾಹಾತಿ
\end{myquote}

\versenum{॥ ೧೨~॥}

\begin{myquote}
ಪೃಥ್ವಿತಲ ಕಾಂಪೆ ಥರಥರ, ಲಕ್ಷ ಅಶ್ವಬಲ ಪುಷ್ಪೆ ವೀರ ಝಾಕಿರಣೆ\\ಭೇದಿ ಧೂಮ ಗೋಲಾಬರಿಶಣ, ಗುಲಿಸ್ವನಸ್ವನ, ಶತೃತೋಪ್ ಆನೆಚಿನೆ
\end{myquote}

\versenum{॥ ೧೩~॥}

\begin{myquote}
ಅಗೆಜಾಯ್ ವೀರ್ಯ–ಪರಿಚಯ ಪತಾಕ–ನಿಚಯ ದಂಡೆ ಝರೆ ರಕ್ತಧಾರಾ\\ಸಂಗೆ ಸಂಗೆ ಪದಾತಿದಲ ಬಂದೂಕ ಪ್ರಬಲ ವೀರಮದ ಮಾತೂ ಯಾರಾ
\end{myquote}

\versenum{॥ ೧೪~॥}

\begin{myquote}
ಐಪಡೆ ವೀರಧ್ವಜದಾರಿ, ಅನ್ಯವೀರ ತಾರಿಧ್ವಜ ಲಯೆ ಆಗೆ ಚಲೆ\\ತಲೆ ತಾರ ಢೇರ ಹಯೆ ಜಾಯ ಮೃತವೀರಕಾಯ ತವು ಪಿಚೆ ನಹಿ ಟೊಲೆ
\end{myquote}

\versenum{॥ ೧೫~॥}

\begin{myquote}
ದೇಹ ಚಾಯ್ ಸುಖೇರ್ ಸಂಗಮ್, ಚಿತ್ತವಿಹಂಗಂ ಸಂಗೀತ–ಸುಧಾರ ಧಾರ\\ಮನ ಚಾಯ ಹಸಿರ ಹಿಂದೋಲ್ ಪ್ರಾಣಸದಾಲೋಲ ಜಾಯಿತ ದುಃಖೇರ್‌ಪಾರ
\end{myquote}

\versenum{॥ ೧೬~॥}

\begin{myquote}
ಛಾಡಿ ಹಿಮ ಶಶಾಂಕ ಚ್ಹಡಾಯ್ ಕೇಬಲ ಚಾಯು ಮಧ್ಯಾಹ್ನ ತಪನಜ್ವಾಲ\\ಪ್ರಾಣಜಾರ ಚಂಡ ದಿವಾಕರ, ಸ್ನಿಗ್ಧ ಶಶಧರ, ಸೇವೋತಬುಲಾಗೆ ಭಾಲೊ
\end{myquote}

\versenum{॥ ೧೭~॥}

\begin{myquote}
ಸುಖತರ ಸಬಾಈ ಕಾತರ ಕೇವಾಸೆ ಪಾಮರ ದುಃಖೆ ಜಾರ ಬಾಲಭಾಸಾ\\ಸುಖೆ ದುಃಖೆ ಅಮೃತೆಗರಳ ಕಂಠೆ ಹಲಾಹಲ ತಬುನಹಿ ಛಾಡೆ ಆಶಾ
\end{myquote}

\versenum{॥ ೧೮~॥}

\begin{myquote}
ರುದ್ರಮುಖೆ ಸಬಾಯಿ ಡೋರಾಯ ಕೇಹನಹಿ ಚಾಯ ಮೃತ್ಯುರೂಪಾ ಎ ಐಲೋಕೇಶೀ\\ಉಷ್ಣಂಭಾರ ರುಧೀರ ಉದ್ಗಾರ, ಭೀಮತರವಾದ ಖಸಾಯಿಯೆ ದೇಯ ಬಾಂಶಿ
\end{myquote}

\versenum{॥ ೧೯~॥}

\begin{myquote}
ಸತ್ಯತುಮಿ ಮೃತ್ಯುರೂಪಾಕಾಳಿ ಸುಖವನಮಾಲಿ ತುಮಾರ್‌ ಮಾಯಾರ ಛಾಯ\\ಕರಾಳಿನ ಕರ ಮರ್ಮಚ್ಛೇದ ಹೋಕ ಮಾಯಾ ಭೇದ ಸುಖಸ್ವಪ್ನ ದೇಹ ದಯಾ
\end{myquote}

\versenum{॥ ೨೦~॥}

\begin{myquote}
ಮುಂಡಮಾಲ, ಪರಾಯೆ, ತುಮಾಯ, ಭಯ, ಫಿರೆಚಾಯ ನಾಮದೇಯ ದಯಾಮಯಿ\\ಪ್ರಾಣಕಾಫೆ ಭೀಮ ಅಟ್ಟಹಾಸ, ನಗ್ನದಿಕ್‌ವಾಸ ಬಲೆ ಮಾ ದಾನವ ಜಿಯೀ
\end{myquote}

\versenum{॥ ೨೧~॥}

\begin{myquote}
ಮುಖೆಬೋಲೆ ದೇಖೀಬೆ ತುಮಾಯ ಅಸಿಲೆ ಸಮಯಕೊಥಾ ಜಾಯಕೇಯಜಾನೆ~॥\\ಮೃತ್ಯುತಮಿ, ರೋಗ ಮಹಾಮಾರಿ ವಿಶಕುಂಬ ಭರೀ ವಿತರಿಛ ಜನೆ ಜನೆ
\end{myquote}

\versenum{॥ ೨೨~॥}

\begin{myquote}
ರೇ ಉನ್ಮಾದ, ಅಪನಬುಲಾವೊ ಫಿರೆನಯಿ, ಚಾನೊ ಪಾಛೆ ದೇಖ, ಭಯಂಕರ\\ದುಃಖಚಾವೋ, ಸುಖ ಹಬೆ ಬೋಲೆ, ಭಕ್ತಿ ಪೂಜಾಶ್ಚಲೆ ಸ್ವಾರ್ಥಸಿದ್ಧಿ ಮನೆ ಭರಾ
\end{myquote}

\versenum{॥ ೨೩~॥}

\begin{myquote}
ಛಾಗಕಂಠ ರುಧೀರೇರ ಧಾರ ಭಯೆರ್ ಸಂಚಾತ ದೇಖೆ ತೋರ ಹಿಯಾ ಕಾಂಪೆ\\ಕಾಪುರುಷ್~। ದಯಾರಾಥಾರ್, ಧನ್ಯವ್ಯವಹಾರ ಮರ್ಮ ಕಥಾ ಬಲಿ ಕಾಕೆ?
\end{myquote}

\versenum{॥ ೨೪~॥}

\begin{myquote}
ಭಂಗವೀಣ ಪ್ರೇಮಸುಧಾಪಾನ ಮಹಾ ಆಕರ್ಷಣ ದೂರಕರ ನಾರೀ ಮಾಯಾ\\ಅಗುವಾನ್, ಸಿಂಧು ರೋಲೆ ಗಾನ, ಅಶ್ರುಜಲಪಾನ, ಪ್ರಾಣಪನ ಯಾಕ್ ಕಾಯಾ
\end{myquote}

\versenum{॥ ೨೫~॥}

\begin{myquote}
ಜಾಗೊ ವೀರ, ಘೂಚಾಯೆ ಸ್ವಪನ, ಶಿಯಾರೆ ಶಮನ, ಭಯಕಿ ತುಮಾರ ಸಾಜೆ~।\\ದುಃಖಭಾರ ಯೇ ಭವ ಈಶ್ವರ ಮಂದಿರ ತಾಹಾರ ಪ್ರೇತಭೂಮಿ ಚಿತಾಮಾಝೆ
\end{myquote}

\versenum{॥ ೨೬~॥}

\begin{myquote}
ಪೂಜಾ ತಾರ ಸಂಗ್ರಾಮ ಅಪಾರ ಸದಾಪರಾಜಯ ತಹನಾ ಡರಾಕ ತೋಮಾ\\ಚೂರ್ಣ ಹೋಕ್ ಸ್ವಾರ್ಥ ಸಾಧಮಾನ ಹೃದಯಸ್ಮಶಾನ ನಾಚುಕ್ ತಾಹತೆ ಶ್ಯಾಮಾ
\end{myquote}

\versenum{॥ ೨೭~॥}

\begin{center}
\textbf{ಶ್ಯಾಮೆಯು ನರ್ತಿಸಲಿ!}
\end{center}

ಇದರ ಮೂಲ ನಾಚುಕ್ ತಾಹಾತೇ ಶ್ಯಾಮಾ' ಎಂಬ ಹೆಸರಿನಲ್ಲಿರುವ ಬಂಗಾಲಿ ಕವನ; ಇಪ್ಪತ್ತೇಳು ನುಡಿಗಳಿಂದ ಕೂಡಿದೆ. ಇದನ್ನು ಇಲ್ಲಿ ಭಾವಕ್ಕೆ ಅನುಗುಣವಾಗಿ ಎಂಟು ಭಾಗಗಳನ್ನಾಗಿ ವಿಂಗಡಿಸಿಕೊಂಡಿದೆ.

ಮೂಲ ಬಂಗಾಳಿ ಕವನ 'ಉದ್ಭೋಧನ' ಪತ್ರಿಕೆಯಲ್ಲಿ ಪ್ರಕಟವಾಯಿತು (ಸಂಪುಟ: ೨, ಸಂಚಿಕೆ:೧). ಇದನ್ನು ಸ್ವಾಮಿಜಿಯವರು ರಚಿಸಿದ್ದು ಬಹುಶಃ ೧೮೯೯ರ ಜುಲೈ ತಿಂಗಳಿನ ೧೩ ಹಾಗೂ ೧೯ನೆಯ ದಿನಾಂಕಗಳ ನಡುವೆ, ಹಡಗಿನಲ್ಲಿ ಕೆಂಪು ಸಮುದ್ರದ ಮೂಲಕವಾಗಿ ಪ್ರಯಾಣಿಸುತ್ತಿದ್ದಾಗ.

'ತಾಯಿ ಕಾಳಿ' ಕವನದಲ್ಲಿ ಮೈದಾಳಿರುವ ಭಾವವೇ ಇಲ್ಲಿ ಮತ್ತೊಂದು ಬಗೆಯಲ್ಲಿ ವ್ಯಕ್ತಗೊಂಡಿರುವುದನ್ನು ಕಾಣುತ್ತೇವೆ. 'ರೌದ್ರದ ಆರಾಧನೆ'ಯೇ ಇಲ್ಲಿಯೂ ಕೇಂದ್ರವಸ್ತುವಾಗಿದೆ. ಇದನ್ನು ಹಲವಾರು ದೃಶ್ಯಚಿತ್ರಗಳ ಮೂಲಕ ಕವನ ನಾಟಕೀಯವಾಗಿ ಹಿಡಿದಿಟ್ಟಿದೆ. ಕೋಮಲತೆ ಹಾಗೂ ರೌದ್ರತೆಗಳ ವೈಸಾದೃಶ್ಯವನ್ನು ಇಲ್ಲಿ ಪ್ರಧಾನವಾಗಿ ಕಾಣಬಹುದಾಗಿದೆ.

ಭಾಗ ೧: ಮಾಧುರ್ಯವೇ ರೂಪುದಾಳಿದಂತಿರುವ ನಿಸರ್ಗದ ರಮಣೀಯತೆ ಇಲ್ಲಿದೆ.

ಭಾಗ ೨: ನಿಸರ್ಗ ರಮಣೀಯ ಮಾತ್ರವೇ ಅಲ್ಲ, ಭೀಷಣವೂ ಹೌದು. ಈ ಭೀಕರತೆಯನ್ನು ಇಲ್ಲಿ ಚಿತ್ರಿಸಿದೆ.

ಭಾಗ ೩: ಮನುಷ್ಯನ ಮನಸ್ಸನ್ನು ಸಹಜವಾಗಿಯೇ ಸೆಳೆಯುವ ಮೋಹಕ ಸನ್ನಿವೇಶ ಹಾಗೂ ಅದರಿಂದ ಉದ್ದೀಪನಗೊಳ್ಳುವ ಭಾವತೀವ್ರತೆ ಇಲ್ಲಿ ನಿರೂಪಿತವಾಗಿದೆ.

ಭಾಗ ೪: ರಣರಂಗದ ಸನ್ನಿವೇಶವನ್ನು ಇಲ್ಲಿ ಕ್ರಿಯಾತ್ಮಕವಾಗಿ ಹಿಡಿದಿಡಲಾಗಿದೆ. ಇದು ಬಹಿರಂಗದ ರಾಜ್ಯ ವಿಸ್ತರಣೆಯ ರಣರಂಗ ಮಾತ್ರವಲ್ಲ, ಅಂತರಂಗದ ಹೋರಾಟದ ಭೂಮಿಕೆಯೂ ಹೌದು. ಸ್ವಾಮಿಜಿ ಮಾನವನ ಅಸ್ತಿತ್ವವನ್ನು ಗುರುತಿಸುವುದೇ ಹೋರಾಟದ ನೆಲೆಯಲ್ಲಿ. ಅವರ ಪ್ರಕಾರ ತಾಟಸ್ಥ್ಯವೆನ್ನುವುದು ಮೃತ್ಯುವಿಗೆ ಸಮ.

ಭಾಗ ೫: ಸಹಜ ಆಕರ್ಷಣೆಗಳಿಗೆ ಸೋಲುವ ಮಾನವಸ್ವಭಾವದ ಚಿತ್ರಣ. ದುಃಖದ ಅನಿವಾರ್ಯ ಪರಿಣಾಮವನ್ನು ಅರಿತೂ ಸುಖವನ್ನು ತಿರಸ್ಕರಿಸಲಾಗದ ಪರಿಸ್ಥಿತಿ.

ಭಾಗ ೬: ಮೃತ್ಯುವನ್ನು ಮೃತ್ಯುವಾಗಿಯೇ, ರೌದ್ರವನ್ನು ರೌದ್ರವಾಗಿಯೇ ನೋಡಬೇಕೆಂಬ ಸತ್ಯದ ದಿಟ್ಟತನ.

ಭಾಗ ೭: ಮಾಧುರ್ಯದ ಮಸಣ ಮಾಡಿ ರೌದ್ರವನ್ನು ಆಹ್ವಾನಿಸಿದಾಗಲೇ ಸತ್ಯದ ಆನಂದನರ್ತನ ಸಿದ್ಧಿಸುವುದೆಂಬ ಭಾವ.

\begin{center}
–೧–
\end{center}

ಕುಸುಮಗಳರಳಿರೆ ಸುಗಂಧ ಬೀರಿ\\ಮತ್ತ ತುಂಬಿಗಳು ಮುತ್ತಿಹವು;\\ಬೆಳ್ಳಿಯ ಚಂದ್ರನ ಹೂನಗೆಮಳೆಯನು\\ಸಗ್ಗಿಗರದೊ ತಿರೆಗೆರೆದಿಹರು.\\ಮಲಯಮಾರುತನ ಮಾಂತ್ರಿಕ ಸ್ಪರ್ಶದಿ\\ದೂರದ ಸ್ಮೃತಿಗಳ ಪದರ ಪದರಗಳ\\ಕಣ್ಣೆದುರಿಗೆ ತಂದರಳಿಸಿದೆ.\\ನದ–ನದಿ–ಸರಸಿಗಳವು ಮೊರೆಯುತಲಿರೆ\\ಅರಳಿದ ಸಾಸಿರ ಕಮಲವ ಸುತ್ತಿವೆ\\ಭ್ರಮರಾಳಿಗಳವು ಸಂಭ್ರಮದಿ.\\ನೊರೆನೊರೆಯಲೆಗಳ ತಾನ ತರಂಗಕೆ\\ಗಿರಿಗುಹೆ ಮಾರ್ದನಿಗುಡುತಲಿದೆ.\\ಮರಗಳ ಎಲೆವನೆ ಮರೆಯಿಂ ಖಗಕುಲ\\ಎದೆಯಾಳದ ಸೊಗದೊಲವನು ಚಿಮ್ಮಿದ\\ಸುಸ್ವರ ಸುಶ್ರುತ ಗಾನದಲಿ.\\ತರುಣ ಭಾಸ್ಕರನ ಹೊನ್ನಿನ ಕುಂಚವು\\ಧಾರಿಣಿ ಪಟವನು ತಾಕಿರೆ ಮೆಲ್ಲನೆ\\ಬಣ್ಣ ಬಣ್ಣಗಳು ಚಿಗಿತು ಚಿಮ್ಮುತಿವೆ\\ನಿಸರ್ಗಮಾತೆಯ ವಕ್ಷದಲಿ,\\ಭಾವಸಾಗರವ ಕಲಕುತಲಿ!

\begin{center}
–೨–
\end{center}

ಮೇಘ ಮೇಘಗಳು ಘಟ್ಟಿಸೆ ಗುಡುಗದು\\ಧರೆಯಾಗಸಗಳ ತುಂಬಿಹುದು.\\ಕತ್ತಲೆ ಕತ್ತಲನುಗುಳಿಹುದು.\\ಪ್ರಳಯಪ್ರಭಂಜನ ಹೂಂಕರಿಸಿಹುದು.\\ರುಧಿರಾರುಣವಹ ಮಿಂಚಿನ ಜ್ವಾಲೆಯು\\ಹೊಳೆಹೊಳೆಯುತಲಿರೆ ವ್ಯೋಮದಲಿ,\\ದೈತ್ಯತರಂಗಗಳೇಳುತ ಮೊರೆದಿವೆ\\ಗಿರಿಗಳ ದಾಂಟುವ ರಭಸದಲಿ!\\ಭೀಮಘೋಷದಲಿ ಗರ್ಜಿಸಿ, ಭೂತಲ\\ಪಾತಾಳಕೆ ಜಾರುತಲಿಹುದು.\\ನೆಲವ ಸೀಳಿ ಘನ ಜ್ವಾಲೆ ಸಿಡಿಯುತಿರೆ\\ಪರ್ವತ ಪುಡಿಪುಡಿಯಾಗಿಹುದು!

\begin{center}
–೩–
\end{center}

ಕಮಲದಿ ತುಂಬಿಹ ನೀಲ ಸರೋವರ\\ತೀರದೊಳೆಸೆದಿಹ ಭವನದೊಳಲ್ಲಿ\\ಹೃದಯರುಧಿರರಸ ಮದಿರೆಯುಕ್ಕುತಿದೆ\\ನೇಹದೊಳುಸುರಿದ ಪಿಸುದನಿಯಲ್ಲಿ;\\ವೀಣಾರವವದು ಕಿವಿಯನು ತುಂಬಿ\\ರಾಗತಾಳಲಯದಲೆಯಲೆಯಲ್ಲಿ\\ಆಶೆರಾಶಿಯನು ಕೆರಳಿಸಿದೆ.\\ತಪ್ತ ಗೋಪಿಯರ ನಿಟ್ಟುಸಿರೆನಿತೊ\\ಭಾವದಾಳವನು ಕಲಕಿರೆ ಹೃದಯವೆ\\ಹರಿದಿದೆ ಕಂಬನಿಧಾರೆಯಲಿ!...\\ಬಿಂಬಾಧರಗಳೊ ನೀಲನಯನಗಳೊ\\ಭಾವಸಾಗರವನುಕ್ಕಿಸಿದೆ.\\ಪ್ರೇಮಪಂಜರದಿ ಪ್ರಾಣಪಕ್ಷಿಯನು\\ಹಿಡಿದಿದೆ ತೋಳಿನ ಬಲೆಯಲ್ಲಿ!

\begin{center}
–೪–
\end{center}

ಭೇರಿಯು ಗುಡುಗಿರೆ, ಕಹಳೆಯು ಮೊಳಗಿರೆ,\\ಯೋಧರ ಪಾದಾಘಾತದಿ ಧಾರಿಣಿ\\ತತ್ತರಿಸುತಲಿರೆ ನಡನಡುಗಿ,\\ತೋಪು ಗುಂಡುಗಳ ಮಳೆಗರೆಯುತಲಿದೆ,\\ಕೋವಿಯುಗುಳುತಿದೆ ಬೆಂಕಿಯನು!

ಭೀಮರಣಸ್ಥಲ ಧೂಮಾವೃತದಲಿ\\ಶತ ಜ್ವಾಲಾಮುಖಿ ಸಿಡಿದಿಹ ತೆರದಲಿ\\ಅಸ್ತ್ರಗಳವು ಕಿಡಿಗರೆಯುತಿವೆ,\\ಮರ್ಮಾಘಾತದಿ ತಾಗುತಿವೆ,\\ಆನೆ–ಕುದುರೆ ಧೂಳೀಪಟವಾಗಿದೆ,\\ಭೂಮಿಯು ಥರಥರ ಬಿರಿಯುತಿದೆ!\\ಸಹಸ್ರ ಸಂಖ್ಯೆಯ ಅಶ್ವಾರೂಢರು,\\ಧೂಮವಲಯವನ್ನು ಭೇದಿಸಿ ನುಗ್ಗುವ,\\ಗುಂಡಿಗೆ ಮಣಿಯದ ಗುಂಡಿಗೆಯವರು\\ಅರಿಗಳ ತೋಪನು ಮುತ್ತಿಹರು!\\
 ನೆತ್ತರ ಧಾರೆಯ ವಿಜಯಧ್ವಜವದೊ\\ಮುಂದಕೆ ಮುಂದಕೆ ನುಗ್ಗುತಿದೆ;\\ವೀರೋನ್ಮಾದದಿ ಮತ್ತ ಯೋಧದಲ\\ಜೊತೆಜೊತೆಯಲ್ಲಿಯೆ ಸಾಗುತಿದೆ!

ವೀರಧ್ವಜವನ್ನು ಹಿಡಿದವನುರುಳಿದ!\\ಆದರೇನು, ಮತ್ತೊಬ್ಬನ ಹೆಗಲನು\\ಏರಿದ ಧ್ವಜ ಮುಂಬರಿದಿಹುದು!\\ಧ್ವಜಧಾರಿಯ ಕಾಲಡಿಯಲಿ ಹೆಣಗಳ\\ರಾಶಿರಾಶಿಗಳೆ ಸಿಕ್ಕಿದರೂ\\ದಟ್ಟಡಿಯಿಡುತಲಿ ಸಾಗಿಹನು!

\begin{center}
–೫–
\end{center}

ಸುಖಭೋಗಗಳಿಗೆ ಎಳಸಿದೆ ತನುವು,\\ಗಾನಸುಧೆಗೆ ಪರಿತಪಿಸಿದೆ ಮನವು,\\ಮೃದು ಮಾಧುರ್ಯಕೆ ಜೀವ ಮಿಡಿಯುತಿರೆ,\\ಎದೆಯು ಚಿಮ್ಮುತಿದೆ ದುಃಖದಾಚೆಗೆ!

ಶೀತಲ ಚಂದ್ರನ ತಂಪನು ತ್ಯಜಿಸುತ\\ನಡುಬಿಸಿಲಿನ ಧಗೆಗೆಳಸುವನಾರು?\\ಚಂಡ ದಿವಾಕರ ದಗ್ಧ ಹೃದಯನೂ\\ಸ್ನಿಗ್ಧ ಶಶಿಯ ಸಿಂಚನ ಬಯಸುವನು!\\
 ಸುಖವನು ಹೀರಲು ಕಾತರರೆಲ್ಲರು –\\ದುಃಖವ ತಬ್ಬುವ ಮರುಳನಾವನು?\\ಸುಖದ ಬಟ್ಟಲಲಿ ದುಃಖ ತುಂಬಿದರು,\\ಅಮೃತದಿ ಹಾಲಾಹಲವೆ ಇದ್ದರೂ,\\ಆಶೆಯನಾವನು ತಳ್ಳುವನು?

\begin{center}
–೬–
\end{center}

ರುದ್ರಮುಖದಿ ಭಯಭೀತರೆಲ್ಲರೂ,\\ಬಿಡುಮುಡಿಗೂದಲ ಮೃತ್ಯುರೂಪಿಯನು\\ಬಯಸರು ಯಾರೂ!\\ನೆತ್ತರು ಬಸಿಯುವ ಖಡ್ಗವನೊಲ್ಲದೆ\\ಕಾಳಿಯ ಕರದಲಿ ಕೊಳಲನ್ನಿಡುವರು!

ಮೃತ್ಯುರೂಪಿಯಹ ಮಹಾಕಾಳಿಯೇ,\\ನೀನೆ ಸತ್ಯ, ಸತ್ಯ!~।\\ಸುಖವನಮಾಲಿಯು ನಿನ್ನಯ ಮಾಯೆಯ\\ನೆರಳು – ಮಿಥ್ಯ, ಮಿಥ್ಯ!\\ಕನಸ ಕಡಿದು, ಓ ಕಾಳಿ ಕರಾಳಿನಿ,\\ದೇಹಸುಖವ ಬಿಡಿಸು,\\ತಾಯೇ, ಭ್ರಮೆಯನು ತುಂಡರಿಸು!\\ಕೊರಳಿಗೆ ತೊಡಿಸುತ ಮುಂಡಮಾಲೆಯನು\\ದೂರಕೆ ಸರಿಯುವರು, ಕರುಣಾಮಯಿ ನೀನೆನ್ನುವರು!\\ಭೀಮರುದ್ರಮುಖದಟ್ಟಹಾಸದಲಿ\\ದಿಗ್ವಸನವ ನೀ ಧರಿಸಿ ನಡೆದುಬರೆ\\ಕಂಪಿಸಿ ಕರಗುವರು, ದಾನವ–\\ನಾಶಿನಿಯೆನ್ನುವರು!\\ದರುಶನದಾಸೆಯ ಬಾಯಲಿ ನುಡಿದರು\\ಕಂಡು ಓಡುತಿಹರು!\\ಮೃತ್ಯುರೂಪಿ ನೀ, ವಿಷದ ಬಟ್ಟಲನು\\ನೀನೆ ತುಂಬುತಿರುವೆ; ಮಾರಿಯ\\ನೀನೆ ಹರಡುತಿರುವೆ!

\begin{center}
–೭–
\end{center}

ನಿನ್ನಿಂದಲೇ ನೀ ಮೋಸ ಹೋಗುತಿಹ\\ಮರುಳನೆ ನೀ ಕೇಳು:\\ಕಾಳಿಯ ಕಾಣುವ ಭಯದಿಂದಲಿ ನೀ\\ಕಣ್ಣ ಹೊರಳಿಸಿರುವೆ;\\ಸುಖವ ನಿರಕಿಸುತ ದುಃಖವ ಸಹಿಸುವೆ,\\ಸ್ವಾರ್ಥದ ಸಿದ್ಧಿಗೆ ಭಕ್ತಿಯ ನಟಿಸುವೆ,\\ಉನ್ಮತ್ತನೊ ನೀನು!

ಬಲಿಯ ಕೊರಳಿನಿಂ ಹರಿಯುವ ರಕ್ತವು\\ನಿನ್ನ ನಡುಗಿಸುವುದು; ಅದ ನೀ\\ದಯೆಯೆನ್ನುವೆಯೇನು?\\ಆಹಾ, ಧನ್ಯ, ಧನ್ಯ, ನೀನು!\\
 ನಿಜದಲಿ ಯಾರು ಅರಿವರಿದನು?

ಪ್ರೇಮಸುಧಾಪಾನದ ವೀಣಾರವ\\ಕಡಿದು ನಿಂತುಬಿಡಲಿ; ನಾರೀ\\ಮಾಯೆ ದೂರ ನಿಲಲಿ!\\ಸಾಗರದೊಲು ನೀ ಗರ್ಜಿಸಿ ಮುನ್ನಡೆ,\\ಪಣಕಿಡು ಪ್ರಾಣವನೆ; ಕುಡಿ ನೀ\\ಕಂಬನಿಧಾರೆಯನೆ; ಸವೆಯಿಸು\\ಸಿದ್ಧಿಗೆ ದೇಹವನೆ!

\begin{center}
–೮–
\end{center}

ಏಳು, ಏಳು, ನೀ, ಓ ವೀರಾತ್ಮನೆ,\\ಕನಸುಗಳನು ಕೊಡಹು;\\ನಿನ್ನ ತಲೆಯ ಬುಡದಲ್ಲಿಯೆ ಸಾವಿದೆ –\\ಅದಕೆ ಅಂಜಲೇಕೆ? ದುಃಖಭಾರದಲಿ ತುಂಬಿಹ ಈ ಭವ\\ಈಶನ ಮಂದಿರವು! ಮಸಣವೆ\\ಆತನಿರುವ ನೆಲೆಯು!\\ತುದಿಯನು ಕಾಣದ ಸಂಗರ, ಚಿತಾಗ್ನಿ –\\ಇದುವೆ ಈಶಪೂಜೆ!\\ಸತತವು ಸೋಲುತಲಿದ್ದರು ನೀನದ\\ಲೆಕ್ಕಿಸದಲೆ ನಡೆಯೊ!\\ಸ್ವಾರ್ಥ – ಮಾನ – ಮದ ಪುಡಿಪುಡಿಯಾಗಲಿ,\\ಎದೆಯ ಮಸಣ ಮಾಡು – ಅಲ್ಲಿಯ\\ನರ್ತನವನು ನೋಡು – ಶ್ಯಾಮೆಯು\\ನರ್ತಿಸುವುದ ನೋಡು!

\begin{center}
\textbf{ಗಾಯಿ ಗೀತ ಶುನಾತೆ ತೊಮಾಯ}\footnote{\engfoot{C.W, Vol. IV, P. 511}}
\end{center}

ಗಾಯಿ ಗೀತ ಶುನಾತೆ ತೊಮಾಯ\\ಭಾಲೊ ಮಣ್ದ ನಾಹಿ ಗಣಿ\\ನಾಹಿ ಗಣಿ ಲೋಕ ನಿಂದ ಯಶ ಕಥಾ!\\ದಾಸ ತೊಮಾ ದೋಹಾಕಾರ\\ಶಶಕ್ತಿ ಕ ನಮಿ ತತ ಪದೆ\\ಅಭ ತುಮಿ ಪಿಛೆ ದಾಡಾಯಿಯೆ\\ತಾಯಿ ಫಿರೆ ದೇಖಿ ತವ ಹಾಸಿಮುಖ~।\\ಫಿರೆಫಿರೆ ಗಾಯಿ ಕಾರೆ ನ ಡೊರಾಯಿ\\ಜನ್ಮ ಮೃತ್ಯು ಮೋರ ಪದತಲೆ\\ದಾಸ ತವ ಜನುಮೆ ಜನುಮೆ ದಯಾನಿಧೆ\\ತವಗತಿ ನಹಿ ಜಾನಿ!\\ಮಮಗತಿ–ತಹವೊ ನ ಜಾನಿ\\ಕೇವಾ ಚಾಯ ಜಾನಿವಾರೆ?\\ಭುಕ್ತಿ ಮುಕ್ತಿ ಭಕ್ತಿ ಆದಿ ಯತ\\ಜಪತಪ ಸಾಧನ ಭಜನ\\ಅಜ್ಜ ತವ ದಿಯೇಛಿ ತಾಡಾಯ\\ಆಛೆ ಮಾತ್ರ ಜಾನಜಾನಿ ಆಶ\\ತಾವೂ ಪ್ರಭು ಕರೊ ಪಾರ~।\\ಚಕ್ಷು ದೇಖೆ ಅಖಿಲ ಜಗತ್\\ನ ಚಾಹೆ ದೇಖಿತೆ ಅಪನಾಯ\\ಕೇನ ವಾ ದೇಖೀಬೆ~।\\ದೇಖೆ ನಿಜರೂಪ ತೇಖಿಲೆ ಪರೇರ್ ಮುಖ!\\ತುಮಿ ಆಖಿ ಮಮ ತವರೂಪ ಸರ್ವ ಘಟೆ\\ಛೆಲೆ ಥೇಲ್ ಕರಿ ತವ ಸನೆ\\ಕಬು ಕ್ರೋಧ ಕರಿ ತುಮಾ ಪರೆ\\ಜೇತೆ ಚಾಯಿ ದುರೇ ಪಲಾಯಿಯೆ;\\ಶಿಯರೆ ದಾಡಾಯೆ ತುಮಿ ರೇತೆ,\\ನಿರ್ವಾಕ ಅನನ ಛಲ ಛಲ ಆಖಿ\\ಚಾಹಾ ಮಮ ಮುಖ ಪಾನೆ\\ಅಮನಿ ಯೆ ಫಿರಿ ತವ ಪಾಯೆ ಧರಿ\\ಕಿನ್ತು ಕ್ಷಮಾ ನಹಿ ಮಾಗಿ\\ತುಮಿ ನಹಿ ಕರ ರೋಷ~।\\ಪುತ್ರತವ ಅನ್ಯಕಿ ಸಹಿಬೆ ಪ್ರಗಲ್ಭತಾ?\\ಪ್ರಭು ತುಮಿ, ಪ್ರಾಣಸಖಾ, ತುಮಿ ಮೋರ~।\\ಕಬು ದೇಖಿ ಆಮಿ ತುಮಿ ತುಮಿ ಆಮಿ\\ವಾಣಿ ತುಮಿ~।\\ವೀಣಾಪಾಣೀ ಕಂಠಮೋರ\\ತಠಂಗೆ ತೊಮಾರ ಭೇಸೆ ಜಾಯ ನರನಾರೀ\\ಸಿಂಧುರೋಲೆ ತವ ಹುಹೂಂಕಾರ,\\ಚಂದ್ರಸೂರ್ಯ ತೊಮಾರಿ ವಚನ\\ಮೃದು ಮಂದ ಪವನ ಆಲಾಪ\\ಏ ಸಕಲ ಸತ್ಯ ಕಥಾ\\ಕಿನ್ತು ಮನಿ ಅತಿ ಸ್ಥೂಲ ಭಾವ\\ತತ್ತ್ವಜ್ಞೇರ ಎ ನ ಹೆ ಭಾರತ~।\\ಸೂರ್ಯ ಚಂದ್ರಚಲ ಗ್ರಹ ತಾರಾ\\ಕೋಟಿ ಕೋಟಿ ಮಂಡಲ ನಿವಾಸ\\ಧೂಮಕೇತು ಬಿಜಲಿ ಆಭಾಸ\\ಸುವಿಸ್ತ್ರತ ಅನಂತ ಆಕಾಶ ಮನ ದೇಖೇ!\\
 ಕಾಮಕ್ರೋಧ ಲೋಭಮೋಹ ಆದಿ\\ಭಂಗ ಯಥಾ ತರಂಗಲೀಲಾರ\\ವಿದ್ಯಾ ಅವಿದ್ಯಾರ ಘರ\\ಜನ್ಮ ಜರಾ ಜೀವನ ಮರಣ\\ಸುಖ ದುಃಖ ದ್ವಂದ್ವ ಭರಾ\\ಕೇಂದ್ರ ಯಾರ್ ಅಹಂ ಅಹಂ ಇತಿ\\ಭುಜದ್ವಯ ಬಾಹಿರಂತರ\\ಆಸಮುದ್ರ ಆಸೂರ್ಯ ಚಂದ್ರಮಾ,\\ಆತಾರಕ ಅನಂತ ಆಕಾಶ,\\ಮನಬುದ್ಧಿ ಚಿತ್ತ ಅಹಂಕಾರ\\ದೇವ ಯಕ್ಷ ಮಾನವ ದಾನವ\\ಪಶುಪಕ್ಷಿ ಕ್ರಿಮಿಕೀಟ ಗಣ\\ಅನುಕ ದ್ವನುಕ ಜಡಜೀವ\\ಸೇಯಿ ಸಮಕ್ಷೇತ್ರೆ ಅವಸ್ಥಿತ!\\ಸ್ಥೂಲ ಅತೀವ ವಾಜ್ಯ ವಿಕಾಸ\\ಕೇಶ ಯಾಥಾ ಶಿರ ಪರೆ~।\\ಮೇರು ತಟೆ ಹಿಮಾನೀ ಪರ್ವತ\\ಯೋಜನ ಯೋಜನ ಸೇವಿಸ್ತಾರ\\ಅಭ್ರಭೇದಿ ನಿರಭ್ರ ಆಕಾಶೆ\\ಶತ ಉಠೆ ಚೂಡಾತಾರ~।\\
 ಝಗಮುಖಿ ಜ್ವಲೆ ಹಿಮಶಿಲ\\ಶತಶತ ಬಿಜಲಿ ಪ್ರಕಾಶ!\\ಉತ್ತರ ಅಯನೆ ವಿವಸ್ವನ\\ಏಕೀ ಭೂತ ಸಹಸ್ರ ಕಿರಣ,\\ಕೋಟಿ ವಜ್ರಸಮ ಕರಧಾರಾ\\ಢಾಲೆ ಯಬೆ ತಾಹಾರ ಉಪರ\\ಶೃಂಗೆ ಶೃಂಗೆ ಮೂರ್ಚಿತ ಭಾಸ್ಕರ\\ಗಲೆ ಚೂಡಾ ಶಿಖರ ಗಹ್ವರ\\ವಿಕಟ ನಿನಾದೆ ಖಸೆ ಪರೆ ಗಿರವರ,\\ಸ್ವಪ್ನಸಮ ಜಲೆಜಿಲ ಯಾಯ ಮಿಲೆ~।\\
 ಸರ್ವ ವೃತ್ತಿ ಮನೇರ ಯಾಖನ\\ಏಕೀ ಭೂತ ತುಮಾರ ಕೃಪಾಯ\\ಕೋಟಿ ಸೂರ್ಯ ಅತೀತ ಪ್ರಕಾಶ\\ಚಿತ್ ಸೂರ್ಯ ಹಯ ವೆ ವಿಕಾಶ\\ಗಲೆ ಯಾಯ ರವಿ ಶಶಿ ತಾರಾ\\ಆಕಾಶ ಪಾತಾಲ ತಲಾತಲ\\ಏ ಬ್ರಹ್ಮಾಂಡ ಗೋಶ್ಪದ ಸಮಾನ\\ವಾಜ್ಯಭೂಮಿ ಅತೀತ ಗಮನ\\ಶಾಂತ ಧಾತು ಮನ ಅಸ್ವಾಲನ ನಹಿ ಕರೆ\\ಶ್ಲತ ಹೃದಯೇರ ತಂತ್ರಿ ಯತ\\ಖುಲೆ ಯಾಯ ಸಕಲ ಬಂಧನ\\ಮಾಯಾ ಮೋಹ ಹಯ ದೂರ\\ಬಾಜೆ ತಥಾ ಅನಾಹತ ನಾದ ಧ್ವನಿ ತವ ವಾಣಿ;\\ಸುನಿ ಸ್ವಸಂಭ್ರಮ ದಾಸ ತವ ಪ್ರಸ್ತುತ ಸತತ\\ಸಾಧಿತೆ ತೊಮಾರ ಕಾಜ\\ಅಮಿ ವರ್ತಮಾನ~।\\ಪ್ರಲಯೇರ್‌ಕಾಲೆ ಅನಂತ ಬ್ರಹ್ಮಾಂಡ ಗ್ರಾಸಿ ಯವೆ\\ಜ್ಞಾನಜ್ಞೇಯ ಜ್ಞಾತಾಲಯ\\ಅಕಕ್ಷಣ ಅತರ್ಕ್ಯ ಜಗತ್\\ನಹಿ ತಾಕೆ ರವಿ ಶಶಿ ತಾರಾ\\ಸೇ ಮಹಾನಿರ್ವಾಣ ನಹಿ ಕರ್ಮ ಕರಣ ಕಾರಣ\\ಮಹಾ ಅಂಧಕಾರ ಫೇರೆ ಅಂಧಕಾರ ಬೂಕೆ\\ಆಮಿ ವರ್ತಮಾನ~।\\
 ಮಹಾ ಅಂಧಕಾರ ಫೇರೆ ಅಂಧಕಾರ ಬೂಕೆ\\ತ್ರಿಶೂನ್ಯ ಜಗತ್ ಶಾಂತ ಸರ್ವಗುಣ ಭೇದ\\ಏಕಾಕಾರ ಸೂಕ್ಷ್ಮರೂಪ ಶುದ್ಧ ಪರಮಾಣು ಕಾಯ\\ಆಮಿ ವರ್ತಮಾನ~।\\ಆಮಿ ಹೊಯಿ ವಿಕಾಸ ಆಬಾರ\\ಮಮಶಕ್ತಿ ಪ್ರಥಮ ವಿಕಾರ\\ಆದಿವಾಣಿ ಪ್ರಣವ ಓಂಕಾರ\\ಬಾಜೆ ಮಹಾಶೂನ್ಯ ಪಥೆ\\ಅನಂತ ಆಕಾಶ ಶುನೆ ಮಹಾನಾದ ಧ್ವನಿ\\ತ್ಯಾಜೆ ನಿದ್ರ ಕಾರಣ ಮಂಡಲಿ~।\\ಪ್ರಾಯ ನವಪ್ರಾಣ ಅನಂತ ಅನಂತ ಪರಮಾಣು\\ಲಂಪ ಝಂಫ ಆವರ್ತ ಉಚ್ವಾಸ\\ಚಲೆ ಕೇಂದ್ರ ಪ್ರದಿದೂರ ಅತಿದೂರ ಹತೆ\\ಚೇತನ ಪವನ ತೋಲೆ ಊರ್ಮಿಮಾಲಾ\\ಮಹಾಭೂತ ಸಿಂಧು ಪರೆ\\ಪರಮಾಣು ಆವರ್ತ ವಿಕಾಸ\\ಆಸ್ವಾಲನ ಪತನ ಉಚ್ವಾಸ\\ಮಹಾ ವೇಗ ಧಾಯ ಸೆ ತರಂಗ ರಾಜಿ\\ಅನಂತ ಅನಂತ ಖಂಡ ತಾರ\\ಉತಸಾರಿಕ ಪ್ರತಿಘಾತ ಬಲೆ,\\ಚೋಟಿ ಶೂನ್ಯ ಪಥೆ ಖಗೋಲ ಮಂಡಲರೂಪೆ\\ಥಾಯ ಗ್ರಹತಾರಾ,\\ಫೇರೆ ಪೃಥ್ವಿಮನುಷ್ಯ ಆವಾಸ~।\\'ಆಮಿ ಆದಿ ಕವಿ\\ಮಮಶಕ್ತಿ ವಿಕಾಸ ರಚನ\\ಜಡಜೀವ ಆದಿ ಯತ\\ಆಮಿ ಕರಿಖೇಲ ಶಕ್ತಿರೂಪ ಮಮ ಮಾಯಾ–ಸನೆ\\ಏಕಾ ಅಯಿ ಹಯಿ ಬಹು ದೇಖಿತೆ ಆಪವನರೂಪ'\\ಆಮಿ ಆದಿ ಕವಿ\\ಮಮಶಕ್ತಿ ವಿಕಾಶ ತಚನಾ\\ಜಡಜೀವ ಆದಿ ಯತ~।\\ಮಮ ಆಜ್ಞಾ ಬಲೆ\\ಮಹೆ ಜಂಝೆ ಪೃಥ್ವಿ ಉಪರ್\\ಗರ್ಜೆ ಮೇಘ ಅಶನಿ ನಿನಾದ\\ಮೃದು ಮಂದ ಮಲಯ ಪವನ\\ಆಸೆ ಜಾಯ ನಿಚ್ವಾಸ್ ಪ್ರಜ್ವಾಸ್ ರೂಪೆ.\\ಢಾಲೆ ಶಶಿ ಹಿಮಕರ ಧಾರಾ\\ತರುಲತಾ ಕರೆ ಆಚ್ಛಾದನ ಧರಾಬಪು\\ತೊಲೆ ಮುಖ ಶಿಶಿರ ಮಾರ್ಜಿತ\\ಪುಲ್ಲ ಪುಲ ರವಿ ಪಾನೆ~।

\begin{center}
\textbf{ಹಾಡೊಂದ ಹಾಡುವೆ ನಾ ನಿನಗೆ}
\end{center}

'ಗಾಯಿ ಗೀತ ಶುನಾತೇ ತೋಮಾಮ್' ಎಂಬ ಹೆಸರಿನ ಈ ಮೂಲ ಬಂಗಾಳಿ ಕವನ 'ಉದ್ಭೋಧನ' ಮಾಸಪತ್ರಿಕೆಯಲ್ಲಿ ಮೊದಲು ಪ್ರಕಟವಾಯಿತು (ಸಂಪುಟ: ೪, ಸಂಚಿಕೆ: ೯), ಸ್ವಾಮಿಜಿಯವರು ೧೮೯೪ರಲ್ಲಿ ಅಮೆರಿಕದಿಂದ ಸ್ವಾಮಿ ರಾಮಕೃಷ್ಣಾನಂದರಿಗೆ ಬಾರಾನಗರದ ಮಠಕ್ಕೆ ಬರದ ಪತ್ರದಲ್ಲಿಯೂ ಈ ಕವನದ ಕೆಲವು ಭಾಗಗಳಿವೆ.

ಸ್ವಾಮಿಜಿಯವರ ಜೀವನದ ವೈಯಕ್ತಿಕ ಅನುಭವದ ಹಿನ್ನೆಲೆಯೊಂದು ಈ ಕವನಕ್ಕಿದೆ. ಪರಿವ್ರಜನದ ಸಂದರ್ಭದಲ್ಲಿ ಸ್ವಾಮಿಜಿಯವರು ಉತ್ತರಪ್ರದೇಶದ ಘಾಜಿಪುರ ಎಂಬಲ್ಲಿಗೆ ಭೇಟಿನೀಡಿದಾಗ ಅಲ್ಲಿದ್ದ ಸಂತ ಪವಾಹಾರಿ ಬಾಬನಿಂದ ರಾಜಯೋಗದ ದೀಕ್ಷೆಯನ್ನು ಪಡೆಯಲು ಅಪೇಕ್ಷಿಸುತ್ತಾರೆ. ಆದರೆ ಈ ಸಂದರ್ಭದಲ್ಲಿ ಅವರಿಗೆ ಕೆಲವು ದಿನಗಳವರೆಗೆ ಮತ್ತೆ ಮತ್ತೆ ಶ‍್ರೀರಾಮಕೃಷ್ಣರು ದರ್ಶನದಲ್ಲಿ ಕಾಣಿಸಿಕೊಳ್ಳುತ್ತಾರೆ. ಅವರ ಮುಖದಲ್ಲೊಂದು ನೋವಿನ ಛಾಯೆಯನ್ನೂ ಸ್ವಾಮೀಜಿ ಗುರುತಿಸುತ್ತಾರೆ. ಹೀಗಾಗಿ ಈ ನಿರ್ಧಾರವನ್ನು ಕೈಬಿಡುತ್ತಾರೆ.

ಹಾಡೊಂದ ಹಾಡುವೆನು, ನಿನಗಾಗಿ ನಾನು,\\ಜನದ ಟೀಕೆಯನೆಲ್ಲ ನಾ ಗಣಿಸೆನು,\\ಹೊಗಳಿಕೆಯೊ ತೆಗಳಿಕೆಯೊ ಲೆಕ್ಕ ನನಗಿಲ್ಲ.\\ಶಿವಶಕ್ತಿಯರೆ ನಿಮ್ಮ ನಿಜದ ಸೇವಕ ನಾನು,\\ನಿಮ್ಮ ಪದತಲದಲ್ಲಿ ಎರಗುತಿಹೆನು.

ಅನವರತ ನೀನೆನ್ನ ಹಿಂದೆಯೇ ಇರುತಿರುವೆ\\ಮೃದು ಮಧುರ ಮಂದಹಾಸವನು ಬೀರಿ:\\ಅದರಿಂದಲೇ ನಾನು ತಿರುತಿರುಗಿ ಹಾಡುವೆನು,\\ಅಂಜಿಕೆಯದೆನಗೆಲ್ಲಿ, ಜನನ ಮರಣವು ಕೂಡ\\ಕಾಲಿನಡಿ ಬಿದ್ದಿಹುದು ಮೈಯ ಮುದುರಿ!

ಜನುಮ ಜನುಮಗಳಲ್ಲು ನಾ ನಿನ್ನ ಸೇವಕನು,\\ಹೇ ದಯಾನಿಧಿ, ನಿನ್ನ ಬಗೆಯನರಿಯೆ;\\ನನ್ನ ಮುಂದನು ಕೂಡ ತಿಳಿಯದಿರುವೆನು ನಾನು,\\ತಿಳಿಯುವಾಸೆಯು ಕೂಡ ಇಲ್ಲವೆನಗೆ.\\ಭುಕ್ತಿಮುಕ್ತಿಗಳನ್ನು, ಭಕ್ತಿ ಜಪತಪಗಳನ್ನು\\ನಿನ್ನ ಆಣತಿಯಂತೆ ತೊರೆದಿರುವೆನು;\\ಒಂದೆ ಬಯಕೆಯು ಎನ್ನ ಎದೆಯೊಳುಳಿದಿಹುದಿನ್ನು –\\ನಿನ್ನ ನನ್ನವನೆಂದು, ನನ್ನ ನಿನ್ನವನೆಂದು,\\ಓ ಪ್ರಭುವೆ, ಸೆಳೆದುಕೋ ನಿನ್ನ ಬಳಿಗೆ.\\ಆಸೆಗಳ ಗಡಿಯೇಕೆ ನಮ್ಮಿರ್ವರೊಳಗೆ?

ಜಗವ ಕಾಣುವ ಕಣ್ಣು ತನ್ನ ತಾ ಕಾಣದಿದೆ,\\ಕಾಣುವುದು ತನ್ನನ್ನು ಜಗದಲ್ಲಿಯೆ;\\ನೀನೆ ನನ್ನಯ ಕಣ್ಣು, ನಿನ್ನ ನಾ ಕಾಣುವೆನು\\ಕಾಣುತಿಹ ಒಂದೊಂದು ಮುಖದಲ್ಲಿಯೆ!

ಆಟದೊಳು ಮುಳುಗಿರುವ ಪುಟ್ಟ ಮಗುವಿನ ಹಾಗೆ\\ನಾ ನಿನಗೆ ತೋರುತಿರುವೆ;\\ಕೆಲವೊಮ್ಮೆ ನಿನ್ನಿಂದ ದೂರಾಗಿ ಅಲೆಯುವೆನು,\\ಕೆಲವೊಮ್ಮೆ ನಿನ್ನೊಡನೆ ಮುನಿಸಿಕೊಳುವೆ\\ಆ ಇರುಳ ಕತ್ತಲಲಿ ಮೌನಾಶ್ರುಧಾರೆಯಲಿ\\ತುಂಬಿ ಬಹ ಕಂಗಳಿಂ ನೀ ನೋಡುವೆ;\\ಆ ನಿನ್ನ ಮುದ್ದು ಮುಖ ಮಣಿಸುವುದು ನನ್ನನ್ನು,\\ನಿನ್ನ ಪದದಡಿ ನಾನು ಶರಣು ಬರುವೆ!

ನಿನ್ನ ಕ್ಷಮೆಯನ್ನೇಕೆ ನಾನು ಕೋರಲಿ? – ಪ್ರಭುವೆ,\\ಕೋಪ ನಿನಗಿಲ್ಲ ಈ ಸುತನ ಮೇಲೆ;\\ನೀನೆನ್ನ ತಾಯ್ತಂದೆ, ನಿನ್ನ ಚಿರಪುತ್ರ ನಾ,\\ನನ್ನ ನೀನೆಂದೆಂದು ಸಹಿಸುತಿರುವೆ!

ನೀನೆ ನನ್ನಯ ಪ್ರಭುವು, ನೀನೆನ್ನ ಪ್ರಾಣಸಖ,\\ನೀನೆ ನಾನಾಗಿರುವೆ, ನಾನೆ ನೀನು!\\ನೀನೆ ನನ್ನಯ ವಾಣಿ – ನನ್ನ ಕಂಠದಿ ನುಡಿವೆ,\\ನೀನೆ ವೀಣಾಪಾಣಿ – ನನ್ನ ನುಡಿಯು!

ಸಾಗರದ ಭೀಮಗರ್ಜನೆ ನಿನ್ನ ಹೂಂಕಾರ,\\ಸೂರ್ಯಚಂದ್ರರೆ ನಿನ್ನ ಶ‍್ರೀವಚನವು;\\ನಿನ್ನ ಆಲಾಪನವೆ ಮೃದುಮಂದ ಪವನವದು,\\ನಿನ್ನಾಜ್ಞೆಯಂತೆಯೇ ಸಕಲ ಜಗವು!

ರವಿ–ಚಂದ್ರ–ತಾರೆಗಳು, ಕೋಟಿ ನೀಹಾರಿಕೆಯು\\ಧೂಮಕೇತುವೊ ಮಿಂಚೊ ಕೊನೆಯಿಲ್ಲದಾಕಾಶ\\ಈ ಎಲ್ಲವನು ನೀನು ನೋಡುತಿರುವೆ!

ಷಡ್ರಿಪುಗಳೆಲ್ಲವೂ ಸಂಸಾರದಲೆಗಳವು\\ಆಡುವಾಟವೆ ಎನುಲು\\ವಿದ್ಯೆ – ಅವಿದ್ಯೆಗಳು ಮನೆಯ ಮಾಡಿಹವಿಲ್ಲಿ;\\ಜನನಮರಣಗಳೆಂಬ ಸುಖದುಃಖ ತಾನೆಂಬ~।\\
 ದ್ವಂದ್ವಗಳು ತಾವಿಲ್ಲಿ ಆವರಿಸಿವೆ;\\'ನಾನು, ನಾನೆಂ'ದೆಂಬ ಅಲ್ಪ ಅಹಮಿಹುದಿಲ್ಲಿ\\ಒಳಗು – ಹೊರಗುಗಳೆಂಬ ಭುಜವ ಚಾಚಿ!

ಸಾಗರದ ಆಳದಿಂ ಪಾರವಿಲ್ಲದ ನೀಲ ಗಗನದೊಳು ನೆಟ್ಟಿರುವ\\ಚಂದ್ರತಾರಗೆ ತನಕ ಎಲ್ಲವೂ, ಎಲ್ಲವೂ\\ಮನಸು ಬುದ್ಧಿಯು ಮತ್ತೆ ಚಿತ್ರಹಂಕಾರಗಳು\\ದೇವ ಯಕ್ಷನು ನರನು ದಾನವರು ತಾವೆಲ್ಲ\\ಪಶು ಪಕ್ಷಿ ಅಣು ಕೀಟ ಸೂಕ್ಷ್ಮಾಣು ಜಡ ಜೀವ\\ತಾವೆಲ್ಲವೂ\\ಒಳಗಿರಲಿ ಹೊರಗಿರಲಿ\\ಒಂದೆ ಇರವಿನ ಸ್ಥೂಲ ನೆಲೆಯಲಿಹವು–\\ನರನ ದೇಹದ ಹೊರಗೆ ಹೂಮ್ಮಿ ಕಾಣುವ ಕೇಶರಾಶಿಯಂತೆ!

ಮೇರುವಿನ ಮಡಿಲಿನಲಿ ಹಿಮಗಿರಿಯು ಹಬ್ಬಿಹುದು\\ಪಾರವಿಲ್ಲದಲೆ;\\ನೂರು ಸಾವಿರ ಶಿಖರ ಮೇಘಮಾಲೆಯನಿರಿದು\\ಆಗಸವ ತಿವಿಯುತಿವೆ,\\ಕೋಟಿ ಮಿಂಚುಗಳಂತೆ ಕೋರೈಸಿವೆ!

ಉತ್ತರಾಯಣರವಿಯ ಶತಕೋಟಿ ಕಿರಣಗಳ\\ವಜ್ರದಂಡದ ಘಾತ ಕಿರಣಗಳಿಗೆ\\ವಿಕಟಾಟ್ಟಹಾಸದಲಿ ಎರಗುತಿರಲು\\ಹಿಮಶಿಖರ ಕರಕರಗಿ ಭೀಮಗರ್ಜನೆಯಿಂದ\\ಜಾರುತಿಹುದು;\\ನೀರು ನೀರೊಳು ಕರಗಿ ಕನಸೆನುವ ತೆರದಲ್ಲಿ\\ಲಯವಾದುದು!

ನಿನ್ನ ಕೃಪೆಯಿಂ ಪ್ರಭುವೆ, ಮನದಲೆಗಳೆಲ್ಲವೂ\\ಲಯವಾಗಲು\\ಕೋಟಿ ಸೂರ್ಯರ ಬೆಳಕ ಹೊಮ್ಮಿಸುತ ಚಿದ್ರವಿಯು\\ತಾನೆ ಬೆಳಗುವನು!\\ರವಿ ಚಂದ್ರ ತಾರೆಗಳು ಪಾತಾಳ ಸ್ವರ್ಗಗಳು\\ಕರಗಿ ಮುಳುಗುವುವು!\\ಬ್ರಹ್ಮಾಂಡವೆಲ್ಲವೂ ಗೋಷ್ಟದದ ತೆರನಾಗೆ\\ಹೊರಗಿನಿರವನು ಮೀರ್ದ ತಾಣ ತೋರುವುದು!\\
 ಕರಣದಾರ್ಭಟ ಕುಗ್ಗಿ, ಮಮಕಾರ ತಗ್ಗಿಹುದು,\\ಎದೆಯ ಬಿಗಿದೆಳೆಗಳವು ತುಂಡಾದವು;\\ಮಾಯೆ – ಮೋಹಗಳಂತು ದೂರದಲಿ ನಿಂದಿಹವು,\\ತವ ಅನಾಹತನಾದ ಕೇಳುತಿಹುದು;\\ಅದರಿಂದ, ದಾಸನಿದೊ, ನಿನ್ನಾಣತಿಯ ನಡೆಸೆ\\ಸಿದ್ಧನಿಹನು!

ನಾನಿರುವೆ!\\ಪ್ರಳಯದಲಿ ಅರಿವು – ಅರಿವಿನ ಮೂಲ – ಅರಿವವನು\\ಅಳಿದರೂ ನಾನಿರುವೆನು!\\ಜಗವೆಂಬುದಲ್ಲಿಲ್ಲ, ಚಿಂತನೆಗು ನಿಲುಕದದು,\\ಆ ಮಹಾನಿರ್ವಾಣ ರವಿ ಚಂದ್ರ ತಾರೆಗಳ\\ನುಂಗಿರುವುದು!\\ಕರಣ– ಕಾರಣವಿಲ್ಲ, ಕರ್ಮಿ ತಾನಲ್ಲಿಲ್ಲ,\\ಕತ್ತಲನು ಕತ್ತಲೆಯೆ ಕಬಳಿಸಿಹುದು!\\ಆಕಾರ ತಾನೊಂದೆ, ಸೂಕ್ಷ್ಮರೂಪವ ತಳೆದ\\ಪರಮಾಣುಕಾಯ!\\ನಾನಿರುವೆನಲ್ಲಿಯೂ!

ಮರಳಿ ನಾ ವಿಕಸಿಸುವೆ!\\'ನಾನು, ನಾನೆಂ'ದೆಂಬ ಅಹಮಿಕೆಯೆ ಮೊದಲ ಅಲೆ,\\ಆದಿವಾಣಿಯು ಪ್ರಣವ – ಆ ಮಹಾ ಶೂನ್ಯದಲಿ\\ದನಿಗೈವುದು!\\ಕೊನೆಯಿಲ್ಲದಾಕಾಶ ಆ ದನಿಗೆ ಕಿವಿಗೊಡಲು\\ಮೂಲಭೂತಗಳೆಲ್ಲ ನಿದ್ರೆ ತಿಳಿಯುವುವು;\\ಸ್ಪಂದಿಸಲು ಪ್ರಾಣವದು ಕೇಂದ್ರದೆಡೆ ತಾವೆಲ್ಲ\\ತಿರುಗ ತೊಡಗುವವು!

ಪಂಚಭೂತದ ಮಹಾಸಾಗರದಿ ಅಲೆಯೆದ್ದು\\ಒಂದೊಂದನಪ್ಪಳಿಸಿ ಮೇಲೆ ಸಾಗುವವು!\\ಪ್ರತಿಘಾತದಿಂದೆದ್ದು ಶೂನ್ಯಪಥದಲಿ ಬಿದ್ದ\\ಗಗನಮಂಡಲವೆಲ್ಲ, ಗ್ರಹತಾರೆ ತಾವೆಲ್ಲ\\ಭ್ರಮಣಕೆಳಸುವವು!\\ಈ ನೆಲವು – ನರನ ಮನೆ\\ತಾನೆ ತಿರುಗುವುದು!

"ಆದಿಕವಿ ನಾನು!"\\ಈ ಜಡವ, ಚೇತನವನೊಳಗೊಂಡ ರೂಪಗಳು\\ವಿಕಸಿಸಿವೆ ತಾವೆನ್ನ ಶಕ್ತಿಯಿಂದ!\\ನನ್ನ ಮಾಯೆಯ ಕೂಡೆ ಆಟವಾಡುವೆ ನಾನೆ,\\ಏಕವಾಗಿಹ ನಾನೆ\\ನನ್ನನ್ನೆ ನಾ ನೋಡೆ\\ಆಗುವೆನನೇಕ!\\

"ಆದಿಕವಿ ನಾನು!\\ಈ ಜಡವ, ಚೇತನವನೊಳಗೊಂಡ ಲೋಕವಿದು\\ವಿಕಸಿಸಿದೆ ತಾನೆನ್ನ ಶಕ್ತಿಯಿಂದ!

ನನ್ನ ಆಣತಿಯಂತೆ\\ಗಾಳಿಯದು ಬೀಸುವುದು,\\ಮೋಡವದು ಗುಡುಗುವುದು,\\ಮಿಂಚು ಹೊಳೆಯುವುದು!\\
 ಮೃದುಪವನ ಚಲಿಸುವನು,\\ಶಶಿಯು ಶೀತಲ ಕಿರಣ ಪಸರಿಸುವನು!\\ತರುಲತೆಗಳನ್ನುಟ್ಟು, ಹಿಮಮಣಿಗಳಿಂದಿಡಿದ\\ಹೂವುಗಳ ತಾ ತೊಟ್ಟು\\ಭೂರಮಣಿ ರವಿಯೆಡೆಗೆ\\ವದನಾರವಿಂದವನು ಎತ್ತಿತೋರುವಳು!"

\begin{center}
\textbf{ಸಾಗರ – ವಕ್ಷೇ}\footnote{\engfoot{C.W, Vol, VI, P.180}}
\end{center}

ನೀಲಾಕಾಶ ಭಾಸೇ ಮೇಘಕುಲ\\ಶ್ವೇತ ಕೃಷ್ಣ ವಿವಿಧ ವರಣ್ –\\ತಾಹೇ ತಾರತಮ್ಯ ತಾರಲ್ಯೇರ್\\ಪೀತ ಭಾನು ಮಾಂಗಿಛ ವಿದಾಯ್\\ರಾಗಚ್ಛಟಾ ಜಲದ ದೇಖಾಯ್~।

ಬಹೇ ವಾಯು ಅಪನಾರ್ ಮನೇ\\ಪ್ರಭಂಜನ್ ಕೊರಿಛೆ ಗಠನ್ –\\ಕ್ಷಣೇ ಗಡೇ, ಭಾಂಗೇ ಆರ್ ಕ್ಷಣೇ –\\ಕೊತೊಮತೋ ಸತ್ಯ ಅಸಂಭವ್ –\\ಜಡ, ಜೀವ, ವರ್ಣ, ರೂಪ, ಭಾವ್~।

ಒಯ್ ಆಶೇ ತುಲಾರಾಶಿ ಸಮ್\\ಪರಕ್ಷಣೇ ಹೇರ್‌ ಮಹಾನಾಗ್,\\ದೇಖೊ ಸಿಂಹ ವಿಕಾಶೇ ವಿಕ್ರಮ್\\ಆರ್ ದೇಖೊ ಪ್ರಣಯಿ ಯುಗಲ್;\\ಶೇಷೆ ಸಬ್ ಆಕಾಶೇ ಮಿಲಾಯ್~॥

ನೀಚೇ ಸಿಂಧು ಗಾಯ್ ನಾನಾ ತಾನ್;\\ಮಹೀಯಾನ್ ಸೇ ನಹೇ ಭಾರತ್!\\ಅಂಬುರಾಶಿ ವಿಖ್ಯಾತ್ ತೋಮಾರ್;\\ರೂಪ–ರಾಗ ಹೊಯೇ ಜಲಮಯ್\\ಗಾಯ್ ಹೇಥಾ, ನಾ ಕೊರೇ ಗರ್ಜನ್

\begin{center}
\textbf{ಕಡಲಿನೆದೆಯ ಮೇಲೆ}
\end{center}

ಸ್ವಾಮಿಜಿಯವರು ಎರಡನೆಯ ಬಾರಿಗೆ ಪಾಶ್ಚಾತ್ಯ ರಾಷ್ಟ್ರಗಳ ಸಂದರ್ಶನದಿಂದ ಹಡಗಿನಲ್ಲಿ ಹಿಂದಿರುವಾಗ ಬರೆದ ಕವನ ಇದು. ಇದನ್ನು ಬರೆಯುವಾಗ ಅವರ ಹಡಗು ಬಹುಶಃ ಪೂರ್ವ ಮೆಡಿಟರೇನಿಯನ್‌ ಅನ್ನು ದಾಟುತ್ತಿತ್ತು. ಇಲ್ಲಿ ವಿವಿಧ ಹೋಲಿಕೆಗಳೊಂದಿಗೆ ಮೋಡಗಳ ವರ್ಣನೆ ಮನೋಜ್ಞವಾಗಿ ಮೂಡಿಬಂದಿದೆ.

ನೀಲಿಯಾಗಸದಲ್ಲಿ ತೇಲಿದೆ ಮೇಘಮಾಲೆಯ ಸಂಕುಲ,\\ಶ್ವೇತ–ಶ್ಯಾಮಲ ವರ್ಣ ವಿಧವಿಧ; ಸಣ್ಣದಿದು, ಅದೊ ವಿಸ್ತರ!\\ಪಡುವಣದ ಮನೆಯೆಡೆಗೆ ನಡೆದಿಹ ಕೆಂಪುಸೂರ್ಯನ ಕರದಲಿ\\ಚಿಮ್ಮಿಚೆಲ್ಲಿದೆ ರಂಗಿನೋಕುಳಿ ಮೋಡಮೋಡಗಳಂಚಲಿ!

ದಿಕ್ಕುದಿಕ್ಕಿಗೆ ಮೊರೆದು ನುಗ್ಗಿದ ಚಂಡಮಾರುತ ಭರದಲಿ\\ಈಗ ಕೂಡಿಸಿ ಮೋಡವೆಲ್ಲವನೀಗ ತುಂಡರಿಸುತ್ತಲಿ;\\ಜೀವವಿಲ್ಲದ, ಜೀವ ತುಂಬಿದ ವರ್ಣ ಬಗೆಬಗೆ ರೂಪವು\\ತೋರುತಿದ್ದರು ದೃಶ್ಯಕೋಟಿಯು ಸ್ವಪ್ನಲೋಕದ ಸತ್ಯವು!

ಹಗುರಮೋಡಗಳಲ್ಲಿ ಹರಡಿವೆ ಹಿಂಜಿದರಳೆಯ ತೆರದಲಿ,\\ಹೆಡೆಯ ಬಿಚ್ಚಿಹ ಕಾಳನಾಗರ, ಸಿಂಹವದು ಮರುಚಣದಲಿ;\\ನೇಹದಲಿ ಒಂದಾದ ಜೋಡಿಯದಲ್ಲಿ ಆ ಕಡೆ ತೇಲಿದೆ\\ಎಲ್ಲವೂ ಕಡೆಗಲ್ಲಿ ಕರಗುತ ವ್ಯೋಮದೊಳಗೊಂದಾಗಿವೆ!

ಕೆಳಗೆ ವಾರಿಧಿಗಾನ ಹೊಮ್ಮಿದೆ ನೂರು ದನಿಯಲಿ ಸಂತತ,\\ಶಬ್ದ ವೇಳುತ ಮುಳುಗುತಿದ್ದರು ಆರ್ಭಟಿಸದಿದೆ ಸಾಗುತ;\\ದಿಗ್ದಿಗಂತಕೆ ಚಾಚುತಿದ್ದರು ಶಾಂತಮರ್ಮರ ಸ್ವರದಲಿ\\ಸದ್ದುಗದ್ದಲವಡಗಿ ಚಿಮ್ಮಿದೆ ಸೂಗದ ಸುಧೆ ಇಂಚರದಲಿ!

\begin{center}
\textbf{ಶ‍್ರೀ ಕೃಷ್ಣ ಸಂಗೀತ}\footnote{\engfoot{C.W, Vol. VII, P. 171}}
\end{center}

\begin{center}
(ಹಿಂದಿ)
\end{center}

\begin{myquote}
ಮೂಜೆ ಬಾರಿ ಬನೊಯಾರಿ ಸಂಯಿಯಾ
\end{myquote}

\begin{flushright}
ಜಾನೆಕೊ ದೇ
\end{flushright}

\begin{myquote}
ಜಾನೆಕೊ ದೇರೆ ಸಂಯಿಯಾ
\end{myquote}

\begin{flushright}
ಜಾನಕೋ ದೇ (ಆಜು ಬಾಲ)
\end{flushright}

\begin{myquote}
ಮೇರಾ ಬನೋಯಾರಿ ಬಾಂದಿ ತುಹಾರಿ\\ಛೋಡೇ ಜತುರಾಯೀ ಸಂಯಿಯಾ
\end{myquote}

\begin{flushright}
ಜಾನಕೊ (ಆಜು ಬಾಲ)\\(ಮೇರೆ ಸಂಯಿಯಾ)
\end{flushright}

\begin{myquote}
ಯುಮುನಾಕಿ ನೀರ ಬಾರೊ ಗಾಗರೀಯ\\ಜೋತಿ ಕಹತ ಸಂಯಿಯಾ
\end{myquote}

\begin{flushright}
ಜಾನೆಕೊ ದೇ~॥
\end{flushright}

\begin{center}
\textbf{ಕೃಷ್ಣಗೀತೆ}
\end{center}

ಯಮುನೆಗೆ ನೀರನ್ನು ತರಲು ಹೋಗುತ್ತಿರುವ ಗೋಪಿಕೆಯೊಬ್ಬಳನ್ನು ಶ‍್ರೀಕೃಷ್ಣನು ತಡೆದಾಗ ಅವಳು ಆತನಲ್ಲಿ ಮಾಡಿಕೊಳ್ಳುವ ವಿನಂತಿ ಈ ಗೀತೆಯಲ್ಲಿ ನಿರೂಪಿತವಾಗಿದೆ.

\begin{myquote}
ನೀರ ತರಲು ಬಿಡು, ಸಖ ಶ‍್ರೀಕೃಷ್ಣನೆ,\\ನೀರ ತರಲು ಬಿಡು ನೀ,\\ಯಮುನೆಗೆ ನೀರ ತರಲು ಬಿಡು ನೀ!\\ಎಂದೋ ನಿನ್ನಯ ದಾಸಿಯಾಗಿರುವೆ,\\ಪೀಡಿಸಲೇಕಿನ್ನು? – ಎನ್ನನು\\ಪೀಡಿಸಲೇಕಿನ್ನು?\\ಯಮುನೆಯ ನೀರನು ತರಲು ಹೋಗುವೆನು,\\ಸಖನೇ ಬಿಡು ನನ್ನ ಕೈಗಳ\\ಮುಗಿಯುವೆ, ಬಿಡು ನನ್ನ!
\end{myquote}

