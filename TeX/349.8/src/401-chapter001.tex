
\chapter[ಅಧ್ಯಾಯ ೧]{ಅಧ್ಯಾಯ ೧\protect\footnote{\engfoot{C.w, Vol. V, P. 329}} ಸಂಸ್ಕೃತ ಸ್ತೋತ್ರಗಳು}

ಒಂದು ದಿನ ಬೇಲೂರು ಮಠಕ್ಕೆ ಸ್ವಾಮೀಜಿಯನ್ನು ನೋಡಲು ನನ್ನ ಕೆಲವು ವಿಶ್ವವಿದ್ಯಾನಿಲಯದ ಸ್ನೇಹಿತರೊಡನೆ ಹೋದೆ. ನಾವು ಅವರ ಸುತ್ತಲೂ ಕುಳಿತುಕೊಂಡೆವು. ಅನೇಕ ವಿಷಯಗಳ ಮೇಲೆ ಮಾತುಕತೆಯಾಗುತ್ತಿತ್ತು. ಯಾವ ಪ್ರಶ್ನೆಯನ್ನೇ ಹಾಕಲಿ ಮರುಕ್ಷಣದಲ್ಲಿಯೇ ತೃಪ್ತಿಕರವಾದ ಉತ್ತರವನ್ನು ಸ್ವಾಮೀಜಿ ಕೊಡುತ್ತಿದ್ದರು. ಇದ್ದಕ್ಕಿದ್ದಂತೆಯೇ ಅವರು ನಮ್ಮ ಕಡೆಗೆ ಬೆರಳು ತೋರಿಸಿ "ನೀವೆಲ್ಲಾ ಅನೇಕ ಬಗೆಯ ಐರೋಪ್ಯ ತತ್ತ್ವಶಾಸ್ತ್ರ, ಕಲೆ, ವಿಜ್ಞಾನ ಮುಂತಾದುವನ್ನು ವ್ಯಾಸಂಗ ಮಾಡುತ್ತಿದ್ದೀರಿ. ವಿವಿಧ ಜನಾಂಗಗಳ ದೇಶಗಳ ವಿಷಯವಾಗಿ ಹೊಸ ಹೊಸ ವಿಚಾರಗಳನ್ನು ಕಲಿಯುತ್ತಿದ್ದೀರಿ. ಜೀವನದಲ್ಲಿ ಶ್ರೇಷ್ಠತಮ ಸತ್ಯ ಯಾವುದು ಹೇಳಬಲ್ಲಿರಾ?" ಎಂದು ಕೇಳಿದರು.

ನಾವೆಲ್ಲಾ ಯೋಚಿಸಲಾರಂಭಿಸಿದೆವು. ಅವರು ನಮ್ಮಿಂದ ಏನು ಉತ್ತರವನ್ನು ಅಪೇಕ್ಷಿಸುತ್ತಿರುವರೆಂದು ಗೊತ್ತುಹಿಡಿಯಲಾರದೆ ಹೋದೆವು. ಯಾರಿಂದಲೂ ಉತ್ತರ ಬಾರದಿರಲು ಅವರು ತಮ್ಮ ಸ್ಫೂರ್ತಿಯುತವಾದ ಭಾಷೆಯಲ್ಲಿ ಹೀಗೆ ಹೇಳಿದರು:

“ಇಲ್ಲಿ ನೋಡಿ – ನಾವೆಲ್ಲಾ ಸಾಯುತ್ತೇವೆ! ಈ ಸತ್ಯವನ್ನು ಯಾವಾಗಲೂ ಮನಸ್ಸಿನಲ್ಲಿಡಿ. ಆಗ ಮಾತ್ರ ಅಂತರಾತ್ಮ ಜಾಗೃತನಾಗುವನು. ಆಗ ಮಾತ್ರ ನಿಮ್ಮಿಂದ ಕ್ಷುದ್ರತನ ಮಾಯವಾಗುವುದು. ಕಾರ್ಯೋನ್ಮುಖರಾಗುವಿರಿ. ನಿಮ್ಮ ಮನಸ್ಸು ಮತ್ತು ದೇಹದಲ್ಲಿ ನವಚೈತನ್ಯ ಮೂಡುವುದು. ಯಾರು ನಿಮ್ಮ ಸಂಪರ್ಕ ಮಾಡುವರೋ ಅವರಿಗೂ ಕೂಡ ನಿಮ್ಮೊಡನಿರುವಾಗ ಏನೋ ಒಂದು ಬಗೆಯ ಆತ್ಮೋನ್ನತಿಯನ್ನು ಪಡೆವಂತೆ ಭಾಸವಾಗುವುದು.”

ನಂತರ ನನಗೂ ಸ್ವಾಮೀಜಿಗೂ ಕೆಳಗೆ ಕಂಡಂತೆ ಸಂಭಾಷಣೆ ಜರುಗಿತು.

ನಾನು: ಆದರೆ, ಸ್ವಾಮೀಜಿ, ಸಾವಿನ ಯೋಚನೆ ಬಂದ ಕೂಡಲೇ ನಮ್ಮ ಉತ್ಸಾಹವೆಲ್ಲಾ ಕುಗ್ಗಿ ನಿರಾಶೆಯಿಂದ ಎದೆಗುಂದುವುದಿಲ್ಲವೆ?

ಸ್ವಾಮೀಜಿ: ಅದೇನೋ ನಿಜ. ಮೊದಲು ಎದೆಯೊಡೆದಂತಾಗಿ ನಿರಾಶೆ ಮತ್ತು ಗ್ಲಾನಿಕರವಾದ ಆಲೋಚನೆಗಳು ಮನಸ್ಸಿಗೆ ಬರುವುವು. ಆದರೆ ದೃಢವಾಗಿ ನಿಲ್ಲು – ದಿನಗಳು ಅಂತೆಯೇ ಕಳೆಯಲಿ. ನಂತರ ನಿನ್ನ ಹೃದಯಕ್ಕೆ ನವಶಕ್ತಿ ಬರುತ್ತದೆ, ಮೃತ್ಯುವಿನ ಬಗ್ಗೆ ನಿರಂತರ ಚಿಂತನೆ ನಿನಗೆ ಹೊಸದೊಂದು ಜೀವನವನ್ನೇ ಕೊಡುತ್ತದೆ; ಪ್ರತಿಕ್ಷಣದಲ್ಲಿಯೂ ನಿನ್ನ ಮನಶ್ಚಕ್ಷುವಿನ ಮುಂದೆ ಮಾನವನ ಎಲ್ಲ ವೈಭವ ಅಚಿರ, ನೀರಗುಳ್ಳೆಯಂತೆ ನಿಸ್ಸಾರ ಎಂಬ ಸತ್ಯವು ನಿನ್ನನ್ನು ಹೆಚ್ಚು ಹೆಚ್ಚಾಗಿ ಆಲೋಚನಾ ಮಗ್ನನಾಗುವಂತೆ ಮಾಡುತ್ತದೆ ಎಂಬುದು ನಿನಗೇ ತಿಳಿಯುವುದು. ಕೊಂಚ ತಾಳು! ದಿನಗಳು ತಿಂಗಳು ವರುಷಗಳು ಉರುಳಿಹೋಗಲಿ. ನಿನ್ನ ಅಂತಃಸ್ಫೂರ್ತಿ ಸಿಂಹಸದೃಶ ಶಕ್ತಿಯಿಂದ ಜಾಗೃತಗೊಂಡಿರುವುದು, ಮತ್ತು ನಿನ್ನ ಅಂತರಂಗಶಕ್ತಿ ಒಂದು ಮಹತ್ತಾದ ಶಕ್ತಿಯಾಗಿ ಮಾರ್ಪಟ್ಟಿರುವುದು ನಿನಗೇ ಅರಿವಾಗುತ್ತದೆ! ನಿರಂತರ ಮೃತ್ಯು ಚಿಂತನೆ ಮಾಡು. ಆಗ ನಾನು ಹೇಳುವ ಪ್ರತಿಯೊಂದು ಮಾತಿನ ಸತ್ಯವನ್ನೂ ಅರಿಯುವೆ. ಇನ್ನು ಮಾತಿನ ಮೂಲಕ ನಾನು ಹೆಚ್ಚೇನನ್ನು ಹೇಳಲಿ!~।

ನನ್ನ ಸ್ನೇಹಿತರಲ್ಲೊಬ್ಬರು ಸ್ವಾಮೀಜಿಯನ್ನು ಮೆಲ್ಲಗೆ ಹೊಗಳುತ್ತಿದ್ದರು.

ಸ್ವಾಮೀಜಿ: ನನ್ನನ್ನು ಹೊಗಳಬೇಡಿ. ನಮ್ಮ ಜಗತ್ತಿನಲ್ಲಿ ಹೊಗಳಿಕೆ ತೆಗಳಿಕೆಗೆ ಯಾವ ಬೆಲೆಯೂ ಇಲ್ಲ. ಅವು ಮನುಷ್ಯನನ್ನು ಉಯ್ಯಾಲೆಯಂತೆ ತೂಗಿಸುವುವು. ನನಗೆ ಸಾಕಾದಷ್ಟು ಹೊಗಳಿಕೆ ಬಂದಿದೆ. ದೋಷಾರೋಪಣೆಗಳ ಸುರಿಮಳೆಯನ್ನೂ ನಾನು ಸಹಿಸಿಕೊಳ್ಳಬೇಕಾಯಿತು. ಆದರೆ ಅವುಗಳನ್ನು ಕುರಿತು ಯೋಚಿಸುತ್ತಾ ಹೋದರೆ ಏನು ಬಂದ ಹಾಗಾಯಿತು! ಪ್ರತಿಯೊಬ್ಬರೂ ನಿರಾತಂಕವಾಗಿ ತಮ್ಮ ಕರ್ತವ್ಯಗಳನ್ನು ಮಾಡುತ್ತಾ ಹೋಗಿ, ಕಡೆಯ ಗಳಿಗೆ ಬಂದಾಗ ನಿನಗಾಗಲಿ ನನಗಾಗಲಿ, ಮತ್ಯಾರಿಗೇ ಆಗಲಿ ಈ ಹೊಗಳಿಕೆ ತೆಗಳಿಕೆಗಳೆರಡೂ ಒಂದೆ. ನಾವು ಕೆಲಸ ಮಾಡಲು ಇಲ್ಲಿದ್ದೇವೆ. ಕರೆ ಬಂದಾಗ ಎಲ್ಲವನ್ನೂ ಬಿಟ್ಟು ಹೊರಡಬೇಕು.

ನಾನು: ನಾವೆಷ್ಟು ಅಲ್ಪರು, ಸ್ವಾಮಿಜಿ!

ಸ್ವಾಮೀಜಿ: ನಿಜ! ನೀನು ಸರಿಯಾಗಿ ಹೇಳಿದೆ! ಲಕ್ಷಾಂತರ ಸೂರ್ಯ ಮತ್ತು ಸೌರವ್ಯೂಹಗಳನ್ನೊಳಗೊಂಡ ಈ ಅಪಾರ ಭೂಮಂಡಲವನ್ನು ಕುರಿತು ಯೋಚಿಸು. ಇವೆಲ್ಲಾ ಯಾವ ಅನಂತ ಅಪರಿಮಿತವಾದ ಶಕ್ತಿಯಿಂದ ಪ್ರಚೋದಿತವಾಗಿ ಆ ಏಕಮಾತ್ರ ಅಗೋಚರ ಪಾದಗಳನ್ನು ಸ್ಪರ್ಶಿಸಲೋಸುಗ ಓಡುತ್ತಿವೆ ಎಂಬುದನ್ನು ಯೋಚಿಸು. ಮತ್ತೆ ನಾವೆಷ್ಟು ಕ್ಷುದ್ರರು! ದೂಷಣೆ, ತುಚ್ಛತನದ ಲೋಲುಪ್ತಿಯಲ್ಲಿ ಮಗ್ನರಾಗಲು ಸ್ಥಳವೆಲ್ಲಿದೆ? ಪರಸ್ಪರ ಹಗೆತನ ಪ್ರತ್ಯೇಕ ಮನೋಭಾವನೆಯನ್ನು ಬೆಳೆಸಿದರೆ ನಮಗಾಗುವ ಲಾಭವೇನು? ನನ್ನ ಬುದ್ಧಿವಾದವನ್ನು ಕೇಳಿ: ನೀವು ಪ್ರೌಢ ಶಿಕ್ಷಣ ಮುಗಿಸಿದ ಮೇಲೆ ಇತರರ ಸೇವೆಗಾಗಿ ನಿಮ್ಮ ಬಾಳನ್ನು ಮೀಸಲಾಗಿಡಿ. ನನ್ನ ಮಾತನ್ನು ನಂಬಿ, ಹಣದಿಂದ ಅಮೂಲ್ಯ ವಸ್ತುಗಳಿಂದ ತುಂಬಿದ ಇಡೀ ಬೊಕ್ಕಸ ನಿಮ್ಮ ಅಧೀನದಲ್ಲಿದ್ದಾಗ ಆಗುವ ಹರ್ಷಕ್ಕಿಂತಲೂ ಅಪಾರವಾದ ಹರ್ಷ ನಿಮಗೆ ಇದರಿಂದ ಉಂಟಾಗುವುದು. ಇತರರಿಗೆ ಸೇವೆ ಸಲ್ಲಿಸುತ್ತಾ ಹೋದಂತೆಲ್ಲಾ ಜ್ಞಾನದ ಹಾದಿಯಲ್ಲೂ ಅಷ್ಟೇ ಮುಂದುವರಿಯುವಿರಿ.

ನಾನು: ನಾವು ತುಂಬಾ ಬಡವರು ಸ್ವಾಮೀಜಿ!

ಸ್ವಾಮೀಜಿ: ನಿನ್ನ ಬಡತನದ ಆಲೋಚನೆಗಳನ್ನು ಒಂದು ಕಡೆ ಕಟ್ಟಿಡು. ಯಾವ ಭಾಗದಲ್ಲಿ ನೀನು ಬಡವ! ನಿನ್ನ ಹತ್ತಿರ ಒಂದು ಜೊತೆ ಕೋಡುಬಂಡಿಯಿಲ್ಲ, ಸೇವಕರ ತಂಡವಿಲ್ಲ ಎಂದೇ? ಅದರಿಂದೇನು? ನಿನ್ನ ಎದೆ ರಕ್ತವನ್ನು ಬಸಿದು ಹಗಲೂ ರಾತ್ರಿ ಇತರರ ಸೇವೆಗಾಗಿ ದುಡಿ. ಈ ಜೀವನದಲ್ಲಿ ನಿನಗೆ ಅಸಾಧ್ಯವಾದುದು ಯಾವುದೂ ಇಲ್ಲವೆಂಬುದನ್ನು ನೀನು ಕೊಂಚವೂ ಅರಿಯೆ! ಅಗೋ! ಅಲ್ಲಿ ನೋಡು! ಪವಿತ್ರಜೀವನದಿಯ ಮತ್ತೊಂದು ದಡ ನಿನ್ನ ಕಣ್ಣೆದುರಿಗೇ ಪ್ರತ್ಯಕ್ಷವಾಗಿದೆ, ಮೃತ್ಯುವಿನ ತೆರೆ ಮಾಯವಾಗಿದೆ! ಅದ್ಭುತವಾದ ಅಮೃತ ರಾಜ್ಯದ ಉತ್ತರಾಧಿಕಾರಿಗಳು ನೀವು!

ನಾನು: ಓ! ಸ್ವಾಮೀಜಿ, ನಿಮ್ಮ ಮುಂದೆ ಕುಳಿತು ನಿಮ್ಮ ಸ್ಫೂರ್ತಿದಾಯಕ ನುಡಿಗಳನ್ನು ಕೇಳುತ್ತಾ ನಾವೆಷ್ಟು ಆನಂದಿಸುತ್ತಿದ್ದೇವೆ!

ಸ್ವಾಮೀಜಿ: ನೋಡಿ, ಭರತಖಂಡದಲ್ಲಿ ಇಷ್ಟು ವರ್ಷಗಳು ನಾನು ಪ್ರಯಾಣಮಾಡಿದಾಗ ಅನೇಕ ಮಹಾತ್ಮರನ್ನು ಸಂದರ್ಶಿಸಿದೆ. ಅವರ ಹೃದಯದಲ್ಲಿ ಪ್ರೇಮ ತುಂಬಿ ತುಳುಕುತ್ತಿತ್ತು. ಅವರ ಪಾದದಡಿ ಕುಳಿತಾಗ ಒಂದು ಮಹಾಶಕ್ತಿಯ ಪ್ರವಾಹ ನನ್ನ ಹೃದಯದಲ್ಲಿ ಸಂಚರಿಸಿದಂತೆ ಭಾಸವಾಗುತ್ತಿತ್ತು. ಅವರ ಸಂಪರ್ಕದಿಂದ ಉದ್ಭವಿಸಿದ ಆ ಪ್ರವಾಹದ ಶಕ್ತಿಯಿಂದ ಮಾತ್ರ ನಿಮಗೆ ನಾಲ್ಕು ಮಾತುಗಳನ್ನು ಹೇಳಿದೆ. ನಾನೇನೋ ಒಬ್ಬ ಮಹಾತ್ಮನೆಂದು ತಿಳಿಯಬೇಡಿ!

ನಾನು: ಆದರೆ, ಸ್ವಾಮೀಜಿ, ನೀವು ಭಗವತ್ಸಾಕ್ಷಾತ್ಕಾರ ಪಡೆದವರೆಂದು ನಾವೆಲ್ಲಾ ಭಾವಿಸುತ್ತೇವೆ.

ನಾನೀ ಮಾತನ್ನು ಹೇಳಿದ ತಕ್ಷಣವೆ ಸ್ವಾಮೀಜಿಯ ಮೋಹಮುಗ್ಧಕರ ಕಣ್ಣುಗಳು ಅಶ್ರುಜಲದಿಂದ ತುಂಬಿದವು. (ಓ! ಈಗಲೂ ಆ ನೋಟ ಅಚ್ಚಳಿಯದೆ ನನ್ನ ಕಣ್ಣಿಗೆ ಕಟ್ಟಿದಂತಿದೆ.) ಪ್ರೇಮದಿಂದ ತುಂಬಿ ತುಳುಕುತ್ತಿದ್ದ ಹೃದಯದಿಂದ ಸ್ವಾಮೀಜಿ ಮೃದುವಾಗಿ ದಯಾಪೂರಿತ ಧ್ವನಿಯಲ್ಲಿ ಹೀಗೆಂದರು, "ಜ್ಞಾನಿಗಳು ಅರಸುವ ಆ ಜ್ಞಾನದ ಪರಿಪೂರ್ಣತೆಯೂ ಆ ಪವಿತ್ರ ಪಾದಗಳಲ್ಲಿದೆ! ಪ್ರೇಮಿಗಳು ಅರಸುವ ಪ್ರೇಮದ ಪೂರ್ಣತೆಯೂ ಆ ಪವಿತ್ರ ಪಾದಗಳಲ್ಲಿದೆ! ಓ ಹೇಳು, ಸ್ತ್ರೀ ಪುರುಷರು ಆ ಪವಿತ್ರ ಪಾದಗಳನ್ನು ಶರಣು ಹೋಗುವರಲ್ಲದೆ ಮತ್ತೆಲ್ಲಿ ಹೋಗುವರು!" ಕೊಂಚ ಹೊತ್ತಿನ ತರುವಾಯ ಹೇಳಿದರು “ಅಯ್ಯೋ! ಪ್ರಪಂಚದಲ್ಲಿ ಜನರು ಈಗ ಮಾಡುತ್ತಿರುವಂತೆ ಹೋರಾಡುತ್ತಾ ಕಾದಾಡುತ್ತಾ ದಿನಗಳನ್ನು ಕಳೆಯುತ್ತಿರುವುದು ಎಂತಹ ತಿಳಿಗೇಡಿತನ! ಆದರೆ ಎಷ್ಟು ಕಾಲ ಅವರು ಹೀಗೆ ಇರಬಲ್ಲರು? ಜೀವನದ ಸಂಧ್ಯೆಯಲ್ಲಿ ಎಲ್ಲರೂ ಮನೆಗೆ, ಜಗನ್ಮಾತೆಯ ವಾಸಸ್ಥಳಕ್ಕೆ ಬಂದೇ ಬರುವರು.”

\newpage

\chapter[ಚಿಂತನೆಯ ಹನಿಗಳು—೧]{ಚಿಂತನೆಯ ಹನಿಗಳು—೧\protect\footnote{\engfoot{CW, Vol. V.P. 409}}}

೧. ಮಾನವನು ಪ್ರಕೃತಿಯನ್ನು ಆಳುವುದಕ್ಕೆ ಅವತರಿಸುವುದು, ಅದನ್ನು ಅನುಸರಿಸುವುದಕ್ಕಲ್ಲ.

೨. ನೀನು ದೇಹವೆಂದು ಭಾವಿಸಿದಾಗ ವಿಶ್ವದ ಒಂದು ಅಂಶ, ನೀನು ಜೀವವೆಂದೆಣಿಸಿದಾಗ ಆ ಸನಾತನ ಪುರುಷನ ಒಂದು ಕಿಡಿ, ನೀನು ಆತ್ಮನೆಂದು ಬಗೆದಾಗ ನೀನೇ ಸರ್ವವೂ ಆಗಿರುವೆ.

೩. ಇಚ್ಛೆ ಸ್ವತಂತ್ರವಲ್ಲ, ಕಾರ್ಯಕಾರಣಗಳ ಪರಂಪರೆಯಿಂದ ಬದ್ಧವಾದ ಒಂದು ಘಟನೆ. ಆದರೆ ಇಚ್ಛೆಯ ಹಿಂದೆ ಸ್ವತಂತ್ರವಾಗಿರುವುದೊಂದು ಇದೆ.

೪. ಶಕ್ತಿ ಇರುವುದು ಸಾಧುಸ್ವಭಾವದಲ್ಲಿ, ಚಾರಿತ್ರ್ಯ ಶುದ್ಧಿಯಲ್ಲಿ.

೫. ವಿಶ್ವವೇ ದೃಶ್ಯರೂಪವನ್ನು ಧರಿಸಿರುವ ದೇವರು.

೬. ನಿಮ್ಮಲ್ಲಿ ನಿಮಗೆ ಶ್ರದ್ಧೆ ಉದಯಿಸುವ ತನಕ ದೇವರಲ್ಲಿ ಶ್ರದ್ಧೆ ಉದಯಿಸಲಾರದು.

೭. ನಾನು ದೇಹವೆಂದು ಭಾವಿಸುವುದೇ ಪಾಪದ ಮೂಲ. ಯಾವುದಾದರೂ ಮೂಲ ಪಾಪವಿದ್ದರೆ ಅದೇ ಇದು.

೮. ಒಂದು ಪಕ್ಷದವರು ದ್ರವ್ಯವು \enginline{(matter)} ಭಾವನೆಗೆ \enginline{(idea)} ಕಾರಣ ಎನ್ನುತ್ತಾರೆ, ಮತ್ತೊಂದು ಪಕ್ಷದವರು ಭಾವನೆ ದ್ರವ್ಯಕ್ಕೆ ಎನ್ನುತ್ತಾರೆ. ಇಬ್ಬರೂ ತಪ್ಪು. ದ್ರವ್ಯ ಮತ್ತು ಭಾವನೆ ಏಕಕಾಲದಲ್ಲಿ ಇರುವುವು, ಭಾವನೆ ಮತ್ತು ದ್ರವ್ಯಗಳೆರಡೂ ಮತ್ತಾವುದೋ ಒಂದು ಮೂರನೆಯ ವಸ್ತುವಿನಿಂದ ಆಗಿವೆ.

೯. ದೇಹದಲ್ಲಿ ಹೇಗೆ ದ್ರವ್ಯಕಣಗಳು ಒಂದು ಕಡೆ ಸೇರುವುವೋ ಹಾಗೆಯೇ ಕಾಲದಲ್ಲಿ ಆಲೋಚನಾ ತರಂಗಗಳು ಒಂದುಗೂಡುವುವು.

೧೦. ದೇವರನ್ನು ವಿವರಿಸಲು ಯತ್ನಿಸುವುದು ಸತ್ತ ಹಾವನ್ನು ಕೊಲ್ಲಲು ಯತ್ನಿಸುವಂತೆ, ಏಕೆಂದರೆ ನಮಗೆ ಗೊತ್ತಿರುವ ಏಕಮಾತ್ರ ವಸ್ತುವೇ ದೇವರು.

೧೧. ಮೃಗಸದೃಶನಾದವನನ್ನು ಮಾನವನನ್ನಾಗಿ ಮಾಡುವ, ಮಾನವನನ್ನು ದೇವನನ್ನಾಗಿ ಮಾಡುವ ಭಾವನೆಯೇ ದೇವರು.

೧೨. ಆಂತರಿಕ ಪ್ರಕೃತಿಯನ್ನು ಬಾಹ್ಯ ಆಕಾರದಲ್ಲಿ ಕಾಣುವಂತೆ ಮಾಡುವುದೇ ಬಾಹ್ಯ ಪ್ರಕೃತಿ.

೧೩. ನಿನ್ನ ಕರ್ಮದ ಯೋಗ್ಯತೆಯನ್ನು ನಿರ್ಧರಿಸುವುದು ಅದರ ಹಿಂದೆ ಇರುವ ಉದ್ದೇಶ. ನೀನೂ ದೇವರು ಮತ್ತು ಅತಿ ಕೀಳು ಮಾನವನು ಕೂಡ ದೇವರು – ಎಂಬ ಭಾವನೆಗಿಂತ ಯಾವ ಉದ್ದೇಶ ಮಿಗಿಲಾಗಬಲ್ಲದು?

೧೪. ಮನೋಪ್ರಪಂಚವನ್ನು ಈಕ್ಷಿಸುವವನು ಬಲಶಾಲಿಯಾಗಿರಬೇಕು ಮತ್ತು ವೈಜ್ಞಾನಿಕ ಶಿಕ್ಷಣವುಳ್ಳವನಿರಬೇಕು.

೧೫. ಮನಸ್ಸೇ ಸರ್ವವು, ಆಲೋಚನೆಯೇ ಸರ್ವವು, ಎಂಬುದು ಮೇಲು ತರದ ಜಡವಾದವಷ್ಟೆ.

೧೬. ಈ ಪ್ರಪಂಚ ಒಂದು ದೊಡ್ಡ ಗರಡಿಯ ಮನೆ. ನಾವಿಲ್ಲಿ ಬಲಿಷ್ಠರಾಗುವುದಕ್ಕೆ ಬಂದಿರುವೆವು.

೧೭. ಮಗುವನ್ನು ತರಬೇತು ಮಾಡುವುದು ಒಂದು ಸಸಿಯನ್ನು ಬೆಳೆಸಿದಂತೆ. ಅದಕ್ಕಿಂತ ಹೆಚ್ಚಿಗೆ ಮಾಡಲಾರೆವು, ನೀವು ಅದಕ್ಕೆ ಇರುವ ಅಡಚಣೆಗಳನ್ನು ನಿವಾರಿಸಬೇಕಾಗಿದೆ. ಅದಕ್ಕೆ ನೀವು ಸಹಾಯವನ್ನು ಮಾತ್ರ ಮಾಡಬಲ್ಲಿರಿ. ಅದು ಆಂತರ್ಯದಿಂದ ವಿಕಾಸವಾಗುವುದು. ಅದು ತನ್ನ ಸ್ವಭಾವವನ್ನು ತಾನೇ ಸೃಷ್ಟಿಸಿಕೊಳ್ಳುವುದು. ಅದಕ್ಕೆ ಇರುವ ಆತಂಕಗಳನ್ನು ಮಾತ್ರ ನಿವಾರಿಸಬಹುದು.

೧೮. ನೀವು ಒಂದು ಪಂಗಡವನ್ನು ಮಾಡಿದೊಡನೆಯೆ ವಿಶ್ವ ಸಹೋದರತ್ವವನ್ನು ಧಿಃಕರಿಸುತ್ತೀರಿ. ಯಾರು ನಿಜವಾಗಿಯೂ ವಿಶ್ವ ಸಹೋದರತ್ವದ ಭಾವನೆಯನ್ನು ಅನುಭವಿಸುವರೋ ಅವರು ಹೆಚ್ಚು ಮಾತನಾಡುವುದಿಲ್ಲ. ಆದರೆ ಅವರ ನಡುವಳಿಕೆಯೇ ಅದನ್ನು ಸಾರಿ ಹೇಳುವುದು.

೧೯. ಸತ್ಯವನ್ನು ಸಹಸ್ರಾರು ವಿಧಗಳಲ್ಲಿ ಹೇಳಬಹುದು, ಪ್ರತಿಯೊಂದು ವಿಧವೂ ಸತ್ಯವಾಗಿರಬಹುದು.

೨೦. ನೀವು ಒಳಗಿನಿಂದ ಹೊರಗೆ ಬೆಳೆಯಬೇಕಾಗಿದೆ. ಯಾರೂ ನಿಮಗೆ ಬೋಧಿಸಲಾರರು. ಯಾರೂ ನಿಮ್ಮನ್ನು ಆಧ್ಯಾತ್ಮಿಕ ಜೀವಿಗಳನ್ನಾಗಿ ಮಾಡಲಾರರು. ನಿಮಗೆ ನಿಮ್ಮ ಆತ್ಮನಲ್ಲದೆ ಬೇರೆ ಗುರುವಿಲ್ಲ.

೨೧. ಒಂದು ಅನಂತ ಸರಪಳಿಯಲ್ಲಿ ಕೆಲವು ಕೊಂಡಿಗಳನ್ನು ವಿವರಿಸಲು ಸಾಧ್ಯವಾದರೆ ಅದರ ಆಧಾರದ ಮೇಲೆಯೇ ಇಡಿಯ ಸರಪಳಿಯನ್ನೂ ವಿವರಿಸಬಹುದು.

೨೨. ಪ್ರಪಂಚದ ಯಾವ ಘಟನೆಗಳಿಂದಲೂ ವಿಚಲಿತನಾಗದವನು ಅಮೃತಾತ್ಮನಾಗಿರುವನು.

೨೩. ಸತ್ಯಕ್ಕಾಗಿ ಸರ್ವವನ್ನೂ ಸಮರ್ಪಿಸಬಹುದು, ಆದರೆ ಸತ್ಯವನ್ನು ಮತ್ತಾವುದಕ್ಕೂ ತೆರುವುದಕ್ಕೆ ಆಗುವುದಿಲ್ಲ.

೨೪. ಸತ್ಯಾನ್ವೇಷಣೆ ಶಕ್ತಿಯ ಚಿಹ್ನೆ, ಅದು ದುರ್ಬಲನಾದ ಅಂಧನ ತಡಕಾಟವಲ್ಲ.

೨೫. ದೇವರು ಮಾನವನಾಗಿರುವನು; ಪುನಃ ಮಾನವ ದೇವರಾಗುವನು.

೨೬. ಮನುಷ್ಯ ಸಾಯುವನು, ಸ್ವರ್ಗಕ್ಕೆ ಹೋಗುವನು ಎಂಬ ಮುಂತಾದುವೆಲ್ಲ ಮಕ್ಕಳಾಟ. ನಾವೆಂದೂ ಬರುವುದೂ ಇಲ್ಲ, ಹೋಗುವುದೂ ಇಲ್ಲ. ನಾವಿರುವೆಡೆಯಲ್ಲಿಯೇ ಇರುವೆವು. ಹಿಂದೆ ಇದ್ದ, ಈಗಿರುವ, ಮುಂದೆ ಬರುವ ಆತ್ಮರೆಲ್ಲ ಒಂದೇ ಕಡೆ ಇರುವರು.

೨೭. ಯಾರ ಹೃದಯಗ್ರಂಥ ತೆರೆದಿದೆಯೋ ಅವರಿಗೆ ಬೇರೆ ಗ್ರಂಥಗಳು ಬೇಕಾಗಿಲ್ಲ. ನಮ್ಮಲ್ಲಿ ಆಸೆಯನ್ನು ಪ್ರಚೋದಿಸುವುದು ಮಾತ್ರವೇ ಅವುಗಳ ಗುರಿಯಾಗಿದೆ! ಅವೆಲ್ಲ ಬೇರೆಯವರ ಅನುಭವ ಮಾತ್ರ.

೨೮. ಎಲ್ಲರಿಗೂ ಅನುಕಂಪವನ್ನು ತೋರು, ಕಷ್ಟದಲ್ಲಿರುವವರಿಗಾಗಿ ಮರುಗು. ಎಲ್ಲರನ್ನೂ ಪ್ರೀತಿಸು, ಯಾರ ಮೇಲೂ ಅಸೂಯೆ ಇಲ್ಲದಿರಲಿ. ಇತರರ ತಪ್ಪನ್ನು ಗಮನಿಸಬೇಡ.

೨೯. ಮನುಷ್ಯ ಎಂದಿಗೂ ಸಾಯುವುದಿಲ್ಲ, ಅವನೆಂದಿಗೂ ಹುಟ್ಟಿಯೂ ಇಲ್ಲ. ದೇಹವು ನಾಶವಾಗುವುದು, ಆದರೆ ಅವನೆಂದೂ ನಾಶವಾಗುವುದಿಲ್ಲ.

೩೦. ಯಾರೂ ಧರ್ಮಕ್ಕಾಗಿ ಹುಟ್ಟುವುದಿಲ್ಲ, ಅವರು ಅನಂತರ ಅದನ್ನು ಆರಿಸಿಕೊಳ್ಳಬೇಕಾಗಿದೆ.

೩೧. ವಿಶ್ವದಲ್ಲಿ ಸತ್ಯವಾಗಿರುವುದು ಒಂದೇ ಆತ್ಮ, ಉಳಿದುವುಗಳೆಲ್ಲ ಅದರ ಆವಿರ್ಭಾವಗಳು ಮಾತ್ರ.

೩೨. ಭಕ್ತರಲ್ಲಿ ಸಾಮಾನ್ಯಕೋಟಿ ಮತ್ತು ಧೀರರಾದ ಕೆಲವರು ಎಂದು ಎರಡು ವಿಭಾಗಗಳಿವೆ.

೩೩. ಈಗ ಇಲ್ಲಿ ಪೂರ್ಣತೆಯನ್ನು ಪಡೆಯುವುದಕ್ಕೆ ಆಗದೆ ಇದ್ದರೆ, ಇನ್ನಾವುದೋ ಜನ್ಮದಲ್ಲಿ ಅದನ್ನು ಪಡೆಯುತ್ತೇನೆ ಎನ್ನುವುದಕ್ಕೆ ದಾಖಲೆಯಿಲ್ಲ.

೩೪. ನಾನು ಒಂದು ಹಿಡಿ ಮೃತ್ತಿಕೆಯನ್ನು ಚೆನ್ನಾಗಿ ತಿಳಿದುಕೊಂಡರೆ, ಪ್ರಪಂಚದಲ್ಲಿರುವ ಮೃತ್ತಿಕೆಯನ್ನೆಲ್ಲ ತಿಳಿದುಕೊಂಡಂತೆ. ಇದು ಮೂಲಸಿದ್ಧಾಂತಗಳ ಜ್ಞಾನ, ಆದರೆ ಅವು ಒಂದೊಂದು ವಾತಾವರಣದಲ್ಲಿ ಒಂದೊಂದು ಬಗೆಯಾಗಿ ರೂಪತಾಳುವುವು. ನೀನು ನಿನ್ನನ್ನು ಅರಿತರೆ ಎಲ್ಲವನ್ನೂ ಅರಿತಂತೆ.

೩೫. ನಾನು ವೇದದಲ್ಲಿ ಯಾವುದು ಯುಕ್ತಿಗೆ ಸಮಂಜಸವಾಗಿರುವುದೋ ಅದನ್ನು ಮಾತ್ರ ಸ್ವೀಕರಿಸುತ್ತೇನೆ. ವೇದಗಳಲ್ಲಿ ಎಷ್ಟೋ ಕಡೆ ವಿರೋಧಾಭಾಸವಿದೆ. ಪಾಶ್ಚಾತ್ಯ ದೃಷ್ಟಿಯಲ್ಲಿ ಅವು ಸ್ಫೂರ್ತಿಗೊಂಡು ಬಂದವುಗಳಲ್ಲ; ಆದರೆ ಅವೆಲ್ಲ ಭಗವಂತನ ಒಟ್ಟು ಜ್ಞಾನ–ಸರ್ವಜ್ಞಾನಗಳೂ ಅಲ್ಲಿವೆ. ಪ್ರತಿ ಕಲ್ಪದ ಆದಿಯಲ್ಲಿಯೂ ಈ ಜ್ಞಾನ ವ್ಯಕ್ತವಾಗುವುದು. ಕಲ್ಪ ಕೊನೆಗೊಂಡಾಗ ಅದು ಸೂಕ್ಷ್ಮ ಅವಸ್ಥೆಗೆ ಹೋಗುವುದು. ಸೃಷ್ಟಿ ಪುನಃ ವ್ಯಕ್ತವಾದಾಗ ಆ ಜ್ಞಾನವೂ ವ್ಯಕ್ತವಾಗುವುದು. ಈ ದೃಷ್ಟಿಯಿಂದ ಸಿದ್ಧಾಂತ ಸರಿಯಾಗಿರುವುದು. ಆದರೆ ವೇದಗಳೆಂದು ಕರೆಯಲ್ಪಡುವ ಈ ಗ್ರಂಥಗಳು ಮಾತ್ರ ಭಗವಂತನ ಜ್ಞಾನವೆಂದು ಹೇಳುವುದು ಕುತರ್ಕ. ಮನು ಒಂದು ಕಡೆ ವೇದದ ಯಾವ ಭಾಗ ಯುಕ್ತಿಗೆ ಸಮಂಜಸವಾಗಿರುವುದೋ ಅದು ಮಾತ್ರ ಸರಿ, ಮಿಕ್ಕವುಗಳು ಅಲ್ಲ ಎನ್ನುವನು. ನಮ್ಮಲ್ಲಿ ಹಲವು ದಾರ್ಶನಿಕರು ಈ ದೃಷ್ಟಿಯಿಂದ ನೋಡುವರು.

೩೬. ಪ್ರಪಂಚದ ಶಾಸ್ತ್ರಗಳಲ್ಲೆಲ್ಲಾ ವೇದ ಒಂದೇ. ವೇದಾಧ್ಯಯನವೂ ಓದುವುದೂ ಕೂಡ ಗೌಣ ಎಂದು ಸಾರುವುದು. ನಿಜವಾದ ಅಧ್ಯಯನ ಯಾವುದರಿಂದ ನಾವು ನಿತ್ಯವಾದುದನ್ನು ತಿಳಿಯುವೆವೊ ಅದು. ಅದು ಓದುವುದೂ ಅಲ್ಲ, ನಂಬಿಕೆಯೂ ಅಲ್ಲ, ಯುಕ್ತಿಯೂ ಅಲ್ಲ, ಅದು ಸಮಾಧಿ.

೩೭. ನಾವೆಲ್ಲಾ ಒಂದಾನೊಂದು ಕಾಲದಲ್ಲಿ ಕೀಳು ಪ್ರಾಣಿಗಳಾಗಿದ್ದೆವು. ಅವು ನಮ್ಮಿಂದ ಸಂಪೂರ್ಣ ಬೇರೆ ಎಂದು ಭಾವಿಸುವೆವು. ಪಾಶ್ಚಾತ್ಯರು ಪ್ರಪಂಚ ನಮಗಾಗಿ ಸೃಷ್ಟಿಯಾಗಿದೆ ಎನ್ನುವರು. ಹುಲಿಗಳೇನಾದರೂ ಗ್ರಂಥ ಬರೆದರೆ, ಮನುಷ್ಯ ತನಗಾಗಿ ಸೃಷ್ಟಿಯಾಗಿದ್ದು, ಅವನೊಬ್ಬ ಮಹಾಪಾಪಿ, ಏಕೆಂದರೆ ಅವನು ಸುಲಭವಾಗಿ ಸಿಕ್ಕುವುದಿಲ್ಲ ಎಂದು ಬರೆಯಬಹುದು. ಇಂದು ನಿನ್ನ ಕಾಲಿನ ಕೆಳಗೆ ಹರಿದಾಡುತ್ತಿರುವ ಕೀಟ ಭವಿಷ್ಯದ ದೇವರು.

೩೮. ಸ್ವಾಮಿ ವಿವೇಕಾನಂದರು ನ್ಯೂಯಾರ್ಕಿನಲ್ಲಿ ಹೀಗೆ ಹೇಳಿದರು: "ನಮ್ಮ ಸ್ತ್ರೀಯರು ನಿಮ್ಮಷ್ಟು ಚತುರರಾಗಲಿ ಎಂದು ಆಶಿಸುವೆನು. ಆದರೆ ಇದು ಚಾರಿತ್ರ್ಯದ ಶುದ್ಧಿಗೆ ಕಳಂಕವಿಲ್ಲದೆ ಆಗಬೇಕು. ನಿಮ್ಮ ಜ್ಞಾನವನ್ನು ನೋಡಿ ನಾನು ತಲೆದೂಗುವೆನು. ಆದರೆ ನೀವು ಕೊಳೆತು ನಾರುವ ವಸ್ತುವಿನ ಮೇಲೆ ಗುಲಾಬಿಯ ಹೂವಿನ ರಾಶಿಯನ್ನು ಹೇರಿ ಅದು ಚೆನ್ನಾಗಿದೆ ಎಂದು ಹೇಳುವುದನ್ನು ನಾನು ಅನುಮೋದಿಸುವುದಿಲ್ಲ. ಚತುರತೆಯು ಶ್ರೇಷ್ಠವಲ್ಲ. ನೀತಿ ಮತ್ತು ಧರ್ಮ ಇವಕ್ಕಾಗಿ ನಾವು ಸಾಧನೆ ಮಾಡುವುದು. ನಮ್ಮ ಸ್ತ್ರೀಯರು ಹೆಚ್ಚು ವಿದ್ಯಾವಂತರಲ್ಲ, ಆದರೆ ಅವರು ಹೆಚ್ಚು ಪರಿಶುದ್ಧರು."

ಸ್ತ್ರೀಗೆ, ತನ್ನ ಗಂಡನಲ್ಲದವರೆಲ್ಲಾ ಮಕ್ಕಳಂತೆ ಇರಬೇಕು. "ಪ್ರತಿಯೊಬ್ಬ ಪುರುಷನಿಗೂ ತನ್ನ ಹೆಂಡತಿಯಲ್ಲದೆ ಉಳಿದ ಸ್ತ್ರೀಯರೆಲ್ಲ ತನ್ನ ತಾಯಿಯಂತೆ ಇರಬೇಕು. ನೀವು ಯಾವುದನ್ನು ಪೌರುಷ \enginline{(gallantry)} ಎನ್ನುವಿರೋ ಅದನ್ನು ನೋಡಿದಾಗ ನನಗೆ ಜುಗುಪ್ಸೆ ಹುಟ್ಟುವುದು. ನೀವು ಲಿಂಗಭೇದವನ್ನು ಮರೆತು ಮಾನವ ಕೋಟಿಯನ್ನು ಸಮಾನ ದೃಷ್ಟಿಯಿಂದ ನೋಡಿದಾಗ ಮಾತ್ರ ನಿಮ್ಮ ಸ್ತ್ರೀಯರು ಮುಂದುವರಿಯುವರು. ಅಲ್ಲಿಯವರೆಗೆ ಅವರು ನಿಮ್ಮ ಕ್ರೀಡಾವಸ್ತುಗಳು, ಅದಕ್ಕಿಂತ ಹೆಚ್ಚಲ್ಲ. ಇದೇ ದಾಂಪತ್ಯವಿಚ್ಛೇದನಕ್ಕೆ ಕಾರಣ. ನಿಮ್ಮ ಪುರುಷರು ಬಾಗಿ ನಿಮಗೆ ಕುಳಿತುಕೊಳ್ಳಲು ಕುರ್ಚಿ ಕೊಡುವರು. ಆದರೆ ಮರುಕ್ಷಣವೇ ನಿಮ್ಮನ್ನು ಹೊಗಳುವರು. ಅವರು ಓ ತರುಣಿ, ನಿನ್ನ ಕಂಗಳು ಎಷ್ಟು ಸುಂದರವಾಗಿವೆ ಎನ್ನುವರು. ಇದನ್ನು ಮಾಡುವುದಕ್ಕೆ ಅವರಿಗೆ ಏನು ಅಧಿಕಾರವಿದೆ? ಇಷ್ಟು ದೂರ ಹೋಗುವುದಕ್ಕೆ ಗಂಡಸಿಗೆ ಎಷ್ಟು ಧೈರ್ಯ? ನಿಮ್ಮ ಸ್ತ್ರೀಯರು ಕೂಡ ಇದಕ್ಕೆ ಹೇಗೆ ಅನುಮತಿಯನ್ನು ಕೊಡುವರು? ಇದು ಭವ್ಯಭಾವನೋದ್ದೀಪಕವಲ್ಲ; ಇದು ಹೀನ ಆಸೆಗಳನ್ನು ಪ್ರಚೋದಿಸುವುದು."

“ನಾವು ಸ್ತ್ರೀಪುರುಷರೆಂದು ಭಾವಿಸಕೂಡದು. ಒಬ್ಬರು ಇನ್ನೊಬ್ಬರಿಗೆ ನೆರವಾಗುವ ಮಾನವರು ಎಂಬ ದೃಷ್ಟಿಯಿಂದ ನೋಡಬೇಕು. ತರುಣತರುಣಿಯರು ಇಬ್ಬರೇ ಇದ್ದರೆ ಸಾಕು, ಆಗಲೇ ಯುವಕ ಅವಳನ್ನು ಶ್ಲಾಘಿಸುವನು. ಅವನು ಒಬ್ಬಳನ್ನು ಮದುವೆಯಾಗುವುದಕ್ಕೆ ಮುಂಚೆ ಆಗಲೇ ಇನ್ನೂರು ಸ್ತ್ರೀಯರನ್ನು ಪ್ರೀತಿಸಿರಬಹುದು. ನಾನು ಏನಾದರೂ ಮದುವೆಯಾಗುವ ಗುಂಪಿಗೆ ಸೇರಿದ್ದರೆ, ಇದಾವುದೂ ಇಲ್ಲದೆ ಸ್ತ್ರೀಯನ್ನು ಪ್ರೀತಿಸುತ್ತಿದ್ದೆ.”

"ನಾನು ಭರತಖಂಡದಲ್ಲಿದ್ದಾಗ ಇದನ್ನೆಲ್ಲಾ ದೂರದಿಂದ ನೋಡಿದಾಗ, ಇದೆಲ್ಲ ಸರಿ, ಇದೊಂದು ತಮಾಷೆ ಎಂದು ಹೇಳಿದ್ದರು. ನಾನೂ ಇದನ್ನು ನಂಬಿದ್ದೆ. ಆದರೆ ನಾನು ಅನಂತರ ಪ್ರಪಂಚವನ್ನು ಅಲೆದು ನೋಡಿದ ಮೇಲೆ ಇದು ಸರಿಯಲ್ಲ ಎಂದು ತೋರುವುದು. ಇದು ತಪ್ಪು, ಪಾಶ್ಚಾತ್ಯರಾದ ನೀವು ಕಣ್ಣು ಮುಚ್ಚಿಕೊಂಡು ಇದು ಸರಿ ಎನ್ನುವಿರಿ. ಪಾಶ್ಚಾತ್ಯರಲ್ಲಿರುವ ತೊಂದರೆ ಏನು ಎಂದರೆ ಅವರಿನ್ನೂ ಬಾಲಬುದ್ಧಿಯವರು, ಹುಡುಗಾಟಿಕೆಯವರು, ಚಂಚಲ ಚಿತ್ತರು; ಅವರಿಗೆ ಬೇಕಾದಷ್ಟು ಐಶ್ವರ್ಯ ಬೇರೆ ಇದೆ. ಇವುಗಳಲ್ಲಿ ಯಾವುದಾದರೂ ಒಂದು ಇದ್ದರೂ ಸಾಕು ನಮ್ಮ ಸರ್ವನಾಶಕ್ಕೆ. ಆದರೆ ಈ ಮೂರು ನಾಲ್ಕು ಎಲ್ಲಾ ಒಟ್ಟಿಗೆ ಸೇರಿದರೆ ಏನು ಗತಿ?"

ಸ್ವಾಮಿಜಿಯವರು ಎಲ್ಲರ ಮೇಲೂ ನಿಷ್ಟುರದಿಂದಿದ್ದರೂ, ಬಾಸ್ಟನ್ ಜನರ ವಿಷಯದಲ್ಲಿ ಬಹಳ ಉಗ್ರವಾಗಿದ್ದರು.

"ನಗರಗಳಲ್ಲಿ ಬಾಸ್ಟನ್ ಅತಿ ಹೋಲಸು, ಇಲ್ಲಿನ ಹೆಂಗಸರಿಗೆಲ್ಲಾ ಏನಾದರೂ ಒಂದು ಷೋಕಿ (ಕಯಾಲಿ) ಬೇಕು. ಅವರು ಚಂಚಲ ಚಿತ್ತರು. ಏನಾದರೂ ಕೌತುಕವಾಗಿರುವುದನ್ನು ನವೀನವಾಗಿರುವುದನ್ನು ಯಾವಾಗಲೂ ಅನುಸರಿಸುವುದಕ್ಕೆ ಕಾತರರಾಗಿರುವರು."

೩೯. "ನಾಗರಿಕತೆಯಲ್ಲಿ ತನಗೆ ಸಮಾನರು ಯಾರೂ ಇಲ್ಲ ಎಂದು ಹೆಮ್ಮೆ ಪಡುವ ಈ ಅಮೆರಿಕಾ ದೇಶದಲ್ಲಿ ಆಧ್ಯಾತ್ಮಿಕತೆ ಎಲ್ಲಿ?” ಎಂದು ಸ್ವಾಮಿ ವಿವೇಕಾನಂದರು ಕೇಳಿದರು.

೪೦. 'ಇಲ್ಲಿ' ಮತ್ತು 'ಅಲ್ಲಿ' ಇವೆಲ್ಲ ಮಕ್ಕಳನ್ನು ಅಂಜಿಸುವ ಪದಗಳು. ಇವೆಲ್ಲ ಇರುವುದಿಲ್ಲ. ಇಲ್ಲಿ ಈ ದೇಹದಲ್ಲಿರುವಾಗ ದೇವರಲ್ಲಿ ಬಾಳಬೇಕು ಬದುಕಬೇಕು. ಸ್ವಾರ್ಥತೆ ಎಲ್ಲ ನಾಶವಾಗಬೇಕು, ಮೌಡ್ಯವನ್ನೆಲ್ಲ ತ್ಯಜಿಸಬೇಕು. ಅಂತಹ ಜನರು ಭರತಖಂಡದಲ್ಲಿರುವರು. ಆದರೆ ನಿಮ್ಮ ದೇಶದಲ್ಲಿ (ಅಮೆರಿಕಾದಲ್ಲಿ) ಅವರೆಲ್ಲಿ? ನಿಮ್ಮ ಪಾದ್ರಿಗಳು ಕನಸುಣಿಗಳ ವಿರುದ್ಧವಾಗಿ ಮಾತನಾಡುತ್ತಾರೆ. ಇಂತಹ ಕನಸುಣಿಗಳಿದ್ದರೆ ನಿಮ್ಮ ದೇಶ ಎಷ್ಟೋ ಉತ್ತಮವಾಗುತ್ತಿತ್ತು. ಕನಸುಣಿಗೂ, ಹತ್ತೊಂಬತ್ತನೇ ಶತಮಾನದಲ್ಲಿ ಹೆಮ್ಮೆಪಡುವವರಿಗೂ ಎಷ್ಟೋ ವ್ಯತ್ಯಾಸವಿದೆ. ಪ್ರಪಂಚ ದೇವರಿಂದ ತುಂಬಿತುಳುಕಾಡುತ್ತಿದೆ, ಪಾಪದಿಂದ ಅಲ್ಲ. ಒಬ್ಬರು ಮತ್ತೊಬ್ಬರಿಗೆ ನೆರವಾಗಲಿ, ಒಬ್ಬರು ಮತ್ತೊಬ್ಬರನ್ನು ಪ್ರೀತಿಸಲಿ.

೪೧. ನನ್ನ ಗುರುವಿನಂತೆ, ನಿಜವಾದ ಸಂನ್ಯಾಸಿಯಂತೆ, ನಾನು ಸಾಯಲಿಚ್ಛಿಸುವೆನು. ಅವರು ಕಾಮಕಾಂಚನಗಳಿಗೆ, ಯಶಸ್ಸಿನ ಆಸೆಗೆ ಸ್ವಲ್ಪವೂ ಗಮನಕೊಟ್ಟವರಲ್ಲ. ಇವುಗಳಲ್ಲಿ ಅತಿ ಹೀನವಾದುದೇ ಕೀರ್ತಿಯ ಆಸೆ.

೪೨. ನಾನು ಎಂದಿಗೂ ಛಲವನ್ನು ತೀರಿಸಿಕೊಳ್ಳುವ ದೃಷ್ಟಿಯಿಂದ ಮಾತನಾಡಲಿಲ್ಲ. ನಾನು ಯಾವಾಗಲೂ ಶೌರ್ಯವನ್ನು ಹೇಳುತ್ತಿದ್ದೆ. ಈ ಒಂದು ಕ್ಷಣಿಕ ಸಾಗರದ ನೊರೆಯ ಮೇಲಿರುವ ನಾವು ಛಲವನ್ನು ತೀರಿಸಿಕೊಳ್ಳುವ ಕನಸನ್ನಾದರೂ ಕಾಣುವೆವೆ? ಇಲ್ಲ, ಆದರೆ ಸೊಳ್ಳೆಗೆ ಅದೊಂದು ಬಹಳ ಪ್ರಮುಖವಾದುದು.

೪೩. “ಇದೊಂದು (ಅಮೆರಿಕಾ) ದೊಡ್ಡ ದೇಶ. ಆದರೆ ನನಗೆ ಇಲ್ಲಿರಲು ಇಚ್ಛೆಯಿಲ್ಲ. ಅಮೆರಿಕಾದವರಿಗೆ ಹಣವೇ ಪ್ರಧಾನ. ಅವರಿಗೆ ಎಲ್ಲಕ್ಕಿಂತ ಇದೇ ಮುಖ್ಯ. ನಿಮ್ಮ ಜನರು ಇನ್ನೂ ಕಲಿಯಬೇಕಾದ್ದು ಬೇಕಾದಷ್ಟು ಇದೆ. ನಿಮಗೂ ನಮ್ಮಷ್ಟೇ ವಯಸ್ಸಾದಾಗ ನೀವು ಜ್ಞಾನವೃದ್ಧರಾಗುವಿರಿ.”

೪೪. ನಾನು ಈ ದೇಹಾತೀತವಾಗುವುದಕ್ಕೆ, ಈ ದೇಹವನ್ನು ಹಳೆಯ ಬಟ್ಟೆಯಂತೆ ಕಿತ್ತೊಗೆಯುವುದು ನನಗೆ ಮೇಲಾಗಬಹುದು. ಆದರೂ ನಾನು ಕೆಲಸ ಮಾಡುವುದನ್ನು ಬಿಡುವುದಿಲ್ಲ. ಈ ಪ್ರಪಂಚದಲ್ಲಿರುವವರೆಲ್ಲ ತಾವು ದೇವರು ಎಂದು ಭಾವಿಸುವವರೆಗೆ ಎಲ್ಲರಿಗೂ ನಾನು ಸ್ಫೂರ್ತಿಯನ್ನು ಕೊಡುವೆನು.

೪೫. ನಾನು ಏನಾಗಿರುವೆನೋ ಅದು ಮತ್ತು ಮುಂದೆ ಪ್ರಪಂಚ ಏನಾಗುವುದೊ ಅದು ನನ್ನ ಗುರುದೇವ ಶ‍್ರೀ ರಾಮಕೃಷ್ಣರ ಅನುಗ್ರಹದಿಂದ. ಅವರು ಅವತರಿಸಿ ಎಲ್ಲಾ ವಸ್ತುಗಳ ಅಂತರಾಳದಲ್ಲಿರುವ ಅದ್ಭುತವಾದ ಸತ್ಯವನ್ನು ಅನುಭವಿಸಿ ಒಂದೇ ಸತ್ಯ ಹಿಂದೂ–ಮುಸಲ್ಮಾನ–ಕ್ರಿಸ್ತಧರ್ಮಗಳಲ್ಲಿದೆ ಎಂಬುದನ್ನು ಬೋಧಿಸಿದರು.

೪೬. ನೀವು ನಾಲಗೆಯನ್ನು ಹರಿಯಬಿಟ್ಟರೆ ಅನಂತರ ಇತರ ಇಂದ್ರಿಯಗಳೂ ತಮ್ಮಿಚ್ಛೆ ಬಂದಂತೆ ಹೋಗುವುವು.

೪೭. ಜ್ಞಾನ ಭಕ್ತಿ ಯೋಗ ಕರ್ಮ–ಇವು ನಾಲ್ಕೂ ಮೋಕ್ಷ ಮಾರ್ಗಗಳು. ಒಬ್ಬ ತಾನು ಯಾವುದಕ್ಕೆ ಯೋಗ್ಯನೋ ಆ ಮಾರ್ಗವನ್ನು ಅನುಸರಿಸಬೇಕು. ಆದರೆ ಈ ಯುಗದಲ್ಲಿ ಕರ್ಮಕ್ಕೆ ಹೆಚ್ಚು ಪ್ರಾಧಾನ್ಯವನ್ನು ಕೊಡಬೇಕು.

೪೮. ಧರ್ಮ ಊಹೆಯಲ್ಲ, ಪ್ರತ್ಯಕ್ಷ ಅನುಭವ. ಯಾರು ಒಂದು ಪ್ರೇತವನ್ನಾದರೂ ನೋಡಿರುವನೋ ಅವನು ಅರಗಿಳಿಯಂತೆ ಇರುವ ಹಲವು ಪಂಡಿತರಿಗಿಂತ ಮೇಲು.

೪೯. ಒಂದು ಸಲ ಸ್ವಾಮೀಜಿ ಅವರು ಯಾರನ್ನೋ ಶ್ಲಾಘಿಸುತ್ತಿದ್ದರು. ಆಗ ಅವರ ಪಕ್ಕದಲ್ಲಿದ್ದವರೊಬ್ಬರು "ಅವರು ನಿಮ್ಮನ್ನು ನಂಬುವುದಿಲ್ಲ" ಎಂದರು. ಆಗ ಸ್ವಾಮೀಜಿ ತಕ್ಷಣ "ಅವನು ಹೀಗೆ ನಂಬಬೇಕೆಂದು ಏನಾದರೂ ಕಾನೂನು ಇದೆಯೊ? ಅವನು ಒಳ್ಳೆಯ ಕೆಲಸ ಮಾಡುತ್ತಿರುವುದರಿಂದ ಶ್ಲಾಘನೆಗೆ ಯೋಗ್ಯ" ಎಂದರು.

೫೦. ನಿಜವಾದ ಧಾರ್ಮಿಕ ಪ್ರಪಂಚದಲ್ಲಿ ಪುಸ್ತಕ ಪಾಂಡಿತ್ಯಕ್ಕೆ ಪ್ರವೇಶವಿಲ್ಲ.

೫೧. ಶ‍್ರೀಮಂತರ ಆರಾಧನೆಯು ಪ್ರವೇಶಿಸಿದೊಡನೆಯೇ ಒಂದು ಧಾರ್ಮಿಕ ಪಂಗಡದ ಅವನತಿ ಪ್ರಾರಂಭವಾದಂತೆ.

೫೨. ನೀನು ಏನಾದರೂ ದುಷ್ಕೃತ್ಯವನ್ನು ಮಾಡಬೇಕಾದರೆ ಹಿರಿಯರೆದುರಿಗೆ ಮಾಡು.

೫೩. ಗುರುಗಳ ಅನುಗ್ರಹದಿಂದ ಶಿಷ್ಯ, ಯಾವ ಗ್ರಂಥವನ್ನೂ ಓದದೇ ಇದ್ದರೂ, ಉದ್ಧಾಮ ಪಂಡಿತನಾಗುವನು.

೫೪. ಪಾಪಪುಣ್ಯಗಳಾವುವೂ ಇಲ್ಲ. ಬರಿಯ ಅಜ್ಞಾನ ಇರುವುದು. ಅದ್ವೈತ ಜ್ಞಾನದಿಂದ ಅಜ್ಞಾನ ನಿವಾರಣೆ ಆಗುವುದು.

೫೫. ಧಾರ್ಮಿಕ ಪಂಗಡಗಳು ಒಟ್ಟಿಗೆ ಬರುವುವು. ಪ್ರತಿಯೊಂದೂ ಉಳಿದವುಗಳಿಗಿಂತ ಮೇಲುಬರಲು ಯತ್ನಿಸುವುದು. ಆದರೆ ಸಾಧಾರಣವಾಗಿ ಅವುಗಳಲ್ಲಿ ಒಂದು ಮಾತ್ರ ಪ್ರಬಲವಾಗುವುದು. ಇದೇ ಕಾಲ ಕಳೆದಂತೆ ತನ್ನ ಸಮಕಾಲೀನ ಪಂಗಡಗಳನ್ನೆಲ್ಲ ಆಪೋಶನವಾಗಿ ತೆಗೆದುಕೊಳ್ಳುವುದು.

೫೬. ಸ್ವಾಮೀಜಿಯವರು ರಾಮನಾಡಿನಲ್ಲಿದ್ದಾಗ ಸಂಭಾಷಣೆಯ ಸಮಯದಲ್ಲಿ ಹೀಗೆ ಹೇಳಿದರು: ಶ‍್ರೀರಾಮ ಪರಮಾತ್ಮ, ಸೀತೆ ಜೀವಾತ್ಮ; ಪ್ರತಿಯೊಬ್ಬ ಸ್ತ್ರೀಪುರುಷರ ದೇಹವೇ ಲಂಕೆ. ಲಂಕಾನಗರದಲ್ಲಿ ಬಂಧಿಯಾದ ಜೀವಾತ್ಮ ಪರಮಾತ್ಮನಾದ ರಾಮನ ಹತ್ತಿರ ಇರಲು ಬಯಸುತ್ತಿತ್ತು. ಆದರೆ ರಾಕ್ಷಸರು ಅದಕ್ಕೆ ಅವಕಾಶಕೊಡುತ್ತಿರಲಿಲ್ಲ. ರಾಕ್ಷಸರು ಎಂದರೆ ಮಾನವನ ಕೆಲವು ವೃತ್ತಿಗಳು, ವಿಭೀಷಣ ಸತ್ತ್ವಗುಣಕ್ಕೆ, ರಾವಣ ರಾಜಸಕ್ಕೆ, ಕುಂಭಕರ್ಣ ತಾಮಸಗುಣಕ್ಕೆ ಪ್ರತಿನಿಧಿಯಾಗಿರುವರು. ಸತ್ತ್ವಗುಣ ಒಳ್ಳೆಯದು. ರಜಸ್ಸು ಕಾಮಕ್ರೋಧಗಳಿಗೆ, ತಮಸ್ಸು ಅಜ್ಞಾನ ಆಲಸ್ಯ ಆಸೆ ದ್ವೇಷ ಮುಂತಾದವಕ್ಕೆ ನಿಂತಿರುವುದು. ಈ ಗುಣಗಳೇ ಲಂಕೆಯಲ್ಲಿರುವ ಸೀತೆ ರಾಮನನ್ನು ಸೇರದಂತೆ ತಡೆಗಟ್ಟಿವೆ. ಹೀಗೆ ಬಂಧನದಲ್ಲಿರುವ ಸೀತೆ ತನ್ನ ಪತಿಯನ್ನು ಸೇರಬೇಕೆಂದು ಕಾತರಿಸುತ್ತಿದ್ದಾಗ ಹನುಮಂತ ಬರುವನು. ಅವನೇ ಗುರು. ಅವನು ರಾಮನ ಉಂಗುರವನ್ನು ತೋರುವನು. ಅದೇ ಅಜ್ಞಾನವನ್ನು ನಾಶಮಾಡುವ ಬ್ರಹ್ಮಜ್ಞಾನ. ಅದರಿಂದ ಸೀತೆಗೆ ರಾಮನೊಂದಿಗೆ ಇರಲು ಸಾಧ್ಯವಾಯಿತು. ಅಥವಾ ಜೀವಾತ್ಮ ಪರಮಾತ್ಮನೊಡನೆ ಇರಲು ಸಾಧ್ಯವಾಯಿತು.

೫೭. ನಿಜವಾದ ಕ್ರಿಸ್ತನು ನಿಜವಾದ ಹಿಂದೂವಿನಂತೆ; ನಿಜವಾದ ಹಿಂದೂವು ನಿಜವಾದ ಕ್ರಿಸ್ತನಂತೆ.

೫೮. ಹಿತವಾದ ಸಾಮಾಜಿಕ ಬದಲಾವಣೆಗಳೆಲ್ಲ ನಮ್ಮ ಆಂತರ್ಯದಲ್ಲಿ ಕೆಲಸ ಮಾಡುತ್ತಿರುವ ಆಧ್ಯಾತ್ಮಿಕ ಪ್ರವೃತ್ತಿಗಳ ಅಭಿವ್ಯಕ್ತಿಯೇ ಆಗಿವೆ. ಅವು ಬಲಶಾಲಿಯಾಗಿದ್ದರೆ, ಹೊಂದಿಕೊಂಡಿದ್ದರೆ, ಸಮಾಜ ಅದಕ್ಕೆ ಸರಿಯಾಗಿ ಅಣಿಯಾಗುವುದು. ಪ್ರತಿಯೊಬ್ಬನೂ ತನ್ನ ಮೋಕ್ಷಕ್ಕೆ ತಾನೇ ದುಡಿಯಬೇಕಾಗಿದೆ. ಬೇರೆ ಮಾರ್ಗವೇ ಇಲ್ಲ. ಇದರಂತೆಯೇ ದೇಶಗಳೂ ಕೂಡ. ಇದರಂತೆಯೇ ಪ್ರತಿಯೊಂದು ದೇಶದ ದೊಡ್ಡ ದೊಡ್ಡ ಸಂಸ್ಥೆಗಳು ಕೂಡ ತಮ್ಮ ಸ್ಥಿತಿಯಲ್ಲಿಯೇ ಇರಬೇಕಾಗಿವೆ. ಇತರ ಜನಾಂಗಗಳ ಆದರ್ಶಗಳಿಗೆ ತಕ್ಕಂತೆ ಅದನ್ನು ಸರಿಪಡಿಸಲು ಆಗುವುದಿಲ್ಲ. ಉತ್ತಮ ಸಂಸ್ಥೆಗಳು ಬರುವವರೆಗೆ ಹಳೆಯ ಸಂಸ್ಥೆಗಳನ್ನು ನಾಶಮಾಡುವುದು ವಿನಾಶಕಾರಿ. ಬೆಳವಣಿಗೆ ಯಾವಾಗಲೂ ಕ್ರಮವಾಗಿ ಆಗುವುದು.

ಸಂಸ್ಥೆಯ ಕುಂದುಕೊರತೆಗಳನ್ನು ಎತ್ತಿತೋರುವುದು ಸುಲಭ. ಏಕೆಂದರೆ ಸಂಸ್ಥೆಗಳಲ್ಲೆಲ್ಲಾ ಅಪೂರ್ಣತೆ ಇದ್ದೇ ತೀರುವುದು. ಆದರೆ ಎಂತಹ ಸಂಸ್ಥೆಯಲ್ಲಿದ್ದರೂ ಉತ್ತಮನಾಗಲು ಯಾರು ಸಹಾಯ ಮಾಡಬಲ್ಲರೋ ಅಂತಹವರೇ ಮಾನವಕೋಟಿಯ ಹಿತಚಿಂತಕರು. ವ್ಯಕ್ತಿಗಳನ್ನು ಮೇಲೆತ್ತಿದರೆ, ಅವರಿರುವ ದೇಶ ಮತು ಸಂಸ್ಥೆಗಳು ಮೇಲೇರುವುದರಲ್ಲಿ ಸಂದೇಹವಿಲ್ಲ. ಧರ್ಮಾತ್ಮರು ಹೀನ ಚಾಳಿಗಳನ್ನು ಮತ್ತು ನಿಯಮಗಳನ್ನು ಗಮನಿಸುವುದಿಲ್ಲ. ಅವುಗಳ ಸ್ಥಾನದಲ್ಲಿ ಲಿಪಿಬದ್ಧವಲ್ಲದ ಆದರೆ ಪ್ರಬಲವಾದ ಪ್ರೀತಿ ಕರುಣೆ ಮತ್ತು ಸತ್ಯ – ಈ ನಿಯಮಗಳು ಬರುವುವು. ಯಾವ ದೇಶಕ್ಕೆ ನಿಯಮಾವಳಿಗಳ ಪುಸ್ತಕದ ಆವಶ್ಯಕತೆ ಬಹಳ ಕಡಿಮೆ ಇರುವುದೋ ಅದು ಧನ್ಯ. ಆ ದೇಶವು, ಆ ಸಂಸ್ಥೆಯನ್ನು ಈ ಸಂಸ್ಥೆಯನ್ನು ಕುರಿತು ಚಿಂತಿಸುತ್ತಿರಬೇಕಾಗಿಲ್ಲ. ಉತ್ತಮ ವ್ಯಕ್ತಿಗಳು ಎಲ್ಲ ನಿಯಮಗಳನ್ನು ಮೀರಿಹೋಗುತ್ತಾರೆ ಮತ್ತು ಇತರರಿಗೆ, ಅವರು ಯಾವುದೇ ಸ್ಥಿತಿಯಲ್ಲಿರಲಿ, ನಿಯಮಗಳನ್ನು ಮೀರಿ ಹೋಗಲು ಸಹಾಯ ಮಾಡುತ್ತಾರೆ.

ಭರತಖಂಡದ ಹಿತ ನಿಂತಿರುವುದು ಪ್ರತಿ ವ್ಯಕ್ತಿಯ ಶಕ್ತಿಯ ಮೇಲೆ ಮತ್ತು ತಾನು ಪವಿತ್ರಾತ್ಮ ಎಂದು ಭಾವಿಸುವುದರ ಮೇಲೆ.

೫೯. ಪ್ರಾಪಂಚಿಕತೆ ನಾಶವಾಗುವವರೆಗೆ ಧಾರ್ಮಿಕತೆ ಬರಲಾರದು.

೬೦. ಗೀತೆಯಲ್ಲಿ ಬರುವ ಪ್ರಥಮ ಸಂಭಾಷಣೆಯನ್ನು ರೂಪಕವಾಗಿ ಬೇಕಾದರೆ ನೋಡಬಹುದು.

೬೧. ಎಲ್ಲಿ ಸ್ಟಿಮರ್ ತಪ್ಪಿಹೋಗುವುದೋ ಎಂದು ಕಾತರನಾಗಿದ್ದ ಅಮೆರಿಕಾದೇಶೀಯನೊಬ್ಬ "ಸ್ವಾಮಿ, ನಿಮಗೆ ಕಾಲದ ಅರಿವೇ ಇಲ್ಲ," ಎಂದ. ಅದಕ್ಕೆ ಸ್ವಾಮಿಗಳು "ಹೌದು, ಇಲ್ಲ. ನೀವು ಕಾಲದಲ್ಲಿರುವಿರಿ. ನಾವು ಅನಂತತೆಯಲ್ಲಿರುವೆವು" ಎಂದರು.

೬೨. ನಾವು ಯಾವಾಗಲೂ ಕರ್ತವ್ಯದ ಬದಲು ಉದ್ವೇಗವಶರಾಗಿ ನಿಜವಾಗಿ ಪ್ರೇಮದಿಂದ ಪ್ರೇರಿತರಾಗಿ ಕೆಲಸ ಮಾಡುತ್ತಿರುವೆವು ಎನ್ನುವೆವು.

೬೩. ನಮಗೆ ತ್ಯಾಗ ಮಾಡುವ ಧೈರ್ಯ ಬೇಕಾದರೆ ನಾವು ಉದ್ವೇಗವಶರಾಗಕೂಡದು. ಉದ್ವೇಗ ಕೇವಲ ಪ್ರಾಣಿಗಳಿಗೆ ಸೇರಿದ್ದು, ಅವು ಸಂಪೂರ್ಣವಾಗಿ ಉದ್ವೇಗದ ಅಧೀನದಲ್ಲಿರುವುವು.

೬೪. ತನ್ನ ಮಕ್ಕಳುಮರಿಗಳಿಗಾಗಿ ಪ್ರಾಣತೊರೆಯುವುದೇನೂ ಮಹೋನ್ನತ ತ್ಯಾಗವಲ್ಲ. ಮಾನವರ ತಾಯಂದಿರಂತೆ ಪ್ರಾಣಿಗಳೂ ತಮ್ಮ ಮರಿಗಳಿಗಾಗಿ ಸಾಯಲು ಸಿದ್ಧವಾಗಿರುತ್ತವೆ. ಹೀಗೆ ಮಾಡುವುದೇನು ಪರಮ ಪ್ರೇಮದ ಚಿಹ್ನೆಯಲ್ಲ. ಇದು ಕೇವಲ ಅಂಧ ಅಭಿಮಾನವಷ್ಟೆ.

೬೫. ನಾವು ಯಾವಾಗಲೂ ನಮ್ಮ ದುರ್ಬಲತೆ ಶಕ್ತಿಯಂತೆ ಕಾಣುವಂತೆ, ಉದ್ವೇಗ ಪ್ರೀತಿಯಂತೆ ಕಾಣುವಂತೆ, ನಮ್ಮ ಹೇಡಿತನ ಧೈರ್ಯದಂತೆ ಕಾಣುವಂತೆ ಮಾಡಲು ಯತ್ನಿಸುತ್ತಿರುವೆವು.

೬೬. ದುರ್ಬಲತೆ ವ್ಯಾಮೋಹ ಮೊದಲಾದುವನ್ನು ಕುರಿತು ನಿನ್ನ ಆತ್ಮನಿಗೆ 'ಇದು ನಿನಗೆ ತರವಲ್ಲ, ಇದು ನಿನಗೆ ತರವಲ್ಲ' ಎಂದು ಹೇಳಿ.

೬೭. ಎಂದಿಗೂ ಪತಿಯು ಸತಿಯನ್ನು ಸತಿಗಾಗಿ ಪ್ರೀತಿಸಲಿಲ್ಲ. ಅಥವಾ ಸತಿಯು ಪತಿಯನ್ನು ಪತಿಗಾಗಿ ಪ್ರೀತಿಸಲಿಲ್ಲ. ಸತಿಯಲ್ಲಿರುವ ದೇವರನ್ನು ಪತಿ ಪ್ರೀತಿಸುವುದು, ಪತಿಯಲ್ಲಿರುವ ದೇವರನ್ನು ಸತಿ ಪ್ರೀತಿಸುವುದು. ಎಲ್ಲರಲ್ಲಿರುವ ದೇವರೇ ನಮ್ಮನ್ನು ಪ್ರೀತಿಸುವ ವಸ್ತುವಿನ ಎಡೆಗೆ ನಮ್ಮನ್ನು ಆಕರ್ಷಿಸುವುದು. ಪ್ರತಿಯೊಂದು ವಸ್ತುವಿನಲ್ಲಿರುವ ದೇವರೇ ನಮ್ಮನ್ನು ಪ್ರೀತಿಸುವಂತೆ ಮಾಡುವುದು. ದೇವರೊಬ್ಬನೇ ಪ್ರೀತಿ.

೬೮. ಓ, ನಿಮ್ಮ ನೈಜಸ್ಥಿತಿಯನ್ನು ನೀವು ಅರಿತರೆ! ನೀವು ಆತ್ಮ, ದೇವರು. ನಾನು ಯಾವಾಗಲಾದರು ಈಶ್ವರನಿಂದೆ ಮಾಡಿದರೆ ಅದೇ ನಾನು ನಿಮ್ಮನ್ನು ಮನುಷ್ಯ ಎಂದು ಕರೆದಾಗ.

೬೯. ಪ್ರತಿಯೊಬ್ಬರಲ್ಲಿಯೂ ದೇವರು ಇರುವನು, ಉಳಿದುವೆಲ್ಲ ಭ್ರಾಂತಿ ಕನಸು.

೭೦. ನನಗೆ ಆಧ್ಯಾತ್ಮದಲ್ಲಿ ಆನಂದ ದೊರಕದಿದ್ದರೆ ಇಂದ್ರಿಯದಲ್ಲಿ ಆನಂದವನ್ನು ಅರಸಬೇಕೆ? ಅಮೃತ ಸಿಕ್ಕದೆ ಇದ್ದರೆ ಚರಂಡಿಯ ನೀರನ್ನು ಕುಡಿಯಬೇಕೆ? ಜಾತಕಪಕ್ಷಿ ಯಾವಾಗಲೂ ಮಳೆಯ ಹನಿಯನ್ನು ಮಾತ್ರ ಕುಡಿಯುವುದು; ನೀರು, ಶುದ್ಧ ನೀರು, ಬೇಕೆಂದು ಹಾರಾಡುತ್ತಿರುವುದು. ಎಂತಹ ಬಿರುಗಾಳಿ ಬರಲಿ, ಏನೇ ಆಗಲಿ, ಅದು ನೆಲಕ್ಕೆ ಇಳಿದು ಇಲ್ಲಿರುವ ನೀರನ್ನು ಕುಡಿಯುವುದಿಲ್ಲ.

೭೧. ಭಗವಂತನ ಸಾಕ್ಷಾತ್ಕಾರಕ್ಕೆ ಯಾವ ಪಂಗಡ ಸಹಾಯ ಮಾಡಿದರೂ ಚಿಂತೆಯಿಲ್ಲ, ಅದನ್ನು ಸ್ವೀಕರಿಸಿ. ದೇವರ ಸಾಕ್ಷಾತ್ಕಾರವೇ ಧರ್ಮ.

೭೨. ನಾಸ್ತಿಕ ದಾನಿಯಾಗಬಹುದು, ಆದರೆ ಧಾರ್ಮಿಕನಾಗಲಾರ. ಆದರೆ ಧಾರ್ಮಿಕಜೀವಿ ದಾನಿಯಾಗಿಯೇ ಇರಬೇಕು.

೭೩. ಯಾರು ಗುರುಗಳಾಗುವುದಕ್ಕೆ ಜನ್ಮತಾಳಿರುವರೋ ಅವರ ವಿನಃ ಉಳಿದವರು ಪ್ರಯತ್ನಿಸಿದರೆ ಗುರುತ್ವವೆಂಬ ಬಂಡೆಗೆ ತಗಲಿ ನುಚ್ಚುನೂರಾಗುವರು.

೭೪, ಪ್ರಾಣಿ, ಮನುಷ್ಯ, ದೇವ ಇವುಗಳೆಲ್ಲದರ ಮಿಶ್ರಣ ಈಗ ಮಾನವನಾಗಿರುವವನು.

೭೫. ಸಮಾಜಪ್ರಗತಿ ಎಂಬುದು ಬೆಚ್ಚಗಿರುವ ಹಿಮ, ಬೆಳ್ಳಗಿರುವ ಕತ್ತಲೆ ಎಂಬಷ್ಟೇ ವಿರೋಧಾಭಾಸವಾಗಿದೆ. ಕೊನೆಗೆ ಸಮಾಜಪ್ರಗತಿ ಎಂಬುದಿಲ್ಲ!

೭೬. ಪರಿಸ್ಥಿತಿಗಳನ್ನು ಉತ್ತಮಪಡಿಸಲಾಗುವುದಿಲ್ಲ. ಆದರೆ ಅವುಗಳನ್ನು ಬದಲಾಯಿಸುವುದರಿಂದ ನಾವು ಉತ್ತಮರಾಗುತ್ತೇವೆ.

೭೭. ನನ್ನ ಜನರಿಗೆ ನಾನು ನೆರವು ನೀಡಬಯಸುವೆನು. ನಾನು ಇಚ್ಛಿಸುವುದು ಇಷ್ಟೇ.

೭೮. ನಾನು ರಹಸ್ಯವನ್ನು ನಂಬುವುದಿಲ್ಲ, ಒಂದು ವಸ್ತು ಅಸತ್ಯವಾದರೆ ಅದು ಇಲ್ಲ, ಅಷ್ಟೇ. ಅಸತ್ಯವಾಗಿರುವುದು ಇಲ್ಲ. ವಿಚಿತ್ರ ಪ್ರಸಂಗಗಳೂ ಸ್ವಾಭಾವಿಕ ಘಟನೆಗಳು – ಅವೆಲ್ಲ ವೈಜ್ಞಾನಿಕ ವಿಷಯಗಳು. ಆದರೆ ಅವು ನನಗೆ ರಹಸ್ಯವಲ್ಲ. ನಾನು ರಹಸ್ಯ ಸಂಘಗಳನ್ನೂ ನಂಬುವುದಿಲ್ಲ. ಅವುಗಳಿಂದ ಏನೂ ಒಳ್ಳೆಯದಾಗುವುದಿಲ್ಲ, ಎಂದೆಂದಿಗೂ ಒಳ್ಳೆಯದಾಗುವುದಿಲ್ಲ.

೭೯. ಸಾಧಾರಣವಾಗಿ ನಾಲ್ಕು ಬಗೆಯ ಮನುಷ್ಯರು ಇರುವರು–ಜ್ಞಾನಿಗಳು, ಭಕ್ತರು, ಯೋಗಿಗಳು, ಕರ್ಮಿಗಳು. ಇವರಲ್ಲಿ ಪ್ರತಿಯೊಬ್ಬರಿಗೂ ಯೋಗ್ಯವಾದ ಉಪಾಸನೆಯನ್ನು ನಾವು ಒದಗಿಸಿಕೊಡಬೇಕು. ಜ್ಞಾನಿ ಬಂದು, ನನಗೆ ಈ ಪೂಜೆ ಪುರಸ್ಕಾರ ಬೇಕಿಲ್ಲ, ನಾನು ಮೆಚ್ಚಬಲ್ಲ ತತ್ತ್ವವನ್ನು ಕೊಡಿ, ಯುಕ್ತಿಯನ್ನು ಕೊಡಿ ಎನ್ನುವನು. ಆದಕಾರಣವೇ ಜ್ಞಾನಿಗೆ ಜ್ಞಾನೋಪಾಸನೆ. ಕರ್ಮಯೋಗಿ "ನನಗೆ ಜ್ಞಾನಿಯ ಉಪಾಸನೆ ಬೇಕಿಲ್ಲ. ನನ್ನ ಸಹೋದರರಿಗಾಗಿ ಮಾಡುವುದಕ್ಕೆ ಏನಾದರೂ ಕರ್ಮವನ್ನು ಕೊಡಿ" ಎನ್ನುವನು. ಅವನಿಗೆ ಕರ್ಮವೇ ಉಪಾಸನೆಯಾಗುವುದು. ಯೋಗಿಗಳಿಗೆ ಮತ್ತು ಭಕ್ತರಿಗೆ ಯೋಗ್ಯವಾದ ಉಪಾಸನೆಗಳಿವೆ. ಇವರೆಲ್ಲರ ಶ್ರದ್ದೆಗೂ ಧರ್ಮದಲ್ಲಿ ಎಡೆಯಿದೆ.

೮೦. ನಾನು ಸತ್ಯಕ್ಕಾಗಿ ಹೋರಾಡುತ್ತೇನೆ. ಸತ್ಯ ಎಂದಿಗೂ ಮಿಥ್ಯೆಯೊಂದಿಗೆ ರಾಜಿ ಮಾಡಿಕೊಳ್ಳುವುದಿಲ್ಲ. ಪ್ರಪಂಚವೇ ನನಗೆ ವಿರೋಧವಾಗಿ ನಿಂತರೂ ಕೊನೆಗೆ ಸತ್ಯವೇ ಜಯಿಸಬೇಕು.

೮೧. ಜನಸಾಮಾನ್ಯರ ಕೈಗೆ ಎಂದು ಪರೋಪಕಾರದ ಮಹಾ ಭಾವನೆಗಳು ಬೀಳುವುವೋ ಆಗ ಮೊದಲು ಅವು ಅಧೋಗತಿಗೆ ಬರುವುವು. ಬುದ್ಧಿ ಮತ್ತು ಪಾಂಡಿತ್ಯ ಇವು ಭಾವನೆಗಳನ್ನು ಶ್ರೇಷ್ಠ ಮಟ್ಟದಲ್ಲಿ ಇಡುವುವು. ದೇಶದ ಸುಸಂಸ್ಕೃತರ ರಕ್ಷಣೆಯಲ್ಲಿ ಮಾತ್ರ ಧರ್ಮ ಮತ್ತು ತತ್ತ್ವ ಪರಿಶುದ್ಧವಾಗಿರಬಲ್ಲವು. ಆ ಪರಿಶುದ್ಧ ಸ್ಥಿತಿಯೇ ಸಮಾಜದ ಬೌದ್ಧಿಕ ಮತ್ತು ನೈತಿಕ ಮಟ್ಟವನ್ನು ತೋರುವ ಪ್ರಮಾಣವಾಗಿದೆ.

೮೨. ನಾನು ನಿಮ್ಮನ್ನು ಹೊಸ ಧರ್ಮಕ್ಕೆ ಸೇರಿಸುವುದಕ್ಕೆ ಬರಲಿಲ್ಲ. ನಿಮ್ಮ ಧರ್ಮವನ್ನು (ಅಮೆರಿಕಾ ದೇಶದ ಜನರ ಧರ್ಮವನ್ನು) ನೀವು ಇಟ್ಟುಕೊಳ್ಳಿ. ಮೆಥಾಡಿಸ್ಟರು ಉತ್ತಮ ಮೆಥಾಡಿಸ್ಟರಾಗುವುದಕ್ಕೆ, ಪ್ರಿಸ್ಬಿಟೇರಿಯನ್ನರು ಉತ್ತಮ ಪ್ರಿಸ್ಬಿಟೇರಿಯನ್ನರಾಗುವುದಕ್ಕೆ, ಯೂನಿಟೇರಿಯನ್ನರು ಉತ್ತಮ ಯೂನಿಟೇರಿಯನ್ನರಾಗುವುದಕ್ಕೆ ನಾನು ಸಹಾಯ ಮಾಡಲಿಚ್ಛಿಸುವೆನು. ಸತ್ಯದಂತೆ ಬಾಳಿ, ಅದನ್ನು ನಿಮ್ಮ ಜೀವನದಲ್ಲಿ ವ್ಯಕ್ತಗೊಳಿಸಿ ಎಂದು ನಿಮಗೆ ಬೋಧಿಸುವೆನು.

೮೩. ಸುಖವು ದುಃಖದ ಕಿರೀಟವನ್ನು ಧರಿಸಿ ಮಾನವನೆದುರಿಗೆ ಬಂದು ನಿಲ್ಲುವುದು. ಯಾರಿಗೆ ಸುಖ ಬೇಕೋ ಅವರು ದುಃಖವನ್ನೂ ಸ್ವೀಕರಿಸಬೇಕು.

೮೪, ಯಾರು ಪ್ರಪಂಚವನ್ನು ನಿಕೃಷ್ಟ ದೃಷ್ಟಿಯಿಂದ ನೋಡುವರೊ, ಯಾರು ಎಲ್ಲವನ್ನೂ ತ್ಯಜಿಸುವರೊ, ಯಾರು ಇಂದ್ರಿಯಜಿತರೊ, ಯಾರು ಶಾಂತಿಗೆ ಹಾತೊರೆಯುತ್ತಿರುವರೊ ಅವರೇ ಮುಕ್ತರು, ಅವರೇ ಮಹಾತ್ಮರು. ಒಬ್ಬನಿಗೆ ರಾಜಕೀಯ ಮತ್ತು ಸಾಮಾಜಿಕ ಸ್ವಾತಂತ್ರ್ಯ ಸಿಕ್ಕಬಹುದು. ಆದರೆ ಅವನು ತನ್ನ ಕಾಮಕ್ರೋಧಗಳಿಗೆ ದಾಸನಾದರೆ ಅವನು ನಿಜವಾದ ಸ್ವಾತಂತ್ರ್ಯದ ಪರಿಶುದ್ಧ ಆನಂದವನ್ನು ಸವಿಯಲಾರನು.

೮೫. ಇತರರಿಗೆ ಒಳ್ಳೆಯದನ್ನು ಮಾಡುವುದು ಧರ್ಮ, ಅವರನ್ನು ಹಿಂಸಿಸುವುದು ಅಧರ್ಮ, ಪೌರುಷ ಮತ್ತು ಶಕ್ತಿ ಇವೇ ಧರ್ಮ; ಹೇಡಿತನ ಮತ್ತು ದೌರ್ಬಲ್ಯವೇ ಅಧರ್ಮ, ಸ್ವಾತಂತ್ರ್ಯವೇ ಪುಣ್ಯ, ಬಂಧನವೇ ಪಾಪ, ಆತ್ಮನಲ್ಲಿ ಮತ್ತು ದೇವರಲ್ಲಿ ಶ್ರದ್ಧೆಯೇ ಪುಣ್ಯ, ಅಶ್ರದ್ಧೆಯೇ ಪಾಪ, ಏಕತೆಯನ್ನು ನೋಡುವುದೇ ಪುಣ್ಯ, ವೈವಿಧ್ಯವನ್ನು ನೋಡುವುದೇ ಪಾಪ. ಹಲವು ಶಾಸ್ತ್ರಗಳು ಧರ್ಮಾತ್ಮರಾಗುವುದಕ್ಕೆ ದಾರಿಯನ್ನು ಮಾತ್ರ ತೋರುತ್ತವೆ.

೮೬. ಯುಕ್ತಿಯ ಮೂಲಕ ಬುದ್ಧಿಯು ಸತ್ಯವನ್ನು ಅರಿತ ಮೇಲೆ ಅದನ್ನು ಭಾವಗಳ \enginline{(feeling)} ಮೂಲವಾದ ಹೃದಯದಲ್ಲಿ ಅನುಭವಿಸುತ್ತೇವೆ. ಜ್ಞಾನ ಮತ್ತು ಭಾವಗಳೆರಡೂ ಏಕಕಾಲದಲ್ಲಿಯೇ ಬೆಳಕನ್ನು ಕಾಣುತ್ತವೆ. ಆಗ ಮಾತ್ರ ಉಪನಿಷತ್ತು ಹೇಳುವಂತೆ ಹೃದಯ ಗ್ರಂಥಿ ಸಡಿಲವಾಗುವುದು, ಸಂಶಯಗಳೆಲ್ಲ ನಾಶವಾಗುವುವು.

ಪೂರ್ವಕಾಲದಲ್ಲಿ ಋಷಿಗಳ ಹೃದಯದಲ್ಲಿ ಜ್ಞಾನ ಮತ್ತು ಭಾವಗಳೆರಡೂ ವಿಕಸಿತವಾದಾಗ ಪರಮಸತ್ಯ ಕಾವ್ಯಮಯವಾಯಿತು. ಆಗಲೇ ವೇದ ಮುಂತಾದುವು ರಚಿತವಾದುವು. ಆದಕಾರಣದಿಂದಲೇ ನಾವು ಅವನ್ನು ಅಧ್ಯಯನ ಮಾಡುತ್ತಿರುವಾಗ ಭಾವ ಮತ್ತು ಜ್ಞಾನಗಳೆಂಬ ಸಮಾನಾಂತರ ರೇಖೆಗಳೆರಡೂ ಕೊನೆಗೆ ಅವುಗಳಲ್ಲಿ ಸಂಗಮವಾಗಿ ಬೇರ್ಪಡಿಸಲಾಗದಂತೆ ಕಾಣುವುವು.

೮೭. ಹಲವು ಧರ್ಮ ಶಾಸ್ತ್ರಗಳಲ್ಲಿ ವಿಶ್ವಪ್ರೇಮ, ಸ್ವಾತಂತ್ರ್ಯ, ಪೌರುಷ, ನಿಃಸ್ವಾರ್ಥತೆ, ಕರುಣೆ ಇವನ್ನು ಪಡೆಯುವುದಕ್ಕೆ ವಿಭಿನ್ನ ಮಾರ್ಗಗಳನ್ನು ಹೇಳಿದೆ. ಯಾವುದು ಪುಣ್ಯ ಮತ್ತು ಯಾವುದು ಪಾಪವೆಂಬ ವಿಷಯದಲ್ಲಿ ಹಲವು ಧರ್ಮಗಳೊಳಗೆ ಭಿನ್ನಾಭಿಪ್ರಾಯಗಳಿವೆ. ಗುರಿಯನ್ನು ಸೇರುವುದರ ಮೇಲೆ ಗಮನ ಕೊಡದೆ, ಪಾಪವನ್ನು ತ್ಯಜಿಸಿ ಪುಣ್ಯವನ್ನೇ ಅರ್ಜಿಸುವುದಕ್ಕಿರುವ ಸಾಧನವನ್ನು ಕುರಿತು ಕಾದಾಡುತ್ತಿರುವರು. ಪ್ರತಿಯೊಂದು ಮಾರ್ಗವೂ ಒಳ್ಳೆಯದೆ. ಗೀತೆ ಹೇಳುವಂತೆ ಪ್ರತಿಯೊಂದು ಮಾರ್ಗದಲ್ಲಿಯೂ ಸ್ವಲ್ಪ ದೋಷವಿದ್ದೇ ಇರುತ್ತದೆ. ಹೇಗೆ ಬೆಂಕಿ ಹೊಗೆಯಿಂದ ಆವೃತ್ತವಾಗಿದೆಯೋ ಹಾಗೇ ಮಾರ್ಗದಲ್ಲಿ ಕುಂದುಕೊರತೆಗಳು ಸ್ವಲ್ಪ ಹೆಚ್ಚು ಕಡಿಮೆ ಇರುವಂತೆ ಕಾಣುವುದು. ಆದರೆ ನಮ್ಮ ನಮ್ಮ ಶಾಸ್ತ್ರಗಳು ಹೇಳುವಂತೆ ನಾವು ಶ್ರೇಷ್ಠ ಗುಣಗಳನ್ನು ಪಡೆಯಬೇಕಾಗಿರುವುದರಿಂದ, ಸಾಧ್ಯವಾದಷ್ಟು ಅವನ್ನು ಅನುಸರಿಸಲು ಪ್ರಯತಿಸೋಣ. ಆದರೆ ಇವುಗಳಲ್ಲಿ ಸ್ವಲ್ಪ ಯುಕ್ತಿ ಮತ್ತು ವಿವೇಚನೆ ಇರಬೇಕು. ನಾವು ಮುಂದುವರಿದಂತೆ ಧರ್ಮ–ಅಧರ್ಮಗಳ ಸಮಸ್ಯೆ ತಾನೇ ಪರಿಹಾರವಾಗುವುದು.

೮೮. ನಮ್ಮ ದೇಶದಲ್ಲಿ ಈಗ ನಿಜವಾಗಿ ಎಷ್ಟು ಜನರು ಶಾಸ್ತ್ರವನ್ನು ಅರ್ಥಮಾಡಿಕೊಳ್ಳಬಲ್ಲರು? ಅವರಿಗೆ ಬ್ರಹ್ಮ ಮಾಯೆ ಪ್ರಕೃತಿ ಮುಂತಾದ ಪದಗಳು ಮಾತ್ರ ಗೊತ್ತಿವೆ. ಸುಮ್ಮನೆ ಈ ಪದಗಳಿಂದ ತಮ್ಮ ತಲೆಕೆಡಿಸಿಕೊಳ್ಳುವರು. ಶಾಸ್ತ್ರದ ನಿಜವಾದ ಅರ್ಥವನ್ನು ಮತ್ತು ಗುರಿಯನ್ನು ಮರೆತು ಕೇವಲ ಪದಗಳಿಗಾಗಿ ಹೋರಾಡುವರು. ಎಲ್ಲಾ ಮಾನವರಿಗೂ, ಎಲ್ಲಾ ಪರಿಸ್ಥಿತಿಗಳಲ್ಲಿಯೂ, ಎಲ್ಲಾ ಕಾಲಗಳಲ್ಲಿಯೂ ಶಾಸ್ತ್ರ ಸಹಾಯಕ್ಕೆ ಬಾರದಿದ್ದರೆ ಅದರಿಂದ ಏನು ಪ್ರಯೋಜನ? ಶಾಸ್ತ್ರ ಕೇವಲ ಸಂನ್ಯಾಸಿಗೆ ಮಾತ್ರ ಮಾರ್ಗವನ್ನು ತೋರಿ, ಗೃಹಸ್ಥನಿಗೆ ತೋರದೆ ಇದ್ದರೆ, ಇಂತಹ ಶಾಸ್ತ್ರದಿಂದ ಗೃಹಸ್ಥನಿಗೆ ಪ್ರಯೋಜನವೇನು? ಕರ್ಮವನ್ನು ತ್ಯಜಿಸಿ ಕಾಡಿಗೆ ಹೋದವರಿಗೆ ಮಾತ್ರ ಶಾಸ್ತ್ರ ಸಹಾಯ ಮಾಡಿ, ಕರ್ಮಪ್ರಪಂಚದಲ್ಲಿ ಮೈಮರೆತವರಿಗೆ, ರೋಗರುಜಿನ, ದುಃಖ ದಾರಿದ್ರ್ಯಗಳಲ್ಲಿ ತಪ್ತರಾದವರಿಗೆ, ನಿರಾಶೆಯಲ್ಲಿ ತೊಳಲುತ್ತಿರುವ ಪಾಪಿಗೆ, ದಬ್ಬಾಳಿಕೆಗೆ ತುತ್ತಾಗಿ ತನ್ನ ಗ್ರಹಚಾರವನ್ನು ಹಳಿದುಕೊಳ್ಳುತ್ತಿರುವವನಿಗೆ, ಸಮರಾಂಗಣದ ಭೀತಿಯಲ್ಲಿರುವವರಿಗೆ, ಕಾಮ ಕ್ರೋಧಗಳ ಉಪಟಳದಲ್ಲಿರುವವರಿಗೆ, ಜಯದ ಸಂತೋಷದಲ್ಲಿ ಉಬ್ಬಿರುವವರಿಗೆ, ಅಪಜಯದ ದುಃಖದಲ್ಲಿ ಕುಗ್ಗಿರುವವರಿಗೆ, ಕೊನೆಗೆ ಭಯಾನಕವಾದ ಮೃತ್ಯು ಸಮೀಪದಲ್ಲಿರುವವರಿಗೆ ಶಾಸ್ತ್ರ ಸಹಾಯಕ್ಕೆ ಬಾರದಿದ್ದರೆ, ಅವರ ಹೃದಯದಲ್ಲಿ ಒಂದು ಭರವಸೆಯ ಹಣತೆಯನ್ನು ಹಚ್ಚದೇ ಇದ್ದರೆ–ದೀನ ಮಾನವ ಕೋಟಿಗೆ ಇಂತಹ ಶಾಸ್ತ್ರದಿಂದ ಏನೂ ಪ್ರಯೋಜನವಿಲ್ಲ, ಇಂತಹ ಶಾಸ್ತ್ರಗಳು ಶಾಸ್ತ್ರಗಳೇ ಅಲ್ಲ.

೮೯. ಭೋಗದ ಮೂಲಕ ಯೋಗ ಕಾಲಕ್ರಮೇಣ ಬರುವುದು. ಆದರೆ ನನ್ನ ದೇಶದ ಜನರ ಗ್ರಹಚಾರ ಇದು: ಯೋಗದ ಮಾತೇಕೆ, ಅವರಿಗೆ ಸ್ವಲ್ಪ ಭೋಗವೂ ಇಲ್ಲ. ಎಲ್ಲಾ ಅಪಮಾನಗಳನ್ನೂ ಸಹಿಸಿಯೂ ಜೀವನಕ್ಕೆ ಅತ್ಯಾವಶ್ಯಕವಾಗಿ ಬೇಕಾದುದರಲ್ಲಿ ಅಲ್ಪಸ್ವಲ್ಪ ಮಾತ್ರ ದೊರಕುವುದು. ಅದು ಕೂಡ ಎಲ್ಲರಿಗೂ ಇಲ್ಲ. ಇಂತಹ ಸ್ಥಿತಿ ನಮ್ಮ ನಿದ್ರೆಗೆ ಭಂಗತಾರದೆ, ನಮ್ಮನ್ನು ಕಾರ್ಯತತ್ಪರರನ್ನಾಗಿ ಮಾಡದೆ ಇರುವುದು ಸೋಜಿಗ!

೯೦. ನಿಮ್ಮ ಹಕ್ಕುಬಾಧ್ಯತೆಗಳಿಗಾಗಿ ನೀವು ಬೇಕಾದಷ್ಟು ಕೂಗಾಡಿ; ಆದರೆ ಇಡೀಯ ಜನಾಂಗದಲ್ಲಿ ಆತ್ಮಗೌರವ ಉದಯಿಸುವವರೆಗೆ ಇವೆಲ್ಲ ಒಂದು ವ್ಯರ್ಥ ಆಲಾಪ.

೯೧. ಮಹಾಪ್ರತಿಭೆಯಿಂದ ಅಥವಾ ಯಾವುದಾದರೂ ಒಂದು ಅದ್ಭುತ ಶಕ್ತಿಯಿಂದ ಒಂದು ಜೀವಿ ಅವತರಿಸಿದಾಗ, ಅವನು ಆ ವಂಶದಲ್ಲಿರುವ ಉತ್ತಮ ಶೀಲವನ್ನೆಲ್ಲಾ, ಕಲ್ಪನಾಶಕ್ತಿಯನ್ನೆಲ್ಲಾ ಹೀರಿಬಿಡುವನು. ಆದಕಾರಣವೇ ಅಂತಹ ಕುಲದಲ್ಲಿ ಅನಂತರ ಹುಟ್ಟುವವರು ಮೂರ್ಖರು ಇಲ್ಲವೆ ಸಾಧಾರಣ ದರ್ಜೆಯ ಮನುಷ್ಯರು ಆಗಿರುವರು. ಕಾಲಕ್ರಮೇಣ ಅನೇಕ ವೇಳೆ ಅಂತಹ ವಂಶಗಳು ನಿರ್ನಾಮವಾಗುವುವು.

೯೨. ನೀನು ಈ ಜನ್ಮದಲ್ಲಿ ಮುಕ್ತಿಯನ್ನು ಪಡೆಯದೆ ಇದ್ದರೆ, ಮುಂದಿನ ಜನ್ಮದಲ್ಲಿ ಅದನ್ನು ಪಡೆಯುವೆ ಎಂಬ ನೆಚ್ಚಿಗೆ ಏನು?

೯೩. ಸ್ವಾಮಿ ವಿವೇಕಾನಂದರು ತಾಜ್‌ಮಹಲನ್ನು ನೋಡಿದಾಗ ಹೀಗೆಂದರು: “ನೀನು ಅಮೃತಶಿಲೆಯ ಚೂರೊಂದನ್ನು ಹಿಂಡಿದರೆ ರಾಜನ ಪ್ರೇಮ ಮತ್ತು ದುಃಖದ ಹನಿಗಳು ಜಿನುಗುವುವು. ಒಳಗಿರುವ ಸೌಂದರ್ಯವನ್ನು ಸರಿಯಾಗಿ ನೋಡಬೇಕಾದರೆ ಒಂದು ಇಂಚಿಗೆ ಆರು ತಿಂಗಳುಗಳು ಹಿಡಿಯುವುವು.”

೯೪, ನಿಜವಾದ ಭರತಖಂಡದ ಚರಿತ್ರೆ ಬೆಳಕಿಗೆ ಬಂದಾಗ ಗೊತ್ತಾಗುವುದು ಧಾರ್ಮಿಕ ವಿಷಯಗಳಲ್ಲಿ ಮಾತ್ರವಲ್ಲ, ಲಲಿತ ಕಲೆಗಳಲ್ಲಿ ಕೂಡ ಭರತಖಂಡವು ಇಡೀಯ ಪ್ರಪಂಚಕ್ಕೆ ಆದಿಗುರು ಎಂಬುದು.

೯೫. ವಾಸ್ತುಶಾಸ್ತ್ರದ \enginline{(Architecture)} ವಿಷಯವಾಗಿ ಸ್ವಾಮೀಜಿ ಹೀಗೆ ಹೇಳಿದರು: "ಜನರು ಕಲ್ಕತ್ತೆಯನ್ನು ಅರಮನೆಗಳ ನಗರ ಎನ್ನುವರು. ಆದರೆ ಮನೆಗಳು ಸುಮ್ಮನೆ ಒಂದು ಪೆಟ್ಟಿಗೆಯನ್ನು ಮತ್ತೊಂದು ಪೆಟ್ಟಿಗೆಯ ಮೇಲೆ ಇಟ್ಟಂತೆ ಕಾಣುವುವು. ಇವು ಯಾವ ಭಾವನೆಯನ್ನೂ ವ್ಯಕ್ತಗೊಳಿಸುವುದಿಲ್ಲ. ರಾಜಪುತಾನದಲ್ಲಿ ಈಗಲೂ ನಿಜವಾದ ಭಾರತೀಯ ವಾಸ್ತುಕಲೆಯನ್ನು ನೋಡಬಹುದು. ನೀವು ಒಂದು ಧರ್ಮಶಾಲೆಯನ್ನು ನೋಡಿದರೆ ಅದು ನಿಮ್ಮನ್ನು ಕೂಗಿ ಕರೆಯುತ್ತಿರುವಂತೆ, ನಿರ್ಬಂಧವಾದ ಆತಿಥ್ಯವನ್ನು ಸ್ವೀಕರಿಸು ಎಂದು ಹೇಳುತ್ತಿದೆಯೋ ಎಂಬಂತೆ ಭಾಸವಾಗುವುದು. ನೀವೊಂದು ದೇವಸ್ಥಾನವನ್ನು ನೋಡಿದರೆ ಸುತ್ತಮುತ್ತಲೂ ಆ ಪವಿತ್ರತೆ ನಿಮ್ಮ ಅನುಭವಕ್ಕೆ ಬರುವುದು. ನೀವೊಂದು ಹಳ್ಳಿಯ ಗುಡಿಸಲನ್ನು ನೋಡಿದರೆ, ಅದರ ಬೇರೆ ಬೇರೆ ಭಾಗಗಳ ಪ್ರಯೋಜನ ತಕ್ಷಣ ಹೊಳೆಯುವುದು. ಆ ಮನೆಯ ಯಜಮಾನರ ಸ್ವಭಾವವನ್ನು ಮತ್ತು ಆದರ್ಶವನ್ನು ಅದು ತೋರುವುದು. ಇಂತಹ ಭಾವವನ್ನು ವ್ಯಕ್ತಗೊಳಿಸುವ ವಾಸ್ತುಕಲೆಯನ್ನು ಇಟಲಿಯಲ್ಲಿ ವಿನಃ ನಾನು ಮತ್ತೆಲ್ಲಿಯೂ ನೋಡಲಿಲ್ಲ."

\newpage

\chapter[ಚಿಂತನೆಯ ಹನಿಗಳು—೨]{ಚಿಂತನೆಯ ಹನಿಗಳು—೨\protect\footnote{\engfoot{CW, Vol. VIII, P. 259}}}

೯೬. ಹಿಂದೂಗಳು ಏಕವೇ ಸತ್ಯ, ಅನೇಕವು ಮಿಥ್ಯ ಎಂದು ಸಾರಿದರೆ, ಬುದ್ಧ ಅನೇಕ ಸತ್ಯ, ಏಕವಾದ ಅಹಂಕಾರ ಮಿಥ್ಯ ಎಂದು ಬೋಧಿಸಿದನೇ ಎಂದು ಯಾರೋ ಸ್ವಾಮಿಗಳನ್ನು ಕೇಳಿದರು. ಅದಕ್ಕೆ ಸ್ವಾಮಿಗಳು “ಹೌದು, ಇದಕ್ಕೆ ಶ‍್ರೀರಾಮಕೃಷ್ಣ ಪರಮಹಂಸರು ಮತ್ತು ನಾವು ಸೇರಿಸಿರುವುದೇ, ಒಂದು ಮತ್ತು ಹಲವು ಎರಡೂ ಒಂದೇ ಸತ್ಯ, ಒಂದೇ ಮನಸ್ಸು ಭಿನ್ನಭಿನ್ನ ಕಾಲಗಳಲ್ಲಿ ಅವನ್ನು ಕಂಡದ್ದು” ಎಂದರು.

೯೭. ಜ್ಞಾಪಕದಲ್ಲಿಡಿ, ಭರತಖಂಡದ ಸಂದೇಶ ಯಾವಾಗಲೂ ಪ್ರಕೃತಿಗಾಗಿ ಅಲ್ಲ ಆತ್ಮನಿರುವುದು, ಆತ್ಮನಿಗಾಗಿ ಪ್ರಕೃತಿ ಇರುವುದು, ಎಂಬುದು.

೯೮. ಪ್ರಪಂಚಕ್ಕೆ ಇಂದು ಅಲ್ಲಿ ದಾರಿಯಲ್ಲಿ ನಿಂತು, ನಮಗೆ ದೇವರಲ್ಲದೆ ಮತ್ತೇನೂ ಬೇಕಾಗಿಲ್ಲ ಎಂದು ಹೇಳುವಂತಹ ಇಪ್ಪತ್ತು ಮಂದಿ ಸ್ತ್ರೀಪುರುಷರು ಬೇಕಾಗಿರುವರು. ಯಾರು ಇದನ್ನು ಮಾಡಬಲ್ಲರು? ಅಂಜುವುದು ಏತಕ್ಕೆ? ಇದು ಸತ್ಯವಾದರೆ ಮತ್ತೇನಾದರೇನಂತೆ? ಇದು ಸತ್ಯವಲ್ಲದೇ ಇದ್ದರೆ ನಮ್ಮ ಜೀವನವಿದ್ದು ಪ್ರಯೋಜನವೇನು?

೯೯. ಮಾನವನ ಪವಿತ್ರತೆಯನ್ನು ನಿಜವಾಗಿ ತಿಳಿದುಕೊಂಡವನು ಮಾಡುವ ಕೆಲಸ ಎಷ್ಟು ಶಾಂತಿಯಿಂದ ಕೂಡಿರುವುದು! ಅಂತವರು ಮತ್ತೇನೂ ಮಾಡಬೇಕಾಗಿಲ್ಲ, ಜನರ ಅಜ್ಞಾನವನ್ನು ಮಾತ್ರ ಪರಿಹರಿಸಬೇಕಾಗಿದೆ. ಉಳಿದುವುಗಳೆಲ್ಲ ತಾನೇ ಅಣಿಯಾಗುವುವು.

೧೦೦. ಶ‍್ರೀರಾಮಕೃಷ್ಣರು ಉತ್ತಮ ಜೀವನವನ್ನು ಬಾಳುವುದರಲ್ಲೇ ಕೃತಾರ್ಥರಾದರು; ಅದಕ್ಕೆ ವಿವರಣೆ ಕೊಡುವುದನ್ನು ಇತರರಿಗೆ ಬಿಟ್ಟರು.

೧೦೧. ಯಾರೋ ಒಬ್ಬರು ಶಿಷ್ಯರು ಅವರಿಗೆ ಸ್ವಲ್ಪ ಪ್ರಾಪಂಚಿಕ ಬುದ್ಧಿವಾದವನ್ನು ಕೊಡಲು ಯತ್ನಿಸಿದಾಗ ಹೀಗೆಂದರು: “ಯೋಜನೆ, ಯೋಜನೆ, ಆದಕಾರಣವೇ ಪಾಶ್ಚಾತ್ಯರು ಒಂದು ಧರ್ಮವನ್ನೂ ಸೃಷ್ಟಿಸಲಾರರು. ನಿಮ್ಮಲ್ಲಿ ಯಾರಾದರೂ ಧರ್ಮಕ್ಕೆ ಪ್ರಯತ್ನ ಪಟ್ಟಿದ್ದರೆ ಅವರೇ ಯಾವ ಯೋಜನೆಯೂ ಇಲ್ಲದ ಕ್ಯಾಥೋಲಿಕ್ ಮಹಾತ್ಮರು. ಯೋಜನೆ ಮಾಡುವವರು ಎಂದಿಗೂ ಧರ್ಮವನ್ನು ಸಾರಲಾರರು.

೧೦೨. ಪಾಶ್ಚಾತ್ಯರಲ್ಲಿ ಸಾಮಾಜಿಕ ಜೀವನ ಒಂದು ನಗೆಯ ಬುಗ್ಗೆಯಂತೆ ಇದೆ. ಆದರೆ ಅದರ ಕೆಳಗೆ ಇರುವುದು ಗೋಳು. ಅದೊಂದು ದುರಂತದಲ್ಲಿ ಕೊನೆಗಾಣುವುದು. ತಮಾಷೆ ನಗು ಇವೆಲ್ಲ ಮೇಲೆ ಮಾತ್ರ. ಆದರೆ ನಿಜವಾಗಿ ಅದರಲ್ಲಿ ತುಂಬಿರುವುದು ಯಾತನೆ. ಆದರೆ ಇಲ್ಲಿ ಭರತಖಂಡದಲ್ಲಿ ಹೊರಗೆಲ್ಲ ದುಃಖ, ಸಂಕಟ; ಆದರೆ ಒಳಗೆ ಸ್ವಚ್ಛಂದ, ಆನಂದ.”

ಹಿಂದೂಗಳಲ್ಲಿ ಒಂದು ಸಿದ್ಧಾಂತವಿದೆ: ದೇವರು ಕೇವಲ ಲೀಲೆಗಾಗಿ ಪ್ರಪಂಚವನ್ನು ಸೃಷ್ಟಿಸಿದ ಎಂದು. ಕೇವಲ ಲೀಲೆಗಾಗಿ ಅವತಾರಗಳು ಆದವು. ಇದೆಲ್ಲ ಒಂದು ಆಟ, ಬರಿದು ಆಟ. ಕ್ರಿಸ್ತನನ್ನು ಏತಕ್ಕೆ ಶಿಲುಬೆಗೆ ಏರಿಸಿದರು? ಇದೆಲ್ಲ ಬರಿಯ ಆಟ. ಇದರಂತೆಯೇ ಜೀವನ ಕೂಡ. ದೇವರೊಂದಿಗೆ ಆಟವಾಡಿ. ಇದೆಲ್ಲ ಲೀಲೆ, ಲೀಲೆ. ನೀನು ಏನಾದರೂ ಮಾಡುತ್ತಿಯೇನು?

೧೦೩. ನಾಯಕನಾಗುವುದಕ್ಕೆ ಒಂದು ಜನ್ಮ ಸಾಲದು. ಅವನು ಅದಕ್ಕಾಗಿಯೇ ಹುಟ್ಟಬೇಕು. ಸಂಸ್ಥೆಯನ್ನು ಕಟ್ಟುವುದು ಮತ್ತು ಯೋಜನೆಯನ್ನು ರೂಪಿಸುವುದು ಏನೂ ಕಷ್ಟವಲ್ಲ. ನಾಯಕನ ನಿಜವಾದ ಪರೀಕ್ಷೆ ಇರುವುದು ಭಿನ್ನ ಭಿನ್ನ ಪ್ರಕೃತಿಯವರನ್ನೆಲ್ಲಾ ಅವರಲ್ಲಿರುವ ಸಾಮಾನ್ಯ ಗುಣಗಳಿಗೆ ಅನುಸಾರವಾಗಿ ಒಟ್ಟಿಗೆ ಸೇರಿಸುವುದರಲ್ಲಿ. ಇದನ್ನು ನಾವು ಪ್ರಯತ್ನ ಪೂರ್ವಕ ಮಾಡಲಾಗುವುದಿಲ್ಲ, ಸ್ವಾಭಾವಿಕವಾಗಿಯೇ ಇದು ಆಗುತ್ತಿರಬೇಕು.

೧೦೪, ಪ್ಲೇಟೋವಿನ ಭಾವನಾಸಿದ್ಧಾಂತವನ್ನು \enginline{(Doctrine of Idea)} ಕುರಿತು ಸ್ವಾಮೀಜಿ ಹೀಗೆ ಹೇಳಿದರು.: "ನೋಡಿ ಇವೆಲ್ಲ ಮಹಾಭಾವನೆಗಳ ಅಸ್ಪಷ್ಟ ಆವಿರ್ಭಾವಗಳು. ಆ ಭಾವನೆಗಳು ಮಾತ್ರ ಪೂರ್ಣ ಸತ್ಯ. ಎಲ್ಲೋ ಒಂದು ಕಡೆ ಆದರ್ಶ ಎಂಬುದಿದೆ. ಇಲ್ಲಿ ಅದನ್ನು ವ್ಯಕ್ತಗೊಳಿಸುವ ಒಂದು ಪ್ರಯತ್ನವಿದೆ. ಹಲವು ವಿಧಗಳಲ್ಲಿ ಪ್ರಯತ್ನ ಯಶಸ್ವಿಯಾಗಿಲ್ಲ. ಆದರೂ ಪ್ರಯತ್ನ ಪಡಿ. ಎಂದಾದರೊಂದು ದಿನ ನೀವು ಆದರ್ಶವನ್ನು ಚೆನ್ನಾಗಿ ವ್ಯಕ್ತಗೊಳಿಸುತ್ತೀರಿ."

೧೦೫. ಈ ಜೀವನದಿಂದ ಪಾರಾಗಿ ಮುಕ್ತಿಯನ್ನು ಪಡೆಯುವುದಕ್ಕಿಂತ ಪುನಃ ಪುನಃ ಈ ಪ್ರಪಂಚಕ್ಕೆ ಬಂದು ತನಗೆ ಯಾವುದು ಮೇಲೆಂದು ತೋರುವುದೋ ಅದನ್ನು ಮಾಡಲು ಆಸೆ ಎಂದು ಬರೆದ ಶಿಷ್ಯರೊಬ್ಬರನ್ನು ಕುರಿತು ಹೀಗೆ ಬರೆದರು: "ಇದು ಏತಕ್ಕೆ ಎಂದರೆ ಪ್ರಗತಿ ಎಂಬ ಹುಚ್ಚಿನಿಂದ ನೀನು ಪಾರಾಗಲಾರೆ. ಆದರೆ ಪ್ರಪಂಚ ಮೇಲಾಗುವುದಿಲ್ಲ. ಅದು ಎಂದಿನಂತೆಯೇ ಇರುವುದು. ಪ್ರಪಂಚದಲ್ಲಿ ಮಾಡಿದ ಬದಲಾವಣೆಯಿಂದ ನಾವು ಮೇಲಾಗುವೆವು.

೧೦೬. ಸ್ವಾಮೀಜಿಯವರು ಆಲ್ಮೋರಾದಲ್ಲಿದ್ದಾಗ ಹಿರಿಯ ಮತ್ತು ದುರ್ಬಲನಾದ ವ್ಯಕ್ತಿಯೊಬ್ಬನು ಕರ್ಮದ ವಿಷಯವಾಗಿ ಒಂದು ಪ್ರಶ್ನೆಯನ್ನು ಹಾಕಿದರು. ಬಲಾಢ್ಯರು ಬಲಹೀನರನ್ನು ಪೀಡಿಸುವುದನ್ನು ನೋಡಿದಾಗ ಏನು ಮಾಡಬೇಕು ಎಂದು ಕೇಳಿದರು. 'ಬಲಾಢ್ಯರನ್ನು ಸದೆಬಡಿಯಿರಿ. ಕರ್ಮದಲ್ಲಿ ನಿಮ್ಮ ಪಾತ್ರವನ್ನು ಮರೆತಂತೆ ಕಾಣುವುದು. ಯಾವಾಗಲೂ ನಿಮಗೆ ದಂಗೆ ಏಳಲು ಸ್ವಾತಂತ್ರ್ಯವಿದೆ' ಎಂದರು ಸ್ವಾಮಿಜಿ.

೧೦೭. ಒಬ್ಬ "ನನ್ನ ರಕ್ಷಣೆಗಾಗಿ ಮೃತ್ಯುವನ್ನು ಎದುರಿಸುವುದು ಮೇಲೋ ಅಥವಾ ಗೀತೆಯಂತೆ, ಏನಾದರೂ ಆಗಲಿ ಎಂದು ಸುಮ್ಮನೆ ಇರುವುದು ಮೇಲೋ?" ಎಂದು ಕೇಳಿದ, ಸ್ವಾಮಿಗಳು "ನನ್ನ ದೃಷ್ಟಿಯಿಂದ ಯಾವ ಪ್ರತಿಕ್ರಿಯೆಯನ್ನೂ ಕೈಗೊಳ್ಳದೆ ಇರುವುದು" ಎಂದು, ಅನಂತರ ನಿಧಾನವಾಗಿ ಮೆಲ್ಲಗೆ ಹೇಳಿದರು; "ಆದರೆ ಇದು ಸಂನ್ಯಾಸಿಗಳಿಗೆ, ಗೃಹಸ್ಥರಿಗೆ ಆತ್ಮರಕ್ಷಣೆ ಇರಬೇಕು."

೧೦೮. ಎಲ್ಲರಿಗೂ ಸುಖವೇ ಪರಮಗುರಿಯೆಂದು ಆಲೋಚಿಸುವುದು ತಪ್ಪು. ಅನೇಕರು ದುಃಖವನ್ನೇ ಅರಸುವಂತೆ ಕಾಣುವುದು. ರೌದ್ರವನ್ನು ರೌದ್ರಕ್ಕಾಗಿ ಆರಾಧಿಸೋಣ.

೧೦೯. ನಾವು ಎಲ್ಲರೊಂದಿಗೂ ಅವರವರ ಭಾಷೆಯಲ್ಲಿ ಮಾತನಾಡಬೇಕು ಎಂದು ಹೇಳಲು ಧೈರ್ಯವಿದ್ದುದು ಶ‍್ರೀರಾಮಕೃಷ್ಣ ಪರಮಹಂಸರಿಗೆ ಮಾತ್ರ.

೧೧೦. ಕಾಳಿಕಾಮಾತೆಯನ್ನು ಸ್ವೀಕರಿಸುವಲ್ಲಿ ತಮಗಿದ್ದ ಹಿಂದಿನ ಸಂಶಯವನ್ನು ಕುರಿತು, ಸ್ವಾಮಿ ವಿವೇಕಾನಂದರು ಹೀಗೆ ಹೇಳುತ್ತಿದ್ದರು: "ನಾನು ಕಾಳಿಯನ್ನು ಹೇಗೆ ದ್ವೇಷಿಸುತ್ತಿದ್ದೆ! ಅವಳ ರೀತಿಯನ್ನೆಲ್ಲಾ ದ್ವೇಷಿಸುತ್ತಿದ್ದೆ. ನಾನು ಅವಳನ್ನು ಸ್ವೀಕರಿಸಲಿಲ್ಲ. ಅದಕ್ಕೆ ಆರು ವರುಷಗಳವರೆಗೆ ಹೋರಾಡಬೇಕಾಯಿತು. ಆದರೆ ಕೊನೆಗೆ ನಾನು ಅವಳನ್ನು ಸ್ವೀಕರಿಸಬೇಕಾಯಿತು. ಶ‍್ರೀರಾಮಕೃಷ್ಣ ಪರಮಹಂಸರು ನನ್ನನ್ನು ಅವಳಿಗೆ ಅರ್ಪಿಸಿದರು. ನಾನು ಮಾಡುತ್ತಿರುವುದನ್ನೆಲ್ಲಾ ಪ್ರೇರೇಪಿಸುತ್ತಿರುವವಳು ಅವಳೇ ಎಂದು ಈಗ ನಾನು ನಂಬುತ್ತೇನೆ. ಅವಳು ನನ್ನನ್ನು ಇಚ್ಛೆಬಂದಂತೆ ತಿರುಗಿಸುತ್ತಾಳೆ. ಆದರೂ ನಾನು ಅಷ್ಟು ದೀರ್ಘಕಾಲ ಹೋರಾಡಿದೆ. ನಾನು ಅವರನ್ನು ಪ್ರೀತಿಸುತ್ತಿದ್ದೆ. ಆದಕಾರಣವೇ ನಾನು ಬದ್ಧನಾಗಿದ್ದೆ. ಅವರ ಅದ್ಭುತವಾದ ಪವಿತ್ರತೆಯನ್ನು ನೋಡಿದೆ, ಅವರ ಅಮೋಘವಾದ ಪ್ರೇಮವನ್ನು ನೋಡಿದೆ. ಅವರ ಮಹಿಮೆ ಆಗ ನನಗೆ ಇನ್ನೂ ಗೋಚರವಾಗಿರಲಿಲ್ಲ. ನಾನು ಅವರಿಗೆ ಶರಣಾದ ಮೇಲೆ ಅದೆಲ್ಲಾ ಅರ್ಥವಾಯಿತು. ಆಗ ಅವರೊಬ್ಬರು ತಲೆ ಕೆಟ್ಟು ಹೋದ ಮನುಷ್ಯನೆಂದು ಭಾವಿಸಿದ್ದೆ. ಯಾವಾಗಲೂ ಯಾರಾದರೂ ದೇವದೇವಿಯರನ್ನು ಅವರು ನೋಡುತ್ತಿದ್ದರು. ನಾನು ಅದನ್ನು ದ್ವೇಷಿಸುತ್ತಿದ್ದೆ. ಕೊನೆಗೆ ನಾನೂ ಅವಳನ್ನು ಸ್ವೀಕರಿಸಬೇಕಾಯಿತು.

ನಾನು ಹಾಗೇಕೆ ಮಾಡಿದೆ ಎನ್ನುವುದೊಂದು ರಹಸ್ಯ, ಅದು ನನ್ನೊಡನೆಯೇ ಹೋಗುವುದು. ಆಗ ನನಗೆ ಬಹಳ ಕಷ್ಟವಾಗಿತ್ತು... ಅದೊಂದು ಅವಕಾಶವಾಗಿತ್ತು... ಅವಳು ನನ್ನನ್ನು ಕೊಂಡುಕೊಂಡಳು. ನನ್ನನ್ನು ಗುಲಾಮನನ್ನಾಗಿ ಮಾಡಿದಳು. ಇದೇ ಪದಗಳು... ಶ‍್ರೀರಾಮಕೃಷ್ಣ ಪರಮಹಂಸರು ನನ್ನನ್ನು ಅವಳಿಗೆ ಅರ್ಪಿಸಿದರು. ವಿಚಿತ್ರ! ಇದಾದಮೇಲೆ ಎಲ್ಲೊ ಎರಡು ವರುಷ ಅವರು ಬಾಳಿದರು ಮತ್ತು ಯಾವಾಗಲೂ ನರಳುತ್ತಿದ್ದರು. ಆರು ತಿಂಗಳಿಗಿಂತ ಮೇಲೆ ಅವರ ಆರೋಗ್ಯ ಎಂದೂ ಚೆನ್ನಾಗಿರಲಿಲ್ಲ.

ಗುರು ನಾನಕರು ಹೀಗೆಯೇ. ತಮ್ಮ ಶಕ್ತಿಯನ್ನೆಲ್ಲಾ ಕೊಡುವುದಕ್ಕೆ ಒಬ್ಬ ಶಿಷ್ಯನಿಗೆ ಕಾಯುತ್ತಿದ್ದರು. ಅವರು ತಮ್ಮ ಮನೆ ಮಕ್ಕಳು ಎಲ್ಲರನ್ನು ಮರೆತು ತಾವು ಕೊಡುವುದನ್ನು ಸ್ವೀಕರಿಸಲು ಯೋಗ್ಯನಾದ ಶಿಷ್ಯ ದೊರಕಿದ ಮೇಲೆ ಅವನಿಗೆ ಕೊಟ್ಟು ನಿರಾತಂಕವಾಗಿ ಸಾಯಲು ಸಿದ್ಧರಾದರು.

ಭವಿಷ್ಯದಲ್ಲಿ ಜನರು ಶ‍್ರೀರಾಮಕೃಷ್ಣ ಪರಮಹಂಸರನ್ನು ಕಾಳಿಕಾಮಾತೆಯ ಅವತಾರವೆನ್ನುತ್ತಾರೆ ಎನ್ನುವಿರಾ? ಹೌದು, ಅದರಲ್ಲಿ ಸಂಶಯವೇ ಇಲ್ಲ. ಅವಳು ಕೇವಲ ತನ್ನ ಕೆಲಸಕ್ಕಾಗಿ ಶ‍್ರೀರಾಮಕೃಷ್ಣರೆಂಬ ದೇಹವನ್ನು ಸೃಷ್ಟಿಸಿದಳೆಂದು ನಾನು ಭಾವಿಸುತ್ತೇನೆ. – ಕಾಳಿ ಮತ್ತು ತಾಯಿ ಎನ್ನುವ ಶಕ್ತಿ ಪ್ರಪಂಚದಲ್ಲಿ ಎಲ್ಲೋ ಒಂದು ಕಡೆ ಇದೆ ಎಂದು ನಾನು ಯೋಚಿಸದೆ ಇರಲಾರೆ... ನಾನು ಬ್ರಹ್ಮವನ್ನು ಕೂಡ ನಂಬುತ್ತೇನೆ. ಆದರೆ ಇದು ಯಾವಾಗಲೂ ಹೀಗೆಯೇ ಅಲ್ಲವೆ? ದೇಹದಲ್ಲಿರುವ ಹಲವು ಜೀವಾಣುಗಳಲ್ಲವೇ ನಮ್ಮ ವ್ಯಕ್ತಿತ್ವಕ್ಕೆ ಕಾರಣ? ನಮ್ಮ ಪ್ರಜ್ಞೆಗೆ ಕಾರಣ ಮಿದುಳಿನಲ್ಲಿರುವ ಹಲವು ಕೇಂದ್ರಗಳೇ ಹೊರತು ಯಾವುದೋ ಒಂದು ಕೇಂದ್ರವಲ್ಲ. ವೈವಿಧ್ಯದಲ್ಲಿ ಏಕತೆ. ಅದರಂತೆಯೇ ಇದು ಬ್ರಹ್ಮನ ವಿಷಯದಲ್ಲಿ ಏತಕ್ಕೆ ಬೇರೆ ಯಾಗಬೇಕು? ಈ ಬ್ರಹ್ಮ ಏಕ; ಆದರೂ ಇದು ಹಲವು ದೇವರುಗಳು ಕೂಡಿ ಆಗಿದೆ.

೧೧೧. ವಯಸ್ಸಾದಂತೆಲ್ಲಾ ಪೌರುಷದಲ್ಲಿ ಎಲ್ಲವೂ ಇರುವಂತೆ ನನಗೆ ತೋರುತ್ತದೆ. ಇದೇ ನನ್ನ ಹೊಸ ಸಂದೇಶ.

೧೧೨. ಸಮಾಜದಲ್ಲಿ ಕೆಲವು ಕಡೆಗಳಲ್ಲಿ ನರಭಕ್ಷಣೆ ಸ್ವಾಭಾವಿಕ ಎಂದು ಪಾಶ್ಚಾತ್ಯರು ಅಭಿಪ್ರಾಯಪಡುವರು. ಇದನ್ನು ಕುರಿತು ಸ್ವಾಮೀಜಿ ಹೀಗೆ ಹೇಳಿದರು: “ಇದು ನಿಜವಲ್ಲ. ಯಾವ ದೇಶದಲ್ಲೂ ಜನರು ನರಮಾಂಸವನ್ನು ಒಂದು ಯಾಗದಲ್ಲಿ ಇಲ್ಲವೆ ಯುದ್ಧದಲ್ಲಿ ದ್ವೇಷದಿಂದ ತಿಂದರೇ ವಿನಃ ಬೇರೆಯಲ್ಲ. ಇದು ಸಾಮಾಜಿಕ ಜೀವನಕ್ಕೇ ಕೊಡಲಿಯ ಪೆಟ್ಟು ಹಾಕಿದಂತೆ.”

೧೧೩. ಕಾಮ ಮತ್ತು ಸೃಷ್ಟಿ! ಇವೇ ಮುಕ್ಕಾಲು ಪಾಲು ಧರ್ಮಗಳ ಮೂಲದಲ್ಲಿರುವುದು. ಇದನ್ನೇ ಇಂಡಿಯಾ ದೇಶದಲ್ಲಿ ವೈಷ್ಣವರೆಂತಲೂ ಪಾಶ್ಚಾತ್ಯ ರಲ್ಲಿ ಕ್ರೈಸ್ತರೆಂತಲೂ ಹೇಳುವುದು. ಕಾಳಿಯನ್ನು ಅಥವಾ ಮೃತ್ಯುವನ್ನು ಎಷ್ಟು ಜನರು ಧೈರ್ಯವಾಗಿ ಆರಾಧಿಸಬಲ್ಲರು! ರೌದ್ರವನ್ನು ರೌದ್ರವೆಂದು ತಿಳಿದು ಆರಾಧಿಸೋಣ. ಅದರ ರೌದ್ರತೆಯನ್ನು ತಗ್ಗಿಸೆಂದು ಕೇಳಿಕೊಳ್ಳಕೂಡದು. ದುಃಖವನ್ನು ದುಃಖಕ್ಕಾಗಿ ಸ್ವೀಕರಿಸೋಣ.

೧೧೪, ಬೌದ್ಧ ಧರ್ಮದ ಮೂರು ಕಾಲಗಳೇ, ಐನೂರು ವರ್ಷಗಳ ಧರ್ಮ, ಐನೂರು ವರ್ಷಗಳ ವಿಗ್ರಹ, ಐನೂರು ವರ್ಷಗಳ ತಂತ್ರ. ಹಿಂದೂ ದೇಶದಲ್ಲಿ, ಬೌದ್ಧ ಧರ್ಮ ಎಂದೂ ತನ್ನದೇ ಪೂಜಾರಿ ಮತ್ತು ದೇವಸ್ಥಾನಗಳನ್ನೊಳಗೊಂಡ ಒಂದು ಪ್ರತ್ಯೇಕ ಧರ್ಮವಾಗಿತ್ತು ಎಂದು ತಪ್ಪು ತಿಳಿಯಬೇಡಿ. ಇಂತಹುದು ಯಾವುದೂ ಇರಲಿಲ್ಲ. ಇದು ಯಾವಾಗಲೂ ಹಿಂದೂ ಧರ್ಮದಲ್ಲಿಯೇ ಇತ್ತು. ಒಂದು ಕಾಲದಲ್ಲಿ ಬುದ್ಧನ ಪ್ರಭಾವ ಬಹಳ ಹೆಚ್ಚಾಗಿತ್ತು. ಅದರಿಂದ ದೇಶದಲ್ಲಿ ಸಂನ್ಯಾಸಿಗಳ ಸಂಖ್ಯೆ ಜಾಸ್ತಿಯಾಯಿತು.

೧೧೫. ಪೂರ್ವಸಂಪ್ರದಾಯದವರು ಯಾವಾಗಲೂ ಮತ್ತೊಂದರ ಎದುರಿಗೆ ಬಾಗುವ ಸ್ವಭಾವದವರು. ಆದರೆ ನಿಮ್ಮ ಸ್ವಭಾವವಾದರೋ ಹೋರಾಟ. ಆದಕಾರಣ ಜೀವನವನ್ನೇ ಆನಂದಿಸುವವರು ನಾವು, ನೀವಲ್ಲ. ನೀವು ಯಾವಾಗಲೂ ನಿಮ್ಮಲ್ಲಿರುವುದನ್ನು ಉತ್ತಮ ಸ್ಥಿತಿಗೆ ತರುವುದಕ್ಕೆ ಯತ್ನಿಸುವವರು, ಕೋಟಿಯ ಒಂದುಭಾಗ ಬದಲಾವಣೆ ಆಗುವುದೇ ತಡ ನೀವು ಕಾಲವಾಗುವಿರಿ. ಪಾಶ್ಚಾತ್ಯರ ಆದರ್ಶ ಏನನ್ನಾದರೂ ಸಾಧಿಸುವುದು, ಪ್ರಾಚ್ಯರ ಆದರ್ಶವಾದರೊ ಸಹಿಸುವುದು. ಪೂರ್ಣ ಜೀವನವಾದರೋ ಸಾಧನೆ ಮತ್ತು ಸಹನೆ ಇವುಗಳ ಒಂದು ಅಪೂರ್ವ ಸಮನ್ವಯ. ಆದರೆ ಇದು ಎಂದಿಗೂ ಸಾಧ್ಯವಿಲ್ಲ.

"ನಮ್ಮ ಸಿದ್ಧಾಂತವು ಮನುಷ್ಯ ಆಶಿಸುವುದನ್ನೆಲ್ಲ ಪಡೆಯಲಾರ ಎಂಬುದನ್ನು ಒಪ್ಪಿಕೊಳ್ಳುವುದು. ಜೀವನದಲ್ಲಿ ಎಷ್ಟೋ ವಿಧಿನಿಷೇಧಗಳಿವೆ. ಇದೇನೋ ಅಷ್ಟು ಹಿತಕರವಲ್ಲವಾದರೂ, ಅದು ಬೆಳಕನ್ನೂ ಶಕ್ತಿಯನ್ನೂ ಹೊರತರುತ್ತದೆ. ನಮ್ಮಲ್ಲಿ ಸ್ವತಂತ್ರ ಮನೋಭಾವದವರು ಈ ಹಿತಕರವಲ್ಲದುದನ್ನು ಮಾತ್ರ ನೋಡಿ ಅದನ್ನು ಆಚೆಗೆ ಒಗೆಯಲು ಯತ್ನಿಸುವರು. ಆದರೆ ಹಿಂದೆ ಇದ್ದ ಸ್ಥಿತಿಯಷ್ಟೇ ಕೆಟ್ಟದ್ದನ್ನು ಅದರ ಸ್ಥಳದಲ್ಲಿಡುವರು ಮತ್ತು ಹೊಸ ಆಚಾರದಲ್ಲಿರುವ ಒಳ್ಳೆಯದನ್ನು ಉಪಯೋಗಿಸಿಕೊಳ್ಳಬೇಕಾದರೆ ಹಿಂದಿನಷ್ಟೇ ಕಾಲವಾಗುವುದು."

ಬದಲಾವಣೆಯಿಂದ ಇಚ್ಚಾಶಕ್ತಿ ಬಲಶಾಲಿಯಾಗುವುದಿಲ್ಲ. ಅದರ ಬದಲು ಅದು ದುರ್ಬಲವಾಗಿ ಬದಲಾವಣೆಯ ಅಧೀನಕ್ಕೆ ಬರುವುದು. ಆದರೆ ನಾವು ಯಾವಾಗಲೂ ಒಳ್ಳೆಯದನ್ನು ಸ್ವೀಕರಿಸುವ ಸ್ಥಿತಿಯಲ್ಲಿರಬೇಕು. ಇಚ್ಛಾಶಕಿ, ಒಳ್ಳೆಯದನ್ನು ಸ್ವೀಕರಿಸುವುದರಿಂದ ಬಲಶಾಲಿಯಾಗುವುದು. ತಿಳಿದೊ ತಿಳಿಯದೆಯೊ ನಾವು ಮೆಚ್ಚುವ ಏಕಮಾತ್ರ ವಸ್ತುವೇ ಇಚ್ಛೆ. ಪ್ರಪಂಚದ ಕಣ್ಣಿಗೆ 'ಸತಿ' ಮಹತ್ತರವಾದುದು. ಏಕೆಂದರೆ ಅದರ ಹಿಂದೆ ಅದ್ಭುತವಾದ ಇಚ್ಛಾಶಕ್ತಿಯಿದೆ.

ಸ್ವಾರ್ಥತೆಯನ್ನು ನಾವು ಹೊರದೂಡಬೇಕಾಗಿದೆ. ನನ್ನ ಜೀವನದಲ್ಲಿ ತಪ್ಪು ಮಾಡಿದಾಗಲೆಲ್ಲ ವಿಮರ್ಶಿಸಿ ನೋಡಿದಾಗ ಸ್ವಾರ್ಥತೆ ಪ್ರವೇಶಿಸಿದ್ದುದರಿಂದ ಹಾಗೆ ಆಯಿತು ಎಂದು ಗೊತ್ತಾಗುವುದು. ಎಲ್ಲಿ ಸ್ವಾರ್ಥತೆ ಇಲ್ಲವೋ ಅಲ್ಲೆಲ್ಲ ನಾನು ಮಾಡಿರುವುದು ಸರಿಯಾಗಿರುತ್ತದೆ.

ನಾನೆಂಬುದಿಲ್ಲದೇ ಇದ್ದರೆ ಯಾವ ಧಾರ್ಮಿಕ ಸಿದ್ಧಾಂತವೂ ಇರುತ್ತಿರಲಿಲ್ಲ. ಮನುಷ್ಯನಿಗೆ ಏನೂ ಬೇಕಿಲ್ಲದೆ ಇದ್ದರೆ ಅವನು ಪೂಜೆ ಪ್ರಾರ್ಥನೆ ಮುಂತಾದುವನ್ನೆಲ್ಲ ಮಾಡುತ್ತಿದ್ದ ಎಂದು ಭಾವಿಸುವಿರೇನು? ಇಲ್ಲ, ಅವನು ದೇವರನ್ನು ಕುರಿತು ಆಲೋಚಿಸುತ್ತಲೇ ಇರಲಿಲ್ಲ. ಎಲ್ಲೋ ಯಾವಾಗಲಾದರೊಮ್ಮೆ ಸುಂದರವಾದ ದೃಶ್ಯವನ್ನು ನೋಡಿದಾಗ ದೇವರನ್ನು ಹೊಗಳುತ್ತಿದ್ದಿರಬಹುದು, ಅಷ್ಟೇ. ನಮ್ಮಲ್ಲಿ ನಿಜವಾಗಿ ಇರಬೇಕಾದ ದೃಷ್ಟಿಯೂ ಇದೇ ಆಗಿರಬೇಕು. ಎಲ್ಲಾ ಪ್ರಾರ್ಥನೆ ಮತ್ತು ಧನ್ಯವಾದಗಳೇ ಆಗಬೇಕು. ನಾವು ಈ ಸ್ವಾರ್ಥತೆಯಿಂದ ಪಾರಾಗಿದ್ದಿದ್ದರೆ!

ಹೋರಾಟ ಬೆಳವಣಿಗೆಯ ಚಿಹ್ನೆ ಎಂದು ಭಾವಿಸಿದರೆ ಅದು ತಪ್ಪು, ಅದೆಂದಿಗೂ ಇಲ್ಲ. ಹೀರಿಕೊಳ್ಳುವುದೇ ಬೆಳವಣಿಗೆಯ ಚಿಹ್ನೆ. ಒಳ್ಳೆಯದನ್ನು ಹೀರಿಕೊಳ್ಳುವ ಗುಣ ಅಪೂರ್ವವಾಗಿದೆ. ಹಿಂದೂಧರ್ಮದಲ್ಲಿ ನಾವೆಂದಿಗೂ ಹೋರಾಟವನ್ನು ಇಚ್ಛಿಸಿದವರಲ್ಲ. ಆದರೂ ನಮ್ಮ ಆತ್ಮರಕ್ಷಣೆಗಾಗಿ ಕೆಲವು ವೇಳೆ ಹೋರಾಡಿರುವೆವು. ಇದು ಧರ್ಮ. ಆದರೆ ಎಂದಿಗೂ ಹೋರಾಟಕ್ಕೆ ಕಾಲುಕೆರೆದವರಲ್ಲ. ಪ್ರತಿಯೊಬ್ಬರೂ ಇದನ್ನು ಕಲಿಯಬೇಕಾಯಿತು. ಹೊಸ ಜನಾಂಗಗಳು ಬೇಕಾದಷ್ಟು ನೆಗೆದಾಡಲಿ, ಕೊನೆಗೆ ಹಿಂದೂಧರ್ಮ ಇವನ್ನೆಲ್ಲ ಹೀರಿಕೊಳ್ಳುವುದು.

೧೧೬. ಎಲ್ಲಾ ಜೀವಗಳ ಮೊತ್ತವೇ ಈಶ್ವರ. ಸಮಷ್ಟಿಯ ಇಚ್ಛಾಶಕ್ತಿಯನ್ನು ಯಾರೂ ಎದುರಿಸಲಾರರು. ನಿಯಮದಂತೆ ಇರುವುದು ಇದೇ. ಇದನ್ನೇ ಶಿವ, ಕಾಳಿ ಎಂದು ಮುಂತಾಗಿ ಕರೆಯುವುದು.

೧೧೭. ರೌದ್ರವನ್ನು ಆರಾಧಿಸು! ಮೃತ್ಯುವನ್ನು ಆರಾಧಿಸು. ಉಳಿದವುಗಳೆಲ್ಲ ವ್ಯರ್ಥ. ಹೋರಾಟವೆಲ್ಲ ವ್ಯರ್ಥ. ಇದೇ ಕೊನೆಯ ಪಾಠ. ಆದರೂ ಇದು ಹೇಡಿಯ ಮೃತ್ಯು ಪ್ರೇಮವಲ್ಲ. ಪ್ರತಿಯೊಂದರ ಆಳವನ್ನೂ ಪರೀಕ್ಷಿಸಿ ಬೇರೇನೂ ಇಲ್ಲ ಎಂದು ಅರಿತ ಮಹಾ ಬಲಾಢ್ಯನ ಸ್ವಾಗತ ಇದು.

೧೧೮. ನಮ್ಮ ಜನರಿಗೆ ಹಳೆಯ ಮೂಢನಂಬಿಕೆಗಳನ್ನು ಕೊಡುವವರನ್ನು ನಾನು ಮೆಚ್ಚುವುದಿಲ್ಲ. ಇಜಿಪ್ಟಾಲಿಸ್ಟನು ಈಜಿಪ್ಟ್ ಜನಾಂಗದ ಮೇಲೆ ತೋರುವ ಸ್ವಾರ್ಥದ ಆಸಕ್ತಿಯಂತೆ ಇದು. ಕೇವಲ ಸ್ವಾರ್ಥದೃಷ್ಟಿಯಿಂದ ಭರತಖಂಡಕ್ಕೆ ತೋರುವ ಆಸಕ್ತಿಯಲ್ಲ ನಮಗೆ ಬೇಕಾಗಿರುವುದು. ಒಬ್ಬನು ತನ್ನದೇ ಮನೋಭಾವಕ್ಕನುಗುಣವಾದ ಭಾರತವನ್ನು ನೋಡಲಿಚ್ಛಿಸಬಹುದು. ಆದರೆ ಭಾರತದಲ್ಲಿ ಯಾವುದು ಬಲಯುತವಾದುದೊ ಅದು ಈ ಯುಗದ ಉತ್ತಮವಾದುದರೊಂದಿಗೆ ಸ್ವಾಭಾವಿಕವಾಗಿ ಬೆರೆಯಬೇಕೆಂಬುದು ನನ್ನ ಆಕಾಂಕ್ಷೆ. ಈ ಹೊಸ ಸ್ಥಿತಿ ಒಳಗಿನಿಂದ ಬೆಳೆದು ಬಂದುದಾಗಿರಬೇಕು.

ನಾನು ಉಪನಿಷತ್ತುಗಳನ್ನು ಮಾತ್ರ ಬೋಧಿಸುತ್ತೇನೆ. ನೀವು ಪರೀಕ್ಷಿಸಿದರೆ ಉಪನಿಷತ್ತಲ್ಲದೆ ನಾನು ಬೇರೇನನ್ನೂ ಹೇಳಿಲ್ಲ ಎಂದು ಗೊತ್ತಾಗುವುದು. ಉಪನಿಷತ್ತುಗಳಲ್ಲಿಯೂ ಒಂದು ಭಾವನೆಯನ್ನು ಮಾತ್ರ ಬೋಧಿಸಿರುವೆನು, ಅದೇ ಬಲ. ವೇದವೇದಾಂತಗಳ ಸಾರವೆಲ್ಲ ಆ ಒಂದು ಪದದಲ್ಲಿದೆ. ಬುದ್ಧ ಅಹಿಂಸೆಯನ್ನು ಬೋಧಿಸಿದ. ಆದರೆ ಆ ಸಂದೇಶವನ್ನು ಬೋಧಿಸುವ ಉತ್ತಮ ಮಾರ್ಗವೇ ಇದು ಎಂದು ತೋರುವುದು. ಆ ಅಹಿಂಸೆಯ ಹಿಂದೆ ಅತಿ ದುರ್ಬಲತೆ ಇತ್ತು. ಪ್ರತಿಕ್ರಿಯೆಯ ಭಾವನೆಯೇ ಒಂದು ದುರ್ಬಲತೆ. ನಾನು ಬರಿಯ ನೊರೆಯಿಂದ ತಪ್ಪಿಸಿಕೊಳ್ಳಬೇಕು ಅಥವಾ ಅದನ್ನು ಶಿಕ್ಷಿಸಬೇಕು ಎಂದು ಆಲೋಚಿಸುವುದಿಲ್ಲ. ನಾನು ಅದನ್ನು ಗಣನೆಗೆ ತರುವುದಿಲ್ಲ. ಆದರೆ ಸೊಳ್ಳೆಗೆ ಅದೇ ಒಂದು ತುಂಬಾ ದೊಡ್ಡ ವಿಷಯ. ನಾನು ಎಲ್ಲಾ ಹಿಂಸೆಯನ್ನೂ ಇಷ್ಟೇ ಲಘುವಾಗಿ ನೋಡುತ್ತೇನೆ. ಶಕ್ತಿ ಮತ್ತು ಧೈರ್ಯ! ನನ್ನ ಆದರ್ಶ ಸಿಪಾಯಿದಂಗೆಯಲ್ಲಿ ಸತ್ತ ಆ ಸಾಧು. ಯಾರೋ ಅವನನ್ನು ಭರ್ಜಿಯಿಂದ ತಿವಿದರು. ಆಗ ಅವನು "ನೀನು ಕೂಡ ಅವನೆ" ಎಂದು ಉಚ್ಚರಿಸಿ ತೆಪ್ಪಗಾದ.

ಇಲ್ಲಿ ಶ‍್ರೀರಾಮಕೃಷ್ಣರ ಸ್ಥಾನವಾವುದು ಎಂದು ನೀವು ಕೇಳಬಹುದು. ಅವರೇ ಮಾರ್ಗ, ಅದ್ಭುತವಾದ, ಅರಿವಿಗೆ ನಿಲುಕದ ಮಾರ್ಗ. ಅವರಿಗೆ ತಾವಾರು ಎಂಬುದೇ ಗೊತ್ತಿರಲಿಲ್ಲ. ಅವರಿಗೆ ಇಂಗ್ಲೆಂಡ್ ಎಲ್ಲಿದೆ, ಇಂಗ್ಲೀಷರು ಯಾರು, ಎಂಬುದೇ ಗೊತ್ತಿರಲಿಲ್ಲ. ಅವರೆಲ್ಲೊ ಸಮುದ್ರದಾಚೆಯಿಂದ ಬಂದ ವಿಚಿತ್ರ ಜನಾಂಗ ಎಂದು ತಿಳಿದಿದ್ದರು. ಆದರೆ ಅವರು ಭವ್ಯಜೀವನವನ್ನು ನಡೆಸಿದರು. ನಾನು ಅದರ ಅರ್ಥವನ್ನು ಓದಿದೆ. ಅವರು ಯಾರನ್ನೂ ಒಮ್ಮೆಯೂ ದೂರಿದವರಲ್ಲ. ನಾನು ಒಂದು ದಿನ ಕರ್ತಭಾಜ ಎಂಬ ಒಂದು ಪಂಗಡವನ್ನು ಟೀಕಿಸುತ್ತಿದ್ದೆ. ನಾನು ಮೂರು ಗಂಟೆಗಳ ಕಾಲ ಮನಸ್ಸಿಗೆ ಬಂದಂತೆ ಹರಟುತ್ತಿದ್ದೆ. ಅವರು ಸಾವಧಾನದಿಂದ ಎಲ್ಲವನ್ನೂ ಕೇಳಿದರು. ನಾನು ಮಾತನ್ನು ಮುಗಿಸಿದ ಮೇಲೆ ಆ ವೃದ್ಧರು, "ಆಗಲಿ, ಆಗಲಿ, ಪ್ರತಿಯೊಂದು ಮನೆಗೂ ಒಂದು ಹಿಂದಿನ ಬಾಗಿಲಿರಬಹುದು, ಯಾರಿಗೆ ಗೊತ್ತು?" ಎಂದರು.

ಇದುವರೆಗೆ ನಮ್ಮ ಹಿಂದೂಧರ್ಮದಲ್ಲಿದ್ದ ದೊಡ್ಡದೊಂದು ಕೊರತೆಯೇ ಅದಕ್ಕೆ ಎರಡು ಪದಗಳು ಮಾತ್ರ ಗೊತ್ತಿದ್ದುದು. ಅದೇ ತ್ಯಾಗ ಮತ್ತು ಮುಕ್ತಿ. ಇಲ್ಲಿ ಮುಕ್ತಿ ಮಾತ್ರ! ಗೃಹಸ್ಥರಿಗೆ ಏನೂ ಇಲ್ಲ!

ಆದರೆ ನಾನು ಸಹಾಯ ಮಾಡಬೇಕೆಂದಿರುವುದೇ ಇವರಿಗೆ. ಎಲ್ಲರ ಆತ್ಮವೂ ಒಂದೇ ಅಲ್ಲವೆ? ಎಲ್ಲರ ಗುರಿಯೂ ಒಂದೇ ಅಲ್ಲವೆ? ವಿದ್ಯಾಭ್ಯಾಸದ ಮೂಲಕ ಇಡೀಯ ದೇಶಕ್ಕೆ ಶಕ್ತಿ ಬರಬೇಕಾಗಿದೆ.

೧೧೯. ಜನಸಾಮಾನ್ಯರಿಗೆ ಶ್ರೇಷ್ಠ ಭಾವನೆಗಳನ್ನು ತರುವುದೇ ಪುರಾಣದ ಗುರಿಯಾಗಿತ್ತು. ಈ ಆವಶ್ಯಕತೆಯನ್ನು ಮನಗಂಡ ಏಕ ಮಾತ್ರ ಮಹಾತ್ಮನೇ ಮಾನವಕೋಟಿಯ ಶ್ರೇಷ್ಠತಮ ಮಾನವನಾದ ಶ‍್ರೀಕೃಷ್ಣನಿರಬಹುದು.

ಇದರಿಂದ ಜೀವನವನ್ನು ಉಳಿಸಿಕೊಂಡು ಅನುಭವಿಸುವುದರ ಮೂಲಕ ಭಗತ್ಸಾಕ್ಷಾತ್ಕಾರವನ್ನು ಪಡೆಯುವ ವಿಷ್ಣುವಿನ ಆರಾಧನೆಯ ರೂಪದ ಧರ್ಮವು ಹುಟ್ಟಿಕೊಂಡಿತು. ನಮ್ಮ ಕೊನೆಯ ಧಾರ್ಮಿಕ ಆಂದೋಳನವಾದ ಚೈತನ್ಯ ಪಂಥದ ಗುರಿಯು ಸುಖಾನುಭವ. ಆದರೆ ಜೈನರು ಮತ್ತೊಂದು ಅತಿಗೆ ಸೇರಿದವರು, ಆತ್ಮ ಹಿಂಸೆಯಿಂದ ನಿಧಾನವಾಗಿ ಸಾಯುವುದೇ ಅವರ ಗುರಿ. ಆದಕಾರಣ ಬೌದ್ಧ ಧರ್ಮ ಸುಧಾರಿತ ಜೈನ ಪಂಥವಾಗಿದೆ. ಬುದ್ಧ ಐದು ಜನ ಮುನಿಗಳ ಸಹವಾಸವನ್ನು ಬಿಟ್ಟುದರ ಅರ್ಥ ಇದು. ಭರತಖಂಡದಲ್ಲಿ ಪ್ರತಿಯೊಂದು ಕಾಲದಲ್ಲಿಯೂ ದೇಹ ದಂಡಿಸಿಕೊಳ್ಳುವುದರಿಂದ ಹಿಡಿದು, ಭೋಗದ ಪರಾಕಾಷ್ಠೆಯವರೆಗೆ ಹಲವು ಪಂಗಡಗಳು ಇವೆ. ಅದೇ ಕಾಲದಲ್ಲಿ ಹಲವು ತತ್ತ್ವ ಸಂಪ್ರದಾಯಗಳೂ ಜಾರಿಗೆ ಬಂದಿವೆ. ಇಂದ್ರಿಯ ನಾಶದಿಂದ ಹಿಡಿದು ಇಂದ್ರಿಯ ತೃಪ್ತಿಯವರೆಗೆ ಎಲ್ಲಾ ಮಾರ್ಗಗಳ ಮೂಲಕ ಭಗವಂತನ ಸಾಕ್ಷಾತ್ಕಾರಕ್ಕೆ ಯತ್ನಿಸುವರು. ಹಿಂದೂಧರ್ಮದಲ್ಲಿ ಒಂದೇ ಕೇಂದ್ರದ ಸುತ್ತಲೂ ಎರಡು ವೃತ್ತಗಳು ಇರುವಂತೆ ಕಾಣುವುವು.

"ತಂದೆ ತಾಯಿ ಸಹೋದರ ಗಂಡ ಮಗು ಇವರಿಗೆ ಇರುವ ಅದ್ಭುತ ಪ್ರೇಮವೇನೋ ಸರಿ. ಆದರೆ ಶ‍್ರೀಕೃಷ್ಣನನ್ನೇ ಈ ಸ್ಥಳಕ್ಕೆ ತಂದು ಅವನೇ ನಮ್ಮ ಮಗು, ಅವನಿಗೆ ನಾವು ಊಟವನ್ನು ಕೊಡುತ್ತಿರುವೆವು ಎಂದು ಭಾವಿಸಿ" ಎನ್ನುವರು ವೈಷ್ಣವರು. ಇಂದ್ರಿಯಗಳ ಮೂಲಕ ಭಗವಂತನನ್ನು ಆರಾಧಿಸಿ ಎನ್ನುವುದೇ ವೈಷ್ಣವರ ಮತ. "ಇಂದ್ರಿಯಗಳನ್ನೇ ನಿಗ್ರಹಿಸಿ, ಅವುಗಳನ್ನು ನಿಮ್ಮ ಅಧೀನದಲ್ಲಿಟ್ಟುಕೊಳ್ಳಿ" ಎನ್ನುವುದು ವೇದಾಂತ ಮತ.

ಇಂಡಿಯಾದೇಶ ಇನ್ನೂ ತಾರುಣ್ಯಾವಸ್ಥೆಯಲ್ಲಿದೆ, ಇನ್ನೂ ಬದುಕಿದೆ. ಯೂರೋಪು ಖಂಡವೂ ತಾರುಣ್ಯದಲ್ಲಿದೆ ಮತ್ತು ಬದುಕಿದೆ. ಯಾರೂ ಮತ್ತೊಬ್ಬರನ್ನು ದೂರುವ ಬೆಳವಣಿಗೆಯ ಸ್ಥಿತಿಗೆ ಇನ್ನೂ ಬಂದಿಲ್ಲ. ಇವೆರಡೂ ದೊಡ್ಡ ಪ್ರಯೋಗಗಳು. ಇವೆರಡರಲ್ಲಿ ಯಾವುದೊಂದೂ ಇನ್ನೂ ಪೂರ್ಣವಾಗಿಲ್ಲ, ಇಂಡಿಯಾ ದೇಶದಲ್ಲಿ ಅದ್ವೈತ ಜ್ಞಾನದೊಂದಿಗೆ ಸಾಮಾಜಿಕ ಸಮತಾವಾದ ಇದೆ. ಆಧ್ಯಾತ್ಮಿಕ ವೈಯಕ್ತಿಕತೆಯನ್ನು ಪ್ರತಿಪಾದಿಸುವ ಯುರೋಪಿನಲ್ಲಿ ನೀವು ಸಾಮಾಜಿಕ ದೃಷ್ಟಿಯಿಂದ ಬೇರೆ ಬೇರೆ. ಆದರೆ ಆಲೋಚನೆಯಲ್ಲಿ ನೀವು ದ್ವೈತಿಗಳು. ಇದೇ ಆಧ್ಯಾತ್ಮಿಕ ಸಮತಾವಾದ. ಒಂದರಲ್ಲಿ ವೈಯಕ್ತಿಕ ಸ್ವಾತಂತ್ರ್ಯದಿಂದ ರಕ್ಷಿಸಲ್ಪಟ್ಟ ಸಾಮಾಜಿಕ ಸಂಸ್ಥೆಗಳಿವೆ. ಮತ್ತೊಂದರಲ್ಲಿ ಸಮತಾವಾದದಿಂದ ರಕ್ಷಿಸಲ್ಪಟ್ಟ ವೈಯಕ್ತಿಕ ಸಂಸ್ಥೆಗಳಿವೆ.

ಇಂಡಿಯಾದೇಶದಲ್ಲಿ ಈಗ ಪ್ರಯೋಗ ಹೇಗೆ ನಡೆಯುತ್ತಿದೆಯೋ ಅದಕ್ಕೆ ಅನುಗುಣವಾಗಿ ಸಹಾಯ ಮಾಡಬೇಕು. ಈಗ ಇರುವ ಸ್ಥಿತಿಗನುಗುಣವಾಗಿ ಅದಕ್ಕೆ ಸಹಾಯ ಮಾಡದೆ ಇದ್ದರೆ ಏನೂ ಪ್ರಯೋಜನವಿಲ್ಲ. ಯುರೋಪಿನಲ್ಲಿ ನಾನು ಮದುವೆಯಾಗದಿರುವವರನ್ನು ಮೆಚ್ಚುವಂತೆಯೇ ಮದುವೆ ಆಗಿರುವವರನ್ನೂ ಮೆಚ್ಚುತ್ತೇನೆ. ಮನುಷ್ಯನಲ್ಲಿರುವ ಗುಣಗಳಷ್ಟೇ ಅವಗುಣಗಳು ಕೂಡ ಅವನನ್ನು ಪೂರ್ಣಾತ್ಮನನ್ನಾಗಿ ಮತ್ತು ಮಹಾತ್ಮನನ್ನಾಗಿ ಮಾಡುತ್ತವೆ ಎಂಬುದನ್ನು ಮರೆಯಬೇಡಿ. ಆದಕಾರಣವೆ ಒಂದು ಜನಾಂಗದ ವೈಶಿಷ್ಟ್ಯವನ್ನು, ಅದೆಲ್ಲ ತಪ್ಪು ಎಂದು ತೋರಿದರೂ, ಎಂದಿಗೂ ಹಾಳುಮಾಡಕೂಡದು.

೧೨೦. ವಿಗ್ರಹದಲ್ಲಿ ದೇವರು ಇದ್ದಾನೆ ಎಂದು ಬೇಕಾದರೆ ನೀನು ಹೇಳಬಹುದು. ಆದರೆ ದೇವರೇ ವಿಗ್ರಹ ಎಂದು ಭಾವಿಸುವ ತಪ್ಪನ್ನು ಮಾಡಬಾರದು.

೧೨೧. ಹಾಟೆಂಟಾಟ್ ಜನರ ವಸ್ತುಪೂಜೆಯನ್ನು ನೀವು ಖಂಡಿಸಿ ಎಂದು ಸ್ವಾಮೀಜಿ ಅವರಿಗೆ ಯಾರೋ ಹೇಳಿದರು. ಸ್ವಾಮೀಜಿಯವರು "ಹಾಗೆಂದರೆ ಏನೋ ನನಗೆ ಗೊತ್ತಿಲ್ಲ" ಎಂದರು. ಆಗ ಹಾಗೆಂದರೆ ಏನೆಂಬುದನ್ನು ವಿವರಿಸಲಾಯಿತು: ಒಂದು ವಸ್ತುವನ್ನು ಪೂಜಿಸುವುದು, ಅನಂತರ ಅದನ್ನೇ ಹೊಡೆಯುವುದು, ಪುನಃ ಅದನ್ನು ಪೂಜಿಸುವುದು ಇತ್ಯಾದಿಯಾಗಿ ವಿವರಿಸಿದರು. ಆಗ ಸ್ವಾಮೀಜಿಯವರು "ನಾನು ಅದನ್ನೇ ಮಾಡುತ್ತೇನೆ; ನಿಮಗೆ ಗೊತ್ತಿಲ್ಲವೆ" ಎಂದರು. ದೀನರಿಗೆ, ಮಂದಬುದ್ಧಿಯವರಿಗೆ ಜನರು ಮಾಡುತ್ತಿರುವ ಅನ್ಯಾಯವನ್ನು ಸ್ಮರಿಸಿ ಅವರು ಹೀಗೆಂದರು: "ಇಲ್ಲಿ ವಸ್ತುಪೂಜೆ ಏನೂ ಇಲ್ಲ ಎನ್ನುವುದು ನಿಮಗೆ ಗೊತ್ತಾಗುವುದಿಲ್ಲವೆ? ನಿಮ್ಮ ಹೃದಯವೆಲ್ಲ ಬತ್ತಿಹೋಗಿದೆ. ಮಗು ಮಾಡುವುದು ಸರಿ ಎನ್ನುವುದು ನಿಮಗೆ ಗೊತ್ತಿಲ್ಲ. ಮಗು ಎಲ್ಲಾ ಕಡೆಗಳಲ್ಲಿಯೂ ವ್ಯಕ್ತಿಯನ್ನು ನೋಡುವುದು. ಜ್ಞಾನ ನಮಗೆ ಮಗುವಿನ ದೃಷ್ಟಿ ಇಲ್ಲದಂತೆ ಮಾಡುವುದು. ಆದರೆ ಕೊನೆಗೆ ನಾವು ಪರಮ ಜ್ಞಾನದಿಂದ ಅದನ್ನು ಪಡೆಯಬಹುದು. ಬಂಡೆ ಕೋಲು ಮರಗಳೆಲ್ಲಾ ಅವನಿಗೆ ಸಜೀವವಾಗಿ ಕಾಣುವುವು. ಅದರ ಹಿಂದೆ ಒಂದು ಸಜೀವ ಶಕ್ತಿ ಇಲ್ಲವೆ? ಇದೊಂದು ಪ್ರತೀಕ, ವಸ್ತುಪೂಜೆ ಅಲ್ಲ. ಇದು ನಿಮಗೆ ಕಾಣುವುದಿಲ್ಲವೆ?"

೧೨೨. ಸ್ವಾಮೀಜಿಯವರು ಒಂದು ದಿನ ಸತ್ಯಭಾಮೆಯ ತ್ಯಾಗವನ್ನು ಕುರಿತು ಮಾತನಾಡಿದರು: ಒಂದು ಸಲ ಕೃಷ್ಣನನ್ನು ತಕ್ಕಡಿಯಲ್ಲಿಟ್ಟು ತೂಗುತ್ತಿದ್ದಾಗ, ಅವನ ಹೆಸರನ್ನು ಬರೆದ ಚೀಟಿಯನ್ನು ಒಂದು ಕಡೆ ಇಟ್ಟಾಗ ಅದೇ ಕೃಷ್ಣನಿಗಿಂತ ಭಾರವಾಗಿ ಹೋಯಿತು. ಸಾಂಪ್ರದಾಯಿಕ ಹಿಂದೂಗಳು ಶ್ರುತಿ, ಅಂದರೆ ಶಬ್ದವನ್ನೇ ಮುಖ್ಯವಾಗಿ ಮಾಡುವರು. ಎದುರಿಗಿರುವ ವಸ್ತು ಸನಾತನವಾದ ಭಾವನೆಯ ಅಸ್ಪಷ್ಟವಾದ ಆವಿರ್ಭಾವ ಅಷ್ಟೆ. ಆದ್ದರಿಂದ ದೇವರ ಹೆಸರೇ ಸರ್ವವೂ. ಸನಾತನ ಮನಸ್ಸಿನಲ್ಲಿರುವ ಭಾವನೆಯ ವಿಷಯೀಕರಣವೇ ದೇವರು. ನಿಮಗಿಂತ ನಿಮ್ಮ ಹೆಸರು ಅನಂತವಾಗಿ ಪೂರ್ಣವಾಗಿದೆ. ದೇವರ ಹೆಸರು ದೇವರಿಗಿಂತ ದೊಡ್ಡದು. ನಿಮ್ಮ ಮಾತಿನಲ್ಲಿ ಜೋಪಾನವಾಗಿರಿ.

೧೨೩. ಗ್ರೀಕರ ದೇವತೆಗಳನ್ನು ಕೂಡ ನಾನು ಪೂಜಿಸುವುದಿಲ್ಲ. ಏಕೆಂದರೆ ಅವರು ಮಾನವಕೋಟಿಯಿಂದ ಬೇರೆಯಾಗಿದ್ದರು. ಯಾರು ನಮ್ಮಂತೆ ಇದ್ದು ನಮಗಿಂತ ಮೇಲಾಗಿರುವರೊ ಅವರನ್ನು ಮಾತ್ರ ಪೂಜಿಸಬೇಕು. ನನಗೂ ದೇವತೆಗಳಿಗೂ ಇರುವ ವ್ಯತ್ಯಾಸ ಕೇವಲ ತರತಮದಲ್ಲಿರಬೇಕು.

೧೨೪. ಕಲ್ಲೊಂದು ಬಿದ್ದು ಕೀಟವನ್ನು ಕೊಲ್ಲುವುದು. ಆದಕಾರಣ ಬೀಳುವ ಕಲ್ಲುಗಳೆಲ್ಲಾ ಕೀಟಗಳನ್ನು ಕೊಲ್ಲುವುವು ಎಂದು ನಿರ್ಧರಿಸುತ್ತೇವೆ. ನಾವು ಏತಕ್ಕೆ ತಕ್ಷಣ ನಿರ್ಧರಿಸುತ್ತೇವೆ? ಅನುಭವದಿಂದ ಎಂದು ಒಬ್ಬರು ಹೇಳುವರು. ಆದರೆ ಈ ಅನುಭವ ಮೊದಲನೆಯ ವೇಳೆಯೇ ಆಗುತ್ತದೆ ಎಂದು ಭಾವಿಸಿ, ಮಗುವನ್ನು ನೀವು ಮೇಲಕ್ಕೆ ಎಸೆದರೆ ಅದು ಅಳುವುದು. ಹಿಂದಿನ ಜನ್ಮದ ಅನುಭವವೇ ಇದಕ್ಕೆ ಕಾರಣವೇ? ಆದರೆ ಅದನ್ನು ಮುಂದಿನ ಜನ್ಮಕ್ಕೆ ಏತಕ್ಕೆ ಅನ್ವಯಿಸುತ್ತೀರಿ? ಏಕೆಂದರೆ ಹಿಂದಕ್ಕೂ ಮುಂದಕ್ಕೂ ಒಂದು ಸಂಬಂಧವಿದೆ. ಹಿಂದಿನಂತೆಯೇ ಮುಂದೆ ಆಗುವುದು. ಆದರೆ ಕಾರಣ ಹೆಚ್ಚು ಕಡಿಮೆಯಾಗದಂತೆ ನಾವು ನೋಡಿಕೊಳ್ಳಬೇಕು ಅಷ್ಟೆ. ಈ ವಿವೇಚನೆಯ ಮೇಲೆಯೇ ಮಾನವ ಜ್ಞಾನವೆಲ್ಲ ನಿಂತಿರುವುದು.

ಪ್ರತ್ಯಕ್ಷಾನುಭವ ಒಂದು ಪ್ರಮಾಣ ಯಾವಾಗ ಆಗುವುದು ಎಂದರೆ, ಕರಣ, ವಿಧಾನ, ಇಂದ್ರಿಯಗ್ರಹಣ ಇವುಗಳೆಲ್ಲ ಪರಿಶುದ್ಧವಾದಾಗ. ರೋಗ ಅಥವಾ ಭಾವೋದ್ವೇಗಗಳು ಇಂದ್ರೀಯ ಗ್ರಹಣವನ್ನು ವಿಕೃತಗೊಳಿಸುತ್ತವೆ. ಆದಕಾರಣ ಪ್ರತ್ಯಕ್ಷಪ್ರಮಾಣ ಕೂಡ ಒಂದು ರೀತಿಯ ಅನುಮಾನವೇ. ಆದಕಾರಣ ಎಲ್ಲಾ ಮಾನವವಿಧ್ಯೆಯೂ ಅನುಮಾನಾಸ್ಪದ ಮತ್ತು ಸುಳ್ಳಾಗಿ ಇರಬಹುದು. ನಿಜವಾದ ಸಾಕ್ಷಿ ಯಾರು? ಯಾವನಿಗೆ ಒಂದು ವಸ್ತುವು ಪ್ರತ್ಯಕ್ಷವಾಗಿ ಕಾಣುತ್ತದೊ ಅವನೇ ಅದರ ಸಾಕ್ಷಿ. ಆದಕಾರಣವೇ ವೇದ ಸತ್ಯ. ಏಕೆಂದರೆ ಅದು ಆಪ್ತವಾಕ್ಯಗಳಿಂದ ಕೂಡಿದೆ. ಆದರೆ ಈ ಗ್ರಹಣಶಕ್ತಿ ಎಲ್ಲೋ ಕೆಲವರಿಗೆ ಮಾತ್ರ ಮೀಸಲೆ? ಇಲ್ಲ! ಇದು ಋಷಿಗಳು ಆರ್ಯರು ಮ್ಲೇಚ್ಛರು ಇವರೆಲ್ಲರಲ್ಲೂ ಇದೆ.

ಆಪ್ತವಾಕ್ಯ ಕೂಡ ಮತ್ತೊಂದು ಬಗೆಯ ಪ್ರತ್ಯಕ್ಷಪ್ರಮಾಣ ಎಂದು ಆಧುನಿಕ ವಂಶೀಯರು ಭಾವಿಸುವರು. ಉಪಮಾನ ಮತ್ತು ಯುಕ್ತಿ ಅಷ್ಟೇನೂ ಸಾಧುವಾದ ಪ್ರಮಾಣಗಳಲ್ಲ. ಆದಕಾರಣ ನಿಜವಾದ ಪ್ರಮಾಣಗಳು ಎರಡು ಮಾತ್ರ: ಪ್ರತ್ಯಕ್ಷ ಮತ್ತು ಅನುಮಾನ.

೧೨೫. ಕೆಲವರು ಬಾಹ್ಯ ಆವಿರ್ಭಾವಕ್ಕೆ ಪ್ರಾಧಾನ್ಯಕೊಡುವರು. ಮತ್ತೆ ಕೆಲವರು ಅಂತರಿಕ ಭಾವಕ್ಕೆ ಪ್ರಾಧಾನ್ಯ ಕೊಡುವರು. ಇದರಲ್ಲಿ ಯಾವುದು ಮೊದಲು? ಮೊಟ್ಟೆಗೆ ಮುಂಚೆ ಹಕ್ಕಿಯೆ ಅಥವಾ ಹಕ್ಕಿಗೆ ಮುಂಚೆ ಮೊಟ್ಟೆಯೆ? ಎಣ್ಣೆ ಪಾತ್ರೆಗೆ ಆಧಾರವೆ, ಪಾತ್ರೆ ಎಣ್ಣೆಗೆ ಆಧಾರವೆ? ಈ ಸಮಸ್ಯೆಗೆ ಪರಿಹಾರವೇ ಇಲ್ಲ. ಇದನ್ನು ಬಿಟ್ಟುಬಿಡಿ. ಮಾಯೆಯಿಂದ ಪಾರಾಗಿ.

ಪ್ರಪಂಚವೇ ಮಾಯವಾಗಿ ಹೋದರೆ ಅದರಿಂದ ನನಗೇನಂತೆ? ನಿಮಗೆ ತಿಳಿದಂತೆ ನನ್ನ ತತ್ತ್ವ ದೃಷ್ಟಿಯಿಂದ ಇದು ಬಹಳ ಉತ್ತಮ. ಆದರೆ ಯಾವುದು ನನಗೆ ವಿರೋಧವಾಗಿರುವುದೋ ಅದೆಲ್ಲ ಕೊನೆಗೆ ನನ್ನ ಹತ್ತಿರವೇ ಇರಬೇಕು. ನಾನು ಜಗನ್ಮಾತೆಯ ಯೋಧನಲ್ಲವೆ?

೧೨೬. ಹೌದು, ನನ್ನ ಜೀವನವೇ ಒಬ್ಬ ಮಹಾವ್ಯಕ್ತಿಯ ಸ್ಫೂರ್ತಿಯಿಂದ ರೂಪಗೊಂಡಿದೆ. ಆದರೇನಂತೆ? ಪ್ರಪಂಚಕ್ಕೆ ಸ್ಫೂರ್ತಿಯೆಲ್ಲ ಯಾವುದೋ ಒಂದು ವ್ಯಕ್ತಿಯಿಂದ ಮಾತ್ರ ಬರಲಿಲ್ಲ.

ಶ‍್ರೀರಾಮಕೃಷ್ಣ ಪರಮಹಂಸರು ಸ್ಫೂರ್ತಿಯನ್ನು ಪಡೆದವರೆಂದು ನಾನು ನಂಬುವೆನು. ಆದರೆ ನನ್ನಲ್ಲಿಯೂ ಅದೇ ಸ್ಫೂರ್ತಿ ಇದೆ, ನಿಮ್ಮಲ್ಲಿಯೂ ಅದೇ ಸ್ಫೂರ್ತಿಯಿದೆ. ಅದರಂತೆಯೇ ನಿಮ್ಮ ಶಿಷ್ಯರು, ಅನಂತರ ಅವರ ಶಿಷ್ಯರು ಕೂಡ. ಹೀಗೆಯೇ ಕೊನೆಯತನಕ ಹೋಗುವುದು.

ರಹಸ್ಯ ವಿವರಣೆಗಳ ಕಾಲ ಆಗಿಹೋಯಿತೆಂಬುದನ್ನು ನೀವು ಗಮನಿಸುವುದಿಲ್ಲವೆ? ಒಳ್ಳೆಯದಕ್ಕೂ ಕೆಟ್ಟದ್ದಕ್ಕೂ ಆ ಕಾಲ ಹೋಯಿತು. ಪುನಃ ಬರುವಂತಿಲ್ಲ. ಭವಿಷ್ಯದಲ್ಲಿ ಸತ್ಯ ಎಲ್ಲರಿಗೂ ಕಾಣುವಂತೆ ಇರಬೇಕು.

೧೨೭. ಉಪನಿಷತ್ತಿನ ಆದರ್ಶಕ್ಕೆ ತಕ್ಕಂತೆ ಇಡೀ ಪ್ರಪಂಚವನ್ನು ಮೇಲೆತ್ತ ಬಹುದು ಎಂದು ಬುದ್ಧ ಬಹಳ ದೊಡ್ಡ ತಪ್ಪನ್ನು ಮಾಡಿದ. ಕೊನೆಗೆ ಸ್ವಾರ್ಥತೆ ಇದನ್ನೆಲ್ಲಾ ಹಾಳುಮಾಡಿತು. ಕೃಷ್ಣ ಬುದ್ಧನಿಗಿಂತ ಬುದ್ದಿವಂತ. ಅವನಲ್ಲಿ ಬಹಳ ಮುಂದಾಲೋಚನೆ ಇತ್ತು. ಆದರೆ ಬುದ್ಧನಿಗೆ ರಾಜಿಯನ್ನು ಕಂಡರೆ ಆಗುತ್ತಿರಲಿಲ್ಲ. ಅವತಾರವ್ಯಕ್ತಿ ಕೂಡ ರಾಜಿಗೆ ಒಪ್ಪದೆ ಯಮಯಾತನೆಗೆ ಗುರಿಯಾಗಿ ಕಾಲವಾಗಿ ಹೋಗಿರುವುದನ್ನು ಇದುವರೆಗೆ ಪ್ರಪಂಚದಲ್ಲಿ ನೋಡಿರುವೆವು. ಬುದ್ಧನು ಒಂದು ಕ್ಷಣ ರಾಜಿಮಾಡಿಕೊಂಡಿದ್ದರೆ ಅವನ ಜೀವಿತಕಾಲದಲ್ಲೇ ಏಷ್ಯಾಖಂಡವೇ ಅವನನ್ನು ದೇವರೆಂದು ಆರಾಧಿಸುತ್ತಿತ್ತು. ಬುದ್ಧತ್ವ ಎನ್ನುವುದು ಒಂದು ಸ್ಥಿತಿಯೇ ಹೊರತು ಅದೊಂದು ವ್ಯಕ್ತಿಯಲ್ಲ ಎಂದು ಅವನು ಹೇಳುತ್ತಿದ್ದನು. ಪ್ರಪಂಚದಲ್ಲೆಲ್ಲಾ ಪ್ರಾಜ್ಞನಾದವನು ಅವನೊಬ್ಬನೇ ಎಂಬುದು ನಿಜ.

೧೨೮. ಪಾಶ್ಚಾತ್ಯ ದೇಶದಲ್ಲಿದ್ದಾಗ ವಿವೇಕಾನಂದರಿಗೆ ಅಲ್ಲಿಯ ಜನರು, ಬುದ್ಧನನ್ನು ಶಿಲುಬೆಗೆ ಏರಿಸಿದ್ದರೆ ಇನ್ನೂ ಅವನ ಜೀವನಕ್ಕೆ ಕಳೆ ಬರುತ್ತಿತ್ತು, ಜನರಿಗೆ ಅವನು ಇನ್ನೂ ಹೆಚ್ಚು ಆಕರ್ಷಣೀಯನಾಗುತ್ತಿದ್ದ ಎಂದರು. ಇದು ರೋಮನ್ನರ ಕ್ರೌರ್ಯ ಎಂದು ಭಂಗಿಸಿ ಅವರು ಹೀಗೆ ಹೇಳಿದರು: ಅತಿ ಹೀನವಾದ ಮೃಗೀಯ ಸ್ವಭಾವ ಕಾರ್ಯ–ರಭಸಕ್ಕೆ ಬೆರಗಾಗುವುದು. ಆದಕಾರಣವೆ ಪ್ರಪಂಚ ಯಾವಾಗಲೂ ಸಾಹಸ ಪ್ರಧಾನವಾಗಿರುವ ಕಾವ್ಯವನ್ನೇ ಮೆಚ್ಚುವುದು. ಅದೃಷ್ಟವಶಾತ್ ಭರತಖಂಡ \enginline{"headlong down the steep abyss"} ಎಂದು ಬರೆದ ಮಿಲ್ಟನ್ನಿನಂತೆ ಇರುವ ಕವಿಯನ್ನು ಸೃಷ್ಟಿಸಲಿಲ್ಲ. ಆದರೆ ಬ್ರೌನಿಂಗನ ಕೆಲವು ಪಂಕ್ತಿಗಳನ್ನು ಇವುಗಳಿಗೆ ಬದಲಾಗಿ ಹಾಕಬಹುದು. ರೋಮನ್ನರಿಗೆ ರುಚಿಸಿದ್ದು ಕ್ರಿಸ್ತನ ಜೀವನದಲ್ಲಿರುವ ಸಾಹಸದ ಘಟನೆಗಳು. ಕ್ರೈಸ್ತಧರ್ಮ ರೋಮನ್ ಚಕ್ರಾಧಿಪತ್ಯದ ಮೇಲೆ ತನ್ನ ಪ್ರಭಾವವನ್ನು ಬೀರಿದ್ದಕ್ಕೆ ಕಾರಣ ಅವನನ್ನು ಶಿಲುಬೆಗೆ ಏರಿಸಿದ ಘಟನೆ. ಹೌದು, ಹೌದು, ನೀವು ಪಾಶ್ಚಾತ್ಯರು ಘಟನೆಗಳ ರಭಸವನ್ನು ಮೆಚ್ಚುತ್ತೀರಿ. ನಿತ್ಯಜೀವನದ ಸಾಮಾನ್ಯ ಘಟನೆಗಳ ಮಹಿಮೆಯನ್ನು ನೀವು ಇನ್ನೂ ಅರಿಯಲಾರಿರಿ. ಕಾಲವಾದ ತನ್ನ ಮಗುವನ್ನು ಬುದ್ಧನ ಸಮೀಪಕ್ಕೆ ಹೊತ್ತು ತಂದ ಆ ಸಣ್ಣ ತಾಯಿಯ ಘಟನೆಗಿಂತ ಯಾವುದು ಹೆಚ್ಚು ರಸವತ್ತಾಗಿರಬಲ್ಲದು? ಇಲ್ಲವೇ ಬುದ್ಧನ ಜೀವನದಲ್ಲಿ ಬರುವ ಮೇಕೆಗಳ ಘಟನೆಯನ್ನು ತೆಗೆದುಕೊಳ್ಳಿ. ಮಹಾತ್ಯಾಗ ಭರತಖಂಡಕ್ಕೆ ಹೊಸದೇನೂ ಅಲ್ಲ. ಆದರೆ ನಿರ್ವಾಣಾನಂತರ ಆ ಕಾವ್ಯಮಯವಾದ ಜೀವನವನ್ನು ನೋಡಿ!

ಅಂದು ಮಳೆ ಬರುತ್ತಿದ್ದ ರಾತ್ರಿ, ಬುದ್ಧ ಒಂದು ಗುಡಿಸಲಿಗೆ ಬಂದು ಗೋಡೆಗೆ ಒರಗಿ ನಿಂತುಕೊಳ್ಳುವನು. ಮಳೆ ಜೋರಾಗಿ ಬೀಳುತ್ತಿದೆ, ಗಾಳಿ ಬಲವಾಗಿ ಬೀಸುತ್ತಿದೆ. ಒಳಗೆ ಇರುವ ದನಗಾಹಿಗೆ ಒಂದು ಕಿಟಕಿಯ ಮೂಲಕ ಬುದ್ಧ ಗೋಚರಿಸುವನು. "ಓಹೋ, ಕಾವಿ ಬಟ್ಟೆಯೆ! ಅಲ್ಲೇ ಇರು. ಅದು ಸಾಕು ನಿನಗೆ" ಎಂದು ಹೇಳಿ ಹೀಗೆ ಹಾಡುವನು:

"ನನ್ನ ದನಗಳೆಲ್ಲಾ ಮನೆಗೆ ಬಂದಾಯಿತು. ಒಲೆ ಚೆನ್ನಾಗಿ ಉರಿಯುತ್ತಿದೆ. ನನ್ನ ಹೆಂಡತಿ ಸುರಕ್ಷಿತಳಾಗಿರುವಳು. ಮಕ್ಕಳೆಲ್ಲಾ ಚೆನ್ನಾಗಿ ಮಲಗಿವೆ. ಓ ಮೇಘಮಾಲೆ, ಬೇಕಾದರೆ ನೀನು ಮಳೆಗರಿ" ಎನ್ನುವನು.

ಬುದ್ಧ ಹೊರಗಿನಿಂದ ಹೀಗೆ ಹೇಳುವನು: “ನನ್ನ ಮನಸ್ಸು ನಿಗ್ರಹಿಸಲ್ಪಟ್ಟಿದೆ, ಇಂದ್ರಿಯಗಳನ್ನು ಜಯಿಸಿ ಆಗಿದೆ. ಹೃದಯ ದೃಢವಾಗಿದೆ. ಓ ಮೇಘಮಾಲೆ, ಇಚ್ಛೆ ಬಂದರೆ ಮಳೆಗರಿ ಈ ರಾತ್ರಿ.”

ದನಗಾಹಿ: "ಪೈರನ್ನು ಕೊಯ್ದು ಆಗಿದೆ. ಎಲ್ಲಾ ಕಣಜದಲ್ಲಿ ಬಂದು ಬಿದ್ದಿದೆ. ಹೊಳೆ ತುಂಬಿ ಹರಿಯುತ್ತಿದೆ. ರಸ್ತೆ ಬಲವಾಗಿದೆ. ಓ ಮೇಘಮಾಲೆ, ಇಚ್ಛೆ ಬಂದರೆ ಮಳೆಗರಿ ಈ ರಾತ್ರಿ" ಹೀಗೆ ಮುಂದೆ ಸಾಗುತ್ತಿರುವುದು, ಕೊನೆಗೆ ದನಗಾಹಿ ಪಶ್ಚಾತ್ತಾಪವನ್ನೂ ವಿಸ್ಮಯವನ್ನೂ ಹೊಂದಿ ಬುದ್ಧನ ಶಿಷ್ಯನಾಗುವನು.

ಅಥವಾ ಆ ಕ್ಷೌರಿಕನ ಕಥೆಗಿಂತ ಮತ್ತಾವುದು ಸುಂದರವಾಗಿರುವುದು? "ಭಗವಾನ್ ಬುದ್ಧ ನನ್ನ ಮನೆಯ ಮುಂದೆ, ಅಂದರೆ ಕ್ಷೌರಿಕನ ಮನೆಯ ಮುಂದೆ ಹೋದನು. ನಾನು ಹಿಂದಿನಿಂದ ಓಡಿದೆ. ಅವನು ನನ್ನನ್ನು ನೋಡಿ ಕ್ಷೌರಿಕನಾದ ನನಗಾಗಿ ಕಾದನು. ಸ್ವಾಮಿ ನಾನು ನಿಮ್ಮೊಡನೆ ಮಾತಾಡಲೆ ಎಂದೆ, ಆತ ಆಗಲಿ ಎಂದ, ಕ್ಷೌರಿಕನಿಗೆ ಆಗಲಿ ಎಂದ."

"ನಿರ್ವಾಣ ನನ್ನಂತಹನಿಗೆ ಸಾಧ್ಯವೆ?" ಎಂದೆ.

ಅವನು, "ಹೌದು, ನಿನ್ನಂತಹ ಕ್ಷೌರಿಕನಿಗೂ ಸಾಧ್ಯ" ಎಂದನು.

"ನಾನು ನಿಮ್ಮನ್ನು ಅನುಸರಿಸಲೆ?" ಎಂದೆ.

"ಓ ಅಗತ್ಯವಾಗಿ, ಕ್ಷೌರಿಕನಾದ ನೀನೂ ಕೂಡ ನನ್ನನ್ನು ಅನುಸರಿಸಬಹುದು" ಎಂದನು.

೧೨೯. ಹಿಂದೂಧರ್ಮಕ್ಕೂ ಬೌದ್ಧ ಧರ್ಮಕ್ಕೂ ಇರುವ ವ್ಯತ್ಯಾಸವೇ ಇದು: ಬೌದ್ಧರು ಇದೆಲ್ಲ ಭ್ರಾಂತಿ ಎಂದು ಅರಿಯಿರಿ ಎಂದರು. ಈ ಭ್ರಾಂತಿಯ ಅಂತರಾಳದಲ್ಲಿ ಸತ್ಯವಿದೆ ಎನ್ನುವುದು ಹಿಂದೂಧರ್ಮ. ಆದರೆ ಇದನ್ನು ಸಾಧಿಸುವುದು ಹೇಗೆ? ಹಿಂದೂಧರ್ಮ ಒಂದು ಕಠಿಣವಾದ ನಿಯಮಾವಳಿಯನ್ನು ಸೂಚಿಸಿಲ್ಲ. ಬುದ್ಧನ ನಿಯಮಗಳನ್ನು ಸಂನ್ಯಾಸಿಗಳು ಮಾತ್ರ ಪಾಲಿಸಬಹುದು. ಎಲ್ಲಾ ಹಾದಿಗಳೂ ಸತ್ಯಕ್ಕೆ ಒಯ್ಯುವ ಹಾದಿಗಳೆ. ಒಬ್ಬ ಹುಟ್ಟು ಸಂನ್ಯಾಸಿಗೆ ಗೃಹಿಣಿಯೊಬ್ಬಳಿಂದ ಆಜ್ಞಾಪಿತನಾಗಿ ಕಟುಕನೊಬ್ಬನು ಶ್ರೇಷ್ಠ ತತ್ತ್ವವನ್ನು ವಿವರಿಸುವನು. ಆದಕಾರಣ ಬೌದ್ಧಧರ್ಮ ಸಂನ್ಯಾಸಿಗಳ ಧರ್ಮವಾಗುವುದು. ಆದರೆ ಹಿಂದೂಧರ್ಮ ಯತಿ–ಜೀವನವನ್ನು ಪೂಜ್ಯದೃಷ್ಟಿಯಿಂದ ನೋಡಿದರೂ ಸಾಧಾರಣ ಜನರಿಗೆ ಕರ್ತವ್ಯಪರಾಯಣತೆಯನ್ನು ಬೋಧಿಸುವುದು. ಇದರ ಮೂಲಕ ಅವರೂ ದೇವರನ್ನು ಸೇರುವರು ಎನ್ನುವುದು.

೧೩೦. ಸ್ತ್ರೀಯರಿಗೆ ಸಂನ್ಯಾಸದ ಆದರ್ಶದ ವಿಷಯವಾಗಿ ಹೀಗೆ ಹೇಳಿದರು: "ನಿಮ್ಮ ಭಾವನೆಗಳನ್ನು ವಿಶದಪಡಿಸಿ ನಿಯಮಗಳನ್ನು ಮಾಡಿ. ಸಾಧ್ಯವಿದ್ದರೆ ವಿಶ್ವಧರ್ಮದ ಭಾವನೆಯನ್ನು ಸ್ವಲ್ಪ ಕೊಡಿ. ಆದರೆ ಇದನ್ನು ಜ್ಞಾಪಕದಲ್ಲಿಡಿ, ಪ್ರಪಂಚದಲ್ಲೆಲ್ಲಾ ಯಾವ ಕಾಲದಲ್ಲೂ ಇಂತಹ ಆದರ್ಶವನ್ನು ಪಾಲಿಸುವುದಕ್ಕೆ ಆರು ಜನಕ್ಕಿಂತ ಜಾಸ್ತಿ ಸಿಕ್ಕಲಾರರು. ಪಂಗಡಗಳಿಗೆ ಮತ್ತು ಪಂಗಡಗಳನ್ನು ಮೀರಿ ಹೋಗುವುದಕ್ಕೆ ಅವಕಾಶವಿರಬೇಕು. ನಿಮಗೆ ಬೇಕಾದುದನ್ನೆಲ್ಲಾ ನೀವೇ ತಯಾರು ಮಾಡಿಕೊಳ್ಳಬೇಕಾಗಿದೆ. ನಿಯಮಗಳನ್ನು ಮಾಡಿ. ಆದರೆ ಅವನ್ನು ಜೋಡಿಸುವಾಗ, ಯಾವಾಗ ಒಬ್ಬ ಅದನ್ನು ಮೀರಿಹೋಗಲು ಸಾಧ್ಯವಾಗುವುದೋ ಆಗ ಸುಲಭವಾಗಿ ಮೀರಿಹೋಗುವಂತೆ ಅಣಿಮಾಡಿ. ಪೂರ್ಣ ಸ್ವಾತಂತ್ರ್ಯ ಮತ್ತು ಪೂರ್ಣ ವಿಧೇಯತೆ ಇವೆರಡರ ಸಾಮರಸ್ಯವೇ ನಮ್ಮ ವೈಶಿಷ್ಟ್ಯ. ಇದು ಮಠಗಳಲ್ಲಿ ಕೂಡ ಸಾಧ್ಯ."

೧೩೧. ಎರಡು ಭಿನ್ನ ಭಿನ್ನ ಜನಾಂಗಗಳು ಒಟ್ಟಿಗೆ ಕಲೆತಾಗ ಪ್ರತ್ಯೇಕವಾದ ಬಲವಾದ ಮೂರನೆಯದೊಂದು ಅಸ್ತಿತ್ವಕ್ಕೆ ಬರುವುದು. ಇದು ಇನ್ನೊಂದರೊಡನೆ ತಾನು ಬೆರೆಯದಿರುವಂತೆ ನೋಡಿಕೊಳ್ಳುವುದು. ವರ್ಣಗಳು ಪ್ರಾರಂಭವಾಗುವುದು ಇಲ್ಲೆ. ಮಿಶ್ರಣದಿಂದ ಶ್ರೇಷ್ಠವಾದುದೊಂದು ಜಾತಿ ಹುಟ್ಟುವುದು. ಒಂದು ಸಲ ಹಾಗೆ ಹುಟ್ಟಿದ ಮೇಲೆ ಆ ವೈಶಿಷ್ಟ್ಯವನ್ನು ರಕ್ಷಿಸಲು ಪ್ರಯತ್ನಿಸುತ್ತೇವೆ.

೧೩೨. ಭರತಖಂಡದಲ್ಲಿ ಬಾಲಿಕೆಯರ ವಿದ್ಯಾಭ್ಯಾಸದ ವಿಷಯದಲ್ಲಿ ಅವರು ಹೀಗೆ ಹೇಳಿದರು: ದೇವರನ್ನು ಪೂಜಿಸುವಾಗ ನೀವು ವಿಗ್ರಹಗಳನ್ನು ಅನಿವಾರ್ಯವಾಗಿ ಉಪಯೋಗಿಸಲೇಬೇಕಾಗಿದೆ. ಆದರೆ ನೀವು ಇದನ್ನು ಬೇಕಾದರೆ ಬದಲಿಸಬಹುದು: ಕಾಳಿಕಾದೇವಿ ಯಾವಾಗಲೂ ಒಂದೇ ರೀತಿ ಇರಬೇಕಾಗಿಲ್ಲ. ಬೇರೆ ಬೇರೆ ವಿಧಗಳಲ್ಲಿ ಜಗನ್ಮಾತೆಯನ್ನು ಕಲ್ಪಿಸಿಕೊಳ್ಳಲು ಬಾಲಿಕೆಯರಿಗೆ ಪ್ರೋತ್ಸಾಹಿಸಿ, ಸರಸ್ವತಿಯ ನೂರಾರು ಆಕಾರಗಳನ್ನು ಕಲ್ಪಿಸಿಕೊಳ್ಳಲಿ. ತಮ್ಮ ಭಾವನೆಗಳನ್ನು ತಾವೇ ಚಿತ್ರಿಸಲಿ, ಮಣ್ಣಿನಲ್ಲಿ ಮಾಡಲಿ, ಅದಕ್ಕೆ ಬಣ್ಣ ಕೊಡಲಿ.

ದೇವಸ್ಥಾನದಲ್ಲಿ ಪೀಠದ ಕೆಳಗೆ ಮೆಟ್ಟಲಿನ ಮೇಲೆ ಒಂದು ನೀರಿನಿಂದ ತುಂಬಿದ ಪಾತ್ರೆ ಇರಬೇಕು. ದೀಪಸ್ತಂಭದಲ್ಲಿ ದೀಪ ಯಾವಾಗಲೂ ಉರಿಯುತ್ತಿರಬೇಕು. ಇದರ ಜೊತೆಗೆ ಸದಾ ಪೂಜೆ ಆಗುತ್ತಿದ್ದರೆ ಹಿಂದೂ ಭಾವನೆಗೆ ಇದಕ್ಕಿಂತ ಪವಿತ್ರವಾದುದೇ ಇಲ್ಲ.

ಆದರೆ ಆ ಸಮಯದಲ್ಲಿ ಯಾವಾಗಲೂ ವೈದಿಕ ಆಚಾರಗಳೇ ಇರಬೇಕು. ಒಂದು ವೈದಿಕ ಪೀಠವಿರಬೇಕು. ಪೂಜೆಯ ಸಮಯದಲ್ಲಿ ಅಲ್ಲಿ ಬೆಂಕಿಯನ್ನು ಹಚ್ಚಬೇಕು. ಮಂಗಳಾರತಿಯ ಸಮಯದಲ್ಲಿ ಮಕ್ಕಳೆಲ್ಲ ಒಟ್ಟಿಗೆ ನೆರೆಯಬೇಕು. ಇಡೀ ಭರತಖಂಡದ ಜನರೆಲ್ಲ ಇದನ್ನು ಗೌರವಿಸಬೇಕು. – ನಿಮ್ಮ ಸುತ್ತಮುತ್ತಲಿರುವ ಪ್ರಾಣಿಗಳನ್ನೆಲ್ಲಾ ಒಟ್ಟಿಗೆ ಸೇರಿಸಿ. ಮೊದಲು ಗೋವನ್ನು ತನ್ನಿ. ಇದು ಒಳ್ಳೆಯದು. ನಾಯಿ, ಬೆಕ್ಕು, ಹಕ್ಕಿ ಮುಂತಾದುವೂ ಇರಲಿ. ಮಕ್ಕಳು ಅವುಗಳಿಗೆ ಆಹಾರವನ್ನು ಕೊಡಲಿ, ಅವುಗಳನ್ನು ನೋಡಿಕೊಳ್ಳಲಿ.

ಅನಂತರ ಜ್ಞಾನಯಜ್ಞವಿದೆ. ಇದೇ ಎಲ್ಲಕ್ಕಿಂತಲೂ ಸುಂದರವಾದುದು. ಭರತಖಂಡದಲ್ಲಿ ಪ್ರತಿಯೊಂದು ಪುಸ್ತಕವೂ ಪವಿತ್ರ ಎಂಬುದು ನಿಮಗೆ ಗೊತ್ತೆ? ವೇದ ಮಾತ್ರವಲ್ಲ, ಇಂಗ್ಲಿಷರ ಮತ್ತು ಮಹಮ್ಮದೀಯರ ಗ್ರಂಥಗಳೂ ಕೂಡ ಪವಿತ್ರ. ಎಲ್ಲವೂ ಪವಿತ್ರ.

ಪುರಾತನ ಕಲೆಗಳನ್ನು ಬಳಕೆಗೆ ತರಬೇಕು. ನಿಮ್ಮ ಹುಡುಗಿಯರಿಗೆ ಹಣ್ಣಿನಿಂದ ರಸಾಯನ ಮೊರಬ್ಬ ಮುಂತಾದುವನ್ನು ಮಾಡುವುದನ್ನು ಕಲಿಸಿ. ರುಚಿ ರುಚಿಯಾಗಿ ಅಡುಗೆ ಮಾಡುವುದು, ಹೊಲಿಯುವುದು ಮೊದಲಾದುವುಗಳನ್ನು ಕಲಿಸಿ. ಚಿತ್ರ ಬರೆಯುವುದು, ಫೋಟೋ ತೆಗೆಯುವುದು, ಕಸೂತಿ ನಕಾಸೆ ಮುಂತಾದುವನ್ನು ಕಲಿಸಿ. ಸಮಯ ಇದ್ದರೆ ತಮ್ಮ ಜೀವನೋಪಾಯಕ್ಕೆ ಸಂಪಾದಿಸಿಕೊಳ್ಳುವಂತಹ ಯಾವುದಾದರೂ ಕೆಲಸವನ್ನು ಅರಿತುಕೊಳ್ಳಲಿ.

ಎಂದಿಗೂ ಮಾನವಕೋಟಿಯನ್ನು ಮರೆಯಬೇಡಿ. ಭರತಖಂಡದಲ್ಲಿ ಮಾನವನನ್ನು ದೇವರೆಂದು ಪೂಜಿಸುವ ಅಭ್ಯಾಸ ಇದೆ. ಆದರೆ ಎಂದಿಗೂ ಇದಕ್ಕೆ ಹೆಚ್ಚು ಗಮನ ಕೊಟ್ಟಿಲ್ಲ. ನಿಮ್ಮ ವಿದ್ಯಾರ್ಥಿಗಳು ಇದನ್ನು ಅಭ್ಯಾಸ ಮಾಡಲಿ. ಈ ಕಾರ್ಯದಲ್ಲಿ ರಸವನ್ನು ತುಂಬಿಸಿ ಇದನ್ನು ಒಂದು ಕಲೆಯನ್ನಾಗಿ ಮಾಡಿ. ಸ್ನಾನವಾದ ಮೇಲೆ ಊಟ ಮಾಡುವುದಕ್ಕೆ ಮುಂಚೆ, ಭಿಕ್ಷುಕರ ಪಾದವನ್ನು ಪೂಜಿಸುವುದು ನಮ್ಮ ಭಾವಕ್ಕೆ ಮತ್ತು ಕ್ರಿಯೆಗೆ ಬಹಳ ಸಹಕಾರಿಯಾಗುವುದು. ಕೆಲವು ದಿನ ಮಕ್ಕಳನ್ನು ನಿಮ್ಮ ವಿದ್ಯಾರ್ಥಿಗಳನ್ನು ಪೂಜಿಸಿ. ಎಳೆಮಕ್ಕಳನ್ನು ತಂದು ಅವಕ್ಕೆ ಸ್ನಾನ ಮಾಡಿಸಿ ಹಾಲು ಕೊಡಬಹುದು. ಮಾತಾಜಿ ನನಗೆ ಏನು ಹೇಳಿದರು ಗೊತ್ತೆ? "ಸ್ವಾಮೀಜಿ, ನನಗೆ ಯಾವ ಸಹಾಯವೂ ಇಲ್ಲ. ಆದರೆ ನಾನು ಪೂಜಿಸುವ ಈ ಪುಣ್ಯಾತ್ಮರು ನನ್ನನ್ನು ಮುಕ್ತಿಗೆ ಒಯ್ಯುವರು." ಅವರು "ಕುಮಾರಿಗಳಲ್ಲಿ ಉಮೆಯನ್ನು ಪೂಜಿಸುತ್ತಿರುವೆ" ಎಂದು ಭಾವಿಸುವರು. ಶಾಲೆಯನ್ನು ಪ್ರಾರಂಭಿಸುವುದಕ್ಕೆ ಇದೊಂದು ಅದ್ಭುತವಾದ ಭಾವನೆ.

೧೩೩. ಪ್ರೇಮ ಯಾವಾಗಲೂ ಆನಂದದ ಒಂದು ಆವಿರ್ಭಾವ. ಅಲ್ಲಿ ಸ್ವಲ್ಪ ನೋವು ಕಂಡರೆ ಅದು ಸ್ವಾರ್ಥದಿಂದ, ದೇಹದ ಮೇಲಿನ ವ್ಯಾಮೋಹದಿಂದ ಬಂದುದು.

೧೩೪. ಪಾಶ್ಚಾತ್ಯರಲ್ಲಿ ಒಬ್ಬರು ಇನ್ನೊಬ್ಬರನ್ನು ಮದುವೆಯಾಗುವ ಪದ್ಧತಿ ಎಲ್ಲಾ ಕಾನೂನುಗಳನ್ನು ಮೀರಿಹೋಗಿದೆ. ಆದರೆ ಭರತಖಂಡದಲ್ಲಿ ಸಮಾಜವೇ ಗಂಡ ಹೆಂಡತಿಯರಿಗೆ ಒಂದು ಸಂಬಂಧವನ್ನು ಕಲ್ಪಿಸುವುದು – ಇದು ಅನಂತಕಾಲದವರೆಗಿನ ಬಂಧನ. ಪ್ರತಿಯೊಂದು ಜನ್ಮದಲ್ಲಿಯೂ, ತಮಗಿಷ್ಟವಿರಲಿ ಬಿಡಲಿ, ಆ ಇಬ್ಬರು ಪರಸ್ಪರ ಮದುವೆಯಾಗಲೇಬೇಕು. ಒಬ್ಬರು ಇನ್ನೊಬ್ಬರ ಪುಣ್ಯಕ್ಕೆ ಭಾಗಿಗಳು. ಈ ಜನ್ಮದಲ್ಲಿ ಒಬ್ಬರು ಬಹಳ ಹಿಂದೆ ಬಿದ್ದರೆ ಮತ್ತೊಬ್ಬರು ಅವರು ಬರುವವರೆಗೆ ಕಾಯಬೇಕು!

೧೩೫. ಪ್ರಜ್ಞೆ ಎನ್ನುವುದು ಅಪ್ರಜ್ಞೆ \enginline{(sub–conscious)} ಮತ್ತು ಅತಿ ಪ್ರಜ್ಞೆ \enginline{(super–conscious)} ಎಂಬ ಮಹಾಸಾಗರದ ಮಧ್ಯೆ ಇರುವ ಒಂದು ತೆಳುವಾದ ತೆರೆಯಂತೆ.

೧೩೬. ಪಾಶ್ಚಾತ್ಯರು ಪ್ರಜ್ಞೆಯ ವಿಷಯವಾಗಿ ಅಷ್ಟೊಂದು ಮಾತಾಡುತ್ತಿರುವುದನ್ನು ಕೇಳಿದಾಗ ನನಗೆ ಅದನ್ನು ನೆಚ್ಚಲೇ ಆಗಲಿಲ್ಲ. ಪ್ರಜ್ಞೆಯೆಂದರೆ ಏನು? ಅಪ್ರಜ್ಞೆಯ ಆಳದೊಂದಿಗೆ ಮತ್ತು ಅತಿಪ್ರಜ್ಞೆಯ ಎತ್ತರದೊಂದಿಗೆ ಅದನ್ನು ಹೋಲಿಸುವುದಕ್ಕೆ ಆಗುವುದಿಲ್ಲ. ಈ ವಿಷಯದಲ್ಲಿ ನಾನು ಎಂದಿಗೂ ತಪ್ಪುವುದಿಲ್ಲ. ಶ‍್ರೀರಾಮಕೃಷ್ಣ ಪರಮಹಂಸರು ಹತ್ತು ನಿಮಿಷಗಳಲ್ಲಿ ಒಬ್ಬನ ಅಪ್ರಜ್ಞೆಯ ಆಳಕ್ಕೆ ಹೋಗಿ, ಅವನ ಹಿಂದಿನ ಜೀವನವನ್ನೆಲ್ಲಾ ತಿಳಿದು, ಅದರಿಂದ ಅವನಲ್ಲಿರುವ ಶಕ್ತಿ ಏನು ಮತ್ತು ಅವನ ಭವಿಷ್ಯವೇನು ಎಂದು ನಿರ್ಧರಿಸುವುದನ್ನು ನಾನು ನೋಡಿಲ್ಲವೆ?

೧೩೭. ಈ ದರ್ಶನಗಳೆಲ್ಲಾ ಗೌಣ. ಇದು ನಿಜವಾದ ಯೋಗವಲ್ಲ. ನಾವು ಹೇಳುವುದನ್ನು ಪರೋಕ್ಷವಾಗಿ ಸಮರ್ಥಿಸುವುದಕ್ಕೆ ಇವುಗಳಿಂದ ಸ್ವಲ್ಪ ಸಹಾಯ ಆಗಬಹುದು. ಒಂದು ಕ್ಷಣಿಕ ದರ್ಶನ ಕೂಡ ಈ ಸ್ಥೂಲ ಜಗತ್ತಿನ ಹಿಂದೆ ಏನೋ ಒಂದು ಇದೆ ಎಂದು ನಂಬುವುದಕ್ಕೆ ಸಹಕಾರಿಯಾಗುವುದು. ಆದರೂ ಯಾರೂ ಇವುಗಳನ್ನೇ ಕುರಿತು ಆಲೋಚಿಸುತ್ತಿರುವರೋ ಅವರು ಬಹಳ ಅಪಾಯಕ್ಕೆ ತುತ್ತಾಗುವರು.

ಈ ಮನಃಶಕ್ತಿಗಳೆಲ್ಲಾ ಸೀಮಾ ಪ್ರಶ್ನೆಗಳು \enginline{(frontier questions).} ಇವುಗಳ ಮೂಲಕ ಬಂದ ತಿಳುವಳಿಕೆ ನಿಸ್ಸಂಶಯವಾದುದೂ ಅಲ್ಲ, ಸ್ಥಿರವೂ ಅಲ್ಲ. ಇವೆಲ್ಲಾ ಸೀಮಾ ಪ್ರಶ್ನೆಗಳೆಂದು ನಾನು ಹೇಳಲಿಲ್ಲವೆ? ಎಲೆಯ ರೇಖೆ ಯಾವಾಗಲೂ ಬದಲಾಗುತ್ತಿರುತ್ತದೆ.

೧೩೮. ಅದ್ವೈತಿಗಳು; ಆತ್ಮ ಬರುವುದೂ ಇಲ್ಲ, ಹೋಗುವುದೂ ಇಲ್ಲ; ಹಲವು ಪ್ರಪಂಚಗಳೆಲ್ಲ ಪ್ರಾಣದ ಮತ್ತು ಆಕಾಶದ ಕ್ರಿಯೆ ಮತ್ತು ಪ್ರತಿಕ್ರಿಯೆಗಳಿಂದ ಉಂಟಾದ ಹಲವು ವಸ್ತುಗಳು ಎನ್ನುವರು. ಅತಿ ಕೆಳಗಿರುವುದೇ, ಘನೀಭೂತವಾಗಿರುವುದೇ ಸೂರ್ಯಲೋಕ. ಇದು ನಮ್ಮ ಕಣ್ಣಿಗೆ ಕಾಣುವುದು. ಇಲ್ಲಿ ಪ್ರಾಣ ಭೌತಿಕಶಕ್ತಿಯಂತೆಯೂ, ಆಕಾಶ ಇಂದ್ರಿಯಗ್ರಾಹ್ಯ ದ್ರವ್ಯದಂತೆಯೂ ಕಾಣುವುದು. ಅನಂತರದ್ದೆ ಚಂದ್ರಲೋಕ. ಇದು ಸೂರ್ಯಲೋಕವನ್ನು ಆವರಿಸಿಕೊಂಡಿದೆ. ಇದು ಚಂದ್ರ ಅಲ್ಲವೇ ಅಲ್ಲ; ಇದು ದೇವತೆಗಳ ಲೋಕ. ಇಲ್ಲಿ ಪ್ರಾಣ ಮನಃಶಕ್ತಿಯಂತೆಯೂ, ಆಕಾಶ ತನ್ಮಾತ್ರದಂತೆಯೂ ಇದೆ. ಇದರ ಆಚೆ ವಿದ್ಯುತ್‌ ಲೋಕವಿದೆ. ಅಂದರೆ ಈ ಸ್ಥಿತಿಯಲ್ಲಿ ಆಕಾಶದಿಂದ ಇದನ್ನು ಬೇರ್ಪಡಿಸಲಾಗುವುದಿಲ್ಲ. ವಿದ್ಯುತ್ ಅನ್ನು ಅದು ಶಕ್ತಿಯೇ ಅಥವಾ ದ್ರವ್ಯವೇ ಎಂದು ಹೇಳುವುದು ಬಹಳ ಕಷ್ಟ. ಅನಂತರವೇ ಬ್ರಹ್ಮಲೋಕ. ಅಲ್ಲಿ ಆಕಾಶವೂ ಇಲ್ಲ ಪ್ರಾಣವೂ ಇಲ್ಲ. ಎರಡೂ ಮನಸ್ಸೆಂಬ ಮೂಲ ಪ್ರಕೃತಿಯಲ್ಲಿ ಐಕ್ಯವಾಗುವುವು. ಇಲ್ಲಿ ಪ್ರಾಣ ಮತ್ತು ಆಕಾಶ ಇಲ್ಲದೆ ಇರುವುದರಿಂದ ಜೀವ ಇಡೀ ಬ್ರಹ್ಮಾಂಡವನ್ನು ಸಮಷ್ಟಿಯಂತೆ ಅಥವಾ ಮನಸ್ಸಿನಂತೆ ಭಾವಿಸುವುದು. ಜೀವ ಇಲ್ಲಿ ಪುರುಷನಂತೆ ಇರುವುದು, ಆದರೂ ಇದು ಅಖಂಡವಲ್ಲ; ಏಕೆಂದರೆ ಇಲ್ಲಿ ಇನ್ನೂ ವೈವಿಧ್ಯ ಇದೆ. ಇಲ್ಲಿಂದ ಜೀವ ತನ್ನ ಗುರಿಯಾದ ಏಕತೆಯನ್ನು ನೋಡುವುದು. ಇವುಗಳೆಲ್ಲ ಕ್ರಮವಾಗಿ ಜೀವಾತ್ಮನ ಮುಂದೆ ಏಳುವ ದೃಶ್ಯಗಳು. ಆದರೆ ಜೀವಾತ್ಮನು ಸ್ವತಃ ಬರುವುದೂ ಇಲ್ಲ ಹೋಗುವುದೂ ಇಲ್ಲ. ಇದರಂತೆಯೇ ಈಗ ಕಾಣುವ ದೃಶ್ಯ ಕೂಡ ಸೃಷ್ಟಿಸಲ್ಪಟ್ಟಿದೆ. ಸೃಷ್ಟಿ ಪ್ರಳಯಗಳೆರಡೂ ಇದೇ ಕ್ರಮದಲ್ಲಿ ಆಗಬೇಕು. ಸೃಷ್ಟಿ ಎಂದರೆ ಒಳಗಿನಿಂದ ಬರುವುದು, ಪ್ರಳಯವೆಂದರೆ ಒಳಗೆ ಹೋಗುವುದು ಎಂದು ಅರ್ಥ.

ಪ್ರತಿಯೊಬ್ಬನು ತನ್ನ ಜಗತ್ತನ್ನು ಮಾತ್ರ ನೋಡಲು ಸಾಧ್ಯ. ಆ ಜಗತ್ತು ಅವನು ಬದ್ಧನಾಗಿರುವುದರಿಂದ ಸೃಷ್ಟಿಯಾಗಿದೆ. ಅವನು ಮುಕ್ತನಾದ ಮೇಲೆ ಹೊರಟು ಹೋಗುವುದು. ಆದರೆ ಇತರ ಬದ್ಧರಿಗೆ ಅದು ಇದ್ದೇ ಇರುವುದು. ನಾಮರೂಪಗಳೇ ಪ್ರಪಂಚ. ಸಮುದ್ರದಲ್ಲಿ ಇರುವ ಒಂದು ಅಲೆ ಎಲ್ಲಿಯವರೆಗೆ ನಾಮರೂಪಗಳಿಗೆ ಒಳಪಟ್ಟಿರುವುದೋ ಅಲ್ಲಿಯವರೆಗೆ ಮಾತ್ರ ಅದೊಂದು ಅಲೆ. ಅಲೆ ಇಳಿದೊಡನೆಯೇ ಅದು ಸಾಗರವಾಗುವುದು. ಆಗ ನಾಮರೂಪಗಳು ಮಾಯವಾಗಿ ಹೋಗುವುವು. ನೀರೇ ನಾಮರೂಪಗಳನ್ನು ರಚಿಸಿದ್ದು. ಆದರೂ ಈ ನಾಮರೂಪಗಳೇ ಅಲೆಯಲ್ಲ. ಅಲೆಯು ನೀರಿನೊಡನೆ ಒಂದಾದ ಕೂಡಲೇ ನಾಮ ರೂಪಗಳು ಮಾಯವಾಗುವುವು. ಆದರೆ ಬೇರೆ ಅಲೆಗಳ ನಾಮರೂಪಗಳು ಹಾಗೆಯೇ ಇರುತ್ತವೆ. ಈ ನಾಮರೂಪಗಳೇ ಮಾಯೆ, ನೀರೇ ಬ್ರಹ್ಮ. ಅಲೆ ಯಾವಾಗಲೂ ನೀರಾಗಿತ್ತು. ಆದರೂ ಅಲೆಯಾದುದರಿಂದ ಅದಕ್ಕೆ ನಾಮ ರೂಪಗಳು ಇದ್ದವು. ಅಲೆ ಇಲ್ಲದೆ ಇದ್ದರೆ ಈ ನಾಮರೂಪಗಳು ಒಂದು ಕ್ಷಣವೂ ಇರಲಾರವು. ಆದರೆ ಅಲೆಯಲ್ಲಿರುವ ನೀರು ಮಾತ್ರ ನಾಮರೂಪಗಳಿಲ್ಲದೇ ಇರಬಲ್ಲದು. ನಾಮರೂಪಗಳು ಒಂದನ್ನು ಒಂದು ಬಿಟ್ಟು ಇಲ್ಲದೇ ಇರುವುದರಿಂದ ಅವು ಸತ್ಯವಾಗಿರಲಾರವು. ಆದರೂ ಅವು ಶೂನ್ಯವಲ್ಲ. ಅವೇ ಮಾಯೆ.

೧೩೯. ನಾನು ಬುದ್ಧನ ಸೇವಕರ ಸೇವಕ. ಅವನಂತೆ ಯಾರು ಹಿಂದೆ ಇದ್ದರು? ಭಗವಾನ್ ಬುದ್ಧ ತನ್ನ ಸ್ವಂತಕ್ಕಾಗಿ ಏನನ್ನೂ ಮಾಡಿದವನಲ್ಲ. ಇಡೀಯ ಪ್ರಪಂಚವನ್ನು ಬಾಚಿ ತಬ್ಬುವಂತಹ ಅನುಕಂಪ, ಎಂತಹ ಕರುಣೆ ಆ ಯತಿರಾಜನಿಗೆ! ಒಂದು ಕುರಿಮರಿಗಾಗಿ ತನ್ನ ಪ್ರಾಣವನ್ನೇ ಕೊಡಲು ಸಿದ್ಧನಾಗಿದ್ದ. ಒಂದು ಹಸಿದ ಹುಲಿಗಾಗಿ ತನ್ನನ್ನೇ ಅರ್ಪಿಸಿಕೊಳ್ಳುವುದರಲ್ಲಿದ್ದ. ಚಂಡಾಲನ ಆತಿಥ್ಯವನ್ನು ಒಪ್ಪಿಕೊಂಡು ಅವನನ್ನು ಹರಸಿದನು. ನಾನು ಸಣ್ಣ ಹುಡುಗನಾಗಿದ್ದಾಗ ಅವನು ನನ್ನ ಕೋಣೆಯೊಳಗೆ ಬಂದನು. ಅವನಿಗೆ ನಾನು ನಮಸ್ಕರಿಸಿದೆ! ಏಕೆಂದರೆ ಅವನು ಸ್ವಯಂ ಭಗವಂತನೇ ಎಂಬುದು ನನಗೆ ಗೊತ್ತಿತ್ತು!

೧೪೦. ಶ‍್ರೀಶುಕನು ಆದರ್ಶ ಪರಮಹಂಸ. ಮಾನವರಲ್ಲಿ ಅವನೊಬ್ಬನಿಗೆ ಮಾತ್ರ ಅಖಂಡ ಬ್ರಹ್ಮಾನಂದದ ಕಡಲಿನಲ್ಲಿ ಒಂದು ಬೊಗಸೆ ಸಚ್ಚಿದಾನಂದವನ್ನು ಕುಡಿಯಲು ಸಾಧ್ಯವಾಯಿತು. ಹಲವು ಮಹಾತ್ಮರು ಆ ಸಾಗರದ ಅಲೆಯ ಗರ್ಜನೆಯನ್ನು ಕೇಳಿಯೇ ಕಾಲವಾಗುವರು. ಕೆಲವರು ಅದನ್ನು ನೋಡುವರು, ಅದಕ್ಕೂ ಕಡಮೆ ಜನರು ಅದರ ರುಚಿ ನೋಡಿರುವರು. ಆದರೆ ಶುಕ ಅದನ್ನು ಪಾನಮಾಡಿದ.

೧೪೧. ತ್ಯಾಗವಿಲ್ಲದ ಭಕ್ತಿ ಎಂತಹುದು? ಅದು ವಿನಾಶಕಾರಿ.

೧೪೨. ನಾವು ಸುಖವನ್ನೂ ಕೋರುವುದಿಲ್ಲ; ದುಃಖವನ್ನೂ ಕೋರುವುದಿಲ್ಲ. ಅವುಗಳ ಮೂಲಕ ಅವನ್ನು ಮೀರಿರುವುದನ್ನು ಪಡೆಯಲು ಯತ್ನಿಸುತ್ತೇವೆ.

೧೪೩. ಶಂಕರಾಚಾರ್ಯರಿಗೆ ವೇದೋಚ್ಚಾರಣೆಯ ಧ್ವನಿ ಗೊತ್ತಿತ್ತು. ನನಗೆ ಬಾಲ್ಯದಲ್ಲಿ ಕಂಡಂತೆ ಅವರಿಗೂ ಒಂದು ದೃಶ್ಯ ಕಂಡಿರಬೇಕು. ಅದರಿಂದ ಅವರು ಪೂರ್ವಕಾಲದ ಸಂಗೀತವನ್ನು ಕಲಿತರು ಎಂದು ಅನೇಕ ವೇಳೆ ನಾನು ಯೋಚಿಸುತ್ತೇನೆ. ಏನಾದರೂ ಆಗಲಿ, ಅವರ ಇಡೀ ಜೀವನದಲ್ಲಿಯೇ ವೇದ–ಉಪನಿಷತ್ತುಗಳ ಸೌಂದರ್ಯ ಓತಪ್ರೋತವಾಗಿದೆ.

೧೪೪. ಒಂದು ದೃಷ್ಟಿಯಲ್ಲಿ ತಾಯಿಯ ಪ್ರೇಮ ಮೇಲಾದರೂ, ಜನರು ಸ್ತ್ರೀ–ಪುರುಷರ ಪ್ರೇಮವನ್ನು ಭಗವಂತನ ಪ್ರೇಮಕ್ಕೆ ಆದರ್ಶವಾಗಿ ತೆಗೆದುಕೊಳ್ಳುವರು. ಮತ್ತಾವ ಸಂಬಂಧದಲ್ಲಿಯೂ ಇಷ್ಟು ಅದ್ಭುತವಾದ ಉದ್ವೇಗಪರವಶತೆ ಇರುವುದಿಲ್ಲ. ಪ್ರೀತಿಸುವವನು ತಾನು ಕಲ್ಪಿಸಿಕೊಂಡಂತೆ ಆಗುತ್ತಾನೆ. ಈ ಪ್ರೇಮ ಪ್ರೀತಿಸುವ ವಸ್ತುವನ್ನು ಮಾರ್ಪಡಿಸುವುದು.

೧೪೫. ಜನಕನಂತೆ ಆಗುವುದು ಅಷ್ಟು ಸುಲಭವೆ? ಸಿಂಹಾಸನದ ಮೇಲೆ ಸ್ವಲ್ಪವೂ ಆಸಕ್ತಿಯಿಲ್ಲದೆ ಕುಳಿತುಕೊಳ್ಳುವುದು ಅಷ್ಟು ಸುಲಭವೆ? ಐಶ್ವರ್ಯ ಕೀರ್ತಿ ಹೆಂಡತಿ ಮಕ್ಕಳು ಯಾವುದನ್ನೂ ಗಣನೆಗೆ ತಾರದೆ ಇರುವುದು ಸಾಧ್ಯವೆ? ಪಾಶ್ಚಾತ್ಯರಲ್ಲಿ ಅನೇಕ ಜನರು ತಾವು ಜನಕನಂತೆ ಇರುವೆವು ಎಂದು ಹೇಳುವರು. ಆದರೆ "ಅವನಂತಹ ಮಹಾಪುರುಷರು ಇಂಡಿಯಾದೇಶದಲ್ಲಿ ಇಲ್ಲ" ಎಂದಷ್ಟೇ ಹೇಳಬೇಕಾಯಿತು.

೧೪೬. ಮುಂದೆ ಬರುವ ಭಾವಗಳನ್ನು ನೀವು ಮನನ ಮಾಡುವುದಕ್ಕೆ ಮತ್ತು ನಿಮ್ಮ ಮಕ್ಕಳಿಗೆ ಕಲಿಸುವುದಕ್ಕೆ ಎಂದಿಗೂ ಮರೆಯಬೇಡಿ: ಒಂದು ಮಿಂಚು ಹುಳುವಿಗೂ, ಚಂಡಕಾಂತಿಯ ಸೂರ್ಯನಿಗೂ ಯಾವ ವ್ಯತ್ಯಾಸವಿದೆಯೋ, ಒಂದು ಅನಂತಸಾಗರಕ್ಕೂ ಒಂದು ಸಣ್ಣ ಕೊಳಕ್ಕೂ ಯಾವ ವ್ಯತ್ಯಾಸವಿದೆಯೊ, ಒಂದು ಸಾಸಿವೆಕಾಳಿಗೂ ಮೇರು ಪರ್ವತಕ್ಕೂ ಯಾವ ವ್ಯತ್ಯಾಸವಿದೆಯೋ ಅದೇ ವ್ಯತ್ಯಾಸ ಸಂನ್ಯಾಸಿಗೂ ಗೃಹಸ್ಥನಿಗೂ ಇರುವುದು.

ಪ್ರಪಂಚದಲ್ಲಿ ಎಲ್ಲವೂ ಭಯದಿಂದ ತುಂಬಿದೆ. ವೈರಾಗ್ಯದಲ್ಲಿ ಮಾತ್ರ ಭಯವಿಲ್ಲ.

ಕಪಟ ಸಂನ್ಯಾಸಿಗಳು ಮತ್ತು ಪತಿತರಾದವರೂ ಕೂಡ ಧನ್ಯರು. ಏಕೆಂದರೆ ಅವರೂ ಆ ಪರಮ ಆದರ್ಶಕ್ಕೆ ಸಾಕ್ಷಿಯಾಗಿರುವರು. ಪರೋಕ್ಷವಾಗಿ ಇತರರ ಧ್ಯೇಯಸಾಧನೆಗೆ ಅವರು ಸಹಾಯ ಮಾಡುವರು.

ನಾವು ಎಂದಿಗೂ ನಮ್ಮ ಆದರ್ಶವನ್ನು ಮರೆಯದಿರೋಣ!

೧೪೭. ಹರಿಯುತ್ತಿರುವ ನದಿ ಪರಿಶುದ್ಧ, ಸಂಚರಿಸುತ್ತಿರುವ ಸಾಧು ಪರಿಶುದ್ದ.

೧೪೮, ಹೊನ್ನನ್ನು ಆಶಿಸುವ ಸಂನ್ಯಾಸಿ ಆತ್ಮಹತ್ಯೆ ಮಾಡಿಕೊಂಡಂತೆ.

೧೪೯. ಮಹಮ್ಮದ್ ಅಥವಾ ಬುದ್ಧನು ಒಳ್ಳೆಯವನಾಗಿದ್ದರೆ ಅದರಿಂದ ನನಗೆ ಆದ ಪ್ರಯೋಜನವೇನು? ಅದು ನನ್ನ ಪಾಪಪುಣ್ಯಗಳನ್ನು ಬದಲಾಯಿಸಬಲ್ಲದೆ? ನಾವು ನಮಗಾಗಿ ಒಳ್ಳೆಯವರಾಗೋಣ. ನಾವು ನಮ್ಮ ಜವಾಬ್ದಾರಿಯ ಮೇಲೆ ಒಳ್ಳೆಯವರಾಗೋಣ.

೧೫೦. ಈ ದೇಶದಲ್ಲಿರುವ ನೀವು ನಿಮ್ಮ ವ್ಯಕ್ತಿತ್ವವನ್ನು ಕಳೆದುಕೊಳ್ಳುವುದಕ್ಕೆ ತುಂಬಾ ಅಂಜುತ್ತಿರುವಿರಿ. ಅಷ್ಟೇಕೆ ಅಂಜುವಿರಿ? ನೀವು ಇನ್ನೂ ಒಂದು ವ್ಯಕ್ತಿ ಆಗಿಲ್ಲ. ನೀವು ನಿಮ್ಮ ಸ್ವಭಾವವನ್ನೆಲ್ಲ ಅರಿತಾಗ ನೀವು ನಿಮ್ಮ ನಿಜವಾದ ವ್ಯಕ್ತಿತ್ವವನ್ನು ಪಡೆಯುತ್ತೀರಿ. ಅದಕ್ಕೆ ಮುಂಚೆ ಅಲ್ಲ. ನಾನು ಈ ದೇಶದಲ್ಲಿ ಮತ್ತೊಂದನ್ನು ಪದೇ ಪದೇ ಕೇಳುತ್ತಿರುವೆನು. ಅದೇ ನಾವು ಪ್ರಕೃತಿಯೊಡನೆ ಸೌಹಾರ್ದದಿಂದ ಬಾಳುವುದು ಎಂಬುದು. ಈ ಪ್ರಪಂಚದಲ್ಲಿ ಆದ ಪ್ರಗತಿಯೆಲ್ಲ ಪ್ರಕೃತಿಯನ್ನು ಗೆದ್ದುದರಿಂದ ಆಗಿದೆ ಎಂಬುದು ನಿಮಗೆ ಗೊತ್ತಿಲ್ಲವೆ? ನಾವು ಏನಾದರೂ ಸ್ವಲ್ಪ ಮುಂದುವರಿಯಬೇಕಾದರೆ ಹೆಜ್ಜೆ ಹೆಜ್ಜೆಗೆ ಪ್ರಕೃತಿಯನ್ನು ವಿರೋಧಿಸಬೇಕಾಗಿದೆ.

೧೫೧. ಇಂಡಿಯಾದೇಶದಲ್ಲಿ ಜನಸಾಮಾನ್ಯರಿಗೆ ಅದ್ವೈತವನ್ನು ಬೋಧನೆ ಮಾಡಬೇಡಿ ಎಂದು ನನಗೆ ಹೇಳುತ್ತಾರೆ. ಆದರೆ ಒಂದು ಮಗುವಿಗೆ ಕೂಡ ಅದನ್ನು ಅರ್ಥವಾಗುವಂತೆ ನಾನು ಹೇಳಬಲ್ಲೆ ಎಂದು ಹೇಳುತ್ತೇನೆ. ಉನ್ನತ ಆಧ್ಯಾತ್ಮಿಕ ಸತ್ಯಗಳನ್ನು ಎಲ್ಲರೂ ಗ್ರಹಿಸಬಹುದು.

೧೫೨. ನೀವು ಎಷ್ಟು ಕಡಮೆ ಓದಿದರೆ ಅಷ್ಟು ಒಳ್ಳೆಯದು. ಗೀತೆ ಮತ್ತು ವೇದಾಂತಕ್ಕೆ ಸಂಬಂಧಿಸಿದ ಕೆಲವು ಸದ್ ಗ್ರಂಥಗಳನ್ನು ಓದಿ. ನಿಮಗೆ ಬೇಕಾಗಿರುವುದು ಇಷ್ಟೇ. ಆಧುನಿಕ ವಿದ್ಯಾಭ್ಯಾಸವೆಲ್ಲ ಕುಲಗೆಟ್ಟು ಹೋಗಿದೆ. ಆಲೋಚಿಸುವುದಕ್ಕೆ ಮುಂಚೆ ಹಲವಾರು ವಿಷಯಗಳನ್ನು ಸಂಗ್ರಹಿಸುತ್ತಾರೆ. ಮನಸ್ಸಿನ ನಿಗ್ರಹವನ್ನು ಮೊದಲು ಬೋಧಿಸಬೇಕು. ನಾನೇನಾದರೂ ವಿದ್ಯಾಭ್ಯಾಸವನ್ನು ಪುನಃ ಪ್ರಾರಂಭಮಾಡಬೇಕಾಗಿ ಬಂದರೆ, ಅದರಲ್ಲಿ ನನಗೆ ಏನಾದರೂ ಸ್ವಲ್ಪ ಸ್ವಾತಂತ್ರ್ಯವಿದ್ದರೆ, ನಾನು ಮೊದಲು ಮನಸ್ಸನ್ನು ನಿಗ್ರಹಿಸುವುದನ್ನು ಕಲಿತು ಅನಂತರ ನನಗೆ ಬೇಕಾದರೆ ವಿಷಯವನ್ನು ಸಂಗ್ರಹಿಸುವೆನು. ಜನರಿಗೆ ವಿಷಯಗಳನ್ನು ಕಲಿತುಕೊಳ್ಳುವುದಕ್ಕೆ ಬಹಳಕಾಲ ಹಿಡಿಯುವುದು. ಏಕೆಂದರೆ ತಮ್ಮ ಇಚ್ಛಾ ಪ್ರಕಾರ ಅವರು ತಮ್ಮ ಮನಸ್ಸನ್ನು ನಿಗ್ರಹಿಸಲಾರರು.

೧೫೩. ಒಂದು ಕೆಟ್ಟ ಕಾಲ ಬಂದರೆ ಅದರಿಂದ ಏನು? ಗಡಿಯಾರದ ಪೆಂಡುಲಂ ಒಂದು ಕಡೆಯಿಂದ ಮತ್ತೊಂದು ಕಡೆಗೆ ಬರಲೇಬೇಕು. ಆದರೆ ಅದೂ ಏನೂ ಒಳ್ಳೆಯದಲ್ಲ. ಅದನ್ನು ನಿಲ್ಲುವಂತೆ ಮಾಡುವುದೇ ಬೇಕಾಗಿರುವುದು.

\newpage

\chapter[ಬೌದ್ಧ ಭಾರತ]{ಬೌದ್ಧ ಭಾರತ\protect\footnote{\engfoot{C.W, Vol. III, P.511}}}

\begin{center}
(೧೯೦೦ರ ಫೆಬ್ರವರಿ ೨ರಂದು ಸಂಜೆ ಷೇಕ್ಸ್ ಪಿಯರ್ ಕ್ಲಬ್, ಪ್ಯಾಸಡೇನಾ, ಕ್ಯಾಲಿಫೋರ್ನಿಯಾದಲ್ಲಿ ಕೊಟ್ಟ ಉಪನ್ಯಾಸ.)
\end{center}

ಈ ಸಂಜೆ ನಮ್ಮ ಮುಂದಿರುವ ವಿಷಯ ಬೌದ್ಧ ಭಾರತ. ಪ್ರಾಯಶಃ ನಿಮ್ಮಲ್ಲಿ ಹೆಚ್ಚುಕಡಿಮೆ ಎಲ್ಲರೂ, ಎಡ್ವಿನ್ ಆರ್ನಾಲ್ಡ್ ರ ಪದ್ಯರೂಪದಲ್ಲಿರುವ ಬುದ್ಧನ ಜೀವನವನ್ನು ಓದಿರಬಹುದು. ನಿಮ್ಮಲ್ಲಿ ಇನ್ನೂ ಕೆಲವರು ಈ ವಿಷಯದ ಅಧ್ಯಯನವನ್ನು ಹೆಚ್ಚು ವಿದ್ವತ್ ಪೂರ್ಣವಾದ ಆಸಕ್ತಿಯಿಂದ ಮಾಡಿರಬಹುದು. ಕಾರಣ ಇಂಗ್ಲಿಷ್, ಫ್ರೆಂಚ್ ಹಾಗೂ ಜರ್ಮನ್ ಭಾಷೆಗಳಲ್ಲಿ ಬೌದ್ಧ ಸಾಹಿತ್ಯ ಸಾಕಷ್ಟು ವಿಪುಲವಾಗಿದೆ. ಮೊಟ್ಟಮೊದಲ ಬಾರಿಗೆ ಐತಿಹಾಸಿಕವಾಗಿ ಸಿಡಿದೆದ್ದ ಜಾಗತಿಕ ಧರ್ಮವಾದ, ಬೌದ್ಧ ಧರ್ಮವು ನಮ್ಮ ಅತ್ಯಂತ ಆಸಕ್ತಿ ಮತ್ತು ಅಧ್ಯಯನಕ್ಕೆ ಕಾರಣವಾಗಿರುವ ಮಹತ್ತ್ವದ ವಿಷಯವಾಗಿದೆ. ಬೌದ್ಧ ಧರ್ಮ ಉದಯವಾಗುವ ಮುಂಚೆ ಭಾರತದಲ್ಲಿ ಮತ್ತು ಇನ್ನೂ ಕೆಲವೆಡೆ ಶ್ರೇಷ್ಠವಾದ ಧರ್ಮಗಳೇನೋ ಇದ್ದವು – ಆದರೆ ಅವೆಲ್ಲಾ ತಮ್ಮ ತಮ್ಮ ಜನಾಂಗಕ್ಕೆ ಸೀಮಿತವಾಗಿದ್ದವು. ಪುರಾತನ ಹಿಂದೂಗಳಲ್ಲಿ ಅಥವಾ ಪುರಾತನ ಯಹೂದಿಗಳಲ್ಲಿ ಅಥವಾ ಪುರಾತನ ಪಾರ್ಸಿಗಳಲ್ಲಿ ಪ್ರತಿಯೊಬ್ಬರಲ್ಲೂ ಬಹಳ ಶ್ರೇಷ್ಠವಾದ ಧರ್ಮಗಳೇ ಇದ್ದುವು. ಆದರೆ ಈ ಧರ್ಮಗಳೆಲ್ಲ ಹೆಚ್ಚು ಕಡಿಮೆ ಜಾತೀಯ ಧರ್ಮಗಳಷ್ಟೇ. ಬೌದ್ಧಮತದ ಅಭ್ಯುತ್ಥಾನದೊಂದಿಗೆ ಜಗತ್ತನ್ನೇ ಗೆಲ್ಲಲು ಹೊರಟಿರುವ ಧರ್ಮದ ಒಂದು ವಿಚಿತ್ರವಾದ ಅಭಿಯಾನದ ಸೂತ್ರಪಾತವಾಯಿತು. ಬೌದ್ಧಮತ ಬೋಧಿಸಿದ ಸಿದ್ಧಾಂತಗಳನ್ನೂ, ಸತ್ಯಗಳನ್ನೂ ಹಾಗೂ ಅದು ಸಾರಬೇಕಾಗಿದ್ದ ಸಂದೇಶವನ್ನೂ ನಾವು ಬದಿಗಿಟ್ಟರೂ, ಈ ಧರ್ಮದಿಂದ ಜಗತ್ತಿನಲ್ಲಾದ ಒಂದು ಪ್ರಚಂಡ ವಿಪ್ಲವದೊಂದಿಗೆ ನಾವು ಮುಖಾಮುಖಿಯಾಗಿ ನಿಲ್ಲುತ್ತೇವೆ. ಆ ಧರ್ಮ ಉದಿಸಿದ ಕೆಲವೇ ಶತಮಾನಗಳಲ್ಲಿ ಬರಿಗಾಲಿನ, ಬೋಳು ತಲೆಯ ಬೌದ್ಧಶ್ರಮಣರು ಆ ಕಾಲದ ಸಭ್ಯ ನಾಗರಿಕ ಜಗತ್ತಿನಲ್ಲಿ ಎಲ್ಲೆಲ್ಲೂ ಹರಡಿಕೊಂಡಿದ್ದರು. ಅಷ್ಟೇ ಅಲ್ಲದೆ ಅವರು ಮತ್ತೂ ಮುಂದುವರಿದು ಲ್ಯಾಪ್‌ಲ್ಯಾಂಡಿನ ಒಂದು ಪ್ರಾಂತದಿಂದ ಫಿಲಿಫೈನ್ಸ್ ದ್ವೀಪ ಪುಂಜದ ಮತ್ತೊಂದು ಪ್ರಾಂತದವರೆಗೂ ನುಗ್ಗಿದ್ದರು. ಬುದ್ಧದೇವನು ಜನಿಸಿದ ಕೆಲವೇ ಶತಮಾನಗಳಲ್ಲಿ ಅವರು ವ್ಯಾಪಕವಾಗಿ ಹರಡಿಕೊಂಡಿದ್ದರು. ಒಂದು ಕಾಲದಲ್ಲಿ ಭರತವರ್ಷದಲ್ಲೇ ಆ ದೇಶದ ಮೂರನೇ ಎರಡು ಭಾಗದಷ್ಟು ಜನತೆ ಆ ಧರ್ಮದ ಪ್ರವಾಹದಲ್ಲಿ ಕೊಚ್ಚಿಹೋಗಿತ್ತು.

ಆದರೂ ಸಮಗ್ರ ಭಾರತವರ್ಷ ಎಂದೂ ಸಂಪೂರ್ಣವಾಗಿ ಬೌದ್ಧ ಮತಾವಲಂಬಿಯಾಗಿರಲಿಲ್ಲ. ಅದು ಆ ಧರ್ಮದಿಂದ ಪೃಥಕ್ಕಾಗಿಯೇ ನಿಂತಿತ್ತು. ಭಾರತದ ಪುರಾತನ ಧರ್ಮವಾಹಿನಿ ನಿರಂತರವಾಗಿ ಹರಿಯುತ್ತಲೇ ಇತ್ತು. ಪುರಾತನ ಯಹೂದಿಗಳಿಂದ ಕ್ರೈಸ್ತಮತಕ್ಕೊದಗಿದ ಗತಿಯೇ ಭಾರತದಲ್ಲಿ ಬೌದ್ಧ ಮತಕ್ಕೂ ಒದಗಿತು. ಅಧಿಕಾಂಶ ಯಹೂದಿಗಳು ಕ್ರೈಸ್ತಮತದಿಂದ ಆಚೆಯೇ ಉಳಿದರು. ಆದರೆ ಈ ಹೋಲಿಕೆಯು ಇಲ್ಲಿಯೇ ನಿಲ್ಲುತ್ತದೆ. ಕ್ರೈಸ್ತಮತ ಎಲ್ಲ ಯಹೂದಿ ಜನಾಂಗವನ್ನು ತನ್ನ ತೆಕ್ಕೆಯೊಳಗೆ ಹಾಕಿಕೊಳ್ಳಲಾಗದಿದ್ದರೂ, ಆ ಮತ ಸಮಗ್ರ ದೇಶವನ್ನೇ ಕಬಳಿಸುತ್ತಿತ್ತು. ಎಲ್ಲೆಲ್ಲಿ ಪ್ರಾಚೀನ ಧರ್ಮ – ಅಂದರೆ ಯಹೂದಿಗಳ ಧರ್ಮ ಅಸ್ತಿತ್ವದಲ್ಲಿತ್ತೋ ಅವುಗಳೆಲ್ಲ ಅತ್ಯಲ್ಪ ಕಾಲದಲ್ಲೇ ಕ್ರೈಸ್ತ ಮತೀಯರಿಂದ ಮುತ್ತಿಗೆ ಹಾಕಲ್ಪಟ್ಟು, ಆ ಪ್ರಾಚೀನ ಮತಾವಲಂಬಿಗಳು ಹರಿದು ಹಂಚಿಹೋದರು. ಹೀಗಾಗಿ ಯಹೂದಿಗಳ ಧರ್ಮ ಪ್ರಪಂಚದಲ್ಲಿ ಬೇರೆ ಬೇರೆ ಕಡೆ ವಿಕ್ಷಿಪ್ತ ಭಾವದಲ್ಲಿ ಇದೆ. ಆದರೆ ಭರತವರ್ಷದಲ್ಲಿ ಹುಟ್ಟಿದ ಈ ಬೃಹತ್ ಶಿಶುವಾಗಿ ಬೌದ್ಧಮತವನ್ನು ಕಟ್ಟಕಡೆಯಲ್ಲಿ ಅದರ ಜನ್ಮದಾತೆಯೇ ಅದನ್ನು ಆಪೋಶಿಸಿದ್ದರಿಂದ ಅಲ್ಲಿ ಆ ಮತ ಹೇಳ ಹೆಸರಿಲ್ಲದಂತಾಯಿತು. ಇಂದು ಇಡೀ ಭಾರತವರ್ಷದಲ್ಲಿ ಬುದ್ಧನ ಮತದ ಹೆಸರೇ ಲುಪ್ತವಾಗಿದೆಯೆಂದರೂ ಸರಿಯೇ. ಬೌದ್ಧಮತದ ಪ್ರಾಮುಖ್ಯತೆ ಹೊರದೇಶಗಳಲ್ಲಿ ಕಂಡುಬರುತ್ತದೆಯೇ ಹೊರತು ಭಾರತದಲ್ಲಿ ಅಷ್ಟಿಲ್ಲ. ಭಾರತದಲ್ಲಿನ ಶೇ. ೯೯ ಭಾಗದ ಜನರಿಗಿಂತ “ನೀವೇ” ಹೆಚ್ಚಾಗಿ ಬೌದ್ಧಮತದ ಬಗ್ಗೆ ತಿಳಿದಿರುವಿರಿ. ಭಾರತದ ಜನತೆ ಹೆಚ್ಚೆಂದರೆ ಆತನ ಹೆಸರನ್ನು ಮಾತ್ರ ಬಲ್ಲರು – "ಹೋ! ಆತನೊಬ್ಬ ಮಹಾಪ್ರವಾದಿ, ಭಗವಂತನ ಅಪರಾವತಾರವೇ ಆತ"– ಎಂದೆನ್ನುವುದರೊಂದಿಗೆ ಮುಗಿಯಿತು. ಶ‍್ರೀಲಂಕೆ ಇಂದಿಗೂ ಬುದ್ಧನ ತಾಣವಾಗಿಯೇ ಉಳಿದಿದೆ. ಹಿಮಾಲಯ ಶ್ರೇಣಿಗಳಾಚೆ ಕೆಲವು ಪ್ರದೇಶಗಳಲ್ಲಿ ಬೌದ್ಧಮತಾವಲಂಬಿಗಳು ಇನ್ನೂ ಇರುವರು ಅಷ್ಟೇ. ಅದರಿಂದಾಚೆಗೆ ಭಾರತದಲ್ಲಿ ಯಾರೂ ಇಲ್ಲ. ಆದರೆ ಬೌದ್ಧಮತ ಏಷ್ಯಾದ ಮಿಕ್ಕೆಲ್ಲ ಭಾಗಗಳಲ್ಲಿ ಹರಡಿದೆ.

ಆದರೂ ಬೌದ್ಧ ಧರ್ಮದ ಅನುಯಾಯಿಗಳಷ್ಟು ಬಹುಸಂಖ್ಯಾತರು ಮತ್ತಾವ ಧರ್ಮದ ಅನುಯಾಯಿಗಳಲ್ಲಿಯೂ ಇಲ್ಲ. ಈ ಧರ್ಮ ಪರೋಕ್ಷವಾಗಿ ಇತರೆ ಎಲ್ಲ ಧರ್ಮಗಳ ಬೋಧನೆಗಳನ್ನು ಸಾಕಷ್ಟು ಸಂಸ್ಕಾರಗೊಳಿಸಿದೆ. ಬಹಳಷ್ಟು ಬೌದ್ಧ ಮತ ಏಷ್ಯಾ ಮೈನರ್‌ ವಲಯವನ್ನೂ ಪ್ರವೇಶಿಸಿತ್ತು. ಒಂದು ಅವಧಿಯಲ್ಲಿ ಬೌದ್ಧಮತ ಮತ್ತು ನಂತರ ಬಂದ ಕ್ರೈಸ್ತಮತದ ಪಂಗಡಗಳ ನಡುವೆ ಅಸ್ತಿತ್ವ ಮತ್ತು ಪ್ರಾಬಲ್ಯಕ್ಕಾಗಿ ನಿರಂತರವಾದ ತಿಕ್ಕಾಟವಿತ್ತು. ನಾಷ್ಟಿಕರು \enginline{(Gnosties)} ಹಾಗೂ ಕ್ರೈಸ್ತ ಮತದ ಆದಿಯಲ್ಲಿದ್ದ ಹೆಚ್ಚು ಕಡಿಮೆ ಇತರೆ ಎಲ್ಲ ಪಂಗಡಗಳೂ, ಸಂಪ್ರದಾಯಗಳೂ ಬೌದ್ಧ ಮತದ ಸಂಸ್ಕಾರವನ್ನೇ ಹೊಂದಿದ್ದುವು. ಇವೆಲ್ಲವೂ ಆ ಅದ್ಭುತ ನಗರಿಯಾದ ಅಲೆಕ್ಸಾಂಡ್ರಿಯಾದಲ್ಲಿ ಬೆಸುಗೆಗೊಂಡು ರೋಮನರ ಅಧಿಪತ್ಯದಡಿ ಈಗಿನ ಕ್ರೈಸ್ತಮತದ ಉದ್ಭವವಾಯಿತು. ಬೌದ್ಧ ಮತದ ಸಾಮಾಜಿಕ ಹಾಗೂ ರಾಜಕೀಯ ಮಜಲುಗಳು ಅದರ ಸಿದ್ಧಾಂತ ಮತ್ತು ಮತಾಚರಣೆಗಳಿಗಿಂತಲೂ ಹೆಚ್ಚು ಆಸಕ್ತಿಕರವಾಗಿವೆ. ಮೊಟ್ಟ ಮೊದಲಬಾರಿಗೆ ಸಿಡಿದೆದ್ದ ಧರ್ಮದ ಪ್ರಚಂಡ ವಿಶ್ವವಿಜೇತ ಶಕ್ತಿಯ ಅಭಿವ್ಯಕ್ತಿಯಾಗಿಯೂ ಬೌದ್ಧಮತ ಬಹಳ ಆಕರ್ಷಣೀಯವಾಗಿದೆ.

ಈ ದಿನದ ಉಪನ್ಯಾಸದಲ್ಲಿ ನಾನು ಚರ್ಚಿಸಬೇಕೆಂದಿರುವ ವಿಷಯ ಭರತವರ್ಷ ಯಾವರೀತಿ ಬೌದ್ಧಮತದಿಂದ ಪ್ರಭಾವಗೊಂಡಿದೆ ಎಂಬುದು. ಬೌದ್ಧಮತ ಮತ್ತು ಅದರ ಅಭ್ಯುತ್ಥಾನವನ್ನು ಸ್ವಲ್ಪಮಟ್ಟಿಗೆ ಅರ್ಥಮಾಡಿಕೊಳ್ಳಲು, ನಾವು ಈ ಮಹಾನ್ ಧರ್ಮಗುರು ಜನ್ಮತಳೆದಾಗ ಭಾರತದ ಸ್ಥಿತಿ ಹೇಗಿತ್ತು ಎಂಬುದರ ಬಗ್ಗೆ ಕಿಂಚಿತ್ತಾದರೂ ತಿಳಿದಿರಬೇಕು.

ಆಗಿನ ಕಾಲದ ಭರತವರ್ಷದಲ್ಲಿ ಬಹು ವಿಸ್ತೃತವಾದ ವಿಶಾಲವಾದ ಧರ್ಮವೊಂದು ಆಗಲೇ ಸುಪ್ರತಿಷ್ಠಿತವಾಗಿತ್ತು. ಆ ಧರ್ಮದ ಸುವ್ಯವಸ್ಥಿತವಾದ ಶಾಸ್ತ್ರಗ್ರಂಥಗಳೇ ವೇದಗಳು. ಬೈಬಲ್ಲಿನ ಹಳೆಯ ಒಡಂಬಡಿಕೆಯಂತೆ ವೇದಸಮೂಹವೆಂಬುದು ಒಂದು ಜ್ಞಾನರಾಶಿಯಾಗಿ ಅಸ್ತಿತ್ವದಲ್ಲಿತ್ತೇ ವಿನಃ ಗ್ರಂಥದಂತಿರಲಿಲ್ಲ. ಬೈಬಲ್ ಸಹ ಬೇರೆ ಬೇರೆ ಕಾಲದಲ್ಲಿನ ವಿಭಿನ್ನ ವ್ಯಕ್ತಿಗಳು ರಚಿಸಿದ ಒಂದು ಶಾಸ್ತ್ರ ಸಮೂಹವೇ ಆಗಿದೆ. ಅದೊಂದು ಸಂಗ್ರಹ. ವೇದಗಳೆಂಬುದು ಅತಿ ವಿಶಾಲವಾದ ಜ್ಞಾನಭಂಡಾರ ಮತ್ತು ಅದರ ಎಲ್ಲ ಗ್ರಂಥಗಳೂ ದೊರಕಿದೆಯೇ ಎಂಬುದು ನಾನಂತೂ ಅರಿಯೆ. ಯಾರೂ ಈ ಗ್ರಂಥಗಳೆಲ್ಲವನ್ನೂ ಕಂಡಿದ್ದಿಲ್ಲ. ಅಷ್ಟೇಕೆ, ಭಾರತದಲ್ಲಿಯೂ ಸಹ ಯಾರೊಬ್ಬರೂ ಆ ವೈದಿಕ ಗ್ರಂಥಗಳೆಲ್ಲವನ್ನೂ ನೋಡಿಲ್ಲ. ಅವುಗಳೆಲ್ಲ ಏನಾದರೂ ಲಭ್ಯವಿದ್ದಿದ್ದರೆ ಈ ಕೋಣೆಯೂ ಸಹ ಅವುಗಳನ್ನೆಲ್ಲ ಇಡಲು ಸಾಲುತ್ತಿರಲಿಲ್ಲ. ಅದೊಂದು ಭಗವಂತನಿಂದಲೇ ನಿಃಶ್ವಾಸಿಸಲ್ಪಟ್ಟು ತಲತಲಾಂತರದಿಂದ ಹಸ್ತಾಂತರಿತವಾದ ವಿಪುಲವಾದ ವಿರಾಟ್ ಸಾಹಿತ್ಯ ರಾಶಿ. ಹಾಗಾಗಿಯೇ ಭರತವರ್ಷದಲ್ಲಿ ಗ್ರಂಥಗಳ ಬಗೆಗಿನ ಪೂಜ್ಯಭಾವನೆ ವಿಪರೀತವಾಗಿ ಸಂಪ್ರದಾಯಬದ್ಧವಾಗಿತ್ತು. ಗ್ರಂಥಗಳನ್ನೇ ಪೂಜಿಸುವ ನಿಮ್ಮ ಸಂಪ್ರದಾಯಗಳನ್ನು ಕುರಿತು ನೀವು ಗೋಳಾಡುತ್ತೀರಿ. ಆದರೆ ಹಿಂದೂಗಳಿಗೆ ತಮ್ಮ ಪವಿತ್ರ ಗ್ರಂಥಗಳ ಬಗ್ಗೆ ಎಷ್ಟು ಪೂಜ್ಯಭಾವನೆಯಿದೆ ಎಂಬುದು ನಿಮಗೆ ಗೊತ್ತಾದರೆ ನಿಮಗಾಗ ಹೇಗೆನ್ನಿಸುತ್ತದೋ ನನಗೆ ಗೊತ್ತಿಲ್ಲ. ಹಿಂದೂಗಳು, ತಮ್ಮ ಪವಿತ್ರ ವೇದಗಳು ಪ್ರತ್ಯಕ್ಷವಾಗಿ ಭಗವಂತನ ಶ‍್ರೀಮುಖದಿಂದಲೇ ನಿಃಸೃತವಾದ ಜ್ಞಾನವೆಂದೂ, ಈ ವೇದಗಳ ಮೂಲಕವೇ ದೇವರು ಈ ಸಮಸ್ತ ಚರಾಚರ ವಿಶ್ವವನ್ನು ಸೃಷ್ಟಿಸಿರುವನೆಂದೂ, ವೇದಗಳಲ್ಲಿ ಸೂಚಿತವಾಗಿರುವುದರಿಂದಲೇ ಈ ಸಮಸ್ತ ವಿಶ್ವವೂ ಅಸ್ತಿತ್ವದಲ್ಲಿರುವುದೆಂದೂ ಶ್ರದ್ಧೆಯಿಂದ ನಂಬುತ್ತಾರೆ. ಆ ವೇದದಲ್ಲೇ ಜಗತ್ತಿನ ಎಲ್ಲವೂ ವಿಕೃತವಾಗಿದೆ, ಅದರ ಹೊರಗೆ ಏನೂ ಇಲ್ಲ. 'ಹಸು' ಎಂಬ ಶಬ್ದ ವೇದದಲ್ಲಿರುವುದರಿಂದಲೇ ಹೊರಗೆ ಹಸುವು ಅಸ್ತಿತ್ವದಲ್ಲಿದೆ. ಮನುಷ್ಯನ ಅಸ್ತಿತ್ವ ಈ ಪಾರ್ಥಿವ ಜಗತ್ತಿನಲ್ಲಿರುವುದಕ್ಕೆ ಕಾರಣ ಆ ಶಬ್ದ ವೇದದಲ್ಲಿರುವುದು. ಇಲ್ಲಿಯೇ ನೀವು 'ಸೃಷ್ಟಿಯ ಆದಿಯಲ್ಲಿದ್ದುದು ಶಬ್ದವೊಂದೇ ಮತ್ತು ಆ ಶಬ್ದವೇ ಈಶ್ವರ ಅಥವಾ ದೇವರು' ಎಂದು ಕ್ರಿಶ್ಚಿಯನ್ನರು ಮುಂದೆ ಆವಿಷ್ಕರಿಸಿ, ವ್ಯಕ್ತಪಡಿಸಿದ ಸಿದ್ಧಾಂತದ ಆದಿಯನ್ನು ನೋಡುತ್ತೀರಿ. ಇದು ಭಾರತದ ಅತ್ಯಂತ ಪ್ರಾಚೀನ ಸಿದ್ಧಾಂತ. ಈ ಸಿದ್ಧಾಂತದ ಮೇಲೆಯೇ ಶಾಸ್ತ್ರಗಳ ಇಡೀ ಭಾವನೆ ನಿಂತಿರುವುದು. ಪ್ರತಿಯೊಂದು ಶಬ್ದವೂ ಭಗವಂತನ ಶಕ್ತಿ ಎಂಬುದನ್ನು ನೆನಪಿನಲ್ಲಿಡಿ. ಶಬ್ದವೇನಿದ್ದರೂ ಈ ಸ್ಥೂಲ ಭೌತಿಕ ಜಗತ್ತಿನಲ್ಲಿ ಆ ಶಕ್ತಿಯ ಬಾಹ್ಯ ಅಭಿವ್ಯಕ್ತಿಯಷ್ಟೇ. ಆದ್ದರಿಂದ ಎಲ್ಲ ಅಭಿವ್ಯಕ್ತಿಯೂ ಸ್ಥೂಲ ಭೌತಿಕಸ್ತರಕ್ಕೆ ಸಂಬಂಧಿಸಿದುದು ಮತ್ತು ಆ ಶಬ್ದವೇ ವೇದ. ಹಾಗೂ ಸಂಸ್ಕೃತವೇ ದೇವಭಾಷೆ. ಒಮ್ಮೆ ದೇವರ ಬಾಯಿಂದ ಭಾಷೆಯೊಂದು ನಿಃಸೃತವಾಯಿತು. ಆ ಭಾಷೆ ಸಂಸ್ಕೃತ ಭಾಷೆ ಅಥವಾ ದೇವಭಾಷೆ. ಆದ್ದರಿಂದಲೇ ಭಾರತೀಯರ ದೃಷ್ಟಿಯಲ್ಲಿ ಸಂಸ್ಕೃತ ಬಿಟ್ಟು ಅನ್ಯ ಭಾಷೆಯೆಲ್ಲ ಪಶುಗಳ ಅರಚಾಟವಷ್ಟೇ. ಅದನ್ನು ಸೂಚಿಸಲು ಸಂಸ್ಕೃತವನ್ನು ಮಾತನಾಡದ ಪ್ರತಿಯೊಂದು ದೇಶದ ಜನರನ್ನೂ 'ಮ್ಲೇಚ್ಛ'ರೆಂದು ಕರೆಯುವರು. ಹೇಗೆ ಗ್ರೀಕರ ಪರಿಭಾಷೆಯಲ್ಲಿ ಬರ್ಬರ ಶಬ್ದವಿದೆಯೋ ಸಂಸ್ಕೃತದಲ್ಲಿ ಅದೇರೀತಿ 'ಮ್ಲೇಚ್ಛ' ಶಬ್ದವು ಬಳಸಲ್ಪಡುತ್ತದೆ. ಬರ್ಬರರಿಗೆ ವಾಕ್‌ಸಾಮರ್ಥ್ಯ ಇಲ್ಲ. ಅವರು ಅರಚುತ್ತಾರೆ ಅಷ್ಟೆ; ಸಂಸ್ಕೃತವೊಂದೇ ದೇವಭಾಷೆ.

ಇನ್ನು ವೇದಗಳು ಯಾರಿಂದಲೂ ರಚಿಸಲ್ಪಟ್ಟಿದ್ದಲ್ಲ. ಅವು ಭಗವಂತನೊಟ್ಟಿಗೇ ಅವಚ್ಛಿನ್ನವಾಗಿ ಸಹಭಾವಿಯಾಗಿದ್ದುವು. ದೇವರು ಅನಂತ; ಹಾಗೆ ಜ್ಞಾನವೂ ಅನಂತ ಮತ್ತು ಈ ಅನಂತ ಜ್ಞಾನದ ಮೂಲಕವೇ ಜಗತ್ತಿನ ಸೃಷ್ಟಿಯಾಗಿದೆ. ಅವರಿಗಿರುವ ನೈತಿಕ ಪರಿಜ್ಞಾನವೆಲ್ಲ – ಅಂದರೆ ಒಳ್ಳೆಯದು ಕೆಟ್ಟದ್ದು – ಎಂಬ ವಿಚಾರವೆಲ್ಲ ನಿಂತಿರುವುದು ವೇದಗಳ ಅನುಶಾಸನದ ಮೇಲೆಯೇ. ಪ್ರತಿಯೊಂದೂ ವೇದಗ್ರಂಥಗಳ ವ್ಯಾಪ್ತಿಯೊಳಗೇ ಇದೆ. ಅದರಾಚೆಗೆ ಯಾವುದೂ ಹೊಗಲಾರದು. ಕಾರಣ ಭಗವದ್‌ಜ್ಞಾನದಾಚೆ ನೀವೇನನ್ನೂ ಪಡೆಯಲಾರಿರಿ. ಇದೇ ಭಾರತೀಯರ, ಪೂರ್ವಾಚಾರ ಪರಾಯಣತೆ ಅಥವಾ ಸಂಪ್ರದಾಯಬದ್ಧತೆ.

ವೇದಗಳ ಉತ್ತರ ಭಾಗದಲ್ಲಿ ನೀವು ಉಚ್ಚತಮ ಆಧ್ಯಾತ್ಮಿಕತೆಯನ್ನು ಕಾಣುತ್ತೀರಿ. ವೇದಗಳ ಮೊದಲ ಭಾಗಗಳಲ್ಲಿ ಬಹಳಷ್ಟು ಸ್ಥೂಲಾಂಶ ಇದೆ. ವೇದಗಳಿಂದ ಒಂದು ಭಾಗವನ್ನು ಉದ್ಧರಿಸಿ 'ಇದು ಒಳ್ಳೆಯದಲ್ಲ' ಎಂದು ಹೇಳುತ್ತೀರಿ. ಏಕೆಂದರೆ ಅದರಲ್ಲಿ ದುಷ್ಕೃತ್ಯವು ವಿಧಿಸಲ್ಪಟ್ಟಿದೆ. ಇದೇ ಮಾತು ಹಳೆಯ ಒಡಂಬಡಿಕೆಗೂ ಅನ್ವಯಿಸುತ್ತದೆ. ಈ ಎಲ್ಲ ಹಳೆಯ ಗ್ರಂಥಗಳಲ್ಲಿ ಕಂಡುಬರುವ ಇಂತಹ ಅನೇಕ ಸಂಗತಿಗಳನ್ನೂ ವಿಚಿತ್ರವಾದ ವಿಚಾರಗಳನ್ನೂ ಈಗಿನ ದಿನಗಳಲ್ಲಿ ನಾವು ಇಷ್ಟಪಡದೆ ಇರಬಹುದು. ನೀವು ಹೇಳುತ್ತೀರಿ: 'ಈ ಮತತತ್ತ್ವ ಸರ್ವಥಾ ಒಳ್ಳೆಯದಲ್ಲ. ಇದು ನನ್ನ ನೀತಿಸಂಹಿತೆಗೆ ಬಾಧಕವಾಗಿದೆ' ಎಂದು. ನಿಮ್ಮ ಸಂಹಿತೆಯನ್ನು ನೀವು ಹೇಗೆ ಪಡೆದಿರಿ? ಬರೀ ನಿಮ್ಮ ಸ್ವಂತ ಚಿಂತನೆಯಿಂದಲೇ? ಹಾಗಿದ್ದರೆ ದೂರನಿಲ್ಲಿ. ಯಾವುದು ದೇವರಿಂದ ಅನುಶಾಸಿಸಲ್ಪಟ್ಟಿದೆಯೋ ಅದನ್ನು ಪ್ರಶ್ನಿಸಲು ನಿಮಗ್ಯಾವ ಹಕ್ಕಿದೆ? ವೇದಗಳು "ಇದನ್ನು ಮಾಡಬೇಡಿ, ಇದು ನೀತಿಗೆಟ್ಟ ಕೆಲಸ" ಮುಂತಾಗಿ ಹೇಳಿದಾಗ ನಿಮಗದನ್ನು ಪ್ರಶ್ನಿಸಲು ಯಾವುದೇ ಹಕ್ಕಿಲ್ಲ.

ಆದರೆ ಕಷ್ಟವಿರುವುದು ಇಲ್ಲಿಯೇ. ನೀವೇನಾದರೂ "ನಮ್ಮ ಬೈಬಲ್ಲಿನಲ್ಲಿ ಹೀಗೆ ಹೇಳಿಲ್ಲ" ಎಂದರೆ ಅವರು ಅದಕ್ಕುತ್ತರವಾಗಿ "ಹೋ! ನಿಮ್ಮ ಬೈಬಲ್ಲೇ! ಅದೇನಿದ್ದರೂ ಇತಿಹಾಸದಲ್ಲಿ ನಿನ್ನೆ ಮೊನ್ನೆಯ ಶಿಶು. ವೇದಗಳ ಹೊರತು ಬೇರಿನ್ನಾವ ಬೈಬಲ್ (ಅಥವಾ ದೇವವಾಣಿ)ಯಿರಲು ಸಾಧ್ಯ? ಬೇರಿನ್ನಾವ ಗ್ರಂಥತಾನೇ ಇರಲು ಸಾಧ್ಯ? ಭಗವಂತನೇ ಎಲ್ಲ ಜ್ಞಾನದ ಆಕರ. ದೇವರೇನು ಎರಡು ಮೂರು ಬೇರೆ ಬೇರೆ ಬೈಬಲ್‌ಗಳಿಂದ ನಮಗೆ ಬೋಧಿಸುತ್ತಾನೆಯೇ? ವೇದಗ್ರಂಥಗಳ ಮೂಲಕವೇ ಅವನ ಜ್ಞಾನಬೋಧನೆ ಪ್ರಕಾಶವಾಯಿತು. ಹಾಗಾದರೆ ದೇವರು ಮೊದಲು ವೇದಗಳಲ್ಲಿ ತಪ್ಪು ಮಾಡಿದನೆಂದು ನಿಮ್ಮ ಅಭಿಪ್ರಾಯವೆ? ನಂತರ ಅದನ್ನು ಸರಿಪಡಿಸಲು, ಉತ್ತಮಗೊಳಿಸಲು ಮತ್ತೊಂದು ಬೈಬಲ್ಲನ್ನು ಮತ್ತೊಂದು ದೇಶಕ್ಕೆ ಬೋಧಿಸಿದನ್ನೆನ್ನುವಿರಾ? ವೇದಗಳಿಗಿಂತ ಪ್ರಾಚೀನತಮವಾದ ಗ್ರಂಥವನ್ನು ನೀವು ತರಲಾರಿರಿ. ಆಮೇಲಿನ ಬೇರೆ ಎಲ್ಲ ಗ್ರಂಥಗಳು ವೇದದಲ್ಲಿ ಹೇಳಿರುವುದನ್ನು ಅನುಕರಣ ಮಾಡಿ ರಚಿಸಲ್ಪಟ್ಟಿದ್ದು" ಎನ್ನುವರು. ಅವರು ನಿಮ್ಮ ಮಾತನ್ನು ಸುತಾರಾಂ ಕೇಳುವುದಿಲ್ಲ. ಕ್ರಿಶ್ಚಿಯನ್ ತನ್ನ ಬೈಬಲ್ಲನ್ನು ತಂದಾಗ, ಅವರು "ಇಲ್ಲ, ಇವೆಲ್ಲ ಕುಯುಕ್ತಿ, ದೇವರು ಒಂದೇ ಬಾರಿ ಮಾತನಾಡಿದ್ದು. ಅವನೆಂದೂ ತಪ್ಪು ಮಾಡಲಾರ" ಎನ್ನುವರು.

ಇವೆಲ್ಲವನ್ನೂ ಸ್ವಲ್ಪ ಪರಿಭಾವಿಸಿ ನೋಡಿ. ಈ ಪೂರ್ವಾಚಾರಪರಾಯಣತೆಯಂತೂ ಅತಿರೇಕವಾದುದು. ನೀವೇನಾದರೂ ಹಿಂದೂವಿಗೆ, ನಿಮ್ಮ ಸಮಾಜವನ್ನು ನೀವು ಸುಧಾರಣೆ ಮಾಡಬೇಕು ಎಂದು ಮುಂತಾಗಿ ಹೇಳಿದರೆ ಅವನು ಹೇಳುತ್ತಾನೆ: "ತಾಳಿ, ಅದೇನಾದರೂ ನಮ್ಮ ಗ್ರಂಥಗಳಲ್ಲಿದೆಯೇ? ಹಾಗೇನಾದರೂ ಇಲ್ಲವಾದರೆ ನಾನದನ್ನು ಬದಲಾಯಿಸಲು ತಲೆಕೆಡಿಸಿಕೊಳ್ಳುವುದಿಲ್ಲ; ಸ್ವಲ್ಪ ತಾಳಿ. ಇನ್ನೊಂದು ಐನೂರು ವರ್ಷಗಳಲ್ಲಿ ಇದೇ ಒಳ್ಳೆಯದೆಂಬುದನ್ನು ನೀವು ನೋಡುತ್ತೀರಿ" ಎನ್ನುವನು. ನೀವೇನಾದರೂ ಅವನಿಗೆ, “ನಿಮ್ಮಲ್ಲಿರುವ ಸಾಮಾಜಿಕ ವ್ಯವಸ್ಥೆಯೇ ಸರಿಯಲ್ಲ” ಎಂದರೆ ಅವನು ಹೇಳುತ್ತಾನೆ: "ನಿಮಗೆ ಅದು ಹೇಗೆ ಗೊತ್ತು?" "ನೋಡಿ, ಈ ವಿಷಯದಲ್ಲಿ ನಮ್ಮ ಸಾಮಾಜಿಕ ವ್ಯವಸ್ಥೆ, ಎಷ್ಟೋ ಉತ್ತಮ. ಸ್ವಲ್ಪ ಕಾಲ, ಐನೂರು ವರ್ಷತಾಳಿ, ನಿಮ್ಮ ಸಾಮಾಜಿಕ ಸಂಸ್ಥೆಗಳೆಲ್ಲ ನಿರ್ನಾಮವಾಗುವುವು. ಯೋಗ್ಯತರವಾದದ್ದು ಉಳಿಯುವುದು ಎಂಬುದೇ ನಿಜವಾದ ಪರೀಕ್ಷೆ. ನೀವಿದ್ದೀರಿ; ಆದರೆ ಐನೂರು ವರ್ಷಗಳು ಸತತವಾಗಿ ಒಟ್ಟಿಗೆ ಬದುಕಿದ ಯಾವುದೇ ಒಂದು ಸಮುದಾಯವೂ ಈ ಜಗತ್ತಿನಲ್ಲಿಲ್ಲ. ಆದರೆ ನಮ್ಮನ್ನು ನೋಡಿ! ಎಲ್ಲ ಕಾಲದಲ್ಲೂ ಕಾಲಚಕ್ರಕ್ಕೆ ಸಿಕ್ಕಿ ಜಜ್ಜಿ ಹೋಗದೆ ನಿಂತಿದ್ದೇವೆ. ಇದೊಂದು ಉತ್ಕಟವಾದ ಪೂರ್ವಾಚಾರ ಪರಾಯಣತೆಯೇ! ದೇವರ ದಯೆಯಿಂದ ನಾನು ಈ ಮಹಾ ಸಂಕಟಸಮುದ್ರವನ್ನು ದಾಟಿದ್ದೇನೆ.

ಭರತಖಂಡದಲ್ಲಿ ಆಗ ಇದ್ದುದು ಈ ರೀತಿಯ ಅತಿರೇಕವಾದ ಪೂರ್ವಾಚಾರ ಪರಾಯಣತೆ. ಅದಲ್ಲದೆ ಮತ್ತಿನ್ನೇನು ಇತ್ತು? ಪ್ರತಿಯೊಂದೂ ವಿಚ್ಛಿನ್ನವಾಗಿತ್ತು. ಇಡೀ ಸಮಾಜವೆಲ್ಲಾ ಈಗಿರುವಂತೆಯೇ – ಇನ್ನೂ ಕಟ್ಟುನಿಟ್ಟಾದ ರೂಪದಲ್ಲಿ ಜಾತಿಗಳಾಗಿ ವಿಭಜನೆಗೊಂಡಿತ್ತು. ನಾವು ಪ್ರಾಸಂಗಿಕವಾಗಿ ಇಲ್ಲಿ ಗಮನಿಸಬೇಕಾದ ವಿಷಯವೊಂದಿದೆ. ಇಲ್ಲಿ ಪಾಶ್ಚಾತ್ಯರಲ್ಲಿ ವರ್ಗಗಳನ್ನು ಸೃಷ್ಟಿಸುವ ಹೊಸ ಪ್ರವೃತ್ತಿ ಈಗೀಗ ಕಾಣಬರುತ್ತಿದೆ. ನನ್ನ ಮಟ್ಟಿಗೆ ಹೇಳುವುದಾದರೆ ನಾನಂತೂ ಜಾತಿಬಾಹಿರ. ನಾನು ಜಾತಿಯ ಕಟ್ಟಳೆಗಳನ್ನೆಲ್ಲಾ ಮುರಿದಿದ್ದೇನೆ. ಜಾತಿಬದ್ದನಾಗಿರುವುದರಲ್ಲಿ ವೈಯಕ್ತಿಕವಾಗಿ ನನಗೇನೂ ವಿಶ್ವಾಸವಿಲ್ಲ. ಅದರಲ್ಲಿ ಕೆಲವು ಬಹಳ ಒಳ್ಳೆಯ ಅಂಶಗಳಿವೆ. ಆದರೆ ನನ್ನ ಮಟ್ಟಿಗೆ ಭಗವಂತನ ದಯೆಯಿದ್ದಲ್ಲಿ ನಾನು ಯಾವ ಜಾತಿಬಂಧನದಿಂದಲೂ ಬದ್ಧನಾಗಿರುವುದಿಲ್ಲ. ಜಾತಿ ಎಂಬ ಶಬ್ದವನ್ನು ನಾನು ಯಾವ ಅರ್ಥದಲ್ಲಿ ಉಪಯೋಗಿಸುತ್ತೇನೆ ಎಂಬುದು ನಿಮಗೆ ಅರ್ಥವಾಗುತ್ತದೆ. ಏಕೆಂದರೆ ಅಂಥದೇ ಜಾತಿಗಳನ್ನು ತ್ವರೆಯಿಂದ ನೀವು ಸೃಷ್ಟಿಸುತ್ತಿದ್ದೀರಿ. ಹಿಂದೂವಿಗೆ ಅದು ವಂಶ ಪಾರಂಪರ್ಯವಾಗಿ ಬಂದ ಒಂದು ವೃತ್ತಿ. ಪ್ರಾಚೀನ ಕಾಲದಲ್ಲಿ, ಹಿಂದೂವು ಬದುಕು ಸರಾಗವಾಗಿ, ಸುಗಮವಾಗಿರಬೇಕೆಂದೆನ್ನುತ್ತಿದ್ದನು. ಪ್ರತಿಯೊಂದಕ್ಕೂ ಜೀವಕಳೆಯನ್ನು ತುಂಬುವುದು ಯಾವುದು? ಸ್ಪರ್ಧೆ, ಹೋರಾಟ. ವಂಶಪಾರಂಪರ್ಯವಾದ ವೃತ್ತಿ ಸ್ಪರ್ಧೆಗೆ ಮಾರಕವಾಗಿದೆ. ನೀವೊಬ್ಬರು ಬಡಗಿಯಾಗಿದ್ದೀರಾ? ಒಳ್ಳೆಯದು, ನಿಮ್ಮ ಮಗನೂ ಬಡಗಿಮಾತ್ರವೇ ಆಗಬೇಕು. ನೀವು ಬಣಗಿಗಳೆ? ಬಣಗಿತನವೇ ಒಂದು ಜಾತಿಯಾಗುತ್ತದೆ. ನಿಮ್ಮ ಮಕ್ಕಳು ಬಣಗಿಗಳಾಗುತ್ತಾರೆ. ನಾವಿನ್ನಾರನ್ನೂ ಈ ವೃತ್ತಿಗೆ ಬರಲು ಬಿಡುವುದಿಲ್ಲ. ಆದ್ದರಿಂದ ನೀವು ಅಲ್ಲೇ ನಿಶ್ಚಿಂತೆಯಿಂದ ಇರಬಹುದು. ನಿಮ್ಮಲ್ಲೊಬ್ಬರು ಸೈನಿಕರು, ಯೋಧರಾಗಿದ್ದರೆ, ಅದನ್ನೇ ಒಂದು ಜಾತಿಯನ್ನಾಗಿ ಮಾಡಿ. ನಿಮ್ಮಲ್ಲಿ ಯಾರಾದರೂ ಪುರೋಹಿತರೇ? ಅದನ್ನೇ ಒಂದು ಜಾತಿಯನ್ನಾಗಿ ಮಾಡಿ. ಪೌರೋಹಿತ್ಯವಂತೂ ವಂಶಪಾರಂಪರ್ಯವಾಗಿಯೇ ಬರಬೇಕು. ಈ ಧಾಟಿಯಲ್ಲೇ ವೃತ್ತಿಗಳನ್ನು ಆಧರಿಸಿ ಬಹಳ ಕಟ್ಟುನಿಟ್ಟಾದ ಜಾತಿವಿಭಜನೆಯಿತ್ತು. ಇದೊಂದು ಅಧಿಕಾರಶಾಹಿಯೇ ಸರಿ! ಅದರಲ್ಲಿರುವ ಒಂದು ಶ್ರೇಷ್ಠಾಂಶವೆಂದರೆ ಅದು ಸ್ಪರ್ಧೆಯನ್ನು ತಿರಸ್ಕರಿಸುತ್ತದೆ. ಈ ಜಾತಿ–ವೈಶಿಷ್ಟ್ಯವೇ, ಇತರ ಎಲ್ಲ ದೇಶಗಳೂ ಅಳಿದುಹೋಗಿರುವಾಗ ಈ ದೇಶವನ್ನು ಉಳಿಯುವಂತೆ ಮಾಡಿರುವುದು.

ಆದರೆ ಅದರಲ್ಲಿ ದೊಡ್ಡ ಕೆಡುಕೊಂದಿದೆ: ವ್ಯಕ್ತಿ–ವೈಶಿಷ್ಟ್ಯತೆ ಅಥವಾ ವ್ಯಕ್ತಿಸ್ವಾತಂತ್ರ್ಯಕ್ಕೆ ಅದು ಪ್ರತಿಬಂಧಕವಾಗಿದೆ. ನಾನು ಬಡಗಿಯಾಗಿ ಹುಟ್ಟಿದ್ದರಿಂದ, ನನಗಿಷ್ಟವಿಲ್ಲದಿದ್ದರೂ ನಾನು ಬಡಗಿಯೇ ಆಗಬೇಕು. ಈ ರೀತಿಯ ಜಾತಿ–ನಿಯಮಗಳು ನಮ್ಮ ಗ್ರಂಥಗಳಲ್ಲಿವೆ – ಇವೆಲ್ಲವೂ ಬುದ್ಧದೇವನು ಹುಟ್ಟುವುದಕ್ಕೆ ಮುಂಚೆ ಇದ್ದ ಚಿತ್ರ. ನಾನು ನಿಮಗೆ ಬುದ್ಧಪೂರ್ವದ ಭಾರತದಲ್ಲಿನ ಪರಿಸ್ಥಿತಿ ಬಗ್ಗೆ ಹೇಳುತ್ತಿದ್ದೇನೆ. ನೀವಾದರೋ ಈಗಿನ್ನೂ ಸಮಾಜವಾದವೆಂದು ಕರೆಯಲ್ಪಡುವ ವ್ಯವಸ್ಥೆಗಾಗಿ ಹೆಣಗಾಡುತ್ತಿದ್ದೀರಿ. ಅದರಿಂದ ಒಳ್ಳೆಯದೇನೋ ಆಗುತ್ತದೆ. ಆದರೆ ಕಟ್ಟಕಡೆಯಲ್ಲಿ ಅದು ನಮ್ಮ ಜನಾಂಗಕ್ಕೆ ಒಂದು ವ್ಯಾಧಿಯಾಗಿ ಪರಿಣಮಿಸುತ್ತದೆ. ಯಾವಾಗಲೂ ಸ್ವಾಧೀನತೆ ಅಥವಾ ಮುಕ್ತಿಯೇ ಮೂಲಮಂತ್ರವಾಗಿರಲಿ. ಮುಕ್ತರಾಗಿ! ಮುಕ್ತವಾದ ಶರೀರ, ಮುಕ್ತವಾದ ಮನಸ್ಸು ಮತ್ತು ಮುಕ್ತವಾದ ಆತ್ಮ! ನಾನಾದರೋ ಯಾವಜ್ಜೀವವೂ ಹೀಗೆಯೇ ಮುಕ್ತನಾಗಿ, ಸ್ವತಂತ್ರನಾಗಿಯೇ ಇದ್ದೇನೆ. ಬಂಧನಕ್ಕೊಳಗಾಗಿ, ಪರಾಧೀನನಾಗಿ ಒಳ್ಳೆಯದನ್ನು ಮಾಡುವುದಕ್ಕಿಂತ, ನಾನು ಸ್ವತಂತ್ರನಾಗಿ ಸ್ವಾಧೀನತಾಭಾವನೆಯಿಂದ ಕೆಟ್ಟದ್ದನ್ನಾದರೂ ಮಾಡಿಯೇನು!

ಅದು ಹಾಗಿರಲಿ, ವರ್ತಮಾನ ಕಾಲದ ಪಾಶ್ಚಾತ್ಯ ದೇಶಗಳಲ್ಲಿ ಯಾವುದಕ್ಕಾಗಿ ಜನರು ಒರಲಿಡುತ್ತಿದ್ದಾರೋ ಅದೆಲ್ಲವನ್ನೂ ಯುಗ–ಯುಗಗಳ ಹಿಂದೆಯೇ ಭಾರತದಲ್ಲಿ ಕಾರ್ಯಗತ ಮಾಡಲಾಗಿತ್ತು. ಜಮೀನೆಲ್ಲವೂ ರಾಷ್ಟ್ರೀಕರಣಗೊಂಡಿತ್ತು. ಇಂತಹ ಇನ್ನೂ ಸಾವಿರಾರು ಕ್ರಮಗಳನ್ನು ತೆಗೆದುಕೊಳ್ಳಲಾಗಿತ್ತು. ಈ ಜಾತಿ ವ್ಯವಸ್ಥೆಯ ಬಗ್ಗೆ ಸಂಕುಚಿತತೆಯ ಆರೋಪವಿದೆ. ನಿಜ, ಭಾರತೀಯರಾದರೋ ಉತ್ಕಟವಾದ ಸಮಾಜವಾದಿಗಳು. ಆದರೆ ಅದಕ್ಕೂ ಆಚೆಗೆ, ವ್ಯಕ್ತಿವೈಶಿಷ್ಟ್ಯತೆ, ವ್ಯಕ್ತಿಸ್ವಾಮ್ಯತೆ ಅಲ್ಲಿ ಶ‍್ರೀಮಂತವಾಗಿತ್ತು. ಅವರು ಪ್ರಚಂಡ ವ್ಯಕ್ತಿವೈಶಿಷ್ಟ್ಯತೆ, ವ್ಯಕ್ತಿಸ್ವಾಮ್ಯತೆಯುಳ್ಳವರಾಗಿದ್ದರು – ಅಂದರೆ ಅತ್ಯಂತ ಸಣ್ಣ ವಿಷಯಗಳಲ್ಲಿರುವ ಕಟ್ಟುಪಾಡುಗಳನ್ನೆಲ್ಲಾ ಅವರು ತ್ಯಜಿಸಿದ ನಂತರ ಇವಾವುದಕ್ಕೂ ಅವರು ಮಣಿಯುತ್ತಿರಲಿಲ್ಲ. ಭಾರತೀಯರಂತೂ ನೀವು ಹೇಗೆ ತಿನ್ನಬೇಕು, ಹೇಗೆ ಕುಡಿಯಬೇಕು, ಹೇಗೆ ಮಲಗಬೇಕು – ಅಷ್ಟೇಕೆ ಹೇಗೆ ಸಾಯಬೇಕು – ಪ್ರತಿಯೊಂದನ್ನೂ ಕಟ್ಟುಪಾಡಿನಲ್ಲಿಟ್ಟಿದ್ದಾರೆ. ಮುಂಜಾನೆ ಹಾಸಿಗೆಯಿಂದ ಎದ್ದಲಾಗಾಯ್ತು, ನೀವು ರಾತ್ರಿ ಹಾಸಿಗೆಗೆ ಹಿಂತಿರುಗಿ ಮಲಗುವವರೆಗೂ ಪ್ರತಿಯೊಂದು ಕ್ಷಣವೂ ಪ್ರತಿಯೊಂದು ಕೆಲಸದಲ್ಲೂ ನೀವು ಶಾಸ್ತ್ರೀಯ ನಿಯಮವನ್ನನುಸರಿಸಬೇಕು. ನಿಯಮ, ನಿಯಮ, ನಿಯಮ! ಒಂದು ದೇಶ ಅಷ್ಟೊಂದು ನಿಯಮ ಶೃಂಖಲೆಗಳಲ್ಲಿ ಹೇಗೆ ಬದುಕಿದೆ ಎಂಬುದೊಂದು ಆಶ್ಚರ್ಯವಲ್ಲವೇ? ನಿಯಮವೆಂದರೆ ಪ್ರಾಣಹೀನತೆ! ಯಾವ ದೇಶದಲ್ಲಿ ಎಷ್ಟು ಹೆಚ್ಚು ನಿಯಮವಿರುತ್ತದೋ ಆ ದೇಶಕ್ಕೆ ಅಷ್ಟೇ ಹಾನಿ. ಆದರೆ ವ್ಯಕ್ತಿಸ್ವಾಮ್ಯತೆ ಗಳಿಸಲು, ನಾವು ಪರ್ವತಗಳಿಗೆ ಹೋಗುತ್ತೇವೆ – ಅಲ್ಲಿ ಯಾವ ನಿಯಮವಾಗಲೀ ಸರ್ಕಾರವಾಗಲೀ ಇರುವುದಿಲ್ಲ. ನೀವು ಹೆಚ್ಚು ಹೆಚ್ಚು ನಿಯಮಗಳನ್ನು ಮಾಡಿದಷ್ಟೂ – ಅಲ್ಲೆಲ್ಲ ಹೆಚ್ಚಿನ ಸಂಖ್ಯೆಯಲ್ಲಿ ಪೊಲೀಸರೂ, ಹೆಚ್ಚು ಸಾಮಾಜಿಕತೆಯೂ ಆವಶ್ಯ. ಕಾರಣ, ಅಲ್ಲಿ ಪುಂಡರ ಹಾವಳಿಯೂ ಹೆಚ್ಚು. ಭಾರತದಲ್ಲಂತೂ ಹೀಗೆ ಪ್ರಚಂಡವಾದ ವಿಧಿನಿಯಮಗಳಿತ್ತು. ಒಂದು ಮಗು ಜನ್ಮತಾಳುತ್ತಲೇ, ಅದು ಹುಟ್ಟಿನಿಂದಲೇ ಗುಲಾಮನಾಗಿರುತ್ತದೆ – ಮೊದಲನೆಯದಾಗಿಯೇ ತನ್ನ ಜಾತಿಗೆ, ನಂತರ ತನ್ನ ದೇಶಕ್ಕೆ, ಆಹಾರ, ವಿಹಾರ ಪ್ರತಿಯೊಂದರಲ್ಲೂ ಗುಲಾಮಗಿರಿ, ಗುಲಾಮಗಿರಿ, ಗುಲಾಮಗಿರಿ! ಅವನ ಆಹಾರಸೇವನೆ, ಪಾನೀಯ ಪ್ರತಿಯೊಂದೂ ನಿಯಮಬದ್ಧ. ಅವನು ನಿಯಮಿತ ವಿಧಾನದಲ್ಲೇ ತಿನ್ನಬೇಕು. ಮೊದಲನೇ ತುತ್ತಿನೊಡನೆ ಈ ಮಂತ್ರ, ಎರಡನೇ ತುತ್ತಿನೊಡನೆ ಮತ್ತೊಂದು, ಮೂರನೇ ತುತ್ತಿನೊಡನೆ ಮಗದೊಂದು!ಹಾಗೆಯೇ ನೀರು ಕುಡಿಯುವಾಗ ಮತ್ತೊಂದು ಮಂತ್ರ. ಒಂದೇ ಎರಡೇ! ಕುಳಿತರೆ ನಿಯಮ, ನಿಂತರೆ ನಿಯಮ. ಇದನ್ನೆಲ್ಲಾ ಸ್ವಲ್ಪ ಪರಿಭಾವಿಸಿ – ನೋಡಿ. ದಿನದಿಂದ ದಿನಕ್ಕೆ ಈ ರೀತಿ ನಿಯಮಗಳಿಂದ ನೂಕಲ್ಪಡುವ ಬಾಳನ್ನು!

ಆದರೆ ಅವರು ಚಿಂತನಶೀಲರಾಗಿದ್ದರು. ಅವರಿಗೆ ಗೊತ್ತಿತ್ತು, ಇಂತಹ ಕಟ್ಟುಪಾಡು ಯಾವುದೇ ನಿಜವಾದ ಮಹತ್ತಿಗೆ ತಮ್ಮನ್ನು ಕೊಂಡೊಯ್ಯಲಾರದೆಂದು. ಅವರು ಅಂತಹ ಮಹತಿಯನ್ನರಸುವವರಿಗೆ ಒಂದು ದಾರಿಯನ್ನು ಬಿಟ್ಟುಕೊಟ್ಟರು. ಎಷ್ಟೇ ಆಗಲಿ, ಈ ನಿಬಂಧನೆಗಳೆಲ್ಲಾ ಏನಿದ್ದರೂ ಈ ಪ್ರಪಂಚಕ್ಕೆ ಮತ್ತು ಈ ಸಾಂಸಾರಿಕ ಬದುಕನ್ನು ಬಾಳಬೇಕೆನ್ನುವವರಿಗಷ್ಟೇ ಎಂಬುದನ್ನು ಅವರು ಕಂಡುಕೊಂಡರು. ನಿಮಗೆ ಹಣ ಬೇಡವೆಂದೆನಿಸಿದೊಡನೆಯೇ ಮತ್ತು ಮಕ್ಕಳು–ಮರಿಗಳಾಗಲೀ, ಈ ಪ್ರಪಂಚದ ವ್ಯವಹಾರಗಳಾಗಲೀ ಬೇಡವೆಂದೆನಿಸಿದೊಡನೆಯೇ, ನೀವು ಸಂಪೂರ್ಣವಾಗಿ ಮುಕ್ತರಾಗಿ ಈ ಸಂಸಾರದಿಂದ ನಿರ್ಗಮಿಸಬಹುದು. ಆ ರೀತಿ ನಿರ್ಗಮಿಸಿದವರನ್ನು ಎಲ್ಲವನ್ನೂ ಪರಿತ್ಯಾಗ ಮಾಡಿದ ಸಂನ್ಯಾಸಿಗಳೆಂದು ಕರೆಯುತ್ತಿದ್ದರು. ಅವರೆಂದೂ ಸಂಘಟನೆಗೊಳ್ಳಲಿಲ್ಲ. ಈಗಲೂ ಸಹ ಅಷ್ಟೇ. ಈ ಮುಕ್ತವಾದ ಸಮುದಾಯಕ್ಕೆ ಸೇರಿದ ಯಾರೇ ಆಗಲೀ – ಪುರುಷರೇ ಆಗಲಿ, ಸ್ತ್ರೀಯರೇ ಆಗಲಿ – ವಿವಾಹವಾಗಲು ನಿರಾಕರಿಸುತ್ತಾರೆ. ಅವರು ಯಾವುದೇ ದ್ರವ್ಯಸಂಚಯನವನ್ನೂ ಮಾಡುವುದಿಲ್ಲ; ಆಸ್ತಿಯನ್ನೂ ಇಟ್ಟುಕೊಳ್ಳುವುದಿಲ್ಲ. ಅವರಿಗೆ ಯಾವ ನಿಯಮವೂ ಇಲ್ಲ. ವೇದ ವಿಧಿಗಳನ್ನು ಅವರು ಪಾಲಿಸಬೇಕಿಲ್ಲ, ವೇದಗಳೂ ಅವರನ್ನು ಬಂಧಿಸಲಾರವು. ಅವರೇ ವೇದಗಳ ಮಕುಟದಲ್ಲಿರುತ್ತಾರೆ. ಅವರು ನಮ್ಮ ಸಾಮಾಜಿಕ ಸಂಸ್ಥಾ ಸಮೂಹಕ್ಕೆಲ್ಲ ವ್ಯತಿರಿಕ್ತವಾದ ಮತ್ತೊಂದು ಧ್ರುವದಲ್ಲಿರುತ್ತಾರೆ. ಜಾತಿಗತ ವಿಧಿನಿಯಮಗಳೆಲ್ಲವನ್ನೂ ಅವರು ದಾಟಿರುತ್ತಾರೆ. ಕಾರಣ, ಅವರು ಅದರಿಂದಾಚೆಗೆ ಬೆಳೆದವರಾಗಿರುತ್ತಾರೆ. ಈ ಚಿಕ್ಕಪುಟ್ಟ ವಿಧಿನಿಷೇಧ ನಿಬಂಧನೆಗಳಿಂದ ಬಂಧಿಸಲಾಗದಷ್ಟು ಮೇರುವ್ಯಕ್ತಿತ್ವ ಅವರದು. ಅವರಿಗೆ ಅವಶ್ಯವಾಗಿರುವುದು ಎರಡು ವ್ರತಗಳಷ್ಟೇ: ಒಂದು ಅವರಾವ ಆಸ್ತಿಯನ್ನೂ ಹೊಂದಿರಬಾರದು ಮತ್ತು ಎರಡನೆಯದಾಗಿ ಅವರೆಂದೂ ವಿವಾಹವಾಗ ಬಾರದು. ನೀವೇನಾದರೂ ವಿವಾಹವಾಗಿ ಸಂಸಾರ ಹೂಡಿದರೆ ಅಥವಾ ದ್ರವ್ಯಾರ್ಜನೆ ಮಾಡಿದರೆ – ಆ ಕ್ಷಣವೇ ಈ ವಿಧಿನಿಯಮಗಳೆಲ್ಲ ನಿಮ್ಮ ಮೇಲೆ ಹೇರಲ್ಪಡುತ್ತವೆ. ಆದರೆ ಇವೆರಡರಲ್ಲಿ ನೀವ್ಯಾವುದನ್ನೂ ಮಾಡದಿದ್ದಲ್ಲಿ ನೀವು ಮುಕ್ತರಾಗುತ್ತೀರಿ. ಈ ಸಂನ್ಯಾಸಿಗಳೇ ನಮ್ಮ ಜನಾಂಗದ ಜೀವಂತ ದೇವತೆಗಳಾಗಿದ್ದರು. ನಮ್ಮಲ್ಲಿನ ಶೇಕಡಾ ತೊಂಬತ್ತೊಂಬತ್ತರಷ್ಟು ಸಂತ ಮಹಾಪುರುಷರೆಲ್ಲಾ ಈ ಸಂನ್ಯಾಸಿಗಳ ವೃಂದದಲ್ಲೇ ಕಾಣಸಿಗುವುದು.

ಯಾವುದೇ ದೇಶದಲ್ಲಾಗಲೀ ಜೀವಿಯ ನಿಜವಾದ ಮಹತಿಯಿರುವುದು ಅದರ ಅಸಾಮಾನ್ಯವಾದ ವ್ಯಕ್ತಿವೈಶಿಷ್ಟ್ಯತೆ ಅಥವಾ ವ್ಯಕ್ತಿಸ್ವಾಮ್ಯತೆಯಲ್ಲಿ. ಆದರೆ ಆ ವ್ಯಕ್ತಿಸ್ವಾಮ್ಯತೆಯನ್ನು ನೀವು ಸಮಾಜದಲ್ಲಿ ಪಡೆಯಲಾಗುವುದಿಲ್ಲ. ಅಂತಹ ವ್ಯಕ್ತಿಸ್ವಾಮ್ಯತೆಯುಳ್ಳ ಜೀವಿ ಸಮಾಜದಲ್ಲಿದ್ದರೆ, ಆತ ಒಳಗೊಳಗೇ ಬೆಂದು, ಸಮಾಜದ ಬಂಧನಗಳನ್ನೆಲ್ಲಾ ಕಿತ್ತೆಸೆದು ಸಮಾಜದ ಘಟಕವನ್ನೇ ಸ್ಫೋಟಿಸಲು ಬಯಸುತ್ತಾನೆ. ಸಮಾಜವೇನಾದರೂ ಆತನನ್ನು ಅದುಮಿಡಲು ಪ್ರಯತ್ನಿಸಿದರೆ, ಆ ವ್ಯಕ್ತಿಯು ವ್ಯಕ್ತಿವೈಶಿಷ್ಟ್ಯವುಳ್ಳ, ವ್ಯಕ್ತಿಸ್ವಾಮ್ಯತೆಯುಳ್ಳ ಜೀವಿ ಸಮಾಜವೇ ಚೂರು ಚೂರಾಗಿ ಸ್ಫೋಟಗೊಳ್ಳುವಂತೆ ಮಾಡುತ್ತಾನೆ. ಆದ್ದರಿಂದಲೇ ಅಂತಹವರಿಗೆ ಭಾರತೀಯರು ಒಂದು ಸುಗಮವಾದ ದಾರಿಯನ್ನು ಮಾಡಿದರು. ಅಂತಹವರಿಗೆ ಅವರು, 'ಒಳ್ಳೆಯದು, ಒಮ್ಮೆ ನೀವು ಸಮಾಜದಿಂದ ಹೊರಬಿದ್ದ ಮೇಲೆ ನಿಮಗಿಷ್ಟ ಬಂದದ್ದೆಲ್ಲವನ್ನೂ ಬೋಧಿಸಿ ಅಥವಾ ಪ್ರಚಾರಮಾಡಿ. ನಿಮ್ಮನ್ನು ನಾವು ದೂರದಿಂದಲೇ ವಂದಿಸುತ್ತೇವೆ' ಎಂದರು. ಅದರ ಪರಿಣಾಮವಾಗಿಯೇ ಅತ್ಯಂತ ಪ್ರಚಂಡವಾದ ವ್ಯಕ್ತಿವೈಶಿಷ್ಟ್ಯತೆಯುಳ್ಳ ನರನಾರಿಯರಿರಲು ಸಾಧ್ಯವಾಗಿತ್ತು. ಯಾವುದೇ ಸಮಾಜದಲ್ಲಾಗಲೀ ಅಂತಹ ಸಂನ್ಯಾಸಿಗಳೇ ಸರ್ವಶ್ರೇಷ್ಠರಾದ ವ್ಯಕ್ತಿಗಳಾಗಿದ್ದರು. ಅಂತಹ ಗೈರಿಕವಸನಧಾರಿಯಾದ, ಮುಂಡಿತ ಮಸ್ತಕದ ಸಂನ್ಯಾಸಿಯ ಆಗಮನವಾಯಿತೆಂದರೆ ರಾಜಕುಮಾರನಿಗೂ ಸಹ ಆತನ ಸನ್ನಿಧಿಯಲ್ಲಿ ಕುಳಿತುಕೊಳ್ಳುವ ಧೈರ್ಯವಿರಲಿಲ್ಲ; ಅವನು ನಿಂತೇ ಇರಬೇಕು. ಮುಂದಿನ ಕ್ಷಣದಲ್ಲಿ ಈ ಸಂನ್ಯಾಸಿ ಆ ರಾಜನ ಅಧೀನದಲ್ಲಿರುವ ಅತ್ಯಂತ ಕಡುಬಡವನಾದ ದರಿದ್ರನೊಬ್ಬನ ಬಾಗಿಲಿನ ಮುಂದೆ ಒಂದು ತುಣುಕು ರೊಟ್ಟಿಗಾಗಿ ನಿಂತಿರಬಹುದು. ಈ ಗೈರಿಕವಸನಧಾರಿ ಸಮಾಜದ ಎಲ್ಲ ಸ್ತರದ ಜನರೊಡನೆಯೂ ಬೆರೆಯಬೇಕು. ಈ ದಿನ ಬಡವನ ಗುಡಿಸಿಲಿನಲ್ಲಿ ಅವನೊಂದಿಗೆ ಕಳೆದರೆ, ನಾಳೆ ರಾಜನ ಅರಮನೆಯಲ್ಲಿ ಸುಪ್ಪತ್ತಿಗೆಯಲ್ಲಿ ಅವನು ನಿದ್ರಿಸಬಹುದು. ಒಂದುದಿನ ರಾಜಗೃಹದಲ್ಲಿ ಸ್ವರ್ಣಪಾತ್ರೆಯಲ್ಲಿ ಮೃಷ್ಟಾನ್ನ ಉಣ್ಣುತ್ತಿದ್ದರೆ, ಮರುದಿನವೇ ಅವನಿಗೆ ತಿನ್ನಲು ಊಟವೇ ಇಲ್ಲದೆ ಉಪವಾಸದಿಂದ ಯಾವುದೋ ಮರದ ಕೆಳಗೆ ಮಲಗಿರಬಹುದು. ಸಮಾಜ ಈ ಜನರನ್ನು ಬಹಳ ಮರ್ಯಾದೆಯಿಂದ ನೋಡುತ್ತದೆ. ಇವರಲ್ಲಿ ಕೆಲವರಂತೂ, ತಮ್ಮ ಅಸಾಮಾನ್ಯವಾದ ವ್ಯಕ್ತಿವೈಶಿಷ್ಟ್ಯತೆಯ ನಿದರ್ಶನವೋ ಎಂಬಂತೆ, ಸಾರ್ವಜನಿಕ ನಿಯಮಗಳನ್ನು ಬೇಕಂತಲೇ ಉಲ್ಲಘಿಸಿ ಜನರನ್ನು ದಿಗ್ಭ್ರಮೆಗೊಳಿಸುತ್ತಾರೆ. ಆದರೆ ಅವರೆಲ್ಲಿಯವರೆವಿಗೂ – ಪರಿಪೂರ್ಣವಾದ ಪಾವಿತ್ರ್ಯತೆ ಮತ್ತು ದ್ರವ್ಯದ ಅಕಿಂಚನತೆ – ಈ ಎರಡು ತತ್ತ್ವಗಳಿಗೆ ನಿಷ್ಠರಾಗಿರುವರೋ ಅಲ್ಲಿಯವರೆವಿಗೂ ಜನತೆಯೆಂದೂ ಅವುಗಳಿಂದ ಅಷ್ಟೊಂದು ಘಾಸಿಗೊಳ್ಳುವುದಿಲ್ಲ.

ಈ ಸಂನ್ಯಾಸಿಗಳು ಪ್ರಚಂಡವಾದ ವ್ಯಕ್ತಿಸ್ವಾಮ್ಯತೆಯುಳ್ಳವರಾದ್ದರಿಂದ ಪ್ರತಿಯೊಂದು ದೇಶದಲ್ಲೂ ಸಂಚರಿಸಿ, ಹೊಸ ಹೊಸ ಸಿದ್ಧಾಂತಗಳ ಯೋಜನೆಗಳ ಸಂಧಾನದಲ್ಲಿಯೇ ಇರುತ್ತಾರೆ. ಅವರು ಯಾವಾಗಲೂ ಹೊಸತಾದದ್ದೇನನ್ನಾದರೂ ಆವಿಷ್ಕರಿಸುತ್ತಲೇ ಇರಬೇಕು. ಅವರೆಂದಿಗೂ ಹಳೆಯ ಜಾಡಿನಲ್ಲಿರಲಾರರು, ಇತರರು ನಮ್ಮನ್ನೆಲ್ಲಾ ಹಳೆಯ ಜಾಡಿನಲ್ಲಿಯೇ ಧಾವಿಸುವಂತೆ ಮಾಡಿ, ನಾವೆಲ್ಲರೂ ಒಂದೇ ರೀತಿಯಲ್ಲಿ ಯೋಚಿಸುವಂತೆ ಬಲಾತ್ಕರಿಸುತ್ತಾರೆ. ಆದರೆ ಮನುಷ್ಯ ಸ್ವಭಾವವೆನ್ನುವುದು ಮನುಷ್ಯನ ಮೂರ್ಖತೆಯನ್ನು ಮೆಟ್ಟಿನಿಲ್ಲುವುದು. ನಮ್ಮ ದೌರ್ಬಲ್ಯಗಳಿಗಿಂತಲೂ ನಮ್ಮಲ್ಲಿರುವ ಘನತೆ, ಉತ್ಕೃಷ್ಟತೆ ಪ್ರಬಲವಾದದ್ದು; ಒಳ್ಳೆಯದು ಎಂದೆಂದಿಗೂ ಕೆಟ್ಟದ್ದಕ್ಕಿಂತಲೂ ಶಕ್ತಿಶಾಲಿಯಾದದ್ದು. ಒಂದುವೇಳೆ ಇತರರು ನಾವೆಲ್ಲಾ ಹಳೆಯ ಜಾಡಿನಲ್ಲಿಯೇ ಯೋಚಿಸುವಂತೆ ಮಾಡುವುದರಲ್ಲಿ ಯಶಸ್ವಿಯಾದರೆಂದುಕೊಳ್ಳೋಣ – ಆಗ ಯೋಚಿಸಲು ನಮಗೇನೂ ಇರುವುದಿಲ್ಲ. ನಾವೆಲ್ಲರೂ ಸತ್ತಂತೆಯೇ

ಒಂದುಕಡೆ ತನ್ನ ಅತ್ಯಂತ ಬಿಗಿಯಾದ ನಿಯಮಶೃಂಖಲೆಯಲ್ಲಿ ತುಳಿತಕ್ಕೊಳಗಾದವರಿಂದ ಕೂಡಿದ, ಹೆಚ್ಚುಕಡಿಮೆ ಸತ್ತ್ವವನ್ನು ಕಳೆದುಕೊಂಡ ಸಾಮಾಜಿಕ ವ್ಯವಸ್ಥೆ. ಮತ್ತೊಂದುಕಡೆ ಈ ಶೃಂಖಲೆಗಳೆಲ್ಲವನ್ನೂ ಕಳಚಿಕೊಂಡ ಸಂನ್ಯಾಸಿವೃಂದ. ಇವರಿಬ್ಬರೂ ಪರಸ್ಪರ ಸಹಾಯಕ್ಕೆ ನಿಂತುಕೊಳ್ಳುವುದು ಅನಿವಾರ್ಯವಾಯಿತು. ಒಂದುಕಡೆ ಭಯಂಕರವಾದ ಕಠೋರ ವಿಧಿನಿಯಮಗಳು – ಹೇಗೆ ಉಸಿರಾಡಬೇಕು, ಹೇಗೆ ಮುಖ ಪ್ರಕ್ಷಾಲನ ಮಾಡಬೇಕು, ಹೇಗೆ ಹಸ್ತಪ್ರಕ್ಷಾಲನ ಮಾಡಬೇಕು, ಹೇಗೆ ದಂತಮಾರ್ಜನ ಮಾಡಬೇಕು, ಹೇಗೆ ಸ್ನಾನ ಮಾಡಬೇಕು – ಮುಂತಾದ ಸಣ್ಣ ವಿಷಯಗಳಿಂದ ಹಿಡಿದು ಸಾವಿನ ಕೊನೆಯ ಕ್ಷಣದವರೆಗೂ ಇರುವ ವಿಧಿನಿಬಂಧನೆಗಳಿಗೆ ಲೆಕ್ಕವೇ ಇಲ್ಲ. ಈ ವಿಧಿನಿಯಮಗಳಿಂದಾಚೆಗೆ ಸಂನ್ಯಾಸಿಯ ಅದ್ಭುತವಾದ ವ್ಯಕ್ತಿಸ್ವಾಮ್ಯತೆ ಮತ್ತೊಂದುಕಡೆ. ಆತನಾದರೋ ಆ ವಿಧಿನಿಬಂಧನೆಗಳಾಚೆಯೇ ವಿಹರಿಸುತ್ತಿದ್ದನು. ಈ ಪ್ರಬಲವಾದ ವ್ಯಕ್ತಿ ಸ್ವಾಧೀನತೆಯುಳ್ಳ ಪುರುಷರ ಅಥವಾ ಸ್ತ್ರೀಯರ ಸಂನ್ಯಾಸಿವೃಂದದಿಂದಲೇ ದಿನಕ್ಕೊಂದು ಹೊಸ ಸಂಪ್ರದಾಯಗಳು ಹುಟ್ಟಿಕೊಳ್ಳುತ್ತಿದ್ದುವು. ಇವರೆಷ್ಟು ಸಮಾಜದಲ್ಲಿ ಎದ್ದು ಕಾಣುತ್ತಿದ್ದರು ಎಂಬುದರ ಬಗ್ಗೆ ಪ್ರಾಚೀನ ಸಂಸ್ಕೃತ ಸಾಹಿತ್ಯದಲ್ಲಿ ಉಲ್ಲೇಖವಿದೆ. ವಿಚಿತ್ರ ವರ್ತನೆಯ ವೃದ್ಧ ಮಹಿಳೆಯೊಬ್ಬಳಿದ್ದಳು. ಆಕೆ ಮಾತನಾಡಿದಾಗಲೆಲ್ಲ ಅದರಲ್ಲಿ ಹೊಸತಾದುದೇನಾದರೂ ಇರುತ್ತಿತ್ತು. ಅವಶ್ಯವಾಗಿ ಇತರರು ಆಕೆಯನ್ನು ಕುರಿತು ಕೆಲವೊಮ್ಮೆ ಆಡಿಕೊಳ್ಳುತ್ತಿದ್ದರೂ, ಜನರಿಗೆ ಆಕೆಯನ್ನು ಕಂಡರೆ ಭಯಗೌರವಗಳಿತ್ತು. ಆಕೆಯ ನಿರ್ದೇಶನಗಳನ್ನು ಮಾತಿಲ್ಲದೆ ಪಾಲನೆ ಮಾಡುತ್ತಿದ್ದರು. ಹೀಗೆ ಪ್ರಾಚೀನ ಕಾಲದಲ್ಲಿ ಈ ರೀತಿಯ ಉತ್ಕೃಷ್ಟವಾದ ನರ–ನಾರಿಯರು ಅಧಿಕವಾಗಿಯೇ ಇದ್ದರು.

ಈ ರೀತಿ ವಿಧಿನಿಬಂಧನೆಗಳಿಂದ ತುಳಿತಕ್ಕೊಳಗಾದ ಸಮಾಜದಲ್ಲಿ ಅಧಿಕಾರವು ಪುರೋಹಿತರ ಕೈಯಲ್ಲಿತ್ತು. ನಮ್ಮ ಸಮಾಜದ ಸ್ತರವಿನ್ಯಾಸದಲ್ಲಿ ಅತ್ಯಂತ ಶ್ರೇಷ್ಠವಾದ ಜಾತಿ ಎಂದರೆ ಪುರೋಹಿತನ ಉದ್ಯಮವಾಗಿತ್ತು. ನನಗದಕ್ಕಿಂತಲೂ ಬೇರೆಯ ಸೂಕ್ತವಾದ ಪದ ಗೊತ್ತಿಲ್ಲದ ಕಾರಣ ನಾನು 'ಪುರೋಹಿತ' ಶಬ್ದವನ್ನೇ ಬಳಸುತ್ತೇನೆ. ಈ ದೇಶದಲ್ಲಿ \enginline{Priest} ಪದ ಯಾವ ಅರ್ಥದಲ್ಲಿ ವ್ಯವಹರಿಸಲ್ಪಡುವುದೋ ನಮ್ಮ ದೇಶದಲ್ಲಿ ಆ ಅರ್ಥದಲ್ಲಿ ಅದು ವ್ಯವಹರಿಸಲ್ಪಡುವುದಿಲ್ಲ. ಕಾರಣ ನಮ್ಮ ಪುರೋಹಿತ ಯಾವುದೇ ಧರ್ಮ ಅಥವಾ ದರ್ಶನದ ಬಗ್ಗೆ ಬೋಧಿಸುವುದಿಲ್ಲ. ಪುರೋಹಿತನ ಕೆಲಸವೇನಿದ್ದರೂ ನಿಬಂಧಿಸಲ್ಪಟ್ಟ ವಿಧಿವಿಧಾನಗಳೆಲ್ಲವನ್ನೂ ಚಾಚೂ ತಪ್ಪದೆ ಅದರ ಅತ್ಯಂತ ಸಣ್ಣ ವಿವರಗಳೊಂದಿಗೆ ನಡೆಸುವುದು. ಆತ ಈ ವಿಧಿನಿಬಂಧನೆಗಳನ್ನು ಅನುಸರಿಸುವುದರಲ್ಲಿ ಸಹಾಯ ಮಾಡುತ್ತಾನೆ. ಆತ ನಿಮ್ಮ ವಿವಾಹಾದಿಗಳನ್ನು ಮಾಡಿಸುತ್ತಾನೆ. ನಿಮ್ಮ ಅಂತ್ಯಕ್ರಿಯೆಯಲ್ಲಿ ಶ್ರಾದ್ಧಾದಿಗಳನ್ನು ಮಾಡಿಸಲು ಅವನು ಬರುತ್ತಾನೆ. ಸಮಾಜದಲ್ಲಿ ಪುರುಷ –ಸ್ತ್ರೀಯರಿಗೆ ಸಂಬಂಧಿಸಿದ ವಿಧಿ ವಿಹಿತವಾದ ಶಾಸ್ತ್ರೋಕ್ತವಾದ ಕರ್ಮಾಚರಣೆಯ ನಿರ್ವಹಣೆಯಲ್ಲಿ ಪುರೋಹಿತನಿರಲೇಬೇಕು. ಸಮಾಜ ವ್ಯವಸ್ಥೆಯಲ್ಲಿ ಮದುವೆಯೇ ಆದರ್ಶ. ಪ್ರತಿಯೊಬ್ಬನೂ ವಿವಾಹವಾಗಲೇ ಬೇಕಿತ್ತು. ವಿವಾಹವಾಗದೆ ಯಾವುದೇ ಧಾರ್ಮಿಕ ಕ್ರಿಯಾಕಲಾಪಗಳನ್ನು ಮಾಡಲು ಮನುಷ್ಯನು ಸಮರ್ಥನಲ್ಲ. ಅವನೇನಿದ್ದರೂ ಅರೆಮನುಷ್ಯ. ಅವನ್ಯಾವುದೇ ಧಾರ್ಮಿಕ ಅನುಷ್ಠಾನಗಳಿಗೂ ಯೋಗ್ಯನಲ್ಲ. ಪುರೋಹಿತನು ಅವಿವಾಹಿತನಾಗಿದ್ದರೆ ಯಾವ ಧಾರ್ಮಿಕ ಕ್ರಿಯಾಕಲಾಪಗಳನ್ನು ನಡೆಸುವುದಕ್ಕೂ ಯೋಗ್ಯನಲ್ಲ. ಅರೆಮನುಷ್ಯ ಅಂದರೆ ಅವಿವಾಹಿತನು ಸಮಾಜದಲ್ಲಿರಲು ಯೋಗ್ಯವಲ್ಲವೆಂದೇ ಎಣಿಸಲಾಗಿತ್ತು.

ಇತ್ತ ಪುರೋಹಿತ ವರ್ಗದ ಅಧಿಕಾರ ವಿಪರೀತವಾಗಿ ಹೆಚ್ಚುತ್ತಿತ್ತು... ನಮ್ಮ ದೇಶದ ಶಾಸನಕಾರರೆಲ್ಲರ ಸಾಮಾನ್ಯ ನೀತಿಯೇ ಈ ಪುರೋಹಿತರಿಗೆ ವಿಶೇಷವಾದ ಪ್ರಾಶಸ್ತ್ಯ, ಮನ್ನಣೆ ಕೊಡುವುದಾಗಿತ್ತು. ಅವರ ಸಾಮಾಜಿಕ ತಂತ್ರ – ನಿಮ್ಮ ಸಮಾಜತಂತ್ರದಲ್ಲಿ ಹೇಗೋ ಹಾಗೆ – ಪುರೋಹಿತನ ಕೈಯಲ್ಲಿ ಹಣ ಶೇಖರವಾಗುವುದನ್ನು ತಡೆಗಟ್ಟುವುದೇ ಆಗಿತ್ತು. ಅದೆಲ್ಲದರ ಉದ್ದೇಶವಾದರೂ ಏನು? ಪುರೋಹಿತನು ಖಂಡಿತವಾಗಿಯೂ ಸಾಮಾಜಿಕ ಮನ್ನಣೆ, ಸನ್ಮಾನಕ್ಕೆ ಅರ್ಹನಾಗಿದ್ದನು. ನೀವು ಗಮನಿಸಬೇಕಾದ ಮತ್ತೊಂದು ಅಂಶವೆಂದರೆ ಎಲ್ಲ ದೇಶಗಳಲ್ಲೂ ಸಮಾಜದ ಶ್ರೇಣಿಯಲ್ಲಿ ಪುರೋಹಿತನ ಸ್ಥಾನ ಅತ್ಯುಚ್ಚವಾದುದು. ಭಾರತದಲ್ಲಂತೂ ಅತ್ಯಂತ ಕಡುಬಡವನಾದ ಬ್ರಾಹ್ಮಣನೂ, ತನ್ನ ಜನ್ಮವಿಶೇಷದಿಂದ, ದೇಶದ ರಾಜಾಧಿರಾಜನಿಗಿಂತಲೂ ಅಧಿಕವಾದ ಸನ್ಮಾನ ಮರ್ಯಾದೆಗಳಿಗೆ ಪಾತ್ರನಾಗಿದ್ದನು. ಭಾರತದಲ್ಲಂತೂ ವಿಪ್ರರೆಂದರೆ ಮರ್ಯಾದಾಸಂಪನ್ನ ಮತ್ತು ಅತ್ಯಂತ ಸಂಭಾವಿತ. ಆದರೆ ಸಮಾಜದ ಕಾನೂನಿನ ವ್ಯವಸ್ಥೆ ಅವನನ್ನೆಂದಿಗೂ ಧನಿಕನಾಗಲು ಬಿಡುತ್ತಿರಲಿಲ್ಲ. ಈ ನಿಯಮಗಳು ಅವನನ್ನು ಚಿರ ದಾರಿದ್ರ್ಯದಲ್ಲಿಯೇ ತನ್ನ ಬಾಳನ್ನು ತೇಯುವಂತೆ ಮಾಡುತ್ತದೆ. ಅವನಿಗೆ ಸಲ್ಲುವುದೇನಿದ್ದರೂ ಈ ಶ್ರೇಷ್ಠವಾದ ಸನ್ಮಾನ, ಮರ್ಯಾದೆಯಷ್ಟೇ. ಅವನು ತನಗಿಷ್ಟಬಂದದ್ದನ್ನೆಲ್ಲಾ ಮಾಡುವಂತಿಲ್ಲ. ಸಮಾಜದ ಸ್ತರವಿನ್ಯಾಸದಲ್ಲಿ ಜಾತಿ ಎಷ್ಟು ಉಚ್ಚವಾಗಿರುತ್ತದೋ ಅಷ್ಟೂ ಆ ಜಾತಿ ಅನುಭವಿಸಬಹುದಾದ ಭೋಗಸುಖಗಳ ಮೇಲೆ ಕಡಿತಗಳಿದ್ದವು. ಜಾತಿ ಎಷ್ಟು ಉಚ್ಚವಾಗಿತ್ತೊ, ಅವನ ಆಹಾರದ ಬಗೆಗಳೂ ಕಡಿಮೆ ಹಾಗೂ ತೆಗೆದುಕೊಳ್ಳಬೇಕಾಗಿದ್ದ ಆಹಾರದ ಪ್ರಮಾಣವೂ ಕಡಮೆಯೇ, ಅವನು ಅವಲಂಬಿಸಬಹುದಾದ ವೃತ್ತಿಗಳೂ ಕಡಿಮೆ. ನಿಮಗೆ ಅವನ ಜೀವನ ಒಂದು ಕಷ್ಟ ಪರಂಪರೆಗಳ ಬವಣೆಯಲ್ಲದೆ ಮತ್ತೇನೂ ಅಲ್ಲ ಎನ್ನಿಸುವುದು ಸಹಜ. ತಿನ್ನುವುದರಲ್ಲಿ, ಕುಡಿಯುವುದರಲ್ಲಿ, ಪ್ರತಿಯೊಂದರಲ್ಲೂ ಅದೊಂದು ನಿರಂತರವಾದ ಕಠೋರ ನಿಯಮಗಳಿಂದ ಬದ್ಧವಾದ ಬಾಳು ಅವನದಾಗಿತ್ತು. ಯಾವುದೇ ತಪ್ಪಿಗಾಗಿ ಕೆಳಜಾತಿಯವನು ಕೊಡಬೇಕಾದ ದಂಡಕ್ಕಿಂತಲೂ ಉಚ್ಚಜಾತಿಯವನು ಅದರ ಹತ್ತರಷ್ಟು ತೆರಬೇಕು. ಕೆಳಜಾತಿಯವನು ಸುಳ್ಳು ಹೇಳಿದರೆ ಅವನಿಗೆ ಒಂದು ಡಾಲರ್ ದಂಡವೆಂದಾದರೆ, ಒಬ್ಬ ಬ್ರಾಹ್ಮಣ ಅದೇ ತಪ್ಪಿಗೆ ನೂರು ಡಾಲರ್ ದಂಡ ತೆರಬೇಕಿತ್ತು. ಕಾರಣ ಅವನು ಚೆನ್ನಾಗಿ ತಿಳಿದೂ ಮಾಡಿದ ತಪ್ಪು ಅದು. ಆರಂಭದಲ್ಲಿಯೇನೋ ಈ ವ್ಯವಸ್ಥೆ ಭವ್ಯವಾಗಿಯೇ ಇತ್ತು. ಆದರೆ ನಂತರ ಈ ಪುರೋಹಿತರೇ ಎಲ್ಲ ಅಧಿಕಾರವನ್ನು ತಮ್ಮ ಕೈಗೆ ತೆಗೆದುಕೊಳ್ಳಲು ಹೊರಟ ಒಂದುಕಾಲ ಬಂದಿತು. ಅವರು ತಮ್ಮ ಅಧಿಕಾರದ ಮೂಲರಹಸ್ಯ – ಧನವರ್ಜನೆ ಎಂಬುದನ್ನೇ ಮರೆತರು. ಅವರಿಗೆ ಅನ್ನ ವಸ್ತ್ರಾದಿಗಳನ್ನೆಲ್ಲ ಕೊಟ್ಟು ಸಮಾಜ ಅವರನ್ನು ಪೋಷಿಸುತ್ತಿದ್ದುದರ ಉದ್ದೇಶ ಇಷ್ಟೆ; ಅವರು ಅಧ್ಯಯನದಲ್ಲಿ, ಬೋಧನೆಯಲ್ಲಿ, ಗಹನವಾದ ಚಿಂತನೆಗಳಲ್ಲಿ ತಮ್ಮನ್ನು ತಾವೇ ತೊಡಗಿಸಿಕೊಳ್ಳಲಿ ಎಂದು. ಅವರು ಅದಕ್ಕೆ ಬದಲಾಗಿ ತಮ್ಮ ಚಾಚಿದ ಹಸ್ತಗಳಿಂದ ಸಮಾಜದ ಸಂಪತ್ತನ್ನು ತಮ್ಮ ಮುಷ್ಟಿಯಲ್ಲಿಟ್ಟುಕೊಳ್ಳಲು ಪ್ರಾರಂಭಿಸಿದರು. ನಿಮ್ಮ ಭಾಷೆಯಲ್ಲಿ ಹೇಳಬೇಕೆಂದರೆ ಅವರು ಹಣವನ್ನು ದೋಚುವ ಧನಪಿಶಾಚಿಗಳಾಗಿ ತಮ್ಮ ಘನವಾದ ಧ್ಯೇಯೋದ್ದೇಶಗಳನ್ನು ಮರೆತರು.

ಬ್ರಾಹ್ಮಣರ ನಂತರ ಇದ್ದ ಎರಡನೆಯ ವರ್ಣವೇ ರಾಜರದು, ಅಂದರೆ ಕ್ಷತ್ರಿಯರದು. ವಾಸ್ತವವಾಗಿ ಅಧಿಕಾರ ಇದ್ದುದು ಅವರ ಕೈಯಲ್ಲಿಯೇ. ಅಷ್ಟೇ ಅಲ್ಲ; ಅವರಲ್ಲಿಯೇ ನಮ್ಮ ದೇಶದ ಎಲ್ಲ ಪ್ರಾಜ್ಞರೂ ಶ್ರೇಷ್ಠ ಚಿಂತಕರೂ ಹುಟ್ಟಿರುವುದು – ಬ್ರಾಹ್ಮಣವರ್ಗದಲ್ಲಲ್ಲ. ಇದೊಂದು ಕೌತುಕವೇ ಸರಿ. ನಮ್ಮ ದೇಶದ ಎಲ್ಲ ದೈವಜ್ಞರೂ – ಒಬ್ಬರಾದರೂ ವ್ಯತಿಕ್ರಮವಿಲ್ಲದೆ – ಎಲ್ಲರೂ ಈ ರಾಜವಂಶಕ್ಕೆ ಸೇರಿದವರಾಗಿರುತ್ತಾರೆ. ಮಹಾಮನೀಷಿಯಾದ ಶ‍್ರೀಕೃಷ್ಣನೂ ಕ್ಷತ್ರಿಯಕುಲಕ್ಕೇ ಸೇರಿದವನು; ಮತ್ತು ಶ‍್ರೀರಾಮನೂ ಸಹ ಕ್ಷತ್ರಿಯ ಸಂಜಾತನೇ ಮತ್ತು ನಮ್ಮ ಶ್ರೇಷ್ಠ ದಾರ್ಶನಿಕರಲ್ಲಿ ಹೆಚ್ಚುಕಡಿಮೆ ಎಲ್ಲರೂ ಸಿಂಹಾಸನದ ಮೇಲೆ ಕುಳಿತಿದ್ದವರೇ. ಈ ರಾಜಸಿಂಹಾಸನದಿಂದಲೇ 'ತ್ಯಾಗಮಾಡಿ' ಎಂಬ ಕೂಗು ಧ್ವನಿತವಾದದ್ದು. ಈ ಕ್ಷತ್ರಿಯರೇ ಅವರ ರಾಜರಾಗಿದ್ದರು ಮತ್ತು ಇವರೇ ಅವರ ದಾರ್ಶನಿಕರೂ ಆಗಿದ್ದರು. ಇವರೇ ನಮ್ಮ ಉಪನಿಷತ್ತುಗಳಲ್ಲಿ ಮಾತನಾಡುವವರು. ಇವರು ತಮ್ಮ ಬುದ್ಧಿಶಕ್ತಿಯಲ್ಲಿ ಮತ್ತು ವಿಚಾರಗಳಲ್ಲಿ ಪುರೋಹಿತರಿಗಿಂತಲೂ ಹೆಚ್ಚು ಮೇಧಾವಿಗಳೂ ಹೆಚ್ಚು ವರ್ಚಸ್ಸುಳ್ಳವರೂ ಆಗಿದ್ದರು. ಅದಲ್ಲದೆ ಇವರು ತಮ್ಮ ಕೈಯಲ್ಲಿ ಆಡಳಿತವನ್ನಿಟ್ಟುಕೊಂಡಿದ್ದ ರಾಜರು ಬೇರೆ. ಇಷ್ಟಾದರೂ ಪುರೋಹಿತರು ಎಲ್ಲ ಅಧಿಕಾರವನ್ನೂ ಪಡೆದು ಇವರ ಮೇಲೆ ದಬ್ಬಾಳಿಕೆ ನಡೆಸುತ್ತಿದ್ದರು. ಈ ರೀತಿ ಮೊದಲ ಎರಡು ವರ್ಣದವರ– ಅಂದರೆ ಬ್ರಾಹ್ಮಣರ ಮತ್ತು ಕ್ಷತ್ರಿಯರ ನಡುವೆ ಅಧಿಕಾರಕ್ಕಾಗಿ ರಾಜಕೀಯ ಪೈಪೋಟಿಯ ಶೀತಲ ಯುದ್ಧ ನಡೆಯುತ್ತಲೇ ಇತ್ತು.

ಅಲ್ಲಿ ಮತ್ತೊಂದು ವಿಶೇಷವಿತ್ತು. ನನ್ನ ಇಲ್ಲಿನ ಮೊದಲ ಉಪನ್ಯಾಸಕ್ಕೆ ಬಂದಿದ್ದವರಿಗೆ, ನಾನು ಭಾರತದಲ್ಲಿನ ಎರಡು ದೊಡ್ಡ ಜನಾಂಗಗಳಾದ ಆರ್ಯರು ಮತ್ತು ಆರ್ಯೇತರರ ಬಗ್ಗೆ ನನ್ನ ಉಪನ್ಯಾಸದಲ್ಲಿ ಉಲ್ಲೇಖಿಸಿದ್ದು ನೆನಪಿರಬೇಕು. ಈ ಆರ್ಯ ಜನಾಂಗದಲ್ಲಿಯೇ ಮೇಲಿನ ಮೂರು ವರ್ಣದವರೂ ಇರುವುದು. ಮಿಕ್ಕವರೆಲ್ಲರೂ ಅಂತ್ಯಜರು ಅಥವಾ ಶೂದ್ರರೆಂಬ ಹೆಸರಿನಡಿ ಬರುತ್ತಾರೆ. ಅವರು ಆರ್ಯರೇ ಅಲ್ಲ. (ಹೊರಗಿನಿಂದ ಭಾರತಕ್ಕೆ ಬಂದ ಹೆಚ್ಚಿನ ವಿದೇಶಿಗಳು ಅಲ್ಲಿ ಈ ಶೂದ್ರರನ್ನು ಅಂದರೆ ಮೂಲನಿವಾಸಿಗಳನ್ನು ಕಂಡರು.) ಅದು ಹೇಗಾದರಿರಲಿ, ಈ ಆರ್ಯರಲ್ಲದ ದೊಡ್ಡ ಜನಸಮೂಹ ಮತ್ತು ಅವರೊಂದಿಗೆ ಬೆರೆತ ಇತರರು ಕಾಲಕ್ರಮದಲ್ಲಿ ಸಭ್ಯರಾದಾಗ ಅವರೆಲ್ಲರೂ ಆರ್ಯರೊಡನೆ ಸಮಾನ ಹಕ್ಕುಗಳಿಗೋಸ್ಕರ ಒಳಸಂಚು ನಡೆಸತೊಡಗಿದ ಈ ಆರ್ಯರಲ್ಲದವರು ಮೇಲ್ವರ್ಗದ ಆರ್ಯರ ಶಾಲಾ–ಕಾಲೇಜುಗಳನ್ನು ಸೇರಬಯಸಿದರು. ಅವರು ಆಚರಿಸುವ ಧಾರ್ಮಿಕ ವಿಧಿ ವಿಧಾನಗಳೆಲ್ಲವನ್ನೂ ಅವರೂ ಆಚರಿಸಬಯಸಿದರು. ಧರ್ಮದಲ್ಲಿ, ರಾಜಕೀಯದಲ್ಲಿ ಆರ್ಯರಿಗೆ ಯಾವ ಹಕ್ಕುಗಳಿದ್ದಿತೋ ಅದೇ ರೀತಿಯ ಸಮಾನ ಹಕ್ಕುಗಳು ತಮಗೂ ಇರಬೇಕೆಂದರು. ಆದರೆ ಅಂತಹ ಹಕ್ಕು ಸಾಧನೆಗೆ ಬ್ರಾಹ್ಮಣ ಪುರೋಹಿತರ ಪ್ರಬಲವಾದ ವಿರೋಧವಿತ್ತು. ಇಲ್ಲಿ ನೀವು ಗಮನಿಸಬೇಕಾದ ಅಂಶವೆಂದರೆ, ಪ್ರತಿಯೊಂದು ದೇಶದಲ್ಲೂ ಈ ಪುರೋಹಿತರ ಸ್ವಭಾವವೇ ಅಂತಹುದು. ಅವರು ಸ್ವಭಾವತಃ ಅತ್ಯಂತ ಸಂಪ್ರದಾಯವಾದಿಗಳು. ಎಲ್ಲಿಯವರೆವಿಗೆ ಪೌರೋಹಿತ್ಯ ಒಂದು ವೃತ್ತಿಯಾಗಿರುತ್ತದೋ ಅಲ್ಲಿಯವರೆವಿಗೂ ಇದು ಅವಶ್ಯವಾಗಿ ಹಾಗೆಯೇ ಇರುತ್ತದೆ. ಸಂಪ್ರದಾಯವಾದಿಗಳಾಗಿರುವುದರಲ್ಲಿ ಪುರೋಹಿತರ ಸ್ವಾರ್ಥ ಅಡಗಿದೆ. ಹೀಗಾಗಿ ಈ ಆರ್ಯ ಜನಾಂಗದ ಹೊರಗಿನಿಂದ ಉಕ್ಕಿ ಬರುತ್ತಿದ್ದ ಕ್ಷೋಭೆ ಅಸಮಾಧಾನಗಳನ್ನು ಪುರೋಹಿತರು ತಮ್ಮ ಎಲ್ಲ ಶಕ್ತಿಯಿಂದಲೂ ಹತ್ತಿಕ್ಕಲು ಪ್ರಯತ್ನಿಸುತ್ತಿದ್ದರು. ಅಲ್ಲದೆ ಈ ಆರ್ಯಜನಾಂಗದೊಳಗೇ ಪ್ರಚಂಡವಾದ ಧಾರ್ಮಿಕ ಕ್ಷೋಭೆ ಇತ್ತು ಮತ್ತು ಅದಕ್ಕೆ ಈ ಕ್ಷಾತ್ರ ವರ್ಣದವರೇ ಮುಂದಾಳುಗಳಾಗಿದ್ದರು.

ಇಷ್ಟಲ್ಲದೆ ಭಾರತದಲ್ಲಿ ಆಗಲೇ ಕಟ್ಟಾ ಆಚಾರವಂತರಾದ ಜೈನ ಸಂಪ್ರದಾಯವೊಂದಿತ್ತು. ಅವರು ಇವತ್ತಿಗೂ ಹಾಗೆಯೇ. ಅದು ಅತ್ಯಂತ ಪ್ರಾಚೀನವಾದ ಪಂಗಡವಾಗಿತ್ತು. ಅವರು ಹಿಂದೂಗಳ ಆರ್ಷೇಯ ಗ್ರಂಥಗಳಾದ ವೇದಗಳು ಪ್ರಮಾಣಸಮ್ಮತವಲ್ಲವೆಂದು ಘೋಷಿಸಿದರು. ತಾವೇ ಗ್ರಂಥಗಳನ್ನ ರಚಿಸಿ, "ಇವುಗಳು ಮಾತ್ರವೇ ಮೂಲ ಗ್ರಂಥಗಳು, ಮೂಲ ವೇದಗಳು. ಈಗ ವೇದಗಳೆಂಬ ಹೆಸರಿನಲ್ಲಿ ಚಾಲ್ತಿಯಲ್ಲಿರುವ ಗ್ರಂಥಗಳೆಲ್ಲ ಈ ಬ್ರಾಹ್ಮಣರಿಂದ ಜನರನ್ನು ಮರುಳು ಮಾಡಲು ರಚಿಸಲ್ಪಟ್ಟ ಗ್ರಂಥಗಳು" ಎಂದು ಹೇಳಿದರು. ಅವರೂ ಅದೇ ರೀತಿಯ ಜೀವನ ಯೋಜನೆಯನ್ನು ಹಾಕಿಟ್ಟಿದ್ದರು. ಶಾಸ್ತ್ರಗಳ ಬಗೆಗಿರುವ ಹಿಂದೂಗಳ ವಾದವನ್ನು ಎದುರಿಸುವುದು ಬಹಳ ಕಷ್ಟ. ಈ ಜೈನರೂ ಸಹ ತಮ್ಮ ಗ್ರಂಥಗಳ ಮೂಲಕವೇ ಈ ಪ್ರಪಂಚ ಸೃಷ್ಟಿಯಾದದ್ದೆಂದರು. ಅವರ ಗ್ರಂಥಗಳೆಲ್ಲ ಹೆಚ್ಚು ಲೋಕಾರೂಢಿಯಾದ ಭಾಷೆಯಲ್ಲೇ ರಚಿತವಾಗಿದ್ದಿತು. ಸಂಸ್ಕೃತ ಅಷ್ಟು ಹಿಂದಿನ ಕಾಲದಲ್ಲಿಯೇ ಆಡುಭಾಷೆಯಾಗಿ ವ್ಯವಹರಿಸಲ್ಪಡುವುದು ನಿಂತುಹೋಗಿತ್ತು. ಆಗಿನ ಆಡುಭಾಷೆಯೊಂದಿಗಿನ ಸಂಸ್ಕೃತದ ಸಂಬಂಧ ನಿಮ್ಮ ಲ್ಯಾಟಿನ್ ಭಾಷೆ ಆಧುನಿಕ ಇಟಾಲಿಯನ್ ಭಾಷೆಯೊಡನೆ ಹೊಂದಿರುವ ಸಂಬಂಧದಂತಿತ್ತು. ಈ ಜೈನರು ತಮ್ಮ ಗ್ರಂಥಗಳೆಲ್ಲವನ್ನೂ ಪಾಲಿ ಭಾಷೆಯಲ್ಲೇ ರಚಿಸಿದರು. ಬ್ರಾಹ್ಮಣರು ಅವರನ್ನು 'ನಿಮ್ಮ ಪುಸ್ತಕಗಳನ್ನು ಬಿಡಿ, ಅದು ಪಾಲಿಭಾಷೆಯಲ್ಲಿ ತಾನೇ ಇರುವುದು' ಎಂದು ಆಕ್ಷೇಪಿಸಿದಾಗ ಅವರು 'ಸಂಸ್ಕೃತ ಮೃತವಾದ ಭಾಷೆ' ಎಂದು ಚುಚ್ಚು ನುಡಿಯುತ್ತಿದ್ದರು.

ಈ ಜೈನರು ಅವರ ರೀತಿ–ನೀತಿಗಳಲ್ಲಿ ವೈದಿಕರಿಗಿಂತ ಭಿನ್ನರಾಗಿದ್ದರು. ವಸ್ತುತಃ ನೀವಿಲ್ಲಿ ಗಮನಿಸಬೇಕಾದುದು, ಈ ಹಿಂದೂಗಳ ಧರ್ಮಗ್ರಂಥಗಳಾದ ವೇದವಾಙ್ಮಯವು ಒಂದು ಬೃಹತ್ ಸಂಗ್ರಹ. ಧರ್ಮದ ಪರ್ಯಾಲೋಚನೆ, ಅಂದರೆ ಅಧ್ಯಾತ್ಮದ ಭಾಗ ಬರುವವರೆಗೂ ಕೆಲವು ಪ್ರಾರಂಭಿಕ ಭಾಗಗಳಲ್ಲಂತೂ ದೃಷ್ಟಿ ಸ್ಥೂಲವಾಗಿದೆ. ನಾನೀಗ ಹೇಳುತ್ತಿರುವ ಈ ಜೈನ ಮುಂತಾದ ಪಂಗಡಗಳೆಲ್ಲಾ ವೇದಗಳ ಈ ಅಧ್ಯಾತ್ಮ ಭಾಗವನ್ನೇ ತಾವು ಬೋಧಿಸುತ್ತಿರುವುದು ಎಂದು ಒತ್ತಿ ಹೇಳುತ್ತಿದ್ದರು. ಈ ಪುರಾತನ ವೇದಗಳಲ್ಲಿ ಮೂರು ಸೋಪಾನಗಳಿವೆ: ಮೊದಲನೆಯದು ಕರ್ಮಾನುಷ್ಠಾನ, ಎರಡನೆಯದು ಉಪಾಸನೆ, ಮೂರನೆಯದೇ ಜ್ಞಾನ. ವ್ಯಕ್ತಿ ಯಾವಾಗ ಕರ್ಮ ಮತ್ತು ಉಪಾಸನೆಗಳಿಂದ ತನ್ನನ್ನು ತಾನು ಪುನೀತನಾಗಿಸಿಕೊಳ್ಳುತ್ತಾನೋ ಅಂತಹವನ ಆಂತರದಲ್ಲಿ ಭಗವಂತನು ಪ್ರಕಾಶನಾಗುತ್ತಾನೆ. ಆಗ ಆ ವ್ಯಕ್ತಿ ಭಗವಂತನು ತನ್ನಲ್ಲೇ ನೆಲೆಸಿರುವುದನ್ನು ಸಾಕ್ಷಾತ್ಕಾರ ಮಾಡಿಕೊಳ್ಳುತ್ತಾನೆ. ಅಂತಹವನು ಮಾತ್ರವೇ ಭಗವಂತನನ್ನು ಕಂಡಿರಲು ಸಾಧ್ಯ. ಕಾರಣ ಆತನ ಅಂತರಂಗ ಪರಿಶುದ್ಧವಾಗಿದೆ. ಏನಿದ್ದರೂ ಕರ್ಮಾನುಷ್ಠಾನ ಮತ್ತು ಉಪಾಸನೆಯಿಂದ ಮಾತ್ರವೇ ಈ ಚಿತ್ತಶುದ್ಧಿ ಬರುವುದು. ಅಷ್ಟೇ ಹೊರತು ಮುಕ್ತಿ ಎಂಬುದು ಆಗಲೇ ಇದೆ. ಅದು ನಮಗೆ ಗೊತ್ತಿಲ್ಲ. ಆದ್ದರಿಂದಲೇ ಈ ಕರ್ಮ, ಉಪಾಸನೆ ಮತ್ತು ಜ್ಞಾನ – ಇವು ಮೂರೂ ಸೋಪಾನಗಳು.

(ಈ ಜೈನ ಸಂಪ್ರದಾಯದವರು) ಕರ್ಮಾನುಷ್ಠಾನ ಎಂದರೆ ಪರೋಪಕಾರವೆಂದೇ ಅರ್ಥೈಸಿದರು. ಕರ್ಮಾನುಷ್ಠಾನ ಎಂಬುದರ ಇಂಗಿತವೇನೋ ಸ್ವಲ್ಪಮಟ್ಟಿಗೆ ಹಾಗಿದ್ದಿರಬಹುದು. ಆದರೆ ಬ್ರಾಹ್ಮಣರಿಗೆ ಕರ್ಮಾನುಷ್ಠಾನವೆಂದರೆ ಬಹುಮಟ್ಟಿಗೆ ಈ ವಿಧಿವಿಹಿತವಾದ ಕ್ರಿಯಾದಿಗಳನ್ನು ವಿಸ್ತೃತವಾಗಿ ನಡೆಸುವುದೆಂದೇ ಆಗಿತ್ತು. ಅವು ಹಸು, ಕೋಣ, ಮೇಕೆ ಮುಂತಾದ ಪ್ರಾಣಿಗಳೆಲ್ಲವನ್ನು ಬಲಿ ಕೊಟ್ಟು, ಅವುಗಳು ಹಸಿಹಸಿಯಾಗಿರುವಾಗಲೇ ಯಜ್ಞಾಗ್ನಿಗೆ ಆಹುತಿ ನೀಡುವುದು ಮುಂತಾದ ಅನುಷ್ಠಾನಗಳೇ ಆಗಿದ್ದುವು. ಇಂತಹ ಸಮಯದಲ್ಲಿ ಜೈನರು, "ಇವೆಲ್ಲಾ ಕರ್ಮಾನುಷ್ಠಾನವಲ್ಲವೇ ಅಲ್ಲ. ಕಾರಣ ಇತರರನ್ನು ನೋಯಿಸುವುದು ಎಂದಿಗೂ ಯಾವುದೇ ಸತ್ಕರ್ಮವೂ ಆಗಲಾರದು" ಎಂದು ಘೋಷಿಸಿದರು. ಈ ಜೈನರು ಇಂತಹ ಕರ್ಮಾನುಷ್ಠಾನಗಳ ಕಡೆ ಬೆರಳುಮಾಡಿ ತೋರಿಸುತ್ತಾ, "ನಿಮ್ಮ ವೇದಗಳೆಲ್ಲಾ ಈ ಪುರೋಹಿತರಿಂದ ತಯಾರಿಸಲ್ಪಟ್ಟ ಸುಳ್ಳು ಕಂತೆಗಳು ಎಂಬುದಕ್ಕೆ ಇದೇ ನಿದರ್ಶನ. ಯಾವುದೇ ಸದ್ಗ್ರಂಥವೂ ಪ್ರಾಣಿಬಲಿ ಕೊಡುವುದು ಇವೇ ಮುಂತಾದ ಕರ್ಮಗಳನ್ನು ವಿಧಿಸುವುದಿಲ್ಲವಷ್ಟೆ! ಇದು ನಂಬಲಸಾಧ್ಯವಾದದ್ದು. ಆದ್ದರಿಂದ ನಿಮ್ಮ ವೇದಗಳಲ್ಲಿ ಲಿಖಿತವಾಗಿರುವ ಈ ಪಶುಬಲಿ ಮತ್ತು ಇತರೇ ಕರ್ಮಾನುಷ್ಠಾನಗಳು ಬ್ರಾಹ್ಮಣರ ಕಲ್ಪನೆಗಳು. ಕಾರಣ ಇದರಿಂದ ಲಾಭ ಪಡೆಯುವವರು ಅವರೊಬ್ಬರೇ. ಇವುಗಳೆಲ್ಲದರಿಂದ ಕೇವಲ ಪುರೋಹಿತನೊಬ್ಬನೇ ಹಣದ ಗಂಟು ಮಾಡಿ ಹಾಯಾಗಿ ಮನೆಯಕಡೆ ನಡೆಯುತ್ತಾನೆ. ಆದ್ದರಿಂದ ಇವೆಲ್ಲ ಪುರೋಹಿತಶಾಹಿ" ಎಂದು ವೇದಗಳನ್ನೆಲ್ಲಾ ಉಡಾಯಿಸುತ್ತಿದ್ದರು.

ಈ ಜೈನಮತದ ಒಂದು ಸಿದ್ಧಾಂತವೆಂದರೆ, ಯಾವುದೇ ದೇವರಾಗಲೀ ಇರಲು ಸಾಧ್ಯವಿಲ್ಲ ಎಂಬುದು ಅವರ ವಾದ: "ಈ ಪುರೋಹಿತರೇ ದೇವರೆಂಬುವವನನ್ನು ಹುಟ್ಟುಹಾಕಿದ್ದಾರೆ; ಕಾರಣ, ತಿಳಿಯದ ಜನರು ದೇವರಲ್ಲಿ ನಂಬಿಕೆಯಿಟ್ಟು ತಮಗೆ ದಕ್ಷಿಣೆ ಕೊಡಲಿ ಎಂದು. ಇವೆಲ್ಲಾ ಶುದ್ಧ ಅನರ್ಥ! ದೇವರೂ ಇಲ್ಲ, ದಿಂಡರೂ ಇಲ್ಲ. ಇರುವುದೇನಿದ್ದರೂ ಪ್ರಕೃತಿ ಮತ್ತು ಜೀವಿಗಳು ಅಷ್ಟೆ! ಜೀವಿಗಳು ಬದುಕಿನಲ್ಲಿ ಸಿಲುಕಿಕೊಂಡು ಮನುಷ್ಯ ಶರೀರವೆಂದು ಕರೆಯಲ್ಪಡುವ ಈ ಅರಿವೆಯನ್ನು ಹಾಕಿಕೊಂಡಿದ್ದಾರೆ" ಎಂಬುದಾಗಿತ್ತು. ಆದರೆ ಇಂತಹ ನಿಲುವಿನಿಂದ ಭೌತಿಕವಾದದ್ದೆಲ್ಲವೂ ಹೇಯವಾದುದು ಎಂಬ ಸಿದ್ಧಾಂತ ಸಹಜವಾಗಿ ಮೂಡಿತು. ದೇಹವನ್ನು ದಂಡಿಸಬೇಕು ಎಂಬ ಯಾತನಾಮಯವಾದ ತಪಸ್ಸಿನ ಸಿದ್ಧಾಂತದ ಪ್ರಪ್ರಥಮ ಪ್ರವರ್ತಕರು ಇವರೇ. ಅಶುದ್ಧವಾದುದೆಲ್ಲದರ ಪರಿಣಾಮವೇ ದೇಹವಾದುದರಿಂದ, ಈ ದೇಹವೂ ಹೇಯವಾದುದಲ್ಲವೇ ಎಂಬುದೇ ಅವರ ವಿಚಾರ. ಆದ್ದರಿಂದಲೇ ಯಾವುದೇ ಒಬ್ಬ ವ್ಯಕ್ತಿ ಒಂಟಿ ಕಾಲಿನಲ್ಲಿ ಸ್ವಲ್ಪಹೊತ್ತು ನಿಂತಿದ್ದರೆ – "ಒಳ್ಳೆಯದು, ಇದು ದೇಹಕ್ಕೆ ತಕ್ಕ ಶಾಸ್ತಿ" ಎನ್ನುತ್ತಿದ್ದರು. ನೆತ್ತಿಯೇನಾದರೂ ಗೋಡೆಗೆ ಹೊಡೆದು ಬಾತುಕೊಂಡರೆ – "ಇದರಿಂದ ಹಿಗ್ಗಬೇಕು! ಕಾರಣ ಇದು ಬಹಳ ಒಳ್ಳೆಯ ಶಾಸ್ತಿ" ಎಂದು ಉಪದೇಶಿಸುವರು. ಒಮ್ಮೆ ಫ್ರಾನ್ಸಿಸ್ಕನ್ ಸಂಪ್ರದಾಯದ ಕೆಲವು ಮೂಲಪುರುಷರು, ಸಂತ ಫ್ರಾನ್ಸಿಸ್ ನೊಡಗೂಡಿ ಯಾರನ್ನೋ ಕಾಣಲು ಒಂದು ಸ್ಥಳಕ್ಕೆ ಹೋಗುತ್ತಿದ್ದರು. ಸಂತ ಫ್ರಾನ್ಸಿಸ್ ತನ್ನೊಡನಿದ್ದ ಸಹಚರನೊಡನೆ ಹೀಗೆಯೇ ಮಾತನಾಡುತ್ತಾ ತಾವು ನೋಡಲು ಹೋಗುತ್ತಿರುವ ಆ ಯಜಮಾನ ತಮ್ಮನ್ನು ಬರಮಾಡಿಕೊಳ್ಳುವನೋ ಇಲ್ಲವೋ ಎಂದನು. ಆಗ ಆ ಸಹಚರನು ಪ್ರಾಯಶಃ ಆ ಯಜಮಾನ ತಮ್ಮನ್ನು ತಿರಸ್ಕರಿಸಬಹುದು ಎಂದು ಸೂಚಿಸಿದನು. ಆಗ ಸಂತ ಫ್ರಾನ್ಸಿಸ್, "ಸಹೋದರ, ಅಷ್ಟೇ ಆದರೆ ಸಾಲದು. ನಾವು ಹೋಗಿ ಅವನ ಬಾಗಿಲನ್ನು ತಟ್ಟಿದಾಗ ಅವನು ಈಚೆ ಬಂದು ನಮ್ಮನ್ನು ಹೊರಗೆ ಅಟ್ಟಿದನೆಂದುಕೋ – ಅದೇನೇನೂ ಅಲ್ಲ. ಆದರೆ ನಮ್ಮನ್ನೆಲ್ಲಾ ಕಟ್ಟಿಹಾಕಿ ಸರಿಯಾಗಿ ನಾಲ್ಕು ಬಿಗಿದನೆಂದುಕೋ, ಅದೂ ಸಹ ಏನೇನೂ ಅಲ್ಲ. ಅವನೇನಾದರೂ ನಮ್ಮ ಕೈಕಾಲುಗಳನ್ನೆಲ್ಲಾ ಕಟ್ಟಿ, ಮೈಯ ರೋಮ ರೋಮಗಳಿಂದಲೂ ರಕ್ತ ಚಿಮ್ಮುವಂತೆ ನಮ್ಮನ್ನು ಚೆನ್ನಾಗಿ ಥಳಿಸಿ, ಹೊರಗಡೆ ಹಿಮದ ಮೇಲೆ ನಮ್ಮನ್ನು ಬಿಸಾಕಿದ ಎಂದುಕೊ – ಆಗ ಸಾಕಷ್ಟು ಪುರಸ್ಕಾರ ಆಯಿತೆನ್ನಬಹುದು" ಎಂದನು.

ಇಂತಹ ಕಠೋರ ತಪಸ್ಸಿನ ವಿಚಾರಗಳು ಆಗಿನ ಕಾಲದಲ್ಲಿ ಪ್ರಚಲಿತವಾಗಿದ್ದುವು. ಈ ಜೈನರು ಇಂತಹ ತಪಸ್ಸಿನ ಪ್ರಪ್ರಥಮ ಮಹಾನ್ ಗುರುಗಳು. ಆದರೆ ಅವರು ಸ್ವಲ್ಪ ಮಹತ್ಕಾರ್ಯವನ್ನೂ ಮಾಡಿದರು. ಅವರು ಜನರಿಗೆ "ಯಾರನ್ನೂ ಹಿಂಸಿಸಬೇಡಿ, ಎಲ್ಲರಿಗೂ ನಿಮಗೆ ಸಾಧ್ಯವಿದ್ದಷ್ಟೂ ಒಳ್ಳೆಯದನ್ನು ಮಾಡಿ – ಇದೇ ಎಲ್ಲ ನೀತಿ, ಎಲ್ಲ ಸದಾಚಾರ; ಇವಿಷ್ಟೇ ನೀವು ಆಚರಿಸಬೇಕಾದ ಕರ್ಮ; ಮಿಕ್ಕದ್ದೆಲ್ಲವೂ ಶುದ್ಧ ಅನರ್ಥವಾದದ್ದು. ಇವೆಲ್ಲ ಬ್ರಾಹ್ಮಣರ ಕೆಲಸ, ಅವೆಲ್ಲವನ್ನೂ ಕಿತ್ತೊಗೆಯಿರಿ" ಎಂದು ಉಪದೇಶಿಸಿದರು. ಅಷ್ಟಕ್ಕೇ ನಿಲ್ಲದೆ ಈ ಜೈನರು ಕಾರ್ಯೋನ್ಮುಖರಾಗಿ ಈ ತತ್ತ್ವವೊಂದನ್ನೇ ವಿಸ್ತೃತವಾಗಿ ಆಚರಣೆಯಲ್ಲಿ ತಂದರು. ಇದೊಂದು ಅದ್ಭುತವಾದ ಆದರ್ಶ. ಅದರಲ್ಲೂ ಅಹಿಂಸೆ ಮತ್ತು ಪರೋಪಕಾರವೆಂಬ ಒಂದು ಅಮೋಘವಾದ ತತ್ತ್ವದಿಂದ, ನಾವು ನೈತಿಕ ಆದರ್ಶವೆಂದು ಕರೆಯುವ ಸಕಲವನ್ನೂ ಅವರು ಹೇಗೆ ಸಿದ್ಧಪಡಿಸುತ್ತಾರೆಂಬುದು ಒಂದು ಚಮತ್ಕಾರವೇ ಸರಿ!

ಈ ಜೈನ ಸಂಪ್ರದಾಯ ಬುದ್ಧನಿಗಿಂತಲೂ ಕೊನೆಯಪಕ್ಷ ೫೦೦ ವರ್ಷಗಳಿಗಿಂತಲೂ ಹಿಂದೆಯೇ ಅಸ್ತಿತ್ವದಲ್ಲಿತ್ತು; ಬುದ್ಧದೇವನಂತೂ ಕ್ರಿಸ್ತನಿಗಿಂತ ೫೫೦ ವರ್ಷ ಹಿಂದಿನವನಾಗಿದ್ದನು. (ಈ ಜೈನ ಸಂಪ್ರದಾಯದವರು) ಸಮಸ್ತ ಪ್ರಾಣಿಸಂಕುಲವನ್ನೂ ಐದು ಭಾಗಗಳಾಗಿ ವಿಂಗಡಿಸುತ್ತಾರೆ: ಅವುಗಳಲ್ಲಿ ಅತ್ಯಂತ ನಿಮ್ನ ಸ್ತರದಲ್ಲಿರುವ ಜೀವಿಗಳಿಗೆ, ಕೇವಲ ಒಂದೇ ಇಂದ್ರಿಯ – ಅಂದರೆ ಸ್ಪರ್ಶೇಂದ್ರಿಯ ಮಾತ್ರವೇ ಇರುವುದು. ಅದಕ್ಕಿಂತ ಮೇಲಿನ ಸ್ತರದಲ್ಲಿರುವ ಜೀವಿಗಳಿಗೆ ಸ್ಪರ್ಶ ಮತ್ತು ಸ್ವಾದ – ಎರಡು ಇಂದ್ರಿಯಗಳಿವೆ. ತದನಂತರದ ಜೀವಿಗಳಿಗೆ, ಸ್ಪರ್ಶ, ಸ್ವಾದ ಮತ್ತು ಶ್ರವಣೇಂದ್ರಿಯಗಳಿವೆ. ಚತುರ್ಥಸ್ತರದಲ್ಲಿರುವ ಜೀವಿಗಳಿಗೆ ಸ್ಪರ್ಶ, ಸ್ವಾದ, ಶ್ರವಣ ಮತ್ತು ದರ್ಶನೇಂದ್ರಿಯಗಳಿವೆ. ಎಲ್ಲಕ್ಕಿಂತಲೂ ಮೇಲಿನ ಸ್ತರದಲ್ಲಿರುವ ಜೀವಿಗಳಿಗೆ ಪಂಚೇಂದ್ರಿಯಗಳಿರುತ್ತವೆ. ಮೊದಲ ಎರಡು ಸ್ತರದಲ್ಲಿರುವ – ಒಂದು ಇಂದ್ರಿಯ ಮತ್ತು ಎರಡು ಇಂದ್ರಿಯಗಳಿರುವ – ಜೀವಿಗಳು ಬರಿದೇ ಕಣ್ಣುಗಳಿಗೆ ಗೋಚರವಲ್ಲ. ಅವು ಜಲರಾಶಿಯಲ್ಲೆಲ್ಲಾ ಇರುತ್ತವೆ. ಜೈನರ ಮತದಲ್ಲಿ (ಈ ನಿಮ್ನ ಸ್ತರದಲ್ಲಿರುವ ಜೀವರಾಶಿಯನ್ನು) ಕೊಲ್ಲುವುದು ಘೋರವಾದ ಪಾಪ! ಈ ಅತ್ಯಂತ ಸಣ್ಣ ಕ್ರಿಮಿಗಳ ಬಗ್ಗೆ ತಿಳುವಳಿಕೆ ಆಧುನಿಕ ಪ್ರಪಂಚದಲ್ಲಿ ಕಳೆದ ಇಪ್ಪತ್ತು ವರ್ಷಗಳಿಂದಷ್ಟೇ ಮೂಡಿದೆ; ಆದ್ದರಿಂದ ಅವುಗಳ ಬಗ್ಗೆ ಯಾರಿಗೂ ಏನೂ ಗೊತ್ತಿರಲಿಲ್ಲ. ಜೈನರ ಮತದಲ್ಲಿ ಈ ಅತ್ಯಂತ ನಿಮ್ನ ಸ್ತರದ ಪ್ರಾಣಿಗಳು ಏಕೇಂದ್ರಿಯವುಳ್ಳವು – ಸ್ಪರ್ಶಾನುಭೂತಿಯನ್ನು ಬಿಟ್ಟರೆ ಅವಕ್ಕೆ ಬೇರೇನೂ ಇಲ್ಲ. ಅದರ ಮೇಲಿನ ಸ್ತರದಲ್ಲಿರುವ ಪ್ರಾಣಿಗಳೂ ಸಹ ಗೋಚರವಲ್ಲ. ಆದ್ದರಿಂದಲೇ ನೀರನ್ನು ಕುದಿಸಿದಾಗ ಈ ಪ್ರಾಣಿಗಳೆಲ್ಲ ಕೊಲ್ಲಲ್ಪಡುತ್ತವೆ ಎಂಬುದು ಅವರಿಗೆಲ್ಲ ಗೊತ್ತಿತ್ತು. ಆದ್ದರಿಂದ ಈ ಜೈನಶ್ರಮಣರು ನೀರಡಿಕೆಯಿಂದ ಸತ್ತರೂ ಸರಿಯೇ ಕುದಿಸಿದ ನೀರನ್ನು ಕುಡಿದು ಈ ಪ್ರಾಣಿಗಳನ್ನೆಂದೂ ಕೊಲ್ಲುತ್ತಿರಲಿಲ್ಲ. ಆದರೆ (ಅದೇ ಜೈನಶ್ರಮಣ) ನಿಮ್ಮ ಮನೆಬಾಗಿಲಿಗೆ ಬಂದಾಗ ನೀವೇನಾದರೂ ಆತನಿಗೆ ಕುಡಿಯಲು ಕುದಿಸಿ ಆರಿಸಿದ ನೀರನ್ನು ಕೊಟ್ಟರೆ – ಆ ನೀರಿನ ಸೌಲಭ್ಯ ಶ್ರಮಣನಿಗಾಯಿತೇ ವಿನಃ ಆ ಪ್ರಾಣಿಗಳನ್ನು ಕೊಂದ ಪಾಪ ನಿಮಗೇ ತಟ್ಟುವುದು; ಅವನಿಗಲ್ಲ. ಅವರು ಈ ರೀತಿಯ ವಿಚಾರಗಳನ್ನು ವಿಪರೀತಕ್ಕೊಯ್ದದ್ದರಿಂದ ಅವು ಹಾಸ್ಯಾಸ್ಪದವಾದುವು. ಉದಾಹರಣೆಗೆ, ಆ ಶ್ರಮಣ ನೇನಾದರೂ ಸ್ನಾನ ಮಾಡಿದರೆ – ಮೈ ಉಜ್ಜುವಾಗ – ಈ ರೀತಿಯ ಅನೇಕ ಸೂಕ್ಷ್ಮಜೀವಿಗಳನ್ನು ಕೊಲ್ಲಬೇಕಾಗುತ್ತದೆ. ಆದ್ದರಿಂದ ಅವನೆಂದಿಗೂ ಸ್ನಾನವನ್ನೇ ಮಾಡುವುದಿಲ್ಲ. ಬದಲಾಗಿ ತಾನೇ ಕೊಲ್ಲಲ್ಪಟ್ಟರೂ ಸಹ ಅದು ಸರಿಯೆಂದೇ ಹೇಳುತ್ತಾನೆ. ಇವರಿಗೆ ಬದುಕಬೇಕೆಂಬ ಕಾಳಜಿಯಿರಲಿಲ್ಲ. ತಾನು ಸಾಯಿಸಲ್ಪಟ್ಟರೂ ಮತ್ತೊಂದು ಪ್ರಾಣಿಯನ್ನು ರಕ್ಷಿಸುವ ಗೀಳು ಇವರದು.

ಆ ಕಾಲದಲ್ಲಿ ಈ ರೀತಿಯ ಜೈನರಿದ್ದರು. ಅವರಲ್ಲದೆ, ಕಠೋರವಾದ ದೇಹದಂಡನಾ ಕ್ರಮಗಳನ್ನನುಸರಿಸುತ್ತಿದ್ದ ಇನ್ನೂ ಇತರೆ ತಾಪಸಿಗಳ ಪಂಗಡಗಳೂ ಇದ್ದವು. ಇವೆಲ್ಲ ಒಂದುಕಡೆ ಹೀಗಿರುತ್ತಿದ್ದರೆ ಪುರೋಹಿತ ವರ್ಗದ ಮತ್ತು ರಾಜವರ್ಗದ ಮಧ್ಯ ರಾಜಕೀಯ ಮಾತ್ಸರ್ಯ, ವೈಮನಸ್ಯ, ಪೈಪೋಟಿಯಂತ ಇದ್ದೇ ಇತ್ತು. ಅದರ ಜೊತೆ ಜೊತೆಯಲ್ಲೇ ವಿಕ್ಷುಬ್ಧವಾದ ಸಂಪ್ರದಾಯಗಳೂ ತಲೆಯೆತ್ತುತ್ತಿದ್ದವು. ಇವೆಲ್ಲಕ್ಕಿಂತಲೂ ದೊಡ್ಡದಾದ ಸಮಸ್ಯೆ ಮತ್ತೊಂದಿತ್ತು: ಬಹುಸಂಖ್ಯಾತರಾಗಿದ್ದ ಸಾಧಾರಣ ಜನತೆ, ಆರ್ಯರಿಗಿರುವಂತೆಯೇ ತಮಗೂ ಸಮಾನವಾದ ಹಕ್ಕುಗಳು ಇರಬೇಕೆಂದು ಹಂಬಲಿಸುತ್ತಿದ್ದರು. ತಮ್ಮ ಬದಿಯಲ್ಲೇ ಜೀವದಾನ ಮಾಡುವ ಪ್ರಕೃತಿಯ ನಿತ್ಯವಾಹಿನಿ ಹರಿಯುತ್ತಿದ್ದರೂ, ಅದರಿಂದ ಒಂದು ಹನಿಯನ್ನೂ ಕುಡಿಯಲು ಹಕ್ಕಿಲ್ಲದೆ, ಬಾಯಾರಿಕೆಯಿಂದಲೇ ಸಾಯಬೇಕಾಗಿದ್ದ ಜನಸ್ತೋಮವು ಒಂದು ಪ್ರಕ್ಷುಬ್ಧ ಸ್ಥಿತಿಯಲ್ಲಿತ್ತು.

ಇಂತಹ ಒಂದು ಮಹಾಸಂದಿಗ್ಧ ಕಾಲದಲ್ಲಿಯೇ ಈ ಮಹಾನ್ ಪುರುಷ ಬುದ್ಧದೇವನ ಆವಿರ್ಭಾವವಾಯಿತು. ನಿಮ್ಮಲ್ಲಿ ಬಹುತೇಕರು ಆತನ ಬಗ್ಗೆ, ಆತನ ಜೀವನದ ಬಗ್ಗೆ ತಿಳಿದಿದ್ದೀರಿ. ಯಾವುದೇ ಮಹಾಪುರುಷನ ಸುತ್ತ ಹೆಣೆಯಲ್ಪಡುವಂತೆ ದಂತಕತೆಗಳು ಪವಾಡಗಳೆಲ್ಲವೂ ಇವನ ಸುತ್ತಲೂ ಇದ್ದಾಗ್ಯೂ, ಮೊಟ್ಟ ಮೊದಲನೆಯದಾಗಿ ಈತ ಜಗತ್ತಿನ ಐತಿಹಾಸಿಕ ಪ್ರವಾದಿಗಳಲ್ಲಿ ಅತ್ಯಂತ ಪ್ರಮುಖನಾದವನು. ಜಗತ್ತಿನ ಪ್ರವಾದಿಗಳಲ್ಲಿ ಇಬ್ಬರು ಪ್ರವಾದಿಗಳಿಗಂತೂ ಅತ್ಯಂತ ಹೆಚ್ಚಿನ ಪುರಾವೆಗಳಿರುವ ಐತಿಹಾಸಿಕತೆಯಿದೆ. ಒಬ್ಬಾತ ಅತ್ಯಂತ ಪ್ರಾಚೀನವಾದ ಬುದ್ಧ ಮತ್ತೊಬ್ಬನೇ ಮಹಮ್ಮದ್. ಕಾರಣ ಇವರಿಬ್ಬರ ಬಗ್ಗೆ ಅವರ ಮಿತ್ರರಲ್ಲಿ ಹಾಗೂ ವೈರಿಗಳಲ್ಲಿ ಒಮ್ಮತವಿರುವುದು ನಮಗೆ ಕಂಡುಬರುತ್ತದೆ. ಆದ್ದರಿಂದ ಅಂತಹ ವ್ಯಕ್ತಿಗಳಿದ್ದರೆಂಬುದಂತೂ ನಮಗೆ ಖಚಿತ. ಇನ್ನು ಮಿಕ್ಕ ಇತರರ ಬಗ್ಗೆಯಾದರೋ ಅವರ ಶಿಷ್ಯರು ಏನು ಹೇಳುತ್ತಾರೋ ಅದನ್ನೇ ನಾವು ಒಪ್ಪಿಕೊಳ್ಳಬೇಕಲ್ಲದೆ, ಬೇರೆ ಹೆಚ್ಚಾಗಿ ಅವರ ಬಗ್ಗೆ ನಮಗೇನೂ ಗೊತ್ತಿಲ್ಲ. ನಿಮಗೆಲ್ಲ ಗೊತ್ತಿರುವಂತೆ ನಮ್ಮ ಹಿಂದೂ ಅವತಾರವಾದ ಶ‍್ರೀಕೃಷ್ಣನಂತೂ ಪುರುಷನೆಂದು ಅಪ್ಪಟ ಪುರಾಣ ಪುರುಷನೇ ಸರಿ. ಅವನ ಜೀವನದ ಬಗ್ಗೆ ಪೌರಾಣಿಕತೆ ದಟ್ಟವಾಗಿದೆ. ಅವನ ಬಗೆಗಿನ ಪ್ರತಿಯೊಂದೂ, ಅವನ ಜೀವನಕಥೆಯ ಬಹುಭಾಗ ಬರೀ ಅವನ ಶಿಷ್ಯರಿಂದಲೇ ಬರೆಯಲ್ಪಟ್ಟಿದ್ದು, ಕೆಲವೊಮ್ಮೆಯಂತೂ ಒಂದೇ ವ್ಯಕ್ತಿಯಲ್ಲಿ ಮೂರು–ನಾಲ್ಕು ವ್ಯಕ್ತಿಗಳು ಮಿಳಿತವಾಗಿರುತ್ತಾರೆ. ಹಾಗೆ ನೋಡಿದರೆ ಅನೇಕ ಅವತಾರ ಪುರುಷರ ಬಗೆಗೆ ಅಷ್ಟೊಂದು ಸ್ಪಷ್ಟವಾಗಿ ನಮಗೇನೂ ಗೊತ್ತಿಲ್ಲ. ಆದರೆ ಬುದ್ಧನ ಸಂಬಂಧವಾಗಿ ಈತನ ಮಿತ್ರರೂ ಮತ್ತು ವೈರಿಗಳಿಬ್ಬರೂ ಬರೆದಿರುವುದರಿಂದ ಇಂತಹ ಐತಿಹಾಸಿಕ ವ್ಯಕ್ತಿ ಇದ್ದನೆಂಬುದಂತೂ ಖಚಿತ. ಸಾಮಾನ್ಯವಾಗಿ ಈ ಪ್ರಪಂಚದಲ್ಲಿ ಒಬ್ಬ ಮಹಾನ್ ವ್ಯಕ್ತಿಯ ಬಗ್ಗೆ ರಾಶಿರಾಶಿಯಾಗಿ ಹೆಣೆಯಲ್ಪಡುವ ಕಟ್ಟುಕತೆಗಳು, ಅಲೌಕಿಕತೆ, ಪವಾಡಗಳು ಮತ್ತು ಇತರೆ ಉಪಾಖ್ಯಾನಗಳು – ಇವುಗಳೆಲ್ಲವನ್ನೂ ವಿಶ್ಲೇಷಿಸಿ ನೋಡಿದಾಗ ಹೊರಗಿನ ಕಥಾ ಸಮೂಹದ ಅಂತರಾಳದಲ್ಲಿ, ಪ್ರತಿಯೊಂದರಲ್ಲೂ ಒಂದು ಅಂತಸ್ಸತ್ವವಿರುವುದು ಕಂಡುಬರುತ್ತದೆ. ಹಾಗೆಯೇ ಈ ವ್ಯಕ್ತಿಯ ಬಗ್ಗೆ ಇರುವ ಕಥಾನಕದಲ್ಲಿ ಒಂದು ಅಂತಸ್ಸತ್ವ ಎದ್ದು ಕಾಣುತ್ತದೆ. ಅದೆಂದರೆ ಈತ ತನಗಾಗಿ, ತನ್ನ ಸ್ವಾರ್ಥಕ್ಕಾಗಿ ಎಂದಿಗೂ ಏನನ್ನೂ ಮಾಡಿದವನಲ್ಲ! ಇದನ್ನು ನಾವು ಹೇಗೆ ತಿಳಿಯಬಹುದೆಂದರೆ ಒಬ್ಬ ವ್ಯಕ್ತಿಯ ಸುತ್ತ ದಂತಕತೆಗಳನ್ನು ಹೆಣೆಯುವಾಗ, ಆ ದಂತಕಥೆಗಳಿಗೆ ಆ ವ್ಯಕ್ತಿಯ ಸಾಮಾನ್ಯ ಮನೋಧರ್ಮ ಮತ್ತು ಚಲನವಲನಗಳ ಲೇಪವಿರಲೇಬೇಕು. ಬುದ್ಧನ ಸಂಬಂಧದ ಒಂದೇ ಒಂದು ದಂತಕಥೆಯಲ್ಲಾಗಲೀ ಆತನ ಮೇಲೆ ಯಾವುದೇ ದುರ್ನಿತಿಯ ಅಥವಾ ದುರಾಚರಣೆಯ ಅಥವಾ ನೀಚಪ್ರವೃತ್ತಿಯ ಆರೋಪಣೆ ಇರುವುದು ಕಾಣಬರುವುದಿಲ್ಲ. ಆತನ ವೈರಿಗಳೂ ಸಹ ಆತನ ಬಗ್ಗೆ ಮೆಚ್ಚಬಹುದಾದ ಒಂದು ಚಿತ್ರಣವನ್ನೇ ಕೊಟ್ಟಿರುತ್ತಾರೆ.

ಬುದ್ಧದೇವನ ಜನ್ಮಗ್ರಹಣವಾದಾಗ ಆತನ ಮುಖಮುದ್ರೆ ಎಷ್ಟು ನಿರ್ಮಲವಾಗಿತ್ತೆಂದರೆ, ಯಾರೇ ಆಗಲಿ ಆತನ ವದನವನ್ನು ದೂರದಿಂದ ನೋಡಿದರೂ ಆ ವ್ಯಕ್ತಿ ಈ ಬಾಹ್ಯಾಚರಣೆಗಳ ಆಡಂಬರದ ಧರ್ಮಕ್ಕೆ ವಿದಾಯ ಹೇಳಿ, ಸಂನ್ಯಾಸಿಯಾಗಿ, ಮುಕ್ತನಾಗುತ್ತಿದ್ದನು. ಆಗ ದೇವತೆಗಳೆಲ್ಲ ಕಂಗಾಲಾಗಿ, "ಇನ್ನು ನಮಗೆ ಉಳಿಗಾಲವಿಲ್ಲ" ಎಂಬುದನ್ನು ಕಂಡರು. ಕಾರಣ, ಈ ದೇವತೆಗಳೆಲ್ಲ ಹೆಚ್ಚುಕಡಿಮೆ, ಈ ಬಾಹ್ಯಾಚರಣೆಯ ಯಜ್ಞಯಾಗಾದಿಗಳ ಹವಿಸ್ಸಿನ ಮೇಲೆಯೇ ಬದುಕಿರುತ್ತಾರೆ. ಈ ಯಜ್ಞಗಳ ಹವಿಸ್ಸೆಲ್ಲ ದೇವತೆಗಳಿಗೆ ಹೋಗುತ್ತಿತ್ತು ಮತ್ತು ಈ ಯಜ್ಞಯಾಗಾದಿಗಳೆಲ್ಲ ನಿಂತುಹೋದೊಡನೆ ದೇವತೆಗಳೆಲ್ಲ ಹಸಿವಿನಿಂದ ಕಂಗಾಲಾಗಿ ಸೊರಗಿ ಸಾಯುವ ಸ್ಥಿತಿಯಲ್ಲಿದ್ದರು; ಕಾರಣ, ಅವರು ತಮ್ಮ ಶಕ್ತಿಯನ್ನೆಲ್ಲಾ ಕಳೆದುಕೊಂಡಿದ್ದರು. ಆಗ ದೇವತೆಗಳೆಲ್ಲಾ ಸೇರಿ, "ಹೇಗಾದರೂ ಸರಿಯೇ ಈ ಮನುಷ್ಯನನ್ನು ನಾವು ಹುಟ್ಟಡಗಿಸಲೇಬೇಕು. ಅವನ ಪವಿತ್ರತೆಯ ತಾಪ ನಮ್ಮ ಜೀವನಕ್ಕೆ ಸಂಚಕಾರವಾಗಿದೆ" ಎಂದು ಸಮಾಲೋಚಿಸಿದರು. ದೇವತೆಗಳೆಲ್ಲಾ ಸೇರಿ ಒಂದು ಸಂಚನ್ನು ಹೂಡಿ ಬುದ್ಧದೇವನ ಬಳಿಗೆ ಬಂದು "ಹೇ ಸೌಮ್ಯ, ನಿನ್ನಲ್ಲಿ ನಮ್ಮದೊಂದು ಕೋರಿಕೆಯಿದೆ. ನಾವೊಂದು ಮಹಾಯಾಗವನ್ನು ಮಾಡಲು ಹೊರಟಿದ್ದೇವೆ. ಅಂದರೆ ನಾವೊಂದು ದೊಡ್ಡ ಅಗ್ನಿಕುಂಡವನ್ನು ಮಾಡಬೇಕಾಗಿದೆ. ಆ ಅಗ್ನಿಯನ್ನು ಪ್ರಜ್ವಲಿಸಲು ಒಂದು ಪವಿತ್ರ ಸ್ಥಾನಕ್ಕಾಗಿ ಇಡೀ ಪ್ರಪಂಚದಲ್ಲೆಲ್ಲಾ ಹುಡುಕಾಟ ನಡೆಸಿದರೂ ಅಂತಹ ಸ್ಥಾನ ನಮಗೆ ಕಾಣ ಸಿಗಲಿಲ್ಲ. ಕೊನೆಗೆ ಈಗ ಅಂತಹ ಸ್ಥಾನವೊಂದು ಪ್ರಾಪ್ತವಾಗಿದೆ. ನೀನೇನಾದರೂ ಅಂಗಾತ ಮಲಗಿದರೆ ನಿನ್ನ ಹೃತ್ಪ್ರದೇಶದ ಮೇಲೆ ನಾವು ದೊಡ್ಡ ಅಗ್ನಿಯನ್ನು ಪ್ರಜ್ವಲಿಸುತ್ತೇವೆ" ಎಂದು ನುಡಿದರು. ಅದಕ್ಕೆ ಬುದ್ಧದೇವನು "ಹಾಗೆಯೇ ಆಗಲಿ, ನಿಮ್ಮ ಯಾಗವನ್ನು ಆರಂಭಿಸಿ" ಎಂದನು. ಆಗ ಆ ದೇವತೆಗಳು ಆತನ ವಿಶಾಲಹೃದಯದ ಮೇಲೆ ಅಗಾಧವಾದ ಅಗ್ನಿಯನ್ನು ಪ್ರಜ್ವಲಿಸಿದರು. ದೇವತೆಗಳೆಂದುಕೊಂಡರು, ಬುದ್ಧದೇವನು ಸತ್ತನೆಂದು. ಆದರೆ ಅತ ಸತ್ತಿರಲಿಲ್ಲ. ಆಗ ದೇವತೆಗಳೆಲ್ಲಾ ಹತಾಶರಾಗಿ, "ಇನ್ನು ನಮ್ಮ ಕತೆ ಮುಗಿಯಿತು" ಎಂದು ಗೊಂದಲದಲ್ಲಿ ಮುಳುಗಿದರು. ನಂತರ ಏನುಮಾಡಲೂ ತೋಚದೆ, ಆ ದೇವತೆಗಳೆಲ್ಲಾ ಸೇರಿ ಆತನ ಮೇಲೆ ಪ್ರಹಾರಗಳ ಸುರಿಮಳೆಯನ್ನೇ ಆರಂಭಿಸಿದರು. ಅದರಿಂದಲೂ ಏನೂ ಪ್ರಯೋಜನವಾಗಲಿಲ್ಲ. ಅವರು ಅವನನ್ನು ಕೊಲ್ಲಲಾಗಲಿಲ್ಲ. ಆಗ ಆ ಅಗ್ನಿಕುಂಡದಿಂದ ಧ್ವನಿಯೊಂದು ಮೂಡಿ ಬರುತ್ತದೆ: "ನೀವೇಕೆ ಇಷ್ಟೆಲ್ಲಾ ವೃಥಾ ಪ್ರಯಾಸ ಮಾಡುತ್ತಿದ್ದೀರಿ?" ಆಗ ದೇವತೆಗಳು "ನಾವೇನು ಮಾಡುವುದು? ನಿನ್ನ ಪವಿತ್ರ ಮುಖಮುದ್ರೆಯನ್ನು ನೋಡಿದ ಯಾರೇ ಆಗಲಿ ಪೂತಾತ್ಮರಾಗಿ ಮುಕ್ತರಾಗಿ ಬಿಡುತ್ತಾರೆ. ನಮ್ಮ ಉಪಾಸನೆ ಮಾಡಿ ನಮಗೆ ಹವಿಸ್ಸನ್ನು ಕೊಡುವವರು ಯಾರೂ ಇಲ್ಲ" ಎಂದರು. ಆಗ ಆ ಧ್ವನಿ "ಹಾಗಾದರೆ ನಿಮ್ಮ ಶ್ರಮ ವ್ಯರ್ಥವೇ ಸರಿ. ಕಾರಣ, ಪವಿತ್ರತೆಯನ್ನು ಎಂದಿಗೂ ಕೊಲ್ಲಲಾಗುವುದಿಲ್ಲ" ಎನ್ನುತ್ತದೆ. ಈ ಪೌರಾಣಿಕ ದಂತಕಥೆ ಅವನಿಗೆ ಆಗದವರಿಂದ, ಅವನ ವೈರಿಗಳಿಂದ ಬರೆಯಲ್ಪಟ್ಟಿದೆ. ಆದರೂ ಬುದ್ಧನಿಗೆ ಅಂಟಿಸಲಾಗಿರುವ ಒಂದೇ ಒಂದು ಕಳಂಕವೆಂದರೆ, ಅವನೊಬ್ಬ ಪವಿತ್ರತೆಯ ಅದ್ಭುತವಾದ ಬೋಧಕನಾಗಿದ್ದನೆಂಬುದು.

ಆತನ ಸಿದ್ಧಾಂತಗಳ ಬಗ್ಗೆ ಸ್ವಲ್ಪಮಟ್ಟಿಗೆ ನಿಮ್ಮಲ್ಲಿ ಕೆಲವರಿಗೆ ಗೊತ್ತಿರಬಹುದು. ಅಜೇಯತಾವಾದಿಗಳೆಂದೆನಿಸಿಕೊಳ್ಳುವ ಈ ಆಧುನಿಕ ಚಿಂತನಶೀಲರನ್ನು ಆಕರ್ಷಿಸುವುದು ಆತನ ಮತದ ಸಿದ್ಧಾಂತಗಳೇ. ವಿಶ್ವಭ್ರಾತೃತ್ವದ ಮಹಾನ್ ಪ್ರತಿಪಾದಕನಾಗಿದ್ದ ಆತ, "ಆರ್ಯನಿಗಾಗಲೀ ಅಥವಾ ಆರ್ಯನಲ್ಲದವನಿಗಾಗಲೀ, ಜಾತಿಯಿರಲಿ ಅಥವಾ ಜಾತಿಹೀನನಾಗಿರಲಿ, ಸಂಪ್ರದಾಯಸ್ಥನಾಗಿರಲಿ ಅಥವಾ ಸಂಪ್ರದಾಯಶೂನ್ಯನಾಗಿರಲಿ – ಪ್ರತಿಯೊಬ್ಬರಿಗೂ ದೇವರ ಮೇಲೆ, ಧರ್ಮದ ಮೇಲೆ ಹಾಗೂ ಮುಕ್ತಿಯ ಮೇಲೆ ಸಮಾನವಾದ ಅಧಿಕಾರವಿದೆ. ಆದ್ದರಿಂದ ನೀವೆಲ್ಲರೂ ಬನ್ನಿರಿ" ಎಂದು ಕರೆಯಿತ್ತನು. ಆದರೆ ಮಿಕ್ಕೆಲ್ಲ ವಿಷಯದಲ್ಲಿ ಆತ ನಿಷ್ಟುರನಾದ ಅಜೇಯತಾವಾದಿಯಾಗಿದ್ದನು. ಆತನು ಸರ್ವದಾ "ವಾಸ್ತವಿಕ ಬುದ್ಧಿಸಂಪನ್ನರಾಗಿ" ಎಂದು ಉಪದೇಶಿಸುತ್ತಿದ್ದನು. ಒಮ್ಮೆ ಬ್ರಾಹ್ಮಣಸಂಜಾತರಾದ ಐದುಜನ ಯುವಕರು ಒಂದು ಸಮಸ್ಯೆಯ ಬಗ್ಗೆ ಕಿತ್ತಾಡುತ್ತಾ ಅವನ ಮೊರೆಹೊಕ್ಕರು. ಅವರೆಲ್ಲ ಬುದ್ಧನನ್ನು ಸತ್ಯಕ್ಕೆ ಮಾರ್ಗವಾವುದು ಎಂಬುದನ್ನು ಕೇಳಲು ಬಂದಿದ್ದರು. ಅವರಲ್ಲಿ ಒಬ್ಬನೆಂದನು “ನನ್ನ ಹಿರಿಯರು ಬೋಧಿಸಿರುವುದು ಇದನ್ನೇ. ಸತ್ಯಕ್ಕೆ ಇದೊಂದೇ ಮಾರ್ಗವಿರುವುದು.” ಮತ್ತೊಬ್ಬಾತ “ಆದರೆ ನನಗೆ ಬೇರೆ ರೀತಿಯಾಗಿ ಬೋಧಿಸಲಾಗಿದೆ ಮತ್ತು ಅದು ಮಾತ್ರವೇ ಸತ್ಯಕ್ಕಿರುವ ಏಕೈಕ ಮಾರ್ಗ” ಎಂದನು. ಹೀಗೆಯೇ ಒಬ್ಬೊಬ್ಬರೂ ನುಡಿದು, "ಹೇ ಮಹಾಶಯ! – ಇವುಗಳಲ್ಲಿ ಯಾವುದು ಸರಿಯಾದ ಮಾರ್ಗ?" ಎಂದು ಕೇಳಿದರು. ಹೀಗೆ ಅವರೆಲ್ಲರೂ ಬುದ್ಧನಲ್ಲಿ ಈ ರೀತಿಯ ತಮ್ಮ ಸಮಸ್ಯೆಯನ್ನು ಅರುಹಿದ ಮೇಲೆ ಬುದ್ಧನು "ಅದು ಸರಿ, ಇದೇ ಸತ್ಯ, ಇದೇ ಭಗವಂತನೆಡೆಗೆ ಇರುವ ಮಾರ್ಗ ಎಂದು ನಿಮ್ಮ ಹಿರಿಯರು ಒಬ್ಬೊಬ್ಬರಿಗೆ ಒಂದೊಂದು ಪಥವನ್ನು ನಿರ್ದೆಶಿಸಿದ್ದಾರೆಂದಲ್ಲವೇ ನೀವು ಹೇಳುವುದು!" ಎನ್ನುತ್ತಾ ಆ ಐವರಲ್ಲಿ ಒಬ್ಬನನ್ನು ಕುರಿತು "ನೀನು ದೇವರನ್ನು ನೋಡಿದ್ದೀಯಾ" ಎಂದನು. ಅದಕ್ಕಾತ “ಇಲ್ಲ” ಎನ್ನಲು, ಬುದ್ಧನು “ನಿಮ್ಮ ತಂದೆ?” ಎಂದನು. “ಇಲ್ಲ, ಆತನೂ ನೋಡಿಲ್ಲ” ಎಂದನು. “ನಿಮ್ಮ ತಾತ”, "ಇಲ್ಲ ಮಹಾಶಯ, ಆತನೂ ಕಂಡಿಲ್ಲ" ಎಂದನು. ಆಗ ಬುದ್ಧನು "ಅವರಾರೂ ಆ ದೇವರನ್ನು ಕಂಡಿಲ್ಲವೇ?" ಎಂದು ಕೇಳಿದಾಗ "ಇಲ್ಲ" ಎಂದನು. ಆಗ ಬುದ್ಧನು "ಸರಿ, ನಿಮ್ಮ ಬೋಧಕರು – ಅವರಲ್ಲಿಯೂ ಯಾರೂ ದೇವರನ್ನು ಕಂಡಿಲ್ಲವೇ?" ಎಂದನು. ಅದಕ್ಕಾತ, ಇಲ್ಲವೆಂದನು. ಇದೇ ರೀತಿಯ ಪ್ರಶ್ನೆಗಳನ್ನು ಮಿಕ್ಕೆಲ್ಲರಿಗೂ ಕೇಳಿದಾಗ ಅವರೆಲ್ಲರೂ ಸಹ ಯಾರೊಬ್ಬರೂ ದೇವರನ್ನು ಕಂಡೇ ಇಲ್ಲ ಎಂದು ದೃಢವಾಗಿ ಸಾರಿ ಹೇಳಿದರು. ಆಗ ಬುದ್ಧನು ಅವರಿಗೆಲ್ಲಾ ಒಂದು ದೃಷ್ಟಾಂತ ಕಥೆಯನ್ನು ಹೇಳಿದನು: ಒಂದಾನೊಂದು ಹಳ್ಳಿಗೆ ಒಬ್ಬ ಯುವಕ ಬಂದು 'ಅಯ್ಯೊ! ನಾನೆಷ್ಟು ಅವಳನ್ನು ಪ್ರೀತಿಸುತ್ತೇನೆ', 'ಅಯ್ಯೋ! ಅವಳನ್ನೆಷ್ಟು ಉತ್ಕಟವಾಗಿ ಪ್ರೀತಿಸುತ್ತೇನೆ' ಎಂದು ಪ್ರಲಾಪಿಸುತ್ತಾ ಹುಯಿಲಿಡುತ್ತಿದ್ದನು. ಅದನ್ನು ಕೇಳಿ ಹಳ್ಳಿಯವರೆಲ್ಲ ಬಂದು ಸೇರಿದರು. ಅವನು ಹೇಳುತ್ತಿದ್ದುದೊಂದೇ, ತಾನು ಅವಳನ್ನೆಷ್ಟು ಪ್ರೀತಿಸುತ್ತಾನೆಂದು. ಅವರೆಲ್ಲ, 'ನೀನು ಅಷ್ಟೊಂದು ಪ್ರೇಮಿಸಿದ ಆ ನಿನ್ನ ಪ್ರೇಯಸಿ ಯಾರು?' ಎಂದು ಕೇಳಿದರು. ಅದಕ್ಕವನು 'ನನಗೆ ಗೊತ್ತಿಲ್ಲ' ಎಂದನು. 'ಆಕೆ ಎಲ್ಲಿ ವಾಸವಾಗಿದ್ದಾಳೆ?' ಎಂದಾಗ, 'ಅದೂ ನನಗೆ ಗೊತ್ತಿಲ್ಲ' ಎಂದನು. ಹಳ್ಳಿಯವರು, 'ಹೋಗಲಿ, ಆಕೆ ನೋಡಲು ಹೇಗಿದ್ದಾಳೆ?' ಎಂದಾಗ, ಅದಕ್ಕೂ ಸಹ ಆತ 'ಇಲ್ಲ ಅದೂ ನನಗೆ ಗೊತ್ತಿಲ್ಲ. ಆದರೆ ಅವಳನ್ನು ನಾನು ಬಹಳವಾಗಿ ಪ್ರೀತಿಸಿದೆ' ಎಂದನು. ನಂತರ ಬುದ್ಧನು ಆ ಯುವಕರನ್ನು ಕುರಿತು, 'ಮಿತ್ರರೇ, ಹಾಗೆ ಪ್ರಲಾಪಿಸುತ್ತಿರುವ ಆ ತರುಣನ ಬಗ್ಗೆ ಏನೆನ್ನಬೇಕು?' ಎಂದನು. ಅವರೆಲ್ಲಾ 'ಅವನೊಬ್ಬ ಶುದ್ಧ ಮುಠ್ಠಾಳನೇ ಸರಿ! ತಾನು ಅವಳನ್ನು ಪ್ರೀತಿಸುತ್ತೇನೆ ಎಂದೆಲ್ಲಾ ಯಾರಿಗಾಗಿ ಪ್ರಲಾಪಿಸುತ್ತಿದ್ದಾನೋ, ಅವಳನ್ನು ಅವನೆಂದೂ ಕಂಡಿಲ್ಲ. ಅಂತಹವಳಿದ್ದಾಳೆಂದಾಗಲೀ, ಅವಳ ಬಗ್ಗೆಯಾಗಲೀ ಏನೂ ಗೊತ್ತಿಲ್ಲದಿರುವಾಗ ಅಂತಹವಳಿಗಾಗಿ ಅವನು ರೋದಿಸಬೇಕಾದರೆ ಅವನೊಬ್ಬ ಶುದ್ಧ ಮುಠ್ಠಾಳನಲ್ಲದೆ ಮತ್ತೇನು?' ಎಂದು ದೃಢವಾಗಿ ಹೇಳಿದರು. ಆಗ ಬುದ್ಧದೇವನು, "ನೀವೂ ಹಾಗೆಯೇ ಅಲ್ಲವೇ? ನೀವೇ ಹೇಳುವಂತೆ, ನಿಮ್ಮ ತಂದೆಯಾಗಲೀ, ನಿಮ್ಮ ತಾತನಾಗಲೀ ಯಾರನ್ನು ನೋಡಿಯೇ ಇಲ್ಲವೋ, ನಿಮಗಾಗಲೀ, ನಿಮ್ಮ ಪೂರ್ವಜರಿಗಾಗಲೀ, ಯಾರ ಬಗ್ಗೆ ಏನೂ ಗೊತ್ತಿಲ್ಲವೋ ಅಂತಹ ದೇವರಿಗೋಸ್ಕರ ಕಿತ್ತಾಡುತ್ತಾ ಪರಸ್ಪರ ಒಬ್ಬರನ್ನೊಬ್ಬರು ಬಡಿದುಹಾಕಲು ಉದ್ಯುಕ್ತರಾಗಿರುವ ನಿಮ್ಮನ್ನೇನನ್ನಬೇಕು!" ಎಂದಾಗ ಆ ಯುವಕರು "ಹಾಗಾದರೆ ನಾವೇನು ಮಾಡಬೇಕು?" ಎಂದು ಕೇಳಿದರು. ಬುದ್ಧನು "ಈಗ ಹೇಳಿ, ದೇವರು ಎಂದಾದರೂ ಕೋಪಿಸಿಕೊಳ್ಳುವನೆಂದು ನಿಮ್ಮ ತಂದೆ ಬೋಧಿಸಿರುವನೆ?" ಎಂದು ಕೇಳಿದನು. ಆಗ ಅವರೆಲ್ಲ 'ಇಲ್ಲ' ಎಂದರು. "ನಿಮ್ಮ ತಂದೆ, ಆ ದೇವರು ದುಷ್ಟ ಎಂದು ಎಂದಾದರೂ ಬೋಧಿಸಿದ್ದಾನೆಯೆ?" ಎಂದಾಗ ಅವರೆಲ್ಲ "ಇಲ್ಲ ಎಂದಿಗೂ ಇಲ್ಲ." ಆತನು ಯಾವಾಗಲೂ ಪರಿಶುದ್ಧ ಎಂದೇ ಬೋಧಿಸಿದ್ದಾರೆ" ಎಂದರು. ಆಗ ಬುದ್ಧನು "ಈಗ ನೀವೆಲ್ಲ ಹೀಗೆ ವಾದ ಮಾಡಿ ಕಿತ್ತಾಡುತ್ತಾ ಒಬ್ಬರನ್ನೊಬ್ಬರ ಬಡಿದಾಡಿಕೊಳ್ಳುವುದಕ್ಕಿಂತಲೂ, ಪರಿಶುದ್ಧರಾಗಿ, ಸನ್ಮಾರ್ಗಾವಲಂಬಿಗಳಾದರೆ, ಆ ದೇವರ ಸಮೀಪಕ್ಕೆ ನೀವು ಬರುವ ಸಂಭವ ಹೆಚ್ಚು ಎಂದೆನಿಸುವುದಿಲ್ಲವೆ? ಆದ್ದರಿಂದ ನಾನು ಹೇಳುವುದಿಷ್ಟೆ: ನೀವೆಲ್ಲ ಪರಿಶುದ್ಧರಾಗಿ, ಪರಿಶುದ್ಧವಾದ ಅಂತಃಕರಣದಿಂದ ಪ್ರತಿಯೊಬ್ಬರನ್ನೂ ಪ್ರೀತಿಸಿ, ಅಷ್ಟೇ ನೀವು ಮಾಡಬೇಕಾದುದು."

ಬುದ್ಧದೇವನ ಜನನಕಾಲದಲ್ಲಿ ಪ್ರಾಣಿಹತ್ಯೆ ಮಾಡದಿರುವುದು, ಹಾಗೂ ಸರ್ವಪ್ರಾಣಿಗಳಲ್ಲೂ ದಯೆ ತೋರುವುದು – ಈ ಬಗೆಯ ಸಿದ್ಧಾಂತ ಆಗಲೇ ಅಸ್ತಿತ್ವದಲ್ಲಿತ್ತು. ಆದರೆ ಜಾತಿನಿರ್ಮೂಲನಕ್ಕೋಸ್ಕರ ಒಂದು ಪ್ರಬಲವಾದ ಆಂದೋಲನವನ್ನೇ ಎಬ್ಬಿಸಿದ್ದು ಬುದ್ಧನಿಂದಾದ ಹೊಸ ಕೊಡುಗೆಯೆಂದೇ ಹೇಳಬೇಕು. ಮತ್ತೂ ಒಂದು ನಾವೀನ್ಯತೆಯೆಂದರೆ, ತನ್ನ ನಲವತ್ತು ಜನ ಶಿಷ್ಯರನ್ನು ಪ್ರೇರೇಪಿಸಿ ಅವರನ್ನೆಲ್ಲಾ ವಿಶ್ವದಾದ್ಯಂತ ಕಳುಹಿಸುತ್ತಾ, ಹೀಗೆ ಹೇಳಿದನು: "ನೀವೆಲ್ಲಾ ಎಲ್ಲೆಂದರಲ್ಲಿ ಹೋಗಿ, ಎಲ್ಲ ಜನಾಂಗದವರೊಡನೆ, ದೇಶದವರೊಡನೆ ಬೆರೆಯಿರಿ. ಎಲ್ಲರ ಹಿತಕ್ಕಾಗಿ ಎಲ್ಲರ ಸುಖಕ್ಕಾಗಿ ಈ ಉತ್ಕೃಷ್ಟವಾದ ಧರ್ಮವನ್ನು ಬೋಧಿಸಿ." ಇನ್ನು ಬುದ್ಧದೇವನಿಗಾದರೋ ಹಿಂದೂಗಳಿಂದ ಖಂಡಿತ ಉಪದ್ರವ ಉಂಟಾಗಲಿಲ್ಲ. ಆತ ತನ್ನ ಪೂರ್ಣಾಯುಷ್ಯ ಕಳೆದ ನಂತರವೇ ದೇಹತ್ಯಾಗ ಮಾಡಿದನು. ತನ್ನ ಜೀವನದಾದ್ಯಂತ ಆತನೊಬ್ಬ ಅತ್ಯಂತ ಕಟ್ಟುನಿಟ್ಟಾದ ವ್ಯಕ್ತಿಯಾಗಿದ್ದನು. ದೌರ್ಬಲ್ಯದೊಂದಿಗೆ ಅವನೆಂದೂ ರಾಜಿ ಮಾಡಿಕೊಂಡವನಾಗಿರಲಿಲ್ಲ. ಆತನ ಮತದ ಅನೇಕ ಸಿದ್ಧಾಂತಗಳಲ್ಲಿ ನನಗೆ ವಿಶ್ವಾಸವೇ ಇಲ್ಲ. ಅವುಗಳನ್ನು ಸರ್ವಥಾ ನಾನೊಪ್ಪುವುದೇ ಇಲ್ಲ. ಕಾರಣ, ಪುರಾತನ ಹಿಂದೂಗಳ ಆ ವೇದಾಂತ ತತ್ತ್ವವು ಅವನ ಸಿದ್ಧಾಂತಕ್ಕಿಂತಲೂ ಎಷ್ಟೋ ಆಲೋಚನಾಪರವಾದ, ಎಷ್ಟೊ ಭವ್ಯತರವಾದ ಬಾಳಿನ ದರ್ಶನವೊಂದನ್ನು ನಮ್ಮ ಮುಂದಿಟ್ಟಿದೆಯೆಂದೇ ನನ್ನ ಬಲವಾದ ನಂಬಿಕೆ. ಆದರೆ ಅವನ ಕಾರ್ಯವೈಖರಿಯನ್ನಂತೂ ನಾನು ಮೆಚ್ಚುತ್ತೇನೆ. ಆ ವ್ಯಕ್ತಿಯಲ್ಲಿ ನನ್ನನ್ನು ಬಹಳವಾಗಿ ಸೂರೆಗೊಳ್ಳುವ ಅಂಶವೆಂದರೆ, ಮನುಷ್ಯಜಾತಿಯ ಅವತಾರ ಪುರುಷರಲ್ಲೆಲ್ಲಾ ಈತನಲ್ಲಿ ಮಾತ್ರವೇ ಯಾವುದೇ ಗೋಜು ತೊಡಕುಗಳಿಲ್ಲದ ತಿಳಿಯಾದ ಮಸ್ತಿಷ್ಕವಿತ್ತು. ಆತ ತನ್ನ ಬುದ್ಧಿಯನ್ನು ತನ್ನ ಸ್ವಾಧೀನದಲ್ಲಿರಿಸಿಕೊಂಡಿದ್ದ ಅದ್ಭುತ ಶಕ್ತಿಸಂಪನ್ನನಾದ ವ್ಯಕ್ತಿಯಾಗಿದ್ದನು. ಸಾಮ್ರಾಜ್ಯಗಳೇ ಆತನ ಪದತಲದಲ್ಲಿದ್ದಾಗಲೂ ಸಹ, 'ಮನುಷ್ಯರಲ್ಲಿ ನಾನೂ ಒಬ್ಬ ಸಾಮಾನ್ಯ ಮನುಷ್ಯನಷ್ಟೇ!' ಎಂಬ ದೃಢವಾದ ನಿಲುವಿದ್ದ ಒಂದೇ ರೀತಿಯ ಮನುಷ್ಯನಾಗಿದ್ದನು.

ಹಿಂದೂಗಳಾದರೋ ಯಾರನ್ನಾದರೂ ಆರಾಧಿಸಲು ತುದಿಗಾಲಿನಲ್ಲಿ ನಿಂತಿರುತ್ತಾರೆ. ವ್ಯಕ್ತಿ–ಪೂಜೆಯಲ್ಲಿ ಅವರು ನಿಸ್ಸೀಮರು. ಅಷ್ಟೇಕೆ, ನೀವೇನಾದರೂ ಇನ್ನೂ ಬಹಳಕಾಲ ಬದುಕಿದ್ದರೆ, ನಮ್ಮ ಜನರು ನನ್ನನ್ನೇ ಆರಾಧಿಸುವುದನ್ನು ನೀವು ನೋಡುತ್ತೀರಿ. ನೀವು ಭಾರತಕ್ಕೆ ಹೋಗಿ ಧರ್ಮದ ಬಗ್ಗೆ ಏನಾದರೂ ಬೋಧಿಸಿದರೆ, ನೀವು ಸಾಯುವ ಮುನ್ನವೇ ನಿಮ್ಮನ್ನು ಪೂಜಿಸುವ ಜನ ಅವರು. ಯಾವಾಗಲೂ ಅವರು ಯಾರನ್ನಾದರೂ ಆರಾಧಿಸುವ ತವಕದಲ್ಲೇ ಇರುತ್ತಾರೆ. ಇಂತಹ ಜನರ ಮಧ್ಯೆ ಇದ್ದು, ವಿಶ್ವವಂದ್ಯನಾದ ಬುದ್ಧ ತಾನು ಸಾಮಾನ್ಯ ಮನುಷ್ಯನಷ್ಟೇ ಎಂದು ಘೋಷಿಸುತ್ತಲೇ ಕೊನೆಯುಸಿರೆಳೆಯುತ್ತಾನೆ. ಇತರೆ ಯಾವುದೇ ಮನುಷ್ಯನಿಗಿಂತಲೂ ತಾನು ಬೇರೆ ಎಂಬ ಒಂದೇ ಒಂದು ಹೇಳಿಕೆಯನ್ನಾಗಲೀ, ಸೂಚನೆಯನ್ನಾಗಲೀ ಆತನ ಬಾಯಿಂದ ಹೊರಪಡಿಸಲು, ಆತನ ಯಾರೊಬ್ಬ ಅನುಯಾಯಿಗಳಿಂದಲೂ ಸಾಧ್ಯವಾಗಲಿಲ್ಲ.

ಆತ ಪ್ರಾಣತ್ಯಾಗ ಮಾಡುವಾಗ ಆಡಿದ ಕೊನೆಯ ಮಾತುಗಳಂತೂ ಚಿರಂತನವಾಗಿ ನನ್ನ ಆಂತರದಲ್ಲಿ ತುಡಿಯುತ್ತಿರುತ್ತಿವೆ. ಅವನಿಗೆ ಸಾಕಷ್ಟು ವಯಸ್ಸಾಗಿತ್ತು. ರುಗ್ಣಾವಸ್ಥೆಯಲ್ಲಿದ್ದ ಆತನ ಅಂತ್ಯ ಸಮೀಪಿಸುತ್ತಿತ್ತು. ಅದೇ ಹೊತ್ತಿನಲ್ಲಿಯೇ, ಸತ್ತಪ್ರಾಣಿಗಳ ಕೊಳೆತ ಮಾಂಸ ತಿನ್ನುವ ಅಂತ್ಯಜನೊಬ್ಬ ಬರುತ್ತಾನೆ. ಈ ಅಂತ್ಯಜರನ್ನು ಹಿಂದೂ ಸಮಾಜದಲ್ಲಿ ಊರಿನೊಳಗೆ ಬರಲು ಬಿಡುವುದಿಲ್ಲ; ಅವರು ಊರಿನಾಚೆಯೇ ಇರುತ್ತಾರೆ. ಪಾಪ, ಚಂಡ ಎಂಬ ಈ ಅಂತ್ಯಜ ಬುದ್ಧದೇವನನ್ನು ತನ್ನ ಮನೆಗೆ ಊಟಕ್ಕೆ ಕರೆದೊಯ್ಯಲು ಬಂದಿದ್ದಾನೆ. ಆತನ ಆಮಂತ್ರಣದ ಮೇರೆಗೆ ಬುದ್ಧ ತನ್ನ ಶಿಷ್ಯರೊಡಗೂಡಿ ಅವನ ಮನೆಗೆ ಬರುತ್ತಾನೆ. ತನಗೆ ಉತ್ಕೃಷ್ಟ ಎಂದು ತೋಚಿದ ರೀತಿಯಲ್ಲಿ ಬುದ್ಧನನ್ನು ಸತ್ಕರಿಸುವುದು ಆ ಚಂಡನ ಆಸೆ. ಅಂತೆಯೇ ಅವನು ಯಥೇಷ್ಟವಾಗಿ ಹಂದಿಯ ಮಾಂಸವನ್ನು ಮತ್ತು ಅನ್ನವನ್ನು ಬುದ್ಧನಿಗೆ ಮತ್ತು ಅವನ ಶಿಷ್ಯರಿಗೆ ಬಡಿಸುತ್ತಾನೆ. ಬುದ್ಧದೇವನು ಆ ಆಹಾರದ ಕಡೆ ಒಮ್ಮೆ ಕಣ್ಣು ಹಾಯಿಸುತ್ತಾನೆ. ಆತನ ಶಿಷ್ಯರೆಲ್ಲರೂ ಇದನ್ನು ಹೇಗೆ ತಿನ್ನುವುದು ಎಂದು ಹಿಂದೇಟು ಹಾಕುತ್ತಿದ್ದಾಗ, ಬುದ್ಧದೇವ ಅವರಿಗೆ ಹೇಳುತ್ತಾನೆ: "ಈ ಆಹಾರವನ್ನು ನೀವೇನೂ ತಿನ್ನಬೇಕಾಗಿಲ್ಲ, ನಿಮಗಿದರಿಂದ ಹೇಸಿಗೆ ಉಂಟಾಗುತ್ತದೆ." ಆದರೆ ಬುದ್ಧನು ಮಾತ್ರ ಶಾಂತಚಿತ್ತದಿಂದ ಆ ಆಹಾರವನ್ನು ಕುಳಿತು ತಿನ್ನಲು ಉದ್ಯುಕ್ತನಾಗುತ್ತಾನೆ. ಸಮದರ್ಶಿತ್ವದ ಅಪೂರ್ವ ಆಚಾರ್ಯನೇ ಆಗಿದ್ದ ಆತ ಈ ಜಾತಿಬಾಹಿರ, ಅಂತ್ಯಜ ಚಂಡನ ಆಹಾರವನ್ನು ಸ್ವೀಕರಿಸಲೇಬೇಕು – ಹೌದು, ಹಂದಿಯ ಮಾಂಸವನ್ನೂ ಕೂಡ. ಆದ್ದರಿಂದ ಆತ ಕುಳಿತು ಊಟ ಮಾಡುತ್ತಾನೆ.

ಬುದ್ಧದೇವ ಆಗಲೇ ಮರಣಾಪನ್ನನಾಗಿದ್ದ. ಈಗಂತೂ ಮೃತ್ಯು ತೀರಾ ಸನ್ನಿಹಿತವಾಗಿದೆ. ಆಗ ಆತನ ಶಿಷ್ಯರಿಗೆ, "ಈ ಮರದಡಿಯಲ್ಲಿ ನನಗೆ ಮಲಗಲು ಏನನ್ನಾದರೂ ಹಾಸಿ, ಕಾರಣ ನನ್ನ ಅಂತ್ಯ ಬಂತೆಂದು ನನಗನ್ನಿಸುತ್ತಿದೆ" ಎಂದು ಹೇಳುತ್ತಾನೆ. ಆ ಮರದಡಿಯಲ್ಲಿ ಸ್ವಲ್ಪಹೊತ್ತು ಹಾಗೆಯೇ ಇದ್ದು ಮಲಗುತ್ತಾನೆ. ಕುಳಿತುಕೊಳ್ಳಲೂ ಸಹ ಆತನ ಕೈಯಿಂದ ಆಗುತ್ತಿಲ್ಲ. ಅವನು ಹಾಗೆಯೇ ಮೈಚಾಚಿದ ನಂತರ ಅವನು ನುಡಿದ ಮೊದಲ ಮಾತುಗಳೆಂದರೆ, "ಆ ಚಂಡನ ಬಳಿಗೆ ಹೋಗಿ ನನಗೆ ಉಪಕಾರವೆಸಗಿದವರಲ್ಲೆಲ್ಲಾ ಆತ ಅತ್ಯಂತ ಮಹದುಪಕಾರಿ ಎಂದು ತಿಳಿಸಿ. ಕಾರಣ ನನಗೆ ಆತನಿಂದ ಭಿಕ್ಷೆ ಪಡೆಯುವ ಸುಯೋಗವಿತ್ತು; ನಾನು ನಿರ್ವಾಣದತ್ತ ಸಾಗುತ್ತಿದ್ದೇನೆ." ನಂತರ ಆತನಿಂದ ಉಪದೇಶ ಪಡೆಯಲು ಕೆಲವರು ಬಂದರು. ಅವರಿಗೆ ಬುದ್ಧನ ಶಿಷ್ಯನೊಬ್ಬ – "ಪ್ರಭುವಿನ ಬಳಿಗೆ ಹೋಗಬೇಡಿ, ಆತ ಮಹಾಸಮಾಧಿಯಲ್ಲಿ ವಿಮಗ್ನನಾಗಿದ್ದಾನೆ" ಎಂದನು. ಈ ಮಾತುಗಳು ಬುದ್ಧನ ಕಿವಿಗೆ ಬಿದ್ದೊಡನೆಯೇ ಆತ ತನ್ನ ಶಿಷ್ಯನಿಗೆ, "ಅವರನ್ನು ತಡೆಯಬೇಡ, ಅವರು ಒಳಕ್ಕೆ ಬರಲಿ" ಎಂದು ಹೇಳುತ್ತಾನೆ. ನಂತರ ಮತ್ತೊಬ್ಬರಾರೋ ಬಂದಾಗ ಅವರನ್ನೂ ಸಹ ಶಿಷ್ಯರು ಒಳಗೆ ಬಿಡುವುದಿಲ್ಲ. ಆದರೆ ಈ ಬಾರಿಯೂ ಬುದ್ಧ ಅವರನ್ನು ತನ್ನ ಬಳಿ ಬರುವಂತೆ ಹೇಳುತ್ತಾನೆ. ಅವರೂ ಅವನ ಬಳಿಗೆ ಹೋಗುತ್ತಾರೆ. ನಂತರ ಬುದ್ಧ ತನ್ನ ಪ್ರಮುಖ ಶಿಷ್ಯನಾದ ಆನಂದನನ್ನುದ್ದೇಶಿಸಿ ಅಂತಿಮವಾಗಿ ಹೀಗೆ ಹೇಳುತ್ತಾನೆ: "ಹೇ ಆನಂದ, ನಾನು ಇಲ್ಲಿಂದ ನಿರ್ಗಮಿಸುತ್ತಿದ್ದೇನೆ, ನನಗೋಸ್ಕರ ಶೋಕಿಸಬೇಡ. ನನ್ನನ್ನು ಕುರಿತು ಚಿಂತೆ ಬೇಡ. ನಾನಂತೂ ತೆರಳಿದಂತೆಯೇ, ನಿಮ್ಮ ಮುಕ್ತಿಯನ್ನು ನೀವೇ ಮುತುವರ್ಜಿಯಿಂದ ಶ್ರಮಪಟ್ಟು ಸಾಧಿಸಿ. ನಿಮ್ಮಲ್ಲಿನ ಪ್ರತಿಯೊಬ್ಬರೂ ನನ್ನಷ್ಟೇ ಸಮರ್ಥರು. ನಾನೂ ಸಹ ನಿಮ್ಮಂತಹವನೇ ಒಬ್ಬನಲ್ಲದೇ ಬೇರೇನೂ ಅಲ್ಲ. ನಾನು ಇಂದು ಏನಾಗಿದ್ದೇನೋ ಅದು ನನ್ನನ್ನು ನಾನೇ ಸಾಧನೆಯಿಂದ ರೂಪಿಸಿಕೊಂಡಿದ್ದಷ್ಟೇ. ನೀವೂ ಸಹ ಹೋರಾಟ ಮಾಡಿ, ಸಾಧನೆ ಮಾಡಿ ನಿಮ್ಮನ್ನು ನೀವೇ ರೂಪಿಸಿಕೊಳ್ಳಿ..."

ಬುದ್ಧನ ಚಿರಸ್ಮರಣೀಯವಾದ ನುಡಿಮುತ್ತುಗಳು ಇವು: "ಪ್ರಾಚೀನ ಗ್ರಂಥವೊಂದನ್ನು ಪ್ರಮಾಣವಾಗಿ ಉದ್ಧೃತಗೊಳಿಸಲಾಗಿದೆಯೆಂದು ಯಾವುದನ್ನೂ ನಂಬಬೇಡಿ. ನಿಮ್ಮ ಪೂರ್ವಜರು ಹೇಳಿದ್ದಾರೆಂದ ಮಾತ್ರಕ್ಕೆ ಯಾವುದನ್ನೂ ನಂಬಬೇಡಿ. ಇತರರು ನಿಮ್ಮನ್ನು ಮೆಚ್ಚುತ್ತಾರೆಂಬ ಕಾರಣಕ್ಕಾಗಿ ಯಾವುದನ್ನೂ ನಂಬಬೇಡಿ. ಪ್ರತಿಯೊಂದನ್ನು ಪರೀಕ್ಷಿಸಿ. ಹೌದು, ಪ್ರತಿಯೊಂದನ್ನೂ ಪರೀಕ್ಷಿಸಿದ ನಂತರವೇ ಅದು ನಾಲ್ಕು ಜನಕ್ಕೆ ಒಳ್ಳೆಯದೆಂದು ಕಂಡುಬಂದರೆ ಅದನ್ನು ನಂಬಿ, ಅದನ್ನು ಎಲ್ಲರಿಗೂ ಬಿತ್ತರಿಸಿ." ಈ ಕೊನೆಯ ನುಡಿಗಳೊಂದಿಗೆ ಬುದ್ಧನ ದೇಹತ್ಯಾಗವಾಯಿತು.

ಆ ವ್ಯಕ್ತಿಯಲ್ಲಿದ್ದ ಅದ್ಭುತವಾದ ಬುದ್ಧಿಸ್ವಾಧೀನತೆಯನ್ನು ನೋಡಿ. ಯಾವುದೇ ದೇವತೆಗಳಾಗಲೀ, ದೇವದೂತರಾಗಲೀ, ದಾನವರಾಗಲೀ–ಯಾರೂ ಇಲ್ಲ. ಅಂತಹವರ ಯಾರ ಗೊಡವೆಯೂ ಆತನಿಗಿರಲಿಲ್ಲ. ಸಾವಿನ ಗಳಿಗೆಯಲ್ಲೂ ಸಹ ನಿರ್ದಾಕ್ಷಿಣ್ಯವಾಗಿ ತನ್ನ ಬುದ್ಧಿಯನ್ನು ತನ್ನ ಸ್ವಾಧೀನದಲ್ಲಿರಿಸಿಕೊಂಡಿದ್ದ. ಆತನ ಮಸ್ತಿಷ್ಕದ ಪ್ರತಿಯೊಂದು ಜೀವಕೋಶವೂ ಪರಿಪೂರ್ಣವಾಗಿ ಸಚೇತನವಾಗಿದ್ದುವು. ಯಾವುದೇ ಭ್ರಾಂತಿ–ಭ್ರಮೆಕಿಂಚಿತ್ತೂ ಇಲ್ಲ. ನನಗಂತೂ ಆತನ ಅನೇಕ ಸಿದ್ಧಾಂತಗಳೊಂದಿಗೆ ಸಹಮತವಿಲ್ಲ. ನಿಮಗೂ ಇಲ್ಲದಿರಬಹುದು. ಆದರೆ ಆತನಲ್ಲಿದ್ದ ಅದ್ಭುತ ಶಕ್ತಿಯ ಒಂದೇ ಒಂದು ಕಣವಾದರೂ ನನ್ನಲ್ಲಿದ್ದಿದ್ದರೆ ಎಂದೆನಿಸುತ್ತದೆ! ನಿಸ್ಸಂದೇಹವಾಗಿ ಜಗತ್ತು ಕಂಡ ಅಪ್ರತಿಮನಾದ ಶ್ರೇಷ್ಠ ದಾರ್ಶನಿಕ ಆತ! ಜಗತ್ತಿನ ಧರ್ಮಗುರುಗಳಲ್ಲೆಲ್ಲ ಅತ್ಯಂತ ವಿವೇಚನಾಶೀಲನಾದ, ಸರ್ವಶ್ರೇಷ್ಠನಾದ ಆಚಾರ್ಯ ಆತ. ಪುರೋಹಿತಶಾಹಿ ಮತ್ತು ಬ್ರಾಹ್ಮಣರ ದಬ್ಬಾಳಿಕೆಗೆ ಆತ ಎಂದೂ ಮಣಿಯಲಿಲ್ಲ. ಇಲ್ಲ, ಆ ವ್ಯಕ್ತಿ ಒಮ್ಮೆಯೂ ಅದಕ್ಕೆ ತಲೆಬಾಗಿದವನಲ್ಲ. ಎಲ್ಲೆಲ್ಲೂ ಒಂದೇ ಸಮನಾಗಿದ್ದ ವ್ಯಕ್ತಿ ಆತ. ದುಃಖಿಗಳೊಡನೆ ಅವರಿಗಾಗಿ ಮರುಗುತ್ತಾ ಅವರನ್ನು ಸಂತೈಸಿ ಅವರಿಗೆ ಸಹಾಯಹಸ್ತ ನೀಡುತ್ತಾ, ಹಾಡಿ ನಲಿಯುತ್ತಿರುವವರೊಡನೆ, ಹಿಗ್ಗಿ ಹಾಡುತ್ತಾ ಬಲಶಾಲಿಗೊಳೊಡನೆ ಅದ್ಭುತ ಶಕ್ತಿ ತೇಜಸ್ಸನ್ನು ಹೊರಸೂಸುತ್ತಾ ಎಲ್ಲೆಲ್ಲಿಯೂ ತನ್ನ ಬುದ್ಧಿಯನ್ನು ತನ್ನ ಸ್ವಾಧೀನದಲ್ಲಿರಿಸಿಕೊಂಡಿದ್ದ ದಕ್ಷ ವ್ಯಕ್ತಿಯಾಗಿದ್ದನು ಆತ.

ಇವೆಲ್ಲವೂ ದಿಟವೇನೋ ಹೌದು. ಆದರೆ ಈ ಅವನು ಎಷ್ಟೇ ಶೀಲಸಂಪನ್ನನಾಗಿದ್ದರೂ ಆತನ ಸಿದ್ಧಾಂತವಂತೂ ನನಗೆ ಅರ್ಥವಾಗುವುದಿಲ್ಲ. ನಿಮಗೆಲ್ಲ ಗೊತ್ತಿರುವ ಹಾಗೆ, ಆತ ಮನುಷ್ಯನಲ್ಲಿ ಆತ್ಮವಿದೆ ಎಂಬುದನ್ನು ನಿರಾಕರಿಸಿದನು. ನಾವು ಹಿಂದೂಗಳೆಲ್ಲ, ಮನುಷ್ಯನಲ್ಲಿ ಶಾಶ್ವತವಾದ, ನಿರ್ವಿಕಾರವಾದ, ಅನಂತವಾದ ಯಾವುದೋ ಒಂದು ನಿತ್ಯವಾದುದರಲ್ಲಿ ಅಚಲವಾದ ಶ್ರದ್ಧೆಯನ್ನಿಟ್ಟಿದ್ದೇವೆ. ಮನುಷ್ಯನಲ್ಲಿರುವ ಈ ನಿತ್ಯವಾದುದನ್ನೇ ನಾವು ಆತ್ಮವೆನ್ನುತ್ತೇವೆ – ಇದು ಅನಾದಿ, ಅನಂತ. ಅದೇರೀತಿ ಪ್ರಕೃತಿಯಲ್ಲೂ ಚಿರಸ್ಥಾಯಿಯಾಗಿ ಬ್ರಹ್ಮ ಎಂದು ಕರೆಯಲ್ಪಡುವ ಅನಾದಿ ಅನಂತವಾಗಿ ನಿಹಿತವಾಗಿರುವುದನ್ನು ನಾವು ನಂಬುತ್ತೇವೆ. ಬುದ್ಧನು ಈ ಎರಡನ್ನೂ ನಿರಾಕರಿಸಿದನು. ಯಾವುದೇ ಚಿರಸ್ಥಾಯಿಯಾಗಿರುವುದರ ಅಸ್ತಿತ್ವಕ್ಕೆ ಪ್ರಮಾಣವೇ ಇಲ್ಲವೆಂದು ಅವನೆಂದನು. ಎಲ್ಲವೂ ಪರಿವರ್ತನಶೀಲವಾಗಿರುವ ಒಂದು ಸಮಷ್ಟಿಯಷ್ಟೆ. ನಿರಂತರವಾಗಿ ಬದಲಾಗುತ್ತಿರುವ ಒಂದು ಚಿಂತಾ ಸಮೂಹವನ್ನೇ – ನಾವು ಮನಸ್ಸೆನ್ನುವುದು ಎಂದನು. ಬೆಂಕಿಯ ಪಂಜೊಂದು ನಿರಂತರವಾಗಿ ಸುತ್ತುತ್ತಿರುವಾಗ ಉಂಟಾಗುವ ಬೆಳಕಿನ ವೃತ್ತ ಒಂದು ಭ್ರಾಂತಿಯಷ್ಟೆ ಅಥವಾ ಒಂದು ನದಿಯ ಉದಾಹರಣೆಯನ್ನೇ ತೆಗೆದುಕೊಂಡರೆ ಅದೊಂದು ನಿರಂತರವಾಗಿ ಹರಿದು ಹೋಗುತ್ತಿರುವ ಜಲರಾಶಿ. ಪ್ರತಿಯೊಂದು ಕ್ಷಣದಲ್ಲೂ ಹೊಸ ನೀರು ಬಂದು ಹರಿದುಹೋಗುತ್ತಿರುತ್ತದೆ. ಇದೇ ಜೀವನ; ಇದೇ ಈ ಎಲ್ಲ ದೇಹ ಧರ್ಮ; ಅದೇರೀತಿ ಎಲ್ಲ ಮನಸ್ಸೂ ಸಹ.

ಏನೇ ಆಗಲಿ, ಅವನ ಸಿದ್ಧಾಂತವಂತೂ ನನಗೆ ಅರ್ಥವಾಗಿಲ್ಲ. ನಾವು ಹಿಂದೂಗಳು ಅವನ ಸಿದ್ಧಾಂತವನ್ನು ಎಂದೂ ಗ್ರಹಿಸಿಯೇ ಇಲ್ಲ. ಆದರೆ ಆ ಸಿದ್ಧಾಂತದಲ್ಲಿ ನಿಗೂಢವಾಗಿರುವ ಉದ್ದೇಶ ಮಾತ್ರ ನನಗರ್ಥವಾಗುತ್ತದೆ. ಹೋ! ಎಂತಹ ಒಂದು ಘನವಾದ, ಭವ್ಯವಾದ ಉದ್ದೇಶ ಅದು! ಬುದ್ಧ ದೇವನು ಹೇಳುತ್ತಿದ್ದುದೇನೆಂದರೆ, ಈ ಪ್ರಪಂಚದ ಏಕೈಕ ದೊಡ್ಡ ಅಭಿಶಾಪವೆಂದರೆ ಸ್ವಾರ್ಥಪರತೆಯೇ! ನಾವು ನಿಕೃಷ್ಟವಾದ ಸ್ವಾರ್ಥಜೀವಿಗಳು ಮತ್ತು ಆ ನಮ್ಮ ಸ್ವಾರ್ಥಪರತೆಯಲ್ಲಿ ಬಾಳಿನ ಎಲ್ಲ ಅಭಿಶಾಪವೂ ಅಡಗಿದೆ. ಸ್ವಾರ್ಥೋದ್ದೇಶ ಕಿಂಚಿತ್ತೂ ಇರಬಾರದು ಎಂಬುದೇ ನಮ್ಮ ಗುರಿಯಾಗಿರಬೇಕು. ನೀವೊಂದು ಹರಿಯುತ್ತಿರುವ ನೀರಿನಂತೆ – ದೇವರನ್ನಾಗಲೀ, ಆತ್ಮವನ್ನಾಗಲೀ, ಯಾವುದನ್ನೂ ಊರುಗೋಲನ್ನಾಗಿಟ್ಟುಕೊಳ್ಳಬೇಡಿ. ನಿಮ್ಮ ಕಾಲಮೇಲೇ ನೀವು ನಿಂತು, ಒಳ್ಳೆಯದನ್ನು ಅದು ಒಳ್ಳೆಯದು ಎಂಬ ಏಕೈಕ ಕಾರಣಕ್ಕಾಗಿಯೇ ಮಾಡಿ. ಹಾಗಲ್ಲದೇ, ಯಾವುದೊ ಶಿಕ್ಷೆಗೊಳಗಾಗುವ ಭಯಕ್ಕೋ ಅಥವಾ ಬೇರೆ ಯಾವುದಾದರೂ ಲೋಕಕ್ಕೆ ಹೋಗುವ ಆಸೆಯಿಂದ ಒಳ್ಳೆಯದನ್ನು ಮಾಡಬೇಕಾಗಿಲ್ಲ. ಎಂದೆಂದೂ ನಿಮ್ಮ ಬುದ್ಧಿಯನ್ನು ನಿಮ್ಮ ಕೈಯಲ್ಲಿಟ್ಟುಕೊಂಡು ಸ್ವಾರ್ಥಶೂನ್ಯರಾಗಿ, ಇದೇ ನಿಮಗಿರಬೇಕಾದ ಗುರಿ ಅಥವಾ ಉದ್ದೇಶ: ನಾನು ಸತ್ಕರ್ಮವನ್ನು ಮಾಡಬಯಸುತ್ತೇನೆ; ಕಾರಣ ಸತ್ಕರ್ಮವನ್ನು ಮಾಡುವುದು ಯಾವಾಗಲೂ ಒಳ್ಳೆಯದೇ. ಇದೊಂದು ಅದ್ಭುತವಾದ ಹಾಗೂ ಭವ್ಯವಾದ ನಿಲುವೇ ಸರಿ! ನನಗೆ ಆತನ ತಾರ್ಕಿಕ ವಿಚಾರಗಳೊಂದಿಗೆ ಕಿಂಚಿತ್ತೂ ಸಹಾನುಭೂತಿಯಿಲ್ಲ. ಆದರೆ ಆತ ಸುರಿಸಿದ ನೈತಿಕ ಶಕ್ತಿಯನ್ನು ನೆನೆದಾಗಲಂತೂ ನನ್ನ ಮನಸ್ಸು ಈರ್ಷ್ಯೆಗೊಳಗಾಗುತ್ತದೆ. ಆ ವ್ಯಕ್ತಿಯಂತೆ ನಿರ್ಭಿತರಾಗಿ, ಬರೀ ನೈತಿಕ ಶಕ್ತಿಯಿಂದಲೇ, ಸ್ವಾತ್ಮನಿರ್ಭರರಾಗಿ ನಿಮ್ಮಲ್ಲಿ ಯಾರು ಒಂದು ಗಂಟೆ ಇರಬಲ್ಲಿರಿ ಎಂದು ನಿಮ್ಮನ್ನು ನೀವೇ ಕೇಳಿಕೊಳ್ಳಿ. ನನಗಂತೂ ಐದು ನಿಮಿಷವೂ ಸಾಧ್ಯವಿಲ್ಲ. ಅಂಜುಬುರುಕರಂತೆ ಯಾವುದಾದರೂ ಆಧಾರಕ್ಕಾಗಿ ತಡಕಾಡುವವರು ನಾವು. ನಾವು ದುರ್ಬಲರು; ಊರುಗೋಲಿಲ್ಲದಾಗ ತತ್ತರಿಸುವ ಅಂಜುಬುರುಕರು ನಾವು. ನನಗಂತೂ ನೈತಿಕತೆಯ ಸಾಕಾರವೇ ಆದ ಆ ಅದ್ಭುತ ವ್ಯಕ್ತಿಯನ್ನು ಕುರಿತು ಚಿಂತಿಸುತ್ತಿದ್ದರೇನೇ ನನ್ನ ಹೃದಯದಲ್ಲಿ ಅನುರಾಗದ ತರಂಗಗಳೇಳುತ್ತವೆ. ಆ ನೈತಿಕ ತೇಜಸ್ಸಿನ ಹತ್ತಿರ ಸುಳಿಯಲೂ ಸಹ ನಮಗೆ ಸಾಧ್ಯವಿಲ್ಲ. ಆ ಅದ್ಭುತ ನೈತಿಕ ತೇಜಸ್ಸಿಗೆ ಸರಿ ಸಾಟಿಯಾದ ಯಾವುದನ್ನೂ ಈ ಜಗತ್ತು ಎಂದೂ ಕಂಡೇ ಇಲ್ಲ. ನಾನಂತೂ ಆ ಶಕ್ತಿಗೆ ಸರಿಸಮನಾದ ಮತ್ತೊಂದನ್ನು ಈವರೆಗೂ ನೋಡೇ ಇಲ್ಲ. ನಾವೆಲ್ಲ ಹುಟ್ಟು ಹೇಡಿಗಳು. ನಮ್ಮನ್ನು ನಾವು ಬಚಾಯಿಸಿಕೊಂಡರೆ ಸಾಕು – (ಇನ್ನೊಬ್ಬರ ಹಿತದ ಬಗ್ಗೆಯೂ ತಲೆ ಕೆಡಿಸಿಕೊಳ್ಳುವವರಲ್ಲ). ನಮ್ಮ ಅಂತರದಲ್ಲಿ ಭಯಂಕರ ಭೀತಿ ಕಾಡುತ್ತಿದೆ; ಭಯಂಕರವಾದ ಸ್ವಾರ್ಥೋದ್ದೇಶ ನಿರಂತರ ಮನೆಮಾಡಿದೆ. ನಮ್ಮ ಸ್ವಾರ್ಥವೇ ನಮ್ಮನ್ನು ಅತ್ಯಂತ ಕೀಳುದರ್ಜೆಯ ಹೇಡಿಗಳನ್ನಾಗಿ ಮಾಡಿದೆ; ನಮ್ಮ ಸ್ವಾರ್ಥವೇ ನಮ್ಮಲ್ಲಿನ ಭೀತಿ ಮತ್ತು ಹೇಡಿತನಕ್ಕೆ ದೊಡ್ಡ ಕಾರಣ. ಆದರೆ ಬುದ್ಧನ ನಿಲುವಿನ ಗಾಂಭೀರ್ಯವೇ ಬೇರೆ. "ಸತ್ಕರ್ಮವನ್ನು ಒಳ್ಳೆಯದೆಂಬ ಕಾರಣಕ್ಕಾಗಿಯೇ ಮಾಡಿ. ಅಷ್ಟು ಸಾಕು, ನಿರರ್ಥಕವಾದ ಪ್ರಶ್ನೆಗಳನ್ನು ಕೇಳದಿರಿ. ಒಂದು ಕಟ್ಟುಕತೆಯಿಂದಾಗಲೀ, ಒಂದು ಉಪಾಖ್ಯಾನದಿಂದಾಗಲೀ ಒಂದು ಮೂಢನಂಬಿಕೆಯಿಂದಾಗಲೀ ಒಳ್ಳೆಯದನ್ನು ಮಾಡಲು ಪುಸಲಾಯಿಸಲ್ಪಟ್ಟ ವ್ಯಕ್ತಿ – ಅವಕಾಶ ದೊರೆತೊಡನೆಯೇ ಕೆಟ್ಟದ್ದನ್ನು ಮಾಡುತ್ತಾನೆ. ಒಳ್ಳೆಯದಾಗಲೆಂಬ ಕಾರಣಕ್ಕಾಗಿ ಯಾರು ಒಳ್ಳೆಯದನ್ನು ಮಾಡುತ್ತಾನೋ – ಅವನೊಬ್ಬನೇ ನಿಜವಾಗಿಯೂ ಒಳ್ಳೆಯವನು. ಅದೇ ಆ ವ್ಯಕ್ತಿಯ ಚಾರಿತ್ರ್ಯವಾಗಿರುತ್ತದೆ."

ಒಮ್ಮೆ ಬುದ್ಧನನ್ನು "ಮನುಷ್ಯನಲ್ಲಿ ಶಾಶ್ವತವಾಗಿರುವುದು ಯಾವುದು?" ಎಂದು ಯಾರೋ ಕೇಳಿದಾಗ "ಆತನಲ್ಲಿದ್ದ ಪ್ರತಿಯೊಂದೂ, ಪ್ರತಿಯೊಂದೂ" ಎಂದನು. ಆದರೆ ಮನುಷ್ಯನಲ್ಲಿ ಅಕ್ಷಯವಾಗಿರುವುದಾದರೂ ಏನು? ದೇಹವಲ್ಲ, ಆತ್ಮವಲ್ಲ, ಆದರೆ ಚಾರಿತ್ರ್ಯ; ಮತ್ತು ಯುಗ ಯುಗಗಳಾದರೂ ಉಳಿಯುವುದು ಅದೊಂದೇ. ನಮ್ಮನ್ನಗಲಿ ಸಾವನ್ನಪ್ಪಿದವರೆಲ್ಲ ನಮಗಾಗಿ ಬಳುವಳಿಯಾಗಿ ಬಿಟ್ಟಿರುವುದು ಅವನ ಚಾರಿತ್ರ್ಯವೊಂದನ್ನೇ – ಅದೇ ಮನುಷ್ಯಜನಾಂಗದ ಅಮೂಲ್ಯವಾದ ಶಾಶ್ವತವಾದ ಸಂಪತ್ತು. ಈ ಚಾರಿತ್ರ್ಯಗಳೇ ಇತಿಹಾಸದುದ್ದಕ್ಕೂ ಯಾವತ್ತೂ ಕ್ರಿಯಾಶೀಲವಾಗಿ ತುಡಿಯುತ್ತಿರುವುದು. ಬುದ್ಧನ ಬಗ್ಗೆಯಾಗಲೀ, ನಾಜರೆತ್‌ನ ಜೀಸಸ್ ಬಗ್ಗೆಯಾಗಲೀ ಹೇಳುವುದಕ್ಕೇನಿದೆ? ಅವರ ಚಾರಿತ್ರ್ಯ ಮಹಿಮೆಯಿಂದಲೇ ಈ ಪ್ರಪಂಚ ಸತ್ತ್ವಪೂರ್ಣವಾಗಿದೆ. ಇದೊಂದು ಪ್ರಚಂಡಶಕ್ತಿನಿಹಿತವಾದ ಅಪೂರ್ವ ಸಿದ್ಧಾಂತವೇ ಸರಿ!

ಇನ್ನು ನಾವು ಈ ವ್ಯಕ್ತಿಯ ಔನ್ನತ್ಯದ ಧ್ಯಾನದಿಂದ ಸ್ವಲ್ಪ ವಾಸ್ತವಕ್ಕಿಳಿಯೋಣ – ಕಾರಣ ನಾವಿನ್ನೂ ಅಸಲು ವಿಷಯದ ಚರ್ಚೆಗೆ ಬಂದಿಲ್ಲ (ನಗುವಿನ ಹೊನಲು). ನಾನು ನಿಮಗೆ ಹೇಳಬೇಕಾಗಿರುವುದು ಇನ್ನೂ ಸಾಕಷ್ಟು ಇದೆ.

ಇನ್ನು ಬುದ್ಧನ ಕೆಲವು ಮಹತ್ಕಾರ್ಯಗಳ ಬಗ್ಗೆ. ಮೊದಲನೆಯದಾಗಿ ಆತನ ಕಾರ್ಯವೈಖರಿಯಲ್ಲಿ ಪ್ರಮುಖವಾದದ್ದು ಸಂಘಸ್ಥಾಪನೆ. ಚರ್ಚ್ ಬಗೆಗಿನ ಇಂದಿನ ನಿಮ್ಮ ವಿಚಾರಗಳೆಲ್ಲ ಆತನ ಚಾರಿತ್ರ್ಯದ ಕೊಡುಗೆಗಳೇ. ಆತನ ಚಾರಿತ್ರ್ಯದ ಮುದ್ರೆಯಿರುವ ಧರ್ಮಸಂಸ್ಥೆಯೊಂದನ್ನು ಆತ ಬಿಟ್ಟುಹೋದನು. ಆತ ಈ ಸಂನ್ಯಾಸಿಗಳನ್ನೆಲ್ಲಾ ಸಂಘಟಿಸಿ ಅವರಿಂದ ಒಂದು ಸಾಮೂಹಿಕ ಮಂಡಲಿಯೊಂದನ್ನು ಅಂದರೆ ಬೌದ್ಧ ಸಂಘವೊಂದನ್ನು ಅಸ್ತಿತ್ವಕ್ಕೆ ತಂದನು. ಆ ಸಂಘದಲ್ಲಿ – ಕ್ರಿ.ಪೂ. ೫೬೦ ವರ್ಷಗಳ ಮುಂಚೆಯೇ – ಗುರುತಿನ ಚೀಟಿಯ ಮೂಲಕ ಮತ ಚಲಾವಣೆ ಕೂಡ ಚಾಲನೆಯಲ್ಲಿತ್ತು. ಅಂದರೆ ಪ್ರತಿಯೊಂದು ಸಣ್ಣ ವಿಷಯದಲ್ಲೂ ಅದ್ಭುತ ಸಂಘಟನಾ ಕೌಶಲ್ಯ ಎದ್ದು ಕಾಣುತ್ತಿತ್ತು. ಸಂಘವೇ ಆತನ ಸ್ಥಾನದಲ್ಲಿ ಅವಶೇಷವಾಗಿತ್ತು ಮತ್ತು ಅದು ಪ್ರಚಂಡಶಕ್ತಿ ಹಾಗೂ ಅಧಿಕಾರದ ಕೇಂದ್ರವಾಯಿತು. ಈ ಸಂಘವೇ ಭರತವರ್ಷದ ಒಳಗೂ ಮತ್ತು ಹೊರಗೂ ಧರ್ಮಪ್ರಚಾರದ ಮಹತ್ ಕಾರ್ಯದಲ್ಲಿ ತೊಡಗಿತು. ಆತನ ತರುವಾಯ ಮುನ್ನೂರು ವರ್ಷಗಳ ನಂತರ ಅಂದರೆ ಕ್ರಿ. ಪೂ. ಇನ್ನೂರರಲ್ಲಿ ಸಾಮ್ರಾಟ್ ಅಶೋಕನ ಆಳ್ವಿಕೆ ಪ್ರಾರಂಭವಾಯಿತು. ನಮ್ಮ ಪಾಶ್ಚಾತ್ಯ ಇತಿಹಾಸಕಾರರೆಲ್ಲರೂ ಕರೆದಿರುವಂತೆ ಚಕ್ರವರ್ತಿಗಳಲ್ಲೆಲ್ಲಾ ಸರ್ವಶ್ರೇಷ್ಠ ದೈವೀಗುಣಸಂಪನ್ನನಾದ ಆತ ಬುದ್ಧನ ವಿಚಾರಗಳಿಗೆ ಸಂಪೂರ್ಣವಾಗಿ ಪರಿವರ್ತಿತನಾಗಿದ್ದನು. ಅಷ್ಟೇ ಅಲ್ಲ, ಆಗಿನ ಕಾಲದ ಚಕ್ರವರ್ತಿಗಳಲ್ಲೆಲ್ಲಾ ಸರ್ವಶ್ರೇಷ್ಠನಾದ ಸಾರ್ವಭೌಮನಾಗಿದ್ದನು. ಈತನ ತಾತ (ಬಿಂಬಸಾರನು) ಅಲೆಕ್ಸಾಂಡರನ ಸಮಕಾಲೀನನಾಗಿದ್ದನು ಮತ್ತು ಅಲೆಕ್ಸಾಂಡರನ ಸಮಯದಿಂದಲೇ ಭಾರತವು ಗ್ರೀಸ್‌ನೊಂದಿಗೆ ನಿಕಟವಾದ ಸಂಬಂಧವನ್ನು ಹೊಂದಿತ್ತು... ಪ್ರತಿದಿನವೂ ಮಧ್ಯ ಏಷ್ಯಾದಲ್ಲಿ ಕಾಣಸಿಗುತ್ತಿರುವ ಒಂದಲ್ಲ ಒಂದು ಶಿಲಾಶಾಸನ ಇದನ್ನು ದೃಢೀಕರಿಸುತ್ತದೆ. ಭರತವರ್ಷವಾದರೋ ಬುದ್ಧ, ಅಶೋಕ, ಈ ಎಲ್ಲರ ವಿಷಯವನ್ನೆಲ್ಲಾ ಮರತೇಬಿಟ್ಟಿತ್ತು. ಆದರೆ ಯಾರೂ ಓದಲಾಗದ ಪ್ರಾಚೀನ ಲಿಪಿಗಳನ್ನೊಳಗೊಂಡ ಶಿಲಾಶಾಸನಗಳು, ವಿಧವಿಧ ಆಕಾರದ ನಿಲುಗಂಬದ ಸಾಲುಗಳು ಅಲ್ಲಿತ್ತು. ಕೆಲವು ಹಳೆಯ ಮೊಗಲ್ ಚಕ್ರವರ್ತಿಗಳು ಇವುಗಳನ್ನು ಓದಿದವರಿಗೆ ಲಕ್ಷಾಂತರ ನಿಧಿಗಳನ್ನು ಕೊಡುತ್ತೇವೆಂದು ಸಾರಿದರು. ಆದರೂ ಯಾರೂ ಓದಲಾಗಿರಲಿಲ್ಲ. ಆದರೆ ಕಳೆದ ೩೦ ವರ್ಷಗಳಿಂದೀಚೆಗೆ ಅವುಗಳನ್ನು ಓದಲಾಗಿದೆ. ಅವೆಲ್ಲವೂ ಪಾಲೀ ಭಾಷೆಯಲ್ಲಿದೆ.

ಮೊದಲನೆಯ ಶಿಲಾಶಾಸನದ ಶೀರ್ಷಿಕೆ ಈ ರೀತಿಯಿದೆ: "...."

ತದನಂತರ ಯುದ್ಧದ ಘೋರ ಪರಿಣಾಮವನ್ನೂ, ದುಃಖದ ಗಾಥೆಯನ್ನೂ ವಿವರಿಸಿ, ತಾನು ಬೌದ್ಧ ಧರ್ಮಕ್ಕೆ ಮತಾಂತರಗೊಂಡಿದ್ದನ್ನು ಉಲ್ಲೇಖಿಸಿ, ಈ ಶಾಸನವನ್ನು ಬರೆಯುತ್ತಾ ಹೋಗುತ್ತಾನೆ. ಅನಂತರ ಹೀಗೆ ಬರೆಯುತ್ತಾನೆ: "ಇಲ್ಲಿಂದಾಚೆಗೆ ನನ್ನ ಸಂತತಿಯವರು ಯಾರೂ ಮತ್ತೊಂದು ಜನಾಂಗದವರನ್ನು ಯುದ್ಧದಲ್ಲಿ ಜಯಿಸಿ ಖ್ಯಾತಿ ಸಂಪಾದಿಸುವುದು ಬೇಡ. ಅವರಿಗೆ ಖ್ಯಾತಿ ಬೇಕಾಗಿದ್ದರೆ ಮತ್ತೊಂದು ಜನಾಂಗದವರಿಗೆ ಸಹಾಯ ಮಾಡಲಿ, ಧರ್ಮದ ಹಾಗೂ ವಿವಿಧ ಜ್ಞಾನವಿಜ್ಞಾನಗಳ ಬೋಧಕರನ್ನು ಅವರಲ್ಲಿಗೆ ಕಳುಹಿಸಲಿ, ಕತ್ತಿಯ ಝಳಪಿನಿಂದ ಸಂಪಾದಿಸಿದ ಯಶಸ್ಸು ಯಶಸ್ಸೇ ಅಲ್ಲ." ನಂತರ ಆತ ಅಲೆಕ್ಸಾಂಡ್ರಿಯಾದವರೆಗೂ ಕೂಡ ಧರ್ಮಪ್ರಚಾರಕರನ್ನು ಕಳುಹಿಸುವುದನ್ನು ನೀವು ನೋಡಬಹುದು... ಒಡನೆಯೇ ಆ ದೇಶದ ಥೆರಾಪುಟೇ, ಎಸ್ಸಿನ್ಸ್ ಎಂಬುವೇ ಮುಂತಾದ ಭಾಗಗಳಲ್ಲೆಲ್ಲಾ ತೀವ್ರ ಸಸ್ಯಾಹಾರಿಗಳ ಪಂಗಡಗಳು ತಲೆಯೆತ್ತುತ್ತಿದ್ದುದನ್ನು ನೀವು ಗಮನಿಸಿದರೆ ನಿಮಗೆ ಆಶ್ಚರ್ಯವೆನಿಸುತ್ತದೆ. ಇನ್ನು ಈ ಮಹಾಸಾಮ್ರಾಟ ಅಶೋಕನು ಜನರಿಗೆ ಹಾಗೂ ಪಶುಗಳಿಗೆ ಆಸ್ಪತ್ರೆಗಳನ್ನು ಕಟ್ಟಿಸಿದನು. ಈ ಶಿಲಾಲಿಪಿಗಳು ಹೇಗೆ ಜನರಿಗಾಗಿ ಮತ್ತು ಪಶುಗಳಿಗಾಗಿ ಆಸ್ಪತ್ರೆಗಳನ್ನು ಕಟ್ಟಲು ಅವರೆಲ್ಲ ಶಾಸನ ಮಾಡುತ್ತಿದ್ದರು ಎಂಬುದನ್ನು ತೋರುತ್ತವೆ. ಅಂದರೆ ಪಶುಗಳಿಗೆ ಮುದಿತನ ಪ್ರಾಪ್ತವಾದಾಗ ಇನ್ನು ಅವುಗಳನ್ನು ತಮ್ಮಲ್ಲಿಟ್ಟುಕೊಂಡು ಸಾಕಲಾಗದಿದ್ದರೆ ಅವುಗಳನ್ನು ದಯೆಯಿಂದ ಗುಂಡಿಟ್ಟು ಕೊಲ್ಲುತ್ತಿರಲಿಲ್ಲ. ಅವುಗಳಿಗೆ ಈ ಆಸ್ಪತ್ರೆಗಳಲ್ಲಿ ಆಶ್ರಯವಿರುತ್ತಿತ್ತು. ಈ ಆಸ್ಪತ್ರೆಗಳು ಸಾರ್ವಜನಿಕರ ದಾನದಿಂದ ನಿರ್ವಹಿಸಲ್ಪಡುತ್ತಿದ್ದುವು.

ಹಡಗು ಸಂಚಾರದ ಸಗಟು ವ್ಯಾಪಾರಿಗಳು ತಾವು ಮಾರಾಟ ಮಾಡುತ್ತಿದ್ದ ಸರಕಿನ ನೂರುಭಾಗಕ್ಕಿಷ್ಟು ಎಂದು ಶುಲ್ಕವನ್ನು ಕೊಡುತ್ತಿದ್ದರು. ಅವೆಲ್ಲಾ ಈ ಆಸ್ಪತ್ರೆಗಳಿಗೆ ಹೋಗುತ್ತಿತ್ತು. ಹೀಗಾಗಿ ಆ ಆಸ್ಪತ್ರೆಗಳು ಯಾರಮೇಲೂ ಹೊಣೆಯಾಗಿರಲಿಲ್ಲ. ಮುದಿಯಾದ ಹಸುವೋ ಅಥವಾ ಮತ್ತಾವ ಪ್ರಾಣಿಯೋ ಇದ್ದರೆ ಅದನ್ನು ಇನ್ನು ಸಾಕುವುದು ಬೇಡವೆಂದೆನಿಸಿದರೆ ಈ ಆಸ್ಪತ್ರೆಗೆ ಕಳುಹಿಸಲಾಗುತ್ತಿತ್ತು. ಆ ಆಸ್ಪತ್ರೆಗಳಲ್ಲಿ ಇಲಿಗಳ ಪರ್ಯಂತ ಏನನ್ನು ಕಳುಹಿಸಿದರೂ ಇಟ್ಟುಕೊಳ್ಳುತ್ತಿದ್ದರು. ನಿಮಗೆ ಗೊತ್ತಿರುವ ಹಾಗೆ ಗೃಹಕೃತ್ಯಗಳಲ್ಲಿ ತೊಡಗಿರುವ ಮಹಿಳೆಯರಷ್ಟೇ ಈ ಪ್ರಾಣಿಗಳನ್ನು ಸಾಯಿಸುತ್ತಾರೆ. ಎಲ್ಲ ರೀತಿಯ ಪಾಷಾಣಗಳನ್ನು ಪಡೆದು, ಆ ಪ್ರಾಣಿಗಳ ತಿನಿಸಿನಲ್ಲಿ ಸೇರಿಸಿದಾಗ, ಅದನ್ನು ತಿಂದು ಪ್ರಾಣಿಗಳು ಸಾಯುತ್ತವೆ. ಅಶೋಕನಾದರೋ ಮನುಷ್ಯನಂತೆಯೇ, ಈ ಪ್ರಾಣಿಗಳೂ ಸಹ ಸರ್ಕಾರದ ಸಂರಕ್ಷಣೆಯಲ್ಲಿರಬೇಕೆಂದು ಪ್ರತಿಪಾದಿಸುತ್ತಿದ್ದನು. ಪ್ರಾಣಿಗಳನ್ನು ಕೊಲ್ಲಲು ಏತಕ್ಕೆ ಅನುವು ಮಾಡಿಕೊಡಬೇಕು? ಯಾವುದೇ ಕಾರಣವೂ ಇಲ್ಲ. ಆತ ಹೇಳುತ್ತಿದ್ದುದೇನೆಂದರೆ ನಮ್ಮ ಆಹಾರಕ್ಕಾಗಿಯಾದರೂ ಪ್ರಾಣಿಹತ್ಯೆಯನ್ನು ನಿಷೇಧಿಸುವ ಮೊದಲು ಜನರಿಗೆ ಎಲ್ಲ ವಿಧದ ತರಕಾರಿಗಳನ್ನು ಒದಗಿಸಬೇಕೆಂದು. ಆದ್ದರಿಂದ ಆತ ತನ್ನ ಜನರನ್ನು ನಾಲ್ಕೂಕಡೆ ಕಳುಹಿಸಿ ಎಲ್ಲ ರೀತಿಯ ತರಕಾರಿಗಳನ್ನು ಸಂಗ್ರಹಿಸಿ ಅವುಗಳನ್ನು ಭಾರತದಲ್ಲಿ ಬೆಳೆಸಿದನು. ನಂತರ ಈ ರೀತಿಯ ಕ್ರಮವನ್ನು ತೆಗೆದುಕೊಂಡೊಡನೆಯೇ ಒಂದು ಆಜ್ಞೆಯನ್ನು ಹೊರಡಿಸಿದನು: ಇಲ್ಲಿಂದಾಚೆಗೆ ಯಾರು ಪ್ರಾಣಿಗಳನ್ನು ಕೊಲ್ಲುತ್ತಾರೋ ಅವರನ್ನು ಶಿಕ್ಷಿಸಲಾಗುತ್ತದೆ. ಸರ್ಕಾರ ತನ್ನ ಘನತೆಗೆ ತಕ್ಕಂತಿರಬೇಕು; ಪ್ರಾಣಿಗಳನ್ನು ಸಂರಕ್ಷಣೆ ಮಾಡಲೇಬೇಕು. ಒಂದು ಹಸುವನ್ನಾಗಲಿ, ಒಂದು ಮೇಕೆಯನ್ನೇ ಆಗಲಿ ಅಥವಾ ಯಾವುದೇ ಪ್ರಾಣಿಯನ್ನಾಗಲೀ ತನ್ನ ಆಹಾರಕ್ಕಾಗಿ ಕೊಲ್ಲಲು ಮನುಷ್ಯನಿಗೆ ಯಾವ ಅಧಿಕಾರವಿದೆ?

ಈ ರೀತಿ ಬೌದ್ಧ ಧರ್ಮ ರಾಜಕೀಯ ಕ್ಷೇತ್ರದಲ್ಲಿ ಒಂದು ಅಗಾಧವಾದ ಶಕ್ತಿಯಾಗಿ ತನ್ನ ಪ್ರಭಾವವನ್ನು ವಿಸ್ತೃತಗೊಳಿಸಿತ್ತು. ಆದರೆ ಕಾಲ ಕಳೆದಂತೆ ಈ ಪ್ರಚಂಡ ಧರ್ಮಪ್ರಚಾರದ ಒಂದು ಸಾಹಸಮಯ ಪ್ರಯತ್ನವೂ ಕೂಡ ಛಿದ್ರಛಿದ್ರವಾಯಿತು. ಆದರೆ ಧರ್ಮವನ್ನು ಪ್ರಚಾರಮಾಡಲು ಕತ್ತಿಯನ್ನೆಂದೂ ತೆಗೆದುಕೊಳ್ಳಲಿಲ್ಲ ಎಂಬುದರ ಕೀರ್ತಿ ಬೌದ್ಧರಿಗೇ ಸಲ್ಲಬೇಕು. ಇಡೀ ಜಗತ್ತಿನ ಧರ್ಮಗಳಲ್ಲೆಲ್ಲಾ – ಬೌದ್ಧ ಧರ್ಮವೊಂದನ್ನು ಬಿಟ್ಟರೆ – ರಕ್ತಪಾತವಿಲ್ಲದೆ ಒಂದೇ ಒಂದು ಹೆಜ್ಜೆಯನ್ನೂ ಸಹ ತಮ್ಮ ವ್ಯಾಪ್ತಿಯನ್ನು ವಿಸ್ತರಿಸುವಲ್ಲಿ ಯಶಸ್ವಿಯಾದ ಧರ್ಮ ಯಾವುದೂ ಇಲ್ಲ. ಬರೀ ಬೌದ್ಧಿಕ ತೇಜಸ್ಸಿನಿಂದಷ್ಟೇ ಲಕ್ಷಾಂತರ ಜನರನ್ನು ಮತಾಂತರಗೊಳಿಸಿ ತನ್ನ ತೆಕ್ಕೆಗೆ ಹಾಕಿಕೊಳ್ಳುವುದರಲ್ಲಿ ಯಶಸ್ವಿಯಾದ ಧರ್ಮ ಬೌದ್ಧ ಧರ್ಮವೊಂದೇ. ಇತಿಹಾಸದುದ್ದಕ್ಕೂ ಇದಕ್ಕೊಂದು ಅಪವಾದವನ್ನು ಸಹ ನೀವು ಕಾಣಲಾರಿರಿ. ಈಗಲೂ ಸಹ ನೀವು ಫಿಲಿಫೈನ್ಸ್ ದ್ವೀಪಪುಂಜಗಳಲ್ಲಿ ಮಾಡಹೊರಟಿರುವುದೂ ಈ ಬಲಪ್ರಯೋಗದಿಂದ ಮತಾಂತರವನ್ನೇ. ಅದು ನಿಮ್ಮ ಕಾರ್ಯವೈಖರಿ. ಕತ್ತಿಯ ಝಳಪಿನಿಂದ ಅವರನ್ನು ಧಾರ್ಮಿಕರನ್ನಾಗಿ ಮಾಡುವುದು! ನಿಮ್ಮ ಧರ್ಮಪ್ರಚಾರಕರು ಬೋಧಿಸುತ್ತಿರುವುದೂ ಅದನ್ನೇ. ಅವರನ್ನು ಮುತ್ತಿಗೆಹಾಕಿ, ಅವರನ್ನು ಸಾಯಿಸಿ – ಕೊನೆಗೆ ಹಾಗಾದರೂ ಅವರಿಗೆ ಧರ್ಮ ಪ್ರಾಪ್ತವಾಗಲಿ. ನಿಜವಾಗಲೂ ಧರ್ಮಪ್ರಚಾರದ ಒಂದು ಅದ್ಭುತ ವೈಖರಿಯೇ ಸರಿ!

ಅಶೋಕನು ಹೇಗೆ ಮತಾಂತರಗೊಂಡನೆಂಬುದೆಲ್ಲಾ ನಿಮಗೆ ಗೊತ್ತಿರಬೇಕು. ತನ್ನ ಯೌವನದಲ್ಲಿ ಈ ಸಾಮ್ರಾಟನು ಅಷ್ಟೇನು ಒಳ್ಳೆಯವನಾಗಿರಲಿಲ್ಲ. (ಆತನಿಗೊಬ್ಬ ಸಹೋದರನಿದ್ದ). ಈ ಸಹೋದರರಿಬ್ಬರೂ ಕಾದಾಡಿ ಈತ (ಅಶೋಕ) ಆತನ ಸಹೋದರನಿಂದಲೇ ಸೋಲಿಸಲ್ಪಟ್ಟನು. ಆ ಜಿದ್ದಿನಿಂದ ಈ ಸಾಮ್ರಾಟನು ಅವನನ್ನು ಕೊಲ್ಲಲು ಉದ್ಯುಕ್ತನಾದನು. ಈ ಅಶೋಕನಿಗೆ ತನ್ನ ಸಹೋದರ ಒಬ್ಬ ಬೌದ್ಧ ಭಿಕ್ಷುವಿನಲ್ಲಿ ಆಶ್ರಯ ಪಡೆದಿದ್ದಾನೆ ಎಂಬ ಸಮಾಚಾರ ಹೇಗೋ ತಲುಪಿತು. ನಮ್ಮಲ್ಲಿ ಸಂನ್ಯಾಸಿ ಅಥವಾ ಭಿಕ್ಷುಗಳನ್ನು ಹೇಗೆ ಪವಿತ್ರತಮರೆಂದು ಕಾಣಲಾಗುತ್ತದೆ ಎಂಬುದನ್ನು ನಿಮಗಾಗಲೇ ಹೇಳಿದ್ದೇನೆ; ಯಾರೂ ಸಹ ಅವರ ಸನಿಹದಲ್ಲಿಯೂ ಸುಳಿಯುವುದಿಲ್ಲ. ಸಮಾಚಾರ ತಲುಪಿದೊಡನೆಯೇ ಸಾಮ್ರಾಟನೇ ಸ್ವತಃ ಬಂದು, "ನಿನ್ನಲ್ಲಾಶ್ರಯ ಹೊಂದಿರುವ ಅವನನ್ನು ನನಗೆ ಒಪ್ಪಿಸು" ಎನ್ನುತ್ತಾನೆ. ಆಗ ಬೌದ್ಧ ಭಿಕ್ಷು ಆ ಸಾಮ್ರಾಟನಿಗೆ ಉಪದೇಶ ಕೊಡುತ್ತಾ “ಪ್ರತೀಕಾರ ಕೆಟ್ಟದ್ದು, ಕ್ರೋಧವನ್ನು ಪ್ರೀತಿಯಿಂದ ನಿಸ್ತೇಜಗೊಳಿಸಬೇಕು. ಕ್ರೋಧವೆಂದೂ ಕ್ರೋಧದಿಂದ ಶಮನವಾಗುವುದಿಲ್ಲ; ದ್ವೇಷವೆಂದೂ ದ್ವೇಷದಿಂದ ದೂರವಾಗುವುದಿಲ್ಲ. ನಿನ್ನ ಪ್ರೀತಿಯಲ್ಲಿ ಕ್ರೋಧವೆಲ್ಲ ಕರಗಿಹೋಗಲಿ, ನಿನ್ನ ಪ್ರೀತಿಯಿಂದ ದ್ವೇಷವೆಲ್ಲ ಶಮನವಾಗಲಿ, ಮಿತ್ರನೇ, ನಿನಗಾದ ಒಂದು ಕೇಡಿಗೆ ಮತ್ತೊಂದು ಕೇಡಿನಿಂದ ಪ್ರತೀಕಾರಗೊಳಿಸಲು ಯತ್ನಿಸಿದರೆ, ನಿನಗಾದ ಮೊದಲ ಕೇಡನ್ನು ನೀನು ಖಂಡಿತ ಶಮನಗೊಳಿಸಿದಂತಾಗುವುದಿಲ್ಲ. ಬದಲಾಗಿ ಪ್ರಪಂಚದಲ್ಲಿರುವ ಕೆಡುಕಿಗೆ ಮತ್ತೊಂದನ್ನು ಸೇರಿಸಿದಂತಾಗುತ್ತದಷ್ಟೇ!?” ಎಂದನು. ಆಗ ಆ ಸಾಮ್ರಾಟನು, "ಅದೆಲ್ಲಾ ಇರಲಿ, ನೀನಾದರೋ ಒಬ್ಬ ಮೂರ್ಖನೇ ಸರಿ. ಈಗ ಹೇಳು, ಆ ಮನುಷ್ಯನಿಗೋಸ್ಕರ ನಿನ್ನನ್ನೇ, ನಿನ್ನ ಪ್ರಾಣವನ್ನೇ ನೀನು ಬಲಿಕೊಡಲು ಸಿದ್ಧನಾಗಿರುವೆಯಾ?" ಎಂದನು. ಆ ಭಿಕ್ಷು "ನಾನಂತೂ ಸಿದ್ಧ" ಎಂದು ತನ್ನ ವಿಹಾರದಿಂದ ಹೊರಗೆ ಬಂದು ಅವನೆದುರಿಗೆ ನಿಂತನು. ಆಗ ಆ ಸಾಮ್ರಾಟನು ತನ್ನ ಝಳಪಿಸುವ ಖಡ್ಗವನ್ನು ಹೊರತೆಗೆದು, ಆ ಭಿಕ್ಷುವಿಗೆ, 'ಸಿದ್ದನಾಗು' ಎಂದು ಆದೇಶಿಸಿದನು. ಇನ್ನೇನು, ಆ ಬೌದ್ಧ ಭಿಕ್ಷುವಿನ ತಲೆಯುರುಳಿಸಲು ಆ ಸಾಮ್ರಾಟನು, ಖಡ್ಗ ಪ್ರಹಾರ ಮಾಡಬೇಕು ಎನ್ನುವ ಆ ಕ್ಷಣದಲ್ಲಿ ಆ ಬೌದ್ಧ ಭಿಕ್ಷುವಿನ ಮುಖದ ಕಡೆ ಒಮ್ಮೆ ನೋಡಿದನು. ಆ ಭಿಕ್ಷುವಿನ ಕಣ್ಣುಗಳಲ್ಲಿ ಎವೆಯಿಕ್ಕದ ಸ್ಥಿರತೆಯಿತ್ತು, ಶಾಂತವಾದ ಮುಖಮುದ್ರೆಯಿತ್ತು. ಆಗ ಅಶೋಕನು ತನ್ನ ದುಸ್ಸಾಹಸವನ್ನು ನಿಲ್ಲಿಸಿ ಹೇಳುತ್ತಾನೆ: "ಎಲೈ ಭಿಕ್ಷು, ದೀನ ಭಿಕಾರಿಯಾದ ನೀನು ಕಣ್ಣುರೆಪ್ಪೆಗಳೂ ಸಹ ಅಲುಗಾಡದಿರುವ ಇಂತಹ ಆತ್ಮಬಲವನ್ನು ಎಲ್ಲಿಂದ ಪಡೆದೆ?" ಆಗ ಆ ಬೌದ್ಧ ಭಿಕ್ಷು ಮತ್ತೆ ಆತನಿಗೆ ಉಪದೇಶ ಕೊಡುತ್ತಾ ಹೋಗುತ್ತಾನೆ. ಸಾಮ್ರಾಟನು ಆನಂದದಿಂದ "ನಿನ್ನ ಮಾತುಗಳು ಬಹಳ ಆಪ್ಯಾಯಮಾನವಾಗಿವೆ. ಮುಂದುವರಿಸು ನಿನ್ನ ಉಪದೇಶವನ್ನು" ಎಂದರಹುತ್ತಾನೆ. ಈ ರೀತಿಯಾಗಿ ಆತ ಪ್ರಭುವಿನ ಮಾಂತ್ರಿಕ ಪ್ರಭಾವಕ್ಕೆ ಅಂದರೆ ಬುದ್ಧನ ಅಪೂರ್ವ ಮಾಂತ್ರಿಕತೆಗೆ ಶರಣಾಗುತ್ತಾನೆ.

ಬೌದ್ಧ ಮತದಲ್ಲಿ ಮೂರು ಮುಖ್ಯವಾದ ಅಂಗಗಳಿವೆ: ಮೊದಲನೆಯದು ಸ್ವಯಂ ಬುದ್ಧದೇವನೇ, ಎರಡನೆಯದೇ ಆತನ ಶಾಸನ ಅಥವಾ ಧರ್ಮ ಮತ್ತು ಕೊನೆಯದಾಗಿ ಆತನ ಸಂಘ. ಆರಂಭದಲ್ಲಿ ಇದೆಲ್ಲ ತೀರಾ ಸರಳವಾಗಿಯೇ ಇತ್ತು. ಬುದ್ಧದೇವನ ಅವಸಾನಕ್ಕೆ ಸ್ವಲ್ಪ ಮುಂಚೆ, ಆತನ ಶಿಷ್ಯರು, “ಪ್ರಭು, ನಿಮ್ಮ (ಪವಿತ್ರ ಅಸ್ಥಿಯ)ನ್ನು ನಾವೇನು ಮಾಡಬೇಕು?” ಎಂದು ಕೇಳಿದರು. "ಏನೂ ಮಾಡಬೇಕಿಲ್ಲ" ಎಂಬ ಉತ್ತರ ಬಂದಿತು. "ನಿಮ್ಮ (ಪವಿತ್ರ ದೇಹದ) ಮೇಲೆ ಎಂತಹ ಸ್ಮಾರಕವನ್ನು ನಾವು ನಿರ್ಮಿಸಬೇಕು?" ಎಂದು ಮತ್ತೊಮ್ಮೆ ಕೇಳಿದಾಗ, ಬುದ್ಧದೇವನು "ನಿಮಗೆ ಹಾಗೆ ಬೇಕೆನಿಸಿದರೆ ಅಸ್ಥಿಯನ್ನೇ ರಾಶಿಮಾಡಿ ಸ್ತೂಪವನ್ನು ನಿರ್ಮಿಸಬಹುದು ಅಥವಾ ಏನೂ ಮಾಡದೇ ಇದ್ದರೂ ಆಯಿತು" ಎಂದನು. ಆದರೆ ಕಾಲಕ್ರಮದಲ್ಲಿ ವಿಶಾಲವಾದ ಮಂದಿರಗಳು, ಬುದ್ಧನ ಸ್ಮೃತಿ ಚಿಹ್ನೆಯಾಗಿ ನಿರ್ಮಿತವಾದ ದೊಡ್ಡ ದೊಡ್ಡ ಸ್ತೂಪಗಳು ಮತ್ತು ಎಲ್ಲ ರೀತಿಯ ದೀರ್ಘವಾದ ಪೂಜೋಪಚಾರಗಳು ಒಂದೊಂದಾಗಿ ತಲೆಯೆತ್ತಿದುವು. ಆಶ್ಚರ್ಯವೆಂದರೆ ಬೌದ್ಧಯುಗಕ್ಕಿಂತ ಮುಂಚೆ ಆರಾಧನೆಗಾಗಿ, ಪೂಜೆಗಾಗಿ ವಿಗ್ರಹಗಳ, ಮೂರ್ತಿಗಳ, ಬಳಕೆ ಯಾರಿಗೂ ತಿಳಿದೇ ಇರಲಿಲ್ಲ. ಮೊಟ್ಟ ಮೊದಲಿಗೆ ವಿಗ್ರಹಗಳನ್ನು ಬಳಕೆಗೆ ತಂದವರೇ ಬೌದ್ಧರು ಎಂದು ನಾನು ಖಚಿತವಾಗಿ ಹೇಳುತ್ತೇನೆ. ಎಲ್ಲ ವಿಧವಾದ ಬುದ್ಧನ ವಿಗ್ರಹಗಳೂ ಹಾಗೂ ಬುದ್ಧನ ಪದದಡಿ ಶ್ರಮಣರೆಲ್ಲರೂ ಕುಳಿತು ಪ್ರಾರ್ಥಿಸುತ್ತಿರುವ ವಿಗ್ರಹಗಳೆಲ್ಲವೂ ಇವೆ. ಈ ಬೌದ್ಧ ಸಂಘದೊಡನೆ ಈ ಪೂಜೋಪಚಾರಗಳೂ ಕಟ್ಟಳೆಗಳೂ ದಿನೇದಿನೇ ದ್ವಿಗುಣವಾಗುತ್ತಾ ಹೋದುವು. ನಂತರ ಈ ಬೌದ್ಧ ಸಂನ್ಯಾಸಿ ಅಥವಾ ಬೌದ್ಧ ಭಿಕ್ಷುಗಳಿದ್ದ ಬೌದ್ಧ ಮಠಗಳಲ್ಲೆಲ್ಲಾ ಹೇರಳವಾಗಿ ಹಣ, ಸಂಪತ್ತು ಶೇಖರವಾಯಿತು. ಅದರ ಅವನತಿಯ ಮೂಲ ಇದ್ದದ್ದು ಇಲ್ಲಿಯೇ. ಸಂನ್ಯಾಸ ಜೀವನವೇನೋ ಕೆಲವರಿಗಂತೂ ಎಲ್ಲ ರೀತಿಯಿಂದಲೂ ಬಹಳ ಒಳ್ಳೆಯದೇ. ಅದರಲ್ಲಿ ಸಂದೇಹವೇ ಇಲ್ಲ. ಆದರೆ ಯಾರೆಂದರೆ ಅವರು ಪ್ರತಿಯೊಬ್ಬ ನರ–ನಾರಿಯೂ ಸುಮ್ಮನೆ ಮನಸ್ಸಿಗೆ ಬಂದೊಡನೆಯೇ ಸಾಮಾಜಿಕ ಬದುಕನ್ನು ಬದಿಗೊತ್ತಿ, ಸಂಸಾರವನ್ನು ತ್ಯಜಿಸಿ ಸಂನ್ಯಾಸ ಜೀವನವನ್ನು ತೆಗೆದುಕೊಳ್ಳುವಂತೆ ಸಿಕ್ಕ ಸಿಕ್ಕವರಿಗೆಲ್ಲಾ ಉಪದೇಶ ಮಾಡಿದಾಗಲೇ ನಿಜವಾದ ತೊಂದರೆಯಾಗುವುದು. ಒಮ್ಮೆ ಭರತವರ್ಷದಲ್ಲೆಲ್ಲಾ ಬೌದ್ಧ ಮಠಗಳನ್ನೇ ನೋಡಬಹುದಾಗಿತ್ತು. ಈ ಮಠಗಳು ಹಾಗೂ ವಿಹಾರಗಳು ಎಷ್ಟು ವಿಶಾಲವಾಗಿದ್ದು ಭೀಮಾಕಾರವಾದ ಕಟ್ಟಡಗಳನ್ನು ಹೊಂದಿದ್ದುವೆಂದರೆ – ಒಂದೊಂದು ಮಠದಲ್ಲಿ ಒಮ್ಮೊಮ್ಮೆ ಇಪ್ಪತ್ತು ಸಾವಿರ ಬೌದ್ಧ ಭಿಕ್ಷುಗಳು, ಇನ್ನು ಕೆಲವೊಂದರಲ್ಲಿ ಶತಸಹಸ್ರ ಬೌದ್ಧ ಭಿಕ್ಷುಗಳು ವಾಸಿಸುತ್ತಿದ್ದರು. ಇಂತಹ ವಿಶಾಲವಾದ ಬೌದ್ಧ ಮಠಗಳು ಸಮಗ್ರ ಭಾರತದಲ್ಲಿ ಹರಡಿ ಹಂಚಿಹೋಗಿದ್ದುವು. ಅವುಗಳು ಅವಶ್ಯವಾಗಿ ಜ್ಞಾನಾರ್ಜನೆಯ ಕೇಂದ್ರಗಳಾಗಿದ್ದುವೆಂಬುದೇನೋ ನಿಜ. ಆದರೆ ಮುಂದಿನ ಪೀಳಿಗೆಯೆನಿಸುವ ತಮ್ಮ ಸಂತಾನವನ್ನು ಮುಂದುವರಿಸಲು ಸಮಾಜದಲ್ಲಿ ಉಳಿದ ವ್ಯಕ್ತಿಗಳಾದರೂ ಎಂಥವರು? ಕೇವಲ ದುರ್ಬಲರು, ಶಕ್ತಿಹೀನರು. ಸದೃಢವಾದ ತೇಜೋಬಲದಿಂದಿರುವ ವೀರ್ಯವಂತರೆಲ್ಲಾ ಸಮಾಜದಿಂದ ಹೊರಗೇ ಹೊರಟು ಹೋದದ್ದರಿಂದ ನಂತರ ಬರುತ್ತಿದ್ದ ಪೀಳಿಗೆಗಳೆಲ್ಲಾ ಸತ್ತ್ವಹೀನವಾದುವು. ತರುವಾಯ, ಹೀಗೆ ಉಂಟಾದ ಬರೀ ತೇಜೋಹೀನತೆಯಿಂದಲೇ ರಾಷ್ಟ್ರೀಯ ಅವನತಿ ಪ್ರಾರಂಭವಾಯಿತು.

ಇನ್ನು ಬೌದ್ಧ ಸಂಘದಲ್ಲಿನ ಅದ್ಭುತವಾದ ಸಂನ್ಯಾಸ–ಸಮುದಾಯದ ಬಗ್ಗೆ ಒಂದಿಷ್ಟು ಹೇಳಬೇಕು. ಅದು ನಿಜವಾಗಿಯೂ ಭವ್ಯವಾದುದೇ. ಆದರೆ ಚಿಂತನೆ ಮತ್ತು ಸಿದ್ಧಾಂತಗಳನ್ನು ರೂಪಿಸುವ ಮಾತು ಬೇರೆ; ಅದನ್ನು ವಾಸ್ತವಿಕತೆಯಲ್ಲಿ ಕೃತಿಗಿಳಿಸುವುದಿದೆಯಲ್ಲ ಅದೇ ಬೇರೆ. ಪ್ರತಿರೋಧವಿಲ್ಲದಿರುವಿಕೆ, ಅಂದರೆ ಯಾವುದನ್ನೂ ಪ್ರತಿಭಟಿಸದೆ ನುಂಗಿಕೊಳ್ಳಬೇಕು ಮುಂತಾದ ವಿಚಾರ, ಆದರ್ಶಗಳೆಲ್ಲಾ ಅಮೋಘವಾದುದೇನೋ ಹೌದು. ಆದರೆ ನಾವೆಲ್ಲರೂ ಅಪ್ರತಿಭಟನೆ ಇವೇ ಮುಂತಾದ ಆದರ್ಶಗಳನ್ನು ಕೃತಿಗಿಳಿಸಲು ತೀರ್ಮಾನಿಸಿ ರಸ್ತೆಗಿಳಿದರೆ ವಾಸ್ತವಿಕವಾಗಿ ಈ ನಗರದಲ್ಲಿ ಏನೂ ಉಳಿಯಲಾರದು. ಅಂದರೆ ಒಂದು ಸಿದ್ಧಾಂತವಾಗಿ, ತಾತ್ತ್ವಿಕವಾಗಿ ಈ ಆದರ್ಶಗಳೆಲ್ಲಾ ಸರಿಯೇ; ಆದರೆ ಅಂತಹ ಚರಮ ಆದರ್ಶ ಸ್ಥಿತಿಯನ್ನು ವಾಸ್ತವಿಕ ಪ್ರಪಂಚದಲ್ಲಿ ಹೇಗೆ ಪಡೆಯಬಹುದು ಎಂಬುದಕ್ಕೆ ಯಾರೂ ಕ್ರಿಯಾತ್ಮಕವಾದ ಪರಿಹಾರವನ್ನು ಇದುವರೆವಿಗೂ ಕಂಡುಹಿಡಿದಿಲ್ಲ.

ಈ ಜಾತಿ ಎಂಬುದರ ಅರ್ಥವು ಎಲ್ಲಿಯವರೆವಿಗೂ ರಕ್ತಪಾವಿತ್ರ್ಯತೆಯನ್ನು ಸಂರಕ್ಷಿಸುವುದೆಂದಾಗುತ್ತದೋ ಅಲ್ಲಿಯವರೆವಿಗೂ ಈ ಜಾತಿಪ್ರತಿಷ್ಠೆ ಅಥವಾ ಕುಲಪ್ರತಿಷ್ಠೆಯಲ್ಲಿ ಏನೋ ಒಂದು ಅರ್ಥವಿದೆ. ಆನುವಂಶಿಕತೆ ಎನ್ನುವುದಂತೂ ನಿಸ್ಸಂದೇಹವಾಗಿ ಇದ್ದೇ ಇದೆ. ಇದನ್ನು ಸ್ವಲ್ಪ ಅರ್ಥಮಾಡಿಕೊಳ್ಳಲು ಪ್ರಯತ್ನಿಸಿ. ನೀವೇಕೆ ನೀಗ್ರೋಗಳೊಂದಿಗೆ ಮತ್ತು ಅಮೆರಿಕನ್ ಇಂಡಿಯನ್ ರ ಜೊತೆ ರಕ್ತಸಂಪರ್ಕವನ್ನು ಮಾಡುವುದಿಲ್ಲ? ಪ್ರಕೃತಿಯೇ ನಿಮ್ಮನ್ನು ಹಾಗೆ ಮಾಡಲು ಬಿಡುವುದಿಲ್ಲ. ಪ್ರಕೃತಿಯ ಪ್ರತಿಬಂಧನೆಯಿಂದಲೇ ನೀವು ಅವರೊಂದಿಗೆ ಆ ರೀತಿ ರಕ್ತಸಂಬಂಧವನ್ನು ಇಟ್ಟುಕೊಳ್ಳಲಾರಿರಿ. ಸುಪ್ತಪ್ರಜ್ಞೆಯಲ್ಲಿರುವ ಯಾವುದೋ ಅಗೋಚರವಾದ ಶಕ್ತಿಗಳು ಕ್ರಿಯಾಶೀಲವಾಗಿದ್ದು ಈ ರೀತಿ ಜಾತಿ ಸಂರಕ್ಷಣೆಯಲ್ಲಿ ತೊಡಗಿರುತ್ತವೆ. ಅದೇ ಆರ್ಯರ ಜಾತಿಪದ್ಧತಿ. ಇತರರು (ಆರ್ಯರಲ್ಲದವರು) ನಮ್ಮ ಸಮಾನರಲ್ಲ ಎಂದು ನಾನು ಹೇಳುತ್ತಿಲ್ಲ ಎಂಬುದನ್ನು ಗಮನಿಸಿ. ಇತರರೂ ಅದೇರೀತಿಯ ಸವಲತ್ತುಗಳನ್ನೂ, ಎಲ್ಲ ರೀತಿಯ ಅನುಕೂಲಗಳನ್ನೂ ಹೊಂದಿರಲೇಬೇಕು; ಆದರೆ ಕೆಲವು ಜನಾಂಗ ಹಾಗೂ ವರ್ಣಗಳ ಅಬಾಧಿತವಾದ ಮಿಶ್ರಣದಿಂದ ಆ ವರ್ಣಗಳು ಸ್ವಲ್ಪಕಾಲ ಅವನತಿ ಹೊಂದುತ್ತವೆ ಎಂಬುದು ನಮಗೆ ಗೊತ್ತು. ಆರ್ಯರು ಮತ್ತು ಆರ್ಯರಲ್ಲದವರ ಮಧ್ಯೆ ಜಾತಿಯ ಕಠೋರವಾದ ಕಟ್ಟುಪಾಡುಗಳಿದ್ದಾಗ್ಯೂ, ಆ ಪ್ರತಿಬಂಧಕವಾದ ಗೋಡೆಯನ್ನು ಸ್ವಲ್ಪಮಟ್ಟಿಗೆ ಉರುಳಿಸಲಾಯಿತು. ಇದರ ಪರಿಣಾಮವಾಗಿ ವಿದೇಶೀಯರಾದ ವಿಲಕ್ಷಣ ಜಾತಿಯ ಅಲೆಮಾರಿ ತಂಡಗಳು ತಮ್ಮೆಲ್ಲ ಮೂಢನಂಬಿಕೆಗಳು, ವಿಚಿತ್ರವಾದ ಆಚಾರವ್ಯವಹಾರಗಳ ಪದ್ಧತಿಗಳೊಂದಿಗೆ ಬಂದರು. ಸ್ವಲ್ಪ ಯೋಚಿಸಿನೋಡಿ. ಬಟ್ಟೆಗಳನ್ನು ಹಾಕಿಕೊಳ್ಳುವಷ್ಟು ಮರ್ಯಾದೆಯೂ ಅವರಲ್ಲಿರಲಿಲ್ಲ. ಇವರೆಲ್ಲ ಮೃತ ಪ್ರಾಣಿಗಳ ಕೊಳೆತ ಮಾಂಸವನ್ನು ತಿನ್ನುವುದು ಇವೇ ಮುಂತಾದ ಕುಸಂಸ್ಕಾರಗಳಿಂದ ಕೂಡಿದವರಾಗಿದ್ದರು. ಅವರ ಹಿಂದೆಯೇ ಅವರ ಮೂಢನಂಬಿಕೆಗಳು, ಮೂಡಾರಾಧನೆಯ ಮಾಂತ್ರಿಕ ವಸ್ತುಗಳು, ನರಬಲಿ, ಅವರ ಪಿಶಾಚಿಗಳಲ್ಲಿನ ನಂಬಿಕೆ, ಪೈಶಾಚಿಕ ವರ್ತನೆ – ಎಲ್ಲವೂ ಬಂದವು. ಅವರು ಅವನ್ನೆಲ್ಲಾ ತಮ್ಮ ಹಿಂದೆಯೇ ಕಟ್ಟಿಕೊಂಡು ಕೆಲವು ವರ್ಷಗಳೇನೋ ಸಭ್ಯರಂತಿದ್ದರು. ಮೊದಲು ಮೊದಲು ಅವರ ಹೀನ ಆಚಾರಗಳೆಲ್ಲವನ್ನೂ ಸ್ವಲ್ಪ ಮುಚ್ಚುಮರೆಯಲ್ಲಿ ಮಾಡುತ್ತಿದ್ದರು. ಅನಂತರ ತಮ್ಮ ಗೊಬ್ಬು ಆಚಾರ–ವ್ಯವಹಾರಗಳು, ಮೂಢನಂಬಿಕೆಗಳು, ಪೈಶಾಚಿಕ ವರ್ತನೆ – ಇವುಗಳೆಲ್ಲವನ್ನೂ ಹೊರಗೆ ಎಲ್ಲರ ಮುಂದೆ ಯಾವುದೇ ನಾಚಿಕೆ ಸಂಕೋಚಗಳಿಲ್ಲದೆ ವ್ಯಕ್ತಪಡಿಸಲು ಮೊದಲು ಮಾಡಿದರು. ಇದು ಇಡೀ ಜನಾಂಗವನ್ನೇ ಅಧೋಗತಿಗೆ ತಂದಿತು ಇದಾದ ನಂತರ ರಕ್ತಗಳು ಬೆರೆತವು. ಯಾರ್ಯಾರೊಂದಿಗೆ ಬೆರೆಯಬಾರದೋ ಅಂತಹವರೊಂದಿಗೆ ಅಂತರ್ಜಾತೀಯ ವಿವಾಹಗಳು ನಡೆಯುತ್ತಿದ್ದುವು. ಇಡೀ ಜನಾಂಗವೇ ಹೀನ ಸಂಸ್ಕಾರದ ಬಲದಿಂದ ಕುಸಿದುಬಿತ್ತು. ಆದರೆ ಕಟ್ಟಕಡೆಯಲ್ಲಿ ಬಹಳ ದೂರದ ಭವಿಷ್ಯದಲ್ಲಿ ಅದರಿಂದ ಒಳ್ಳೆಯದೇನೋ ಸಾಧ್ಯವಾಯಿತು. ನೀವು ನೀಗ್ರೋಗಳೊಡನೆ ಮತ್ತು ಅಮೇರಿಕನ್ ಇಂಡಿಯನ್‌ರೊಡನೆ ಬೆರೆತರೆ, ಖಂಡಿತವಾಗಿ ನಿಮ್ಮ ನಾಗರಿಕತೆ, ಸಭ್ಯತೆ ಕುಸಿಯುತ್ತದೆ ನಿಜ. ಆದರೆ ನೂರಾರು, ಸಾವಿರಾರು ವರ್ಷಗಳ ನಂತರ, ಈ ಸಮ್ಮಿಶ್ರಣದ ಪರಿಣಾಮವಾಗಿ ಎಂದಿಗಿಂತಲೂ, ಒಂದು ಪ್ರಚಂಡ ಶಕ್ತಿಯಿಂದ ಕೂಡಿದ ಅದ್ಭುತವಾದ ಜನಾಂಗವೊಂದು ಮತ್ತೊಮ್ಮೆ ಉದ್ಭೂತವಾಗುತ್ತದೆ. ಆದರೆ ಸದ್ಯಕ್ಕಂತೂ ನೀವು ನೋವು–ಕಷ್ಟಗಳನ್ನನುಭವಿಸಲೇ ಬೇಕಾಗುತ್ತದೆ.

ಹಿಂದೂಗಳಲ್ಲಿ ಒಂದು ನಂಬಿಕೆಯಿದೆ: ಈ ಜಗತ್ತಿನಲ್ಲಿ ಸುಸಭ್ಯವಾದ ಜನಾಂಗವಿದ್ದದ್ದು ಒಂದೇ ಒಂದು – ಅದೇ ಆರ್ಯಜನಾಂಗ. ಇದೊಂದು ವಿಚಿತ್ರವಾದ ನಂಬಿಕೆ ಎಂದೇ ನನಗನ್ನಿಸಿದರೂ, ಅವರ ನಂಬಿಕೆಗೆ ವಿರುದ್ಧವಾಗಿ ನಾನೇನೂ ಹೇಳಲಾರೆ. ಕಾರಣ ಅವರ ನಂಬಿಕೆ ಸರಿಯಲ್ಲ ಎಂದು ತೋರಿಸುವ ಒಂದೇ ಒಂದು ಪ್ರಮಾಣವನ್ನೂ ನಾನಿದುವರೆವಿಗೂ ಕಂಡಿಲ್ಲ. ಈ ಆರ್ಯಜನಾಂಗದವನು ತನ್ನ ರಕ್ತವನ್ನು ದಾನಮಾಡುವವರೆವಿಗೂ ಇತರೇ ಯಾವುದೇ ಜನಾಂಗ ಸಭ್ಯ ಜನಾಂಗವಾಗಿರಲು ಸಾಧ್ಯವಿಲ್ಲ. ಈ ಸಂಸ್ಕರಣವನ್ನು ಯಾವುದೇ ಉಪದೇಶ, ಬೋಧನೆಯೂ ಮಾಡಲಾರದು. ಅತ್ಯಂತ ಆವಶ್ಯಕವಾಗಿ ಸಾರಭೂತವಾದದ್ದೆಂದರೆ, ಒಂದು ಜನಾಂಗ ಆರ್ಯನಿಂದ ರಕ್ತವನ್ನು ಪಡೆಯಬೇಕು. ನಂತರವೇ ಅದು ಸಭ್ಯತೆಯನ್ನು ತಲುಪುವುದು. ಬರೀ ಬೋಧನೆ, ಉಪದೇಶಗಳಿಂದ ಏನೂ ಪ್ರಯೋಜನವಾಗಲಾರದು. ಆರ್ಯರು ಯಾವುದೋ ದೇಶದಲ್ಲಾಗಲೀ ಸಭ್ಯತೆಗೇ ಒಂದು ನಿದರ್ಶನವಾಗಿರುತ್ತಾರೆ. ನಿಮ್ಮ ದೇಶದ ಮಾತೇ ತೆಗೆದುಕೊಂಡರೂ, ನೀವು ನೀಗ್ರೋ ಜನಾಂಗದೊಂದಿಗೆ ನಿಮ್ಮ ರಕ್ತವನ್ನು ಸಮ್ಮಿಶ್ರಗೊಳಿಸಲು ಸಮ್ಮತಿಸುತ್ತೀರಾ? ಆಗ ಉಚ್ಚತರವಾದ ಸಂಸ್ಕೃತಿಯನ್ನು ಅವರು ಪಡೆಯಲು ಸಾಧ್ಯ.

ಹಿಂದೂಗಳಿಗೆ ವರ್ಣವಿಭಜನೆ, ಜಾತಿ ವ್ಯವಸ್ಥೆಯ ಬಗ್ಗೆ ವಿಶೇಷವಾದ ಮಮತೆ. ನನ್ನಲ್ಲೂ ಸಹ ಆ ಜನಾಂಗದ ಈ ಮೂಢನಂಬಿಕೆಯ ಸೋಂಕು ಅಥವಾ ಕಲೆ ಇರಬಹುದೋ ಏನೋ ನನಗೆ ಗೊತ್ತಿಲ್ಲ. ನಾನಂತೂ ಬುದ್ಧದೇವನ ಆದರ್ಶವನ್ನು ಉತ್ಕಟವಾಗಿ ಪ್ರೀತಿಸುತ್ತೇನೆ. ಅದೊಂದು ಮಹಾನ್ ಆದರ್ಶ! ಆದರೆ, ನನ್ನ ಮಟ್ಟಿಗೆ ಹೇಳುವುದಾದರೆ, ಆ ಆದರ್ಶವನ್ನು ಕಾರ್ಯಾಚರಣೆಗೆ ತಂದರೀತಿ ಅಷ್ಟೊಂದು ಸೂಕ್ತವಾಗಿತ್ತು ಎಂದು ನನಗನ್ನಿಸುವುದಿಲ್ಲ. ಕಟ್ಟಕಡೆಯಲ್ಲಿ ಭಾರತದ ರಾಷ್ಟ್ರೀಯ ಅಧಃಪತನಕ್ಕೆ ಅದೇ ದೊಡ್ಡ ಕಾರಣಗಳಲ್ಲೊಂದಾಯಿತು. ಆದರೆ ಬೌದ್ಧ ಧರ್ಮದಿಂದ ಒಂದು ಅದ್ಭುತವಾದ ಜನಾಂಗೀಯ ಮಿಶ್ರಣವುಂಟಾಯಿತು. ಅನೇಕ ಜನಾಂಗಗಳು ಒಂದುಗೂಡಲು ಸಾಧ್ಯವಾಯಿತು. ವಿಭಿನ್ನ ಜನಾಂಗಳು ಒಂದುಗೂಡಿದಾಗ; ಬಿಳಿಯರು, ಕರಿಯರು ಮತ್ತು ವಿವಿಧ ಬಗೆಯ ಜನರು ತಮ್ಮೆಲ್ಲ ವೈಶಿಷ್ಟ್ಯ ಮತ್ತು ಆಚಾರವ್ಯವಹಾರಗಳೊಂದಿಗೆ ಬೆರೆತಾಗ ಅಂತಹ ಬೆಸುಗೆಯ ಪರಿಣಾಮವಾಗಿ ಕಾಲಕ್ರಮದಲ್ಲಿ ಖಂಡಿತವಾಗಿಯೂ ಒಂದು ಮಹಾ ವಿಪ್ಲವ ಉಂಟಾಗುತ್ತದೆ. ಆದರೆ ಸದ್ಯಕ್ಕಂತೂ ಈ ಮಹಾದೈತ್ಯ ನಿದ್ರಿಸಲೇಬೇಕು. ಅಂತಹ ಸಮ್ಮಿಶ್ರಣದ ಪರಿಣಾಮವೆಲ್ಲವೂ ಹಾಗೆಯೇ.

ಬೌದ್ಧಧರ್ಮ ಯಾವಾಗ ಈ ರೀತಿ ಅಧೋಮುಖವಾಗಿ ಸಾಗುತ್ತಿತ್ತೋ, ಅನಿವಾರ್ಯವಾದ ಪ್ರತಿಕ್ರಿಯೆಯೊಂದು ಭರತವರ್ಷದಲ್ಲಿ ಉಂಟಾಯಿತು. ಇಡೀ ಪ್ರಪಂಚದಲ್ಲೆಲ್ಲಾ ಇರುವುದು ಒಂದೇ ಸತ್ತೆ, ಸಮಗ್ರ ಜಗತ್ತೆಲ್ಲಾ ಈ ಒಂದು ಐಕ್ಯ ಸೂತ್ರದಿಂದ ವಿಕೃತವಾಗಿದೆ. ಈ ಜಗತ್ತು ಒಂದು ಏಕತೆಯ ಘಟಕವೇ ಸರಿ. ನಾನಾ ಅಥವಾ ಬಹುತ್ವ ಕೇವಲ ತೋರಿಕೆಗಷ್ಟೇ. ಇರುವುದೊಂದೇ. ನಾವು ಯಾವುದನ್ನು ದ್ವೈತವಿಹೀನವಾದ ಅದ್ವೈತ ಎಂದು ಕರೆಯುತ್ತೇವೋ ಅದು ಈ ಏಕತೆಯ ಭಾವನೆಯನ್ನೇ, ಇದೇ ಭಾರತೀಯರ ಭಾವನೆ. ಈ ಸಿದ್ಧಾಂತ ಭಾರತದಲ್ಲಿ ಯಾವಾಗಲೂ ಇತ್ತು. ಯಾವಾಗೆಲ್ಲಾ ಸಂದೇಹವಾದ, ಭೌತವಾದ ಮುಂತಾದುವು ಎಲ್ಲವನ್ನೂ ನಾಶಮಾಡುತ್ತಿದ್ದುವೋ, ಆಗೆಲ್ಲಾ ಈ ಸಿದ್ಧಾಂತ ಮುಂದೆ ಬಂದು ಕ್ರಿಯಾಶೀಲವಾಗಿ ಪರಿಸ್ಥಿತಿಯನ್ನು ಸುಧಾರಿಸುತ್ತದೆ. ಯಾವಾಗ ಬೌದ್ಧ ಮತದ ಮುಕ್ತದ್ವಾರದಿಂದ ಎಲ್ಲ ರೀತಿಯ ವಿದೇಶೀ ಬರ್ಬರರು, ಅವರ ವಿಚಿತ್ರ ಆಚಾರ ವ್ಯವಹಾರಗಳು, ಪದ್ಧತಿಗಳು, ಸ್ವೇಚ್ಛಾಚಾರಗಳೊಂದಿಗೆ – ಭಾರತದಲ್ಲಿ ಪ್ರವೇಶ ಮಾಡಿ ಭರತವರ್ಷದಲ್ಲಿದ್ದ ಸಾಮಾಜಿಕ ರಚನಾ ವಿನ್ಯಾಸವೆಲ್ಲವನ್ನೂ ಛಿದ್ರ ಛಿದ್ರಗೊಳಿಸಿದರೋ – ಆಗ ಭಾರತೀಯ ಸಮಾಜದೊಳಗಿಂದಲೇ ಒಂದು ವ್ಯಾಪಕವಾದ ಪ್ರತಿಕ್ರಿಯೆಯನ್ನು ನೋಡಬಹುದಿತ್ತು. ಈ ಪ್ರತಿಕ್ರಿಯೆಯ ಮುಂದಾಳುತನ ಒಬ್ಬ ತರುಣ ಸಂನ್ಯಾಸಿಯ ಕೈಯಲ್ಲಿತ್ತು – ಆತನೇ ಶಂಕರಾಚಾರ್ಯ. ಯಾವುದೇ ಹೊಸ ಸಿದ್ಧಾಂತಗಳನ್ನಾಗಲೀ ಬೋಧಿಸದೆ ಅಥವಾ ಯಾವುದೇ ಹೊಸ ಚಿಂತಾಪ್ರಣಾಲಿಯನ್ನಾಗಲೀ ಹೊಸ ಪಂಗಡಗಳನ್ನಾಗಲೀ ಮಾಡದೆ – ಆತ ವೈದಿಕ ಧರ್ಮವನ್ನೇ ಮತ್ತೊಮ್ಮೆ ವಾಸ್ತವಿಕ ಬದುಕಿನಲ್ಲಿ ಪುನರುಜ್ಜೀವನಗೊಳಿಸಿದನು. ಪರಿಣಾಮವಾಗಿಯೇ ಆಧುನಿಕ ಹಿಂದೂಧರ್ಮ, ತನ್ನ ಪ್ರಾಚೀನ ಹಿಂದೂಧರ್ಮವನ್ನು ಬೆರೆಸಿಕೊಂಡಿದ್ದಲ್ಲದೆ ಹೆಚ್ಚು ಹೆಚ್ಚು ವೇದಾಂತ ಪ್ರಧಾನವಾಗಿದೆ.

ಆದರೆ ಯಾವುದೇ ಆಗಲೀ ಒಮ್ಮೆ ಅಳಿದುಹೋಗುವುದೋ ಅದು ಮತ್ತೆಂದೂ ಪ್ರಾಣಶಕ್ತಿಯನ್ನು ಪಡೆದು ಬದುಕಿಗೆ ಹಿಂತಿರುಗಲಾರದು. ಅದೇರೀತಿ ಹಿಂದೂಧರ್ಮದಲ್ಲಿನ ಕರ್ಮಕಾಂಡದ ಈ ವಿಸ್ತಾರವಾದ ಯಜ್ಞಯಾಗಾದಿ ವಿಧಿಗಳು ಮತ್ತೆಂದೂ ಜೀವಂತವಾಗಿ ಹಿಂತಿರುಗಲಿಲ್ಲ. ಪುರಾತನ ವಿಧಿವಿಹಿತವಾದ ಯಜ್ಞಯಾಗಾದಿಗಳ ಪ್ರಕಾರ ಯಾರು ಗೋಮಾಂಸವನ್ನು ತಿನ್ನುವುದಿಲ್ಲವೋ ಅವನು ಕಟ್ಟುನಿಟ್ಟಿನ ಹಿಂದೂವೇ ಅಲ್ಲ ಎಂದು ನಾನು ನಿಮಗೆ ಹೇಳಿದರೆ ನಿಮಗೆ ಆಶ್ಚರ್ಯವಾಗಬಹುದು. ಕೆಲವು ಸಂದರ್ಭಗಳಲ್ಲಿ ಒಂದು ಎತ್ತನ್ನೋ ಅಥವಾ ಗೋವನ್ನೋ ಬಲಿಕೊಟ್ಟು ಅದನ್ನು ತಿನ್ನಬೇಕಿತ್ತು. ಆದರೆ ಇಂದಿನ ಹಿಂದೂವಿಗೆ ಅಂತಹುದು ಹೇಸಿಗೆಯೆನಿಸುತ್ತದೆ. ಭಾರತೀಯರಲ್ಲಿ, ಅವರವರ ಸಂಪ್ರದಾಯಗಳಲ್ಲಿ ಪರಸ್ಪರ ಎಷ್ಟೇ ಭಿನ್ನಾಭಿಪ್ರಾಯವಿದ್ದರೂ, ಒಂದು ವಿಷಯದಲ್ಲಂತೂ ಅವರೆಲ್ಲ ಒಂದೇ. ಅದೆಂದರೆ ಅವರೆಂದೂ ಗೋಮಾಂಸವನ್ನು ತಿನ್ನುವುದಿಲ್ಲ. ಪುರಾತನ ಯಜ್ಞಯಾಗಾದಿಗಳು ಮತ್ತು ಪುರಾತನ ದೇವತೆಗಳು ಎಂದೆಂದಿಗೂ ಹಿಂದಿರುಗಿ ಬಾರದಂತೆ ನಿರ್ಗಮಿಸಿದ್ದಾರೆ. ಆಧುನಿಕ ಭಾರತ ವೇದಗಳ ಅಧ್ಯಾತ್ಮ ಭಾಗಕ್ಕೆ ಸೇರಿದ್ದಾಗಿದೆ.

ಭಾರತದಲ್ಲಿ ಬೌದ್ಧ ಧರ್ಮವೇ ಮೊಟ್ಟಮೊದಲ ಮತೀಯ ಪಂಥವಾಗಿತ್ತು. “ಪರಿಪೂರ್ಣತೆಗೆ ಇರುವುದು ನಮ್ಮದೊಂದೇ ಮಾರ್ಗ. ನಮ್ಮ ಸಂಘವನ್ನು ಸೇರಿದ ಹೊರತು ನಿಮಗೆ ರಕ್ಷೆಯೇ ಇಲ್ಲ” – ಎಂದು ನುಡಿದವರಲ್ಲಿ ಇವರೇ ಮೊದಲಿಗರು. ಅವರು ಹೇಳುತ್ತಿದ್ದುದೊಂದೇ ಮಾತು: "ಬೌದ್ಧಮತವೊಂದೇ ಸರಿಯಾದ ಮಾರ್ಗ" ಎಂದು. ಆದರೆ ಅವರ ಧಮನಿಗಳಲ್ಲಿ ಹರಿಯುತ್ತಿದ್ದುದು ಹಿಂದೂರಕ್ತವಾದುದರಿಂದ ಇತರೆ ದೇಶಗಳಲ್ಲಿನ ಮತೀಯ ಪ್ರಚಾರಕರಷ್ಟು ಕಲ್ಲೆದೆಯವರಾಗಿರಲಿಲ್ಲ. ಇಲ್ಲ ಸಹಸಾ ಅವರು ಅನ್ಯದೇಶಗಳ ಮತಪ್ರಚಾರಕರ ಹಾಗಿರಲಿಲ್ಲ. ನಿಮಗೆ ಮುಕ್ತಿ, ವಿಮೋಚನೆ ಬಂದೇ ಬರುತ್ತದೆ. ಯಾರೂ ಸಹ ಶಾಶ್ವತವಾಗಿ ಅಡ್ಡದಾರಿಯಲ್ಲಿರಲಾರರು ಎನ್ನುತ್ತಿದ್ದ ಅವರಲ್ಲಿ ಯಥೇಷ್ಟವಾಗಿ ಹಿಂದೂರಕ್ತವೇ ಹರಿಯುತ್ತಿದ್ದುದರಿಂದ ಅವರ ಹೃದಯ ಕಲ್ಲಿನಂತಿರಲಿಲ್ಲ. ಆದರೆ ನೀವು ಅವರನ್ನು ಸೇರಲೇಬೇಕು.

ಆದರೆ ನಿಮಗೆ ಗೊತ್ತಿರುವ ಹಾಗೆ ಹಿಂದೂಗಳ ಭಾವನೆಯಾದರೋ ಯಾವುದೇ ಹೊಸ ಮತ ಪಂಥಗಳನ್ನು ಸೇರದಿರುವುದು. ನೀವೆಲ್ಲಿದ್ದರೂ ಅಲ್ಲಿಂದಲೇ, ಪರಿಧಿಯ ಮೇಲಿನ ಆ ಬಿಂದುವಿನಿಂದಲೇ ಗುರಿ ಅಥವಾ ಕೇಂದ್ರದೆಡೆಗೆ ಪಯಣ ಬೆಳೆಸಬಹುದು. ಆಗ ಎಲ್ಲವೂ ಸರಿಯಾಗಿರುತ್ತದೆ. ಈ ಹಿಂದೂಧರ್ಮದಲ್ಲಿ ಇದೇ ಒಂದು ದೊಡ್ಡ ಅನುಕೂಲ. ಈ ಹಿಂದೂಧರ್ಮದಲ್ಲಿನ ಒಂದು ಗುಟ್ಟೆಂದರೆ, ಸಿದ್ಧಾಂತಗಳೂ, ಮತವಾದಗಳೂ ಇವುಗಳೆಲ್ಲಾ ಮಹತ್ತ್ವವುಳ್ಳ ಅಂತಹ ಪ್ರಧಾನ ವಿಷಯಗಳಲ್ಲವೇ ಅಲ್ಲ. ಮುಖ್ಯವಾದದ್ದು ಜೀವನ. ಯಾರಾದರೂ ಈ ಜಗತ್ತಿನ ಸರ್ವಶ್ರೇಷ್ಠವಾದ ಮತವಾದ ಅಥವಾ ಸಿದ್ಧಾಂತಗಳೆಲ್ಲವನ್ನೂ ಕುರಿತು ಮಾತನಾಡಿದರೂ, ಅವರ ನಡೆ–ನುಡಿಗಳಲ್ಲಿ ಮೂರ್ಖರಂತಿದ್ದರೆ ಅವರ ಪಾಂಡಿತ್ಯ ಯಾವ ಕೆಲಸಕ್ಕೂ ಬಾರದು. ಆದರೆ ಯಾರು ತಮ್ಮ ನಡೆ–ನುಡಿಗಳಲ್ಲಿ ಶೀಲವಂತರಾಗಿರುತ್ತಾರೋ ಅವರಿಗೆ ಹೆಚ್ಚಿನ ಅವಕಾಶಗಳೂ, ಒಳ್ಳೆಯ ಭವಿಷ್ಯವೂ ಇದೆ. ವಸ್ತುಸ್ಥಿತಿ ಹೀಗಿರುವಾಗ ಪ್ರತಿಯೊಬ್ಬರ ವಿಕಾಸಕ್ಕಾಗಿಯೂ ವೇದಾಂತಿ ತಾಳ್ಮೆಯಿಂದ ಕಾಯಬಲ್ಲನು. ವೇದಾಂತದ ಮುಖ್ಯ ಪ್ರತಿಪಾದನೆಯೆಂದರೆ – ಇರುವುದು ಒಂದೇ ಸತ್ತೆ; ನಿತ್ಯವಾಗಿರುವುದು ಒಂದೇ ಒಂದು – ಮತ್ತು ಅದೇ ಈಶ್ವರ ಅಥವಾ ದೇವರು. ಅವನ ಎಲ್ಲೆಯನ್ನು ನಾವೆಂದೂ ನಿಖರವಾಗಿ ನಿಷ್ಕರ್ಷಿಸಲು ಸಾಧ್ಯವಿಲ್ಲ. ಅವನು ದೇಶ–ಕಾಲ–ಕಾರ್ಯಕಾರಣಗಳೆಲ್ಲದರ ಆಚೆ ಇರುವನು. ಅವನು ಸಚ್ಚಿದಾನಂದಸ್ವರೂಪ ಎಂಬುದರ ಹೊರತು ಅವನು ಏನಾಗಿದ್ದಾನೆಂದು ನಾವೆಂದೂ ಹೇಳಲಾರೆವು. ಅವನೊಬ್ಬನೇ ನಿತ್ಯ ಸತ್ಯ. ಪ್ರತಿಯೊಂದರಲ್ಲಿ – ನಿಮ್ಮಲ್ಲಿ, ನನ್ನಲ್ಲಿ ಅಷ್ಟೇಕೆ ಈ ಗೋಡೆಯಲ್ಲಿ, ಪ್ರತಿಯೊಂದರಲ್ಲಿಯೂ ಎಲ್ಲೆಲ್ಲೂ ನಿತ್ಯ ಸತ್ಯವಾಗಿ ಅನುಸ್ಯೂತನಾಗಿರುವವನು ಅವನೇ. ಅವನ, ಆ ಪರತತ್ತ್ವದ ಜ್ಞಾನದ ಮೇಲೆಯೇ ನಮ್ಮೆಲ್ಲರ ಸಮಷ್ಟಿ ಜ್ಞಾನ ಆಧರಿಸಿರುವುದು. ಅವನ ಪರಮಾನಂದದ ಮೇಲೆಯೇ ನಮ್ಮ ಸುಖ–ಸಂತೋಷಗಳೆಲ್ಲ ವಿಧೃತವಾಗಿರುವುದು. ಅವನೊಬ್ಬನೇ ಸತ್ಯ. ಮನುಷ್ಯ ಯಾವಾಗ ಇದರ ಅನುಭೂತಿಯನ್ನು ಪಡೆಯುತ್ತಾನೋ ಆಗ, 'ನಾನೊಬ್ಬನೇ ಸತ್ಯ. ನನ್ನಲ್ಲಿ ಸತ್ಯವಾಗಿರುವವನೂ ಅವನೇ. ನಾನೇ ಅವನು, ಸೋಽಹಂ' ಎಂಬುದನ್ನು ಅರಿಯುತ್ತಾನೆ. ಯಾವಾಗ ವ್ಯಕ್ತಿ ಪರಿಪೂರ್ಣವಾದ ಚಿತ್ತಶುದ್ದಿಯನ್ನೂ ಮತ್ತು ಸಂಸ್ಕಾರಗಳನ್ನೂ ಪಡೆದು ಸ್ಥೂಲವಾದುದೆಲ್ಲವನ್ನೂ ದಾಟಿರುತ್ತಾನೋ, ಆಗ ಜೀಸಸ್ ಹೇಳಿದಂತೆ "ನಾನು ಮತ್ತು ನನ್ನ ತಂದೆ ಒಂದೇ–ಏಕ ಹಾಗೂ ಅಭಿನ್ನ" ಎಂಬ ಸ್ಥಿತಿಯನ್ನು ತಲುಪುತ್ತಾನೆ. ವೇದಾಂತಿಗೆ ಪ್ರತಿಯೊಬ್ಬರಿಗಾಗಿ ಕಾಯುವ ತಾಳ್ಮೆ ಇದೆ. ನೀವೆಲ್ಲೇ ಇರಿ, ಸರ್ವೊಚ್ಚ ಆದರ್ಶವೆಂದರೆ, "ನಾನು ಮತ್ತು ನನ್ನ ತಂದೆಯೂ ಒಂದೇ" ಎನ್ನುವುದು. ಇದನ್ನು ಸಾಕ್ಷಾತ್ಕರಿಸಿಕೊಳ್ಳಿ. ಇದಕ್ಕೆ ಮೂರ್ತಿಪೂಜೆ ಸಹಾಯ ಮಾಡಿದರೆ ಮೂರ್ತಿಗಳಿಗೆ ಆಹ್ವಾನವಿದೆ. ಇಲ್ಲ, ಮಹಾಪುರುಷರ ಆರಾಧನೆ ನಿಮಗೆ ಅನುಕೂಲವೆಂದಾದರೆ ಅವರನ್ನು ಆರಾಧಿಸಿ. ಮೊಹಮ್ಮದ್‌ನ ಪೂಜೆ ನಿಮಗೆ ಸಹಾಯ ಮಾಡಿದರೆ ಹಾಗೆಯೇ ಮಾಡಿ. ಆದರೆ ಯಾವುದೇ ಅನುಷ್ಠಾನ ಕೈಗೊಂಡರೂ ಅದರಲ್ಲಿ ಪ್ರಾಮಾಣಿಕತೆಯಿರಲಿ, ಅಷ್ಟೇ. ನೀವು ಪ್ರಾಮಾಣಿಕರಾಗಿದ್ದರೆ, ನೀವು ಖಂಡಿತ ಗುರಿಯೆಡೆಗೆ ಬಂದೇ ಬರುತ್ತೀರೆಂದೇ ವೇದಾಂತಧರ್ಮ ಒತ್ತಿ ಹೇಳುವುದು. ಯಾರೂಕೂಡಾ ಕಡೆಗಣಿಸಲ್ಪಡುವುದಿಲ್ಲ. "ನಾನು ಮತ್ತು ನನ್ನ ತಂದೆಯು ಒಂದೇ" ಎಂಬ ಅಂತಿಮ ಸತ್ಯವನ್ನು ನೀವು ತಿಳಿಯುವವರೆಗೂ, ಸರ್ವಸತ್ಯವೂ ನಿಹಿತವಾಗಿರುವ ನಿಮ್ಮ ಹೃದಯವು ಅಧ್ಯಾಯದ ನಂತರ ಅಧ್ಯಾಯ ಬಿಚ್ಚಿಕೊಳ್ಳುತ್ತಾ ಹೋಗುತ್ತದೆ. ಅಲ್ಲದೆ ಮುಕ್ತಿ ಅಥವಾ ವಿಮೋಚನೆ ಎನ್ನುವುದಾದರೂ ಏನು? ಭಗವಂತನೊಡನೆ ಇರುವುದು ಎಂದಷ್ಟೇ. ಎಲ್ಲಿ? ಸರ್ವಸ್ಥಾನದಲ್ಲಿಯೂ, ಸರ್ವಕಾಲದಲ್ಲಿಯೂ, ಈ ವರ್ತಮಾನ ಕ್ಷಣವನ್ನೇ ತೆಗೆದುಕೊಳ್ಳಿ. ಅನಂತ ಕಾಲದಲ್ಲಿ ಯಾವುದೇ ಒಂದು ಕ್ಷಣವೂ, ಯಾವುದೇ ಮತ್ತೊಂದು ಕ್ಷಣದಷ್ಟೇ ಅಮೂಲ್ಯವಾದದ್ದು. ನೀವೇ ಗಮನಿಸಬಹುದಾದಂತೆ, ಇದೇ ವೇದಗಳ ಪ್ರಾಚೀನ ಸಿದ್ಧಾಂತ, (ಬೌದ್ಧ ಧರ್ಮದ ಪ್ರಭಾವದಿಂದ) ಇದು ಪುನರುಜ್ಜೀವವಾಯಿತು. ಬೌದ್ಧ ಧರ್ಮ ಭಾರತದಿಂದ ನಶಿಸಿಹೋಯಿತು ನಿಜ. ಆದರೆ ಅದು ಭರತಖಂಡದ ಜನ ಜೀವನದಲ್ಲಿ ದಾನಪರತೆ, ಪ್ರಾಣಿದಯೆ ಇವೇ ಮುಂತಾದ ಅಳಿಸಲಾಗದ ಸತ್ ಸಂಸ್ಕಾರಗಳ ಚಿರಮುದ್ರೆಯನ್ನೊತ್ತಿ ಹೋಗಿದೆ. ವೇದಾಂತ ಧರ್ಮವೇ, ಭರತಖಂಡದ ಒಂದು ತುದಿಯಿಂದ ಮತ್ತೊಂದು ತುದಿಯವರೆಗೂ, ಮತ್ತೊಮ್ಮೆ ತನ್ನ ದಿಗ್ವಿಜಯದ ಕಹಳೆಯನ್ನು ಮೊಳಗುತ್ತಿದೆ.

\newpage

\chapter[ಭಾವನೆಯ ಹನಿಗಳು]{ಭಾವನೆಯ ಹನಿಗಳು\protect\footnote{\engfoot{C.W, Vol. III, P.511}}}

\begin{center}
(ಸ್ವಾಮಿಜಿ ಅವರು ಮದ್ರಾಸಿನಲ್ಲಿ ನಡೆದ ಮಾತುಕತೆಗಳ ಸಂಗ್ರಹ. ಕಾಲ ೧೮೯೨ – ೩)
\end{center}

ಹಿಂದೂಧರ್ಮದ ಮೂರು ಮುಖ್ಯವಾದ ನಂಬಿಕೆಗಳೆಂದರೆ, ದೇವರಲ್ಲಿ ನಂಬಿಕೆ, ವೇದದಲ್ಲಿ ನಂಬಿಕೆ, ಕರ್ಮದಲ್ಲಿ ನಂಬಿಕೆ.

ವೇದವನ್ನು ವಿಶದವಾಗಿ ಓದಿದರೆ ಅಲ್ಲಿ ಒಂದು ಸಾಮರಸ್ಯವಿರುವುದು ನಮಗೆ ಕಾಣುವುದು, ಹಿಂದೂಧರ್ಮದಲ್ಲಿ ಒಬ್ಬ ಒಂದು ಸತ್ಯದಿಂದ ಮತ್ತೊಂದು ಸತ್ಯಕ್ಕೆ ಪ್ರಯಾಣ ಮಾಡುವನು. ಸಣ್ಣ ಸತ್ಯದಿಂದ ದೊಡ್ಡ ಸತ್ಯದೆಡೆಗೆ ಪ್ರಯಾಣ ಮಾಡುವನು. ಅವನೆಂದಿಗೂ ಅಸತ್ಯದಿಂದ ಸತ್ಯಕ್ಕೆ ಹೋಗುವುದಿಲ್ಲ.

ವೇದವನ್ನು ನಾವು ವಿಕಸನದ ದೃಷ್ಟಿಯಿಂದ ಓದಬೇಕಾಗಿದೆ. ಧರ್ಮವು ಏಕತೆಯಲ್ಲಿ ಪರಿಪೂರ್ಣತೆಯನ್ನು ಪಡೆಯುವ ತನಕ ಅದರಲ್ಲಿ ಆದ ಬೆಳವಣಿಗೆಯ ಇಡೀ ಇತಿಹಾಸವೇ ಅಲ್ಲಿದೆ. ವೇದಗಳು ಅನಾದಿ. ಸಾಧಾರಣವಾಗಿ ಜನರು ತಪ್ಪು ತಿಳಿದುಕೊಂಡಿರುವಂತೆ ಅಲ್ಲಿರುವ ಪದಗಳಲ್ಲ ಅನಾದಿ. ಅಲ್ಲಿರುವ ಆಧ್ಯಾತ್ಮಿಕ ನಿಯಮಗಳು ಮಾತ್ರ ಅನಾದಿ. ನಿತ್ಯವಾದ ಸನಾತನವಾದ ಆ ನಿಯಮಗಳನ್ನು ಕಾಲಕಾಲಕ್ಕೆ ಹಲವು ಋಷಿಗಳು ಕಂಡುಹಿಡಿದರು. ಅವರಲ್ಲಿ ಈಗ ಹಲವರ ಹೆಸರು ಮರೆತು ಹೋಗಿವೆ; ಕೆಲವು ಜ್ಞಾಪಕದಲ್ಲಿವೆ.

ಹಲವು ಜನರು ಸಮುದ್ರವನ್ನು ಬೇರೆ ಬೇರೆ ದೃಷ್ಟಿಕೋನಗಳಿಂದ ನೋಡಿದಾಗ ಪ್ರತಿಯೊಬ್ಬರೂ ತಮ್ಮ ತಮ್ಮ ದೃಷ್ಟಿಗೆ ತಕ್ಕಂತೆ ಅದನ್ನು ನೋಡುವರು. ಪ್ರತಿಯೊಬ್ಬರೂ ತಮಗೆ ಕಾಣುವುದೇ ನಿಜವಾದ ಸಮುದ್ರವೆಂದರೂ ಎಲ್ಲರೂ ಸತ್ಯವನ್ನೇ ಹೇಳುತ್ತಿರುವರು. ಏಕೆಂದರೆ ಎಲ್ಲರೂ ಅನಂತಸಾಗರದ ಅಂಶವನ್ನೇ ನೋಡುತ್ತಿರುವರು. ಇದರಂತೆಯೇ ಹಲವು ಧರ್ಮಗಳು ಸಾರುವುದು ಪರಸ್ಪರ ವಿರೋಧವಾಗಿ ಒಂದೇ ಅನಂತವಾದ ಸತ್ಯವನ್ನೇ ವಿವರಿಸುತ್ತಿರುವರು.

ಒಬ್ಬ ಮರೀಚಿಕೆಯನ್ನು ಮೊದಲು ನೋಡಿದಾಗ ಅದನ್ನು ಸತ್ಯವೆಂದು ಭ್ರಮಿಸುವನು. ಅದರಲ್ಲಿ ಬಾಯಾರಿಕೆಯನ್ನು ತಣಿಸಲು ವೃಥಾಯತ್ನಿಸಿ, ಅದೊಂದು ಮರೀಚಿಕೆ ಎಂದು ಅರಿಯುವನು. ಅವನು ಪುನಃ ಮುಂದೆ ಇನ್ನೊಂದು ಮರೀಚಿಕೆಯನ್ನು ಕಂಡಾಗ ಅದೊಂದು ಭ್ರಾಂತಿ ಎಂಬ ಭಾವನೆಯು ಅವನಿಗೆ ತತ್‌ಕ್ಷಣ ಹೊಳೆಯುವುದು. ಇದರಂತೆಯೇ ಜೀವನ್ಮುಕ್ತನಿಗೆ ಮಾಯೆ.

ಸ್ವಾಭಾವಿಕವಾಗಿ ಕೆಲವು ಶಕ್ತಿಗಳು ಕೆಲವರಲ್ಲಿ ಮಾತ್ರ ಇರುವಂತೆ ಕೆಲವು ವೈದಿಕ ರಹಸ್ಯಗಳು ಕೆಲವು ವಂಶದವರಿಗೆ ಮಾತ್ರ ಗೊತ್ತಿದ್ದುವು. ಈ ವಂಶಗಳು ನಿರ್ನಾಮವಾದ ಮೇಲೆ ಈ ರಹಸ್ಯಗಳೂ ಮಾಯವಾಗಿ ಹೋದುವು.

ಆಯುರ್ವೇದದ ದೇಹ ರಚನಾಶಾಸ್ತ್ರ \enginline{Anatomy} ದಷ್ಟೇ ವೇದಕಾಲದ (ದೇಹ ರಚನಾಶಾಸ್ತ್ರವೂ) ಅಭಿವೃದ್ದಿಯಾಗಿತ್ತು. ದೇಹದ ಹಲವು ಭಾಗಗಳಿಗೆ ಹಲವು ಹೆಸರುಗಳಿದ್ದುವು. ಅವರು ಯಜ್ಞಗಳಿಗೆ ಪ್ರಾಣಿಗಳನ್ನು ಬಲಿಕೊಡಬೇಕಾಗಿತ್ತು. ಸಮುದ್ರವು ಹಡಗುಗಳಿಂದ ತುಂಬಿದೆ ಎನ್ನುವರು. ಸಮುದ್ರಯಾನವನ್ನು ಅನಂತರ ಬಹಿಷ್ಕರಿಸಿದರು. ಏಕೆಂದರೆ ಅವರೆಲ್ಲ ಎಲ್ಲಿ ಬೌದ್ಧರಾಗಿ ಹೋಗುವರೋ ಎಂದು.

ವೇದಕಾಲದ ಪುರೋಹಿತರಿಗೆ ವಿರೋಧವಾಗಿ ದಂಗೆಯೆದ್ದ ಕ್ಷತ್ರಿಯರ ಚಳುವಳಿಯೇ ಬೌದ್ಧ ಧರ್ಮ. ಹಿಂದೂಧರ್ಮವು ಬೌದ್ಧ ಧರ್ಮದ ಸಾರವನ್ನೆಲ್ಲಾ ಹೀರಿ ಅದನ್ನು ಆಚೆಗೆ ಎಸೆಯಿತು. ದಾಕ್ಷಿಣಾತ್ಯ ಆಚಾರ್ಯರೆಲ್ಲ ಇಬ್ಬರಿಗೂ ರಾಜಿ ಮಾಡಲು ಯತ್ನಿಸುತ್ತಿದ್ದರು. ಶಂಕರಾಚಾರ್ಯರ ಬೋಧನೆಯಲ್ಲಿ ಬೌದ್ಧರ ಪ್ರಭಾವವನ್ನು ನೋಡುತ್ತೇವೆ. ಶಂಕರಾಚಾರ್ಯರ ಶಿಷ್ಯರು ಸರಿಯಾಗಿ ಅವರ ಗುರುಗಳನ್ನು ಅರ್ಥಮಾಡಿಕೊಳ್ಳಲಿಲ್ಲ. ಆದಕಾರಣವೇ ಅವರನ್ನು ಪ್ರಚ್ಛನ್ನ ಬೌದ್ಧರೆಂದು ಕರೆದರು.

ಸ್ಪೆನ್ಸರ್‌ನ ಅಜ್ಞೇಯ ಎಂದರೇನು? ಅದೇ ನಮ್ಮ ಮಾಯೆ. ಪಾಶ್ಚಾತ್ಯ ದಾರ್ಶನಿಕರಿಗೆ ಅಜ್ಞೇಯವೆಂದರೆ ಅಂಜಿಕೆ. ಆದರೆ ನಮ್ಮ ದಾರ್ಶನಿಕರು ಕಡುಗೆಚ್ಚಿನಿಂದ ಅದಕ್ಕೆ ಹಾರಿ ಅದನ್ನು ಜನಿಸಿರುವರು. ಪಾಶ್ಚಾತ್ಯ ದಾರ್ಶನಿಕರು ಮೇಲೆ ಆಕಾಶದಲ್ಲಿ ಹಾರುತ್ತಿರುವ ರಣಹದ್ದುಗಳಂತೆ; ಅವರ ಮನಸ್ಸೆಲ್ಲಾ ಭೂಮಿಯ ಮೇಲೆ ನಾರುತ್ತಿರುವ ಹೆಣಗಳ ಮೇಲಿರುವುದು. ಅವರು ಅಜೇಯವನ್ನು ಜಯಿಸಲಾರರು. ಆದಕಾರಣವೇ ಹಿಂದಿರುಗಿ ಭೂಮಿಗೆ ಬಂದು ಧನಲಕ್ಷ್ಮಿಯನ್ನು ಆರಾಧಿಸುವರು.

ಈ ಪ್ರಪಂಚದಲ್ಲಿ ಜನರು ಎರಡು ದಾರಿಗಳಲ್ಲಿ ಮುಂದುವರಿದಿರುವರು. ಒಂದು ರಾಜಕೀಯ ಮತ್ತೊಂದು ಧಾರ್ಮಿಕ. ಮೊದಲನೆಯದರಲ್ಲಿ ಗ್ರೀಕರು ಪರಿಣತರು. ಆಧುನಿಕ ರಾಜಕೀಯ ಸಂಸ್ಥೆಗಳೆಲ್ಲ ಪೂರ್ವದ ಗ್ರೀಕರ ಸಂಸ್ಥೆಗಳ ಬೆಳವಣಿಗೆಗಳಷ್ಟೆ. ಧಾರ್ಮಿಕ ಕ್ಷೇತ್ರದಲ್ಲಿ ಹಿಂದೂಗಳು ಪರಿಣಿತರು.

ನಮ್ಮ ಧರ್ಮತರುವಿನಲ್ಲಿ ದೂರದಲ್ಲಿ ಕವಲೊಡೆದ ಶಾಖೆಯೇ ಕ್ರೈಸ್ತಧರ್ಮ. ಬೌದ್ಧ ಧರ್ಮವು ಹಿಂದೂಧರ್ಮದಿಂದ ಬಲಾತ್ಕಾರವಾಗಿ ಬೇರೆಯಾದ ಮಗುವಿನಂತೆ ಇದೆ.

ಯಾವ ಒಂದು ವಸ್ತುವಿನಿಂದ ಉಳಿದವೆಲ್ಲ ಆಗಿರುವುವೋ ಅದನ್ನು ಅರಿತಾದ ಮೇಲೆ ರಸಾಯನಶಾಸ್ತ್ರ ಇನ್ನು ಮುಂದುವರಿಯಲಾರದು. ಅದರಂತೆಯೇ ಒಂದು ಏಕತೆಯನ್ನು ಅರಿತು ಧರ್ಮ ಇನ್ನು ಮುಂದುವರಿಯಲಾರದು. ಹಿಂದೂ ಧರ್ಮದ ಸ್ಥಿತಿ ಹೀಗಿರುವುದು.

ವೇದಗಳಲ್ಲಿ ಇಲ್ಲದ ಧಾರ್ಮಿಕ ಭಾವನೆಯನ್ನು ಯಾರೂ ಎಲ್ಲಿಯೂ ಬೋಧಿಸಿಲ್ಲ. ಎಲ್ಲಾ ಕಡೆಗಳಲ್ಲಿಯೂ ಎರಡು ವಿಧದ ಬೆಳವಣಿಗೆ ಇದೆ. ಒಂದು ವಿಭಜನಾತ್ಮಕವಾದುದು \enginline{(Analytical)} ಮತ್ತೊಂದು ಸಂಯೋಜನಾತ್ಮಕವಾದುದು \enginline{(Synthetical).} ಮೊದಲನೆಯದರಲ್ಲಿ ಹಿಂದೂಗಳನ್ನು ಯಾರೂ ಮೀರಿಸಿರುವುದಿಲ್ಲ. ಎರಡನೆಯದರಲ್ಲಿ ಅವರನ್ನು ಕೇಳುವವರೇ ಇಲ್ಲ.

ಹಿಂದೂಗಳು ವಿಭಜನೆ ಮಾಡುವುದನ್ನು ಮತ್ತು ಒಂದು ವಸ್ತುವಿನ ಮೂಲಕ್ಕೆ ಹೋಗುವುದನ್ನು ರೂಢಿಸಿರುವರು. ಪ್ರಪಂಚದಲ್ಲಿ ಮತ್ತಾವ ದೇಶವೂ ಪಾಣಿನಿಯ ವ್ಯಾಕರಣದಂತಹ ಮತ್ತೊಂದು ವ್ಯಾಕರಣವನ್ನು ಸೃಷ್ಟಿಸಿಲ್ಲ.

ರಾಮಾನುಜರ ಮುಖ್ಯ ಕೆಲಸವೇ ಜೈನರನ್ನು ಮತ್ತು ಬೌದ್ಧರನ್ನು ಹಿಂದೂ ಧರ್ಮಕ್ಕೆ ಸೇರಿಸಿದ್ದು. ಅವರು ವಿಗ್ರಹಾರಾಧನೆಗೆ ಹೆಚ್ಚು ಪ್ರೋತ್ಸಾಹವನ್ನು ಕೊಡುತ್ತಿದ್ದರು. ಮುಕ್ತಿಗೆ ಶ್ರದ್ಧಾಭಕ್ತಿಗಳು ಅತ್ಯಂತ ಮುಖ್ಯ ಎಂಬುದನ್ನು ಅವರು ಬಳಕೆಗೆ ತಂದರು.

ಭಾಗವತದಲ್ಲಿ ಕೂಡ ಜೈನರ ಇಪ್ಪತ್ತನಾಲ್ಕು ತೀರ್ಥಂಕರರಂತೆ ಇಪ್ಪತ್ತನಾಲ್ಕು ಅವತಾರಗಳನ್ನು ಹೇಳುವರು. ವೃಷಭದೇವ ಇಬ್ಬರಿಗೂ ಸಾಮಾನ್ಯನಾಗಿರುವನು.

ಯೋಗಾಭ್ಯಾಸದ ಬಲದಿಂದ ಮನಸ್ಸನ್ನು ಒಂದು ವಸ್ತುವಿನಿಂದ ಹಿಂದಕ್ಕೆ ತೆಗೆದುಕೊಳ್ಳುವುದು ಸಾಧ್ಯವಾಗುವುದು. ಸಿದ್ಧನು ಒಂದು ವಸ್ತುವಿನ ಗುಣಗಳನ್ನೆಲ್ಲ ಕಳೆದು ಅದರ ಮೂಲ ದ್ರವ್ಯವನ್ನು ಮಾತ್ರ ಕುರಿತು ಯೋಚಿಸಬಲ್ಲ. ಇದೇ ಅವನ ವೈಶಿಷ್ಟ್ಯ. ಅವನು ದ್ರವ್ಯವನ್ನೇ ಬೇರೊಂದು ದೃಶ್ಯವಸ್ತುವೆಂಬಂತೆ ಗ್ರಹಿಸಬಲ್ಲ.

ವಿರೋಧವಾದ ಅತಿರೇಕಗಳೆರಡೂ ಒಂದನ್ನೊಂದು ಹೋಲುವುವು. ಅನಂತ ಬ್ರಹ್ಮನ ಚಿಂತನೆಯಲ್ಲಿ ಸಂಪೂರ್ಣ ಮೈಮರೆತ ಭಕ್ತ ಮತ್ತು ಅತಿ ನೀಚನಾದ ಕುಡುಕ ಇಬ್ಬರೂ ನೋಟಕ್ಕೆ ಒಂದೇ ಸಮನಾಗಿ ಕಾಣುವರು. ಕೆಲವು ವೇಳೆ ನಮಗೆ ಈ ಹೋಲಿಕೆಯಿಂದ ಆಶ್ಚರ್ಯವಾಗುವುದು.

ತುಂಬ ದೃಢಚಿತ್ತರು ಆಧ್ಯಾತ್ಮಿಕ ಜೀವನದಲ್ಲಿ ಜಯಶೀಲರಾಗುವರು. ಅವರು ಯಾವುದನ್ನು ತೆಗೆದುಕೊಳ್ಳಲಿ ಅದರಲ್ಲಿ ತನ್ಮಯರಾಗಿ ಹೋಗುವರು.

"ಪ್ರಪಂಚದಲ್ಲಿರುವವರು ಎಲ್ಲರೂ ಹುಚ್ಚರೆ. ಕೆಲವರಿಗೆ ಹೊನ್ನಿನ ಹುಚ್ಚು, ಕೆಲವರಿಗೆ ಹೆಣ್ಣಿನ ಹುಚ್ಚು, ಮತ್ತೆ ಕೆಲವರಿಗೆ ದೇವರ ಹುಚ್ಚು. ಹೇಗಿದ್ದರೂ ಮುಳುಗಿ ಸಾಯಬೇಕಾಗಿರುವಾಗ ಒಂದು ಗೊಬ್ಬರದ ಗುಂಡಿಯಲ್ಲಿ ಬಿದ್ದು ಸಾಯುವುದಕ್ಕಿಂತ ಕ್ಷೀರಸಾಗರದಲ್ಲಿ ಮುಳುಗಿ ಸಾಯುವುದು ಮೇಲು" ಎಂದು ದೈವೋನ್ಮಾದದಲ್ಲಿದ್ದ ಭಕ್ತರೊಬ್ಬರು ಹೇಳಿದರು.

ಅನಂತಪ್ರೇಮದ ಈಶ್ವರನನ್ನು ಪರಮಪ್ರೇಮಕ್ಕೆ ಪಾತ್ರನಾದ ವಸ್ತುವನ್ನೂ ಅನಂತವನ್ನೂ ನೀಲಿಯಾಗಿ ಚಿತ್ರಿಸುವರು. ಕೃಷ್ಣನನ್ನು ನೀಲಿಯಾಗಿ ಚಿತ್ರಿಸುವರು. ಹಾಗೆಯೇ ಸಾಲೊಮನ್ನಿನ ಪ್ರೇಮೇಶ್ವರನ್ನೂ ಕೂಡ ನೀಲಿಯಾಗಿಯೇ ಚಿತ್ರಿಸುವರು. ಯಾವುದು ಅನಂತವಾಗಿರುವುದೋ, ಅಮೋಘವಾಗಿರುವುದೋ ಅದನ್ನು ನೀಲಿ ಬಣ್ಣದೊಂದಿಗೆ ಹೋಲಿಸುವುದು ಸ್ವಾಭಾವಿಕ. ಒಂದು ಬೊಗಸೆ ನೀರನ್ನು ತೆಗೆದುಕೊಳ್ಳಿ. ಅದಕ್ಕೆ ಯಾವ ಬಣ್ಣವೂ ಇಲ್ಲ. ಆದರೆ ಅನಂತ ಆಕಾಶವನ್ನು ನೋಡಿ ಅದು ನೀಲಿಯಾಗಿರುವುದು.

ಹಿಂದೂಗಳು ಭಾವನೆಯಲ್ಲಿ ತತ್ಪರರಾಗಿ ವಾಸ್ತವಿಕತೆಯನ್ನು ಮರೆತರು ಎಂಬುದು ಇದರಿಂದ ಕಾಣುವುದು. ಶಿಲ್ಪಕಲೆಯನ್ನು ಮತ್ತು ಚಿತ್ರಕಲೆಯನ್ನು ತೆಗೆದುಕೊಳ್ಳಿ. ಹಿಂದೂಗಳ ಚಿತ್ರಕಲೆಯಲ್ಲಿ ನೀವು ಏನನ್ನು ನೋಡುತ್ತೀರಿ? ಹಲವು ಬಗೆಯ ಚಿತ್ರವಿಚಿತ್ರವಾದ ಅಸ್ವಾಭಾವಿಕವಾದ ಭಾವಭಂಗಿಗಳನ್ನು. ಹಿಂದೂ ದೇವಸ್ಥಾನದಲ್ಲಿ ನೀವು ಏನನ್ನು ನೋಡುತ್ತೀರಿ? ಚತುರ್ಭುಜ ನಾರಾಯಣ ಮುಂತಾದ ಚಿತ್ರಗಳನ್ನು ನೋಡುತ್ತೀರಿ. ಆದರೆ ಇಟಲಿಯವರ ಚಿತ್ರಗಳನ್ನೋ, ಗ್ರೀಕರ ಶಿಲ್ಪಗಳನ್ನೋ ತೆಗೆದುಕೊಳ್ಳಿ. ಅವುಗಳಲ್ಲಿ ನಿಸರ್ಗವನ್ನು ಎಷ್ಟು ಕೂಲಂಕಷವಾಗಿ ವಿವರಿಸುವರು! ಒಬ್ಬ ಕೈಯಲ್ಲಿ ಕ್ಯಾಂಡಲನ್ನು ಹಿಡಿದುಕೊಂಡು ಹೋಗುತ್ತಿರುವ ಹೆಂಗಸನ್ನು ಚಿತ್ರಿಸಲು ಇಪ್ಪತ್ತು ವರ್ಷಗಳವರೆಗೆ ಕ್ಯಾಂಡಲನ್ನು ಹಚ್ಚುತ್ತ ತಾನೇ ಕುಳಿತ!

ಹಿಂದೂಗಳು ಭಾವಪ್ರಧಾನ ಶಾಸ್ತ್ರಗಳಲ್ಲಿ ಮುಂದುವರಿದರು. ಮನುಷ್ಯನ ಸ್ವಭಾವಕ್ಕೆ ತಕ್ಕಂತೆ ಹಲವು ಆಚಾರಗಳನ್ನು ವೇದ ಸಾರುವುದು. ಯಾವುದನ್ನು ವಯಸ್ಕರಿಗೆ ಬೋಧಿಸುವರೋ ಅವನ್ನು ಹುಡುಗರಿಗೆ ಬೋಧಿಸುವುದಿಲ್ಲ.

ಗುರು ಭವವೈದ್ಯನಾಗಬೇಕು. ಅವನು ತನ್ನ ಶಿಷ್ಯನ ಸ್ವಭಾವವನ್ನು ಚೆನ್ನಾಗಿ ಅರಿತು ಅವನ ಸ್ವಭಾವಕ್ಕೆ ತಕ್ಕ ಬೋಧನೆಯನ್ನು ಮಾಡಬೇಕು.

ಯೋಗಾಭ್ಯಾಸಕ್ಕೆ ಹಲವು ಮಾರ್ಗಗಳಿವೆ. ಕೆಲವರಿಗೆ ಕೆಲವು ಮಾರ್ಗಗಳು ಜಯಪ್ರದವಾಗಿವೆ. ಆದರೆ ಎಲ್ಲದರಲ್ಲೂ ಎರಡು ಮುಖ್ಯ ವಿಷಯಗಳಿವೆ: ಮೊದಲನೆಯದು ಎಲ್ಲ ಅನುಭವಗಳನ್ನೂ ನಿರಾಕರಿಸುತ್ತ ಹೋಗುವುದು; ಎರಡನೆಯದು ನೀವೇ ಎಲ್ಲವೂ, ಇಡೀ ಜಗತ್ತೂ ನೀವೆ ಎಂದು ಭಾವಿಸುವುದು. ಎರಡನೆಯದು ಮೊದಲನೆಯದಕ್ಕಿಂತ ಬೇಗ ಗುರಿಯ ಎಡೆಗೆ ಒಯ್ದರೂ ಸುರಕ್ಷಿತವಾದ ಮಾರ್ಗವಲ್ಲ. ಇದರಲ್ಲಿ ಹಲವು ಅಪಾಯಗಳಿವೆ. ಸಾಧಕನನ್ನು ಅಡ್ಡಹಾದಿಗೆ ಎಳೆದು ಗುರಿಯನ್ನು ಸೇರದಂತೆ ಮಾಡುವ ಅಂಶಗಳಿವೆ.

ಹಿಂದೂಗಳು ಬೋಧಿಸುವ ಪ್ರೀತಿಗೂ ಕ್ರೈಸ್ತರು ಬೋಧಿಸುವ ಪ್ರೀತಿಗೂ ಈ ವ್ಯತ್ಯಾಸವಿದೆ: ನೆರೆಯವರು ಹೇಗೆ ನಮ್ಮನ್ನು ಪ್ರೀತಿಸಬೇಕೆಂದು ಆಶಿಸುವೆವೋ ಹಾಗೆ ನೀವು ನೆರೆಯವರನ್ನು ಪ್ರೀತಿಸಿ ಎಂದು ಕ್ರೈಸ್ತರು ಬೋಧಿಸುವರು. ಹಿಂದೂಗಳು ಇತರರನ್ನು 'ನಾವೇ ಅವರು' ಎಂದು ಪ್ರೀತಿಸಿ ಎನ್ನುವರು, ಅವರಲ್ಲಿ ನಮ್ಮನ್ನೇ ನೋಡಬೇಕು ಎನ್ನುವರು.

ಮುಂಗುಸಿಯನ್ನು ಮೇಲೆ ಒಂದು ಗ್ಲಾಸಿನ ಪೆಟ್ಟಿಗೆಯಲ್ಲಿಟ್ಟು ಅದಕ್ಕೆ ಒಂದು ದೊಡ್ಡ ಸರಪಳಿಯನ್ನು ಕಟ್ಟಿರುವರು. ಅದು ಸ್ವತಂತ್ರವಾಗಿ ಸಂಚರಿಸಬಹುದು. ಅಪಾಯದ ಸುಳಿವು ಗೊತ್ತಾದೊಡನೆಯೇ ಅದು ಗ್ಲಾಸಿನ ಗೂಡಿಗೆ ನೆಗೆಯುವುದು. ಯೋಗಿಯೂ ಕೂಡ ಇದರಂತೆಯೇ ಪ್ರಪಂಚದಲ್ಲಿ ಇರುವನು.

ಇಡೀ ಬ್ರಹ್ಮಾಂಡ ಒಂದು ದೊಡ್ಡ ಅಸ್ತಿತ್ವದ ಸರಪಳಿಯಂತೆ ಇರುವುದು. ಅದರಲ್ಲಿ ದ್ರವ್ಯ \enginline{(Matter)} ಒಂದು ತುದಿ, ದೇವರು ಮತ್ತೊಂದು ತುದಿ. ವಿಶಿಷ್ಟಾದ್ವೈತವನ್ನು ಇಂತಹ ಉಪಮಾನದ ಮೂಲಕ ವಿವರಿಸಬಹುದು.

ವೇದದಲ್ಲಿ ಸಾಕಾರ ದೇವರನ್ನು ಸಮರ್ಥಿಸುವ ಹಲವು ಮಂತ್ರಗಳಿವೆ. ದೀರ್ಘಧ್ಯಾನದಿಂದ ದೇವರನ್ನು ನೋಡಿದ ಋಷಿಗಳು ಪ್ರಪಂಚದಲ್ಲಿ ಅಜ್ಞೇಯವಾದುದನ್ನು ಅರಿತಿದ್ದರು. ಅವರು ಪ್ರಪಂಚಕ್ಕೆ ಸವಾಲನ್ನು ಹಾಕಿರುವರು. ಯಾರು ಆ ಋಷಿಗಳು ಹೇಳಿದ ಮಾರ್ಗದಲ್ಲಿ ನಡೆದಿಲ್ಲವೋ, ಅವರ ಬೋಧನೆಯನ್ನು ಕೇಳಿಲ್ಲವೋ, ಅಂತಹ ಹುರುಳಿಲ್ಲದ ಮನುಷ್ಯರು ಮಾತ್ರ ಅವರನ್ನು ಜರಿಯುವರು. ಯಾವಾಗಲೂ ನಾವು ಅವರು ಹೇಳಿದ ಮಾರ್ಗದಲ್ಲಿ ನಡೆದಿರುವೆವು. ಆದರೂ ನಮಗೆ ಏನೂ ಕಾಣಿಸಲಿಲ್ಲ, ಅವರು ಹೇಳಿರುವುದು ಸುಳ್ಳು ಎಂದು ಹೇಳುವ ಎದೆಗಾರಿಕೆಯಿಲ್ಲ. ಹಲವು ಸಾಧಕರು ಪರೀಕ್ಷೆಗೆ ಒಳಪಟ್ಟಿರುವರು. ಆದರೆ ದೇವರು ಕೈಬಿಡಲಿಲ್ಲ. ಅವರಿಗೆ ಇದರ ಅನುಭವ ಆಗಿದೆ. ದೇವರ ಮೇಲೆ ಇಟ್ಟ ಭರವಸೆಯಿಂದ ನಮಗೆ ಸಮಾಧಾನ ಸಿಕ್ಕದೆ ಇದ್ದರೆ ಈ ಪ್ರಪಂಚದಲ್ಲಿ ಆತ್ಮಹತ್ಯೆಯೇ ಮೇಲು.

ಒಬ್ಬ ಸಾಧು ಸ್ವಭಾವದ ಪಾದ್ರಿ ಕೆಲಸದ ಮೇಲೆ ಹೊರಟುಹೋದ: ಇದ್ದಕ್ಕಿದ್ದಂತೆಯೇ ಅವನ ಮೂರು ಜನ ಮಕ್ಕಳು ಕಾಲರಾ ರೋಗದಿಂದ ಕಾಲವಾದರು. ಅವನ ಹೆಂಡತಿ ಮೂರು ಮಕ್ಕಳ ಶವಗಳ ಮೇಲೆ ಬಟ್ಟೆಯನ್ನು ಹೊದಿಸಿ ಗಂಡನಿಗಾಗಿ ಕಾಯುತ್ತಿದ್ದಳು. ಅವನು ಹಿಂದಿರುಗಿ ಬಂದಾಗ ಅವನನ್ನು ಬಾಗಿಲಲ್ಲಿ ತಡೆದು ಹೀಗೆ ಪ್ರಶ್ನೆ ಹಾಕಿದಳು – "ನೋಡಿ, ಯಾರೋ ನಿಮ್ಮ ವಶಕ್ಕೆ ಏನನ್ನೋ ಇಟ್ಟುಕೊಂಡಿರು ಎಂದು ಕೊಡುತ್ತಾರೆ. ನೀವು ಇಲ್ಲದಾಗ ಅದನ್ನು ಅವರು ತೆಗೆದುಕೊಂಡು ಹೋಗಿಬಿಡುವರು. ಇದರಿಂದ ನಿಮಗೆ ವ್ಯಥೆಯಾಗುವುದೇ?" “ನಿಜವಾಗಿ ವ್ಯಥೆಯಾಗುವುದಿಲ್ಲ” ಎಂದ ಗಂಡ. ಅನಂತರ ಗಂಡನನ್ನು ಒಳಗೆ ಕರೆದುಕೊಂಡು ಹೋಗಿ ಮಕ್ಕಳ ಶವಗಳನ್ನು ತೋರಿಸಿದಳು. ಅವನು ಇದನ್ನು ಬಹಳ ಸ್ಥೈರ್ಯದಿಂದ ಸಹಿಸಿ ಶವಸಂಸ್ಕಾರವನ್ನು ಮಾಡಿದನು. ದಯಾಮಯನಾದ ಭಗವಂತನೊಬ್ಬನು ಪ್ರಪಂಚದಲ್ಲಿ ಪ್ರತಿಯೊಂದನ್ನೂ ತನ್ನ ಇಚ್ಛಾನುಸಾರವಾಗಿ ಮಾಡುತ್ತಿರುವನು ಎಂದು ನಂಬಿರುವ ಶ್ರದ್ಧೆ ಇಂತಹುದು.

ಅಖಂಡವನ್ನು ಕುರಿತು ನಾವು ಎಂದಿಗೂ ಆಲೋಚಿಸಲಾರೆವು. ಒಂದು ವಸ್ತು ಸಾಂತವಾಗದೆ ಅದು ನಮಗೆ ಗೊತ್ತಾಗುವಂತೆ ಇಲ್ಲ. ಅನಂತನಾದ ದೇವರನ್ನು ಸಾಂತವಾಗಿ ಮಾತ್ರ ಭಾವಿಸಬಹುದು, ಪೂಜಿಸಬಹುದು.

ಜಾನ್ ದಿ ಬ್ಯಾಪ್ಟಿಸ್ಟ್ ಎಂಬುವನು ಒಬ್ಬ ಯಸೀನಿಯವನು. (ಪ್ರಾಚೀನ ಯಹೂದಿ ಸಂನ್ಯಾಸಿ ಪಂಗಡ) ಅದೊಂದು ನಿಜವಾಗಿಯೂ ಬೌದ್ಧರ ಪಂಗಡ. ಕ್ರೈಸ್ತರ ಶಿಲುಬೆ ಎರಡು ಶಿವಲಿಂಗಗಳನ್ನು ಜೋಡಿಸಿದಂತೆ ಇದೆ. ಬೌದ್ಧರ ಪೂಜೆಯ ಅವಶೇಷಗಳು ಪುರಾತನ ರೋಮನ್ನರ ಅವಶೇಷಗಳಲ್ಲಿವೆ.

ದಕ್ಷಿಣ ದೇಶದಲ್ಲಿ ಕೆಲವು ರಾಗಗಳನ್ನು ಅವು ಸ್ವತಂತ್ರ ಎಂದು ಭಾವಿಸಿ ಹಾಡುವರು. ಆದರೆ ಅವು ಮುಖ್ಯ ಸಪ್ತಸ್ವರಗಳಿಂದ ಜನ್ಯವಾದ ರಾಗಗಳು. ದಾಕ್ಷಿಣಾತ್ಯ ಸಂಗೀತದಲ್ಲಿ ಬಹಳ ಕಡಮೆ ಮೂರ್ಛನಾ ಇರುವುದು. ನಿಜವಾದ ಸಂಗೀತಕ್ಕೆ ಸಂಬಂಧಪಟ್ಟ ವಾದ್ಯ ಕೂಡ ಬಹಳ ಅಪರೂಪ. ದಾಕ್ಷಿಣಾತ್ಯರ ವೀಣೆ ನಿಜವಾದ ವೀಣೆಯಲ್ಲ. ನಮ್ಮ ಸಂಗೀತದಲ್ಲಿ ವೀರರಸವಿಲ್ಲ; ಕವಿತೆಯಲ್ಲೂ ವೀರರಸವಿಲ್ಲ. ಭವಭೂತಿಯಲ್ಲಿ ಸ್ವಲ್ಪ ವೀರರಸವನ್ನು ನೋಡುತ್ತೇವೆ.

\delimiter

ಕ್ರಿಸ್ತ ಸಂನ್ಯಾಸಿಯಾಗಿದ್ದ. ಅವನ ಧರ್ಮ ಮುಖ್ಯವಾಗಿ ಸಂನ್ಯಾಸಿಗಳಿಗೆ ಮಾತ್ರ. ಅವನ ಸಂದೇಶವನ್ನೆಲ್ಲ 'ತ್ಯಜಿಸಿ' ಎಂಬ ಒಂದು ಮಾತಿನಲ್ಲಿ ಸಂಗ್ರಹಿಸಬಹುದು ಇದಕ್ಕಿಂತ ಹೆಚ್ಚು ಇಲ್ಲ. ಎಲ್ಲೋ ಕೆಲವರಿಗೆ ಮಾತ್ರ ಇದು ಸಾಧ್ಯ.

“ಮತ್ತೊಂದು ಕೆನ್ನೆಯನ್ನು ತೋರಿ.” ಇದು ಅಸಾಧ್ಯ; ಇದು ವ್ಯವಹಾರ ಸಾಧ್ಯವೇ ಅಲ್ಲ! ಪಾಶ್ಚಾತ್ಯರಿಗೆ ಇದು ಗೊತ್ತಿದೆ. ಯಾರು ಪೂರ್ಣತೆಯನ್ನು ಪಡೆಯಬೇಕೆಂದು, ಧಾರ್ಮಿಕರಾಗಬೇಕೆಂದು ಅಕಾಂಕ್ಷೆಯುಳ್ಳವರಾಗಿರುವರೋ ಅವರಿಗೆ ಮಾತ್ರ ಇದು ಸಾಧ್ಯ.

"ನಿಮ್ಮ ಹಕ್ಕುಗಳನ್ನು ಬಿಡಬೇಡಿ" ಎಂಬುದು ಸಾಧಾರಣ ಮನುಷ್ಯನಿಗೆ ಕೊಡುವ ಬೋಧನೆ. ಗೃಹಸ್ಥರಿಗೆ ಮತ್ತು ಸಂನ್ಯಾಸಿಗಳಿಗೆ ಒಂದೇ ನಿಯಮಗಳನ್ನು ಹೇಳಲು ಸಾಧ್ಯವಿಲ್ಲ. ಆದರೆ ವಿಜ್ಞಾನ ಇದಕ್ಕೆ ವಿರೋಧವಾಗಿದೆ. ಎರಡು ದೇಹಗಳಿಗಿಂತ ಎರಡು ಮನಸ್ಸುಗಳಲ್ಲಿ ಎಷ್ಟೋ ವ್ಯತ್ಯಾಸಗಳಿರುತ್ತವೆ. ಜನರೆಲ್ಲ ಬೇರೆ ಬೇರೆ ಎಂಬುದೇ ಹಿಂದೂಧರ್ಮದ ಮುಖ್ಯ ಸಿದ್ಧಾಂತ. ವೈವಿಧ್ಯದಲ್ಲಿ ಮಾತ್ರ ಏಕತೆಯಿದೆ. ಕುಡುಕನಿಗೂ ಕೆಲವು ಮಂತ್ರಗಳಿವೆ. ವಾರಾಂಗನೆಯ ಮನೆಗೆ ಹೋಗುವವನಿಗೂ ಕೆಲವು ಮಂತ್ರಗಳಿವೆ.

ನೀತಿ ಎಂಬುದು ಸಾಪೇಕ್ಷ ಪದ, ಪ್ರಪಂಚದಲ್ಲಿ ನಿರಪೇಕ್ಷವಾದ ನೀತಿ ಯಾವುದಾದರೂ ಇದೆ ಎಂದು ಭಾವಿಸಿದಿರೇನು? ಇದೊಂದು ಮೌಢ್ಯ. ಪ್ರತಿಯೊಬ್ಬರನ್ನೂ ಪ್ರತಿಯೊಂದು ಕಾಲದಲ್ಲಿಯೂ ಒಂದೇ ದೃಷ್ಟಿಯಿಂದ ಅಳೆಯುವುದಕ್ಕೆ ಆಗುವುದಿಲ್ಲ.

ಪ್ರತಿಯೊಬ್ಬರೂ ಪ್ರತಿಯೊಂದು ಕಾಲದಲ್ಲಿಯೂ, ಪ್ರತಿಯೊಂದು ದೇಶದಲ್ಲಿಯೂ ಬೇರೆ ಬೇರೆ ಸ್ಥಿತಿಯಲ್ಲಿರುವರು. ಸ್ಥಿತಿ ಬದಲಾದರೆ ಭಾವನೆಯೂ ಬದಲಾಗುವುದು. ದನದ ಮಾಂಸವನ್ನು ತಿನ್ನುವುದು ಧರ್ಮಕ್ಕೆ ವಿರೋಧವಾಗಿರಲಿಲ್ಲ. ಹಿಂದೆ ಚಳಿ ಜಾಸ್ತಿ ಇತ್ತು, ದವಸಧಾನ್ಯಗಳು ಇಷ್ಟು ಇರಲಿಲ್ಲ. ಆಗ ಮಾಂಸ ಒಂದೇ ಮುಖ್ಯ ಆಹಾರವಾಗಿತ್ತು. ಆ ದೇಶದಲ್ಲಿ ಆ ಕಾಲದಲ್ಲಿ ದನದ ಮಾಂಸ ತಿನ್ನುವುದಕ್ಕೆ ಅನುಮತಿ ಇತ್ತು. ಆದರೆ ಈಗ ಅದನ್ನು ಅಧರ್ಮ ಎಂದು ಭಾವಿಸುವರು.

ಬದಲಾಗದೇ ಇರುವವನು ದೇವರೊಬ್ಬನೇ. ಸಮಾಜ ಬದಲಾಗುತ್ತದೆ. ಜಗತ್ ಎಂದರೆ ಚಲಿಸುತ್ತಿರುವುದು ಎಂದು ಅರ್ಥ. ದೇವರೊಬ್ಬನೇ ಅಚಲನು.

ನಾನು ಸುಧಾರಣೆ ಹೊಂದಿ ಎನ್ನುವುದಿಲ್ಲ. ಮುಂದುವರಿಯಿರಿ ಎನ್ನುತ್ತೇನೆ. ಸುಧಾರಣೆಗೆ ಅಯೋಗ್ಯವಾಗಿರುವುದು ಯಾವುದೂ ಇಲ್ಲ. ಹೊಂದಾಣಿಕೆಯೇ ಜೀವನದ ರಹಸ್ಯ, ಜೀವವಿಕಾಸದ ಹಿನ್ನೆಲೆಯಲ್ಲಿರುವುದು ಅದೇ ತತ್ತ್ವ ಅದೇ ಹಿಂದೆ ಇರುವ ನಿಯಮ. ಜೀವನಿಗೂ ಮತ್ತು ಇವನನ್ನು ನಾಶಪಡಿಸಲು ಯತ್ನಿಸುವ ಪರಿಸರಕ್ಕೂ ನಡೆಯುವ ಘರ್ಷಣೆಯಿಂದಲೇ ಹೊಂದಾಣಿಕೆ ಹುಟ್ಟುವುದು. ಯಾರು ಚೆನ್ನಾಗಿ ಹೊಂದಿಕೊಂಡು ಹೋಗುವರೋ ಅವರೇ ದೀರ್ಘಕಾಲ ಬಾಳುವರು. ನಾನು ಇದನ್ನು ಬೋಧಿಸದೇ ಇದ್ದರೂ ಸಮಾಜ ಬದಲಾಗುತ್ತಿದೆ. ಅದು ಬದಲಾಗಬೇಕು. ಕ್ರೈಸ್ತಧರ್ಮವೂ ಅಲ್ಲ, ವಿಜ್ಞಾನವೂ ಅಲ್ಲ ಕೆಲಸ ಮಾಡುತ್ತಿರುವುದು. ಜನರು ಬಾಳಬೇಕು, ಇಲ್ಲವೇ ಅಳಿಯಬೇಕು ಎಂಬ ಉದ್ದೇಶವೇ ಇದರ ಹಿಂದೆ ಕೆಲಸ ಮಾಡುತ್ತಿರುವುದು.

ಪ್ರಪಂಚದ ಶ್ರೇಷ್ಠತಮ ದೃಶ್ಯಗಳನ್ನು ಹಿಮಾಲಯದ ಮೇಲಿನಿಂದ ನೋಡಬಹುದು. ಒಬ್ಬ ಅಲ್ಲಿ ಕೆಲವು ಕಾಲವಿದ್ದರೆ ಮುಂಚೆ ಅವನೆಷ್ಟೇ ಚಂಚಲನಾಗಿದ್ದರೂ ಅವನಿಗೆ ಸ್ವಲ್ಪ ಶಾಂತಿ ದೊರಕುವುದರಲ್ಲಿ ಸಂದೇಹವಿಲ್ಲ.

ಈಶ್ವರನೇ ಶ್ರೇಷ್ಠ ಸಾರ್ವತ್ರಿಕ ನಿಯಮ. ಒಮ್ಮೆ ಈ ನಿಯಮವನ್ನು ಅರಿತರೆ ಅನಂತರ ಉಳಿದವೆಲ್ಲ ಅದಕ್ಕೆ ಅಧೀನ ಎಂದು ಅರಿಯಬಹುದು. ಬೀಳುವ ವಸ್ತುಗಳಿಗೆ ನ್ಯೂಟನ್ನಿನ ಆಕರ್ಷಣ ಸಿದ್ಧಾಂತವಿದ್ದಂತೆ ಧರ್ಮಕ್ಕೆ ದೇವರು.

ಪೂಜೆಯೇ ಪರಮಪ್ರಾರ್ಥನೆ, ಧ್ಯಾನಮಾಡುವುದಕ್ಕೆ ಸಾಧ್ಯವಿಲ್ಲದೆ ಇದ್ದರೆ ಪೂಜೆ ಆವಶ್ಯಕ. ಅವನಿಗೆ ಸ್ಥೂಲವಾದದ್ದು ಏನಾದರೂ ಬೇಕಾಗುವುದು. ಧೀರರು ಮಾತ್ರ ನಿಷ್ಕಪಟಿಗಳಾಗಿರಬಹುದು. ಸಿಂಹವನ್ನೂ ನರಿಯನ್ನೂ ಹೋಲಿಸಿ ನೋಡಿ.

ದೇವರನ್ನು ಮತ್ತು ಪ್ರಕೃತಿಯಲ್ಲಿ ಕೇವಲ ಒಳ್ಳೆಯದನ್ನು ಮಾತ್ರ ಪ್ರೀತಿಸುವುದನ್ನು ಒಂದು ಮಗುವೂ ಮಾಡಬಲ್ಲದು. ನೀನು ಭಯಾನಕವಾಗಿರುವುದನ್ನೂ, ದುಃಖಕರವಾಗಿರುವುದನ್ನೂ ಪ್ರೀತಿಸಬೇಕು. ತಂದೆಯು ಮಗು ತನಗೆ ಕಷ್ಟ ಕೊಡುವಾಗಲೂ ಅದನ್ನು ಪ್ರೀತಿಸುವನು.

ಶ‍್ರೀಕೃಷ್ಣ ಮಾನವಕೋಟಿಯ ಉದ್ಧಾರಕ್ಕೆ ಅವತಾರವೆತ್ತಿ ಬಂದ ದೇವರು; ಗೋಪೀಲೀಲೆ ಪರಾಭಕ್ತಿಯ ಪರಾಕಾಷ್ಠೆ. ಅಲ್ಲಿ ವ್ಯಕ್ತಿತ್ವ ಅಳಿಸಿಹೋಗಿ ದೇವರಲ್ಲಿ ಲಯವಾಗುವುದು. ಈ ಲೀಲೆಯಲ್ಲಿಯೇ ಶ‍್ರೀಕೃಷ್ಣ ಸರ್ವಧರ್ಮಗಳನ್ನೂ ನನಗಾಗಿ ಪರಿತ್ಯಜಿಸಿ ಎಂಬುದುನ್ನು ಬೋಧಿಸಿರುವನು. ಭಕ್ತಿಯನ್ನು ಅರಿಯಬೇಕಾದರೆ ಬೃಂದಾವನ ಲೀಲೆಯಲ್ಲಿ ಶರಣಾಗಿ. ಈ ವಿಷಯದ ಮೇಲೆ ಎಷ್ಟೋ ಗ್ರಂಥಗಳಿವೆ. ಇದೇ ಇಂಡಿಯಾದೇಶದ ಧರ್ಮ. ಹಿಂದೂಗಳಲ್ಲಿ ಬಹುಪಾಲು ಮಂದಿ ಜನರು ಶ‍್ರೀಕೃಷ್ಣನ ಅನುಯಾಯಿಗಳು.

ಶ‍್ರೀಕೃಷ್ಣನು ದೀನರಿಗೆ ದರಿದ್ರರಿಗೆ ಪಾಪಿಗಳಿಗೆ ದೇವರು. ತಂದೆ ತಾಯಿ ಮಗ ಇವರೆಲ್ಲರಿಗೂ ಬೇಕಾದವನು. ಅವನು ನಮ್ಮ ಮಾನವ ಸಂಬಂಧಗಳನ್ನೆಲ್ಲಾ ನಿಕಟವಾಗಿ ಪ್ರವೇಶಿಸಿ ಪ್ರತಿಯೊಂದನ್ನೂ ಪಾವನಗೊಳಿಸಿ, ಕೊನೆಗೆ ಮುಕ್ತಿಯನ್ನು ನೀಡುವನು. ಅವನು ಪಂಡಿತರಿಗೆ ಮತ್ತು ತಾತ್ತ್ವಿಕರಿಗೆ ಕಾಣದೆ ಅವಿತುಕೊಳ್ಳುವನು. ಮಕ್ಕಳಿಗೆ ಮತ್ತು ಅಜ್ಞರಿಗೆ ಕಾಣಿಸಿಕೊಳ್ಳುವನು. ಅವನು ಶ್ರದ್ಧೆಗೆ ಮತ್ತು ಪ್ರೀತಿಗೆ ಒಲಿಯುವ ದೇವರು. ಪಾಂಡಿತ್ಯಕ್ಕೆ ಒಲಿಯುವವನಲ್ಲ. ಗೋಪಿಯರಿಗೆ ಪ್ರೀತಿ ಮತ್ತು ದೇವರು ಎರಡೂ ಒಂದೇ ಆಗಿದ್ದುವು. ಅವರು ಸಾಕ್ಷಾತ್ ಪ್ರೇಮವೇ ಶ‍್ರೀಕೃಷ್ಣ ಎಂದು ಮಾತ್ರ ಅರಿತಿದ್ದರು.

ಶ‍್ರೀಕೃಷ್ಣ ದ್ವಾರಕೆಯಲ್ಲಿ ಕರ್ತವ್ಯವನ್ನು ಬೋಧಿಸುವನು; ಬೃಂದಾವನದಲ್ಲಿ ಪ್ರೀತಿಯನ್ನು ಬೋಧಿಸುವನು. ತನ್ನ ಮಕ್ಕಳು ದುರ್ಜನರಾದುದರಿಂದ ಅವರು ಒಬ್ಬರೊಡನೊಬ್ಬರು ಕಾದಾಡಿ ಸಾಯಲಿಕ್ಕೆ ಬಿಟ್ಟನು.

ಯೆಹೂದ್ಯರ ಮತ್ತು ಮಹಮ್ಮದೀಯರ ಭಾವನೆಯಲ್ಲಿ ದೇವರು ದೊಡ್ಡ ನ್ಯಾಯಾಧಿಪತಿಯಂತೆ ಇರುವನು. ನಮ್ಮ ದೇವರು ನೋಡುವುದಕ್ಕೆ ನಿಷ್ಟುರನು, ಆದರೆ ಅವನ ಒಳಗೆ ಪ್ರೀತಿ ಮತ್ತು ದಯೆ ತುಂಬಿ ತುಳುಕುತ್ತಿವೆ.

ಕೆಲವರಿಗೆ ಅದ್ವೈತ ಗೊತ್ತಾಗುವುದಿಲ್ಲ. ಅವರು ಸುಮ್ಮನೆ ಅದನ್ನು ಅಪಹಾಸ್ಯಕ್ಕೆ ಈಡು ಮಾಡುವರು. ಶುದ್ಧ ಏನು, ಅಶುದ್ಧ ಏನು, ಪಾಪವೇನು ಪುಣ್ಯವೇನು? ಇವೆಲ್ಲಾ ಮೂಢನಂಬಿಕೆ ಎಂದು ಯಾವ ನೀತಿ ನಿಯಮಗಳನ್ನೂ ಅವರು ಪಾಲಿಸುವುದಿಲ್ಲ. ಇದೊಂದು ಠಕ್ಕತನ. ಇಂತಹ ಬೋಧನೆಯಿಂದ ಬೇಕಾದಷ್ಟು ಅನಾಹುತ ಉಂಟಾಗಿದೆ.

ಈ ದೇಹವು ಪಾಪಪುಣ್ಯಗಳೆಂಬ ಎರಡು ಬಗೆಯ ಕರ್ಮಗಳಿಂದ ಆಗಿದೆ. ಒಂದು ಮುಳ್ಳು ನನ್ನ ಕಾಲಿಗೆ ಹೊಕ್ಕಿದ್ದರೆ ನಾನು ಮತ್ತೊಂದು ಮುಳ್ಳನ್ನು ತೆಗೆದುಕೊಂಡು ಅದನ್ನು ಹೊರಗೆ ತೆಗೆಯಬೇಕಾಗಿದೆ. ಅನಂತರ ಎರಡು ಮುಳ್ಳುಗಳನ್ನೂ ಆಚೆಗೆ ಎಸೆಯಬೇಕಾಗಿದೆ. ಮುಕ್ತನಾಗಬೇಕೆಂದಿರುವವನು ಪುಣ್ಯದ ಮುಳ್ಳಿನಿಂದ ಪಾಪದ ಮುಳ್ಳನ್ನು ತೆಗೆದುಹಾಕುವನು. ಅವನು ಮಾಡುವ ಇನ್ನೂ ಜೀವಿಸಿರುವುದರಿಂದ, ಅವನಲ್ಲಿ ಪುಣ್ಯ ಮಾತ್ರ ಇರುವುದರಿಂದ ಅವನು ಮಾಡುವ ಕರ್ಮಗಳು ಪುಣ್ಯ ಕರ್ಮಗಳಾಗಿರುವುವು. ಜೀವನ್ಮುಕ್ತನಲ್ಲಿ ಸ್ವಲ್ಪ ಪುಣ್ಯ ಮಿಕ್ಕಿರುವುದು; ಅದಕ್ಕೆ ಅವನಿನ್ನೂ ಜೀವಿಸಿರುವುದು. ಅವನು ಮಾಡುವುದೆಲ್ಲಾ ಪುಣ್ಯವಾಗಿರಬೇಕು.

ಪುಣ್ಯ ನಮ್ಮನ್ನು ಉದ್ಧರಿಸುವುದು. ಪಾಪ ನಮ್ಮನ್ನು ಅವನತಿಗೆ ಒಯ್ಯುವುದು. ಮಾನವ ಮೂರು ಗುಣಗಳಿಂದಾಗಿರುವನು. ಮೃಗಗುಣ, ಮಾನವಗುಣ, ದೇವರಗುಣಗಳೇ ಅವು. ಯಾವುದು ನಿನ್ನಲ್ಲಿ ಪವಿತ್ರತೆಯನ್ನು ವೃದ್ಧಿ ಮಾಡುವುದೋ ಅದು ಪುಣ್ಯ. ಯಾವುದು ನಿನ್ನಲ್ಲಿ ಮೃಗೀಯ ಗುಣಗಳನ್ನು ವೃದ್ಧಿ ಮಾಡುವುದೋ ಅದೇ ಪಾಪ. ನೀನು ಮೃಗೀಯ ಭಾವನೆಯನ್ನು ನಾಶಮಾಡಿ ಮಾನವನಾಗಬೇಕು; ಅಂದರೆ ನಿನ್ನಲ್ಲಿ ಪ್ರೀತಿ ವಿಶ್ವಾಸ ದಾನ ಮುಂತಾದುವುಗಳು ಇರಬೇಕು. ನೀನು ಅದನ್ನೂ ದಾಟಬೇಕು. ಸಚ್ಚಿದಾನಂದ ಸ್ವರೂಪನಾಗಬೇಕು. ಸುಡದ ಬೆಂಕಿಯಂತೆ ಇರಬೇಕು; ಅದ್ಭುತವಾಗಿ ಪ್ರೇಮಮಯಿಯಾಗಿರಬೇಕು. ಆದರೆ ನಿನ್ನಲ್ಲಿ ಮಾನವ ದುರ್ಬಲತೆ ಇರಕೂಡದು, ದುಃಖವಿರಕೂಡದು.

ಭಕ್ತಿಯನ್ನು ವೈಧೀ ಮತ್ತು ರಾಗಾನುಗಾ ಎಂದು ಭಾಗಗಳಾಗಿ ಮಾಡುವರು. ರಾಗಾನುಗಾ ಭಕ್ತಿಯಲ್ಲಿ ಐದು ವಿಧಗಳಿವೆ. ಅದೇ ಶಾಂತ, ದಾಸ್ಯ, ಸಖ್ಯ, ವಾತ್ಸಲ್ಯ ಮತ್ತು ಮಧುರ.

ಕೇಶವ ಚಂದ್ರಸೇನನು ಸಮಾಜವನ್ನು ಒಂದು ಕಡೆ ವೃತ್ತಕ್ಕೆ ಹೋಲಿಸುವನು. ದೇವರೇ ಮಧ್ಯದಲ್ಲಿರುವ ಸೂರ್ಯ. ಸಮಾಜ ಕೆಲವು ವೇಳೆ ಸೂರ್ಯನಿಗೆ ಹತ್ತಿರದಲ್ಲಿರುವುದು, ಮತ್ತೆ ಕೆಲವು ವೇಳೆ ಸೂರ್ಯನಿಗೆ ದೂರದಲ್ಲಿರುವುದು. ಅವತಾರ ಪುರುಷನು ಸಮಾಜವನ್ನು ದೇವರ ಸಮೀಪಕ್ಕೆ ಒಯ್ಯುವನು. ಪುನಃ ಸಮಾಜ ಹಿಂದೆ ಹೋಗುವುದು, ಇದು ಏತಕ್ಕೆ ಇರಬೇಕು? ನಮ್ಮನ್ನೆಲ್ಲಾ ಪೂರ್ಣಾತ್ಮರನ್ನಾಗಿ ಏತಕ್ಕೆ ಸೃಷ್ಟಿಸಲಿಲ್ಲ? ಇದು ಅವನ ಲೀಲೆ, ನಮಗೆ ಗೊತ್ತಿಲ್ಲ.

ಮಾನವ ಬ್ರಹ್ಮನಾಗಬಲ್ಲ, ದೇವರಾಗಲಾರ. ಯಾರಾದರೂ ದೇವರಾದರೆ ಅವನು ಏನು ಸೃಷ್ಟಿಸಬಲ್ಲ ತೋರಿ. ವಿಶ್ವಾಮಿತ್ರನ ಆಜ್ಞೆಯನ್ನು ಮಾತ್ರ ಪಾಲಿಸುತ್ತಿತ್ತು. ಯಾರಾದರೂ ಮತ್ತೊಬ್ಬನ ನಿಯಮವನ್ನು ಅಲ್ಲಗಳೆಯುವನು. ಈ ಪ್ರಪಂಚ ಎಷ್ಟರ ಮಟ್ಟಿಗೆ ಒಂದನ್ನು ಮತ್ತೊಂದು ಅನುಸರಿಸಿಕೊಂಡಿದೆ ಎಂದರೆ ಅದರಲ್ಲಿ ಒಂದು ಸಣ್ಣ ಅಣುವನ್ನು ಕದಲಿಸಿದರೂ ಸೃಷ್ಟಿಯೇ ನಾಶವಾಗುವುದು.

ಹಿಂದೆ ಮಹಾಪುರುಷರಿದ್ದರು. ಅವರು ನಮಗಿಂತ ಎಷ್ಟು ಮಹಾಮಹಿಮರಾಗಿದ್ದರೋ ಅದನ್ನು ಯಾವ ಗಣಿತವೂ ಹೇಳುವ ಸ್ಥಿತಿಯಲ್ಲಿರಲಿಲ್ಲ. ಆದರೆ ಅವರನ್ನು ದೇವರೊಡನೆ ಹೋಲಿಸಿದರೆ ಅವರು ಬರಿಯ ಚುಕ್ಕೆಯಂತೆ ಇದ್ದರು, ಅನಂತದೊಂದಿಗೆ ಹೋಲಿಸಿದರೆ ಯಾವುದೇ ಕೆಲಸಕ್ಕೆ ಬರುವುದಿಲ್ಲ. ದೇವರೊಡನೆ ಹೋಲಿಸಿದರೆ ವಿಶ್ವಾಮಿತ್ರ ಯಾರು? ಬರಿಯ ಒಂದು ಮಾನವ ಚಿಟ್ಟೆ.

ಪತಂಜಲಿಯೇ ಭೌತಿಕ ಮತ್ತು ಆಧ್ಯಾತ್ಮಿಕ ಪರಿಣಾಮವಾದಕ್ಕೆ ಪಿತನಂತೆ ಇರುವನು. ಸಾಧಾರಣವಾಗಿ ಒಂದು ದೇಹ ಸುತ್ತಲಿರುವ ವಾತಾವರಣಕ್ಕಿಂತ ದುರ್ಬಲವಾಗಿರುವುದು. ದೇಹ ಹೊಂದಿಕೊಳ್ಳಲು ವಾತಾವರಣದೊಡನೆ ಹೋರಾಡುತ್ತಿರುವುದು. ಕೆಲವು ವೇಳೆ ದೇಹ ಅತಿಯಾಗಿ ಹೊಂದಿಕೊಳ್ಳುವುದು. ಆಗ ದೇಹ ಬದಲಾಗಿ ಬೇರೊಂದು ರೂಪವನ್ನು, ಧರಿಸುವುದು. ನಂದಿ ಒಬ್ಬ ಮನುಷ್ಯನಾಗಿದ್ದ. ಆದರೆ ಅವನಲ್ಲಿ ಎಷ್ಟೊಂದು ಪವಿತ್ರತೆ ಇತ್ತೆಂದರೆ ಅದನ್ನು ವ್ಯಕ್ತಪಡಿಸುವ ಶಕ್ತಿ ಅವನ ದೇಹಕ್ಕಿರಲಿಲ್ಲ. ಆದಕಾರಣವೇ ಅವನ ದೇಹ ದೇವನ ದೇಹವಾಗಿ ಬದಲಾಯಿತು.

ಸ್ಪರ್ಧೆ ಎಂಬ ಪ್ರಚಂಡ ಯಂತ್ರ ಎಲ್ಲವನ್ನೂ ನಾಶ ಮಾಡುವುದು. ನೀವೇನಾದರೂ ಬದುಕಬೇಕಾದರೆ ನೀವು ಕಾಲಕ್ಕೆ ಹೊಂದಿಕೊಂಡು ಹೋಗಬೇಕು. ನಾವೇನಾದರೂ ಜೀವಂತವಾಗಿರಬೇಕಾದರೆ ವೈಜ್ಞಾನಿಕ ಜನಾಂಗವಾಗಿರಬೇಕು. ಬುದ್ಧಿಯೇ ಒಂದು ಶಕ್ತಿ. ಐರೋಪ್ಯರ ಸಂಘಟನಾಶಕ್ತಿಯನ್ನು ನೀವು ಕಲಿಯಬೇಕು. ನೀವು ವಿದ್ಯಾವಂತರಾಗಬೇಕು. ನಿಮ್ಮ ವಿದ್ಯಾಭ್ಯಾಸವನ್ನು ಕೊಡಬೇಕು. ಬಾಲ್ಯ ವಿವಾಹವನ್ನು ನೀವು ವಿರೋಧಿಸಬೇಕು.

ಮೇಲಿನ ಭಾವನೆಗಳೆಲ್ಲ ಸಮಾಜದಲ್ಲಿ ಸಂಚರಿಸುತ್ತಿವೆ. ಇದು ನಿಮಗೆಲ್ಲ ಗೊತ್ತಿದೆ. ಆದರೂ ಅದನ್ನು ಕಾರ್ಯರೂಪಕ್ಕೆ ತರಲು ಧೈರ್ಯವಿಲ್ಲ. ಬೆಕ್ಕಿಗೆ ಗಂಟೆ ಕಟ್ಟುವವರಾರು? ಎಂದಾದರೊಂದು ದಿನ ಒಬ್ಬ ಮಹಾಪುರುಷ ಬರುವನು. ಆಗ ಇಲಿಗಳೆಲ್ಲ ಧೈರ್ಯಶಾಲಿಗಳಾಗುವುವು.

ಮಹಾಪುರುಷನೊಬ್ಬ ಯಾವಾಗಲಾದರೂ ಅವತರಿಸಿದರೆ ಆಗ ಸಮಯ ಸನ್ನಿಹಿತವಾಗಿರುವುದು. ಘಟನೆ ಪ್ರಾರಂಭವಾಗುವುದಕ್ಕೆ, ಅವನು ಒಂದು ನಿಮಿತ್ತ ಮಾತ್ರ. ಮದ್ದಿಗೆ ತಾಕುವ ಒಂದು ಕಿಡಿಯಂತೆ. ನಾವು ಮಾತನಾಡುವುದರಲ್ಲಿಯೂ ಒಂದು ಪ್ರಯೋಜನವಿದೆ. ನಾವೂ ಆ ಮಹಾಪುರುಷನ ಆಗಮನಕ್ಕೆ ಅಣಿಯಾಗುವೆವು.

ಕೃಷ್ಣ ತಂತ್ರಗಾರನೇ? ಅಲ್ಲ, ಅವನು ಯುದ್ಧವನ್ನು ನಿಲ್ಲಿಸಲು ಸಾಕಷ್ಟು ಪ್ರಯತ್ನಪಟ್ಟನು. ಯುದ್ಧಕ್ಕೆ ಕಾಲುಕೆರೆದವನು ದುರ್ಯೋಧನನೆ; ಆದರೆ ಒಮ್ಮೆ ನೀನು ಒಂದರಲ್ಲಿ ನಿರತನಾದ ಮೇಲೆ ಅಲ್ಲಿಂದ ಹಿಂದಿರುಗುವಂತಿಲ್ಲ. ಇದು ಕಾರ್ಯತತ್ಪರನ ಲಕ್ಷಣ. ಓಡಿ ಹೋಗಬೇಡಿ, ಇದು ಹೇಡಿತನ. ನೀವು ಕೆಲಸದ ಮಧ್ಯೆ ಇರುವಾಗ ಅದನ್ನು ಮಾಡಬೇಕು. ನೀವು ಸ್ವಲ್ಪವೂ ಹಿಂದೆ ಸರಿಯಕೂಡದು. ನಿಜವಾದ ಕರ್ತವ್ಯದಿಂದ ಹಿಮ್ಮುಖರಾಗಕೂಡದು. ಇದೊಂದು ಧರ್ಮಯುದ್ಧ.

ಸೈತಾನ ಹಲವು ಆಕಾರಗಣನ್ನು ತಾಳಿಬರುವನು. ಕೋಪ ನ್ಯಾಯದಂತೆ ಬರುವುದು, ಕಾಮ ಕರ್ತವ್ಯದ ಸೋಗಿನಲ್ಲಿ ಬರುವುದು. ಇದು ಮೊದಲು ಬಂದಾಗ ಅರಿವಾಗುವುದು. ಆದರೆ ಅನಂತರ ಮನುಷ್ಯ ಮರೆಯುವನು. ಇವೆಲ್ಲ ನಿಮ್ಮ ವಕೀಲನ ಹೃದಯದಂತೆ. ಮೊದಲ ಕಕ್ಷಿ ಹೇಳುವುದೆಲ್ಲ ಸುಳ್ಳು ಎಂದು ಅವನಿಗೆ ಗೊತ್ತಿರುವುದು. ಅನಂತರ ಕಕ್ಷಿಗಳ ಬಗ್ಗೆ ಇದು ತಮ್ಮ ಕರ್ತವ್ಯವೆಂದು ತಿಳಿದು ಅವರು ನ್ಯಾಯಾನ್ಯಾಯವನ್ನು ಗಣನೆಗೇ ತರುವುದಿಲ್ಲ.

ನರ್ಮದಾನದಿಯ ತೀರದಲ್ಲಿ ಯೋಗಿಗಳು ವಾಸಿಸುವರು. ಅಲ್ಲಿ ಹವೆ ಹಿತಕರವಾಗಿರುವುದು. ಭಕ್ತರು ವೃಂದಾವನದಲ್ಲಿರುವರು.

ಸಿಪಾಯಿಗಳು ಬೇಗ ಮೃತ್ಯುವಶರಾಗುವರು. ಪ್ರಕೃತಿಯಲ್ಲಿ ಎಷ್ಟೋ ಕುಂದುಗಳಿವೆ. ಪೈಲ್ವಾನ ಬೇಗ ಸಾಯುವನು. ಉತ್ತಮ ಮನುಷ್ಯರು \enginline{(Gentlemen)} ಬಲಶಾಲಿಗಳು, ಆದರೆ ಬಡವರು ಕಷ್ಟಸಹಿಷ್ಣುಗಳು. ಮಲಬದ್ಧತೆಯವನಿಗೆ ಹಣ್ಣು ಒಳ್ಳೆಯದು. ಮಿದುಳಿನ ಕೆಲಸ ಮಾಡಬೇಕಾದರೆ ಹೆಚ್ಚು ವಿಶ್ರಾಂತಿ ಬೇಕು. ಆಹಾರದಲ್ಲಿ ವ್ಯಂಜನಾದಿಗಳನ್ನು ಉಪಯೋಗಿಸಬೇಕು.

ಕಾಡುಮನುಷ್ಯರು ದಿನಕ್ಕೆ ನಲವತ್ತು ಐವತ್ತು ಮೈಲುಗಳಷ್ಟು ಬೇಕಾದರೆ ನಡೆಯುತ್ತಾರೆ. ಅವರು ಎಂತಹ ಆಹಾರವನ್ನಾದರೂ ತಿನ್ನಬಲ್ಲರು.

ನಮ್ಮ ಹಣ್ಣುಗಳೆಲ್ಲ ಕೃತಕ, ಸ್ವಾಭಾವಿಕವಾದ ಯಾವ ಮಾವಿನಹಣ್ಣು ಕೂಡ ಅಷ್ಟೇನೂ ರುಚಿಯಿಲ್ಲ. ಗೋಧಿ ಕೂಡ ಕೃತಕ.

ಬ್ರಹ್ಮಚರ್ಯದಿಂದ ಆತ್ಮಶಕ್ತಿಯನ್ನು ನಿಮ್ಮ ದೇಹದಲ್ಲಿ ಸಂಗ್ರಹಿಸಿ. ಗೃಹಸ್ಥ ತನ್ನ ವರಮಾನವನ್ನು ಹೀಗೆ ವ್ಯಯಮಾಡಬೇಕು; ಕಾಲುಭಾಗ ಮನೆಗೆ, ಕಾಲುಭಾಗ ಧರ್ಮಕ್ಕೆ, ಕಾಲುಭಾಗ ಆಪತ್ಕಾಲದ ಸಂಗ್ರಹ, ಕಾಲುಭಾಗ ತನಗಾಗಿ.

ಅನೇಕತೆಯಲ್ಲಿ ಏಕತೆಯೇ ಸಾರ್ವತ್ರಿಕತೆಯಲ್ಲಿ ವೈಯಕ್ತಿಕತೆಯೇ ಸೃಷ್ಟಿಯ ವಿಧಾನ.

ಕಾರಣವನ್ನು ಮಾತ್ರ ಏತಕ್ಕೆ ಅಲ್ಲಗಳೆಯುತ್ತೀರಿ? ಪರಿಣಾಮವನ್ನೂ ಅಲ್ಲಗಳೆಯಿರಿ. ಪರಿಣಾಮದಲ್ಲಿರುವುದೆಲ್ಲ ಕಾರಣದಲ್ಲಿ ಇರಬೇಕು.

ಯೇಸುವು ಸಮಾಜದಲ್ಲಿ ಕೆಲಸಮಾಡಿದ್ದು ಹದಿನೆಂಟು ತಿಂಗಳು ಮಾತ್ರ. ಇದಕ್ಕಾಗಿ ಮೂವತ್ತೆರಡು ತಿಂಗಳು ಮೌನವಾಗಿ ಅವನು ಸಾಧನೆ ಮಾಡುತ್ತಿದ್ದ. ಮಹಮ್ಮದ್ ಸಮಾಜದಲ್ಲಿ ಕೆಲಸಮಾಡಲು ಯತ್ನಿಸಿದಾಗ ನಲವತ್ತು ವರುಷಗಳಾಗಿದ್ದುವು.

ಸಮಾಜದಲ್ಲಿ ಜಾತಿ ಸ್ವಾಭಾವಿಕವಾಗಿ ಆಗುವುದು ನಿಜ. ಯಾರಿಗೆ ಒಂದು ಬಗೆಯ ಕೆಲಸ ಮಾಡುವುದರಲ್ಲಿ ಅಭಿರುಚಿ ಇರುವುದೋ ಅವರೆಲ್ಲಾ ಒಂದು ಪಂಗಡವಾಗುವರು. ಆದರೆ ಒಬ್ಬ ವ್ಯಕ್ತಿಯ ಜಾತಿಯನ್ನು ಯಾರು ನಿಷ್ಕರ್ಷಿಸುವರು? ಬ್ರಾಹ್ಮಣ ತನಗೆ ಅಧ್ಯಾತ್ಮದಲ್ಲಿ ವಿಶೇಷ ಅಭಿರುಚಿ ಇದೆ ಎಂದು ಭಾವಿಸಿದರೆ ಶೂದ್ರನೊಂದಿಗೆ ಏತಕ್ಕೆ ಆ ಕ್ಷೇತ್ರದಲ್ಲಿ ನಿಲ್ಲಲು ಅಂಜಬೇಕು? ಕುದುರೆಯು ಕತ್ತೆಯೊಂದಿಗೆ ಓಡುವುದಕ್ಕೆ ಅಂಜುವುದೇ?

ಕೃಷ್ಣ –ಕರ್ಣಾಮೃತದ ಕರ್ತೃವಾದ ಬಿಲ್ವಮಂಗಲನ ಜೀವನವನ್ನು ಓದಿ ನೋಡಿ. ತಾನು ದೇವರನ್ನು ನೋಡಲಿಲ್ಲವಲ್ಲ ಎಂದು ತನ್ನ ಕಣ್ಣುಗಳನ್ನು ಕಿತ್ತು ಹಾಕಿದನು. ಅಯೋಗ್ಯವಾದ ದಾರಿಗೆ ತಿರುಗಿದ ಪ್ರೇಮ ಕೂಡ ಕೊನೆಗೆ ಒಳ್ಳೆಯ ದಾರಿಯನ್ನು ಹಿಡಿಯುವುದು ಎಂಬುದಕ್ಕೆ ಅವನ ಜೀವನ ಒಂದು ಉದಾಹರಣೆಯಾಗಿದೆ.

ಹಿಂದೂಗಳು ಅತಿ ಹಿಂದೆಯೇ ಆಧ್ಯಾತ್ಮಿಕ ಪ್ರಪಂಚದಲ್ಲಿ ಪ್ರಗತಿಪರರಾದರು ಮತ್ತು ಆದರ್ಶಗಳಲ್ಲಿ ಆಸಕ್ತರಾದರು. ಅದೇ ಅವರ ಈಗಿನ ಸ್ಥಿತಿಗೆ ಕಾರಣ. ಹಿಂದೂಗಳು ಪಾಶ್ಚಾತ್ಯರಿಂದ ಪ್ರಪಂಚಕ್ಕೆ ಬೇಕಾದ ಸ್ವಲ್ಪ ವಿಷಯಗಳನ್ನು ಕಲಿಯಬೇಕು; ಮತ್ತು ಪಾಶ್ಚಾತ್ಯರಿಗೆ ಸ್ವಲ್ಪ ಆಧ್ಯಾತ್ಮಿಕ ವಿಷಯಗಳನ್ನು ಹೇಳಿಕೊಡಬೇಕು.

ಮೊದಲು ನಿಮ್ಮ ಸ್ತ್ರೀಯರನ್ನು ವಿದ್ಯಾವಂತರನ್ನಾಗಿ ಮಾಡಿ, ಅವರಿಗೆ ಸ್ವಾತಂತ್ರ್ಯವನ್ನು ಕೊಡಿ. ಅನಂತರ ಯಾವ ಸುಧಾರಣೆ ತಮಗೆ ಆವಶ್ಯಕ ಎಂಬುದನ್ನು ಅವರೇ ಹೇಳುವರು. ಅವರಿಗೆ ಸಂಬಂಧಪಟ್ಟ ವಿಷಯಗಳನ್ನು ನಿಶ್ಚಯಿಸಲು ನೀವಾರು?

ಝಾಡಮಾಲಿಗಳನ್ನು ಮತ್ತು ಹೊಲೆಯರನ್ನು ಈಗಿರುವ ಹೀನ ಸ್ಥಿತಿಗೆ ಯಾರು ತಂದರು? ನಮ್ಮ ನಡತೆಯ ನಿರ್ದಯತೆಯೇ ಕಾರಣ. ಜೊತೆಗೆ ಅದ್ಭುತವಾದ ಅದ್ವೈತವನ್ನು ಬೇರೆ ಬೋಧಿಸುವುದು! ಇದು ಗಾಯಕ್ಕೆ ಉಪ್ಪು ನೀರನ್ನು ಹಾಕಿದಂತಲ್ಲವೆ?

ಸಾಕಾರ ನಿರಾಕಾರಗಳೆರಡೂ ಜೀವನದಲ್ಲಿ ಹಾಸುಹೊಕ್ಕಾಗಿವೆ. ನಿರಾಕಾರವನ್ನು ಸಾಕಾರದೊಂದಿಗೆ ಮಾತ್ರ ನಾವು ಆಲೋಚಿಸಬಹುದು. ಸಾಕಾರವನ್ನು ನಿರಾಕಾರದೊಂದಿಗೆ ಮಾತ್ರ ನಾವು ಆಲೋಚಿಸಬಹುದು. ಈ ಪ್ರಪಂಚ ನಮ್ಮ ಆಲೋಚನೆಯ ಮೂರ್ತರೂಪ. ಧರ್ಮದ ಮೂರ್ತರೂಪವೇ ವಿಗ್ರಹ.

ದೇವರಿಗೆ ಎಲ್ಲಾ ಸ್ವಭಾವಗಳೂ ಸಾಧ್ಯ. ಆದರೆ ನಾವು ತಂದೆಯನ್ನು ಮತ್ತು ಮಗನನ್ನು ಪ್ರೀತಿಸುವಂತೆ ಅವನನ್ನು ಪ್ರೀತಿಸಬಹುದು. ಪ್ರಪಂಚದಲ್ಲಿ ಅತಿ ತೀವ್ರವಾಗಿರುವ ಪ್ರೀತಿಯೇ ಸ್ತ್ರೀಪುರುಷರೊಳಗೆ ಇರುವ ಪ್ರೀತಿ. ಅದರಲ್ಲಿರೂ ಪರಸತಿಯ ಮೇಲೆ, ಪರಪುರುಷನ ಮೇಲೆ ಇರುವ ಪ್ರೀತಿ ಅದಕ್ಕಿಂತಲೂ ತೀವ್ರವಾಗಿರವುದು. ನಾವು ಇದನ್ನು ರಾಧಾಕೃಷ್ಣರ ಪ್ರೇಮದಲ್ಲಿ ನೋಡಬಹುದು.

ವೇದದಲ್ಲಿ ಎಲ್ಲಿಯೂ ಮಾನವ ಜನ್ಮತಃ ಪಾಪಿ ಎಂದು ಹೇಳಿಲ್ಲ. ಹಾಗೆ ಹೇಳುವುದು ಮಾನವ ಸ್ವಭಾವಕ್ಕೆ ಅನ್ಯಾಯ ಮಾಡಿದಂತೆ.

ಸತ್ಯವನ್ನು ಹೇಗಿದೆ ಹಾಗೆ ನೋಡುವುದು ಅಷ್ಟು ಸುಲಭವಲ್ಲ. ಅಂದಿನ ದಿನ ಚಿತ್ರದಲ್ಲಿ ಬೆಕ್ಕು ಬಹಳ ಪ್ರಮುಖವಾದ ಸ್ಥಾನದಲ್ಲಿದ್ದರೂ ಅದು ಎಲ್ಲಿರುವುದೆಂದು ಕಂಡುಹಿಡಿಯಲು ಯಾರಿಗೂ ಸಾಧ್ಯವಾಗಲಿಲ್ಲ.

ನೀನು ಇನ್ನೊಬ್ಬರಿಗೆ ತೊಂದರೆ ಕೊಟ್ಟು ಸುಮ್ಮನೆ ಕುಳಿತುಕೊಳ್ಳಲಾರೆ. ಕರ್ಮ ಒಂದು ಅದ್ಭುತವಾದ ಯಂತ್ರ. ದೇವರು ನಿನಗೆ ಕೊಡುವ ಶಿಕ್ಷೆಯನ್ನು ಅನುಭವಿಸಲೇ ಬೇಕಾಗುವುದು.

ಕಾಮಕ್ಕೆ ಕಣ್ಣಿಲ್ಲ. ಅದು ನರಕಕ್ಕೆ ಒಯ್ಯುವುದು. ಪ್ರೇಮವೆ ವಿಶ್ವಾಸ. ಅದು ಸ್ವರ್ಗಕ್ಕೆ ಒಯ್ಯುವುದು. ರಾಧಾಕೃಷ್ಣರ ಪ್ರೇಮದಲ್ಲಿ ಕಾಮವಿಲ್ಲ. ರಾಧೆ ಶ‍್ರೀಕೃಷ್ಣನಿಗೆ 'ನೀನು ನಿನ್ನ ಪಾದಕಮಲಗಳನ್ನು ನನ್ನ ಹೃದಯದ ಮೇಲೆ ಇಟ್ಟರೆ ನನ್ನ ಕಾಮವೆಲ್ಲ ನಾಶವಾಗುವುದು' ಎಂದು ಹೇಳುವಳು. ನಾವು ಯಾವಾಗ ಆಕಾರವನ್ನು ಮೀರಿ ಆತ್ಮ ಭಾವನೆಗೆ ಹೋಗುವೆವೋ ಆಗ ಕಾಮ ನಾಶವಾಗುವುದು; ಪ್ರೇಮ ಒಂದೇ ಉಳಿಯುವುದು. ಕವಿಯೊಬ್ಬ ಅಗಸಗಿತ್ತಿಯನ್ನು ಪ್ರೀತಿಸುತ್ತಿದ್ದ. ಅವಳ ಕಾಲಮೇಲೆ ಕುದಿಯುತ್ತಿರುವ ಸಾರು ಬಿತ್ತು. ಅದಕ್ಕೆ ಕವಿಯ ಕಾಲು ಸುಟ್ಟುಹೋಯಿತು!

ಶಿವನು ಭಗವಂತನ ಘನಗಂಭೀರವಾದ ಭಾವ. ಶ‍್ರೀಕೃಷ್ಣನು ಭಗವಂತನ ಸೌಂದರ್ಯದ ಅಂಶ. ಪ್ರೇಮವು ನೀಲಿ ಬಣ್ಣವನ್ನು ತಾಳುವುದು. ನೀಲಿಯು ತೀವ್ರ ಪ್ರೇಮವನ್ನು ಸೂಚಿಸುವುದು. ಸಾಲೋಮನ್ ಕೃಷ್ಣನನ್ನು ನೋಡಿದ. ಆದರೆ ಭರತಖಂಡದಲ್ಲಿ ಕೃಷ್ಣನು ಎಲ್ಲರಿಗೂ ಕಾಣಿಸಿಕೊಳ್ಳಲು ಬಂದನು. ಈಗಲೂ ನಿಮಗೆ ಪ್ರೇಮವಿದ್ದರೆ ರಾಧೆಯನ್ನು ನೋಡಬಲ್ಲಿರಿ. ರಾಧೆಯಾಗಿ ಮುಕ್ತಿ ಹೊಂದಿ, ಬೇರೆ ಮಾರ್ಗವೇ ಇಲ್ಲ. ಕ್ರೈಸ್ತರಿಗೆ ಸಾಲೋಮನ್ನನ ಹಾಡು ಅರ್ಥವಾಗಲಿಲ್ಲ. ಅದು ಕ್ರಿಸ್ತನಿಗೆ ಚರ್ಚಿನ ಮೇಲೆ ಇರುವ ಪ್ರೀತಿಯ ಒಂದು ಸಂಕೇತವಾಗಿರುವ ಭವಿಷ್ಯವಾಣಿ ಎಂದು ಅವರು ಹೇಳುವರು. ಹಿಂದೂಗಳು ಬುದ್ಧನನ್ನು ಅವತಾರವೆಂದು ನಂಬುವರು. ಹಿಂದೂಗಳು ನಿಸ್ಸಂಶಯವಾಗಿ ದೇವರಲ್ಲಿ ನಂಬುವರು. ಬೌದ್ಧರು ದೇವರಿರುವನೇ ಇಲ್ಲವೇ ಎಂಬುದನ್ನು ಅರಿಯುವುದಕ್ಕೆ ಪ್ರಯತ್ನಿಸುವುದಿಲ್ಲ.

ಒಳ್ಳೆಯವರಾಗಿ, ಕಾಮವನ್ನು ಕೊಲ್ಲಿ ಎಂದು ನಮ್ಮನ್ನು ಅನುಷ್ಠಾನ ಪ್ರಪಂಚದಲ್ಲಿ ಜಾಗೃತರನ್ನಾಗಿ ಮಾಡಲು ಬುದ್ಧ ಬಂದ. ಆಗ ದ್ವೈತ ನಿಜವೇ, ಅದ್ವೈತ ನಿಜವೇ, ಇರುವುದು ಏಕವೇ ಅನೇಕವೇ ಎಂಬುದು ನಮಗೆ ಗೊತ್ತಾಗುವುದು. ಬುದ್ಧನು ಹಿಂದೂಧರ್ಮದ ಸುಧಾರಕ.

ಒಬ್ಬನೇ ಮನುಷ್ಯ ತಾಯಿಗೆ ಒಂದು ಬಗೆಯಾಗಿ ಕಾಣಿಸುವನು. ಹೆಂಡತಿಗೆ ಇನ್ನೊಂದು ಬಗೆಯಾಗಿ ಕಾಣಿಸುವನು. ದುರ್ಜನರು ದೇವರಲ್ಲಿ ದೌರ್ಜನ್ಯವನ್ನು ನೋಡುವರು. ಸಜ್ಜನರು ದೇವರಲ್ಲಿ ಸೌಜನ್ಯವನ್ನು ನೋಡುವರು. ದೇವರನ್ನು ಯಾವ ಆಕಾರದಲ್ಲಿ ಬೇಕಾದರೂ ನೋಡಬಹುದು. ಪ್ರತಿಯೊಬ್ಬರ ಭಾವನೆಗೆ ತಕ್ಕಂತೆ ಅವನು ರೂಪವನ್ನು ಧರಿಸುವನು. ನೀರು ಹಲವು ಪಾತ್ರೆಗಳಲ್ಲಿ ಹಲವು ಆಕಾರಗಳನ್ನು ತಾಳುವುದು. ಎಲ್ಲದರಲ್ಲಿಯೂ ನೀರೇ ಇರುವುದು. ಆದಕಾರಣ ಎಲ್ಲಾ ಧರ್ಮಗಳೂ ಸತ್ಯವೇ.

ದೇವರು ಕ್ರೂರಿ, ಆದರೂ ಕ್ರೂರಿಯಲ್ಲ. ಅವನು ಎಲ್ಲಾ ವಸ್ತುಗಳಲ್ಲಿಯೂ ಇರುವನು, ಅವನು ಯಾವ ವಸ್ತುವೂ ಅಲ್ಲ. ಅವನೊಂದು ವಿರೋಧಾಭಾಸದ ಕಂತೆ. ಪ್ರಕೃತಿ ಕೂಡ ಅಷ್ಟೆ, ಒಂದು ವಿರೋಧಾಭಾಸದ ಹೊರೆ.

ಸ್ವತಂತ್ರೇಚ್ಛೆ–ಅಂದರೆ ನಿಮಗೆ ತೋಚಿದುದನ್ನು ಮಾಡುವುದಕ್ಕೆ ಸ್ವಾತಂತ್ರ್ಯ ಇರುವುದು. ಆದರೆ ಈ ಸ್ವಾತಂತ್ರ್ಯ ಒಂದು ಆವಶ್ಯಕತೆಯಿಂದ ಜನಿಸಿದುದು. ಆಲೋಚನೆ ಮತ್ತು ಕರ್ಮ–ಇವುಗಳ ಆದಿ ಮಧ್ಯ ಮತ್ತು ಅಂತ್ಯದಲ್ಲಿ ಒಂದು ಅನಂತವಾದ ಸರಪಳಿ ಇದೆ. ಆಲೋಚನೆಯನ್ನೇ ನಾವು ಸ್ವಾತಂತ್ರ್ಯ ಎನ್ನುವುದು. ಅದು ಒಂದು ಹಕ್ಕಿಯು ಬೆಳಕಿನ ಕೋಣೆಯ ಮೂಲಕ ಹಾದುಹೋದಂತೆ. ನಮಗೆ ಸ್ವಾತಂತ್ರ್ಯವಿದೆ ಎಂದು ಭಾವಿಸುವೆವು. ಅದಕ್ಕೆ ಮತ್ತಾವ ಕಾರಣವೂ ಇಲ್ಲ ಎಂದು ತಿಳಿಯುವೆವು. ನಮ್ಮ ಪ್ರಜ್ಞೆಗೆ ಅತೀತವಾಗಿ ನಾವು ಹೋಗಲಾರೆವು. ಆದುದರಿಂದ ನಾವು ಸ್ವತಂತ್ರರು ಎಂದು ಭಾವಿಸುತ್ತೇವೆ. ದೇವರೊಬ್ಬನು ಮಾತ್ರ ಸ್ವಾತಂತ್ರ್ಯವನ್ನು ಅನುಭವಿಸಬಲ್ಲ. ಮಹಾಪುರುಷರು ದೇವರಲ್ಲಿ ಏಕತಾಭಾವನೆಯನ್ನು ಪಡೆದಿರುವುದರಿಂದ ಅವರೂ ಕೂಡ ಸ್ವಾತಂತ್ರ್ಯವನ್ನು ಅನುಭವಿಸುವರು.

ಒಂದು ಕೊಳದಿಂದ ಹೊರಗೆ ಹರಿಯುವ ನೀರನ್ನು ತಡೆದು ಅದು ಹೊರಗೆ ಹರಿಯದಂತೆ ನೀವು ಮಾಡಬಹುದು. ಇದಕ್ಕಿಂತ ಹೆಚ್ಚು ಸ್ವಾತಂತ್ರ್ಯ ನಿಮಗೆ ಇಲ್ಲ. ಅದರ ಮೂಲ ಹಾಗೆಯೇ ಇರುವುದು. ಎಲ್ಲವೂ ಪೂರ್ವನಿಶ್ಚಿತವಾಗಿದೆ. ಆ ಪೂರ್ವ ನಿಶ್ಚಿತವಾದುದರಲ್ಲಿ ಸ್ವಲ್ಪ ಭಾಗವೆಂದರೆ ನನಗೆ ಸ್ವಾತಂತ್ರ್ಯವಿದೆ ಎಂದು ಭಾವಿಸುವುದು. ನಾನೇ ನನ್ನ ಕರ್ಮಗಳನ್ನು ಮಾಡುತ್ತಿರುವೆ. ಅದರ ಪ್ರತಿಕ್ರಿಯೆಯೇ ಜವಾಬ್ದಾರಿ. ಎಲ್ಲಿಯೂ ನಿರಂಕುಶ ಅಧಿಕಾರವಿಲ್ಲ. ಅದರಿಂದ ನಾನು ಕೆಲಸ ಮಾಡುತ್ತಿರುವೆ ಎಂದು ಮನುಷ್ಯ ಭಾವಿಸುವನು. ಈ ಭಾವನೆಯನ್ನೇ ನಾವು ಸ್ವಾತಂತ್ರ್ಯ ಎನ್ನುವುದು. ಎಲ್ಲಿ ಈ ಅಧಿಕಾರವಿರುವುದೋ ಅಲ್ಲಿ ಜವಾಬ್ದಾರಿ ಇರುವುದು. ಪ್ರಾರಬ್ಧದಿಂದ ಪ್ರೇರಿತರಾಗಿ ನಾವು ಏನನ್ನು ಮಾಡಿದರೂ ಅದರ ಪ್ರತಿಕ್ರಿಯೆಯನ್ನು ಅನುಭವಿಸಬೇಕು. ಯಾರೇ ಎಸೆದಿರಲಿ, ಚೆಂಡು ಪ್ರತಿಕ್ರಿಯೆಗೆ ಗುರಿಯಾಗುತ್ತದೆ.

ಜನ್ಮದತ್ತವಾಗಿ ಬಂದಿರುವ ಸ್ವಾತಂತ್ರ್ಯವೆಂಬ ಈ ಆವಶ್ಯಕತೆಯ ಸುತ್ತಮುತ್ತಲೂ ನಾವು ಮಾಡಿಕೊಳ್ಳುವ ಸಂಬಂಧಗಳ ಮೇಲೆ ಯಾವ ಪರಿಣಾಮವನ್ನೂ ಉಂಟುಮಾಡುವುದಿಲ್ಲ. ಸಾಪೇಕ್ಷತೆಯಲ್ಲಿ ವ್ಯತ್ಯಾಸವಾಗುವುದಿಲ್ಲ. ಪ್ರತಿಯೊಬ್ಬರೂ ಸ್ವತಂತ್ರರು ಇಲ್ಲವೇ ಯಾವುದಕ್ಕೊ ಸಮವಾಗಿ ಉಳಿದಿರುವುದು. ಪಾಪ ಪುಣ್ಯ ಇದ್ದೇ ಇರುವುವು. ಕಳ್ಳ ತಾನು ನಿರ್ವಾಹವಿಲ್ಲದೆ ಕದಿಯಬೇಕಾಯಿತು ಎಂದರೆ ನ್ಯಾಯಾಧಿಪತಿ ನಿರ್ವಾಹವಿಲ್ಲದೆ ಶಿಕ್ಷಿಸಬೇಕಾಯಿತು ಎನ್ನುವನು. ನಾವು ಒಂದು ರೈಲಿನಲ್ಲಿ ಕುಳಿತುಕೊಂಡಿರುವೆವು. ಇಡೀ ರೈಲು ಸಂಚರಿಸುತ್ತಿದ್ದರೂ ಒಳಗೆ ಕುಳಿತಿರುವವರ ಸಂಬಂಧ ಹಿಂದೆ ಇದ್ದಂತೆಯೇ ಇರುವುದು. ಕಾರ್ಯ ಕಾರಣಗಳ ಅನಂತ ಪರಿಣಾಮದಿಂದ ಪಾರಾಗುವುದೇ ಮುಕ್ತಿ. ಮುಕ್ತರಿಗೆ ಯಾವ ಆವಶ್ಯಕತೆಯೂ ಇಲ್ಲ. ಅವರು ದೇವರಂತೆ. ಆವಶ್ಯಕತೆಯೇ ಕಾರ್ಯಕಾರಣಗಳ ಸಂಬಂಧಕ್ಕೆ ಮೂಲ. ದೇವರೊಬ್ಬನೇ ಸ್ವತಂತ್ರನು. ಅವನೇ ಇಚ್ಛೆಗೆ ಮೂಲ; ಮುಕ್ತಾತ್ಮರಿಗೆ ಇದು ಗೊತ್ತು.

ಆವಶ್ಯಕತೆಯೇ ನಿಜವಾದ ಪ್ರಾರ್ಥನೆ, ಬರಿಯ ಮಾತಲ್ಲ. ನಿಮ್ಮ ನಿಮ್ಮ ಪ್ರಾರ್ಥನೆ ಈಡೇರಿದೆಯೇ ಇಲ್ಲವೇ ಎಂಬುದನ್ನು ನೋಡಬೇಕಾದರೆ ತಾಳ್ಮೆ ಇರಬೇಕು.

ನಿಮ್ಮ ಪಾಲಿಗೆ ಬಂದ ಕರ್ತವ್ಯವನ್ನು ನೀವು ಮಾಡಿ ಶುದ್ಧ ಚಾರಿತ್ರ್ಯವನ್ನು ನೀವು ರೂಢಿಸಬೇಕು. ಕರ್ತವ್ಯವನ್ನು ಮಾಡಿದರೆ ಕರ್ತವ್ಯಭಾರದಿಂದ ಪಾರಾಗುತ್ತೇವೆ. ಆಗ ಮಾತ್ರ ಎಲ್ಲವನ್ನೂ ದೇವರೇ ಮಾಡುತ್ತಿರುವನು ಎಂಬ ಭಾವನೆ ಬರುವುದು. ನಾವು ಅವನ ಕೈಯಲ್ಲಿ ಒಂದು ಯಂತ್ರ ಮಾತ್ರ. ಈ ದೇಹ ಒಂದು ಗ್ಲಾಸಿನಂತೆ, ದೇವರೇ ಅದರಲ್ಲಿರುವ ದೀಪ. ಯಾವುದು ದೇಹದ ಮೂಲಕ ಹೊರಗೆ ಹೋಗುತ್ತಿದೆಯೋ ಅದೆಲ್ಲಾ ದೇವರೆ; ನಿಮಗೆ ಗೊತ್ತಾಗುವುದಿಲ್ಲ. ನಾನು ಎಂದು ಭಾವಿಸುವಿರಿ. ಇದೊಂದು ಭ್ರಾಂತಿ, ಭಗವದಿಚ್ಛೆಯಿದ್ದಂತೆ ಆಗಲಿ ಎಂದು ಅವನಿಗೆ ಶರಣಾಗುವುದನ್ನು ಕಲಿಯಬೇಕು. ಇದನ್ನು ಕಲಿಯುವುದಕ್ಕೆ ಕರ್ತವ್ಯ ಪರಿಪಾಲನೆಯೇ ಮಾರ್ಗ. ಈ ಕರ್ತವ್ಯವೇ ನೀತಿ. ಸಂಪೂರ್ಣವಾಗಿ ಶರಣಾಗುವುದನ್ನು ಅಭ್ಯಾಸ ಮಾಡಿ. ನಾನೆಂಬುದರಿಂದ ಪಾರಾಗಿ, ವಂಚನೆಯಿಲ್ಲದೆ ಇರಿ. ಆಗ ಮಾತ್ರ ಕರ್ತವ್ಯ ಎಂಬ ಭಾವನೆಯಿಂದ ಪಾರಾಗಬಲ್ಲಿರಿ.

ಏಕೆಂದರೆ ಎಲ್ಲಾ ಮಾಡುತ್ತಿರುವವನೂ ಅವನೇ. ಅನಂತರ ನಿಮ್ಮ ಸ್ವಭಾವಕ್ಕೆ ತಕ್ಕಂತೆ ಮಾಡುತ್ತ ಹೋಗಿ. ಇತರರು ಮಾಡಿದ ತಪ್ಪನ್ನು ಮನ್ನಿಸಿ ಮರೆಯಿರಿ.

ನಮ್ಮ ಧರ್ಮ ಹಲವರಿಗೆ ಹಲವು ಬಗೆಯ ಕರ್ತವ್ಯದ ಆದರ್ಶಗಳನ್ನು ನೀಡುವುದು. ಜ್ಯೋತಿ ಎಲ್ಲಾ ಕಡೆಗಳಲ್ಲಿಯೂ ಇರುವುದು. ಆದರೆ ಮಹಾಪುರುಷರಲ್ಲಿ ಮಾತ್ರ ಇದು ವ್ಯಕ್ತವಾಗುವುದು. ಮಾಹಾಪುರುಷರು ಒಂದು ಸ್ಪಟಿಕದಂತೆ. ಬೆಳಕು ನಿರಾತಂಕವಾಗಿ ಅದರ ಮೂಲಕ ಹೋಗಿ ಬರುವುದು. ಜೀವನ್ಮುಕ್ತನನ್ನು ಏತಕ್ಕೆ ನೀವು ಪೂಜಿಸಬಾರದು?

ಸಾಧುಸಂಗ ಒಳ್ಳೆಯದು. ನೀವು ಸಾಧುಗಳ ಸಮೀಪಕ್ಕೆ ಹೋದರೆ ಸಾಕು, ಪ್ರಯತ್ನಿಸದೇ ಇದ್ದರೂ ಸುತ್ತಲೂ ಪವಿತ್ರತೆ ತುಂಬಿ ತುಳುಕಾಡುವುದನ್ನು ನೋಡುವಿರಿ.

ನಿಮಗೆ ಯಾರಾದರೂ ತೊಂದರೆಯನ್ನು ಕೊಟ್ಟರೆ ಅದನ್ನು ಎದುರಿಸಬೇಕಾಗಿಲ್ಲ. ಆದರೆ ಇತರರಿಗೆ ಅನ್ಯಾಯವಾದರೆ ನೀವು ಅದನ್ನು ವಿರೋಧಿಸಬಹುದು. ನೀವು ಸಾಧುವಾಗಬೇಕೆಂದು ಇಚ್ಛಿಸಿದರೆ ನೀವು ಎಲ್ಲಾ ಬಗೆಯ ಸುಖವನ್ನೂ ತ್ಯಜಿಸಬೇಕು. ಹಾಗಲ್ಲದೆ ಸಾಧಾರಣವಾಗಿದ್ದರೆ ನೀವೂ ಎಲ್ಲವನ್ನೂ ಅನುಭವಿಸಬಹುದು. ಆದರೆ ದಾರಿತೋರೆಂದು ದೇವರನ್ನು ಪ್ರಾರ್ಥಿಸಿ, ಅವನು ನಿಮಗೆ ದಾರಿ ತೋರುವನು.

ಈ ಪ್ರಪಂಚವು ಹೃದಯದ ಸ್ವಲ್ಪ ಭಾಗವನ್ನು ಮಾತ್ರ ತುಂಬಬಲ್ಲುದು. ಹೃದಯವು ಪ್ರಪಂಚಕ್ಕೆ ಅತೀತವಾಗಿರುವುದನ್ನು ಆಶಿಸುವುದು.

ಸ್ವಾರ್ಥತೆಯೇ ಪ್ರತಿಯೊಬ್ಬನಲ್ಲಿಯೂ ಇರುವ ಪ್ರತ್ಯಕ್ಷ ರಾಕ್ಷಸ. ಪ್ರತಿಯೊಂದು ಬಗೆಯ ಸ್ವಾರ್ಥವೂ ಸೈತಾನನೇ. ಕಡೆಯಿಂದ ದೇವರು ಬರುವನು, ಸ್ವಾರ್ಥತೆಯನ್ನು ತೆಗೆದುಹಾಕಿದ ಮೇಲೆ ದೇವರೊಬ್ಬನೇ ಉಳಿಯುವನು. ಕತ್ತಲೆ ಬೆಳಕು ಒಟ್ಟಿಗೆಯೇ ಇರಲಾರವು.

ನಾನು ಎಂಬ ಅಲ್ಪ ಅಹಂಕಾರವನ್ನು ಮರೆಯುವುದು ಸ್ವಸ್ಥವಾದ ಪರಿಶುದ್ಧ ಮನಸ್ಸಿನ ಸೂಚನೆ. ಆರೋಗ್ಯವಂತವಾದ ಮಗು ತನಗೆ ದೇಹವಿದೆ ಎಂಬುದನ್ನು ಮರೆಯುವುದು. ಸೀತೆ! ಅವಳನ್ನು ಪರಿಶುದ್ಧಳೆಂದು ಹೇಳುವುದಷ್ಟೇ ಅಲ್ಲ, ಅವಳು ರೂಪತಾಳಿದ ಪಾತಿವ್ರತ್ಯವೇ ಆಗಿದ್ದಾಳೆ. ಪ್ರಪಂಚದಲ್ಲೇ ಇದುವರೆಗೆ ಅಂತಹ ಮತ್ತೊಬ್ಬ ವ್ಯಕ್ತಿಯಿರಲಿಲ್ಲ.

ಭಕ್ತನು ಶ‍್ರೀರಾಮನ ಸಮೀಪದಲ್ಲಿರುವ ಸೀತೆಯಂತೆ ಇರಬೇಕು. ಭಕ್ತ ಹಲವು ಕಷ್ಟಗಳಿಗೆ ಬೀಳಬಹುದು. ಸೀತೆ ತನ್ನ ಕಷ್ಟವನ್ನು ಗಣನೆಗೆ ತರಲಿಲ್ಲ. ಯಾವಾಗಲೂ ರಾಮನನ್ನೇ ಚಿಂತಿಸುತ್ತಿದ್ದಳು.

ನಿಜವಾಗಿ ಪ್ರಪಂಚದಲ್ಲಿ ಏನಿದೆ ಎಂಬುದನ್ನು ಬೌದ್ಧ ಧರ್ಮ ಹೇಳುವುದಿಲ್ಲ. ಪ್ರವಾಹದಲ್ಲಿ ನೀರು ಬದಲಾಗುತ್ತಿರುವುದು. ನಾವೇ ಪ್ರವಾಹವೆಂದು ಹೇಳುವುದಕ್ಕೆ ಆಗುವುದೇ ಇಲ್ಲ. ಬೌದ್ಧರು ಒಂದಲ್ಲ, ಹಲವು ಇವೆ ಎನ್ನುವರು. ನಾವು ಒಂದು ನಿಜ; ಹಲವು ಇಲ್ಲ ಎನ್ನುವೆವು, ಬೌದ್ಧರ ಪ್ರಕಾರ ಮಾನವ ಒಂದು ತರಂಗಮಾಲೆಯಂತೆ. ಪ್ರತಿಯೊಂದು ಅಲೆಯೂ ನಾಶವಾಗುವುದು. ಆದರೆ ಹೇಗೋ ಅದು ಮತ್ತೊಂದು ಅಲೆಗೆ ಕಾರಣವಾಗುವುದು. ಎರಡನೆಯ ಅಲೆಯೂ ಮೊದಲನೆಯದಂತಿದೆ ಎಂಬುದೊಂದು ಭ್ರಾಂತಿ. ಭ್ರಾಂತಿಯಿಂದ ಪಾರಾಗುವುದಕ್ಕೆ ಸತ್ಕರ್ಮ ಆವಶ್ಯಕ. ನಾವು ಪ್ರಪಂಚದ ಹಿಂದೆ ಅದ್ವೈತವಿದೆ ಎನ್ನುತ್ತೇವೆ. ಪ್ರಪಂಚದಲ್ಲಿ ದುಃಖವಿದೆ, ನಾವು ಅದರಿಂದ ಪಾರಾದರೆ ಸಾಕು, ನಮಗೆ ಸುಖ ಸಿಕ್ಕುವುದೋ ಇಲ್ಲವೋ ಗೊತ್ತಿಲ್ಲ, ಎಂದು ಬೌದ್ಧರು ಹೇಳುವರು. ಇತರರು ಬೋಧಿಸಿದಂತೆ ಬುದ್ಧನು ಈ ಆತ್ಮನನ್ನು ವಿವರಿಸಲಿಲ್ಲ. ಹಿಂದೂಗಳ ದೃಷ್ಟಿಯಲ್ಲಿ ಆತ್ಮ ನಿಜವಾಗಿದೆ. ದೇವರು ನಿರಪೇಕ್ಷನಾಗಿರುವನು. ಸಾಪೇಕ್ಷತೆಯ ಭಾವನೆಯನ್ನು ನಾಶಪಡಿಸುತ್ತವೆ ಎಂಬ ಅಂಶವನ್ನು ಇಬ್ಬರೂ ಒಪ್ಪುತ್ತಾರೆ. ಬಹುತ್ವ ನಾಶವಾದ ಮೇಲೆ ಏನು ಉಳಿಯುವುದು ಎಂಬುದನ್ನು ಬೌದ್ಧರು ಹೇಳುವುದಿಲ್ಲ.

ಆಧುನಿಕ ಹಿಂದೂಧರ್ಮ ಮತ್ತು ಬೌದ್ಧ ಧರ್ಮಗಳೆರಡೂ ಒಂದೇ ಮೂಲದಿಂದ ಕವಲೊಡೆದ ಶಾಖೆಗಳು. ಬೌದ್ಧಧರ್ಮ ಅವನತಿಗೆ ಇಳಿಯಿತು. ಆಗ ಶಂಕರಾಚಾರ್ಯರು ಅದನ್ನು ಸಂಪೂರ್ಣ ಕಡಿದೇ ಹಾಕಿದರು.

ಬುದ್ಧನು ವೇದವನ್ನು ಒಪ್ಪಿಕೊಳ್ಳಲಿಲ್ಲ ಎನ್ನುತ್ತಾರೆ. ಏಕೆಂದರೆ ಅಲ್ಲಿ ಬೇಕಾದಷ್ಟು ಪ್ರಾಣಿವಧೆ ಹೇಳಿದೆಯಂತೆ. ಬೌದ್ಧರು ಹಿಂದೂಗಳ ಕರ್ಮಕಾಂಡದೊಂದಿಗೆ ಹೆಜ್ಜೆ ಹೆಜ್ಜೆಗೆ ಹೋರಾಡಿದರು. ಆದರೆ ಬುದ್ಧನಿಗೆ ಹಾಗೆ ಮಾಡುವುದಕ್ಕೆ ಅಧಿಕಾರವಿರಲಿಲ್ಲ.

ಬುದ್ಧನು ದೇವರ ವಿಷಯದಲ್ಲಿ ಏನನ್ನೂ ಹೇಳಲಿಲ್ಲ. ಆದರೆ ನಮ್ಮ ಧರ್ಮಗಳೆಲ್ಲ ದೇವರಿದ್ದಾನೆಂದು ಸಾರುವುವು. ವೇದಗಳು ಸಾಕಾರ ಮತ್ತು ನಿರಾಕಾರ ದೇವರನ್ನು ಹೇಳುವುವು. ಗೀತೆಯಲ್ಲೆಲ್ಲ ದೇವರ ವಿಷಯವಿದೆ. ದೇವರಿಲ್ಲದೆ ಹಿಂದೂಧರ್ಮವೇ ಇಲ್ಲ. ದೇವರಿಲ್ಲದೆ ವೇದಗಳೇ ಇಲ್ಲ. ಮುಕ್ತಿಗೆ ಇದೊಂದೇ ಮಾರ್ಗ. ಸಂನ್ಯಾಸಿಗಳು ಕೆಳಗೆ ಬರುವ ಭಾವನೆಗಳನ್ನು ಹಲವು ವೇಳೆ ಉಚ್ಚರಿಸಬೇಕು: “ನಾನು ಮುಕ್ತಿಗಾಗಿ ದೇವರನ್ನು ಆರಾಧಿಸುತ್ತೇನೆ. ಯಾವನು ಈ ಪ್ರಪಂಚವನ್ನು ಸೃಷ್ಟಿಸಿದನೊ ಮತ್ತು ಯಾವನು ವೇದವನ್ನು ಕೊಟ್ಟನೋ ಅವನಲ್ಲಿ ಶರಣಾಗುತ್ತೇನೆ”.

ಬುದ್ಧನು ಬೇರೆ ಬೇರೆ ಧರ್ಮಗಳೊಳಗೆ ಇರುವ ಸೌಹಾರ್ದತೆಯನ್ನು ಅರಿಯಬೇಕಾಗಿತ್ತು ಎಂದು ನಾವು ಈಗ ಭಾವಿಸುತ್ತೇವೆ. ಅವನು ಮತೀಯ ಭಾವನೆಯನ್ನು ತಂದನು.

ಈಗಿರುವ ಹಿಂದೂ, ಬೌದ್ಧ ಮತ್ತು ಜೈನ ಧರ್ಮಗಳೆಲ್ಲ ಏಕಕಾಲದಲ್ಲೇ ಬೇರೆಯಾದವು. ಕೆಲವು ಕಾಲದವರೆಗೆ ಕಾಪಟ್ಯ ಮತ್ತು ವಿಲಕ್ಷಣತೆಗಳಲ್ಲಿ ಒಂದೊಂದು ಧರ್ಮವೂ ಉಳಿದೆರಡನ್ನು ಮೀರಿಸುವಂತೆ ವರ್ತಿಸುತ್ತಿದ್ದುವು.

ದೇವರಿಲ್ಲದೆ ಇರುವುದನ್ನು ನಾವು ಕಲ್ಪಿಸಿಕೊಳ್ಳಲಿಕ್ಕೆಯೇ ಆಗುವುದಿಲ್ಲ. ನಾವು ಪಂಚೇಂದ್ರಿಯಗಳ ಮೂಲಕ ಏನೇನು ಕಲ್ಪಿಸಿಕೊಳ್ಳುವೆವೋ ಅದೆಲ್ಲವೂ ಅವನೇ ಆಗಿ, ಒಂದು ಗೋಸುಂಬೆಯಂತೆ ಅದನ್ನೂ ಮೀರಿರುವನು. ಪ್ರತಿಯೊಬ್ಬನೂ, ಪ್ರತಿಯೊಂದು ಜನಾಂಗವೂ ಒಂದೊಂದು ಕಾಲದಲ್ಲಿ ಅದರ ಒಂದೊಂದು ಬಣ್ಣವನ್ನು ನೋಡುವುದು. ಪ್ರತಿಯೊಬ್ಬನೂ ದೇವರಲ್ಲಿ ತನಗೆ ಯಾವುದು ಯೋಗ್ಯವೋ ಅದನ್ನು ತೆಗೆದುಕೊಳ್ಳಲಿ; ಪ್ರತಿಯೊಂದು ಪ್ರಾಣಿಯೂ ತಾನು ಯಾವ ಆಹಾರವನ್ನು ಜೀರ್ಣಿಸಿ ಕೊಳ್ಳಬಲ್ಲುದೋ ಆ ಆಹಾರವನ್ನು ಸೇವಿಸುವಂತೆ.

ಕ್ರೈಸ್ತ ಧರ್ಮದಂತಹ ಧರ್ಮಗಳಲ್ಲಿ ಇರುವ ಒಂದು ದೋಷವೆಂದರೆ ಎಲ್ಲರಿಗೂ ಒಂದೇ ನಿಯಮವನ್ನು ಪ್ರಯೋಗಿಸುವುದು. ಆದರೆ ಹಿಂದೂಗಳಲ್ಲಿ ಬಗೆಬಗೆಯ ಮನೋಧರ್ಮದವರಿಗೆಲ್ಲ ಬೇರೆ ಬೇರೆ ನಿಯಮಗಳಿವೆ. ಎಲ್ಲರಿಗೂ ಮುಂದುವರಿಯುವ ಅವಕಾಶವಿದೆ. ಎಲ್ಲಾ ಭಾವನೆಗಳ ಪರಿಪೂರ್ಣತೆಯನ್ನು ನಾವಿಲ್ಲಿ ನೋಡುತ್ತೇವೆ. ಉದಾಹರಣೆಗೆ ಶಾಂತಭಾವವನ್ನು ವಸಿಷ್ಠನಲ್ಲಿ ನೋಡುವೆವು; ಕೃಷ್ಣನಲ್ಲಿ ಪ್ರೀತಿಯನ್ನು ನೋಡುತ್ತೇವೆ. ರಾಮ–ಸೀತೆಯರಲ್ಲಿ ಕರ್ತವ್ಯ ಪರಾಯಣತೆಯನ್ನು ನೋಡುತ್ತೇವೆ. ಶುಕದೇವನಲ್ಲಿ ಜ್ಞಾನವನ್ನು ನೋಡುತ್ತೇವೆ. ಇದನ್ನು ಮತ್ತು ಇತರ ಆದರ್ಶ ಜೀವನಗಳನ್ನು ನೋಡಿ. ನಿಮಗೆ ಯಾವುದು ಹಿಡಿಸುವುದೋ ಅದನ್ನು ತೆಗೆದುಕೊಳ್ಳಿ.

ಸತ್ಯವು ನಿಮ್ಮನ್ನು ಎಲ್ಲಿಗೆ ಒಯ್ದರೂ ನೀವು ಅಲ್ಲಿಗೆ ಹೋಗಿ. ನಿಮ್ಮ ಭಾವನೆಗಳ ಚರಮ ಸೀಮೆಯನ್ನು ಮುಟ್ಟಿ. ಹೇಡಿಗಳಾಗಬೇಡಿ. ಆಷಾಢಭೂತಿಗಳಾಗಬೇಡಿ.

ನಿಮ್ಮ ಜೀವನದ ಆದರ್ಶದ ಮೇಲೆ ನಿಮಗೆ ಶ್ರದ್ಧೆ ಇರಬೇಕು. ಕ್ಷಣಿಕ ಶ್ರದ್ಧೆಯಲ್ಲ. ಉದ್ವೇಗವಿಲ್ಲದ, ಎಂದಿಗೂ ಬಿಡದ, ಒಂದೇ ಸಮನಾಗಿರುವ ಶ್ರದ್ಧೆ ಇರಬೇಕು. ಚಾತಕಪಕ್ಷಿ ಮಿಂಚುಗುಡುಗುಗಳು ಬರುವಾಗ ಮಳೆನೀರಿಗಾಗಿ ಕಾಯುತ್ತಿರುವುದೇ ಹೊರತು ಬೇರೆ ನೀರನ್ನೇ ಕುಡಿಯುವುದಿಲ್ಲ. ಪವಿತ್ರಾತ್ಮನಾಗಲು ಸಾಧನೆ ಮಾಡುತ್ತಾ. ನಾಶವಾದರೂ ಚಿಂತೆಯಿಲ್ಲ; ಸಾವಿರಾರು ವೇಳೆ ಮೃತ್ಯುವನ್ನು ಬೇಕಾದರೆ ಆಲಿಂಗಿಸು. ಎಂದಿಗೂ ನಿರಾಶನಾಗದಿರು. ಅಮೃತ ಸಿಕ್ಕಲಿಲ್ಲವೆಂದು ವಿಷಪಾನ ಮಾಡುವುದಕ್ಕೆ ಆಗುವುದಿಲ್ಲ. ಬೇರೆ ವಿಧಿಯೇ ಇಲ್ಲ. ಈ ಲೋಕ ಆ ಲೋಕದಷ್ಟೇ ಅನಿಶ್ಚಿತ.

ದಾನ ಎಂದಿಗೂ ನಿಷ್ಪ್ರಯೋಜಕವಾಗುವುದಿಲ್ಲ. ಒಂದು ಆದರ್ಶದಲ್ಲಿ ನಾವಿಟ್ಟಿರುವ ಶ್ರದ್ಧೆ ಎಂದಿಗೂ ನಿಷ್ಪ್ರಯೋಜಕವಾಗುವುದಿಲ್ಲ. ಇತರರಿಗೆ ಎಂದೆಂದಿಗೂ ಅನುಕಂಪ ತೋರುವುದನ್ನು ಮರೆಯಬೇಡ. ವೈರಿಗಳನ್ನು ಪ್ರೀತಿಸುವುದು ಸಾಧಾರಣ ಮಾನವರಿಗೆ ಸಾಧ್ಯವಿಲ್ಲ. ತಾವು ಬದುಕಬೇಕೆಂದು ಇತರರನ್ನು ಆಚೆಗೆ ಓಡಿಸುವರು. ಪ್ರಪಂಚದಲ್ಲಿ ಎಲ್ಲೋ ಕೆಲವು ಮಂದಿ ಮಾತ್ರ ನಾವೂ ಬದುಕೋಣ, ಇತರರೂ ಬದುಕಲಿ ಎಂದು ಅನುಷ್ಠಾನ ಮಾಡಿದರು. ಜನಕರಾಜ ಅಂತಹವನೊಬ್ಬ. ಅಂತಹವರು ಸಂನ್ಯಾಸಿಗಳಿಗಿಂತಲೂ ಮೇಲು. ತ್ಯಾಗದ ಮತ್ತು ಪವಿತ್ರತೆಯ ಸಾಕಾರಮೂರ್ತಿಯಂತಿದ್ದ ಶುಕದೇವ ಜನಕನನ್ನು ತನ್ನ ಗುರುವಾಗಿ ಆರಿಸಿಕೊಳ್ಳುವನು. ಜನಕನು ಅವನಿಗೆ "ನೀನು ಆಜನ್ಮಸಿದ್ಧ. ನಿನಗೆ ಏನು ಗೊತ್ತಿದೆಯೋ, ನಿನ್ನ ತಂದೆ ಏನನ್ನು ಹೇಳಿದನೋ, ಅದೆಲ್ಲಾ ಸತ್ಯ ಎಂದು ನಾನು ಹೇಳುತ್ತೇನೆ" ಎಂದನು.

ಸಮಷ್ಟಿಯಲ್ಲಿ ವ್ಯಷ್ಟಿಯೇ ಸೃಷ್ಟಿಯ ನಿಯಮ. ದೇಹದ ಪ್ರತಿಯೊಂದು ಜೀವಾಣುವೂ \enginline{(Cell)} ನಮ್ಮ ಪ್ರಜ್ಞೆಗೆ ಕಾರಣ. ಮನುಷ್ಯನು ಏಕಕಾಲದಲ್ಲಿ ವ್ಯಷ್ಟಿಯೂ ಆಗಿರುವನು, ಸಮಷ್ಟಿಯೂ ಆಗಿರುವನು. ನಮ್ಮ ವಿಶೇಷ ಗುಣವನ್ನು ಅರಿಯುವಾಗಲೇ ನಮ್ಮ ಜನಾಂಗದ ಮತ್ತು ವಿಶ್ವದ ಭಾವನೆಯನ್ನೂ ಅರಿಯುತ್ತೇವೆ. ಪ್ರತಿಯೊಂದೂ ಅನಂತವೃತ್ತದಂತೆ. ಕೇಂದ್ರ ಎಲ್ಲೆಲ್ಲಿಯೂ ಇರುವುದು, ಪರಿಧಿ ಎಲ್ಲಿಯೂ ಇಲ್ಲ. ಒಬ್ಬನು ಮನಸ್ಸು ಮಾಡಿದರೆ ವಿಶ್ವಾತ್ಮನನ್ನು ಅರಿಯಬಹುದು. ಇದೇ ಹಿಂದೂಧರ್ಮದ ಸಾರ. ಪ್ರತಿಯೊಂದರಲ್ಲಿಯೂ ಯಾರು ತಮ್ಮ ಆತ್ಮವನ್ನೇ ನೋಡಬಲ್ಲರೋ ಅವರೇ ಜ್ಞಾನಿಗಳು.

ಆಧ್ಯಾತ್ಮಿಕ ನಿಯಮಗಳನ್ನು ಕಂಡುಹಿಡಿದವರೇ ಋಷಿಗಳು. ಅದ್ವೈತ ತತ್ತ್ವದಲ್ಲಿ ಜೀವಾತ್ಮನಿಲ್ಲ. ಅದೊಂದು ಭ್ರಾಂತಿ, ದ್ವೈತ ತತ್ತ್ವದಲ್ಲಿ ಜೀವಾತ್ಮನಿರುವನು. ಅವನು ಪರಮಾತ್ಮನಿಗಿಂತ ಸಂಪೂರ್ಣ ಬೇರೆ. ಎರಡೂ ಸತ್ಯವೇ. ಒಬ್ಬ ನೀರಿನ ಮೂಲಕ್ಕೆ ಹೋಗುತ್ತಾನೆ; ಮತ್ತೊಬ್ಬ ಆ ನೀರೆಲ್ಲಾ ಸೇರಿರುವ ಕೆರೆಗೆ ಹೋಗುತ್ತಾನೆ. ತೋರಿಕೆಗೆ ನಾವೆಲ್ಲಾ ಅದ್ವೈತಿಗಳು; ನಿಜವಾಗಿ ಇದೊಂದೇ ಸತ್ಯ. ಅದ್ವೈತದ ಪ್ರಕಾರ ನೀನು ಎಲ್ಲರನ್ನೂ ನಿನ್ನವರಂತೆ ಕಾಣಬೇಕು. ಕ್ರೈಸ್ತರು ಹೇಳುವಂತೆ ಇತರರನ್ನು ನಿನ್ನ ಸಹೋದರರಂತೆ ನೋಡುವುದಲ್ಲ. ಸಹೋದರತ್ವದ ಭಾವನೆ ವಿಶ್ವಾತ್ಮ ಭಾವನೆಗೆ ಎಡೆಗೊಡಬೇಕು. ವಿಶ್ವಾತ್ಮ ಭಾವನೆಯೇ ನಮ್ಮ ಗುರಿ. ವಿಶ್ವ ಸಹೋದರತ್ವದ ಭಾವನೆಯಲ್ಲ. ಅದ್ವೈತದಲ್ಲಿ ಪರಮಸುಖದ ಸಿದ್ಧಾಂತವೂ ಸೇರಿದೆ.

'ಸೋಽಹಂ' – ನಾನು ಅವನೇ. ಇದನ್ನು ಅನವರತವೂ ಮನನ ಮಾಡಿ. ಮೊದಲು ಇದನ್ನು ಮನಸ್ಸಿಟ್ಟು ಮಾಡಿ. ಅನಂತರ ಇದು ನಮಗೆ ಅಭ್ಯಾಸವಾಗಿ ಹೋಗುವುದು, ಯಾಂತ್ರಿಕವಾಗುವುದು. ಈ ಭಾವನೆ ನಮ್ಮ ನರಗಳಿಗೂ ಇಳಿದು ಹೋಗುವುದು. ಇದನ್ನು ಕಂಠಪಾಠ ಮಾಡಿ ಪದೇ ಪದೇ ಮನನ ಮಾಡುತ್ತಾ ನಮ್ಮ ಅನೈತಿಕ ನರಗಳೂ ಕೂಡ ಇದನ್ನು ಮಾಡುವಂತೆ ಇರಬೇಕು.

ಇದು ಸಾಧ್ಯವಿಲ್ಲದೆ ಇದ್ದರೆ ದ್ವೈತದಿಂದ ಪ್ರಾರಂಭ ಮಾಡಿ. ಇದು ನಿಮ್ಮ ಅರಿವಿಗೆ ನಿಲುಕುವುದು. ಎರಡನೆಯದೇ ವಿಶಿಷ್ಟಾದ್ವೈತತತ್ತ್ವ – ನಾನು ನಿನ್ನಲ್ಲಿ, ನೀನು ನನ್ನಲ್ಲಿ, ಎಲ್ಲರೂ ಭಗವಂತನಲ್ಲಿ ಎನ್ನುತ್ತದೆ ಇದು. ಇದೇ ಕ್ರಿಸ್ತನ ಬೋಧನೆ.

ಪರಮಾದ್ವೈತವನ್ನು ಅನುಷ್ಠಾನದ ಭೂಮಿಕೆಗೆ ತರುವುದಕ್ಕೆ ಆಗುವುದಿಲ್ಲ. ಅದ್ವೈತವನ್ನು ಅನುಷ್ಠಾನಕ್ಕೆ ತಂದಾಗ ಅದು ವಿಶಿಷ್ಟಾದ್ವೈತದ ಭೂಮಿಕೆಯಲ್ಲಿ ಕೆಲಸ ಮಾಡುವುದು. ದ್ವೈತ ಎಂದರೆ ಸಣ್ಣ ವೃತ್ತ (ಜೀವಾತ್ಮ) ಮತ್ತು ದೊಡ್ಡ ವೃತ್ತ (ಪರಮಾತ್ಮ) ಎರಡೂ ಬೇರೆ ಬೇರೆ ಎಂಬುದು. ಭಕ್ತಿ ಇವೆರಡಕ್ಕೂ ಒಂದು ಸಂಬಂಧವನ್ನು ಕಲ್ಪಿಸುವುದು. ವಿಶಿಷ್ಟಾದ್ವೈತವೆಂದರೆ ದೊಡ್ಡ ವೃತ್ತದೊಳಗೆ ಸಣ್ಣ ವೃತ್ತ ಇದೆ ಎಂಬುದು. ಸಣ್ಣ ವೃತ್ತದ ಚಲನೆಯೆಲ್ಲ ದೊಡ್ಡ ವೃತ್ತದ ಆಳ್ವಿಕೆಗೆ ಒಳಪಟ್ಟಿದೆ. ಅದ್ವೈತ ಎಂದರೆ ಸಣ್ಣ ವೃತ್ತ ವಿಕಾಸವಾಗಿ ದೊಡ್ಡ ವೃತ್ತದಲ್ಲಿ ಐಕ್ಯವಾಗುವುದು. ಅದ್ವೈತದಲ್ಲಿ 'ನಾನು' ಎಂಬುದು ದೇವರಲ್ಲಿ ಲೀನವಾಗಿ ಹೋಗುವುದು. ದೇವರು ಇಲ್ಲಿಯೂ ಇರುವನು, ಅಲ್ಲಿಯೂ ಇರುವನು; ದೇವರೇ ನಾನು.

ಭಕ್ತಿಯನ್ನು ಪಡೆಯಬೇಕಾದರೆ ಭಗವಂತನ ನಾಮಜಪವನ್ನು ಮಾಡಬೇಕು. ಬರಿ ಮಂತ್ರೋಚ್ಛಾರಣೆಯಿಂದಲೇ ಎಷ್ಟೋ ಪ್ರಯೋಜನವಿದೆ. ಶಕ್ತಿಗೆಲ್ಲಾ ಕಾರಣ ಮಂತ್ರೋಚ್ಛಾರಣೆ. ಹಿಂದಿನ ಕಾಲದ ಅಸ್ತ್ರಶಕ್ತಿಯ ಹಿಂದೆಲ್ಲ ಮಂತ್ರಶಕ್ತಿಯಿತ್ತು. ನಮ್ಮ ಶಾಸ್ತ್ರದಲ್ಲೆಲ್ಲ ಅದನ್ನು ನಂಬುವರು. ಈ ಶಾಸ್ತ್ರವೆಲ್ಲ ಬರಿಯ ಕಲ್ಪನೆ ಎಂದು ಹೇಳುವುದು ಒಂದು ಮೂಢನಂಬಿಕೆ.

ಭಕ್ತಿಯನ್ನು ಪಡೆಯಬೇಕಾದರೆ ಸಾಧುಗಳ ಸಂಗವನ್ನು ಮಾಡಿ. ಗೀತೆ, ಇಮಿಟೇಷನ್ ಆಫ್ ಕ್ರೈಸ್ಟ್, \enginline{(Imitation of Christ)} ಮುಂತಾದ ಪುಸ್ತಕಗಳನ್ನೋದಿ. ಯಾವಾಗಲೂ ಭಗವಂತನ ಮಹಿಮೆಯನ್ನು ಮನನ ಮಾಡುತ್ತಿರಿ.

ವೇದಗಳಲ್ಲಿ ಕೇವಲ ಭಕ್ತಿಯನ್ನು ಪಡೆಯುವುದು ಹೇಗೆ ಎಂಬುದನ್ನು ಮಾತ್ರ ಹೇಳಿಲ್ಲ, ಪ್ರಪಂಚದ ಒಳ್ಳೆಯ ಮತ್ತು ಕೆಟ್ಟ ವಸ್ತುಗಳನ್ನು ಹೇಗೆ ಪಡೆಯಬೇಕೆಂಬುದನ್ನು ಹೇಳಿದೆ. ನಿಮಗೆ ಬೇಕಾದುದನ್ನು ಆರಿಸಿಕೊಳ್ಳಿ.

ಬಂಗಾಳವು ಭಕ್ತರ ದೇಶ. ಜಗನ್ನಾಥನ ದರ್ಶನಕ್ಕಾಗಿ ಚೈತನ್ಯ ದೇವನು ನಿಂತುಕೊಳ್ಳುತ್ತಿದ್ದ ಕಲ್ಲು, ಅವನು ಸುರಿಸುತ್ತಿದ್ದ ಭಕ್ತಿ ಮತ್ತು ಪ್ರೇಮಾಶ್ರುಗಳಿಂದ ಸವೆದುಹೋಯಿತು. ಅವನು ಸಂನ್ಯಾಸವನ್ನು ಸ್ವೀಕರಿಸಿದಾಗ ತನ್ನ ಗುರುಗಳಿಗೆ ತನ್ನ ಬಾಯಲ್ಲಿ ಸಕ್ಕರೆ ಸ್ವಲ್ಪ ಕಾಲ ಕರಗದೆ ಇದ್ದುದನ್ನು ತೋರಿಸಿ ತಾನು ಹೇಗೆ ಸಂನ್ಯಾಸಕ್ಕೆ ಯೋಗ್ಯ ಎಂಬುದನ್ನು ದೃಢಪಡಿಸಿದನು. ಭಕ್ತಿಯ ಅಂತರ್‌ಜ್ಞಾನದಿಂದ ಅವನು ವೃಂದಾವನವನ್ನು ಕಂಡುಹಿಡಿದನು.

ನಿಮ್ಮ ಶ್ರೇಯಸ್ಸಿಗಾಗಿ ಕೆಲವು ವಿಷಯಗಳನ್ನು ಹೇಳುತ್ತೇನೆ. ಭರತಖಂಡದಿಂದ ಬರುವ ಪ್ರತಿಯೊಂದನ್ನೂ, ಅದನ್ನು ನಂಬದೇ ಇರುವುದಕ್ಕೆ ಆಧಾರ ಸಿಕ್ಕುವ ತನಕ, ಸತ್ಯ ಎಂದು ನಂಬಿ. ಯೂರೋಪಿನಿಂದ ಬರುವ ಪ್ರತಿಯೊಂದನ್ನೂ ಅದನ್ನು ನಂಬುವುದಕ್ಕೆ ಸಾಕಷ್ಟು ಪ್ರಮಾಣ ದೊರಕುವವರೆಗೆ ಸುಳ್ಳು ಎಂದು ತಿಳಿಯಿರಿ.

ಐರೋಪ್ಯರ ಮೌಢ್ಯತೆಗೆ ಪರವಶರಾಗಬೇಡಿ. ನೀವೇ ಆಲೋಚನೆ ಮಾಡಿ. ನಿಮಗೆ ಬೇಕಾಗಿರುವುದು ಒಂದೇ. ನೀವು ದಾಸ್ಯದಲ್ಲಿರುವಿರಿ. ಐರೋಪ್ಯರು ಮಾಡುವುದನ್ನು ಅನುಕರಿಸುವಿರಿ. ಇದು ಮನಸ್ಸಿನ ತಾಮಸಿಕ ಸ್ಥಿತಿ.

ಸಮಾಜವು ಎಲ್ಲಿಂದ ಬೇಕಾದರೂ ತನ್ನ ಬೆಳವಣಿಗೆಗೆ ವಸ್ತುಗಳನ್ನು ಸಂಗ್ರಹಿಸಬಹುದು. ಆದರೆ ಅದು ತನ್ನ ರೀತಿಯಲ್ಲೇ ಬೆಳೆಯಬೇಕು.

ಹೊಸ ಆಚಾರಕ್ಕೆ ಅಂಜುವುದೇ ಮೌಢ್ಯಕ್ಕೆ ಮೂಲ. ಇದೇ ನರಕಕ್ಕೆ ಹಾದಿ. ಇದರಿಂದಲೇ ಮತಭ್ರಾಂತಿ ಹುಟ್ಟುವುದು. ಸತ್ಯವೇ ಸ್ವರ್ಗ, ಮತಭ್ರಾಂತಿಯೇ ನರಕ.

