
\chapter[ಅಧ್ಯಾಯ ೧]{ಅಧ್ಯಾಯ ೧\protect\footnote{\engfoot{Complete Works of Swami Vivekananda, Volume VI, Page 445}}}

\centerline{{\fontsize{11}{13}\selectfont ಸ್ಥಳ: ಬಾಗ್‌ಬಜಾರ್, ಕಲ್ಕತ್ತ, ಪ್ರಿಯನಾಥ ಮುಖ್ಯೋಪಾಧ್ಯಾಯರ ಮನೆ, ಕ್ರಿ.ಶ. ೧೮೯೭.}}

ಸ್ವಾಮಿ ವಿವೇಕಾನಂದರು, ಮೊದಲ ಸಾರಿ ವಿಲಾಯಿತಿಯಿಂದ ಹಿಂತಿರುಗಿ ಹಿಂದೂಸ್ತಾನಕ್ಕೆ ಬಂದು ಕಲ್ಕತ್ತ ಸೇರಿ ಈಗ ಮೂರು ನಾಲ್ಕು ದಿನಗಳಾಗಿವೆ. ಬಹುಕಾಲವಾದ ಬಳಿಕ ಅವರ ಪುಣ್ಯದರ್ಶನವನ್ನು ಪಡೆದ ಶ‍್ರೀರಾಮಕೃಷ್ಣ ಪರಮಹಂಸರ ಭಕ್ತರಿಗೆಲ್ಲಾ ಉಂಟಾಗಿದ್ದ ಆನಂದಕ್ಕೆ ಎಲ್ಲೆಯೇ ಇಲ್ಲ. ಅವರಲ್ಲಿ ಅನುಕೂಲವಿದ್ದವರು ಈಗ ಸ್ವಾಮೀಜಿಯನ್ನು ತಮ್ಮ ತಮ್ಮ ಮನೆಗೆ ಭಿಕ್ಷೆಗೆ ಬರಮಾಡಿಕೊಂಡು ತಾವು ಧನ್ಯರಾದೆವೆಂದು ತಿಳಿಯುತ್ತಿದ್ದರು. ಈ ದಿವಸ ಮಧ್ಯಾಹ್ನ ಬಾಗ್‌ಬಜಾರಿನ ರಾಜವಲ್ಲಭ ಮೊಹಲ್ಲದಲ್ಲಿದ್ದ ಶ‍್ರೀರಾಮಕೃಷ್ಣ ಭಕ್ತರಾದ ಶ‍್ರೀಯುತ ಪ್ರಿಯನಾಥ ಮುಖ್ಯೋಪಾಧ್ಯಾಯರ ಮನೆಯಿಂದ ಸ್ವಾಮೀಜಿಗೆ ಆಹ್ವಾನ. ಈ ವರ್ತಮಾನವನ್ನು ತಿಳಿದು ಅನೇಕ ಭಕ್ತ ಜನರು ಅವರ ಮನೆಗೆ ಬಂದಿದ್ದಾರೆ.

ಶಿಷ್ಯನು ಜನರ ಮೂಲಕ ಈ ಸುದ್ದಿಯನ್ನರಿತು ಮುಖ್ಯೋಪಾಧ್ಯಾಯರ ಮನೆಗೆ ಸುಮಾರು ಎರಡೂವರೆ ಗಂಟೆಯ ಹೊತ್ತಿಗೆ ಹೋಗಿದ್ದಾನೆ. ಸ್ವಾಮೀಜಿಯೊಡನೆ ಇದುವರೆಗೂ ಶಿಷ್ಯನು ಮಾತನಾಡಿದುದಿಲ್ಲ. ಶಿಷ್ಯನ ಜೀವನದಲ್ಲಿ ಸ್ವಾಮೀಜಿಯ ದರ್ಶನಲಾಭ ಇದೇ ಮೊದಲು.

ಶಿಷ್ಯನು ಬಂದೊಡನೆಯೇ ತುರೀಯಾನಂದ ಸ್ವಾಮಿಗಳು, ಅವನನ್ನು ಸ್ವಾಮಿಗಳ ಹತ್ತಿರಕ್ಕೆ ಕರೆದುಕೊಂಡು ಹೋಗಿ ಪರಿಚಯಮಾಡಿಸಿದರು. ಸ್ವಾಮಿಗಳು ಮಠಕ್ಕೆ ಬಂದು ಶಿಷ್ಯರಚಿತವಾದ ಒಂದು ಶ‍್ರೀರಾಮಕೃಷ್ಣ ಸ್ತೋತ್ರವನ್ನು ಓದಿ ಇದಕ್ಕೆ ಹಿಂದೆಯೇ ಅವನ ವಿಷಯವನ್ನು ಕೇಳಿದ್ದರು.

ಶ‍್ರೀರಾಮಕೃಷ್ಣ ಭಗವಾನರ ಭಕ್ತಶ್ರೇಷ್ಠರಾದ ನಾಗಮಹಾಶಯರ ಹತ್ತಿರ ಅವನು ಹೋಗಿ ಬರುತ್ತಿದ್ದನೆಂಬುದೂ ಸ್ವಾಮಿಗಳಿಗೆ ಗೊತ್ತಾಗಿತ್ತು.

ಶಿಷ್ಯನು ಸ್ವಾಮಿಜಿಗೆ ನಮಸ್ಕಾರಮಾಡಿ ಕುಳಿತುಕೊಳ್ಳಲು ಸ್ವಾಮೀಜಿ ಅವನೊಡನೆ ಸಂಸ್ಕೃತದಲ್ಲಿ ಮಾತನಾಡುತ್ತ, ನಾಗಮಹಾಶಯರ ಯೋಗಕ್ಷೇಮಾದಿಗಳನ್ನು ಕೇಳಿದರು; ಮತ್ತು ಅವರ ಅಸಾಧಾರಣ ತ್ಯಾಗ, ಉದ್ದಾಮ ಭಗವದನುರಾಗ ಮತ್ತು ದೈನ್ಯ ಇವುಗಳ ವಿಷಯವನ್ನು ಕುರಿತು ಮಾತನಾಡುತ್ತ “ವಯಂ ತತ್ತ್ವಾನ್ವೇಷಾತ್ ಹತಾಃ ಮಧುಕರ ತ್ವಂ ಖಲು ಕೃತೀ?"\footnote{ಅಭಿಜ್ಞಾನ ಶಾಕುಂತಲ} (ನಾವು ತತ್ತ್ವಾನ್ವೇಷಣ ಪ್ರಯತ್ನದಿಂದ ನಾಶವಾಗಿದ್ದೇವೆ. ಎಲೈ ದುಂಬಿಯೇ, ನೀನಾದರೋ ಮಧುವನ್ನು ಪಾನಮಾಡಿ ಕೃತಕೃತ್ಯನಾಗಿದ್ದೀಯೆ.) ಎಂದು ಹೇಳಿ ಈ ವಿಷಯಗಳನ್ನೆಲ್ಲಾ ನಾಗಮಹಾಶಯರಿಗೆ ಕಾಗದ ಬರೆದು ತಿಳಿಸುವಹಾಗೆ ಶಿಷ್ಯನಿಗೆ ಅಪ್ಪಣೆ ಮಾಡಿದರು. ಆಮೇಲೆ ಜನರ ಗದ್ದಲದಲ್ಲಿ ಮಾತನಾಡುವುದಕ್ಕೆ ಅನುಕೂಲಿಸುವುದಿಲ್ಲವೆಂಬುದನ್ನು ತಿಳಿದು ಅವನನ್ನೂ ತುರೀಯಾನಂದ ಸ್ವಾಮಿಗಳನ್ನೂ ಪಶ್ಚಿಮ ದಿಕ್ಕಿನ ಚಿಕ್ಕ ಕೊಠಡಿಗೆ ಕರೆದುಕೊಂಡು ಹೋಗಿ ಶಿಷ್ಯನ ಕಡೆಗೆ ತಿರುಗಿ, ‘ವಿವೇಕ ಚೂಡಾಮಣಿ’ಯ ಈ ವಾಕ್ಯಗಳನ್ನು ಹೇಳುವುದಕ್ಕೆ ಮೊದಲು ಮಾಡಿದರು -

\begin{verse}
“ಮಾ ಭೈಷ್ವ ವಿದ್ವಂಸ್ತವನಾಸ್ತ್ಯಪಾಯಃ\\ಸಂಸಾರ ಸಿಂಧೋಸ್ತರಣೇಽಸ್ತ್ಯುಪಾಯಃ~।\\ಯೇನೈವ ಯಾತಾ ಯತಯೋಽಸ್ಯ ಪಾರಂ\\ತಮೇವ ಮಾರ್ಗಂ ತವ ನಿರ್ದಿಶಾಮಿ~॥”
\end{verse}

“ಎಲೈ ವಿದ್ವಾಂಸನೇ, ಭಯಪಡಬೇಡ, ನಿನಗೆ ಅಪಾಯವಿಲ್ಲ; ಸಂಸಾರ ಸಾಗರವನ್ನು ದಾಟುವುದಕ್ಕೆ ಉಪಾಯವಿದೆ; ಯಾವುದನ್ನು ಅವಲಂಬನ ಮಾಡಿಕೊಂಡು ಶುದ್ಧಸತ್ತ್ವರಾದ ಯೋಗಿಗಳು ಸಂಸಾರ ಸಾಗರವನ್ನು ದಾಟಿದರೋ ಅದೇ ಉತ್ಕೃಷ್ಟವಾದ ಮಾರ್ಗವನ್ನೇ ನಾನು ತೋರಿಸಿಕೊಡುತ್ತೇನೆ.”

ಶಂಕರಾಚಾರ್ಯರ ‘ವಿವೇಕ ಚೂಡಾಮಣಿ’ ಎಂಬ ಗ್ರಂಥವನ್ನು ಓದುವಂತೆ ಅವನಿಗೆ ಅಪ್ಪಣೆ ಮಾಡಿದರು.

ಈ ಮಾತುಗಳನ್ನು ಕೇಳಿ, ಶಿಷ್ಯನು “ಸ್ವಾಮೀಜಿ ಈ ರೀತಿಯಲ್ಲಿ ಮಂತ್ರ ದೀಕ್ಷೆಯನ್ನು ಗ್ರಹಣಮಾಡಲು ನನಗೆ ಸೂಚಿಸಿದರೇನು?" ಎಂದು ಯೋಚಿಸುವುದಕ್ಕೆ ಮೊದಲುಮಾಡಿದನು. ಶಿಷ್ಯನು ಆಗ ತುಂಬ ಆಚಾರಶೀಲನೂ ಮತ್ತು ವೇದಾಂತ ಮತವಾದಿಯೂ ಆಗಿದ್ದನು. ಗುರುವನ್ನು ಆರಿಸಿಕೊಳ್ಳುವುದರಲ್ಲಿ ಅವನ ಬುದ್ಧಿಯು ಇನ್ನೂ ಸ್ಥಿರವಾಗಿರಲಿಲ್ಲ; ಮತ್ತು ಅವನಿಗೆ ವರ್ಣಾಶ್ರಮ ಧರ್ಮದಲ್ಲಿ ತುಂಬ ಪಕ್ಷಪಾತ.

ನಾನಾ ಪ್ರಸ್ತಾಪಗಳು ಬಂದವು. ಈ ಸಮಯದಲ್ಲಿ ಒಬ್ಬರು ಬಂದು “ಮಿರರ್ ಪತ್ರಿಕೆಯ ಸಂಪಾದಕರಾದ ಶ‍್ರೀಯುತ ನರೇಂದ್ರನಾಥಸೇನರು ಸ್ವಾಮೀಜಿಯವರ ಸಂದರ್ಶನ ತೆಗೆದುಕೊಳ್ಳುವುದಕ್ಕೆ ಬಂದಿದ್ದಾರೆ" ಎಂದು ತಿಳಿಸಿದರು. ಸ್ವಾಮಿಜಿ ಈ ವರ್ತಮಾನವನ್ನು ತಂದಿದ್ದವರೊಡನೆ “ಅವರನ್ನು ಇಲ್ಲಿಗೆ ಕರೆದುಕೊಂಡು ಬಾ" ಎಂದು ಹೇಳಿದರು. ನರೇಂದ್ರಬಾಬು ಚಿಕ್ಕಮನೆಗೆ ಬಂದು ಕುಳಿತುಕೊಂಡು ಅಮೆರಿಕಾ ಇಂಗ್ಲೆಂಡ್ ಸಂಬಂಧವಾಗಿ ನಾನಾ ಪ್ರಶ್ನೆಗಳನ್ನು ಹಾಕುತ್ತಿದ್ದರು. ಈ ಪ್ರಶೋತ್ತರಗಳಲ್ಲಿ ಸ್ವಾಮೀಜಿ ಅಮೆರಿಕಾವಾಸಿಗಳಂಥ ಸಹೃದಯರೂ, ಉದಾರಚಿತ್ತರೂ, ಅತಿಥಿ ಸತ್ಕಾರ\break ಪರಾಯಣರೂ, ಹೊಸ ಹೊಸ ಭಾವಗಳನ್ನು ಗ್ರಹಿಸುವುದರಲ್ಲಿ ವಿಶೇಷ ಉತ್ಸುಕರೂ, ಆದ ಜನರು ಜಗತ್ತಿನಲ್ಲಿ ಬೇರೆಲ್ಲಿಯೂ ಕಂಡುಬರಲಾರರು - ಎಂದು ಅಭಿಪ್ರಾಯ ಪಟ್ಟರಲ್ಲದೆ, "ಅಮೆರಿಕಾದಲ್ಲಿ ಅಲ್ಪಸ್ವಲ್ಪವಾದರೂ ಕಾರ್ಯವು ಆಗಿರುವುದು ನನ್ನ ಸಾಮರ್ಥ್ಯದಿಂದಲ್ಲ; ಅಮೆರಿಕಾ ದೇಶದ ಜನರು ಅಷ್ಟು ಒಳ್ಳೆಯ ಹೃದಯವುಳ್ಳವರಾದ್ದರಿಂದಲೇ ವೇದಾಂತದ ಅಭಿಪ್ರಾಯಗಳು ಅವರ ಮನಸ್ಸಿಗೆ ಹಿಡಿಸಿದುವು'' ಎಂದು ಹೇಳಿದರು. ಇಂಗ್ಲೆಂಡಿನ ವಿಚಾರವನ್ನು ಕುರಿತು ಅವರು ಹೇಳುವುದೇನೆಂದರೆ - “ಇಂಗ್ಲಿಷಿನವರಂಥ ಪ್ರಾಚೀನ ರೀತಿನೀತಿಗಳ ಮೇಲೆ ಪಕ್ಷಪಾತವಿರುವ ಜನರು ಜಗತ್ತಿನಲ್ಲಿ ಇನ್ನೊಬ್ಬರು ಇಲ್ಲ. ಅವರು ಯಾವ ಹೊಸ ಭಾವವನ್ನೂ ಸುಲಭವಾಗಿ ಗ್ರಹಿಸಲು ಶಕ್ತರಲ್ಲ. ಆದರೆ ಪಟ್ಟು ಹಿಡಿದು ಒಂದು ಸಾರಿ ಯಾವುದಾದರೊಂದು ಅಭಿಪ್ರಾಯವನ್ನು ಅವರಿಗೆ ತಿಳಿಸಿಬಿಟ್ಟರೆ ಆಮೇಲೆ ಏನು ಮಾಡಿದರೂ ಅವರು ಅದನ್ನು ಬಿಡುವುದಿಲ್ಲ. ಇಂಥ ದೃಢ ಪ್ರತಿಜ್ಞತೆ ಮತ್ತಾವ ಜನರಲ್ಲಿಯೂ ದೊರೆಯುವುದಿಲ್ಲ. ಇದಕ್ಕೋಸ್ಕರವೇ ಅವರು ನಾಗರಿಕತೆಯಲ್ಲಿಯೂ ಸಾಮರ್ಥ್ಯದಲ್ಲಿಯೂ ಸರ್ವಶ್ರೇಷ್ಠವಾದ ಸ್ಥಾನವನ್ನು ಪಡೆದು ಅಧಿಕಾರ ಮಾಡುತ್ತಾ ಇದ್ದಾರೆ.”

ಆಮೇಲೆ ಸಮರ್ಥರಾದ ಪ್ರಚಾರಕರು ದೊರಕಿದರೆ ಅಮೆರಿಕಾಕ್ಕಿಂತಲೂ ಇಂಗ್ಲೆಂಡಿನಲ್ಲಿಯೇ ವೇದಾಂತ ಪ್ರಚಾರಕಾರ್ಯ ಸ್ಥಾಯಿಯಾಗಿ ನಿಲ್ಲುವುದು ಹೆಚ್ಚು ಸಂಭವವೆಂದು ತಿಳಿಸಿ, “ನಾನು ಕಾರ್ಯವನ್ನು ಆರಂಭಿಸಿ ಮಾತ್ರ ಬಂದಿದ್ದೇನೆ ಅಷ್ಟೆ; ಮುಂದೆ ಪ್ರಚಾರಕರು ಇದೇ ಮಾರ್ಗವನ್ನು ಅನುಸರಿಸಿದರೆ, ಸಕಾಲದಲ್ಲಿ ತುಂಬ ಕೆಲಸ ಆಗುವುದು" ಎಂದರು.

ನರೇಂದ್ರಬಾಬು “ಹೀಗೆ ಧರ್ಮಪ್ರಚಾರ ಮಾಡುವುದರ ಮೂಲಕ ಮುಂದೆ ನಮಗೆ ಏನು ಉಪಕಾರ ಆಗಬಹುದು?" ಎಂದು ಪ್ರಶ್ನೆ ಮಾಡಿದರು.

ಸ್ವಾಮೀಜಿ: ನಮ್ಮ ದೇಶದಲ್ಲಿ ಇರುವುದು ಇದೊಂದು ವೇದಾಂತಧರ್ಮ ಮಾತ್ರ; ಪಾಶ್ಚಾತ್ಯ ನಾಗರಿಕತೆಯೊಡನೆ ಹೋಲಿಸಿ ನೋಡಿದರೆ ನಮಗೆ ಮತ್ತೇನೂ ಇಲ್ಲವೇ ಇಲ್ಲವೆಂದು ಹೇಳಬಹುದು. ಆದರೆ ಸಕಲ ಮತಗಳ ಮತ್ತು ಸಕಲ ಪಂಥಗಳ ಜನರಿಗೂ ಧರ್ಮಲಾಭದಲ್ಲಿ ಸಮಾನ ಅಧಿಕಾರವನ್ನು ಕೊಡುವ ಈ ಸಾರ್ವಭೌಮಿಕ ವೇದಾಂತವಿದೆಯೆಲ್ಲಾ, ಇದರ ಪ್ರಚಾರದಿಂದ ಪಾಶ್ಚಾತ್ಯ ನಾಗರಿಕ ಜಗತ್ತಿಗೆ, ಭರತ ಖಂಡದಲ್ಲಿ ಒಂದಾನೊಂದು ಕಾಲದಲ್ಲಿ ಎಂಥ ಆಶ್ಚರ್ಯಕರವಾದ ಧರ್ಮಭಾವದ ಸ್ಫೂರ್ತಿಯುಂಟಾಗಿತ್ತು ಮತ್ತು ಈಗಲೂ ಇದೆ ಎಂಬುದು ಗೊತ್ತಾಗುತ್ತದೆ. ಈ ಮತದ ಪ್ರಚಾರದಿಂದ ಪಾಶ್ಚಾತ್ಯ ಜನರಿಗೆ ನಮ್ಮ ವಿಷಯದಲ್ಲಿ ಗೌರವವೂ ಸಹಾನುಭೂತಿಯೂ. ಉಂಟಾದರೆ ಒಂದುಕಡೆ ನಾವು ಅವರಿಂದ ಐಹಿಕ ಜೀವನಕ್ಕೆ ಬೇಕಾದ ವಿಜ್ಞಾನ ಶಾಸ್ತ್ರಾದಿಗಳನ್ನು ಕಲಿತುಕೊಂಡು ಜೀವನ ಸಂಗ್ರಾಮದಲ್ಲಿ ಹೆಚ್ಚು ಚುರುಕಿನಿಂದ ಕೆಲಸ ಮಾಡುವೆವು. ಮತ್ತೊಂದು ಕಡೆ ಅವರು ನಮ್ಮಿಂದ ವೇದಾಂತ ಮತವನ್ನು ಕಲಿತು ಪಾರಮಾರ್ಥಿಕವಾದ ಕಲ್ಯಾಣವನ್ನು ಪಡೆಯುವರು.

ನರೇಂದ್ರಬಾಬು: ಹೀಗೆ ಕೊಳು ಕೊಡುಗೆಗಳಿಂದ ಯಾವುದಾದರೂ ಒಂದು ವಿಧವಾದ ರಾಜಕೀಯ ಉನ್ನತಿಯುಂಟಾಗುವ ಪ್ರತ್ಯಾಶೆಯುಂಟೆ? ಎಂದು ಕೇಳಿದರು.

ಸ್ವಾಮೀಜಿ ಹೀಗೆ ಉತ್ತರ ಕೊಟ್ಟರು: ಅವರು, ಪಾಶ್ಚಾತ್ಯರು ಮಹಾ ಪರಾಕ್ರಮಶಾಲಿಯಾದ ವಿರೋಚನನ ಸಂತಾನದವರು. ಅವರ ಸಾಮರ್ಥ್ಯದಿಂದ ಪಂಚಭೂತಗಳೂ ಆಟದ ಬೊಂಬೆಗಳಂತೆ ಕೆಲಸ ಮಾಡುತ್ತವೆ; ಅವರೊಡನೆ ಸೆಣಸಿ ಈ ಸ್ಥೂಲ ಪಾಂಚಭೌತಿಕ ಶಕ್ತಿಯನ್ನು ಪ್ರಯೋಗಿಸಿಯೇ ಸ್ವತಂತ್ರರಾಗುತ್ತೇವೆಂದು ನೀವು ತಿಳಿದುಕೊಂಡಿದ್ದರೆ ಅದು ದೊಡ್ಡ ತಪ್ಪು. ಹಿಮಾಲಯದ ಮುಂದೆ ಒಂದು ಸಾಮಾನ್ಯವಾದ ಕಲ್ಲಿನ ಚೂರು ಹೇಗೊ ಈ ಶಕ್ತಿಪ್ರಯೋಗ ಕುಶಲತೆಯಲ್ಲಿ ಅವರಿಗೂ ನಮಗೂ ಅಷ್ಟು ಭೇದ. ನನ್ನ ಅಭಿಪ್ರಾಯವೇನು ಬಲ್ಲಿರಾ? ನಾವು ಈ ರೀತಿಯಲ್ಲಿ ವೇದಾಂತೋಕ್ತ ಧರ್ಮದ ಗೂಢ ರಹಸ್ಯವನ್ನು ಪಾಶ್ಚಾತ್ಯ ಜಗತ್ತಿನಲ್ಲಿ ಪ್ರಚಾರ ಮಾಡಿ ಆ ಮಹಾ ಶಕ್ತಿಶಾಲಿಗಳ ವಿಶ್ವಾಸವನ್ನೂ, ಸಹಾನುಭೂತಿಯನ್ನೂ ಆಕರ್ಷಣೆ ಮಾಡಿಕೊಂಡು ಧಾರ್ಮಿಕ ವಿಚಾರದಲ್ಲಿ ಬಹುಕಾಲ ಅವರಿಗೆ ಗುರುಸ್ಥಾನದಲ್ಲಿರುವೆವು; ಅವರೂ ಇಹಲೋಕದ ನಾನಾ ವಿಷಯಗಳಲ್ಲಿ ನಮಗೆ ಗುರುಗಳಾಗಿರುವರು. ಧರ್ಮವೆಂಬ ಪದಾರ್ಥವನ್ನು ಅವರ ಕೈಗೆ ಹಾಕಿಬಿಟ್ಟು, ಭಾರತೀಯರು ಎಂದು ಪಾಶ್ಚಾತ್ಯರ ಪದತಲದಲ್ಲಿ ಧರ್ಮವನ್ನು ಕಲಿಯಲು ಕುಳಿತುಕೊಳ್ಳುವರೊ ಅಂದೆ ಈ ಅಧಃಪತಿತರಾದ ಜನರ ರಾಷ್ಟ್ರೀಯತ್ವವು ಸತ್ತಂತಾಗುವುದು. ಒಟ್ಟಿಗೆ ಅರಚುತ್ತ ಅವರನ್ನು ‘ಇದು ಕೊಡಿ’ ‘ಅದು ಕೊಡಿ’ ಎಂದು ಕೇಳುತ್ತಿದ್ದರೆ ಏನೂ ಆಗುವುದಿಲ್ಲ. ಈ ಕೊಟ್ಟು ತೆಗೆದುಕೊಳ್ಳುವ ಕಾರ್ಯದ ಮೂಲಕ ಯಾವಾಗ ಎರಡು ಪಕ್ಷದವರಲ್ಲಿಯೂ ಗೌರವವೂ ಸಹಾನುಭೂತಿಯೂ ಪ್ರಬಲವಾಗಿ ನಿಲ್ಲುತ್ತದೋ ಆಗ ಗದ್ದಲ ನಿಲ್ಲುವುದು. ಅವರು ತಾವೇ ಎಲ್ಲವನ್ನೂ ಮಾಡುವರು. ಹೀಗೆ ಧಾರ್ಮಿಕ ಚರ್ಚೆಯಿಂದಲೂ ವೇದಾಂತ ಧರ್ಮದ ಅತಿಶಯವಾದ ಪ್ರಚಾರದಿಂದಲೂ ಆ ದೇಶ ಮತ್ತು ಪಾಶ್ಚಾತ್ಯ ದೇಶ ಎರಡಕ್ಕೂ ವಿಶೇಷ ಲಾಭವಾಗುವುದೆಂದು ನನ್ನ ನಂಬುಗೆ. ಇದರೊಡನೆ ಹೋಲಿಸಿ ನೋಡಿದರೆ ರಾಜನೀತಿ ವಿಷಯಕವಾದ ಚರ್ಚೆ ಗೌಣ ಮಾರ್ಗವೆಂದು ತೋರುವುದು. ನಾನು ಈ ನಂಬುಗೆಯನ್ನು ಕಾರ್ಯರೂಪವಾಗಿ ಪರಿಣಾಮಗೊಳಿಸುವುದಕ್ಕಾಗಿಯೇ ನನ್ನ ಜೀವನವನ್ನು ಅರ್ಪಿಸುವೆನು. ಭರತಖಂಡಕ್ಕೆ ಮತ್ತಾವುದಾದರೂ ವಿಧದಲ್ಲಿ ಮಂಗಳವಾಗುವುದೆಂದು ತಾವು ತಿಳಿದುಕೊಂಡಿದ್ದರೆ ನೀವು ಬೇರೆ ವಿಧದಲ್ಲಿ ಕೆಲಸಮಾಡುತ್ತಾ ಹೋಗಿ.

ನರೇಂದ್ರಬಾಬು ಸ್ಯಾಮಾಜಿಯವರ ಮಾತಿಗೆ ಪ್ರತಿ ಹೇಳದೆ ಒಪ್ಪಿಕೊಂಡು ಸ್ವಲ್ಪ ಹೊತ್ತಿನ ಮೇಲೆ ಎದ್ದು ಹೊರಟುಹೋದರು. ಶಿಷ್ಯನು ಸ್ವಾಮಿಗಳ ಪೂರ್ವೋಕ್ತವಾದ ಮಾತನ್ನೆಲ್ಲಾ ಕೇಳಿ ಸ್ಥಂಭೀಭೂತನಾಗಿ ಅವರ ದೀಪ್ತವಾದ ಮೂರ್ತಿಯ ಕಡೆಗೆ ಎವೆಯಿಕ್ಕದೆ ನೋಡುತ್ತಿದ್ದನು.

ನರೇಂದ್ರಬಾಬುಗಳು ಹೊರಟುಹೋದ ಮೇಲೆ, ಗೋರಕ್ಷಣಾ ಸಭೆಗಾಗಿ ಕೆಲಸಮಾಡುತ್ತಿದ್ದ ಪ್ರಚಾರಕನೊಬ್ಬನು ಸ್ವಾಮಿಜಿಯವರ ದರ್ಶನಕ್ಕಾಗಿ ಬಂದನು. ಅವನ ವೇಷಭೂಷಾದಿಗಳು ಪೂರ್ತಿಯಾಗಿ ಅಲ್ಲದಿದ್ದರೂ ಬಹಳಮಟ್ಟಿಗೆ ಸಂನ್ಯಾಸಿಗಳ ಹಾಗೆ; ತಲೆಗೆ ಕಾವಿಯ ಬಣ್ಣದ ರುಮಾಲು; ನೋಡಿದ ಕೂಡಲೆ ಉತ್ತರ ಹಿಂದೂಸ್ಥಾನದವನೆಂದು ಹೇಳಬಹುದು. ಈ ಗೋರಕ್ಷಾ ಪ್ರಚಾರಕನು ಬಂದಿದ್ದ ವರ್ತಮಾನವನ್ನು ಕೇಳಿ ಸ್ವಾಮಿಜಿ ಬೈಠಕ್ ಖಾನೆಗೆ ಬಂದರು. ಪ್ರಚಾರಕನು ಸ್ವಾಮೀಜಿಯನ್ನು ಅಭಿವಂದಿಸಿ ಗೋಮಾತೆಯ ಒಂದು ಚಿತ್ರವನ್ನು ಅವರಿಗೆ ಸಮರ್ಪಿಸಿದನು. ಸ್ವಾಮೀಜಿ ಅದನ್ನು ತೆಗೆದುಕೊಂಡು ಹತ್ತಿರದಲ್ಲಿದ್ದ ಒಬ್ಬರ ಕೈಯಲ್ಲಿ ಕೊಟ್ಟು ಆತನೊಡನೆ ಮುಂದೆ ಹೇಳುವಂತೆ ಸಂಭಾಷಣೆಯನ್ನು ಮಾಡಿದರು.

ಸ್ವಾಮಿಜಿ: ತಮ್ಮ ಸಭೆಯ ಉದ್ದೇಶವೇನು?

ಪ್ರಚಾರಕ: ನಮ್ಮ ದೇಶದ ಗೋಮಾತೆಯನ್ನು ಕಟುಕರ ಕೈಯಿಂದ ತಪ್ಪಿಸಿ ಕಾಪಾಡುತ್ತಿದ್ದೇವೆ. ಅಲ್ಲಲ್ಲಿ ಪಿಂಜರಾಪೋಲುಗಳು ಸ್ಥಾಪಿಸಲ್ಪಟ್ಟಿವೆ. ಅವುಗಳಲ್ಲಿ ಖಾಯಿಲೆಯ, ಕೈಲಾಗದ ಮತ್ತು ಕಟುಕರಿಂದ ಕೊಂಡುತಂದ ಗೋಮಾತೆಯರನ್ನು ರಕ್ಷಿಸುತ್ತೇವೆ.

ಸ್ವಾಮೀಜಿ: ಇದು ಬಹು ಒಳ್ಳೆಯ ಕೆಲಸ; ತಮ್ಮ ಸಂಪಾದನೆಗೆ ಮಾರ್ಗ?

ಪ್ರಚಾರಕ: ದಯಾಪರರಾದ ತಮ್ಮಂಥವರು ಏನಾದರೂ ಕೊಡುತ್ತಾರೆಯಲ್ಲ, ಅದರಿಂದಲೇ ಸಭೆಯ ಈ ಕಾರ್ಯ ನಡೆಯುವುದು.

ಸ್ವಾಮೀಜಿ: ನಿಮ್ಮ ಹತ್ತಿರ ಮೂಲಧನ ಎಷ್ಟು ರೂಪಾಯಿ ಇದೆ?

ಪ್ರಚಾರಕ: ಮಾರವಾಡಿ ವರ್ತಕರು ಈ ಕಾರ್ಯಕ್ಕೆ ಒಳ್ಳೆಯ ಪೋಷಕರಾಗಿದ್ದಾರೆ. ಅವರು ಈ ಸತ್ಕಾರ್ಯಕ್ಕಾಗಿ ಹೆಚ್ಚು ದ್ರವ್ಯವನ್ನು ಕೊಟ್ಟಿದ್ದಾರೆ.

ಸ್ವಾಮೀಜಿ: ಮಧ್ಯ ಹಿಂದೂಸ್ಥಾನದಲ್ಲಿ ಎಂಥ ಭಯಂಕರವಾದ ಕ್ಷಾಮ ಬಂದಿದೆ! ಹೊಟ್ಟೆಗಿಲ್ಲದೆ ಒಂಬತ್ತು ಲಕ್ಷ ಜನರು ಸತ್ತುಹೋದರೆಂದು ಇಂಡಿಯಾ ಸರ್ಕಾರದವರು ಪಟ್ಟಿ ಕೊಟ್ಟಿದ್ದಾರೆ. ನಿಮ್ಮ ಸಭೆ ಈ ದುರ್ಭಿಕ್ಷಕಾಲದಲ್ಲಿ ಏನಾದರೂ ಸಹಾಯ ಮಾಡುವುದಕ್ಕೆ ಏರ್ಪಾಡು ಮಾಡಿದೆಯೇನು?

ಪ್ರಚಾರಕ: ನಾವು ದುರ್ಭಿಕ್ಷ ಮೊದಲಾದುವುಗಳಲ್ಲಿ ಸಹಾಯ ಮಾಡುವುದಿಲ್ಲ. ಕೇವಲ ಗೋಮಾತೆಯ ರಕ್ಷಣೆಗೆ ಈ ಸಭೆ ಸ್ಥಾಪಿಸಲ್ಪಟ್ಟಿರುವುದು.

ಸ್ವಾಮಿಜಿ: ಅಣ್ಣ ತಮ್ಮಂದಿರಾದ ನಿಮ್ಮ ದೇಶದ ಜನರು ಲಕ್ಷಗಟ್ಟಲೆ ಮೃತ್ಯುವಿನ ಬಾಯಲ್ಲಿ ಬೀಳುತ್ತಿರಲು, ಕೈಯಲ್ಲಾಗುತ್ತಿದ್ದರೂ ಇಂಥ ಭಯಂಕರವಾದ ದುಷ್ಕಾಲದಲ್ಲಿ ಅವರಿಗೆ ಅನ್ನ ಕೊಟ್ಟು ಸಹಾಯ ಮಾಡುವುದು ಯುಕ್ತವೆಂಬುದು ಮನಸ್ಸಿಗೆ ತೋರಲಿಲ್ಲವೋ?

ಪ್ರಚಾರಕ: ಇಲ್ಲ; ಜನರ ಕರ್ಮಫಲದಿಂದ, ಪಾಪದಿಂದ ಈ ಕ್ಷಾಮ ಬಂದಿದೆ; ಕರ್ಮಕ್ಕೆ ತಕ್ಕ ಫಲವಾಗಿದೆ.

ಪ್ರಚಾರಕನ ಮಾತನ್ನು ಕೇಳಿ ಸ್ವಾಮೀಜಿಯ ಆ ವಿಶಾಲವಾದ ಕಣ್ಣುಗಳಲ್ಲಿ ಬೆಂಕಿಯ ಕಿಡಿಗಳು ಉದುರುವಂತೆ ಕಂಡಿತು; ಮುಖ ಕೆಂಪಗಾಯಿತು. ಆದರೆ ಮನಸ್ಸಿನ ಭಾವವನ್ನು ಅದುಮಿಕೊಂಡು ಹೇಳಿದ್ದೇನೆಂದರೆ: ಯಾವ ಸಭಾ ಸಮಿತಿಗಳು ಮನುಷ್ಯರಲ್ಲಿ ಸಹಾನುಭೂತಿಯನ್ನು ತೋರಿಸದೆ, ತಮ್ಮ ಅಣ್ಣತಮ್ಮಂದಿರು ಹೊಟ್ಟೆಗಿಲ್ಲದೆ ಸಾಯುತ್ತಿದ್ದಾರೆಂದು ನೋಡಿಯೂ ಅವರ ಜೀವವನ್ನು ಉಳಿಸುವುದಕ್ಕಾಗಿ ಒಂದು ತುತ್ತು ಅನ್ನವನ್ನೂ ಕೊಡದೆ ಪಶುಪಕ್ಷಿಗಳ ರಕ್ಷಣೆಗಾಗಿ ರಾಶಿ ರಾಶಿ ಅನ್ನವನ್ನು ದಾನಮಾಡುತ್ತವೆಯೋ, ಅವುಗಳೊಡನೆ ನನಗೆ ಸ್ವಲ್ಪವೂ ಸಹಾನುಭೂತಿಯಿಲ್ಲ. ಅವುಗಳಿಂದ ಸಮಾಜಕ್ಕೆ ಹೆಚ್ಚು ಉಪಕಾರವಾಗುತ್ತದೆಂದು ನಾನು ನಂಬುವುದಿಲ್ಲ. ಕರ್ಮಫಲದಿಂದ ಜನರು ಸಾಯುತ್ತಾರೆ! ಹೀಗೆ ಕರ್ಮದ ನೆವವನ್ನು ಹೇಳುವುದಾದರೆ ಜಗತ್ತಿನಲ್ಲಿ ಯಾವ ವಿಷಯದಲ್ಲಿಯೂ ಕೆಲಸಮಾಡುವುದೇ ನಿಷ್ಪ್ರಯೋಜಕವೆಂದು ಒಟ್ಟಿಗೆ ನಿಶ್ಚಯಿಸಬಹುದು. ನಿಮ್ಮ ಪಶುರಕ್ಷಣೆಯ ಕೆಲಸವೂ ಆಮೇಲೆ ನಡೆಯುವುದಿಲ್ಲ. ಈ ಕೆಲಸದ ವಿಚಾರದಲ್ಲಿಯೂ “ಗೋಮಾತೆಗಳು ತಮ್ಮ ತಮ್ಮ ಕರ್ಮಫಲದಿಂದಲೇ ಕಟುಕರ ಕೈಗೆ ಹೋಗುತ್ತವೆ ಮತ್ತು ಸಾಯುತ್ತವೆ; ಆದ್ದರಿಂದ ಅದಕ್ಕೆ ನಾವು ಏನೂ ಮಾಡಬೇಕಾದ ಆವಶ್ಯಕತೆ ಇಲ್ಲ" ಎಂದು ಹೇಳಬಹುದು.

ಪ್ರಚಾರಕನು ಸ್ವಲ್ಪ ಅಪ್ರತಿಭನಾಗಿ ಹೇಳಿದ್ದೇನೆಂದರೆ: ತಾವು ಹೇಳುವುದೇನೋ ನಿಜ; ಆದರೆ ‘ಹಸು ನಮಗೆ ತಾಯಿ’ಯೆಂದು ಶಾಸ್ತ್ರ ಹೇಳುತ್ತದೆ.

ಸ್ವಾಮಿಜಿ ನಗುತ್ತ ನಗುತ್ತ: ಹಸು ನಮ್ಮ ತಾಯಿಯೆಂಬುದನ್ನು ನಾನು ಅರ್ಥಮಾಡಿಕೊಂಡಿದ್ದೇನೆ - ಇಲ್ಲದಿದ್ದರೆ ಇಂಥ ಧನ್ಯರಾದ ಪುತ್ರರನ್ನೆಲ್ಲಾ ಇನ್ನು ಯಾರು ಹೆತ್ತಾರು?

ಆ ಹಿಂದೂಸ್ಥಾನಿ ಪ್ರಚಾರಕನು ಈ ವಿಷಯದಲ್ಲಿ ಮತ್ತೇನನ್ನೂ ಹೇಳದೆ - (ಸ್ವಾಮೀಜಿ ವ್ಯಂಗ್ಯವಾಗಿ ಅಪಹಾಸ್ಯ ಮಾಡಿದ್ದನ್ನು ಆತನು ಅರ್ಥಮಾಡಿಕೊಳ್ಳಲಾರದೆ ಹೋದನೆಂದು ತೋರುತ್ತದೆ) - ಸ್ವಾಮೀಜಿಯನ್ನು ಕುರಿತು ಈ ಸಭೆಗಾಗಿ ಅವರಿಂದ ತಾನು ಏನನ್ನಾದರೂ ಬೇಡುವುದಾಗಿ ತಿಳಿಸಿದನು.

ಸ್ವಾಮೀಜಿ: ನಾನು ಸಂನ್ಯಾಸಿ; ಭಿಕ್ಷು; ನಿಮಗೆ ಸಹಾಯ ಮಾಡುವುದಕ್ಕೆ ನಾನು ಹಣವನ್ನು ಎಲ್ಲಿಂದ ತರಲಿ? ನನ್ನ ಕೈಯಲ್ಲಿ ಏನಾದರೂ ಯಾವಾಗಲಾದರೂ ಹಣವಿದ್ದರೆ ಅದನ್ನು ಮೊದಲು ಮನುಷ್ಯ ಸೇವೆಯಲ್ಲಿ ವೆಚ್ಚ ಮಾಡುತ್ತೇನೆ; ಮನುಷ್ಯನನ್ನು ಮೊದಲು ಬದುಕಿಸಿಕೊಳ್ಳಬೇಕು - ಅನ್ನದಾನ, ವಿದ್ಯಾದಾನ, ಧರ್ಮದಾನಗಳನ್ನು ಮಾಡಬೇಕು. ಇದೆಲ್ಲವನ್ನೂ ಮಾಡಿ ಹಣ ಮಿಕ್ಕರೆ, ಆಗ ನಿಮ್ಮ ಸಭೆಗೆ ಏನಾದರೂ ಸ್ವಲ್ಪ ಕೊಡುವುದಕ್ಕೆ ಬಂದೇನು.

ಈ ಮಾತನ್ನು ಕೇಳಿ ಪ್ರಚಾರಕ ಮಹಾಶಯನು ಸ್ವಾಮಿಗಳಿಗೆ ನಮಸ್ಕರಿಸಿ ಹೊರಟುಹೋದನು. ಆಗ ಸ್ವಾಮೀಜಿ ನಮಗೆ ಹೇಳುವುದಕ್ಕೆ ಮೊದಲು ಮಾಡಿದರು - “ಆತನು ಹೇಳಿದ್ದೇನು! ಕರ್ಮಫಲದಿಂದ ಮನುಷ್ಯರು ಸತ್ತುಹೋಗುತ್ತಾರೆ, ಅವರಿಗೆ ದಯೆ ತೋರಿಸಿದರೆ ಆಗುವುದೇನು - ಎಂದು ಹೇಳಿದನಲ್ಲವೆ? ದೇಶವು ತೀರ ಹೀನಸ್ಥಿತಿಗೆ ಬಂದಿದೆಯೆಂಬುದಕ್ಕೆ ಇದು ದೊಡ್ಡ ಸಾಕ್ಷಿಯಾಗಿದೆ. ನಿಮ್ಮ ಹಿಂದೂ ಧರ್ಮದ ಕರ್ಮವಾದವು ಎಲ್ಲಿಗೆ ಬಂದು ನಿಂತುಕೊಂಡಿದೆ ನೋಡಿದಿಯೊ? ಮನುಷ್ಯರ ಕಷ್ಟಕ್ಕಾಗಿ ಅನುಕಂಪದಿಂದ ದುಃಖಿಸದ ಮನುಷ್ಯನಿದ್ದರೆ ಅಂಥವನು ಮನುಷ್ಯನೇ?" - ಈ ಮಾತುಗಳನ್ನು ಆಡುತ್ತ ಆಡುತ್ತ ಸ್ವಾಮೀಜಿಯ ದೇಹವು ಚಿತ್ತಕ್ಷೋಭದಿಂದಲೂ ದುಃಖದಿಂದಲೂ ನಡುಗತೊಡಗಿತು.

ಅನಂತರ ಸ್ವಾಮಿಜಿ ಧೂಮಪಾನ ಮಾಡುತ್ತಾ ಶಿಷ್ಯನನ್ನು ಕುರಿತು ಪುನಃ ಬಂದು ನನ್ನನ್ನು ನೋಡು ಎಂದು ಹೇಳಿದರು.

\newpage

ಶಿಷ್ಯ: ತಾವು ಎಲ್ಲಿರುತ್ತೀರಿ? ಯಾರೋ ದೊಡ್ಡ ಮನುಷ್ಯರ ಮನೆಯಲ್ಲಿ ಇರುತ್ತೀರಿ ಎಂದು ತೋರುತ್ತದೆ. ಅಲ್ಲಿಗೆಲ್ಲಾ ನಾನು ಬರಬಹುದೆ?

ಸ್ವಾಮಿಜಿ: ಸದ್ಯದಲ್ಲಿ ನಾನು ಆಲಂಬಜಾರ್‌ ಮಠದಲ್ಲಿ ಸ್ವಲ್ಪ ಕಾಲ, ಗೋಪಾಲಲಾಲ ಶೀಲರ ತೋಟದಲ್ಲಿರುವ ಮನೆಯಲ್ಲಿ ಸ್ವಲ್ಪ ಕಾಲ ಇರುತ್ತೇನೆ. ನೀನು ಅಲ್ಲಿಗೆ ಬರಬಹುದು.

ಶಿಷ್ಯ: ಮಹಾಶಯರೇ, ತಮ್ಮ ಜೊತೆಯಲ್ಲಿ ಯಾರೂ ಇಲ್ಲದಾಗ ಮಾತನಾಡಬೇಕೆಂದು ತುಂಬಾ ಆಶೆಯಿದೆ.

ಸ್ವಾಮಿಜಿ: ಹಾಗೇ ಆಗಲಿ, ಒಂದು ದಿನ ರಾತ್ರಿ ಬಂದುಬಿಡು. ಬರಿಯ ವೇದಾಂತ ವಿಚಾರವೇ ಆಗಲಿ!

ಶಿಷ್ಯ: ಮಹಾಶಯರೇ! ತಮ್ಮೊಡನೆ ಕೆಲವು ಇಂಗ್ಲಿಷ್ ಮತ್ತು ಅಮೆರಿಕನ್ ಜನರು ಬಂದಿದ್ದಾರೆಂದು ಕೇಳಿದೆ; ಅವರು ನನ್ನ ವೇಷಭೂಷಣಗಳನ್ನೂ ಮಾತು ಕಥೆಗಳನ್ನೂ ನೋಡಿ ಅಸಮಾಧಾನ ಪಟ್ಟುಕೊಳ್ಳುವುದಿಲ್ಲವಷ್ಟೆ?

ಸ್ವಾಮೀಜಿ: ಅವರೂ ಮನುಷ್ಯರೆ; ಅದರಲ್ಲಿಯೂ ವೇದಾಂತ ಧರ್ಮನಿಷ್ಠರು; ನಿನ್ನೊಡನೆ ಮಾತನಾಡುವ ವಿಷಯದಲ್ಲಿ ಅವರಿಗೆ ಸಂತೋಷವೆ.

ಶಿಷ್ಯ: ಮಹಾಶಯರೇ, ವೇದಾಂತ ಎಂಬುದು ಎಲ್ಲಾ ಭಾಗದಲ್ಲಿಯೂ ಅಧಿಕಾರಿಗಳಾಗಿರುವವರಿಗೆ ಬರತಕ್ಕದ್ದು. ಅದು ನಿಮ್ಮ ಪಾಶ್ಚಾತ್ಯ ಶಿಷ್ಯರಿಗೆ ಬಂದದ್ದು ಹೇಗೆ? ಶಾಸ್ತ್ರದಲ್ಲಿ 'ವೇದ ವೇದಾಂತಗಳನ್ನು ವ್ಯಾಸಂಗಮಾಡಿ, ಪ್ರಾಯಶ್ಚಿತ್ತಗಳನ್ನು ಮಾಡಿಕೊಂಡು, ನಿತ್ಯನೈಮಿತ್ತಿಕ ಕರ್ಮಾನುಷ್ಠಾನಗಳನ್ನು ಆಚರಿಸುತ್ತ, ಆಹಾರ ವಿಹಾರಗಳಲ್ಲಿ ವಿಶೇಷವಾಗಿ ಸಂಯಮವುಳ್ಳವನಾಗಿದ್ದು, ಅದರಲ್ಲಿಯೂ ಸಾಧನ ಚತುಷ್ಟಯ ಸಂಪನ್ನನಾಗಿರದ ಹೊರತು ವೇದಾಂತಕ್ಕೆ ಅಧಿಕಾರಿಯಾಗುವುದಿಲ್ಲ - ಎಂದು ಹೇಳಿದೆ. ತಮ್ಮ ಪಾಶ್ಚಾತ್ಯ ಶಿಷ್ಯರಲ್ಲಿ ಪ್ರತಿಯೊಬ್ಬರೂ ಅಬ್ರಾಹ್ಮಣರು; ಆದ್ದರಿಂದ ಆಹಾರ ವಿಹಾರಗಳಲ್ಲಿ ಅನಾಚಾರಿಗಳು; ಅವರು ವೇದಾಂತ ಮತವನ್ನು ತಿಳಿದುಕೊಂಡದ್ದು ಹೇಗೆ?

ಸ್ವಾಮೀಜಿ: ಅವರೊಡನೆ ಮಾತನಾಡಿ ನೋಡಿದರೆ ಆಗ ಗೊತ್ತಾಗುತ್ತದೆ, ಅವರು ವೇದಾಂತವನ್ನು ತಿಳಿದುಕೊಂಡಿದ್ದಾರೆಯೋ ಇಲ್ಲವೋ ಎಂಬುದು.

ಪ್ರಾಯಶಃ ಇಷ್ಟು ಹೊತ್ತಿಗೆ, ಶಿಷ್ಯನು ಒಬ್ಬ ನಿಷ್ಠಾವಂತನೂ ಆಚಾರಶೀಲನೂ ಆದ ಹಿಂದುವೆಂದು ಸ್ವಾಮೀಜಿ ಗ್ರಹಿಸಿದ್ದರು. ಅನಂತರ ಅವರು ಕೆಲವರು ರಾಮಕೃಷ್ಣ ಭಕ್ತರೊಡನೆ ಬಾಗಬಜಾರಿನ ಶ‍್ರೀಯುತ ಬಲರಾಮಬಸು ಮಹಾಶಯರ ಮನೆಗೆ ಹೋದರು. ಶಿಷ್ಯನು ವಟತಲದಲ್ಲಿ ‘ವಿವೇಕಚೂಡಾಮಣಿ’ಯ ಒಂದು ಪ್ರತಿಯನ್ನು ಕೊಂಡುಕೊಂಡು ದರ್ಜಿಯ ಮೊಹಲ್ಲೆಯಲ್ಲಿದ್ದ ತನ್ನ ಮನೆಯ ಕಡೆಗೆ ಹೋದನು.

\newpage

\chapter[ಅಧ್ಯಾಯ ೨]{ಅಧ್ಯಾಯ ೨\protect\footnote{\engfoot{C.W. Vol. VI, P. 453}}}

\begin{center}
ಸ್ಥಳ: ಕಲ್ಕತ್ತೆಯಿಂದ ಕಾಶೀಪುರಕ್ಕೆ ಹೋಗುವ ದಾರಿ ಮತ್ತು ಗೋಪಾಲಲಾಲ ಶೀಲರ ತೋಟ, ವರ್ಷ: ಕ್ರಿ.ಶ. ೧೮೯೭.
\end{center}

ಸ್ವಾಮೀಜಿ ಈ ದಿನ ಶ‍್ರೀಯುತ ಗಿರೀಶಚಂದ್ರ ಘೋಷ ಮಹಾಶಯರ\footnote{ಕಲ್ಕತ್ತೆಯ ಸುವಿಖ್ಯಾತ ನಟರೂ, ನಾಟಕಕರ್ತರೂ, ರಾಮಕೃಷ್ಣರ ಭಕ್ತಾಗ್ರಣಿಗಳೂ ಆದ ಗಿರೀಶಚಂದ್ರ ಘೋಷರು.} ಮನೆಯಲ್ಲಿ ಮಧ್ಯಾಹ್ನ ವಿಶ್ರಮಿಸಿಕೊಳ್ಳುತ್ತಿದ್ದರು. ಶಿಷ್ಯನು ಅಲ್ಲಿಗೆ ಹೋಗಿ ನಮಸ್ಕಾರ ಮಾಡಿದನು. ನೋಡಲಾಗಿ ಸ್ವಾಮೀಜಿ ಆಗ ಗೋಪಾಲಲಾಲ ಶೀಲರ ಉದ್ಯಾನಗೃಹಕ್ಕೆ ಹೋಗಲು ಸಿದ್ಧರಾಗಿದ್ದರು. ಗಾಡಿ ಬಂದು ನಿಂತುಕೊಂಡಿತ್ತು. ಸ್ವಾಮೀಜಿ ಶಿಷ್ಯನನ್ನು ಕುರಿತು “ನನ್ನ ಜೊತೆಯಲ್ಲಿ ಬಾ ಹೋಗೋಣ" ಎಂದರು. ಶಿಷ್ಯನು ಸಮ್ಮತಿಸಲು, ಸ್ವಾಮಿಜಿ ಅವನನ್ನು ಜೊತೆಯಲ್ಲಿ ಕರೆದುಕೊಂಡು ಗಾಡಿಯನ್ನು ಹತ್ತಿದರು. ಗಾಡಿ ಹೊರಟಿತು. ಚಿತ್ಪುರದ ರಸ್ತೆಗೆ ಬಂದಾಗ ಗಂಗಾದರ್ಶನವಾಗಲು ಒಡನೆಯೇ ಸ್ವಾಮೀಜಿ ತಮ್ಮಷ್ಟಕ್ಕೆ ತಾವೇ ರಾಗವಾಗಿ “ಗಂಗಾ ತರಂಗ ರಮಣೀಯ ಜಟಾ ಕಲಾಪಂ..." ಎಂಬುದಾಗಿ ಹೇಳುವುದಕ್ಕೆ ಮೊದಲು ಮಾಡಿದರು. ಶಿಷ್ಯನು ಬೆರಗಾಗಿ ಆ ಅದ್ಭುತವಾದ ಸ್ವರಲಹರಿಯನ್ನು ನಿಶ್ಶಬ್ದವಾಗಿ ಕೇಳುತ್ತಿದ್ದನು. ಹೀಗೆ ಸ್ವಲ್ಪ ಹೊತ್ತು ಕಳೆದ ಮೇಲೆ ಒಂದು ರೈಲ್ವೆ ಇಂಜಿನ್ ಗಾಡಿ ಚಿತ್ಪುರದ ಹೈಡ್ರಾಲಿಕ್ ಸೇತುವೆಯ ಕಡೆಗೆ ಹೋಗುತ್ತಿದ್ದುದನ್ನು ನೋಡಿ ಸ್ವಾಮೀಜಿ ಶಿಷ್ಯನನ್ನು ಕುರಿತು “ನೋಡಿದೆಯಾ, ಹೇಗೆ ಸಿಂಹದ ಹಾಗೆ ಹೋಗುತ್ತಿದೆ!" ಎಂದರು.

ಶಿಷ್ಯ: ಅದು ಕೇವಲ ಜಡ, ಅದರ ಹಿಂದೆ ಮನುಷ್ಯನ ಚೇತನ ಶಕ್ತಿ ಕೆಲಸ ಮಾಡದಿದ್ದರೆ ಅದು ನಡೆದೀತೆ? ಅದು ಹೀಗೆ ಹೋಗುವುದರಲ್ಲಿ ತನ್ನ ಮಹತ್ವವೇನಾದರೂ ಇದೆಯೇನು?

ಸ್ವಾಮೀಜಿ: ಚೇತನದ ಲಕ್ಷಣವೇನೆಂದು ಹೇಳಬಲ್ಲೆಯಾ?

ಶಿಷ್ಯ: ಏಕೆ ಮಹಾಶಯರೇ! ಯಾವುದರಲ್ಲಿ ಬುದ್ಧಿಪೂರ್ವಕವಾದ ಕಾರ್ಯ ಕಂಡುಬರುತ್ತದೆಯೋ ಅದು ಚೇತನ.

ಸ್ವಾಮೀಜಿ: ಯಾವುದು ಪ್ರಕೃತಿಗೆ ವಿರೋಧವಾಗಿ ನಿಂತು ಪ್ರತಿಭಟಿಸುತ್ತದೆಯೋ ಅದೇ ಚೇತನ; ಅದರಲ್ಲಿಯೇ ಚೈತನ್ಯದ ವಿಕಾಸವಿರುತ್ತದೆ. ನೋಡು, ಒಂದು ಸಾಮಾನ್ಯವಾದ ಇರುವೆಯನ್ನು ಕೊಲ್ಲುವುದಕ್ಕೆ ಹೋಗು, ಅದೂ ಜೀವವನ್ನು ಉಳಿಸಿಕೊಳ್ಳುವುದಕ್ಕೆ ಒಂದು ಸಾರಿ ಎದುರು ಬೀಳುತ್ತದೆ. ಎಲ್ಲಿ ಕ್ರಿಯೆ ಅಥವಾ ಪುರುಷ ಪ್ರಯತ್ನ ಇದೆಯೋ, ಎಲ್ಲಿ ಎದುರುಬೀಳುವಿಕೆ ಇದೆಯೋ, ಅಲ್ಲಿಯೇ ಜೀವನ ಚಿಹ್ನೆ, ಅಲ್ಲಿಯೇ ಚೈತನ್ಯದ ವಿಕಾಸ.

ಶಿಷ್ಯ: ಮನುಷ್ಯನಿಗೂ ಮತ್ತು ಮನುಷ್ಯ ಜಾತಿಗೂ ಈ ನಿಯಮವು ಅನ್ವಯಿಸುತ್ತದೆಯೆ ಮಹಾಶಯರೆ?

ಸ್ವಾಮೀಜಿ: ಅನ್ವಯಿಸುತ್ತದೆಯೋ ಇಲ್ಲವೋ ಒಂದು ಸಾರಿ ಜಗತ್ತಿನ ಚರಿತ್ರೆಯನ್ನು ಓದಿ ನೋಡು; ನಿಮ್ಮನ್ನು ಬಿಟ್ಟು ಮಿಕ್ಕ ಎಲ್ಲಾ ದೇಶದ ಜನರಿಗೂ ಇದು ಅನ್ವಯಿಸುತ್ತದೆ. ನೀವು ಮಾತ್ರ ಪ್ರಪಂಚದಲ್ಲಿ ಈಗ ಜಡದ ಹಾಗೆ ಬಿದ್ದಿದ್ದೀರಿ. ನಿಮ್ಮನ್ನು ಮಂತ್ರಮುಗ್ಧರನ್ನಾಗಿ ಮಾಡಿಬಿಟ್ಟಿದ್ದಾರೆ: ಬಹಳ ಹಿಂದಿನ ಕಾಲದಿಂದಲೂ ಇತರರು ನಿಮಗೆ ‘ನೀವು ಹೀನರು, ನಿಮಗೆ ಯಾವ ಶಕ್ತಿಯೂ ಇಲ್ಲ’ ಎಂದು ಹೇಳುತ್ತ ಬಂದರು; ನೀವೂ ಅದನ್ನೇ ಕೇಳುತ್ತಾ ಈಗ ಒಂದು ಸಾವಿರ ವರ್ಷದಿಂದ ‘ನಾವು ಹೀನರು, ಯಾವ ಕೆಲಸಕ್ಕೂ ಕೈಲಾಗದವರು’ ಎಂದು ಭಾವಿಸಿಕೊಂಡಿರುವಿರಿ. ಭಾವಿಸಿಕೊಂಡು ಹಾಗೆಯೇ ಆಗಿರುವಿರಿ. (ತಮ್ಮ ಶರೀರವನ್ನು ತೋರಿಸಿ) ಈ ದೇಹವೂ ನಿಮ್ಮ ದೇಶದ ಮಣ್ಣಿನಿಂದಲೇ ತಾನೇ ಆಗಿದೆ? - ಆದರೆ ನಾನು ಮಾತ್ರ ಯಾವಾಗಲೂ ಹೀಗೆ ಭಾವಿಸಿಕೊಂಡಿಲ್ಲ. ಅದಕ್ಕೇ ನೋಡು ಈಶ್ವರನ ಇಚ್ಛೆಯಿಂದ, ಯಾರು ನನ್ನನ್ನು ಬಹುದಿವಸದಿಂದಲೂ ಕೀಳಾಗಿ ತಿಳಿಯುತ್ತಿದ್ದರೋ ಅವರೇ ದೇವರನ್ನು ಕಂಡರೆ ಹೇಗೋ ಹಾಗೆ ನನಗೆ ಗೌರವ ನೀಡಿದ್ದಾರೆ ಮತ್ತು ನೀಡುತ್ತಿದ್ದಾರೆ. ನೀವೂ ಹೀಗೆ ನಿಮ್ಮೊಳಗೆ ಅನಂತ ಶಕ್ತಿಯೂ, ಅಪಾರ ಜ್ಞಾನವೂ, ಅದಮ್ಯ ಉತ್ಸಾಹವೂ ಇದೆಯೆಂದು ಭಾವಿಸಿಕೊಂಡು ಒಳಗಿರುವ ಈ ಶಕ್ತಿಯನ್ನು ಎಬ್ಬಿಸಲು ಸಮರ್ಥರಾದರೆ ಆಗ ನೀವೂ ನನ್ನ ಹಾಗೆ ಆಗಬಲ್ಲಿರಿ.

ಶಿಷ್ಯ: ಹೀಗೆ ಭಾವಿಸಿಕೊಳ್ಳುವುದಕ್ಕೆ ಶಕ್ತಿ ಎಲ್ಲಿದೆ ಮಹಾಶಯರೆ? ಬಾಲ್ಯ ಕಾಲದಿಂದಲೂ ಈ ಮಾತುಗಳನ್ನು ಹೇಳಬೇಕು ಮತ್ತು ತಿಳಿಸಿಕೊಡಬೇಕು; ಅಂಥ ಶಿಕ್ಷಕ ಅಥವಾ ಗುರು ತಾನೆ ಎಲ್ಲಿದ್ದಾನೆ? ಓದು ಬರಹಗಳು ಈಗಿನ ಕಾಲದಲ್ಲಿ ಕೇವಲ ನೌಕರಿ ಸಂಪಾದಿಸುವುದಕ್ಕೋಸ್ಕರ – ಎಂಬ ಮಾತನ್ನೇ ನಾವು ಎಲ್ಲರಿಂದಲೂ ಕೇಳಿದ್ದೇವೆ ಮತ್ತು ಕಲಿತುಕೊಂಡಿದ್ದೇವೆ.

ಸ್ವಾಮೀಜಿ: ಅದಕ್ಕೇ ನಾವು ಬಂದಿರುವುದು - ಬೇರೆ ವಿಧವಾಗಿ ಹೇಳಿ ಕೊಡುವುದಕ್ಕೆ ಮತ್ತು ತೋರಿಸಿಕೊಡುವುದಕ್ಕೆ. ನೀವು ನನ್ನಿಂದ ಈ ತತ್ತ್ವವನ್ನು ಕಲಿತುಕೊಳ್ಳಿ, ತಿಳಿದುಕೊಳ್ಳಿ, ಅನುಭವಕ್ಕೆ ತಂದುಕೊಳ್ಳಿ. ಆಮೇಲೆ ಗ್ರಾಮ ಗ್ರಾಮಗಳಲ್ಲಿಯೂ ಊರು ಊರುಗಳಲ್ಲಿಯೂ ಹಳ್ಳಿ ಹಳ್ಳಿಗಳಲ್ಲಿಯೂ ಈ ಅಭಿಪ್ರಾಯವನ್ನು ತೆಗೆದುಕೊಂಡುಹೋಗಿ ಹರಡಿ. ಎಲ್ಲರ ಹತ್ತಿರಕ್ಕೂ ಹೋಗಿ, ‘ಏಳಿ ಎಚ್ಚರಗೊಳ್ಳಿ, ಇನ್ನು ನಿದ್ದೆ ಮಾಡಬೇಡಿ; ಎಲ್ಲಾ ಅಭಾವಗಳನ್ನೂ ದುಃಖಗಳನ್ನೂ ನೀಗಿಸುವ ಶಕ್ತಿ ನಿಮ್ಮ ಒಳಗೇ ಇದೆ; ಈ ಮಾತನ್ನು ನಂಬಿ, ಹಾಗಾದರೇ ಈ ಶಕ್ತಿ ಉದ್ಭುದ್ಧವಾಗುತ್ತದೆ’ ಎಂದು ಹೇಳಿ. ಈ ವಿಷಯವನ್ನು ಎಲ್ಲರಿಗೂ ಹೇಳಿ; ಮತ್ತು ಅದರ ಜೊತೆಗೆ ಸುಲಭವಾದ ಮಾತಿನಲ್ಲಿ ವಿಜ್ಞಾನ, ದರ್ಶನ, ಭೂಗೋಳ ಮತ್ತು ಚರಿತ್ರೆಗಳ ಮುಖ್ಯ ತತ್ತ್ವಗಳನ್ನು ಸಾಧಾರಣ ಜನರ ಮನಸ್ಸಿಗೆ ಅಂಟುವಂತೆ ತಿಳಿಸಿ. ನಾನು ಅವಿವಾಹಿತರಾದ ಯುವಕರನ್ನು ಸೇರಿಸಿ ಒಂದು ಕೇಂದ್ರವನ್ನು ಮಾಡುತ್ತೇನೆ - ಮೊದಲು ಅವರಿಗೆ ಶಿಕ್ಷಣವನ್ನು ಕೊಟ್ಟು ಆಮೇಲೆ ಅವರ ಮೂಲಕ ಈ ಕೆಲಸವನ್ನು ಮಾಡಿಸುತ್ತೇನೆ - ಇದು ನಾನು ಮಾಡಬೇಕೆಂದಿರುವ ಏರ್ಪಾಡು.

ಶಿಷ್ಯ: ಆದರೆ, ಮಹಾಶಯರೆ! ಹೀಗೆ ಮಾಡುವುದಕ್ಕೆ ಬಹಳ ಹಣ ಬೇಕಾಗುತ್ತದೆ; ಹಣವನ್ನು ಎಲ್ಲಿಂದ ತರುವಿರಿ?

ಸ್ವಾಮೀಜಿ: ಏನು ನೀನು ಹೇಳುವುದು! ಮನುಷ್ಯನಿಂದ ತಾನೇ ರೂಪಾಯಿ ಆಗುವುದು? ರೂಪಾಯಿಯಿಂದ ಮನುಷ್ಯನಾಗುತ್ತಾನೆಂಬ ಮಾತನ್ನು ಯಾವಾಗಲಾದರೂ ಯಾರಿಂದಲಾದರೂ ಕೇಳಿದ್ದೀಯೇನು? ನೀವು ಬಾಯಲ್ಲಿ ಆಡುವುದೂ ಮನಸ್ಸಿನಲ್ಲಿ ಯೋಚಿಸುವುದೂ ಒಂದೇ ಆಗುವುದಾದರೆ, ಆಡುವ ಮಾತಿನಲ್ಲಿಯೂ ಮಾಡುವ ಕೆಲಸದಲ್ಲಿಯೂ ಒಂದೇ ಆಗುವುದಾದರೆ, ಆಗ ಹಣವು ತನ್ನಷ್ಟಕ್ಕೆ ತಾನೇ ನೀರಿನಂತೆ ನಿಮ್ಮ ಹತ್ತಿರ ಬಂದು ಸುರಿಯುವುದು.

ಶಿಷ್ಯ: ಒಳ್ಳೆಯದು ಮಹಾಶಯರೆ! ಹಣ ಬಂತೆಂದು ಒಪ್ಪಿಕೊಳ್ಳೋಣ, ಮತ್ತು ತಾವು ಹಾಗೆ ಸತ್ಕಾರ್ಯವನ್ನು ಕೈಕೊಂಡು ಮಾಡಿದಿರೆಂದೂ ಇಟ್ಟುಕೊಳ್ಳೋಣ; ಅದರಿಂದ ತಾನೆ ಆಗುವುದು ಏನು? ಇದಕ್ಕೆ ಹಿಂದೆಯೂ ಎಷ್ಟೆಷ್ಟೋ ಜನ ಮಹಾಪುರುಷರು ಎಷ್ಟೆಷ್ಟೋ ಒಳ್ಳೆಯದನ್ನು ಮಾಡಿ ಹೋಗಿದ್ದಾರೆ - ಈಗ ಅದೆಲ್ಲಾ ಎಲ್ಲಿ? ತಾವು ಮಾಡುವ ಕಾರ್ಯಕ್ಕೂ ಕಾಲಾಂತರದಲ್ಲಿ ಇಂಥ ಅವಸ್ಥೆಯೇ ಬರುವುದು; ಇದು ನಿಜ. ಹೀಗಿರಲು ಈ ವಿಧವಾದ ಪ್ರಯತ್ನದ ಆವಶ್ಯಕತೆ ಏನು?

ಸ್ವಾಮೀಜಿ: ಮುಂದೆ ಏನಾಗುತ್ತದೆ ಎಂಬುದನ್ನೇ ಯಾವಾಗಲೂ ಯೋಚಿಸುತ್ತಾ ಅದನ್ನು ನೋಡಿಕೊಂಡೇ ಯಾವ ಕೆಲಸವನ್ನೂ ಮಾಡುವುದಕ್ಕಾಗುವುದಿಲ್ಲ. ಯಾವುದು ಸತ್ಯ, ಸರಿಯೆಂದು ತಿಳಿದುಕೊಂಡಿದ್ದೀಯೋ ಅದನ್ನು ಈಗ ಮಾಡಿ ಹಾಕಿಬಿಡು; ಮುಂದೆ ಏನಾಗುತ್ತದೆ ಏನಾಗುವುದಿಲ್ಲ ಎಂಬುದನ್ನು ಏಕೆ ಯೋಚಿಸಬೇಕು? ಜೀವನವಿರುವುದು ಕೆಲವು ವರ್ಷ - ಅದರಲ್ಲಿ ಫಲಾಫಲಗಳನ್ನು ಎಣಿಸುತ್ತಾ ಯಾವನಾದರೂ ಕೆಲಸ ಮಾಡುವುದು ಸಾಧ್ಯವೆ? ಫಲಾಫಲಗಳನ್ನು ಕೊಡತಕ್ಕವನು ಈಶ್ವರ ಒಬ್ಬನೆ; ಏನು ಬರಬೇಕಾದದ್ದಿದೆಯೋ ಅದನ್ನು ಕೊಡುತ್ತಾನೆ; ಆ ವಿಚಾರವನ್ನು ಕಟ್ಟಿಕೊಂಡು ನಿನಗೆ ಈಗ ಆಗಬೇಕಾದ್ದೇನು? ನೀನು ಆ ಕಡೆಗೆ ದೃಷ್ಟಿ ಇಡದೆ ಸುಮ್ಮನೆ ಕೆಲಸವನ್ನು ಮಾತ್ರ ಮಾಡಿಕೊಂಡು ಹೋಗು.

ಹೀಗೆ ಮಾತನಾಡುತ್ತಿರಲು, ಗಾಡಿ ಉದ್ಯಾನ ಗೃಹಕ್ಕೆ ತಲುಪಿತು. ಕಲ್ಕತ್ತೆಯಿಂದ ಅನೇಕ ಜನರು ಸ್ವಾಮೀಜಿಯವರ ದರ್ಶನಕ್ಕಾಗಿ ಆ ದಿವಸ ಈ ತೋಟಕ್ಕೆ ಬಂದರು. ಸ್ವಾಮೀಜಿ ಗಾಡಿಯಿಂದ ಇಳಿದು ಮನೆಯೊಳಕ್ಕೆ ಹೋಗಿ ಕುಳಿತರು. ಅವರೊಡನೆ ಮಾತಾಡುವುದಕ್ಕೆ ತೊಡಗಿದರು. ಸ್ವಾಮೀಜಿಯ ವಿಲಾಯಿತಿಯ ಶಿಷ್ಯರಾದ ಗುಡ್ವಿನ್ ಸಾಹೇಬರು ಸೇವೆಯೇ ಮೂರ್ತಿವೆತ್ತಂತೆ ಹತ್ತಿರದಲ್ಲಿ ನಿಂತುಕೊಂಡಿದ್ದರು. ಇದಕ್ಕೆ ಮುಂಚೆಯೇ ಅವರೊಡನೆ ಪರಿಚಯವಾಗಿದ್ದರಿಂದ ಶಿಷ್ಯನು ಅವರ ಹತ್ತಿರಕ್ಕೆ ಹೋದನು. ಇಬ್ಬರೂ ಸೇರಿ ಸ್ವಾಮೀಜಿ ವಿಷಯವಾಗಿ ನಾನಾ ಮಾತುಗಳಲ್ಲಿ ಪ್ರವೃತ್ತರಾದರು. ಸಂಜೆಯಾದ ಮೇಲೆ ಸ್ವಾಮೀಜಿ ಶಿಷ್ಯನನ್ನು ಕರೆದು ‘ನೀನು ಕಠೋಪನಿಷತ್ತನ್ನು ಕಂಠಪಾಠಮಾಡಿದ್ದೀಯಾ’ ಎಂದು ಕೇಳಿದರು.

ಶಿಷ್ಯ: ಇಲ್ಲ ಮಹಾಶಯರೆ! ಶಂಕರಭಾಷ್ಯದೊಡನೆ ಓದಿದ್ದೇನೆ ಅಷ್ಟೆ.

ಸ್ವಾಮೀಜಿ: ಉಪನಿಷತ್ತುಗಳಲ್ಲಿ ಇಂಥ ಸುಂದರವಾದ ಗ್ರಂಥ ಮತ್ತೊಂದು ಕಂಡುಬರುವುದಿಲ್ಲ. ಅದನ್ನು ನೀವು ಗಟ್ಟಿಮಾಡಿರಬೇಕೆಂದು ನನ್ನ ಅಪೇಕ್ಷೆ. ನಚಿಕೇತನ ಹಾಗೆ ಶ್ರದ್ಧೆ, ಸಾಹಸ, ವಿಚಾರ ಮತ್ತು ವೈರಾಗ್ಯವನ್ನು ಜೀವನದಲ್ಲಿ ಸಂಪಾದಿಸುವ ಪ್ರಯತ್ನ ಮಾಡಿ; ಸುಮ್ಮನೆ ಓದಿದರೆ ಏನಾಗುತ್ತದೆ?

ಶಿಷ್ಯ: ಸೇವಕನಿಗೆ ಇದೆಲ್ಲಾ ಅನುಭವಕ್ಕೆ ಬರುವಂತೆ ಕೃಪೆಮಾಡಬೇಕು.

ಸ್ವಾಮೀಜಿ: ಪರಮಹಂಸರ ಮಾತನ್ನು ಕೇಳಿದ್ದೀಯಷ್ಟೆ? ಅವರು ‘ಕೃಪೆಯ ಗಾಳಿ ಬೀಸುತ್ತಲೇ ಇರುತ್ತದೆ, ನೀನು ಮಾತ್ರ ನಿನ್ನ ದೋಣಿಯ ಪಟವನ್ನು ಎತ್ತಿ ಕಟ್ಟಬೇಕು’ ಎಂದು ಹೇಳುತ್ತಿದ್ದರು. ಯಾರಾದರೂ ಏನಾದರೂ ಮಾಡಿಕೊಡುವುದಕ್ಕೆ ಸಾಧ್ಯವೇನಪ್ಪಾ? ಅವನವನ ಅದೃಷ್ಟ ಅವನವನ ಕೈಯಲ್ಲಿ - ಗುರು ಇದಿಷ್ಟನ್ನು ಮಾತ್ರ ತಿಳಿಸಿಕೊಡುತ್ತಾನೆ, ಅಷ್ಟೆ. ಬೀಜದ ಶಕ್ತಿಯಿಂದಲೇ ಗಿಡವಾಗುವುದು; ನೀರು ಗಾಳಿ ಇವೆಲ್ಲಾ ಅದಕ್ಕೆ ಸಹಾಯ ಮಾತ್ರ.

ಶಿಷ್ಯ: ಹೊರಗಿನ ಸಹಾಯದ ಆವಶ್ಯಕತೆಯೂ ಇದೆಯಷ್ಟೆ, ಮಹಾಶಯರೆ?

ಸ್ವಾಮೀಜಿ: ಅದೇನೋ ಉಂಟು: ಆದರೆ ಏನೆಂದುಕೊಂಡಿದ್ದೀಯೆ? ಒಳಗೆ ಸಾರವಿಲ್ಲದಿದ್ದರೆ ನೂರಾರು ಸಹಾಯಗಳಿಂದ ಏನೂ ಆಗುವುದಿಲ್ಲ. ಆದರೆ ಎಲ್ಲರಿಗೂ ಆತ್ಮಾನುಭವವಾಗುವ ಒಂದೊಂದು ಕಾಲ ಬರುತ್ತದೆ; ಏಕೆಂದರೆ ಎಲ್ಲರೂ ಬ್ರಹ್ಮ, ಮೇಲು ಕೀಳು ಎಂದು ವಿಭಾಗ ಮಾಡುವುದು ಕೇವಲ ಈ ಬ್ರಹ್ಮ ವಿಕಾಸದ ತಾರತಮ್ಯದಿಂದಲೇ; ಸಕಾಲದಲ್ಲಿ ಎಲ್ಲರಲ್ಲಿಯೂ ಅದು ಪೂರ್ಣವಿಕಾಸವಾಗುತ್ತದೆ. ಅದನ್ನೇ ಶಾಸ್ತ್ರ ‘ಕಾಲೇನಾತ್ಮನಿ ವಿಂದತಿ’ - ಎಂದರೆ ಸಕಾಲದಲ್ಲಿ ಆತ್ಮಜ್ಞಾನವು ದೊರಕುತ್ತದೆ ಎಂದು ಹೇಳಿದೆ.

ಶಿಷ್ಯ: ಹಾಗಾದರೆ ಹೀಗೆ ಆಗುವುದು ಯಾವಾಗ ಮಹಾಶಯರೆ? ಎಷ್ಟೋ ಜನ್ಮಗಳನ್ನು ಅಜ್ಞಾನದಲ್ಲಿ ಕಳೆದುಬಿಟ್ಟಿದ್ದೇವೆಂದು ಶಾಸ್ತ್ರಮುಖದಿಂದ ಕೇಳಿದ್ದೇನೆ.

ಸ್ವಾಮೀಜಿ: ಭಯವೇನು? ಯಾವಾಗ ನೀನಿಲ್ಲಿಗೆ ಈ ಸಲ ಬಂದೆಯೊ, ಈ ಜನ್ಮದಲ್ಲೇ ಮುಕ್ತಿಯನ್ನು ಗಳಿಸುವೆ. ಮುಕ್ತಿ, ಸಮಾಧಿ – ಇವೆಲ್ಲಾ ಬ್ರಹ್ಮಪ್ರಕಾಶದ ಮಾರ್ಗದಲ್ಲಿರುವ ಪ್ರತಿಬಂಧಕಗಳನ್ನು ನಿವಾರಣೆ ಮಾಡುವುವು, ಅಷ್ಟೆ; ಆತ್ಮನು ಮಾತ್ರ ಸೂರ್ಯನ ಹಾಗೆ ಯಾವಾಗಲು ಪ್ರಜ್ವಲಿಸುತ್ತಾ ಇದ್ದಾನೆ. ಅಜ್ಞಾನದ ಮೋಡ ಅದನ್ನು ಮುಚ್ಚಿಕೊಂಡಿದೆ - ಆ ಮೋಡವನ್ನು ತಳ್ಳಿಬಿಟ್ಟರೆ ಸೂರ್ಯನ ಪ್ರಕಾಶ ಗೋಚರಿಸುತ್ತದೆ, ಆಗ ‘ಭಿದ್ಯತೇ ಹೃದಯಗ್ರಂಥಿಃ’ ಮುಂತಾದ ಸ್ಥಿತಿ ಬರುತ್ತದೆ. ಎಷ್ಟು ಮತಗಳನ್ನು ಬೇಕಾದರೂ ನೋಡು, ಅವು ಈ ಮಾರ್ಗದಲ್ಲಿ ಬರುವ ಪ್ರತಿಬಂಧಕಗಳನ್ನು ದೂರಮಾಡುವಂತೆ ಉಪದೇಶಿಸುತ್ತವೆ. ಯಾರು ಯಾವ ಭಾವದಲ್ಲಿ ಆತ್ಮಾನುಭವ ಮಾಡಿಕೊಂಡರೆ ಅವರು ಆ ಭಾವವನ್ನೆ ಉಪದೇಶಿಸಿಬಿಟ್ಟಿದ್ದಾರೆ. ಆದರೆ ಎಲ್ಲ ಉದ್ದೇಶವೂ ಆತ್ಮಜ್ಞಾನವೆ - ಆತ್ಮದರ್ಶನವೆ. ಇದಕ್ಕೆ ಎಲ್ಲಾ ಜಾತಿಗೆ, ಎಲ್ಲಾ ಜೀವಿಗಳಿಗೂ ಸಮಾನವಾದ ಅಧಿಕಾರ. ಇದು ಎಲ್ಲರಿಗೂ ಒಪ್ಪಿಗೆಯಾಗುವ ಅಭಿಪ್ರಾಯ.

ಶಿಷ್ಯ: ಮಹಾಶಯರೆ! ಶಾಸ್ತ್ರದ ಈ ಮಾತನ್ನು ಕೇಳಿದಾಗ ಅಥವಾ ಓದಿದಾಗ, ಇನ್ನೂ ಆತ್ಮವಸ್ತುವು ಪ್ರತ್ಯಕ್ಷವಾಗಿಲ್ಲವಲ್ಲಾ ಎಂದುಕೊಂಡು ಮನಸ್ಸು ವ್ಯಥೆಪಡುವುದು.

ಸ್ವಾಮೀಜಿ: ಇದಕ್ಕೆ ವ್ಯಾಕುಲತೆಯೆಂದು ಹೆಸರು. ಇದು ಎಷ್ಟೆಷ್ಟು ಬೆಳೆಯುತ್ತ ಹೋಗುತ್ತದೆಯೋ ಪ್ರತಿಬಂಧಕರೂಪವಾದ ಆ ಮೋಡ ಅಷ್ಟಷ್ಟು ಕಳೆಯುತ್ತಾ ಹೋಗುತ್ತದೆ. ಆಗಲೇ ಶ್ರದ್ಧೆ ದೃಢವಾಗುತ್ತದೆ ಮತ್ತು ಕ್ರಮವಾಗಿ ಆತ್ಮನು ಕರತಲಾಮಲಕದ ಹಾಗೆ ಪ್ರತ್ಯಕ್ಷನಾಗುತ್ತಾನೆ. ಅನುಭವವೇ ಧರ್ಮದ ಜೀವ. ಕೆಲವು ಆಚಾರನಿಯಮಗಳನ್ನು ಎಲ್ಲರೂ ನಡೆಸಿಕೊಂಡು ಹೋಗಬಹುದು. ಆದರೆ ಅನುಭವಕ್ಕೋಸ್ಕರ ಎಷ್ಟು ಜನ ವ್ಯಾಕುಲರಾಗುತ್ತಾರೆ? ವ್ಯಾಕುಲತೆ - ಈಶ್ವರ ಲಾಭ ಅಥವಾ ಆತ್ಮಜ್ಞಾನಕ್ಕೋಸ್ಕರ ಹುಚ್ಚು ಹಿಡಿದಂತಾಗುವುದು - ಇದೇ ಧರ್ಮದ ಯಥಾರ್ಥ ಪ್ರಾಣ.

“ಶ‍್ರೀಕೃಷ್ಣ ಪರಮಾತ್ಮನಿಗೋಸ್ಕರ ಗೋಪಿಯರಿಗೆ ಯಾವ ವಿಧವಾದ ವ್ಯಾಕುಲತೆ, ಉದ್ದಾಮ ಉನ್ಮತ್ತತೆಯುಂಟಾಗಿತ್ತೊ ಆತ್ಮದರ್ಶನಕ್ಕಾಗಿ ಆ ವಿಧವಾದ ವ್ಯಾಕುಲತೆ ಬೇಕು. ಗೋಪಿಯರ ಮನಸ್ಸಿನಲ್ಲಿಯೂ ಸ್ವಲ್ಪ ಸ್ವಲ್ಪ ಹೆಂಗಸು ಗಂಡಸು ಎಂಬ ಭೇದವಿತ್ತು. ನಿಜವಾದ ಆತ್ಮಜ್ಞಾನವಾದರೆ ಸ್ತ್ರೀಪುರುಷರೆಂಬ ಲಿಂಗಭೇದ ಪೂರ್ತಿ ಅಳಿದುಹೋಗುವುದು."

ಹೀಗೆಂದು ಹೇಳುತ್ತ ಹೇಳುತ್ತ ‘ಗೀತಗೋವಿಂದ’ದ ಪ್ರಸ್ತಾಪವನ್ನೆತ್ತಿಕೊಂಡು ಸ್ವಾಮೀಜಿ ಹೇಳಿದರು -

“ಜಯದೇವನೇ ಸಂಸ್ಕೃತ ಭಾಷೆಯಲ್ಲಿ ಕೊನೆಯ ಕವಿ. ಆದರೆ, ಜಯದೇವನು ಅನೇಕ ಸ್ಥಳಗಳಲ್ಲಿ ಭಾವಕ್ಕಿಂತಲೂ ಕಿವಿಗೆ ಇಂಪಾದ ವಾಕ್ಯ ವಿನ್ಯಾಸಗಳ ಕಡೆಗೆ ಹೆಚ್ಚು ಲಕ್ಷ್ಯವನ್ನು ಕೊಟ್ಟಿದ್ದಾನೆ. ‘ಗೀತಗೋವಿಂದ’ದ ‘ಪತತಿ ಪತತ್ರೇ...’ ಎಂಬ ಶ್ಲೋಕದಲ್ಲಿ ಅನುರಾಗ ಮತ್ತು ವ್ಯಾಕುಲತೆಯ ಎಂಥ ಪರಾಕಾಷ್ಠೆಯನ್ನು ಕವಿ ತೋರಿಸಿದ್ದಾನೆ ನೋಡಿದೆಯಾ? ಆತ್ಮ ದರ್ಶನಕ್ಕೋಸ್ಕರ ಈ ವಿಧವಾದ ಅನುರಾಗ ಉಂಟಾಗಬೇಕು. ಜೀವನದಲ್ಲಿ ವಿಲವಿಲನೆ ಒದೆದಾಟ ಉಂಟಾಗಬೇಕು; ಹಾಗೆ ಬೃಂದಾವನ ಲೀಲೆಯ ಪ್ರಸ್ತಾಪವನ್ನು ಬಿಟ್ಟು ಬಂದರೆ ಕುರುಕ್ಷೇತ್ರದ ಕೃಷ್ಣನು ಎಷ್ಟು ಮನೋಹರವಾಗಿದ್ದಾನೆ! ಅದನ್ನೂ ನೋಡು - ಅಂಥ ಭಯಂಕರವಾದ ಯುದ್ಧ ಕೋಲಾಹಲದಲ್ಲಿಯೂ ಕೃಷ್ಣನು ಎಷ್ಟು ಸ್ಥಿರ, ಗಂಭೀರ, ಶಾಂತ! ಯುದ್ಧರಂಗದಲ್ಲಿಯೇ ಅರ್ಜುನನಿಗೆ ಗೀತೆಯನ್ನು ಹೇಳಿದನು! ಕ್ಷತ್ರಿಯ ಧರ್ಮದಲ್ಲಿಯೇ ಯುದ್ಧವನ್ನು ಮಾಡುವಂತೆ ಮಾಡಿಬಿಟ್ಟನು! ಇಂಥ ಭಯಂಕರವಾದ ಯುದ್ಧಕ್ಕೆ ಪ್ರವರ್ತಕನಾದರೂ ಶ‍್ರೀಕೃಷ್ಣನು ತಾನು ಮಾತ್ರ ಹೇಗೆ ಕರ್ಮಹೀನನಾಗಿದ್ದ, ಅಸ್ತ್ರವನ್ನು ಹಿಡಿಯಲಿಲ್ಲ. ಯಾವ ಕಡೆಯಿಂದ ನೋಡಿದರೂ ಶ‍್ರೀಕೃಷ್ಣ ಚರಿತ್ರೆ ಸರ್ವಾಂಗ ಸಂಪೂರ್ಣವಾಗಿ ಕಂಡುಬರುತ್ತದೆ. ಅವನು ಜ್ಞಾನ ಕರ್ಮ ಭಕ್ತಿ ಯೋಗ ಇವುಗಳ ಮೂರ್ತಿವೆತ್ತ ವಿಗ್ರಹವೊ ಎಂಬಂತಿದ್ದಾನೆ. ಶ‍್ರೀಕೃಷ್ಣನ ಈ ಭಾವವನ್ನೇ ಈಗಿನ ಕಾಲದಲ್ಲಿ ವಿಶೇಷವಾಗಿ ಭಾವಿಸಬೇಕು. ಈಗ ಬೃಂದಾವನದಲ್ಲಿ ಕೊಳಲನ್ನೂದುವ ಕೃಷ್ಣನನ್ನೇ ನೋಡಿಕೊಂಡು ಕುಳಿತರೆ ನಡೆಯುವುದಿಲ್ಲ. ಅದರಿಂದ ಜೀವನ ಉದ್ದಾರವಾಗುವುದಿಲ್ಲ. ಈಗಿನ ಕಾಲಕ್ಕೆ ಬೇಕಾದದ್ದು ಗೀತೆಯ ರೂಪದಲ್ಲಿ ಸಿಂಹನಾದ ಮಾಡುತ್ತಿರುವ ಶ‍್ರೀಕೃಷ್ಣನ ಪೂಜೆ! ಧನುರ್ಧಾರಿಯಾದ ರಾಮ, ಮಹಾವೀರ, ಕಾಳಿಮಾತೆ ಇವರ ಪೂಜೆ! ಹಾಗಾದರೆ ಜನರು ವಿಶೇಷವಾಗಿ ಉದ್ಯಮಶೀಲರೂ ಕರ್ಮಶೀಲರೂ ಆಗಿ ಶಕ್ತಿವಂತರಾಗಿ ಎದ್ದು ನಿಲ್ಲುವರು, ನಾನು ಚೆನ್ನಾಗಿ ನೋಡಿಬಿಟ್ಟಿದ್ದೇನೆ - ಈ ದೇಶದಲ್ಲಿ ಈಗ ಯಾರು ‘ಧರ್ಮ, ಧರ್ಮ’ ಎನ್ನುತ್ತಿದ್ದಾರೆಯೋ ಅವರಲ್ಲಿ ಅನೇಕರು ದುರ್ಬಲರಾದ ರೋಗಿಗಳು, ಹುಳುಕು ಮೆದುಳಿನವರು ಅಥವಾ ವಿಚಾರಶೂನ್ಯರಾದ ಹುಂಬರು. ಈಗ ಮಹಾ ರಜೋಗುಣವು ಉದ್ದೀಪ್ತವಾದ ಹೂರತು ನಿಮಗೆ ಇಹವೂ ಇಲ್ಲ, ಪರವೂ ಇಲ್ಲ. ದೇಶವು ಘೋರವಾದ ತಮೋಗುಣದಿಂದ ಮುಚ್ಚಿದೆ. ಇದರ ಪರಿಣಾಮ - ಇಹಜೀವನದಲ್ಲಿ ದಾಸತ್ವ, ಪರಲೋಕದಲ್ಲಿ ನರಕ!"

ಶಿಷ್ಯ: ಪಾಶ್ಚಾತ್ಯ ದೇಶಿಯರ ರಜೋಭಾವವನ್ನು ನೋಡಿ ತಮಗೆ ಏನು ಪ್ರತ್ಯಾಶೆಯುಂಟಾಗುತ್ತಿದೆ? ಅವರು ಕ್ರಮವಾಗಿ ಸಾತ್ವಿಕರಾಗುತ್ತಾರೆಯೆ?

ಸ್ವಾಮೀಜಿ: ನಿಜವಾಗಿ, ಮಹಾ ರಜೋಗುಣ ಸಂಪನ್ನರಾದ ಅವರು ಈಗ ಭೋಗದ ಕೊನೆಯ ಅಂತಸ್ತಿಗೆ ಹತ್ತಿಬಿಟ್ಟಿದ್ದಾರೆ. ಅವರಿಗೆ ಯೋಗ ಪ್ರಾಪ್ತಿಯಾಗದೆ ನಾಲಗೆಯ ಚಾಪಲ್ಯವಿರುವ ನಿಮಗಾಗುವುದೆಂದು ಭಾವಿಸಿರುವಿರಾ? ಅವರ ಉತ್ಕೃಷ್ಟ ಭೋಗವನ್ನು ನೋಡಿದ ನನಗೆ ‘ಮೇಘದೂತ’ದ “ವಿದ್ಯುತ್ವಂತಂ ಲಲಿತವಃ..." ಎಂಬ ಚಿತ್ರ ಮನಸ್ಸಿಗೆ ಗೋಚರವಾಗುತ್ತದೆ. ಅಲ್ಲದೆ, ನಿಮ್ಮ ಭೋಗದಲ್ಲಿರುವುದೇನು? - ಚೌಗುನೆಲದ ಮನೆಗಳಲ್ಲಿ ಚಿಂದಿಗಳ ಮೇಲೆ ಮಲಗಿಕೊಂಡು ವರ್ಷವರ್ಷವೂ ಹಂದಿಗಳ ಹಾಗೆ ವಂಶವೃದ್ಧಿ ಮಾಡುವುದು! ಹೊಟ್ಟೆಗಿಲ್ಲದೆ ಸಾಯುತ್ತಿರುವ ತಿರುಕನನ್ನೂ ಗುಲಾಮರನ್ನೂ ಪಡೆಯುವುದು. ಆದ್ದರಿಂದ ಈಗ ಜನರಲ್ಲಿ ರಜೋ ಗುಣವನ್ನು ಉದ್ದೀಪನಗೊಳಿಸಿ ಅವರನ್ನು ಕರ್ಮಪ್ರಾಣರನ್ನಾಗಿ ಮಾಡಬೇಕಾಗಿದೆಯೆಂದು ಹೇಳಿದ್ದು. ಕರ್ಮ, ಕರ್ಮ, ಕರ್ಮ, ಈಗ “ನಾನ್ಯಃ ಪಂಥಾ ವಿದ್ಯತೇ ಯನಾಯ," ಅದನ್ನು ಬಿಟ್ಟರೆ ಉದ್ಧಾರವಾಗುವುದಕ್ಕೆ ಬೇರೆ ಮಾರ್ಗವೇ ಇಲ್ಲ.

ಶಿಷ್ಯ: ಮಹಾಶಯರೇ! ನಮ್ಮ ಪೂರ್ವಿಕರೇನು ರಜೋಗುಣ ಸಂಪನ್ನರಾಗಿದ್ದರೆ?

ಸ್ವಾಮೀಜಿ: ಇರಲಿಲ್ಲವೆ? ಚರಿತ್ರೆ ಹೇಳುತ್ತದೆ. ಅವರು ಎಷ್ಟೋ ದೇಶಗಳಲ್ಲಿ ವಸಾಹತುಗಳನ್ನು ಸ್ಥಾಪನೆ ಮಾಡಿದರು. ತಿಬೇಟ್, ಚೀನಾ, ಸುಮಾತ್ರ, ಬಹು ದೂರದಲ್ಲಿರುವ ಜಪಾನ್‌ವರೆಗೂ ಧರ್ಮಪ್ರಚಾರಕರನ್ನು ಕಳುಹಿಸಿದರು. ರಜೋಗುಣದ ಮೂಲಕವಲ್ಲದೆ ಪ್ರಗತಿಯುಂಟಾಗುವ ಸಂಭವವುಂಟೆ?

ಹೀಗೆ ಮಾತನಾಡುತ್ತ ಮಾತನಾಡುತ್ತಾ ರಾತ್ರಿಯಾಯಿತು. ಇಷ್ಟು ಹೊತ್ತಿಗೆ ಮಿಸ್ ಮುಲ್ಲರ್ ಬಂದಳು. ಆಕೆ ಒಬ್ಬ ಇಂಗ್ಲಿಷ್ ಹೆಂಗಸು, ಸ್ವಾಮಿಗಳಲ್ಲಿ ತುಂಬ ಶ್ರದ್ಧಾಭಕ್ತಿಯುಳ್ಳವಳು. ಸ್ವಾಮಿಜಿ ಶಿಷ್ಯನಿಗೆ ಆಕೆಯ ಪರಿಚಯವನ್ನು ಮಾಡಿಕೊಟ್ಟರು. ಸ್ವಲ್ಪ ಹೊತ್ತು ಮಾತನಾಡಿದ ಬಳಿಕ ಆಕೆ ಎದ್ದು ಮಹಡಿಗೆ ಹೋದಳು.

ಸ್ವಾಮೀಜಿ: ನೋಡಿದೆಯೊ, ಎಂಥ ವೀರ ವಂಶದವರು ಇವರು? ಮನೆ ಮಠ ಇರುವುದೆಲ್ಲಿ? ಸಾಲದ್ದಕ್ಕೆ ದೊಡ್ಡ ಮನುಷ್ಯರ ಮನೆ ಹೆಂಗಸು. ಆದರೂ ಧರ್ಮಲಾಭದ ಆಶೆಯನ್ನಿಟ್ಟುಕೊಂಡು ಬಂದಿರುವುದು ಎಲ್ಲಿಗೆ?

ಶಿಷ್ಯ: ಹೌದು ಮಹಾಶಯರೆ! ಆದರೆ ತಮ್ಮ ಕಾವ್ಯಕಲಾಪಗಳು ಇನ್ನೂ ಅದ್ಭುತವಾದುವು. ಎಷ್ಟು ಜನ ಪಶ್ಚಿಮದ ಹೆಂಗಸರು ತಮ್ಮ ಸೇವೆಗಾಗಿ ಸರ್ವದಾ ಸಿದ್ಧರಾಗಿದ್ದಾರೆ! ಈ ಕಾಲದಲ್ಲಿ ಇದು ನಿಜವಾಗಿಯೂ ತುಂಬ ಆಶ್ಚರ್ಯಕರವಾದ ವಿಷಯ!

ಸ್ವಾಮೀಜಿ: (ತಮ್ಮ ದೇಹವನ್ನು ತೋರಿಸಿ) ಈ ಶರೀರ ಇದ್ದರೆ ಇನ್ನೂ ಎಷ್ಟೋ ನೋಡುವೆ; ಉತ್ಸಾಹಿಗಳೂ ಅಭಿಮಾನಶಾಲಿಗಳೂ ಆದ ಕೆಲವು ಯುವಕರು ಸಿಕ್ಕಿದರೆ ನಾನು ದೇಶವನ್ನು ಅಲ್ಲೋಲಕಲ್ಲೋಲ ಮಾಡಿಬಿಟ್ಟೇನು. ಮದ್ರಾಸಿನಲ್ಲಿ ಕೆಲವು ಜನರಿದ್ದಾರೆ. ಆದರೆ ನನಗೆ ಬಂಗಾಳದಲ್ಲಿ ಆಸೆ ಬಹಳ. ಇಂಥ ತಿಳಿಯಾದ ಬುದ್ಧಿ ಪ್ರಾಯಶಃ ಮತ್ತಾರಿಗೂ ಲಭ್ಯವಾಗಿಲ್ಲ. ಆದರೆ ಇವರ ಮಾಂಸಖಂಡಗಳಲ್ಲಿ ಶಕ್ತಿಯಿಲ್ಲ. ಮಿದುಳು ಮತ್ತು ಮಾಂಸಖಂಡಗಳು ಇವೆರಡೂ ಸಮವಾಗಿ ಪೂರ್ಣ ವಿಕಾಸ ಉಳ್ಳವು ಆಗಬೇಕು. ದೃಢಬದ್ಧವಾದ ಶರೀರವೂ ವಿಶೇಷವಾದ ಬುದ್ಧಿಯೂ ಇದ್ದರೆ ಸಾಕು, ಜಗತ್ತೇ ಬಂದು ನಮ್ಮ ಕಾಲಿಗೆ ಬೀಳುತ್ತದೆ.

ಸ್ವಾಮಿಜಿಯವರ ಭೋಜನಕ್ಕೆ ಎಲ್ಲವೂ ಸಿದ್ಧವಾಗಿದೆಯೆಂದು ವರ್ತಮಾನ ಬಂತು. ಸ್ವಾಮೀಜಿ ಶಿಷ್ಯನನ್ನು ಕುರಿತು “ಬಾ! ನಾನು ಊಟಮಾಡುವುದನ್ನು ನೋಡುವಿಯಂತೆ" ಎಂದು ಹೇಳಿದರು. ಊಟಮಾಡುತ್ತ “ಎಣ್ಣೆ, ಕೊಬ್ಬು ಇವುಗಳನ್ನೆಲ್ಲಾ ಅತಿಯಾಗಿ ತಿನ್ನುವುದು ಒಳ್ಳೆಯದಲ್ಲ; ಪೂರಿಗಿಂತಲೂ ರೊಟ್ಟಿ ಒಳ್ಳೆಯದು. ಪೂರಿ ಕಾಯಿಲೆಯವರ ಆಹಾರ; ಹೊಸದಾಗಿರುವ ಕಾಯಿಪಲ್ಯಗಳನ್ನು ತಿನ್ನು, ಸಿಹಿ ಪದಾರ್ಥವನ್ನು ಕಡಿಮೆಮಾಡು" ಎಂದರು. ಹೀಗೆ ಮಾತನಾಡುತ್ತ ಆಡುತ್ತ “ಏನಯ್ಯಾ ನಾನು ಎಷ್ಟು ರೊಟ್ಟಿ ತಿಂದೆ? ಇನ್ನೂ ತಿನ್ನಬೇಕೇನು?" ಎಂದು ಕೇಳಿದರು. ಎಷ್ಟು ತಿಂದರೊ ಅದು ಗೊತ್ತಿಲ್ಲ, ಇನ್ನೂ ಹಸಿವು ಇತ್ತೊ ಇಲ್ಲವೊ ಅದನ್ನು ತಿಳಿದುಕೊಳ್ಳಲಾರದವರಾಗಿದ್ದರು! ಮಾತನಾಡುತ್ತ ಆಡುತ್ತ ದೇಹ ಬುದ್ಧಿ ಅಷ್ಟು ಕಡಿಮೆಯಾಗಿ ಹೋಗಿತ್ತು.

ಇನ್ನೊಂದು ಸ್ವಲ್ಪ ತಿಂದುಬಿಟ್ಟು ಸ್ವಾಮೀಜಿ ಊಟವನ್ನು ಮುಗಿಸಿದರು. ಶಿಷ್ಯನೂ ಅಪ್ಪಣೆಯನ್ನು ಪಡೆದು ಕಲ್ಕತ್ತೆಗೆ ಹಿಂತಿರುಗಿದನು. ಗಾಡಿ ಸಿಕ್ಕದೆ ಇದ್ದುದರಿಂದ ನಡೆದುಕೊಂಡೆ ಹೋಗಬೇಕಾಯಿತು. ಹೋಗುತ್ತ ಹೋಗುತ್ತ ಮರು ದಿನ ಮತ್ತೆ ಸ್ವಾಮಿಗಳನ್ನು ನೋಡುವುದಕ್ಕೆ ಎಷ್ಟು ಬೇಗ ಬರಬೇಕೆಂದು ಯೋಚನೆ ಮಾಡಿಕೊಂಡನು.

\newpage

\chapter[ಅಧ್ಯಾಯ ೩]{ಅಧ್ಯಾಯ ೩\protect\footnote{\engfoot{C.W, Vol. VI, P. 461}}}

\begin{center}
ಸ್ಥಳ: ಕಾಶೀಪುರ; ಗೋಪಾಲಲಾಲ ತೋಟ, ವರ್ಷ: ಕ್ರಿ.ಶ. ೧೮೯೭.
\end{center}

ಸ್ವಾಮೀಜಿ ಮೊದಲಸಾರಿ ವಿಲಾಯಿತಿಯಿಂದ ಹಿಂತಿರುಗಿ ಬಂದು ಕೆಲವು ದಿವಸ ಕಾಶೀಪುರದ ಗೋಪಾಲಲಾಲ ತೋಟದಲ್ಲಿ\footnote{ಇದೇ ತೋಟದಲ್ಲಿರುವಾಗ ಸ್ವಾಮೀಜಿ ಒಂದು ದಿನ ತಲೆಯಿಲ್ಲದ ಪ್ರೇತವನ್ನು ನೋಡಿದರು. ಅದು ಆಗತಾನೆ ಒದಗಿದ್ದ ಮೃತ್ಯುವಿನ ದೆಸೆಯಿಂದ ದೀನಸ್ವರದಲ್ಲಿ ಪ್ರಾಣಭಿಕ್ಷೆಯನ್ನು ಬೇಡಿಕೊಳ್ಳುವಂತಿತ್ತು. ವಿಚಾರಮಾಡಲು, ನಿಜವಾಗಿಯೂ ಈ ತೋಟದಲ್ಲಿ ಯಾವನೋ ಒಬ್ಬ ಬ್ರಾಹ್ಮಣನು ಹತನಾಗಿ ಸತ್ತುಹೋಗಿದ್ದನೆಂದು ಆಮೇಲೆ ಗೊತ್ತಾಯಿತು. ಈ ಘಟನೆಯನ್ನು ಅವರು ಆಮೇಲೆ ತಮ್ಮ ಗುರುಭ್ರಾತೃಗಳೊಡನೆ ಹೇಳಿದರು.} ಇದ್ದರು. ಶಿಷ್ಯನು ಆಗ ಪ್ರತಿ ನಿತ್ಯವೂ ಅಲ್ಲಿಗೆ ಹೋಗಿ ಬರುತ್ತಿದ್ದನು. ಶಿಷ್ಯನೇ ಏಕೆ, ಬಹುಜನ ಉತ್ಸಾಹಶಾಲಿಗಳಾದ ಯುವಕರೂ ಆಗ ಸ್ವಾಮೀಜಿಯವರ ದರ್ಶನ ಮಾಡಬೇಕೆಂದು ಅಲ್ಲಿಗೆ ಬಂದು ಗುಂಪು ಸೇರುತ್ತಿದ್ದರು. ಮಿಸ್ ಮುಲ್ಲರ್ ಸ್ವಾಮಿಗಳ ಜೊತೆಯಲ್ಲಿ ಬಂದು ಇಲ್ಲಿಯೇ ಮೊದಲು ಉಳಿದುಕೊಂಡಿದ್ದಳು. ಶಿಷ್ಯನ ಗುರುಭ್ರಾತೃವಾದ ಗುಡ್ವಿನ್ ಸಾಹೇಬರೂ ಈ ತೋಟದಲ್ಲಿಯೇ ಸ್ವಾಮೀಜಿಯವರೊಡನೆ ಇರುತ್ತಿದ್ದರು.

ಸ್ವಾಮಿಜಿಯವರ ಪ್ರಖ್ಯಾತಿ ಆಗ ಭರತಖಂಡದ ಒಂದು ತುದಿಯಿಂದ ಮತ್ತೊಂದು ತುದಿಯವರೆಗೂ ಪ್ರತಿಧ್ವನಿತವಾಗುತ್ತಿತ್ತು. ಆದ್ದರಿಂದ ಕೆಲವರು ಕುತೂಹಲ ಪರವಶರಾಗಿಯೂ, ಕೆಲವರು ತತ್ಯಾನ್ವೇಷಿಗಳಾಗಿಯೂ, ಮತ್ತೆ ಕೆಲವರು ಸ್ವಾಮಿಗಳ ಜ್ಞಾನ ಮಹತ್ವವನ್ನು ಪರೀಕ್ಷೆ ಮಾಡುವುದಕ್ಕಾಗಿಯೂ, ಆಗ ಸ್ವಾಮಿಗಳ ಹತ್ತಿರಕ್ಕೆ ಬರುತ್ತಿದ್ದರು. ಶಿಷ್ಯನು ನೋಡುತ್ತಿದ್ದ ಹಾಗೆಯೇ ಪ್ರಶ್ನೆ ಮಾಡುವವರು ಸ್ವಾಮಿಗಳ ಶಾಸ್ತ್ರ ವ್ಯಾಖ್ಯಾನವನ್ನು ಕೇಳಿ ಬೆರಗಾಗಿ ಹೋಗುತ್ತಿದ್ದರು. ಅವರ ಅಸಾಧಾರಣ ಪ್ರತಿಭೆಯನ್ನು ನೋಡಿ ದೊಡ್ಡ ದೊಡ್ಡ ದಾರ್ಶನಿಕರೂ ವಿಶ್ವವಿದ್ಯಾನಿಲಯದ ಪ್ರಸಿದ್ಧ ಪಂಡಿತರೂ ಮಂಕುಬಡಿದು ಕುಳಿತುಕೊಳ್ಳುತ್ತಿದ್ದರು. ಸ್ವಾಮಿಗಳ ಕಂಠದಲ್ಲಿ ಸರಸ್ವತಿ ಸರ್ವದಾ ಇರುತ್ತಿದ್ದಳೆಂದು ತೋರುತ್ತದೆ. ಈ ತೋಟದಲ್ಲಿ ಇದ್ದಾಗಲೇ ಅವರ ಅಲೌಕಿಕವಾದ ಯೋಗದೃಷ್ಟಿಯ ಪರಿಚಯವು ಆಗಾಗ್ಗೆ ಉಂಟಾಗುತ್ತಿತ್ತು.

ಕಲ್ಕತ್ತೆಯ ಬಡೊಬಾಜಾರಿನಲ್ಲಿ ಬಹು ಪಂಡಿತರು ವಾಸಿಸುವರು. ಹಣವಂತರಾದ ಮಾರವಾಡಿ ವರ್ತಕರ ಅನ್ನದಿಂದ ಇವರು ಪೋಷಿತರು. ಈ ವೇದಶಾಸ್ತ್ರಜ್ಞರಾದ ಪಂಡಿತರೆಲ್ಲಾ ಈ ಕಾಲದಲ್ಲಿ ಸ್ವಾಮಿಗಳ ಕೀರ್ತಿಯನ್ನು ಕೇಳಿದ್ದರು. ಅವರಲ್ಲಿ ಕೆಲವರು ದೊಡ್ಡ ಪಂಡಿತರು ಸ್ವಾಮಿಗಳೊಡನೆ ವಾಕ್ಯಾರ್ಥ ಮಾಡುವ ಅಭಿಲಾಷೆಯಿಂದ ಒಂದು ದಿನ ಈ ತೋಟಕ್ಕೆ ಬಂದರು. ಶಿಷ್ಯನು ಆ ದಿವಸ ಅಲ್ಲಿಗೆ ಹೋಗಿದ್ದನು.

ಬಂದಿದ್ದ ಪಂಡಿತರೆಲ್ಲಾ ಸಂಸ್ಕೃತ ಭಾಷೆಯಲ್ಲಿ ನಿರರ್ಗಳವಾಗಿ ಮಾತನಾಡುವ ಸಾಮರ್ಥ್ಯವುಳ್ಳವರಾಗಿದ್ದರು. ಅವರು ಬಂದಕೂಡಲೆ ಗುಂಪಿನಿಂದ ಸುತ್ತುವರಿಯಲ್ಪಟ್ಟಿದ್ದ ಸ್ವಾಮಿಗಳನ್ನು ಮಾತನಾಡಿಸಿ ಅವರೊಡನೆ ಸಂಸ್ಕೃತದಲ್ಲಿ ಮಾತು ಮೊದಲು ಮಾಡಿದರು. ಸ್ವಾಮೀಜಿ ಸಂಸ್ಕೃತದಲ್ಲಿಯೇ ಅವರಿಗೆ ಉತ್ತರ ಕೊಡುತ್ತಿದ್ದರು. ಯಾವ ವಿಷಯವನ್ನು ಕುರಿತು ಸ್ವಾಮೀಜಿಯೊಡನೆ ಆ ದಿನ ಪಂಡಿತರು ವಾಕ್ಯಾರ್ಥ ಮಾಡಿದರೊ ಅದು ಈಗ ಶಿಷ್ಯನಿಗೆ ಜ್ಞಾಪಕವಿಲ್ಲ. ಆದರೆ ಇಷ್ಟು ಮಾತ್ರ ಜ್ಞಾಪಕವಿದೆ, ಪಂಡಿತರೆಲ್ಲರೂ ಸಾಮಾನ್ಯವಾಗಿ ಒಟ್ಟಿಗೆ ಗದ್ದಲಮಾಡಿ ಸ್ವಾಮಿಜಿಗೆ ದರ್ಶನಗಳ ಕೂಟಪ್ರಶ್ನೆಗಳನ್ನು ಹಾಕುತ್ತಿದ್ದರು. ಸ್ವಾಮೀಜಿ ಪ್ರಶಾಂತ ಗಂಭೀರಭಾವದಿಂದ ನಿಧಾನವಾಗಿ ಈ ವಿಷಯಗಳಲ್ಲಿ ತಮ್ಮ ವಿಚಾರಪೂರ್ವಕವಾದ ಸಿದ್ಧಾಂತವನ್ನು ಹೇಳುತ್ತಿದ್ದರು. ಒಂದು ಅಂಶ ಚೆನ್ನಾಗಿ ಜ್ಞಾಪಕವಿದೆ. ಅದೇನೆಂದರೆ - ಸ್ವಾಮೀಜಿಯ ಸಂಸ್ಕೃತ ಮಾತು ಪಂಡಿತರ ಮಾತಿಗಿಂತಲೂ ಕಿವಿಗೆ ಇಂಪಾಗಿಯೂ ಸುಲಲಿತವಾಗಿಯೂ ಇತ್ತು. ಪಂಡಿತರು ಆಮೇಲೆ ಇದನ್ನು ಒಪ್ಪಿಕೊಂಡರು.

ಸಂಸ್ಕೃತ ಭಾಷೆಯಲ್ಲಿ ಸ್ವಾಮೀಜಿ ಹೀಗೆ ನಿರರ್ಗಳವಾಗಿ ಮಾತನಾಡುವುದನ್ನು ನೋಡಿ ಅವರ ಗುರುಭ್ರಾತೃಗಳೂ ಆ ದಿನ ಸ್ತಂಭೀಭೂತರಾಗಿದ್ದರು. ಏಕೆಂದರೆ, ಇದಕ್ಕೆ ಹಿಂದೆ ಆರು ವರ್ಷಕಾಲ ಯೂರೋಪ್ ಮತ್ತು ಅಮೆರಿಕಾದಲ್ಲಿ ಇರುವಾಗ ಸಂಸ್ಕೃತವನ್ನು ಮಾತನಾಡುವುದಕ್ಕೆ ಸ್ವಾಮೀಜಿಗೆ ಅಷ್ಟು ಅವಕಾಶ ದೊರೆಯಲಿಲ್ಲವೆಂಬುದು ಅವರಿಗೆ ಗೊತ್ತಿತ್ತು. ಶಾಸ್ತ್ರವನ್ನು ತಿಳಿದ ಈ ಪಂಡಿತ ಸಮೂಹದೊಡನೆ ಹೀಗೆ ವಾಕ್ಯಾರ್ಥ ಮಾಡಿದ್ದರಿಂದ, ಸ್ವಾಮಿಗಳಲ್ಲಿ ಅದ್ಭುತಶಕ್ತಿ ಸ್ಫುರಿಸಿತ್ತೆಂದು ಅವರು ತಿಳಿದುಕೊಂಡರು. ಆ ದಿನ ಈ ಸಭೆಯಲ್ಲಿ ರಾಮಕೃಷ್ಣಾನಂದ, ಯೋಗಾನಂದ, ನಿರ್ಮಲಾನಂದ, ತುರೀಯಾನಂದ ಮತ್ತು ಶಿವಾನಂದ ಸ್ವಾಮಿಗಳೂ ಇದ್ದರು.

ಸ್ವಾಮೀಜಿ ಪಂಡಿತರೊಡನೆ ವಾದಮಾಡುವಾಗ ಸಿದ್ಧಾಂತ ಪಕ್ಷವನ್ನು ಹಿಡಿದಿದ್ದರು; ಪಂಡಿತರು ಪೂರ್ವಪಕ್ಷ ವಹಿಸಿದ್ದರು. ಶಿಷ್ಯನಿಗೆ ಒಂದು ವಿಷಯ ಜ್ಞಾಪಕಕ್ಕೆ ಬಂತು; ಏನೆಂದರೆ, ಬಹಳ ಹೊತ್ತಿನಮೇಲೆ ಸ್ವಾಮೀಜಿ ಒಂದು ಕಡೆ ‘ಸ್ವಸ್ತಿ’ ಎಂದು ಹೇಳಬೇಕಾಗಿದ್ದಾಗ ‘ಅಸ್ತಿ’ ಎನ್ನಲು, ಪಂಡಿತರು ನಗುವುದಕ್ಕೆ ತೊಡಗಿದರು. ಅದನ್ನು ನೋಡಿ ಸ್ವಾಮೀಜಿ ಆ ಕ್ಷಣವೆ “ಪಂಡಿತಾನಾಂ ದಾಸೋಽಹಂ ಕ್ಷಂತವ್ಯ ಮೇ ತತ್ ಸ್ಖಲನಂ" (ನಾನು ಪಂಡಿತರ ದಾಸ, ನನ್ನ ವ್ಯಾಕರಣದ ತಪ್ಪನ್ನು ಕ್ಷಮಿಸಬೇಕು) ಎಂದರು. ಪಂಡಿತರೂ ಸ್ವಾಮಿಗಳ ಈ ವಿಧವಾದ ನಮ್ರತೆಯನ್ನು ನೋಡಿ ಬೆರಗಾಗಿ ಹೋದರು. ಬಹಳ ಹೊತ್ತು ವಾದವಾದ ಮೇಲೆ ಸಿದ್ಧಾಂತ ಪಕ್ಷವನ್ನು ಹಿಡಿದು ವಿಚಾರಮಾಡಿದ್ದು ಸಮರ್ಪಕವಾಯಿತೆಂದು ಪಂಡಿತರು ಒಪ್ಪಿಕೊಂಡು, ಸಂತೋಷದಿಂದ ಮಾತನಾಡಿ ಹೊರಡುವುದಕ್ಕೆ ಸಿದ್ಧರಾದರು. ಅಲ್ಲಿಗೆ ಬಂದಿದ್ದ ನಾಲಕೈದು ಜನ ದೊಡ್ಡ ಮನುಷ್ಯರು ಆಗ ಅವರ ಹಿಂದೆ ಹೋಗಿ “ಮಹಾಶಯರೆ, ಸ್ವಾಮಿಜಿ ವಿಷಯವಾಗಿ ತಾವೇನು ತಿಳಿದುಕೊಳ್ಳೋಣವಾಯಿತು?" ಎಂದು ಕೇಳಿದರು. ಅದಕ್ಕೆ ಅವರಲ್ಲಿ ಎಲ್ಲರಿಗಿಂತಲೂ ಹೆಚ್ಚು ವಯಸ್ಸಾಗಿದ್ದ ಪಂಡಿತರು “ವ್ಯಾಕರಣದಲ್ಲಿ ಗಂಭೀರವಾದ ವ್ಯುತ್ಪತ್ತಿ ಇಲ್ಲದಿದ್ದರೂ ಸ್ವಾಮಿಗಳು ಶಾಸ್ತ್ರದ ಗೂಢಾರ್ಥವನ್ನು ಬಲ್ಲವರು; ವಾಕ್ಯಾರ್ಥ ಮಾಡುವುದರಲ್ಲಿ ಅದ್ವಿತೀಯರು; ಮತ್ತು ಪ್ರತಿಭಾಬಲದಿಂದ ಖಂಡನೆ ಮಾಡುತ್ತಾ ಅದ್ಭುತ ಪಾಂಡಿತ್ಯವನ್ನು ತೋರಿಸಿದ್ದಾರೆ" ಎಂದು ಉತ್ತರ ಕೊಟ್ಟರು.

ಸ್ವಾಮಿಜಿಯವರ ಮೇಲೆ ಅವರ ಗುರುಭ್ರಾತೃಗಳಿಗೆ ಯಾವಾಗಲೂ ಎಂಥ ಅದ್ಭುತವಾದ ಪ್ರೀತಿ ಇರುತ್ತಿತ್ತು! ಪಂಡಿತರೊಡನೆ ಸ್ವಾಮಿಗಳಿಗೆ ಜಟಿಲ ವಿಚಾರ ಹತ್ತಿಕೊಂಡಾಗ, ಅವರು ಕುಳಿತಿದ್ದ ತೊಟ್ಟಿಗೆ ಉತ್ತರದಲ್ಲಿದ್ದ ಕೊಠಡಿಯಲ್ಲಿ ರಾಮಕೃಷ್ಣಾನಂದ ಸ್ವಾಮಿಗಳು ಜಪಮಾಡುತ್ತಿದ್ದುದು ಶಿಷ್ಯನ ಕಣ್ಣಿಗೆ ಬಿತ್ತು. ಪಂಡಿತರೆಲ್ಲರೂ ಹೊರಟು ಹೋದ ಮೇಲೆ ಇದರ ಕಾರಣವನ್ನು ವಿಚಾರಮಾಡಲಾಗಿ, ಸ್ವಾಮಿಗಳಿಗೆ ವಿಜಯವಾಗುವುದಕ್ಕಾಗಿಯೆ ಒಂದೇ ಮನಸ್ಸಿನಿಂದ ಪರಮಹಂಸರ ಪಾದಪದ್ಮದಲ್ಲಿ ಅವರು ಪ್ರಾರ್ಥನೆ ಮಾಡುತ್ತಿದ್ದರೆಂದು ತಿಳಿದುಬಂತು!

ಪಂಡಿತರು ಹೊರಟುಹೋದ ಮೇಲೆ, ಪೂರ್ವಪಕ್ಷ ಮಾಡುತ್ತಿದ್ದ ಆ ಪಂಡಿತರು ಪೂರ್ವಮೀಮಾಂಸಾ ಶಾಸ್ತ್ರದಲ್ಲಿ ದೊಡ್ಡ ವಿದ್ವಾಂಸರೆಂದು, ಶಿಷ್ಯನು ಸ್ವಾಮಿಗಳಿಂದ ತಿಳಿದುಕೊಂಡ. ಸ್ವಾಮೀಜಿ ಉತ್ತರಮೀಮಾಂಸಾ ಪಕ್ಷವನ್ನು ಹಿಡಿದು ಅವರಿಗೆ ಜ್ಞಾನಕಾಂಡದ ಶ್ರೇಷ್ಠತೆಯನ್ನು ತೋರಿಸಿದರು. ಪಂಡಿತರೂ ಸ್ವಾಮಿಜಿಯವರ ಸಿದ್ಧಾಂತವನ್ನು ಒಪ್ಪಿಕೊಳ್ಳಲೇಬೇಕಾಯಿತು.

ವ್ಯಾಕರಣದ ಒಂದು ತಪ್ಪನ್ನು ಹಿಡಿದುಕೊಂಡು ಪಂಡಿತರು ಸ್ವಾಮಿಗಳನ್ನು ಹಾಸ್ಯಮಾಡಿದರಷ್ಟೆ, ಆ ವಿಚಾರದಲ್ಲಿ ಸ್ವಾಮೀಜಿ “ಬಹು ವರ್ಷಗಳಿಂದಲೂ ಸಂಸ್ಕೃತದಲ್ಲಿ ಮಾತು ಕಥೆಗಳನ್ನಾಡದೆ ಇದ್ದದ್ದರಿಂದ ಹೀಗೆ ತಪ್ಪು ಬಂದು ಬಿಟ್ಟಿತು" ಎಂದು ಹೇಳಿದರು. ಪಂಡಿತರನ್ನು ಇದಕ್ಕಾಗಿ ಅವರು ಸ್ವಲ್ಪವೂ ಆಕ್ಷೇಪಣೆಮಾಡಲಿಲ್ಲ. ಆದರೆ ಈ ವಿಷಯವಾಗಿ ಸ್ವಾಮಿಗಳು ಈ ಒಂದು ಅಂಶವನ್ನೇನೊ ಹೇಳಿದರು; ಏನೆಂದರೆ - ಪಾಶ್ಚಾತ್ಯ ದೇಶದಲ್ಲಿ ವಾದ ಮಾಡುತ್ತಿರುವಾಗ ಮೂಲ ವಿಷಯವನ್ನು ಬಿಟ್ಟು ಹೀಗೆ ಭಾಷೆಯ ಸಾಮಾನ್ಯವಾದ ಒಂದು ದೋಷವನ್ನು ಹಿಡಿಯುವುದು ಪ್ರತಿಪಕ್ಷದವರ ಮಹಾ ಅಸೌಜನ್ಯವೆಂದು ಪರಿಗಣಿಸಲ್ಪಡುತ್ತದೆ. ಭಾಷೆಯ ಕಡೆಗೆ ಲಕ್ಷ್ಯವನ್ನೇ ಕೊಡುವುದಿಲ್ಲ. “ನಿಮ್ಮ ದೇಶದಲ್ಲಿಯಾದರೋ ಹೊಟ್ಟೆಗಾಗಿಯೇ ಹೊಡೆದಾಟವಾಗುತ್ತದೆ - ಒಳಗಿರುವ ತಿರುಳನ್ನು ಕೇಳುವವರೇ ಇಲ್ಲ" – ಎಂದು ಹೇಳಿ ಸ್ವಾಮೀಜಿ ಶಿಷ್ಯನೊಡನೆ ಆ ದಿವಸ ಸಂಸ್ಕೃತದಲ್ಲಿ ಮಾತನಾಡುವುದಕ್ಕೆ ಮೊದಲುಮಾಡಿದರು. ಶಿಷ್ಯನೂ ಹರಕು ಮುರುಕು ಸಂಸ್ಕೃತದಲ್ಲಿ ಉತ್ತರ ಕೊಡುತ್ತಿದ್ದನು. ಅಂದಿನಿಂದ ಶಿಷ್ಯನು ಸ್ವಾಮಿಜಿಯವರ ಇಷ್ಟದಂತೆ ಅವರೊಡನೆ ಸಾಮಾನ್ಯವಾಗಿ ಆಗಾಗ್ಗೆ ದೇವಭಾಷೆಯಲ್ಲಿ (ಸಂಸ್ಕೃತದಲ್ಲಿ) ಮಾತನಾಡುತ್ತಿದ್ದನು.

“ನಾಗರಿಕತೆ” ಎಂಬುದು ಯಾವುದು? ಎಂಬ ಪ್ರಶ್ನೆಗೆ ಉತ್ತರವಾಗಿ ಅಂದು ಸ್ವಾಮೀಜಿ ಹೀಗೆ ಹೇಳಿದ್ದರು: “ಯಾವ ರಾಷ್ಟ್ರ ಅಥವಾ ಸಮಾಜ ಆಧ್ಯಾತ್ಮಿಕ ಭಾವದಲ್ಲಿ ಎಷ್ಟೆಷ್ಟು ಮುಂದಾಗಿದೆಯೊ ಆ ರಾಷ್ಟ್ರ ಮತ್ತು ಸಮಾಜ ಅಷ್ಟಷ್ಟು ನಾಗರಿಕ. ನಾನಾ ಕಾರ್ಖಾನೆಗಳನ್ನು ಇಟ್ಟುಕೊಂಡು ಐಹಿಕ ಜೀವನದ ಸುಖ ಸ್ವಾಚ್ಛಂದ್ಯಗಳನ್ನು ಹೆಚ್ಚಿಸಿಕೊಂಡುಬಿಟ್ಟ ಮಾತ್ರದಿಂದಲೆ ಯಾವುದಾದರೊಂದು ರಾಷ್ಟ್ರವು ನಾಗರಿಕವಾಯಿತೆಂದು ಹೇಳುವುದಕ್ಕಾಗುವುದಿಲ್ಲ. ಈಗಿನ ಪಾಶ್ಚಾತ್ಯ ನಾಗರಿಕತೆ ಜನರ ಹಾಹಾಕಾರವನ್ನೂ ಅಭಾವವನ್ನೂ ದಿನೇ ದಿನೇ ಹೆಚ್ಚಿಸುತ್ತಿದೆ. ಆದರೆ ಭಾರತೀಯ ಪ್ರಾಚೀನ ನಾಗರಿಕತೆ ಸರ್ವಸಾಧಾರಣರಿಗೂ ಆಧ್ಯಾತ್ಮಿಕ ಉನ್ನತಿಯ ಮಾರ್ಗವನ್ನು ತೋರಿಸುತ್ತಾ, ಜನಗಳ ಐಹಿಕ ಅಭಾವವನ್ನು ಒಟ್ಟಿಗೆ ಹೋಗಲಾಡಿಸಲಾರದಿದ್ದರೂ ಬಹಳಮಟ್ಟಿಗೆ ಅದನ್ನು ಕಡಿಮೆ ಮಾಡುತ್ತಾ ಇತ್ತು; ಅದರಲ್ಲಿ ಸಂದೇಹವಿಲ್ಲ. ಈಗಿನ ಕಾಲದಲ್ಲಿ ಈ ಎರಡು ವಿಧವಾದ ನಾಗರಿಕತೆಯನ್ನೂ ಒಂದು ಕಡೆ ಸೇರಿಸುವುದಕ್ಕೆ ಭಗವಾನ್ ಶ‍್ರೀ ರಾಮಕೃಷ್ಣ ಪರಮಹಂಸರು ಅವತಾರಮಾಡಿದರು. ಈಗಿನ ಕಾಲದಲ್ಲಿ ಜನರು ಒಂದು ಕಡೆಯಲ್ಲಿ ಹೇಗೆ ಕರ್ಮತತ್ಪರರಾಗಬೇಕೊ, ಹಾಗೆ ಮತ್ತೊಂದು ಕಡೆಯಲ್ಲಿ ಗಂಭೀರವಾದ ಆಧ್ಯಾತ್ಮಿಕ ಜ್ಞಾನವನ್ನು ಸಂಪಾದಿಸಬೇಕು." ಹೀಗೆ ಭಾರತೀಯ ಮತ್ತು ಪಾಶ್ಚಾತ್ಯ ನಾಗರಿಕತೆಯ ಪರಸ್ಪರ ಸಮ್ಮಿಶ್ರಣದಿಂದ ಜಗತ್ತಿನಲ್ಲಿ ಹೊಸ ಯುಗ ಆರಂಭವಾಗುತ್ತದೆಂಬ ವಿಷಯವನ್ನೂ ಆ ದಿನ ಸ್ವಾಮೀಜಿ ವಿಶೇಷವಾಗಿ ತಿಳಿಸಿದರು. ಇದನ್ನು ತಿಳಿಸುತ್ತ ತಿಳಿಸುತ್ತ ಸ್ವಾಮೀಜಿ ಒಂದು ಕಡೆ ಹೀಗೆ ಹೇಳಿದರು: “ಮತೊಂದು ವಿಷಯ - ಧರ್ಮಪರಾಯಣರಾದಷ್ಟೂ ಜನರು ಹೊರಗಿನ ಆಚಾರ ವ್ಯವಹಾರಗಳಲ್ಲಿ ಅಷ್ಟು ಗಂಭೀರರಾಗಬೇಕೆಂದು ಆ ದೇಶದ ಜನರ ಭಾವನೆ; ಬಾಯಲ್ಲಿ ಬೇರೆ ಮಾತೇ ಬರುವುದಿಲ್ಲ. ಒಂದು ಕಡೆಯಲ್ಲಿ ನನ್ನ ಬಾಯಿಂದ ಉದಾರವಾದ ಧಾರ್ಮಿಕ ವಿಷಯಗಳನ್ನು ಕೇಳಿ ಆ ದೇಶದ ಧರ್ಮಪ್ರಚಾರಕರು ಎಷ್ಟೇ ಬೆರಗಾಗುತ್ತಿದ್ದರೊ, ಉಪನ್ಯಾಸವಾದ ಮೇಲೆ ಇಷ್ಟಮಿತ್ರರೊಡನೆ ನಾನು ತಿಂಡಿ ತೀರ್ಥಗಳನ್ನು ಮಾಡುವುದನ್ನು ನೋಡಿ ಅಷ್ಟೇ ಬೆರಗಾಗುತ್ತಿದ್ದರು. ಬಾಯಿಬಿಟ್ಟು ಕೆಲವು ಸಾರಿ ‘ಸ್ವಾಮೀಜಿ! ತಾವು ಧರ್ಮಪ್ರಚಾರಕರು; ಸಾಧಾರಣ ಜನರಂತೆ ಹೀಗೆ ಹಾಸ್ಯ ಪರಿಹಾಸ್ಯಗಳನ್ನು ಮಾಡುವುದು ತಮಗೆ ಯೋಗ್ಯವಲ್ಲ, ಈ ವಿಧವಾದ ಚಪಲತೆ ತಮಗೆ ಗೌರವ ಜನಕವಲ್ಲ’ ಎಂದು ಹೇಳಿಯೂ ಬಿಡುತ್ತಿದ್ದರು. ಅದಕ್ಕೆ ನಾನು ‘ನಾವು ಬ್ರಹ್ಮಾನಂದದ ಮಕ್ಕಳು, ಮುಖವನ್ನೇಕೆ ಜೋಲು ಬಿಟ್ಟುಕೊಂಡಿರಬೇಕು?’ ಎಂದು ಉತ್ತರ ಕೊಡುತ್ತಿದ್ದೆನು. ಈ ಮಾತನ್ನು ಕೇಳಿ ಅವರು ಅದರ ಗೂಢಾರ್ಥವನ್ನು ಗ್ರಹಿಸುತ್ತಿದ್ದರೋ ಇಲ್ಲವೊ ಸಂದೇಹ."

ಆ ದಿನ ಸ್ವಾಮೀಜಿ ಭಾವಸಮಾಧಿ ಮತ್ತು ನಿರ್ವಿಕಲ್ಪ ಸಮಾಧಿಗಳ ವಿಚಾರವಾಗಿಯೂ ನಾನಾ ವಿಷಯಗಳನ್ನು ಹೇಳಿದರು. ಸಾಧ್ಯವಾದಷ್ಟು ಮಟ್ಟಿಗೆ ಅದು ಮುಂದೆ ನಿರೂಪಿಸಲ್ಪಟ್ಟಿದೆ:

“ಒಬ್ಬನು ಹನುಮಂತನ ಹಾಗೆ ಭಕ್ತಿಭಾವದಲ್ಲಿ ಸಾಧನೆಯನ್ನು ಮಾಡುತ್ತಿದ್ದಾನೆಂದುಕೊ! ಭಾವ ಹೇಗೆ ಹೇಗೆ ಗಾಢವಾಗುತ್ತ ಹೋಗುತ್ತದೆಯೋ, ಹಾಗೆ ಹಾಗೆ ಆ ಸಾಧಕನ ನಡೆ ನುಡಿ ಭಾವ ಭಂಗಿ - ಅಷ್ಟೆ ಏಕೆ? - ದೇಹದ ಅಂಗಾಂಗ ರಚನೆಯೂ ಆಗುತ್ತದೆ. "ಜಾತ್ಯಂತರ ಪರಿಣಾಮವು“ ಹೀಗೆಯೇ ಆಗುವುದು. ಹೀಗೆ ಒಂದು ಭಾವವನ್ನು ಹಿಡಿದ ಸಾಧಕನು ಕ್ರಮವಾಗಿ ಅದೇ ರೂಪವಾಗಿ ಪರಿಣಮಿಸಿಬಿಡುವನು. ಯಾವುದಾದರೂ ಒಂದು ವಿಧವಾದ ಭಾವದ ಕೊನೆಯ ಅವಸ್ಥೆಗೇ ‘ಭಾವಸಮಾಧಿ’ ಎಂದು ಹೆಸರು. ಆಮೇಲೆ ‘ನಾನು ದೇಹವಲ್ಲ’ ‘ಮನಸ್ಸಲ್ಲ’ ‘ಬುದ್ಧಿಯಲ್ಲ’ ಎಂದು ‘ನೇತಿ’ ‘ನೇತಿ’ ಮಾರ್ಗವನ್ನು ಅನುಸರಿಸಿ ಜ್ಞಾನಿಯಾದ ಸಾಧಕನು ಚಿನ್ಮಾತ್ರ ಸ್ಥಿತಿಯಲ್ಲಿ ಇದ್ದು ಬಿಟ್ಟರೆ, ನಿರ್ವಿಕಲ್ಪ ಸಮಾಧಿಯುಂಟಾಗುತ್ತದೆ. ಒಂದೊಂದೇ ಭಾವವನ್ನು ತೆಗೆದುಕೊಂಡು ಅದರಲ್ಲಿ ಸಿದ್ಧನಾಗಬೇಕಾದರೆ - ಅಥವಾ ಆ ಭಾವದ ಕೊನೆಯ ಅವಸ್ಥೆಗೆ ಹೋಗಬೇಕಾದರೆ - ಎಷ್ಟೋ ಜನ್ಮಗಳು ಹಿಡಿಯುತ್ತವೆ. ಹೀಗಿರಲು ಭಾವರಾಜ್ಯದ ರಾಜರಾದ ನಮ್ಮ ಪರಮಹಂಸರಾದರೊ ಹದಿನೆಂಟು ಭಾವಗಳಲ್ಲಿ ಸಿದ್ಧಿಯನ್ನು ಪಡೆದರು! ಭಾವಮುಖದಲ್ಲಿ ಇಲ್ಲದೆ ಇದ್ದರೆ ಅವರ ಶರೀರ ನಿಲ್ಲುತ್ತಿರಲಿಲ್ಲ - ಎಂದೂ ಅವರು ಹೇಳುತ್ತಿದ್ದರು."

ಮಾತುಕತೆಯಾದ ಮೇಲೆ ಶಿಷ್ಯನು ಆ ದಿವಸ “ಮಹಾಶಯರೆ! ಆ ದೇಶದಲ್ಲಿ ಊಟ ಉಪಚಾರಾದಿಗಳನ್ನು ಹೇಗೆ ಮಾಡುತ್ತಿದ್ದಿರಿ?" ಎಂದು ಕೇಳಿದನು.

ಸ್ವಾಮಿಜಿ: ಆ ದೇಶದವರಂತೆಯೆ ಮಾಡುತ್ತಿದ್ದೆ; ನಾನು ಸಂನ್ಯಾಸಿ; ನನಗೆ ಯಾತರಿಂದಲೂ ಜಾತಿ ಹೋಗುವುದಿಲ್ಲ.

ಈ ದೇಶದಲ್ಲಿ ಅವರು ಮುಂದೆ ಯಾವ ರೀತಿ ಕಾರ್ಯ ಮಾಡಬೇಕೆಂದಿದ್ದರೊ ಆ ವಿಷಯವಾಗಿಯೂ ಆ ದಿನ ಸ್ವಾಮೀಜಿ ಹೀಗೆಂದರು:

“ಮದ್ರಾಸು ಮತ್ತು ಕಲ್ಕತ್ತ ಈ ಎರಡೂ ಕಡೆ ಕೇಂದ್ರಗಳನ್ನು ಸ್ಥಾಪಿಸಿ ಸಮಸ್ತ ವಿಧವಾದ ಲೋಕಕಲ್ಯಾಣಕ್ಕೋಸ್ಕರ ಹೊಸಪರಿಯ ಸಾಧುಸಂನ್ಯಾಸಿಗಳನ್ನು ಸಿದ್ಧಪಡಿಸುತ್ತೇನೆ. ಮತ್ತೊಂದು ವಿಷಯವೇನೆಂದರೆ, ಧ್ವಂಸದ ಮೂಲಕ - ಅಥವಾ ಪ್ರಾಚೀನ ರೀತಿನೀತಿಗಳನ್ನು ತಲೆಕೆಳಗು ಮಾಡಿ - ಸಮಾಜದ ಅಥವಾ ದೇಶದ ಉನ್ನತಿಯನ್ನು ಸಾಧಿಸುವುದಕ್ಕಾಗುವುದಿಲ್ಲ. ಎಲ್ಲಾ ಕಾಲಗಳಲ್ಲಿಯೂ ಉನ್ನತಿಯು ನಿರ್ಮಾಣ ಮಾರ್ಗದ ಮೂಲಕ, ಎಂದರೆ, ಪ್ರಾಚೀನರೀತಿ ಮುಂತಾದ್ದನ್ನು ಹೊಸ ವಿಧದಲ್ಲಿ ಬದಲಾಯಿಸಿಕೊಂಡೇ ಬಂದಿರುವುದು. ಭರತಖಂಡದ ಧರ್ಮ ಪ್ರಚಾರಕರು ಮಾತ್ರವೇ ಹಿಂದಿನ ಕಾಲಗಳಲ್ಲಿ ಹೀಗೆ ಮಾಡಿದ್ದಾರೆ. ಬುದ್ಧದೇವನ ಧರ್ಮ ಒಂದು ಮಾತ್ರ ಪ್ರಾಚೀನ ರೀತಿನೀತಿಗಳನ್ನು ಧ್ವಂಸಪಡಿಸತಕ್ಕದ್ದು ಆಗಿತ್ತು. ಅದಕ್ಕೋಸ್ಕರವೆ ಈ ಮತ ಭರತಖಂಡದಲ್ಲಿ ನಿರ್ನಾಮವಾಗಿ ಹೋಗಿದೆ."

ಸ್ವಾಮಿಜಿ ಈ ರೀತಿಯಲ್ಲಿ ಮಾತನಾಡುತ್ತ ಆಡುತ್ತ ಮತ್ತೊಂದು ವಿಷಯವನ್ನು ಹೇಳುತ್ತಿದ್ದದ್ದು ಶಿಷ್ಯನಿಗೆ ಜ್ಞಾಪಕವಿದೆ - ಒಬ್ಬ ಮನುಷ್ಯನಲ್ಲಿ ಬ್ರಹ್ಮವಿಕಾಸವಾದರೆ ಸಾವಿರಾರು ಜನರು ಅದರ ಬೆಳಕಿನಲ್ಲಿ ದಾರಿಯನ್ನು ಕಂಡುಕೊಂಡು ಮುಂದುವರಿಯುತ್ತಾರೆ. ಬ್ರಹ್ಮಜ್ಞ ಪುರುಷರು ಮಾತ್ರವೇ ಲೋಕ ಗುರುಗಳು; ಈ ವಿಷಯವು ಸರ್ವಶಾಸ್ತ್ರ ಮತ್ತು ಯುಕ್ತಿಗಳಿಂದ ಸಾಧಿಸಲ್ಪಟ್ಟಿದೆ. ಅವೈದಿಕವೂ ಅಶಾಸ್ತ್ರೀಯವೂ ಆದ ಕುಲಗುರು ಪದ್ದತಿಯನ್ನು ಸ್ವಾರ್ಥಪರರಾದ ಬ್ರಾಹ್ಮಣರೆ ಈ ದೇಶದಲ್ಲಿ ಪ್ರಚಾರಕ್ಕೆ ತಂದಿದ್ದಾರೆ. ಇದರಿಂದಲೆ ಸಾಧನೆ ಮಾಡಿದರೂ ಜನರು ಈಗ - ಸಿದ್ಧರು ಅಥವಾ ಬ್ರಹ್ಮಜ್ಞರು ಆಗಲಾರದವರಾಗಿದ್ದಾರೆ. ಧರ್ಮದ ಈ ಗ್ಲಾನಿಯನ್ನೆಲ್ಲಾ ದೂರಮಾಡುವುದಕ್ಕೆ ಶ‍್ರೀರಾಮಕೃಷ್ಣ ಪರಮಹಂಸರು ದೇಹಧಾರಣೆ ಮಾಡಿ ಈಗ ಪ್ರಪಂಚದಲ್ಲಿ ಅವತಾರ ಮಾಡಿದರು. ಅವರು ತೋರಿಸಿಕೊಟ್ಟಿರುವ ಸಾರ್ವಭೌಮಿಕ ಮತ ಜಗತ್ತಿನಲ್ಲಿ ಪ್ರಚಾರಕ್ಕೆ ಬಂದರೆ ಜಗತ್ತಿಗೂ ಜನರಿಗೂ ಮಂಗಳವಾಗುತ್ತದೆ. ಇಂಥ ಅದ್ಭುತ ಮಹಾ ಸಮನ್ವಯಾಚಾರ್ಯರು ಯಾರೂ ಅನೇಕ ಶತಮಾನಗಳಿಂದ ಈಚೆಗೆ ಭರತಖಂಡದಲ್ಲಿ ಅವತಾರ ಮಾಡಿರಲಿಲ್ಲ.

ಸ್ವಾಮಿಜಿಯವರ ಗುರುಭ್ರಾತೃಗಳಲ್ಲಿ ಒಬ್ಬರು ಈ ಸಮಯದಲ್ಲಿ “ನೀವು ಆ ದೇಶದಲ್ಲಿ ಯಾವಾಗಲೂ ಎಲ್ಲರೆದುರಿಗೂ ಪರಮಹಂಸರು ಅವತಾರ ಪುರುಷರೆಂದು ಏಕೆ ಪ್ರಚಾರಮಾಡಲಿಲ್ಲ?" ಎಂದು ಕೇಳಿದರು.

ಸ್ವಾಮೀಜಿ: ಅವರು ದರ್ಶನ ವಿಜ್ಞಾನಗಳನ್ನು ದೊಡ್ಡದುಮಾಡಿ ಮುಂದೆ ತಂದಿಡುತ್ತಾರೆ. ಆದ್ದರಿಂದ ಯುಕ್ತಿ ತರ್ಕ ದರ್ಶನ ವಿಜ್ಞಾನ ಇವುಗಳ ಮೂಲಕವೆ ಅವರ ಪಾಂಡಿತ್ಯ ಪ್ರತಿಷ್ಠೆ ಪುಡಿಪುಡಿ ಮಾಡದಿದ್ದರೆ, ಯಾವ ವಿಧವಾದ ಮರ್ಯಾದೆಯೂ ಬರುವುದಿಲ್ಲ. ಯುಕ್ತಿ ತರ್ಕಗಳಿಗೆ ಸೋತು ನಿಜವಾಗಿ ತತ್ಯಾನ್ವೇಷಿಗಳಾಗಿ ಯಾರು ನನ್ನ ಹತ್ತಿರಕ್ಕೆ ಬರುತ್ತಿದ್ದರೊ ಅವರಿಗೆ ಪರಮಹಂಸರ ವಿಚಾರವನ್ನು ಹೇಳುತ್ತಿದ್ದೆ. ಹಾಗಲ್ಲದೆ ಥಟ್ಟನೆ ಅವತಾರದ ವಿಷಯವನ್ನು ಹೇಳುತ್ತಿದ್ದರೆ ಅವರು “ಅದೇನು ಹೊಸ ವಿಷಯ ನೀವು ಹೇಳುವುದು? - ನಮ್ಮ ಪ್ರಭು ಏಸು ಕ್ರಿಸ್ತನೂ ಇದ್ದಾನೆ" ಎಂದು ಬಿಡುತ್ತಿದ್ದರು.

ಮೂರು ನಾಲ್ಕು ಗಂಟೆಯ ಹೊತ್ತು ಹೀಗೆ ಅತ್ಯಾನಂದದಲ್ಲಿ ಕಳೆದು ಶಿಷ್ಯನು ಆ ದಿನ ಅಲ್ಲಿಗೆ ಬಂದಿದ್ದ ಇತರ ಜನರೊಡನೆ ಕಲ್ಕತ್ತೆಗೆ ಹಿಂತಿರುಗಿದನು.

\newpage

\chapter[ಅಧ್ಯಾಯ ೪]{ಅಧ್ಯಾಯ ೪\protect\footnote{\engfoot{C W, Vol. 1, P512}}}

\begin{center}
ಸ್ಪಳ: ಶ‍್ರೀ ನವಗೋಪಾಲ ಘೋಷರ ಮನೆ, ರಾಮಕೃಷ್ಣಪುರ, ಹೌರಾ, ಕ್ರಿ.ಶ. ೧೮೯೭, ಫಬ್ರವರಿ ೬.
\end{center}

ಶ‍್ರೀ ಶ‍್ರೀರಾಮಕೃಷ್ಣ ಪರಮಹಂಸರ ಪರಮಭಕ್ತರಾದ ಶ‍್ರೀಯುತ ಬಾಬೂ ನವಗೋಪಾಲ ಘೋಷ್ ಮಹಾಶಯರು ಭಾಗೀರಥಿಯ ಪಶ್ಚಿಮ ದಡದಲ್ಲಿದ್ದ ಹೌರಾಕ್ಕೆ ಸೇರಿದ ರಾಮಕೃಷ್ಣಪುರದಲ್ಲಿ ಹೊಸದಾಗಿ ಒಂದು ಮನೆಯನ್ನು ಕಟ್ಟಿಸಿದ್ದರು. ಈ ಮನೆಗಾಗಿ ಸ್ಥಳವನ್ನು ಕೊಂಡುಕೊಳ್ಳುವಾಗ ಆ ಭಾಗಕ್ಕೆ ‘ರಾಮಕೃಷ್ಣಪುರ’ವೆಂದು ಹೆಸರಿರುವುದಾಗಿ ತಿಳಿದು ತುಂಬ ಸಂತೋಷವಾಯಿತು. ಏಕೆಂದರೆ ಈ ಗ್ರಾಮದ ಹೆಸರು ಹೇಳಿದರೆ ತಮ್ಮ ಇಷ್ಟದೇವತೆ ಜ್ಞಾಪಕ ಬರುವುದು. ಮನೆ ಕಟ್ಟಿ ಮುಗಿದ ಸ್ವಲ್ಪ ದಿನಕ್ಕೆ ಸ್ವಾಮೀಜಿಯವರು ಮೊದಲನೆಯ ಸಲ ವಿಲಾಯಿತಿಯಿಂದ ಕಲ್ಕತ್ತೆಗೆ ಹಿಂತಿರುಗಿ ಬಂದರು. ಆದ್ದರಿಂದ ಘೋಷರಿಗೂ ಅವರ ಹೆಂಡತಿಗೂ, ಸ್ವಾಮಿಜಿಯಿಂದ ಆ ಮನೆಯಲ್ಲಿ ರಾಮಕೃಷ್ಣ ವಿಗ್ರಹವನ್ನು ಪ್ರತಿಷ್ಠೆ ಮಾಡಿಸಬೇಕೆಂಬುದೊಂದು ದೊಡ್ಡ ಅಪೇಕ್ಷೆಯಾಗಿತ್ತು. ಘೋಷರು ಮಠಕ್ಕೆ ಹೋಗಿ ಕೆಲವು ದಿನಗಳ ಮುಂಚೆ ಈ ಪ್ರಸ್ತಾಪವನ್ನು ಎತ್ತಿದ್ದರು. ಸ್ವಾಮಿಜಿ ಅದಕ್ಕೆ ಸಮ್ಮತಿಯನ್ನು ಕೊಟ್ಟಿದ್ದರು. ನವಗೋಪಾಲ ಬಾಬುಗಳ ಮನೆಯಲ್ಲಿ ಇಂದು ಆ ಸಂಬಂಧವಾದ ಉತ್ಸವ - ಮಠಕ್ಕೆ ಸೇರಿದ ಸಂನ್ಯಾಸಿಗಳೂ ಪರಮಹಂಸರ ಗೃಹಸ್ಥ ಭಕ್ತರೂ ಎಲ್ಲರೂ ಇಂದು ಅಲ್ಲಿಗೆ ವಿಶ್ವಾಸಪೂರ್ವಕವಾಗಿ ಆಹ್ವಾನಿಸಲ್ಪಟ್ಟಿದ್ದರು. ಮನೆ ಈಗ ತಳಿರು ತೋರಣಗಳಿಂದ ಪರಿಶೋಭಿತವಾಗಿದೆ - ಮುಂಬಾಗಿಲಿನಲ್ಲಿ ಪೂರ್ಣಕುಂಭ, ಬಾಳೆಯಕಂಬ, ದೇವದಾರು ಪತ್ರದ ತೋರಣ, ಮಾವಿನ ಕೊನೆಯ ಮತ್ತು ಹುವ್ವಿನ ಗೊಂಚಲಿನ ಸಾಲು. “ಜಯ ರಾಮಕೃಷ್ಣ!” ಎಂಬ ಶಬ್ದದಿಂದ ರಾಮಕೃಷ್ಣಪುರ ಈ ದಿನ ಅನುರಣಿತವಾಗುತ್ತಿದೆ.

ಮಠದಿಂದ ಮೂರು ದೋಣಿಗಳನ್ನು ಬಾಡಿಗೆಗೆ ಗೊತ್ತುಮಾಡಿಕೊಂಡು ಸ್ವಾಮಿಗಳೊಡನೆ ಮಠದ ಸಂನ್ಯಾಸಿಗಳು ಬ್ರಹ್ಮಚಾರಿಗಳು ಎಲ್ಲರೂ ರಾಮಕೃಷ್ಣ ಪುರದ ಘಾಟಿಗೆ ಬಂದು ತಲುಪಿದರು. ಸ್ವಾಮಿಗಳ ಪೋಷಾಕೆಲ್ಲಾ ಕಾವಿಯ ರಂಗಿನ ಧೋತ್ರ, ತಲೆಗೆ ಒಂದು ಕುಲಾವಿ, ಇಷ್ಟೆ - ಕಾಲಿನಲ್ಲಿ ಏನೂ ಇಲ್ಲ. ರಾಮಕೃಷ್ಣಪುರದ ಘಾಟಿನಿಂದ ಅವರು ಯಾವ ದಾರಿಯಲ್ಲಿ ನವಗೋಪಾಲ ಬಾಬುಗಳ ಮನೆಗೆ ಹೋಗುವವರಾಗಿದ್ದರೊ ಆ ದಾರಿಯ ಎರಡೂ ಕಡೆಯಲ್ಲಿ ಲೆಕ್ಕಿಸಲಾರದಷ್ಟು ಜನರು ಅವರ ದರ್ಶನಾಕಾಂಕ್ಷಿಗಳಾಗಿ ನಿಂತಿದ್ದರು. ಘಾಟಿನಲ್ಲಿ ಇಳಿಯುತ್ತಿದ್ದ ಹಾಗೆಯೇ ಸ್ವಾಮೀಜಿ “ದುಃಖಿನೀ ಬ್ರಾಹ್ಮಣೀ ಕೊಲೇ ಕೇ ಶುಯೇಛೆ ಆಲೋಕರೇ, ಕೇರೇ ಜರೇ ದಿಗಂಬರ ಏಸೇಛೆ ಕುಟಿರ ಘರೆ"\footnote{ವಿಪ್ರವನಿತೆ ಅಂಕದಲ್ಲಿ, ನಲಿವನಾರೆ ಬೆಳಕಚೆಲ್ಲಿ. ಧರಿಸಿ ದಿಸೆಯ ಚಲುವ ಕುವರ, ಇಳಿದು ಬಂದೆ ಬಡಕುಟೀರ.} ಎಂಬ ಕೀರ್ತನವನ್ನು ಆರಂಭಿಸಿ ತಾವೇ ಮೃದಂಗವನ್ನು ಬಾರಿಸುತ್ತ ಮುಂದೆ ಹೊರಟರು. ಇನ್ನಿಬ್ಬರು ಮೂರು ಜನರೂ ಮೃದಂಗವನ್ನು ಜೊತೆಯಲ್ಲಿ ಬಾರಿಸಲು ಮೊದಲುಮಾಡಿದರು. ಅಲ್ಲಿ ಸೇರಿದ್ದ ಭಕ್ತರೆಲ್ಲರೂ ಸಮಸ್ವರದಲ್ಲಿ ಆ ಕೀರ್ತನೆಯನ್ನು ಹಾಡುತ್ತ ಅವರ ಹಿಂದೆ ಹೊರಟರು.

ಉದ್ದಾಮ ನೃತ್ಯದಿಂದಲೂ ಮೃದಂಗ ಧ್ವನಿಯಿಂದಲೂ ರಸ್ತೆಯೂ ಘಾಟಿಯೂ ಮುಖರಿತವಾಗುತ್ತಿತ್ತು. ಹೋಗುತ್ತ ಹೋಗುತ್ತ ಈ ಗುಂಪು ಶ‍್ರೀಯುತ ಡಾಕ್ಟರ್ ರಾಮಲಾಲ ಬಾಬುಗಳ ಮನೆಯ ಹತ್ತಿರ ಸ್ವಲ್ಪ ಹೊತ್ತು ನಿಂತುಕೊಂಡಿತು. ರಾಮಲಾಲ ಬಾಬುಗಳೂ ಬಹುಬೇಗ ಮನೆಯಿಂದ ಹೊರಕ್ಕೆ ಓಡಿಬಂದು ಜೊತೆಯಲ್ಲಿ ಹೊರಟರು. ಜನರು ಮನಸ್ಸಿನಲ್ಲಿ ನೆನೆಸಿಕೊಂಡಿದ್ದದ್ದೇನೆಂದರೆ - ಸ್ವಾಮೀಜಿ ಎಷ್ಟೋ ವೇಷಭೂಷಣಗಳಿಂದಲಂಕೃತರಾಗಿ ಆಡಂಬರದಿಂದ ಹೋಗುತ್ತಾರೆಂದು. ಆದರೆ, ಅವರು ಮಠದ ಇತರ ಸಂನ್ಯಾಸಿಗಳಂತೆಯೇ ಸಾಮಾನ್ಯವಾದ ಉಡುಪಿನಲ್ಲಿಯೂ ಬರಿಯ ಕಾಲಿನಲ್ಲಿಯೂ ಮೃದಂಗವನ್ನು ಬಾರಿಸುತ್ತ ಬಾರಿಸುತ್ತ ಬಂದದ್ದನ್ನು ನೋಡಿ ಅನೇಕರಿಗೆ ಮೊದಲು ಅವರ ಗುರುತು ಹಿಡಿಯುವುದಕ್ಕೆ ಆಗದೆ ಆಮೇಲೆ ಇತರರನ್ನು ಕೇಳಿ ತಿಳಿದುಕೊಂಡು “ಇವರೆಯೇ ಪ್ರಪಂಚವನ್ನು ಗೆದ್ದ ವಿವೇಕಾನಂದ ಸ್ವಾಮಿಗಳು!" ಎನ್ನುವುದಕ್ಕೆ ಮೊದಲು ಮಾಡಿದರು. ಸ್ವಾಮೀಜಿಯ ಈ ಅಸಾಮಾನ್ಯವಾದ ನಮ್ರತೆಯನ್ನು ನೋಡಿ ಎಲ್ಲರೂ ಅವರನ್ನು ಒಟ್ಟಿಗೆ ಹೊಗಳುವುದಕ್ಕೂ, “ಜಯ ರಾಮಕೃಷ್ಣ" ಎಂಬ ಧ್ವನಿಯಿಂದ ಹೋಗುತ್ತಿದ್ದ ದಾರಿಯನ್ನು ನಡುಗಿಸುವುದಕ್ಕೂ ಮೊದಲುಮಾಡಿದರು.

ಗೃಹಸ್ಥರಿಗೆ ಆದರ್ಶಸ್ವರೂಪವಾದ ನವಗೋಪಾಲ ಬಾಬುಗಳ ಹೃದಯ ಅಂದು ಆನಂದದಿಂದ ತುಂಬಿಹೋಗಿತ್ತು. ಅವರು ಪರಮಹಂಸರ ಮತ್ತು ಅವರಿಗೆ ಸಂಬಂಧಪಟ್ಟ ಎಲ್ಲಾ ಜನರ ಪೂಜೆಗೋಸ್ಕರ ಯಥೇಚ್ಛವಾಗಿ ಸಾಮಾನುಗಳನ್ನು ಸಿದ್ಧಪಡಿಸಿಕೊಂಡು ನಾಲ್ಕೂ ಕಡೆಗೆ ಸಡಗರದಿಂದ ಓಡಿಯಾಡುತ್ತ ಮಧ್ಯೆ ಮಧ್ಯೆ ಉಲ್ಲಾಸದಿಂದ ‘ಜಯರಾಮ್’ ‘ಜಯರಾಮ್’ ಎಂದು ಉದ್ಘೋಷಿಸುತ್ತ ಇದ್ದರು.

ಕ್ರಮವಾಗಿ ಗುಂಪು ನವಗೋಪಾಲ ಬಾಬುಗಳ ಮನೆಯ ಬಾಗಿಲಿಗೆ ಬಂದಕೂಡಲೇ ಮನೆಯ ಒಳಗೆ ಶಂಖಧ್ವನಿಯೂ ಆಯಿತು. ಸ್ವಾಮೀಜಿ ಮೃದಂಗವನ್ನು ಕೆಳಗಿಟ್ಟು ಬೈಠಕ್ ಖಾನೆಯಲ್ಲಿ ಸ್ವಲ್ಪ ಕಾಲ ವಿಶ್ರಮಿಸಿಕೊಂಡು ಅನಂತರ ಪೂಜಾಗೃಹವನ್ನು ನೋಡಲು ಮೇಲಕ್ಕೆ ಹೋದರು. ಪೂಜಾಗೃಹವು ಅಮೃತಶಿಲೆಯಿಂದ ರಚಿತವಾಗಿತ್ತು. ಮಧ್ಯದಲ್ಲಿ ಸಿಂಹಾಸನ; ಅದರ ಮೇಲೆ ಪರಮಹಂಸರ ಪೊರ್ಸಿಲೇನಿನ ಮೂರ್ತಿ. ಹಿಂದೂಗಳ ದೇವರ ಪೂಜೆಗೆ ಯಾವ ಯಾವ ಉಪಕರಣಗಳು ಬೇಕೋ ಇವೆಲ್ಲವೂ ಸಿದ್ಧವಾಗಿದ್ದುವು - ಯಾವ ಭಾಗದಲ್ಲಿಯೂ ಏನೂ ಕಡಮೆಯಾಗಿರಲಿಲ್ಲ. ಸ್ವಾಮೀಜಿ ಇದನ್ನು ನೋಡಿ ಬಹುವಾಗಿ ಸಂತುಷ್ಟರಾದರು.

ನವಗೋಪಾಲ ಬಾಬುಗಳ ಪತ್ನಿ ಇತರ ಹೆಂಗಸರೊಡನೆ ಸ್ವಾಮಿಗಳಿಗೆ ಸಾಷ್ಟಾಂಗ ಪ್ರಣಾಮಮಾಡಿ ಬೀಸಣಿಗೆಯನ್ನು ತೆಗೆದುಕೊಂಡು ಅವರಿಗೆ ಗಾಳಿ ಹಾಕುವುದಕ್ಕೆ ಮೊದಲುಮಾಡಿದರು. ಸ್ವಾಮೀಜಿಯವರ ಬಾಯಿಂದ ಎಲ್ಲಾ ವಿಚಾರಗಳ ಪ್ರಶಂಸೆಯನ್ನೂ ಕೇಳಿ ಮನೆಯ ಯಜಮಾನಿಯು ಅವರನ್ನು ಸಂಬೋಧಿಸಿ ಹೀಗೆಂದು ಹೇಳಿದರು - “ನಾವು ಪರಮಹಂಸರ ಸೇವೆ ಮಾಡುವ ಅಧಿಕಾರವನ್ನು ಪಡೆಯುವುದೆಂದರೇನು? ಸಾಮಾನ್ಯವಾದ ಮನೆ, ಸಾಮಾನ್ಯವಾದ ಸಂಪತ್ತು. ತಾವು ಈ ದಿವಸ ಸ್ವಂತ ನಿಂತುಕೊಂಡು ಕೃಪೆಮಾಡಿ ಪರಮಹಂಸರ ಪ್ರತಿಷ್ಠೆಯನ್ನು ಮಾಡಿಸಿ ನಮ್ಮನ್ನು ಧನ್ಯರನ್ನಾಗಿ ಮಾಡಬೇಕು."

ಸ್ವಾಮಿಗಳು ಅದಕ್ಕೆ ಉತ್ತರವಾಗಿ ಪರಿಹಾಸ್ಯ ಮಾಡುತ್ತ, “ನಿಮ್ಮ ಪರಮಹಂಸರು ಇಂತಹ ಅಮೃತಶಿಲೆಯ ಮಹಡಿಯ ಮನೆಯಲ್ಲಿ ಹದಿನಾಲ್ಕು ತಲೆಗಳಿಂದ ವಾಸಮಾಡಿರಲಿಲ್ಲ. ಹುಟ್ಟಿದ್ದು ಒಂದು ಹಳ್ಳಿಯ ಗುಡಿಸಲಲ್ಲಿ; ಹಾಗೂ ಹೀಗೂ ದಿನಗಳನ್ನು ಕಳೆದದ್ದಾಯಿತು. ಇಲ್ಲಿ, ಇಂಥ ಉತ್ತಮವಾದ ಸೇವೆಯಲ್ಲಿ ಇರದಿದ್ದರೆ ಮತ್ತೆಲ್ಲಿ ಇರುವರು?" ಎಂದರು. ಎಲ್ಲರೂ ಸ್ವಾಮಿಗಳ ಮಾತನ್ನು ಕೇಳಿ ನಗುವುದಕ್ಕೆ ಮೊದಲುಮಾಡಿದರು. ಈಗ ವಿಭೂತಿಭೂಷಿತಾಂಗರಾದ ಸ್ವಾಮೀಜಿ ಸಾಕ್ಷಾತ್ ಮಹಾದೇವನ ಹಾಗೆ ಪೂಜಕನ ಆಸನದಲ್ಲಿ ಕುಳಿತು ಪರಮಹಂಸರನ್ನು ಆವಾಹನೆ ಮಾಡುವುದಕ್ಕೆ ಹೊರಟರು.

ಪ್ರಕಾಶಾನಂದ ಸ್ವಾಮಿಗಳು ಸ್ವಾಮೀಜಿ ಹತ್ತಿರ ಕುಳಿತುಕೊಂಡು ಮಂತ್ರ ಮುಂತಾದವುಗಳನ್ನು ಹೇಳುತ್ತಿದ್ದರು. ಪೂಜೆಯ ನಾನಾ ಅಂಗಗಳು ಕ್ರಮವಾಗಿ ಮುಗಿದುವು, ಮಂಗಳಾರತಿಗಾಗಿ ಶಂಖ ಗಂಟೆಗಳ ಧ್ವನಿ ಮೊದಲಾಯಿತು. ಪ್ರಕಾಶಾನಂದ ಸ್ವಾಮಿಗಳೇ ಅದನ್ನು ಬಾರಿಸಿದರು.

ಮಂಗಳಾರತಿಯಾದ ಮೇಲೆ ಸ್ವಾಮಿಗಳು ಪೂಜಾಗೃಹದಲ್ಲಿ ಕುಳಿತುಕೊಂಡಿದ್ದ ಹಾಗೆಯೆ ಶ‍್ರೀರಾಮಕೃಷ್ಣ ದೇವರಿಗೆ ನಮಸ್ಕಾರಮಾಡುವ ಮಂತ್ರವನ್ನು ಬಾಯಲ್ಲಿಯೇ ರಚಿಸಿ ಹೀಗೆ ಹೇಳಿಬಿಟ್ಟರು:

\begin{verse}
“ಸ್ಥಾಪಕಾಯ ಚ ಧರ್ಮಸ್ಯ ಸರ್ವಧರ್ಮಸ್ವರೂಪಿಣೇ~।\\ಅವತಾರವರಿಷ್ಠಾಯ ರಾಮಕೃಷ್ಣಾಯ ತೇ ನಮಃ~॥”
\end{verse}

ಸಕಲರೂ ಈ ಮಂತ್ರವನ್ನು ಹೇಳಿ ಪರಮಹಂಸರಿಗೆ ನಮಸ್ಕಾರ ಮಾಡಲು ಶಿಷ್ಯನು ಪರಮಹಂಸರ ಮೇಲಣ ಒಂದು ಸ್ತೋತ್ರವನ್ನು ಓದಿದನು. ಹೀಗೆ ಪೂಜೆ ಮುಗಿಯಿತು. ಭಕ್ತಮಂಡಲಿಯು ಕೆಳಗೆ ಇಳಿದು ಬಂದು ಸ್ವಲ್ಪ ಫಲಾಹಾರ ಮಾಡಿ ಭಜನೆಗೆ ಆರಂಭಮಾಡಿತು. ಸ್ವಾಮೀಜಿ ಮೇಲೆಯೇ ಇದ್ದರು. ಮನೆಯಲ್ಲಿದ್ದ ಹೆಂಗಸರು ಸ್ವಾಮೀಜಿಗೆ ನಮಸ್ಕಾರಮಾಡಿ ನಾನಾ ಧಾರ್ಮಿಕ ವಿಚಾರಗಳನ್ನು ಕೇಳಿ ತಿಳಿದುಕೊಳ್ಳುತ್ತಲೂ ಆಶೀರ್ವಾದವನ್ನು ಪಡೆಯುತ್ತಲೂ ಇದ್ದರು. ಆ ಪರಿವಾರದಲ್ಲಿದ್ದವರೆಲ್ಲರೂ ರಾಮಕೃಷ್ಣ ಪರಮಹಂಸರಲ್ಲಿ ಪ್ರಾಣವನ್ನು ಇಟ್ಟುಕೊಂಡಿದ್ದುದನ್ನು ನೋಡಿ ಶಿಷ್ಯ ಬೆರಗಾಗಿ ನಿಂತು ಅವರ ಸಹವಾಸದಿಂದ ತನ್ನ ಮನುಷ್ಯ ಜನ್ಮ ಸಾರ್ಥಕವಾಯಿತೆಂದು ಭಾವಿಸಿಕೊಂಡನು.

ಅನಂತರ ಭಕ್ತಾದಿಗಳು ಪ್ರಸಾದವನ್ನು ಸ್ವೀಕರಿಸಿ ಕೈತೊಳೆದುಕೊಂಡು ಕೆಳಗೆ ಹೋಗಿ ಸ್ವಲ್ಪ ವಿಶ್ರಾಂತಿಯನ್ನು ತೆಗೆದುಕೊಂಡರು. ಕ್ರಮವಾಗಿ ಸಾಯಂಕಾಲವಾಗಲು ಅವರೆಲ್ಲರೂ ಸಣ್ಣ ಸಣ್ಣ ಗುಂಪಾಗಿ ತಮ್ಮ ತಮ್ಮ ಮನೆಗೆ ಹಿಂತಿರುಗಿ ಹೊರಟರು. ಶಿಷ್ಯನೂ ಸ್ವಾಮಿಗಳ ಸಂಗಡ ಗಾಡಿಯಲ್ಲಿ ಕುಳಿತುಕೊಂಡು ಹೋಗಿ ರಾಮಕೃಷ್ಣಪುರದ ಘಾಟಿನಲ್ಲಿ ದೋಣಿಯನ್ನು ಹತ್ತಿ ಆನಂದದಿಂದ ನಾನಾ ಮಾತುಗಳನ್ನಾಡುತ್ತ ಬಾಗ್‌ಬಜಾರಿನ ಕಡೆಗೆ ಹೋದನು.

\newpage

\chapter[ಅಧ್ಯಾಯ ೫]{ಅಧ್ಯಾಯ ೫\protect\footnote{\engfoot{C.W, Vol. VI, P. 4G5}}}

\begin{center}
ಸ್ಥಳ: ದಕ್ಷಿಣೇಶ್ವರದ ಕಾಳಿ ದೇವಸ್ಥಾನ, ಆಲಂಬಜಾರಿನ ಮಠ, ವರ್ಷ: ಕ್ರಿ.ಶ. ೧೮೯೭ನೆಯ ಮಾರ್ಚಿ.
\end{center}

ಸ್ವಾಮಿಜಿ ಇಂಗ್ಲೆಂಡಿನಿಂದ ಮೊದಲಬಾರಿ ಹಿಂತಿರುಗಿ ಬಂದಾಗ ಆಲಂಬಜಾರಿನ ರಾಮಕೃಷ್ಣ ಮಠ ಪ್ರತಿಷ್ಠಾಪಿತವಾಗಿತ್ತು. ಈ ಮಠದ ಮನೆಯನ್ನು ಜನರು “ಭೂತದ ಮನೆ" ಎಂದು ಕರೆಯುತ್ತಿದ್ದರು. ಆದರೆ ಸಂನ್ಯಾಸಿಗಳ ಸಂಸರ್ಗದಿಂದ ಈ ಭೂತದ ಮನೆ ರಾಮಕೃಷ್ಣ ಯಾತ್ರಾಸ್ಥಳವಾಗಿ ಪರಿಣಮಿಸಿತು. ಅಲ್ಲಿ ಎಷ್ಟು ಶಾಸ್ತ್ರಪ್ರಸಂಗಗಳು, ಎಷ್ಟು ನಾಮಕೀರ್ತನೆಗಳು ನಡೆದುವೋ ಅವಕ್ಕೆ ಎಲ್ಲೆಯೇ ಇಲ್ಲ. ಕಲ್ಕತ್ತೆಯಲ್ಲಿ ರಾಜಯೋಗ್ಯವಾದ ಸ್ವಾಗತವನ್ನು ಪಡೆದ ಸ್ವಾಮಿಗಳು ಈ ಮುರುಕು ಮಠದಲ್ಲಿಯೇ ಇರುತ್ತಿದ್ದರು. ಕಲ್ಕತ್ತೆಯ ಜನರು ಅವರಲ್ಲಿ ಶ್ರದ್ಧಾಭಕ್ತಿಯುಳ್ಳವರಾಗಿ ಒಂದು ತಿಂಗಳು ಇರಬೇಕೆಂದು ಅವರಿಗೋಸ್ಕರ ಕಲ್ಕತ್ತೆಯ ಉತ್ತರಕ್ಕೆ ಕಾಶೀಪುರದಲ್ಲಿ ಗೋಪಾಲಲಾಲ ಶೀಲರ ತೋಟದ ಮನೆಯಲ್ಲಿ ಸ್ಥಾನವನ್ನು ಗೊತ್ತು ಮಾಡಿಕೊಟ್ಟಿದ್ದರಷ್ಟೆ. ಅಲ್ಲಿಗೂ ಮಧ್ಯೆ ಮಧ್ಯೆ ಹೋಗುತ್ತ ಇರುತ್ತ ದರ್ಶನೋತ್ಸುಕರಾದ ಜನಸಂಘದೊಡನೆ ಧರ್ಮಾಲಾಪಾದಿಗಳನ್ನು ಮಾಡಿ ಅವರ ಹೃದಯಾಕಾಂಕ್ಷೆಯನ್ನು ಪೂರ್ಣಮಾಡುತ್ತಿದ್ದರು.

ಶ‍್ರೀರಾಮಕೃಷ್ಣ ಪರಮಹಂಸರ ಜನ್ಮೋತ್ಸವ ಹತ್ತಿರಕ್ಕೆ ಬಂತು. ದಕ್ಷಿಣೇಶ್ವರದಲ್ಲಿರುವ ರಾಣಿರಾಸಮಣಿಯ ಕಾಳಿ ದೇವಸ್ಥಾನದಲ್ಲಿ ಈ ಸಾರಿ ಉತ್ಸವಕ್ಕಾಗಿ ಅಗಾಧವಾದ ಏರ್ಪಾಡು ನಡೆಯುತ್ತಿದೆ. ಶ‍್ರೀ ರಾಮಕೃಷ್ಣರ ಸೇವಕರ ಮಾತು ಹಾಗಿರಲಿ, ಧರ್ಮಪಿಪಾಸುಗಳಾದ ಜನಸಾಮಾನ್ಯರಿಗೂ ಆನಂದ ಉತ್ಸಾಹಗಳು ಮೇರೆಯಿಲ್ಲದಂತಿದ್ದವು. ಏಕೆಂದರೆ ವಿಶ್ವವಿಜಯಿಗಳಾದ ಸ್ವಾಮಿಜಿ ಶ‍್ರೀರಾಮಕೃಷ್ಣರ ಭವಿಷ್ಯ ವಾಣಿಯನ್ನು ಸಫಲಮಾಡಿ ಈ ವರ್ಷ ಹಿಂತಿರುಗಿ ಬಂದಿದ್ದಾರೆ; ಅವರ ಗುರು ಭ್ರಾತೃಗಳು ಅಂದು ಅವರನ್ನು ಪಡೆದು ಶ‍್ರೀರಾಮಕೃಷ್ಣರ ಸಂಗಸುಖವನ್ನು ಅನುಭವಿಸುತ್ತಿರುವಂತೆ ತೋರುತ್ತಿದ್ದರು. ಕಾಳೀಮಂದಿರದ ದಕ್ಷಿಣ ಭಾಗದ ದೊಡ್ಡ ಪಾಕಶಾಲೆಯಲ್ಲಿ ಅಡಿಗೆಯಾಗುತ್ತಿದೆ. ಸ್ವಾಮಿಜಿಯವರು ಕೆಲವು ಗುರು ಭ್ರಾತೃಗಳೊಡನೆ ಸುಮಾರು ಒಂಬತ್ತು ಗಂಟೆಯ ಹೊತ್ತಿಗೆ ಬಂದರು. ಬರಿಯ ಕಾಲು; ತಲೆಗೆ ಕಾವಿಯ ಬಣ್ಣದ ಪಾಗು. ಜನಸಂಘ ಅವರನ್ನು ನೋಡುತ್ತಾ ಅಲ್ಲಿಂದಿಲ್ಲಿಗೆ ಓಡಿಯಾಡುತ್ತಿದೆ. ಅವರ ಆ ಅನಿಂದಿತವಾದ ರೂಪವನ್ನು ದರ್ಶನ ಮಾಡೋಣ, ಆ ಪಾದಪದ್ಮವನ್ನು ಸ್ಪರ್ಶಮಾಡೋಣ, ಮತ್ತು ಅವರ ಬಾಯಿಂದ ಜ್ವಲಿಸುತ್ತಿರುವ ಅಗ್ನಿಯ ಜ್ವಾಲೆಯಂತಿರುವ ಮಾತನ್ನು ಕೇಳಿ ಧನ್ಯರಾಗೋಣ, ಎಂದು ಜನ ಕುತೂಹಲಿಗಳಾಗಿರುವರು. ಅದರಿಂದಲೆ ಸ್ವಾಮೀಜಿಗೆ ಈ ದಿವಸ ಒಂದು ನಿಮಿಷ ಅವಕಾಶವೂ ಇಲ್ಲ. ಕಾಳೀಮಾತೆಯ ಮಂದಿರದ ಮುಂದೆ ಅಸಂಖ್ಯವಾದ ಜನ. ಸ್ವಾಮೀಜಿ ಜಗನ್ಮಾತೆಗೆ ನೆಲದ ಮೇಲೆ ಉದ್ದಕ್ಕೆ ಪ್ರಣಾಮ ಮಾಡಿದರು - ಜೊತೆಯಲ್ಲಿಯೇ ಸಾವಿರಾರು ತಲೆಗಳು ಬಗ್ಗಿದುವು. ಆಮೇಲೆ ರಾಧಾಕಾಂತ ಸ್ವಾಮಿಗೆ ಪ್ರಣಾಮ ಮಾಡಿ ಅವರು ಪರಮಹಂಸರ ವಾಸಗೃಹಕ್ಕೆ ಬಂದರು. ಆ ತೊಟ್ಟಿಯಲ್ಲಿ ಈಗ ಒಂದು ಎಳ್ಳು ಹಾಕಿದರೆ ಹಿಡಿಸುವಷ್ಟು ಜಾಗವೂ ಇಲ್ಲ. ‘ಜಯರಾಮಕೃಷ್ಣ!’ ಎಂಬ ಧ್ವನಿಯಿಂದ ಕಾಳಿ ದೇವಸ್ಥಾನದ ದಿಕ್ಕುಗಳೆಲ್ಲವೂ ಪ್ರತಿಧ್ವನಿತವಾಗುತ್ತಿವೆ. ಸಾವಿರಾರು ನೋಟಕರನ್ನು ಕೂರಿಸಿಕೊಂಡು ಹೋರ್‌ ಮಿಲರ್ ಕಂಪೆನಿಯ ಜಹಜು ಬಾರಿಬಾರಿಗೂ ಕಲ್ಕತ್ತೆಯಿಂದ ಓಡಿಯಾಡುತ್ತಿದೆ. ನಗಾರಿ ನೌಪತ್ತಿನ ವಾದ್ಯಧ್ವನಿಯಿಂದ ಗಂಗಾನದಿ ನರ್ತನ ಮಾಡುತ್ತಿದೆ. ಉತ್ಸಾಹ, ಆಕಾಂಕ್ಷೆ, ಧರ್ಮಪಿಪಾಸೆ ಮತ್ತು ಅನುರಾಗ ಇವು ಮೂರ್ತಿಮತ್ತಾಗಿ ಶ‍್ರೀರಾಮಕೃಷ್ಣರ ಭಕ್ತರ ರೂಪದಲ್ಲಿ ಅಲ್ಲಲ್ಲಿ ಬೆಳಗುತ್ತಿವೆ. ಈ ಸಲದ ಈ ಉತ್ಸವ ಹೃದಯದಲ್ಲಿ ಅನುಭವಮಾಡಿಕೊಳ್ಳಬೇಕಾದ ವಿಷಯ - ಮಾತಿನಲ್ಲಿ ವ್ಯಕ್ತಪಡಿಸಲು ಅಸಾಧ್ಯ.

ಸ್ವಾಮಿಜಿಯೊಡನೆ ಬಂದಿದ್ದ ಇಬ್ಬರು ಇಂಗ್ಲಿಷ್ ಸ್ತ್ರೀಯರೂ ಉತ್ಸವಕ್ಕೆ ಬಂದಿದ್ದರು. ಶಿಷ್ಯನಿಗೆ ಅವರೊಡನೆ ಇನ್ನೂ ಪರಿಚಯವಾಗಿರಲಿಲ್ಲ. ಸ್ವಾಮೀಜಿ ಅವರನ್ನು ಜೊತೆಯಲ್ಲಿ ಕರೆದುಕೊಂಡು ಹೋಗಿ ಪವಿತ್ರವಾದ ಪಂಚವಟಿ ಮತ್ತು ಬಿಲ್ವಮೂಲಗಳನ್ನು ದರ್ಶನ ಮಾಡಿಸುತ್ತಿದ್ದರು. ಸ್ವಾಮಿಜಿಯೊಡನೆ ಈಗಲೂ ಅಷ್ಟೊಂದು ವಿಶೇಷ ಪರಿಚಯವಿಲ್ಲದಿದ್ದರೂ ಶಿಷ್ಯನು ಅವರ ಹಿಂದೆ ಹಿಂದೆ ಹೋಗಿ ಈ ಉತ್ಸವದ ಸಂಬಂಧವಾಗಿ ತಾನು ರಚಿಸಿದ ಒಂದು ಸಂಸ್ಕೃತ ಸ್ತೋತ್ರವನ್ನು ಸ್ವಾಮೀಜಿ ಕೈಗೆ ಕೊಟ್ಟನು. ಸ್ವಾಮೀಜಿ ಅದನ್ನು ಓದುತ್ತ ಓದುತ್ತ ಪಂಚವಟಿಯ ಕಡೆಗೆ ಹೊರಟರು. ಹೋಗುತ್ತ ಹೋಗುತ್ತ ಶಿಷ್ಯನ ಕಡೆಗೆ ಒಂದು ಸಾರಿ ತಿರುಗಿ ನೋಡಿ “ಚೆನ್ನಾಗಿದೆ, ಇನ್ನೂ ಬರಿ" ಎಂದು ಹೇಳಿದರು.

ಪಂಚವಟಿಯ ಒಂದು ಕಡೆಯಲ್ಲಿ ಪರಮಹಂಸರ ಗೃಹಸ್ಥ ಭಕ್ತರು ಇಳಿದುಕೊಂಡಿದ್ದರು. ಗಿರೀಶಬಾಬುಗಳು ಪಂಚವಟಿಯ ಉತ್ತರದಲ್ಲಿ ಗಂಗೆಯ ಕಡೆಗೆ ತಿರುಗಿಕೊಂಡು ಕುಳಿತಿದ್ದರು ಮತ್ತು ಅವರನ್ನು ಸುತ್ತಿಕೊಂಡು ಮಿಕ್ಕ ಭಕ್ತಜನರು ಶ‍್ರೀರಾಮಕೃಷ್ಣರ ಗುಣಗಾನದಲ್ಲಿಯೂ ಕಥಾ ಪ್ರಸಂಗದಲ್ಲಿಯೂ ಮೈಮರೆತು ಕುಳಿತುಕೊಂಡಿದ್ದರು. ಈ ಸಮಯದಲ್ಲಿ ಸ್ವಾಮೀಜಿ ಬಹುಜನರೊಡನೆ ಗಿರೀಶಬಾಬುಗಳ ಹತ್ತಿರಕ್ಕೆ ಬಂದು “ಏನ್ರೀ - ಘೋಷರೆ" ಎನ್ನುತ್ತ ಗಿರೀಶಬಾಬುಗಳಿಗೆ ನಮಸ್ಕಾರ ಮಾಡಿದರು. ಗಿರೀಶಬಾಬುಗಳೂ ಅವರಿಗೆ ಕೈ ಜೋಡಿಸಿ ಪ್ರತಿನಮಸ್ಕಾರ ಮಾಡಿದರು. ಸ್ವಾಮೀಜಿ ಗಿರೀಶಬಾಬುಗಳಿಗೆ ಹಿಂದಿನ ವೃತ್ತಾಂತವನ್ನು ಜ್ಞಾಪಿಸಿ “ಘೋಷರೆ, ಅದೇ ಒಂದು ಕಾಲ" ಎಂದರು. ಗಿರೀಶಬಾಬುಗಳೂ ಸ್ವಾಮೀಜಿ ಮಾತಿಗೆ ಸಮ್ಮತಿಸಿ “ಅದೇನೋ ನಿಜ; ಆದರೆ ಈಗಲೂ ಅದನ್ನೆ ಇನ್ನೂ ಹೆಚ್ಚಾಗಿ ನೋಡಬೇಕೆಂದು ಮನಸ್ಸು ಹಂಬಲಿಸುತ್ತದೆ" ಎಂದರು.

ಹೀಗೆ ಅವರಿಬ್ಬರಿಗೂ ಏನು ಮಾತು ನಡೆಯಿತೊ ಅದರ ಮರ್ಮವನ್ನು ಹೊರಗಿನ ಜನರನೇಕರು ಅರ್ಥಮಾಡಿಕೊಳ್ಳಲಾರದೆ ಹೋದರು. ಸ್ವಲ್ಪ ಹೊತ್ತು ಮಾತುಕತೆ ನಡೆದ ಮೇಲೆ ಸ್ವಾಮೀಜಿ ಪಂಚವಟಿಯ ಈಶಾನ್ಯದಲ್ಲಿರುವ ಬಿಲ್ವ ವೃಕ್ಷದ ಕಡೆಗೆ ಹೊರಟರು. ಸ್ವಾಮಿಜಿ ಹೊರಟುಹೋದ ಮೇಲೆ ಗಿರೀಶ ಬಾಬುಗಳು ಬಂದಿದ್ದ ಭಕ್ತಮಂಡಲಿಯನ್ನು ಸಂಬೋಧಿಸಿ ಹೀಗೆಂದು ಹೇಳಿದರು: "ಒಂದು ದಿನ ಹರಮೋಹನ ಮಿತ್ರರು, ವರ್ತಮಾನ ಪತ್ರಿಕೆಯನ್ನು ನೋಡಿಕೊಂಡೊ ಏನೊ ಬಂದು, ಅಮೆರಿಕಾದಲ್ಲಿ ಸ್ವಾಮಿಗಳ ವಿಚಾರವಾಗಿ ಅಪಯಶಸ್ಸು ಹರಡಿಕೊಂಡಿತ್ತೆಂದು ಹೇಳಿದರು. ಆಗ ನಾನು ಅವರಿಗೆ - 'ನರೇನನು\footnote{ವಿವೇಕಾನಂದ ಸ್ವಾಮಿಗಳು.} ಏನಾದರೂ ಕೆಟ್ಟದ್ದು ಮಾಡಿದ್ದನ್ನು ನನ್ನ ಕಣ್ಣಿನಿಂದಲೆ ನೋಡಿದರೂ ಆಗ ಅದು ನನ್ನ ಕಣ್ಣಿನ ದೋಷವೆಂದು ಹೇಳಿಬಿಡುತ್ತೇನೆ - ಕಣ್ಣಿನ ಮೇಲೆ ಹಾಕಿಬಿಡುತ್ತೇನೆ. ಅವರು ಹೊತ್ತು ಹುಟ್ಟುವುದಕ್ಕಿಂತ ಮುಂಚೆ ತೆಗೆದ ಬೆಣ್ಣೆ; ಅವರು ನೀರಿನಲ್ಲಿ ಕರಗಿ ಹೋಗುತ್ತಾರೆಯೇನು? ಅವರಲ್ಲಿ ಯಾರಾದರೂ ಯಾವುದಾದರೂ ದೋಷವನ್ನು ಎತ್ತಿಯಾಡಿದರೆ ಅಂಥವರಿಗೆ ನರಕ ಪ್ರಾಪ್ತಿಯಾಗುತ್ತದೆ' ಎಂದು ಹೇಳಿದೆನು." ಹೀಗೆ ಮಾತುಕತೆಗಳು ನಡೆಯುತ್ತಿದ್ದವು. ಆಗ ನಿರಂಜನಾನಂದ ಸ್ವಾಮಿಗಳು ಗಿರೀಶಘೋಷ ಮಹಾಶಯರಿದ್ದೆಡೆಗೆ ಬಂದರು. ಒಂದು ಹುಕ್ಕವನ್ನು ತೆಗೆದುಕೊಂಡು ತಂಬಾಕನ್ನು ಸೇವಿಸುತ್ತಾ ಕೊಲೊಂಬೋವಿನಿಂದ ಕಲ್ಕತ್ತೆಗೆ ಹಿಂತಿರುಗಿ ಬರುವವರೆಗೆ ಹಿಂದೂಸ್ಥಾನದ ಬೇರೆ ಬೇರೆ ಸ್ಥಳಗಳಲ್ಲಿ ಸಾಧಾರಣ ಜನರು ಸ್ವಾಮಿಗಳಿಗೆ ಎಷ್ಟು ಅಪೂರ್ವವಾದ ಆದರಾತಿಥ್ಯಗಳನ್ನು ಮಾಡಿದರು, ಮತ್ತು ಸ್ವಾಮಾಜಿ ಅವರೆಲ್ಲರಿಗೂ ಹೇಗೆ ಉಪನ್ಯಾಸಗಳ ಮೂಲಕ ಅಮೂಲ್ಯವಾದ ಉಪದೇಶವನ್ನು ಕೊಟ್ಟರು ಎಂಬುದನ್ನು ತಿಳಿಸಿ ಅವುಗಳಲ್ಲಿ ಕೆಲವನ್ನು ವರ್ಣಿಸುತ್ತಿದ್ದರು. ಗಿರೀಶಬಾಬುಗಳು ಅದನ್ನು ಕೇಳುತ್ತ ಕೇಳುತ್ತ ಸ್ತಂಭೀಭೂತರಾಗಿ ಕುಳಿತುಕೊಂಡುಬಿಟ್ಟರು.

ಆ ದಿನ ದಕ್ಷಿಣೇಶ್ವರದ ದೇವಾಲಯದಲ್ಲಿ ಎಲ್ಲೆಲ್ಲಿಯೂ ಹೀಗೆ ಒಂದು ದಿವ್ಯಭಾವದ ಪ್ರವಾಹ ಹರಿಯುತ್ತಿತ್ತು. ಈಗ ಜನಸ್ತೋಮ ಸ್ವಾಮೀಜಿಯ ಉಪನ್ಯಾಸವನ್ನು ಕೇಳುವುದಕ್ಕೆ ಉತ್ಸುಕವಾಗಿ ನಿಂತುಕೊಂಡಿದೆ – ಆದರೆ ಬಹುಪ್ರಯತ್ನ ಮಾಡಿದರೂ ಸ್ವಾಮೀಜಿ ಜನರ ಗುಜುಗುಜು ಶಬ್ದವನ್ನು ಮೀರುವಷ್ಟು ಗಟ್ಟಿಯಾಗಿ ಮಾತನಾಡಲಾರದೆ ಹೋದರು. ವಿಧಿಯಿಲ್ಲದೆ, ಉಪನ್ಯಾಸದ ಪ್ರಯತ್ನವನ್ನು ನಿಲ್ಲಿಸಿ ಅವರು ತಮ್ಮೊಡನೆ ಇಬ್ಬರು ಇಂಗ್ಲಿಷ್ ಸ್ತ್ರೀಯರನ್ನು ಜೊತೆಯಲ್ಲಿ ಕರೆದುಕೊಂಡು ಪರಮಹಂಸರ ಸಾಧನೆಯ ಸ್ಥಾನವನ್ನು ತೋರಿಸುತ್ತಲೂ, ಪರಮಹಂಸರ ಮುಖ್ಯ ಭಕ್ತರ ಮತ್ತು ಅಂತರಂಗದ ಜನರೊಡನೆ ಮಾತನಾಡುತ್ತಲೂ ಇದ್ದು ಬಿಟ್ಟರು. ಇಂಗ್ಲಿಷ್ ಸ್ತ್ರೀಯರು ಧರ್ಮಶಿಕ್ಷಣಕ್ಕಾಗಿ ಅವರೊಡನೆ ದೂರದೇಶದಿಂದ ಬಂದಿದ್ದನ್ನು ನೋಡಿ ಪ್ರೇಕ್ಷಕರಲ್ಲಿ ಕೆಲವರು ಆಶ್ಚರ್ಯಯುಕ್ತರಾಗಿ ಸ್ವಾಮೀಜಿಯ ಅದ್ಭುತ ಸಾಮರ್ಥ್ಯದ ವಿಚಾರವಾಗಿ ಮಾತನಾಡಿಕೊಳ್ಳುವುದಕ್ಕೆ ತೊಡಗಿದರು.

ಮೂರು ಗಂಟೆಯಾದ ಮೇಲೆ ಸ್ವಾಮೀಜಿ ಶಿಷ್ಯನನ್ನು ಕುರಿತು “ಒಂದು ಬಂಡಿಯನ್ನು ನೋಡು - ಮಠಕ್ಕೆ ಹೋಗಬೇಕು" ಎಂದರು. ಅನಂತರದಲ್ಲಿ ಆಲಂಬಜಾರಿನವರೆಗೂ ಹೋಗುವುದಕ್ಕೆ ಎರಡಾಣೆ ಬಾಡಿಗೆಯನ್ನು ನಿಷ್ಕರ್ಷೆ ಮಾಡಿ, ಶಿಷ್ಯನು ಒಂದು ಗಾಡಿಯನ್ನು ತೆಗೆದುಕೊಂಡು ಬರಲು, ಸ್ವಾಮಿಗಳು ಗಾಡಿಯಲ್ಲಿ ತಾವು ಒಂದು ಕಡೆ ಕುಳಿತುಕೊಂಡು ನಿರಂಜನಾನಂದ ಸ್ವಾಮಿಗಳನ್ನೂ ಶಿಷ್ಯನನ್ನೂ ಮತ್ತೊಂದು ಕಡೆ ಕೂರಿಸಿಕೊಂಡು ಆಲಂಬಜಾರಿನ ಮಠದ ಕಡೆಗೆ ಸಂತೋಷದಿಂದ ಹೊರಟರು. ಹೊಗುತ್ತ ಹೋಗುತ್ತ ಶಿಷ್ಯನನ್ನು ಕುರಿತು ಹೀಗೆಂದರು: “ಜೀವನದಲ್ಲಿಯೂ ಕಾರ್ಯದಲ್ಲಿಯೂ ಪರಿಣತವಾಗದ ಭಾವವನ್ನು ಹಿಡಿದುಕೊಂಡು ಕುಳಿತರೆ ಏನಾಗುತ್ತದೆ? ಈ ಉತ್ಸವ ಮೊದಲಾದವುಗಳೂ ಬೇಕು, ಅವು ಇದ್ದರೆ ತಾನೆ ಕ್ರಮವಾಗಿ ಈ ಭಾವಗಳೆಲ್ಲ ಜನಸಾಮಾನ್ಯರ ಮನಸ್ಸಿಗೆ ಹತ್ತುತ್ತವೆ. ಹಿಂದೂಗಳಲ್ಲಿ ಹನ್ನೆರಡು ತಿಂಗಳಿಗೆ ಹದಿಮೂರು ಹಬ್ಬಗಳುಂಟಷ್ಟೆ - ಇವುಗಳ ಆಚರಣೆಯೆ ಧರ್ಮದ ದೊಡ್ಡ ದೊಡ್ಡ ಭಾವಗಳು ಕ್ರಮೇಣ ಜನರಲ್ಲಿ ಪ್ರವೇಶಿಸುವಂತೆ ಮಾಡುವುದು. ಅವುಗಳಲ್ಲಿ ಒಂದು ದೋಷವೂ ಇದೆ. ಸಾಧಾರಣ ಜನರು ಇವುಗಳ ಒಳ ತಿರುಳನ್ನು ತಿಳಿದುಕೊಳ್ಳದೆ ಇವುಗಳಲ್ಲಿಯೇ ಮೈಮರೆತು ಹೋಗುತ್ತಾರೆ ಮತ್ತು ಈ ಉತ್ಸವಾಮೋದಗಳು ನಿಂತುಹೋದರೆ ಆಮೇಲೆ ಎಂದಿನಂತೆಯೇ ಇದ್ದು ಬಿಡುತ್ತಾರೆ. ಆದ್ದರಿಂದ ಅವು ಧರ್ಮದ ಬಹಿರಾವರಣವೆಂಬುದೂ, ಈಗ ಧರ್ಮ ಮತ್ತು ಆತ್ಮಜ್ಞಾನಗಳನ್ನು ಮುಚ್ಚಿಕೊಂಡಿರುತ್ತವೆಂಬುದೂ ನಿಜ."

ಆದರೆ ಯಾರು ಧರ್ಮವೆಂದರೇನು, ಆತ್ಮವೆಂದರೇನು ಎಂಬುದನ್ನೆಲ್ಲಾ ಸ್ವಲ್ಪವೂ ತಿಳಿದುಕೊಳ್ಳಲಾರರೊ - ಅವರು ಈ ಉತ್ಸವಾಮೋದಗಳಿಂದ ಕ್ರಮೇಣ ಧರ್ಮವನ್ನು ತಿಳಿದುಕೊಳ್ಳುವುದಕ್ಕೆ ಪ್ರಯತ್ನ ಮಾಡುತ್ತಾರೆ. ನೋಡು ಇಂದು ಪರಮಹಂಸರ ಜನ್ಮೋತ್ಸವ ನಡೆಯಿತಷ್ಟೆ, ಇದಕ್ಕೆ ಯಾರು ಯಾರು ಬಂದಿದ್ದಾರೆ ಅವರು ಪರಮಹಂಸರ ವಿಷಯವಾಗಿ ಒಂದು ಸಾರಿಯಾದರೂ ಯೋಚಿಸುವರು. ಯಾರ ಹೆಸರಿನಲ್ಲಿ ಇಷ್ಟು ಜನರು ಒಂದುಕಡೆ ಸೇರಿದ್ದರು, ಆತನು ಯಾರು, ಆತನ ಹೆಸರಿನಲ್ಲಿ ಇಷ್ಟು ಜನ ಏಕೆ ಬಂದರು - ಎಂಬ ಅಂಶ ಅವರ ಮನಸ್ಸಿಗೆ ಹೊಳೆಯುತ್ತದೆ. ಯಾರಿಗೆ ಅದೂ ಇಲ್ಲವೊ, ಅಂಥವರೂ ಕೂಡ ಭಜನೆಯನ್ನು ಕೇಳುವುದಕ್ಕೆ ಮತ್ತು ಪ್ರಸಾದವನ್ನು ತೆಗೆದುಕೊಳ್ಳುವುದಕ್ಕೆ ಕೊನೆಯ ಪಕ್ಷ ವರ್ಷಕ್ಕೊಂದು ಸಾರಿಯಾದರೂ ಬರುತ್ತಾರೆ, ಬಂದು ಪರಮಹಂಸರ ಭಕ್ತರ ದರ್ಶನ ಮಾಡುತ್ತಾರೆ; ಅದರಿಂದ ಅವರಿಗೆ ಉಪಕಾರವೇ ಹೊರತು ಅಪಕಾರವಾಗುವುದಿಲ್ಲ.

ಶಿಷ್ಯ: ಆದರೆ, ಮಹಾಶಯರೆ, ಈ ಉತ್ಸವ ಕೀರ್ತನೆಗಳೇ ಸಾರವೆಂದು ಯಾರಾದರು ತಿಳಿದುಕೊಂಡುಬಿಟ್ಟರೆ, ಅಂಥವರು ಮುಂದಕ್ಕೆ ಹೋಗಬಲ್ಲರೆ? ನಮ್ಮ ದೇಶದಲ್ಲಿ ಷಷ್ಠಿಯಪೂಜೆ ಮಂಗಳಗೌರಿಯ ಪೂಜೆ ಮುಂತಾದುವು ಹೇಗೆ ನಿತ್ಯನೈಮಿತ್ತಿಕಗಳಾಗಿಬಿಟ್ಟಿವೆಯೊ ಹಾಗೆಯೆ ಇದೂ ಒಂದು ಆಗುತ್ತದೆ. ಸಾಯುವತನಕ ಜನರು ಇವುಗಳನ್ನೆಲ್ಲಾ ಮಾಡುತ್ತಿರುತ್ತಾರೆ. ಆದರೆ ಎಲ್ಲಿಯೂ ಇಂಥ ಪೂಜೆಗಳನ್ನು ಮಾಡುತ್ತ ಮಾಡುತ್ತ ಅವರು ಬ್ರಹ್ಮಜ್ಞರಾಗಿರುವುದನ್ನು ನಾನು ನೋಡಿಲ್ಲ.

ಸ್ವಾಮೀಜಿ: ಏಕೆ? ಹಿಂದೂಸ್ತಾನದಲ್ಲಿ ಎಷ್ಟೋ ಜನ ಧರ್ಮವೀರರು ಹುಟ್ಟಿದ್ದಾರೆ - ಅವರೆಲ್ಲರೂ ಇದನ್ನು ಅನುಸರಿಸಿಯೇ ಮುಂದುವರಿದು ಹೋಗಿ ದೊಡ್ಡವರಾಗಿದ್ದರಲ್ಲಾ? ಇವನ್ನು ಅನುಸರಿಸಿ ಸಾಧನೆ ಮಾಡುತ್ತಲೇ ಆತ್ಮ ಸಾಕ್ಷಾತ್ಕಾರವನ್ನು ಪಡೆದರು. ಅವರು ಅವುಗಳಲ್ಲೇ ಉಳಿದುಬಿಡಲಿಲ್ಲ. ಆದರೂ ಅವತಾರ ಸಮಾನರಾದ ಮಹಾಪುರುಷರು ಲೋಕಸಂಗ್ರಹಾರ್ಥವಾಗಿ ಇವನ್ನು ನಡೆಸಿಕೊಂಡು ಹೋಗುತ್ತಾರೆ.

ಶಿಷ್ಯ: ಲೋಕಸಂಗ್ರಹಾರ್ಥವಾಗಿ ನಡೆಸಿಕೊಂಡು ಹೋಗುತ್ತಾರೆ. ಆದರೆ ಆತ್ಮಜ್ಞಾನಿಗೆ ಯಾವಾಗ ಈ ಸಂಸಾರ ಇಂದ್ರಜಾಲದಂತೆ ಮಿಥ್ಯೆಯೆಂದು ತಿಳಿವಳಿಕೆಯುಂಟಾಗುತ್ತದೋ ಆಗ ಅವನಿಗೆ ಪುನಃ ಈ ಬಾಹ್ಯ ಲೋಕ ವ್ಯವಹಾರಗಳೆಲ್ಲವೂ ಸತ್ಯವೆಂದು ತೋರುವುದು ಸಾಧ್ಯವೇ?

ಸ್ವಾಮೀಜಿ: ಏಕೆ ಸಾಧ್ಯವಿಲ್ಲ? ನಾವು ಯಾವುದನ್ನು ಸತ್ಯವೆಂದು ತಿಳಿದುಕೊಂಡಿದ್ದೇವೆಯೊ ಅದು ದೇಶ ಕಾಲ ಪಾತ್ರಭೇದಗಳಿಂದ ಬೇರೆ ಬೇರೆಯಾಗಿದೆ ತಾನೆ? ಆದ್ದರಿಂದಲೇ, ಅಧಿಕಾರಭೇದದ ಮೇಲೆ ಸಮಸ್ತ ವಿಧವಾದ ವ್ಯವಹಾರಗಳಿಗೆ ಆವಶ್ಯಕತೆಯುಂಟಾಗುವುದು. ಪರಮಹಂಸರು ‘ತಾಯಿಯಾದವಳು ಒಬ್ಬ ಹುಡುಗನಿಗೆ ಪಲಾವನ್ನು ಬೇಯಿಸಿ ಕೊಡುತ್ತಾಳೆ; ಮತ್ತೊಬ್ಬ ಹುಡುಗನಿಗೆ ಸೀಮೆ ಅಕ್ಕಿಯ ಗಂಜಿಯನ್ನು ಮಾಡಿಕೊಡುತ್ತಾಳೆ’ ಎಂದು ಹೇಳುತ್ತಿದ್ದರಲ್ಲ, ಹಾಗೆ.

ಶಿಷ್ಯನು ವಿಷಯವನ್ನು ಇಷ್ಟು ಹೊತ್ತಿಗೆ ತಿಳಿದುಕೊಂಡು ಸುಮ್ಮನಾದನು. ನೋಡುತ್ತ ನೋಡುತ್ತ ಗಾಡಿಯು ಆಲಂಬಜಾರಿನ ಮಠಕ್ಕೆ ಬಂತು. ಶಿಷ್ಯನು ಗಾಡಿಯ ಬಾಡಿಗೆಯನ್ನು ಕೊಟ್ಟು ಸ್ವಾಮೀಜಿ ಜೊತೆಯಲ್ಲಿ ಮಠದೊಳಕ್ಕೆ ಹೋಗಿ ಸ್ವಾಮೀಜಿಗೆ ಬಾಯಾರಿಕೆಯಾಗಲು ನೀರನ್ನು ತಂದುಕೊಟ್ಟನು. ಸ್ವಾಮೀಜಿ ನೀರನ್ನು ಕುಡಿದು ಕೋಟನ್ನು ಬಿಚ್ಚಿಹಾಕಿದರು ಮತ್ತು ನೆಲದ ಮೇಲೆ ಹಾಸಿದ್ದ ಜಮಖಾನದ ಮೇಲೆ ಅರ್ಧ ಉರುಳಿಕೊಂಡ ಅವಸ್ಥೆಯಲ್ಲಿ ಕುಳಿತುಕೊಂಡರು. ನಿರಂಜನಾನಂದ ಸ್ವಾಮಿಗಳು ಪಕ್ಕದಲ್ಲಿ ಕುಳಿತುಕೊಂಡು - “ಉತ್ಸವಕ್ಕೆ ಇಷ್ಟು ಗುಂಪು ಯಾವಾಗಲೂ ಸೇರಿರಲಿಲ್ಲ. ಕಲ್ಕತ್ತೆಯೆ ಬಂದುಬಿಟ್ಟಿದ್ದಂತಿತ್ತು" ಎಂದರು.

ಸ್ವಾಮೀಜಿ: ಆಗುವುದು ಅಷ್ಟೇ ಅಲ್ಲ! ಇನ್ನೂ ಮುಂದೆ ಆಗುವುದದೆಷ್ಟೋ?!

ಶಿಷ್ಯ: ಮಹಾಶಯರೆ, ಪ್ರತಿಯೊಂದು ಧರ್ಮಸಂಪ್ರದಾಯದಲ್ಲಿಯೂ ಒಂದಲ್ಲದಿದ್ದರೆ ಮತ್ತೊಂದು ಬಾಹ್ಯ ಉತ್ಸವ ಅಥವಾ ಆಮೋದವು ಇದ್ದೇ ಇರುತ್ತದೆ. ಆದರೆ ಯಾರ ಜೊತೆಗೂ ಯಾರೂ ಸೇರುವುದಿಲ್ಲ. ಮಹಮ್ಮದನ ಮತ ಇಂಥ ಉದಾರವಾದ ಮತ! ಅದರಲ್ಲಿಯೂ ಷಿಯಾ ಸುನ್ನಿಗಳಿಗೆ ಢಾಕಾಪಟ್ಟಣದಲ್ಲಿ ಮಾರಾಮಾರಿಯಾದ್ದನ್ನು ನೋಡಿದ್ದೇನೆ.

ಸ್ವಾಮೀಜಿ: ಸಂಪ್ರದಾಯ (ಪಂಗಡ) ಹುಟ್ಟಿದರೆ ಸ್ವಲ್ಪ ಹೆಚ್ಚು ಕಡಿಮೆ ಹೀಗೆಲ್ಲ ಆಗುತ್ತವೆ. ಆದರೆ ಇಲ್ಲಿಯ ಭಾವವೇನು ಗೊತ್ತೆ? - ಸಂಪ್ರದಾಯವಿಹೀನತೆ!! ನಮ್ಮ ಪರಮಹಂಸರು - ಇದನ್ನೇ ತೋರಿಸುವುದಕ್ಕೆ ಅವತಾರ ಮಾಡಿದ್ದು. ಅವರು ಎಲ್ಲಕ್ಕೂ ಗೌರವ ಕೊಡುತ್ತಿದ್ದರು. ಮತ್ತೆ ‘ಬ್ರಹ್ಮಜ್ಞಾನದ ಕಡೆಯಿಂದ ನೋಡಿದರೆ ಇವೆಲ್ಲಾ ಮಿಥ್ಯಾ ಮಾಯಾಮಾತ್ರ’ ಎಂದು ಹೇಳುತ್ತಿದ್ದರು.

ಶಿಷ್ಯ: ಮಹಾಶಯರೆ, ತಮ್ಮ ಮಾತು ನನಗೆ ಅರ್ಥವಾಗಲಿಲ್ಲ; ಮಧ್ಯ ಮಧ್ಯದಲ್ಲಿ ನನ್ನ ಮನಸ್ಸಿಗೆ ಎನ್ನಿಸುತ್ತದೆ - ತಾವೂ ಹೀಗೆ ಉತ್ಸವ ಪ್ರಚಾರಾದಿಗಳನ್ನು ಮಾಡಿ ಪರಮಹಂಸರ ಹೆಸರಿಟ್ಟು ಒಂದು ಸಂಪ್ರದಾಯವನ್ನು ಆರಂಭಿಸುತ್ತೀರಿ ಎಂದು. ನಾನು ನಾಗಮಹಾಶಯರ ಬಾಯಿಯಿಂದ, ಪರಮಹಂಸರು ಯಾವ ಪಂಗಡಕ್ಕೂ ಸೇರಿರಲಿಲ್ಲವೆಂದೂ ಅವರು ಶಾಸ್ತ್ರ, ವೈಷ್ಣವ, ಬ್ರಾಹ್ಮೋಗಳ, ಮುಸಲ್ಮಾನ, ಕ್ರೈಸ್ತ ಇವರೆಲ್ಲರ ಮತಕ್ಕೂ ಗೌರವ ಕೊಡುತ್ತಿದ್ದರೆಂದೂ ಕೇಳಿದ್ದೇನೆ.

ಸ್ವಾಮೀಜಿ: ನಾನು ಸಕಲ ಧರ್ಮಗಳನ್ನೂ ಹಾಗೆ ಗೌರವಿಸುವುದಿಲ್ಲವೆಂದು ನೀನು ಹೇಗೆ ತಿಳಿದುಕೊಂಡೆ?

ಹೀಗೆಂದು ಹೇಳಿ ಸ್ವಾಮೀಜಿ ನಿರಂಜನ ಮಹಾರಾಜರನ್ನು ಕುರಿತು ನಗುತ್ತ ನಗುತ್ತ, “ಏನಯ್ಯಾ, ಈ ಬಂಗಾಳನು (ಪೂರ್ವ ಬಂಗಾಳದವನು) ಹೇಳುತ್ತಿರುವುದೇನು?" ಎಂದು ಕೇಳಿದರು.

ಶಿಷ್ಯ: ಮಹಾಶಯರೆ, ದಯೆಯಿಟ್ಟು ಈ ವಿಷಯವನ್ನು ನನಗೆ ತಿಳಿಸಿಕೊಡಬೇಕು.

ಸ್ವಾಮೀಜಿ: ನೀನು ನನ್ನ ಉಪನ್ಯಾಸಗಳನ್ನು ಓದಿದ್ದೀಯಷ್ಟೆ; ಹೇಳು ನೋಡೋಣ, ನಾನು ಎಲ್ಲಿ ಪರಮಹಂಸರ ಹೆಸರನ್ನು ಎತ್ತಿದ್ದೇನೆ? ಶುದ್ಧ ಉಪನಿಷದ್ಧರ್ಮವನ್ನೆ ಹೇಳುತ್ತ ಜಗತ್ತಿನಲ್ಲಿ ಸಂಚಾರ ಮಾಡಿದ್ದೇನೆ.

ಶಿಷ್ಯ: ಅದೇನೋ ನಿಜ, ಆದರೆ ತಮ್ಮೊಡನೆ ಪರಿಚಿತನಾಗಿ ನೋಡಿದ್ದೇನೆ - ತಾವು ರಾಮಕೃಷ್ಣಗತಪ್ರಾಣರೇ. ಪರಮಹಂಸರನ್ನು ತಾವು ದೇವರೆಂದು ತಿಳಿದುಕೊಂಡಿದ್ದರೆ ಇತರ ಜನರಿಗೂ ಅದನ್ನು ಒಟ್ಟಿಗೆ ಹೇಳಿಬಿಡಬಾರದೇಕೆ?

ಸ್ವಾಮೀಜಿ: ನಾನು ಏನು ತಿಳಿದುಕೊಂಡಿದ್ದೇನೆಯೊ ಅದನ್ನು ಹೇಳುತ್ತೇನೆ. ನೀನು ಅದ್ವೈತ ವೇದಾಂತ ಮತವನ್ನು ಸರಿಯಾದ ಧರ್ಮವೆಂದು ತಿಳಿದುಕೊಂಡಿದ್ದರೆ, ಜನರಿಗೆ ಅದನ್ನೇ ತಿಳಿಸಬಾರದೇಕೆ?

ಶಿಷ್ಯ: ಮೊದಲು ಅನುಭವಕ್ಕೆ ತಂದುಕೊಳ್ಳುತ್ತೇನೆ. ಆಮೇಲೆ ತಿಳಿಸುತ್ತೇನೆ; ಈ ಮತವನ್ನು ನಾನು ಸುಮ್ಮನೆ ಓದಿದ್ದೇನೆ ಅಷ್ಟೇ.

ಸ್ವಾಮೀಜಿ: ಹಾಗಾದರೆ ಮೊದಲು ಅನುಭವಕ್ಕೆ ತಂದುಕೊ, ಆಮೇಲೆ ಜನರಿಗೆ ತಿಳಿಸುವೆಯಂತೆ. ಈಗ ಜನರಲ್ಲಿ ಪ್ರತಿಯೊಬ್ಬನೂ ಒಂದೊಂದು ಮತವನ್ನು ನಂಬಿಕೊಂಡು ಹೋಗುತ್ತಿದ್ದಾನೆ - ಆದ್ದರಿಂದ ಹೇಳುವುದಕ್ಕೆ ನಿನಗೂ ಯಾವ ಅಧಿಕಾರವೂ ಇಲ್ಲ. ಏಕೆಂದರೆ, ನೀನೂ ಈಗ ಅವರ ಹಾಗೆ ಒಂದು ಧರ್ಮ ಮತವನ್ನು ನಂಬಿಕೊಂಡು ಹೋಗುತ್ತಿದ್ದೀಯೆ ಹೊರತು ಮತ್ತೆ ಬೇರೆ ಅಲ್ಲ.

ಶಿಷ್ಯ: ಹೌದು - ನಾನೂ ಒಂದನ್ನು ನಂಬಿಕೊಂಡೆ ಹೋಗುತ್ತಿದ್ದೇನೆ ನಿಜ; ಆದರೆ ನನ್ನ ಪ್ರಮಾಣ ಶಾಸ್ತ್ರ, ನಾನು ಶಾಸ್ತ್ರಕ್ಕೆ ವಿರೋಧಿಯಾದ ಮತಕ್ಕೆ ಗೌರವ ಕೊಡುವುದಿಲ್ಲ.

ಸ್ವಾಮೀಜಿ: ಶಾಸ್ತ್ರಕ್ಕೆ ಗೌರವ ಕೊಡುತ್ತೀಯೋ? ಉಪನಿಷತ್ತು ಪ್ರಮಾಣವಾದರೆ, ಬೈಬಲ್, ಜೆಂದವಸ್ತಗಳು ಏಕೆ ಪ್ರಮಾಣವಾಗಬಾರದು?

ಶಿಷ್ಯ: ಈ ಗ್ರಂಥಗಳಿಗೆ ಪ್ರಮಾಣ್ಯವನ್ನು ಒಪ್ಪಿಕೊಂಡರೂ ವೇದದ ಹಾಗೆ ಪ್ರಾಚೀನ ಗ್ರಂಥಗಳಲ್ಲಿ ಮತ್ತು ಆತ್ಮತತ್ತ್ವವಿಚಾರ ವೇದದಲ್ಲಿ ಎಷ್ಟರಮಟ್ಟಿಗಿದೆಯೋ ಅಷ್ಟರಮಟ್ಟಿಗೆ ಮತ್ತೆಲ್ಲಿಯೂ ಇಲ್ಲ.

ಸ್ವಾಮೀಜಿ: ನಿನ್ನ ಮಾತನ್ನು ಒಪ್ಪಿಕೊಂಡರೂ ವೇದವನ್ನು ಬಿಟ್ಟು ಮತ್ತೆಲ್ಲಿಯೂ ಸತ್ಯವು ಇಲ್ಲವೆಂದು ಹೇಳುವುದಕ್ಕೆ ನಿನಗೇನು ಅಧಿಕಾರ?

ಶಿಷ್ಯ: ವೇದವನ್ನು ಬಿಟ್ಟು ಇತರ ಧರ್ಮಗ್ರಂಥಗಳಲ್ಲಿ ಸತ್ಯವಿರಬಹುದು; ಆ ವಿಷಯದಲ್ಲಿ ವಿರುದ್ಧವಾಗಿ ನಾನು ಏನೂ ಹೇಳುವುದಿಲ್ಲ. ಆದರೆ ನಾನು ಉಪನಿಷತ್ತಿನ ಮತವನ್ನೇ ಅನುಸರಿಸಿಕೊಂಡು ಹೋಗುತ್ತೇನೆ; ನನಗೆ ಅದರಲ್ಲಿ ತುಂಬ ನಂಬಿಕೆ.

ಸ್ವಾಮೀಜಿ: ಹಾಗೆ ಮಾಡು; ಆದರೆ ಮತ್ತಾರಿಗಾದರೂ ಹೀಗೆ ಯಾವುದಾದರೂ ಮತದಲ್ಲಿ ತುಂಬ ನಂಬಿಕೆ ಇದ್ದರೆ ಅವರೂ ಆ ನಂಬಿಕೆಯಿಂದಲೆ ಮುಂದುವರಿಯುವುದಕ್ಕೆ ಅವಕಾಶ ಕೊಡು. ಆಮೇಲೆ ನೀನೂ ಆತನೂ ಒಂದು ಕಡೆಯಲ್ಲಿ ಸೇರುವಿರೆಂಬುದು ಗೊತ್ತಾಗುತ್ತದೆ. ಶಿವಮಹಿಮ್ನ ಸ್ತೋತ್ರದಲ್ಲಿ ‘ತ್ವಮಸಿ ಪಯಸಾಮರ್ಣವ ಇವ’ - ಹರಿಯುವ ನದಿಗೆ ಸಮುದ್ರವು ಹೇಗೋ ಹಾಗೆ ನೀನು ಎಂಬುದನ್ನು ಓದಿಲ್ಲವೆ?

\newpage

\chapter[ಅಧ್ಯಾಯ ೬]{ಅಧ್ಯಾಯ ೬\protect\footnote{\engfoot{C.W, Vol. VI, P 471}}}

\begin{center}
ಸ್ಥಳ: ಆಲಂಬಜಾರಿನ ಮಠ, ಕಲ್ಕತ್ತ, ವರ್ಷ: ಕ್ರಿ.ಶ. ೧೮೯೭ನೆಯ ಮೇ ತಿಂಗಳು.
\end{center}

ಸ್ವಾಮೀಜಿ ಡಾರ್ಜಿಲಿಂಗ್‌ನಿಂದ ಕಲ್ಕತ್ತೆಗೆ ಹಿಂತಿರುಗಿ ಬಂದಿದ್ದಾರೆ. ಆಲಂಬಜಾರಿನ ಮಠದಲ್ಲಿಯೇ ಇರುತ್ತಿದ್ದಾರೆ. ಗಂಗಾತೀರದಲ್ಲಿರುವ ಯಾವುದಾದರೊಂದು ಕಡೆಗೆ ಮಠವನ್ನು ತೆಗೆದುಕೊಂಡು ಹೋಗುವ ಪ್ರಸ್ತಾಪ ನಡೆಯುತ್ತಿದೆ. ಶಿಷ್ಯನು ಈಗ ಮಠದಲ್ಲಿ ಅವರ ಹತ್ತಿರಕ್ಕೆ ವಿಶೇಷವಾಗಿ ಹೋಗಿಬರುತ್ತಿದ್ದಾನೆ. ಮತ್ತು ಮಧ್ಯೆ ಮಧ್ಯೆ ರಾತ್ರಿ ಅಲ್ಲಿಯೇ ಇದ್ದು ಬಿಡುತ್ತಾನೆ. ಶಿಷ್ಯನ ಜೀವನದಲ್ಲಿ ಪ್ರಥಮ ಮಾರ್ಗದರ್ಶಕರಾದ ನಾಗಮಹಾಶಯರು ಅವನಿಗೆ ಮಂತ್ರ ದೀಕ್ಷೆಯನ್ನು ಕೊಡಲಿಲ್ಲ; ಮತ್ತು ಮಂತ್ರೋಪದೇಶದ ಮಾತು ಬಂದಾಗ ಸ್ವಾಮೀಜಿಯ ಪ್ರಸ್ತಾಪ ಬಂದು “ಸ್ವಾಮಿಗಳೆ ಜಗತ್ತಿಗೆ ಗುರುವಾಗುವುದಕ್ಕೆ ಯೋಗ್ಯರು!" ಎಂದು ಅವನಿಗೆ ಹೇಳುತ್ತಿದ್ದರು. ದೀಕ್ಷಾಗ್ರಹಣದಲ್ಲಿ ಕೃತಸಂಕಲ್ಪನಾಗಿ ಶಿಷ್ಯನು ಇದಕ್ಕೋಸ್ಕರವೆ ಡಾರ್ಜಿಲಿಂಗಿಗೆ ಸ್ವಾಮೀಜಿಗೆ ಒಂದು ಕಾಗದ ಬರೆದು ತಿಳಿಸಿದ್ದನು. ಸ್ವಾಮೀಜಿ ಅದಕ್ಕೆ ಉತ್ತರವಾಗಿ “ನಾಗಮಹಾಶಯರ ಆಕ್ಷೇಪಣೆ ಇಲ್ಲದಿದ್ದರೆ ನಿನಗೆ ಅತ್ಯಾನಂದದಿಂದ ದೀಕ್ಷೆಯನ್ನು ಕೊಡುತ್ತೇನೆ" ಎಂದು ಬರೆದರು. ಆ ಕಾಗದ ಶಿಷ್ಯನ ಹತ್ತಿರದಲ್ಲಿ ಈಗಲೂ ಇದೆ.

೧೩೧೩ನೆಯ ಬಂಗಾಳಿ ಸಂವತ್ಸರದ ೧೯ನೆಯ ವೈಶಾಖ.\footnote{ಏಪ್ರಿಲ್ - ಮೇ.} ಸ್ವಾಮಿಜಿ ಈ ದಿನ ಶಿಷ್ಯನಿಗೆ ದೀಕ್ಷೆಯನ್ನು ಕೊಡುತ್ತೇನೆಂದು ಮಾತು ಕೊಟ್ಟಿದ್ದಾರೆ. ಈ ದಿನ ಶಿಷ್ಯನ ಜೀವನದಲ್ಲಿ ಅತ್ಯಂತ ವಿಶೇಷ ದಿನ! ಶಿಷ್ಯನು ಬೆಳಿಗ್ಗೆ ಹೊತ್ತಿಗೆ ಮುಂಚೆ ಗಂಗೆಯಲ್ಲಿ ಸ್ನಾನಮಾಡಿ ಲಿಚಿ ಹಣ್ಣು ಮತ್ತು ಇತರ ಪದಾರ್ಥಗಳನ್ನು ಕೊಂಡು ಸುಮಾರು ಎಂಟು ಗಂಟೆಯ ಹೊತ್ತಿಗೆ ಆಲಂಬಜಾರಿನ ಮಠಕ್ಕೆ ಬಂದಿದ್ದಾನೆ. ಶಿಷ್ಯನನ್ನು ನೋಡಿ ಸ್ವಾಮೀಜಿ ಗೋಪ್ಯವಾಗಿ “ಇಂದು ನಿನ್ನನ್ನು ‘ಬಲಿ’ ಕೊಡಬೇಕು - ಅಲ್ಲವೆ?" ಎಂದರು.

ಸ್ವಾಮಿಜಿ ಶಿಷ್ಯನಿಗೆ ಈ ಮಾತನ್ನು ಹೇಳಿ ಮತ್ತೆ ನಗುನಗುತ್ತ ಎಲ್ಲರೊಡನೆಯೂ ಅಮೆರಿಕಾದ ನಾನಾ ಪ್ರಸ್ತಾಪಗಳನ್ನು ಕುರಿತು ಮಾತನಾಡತೊಡಗಿದರು. ಧಾರ್ಮಿಕ ಜೀವನವನ್ನು ಬೆಳೆಸಬೇಕಾದರೆ ಹೇಗೆ ಅಚಲವಾದ ನಂಬಿಕೆ ಮತ್ತು ದೃಢವಾದ ಭಕ್ತಿಭಾವವನ್ನು ಇಟ್ಟಿರಬೇಕು, ಗುರುವಿಗಾಗಿ ಹೇಗೆ ಪ್ರಾಣವನ್ನು ಕೊಡುವುದಕ್ಕೂ ಸಿದ್ಧನಾಗಿರಬೇಕು ಈ ಮಾತುಗಳೆಲ್ಲಾ ಜೊತೆಯಲ್ಲಿ ಬಂದುವು. ಅನಂತರದಲ್ಲಿ ಶಿಷ್ಯನಿಗೆ ಕೆಲವು ಪ್ರಶ್ನೆಗಳನ್ನು ಹಾಕಿ ಅವನ ಹೃದಯವನ್ನು ಪರೀಕ್ಷಿಸುವುದಕ್ಕೆ ಮೊದಲುಮಾಡಿದರು: “ನಾನು ನಿನಗೆ ಯಾವಾಗ ಯಾವ ಕೆಲಸವನ್ನು ಮಾಡು ಎಂದು ಹೇಳುತ್ತೇನೆಯೋ ಆವಾಗ ಅದನ್ನು ಸಾಧ್ಯವಾದಷ್ಟು ಮಾಡುತ್ತಿಯೋ? ಗಂಗೆಯಲ್ಲಿ ಧುಮುಕಿದರೆ ಅಥವಾ ಮನೆಯ ಛಾವಣಿಯಿಂದ ಕೆಳಕ್ಕೆ ಹಾರಿದರೆ ನಿನಗೆ ಮಂಗಳವಾಗುವುದೆಂದು ತಿಳಿದುಕೊಂಡು ಅದನ್ನೇ ಮಾಡು ಎಂದು ಹೇಳಿದರೆ ಅದನ್ನೂ ಹಿಂದು ಮುಂದು ನೋಡದೆ ಮಾಡಬಲ್ಲೆಯೋ? ಈಗಲೂ ಯೋಚಿಸಿನೋಡು. ಇಲ್ಲದಿದ್ದರೆ ಭಾವಾವೇಶಕ್ಕೆ ಒಳಗಾಗಿ ನನ್ನನ್ನು ಗುರುವೆಂದು ಸ್ವೀಕರಿಸಲು ಮುನ್ನುಗ್ಗಬೇಡ.” ಹೀಗೆ ಕೆಲವು ಪ್ರಶ್ನೆಗಳನ್ನು ಹಾಕಿ ಸ್ವಾಮೀಜಿ ಶಿಷ್ಯನ ಮನಸ್ಸಿನಲ್ಲಿದ್ದ ನಂಬಿಕೆಯ ದಾರ್ಢ್ಯವನ್ನು ತಿಳಿದುಕೊಂಡರು. ಶಿಷ್ಯನು ತಲೆಯನ್ನು ಬಗ್ಗಿಸಿ “ಮಾಡಬಲ್ಲೆ" ಎಂದು ಪ್ರತಿಯೊಂದು ಪ್ರಶ್ನೆಗೂ ಉತ್ತರ ಕೊಡುತ್ತ ಬಂದನು.

ಸ್ವಾಮೀಜಿ ಹೇಳುವುದಕ್ಕೆ ಮೊದಲುಮಾಡಿದರು: “ಯಾರು ಈ ಕೊನೆ ಮೊದಲಿಲ್ಲದ ಹುಟ್ಟು ಸಾವುಗಳ ಮಾಯೆಯಿಂದ ಹೊರಕ್ಕೆ ಕರೆದುಕೊಂಡು ಹೋಗಿಬಿಡುವನೊ, ಯಾರು ಕೃಪೆಮಾಡಿ ಸಮಸ್ತ ಮಾನಸಿಕ ಆಧಿವ್ಯಾಧಿಗಳನ್ನೂ ಹೋಗಲಾಡಿಸುವನೊ ಅವನೆ ನಿಜವಾದ ಗುರು. ಮೊದಲು ಶಿಷ್ಯರು ಸಮಿತ್ತನ್ನು ಹಿಡಿದುಕೊಂಡು ಗುರುವಿನ ಆಶ್ರಮಕ್ಕೆ ಹೋಗುತ್ತಿದ್ದರು. ಗುರುವಿಗೆ ಅಧಿಕಾರಿಯೆಂದು ತಿಳಿದರೆ ಆತನು ಅಂಥವನಿಗೆ ದೀಕ್ಷೆ ಕೊಟ್ಟು ವೇದವನ್ನು ಹೇಳಿ ಮನೋವಾಕ್ಕಾಯ ದಂಡನರೂಪವಾದ ವ್ರತಕ್ಕೆ ಚಿಹ್ನೆಯಾಗಿ ಮೂರು ಸುತ್ತು ಮೌಂಜಿ ಮೇಖಲೆಯನ್ನು ಅವರ ಸೊಂಟದಲ್ಲಿ ಕಟ್ಟುತ್ತಿದ್ದನು. ಆ ಮೌಂಜಿ ಮೇಖಲೆಗೆ ಬದಲಾಗಿ ಆ ಮೇಲೆ ಯಜ್ಞಸೂತ್ರ ಅಥವಾ ಯಜ್ಞೋಪವೀತವನ್ನು ಹಾಕಿಕೊಳ್ಳುವ ಪದ್ಧತಿ ಬಂತು."

ಶಿಷ್ಯ: ಹಾಗಾದರೇನು, ಮಹಾಶಯರೆ, ನಮ್ಮ ಹಾಗೆ ದಾರದ ಯಜ್ಞೋಪವೀತವನ್ನು ಹಾಕಿಕೊಳ್ಳುವುದು ವೇದೋಕ್ತ ಮಾರ್ಗವಲ್ಲವೆ?

ಸ್ವಾಮೀಜಿ: ವೇದದಲ್ಲಿ ಎಲ್ಲಿಯೂ ಯಜ್ಞೋಪವೀತಸೂತ್ರದ ಪ್ರಸ್ತಾಪವಿಲ್ಲ. ಆಧುನಿಕ ಸ್ಮೃತಿಕಾರರಾದ ರಘುನಂದನ ಭಟ್ಟಾಚಾರ್ಯರು: ‘ಈ ಸಂದರ್ಭದಲ್ಲಿಯೇ ಯಜ್ಞಸೂತ್ರವನ್ನು ಧರಿಸಬೇಕು’ ಎಂದು ಬರೆದಿದ್ದಾರೆ. ಯಜ್ಞೋಪವೀತದ ಪ್ರಸ್ತಾಪವು ಗೋಭಿಲ ಗೃಹ್ಯಸೂತ್ರದಲ್ಲಿಯೂ ಇಲ್ಲ. ಗುರುವಿನ ಹತ್ತಿರ ಆಗುವ ಈ ಮೊದಲನೆಯ ವೈದಿಕ ಸಂಸ್ಕಾರವೆ ‘ಉಪನಯನ’ವೆಂದು ಶಾಸ್ತ್ರದಲ್ಲಿ ಉಕ್ತವಾಗಿರುವುದು. ಆದರೆ ಈಗಿನ ಕಾಲದಲ್ಲಿ ದೇಶಕ್ಕೆ ಎಂಥ ದುರವಸ್ಥೆಯುಂಟಾಗಿ ಹೋಗಿದೆ! ಶಾಸ್ರೋಕ್ತಮಾರ್ಗವನ್ನು ಬಿಟ್ಟು ಕೇವಲ ಕೆಲವು ದೇಶಾಚಾರ, ಲೋಕಾಚಾರ, ಮತ್ತು ಹೆಂಗಸರ ಆಚಾರಗಳಲ್ಲಿ ದೇಶ ಮುಳುಗಿಹೋಗಿದೆ. ಆದ್ದರಿಂದಲೇ ನೀವು ಪ್ರಾಚೀನ ಕಾಲದಂತೆ ಶಾಸ್ತ್ರ ಮಾರ್ಗವನ್ನನುಸರಿಸಿ ಹೋಗಿರೆಂದು ನಿಮಗೆ ಹೇಳುತ್ತೇನೆ; ನೀವು ಶ್ರದ್ಧಾವಂತರಾಗಿ ದೇಶದಲ್ಲಿಯೂ ಶ್ರದ್ಧೆಯನ್ನು ಉಂಟುಮಾಡಿ ನಚಿಕೇತನ ಹಾಗೆ ಯಮಲೋಕಕ್ಕೆ ಹೋಗಿ, ಆತ್ಮತತ್ತ್ವವನ್ನು ತಿಳಿದುಕೊಳ್ಳುವುದಕ್ಕೋಸ್ಕರ ಆತ್ಮೋದ್ಧಾರಕ್ಕೋಸ್ಕರ, ಈ ಜನನ ಮರಣ ಸಮಸ್ಯೆಯ ವಿಚಾರಕ್ಕೋಸ್ಕರ, ಯಮನ ಬಾಯಿಗೆ ಹೋದರೆ ತತ್ತ್ವ ತಿಳಿದುಬರುವ ಹಾಗಿದ್ದ ಪಕ್ಷದಲ್ಲಿ ಅದಕ್ಕೋಸ್ಕರ ನಿರ್ಭೀತರಾಗಿ ಯಮನ ಬಾಯಿಗೆ ಹೋಗಿ ಬೀಳಬೇಕು. ಭಯವೇ ಮೃತ್ಯು, ಭಯದ ಆಚೆಯ ದಡಕ್ಕೆ ಹೋಗಬೇಕು. ಇಂದಿನಿಂದ ಭಯರಹಿತನಾಗು, ನಡೆ, ಹೊರಡು - ನಿನ್ನ ಮೋಕ್ಷಕ್ಕೋಸ್ಕರವೂ ಪರಪ್ರಯೋಜನಕ್ಕಾಗಿಯೂ ದೇಹವನ್ನು ಕೊಡುವುದಕ್ಕೆ ಆಗುವುದೇನು! ಕೆಲವು ರಕ್ತ ಮೂಳೆ ಮಾಂಸಗಳ ಮುದ್ದೆಗಳನ್ನು ಹೊತ್ತುಕೊಂಡು ಏನು ಪ್ರಯೋಜನ? ಈಶ್ವರಾರ್ಥವಾಗಿ ಸರ್ವಸ್ವ ತ್ಯಾಗರೂಪವಾದ ಮಂತ್ರ ದೀಕ್ಷೆಯನ್ನು ಪಡೆದು ದಧೀಚಿ ಮುನಿಯ ಹಾಗೆ ಪರರಿಗೋಸ್ಕರ ಮೂಳೆ ಮಾಂಸಗಳನ್ನು ದಾನಮಾಡು. ಯಾರು ವೇದ ವೇದಾಂತಗಳನ್ನು ವ್ಯಾಸಂಗ ಮಾಡಿರುವನೊ ಯಾರು ಬ್ರಹ್ಮಜ್ಞನೊ ಯಾರು ಪರರನ್ನು ಅಭಯದ ತೀರಕ್ಕೆ ಕರೆದುಕೊಂಡು ಹೋಗಲು ಸಮರ್ಥನೊ, ಅವನೆ ನಿಜವಾದ ಗುರು; ಅಂಥವನ ಹತ್ತಿರವೆ ದೀಕ್ಷೆಯನ್ನು ಪಡೆಯಬೇಕು - ‘ನಾತ್ರ ಕಾರ್ಯಾ ವಿಚಾರಣಾ’ - ಅಂಥಲ್ಲಿ ಯಾವುದನ್ನೂ ವಿಚಾರಿಸಬೇಕಾಗಿಲ್ಲ. ಈಗ ಅದು ಏನಾಗಿದೆ ಬಲ್ಲೆಯಾ? - “ಅಂಧೇನೈವ ನೀಯಮಾನಾ ಯಥಾಂಧಾಃ" - ಕುರುಡನು ಕುರುಡನಿಗೆ ದಾರಿ ತೋರಿಸಿದಂತೆ.

ಹೊತ್ತು ಸುಮಾರು ಒಂಬತ್ತು ಗಂಟೆಯಾಗಿದೆ. ಸ್ವಾಮೀಜಿ ಇಂದು ಗಂಗೆಗೆ ಹೋಗದೆ ಮನೆಯಲ್ಲಿಯೇ ಸ್ನಾನಮಾಡಿದ್ದಾರೆ. ಸ್ನಾನವಾದ ಮೇಲೆ ಒಂದು ಮಡಿ ಕಾವಿಯ ಬಟ್ಟೆಯನ್ನು ಉಟ್ಟುಕೊಂಡು ಮೆಲ್ಲಗೆ ದೇವರ ಮನೆಗೆ ಹೋಗಿ ಪೂಜಾಸನದಲ್ಲಿ ಕುಳಿತುಕೊಂಡರು. ಶಿಷ್ಯನು ದೇವರ ಮನೆಗೆ ಹೋಗದೆ ಹೊರಗೆ ಕಾದುಕೊಂಡಿದ್ದನು - ಸ್ವಾಮೀಜಿ ಕರೆದರೆ ಆಮೇಲೆ ಹೋಗೋಣವೆಂದು. ಈಗ ಸ್ವಾಮೀಜಿ ಧ್ಯಾನಮಗ್ನರಾದರು; ಮುಕ್ತ ಪದ್ಮಾಸನ; ಕಣ್ಣುಗಳು ಅರ್ಧ ಮುಚ್ಚಿವೆ. ದೇಹ ಮನಸ್ಸು ಪ್ರಾಣ ಇವೆಲ್ಲ ನಿಶ್ಚಲವಾಗಿ ಹೋಗಿರುವಂತಿವೆ. ಧ್ಯಾನ ಮುಗಿದ ಮೇಲೆ ಸ್ವಾಮೀಜಿ ಶಿಷ್ಯನನ್ನು “ಬಾಪ್ಪಾ!" ಎಂದು ಕರೆದರು. ಶಿಷ್ಯನು ಸ್ವಾಮೀಜಿಯ ಅಂತಃಕರಣಪೂರ್ವಕವಾದ ಕರೆಗೆ ಮುಗ್ಧನಾಗಿ ದೇವರ ಮನೆಯೊಳಕ್ಕೆ ಹೋದನು. ದೇವರ ಮನೆಯೊಳಕ್ಕೆ ಪ್ರವೇಶ ಮಾಡುತ್ತಿದ್ದ ಹಾಗೆಯೇ “ಬಾಗಿಲ ಚಿಲಕವನ್ನು ಹಾಕಿಬಿಡು" ಎಂದು ಶಿಷ್ಯನಿಗೆ ಹೇಳಿದರು. ಹಾಗೆ ಮಾಡಿದ ಕೂಡಲೆ “ಸ್ಥಿರವಾಗಿ ನನ್ನ ಎಡಭಾಗದಲ್ಲಿ ಕುಳಿತುಕೊ" ಎಂದರು. ಸ್ವಾಮಿಗಳ ಆಜ್ಞೆಯನ್ನು ಶಿರಸಾವಹಿಸಿ ಶಿಷ್ಯನು ಆಸನದಲ್ಲಿ ಕುಳಿತುಕೊಂಡನು. ಅವನ ಹೃತ್ಪಿಂಡವು ಆಗ ಒಂದು ವಿಧವಾದ ಅನಿರ್ವಚನೀಯವಾದ ಅಪೂರ್ವ ಭಾವದಲ್ಲಿ ಡವಡವನೆ ಬಡಿದುಕೊಳ್ಳುತ್ತಿತ್ತು. ಅನಂತರದಲ್ಲಿ ಸ್ವಾಮೀಜಿ ತಮ್ಮ ಕರಕಮಲವನ್ನು ಶಿಷ್ಯನ ತಲೆಯ ಮೇಲೆ ಇಟ್ಟು ಶಿಷ್ಯನನ್ನು ಕೆಲವು ಗೋಪ್ಯವಾದ ವಿಷಯಗಳನ್ನು ಕೇಳಿದರು; ಮತ್ತು ಶಿಷ್ಯನು ಈ ವಿಷಯಗಳಿಗೆ ಸಾಧ್ಯವಾದ ಮಟ್ಟಿಗೆ ಉತ್ತರವನ್ನು ಕೊಟ್ಟ ಮೇಲೆ ಮಹಾಬೀಜ ಮಂತ್ರವನ್ನು ಅವನ ಕಿವಿಯಲ್ಲಿ ಮೂರು ಸಾರಿ ಉಚ್ಚರಿಸಿದರು, ಆಮೇಲೆ ಅದನ್ನು ಮೂರು ಸಾರಿ ಶಿಷ್ಯನಿಂದ ಹೇಳಿಸಿದರು. ಅನಂತರ ಸಾಧನದ ವಿಚಾರವಾಗಿ ಸಾಮಾನ್ಯ ಉಪದೇಶವನ್ನು ಕೊಟ್ಟು ಸ್ಥಿರರಾಗಿ ಬಿಡುಗಣ್ಣಿನಿಂದ ಸ್ವಲ್ಪ ಹೊತ್ತು ಶಿಷ್ಯನ ಕಣ್ಣುಗಳನ್ನು ದುರುಗುಟ್ಟಿಕೊಂಡು ನೋಡುತ್ತಿದ್ದರು. ಶಿಷ್ಯನ ಮನಸ್ಸು ಈಗ ಸ್ತಬ್ಧವಾಗಿ ಏಕಾಗ್ರವಾಗಲು ಅವನು ಒಂದು ಅನಿರ್ವಚನೀಯವಾದ ಭಾವದಲ್ಲಿ ಸ್ಥಿರವಾಗಿ ಕುಳಿತುಕೊಂಡನು. ಎಷ್ಟು ಹೊತ್ತು ಹೀಗೆ ಕಳೆಯಿತೊ ಅದನ್ನು ತಿಳಿಯುವುದಕ್ಕೆ ಸಾಧ್ಯವಾಗಲಿಲ್ಲ. ಆಮೇಲೆ ಸ್ವಾಮೀಜಿ “ಗುರುದಕ್ಷಿಣೆಯನ್ನು ಕೊಡು" ಎಂದರು. ಶಿಷ್ಯನು “ಏನು ಕೊಡಲಿ?" ಎಂದು ಕೇಳಿದನು. ಅದನ್ನು ಕೇಳಿ ಸ್ವಾಮೀಜಿ “ಹೋಗು, ಉಗ್ರಾಣದಿಂದ ಯಾವುದಾದರೂ ಹಣ್ಣನ್ನು ತೆಗೆದುಕೊಂಡು ಬಾ" ಎಂದು ಆಜ್ಞಾಪಿಸಿದರು. ಶಿಷ್ಯನು ಉಗ್ರಾಣಕ್ಕೆ ಓಡಿಹೋಗಿ ಹತ್ತು ಹದಿನೈದು ಲಿಚಿ ಹಣ್ಣುಗಳನ್ನು ತೆಗೆದುಕೊಂಡು ಬಂದನು. ಸ್ವಾಮಿಜಿ ಕೈಗೆ ಅದನ್ನು ಕೊಡುತ್ತಿದ್ದ ಹಾಗೆಯೆ ಅವರು ಹಾಗೆಯೆ ಅವುಗಳನ್ನೆಲ್ಲಾ ಒಂದೊಂದಾಗಿ ತಿಂದುಬಿಟ್ಟು “ನೀನು ಗುರುದಕ್ಷಿಣೆಯನ್ನು ಕೊಟ್ಟು ಆಯಿತು, ನಡೆ!" ಎಂದರು.

ಶಿಷ್ಯನು ದೇವರ ಮನೆಯಲ್ಲಿ ದೀಕ್ಷೆಯನ್ನು ಪಡೆದಾಗ ಮಠಕ್ಕೆ ಸೇರಿದ ಮತ್ತೊಬ್ಬ ವ್ಯಕ್ತಿಯೂ (ಶುದ್ಧಾನಂದ ಸ್ವಾಮಿಗಳು) ತಟ್ಟನೆ ದೀಕ್ಷೆ ಪಡೆಯಬೇಕೆಂದು ಸಂಕಲ್ಪ ಮಾಡಿಕೊಂಡು ಬಾಗಿಲ ಹತ್ತಿರ ನಿಂತುಕೊಂಡಿದ್ದರು. ಅವರು ಬ್ರಹ್ಮಚಾರಿಯಾಗಿ ಮಠದಲ್ಲಿ ಸೇರಿದ್ದರೂ ಇದಕ್ಕೆ ಹಿಂದೆ ಶಾಸ್ತ್ರೀಯ ದೀಕ್ಷೆಯನ್ನು ಪಡೆದಿರಲಿಲ್ಲ. ಶಿಷ್ಯನು ಈ ದಿವಸ ಹೀಗೆ ದೀಕ್ಷೆ ಪಡೆದದ್ದನ್ನು ನೋಡಿ ಅವರೂ ಈಗ ಈ ವಿಚಾರದಲ್ಲಿ ಉತ್ಸಾಹಿತರಾಗಿ, ಶಿಷ್ಯನು ದೀಕ್ಷೆ ಪಡೆದು ದೇವರ ಮನೆಯಿಂದ ಹೊರಗೆ ಹೊರಡುತ್ತಿದ್ದ ಹಾಗೆ, ಆ ಮನೆಯೊಳಗೆ ಸ್ವಾಮಿಗಳ ಹತ್ತಿರ ಹೋಗಿ ತಮ್ಮ ಅಭಿಪ್ರಾಯವನ್ನು ವಿಜ್ಞಾಪಿಸಿಕೊಂಡರು. ಸ್ವಾಮೀಜಿ ಶುದ್ಧಾನಂದರ ಅತಿಶಯವಾದ ಆಸಕ್ತಿಯನ್ನು ನೋಡಿ ಅದಕ್ಕೆ ಒಪ್ಪಿಕೊಂಡು ಪುನಃ ಪೂಜಾಸನದಲ್ಲಿ ಕುಳಿತುಕೊಂಡರು.

ಅನಂತರದಲ್ಲಿ ಶುದ್ಧಾನಂದರಿಗೆ ದೀಕ್ಷೆಯನ್ನು ಕೊಟ್ಟು ಸ್ವಾಮಿಜಿ ಸ್ವಲ್ಪ ಹೊತ್ತಿನ ಮೇಲೆ ಹೊರಗೆ ಬಂದರು. ಊಟವಾದ ಮೇಲೆ ಮಲಗಿ ಸ್ವಲ್ಪ ವಿಶ್ರಾಂತಿ ಪಡೆದರು. ಶಿಷ್ಯನೂ ಈ ಮಧ್ಯೆ ಶುದ್ಧಾನಂದರೊಡನೆ ಸ್ವಾಮಿಜಿಯ ಭುಕ್ತಶೇಷವನ್ನು ಆನಂದದಿಂದ ಸ್ವೀಕರಿಸಿ ಬಂದು ಅವರ ಪದತಲದಲ್ಲಿ ಕುಳಿತುಕೊಂಡು ಮೆಲ್ಲ ಮೆಲ್ಲನೆ ಅವರ ಕಾಲುಗಳನ್ನು ಒತ್ತುತ್ತಿದ್ದನು.

ವಿಶ್ರಾಂತಿಯಾದ ಮೇಲೆ ಸ್ವಾಮೀಜಿ ಮೇಲಿನ ಬೈಠಕ್ ಖಾನೆಗೆ ಬಂದು ಕುಳಿತುಕೊಂಡರು. ಶಿಷ್ಯನು ಈ ಸಮಯದಲ್ಲಿ ಅವಕಾಶವನ್ನು ಮಾಡಿಕೊಂಡು ಅವರನ್ನು ಕುರಿತು “ಮಹಾಶಯರೆ, ಪಾಪಪುಣ್ಯಗಳ ಭಾವ ಎಲ್ಲಿಂದ ಬಂತು?" ಎಂದು ಕೇಳಿದನು.

ಸ್ವಾಮೀಜಿ: ಬಹುತ್ವಭಾವದಿಂದಲೇ ಇದೆಲ್ಲಾ ಬಂದಿದೆ. ಮನುಷ್ಯ ಏಕತ್ವದ ಕಡೆಗೆ ಎಷ್ಟೆಷ್ಟು ಮುಂದುವರಿಯುತ್ತಾನೆಯೋ ಧರ್ಮಾಧರ್ಮ ದ್ವಂದ್ವಭಾವಗಳಿಗೆಲ್ಲಾ ಕಾರಣವಾದ ‘ನಾನು ನೀನು’ ಎಂಬ ಭಾವವು ಅಷ್ಟಷ್ಟು ಕಡಿಮೆಯಾಗುತ್ತ ಹೋಗುತ್ತದೆ. ನನಗಿಂತ ಇವನು ಬೇರೆ - ಎಂಬ ಭಾವ ಮನಸ್ಸಿಗೆ ಬಂದರೆ, ಆಗ ಇತರ ದ್ವಂದ್ವಭಾವಗಳೆಲ್ಲಾ ವಿಕಾಸವನ್ನು ಪಡೆಯುತ್ತ ಹೋಗುತ್ತವೆ; ಮತ್ತು ಏಕತ್ವದ ಸಂಪೂರ್ಣವಾದ ಅನುಭವವಾದರೆ ಮನುಷ್ಯನಿಗೆ ಆಮೇಲೆ ಶೋಕಮೋಹಗಳು ಇರುವುದಿಲ್ಲ - ‘ತತ್ರ ಕೋ ಮೋಹಃ ಕಃ ಶೋಕಃ ಏಕತ್ವಮನುಪಶ್ಯತಃ?’

ಯಾವುದಾದರೂ ವಿಧವಾದ ದೌರ್ಬಲ್ಯಕ್ಕೆ ಪಾಪವೆಂದು ಹೆಸರು. ಈ ದೌರ್ಬಲ್ಯದಿಂದಲೆ ಹಿಂಸಾದೋಷಾದಿಗಳು ಹುಟ್ಟುವುವು. ಈ ದೌರ್ಬಲ್ಯದ ಹೆಸರೇ ಪಾಪ. ಒಳಗೆ ಆತ್ಮ ಸರ್ವದಾ ಧಗಧಗನೆ ಪ್ರಜ್ವಲಿಸುತ್ತಿದೆ. ಆ ಕಡೆಗೆ ದೃಷ್ಟಿ ಕೊಡದೆ ಈ ವಿಕಾರ ಸ್ವರೂಪಿನ ಅಸ್ಥಿ ಮಾಂಸಪಂಜರವಾದ ಜಡಶರೀರದ ಕಡೆಗೇ ಎಲ್ಲರೂ ದೃಷ್ಟಿ ಕೊಟ್ಟು ‘ನಾನು, ನಾನು’ ಎನ್ನುವರು. ಇದೇ ಸಕಲ ವಿಧವಾದ ದೌರ್ಬಲ್ಯಕ್ಕೂ ಮೂಲ. ಈ ಅಭ್ಯಾಸದಿಂದಲೆ ಜಗತ್ತಿನಲ್ಲಿ ವ್ಯಾವಹಾರಿಕ ಭಾವವುಂಟಾಗಿದೆ. ಪಾರಮಾರ್ಥಿಕ ಭಾವವು ಈ ದ್ವಂದ್ವದ ಹೊರಗೆ ಇರುವುದು.

ಶಿಷ್ಯ: ಹಾಗಾದರೆ ಈ ವ್ಯಾವಹಾರಿಕ ಸತ್ತೆ (ಸ್ಥಿತಿ)ಯೆಲ್ಲಾ ಸತ್ಯವಾದುದ್ದಲ್ಲವೆ?

ಸ್ವಾಮೀಜಿ: ಎಷ್ಟು ಹೊತ್ತು ‘ಅಹಂ’ನ ಜ್ಞಾನವಿರುತ್ತದೆಯೊ ಅಷ್ಟು ಹೊತ್ತು ಸತ್ಯ. ಮತ್ತೆ ಯಾವಾಗ ನಾನು ‘ಆತ್ಮ’ ಎಂಬ ಅನುಭವವಿರುತ್ತದೆಯೋ ಆಗ ಅಸತ್ಯ. ಜನರು ಯಾವುದನ್ನು ಪಾಪ ಪಾಪವೆಂದು ಹೇಳುತ್ತಾರೆಯೋ ಅದು ದೌರ್ಬಲ್ಯದ ಫಲ - ‘ನಾನು ದೇಹ’ ಎಂಬುವುದು ಅಹಂಭಾವದ ರೂಪಾಂತರ. ಯಾವಾಗ ನಾನು ಆತ್ಮ ಎಂಬ ಭಾವದಲ್ಲಿ ಮನಸ್ಸು ನಿಶ್ಚಲವಾಗುತ್ತದೆಯೋ, ಆಗ ನೀನು ಪಾಪ ಪುಣ್ಯ ಧರ್ಮಾಧರ್ಮಗಳನ್ನು ದಾಟಿ ಹೋಗುವೆ. ಪರಮಹಂಸರು, ‘ನಾನು’ ಸತ್ತರೆ ತೊಂದರೆಯೆಲ್ಲಾ ಪರಿಹಾರವಾಗುತ್ತದೆ ಎಂದು ಹೇಳುತ್ತಿದ್ದರು.

ಶಿಷ್ಯ: ಮಹಾಶಯರೆ ‘ಅಹಂ’ ಎನ್ನುವುದನ್ನು ಸಾಯಿಸಿದರೂ ಸಾಯುವುದಿಲ್ಲ. ಅದನ್ನು ಸಾಯಿಸುವುದು ತುಂಬ ಕಷ್ಟ

ಸ್ವಾಮೀಜಿ: ಒಂದು ವಿಧದಲ್ಲಿ ತುಂಬ ಕಷ್ಟ, ಮತ್ತೊಂದು ವಿಧದಲ್ಲಿ ತುಂಬ ಸುಲಭ. ‘ಅಹಂ’ ಪದಾರ್ಥ ಎಲ್ಲಿದೆ ಎಂದು ತಿಳಿದುಕೊಂಡಿದ್ದೀಯೆ? ಯಾವ ಪದಾರ್ಥ ಇಲ್ಲವೊ ಅದನ್ನು ಸಾಯಿಸುವುದು ಎಂದರೇನು? ‘ಅಹಂ’ ರೂಪದ ಒಂದು ಮಿಥ್ಯಾ ಭಾವದಲ್ಲಿ ಮನುಷ್ಯನು ಮಂತ್ರಮುಗ್ಧನಾಗಿದ್ದಾನೆ; ಅಷ್ಟೆ. ಈ ಭೂತ ಬಿಟ್ಟು ಹೋದರೆ ಸ್ವಪ್ನವೆಲ್ಲಾ ಕಳೆದುಹೋಗಿ ಒಂದು ಆತ್ಮವು ಆಬ್ರಹ್ಮಸ್ತಂಭ ಪರ್ಯಂತವಾಗಿ ಎಲ್ಲದರಲ್ಲಿಯೂ ಇರುವುದು ಗೊತ್ತಾಗುತ್ತದೆ. ಇದನ್ನೇ ತಿಳಿದುಕೊಳ್ಳಬೇಕು - ಪ್ರತ್ಯಕ್ಷ ಮಾಡಿಕೊಳ್ಳಬೇಕು. ಸಾಧನೆ ಭಜನೆ ಎಷ್ಟಿವೆಯೋ ಅವೆಲ್ಲಾ ಈ ಆವರಣವನ್ನು ಕತ್ತರಿಸಿಹಾಕುವುದಕ್ಕೋಸ್ಕರ. ಅದು ಹೋದರೆಯೆ ಚಿತ್ಸೂರ್ಯನು ತನ್ನ ಪ್ರಭೆಯಲ್ಲಿ ತಾನು ಪ್ರಜ್ವಲಿಸುತ್ತಿರುವುದನ್ನು ನೋಡಲು ಸಮರ್ಥನಾಗುವೆ. ಯಾವ ವಸ್ತು ಸ್ವಸಂವೇದ್ಯವೊ ಅದನ್ನು ಮತ್ತಾವುದಾದರೊಂದರ ಸಹಾಯದಿಂದ ತಿಳಿದುಕೊಳ್ಳುವುದು ಹೇಗೆ? ಶ್ರುತಿ ಅದನ್ನೇ ಹೇಳುವುದು - ‘ವಿಜ್ಞಾತಾರವರೇ ಕೇನ ವಿಜಾನೀಯಾತ್.’ - ಯಾವುದು ತಿಳಿಯುತ್ತದೆಯೋ ಅದನ್ನು ಯಾವುದರಿಂದ ತಿಳಿಯುವುದು? ಏನು ಅಲ್ಪ ಸ್ವಲ್ಪ ತಿಳಿದುಕೊಂಡಿದ್ದೀಯೊ ಅದು ಮನೋರೂಪವಾದ ಕರಣದ ಸಹಾಯದಿಂದ. ಮನಸ್ಸು ಜಡ; ಅದರ ಹಿಂದೆ ಶುದ್ಧ ಆತ್ಮ ಇರುವುದರಿಂದಲೆ ಮನಸ್ಸಿನ ಮೂಲಕ ಕಾರ್ಯ ನಡೆಯುತ್ತದೆ. ಆದ್ದರಿಂದ ಮನಸ್ಸಿನ ಮೂಲಕ ಆ ಆತ್ಮವನ್ನು ಹೇಗೆ ತಿಳಿಯುವಿ? ಆದರೆ ಇಷ್ಟು ಮಾತ್ರ ತಿಳಿಯುತ್ತದೆ; ಏನೆಂದರೆ, ಮನಸ್ಸು ಶುದ್ಧ ಆತ್ಮದ ಹತ್ತಿರ ಹೋಗಲಾರದು, ಬುದ್ಧಿಯೂ ಹೋಗಲಾರದು. ತಿಳಿದುಕೊಳ್ಳುವುದು ಇಲ್ಲಿಯವರೆಗೆ ಮಾತ್ರ. ಅದರ ಮೇಲೆ, ಯಾವಾಗ ಮನಸ್ಸು ನಿರ್ವಿಕಲ್ಪ ಅಥವಾ ವೃತ್ತಿಹೀನವಾಗುತ್ತದೆಯೋ ಆಗಲೆ ಮನಸ್ಸು ಲುಪ್ತವಾಗುತ್ತದೆ, ಆಗಲೆ ಆತ್ಮ ಪ್ರತ್ಯಕ್ಷವಾಗುವುದು. ಈ ಸ್ಥಿತಿಯನ್ನೇ ಭಾಷ್ಯಕಾರರಾದ ಶಂಕರಾಚಾರ್ಯರು ‘ಅಪರೋಕ್ಷಾನುಭೂತಿ’ಯೆಂದು ವರ್ಣಿಸಿದ್ದಾರೆ.

ಶಿಷ್ಯ: ಆದರೆ, ಮಹಾಶಯರೆ, ಮನಸ್ಸೆ ‘ಅಹಂ’. ಆ ಮನಸ್ಸು ಲುಪ್ತವಾದರೆ ಆಮೇಲೆ ‘ಅಹಂ’ ಎಂಬುದೂ ಇರುವುದಿಲ್ಲ.

ಸ್ವಾಮೀಜಿ: ಆಗ ಯಾವ ಸ್ಥಿತಿ ಇರುತ್ತದೆಯೋ ಅದೆ ನಿಜವಾದ ‘ಅಹಂ’ ಸ್ವರೂಪ. ಆಗ ಯಾವ ಅಹಂ ಇರುತ್ತದೆಯೋ ಅದು ಸರ್ವಭೂತಸ್ಥ ಸರ್ವಗತ-ಸರ್ವಾಂತರಾತ್ಮಾ. ಘಟಾಕಾಶ ಹೋಗಿ ಮಹಾಕಾಶವಿದ್ದ ಹಾಗೆ - ಘಟ ಒಡೆದು ಹೋದರೆ ಅದರ ಒಳಗಿರುವ ಆಕಾಶವೂ ನಾಶವಾಗಿ ಬಿಡುವುದೆ? ಯಾವ ಕ್ಷುದ್ರ ಅಹಂಕಾರವನ್ನು ನೀನು ದೇಹಬದ್ಧವೆಂದು ತಿಳಿದುಕೊಂಡಿದ್ದೀಯೋ ಅದೇ ವಿಸ್ತೃತವಾಗಿ ಹೀಗೆ ಸರ್ವಗತ ಅಹಂಕಾರ ಅಥವಾ ಆತ್ಮರೂಪದಲ್ಲಿ ಪ್ರತ್ಯಕ್ಷವಾಗುತ್ತದೆ. ಆದ್ದರಿಂದ ಮನಸ್ಸು ಇದ್ದರೆ ಅಥವಾ ಹೋದರೆ ಅದರಿಂದ ಯಥಾರ್ಥವಾದ ‘ಅಹಂ’ ಅಥವಾ ಆತ್ಮಕ್ಕೆ ಆಗುವುದೇನು?

ನಾನು ಹೇಳುವುದು ಸಕಾಲದಲ್ಲಿ ಪ್ರತ್ಯಕ್ಷವಾಗುವುದು - ‘ಕಾಲೇನಾತ್ಮನಿ ವಿಂದತಿ.’ ಶ್ರವಣಮನನಗಳನ್ನು ಮಾಡುತ್ತ ಮಾಡುತ್ತ ಇದ್ದರೆ ಕಾಲಕ್ರಮೇಣ ಈ ವಿಷಯ ತಾನೇ ಅರ್ಥವಾಗುವುದು - ಮತ್ತು ನೀನು ಮನಸ್ಸಿನ ಆಚೆಗೆ ಹೋಗುವೆ. ಆಗ ಈ ಪ್ರಶ್ನೆಯನ್ನು ಕೇಳುವ ಆವಶ್ಯಕತೆ ಇರುವುದಿಲ್ಲ.

ಶಿಷ್ಯನು ಇದನ್ನು ಕೇಳಿ ಮೌನವಾಗಿ ಕುಳಿತುಕೊಂಡಿದ್ದನು. ಸ್ವಾಮಿಜಿ ಮೆಲ್ಲಗೆ ಧೂಮಪಾನ ಮಾಡುತ್ತ ಮಾಡುತ್ತ ಪುನಃ ಹೀಗೆಂದರು: “ಈ ಸಹಜವಾದ ವಿಷಯವನ್ನು ತಿಳಿಸುವುದಕ್ಕೆ ಎಷ್ಟೆಷ್ಟು ಶಾಸ್ತ್ರಗಳು ಬರೆಯಲ್ಪಟ್ಟಿವೆ, ಆದರೂ ಜನರು ತಿಳಿದುಕೊಳ್ಳಲಾರರು! ಮೇಲೆ ಮೇಲೆ ಮಧುರವಾಗಿರುವ ಕೆಲವು ಬೆಳ್ಳಿಯ ನಾಣ್ಯಗಳನ್ನು ಹೆಂಗಸರ ಕ್ಷಣಭಂಗುರವಾದ ರೂಪವನ್ನೂ ಕಟ್ಟಿಕೊಂಡು ಈ ದುರ್ಲಭವಾದ ಮಾನುಷ ಜನ್ಮವನ್ನು ಹೇಗೆ ಕಳೆದುಬಿಡುತ್ತಾರೆ! ಮಹಾಮಾಯೆಯ ಆಶ್ಚರ್ಯಕರವಾದ ಪ್ರಭಾವ! ಅಮ್ಮಮ್ಮ!"

\newpage

\chapter[ಅಧ್ಯಾಯ ೭]{ಅಧ್ಯಾಯ ೭\protect\footnote{\engfoot{C.W, Vol. VI, P. 476}}}

\begin{center}
ಸ್ಥಳ: ಬಾಗ್‌ಬಜಾರ್, ಕಲ್ಕತ್ತ, ವರ್ಷ: ೧೮೯೭.
\end{center}

ಸ್ವಾಮೀಜಿ ಕೆಲವು ದಿನಗಳಿಂದ ಬಾಗಬಜಾರಿನ ಬಲರಾಮ ಬಾಬುಗಳ ಮನೆಯಲ್ಲಿ ಇದ್ದರು. ಇಂದು ಅವರು ಪರಮಹಂಸರ ಗೃಹಸ್ಥಭಕ್ತರು ಒಂದು ಕಡೆ ಸೇರುವಂತೆ ಹೇಳಿ ಕಳುಹಿಸಿದ್ದರಿಂದ ಮೂರು ಗಂಟೆಯ ಮೇಲೆ ಸಾಯಂಕಾಲದ ಹೊತ್ತಿಗೆ ಪರಮಹಂಸರ ಬಹು ಭಕ್ತರು ಈ ಮನೆಯಲ್ಲಿ ಸೇರಿದ್ದಾರೆ. ಯೋಗಾನಂದರೂ ಅಲ್ಲಿದ್ದರು. ಸ್ವಾಮಿಜಿ ಹೀಗೆಂದರು:

“ನಾನಾ ದೇಶಗಳನ್ನು ಸುತ್ತಿ ನನಗೆ ದೃಢವಾಗಿ ನಂಬುಗೆ ಹುಟ್ಟಿಹೋಗಿದೆ, ಏನೆಂದರೆ - ಸಂಘವಿಲ್ಲದೆ ಯಾವ ದೊಡ್ಡ ಕಾರ್ಯವೂ ನಡೆಯಲಾರದು. ಆದರೆ ನಮ್ಮದರಂಥ ದೇಶಗಳಲ್ಲಿ ಮೊದಲಿನಿಂದಲೂ ಸಾಧಾರಣ ರೀತಿಯಲ್ಲಿ ಸಂಘವನ್ನು ಏರ್ಪಡಿಸುವುದು ಅಥವಾ ಸಾಧಾರಣ ಜನರ ಸಮ್ಮತಿಯನ್ನು ತೆಗೆದುಕೊಂಡು ಕೆಲಸ ಮಾಡುವುದು ಅಷ್ಟು ಸುಲಭವೆಂದು ತೋರುವುದಿಲ್ಲ. ಪಾಶ್ಚಾತ್ಯ ನರನಾರಿಯರು ಸುಶಿಕ್ಷಿತರಾಗಿದ್ದಾರೆ. ನಮ್ಮ ಹಾಗೆ ದ್ವೇಷಪರಾಯಣರಲ್ಲ; ಅವರು ಗುಣಕ್ಕೆ ಗೌರವ ಕೊಡುವುದನ್ನು ಕಲಿತುಕೊಂಡಿದ್ದಾರೆ. ಇದನ್ನು ಆ ದೇಶದಲ್ಲಿ ಎಷ್ಟು ಆದರಿಸುತ್ತಾರೆ! ನನ್ನ ಉದಾಹರಣೆಯನ್ನೇ ತೆಗೆದುಕೊಳ್ಳಿ. ನನ್ನಂತಹ ಯಃಕಶ್ಚಿತ್ ವ್ಯಕ್ತಿಯನ್ನು ಎಷ್ಟೊಂದು ಆದರದಿಂದ ಅವರು ಬರಮಾಡಿಕೊಂಡು ನನಗೆ ಎಷ್ಟೊಂದು ಅನುಕೂಲಗಳನ್ನು ಕಲ್ಪಿಸಿಕೊಟ್ಟರು! ಈ ದೇಶದಲ್ಲಿ ಶಿಕ್ಷಣ ಪ್ರಸರಿಸಿ, ಅದರಿಂದ ಯಾವಾಗ ಮತಗಿತಗಳ ಕಿಷ್ಕಂಧವಾದ ಮೇರೆಗಳ ಹೊರಗೆ ಮನಸ್ಸು ಕೊಡುವುದನ್ನು ಕಲಿಯುತ್ತಾರೆಯೋ, ಆಗ ಪ್ರಜಾಭಿಪ್ರಾಯವನ್ನು ಅನುಸರಿಸಿ ಸಂಘದ ಕೆಲಸವು ನಡೆಯಬಲ್ಲದು. ಅದಕೋಸ್ಕರವೇ ಈ ಸಂಘಕ್ಕೆ ಒಬ್ಬ ಪ್ರಧಾನ ಸರ್ವಾಧಿಕಾರಿ ಇರಬೇಕು. ಎಲ್ಲರೂ ಆತನ ಆಜ್ಞೆಗೆ ಗೌರವ ಕೊಟ್ಟುಕೊಂಡು ಹೋಗಬೇಕು. ಆಮೇಲೆ ಕ್ರಮೇಣ ಎಲ್ಲರ ಮತವನ್ನೂ ತಿಳಿದುಕೊಂಡು ಕಾರ್ಯವನ್ನು ಮಾಡಬಹುದು. ನಾನು ಯಾರ ಹೆಸರಿನಲ್ಲಿ ಸಂನ್ಯಾಸಿಯಾಗಿದ್ದೇನೆಯೊ, ತಾವು ಯಾರನ್ನು ಜೀವನಕ್ಕೆ ಆದರ್ಶವನ್ನಾಗಿಟ್ಟುಕೊಂಡು ಸಂಸಾರಾಶ್ರಮವನ್ನು ಸ್ವೀಕರಿಸಿ ಕಾರ್ಯಕ್ಷೇತ್ರ ದಲ್ಲಿರುವಿರೋ, ಯಾರ ಪುಣ್ಯನಾಮವೂ ಅದ್ಭುತ ಜೀವನವೂ ದೇಹಾವಸಾನವಾದ ಇಪ್ಪತ್ತು ವರ್ಷಗಳ ಒಳಗೆ ಪ್ರಾಚ್ಯ ಮತ್ತು ಪಾಶ್ಚಾತ್ಯ ಜಗತ್ತಿನಲ್ಲಿ ಆಶ್ಚರ್ಯಕರವಾದ ರೀತಿಯಲ್ಲಿ ಪ್ರಸಾರವಾಯಿತೋ ಆತನ ಹೆಸರಿನಲ್ಲಿಯೆ ಈ ಸಂಘ ಪ್ರತಿಷ್ಠಿತವಾಗುವುದಾಗಿದೆ. ನಾನು ಪ್ರಭುವಿನ ದಾಸ, ತಾವು ಈ ಕಾರ್ಯದಲ್ಲಿ ಸಹಾಯ ಮಾಡಬೇಕು."

ಶ‍್ರೀಯುತ ಗಿರೀಶಚಂದ್ರ ಘೋಷರೆ ಮುಂತಾಗಿ ಅಲ್ಲಿ ಬಂದಿದ್ದ ಗೃಹಸ್ಥರು ಈ ಪ್ರಸ್ತಾಪವನ್ನು ಅನುಮೋದಿಸಲು, ರಾಮಕೃಷ್ಣ ಸಂಘದ ಮುಂದಿನ ಕಾರ್ಯಕ್ರಮ ಚರ್ಚೆಗೆ ಬಂತು. ಸಂಘಕ್ಕೆ ರಾಮಕೃಷ್ಣ ಪ್ರಚಾರ ಸಂಸ್ಥೆ ಅಥವಾ ರಾಮಕೃಷ್ಣ ಮಿಷನ್ ಎಂದು ಹೆಸರಿಟ್ಟಿದ್ದಾಯಿತು. ಅದರ ಉದ್ದೇಶ ಮುಂತಾದುವನ್ನು ಅದರ ಮುದ್ರಿತ ವಿಜ್ಞಾಪನ ಪತ್ರಿಕೆಯಿಂದ ತೆಗೆದು ಇಲ್ಲಿ ಕೊಟ್ಟಿದ್ದೇವೆ.

\textbf{ಉದ್ದೇಶ:} ಮಾನವಹಿತಾರ್ಥವಾಗಿ ಶ‍್ರೀರಾಮಕೃಷ್ಣ ಪರಮಹಂಸರು ಯಾವ ತತ್ತ್ವಗಳನ್ನು ವಿವರಿಸಿದರೆ ಮತ್ತು ತಮ್ಮ ಜೀವನದಲ್ಲಿ ಕಾರ್ಯರೂಪವಾಗಿ ಉದಾಹರಿಸಿದರೆ ಅವುಗಳ ಪ್ರಚಾರವೂ, ಮತ್ತು ಈ ತತ್ತ್ವಗಳನ್ನು ಅನುಭವಕ್ಕೆ ತಂದುಕೊಳ್ಳಲು ಸಹಕಾರಿಗಳಾದ ಮನುಷ್ಯನ ದೈಹಿಕ ಮಾನಸಿಕ ಮತ್ತು ಪಾರಮಾರ್ಥಿಕ ಉನ್ನತಿಗಳಲ್ಲಿ ಸಹಾಯ ಮಾಡುವುದೂ ಈ ಮಿಷನ್ನಿನ ಉದ್ದೇಶ.

\textbf{ವ್ರತ:} ಜಗತ್ತಿನ ಎಲ್ಲಾ ಧರ್ಮಮತಗಳೂ ಒಂದು ಅಕ್ಷಯ ಸನಾತನ ಧರ್ಮದ ರೂಪಾಂತರ ಮಾತ್ರ ಎಂಬ ಜ್ಞಾನದಿಂದ ಸಕಲ ಧರ್ಮಾನುಯಾಯಿಗಳಲ್ಲಿಯೂ ಆತ್ಮೀಯತೆಯನ್ನು ಇರಿಸುವುದಕ್ಕೋಸ್ಕರ ಶ‍್ರೀರಾಮಕೃಷ್ಣ ಪರಮಹಂಸರು ಯಾವ ಕಾರ್ಯವನ್ನು ಆರಂಭಿಸಿದರೆ ಅದರ ಪರಿಚಾಲನವೆ ಈ ಮಿಷನ್ನಿನ ವ್ರತ.

\textbf{ಕಾರ್ಯಕ್ರಮ:} ಮನುಷ್ಯನ ಸಾಂಸಾರಿಕ ಮತ್ತು ಆಧ್ಯಾತ್ಮಿಕ ಉನ್ನತಿಗೋಸ್ಕರ ವಿದ್ಯೆಯನ್ನು ದಾನ ಮಾಡುವುದಕ್ಕೆ ಉಪಯೋಗವಾಗುವಂತೆ ಜನರನ್ನು ಶಿಕ್ಷಣ ಕೊಟ್ಟು ತಯಾರುಮಾಡುವುದು. ಕುಶಲ ಕಲೆಗಳಲ್ಲಿಯೂ, ಕಷ್ಟಪಟ್ಟು ಕೆಲಸ ಮಾಡಬೇಕಾದ ಕಸುಬುಗಳಲ್ಲಿಯೂ ಉತ್ಸಾಹವನ್ನು ಹೆಚ್ಚಿಸುವುದು, ಮತ್ತು ಇತರ ಧರ್ಮಭಾವವು ಶ‍್ರೀರಾಮಕೃಷ್ಣರ ಜೀವನದಲ್ಲಿ ಹೇಗೆ ವಿವರಿಸಲ್ಪಟ್ಟುವೊ ಅವುಗಳನ್ನು ಜನಸಮಾಜದಲ್ಲಿ ಪ್ರವರ್ತಿಸುವಂತೆ ಮಾಡುವುದು.

\textbf{ಭರತಖಂಡದಲ್ಲಿನ ಕಾರ್ಯ:} ಭರತಖಂಡದ ಪ್ರತಿನಗರದಲ್ಲಿಯೂ ಬೋಧಕರಾಗಲು ಇಚ್ಛಿಸುವ ಗೃಹಸ್ಥ ಮತ್ತು ಸಂನ್ಯಾಸಿಗಳಿಗೆ ಶಿಕ್ಷಣ ಕೊಡುವುದಕ್ಕೋಸ್ಕರ ಆಶ್ರಮವನ್ನು ಸ್ಥಾಪಿಸುವುದು ಮತ್ತು ಅವರು ಯಾವ ರೀತಿಯಲ್ಲಿ ದೇಶ ದೇಶಾಂತರಗಳಿಗೆ ಹೋಗಿ ಜನಗಳನ್ನು ಶಿಕ್ಷಿತರನ್ನಾಗಿ ಮಾಡಲು ಸಾಧ್ಯವೋ ಅದನ್ನು ಅನುಸರಿಸುವುದು.

\textbf{ವಿದೇಶೀಯ ಕಾರ್ಯವಿಭಾಗ:} ಭರತಖಂಡದ ಹೊರಗೆ ಇರುವ ಪ್ರದೇಶಗಳಲ್ಲಿ (ಜನರು) ‘ವ್ರತಧಾರಿ’ಗಳಾಗುವಂತೆ ಪ್ರೇರೇಪಿಸುವುದು ಮತ್ತು ಆಯಾ ದೇಶಗಳಲ್ಲಿ ಸ್ಥಾಪಿಸಲ್ಪಟ್ಟ ಆಶ್ರಮಗಳೊಡನೆ ಭಾರತೀಯ ಆಶ್ರಮಗಳಿಗೆ ನಿಕಟ ಸಂಬಂಧವೂ ಸಹಾನುಭೂತಿಯೂ ಬೆಳೆಯುವಂತೆ ಮಾಡುವುದು ಮತ್ತು ಹೊಸ ಆಶ್ರಮಗಳನ್ನು ಸ್ಥಾಪಿಸುವುದು

ಸ್ವಾಮೀಜಿ ತಾವೇ ಮೇಲೆ ಹೇಳಿದ ಸಮಿತಿಯ ಗೌರವಾಧ್ಯಕ್ಷರಾದರು. ಬ್ರಹ್ಮಾನಂದ ಸ್ವಾಮಿಗಳು ಕಲ್ಕತ್ತಾ ಕೇಂದ್ರಕ್ಕೆ ಅಧ್ಯಕ್ಷರೂ ಮತ್ತು ಯೋಗಾನಂದ ಸ್ವಾಮಿಗಳು ಅವರ ಸಹಕಾರಿಗಳೂ ಆದರು. ಅಟಾರ್ನಿ ಬಾಬು ನರೇಂದ್ರನಾಥ ಮಿತ್ರರು ಇದರ ಕಾರ್ಯದರ್ಶಿಗಳಾಗಿಯೂ ಡಾಕ್ಟರ್ ಶಶಿಭೂಷಣ ಘೋಷರು ಮತ್ತು ಶರಚ್ಚಂದ್ರ ಸರಕಾರರು ಉಪಕಾರದರ್ಶಿಗಳಾಗಿಯೂ ಶಿಷ್ಯನು ಶಾಸ್ತ್ರಪಾಠಕನಾಗಿಯೂ ಚುನಾಯಿಸಲ್ಪಟ್ಟರು. ಇದರ ಜೊತೆಯಲ್ಲಿ ಪ್ರತಿ ಭಾನುವಾರವೂ ನಾಲ್ಕು ಗಂಟೆಯ ಮೇಲೆ ಬಲರಾಮಬಾಬುಗಳ ಮನೆಯಲ್ಲಿ ಸಮಿತಿ ಸೇರಬೇಕೆಂಬ ಮತ್ತೊಂದು ನಿಯಮವೂ ಮಾಡಲ್ಪಟ್ಟಿತು. ಹಿಂದೆ ಹೇಳಿದ ಸಭೆಯು ಏರ್ಪಾಡಾದ ನಂತರ ಮೂರು ವರ್ಷಗಳ ಕಾಲ ಬಲರಾಮವಸು ಅವರ ಮನೆಯಲ್ಲಿ ಸಮಿತಿಯ ಅಧಿವೇಶನವು ನಡೆಯಿತು. ಮತ್ತೆ ವಿಲಾಯತಿಗೆ ಹೋಗುವತನಕ ಸ್ವಾಮೀಜಿ ಅನುಕೂಲವಾದಾಗ ಸಮಿತಿಯ ಕೂಟಕ್ಕೆ ಬರುತ್ತ ಆಗಾಗ್ಗೆ ಉಪದೇಶ ಮಾಡುತ್ತಲೂ ಅಥವಾ ಕಿನ್ನರ ಕಂಠದಲ್ಲಿ ಗಾನ ಮಾಡುತ್ತ ಶೋತೃಗಳನ್ನು ಮೋಹಿತರನ್ನಾಗಿ ಮಾಡುತ್ತಲೂ ಇದ್ದರೆಂದು ಬೇರೆಯಾಗಿ ಹೇಳಬೇಕಾಗಿಲ್ಲ. ಸಭೆ ಮುಗಿದ ಕೂಡಲೆ ಸಭಿಕರೆಲ್ಲರೂ ಹೊರಟು ಹೋಗಲು ಸ್ವಾಮಿಜಿ ಯೋಗಾನಂದಸ್ವಾಮಿಗಳನ್ನು ಕುರಿತು “ಕಾರ್ಯವೇನೊ ಹೀಗೆ ಆರಂಭವಾಗಿ ಹೋಯಿತು; ಈಗ ನೋಡು, ಪರಮಹಂಸರ ದಯೆಯಿಂದ ಕಾರ್ಯವು ಎಷ್ಟು ದೂರ ಮುಂದುವರಿಯುವುದು" ಎಂದು ಹೇಳಿದರು.

ಸ್ವಾಮಿ ಯೋಗಾನಂದ: ನಿಮ್ಮ ಈ ಏರ್ಪಾಡು ಎಲ್ಲಾ ವಿದೇಶೀಯ ರೀತಿಯಲ್ಲಿ ಮಾಡಲ್ಪಟ್ಟಿದೆ. ಪರಮಹಂಸರ ಉಪದೇಶವೇನು ಈ ರೂಪವಾಗಿತ್ತೇ?

ಸಾಮಿಜಿ: ಇದೆಲ್ಲಾ ಪರಮಹಂಸರ ಅಭಿಪ್ರಾಯವಾಗಿರಲಿಲ್ಲವೆಂದು ನೀನು ಹೇಗೆ ತಿಳಿದುಕೊಂಡೆ? ಅನಂತಭಾವಮಯರಾದ ಪರಮಹಂಸರನ್ನು ನೀವು ನಿಮ್ಮ ಎಲ್ಲೆಗಳಲ್ಲಿ ಕಟ್ಟಿಹಾಕಲಿಚ್ಚಿಸುವಿರೆಂದು ತೋರುತ್ತದೆ. ನಾನು ಈ ಎಲ್ಲೆಯನ್ನು ಒಡೆದುಹಾಕಿ ಅವರ ಭಾವ ಜಗತ್ತಿನಲ್ಲೆಲ್ಲಾ ವ್ಯಾಪಿಸುವಂತೆ ಮಾಡಿಬಿಡುತ್ತೇನೆ. ಪರಮಹಂಸರು ತಮ್ಮ ಪೂಜೆಯನ್ನು ಮಾಡಬೇಕೆಂದು ನನಗೆ ಯಾವಾಗಲೂ ಉಪದೇಶ ಕೊಡಲಿಲ್ಲ. ಸಾಧನೆ, ಭಜನೆ, ಧ್ಯಾನಧಾರಣ ಮತ್ತು ಇತರ ಉಚ್ಛಧರ್ಮಭಾವಗಳಿಗೆ ಸಂಬಂಧಪಟ್ಟ ಯಾವ ಯಾವ ಉಪದೇಶಗಳನ್ನು ಅವರು ಕೊಟ್ಟು ಹೋದರೊ ಅವುಗಳನ್ನೆ ಪಡೆದುಕೊಂಡು ಜೀವರಿಗೆ ಶಿಕ್ಷಣವನ್ನು ಕೊಡಬೇಕು. ಅನಂತ ಮತ ಅನಂತ ಪಥ. ಸಂಪ್ರದಾಯಗಳಿಂದ ತುಂಬಿಹೋಗಿರುವ ಜಗತ್ತಿನಲ್ಲಿ ಮತ್ತೊಂದು ನೂತನ ಸಂಪ್ರದಾಯವನ್ನು ಕಲ್ಪಿಸುವುದಕ್ಕೆ ಹೋಗುವುದು ನನ್ನ ಜನ್ಮದಲ್ಲಿಯೆ ಆಗಲಾರದು. ಪ್ರಭುದೇವನ ಪದತಲದಲ್ಲಿ ಆಶ್ರಯವನ್ನು ಪಡೆದು ನಾನು ಧನ್ಯನಾಗಿದ್ದೇನೆ. ಮೂರು ಜಗತ್ತಿನ ಜನರಿಗೂ ಅವರ ಭಾವಗಳನ್ನು ಕೊಡುವುದಕ್ಕೆ ನಾನು ಹುಟ್ಟಿರುವುದು.

ಯೋಗಾನಂದಸ್ವಾಮಿಗಳು ಈ ಮಾತಿಗೆ ಬದಲು ಹೇಳದೆ ಇದ್ದದ್ದರಿಂದ ಸ್ವಾಮಾಜಿ ಮತ್ತೂ ಹೇಳತೊಡಗಿದರು: “ಪ್ರಭುವಿನ ದಯೆಗೆ ನಿದರ್ಶನವನ್ನು ಮೇಲಿಂದ ಮೇಲೆ ಈ ಜೀವನದಲ್ಲಿ ಪಡೆದಿದ್ದೇನೆ. ಆತನು ಹಿಂದೆ ನಿಂತುಕೊಂಡೆ ಈ ಕೆಲಸವನ್ನೆಲ್ಲಾ ಮಾಡಿಸುತ್ತಾನೆ. ಯಾವಾಗ ಹಸಿವಿನಿಂದ ಬಳಲಿ ಬೇಸತ್ತು ಮರದ ಕೆಳಗೆ ಬಿದ್ದಿರುತ್ತಿದ್ದೆನೊ, ಯಾವಾಗ ಕೌಪೀನ ಹಾಕಿಕೊಳ್ಳುವುದಕ್ಕೂ ಬಟ್ಟೆಯಿಲ್ಲದೆ ಇದ್ದೆನೊ, ಯಾವಾಗ ಒಂದು ಕುರುಡು ಕವಡೆಯೂ ಇಲ್ಲದೆ ಭೂಸಂಚಾರ ಮಾಡಬೇಕೆಂದು ನಿಶ್ಚಯಿಸಿಕೊಂಡೆನೋ ಆವಾಗಲೂ ಪರಮಹಂಸರ ದಯೆಯಿಂದ ಎಲ್ಲಾ ವಿಧದಲ್ಲಿಯೂ ಸಹಾಯವನ್ನು ಪಡೆದಿದ್ದೇನೆ, ಮತ್ತು ಯಾವಾಗ ಇದೇ ವಿವೇಕಾನಂದನ ದರ್ಶನಮಾಡಬೇಕೆಂದು ಚಿಕಾಗೊವಿನ ರಸ್ತೆಯಲ್ಲಿ ಬಡಿದಾಟಗಳಾಗುತ್ತಿದ್ದುವೊ, ಯಾವ ಆ ಸನ್ಮಾನದ ನೂರರಲ್ಲಿ ಒಂದು ಭಾಗವನ್ನು ಪಡೆದರೂ ಸಾಧಾರಣ ಮನುಷ್ಯನು ಉನ್ಮತ್ತನಾಗಿಬಿಡುವನೋ ಅಂಥ ಸನ್ಮಾನವನ್ನೂ ಆಗ ಪರಮಹಂಸರ ದಯೆಯಿಂದ ಸುಲಭವಾಗಿ ಅರಗಿಸಿಕೊಂಡಿದ್ದೇನೆ. ಪ್ರಭುವಿನ ಇಚ್ಛೆಯಿಂದ ಎಲ್ಲೆಲ್ಲಿಯೂ ವಿಜಯ. ಈಗ ಈ ದೇಶದಲ್ಲಿ ಸ್ವಲ್ಪ ಕಾರ್ಯವನ್ನು ಮಾಡುವುದಕ್ಕೆ ಹೊರಟಿದ್ದೇನೆ; ನೀವು ಸಂಶಯವನ್ನು ಬಿಟ್ಟು ನನ್ನ ಕಾರ್ಯದಲ್ಲಿ ಸಹಾಯಮಾಡಿ; ಅವರ ಇಚ್ಛೆಯಿಂದ ಎಲ್ಲಾ ಕೈಗೂಡುತ್ತದೆಂಬುದನ್ನು ನೀವೇ ನೋಡುವಿರಿ.”

ಯೋಗಾನಂದ: ನೀವು ಯಾವುದನ್ನು ಮನಸ್ಸಿನಲ್ಲಿ ತರುತ್ತೀರೋ ಅದೇ ಆಗುತ್ತದೆ. ನಾವಾದರೊ ಚಿರಕಾಲ ನಿಮ್ಮ ಆಜ್ಞಾನುವರ್ತಿಗಳು, ಪರಮಹಂಸರು ನಿಮ್ಮ ಮೂಲಕ ಇದೆಲ್ಲವನ್ನೂ ಮಾಡಿಸುತ್ತಾರೆಂಬುದನ್ನು ಮಧ್ಯೆ ಮಧ್ಯೆ ನಾನು ಚೆನ್ನಾಗಿ ಕಂಡುಕೊಂಡಿದ್ದೇನೆ. ಆದರೆ ನಿಜ ಹೇಳಬೇಕೆಂದರೆ, ಪರಮಹಂಸರ ಕೆಲಸ ಕಾರ್ಯಗಳು ಬೇರೆ ವಿಧವಾಗಿದ್ದುದನ್ನು ನೋಡಿರಲಿಲ್ಲವೆ ಎಂಬುದಾಗಿ ಮಧ್ಯೆ ಮಧ್ಯೆ ಎಂಥ ಸಂಶಯವು ಬಂದುಬಿಡುತ್ತದೆ! ಅದಕ್ಕೋಸ್ಕರವೆ ನಾವು ಅವರ ಉಪದೇಶವನ್ನು ಬಿಟ್ಟು ಬೇರೆ ಮಾರ್ಗದಲ್ಲಿ ಹೋಗುತ್ತಿದ್ದೇವೆಯೇನೊ ಎನ್ನಿಸುತ್ತದೆ. ಆದ್ದರಿಂದ ನಿಮಗೆ ಪ್ರತಿಯಾಗಿ ಹೇಳಿ ಎಚ್ಚರಿಕೆ ಕೊಟ್ಟೆ.

ಸ್ವಾಮೀಜಿ: ನೀನು ಏನು ತಿಳಿದುಕೊಂಡಿದ್ದೀಯೆ? ಸಾಧಾರಣ ಭಕ್ತರು ಪರಮಹಂಸರನ್ನು ಎಷ್ಟು ತಿಳಿದುಕೊಂಡಿದ್ದಾರೆಯೊ ಅವರು ನಿಜವಾಗಿ ಇರುವುದು ಅಷ್ಟೇ ಅಲ್ಲ, ಅವರು ಅನಂತಭಾವಮಯರಾದವರು. ಬ್ರಹ್ಮಜ್ಞಾನಕ್ಕಾದರೂ ಎಲ್ಲೆಯುಂಟು, ಪ್ರಭುವಿನ ದುರ್ಜ್ಞೇಯವಾದ ಭಾವಕ್ಕೆ ಎಲ್ಲೆಯಿಲ್ಲ. ಅವರ ಕೃಪಾಕಟಾಕ್ಷದಿಂದ ಲಕ್ಷ ವಿವೇಕಾನಂದರು ಈಗ ತಯಾರಾಗಬಲ್ಲರು. ಆದರೆ ಅವರು ಹಾಗೆ ಮಾಡದೆ, ಬೇಕೆಂದು ನನ್ನನ್ನು ಉಪಕರಣ ರೂಪವಾಗಿ ಮಾಡಿಕೊಂಡು ನನ್ನ ಮೂಲಕ ಹೀಗೆ ಮಾಡಿಸುತ್ತಿದ್ದಾರೆ - ಅದಕ್ಕೆ ನಾನೇನು ಮಾಡಲಿ ಹೇಳು!

ಹೀಗೆಂದು ಹೇಳಿ ಸ್ವಾಮಿಜಿ ಬೇರೆ ಕೆಲಸಕ್ಕಾಗಿ ಮತ್ತೆಲ್ಲಿಗೊ ಹೋದರು. ಯೋಗಾನಂದ ಸ್ವಾಮಿಗಳು ಶಿಷ್ಯನನ್ನು ಕುರಿತು “ಆಹಾ! ವಿವೇಕಾನಂದ ಸ್ವಾಮಿಗಳ ನಂಬಿಕೆಯನ್ನು ಕೇಳಿದೆಯೊ? ಪರಮಹಂಸರ ಕೃಪಾಕಟಾಕ್ಷದಿಂದ ಲಕ್ಷ ವಿವೇಕಾನಂದರು ತಯಾರು ಆಗಬಲ್ಲರೆಂದು ಹೇಳಿದರಲ್ಲವೆ! ಏನು ಗುರುಭಕ್ತಿ! ನಮಗೆ ಅದರ ನೂರರಲ್ಲಿ ಒಂದು ಭಾಗ ಭಕ್ತಿ ಇದ್ದರೂ ಧನ್ಯರಾದೇವು" ಎಂದು ಹೇಳಿದರು.

ಶಿಷ್ಯ: ಮಹಾಶಯರೆ, ಸ್ವಾಮಿಗಳ ವಿಚಾರವಾಗಿ ಪರಮಹಂಸರು ಏನು ಹೇಳುತ್ತಿದ್ದರು?

ಯೋಗಾನಂದ: ಇಂಥ ಆಧಾರವು ಈ ಯುಗದಲ್ಲಿ ಜಗತ್ತಿನಲ್ಲಿ ಮತ್ಯಾವಾಗಲೂ ಬಂದಿರಲಿಲ್ಲ - ಎಂದು ಹೇಳುತ್ತಿದ್ದರು. ಕೆಲವು ವೇಳೆ ನರೇನನು ಪುರುಷ ತಾವು ಪ್ರಕೃತಿ - ನರೇನನು ತಮ್ಮ ಮಾವನ ಮನೆಯೆಂದು ಹೇಳುತ್ತಿದ್ದರು. ಕೆಲವು ವೇಳೆ ‘ಅಖಂಡವರ್ಗ’ ಎಂದು ಹೇಳುತ್ತಿದ್ದರು. ಮತ್ತೆ ಕೆಲವು ವೇಳೆ ‘ಅಖಂಡದ ಗೃಹದಲ್ಲಿ - ಎಲ್ಲಿ ದೇವದೇವಿಯರೆಲ್ಲಾ ಬ್ರಹ್ಮನಿಂದ ತಮ್ಮ ತಮ್ಮ ಅಸ್ತಿತ್ವವನ್ನು ಬೇರೆಯಾಗಿಟ್ಟುಕೊಳ್ಳಲು ಆಗುತ್ತಿರಲಿಲ್ಲವೋ, ಅದರಲ್ಲಿಯೇ ಲೀನರಾಗಿ ಹೋಗಿಬಿಡುತ್ತಿದ್ದರೂ, ಅಲ್ಲಿ ಏಳು ಜನ ಋಷಿಗಳು ತಮ್ಮ ತಮ್ಮ ಅಸ್ತಿತ್ವವನ್ನು ಬೇರೆ ಇಟ್ಟುಕೊಂಡು ಧ್ಯಾನದಲ್ಲಿ ನಿಮಗ್ನರಾಗಿದ್ದುದನ್ನು ನೋಡಿದೆ; ನರೇನನು ಅವರಲ್ಲಿ ಒಬ್ಬರ ಅಂಶಾವತಾರ’ ಎಂದು ಹೇಳುತ್ತಿದ್ದರು. ಮತ್ತೆ ಕೆಲವು ವೇಳೆ ‘ಜಗತ್ಪಾಲಕನಾದ ನಾರಾಯಣನು ನರ ಮತ್ತು ನಾರಾಯಣನೆಂಬ ಹೆಸರಿನಿಂದ ಯಾವ ಎರಡು ಋಷಿರೂಪಗಳನ್ನು ಧಾರಣಮಾಡಿ ಲೋಕಕಲ್ಯಾಣಕ್ಕೋಸ್ಕರ ತಪಸ್ಸು ಮಾಡಿದನೊ ನರೇನನು ಅದೇ ಆ ನರ ಋಷಿಯ ಅವತಾರ’ ಎಂದು ಹೇಳುತ್ತಿದ್ದರು. ಮತ್ತೆ ಕೆಲವು ವೇಳೆ ‘ಶುಕದೇವನ ಹಾಗೆ; ಮಾಯೆಯು ಅವನನ್ನು ಮುಟ್ಟಲಾರದೆ ಹೋಯಿತು’ ಎಂದು ಹೇಳುತ್ತಿದ್ದರು.

ಶಿಷ್ಯ: ಈ ಮಾತುಗಳು ನಿಜವೇ? ಅಥವಾ ಪರಮಹಂಸರ ಭಾವಮುಖದಲ್ಲಿ ಒಂದೊಂದು ಸಮಯದಲ್ಲಿ ಒಂದೊಂದು ವಿಧವಾಗಿ ಹೇಳುತ್ತಿದ್ದರೊ?

ಯೋಗಾನಂದ: ಅವರ ಮಾತೆಲ್ಲ ನಿಜ. ಅವರ ಶ‍್ರೀಮುಖದಲ್ಲಿ ಭ್ರಾಂತಿಯಿಂದಲೂ ಸುಳ್ಳು ಯಾವಾಗಲೂ ಬರುತ್ತಿರಲಿಲ್ಲ.

ಶಿಷ್ಯ: ಹಾಗಾದರೆ ಹೀಗೆ ಒಂದೊಂದು ಸಮಯದಲ್ಲಿ ಒಂದೊಂದು ವಿಧವಾಗಿ ಏಕೆ ಹೇಳಿದರು?

ಯೋಗಾನಂದ: ನಿನಗೆ ತಿಳಿದುಕೊಳ್ಳುವುದಕ್ಕಾಗಲಿಲ್ಲ. ಅವರು ನರೇನನನ್ನು ಇವರೆಲ್ಲರ ಸಮಷ್ಟಿಪ್ರಕಾಶವೆಂದು ಹೇಳುತ್ತಿದ್ದರು. ನರೇನನಲ್ಲಿ ಋಷಿಯ ವೇದಜ್ಞಾನ, ಶಂಕರನ ತ್ಯಾಗ, ಬುದ್ಧನ ಹೃದಯ, ಶುಕದೇವನ ಮಾಯಾರಾಹಿತ್ಯ ಮತ್ತು ಬ್ರಹ್ಮಜ್ಞಾನದ ಪೂರ್ಣವಿಕಾಸ – ಇವೆಲ್ಲವೂ ಒಟ್ಟಿಗೆ ಇರುವುದನ್ನು ನೋಡಲಾರದೆ ಹೋದೆಯೇನು? ಪರಮಹಂಸರು ಅದನ್ನೆ ಮಧ್ಯೆ ಮಧ್ಯೆ ಹೀಗೆ ವಿಧವಿಧವಾಗಿ ಹೇಳುತ್ತಿದ್ದರು. ಏನು ಹೇಳುತ್ತಿದ್ದರೊ ಅದೆಲ್ಲಾ ಸತ್ಯ.

ಶಿಷ್ಯನು ಕೇಳಿ ಮಾತು ಹೊರಡದೆ ಬೆರಗಾಗಿದ್ದನು. ಈ ಮಧ್ಯೆ ಸ್ವಾಮೀಜಿ ಹಿಂತಿರುಗಿಬಂದು ಶಿಷ್ಯನನ್ನು “ನಿಮ್ಮ ಆ ದೇಶದಲ್ಲಿ ಪರಮಹಂಸರ ಹೆಸರನ್ನು ವಿಶೇಷವಾಗಿ ಜನರು ಕೇಳಿಬಲ್ಲರೆ?" ಎಂದು ಪ್ರಶ್ನೆ ಮಾಡಿದರು.

ಶಿಷ್ಯ: ಮಹಾಶಯರೆ, ನಾಗಮಹಾಶಯರು ಒಬ್ಬರು ಮಾತ್ರ ಆ ದೇಶದಿಂದ ಪರಮಹಂಸರ ಹತ್ತಿರಕ್ಕೆ ಬಂದಿದ್ದರು. ಅವರಿಂದ ಕೇಳಿ ಈಗ ಅನೇಕರಿಗೆ ಪರಮಹಂಸರ ವಿಷಯವನ್ನು ತಿಳಿದುಕೊಳ್ಳುವುದಕ್ಕೆ ಕುತೂಹಲವುಂಟಾಗಿದೆ. ಆದರೆ ಪರಮಹಂಸರು ಈಶ್ವರನ ಅವತಾರವೆಂಬ ವಿಷಯವನ್ನು ಆ ದೇಶದ ಜನರು ಇನ್ನೂ ತಿಳಿದುಕೊಂಡಿಲ್ಲ. ಅದನ್ನು ಕೇಳಿದರೂ ಕೆಲವರು ನಂಬುವುದಿಲ್ಲ.

ಸ್ವಾಮೀಜಿ: ಆ ವಿಷಯವನ್ನು ನಂಬುವುದೇನು ಸುಲಭವಾದ ಕೆಲಸವೊ? ನಾನು ಅವರನ್ನು ಕೈಯಾರ ಪರೀಕ್ಷಿಸಿ ನೋಡಿದ್ದೇನೆ. ಅವರ ಬಾಯಿಯಿಂದಲೆ, ಈ ಬಗ್ಗೆ ಆಗಾಗ್ಗೆ ಕೇಳಿದ್ದೇನೆ, ಇಪ್ಪತ್ತುನಾಲ್ಕು ಗಂಟೆಯ ಹೊತ್ತು ಅವರ ಜೊತೆಯಲ್ಲಿ ಕುಳಿತು ನಿಂತು ನೋಡಿದ್ದೇನೆ, ಆದರೂ ಮಧ್ಯೆ ಮಧ್ಯೆ ನನಗೂ ಸಂದೇಹವು ಬರುತ್ತದೆ. ಹೀಗಿರಲು ಮತ್ತೊಬ್ಬರ ವಿಚಾರವಾಗಿ ಹೇಳಬೇಕಾದ್ದೇನು?

ಶಿಷ್ಯ: ಮಹಾಶಯರೇ, ಪರಮಹಂಸರು ತಾವು ಪೂರ್ಣಬ್ರಹ್ಮ ಭಗವಂತನೆಂಬ ವಿಷಯವನ್ನು ಅವರೇ ತಮ್ಮ ಸ್ವಂತ ಬಾಯಿಂದಲೆ ತಮಗೆ ಯಾವಾಗಲಾದರೂ ಹೇಳಿದ್ದಾರೆಯೇನು?

ಸ್ವಾಮೀಜಿ: ಎಷ್ಟೋ ಸಲ ಹೇಳಿದ್ದಾರೆ. ನಮ್ಮೆಲ್ಲರಿಗೂ ಹೇಳಿದ್ದಾರೆ. ಅವರು ಕಾಶಿಪುರದ ತೋಟದಲ್ಲಿ - ಇನ್ನೇನು ದೇಹ ಬಿದ್ದು ಹೋಗಬೇಕು ಎನ್ನುವಹಾಗೆ ಇದ್ದಾಗ ಒಂದು ದಿನ ಅವರ ಹಾಸಿಗೆ ಪಕ್ಕದಲ್ಲಿ ಕುಳಿತುಕೊಂಡು ‘ಈ ಸಮಯದಲ್ಲಿ “ನಾನು ದೇವರು" ಎಂದು ನೀವು ಹೇಳಬಲ್ಲಿರಾದರೆ ನೀವು ನಿಜವಾಗಿಯೂ ದೇವರು ಎಂದು ನಂಬುವೆನು’ ಎಂದು ಮನಸ್ಸಿನಲ್ಲಿಯೆ ಹೇಳಿಕೊಂಡೆನು. ಆವಾಗ ದೇಹ ಹೋಗುವುದಕ್ಕೆ ಎರಡು ದಿನ ಮಾತ್ರ ಮಿಕ್ಕಿತ್ತು. ಪರಮಹಂಸರು ಆಗ ಹಠಾತ್ತಾಗಿ ನನ್ನ ಕಡೆಗೆ ತಿರುಗಿಕೊಂಡು “ಯಾರು ರಾಮನೊ ಯಾರು ಕೃಷ್ಣನೊ - ಅವನೆ ಈಗ ಈ ಶರೀರದಲ್ಲಿ ರಾಮಕೃಷ್ಣ - ನಿಮ್ಮ ವೇದಾಂತ ದೃಷ್ಟಿಯಿಂದಲ್ಲ" ಎಂದು ಹೇಳಿಬಿಟ್ಟರು. ಇದನ್ನು ಕೇಳಿ ನನಗೆ ಮಾತು ಅಡಗಿಹೋಯಿತು. ಪ್ರಭುವಿನ ಶ‍್ರೀಮುಖದಲ್ಲಿ ಪದೇಪದೇ ಕೇಳಿದ್ದರೂ ನನಗೇ ಅಲ್ಲಿಯವರೆಗೆ ಪೂರ್ಣವಾದ ನಂಬುಗೆ ಬಂದಿತ್ತೊ ಇಲ್ಲವೊ; ಸಂದೇಹ, ನಿರಾಶೆಯಲ್ಲಿ ಮನಸ್ಸು ಮಧ್ಯೆ ಮಧ್ಯೆ ಅಲ್ಲಾಡುತ್ತಿತ್ತು. ಇತರರ ವಿಚಾರವನ್ನು ಮತ್ತೆ ಹೇಳುವುದೇನು? ನಮ್ಮ ಹಾಗೆ ದೇಹವುಳ್ಳ ಒಬ್ಬ ವ್ಯಕ್ತಿಯನ್ನು ದೇವರೆಂದು ಹೇಳುವುದೂ ನಂಬುವುದೂ ತುಂಬ ಕಷ್ಟವಾದ ಕೆಲಸ. ಸಿದ್ಧರು ಬ್ರಹ್ಮಜ್ಞರು - ಎಂದು ಹೇಳುವಷ್ಟು ನಮ್ಮ ಮನಸ್ಸು ಹೋಗುವುದು. ಅವರನ್ನು ಸಂತರೆಂದಾದರೂ ಕರೆಯಲಿ, ಬ್ರಹ್ಮಜ್ಞರೆಂದಾದರೂ ಕರೆಯಲಿ, ಅದು ಮುಖ್ಯವಲ್ಲ. ಆದರೆ ಪರಮಹಂಸರಂಥ ಪರಮ ಪರಿಪೂರ್ಣರು ಜಗತ್ತಿನಲ್ಲಿ ಈ ಹಿಂದೆ ಯಾವಾಗಲೂ ಬಂದಿರಲಿಲ್ಲ. ಸಂಸಾರದ ಘೋರಾಂಧಕಾರದಲ್ಲಿ ಈಗ ಈ ಮಹಾಪುರುಷರ ಜ್ಯೋತಿಸ್ತಂಭರೂಪ! ಇದರ ಬೆಳಕಿನಲ್ಲಿಯೆ ಈಗ ಮನುಷ್ಯನು ಸಂಸಾರ ಸಮುದ್ರದ ದಡವನ್ನು ಸೇರುವನು.

ಶಿಷ್ಯ: ಮಹಾಶಯರೆ, ಏನನ್ನಾದರೂ ಮಹಾತ್ಮ್ಯವನ್ನು ನೋಡದೆ ಕೇಳದೆ ಹೋದರೆ ನಿಜವಾದ ನಂಬುಗೆ ಹುಟ್ಟುವುದಿಲ್ಲವೆಂದು ನನಗೆ ತೋರುತ್ತದೆ. ಮಥುರ ಬಾಬುಗಳು ಪರಮಹಂಸರ ಸಂಬಂಧವಾಗಿ ಎಷ್ಟೆಷ್ಟನ್ನು ನೋಡಿದ್ದರೆಂದು ಕೇಳಿದ್ದೇನೆ. ಅದಕ್ಕೋಸ್ಕರವೆ ಆತನಿಗೆ ಪರಮಹಂಸರಲ್ಲಿ ಅಷ್ಟು ನಂಬುಗೆಯಿತ್ತು.

ಸ್ವಾಮೀಜಿ: ಯಾರಿಗೆ ನಂಬುಗೆ ಉಂಟಾಗುವುದಿಲ್ಲವೋ ಅವರಿಗೆ ಕಣ್ಣಾರೆ ಕಂಡರೂ ನಂಬುಗೆಯಾಗುವುದಿಲ್ಲ; ಚಿತ್ತಭ್ರಮಣೆ, ಸ್ವಪ್ನ ಎಂದುಕೊಂಡುಬಿಡುತ್ತಾರೆ. ದುರ್ಯೋಧನನೂ ವಿಶ್ವರೂಪವನ್ನು ನೋಡಿದನು - ಅರ್ಜುನನೂ ನೋಡಿದನು. ಅರ್ಜುನನಿಗೆ ನಂಬುಗೆ ಹುಟ್ಟಿತು. ದುರ್ಯೋಧನನು ಇಂದ್ರಜಾಲವೆಂದು ಭಾವಿಸಿದನು. ದೇವರೇ ತಿಳಿಸಿಕೊಡದಿದ್ದರೆ ಏನನ್ನು ಹೇಳುವುದೂ ತಿಳಿಸುವುದೂ ಸಾಧ್ಯವಲ್ಲ. ನೋಡದೆ ಕೇಳದೆ ಕೆಲವರಿಗೆ ಹದಿನಾರಾಣೆಯ ನಂಬುಗೆಯುಂಟಾಗುತ್ತದೆ. ಕೆಲವರು ಹನ್ನೆರಡು ವರ್ಷಕಾಲ ಎದುರಿಗಿದ್ದುಕೊಂಡು ನಾನಾ ದಿವ್ಯ ವಿಭೂತಿಗಳನ್ನು ನೋಡಿದರೂ ಸಂದೇಹದಲ್ಲಿ ಮುಳುಗಿರುತ್ತಾರೆ. ಸಾರವಾದ ಮಾತೆಂದರೆ - ಅವನ ಕೃಪೆ; ಆದರೆ ಬಿಡದೆ ಹಿಡಿದುಕೊಂಡು ಇರಬೇಕು ಅಷ್ಟೆ; ಹಾಗಾದರೆ ಅವನ ಕೃಪೆಯುಂಟಾಗುತ್ತದೆ.

ಶಿಷ್ಯ: ಕೃಪೆಗೆ ಏನಾದರೂ ನಿಯಮವಿದೆಯೇನು, ಮಹಾಶಯರೇ?

ಸ್ವಾಮೀಜಿ: ಇರುವುದೂ ನಿಜ; ಇಲ್ಲದಿರುವುದೂ ನಿಜ.

ಶಿಷ್ಯ: ಹೇಗೆ?

ಸ್ವಾಮೀಜಿ: ಯಾರು ಮನೋವಾಕ್ಕಾಯಗಳಲ್ಲಿ ಸದಾ ಪವಿತ್ರರಾಗಿರುತ್ತಾರೆಯೋ, ಯಾರ ಭಕ್ತಿ ಪ್ರಬಲವಾಗಿರುತ್ತದೆಯೋ; ಯಾರು ಸದಸದ್ವಿಚಾರವಂತರೊ ಮತ್ತು ಧ್ಯಾನಧಾರಣಗಳಲ್ಲಿ ಆಸಕ್ತರೋ ಅವರ ಮೇಲೆಯೆ ಭಗವಂತನ ಕೃಪೆಯುಂಟಾಗುತ್ತದೆ. ಆದರೆ ಭಗವಂತನು ಪ್ರಕೃತಿಯ ಸಕಲ ನಿಯಮಗಳ ಹೊರಗಿದ್ದಾನೆ; ಯಾವ ನಿಯಮ ನೀತಿಗಳಿಗೂ ವಶೀಭೂತನಲ್ಲ. ಪರಮಹಂಸರು ‘ಅವನ ಸ್ವಭಾವವು ಹುಡುಗರ ಸ್ವಭಾವದ ಹಾಗೆ’ ಎಂದು ಹೇಳುತ್ತಿದ್ದರು. ಅದಕ್ಕೋಸ್ಕರವೆ ಕೆಲವರು ಕೋಟಿ ಜನ್ಮಗಳನ್ನು ಎತ್ತಿ ಎತ್ತಿ ಬಂದರೂ ಅವನ ಸುಳಿವೆ ಗೊತ್ತಾಗದಿರುವುದು ಕಾಣಬರುತ್ತದೆ; ಮತ್ತು ಯಾರನ್ನು ನಾವು ಪಾಪಿ ಪಾಪಿ ನಾಸ್ತಿಕ ಎನ್ನುತ್ತೇವೆಯೋ, ಅವರಲ್ಲಿ ಬೇಗನೆ ಚಿತ್ಪ್ರಕಾಶವುಂಟಾಗಿಬಿಡುತ್ತದೆ - ಅವರಿಗೆ ಭಗವಂತನು ಅಯಾಚಿತವಾಗಿ ಕೃಪೆಮಾಡುತ್ತಾನೆ. ಅವರಿಗೆ ಪೂರ್ವಜನ್ಮದ ಪುಣ್ಯವಿತ್ತು ಎಂದು ನೀನು ಹೇಳಬಹುದು; ಆದರೆ ಈ ರಹಸ್ಯವನ್ನು ತಿಳಿಯುವುದು ಕಷ್ಟ. ಪರಮಹಂಸರು ಕೆಲವು ವೇಳೆ “ಅವನ ಮೇಲೆ ಭಾರ ಹಾಕಿದ್ದುಬಿಡು - ಬಿರುಗಾಳಿಗೆ ಹಾರಾಡುವ ತರಗೆಲೆಯಾಗಿ ಬಿಡು"\footnote{ಎಂದರೆ ತರಗೆಲೆಗಳು ಗಾಳಿಗೆ ಸಿಕ್ಕಿದಾಗ ಅದು ಅವುಗಳನ್ನು ಎತ್ತಕಡೆ ಎತ್ತಿಕೊಂಡು ಹೋದರೆ ಅತ್ತಕಡೆ ಹೇಗೆ ಹೋಗುತ್ತವೆಯೋ, ಹಾಗೆ ಮನುಷ್ಯನೂ ಭಗವಂತನ ಇಚ್ಛೆ ತನ್ನನ್ನು ಹೇಗೆ ಹೇಗೆ ಎಳೆದುಕೊಂಡು ಹೋದರೆ ಹಾಗೆ ಗೊಣಗುಟ್ಟದೆ ಹೊರಟು ಹೋಗಬೇಕೆಂದು ಅರ್ಥ.} ಎಂದೂ, ಕೆಲವು ವೇಳೆ ‘ಅವನ ಕೃಪಾಮಾರುತವು ಬೀಸುತ್ತಲೇ ಇರುತ್ತದೆ, ನೀನು ನಿನ್ನ ದೋಣಿಯ ಪಟವನ್ನು ಎತ್ತಿಕಟ್ಟುವುದಿಲ್ಲ’ ಎಂದೂ ಹೇಳುತ್ತಿದ್ದರು.

ಶಿಷ್ಯ: ಮಹಾಶಯರೆ, ಇದಂತೂ ಬಹಳ ಕಠಿಣವಾದ ವಿಷಯ; ಯಾವ ಯುಕ್ತಿಯೂ ಇದರ ಮುಂದೆ ನಿಲ್ಲುವ ಹಾಗಿಲ್ಲ.

ಸ್ವಾಮಿಜಿ: ಯುಕ್ತಿತರ್ಕಗಳ ಎಲ್ಲೆಯು ಮಾಯಾ ವ್ಯಾಪ್ತವಾದ ಜಗತ್ತಿನಲ್ಲಿಯೆ, ದೇಶಕಾಲ ನಿಮಿತ್ತಗಳ ಗಡಿಯ ಮಧ್ಯದಲ್ಲಿಯೆ ಇರತಕ್ಕದ್ದು. ಅವನು ದೇಶಕಾಲಾತೀತ. ಅವನ ನಿಯಮ ಇರುವುದೂ ನಿಜ ಮತ್ತು ಅವನು ನಿಯಮದ ಹೊರಗೆ ಇರುವುದೂ ನಿಜ. ಪ್ರಕೃತಿಯ ಯಾವ ಕೆಲವು ನಿಯಮಗಳಿವೆಯೋ ಅವುಗಳನ್ನು ಅವನೇ ಮಾಡಿದ್ದಾನೆ. ಅವುಗಳೂ ಅವನೇ ಆಗಿದ್ದಾನೆ; ಮತ್ತು ಅವುಗಳೆಲ್ಲವುಗಳ ಹೊರಗೂ ಇದ್ದಾನೆ. ಅವನು ಯಾರಿಗೆ ಕೃಪೆ ಮಾಡುತ್ತಾನೆಯೋ, ಅವನು ಆ ಮುಹೂರ್ತದಲ್ಲಿಯೆ ಗಡಿಯ ಹೊರಗಡೆ ಹೊರಟು ಹೋಗುತ್ತಾನೆ. ಅದಕ್ಕೋಸ್ಕರವೆ ಕೃಪೆಗೆ ಯಾವ ಬಂಧಿಸತಕ್ಕ ನಿಯಮವೂ ಇಲ್ಲ. ಕೃಪೆಯು ಅವನ ಲೀಲೆ, ಈ ಜಗತ್ಸೃಷ್ಟಿ ಎಲ್ಲಾ ಅವನ ಲೀಲೆ - ‘ಲೋಕವತ್ತು ಲೀಲಾ ಕೈವಲ್ಯಂ’\footnote{ವೇದಾಂತ ಸೂತ್ರ ೨, ೧, ೩೩.} ಯಾರು ಲೀಲೆಯಾಗಿ ಇಂಥ ಜಗತ್ತನ್ನು ಸೃಷ್ಟಿಸುವುದಕ್ಕೂ ನಾಶಪಡಿಸುವುದಕ್ಕೂ ಸಮರ್ಥನೊ ಅವನು ಮಹಾಪಾಪಿಗೂ ಕೃಪೆಮಾಡಿ ಮುಕ್ತಿ ಕೊಡಲಾರನೇನು? ಆದರೆ ಕೆಲವರಿಗೆ ಸಾಧನ ಭಜನಗಳನ್ನು ಮಾಡಿಸುತ್ತಾನೆ. ಕೆಲವರಿಗೆ ಮಾಡಿಸುವುದಿಲ್ಲ - ಅದೂ ಅವನ ಲೀಲೆ - ಅವನ ಇಚ್ಛೆ.

ಶಿಷ್ಯ: ಮಹಾಶಯರೆ, ತಿಳಿದುಕೊಳ್ಳುವುದಕ್ಕಾಗಲಿಲ್ಲ.

ಸ್ವಾಮೀಜಿ: ತಿಳಿದುಕೊಳ್ಳುವುದಕ್ಕೆ ಇನ್ನು ಆಗುತ್ತದೆಯೇನು? ಎಷ್ಟು ತಿಳಿದುಕೊಂಡೆಯೊ ಅಷ್ಟನ್ನು ಮನಸ್ಸಿನಲ್ಲಿ ನಿಲ್ಲಿಸಿಕೊ. ಹಾಗಾದರೆ ಈ ಜಗತ್ತೆಂಬ ಇಂದ್ರಜಾಲ ತನ್ನಷ್ಟಕ್ಕೆ ತಾನೆ ಮಾಯವಾಗುವುದು. ಆದರೆ ತಿಳಿದಷ್ಟನ್ನು ನಿಲ್ಲಿಸಿ ಹಿಡಿದುಕೊಂಡಿರಬೇಕು. ಕಾಮಕಾಂಚನಗಳಿಂದ ಮನಸ್ಸನ್ನು ಸೆಳೆದುಕೊಳ್ಳಬೇಕು, ಸದಸದ್ವಿಚಾರವನ್ನು ಸರ್ವದಾ ಮಾಡುತ್ತಿರಬೇಕು, ‘ನಾನು ದೇಹವಲ್ಲ’ ಎಂಬ ವಿದೇಹ ಭಾವದಲ್ಲಿ ಇರಬೇಕು, ‘ನಾನು ಸರ್ವಗತನಾದ ಆತ್ಮಾ’ ಎಂಬುದನ್ನು ಅನುಭವ ಮಾಡಿಕೊಳ್ಳಬೇಕು. ಹೀಗೆ ಪಟ್ಟು ಹಿಡಿದಿರುವುದರ ಹೆಸರೆ ಪುರುಷ ಪ್ರಯತ್ನ. ಈ ವಿಧವಾದ ಪುರುಷಕಾರದ ಸಹಾಯದಿಂದ ಅವನ ಮೇಲೆ ನಿರ್ಭರತೆ ಉಂಟಾಗುತ್ತದೆ. ಇದೇ ಪರಮ ಪುರುಷಾರ್ಥವಾಯಿತು.

ಸ್ವಾಮೀಜಿ ಮತ್ತೆ ಹೇಳತೊಡಗಿದರು: ಅವನ ಕೃಪೆಯು ನಿನ್ನ ಮೇಲೆ ಇಲ್ಲದಿದ್ದರೆ ನೀನು ಇಲ್ಲಿಗೆ ಬರುತ್ತಿದ್ದೆಯೋ? ಪರಮಹಂಸರು ‘ಯಾರ ಮೇಲೆ ಈಶ್ವರನ ಕೃಪೆಯಿದೆಯೋ ಅವರು ಇಲ್ಲಿಗೆ ಬಂದೇ ಬರುತ್ತಾರೆ; ಎಲ್ಲಿಯಾದರೂ ಇರಲಿ ಏನಾದರೂ ಮಾಡಲಿ ಇಲ್ಲಿರುವವನ (ಪರಮಹಂಸರ) ಮಾತಿನಿಂದಲೂ ಭಾವದಿಂದಲೂ ಅವರು ಪ್ರಭಾವಿತರಾಗಿಯೇ ಆಗುತ್ತಾರೆ’ ಎಂದು ಹೇಳುತ್ತಿದ್ದರು. ನಿನ್ನ ವಿಚಾರವನ್ನೆ ಯೋಚಿಸಿ ನೋಡು! ಯಾರು ಕೃಪಾಬಲದಿಂದ ಸಿದ್ಧಪುರುಷರಾದರೊ ಯಾರು ಪ್ರಭುವಿನ ಕೃಪೆಯನ್ನು ಚೆನ್ನಾಗಿ ತಿಳಿದುಕೊಂಡಿದ್ದಾರೆಯೊ ಅಂಥ ನಾಗಮಹಾಶಯರ ಸಂಗಲಾಭ ಈಶ್ವರನ ಕೃಪೆಯಲ್ಲದೆ ಬೇರೆ ಏನು? ‘ಅನೇಕ ಜನ್ಮಸಂಸಿದ್ಧಸ್ತತೋ ಯಾತಿ ಪರಾಂ ಗತಿಂ.’ (ಗೀತೆ ೬.೪೫) ಜನ್ಮ ಜನ್ಮಾಂತರದ ಪುಣ್ಯವಿದ್ದರೆ ಆಗ ಅಂಥ ಮಹಾಪುರುಷರ ದರ್ಶನಲಾಭವಾಗುತ್ತದೆ. ಶಾಸ್ತ್ರದಿಂದ ಉತ್ತಮ ಭಕ್ತಿಯ ಲಕ್ಷಣಗಳೆಂದು ಯಾವ ಯಾವುವು ತಿಳಿದುಬರುತ್ತವೆಯೋ ಅವೆಲ್ಲಾ ನಾಗಮಹಾಶಯರಲ್ಲಿ ಸ್ಪಷ್ಟವಾಗಿ ಕಂಡುಬಂದಿವೆ. ‘ತೃಣಾದಪಿ ಸುನೀಚೇನ’ ಎಂದು ಹೇಳುತ್ತಾರೆಯಲ್ಲಾ ಅದು ನಾಗಮಹಾಶಯರೊಬ್ಬರಲ್ಲಿ ಮಾತ್ರವೇ ಪ್ರತ್ಯಕ್ಷವಾದದ್ದು. ನಿಮ್ಮ ಪೂರ್ವ ಬಂಗಾಳದೇಶ ಧನ್ಯ - ನಾಗಮಹಾಶಯರ ಪದಸ್ಪರ್ಶದಿಂದ ಪವಿತ್ರವಾಗಿ ಹೋಗಿದೆ.

ಹೀಗೆಂದು ಹೇಳುತ್ತ ಹೇಳುತ್ತ ಸ್ವಾಮೀಜಿ ಮಹಾಕವಿ ಶ‍್ರೀಯುತ ಗಿರೀಶಚಂದ್ರ ಘೋಷರ ಮನೆಗೆ ಹೋಗಿಬರುವುದಕ್ಕೆಂದು ಹೊರಟರು; ಜೊತೆಯಲ್ಲಿ ಯೋಗಾನಂದ ಸ್ವಾಮಿ ಮತ್ತು ಶಿಷ್ಯ ಇದ್ದರು. ಗಿರೀಶ ಬಾಬುಗಳ ಮನೆಗೆ ಹೋಗಿ ಕುಳಿತುಕೊಂಡು ಸ್ವಾಮಿಗಳು “ಜಿ.ಸಿ., ಈಗ ಅದು ಮಾಡೋಣ ಇದು ಮಾಡೋಣ, ಅವರ ವಿಷಯವನ್ನು ಪ್ರಪಂಚದಲ್ಲಿ ಹರಡಿ ಬಿಡೋಣ ಎಂಬುದೆ ಮನಸ್ಸಿಗೆ ಬರುತ್ತಿರುತ್ತದೆ. ಮತ್ತೆ ಇದರಿಂದ ಭರತಖಂಡದಲ್ಲಿ ಮತ್ತೊಂದು ಸಂಪ್ರದಾಯವು ಸೃಷ್ಟಿಯಾಗಿ ಬಿಡುವುದೆಂದೂ ಭಾವಿಸಿಕೊಳ್ಳುತ್ತೇನೆ. ಅದಕ್ಕೇ ಬಹುಮಟ್ಟಿಗೆ ಎಚ್ಚರಿಕೆಯಿಂದ ನಡೆಯುತ್ತೇನೆ. ಕೆಲವು ಸಮಯ, ಸಂಪ್ರದಾಯವಾಗಿ ಹೋಗಲಿ ಎಂದುಕೊಳ್ಳುತ್ತೇನೆ. ಮತ್ತೆ - ಹಾಗಲ್ಲ, ಅವರು ಯಾವ ಭಾವವನ್ನೂ ಯಾವಾಗಲೂ ನಷ್ಟಪಡಿಸಲಿಲ್ಲ. ಸಮದೃಷ್ಟಿಯೆ ಅವರ ಭಾವ - ಹೀಗೆಂದುಕೊಂಡು ಅನೇಕಸಾರಿ ಮನಸ್ಸಿನ ಪ್ರವೃತ್ತಿಯನ್ನು ನಿಗ್ರಹಿಸಿಕೊಂಡು ಹೋಗುತ್ತೇನೆ. ನಿನ್ನ ಅಭಿಪ್ರಾಯವೇನು?" ಎಂದರು.

ಗಿರೀಶಬಾಬು: ನಾನು ಇನ್ನೇನು ಹೇಳಲಿ? ನೀನು ಅವರ ಕೈಯಿನ ಉಪಕರಣ; ಏನನ್ನು ಮಾಡಿಸುತ್ತಾರೆಯೋ ಅದನ್ನೆ ನೀನು ಮಾಡಬೇಕು. ನನಗೆ ಅಷ್ಟು ವಿಷಯಗಳು ತಿಳಿಯುವುದಿಲ್ಲ. ಆದರೆ ನಿನ್ನ ಮೂಲಕ ಪ್ರಭುವಿನ ಶಕ್ತಿ ಕಾರ್ಯವನ್ನು ಮಾಡಿಸಿಕೊಂಡು ಹೋಗುತ್ತಿದೆಯೆಂಬುದು ನನ್ನ ಕಣ್ಣಿಗೆ ಕಾಣುತ್ತಿದೆ - ಸುಸ್ಪಷ್ಟವಾಗಿ ಕಾಣುತ್ತಿದೆ.

ಸ್ವಾಮೀಜಿ: ನನ್ನ ಇಷ್ಟದಂತೆ ನಾನು ಕಾರ್ಯಮಾಡಿಕೊಂಡು ಹೋಗುತ್ತಿದ್ದೇನೆಂದು ನನಗೆ ಕಾಣುತ್ತದೆ. ಆದರೆ ಆಪತ್ತಿನಲ್ಲಿಯೂ ವಿಪತ್ತಿನಲ್ಲಿಯೂ ಅಭಾವದಲ್ಲಿಯೂ ದಾರಿದ್ರ್ಯದಲ್ಲಿಯೂ ಅವರು ನೋಡಿಕೊಳ್ಳುತ್ತಿದ್ದು, ಸರಿಯಾದ ದಾರಿಯಲ್ಲಿ ನನ್ನನ್ನು ನಡೆಸಿಕೊಂಡು ಹೋಗುತ್ತಿದ್ದಾರೆಂಬುದನ್ನು ಕಂಡಿದ್ದೇನೆ. ಆದರೆ ಪ್ರಭುವಿನ ಶಕ್ತಿಯ ಪರಿಮಾಣವು ಇಷ್ಟೇ ಎಂದು ಸ್ವಲ್ಪಮಟ್ಟಿಗಾದರೂ ಗೊತ್ತುಮಾಡಲಾರದವನಾಗಿದ್ದೇನೆ.

ಗಿರೀಶಬಾಬು: ಅವರು, ಎಲ್ಲವೂ ಪೂರ್ಣವಾಗಿ ತಿಳಿದರೆ ಕೂಡಲೆ ಎಲ್ಲಾ ಮಾಯವಾಗಿ ಹೋಗುತ್ತದೆ; ಮಾಡುವವನು ಯಾರು? ಮಾಡಿಸುವವನು ಯಾರು? ಎಂದು ಹೇಳಿದ್ದಾರೆ. ಹೀಗೆ ಮಾತುಕತೆಗಳು ನಡೆದ ಬಳಿಕ ಅಮೆರಿಕಾದ ಪ್ರಸ್ತಾಪ ಆರಂಭವಾಯಿತು. ಗಿರೀಶಬಾಬುಗಳು ಬೇಕೆಂದೇ ಸ್ವಾಮಿಗಳ ಮನಸ್ಸನ್ನು ಬೇರೊಂದು ಸಂಗತಿಗೆ ತಿರುಗಿಸಿದಂತೆ ಕಾಣುತ್ತದೆ. ಹೀಗೆ ಮಾಡಿದ್ದಕ್ಕೆ ಕಾರಣವನ್ನು ಕೇಳಲು ಗಿರೀಶಬಾಬುಗಳು ನನಗೆ ಬೇರೊಂದು ಸಮಯದಲ್ಲಿ ಹೇಳಿದರು: “ಪರಮಹಂಸರ ಶ‍್ರೀಮುಖದಿಂದ ಕೇಳಿದ್ದೇನೆ – ಇಂಥ ಮಾತನ್ನು ಹೆಚ್ಚಾಗಿ ಆಡುತ್ತ ಆಡುತ್ತ ಸ್ವಾಮೀಜಿಗಳಿಗೆ ಸಂಸಾರ ವೈರಾಗ್ಯವೂ ಈಶ್ವರೋದ್ದೀಪನವೂ ಆಗುತ್ತದೆ. ಒಟ್ಟಿಗೇನಾದರೂ ಸ್ವಸ್ವರೂಪದ ದರ್ಶನವಾಗಿಬಿಟ್ಟರೆ - ಅವನಿಗೂ ಈ ವಿಷಯವು ಗೊತ್ತೋ ಏನೋ - ಆಮೇಲೆ ಒಂದು ಮುಹೂರ್ತವೂ ಅವನ ದೇಹವು ನಿಲ್ಲುವುದಿಲ್ಲ ಎಂದು. ಅದಕ್ಕೋಸ್ಕರವೇ ಸ್ವಾಮಿಗಳ ಗುರುಭ್ರಾತೃಗಳೂ ಕೂಡ ಸ್ವಾಮೀಜಿ ಇಪ್ಪತ್ತುನಾಲ್ಕು ಗಂಟೆಯೂ ಪರಮಹಂಸರ ಮಾತುಕತೆಗಳನ್ನೇ ಆಡುವುದಕ್ಕೆ ಮೊದಲು ಮಾಡಿದರೆ, ಸ್ವಾಮಿಗಳನ್ನು ಬೇರೆ ವಿಷಯದಲ್ಲಿ ಮನಸ್ಸಿಡುವಂತೆ ಮಾಡುತ್ತಿದ್ದದ್ದನ್ನು ನೋಡಿದ್ದೇನೆ." ಅದು ಹೇಗಾದರೂ ಇರಲಿ ಅಮೆರಿಕಾದ ಪ್ರಸ್ತಾಪ ಮಾಡುತ್ತ ಮಾಡುತ್ತ ಅದರಲ್ಲಿಯೆ ಮಗ್ನರಾಗಿ ಹೋದರು. ಆ ದೇಶದ ಸಮೃದ್ಧಿ, ಸ್ತ್ರೀಪುರುಷರ ಗುಣಾವಗುಣ, ಭೋಗ ವಿಲಾಸಗಳು ಮುಂತಾದ ನಾನಾ ಸಂಗತಿಗಳನ್ನು ವರ್ಣನೆ ಮಾಡತೊಡಗಿದರು.

\newpage

\chapter[ಅಧ್ಯಾಯ ೮]{ಅಧ್ಯಾಯ ೮\protect\footnote{\enginline{C.W, Vol. VI, P, 484}}}

\begin{center}
ಸ್ಥಳ: ಕಲ್ಕತ್ತ, ವರ್ಷ: ಕ್ರಿ.ಶ. ೧೮೯೭.
\end{center}

ಸ್ವಾಮೀಜಿ ಬಾಗಬಜಾರಿನ ಬಲರಾಮ ಬಸುಗಳ ಮನೆಯಲ್ಲಿ ವಾಸಮಾಡುವುದಕ್ಕೆ ಮೊದಲುಮಾಡಿ ಕೆಲವು ದಿನಗಳಾದುವು. ಬೆಳಗ್ಗೆಯಾಗಲಿ ಮಧ್ಯಾಹ್ನವಾಗಲಿ ಸಾಯಂಕಾಲವಾಗಲಿ ಅವರಿಗೆ ಸ್ವಲ್ಪವೂ ವಿರಾಮವಿಲ್ಲ; ಏಕೆಂದರೆ ಉತ್ಸಾಹಶಾಲಿಗಳಾದ ಅನೇಕ ಯುವಕರು, ಕಾಲೇಜಿನ ಬಹುಜನ ವಿದ್ಯಾರ್ಥಿಗಳು, ಅವರು ಈಗ ಎಲ್ಲಿದ್ದರೂ ಸರಿಯೆ, ಅವರ ದರ್ಶನ ಮಾಡುವುದಕ್ಕೆಂದು ಬರುತ್ತಿದ್ದರು; ಸ್ವಾಮಿಜಿ ಕೂಡ ಅವರೆಲ್ಲರಿಗೂ ಆದರದಿಂದ ಧರ್ಮ ಮತ್ತು ದರ್ಶನಗಳಿಗೆ ಸಂಬಂಧಪಟ್ಟ ಜಟಿಲ ತತ್ತ್ವಗಳನ್ನು ಸುಲಭವಾದ ಮಾತಿನಲ್ಲಿ ತಿಳಿಸಿಕೊಡುತ್ತಿದ್ದರು; ಸ್ವಾಮಿಜಿಯ ಮೇಧಾಶಕ್ತಿಯ ಮುಂದೆ ಅವರೆಲ್ಲರೂ ಪರಾಜಿತರಾದಂತಾಗಿ ಸದ್ದುಗದ್ದಲವಿಲ್ಲದೆ ಇರುತ್ತಿದ್ದರು.

ಇಂದು ಸೂರ್ಯಗ್ರಹಣ - ಪೂರ್ಣಗ್ರಹಣ. ಜ್ಯೋತಿಷ್ಯರು ಗ್ರಹಣವನ್ನು ನೋಡುವುದಕ್ಕೆ ನಾನಾ ಸ್ಥಳಗಳಿಗೆ ಹೋಗಿದ್ದಾರೆ. ಆಚಾರಶೀಲ ಸ್ತ್ರೀಪುರುಷರು ಗಂಗಾಸ್ನಾನಕ್ಕಾಗಿ ಬಹುದೂರದಿಂದ ಬಂದು ಉತ್ಸುಕರಾಗಿ ಗ್ರಹಣವಾಗುವ ಕಾಲವನ್ನು ಎದುರುನೋಡುತ್ತಿದ್ದಾರೆ. ಸ್ವಾಮೀಜಿಗೆ ಮಾತ್ರ ಗ್ರಹಣದ ವಿಚಾರವಾಗಿ ವಿಶೇಷವಾದ ಉತ್ಸಾಹವೇನೂ ಇಲ್ಲ. ಶಿಷ್ಯನು ಈ ದಿವಸ ಸ್ವಾಮೀಜಿಗೆ ತನ್ನ ಕೈಯಿಂದ ಅಡಿಗೆ ಮಾಡಿ ಬಡಿಸುತ್ತಾನೆ – ಹಾಗೆ ಸ್ವಾಮಿಗಳ ಅಪ್ಪಣೆಯಾಗಿದೆ. ತರಕಾರಿ ಮತ್ತು ಅಡಿಗೆಗೆ ಬೇಕಾದ ಇತರ ಪದಾರ್ಥಗಳನ್ನು ತಂದುಕೊಂಡು ಅವನು ಸುಮಾರು ಎಂಟು ಗಂಟೆಯ ಹೊತ್ತಿಗೆ ಬಲರಾಮ ಬಾಬುಗಳ ಮನೆಗೆ ಬಂದಿದ್ದಾನೆ. ಅವನನ್ನು ನೋಡಿ ಸ್ವಾಮೀಜಿ “ನಿಮ್ಮ ದೇಶದ ಅಡಿಗೆ ಮಾಡಬೇಕು; ಮತ್ತು ಗ್ರಹಣವಾಗುವುದಕ್ಕೆ ಮುಂಚೆಯೇ ತಿನ್ನುವುದು ಕುಡಿಯುವುದು ಎಲ್ಲಾ ಮುಗಿದುಹೋಗಿರಬೇಕು" ಎಂದರು.

ಬಲರಾಮ ಬಾಬುಗಳ ಮನೆಯ ಹೆಂಗಸರು ಯಾರೂ ಈಗ ಕಲ್ಕತ್ತೆಯಲ್ಲಿಲ್ಲ. ಆದ್ದರಿಂದ ಮನೆ ಪೂರ್ತಿಯಾಗಿ ಬಿಡುವಾಗಿತ್ತು. ಶಿಷ್ಯ ಮನೆಯ ಒಳಗೆ ಹೋಗಿ ಅಡುಗೆಗೆ ಮೊದಲು ಮಾಡಿದನು. ಶ‍್ರೀರಾಮಕೃಷ್ಣ ಗತಪ್ರಾಣಳಾದ ಯೋಗೀನ್ ಮಾತೆಯು ಹತ್ತಿರ ನಿಂತುಕೊಂಡು ಅಡಿಗೆಗೆ ಬೇಕಾದ ಪದಾರ್ಥಗಳನ್ನು ಶಿಷ್ಯನಿಗೆ ಒದಗಿಸಿಕೊಡುತ್ತ ಸಹಾಯ ಮಾಡುತ್ತಿದ್ದಳು. ಸ್ವಾಮೀಜಿ ಮಧ್ಯೆ ಮಧ್ಯೆ ಒಳಕ್ಕೆ ಬಂದು ಅಡುಗೆಯನ್ನು ನೋಡಿ ಅವನಿಗೆ ಉತ್ಸಾಹ ಕೊಡುತ್ತಿದ್ದರು; ಮತ್ತು ಆಗಾಗೆ, “ನೋಡು, ಅಡುಗೆ ಪೂರ್ವಬಂಗಾಳದ ಕಡೆಯವರ ಹಾಗೆ ಇರಬೇಕು" ಎಂದು ಮುಂತಾಗಿ ಹಾಸ್ಯ ಮಾಡುತ್ತಿದ್ದರು.

ಅನ್ನ ಸಾರು ತೊವ್ವೆ ಪಲ್ಯ ಮುಂತಾದ ಅಡುಗೆ ಇನ್ನೇನು ಮುಗಿದಂತಾಯಿತು. ಈ ಸಮಯದಲ್ಲಿ ಸ್ವಾಮೀಜಿ ಸ್ನಾನಮಾಡಿಕೊಂಡು ಬಂದು ತಾವೆ ಎಲೆ ಹಾಕಿಕೊಂಡು ಊಟಕ್ಕೆ ಕುಳಿತರು. “ಇನ್ನೂ ಸ್ವಲ್ಪ ಅಡುಗೆಯಾಗುವುದು ಉಳಿದಿದೆ" ಎಂದು ಹೇಳಿದರೂ ಕೇಳದೆ ಹಟಮಾಡಿಕೊಂಡು ಹುಡುಗರ ಹಾಗೆ “ಏನಾಗಿದೆಯೋ ಅದನ್ನೇ ಬೇಗ ತಂದುಹಾಕಿಬಿಡು, ಇನ್ನು ನಾನು ಕಾದುಕೊಂಡಿರಲಾರೆ, ಹೊಟ್ಟೆ ಹಸಿವಿನಿಂದ ಉರಿದು ಹೋಗುತ್ತಿದ್ದೇನೆ" ಎಂದರು. ಆದ್ದರಿಂದ ಶಿಷ್ಯನು ಗಡಿಬಿಡಿಯಿಂದ ಮೊದಲು ಪಲ್ಯ ಅನ್ನಗಳನ್ನು ಬಡಿಸಿದನು; ಸ್ವಾಮೀಜಿ ಆ ಕ್ಷಣವೆ ಊಟಕ್ಕೆ ಮೊದಲುಮಾಡಿದರು. ಆಮೇಲೆ ಶಿಷ್ಯನು ಮಿಕ್ಕ ಅಡುಗೆಗಳನ್ನೆಲ್ಲಾ ಸ್ವಾಮೀಜಿಗೆ ಬಡಿಸಿದ ಮೇಲೆ ಯೋಗಾನಂದ ಪ್ರೇಮಾನಂದ ಮುಂತಾದ ಮಿಕ್ಕ ಸಂನ್ಯಾಸಿ ಗುರುಗಳಿಗೆ ಬಡಿಸುವುದಕ್ಕೆ ಹೊರಟನು. ಶಿಷ್ಯನು ಯಾವತ್ತೂ ಅಡುಗೆಯಲ್ಲಿ ಬುದ್ಧಿವಂತನೆನಿಸಿಕೊಂಡವನಲ್ಲ. ಆದರೆ ಸ್ವಾಮೀಜಿ ಇಂದು ಅವನ ಅಡುಗೆಯನ್ನು ಬಹುವಾಗಿ ಹೊಗಳುವುದಕ್ಕೆ ಮೊದಲುಮಾಡಿದರು. ಕಲ್ಕತ್ತೆಯ ಜನರು ನಾನು ಮಾಡಿದ್ದ ಅಡುಗೆಯ ಹೆಸರು ಕೇಳಿದೊಡನೆಯೇ ಹಾಸ್ಯ ಮಾಡಿಕೊಂಡು ನಗುವರು. ಆದರೆ ಅವರು ಮಾತ್ರ ಆ ಪಲ್ಯವನ್ನು ತಿಂದು ಸಂತೋಷಪಟ್ಟು “ಇಂಥಾದ್ದನ್ನು ಯಾವಾಗಲೂ ತಿಂದಿರಲಿಲ್ಲ" ಎಂದರು. ಆಮೇಲೆ ಮೊಸರು ಹಾಕಿಸಿಕೊಂಡು ಊಟಮಾಡಿ ಮುಗಿಸಿ ಆಚಮನಾನಂತರ ಒಳಗೆ ಇದ್ದ ಮಂಚದಮೇಲೆ ಹೋಗಿ ಕುಳಿತುಕೊಂಡರು. ಶಿಷ್ಯನು ಸ್ವಾಮೀಜಿ ಎದುರಿಗೆ ನಡುಮನೆಯಲ್ಲಿ ಪ್ರಸಾದವನ್ನು ತೆಗೆದುಕೊಳ್ಳುತ್ತ ಕುಳಿತನು. ತಂಬಾಕನ್ನು ತೀಡುತ್ತ ತೀಡುತ್ತ ಅವನನ್ನು ಕುರಿತು “ಯಾರು ಚೆನ್ನಾಗಿ ಅಡುಗೆ ಮಾಡಲಾರರೊ ಅವರು ಒಳ್ಳೆಯ ಸಾಧುಗಳಾಗಲಾರರು - ಮನಸ್ಸು ಶುದ್ಧವಾಗಿಲ್ಲದಿದ್ದರೆ ಒಳ್ಳೆಯ ರುಚಿಯಾದ ಅಡುಗೆಯಾಗುವುದಿಲ್ಲ" ಎಂದು ಹೇಳಿದರು.

ಸ್ವಲ್ಪ ಹೊತ್ತಿನ ಮೇಲೆ ನಾಲ್ಕು ಕಡೆಯಲ್ಲಿಯೂ ಶಂಖದ ಮತ್ತು ಗಂಟೆಯ ಧ್ವನಿಯೆದ್ದಿತು. ಸ್ತ್ರೀಯರ ‘ಉಲು’\footnote{ವಂಗ ದೇಶದಲ್ಲಿ ಸ್ತ್ರೀಯರು ಮಂಗಳಸೂಚಕಾರ್ಥವಾಗಿ ಮಾಡುವ ಧ್ವನಿಗೆ 'ಉಲು' ಎಂದು ಹೆಸರು.} ಧ್ವನಿಯು ಕೇಳುವುದಕ್ಕೆ ಮೊದಲಾಯಿತು. ಸ್ವಾಮೀಜಿ “ಅಯ್ಯಾ, ಗ್ರಹಣ ಹಿಡಿಯಿತು - ನಾನು ಮಲಗಿಕೊಳ್ಳುತ್ತೇನೆ. ಸ್ವಲ್ಪ ಕಾಲು ಹಿಸುಕು" ಎಂದು ಹೇಳಿ ನಿದ್ರೆಮಾಡತೊಡಗಿದರು. ಶಿಷ್ಯನೂ ಅವರ ಪಾದಸೇವೆ ಮಾಡುತ್ತ “ಈ ಪುಣ್ಯಕಾಲದಲ್ಲಿ ಗುರುಪದ ಸೇವೆಯೆ ನನಗೆ ಗಂಗಾಸ್ನಾನ ಮತ್ತು ಜಪ"ವೆಂದು ಭಾವಿಸಿಕೊಂಡನು. ಹೀಗೆಂದುಕೊಂಡು ಶಿಷ್ಯನು ಶಾಂತಮನಸ್ಕನಾಗಿ ಸ್ವಾಮೀಜಿಯವರ ಪಾದಸೇವೆ ಮಾಡತೊಡಗಿದನು. ಗ್ರಹಣ ಪೂರ್ತ ಗ್ರಾಸವಾಗಿ ಕ್ರಮೇಣ ನಾಲ್ಕು ದಿಕ್ಕುಗಳಲ್ಲಿಯೂ ಸಾಯಂಕಾಲದ ಹಾಗೆ ಕತ್ತಲೆ ಕವಿದುಕೊಂಡಿತು.

ಗ್ರಹಣ ಬಿಡುವುದಕ್ಕೆ ಇನ್ನು ೧೫-೨೦ ನಿಮಿಷವಿದೆ ಎನ್ನುವಾಗ ಸ್ವಾಮೀಜಿ ಎದ್ದು ಕೈಕಾಲು ಮುಖಗಳನ್ನು ತೊಳೆದುಕೊಂಡು ತಂಬಾಕನ್ನು ಸೇವಿಸುತ್ತ ಸೇವಿಸುತ್ತ ಶಿಷ್ಯನನ್ನು ಕುರಿತು ಹಾಸ್ಯಮಾಡುತ್ತ “ಗ್ರಹಣ ಕಾಲದಲ್ಲಿ ಯಾರು ಏನು ಮಾಡುತ್ತಾರೋ ಅವರು ಅದನ್ನು ಕೋಟಿ ಪಾಲಿನಷ್ಟು ಪಡೆಯುತ್ತಾರೆಂದು ಜನರು ಹೇಳುತ್ತಾರೆ. ಅದಕ್ಕೋಸ್ಕರವೇ ಎಂದುಕೊಂಡೆ - ಮಹಾಮಾಯೆ ಈ ಶರೀರದಲ್ಲಿ ಒಳ್ಳೆಯ ನಿದ್ದೆಯನ್ನು ಕೊಡಲಿಲ್ಲ; ಈ ಸಮಯದಲ್ಲಿ ಒಂದಿಷ್ಟು ನಿದ್ದೆ ಮಾಡಿದರೆ ಆಮೇಲೆ ಚೆನ್ನಾಗಿ ನಿದ್ದೆ ಬರುವುದು - ಎಂದು. ಆದರೆ ಅದು ಆಗಲಿಲ್ಲ; ಹೆಚ್ಚೆಂದರೆ ಹದಿನೈದು ನಿಮಿಷ ನಿದ್ದೆಯಾಗಿರಬಹುದು" ಎಂದರು.

ಅನಂತರ ಎಲ್ಲರೂ ಸ್ವಾಮೀಜಿ ಹತ್ತಿರ ಬಂದು ಕುಳಿತುಕೊಳ್ಳಲು ಸ್ವಾಮಾಜಿ ಶಿಷ್ಯನನ್ನು ಕುರಿತು ಉಪನಿಷತ್‌ ಸಂಬಂಧವಾಗಿ ಏನಾದರೂ ಹೇಳುವಂತೆ ಅಪ್ಪಣೆ ಮಾಡಿದರು. ಶಿಷ್ಯನು ಇದಕ್ಕೆ ಹಿಂದೆ ಎಂದೂ ಸ್ವಾಮಿಜಿ ಮುಂದೆ ಉಪನ್ಯಾಸ ಮಾಡಿದ್ದಿಲ್ಲ. ಆದ್ದರಿಂದ ಅವನ ಎದೆ ಡವಡವನೆ ಹೊಡೆದುಕೊಳ್ಳಲು ಮೊದಲುಮಾಡಿತು. ಆದರೆ ಸ್ವಾಮಿಜಿ ಬಿಡುವಂಥವರಲ್ಲ. ಆದ್ದರಿಂದ ಶಿಷ್ಯನು ಎದ್ದು “ಪರಾಂಚಿ ಖಾನಿ ವ್ಯತೃಣತ್ ಸ್ವಯಂಭೂಃ" ಎಂಬ ವಾಕ್ಯವನ್ನು ವಿವರಿಸುವುದಕ್ಕೆ ಮೊದಲುಮಾಡಿದನು; ಆಮೇಲೆ ‘ಗುರುಭಕ್ತಿ’ ಮತ್ತು ‘ತ್ಯಾಗ’ದ ಮಹಿಮೆಯನ್ನು ವರ್ಣನೆಮಾಡಿ ಬ್ರಹ್ಮಜ್ಞಾನವೆ ಪರಮಪುರುಷಾರ್ಥವೆಂಬುದನ್ನು ವಿಚಾರಮಾಡಿ ಕುಳಿತುಕೊಂಡನು. ಸ್ವಾಮಿಜಿ ಪುನಃ ಪುನಃ ಚಪ್ಪಾಳೆ ತಟ್ಟುವುದರ ಮೂಲಕ ಶಿಷ್ಯನನ್ನು ಉತ್ಸಾಹಗೊಳಿಸಿ “ಆಹಾ! ಬಲು ಸೊಗಸಾಗಿ ಹೇಳಿದೆ" ಎಂದರು.

ಆಮೇಲೆ ಸ್ವಾಮೀಜಿ ಶುದ್ಧಾನಂದ ಪ್ರಕಾಶಾನಂದ ಮುಂತಾದ ಕೆಲವು ಸಂನ್ಯಾಸಿಗಳಿಗೆ ಏನಾದರೂ ಸ್ವಲ್ಪಹೇಳುವಂತೆ ಅಪ್ಪಣೆ ಮಾಡಿದರು. ಶುದ್ಧಾನಂದ ಸ್ವಾಮಿಗಳು ಓಜಃಪೂರ್ಣವಾದ ಭಾಷೆಯಲ್ಲಿ ‘ಧ್ಯಾನ’ದ ಸಂಬಂಧವಾಗಿ ಅಷ್ಟೇನೂ ಉದ್ದವಲ್ಲದ ಒಂದು ಉಪನ್ಯಾಸ ಮಾಡಿದರು. ಆಮೇಲೆ ಪ್ರಕಾಶಾನಂದ ಮುಂತಾದವರೂ ಹೀಗೆಯೇ ಮಾಡಿದಮೇಲೆ ಸ್ವಾಮೀಜಿ ಎದ್ದು ಹೊರಗಿನ ಬೈಠಕ್ ಖಾನೆಗೆ ಬಂದರು. ಆಗಲೂ ಸಾಯಂಕಾಲವಾಗುವುದಕ್ಕೆ ಇನ್ನೂ ಸುಮಾರು ಒಂದು ಘಂಟೆಯಿತ್ತು. ಎಲ್ಲರೂ ಅಲ್ಲಿಗೆ ಬಂದ ಮೇಲೆ ಸ್ವಾಮಿಜಿ “ನೀವು ಕೇಳತಕ್ಕದ್ದೇನಾದರೂ ಇದ್ದರೆ ಹೇಳಿ" ಎಂದರು.

ಶುದ್ಧಾನಂದ ಸ್ವಾಮಿಗಳು ‘ಮಹಾಶಯರೆ, ಧ್ಯಾನದ ಸ್ವರೂಪವೇನು?’ ಎಂದು ಕೇಳಿದರು.

ಸ್ವಾಮೀಜಿ: ಯಾವುದಾದರೂ ಒಂದು ವಿಷಯದಲ್ಲಿ ಮನಸ್ಸನ್ನು ಕೇಂದ್ರೀಕರಿಸುವುದರ ಹೆಸರೇ ಧ್ಯಾನ. ಒಂದು ವಿಷಯದಲ್ಲಿ ಮನಸ್ಸನ್ನು ಏಕಾಗ್ರಮಾಡುವುದಕ್ಕೆ ಸಮರ್ಥನಾದರೆ ಆ ಮನಸ್ಸನ್ನು ಯಾವ ವಿಷಯದಲ್ಲಿ ಬೇಕಾದರೂ ಏಕಾಗ್ರ ಮಾಡಬಹುದು.

ಶಿಷ್ಯ: ಶಾಸ್ತ್ರದಲ್ಲಿ ವಿಷಯ ಮತ್ತು ನಿರ್ವಿಷಯ ಎಂಬ ಭೇದವನ್ನಿಟ್ಟುಕೊಂಡು ಎರಡು ವಿಧವಾದ ಧ್ಯಾನ ಹೇಳಲ್ಪಟ್ಟಿದೆಯಲ್ಲಾ ಅದರ ಅರ್ಥವೇನು? ಅವುಗಳಲ್ಲಿ ಯಾವುದು ಶ್ರೇಷ್ಠ?

ಸ್ವಾಮೀಜಿ: ಮೊದಲು ಯಾವುದಾದರೂ ವಿಷಯವನ್ನು ತೆಗೆದುಕೊಂಡು ಧ್ಯಾನವನ್ನು ಅಭ್ಯಾಸ ಮಾಡಬೇಕು. ಒಂದು ಕಾಲದಲ್ಲಿ ನಾನು ಒಂದು ಕಪ್ಪು ಬಟ್ಟಿನಲ್ಲಿ ಮನಸ್ಸನ್ನು ನಿಲ್ಲಿಸುತ್ತಿದ್ದೆನು. ಆಗ ಕೊನೆಯಲ್ಲಿ ಬಟ್ಟು ಕಾಣದೇ ಹೋಗುತ್ತಿತ್ತು. ಅಥವಾ ಮುಂದೆ ಏನಿದೆ ಅದನ್ನು ತಿಳಿದುಕೊಳ್ಳಲಾರದೆ ಹೋಗುತ್ತಿದ್ದೆನು; ಮನಸ್ಸು ಸ್ತಬ್ಧವಾಗಿಬಿಡುತ್ತಿತ್ತು. ಯಾವ ವೃತ್ತಿಯ ತರಂಗವೂ ಉಂಟಾಗುತ್ತಿರಲಿಲ್ಲ - ಗಾಳಿಯಿಲ್ಲದ ಸಾಗರದಂತಿರುತ್ತಿತ್ತು. ಈ ಅವಸ್ಥೆಯಲ್ಲಿ ಅತೀಂದ್ರಿಯ ಸತ್ಯದ ಛಾಯೆಯನ್ನು ಕೊಂಚ ಕೊಂಚ ನೋಡಬಲ್ಲವನಾಗಿದ್ದೆನು. ಅದಕ್ಕೋಸ್ಕರವೆ, ಯಾವುದಾದರೊಂದು ಸಾಮಾನ್ಯವಾದ ಬಾಹ್ಯ ವಿಷಯವನ್ನು ಇಟ್ಟುಕೊಂಡು ಧ್ಯಾನವನ್ನು ಅಭ್ಯಾಸ ಮಾಡಿದರೂ ಮನಸ್ಸು ಏಕಾಗ್ರ ಅಥವಾ ಧ್ಯಾನಸ್ಥವಾಗುತ್ತದೆಂದು ತೋರುತ್ತದೆ. ಆದರೆ ಯಾವುದರಲ್ಲಿ ಯಾರ ಮನಸ್ಸು ನಿಲ್ಲುತ್ತದೆಯೋ ಅದನ್ನು ಅವರು ಅವಲಂಬಿಸಿಕೊಂಡು ಧ್ಯಾನವನ್ನು ಅಭ್ಯಾಸ ಮಾಡಿದರೆ ಮನಸ್ಸು ಶೀಘ್ರದಲ್ಲಿ ಸ್ಥಿರವಾಗಿ ಹೋಗುತ್ತದೆ. ಅದಕ್ಕೋಸ್ಕರವೆ ಈ ದೇಶದಲ್ಲಿ ಇಷ್ಟು ದೇವ ದೇವೀ ಮೂರ್ತಿಗಳ ಪೂಜೆ, ಅಲ್ಲದೆ ಈ ದೇವ ದೇವಿಯರ ಪೂಜೆಯಿಂದ ಎಂಥ ಶಿಲ್ಪಕಲೆಯ ಉನ್ನತಿ ಆಯಿತು! ಈಗ ಆ ಮಾತು ಹಾಗಿರಲಿ, ಈಗ್ಗೆ ಬೇಕಾದ ವಿಷಯವೇನೆಂದರೆ, ಧ್ಯಾನದ ಬಾಹ್ಯದ ಅವಲಂಬನೆ ಎಲ್ಲರಿಗೂ ಸಮಾನ ಅಥವಾ ಒಂದೇ ಆಗುವುದು ಸಾಧ್ಯವಿಲ್ಲ. ಯಾರು ಯಾವ ವಿಷಯವನ್ನು ಅವಲಂಬಿಸಿಕೊಂಡು ಧ್ಯಾನಸಿದ್ಧನಾಗುತ್ತಾನೆಯೋ ಆತನು ಆ ಬಾಹ್ಯಾವಲಂಬನವನ್ನೇ ವರ್ಣನೆ ಮಾಡುತ್ತಲೂ ಪ್ರಚಾರಮಾಡುತ್ತಲೂ ಹೋಗುವನು. ಆಮೇಲೆ ಕ್ರಮೇಣ ಅದರಿಂದ ಮನಸ್ಸನ್ನು ಸ್ಥಿರಪಡಿಸಿಕೊಳ್ಳಬೇಕೆಂಬ ಸಂಗತಿ ಮರೆತುಹೋಗಿ ಆ ಬಾಹ್ಯಾವಲಂಬನವೆ ದೊಡ್ಡದಾಗಿ ಕುಳಿತುಕೊಳ್ಳುತ್ತದೆ. ಉಪಾಯವನ್ನು ಹಿಡಿದುಕೊಂಡೆ ಜನರು ಒದ್ದಾಡುತ್ತಾ ಕುಳಿತಿದ್ದಾರೆ, ಉದ್ದೇಶದ ಕಡೆಗೆ ಲಕ್ಷ್ಯವೇ ಕಡಿಮೆಯಾಗಿಹೋಗಿದೆ. ಉದ್ದೇಶವೇನೆಂದರೆ ಮನಸ್ಸನ್ನು ವೃತ್ತಿಶೂನ್ಯ ಮಾಡುವುದು. ಆದರೆ, ಯಾವುದಾದರೂ ಒಂದು ವಿಷಯದಲ್ಲಿ ತನ್ಮಯನಾಗಿಹೋಗದೆ ಅದು ಸಾಧ್ಯವಿಲ್ಲ.

ಶಿಷ್ಯ: ಮನೋವೃತ್ತಿಯು ವಿಷಯಾಕಾರವನ್ನು ಪಡೆದರೆ ಅದರಿಂದ ಬ್ರಹ್ಮವು ಅನುಭವಕ್ಕೆ ಬರುವುದು ಹೇಗೆ?

ಸ್ವಾಮೀಜಿ: ವೃತ್ತಿ ಮೊದಲು ವಿಷಯಾಕಾರವಾಗಿರುವುದೇನೋ ನಿಜ; ಆದರೆ ಆಮೇಲೆ ಈ ವಿಷಯದ ಜ್ಞಾನವಿರುವುದಿಲ್ಲ. ಆಮೇಲೆ ಶುದ್ಧ “ಅಸ್ತಿ" (ಇದೆ) ಎಂಬುದಿಷ್ಟು ಮಾತ್ರ ಜ್ಞಾನವಿರುತ್ತದೆ.

ಶಿಷ್ಯ: ಮಹಾಶಯರೇ, ಮನಸ್ಸಿಗೆ ಏಕಾಗ್ರತೆಯುಂಟಾದರೂ ಕಾಮನೆ, ವಾಸನಾದಿಗಳು ಬರುತ್ತವೆ, ಏಕೆ?

ಸ್ವಾಮೀಜಿ: ಅವು ಹಿಂದಿನ ಸಂಸ್ಕಾರದಿಂದ ಬರುತ್ತವೆ. ಬುದ್ಧದೇವನು ಸಮಾಧಿಸ್ಥನಾಗಲು ಹೋದಾಗ ಮಾರನು ತಲೆ ಹಾಕುತ್ತಿದ್ದನು. ಮಾರನೆಂದರೆ ಹೊರಗೆ ಏನೂ ಇರುತ್ತಿರಲಿಲ್ಲ. ಮನಸ್ಸಿನ ಪೂರ್ವ ಸಂಸ್ಕಾರವೆ ಛಾಯಾರೂಪವಾಗಿ ಹೊರಗೆ ಕಾಣುತ್ತಿತ್ತು.

ಶಿಷ್ಯ: ಹಾಗಾದರೆ ಸಿದ್ಧನಾಗುವುದಕ್ಕೆ ಮುಂಚೆ ನಾನಾ ಹೆದರಿಕೆಯನ್ನು ಉಂಟುಮಾಡುವ ನಾನಾ ರೂಪಗಳು ಕಂಡುಬರುತ್ತವೆಯೆಂದು ಕೇಳಿದ್ದೇವೆಯಲ್ಲಾ, ಅವೇನು ಮನಃಕಲ್ಪಿತವೇ?

ಸ್ವಾಮಿಜಿ: ಅಲ್ಲದೆ ಮತ್ತೇನು? ಸಾಧಕನು ಆಗ ಇವು ತನ್ನ ಮನಸ್ಸಿನ ಬಹಿಃಪ್ರಕಾಶವೇ ಎಂಬುದನ್ನು ತಿಳಿದುಕೊಳ್ಳಲಾರನು. ಆದರೆ ಹೊರಗೆ ಏನೂ ಇರುವುದಿಲ್ಲ. ನೀನು ನೋಡುತ್ತಿರುವ ಈ ಜಗತ್ತೂ ಇರುವುದಿಲ್ಲ. ಎಲ್ಲಾ ಮನಸ್ಸಿನ ಕಲ್ಪನೆ. ಮನಸ್ಸು ಯಾವಾಗ ವೃತ್ತಿಶೂನ್ಯವಾಗುತ್ತದೆಯೋ ಆವಾಗ ಅದರಿಂದ ಬ್ರಹ್ಮಾಭಾಸದ ದರ್ಶನವುಂಟಾಗುತ್ತದೆ. ಆಗ ‘ಯಂ ಯಂ ಲೋಕಂ ಮನಸಾ ಸಂವಿಭಾತಿ’ (ಮುಂಡಕ, ೩.೧.೧೦) ಆ ಆ ಲೋಕದ ದರ್ಶನವಾಗುತ್ತದೆ. ಯಾವುದನ್ನು ಸಂಕಲ್ಪ ಮಾಡಿದರೂ ಅದು ಸಿದ್ಧವಾಗುತ್ತದೆ. ಈ ವಿಧವಾದ ಸತ್ಯಸಂಕಲ್ಪಾವಸ್ಥೆಯುಂಟಾದರೂ ಮನಸ್ಸನ್ನು ಅಧೀನದಲ್ಲಿ ಇಟ್ಟುಕೊಂಡು ಯಾವ ಆಸೆಗೂ ದಾಸನಾಗದೆ ಇರಬಲ್ಲವನಾದರೆ ಅಂಥವನು ಬ್ರಹ್ಮಜ್ಞಾನವನ್ನು ಪಡೆಯುತ್ತಾನೆ. ಈ ಅವಸ್ಥೆಯನ್ನು ಪಡೆದು ಯಾರು ಚಂಚಲವಾಗುತ್ತಾರೋ ಅವರು ನಾನಾ ಸಿದ್ಧಿಗಳನ್ನು ಪಡೆದು ಪರಮಾರ್ಥದಿಂದ ಭ್ರಷ್ಟರಾಗುತ್ತಾರೆ.

ಈ ಮಾತನ್ನು ಹೇಳುತ್ತ ಹೇಳುತ್ತ ಸ್ವಾಮೀಜಿ ‘ಶಿವ’ ‘ಶಿವ’ ಎಂದು ಶಿವನಾಮವನ್ನು ಉಚ್ಚಾರಣೆ ಮಾಡತೊಡಗಿದರು. ಕೊನೆಗೆ ಹೇಳಿದ್ದೇನೆಂದರೆ “ತ್ಯಾಗವನ್ನು ಬಿಟ್ಟು ಮತ್ತಾವುದರಿಂದಲೂ ಈ ಗಂಭೀರವಾದ ಜೀವನ ಸಮಸ್ಯೆಯ ರಹಸ್ಯವನ್ನು ಭೇದಿಸುವುದಕ್ಕಾಗುವುದಿಲ್ಲ. ತ್ಯಾಗ-ತ್ಯಾಗ-ತ್ಯಾಗ! ಇದೇ ನಿಮ್ಮ ಜೀವನದ ಮೂಲಮಂತ್ರವಾಗಿರಲಿ." ‘ಸರ್ವಂ ವಸ್ತು ಭಯಾನ್ವಿತಂ ಭುವಿ ನೃಣಾಂ ವೈರಾಗ್ಯಮೇವಾಭಯಂ.’ (ವೈರಾಗ್ಯಶತಕ) - ಈ ಜಗತ್ತಿನಲ್ಲಿ ಎಲ್ಲವೂ ಭಯದಿಂದ ಕೂಡಿರುವುದು. ವೈರಾಗ್ಯವೊಂದೇ ಭಯರಹಿತವಾದುದು.

\newpage

\chapter[ಅಧ್ಯಾಯ ೯]{ಅಧ್ಯಾಯ ೯\protect\footnote{\enginline{C.W, Vol. VI, P. 488}}}

\begin{center}
ಸ್ಥಳ: ಕಲ್ಕತ್ತ, ವರ್ಷ: ಕ್ರಿ.ಶ. ೧೮೯೭, ಮಾರ್ಚಿ ಅಥವಾ ಏಪ್ರಿಲ್.
\end{center}

ಸ್ವಾಮೀಜಿ ಅಮೆರಿಕಾದಿಂದ ಹಿಂತಿರುಗಿ ಬಂದ ಕೆಲವು ದಿನ ಕಲ್ಕತ್ತೆಯಲ್ಲೇ ಇರುತ್ತಿದ್ದರು - ಬಾಗ್‌ಬಜಾರಿನ ಬಲರಾಮ ಬಸು ಮಹಾಶಯರ ಮನೆಯಲ್ಲಿಯೇ ಇದ್ದರು. ಮಧ್ಯೆ ಮಧ್ಯೆ ಪರಿಚಿತರಾದ ಜನರ ಮನೆಗೂ ಹೋಗಿ ಬರುತ್ತಿದ್ದರು. ಈ ದಿವಸ ಬೆಳಿಗ್ಗೆ ಶಿಷ್ಯನು ಸ್ವಾಮಿಜಿ ಬಳಿಗೆ ಬಂದು ನೋಡಲು, ಸ್ವಾಮಿಜಿ ಹೊರಗೆ ಎಲ್ಲಿಗೊ ಹೋಗುವುದಕ್ಕೆ ಸಿದ್ಧರಾಗಿದ್ದರು. ಶಿಷ್ಯನಿಗೆ “ನಡೆ, ನನ್ನ ಜೊತೆಯಲ್ಲಿ ಬರುವೆಯಂತೆ" ಎಂದು ಹೇಳಿದರು. ಹೇಳುತ್ತ ಹೇಳುತ್ತ ಕೆಳಗೆ ಇಳಿದರು. ಶಿಷ್ಯನು ಹಿಂದೆಹಿಂದೆಯೆ ಹೋದನು. ಅವರು ಒಂದು ಬಾಡಿಗೆಯ ಗಾಡಿಯೊಳಕ್ಕೆ ಶಿಷ್ಯನೊಡನೆ ಹೋಗಿ ಕುಳಿತುಕೊಂಡರು; ಗಾಡಿ ದಕ್ಷಿಣಾಭಿಮುಖವಾಗಿ ಹೊರಟಿತು.

ಶಿಷ್ಯ: ಮಹಾಶಯರೆ ಎಲ್ಲಿಗೆ ಹೋಗಬೇಕು?

ಸ್ವಾಮೀಜಿ: ನಡೆ, ಗೊತ್ತಾಗುತ್ತದೆ ಈವಾಗ.

ಹೀಗೆ ಸ್ವಾಮಿಜಿ ಎಲ್ಲಿಗೆ ಹೋಗಬೇಕೆಂಬುದರ ವಿಚಾರವಾಗಿ ಶಿಷ್ಯನಿಗೆ ಏನನ್ನೂ ಹೇಳದೆ, ಗಾಡಿಯು ಬೀಡನ್ ರಸ್ತೆಗೆ ಬಂದಾಗ ಮಾತಿಗೆ ಮಾತು ಬಂದು ಹೀಗೆಂದರು: “ನಿಮ್ಮ ದೇಶದಲ್ಲಿ ಹೆಂಗಸರಿಗೆ ಓದು ಬರಹವನ್ನು ಕಲಿಸುವುದಕ್ಕಾಗಿ ಸ್ವಲ್ಪ ಪ್ರಯತ್ನವೂ ಕಂಡುಬರುವುದಿಲ್ಲ. ನೀವು ಓದುಬರಹಗಳನ್ನು ಕಲಿತು ದೊಡ್ಡವರಾಗುತ್ತೀರಿ. ಆದರೆ ಯಾರು ನಿಮ್ಮ ಸುಖಗಳಲ್ಲಿ ಭಾಗಿಗಳೊ - ಸಕಲಕಾಲದಲ್ಲಿಯೂ ಮನಃಪೂರ್ವಕವಾಗಿ ಸೇವೆ ಮಾಡುತ್ತಾರೆಯೋ, ಅವರಿಗೆ ಶಿಕ್ಷಣ ಕೊಡುವುದಕ್ಕೆ - ಅವರಿಗೆ ಉನ್ನತಿಯುಂಟುಮಾಡಿಕೊಡುವುದಕ್ಕೆ ನೀವು ಏನು ಮಾಡುತ್ತೀರಿ?"

ಶಿಷ್ಯ: ಏಕೆ ಮಹಾಶಯರೆ! ಈಗಿನ ಕಾಲದಲ್ಲಿ ಹೆಂಗಸರಿಗೋಸ್ಕರ ಎಷ್ಟು ಸ್ಕೂಲುಗಳೂ ಕಾಲೇಜುಗಳೂ ಇವೆ. ಎಷ್ಟು ಹೆಂಗಸರು ಬಿ.ಎ., ಎಂ.ಎ. ಪ್ಯಾಸು ಮಾಡುತ್ತಿದ್ದಾರೆ!

ಸ್ವಾಮೀಜಿ: ಅದು ವಿಲಾಯಿತಿಯ ರೀತಿ! ನಿಮ್ಮ ಧರ್ಮಶಾಸ್ತ್ರಾನುಸಾರವಾಗಿ ನಿಮ್ಮ ದೇಶಾನುಸಾರವಾಗಿ ನಡೆಯುವ ಸ್ಕೂಲು ಎಲ್ಲೆಲ್ಲಿ ಎಷ್ಟೆಷ್ಟಿವೆ? ದೇಶದಲ್ಲಿ ಗಂಡಸರಿಗೂ ಇಂಥ ಶಿಕ್ಷಣ ಪ್ರಚಾರಕ್ಕೆ ಬಂದಿಲ್ಲ. ಹೀಗಿರಲು ಹೆಂಗಸರಲ್ಲಿ ಕೇಳಬೇಕೆ! ಸರ್ಕಾರದ ಲೆಕ್ಕದ ಪಟ್ಟಿಯನ್ನು ನೋಡಿದರೆ ಭರತಖಂಡದಲ್ಲಿ ಶೇಕಡ ಹತ್ತು ಹನ್ನೆರಡು ಜನ ಮಾತ್ರ ಶಿಕ್ಷಿತರೆಂದು ಗೊತ್ತಾಗುತ್ತದೆ. ಹೆಂಗಸರಲ್ಲಿ ಶೇಕಡ ಒಬ್ಬರು ಇದ್ದಾರೆಯೊ ಇಲ್ಲವೊ ಎಂದು ತೋರುತ್ತದೆ.

ಹಾಗಿಲ್ಲದಿದ್ದರೆ ದೇಶಕ್ಕೆ ಇಂಥ ದುರ್ದೆಶೆಯುಂಟಾಗುತ್ತಿತ್ತೆ? ಶಿಕ್ಷಣ ಪ್ರಚಾರ, ಜ್ಞಾನೋದಯ, ಇವೆಲ್ಲಾ ಆಗದಿದ್ದರೆ ದೇಶದ ಉನ್ನತಿ ಹೇಗೆ ಆಗುತ್ತದೆ? ದೇಶದಲ್ಲಿ ನೀವು ಕೆಲವರು ಓದುಬರಹಗಳನ್ನು ಕಲಿತುಕೊಂಡಿದ್ದೀರಷ್ಟೆ - ದೇಶದ ಭವಿಷ್ಯದಾಚೆಗೆ ಕಾರಣರಾದವರು - ಈ ಕೆಲವು ಜನರಲ್ಲಿಯೂ ಈ ವಿಷಯವಾಗಿ ಯಾವ ಕಾರ್ಯವಾಗಲಿ ಪ್ರಯತ್ನವಾಗಲಿ ಕಂಡುಬರುವುದಿಲ್ಲ. ಆದರೆ ಇದನ್ನು ತಿಳಿದುಕೊ, ಜನಸಾಧಾರಣರಲ್ಲಿಯೂ ಹೆಂಗಸರಲ್ಲಿಯೂ ಶಿಕ್ಷಣಪ್ರಚಾರವಾಗದಿದ್ದರೆ ಏನೂ ಸಾಧ್ಯವಿಲ್ಲ. ಆದ್ದರಿಂದ ನನ್ನ ಇಷ್ಟವೇನೆಂದರೆ - ಕೆಲವು ಜನ ಬ್ರಹ್ಮಚಾರಿ ಮತ್ತು ಬ್ರಹ್ಮಚಾರಿಣಿಯರನ್ನು ತಯಾರುಮಾಡಬೇಕು. ಬ್ರಹ್ಮಚಾರಿಗಳು ಸಕಾಲದಲ್ಲಿ ಸಂನ್ಯಾಸವನ್ನು ಪಡೆದು ದೇಶದ ಎಲ್ಲಾ ಹಳ್ಳಿ ಹಳ್ಳಿಗೂ ಹೋಗಿ ಸಾಧಾರಣ ಜನರಿಗೆ ಶಿಕ್ಷಣ ಕೊಡುವುದಕ್ಕೆ ಪ್ರಯತ್ನಿಸಬೇಕು. ಬ್ರಹ್ಮಚಾರಿಣಿಯರೂ, ಹೆಂಗಸರಲ್ಲಿ ಶಿಕ್ಷಣವನ್ನು ಪ್ರಚಾರಮಾಡಬೇಕು. ಆದರೆ ದೇಶೀಯ ರೀತಿಯಲ್ಲಿ ಈ ಕೆಲಸವನ್ನು ಮಾಡಬೇಕು. ಗಂಡಸರಿಗೋಸ್ಕರ ಹೇಗೆ ಕೆಲವು ಶಿಕ್ಷಣ ಕೇಂದ್ರಗಳನ್ನು ಮಾಡಬೇಕೊ, ಹೆಂಗಸರಿಗೋಸ್ಕರವೂ ಹಾಗೆಯೆ ಕೆಲವು ಕೇಂದ್ರಗಳನ್ನು ಮಾಡಬೇಕು. ಶಿಕ್ಷಿತೆಯರೂ ಸಚ್ಚರಿತ್ರೆಯರೂ ಆದ ಬ್ರಹ್ಮಚಾರಿಣಿಯರು ಈ ಕೇಂದ್ರಗಳಲ್ಲಿ ಹೆಂಗಸರ ಶಿಕ್ಷಣದ ಭಾರವನ್ನು ವಹಿಸಬೇಕು. ಪುರಾಣ ಇತಿಹಾಸ ಗೃಹಕಾರ್ಯ ಶಿಲ್ಪ ಗೃಹಕೃತ್ಯನಿಯಮಗಳು ಆದರ್ಶಜೀವನವನ್ನು ಬೆಳೆಸುವುದಕ್ಕೆ ಸಹಕಾರಿಗಳಾದ ನೀತಿಗಳು – ಇವುಗಳ ವಿಚಾರವಾಗಿ ಈಗಿನ ವಿಜ್ಞಾನದ ಸಹಾಯದಿಂದ ಶಿಕ್ಷಣ ಕೊಡಬೇಕು. ಶಿಷ್ಯರನ್ನು ಧರ್ಮಪರಾಯಣರನ್ನಾಗಿಯೂ ನೀತಿಪರಾಯಣರನ್ನಾಗಿಯೂ ಮಾಡಬೇಕು. ಮುಂದೆ ಅವರು ಏತರಿಂದ ಒಳ್ಳೆಯ ಗೃಹಿಣಿಯರಾಗುವರೊ ಅದನ್ನೇ ಹೇಳಿಕೊಡಬೇಕು. ಈ ಸ್ತ್ರೀಯರ ಸಂತತಿಗಳು ಆಮೇಲೆ ಈ ವಿಷಯಗಳಲ್ಲಿ ಮತ್ತೂ ಉನ್ನತಿಯನ್ನು ಪಡೆಯಲು ಸಮರ್ಥರಾಗುವರು. ಯಾವ ತಾಯಿಯು ಸುಶಿಕ್ಷಿತಳೂ ನೀತಿಪರಾಯಣಳಾಗಿಯೂ ಇರುತ್ತಾಳೆಯೋ ಅವರ ಮನೆಯಲ್ಲಿಯೇ ಶ್ರೇಷ್ಠರಾದ ಜನರು ಹುಟ್ಟುವರು. ಈಗ ನೀವು ಹೆಂಗಸರನ್ನು ಕೆಲಸಮಾಡುವ ಯಂತ್ರಗಳನ್ನಾಗಿ ಮಾಡಿದ್ದೀರಿ. ರಾಮರಾಮ! ಇದೇ ಏನು ನಿಮ್ಮ ಶಿಕ್ಷಣದ ಫಲ? ಮೊದಲು ಹೆಂಗಸರನ್ನು ಉತ್ತಮ ಸ್ಥಿತಿಗೆ ತರಬೇಕು. ಸಾಧಾರಣ ಜನರನ್ನು ಎಚ್ಚರಗೊಳಿಸಬೇಕು; ಹಾಗಾದರೆ ತಾನೆ ದೇಶಕ್ಕೆ ಕಲ್ಯಾಣ, ಭರತಖಂಡಕ್ಕೆ ಕಲ್ಯಾಣ.

ಗಾಡಿ ಈಗ ಕಾರ್ನವಾಲೀಸ್ ರಸ್ತೆಯ ಬ್ರಹ್ಮಸಮಾಜವನ್ನು ಬಿಟ್ಟು ಮುಂದಕ್ಕೆ ಹೋಗುತ್ತಿದ್ದುದನ್ನು ನೋಡಿ ಸ್ವಾಮಿಜಿ ಗಾಡಿಯವನಿಗೆ ‘ಚೋರಬಾಗಾನಿನ ರಸ್ತೆಯಲ್ಲಿ ಹೊಡಿ’ ಎಂದು ಹೇಳಿದರು. ಗಾಡಿ ಆ ರಸ್ತೆಗೆ ಪ್ರವೇಶ ಮಾಡಿದಾಗ ಸ್ವಾಮೀಜಿ ಮಹಾಕಾಳಿ ಪಾಠ ಶಾಲೆಯ ಸ್ಥಾಪಕಳಾದ ತಪಸ್ವಿನೀಮಾತೆ ತಮ್ಮ ಪಾಠಶಾಲೆಯನ್ನು ಬಂದುನೋಡಬೇಕೆಂದು ಆಹ್ವಾನಮಾಡಿ ಚೀಟಿ ಬರೆದು ಕಳುಹಿಸಿದುದಾಗಿ ಬಾಯಿಬಿಟ್ಟು ಹೇಳಿದರು. ಈ ಪಾಠಶಾಲೆ ಆಗ ಚೋರಬಾಗಾನಿನ ರಾಜೇಂದ್ರಮಲ್ಲಿಕ ಮಹಾಶಯರ ಮನೆಗೆ ಸ್ವಲ್ಪ ಪೂರ್ವದಲ್ಲಿದ್ದ ಎರಡಂತಸ್ತಿನ ಒಂದು ಬಾಡಿಗೆಯ ಮನೆಯಲ್ಲಿತ್ತು. ಗಾಡಿ ನಿಲ್ಲಲು ನಾಲ್ಕೈದು ಜನ ದೊಡ್ಡ ಮನುಷ್ಯರು ಅವರಿಗೆ ಪ್ರಣಾಮಮಾಡಿ ಮೇಲಕ್ಕೆ ಕರೆದುಕೊಂಡು ಹೋದರು. ತಪಸ್ವಿನೀಮಾತೆ ಅವರನ್ನು ಪ್ರೀತಿಪೂರ್ವಕ ಬರಮಾಡಿಕೊಂಡರು. ಸ್ವಲ್ಪ ಹೊತ್ತಿನ ಮೇಲೆ ತಪಸ್ವಿನೀಮಾತೆ ಸ್ವಾಮಿಜಿಯನ್ನು ತರಗತಿಗೆ ತಮ್ಮ ಜೊತೆಯಲ್ಲಿ ಕರೆದುಕೊಂಡು ಹೋದರು. ಹುಡುಗಿಯರು ಎದ್ದು ನಿಂತು ಸ್ವಾಮೀಜಿಗೆ ಪ್ರಣಾಮಮಾಡಿದರು. ಮತ್ತೆ ಮಾತೆಯ ಅಪ್ಪಣೆಯಂತೆ ಮೊದಲು ರಾಗವಾಗಿ ಶಿವಸ್ತೋತ್ರವನ್ನು ಹೇಳುವುದಕ್ಕೆ ಮೊದಲು ಮಾಡಿದರು. ಆಮೇಲೆ ಪೂಜೆ ಮುಂತಾದ ವಿಷಯಗಳಲ್ಲಿ ಪಾಠಶಾಲೆಯಲ್ಲಿ ಶಿಕ್ಷಣ ಹೇಗೆ ಕೊಡಲ್ಪಡುತ್ತಿತ್ತೆಂಬುದನ್ನು ಮಾತಾಜಿಯವರ ಅಪ್ಪಣೆ ಮೇರೆ ಹುಡುಗಿಯರು ನಡೆಸಿ ತೋರಿಸಿದರು. ಸ್ವಾಮೀಜಿ ಉತ್ಫುಲ್ಲ ನಯನದಿಂದ ಇವೆಲ್ಲವನ್ನೂ ನೋಡಿ ಮತ್ತೊಂದು ತರಗತಿಯ ವಿದ್ಯಾರ್ಥಿನಿಯರನ್ನು ನೋಡುವುದಕ್ಕೆ ಹೊರಟರು. ವೃದ್ಧೆಯಾದ ಮಾತಾಜಿಯು ಸ್ವಾಮೀಜಿ ಜೊತೆಯಲ್ಲಿ ಎಲ್ಲ ತರಗತಿಗಳಿಗೂ ತಿರುಗಲಾರೆನೆಂದು ಆ ಪಾಠಶಾಲೆಯ ಇಬ್ಬರು ಮೂರುಜನ ಉಪಾಧ್ಯಾಯರನ್ನು ಕರೆದು ಎಲ್ಲಾ ತರಗತಿಗಳನ್ನೂ ಸ್ವಾಮೀಜಿಗೆ ಚೆನ್ನಾಗಿ ತೋರಿಸುವಂತೆ ಹೇಳಿದರು. ಅನಂತರ ಸ್ವಾಮೀಜಿ ಎಲ್ಲಾ ತರಗತಿಗೂ ಹೋಗಿ ಮತ್ತೆ ಮಾತಾಜಿಯ ಬಳಿಗೆ ಬರಲು, ಆಕೆಯು ಒಬ್ಬ ಹುಡುಗಿಯನ್ನು ಅಲ್ಲಿಗೆ ಕರೆಸಿಕೊಂಡು ರಘುವಂಶದ ಮೂರನೆಯ ಸರ್ಗದಲ್ಲಿ ಮೊದಲನೆಯ ಶ್ಲೋಕಕ್ಕೆ ಅರ್ಥ ಹೇಳುವಂತೆ ಹೇಳಿದರು. ವಿದ್ಯಾರ್ಥಿನಿ ಅದಕ್ಕೆ ಸಂಸ್ಕೃತದಲ್ಲಿಯೆ ಅರ್ಥವನ್ನು ಹೇಳಿ ಸ್ವಾಮೀಜಿಗೆ ಒಪ್ಪಿಸಿದಳು. ಸ್ವಾಮಿಜಿ ಕೇಳಿ ತಮ್ಮ ಸಂತೋಷವನ್ನು ಸೂಚಿಸಿ, ಶಿಕ್ಷಣ ಪ್ರಚಾರದಲ್ಲಿ ಮಾತಾಜಿಯ ಪ್ರಯತ್ನ ಮತ್ತು ವ್ಯವಸಾಯಗಳು ಇಷ್ಟು ದೂರ ಸಫಲವಾದದ್ದನ್ನು ನೋಡಿ ಅವರನ್ನು ವಿಶೇಷವಾಗಿ ಹೊಗಳುವುದಕ್ಕೆ ಮೊದಲುಮಾಡಿದರು. ಅದಕ್ಕೆ ಮಾತಾಜಿಯು ವಿನೀತಭಾವದಿಂದ ‘ನಾನು ಭಗವತೀ ಬುದ್ಧಿಯನ್ನಿಟ್ಟುಕೊಂಡು ವಿದ್ಯಾರ್ಥಿನಿಯರ ಸೇವೆಯನ್ನು ಮಾಡಿಕೊಂಡಿದ್ದೇನೆ. ಅಷ್ಟು ಹೊರತು ಪಾಠಶಾಲೆಯನ್ನಿಟ್ಟು ಯಶಸ್ಸು ಪಡೆಯುವುದಾಗಲಿ ಮತ್ತಾವ ಉದ್ದೇಶವಾಗಲಿ ಇಲ್ಲ’ ಎಂದರು.

ವಿದ್ಯಾಲಯ ಸಂಬಂಧವಾಗಿ ಮಾತುಕತೆಗಳನ್ನು ಮುಗಿಸಿ ಸ್ವಾಮೀಜಿ ಹೊರಡುವುದಕ್ಕೆ ಸಿದ್ಧರಾಗಲು ಮಾತಾಜಿಯವರ ಕೋರಿಕೆಯ ಮೇರೆಗೆ, ಪಾಠಶಾಲೆಯ ವಿಚಾರವಾಗಿ ದರ್ಶಕರ ಅಭಿಪ್ರಾಯವನ್ನು ವಿಸ್ತಾರವಾಗಿ ಬರೆದಿಟ್ಟರು. ಬರೆದ ವಿಷಯದಲ್ಲಿ ಕೊನೆಯ ಪಂಕ್ತಿಯು ಶಿಷ್ಯನಿಗೆ ಈಗಲೂ ಮನಸ್ಸಿನಲ್ಲಿ ನಿಂತಿದೆ. ಅದೇನೆಂದರೆ “ಕೆಲಸ ಸರಿಯಾದ ಮಾರ್ಗದಲ್ಲಿದೆ."

ಅನಂತರ ಮಾತಾಜಿಗೆ ಅಭಿವಾದನ ಮಾಡಿದ ಮೇಲೆ ಸ್ವಾಮೀಜಿ ಮತ್ತೆ ಗಾಡಿಯೊಳಗೆ ಕುಳಿತುಕೊಂಡು ಶಿಕ್ಷಣ ಸಂಬಂಧವಾಗಿ ನಾನಾ ಕಥೋಪಕಥನಗಳನ್ನು ಮಾಡುತ್ತ ಮಾಡುತ್ತ ಬಾಗಬಜಾರಿನ ಕಡೆಗೆ ಹೊರಟರು. ಈ ಕೆಳಗೆ ಬರೆದಿರುವುದು ಅದರ ಅಲ್ಪಸ್ವಲ್ಪ ವಿವರಣೆ.

ಸ್ವಾಮೀಜಿ: ಈಕೆಯು (ಮಾತಾಜಿ) ಹುಟ್ಟಿದ್ದು ಎಲ್ಲಿ! ಸರ್ವಸ್ವವನ್ನೂ ತ್ಯಾಗ ಮಾಡಿದ್ದಾಳೆ! ಆದರೆ ಲೋಕಹಿತಾರ್ಥವಾಗಿ ಎಷ್ಟು ಪ್ರಯತ್ನ! ಹೆಂಗಸರಲ್ಲದಿದ್ದರೆ ಹುಡುಗಿಯರಿಗೆ ಹೀಗೆ ಶಿಕ್ಷಣ ಕೊಡುವುದಕ್ಕಾಗುತ್ತದೆಯೇನು? ನಾನು ನೋಡಿದ್ದು ಎಲ್ಲಾ ಚೆನ್ನಾಗಿತ್ತು. ಆದರೆ ಕೆಲವು ಸಂಸಾರಿಗಳಾದ ಗಂಡಸರು ಉಪಾಧ್ಯಾಯರು ಗಳಾಗಿದ್ದಾರಲ್ಲಾ ಇದುಮಾತ್ರ ಚೆನ್ನಾಗಿ ತೋರುವುದಿಲ್ಲ. ಶಿಕ್ಷಿತರಾದ ವಿಧವೆಯರು ಮತ್ತು ಬ್ರಹ್ಮಚಾರಿಣಿಯರು ಇವರ ಮೇಲೆಯೇ ಪಾಠಶಾಲೆಯ ಶಿಕ್ಷಣದ ಭಾರವನ್ನು ಪೂರ್ತಿಯಾಗಿ ಹಾಕುವುದು ಉಚಿತ. ಈ ದೇಶದಲ್ಲಿ ಸ್ತ್ರೀ ವಿದ್ಯಾಲಯದಲ್ಲಿ ಗಂಡಸರನ್ನು ಸೇರಿಸದೆ ಇರುವುದೇ ಒಳ್ಳೆಯದು.

ಶಿಷ್ಯ: ಆದರೆ, ಮಹಾಶಯರೇ, ಗಾರ್ಗಿ, ಖನಾ, ಲೀಲಾವತಿ ಇವರ ಹಾಗೆ ಗುಣವತಿಯರಾದ ಶಿಕ್ಷಿತ ಸ್ತ್ರೀಯರು ದೇಶದಲ್ಲಿ ಎಲ್ಲಿ ಸಿಕ್ಕುತ್ತಾರೆ?

ಸ್ವಾಮೀಜಿ: ದೇಶದಲ್ಲಿ ಈಗಲೂ ಇಂಥ ಹೆಂಗಸರಿಲ್ಲವೆ? ಈ ಸೀತಾ ಸಾವಿತ್ರಿಯರ ದೇಶದಲ್ಲಿ, ಪುಣ್ಯಕ್ಷೇತ್ರವಾದ ಭರತಖಂಡದಲ್ಲಿ, ಈಗಲೂ ಹೆಂಗಸರಲ್ಲಿ ಎಂಥ ಗುಣ, ಸೇವಾಭಾವ, ಸ್ನೇಹ, ದಯೆ, ತೃಪ್ತಿ ಮತ್ತು ಭಕ್ತಿ ಇವು ಕಂಡು ಬರುತ್ತದೆಯೋ ಅವು ಅಷ್ಟರಮಟ್ಟಿಗೆ ಪೃಥ್ವಿಯಲ್ಲಿ ಮತ್ತೆಲ್ಲಿಯೂ ಇರುವುದನ್ನು ನಾನು ನೋಡಿಲ್ಲ. ಆ ಪಾಶ್ಚಾತ್ಯ ದೇಶದ ಹೆಂಗಸರನ್ನು ನೋಡಿದರೆ ಅನೇಕವೇಳೆ ನನಗೆ ಅವರನ್ನು ಹೆಂಗಸರೆಂದೇ ಹೇಳುವುದಕ್ಕಾಗುವುದಿಲ್ಲ. ಅವರು ಮಾಡುವ ಕೆಲಸಗಳು ಎಲ್ಲಾ ಗಂಡಸರ ಹಾಗೆಯೇ? ಗಾಡಿ ಹೊಡೆಯುತ್ತಾರೆ, ಆಫೀಸಿಗೆ ಹೋಗುತ್ತಾರೆ, ಸ್ಕೂಲಿಗೆ ಹೋಗುತ್ತಾರೆ, ಎಲ್ಲಾ ಉದ್ಯೋಗಗಳನ್ನೂ ಮಾಡುತ್ತಾರೆ! ಭರತಖಂಡ ಒಂದರಲ್ಲಿ ಮಾತ್ರ ಹೆಂಗಸರ ಲಜ್ಜೆ ವಿನಯ ಮುಂತಾದುವು ಕಣ್ಣಿಗೆ ಹಿತವನ್ನು ಉಂಟುಮಾಡುತ್ತವೆ. ಇಂಥ ಆಧಾರಗಳೆಲ್ಲ ಇದ್ದರೂ ನೀವು ಇವರನ್ನು ಉನ್ನತರನ್ನಾಗಿ ಮಾಡಲಾರದೆ ಹೋದಿರಿ! ಇವರಿಗೆ ಜ್ಞಾನದ ಬೆಳಕನ್ನು ಕೊಡುವುದಕ್ಕೆ ಪ್ರಯತ್ನ ಮಾಡಲಿಲ್ಲ! ಸರಿಯಾದ ಶಿಕ್ಷಣವನ್ನು ಪಡೆದರೆ ಇವರು ಆದರ್ಶ ಸ್ತ್ರೀಯರಾಗಬಲ್ಲರು.

ಶಿಷ್ಯ: ಮಹಾಶಯರೇ, ಮಾತಾಜಿಯು ವಿದ್ಯಾರ್ಥಿನಿಯರಿಗೆ ಯಾವ ರೀತಿಯಲ್ಲಿ ಜ್ಞಾನ ಶಿಕ್ಷಣವನ್ನು ಕೊಡುತ್ತಿದ್ದಾರೆಯೊ ಅದರಿಂದ ಇಂಥ ಫಲವಾಗುವುದೇನು? ಈ ಹುಡುಗಿಯರೆಲ್ಲಾ ದೊಡ್ಡವರಾಗಿ ಮದುವೆ ಮಾಡಿಕೊಳ್ಳುತ್ತಾರೆ. ಅದಾದಮೇಲೆ ಇನ್ನೂ ಸ್ವಲ್ಪಕಾಲಕ್ಕೆ ಮಿಕ್ಕ ಹೆಂಗಸರ ಹಾಗೇ ಆಗಿಬಿಡುತ್ತಾರೆ. ಇವರು ಬ್ರಹ್ಮಚರ್ಯವನ್ನು ಅವಲಂಬಿಸುವಂತೆ ಮಾಡಬಲ್ಲರಾದರೆ, ಇವರು ಸಮಾಜದ ಮತ್ತು ದೇಶದ ಉನ್ನತಿಗಾಗಿ ಜೀವನವನ್ನು ಧಾರೆಯೆರೆಯುವುದಕ್ಕೂ ಮತ್ತು ಶಾಸ್ರೋಕ್ತವಾದ ಉನ್ನತ ಆದರ್ಶವನ್ನು ಸಾಧಿಸುವುದಕ್ಕೂ ಸಮರ್ಥರಾಗಬಲ್ಲರೆಂದು ನನಗೆ ತೋರುತ್ತದೆ.

ಸ್ವಾಮೀಜಿ: ಕ್ರಮೇಣ ಎಲ್ಲಾ ಆಗುತ್ತದೆ, ಸಮಾಜದಂಡನೆಗೆ ಹೆದರದೆ ತಮ್ಮ ಹೆಣ್ಣುಮಕ್ಕಳನ್ನು ಮದುವೆಮಾಡದೆ ನಿಲ್ಲಿಸಿಕೊಳ್ಳುವಷ್ಟು ಶಿಕ್ಷಿತರಾದ ಜನರು ದೇಶದಲ್ಲಿ ಇನ್ನೂ ಹುಟ್ಟಿಲ್ಲ. ಇದನ್ನೇ ನೋಡು - ಈಗಲೂ ಹುಡುಗಿಯರಿಗೆ ಹನ್ನೆರಡು ಹದಿಮೂರು ವರ್ಷ ತುಂಬಿತೋ ತುಂಬಲಿಲ್ಲವೋ, ಆಗಲೇ ಲೋಕಕ್ಕೆ ಹೆದರಿಕೊಂಡು, ಸಮಾಜಕ್ಕೆ ಹೆದರಿಕೊಂಡು, ಅವರಿಗೆ ಮದುವೆ ಮಾಡಿಬಿಡುತ್ತಾರೆ. ಮೊನ್ನೆ ಮೊನ್ನೆ ವಿವಾಹದ ವಯಸ್ಸಿಗೆ ಸಂಬಂಧಿಸಿದ ಕಾನೂನನ್ನು ಜಾರಿಗೆ ತರಲು ಪ್ರಯತ್ನಿಸಿದಾಗ ಸಮಾಜದ ಮುಖಂಡರು ಲಕ್ಕೋಪಲಕ್ಷ ಜನರನ್ನು ಸೇರಿಸಿ ‘ನಮಗೆ ಆ ಕಾಯಿದೆ ಬೇಡ!’ ಎಂದು ಅರಚಿದರು. ಮತ್ತೊಂದು ದೇಶವಾಗಿದ್ದರೆ, ಸಭೆ ಮಾಡಿ ಅರಚಿಕೊಳ್ಳುವುದು ಹಾಗಿರಲಿ, ನಾಚಿಕೆಯಿಂದ ಮುಖವನ್ನು ತಗ್ಗಿಸಿಕೊಂಡು ಜನರು ಮನೆಯಲ್ಲಿ ಕುಳಿತುಕೊಂಡಿರುತ್ತಿದ್ದರು ಮತ್ತು ಸಮಾಜದಲ್ಲಿ ಈಗಲೂ ಇಂಥ ಕಳಂಕವಿದೆಯೆಲ್ಲಾ ಎಂದುಕೊಳ್ಳುತ್ತಿದ್ದರು.

ಶಿಷ್ಯ: ಹಾಗಾದರೆ, ಮಹಾಶಯರೆ, ಸಂಹಿತಾಕಾರರು ಸ್ವಲ್ಪವೂ ಯೋಚಿಸದೆ ಬಾಲ್ಯವಿವಾಹವನ್ನು ಅನುಮೋದಿಸಿದ್ದಾರೆಯೇನು? ಖಂಡಿತವಾಗಿಯೂ ಅದರೊಳಗೆ ಏನೋ ರಹಸ್ಯವಿದೆ?

ಸ್ವಾಮೀಜಿ: ಏನು ರಹಸ್ಯವಿದೆ?

ಶಿಷ್ಯ: ಇದನ್ನೇ ನೋಡಿ; ಚಿಕ್ಕ ವಯಸ್ಸಿನಲ್ಲಿ ಹೆಂಗಸರಿಗೆ ಮದುವೆ ಮಾಡಿದರೆ ಅವರು ಗಂಡನ ಮನೆಗೆ ಬಂದು ಕುಲಧರ್ಮಗಳನ್ನು ಬಾಲ್ಯಕಾಲದಲ್ಲಿಯೇ ಕಲಿತುಕೊಳ್ಳಬಹುದು. ಅತ್ತೆ ಮಾವಂದಿರ ಆಶ್ರಯದಲ್ಲಿದ್ದು ಗೃಹಕೃತ್ಯದಲ್ಲಿ ನಿಪುಣರಾಗಬಹುದು. ಅದರಿಂದ ಲಜ್ಜೆ, ಸಹಿಷ್ಣುತೆ, ಶ್ರಮಶೀಲತೆ ಮುಂತಾದ ಸ್ತ್ರೀಯರಿಗೆ ಅನುರೂಪವಾದ ಗುಣಗಳು ಹೊರಹೊಮ್ಮುತ್ತವೆ. ತವರುಮನೆಯಲ್ಲಿ ವಯಸ್ಸಾದ ಹುಡುಗಿಯು ದಾರಿತಪ್ಪಿ ಹೋಗುವುದಕ್ಕೆ ಹೆಚ್ಚು ಅವಕಾಶವಿದೆ.

ಸ್ವಾಮೀಜಿ: ಹಾಗೆಯೇ ಎದುರು ಪಕ್ಷದ ಪರವಾಗಿಯೂ ಹೇಳಬಹುದೇನೆಂದರೆ - ಬಾಲ್ಯವಿವಾಹದಿಂದ ಹೆಂಗಸರು ಅಕಾಲದಲ್ಲಿ ಸಂತಾನವನ್ನು ಪಡೆದು, ಅನೇಕರು ಮೃತ್ಯುವಿನ ಬಾಯಿಗೆ ಬೀಳುವರು; ಅವರ ಮಕ್ಕಳು ಮರಿಗಳೂ ಕ್ಷೀಣ ಜೀವಿಗಳಾಗಿ ದೇಶದ ತಿರುಕರ ಸಂಖ್ಯೆಯನ್ನು ಹೆಚ್ಚಿಸುವರು. ಏಕೆಂದರೆ ತಾಯಿತಂದೆಗಳ ಶರೀರವು ಸಂಪೂರ್ಣ ಸಾಮರ್ಥ್ಯವನ್ನೂ ಬಲವನ್ನೂ ಪಡೆದ ಹೊರತು ಆರೋಗ್ಯವಂತರೂ ದೃಢಕಾಯರೂ ಆದ ಮಕ್ಕಳು ಹೇಗೆ ಹುಟ್ಟುವರು? ವಿದ್ಯೆಯನ್ನು ಕಲಿತು ಸ್ವಲ್ಪ ವಯಸ್ಸಾದ ಮೇಲೆ ಮದುವೆ ಮಾಡಿಕೊಳ್ಳುವ ಹೆಂಗಸರ ಹೊಟ್ಟೆಯಲ್ಲಿ ಯಾವ ಮಕ್ಕಳು ಹುಟ್ಟುತ್ತಾರೆಯೋ ಅವರಿಂದ ದೇಶಕ್ಕೆ ಕಲ್ಯಾಣವಾಗುತ್ತದೆ. ನಿಮ್ಮ ಮನೆಮನೆಗೂ ಇಷ್ಟು ವಿಧವೆಯರು ಇರುವುದಕ್ಕೆ ಕಾರಣವೆ ಬಾಲ್ಯವಿವಾಹ. ಬಾಲ್ಯವಿವಾಹವು ಕಡಿಮೆಯಾಗಿ ಹೋದರೆ ವಿಧವೆಯರ ಸಂಖ್ಯೆಯೂ ಕಡಿಮೆಯಾಗಿ ಹೋಗುತ್ತದೆ.

ಶಿಷ್ಯ: ಆದರೆ ಮಹಾಶಯರೆ, ವಯಸ್ಸಾದಮೇಲೆ ಮದುವೆ ಮಾಡಿದರೆ ಹೆಂಗಸರು ಮನೆಯ ಕೆಲಸಗಳಿಗೆ ಅಷ್ಟು ಮನಸ್ಸು ಕೊಡುವುದಿಲ್ಲವೆಂದು ತೋರುತ್ತದೆ. ಕಲ್ಕತ್ತೆಯ ಅನೇಕ ಮನೆಗಳಲ್ಲಿ ಅತ್ತೆಯರು ಅಡಿಗೆ ಮಾಡುತ್ತಾರೆಂದೂ ಮತ್ತು ವಿದ್ಯಾವಂತರಾದ ಸೊಸೆಯರು ಸೋಮಾರಿಗಳಾಗಿ ಕುಳಿತು ಕಾಲಿಗೆ ಅರಗಿನ ರಸವನ್ನು ಹಚ್ಚಿಸಿಕೊಂಡು ಕುಳಿತಿರುತ್ತಾರೆಂದೂ ಕೇಳಿದ್ದೇನೆ. ನಮ್ಮ ಪೂರ್ವ ಬಂಗಾಳದೇಶದಲ್ಲಿ ಹೀಗೆ ಎಂದಿಗೂ ಆದದ್ದಿಲ್ಲ.

ಸ್ವಾಮೀಜಿ: ಒಳ್ಳೆಯದು ಕೆಟ್ಟದ್ದು ಎಲ್ಲಾ ದೇಶದಲ್ಲಿಯೂ ಇರುತ್ತವೆ. ಪ್ರತಿಯೊಂದು ದೇಶದಲ್ಲಿಯೂ ಸಮಾಜವು ತನ್ನಷ್ಟಕ್ಕೆ ತಾನೇ ರೂಪುಗೊಳ್ಳುತ್ತದೆ. ಅದಕ್ಕೋಸ್ಕರವೆ, ಬಾಲ್ಯ ವಿವಾಹವನ್ನು ಕಿತ್ತುಹಾಕುವುದು ವಿಧವೆಯರಿಗೆ ಮದುವೆಮಾಡುವುದು ಮುಂತಾದ ವಿಷಯಗಳನ್ನು ತೆಗೆದುಕಕೊಂಡು ನಾವು ತಲೆ ಚಚ್ಚಿಕೊಳ್ಳಬೇಕಾದ್ದಿಲ್ಲ. ನಮ್ಮ ಕೆಲಸವೇನೆಂದರೆ, ಹೆಂಗಸರು ಗಂಡಸರು ಸಮಾಜ ಎಲ್ಲಕ್ಕೂ ಶಿಕ್ಷಣವನ್ನು ಕೊಡುವುದು; ಆ ಶಿಕ್ಷಣದ ಫಲದಿಂದ ಅವರು ತಾವೇ ಯಾವುದು ಒಳ್ಳೆಯದು ಯಾವುದು ಕೆಟ್ಟದ್ದು ಎಂದು ಎಲ್ಲವನ್ನು ತಿಳಿದುಕೊಳ್ಳಲು ಸಮರ್ಥರಾಗುತ್ತಾರೆ. ಮತ್ತು ತಾವೇ ಕೆಟ್ಟದ್ದು ಮಾಡುವುದನ್ನು ಬಿಟ್ಟು ಬಿಡುತ್ತಾರೆ. ಆಗ ಸಮಾಜದ ಯಾವದೊಂದು ವಿಚಾರವನ್ನೂ ಬಲವಂತವಾಗಿ ಕಿತ್ತುಹಾಕಿ ಬದಲಾವಣೆ ಮಾಡಬೇಕಾಗಿಲ್ಲ.

ಶಿಷ್ಯ: ಹೆಂಗಸರಿಗೆ ಈಗ ಎಂಥ ಶಿಕ್ಷಣ ಅವಶ್ಯಕವಾದದ್ದು?

ಸ್ವಾಮೀಜಿ: ಧರ್ಮ, ಶಿಲ್ಪ, ವಿಜ್ಞಾನ, ಗೃಹಕೃತ್ಯ, ಅಡುಗೆ, ಹೊಲಿಗೆ ಶರೀರಪಾಲನ - ಈ ವಿಷಯಗಳನ್ನು ಸುಮಾರಾಗಿ ಹೆಂಗಸರಿಗೆ ಹೇಳಿಕೊಡುವುದು ಉಚಿತ. ಕಥೆ ಕಾದಂಬರಿಗಳನ್ನು ಅವರ ಕೈಗೆ ಕೊಡುವುದು ಉಚಿತವಲ್ಲ. ಮಹಾಕಾಳಿ ಪಾಠಶಾಲೆ ಬಹುಮಟ್ಟಿಗೆ ಸರಿಯಾದ ದಾರಿಯಲ್ಲಿ ಹೋಗುತ್ತಿದೆ. ಆದರೆ ಬರಿಯ ಪೂಜಾ ಪದ್ಧತಿಯನ್ನು ಹೇಳಿಕೊಟ್ಟರೇ ಸಾಲದು. ಎಲ್ಲಾ ವಿಷಯಗಳಲ್ಲಿಯೂ ಕಣ್ಣು ತೆರೆಯುವಂತೆ ಮಾಡಬೇಕು. ಆದರ್ಶನಾರಿಯರ ಚರಿತ್ರೆಗಳನ್ನು ವಿದ್ಯಾರ್ಥಿನಿಯರ ಮುಂದೆ ಸರ್ವದಾ ಇಟ್ಟು ಉಚ್ಚ ತ್ಯಾಗರೂಪವಾದ ವ್ರತದಲ್ಲಿ ಅವರಿಗೆ ಅನುರಾಗ ಹುಟ್ಟುವಂತೆ ಮಾಡಬೇಕು. ಸೀತೆ, ಸಾವಿತ್ರಿ, ದಮಯಂತಿ, ಲೀಲಾವತಿ, ಖನಾ, ಮೀರಾ - ಇವರ ಜೀವನಚರಿತ್ರೆಗಳನ್ನು ಹೆಂಗಸರಿಗೆ ತಿಳಿಯಪಡಿಸಿ ತಮ್ಮ ಜೀವನಗಳನ್ನೂ ಹೀಗೆಯೆ ನಡೆಸಿಕೊಂಡು ಬರಬೇಕೆಂದು ಹೇಳಬೇಕು.

ಗಾಡಿ ಈಗ ಬಾಗಬಜಾರಿನ ಬಲರಾಮ ಬಸು ಮಹಾಶಯರ ಮನೆಗೆ ತಲುಪಿತು. ಸ್ವಾಮಿಜಿ ಗಾಡಿಯಿಂದ ಇಳಿದು ಮೇಲಕ್ಕೆ ಹೋಗಿ ಅವರ ದರ್ಶನ ತೆಗೆದುಕೊಳ್ಳುವುದಕ್ಕೆಂದು ಅಲ್ಲಿಗೆ ಬಂದಿದ್ದವರಿಗೆ ಮಹಾಕಾಳಿ ಪಾಠಶಾಲೆಯ ವೃತ್ತಾಂತವನ್ನು ಆದ್ಯಂತವಾಗಿ ತಿಳಿಸಿದರು.

ಆಮೇಲೆ ಹೊಸದಾಗಿ ಸ್ಥಾಪಿಸಿದ್ದ “ರಾಮಕೃಷ್ಣ ಮಿಷನ್"ನ ಸದಸ್ಯರು ಯಾವ ಯಾವ ಕಾರ್ಯವನ್ನು ಕೈಗೊಳ್ಳಬೇಕೆಂಬ ವಿಷಯವನ್ನು ವಿಚಾರಮಾಡುತ್ತ ಮಾಡುತ್ತಾ “ವಿದ್ಯಾದಾನ" ಮತ್ತು “ಜ್ಞಾನದಾನ"ದ ಶ್ರೇಷ್ಠತ್ವವನ್ನು ನಾನಾ ವಿಧವಾಗಿ ಪ್ರತಿಪಾದನ ಮಾಡುತ್ತ ಹೋದರು. ಶಿಷ್ಯನ ಕಡೆಗೆ ಲಕ್ಷ್ಯಕೊಟ್ಟು “ವಿದ್ಯೆ ಕಲಿಸಿ, ವಿದ್ಯೆ ಕಲಿಸಿ! - ನಾನ್ಯಃ ಪಂಥಾ ವಿದ್ಯತೇಽಯನಾಯ" ಎಂದರು. ವಿದ್ಯಾದಾನಕ್ಕೆ ವಿರೋಧಿಗಳಾದವರನ್ನು ನಿರ್ಧೇಶಿಸಿ “ಪ್ರಹ್ಲಾದನ ಗುಂಪಿಗೆ ಸೇರಿಲ್ಲವಲ್ಲ" ಎಂದರು. ಈ ಮಾತಿಗೆ ಅರ್ಥವೇನೆಂದು ಕೇಳಲು “ಕೇಳಿಲ್ಲವೆ? ‘ಕ’ ಎಂಬ ಅಕ್ಷರವನ್ನು ನೋಡಿದೊಡನೆಯೆ ಪ್ರಹ್ಲಾದನಿಗೆ ಕಣ್ಣಿನಲ್ಲಿ ನೀರು ಬಂತು - ಇನ್ನು ಓದುವುದು ಕೇಳುವುದು ಹೇಗೆ ನಡೆದೀತು? ಪ್ರಹ್ಲಾದನ ಕಣ್ಣಿನಲ್ಲಿ ನೀರು ಬಂದದ್ದು ನಿಸ್ಸಂದೇಹವಾಗಿಯೂ ಆನಂದದಿಂದ. ಆದರೆ ಮೂರ್ಖರ ಕಣ್ಣಿನಲ್ಲಿ ಭಯದಿಂದ ನೀರು ಸುರಿಯುತ್ತಿರುತ್ತದೆ. ಭಕ್ತರಲ್ಲಿಯೂ ಅನೇಕರು ಇಂಥವರು ಇದ್ದಾರೆ" ಎಂದುತ್ತರಿಸಿದರು. ಎಲ್ಲರೂ ಈ ಮಾತನ್ನು ಕೇಳಿ ನಕ್ಕರು. “ನಿಮ್ಮ ಮನಸ್ಸು ಯಾವ ವಿಷಯದ ಕಡೆ ತಿರುಗುತ್ತದೆಯೋ ಅದು ಪೂರ್ತಿಯಾಗುವವರೆಗೂ ನಿಮಗೆ ಶಾಂತಿ ಇರದು. ಈಗ ನಿಮ್ಮ ಇಷ್ಟ ಯಾವುದೋ ಅದೇ ನಡೆದು ಹೋಗುತ್ತದೆ" ಎಂದರು ಯೋಗಾನಂದರು.

\newpage

\chapter[ಅಧ್ಯಾಯ ೧೦]{ಅಧ್ಯಾಯ ೧೦\protect\footnote{\engfoot{Complete Works of Swami Vivekananda, Volume VI, Page 445}}}

\begin{center}
ಸ್ಥಳ: ಕಲ್ಕತ್ತ, ವರ್ಷ: ಕ್ರಿ.ಶ. ೧೮೯೭.
\end{center}

ಇಂದಿಗೆ ಹತ್ತುದಿನದಿಂದ ಶಿಷ್ಯನು ಸ್ವಾಮೀಜಿ ಹತ್ತಿರ ಋಗ್ವೇದದ ಸಾಯಣಭಾಷ್ಯವನ್ನು ಓದುತ್ತಿದ್ದಾನೆ. ಸ್ವಾಮೀಜಿ ಬಾಗಬಜಾರಿನಲ್ಲಿ ಬಲರಾಮ ಬಸುಗಳ ಮನೆಯಲ್ಲಿರುತ್ತಿದ್ದಾರೆ. ಮ್ಯಾಕ್ಸ್‌ಮುಲ್ಲರ್‌ನಿಂದ ಮುದ್ರಿತವಾದ, ಬಹು ಸಂಪುಟಗಳುಳ್ಳ, ಋಗ್ವೇದ ಗ್ರಂಥವೊಂದನ್ನು ಯಾರೋ ಒಬ್ಬ ದೊಡ್ಡ ಮನುಷ್ಯರ ಮನೆಯಿಂದ ತರಲಾಗಿತ್ತು. ಹೊಸ ಗ್ರಂಥ, ಅದರಲ್ಲಿಯೂ ವೈದಿಕಭಾಷೆ. ಆದ್ದರಿಂದ ಓದುತ್ತ ಶಿಷ್ಯನಿಗೆ ಅನೇಕ ಕಡೆಗಳಲ್ಲಿ ತಡೆಯಾಗುತ್ತಿತ್ತು. ಅದನ್ನು ಕಂಡು ಸ್ವಾಮೀಜಿ ಸ್ನೇಹಪೂರ್ವಕವಾಗಿ ಆಗಾಗ್ಯೆ ಬಂಗಾಲ್ ಎಂದು ಹಾಸ್ಯಮಾಡುತ್ತಿದ್ದರು ಮತ್ತು ಈ ಸ್ಥಳಗಳಲ್ಲಿ ಉಚ್ಚಾರಣೆಯ ಪಾಠಗಳನ್ನು ಹೇಳಿಕೊಡುತ್ತಿದ್ದರು. ವೇದದ ಅನಾದಿತ್ವವನ್ನು ಸ್ಥಾಪಿಸುವುದಕ್ಕೆ ಸಾಯಣಾಚಾರ್ಯರು ತಮ್ಮ ಯಾವ ಅದ್ಭುತ ಬುದ್ಧಿ ಕೌಶಲ್ಯವನ್ನು ತೋರಿಸಿದ್ದಾರೆಯೊ ಅದರ ಮೇಲೆ ಸ್ವಾಮೀಜಿ ವ್ಯಾಖ್ಯಾನ ಮಾಡುತ್ತ ಕೆಲವು ವೇಳೆ ಭಾಷ್ಯಕಾರರನ್ನು ಬಹಳವಾಗಿ ಹೊಗಳುತ್ತಿದ್ದರು; ಮತ್ತೆ ಕೆಲವೆಡೆ ಪ್ರಮಾಣ ಪ್ರಯೋಗದಲ್ಲಿ ಅಥವಾ ಪದದ ಗೂಢಾರ್ಥ ವಿಚಾರವಾಗಿ ತಮ್ಮ ಅಭಿಪ್ರಾಯವು ಬೇರೆಯಾಗಿರುವುದನ್ನು ತಿಳಿಸಿ ಸಾಯಣಾಚಾರ್ಯರನ್ನು ವಿನೋದವಾಗಿ ಹಾಸ್ಯ ಮಾಡುತ್ತಿದ್ದರು.

ಹೀಗೆ ಸ್ವಲ್ಪ ಹೊತ್ತು ಪಾಠ ನಡೆದ ಬಳಿಕ ಸ್ವಾಮೀಜಿ ಮ್ಯಾಕ್ಸ್‌ಮುಲ್ಲರ್‌ನ ಪ್ರಸ್ತಾಪವನ್ನು ಎತ್ತಿ ಹೀಗೆಂದು ಹೇಳಿದರು: “ನನಗೆ ಏನು ತೋರುತ್ತದೆ ಬಲ್ಲೆಯಾ? ಸಾಯಣರೆ ತಮ್ಮ ಭಾಷ್ಯವನ್ನು ತಾವೇ ಸ್ವಂತವಾಗಿ ಉದ್ಧಾರ ಮಾಡುವುದಕ್ಕಾಗಿ ಮ್ಯಾಕ್ಸ್‌ಮುಲ್ಲರ್‌ನ ರೂಪದಲ್ಲಿ ಪುನಃ ಹುಟ್ಟಿರಬೇಕು! ನನಗೆ ಬಹುದಿನದಿಂದಲೂ ಈ ಅಭಿಪ್ರಾಯವಿದೆ. ಮ್ಯಾಕ್ಸ್ ಮುಲ್ಲರನ್ನು ನೋಡಿ ಆ ಅಭಿಪ್ರಾಯ ಮತ್ತಷ್ಟು ದೃಢವಾದಂತಾಯಿತು. ಅಂಥ ಅಧ್ಯವಸಾಯವುಳ್ಳ, ವೇದವೇದಾಂತಸಿದ್ಧ ಪಂಡಿತರು ಈ ದೇಶದಲ್ಲಿಯೂ ಕಂಡುಬರುವುದಿಲ್ಲ. ಇದರ ಜೊತೆಗೆ ಪರಮಹಂಸರ ಮೇಲೆ ಎಂಥ ಅಗಾಧ ಭಕ್ತಿ! ಅವರನ್ನು ಅವತಾರವೆಂದು ನಂಬುತ್ತಿದ್ದನಯ್ಯ! ಅವರ ಮನೆಯಲ್ಲಿ ಅತಿಥಿಯಾಗಿದ್ದೆ - ಏನು ಆದರ ಉಪಚಾರಗಳನ್ನು ಮಾಡಿದರು! ಆ ಮುದುಕ ಮುದುಕಿಯರನ್ನು ನೋಡಿದಾಗ ವಸಿಷ್ಠ ಅರುಂಧತಿಯರಂತೆ ಇಬ್ಬರು ಸಂಸಾರ ಮಾಡಿಕೊಂಡಿದ್ದ ಹಾಗೆ ತೋರಿತು. ನನಗೆ ಹೊರಡುವುದಕ್ಕೆ ಅಪ್ಪಣೆಯನ್ನು ಕೊಟ್ಟಾಗ, ಆ ವೃದ್ಧನ ಕಣ್ಣಿನಲ್ಲಿ ನೀರು ಬಂತು!"

ಶಿಷ್ಯ: ಒಳ್ಳೆಯದು ಮಹಾಶಯರೆ, ಸಾಯಣರೆ ಮಾಕ್ಸ್‌ಮುಲ್ಲರ್ ಆಗಿದ್ದರೆ ಪುಣ್ಯಭೂಮಿಯಾದ ಭರತಖಂಡದಲ್ಲಿ ಹುಟ್ಟದೆ ಮ್ಲೇಚ್ಛರಾಗಿ ಏಕೆ ಹುಟ್ಟಿದರು?

ಸ್ವಾಮೀಜಿ: ಅಜ್ಞಾನದಿಂದಲೆ ಮನುಷ್ಯನು ‘ನಾನು ಆರ್ಯ, ಅವನು ಮ್ಲೇಚ್ಛ’ ಮುಂತಾಗಿ ಭಾವಿಸಿಕೊಳ್ಳುವನು ಮತ್ತು ಭೇದಮಾಡುವನು. ಆದರೆ ಯಾರು ವೇದದ ಭಾಷ್ಯಕಾರರೊ, ಜ್ಞಾನದ ಉಜ್ವಲ ಮೂರ್ತಿಗಳೋ ಅವರ ಬಳಿಯಲ್ಲಿ ಅದೆಲ್ಲಾ ಸಂಪೂರ್ಣವಾಗಿ ಅರ್ಥಶೂನ್ಯವಾದದ್ದು. ಜೀವಿಯ ಉಪಕಾರಕ್ಕೋಸ್ಕರ ಅವರು ಹೇಗೆ ಮನಸ್ಸು ಬಂದರೆ ಹಾಗೆ ಹುಟ್ಟಲು ಸಮರ್ಥರು. ಅದರಲ್ಲಿಯೂ ಯಾವ ದೇಶದಲ್ಲಿ ವಿದ್ಯೆ ಮತ್ತು ಸಂಪತ್ತು ಇವೆರಡೂ ಇವೆಯೋ ಅಲ್ಲಿ ಹುಟ್ಟದಿದ್ದರೆ ಈ ದೊಡ್ಡ ಗ್ರಂಥವು ಅಚ್ಚಾಗುವುದಕ್ಕೆ ಬೇಕಾದ ಖರ್ಚನ್ನು ತರುವುದು ಎಲ್ಲಿಂದ? ಕೇಳಿಲ್ಲವೆ? ಈಸ್ಟ್ ಇಂಡಿಯಾ ಕಂಪನಿಯವರು ಈ ಋಗ್ವೇದವನ್ನು ಅಚ್ಚು ಮಾಡಿಸುವುದಕ್ಕೆ ಒಂಬತ್ತು ಲಕ್ಷ ರೂಪಾಯಿಗಳನ್ನು ನಗದಾಗಿ ಕೊಟ್ಟರು. ಅಷ್ಟು ಹಣವೂ ಸಾಲದೆ ಈ ದೇಶದಲ್ಲಿ ನೂರಾರು ಪಂಡಿತರುಗಳಿಗೆ ಮಾಸಾಶನಗಳನ್ನು ಕೊಟ್ಟು ಈ ಕಾರ್ಯಕ್ಕೆ ನಿಯಮಿಸಬೇಕಾಯಿತು. ವಿದ್ಯೆ ಮತ್ತು ಜ್ಞಾನ ಇವುಗಳಿಗಾಗಿ ಹೀಗೆ ಯಥೇಚ್ಛವಾಗಿ ಹಣವನ್ನು ಖರ್ಚುಮಾಡುವುದು, ಇಂಥ ಪ್ರಬಲವಾದ ಜ್ಞಾನತೃಷ್ಣೆ ಇರುವುದನ್ನು ಈ ದೇಶದಲ್ಲಿ ಈ ಕಾಲದಲ್ಲಿ ಯಾರಾದರೂ ಎಲ್ಲಿಯಾದರೂ ನೋಡಿದ್ದಾರೇನು? ಮ್ಯಾಕ್ಸ್‌ಮುಲ್ಲರ್ ತಾವೇ ಭೂಮಿಕೆಯಲ್ಲಿ ಹೀಗೆ ಬರೆದಿದ್ದಾನೆ: ಆತನು ೨೫ ವರ್ಷಕಾಲ ಕೇವಲ ಹಸ್ಥಪ್ರತಿಯನ್ನು ಸಿದ್ಧಪಡಿಸುತ್ತಿದ್ದನಂತೆ! ಆಮೇಲೆ ಅಚ್ಚು ಹಾಕಿಸುವುದಕ್ಕೆ ಇಪ್ಪತ್ತು ವರ್ಷ ಹಿಡಿಯಿತಂತೆ; ೪೫ ವರ್ಷ ಒಂದು ಪುಸ್ತಕವನ್ನು ಈ ರೀತಿ ಹಿಡಿದುಕೊಂಡು ಕುಳಿತುಕೊಳ್ಳುವುದೆಂದರೆ ಅದು ಸಾಮಾನ್ಯ ಮನುಷ್ಯನ ಕೆಲಸವಲ್ಲ. ಇದರಿಂದಲೆ ತಿಳಿದುಕೊ ಆತನು ಸಾಯಣರೆಂದು.

ಮ್ಯಾಕ್ಸ್‌ಮುಲ್ಲರ್ ಸಂಬಂಧದಲ್ಲಿ ಈ ವಿಧವಾಗಿ ಮಾತುಕಥೆಗಳು ನಡೆದ ಮೇಲೆ ಮತ್ತೆ ಪಾಠ ನಡೆಯುವುದಕ್ಕೆ ಮೊದಲಾಯಿತು. ಈಗ, ವೇದವನ್ನು ಅವಲಂಬಿಸಿಕೊಂಡೇ ಸೃಷ್ಟಿಯ ವಿಕಾಸವಾಯಿತೆಂಬೀ ಸಾಯಣರ ಮತವನ್ನು ಸ್ವಾಮಿಜಿ ಪೂರ್ತಿ ಸಮರ್ಥಿಸುವುದಕ್ಕೆ ಹೊರಟರು. ಅವರು ಹೇಳಿದ್ದೇನೆಂದರೆ: "ವೇದ ಅಂದರೆ ಅನಾದಿ ಸತ್ಯದ ಸಮಷ್ಟಿ. ವೇದಪಾರಂಗತರಾದ ಋಷಿಗಳು ಈ ಸತ್ಯಗಳನ್ನು ಪ್ರತ್ಯಕ್ಷ ಮಾಡಿಕೊಂಡರು; ಅತೀಂದ್ರಿಯದರ್ಶಿಗಳು ಹೊರತು ನಮ್ಮಂಥ ಸಾಧಾರಣ ಜನರ ದೃಷ್ಟಿಗೆ ಅವು ಪ್ರತ್ಯಕ್ಷವಾಗುವುದಿಲ್ಲ: ಅದಕ್ಕೋಸ್ಕರವೆ ವೇದದಲ್ಲಿ ‘ಋಷಿ’ ಶಬ್ದಕ್ಕೆ ಮಂತ್ರಾರ್ಥದ್ರಷ್ಟೃ (ಮಂತ್ರಾರ್ಥಗಳನ್ನು ಕಂಡವನು) ಎಂದು ಅರ್ಥ. ಯಜ್ಞೋಪವೀತ ಕತ್ತಿನಲ್ಲಿದ್ದ ಮಾತ್ರಕ್ಕೆ ಬ್ರಾಹ್ಮಣನಲ್ಲ. ಬ್ರಾಹ್ಮಣಾದಿ ಜಾತಿವಿಭಾಗವು ಆಮೇಲೆ ಆಯಿತು. ವೇದ ಶಬ್ದಾತ್ಮಕ ಎಂದರೆ ಭಾವಾತ್ಮಕ ಅಥವಾ ಅನಂತ ಭಾವರಾಶಿಯ ಸಮಷ್ಟಿ ಮಾತ್ರ. ‘ಶಬ್ದ’ ಪದಕ್ಕೆ ವೈದಿಕ ಪ್ರಾಚೀನ ಅರ್ಥ ‘ಸೂಕ್ಷ್ಮಭಾವ’ ಎಂದು. ಅದು ಆಮೇಲೆ ಸ್ಥೂಲಾಕಾರವನ್ನು ಪಡೆದು ತನ್ನನ್ನು ತಾನು ಪ್ರಕಾಶಕ್ಕೆ ತಂದುಕೊಳ್ಳುವುದು. ಆದ್ದರಿಂದ ಪ್ರಳಯವಾಗುವಾಗ ಮುಂದಿನ ಸೃಷ್ಟಿಯ ಸೂಕ್ಷ್ಮ ಬೀಜಗಳು ವೇದದಲ್ಲಿಯೆ ಅಡಗಿರುತ್ತವೆ. ಅದರಿಂದಲೆ ಪುರಾಣದಲ್ಲಿ ಮೊದಲು ಮತ್ಸ್ಯಾವತಾರದಿಂದ ವೇದೋದ್ಧಾರವಾಗಿರುವುದು ಕಂಡುಬರುತ್ತದೆ. ಪ್ರಥಮಾವತಾರದಿಂದ ವೇದದ ಉದ್ಧಾರ ಸಾಧಿತವಾಯಿತು. ಆಮೇಲೆ ಅದೇ ವೇದದಿಂದಲೇ ಕ್ರಮವಾಗಿ ಸೃಷ್ಟಿ ವಿಕಾಸವಾಗತೊಡಗಿತು ಎಂದರೆ, ವೇದದಲ್ಲಿದ್ದ ಶಬ್ದವನ್ನು ಅವಲಂಬಿಸಿಕೊಂಡು ಪ್ರಪಂಚದ ಸಕಲಸ್ಥೂಲ ಪದಾರ್ಥಗಳು ಒಂದೊಂದಾಗಿ ನಿರ್ಮಿತವಾಗುತ್ತ ಬಂದವು. ಏಕೆಂದರೆ, ಸಕಲ ಸ್ಥೂಲಪದಾರ್ಥಗಳ ಸೂಕ್ಷ್ಮರೂಪವು ಶಬ್ದ ಅಥವಾ ಭಾವವೇ ಆಗಿದೆ. ಎಲ್ಲ ಹಿಂದಿನ ಕಲ್ಪಗಳಲ್ಲಿಯೂ ಹೀಗೇ ಸೃಷ್ಟಿಯಾಯಿತು. ಈ ಸಂಗತಿ ವೈದಿಕ ಸಂಧ್ಯಾವಂದನೆಯ ಮಂತ್ರದಲ್ಲಿಯೂ ಇದೆ - ‘ಸೂರ್ಯಾಚಂದ್ರಮಸೌ ಧಾತಾ ಯಥಾ ಪೂರ್ವಮಕಲ್ಪಯತ್, ಪೃಥಿವೀಂ ದಿವಂ ಚಾಂತರಿಕ್ಷಮಥೋ ಸ್ವಃ.’ ಗೊತ್ತಾಯಿತೆ?"

ಶಿಷ್ಯ: ಆದರೆ ಮಹಾಶಯರೆ, ಯಾವ ಪದಾರ್ಥವೂ ಇಲ್ಲದಿದ್ದರೆ ಏತಕ್ಕೋಸ್ಕರ ಶಬ್ದ ಪ್ರಯೋಗಿಸಲ್ಪಡುತ್ತದೆ? ಮತ್ತು ಪದಾರ್ಥಗಳ ಹೆಸರುತಾನೆ ಹೇಗೆ ಬರುತ್ತವೆ?

ಸ್ವಾಮೀಜಿ: ಮೇಲುಮೇಲೆ ನೋಡಿದರೆ ಈ ವಿಷಯವನ್ನು ಅರ್ಥಮಾಡಿಕೊಳ್ಳುವುದು ಕಷ್ಟ. ಆದರೆ ನೋಡು; ಈ ಘಟ ಒಡೆದು ಹೋದರೆ ಘಟತ್ವ ನಾಶವಾಗಿ ಹೋಗುತ್ತದೆಯೇನು? ಇಲ್ಲ; ಏಕೆಂದರೆ ಘಟ ಸ್ಥೂಲ; ಆದರೆ ಘಟತ್ವ ಘಟದ ಸೂಕ್ಷ್ಮಾವಸ್ಥೆ ಅಥವಾ ಶಬ್ದಾವಸ್ಥೆ. ಹೀಗೆ ಸಕಲ ಪದಾರ್ಥಗಳ ಶಬ್ದಾವಸ್ಥೆಯೆ ಈ ಸಕಲ ಪದಾರ್ಥಗಳ ಸೂಕ್ಷ್ಮಾವಸ್ಥೆ. ನಾವು ಯಾವ ಪದಾರ್ಥಗಳನ್ನು ನೋಡಿ ಕೇಳಿ ಮುಟ್ಟಿ ಮುಂತಾದ್ದನ್ನು ಮಾಡುತ್ತೇವೆಯೋ ಅವು ಇಂಥ ಸೂಕ್ಷ್ಮ ಅಥವಾ ಶಬ್ದಾವಸ್ಥೆಯಲ್ಲಿರುವ ಪದಾರ್ಥಗಳ ಸ್ಥೂಲವಿಕಾಸ, ಕಾರ್ಯಕಾರಣ ಸಂಬಂಧವಿದ್ದಂತೆ. ಜಗತ್ತು ಧ್ವಂಸವಾಗಿ ಹೋದರೂ ಜಗದ್ಭೋಧಾತ್ಮಕವಾದ ಶಬ್ದ ಅಥವಾ ಸ್ಥೂಲ ಪದಾರ್ಥಗಳ ಸೂಕ್ಷ್ಮ ಸ್ವರೂಪ ಬ್ರಹ್ಮದಲ್ಲಿ ಕಾರಣರೂಪವಾಗಿರುತ್ತವೆ. ಜಗದ್ವಿಕಾಸವಾಗುವುದಕ್ಕೆ ಹಿಂದೆ, ಮೊದಲು ಸೂಕ್ಷ್ಮಸ್ವರೂಪಗಳ ಸಮಷ್ಟಿ ರೂಪವಾದ ಈ ಪದಾರ್ಥ ಸ್ಪಂದಿಸಿ ಅದರಲ್ಲಿಯೇ ಪ್ರಕೃತಿ ಸ್ವರೂಪವೂ ಶಬ್ದ ಗರ್ಭಾತ್ಮಕವೂ ಆದ ಅನಾದಿಯಾದ ‘ಓಂ’ ಕಾರವು ತನ್ನಷ್ಟಕ್ಕೆ ತಾನೇ ಬರುತ್ತಿರುತ್ತದೆ. ಕ್ರಮವಾಗಿ ಈ ಸಮಷ್ಟಿಯಿಂದ ಪ್ರತಿಯೊಂದು ವಿಶೇಷ ಪದಾರ್ಥವೂ ಮೊದಲು ಸೂಕ್ಷ್ಮ ಅಥವಾ ಶಾಬ್ದಿಕ ರೂಪದಲ್ಲಿ ಅನಂತರ ಸ್ಥೂಲರೂಪದಲ್ಲಿ ಪ್ರಕಾಶಿತವಾಗುತ್ತದೆ. ಈ ಶಬ್ದವೇ ಬ್ರಹ್ಮ - ಶಬ್ದವೇ ವೇದ. ಇದೇ ಸಾಯಣರ ಅಭಿಪ್ರಾಯ ತಿಳಿಯಿತೆ?

ಶಿಷ್ಯ; ಮಹಾಶಯರೆ, ಚೆನ್ನಾಗಿ ತಿಳಿದುಕೊಳ್ಳುವುದಕ್ಕಾಗಲಿಲ್ಲ.

ಸ್ವಾಮಿಜಿ: ಜಗತ್ತಿನಲ್ಲಿ ಎಷ್ಟು ಗಡಿಗೆಗಳಿವೆಯೋ ಅವೆಲ್ಲಾ ನಷ್ಟವಾದರೂ ಘಟ ಶಬ್ದವಿರಬಲ್ಲವೆಂದು ಗೊತ್ತಿದೆಯೊ? ಹಾಗಾದರೆ ಜಗತ್ತು ಧ್ವಂಸವಾದರೂ ಎಂದರೆ ಯಾವ ಪದಾರ್ಥಗಳು ಸೇರಿ ಜಗತ್ತಾಗಿದೆಯೋ ಅವೆಲ್ಲಾ ಒಡೆದು ಪುಡಿಪುಡಿಯಾಗಿ ಹೋದರೂ, ಅವುಗಳನ್ನು ಬೋಧಿಸುವ ಶಬ್ದಗಳು ಏಕೆ ಇರಲಾರವು? ಮತ್ತು ಅವುಗಳಿಂದ ಪುನಃ ಸೃಷ್ಟಿ ಏಕೆ ಆಗಲಾರದು?

ಶಿಷ್ಯ: ಆದರೆ, ಮಹಾಶಯರೇ, ‘ಘಟ’ ‘ಘಟ’ ಎಂದು ಕೂಗಿಕೊಂಡ ಮಾತ್ರದಿಂದ ಘಟ ತಯಾರಾಗುವುದಿಲ್ಲ.

ಸ್ವಾಮೀಜಿ: ನೀನೂ ನಾನೂ ಹೀಗೆ ಕೂಗಿಕೊಂಡರೆ ಆಗುವುದಿಲ್ಲ. ಆದರೆ ಸಿದ್ಧ ಸಂಕಲ್ಪ ಬ್ರಹ್ಮದಲ್ಲಿ ಘಟಸ್ಮೃತಿ ಉಂಟಾದ ಮಾತ್ರದಿಂದಲೆ ಘಟ ಪ್ರಕಾಶಿತವಾಗುತ್ತದೆ. ಸಾಮಾನ್ಯ ಸಾಧಕನ ಇಚ್ಚೆಯಿಂದಲೆ ನಾನಾ ಅಘಟನಘಟನಗಳು ನಡೆಯುವುದು ಸಾಧ್ಯವಾಗಿರುವಾಗ ಸಿದ್ಧ ಸಂಕಲ್ಪವಾದ ಬ್ರಹ್ಮದ ವಿಚಾರವಾಗಿ ಹೇಳತಕ್ಕದ್ದೇನು? ಸೃಷ್ಟಿಗೆ ಮೊದಲು ಬ್ರಹ್ಮ ಶಬ್ದಾತ್ಮಕವಾಗಿರುತ್ತದೆ; ಆಮೇಲೆ ಓಂಕಾರಾತ್ಮಕ ಅಥವಾ ನಾದಾತ್ಮಕವಾಗಿಬಿಡುತ್ತದೆ; ಆಮೇಲೆ ಪೂರ್ವ ಕಲ್ಪದ ನಾನಾ ವಿಶೇಷ ಶಬ್ದಗಳು - ಎಂದರೆ ಭೂಃ, ಭುವಃ, ಸ್ವಃ, ಅಥವಾ ಗೋವು, ಮನುಷ್ಯ, ಪಟ, ಘಟ ಮುಂತಾದುವುಗಳು - ಓಂಕಾರದಿಂದ ಹೊರಡುತ್ತವೆ. ಸಿದ್ಧಸಂಕಲ್ಪ ಬ್ರಹ್ಮದಲ್ಲಿ ಈ ಶಬ್ದಗಳು ಕ್ರಮಕ್ರಮವಾಗಿ ಒಂದೊಂದಾಗಿ ಉಂಟಾದ ಮಾತ್ರದಿಂದಲೇ ಆಯಾ ಪದಾರ್ಥಗಳು ಅಲ್ಲಿಂದ ಹೊರಗೆ ಬಂದು ಕ್ರಮೇಣ ವೈವಿಧ್ಯಮಯವಾದ ಜಗತ್ತು ಪ್ರಕಾಶಕ್ಕೆ ಬರುತ್ತದೆ. ಈಗ ತಿಳಿಯಿತೆ, ಶಬ್ದವೆ ಹೇಗೆ ಸೃಷ್ಟಿಯ ಮೂಲವೆಂಬುದು?

ಶಿಷ್ಯ: ಹುಂ, ಏನೊ ಒಂದು ವಿಧದಲ್ಲಿ ತಿಳಿದುಕೊಂಡೆ ನಿಜ; ಆದರೆ ಸರಿಯಾಗಿ ಮನಸ್ಸಿಗೆ ಹತ್ತಲಿಲ್ಲ.

ಸ್ವಾಮೀಜಿ: ಮನಸ್ಸಿಗೆ ಹತ್ತುವುದು - ಪ್ರತ್ಯಕ್ಷವಾಗಿ ಅನುಭವ ಮಾಡಿಕೊಳ್ಳುವುದು - ಏನು ಸುಲಭವೇನಪ್ಪಾ? ಮನಸ್ಸು ಯಾವಾಗ ಬ್ರಹ್ಮದಲ್ಲಿ ಪ್ರವೇಶಿಸುತ್ತ ಹೋಗುತ್ತದೆಯೋ ಆಗ ಒಂದಾದಮೇಲೆ ಒಂದರಂತೆ ಈ ಸ್ಥಿತಿಗಳ ಮೂಲಕ ಹೋಗಿ ಕೊನೆಗೆ ನಿರ್ವಿಕಲ್ಪಕ್ಕೆ ಹೋಗುತ್ತದೆ. ಸಮಾಧಿ ಮುಖದಲ್ಲಿ ಮೊದಲು ಗೊತ್ತಾಗುತ್ತದೆ - ಜಗತ್ತು ಭಾವಮಯ ಎಂದು. ಆಮೇಲೆ ಗಂಭೀರವಾದ ಓಂಕಾರ ಧ್ವನಿಯಲ್ಲಿ ಎಲ್ಲವೂ ಸೇರಿಹೋಗುತ್ತದೆ - ಆಮೇಲೆ ಅದೂ ಕೇಳುವುದಿಲ್ಲ - ಅದೂ ಇದೆಯೋ ಇಲ್ಲವೋ ಎನ್ನಿಸುತ್ತದೆ! ಇದೇ ಅನಾದಿನಾದವೆಂಬುದು. ಆಮೇಲೆ ಮನಸ್ಸು ಬ್ರಹ್ಮಸತ್ಯದಲ್ಲಿ ಲೀನವಾಗಿ ಹೋಗುತ್ತದೆ. ಆಯಿತು - ಎಲ್ಲ ಅಲ್ಲಿಗೆ ನಿಂತುಹೋಯಿತು.

ಸ್ವಾಮೀಜಿಯ ಮಾತಿನಿಂದ ಅವರು ಈ ಸ್ಥಿತಿಗಳನ್ನೆಲ್ಲಾ ಅನುಭವ ಮಾಡಿಕೊಂಡು ಅನೇಕ ಸಾರಿ ಸಮಾಧಿಸ್ಥಿತಿಗೆ ಹೋಗಿಬಂದಿದ್ದರೆಂದು ಶಿಷ್ಯನಿಗೆ ಸ್ಪಷ್ಟವಾಗಿ ತಿಳಿಯುತ್ತ ಬಂತು. ಇಲ್ಲದಿದ್ದರೆ ಇಷ್ಟು ವಿಶದವಾಗಿ ಈ ವಿಷಯಗಳನ್ನೆಲ್ಲಾ ಹೇಗೆ ತಿಳಿಸಿಹೇಳುತ್ತಿದ್ದರು? ತಾವೆ ನೋಡಿ, ಕೇಳಿ, ಮಾಡುವ ವಿಷಯವಲ್ಲದಿದ್ದರೆ ಯಾರೂ ಎಂದಿಗೂ ಹೀಗೆ ಹೇಳಲೂ ತಿಳಿಸಲೂ ಆರರೆಂದು ಶಿಷ್ಯನು ಇದನ್ನೆಲ್ಲಾ ಮಾತಿಲ್ಲದೆ ಬೆರಗಾಗಿ ಕೇಳುತ್ತ ಭಾವಿಸಿಕೊಂಡನು.

ಸ್ವಾಮೀಜಿ ಇನ್ನೂ ಹೇಳಿದ್ದೇನೆಂದರೆ: ಅವತಾರಸಮಾನರಾದ ಮಹಾಪುರುಷರು ಸಮಾಧಿಸ್ಥಿತಿಯಿಂದ ‘ನಾನು ನನ್ನದು’ ಎಂಬ ಸ್ಥಿತಿಗೆ ಇಳಿದುಬಂದಾಗ ಮೊಟ್ಟಮೊದಲು ಅವ್ಯಕ್ತ ನಾದದ ಅನುಭವವನ್ನು ಪಡೆಯುವರು; ಕ್ರಮವಾಗಿ ನಾದವು ಸ್ಪಷ್ಟವಾಗಿ ಓಂಕಾರದ ಅನುಭವವಾಗುವುದು; ಓಂಕಾರದಿಂದ ಆಮೇಲೆ ಶಬ್ದಮಯವಾದ ಜಗತ್ತಿನ ಬೋಧೆಯುಂಟಾಗುವುದು; ಆಮೇಲೆ ಎಲ್ಲಕ್ಕೂ ಕೊನೆಯಲ್ಲಿ ಸ್ಥೂಲವಾದ ಭೌತಿಕ ಜಗತ್ತಿನ ಪ್ರತ್ಯಕ್ಷವಾಗುವುದು. ಸಾಮಾನ್ಯ ಸಾಧಕರು ಹೇಗೆ ಹೇಗೋ ಬಹು ಕಷ್ಟಪಟ್ಟುಕೊಂಡು ನಾದದಿಂದ ಮುಂದಕ್ಕೆ ಹೋಗಿ ಬ್ರಹ್ಮದ ಉಪಲಬ್ಧಿಯನ್ನು ಸಾಕ್ಷಾತ್ತಾಗಿ ಪಡೆದರೆ, ಯಾವ ಕೆಳಗಿನ ಭೂಮಿಯಲ್ಲಿ ಸ್ಥೂಲ ಜಗತ್ತಿನ ಪ್ರತ್ಯಕ್ಷವಾಗುವುದೋ ಅಲ್ಲಿಗೆ ಪುನಃ ಇಳಿದುಬರಲಾರರು, ಬ್ರಹ್ಮದಲ್ಲಿಯೇ ಸೇರಿ ಹೋಗುತ್ತಾರೆ - ‘ಕ್ಷೀರೇ ನೀರವತ್’ - ನೀರು ಹಾಲಿನಲ್ಲಿ ಸೇರುವಂತೆ.

ಈ ಮಾತುಗಳೆಲ್ಲಾ ನಡೆಯುತ್ತಿರುವಾಗ ಶ್ರೇಷ್ಠ ನಾಟಕಕಾರ ಶ‍್ರೀಯುತ ಗಿರೀಶಚಂದ್ರ ಘೋಷ ಮಹಾಶಯ ಅಲ್ಲಿಗೆ ಬಂದರು. ಸ್ವಾಮೀಜಿ ಅವರನ್ನು ಅಭಿನಂದಿಸಿ ಕುಶಲ ಪ್ರಶ್ನಾದಿಗಳನ್ನು ಮಾಡಿ ಪುನಃ ಶಿಷ್ಯನಿಗೆ ಪಾಠ ಹೇಳತೊಡಗಿದರು. ಗಿರೀಶಬಾಬುಗಳೂ ಅದನ್ನು ಮನಸ್ಸುಕೊಟ್ಟು ಕೇಳುತ್ತ ಸ್ವಾಮೀಜಿ ಹೀಗೆ ವೇದಕ್ಕೆ ವಿಶದವಾಗಿ ವ್ಯಾಖ್ಯಾನ ಮಾಡುತ್ತಿರುವುದನ್ನು ನೋಡಿ ಬೆರಗಾಗಿ ಕುಳಿತುಕೊಂಡರು.

ಹಿಂದಿನ ವಿಷಯಗಳನ್ನು ಅನುಸರಣ ಮಾಡಿಕೊಂಡು ಸ್ವಾಮೀಜಿ ಪುನಃ ಹೇಳತೊಡಗಿದರು: ವೈದಿಕ ಮತ್ತು ಲೌಕಿಕವೆಂದು ಶಬ್ದವು ಎರಡು ಭಾಗ. ನ್ಯಾಯ ಗ್ರಂಥವಾದ ‘ಶಬ್ದ ಶಕ್ತಿ ಪ್ರಕಾಶಿಕೆ’ಯಲ್ಲಿ ಈ ವಿಷಯದ ವಿಚಾರವನ್ನು ನೋಡಿದ್ದೇನೆ. ಅಲ್ಲಿ ಮಾಡಿರುವ ವಿಚಾರಗಳು ಗಂಭೀರ ವಿಚಾರವನ್ನೇನೊ ತೋರಿಸುತ್ತವೆ; ಆದರೆ ಪಾರಿಭಾಷಿಕ ಶಬ್ದಗಳ ಅವಾಂತರದಲ್ಲಿ ತಲೆ ಚಿಟ್ಟು ಹಿಡಿದು ಹೋಗುತ್ತದೆ. ಈಗ ಗಿರೀಶಬಾಬುಗಳ ಕಡೆಗೆ ನೋಡಿ ಹೇಳಿದ್ದೇನೆಂದರೆ: ಏನು ಜಿ.ಸಿ., ನೀವು ಇವುಗಳೊಂದನ್ನೂ ಓದಲಿಲ್ಲ, ಕೇವಲ ಪುರಾಣ ಕಥೆಗಳಲ್ಲಿಯೇ ಕಾಲ ಕಳೆದುಹೋಯಿತು.

ಗಿರೀಶಬಾಬು: ಮತ್ತೇನು ಓದೋಣವಣ್ಣ? ಅದರಲ್ಲಿ ಪ್ರವೇಶಿಸುವಷ್ಟು ಅವಕಾಶವೂ ಇಲ್ಲ, ಬುದ್ಧಿಯೂ ಇಲ್ಲ. ಆದರೇನು ಪರಮಹಂಸರ ದಯೆಯಿಂದ ಆ ವೇದವೇದಾಂತಗಳಿಗೆ ನಮಸ್ಕಾರಮಾಡಿ ಈಸಾರಿಗೆ ಕಡೆಹಾಯ್ದು ಬಿಡುತ್ತೇನೆ. ನಿನ್ನಿಂದ ತಮ್ಮ ಕೆಲಸವನ್ನು ತುಂಬ ಮಾಡಿಸಿಕೊಳ್ಳಬೇಕೆಂದು ಅವರು ಅದನ್ನೆಲ್ಲಾ ನಿನಗೆ ಓದಿಸಿದರು. ನನಗೆ ಅದರ ಆವಶ್ಯಕತೆ ಇಲ್ಲ. ಹೀಗೆಂದು ಹೇಳಿ ಗಿರೀಶಬಾಬುಗಳು ಆ ದೊಡ್ಡ ಋಗ್ವೇದ ಗ್ರಂಥಕ್ಕೆ ಮತ್ತೆ ಮತ್ತೆ ಪ್ರಣಾಮ ಮಾಡುತ್ತ ಮಾಡುತ್ತ ‘ಜಯ ವೇದರೂಪೀ ರಾಮಕೃಷ್ಣ ಜಯ!’ ಎಂದು ಹೇಳತೊಡಗಿದರು. ಸ್ವಾಮಿಜಿ ಯಾವಾಗ ಯಾವ ವಿಷಯವನ್ನು ಉಪದೇಶಮಾಡುತ್ತಿದ್ದರೆ ಆಗ ಆ ವಿಷಯವು ಶ್ರೋತೃಗಳ ಮನಸ್ಸಿನಲ್ಲಿ ದೃಢವಾಗಿ ನೆಟ್ಟು ಅವರು ಅದೇ ಎಲ್ಲಕ್ಕಿಂತಲೂ ಸಾರವಾದ ವಸ್ತುವೆಂದು ಭಾವಿಸುತ್ತಿದ್ದರು. ಇದನ್ನು ಪಾಠಕರಿಗೆ ನಾವು ಮತ್ತೊಂದು ಕಡೆಯಲ್ಲಿ ಹೇಳಿದ್ದೇವೆ. ಬ್ರಹ್ಮಜ್ಞಾನ ಸಂಬಂಧವಾಗಿ ಅವರು ಮಾತನಾಡಿದಾಗ, ಅದನ್ನು ಕೇಳುತ್ತಿದ್ದವರು ಅದನ್ನು ಪಡೆಯುವುದೊಂದೇ ಜೀವನದ ಉದ್ದೇಶವೆಂದು ಮನಸ್ಸಿನಲ್ಲಿ ಮಾಡಿಕೊಳ್ಳುತ್ತಿದ್ದರು. ಮತ್ತೆ, ಭಕ್ತಿ, ಕರ್ಮ ಅಥವಾ ಜಾತಿಯ ಉನ್ನತಿ ಮುಂತಾದ ಇತರ ವಿಷಯಗಳನ್ನು ಅವರು ಪ್ರಸ್ತಾಪಿಸಿದಾಗ ಆ ವಿಷಯಕ್ಕೆ ಶ್ರೋತೃಗಳು ತಮ್ಮ ಮನಸ್ಸಿನಲ್ಲಿ ಸರ್ವೋಚ್ಚಾಸನವನ್ನು ಕೊಟ್ಟು ಆಯಾ ವಿಷಯವನ್ನು ಅಭ್ಯಾಸ ಮಾಡುವುದಕ್ಕೆ ತವಕಪಡುತ್ತಿದ್ದರು. ಈಗ ವೇದದ ಪ್ರಸಕ್ತಿಯನ್ನು ತೆಗೆದು ಶಿಷ್ಯ ಮುಂತಾದವರ ಮನಸ್ಸನ್ನು ವೇದೋಕ್ತ ಮಹಿಮೆಯಲ್ಲಿ ಎಷ್ಟರಮಟ್ಟಿಗೆ ಮೋಹಗೊಳಿಸಿಬಿಟ್ಟರೆಂದರೆ, ಅವರು ಈಗ ಅದಕ್ಕಿಂತ ಸಾರತರವೂ ಆವಶ್ಯಕವೂ ಆದ ಮತ್ತೊಂದು ವಸ್ತುವನ್ನು ಅರಿಯಲಾರದೆ ಹೋದರು. ಗಿರೀಶಬಾಬುಗಳು ಇದನ್ನು ಕಂಡರು; ಮತ್ತು ಸ್ವಾಮಿಗಳ ಮಹದುದಾರಭಾವವೂ ಶಿಕ್ಷಣ ಕೊಡುವ ಈ ವಿಧವಾದ ರೀತಿಯೂ ಮೊದಲೆ ಅವರಿಗೆ ತಿಳಿದಿದ್ದರಿಂದ, ಶಿಷ್ಯ ಮುಂತಾದವರಿಗೆ ಜ್ಞಾನ, ಭಕ್ತಿ ಮತ್ತು ಕರ್ಮಗಳ ಸಮಾನವಾದ ಆವಶ್ಯಕತೆಯನ್ನೂ ತೋರಿಸಿಕೊಡುವುದಕ್ಕಾಗಿ ಈಗ ಒಂದು ಯುಕ್ತಿಯನ್ನು ಹೂಡಿದರು.

ಸ್ವಾಮೀಜಿ ಅನ್ಯಮನಸ್ಕರಾಗಿ ಏನೋ ಯೋಚಿಸುತ್ತಿದ್ದರು. ಈ ಮಧ್ಯೆ ಗಿರೀಶಬಾಬುಗಳು ಹೇಳಿದ್ದೇನೆಂದರೆ - ಹೌದೋ ನರೇನ್. ಒಂದು ಮಾತು ಕೇಳಬೇಕೆಂದಿದ್ದೇನೆ. ವೇದ ವೇದಾಂತವನ್ನೇನೊ ಬೇಕಾದಷ್ಟು ಓದಬಹುದು. ಆದರೆ, ಈ ನಮ್ಮ ದೇಶದಲ್ಲಿರುವ ಘೋರವಾದ ಹಾಹಾಕಾರ, ಹೊಟ್ಟೆಗಿಲ್ಲದಿರುವಿಕೆ, ವ್ಯಭಿಚಾರ, ಶಿಶುಹತ್ಯ ಮುಂತಾದ ಮಹಾಪಾಪಗಳು ಇವುಗಳನ್ನೆಲ್ಲ ನಿವಾರಣೆ ಮಾಡುವ ದಾರಿ ನಿಮ್ಮ ವೇದದಲ್ಲಿ ಏನಾದರೂ ಹೇಳಿದೆಯೇನು? ಇಗೋ ಇಂಥವರ ಮನೆಯ ಯಜಮಾನಿ, ಒಂದಾನೊಂದು ಕಾಲದಲ್ಲಿ ಅವರ ಮನೆಯಲ್ಲಿ ದಿನಕ್ಕೆ ಐವತ್ತು ಎಲೆ ಬೀಳುತ್ತಿತ್ತು, ಅಂಥವಳು ಈಗ ಒಲೆಯ ಮೇಲೆ ಮೂರು ದಿನದಿಂದ ತಪ್ಪಲೆಯಿಟ್ಟಿಲ್ಲ. ಇಗೋ ಇಂಥವರ ಮನೆಯ ಕುಲಸ್ತ್ರೀಯನ್ನು ಪುಂಡರು ಕೆಡಿಸಿ ಕೊಂದುಹಾಕಿದರು. ಇಗೋ ಇಂಥವರ ಮನೆಯಲ್ಲಿ ಶಿಶುಹತ್ಯವಾಯಿತು, ಇಂಥವರು ಮೋಸಮಾಡಿ ಗಂಡನು ಸತ್ತ ಹೆಂಗಸಿನ ಸರ್ವಸ್ವವನ್ನೂ ಹೊತ್ತು ಹಾಕಿದರು - ಇವುಗಳೆಲ್ಲಾ ಇಲ್ಲದಂತೆ ಮಾಡುವುದಕ್ಕೆ ಏನಾದರೂ ದಾರಿಯಿದೆಯೆ ನಿಮ್ಮ ವೇದದಲ್ಲಿ? ಗಿರೀಶಬಾಬುಗಳು ಹೀಗೆ ಸಮಾಜದ ಭಯಂಕರವಾದ ಚಿತ್ರಗಳನ್ನು ಮೇಲೆ ಮೇಲೆ ಚಿತ್ರಿಸಿ ತೋರಿಸುವುದಕ್ಕೆ ಮೊದಲು ಮಾಡಲು, ಸ್ವಾಮೀಜಿ ಮಾತು ಹೊರಡದೆ ಕುಳಿತುಕೊಂಡರು. ಜಗತ್ತಿನ ದುಃಖ ಕಷ್ಟಗಳ ವಿಚಾರವನ್ನು ಯೋಚಿಸುತ್ತ ಸ್ವಾಮಿಗಳ ಕಣ್ಣಿನಲ್ಲಿ ನೀರು ಬಂದಿತು. ಅವರು ತಮ್ಮ ಮನಸ್ಸಿನ ಈ ವಿಧವಾದ ಭಾವವನ್ನು ನಮಗೆ ತಿಳಿಸಬಾರದೆಂದಿದ್ದರೊ ಏನೊ, ಅಂತು ಎದ್ದು ಹೊರಗೆ ಹೊರಟುಹೋದರು.

ಈ ಮಧ್ಯೆ ಗಿರೀಶಬಾಬುಗಳು ಶಿಷ್ಯನ ಕಡೆಗೆ ತಿರುಗಿ ಹೇಳಿದ್ದೇನೆಂದರೆ: ನೋಡಿದೆಯೇನಯ್ಯ ಎಂಥ ದೊಡ್ಡ ಹೃದಯ! ನಿಮ್ಮ ಸ್ವಾಮಿಗಳನ್ನು ಕೇವಲ ವೇದಪಂಡಿತನೆಂದು ನಾನು ಗೌರವಿಸುವುದಿಲ್ಲ; ಆದರೆ ಇಗೋ ಈಗ ಪ್ರಾಣಿಗಳ ಕಷ್ಟವನ್ನು ನೋಡಿ ಅಳುತ್ತ ಹೊರಟುಹೋದನಲ್ಲ, ಈ ಮಹಾ ಹೃದಯಕ್ಕೋಸ್ಕರ ಗೌರವಿಸುವುದು. ಕಣ್ಣೆದುರಿಗೆ ನೋಡಿದೆಯಷ್ಟೆ! ಜನರ ದುಃಖಕಷ್ಟಗಳ ಮಾತನ್ನು ಕೇಳಿ ಕರುಣೆಯಿಂದ ಹೃದಯವು ತುಂಬಿಹೋಗಿ ಸ್ವಾಮಿಗಳ ವೇದವೇದಾಂತವೆಲ್ಲಾ ಎಲ್ಲಿಯೋ ಓಡಿಹೋಯಿತು.

ಶಿಷ್ಯ: ಮಹಾಶಯರೆ, ನನಗೆ ಸೊಗಸಾಗಿ ವೇದಪಾಠವು ನಡೆಯುತ್ತಿತ್ತು; ತಾವು ಮಾಯಾ ಜಗತ್ತಿನ ಕೆಲವು ಕೆಲಸಕ್ಕೆ ಬಾರದ ಮಾತುಗಳನ್ನು ತೆಗೆದು ಸ್ವಾಮಿಗಳ ಮನಸ್ಸನ್ನು ಕೆಡಿಸಿಬಿಟ್ಟಿರಿ.

ಗಿರೀಶಬಾಬು: ಜಗತ್ತಿನಲ್ಲಿ ಇಂಥ ದುಃಖಕಷ್ಟಗಳಿವೆ. ಈಗ ಇವನ್ನು ಕೊನೆಗೆ ಒಂದು ಸಾರಿ ಕಣ್ಣು ಬಿಟ್ಟು ಕೂಡ ನೋಡದೆ ಸುಮ್ಮನೆ ಕುಳಿತುಕೊಂಡು ಬರಿಯ ವೇದವನ್ನು ಓದುತ್ತಾನಂತೆ! ನಿನ್ನ ವೇದವೇದಾಂತವನ್ನೆಲ್ಲಾ ಅತ್ತಲಾಗಿ ತೆಗೆದಿಡು.

ಶಿಷ್ಯ: ತಾವು ಭಾವೋದ್ರೇಕವನ್ನುಂಟುಮಾಡುವ ಮಾತುಗಳನ್ನು ಮಾತ್ರವೇ ಕೇಳುವುದಕ್ಕೆ ಇಷ್ಟಪಡುತ್ತೀರಿ; ಏಕೆಂದರೆ ತಾವು ಹೃದಯವುಳ್ಳವರಲ್ಲವೇನು! ಆದರೆ ಇದೆಲ್ಲಾ ಶಾಸ್ತ್ರ, ಇದರಲ್ಲಿ ವಿಚಾರವನ್ನು ಹೂಡಿದರೆ ಪ್ರಪಂಚ ಮರೆತು ಹೋಗುವುದು; ಇದರಲ್ಲಿ ತಮಗೆ ಆಸಕ್ತಿ ಇರುವಂತೆ ನನಗೆ ಕಂಡುಬರಲಿಲ್ಲ; ಹಾಗಿದ್ದಿದ್ದರೆ ತಾವು ಹೀಗೆ ರಸಭಂಗ ಮಾಡುತ್ತಿರಲಿಲ್ಲ.

ಗಿರೀಶಬಾಬು: ಹಾಗಾದರೆ ನಾನು ಹೇಳುತ್ತೇನೆ; ಜ್ಞಾನ ಮತ್ತು ಪ್ರೇಮ ಇವೆರಡೂ ಹೇಗೆ ಬೇರೆಬೇರೆಯೆಂಬುದನ್ನು ನನಗೆ ತೋರಿಸಿಕೊಡು ನೋಡೋಣ. ಇದನ್ನೇ ನೋಡು, ನಿನ್ನ ಗುರು ಇದ್ದಾನಲ್ಲ, ಆತನು ಎಷ್ಟು ಪಂಡಿತನೊ ಅಷ್ಟು ಭಾವುಕ. ನಿನ್ನ ವೇದವೂ ಹೇಳುವುದಿಲ್ಲವೆ ‘ಸತ್ ಚಿತ್ ಆನಂದ’ ಈ ಮೂರೂ ಒಂದೇ ಪದಾರ್ಥವೆಂದು? ಇಲ್ಲಿ ನೋಡಲಿಲ್ಲವೇ? ಸ್ವಾಮೀಜಿ ಅಷ್ಟು ಪಾಂಡಿತ್ಯವನ್ನು ತೋರಿಸಿದರು. ಆದರೆ ಪ್ರಪಂಚದ ದುಃಖದ ವೃತ್ತಾಂತವು ಕಿವಿಯಲ್ಲಿ ಬಿದ್ದು ಮನಸ್ಸಿಗೆ ಬಂದ ಕೂಡಲೆ, ಪ್ರಾಣಿಗಳ ಕಷ್ಟಕ್ಕಾಗಿ ಅಳುವುದಕ್ಕೆ ಮೊದಲು ಮಾಡಿದರು. ವೇದವೇದಾಂತಗಳು ಜ್ಞಾನ ಮತ್ತು ಪ್ರೇಮಗಳಲ್ಲಿ ಭೇದವನ್ನು ಪ್ರಾಮಾಣ್ಯದಿಂದ ಹೇಳುವುದಾದರೆ ಅಂಥ ವೇದವೇದಾಂತಗಳು ನನ್ನ ತಲೆಯನ್ನು ಪ್ರವೇಶಿಸದೆ ಇರಲಿ.

ಶಿಷ್ಯನು ಮಾತೇ ಹೊರಡದೆ ನಿಜವಾಗಿಯೂ ಗಿರೀಶಬಾಬುಗಳ ಸಿದ್ಧಾಂತ ವೇದಕ್ಕೆ ವಿರುದ್ಧವಾದುದಲ್ಲ ಎಂದು ಭಾವಿಸಿಕೊಳ್ಳುತ್ತಿದ್ದನು.

ಈ ಮಧ್ಯೆ ಸ್ವಾಮೀಜಿ ಹಿಂತಿರುಗಿ ಬಂದು ಶಿಷ್ಯನನ್ನು ಕರೆದು “ಏನಯ್ಯಾ ನೀನು ಆಡುತ್ತಿದ್ದ ಮಾತೇನು?" ಎಂದು ಕೇಳಿದರು. ಅದಕ್ಕೆ ಶಿಷ್ಯನು “ಎಲ್ಲಾ ಬರಿಯ ವೇದದ ಮಾತೇ ನಡೆಯುತ್ತಿತ್ತು; ಇವರು ಈ ಗ್ರಂಥವೊಂದನ್ನೂ ಓದಿಲ್ಲ; ಆದರೆ ಅವುಗಳ ಸಿದ್ಧಾಂತಗಳನ್ನು ಮಾತ್ರ ಚೆನ್ನಾಗಿ ಸರಿಯಾಗಿ ತಿಳಿದುಕೊಂಡಿದ್ದಾರೆ. ಇದು ತುಂಬ ಆಶ್ಚರ್ಯಕರವಾದ ಸಂಗತಿ" ಎಂದು ಉತ್ತರ ಕೊಟ್ಟನು.

ಸ್ವಾಮೀಜಿ: ಗುರುಭಕ್ತಿಯಿದ್ದರೆ ಎಲ್ಲಾ ಸಿದ್ಧಾಂತಗಳು ಪ್ರತ್ಯಕ್ಷವಾಗುತ್ತವೆ. ಓದಿ ಕೇಳಿ ಮಾಡಬೇಕಾದ ಆವಶ್ಯಕತೆ ಇಲ್ಲ. ಆದರೆ ಇಂಥ ಭಕ್ತಿವಿಶ್ವಾಸಗಳು ಜಗತ್ತಿನಲ್ಲಿ ದುರ್ಲಭ. ಗಿರೀಶಬಾಬುಗಳಂತೆ ಯಾರಿಗೆ ಭಕ್ತಿ ವಿಶ್ವಾಸಗಳಿವೆಯೋ ಅವರು ಶಾಸ್ತ್ರವನ್ನೋದಬೇಕಾದದ್ದಿಲ್ಲ. ಆದರೆ ಇತರರು ಅವರನ್ನು (ಗಿರೀಶಬಾಬುಗಳನ್ನು) ಅನುಕರಣ ಮಾಡುವುದಕ್ಕೆ ಹೋದರೆ ಅಂಥವರಿಗೆ ಸರ್ವನಾಶವಾದೀತು. ಅವರ ವಿಷಯವನ್ನು ಕಿವಿಯಿಂದ ಕೇಳಿಕೊಂಡು ಹೋಗಿ; ಎಂದಿಗೂ ಅವರನ್ನು ನೋಡಿಕೊಂಡು ಕೆಲಸಮಾಡುವುದಕ್ಕೆ ಹೋಗಬೇಡಿ.

ಶಿಷ್ಯ: ಅಪ್ಪಣೆ.

ಸ್ವಾಮೀಜಿ: ಅಪ್ಪಣೆಯೆಂದರಾಗಲಿಲ್ಲ! ಏನೇನು ಹೇಳಿದೆ ಆ ವಿಷಯಗಳನ್ನೆಲ್ಲಾ ತಿಳಿದುಕೊ, ಮೂರ್ಖನ ಹಾಗೆ ಎಲ್ಲಕ್ಕೂ ಸುಮ್ಮನೆ ರುಜುಹಾಕಿಕೊಂಡು ಹೋಗಬೇಡ. ನಾನು ಹೇಳಿದರೂ ನಂಬಬೇಡ; ತಿಳಿದರೆ, ಆಮೇಲೆ ಒಪ್ಪಿಕೊ. ನನಗೆ ಪರಮಹಂಸರು ತಮ್ಮ ಮಾತುಗಳನ್ನೆಲ್ಲಾ ಗ್ರಹಿಸಿಕೊಳ್ಳಬೇಕೆಂದು ಸರ್ವದಾ ಹೇಳುತ್ತಿದ್ದರು. ಸದ್ಯುಕ್ತಿ ತರ್ಕಶಾಸ್ತ್ರ ಇವೆಲ್ಲಾ ಏನನ್ನು ಹೇಳುತ್ತವೆಯೋ ಅದನ್ನು ಅನುಸರಿಸಿ ಆ ಮಾರ್ಗದಲ್ಲಿ ಹೋಗುತ್ತಿದ್ದರು. ವಿಚಾರ ಮಾಡುತ್ತ ಮಾಡುತ್ತ ಬುದ್ಧಿಯ ಪರಿಷ್ಕಾರವಾಗುತ್ತದೆ, ಆಮೇಲೆ ಅದರಲ್ಲಿ ಬ್ರಹ್ಮ ಪ್ರತಿಫಲಿತವಾಗುತ್ತದೆ. ತಿಳಿಯಿತೆ?

ಶಿಷ್ಯ: ಹುಂ; ಆದರೆ ನಾನಾ ಜನರ ನಾನಾ ಮಾತು ಕೇಳಿ ಬುದ್ಧಿ ಸಮನಾಗಿರುವುದಿಲ್ಲ; ಗಿರೀಶಬಾಬುಗಳು ‘ಅದನ್ನೆಲ್ಲಾ ಓದಿದರೆ ಆಗುವುದೇನು?’ ಎಂದರು; ತಾವು ವಿಚಾರಮಾಡಬೇಕೆಂದು ಹೇಳುವಿರಿ; ಈಗ ಏನು ಮಾಡುವುದು?

ಸ್ವಾಮೀಜಿ: ನಮ್ಮ ಇಬ್ಬರ ಮಾತೂ ನಿಜ. ಆದರೆ ಎರಡು ಬೇರೆ ಬೇರೆ ದೃಷ್ಟಿಗಳಿಂದ ನಮ್ಮಿಬ್ಬರ ಮಾತುಗಳನ್ನು ಆಡಬೇಕಾಗಿದೆ - ಅಷ್ಟೆ. ಒಂದು ಅವಸ್ಥೆ ಇದೆ, ಅಲ್ಲಿ ಯುಕ್ತಿ ತರ್ಕ ಎಲ್ಲಾ ಹೋಗುತ್ತದೆ - ‘ಮೂಕಾಸ್ವಾದನವತ್’ - ಮೂಕನು ಸವಿಯುವಂತೆ. ಮತ್ತೊಂದು ಅವಸ್ಥೆ ಇದೆ. ಅದರಲ್ಲಿ ವೇದಾದಿ ಶಾಸ್ತ್ರಗ್ರಂಥಗಳ ಸಮಾಲೋಚನೆ, ಓದುವುದು, ಕೇಳುವುದು ಎಲ್ಲ ಇರುತ್ತವೆ. ಓದಿ ಕೇಳಿ ಮಾಡುತ್ತ ಮಾಡುತ್ತ ಸತ್ಯವಸ್ತುವು ಪ್ರತ್ಯಕ್ಷವಾಗುತ್ತದೆ. ನೀವು ಇವುಗಳನ್ನು ಓದಿ ಕೇಳಿ ಮಾಡಬೇಕಾಗಿದೆ; ಹಾಗಾದರೆ ನಿನಗೆ ಸತ್ಯ ಪ್ರತ್ಯಕ್ಷವಾಗುತ್ತದೆ - ತಿಳಿಯಿತೆ?

ದಡ್ಡ ಶಿಷ್ಯನು ಸ್ವಾಮಿಗಳ ಈ ವಿಧವಾದ ಅಪ್ಪಣೆಯಿಂದ ಗಿರೀಶಬಾಬುಗಳಿಗೆ ಸೋಲಾಯಿತೆಂದು ಮನಸ್ಸಿನಲ್ಲಿ ಮಾಡಿಕೊಂಡು, ಗಿರೀಶಬಾಬುಗಳ ಕಡೆಗೆ ನೋಡುತ್ತ, “ಮಹಾಶಯರೇ, ನಾನು ವೇದ ವೇದಾಂತಗಳನ್ನು ಓದಿ ವಿಚಾರಮಾಡಬೇಕೆಂದು ಸ್ವಾಮಿಗಳು ಹೇಳಿದ್ದನ್ನು ಕೇಳಿದಿರಷ್ಟೇ!" ಎಂದು ನುಡಿದನು.

ಗಿರೀಶಬಾಬು: ನೀನು ಹಾಗೆಯೇ ಮಾಡು ಹೋಗು. ಸ್ವಾಮಿಗಳ ಆಶೀರ್ವಾದದಿಂದ ಹಾಗೆ ಮಾಡಿಯೆ ನಿನಗೆ ಎಲ್ಲಾ ಒಳ್ಳೆಯದಾಗುತ್ತದೆ.

ಸದಾನಂದ ಸ್ವಾಮಿಗಳು ಈ ಸಮಯದಲ್ಲಿ ಅಲ್ಲಿಗೆ ಬಂದು ಕುಳಿತುಕೊಂಡರು. ಸ್ವಾಮೀಜಿ ಅವರನ್ನು ನೋಡಿ “ಅಯ್ಯಾ, ಈ ಜಿ.ಸಿ.ಯ ಬಾಯಿಯಿಂದ ದೇಶದ ದುರ್ದಶೆಯನ್ನು ಕೇಳಿ ಮನಸ್ಸು ಮಿಲಮಿಲನೆ ಒದ್ದಾಡುತ್ತಿದೆ. ದೇಶಕ್ಕೋಸ್ಕರ ಏನಾದರೂ ಮಾಡಬಲ್ಲಿರಾ?" ಎಂದರು.

ಸದಾನಂದ: ಮಹಾಸ್ವಾಮಿ, ಅಪ್ಪಣೆ - ಸೇವಕನು ಸಿದ್ಧನಾಗಿದ್ದಾನೆ.

ಸ್ವಾಮೀಜಿ: ಮೊದಲು ಸಣ್ಣ ಕ್ರಮದಲ್ಲಿ ಒಂದು ಸೇವಾಶ್ರಮವನ್ನು ಸ್ಥಾಪಿಸಿ. ಅದರಲ್ಲಿ ದರಿದ್ರರೂ ದುಃಖಿಗಳೂ ಎಲ್ಲರೂ ಸಹಾಯ ಪಡೆಯಬೇಕು. ರೋಗಿಗಳಿಗೆ ಉಪಚಾರ ಮಾಡಬೇಕು. ಯಾರಿಗೆ ನೀನು ಎನ್ನುವವರು ಯಾರೂ ಇಲ್ಲವೊ ಅಂಥ ದಿಕ್ಕಿಲ್ಲದ ಜನರಿಗೆ ಜಾತಿವರ್ಣಗಳನ್ನು ಲಕ್ಷ್ಯಮಾಡದೆ ಸೇವೆ ಮಾಡಬೇಕು. ಗೊತ್ತಾಯಿತೆ?

ಸದಾನಂದ: “ಅಪ್ಪಣೆ ಮಹಾಸ್ವಾಮಿ?"

ಸ್ವಾಮೀಜಿ: ಜೀವ ಸೇವೆಗಿಂತ ಮತ್ತೊಂದು ಧರ್ಮವಿಲ್ಲ. ಸೇವಾಧರ್ಮವನ್ನು ಸರಿಯಾಗಿ ನಡೆಸಿಕೊಂಡು ಬರಲು ಸಮರ್ಥನಾದರೆ ಸುಲಭವಾಗಿಯೇ ಸಂಸಾರ ಬಂಧನವು ಕತ್ತರಿಸಿ ಹೋಗುತ್ತದೆ - ‘ಮುಕ್ತಿಃ ಕರಫಲಾಯತೇ’ - ಮುಕ್ತಿಯು ಅಂಗೈನ ನೆಲ್ಲಿಕಾಯಾಗುತ್ತದೆ.

ಈಗ ಗಿರೀಶಬಾಬುಗಳನ್ನು ಸಂಬೋಧಿಸಿ ಸ್ವಾಮಿಗಳು ಹೇಳಿದ್ದೇನೆಂದರೆ - ನೋಡು ಗಿರೀಶಬಾಬು, ನನಗೆ ಎನ್ನಿಸುತ್ತದೆ - ಈ ಜಗತ್ತಿನ ದುಃಖವನ್ನು ಹೋಗಲಾಡಿಸುವುದಕ್ಕೆ ನಾನು ಸಹಸ ಜನ್ಮವನ್ನೆ ತ್ತಬೇಕಾದರೂ ಎತ್ತುವೆನು! ಅದರಿಂದ ಯಾರಿಗಾದರೂ ಒಂದಿಷ್ಟು ದುಃಖಹೋಗುವುದಾದರೂ ಹಾಗೆ ಮಾಡುವೆನು. ಬರಿಯ ಒಬ್ಬನ ಮುಕ್ತಿಯನ್ನು ಕಟ್ಟಿಕೊಂಡು ಆಗುವುದೇನು? ಎನ್ನಿಸುತ್ತದೆ. ಎಲ್ಲರನ್ನೂ ಜೊತೆಯಲ್ಲಿ ಕರೆದುಕೊಂಡು ಆ ಮಾರ್ಗದಲ್ಲಿ ಹೋಗಬೇಕು. ಏಕೆ ಈ ಭಾವ ಉಂಟಾಯಿತು ಹೇಳಬಲ್ಲೆಯಾ?

ಗಿರೀಶಬಾಬು: ಹಾಗಲ್ಲದಿದ್ದರೆ ಪರಮಹಂಸರು ನಿನ್ನನ್ನು ಎಲ್ಲರಿಗಿಂತಲೂ ದೊಡ್ಡ ಆಧ್ಯಾತ್ಮಿಕ ಶಕ್ತಿ ಎಂದು ಹೇಳುತ್ತಿರಲಿಲ್ಲ.

ಇಷ್ಟು ಹೇಳಿ ಗಿರೀಶಬಾಬುಗಳು ಬೇರೆ ಕೆಲಸವಿದೆಯೆಂದು ಅಪ್ಪಣೆ ಪಡೆದು ಹೊರಟುಹೋದರು.

\newpage

\chapter[ಅಧ್ಯಾಯ ೧೧]{ಅಧ್ಯಾಯ ೧೧\protect\footnote{\engfoot{C.W, Vol. VI, P. 503}}}

\begin{center}
ಸ್ಥಳ: ಆಲಂಬಜಾರ್ ಮಠ, ವರ್ಷ: ಕ್ರಿ.ಶ. ೧೮೯೭.
\end{center}

ಸ್ವಾಮೀಜಿ ಮೊದಲಸಾರಿ ವಿಲಾಯಿತಿಯಿಂದ ಹಿಂತಿರುಗಿ ಕಲ್ಕತ್ತೆಗೆ ಬಂದ ಮೇಲೆ ಉತ್ಸಾಹಶಾಲಿಗಳಾದ ಬಹುಜನ ಯುವಕರು ಸ್ವಾಮಿಗಳಲ್ಲಿ ಬಂದು ಹೋಗುತ್ತಿದ್ದರೆಂದು ಹಿಂದೆಯೇ ಹೇಳಿದ್ದೇವೆ. ಈ ಸಮಯದಲ್ಲಿ ಸ್ವಾಮೀಜಿ ಯುವಕರಿಗೆ ಬ್ರಹ್ಮಚರ್ಯ ಮತ್ತು ತ್ಯಾಗದ ವಿಷಯವಾಗಿ ಸರ್ವದಾ ಉಪದೇಶ ಮಾಡುತ್ತಲೂ, ಸಂನ್ಯಾಸವನ್ನು ಎಂದರೆ ತಮ್ಮ ಮೋಕ್ಷ ಮತ್ತು ಜಗತ್ತಿನ ಕಲ್ಯಾಣಕ್ಕಾಗಿ ಸರ್ವಸ್ವವನ್ನೂ ತ್ಯಾಗ ಮಾಡುವಿಕೆಯನ್ನು ಕೈಗೊಳ್ಳಬೇಕೆಂದು ಬಹುವಾಗಿ ಅವರನ್ನು ಉತ್ಸಾಹಗೊಳಿಸುತ್ತಲೂ ಇದ್ದುದನ್ನು ನೋಡಿದ್ದೇವೆ. ಸಂನ್ಯಾಸ ತೆಗೆದುಕೊಳ್ಳದಿದ್ದರೆ ಯಾರಿಗೂ ಯಾಥಾರ್ಥವಾದ ಆತ್ಮಜ್ಞಾನವುಂಟಾಗುವುದು ಸಾಧ್ಯವಲ್ಲ; ಅಷ್ಟೇಕೆ? ಬಹುಜನ ಹಿತಕರವೂ ಬಹುಜನ ಸುಖಕರವೂ ಆದ ಯಾವ ವಿಧವಾದ ಐಹಿಕ ಕಾರ್ಯಕ್ರಮಗಳನ್ನು ಮಾಡುವುದೂ, ಅದರಿಂದ ಸಿದ್ಧಿಯನ್ನು ಪಡೆಯುವುದೂ ಸಂನ್ಯಾಸವಿಲ್ಲದೆ ಆಗುವುದಿಲ್ಲ - ಎಂದು ಅವರು ಅನೇಕವೇಳೆ ಹೇಳಿದ್ದನ್ನು ನಾವು ಕೇಳಿದ್ದೇವೆ. ಅವರು ಯಾವಾಗಲೂ ಉತ್ಸಾಹಶಾಲಿಗಳಾದ ಯುವಕರ ಮುಂದೆ ತ್ಯಾಗದ ಉನ್ನತ ಆದರ್ಶವನ್ನು ಇಡುತ್ತಿದ್ದರು; ಮತ್ತು ಯಾರಾದರೂ ಸಂನ್ಯಾಸ ತೆಗೆದುಕೊಳ್ಳಬೇಕೆಂಬ ಅಭಿಪ್ರಾಯವನ್ನು ತೋರಿಸಿದರೆ ಅವರಿಗೆ ಹೆಚ್ಚು ಉತ್ಸಾಹವನ್ನುಂಟುಮಾಡಿ ಕೃಪೆಯನ್ನು ತೋರಿಸುತ್ತಿದ್ದರು. ಅವರ ಉತ್ಸಾಹದ ಮಾತುಗಳಿಂದ ಆಗ ಕೆಲವು ಜನ ಭಾಗ್ಯಶಾಲಿಗಳಾದ ಯುವಕರು ಸಂಸಾರಾ ಶ್ರಮವನ್ನು ತ್ಯಾಗಮಾಡಿ ಅವರಿಂದಲೇ ಸಂನ್ಯಾಸಾಶ್ರಮವನ್ನು ಕೈಕೊಂಡರು. ಇವರಲ್ಲಿ ಮೊದಲು ನಾಲ್ಕು ಜನರಿಗೆ ಸಂನ್ಯಾಸ ಕೊಟ್ಟರಷ್ಟೆ. ಆ ದಿನ ಶಿಷ್ಯನ ಮನಸ್ಸಿನಲ್ಲಿ ಇವೊತ್ತಿಗೂ ನೆನಪಿನಲ್ಲಿದೆ.

ಸ್ವಾಮಿ ನಿತ್ಯಾನಂದ, ವಿರಜಾನಂದ, ಪ್ರಕಾಶಾನಂದ ಮತ್ತು ನಿರ್ಭಯಾನಂದ ಎಂಬ ಹೆಸರುಗಳನ್ನು ಪಡೆದು ಶ‍್ರೀರಾಮಕೃಷ್ಟ ಮಂಡಲಿಯಲ್ಲಿ ಈಗ ಯಾರು ಸುಪರಿಚಿತರಾಗಿದ್ದಾರೆಯೋ ಅವರೇ ಆ ದಿನ ಸಂನ್ಯಾಸವನ್ನು ಪಡೆದವರು. ಮಠದ ಸಂನ್ಯಾಸಿಗಳಿಂದ ಶಿಷ್ಯನು ಅನೇಕ ಸಲ ಕೇಳಿರುವುದೇನೆಂದರೆ - ಅವರಲ್ಲಿ ಒಬ್ಬರಿಗೆ ಸಂನ್ಯಾಸ ಕೊಡಿಸುವ ವಿಷಯದಲ್ಲಿ ಸ್ವಾಮೀಜಿಯ ಗುರುಭ್ರಾತೃಗಳು ಆಕ್ಷೇಪಿಸಿದ್ದರು. ಸ್ವಾಮಿಜಿ ಅದಕ್ಕೆ ಉತ್ತರವಾಗಿ “ನಾವೂ ಪಾಪಿ ತಾಪಿ ದೀನ ದುಃಖ ಪತಿತ ಇಂಥವರ ಉದ್ಧಾರದ ವಿಷಯದಲ್ಲಿ ಹಿಂಜರಿದರೆ ಅದನ್ನು ಮಾಡುವವರು ಇನ್ಯಾರು? ನೀವು ಈ ವಿಷಯದಲ್ಲಿ ಯಾವ ವಿಧವಾಗಿಯೂ ಅಡ್ಡಿಯಾಗಬೇಡಿ" ಎಂದರು. ಸ್ವಾಮಿಜಿಯವರ ಬಲಿಷ್ಠವಾದ ಇಚ್ಛೆಯೇ ಕೈಗೂಡಿತು. ಅನಾಥ ಶರಣರಾದ ಸ್ವಾಮಿಜಿ ತಮ್ಮ ಕೃಪಾಗುಣದಿಂದ ಆತನಿಗೆ ಸಂನ್ಯಾಸ ಕೊಡುವುದಕ್ಕೆ ಮನಸ್ಸು ಮಾಡಿದರು.

ಶಿಷ್ಯನು ಈಗ ಎರಡು ದಿನಗಳಿಂದ ಮಠದಲ್ಲಿಯೆ ಇದ್ದಾನೆ. ಸ್ವಾಮಿಜಿ ಶಿಷ್ಯನನ್ನು ಕುರಿತು “ನೀನು ಭಟ್ಟಾಚಾರ್ಯ ಬ್ರಾಹ್ಮಣ; ನಾಳೆ ಬೆಳಿಗ್ಗೆ ನೀನೇ ಇವರ ಶ್ರಾದ್ಧ ಮಾಡಿಸು; ಮರುದಿನ ಇವರೆಲ್ಲರಿಗೂ ಸಂನ್ಯಾಸ ಕೊಡಿಸುತ್ತೇನೆ. ಇವತ್ತು ಪಂಚಾಂಗ ಆಗಮಾದಿಗಳನ್ನು ನೋಡಿಬಿಡು" ಎಂದು ಹೇಳಿದರು. ಶಿಷ್ಯನು ಸ್ವಾಮಿಜಿಯವರ ಆಜ್ಞೆಯನ್ನು ಶಿರಸಾವಹಿಸಿದನು.

ಸಂನ್ಯಾಸ ತೆಗೆದುಕೊಳ್ಳುವ ಹಿಂದಿನ ದಿನ ಸಂನ್ಯಾಸವ್ರತವನ್ನು ಸ್ವೀಕರಿಸುವುದಕ್ಕೆ ನಿಶ್ಚಯಮಾಡಿಕೊಂಡು ಮೇಲೆ ಹೇಳಿದ ನಾಲ್ಕುಜನ ಬ್ರಹ್ಮಚಾರಿಗಳು ತಲೆಯನ್ನು ಮುಂಡನ ಮಾಡಿಕೊಂಡು ಗಂಗೆಯಲ್ಲಿ ಸ್ನಾನ ಮಾಡಿ ಶುಭ್ರವಸ್ತ್ರವನ್ನುಟ್ಟು ಸ್ವಾಮಿಜಿಯವರ ಪಾದಪದ್ಮಗಳನ್ನು ವಂದಿಸಿ, ಅವರ ಸ್ನೇಹಾಶೀರ್ವಾದವನ್ನು ಪಡೆದು ಶ್ರಾದ್ಧ ಮಾಡುವುದಕ್ಕೆ ಸಿದ್ಧರಾದರು.

ಇಲ್ಲಿ ಒಂದು ಸಂಗತಿಯನ್ನು ಹೇಳಿದರೆ ಅಸಂಬದ್ಧವಾಗಲಾರದು. ಅದೇನೆಂದರೆ, ಶಾಸ್ತ್ರರೀತಿ ಸಂನ್ಯಾಸಾಶ್ರಮವನ್ನು ತೆಗೆದುಕೊಳ್ಳುವವರು ಆ ಕಾಲದಲ್ಲಿ ತಮ್ಮ ಶ್ರಾದ್ಧವನ್ನು ತಾವೇ ಮಾಡಿಕೊಳ್ಳಬೇಕು. ಏಕೆಂದರೆ ಸಂನ್ಯಾಸವನ್ನು ತೆಗೆದುಕೊಂಡರೆ ಲೌಕಿಕ ಅಥವಾ ವೈದಿಕ ಯಾವ ವಿಷಯದಲ್ಲಿಯೂ ಅವರಿಗೆ ಅಲ್ಲಿಂದ ಮುಂದಕ್ಕೆ ಅಧಿಕಾರವಿರುವುದಿಲ್ಲ. ಮಕ್ಕಳು ಮೊಮ್ಮಕ್ಕಳು ಮುಂತಾದವರು ಮಾಡುವ ಶ್ರಾದ್ಧ ಪಿಂಡ ದಾನಾದಿಗಳ ಫಲ ಅವರನ್ನು ಅಲ್ಲಿಂದ ಮುಂದೆ ಸ್ಪರ್ಶಮಾಡಲಾರವು. ಆದ್ದರಿಂದ ಸಂನ್ಯಾಸ ತೆಗೆದುಕೊಳ್ಳುವುದಕ್ಕೆ ಮೊದಲು ತನ್ನ ಶ್ರಾದ್ಧವನ್ನು ತಾನೆ ಮಾಡಿಕೊಳ್ಳಬೇಕು; ತನ್ನ ಕಾಲಿನ ಮೇಲೆ ತನ್ನ ಪಿಂಡವನ್ನು ಹಾಕಿಕೊಂಡು ಸಂಸಾರದ ಅಷ್ಟೇ ಏಕೆ, ತನ್ನ ದೇಹದ ಪೂರ್ವ ಸಂಬಂಧಾದಿಗಳನ್ನು ಸಂಕಲ್ಪದ್ವಾರಾ ಪೂರ್ತಿಯಾಗಿ ತ್ಯಜಿಸಲು ಸಾಧನ ಮಾಡಬೇಕು. ಇದನ್ನು ಸಂನ್ಯಾಸ ತೆಗೆದುಕೊಳ್ಳುವುದರ ಅಧಿವಾಸಕ್ರಿಯೆಯೆಂದು ಹೇಳಬಹುದು. ಸ್ವಾಮೀಜಿ ಈ ವೈದಿಕ ಕರ್ಮಗಳಲ್ಲೆಲ್ಲಾ ಪೂರ್ಣವಾದ ನಂಬುಗೆಯನ್ನಿಟ್ಟುಕೊಂಡಿದ್ದರು ಮತ್ತು ಈ ಕರ್ಮಗಳೆಲ್ಲಾ ಶಾಸ್ತ್ರೋಕ್ತವಾಗಿ ಸರಿಯಾಗಿ ನಡೆಯದಿದ್ದರೆ ಬಹು ಅಸಮಾಧಾನಪಟ್ಟುಕೊಳ್ಳುತ್ತಿದ್ದರು. ಇದನ್ನು ಶಿಷ್ಯನು ಗಮನಿಸಿದನು. ಈಗಿನ ಕಾಲದಲ್ಲಿ ಕಾವಿಯ ಬಟ್ಟೆಯನ್ನುಟ್ಟು ಹೊರಗೆ ಬಂದುಬಿಟ್ಟರೆ ಅನೇಕರು ಸಂನ್ಯಾಸ ದೀಕ್ಷೆಯನ್ನು ಪಡೆದೆವೆಂದುಕೊಳ್ಳುತ್ತಾರಲ್ಲಾ, ಇದನ್ನು ಸ್ವಾಮೀಜಿ ಒಪ್ಪುತ್ತಿರಲಿಲ್ಲ. ಬಹುಕಾಲದಿಂದ ಬಂದ ಗುರುಪರಂಪರೆಯನ್ನು ಅನುಸರಿಸಿ ಬ್ರಹ್ಮಸಾಧನಕ್ಕೆ ಉಪಯುಕ್ತವಾದ ಸಂನ್ಯಾಸ ವ್ರತವನ್ನು ತೆಗೆದುಕೊಳ್ಳುವುದಕ್ಕೆ ಮೊದಲು ನಡೆಯಬೇಕಾದ ನೈಷ್ಠಿಕ ಸಂಸ್ಕಾರಗಳನ್ನು ಬ್ರಹ್ಮಚಾರಿಗಳಿಂದ ಯಥಾವತ್ತಾಗಿ ನಡೆಸುತ್ತಿದ್ದರು. ನಾವು ಮತ್ತೂ ಒಂದು ಸಂಗತಿಯನ್ನು ಕೇಳಿದ್ದೇವೆ; ಅದೇನೆಂದರೆ, ಪರಮಹಂಸರು ಪರಂಧಾಮವನ್ನು ಪಡೆದಬಳಿಕ ಸ್ವಾಮೀಜಿ ಉಪನಿಷತ್ತು ಮತ್ತು ಶಾಸ್ತ್ರಗಳಲ್ಲಿರುವ ಸಂನ್ಯಾಸ ವಿಧಿಪದ್ಧತಿಗಳನ್ನು ಶೇಖರಿಸಿಕೊಂಡು, ಗುರುಭ್ರಾತೃಗಳೊಡನೆ ಒಂದು ಕಡೆ ಪರಮಹಂಸರ ಚಿತ್ರಪಟದ ಸಮಕ್ಷಮದಲ್ಲಿ ವೈದಿಕ ರೀತಿಯನ್ನನುಸರಿಸಿ ಸಂನ್ಯಾಸವನ್ನು ತೆಗೆದುಕೊಂಡರಂತೆ.

ಆಲಂಬಜಾರ್ ಮಠದ ಮಹಡಿಯ ಮೇಲಿದ್ದ ಕೊಠಡಿಯಲ್ಲಿ ಶ್ರಾದ್ಧಕ್ಕೆ ಬೇಕಾದ ಪದಾರ್ಥಗಳೆಲ್ಲಾ ಸಿದ್ಧಮಾಡಲ್ಪಟ್ಟಿದ್ದುವು. ನಿತ್ಯಾನಂದಸ್ವಾಮಿಗಳು ಶ್ರಾದ್ಧಗಳನ್ನು ಅನೇಕಸಾರಿ ಮಾಡಿಸಿದ್ದಾರೆ; ಆದ್ದರಿಂದ ಬೇಕಾದ ಪದಾರ್ಥಗಳನ್ನು ಒದಗಿಸಿಕೊಳ್ಳುವುದರಲ್ಲಿ ಏನೂ ಕೊರತೆಯಿರಲಿಲ್ಲ. ಶಿಷ್ಯನು ಸ್ನಾನಮಾಡಿಕೊಂಡು ಸ್ವಾಮಿಜಿಯವರ ಅಪ್ಪಣೆಯಂತೆ ಪೌರೋಹಿತ್ಯ ಮಾಡಿಸುವುದಕ್ಕೆ ಕುಳಿತನು. ಮಂತ್ರಗಳನ್ನು ಹೇಳಿಕೊಡುವುದಕ್ಕೂ ಹೇಳುವುದಕ್ಕೂ ಆರಂಭವಾಯಿತು. ಸ್ವಾಮೀಜಿ ಒಂದೊಂದುಸಾರಿ ಬಂದು ಬಂದು ಹೋಗುತ್ತಿದ್ದರು. ಶ್ರಾದ್ಧದ ಕೊನೆಯಲ್ಲಿ ನಾಲ್ಕು ಜನ ಬ್ರಹ್ಮಚಾರಿಗಳೂ ಯಾವಾಗ ತಮ್ಮ ತಮ್ಮ ಪಾದಗಳ ಮೇಲೆ ಪಿಂಡಗಳನ್ನು ಹಾಕಿಕೊಂಡು ಇಂದಿನಿಂದ ಸಂಸಾರದ ಭಾಗಕ್ಕೆ ಸತ್ತವರಂತಾದರೋ ಆಗ ಶಿಷ್ಯನ ಮನಸ್ಸಿನಲ್ಲಿ ತುಂಬ ವ್ಯಾಕುಲತೆಯುಂಟಾಯಿತು. ಸಂನ್ಯಾಸದ ಕಠೋರತೆಯನ್ನು ನೆನೆದುಕೊಂಡು ಮಂಕುಬಡಿದಂತಾಯಿತು. ಪಿಂಡ ಮುಂತಾದುವುಗಳನ್ನು ತೆಗೆದುಕೊಂಡು ಗಂಗೆಗೆ ಹೊರಟುಹೋದ ಬಳಿಕ ಸ್ವಾಮೀಜಿ ಶಿಷ್ಯನ ವ್ಯಾಕುಲತೆಯನ್ನು ನೋಡಿ “ಇದನ್ನೆಲ್ಲಾ ನೋಡಿ ಕೇಳಿ ನಿನ್ನ ಮನಸ್ಸಿನಲ್ಲಿ ತುಂಬ ಭಯವುಂಟಾಗಿದೆ - ಅಲ್ಲವೇನಯ್ಯಾ?" ಎಂದು ಕೇಳಿದರು. ಶಿಷ್ಯನು ತಲೆಯನ್ನು ಬಗ್ಗಿಸಿಕೊಂಡು ಸಮ್ಮತಿಯನ್ನು ಸೂಚಿಸಿದ. ಸ್ವಾಮೀಜಿ ಅವನಿಗೆ ಹೀಗೆಂದು ಹೇಳಿದರು: ಸಂಸಾರದಲ್ಲಿ ಇಂದಿನಿಂದ ಇವರು ಸತ್ತಹಾಗೆ. ನಾಳೆಯಿಂದ ಇವರಿಗೆ ಹೊಸ ದೇಹ ಹೊಸ ಯೋಚನೆ ಹೊಸ ಬಟ್ಟೆಬರೆಗಳು ಬರುತ್ತವೆ. ಇವರು ಬ್ರಹ್ಮವೀರ್ಯದಲ್ಲಿ ಪ್ರದೀಪ್ತರಾದ ಉರಿಯುತ್ತಿರುವ ಬೆಂಕಿಯಂತಿರುತ್ತಾರೆ. “ನ ಕರ್ಮಣಾ ನ ಪ್ರಜಯಾ ಧನೇನ ತ್ಯಾಗೇನೈಕೇ ಅಮೃತತ್ವ ಮಾನಶುಃ” - ಕರ್ಮದಿಂದಾಗಲಿ, ಸಂತತಿಯಿಂದಾಗಲಿ, ಧನದಿಂದಾಗಲಿ ಅಮೃತತ್ವವು ಸಾಧ್ಯವಿಲ್ಲ. ತ್ಯಾಗದಿಂದಲೇ ಕೆಲವರು ಅಮೃತತ್ವವನ್ನು ಪಡೆದರು. ಧನದಿಂದ ಯಜ್ಞಯಾಗಾದಿಗಳಿಂದಲ್ಲ; ತ್ಯಾಗವೊಂದರಿಂದಲೇ ಅಮೃತತ್ವವನ್ನು ಪಡೆದರು.

ಸ್ವಾಮೀಜಿಯವರ ಮಾತನ್ನು ಕೇಳಿ ಶಿಷ್ಯನು ಮಾತು ಹೊರಡದೆ ನಿಂತುಕೊಂಡಿದ್ದನು. ಸಂನ್ಯಾಸದ ಕಠೋರತೆಯನ್ನು ನೆನೆದುಕೊಂಡು ಆತನ ಬುದ್ಧಿಯು ಸ್ತಂಭೀಭೂತವಾಗಿ ಹೋಗಿತ್ತು - ಶಾಸ್ತ್ರಜ್ಞಾನಾಡಂಬರಗಳಿಂದ ದೂರ ಹೋಗಿಬಿಟ್ಟಿತ್ತು. ಅವನು ಬರಿಯ ಮಾತಿಗೂ ಕಾರ್ಯಕ್ಕೂ ಎಷ್ಟು ಭೇದವಿದೆಯೆಂದು ಭಾವಿಸಿಕೊಳ್ಳತೊಡಗಿದನು.

ಶ್ರಾದ್ಧ ಮಾಡಿಕೊಂಡು ನಾಲ್ಕು ಜನ ಬ್ರಹ್ಮಚಾರಿಗಳು ಈ ಮಧ್ಯೆ ಗಂಗೆಯಲ್ಲಿ ಪಿಂಡವನ್ನು ಹಾಕಿ ಬಂದು ಸ್ವಾಮಿಜಿಯವರ ಪಾದಪದ್ಮಗಳಿಗೆ ನಮಸ್ಕಾರ ಮಾಡಿದರು. ಸ್ವಾಮಿಜಿ ಆಶೀರ್ವಾದಮಾಡಿ ನೀವು ಮಾನವ ಜೀವನದ ಶ್ರೇಷ್ಠ ವ್ರತವನ್ನು ಕೈಗೊಳ್ಳುವುದರಲ್ಲಿ ಉತ್ಸಾಹಿತರಾಗಿದ್ದೀರಿ; ನಿಮ್ಮ ಜನ್ಮ ಧನ್ಯ; ಧನ್ಯ ನಿಮ್ಮ ವಂಶ; ಧನ್ಯಳು ನಿಮ್ಮನ್ನು ಹೆತ್ತ ತಾಯಿ; ‘ಕುಲಂ ಪವಿತ್ರಂ ಜನನೀ ಕೃತಾರ್ಥಾ’ ಎಂದು ಹೇಳಿದರು.

ಆ ದಿವಸ ರಾತ್ರಿ ಊಟವಾದ ಮೇಲೆ ಸ್ವಾಮೀಜಿ ಕೇವಲ ಸಂನ್ಯಾಸ ಧರ್ಮದ ವಿಷಯವಾಗಿಯೆ ಮಾತುಕಥೆಗಳನ್ನಾಡುತ್ತಿದ್ದರು. ಸಂನ್ಯಾಸವ್ರತವನ್ನು ಕೈಗೊಳ್ಳುವುದರಲ್ಲಿ ಉತ್ಸುಕರಾಗಿದ್ದ ಬ್ರಹ್ಮಚಾರಿಗಳ ಕಡೆಗೆ ನೋಡುತ್ತ ಹೀಗೆಂದು ಹೇಳಿದರು: “ಆತ್ಮನೋ ಮೋಕ್ಷಾರ್ಥಂ ಜಗದ್ಧಿತಾಯ ಚ" - ತನ್ನ ಮುಕ್ತಿಗಾಗಿ ಮತ್ತು ಜಗತ್ತಿನ ಹಿತಕ್ಕಾಗಿ - ಇದೇ ಸಂನ್ಯಾಸದ ನಿಜವಾದ ಉದ್ದೇಶ, ಸಂನ್ಯಾಸ ವಿಲ್ಲದಿದ್ದರೆ ಯಾರೂ ಯಾವಾಗಲೂ ಬ್ರಹ್ಮಜ್ಞರಾಗಲಾರರು - ಈ ಮಾತನ್ನು ವೇದವೇದಾಂತಗಳು ಉದ್ಘೋಷಿಸಿ ಹೇಳುತ್ತವೆ. “ಈ ಸಂಸಾರವನ್ನು ನಡೆಸಿಕೊಂಡು ಹೋಗುತ್ತೇನೆ, ಬ್ರಹ್ಮಜ್ಞನೂ ಆಗುತ್ತೇನೆ" ಎಂದು ಯಾರು ಹೇಳುತ್ತಾರೆಯೋ ಅವರ ಮಾತಿಗೆ ಕಿವಿ ಕೊಡಬೇಡಿರಿ. ಅವೆಲ್ಲ ಪ್ರಚ್ಛನ್ನ ಭೋಗಿಗಳ ಪೊಳ್ಳು ಮಾತುಗಳು. ಒಂದಿಷ್ಟು ಸಂಸಾರ ಭೋಗೇಚ್ಛೆ ಯಾರಿಗೆ ಇರುತ್ತದೆಯೊ - ಒಂದಿಷ್ಟು ಆಶೆ ಯಾರಿಗೆ ಇರುತ್ತದೆಯೊ - ಅವರಿಗೆ ಇದು ಬಹು ಕಠಿನವಾದ ಮಾರ್ಗವೆಂದು ತೋರಿ ಭಯವಾಗುವುದು. ಅದಕ್ಕೋಸ್ಕರವೇ ತಮ್ಮ ಮನಸ್ಸಿಗೆ ಸಮಾಧಾನ ತಂದುಕೊಳ್ಳುವುದಕ್ಕೋಸ್ಕರ ‘ಈ ಪಕ್ಷ ಆ ಪಕ್ಷ ಎರಡು ಪಕ್ಷಗಳನ್ನೂ ಎಂದರೆ ಭೋಗ ತ್ಯಾಗಗಳೆರಡನ್ನೂ ನಡೆಸಿಕೊಂಡು ಹೋಗಬೇಕು’ ಎಂದು ಹೇಳಿಕೊಂಡು ತಿರುಗುತ್ತಾರೆ. ಅದೆಲ್ಲಾ ಹುಚ್ಚರ ಮಾತು, ಉನ್ಮತ್ತರ ಪ್ರಲಾಪ, ಅಶಾಸ್ತ್ರೀಯ, ಅವೈದಿಕ ಮತ. ತ್ಯಾಗವಿಲ್ಲದೆ ಮುಕ್ತಿಯಿಲ್ಲ. ತ್ಯಾಗವಿಲ್ಲದೆ ಪರಾಭಕ್ತಿ ಬರುವುದಿಲ್ಲ. ತ್ಯಾಗ ತ್ಯಾಗ - ‘ನಾನ್ಯಃ ಪಂಥಾ ವಿದ್ಯತೇ ಅಯನಾಯ’ - ಇದಲ್ಲದೆ ಬೇರೆ ದಾರಿ ಇಲ್ಲ. ಗೀತೆಯಲ್ಲಿಯೂ ಹೇಳಿದೆ - ‘ಕಾಮ್ಯಾನಾಂ ಕರ್ಮಣಾಂ ನ್ಯಾಸಂ ಸಂನ್ಯಾಸಂ ಕವಯೋ ವಿದುಃ’ - ಬಯಕೆಗಳನ್ನು ಗುರಿಯಾಗುಳ್ಳ ಎಲ್ಲ ಕರ್ಮಗಳನ್ನು ತ್ಯಜಿಸುವುದೇ ಸಂನ್ಯಾಸ.

ಸಂಸಾರದ ತಾಪತ್ರಯವನ್ನು ಬಿಟ್ಟುಬಿಡದಿದ್ದರೆ ಯಾರಿಗೂ ಮುಕ್ತಿ ಲಭಿಸುವುದಿಲ್ಲ. ಸಂಸಾರಾಶ್ರಮದಲ್ಲಿ ಯಾರು ಇರುತ್ತಾರೆಯೊ ಇವರು ಒಂದಲ್ಲದಿದ್ದರೆ ಮತ್ತೊಂದು ಆಶೆಗೆ ದಾಸರಾಗಿ ಬಂಧನದಲ್ಲಿ ಸಿಕ್ಕಿಕೊಂಡಿರುತ್ತಾರೆಂಬುದೇ ಇದಕ್ಕೆ ಪ್ರಮಾಣ. ಇಲ್ಲದಿದ್ದರೆ ಸಂಸಾರದಲ್ಲಿ ಹೇಗೆ ಇರುತ್ತಾನೆ? ಕಾಮಿನಿಯ ದಾಸ, ಇಲ್ಲದಿದ್ದರೆ ಕಾಂಚನದ ದಾಸ, ಇಲ್ಲದಿದ್ದರೆ ಮಾನ ಯಶಸ್ಸು ವಿದ್ಯೆ ಪಾಂಡಿತ್ಯ - ಇವುಗಳ ದಾಸ. ಈ ದಾಸತ್ವದಿಂದ ಹೊರಗೆ ಹೋಗಿ ಬಿದ್ದರೆ ಆಗ ಮುಕ್ತಿಮಾರ್ಗದಲ್ಲಿ ಮುಂದಾಗಲು ಸಮರ್ಥನಾದಾನು! ಯಾರು ಎಷ್ಟಾದರೂ ಹೇಳಲಿ, ನಾನೇನೋ ತಿಳಿದುಕೊಂಡಿದ್ದೇನೆ - ಇದನ್ನೆಲ್ಲಾ ಬಿಟ್ಟು ಬಿಡದಿದ್ದರೆ, ಸಂನ್ಯಾಸವನ್ನು ಕೈಗೊಳ್ಳದಿದ್ದರೆ ಯಾವುದರಿಂದಲೂ ಜೀವನಿಗೆ ರಕ್ಷಣವಿಲ್ಲ - ಯಾವುದರಿಂದಲೂ ಬ್ರಹ್ಮಜ್ಞಾನವಾಗುವ ಸಂಭವವಿಲ್ಲ.

ಶಿಷ್ಯ: ಮಹಾಶಯರೆ, ಸಂನ್ಯಾಸವನ್ನು ತೆಗೆದುಕೊಂಡರೆ ತಾನೇ ಸಿದ್ಧಿಯಾಗುತ್ತದೆಯೇ?

ಸ್ವಾಮೀಜಿ: ಸಿದ್ಧಿಯಾಗುತ್ತದೆಯೊ ಬಿಡುತ್ತದೆಯೋ ಅದು ಆಮೇಲಿನ ಮಾತು. ನೀನು ಎಂದಿನವರೆಗೆ ಈ ಭೀಷಣ ಸಂಸಾರದ ಎಲ್ಲೆಯಿಂದ ಹೊರಗೆ ಹೋಗಿ ಬೀಳಲಾರೆಯೊ - ಎಲ್ಲಿಯವರೆಗೆ ಆಶಾವಾಸನೆಗಳ ಗುಲಾಮಗಿರಿಯನ್ನು ಬಿಡಲಾರೆಯೊ - ಅಲ್ಲಿಯವರೆಗೆ ನಿನಗೆ ಭಕ್ತಿ ಮುಕ್ತಿ ಯಾವುದೂ ಬರುವುದಿಲ್ಲ. ಬ್ರಹ್ಮಜ್ಞನಿಗೆ ಸಿದ್ಧಿ ಋದ್ಧಿಗಳೆಲ್ಲಾ ಅತಿ ತುಚ್ಛವಾದುವುಗಳು.

ಶಿಷ್ಯ: ಮಹಾಶಯರೆ, ಸಂನ್ಯಾಸಕ್ಕೆ ಕಾಲಾಕಾಲಗಳಾಗಲಿ, ಪ್ರಕಾರ ಪ್ರಭೇದಗಳಾಗಲೀ ಇವೆಯೆ?

ಸ್ವಾಮೀಜಿ: ಸಂನ್ಯಾಸ ಧರ್ಮಸಾಧನಕ್ಕೆ ಕಾಲಾಕಾಲವಿಲ್ಲ. ಶ್ರುತಿ ಹೇಳುವುದೇನೆಂದರೆ, “ಯದಹರೇವ ವಿರಜೇತ್, ತದಹಹರೇವ ಪ್ರವ್ರಜೇತ್"- ಎಂದು ವೈರಾಗ್ಯ ಹುಟ್ಟುತ್ತದೆಯೋ ಅಂದೇ ಸಂನ್ಯಾಸಿಯಾಗಿ ಹೊರಟು ಹೋಗಬೇಕು. ಯೋಗವಾಸಿಷ್ಠದಲ್ಲಿಯೂ ಹೇಳಿದೆ –

\begin{verse}
“ಯುವೈವ ಧರ್ಮಶೀಲಃ ಸ್ಯಾತ್ ಅನಿತ್ಯಂ ಖಲು ಜೀವಿತಂ~।\\ಕೋ ಹಿ ಜಾನಾತಿ ಕಸ್ಯಾದ್ಯ ಮೃತ್ಯುಕಾಲೋ ಭವಿಷ್ಯತಿ~॥”
\end{verse}

ಜೀವನ ಅನಿತ್ಯವಾದುದರಿಂದ ಯೌವನದಲ್ಲಿಯೇ ಧರ್ಮಶೀಲನಾಗಬೇಕು. ಯಾರಿಗೆ ಯಾವಾಗ ಮೃತ್ಯು ಬಂದೊದಗುವುದೆಂಬುದನ್ನು ಯಾರು ಬಲ್ಲರು?

ಶಾಸ್ತ್ರದಲ್ಲಿ ನಾಲ್ಕು ವಿಧವಾದ ಸಂನ್ಯಾಸ ವಿಧಾನ ಕಂಡುಬರುತ್ತದೆ: (೧) ವಿದ್ವತ್ ಸಂನ್ಯಾಸ, (೨) ವಿವಿದಿಷಾ ಸಂನ್ಯಾಸ, (೩) ಮರ್ಕಟ ಸಂನ್ಯಾಸ ಮತ್ತು (೪) ಆತುರ ಸಂನ್ಯಾಸ. ಹಠಾತ್ತಾಗಿ ನಿಜವಾದ ವೈರಾಗ್ಯ ಹುಟ್ಟಿತು. ಆಗ ಸಂನ್ಯಾಸವನ್ನು ತೆಗೆದುಕೊಂಡು ಹೊರಟುಹೋದರೆ - ಇದು ಪೂರ್ವಜನ್ಮ ಸಂಸ್ಕಾರವಿಲ್ಲದಿದ್ದರೆ ಹುಟ್ಟುವುದಿಲ್ಲ - ಇದರ ಹೆಸರೇ ವಿದ್ವತ್ ಸಂನ್ಯಾಸ. ಆತ್ಮತತ್ತ್ವವನ್ನು ತಿಳಿಯುವ ಪ್ರಬಲವಾದ ಆಶೆಯಿಂದ ಶಾಸ್ತ್ರಪಾಠ ಸಾಧನಾದಿಗಳ ಮೂಲಕ ಸ್ವಸ್ವರೂಪವನ್ನು ತಿಳಿದುಕೊಳ್ಳುವುದಕ್ಕೋಸ್ಕರ ಯಾರಾದರೂ ಬ್ರಹ್ಮಜ್ಞ ಪುರುಷರ ಹತ್ತಿರ ಸಂನ್ಯಾಸವನ್ನು ತೆಗೆದುಕೊಂಡು ಅಧ್ಯಯನ ಸಾಧನ ಭಜನಗಳನ್ನು ಮಾಡುವುದಕ್ಕೆ ತೊಡಗಿದರೆ - ಇದಕ್ಕೆ ವಿವಿದಿಷಾ ಸಂನ್ಯಾಸವೆಂದು ಹೆಸರು. ಸಂಸಾರದಲ್ಲಿ ಏಟು ತಿಂದೊ, ನೆಂಟರಿಷ್ಟರನ್ನು ಕಳೆದುಕೊಂಡೋ ಅಥವಾ ಮತ್ತಾವ ಕಾರಣದಿಂದಲೋ ಕೆಲಕೆಲವರು ಹೊರಟುಹೋಗಿ ಸಂನ್ಯಾಸ ತೆಗೆದುಕೊಳ್ಳುವರು. ಆದರೆ ಈ ವೈರಾಗ್ಯ ಸ್ಥಿರವಾಗಿರುವುದಿಲ್ಲ - ಇದರ ಹೆಸರು ಮರ್ಕಟ ಸಂನ್ಯಾಸ. “ವೈರಾಗ್ಯ ಪಡೆದು ಪಶ್ಚಿಮದ ಕಡೆಗೆ ಹೋದ; ಹಾಗೆಯೇ ಒಂದು ಚಾಕರಿ ಸಂಪಾದಿಸಿಕೊಂಡ; ಆಮೇಲೆ ಬೇಕಾಗಿದ್ದರೆ ತನ್ನ ಪರಿವಾರವನ್ನು ಕರೆಸಿಕೊಂಡ; ಇಲ್ಲದಿದ್ದರೆ ಮರಳಿ ಮದುವೆ ಮಾಡಿಕೊಂಡುಬಿಟ್ಟ" ಎಂದು ಪರಮಹಂಸರು ಹೇಳುತ್ತಿದ್ದರಲ್ಲ ಹಾಗೆ. ಮತ್ತೊಂದು ವಿಧವಾದ ಸಂನ್ಯಾಸವಿದೆ; ಅದು ಹೇಗೆಂದರೆ ಒಬ್ಬನು ಮುಮೂರ್ಷು. ಅಂದರೆ ಬದುಕುವ ಆಶೆಯಿಲ್ಲದೆ ಕಾಯಿಲೆಯಲ್ಲಿ ನರಳುತ್ತ ಹಾಸಗೆಯ ಮೇಲೆ ಮಲಗಿದ್ದಾನೆ. ಅವನಿಗೆ ಆಗ ಸಂನ್ಯಾಸ ಕೊಡುವುದಕ್ಕೆ ವಿಧಿಯಿದೆ. ಅವನು ಸತ್ತರೆ ಪವಿತ್ರವಾದ ಸಂನ್ಯಾಸ ವ್ರತವನ್ನು ಕೈಗೊಂಡು ದೇಹವನ್ನು ಬಿಡುತ್ತಾನೆ; ಮುಂದಿನ ಜನ್ಮದಲ್ಲಿ ಒಳ್ಳೆಯ ಜನ್ಮ ಬರುತ್ತದೆ. ಹಾಗಲ್ಲದೆ ಬದುಕಿಕೊಂಡುಬಿಟ್ಟರೆ ಮತ್ತೆ ಮನೆಗೆ ಹಿಂತಿರುಗಿ ಹೋಗದೆ ಬ್ರಹ್ಮಜ್ಞಾನವನ್ನು ಪಡೆಯುವುದಕ್ಕೋಸ್ಕರ ಪ್ರಯತ್ನ ಮಾಡುತ್ತ ಸಂನ್ಯಾಸಿಯಾಗಿ ಕಾಲ ಕಳೆಯಬೇಕು. ನಿಮ್ಮ ಚಿಕ್ಕಪ್ಪ ಶಿವಾನಂದ ಸ್ವಾಮಿಗಳಿಗೆ ಆತುರ ಸಂನ್ಯಾಸವನ್ನು ಕೊಡಲಾಗಿತ್ತು. ಅವನು ಸತ್ತುಹೋದನು. ಆದರೆ ಹೀಗೆ ಸಂನ್ಯಾಸ ತೆಗೆದುಕೊಂಡದ್ದರಿಂದ ಆತನಿಗೆ ಉತ್ತಮವಾದ ಜನ್ಮ ಬರುತ್ತದೆ. ಸಂನ್ಯಾಸ ತೆಗೆದುಕೊಳ್ಳದಿದ್ದರೆ ಆತ್ಮಜ್ಞಾನ ಬರುವುದಕ್ಕೆ ಬೇರೆ ಮಾರ್ಗವಿಲ್ಲ.

ಶಿಷ್ಯ: ಮಹಾಶಯರೆ, ಹಾಗಾದರೆ ಗೃಹಸ್ಥರಿಗೆ ದಾರಿ ಯಾವುದು?

ಸ್ವಾಮೀಜಿ: ಪುಣ್ಯವಶಾತ್ ಒಂದಲ್ಲದಿದ್ದರೆ ಮತ್ತೊಂದು ಜನ್ಮದಲ್ಲಿ ಅವರಿಗೆ ವೈರಾಗ್ಯವುಂಟಾಗುತ್ತದೆ. ವೈರಾಗ್ಯ ಬಂದುಬಿಟ್ಟರೆ ಮುಗಿದುಹೋಯಿತು - ಜನನ - ಮರಣ - ಮಾಯಾಜಾಲದಿಂದ ತಪ್ಪಿಸಿಕೊಂಡು ಹೋಗುವುದಕ್ಕೆ ಆಮೇಲೆ ಹೆಚ್ಚು ಹೊತ್ತಾಗುವುದಿಲ್ಲ. ಆದರೆ ಸಕಲ ನಿಯಮಗಳಿಗೂ ಒಂದೆರಡು ವಿನಾಯಿತಿಯುಂಟು. ಸರಿಯಾದ ಗೃಹಸ್ಥ ಧರ್ಮವನ್ನು ಪಾಲನೆ ಮಾಡಿಕೊಂಡು ಬಂದರೂ ಒಬ್ಬಿಬ್ಬರು ಮುಕ್ತ ಪುರುಷರಾಗುವುದು ಕಂಡುಬಂದಿದೆ - ನಮ್ಮಲ್ಲಿ ನಾಗಮಹಾಶಯರ ಹಾಗೆ.

ಶಿಷ್ಯ: ಮಹಾಶಯರೆ, ವೈರಾಗ್ಯ ಮತ್ತು ಸಂನ್ಯಾಸ ಈ ವಿಷಯವಾಗಿ ಉಪನಿಷತ್ತು ಮುಂತಾದ ಗ್ರಂಥಗಳಲ್ಲಿಯೂ ವಿಸ್ತಾರವಾದ ಉಪದೇಶ ಸಿಕ್ಕುವುದಿಲ್ಲ.

ಸ್ವಾಮಿಜಿ: ಏನು ಹುಚ್ಚನ ಹಾಗೆ ಹೇಳುತ್ತಿ! ವೈರಾಗ್ಯವೇ ಉಪನಿಷತ್ತಿನ ಪ್ರಾಣ. ವಿಚಾರದಿಂದ ಹುಟ್ಟುವ ಪ್ರಜ್ಞೆಯೇ ಉಪನಿಷತ್ - ಜ್ಞಾನದ ಕೊನೆಯ ಉದ್ದೇಶ. ಆದರೆ ನಾನು ನಂಬಿಕೊಂಡಿರುವುದೇನೆಂದರೆ, ಭಗವಾನ್ ಬುದ್ಧದೇವನಿಂದ ಈಚೆಗೆ ಭರತಖಂಡದಲ್ಲಿ ಈ ತ್ಯಾಗವ್ರತ ವಿಶೇಷವಾಗಿ ಪ್ರಚಾರಕ್ಕೆ ಬಂದಿದೆ, ಮತ್ತು ವೈರಾಗ್ಯ ವಿಷಯ ವಿತೃಷ್ಣೆಗಳೆ ಧರ್ಮದ ಕೊನೆಯ ಉದ್ದೇಶವೆಂದೂ ವಿವೇಚನೆ ಮಾಡಲ್ಪಟ್ಟಿದೆ. ಬೌದ್ಧ ಧರ್ಮದ ಆ ತ್ಯಾಗ ವೈರಾಗ್ಯಗಳನ್ನು ಹಿಂದೂ ಧರ್ಮ ಹೀರಿಕೊಂಡುಬಿಟ್ಟಿದೆ. ಭಗವಾನ್ ಬುದ್ಧನಂಥ ತ್ಯಾಗಿ ಮಹಾಪುರುಷ ಪ್ರಪಂಚದಲ್ಲಿ ಮತ್ತೊಬ್ಬ ಹುಟ್ಟಿಲ್ಲ.

ಶಿಷ್ಯ: ಮಹಾಶಯರೆ, ಹಾಗಾದರೆ ಬುದ್ಧದೇವನು ಹುಟ್ಟುವುದಕ್ಕೆ ಮುಂಚೆ ದೇಶದಲ್ಲಿ ತ್ಯಾಗ ವೈರಾಗ್ಯಗಳು ಎಲ್ಲೋ ಸ್ವಲ್ಪವಿದ್ದುವೇನು? ಮತ್ತು ದೇಶದಲ್ಲಿ ಸಂನ್ಯಾಸಿಗಳೇ ಇರಲಿಲ್ಲವೆ?

ಸ್ವಾಮೀಜಿ: ಹಾಗೆಂದು ಹೇಳಿದವರು ಯಾರು? ಆದರೆ ಅದೇ ಜೀವನದ ಚರಮಲಕ್ಷಣವೆಂಬುದು ಸಾಧಾರಣ ಜನರಿಗೆ ಗೊತ್ತಿರಲಿಲ್ಲ; ವೈರಾಗ್ಯದಲ್ಲಿ ದಾಢ್ಯವಿರಲಿಲ್ಲ; ವಿವೇಕದಲ್ಲಿ ನಿಷ್ಠೆಯಿರಲಿಲ್ಲ. ಆದ್ದರಿಂದ ಬುದ್ಧ ದೇವನು ಎಷ್ಟೋ ಯೋಗಿಗಳ, ಎಷ್ಟೋ ಸಾಧುಗಳ ಹತ್ತಿರ ಹೋದರೂ ಶಾಂತಿಯನ್ನು ಪಡೆಯಲಿಲ್ಲ. ಆಮೇಲೆ ‘ಇಹಾಸನೇ ಶುಷ್ಯತು ಮೇ ಶರೀರಂ’ - ಈ ಆಸನದಲ್ಲಿ ಕುಳಿತೇ ನನ್ನ ದೇಹ ಒಣಗಿ ಹೋಗಲಿ ಎಂದು ಹೇಳಿ ಆತ್ಮಜ್ಞಾನವನ್ನು ಪಡೆಯುವುದಕ್ಕೋಸ್ಕರ ತಾನೇ ಕುಳಿತುಕೊಂಡುಬಿಟ್ಟು ಪ್ರಬುದ್ಧನಾಗಿ ಎದ್ದು ಬಂದನು. ಭರತಖಂಡದಲ್ಲಿ ಸಂನ್ಯಾಸಿಗಳ ಮಠಗಿಠಗಳನ್ನು ಏನೇನು ನೋಡಿದ್ದೀಯೋ ಅವೆಲ್ಲ ಬೌದ್ಧಧರ್ಮದ ಅಧೀನದಲ್ಲಿದ್ದುವು. ಹಿಂದೂಗಳು ಅವುಗಳಿಗೆಲ್ಲ ಈಗ ತಮ್ಮ ಬಣ್ಣವನ್ನು ಹಾಕಿ ಅವುಗಳನ್ನು ತಮ್ಮ ಸ್ವತ್ತನ್ನಾಗಿ ಮಾಡಿಟ್ಟುಕೊಂಡಿದ್ದಾರೆ. ಭಗವಾನ್ ಬುದ್ಧದೇವನಿಂದಲೇ ಯಥಾರ್ಥವಾದ ಸಂನ್ಯಾಸಾಶ್ರಮದ ಸೂತ್ರವು ಬಂದಿದ್ದು. ಅವನೇ ಸಂನ್ಯಾಸಾಶ್ರಮದ ನಿರ್ಜೀವವಾದ ಅಸ್ಥಿಪಂಜರದಲ್ಲಿ ಪ್ರಾಣವನ್ನು ತುಂಬಿಟ್ಟು ಹೋದವನು.

ಸ್ವಾಮೀಜಿಯ ಗುರುಭ್ರಾತೃವಾದ ರಾಮಕೃಷ್ಣಾನಂದ ಸ್ವಾಮಿಗಳು “ಬುದ್ಧದೇವನು ಹುಟ್ಟುವುದಕ್ಕೆ ಮುಂಚೆಯೂ ಭರತಖಂಡದಲ್ಲಿ ನಾಲ್ಕು ಆಶ್ರಮಗಳು ಇದ್ದುವೆಂಬುದಕ್ಕೆ ಸಂಹಿತಾ ಪುರಾಣಾದಿಗಳು ಪ್ರಮಾಣವಾಗಿವೆ" ಎಂದರು. ಅದಕ್ಕೆ ಸ್ವಾಮಿಜಿ, “ಮನು ಮುಂತಾದವರ ಸಂಹಿತೆಗಳು, ಪುರಾಣಗಳ ಬಹುಭಾಗ, ಮತ್ತು ಮಹಾಭಾರತದ ಅನೇಕ ವಿಷಯಗಳು ಇವೆಲ್ಲಾ ಈಚಿನ ಕಾಲದವು. ಆದರೆ ಭಗವಾನ್ ಬುದ್ಧದೇವನು ಅವುಗಳಿಗಿಂತ ಬಹು ಹಿಂದಿನವನು" ಎಂದು ಉತ್ತರ ಕೊಟ್ಟರು. ಸ್ವಾಮಿ ರಾಮಕೃಷ್ಣಾನಂದರು ಹೇಳಿದ್ದೇನೆಂದರೆ, “ಹಾಗಿದ್ದರೆ ವೇದ ಉಪನಿಷತ್ತು ಸಂಹಿತೆ ಪುರಾಣ ಇವುಗಳಲ್ಲಿ ಬೌದ್ಧ ಧರ್ಮದ ಸಮಾಲೋಚನೆಯು ಖಂಡಿತವಾಗಿಯೂ ಇರುತ್ತಿತ್ತು. ಆದರೆ ಈ ಪ್ರಾಚೀನ ಗ್ರಂಥಗಳಲ್ಲಿ ಬೌದ್ಧ ಧರ್ಮದ ಸಮಾಲೋಚನೆ ಕಂಡುಬರದಿರುವಾಗ ಬುದ್ಧ ದೇವನು ಇದಕ್ಕಿಂತ ಮುಂಚೆಯಿದ್ದವನೆಂದು ಹೇಗೆ ಹೇಳಿದೆ? ಎರಡು ಮೂರು ಪ್ರಾಚೀನ ಪುರಾಣಾದಿಗಳಲ್ಲಿ ಬೌದ್ಧ ಮತದ ಸೂಕ್ಷ್ಮ ವರ್ಣನೆಯಿದೆ - ಅದನ್ನು ನೋಡಿ ಅಷ್ಟರಿಂದಲೇ ಹಿಂದೂಗಳ ಸಂಹಿತಾ ಪುರಾಣಾದಿಗಳು ಆಧುನಿಕ ಶಾಸ್ತ್ರಗಳೆಂದು ಹೇಳಲಾಗುವುದಿಲ್ಲ."

ಸ್ವಾಮಿಜಿ: ಚರಿತ್ರೆಯನ್ನು ಓದಿ ನೋಡು. ಹಿಂದೂಧರ್ಮ ಬುದ್ಧದೇವನ ಭಾವಗಳನ್ನೆಲ್ಲ ತನ್ನಲ್ಲಿ ಸೇರಿಸಿಕೊಂಡುಬಿಟ್ಟು, ಇಷ್ಟು ದೊಡ್ಡದಾಗಿದೆ ಎಂದು ಗೊತ್ತಾಗುತ್ತದೆ.

ರಾಮಕೃಷ್ಣಾನಂದ: ತ್ಯಾಗ ವೈರಾಗ್ಯ ಮುಂತಾದುವುಗಳನ್ನು ಜೀವಮಾನದಲ್ಲಿ ಯಥಾರ್ಥವಾಗಿ ಅನುಷ್ಠಾನಮಾಡಿ ಬುದ್ಧದೇವನು ಹಿಂದೂಧರ್ಮದ ಭಾವಗಳನ್ನು ಸಜೀವಗಳಾಗಿ ಮಾಡಿಹೋದನು ಅಷ್ಟೆ ಎಂದು ನನಗೆ ತೋರುತ್ತದೆ.

ಸ್ವಾಮೀಜಿ: ಆದರೆ ಇದಕ್ಕೆ ಪ್ರಮಾಣ ಸಿಕ್ಕುವುದಿಲ್ಲ. ಏಕೆಂದರೆ, ಬುದ್ಧದೇವನು ಹುಟ್ಟುವುದಕ್ಕಿಂತ ಮುಂಚಿನ ಕಾಲದ ಚರಿತ್ರೆ ಯಾವುದೂ ಸಿಕ್ಕುವುದಿಲ್ಲ. ಚರಿತ್ರೆಯನ್ನು ಪ್ರಮಾಣವೆಂದು ಒಪ್ಪಿಕೊಂಡರೆ, ಪುರಾಣ ಕಾಲದ ಘೋರಾಂಧಕಾರದಲ್ಲಿ ಭಗವಾನ್ ಬುದ್ಧದೇವನೊಬ್ಬನೆ ಜ್ಞಾನಾಲೋಕದಿಂದ ಪ್ರದೀಪ್ತನಾಗಿದ್ದವನೆಂದು ಒಪ್ಪಿಕೊಳ್ಳ ಬೇಕಾಗುತ್ತದೆ.

ಈಗ ಮತ್ತೆ ಸಂನ್ಯಾಸಧರ್ಮದ ಪ್ರಸ್ತಾಪ ಬಂತು. ಸ್ವಾಮಿಜಿ, “ಸಂನ್ಯಾಸದ ಮೂಲ ಯಾವುದಾದರೂ ಆಗಲಿ, ಮಾನವ ಜನ್ಮದ ಉದ್ದೇಶ ಏನೆಂದರೆ ಈ ತ್ಯಾಗ ವ್ರತವನ್ನವಲಂಬಿಸಿಕೊಂಡು ಬ್ರಹ್ಮಜ್ಞನಾಗುವುದು. ಸಂನ್ಯಾಸಗ್ರಹಣವೆ ಪರಮ ಪುರುಷಾರ್ಥ. ಯಾರು ವೈರಾಗ್ಯವನ್ನು ಪಡೆದು ಸಂಸಾರದಲ್ಲಿ ವಿರಕ್ತರಾಗುತ್ತಾರೆಯೋ ಅವರೇ ಧನ್ಯರು" ಎಂದರು.

ಶಿಷ್ಯ: ಮಹಾಶಯರೆ, ಪರಿವ್ರಾಜಕ ಸಂನ್ಯಾಸಿಗಳ ಸಂಖ್ಯೆ ಹೆಚ್ಚಿಹೋಗಿರುವುದರಿಂದ ವ್ಯಾವಹಾರಿಕ ವಿಚಾರಗಳಲ್ಲಿ ದೇಶಕ್ಕೆ ಇದ್ದ ಉನ್ನತಿಯು ಕ್ಷೀಣವಾಗಿದೆ ಎಂದು ಈಗ ಅನೇಕರು ಹೇಳುತ್ತಿದ್ದಾರೆ. ಗೃಹಸ್ಥರ ಸಹಾಯವನ್ನು ಅವಲಂಬಿಸಿಕೊಂಡು ಸಾಧುಸಂನ್ಯಾಸಿಗಳು ಕೆಲಸಮಾಡದೆ ತಿರುಗುತ್ತಿದ್ದಾರೆ. ‘ಇದರಿಂದ ಸಮಾಜ ಮತ್ತು ಸ್ವದೇಶಗಳಿಗೆ ಯಾವ ವಿಧದಲ್ಲಿಯೂ ಸಹಾಯವಾಗುವುದಿಲ್ಲ’ ಎಂದು ಅವರು ಹೇಳುತ್ತಾರೆ.

ಸ್ವಾಮೀಜಿ: ಲೌಕಿಕ ಅಥವಾ ವ್ಯಾವಹಾರಿಕ ಉನ್ನತಿ ಎಂಬುದರ ಅರ್ಥವೇನು? ಅದನ್ನು ಮೊದಲು ನನಗೆ ವಿವರಿಸು.

ಶಿಷ್ಯ: ಪಾಶ್ಚಾತ್ಯರು ಹೇಗೆ ವಿದ್ಯೆಯ ಸಹಾಯದಿಂದ ದೇಶದಲ್ಲಿ ಅನ್ನ ವಸ್ತ್ರಗಳಿಗೆ ಏರ್ಪಾಡು ಮಾಡಿಕೊಂಡಿದ್ದಾರೆಯೊ, ವಿಜ್ಞಾನದ ಸಹಾಯದಿಂದ ದೇಶದಲ್ಲಿ ವ್ಯಾಪಾರ ಶಿಲ್ಪ ಅನ್ನ ವಸ್ತ್ರ ರೈಲು ತಂತಿ ಮುಂತಾದ ವಿಷಯಗಳಲ್ಲಿ ಮೇಲ್ಮೆಯನ್ನು ಪಡೆದಿದ್ದಾರೆಯೊ ಹಾಗೆ ಮಾಡುವುದು.

ಸ್ವಾಮೀಜಿ: ಮನುಷ್ಯನಲ್ಲಿ ರಜೋಗುಣವು ಹೆಚ್ಚಾಗಿರದಿದ್ದರೆ ಇದೆಲ್ಲಾ ಆಗುತ್ತದೆಯೇನು? ನಾನು ಭರತಖಂಡವನ್ನು ಸುತ್ತಿ ನೋಡಿಕೊಂಡು ಬಂದಿದ್ದೇನೆ. ಎಲ್ಲಿಯೂ ರಜೋಗುಣದ ವಿಕಾಸವಿಲ್ಲ! ಬರಿಯ ತಮಸ್ಸು. ಘೋರ ತಮೋಗುಣದಲ್ಲಿ ಸಾಧಾರಣ ಜನರೆಲ್ಲಾ ಬಿದ್ದಿದ್ದಾರೆ. ಕೇವಲ ಸಂನ್ಯಾಸಿಗಳಲ್ಲಿ ಮಾತ್ರ ರಜೋಗುಣವೂ ಸತ್ತ್ವಗುಣವೂ ಇರುವುದನ್ನು ನೋಡಿದ್ದೇನೆ - ಇವರೇ ಭರತಖಂಡದ ಮೇರುದಂಡ. ನಿಜವಾದ ಸಂನ್ಯಾಸಿಗಳು ಗೃಹಸ್ಥರಿಗೆ ಉಪದೇಶಕರು. ಅವರ ಉಪದೇಶವನ್ನೂ ಮತ್ತು ಜ್ಞಾನವನ್ನೂ ಪಡೆದೇ ಹಿಂದೆ ಅನೇಕ ವೇಳೆ ಸಂಸಾರಿಗಳು ಜೀವನ ಸಂಗ್ರಾಮದಲ್ಲಿ ಕೃತಕೃತ್ಯರಾಗಿದ್ದಾರೆ. ಸಂನ್ಯಾಸಿಗಳ ಅಮೂಲ್ಯವಾದ ಉಪದೇಶಕ್ಕೆ ಬದಲಾಗಿ ಗೃಹಸ್ಥರು ಅವರಿಗೆ ಅನ್ನ ವಸ್ತ್ರಗಳನ್ನು ಕೊಡುವರು. ಈ ಕೊಟ್ಟು ತೆಗೆದುಕೊಂಡು ಮಾಡುವುದಿಲ್ಲದಿದ್ದರೆ ಭರತಖಂಡದ ಜನರು ಇಷ್ಟು ಹೊತ್ತಿಗೆ ಅಮೆರಿಕಾದ ಆದಿವಾಸಿಗಳ ಹಾಗೆ ನಿರ್ನಾಮವಾಗಿ ಹೋಗುತ್ತಿದ್ದರು. ಸಂನ್ಯಾಸಿಗಳಿಗೆ ಅನ್ನ ಬಟ್ಟೆಗಳನ್ನು ಕೊಡುತ್ತಿರುವುದರಿಂದಲೇ ಗೃಹಸ್ಥರು ಈಗಲೂ ಉನ್ನತಿಗೆ ಸಾಧಕವಾದ ಮಾರ್ಗದಲ್ಲಿ ಹೋಗುತ್ತಿದ್ದಾರೆ. ಸಂನ್ಯಾಸಿಗಳು ಕರ್ಮಹೀನರಲ್ಲ. ಅವರೇ ಕರ್ಮದ ಜನ್ಮಸ್ಥಾನ. ಉಚ್ಚಾದರ್ಶಗಳನ್ನು ಅವರು ಜೀವನದಲ್ಲಿ ಕಾರ್ಯರೂಪಕ್ಕೆ ತಿರುಗಿಸಿ ನಡೆಸಿಕೊಂಡು ಬರುವುದನ್ನು ನೋಡಿಯೂ ಮತ್ತು ಅವರಿಂದ ಈ ಅಭಿಪ್ರಾಯಗಳನ್ನು ತೆಗೆದುಕೊಂಡ ಸಂಸಾರಿಗರು ಕರ್ಮಕ್ಷೇತ್ರದಲ್ಲಿ, ಜೀವನ ಸಂಗ್ರಾಮದಲ್ಲಿ ಸಮರ್ಥರಾಗಿದ್ದಾರೆ ಮತ್ತು ಆಗುತ್ತಾರೆ. ಪವಿತ್ರರಾದ ಸಂನ್ಯಾಸಿಗಳನ್ನು ನೋಡಿಯೇ ಗೃಹಸ್ಥರು ಪವಿತ್ರಭಾವಗಳನ್ನೆಲ್ಲಾ ಜೀವನದಲ್ಲಿ ರೂಢಿಸಿಕೊಂಡು ಬಂದು ನಿಜವಾದ ರೀತಿಯಲ್ಲಿ ಕರ್ಮತತ್ಪರರಾಗುತ್ತಾರೆ. ಸಂನ್ಯಾಸಿಗಳು ತಮ್ಮ ಜೀವನದಲ್ಲಿ ಈಶ್ವರ ಸಾಕ್ಷಾತ್ಕಾರಕ್ಕಾಗಿಯೂ ಲೋಕಕಲ್ಯಾಣಕ್ಕಾಗಿಯೂ ಸರ್ವಸ್ವವನ್ನೂ ತ್ಯಾಗಮಾಡಬೇಕೆಂಬ ತತ್ತ್ವವನ್ನು ತೋರಿಸಿ ಸಂಸಾರಿಗಳಿಗೆ ಎಲ್ಲಾ ಸಂಗತಿಗಳಲ್ಲಿಯೂ ಪ್ರೋತ್ಸಾಹ ಕೊಡುತ್ತಾರೆ. ಅದಕ್ಕೆ ಪ್ರತಿಫಲವಾಗಿ ಇವರು ಅವರಿಗೆ ಅನ್ನ ವಸ್ತ್ರಗಳನ್ನು ಕೊಡುತ್ತಾರೆ. ಈ ಅನ್ನವು ಬರುವುದಕ್ಕೆ ಬೇಕಾದ ಪ್ರವೃತ್ತಿಯೂ ಸಾಮರ್ಥ್ಯವೂ ಸರ್ವತ್ಯಾಗಿಗಳಾದ ಸಂನ್ಯಾಸಿಗಳ ಸ್ನೇಹಾಶೀರ್ವಾದಗಳಿಂದಲೇ ಪ್ರಪಂಚದಲ್ಲಿ ಹೆಚ್ಚುತ್ತದೆ. ಜನರು ಅರಿಯದೆ ಸಂನ್ಯಾಸ ಆಶ್ರಮವನ್ನು ನಿಂದಿಸುತ್ತಾರೆ. ಇತರೆ ದೇಶಗಳಲ್ಲಿ ಏನಾದರೂ ಆಗಿರಲಿ, ಈ ದೇಶದಲ್ಲಿ ಸಂನ್ಯಾಸಿಗಳು ಜನಜೀವನವೆಂಬ ದೋಣಿಯನ್ನು ನಡೆಸುತ್ತಿರುವುದರಿಂದ ಸಂಸಾರ ಸಾಗರದಲ್ಲಿ ಗೃಹಸ್ಥರ ಜೀವಮಾನವೆಂಬ ದೋಣಿಯು ಮುಳುಗುತ್ತಿಲ್ಲ.

ಶಿಷ್ಯ: ಮಹಾಶಯರೆ, ಲೋಕಕಲ್ಯಾಣತತ್ಪರರಾದ ಯಥಾರ್ಥ ಸಂನ್ಯಾಸಿಗಳನ್ನು ನೋಡಬೇಕೆಂದರೆ ಎಷ್ಟು ಜನ ಸಿಕ್ಕಿಯಾರು?

ಸ್ವಾಮೀಜಿ: ಸಾವಿರ ವರ್ಷಕಾಲದಲ್ಲಿ ಪರಮಹಂಸರಂಥ ಒಬ್ಬ ಸಂನ್ಯಾಸಿ ಮಹಾಪುರುಷನು ಬಂದರೆ ಬೇಕಾದಷ್ಟು ಆಗಿಹೋಯಿತು. ಆತನು ಯಾವ ಉಚ್ಚಾದರ್ಶ ಮತ್ತು ಭಾವಗಳನ್ನು ಕೊಟ್ಟು ಹೋಗುತ್ತಾನೆಯೋ, ಅವುಗಳನ್ನು ಜನರು ಆತನ ಕಾಲಾನಂತರವೂ ಸಹಸ್ರ ವರ್ಷಗಳವರೆಗೂ ತೆಗೆದುಕೊಂಡು ಹೋಗುತ್ತಾರೆ. ಈ ಸಂನ್ಯಾಸ ಆಶ್ರಮವೂ ದೇಶದಲ್ಲಿ ಇದ್ದದ್ದರಿಂದಲೆ ಅವರಂಥ ಮಹಾಪುರುಷರು ಈ ದೇಶದಲ್ಲಿ ಹುಟ್ಟಿದ್ದಾರೆ. ಸ್ವಲ್ಪ ಹೆಚ್ಚು ಕಡಿಮೆ ದೋಷವೆಂಬುದು ಎಲ್ಲಾ ಆಶ್ರಮದಲ್ಲೂ ಉಂಟು - ಎಷ್ಟೇ ದೋಷವಿದ್ದರೂ ಇಷ್ಟು ದಿನದವರೆಗೆ ಈ ಆಶ್ರಮ ಮಿಕ್ಕ ಆಶ್ರಮಗಳ ಶಿರಸ್ಥಾನದಲ್ಲಿ ಅಧಿಕಾರ ಮಾಡುತ್ತ ನಿಂತುಕೊಂಡಿದೆಯಲ್ಲಾ ಅದಕ್ಕೆ ಕಾರಣವೇನು? ನಿಜವಾದ ಸಂನ್ಯಾಸಿಗಳು ತಮ್ಮ ಮೋಕ್ಷವನ್ನೂ ಅಲಕ್ಷ್ಯಮಾಡಿಬಿಟ್ಟರು. ಜಗತ್ತಿನ ಕಲ್ಯಾಣವನ್ನುಂಟು ಮಾಡುವುದಕ್ಕೆ ಅವರು ಹುಟ್ಟಿದ್ದು. ಇಂಥ ಸಂನ್ಯಾಸಾಶ್ರಮದ ವಿಚಾರದಲ್ಲಿ ನೀವು ಕೃತಜ್ಞರಾಗಿರದಿದ್ದರೆ ನಿಮಗೆ ಧಿಕ್ – ನೂರು ಬಾರಿ ಧಿಕ್!

ಹೀಗೆಂದು ಹೇಳುತ್ತ ಹೇಳುತ್ತ ಸ್ವಾಮಿಗಳ ಮುಖಮಂಡಲವು ಪ್ರದೀಪ್ತವಾಗಿಬಿಟ್ಟಿತು. ಸಂನ್ಯಾಸಾಶ್ರಮದ ಮಹಿಮೆಯ ಪ್ರಸ್ತಾವದಲ್ಲಿ ಸ್ವಾಮಿಗಳು ಸಾಕಾರವಾದ ಸಂನ್ಯಾಸದಂತೆ ಶಿಷ್ಯನಿಗೆ ಕಂಡುಬಂದಿತು.

ಅನಂತರ ಈ ಆಶ್ರಮದ ಗೌರವವನ್ನು ಅವರು ತಮ್ಮ ಹೃದಯದಲ್ಲಿ ಅನುಭವ ಮಾಡುತ್ತ ಮಾಡುತ್ತ ಅಂತರ್ಮುಖಿಗಳಾದಂತಾಗಿ ತಮ್ಮಷ್ಟಕ್ಕೆ ಮಧುರ ಸ್ವರದಲ್ಲಿ

\begin{verse}
ವೇದಾಂತವಾಕ್ಯೇಷು ಸದಾ ರಮಂತಃ,\\ಭಿಕ್ಷಾನ್ನ ಮಾತ್ರೇಣ ಚ ತುಷ್ಟಿಮಂತಃ~।\\ಅಶೋಕಮಂತಃ ಕರಣೇ ಚರಂತಃ\\ಕೌಪೀನವಂತಃ ಖಲು ಭಾಗ್ಯವಂತಃ~॥
\end{verse}

ಎಂಬ ಶ್ಲೋಕವನ್ನು ಹೇಳಿದರು.

ಆಮೇಲೆ ಪುನಃ ಹೇಳಿದ್ದೇನೆಂದರೆ: ಬಹುಜನರ ಹಿತಕ್ಕೋಸ್ಕರ ಬಹು ಜನರ ಸುಖಕ್ಕೋಸ್ಕರ ಸಂನ್ಯಾಸಿಯ ಜನ್ಮ. ಸಂನ್ಯಾಸವನ್ನು ಪಡೆದು ಯಾರು ಈ ಉಚ್ಚಲಕ್ಷ್ಯವನ್ನು ಮರೆತುಬಿಡುತ್ತಾರೆಯೊ - ‘ವೃಥೈವ ತಸ್ಯ ಜೀವನಂ,’ ಅವನ ಬಾಳು ವ್ಯರ್ಥವೇ ಸರಿ. ಪರರಿಗೋಸ್ಕರ ಪ್ರಾಣ ಕೊಡುವುದಕ್ಕೆ, ಆಕಾಶವನ್ನು ಮುಟ್ಟುತ್ತಿರುವ ಪ್ರಾಣಿಗಳ ಗೋಳಾಟವನ್ನು ಹೋಗಲಾಡಿಸುವುದಕ್ಕೆ, ವಿಧವೆಯರ ಕಣ್ಣೀರನ್ನು ತೊಡೆಯುವುದಕ್ಕೆ, ಪುತ್ರವಿಯೋಗದಿಂದ ಸಂಕಟ ಪಡುತ್ತಿರುವವರ ಹೃದಯದಲ್ಲಿ ಶಾಂತಿಯನ್ನು ಉಂಟುಮಾಡುವುದಕ್ಕೆ, ಅಜ್ಞರಾದ ಸಾಧಾರಣ ಜನರಿಗೆ ಜೀವನ ಸಂಗ್ರಾಮದಲ್ಲಿ ದಕ್ಷತೆಯನ್ನುಂಟುಮಾಡುವುದಕ್ಕೆ, ಶಾಸ್ತ್ರೋಪದೇಶವನ್ನು ಹರಡುವುದರ ಮೂಲಕ ಎಲ್ಲರಿಗೂ ಐಹಿಕ ಮತ್ತು ಪಾರಮಾರ್ಥಿಕ ಮಂಗಳವನ್ನುಂಟುಮಾಡುವುದಕ್ಕೆ ಮತ್ತು ಜ್ಞಾನಾಲೋಕವನ್ನು ಬೆಳಗಿ ಎಲ್ಲರಲ್ಲಿಯೂ ನಿದ್ರಿಸುತ್ತಿರುವ ಬ್ರಹ್ಮಸಿಂಹವನ್ನು ಎಬ್ಬಿಸುವುದಕ್ಕೆ ಜಗತ್ತಿನಲ್ಲಿ ಸಂನ್ಯಾಸಿಯ ಜನ್ಮವಿರುವುದು. ಅನಂತರ ತಮ್ಮ ಭ್ರಾತೃಗಳ ಕಡೆಗೆ ಲಕ್ಷ್ಯವಿಟ್ಟು ಹೇಳಿದ್ದೇನೆಂದರೆ, “ಆತ್ಮನೋ ಮೋಕ್ಷಾರ್ಥಂ ಜಗದ್ಧಿತಾಯ ಚ - ನಾನು ಹುಟ್ಟಿರುವುದು. ಏನು ಮಾಡುತ್ತೀರಿ ಎಲ್ಲರೂ ಸುಮ್ಮನೆ ಕುಳಿತು? ಏಳಿ ಎಚ್ಚರಗೊಳ್ಳಿ, ನೀವು ಎಚ್ಚೆತ್ತುಕೊಂಡು ಎಲ್ಲರನ್ನೂ ಎಚ್ಚರಗೊಳಿಸಿ, ನರಜನ್ಮವನ್ನು ಸಾರ್ಥಕಗೊಳಿಸಿ ಹೋಗಿ; ಉತ್ತಿಷ್ಠತ - ಜಾಗ್ರತ - ಪ್ರಾಪ್ಯವರಾನ್ ನಿಬೋಧತ.”

\newpage

\chapter[ಅಧ್ಯಾಯ ೧೨]{ಅಧ್ಯಾಯ ೧೨\protect\footnote{\engfoot{C.W, Vol. VI, P. 513}}}

\begin{center}
ಸ್ಥಳ: ಕಲ್ಕತ್ತ, ಬಲರಾಮಬಾಬುಗಳ ಮನೆ, ವರ್ಷ: ಕ್ರಿ.ಶ. ೧೮೯೮.
\end{center}

ಸ್ವಾಮೀಜಿ ಈಗ ಎರಡು ದಿನವೆಲ್ಲಾ ಬಾಗಬಜಾರಿನಲ್ಲಿ ಬಲರಾಮ ಬಸುಗಳ ಮನೆಯಲ್ಲಿ ಇದ್ದರು. ಆದ್ದರಿಂದ ಅಲ್ಲಿಗೆ ಪ್ರತಿನಿತ್ಯವೂ ಹೋಗಿಬರುವುದಕ್ಕೆ ಶಿಷ್ಯನಿಗೆ ತುಂಬಾ ಅನುಕೂಲವಾಗಿತ್ತು. ಇಂದು ಸಾಯಂಕಾಲಕ್ಕೆ ಸ್ವಲ್ಪ ಮುಂಚೆ ಸ್ವಾಮೀಜಿ ಈ ಮನೆಯ ಮಹಡಿಯ ಮೇಲೆ ತಿರುಗಾಡುತ್ತಿದ್ದಾರೆ. ಶಿಷ್ಯನೂ, ಬೇರೆ ನಾಲ್ಕೈದು ಜನರೂ ಜೊತೆಯಲ್ಲಿದ್ದಾರೆ. ಬಹು ಸೆಕೆ. ಆದ್ದರಿಂದ ಸ್ವಾಮೀಜಿ ಮೈಯನ್ನು ಬಿಟ್ಟುಕೊಂಡಿದ್ದಾರೆ. ಮೆಲ್ಲ ಮೆಲ್ಲಗೆ ದಕ್ಷಿಣದ ಕಡೆಯಿಂದ ಗಾಳಿ ಬರುತ್ತಿದೆ. ತಿರುಗಾಡುತ್ತ ತಿರುಗಾಡುತ್ತ ಸ್ವಾಮೀಜಿ ಗುರುಗೋವಿಂದರ ಪ್ರಸ್ತಾಪವನ್ನು ತೆಗೆದು ಅವರ ತ್ಯಾಗ ತಪಸ್ಸು ಕ್ಷಮೆ ಮತ್ತು ಜೀವವನ್ನೂ ತೆತ್ತು ಪಟ್ಟ ಶ್ರಮ, ಇವುಗಳ ಫಲವಾಗಿ ಸಿಕ್ಕರ ಜಾತಿಗೆ ಹೇಗೆ ಪುನಃ ಅಭ್ಯುತ್ಥಾನ ಉಂಟಾಯಿತು, ಹೇಗೆ ಅವರು ಮೊದಲು ಮುಸಲ್ಮಾನ ಧರ್ಮವನ್ನವಲಂಬಿಸಿದ್ದ ಜನರಿಗೂ ದೀಕ್ಷೆಯನ್ನು ಕೊಟ್ಟು ಮತ್ತೆ ಅವರನ್ನು ಹಿಂದೂಗಳನ್ನಾಗಿ ಮಾಡಿ ಸಿಕ್ಕರ ಜಾತಿಗೆ ಸೇರಿಸಿಕೊಂಡರು ಮತ್ತು ಹೇಗೆ ನರ್ಮದಾ ತೀರದಲ್ಲಿ ಮಾನವ ದೇಹವನ್ನು ಬಿಟ್ಟರು - ಈ ವಿಷಯಗಳನ್ನೆಲ್ಲಾ ಓಜಸ್ವಿಯಾದ ಭಾಷೆಯಲ್ಲಿ ಸ್ವಲ್ಪ ಸ್ವಲ್ಪ ವರ್ಣಿಸತೊಡಗಿದರು. ಗುರು ಗೋವಿಂದರ ಹತ್ತಿರ ದೀಕ್ಷೆಯನ್ನು ಪಡೆದ ಜನರಲ್ಲಿ ಆಗ ಎಂಥ ಮಹಾಶಕ್ತಿ ಸಂಚರಿಸುವುದಕ್ಕೆ ಮೊದಲಾಯಿತೊ, ಅದನ್ನು ವಿವರಿಸಿ, ಸ್ವಾಮಿಜಿ ಸಿಕ್ಕರ ಜಾತಿಯವರಲ್ಲಿ ರೂಢಿಗೆ ಬಂದಿರುವ ಒಂದು ದೋಹಾ(ಶ್ಲೋಕ)ವನ್ನು ಹೇಳಿದರು -

\begin{verse}
“ಸವೈ ಯಾ ಲಾಖ್ ಪರ ಏಕ ಚಡಾಉ\\ಜಬ್ ಗುರು ಗೋವಿಂದ ನಾಮ ಸುನಾಉ.”
\end{verse}

ಎಂದರೆ, ಗುರುಗೋವಿಂದರ ಹತ್ತಿರ ದೀಕ್ಷೆಯನ್ನು ಪಡೆದು ಒಬ್ಬೊಬ್ಬರಲ್ಲಿಯೂ ಒಂದೂಕಾಲು ಲಕ್ಷ ಜನರಲ್ಲಿರುವುದಕ್ಕಿಂತಲೂ ಹೆಚ್ಚು ಶಕ್ತಿ ಸಂಚರಿಸುತ್ತಿತ್ತು. ಎಂದರೆ ಆತನ ಹತ್ತಿರ ದೀಕ್ಷೆಯನ್ನು ಪಡೆದರೆ ಆತನ ಶಕ್ತಿಯಿಂದ ಜೀವನದಲ್ಲಿ ನಿಜವಾದ ಧರ್ಮಪ್ರಾಣವು ಬಂದು ಗುರುಗೋವಿಂದರ ಪ್ರತಿಯೊಬ್ಬ ಶಿಷ್ಯನೊಳಗೂ ಎಂಥ ವೀರತ್ವ ತುಂಬಿಕೊಳ್ಳುತ್ತಿತ್ತೆಂದರೆ, ಅವನು ಆಗ ಒಂದೂಕಾಲು ಲಕ್ಷ ವಿಧರ್ಮಿಗಳನ್ನು ಸೋಲಿಸಲು ಸಮರ್ಥನಾಗುತ್ತಿದ್ದನು. ಧರ್ಮ ಮಹಿಮೆಯನ್ನು ತಿಳಿಸುವ ಈ ಮಾತುಗಳನ್ನು ಹೇಳುತ್ತ ಹೇಳುತ್ತ ಉತ್ಸಾಹದಿಂದ ಅರಳಿದ ಸ್ವಾಮಿಜಿಯವರ ಕಣ್ಣಿನಲ್ಲಿ ತೇಜಸ್ಸು ಒಡೆದುಕೊಂಡು ಹೊರಗೆ ಬರುತ್ತಿದ್ದಂತೆ ಕಂಡಿತು. ಕೇಳುತ್ತಿದ್ದವರು ಬೆರಗಾಗಿ ಸ್ವಾಮೀಜಿ ಮುಖದ ಕಡೆಗೆ ದೃಷ್ಟಿ ಕೊಟ್ಟು ಅದನ್ನೇ ನೋಡುತ್ತಿದ್ದರು. ಸ್ವಾಮಿಜಿಯಲ್ಲಿ ಎಂಥ ಅದ್ಭುತವಾದ ಉತ್ಸಾಹವೂ ಶಕ್ತಿಯೂ ಇತ್ತೆಂದರೆ, ಅವರು ಯಾವಾಗ ಯಾವ ವಿಷಯವನ್ನು ಕುರಿತು ಹೇಳುತ್ತಿದ್ದರೋ ಆವಾಗ ಅವರು ಅದರಲ್ಲಿ ಅಷ್ಟು ತನ್ಮಯರಾಗಿ ಹೋಗಿಬಿಡುತ್ತಿದ್ದರು; ನಮ್ಮ ಮನಸ್ಸಿಗೆ, ಈ ವಿಷಯವನ್ನೇ ಅವರು ಪ್ರಪಂಚದಲ್ಲಿ ಎಲ್ಲಕ್ಕಿಂತಲೂ ಶ್ರೇಷ್ಠವೆಂಬುದಾಗಿ ತಿಳಿದುಕೊಂಡಿದ್ದರೆಂದೂ, ಅದನ್ನು ಪಡೆಯುವುದೇ ಮನುಷ್ಯ ಜೀವನದ ಏಕಮಾತ್ರ ಉದ್ದೇಶವೆಂಬುದಾಗಿ ನಿರ್ಧರಿಸಿಕೊಂಡಿದ್ದರೆಂದೂ ತೋರುತ್ತಿತ್ತು.

ಸ್ವಲ್ಪ ಹೊತ್ತು ಹೋದಮೇಲೆ ಶಿಷ್ಯನು “ಮಹಾಶಯರೆ, ಇದರಲ್ಲಿ ಅದ್ಭುತವಾದ ವಿಷಯವೇನೆಂದರೆ, ಗುರುಗೋವಿಂದರು ಹಿಂದೂ ಮುಸಲ್ಮಾನರಿಬ್ಬರನ್ನೂ ತಮ್ಮ ಧರ್ಮದಲ್ಲಿ ಸೇರಿಸಿಕೊಂಡು ಒಂದೇ ಉದ್ದೇಶದಿಂದ ಕೆಲಸಮಾಡಿಸಬಲ್ಲವರಾದರು. ಭರತಖಂಡದ ಚರಿತ್ರೆಯಲ್ಲಿ ಇಂಥ ಮತ್ತೊಂದು ದೃಷ್ಟಾಂತ ಕಂಡುಬರುವುದಿಲ್ಲ" ಎಂದನು.

ಸ್ವಾಮೀಜಿ: ಸಾಮಾನ್ಯ ಉದ್ದೇಶವಿಲ್ಲದೆ ಒಂದೇ ಜನರು ಯಾವಾಗಲೂ ಐಕಮತ್ಯವೆಂಬ ಸೂತ್ರದ ಕಟ್ಟಿಗೆ ಸಿಕ್ಕುವುದಿಲ್ಲ. ಸಭೆ ಸಮಿತಿ ಉಪನ್ಯಾಸ ಇವುಗಳನ್ನು ಮಾಡಿ ಸರ್ವಸಾಧಾರಣ ಜನರನ್ನು ಯಾವಾಗಲೂ ಒಟ್ಟುಗೂಡಿಸುವುದು ಆಗಲಾರದು. ಅವರ ಆಸಕ್ತಿ ಒಂದಾಗದೇ ಇದ್ದರೆ ಇವೆಲ್ಲ ಅಪ್ರಯೋಜಕ. ಆಗಿನ ಕಾಲದ ಹಿಂದೂ ಮುಸಲ್ಮಾನ ಎಲ್ಲರೂ ಭಯಂಕರವಾದ ಅತ್ಯಾಚಾರ ಅವಿಚಾರಗಳ ಆಧಿಪತ್ಯದಲ್ಲಿ ವಾಸಮಾಡುತ್ತಿದ್ದರೆಂಬುದನ್ನು ಗುರು ಗೋವಿಂದರು ತೋರಿಸಿಕೊಟ್ಟರು. ಗುರುಗೋವಿಂದರು ಯಾವುದೇ ಹೊಸ ಸಾಮಾನ್ಯ ಉದ್ದೇಶವನ್ನು ಸೃಷ್ಟಿಸಲಿಲ್ಲ. ಆಗಲೇ ಇದ್ದದ್ದನ್ನು ಜನಸಾಧಾರಣರಿಗೆ ತಿಳಿಸಿಕೊಟ್ಟರು - ಅಷ್ಟುಮಾತ್ರ. ಅದರಿಂದಲೆ ಹಿಂದೂ ಮುಸಲ್ಮಾನರೆಲ್ಲರೂ ಅವರನ್ನು ಅನುಸರಿಸಿದರು. ಆತನು ಮಹಾ ಶಕ್ತಿಸಾಧಕನಾಗಿದ್ದನು. ಭರತಖಂಡದ ಚರಿತ್ರೆಯಲ್ಲಿ ಆತನಂಥ ದೃಷ್ಟಾಂತ ಅಪರೂಪ.

ಕತ್ತಲೆಯಾದದ್ದನ್ನು ನೋಡಿ ಸ್ವಾಮಿಜಿ ಎಲ್ಲರನ್ನೂ ಜೊತೆಯಲ್ಲಿ ಕರೆದುಕೊಂಡು ಎರಡನೆಯ ಅಂತಸ್ತಿನ ಬೈಠಕ್ ಖಾನೆಗೆ ಇಳಿದು ಬಂದರು. ಅವರು ಇಲ್ಲಿ ಕುಳಿತುಕೊಂಡೊಡನೆಯೆ ಎಲ್ಲರೂ ಅವರನ್ನು ಸುತ್ತಿಕೊಂಡು ಕುಳಿತರು. ಈ ಸಮಯದಲ್ಲಿ ಸಿದ್ಧಿಯ ಅದ್ಭುತ ಕಾರ್ಯಗಳ ಮಾತು ಬಂತು.

ಸ್ವಾಮಿಜಿ: ಸಿದ್ಧಿ ಅಥವಾ ಅದ್ಭುತ ಕಾರ್ಯಗಳನ್ನು ಮಾಡುವ ಶಕ್ತಿಯನ್ನು ಅತಿ ಸಾಮಾನ್ಯವಾದ ಚಿತ್ತಸಂಯಮದಿಂದಲೇ ಪಡೆಯಬಹುದು ಎಂದರು. ಶಿಷ್ಯನನ್ನು ನೋಡಿ ಮತ್ತೊಬ್ಬರ ಮನಸ್ಸಿನಲ್ಲಿರುವುದನ್ನು ತಿಳಿದುಕೊಳ್ಳುವುದನ್ನು ಕಲಿತುಕೊಳ್ಳಬೇಕೇನು? ನಾಲ್ಕೈದು ದಿನಗಳಲ್ಲಿಯೆ ಈ ವಿದ್ಯೆಯನ್ನು ಕಲಿಸಿಕೊಡಬಲ್ಲೆ ಎಂದು ಹೇಳಿದರು.

ಶಿಷ್ಯ: ಅದರಿಂದ ಉಪಕಾರವಾಗುತ್ತದೆಯೇನು?

ಸ್ವಾಮೀಜಿ: ಏಕೆ? ಅನ್ಯರ ಮನಸ್ಸಿನಲ್ಲಿರುವುದನ್ನು ತಿಳಿದುಕೊಳ್ಳಲು ಶಕ್ತನಾಗುವೆ.

ಶಿಷ್ಯ: ಅದರಿಂದ ಬ್ರಹ್ಮವಿದ್ಯಾಲಾಭಕ್ಕೆ ಏನಾದರೂ ಸಹಾಯವಾಗುತ್ತದೆಯೇನು?

ಸ್ವಾಮೀಜಿ: ಸ್ವಲ್ಪವೂ ಇಲ್ಲ.

ಶಿಷ್ಯ: ಹಾಗಾದರೆ ನನಗೆ ಈ ವಿದ್ಯೆಯನ್ನು ಕಲಿತುಕೊಳ್ಳುವ ಅವಶ್ಯಕತೆಯೇನೂ ಇಲ್ಲ. ಆದರೆ, ಮಹಾಶಯರೆ, ತಾವು ಸಿದ್ಧಿಯ ಸಂಬಂಧವಾಗಿ ಯಾವುದನ್ನು ಪ್ರತ್ಯಕ್ಷ ಮಾಡಿಕೊಂಡಿದ್ದೀರಿ ಅಥವಾ ನೋಡಿದ್ದೀರಿ ಅದರ ವಿಚಾರವಾಗಿ ಕೇಳುವುದಕ್ಕೆ ನನಗೆ ಆಶೆಯಾಗಿದೆ.

ಸ್ವಾಮಿಜಿ: ನಾನು ಒಂದು ಸಲ ಹಿಮಾಲಯದಲ್ಲಿ ಸುತ್ತುತ್ತ ಸುತ್ತುತ ಯಾವುದೋ ಒಂದು ಪರ್ವತ ಗ್ರಾಮದಲ್ಲಿ ಒಂದು ರಾತ್ರಿಯ ಮಟ್ಟಿಗೆ ಇದ್ದೆ. ಸಾಯಂಕಾಲ ಸ್ವಲ್ಪ ಹೊತ್ತಾದ ಮೇಲೆ ಆ ಹಳ್ಳಿಯಲ್ಲಿ ಮದ್ದಲೆಯ ಸದ್ದು ಕೇಳಿಸಿತು. ಆ ಮನೆಯವನನ್ನು ಅದೇನೆಂದು ಕೇಳಲು, ಗ್ರಾಮದ ಯಾವನೋ ಒಬ್ಬನ ಮೇಲೆ ‘ದೆವ್ವ’ ಬಂದಿದೆಯೆಂದು ಗೊತ್ತಾಯಿತು. ಮನೆಯವನ ಉತ್ಸಾಹದಿಂದಲೂ ನನ್ನ ಮನಸ್ಸಿನ ಕುತೂಹಲವನ್ನು ಕಳೆದುಕೊಳ್ಳುವುದಕ್ಕೋಸ್ಕರವೂ ಅದನ್ನು ನೋಡುವುದಕ್ಕೆ ಹೋದೆ. ಹೋಗಿ ನೋಡಿದರೆ, ಅನೇಕ ಜನರು ಗುಂಪುಗೂಡಿದ್ದಾರೆ; ಎತ್ತರವಾಗಿ ಗುಂಗುರು ಕೂದಲಿನ ತಲೆಯುಳ್ಳ ಒಬ್ಬ ಪರ್ವತವಾಸಿಯನ್ನು ತೋರಿಸಿ ‘ಅವನ ಮೇಲೆಯೇ ದೆವ್ವ ಬಂದಿರುವುದು’ ಎಂದು ಹೇಳಿದರು. ಅವನ ಹತ್ತಿರವೆ ಒಂದು ಕೊಡಲಿಯನ್ನು ಬೆಂಕಿಯಲ್ಲಿ ಇಟ್ಟು ಕಾಯಿಸುತ್ತಿದ್ದುದು ಕಂಡಿತು. ಸ್ವಲ್ಪ ಹೊತ್ತಿನ ಮೇಲೆ, ಈ ಬೆಂಕಿಯ ಬಣ್ಣಕ್ಕಾಗಿದ್ದ ಕೊಡಲಿಯಿಂದ ದೆವ್ವ ಬಂದವನ ಮೈಮೇಲೆ ಅಲ್ಲಲ್ಲೆ ಒತ್ತುತ್ತಿದ್ದರು, ತಲೆಕೂದಲಿನಲ್ಲಿಯೂ ತಿವಿಯುತ್ತಿದ್ದರು. ಆದರೆ, ಆಶ್ಚರ್ಯವೇನೆಂದರೆ, ಈ ಕೊಡಲಿಯ ಸ್ಪರ್ಶದಿಂದ ಅವನಿಗೆ ಯಾವ ಅಂಗವಾಗಲಿ ಕೂದಲಾಗಲಿ ಸುಟ್ಟುಹೋಗುತ್ತಿರಲಿಲ್ಲ; ನೋವನ್ನು ಸೂಚಿಸುವ ಚಿಹ್ನೆ ಆತನ ಮುಖದಲ್ಲಿ ತೋರುತ್ತಲೂ ಇರಲಿಲ್ಲ. ನೋಡಿ ಬೆರಗಾಗಿಬಿಟ್ಟೆ. ಈ ಮಧ್ಯದಲ್ಲಿ ಗ್ರಾಮದ ಯಜಮಾನನು ನನ್ನ ಹತ್ತಿರಕ್ಕೆ ಕೈಮುಗಿದುಕೊಂಡು ಬಂದು, “ಮಹಾರಾಜ್ ತಾವು ದಯೆಯಿಟ್ಟು ಇವನಿಗೆ ದೆವ್ವ ಮೆಟ್ಟಿಕೊಂಡಿರುವುದನ್ನು ಬಿಡಿಸಿಕೊಡಬೇಕು" ಎಂದು ಕೇಳಿಕೊಂಡನು. ನಾನು ಇದನ್ನು ಕಂಡು ಸ್ವಲ್ಪ ಅಸ್ಥಿರನಾದೆ! ಏನು ಮಾಡಲಿ - ಎಲ್ಲರೂ ಕೇಳಿಕೊಂಡ ಮೇಲೆ, ಭೂತ ಹಿಡಿದುಕೊಂಡಿದ್ದ ಮನುಷ್ಯನ ಹತ್ತಿರಕ್ಕೆ ಹೋದದ್ದಾಯಿತು. ಹೋದಮೇಲೆ ಮೊದಲು ಕೊಡಲಿಯನ್ನು ಪರೀಕ್ಷಿಸಬೇಕೆಂದು ಇಚ್ಛೆಯುಂಟಾಯಿತು. ಅದನ್ನು ಕೈಯಿಂದ ಹಿಡಿದುಕೊಳ್ಳುತ್ತಿದ್ದ ಹಾಗೆಯೆ ಕೈ ಸುಟ್ಟುಹೋಯಿತು. ಆಗ ಕೊಡಲಿಯೇನೊ ಕಪ್ಪಾಗಿಬಿಟ್ಟಿತ್ತು. ಕೈಯಿನ ಉರಿಯಿಂದ ಅಸ್ಥಿರನಾಗಿದ್ದೆ. ಅದು ಹಾಗಿರಬಹುದು ಹೀಗಿರಬಹುದು ಎಂಬ ಊಹೆ ಗೀಹೆ ಎಲ್ಲಾ ಆಗ ಓಡಿಹೋಯಿತು. ಏನುಮಾಡಲಿ, ಉರಿಯಿಂದ ಅಸ್ಥಿರನಾಗಿದ್ದರೂ ಆ ಮನುಷ್ಯನ ತಲೆಯ ಮೇಲೆ ಕೈಯಿಟ್ಟು ಸ್ವಲ್ಪ ಜಪ ಮಾಡಿದೆ. ಆಶ್ಚರ್ಯವೇನಾಯಿತೆಂದರೆ ಹೀಗೆ ಮಾಡಿದ ಹತ್ತು ಹನ್ನೆರಡು ನಿಮಿಷಗಳೊಳಗೇ ಆತನು ಸ್ವಸ್ಥನಾಗಿಬಿಟ್ಟನು. ಆಗ ನನ್ನ ಮೇಲೆ ಆ ಗ್ರಾಮದ ಜನರ ಭಕ್ತಿಯನ್ನು ನೋಡಬಹುದೆ? ನನ್ನನ್ನು ಒಬ್ಬ ಬಲು ದೊಡ್ಡ ಮನುಷ್ಯನೆಂದು ನಿರ್ಧಾರ ಮಾಡಿಬಿಟ್ಟರು. ನನಗೇನೊ ಈ ಸಂಗತಿ ಸ್ವಲ್ಪವೂ ಅರ್ಥವಾಗದೆ ಹೋಯಿತು. ಮತ್ತೇನೂ ಮಾಡಲಾರದೆ, ನನಗೆ ಆಶ್ರಯ ಕೊಟ್ಟಿದ್ದವನ ಜೊತೆಯಲ್ಲಿ ಸುಮ್ಮನೆ ಅವನ ಗುಡಿಸಲಿಗೆ ಹಿಂತಿರುಗಿ ಬಂದುಬಿಟ್ಟೆ. ಆಗ ರಾತ್ರಿ ಹನ್ನೆರಡು ಘಂಟೆಯಿರಬಹುದು. ಬಂದವನು ಮಲಗಿಬಿಟ್ಟೆ. ಆದರೆ ಕೈಯುರಿಯಿಂದಲೂ ಈ ವಿಚಾರದ ರಹಸ್ಯವನ್ನು ಸ್ವಲ್ಪವೂ ತಿಳಿದುಕೊಳ್ಳಲಾರದೆ ಹೋದೆನಲ್ಲಾ ಎಂಬ ಚಿಂತೆಯಿಂದಲೂ ನಿದ್ರೆ ಬರಲಿಲ್ಲ. ಉರಿಯುತ್ತಿರುವ ಕೊಡಲಿಯಿಂದ ಮನುಷ್ಯನ ದೇಹ ಸುಟ್ಟುಹೋಗದಿದ್ದುದನ್ನು ನೋಡಿ ಇದು ಮಾತ್ರ ಸುಮ್ಮನೆ ಮನಸ್ಸಿಗೆ ಹೊಳೆಯುತ್ತಿತ್ತು, ಏನೆಂದರೆ: \enginline{“There are more things in heaven and earth, than are dreamt of in your philosophy” (} ಪೃಥ್ವಿಯಲ್ಲಿಯೂ ಸ್ವರ್ಗದಲ್ಲಿಯೂ ಇಂಥಿಂಥ ಅನೇಕ ಸಂಗತಿಗಳಿವೆ, ಎಂಥವುಗಳೆಂದರೆ ಅವುಗಳನ್ನು ದರ್ಶನಶಾಸ್ತ್ರಗಳು ಸ್ವಪ್ನದಲ್ಲಿಯೂ ಕಂಡಿಲ್ಲ.)

ಶಿಷ್ಯ: ಆಮೇಲೆ ಈ ವಿಷಯದಲ್ಲಿ ಏನಾದರೂ ತಮಗೆ ಇತ್ಯರ್ಥ ಮಾಡುವುದಕ್ಕಾಯಿತೆ?

ಸ್ವಾಮಿಜಿ: ಇಲ್ಲ; ಈ ಹೊತ್ತು ಮಾತು ಮಾತಿನ ಮೇಲೆ ಜ್ಞಾಪಕಕ್ಕೆ ಬಂತು; ಅದನ್ನೇ ನಿನಗೆ ಹೇಳಿದೆ.

ಆಮೇಲೆ ಸ್ವಾಮಿಜಿ ಮತ್ತೆ ಹೇಳತೊಡಗಿದರು: “ಪರಮಹಂಸರು ಮಾತ್ರ ಸಿದ್ಧಿಗಳನ್ನು ತುಂಬ ಹಳಿಯುತ್ತಿದ್ದರು. ಇವೆಲ್ಲವೂ ಶಕ್ತಿಗಳನ್ನು ತೋರಿಸುವುದರ ಕಡೆಗೆ ಮನಸ್ಸನ್ನೆಳೆದು ಪರಮಾರ್ಥ ತತ್ತ್ವವನ್ನು ಪಡೆಯುವುದಕ್ಕೆ ಅವಕಾಶ ಕೊಡುವುದಿಲ್ಲ ಎಂದು ಹೇಳುತ್ತಿದ್ದರು. ಆದರೆ ಮನುಷ್ಯರ ಮನಸ್ಸು ಎಷ್ಟು ದುರ್ಬಲವಾದದ್ದೆಂದರೆ, ಸಂಸಾರಿಗಳ ಮಾತಿರಲಿ, ಸಾಧುಗಳಲ್ಲಿಯೂ ಶೇಕಡಾ ತೊಂಬತ್ತರಷ್ಟು ಜನರು ಸಿದ್ಧಿಯ ಉಪಾಸಕರಾಗಿ ಪರಿಣಮಿಸುತ್ತಾರೆ. ಪಾಶ್ಚಾತ್ಯ ದೇಶಗಳಲ್ಲಿ ಈ ವಿಧವಾದ ಇಂದ್ರಜಾಲವನ್ನು ನೋಡಿದರೆ ಜನ ಬೆರಗಾಗುತ್ತಾರೆ. ಸಿದ್ಧಿಗಳನ್ನು ಪಡೆಯುವುದು ಹೇಯವಾದದ್ದು, ಧರ್ಮಮಾರ್ಗಕ್ಕೆ ವಿಘ್ನಕಾರಿ ಎಂಬೀ ವಿಷಯವನ್ನು ಪರಮಹಂಸರು ಕೃಪೆಮಾಡಿ ತಿಳಿಸಿಕೊಟ್ಟು ಹೋಗಿರುವುದರಿಂದ ಇದು ನನಗೆ ಗೊತ್ತಿದೆ. ಅದಕ್ಕೋಸ್ಕರವೇ ಪರಮಹಂಸರ ಶಿಷ್ಯರು ಯಾರೂ ಈ ವಿಷಯದಲ್ಲಿ ಆಶೆ ಇಟ್ಟುಕೊಂಡಿರದೆ ಇರುವುದನ್ನು ನೋಡಿಲ್ಲವೇನು?”

ಯೋಗಾನಂದ ಸ್ವಾಮಿಗಳು ಈ ಸಮಯದಲ್ಲಿ ಸ್ವಾಮಿಗಳನ್ನು ಕುರಿತು, “ನಿಮ್ಮ ಜೊತೆಯಲ್ಲಿ ಮದ್ರಾಸಿನಲ್ಲಿದ್ದಾಗ ದೆವ್ವವನ್ನು ನೋಡಿದೆವಲ್ಲ ಆ ವೃತ್ತಾಂತವನ್ನು ಈ ಬಂಗಾಳನಿಗೆ ಹೇಳಿ" ಎಂದರು.

ಶಿಷ್ಯನು ಈ ವೃತ್ತಾಂತವನ್ನು ಹಿಂದೆ ಕೇಳಿರಲಿಲ್ಲ. ಆದ್ದರಿಂದ ಅದನ್ನು ತಿಳಿಸಬೇಕೆಂದು ಸ್ವಾಮಿಗಳ ಹತ್ತಿರ ಹಟಮಾಡುತ್ತ ಕುಳಿತನು. ಸ್ವಾಮೀಜಿ ವಿಧಿಯಿಲ್ಲದೆ ಅವನಿಗೆ ಈ ವೃತ್ತಾಂತವನ್ನು ಹೇಳಿದರು:

“ಮದ್ರಾಸಿನಲ್ಲಿ ಮನ್ಮಥಬಾಬುಗಳ ಮನೆಯಲ್ಲಿದ್ದಾಗ ಒಂದು ದಿನ ನಮ್ಮ ತಾಯಿಯು ಸತ್ತುಹೋಗಿದ್ದಂತೆ ಕನಸಾಯಿತು. ಮನಸ್ಸು ತುಂಬ ವ್ಯಾಕುಲಿತವಾಯಿತು. ಆಗ ಮಠಕ್ಕೂ ಕಾಗದಪತ್ರಗಳನ್ನು ಬಹಳವಾಗಿ ಬರೆಯುತ್ತಿರಲಿಲ್ಲ. ಹೀಗಿರಲು ಮನೆಗೆ ಬರೆಯುವ ವಿಚಾರವನ್ನು ಹೇಳಲು ಅವನು ಆಗಲೇ ಈ ಸಂಬಂಧವಾದ ವರ್ತಮಾನವನ್ನು ತರಿಸುವುದಕ್ಕೋಸ್ಕರ ಕಲ್ಕತ್ತೆಗೆ ತಂತಿ ಕೊಟ್ಟನು. ಏಕೆಂದರೆ, ಸ್ವಪ್ನವನ್ನು ಕಂಡು ನನ್ನ ಮನಸ್ಸು ತುಂಬ ಕಲಕಿಹೋಗಿತ್ತು. ಅಲ್ಲದೆ ಇತ್ತಕಡೆಗೆ ಮದ್ರಾಸಿನ ಸ್ನೇಹಿತರು ಆಗ ನನ್ನನ್ನು ಅಮೆರಿಕಾಕ್ಕೆ ಕಳುಹಿಸುವುದಕ್ಕೆ ಸಿದ್ಧ ಮಾಡುವುದರಲ್ಲಿ ಸಡಗರಪಡುತ್ತಿದ್ದರು. ಆದರೆ ತಾಯಿಯ ದೇಹಾರೋಗ್ಯದ ವಿಚಾರವಾಗಿ ಒಳ್ಳೆಯ ವರ್ತಮಾನ ತಿಳಿಯದಿದ್ದರೆ ನನಗೆ ಹೋಗುವುದಕ್ಕೆ ಇಷ್ಟವಿರಲಿಲ್ಲ. ನನ್ನ ಮನೋಭಾವವನ್ನು ತಿಳಿದು ಮನ್ಮಥಬಾಬು," ಈ ಪಟ್ಟಣದ ಹತ್ತಿರ ಒಬ್ಬ ಪಿಶಾಚ ಸಿದ್ಧನಾದವನು ಇದ್ದಾನೆ, ಅವನು ಜನರ ಶುಭಾಶುಭಗಳು ಭೂತ ಭವಿಷ್ಯತ್ತು ಎಲ್ಲವನ್ನೂ ಸರಿಯಾಗಿ ಹೇಳಬಲ್ಲನು" ಎಂದು ಹೇಳಿದನು. ಅವನು ಕೇಳಿಕೊಂಡದ್ದರಿಂದಲೂ ನನ್ನ ಮನಸ್ಸಿನ ಚಿಂತೆಯನ್ನು ಕಳೆದುಕೊಳ್ಳುವುದಕ್ಕೂ ಆತನ ಹತ್ತಿರ ಹೋಗುವುದಕ್ಕೆ ಒಪ್ಪಿಕೊಂಡೆನು. ಮನ್ಮಥಬಾಬು, ನಾನು, ಅಳಸಿಂಗಾಚಾರ್ಯರು, ಇನ್ನೊಬ್ಬರು ಇಷ್ಟು ಜನವೂ ರೈಲು ಹತ್ತಿ ಹೊರಟು ಆಮೇಲೆ ಅಷ್ಟು ದೂರ ನಡೆದುಕೊಂಡು ಹೋಗಿ ಆ ಸ್ಥಳವನ್ನು ತಲುಪಿದೆವು. ಹೋಗಿ ನೋಡಿದರೆ, ಸ್ಮಶಾನದ ಹತ್ತಿರ ವಿಕಟಾಕಾರವುಳ್ಳ ಒಣಗಿ ಹೋದ, ಕಾಡಿಗೆಯಂತಹ ಒಬ್ಬನು ಮಲಗಿಕೊಂಡೂ ಒಬ್ಬನು ಕುಳಿತುಕೊಂಡೂ ಇದ್ದರು. ಅವರ ಅನುಚರರು ‘ಕೊಣಪಣ’ ಎಂದು ಮದ್ರಾಸಿನ ಮಾತಿನಲ್ಲಿ, ಆತನೇ ಪಿಶಾಚಸಿದ್ಧನಾದ ಮನುಷ್ಯನೆಂದು ತಿಳಿಸಿದರು. ಮೊದಲು ಆತನು ಅಲಕ್ಷ್ಯದಿಂದ ನಮ್ಮ ಕಡೆಗೆ ದೃಷ್ಟಿಯನ್ನೇ ಕೊಡಲಿಲ್ಲ. ಆಮೇಲೆ ನಾವು ಹಿಂತಿರುಗುವುದಕ್ಕೆ ಸಿದ್ಧರಾಗಲು ನಾವು ನಿಲ್ಲಬೇಕೆಂದು ಹೇಳಿದನು. ನಮ್ಮ ಜೊತೆಯಲ್ಲಿದ್ದ ಅಳಸಿಂಗನೆ ಭಾಷಾಂತರಕಾರನ ಕೆಲಸಮಾಡಿ ನಾವು ನಿಲ್ಲಬೇಕೆಂಬ ಮಾತನ್ನು ತಿಳಿಸಿದನು. ಅನಂತರ ಒಂದು ಸೀಸದಕಡ್ಡಿಯನ್ನು ತೆಗೆದುಕೊಂಡು ಆತನು ಸ್ವಲ್ಪ ಹೊತ್ತು ಏನೋ ಗುರುತುಹಾಕುತ್ತಿದ್ದನು. ಆಮೇಲೆ ಆ ಮನುಷ್ಯನು ಚಿತ್ತೈಕಾಗ್ರತೆಯಿಂದಲೋ ಏನೋ ಸ್ಥಿರನಾಗಿಬಿಟ್ಟಿದ್ದನ್ನು ನೋಡಿದೆ. ಆಮೇಲೆ, ಮೊದಲು ನನ್ನ ಹೆಸರನ್ನೂ ನಮ್ಮ ವಂಶದಲ್ಲಿ ಹದಿನಾಲ್ಕು ತಲೆಯವರೆಗಿನ ಹೆಸರನ್ನೂ ಹೇಳಿದನು. ಅದಲ್ಲದೆ ಪರಮಹಂಸರು ನನ್ನ ಜೊತೆಯಲ್ಲಿ ಬೆಂಬಿಡದೆ ಇದ್ದಾರೆಂದೂ ನಮ್ಮ ತಾಯಿ ಕುಶಲವಾಗಿದ್ದಾಳೆಂದೂ ಹೇಳಿದನು; ಮತ್ತು ಧರ್ಮಪ್ರಚಾರ ಮಾಡುವುದಕ್ಕಾಗಿ ನಾನು ಬಹುದೂರಕ್ಕೆ ಅತಿ ಶೀಘ್ರದಲ್ಲಿಯೇ ಹೋಗಬೇಕಾಗಿದೆಯೆಂಬುದನ್ನೂ ಹೇಳಿದನು. ಹೀಗೆ ತಾಯಿಯ ಕ್ಷೇಮ ಸಮಾಚಾರವನ್ನು ತಿಳಿದುಕೊಂಡು ಭಟ್ಟಾಚಾರ್ಯನ ಜೊತೆಯಲ್ಲಿ ಪಟ್ಟಣಕ್ಕೆ ಹಿಂತಿರುಗಿ ಬಂದೆ. ಬಂದ ಮೇಲೆ ಕಲ್ಕತ್ತೆಯಿಂದ ಬಂದ ತಂತಿಯ ವರ್ತಮಾನದಿಂದಲೂ ನಮ್ಮ ತಾಯಿಯ ಕುಶಲ ಸಮಾಚಾರ ತಿಳಿಯಿತು.

ಯೋಗಾನಂದ ಸ್ವಾಮಿಗಳ ಕಡೆಗೆ ನೋಡಿ ಸ್ವಾಮಾಜಿ, “ಅದು ಕಾಕತಾಳೀಯ ನ್ಯಾಯವೇ ಆಗಿರಲಿ ಮತ್ತೆ ಏನಾದರೂ ಆಗಿರಲಿ ಆತನು ಏನೇನು ಹೇಳಿದನೋ ಅದೆಲ್ಲ ಹಾಗೆಹಾಗೇ ನಡೆದೇನೋ ನಡೆಯಿತು" ಎಂದು ಹೇಳಿದರು.

ಯೋಗಾನಂದ ಸ್ವಾಮಿಗಳು ಅದಕ್ಕೆ ಉತ್ತರವಾಗಿ, “ನಾನು ಮೊದಲು ಇದನ್ನೆಲ್ಲ ಸ್ವಲ್ಪವೂ ನಂಬುತ್ತಿರಲಿಲ್ಲ. ಅದಕ್ಕೋಸ್ಕರವೇ ಇದನ್ನೆಲ್ಲ ನೋಡಬೇಕಾದ ಆವಶ್ಯಕತೆಯುಂಟಾಯಿತು" ಎಂದರು.

ಸ್ವಾಮಾಜಿ: ನಾನೇನು ಕಣ್ಣಾರೆ ನೋಡದೆ ಕಿವಿಯಾರೆ ಕೇಳದೆ ಯಾವುದೆಂದರೆ ಅದನ್ನು ನಂಬುವವನೇನು? ಅಂಥ ಮನುಷ್ಯನಲ್ಲವೇ ಅಲ್ಲ. ಮಹಾಮಾಯೆಯ ರಾಜ್ಯಕ್ಕೆ ಬಂದು ಜಗತ್ತಿನಲ್ಲಿ ಎಷ್ಟೊಂದು ಇಂದ್ರಜಾಲ ಮಹೇಂದ್ರ ಜಾಲಗಳನ್ನು ನೋಡಿಬಿಟ್ಟಿದ್ದೇನೆ! ಮಾಯಾ-ಮಾಯಾ!! ರಾಮ ರಾಮ! ಈವೊತ್ತು ಏನು! ಬರೀ ಬೂದಿ ಭಸ್ಮಗಳ ಮಾತೇ ಆಗಿಹೊಯಿತು! ಭೂತವನ್ನು ಮನಸ್ಸಿನಲ್ಲಿ ಭಾವಿಸಿಕೊಳ್ಳುತ್ತ ಭಾವಿಸಿಕೊಳ್ಳುತ್ತ ಜನರು ಭೂತವಾಗಿಬಿಡುತ್ತಾರೆ. ಮತ್ತೆ, ಯಾರು ಹಗಲೂ ರಾತ್ರಿಯೂ, ತಿಳಿದೊ ತಿಳಿಯದೆಯೊ, ‘ನಾನು ನಿತ್ಯಶುದ್ಧ ಬುದ್ಧ ಮುಕ್ತಸ್ವರೂಪನಾದ ಆತ್ಮ’ ಎಂದುಕೊಳ್ಳುತ್ತಾರೆಯೋ ಅವರು ಬ್ರಹ್ಮಜ್ಞರಾಗುವರು.

ಹೀಗೆಂದು ಹೇಳಿ ಸ್ವಾಮೀಜಿ ಸ್ನೇಹಪೂರ್ವಕವಾಗಿ ಶಿಷ್ಯನ ಕಡೆಗೆ ತಿರುಗಿ ಹೀಗೆಂದು ಹೇಳಿದರು: ಕೆಲಸಕ್ಕೆ ಬಾರದ ಮಾತುಗಳಿಗೆಲ್ಲಾ ಮನಸ್ಸಿನಲ್ಲಿ ಸ್ವಲ್ಪವೂ ಎಡೆಗೊಡಬೇಡ. ಕೇವಲ ಸದಸದ್ವಿಚಾರವನ್ನು ಮಾಡು. ಆತ್ಮವನ್ನು ಪ್ರತ್ಯಕ್ಷ ಮಾಡಿಕೊಳ್ಳುವುದಕ್ಕೆ ಪ್ರಾಣವನ್ನಾದರೂ ಕೊಟ್ಟು ಪ್ರಯತ್ನ ಮಾಡು. ಆತ್ಮಜ್ಞಾನಕ್ಕಿಂತ ಶ್ರೇಷ್ಠವಾದದ್ದು ಮತ್ತಾವುದೂ ಇಲ್ಲ. ಮಿಕ್ಕದ್ದೆಲ್ಲಾ ಮಾಯಾ - ಇಂದ್ರಜಾಲ ಮಹೇಂದ್ರಜಾಲ. ಪ್ರತ್ಯಗಾತ್ಮನೊಬ್ಬನೇ ನಿಜವಾಗಿಯೂ ಸತ್ಯ. ಈ ವಿಷಯವನ್ನು ತಿಳಿದುಕೊ; ಅದಕ್ಕೋಸ್ಕರವೇ ನಿಮಗೆ ತಿಳಿಸುವುದಕ್ಕೆ ಒದ್ದಾಡುತ್ತಿದ್ದೇನೆ. “ಏಕಮೇವಾದ್ವಿತೀಯಂ ಬ್ರಹ್ಮ ನೇಹ ನಾನಾಸ್ತಿ ಕಿಂಚನ." ಇರುವುದು ಏಕವಾದ, ಎರಡಿಲ್ಲದ ಬ್ರಹ್ಮವೊಂದೇ, ಬೇರೇನೂ ಇಲ್ಲ. (ಬೃ.ಉ. ೪.೧೬).

ಮಾತು ನಡೆಯುತ್ತ ನಡೆಯುತ್ತ ರಾತ್ರಿ ಹನ್ನೊಂದು ಘಂಟೆ ಹೊಡೆದು ಹೋಯಿತು. ಅನಂತರದಲ್ಲಿ ಸ್ವಾಮೀಜಿ ಊಟಮಾಡಿ ವಿಶ್ರಮಿಸಿಕೊಳ್ಳುವುದಕ್ಕೆ ಎದ್ದರು. ಶಿಷ್ಯನು ಸ್ವಾಮಿಜಿಯವರ ಪಾದಪದ್ಮದಲ್ಲಿ ಪ್ರಣಾಮ ಮಾಡಿ ಅಪ್ಪಣೆಯನ್ನು ಪಡೆದನು. ಸ್ವಾಮೀಜಿ “ನಾಳೆ ಬರುತ್ತೀಯಷ್ಟೆ?" ಎಂದು ಕೇಳಿದರು.

ಶಿಷ್ಯ: ಹೌದು! ಬಂದೇ ಬರುತ್ತೇನೆ. ತಮ್ಮನ್ನು ಸಾಯಂಕಾಲದ ಹೊತ್ತು ನೋಡದಿದ್ದರೆ ಮನಸ್ಸು ವ್ಯಾಕುಲಗೊಂಡು ಮಿಲಮಿಲನೆ ಒದ್ದಾಡುವುದು.

ಶಿಷ್ಯನು ಸ್ವಾಮಿಗಳ ಮಾತನ್ನು ಮನನ ಮಾಡುತ್ತಾ ಮಾಡುತ್ತಾ ರಾತ್ರಿ ಹನ್ನೆರಡು ಘಂಟೆಯ ಹೊತ್ತಿಗೆ ಮನೆಗೆ ಹಿಂತಿರುಗಿ ಬಂದನು.

\newpage

\chapter[ಅಧ್ಯಾಯ ೧೩]{ಅಧ್ಯಾಯ ೧೩\protect\footnote{\engfoot{C.W, Vol. VII, P. 107}}}

\begin{center}
ಸ್ಥಳ: ಬೇಲೂರು ಮಠ (ಬಾಡಿಗೆ ಕಟ್ಟಡ); ವರ್ಷ: ಕ್ರಿ.ಶ. ೧೮೯೮.
\end{center}

ಸ್ವಾಮೀಜಿ ಇಂಗ್ಲೆಂಡಿನಿಂದ ಹಿಂತಿರುಗಿ ಬಂದ ವರ್ಷ ದಕ್ಷಿಣೇಶ್ವರದ ರಾಣಿ ರಾಸಮಣಿಯ ಕಾಳಿದೇವಸ್ಥಾನದಲ್ಲಿ ಶ‍್ರೀ ರಾಮಕೃಷ್ಣ ಪರಮಹಂಸರ ಜನ್ಮೋತ್ಸವ ವಿಜೃಂಭಣೆಯಿಂದ ನಡೆಯಿತು. ಆದರೆ, ನಾನಾ ಕಾರಣದಿಂದ ಮುಂದಿನ ವರ್ಷ ದಕ್ಷಿಣೇಶ್ವರದಲ್ಲಿ ಉತ್ಸವ ನಿಂತುಹೋಯಿತು ಮತ್ತು ಬೇಲೂರಿನಲ್ಲಿ ಗಂಗಾ ತೀರದಲ್ಲಿದ್ದ ಶ‍್ರೀಯುತ ನೀಲಾಂಬರ ಮುಖ್ಯೋಪಾಧ್ಯಾಯರ ಆರಾಮಗೃಹವನ್ನು ಬಾಡಿಗೆಗೆ ತೆಗೆದುಕೊಂಡು ಅಲ್ಲಿಗೆ ಮಠವನ್ನು ಸಾಗಿಸಲಾಗಿತ್ತು. ಅದಾದ ಕೆಲವು ದಿವಸದ ಮೇಲೆ ಈಗಿನ ಮಠ ಇರುವ ಸ್ಥಳ ಖರೀದಿಯಾಯಿತು. ಆದರೂ ಆ ವರ್ಷ ಜನ್ಮೋತ್ಸವ ಹೊಸ ಸ್ಥಳದಲ್ಲಿ ನಡೆಯುವುದಕ್ಕಾಗಲಿಲ್ಲ. ಏಕೆಂದರೆ, ಆಗ ಮಠದ ನಿವೇಶನದ ತುಂಬ ಕಾಡು ಬೆಳೆದುಕೊಂಡಿತ್ತು. ಅನೇಕ ಕಡೆಗಳಲ್ಲಿ ನೆಲ ಸಮವಾಗಿರಲಿಲ್ಲ. ಇದರಿಂದಲೇ ಆ ಸಾರಿ ಶ‍್ರೀರಾಮಕೃಷ್ಣ ಜನ್ಮೋತ್ಸವ ಬೇಲೂರಿನಲ್ಲಿ ದಾಯರವರು ಕಟ್ಟಿಸಿದ್ದ ಗುಡಿಯಲ್ಲಿ ನಡೆಯಿತು. ಈ ಉತ್ಸವಕ್ಕೆ ಹಿಂದಿನ ಫಾಲ್ಗುಣ ಶುದ್ಧ ಬಿದಿಗೆಯಂದು ನೀಲಾಂಬರ ಬಾಬುಗಳ ತೋಟದಲ್ಲಿ ಶ‍್ರೀರಾಮಕೃಷ್ಣ ಪರಮಹಂಸರ ಜನ್ಮತಿಥಿ ಪೂಜೆ ನಡೆಯಿತು. ಇದಾದ ಒಂದೆರಡು ದಿನಕ್ಕೆ ಒಳ್ಳೆಯ ಮುಹೂರ್ತದಲ್ಲಿ, ಮಠಕ್ಕೋಸ್ಕರ ಕೊಂಡುಕೊಂಡಿದ್ದ ಹೊಸ ನೀವೇಶನಕ್ಕೆ ಶ‍್ರೀರಾಮಕೃಷ್ಣರ ಮೂರ್ತಿ ಮುಂತಾದ್ದನ್ನೆಲ್ಲಾ ತೆಗೆದುಕೊಂಡು ಹೋಗಿ ಪೂಜೆ ಹೋಮಾದಿಗಳನ್ನು ಮಾಡಿ ಆಮೇಲೆ ಪರಮಹಂಸರ ಪ್ರತಿಮೆಯನ್ನು ಪ್ರತಿಷ್ಠೆ ಮಾಡಲಾಯಿತು. ಸ್ವಾಮಿಜಿ ಆಗ ಹಿಂದೆ ಹೇಳಿದ ನೀಲಾಂಬರ ಬಾಬುಗಳ ತೋಟದಲ್ಲಿಯೇ ಇರುತ್ತಿದ್ದರು. ದೊಡ್ಡ ಪ್ರಮಾಣದಲ್ಲಿ ಶ‍್ರೀ ರಾಮಕೃಷ್ಣರ ಜನ್ಮತಿಥಿ ಪೂಜೆಗೆ ಎಲ್ಲ ವ್ಯವಸ್ಥೆಯನ್ನು ಮಾಡಲಾಗಿತ್ತು. ಸ್ವಾಮೀಜಿ ಅಪ್ಪಣೆಯನ್ನನುಸರಿಸಿ ದೇವರ ಮನೆಯು ಬೇಕಾದ ಸಲಕರಣೆ ಸಾಮಗ್ರಿಗಳಿಂದ ತುಂಬಿತ್ತು. ಸ್ವಾಮೀಜಿ ಈ ದಿನ ತಾವೇ ಎಲ್ಲಾ ವಿಷಯಗಳನ್ನು ವಿಚಾರಿಸಿ ತಿಳಿದುಕೊಂಡರು.

ಜನ್ಮತಿಥಿಯ ಬೆಳಗ್ಗೆ ಎಲ್ಲರಿಗೂ ಆನಂದ! ಪರಮಹಂಸರ ಮಾತನ್ನು ಬಿಟ್ಟು ಭಕ್ತರ ಬಾಯಲ್ಲಿ ಮತ್ತೊಂದು ಮಾತಿಲ್ಲ. ಈಗ ಸ್ವಾಮೀಜಿ ಪೂಜಾ ಮಂದಿರದ ಮುಂದಕ್ಕೆ ಬಂದು ಪೂಜೆಗೆ ಸಿದ್ಧತೆ ನಡೆಯುತ್ತಿದ್ದದ್ದನ್ನು ನೋಡುತ್ತ ನಿಂತುಕೊಂಡರು.

ಪೂಜೆಗೆ ಆಗುತ್ತಿದ್ದ ಏರ್ಪಾಡನ್ನು ನೋಡಿ ಸ್ವಾಮೀಜಿ ಶಿಷ್ಯನನ್ನು ಕುರಿತು “ಯಜ್ಞೋಪವೀತವನ್ನು ತಂದಿದ್ದೀಯಷ್ಟೇ” ಎಂದು ಕೇಳಿದರು.

ಶಿಷ್ಯ: ತಮ್ಮ ಅಪ್ಪಣೆಯಂತೆ ಎಲ್ಲವೂ ಸಿದ್ಧವಾಗಿದೆ; ಆದರೆ ಇಷ್ಟೊಂದು ಯಜ್ಞೋಪವೀತಗಳನ್ನು ಸಿದ್ಧಪಡಿಸಿಕೊಂಡದ್ದು ಏಕೊ ನನಗೆ ಗೊತ್ತಾಗಲಿಲ್ಲ.

ಸ್ವಾಮೀಜಿ: ದ್ವಿಜರಿಗೆ ಮಾತ್ರ ಉಪನಯನ ಸಂಸ್ಕಾರಕ್ಕೆ ಅಧಿಕಾರ, ಇದಕ್ಕೆ ವೇದವೇ ಪ್ರಮಾಣ. ಇಂದು ಪರಮಹಂಸರ ಜನ್ಮೋತ್ಸವಕ್ಕೆ ಯಾರು ಯಾರು ಬರುತ್ತಾರೆಯೋ ಅವರೆಲ್ಲರಿಗೂ ಜನಿವಾರವನ್ನು ಹಾಕಿಸುತ್ತೇನೆ. ಇವರೆಲ್ಲಾ ಪತಿತರಾಗಿ ಬಿಟ್ಟಿದ್ದಾರೆ. ಪ್ರಾಯಶ್ಚಿತ್ತ ಮಾಡಿಕೊಂಡ ಮೇಲೆ ಉಪನಯನ ಸಂಸ್ಕಾರಕ್ಕೆ ಅರ್ಹನಾಗುವನೆಂದು ಶಾಸ್ತ್ರದಲ್ಲಿ ಹೇಳಿದೆ. ಇಂದು ಪರಮಹಂಸರ ಶುಭಜನ್ಮೋತ್ಸವ, ಎಲ್ಲರೂ ಅವರ ನಾಮೋಚ್ಛಾರಣೆ ಮಾಡಿ ಪರಿಶುದ್ಧರಾಗುವರು. ಅದಕ್ಕೋಸ್ಕರವೇ ಇಂದು ಬರುವ ಭಕ್ತರಿಗೆಲ್ಲ ಜನಿವಾರವನ್ನು ಹಾಕಿಸಬೇಕು; ತಿಳಿಯಿತೇ?

ಶಿಷ್ಯ: ನಾನು ತಮ್ಮ ಅಪ್ಪಣೆಯಂತೆ ಬೇಕಾದಷ್ಟು ಜನಿವಾರವನ್ನು ಕೂಡಿಹಾಕಿಕೊಂಡು ತಂದಿದ್ದೇನೆ. ಪೂಜೆಯಾದ ಮೇಲೆ ತಮ್ಮ ಅನುಮತಿಯನ್ನು ಪಡೆದು, ಬಂದ ಭಕ್ತರಿಗೆಲ್ಲಾ ಅದನ್ನು ಹಾಕಿಸುತ್ತೇನೆ.

ಸ್ವಾಮೀಜಿ: ಬ್ರಾಹ್ಮಣೇತರ ಭಕ್ತರಿಗೆ ಈ ವಿಧದ ಗಾಯತ್ರಿ ಮಂತ್ರವನ್ನು ಹೇಳಿಕೊಡುತ್ತೇನೆ (ಇಲ್ಲಿ ಕ್ಷತ್ರಿಯಾದಿ ದ್ವಿಜರ ಗಾಯತ್ರಿಯನ್ನು ಹೇಳಿದರು). ಕಾಲಕ್ರಮದಲ್ಲಿ ದೇಶವನ್ನೆಲ್ಲ ಬ್ರಾಹ್ಮಣ ಪದವಿಗೆ ಹತ್ತಿಸಬೇಕಾಗಿದೆ. ಪರಮಹಂಸರ ಭಕ್ತರ ಮಾತಂತೂ ಹೇಳಬೇಕಾಗಿಲ್ಲ. ಹಿಂದೂಗಳೆಲ್ಲರೂ ಪರಸ್ಪರ ಅಣ್ಣ ತಮ್ಮಂದಿರು. ಮುಟ್ಟಬೇಡ ಮುಟ್ಟಬೇಡ ಎಂದು ಹೇಳಿ ಹೇಳಿ ನಾವೇ ಅವರನ್ನು ಕೀಳುಮಾಡಿ ಕೂರಿಸಿದ್ದೇವೆ. ದೇಶದ ಹೀನತೆ, ಭೀರುತ್ವ, ಮೂರ್ಖತನ, ಹೇಡಿತನ ಇವುಗಳು ಪರಾಕಾಷ್ಠೆಯನ್ನು ಮುಟ್ಟಿದೆ. ಇವರನ್ನು ಮೇಲಕ್ಕೆ ಎತ್ತಬೇಕು; ಅಭಯದ ಮಾತನ್ನು ಹೇಳಬೇಕು. ‘ನೀವು ನಮ್ಮ ಹಾಗೇ ಮನುಷ್ಯರು. ನಿಮಗೂ ನಮ್ಮ ಹಾಗೇ ಎಲ್ಲಾ ಅಧಿಕಾರವೂ ಇದೆ’ ಎಂದು ಹೇಳಬೇಕು ತಿಳಿಯಿತೆ.

ಶಿಷ್ಯ: ತಿಳಿಯಿತು.

ಸ್ವಾಮೀಜಿ: ಈಗ ಯಾರುಯಾರಿಗೆ ಜನಿವಾರ ಬೇಕೋ ಅವರೆಲ್ಲರೂ ಗಂಗಾಸ್ನಾನ ಮಾಡಿಕೊಂಡು ಬರುವಂತೆ ಹೇಳು; ಆಮೇಲೆ ಪರಮಹಂಸರಿಗೆ ನಮಸ್ಕಾರಮಾಡಿ ಎಲ್ಲರೂ ಜನಿವಾರವನ್ನು ಹಾಕಿಕೊಳ್ಳಲಿ..

ಸುಮಾರು ನಲ್ವತ್ತು ಐವತ್ತು ಜನ ಭಕ್ತರು ಸ್ವಾಮೀಜಿಯ ಅಪ್ಪಣೆಯ ಮೇರೆಗೆ, ಕ್ರಮವಾಗಿ ಗಂಗಾಸ್ನಾನ ಮಾಡಿ ಬಂದು ಶಿಷ್ಯನಿಂದ ಗಾಯತ್ರಿ ಉಪದೇಶವನ್ನು ಹೊಂದಿ ಜನಿವಾರವನ್ನು ಹಾಕಿಕೊಳ್ಳುವುದಕ್ಕೆ ಮೊದಲು ಮಾಡಿದರು. ಮಠದಲ್ಲಿ ಮಂಗಳರವ ಉಂಟಾಯಿತು. ಜನಿವಾರವನ್ನು ಹಾಕಿಕೊಂಡು ಭಕ್ತರು ಪುನಃ ಪರಮಹಂಸರಿಗೆ ಪ್ರಣಾಮ ಮಾಡಿದರು; ಅದನ್ನು ನೋಡಿ ಸ್ವಾಮಿಜಿಯವರ ಮುಖಾರವಿಂದವು ಒಂದಕ್ಕೆ ನೂರರಷ್ಟು ಅರಳಿದಂತಾಯಿತು. ಇದಾದ ಸ್ವಲ್ಪಹೊತ್ತಿಗೆ ಶ‍್ರೀಯುತ ಗಿರೀಶಚಂದ್ರಘೋಷ್ ಮಹಾಶಯರು ಮಠಕ್ಕೆ ಬಂದರು.

ಈಗ ಸ್ವಾಮಿಜಿಯವರ ಅಪ್ಪಣೆಯಂತೆ ಸಂಗೀತ ಆರಂಭವಾಯಿತು. ಮಠದ ಸಂನ್ಯಾಸಿಗಳು ಸ್ವಾಮೀಜಿಗೆ ತಮ್ಮ ಮನದಣಿಯ ವೇಷಭೂಷಣಗಳನ್ನು ಹಾಕತೊಡಗಿದರು. ಅವರ ಕಿವಿಯಲ್ಲಿ ಶಂಖದ ಕುಂಡಲ, ಮೈಯಲ್ಲಿ ಕರ್ಪೂರಧೂಸರಿತವಾದ ಪವಿತ್ರ ವಿಭೂತಿ, ತಲೆಯಲ್ಲಿ ಕಾಲುತನಕ ಜೋಲಾಡುವ ಜಟೆಗಳು, ಎಡಗೈಯಲ್ಲಿ ತ್ರಿಶೂಲ, ಎರಡು ತೋಳುಗಳಲ್ಲಿಯೂ ರುದ್ರಾಕ್ಷಿವಲಯ, ಕತ್ತಿನಲ್ಲಿ ಮಂಡಿಯವರೆಗೆ ಬಂದಿದ್ದ ಮೂರು ಸಾಲು ದೊಡ್ಡ ರುದ್ರಾಕ್ಷಿಯ ಮಾಲೆ - ಮುಂತಾದುವುಗಳನ್ನೆಲ್ಲಾ ಹಾಕಿದರು. ಇವೆಲ್ಲವನ್ನೂ ಧರಿಸಿಕೊಳ್ಳಲು ಸ್ವಾಮಿಗಳ ರೂಪದಲ್ಲಿ ಉಂಟಾದ ರಮಣೀಯತೆಯನ್ನು ಬಾಯಲ್ಲಿ ಹೇಳತೀರದು. ಆ ದಿನ ಯಾರು ಆ ಮೂರ್ತಿಯನ್ನು ನೋಡಿದರೆ ಅವರೆಲ್ಲರೂ, ಸಾಕ್ಷಾತ್ ಭೈರವನೆ ಸ್ವಾಮಿಗಳ ಶರೀರದಿಂದ ಭೂಮಿಯಲ್ಲಿ ಅವತರಿಸಿದ್ದಾನೆಂದು ಹೇಳಿದರು. ಸ್ವಾಮಿಗಳೂ ಇತರ ಸಂನ್ಯಾಸಿಗಳ ದೇಹಕ್ಕೆ ವಿಭೂತಿ ಇಟ್ಟರು. ಅವರು ಸ್ವಾಮೀಜಿಯ ನಾಲ್ಕು ಕಡೆಯಲ್ಲಿಯೂ ಮೂರ್ತಿಮಂತ ಭೈರವ ಗಣಗಳ ಹಾಗೆ ಇದ್ದುಕೊಂಡು, ಮಠಪ್ರದೇಶದಲ್ಲಿ ಕೈಲಾಸಪರ್ವತದ ಅಂದವನ್ನು ಬೀರುತ್ತಿದ್ದರು. ಆ ದೃಶ್ಯವನ್ನು ನೆನೆಸಿಕೊಂಡರೂ ಆನಂದವಾಗುತ್ತದೆ.

ಈಗ ಸ್ವಾಮೀಜಿ ಪಶ್ಚಿಮ ದಿಕ್ಕಿಗೆ ತಿರುಗಿಕೊಂಡು ಮುಕ್ತ ಪದ್ಮಾಸನದಲ್ಲಿ ಕುಳಿತುಕೊಂಡು “ಕೂಜಂತಂ ರಾಮರಾಮೇತಿ" ಎಂಬ ಪ್ರಾರ್ಥನಾ ಶ್ಲೋಕವನ್ನು ಮಧುರಮಧುರವಾಗಿ ಹೇಳುತ್ತ, ಅದು ಮುಗಿದ ಮೇಲೆ ‘ರಾಮ ರಾಮ ಶ‍್ರೀರಾಮರಾಮ’ ಎಂಬುದನ್ನೇ ಪುನಃ ಪುನಃ ಎನ್ನತೊಡಗಿದರು. ಅಕ್ಷರಕ್ಷರದಲ್ಲಿಯೂ ಅಮೃತವು ತೊಟ್ಟಿಕ್ಕುವಂತಿತ್ತು. ಸ್ವಾಮೀಜಿಯವರ ಕಣ್ಣುಗಳು ಅರ್ಧ ಮುಚ್ಚಿವೆ. ಕೈಯಲ್ಲಿ ತಂಬೂರಿಯ ಸ್ವರ ಶ್ರುತಿಗೊಡುತ್ತಿದೆ. ‘ರಾಮರಾಮ ಶ‍್ರೀರಾಮರಾಮ’ ಎಂಬ ಧ್ವನಿಯನ್ನು ಬಿಟ್ಟು ಸ್ವಲ್ಪ ಹೊತ್ತಿನ ತನಕ ಮತ್ತೆ ಯಾವುದೂ ಕೇಳುತ್ತಿರಲಿಲ್ಲ. ಹೀಗೆ ಸುಮಾರು ಅರ್ಧ ಗಂಟೆಗಿಂತ ಹೆಚ್ಚಾಗಿ ಕಳೆಯಿತು; ಆಗಲೂ ಯಾರ ಬಾಯಲ್ಲಿಯೂ ಬೇರೆ ಯಾವ ಮಾತೂ ಇಲ್ಲ. ಸ್ವಾಮಿಜಿಯವರ ಬಾಯಿಂದ ಬರುತ್ತಿದ್ದ ರಾಮ ನಾಮಾಮೃತವನ್ನು ಪಾನ ಮಾಡಿ ಎಲ್ಲರೂ ಇಂದು ಮತ್ತರಾಗಿಬಿಟ್ಟಿದ್ದರು. ಇದೇನು ನಿಜವಾಗಿಯೂ ಇವತ್ತು ಸ್ವಾಮೀಜಿ ಶಿವಭಾವದಲ್ಲಿ ಮತ್ತರಾಗಿ ರಾಮನಾಮವನ್ನು ಹೇಳುವುದಕ್ಕೆ ತೊಡಗಿದರು ಎಂದು ಶಿಷ್ಯ ಮನಸ್ಸಿನಲ್ಲಿ ಭಾವಿಸಿಕೊಂಡನು. ಸ್ವಾಮೀಜಿಯವರ ಮುಖದ ಸ್ವಾಭಾವಿಕ ಗಾಂಭೀರ್ಯ ಇಂದು ನೂರ್ಮಡಿಯಾದ ಗಂಭೀರತೆಯನ್ನು ಪಡೆದಂತಿತ್ತು; ಅರೆಮುಚ್ಚಿದ ಕಣ್ಣಂಚಿನಲ್ಲಿ ಪ್ರಭಾತ ಸೂರ್ಯನ ಕಾಂತಿ ಹೊರಟು ಹೊರಗೆ ಬರುವಂತಿತ್ತು. ಅತ್ಯಂತವಾದ ಆನಂದಮತ್ತತೆಯಲ್ಲಿ ಆ ವಿಪುಲವಾದ ದೇಹ ತೂರಾಡುವಂತಿತ್ತು. ಆ ರೂಪವನ್ನು ವರ್ಣನೆಮಾಡಲು ಸಾಧ್ಯವಾಗುವುದಿಲ್ಲ; ಅನುಭವಕ್ಕೆ ಬರಬೇಕಾದದ್ದು. ನೋಟಕರು “ಚಿತ್ರಾರ್ಪಿತಾರಂಭ ಇವಾವತಸ್ಥೇ" - ಚಿತ್ರದಲ್ಲಿ ಬರೆದವರಂತೆ ಇದ್ದರು.

ರಾಮನಾಮ ಸಂಕೀರ್ತನೆಯು ಮುಗಿದ ಮೇಲೆ ಸ್ವಾಮಿಜಿ ಮೊದಲಿನಂತೆ ಆನಂದ ಪ್ರಮತ್ತತೆಯಲ್ಲಿಯೇ ‘ಸೀತಾಪತಿ ರಾಮಚಂದ್ರ ರಘುರಾಈ’ ಎಂಬುದನ್ನು ಗಾನಮಾಡುವುದಕ್ಕೆ ತೊಡಗಿದರು. ಮೃದಂಗ ಬಾರಿಸುವವನು ಸರಿಯಾಗಿಲ್ಲದ್ದರಿಂದ ಸ್ವಾಮಿಗಳಿಗೆ ರಸಭಂಗವಾದಂತೆ ತೋರಿತು. ಆಗ ಅವರು ಶಾರದಾನಂದ ಸ್ವಾಮಿಗಳನ್ನು ಗಾನಮಾಡುವಂತೆ ಅಪ್ಪಣೆ ಮಾಡಿ ತಾವೇ ಮೃದಂಗವನ್ನು ತೆಗೆದುಕೊಂಡರು. ಶಾರದಾನಂದ ಸ್ವಾಮಿಗಳು ಮೊದಲು “ಏಕರೂಪ ಅರೂಪನಾಮವರಣ” ಎಂಬ ಕೀರ್ತನೆಯನ್ನು ಹಾಡಿದರು. ಮೃದಂಗದ ಸ್ನಿಗ್ಧ ಗಂಭೀರ ನಿರ್ಘೋಷಕ್ಕೆ ಗಂಗೆಯು ಉಕ್ಕಿ ಉಕ್ಕಿ ಬರುವಂತಿತ್ತು. ಶಾರದಾನಂದ ಸ್ವಾಮಿಗಳ ಸುಕಂಠವೂ ಮಧುರ ಆಲಾಪನೆಯೂ ಮನೆಯನ್ನೆಲ್ಲಾ ತುಂಬಿಕೊಂಡಿತು. ಆಮೇಲೆ ಶ‍್ರೀರಾಮಕೃಷ್ಣರು ಹಾಡುತ್ತಿದ್ದ ಹಾಡುಗಳನ್ನೆಲ್ಲಾ ಒಂದೊಂದಾಗಿ ಹಾಡತೊಡಗಿದರು.

ಈಗ ಸ್ವಾಮಿಜಿ ಇದ್ದಕ್ಕಿದ್ದ ಹಾಗೆ ತಮ್ಮ ವೇಷಭೂಷಣಗಳನ್ನೆಲ್ಲಾ ತೆಗೆದುಹಾಕಿ ಗಿರೀಶಬಾಬುಗಳಿಗೆ ಇವುಗಳನ್ನು ತೊಡಿಸುವುದಕ್ಕೆ ಹೊರಟರು. ತಮ್ಮ ಕೈಯಿಂದಲೆ ಗಿರೀಶಬಾಬುಗಳ ವಿಶಾಲವಾದ ದೇಹದಲ್ಲಿ ವಿಭೂತಿಯನ್ನು ಬಳಿದು ಕಿವಿಯಲ್ಲಿ ಕುಂಡಲವನ್ನೂ ತಲೆಯಲ್ಲಿ ಜಟೆಯನ್ನೂ ಕತ್ತಿನಲ್ಲಿ ರುದ್ರಾಕ್ಷಿಯನ್ನೂ ತೋಳುಗಳಲ್ಲಿ ರುದ್ರಾಕ್ಷಿ ವಲಯವನ್ನೂ ಇಟ್ಟರು. ಗಿರೀಶ ಬಾಬುಗಳು ಇವುಗಳಿಂದ ಬೇರೊಬ್ಬ ಮೂರ್ತಿಯಾಗಿ ಪರಿಣಮಿಸಿದಂತಾದರು. ಇದನ್ನು ನೋಡಿದ ಭಕ್ತರಿಗೆ ಬೆರಗಾಯಿತು. ಆಮೇಲೆ ಸ್ವಾಮೀಜಿ ‘ಈತ ಭೈರವನ ಅವತಾರ; ನಮಗೂ ಈತನಿಗೂ ಯಾವ ವ್ಯತ್ಯಾಸವೂ ಇಲ್ಲ’ ಎಂದು ಪರಮಹಂಸರು ಹೇಳುತ್ತಿದ್ದರು - ಎಂದರು. ಗಿರೀಶಬಾಬುಗಳೂ ವಿಸ್ಮಿತರಾಗಿ ಕುಳಿತುಕೊಂಡಿದ್ದರು. ಅವರ ಸಂನ್ಯಾಸಿ ಗುರು ಭ್ರಾತೃಗಳು ಅವರನ್ನು ಇಂದು ಹೇಗೆ ಅಲಂಕರಿಸಬೇಕೆಂದು ಇಷ್ಟಪಟ್ಟರೊ ಆ ವೇಷದಲ್ಲಿ ಅವರು ಒಪ್ಪುತ್ತಿದ್ದರು. ಕೊನೆಗೆ ಸ್ವಾಮೀಜಿಯ ಅಪ್ಪಣೆಯಂತೆ ಒಂದು ಕಾವಿಯಬಟ್ಟೆಯನ್ನು ತಂದು ಗಿರೀಶಬಾಬುಗಳಿಗೆ ಉಡಿಸಿದ್ದಾಯಿತು. ಗಿರೀಶಬಾಬುಗಳು ಯಾವ ಅಡ್ಡಿಯನ್ನೂ ಮಾಡಲಿಲ್ಲ. ಗುರುಭ್ರಾತೃಗಳ ಇಚ್ಛೆಯಂತೆ ಅವರು ಇಂದು ಕೈಕಾಲುಗಳನ್ನು ಧಾರಾಳವಾಗಿ ನೀಡಿದರು. ಈಗ ಸ್ವಾಮೀಜಿ: “ಜಿ.ಸಿ. (ಗಿರೀಶಬಾಬುಗಳನ್ನು ಸ್ವಾಮಿಜಿಗಳು ಜಿ.ಸಿ, ಎಂದು ಕರೆಯುತ್ತಿದ್ದರು) ನೀನು ಈವೊತ್ತು ನಮಗೆ ಪರಮಹಂಸರ ವಿಚಾರವನ್ನು ಹೇಳಬೇಕು. (ಎಲ್ಲರ ಕಡೆಗೂ ನೋಡಿ) ಎಲ್ಲರೂ ಸುಮ್ಮನೆ ಸ್ಥಿರವಾಗಿ ಕುಳಿತುಕೊಳ್ಳಿ" ಎಂದರು. ಗಿರೀಶಬಾಬುಗಳಿಗೆ ಆಗಲೂ ಬಾಯಲ್ಲಿ ಮಾತಿಲ್ಲ. ಯಾರ ಜನ್ಮೋತ್ಸವಕ್ಕಾಗಿ ಇಂದು ಎಲ್ಲರೂ ಸೇರಿದ್ದರೋ ಅವರ ಲೀಲಾದರ್ಶನ ಮತ್ತು ಸಾಕ್ಷಾತ್ ಶಿಷ್ಯರ ಆನಂದ ದರ್ಶನ ಇವುಗಳಿಂದ ಉಂಟಾದ ಸುಖದಲ್ಲಿ ಜಡರಂತಾಗಿ ಬಿಟ್ಟಿದ್ದರು. ಕೊನೆಗೆ ಗಿರೀಶಬಾಬುಗಳು ಹೀಗೆಂದು ಹೇಳಿದರು: “ದಯಾಮಯರಾದ ಪರಮಹಂಸರ ವೃತ್ತಾಂತವನ್ನು ನಾನು ಮತ್ತೇನು ಹೇಳಲಿ? ಕಾಮಕಾಂಚನ ತ್ಯಾಗಿಗಳಾದ ನಿಮ್ಮಂಥ ಬಾಲಸಂನ್ಯಾಸಿಗಳ ಜೊತೆಯಲ್ಲಿ ಅವರು ಈ ಅಧಮನಿಗೆ ಏಕಾಸನದಲ್ಲಿ ಕುಳಿತುಕೊಳ್ಳುವುದಕ್ಕೆ ಅಧಿಕಾರವನ್ನು ಕೊಟ್ಟರಲ್ಲಾ, ಇದರಿಂದಲೇ ಅವರ ಅಪಾರ ಕರುಣೆಯನ್ನು ಅನುಭವ ಮಾಡಿಕೊಂಡಿದ್ದೇನೆ." ಈ ಮಾತನ್ನು ಹೇಳುತ್ತ ಹೇಳುತ್ತ ಗಿರೀಶಬಾಬುಗಳಿಗೆ ಗದ್ಗದ ಸ್ವರದಿಂದ ಮಾತು ಹೊರಡದಂತಾಯಿತು; ಅವರು ಆ ದಿವಸ ಮತ್ತೇನನ್ನೂ ಹೇಳಲಾರದೆ ಹೋದರು.

ಆಮೇಲೆ ಸ್ವಾಮೀಜಿ ಕೆಲವು ಹಿಂದೀ ಹಾಡುಗಳನ್ನು ಹಾಡಿದರು. ಶಿಷ್ಯನು ಸಂಗೀತ ವಿದ್ಯೆಯಲ್ಲಿ ಮಹಾಪಂಡಿತ! ಆದ್ದರಿಂದ ಈ ಗಾನಗಳಲ್ಲಿ ಒಂದಕ್ಷರವೂ ಅವನಿಗೆ ಗೊತ್ತಾಗಲಿಲ್ಲ. ಕೇವಲ ಸ್ವಾಮಿಜಿಯವರ ಮುಖವನ್ನು ಎವೆಯಿಕ್ಕದೆ ನೋಡುತ್ತಿದ್ದನು. ಈ ಸಮಯದಲ್ಲಿ ಮೊದಲನೆಯ ಪೂಜೆ ಮುಗಿಯಲು ಭಕ್ತರನ್ನು ಫಲಾಹಾರಕ್ಕೆ ಕರೆದರು. ಫಲಾಹಾರ ಮುಗಿದನಂತರ ಸ್ವಾಮೀಜಿ ಕೆಳಗಿನ ಬೈಠಕ್ ಖಾನೆಗೆ ಹೋಗಿ ಕುಳಿತುಕೊಂಡರು. ಬಂದಿದ್ದ ಭಕ್ತರೂ ಅವರನ್ನು ಸುತ್ತಿಕೊಂಡು ಕುಳಿತರು. ಜನಿವಾರವನ್ನು ಹಾಕಿಕೊಂಡು ಕುಳಿತಿದ್ದ ಒಬ್ಬ ಗೃಹಸ್ಥನನ್ನು ಸಂಬೋಧಿಸಿ ಸ್ವಾಮೀಜಿ “ನೀವು ದ್ವಿಜರು; ಆದರೆ ಬಹುಕಾಲದ ಹಿಂದಿನಿಂದ ವ್ರಾತ್ಯರಾಗಿದ್ದೀರಿ; ಇಂದಿನಿಂದ ಮತ್ತೆ ದ್ವಿಜರಾದಿರಿ; ಪ್ರತಿನಿತ್ಯವೂ ನೂರು ಗಾಯತ್ರಿ ಜಪವನ್ನು ಮಾಡಿ; ತಿಳಿಯಿತೆ?” ಎಂದರು. ಗೃಹಸ್ಥನು “ಅಪ್ಪಣೆ" ಎಂದು ಸ್ವಾಮೀಜಿಯವರ ಆಜ್ಞೆಯನ್ನು ಶಿರಸಾವಹಿಸಿದನು.

ಈ ಮಧ್ಯೆ ಶ‍್ರೀಯುತ ಮಹೇಂದ್ರನಾಥ ಗುಪ್ತರು (ಮಾಸ್ಟರ್ ಮಹಾಶಯ) ಬಂದರು. ಸ್ವಾಮೀಜಿ ಮಾಸ್ಟರ್ ಮಹಾಶಯರನ್ನು ನೋಡಿ ಅವರನ್ನು ನಾನಾ ಸಾದರ ಸಂಭಾಷಣಗಳಿಂದ ಸಂತೋಷಗೊಳಿಸಿದರು. ಮಹೇಂದ್ರ ಬಾಬುಗಳು ನಮಸ್ಕರಿಸಿ ಒಂದು ಮೂಲೆಯಲ್ಲಿ ನಿಂತುಕೊಂಡರು. ಸ್ವಾಮೀಜಿ ಕುಳಿತುಕೊಳ್ಳಬೇಕೆಂದು ಮೇಲಿಂದ ಮೇಲೆ ಹೇಳಿದ್ದರಿಂದ ಬಹು ಸಂಕೋಚ ಪಟ್ಟುಕೊಂಡು ಒಂದು ಮೂಲೆಯಲ್ಲಿ ಕುಳಿತುಕೊಂಡರು.

ಸ್ವಾಮೀಜಿ: ಮಾಸ್ಟರ್ ಮಹಾಶಯ, ಇವೊತ್ತು ಪರಮಹಂಸರ ಜನ್ಮದಿನ; ಪರಮಹಂಸರ ವೃತ್ತಾಂತವನ್ನು ನಮಗೆ ಏನಾದರೂ ತಿಳಿಸಬೇಕು.

ಮಾಸ್ಟರ್ ಮಹಾಶಯರು ಮೃದುವಾಗಿ ನಗುತ್ತ ತಲೆತಗ್ಗಿಸಿಕೊಂಡು ಇದ್ದರು. ಇಷ್ಟರಲ್ಲಿ ಅಖಂಡಾನಂದ ಸ್ವಾಮಿಗಳು ಮುರ್ಷಿದಾಬಾದಿನಿಂದ ಸುಮಾರು ಒಂದೂವರೆ ಮಣ ತೂಕವುಳ್ಳ ಎರಡು ಪಂತವ (ಒಂದು ಬಗೆ ಮಿಠಾಯಿ)ವನ್ನು ತೆಗೆದುಕೊಂಡು ಬಂದರು. ಈ ಅದ್ಭುತವಾದ ಮಿಠಾಯಿಯನ್ನು ನೋಡುವುದಕ್ಕೆಂದು ಎಲ್ಲರೂ ಎದ್ದರು. ಸ್ವಾಮೀಜಿ ಮುಂತಾದವರಿಗೆ ಅದನ್ನು ತೋರಿಸಿದ ಮೇಲೆ, ಸ್ವಾಮೀಜಿ “ದೇವರ ಮನೆಗೆ ನೈವೇದ್ಯಕ್ಕೆ ತೆಗೆದುಕೊಂಡು ಹೋಗು" ಎಂದರು.

ಅಖಂಡಾನಂದ ಸ್ವಾಮಿಗಳ ಕಡೆಗೆ ನೋಡುತ್ತ ಸ್ವಾಮೀಜಿ ಶಿಷ್ಯನಿಗೆ ಹೀಗೆಂದು ಹೇಳಿದರು: “ನೋಡಿದೆಯೊ ಎಂಥ ಕರ್ಮವೀರ! ಭಯ ಮೃತ್ಯು - ಇದೊಂದರ ಜ್ಞಾನವೂ ಇಲ್ಲ; ಒಂದೇ ಸಮನಾಗಿ ಕರ್ಮ ಮಾಡಿಕೊಂಡು ಹೋಗುತ್ತಿದ್ದಾನೆ - ಬಹುಜನ ಹಿತಾಯ ಬಹುಜನ ಸುಖಾಯ."

ಶಿಷ್ಯ: ಮಹಾಶಯರೆ, ಎಷ್ಟೋ ತಪಸ್ಸಿನ ಪ್ರಭಾವದಿಂದ ಅವರಲ್ಲಿ ಈ ಶಕ್ತಿ ಬಂದಿದೆ.

ಸ್ವಾಮೀಜಿ: ತಪಸ್ಸಿನ ಫಲದಿಂದ ಶಕ್ತಿ ಬರುತ್ತದೆ; ಅದರಂತೆ ಸೇವಾಭಾವದಿಂದ ಕರ್ಮ ಮಾಡಿದರೂ ಅದು ತಪಸ್ಸಾಗುತ್ತದೆ. ಕರ್ಮಯೋಗಿ ಕರ್ಮವನ್ನು ತಪಸ್ಸಿನ ಅಂಗವೆಂದೇ ಹೇಳುವನು. ತಪಸ್ಸನ್ನು ಮಾಡುತ್ತ ಇರುವಾಗ ಪರಹಿತೇಚ್ಛೆಯು ಬಲವತ್ತರವಾಗುತ್ತ ಸಾಧಕನ ಕೈಯಲ್ಲಿ ಕರ್ಮ ಮಾಡಿಸುತ್ತ ಹೋಗುತ್ತದೆ. ಪರಾರ್ಥವಾಗಿ ಕೆಲಸ ಮಾಡುತ್ತ ಅಷ್ಟಷ್ಟು ಚಿತ್ತಶುದ್ಧಿಯೂ ಪರಮಾತ್ಮನ ದರ್ಶನವೂ ಲಭಿಸುತ್ತದೆ.

ಶಿಷ್ಯ: ಆದರೆ; ಮಹಾಶಯರೆ, ಮೊದಲಿನಿಂದಲೂ ಅನ್ಯರಿಗಾಗಿ ಪ್ರಾಣವನ್ನಾದರೂ ಕೊಡುವಂತೆ ಎಷ್ಟು ಜನರು ಕೆಲಸಮಾಡಬಲ್ಲರು? ಜೀವನು ಸ್ವಸುಖೇಚ್ಛೆಯನ್ನು ಬಲಿಕೊಟ್ಟು ಪರಾರ್ಥವಾಗಿ ಪ್ರಾಣ ಕೊಡುವಂತಾಗುವಷ್ಟು ಉದಾರತೆ ಹೇಗೆ ಮನಸ್ಸಿಗೆ ಉಂಟಾದೀತು?

ಸ್ವಾಮಿಜಿ: ತಪಸ್ಸಿನ ಮೇಲೆ ತಾನೇ ಎಷ್ಟು ಜನಕ್ಕೆ ಮನಸ್ಸು ಹೋಗುತ್ತದೆ? ಕಾಮಕಾಂಚನದ ಆಕರ್ಷಣದಲ್ಲಿ ಎಷ್ಟು ಜನರು ತಾನೇ ಭಗವಂತನನ್ನು ಅಪೇಕ್ಷಿಸುವರು? ತಪಸ್ಸು ಎಷ್ಟು ಕಠಿಣವೊ ನಿಷ್ಕಾಮ ಕರ್ಮವೂ ಅಷ್ಟೇ ಕಠಿಣ. ಆದ್ದರಿಂದ ಯಾರು ಪರಹಿತಾರ್ಥವಾಗಿ ಕರ್ಮ ಮಾಡುವುದಕ್ಕೆ ಹೊರಡುತ್ತಾರೆಯೋ ಅವರಿಗೆ ವಿರೋಧವಾಗಿ ನೀನು ಏನು ಹೇಳುವುದಕ್ಕೂ ಅಧಿಕಾರವಿಲ್ಲ. ನಿನಗೆ ತಪಸ್ಸು ಒಗ್ಗಿದರೆ ಅದನ್ನು ಮಾಡುತ್ತಾ ಹೋಗು; ಮತ್ತೊಬ್ಬನಿಗೆ ಕರ್ಮವು ಒಗ್ಗುತ್ತದೆ - ಅವನಿಗೆ ಬೇಡವೆಂದು ಹೇಳುವುದಕ್ಕೆ ನಿನಗೇನು ಅಧಿಕಾರ? ಕರ್ಮ ತಪಸ್ಸಲ್ಲವೆಂಬುದು ನಿನ್ನ ದೃಢವಾದ ಅಭಿಪ್ರಾಯ. ನಾನು ಬಲ್ಲೆ.

ಶಿಷ್ಯ: ಅಪ್ಪಣೆ; ಆದರೆ ಹಿಂದೆ ತಪಸ್ಸನ್ನು ನಾನು ಬೇರೊಂದು ವಿಧವಾಗಿ ಅರ್ಥ ಮಾಡಿಕೊಂಡಿದ್ದೆ.

ಸ್ವಾಮಿಜಿ: ಸಾಧನ ಭಜನಗಳನ್ನು ಅಭ್ಯಾಸ ಮಾಡುತ್ತ ಮಾಡುತ್ತ ಅದರಲ್ಲಿ ಹೇಗೆ ಪ್ರಬಲವಾದ ಪ್ರವೃತ್ತಿ ಹುಟ್ಟುತ್ತದೆಯೋ, ಹಾಗೆಯೆ ಇಚ್ಛೆಯಿಲ್ಲದೆ ಇದ್ದರೂ ಕರ್ಮವನ್ನು ಮಾಡುತ್ತ ಮಾಡುತ್ತ ಹೋದರೆ ಮನಸ್ಸು ಕ್ರಮೇಣ ಅದರಲ್ಲಿಯೇ ಮುಳುಗಿಹೋಗುವುದು. ಕ್ರಮವಾಗಿ ಪರಾರ್ಥ ಕರ್ಮದಲ್ಲಿ ಪ್ರವೃತ್ತಿಯುಂಟಾಗುತ್ತದೆ; ತಿಳಿಯಿತೆ? ಇಷ್ಟವಿಲ್ಲದಿದ್ದರೂ ಒಂದು ಸಾರಿ ಪರರ ಸೇವೆ ಮಾಡಿನೋಡು. ತಪಸ್ಸಿನ ಫಲ ಬರುತ್ತದೆಯೋ ಇಲ್ಲವೋ ನೋಡು. ಪರಾರ್ಥವಾದ ಕರ್ಮದ ಫಲದಿಂದ ಮನಸ್ಸಿನ ಡೊಂಕುಪಂಕುಗಳೆಲ್ಲಾ ನೆಟ್ಟಗಾಗಿ ಮನುಷ್ಯನು ಕಪಟರಹಿತನಾಗಿ ಪರಹಿತಕ್ಕಾಗಿ ಜೀವವನ್ನು ತೆಯ್ಯುವುದಕ್ಕೆ ಹೊರಡುವನು.

ಶಿಷ್ಯ: ಆದರೆ, ಮಹಾಶಯರೆ, ಪರಹಿತ ಏತಕ್ಕೆ ಆಗಬೇಕು?

ಸ್ವಾಮೀಜಿ: ತನ್ನ ಹಿತಕ್ಕೋಸ್ಕರ, ಯಾವ ಈ ದೇಹದಲ್ಲಿ ‘ನಾನು’ ಎಂಬ ಅಭಿಮಾನವನ್ನಿಟ್ಟುಕೊಂಡು ಕುಳಿತಿದ್ದೀಯೊ ಆ ದೇಹವನ್ನು ಪರರಿಗೋಸ್ಕರ ತೆಗೆದಿಟ್ಟಿದ್ದೇನೆ ಎಂಬ ವಿಷಯವನ್ನು ಭಾವಿಸಿಕೊಳ್ಳುತ್ತ ಹೋದರೆ ಈ ಅಹಂಕಾರವನ್ನು ಮರೆಯಬೇಕಾಗುತ್ತದೆ. ಕಡೆಯಲ್ಲಿ ವಿದೇಹಬುದ್ಧಿ ಬರುತ್ತದೆ. ನೀನು ಎಷ್ಟೆಷ್ಟು ಏಕಾಗ್ರತೆಯಿಂದ ಪರರ ವಿಚಾರವನ್ನು ಭಾವಿಸುತ್ತೀಯೋ ಅಷ್ಟಷ್ಟು ನಿನ್ನನ್ನು ಮರೆಯುತ್ತ ಹೋಗುತ್ತೀಯೆ. ಹೀಗೆ ಕರ್ಮದಿಂದ ಚಿತ್ತಶುದ್ಧಿಯುಂಟಾದಾಗ, ನಿನ್ನಲ್ಲಿರುವ ಆತ್ಮವೇ ಸಮಸ್ತ ಜೀವರಾಶಿಯಲ್ಲಿಯೂ ಸಮಸ್ತ ವಸ್ತುಗಳಲ್ಲಿಯೂ ವಿರಾಜಿಸುತ್ತದೆ ಎಂಬ ತತ್ತ್ವವನ್ನು ಅರಿಯಲು ಶಕ್ತನಾಗುವೆ. ಆದ್ದರಿಂದಲೆ, ತನ್ನ ಆತ್ಮವಿಕಾಸಕ್ಕೆ ಪರಹಿತ ಸಾಧನೆಯು ಒಂದು ಮಾರ್ಗ, ಇದೂ ಒಂದು ವಿಧವಾದ ಸಾಧನೆ ಎಂಬುದು ಗೊತ್ತಾಗುತ್ತದೆ. ಇದರ ಉದ್ದೇಶವೂ ಆತ್ಮವಿಕಾಸವೆ. ಜ್ಞಾನ ಭಕ್ತಿ ಮುಂತಾದ ಸಾಧನೆಗಳಿಂದ ಹೇಗೆ ಆತ್ಮವಿಕಾಸವಾಗುತ್ತದೆಯೋ ಪರಾರ್ಥ ಕರ್ಮ ಮಾಡುವುದರಿಂದಲೂ ಹಾಗೆಯೆ ಆಗುತ್ತದೆ.

ಶಿಷ್ಯ: ಆದರೆ ಮಹಾಶಯರೆ, ನಾನು ಹಗಲೂ ರಾತ್ರಿಯೂ ಪರರ ಚಿಂತೆಯನ್ನೇ ಹಚಿಕೊಂಡರೆ ಆತ್ಮ ವಿಚಾರವನ್ನು ಮಾಡುವುದು ಯಾವಾಗ? ಒಂದು ವಿಶೇಷ ಭಾವವನ್ನು ಹಿಡಿದುಕೊಂಡು ಕುಳಿತರೆ ಅಭಾವರೂಪಿಯಾದ ಆತ್ಮವನ್ನು ಸಾಕ್ಷಾತ್ಕಾರ ಮಾಡಿಕೊಳ್ಳುವುದು ಹೇಗೆ?

ಸ್ವಾಮಿಜಿ: ಆತ್ಮಜ್ಞಾನ ಲಾಭವೆ ಸಕಲ ಸಾಧನೆಯ, ಸಕಲ ಮಾರ್ಗದ ಮುಖ್ಯ ಉದ್ದೇಶ. ನೀನು ಸೇವಾಪರನಾಗಿ ಈ ಕರ್ಮಫಲದಿಂದ ಚಿತ್ತಶುದ್ಧಿಯನ್ನು ಪಡೆದು, ಸರ್ವ ಜೀವಿಗಳನ್ನೂ ಆತ್ಮದ ಹಾಗೆ ನೋಡಬಲ್ಲೆಯಾದರೆ ಆತ್ಮದರ್ಶನದಲ್ಲಿ ಇನ್ನು ಉಳಿದದ್ದು ಏನು? ಆತ್ಮದರ್ಶನವೆಂದರೇನು ಜಡವಸ್ತುವಿನ ಹಾಗೆ - ಈ ಗೋಡೆಯ ಹಾಗೆ ಅಥವಾ ಮರದ ತುಂಡಿನ ಹಾಗೆ - ಬಿದ್ದಿರುವುದೆಂದು ತಿಳಿದುಕೊಂಡೆಯೋ?

ಶಿಷ್ಯ: ಹಾಗಲ್ಲದಿದ್ದರೂ, ಸರ್ವವೃತ್ತಿ ಮತ್ತು ಕರ್ಮದ ನಿರೋಧದಿಂದ ತಾನೆ ಆತ್ಮದ ಸ್ವಸ್ವರೂಪಲಾಭವೆಂದು ಶಾಸ್ತ್ರ ಹೇಳುವುದು?

ಸ್ವಾಮೀಜಿ: ಶಾಸ್ತ್ರದಲ್ಲಿ ಯಾವುದನ್ನು ಸಮಾಧಿಯೆಂದು ಹೇಳಿದೆಯೋ ಆ ಸ್ಥಿತಿ ಸುಲಭವಾಗಿ ದೊರೆಯುವುದಿಲ್ಲ. ಯಾವಾಗಲಾದರೂ ಯಾರಿಗಾದರೂ ದೊರೆತರೂ ಹೆಚ್ಚುಕಾಲ ಸ್ಥಾಯಿಯಾಗಿ ಇರುವುದಿಲ್ಲ. ಆಗ ಅವನು ಯಾವುದರ ಆಧಾರದ ಮೇಲೆ ಇರಬೇಕು ಹೇಳು? ಆದ್ದರಿಂದ ಶಾಸ್ತ್ರೋಕ್ತವಾದ ಸ್ಥಿತಿ ಬಂದನಂತರ ಸಾಧಕನು ಪ್ರತಿ ಪ್ರಾಣಿಯಲ್ಲಿಯೂ ಆತ್ಮವನ್ನು ಕಾಣುತ್ತಾನೆ, ಮತ್ತು ಅಭೇದ ಜ್ಞಾನದಿಂದ ಸೇವೆಮಾಡುತ್ತ ಪ್ರಾರಬ್ಧ ಕರ್ಮವನ್ನು ಸವೆಸುತ್ತಾನೆ. ಈ ಅವಸ್ಥೆಗೆ ಶಾಸ್ತ್ರಕಾರರು ಜೀವನ್ಮುಕ್ತಿ ಅವಸ್ಥೆ ಎಂದು ಹೇಳುತ್ತಾರೆ.

ಶಿಷ್ಯ: ಹಾಗಾದರೆ, ಜೀವನ್ಮುಕ್ತಿ ಅವಸ್ಥೆಯನ್ನು ಪಡೆಯದಿದ್ದರೆ, ಸರಿಯಾಗಿ ಪರಾರ್ಥವನ್ನು ಮಾಡುವುದಕ್ಕಾಗುವುದಿಲ್ಲವೆಂದು ಹೇಳಿದ ಹಾಗಾಯಿತು.

ಸ್ವಾಮೀಜಿ: ಶಾಸ್ತ್ರದಲ್ಲಿ ಈ ಮಾತು ಹೇಳಿದೆ; ಇನ್ನೂ ಹೇಳಿರುವುದೇನೆಂದರೆ ಪರಾರ್ಥವಾಗಿ ಸೇವೆಮಾಡುತ್ತ ಮಾಡುತ್ತ ಸಾಧಕನಿಗೆ ಜೀವನ್ಮುಕ್ತಿಯ ಅವಸ್ಥೆಯು ಬರುತ್ತದೆ; ಹಾಗಲ್ಲದಿದ್ದರೆ ‘ಕರ್ಮಯೋಗ’ವೆಂದು ಬೇರೊಂದು ಮಾರ್ಗವನ್ನು ಶಾಸ್ತ್ರವು ಉಪದೇಶ ಮಾಡಬೇಕಾಗಿರಲಿಲ್ಲ.

ಶಿಷ್ಯನು ಇಷ್ಟು ಹೊತ್ತಿಗೆ ತಿಳಿದುಕೊಂಡು ಸಮಾಧಾನಗೊಂಡನು. ಸ್ವಾಮಿಗಳೂ ಈ ವಿಚಾರವನ್ನು ಬಿಟ್ಟು, ಕಿನ್ನರಕಂಠದಲ್ಲಿ ಹಾಡಲು ಆರಂಭಿಸಿದರು.

\newpage

\chapter[ಅಧ್ಯಾಯ ೧೪]{ಅಧ್ಯಾಯ ೧೪\protect\footnote{\engfoot{C.W, Vol. VII, P. 113}}}

\begin{center}
ಸ್ಥಳ: ಬೇಲೂರು ಮಠ (ಬಾಡಿಗೆ ಕಟ್ಟಡ), ವರ್ಷ: ೧೮೯೮.
\end{center}

ಇಂದು ಹೊಸಮಠದ ನಿವೇಶನದಲ್ಲಿ ಸ್ವಾಮೀಜಿ ಹೋಮಮಾಡಿ ಪರಮಹಂಸ ದೇವರ ಪ್ರತಿಷ್ಠೆಯನ್ನು ಮಾಡುವರು. ಶಿಷ್ಯನು ಹಿಂದಿನ ರಾತ್ರಿಯಿಂದ ಮಠದಲ್ಲಿದ್ದಾನೆ. ಪರಮಹಂಸ ದೇವರ ಪ್ರತಿಷ್ಠೆ ಮಾಡುವುದನ್ನು ನೋಡಬೇಕೆಂಬ ಆಸೆ.

ಬೆಳಿಗ್ಗೆ ಗಂಗಾಸ್ನಾನಮಾಡಿ ಸ್ವಾಮೀಜಿ ದೇವರಮನೆಗೆ ಹೋದರು. ಅನಂತರದಲ್ಲಿ ಪೂಜಾಸನದಲ್ಲಿ ಕುಳಿತುಕೊಂಡು ಹೂವಿನ ಬುಟ್ಟಿಯಲ್ಲಿ ಎಷ್ಟು ಪತ್ರೆಗಳಿದ್ದುವೊ ಎಲ್ಲವನ್ನೂ ಒಟ್ಟಿಗೆ ಎರಡು ಕೈಗಳಿಂದಲೂ ತೆಗೆದುಕೊಂಡು ಶ‍್ರೀರಾಮಕೃಷ್ಣರ ಪಾದುಕೆಯ ಮೇಲೆ ಇರಿಸಿ ಧ್ಯಾನಸ್ಥರಾದರು - ಅಪೂರ್ವದರ್ಶನ! ಅವರ ಧರ್ಮ-ಪ್ರಭಾ-ವಿಭಾಸಿತವಾದ ಸ್ನಿಗ್ಧೋಜ್ವಲ ತೇಜಸ್ಸು ದೇವರಮನೆಯನ್ನು ಒಂದು ಅದ್ಭುತವಾದ ಪ್ರಕಾಶದಿಂದ ತುಂಬಿತು! ಪ್ರೇಮಾನಂದರು ಮತ್ತು ಇತರ ಸ್ವಾಮಿಗಳು ಪೂಜಾಗೃಹದ ಬಾಗಿಲಲ್ಲಿ ನಿಂತುಕೊಂಡಿದ್ದರು.

ಧ್ಯಾನ, ಪೂಜೆಗಳಾದ ಮೇಲೆ, ಮಠದ ನಿವೇಶನಕ್ಕೆ ಹೊರಡಲು ಎಲ್ಲವೂ ಸಿದ್ಧವಾಯಿತು. ಶ‍್ರೀರಾಮಕೃಷ್ಣರ ಭಸ್ಮಾಸ್ಥಿಗಳನ್ನು ಇಟ್ಟಿದ್ದ ತಾಮ್ರ ಸಂಪುಟವನ್ನು ಸ್ವಾಮೀಜಿ ತಾವೇ ಬಲಭುಜದ ಮೇಲೆ ಇಟ್ಟುಕೊಂಡು ಮುಂದೆ ಹೊರಟರು. ಇತರ ಸಂನ್ಯಾಸಿಗಳೊಡನೆ ಶಿಷ್ಯನು ಹಿಂದೆ ಹಿಂದೆ ಹೋದನು. ಶಂಖ ಘಂಟೆ ಇವುಗಳ ಧ್ವನಿಯಿಂದ ಗಂಗಾ ತಟಪ್ರದೇಶವು ಅದಿರುತ್ತಿರಲು ಗಂಗಾನದಿಯು ಹಾವಭಾವಗಳನ್ನು ತೋರಿಸುತ್ತ ನರ್ತನ ಮಾಡುವಂತಿತ್ತು. ಹೋಗುತ್ತ ಹೋಗುತ್ತ ದಾರಿಯಲ್ಲಿ ಸ್ವಾಮೀಜಿ ಶಿಷ್ಯನನ್ನು ಕುರಿತು ಹೀಗೆಂದರು: ಪರಮಹಂಸರು ‘ನೀನು ನನ್ನನ್ನು ಭುಜದಮೇಲೆ ಇಟ್ಟುಕೊಂಡು ಎಲ್ಲಿಗೆ ಕರೆದುಕೊಂಡು ಹೋದರೆ ಅಲ್ಲಿಗೆ ನಾನು ಬಂದುಬಿಡುತ್ತೇನೆ - ಅದು ಮರದ ಕೆಳಗೇ ಆಗಿರಲಿ, ಒಂದು ಗುಡಿಸಲೇ ಆಗಿರಲಿ, ಚಿಂತೆಯಿಲ್ಲ’ ಎಂದು ನನಗೆ ಹೇಳಿದ್ದಾರೆ. ಅದಕ್ಕೋಸ್ಕರವೇ ಇವೊತ್ತು ನಾನೇ ಅವರನ್ನು (ಅವರ ಭಸ್ಮಾಸ್ಥಿಗಳನ್ನು) ಹೆಗಲಿನ ಮೇಲೆ ಕೂರಿಸಿಕೊಂಡು ಹೊಸಮಠ ಪ್ರದೇಶಕ್ಕೆ ಹೋಗುತ್ತೇನೆ. ಇದನ್ನು ಖಂಡಿತವಾಗಿ ತಿಳಿದುಕೊ, ಬಹುಕಾಲದವರೆಗೆ ‘ಬಹುಜನರ ಹಿತಸಾಧನೆಗಾಗಿ’ ಪರಮಹಂಸರು ಈ ಸ್ಥಾನದಲ್ಲಿ ನಿಶ್ಚಿಂತೆಯಾಗಿ ನಿಲ್ಲುವರು.

ಶಿಷ್ಯ: ಪರಮಹಂಸರು ತಮಗೆ ಈ ಮಾತನ್ನು ಯಾವಾಗ ಹೇಳಿದರು?

ಸ್ವಾಮೀಜಿ: (ಮಠದ ಸಾಧುಗಳನ್ನು ತೋರಿಸಿ) ಇವರ ಬಾಯಿಂದ ಕೇಳಲಿಲ್ಲವೇನು? - ಕಾಶೀಪುರದ ತೋಟದಲ್ಲಿ.

ಶಿಷ್ಯ: ಓಹೋ! ಆಗಲೇ ಪರಮಹಂಸರ ಗೃಹಸ್ಥ ಮತ್ತು ಸಂನ್ಯಾಸಿ ಭಕ್ತರ ಮಧ್ಯೆ ಸೇವಾಧಿಕಾರ ವಿಚಾರವಾಗಿ ಬೇರೆ ಬೇರೆ ಗುಂಪುಗಳಾದುವೆಂದು ತೋರುತ್ತದೆ.

ಸ್ವಾಮೀಜಿ: ಹೌದು; ಆದರೆ ‘ಗುಂಪುಗಳು’ ಎಂಬುದು ನಿಜವಲ್ಲ. ಎಲ್ಲೋ ಸ್ವಲ್ಪ ಮನಸ್ತಾಪ ಬಂದಿತ್ತು. ಇದನ್ನು ತಿಳಿದುಕೊ - ಯಾರು ಪರಮಹಂಸರ ಭಕ್ತರೊ, ಯಾರು ನಿಜವಾಗಿ ಅವರ ಕೃಪೆಯನ್ನು ಪಡೆದಿರುತ್ತಾರೊ, (ಅವರು ಗೃಹಸ್ಥರೇ ಆಗಿರಲಿ, ಸಂನ್ಯಾಸಿಗಳೇ ಆಗಿರಲಿ) ಅವರಲ್ಲಿ ಗುಂಪು ಪಂಗಡ ಯಾವುದೂ ಇಲ್ಲ. ಇರುವುದಕ್ಕೆ ಆಗುವುದಿಲ್ಲ. ಆದರೆ ಈ ಅಲ್ಪ ಸ್ವಲ್ಪ ಮನಸ್ತಾಪಕ್ಕೆ ಕಾರಣವೇನು ಬಲ್ಲೆಯಾ? ಪ್ರತಿಯೊಬ್ಬ ಭಕ್ತನೂ ಪರಮಹಂಸರಿಗೆ ತಮ್ಮ ತಮ್ಮ ಬಣ್ಣವನ್ನು ಕಟ್ಟಿ ಒಬ್ಬೊಬ್ಬನೂ ಅವರನ್ನು ಒಂದೊಂದು ವಿಧವಾಗಿ ನೋಡಿ ಒಬ್ಬೊಬ್ಬರೂ ಒಂದೊಂದು ರೀತಿ ತಿಳಿದುಕೊಳ್ಳುತ್ತಾರೆ. ಅವರು ಒಂದು ದೊಡ್ಡ ಸೂರ್ಯನಂತಿದ್ದಾರೆ; ನಾವು ಒಬ್ಬೊಬ್ಬರೂ ಒಂದೊಂದು ಬಣ್ಣದ ಗಾಜನ್ನು ಕಣ್ಣಿಗೆ ಕಟ್ಟಿಕೊಂಡು ಆ ಒಬ್ಬ ಸೂರ್ಯನನ್ನೇ ನಾನಾ ಬಣ್ಣಗಳಿಂದ ಕೂಡಿರುವವನೆಂದು ಹೇಳುತ್ತಿರುವಂತಿದ್ದೇವೆ. ಆದ್ದರಿಂದ ಇದೂ ನಿಜ. ಏಕೆಂದರೆ, ಕಾಲಕ್ರಮದಲ್ಲಿ ಇಲ್ಲಿಂದಲೂ ಒಂದು ಪಂಗಡ ನಿರ್ಮಿತವಾಗುತ್ತದೆ. ಆದರೆ ಯಾರು ಭಾಗ್ಯವಶದಿಂದ ಅವತಾರ ಪುರುಷರ ಸಾಕ್ಷಾತ್ ಸಂಪರ್ಕದಲ್ಲಿರುವರೊ ಅವರ ಜೀವನದೆಸೆಯಲ್ಲಿ ಈ ವಿಧವಾದ ‘ಗುಂಪು’ ಸಾಮಾನ್ಯವಾಗಿ ಆಗುವುದಿಲ್ಲ. ಆ ಆತ್ಮಾರಾಮಪುರುಷರ ಬೆಳಕಿನಲ್ಲಿ, ಅವರ ಕಣ್ಣು ಸ್ಪಷ್ಟವಾಗಿ ಕಾಣುತ್ತದೆ. ಅಹಂಕಾರ, ಅಭಿಮಾನ, ಹೀನಬುದ್ಧಿ ಎಲ್ಲಾ ಕೊಚ್ಚಿಹೋಗುತ್ತವೆ. ಆದ್ದರಿಂದ ‘ಗುಂಪು’ ಮಾಡಿಕೊಳ್ಳುವುದಕ್ಕೆ ಅವರಿಗೆ ಅವಕಾಶವಿರುವುದಿಲ್ಲ. ಕೇವಲ ತಮ್ಮ ತಮ್ಮ ಭಾವದಿಂದ ಅವರನ್ನು ಮನಃಪೂರ್ವಕವಾಗಿ ಪೂಜಿಸುವರು.

ಶಿಷ್ಯ: ಮಹಾಶಯರೆ, ಹಾಗಾದರೇನು ಪರಮಹಂಸರ ಭಕ್ತರೆಲ್ಲರೂ ಅವರನ್ನು ದೇವರೆಂದು ತಿಳಿದುಕೊಂಡಿದ್ದರೂ ಆ ಒಂದೇ ಭಗವಂತನ ಸ್ವರೂಪವನ್ನು ಅವರು ಬೇರೆ ಬೇರೆ ಭಾವದಲ್ಲಿ ನೋಡುವುದರಿಂದ ಅವರ ಶಿಷ್ಯ ಪ್ರಶಿಷ್ಯರು ಕ್ರಮೇಣ ಒಂದು ಇಕ್ಕಟ್ಟಾದ ಎಲೆಯೊಳಗೆ ಬಿದ್ದು ಸಣ್ಣ ಗುಂಪು ಅಥವಾ ಸಂಪ್ರದಾಯಗಳಾಗಿ ಪರಿಣಮಿಸುತ್ತಾರೆಯೆ?

ಸ್ವಾಮೀಜಿ: ಹೌದು; ಇದರಿಂದಲೇ ಕಾಲಕ್ರಮದಲ್ಲಿ ಸಂಪ್ರದಾಯವು ಆಗಿಯೇ ಆಗುತ್ತದೆ. ಇದನ್ನೇ ನೋಡು, ಈಗ ಚೈತನ್ಯ ದೇವನ ಎರಡು ಮೂರು ಸಂಪ್ರದಾಯಗಳಿವೆ; ಕ್ರಿಸ್ತನ ಸಾವಿರಾರು ಮತಗಳು ಹೊರಟಿವೆ. ಆದರೆ ಈ ಸಂಪ್ರದಾಯಗಳನ್ನೆಲ್ಲಾ ಚೈತನ್ಯದೇವನೂ ಕ್ರಿಸ್ತನೂ ಒಪ್ಪಿದ್ದರೆ?

ಶಿಷ್ಯ: ಹಾಗಾದರೆ ಶ‍್ರೀರಾಮಕೃಷ್ಣ ಪರಮಹಂಸರ ಭಕ್ತರಲ್ಲಿಯೂ ಕ್ರಮೇಣ ಬಹು ಸಂಪ್ರದಾಯಗಳು ಆಗುತ್ತವೆಂದು ತೋರುತ್ತದೆ.

ಸ್ವಾಮೀಜಿ: ಆಗದೆ ಇನ್ನೇನು? ಆದರೆ ನಮ್ಮ ಈ ಮಠವಿದೆಯಲ್ಲಾ ಇದರಿಂದ ಸಕಲಮತದ ಮತ್ತು ಭಾವಗಳ ಸಮನ್ವಯವು ನಿಂತಿರುತ್ತದೆ. ಇದು ಪರಮಹಂಸರ ಮತವು ಎಷ್ಟು ಉದಾರವಾಗಿತ್ತೋ ಅಷ್ಟೇ ಉದಾರವಾದ ಅದೇ ಭಾವಕ್ಕೆ ಕೇಂದ್ರಸ್ಥಾನವಾಗುತ್ತದೆ. ಈ ಸ್ಥಳದಿಂದ ಹೊರಟು ಉಕ್ಕಿ ಹರಿಯುವ ಸಮನ್ವಯದ ಪ್ರವಾಹದಿಂದ ಪ್ರಪಂಚವೆಲ್ಲಾ ತುಂಬಿಹೋಗುತ್ತದೆ.

ಹೀಗೆ ಮಾತುಕತೆಗಳು ನಡೆಯುತ್ತ ನಡೆಯುತ್ತ ಎಲ್ಲರೂ ಮಠದ ನಿವೇಶನಕ್ಕೆ ಬಂದರು. ಸ್ವಾಮೀಜಿ ಭುಜದ ಮೇಲಿದ್ದ ಸಂಪುಟವನ್ನು ಅಲ್ಲಿ ಹಾಕಿದ್ದ ಆಸನದ ಮೇಲೆ ಇಳಿಸಿ ನೆಲಮುಟ್ಟಿ ನಮಸ್ಕಾರ ಮಾಡಿದರು. ಮಿಕ್ಕವರೂ ನಮಸ್ಕಾರ ಮಾಡಿದರು.

ಆಮೇಲೆ ಸ್ವಾಮಿಜಿ ಪುನಃ ಪೂಜೆಗೆ ಕುಳಿತರು. ಪೂಜೆ ಮುಗಿದ ಮೇಲೆ ಹೋಮಾಗ್ನಿಯನ್ನು ಪ್ರಜ್ವಲಮಾಡಿ ಅದರಲ್ಲಿ ಹೋಮ ಮಾಡಿದರು; ಮತ್ತು ಸಂನ್ಯಾಸಿ ಭ್ರಾತೃಗಳ ಸಹಾಯದಿಂದ ಕೈಯಿಂದಲೇ ಪರಮಾನ್ನವನ್ನು ಮಾಡಿಸಿ ನೈವೇದ್ಯ ಮಾಡಿದರು. ಆವೊತ್ತು ಅಲ್ಲಿ ಅವರು ಕೆಲವು ಜನ ಗೃಹಸ್ಥರಿಗೆ ದೀಕ್ಷೆ ಕೊಟ್ಟರೆಂದು ತೋರುತ್ತದೆ. ಅದು ಹೇಗಾದರೂ ಇರಲಿ, ಪೂಜೆಯನ್ನು ಮುಗಿಸಿಕೊಂಡು ಸ್ವಾಮೀಜಿ ಅಲ್ಲಿದ್ದವರೆಲ್ಲರನ್ನೂ ಆದರದಿಂದ ಬರಮಾಡಿಕೊಂಡು ಅವರನ್ನು ಸಂಬೋಧಿಸಿ “ಮಹಾ ಯುಗಾವತಾರ ಪರಮಹಂಸರು ಇಂದಿನಿಂದ ಬಹುಕಾಲದವರೆಗೆ ‘ಬಹುಜನ ಹಿತಾಯ ಬಹುಜನ ಸುಖಾಯ’ ಈ ಪುಣ್ಯಕ್ಷೇತ್ರದಲ್ಲಿ ನಿಂತು ಇದನ್ನು ಸರ್ವಧರ್ಮದ ಅಪೂರ್ವ ಸಮನ್ವಯಕ್ಕೆ ಕೇಂದ್ರವನ್ನಾಗಿ ಮಾಡಬೇಕೆಂದು ತಾವೆಲ್ಲರೂ ಇವೊತ್ತು ಮನೋವಾಕ್ಕಾಯಗಳಿಂದ ಪರಮಹಂಸರ ಪಾದಪಾದ್ಮದಲ್ಲಿ ಪ್ರಾರ್ಥನೆ ಮಾಡಬೇಕು" ಎಂದರು. ಎಲ್ಲರೂ ಕೈಮುಗಿದುಕೊಂಡು ಹಾಗೆಯೇ ಪ್ರಾರ್ಥನೆ ಮಾಡಿದರು. ಪೂಜೆಯಾದ ಮೇಲೆ ಸ್ವಾಮಿಜಿ ಶಿಷ್ಯನನ್ನು ಕರೆದು “ಪರಮಹಂಸರ ಈ ಸಂಪುಟವನ್ನು ಹಿಂದಕ್ಕೆ ತೆಗೆದುಕೊಂಡು ಹೋಗಲು ನಮ್ಮಲ್ಲಿ (ಸಂನ್ಯಾಸಿಗಳಲ್ಲಿ) ಯಾರಿಗೂ ಅಧಿಕಾರವಿಲ್ಲ; ಏಕೆಂದರೆ ಇವೊತ್ತು ನಾನು ಪರಮಹಂಸರನ್ನು ಇಲ್ಲಿ ಕೂರಿಸಿದ್ದೇನೆ. ಆದ್ದರಿಂದ ನೀನೇ ಈ ಸಂಪುಟವನ್ನು ತಲೆಯ ಮೇಲೆ ಇಟ್ಟುಕೊಂಡು ಹೋಗು" ಎಂದು ಹೇಳಿದರು. ಶಿಷ್ಯನು ಈ ಸಂಪುಟವನ್ನು ಮುಟ್ಟುವುದಕ್ಕೆ ಹಿಂದುಮುಂದು ನೋಡುತ್ತಿರುವುದನ್ನು ನೋಡಿ “ಭಯವಿಲ್ಲ ಹಾಗೆ ಮಾಡು, ನಾನು ಆಜ್ಞೆ ಮಾಡಿದ್ದೇನೆ" ಎಂದರು. ಶಿಷ್ಯನು ಆಗ ಆನಂದಿತಚಿತ್ತನಾಗಿ ಸ್ವಾಮಿಜಿಯ ಆಜ್ಞೆಯನ್ನು ಶಿರಸಾವಹಿಸಿ ಸಂಪುಟವನ್ನು ತಲೆಯಮೇಲೆ ಹೊತ್ತುಕೊಂಡನು; ಮತ್ತು ಶ‍್ರೀಗುರುವಿನ ಆಜ್ಞೆಯಿಂದ ಈ ಸಂಪುಟವನ್ನು ಮುಟ್ಟುವುದಕ್ಕೆ ಅಧಿಕಾರವನ್ನು ಪಡೆದಿದ್ದಕ್ಕಾಗಿ ತಾನು ಧನ್ಯನೆಂದು ಭಾವಿಸಿಕೊಳ್ಳುತ್ತಾ ಹೋದನು. ಮುಂದೆ ಶಿಷ್ಯ, ಹಿಂದೆ ಸ್ವಾಮಿಗಳು, ಆಮೇಲೆ ಮಿಕ್ಕವರು ಹೀಗೆ ಎಲ್ಲರೂ ಹೊರಟರು. ದಾರಿಯಲ್ಲಿ ಅವನಿಗೆ “ಪರಮಹಂಸರು ಇವೊತ್ತು ನಿನ್ನ ತಲೆಯ ಮೇಲೆ ಹತ್ತಿ ನಿಂತು ನಿನಗೆ ಆಶೀರ್ವಾದ ಮಾಡಿದ್ದಾರೆ. ಹುಷಾರ್, ಇಂದಿನಿಂದ ಮತ್ತೆ ಯಾವುದಾದರೂ ಅನಿತ್ಯ ವಿಷಯದಲ್ಲಿ ಮನಸ್ಸು ಕೊಟ್ಟೀಯೆ" ಎಂದು ಹೇಳಿದರು ಸ್ವಾಮೀಜಿ. ಒಂದು ಸಣ್ಣ ಸೇತುವೆಯನ್ನು ದಾಟುವಷ್ಟರೊಳಗಾಗಿ ಪುನಃ ಸ್ವಾಮೀಜಿ ಶಿಷ್ಯನಿಗೆ “ನೋಡು ಇನ್ನು ಮೇಲೆ ಬಹಳ ಹುಷಾರ್! ತುಂಬ ವಿಚಾರಪೂರ್ವಕವಾಗಿ ನಡೆದುಕೊಳ್ಳಬೇಕು" ಎಂದು ಹೇಳಿದರು.

ಹೀಗೆ ನಿರ್ವಿಘ್ನವಾಗಿ ಮಠಕ್ಕೆ ಹೋಗಿ ಎಲ್ಲರೂ ಆನಂದದಲ್ಲಿದ್ದರು. ಸ್ವಾಮೀಜಿ ಈಗ ಶಿಷ್ಯನೊಡನೆ ಮಾತುಕತೆಗಳಿಗೆ ಆರಂಭಮಾಡಿ ಹೇಳಿದ್ದೇನೆಂದರೆ: “ಪರಮಹಂಸರ ಇಚ್ಛೆಯಿಂದ ಇವೊತ್ತು ಅವರ ಧರ್ಮ ಕ್ಷೇತ್ರದ ಪ್ರತಿಷ್ಠೆಯಾಯಿತು. ಹನ್ನೆರಡು ವರ್ಷದ ಚಿಂತಾಭಾರ ಕೆಳಗಿಳಿದಂತಾಯಿತು. ನನ್ನ ಮನಸ್ಸಿನಲ್ಲಿ ಈಗ ಏನಿದೆ ಬಲ್ಲೆಯಾ? ಈ ಮಠ ವಿದ್ಯೆ ಮತ್ತು ಸಾಧನಗಳಿಗೆ ಕೇಂದ್ರಸ್ಥಾನವಾಗುವುದು; ನಿಮ್ಮಂಥ ಧಾರ್ಮಿಕ ಗೃಹಸ್ಥರು ಅದರ ನಾಲ್ಕು ಕಡೆಯ ಸ್ಥಳದಲ್ಲಿಯೂ ಮನೆ ಮಠಗಳನ್ನು ಮಾಡಿ ಕೊಂಡಿರುವರು; ಅಲ್ಲದೆ ಮಠದ ದಕ್ಷಿಣಕ್ಕಿರುವ ಈ ಪ್ರದೇಶದಲ್ಲಿ ಇಂಗ್ಲೆಂಡ್ ಮತ್ತು ಅಮೆರಿಕಾ ದೇಶಗಳ ಭಕ್ತಾದಿಗಳು ಇರುವುದಕ್ಕೆ ಮನೆಯಾಗುವುದು. ಹೀಗಾದರೆ ಹೇಗಿರುತ್ತದೆ ಹೇಳು ನೋಡೋಣ?”

ಶಿಷ್ಯ: ಮಹಾಶಯರೆ, ತಮ್ಮ ಈ ಕಲ್ಪನೆಯು ಅದ್ಭುತವಾದದ್ದು.

ಸ್ವಾಮೀಜಿ: ಕಲ್ಪನೆಯೇನಿದೆಯಯ್ಯಾ? ಸಮಯದಲ್ಲಿ ಎಲ್ಲಾ ಆಗತಕ್ಕದ್ದೆ. ನಾನು ಸುಮ್ಮನೆ ಆರಂಭಮಾಡಿಕೊಡುತ್ತೇನೆ ಅಷ್ಟೆ; ಇದರ ಮೇಲೆ ಇನ್ನೂ ಎಷ್ಟೋ ಆಗಬೇಕಾಗಿದೆ! ನಾನು ಸ್ವಲ್ಪ ಮಾಡಿ ಹೋಗುತ್ತೇನೆ; ಮತ್ತು ನಿಮ್ಮಲ್ಲಿ ನಾನಾ ಅಭಿಪ್ರಾಯ ವಿಶೇಷಗಳನ್ನು ಬಿತ್ತಿ ಹೋಗುತ್ತೇನೆ. ನೀವು ಆಮೇಲೆ ಅವುಗಳನ್ನೆಲ್ಲಾ ಕಾರ್ಯದಲ್ಲಿ ಪರಿಣತವಾಗಿ ಮಾಡುವಿರಿ. ದೊಡ್ಡ ದೊಡ್ಡ ತತ್ತ್ವಗಳನ್ನು ಸುಮ್ಮನೆ ಕೇಳಿಬಿಟ್ಟರೆ ಆಗುವುದೇನು? ಅವುಗಳನ್ನು ಅನುಷ್ಠಾನದಲ್ಲಿ ತರಬೇಕು. ಶಾಸ್ತ್ರದ ಉದ್ದ ಉದ್ದವಾದ ಮಾತುಗಳನ್ನು ಸುಮ್ಮನೆ ಓದಿಬಿಟ್ಟರೆ ಬಂದದ್ದೇನು? ಶಾಸ್ತ್ರ ವಿಷಯಗಳನ್ನು ಮೊದಲು ತಿಳಿದುಕೊಳ್ಳಬೇಕು. ಆಮೇಲೆ ಜೀವನದಲ್ಲಿ ಅವುಗಳನ್ನು ಅನುಷ್ಠಾನಗೊಳಿಸಬೇಕು ತಿಳಿಯಿತೆ? ಇದಕ್ಕೆ ಅನುಷ್ಠಾನಕರ್ಮ ಎಂದು ಹೆಸರು.

ಹೀಗೆ ನಾನಾ ಪ್ರಸ್ತಾವವು ನಡೆಯುತ್ತಾ ಶ‍್ರೀಮದ್‌ಶಂಕರಾಚಾರ್ಯರ ಮಾತು ಬಂತು. ಶಿಷ್ಯನಿಗೆ ಶಂಕರಾಚಾರ್ಯರ ಮೇಲೆ ತುಂಬ ಅಭಿಮಾನ. ಅಷ್ಟೇ ಏಕೆ ಈ ವಿಷಯದಲ್ಲಿ ಅವನನ್ನು ಅತಿ ಪಕ್ಷಪಾತಿ ಎಂದು ಬೇಕಾದರೂ ಹೇಳಬಹುದು. ಅವನು ಶಂಕರಾಚಾರ್ಯರಿಂದ ಸ್ಥಾಪಿತವಾದ ಅದ್ವೈತಮತವು ಸರ್ವದರ್ಶನಗಳಿಗೂ ಶಿರೋರತ್ನಪ್ರಾಯವಾದದ್ದೆಂಬ ಬುದ್ಧಿಯಿಟ್ಟುಕೊಂಡಿದ್ದನು. ಶಂಕರಾಚಾರ್ಯರ ಯಾವ ಮಾತಿನಲ್ಲಿಯಾದರೂ ಯಾರಾದರೂ ಒಂದು ದೋಷವನ್ನು ಕಂಡುಹಿಡಿದರೆ ಅವನ ಹೃದಯವು ಹಾವು ಕಚ್ಚಿದಂತೆ ವ್ಯಥೆಗೊಳ್ಳುತ್ತಿತ್ತು. ಸ್ವಾಮಿಜಿ ಇದನ್ನು ಬಲ್ಲರು. ಯಾವುದಾದರೂ ಒಂದು ಮತದ ಅತಿ ಪಕ್ಷಪಾತಿಗಳಾಗಿದ್ದರೆ ಅದನ್ನು ಅವರು ಸಹಿಸುತ್ತಿರಲಿಲ್ಲ. ಯಾವುದಾದರೊಂದು ವಿಷಯದಲ್ಲಿ ಅತಿ ಪಕ್ಷಪಾತವನ್ನು ಕಂಡರೆ ಸಾಕು, ಅವರು ಅದರ ವಿರುದ್ದ ಪಕ್ಷವನ್ನು ಹಿಡಿದು ಅಜಸ್ರ ಅಮೋಘ ಯುಕ್ತಿಯ ಬಲದಿಂದ ಈ ಅತಿ ಪಕ್ಷಪಾತದ ಕಟ್ಟನ್ನು ಪುಡಿಪುಡಿಮಾಡಿಹಾಕುತ್ತಿದ್ದರು.

ಸ್ವಾಮೀಜಿ: ಶಂಕರಾಚಾರ್ಯರಿಗೆ ಕತ್ತಿಯ ಧಾರೆಯಂಥ ಬುದ್ಧಿ. ವಿಚಾರಕರು ಹೌದು, ಪಂಡಿತರು ಹೌದು. ಆದರೆ ಅವರ ಉದಾರತೆಯು ತುಂಬಾ ಗಂಭೀರ ವಾದದ್ದಾಗಿರಲಿಲ್ಲ. ಹೃದಯವೂ ಹೀಗಿತ್ತೆಂದು ತೋರುತ್ತದೆ; ಮತ್ತು ಬ್ರಾಹ್ಮಣ ಅಭಿಮಾನವೂ ಹೆಚ್ಚಾಗಿತ್ತು. ದಕ್ಷಿಣ ದೇಶದ ವೈದಿಕ ಬ್ರಾಹ್ಮಣರ ಮಗ, ಮತ್ತೇನು? ಬ್ರಾಹ್ಮಣೇತರರಿಗೆ ಬ್ರಹ್ಮಜ್ಞಾನವಾಗುವುದಿಲ್ಲ ಎಂಬೀ ಅಂಶವನ್ನು ವೇದಾಂತ ಭಾಷ್ಯದಲ್ಲಿ ಹೇಗೆ ಸಮರ್ಥನೆ ಮಾಡಿಕೊಂಡು ಹೋಗಿದ್ದಾರೆ! ಸತ್ಯವೆಂಬಂತೆ ಕಾಣುವ ವಿಚಾರ! ವಿದುರನ ಸಂಗತಿಯನ್ನು ಎತ್ತಿಕೊಂಡು, ಅವನ ಪೂರ್ವಜನ್ಮದ ಬ್ರಾಹ್ಮಣ ಶರೀರದ ಫಲದಿಂದ ಅವನು ಬ್ರಹ್ಮಜ್ಞನಾದನೆಂದು ಹೇಳಿದ್ದಾರೆ. ಹಾಗಾದರೆ ನಾನು ಕೇಳುವುದೇನೆಂದರೆ, ಈಗಿನ ಕಾಲದಲ್ಲಿ ಹೀಗೆ ಯಾವನಿಗಾದರೂ ಶೂದ್ರನಿಗೆ ಬ್ರಹ್ಮಜ್ಞಾನವಾದರೆ ಅವನಿಗೆ ಶಂಕರಾಚಾರ್ಯರು ಹೇಳಿರುವಂತೆ ಪೂರ್ವಜನ್ಮದಲ್ಲಿ ಬ್ರಾಹ್ಮಣ್ಯವಿದ್ದುದರಿಂದ ಈಗ ಹೀಗೆ ಆಯಿತೆಂದು ಹೇಳಬೇಕೆ? ಬ್ರಾಹ್ಮಣತ್ವವನ್ನು ಹೀಗೆ ಎಳೆದಾಡುವುದರಿಂದ ಬಂದ ಫಲವೇನಪ್ಪಾ? ವೇದವೇನೊ ತ್ರಿವರ್ಣದವರೆಗೆ ಮಾತ್ರ ವೇದಪಾಠ ಮತ್ತು ಬ್ರಹ್ಮಜ್ಞಾನಗಳ ಅಧಿಕಾರವನ್ನು ಕೊಟ್ಟಿದೆ. ಆದ್ದರಿಂದಲೇ ಈ ವಿಷಯವನ್ನು ತೆಗೆದುಕೊಂಡು ವೇದದ ಮೇಲೆ ತಮ್ಮ ಅದ್ಭುತ ವಿದ್ಯೆಯನ್ನು ಉಪಯೋಗಿಸುವ ಆವಶ್ಯಕತೆ ಏನೂ ಕಾಣಲಿಲ್ಲ. ಅಲ್ಲದೆ ಅವರ ಹೃದಯವೆಂಥದ್ದೆಂದರೆ, ಅವರು ಎಷ್ಟೋ ಜನ ಬೌದ್ಧ ಶ್ರಮಣರನ್ನು ತಮ್ಮ ತರ್ಕದಿಂದ ಸೋಲಿಸಿಬಿಟ್ಟು, ಬೆಂಕಿಯಲ್ಲಿ ಬೀಳಿಸಿ ಕೊಲ್ಲಿಸಿದರು. ಬುದ್ಧಿಹೀನರಾದ ಬೌದ್ಧರೂ ಕೂಡಾ ತರ್ಕದಲ್ಲಿ ಸೋಲನ್ನು ಒಪ್ಪಿಕೊಂಡು ಬೆಂಕಿಯಲ್ಲಿ ಬಿದ್ದು ಸಾಯುವುದಕ್ಕೆ ಮೊದಲು ಮಾಡಿದರು. ಶಂಕರಾಚಾರ್ಯರ ಈ ರೀತಿಯ ಕಾರ್ಯವನ್ನು ಧರ್ಮಾಂಧತೆ ಎನ್ನದೆ ಮತ್ತೇನೆನ್ನಬಹುದು? ಆದರೆ ಬುದ್ಧದೇವನ ಹೃದಯವನ್ನು ನೋಡು! ‘ಬಹುಜನ ಹಿತಾಯ ಬಹುಜನ ಸುಖಾಯ’ ಎಂಬ ಈ ಮಾತನ್ನು ಹೇಳಬೇಕಾದ್ದೇ ಇಲ್ಲ. ಸಾಮಾನ್ಯ ಒಂದು ಆಡಿನ ಮರಿಯನ್ನು ಕಾಪಾಡುವುದಕ್ಕಾಗಿ ತನ್ನ ಪ್ರಾಣವನ್ನು ಕೊಡುವುದಕ್ಕೆ ಸರ್ವದಾ ಸಿದ್ಧನಾಗಿರುತ್ತಿದ್ದನು; ನೋಡು ಎಂಥ ಉದಾರತೆ - ಎಂಥ ದಯೆ!

ಶಿಷ್ಯ: ಬುದ್ಧದೇವನ ಈ ಭಾವವನ್ನು ಒಂದು ವಿಧವಾದ ಹುಚ್ಚೆಂದು ಹೇಳುವುದಕ್ಕಾಗುವುದಿಲ್ಲವೇ ಮಹಾಶಯರೆ? ಒಂದು ಪಶುವಿಗೋಸ್ಕರ ತನ್ನ ತಲೆಯನ್ನು ಕೊಡುವುದಕ್ಕೆ ಹೋದನಲ್ಲವೆ?

ಸ್ವಾಮೀಜಿ: ಆದರೆ ಅವನ ಮತಭ್ರಾಂತಿಯಿಂದ ಜಗತ್ತಿನ ಪ್ರಾಣಿಗಳಿಗೆ ಎಷ್ಟು ಕಲ್ಯಾಣವಾಯಿತು - ಅದನ್ನು ನೋಡು. ಎಷ್ಟು ಆಶ್ರಮ, ಎಷ್ಟು ಸ್ಕೂಲು, ಎಷ್ಟು ಕಾಲೇಜು, ಎಷ್ಟು ಸಾಧಾರಣ ಜನರಿಗಾಗಿ ಏರ್ಪಡಿಸಿದ ಆಸ್ಪತ್ರೆ, ಎಷ್ಟು ಪಶುಶಾಲೆಗಳ ಸ್ಥಾಪನೆ, ಎಷ್ಟು ಗುಡಿಗೋಪುರಾದಿಗಳನ್ನು ಕಟ್ಟುವುದರಲ್ಲಿ ನಿಪುಣತೆ ಇವು ಪ್ರಕಾಶಕ್ಕೆ ಬಂದವು. ಅದನ್ನು ನೋಡು! ಬುದ್ಧದೇವನು ಹುಟ್ಟುವುದಕ್ಕೆ ಮುಂಚೆ ಈ ದೇಶದಲ್ಲಿ ಇದ್ದದ್ದೇನು? ಓಲೆಗರಿಯ ಪುಸ್ತಕಗಳಲ್ಲಿ ಕಟ್ಟುಹಾಕಿದ್ದ ಕೆಲವು ಧರ್ಮ ತತ್ತ್ವಗಳು - ಅದೂ ಗೊತ್ತಿದ್ದುದು ಸ್ವಲ್ಪ ಜನರಿಗೆ ಮಾತ್ರ. ಭಗವಾನ್ ಬುದ್ಧದೇವನು ಅವುಗಳನ್ನು ಅನುಷ್ಠಾನರಂಗಕ್ಕೆ ತಂದನು. ಜನರು ತಮ್ಮ ದೈನಂದಿನ ಜೀವನದಲ್ಲಿ ಅವುಗಳನ್ನು ಹೇಗೆ ಕಾರ್ಯದಲ್ಲಿ ಉಪಯೋಗಿಸಬೇಕೆಂಬುದನ್ನು ತೋರಿಸಿಕೊಟ್ಟನು. ಒಂದು ದೃಷ್ಟಿಯಲ್ಲಿ, ನಿಜವಾದ ವೇದಾಂತದ ಜೀವಂತ ಮೂರ್ತಿಯೇ ಅವನಾಗಿದ್ದನು.

ಶಿಷ್ಯ; ಆದರೆ ಮಹಾಶಯರೆ, ಆತನೇ ವರ್ಣಾಶ್ರಮಧರ್ಮವನ್ನು ನಾಶಪಡಿಸಿ ಭರತಖಂಡದಲ್ಲಿ ಹಿಂದೂಧರ್ಮದ ವಿಪ್ಲವವನ್ನು ತಂದವನು. ಅದಕ್ಕೋಸ್ಕರವೇ ಅವನಿಂದ ಪ್ರಚಾರಿತವಾದ ಧರ್ಮವು ಭರತಖಂಡದಿಂದ ಕಾಲಕ್ರಮೇಣ ಓಡಿಸಲ್ಪಟ್ಟಿತು; ಇದೂ ನಿಜವೆಂದು ಒಪ್ಪಿಕೊಳ್ಳಬೇಕು.

ಸ್ವಾಮೀಜಿ: ಬೌದ್ಧ ಧರ್ಮದ ಈ ವಿಧವಾದ ದುರ್ದೆಶೆಯು ಆತನ ಬೋಧನೆಯ ದೋಷದಿಂದ ಬರಲಿಲ್ಲ; ಆತನ ಶಿಷ್ಯ ಪರಂಪರೆಯ ದೋಷದಿಂದಲೇ ಬಂದದ್ದು. ಅವರು ತುಂಬ ದಾರ್ಶನಿಕರಾಗಿ ಅವರ ಹೃದಯದ ಉದಾರತೆ ಕಡಿಮೆಯಾಗಿ ಹೋಯಿತು. ಆಮೇಲೆ ಕ್ರಮೇಣ ವಾಮಾಚಾರದ ವ್ಯಭಿಚಾರವು ಪ್ರವೇಶಿಸಿ ಬೌದ್ಧಧರ್ಮ ಮಾಯವಾಯಿತು. ಅಂಥ ಅಸಹ್ಯಕರವಾದ ವಾಮಾಚಾರ ಈಗ ಯಾವ ತಂತ್ರದಲ್ಲಿಯೂ ಇಲ್ಲ! ಬೌದ್ಧ ಧರ್ಮದ ಪ್ರಧಾನ ಕೇಂದ್ರ ‘ಜಗನ್ನಾಥಕ್ಷೇತ್ರ’ವಾಗಿತ್ತು - ಅಲ್ಲಿ ಮಂದಿರ ಪ್ರದೇಶದಲ್ಲಿ ಕೆತ್ತಿರುವ ಅಸಹ್ಯಕರವಾದ ಮೂರ್ತಿಗಳನ್ನು ಒಂದು ಸಲ ಹೋಗಿ ನೋಡಿಕೊಂಡು ಬಂದರೆ ಈ ವಿಷಯವನ್ನು ತಿಳಿದುಕೊಳ್ಳಬಲ್ಲೆ. ರಾಮಾನುಜ ಮತ್ತು ಚೈತನ್ಯ ಮಹಾಪ್ರಭುವಿನ ಕಾಲದಿಂದ ಪುರುಷೋತ್ತಮ ಕ್ಷೇತ್ರ ವೈಷ್ಣವರ ವಶಕ್ಕೆ ಬಂದಿದೆ. ಈಗ ಅದು ಈ ಮಹಾಪುರುಷರ ಶಕ್ತಿಸಾಮರ್ಥ್ಯದಿಂದ ರೂಪಾಂತರವನ್ನು ಹೊಂದಿದೆ.

ಶಿಷ್ಯ: ಮಹಾಶಯರೆ, ತೀರ್ಥಾದಿ ಸ್ಥಾನಗಳಿಗೆ ವಿಶೇಷ ಮಹಿಮೆಯಿರುವಂತೆ ಶಾಸ್ತ್ರದಿಂದ ತಿಳಿದುಬರುತ್ತದೆ. ಇದರಲ್ಲಿ ಎಷ್ಟು ನಿಜವಿದೆ?

ಸ್ವಾಮೀಜಿ: ಸಮಗ್ರ ಬ್ರಹ್ಮಾಂಡವೂ ನಿತ್ಯಾತ್ಮನಾದ ಈಶ್ವರನ ವಿರಾಟ್ ಶರೀರವಾದ ಮೇಲೆ ಸ್ಥಾನ ಮಹಾತ್ಮ್ಯವಿರುವುದರಲ್ಲಿ ವೈಚಿತ್ರ್ಯವೇನು? ಸ್ಥಾನ ವಿಶೇಷದಲ್ಲಿ ಅದರ ವಿಶೇಷ ಪ್ರಕಾಶ - ಕೆಲವೆಡೆಗಳಲ್ಲಿ ತಾನಾಗಿ ಮತ್ತೆ ಕೆಲವೆಡೆಗಳಲ್ಲಿ ಶುದ್ಧ ಸತ್ಯವಾದ ಮಾನವ ಮನಸ್ಸಿನ ವ್ಯಾಕುಲಾಗ್ರಹದಿಂದ – ಇದ್ದೇ ಇರುತ್ತದೆ. ಸಾಧಾರಣ ಜನರು ಈ ಸ್ಥಳಗಳಿಗೆ ಜಿಜ್ಞಾಸುಗಳಾಗಿ ಹೋದರೆ ಫಲ ಸುಲಭವಾಗಿ ಬರುತ್ತದೆ. ಇದಕ್ಕೋಸ್ಕರವೇ ತೀರ್ಥಕ್ಷೇತ್ರಾದಿಗಳನ್ನು ಆಶ್ರಯಿಸಿಕೊಂಡಿದ್ದರೆ ಸಕಾಲದಲ್ಲಿ ಆತ್ಮವಿಕಾಸ ಆಗುವುದು ಸಾಧ್ಯವಾಗುತ್ತದೆ. ಈ ಮಾನವ ದೇಹಕ್ಕಿಂತ ಪ್ರಧಾನವಾದ ತೀರ್ಥಸ್ಥಳ - ಮತ್ತೊಂದು ಇಲ್ಲ ಎಂಬುದನ್ನು ಖಂಡಿತವಾಗಿ ತಿಳಿದುಕೊ. ಇಲ್ಲಿ ಆತ್ಮದ ಎಂಥ ವಿಕಾಸವಿದೆಯೊ, ಅಷ್ಟು ಮತ್ತೆಲ್ಲಿಯೂ ಇಲ್ಲ. ಜಗನ್ನಾಥನ ರಥವಿದೆಯಷ್ಟೆ. ಅದೂ ಈ ದೇಹವೆಂಬ ರಥದ ಸ್ಥೂಲರೂಪ ಮಾತ್ರ. ಈ ದೇಹವನ್ನು ಧರಿಸಿ ಆತ್ಮವನ್ನು ದರ್ಶನ ಮಾಡಲೇಬೇಕಾಗಿದೆ. ‘ಆತ್ಮಾನಂ ರಥಿನಂ ವಿದ್ಧಿ’ - ಆತ್ಮನು ರಥದಲ್ಲಿ ಕುಳಿತಿರುವನು ಎಂದು ತಿಳಿ, ಎಂದು ಓದಿಲ್ಲವೆ? ‘ಮಧ್ಯೇ ವಾಮನಮಾಸೀನಂ ವಿಶ್ವೇದೇವಾ ಉಪಾಸತೇ’ – ದೇಹದಲ್ಲಿ ಕುಳಿತಿರುವ ವಾಮನನನ್ನು ಎಲ್ಲ ದೇವತೆಗಳೂ ಪೂಜಿಸುತ್ತಾರೆ. ಈ ವಾಮನರೂಪಿಯಾದ ಆತ್ಮದ ದರ್ಶನವೇ ನಿಜವಾದ ಜಗನ್ನಾಥ ದರ್ಶನ. ‘ರಥೇ ಚ ವಾಮನಂ ದೃಷ್ಟ್ವಾ ಪುನರ್ಜನ್ಮ ನ ವಿದ್ಯತೇ’ ರಥದಲ್ಲಿ ಇರುವ ವಾಮನನನ್ನು ನೋಡಿದರೆ ಪುನರ್ಜನ್ಮ ಇಲ್ಲ ಎಂದು ಹೇಳಿದೆಯಲ್ಲಾ, ಇದರ ಅಭಿಪ್ರಾಯವೇನೆಂದರೆ ಯಾವುದನ್ನು ಉಪೇಕ್ಷೆ ಮಾಡಿ ಈ ವಿಕೃತ ದೇಹರೂಪವಾದ ಜಡಪದಾರ್ಥವನ್ನು ಯಾವಾಗಲೂ ‘ನಾನು’ ಎಂದು ಹಿಡಿದುಕೊಂಡಿರುತ್ತೀಯೋ, ಆ ನಿನ್ನ ಅಂತರಾತ್ಮನನ್ನು ದರ್ಶನ ಮಾಡಲು ಸಮರ್ಥನಾದರೆ ಮತ್ತೆ ಪುನರ್ಜನ್ಮವಿಲ್ಲ - ಎಂದು. ಮಠದ ಮಂಟಪದಲ್ಲಿ ದೇವರನ್ನು ನೋಡುವುದರಿಂದಲೇ ಜೀವನಿಗೆ ಮುಕ್ತಿಯಾಗುವ ಹಾಗಿದ್ದಿದ್ದರೆ, ವರ್ಷವರ್ಷವೂ ಕೋಟಿ ಕೋಟಿ ಜೀವಗಳಿಗೆ ಮುಕ್ತಿಯಾಗುತ್ತಿತ್ತು - ಈಗಿನ ಕಾಲದಲ್ಲಿ ರೈಲಿನಲ್ಲಿ ಹೋಗುವುದು ಎಷ್ಟು ಸುಲಭವೋ ಹಾಗೆ. ಆದರೆ ಜಗನ್ನಾಥನ ಸಾಧಾರಣ ಭಕ್ತ ಜನರಿಗೆ ಇರುವ ನಂಬಿಕೆಯನ್ನು ‘ಕೆಲಸಕ್ಕೆ ಬಾರದ್ದು, ಸುಳ್ಳು’ ಎಂದು ಹೇಳುವುದಿಲ್ಲ. ಒಂದು ಶ್ರೇಣಿಯ ಜನರಿದ್ದಾರೆ. ಅವರು ಈ ಮೂರ್ತಿಯನ್ನು ಅವಲಂಬಿಸಿಕೊಂಡು ಉಚ್ಚಭಾವದಿಂದ ಉಚ್ಚತರ ತತ್ತ್ವಕ್ಕೆ ಹತ್ತಿ ಹೋಗುತ್ತಾರೆ; ಆದ್ದರಿಂದಲೇ ಈ ಮೂರ್ತಿಯಲ್ಲಿ ಭಗವಂತನ ವಿಶೇಷ ಶಕ್ತಿ ಪ್ರಕಾಶಿತವಾಗಿದೆ ಎಂಬುದರಲ್ಲಿ ಸಂದೇಹವಿಲ್ಲ.

ಶಿಷ್ಯ: ಹಾಗಾದರೇನು, ಮಹಾಶಯರೇ, ದಡ್ಡರಿಗೂ ಬುದ್ಧಿವಂತರಿಗೂ ಧರ್ಮವೆಂಬುದು ಬೇರೆಬೇರೆಯೇ?

ಸ್ವಾಮೀಜಿ: ಹಾಗೇ ತಾನೆ; ಇಲ್ಲದಿದ್ದರೆ ನಿಮ್ಮ ಶಾಸ್ತ್ರಗಳು ಕೂಡ ಅಧಿಕಾರಿ ನಿರ್ದೆಶ ಮಾಡುವುದಕ್ಕೆ ಇಷ್ಟೊಂದು ಒದ್ದಾಡುವುದೇಕೆ? ಎಲ್ಲವೂ ಸತ್ಯ. ಆದರೆ ಸಾಪೇಕ್ಷ ಸತ್ಯ. ಮನುಷ್ಯನು ಯಾವ ಯಾವುದನ್ನು ಸತ್ಯವೆಂದು ತಿಳಿದುಕೊಂಡಿದ್ದಾನೆಯೋ ಅದೆಲ್ಲಾ ಹೀಗೆಯೇ; ಕೆಲವು ಸ್ವಲ್ಪ ಸತ್ಯ, ಕೆಲವು ಅದಕ್ಕಿಂತಲೂ ಹೆಚ್ಚು ಸತ್ಯ; ನಿತ್ಯ ಸತ್ಯ ಕೇವಲ ಭಗವಂತ ಒಬ್ಬ ಮಾತ್ರ. ಈ ಆತ್ಮವು ಜಡಪದಾರ್ಥದ ಮಧ್ಯದಲ್ಲಿ ಪೂರ್ತಿಯಾಗಿ ನಿದ್ದೆ ಮಾಡುತ್ತಿದೆ. ಜೀವನೆಂಬ ಮನುಷ್ಯನಲ್ಲಿ ಅದೇ ಸ್ವಲ್ಪ ಜಾಗರಿತವಾಗಿದೆ. ಶ‍್ರೀಕೃಷ್ಣ ಬುದ್ದ ಶಂಕರಾದಿಗಳಲ್ಲಾದರೋ ಈ ಆತ್ಮವು ಪೂರ್ಣವಾಗಿ ಜಾಗರಿತವಾಗಿದೆ. ಇದಕ್ಕೂ ಮೇಲಿನ ಅವಸ್ಥೆಯೊಂದು ಉಂಟು. ಅದನ್ನು ಭಾವಿಸಿಕೊಳ್ಳುವುದಕ್ಕಾಗಲಿ ಮಾತಿನಲ್ಲಿ ಹೇಳುವುದಕ್ಕಾಗಲೀ ಆಗುವುದಿಲ್ಲ - ‘ಅವಾಙ್ಮಾನಸಗೋಚರಂ.’

ಶಿಷ್ಯ: ಮಹಾಶಯರೇ ಕೆಲಕೆಲವು ಸಂಪ್ರದಾಯದವರು ಭಗವಂತನೊಡನೆ ಒಂದೊಂದು ಭಾವ ಅಥವಾ ಸಂಬಂಧವನ್ನು ಕಲ್ಪಿಸಿಕೊಂಡು ಸಾಧನೆ ಮಾಡಬೇಕೆಂದು ಹೇಳುತ್ತಾರೆ. ಆತ್ಮದ ಮಹತ್ವ ಮುಂತಾದುವುಗಳ ವಿಷಯವನ್ನು ಅವರು ಸ್ವಲ್ಪವೂ ಅರಿಯರು. ಕೇಳಿದರೂ ‘ಈ ವಿಷಯಗಳನ್ನೆಲ್ಲಾ ಬಿಟ್ಟು ಯಾವಾಗಲೂ ಭಾವದಲ್ಲಿ ಇರಿ’ ಎಂದು ಹೇಳುತ್ತಾರೆ.

ಸ್ವಾಮೀಜಿ: ಅವರು ಏನು ಹೇಳುತ್ತಾರೆಯೋ ಅದು ಅವರ ವಿಚಾರಕ್ಕೆ ಸತ್ಯವೆ. ಹೀಗೆ ಮಾಡುತ್ತ ಅವರೊಳಗೂ ಒಂದು ದಿನ ಬ್ರಹ್ಮಜಾಗರಿತವಾಗುತ್ತದೆ. ನಾವು ಸಂನ್ಯಾಸಿಗಳು ಮಾಡುವುದು ಮತ್ತೊಂದು ವಿಧವಾದ ಭಾವ. ನಾವು ಸಂಸಾರವನ್ನು ತಾಗಮಾಡಿದ್ದೇವೆ. ಆದ್ದರಿಂದ ನಮ್ಮ ಭಾವವು ಅಮ್ಮ, ಅಪ್ಪ, ಹೆಂಡತಿ, ಮಗ ಮುಂತಾದ ಸಂಬಂಧರೂಪವಾದ ಯಾವುದಾದರೂ ಒಂದು ಭಾವವನ್ನು ಭಗವಂತನಲ್ಲಿ ಆರೋಪಿಸಿ ಸಾಧನೆ ಮಾಡುವ ಭಾವ ಹೇಗಾದೀತು? ಅವೆಲ್ಲಾ ನಮ್ಮ ಮನಸ್ಸಿಗೆ ಸಂಕುಚಿತವೆನಿಸುತ್ತದೆ. ನಿಜವಾಗಿಯೂ, ಸರ್ವಭಾವಾತೀತನಾದ ಭಗವಂತನ ಉಪಾಸನೆ ಮಾಡುವುದು ಬಹು ಕಠಿಣ. ಆದರೆ ಅಮೃತ ಸಿಕ್ಕುತ್ತದೆಯೊ ಇಲ್ಲವೊ ಎಂದು ವಿಷ ಕುಡಿಯುವುದಕ್ಕೆ ಹೋಗಬೇಕೆ? ಈ ಆತ್ಮದ ಸಂಗತಿಯನ್ನು ಸರ್ವದಾ ಹೇಳುತ್ತಿರಬೇಕು, ಕೇಳುತ್ತಿರಬೇಕು, ವಿಚಾರಮಾಡುತ್ತಿರಬೇಕು. ಹೀಗೆ ಮಾಡುತ್ತಾ ಮಾಡುತ್ತಾ ಕಾಲಕ್ರಮದಲ್ಲಿ ನಿನ್ನೊಳಗೂ ಬ್ರಹ್ಮಜ್ಞಾನವಾಗುವುದನ್ನು ನೋಡುವೆ. ಈ ಭಾವ ಚಾಪಲ್ಯಗಳನ್ನೆಲ್ಲ ಬಿಟ್ಟು ನಡೆ. ಇದನ್ನು ಕೇಳು, ಕಠೋಪನಿಷತ್ತಿನಲ್ಲಿ ಯಮನು ಹೇಳಿರುವುದನ್ನು -

\begin{verse}
“ಉತ್ತಿಷ್ಠತ ಜಾಗ್ರತ ಪ್ರಾಪ್ಯವರಾನ್ ನಿಬೋಧತ”
\end{verse}

ಎದ್ದೇಳು, ಎಚ್ಚರವಾಗು, ಶ್ರೇಷ್ಠವಾದವರನ್ನು ಆಶ್ರಯಿಸಿ ಜ್ಞಾನವನ್ನು ಪಡೆ. ಹೀಗೆ ಈ ಪ್ರಸ್ತಾವವು ಮುಗಿಯಿತು. ಮಠದಲ್ಲಿ ಪ್ರಸಾದವನ್ನು ಕೊಡುವ ಗಂಟೆಯಾಯಿತು. ಸ್ವಾಮೀಜಿ ಜೊತೆಯಲ್ಲಿ ಶಿಷ್ಯನೂ ಪ್ರಸಾದವನ್ನು ತೆಗೆದುಕೊಳ್ಳುವುದಕ್ಕೆ ಹೊರಟನು.

\newpage

\chapter[ಅಧ್ಯಾಯ ೧೫]{ಅಧ್ಯಾಯ ೧೫\protect\footnote{\engfoot{C.W, Vol. VII, P 122}}}

\begin{center}
ಸ್ಥಳ: ಬೇಲೂರು ಮಠ (ಬಾಡಿಗೆ ಕಟ್ಟಡ), ವರ್ಷ: ಕ್ರಿ.ಶ. ೧೮೯೮, ಫೆಬ್ರವರಿ ತಿಂಗಳು.
\end{center}

ಬೇಲೂರಿನಲ್ಲಿ ಶ‍್ರೀಯುತ ನೀಲಾಂಬರ ಬಾಬುವಿನ ತೋಟಕ್ಕೆ ಸ್ವಾಮೀಜಿ ಮಠವನ್ನು ವರ್ಗಾಯಿಸಿದರು. ಆಲಂಬಜಾರಿನಿಂದ ಇಲ್ಲಿಗೆ ಬಂದದ್ದಾಯಿತಾದರೂ ಸಾಮಾನು ಸರಂಜಾಮುಗಳನ್ನೆಲ್ಲಾ ಇನ್ನೂ ಜೋಡಿಸಿಟ್ಟಿರಲಿಲ್ಲ. ಎಲ್ಲವೂ ಅಲ್ಲಿ ಇಲ್ಲಿ ಬಿದ್ದಿವೆ. ಸ್ವಾಮೀಜಿ ಹೊಸ ಮನೆಗೆ ಬಂದು ಬಹು ಸಂತೋಷವಾಗಿದ್ದಾರೆ. ಶಿಷ್ಯನು ಅವರ ಹತ್ತಿರ ಹೋದಕೂಡಲೆ “ನೋಡಿದೆಯಾ ಗಂಗೆ ಹೇಗಿದೆ - ಕಟ್ಟಡ ಹೇಗಿದೆ - ಮಠ ಇಂಥ ಸ್ಥಳದಲ್ಲಿ ಇಲ್ಲದಿದ್ದಿದ್ದರೆ ಚೆನ್ನಾಗಿರುತ್ತಿತ್ತೆ?" ಎಂದರು. ಆಗ ಅಪರಾಹ್ನವಾಗಿತ್ತು.

ಸಾಯಂಕಾಲ ಶಿಷ್ಯನು ಮಹಡಿಯ ಮೇಲೆ ಸ್ವಾಮೀಜಿಯ ದರ್ಶನ ತೆಗೆದುಕೊಳ್ಳಲು ಹೋದಾಗ, ನಾನಾ ಪ್ರಸ್ತಾಪಗಳು ಬಂದವು. ಮನೆಯಲ್ಲಿ ಮತ್ತಾರೂ ಇಲ್ಲ; ಶಿಷ್ಯನು ಮಧ್ಯೆ ಮಧ್ಯೆ ಎದ್ದು ಸ್ವಾಮೀಜಿಗೆ ತಂಬಾಕನ್ನು ಸಿದ್ದಮಾಡಿ ಕೊಡುತ್ತಿದ್ದನು. ಅವನು ಪ್ರಶ್ನೆಗಳನ್ನು ಹಾಕುತ್ತ ಕೊನೆಯಲ್ಲಿ ಮಾತಿಗೆ ಮಾತು ಬಂದು ಸ್ವಾಮಿಜಿಯ ಬಾಲ್ಯಕಾಲದ ವಿಷಯವನ್ನು ತಿಳಿದುಕೊಳ್ಳಬೇಕು ಎಂದು ಅಪೇಕ್ಷಿಸಿದನು. ಸ್ವಾಮೀಜಿ “ಚಿಕ್ಕ ವಯಸ್ಸಿನಿಂದ ನಾನು ಭೂತ ಪಿಶಾಚಿಗಳನ್ನು ಹೆದರಿಸಿ ಓಡಿಸುತ್ತಿದ್ದಂಥವನು; ಇಲ್ಲದಿದ್ದರೆ ನನಗೆ ಕೈಯಲ್ಲಿ ಕಾಸಿಲ್ಲದೆ ಪ್ರಪಂಚವನ್ನು ತಿರುಗಿ ಬರುವುದಕ್ಕಾಗುತ್ತಿತ್ತೇನಯ್ಯಾ?" ಎಂದರು.

ಹುಡುಗರಾಗಿದ್ದಾಗ, ರಾಮಾಯಣ ಗಾನವನ್ನು ಕೇಳುವುದಕ್ಕೆ ಅವರಿಗೆ ತುಂಬ ಆಶೆಯಿತ್ತು. ತಾವಿದ್ದ ಭಾಗದಲ್ಲಿ ಹತ್ತಿರ ಎಲ್ಲಿ ರಾಮಾಯಣ ಗಾನವಾದರೂ ಆಟಪಾಠಗಳನ್ನು ಬಿಟ್ಟು ಅಲ್ಲಿಗೆ ಹೋಗುತ್ತಿದ್ದರು. ರಾಮಾಯಣವನ್ನು ಕೇಳುತ್ತ ಕೇಳುತ್ತ ಒಂದೊಂದು ದಿನ ತನ್ಮಯರಾಗಿ ಹೋಗಿ ಮನೆ ಮಠಗಳನ್ನು ಮರೆತು ಬಿಡುತ್ತಿದ್ದರಂತೆ ಮತ್ತು ‘ಕತ್ತಲೆಯಾಯಿತು’ ಎಂದಾಗಲಿ ‘ಮನೆಗೆ ಹೋಗಬೇಕು’ ಎಂದಾಗಲಿ ಯಾವ ವಿಧವಾದ ಚಾಪಲ್ಯವೂ ಇರುತ್ತಿರಲಿಲ್ಲವಂತೆ. ಒಂದು ದಿನ ರಾಮಾಯಣವನ್ನು ಕೇಳುತ್ತಿರುವಾಗ ಹನುಮಂತನು ಬಾಳೆಯ ತೋಟದಲ್ಲಿರುವನೆಂಬುದು ಕಿವಿಗೆ ಬಿತ್ತು. ಆಗ ಆ ಮಾತಿನಲ್ಲಿ ಎಷ್ಟು ನಂಬಿಕೆ ಹುಟ್ಟಿ ಹೋಯಿತೆಂದರೆ, ಅವರು ರಾತ್ರಿ ರಾಮಾಯಣವನ್ನು ಕೇಳಿ ಮನೆಗೆ ಹಿಂದಿರುಗಿ ಹೋಗದೆ ತಮ್ಮ ಮನೆಯ ಹತ್ತಿರದಲ್ಲಿದ್ದ ಯಾವುದೊ ಒಂದು ತೋಟದಲ್ಲಿ ಬಾಳೆಯ ಗಿಡದ ಕೆಳಗೆ ಹನುಮಂತನ ದರ್ಶನದ ಆಶೆಯಿಂದ ಹೊತ್ತನ್ನು ಕಳೆದುಬಿಟ್ಟರು.

ಹನುಮಂತನ ಮೇಲೆ ಸ್ವಾಮಿಗಳಿಗೆ ಅಗಾಧವಾದ ಭಕ್ತಿಯಿತ್ತು. ಸಂನ್ಯಾಸಿಗಳಾದ ಮೇಲೆಯೂ ಆಗಾಗ್ಗೆ ಮಹಾವೀರನ (ಆಂಜನೇಯನ) ಪ್ರಸ್ತಾವದಲ್ಲಿ ಉನ್ಮತ್ತರಂತಾಗಿಬಿಡುತ್ತಿದ್ದರು ಮತ್ತು ಅನೇಕ ವೇಳೆ ಮಠದಲ್ಲಿ ಅವನ ಒಂದು ಸ್ಪಟಿಕ ಶಿಲೆಯ ಪ್ರತಿಮೆಯನ್ನು ಸ್ಥಾಪಿಸಬೇಕೆಂದು ಸಂಕಲ್ಪಿಸಿದ್ದರು.

ವಿದ್ಯಾರ್ಥಿದೆಶೆಯಲ್ಲಿ, ಅವರು ಹಗಲುಹೊತ್ತು ಜೊತೆ ಹುಡುಗರೊಡನೆ ಆಮೋದ ಪ್ರಮೋದಗಳನ್ನು ಮಾಡುತ್ತ ತಿರುಗುತ್ತಿದ್ದರು. ರಾತ್ರಿಯ ಹೊತ್ತು ಮನೆಯ ಬಾಗಿಲನ್ನು ಹಾಕಿಕೊಂಡು ಓದಿ ಬರೆದು ಮಾಡುತ್ತಿದ್ದರು. ಯಾವಾಗ ಬಂದು ಅವರು ಓದಿ ಬರೆದು ಮಾಡುತ್ತಿದ್ದರೆಂಬುದನ್ನು ಯಾರೂ ತಿಳಿದುಕೊಳ್ಳಲಾರದವರಾಗಿದ್ದರು.

\delimiter

ಶಿಷ್ಯ: ಮಹಾಶಯರೆ, ಸ್ಕೂಲಿನಲ್ಲಿ ಓದುವಾಗ ತಮಗೆ ಯಾವಾಗಲಾದರೂ ಏನಾದರೂ “ದರ್ಶನ"ಗಳಾದುವೆ?

ಸ್ವಾಮೀಜಿ: ಸ್ಕೂಲಿನಲ್ಲಿ ಓದುವಾಗ ಒಂದು ರಾತ್ರಿ ಬಾಗಿಲನ್ನು ಹಾಕಿಕೊಂಡು ಧ್ಯಾನ ಮಾಡುತ್ತಾ ಮಾಡುತ್ತಾ ಮನಸ್ಸು ತುಂಬ ತನ್ಮಯವಾಗಿಬಿಟ್ಟಿತು. ಎಷ್ಟು ಹೊತ್ತು ಈ ಭಾವದಲ್ಲಿ ಧ್ಯಾನ ಮಾಡಿದೆನೊ ಹೇಳಲಾರೆ. ಧ್ಯಾನ ಮುಗಿಯಿತು - ಆಗಲೂ ಕುಳಿತುಕೊಂಡಿದ್ದೆ - ಆಗ ಆ ಮನೆಯ ಗೋಡೆಯನ್ನು ಒಡೆದುಕೊಂಡು ಜ್ಯೋತಿರ್ಮಯ ಮೂರ್ತಿಯೊಂದು ಒಳಕ್ಕೆ ಬಂದು ಎದುರಿಗೆ ನಿಂತುಕೊಂಡಿತು. ಮಹಾಶಾಂತ ಸಂನ್ಯಾಸ ಮೂರ್ತಿ, ಮುಂಡನವಾದ ತಲೆ; ಕೈಯಲ್ಲಿ ದಂಡ ಮತ್ತು ಕಮಂಡಲು. ನನ್ನನ್ನು ಸ್ವಲ್ಪ ಹೊತ್ತು ಎವೆಯಿಕ್ಕದೆ ನೋಡುತ್ತಿತ್ತು - ನನಗೆ ಏನೋ ಹೇಳಬೇಕೆಂಬಂತಿದ್ದ ಭಾವ. ನಾನು ಬೆರಗಾಗಿ ಅದರ ಕಡೆಗೆ ನೋಡುತ್ತಿದ್ದೆನು. ಆಮೇಲೆ ಮನಸ್ಸಿಗೆ ಏನೋ ವಿಧವಾದ ದಿಗಿಲು ಉಂಟಾಯಿತು - ತಟ್ಟನೆ ಬಾಗಿಲನ್ನು ತೆಗೆದು ಮನೆಯಿಂದ ಹೊರಕ್ಕೆ ಬಂದು ಬಿಟ್ಟೆ. ಆಮೇಲೆ ನಾನು ಏಕೆ ಹೀಗೆ ದಡ್ಡನಂತೆ ದಿಗಿಲುಪಟ್ಟು ಓಡಿಹೋದೆ, ಪ್ರಾಯಶಃ ಅದು ಏನನ್ನೋ ಹೇಳುತ್ತಿತ್ತು ಎಂದು ಮನಸ್ಸಿಗೆ ಎನ್ನಿಸಿತು. ಆದರೆ, ಪುನಃ ಆ ಮೂರ್ತಿಯನ್ನು ಯಾವಾಗಲೂ ನೋಡಿಲ್ಲ. ಎಷ್ಟೋ ದಿನ ಮನಸ್ಸಿಗೆ ಎನ್ನಿಸಿದೆ, ಅದರ ಸುಳಿವನ್ನು ಕಂಡರೆ ಈ ಸಾರಿ ಹೆದರಿಕೊಳ್ಳುವುದಿಲ್ಲ, ಅದರೊಡನೆ ಮಾತನಾಡುತ್ತೇನೆ ಎಂದು; ಆದರೆ ಪುನಃ ಕಾಣಲಿಲ್ಲ.

ಶಿಷ್ಯ: ಆಮೇಲೆ ಈ ವಿಷಯವನ್ನು ಕುರಿತು ಯೋಚಿಸಲಿಲ್ಲವೆ?

ಸ್ವಾಮೀಜಿ: ಯೋಚಿಸಿದೆ; ಆದರೆ ಯಾವುದೂ ಗೊತ್ತಾಗಲಿಲ್ಲ; ಭಗವಾನ್ ಬುದ್ಧದೇವನನ್ನು ನೋಡಿರಬೇಕೆಂದು ಈಗ ತೋರುತ್ತದೆ.

ಸ್ವಲ್ಪ ಹೊತ್ತಿನ ಮೇಲೆ, ಸ್ವಾಮೀಜಿ ಹೇಳಿದ್ದೇನೆಂದರೆ; ಮನಸ್ಸು ಶುದ್ಧವಾದರೆ, ಕಾಮಕಾಂಚನಗಳಲ್ಲಿ ಆಶೆಯು ತೊಲಗಿದರೆ, ಎಷ್ಟೋ ದಿವ್ಯದರ್ಶನಗಳನ್ನು ನೋಡಬಹುದು - ಅದ್ಭುತ, ಅದ್ಭುತ! ಆದರೆ ಅದರಲ್ಲಿ ಚಾಪಲ್ಯವನ್ನು ಇಟ್ಟುಕೊಳ್ಳಬಾರದು. ಇವುಗಳಲ್ಲಿ ಹಗಲು ರಾತ್ರೆಯೂ ಮನಸ್ಸಿದ್ದರೆ ಸಾಧಕ ಮುಂದುವರಿಯಲಾರ. ಕೇಳಿಲ್ಲವೆ? ಪರಮಹಂಸರು ‘ನನ್ನ ಪ್ರಿಯನಾಥನ ಗರ್ಭಗುಡಿಯ ಹೊರಗೆ ಎಷ್ಟೋ ರತ್ನಗಳು ಬಿದ್ದಿವೆ’ ಎಂದು ಹೇಳುತ್ತಿದ್ದರು. ಆತ್ಮವನ್ನು ಸಾಕ್ಷಾತ್ಕಾರ ಮಾಡಿಕೊಳ್ಳಬೇಕು. ಆ ಚಾಪಲ್ಯಗಳಲ್ಲೆಲ್ಲಾ ಮನಸ್ಸಿಟ್ಟರೆ ಏನಾಗುತ್ತದೆ?

ಈ ಮಾತುಗಳನ್ನು ಹೇಳಿ ಸ್ವಾಮೀಜಿ ತನ್ಮಯರಾಗಿ ಏನೋ ಒಂದು ವಿಷಯವನ್ನು ಯೋಚಿಸುತ್ತ ಸ್ವಲ್ಪ ಹೊತ್ತು ಸುಮ್ಮನೆ ಕುಳಿತಿದ್ದರು. ಆಮೇಲೆ ಪುನಃ ಹೇಳತೊಡಗಿದರು: ನೋಡು! ಅಮೆರಿಕಾದಲ್ಲಿದ್ದಾಗ ನನಗೆ ಕೆಲವು ಅದ್ಭುತ ಶಕ್ತಿಗಳು ಸ್ಪುರಿಸಿದುವು. ಜನರ ಕಣ್ಣನ್ನು ನೋಡಿ ಅವರ ಮನಸ್ಸಿನೊಳಗಿರುವುದನ್ನೆಲ್ಲಾ ತಿಳಿದುಕೊಳ್ಳಬಲ್ಲವನಾಗಿದ್ದೆ - ಒಂದು ಕ್ಷಣಮಾತ್ರದಲ್ಲಿ ಯಾರು ಯಾವುದನ್ನು ಯೋಚಿಸುತ್ತಿದ್ದರು, ಯಾವುದನ್ನು ಯೋಚಿಸುತ್ತಿರಲಿಲ್ಲ, ಇದೆಲ್ಲಾ ಕರತಲಾಮಲಕದ ಹಾಗೆ ಪ್ರತ್ಯಕ್ಷವಾಗಿಬಿಡುತ್ತಿತ್ತು. ಕೆಲಕೆಲವರಿಗೆ ಅದನ್ನು ಹೇಳುತ್ತಿದ್ದೆ; ಯಾರುಯಾರಿಗೆ ಹೇಳುತ್ತಿದ್ದೆನೋ ಅವರಲ್ಲಿ ಅನೇಕರು ಶಿಷ್ಯರಾಗಿಬಿಡುತ್ತಿದ್ದರು. ಯಾರಾದರೂ ಯಾವುದಾದರೂ ವಿಧವಾದ ಕೇಡನ್ನು ಬಗೆದು ನನ್ನ ಜೊತೆಯಲ್ಲಿ ಸೇರುವುದಕ್ಕೆ ಬಂದಿದ್ದರೆ ಅಂಥವರು ಈ ಶಕ್ತಿಯ ಪರಿಚಯವನ್ನು ಪಡೆದು ಮತ್ತೆ ನನ್ನ ಕಡೆಗೇ ಸುಳಿಯುತ್ತಿರಲಿಲ್ಲ.

ಚಿಕಾಗೋ ಮುಂತಾದ ಪಟ್ಟಣಗಳಲ್ಲಿ ಉಪನ್ಯಾಸಕ್ಕೆ ಆರಂಭಿಸಿದಾಗ, ವಾರಕ್ಕೆ ೧೨-೧೩, ಕೆಲವು ವೇಳೆ ಇನ್ನೂ ಹೆಚ್ಚಾಗೆ, ಉಪನ್ಯಾಸಗಳನ್ನು ಕೊಡಬೇಕಾಗಿತ್ತು. ಅತ್ಯಧಿಕವಾದ ಶಾರೀರಿಕ ಮತ್ತು ಮಾನಸಿಕ ಶ್ರಮದಿಂದ ಬಳಲಿ ಬೇಸತ್ತು ಬಿದ್ದು ಬಿಡುತ್ತಿದ್ದೆ. ಉಪನ್ಯಾಸದ ವಿಷಯವೆಲ್ಲಾ ಮುಗಿದುಹೋಗಿ ಬಿಟ್ಟಂತಾಗುತ್ತಿತ್ತು. ಏನು ಮಾಡಲಿ, ಬೆಳಿಗ್ಗೆ ಎಲ್ಲಿಂದ ಯಾವ ಹೊಸ ವಿಷಯವನ್ನು ತಂದು ಹೇಳಲಿ - ಎಂದು ಯೋಚಿಸುತ್ತಿದ್ದೆ. ಹೊಸ ವಿಷಯ ಇನ್ನು ದೊರಕುವುದಿಲ್ಲ ಎನ್ನುವ ಹಾಗೆ ಇರುತ್ತಿತ್ತು. ಒಂದು ದಿನ ಉಪನ್ಯಾಸವಾದ ಮೇಲೆ ಮಲಗಿಕೊಂಡೇ ಯೋಚನೆ ಮಾಡುತ್ತಿದ್ದೆ - ಹಾಗಾದರೆ ಈಗ ಏನು ಉಪಾಯ ಮಾಡಬೇಕು ಎಂದು ಯೋಚಿಸುತ್ತ ಸ್ವಲ್ಪ ಜೋಂಪು ಹತ್ತಿದಂತಾಯಿತು. ಆ ಸ್ಥಿತಿಯಲ್ಲಿಯೇ ನನಗೆ ಕೇಳಿಬಂತು - ಯಾರೋ ಒಬ್ಬರು ನನ್ನ ಪಕ್ಕದಲ್ಲಿ ನಿಂತುಕೊಂಡು ಉಪನ್ಯಾಸ ಮಾಡುತ್ತಿದ್ದ ಹಾಗೆ; ಎಷ್ಟು ನೂತನ ಭಾವಗಳು, ಎಷ್ಟು ವಿಷಯಗಳು! - ಅವುಗಳನ್ನೆಲ್ಲಾ ಈ ಜನ್ಮದಲ್ಲಿ ಕೇಳಿರಲಿಲ್ಲ. ಜೋಂಪಿನಿಂದ ಎದ್ದು ಅವುಗಳನ್ನು ನೆನಪಿಗೆ ತಂದು ಇಟ್ಟುಕೊಂಡೆ; ಉಪನ್ಯಾಸದಲ್ಲಿಯೂ ಅದನ್ನೇ ಹೇಳಿಬಿಟ್ಟೆ. ಹೀಗೆ ಎಷ್ಟು ದಿನಗಳು ಕಳೆದವೊ ಅವಕ್ಕೆ ಲೆಕ್ಕವಿಲ್ಲ. ಮಲಗಿಕೊಂಡಿದ್ದ ಹಾಗೆಯೆ ಇಂಥ ಉಪನ್ಯಾಸಗಳನ್ನು ಎಷ್ಟೋ ದಿನ ಕೇಳಿದ್ದೇನೆ. ಕೆಲವು ವೇಳೆ ಎಷ್ಟು ಗಟ್ಟಿಯಾಗಿ ಉಪನ್ಯಾಸವು ನಡೆಯುತ್ತಿತ್ತೆಂದರೆ, ನೆರೆಮನೆಯವರು ಆ ಮಾತನ್ನು ಕೇಳಿ ಮರುದಿನ ‘ಸ್ವಾಮಿಗಳೆ, ನಿನ್ನೆ ಅಂಥ ರಾತ್ರಿಯಲ್ಲಿ ತಾವು ಯಾರೊಡನೆ ಅಷ್ಟು ಗಟ್ಟಿಯಾಗಿ ಮಾತನಾಡುತ್ತಿದ್ದಿರಿ?’ ಎಂದು ನನ್ನನ್ನು ಕೇಳುತ್ತಿದ್ದರು. ನಾನು ಅವರ ಆ ಪ್ರಶ್ನೆಯನ್ನು ಹೇಗೋ ಮರೆಸಿಬಿಡುತ್ತಿದ್ದೆ. ಅದು ಒಂದು ಅದ್ಭುತದ ಸಂಗತಿ.

ಶಿಷ್ಯನು ಸ್ವಾಮಿಗಳ ಮಾತನ್ನು ಕೇಳಿ ಬೆರಗಾಗಿ ಯೋಚಿಸುತ್ತ ಯೋಚಿಸುತ್ತ ಹೀಗೆಂದನು: “ಮಹಾಶಯರೆ, ಹಾಗಾದರೆ ನನಗೇನು ತೋರುತ್ತದೆ ಎಂದರೆ ತಾವೇ ಸೂಕ್ಷ್ಮ ದೇಹದಿಂದ ಹೀಗೆ ಉಪನ್ಯಾಸ ಮಾಡುತ್ತಿದ್ದಿರಿ; ಮತ್ತು ಸ್ಥೂಲದೇಹದಲ್ಲಿ ಆಗಾಗ್ಗೆ ಅದರ ಪ್ರತಿಧ್ವನಿ ಹೊರಗೆ ಬರುತ್ತಿತ್ತು.”

ಇದನ್ನು ಕೇಳಿ ಸ್ವಾಮೀಜಿ “ಇರಬಹುದು" ಎಂದರು. ಆಮೇಲೆ ಅಮೆರಿಕಾದ ಪ್ರಸ್ತಾಪ ಬಂತು. ಸ್ವಾಮೀಜಿ ಹೀಗೆಂದರು: "ಆ ದೇಶದಲ್ಲಿ ಗಂಡಸರಿಗಿಂತಲೂ ಹೆಂಗಸರು ಶಿಕ್ಷಿತರಾದವರು, ವಿಜ್ಞಾನ, ದರ್ಶನ ಇವುಗಳಲ್ಲಿ ಅವರೆಲ್ಲಾ ಪಂಡಿತೆಯರು; ಅದಕ್ಕೋಸ್ಕರವೆ ಅವರು ನನ್ನನ್ನು ಅಷ್ಟು ಗೌರವಿಸುತ್ತಿದ್ದರು. ಗಂಡಸರು ಹಗಲೂ ಇರುಳೂ ದುಡಿಯುತ್ತಾರೆ; ವಿಶ್ರಾಂತಿಗೆ ಅವಕಾಶವಿಲ್ಲ. ಹೆಂಗಸರು ಪಾಠಶಾಲೆಗಳಲ್ಲಿ ಪಾಠಗಳನ್ನು ಹೇಳುವುದು, ಕೇಳುವುದು ಇವುಗಳನ್ನೆಲ್ಲಾ ಮಾಡುತ್ತ ಮಹಾ ಪಂಡಿತೆಯರಾಗಿಬಿಟ್ಟಿದ್ದಾರೆ. ಅಮೆರಿಕಾದಲ್ಲಿ ಯಾವ ಕಡೆ ನೋಡಿದರೂ ಕೇವಲ ಹೆಂಗಸರ ಪಾರುಪತ್ಯವೆ.

ಶಿಷ್ಯ: ಒಳ್ಳೆಯದು, ಮಹಾಶಯರೆ, ಅತಿಪಕ್ಷಪಾತಿಗಳಾದ ಕ್ರೈಸ್ತರು ಅಲ್ಲಿ ತಮಗೆ ವಿರೋಧಿಗಳಾಗಲಿಲ್ಲವೆ?

ಸ್ವಾಮಿಜಿ: ಆಗದೆ ಮತ್ತೇನು? ಅಲ್ಲದೆ ಜನರು ನನಗೆ ತುಂಬ ಗೌರವವನ್ನು ತೋರಿಸುವುದಕ್ಕೆ ಆರಂಭಿಸಲು, ಪಾದ್ರಿಗಳು ನನ್ನನ್ನು ಬಲವಾಗಿ ಬೆನ್ನಟ್ಟಿದರು; ನನ್ನ ಮೇಲೆ ಬೇಕಾದಷ್ಟು ಅಪವಾದವನ್ನು ಪತ್ರಿಕೆಗಳಲ್ಲಿ ಬರೆದು ಗದ್ದಲ ಹತ್ತಿಸಿದರು. ಎಷ್ಟೋ ಜನರು ಅದನ್ನು ವಿರೋಧಿಸಬೇಕೆಂದು ನನಗೆ ಹೇಳುತ್ತಿದ್ದರು. ಆದರೆ ನಾನು ಮಾತ್ರ ಸ್ವಲ್ಪವೂ ಒಪ್ಪುತ್ತಿರಲಿಲ್ಲ. ನನ್ನ ದೃಢವಾದ ನಂಬುಗೆ ಏನೆಂದರೆ, ಬರಿಯ ತೀಕ್ಷ್ಣ ಬುದ್ಧಿಯಿಂದ ಜಗತ್ತಿನಲ್ಲಿ ಯಾವ ಮಹತ್ಕಾರ್ಯವೂ ಆಗುವುದಿಲ್ಲ. ಆದ್ದರಿಂದಲೇ ಈ ಅಶ್ಲೀಲವಾದ ಕೆಟ್ಟ ಅಪವಾದಗಳನ್ನು ಕಿವಿಗೆ ಹಾಕಿಕೊಳ್ಳದೆ ಮೆಲ್ಲ ಮೆಲ್ಲಗೆ ನನ್ನ ಕೆಲಸವನ್ನು ಮಾಡುತ್ತ ಹೋಗುತ್ತಿದ್ದೆನು. ಅನೇಕ ವೇಳೆ ಯಾರು ನನ್ನ ಮೇಲೆ ಇಲ್ಲದ ಅಪವಾದವನ್ನು ಹೊರಿಸಿದ್ದರೂ ಅವರು ಮನಸ್ಸಿನಲ್ಲಿ ನೊಂದುಕೊಂಡು ನನಗೆ ಶರಣು ಹೋಗುತ್ತಿದ್ದರು ಮತ್ತು ತಾವೇ ಪತ್ರಿಕೆಯಲ್ಲಿ ಪ್ರತಿವಾದ ಮಾಡಿ ಕ್ಷಮೆಯನ್ನು ಬೇಡುತ್ತಿದ್ದರು. ಕೆಲಕೆಲವು ವೇಳೆ ಹೀಗೂ ಆಗಿದೆ: ನನ್ನನ್ನು ಯಾರ ಮನೆಗಾದರೂ ಭೋಜನಕ್ಕೆ ಕರೆದಿದ್ದಾರೆಂದು ಗೊತ್ತಾದರೆ, ಯಾರೋ ನನ್ನ ಮೇಲೆ ಸುಳ್ಳು ಅಪವಾದಗಳನ್ನು ಆ ಮನೆಯವರಿಗೆ ತಿಳಿಸುತ್ತಿದ್ದರು; ಅದನ್ನು ಕೇಳಿ ಅವರು ಮನೆಯ ಬಾಗಿಲನ್ನು ಹಾಕಿಕೊಂಡು ಎಲ್ಲಿಯೋ ಹೊರಟುಹೋಗಿಬಿಡುತ್ತಿದ್ದರು; ನಾನು ಒಪ್ಪಿ ಕೊಂಡಿದ್ದಂತೆ ಹೋಗಿ ನೋಡುತ್ತೇನೆ! - ಎಲ್ಲವೂ ಬಿಮ್ಮೆನ್ನುತ್ತಿದೆ. ಯಾರೂ ಇಲ್ಲ! ಆಮೇಲೆ ಕೆಲವು ದಿನದ ಮೇಲೆ ಅವರೇ, ನಿಜವಾದ ವಿಷಯ ತಿಳಿದುಕೊಂಡು, ಮನಸ್ಸಿನಲ್ಲಿ ಪರಿತಾಪಪಟ್ಟುಕೊಂಡು ನನ್ನ ಶಿಷ್ಯರಾಗುವುದಕ್ಕೆ ಬರುತಿದ್ದರು. ಏನೆಂದು ಕೊಂಡಿದ್ದೀಯಪ್ಪಾ, ಸಂಸಾರದಲ್ಲಿ ಎಲ್ಲಾ ಪ್ರಾಪಂಚಿಕ ಮೋಹವೆ; ನಿಜವಾದ ಸತ್ಸಾಹಸಿಯೂ ಜ್ಞಾನಿಯೂ ಈ ಪ್ರಾಪಂಚಿಕ ಮೋಹದಲ್ಲಿ ಮೈಮರೆತು ಹೋಗುತ್ತಾನೆಯೆ? ಪ್ರಪಂಚ ತನಗೆ ಇಷ್ಟಬಂದದ್ದನ್ನು ಹೇಳಿಕೊಳ್ಳಲಿ, ನನ್ನ ಕರ್ತವ್ಯ ಕಾರ್ಯವನ್ನು ನಾನು ಮಾಡಿಕೊಂಡು ಹೋಗುತ್ತೇನೆ - ಇದೇ ವೀರನ ಕಾರ್ಯವೆಂದು ತಿಳಿದುಕೊ. ಹಾಗಲ್ಲದೆ ಇವನೇನು ಎನ್ನುತ್ತಾನೆ, ಅವನೇನು ಬರೆಯುತ್ತಾನೆ, ಎಂಬಿವುಗಳನ್ನೇ ಹಗಲೂ ಇರುಳೂ ಹಿಡಿದುಕೊಂಡು ಕುಳಿತರೆ, ಪ್ರಪಂಚದಲ್ಲಿ ಯಾವ ಮಹತ್ಕಾರ್ಯವೂ ಆಗುವುದಿಲ್ಲ. ಈ ಶ್ಲೋಕವನ್ನು ಅರಿಯೆಯಾ? -

\begin{verse}
ನಿಂದಂತು ನೀತಿನಿಪುಣಾ ಯದಿ ವಾ ಸ್ತುವಂತು\\ಲರ್ಕ್ಷೀಃ ಸಮಾವಿಶತು ಗಚ್ಛತು ವಾ ಯಥೇಷ್ಟಮ್~।\\ಅದ್ಯೈವ ವಾ ಮರಣಮಸ್ತು ಯುಗಾಂತರೇ ವಾ\\ನ್ಯಾಯ್ಯಾತ್ ಪಥಃ ಪ್ರವಿಚಲಂತಿ ಪದಂ ನ ಧೀರಾಃ~॥ (ಭರ್ತೃಹರಿ; ನೀತಿಶತಕ)
\end{verse}

ಜನರು ನಿನ್ನನ್ನು ಸ್ತುತಿಸಲಿ ನಿಂದಿಸಲಿ, ನಿನ್ನ ಮೇಲೆ ಲಕ್ಷ್ಮಿಯ ಕೃಪೆಯಾಗಲಿ ಆಗದಿರಲಿ, ಇಂದೇ ಮರಣವಾಗಲಿ ಮತ್ತೊಂದು ಯುಗದಲ್ಲಿ ಆಗಲಿ, ನ್ಯಾಯವಾದ ಮಾರ್ಗವನ್ನು ಬಿಟ್ಟು ಹೋಗಕೂಡದು. ಎಷ್ಟೆಷ್ಟೋ ಬಿರುಗಾಳಿ ಸುಂಟರಗಾಳಿಗಳನ್ನು ತಟಾಯಿಸಿದ ಮೇಲೆ ಶಾಂತಿ ರಾಜ್ಯವು ದೊರಕುವುದು! ಯಾರಾರು ಎಷ್ಟೆಷ್ಟು ದೊಡ್ಡವರಾಗಿರುತ್ತಾರೆಯೋ ಅವರವರು ಅಷ್ಟಷ್ಟು ಕಠಿನವಾದ ಕಷ್ಟನಿಷ್ಟುರಗಳ ಪರೀಕ್ಷೆಗೆ ಸಿಕ್ಕಿರುತ್ತಾರೆ. ಈ ಪರೀಕ್ಷೆಯೆಂಬ ಒರೆಗಲ್ಲಿನಲ್ಲಿ ಅವರ ಜೀವನವನ್ನು ಉಜ್ಜಿ ನೋಡಿ ಆಮೇಲೆ ಅವರನ್ನು ಜಗತ್ತು ದೊಡ್ಡವರೆಂದು ಒಪ್ಪಿಕೊಳ್ಳುತ್ತದೆ. ಯಾರು ಅಂಜುಬುರುಕರೊ ಕೀಳು ಜನರೊ ಅವರೇ ಸಮುದ್ರದ ಅಲೆಗಳನ್ನು ನೋಡಿ, ತೀರದಲ್ಲಿ ದೋಣಿಯನ್ನು ಮುಳುಗಿಸುವರು. ಮಹಾ ವೀರನಾದವನು ಯಾವುದನ್ನಾದರೂ ಲಕ್ಷ್ಯ ಮಾಡುವನೇನು? ಏನು ಆದರೆ ಅದು ಆಗಿ ಹೋಗಲಿ, ನನ್ನ ಆದರ್ಶವನ್ನು, ಇಷ್ಟವನ್ನು ಸಾಧಿಸಿಯೇ ತೀರಬೇಕು – ಇದೇ ಪುರುಷಕಾರವೆಂಬುದು. ಇದಿಲ್ಲದಿದ್ದರೆ ನೂರು ದೈವದಿಂದಲೂ ನಿನ್ನ ಜಡತ್ವವನ್ನು ಹೋಗಲಾಡಿಸಲಾಗುವುದಿಲ್ಲ.

ಶಿಷ್ಯ: ಆದರೆ ದೈವದ ಮೇಲೆ ಭಾರಹಾಕುವುದು ದೌರ್ಬಲ್ಯದ ಚಿಹ್ನೆಯೆ?

ಸ್ವಾಮೀಜಿ: ದೇವರಲ್ಲಿ ಪೂರ್ಣ ಶರಣಾಗತಿ ಮಾನವಸಾಧನೆಯ ಆತ್ಯಂತಿಕ ಸ್ಥಿತಿ ಎಂದು ಶಾಸ್ತ್ರಗಳು ಹೇಳುತ್ತವೆ. ಆದರೆ ನಮ್ಮ ದೇಶದಲ್ಲಿ ದೈವ ದೈವ ಎಂದುಕೊಳ್ಳುತ್ತಿರುವುದು ಮೃತ್ಯುವಿನ ಚಿಹ್ನೆ, ಮಹಾ ಕಾಪುರುಷತೆಯ ಪರಿಣಾಮ. ಒಬ್ಬ ವಿಚಿತ್ರ ಈಶ್ವರನನ್ನು ಕಲ್ಪಿಸಿಕೊಂಡು ಅವನ ತಲೆಯ ಮೇಲೆ ತಮ್ಮ ದೋಷಗಳನ್ನೆಲ್ಲಾ ಹೊರಿಸುವ ಪ್ರಯತ್ನ ಮಾತ್ರ. ಪರಮಹಂಸರು ಹೇಳುತ್ತಿದ್ದ ಆ ಗೋಹತ್ಯಾಪಾಪದ ಕಥೆಯನ್ನು ಕೇಳಿದ್ದೀಯಷ್ಟೆ?\footnote{ಒಬ್ಬನು ಸುಂದರವಾದ ತೋಟವೊಂದನ್ನು ಬೆಳೆಸಿದ್ದನು. ಒಂದು ದಿನ ಹಸುವೊಂದು ನುಗ್ಗಿ ಅದನ್ನೆಲ್ಲ ಹಾಳು ಮಾಡಿತು. ಯಜಮಾನನಿಗೆ ಕೋಪ ಬಂದು ಹಸುವಿಗೆ ನಾಲ್ಕು ಬಾರಿಸಿದ. ಹಸು ಸತ್ತೇ ಹೋಯಿತು. ಆಗ ತನಗೆ ಬರುವ ಮಹಾ ಪಾಪವನ್ನು ತಪ್ಪಿಸಿಕೊಳ್ಳುವುದಕ್ಕಾಗಿ ಒಂದು ಉಪಾಯ ಹೂಡಿದ. ಕೈಯ ಅಭಿಮಾನ ದೇವತೆ ಇಂದ್ರ. ಆದ್ದರಿಂದ ತಪ್ಪು ಮಾಡಿದವನು ಇಂದ್ರ ಎಂದು ಹೇಳಿಬಿಟ್ಟ. ಇವನ ಕಪಟವನ್ನು ತಿಳಿದ ಇಂದ್ರ ಅವನ ಬಳಿಗೆ ಬ್ರಾಹ್ಮಣ ವೇಷದಲ್ಲಿ ಬಂದ. ಯಜಮಾನನಿಗೆ ಅನೇಕ ಪ್ರಶ್ನೆಗಳನ್ನು ಹಾಕಿ ತೋಟದ ಪ್ರತಿಯೊಂದು ಭಾಗದ ಸೊಗಸಿಗೂ ಆ ಯಜಮಾನನೇ ಕಾರಣ ಎಂಬುದನ್ನು ಅವನ ಬಾಯಿಂದಲೇ ಹೊರಡಿಸಿದ. ಆಗ ಇಂದ್ರನು ಹೇಳಿದ: 'ಅಯ್ಯಾ, ಇಲ್ಲಿನ ಒಳ್ಳೆಯದೆಲ್ಲ ಆಗಿರುವುದು ನಿನ್ನಿಂದಲೇ. ಆದರೆ ಹಸುವನ್ನು ಕೊಂದವನು ಮಾತ್ರ ಇಂದ್ರ ಅಲ್ಲವೇ?'} ಆ ಗೋಹತ್ಯಾಪಾಪವನ್ನು ಕೊನೆಗೆ ತೋಟದ ಯಜಮಾನನೆ ಅನುಭವಿಸಿ ಸಾಯಬೇಕಾಯಿತು. ಈಗಿನ ಕಾಲದಲ್ಲಿ ಎಲ್ಲರೂ ‘ಯಥಾ ನಿಯುಕ್ತೋಽಸ್ಮಿ ತಥಾ ಕರೋಮಿ’ ಎನ್ನುತ್ತಾ ಪಾಪ ಪುಣ್ಯಗಳೆರಡನ್ನೂ ಈಶ್ವರನ ತಲೆಯಮೇಲೆ ಹೊರಿಸಿಬಿಡುತ್ತಾರೆ. ತಾವೇನೊ ಪದ್ಮ ಪತ್ರದ ಮೇಲಿರುವ ನೀರಿನ ಹಾಗೆ! ಸರ್ವದಾ ಈ ಭಾವದಲ್ಲಿ ಇರಬಲ್ಲನಾದರೆ ಅವನು ಮುಕ್ತನೇ; ಆದರೆ ಒಳ್ಳೆಯ ಕಾಲದಲ್ಲಿ ‘ನಾನು’ ಕೆಟ್ಟಕಾಲದಲ್ಲಿ ‘ನೀನು’! - ನಿನ್ನ ದೈವನಿರ್ಭರತೆ ವಿಚಿತ್ರ; ಪೂರ್ಣ ಪ್ರೇಮ ಅಥವಾ ಜ್ಞಾನವಿಲ್ಲದಿದ್ದರೆ ನಿರ್ಭರಸ್ಥಿತಿ ಬರಲೇ ಆರದು. ಯಾರಿಗೆ ನಿಜವಾದ ನಿರ್ಭರತೆಯು ಉಂಟಾಗಿದೆಯೊ ಅವರಿಗೆ ಒಳ್ಳೆಯದು ಕೆಟ್ಟದ್ದು ಎಂಬ ಬುದ್ಧಿಯೇ ಇರುವುದಿಲ್ಲ - ನಮ್ಮಲ್ಲಿ (ಶ‍್ರೀರಾಮಕೃಷ್ಣರ ಶಿಷ್ಯರಲ್ಲಿ) ಈಗ ಇರುವ, ಇಂಥ ಸ್ಥಿತಿಯ ಉಜ್ಜ್ವಲ ದೃಷ್ಟಾಂತವೆಂದರೆ - ನಾಗಮಹಾಶಯರು.

ಹೀಗೆಂದು ಹೇಳುತ್ತ ಹೇಳುತ್ತ, ನಾಗಮಹಾಶಯರ ಪ್ರಸ್ತಾವಕ್ಕೆ ಆರಂಭವಾಯಿತು.

ಸ್ವಾಮೀಜಿ: ಎಂಥ ಅನುರಾಗಿಯಾದ ಭಕ್ತರು! ಅಂಥವರು ಮತ್ತೊಬ್ಬರಿದ್ದಾರೆಯೆ? ಆಹಾ! ಅವರ ದರ್ಶನವು ಮತ್ತೆ ಆಗುವುದು ಯಾವಾಗ?

ಶಿಷ್ಯ: ಅವರು ಶೀಘ್ರವಾಗಿಯೇ ತಮ್ಮ ದರ್ಶನ ಪಡೆದುಕೊಳ್ಳುವುದಕ್ಕೋಸ್ಕರ ಕಲ್ಕತ್ತಕ್ಕೆ ಬರುತ್ತಾರೆಂದು ಶ‍್ರೀಮಾತೆ (ನಾಗಮಹಾಶಯರ ಪತ್ನಿ) ನನಗೆ ಚೀಟಿ ಬರೆದು ಕಳುಹಿಸಿದ್ದಾರೆ.

ಸ್ವಾಮೀಜಿ: ಪರಮಹಂಸರು ಅವರನ್ನು ಜನಕರಾಜನೊಡನೆ ಹೋಲಿಸುತ್ತಿದ್ದರು. ಅಂಥ ಜಿತೇಂದ್ರಿಯನ ದರ್ಶನದ ಮಾತು ಹಾಗಿರಲಿ, ವರ್ತಮಾನವೂ ಕೇಳಿಬರುವುದಿಲ್ಲ. ಅವರ ಸಂಗವನ್ನು ತುಂಬಾ ಮಾಡು. ಪರಮಹಂಸರ ಅಂತರಂಗಕ್ಕೆ ಸೇರಿದವರಲ್ಲಿ ಅವರೂ ಒಬ್ಬರು.

ಶಿಷ್ಯ: ಮಹಾಶಯರೆ, ಆ ಪ್ರಾಂತದಲ್ಲಿ ಅನೇಕರು ಅವರನ್ನು ಹುಚ್ಚರೆನ್ನುತ್ತಾರೆ. ನಾನು ಮಾತ್ರ ಮೊದಲು ದರ್ಶನ ಮಾಡಿದ ದಿವಸದಿಂದಲೂ ಅವರನ್ನು ಮಹಾಪುರುಷರೆಂದು ಮನಸ್ಸಿನಲ್ಲಿ ತಿಳಿದುಕೊಂಡಿದ್ದೇನೆ. ಅವರಿಗೆ ನನ್ನ ಮೇಲೆ ತುಂಬ ವಿಶ್ವಾಸ ಮತ್ತು ಕೃಪೆ.

ಸ್ವಾಮೀಜಿ: ಅಂಥ ಮಹಾಪುರುಷನ ಸಂಗವನ್ನು ಪಡೆದುಕೊಂಡಿದ್ದೀಯೆ, ಇನ್ನು ಬೇರೆ ಅಂಜಿಕೆ ಎಂಥಾದ್ದಯ್ಯಾ? ಬಹು ಜನ್ಮಗಳ ತಪಸ್ಸು ಇದ್ದರೇ ಅಂಥ ಮಹಾಪುರುಷರ ಸಂಗವೆಲ್ಲಾ ದೊರಕುವುದು. ನಾಗಮಹಾಶಯರು ಮನೆಯಲ್ಲಿ ಯಾವ ರೀತಿಯಲ್ಲಿ ಇರುತ್ತಾರೆ?

ಶಿಷ್ಯ: ಮಹಾಶಯರೆ, ಅವರ ಕೆಲಸಕಾರ್ಯಗಳೊಂದನ್ನೂ ನೋಡಿಲ್ಲ. ಕೇವಲ ಅತಿಥಿ ಸೇವೆಯನ್ನು ಮಾಡಿಕೊಂಡು ಇರುತ್ತಾರೆ. ಪಾಲ್‌ಬಾಬುಗಳು ಕೊಡುವ ಕೆಲವು ರೂಪಾಯಿಗಳನ್ನು ಬಿಟ್ಟರೆ ಹೊಟ್ಟೆ ಬಟ್ಟೆಗಳಿಗೆ ಬೇರೆ ಆದಾಯವಿಲ್ಲ; ಆದರೆ ಖರ್ಚುಪಟ್ಟಿಯನ್ನು ನೋಡಿದರೆ ಒಬ್ಬ ದೊಡ್ಡ ಮನುಷ್ಯರ ಮನೆಯಲ್ಲಿ ಹೇಗೊ ಹಾಗೆ! ಆದರೆ ತಮ್ಮ ಭೋಗಕ್ಕಾಗಿ ಒಂದು ಪಾವಲಿಯ ಖರ್ಚೂ ಇಲ್ಲ - ಅಷ್ಟು ಖರ್ಚೆಲ್ಲವೂ ಕೇವಲ ಪರಸೇವಾರ್ಥವಾಗಿ. ಸೇವೆ ಸೇವೆ - ಇದೇ ಅವರ ಜೀವನದ ಮಹಾ ವ್ರತವೆಂದು ತೋರುತ್ತದೆ. ಪ್ರತಿ ಪ್ರಾಣಿಯಲ್ಲಿಯೂ ಆತ್ಮವನ್ನು ದರ್ಶನಮಾಡಿ ಅವರು ಅಭೇದ ಜ್ಞಾನದಿಂದ ಜಗತ್ತಿನ ಸೇವೆಯನ್ನು ಮಾಡುವುದರಲ್ಲಿ ಆಸಕ್ತರಾಗಿದ್ದಾರೆಂದು ತೋರುತ್ತದೆ. ಸೇವೆಗೋಸ್ಕರ ತಮ್ಮ ಶರೀರವನ್ನು ಶರೀರವೆಂದೇ ಎಣಿಸಿಲ್ಲ - ಅದನ್ನು ಮರೆತೇಹೋಗಿರುವಂತೆ ತೋರುತ್ತದೆ. ನಿಜವಾದ ಶರೀರಜ್ಞಾನ ಅವರಿಗೆ ಇದೆಯೊ ಇಲ್ಲವೋ ಆ ವಿಷಯದಲ್ಲಿ ನನಗೆ ಸಂದೇಹ. ತಾವು ಯಾವ ಸ್ಥಿತಿಯನ್ನು ಜ್ಞಾನಾತೀತ ಅವಸ್ಥೆ ಎಂದು ಹೇಳುತ್ತೀರೊ, ಆ ಸ್ಥಿತಿಯಲ್ಲಿಯೇ ಅವರು ಯಾವಾಗಲೂ ಇರುತ್ತಾರೆಂದು ತೋರುತ್ತದೆ.

ಸ್ವಾಮೀಜಿ: ಹಾಗಾಗದೇ ಮತ್ತೇನು? ಪರಮಹಂಸರು ಅವರನ್ನು ಎಷ್ಟು ನಂಬುತ್ತಿದ್ದರು! ನಿಮ್ಮ ಪೂರ್ವ ಬಂಗಾಳ ದೇಶದಲ್ಲಿ ಈ ಸಾರಿ ಪರಮಹಂಸರ ಸಹವಾಸಿಯೊಬ್ಬರು ಅವತರಿಸಿದ್ದಾರೆ. ಅವರ ಬೆಳಕಿನಿಂದ ಪೂರ್ವ ಬಂಗಾಳ ದೇಶದಲ್ಲಿ ಬೆಳಕು ಹರಿದಿದೆ.

\chapter[ಅಧ್ಯಾಯ ೧೬]{ಅಧ್ಯಾಯ ೧೬\protect\footnote{\engfoot{C.W, Vol. VII, P. 129}}}

\begin{center}
ಸ್ಥಳ: ಬೇಲೂರು ಮಠ, (ಬಾಡಿಗೆ ಕಟ್ಟಡ), ವರ್ಷ: ಕ್ರಿ.ಶ. ೧೮೯೮ ನವೆಂಬರ್.
\end{center}

ಸ್ವಾಮಿಜಿ ಕಾಶ್ಮೀರದಿಂದ ಹಿಂತಿರುಗಿ ಬಂದು ಇಂದಿಗೆ ಎರಡು ಮೂರು ದಿನಗಳಾಗಿದ್ದವು. ದೇಹದಲ್ಲಿ ಸ್ವಲ್ಪ ಸ್ವಸ್ಥವಿಲ್ಲ. ಶಿಷ್ಯನು ಮಠಕ್ಕೆ ಬಂದಕೂಡಲೆ ಬ್ರಹ್ಮಾನಂದ ಸ್ವಾಮಿಗಳು “ಕಾಶ್ಮೀರದಿಂದ ಹಿಂತಿರುಗಿ ಬಂದಾರಭ್ಯ ಸ್ವಾಮೀಜಿ ಯಾರೊಡನೆಯೂ ಯಾವ ಮಾತುಕಥೆಯನ್ನೂ ಆಡಿಲ್ಲ; ಸ್ತಬ್ಧರಾಗಿ ಕುಳಿತು ಬಿಟ್ಟಿದ್ದಾರೆ. ನೀನು ಸ್ವಾಮೀಜಿ ಹತ್ತಿರ ಮಾತುಕಥೆಗಳನ್ನು ಆಡಿ ಅವರ ಮನಸ್ಸನ್ನು ಕೆಳಕ್ಕೆ ಇಳಿಸಲು ಪ್ರಯತ್ನ ಮಾಡಿ ನೋಡು" ಎಂದರು.

ಶಿಷ್ಯನು ಮೇಲೆ ಸ್ವಾಮೀಜಿ ಕೊಠಡಿಗೆ ಹೋಗಿ ನೋಡಿದನು - ಸ್ವಾಮೀಜಿ ಮುಕ್ತ ಪದ್ಮಾಸನದಲ್ಲಿ ಪೂರ್ವಮುಖವಾಗಿ ಗಂಭೀರ ಧ್ಯಾನಮಗ್ನರಾಗಿದ್ದಂತೆ ಕುಳಿತು ಕೊಂಡಿದ್ದರು. ಮುಖದಲ್ಲಿ ನಗುವಿಲ್ಲ, ಪ್ರದೀಪ್ತ ನಯನಗಳಲ್ಲಿ ಬಹಿರ್ಮುಖ ದೃಷ್ಟಿಯಿಲ್ಲ - ಒಳಗೆ ಏನನ್ನೊ ನೋಡುತ್ತಿದ್ದಂತೆ ತೋರುತ್ತಿತ್ತು. ಶಿಷ್ಯನನ್ನು ನೋಡಿದ ಕೂಡಲೆ “ಬಾರಯ್ಯ, ಕೂತುಕೊ" ಎಂದರು - ಅಷ್ಟೆ. ಸ್ವಾಮಿಗಳ ಎಡಗಣ್ಣಿನ ಒಳಭಾಗ ಕೆಂಪೇರಿದ್ದದ್ದನ್ನು ನೋಡಿ ಶಿಷ್ಯನು “ತಮ್ಮ ಕಣ್ಣಿನ ಒಳಭಾಗವೇಕೆ ಕೆಂಪೇರಿದೆ" ಎಂದು ಕೇಳಿದನು. ಸ್ವಾಮೀಜಿ “ಅದೇನೂ ಆಗಿಲ್ಲ” ಎಂದು ಹೇಳಿ ಪುನಃ ಸ್ಥಿರವಾಗಿ ಕುಳಿತುಕೊಂಡರು. ಬಹಳ ಹೊತ್ತು ಕುಳಿತುಕೊಂಡರೂ ಸ್ವಾಮಿಜಿ ಯಾವ ಮಾತನ್ನೂ ಆಡಲಿಲ್ಲ. ಆಗ ಶಿಷ್ಯನು ಕಳವಳಕೊಂಡು ಸ್ವಾಮೀಜಿ ಪಾದಪದ್ಮವನ್ನು ಮುಟ್ಟಿ “ಅಮರನಾಥದಲ್ಲಿ ಏನೇನು ನೋಡಿದರೆ ಅದನ್ನು ನನಗೆ ಹೇಳುವುದಿಲ್ಲವೆ?” ಎಂದನು. ಪಾದಸ್ಪರ್ಶದಿಂದ ಸ್ವಾಮಿಜಿಯವರು ಸ್ವಲ್ಪ ಚಕಿತರಾದರು - ಸ್ವಲ್ಪ ಬಹಿರ್ದೃಷ್ಟಿಯಾದಂತೆ ತೋರಿತು. “ಅಮರನಾಥ ದರ್ಶನದಾರಭ್ಯ ನನ್ನ ಮನಸ್ಸಿನಲ್ಲಿ ಇಪ್ಪತ್ತುನಾಲ್ಕು ಘಂಟೆಯ ಹೊತ್ತೂ ಶಿವನು ತುಂಬಿಕೊಂಡಿರುವಂತಿದೆ; ಏನು ಮಾಡಿದರೂ ಇಳಿದುಬರುವಂತಿಲ್ಲ" ಎಂದರು. ಶಿಷ್ಯನು ಕೇಳಿ ಬೆರಗಾದನು.

ಸ್ವಾಮೀಜಿ: ಅಮರನಾಥದಲ್ಲಿಯೂ ಆಮೇಲೆ ಕ್ಷೀರಭವಾನಿ ಮಂದಿರದಲ್ಲಿಯೂ ತುಂಬ ತಪಸ್ಸು ಮಾಡಿದೆ. ಹೋಗಿ ತಂಬಾಕನ್ನು ಸಿದ್ಧಮಾಡಿಕೊಂಡು ಬಾ. ಎಲ್ಲವನ್ನು ನಿನಗೆ ಹೇಳುತ್ತೇನೆ.

ಶಿಷ್ಯನು ಪ್ರಫುಲ್ಲವಾದ ಮನಸ್ಸಿನಿಂದ ಸ್ವಾಮಿಜಿಯ ಆಜ್ಞೆಯನ್ನು ಶಿರಸಾವಹಿಸಿ ತಂಬಾಕನ್ನು ಸಿದ್ಧಪಡಿಸಿಕೊಟ್ಟನು. ಸ್ವಾಮಿಜಿ ಮೆಲ್ಲ ಮೆಲ್ಲಗೆ ಧೂಮಪಾನ ಮಾಡುತ್ತ ಹೇಳತೊಡಗಿದರು: ಅಮರನಾಥಕ್ಕೆ ಹೋಗುವಾಗ ಪರ್ವತದ ಒಂದು ಕಡಿದಾದ ತಿಟ್ಟನ್ನು ಹತ್ತಿ ಹೋದೆ; ಆ ಮಾರ್ಗದಲ್ಲಿ ಯಾತ್ರಿಕರು ಯಾರೂ ಹೋಗುವುದಿಲ್ಲ; ಬೆಟ್ಟದ ಜನರು ಮಾತ್ರ ಆ ಮಾರ್ಗದಲ್ಲಿ ಹೋಗಿ ಬರುತ್ತಾರೆ. ನನಗೇನೋ ಈ ದಾರಿಯಲ್ಲಿಯೇ ಹೋಗಬೇಕೆಂಬ ಹಟ ಹುಟ್ಟಿತು. ಹಾಗೆಯೇ ಹೊರಟುಹೋದೆ. ಆ ಪ್ರಯತ್ನದಿಂದ ಶರೀರ ಸ್ವಲ್ಪ ಆಯಾಸಗೊಂಡಿದೆ. ಅಲ್ಲಿ ಎಷ್ಟು ಕೊರೆತವೆಂದರೆ ಮೈಯಲ್ಲಿ ಸೂಜಿ ಚುಚ್ಚಿದಂತಾಗುತ್ತದೆ.

ಶಿಷ್ಯ: ಬೆತ್ತಲೆಯಾಗಿ ಹೋಗಿ ಅಮರನಾಥನನ್ನು ದರ್ಶನ ಮಾಡಬೇಕೆಂದು ಕೇಳಿದ್ದೇನೆ; ಸಂಗತಿ ನಿಜವೇ?

ಸ್ವಾಮೀಜಿ: ಹೌದು; ನಾನೂ ಕೌಪೀನವನ್ನು ಮಾತ್ರ ಹಾಕಿಕೊಂಡು ಬೂದಿ ಬಳಿದುಕೊಂಡು ಗುಹೆಯೊಳಕ್ಕೆ ಹೋದೆನು. ಆಗ ಚಳಿ ಸೆಕೆ ಯಾವುದೂ ತೋರಲಿಲ್ಲ. ಆದರೆ ಮಂದಿರದ ಹೊರಕ್ಕೆ ಬಂದಾಗ ಚಳಿಯಿಂದ ಜಡನಾಗಿ ಹೋದೆ.

ಶಿಷ್ಯ: ಪಾರಿವಾಳ ಹಕ್ಕಿಗಳನ್ನು ನೋಡಿದಿರೇನು? ಅಲ್ಲಿ ಚಳಿಯಿಂದ ಯಾವ ಜೀವಜಂತುಗಳೂ ವಾಸಮಾಡುವಂತೆ ಕಾಣಬರುವುದಿಲ್ಲವೆಂದೂ, ಕೇವಲ ಬಿಳಿಯ ಪಾರಿವಾಳ ಪಕ್ಷಿಗಳ ಒಂದು ಹಿಂಡು ಎಲ್ಲಿಂದಲೋ ಆಗಾಗ್ಗೆ ಬರುತ್ತಿರುತ್ತದೆಯೆಂದೂ ಕೇಳಿದ್ದೇನೆ.

ಸ್ವಾಮೀಜಿ: ಹೌದು; ಮೂರು ನಾಲ್ಕು ಬಿಳಿಯ ಪಾರಿವಾಳಗಳನ್ನು ನೋಡಿದೆ. ಅವು ಗುಹೆಯಲ್ಲಿರುವವೊ ಇಲ್ಲದಿದ್ದರೆ ಹತ್ತಿರದಲ್ಲಿರುವ ಪರ್ವತದಲ್ಲಿರುವವೋ ಅದನ್ನು ತಿಳಿದುಕೊಳ್ಳಲಾಗಲಿಲ್ಲ.

ಶಿಷ್ಯ; ಮಹಾಶಯರೆ, ಗುಹೆಯಿಂದ ಹೊರಕ್ಕೆ ಬಂದು ಬಿಳಿಯ ಪಾರಿವಾಳದ ಹಕ್ಕಿಗಳನ್ನು ನೋಡಿದರೆ ನಿಜವಾಗಿಯೂ ಶಿವದರ್ಶನವಾಯಿತೆಂದು ಅರ್ಥ ಎಂದು ಜನರು ಹೇಳುವುದನ್ನು ಕೇಳಿದ್ದೇನೆ.

ಸ್ವಾಮೀಜಿ: ಪಾರಿವಾಳಗಳನ್ನು ನೋಡಿದರೆ ಯಾವ ಉದ್ದೇಶವನ್ನಿಟ್ಟುಕೊಂಡು ಹೋಗಿರುತ್ತಾನೆಯೋ ಅದು ಈಡೇರುತ್ತದೆಯೆಂದು ಕೇಳಿದ್ದೇನೆ.

ಅನಂತರದಲ್ಲಿ ಸ್ವಾಮಿಜಿಯವರು ಯಾತ್ರಿಕರೆಲ್ಲಾ ಬರುವ ರಸ್ತೆಯಿಂದಲೇ ಶ‍್ರೀನಗರಕ್ಕೆ ಬಂದರು. ಶ‍್ರೀನಗರಕ್ಕೆ ಹಿಂತಿರುಗಿ ಬಂದ ಸ್ವಲ್ಪ ದಿನಕ್ಕೆ ಕ್ಷೀರಭವಾನಿಯ ದರ್ಶನ ಮಾಡುವುದಕ್ಕೆ ಹೋದರು ಮತ್ತು ಏಳು ದಿನ ಅಲ್ಲಿದ್ದುಕೊಂಡು ದೇವಿಯನ್ನು ಉದ್ದೇಶಿಸಿ ಹಾಲಿನಿಂದ ಪೂಜೆ ಮತ್ತು ಹೋಮ ಮಾಡಿದರು. ಪ್ರತಿನಿತ್ಯವೂ ಒಂದು ಮಣ ಹಾಲನ್ನು ನೈವೇದ್ಯ ಮಾಡುತ್ತಲೂ ಇದ್ದರು. ಒಂದು ದಿನ ಪೂಜೆ ಮಾಡುತ್ತ ಮಾಡುತ್ತ, ಸ್ವಾಮಿಜಿ ಮನಸ್ಸಿಗೆ ಬಂದದ್ದೇನೆಂದರೆ, "ಭವಾನೀಮಾತೆ ನಿಜವಾಗಿಯೂ ಇಲ್ಲಿ ಎಷ್ಟು ಕಾಲದಿಂದ ಇದ್ದಾಳೆ! ಹಿಂದೆ ಯವನರು ಬಂದು ಅವಳ ಮಂದಿರವನ್ನು ಧ್ವಂಸಮಾಡಿ ಹೋದರು: ಆದರೂ ಆಗಿನವರು ಏನನ್ನೂ ಮಾಡಲಿಲ್ಲ. ಆಹಾ! ನಾನು ಆಗ ಇದ್ದಿದ್ದರೆ ಅದನ್ನೆಲ್ಲಾ ಸುಮ್ಮನೆ ನೋಡಿಕೊಂಡಿರುತ್ತಿದ್ದೆನೆ?"-ಹೀಗೆಂದು ಭಾವಿಸಿಕೊಳ್ಳುತ್ತ ಅವರ ಮನಸ್ಸಿಗೆ ದುಃಖದಿಂದಲೂ ಕ್ಷೋಭೆಯಿಂದಲೂ ತುಂಬ ನೋವಾಗಲು, ಮಾತೆಯು ಹೀಗೆಂದು ಹೇಳಿದ್ದು ಸ್ಪಷ್ಟವಾಗಿ ಕೇಳಿಸಿತು: “ನನ್ನ ಇಚ್ಛಾನುಸಾರವಾಗಿಯೆ ಯವನರು ನಾಶ ಮಾಡಿದರು. ನಾನು ಪಾಳು ದೇವಾಲಯದಲ್ಲಿ ವಾಸಮಾಡಬೇಕೆಂದು ನನ್ನ ಇಷ್ಟ, ಮನಸ್ಸು ಬಂದರೆ ಈಗ ಇಲ್ಲಿ ಏಳು ಅಂತಸ್ತಿನ ಬಂಗಾರದ ಮಂದಿರವನ್ನು ಎಬ್ಬಿಸಲಾರೆನೇನು? ನೀನು ಏನು ಮಾಡಬಲ್ಲೆ? ನಿನ್ನನ್ನು ನಾನು ರಕ್ಷಿಸುವೆನೊ ಅಥವಾ ನನ್ನನ್ನು ನೀನು ರಕ್ಷಿಸುವೆಯೊ?” ಸ್ವಾಮೀಜಿ ಈ ದೇವವಾಣಿಯನ್ನು ಕೇಳಿದಾರಭ್ಯ ನಾನು ಯಾವ ಸಂಕಲ್ಪವನ್ನೂ ಮಾಡಿಕೊಂಡಿಲ್ಲ. ಮಠಗಿಠಗಳನ್ನು ಮಾಡುವ ಸಂಕಲ್ಪವನ್ನೆಲ್ಲಾ ತ್ಯಜಿಸಿಬಿಟ್ಟಿದ್ದೇನೆ; ಮಾತೆಗೆ ಏನು ಇಷ್ಟವೋ ಅದೇ ಆಗಲಿ ಎಂದರು. ಶಿಷ್ಯನು ಬೆರಗಾಗಿ, ಅವರೇ ಒಂದು ದಿನ ‘ನೀನು ನೋಡುವುದು ಕೇಳುವುದು ಎಲ್ಲಾ ನಿನ್ನೊಳಗೆ ಇರುವ ಆತ್ಮ ಪ್ರತಿಧ್ವನಿ ಮಾತ್ರ; ಹೊರಗೆ ಏನೂ ಇಲ್ಲ’ ಎಂದು ಹೇಳಿರಲಿಲ್ಲವೆ ಎಂದು ಯೋಚಿಸಿಕೊಂಡು, ಮಹಾಶಯರೆ, ತಾವೇ ಹೇಳಿದ್ದೀರಲ್ಲಾ ಈ ದೇವವಾಣಿಗಳೆಲ್ಲಾ ನಮ್ಮ ಅಂತರಂಗದ ಭಾವದ ಬಾಹ್ಯ ಪ್ರತಿಧ್ವನಿ ಮಾತ್ರ ಎಂಬುದಾಗಿ ಎಂದು ಸ್ಪಷ್ಟವಾಗಿ ಕೇಳಿಬಿಟ್ಟನು.

ಸ್ವಾಮೀಜಿ ಗಂಭೀರವಾಗಿ ಹೇಳಿದ್ದೇನೆಂದರೆ - “ಅದು ಒಳಗಿನದೆ ಆಗಲಿ ಹೊರಗಿನದೆ ಆಗಲಿ, ನೀನು ನನ್ನ ಹಾಗೆ ನಿನ್ನ ಕಿವಿಗಳಿಂದಲೇ ಇಂಥ ಅಶರೀರವಾಣಿಯನ್ನು ಕೇಳಿದ್ದರೆ, ಅದು ಸುಳ್ಳೆಂದು ಹೇಳುತ್ತಿದ್ದೆಯೇನು? ದೈವವಾಣಿಯು ನಿಜವಾಗಿಯೂ ಕೇಳಿಸುತ್ತದೆ; ನಾವು ಮಾತುಕಥೆಗಳನ್ನಾಡುತ್ತಿದ್ದೇವಲ್ಲಾ ಹಾಗೆಯೆ ಕೇಳಿಸುತ್ತದೆ."

ಶಿಷ್ಯನು ಎರಡನೆಯ ಮಾತನಾಡದೆ ಸ್ವಾಮಿಗಳ ಮಾತನ್ನು ಶಿರಸಾವಹಿಸಿಕೊಂಡನು; ಏಕೆಂದರೆ ಸ್ವಾಮಿಗಳ ಮಾತಿನಲ್ಲಿ ಇಂಥ ಒಂದು ಅದ್ಭುತಶಕ್ತಿ ಇತ್ತು. ಅದನ್ನು ಒಪ್ಪದೆ ಇರುವುದಕ್ಕೆ ಆಗುತ್ತಿರಲಿಲ್ಲ - ಯುಕ್ತಿ ತರ್ಕಗಳೆಲ್ಲ ಎಲ್ಲಿಯೋ ಕೊಚ್ಚಿಹೋಗಿ ಬಿಡುತ್ತಿದ್ದವು.

ಶಿಷ್ಯನು ಈಗ ಪ್ರೇತಾತ್ಮಗಳ ಪ್ರಸ್ತಾವವನ್ನೆತ್ತಿ ಮಹಾಶಯರೆ, ಭೂತ ಪ್ರೇತಗಳ ವಿಚಾರವನ್ನು ಕೇಳಿದ್ದೇವೆಯಲ್ಲಾ - ಶಾಸ್ತ್ರದಲ್ಲಿಯೂ ಮೇಲಿಂದ ಮೇಲೆ ಅದರ ಸಮರ್ಥನೆ ಕಂಡುಬರುತ್ತದೆಯಲ್ಲಾ - ಅವೆಲ್ಲಾ ನಿಜವಾಗಿಯೂ ಇವೆಯೇ ಎಂದನು.

ಸ್ವಾಮೀಜಿ: ನಿಜವಲ್ಲದೆ ಮತ್ತೇನು? ನೀನು ಯಾವುದನ್ನು ನೋಡಿಲ್ಲವೋ ಅದು ನಿಜವಲ್ಲವೇನು? ನಿನ್ನ ದೃಷ್ಟಿಯಿಂದಾಚೆಗೆ ಎಷ್ಟು ಲಕ್ಷಾಂತರ ಬ್ರಹ್ಮಾಂಡಗಳು ಎಷ್ಟು ದೂರದಲ್ಲಿ ತಿರುಗುತ್ತಿವೆ! ನೀನು ನೋಡಲಾರದ್ದರಿಂದ ಅವುಗಳೇ ಇಲ್ಲವೇನು? ಆದರೆ ಈ ಭೂತಗಳ ವಿಚಾರಕ್ಕೆ ಮನಸ್ಸು ಕೊಡಬೇಡ - ಭೂತ ಪ್ರೇತಗಳೇನೊ ಇವೆಯೆಂದು ಇಟುಕೊ. ನಿನ್ನ ಕೆಲಸವೇನೆಂದರೆ, ಈ ಶರೀರ ಮಧ್ಯದಲ್ಲಿರುವ ಆತ್ಮವನ್ನು ಪ್ರತ್ಯಕ್ಷ ಮಾಡಿಕೊಳ್ಳಬಲ್ಲೆಯಾದರೆ, ಭೂತಪ್ರೇತಗಳು ನಿನ್ನ ದಾಸಾನುದಾಸರಾಗಿಬಿಡುವುವು.

ಶಿಷ್ಯ: ಆದರೆ, ಮಹಾಶಯರೆ, ಅವುಗಳನ್ನು ನೋಡಿದರೆ ಪುನರ್ಜನ್ಮಾದಿಗಳಲ್ಲಿ ತುಂಬಾ ನಂಬಿಕೆಯುಂಟಾಗಿ ಪರಲೋಕದಲ್ಲಿ ಅಪನಂಬಿಕೆ ತಪ್ಪಿಹೋಗುವುದೆಂದು ತೋರುತ್ತದೆ.

ಸ್ವಾಮೀಜಿ: ನೀವೋ ಮಹಾಶೂರರು; ನಿಮ್ಮಂಥವರು ಭೂತ ಪ್ರೇತಗಳನ್ನು ನೋಡಿ ಪರಲೋಕದಲ್ಲಿ ದೃಢವಾದ ನಂಬಿಕೆಯನ್ನಿಟ್ಟುಕೊಳ್ಳಬೇಕೇನು? ಇಷ್ಟು ಶಾಸ್ತ್ರ ವಿಜ್ಞಾನಗಳನ್ನು ಓದಿದ್ದೀರಿ - ಈ ವಿರಾಟ್ ವಿಶ್ವದ ಎಷ್ಟೋ ಗೂಢತತ್ತ್ವಗಳನ್ನು ತಿಳಿದುಕೊಂಡಿದ್ದೀರಿ - ಇದರ ಮೇಲೆಯೂ ಆತ್ಮಜ್ಞಾನವನ್ನು ಪಡೆಯುವುದಕ್ಕಾಗಿ ಭೂತಪ್ರೇತಗಳನ್ನು ನೋಡಬೇಕೆ? ಛಿ ಛಿ!

ಶಿಷ್ಯ: ಒಳ್ಳೆಯದು, ಮಹಾಶಯರೆ, ತಾವು ಭೂತಪ್ರೇತಗಳನ್ನು ಯಾವಾಗಲಾದರೂ ಕಣ್ಣಾರೆ ನೋಡಿದ್ದೀರಾ?

ಸ್ವಾಮೀಜಿ ಹೇಳಿದ್ದೇನೆಂದರೆ - ಅವರ ಬಂಧುವರ್ಗಕ್ಕೆ ಸೇರಿದ ಯಾರೋ ಒಬ್ಬರು ಪಿಶಾಚಿಯಾಗಿ ಅವರಿಗೆ ಆಗಾಗ್ಗೆ ಕಾಣಿಸಿಕೊಳ್ಳುತ್ತಿದ್ದರು. ಅದು ಆಗಾಗ್ಗೆ ದೂರ ಪ್ರದೇಶದಿಂದ ವರ್ತಮಾನಗಳನ್ನೂ ತರುತ್ತಿತ್ತಂತೆ. ಆದರೆ ಪರೀಕ್ಷೆ ಮಾಡಿ ನೋಡಿದಾಗ ಅದರ ಮಾತು ಎಲ್ಲಾ ಸಮಯದಲ್ಲಿಯೂ ನಿಜವಾಗುತ್ತಿರಲಿಲ್ಲವಂತೆ. ಆಮೇಲೆ ಯಾವುದೋ ಒಂದು ತೀರ್ಥಕ್ಷೇತ್ರಕ್ಕೆ ಹೋಗಿ “ಅದು ಮುಕ್ತಿ ಪಡೆಯಲಿ" ಎಂದು ಪ್ರಾರ್ಥನೆ ಮಾಡಿದಾರಭ್ಯ ಪುನಃ ಕಂಡು ಬಂದಿಲ್ಲವಂತೆ.

ಶಿಷ್ಯನು ಈಗ ಶ್ರಾದ್ಧಾದಿಗಳಿಂದ ಪ್ರೇತಾತ್ಮಕ್ಕೆ ತೃಪ್ತಿಯಾಗುತ್ತದೆಯೆ ಇಲ್ಲವೆ ಎಂದು ಕೇಳಲು, ಸ್ವಾಮಿಜಿ “ಅದು ಸ್ವಲ್ಪವೂ ಅಸಂಭವವಲ್ಲ" ಎಂದರು. ಶಿಷ್ಯನು ಈ ವಿಷಯದಲ್ಲಿ ಯುಕ್ತಿ ಪ್ರಮಾಣಗಳನ್ನು ಅಪೇಕ್ಷಿಸಲು, ಸ್ವಾಮೀಜಿ “ನಿನಗೆ ಒಂದು ದಿನ ಈ ವಿಷಯವನ್ನು ಚೆನ್ನಾಗಿ ತಿಳಿಸಿಕೊಡುತ್ತೇನೆ. ಶ್ರಾದ್ಧಾದಿಗಳಿಂದ ಪ್ರೇತಾತ್ಮಗಳಿಗೆ ತೃಪ್ತಿಯಾಗುವುದೆಂಬುದಕ್ಕೆ ಬೇಕಾದಷ್ಟು ಯುಕ್ತಿಗಳಿವೆ. ಇಂದು ನನಗೆ ದೇಹಸ್ಥಿತಿ ಸರಿಯಾಗಿಲ್ಲ, ಇನ್ನೊಂದು ದಿನ ಅದನ್ನು ತಿಳಿಸಿಕೊಡುತ್ತೇನೆ" ಎಂದು ಹೇಳಿದರು. ಆದರೆ ಸ್ವಾಮಿಗಳಿಗೆ ಈ ಪ್ರಶ್ನೆ ಹಾಕುವುದಕ್ಕೆ ಶಿಷ್ಯನು ಈ ಜನ್ಮದಲ್ಲಿ ಪುನಃ ಅವಕಾಶವನ್ನು ಪಡೆಯಲಿಲ್ಲ.

\newpage

\chapter[ಅಧ್ಯಾಯ ೧೭]{ಅಧ್ಯಾಯ ೧೭\protect\footnote{\engfoot{C.W, Vol. VII, P 132}}}

\begin{center}
ಸ್ಥಳ: ಬೇಲೂರು ಮಠ (ಬಾಡಿಗೆ ಕಟ್ಟಡ), ವರ್ಷ: ಕ್ರಿ.ಶ. ೧೮೯೮ ನವೆಂಬರ್.
\end{center}

ಬೇಲೂರಿನ ನೀಲಾಂಬರ ಬಾಬುವಿನ ತೋಟದಲ್ಲಿಯೇ ಈಗಲೂ ಮಠ ಇದೆ. ಮಾರ್ಗಶೀರ್ಷಮಾಸ ಕೊನೆಯ ಭಾಗ; ಸ್ವಾಮೀಜಿ ಈಗ ಸಂಸ್ಕೃತ ಶಾಸ್ತ್ರಾದಿ ವ್ಯಾಸಂಗದಲ್ಲಿ ವಿಶೇಷ ತತ್ಪರರಾಗಿದ್ದಾರೆ. “ಆಚಂಡಾಲಾಪ್ರತಿಹತರಯೋ" ಮುಂತಾದ ಎರಡು ಶ್ಲೋಕಗಳನ್ನು ಅವರು ಈ ಕಾಲದಲ್ಲಿಯೇ ರಚಿಸಿದರು. ಇಂದು ಸ್ವಾಮೀಜಿ “ಓಂ ಹ್ರೀಂ ಋತಂ”\footnote{ಈ ಘಟನೆಯಾದ ನಾಲ್ಕು ಐದು ದಿನಗಳಾದ ಮೇಲೆ ಸ್ವಾಮಿಗಳು ಒಂದು ದಿನ ಶಿಷ್ಯನನ್ನು 'ಆ ಸ್ತೋತ್ರದಲ್ಲಿ ಏನಾದರೂ ತಿದ್ದಬೇಕಾಗಿತ್ತೆ?' ಎಂದು ಕೇಳಿದರು. ಅದಕ್ಕೆ ಉತ್ತರವಾಗಿ ತಾನು ಅದನ್ನು ಇನ್ನೂ ಚೆನ್ನಾಗಿ ಓದಿ ನೋಡಿರಲಿಲ್ಲವೆಂದು ಶಿಷ್ಯನು ಹೇಳಿದನು. ಆಮೇಲೆ ಈ ಸ್ತೋತ್ರದ ಮೂಲ ಪ್ರತಿಯನ್ನು ಮಠದಲ್ಲಿ ಬಹುವಾಗಿ ಹುಡುಕಿದರೂ ಸಿಕ್ಕದೆ ಇದ್ದದ್ದರಿಂದ 'ಓಂ ಹ್ರೀಂ ಋತಂ' ಸ್ತೋತ್ರವು ಹೋಗಿಬಿಡುವಂತಿತ್ತು. ಶಿಷ್ಯನ ಹತ್ತಿರವಿದ್ದ ಪ್ರತಿಯು, ಸ್ವಾಮಿಗಳ ಸ್ವರೂಪ ಸಂವರಣವಾದ ಸುಮಾರು ನಾಲ್ಕು ವರ್ಷಗಳ ಮೇಲೆ, ಶಿಷ್ಯನ ಹಳೆಯ ಕಾಗದಗಳನ್ನು ಹುಡುಕುತ್ತಿರುವಾಗ ಸಿಕ್ಕಿತು. ಅನಂತರ ಅದು 'ಉದ್ಭೋಧನ'ದಲ್ಲಿ ಮೊದಲು ಮುದ್ರಿತವಾಯಿತು.} ಮುಂತಾದ ಸ್ತೋತ್ರವನ್ನು ರಚಿಸಿ ಶಿಷ್ಯನ ಕೈಯಲ್ಲಿ ಕೊಟ್ಟು “ನೋಡು ಇದರಲ್ಲಿ ಛಂದೋಭಂಗ ಮುಂತಾದ ದೋಷಗಳೇನಾದರೂ ಇವೆಯೋ ಏನೋ" ಎಂದು ಹೇಳಿದರು. ಶಿಷ್ಯನು ತೆಗೆದುಕೊಂಡು ಅದರ ಒಂದು ಪ್ರತಿಯನ್ನು ಮಾಡಿಕೊಂಡನು.

ಸ್ವಾಮೀಜಿ ಈ ಸ್ತೋತ್ರವನ್ನು ರಚಿಸಿದ ದಿವಸ ಅವರ ನಾಲಗೆಯಲ್ಲಿ ಸರಸ್ವತಿಯೆ ಆರೂಢಳಾಗಿದ್ದಂತಿತ್ತು. ಶಿಷ್ಯನೊಡನೆ ನಿರರ್ಗಳವಾಗಿ ಸುಲಲಿತವಾದ ಸಂಸ್ಕೃತ ಭಾಷೆಯಲ್ಲಿ ಸುಮಾರು ಎರಡು ಗಂಟೆಯ ಹೊತ್ತು ಮಾತನಾಡಿದರು! ಇಷ್ಟು ಸುಲಲಿತವಾದ ವಾಕ್ಯವಿನ್ಯಾಸವನ್ನು ಶಿಷ್ಯನು ಮಹಾ ಮಹಾ ಪಂಡಿತರ ಬಾಯಲ್ಲಿಯೂ ಯಾವಾಗಲೂ ಕೇಳಿಲ್ಲ.

ಅದು ಹೇಗಾದರೂ ಇರಲಿ, ಶಿಷ್ಯನು ಸ್ತೋತ್ರವನ್ನು ನಕಲು ಮಾಡಿಕೊಂಡ ಮೇಲೆ ಸ್ವಾಮಿಜಿ “ನೋಡು, ಭಾವದಲ್ಲಿ ತನ್ಮಯನಾಗಿ ಬರೆಯುತ್ತ ಇರುವಾಗ ಆಗಾಗ್ಗೆ ವ್ಯಾಕರಣ ದೋಷಗಳು ಆಗುವುದುಂಟು. ಅದಕ್ಕೋಸ್ಕರವೇ ನೋಡಿ ಕೇಳಿ ಮಾಡಬೇಕೆಂದು ನಿನಗೆ ಹೇಳುವುದು" ಎಂದರು.

ಶಿಷ್ಯ: ಮಹಾಶಯರೆ, ಅವೆಲ್ಲಾ ದೋಷಗಳಲ್ಲ - ಅವು ಆರ್ಷ ಪ್ರಯೋಗ

ಸ್ವಾಮೀಜಿ: ನೀನೇನೋ ಹೇಳುತ್ತೀಯೆ; ಆದರೆ ಇತರರು ಯಾತಕ್ಕೆ ಹಾಗೆ ತಿಳಿದುಕೊಂಡಾರು? ಆ ದಿನ ‘ಹಿಂದೂ ಧರ್ಮವೆಂದರೇನು?’ ಎಂದು ಒಂದು ಪ್ರಬಂಧವನ್ನು ಬಂಗಾಳಿಯಲ್ಲಿ ಬರೆದೆ - ಅದನ್ನು ನಿಮ್ಮಲ್ಲಿಯೇ ಕೆಲವರು ನೋಡಿ ಕಠಿಣವಾದ ಬಂಗಾಳಿ ಎಂದಿರಿ. ಸಕಲ ಪದಾರ್ಥಗಳ ಹಾಗೆ ಭಾಷೆ ಮತ್ತು ಭಾವಗಳೂ ಕಾಲಕ್ರಮದಲ್ಲಿ ನೂತನತ್ವವಿಲ್ಲದೆ ಹೋಗುವುದೆಂದು ನನ್ನ ಅಭಿಪ್ರಾಯ. ದೇಶದಲ್ಲಿ ಹಾಗೇ ಆಗಿದೆ ಎಂದು ನನಗೆ ತೋರುತ್ತದೆ. ಭಾಷೆ ಮತ್ತು ಭಾವಗಳಲ್ಲಿ ಪರಮಹಂಸರ ಆಗಮನದಿಂದ ಹೊಸತಾಗಿ ಪ್ರವಾಹ ಉಕ್ಕಿಬಂದಂತಾಗಿದೆ. ಈಗ ಎಲ್ಲವೂ ಹೊಸ ಅಚ್ಚಿನಲ್ಲಿ ತಯಾರಾಗಬೇಕಾಗಿದೆ. ನೂತನ ಪ್ರತಿಭೆಯ ಮುದ್ರೆಯನ್ನು ಹಾಕಿ ಎಲ್ಲಾ ವಿಷಯಗಳನ್ನೂ ಪ್ರಚಾರ ಮಾಡಬೇಕಾಗಿದೆ. ಇದನ್ನೇ ನೋಡು - ಹಿಂದಿನ ಕಾಲದ ಸಂನ್ಯಾಸಿಗಳ ಕಾರ್ಯಕ್ರಮಗಳೆಲ್ಲಾ ಹೇಗೆ ಬದಲಾಯಿಸಿ ಈಗ ಹೊಸ ಮಾದರಿಯಾಗಿ ಬಿಟ್ಟಿವೆ. ಸಮಾಜ ಇದಕ್ಕೆ ವಿರುದ್ಧವಾಗಿ ಬೇಕಾದಷ್ಟು ಪ್ರತಿವಾದಿಸಿಯೂ ಇದೆ. ಆದರೆ ಅದರಿಂದ ಆಗುವುದೇನು? - ಏನೂ ಆಗುವುದಿಲ್ಲ. ನಾವೇನು ಅದಕ್ಕೆ ಹೆದರುತ್ತೇವೆಯೆ? ಈಗ ಸಂನ್ಯಾಸಿಗಳೆಲ್ಲಾ ದೂರದಲ್ಲಿ ದೇಶಾಂತರಗಳಿಗೆ ಪ್ರಚಾರಕಾರ್ಯಕ್ಕೋಸ್ಕರ ಹೋಗಬೇಕಾಗಿದೆ - ಬೂದಿ ಬಳಿದುಕೊಂಡು ಅರ್ಧ ಮೈಯನ್ನು ಬಿಟ್ಟುಕೊಂಡು ಹಿಂದಿನ ಸಂನ್ಯಾಸಿಗಳ ವೇಷಭೂಷಣಗಳಿಂದ ಹೋದರೆ ಜಹಜಿನಲ್ಲಿಯೇ ಕೂರಿಸಿಕೊಳ್ಳುವುದಿಲ್ಲ. ಈ ವೇಷದಿಂದ ಹೇಗೋ ಆ ದೇಶಕ್ಕೆ ತಲುಪಿದರೂ ಅಂಥವನು ಬಂದೀಖಾನೆಯಲ್ಲಿರಬೇಕಾಗುವುದು. ದೇಶದ ನಾಗರಿಕತೆಗೂ ಕಾಲಕ್ಕೂ ಅನುಸಾರವಾಗಿ ಎಲ್ಲಾ ವಿಷಯಗಳಲ್ಲಿಯೂ ಸ್ವಲ್ಪ ಸ್ವಲ್ಪ ಬದಲಾವಣೆ ಮಾಡಿಕೊಳ್ಳಬೇಕು. ಇನ್ನು ಮೇಲೆ ಬಂಗಾಳಿ ಭಾಷೆಯಲ್ಲಿ ಪ್ರಬಂಧಗಳನ್ನು ಬರೆಯೋಣವೆಂದು ಮನಸ್ಸಿನಲ್ಲಿ ಮಾಡಿಕೊಂಡಿದ್ದೇನೆ. ಸಾಹಿತ್ಯ ಲೇಖಕರು ಪ್ರಾಯಶಃ ಅವುಗಳನ್ನು ನೋಡಿ ಆಕ್ಷೇಪಣೆ ಮಾಡಬಹುದು. ಮಾಡಿಕೊಂಡು ಹೋಗಲಿ - ಆದರೆ ನಾನು ಬಂಗಾಳಿ ಭಾಷೆಯನ್ನು ಹೊಸ ಅಚ್ಚಿನಲ್ಲಿ ಹಾಕಿ ತೆಗೆದು ಇಡಲು ಯತ್ನ ಮಾಡುತ್ತೇನೆ. ಈಗಿನ ಬಂಗಾಳಿ ಲೇಖಕರು ಬರೆಯುವುದಕ್ಕೆ ಹೊರಟರೆ ತುಂಬಾ ಕ್ರಿಯಾಪದ ಉಪಯೋಗಮಾಡುತ್ತಾರೆ. ಇದರಿಂದ ಭಾಷೆಗೆ ಬಿಗಿ ಬರುವುದಿಲ್ಲ. ವಿಶೇಷಣಗಳಿಂದ ಭಾವಗಳನ್ನು ವ್ಯಕ್ತಪಡಿಸುವುದಾದರೆ ಭಾಷೆಗೆ ತುಂಬಾ ಬಿಗಿ ಬರುತ್ತದೆ. ಇನ್ನು ಮೇಲೆ ಹೀಗೆ ಬರೆಯುವುದಕ್ಕೆ ಪ್ರಯತ್ನ ಮಾಡು ನೋಡೋಣ. ‘ಉದ್ಭೋಧನ’ದಲ್ಲಿ ಇಂಥ ಭಾಷೆಯಿಂದ ಪ್ರಬಂಧಗಳನ್ನು ಬರೆಯುವುದಕ್ಕೆ ಪ್ರಯತ್ನ ಮಾಡಿ. ಭಾಷೆಯಲ್ಲಿ ಕ್ರಿಯಾಪದಗಳ ಅರ್ಥವನ್ನು ನೀನು ಬಲ್ಲೆಯಾ? ಇದರಿಂದ ಭಾವಕ್ಕೆ ವಿರಾಮ ಬರುವುದು. ಆದ್ದರಿಂದ ಭಾಷೆಯಲ್ಲಿ ಕ್ರಿಯಾಪದಗಳನ್ನು ಹೆಚ್ಚಾಗಿ ಉಪಯೋಗಿಸುವುದು ದೊಡ್ಡ ದೊಡ್ಡ ನಿಟ್ಟುಸಿರುಗಳನ್ನು ಬಿಡುವಹಾಗೆ ದುರ್ಬಲತೆಯ ಚಿಹ್ನೆ ಮಾತ್ರ. ಹೀಗೆ ಮಾಡಿದರೆ ಭಾಷೆಗೆ ಉಸಿರೇ ಇಲ್ಲವೆನ್ನಿಸುತ್ತದೆ. ಅದಕ್ಕೋಸ್ಕರವೆ ಬಂಗಾಳಿ ಭಾಷೆಯಲ್ಲಿ ಚೆನ್ನಾಗಿ ಉಪನ್ಯಾಸ ಕೊಡುವುದಕ್ಕಾಗುವುದಿಲ್ಲ. ಭಾಷೆಯ ಮೇಲೆ ಯಾರಿಗೆ ಸಾಮರ್ಥ್ಯ ಇರುವುದೋ ಅವರು ಅಷ್ಟು ಬೇಗ ಬೇಗ ಭಾವಗಳನ್ನು ನಿಲ್ಲಿಸಿಬಿಡುವುದಿಲ್ಲ. ಅನ್ನ ಸಾರನ್ನು ತಿಂದು ನಿಮ್ಮ ಶರೀರ ಹೇಗೆ ಬಿಗಿಯಿಲ್ಲದೆ ಹೋಗಿದೆಯೊ, ಅದರ ಹಾಗೆಯೆ ಭಾಷೆಯೂ ಆಗಿಬಿಟ್ಟಿದೆ. ಆಹಾರ ನಡೆ ನುಡಿ ಭಾವ ಭಾಷೆ ಇವುಗಳಲ್ಲಿ ತೇಜಸ್ಸನ್ನು ತರಬೇಕು; ಎಲ್ಲಾ ಕಡೆಯಲ್ಲಿಯೂ ಪ್ರಾಣ ವಿಸ್ತೃತವಾಗಬೇಕು - ಎಲ್ಲಾ ರಕ್ತನಾಳದಲ್ಲಿಯೂ ಒಂದು ವಿಧವಾದ ಪ್ರಾಣ ಸಂಚಲನೆ ಅನುಭವಕ್ಕೆ ಬರುವಂತೆ ಮಾಡಬೇಕು. ಹಾಗಾದರೇನೆ, ಈ ಘೋರ ಜೀವನ ಸಂಗ್ರಾಮದಲ್ಲಿ ನಮ್ಮ ದೇಶದ ಜನರು ಬದುಕಿ ಬರಬಲ್ಲರು. ಇಲ್ಲದಿದ್ದರೆ, ಸ್ವಲ್ಪದರಲ್ಲಿಯೆ, ಬಹು ಶೀಘ್ರವಾಗಿ ಈ ದೇಶ ಮತ್ತು ಜನಾಂಗಗಳು ಮೃತ್ಯುಚ್ಛಾಯೆಯಲ್ಲಿ ಸೇರಿಹೋಗುವುವು.

ಶಿಷ್ಯ: ಮಹಾಶಯರೆ, ಬಹುಕಾಲದಿಂದ ಈ ದೇಶದ ಜನರ ಸ್ವಭಾವವೇ ಒಂದು ವಿಧವಾಗಿಬಿಟ್ಟಿದೆ; ಅದನ್ನು ಬೇಗ ಬದಲಾಯಿಸುವುದು ಸಾಧ್ಯವೇ?

ಸ್ವಾಮಿಜಿ: ಹಿಂದಿನ ಕ್ರಮವು ಸರಿಯಾಗಿಲ್ಲವೆಂದು ನೀನು ತಿಳಿದುಕೊಂಡಿದ್ದರೆ ನಾನು ಹೇಳುವಂತೆ, ಹೊಸ ರೀತಿಯಲ್ಲಿ ಪ್ರವರ್ತಿಸುವುದನ್ನು ಕಲಿತುಕೊಳ್ಳುವುದಿಲ್ಲವೇಕೆ? ನಿನ್ನನ್ನು ನೋಡಿ ನಿನ್ನಂತೆ ಮಾಡುವರು ಹತ್ತು ಜನ, ಅವರನ್ನು ನೋಡಿ ಅಭ್ಯಾಸ ಮಾಡುವರು ಮತ್ತೆ ಐವತ್ತು ಜನರು - ಹೀಗೆ ಕಾಲಕ್ರಮದಲ್ಲಿ ಜನರಲ್ಲೆಲ್ಲಾ ಈ ನೂತನ ಭಾವವು ಜಾಗೃತವಾಗುತ್ತದೆ. ಇದರ ಮೇಲೆ ತಿಳಿದೂ ತಿಳಿದೂ ನೀವು ಹಾಗೆ ನಡೆದುಕೊಳ್ಳದಿದ್ದರೆ, ನೀವೆಲ್ಲಾ ಕೇವಲ ಮಾತಿನಲ್ಲಿ ಪಂಡಿತರು, ಕೆಲಸದ ವೇಳೆಯಲ್ಲಿ ಮೂರ್ಖರು ಎಂದು ತಿಳಿಯುತ್ತೇನೆ.

"ಹೃದಯದಲ್ಲಿ ಕ್ರಮಕ್ರಮವಾಗಿ ಬಲವನ್ನು ತಂದುಕೊಳ್ಳಬೇಕು. ಒಬ್ಬ ‘ಮನುಷ್ಯ’ ತಯಾರಾದರೆ ಲಕ್ಷ ಉಪನ್ಯಾಸಗಳ ಫಲವಾಗುವುದು. ಮನಸ್ಸನ್ನೂ ವಾಕ್ಕನ್ನೂ ಒಂದುಮಾಡಿ ಭಾವಗಳು ಜೀವನದಲ್ಲಿ ಫಲಿಸುವಂತೆ ಮಾಡಬೇಕು. ಇದನ್ನೇ ‘ಭಾವದ ಮನೆಯಲ್ಲಿ ಕಳ್ಳತನವಿಲ್ಲದಿರುವಿಕೆ’ ಎಂದು ಪರಮಹಂಸರು ಹೇಳುತ್ತಿದ್ದರು. ಎಲ್ಲಾ ಕಡೆಗಳಲ್ಲಿಯೂ ಕರ್ಮದ ಮೂಲಕವಾಗಿ ಅಭಿಪ್ರಾಯ ಅಥವಾ ಭಾವಗಳನ್ನು ಪ್ರಕಾಶಗೊಳಿಸುವಂತೆ ಆಗಬೇಕು. ಬರಿಯ ಸಿದ್ಧಾಂತಗಳಿಂದ ದೇಶ ಹಾಳಾಗಿಹೋಗಿದೆ. ಯಾರು ಪರಮಹಂಸರ ನಿಜವಾದ ಶಿಷ್ಯರೋ ಅವರು ಧರ್ಮವು ಅನುಷ್ಠಾನ ರಂಗದಲ್ಲಿ ಕಾರ್ಯಶೀಲವಾಗುವುದನ್ನು ತೋರಿಸಬೇಕು; ಜನಗಳ ಅಥವಾ ಸಮಾಜದ ಮಾತಿಗೆ ಲಕ್ಷ್ಯ ಕೊಡದೆ ಕಾರ್ಯ ಮಾಡಿಕೊಂಡು ಹೋಗಬೇಕು. ತುಳಸೀದಾಸರ ದೋಹೆಯಲ್ಲಿರುವುದನ್ನು ಕೇಳಿಲ್ಲವೇ-

\begin{verse}
ಹಾತೀ ಚಲೇ ಬಾಜಾರಮೇ ಕುತ್ತೋ ಭುಕೇ ಹಾಜಾರ್,\\ಸಾಧುನ್ ಕೋ ದುರ್ಭಾವ ನೇಹಿ ಜಬ್ ನಿಂದೇ ಸಂಸಾರ್~।
\end{verse}

ಆನೆಯು ರಸ್ತೆಯಲ್ಲಿ ಹೋಗುತ್ತಿದ್ದರೆ ಸಾವಿರ ನಾಯಿಗಳು ಬೊಗಳುತ್ತವೆ; (ಅದರಂತೆ) ಪ್ರಪಂಚವು ನಿಂದಿಸಿದರೆ ಸಾಧುವಿಗೆ ಯಾವ ಲೋಪವೂ ಇಲ್ಲ.

"ಈ ಜನರನ್ನು ಹುಳುಗಳೆಂದು ಭಾವಿಸಿ ನಡೆದುಕೊಳ್ಳಬೇಕು; ಅವರ ಒಳ್ಳೆಯ ಮತ್ತು ಕೆಟ್ಟ ಮಾತುಗಳಿಗೆ ಕಿವಿಗೊಟ್ಟರೆ ಜೀವನದಲ್ಲಿ ಯಾವ ಮಹತ್ಕಾರ್ಯವನ್ನೂ ಮಾಡುವುದಕ್ಕಾಗುವುದಿಲ್ಲ. ‘ನಾಯಮಾತ್ಮಾ ಬಲಹೀನೇನ ಲಭ್ಯಃ’ ಶರೀರದಲ್ಲಿಯೂ ಮನಸ್ಸಿನಲ್ಲಿಯೂ ಬಲವಿಲ್ಲದಿದ್ದರೆ ಆತ್ಮವನ್ನು ಪಡೆಯುವುದಕ್ಕಾಗುವುದಿಲ್ಲ. ಪುಷ್ಟಿಕರವಾದ ಒಳ್ಳೆಯ ಆಹಾರದಿಂದ ಮೊದಲು ಶರೀರವನ್ನು ಬೆಳೆಸಬೇಕು. ಹಾಗಾದರೆ ಮನಸ್ಸು ಬದಲಾಗುತ್ತದೆ. ಮನಸ್ಸು ಶರೀರದ ಸೂಕ್ಷ್ಮಾಂಶ. ಮನಸ್ಸಿನಲ್ಲಿಯೂ ತುಂಬ ಬಲವನ್ನು ತೋರ್ಪಡಿಸಬೇಕು. ‘ನಾನು ಹೀನ’ ‘ನಾನು ಹೀನ’ ಎನ್ನುತ್ತ ಎನ್ನುತ್ತ ಮನುಷ್ಯನು ಹೀನನಾಗಿ ಹೋಗುತ್ತಾನೆ - ಶಾಸ್ತ್ರಕಾರರು ಅದನ್ನೇ ಹೇಳಿದ್ದಾರೆ -

\begin{verse}
ಮುಕ್ತಾಭಿಮಾನೀ ಮುಕ್ತೋ ಹಿ ಬದ್ದೋ ಬದ್ಧಾಭಿಮಾನ್ಯಪಿ~।\\ಕಿಂ ವದಂತೀಹ ಸತ್ಯೋಽಯಂ ಯಾ ಮತಿ: ಸಾ ಗತಿರ್ಭವೇತ್~॥
\end{verse}

ಯಾರು ತಾನು ಬದ್ಧ ಎಂಬ ಅಭಿಮಾನವನ್ನು ಬಿಟ್ಟು ಸರ್ವದಾ ಜಾಗರೂಕನಾಗಿರುತ್ತಾನೆಯೋ ಅವನು ಮುಕ್ತನಾಗಿಯೇ ಬಿಡುತ್ತಾನೆ; ಯಾರು ತಾನು ಬದ್ಧನೆಂದು ಭಾವಿಸಿಕೊಳ್ಳುತ್ತಾನೆಯೋ ಅವನಿಗೆ ಜನ್ಮ ಜನ್ಮದಲ್ಲಿಯೂ ಬದ್ಧ ದೆಶೆಯೆ; ತಿಳಿದುಕೊ. ಐಹಿಕ ಪಾರಮಾರ್ಥಿಕ ಇವೆರಡು ವಿಚಾರಗಳಲ್ಲಿಯೂ ಈ ಮಾತು ನಿಜವೆಂದು ತಿಳಿ. ಇಹ ಜೀವನದಲ್ಲಿ ಯಾರು ಸರ್ವದಾ ಹತಾಶರೋ ಅವರಿಂದ ಯಾವ ಕೆಲಸವೂ ಸಾಗುವುದಿಲ್ಲ; ಅವರು ಜನ್ಮಜನ್ಮದಲ್ಲಿಯೂ ಗೋಳಾಡುತ್ತ ಬರುತ್ತಾರೆ ಹೋಗುತ್ತಾರೆ. ‘ವೀರಭೋಗ್ಯಾ ವಸುಂಧರಾ’ - ವೀರನೇ ವಸುಂಧರೆಯನ್ನು ಅನುಭವಿಸುವವನು; ಈ ಮಾತು ಖಂಡಿತವಾಗಿಯೂ ಸತ್ಯ. ವೀರನಾಗು - ಸರ್ವದಾ ‘ಅಭೀಃ’ ‘ಅಭೀಃ’ ಎಂದು ಹೇಳು. ಎಲ್ಲರಿಗೂ ‘ಮಾಭೈಃ; ’ ‘ಮಾಭೈಃ’ ಎಂದು ಹೇಳು. ಭಯವೇ ಮೃತ್ಯು - ಭಯವೇ ಪಾಪ - ಭಯವೇ ನರಕ - ಭಯವೇ ವ್ಯಭಿಚಾರ. ಜಗತ್ತಿನಲ್ಲಿ ಎಷ್ಟು ಅಸತ್ ಅಥವಾ ಮಿಥ್ಯಾ ಭಾವಗಳು ಇವೆಯೋ, ಅವೆಲ್ಲಾ ಈ ಭಯರೂಪಿ ಸೈತಾನನಿಂದ ಬಂದವು. ಈ ಭಯವೇ ಸೂರ್ಯನ ಸೂರ್ಯತ್ವ - ಭಯವೇ ವಾಯುವಿನ ವಾಯು - ಭಯವೇ ಯಮನ ಯಮತ್ವ. ಅದೇ ಇವುಗಳನ್ನು ಅವುಗಳ ಸ್ಥಾನದಲ್ಲಿ ಇಟ್ಟಿದೆ - ತಮ್ಮ ತಮ್ಮ ಎಲ್ಲೆಯಿಂದ ಯಾರನ್ನೂ ಹೋಗಗೊಡಿಸುವುದಿಲ್ಲ. ಅದನ್ನೇ ಶ್ರುತಿಯು ಹೇಳುವುದು -

\begin{verse}
ಭಯಾದಸ್ಯಾಗ್ನಿಸ್ತಪತಿ ಭಯಾತ್ ತಪತಿ ಸೂರ್ಯಃ~।\\ಭಯಾದಿಂದ್ರಶ್ಚ ವಾಯುಶ್ಚ ಮೃತ್ಯುರ್ಧಾವತಿ ಪಂಚಮಃ~॥
\end{verse}

ಯಾವಾಗ ಇಂದ್ರ ಚಂದ್ರ ವಾಯು ವರುಣರು ಭಯರಹಿತರಾಗುವರೋ ಅಂದೇ ಎಲ್ಲರೂ ಬ್ರಹ್ಮದಲ್ಲಿ ಸೇರಿಹೋಗುವರು - ಸೃಷ್ಟಿರೂಪ ಭ್ರಾಂತಿಯ ಲಯವಾಗಿ ಹೋಗುವುದು. ಅದಕ್ಕೋಸ್ಕರವೇ ಹೇಳುವುದು - ‘ಅಭೀಃ’ ‘ಅಭೀಃ’ ಎಂದು.

ಹೀಗೆಂದು ಹೇಳುತ್ತ ಹೇಳುತ್ತ ಸ್ವಾಮಿಗಳ ಆ ನೀಲೋತ್ಪಲಸದೃಶವಾದ ನಯನ ಪ್ರಾಂತವು ಕೆಂಪೇರಿದಂತಾಯಿತು. ‘ಅಭೀಃ’ ಎಂಬುದೇ ಮೂರ್ತಿಮತ್ತಾಗಿ ಸ್ವಾಮಿಗಳ ರೂಪದಲ್ಲಿ ಶಿಷ್ಯನ ಎದುರಿಗೆ ದೇಹವನ್ನು ಧರಿಸಿ ನಿಂತುಕೊಂಡಿದ್ದಂತಿತ್ತು! ಶಿಷ್ಯನು ಆ ನಿರ್ಭಯಮೂರ್ತಿಯನ್ನು ದರ್ಶನಮಾಡಿ ಮನಸ್ಸಿನಲ್ಲಿಯೇ “ಈ ಮಹಾಪುರುಷರ ಹತ್ತಿರ ಇದ್ದರೆ, ಇವರ ಮಾತು ಕೇಳಿದರೆ, ಮೃತ್ಯುಭಯವು ಎಲ್ಲಿಗೋ ಪಲಾಯನ ಮಾಡುವ ಹಾಗಿದೆ" ಎಂದುಕೊಂಡನು.

ಸ್ವಾಮೀಜಿ ಮತ್ತೂ ಹೇಳತೊಡಗಿದರು: ಆ ದೇಹವನ್ನು ಧಾರಣ ಮಾಡಿ ಎಷ್ಟು ಸುಖದುಃಖಗಳ - ಎಷ್ಟು ಸಂಪತ್ತು ವಿಪತ್ತುಗಳ ತರಂಗಗಳಲ್ಲಿ ಸಿಕ್ಕಿಕೊಂಡಿರುವಿ! ಆದರೆ ತಿಳಿದುಕೊ - ಅವೆಲ್ಲಾ ಕ್ಷಣಮಾತ್ರವಿರತಕ್ಕವು. ಇವುಗಳನ್ನೆಲ್ಲಾ ಗಮನಿಸತಕ್ಕ ಅಂಶಗಳೆಂದು ತಿಳಿಯಬೇಡ. “ನಾನು ಅಜರಾಮರ ಚಿನ್ಮಯ ಆತ್ಮಾ” - ಎಂಬೀ ಭಾವವನ್ನು ಹೃದಯದಲ್ಲಿ ದೃಢವಾಗಿ ಇಟ್ಟುಕೊಂಡು ಜೀವನವನ್ನು ಕಳೆಯಬೇಕು. “ನನಗೆ ಜನ್ಮವಿಲ್ಲ, ಮೃತ್ಯುವಿಲ್ಲ, ನಿರ್ಲೇಪನಾದ ಆತ್ಮ” ಎಂಬೀ ಭಾವದಲ್ಲಿ ಪೂರ್ತಿಯಾಗಿ ತನ್ಮಯನಾಗಿಬಿಡು. ಪೂರ್ತಿ ತನ್ಮಯನಾಗಿ ಬಿಡಬಲ್ಲೆಯಾದರೆ, ದುಃಖಕಷ್ಟಗಳ ಕಾಲದಲ್ಲಿ ಈ ಭಾವ ತನ್ನಷ್ಟಕ್ಕೆ ತಾನೇ ಮನಸ್ಸಿನಲ್ಲಿ ಹುಟ್ಟುತ್ತದೆ - ಪ್ರಯತ್ನಪೂರ್ವಕವಾಗಿ ಈ ಭಾವವನ್ನು ಹುಟ್ಟಿಸಬೇಕಾದುದಿಲ್ಲ. ಆ ದಿವಸ ವೈದ್ಯನಾಥ ಕ್ಷೇತ್ರದಲ್ಲಿ ಪ್ರಿಯನಾಥ ಮುಖ್ಯನ ಮನೆಗೆ ಹೋಗಿದ್ದೆ. ಅಲ್ಲಿ ಉಬ್ಬಸ ಬಂದು ಪ್ರಾಣ ಹೋಗಿಬಿಡುತ್ತದೆಯೋ ಎನ್ನುವ ಹಾಗಾಯಿತು. ಆದರೆ ಒಳಗಿನಿಂದ ಮಾತ್ರ ಪ್ರತಿ ಶ್ವಾಸದಲ್ಲಿಯೂ ಗಂಭೀರ ಧ್ವನಿ ಬರುವುದಕ್ಕೆ ಮೊದಲಾಯಿತು - “ಸೋಽಹಂ ಸೋಽಹಂ” ಎಂದು. ದಿಂಬು ಒರಗಿಕೊಂಡು ಪ್ರಾಣ ಹೋಗುವುದನ್ನೇ ನಿರೀಕ್ಷಿಸುತ್ತಿದ್ದೆ; ಆಗಲೂ ನೋಡಿದ್ದೇನೆಂದರೆ, ಒಳಗಿನಿಂದ ಕೇವಲ “ಸೋಹಂ ಸೋಽಹಂ" ಎಂಬ ಸದ್ದೇ ಬರುತ್ತಿತ್ತು. ಹಾಗೆ ಸುಮ್ಮನೆ ಕೇಳುತ್ತ ಹೋದೆ - “ಏಕಮೇವಾದ್ವಯಂ ಬ್ರಹ್ಮ ನೇಹ ನಾನಾಸ್ತಿ ಕಿಂಚನ" – ಎರಡನೆಯದಿಲ್ಲದ ಬ್ರಹ್ಮವೊಂದೇ ಇರುವುದು ಬೇರಾವುದೂ ಇಲ್ಲ.

ಶಿಷ್ಯನು ಸ್ತಂಭೀಭೂತನಾಗಿ ಮಹಾಶಯರೆ, ತಮ್ಮ ಜೊತೆಯಲ್ಲಿ ಮಾತನಾಡಿ ತಮ್ಮ ಅನುಭವಗಳನ್ನು ಕೇಳಿದಮೇಲೆ ಶಾಸ್ತ್ರ ಓದುವ ಆವಶ್ಯಕತೆಯಿರುವುದಿಲ್ಲ ಎಂದನು.

ಸ್ವಾಮೀಜಿ: ಹಾಗಲ್ಲವಯ್ಯ! ಶಾಸ್ತ್ರವನ್ನೂ ಓದಬೇಕು. ಜ್ಞಾನ ಸಂಪಾದನೆಗೆ ಶಾಸ್ತ್ರವನ್ನು ಓದುವುದು ಅತ್ಯಂತ ಆವಶ್ಯಕ. ನಾನು ಮಠದಲ್ಲಿ ಶೀಘ್ರವಾಗಿಯೆ ಶಾಸ್ತ್ರಗಳನ್ನು ಪಾಠ ಹೇಳುವ ತರಗತಿಯನ್ನು ಆರಂಭಿಸುತ್ತೇನೆ. ವೇದ, ಉಪನಿಷತ್ತು, ಗೀತೆ, ಭಾಗವತ ಇವುಗಳನ್ನೆಲ್ಲಾ ಓದಬೇಕು. ಅಷ್ಟಾಧ್ಯಾಯಿಯನ್ನು ಓದಬೇಕು..

ಶಿಷ್ಯ: ತಾವು ಪಾಣಿನಿಯ ಅಷ್ಟಾಧ್ಯಾಯಿಯನ್ನು ಓದಿದ್ದೀರಾ?

ಸ್ವಾಮೀಜಿ: ಜಯಪುರದಲ್ಲಿದ್ದಾಗ, ಒಬ್ಬ ದೊಡ್ಡ ವೈಯಾಕರಣರ ಗುರುತಾಯಿತು. ಅವರ ಹತ್ತಿರ ವ್ಯಾಕರಣವನ್ನು ಓದಬೇಕೆಂಬ ಆಶೆಯಾಯಿತು. ವ್ಯಾಕರಣದಲ್ಲಿ ಮಹಾ ಪಂಡಿತರಾಗಿದ್ದರೂ ಪಾಠ ಹೇಳುವುದರಲ್ಲಿ ಅವರಿಗೆ ಅಷ್ಟು ಸಾಮರ್ಥ್ಯವಿರಲಿಲ್ಲ. ನನಗೆ ಮೊದಲನೆಯ ಸೂತ್ರದ ಭಾಷ್ಯವನ್ನು ಮೂರು ದಿನ ಪರ್ಯಂತ ಹೇಳಿದರು; ಆದರೂ ನಾನು ಅದನ್ನು ಸ್ವಲ್ಪವೂ ತಿಳಿದುಕೊಳ್ಳಲಾರದೆ ಹೋದೆನು. ನಾಲ್ಕನೆಯ ದಿವಸ ಉಪಾಧ್ಯಾಯರು ಬೇಸರಪಟ್ಟುಕೊಂಡು, ‘ಸ್ವಾಮಿಗಳೆ! ಮೂರು ದಿನವಾದರೂ ತಮಗೆ ಪ್ರಥಮ ಸೂತ್ರದ ಅರ್ಥವನ್ನು ತಿಳಿಸಲಾರದೆ ಹೋದೆ. ನನ್ನ ಹತ್ತಿರ ಓದುವುದರಿಂದ ಏನೂ ಪ್ರಯೋಜನವಾಗುವುದಿಲ್ಲವೆಂದು ತೋರುತ್ತದೆ’ ಎಂದರು. ಈ ಮಾತನ್ನು ಕೇಳಿ ಮನಸ್ಸಿಗೆ ತುಂಬ ನೋವಾಯಿತು. ಆಹಾರ ನಿದ್ರೆಗಳನ್ನು ತ್ಯಜಿಸಿ, ಮೊದಲನೆಯ ಸೂತ್ರದ ಭಾಷ್ಯವನ್ನು ನನ್ನಷ್ಟಕ್ಕೆ ನಾನೇ ಓದುವುದಕ್ಕೆ ಮೊದಲುಮಾಡಿದೆ. ಮೂರು ಗಂಟೆಯ ಒಳಗೆ ಈ ಸೂತ್ರ ಭಾಷ್ಯದ ಅರ್ಥವು ಕರತಲಾಮಲಕದ ಹಾಗೆ ತಿಳಿದುಹೋಯಿತು. ಆಮೇಲೆ ಉಪಾಧ್ಯಾಯರ ಹತ್ತಿರ ಹೋಗಿ ಸಮಸ್ತ ವ್ಯಾಖ್ಯಾನದ ತಾತ್ಪರ್ಯವನ್ನು ಮಾತಾಡಿದಂತೆ ಹೇಳಿ ತಿಳಿಸಿಬಿಟ್ಟೆ. ಉಪಾಧ್ಯಾಯರು ಅದನ್ನು ಕೇಳಿ, ‘ನಾನು ಮೂರು ದಿನ ಹೇಳಿಯೂ ತಿಳಿಸಲಾರದೆ ಇದ್ದದ್ದನ್ನು ತಾವು ಮೂರು ಗಂಟೆಯಲ್ಲಿ ಇಷ್ಟು ಚಮತ್ಕಾರವಾದ ವ್ಯಾಖ್ಯಾನವನ್ನು ಹೇಗೆ ತಿಳಿದುಕೊಂಡಿರಿ?’ ಎಂದರು. ಆಮೇಲೆ ಪ್ರತಿನಿತ್ಯವೂ ಪ್ರವಾಹದ ಜಲದಂತೆ ಅಧ್ಯಾಯದ ಮೇಲೆ ಅಧ್ಯಾಯವನ್ನು ಓದುತ್ತ ಹೋದೆ. ಮನಸ್ಸಿಗೆ ಏಕಾಗ್ರತೆಯಿದ್ದರೆ ಎಲ್ಲವೂ ಸಾಧ್ಯವಾಗುತ್ತದೆ - ಸುಮೇರುವನ್ನು ಬೇಕಾದರೂ ಪುಡಿಮಾಡಬಹುದು.

ಶಿಷ್ಯ: ಮಹಾಶಯರೆ, ತಮ್ಮದೆಲ್ಲಾ ಅದ್ಭುತವೇ!

ಸ್ವಾಮೀಜಿ: ಅದ್ಭುತವೆಂಬ ವಿಶೇಷವೇನೂ ಇಲ್ಲ. ಅಜ್ಞತೆಯೇ ಅಂಧಕಾರ. ಅದು ಎಲ್ಲವನ್ನೂ ಆವರಿಸಿಕೊಂಡಿರುವುದರಿಂದ ಎಲ್ಲ ಅದ್ಭುತವಾಗಿ ಕಾಣುತ್ತದೆ. ಜ್ಞಾನದ ಬೆಳಕಿನಿಂದ ಎಲ್ಲಾ ಸ್ಪಷ್ಟವಾದರೆ, ಆಗ ಯಾವುದರಲ್ಲಿಯೂ ಅದ್ಭುತವಿರುವುದಿಲ್ಲ. ಇಂಥ ‘ಅಘಟನ ಘಟನ ಪಟೀಯಸಿ’ಯಾದ ಮಾಯೆ, ಅದೂ ಲುಪ್ತವಾಗಿಬಿಡುತ್ತದೆ. ಯಾವುದನ್ನು ತಿಳಿದುಕೊಂಡರೆ ಎಲ್ಲವೂ ತಿಳಿಯುತ್ತದೆಯೋ ಅದನ್ನು ತಿಳಿದುಕೊ - ಅದರ ವಿಚಾರವನ್ನು ಯೋಚಿಸು - ಆ ಆತ್ಮವು ಪ್ರತ್ಯಕ್ಷವಾದರೆ ಶಾಸ್ತ್ರಾರ್ಥವು ಕರತಲಾಮಲಕದ ಹಾಗೆ ಪ್ರತ್ಯಕ್ಷವಾಗುತ್ತದೆ. ಪುರಾತನ ಋಷಿಗಳಿಗೆ ಆಯಿತು, ನಮಗೆ ಆಗುವುದಿಲ್ಲವೇ? ನಾವೂ ಮನುಷ್ಯರೆ. ಒಂದು ಸಲ ಒಬ್ಬರ ಜೀವನದಲ್ಲಿ ಯಾವುದು ಆಗಿದೆಯೋ, ಪ್ರಯತ್ನಪಟ್ಟರೆ, ಅದು ಪುನಃ ಮತ್ತೊಬ್ಬರ ಜೀವನದಲ್ಲಿಯೂ ಅವಶ್ಯವಾಗಿ ಆಗಿಯೇ ಆಗುವುದು. ಯಾವುದು ಒಂದು ಸಲ ಆಗಿದೆಯೊ ಅದೇ ಪುನಃ ಪುನಃ ಆಗುತ್ತದೆ. ಆ ಆತ್ಮವು ಸರ್ವಪ್ರಾಣಿಗಳಿಗೂ ಸಮಾನವಾದದ್ದು. ಆದರೆ ಒಂದು ಪ್ರಾಣಿಗೂ ಮತ್ತೊಂದು ಪ್ರಾಣಿಗೂ ಅದರ ವಿಕಾಸದಲ್ಲಿ ತಾರತಮ್ಯವಿರುತ್ತದೆ, ಅಷ್ಟೆ. ಈ ಆತ್ಮನನ್ನು ವಿಕಾಸಗೊಳಿಸಿಕೊಳ್ಳುವ ಯತ್ನ ಮಾಡು. ಆಗ ನೋಡುವೆ, ಬುದ್ಧಿಯು ಎಲ್ಲ ವಿಷಯಗಳಲ್ಲಿಯೂ ಪ್ರವೇಶಿಸುತ್ತದೆ. ಆತ್ಮವನ್ನರಿಯದವನ ಬುದ್ಧಿ ಏಕದರ್ಶಿನಿ. ಅವನು ಅಲ್ಪ ಸ್ವಲ್ಪವನ್ನು ತಿಳಿಯಬಲ್ಲ; ಆತ್ಮವನ್ನರಿತವನ ಬುದ್ಧಿಯು ಸರ್ವಗ್ರಾಸಿನಿ - ಎಲ್ಲವನ್ನೂ ವ್ಯಾಪಿಸಬಲ್ಲ. ಆತ್ಮ ಪ್ರಕಾಶವಾದರೆ, ದರ್ಶನ, ವಿಜ್ಞಾನ ಎಲ್ಲಾ ಅಧೀನವಾಗಿಬಿಡುವುದೆಂಬುದನ್ನು ನೋಡುವೆ. ಸಿಂಹಗರ್ಜನೆಯಂತೆ ಆತ್ಮದ ಮಹಿಮೆಯನ್ನು ಉದ್ಘೋಷಿಸು. ಜೀವಿಗಳಿಗೆ ಅಭಯವನ್ನು ಕೊಟ್ಟು ಹೀಗೆಂದು ಹೇಳು - ‘ಉತ್ತಿಷ್ಠತ ಜಾಗ್ರತ ಪ್ರಾಪ್ಯ ವರಾನ್ ನಿಬೋಧತ’.

\newpage

\chapter[ಅಧ್ಯಾಯ ೧೮]{ಅಧ್ಯಾಯ ೧೮\protect\footnote{\engfoot{Complete Works of Swami Vivekananda, Volume VI, Page 445}}}

\begin{center}
ಸ್ಥಳ: ಬೇಲೂರು ಮಠ (ಬಾಡಿಗೆ ಕಟ್ಟಡ), ವರ್ಷ: ಕ್ರಿ.ಶ. ೧೮೯೮.
\end{center}

ಶಿಷ್ಯನು ಈಗ ಎರಡು ದಿನಗಳಿಂದ ಬೇಲೂರು ನೀಲಾಂಬರ ಬಾಬುಗಳ ಆರಾಮಗೃಹದಲ್ಲಿ ಸ್ವಾಮಿಜಿಯವರ ಹತ್ತಿರ ಇದ್ದಾನೆ. ಕಲ್ಕತ್ತೆಯಿಂದ ಅನೇಕ ಯುವಕರು ಈ ಸಮಯದಲ್ಲಿ ಸ್ವಾಮಿಜಿಯವರ ಹತ್ತಿರ ಹೋಗಿಬರುತ್ತಿದ್ದರಿಂದ ಮಠದಲ್ಲಿ ಈಚೆಗೆ ನಿತ್ಯೋತ್ಸವವಿದ್ದಂತಿದೆ. ಎಷ್ಟೋ ಧರ್ಮ ಚರ್ಚೆ - ಸಾಧನ ಭಜನ ಪ್ರಯತ್ನ - ದೀನದುಃಖಿಗಳ ಕಷ್ಟ ನಿವಾರಣೆಯ ವಿಚಾರ ಇವೆಲ್ಲಾ ನಡೆಯುತ್ತಿವೆ. ಸಂನ್ಯಾಸಿಗಳಾದ ಗುರುಜನರೆಲ್ಲಾ ಮಹಾ ಉತ್ಸಾಹಿಗಳಾಗಿದ್ದಾರೆ - ಮಹಾದೇವನ ಗಣಗಳ ಹಾಗೆ ಸ್ವಾಮೀಜಿಯ ಆಜ್ಞೆಯನ್ನು ನಡೆಸುವುದಕ್ಕೆ ಉತ್ಸುಕರಾಗಿದ್ದಾರೆ. ಪ್ರೇಮಾನಂದ ಸ್ವಾಮಿಗಳು ದೇವರ ಪೂಜೆಯ ಭಾರವನ್ನು ವಹಿಸಿಕೊಂಡಿದ್ದಾರೆ. ಮಠದಲ್ಲಿ ಪೂಜಾ ಪ್ರಸಾದಗಳಿಗೆ ಬೇಕಾದಷ್ಟು ಏರ್ಪಾಡಾಗಿದೆ – ಬರುವ ಭಕ್ತಾದಿಗಳಿಗೋಸ್ಕರ ಪ್ರಸಾದ ಸರ್ವದಾ ಸಿದ್ಧವಾಗಿರುತ್ತದೆ.

ಇಂದು ರಾತ್ರಿ ಶಿಷ್ಯನು ತಮ್ಮ ಜೊತೆಯಲ್ಲಿಯೇ ಇರುವುದಕ್ಕೆ ಸ್ವಾಮಿಗಳು ಅನುಮತಿಯನ್ನು ಕೊಟ್ಟಿದ್ದಾರೆ. ಸ್ವಾಮಿಗಳ ಸೇವಾಧಿಕಾರವನ್ನು ಪಡೆದು ಶಿಷ್ಯನ ಹೃದಯದಲ್ಲಿ ಇಂದು ಹಿಡಿಸಲಾರದಷ್ಟು ಆನಂದ. ಪ್ರಸಾದವನ್ನು ತೆಗೆದುಕೊಂಡ ಮೇಲೆ ಪಾದಸೇವೆಯನ್ನು ಮಾಡುತ್ತಿರಲು ಆ ಸಮಯದಲ್ಲಿ ಸ್ವಾಮಿಜಿ “ಇಂಥ ಸ್ಥಳವನ್ನು ಬಿಟ್ಟು ನೀನು ಕಲ್ಕತ್ತೆಗೆ ಹೋಗಬೇಕೆನ್ನುತ್ತೀಯಲ್ಲಾ! - ಇಲ್ಲಿ ಎಂಥ ಪವಿತ್ರಭಾವ - ಎಂಥ ಗಂಗೆಯ ಮೇಲಣ ಗಾಳಿ - ಎಂಥ ಸಾಧುಸಂತರ ಸಮಾಗಮ! ಇಂಥ ಸ್ಥಳವನ್ನು ಮತ್ತೆ ಎಲ್ಲಿಯಾದರೂ ಹುಡುಕಿಕೊಂಡು ಬರಬಲ್ಲೆಯಾ?”

ಶಿಷ್ಯ: ಮಹಾಶಯರೆ, ಬಹು ಜನ್ಮಗಳ ತಪಸ್ಸಿನಿಂದ ತಮ್ಮ ಸಂಗಲಾಭ ಪ್ರಾಪ್ತವಾಗಿದೆ. ಈಗ ಯಾತರಿಂದ ಮತ್ತೆ ಮಾಯಾಮೋಹಗಳ ಮಧ್ಯೆ ಬೀಳದಿರಬಹುದೋ ಅದನ್ನು ಕೃಪೆಯಿಟ್ಟು ನನಗೆ ಹೇಳಿಕೊಡಬೇಕು. ಈಗ ಪ್ರತ್ಯಕ್ಷಾನುಭವಕ್ಕೋಸ್ಕರ ಮನಸ್ಸು ಆಗಾಗ್ಗೆ ಬಹು ವ್ಯಾಕುಲಗೊಳ್ಳುತ್ತದೆ.

ಸ್ವಾಮೀಜಿ: ನನಗೂ ಹೀಗೆ ಎಷ್ಟೋ ಆಗಿದೆ. ಕಾಶೀಪುರದ ತೋಟದಲ್ಲಿ ಒಂದು ದಿನ ತುಂಬ ವ್ಯಾಕುಲನಾಗಿ ಪರಮಹಂಸರಿಗೆ ನನ್ನ ಪ್ರಾರ್ಥನೆಯನ್ನು ತಿಳಿಸಿದೆ. ಆಮೇಲೆ ಸಾಯಂಕಾಲದ ಹೊತ್ತಿಗೆ ಧ್ಯಾನ ಮಾಡುತ್ತ ನನ್ನ ದೇಹವನ್ನೇ ತಿಳಿದುಕೊಳ್ಳಲಾರದೆ ಹೋದೆ. ದೇಹ ಇಲ್ಲವೇ ಇಲ್ಲವೆಂದು ತೋರಿತು. ಚಂದ್ರ, ಸೂರ್ಯ, ದೇಶ, ಕಾಲ, ಎಲ್ಲಾ ಒಂದೇ ರೂಪವಾಗಿ ಎಲ್ಲಿಯೋ ಸೇರಿಹೋಯಿತು. ದೇಹ ಬುದ್ಧಿ ಮುಂತಾದುವು ಪ್ರಾಯಶಃ ಇರಲೇ ಇಲ್ಲವೆಂದು ಹೇಳಬಹುದು; ಪ್ರಾಯಶಃ ನಾನು ಲಯವಾಗಿ ಹೋಗಿಬಿಟ್ಟಿದ್ದೆನೆಂದು ತೋರುತ್ತದೆ, ಮತ್ತೇನು? ಸ್ವಲ್ಪ ‘ಅಹಂ’ ಇತ್ತು, ಅದರಿಂದಲೇ ಆ ಸಮಾಧಿಯಿಂದ ಹಿಂತಿರುಗಿದೆ. ಆ ವಿಧವಾದ ಸಮಾಧಿಕಾಲದಲ್ಲಿಯೆ ‘ನಾನು’ ಮತ್ತು ‘ಬ್ರಹ್ಮ’ಗಳ ಭೇದ ಹೋಗಿಬಿಡುತ್ತದೆ - ಎಲ್ಲಾ ಒಂದಾಗಿಬಿಡುತ್ತದೆ – ಮಹಾಸಮುದ್ರದ ನೀರಿನಂತೆ - ನೀರು ಹೊರತು ಮತ್ತೇನೂ ಇಲ್ಲ - ಭಾವ ಭಾಷೆ ಎಲ್ಲಾ ಓಡಿಹೋಗಿಬಿಡುತ್ತವೆ. ‘ಅವಾಙ್ಮಾನಸಗೋಚರಂ’ ಎಂಬುದು ಈ ಸಮಯದಲ್ಲಿಯೇ ಸರಿಯಾಗಿ ಗೊತ್ತಾಗುವುದು. ಹಾಗಲ್ಲದೆ, ‘ನಾನು ಬ್ರಹ್ಮ’ ಎಂದು ಸಾಧಕ ಭಾವಿಸಿಕೊಳ್ಳುವಾಗ ಅಥವಾ ಹೇಳುವಾಗ ‘ನಾನು’ ಮತ್ತು ‘ಬ್ರಹ್ಮ’ ಈ ಎರಡು ಪದಾರ್ಥಗಳು ಬೇರೆ ಇರುತ್ತವೆ - ದ್ವೈತಭಾವವಿರುತ್ತದೆ. ಆಮೇಲೆ ಈ ಸ್ಥಿತಿಯನ್ನು ಪಡೆಯುವುದಕ್ಕೆ ಮೇಲಿಂದ ಮೇಲೆ ಪ್ರಯತ್ನಪಟ್ಟರೂ ಪುನಃ ಪಡೆಯಲಾರದೆ ಹೋದೆ. ಪರಮಹಂಸರಿಗೆ ತಿಳಿಸಲು ಅವರು, ‘ಹಗಲೂ ರಾತ್ರಿ ಈ ಸ್ಥಿತಿಯಲ್ಲಿ ಇದ್ದುಬಿಟ್ಟರೆ ಮಾತೆಯ ಕೆಲಸ ಆಗುವುದಿಲ್ಲ; ಆದ್ದರಿಂದ ಈಗ ಮತ್ತೆ ಆ ಸ್ಥಿತಿಯನ್ನು ಪಡೆಯಲಾರೆ; ಕಾರ್ಯ ಮಾಡುವುದೆಲ್ಲಾ ಮುಗಿದರೆ ಆಮೇಲೆ ಆ ಸ್ಥಿತಿ ಬರುತ್ತದೆ’ ಎಂದು ಹೇಳಿದರು.

ಶಿಷ್ಯ: ವಿಶೇಷ ಸಮಾಧಿ ಅಥವಾ ನಿಜವಾದ ನಿರ್ವಿಕಲ್ಪ ಸಮಾಧಿಯಾದರೆ ಆಗ ಯಾರೂ ಮತ್ತೆ ಅಹಂ ಜ್ಞಾನವನ್ನು ಆಶ್ರಯಿಸಿಕೊಂಡು ದ್ವೈತ ಪ್ರಪಂಚಕ್ಕೆ, ಸಂಸಾರಕ್ಕೆ ಹಿಂತಿರುಗಿ ಬರಲಾರರೆ?

ಸ್ವಾಮೀಜಿ: ಪರಮಹಂಸರು ಏನು ಹೇಳುತ್ತಿದ್ದರೆಂದರೆ, - ಅವತಾರ ಪುರುಷರು ಮಾತ್ರ ಜೀವಹಿತೇಚ್ಛೆಯಿಂದ ಈ ಸಮಾಧಿಯನ್ನು ಬಿಟ್ಟು ಕೆಳಗಿಳಿದು ಬರಬಲ್ಲರು. ಸಾಧಾರಣ ಜೀವರಿಗೆ ಅಲ್ಲಿಂದ ಹಿಂದಕ್ಕೆ ಬರುವುದಕ್ಕಾಗುವುದಿಲ್ಲ. ಅವರ ದೇಹ ಇಪ್ಪತ್ತೊಂದು ದಿನ ಮಾತ್ರ ಜೀವಿಸಿಕೊಂಡಿದ್ದು ಆಮೇಲೆ ತರಗೆಲೆಯ ಹಾಗೆ ಸಂಸಾರ ವೃಕ್ಷದಿಂದ ಉದುರಿ ಬಿದ್ದು ಹೋಗುತ್ತದೆ.

ಶಿಷ್ಯ: ಮನಸ್ಸು ಲುಪ್ತವಾಗಿ ಸಮಾಧಿಯಾದಾಗ ಮನಸ್ಸಿನ ಒಂದು ತರಂಗವೂ ಇಲ್ಲದಿರುವಾಗ - ವಿಕ್ಷೇಪದ, ಎಂದರೆ ಅಹಂ ಜ್ಞಾನದಿಂದ ಸಂಸಾರಕ್ಕೆ ಹಿಂತಿರುಗಿಬರುವ ಸಂಭವ ಎಲ್ಲಿಯದು? ಮನಸ್ಸೇ ಇಲ್ಲದಿರುವಾಗ ಏನು ಕಾರಣದಿಂದ ಸಮಾಧಿ ಸ್ಥಿತಿಯನ್ನು ಬಿಟ್ಟು ದ್ವೈತ ಪ್ರಪಂಚಕ್ಕೆ ಇಳಿದು ಬರುವರು?

ಸ್ವಾಮಿಜಿ: ವೇದಾಂತ ಶಾಸ್ತ್ರದ ಅಭಿಪ್ರಾಯವೇನೆಂದರೆ ನಿಶ್ಶೇಷ ನಿರೋಧ ಸಮಾಧಿಯಿಂದ ಹಿಂದಿರುಗುವಿಕೆಯಿಲ್ಲ, ‘ಅನಾವೃತ್ತಿಃ ಶಬ್ದಾತ್’ ಹಿಂದಿರುಗುವಿಕೆ ಇಲ್ಲ, ಇದಕ್ಕೆ ಶಾಸ್ತ್ರವೇ ಪ್ರಮಾಣ. ಆದರೆ ಅವತಾರ ಪುರುಷರು ಜೀವಹಿತಕ್ಕೋಸ್ಕರವಾಗಿ ಅಲ್ಪ ಸ್ವಲ್ಪ ಸಾಮಾನ್ಯ ವಾಸನೆ, ಆಶೆಯನ್ನು ಇಟ್ಟುಕೊಳ್ಳುತ್ತಾರೆ. ಅದನ್ನು ಅವಲಂಬಿಸಿಕೊಂಡೇ ಜ್ಞಾನಾತೀತ ಅದ್ವೈತ ಭೂಮಿಯಿಂದ ‘ನಾನು ನೀನು’ ಎಂಬ ಜ್ಞಾನ ಮೂಲವಾದ ದ್ವೈತಭೂಮಿಗೆ ಬರುತ್ತಾರೆ.

ಶಿಷ್ಯ: ಆದರೆ, ಮಹಾಶಯರೆ, ಅಲ್ಪಸ್ವಲ್ಪ ವಾಸನೆ ಇದ್ದರೆ ಅದನ್ನು ನಿಶ್ಶೇಷನಿರೋಧ ಸಮಾಧಿಯೆಂದು ಹೇಳುವುದು ಹೇಗೆ? ಏಕೆಂದರೆ ನಿಶ್ಶೇಷ, ನಿರ್ವಿಕಲ್ಪ ಸಮಾಧಿಯಲ್ಲಿ ಮನಸ್ಸಿನ ಸರ್ವವೃತ್ತಿಗಳ, ವಾಸನೆಗಳ ನಿರೋಧ ಅಥವಾ ಧ್ವಂಸ ಆಗಿಬಿಡುವುದೆಂದು ಶಾಸ್ತ್ರದಲ್ಲಿ ಹೇಳಿದೆ.

ಸ್ವಾಮೀಜಿ: ಹಾಗಾದರೆ ಮಹಾ ಪ್ರಳಯವಾದ ಮೇಲೆ ಪುನಃ ಸೃಷ್ಟಿಯಾಗುವುದು ಹೇಗೆ? ಮಹಾ ಪ್ರಳಯದಲ್ಲಿ ಎಲ್ಲವೂ ಬ್ರಹ್ಮದಲ್ಲಿ ಸೇರಿ ಹೋಗುವುದಷ್ಟೇ? ಆಮೇಲೂ ಶಾಸ್ತ್ರದಲ್ಲಿ ಸೃಷ್ಟಿಯ ಪ್ರಸ್ತಾವ ಕೇಳಿಬರುತ್ತದೆ - ಸೃಷ್ಟಿಯೂ ಲಯವೂ ಪ್ರವಾಹದಂತೆ ಪುನಃ ಪುನಃ ಆಗಿ ಹೋಗುತ್ತಿರುತ್ತವೆ. ಮಹಾ ಪ್ರಳಯವಾದ ಮೇಲೆ ಸೃಷ್ಟಿ ಮತ್ತು ಲಯಗಳು ಮತ್ತೆ ಬರುವ ಹಾಗೆ ಅವತಾರ ಪುರುಷರ ನಿರೋಧ ಮತ್ತು ವ್ಯುತ್ಥಾನಗಳೂ ಬರುತ್ತವೆ. ಇದು ಅಸಂಗತವೇಕಾಗುತ್ತದೆ?

ಶಿಷ್ಯ: ಲಯಕಾಲದಲ್ಲಿ ಪುನಃ ಸೃಷ್ಟಿಯ ಬೀಜವು ಬ್ರಹ್ಮದಲ್ಲಿ ಲೀನಪ್ರಾಯವಾಗಿರುತ್ತದೆ. ಅದು ಮಹಾಪ್ರಳಯ ಅಥವಾ ನಿರೋಧ ಸಮಾಧಿಯಲ್ಲ; ಆದರೆ ಸೃಷ್ಟಿಯು ಬೀಜ ಮತ್ತು ಶಕ್ತಿಯ ಆಕಾರ ಧಾರಣ ಮಾತ್ರ - ಎಂದು ನಾನು ವಾದಿಸಿದರೆ?

ಸ್ವಾಮೀಜಿ: ಹಾಗಾದರೆ ಅದಕ್ಕೆ ನಾನು ಹೇಳುವುದೆಂದರೆ, ಯಾವ ಬ್ರಹ್ಮದಲ್ಲಿ ಯಾವ ವಿಶೇಷಣದ ಆಭಾಸವೂ ಇಲ್ಲವೊ - ಯಾವುದು ನಿರ್ಲೇಪ ಮತ್ತು ನಿರ್ಗುಣವೋ - ಅದರ ಮೂಲಕ ಈ ಸೃಷ್ಟಿ ಹೇಗೆ ತಾನೇ ಬಹಿರ್ಗತವಾಗಲು ಸಾಧ್ಯವಾಯಿತು? - ಅದಕ್ಕೆ ಉತ್ತರ ಕೊಡು.

ಶಿಷ್ಯ: ಇದು ತೋರಿಕೆಯ ಬಹಿರ್ಗತ. ಈ ಮಾತಿಗೆ ಉತ್ತರವಾಗಿ ಶಾಸ್ತ್ರದಲ್ಲಿ ಹೇಳಿರುವುದೇನೆಂದರೆ, ಬ್ರಹ್ಮದಿಂದ ಸೃಷ್ಟಿ ವಿಕಾಸ ಮರುಮರೀಚಿಕೆಯಂತೆ ಕಂಡುಬರುವುದೇನೋ ನಿಜ: ಆದರೆ ವಸ್ತುತಃ ಸೃಷ್ಟಿ ಮುಂತಾದ್ದು ಯಾವುದೂ ಆಗಿಲ್ಲ. ಭಾವ ವಸ್ತುವಾದ ಬ್ರಹ್ಮದಲ್ಲಿ ಅಭಾವ ಅಥವಾ ಮಿಥ್ಯಾ ಮಾಯಾಶಕ್ತಿಯಿಂದ ಈ ವಿಧವಾದ ಭ್ರಮವು ಕಾಣುತ್ತದೆ.

ಸ್ವಾಮೀಜಿ: ಸೃಷ್ಟಿಯೇ ಮಿಥ್ಯೆಯಾದರೆ, ಜೀವನ, ನಿರ್ವಿಕಲ್ಪ ಸಮಾಧಿ ಮತ್ತು ಸಮಾಧಿಯಿಂದ ವ್ಯುತ್ಥಾನ ಇವುಗಳೂ ಮಿಥ್ಯೆಯೆಂದು ನೀನು ತಿಳಿದುಕೊಳ್ಳಬಲ್ಲೆಯಷ್ಟೆ. ಜೀವವು ಸ್ವತಃ ಬ್ರಹ್ಮಸ್ವರೂಪ; ಅದಕ್ಕೆ ಇನ್ನು ಬಂಧನದ ಅನುಭವವೆಂಥಾದ್ದು? ಹಾಗಾದರೆ, ‘ನಾನು ಆತ್ಮ’ ಎಂಬುದನ್ನು ಅನುಭವ ಮಾಡಿಕೊಳ್ಳುವುದಕ್ಕೆ ನೀನು ಅಪೇಕ್ಷಿಸುತ್ತೀಯಷ್ಟೆ, ಅದೂ ಕೂಡ ಒಂದು ಭ್ರಮೆ - ಏಕೆಂದರೆ ಶಾಸ್ತ್ರವು ನೀನು ಸರ್ವದಾ ಬ್ರಹ್ಮವಾಗಿಯೆ ಇದ್ದೀಯೆ ಎಂದು ಹೇಳುತ್ತದೆ. ಆದ್ದರಿಂದಲೆ “ಅಯಮೇವ ಹಿ ತೇ ಬಂಧಃ ಸಮಾಧಿಮನುತಿಷ್ಠಸಿ” - ನೀನು ಸಮಾಧಿಯನ್ನು ಪಡೆಯುವುದಕ್ಕೆ ಅಪೇಕ್ಷಿಸುತ್ತೀಯಷ್ಟೆ, ಇದೇ ನಿನಗೆ ಬಂಧನ.

ಶಿಷ್ಯ: ಇದು ಒಳ್ಳೆಯ ಕಷ್ಟಕ್ಕೆ ಬಂತು; ನಾನು ಬ್ರಹ್ಮವಾದರೆ, ಈ ವಿಷಯ ಯಾವಾಗಲೂ ಅನುಭವವಾಗುತ್ತಿರುವುದಿಲ್ಲವೇಕೆ?

ಸ್ವಾಮೀಜಿ: ದ್ವೈತ ಭೂಮಿಕೆಯಲ್ಲಿ ಈ ಸಂಗತಿಯನ್ನು ಅನುಭವ ಮಾಡಬೇಕಾದರೆ ಒಂದು ಕರಣ ಬೇಕು. ಮನಸ್ಸೇ ನಮಗೆ ಇರತಕ್ಕ ಅಂತಃಕರಣ. ಆದರೆ ಮನಸ್ಸು ಎಂಬ ಪದಾರ್ಥ ಜಡ. ಹಿಂದೆ ಇರುವ ಆತ್ಮದ ಪ್ರಭೆಯಿಂದ ಮನಸ್ಸು ಚೇತನದ ಹಾಗೆ ಕಾಣುತ್ತದೆ, ಅಷ್ಟೆ. ಪಂಚದಶೀಕಾರರು ಅದನ್ನೇ ಹೇಳಿದ್ದಾರೆ - “ಚಿಚ್ಛಾಯಾವೇಶತಃ ಶಕ್ತಿಶ್ಚೇತನೇವ ವಿಭಾತಿ ಸಾ" ಚಿತ್ಸ್ವರೂಪವಾದ ಆತ್ಮದ ಛಾಯೆ ಅಥವಾ ಪ್ರತಿಬಿಂಬದ ಆವೇಶ ಶಕ್ತಿಯೇ ಚೈತನ್ಯಮಯಿಯೆಂದು ತೋರುತ್ತದೆ ಮತ್ತು ಇದರಿಂದಲೇ ಮನಸ್ಸೂ ಚೇತನ ಪದಾರ್ಥವೆಂದು ಕಾಣುತ್ತದೆ. ಅದರಿಂದಲೆ ಮನಸ್ಸಿನಿಂದ ಶುದ್ಧ ಚೈತನ್ಯ ಸ್ವರೂಪವಾದ ಆತ್ಮವನ್ನು ತಿಳಿದುಕೊಳ್ಳಲಾಗುವುದಿಲ್ಲವೆಂಬ ಮಾತು ನಿಶ್ಚಯ. ಮನಸ್ಸಿನ ಆಚೆಗೆ ಹೋಗಬೇಕು. ಮನಸ್ಸಿನ ಆಚೆಗೆ ಮತ್ತಾವ ಕರಣವೂ ಇಲ್ಲ - ಆತ್ಮವೊಂದೇ ಇರುವುದು; ಆದ್ದರಿಂದ ಯಾವುದನ್ನು ತಿಳಿದುಕೊಳ್ಳಬೇಕೊ ಅದೇ ಕರಣಸ್ಥಾನದಲ್ಲಿದೆ. ಕರ್ತ, ಕರ್ಮ, ಕರಣ ಎಲ್ಲಾ ಒಂದಾಗಿವೆ. ಅದಕ್ಕೋಸ್ಕರವೇ ಶ್ರುತಿ “ವಿಜ್ಞಾತಾರಮರೇ ಕೇನ ವಿಜಾನೀಯಾತ್" - ನಿತ್ಯವಸ್ತುವನ್ನು ಯಾವುದರಿಂದ ತಿಳಿಯುವುದು? ಎಂದು ಹೇಳಿರುವುದು. ಇದರಿಂದ ಒಟ್ಟಾರೆ ಏನು ಹೇಳಿದಹಾಗಾಯಿತೆಂದರೆ, ದ್ವೈತ ಭೂಮಿಕೆಯ ಆಚೆ ಒಂದು ಅವಸ್ಥೆ ಇದೆ; ಅಲ್ಲಿ ಕರ್ತ ಕರ್ಮ ಕರಣಾದಿ ದ್ವೈತಭಾವಗಳಿಲ್ಲ. ಮನಸ್ಸು ನಿರುದ್ಧವಾದರೆ ಇದು ಪ್ರತ್ಯಕ್ಷವಾಗುತ್ತದೆ. ಬೇರೆ ಮಾತಿಲ್ಲದಿದ್ದರಿಂದ ಈ ಅವಸ್ಥೆಯನ್ನು ಪ್ರತ್ಯಕ್ಷ ಮಾಡಿಕೊಳ್ಳುವುದೆಂದು ಹೇಳಿದೆ. ಏಕೆಂದರೆ ಆ ಅನುಭವವನ್ನು ತಿಳಿಸುವುದಕ್ಕೆ ಮಾತುಗಳೇ ಇಲ್ಲ. ಶಂಕರಾಚಾರ್ಯರು ಅದನ್ನು ‘ಅಪರೋಕ್ಷಾನುಭೂತಿ’ ಎಂದು ಕರೆದಿದ್ದಾರೆ. ಈ ಪ್ರತ್ಯಕ್ಷಾನುಭೂತಿ ಅಥವಾ ಅಪರೋಕ್ಷಾನುಭೂತಿಯನ್ನು ಪಡೆದರೂ ಅವತಾರ ಪುರುಷರು ಕೆಳಕ್ಕೆ ಇಳಿದುಬಂದು ದ್ವೈತಭೂಮಿಯಲ್ಲಿ ಅದರ ಆಭಾಸವನ್ನು ತಿಳಿಸುತ್ತಾರೆ - ಅದಕ್ಕೋಸ್ಕರವೇ ಆಪ್ತಪುರುಷರ ಅನುಭವದಿಂದಲೇ ವೇದಾದಿ ಶಾಸ್ತ್ರಗಳ ಉತ್ಪತ್ತಿಯೆಂದು ಹೇಳುವುದು. ಸಾಧಾರಣ ಜೀವನ ಅವಸ್ಥೆಯಾದರೆ ಉಪ್ಪಿನ ಬೊಂಬೆ ಸಮುದ್ರವನ್ನು ಅಳೆಯುವುದಕ್ಕೆ ಹೋಗಿ ಕರಗಿಹೋಗುವ ಹಾಗೆ; ತಿಳಿಯಿತೆ? ಒಟ್ಟಿನ ಮೇಲೆ ಏನು ಹೇಳಬಹುದೆಂದರೆ, “ನೀನು ಸರ್ವದಾ ಬ್ರಹ್ಮ" ಎಂಬ ಸಂಗತಿಯನ್ನು ‘ತಿಳಿಯ’ಬೇಕು ಅಷ್ಟೆ; ನೀನು ಸರ್ವದಾ ಅದೇ ಆಗಿದ್ದೀಯೆ, ಆದರೆ ಮಧ್ಯೆ ಶಾಸ್ತ್ರವು ಮಾಯೆಯೆಂದು ಹೇಳುವ ಜಡಮನಸ್ಸು ಬಂದು ಅದನ್ನು ತಿಳಿಯಗೊಡಿಸುವುದಿಲ್ಲ. ಆ ಸೂಕ್ಷ್ಮ ಜಡರೂಪವಾದ ಕಾರಣವಸ್ತುವಿನಿಂದ ನಿರ್ಮಿತವಾದ ಮನಸ್ಸೆಂಬ ಪದಾರ್ಥ ಪ್ರಶಾಂತವಾದರೆ, ಆತ್ಮದ ಪ್ರಭೆಯಿಂದ ಆತ್ಮ ತಾನೇ ಪ್ರಕಾಶಿತವಾಗುತ್ತದೆ. ಈ ಮಾಯಾ ಅಥವಾ ಮನಸ್ಸು ಮಿಥ್ಯೆಯೆಂಬುದಕ್ಕೆ ಒಂದು ಪ್ರಮಾಣವೇನೆಂದರೆ, ಮನಸ್ಸು ಜಡ ಮತ್ತು ಅಂಧಕಾರಸ್ವರೂಪ; ಹಿಂದಿರುವ ಆತ್ಮದ ಪ್ರಭೆಯಿಂದ ಚೇತನದ ಹಾಗೆ ಗೋಚರವಾಗುತ್ತದೆ. ಇದನ್ನು ಯಾವಾಗ ತಿಳಿದುಕೊಳ್ಳಲು ಸಮರ್ಥನಾಗುವೆಯೋ, ಆಗ ಒಂದು ಅಖಂಡ ಚೇತನದಲ್ಲಿ ಮನಸ್ಸು ಲಯವಾಗಿ ಹೋಗುತ್ತದೆ; ಆಗಲೇ - “ಅಯಮಾತ್ಮಾ ಬ್ರಹ್ಮ" ಎಂಬ ಅನುಭೂತಿಯಾಗುತ್ತದೆ.

ಆಮೇಲೆ ಸ್ವಾಮೀಜಿ, “ನಿನಗೆ ನಿದ್ರೆ ಬರುವಹಾಗಿದೆ ಎಂದು ಕಾಣುತ್ತದೆ - ಹಾಗಾದರೆ ಮಲಗಿಕೊ" ಎಂದರು. ಶಿಷ್ಯನು ಸ್ವಾಮಿಗಳ ಹತ್ತಿರ ಹಾಸಿಗೆಯ ಮೇಲೆ ಮಲಗಿಕೊಂಡು ನಿದ್ರೆ ಮಾಡಲಾರಂಭಿಸಿದನು. ರಾತ್ರಿ ಸ್ವಾಮೀಜಿಗೆ ಚೆನ್ನಾಗಿ ನಿದ್ರೆ ಬರದೆ ಇದ್ದದ್ದರಿಂದ ಮಧ್ಯೆಮಧ್ಯೆ ಏಳುತ್ತಿದ್ದರು. ಶಿಷ್ಯನೂ ಆಗ ನಿದ್ರೆಯಿಂದೆದ್ದು ಅವರಿಗೆ ಅಗತ್ಯವಾದ ಸೇವೆ ಮಾಡುತ್ತಿದ್ದನು. ಆ ರಾತ್ರಿ ಕಳೆಯಲು, ಅದರ ಕಡೆಯ ಭಾಗದಲ್ಲಿ ಒಂದು ಅದ್ಭುತವನ್ನು ನೋಡಿ ನಿದ್ರಾಭಂಗವಾಗಿ ಆನಂದದಿಂದ ಹಾಸಿಗೆ ಬಿಟ್ಟೆದ್ದನು. ಬೆಳಗ್ಗೆ ಗಂಗಾಸ್ನಾನವಾದ ಮೇಲೆ ಶಿಷ್ಯನು ಬಂದು ನೋಡಲು, ಸ್ವಾಮೀಜಿ ಮಠದ ಒಳಭಾಗದಲ್ಲಿ ಒಂದು ದೊಡ್ಡ ಬೆಂಚಿನ ಮೇಲೆ ಪೂರ್ವಮುಖರಾಗಿ ಕುಳಿತಿದ್ದರು. ಹಿಂದಿನ ದಿನ ರಾತ್ರಿ ಆದ ಸ್ವಪ್ನ ವೃತ್ತಾಂತವನ್ನು ಸ್ಮರಿಸಿಕೊಂಡು ಸ್ವಾಮಿಗಳ ಪಾದಪದ್ಮವನ್ನು ಅರ್ಚನೆ ಮಾಡುವುದಕ್ಕೋಸ್ಕರ ಅವನ ಮನಸ್ಸು ಈಗ ಕಾತರಗೊಂಡಿತು. ಈ ಅಭಿಪ್ರಾಯವನ್ನು ತಿಳಿಸಿ ಸ್ವಾಮಿಜಿಯವರ ಅನುಮತಿಯನ್ನು ಬೇಡಿದನು. ಅವನ ವಿಶೇಷವಾದ ಒತ್ತಾಯದಿಂದ ಸ್ವಾಮೀಜಿ ಒಪ್ಪಿಕೊಂಡರು. ಶಿಷ್ಯನು ಕೆಲವು ಪುಷ್ಪಗಳನ್ನು ತೆಗೆದುಕೊಂಡು ಬಂದು ಸ್ವಾಮಿಗಳ ದೇಹದಲ್ಲಿ ಮಹಾಶಿವನ ಅಧಿಷ್ಠಾನವನ್ನು ಭಾವಿಸುತ್ತಾ ವಿಧ್ಯುಕ್ತವಾದ ರೀತಿಯಲ್ಲಿ ಅವರನ್ನು ಪೂಜಿಸಿದನು.

ಪೂಜೆ ಮುಗಿದ ಮೇಲೆ ಸ್ವಾಮೀಜಿ ಶಿಷ್ಯನನ್ನು ಕುರಿತು, “ನಿನ್ನ ಪೂಜೆಯೇನೋ ಆಗಿಹೋಯಿತು; ಆದರೆ ಬಾಬೂರಾಮ (ಪ್ರೇಮಾನಂದ) ಬಂದು ನಿನ್ನನ್ನು ಈಗ ನುಂಗಿಹಾಕಿಬಿಡುತ್ತಾನೆ! ನೀನು ಪರಮಹಂಸರ ಪೂಜೆಗೆ ಇಟ್ಟುಕೊಳ್ಳುವ ಹೂವಿನ ಪಾತ್ರೆಯಲ್ಲಿ ನನ್ನ ಕಾಲನ್ನು ಇಟ್ಟು ಪೂಜೆ ಮಾಡಿದೆ ಅಲ್ಲವೆ?" ಎಂದರು. ಮಾತು ಮುಗಿಯಿತೊ ಮುಗಿಯಲಿಲ್ಲವೊ ಅಷ್ಟರೊಳಗೆ ಪ್ರೇಮಾನಂದ ಸ್ವಾಮಿಗಳು ಅಲ್ಲಿಗೆ ಬಂದರು. ಸ್ವಾಮೀಜಿ ಅವರಿಗೆ “ಅಯ್ಯಾ, ನೋಡು ಇಂದು ಏನು ಭಯಂಕರವಾದ ಕೆಲಸವನ್ನು ಮಾಡಿಬಿಟ್ಟಿದ್ದಾನೆ! ಪರಮಹಂಸರ ಪೂಜೆಯ ತಟ್ಟೆಯಲ್ಲಿ, ಸುಗಂಧ ಮುಂತಾದ್ದನ್ನೆಲ್ಲಾ ತಂದು ಇವೊತ್ತು ನನ್ನ ಪೂಜೆ ಮಾಡಿದ್ದಾನೆ" ಎಂದು ಹೇಳಿಬಿಟ್ಟರು. ಪ್ರೇಮಾನಂದ ಸ್ವಾಮಿಗಳು ನಗುತ್ತ ನಗುತ್ತ, “ಅವನು ಮಾಡಿದ್ದು ಸರಿತಾನೆ; ನೀನೂ ಪರಮಹಂಸರು ಬೇರೆಬೇರೆಯೇನು?" ಎಂದರು. ಈ ಮಾತನ್ನು ಕೇಳಿ ಶಿಷ್ಯನು ನಿರ್ಭಯನಾದನು.

ಶಿಷ್ಯನು ತುಂಬ ಆಚಾರಶೀಲನಾದವನು; ಆದ್ದರಿಂದ ಅಭಕ್ಷ್ಯವಸ್ತುಗಳನ್ನು ತಿನ್ನುವುದಿರಲಿ, ಯಾರಾದರೂ ಮುಟ್ಟಿದ ಪದಾರ್ಥಗಳನ್ನೂ ತಿನ್ನುತ್ತಿರಲಿಲ್ಲ. ಇದಕ್ಕೋಸ್ಕರ ಸ್ವಾಮಿಗಳು ಶಿಷ್ಯನನ್ನು ಆಗಾಗ್ಗೆ “ಭಟ್ಟಾಚಾರ್ಯ ವೈದಿಕ ಬ್ರಾಹ್ಮಣ" ಎಂದು ಕರೆಯುತ್ತಿದ್ದರು. ಬೆಳಿಗ್ಗೆ ಫಲಾಹಾರದ ಸಮಯದಲ್ಲಿ ವಿಲಾಯತಿಯ ಬಿಸ್ಕತ್ತು ಮೊದಲಾದ್ದನ್ನು ತಿನ್ನುತ್ತ ತಿನ್ನುತ್ತ ಸ್ವಾಮಿಗಳು ಸದಾನಂದ ಸ್ವಾಮಿಗಳಿಗೆ “ಭಟ್ಟಾಚಾರ್ಯನನ್ನು ಹಿಡಿದುಕೊಂಡು ಬಾ" ಎಂದು ಹೇಳಿದರು. ಅವರ ಆಜ್ಞೆಯನ್ನು ಕೇಳಿ ಶಿಷ್ಯನು ಹತ್ತಿರ ಬರಲು, ಸ್ವಾಮಿಗಳು ಈ ಪದಾರ್ಥಗಳಲ್ಲಿ ಸ್ವಲ್ಪವನ್ನು ಅವನಿಗೆ ಪ್ರಸಾದದಂತೆ ತಿನ್ನುವುದಕ್ಕೆ ಕೊಟ್ಟರು. ಶಿಷ್ಯನು ಸಂಕೋಚಪಡದೆ ಅದನ್ನು ತಿಂದನು. ಸ್ವಾಮೀಜಿ ಅದನ್ನು ನೋಡಿ ಅವನಿಗೆ “ಈವೊತ್ತು ಏನು ತಿಂದೆ ಬಲ್ಲೆಯಾ? ಇವು ಕೋಳಿ ಮೊಟ್ಟೆಯಿಂದ ಮಾಡಿದುವು!" ಎಂದು ಹೇಳಿದರು. ಅದಕ್ಕೆ ಅವನು “ಏನಾದರೂ ಇರಲಿ ನಾನು ತಿಳಿದುಕೊಳ್ಳಬೇಕಾದುದಿಲ್ಲ; ತಮ್ಮ ಪ್ರಸಾದರೂಪವಾದ ಅಮೃತವನ್ನು ತಿಂದು ಅಮರನಾದೆನು" ಎಂದು ಉತ್ತರ ಕೊಟ್ಟನು. ಅದನ್ನು ಕೇಳಿ ಸ್ವಾಮಿಗಳು “ಇಂದಿನಿಂದ ನಿನ್ನ ಜಾತಿ ವರ್ಣ ಕುಲ ಪಾಪ ಪುಣ್ಯ ಮೊದಲಾದುವುಗಳ ಮೇಲಣ ಎಲ್ಲ ಅಭಿಮಾನವೂ ಹೋಗಿಬಿಡಲಿ - ನಾನು ಆಶೀರ್ವಾದ ಮಾಡಿದ್ದೇನೆ" ಎಂದು ಹೇಳಿದರು.

ಸ್ವಾಮೀಜಿಯ ಅಂದಿನ ಅಯಾಚಿತ ಮತ್ತು ಅಪಾರ ದಯೆಯ ವೃತ್ತಾಂತವನ್ನು ಸ್ಮರಿಸಿಕೊಂಡು ಶಿಷ್ಯನು ತನ್ನ ಮನುಷ್ಯ ಜನ್ಮವು ಸಾರ್ಥಕವಾಯಿತೆಂದು ತಿಳಿದು ಕೊಂಡನು.

ಮಧ್ಯಾಹ್ನದ ಮೇಲೆ ಸ್ವಾಮೀಜಿ ಹತ್ತಿರಕ್ಕೆ ಅಕೌಂಟೆಂಟ್ ಜನರಲ್ ಬಾಬು ಮನ್ಮಥನಾಥ ಭಟ್ಟಾಚಾರ್ಯರು ಬಂದರು. ಅಮೆರಿಕಾಕ್ಕೆ ಹೋಗುವುದಕ್ಕೆ ಮೊದಲು ಮದ್ರಾಸಿನಲ್ಲಿ ಸ್ವಾಮೀಜಿ ಅನೇಕ ದಿನ ಇವರ ಮನೆಯಲ್ಲಿ ಅತಿಥಿಯಾಗಿದ್ದರು; ಆಗಿನಿಂದ ಇವರು ಸ್ವಾಮಿಗಳಲ್ಲಿ ವಿಶೇಷ ಭಕ್ತಿ ಶ್ರದ್ಧೆಗಳನ್ನಿಟ್ಟುಕೊಂಡಿದ್ದರು. ಭಟ್ಟಾಚಾರ್ಯ ಮಹಾಶಯರು ಸ್ವಾಮಿಗಳನ್ನು ಪಾಶ್ಚಾತ್ಯದೇಶ ಮತ್ತು ಭರತ ಖಂಡಗಳ ಸಂಬಂಧವಾಗಿ ನಾನಾ ಸಂಗತಿಗಳನ್ನು ಕೇಳುವುದಕ್ಕೆ ಮೊದಲುಮಾಡಿದರು. ಸ್ವಾಮೀಜಿ ಈ ಪ್ರಶ್ನೆಗಳಿಗೆ ಉತ್ತರ ಕೊಟ್ಟು ಮತ್ತು ಇತರ ನಾನಾ ವಿಧಗಳಲ್ಲಿಯೂ ಅವರನ್ನು ಸಂತೋಷಗೊಳಿಸಿ “ಒಂದು ದಿನ ಇಲ್ಲಿ ಇದ್ದು ಹೋಗಿ" ಎಂದರು. ಮನ್ಮಥಬಾಬುಗಳು ಅದಕ್ಕೆ “ಇನ್ನೊಂದು ದಿನ ಬಂದು ಇದ್ದು ಹೋಗುತ್ತೇನೆ" ಎಂದು ಹೇಳಿ ಅಪ್ಪಣೆ ಪಡೆದು ಕೆಳಕ್ಕೆ ಇಳಿದು ಹೋಗುತ್ತ ಹೋಗುತ್ತ ಸ್ನೇಹಿತರೊಬ್ಬರೊಡನೆ ಹೀಗೆಂದು ಹೇಳಿದರು: “ಇವರು ಭೂಮಂಡಲದಲ್ಲಿ ಒಂದು ದೊಡ್ಡ ಆಂದೋಳನವನ್ನೇ ಉಂಟುಮಾಡುವರೆಂದು ನಾವು ಮೊದಲೇ ಮದ್ರಾಸಿನಲ್ಲಿ ಚೆನ್ನಾಗಿ ತಿಳಿದುಕೊಂಡಿದ್ದೆವು. ಇಂಥ ಸರ್ವತೋಮುಖವಾದ ಪ್ರತಿಭೆ ಮನುಷ್ಯ ಮಾತ್ರದಲ್ಲಿ ಕಂಡುಬರುವುದಿಲ್ಲ."

ಸ್ವಾಮೀಜಿಯವರು ಮನ್ಮಥ ಬಾಬುಗಳ ಜೊತೆಯಲ್ಲಿ ಗಂಗೆಯ ತೀರದವರೆಗೆ ಬಂದು ಅವರಿಗೆ ಅಭಿವಾದನ ಮಾಡಿ ಅವರು ಹೋಗುವುದಕ್ಕೆ ಅಪ್ಪಣೆಯನ್ನು ಕೊಟ್ಟರು. ಅನಂತರ ಮೈದಾನದಲ್ಲಿ ಸ್ವಲ್ಪ ಹೊತ್ತು ತಿರುಗಾಡುತ್ತಿದ್ದು ವಿಶ್ರಾಂತಿ ಪಡೆಯುವುದಕ್ಕೆ ಮೇಲೆ ಹೊರಟುಹೋದರು.

\newpage

\chapter[ಅಧ್ಯಾಯ ೧೯]{ಅಧ್ಯಾಯ ೧೯\protect\footnote{\engfoot{C.W. Vol. VII, P 144}}}

\begin{center}
ಸ್ಥಳ: ಬೇಲೂರು ಮಠ (ಬಾಡಿಗೆ ಕಟ್ಟಡ), ವರ್ಷ: ಕ್ರಿ.ಶ. ೧೮೯೮.
\end{center}

ಶಿಷ್ಯನು ಈ ದಿವಸ ಬೆಳಿಗ್ಗೆ ಮಠಕ್ಕೆ ಬಂದಿದ್ದಾನೆ. ಸ್ವಾಮಿಜಿಯವರ ಪಾದ ಪದ್ಮಕ್ಕೆ ನಮಸ್ಕರಿಸಿ ನಿಂತುಕೊಂಡ ಕೂಡಲೆ, ಸ್ವಾಮೀಜಿ “ಇನ್ನು ನೌಕರಿ ಮಾಡಿ ಆಗುವುದೇನು? ಒಂದು ಕಸುಬನ್ನು ಅವಲಂಬಿಸಬಾರದೆ?" ಎಂದರು. ಶಿಷ್ಯನು ಆಗ ಒಂದು ಕಡೆ ಮನೆಯಲ್ಲಿ ಪಾಠ ಹೇಳಿಕೊಡುತ್ತಿದ್ದನು ಅಷ್ಟೆ. ಸಂಸಾರದ ಭಾರ ಆಗಲೂ ಇನ್ನೂ ಅವನ ಮೇಲೆ ಬಿದ್ದಿರಲಿಲ್ಲ. ಸುಖವಾಗಿ ಕಾಲಕಳೆದುಹೋಗುತ್ತಿತ್ತು. ಶಿಕ್ಷಣ ಕಾರ್ಯ ಸಂಬಂಧವಾಗಿ ಶಿಷ್ಯನು ಪ್ರಶ್ನೆ ಮಾಡಲು, ಸ್ವಾಮೀಜಿ “ಬಹುಕಾಲ ಮೇಷ್ಟರ ಕೆಲಸ ಮಾಡಿದರೆ ಬುದ್ಧಿ ಕೆಟ್ಟು ಹೋಗುತ್ತದೆ. ಜ್ಞಾನವಿಕಾಸ ಉಂಟಾಗುವುದಿಲ್ಲ. ಹಗಲೂ ರಾತ್ರಿ ಹುಡುಗರ ಜತೆಯಲ್ಲಿ ಇದ್ದು ಇದ್ದು ಕ್ರಮೇಣ ಜಡರಂತಾಗಿ ಬಿಡಬೇಕಾಗುತ್ತದೆ. ಇನ್ನು ಮೇಲೆ ಮೇಷ್ಟರ ಕೆಲಸ ಮಾಡಬೇಡ" ಎಂದರು.

ಶಿಷ್ಯ: ಆದರೆ ಏನು ಮಾಡಲಿ?

ಸ್ವಾಮಿಜಿ: ನೀನು ಸಂಸಾರದಲ್ಲೇ ಇದ್ದುಕೊಂಡು ಧನಾರ್ಜನೆಯನ್ನು ಮಾಡಬೇಕೆಂಬ ಯೋಚನೆ ಇದ್ದರೆ, ಹೋಗು - ಅಮೆರಿಕಾಕ್ಕೆ ಹೊರಟುಹೋಗು. ಕಸುಬು ಮಾಡುವುದಕ್ಕೆ ಸಲಹೆಗಳನ್ನು ಕೊಡುವೆನು. ನೋಡು ಐದು ವರ್ಷದಲ್ಲಿ ಎಷ್ಟು ರೂಪಾಯಿ ಶೇಖರಿಸುವೆ!

ಶಿಷ್ಯ: ಏನು ಕಸುಬನ್ನು ಅವಲಂಬಿಸಲಿ? ಹಣವನ್ನು ಎಲ್ಲಿಂದ ತರಲಿ?

ಸ್ವಾಮಿಜಿ: ನೀನೇನು ಹುಚ್ಚನ ಹಾಗೆ ಮಾತನಾಡುತ್ತೀಯೆ? ಒಳಗೆ ಅದಮ್ಯವಾದ ಶಕ್ತಿಯಿದೆ. ಸುಮ್ಮನೆ ‘ನಾನು ಏನೂ ಅಲ್ಲ’ ಎಂದುಕೊಂಡು ನಿರ್ವೀರ್ಯರಾಗಿ ಬಿದ್ದಿದ್ದೀರಿ. ನೀವೇ ಏಕೆ? ಎಲ್ಲಾ ಜಾತಿಯವರೂ ಹಾಗೇ ಆಗಿದ್ದಾರೆ. ಒಂದು ಸಲ ಹೊರಗೆ ಸಂಚರಿಸಿಕೊಂಡು ಬಾ - ಭರತಖಂಡವನ್ನು ಬಿಟ್ಟು ಮಿಕ್ಕ ದೇಶಗಳ ಜನರಲ್ಲಿ ಜೀವನ ಹೇಗೆ ಪ್ರವಹಿಸುತ್ತಿದೆ ಎಂಬುದು ಗೊತ್ತಾಗುತ್ತದೆ. ನೀವು ಮಾಡುವುದೇನು? - ಇಷ್ಟು ವಿದ್ಯೆ ಕಲಿತು ಅನ್ಯರ ಮನೆಯ ಬಾಗಿಲಿನಲ್ಲಿ ಭಿಕಾರಿಯಂತೆ ‘ನೌಕರಿ ಕೊಡಿ, ನೌಕರಿ ಕೊಡಿ’ ಎಂದು ಅರಚಿಕೊಳ್ಳುವುದು. ಎಕ್ಕಡವನ್ನು ತಿಂದು ತಿಂದು - ಗುಲಾಮಗಿರಿಯನ್ನು ಮಾಡಿ ಮಾಡಿ ನೀವೇನು ಮನುಷ್ಯರಾಗಿದ್ದೀರಾ? ನಿಮ್ಮ ಬೆಲೆ ಒಂದು ಕುರುಡು ಕವಡೆಯೂ ಇಲ್ಲ. ಇಂಥ ಸಜಲ ಸಫಲ ದೇಶ; ಇಲ್ಲಿ ಪ್ರಕೃತಿ ಮಿಕ್ಕ ದೇಶಗಳಿಗಿಂತ ಕೋಟ್ಯಧಿಕವಾಗಿ ಧನಧಾನ್ಯವನ್ನು ಕೊಡುತ್ತಿದೆ. ಇಲ್ಲಿ ಪ್ರಾಣವನ್ನು ಧರಿಸಿಕೊಂಡಿರುವುದಕ್ಕೆ ನಿಮ್ಮ ಹೊಟ್ಟೆಗೆ ಅನ್ನವಿಲ್ಲ - ಬೆನ್ನಿನ ಮೇಲೆ ಬಟ್ಟೆಯಿಲ್ಲ! ಯಾವ ದೇಶದ ಧನಧಾನ್ಯಗಳು ಪ್ರಪಂಚದ ಇತರ ದೇಶಗಳಲ್ಲೆಲ್ಲಾ ನಾಗರಿಕತೆಯನ್ನು ಪ್ರಚಾರಪಡಿಸಿತೋ ಆ ಅನ್ನಪೂರ್ಣೆಯ ದೇಶದಲ್ಲಿ ನಿಮಗೆ ಇಂಥ ದುರ್ದಶೆಯೆ? ಒಂದು ಗತಿಗೆಟ್ಟ ನಾಯಿಗಿಂತಲೂ ನಿಮ್ಮ ಸ್ಥಿತಿ ಕಡೆಯಾಗಿದೆ. ನೀವು ನಿಮ್ಮ ವೇದ ವೇದಾಂತಗಳ ಬಡಾಯಿಯನ್ನು ಹೇಳಿಕೊಳ್ಳುತ್ತೀರಿ. ಯಾವ ಜನಾಂಗವು ಸಾಮಾನ್ಯ ಅನ್ನ ಬಟ್ಟೆಗಳನ್ನು ಸಂಪಾದಿಸಲಾರದೆ, ಮತ್ತೊಬ್ಬರ ಮುಖವನ್ನು ನೋಡಿಕೊಂಡೇ ಹೊಟ್ಟೆ ಹೊರೆದುಕೊಳ್ಳುತ್ತದೆಯೋ, ಆ ಜನಾಂಗಕ್ಕೆ ಬಡಾಯಿ ಏಕೆ? ಧರ್ಮ ಕರ್ಮಗಳನ್ನು ಈಗ ಗಂಗೆಯಲ್ಲಿ ತೇಲಿಬಿಟ್ಟು ಮೊದಲು ಜೀವನ ಸಂಗ್ರಾಮದಲ್ಲಿ ಮುಂದಾಗಿರಿ. ಭರತಖಂಡದಲ್ಲಿ ಎಷ್ಟು ಪದಾರ್ಥಗಳು ಉತ್ಪತ್ತಿಯಾಗುತ್ತವೆ! ಅನ್ಯದೇಶದ ಜನರು ಈ ಕಚ್ಚಾಪದಾರ್ಥಗಳಿಂದ ಮತ್ತು ಅವುಗಳ ಸಹಾಯದಿಂದ ಬಂಗಾರವನ್ನು ಪಡೆಯುತ್ತಾರೆ. ನೀವೊ ಮೂಟೆ ಹೊರುವ ಕತ್ತೆಗಳಂತೆ ಅವರ ಪದಾರ್ಥಗಳನ್ನು ಎಳೆದು ಸಾಯುತ್ತೀರಿ. ಭರತಖಂಡದಲ್ಲಿ ಬೆಳೆಯುವ ಕಚ್ಚಾ ಪದಾರ್ಥಗಳನ್ನು ದೇಶವಿದೇಶಗಳ ಜನರು ತೆಗೆದುಕೊಂಡು ಹೋಗಿ ಅವುಗಳ ಮೇಲೆ ತಮ್ಮ ಬುದ್ಧಿಯನ್ನು ಉಪಯೋಗಿಸಿ ನಾನಾ ಪದಾರ್ಥಗಳನ್ನು ತಯಾರುಮಾಡಿ ದೊಡ್ಡವರಾದರು. ನೀವೊ ನಿಮ್ಮ ಬುದ್ಧಿಯನ್ನು ಪೆಟ್ಟಿಗೆಯಲ್ಲಿ ಮುಚ್ಚಿಟ್ಟು ಮನೆಯಲ್ಲಿದ್ದ ಹಣವನ್ನು ಹೊರಗಿನವರಿಗೆ ಕೊಟ್ಟು ಬಿಟ್ಟು ‘ಕವಳ’ ‘ಕವಳ’ ಎಂದು ಅಲೆಯುತ್ತಿದ್ದೀರಿ!"

ಶಿಷ್ಯ: ಯಾವ ಮಾರ್ಗವನ್ನು ಅನುಸರಿಸಿದರೆ ಅನ್ನ ಸಂಪಾದನೆಯಾಗುತ್ತದೆ, ಮಹಾಶಯರೆ?

ಸ್ವಾಮೀಜಿ: ಮಾರ್ಗ ನಿಮ್ಮ ಕೈಯಲ್ಲಿಯೇ ಇದೆ. ಕಣ್ಣನ್ನು ಬಟ್ಟೆಯಿಂದ ಕಟ್ಟಿಕೊಂಡು ‘ನಾನು ಕುರುಡ, ನಾನು ಏನನ್ನೂ ನೋಡಲಾರೆ’ ಎಂದು ಹೇಳುತ್ತಿದ್ದೀರಿ. ಕಣ್ಣಿನ ಕಟ್ಟನ್ನು ಕಿತ್ತುಹಾಕಿ - ನಡುಹಗಲು ಸೂರ್ಯನ ಕಿರಣಗಳಿಂದ ಜಗತ್ತು ಬೆಳಗುತ್ತಿರುವುದು ಕಣ್ಣಿಗೆ ಬೀಳುತ್ತದೆ. ಹಣ ದೊರೆಯದಿದ್ದರೆ ಜಹಜಿನಲ್ಲಿ ಕೂಲಿಯಾಗಿ ವಿದೇಶಕ್ಕೆ ಹೊರಟುಹೋಗಿ. ಸ್ವದೇಶಿ ಬಟ್ಟೆ, ಚೌಕ, ಮೊರ, ಪೊರಕೆ ಇವುಗಳನ್ನು ತಲೆಯ ಮೇಲೆ ಹೊತ್ತುಕೊಂಡು ಅಮೆರಿಕಾ ಯೂರೋಪ್ ಗಳ ರಸ್ತೆಗಳಲ್ಲಿ ತಿರುಗಿ; ಹಿಂದೂದೇಶದ ಪದಾರ್ಥಗಳಿಗೆ ಈಗಲೂ ಎಷ್ಟು ಬೆಲೆಯಿದೆಯೆಂಬುದು ಗೊತ್ತಾಗುತ್ತದೆ. ಅಮೆರಿಕಾದಲ್ಲಿ ನೋಡಿದೆ - ಹೂಗ್ಲಿ ಜಿಲ್ಲೆಯ ಕೆಲವು ಜನ ಮುಸಲ್ಮಾನರು ಹೀಗೆ ತಿರುಗುತ್ತ ಹಣವಂತರಾಗಿಬಿಟ್ಟಿದ್ದಾರೆ. ಅವರದಕ್ಕಿಂತಲೂ ಕಡೆಯೇ ನಿಮ್ಮ ವಿದ್ಯಾಬುದ್ಧಿ? ಇದನ್ನೇ ನೋಡು. ಈ ದೇಶದಲ್ಲಿ ಬನಾರಸ್ ಸೀರೆಯಾಗುತ್ತದೆಯಷ್ಟೆ. ಅಂಥ ಉತ್ಕೃಷ್ಟವಾದ ಬಟ್ಟೆ ಪೃಥ್ವಿಯಲ್ಲಿ ಮತ್ತೆಲ್ಲಿಯೂ ಆಗುವುದಿಲ್ಲ. ಈ ಬಟ್ಟೆಯನ್ನು ತೆಗೆದುಕೊಂಡು ಅಮೆರಿಕಾಕ್ಕೆ ಹೋಗಿಬಿಡು, ಆ ದೇಶದಲ್ಲಿ ಈ ಬಟ್ಟೆಯ ಲಂಗ ತಯಾರು ಮಾಡಿ ಮಾರುತ್ತಾ ಬಾ, ನೋಡು ಎಷ್ಟು ದುಡ್ಡು ಸಿಕ್ಕುತ್ತದೆ.

ಶಿಷ್ಯ: ಮಹಾಶಯರೆ, ಅವರು ಬನಾರಸ್ ಸೀರೆಯ ಲಂಗ ಏಕೆ ಹಾಕಿಕೊಳ್ಳುವರು? ಚಿತ್ರವಿಚಿತ್ರ ಬಟ್ಟೆಗಳನ್ನು ಆ ದೇಶದ ಹೆಂಗಸರು ಮೆಚ್ಚುವುದಿಲ್ಲವೆಂದು ಕೇಳಿದ್ದೇನೆ.

ಸ್ವಾಮಿಜಿ: ತೆಗೆದುಕೊಳ್ಳುತ್ತಾರೆಯೋ ಬಿಡುತ್ತಾರೆಯೋ ಅದನ್ನು ನಾನು ನೋಡಿಕೊಳ್ಳುತ್ತೇನೆ. ನೀನು ಉದ್ಯಮಶೀಲನಾಗಿ ಹೊರಡು. ನನಗೆ ಆ ದೇಶದಲ್ಲಿ ಅನೇಕ ಸ್ನೇಹಿತರೂ ಆಪ್ತರೂ ಇದ್ದಾರೆ. ನಾನು ನಿನಗೆ ಅವರ ಪರಿಚಯ ಮಾಡಿಕೊಡುತ್ತೇನೆ. ಅವರೇ ಇವುಗಳನ್ನು ಮೊದಲು ತೆಗೆದುಕೊಳ್ಳುವಂತೆ ಕೇಳಿಕೊಳ್ಳುತ್ತೇನೆ. ಆಮೇಲೆ ನೋಡುವೆಯಂತೆ - ಎಷ್ಟು ಜನ ಅವರನ್ನು ಅನುಕರಣ ಮಾಡುತ್ತಾರೆ ಅಂತ. ನೀನು ಆಮೇಲೆ ಪದಾರ್ಥಗಳನ್ನು ಒದಗಿಸಲಾರದೆ ಹೋಗುವೆ.

ಶಿಷ್ಯ: ಕಸುಬನ್ನು ಕೈಕೊಳ್ಳುವುದಕ್ಕೆ ಮೂಲಧನವೆಲ್ಲಿಂದ ತರೋಣ?

ಸ್ವಾಮೀಜಿ: ನಾನು ಏನುಮಾಡಬಹುದೆಂದರೆ, ನಿನಗೆ ಕಾರ್ಯಾರಂಭ ಮಾಡಿಕೊಡಬಹುದು. ಆಮೇಲೆ ಮಾತ್ರ ನಿನ್ನ ಸ್ವಂತ ಪ್ರಯತ್ನದ ಮೇಲೆ ಎಲ್ಲವೂ ಹೋಗುತ್ತದೆ. ‘ಹತೋ ವಾ ಪ್ರಾಪ್ಸ್ಯಸಿ ಸ್ವರ್ಗಂ ಜಿತ್ವಾ ವಾ ಭೋಕ್ಷ್ಯಸೇ ಮಹೀಂ’ - ಈ ಪ್ರಯತ್ನದಲ್ಲಿ ಸತ್ತುಹೋದರೆ ಅದೂ ಉತ್ತಮವೆ. ನಿನ್ನನ್ನು ನೋಡಿ ಹತ್ತಾರು ಜನರು ಮುಂದಾಳುಗಳಾಗಿ ಹೊರಡುತ್ತಾರೆ. ಹಾಗಲ್ಲದೆ ಸಫಲವಾದರೆ ಮಹಾ ಭೋಗದಲ್ಲಿ ಜೀವನ ಮಾಡುವೆ.

ಶಿಷ್ಯ: ಅಪ್ಪಣೆ; ಆದರೆ ಬರಿಯ ಸಾಹಸ ಸಾಲದು.

ಸ್ವಾಮೀಜಿ: ಅದಕ್ಕೇ ಅಪ್ಪ ನಾನು ಹೇಳುವುದು, ನಿಮಗೆ ಶ್ರದ್ಧೆಯಿಲ್ಲ; ಆತ್ಮವಿಶ್ವಾಸವಿಲ್ಲ ಎಂದು. ನಿಮ್ಮಿಂದ ಏನು ನಡೆದೀತು? ಸಂಸಾರ ನಡೆಯುವುದಿಲ್ಲ, ಧರ್ಮ ನಡೆಯುವುದಿಲ್ಲ. ಹೀಗೆ ಉದ್ಯೋಗ ಉದ್ಯಮಗಳನ್ನು ಮಾಡಿಕೊಂಡು ಸಂಸಾರದಲ್ಲಿ ಗಣ್ಯರು, ಮಾನ್ಯರು, ಶ‍್ರೀಮಂತರು ಆಗಿ; ಇಲ್ಲವೇ ಎಲ್ಲವನ್ನೂ ಕಿತ್ತು ಒಗೆದು ನಮ್ಮ ದಾರಿಗೆ ಬಂದುಬಿಡಿ; ದೇಶ ವಿದೇಶದ ಜನರಿಗೆ ಧರ್ಮೋಪದೇಶವನ್ನು ಕೊಟ್ಟು ಉಪಕಾರ ಮಾಡಿ; ಹಾಗಾದರೆ ನನಗೆ ಸಿಕ್ಕಿದಂತೆ ನಿಮಗೂ ಭಿಕ್ಷ ಸಿಕ್ಕುವುದು. ಆದಾನ ಪ್ರದಾನಗಳಿಲ್ಲದಿದ್ದರೆ ಯಾರೂ ಮತ್ತೊಬ್ಬರ ಕಡೆಗೆ ನೋಡುವುದಿಲ್ಲ. ನೋಡು, ನಾನು ಒಂದೆರಡು ಧರ್ಮ ವಾಕ್ಯಗಳನ್ನು ಕಿವಿಗೆ ಹಾಕುತ್ತೇನೆ - ಅದಕ್ಕೇ ಗೃಹಸ್ಥರು ನನಗೆರಡು ಹೊತ್ತು ಅನ್ನ ಹಾಕುತ್ತಾರೆ. ನೀವು ಏನೂ ಮಾಡುವುದಿಲ್ಲ. ನಿಮಗೆ ಜನರು ಅನ್ನವನ್ನೇಕೆ ಕೊಡುತ್ತಾರೆ? ನೌಕರಿಯಲ್ಲಿ, ಗುಲಾಮಗಿರಿಯಲ್ಲಿ, ಇಷ್ಟು ಕಷ್ಟವನ್ನು ನೋಡಿಯೂ ನಿಮಗೆ ಬುದ್ಧಿ ಬರುವುದಿಲ್ಲ. ಆದ್ದರಿಂದ ಕಷ್ಟವೂ ಹೋಗುವುದಿಲ್ಲ. ಇದು ನಿಜವಾಗಿಯೂ ದೈವೀಮಾಯೆಯ ಲೀಲೆ! ಆ ದೇಶದಲ್ಲಿ ನಾನು ನೋಡಿದ್ದೇನೆಂದರೆ - ಯಾರು ನೌಕರಿ ಮಾಡುತ್ತಾರೋ ಅವರಿಗೆ ಪ್ರತಿನಿಧಿಸಭೆಯಲ್ಲಿ ಸ್ಥಳವು ಹಿಂದುಗಡೆ ಇರುತ್ತದೆ; ಯಾರು ತಮ್ಮ ಸ್ವಂತ ಪ್ರಯತ್ನದಿಂದಲೂ ವಿದ್ಯಾಬುದ್ಧಿಗಳಿಂದಲೂ ಪ್ರಸಿದ್ಧರಾಗಿದ್ದಾರೆಯೋ ಅವರು ಕುಳಿತುಕೊಳ್ಳುವುದಕ್ಕೆ ಮುಂದುಗಡೆಯಲ್ಲಿರುವ ಸ್ಥಳ. ಆ ದೇಶದಲ್ಲೆಲ್ಲಾ ಜಾತಿಗೀತೆಗಳ ಉತ್ಪಾತವಿಲ್ಲ. ಪ್ರಯತ್ನ ಪರಿಶ್ರಮಗಳಿಂದಾಗಿ ಭಾಗ್ಯಲಕ್ಷ್ಮಿ ಯಾರಿಗೆ ಒಲಿದಿರುವಳೋ ಅವರೇ ದೇಶದ ನೇತೃಗಳು ನಿಯಂತೃಗಳು ಎಂದು ಗಣ್ಯರಾಗಿರುತ್ತಾರೆ. ನಿಮ್ಮ ದೇಶದಲ್ಲಿ ಜಾತಿಯನ್ನು ದೊಡ್ಡದುಮಾಡಿಕೊಂಡು, ನಿಮಗೆ ಅನ್ನವೂ ಹುಟ್ಟದಂತಾಗಿದೆ. ಒಂದು ಸೂಜಿ ತಯಾರು ಮಾಡುವುದಕ್ಕೆ ಯೋಗ್ಯತೆ ಇಲ್ಲ. ನೀವು ಇಂಗ್ಲಿಷರನ್ನು ಟೀಕಿಸುತ್ತೀರಿ. ಬುದ್ಧಿಯಿಲ್ಲದವರು! ಅವರ ಕಾಲು ಕಟ್ಟಿಕೊಂಡು ಜೀವನ ಸಂಗ್ರಾಮೋಪಯೋಗಿಯಾದ ವಿದ್ಯೆಯನ್ನೂ ಶಿಲ್ಪ ವಿಜ್ಞಾನವನ್ನೂ ಕರ್ಮತತ್ಪರತೆಯನ್ನೂ ಕಲಿಯಿರಿ. ಯಾವಾಗ ಉಪಯುಕ್ತವಾಗುವಿರೋ ಆಗ ನಿಮಗೆ ಆದರ ಬರುವುದು. ಆಗ ನಿಮ್ಮ ಮಾತನ್ನು ನಡೆಸುತ್ತಾರೆ. ಯಾವುದೂ ಏನೂ ಇಲ್ಲದೆ ಕೇವಲ ಕಾಂಗ್ರೆಸ್ ಮಾಡಿಕೊಂಡು ಕೂಗಿಟ್ಟರೆ ಏನಾಗುತ್ತದೆ?

ಶಿಷ್ಯ: ಮಹಾಶಯರೆ, ದೇಶದಲ್ಲಿರುವ ವಿದ್ಯಾವಂತರೆಲ್ಲಾ ಅದರಲ್ಲಿ ಸೇರಿದ್ದಾರಲ್ಲಾ?

ಸ್ವಾಮೀಜಿ: ಸ್ವಲ್ಪ ಹರಟೆ ಹೊಡೆದರೆ ಅಥವಾ ಚೆನ್ನಾಗಿ ಮಾತನಾಡಿ ಬಿಟ್ಟರೆ ನಿಮ್ಮ ಪಾಲಿಗೆ ಶಿಕ್ಷಿತರಾದಂತಾಯಿತು! ಯಾವ ವಿದ್ಯೆಯ ಅಭಿವೃದ್ಧಿಯಿಂದ ಇತರ ಸಾಧಾರಣರನ್ನು ಜೀವನ ಸಂಗ್ರಾಮದಲ್ಲಿ ಸಮರ್ಥರನ್ನಾಗಿ ಮಾಡುವುದಕ್ಕಾಗುವುದಿಲ್ಲವೋ, ಯಾವುದು ಮನುಷ್ಯನಿಗೆ ಚಾರಿತ್ರಬಲ, ಪರಸೇವಾತತ್ಪರತೆ, ಸಿಂಹಸಾಹಸಿಕತೆ ಇವುಗಳನ್ನು ಒದಗಿಸುವುದಿಲ್ಲವೋ ಅದು ವಿದ್ಯೆಯೇನು? ಯಾವ ವಿದ್ಯೆಯಿಂದ ಜೀವನದಲ್ಲಿ ತನ್ನ ಕಾಲಿನ ಮೇಲೆ ತಾನು ನಿಂತುಕೊಳ್ಳಬಹುದೋ, ಅದೇ ವಿದ್ಯೆ. ಈಗ ಈ ಸ್ಕೂಲು ಕಾಲೇಜುಗಳಲ್ಲೆಲ್ಲಾ ಓದಿ, ನೀವು ಒಂದು ವಿಧವಾದ ಅಜೀರ್ಣ ರೋಗಗ್ರಸ್ತ ಜನರಾಗಿಬಿಟ್ಟಿದ್ದೀರಿ. ಕೇವಲ ಯಂತ್ರಗಳ ಹಾಗೆ ಕೆಲಸ ಮಾಡುತ್ತೀರಿ: ‘ಜಾಯಸ್ವ’ ‘ಮ್ರಿಯಸ್ವ’ ಹುಟ್ಟುವುದು, ಸಾಯುವುದು ಎಂಬ ವಾಕ್ಯಗಳಿಗೆ ಸಾಕ್ಷಿಸ್ವರೂಪವಾಗಿ ನಿಂತುಕೊಂಡಿದ್ದೀರಿ. ಆರಂಬಗಾರರು, ಜೋಡು ಹೊಲಿಯುವವರು, ಗುಡಿಸುವವರು ಇದ್ದಾರಲ್ಲಾ ಇವರ ಕರ್ಮತತ್ಪರತೆಯೂ ಆತ್ಮನಿಷ್ಠೆಯೂ ನಿಮ್ಮಗಳಲ್ಲಿ ಅನೇಕರದಕ್ಕಿಂತ ಎಷ್ಟೋ ಮೇಲು. ಅವರು ಸದ್ದಿಲ್ಲದೆ ಯಾವಾಗಲೂ ಕೆಲಸ ಮಾಡಿಕೊಂಡು ಹೋಗುತ್ತಿರುತ್ತಾರೆ. ದೇಶದಲ್ಲಿ ಧನಧಾನ್ಯಗಳನ್ನು ಉಂಟುಮಾಡುತ್ತಾರೆ - ಬಾಯಲ್ಲಿ ಮಾತಿಲ್ಲ. ಅವರು ಶೀಘ್ರವಾಗಿಯೇ ನಿಮ್ಮ ಮೇಲೆ ಬಂದುಬಿಡುತ್ತಾರೆ. ಬಂಡವಾಳ ಅವರ ಕೈಯಿಂದ ಬರುತ್ತದೆ - ನಿಮ್ಮ ಹಾಗೆ ಅವರಿಗೆ ಅದಿಲ್ಲ ಇದಿಲ್ಲ ಎಂಬ ಅಭಾವದ ಬಾಧೆಯಿಲ್ಲ. ಈಗಿನ ಶಿಕ್ಷಣ ಕ್ರಮದಿಂದ ನಿಮ್ಮ ಹೊರಗಿನ ಆಚಾರವಿಚಾರಗಳೂ ಸ್ಥಿತಿಗತಿಗಳೂ ಬದಲಾಗಿಬಿಟ್ಟಿವೆ. ಆದರೆ ಉದ್ಭಾವನ ಶಕ್ತಿಯಿಲ್ಲದಿರುವುದರಿಂದ ನಿಮಗೆ ದ್ರವ್ಯಸಂಪಾದನೆ ಮಾರ್ಗವಿಲ್ಲ. ನೀವು ಈ ಸಹಿಷ್ಣುಗಳಾದ ನೀಚ ಜಾತಿಗಳ ಮೇಲೆ ಇಷ್ಟುದಿನ ಅತ್ಯಾಚಾರ ನಡೆಸಿಕೊಂಡು ಬಂದಿದ್ದೀರಿ - ಈಗ ಅವರು ನಿಮ್ಮ ಮೇಲೆ ಮುಯ್ಯಿ ತೀರಿಸಿಕೊಳ್ಳುವರು. ನೀವೊ ‘ಹಾ ನೌಕರಿ; ಹೂ ನೌಕರಿ’ ಎಂದುಕೊಂಡು ನಿರ್ನಾಮವಾಗಿಬಿಡುತ್ತೀರಿ.

ಶಿಷ್ಯ: ಮಹಾಶಯರೆ, ಇತರ ದೇಶದೊಡನೆ ಹೋಲಿಸಿದರೆ ನಮ್ಮ ದೇಶದ ಉದ್ಭಾವನ ಶಕ್ತಿ ಅಲ್ಪವಾದರೂ ಭರತಖಂಡದ ಇತರ ಜಾತಿಯ ಜನಗಳು ಮಾತ್ರ ನಮ್ಮ ಬುದ್ಧಿಯಿಂದಲೇ ನಡೆಸಲ್ಪಡುತ್ತಿದ್ದಾರೆ. ಆದ್ದರಿಂದ ಬ್ರಾಹ್ಮಣ ಕಾಯಸ್ಥ ಮುಂತಾದ ಮೇಲುಜಾತಿಯವರನ್ನು ಜೀವನ ಸಂಗ್ರಾಮದಲ್ಲಿ ಸೋಲಿಸುವ ಶಕ್ತಿಯೂ ಶಿಕ್ಷಣವೂ ಇತರ ಜಾತಿಯವರಿಗೆ ಎಲ್ಲಿಂದ ಬರಬೇಕು?

ಸ್ವಾಮೀಜಿ: ನಿಮ್ಮ ಹಾಗೆ ಅವರು ಕೆಲವು ಪುಸ್ತಕಗಳನ್ನು ಓದಿಲ್ಲದೆ ಇರಬಹುದು, ನಿಮ್ಮ ಹಾಗೆ ಷರ್ಟು ಕೋಟು ಹಾಕಿಕೊಂಡು ದೊಡ್ಡ ಮನುಷ್ಯರಾಗುವುದನ್ನು ಕಲಿತುಕೊಂಡಿಲ್ಲದೆ ಇರಬಹುದು, ಆದರೇನು? ಅವರೇ ಜನಾಂಗದ ಮೇರು ದಂಡ – ಎಲ್ಲಾ ದೇಶದಲ್ಲಿಯೂ ಈ ಇತರ ಶ್ರೇಣಿಯ ಜನರು ಕೆಲಸ ಬಿಟ್ಟು ಕುಳಿತರೆ ನೀವು ಅನ್ನ ಬಟ್ಟೆಗಳನ್ನೆಲ್ಲಿಂದ ತರುವಿರಿ? ಕಲ್ಕತ್ತೆಯಲ್ಲಿ ಒಂದು ದಿವಸ ಗುಡಿಸುವವರು ಕೆಲಸ ಮಾಡದೆ ನಿಂತುಬಿಟ್ಟರೆ ಹಾಹಾಕಾರ ಹುಟ್ಟಿಹೋಗುತ್ತದೆ - ಮೂರು ದಿನ ಅವರು ಕೆಲಸ ಮಾಡದೆ ನಿಂತುಬಿಟ್ಟರೆ, ಮಹಾಮಾರಿಯಿಂದ ಊರಿಗೆ ಊರೇ ಹಾಳಾಗಿ ಹೋಗುತ್ತದೆ. ಕಷ್ಟಜೀವಿಗಳಾದವರು ಕೈ ಕಟ್ಟಿಕೊಂಡು ಕುಳಿತರೆ ನಿಮಗೆ ಅನ್ನ ಬಟ್ಟೆಗಳು ದೊರೆಯುವುದಿಲ್ಲ. ಇವರನ್ನು ನೀವು ಕೀಳು ಜನರೆಂದು ಭಾವಿಸುತ್ತೀರಿ! ನೀವು ವಿದ್ಯಾವಂತರೆಂದು ಬಡಾಯಿಕೊಚ್ಚಿಕೊಳ್ಳುತ್ತೀರಿ!

“ಜೀವನ ಸಂಗ್ರಾಮದಲ್ಲಿಯೇ ಸರ್ವದಾ ಮುಳುಗಿಹೋಗಿರುವುದರಿಂದ ಕೆಳಗಿನ ಶ್ರೇಣಿಯ ಜನರಿಗೆ ಇಷ್ಟು ದಿನವೂ ಜ್ಞಾನೋದಯವಾಗಿಲ್ಲ. ಇವರು ಮಾನವ ಬುದ್ಧಿಯಿಂದ ನಡೆಸಲ್ಪಟ್ಟ ಯಂತ್ರಗಳ ಹಾಗೆ ಒಂದೇ ಭಾವದಲ್ಲಿ ಇಷ್ಟು ದಿನವೂ ಕೆಲಸ ಮಾಡಿಕೊಂಡು ಬಂದಿರುತ್ತಾರೆ ಮತ್ತು ಬುದ್ಧಿವಂತರೂ ಚತುರರೂ ಇವರ ಪರಿಶ್ರಮ ಸಂಪಾದನೆಗಳ ಸಾರವನ್ನು ಪಡೆದುಕೊಂಡಿರುತ್ತಾರೆ. ಸಮಸ್ತ ದೇಶದಲ್ಲಿಯೂ ಹೀಗೇ ಆಗಿದೆ. ಆದರೆ ಈಗ ಆ ಕಾಲ ಹೋಯಿತು. ಇತರ ಜಾತಿಯವರು ಈ ಸಂಗತಿಯನ್ನು ಕ್ರಮೇಣ ತಿಳಿದುಕೊಳ್ಳುತ್ತಾ ಇದ್ದಾರೆ ಮತ್ತು ಅದಕ್ಕೆ ವಿರೋಧವಾಗಿ ಎಲ್ಲರೂ ಸೇರಿ ನಿಂತು ತಮಗೆ ನ್ಯಾಯವಾಗಿ ಸಲ್ಲಬೇಕಾದ ಸ್ಥಾನವನ್ನು ಪಡೆಯುವುದಕ್ಕೆ ದೃಢ ಪ್ರತಿಜ್ಞೆಯನ್ನು ಕೈಕೊಂಡಿದ್ದಾರೆ. ಯೂರೋಪಿನಲ್ಲಿಯೂ ಅಮೆರಿಕಾದಲ್ಲಿಯೂ ಇತರ ಜಾತಿಯವರು ಮೊದಲು ಎಚ್ಚೆತ್ತು ಈ ಜಗಳವನ್ನು ಆರಂಭ ಮಾಡಿದ್ದಾರೆ. ಭರತಖಂಡದಲ್ಲಿಯೂ ಅದರ ಲಕ್ಷಣವು ಕಂಡುಬರುತ್ತಿದೆ. ಕೀಳು ಜಾತಿಯವರಲ್ಲಿ ಈಗ ಎಷ್ಟು ಮುಷ್ಕರವಿದೆ ಎಂಬುದರಿಂದಲೇ ಇದು ಗೊತ್ತಾಗುತ್ತದೆ. ಇನ್ನು ಸಾವಿರ ಸಾರಿ ಪ್ರಯತ್ನ ಪಟ್ಟರೂ ಮೇಲು ಜಾತಿಯವರು ಕೀಳು ಜಾತಿಯವರನ್ನು ತಡೆಯುವುದಕ್ಕಾಗುವುದಿಲ್ಲ. ಈಗ ಇತರ ಜಾತಿಯವರು ತಮಗೆ ನ್ಯಾಯವಾದ ಅಧಿಕಾರವನ್ನು ಪಡೆಯುವುದರಲ್ಲಿ ಸಹಾಯ ಮಾಡಿದರೆ ಅದರಿಂದ ಮೇಲು ಜಾತಿಯವರಿಗೆ ಒಳ್ಳೆಯದಾಗುತ್ತದೆ."

“ಅದಕ್ಕೇ ನಾನು ಹೇಳುವುದು, ಯಾತರಿಂದ ಈ ಸಾಧಾರಣ ಜನರಲ್ಲಿ ಜ್ಞಾನೋದಯವಾಗುತ್ತದೆಯೋ ಅದನ್ನು ಕೈಗೊಳ್ಳಿರಿ ಎಂದು. ಅವರಿಗೆ "ನೀವು ನಮ್ಮ ಸಹೋದರರು, ಶರೀರದ ಒಂದು ಅಂಗ - ನಾವು ನಿಮ್ಮನ್ನು ಪ್ರೀತಿಸುತ್ತೇವೆ, ದ್ವೇಷಿಸುವುದಿಲ್ಲ“ ಎಂದು ತಿಳಿಸಿ. ನಿಮ್ಮಿಂದ ಈ ಸಹಾನುಭೂತಿಯನ್ನು ಪಡೆದರೆ ಅವರು ನೂರರಷ್ಟು ಉತ್ಸಾಹದಿಂದ ಕಾರ್ಯತತ್ಪರರಾಗುತ್ತಾರೆ. ಆಧುನಿಕ ವಿಜ್ಞಾನಶಾಸ್ತ್ರದ ಸಹಾಯದಿಂದ ಅವರಿಗೆ ಜ್ಞಾನೋದಯವನ್ನು ಮಾಡಿಕೊಡಿ. ಚರಿತ್ರೆ, ಭೂಗೋಳ, ವಿಜ್ಞಾನ, ಸಾಹಿತ್ಯ - ಜೊತೆಜೊತೆಯಲ್ಲಿ ಧರ್ಮದ ಗೂಢ ತತ್ತ್ವ ಇವುಗಳನ್ನು ಹೇಳಿಕೊಡಿ. ಈ ವಿದ್ಯೆಗೆ ಫಲವಾಗಿ ಉಪಾಧ್ಯಾಯರ ದಾರಿದ್ರ್ಯವೂ ಅಡಗುತ್ತದೆ. ಆದಾನ ಪ್ರದಾನಗಳಿಂದ ಇಬ್ಬರೂ ಪರಸ್ಪರ ಮಿತ್ರರಂತಾಗುವಿರಿ."

ಶಿಷ್ಯ: ಆದರೆ, ಮಹಾಶಯರೆ, ಅವರಲ್ಲಿ ವಿದ್ಯೆಯು ಹರಡಿದರೆ, ಅವರೂ ಕಾಲಕ್ರಮದಲ್ಲಿ ನಮ್ಮ ಹಾಗೆ ಬುದ್ಧಿ ವಿಕಾಸವನ್ನು ಪಡೆದು ಉದ್ಯಮಹೀನರೂ ಆಲಸರೂ ಆಗಿ ತಮ್ಮ ಕೆಳಗಿರುವ ಜನರು ಪಡುವ ಪರಿಶ್ರಮದ ಸಾರವನ್ನು ತೆಗೆದುಕೊಳ್ಳುವರಲ್ಲವೆ?

ಸ್ವಾಮಿಜಿ: “ಹಾಗೆ ಏಕೆ ಆಗುತ್ತದೆ? ಜ್ಞಾನೋದಯವಾದರೂ ಕುಂಬಾರ ಕುಂಬಾರನಾಗಿಯೆ ಇರುತ್ತಾನೆ. ಬೆಸ್ತ ಬೆಸ್ತನಾಗಿಯೆ ಇರುತ್ತಾನೆ. ರೈತ ರೈತನಾಗಿಯೆ ಇರುತ್ತಾನೆ. ಜಾತಿಯ ಕರ್ಮವನ್ನು ಏಕೆ ಬಿಡುತ್ತಾನೆ? ‘ಸಹಜಂ ಕರ್ಮ ಕೌಂತೇಯ ಸದೋಷಮಪಿ ನ ತ್ಯಜೇತ್’ - ದೋಷಪೂರ್ಣವಾದದ್ದಾದರೂ, ನಿನಗೆ ಸಹಜವಾದ ಕರ್ಮವನ್ನು ಬಿಡಬೇಡ - ಈ ಭಾವದಲ್ಲಿ ಶಿಕ್ಷಣವನ್ನು ಪಡೆದರೆ, ಇವರು ತಮ್ಮ ವೃತ್ತಿಯನ್ನು ಏಕೆ ಬಿಡುವರು? ಜ್ಞಾನ ಬಲದಿಂದ ತಮ್ಮ ಸಹಜ ಕರ್ಮವನ್ನು ಹೇಗೆ ಉತ್ತಮಗೊಳಿಸಬಹುದೊ ಹಾಗೇ ಯತ್ನ ಮಾಡುತ್ತಾರೆ. ಕಾಲಕ್ರಮದಲ್ಲಿ ಹತ್ತಾರು ಜನ ಪ್ರಭಾವಶಾಲಿಗಳಾದವರು ಅವರಲ್ಲಿ ಹುಟ್ಟಿಯೇ ಹುಟ್ಟುತ್ತಾರೆ. ಅವರನ್ನು ನೀವು (ಮೇಲುಜಾತಿಯವರು) ನಿಮ್ಮ ಶ್ರೇಣಿಯೊಳಕ್ಕೆ ತೆಗೆದುಕೊಳ್ಳುತ್ತೀರಿ. ಬ್ರಾಹ್ಮಣರು ತೇಜಸ್ವಿಯಾದ ವಿಶ್ವಾಮಿತ್ರನನ್ನು ಬ್ರಾಹ್ಮಣನೆಂದು ಒಪ್ಪಿಕೊಂಡಾಗ ಕ್ಷತ್ರಿಯರು ಅವರಿಗೆ ಎಷ್ಟು ಕೃತಜ್ಞರಾಗಿದ್ದಿರಬಹುದು ನೋಡು! ಹೀಗೆ ಸಹಾನುಭೂತಿ ಮತ್ತು ಸಹಾಯಗಳನ್ನು ಪಡೆದರೆ ಮನುಷ್ಯರು ಹಾಗಿರಲಿ, ಪಶುಪಕ್ಷಿಗಳೂ ನಮ್ಮವುಗಳಾಗಿಬಿಡುವುವು."

ಶಿಷ್ಯ: ಮಹಾಶಯರೆ, ತಾವು ಹೇಳುವುದು ಸತ್ಯವಾದರೂ ಮೇಲು ಜಾತಿಗೆ ಸೇರಿದವರಲ್ಲಿಯೇ ಈಗಲೂ ಬಹು ತಾರತಮ್ಯವಿದೆ ಎಂದು ಕಾಣುತ್ತದೆ. ಭರತಖಂಡದ ಇತರ ಜಾತಿಯವರು ಮೇಲುಜಾತಿಯವರಿಗೆ ಸಹಾನುಭೂತಿಯನ್ನು ತೋರುವುದು ಬಹು ಕಷ್ಟದ ಕೆಲಸವೆಂದು ತೋರುತ್ತದೆ.

ಸ್ವಾಮೀಜಿ: ಅದು ಆಗದಿದ್ದರೆ ನಿಮಗೆ (ಮೇಲುಜಾತಿಯವರಿಗೆ) ಮಂಗಳ ಉಂಟಾಗುವುದಿಲ್ಲ. ನೀವು ಹಿಂದಿನಿಂದ ಮಾಡಿಕೊಂಡು ಬಂದಿರುವಂತೆ ಒಂದು ಮನೆಯಲ್ಲಿ ಪರಸ್ಪರ ಜಗಳ ಮಾಡಿಕೊಂಡು ಎಲ್ಲರೂ ಧ್ವಂಸವಾಗಿ ಬಿಡುವಿರಿ. ಈ ಸಾಧಾರಣ ಜನರು ಯಾವಾಗ ಎಚ್ಚೆತ್ತುಬಿಡುವರೊ ಆಗ ಅವರು ತಮ್ಮ ಮೇಲೆ ನಡೆದ ನಿಮ್ಮ ಅತ್ಯಾಚಾರವನ್ನು ತಿಳಿದುಕೊಳ್ಳುವರು. ಆಗ ಅವರು ಉಫ್ ಎಂದು ಊದಿದರೆ ನೀವು ಹೋಗಿ ಬೀಳುವಿರಿ! ಅವರೇ ನಿಮಗೆ ನಾಗರಿಕತೆಯನ್ನು ತಂದುಕೊಟ್ಟಿದ್ದಾರೆ, ಅವರೇ ಆಗ ಎಲ್ಲವನ್ನೂ ಹಾಳುಮಾಡುತ್ತಾರೆ. ಯೋಚಿಸಿನೋಡು - ಅಂಥ ಪ್ರಾಚೀನ ರೋಮನ್ ನಾಗರಿಕತೆ, ಗಾಲ್ ಜನಾಂಗದವರಿಂದ ನಾಶವಾಗಿ ಹೋಯಿತು! ಅದಕ್ಕೋಸ್ಕರವೆ ನಾನು ಹೇಳುವುದು, ಈ ಕೀಳುಜಾತಿಯವರಿಗೆಲ್ಲಾ ವಿದ್ಯಾದಾನ ಜ್ಞಾನದಾನಗಳನ್ನು ಮಾಡಿ ಇವರ ನಿದ್ರೆಯನ್ನು ತಪ್ಪಿಸುವುದಕ್ಕೆ ಯತ್ನಶೀಲರಾಗಿ ಎಂದು.

ಇವರು ಎಚ್ಚೆತ್ತುಕೊಂಡರೆಂದರೆ - ಒಂದು ದಿನ ನಿಜವಾಗಿಯೂ ಎಚ್ಚೆತ್ತುಕೊಳ್ಳುವರು - ಆಗ ಅವರೂ ನಿಮ್ಮಿಂದ ಆದ ಉಪಕಾರವನ್ನು ಮರೆಯುವುದಿಲ್ಲ. ನಿಮಗೆ ಕೃತಜ್ಞರಾಗಿರುತ್ತಾರೆ.

ಹೀಗೆ ಕಥೋಪಕಥನಗಳು ಆದಮೇಲೆ ಸ್ವಾಮೀಜಿ ಶಿಷ್ಯನನ್ನು ಕುರಿತು ಹೇಳಿದ್ದೇನೆಂದರೆ - “ಆ ಸಂಗತಿಯೆಲ್ಲಾ ಈಗ ಹಾಗಿರಲಿ - ನೀನು ಈಗ ಏನನ್ನು ಸ್ಥಿರಮಾಡಿಕೊಂಡೆ ಅದನ್ನು ಹೇಳು. ಯಾವುದಾದರೊಂದನ್ನು ಮಾಡು. ಯಾವುದಾದರೊಂದು ಕಸುಬಿನ ಪ್ರಯತ್ನ ಮಾಡು. ಇಲ್ಲದಿದ್ದರೆ ನನ್ನ ಹಾಗೆ ‘ಆತ್ಮನೋ ಮೋಕ್ಷಾರ್ಥಂ ಪರ ಹಿತಾಯ ಚ’ - ನಿಜವಾದ ಸಂನ್ಯಾಸ ಮಾರ್ಗದಲ್ಲಿ ಹೊರಟು ಬಾ. ಈ ಎರಡನೆಯ ಮಾರ್ಗವೆ ನಿಜವಾಗಿಯೂ ಶ್ರೇಷ್ಠವಾದ ಮಾರ್ಗ. ಅಯೋಗ್ಯ ಸಂಸಾರಿಯಾಗಿ ಆಗುವುದೇನು? ವಿಚಾರ ಮಾಡಿದರೆ ಎಲ್ಲವೂ ಕ್ಷಣಿಕವೆಂಬುದು ಗೊತ್ತಾಗುವುದು - ‘ನಲಿನೀದಲಗತ ಜಲಮತಿತರಲಂ ತದ್ವಜ್ಜೀವಿತಮತಿಶಯಚಪಲಂ’ - ತಾವರೆಯ ಮೇಲಿನ ನೀರಿನಂತೆ ಜೀವನ ಕ್ಷಣಿಕ. ಆದ್ದರಿಂದ ಈ ಆತ್ಮಜ್ಞಾನವನ್ನು ಪಡೆಯುವುದಕ್ಕೆ ಉತ್ಸಾಹವಿದ್ದರೆ ಇನ್ನು ಕಾಲವಿಳಂಬ ಮಾಡಬೇಡ; ಮುಂದಾಗು. ‘ಯದಹರೇವ ವಿರಜೇತ್ ತದಹರೇವ ಪ್ರವ್ರಜೇತ್’ - ಯಾವ ದಿನ ಈ ಜಗತ್ತಿನ ವಿಷಯದಲ್ಲಿ ವೈರಾಗ್ಯ ಬರುತ್ತದೆಯೋ ಅದೇ ದಿನ ತ್ಯಾಗ ಮಾಡಿ ಸಂನ್ಯಾಸವನ್ನು ಸ್ವೀಕರಿಸು. ಅನ್ಯರಿಗೋಸ್ಕರ ನಿನ್ನ ಜೀವನವನ್ನು ಬಲಿ ಕೊಟ್ಟು ಜನರ ಮನೆಯ ಬಾಗಿಲು ಬಾಗಿಲಿಗೂ ಹೋಗಿ ಅಭಯವಾಣಿಯನ್ನು ಹೇಳು - ‘ಉತ್ತಿಷ್ಠತ ಜಾಗ್ರತ ಪ್ರಾಪ್ಯವರಾನ್ನಿಬೋಧತ’ ಎಂದು."

\newpage

\chapter[ಅಧ್ಯಾಯ ೨೦]{ಅಧ್ಯಾಯ ೨೦\protect\footnote{\engfoot{C.W, Vol. VII, P. 151}}}

\begin{center}
ಸ್ಥಳ: ಕಲ್ಕತ್ತ, ವರ್ಷ: ಕ್ರಿ.ಶ. ೧೮೯೮.
\end{center}

ಈಗ ಮೂರು ದಿನಗಳಿಂದ ಸ್ವಾಮೀಜಿ ಬಾಗಬಜಾರಿನ ಬಲರಾಮ ಬಸುಗಳ ಮನೆಯಲ್ಲಿದ್ದಾರೆ. ಪ್ರತಿನಿತ್ಯವೂ ಅಸಂಖ್ಯ ಜನರ ಗುಂಪು. ಸ್ವಾಮಿ ಯೋಗಾನಂದರೂ ಸ್ವಾಮೀಜಿ ಜೊತೆಯಲ್ಲಿ ಇರುತ್ತಿದ್ದಾರೆ. ಇಂದು ಸೋದರಿ ನಿವೇದಿತಾಳನ್ನು ಜೊತೆಯಲ್ಲಿ ಕರೆದುಕೊಂಡು ಸ್ವಾಮೀಜಿ ಆಲಿಪುರದ ಮೃಗಶಾಲೆಯನ್ನು ನೋಡುವುದಕ್ಕೆ ಹೋಗುವರು. ಶಿಷ್ಯನು ಬಂದಾಗ ಅವನಿಗೂ ಯೋಗಾನಂದರಿಗೂ “ನೀವು ಮೊದಲು ಹೊರಟು ಹೋಗಿ - ನಾನು ನಿವೇದಿತಾಳನ್ನು ಕರೆದುಕೊಂಡು ಸ್ವಲ್ಪ ಹೊತ್ತು ಬಿಟ್ಟುಕೊಂಡು ಗಾಡಿಯಲ್ಲಿ ಬರುತ್ತೇನೆ” ಎಂದು ಹೇಳಿದರು.

ಯೋಗಾನಂದ ಸ್ವಾಮಿಗಳು ಶಿಷ್ಯನನ್ನು ಜೊತೆಯಲ್ಲಿ ಕರೆದುಕೊಂಡು ಸುಮಾರು ಎರಡೂವರೆ ಗಂಟೆಗೆ ಟ್ರಾಮ್ ಗಾಡಿಯಲ್ಲಿ ಹೊರಟರು. ಸುಮಾರು ನಾಲ್ಕು ಗಂಟೆಯ ಹೊತ್ತಿಗೆ ಮೃಗಶಾಲೆಗೆ ಬಂದು ಸೇರಿ ತೋಟದ ಸೂಪರಿಂಟೆಂಡೆಂಟರಾದ ಬಾಬು ರಾಮಬ್ರಹ್ಮ ಸನ್ಯಾಲ ರಾಯಬಹದ್ದೂರರನ್ನು ನೋಡಿದರು. ಸ್ವಾಮೀಜಿ ಬರುತ್ತಾರೆಂದು ಕೇಳಿದ ರಾಮಬ್ರಹ್ಮಬಾಬುಗಳು ಬಹಳ ಸಂತೋಷಪಟ್ಟು ಸ್ವಾಮೀಜಿಯವರನ್ನು ಆದರದಿಂದ ಬರಮಾಡಿಕೊಳ್ಳುವುದಕ್ಕೋಸ್ಕರ ತೋಟದ ಬಾಗಿಲಲ್ಲಿ ನಿಂತುಕೊಂಡರು. ಸುಮಾರು ನಾಲ್ಕೂವರೆ ಗಂಟೆಯ ಹೊತ್ತಿಗೆ ಸ್ವಾಮೀಜಿಯವರು ನಿವೇದಿತಾಳನ್ನು ಕರೆದುಕೊಂಡು ಅಲ್ಲಿಗೆ ಬಂದರು. ರಾಮಬ್ರಹ್ಮ ಬಾಬುಗಳು ಪರಮ ಆದರದಿಂದ ಸ್ವಾಮೀಜಿಯನ್ನೂ ನಿವೇದಿತಾಳನ್ನೂ ಎದುರುಗೊಂಡು ಮೃಗಶಾಲೆಯೊಳಕ್ಕೆ ಕರೆದುಕೊಂಡು ಹೋದರು; ಮತ್ತು ಸುಮಾರು ಒಂದೂವರೆ ಘಂಟೆಯ ಹೊತ್ತು ಅವರ ಜೊತೆಯಲ್ಲಿ ತಿರುಗುತ್ತ ತೋಟದ ನಾನಾ ಸ್ಥಾನಗಳನ್ನು ತೋರಿಸುತ್ತ ಬಂದರು. ಯೋಗಾನಂದ ಸ್ವಾಮಿಗಳೂ ಶಿಷ್ಯನೊಡನೆ ಅವರ ಹಿಂದೆ ಹಿಂದೆಯೇ ಹೋದರು.

ರಾಮಬ್ರಹ್ಮಬಾಬುಗಳು ವನಸ್ಪತಿ ಶಾಸ್ತ್ರದಲ್ಲಿ ಒಳ್ಳೆಯ ಪಂಡಿತರಾಗಿದ್ದರು. ಆದ್ದರಿಂದ ಆ ತೋಟದಲ್ಲಿದ್ದ ನಾನಾ ವೃಕ್ಷಗಳನ್ನು ತೋರಿಸುತ್ತ ವನಸ್ಪತಿ ಶಾಸ್ತ್ರಾನುಸಾರವಾಗಿ ಮರಗಿಡಗಳಲ್ಲಿ ಕಾಲಕ್ರಮದಿಂದ ಎಂಥ ಕ್ರಮ ಪರಿಣತಿಯುಂಟಾಗುವುದೆಂಬುದನ್ನು ಆಗಾಗ್ಗೆ ವಿಚಾರ ಮಾಡುತ್ತಿದ್ದರು. ನಾನಾ ಜೀವಜಂತುಗಳನ್ನು ನೋಡುತ್ತ ಸ್ವಾಮೀಜಿ ಮಧ್ಯೆ ಮಧ್ಯೆ ಜೀವದ ಉತ್ತರೋತ್ತರ ವಿಕಾಸದ ಸಂಬಂಧವಾಗಿ ಡಾರ್ವಿನ್ ಮತವನ್ನು ವಿಚಾರ ಮಾಡುತ್ತ ಬಂದರು. ಶಿಷ್ಯನಿಗೆ ಇನ್ನೂ ಜ್ಞಾಪಕವಿದೆ, ಏನೆಂದರೆ - ಅವರು ಹಾವಿನ ಮನೆಗೆ ಬಂದು ಅಲ್ಲಿ ಚಕ್ರಾಕಾರವಾಗಿ ಸುತ್ತಿಕೊಂಡಿದ್ದ ಒಂದು ಘಟಸರ್ಪವನ್ನು ತೋರಿಸಿ “ಇದರಿಂದಲೇ ಆಮೇಲೆ ಆಮೆ ಬಂದದ್ದು. ಸರ್ಪ ಬಹುಕಾಲ ಒಂದೇ ಕಡೆ ಬಿದ್ದಿದ್ದು ಗಟ್ಟಿಯಾದ ಬೆನ್ನುಳ್ಳದ್ದಾಗಿದೆ" ಎಂದು ಹೇಳಿದರು. ಈ ಮಾತನ್ನು ಹೇಳಿ ಸ್ವಾಮಿಜಿಯವರು ಶಿಷ್ಯನನ್ನು ಕುರಿತು “ನೀವು ಆಮೆಯನ್ನು ತಿನ್ನುತ್ತೀರಲ್ಲವೇ? ಡಾರ್ವಿನ್ ಮತಾನುಸಾರವಾಗಿ ಈ ಹಾವೇ ಕಾಲಕ್ರಮದಲ್ಲಿ ಆಮೆಯಾಗಿ ವಿಕಾಸವಾಗಿದೆ - ಆದ್ದರಿಂದ ನೀವು ಹಾವನ್ನೂ ತಿಂದ ಹಾಗಾಯಿತು" ಎಂದು ಹಾಸ್ಯ ಮಾಡಿದರು.

ಶಿಷ್ಯನು ಈ ಮಾತನ್ನು ಕೇಳಿ ಅಸಹ್ಯದಿಂದ ಮುಖವನ್ನು ತಿರುಗಿಸಿಕೊಂಡು “ಮಹಾಶಯರೆ, ಒಂದು ಪದಾರ್ಥ ಕ್ರಮವಿಕಾಸದ ಮೂಲಕ ಬೇರೊಂದು ಪದಾರ್ಥವಾಗಿ ಹೋದರೆ ಆಗ ಅದರ ಪೂರ್ವಾಕೃತಿ ಮತ್ತು ಸ್ವಭಾವಗಳು ಇರುವುದಿಲ್ಲ. ಆದ್ದರಿಂದ ಆಮೆಯನ್ನು ತಿಂದರೆ ಹಾವನ್ನು ತಿಂದಹಾಗಾಯಿತು ಎಂದು ಹೇಗೆ ಹೇಳುವಿರಿ?” ಎಂದು ಕೇಳಿದನು.

ಶಿಷ್ಯನ ಮಾತು ಕೇಳಿ ಸ್ವಾಮಿಜಿಯೂ ರಾಮಬ್ರಹ್ಮ ಬಾಬುಗಳೂ ನಗುವುದಕ್ಕೆ ತೊಡಗಿದರು; ಮತ್ತು ಸೋದರಿ ನಿವೇದಿತಾಗೆ ಈ ಮಾತನ್ನು ತಿಳಿಸಲು ಆಕೆಯೂ ನಗುವುದಕ್ಕೆ ಮೊದಲುಮಾಡಿದಳು. ಕ್ರಮವಾಗಿ ಎಲ್ಲರೂ ಹುಲಿ ಸಿಂಹಗಳನ್ನು ಇಟ್ಟಿದ್ದ ಮನೆಯ ಕಡೆಗೆ ತಿರುಗಿದರು.

ರಾಮಬ್ರಹ್ಮ ಬಾಬುಗಳ ಅಪ್ಪಣೆಯಂತೆ ಕಾವಲಿನವರು ಬೇಕಾದಷ್ಟು ಮಾಂಸವನ್ನು ತಂದು ನಮ್ಮಗಳೆದುರಿಗೆ ಹುಲಿ ಸಿಂಹಗಳಿಗೆ ಅದನ್ನು ತಿನ್ನಿಸುವುದಕ್ಕೆ ಮೊದಲುಮಾಡಿದರು. ಅವು ಸಂತೋಷದಿಂದ ಗರ್ಜಿಸುವುದನ್ನೂ ಮತ್ತು ಆತುರದಿಂದ ತಿನ್ನುವುದನ್ನೂ ನೋಡಿದ ನಂತರ ಆ ತೋಟದ ಮಧ್ಯದಲ್ಲಿದ್ದ ರಾಮಬ್ರಹ್ಮಬಾಬುಗಳ ಮನೆಗೆ ನಾವೆಲ್ಲರೂ ಬಂದೆವು. ಅಲ್ಲಿ ಟೀ ಮತ್ತು ಫಲಾಹಾರಕ್ಕೆ ಏರ್ಪಾಡು ಆಗಿತ್ತು. ಸ್ವಾಮೀಜಿ ಸ್ವಲ್ಪವೇ ಸ್ವಲ್ಪ ಟೀಯನ್ನು ಕುಡಿದರು. ನಿವೇದಿತಾಳೂ ಟೀ ಕುಡಿದಳು. ಒಂದೇ ಮೇಜಿನ ಮೇಲೆ ಸಿಸ್ಟರ್ ನಿವೇದಿತಾ ಮುಟ್ಟಿದ ಪಕ್ವಾನ್ನ ಮತ್ತು ಟೀಯನ್ನು ತೆಗೆದುಕೊಳ್ಳುವುದಕ್ಕೆ ಶಿಷ್ಯನು ಸಂಕೋಚಪಡುತ್ತಿದ್ದುದನ್ನು ನೋಡಿ ಸ್ವಾಮೀಜಿ ಅವನಿಗೆ ಪುನಃ ಪುನಃ ಹೇಳಿ ಅದನ್ನು ತಿನ್ನಿಸಿದರು ಮತ್ತು ತಾವು ನೀರು ಕುಡಿದು ಮಿಕ್ಕದ್ದನ್ನು ಶಿಷ್ಯನಿಗೆ ಕುಡಿಯುವುದಕ್ಕೆ ಕೊಟ್ಟರು. ಆಮೇಲೆ ಡಾರ್ವಿನ್‌ನ ಕ್ರಮವಿಕಾಸವಾದದ ಮೇಲೆ ಸ್ವಲ್ಪ ಹೊತ್ತು ಮಾತುಕಥೆಗಳು ನಡೆದುವು.

ರಾಮಬ್ರಹ್ಮಬಾಬು: ಡಾರ್ವಿನ್ ಸಾಹೇಬನು ಕ್ರಮವಿಕಾಸ ಮತ್ತು ಅದಕ್ಕೆ ಕಾರಣ ಇವುಗಳ ಸಂಬಂಧವಾಗಿ ಹೇಳಿರುವುದರ ಮೇಲೆ ತಮ್ಮ ಅಭಿಪ್ರಾಯವೇನು?

- ಸ್ವಾಮೀಜಿ: ಡಾರ್ವಿನ್‌ನ ಮಾತು ನಿಜವಾದರೂ ಕ್ರಮವಿಕಾಸದ ಕಾರಣ ಸಂಬಂಧವಾಗಿ ಆತನು ಹೇಳಿರುವುದೇನು ಚರಮ ಸಿದ್ಧಾಂತವೆಂದು ನಾನು ಒಪ್ಪುವುದಿಲ್ಲ.

ರಾಮಬ್ರಹ್ಮಬಾಬು: ಈ ವಿಷಯದಲ್ಲಿ ನಮ್ಮ ದೇಶದ ಪ್ರಾಚೀನ ಪಂಡಿತರು ಏನಾದರೂ ವಿಚಾರ ಮಾಡಿದ್ದಾರೆಯೆ?

ಸ್ವಾಮೀಜಿ: ಸಾಂಖ್ಯದರ್ಶನದಲ್ಲಿ ಈ ವಿಷಯ ಸೊಗಸಾಗಿ ವಿಚಾರ ಮಾಡಲ್ಪಟ್ಟಿದೆ. ಭರತಖಂಡದ ಪ್ರಾಚೀನ ದಾರ್ಶನಿಕರ ಸಿದ್ಧಾಂತವೇ ಕ್ರಮವಿಕಾಸದ ಕಾರಣ ಸಂಬಂಧವಾಗಿ ಚರಮ ಸಿದ್ಧಾಂತವೆಂಬುದು ನನ್ನ ಅಭಿಪ್ರಾಯ.

ರಾಮಬ್ರಹ್ಮ ಬಾಬು: ಈ ಸಿದ್ಧಾಂತವನ್ನು ಸಂಕ್ಷೇಪವಾಗಿ ತಿಳಿಸುತ್ತಾ ಹೋದರೆ ಕೇಳಬೇಕೆಂದು ನನಗೆ ಆಶೆಯಾಗಿದೆ.

ಸ್ವಾಮೀಜಿ: ನಿಮ್ನ ಜಾತಿಯನ್ನು ಉಚ್ಚ ಜಾತಿಯಾಗಿ ವಿಕಾಸಗೊಳಿಸುವುದಕ್ಕೆ ಪಾಶ್ಚಾತ್ಯರು ಹೇಳುವ ಜೀವನ ಸಂಗ್ರಾಮ, ಯೋಗ್ಯತಮವಾದದ್ದು ಮಾತ್ರ ಉಳಿದುಕೊಳ್ಳುತ್ತದೆ, ಪ್ರಾಕೃತಿಕ ಆಯ್ಕೆ ಮುಂತಾದ ನಿಯಮಗಳೂ ಕಾರಣಗಳೂ ತಮಗೆ ಗೊತ್ತೇ ಇವೆ. ಪತಂಜಲಿ ದರ್ಶನದಲ್ಲಿ ಮಾತ್ರ ಇವೊಂದೂ ಕಾರಣವೆಂದು ತೋರಿಸಿಲ್ಲ. ಪತಂಜಲಿಗಳ ಮತವೇನೆಂದರೆ - ಒಂದು ಜಾತಿಯಿಂದ ಮತ್ತೊಂದು ಜಾತಿಗೆ ವಿಕಾಸವು ‘ಪ್ರಕೃತಿಯ ಆಪೂರಣದಿಂದ’ ಆಗುತ್ತದೆ. ವಾತಾವರಣದೊಡನೆ ಹಗಲೂ ರಾತ್ರಿ ಹೋರಾಡಿ ಅದು ಆಗುವುದೆಂಬುದಲ್ಲ. ನಾನು ವಿವೇಚನೆ ಮಾಡಿರುವುದರಲ್ಲಿ ಹೋರಾಟ ಮತ್ತು ಸ್ಪರ್ಧೆ ಇವು ಜೀವಕ್ಕೆ ಪೂರ್ಣತೆಯುಂಟಾಗುವುದರಲ್ಲಿ ಅನೇಕ ವೇಳೆ ಪ್ರತಿಬಂಧಕವಾಗಿ ನಿಲ್ಲುತ್ತವೆ. ಪಾಶ್ಚಾತ್ಯ ದರ್ಶನದಲ್ಲಿ ಹೇಳಿರುವಂತೆ ಸಾವಿರ ಜೀವರ ಧ್ವಂಸವಾಗಿ ಒಂದು ಜೀವನದ ಕ್ರಮೋನ್ನತಿಯಾಗುವುದಾದರೆ ಈ ಕ್ರಮ ವಿಕಾಸದಿಂದ ಜಗತ್ತಿಗೆ ಹೆಚ್ಚಿನ ಒಳ್ಳೆಯದಾಗುವುದಿಲ್ಲ ಎಂದು ಹೇಳಬೇಕಾಗುತ್ತದೆ. ಲೌಕಿಕ ಉನ್ನತಿಯ ವಿಚಾರದಲ್ಲಿ ಇದನ್ನು ಒಪ್ಪಿಕೊಂಡರೂ ಆಧ್ಯಾತ್ಮಿಕ ವಿಕಾಸಕ್ಕೆ ಅದು ತುಂಬ ಪ್ರತಿಬಂಧಕವೆಂಬುದನ್ನು ಒಪ್ಪಿಕೊಳ್ಳಬೇಕು. ನಮ್ಮ ದೇಶದ ದಾರ್ಶನಿಕರ ಅಭಿಪ್ರಾಯವೇನೆಂದರೆ - ಪ್ರತಿಯೊಂದು ಜೀವಿಯೂ ಪರಿಪೂರ್ಣ ಆತ್ಮ ಮತ್ತು ಪ್ರಕೃತಿಯ ವಿಕಾಸ ಮತ್ತು ಅಭಿವ್ಯಕ್ತಿಯ ವೈವಿಧ್ಯಕ್ಕೆ ಕಾರಣ ಆತ್ಮದ ಅಭಿವ್ಯಕ್ತಿಯ ಪ್ರಮಾಣದಲ್ಲಿರುವ ವ್ಯತ್ಯಾಸ. ಪ್ರಕೃತಿಯ ಅಭಿವ್ಯಕ್ತಿಗೂ ವಿಕಾಸಕ್ಕೂ ಇರುವ ಪ್ರತಿಬಂಧಕಗಳನ್ನು ಪೂರ್ಣವಾಗಿ ತೆಗೆದುಹಾಕಿಬಿಟ್ಟ ಮೇಲೆ ಆತ್ಮವು ಪೂರ್ಣವಾಗಿ ಪ್ರಕಾಶಿತವಾಗುತ್ತದೆ. ಪ್ರಕೃತಿಯ ಅಭಿವ್ಯಕ್ತಿಯು ಕೆಳಗಣ ಮೆಟ್ಟಲುಗಳಲ್ಲಿ ಹೇಗಾದರೂ ಇರಲಿ, ಮೇಲಣ ಮೆಟ್ಟಲುಗಳಲ್ಲಿ ಪ್ರತಿಬಂಧಕಗಳನ್ನು ಅತಿಕ್ರಮಿಸಿ ಹೋಗುವುದು, ಅವುಗಳೊಡನೆ ಹಗಲೂ ರಾತ್ರಿ ಹೋರಾಡುವುದರಿಂದಲ್ಲ; ಅಲ್ಲಿ ಶಿಕ್ಷಣ, ಸಂಸ್ಕಾರ, ಏಕಾಗ್ರತೆ ಮತ್ತು ಮುಖ್ಯವಾಗಿ ತ್ಯಾಗ ಇವುಗಳ ಮೂಲಕ ಪ್ರತಿಬಂಧಕಗಳು ಸರಿದು ಹೋಗುತ್ತವೆ ಅಥವಾ ಆತ್ಮವು ಹೆಚ್ಚು ಪ್ರಕಾಶಿತವಾಗುವುದು ಕಂಡುಬರುತ್ತದೆ. ಆದ್ದರಿಂದ ಪ್ರತಿಬಂಧಕಗಳನ್ನು ಆತ ಪ್ರಕಾಶದ ಕಾರ್ಯವೆಂದು ಹೇಳದೆ ಕಾರಣರೂಪವಾಗಿ ತಂದಿಡುವುದು ಮತ್ತು ಪ್ರಕೃತಿಯ ಈ ವೈವಿಧ್ಯದ ಅಭಿವ್ಯಕ್ತಿಗೆ ಸಹಕಾರಿಗಳೆಂದು ಹೇಳುವುದೂ ಯುಕ್ತಿಯುಕ್ತವಲ್ಲ; ಸಾವಿರ ಪಾಪಿಗಳನ್ನು ಸಂಹಾರಮಾಡಿ ಜಗತ್ತಿನಿಂದ ಪಾಪವನ್ನು ಹೋಗಲಾಡಿಸುವ ಪ್ರಯತ್ನದಿಂದ ಜಗತ್ತಿನಲ್ಲಿ ಪಾಪವು ಇನ್ನೂ ಹೆಚ್ಚುತ್ತದೆ. ಆದರೆ ಆಧ್ಯಾತ್ಮಿಕ ಉಪದೇಶದ ಮೂಲಕ ಜನರು ಪಾಪವನ್ನು ಆಚರಿಸದಂತೆ ಮಾಡಿದರೆ ಆಮೇಲೆ ಜಗತ್ತಿನಲ್ಲಿ ಪಾಪವು ಇರುವುದಿಲ್ಲ. ಈಗ ನೋಡಿ, ಪಾಶ್ಚಾತ್ಯರ ಸಿದ್ಧಾಂತದಂತೆ ಜೀವಗಳು ಪರಸ್ಪರ ಸಂಗ್ರಾಮ ಮತ್ತು ಪೈಪೋಟಿಯ ಮೂಲಕ ಉತ್ತಮ ಸ್ಥಿತಿಯನ್ನು ಪಡೆಯುತ್ತವೆ ಎಂಬ ಸಿದ್ಧಾಂತವು ಎಷ್ಟು ಭಯಂಕರವಾಗಿ ಪರಿಣಮಿಸಿದೆ.

ರಾಮಬ್ರಹ್ಮ ಬಾಬುಗಳು ಸ್ವಾಮಿಜಿಯ ಮಾತನ್ನು ಕೇಳಿ ಸ್ತಂಭೀಭೂತರಾಗಿದ್ದು ಕೊನೆಗೆ ಹೇಳಿದ್ದೇನೆಂದರೆ: “ತಮ್ಮ ಹಾಗೆ ಪ್ರಾಚ್ಯ ಮತ್ತು ಪಾಶ್ಚಾತ್ಯ ದರ್ಶನಗಳಲ್ಲಿ ಅಭಿಜ್ಞರಾದ ಜನಗಳ ಆವಶ್ಯಕತೆಯು ಈಗ ಭರತಖಂಡದಲ್ಲಿ ವಿಶೇಷವಾಗಿದೆ. ಒಂದು ನಾಣ್ಯದ ಒಂದೇ ಮುಖವನ್ನು ನೋಡುವ ಜನರ ಪ್ರಮಾದಗಳನ್ನು ತೋರಿಸಲು ಇಂಥವರು ಸಮರ್ಥರು. ಕ್ರಮ ವಿಕಾಸವಾದದ ತಮ್ಮ ನೂತನ ವ್ಯಾಖ್ಯಾನವನ್ನು ಕೇಳಿ ನನಗೆ ಪರಮ ಸಂತೋಷವಾಯಿತು.”

ಹೊರಡುವಾಗ ರಾಮಬ್ರಹ್ಮಬಾಬುಗಳು ತೋಟದ ಹೊರಬಾಗಿಲಿನವರೆಗೆ ಬಂದು ಸ್ವಾಮಾಜಿಯನ್ನು ಕಳುಹಿಸಿಕೊಟ್ಟು ಇನ್ನೊಂದು ಸಾರಿ ನಿಧಾನವಾಗಿ ಒಬ್ಬರೇ ಇರುವಾಗ, ದರ್ಶನ ತೆಗೆದುಕೊಳ್ಳುವುದಾಗಿ ಮಾತುಕೊಟ್ಟರು. ರಾಮಬ್ರಹ್ಮ ಬಾಬುಗಳಿಗೆ ಈ ಜೀವಮಾನದಲ್ಲಿ ಪುನಃ ಸ್ವಾಮೀಜಿ ಹತ್ತಿರ ಬರುವುದಕ್ಕೆ ಅವಕಾಶ ಸಿಕ್ಕಿತೋ ಇಲ್ಲವೋ ನನಗೆ ತಿಳಿಯದು; ಏಕೆಂದರೆ ಇದಾದ ಸ್ವಲ್ಪ ದಿನದ ಮೇಲೆ ಬಾಬುಗಳು ಸ್ವರ್ಗಸ್ಥರಾದರು.

ಶಿಷ್ಯನು ಯೋಗಾನಂದ ಸ್ವಾಮಿಗಳೊಡನೆ ಟ್ರಾಂನಲ್ಲಿ ರಾತ್ರಿ ಎಂಟು ಗಂಟೆಯ ಹೊತ್ತಿಗೆ ಬಾಗಬಜಾರಿಗೆ ಹಿಂತಿರುಗಿ ಬಂದನು. ಸ್ವಾಮೀಜಿ ಇದಕ್ಕೆ ಸುಮಾರು ಹದಿನೈದು ನಿಮಿಷ ಮುಂಚೆ ಬಂದು ವಿಶ್ರಾಂತಿ ತೆಗೆದುಕೊಳ್ಳುತ್ತಿದ್ದರು. ಸುಮಾರು ಅರ್ಧಗಂಟೆ ವಿಶ್ರಮಿಸಿಕೊಂಡ ಮೇಲೆ ಅವರು ಬೈಠಕ್‌ಖಾನೆಯಲ್ಲಿ ನಾವಿದ್ದ ಕಡೆಗೆ ಬಂದರು. ಆಗ ಅಲ್ಲಿ ಯೋಗಾನಂದ, ಶರಚ್ಚಂದ್ರ ಸರ್ಕಾರ್, ಶಶಿಭೂಷಣ ಘೋಷ್, ವಿಪಿನ ವಿಹಾರಿ ಘೋಷ್, ಶಾಂತಿರಾಮ ಘೋಷ್ ಮುಂತಾದ ಪರಿಚಿತರೂ, ಸ್ನೇಹಿತರೂ ಮತ್ತು ಸ್ವಾಮೀಜಿಯ ದರ್ಶನಾರ್ಥವಾಗಿ ಬಂದಿದ್ದ ಅಪರಿಚಿತರಾದ ಐದು ಆರು ಜನರೂ ಇದ್ದರು. ಸ್ವಾಮೀಜಿ ಇಂದು ಮೃಗಶಾಲೆಯನ್ನು ನೋಡುವುದಕ್ಕೆ ಹೋಗಿ ರಾಮಬ್ರಹ್ಮಬಾಬುಗಳೊಡನೆ ಕ್ರಮವಿಕಾಸವಾದದ ಅಪೂರ್ವ ವಿಚಾರ ಮಾಡಿದ್ದರೆಂದು ಕೇಳಿ ಅವರೆಲ್ಲಾ ಈ ಪ್ರಸ್ತಾವವನ್ನು ವಿಸ್ತಾರವಾಗಿ ಹೇಳಬೇಕೆಂದು ಮೊದಲೇ ಉತ್ಸುಕರಾಗಿದ್ದರು. ಆದ್ದರಿಂದ ಅವರು ಬಂದ ಕೂಡಲೆ ಎಲ್ಲರ ಅಭಿಪ್ರಾಯವನ್ನೂ ತಿಳಿದುಕೊಂಡ ಶಿಷ್ಯನು ಆ ಮಾತನ್ನೇ ಎತ್ತಿದನು.

ಶಿಷ್ಯ: ಮಹಾಶಯರೆ, ಮೃಗಶಾಲೆಯಲ್ಲಿ ಕ್ರಮವಿಕಾಸದ ವಿಚಾರವಾಗಿ ತಾವು ಹೇಳಿದ್ದನ್ನು ನಾನು ಚೆನ್ನಾಗಿ ಅರ್ಥಮಾಡಿಕೊಳ್ಳುವುದಕ್ಕಾಗಲಿಲ್ಲ. ದಯಮಾಡಿ ಸುಲಭವಾದ ಮಾತಿನಲ್ಲಿ ಅದನ್ನು ಪುನಃ ತಿಳಿಸಿಕೊಡುವಿರಾ?

ಸ್ವಾಮೀಜಿ: ಏಕೆ, ಅರ್ಥವಾಗಲಿಲ್ಲವೆ?

ಶಿಷ್ಯ: ತಾವೇ ಅನೇಕಸಲ ನನಗೆ ಹೇಳಿದ್ದೀರಿ, ಹೊರಗಿನ ಶಕ್ತಿಗಳೊಡನೆ ಯುದ್ಧ ಮಾಡುವುದಕ್ಕೆ ಇರುವ ಸಾಮರ್ಥ್ಯವೇ ಜೀವನದ ಚಿಹ್ನೆ, ಅದೇ ಉನ್ನತಿಗೆ ಸೋಪಾನ ಎಂದು. ಈ ದಿವಸ ಅದಕ್ಕೆ ವಿರೋಧವಾಗಿ ಹೇಳಿದಂತಿತ್ತು.

ಸ್ವಾಮೀಜಿ: ವಿರೋಧವಾಗಿ ಯಾಕೆ ಹೇಳಲಿ? ನೀನೇ ತಿಳಿದುಕೊಳ್ಳಲಾರದೆ ಹೋದೆ. ಪ್ರಾಣಿ ಜಗತ್ತಿನಲ್ಲಿ ನಾವು ನಿಜವಾಗಿಯೂ ಜೀವನಸ್ಪರ್ಧೆ ಮತ್ತು ಯೋಗ್ಯತಮವೇ ಉಳಿಯುವುದು ಮುಂತಾದ ನಿಯಮಗಳು ಇರುವುದನ್ನು ಸ್ಪಷ್ಟವಾಗಿ ನೋಡಬಹುದು. ಆದ್ದರಿಂದಲೇ ಡಾರ್ವಿನ್‌ನ ಮತ ಕೆಲವು ಮಟ್ಟಿಗೆ ನಿಜವೆಂದು ತೋರುತ್ತದೆ. ಆದರೆ ಬುದ್ಧಿಯ ವಿಕಾಸವುಳ್ಳ ಮನುಷ್ಯ ಜಗತ್ತಿನಲ್ಲಿ ಈ ನಿಯಮದ ವಿರೋಧವೇ ಕಂಡುಬರುತ್ತದೆ. ಇದನ್ನು ಯೋಚಿಸಿನೋಡು, ಯಾರನ್ನು ನಾವು ನಿಜವಾಗಿಯೂ ಮಹಾತ್ಮರು ಅಥವಾ ಆದರ್ಶ ಪುರುಷರು ಎಂದು ಬಲ್ಲೆವೋ ಅವರಲ್ಲಿ ಬಾಹ್ಯ ಹೋರಾಟ ಸುತರಾಂ ಇರುವುದಿಲ್ಲ. ಮನುಷ್ಯೇತರ ಪ್ರಾಣಿಜಗತ್ತಿನಲ್ಲಿ ಜ್ಞಾನದ ಪ್ರಾಬಲ್ಯ ಅಸ್ವಾಭಾವಿಕ. ಮನುಷ್ಯನಾದರೋ ಎಷ್ಟೆಷ್ಟು ದೊಡ್ಡವನಾದರೆ ಅಷ್ಟಷ್ಟು ಅವನಲ್ಲಿ ಬುದ್ಧಿಯ ವಿಕಾಸ. ಆದ್ದರಿಂದ ಪ್ರಾಣಿಗಳಂತೆ ಮಾನವ ಪ್ರಪಂಚದಲ್ಲೂ ಪರರನ್ನು ಧ್ವಂಸಮಾಡಿ ಅದರ ಮೂಲಕ ಉನ್ನತಿ ಆಗುವುದು ಸಾಧ್ಯವಲ್ಲ. ಮಾನವನ ಸರ್ವಶ್ರೇಷ್ಠ ಪೂರ್ಣ ವಿಕಾಸ ತ್ಯಾಗವೊಂದರಿಂದಲೇ ಸಾಧಿತವಾಗುತ್ತದೆ. ಯಾರು ಅನ್ಯರಿಗೋಸ್ಕರ ಎಷ್ಟೆಷ್ಟು ತ್ಯಾಗ ಮಾಡಬಲ್ಲರೋ ಅವರು ಮನುಷ್ಯರಲ್ಲಿ ಅಷ್ಟಷ್ಟು ಶ್ರೇಷ್ಠರಾಗುವರು; ಕೆಳಗಿನ ಮೆಟ್ಟಲಾದ ಪ್ರಾಣಿ ಜಗತ್ತಿನಲ್ಲಿ ಯಾವುದು ಎಷ್ಟೆಷ್ಟು ನಾಶಪಡಿಸಬಲ್ಲದೊ ಅದು ಅಷ್ಟು ಬಲಶಾಲಿಯಾದ ಮೃಗವಾಗುವುದು. ಆದ್ದರಿಂದ ಜೀವನ ಸಂಗ್ರಾಮವೆಂಬ ಸಿದ್ಧಾಂತವು ಈ ಎರಡು ಪ್ರಪಂಚದಲ್ಲಿಯೂ ಒಂದೇ ಉಪಯೋಗವುಳ್ಳದ್ದಾಗಲಾರದು. ಮನುಷ್ಯನ ಸಂಗ್ರಾಮ ಆಗುವುದು ಮನಸ್ಸಿನಲ್ಲಿ. ಮನಸ್ಸನ್ನು ಯಾರು ಎಷ್ಟೆಷ್ಟು ಸಂಯಮ ಮಾಡಿಕೊಂಡಿರುವರೋ ಅವರು ಅಷ್ಟಷ್ಟು ದೊಡ್ಡವರಾಗಿದ್ದಾರೆ. ಮನಸ್ಸಿನ ವೃತ್ತಿಗಳೆಲ್ಲವೂ ಪೂರ್ತಿಯಾಗಿ ವಿಶ್ರಾಂತಿಯಲ್ಲಿದ್ದರೆ ಆತ್ಮದ ವಿಕಾಸವುಂಟಾಗುವುದು. ಮಾನವೇತರ ಪ್ರಾಣಿ ಜಗತ್ತಿನಲ್ಲಿ ಸ್ಥೂಲ ದೇಹವನ್ನು ಕಾಪಾಡಿಕೊಳ್ಳುವುದಕ್ಕಾಗಿ ಸಂಗ್ರಾಮ ಕಂಡುಬರುತ್ತದೆ. ಮಾನವ ಜಗತ್ತಿನಲ್ಲಿ ಮನಸ್ಸಿನ ಮೇಲೆ ಆಧಿಪತ್ಯವನ್ನು ಪಡೆಯುವುದಕ್ಕೋಸ್ಕರ ಅಥವಾ ಸಾತ್ತ್ವಿಕ ವೃತ್ತಿಯನ್ನು ಪಡೆಯುವುದಕ್ಕೋಸ್ಕರ ಸಂಗ್ರಾಮ ನಡೆಯುವುದು. ಜೀವಂತ ವೃಕ್ಷ ಮತ್ತು ಕೊಳದ ನೀರಿನಲ್ಲಿ ಬಿದ್ದ ವೃಕ್ಷಚ್ಛಾಯೆ ಇವುಗಳ ಹಾಗೆ. ಮನುಷ್ಯೇತರ ಪ್ರಾಣಿ ಜಗತ್ತು ಮತ್ತು ಮನುಷ್ಯ ಜಗತ್ತು ಇವುಗಳಲ್ಲಿ ಸಂಗ್ರಾಮವು ವ್ಯತ್ಯಸ್ತವಾಗಿರುತ್ತದೆ.

ಶಿಷ್ಯ: ಹಾಗಾದರೆ ತಾವು ನಮ್ಮ ಶಾರೀರಿಕ ಉನ್ನತಿಗೋಸ್ಕರ ಸಾಧನೆ ಮಾಡಬೇಕೆಂದು ಅಷ್ಟೊಂದೆಲ್ಲಾ ಹೇಳಿದ್ದೇಕೆ?

ಸ್ವಾಮೀಜಿ: ನೀವೇನು ಮನುಷ್ಯರೇ? ಎಲ್ಲೋ ಸ್ವಲ್ಪ ಬುದ್ಧಿ ಇದೆ ಅಷ್ಟೆ; ದೇಹ ಚನ್ನಾಗಿಲ್ಲದಿದ್ದರೆ ಮನಸ್ಸಿನೊಡನೆ ಸಂಗ್ರಾಮ ಮಾಡುವುದು ಹೇಗೆ? ನೀವೇನು ಜಗತ್ತಿನ ಪೂರ್ಣ ವಿಕಾಸ ಎನ್ನಿಸಿಕೊಳ್ಳುವ ಮನುಷ್ಯರೆಂದುಕೊಂಡಿದ್ದೀರಾ? ಆಹಾರ ನಿದ್ರೆ ಹೆಂಗಸು ಇವುಗಳನ್ನು ಬಿಟ್ಟು ಮತ್ತೇನಿದೆ ನಿಮಗೆ? ಇನ್ನು ನಾಲ್ಕು ಕಾಲಿನಲ್ಲಿ ನಡೆಯುತ್ತಿಲ್ಲವಲ್ಲಾ ಅದೇ ಒಂದು ಪುಣ್ಯ. ಪರಮಹಂಸರು ‘ಯಾರಿಗೆ ಮಾನದ ಮೇಲೆ ಜ್ಞಾನವಿರುತ್ತದೆಯೋ ಅವನೇ ಮಾನುಷ’ ಎಂದು ಹೇಳುತ್ತಿದ್ದರು; ನೀವೊ ‘ಜಾಯಸ್ವ ಮ್ರಿಯಸ್ವ’ ಎಂಬ ವಾಕ್ಯಗಳಿಗೆ ಸಾಕ್ಷೀಭೂತರಾಗಿ ಸ್ವದೇಶ ವಾಸಿಗಳಿಗೆ ಹಿಂಸಾಜನಕರೂ ವಿದೇಶೀ ಜನಗಳ ಅಗೌರವಕ್ಕೆ ಪಾತ್ರರೂ ಆಗಿದ್ದೀರಿ. ನೀವು ಮಾನವ ಮತ್ತು ಮಾನವೇತರ ಪ್ರಾಣಿಗಳ ಮಧ್ಯೆ ಇದ್ದೀರಿ. ಅದಕ್ಕೋಸ್ಕರವೇ ಸಂಗ್ರಾಮ ಮಾಡಿರೆಂದೂ, ಸಿದ್ಧಾಂತಗಿದ್ದಾಂತ ಎಲ್ಲಾ ಕಟ್ಟಿ ಇಡಿ ಎಂದೂ ಹೇಳುವುದು. ನಿಮ್ಮ ದೈನಂದಿನ ಕೆಲಸವನ್ನೂ ನಡವಳಿಕೆಯನ್ನೂ ಸ್ಥಿರಬುದ್ಧಿಯಿಂದ ವಿವೇಚನೆ ಮಾಡಿನೋಡಿ. ನೀವು ಮಾನವ ಮತ್ತು ಮಾನವೇತರ ಸ್ಥಾನಗಳ ಮಧ್ಯದಲ್ಲಿರುವ ಜೀವ ವಿಶೇಷಗಳು ಹೌದೋ ಅಲ್ಲವೋ ಎಂಬುದು ತಿಳಿಯುತ್ತದೆ. ದೇಹವನ್ನು ಮೊದಲು ದೃಢಪಡಿಸಿ. ಆಮೇಲೆ ಮನಸ್ಸಿನ ಮೇಲೆ ಕ್ರಮೇಣ ಆಧಿಪತ್ಯ ಬರುತ್ತದೆ. - ‘ನಾಯಮಾತ್ಮಾ ಬಲಹೀನೇನ ಲಭ್ಯಃ’; ತಿಳಿಯಿತೆ?

ಶಿಷ್ಯ: ಮಹಾಶಯರೆ, ‘ಬಲಹೀನೇನ’ ಎಂಬುದಕ್ಕೆ ಭಾಷ್ಯಕಾರರು ಮಾತ್ರ ‘ಬ್ರಹ್ಮ ಚರ್ಯ ಹೀನೇನ’ ಎಂದು ಅರ್ಥಮಾಡಿದ್ದಾರೆ.

ಸ್ವಾಮಿಜಿ: ಮಾಡಿಕೊಳ್ಳಲಿ; ನಾನು ದುರ್ಬಲವಾದ ಶರೀರವುಳ್ಳವರಿಗೆ ಆತ್ಮಸಾಕ್ಷಾತ್ಕಾರವಾಗುವುದಿಲ್ಲ ಎಂದು ಹೇಳುತ್ತೇನೆ.

ಶಿಷ್ಯ: ಆದರೆ ಬಲವಿರುವ ಶರೀರದಲ್ಲಿ ಅನೇಕ ಸಾರಿ ಮಂದಬುದ್ಧಿಯೂ ಕಂಡುಬರುತ್ತದೆ.

ಸ್ವಾಮೀಜಿ: ಅಂಥವರಿಗೆ ಕಷ್ಟಪಟ್ಟು ನೀನೇನಾದರೂ ಒಳ್ಳೆಯ ಭಾವವನ್ನು ಒಂದು ಸಾರಿ ತಿಳಿಸಿಕೊಟ್ಟೆಯೆಂದರೆ ಅದನ್ನು ಅವರು ಎಷ್ಟು ಶೀಘ್ರವಾಗಿ ಕಾರ್ಯರೂಪಕ್ಕೆ ಇಳಿಸಬಲ್ಲರೋ ಅಷ್ಟು ಶೀಘ್ರವಾಗಿ, ಹೀನವೀರ್ಯರಾದವರು ಮಾಡಲಾರರು. ನೋಡಿಲ್ಲವೆ, ಕ್ಷೀಣವಾದ ದೇಹದಲ್ಲಿ ಕಾಮಕ್ರೋಧಗಳ ವೇಗವನ್ನು ತಡೆಯುವುದಕ್ಕಾಗುವುದಿಲ್ಲ. ಒಣಕಲಾಗಿರುವ ಜನರು ಬಹು ಬೇಗ ರೇಗಿಬಿಡುತ್ತಾರೆ - ಬಹು ಬೇಗ ಕಾಮಮೋಹಿತರಾಗಿಬಿಡುತ್ತಾರೆ.

ಶಿಷ್ಯ: ಆದರೆ ಈ ನಿಯಮದ ವ್ಯತಿಕ್ರಮವೂ ಕಂಡುಬರುತ್ತದೆ.

ಸ್ವಾಮೀಜಿ: ಅದು ಇಲ್ಲವೆಂದು ಯಾರು ಹೇಳುತ್ತಾರೆ? ಮನಸ್ಸಿನ ಮೇಲೆ ಒಂದು ಸಾರಿ ಆಧಿಪತ್ಯ ಬಂದುಬಿಟ್ಟಿತು ಅಂದರೆ ಆಮೇಲೆ ದೇಹ ಬಲವಾಗಿರಲಿ ಒಣಗಿಕೊಂಡೆ ಇರಲಿ, ಅದರಿಂದ ಏನೂ ಬರುವುದೂ ಇಲ್ಲ ಹೋಗುವುದೂ ಇಲ್ಲ. ಒಟ್ಟು ಮಾತೇನೆಂದರೆ, ದೇಹವು ಚೆನ್ನಾಗಿಲ್ಲದಿದ್ದರೆ ಅಂಥವನು ಆತ್ಮಜ್ಞಾನಕ್ಕೆ ಅಧಿಕಾರಿಯೇ ಆಗುವುದಿಲ್ಲ. ಪರಮಹಂಸರು ‘ಶರೀರದಲ್ಲಿ ಸತ್ತ್ವವು ಇಲ್ಲದೆ ಇದ್ದರೆ ಜೀವನು ಸಿದ್ಧನಾಗಲಾರನು’ ಎಂದು ಹೇಳುತ್ತಿದ್ದರು.

ಈ ಮಾತುಗಳನ್ನು ಹೇಳುತ್ತ ಹೇಳುತ್ತ ಸ್ವಾಮಿಜಿ ಉತ್ತೇಜಿತರಾದದ್ದನ್ನು ನೋಡಿ, ಶಿಷ್ಯನು ಧೈರ್ಯವಾಗಿ ಮತ್ತಾವ ಪ್ರಶ್ನೆಯನ್ನೂ ಕೇಳಲಾರದೆ ಹೋದನು. ಸ್ವಾಮೀಜಿಯ ಸಿದ್ಧಾಂತವನ್ನೊಪ್ಪಿಕೊಂಡು ಸುಮ್ಮನಾದನು. ಸ್ವಲ್ಪ ಹೊತ್ತಿನ ಮೇಲೆ ಸ್ವಾಮೀಜಿ ಅಲ್ಲಿದ್ದವರಿಗೆಲ್ಲಾ ಗುಟ್ಟಾಗಿ ಹೇಳಿದ್ದೇನೆಂದರೆ - “ಮತ್ತೊಂದು ಸಂಗತಿಯನ್ನು ಕೇಳಿದ್ದಿರಾ? ಇವೊತ್ತು ಈ ಭಟ್ಟಾಚಾರ್ಯ ಬ್ರಾಹ್ಮಣನು ನಿವೇದಿತೆಯ ಎಂಜಲನ್ನು ತಿಂದು ಬಂದಿದ್ದಾನೆ. ಆಕೆ ಮುಟ್ಟಿದ ಅನ್ನವನ್ನು ತಿನ್ನಲಿಲ್ಲವೆಂದು ತೋರುತ್ತದೆ. ಅದರಿಂದ ಆಗುವುದೇನು ಹೋಗುವುದೇನು - ಆದರೆ ಆಕೆ ಮುಟ್ಟಿದ ನೀರನ್ನು ಕುಡಿದದ್ದು ಹೇಗೆ?"

ಶಿಷ್ಯ: ಅದು ತಮ್ಮ ಆಜ್ಞೆಯಿಂದಲೇ ಮಾಡಿದ್ದು, ಗುರುವಿನ ಆಜ್ಞಾನುಸಾರವಾಗಿ ನಾನು ಏನು ಬೇಕಾದರೂ ಮಾಡುತ್ತೇನೆ. ನೀರು ಕುಡಿಯುವುದಕ್ಕೆ ನನಗೇನು ಇಷ್ಟವಿರಲಿಲ್ಲ; ಆದರೆ ತಾವು ಕುಡಿದುಕೊಟ್ಟಿದ್ದರಿಂದ ಪ್ರಸಾದವೆಂದು ಕುಡಿದೆ.

ಸ್ವಾಮೀಜಿ: ನಿನ್ನ ಜಾತಿ ಹಾಳಾಗಿ ಹೋಯಿತು; ಇನ್ನು ನೀನು ಭಟ್ಟಾಚಾರ್ಯ ಬ್ರಾಹ್ಮಣನೆಂದರೆ ಯಾರೂ ಒಪ್ಪುವುದಿಲ್ಲ.

ಶಿಷ್ಯ: ಒಪ್ಪದಿದ್ದರೆ ಬಿಡಲಿ; ನಾನು ತಮ್ಮ ಆಜ್ಞೆಯನ್ನು ಅನುಸರಿಸಿ ಚಂಡಾಲನ ಮನೆಯ ಅನ್ನವನ್ನೂ ತಿನ್ನಬಲ್ಲೆ.

ಇದನ್ನು ಕೇಳಿ ಸ್ವಾಮಿಗಳೂ ಮತ್ತು ಅಲ್ಲಿದ್ದವರೂ ಗಟ್ಟಿಯಾಗಿ ನಗಲಾರಂಭಿಸಿದರು.

ಈ ಮಾತುಕತೆಗಳಲ್ಲಿ ರಾತ್ರಿ ಸುಮಾರು ಹನ್ನೆರಡೂವರೆ ಗಂಟೆಯಾಗಿ ಬಿಟ್ಟಿತು. ಶಿಷ್ಯನು ತಾನು ವಾಸವಾಗಿದ್ದ ಮನೆಗೆ ಹಿಂತಿರುಗಿ ಬಂದು ನೋಡಲು ಬಾಗಿಲು ಹಾಕಿಬಿಟ್ಟಿತ್ತು; ಎಷ್ಟು ಕೂಗಿದರೂ ಯಾರೂ ಏಳದಿರಲು ವಿಧಿಯಿಲ್ಲದೆ ಮನೆಯ ಜಗಲಿಯ ಮೇಲೆ ಮಲಗಿಕೊಂಡು ರಾತ್ರಿಯನ್ನು ಕಳೆಯಬೇಕಾಗಿ ಬಂತು.

ಕಾಲಚಕ್ರದ ಕಠೋರ ಪರಿವರ್ತನದಿಂದ, ಸ್ವಾಮೀಜಿ, ಯೋಗಾನಂದರೂ, ಸಿಸ್ಟರ್ ನಿವೇದಿತೆಯೂ ಈಗ ನರಶರೀರದಲ್ಲಿಲ್ಲ! ಅವರ ಜೀವನದ ಪವಿತ್ರ ಸ್ಮೃತಿಯೊಂದು ಮಾತ್ರ ಉಳಿದುಕೊಂಡಿದೆ - ಅವರ ಮಾತುಕಥೆಗಳನ್ನು ಅಲ್ಪಸ್ವಲ್ಪ ಬರೆಯಲು ಸಮರ್ಥನಾದ್ದರಿಂದ ಶಿಷ್ಯನು ತಾನು ಧನ್ಯನಾದೆನೆಂದು ಭಾವಿಸುವನು.

\newpage

\chapter[ಅಧ್ಯಾಯ ೨೧]{ಅಧ್ಯಾಯ ೨೧\protect\footnote{\engfoot{Complete Works of Swami Vivekananda, Volume VI, Page 445}}}

\begin{center}
ಸ್ಥಳ: ಬೇಲೂರು ಮಠ (ಬಾಡಿಗೆ ಕಟ್ಟಡ), ವರ್ಷ: ಕ್ರಿ.ಶ. ೧೮೯೮.
\end{center}

ಇಂದು ಸಮಾರು ಎರಡು ಗಂಟೆಯ ಹೊತ್ತಿಗೆ ಶಿಷ್ಯನು ಮಠಕ್ಕೆ ನಡೆದುಕೊಂಡು ಬಂದಿದ್ದಾನೆ. ನೀಲಾಂಬರ ಬಾಬುಗಳ ತೋಟದ ಮನೆಗೆ ಈಗ ಮಠವನ್ನು ವರ್ಗಾಯಿಸಿದೆ. ಈಗಿನ ಮಠದ ನಿವೇಶನವನ್ನು ಕೊಂಡು ಕೆಲವು ದಿನಗಳಾಗಿವೆ. ಸ್ವಾಮೀಜಿ ಸುಮಾರು ನಾಲ್ಕು ಗಂಟೆಯ ಹೊತ್ತಿಗೆ ಶಿಷ್ಯನನ್ನು ಜೊತೆಯಲ್ಲಿ ಕರೆದುಕೊಂಡು ಮಠದ ನಿವೇಶನದಲ್ಲಿ ತಿರುಗಾಡಿಕೊಂಡು ಬರುವುದಕ್ಕೆಂದು ಹೊರಗೆ ಹೊರಟಿದ್ದಾರೆ. ಮಠದ ನಿವೇಶನದಲ್ಲಿ ಆಗ ಕಾಡು ಬೆಳೆದುಕೊಂಡಿತ್ತು. ಅದರ ಉತ್ತರ ಭಾಗದಲ್ಲಿ ಆಗ ಒಂದು ಅಂತಸ್ತಿನ ಗಾರೆಯ ಕಟ್ಟಡವೊಂದಿತ್ತು. ಅದನ್ನು ಉತ್ತಮಪಡಿಸಿಕೊಂಡೇ ಈಗಿನ ಮಠದ ಕಟ್ಟಡವು ನಿರ್ಮಿತವಾಗಿದೆ. ಮಠದ ನಿವೇಶನವನ್ನು ಖರೀದಿಗೆ ಕೊಟ್ಟವನೂ ಸ್ವಾಮೀಜಿ ಜೊತೆಯಲ್ಲಿ ಸ್ವಲ್ಪ ದೂರ ಬಂದು ಹಿಂತಿರುಗಿದನು. ಸ್ವಾಮೀಜಿ ಶಿಷ್ಯನೊಡನೆ ನಿವೇಶನದಲ್ಲಿ ತಿರುಗಾಡುವುದಕ್ಕೆ ಮೊದಲುಮಾಡಿದರು ಮತ್ತು ಮಾತಿಗೆ ಮಾತು ಬಂದು ಭವಿಷ್ಯದಲ್ಲಿ ಮಠದ ಉಪಯುಕ್ತತೆ ಅದರ ವಿಧಿ ವಿಧಾನಗಳು ಇವುಗಳ ವಿಚಾರವನ್ನು ತೆಗೆದುಕೊಂಡರು.

ಕ್ರಮೇಣ ಒಂದಂತಸ್ತಿನ ಮನೆಯ ಪೂರ್ವದ ಕಡೆಯ ಅಂಗಳಕ್ಕೆ ಹೋಗಿ ಅಲ್ಲಿ ತಿರುಗಾಡುತ್ತ ಸ್ವಾಮಿಗಳು ಹೇಳಿದ್ದೇನೆಂದರೆ: ಅಲ್ಲಿ ಸಾಧುಗಳು ಇರುವುದಕ್ಕೆ ಸ್ಥಳವಾಗುವುದು; ಸಾಧನೆ, ಭಜನೆ, ಜ್ಞಾನ, ವಿಚಾರ ಇವುಗಳಿಗೆ ಈ ಮಠ ಪ್ರಧಾನ ಕೇಂದ್ರಸ್ಥಾನವಾಗುವುದು - ಇದೇ ನನ್ನ ಅಭಿಪ್ರಾಯ. ಇಲ್ಲಿ ಉದಯಿಸುವ ಶಕ್ತಿಯಿಂದ ಜಗತ್ತು ತಲ್ಲಣಿಸಿ ಹೋಗುತ್ತದೆ; ಮನುಷ್ಯನ ಜೀವನ ಗತಿ ಬದಲಾಗಿಬಿಡುತ್ತದೆ. ಜ್ಞಾನ, ಭಕ್ತಿ, ಯೋಗ, ಕರ್ಮ ಇವು ಒಂದು ಕಡೆ ಸೇರಿ ಇಲ್ಲಿಂದ ಮಾನವನಿಗೆ ಕಲ್ಯಾಣಪ್ರದವಾದ ಆದರ್ಶ ಹೊರಡುತ್ತವೆ; ಈ ಮಠಕ್ಕೆ ಸೇರಿದ ಜನಗಳ ಇಚ್ಛೆಯಿಂದ ಕಾಲಕ್ರಮದಲ್ಲಿ ದಿಕ್ಕು ದಿಕ್ಕುಗಳಲ್ಲೆಲ್ಲಾ ಪ್ರಾಣವು ಸಂಚರಿಸುವುದು; ಯಥಾರ್ಥವಾದ ಧರ್ಮಶೀಲರೆಲ್ಲಾ ಇಲ್ಲಿ ಬಂದು ಸೇರುವರು - ಮನಸ್ಸಿಗೆ ಇಂಥ ಎಷ್ಟೆಷ್ಟೋ ಯೋಚನೆಗಳು ತೋರುತ್ತವೆ!

“ಮಠದ ದಕ್ಷಿಣ ಭಾಗದಲ್ಲಿ ಸ್ಥಳವಿದೆಯಲ್ಲಾ ಅದು ವಿದ್ಯೆಯ ಕೇಂದ್ರ ಸ್ಥಾನವಾಗುವುದು. ವ್ಯಾಕರಣ, ದರ್ಶನ, ವಿಜ್ಞಾನ, ಕಾವ್ಯ, ಅಲಂಕಾರ, ಸ್ಮೃತಿ, ಭಕ್ತಿ ಶಾಸ್ತ್ರ ಮತ್ತು ರಾಜಕೀಯ ಮುಂತಾದುವು ಇಲ್ಲಿ ಬೋಧಿಸಲ್ಪಡುವುವು. ಹಿಂದಿನ ಹಳೆಯ ಮಠಗಳ ಮಾದರಿಯಲ್ಲಿ ಈ ವಿದ್ಯಾಮಂದಿರವು ಸ್ಥಾಪಿಸಲ್ಪಡುವುದು. ಬಾಲ ಬ್ರಹ್ಮಚಾರಿಗಳು ಇಲ್ಲಿ ವಾಸಮಾಡಿಕೊಂಡು ಶಾಸ್ತ್ರವನ್ನು ಓದುವರು. ಅವರ ಅನ್ನ ಬಟ್ಟೆಗಳೆಲ್ಲವೂ ಮಠದಿಂದ ದೊರೆಯುವುವು. ಈ ಬ್ರಹ್ಮಚಾರಿಗಳೆಲ್ಲಾ ಐದು ವರ್ಷ ಶಿಕ್ಷಣ ಪಡೆದ ಮೇಲೆ, ಇಷ್ಟವಿದ್ದರೆ ಮನೆಗೆ ಹೋಗಿ ಸಂಸಾರಿಗಳಾಗಬಹುದು. ಹಾಗಲ್ಲದೆ ಸಂನ್ಯಾಸದ ಮೇಲೆ ಇಷ್ಟವಿದ್ದು, ಮಠಾಧಿಪತಿಗಳು ಒಪ್ಪಿದರೆ, ಸಂನ್ಯಾಸವನ್ನು ತೆಗೆದುಕೊಳ್ಳಬಹುದು. ಈ ಬ್ರಹ್ಮಚಾರಿಗಳಲ್ಲಿ ಯಾರಲ್ಲಿಯಾದರೂ ಉಚ್ಛೃಂಖಲತೆಯಾಗಲಿ ದುಶ್ಚರಿತ್ರವಾಗಲಿ ಕಂಡುಬಂದರೆ ಅಂಥವರನ್ನು ಮಠದ ಸ್ವಾಮಿಗಳು ಆಗಲೇ ಹೊರಗೆ ಹೊರಡಿಸಿಬಿಡುವರು. ಇಲ್ಲಿ ಜಾತಿ ವರ್ಣಗಳ ಭೇದವಿಲ್ಲದೆ ವ್ಯಾಸಂಗ ನಡೆಯುವುದು. ಇದಕ್ಕೆ ಯಾರ ಆಕ್ಷೇಪಣೆ ಇದೆಯೋ ಅವರನ್ನು ತೆಗೆದುಕೊಳ್ಳುವುದಿಲ್ಲ. ಆದರೆ ಯಾರು ತಮ್ಮ ಜಾತಿ ವರ್ಣಾಶ್ರಮಾಚಾರಗಳನ್ನು ನಡೆಸಿಕೊಂಡು ಹೋಗಲು ಇಷ್ಟಪಡುವರೊ ಅವರು ತಮ್ಮ ಊಟ ಉಪಚಾರಾದಿಗಳ ಏರ್ಪಾಡನ್ನು ತಾವೇ ವಹಿಸಿಕೊಳ್ಳಬೇಕು. ಅವರು ವ್ಯಾಸಂಗವನ್ನು ಮಾತ್ರ ಎಲ್ಲರ ಜೊತೆಯಲ್ಲಿಯೂ ಒಂದೇ ಕಡೆ ಮಾಡುವರು. ಅವರ ಮೇಲೂ ಚಾರಿತ್ರ್ಯ ವಿಚಾರಗಳಲ್ಲಿ ಸ್ವಾಮಿಗಳು ತೀಕ್ಷ್ಣ ದೃಷ್ಟಿಯನ್ನು ಇಟ್ಟೇ ಇರುವರು. ಇಲ್ಲಿ ಶಿಕ್ಷಿತರಾಗದೇ ಹೋದರೆ ಯಾರೂ ಸಂನ್ಯಾಸಕ್ಕೆ ಯೋಗ್ಯರಾಗುವುದಿಲ್ಲ. ಕ್ರಮೇಣ ಹೀಗೆ ಈ ಮಠದಲ್ಲಿ ಕಾರ್ಯ ನಡೆಯುವಾಗ ಹೇಗಿರಬಹುದು!"

ಶಿಷ್ಯ: ಹಾಗಾದರೆ, ತಾವು ಹಿಂದಿನ ಕಾಲದಲ್ಲಿದ್ದಂತೆ ಗುರು ಗೃಹದಲ್ಲಿ ಬ್ರಹ್ಮಚರ್ಯಾಶ್ರಮದ ಆಚರಣೆಯನ್ನು ಪುನಃ ದೇಶದಲ್ಲಿ ಪ್ರಚಾರಕ್ಕೆ ತರಬೇಕೆಂದಿದ್ದೀರಿ?

ಸ್ವಾಮಾಜಿ: ಅಲ್ಲದೆ ಮತ್ತೇನು? ಈಗಿನ ವಿದ್ಯಾಭ್ಯಾಸ ಪದ್ಧತಿಯಲ್ಲಿ ಬ್ರಹ್ಮವಿದ್ಯಾ ವಿಕಾಸಕ್ಕೆ ಅನುಕೂಲ ಸ್ವಲ್ಪವೂ ಇಲ್ಲ. ಹಿಂದಿನಂತೆ ಬ್ರಹ್ಮಚರ್ಯಾಶ್ರಮವನ್ನು ಸ್ಥಾಪಿಸಬೇಕಾಗಿದೆ. ಆದರೆ ಈಗ ಉದಾರಭಾವಗಳ ಮೇಲೆ ಅದರ ತಳಹದಿಯನ್ನು ಹಾಕಬೇಕು. ಎಂದರೆ ಈ ಕಾಲಕ್ಕೆ ಒಗ್ಗುವಂತಹ ಅನೇಕ ಬದಲಾವಣೆಗಳನ್ನು ಅದರಲ್ಲಿ ಸೇರಿಸಬೇಕು. ಅದನ್ನೆಲ್ಲಾ ಆಮೇಲೆ ಹೇಳುತ್ತೇನೆ.

ಸ್ವಾಮೀಜಿ ಪುನಃ ಹೇಳತೊಡಗಿದರು: ಮಠದ ದಕ್ಷಿಣಕ್ಕೆ ಇರುವ ಈ ಸ್ಥಳವನ್ನೂ ಕಾಲಕ್ರಮದಲ್ಲಿ ಕೊಂಡುಕೊಳ್ಳಬೇಕು. ಇಲ್ಲಿ ಮಠದ ಅನ್ನಸತ್ರವಾಗುವುದು. ಅಲ್ಲಿ ನಿಜವಾದ ಬಡಬಗ್ಗರಿಗೆ ನಾರಾಯಣ ಬುದ್ಧಿಯಿಂದ ಸೇವೆ ಮಾಡುವುದಕ್ಕೆ ಏರ್ಪಾಡು ಇರುವುದು. ಈ ಅನ್ನಸತ್ರವು ಪರಮಹಂಸರ ಹೆಸರಿನಲ್ಲಿ ಸ್ಥಾಪಿಸಲ್ಪಡುವುದು. ಸಂಗ್ರಹವಾದ ಹಣಕ್ಕೆ ತಕ್ಕಂತೆ ಮೊದಲು ಅನ್ನಸತ್ರ ಆರಂಭಿಸಲ್ಪಡುವುದು. ಮೊದಲು ಇಬ್ಬರು ಮೂವರನ್ನು ಇಟ್ಟುಕೊಂಡು ಪ್ರಾರಂಭ ಮಾಡಿದರೆ ಸಾಕು. ಉತ್ಸಾಹಿಗಳಾದ ಬ್ರಹ್ಮಚಾರಿಗಳಿಗೆ ಈ ಅನ್ನಸತ್ರ ನಡೆಸುವ ಬಗ್ಗೆ ಶಿಕ್ಷಣ ಕೊಡಬೇಕು. ಅವರು ಅಲ್ಲಿ ಇಲ್ಲಿ ಶೇಖರಿಸುವುದರಿಂದ, ಭಿಕ್ಷೆ ಮಾಡುವುದರಿಂದ - ಈ ಅನ್ನಸತ್ರ ನಡೆಯುವುದು. ಮಠ ಈ ವಿಷಯಕ್ಕೆ ಯಾವ ವಿಧವಾದ ದ್ರವ್ಯಸಹಾಯವನ್ನೂ ಮಾಡುವುದಕ್ಕಾಗುವುದಿಲ್ಲ. ಬ್ರಹ್ಮಚಾರಿಗಳೇ ಅದಕ್ಕೋಸ್ಕರ ದುಡ್ಡನ್ನು ಸೇರಿಸಿಕೊಂಡು ಬರಬೇಕು. ಸೇವಾಸತ್ರದಲ್ಲಿ ಹೀಗೆ ಐದು ವರ್ಷವೂ ವಿದ್ಯಾಶ್ರಮದಲ್ಲಿ ಐದು ವರ್ಷವೂ ಒಟ್ಟು ಹತ್ತು ವರ್ಷ ಶಿಕ್ಷಣವಾದ ಮೇಲೆ ಮಠದ ಸ್ವಾಮಿಗಳಿಂದ ದೀಕ್ಷೆ ಪಡೆದು ಸಂನ್ಯಾಸಾಶ್ರಮವನ್ನು ಪಡೆಯುವುದಕ್ಕೆ ತಕ್ಕವರಾಗುವರು. ಅವರಿಗೂ ಸಂನ್ಯಾಸಿಗಳಾಗುವುದಕ್ಕೆ ಇಷ್ಟವಿದ್ದು, ಇವರು ತಕ್ಕ ಅಧಿಕಾರಿಗಳು, ಇವರಿಂದ ಉಪಯೋಗವಾಗುವುದು ಎಂದು ತಿಳಿದು ಮಠಾಧ್ಯಕ್ಷರು ಸಂನ್ಯಾಸ ಕೊಡುವುದಕ್ಕೆ ಇಷ್ಟಪಟ್ಟರೆ ಆಗಬಹುದು. ಆದರೆ ಮಠಾಧ್ಯಕ್ಷರು ಕೆಲಕೆಲವು ವಿಶೇಷ ಸದ್ಗುಣಸಂಪನ್ನರಾದ ಬ್ರಹ್ಮಚಾರಿಗಳ ವಿಷಯದಲ್ಲಿ ಈ ನಿಯಮವನ್ನು ಮೀರಿ ಇಷ್ಟ ಬಂದಾಗ ಅವರಿಗೆ ಸಂನ್ಯಾಸ ಕೊಡಬಹುದು. ಸಾಧಾರಣ ಬ್ರಹ್ಮಚಾರಿಗಳಿಗೆ ಮಾತ್ರ ಹಿಂದಿನ ರೀತಿಯಲ್ಲಿ ಕ್ರಮಕ್ರಮವಾಗಿ ಸಂನ್ಯಾಸಾಶ್ರಮದಲ್ಲಿ ಪ್ರವೇಶ ಮಾಡಿಸಬೇಕು. ನನ್ನ ಮನಸ್ಸಿನಲ್ಲಿ ಈ ಭಾವನೆಗಳೆಲ್ಲಾ ಇವೆ.

ಶಿಷ್ಯ: ಮಹಾಶಯರೆ, ಮಠದಲ್ಲಿ ಹೀಗೆ ಮೂರು ಶಾಲೆಗಳನ್ನು ಇಡುವುದರ ಉದ್ದೇಶವೇನು?

ಸ್ವಾಮೀಜಿ: ಗೊತ್ತಾಗಲಿಲ್ಲವೆ? ಮೊದಲು ಅನ್ನದಾನ; ಆಮೇಲೆ ವಿದ್ಯಾದಾನ; ಎಲ್ಲಕ್ಕೂ ಮೇಲೆ ಜ್ಞಾನದಾನ. ಈ ಮೂರು ಭಾವಗಳ ಸಮನ್ವಯವನ್ನು ಈ ಮಠದಿಂದ ಮಾಡಬೇಕಾಗಿದೆ. ಅನ್ನದಾನ ಮಾಡುವುದಕ್ಕೆ ಪ್ರಯತ್ನ ಮಾಡುತ್ತ ಮಾಡುತ್ತ ಬ್ರಹ್ಮಚಾರಿಗಳ ಮನಸ್ಸಿನಲ್ಲಿ ಸ್ವಾರ್ಥವಿಲ್ಲದೆ ಕರ್ಮ ಮಾಡುವುದೂ ಈಶ್ವರ ಬುದ್ಧಿಯಿಂದ ಜೀವ ಸೇವೆ ಮಾಡುವುದೂ ದೃಢವಾಗುತ್ತವೆ. ಇದರಿಂದ ಅವರ ಚಿತ್ರ ಕ್ರಮೇಣ ನಿರ್ಮಲವಾಗಿ ಅದರಿಂದ ಸತ್ಯಭಾವ ಅಭಿವೃದ್ಧಿ ಹೊಂದುತ್ತದೆ. ಹಾಗಾದಾಗ ಬ್ರಹ್ಮಚಾರಿಗಳು ಸಕಾಲದಲ್ಲಿ ಬ್ರಹ್ಮವಿದ್ಯೆಯನ್ನು ಪಡೆಯುವುದಕ್ಕೆ ಯೋಗ್ಯತೆಯನ್ನೂ ಸಂನ್ಯಾಸಾಶ್ರಮದಲ್ಲಿ ಪ್ರವೇಶ ಮಾಡುವುದಕ್ಕೆ ಅಧಿಕಾರವನ್ನೂ ಹೊಂದುತ್ತಾರೆ.

ಶಿಷ್ಯ: ಮಹಾಶಯರೆ, ಜ್ಞಾನದಾನವೇ ಶ್ರೇಷ್ಠವಾದದ್ದಾದರೆ, ಅನ್ನದಾನ ವಿದ್ಯಾದಾನದ ಶಾಖೆಗಳನ್ನು ಇಡಬೇಕಾದ ಆವಶ್ಯಕತೆ ಏನು?

ಸ್ವಾಮೀಜಿ: ಇನ್ನಾದರೂ ನೀನು ಈ ಸಂಗತಿಯನ್ನು ತಿಳಿದುಕೊಳ್ಳಲಾರದೆ ಹೋದೆಯಲ್ಲ! ಕೇಳು - ಅನ್ನಕ್ಕೋಸ್ಕರ ಹಾಹಾಕಾರಪಡುತ್ತಿರುವ ಈ ಕಾಲದಲ್ಲಿ ನೀನು ಪರಾರ್ಥವಾಗಿ - ಸೇವೆ ಮಾಡಿಯೊ ಬಡಬಗ್ಗರಿಂದ ಭಿಕ್ಷೆಗಿಕ್ಷೆಗಳನ್ನು ಮಾಡಿಯೊ ಅಂತೂ ಹೇಗೊ - ಎರಡು ಹಿಡಿ ಅನ್ನವನ್ನು ಕೊಡಬಲ್ಲೆಯಾದರೆ, ಜೀವ ಜಗತ್ತಿಗೂ ನಿನಗೂ ಮಂಗಳ ಆಗುತ್ತದೆ. ಜೊತೆಯಲ್ಲಿ ಈ ಸತ್ಕಾರ್ಯಕ್ಕೋಸ್ಕರ ಎಲ್ಲರ ಸಹಾನುಭೂತಿಯನ್ನೂ ಪಡೆಯುವೆ. ಈ ಸತ್ಕಾರ್ಯಕ್ಕೋಸ್ಕರ ನಿನ್ನಲ್ಲಿ ವಿಶ್ವಾಸವನ್ನಿಟ್ಟು ಕಾಮಕಾಂಚನಬದ್ಧರಾದ ಸಂಸಾರಿಕರು ನಿನಗೆ ಸಹಾಯ ಮಾಡುವುದಕ್ಕೆ ಹೊರಡುತ್ತಾರೆ. ನೀನು ವಿದ್ಯಾದಾನ ಅಥವಾ ಜ್ಞಾನದಾನದಿಂದ ಎಷ್ಟು ಜನರನ್ನು ಆಕರ್ಷಿಸಬಲ್ಲೆಯೋ ಅದರ ಸಾವಿರದಷ್ಟು ಜನರು ಅಯಾಚಿತವಾಗಿ ಮಾಡಿದ ಈ ಅನ್ನದಾನದಿಂದ ಆಕರ್ಷಿಸಲ್ಪಡುವರು. ಈ ಕಾರ್ಯದಿಂದ ನೀನು ಎಷ್ಟು ಸಾಮಾನ್ಯ ಜನರ ಸಹಾನುಭೂತಿಯನ್ನು ಪಡೆಯಬಹುದೊ ಅಷ್ಟನ್ನು ಮತ್ಯಾವ ಕಾರ್ಯದಿಂದಲೂ ಪಡೆಯುವುದಕ್ಕಾಗುವುದಿಲ್ಲ. ಯಥಾರ್ಥವಾದ ಸತ್ಕಾರ್ಯಕ್ಕೆ, ಮನುಷ್ಯನೇ ಏಕೆ, ದೇವರೂ ಸಹಾಯ ಮಾಡುವನು. ಹೀಗೆ ಜನರು ಆಕರ್ಷಿತರಾದರೆ, ಅನಂತರ ಅವರಲ್ಲಿ ವಿದ್ಯಾರ್ಜನೆ ಜ್ಞಾನಾರ್ಜನೆಗಳನ್ನು ಮಾಡುವುದಕ್ಕೆ ಆಶೆಯನ್ನು ಉದ್ದೀಪನ ಗೊಳಿಸಬಹುದು. ಆದ್ದರಿಂದಲೇ ಅನ್ನದಾನ ಮೊದಲು.

ಶಿಷ್ಯ: ಅನ್ನಸತ್ರ ಮಾಡಬೇಕಾದರೆ, ಮೊದಲು ಸ್ಥಳ ಬೇಕು; ಆಮೇಲೆ ಇದಕ್ಕೋಸ್ಕರ ಒಂದು ಮನೆಯಾಗಬೇಕು; ಆಮೇಲೆ ಕೆಲಸ ನಡೆಸುವುದಕ್ಕೆ ಹಣ ಬೇಕು; ಇಷ್ಟು ಹಣ ಎಲ್ಲಿಂದ ಬರಬೇಕು?

ಸ್ವಾಮೀಜಿ: ಮಠದ ದಕ್ಷಿಣ ದಿಕ್ಕನ್ನು ನಾನು ಈಗ ಬಿಟ್ಟುಕೊಡುತ್ತೇನೆ ಮತ್ತು ಒಂದು ಸೋಗೆಯ ಗುಡಿಸಲನ್ನು ಹಾಕಿಸಿಕೊಡುತ್ತೇನೆ. ನೀನು ಒಬ್ಬಿಬ್ಬರು ಕುರುಡ ಕುಂಟರನ್ನು ಹುಡುಕಿಕೊಂಡು ಬಂದು ನಾಳಿನಿಂದಲೇ ಅವರ ಸೇವೆ ಮಾಡಿಕೊಂಡು ಬಾ ಹೋಗು ನೋಡೋಣ. ಸ್ವಂತವಾಗಿ ಹೋಗಿ ಅವರಿಗೋಸ್ಕರ ಭಿಕ್ಷೆ ಮಾಡಿಕೊಂಡು ಬರಬೇಕು. ಸ್ವಂತವಾಗಿ ಅಡಿಗೆ ಮಾಡಿ ಅವರಿಗೆ ಬಡಿಸಬೇಕು. ಹೀಗೆ ಕೆಲವು ದಿನ ಮಾಡಿದರೆ ಸಾಕು ನೋಡು - ನಿನ್ನ ಈ ಕೆಲಸದಲ್ಲಿ ಎಷ್ಟು ಜನ ಸಹಾಯ ಮಾಡುವುದಕ್ಕೆ ಮುಂದಾಗಿ ಬರುತ್ತಾರೆ, ಎಷ್ಟು ದುಡ್ಡು ಕಾಸನ್ನು ಕೊಡುತ್ತಾರೆ ಅಂತ. ‘ನಹಿ ಕಲ್ಯಾಣಕೃತ್ ಕಶ್ಚಿತ್ ದುರ್ಗತಿಂ ತಾತ ಗಚ್ಛತಿ’. ಮಗೂ, ಒಳ್ಳೆಯದನ್ನು ಮಾಡುವವನಿಗೆ ಎಂದೂ ದುರ್ಗತಿ ಒದಗುವುದಿಲ್ಲ.

ಶಿಷ್ಯ: ಅದೇನೊ ನಿಜ; ಆದರೆ ಹೀಗೆ ಯಾವಾಗಲೂ ಕರ್ಮ ಮಾಡುತ್ತ ಮಾಡುತ್ತ ಕ್ರಮೇಣ ಕರ್ಮ ಬಂಧನವುಂಟಾಗುತ್ತದೆ.

ಸ್ವಾಮೀಜಿ: ಕರ್ಮದ ಫಲದಲ್ಲಿ ನಿನಗೆ ದೃಷ್ಟಿಯಿಲ್ಲದಿದ್ದರೆ, ಸಕಲ ವಿಧವಾದ ಆಶೆ ಅಭಿಲಾಷೆಗಳನ್ನೂ ದಾಟಿ ಹೋಗಿ ನಿಜವಾದ ಅನುರಾಗವನ್ನು ಪಡೆದಿದ್ದರೆ, ಈ ಸತ್ಕಾರ್ಯಗಳೆಲ್ಲಾ ನಿನ್ನ ಕರ್ಮಬಂಧನವನ್ನು ಬಿಡಿಸುವುದಕ್ಕೆ ಸಹಾಯ ಮಾಡುತ್ತವೆ. ಇಂಥ ಕರ್ಮದಿಂದ ಬಂಧನವುಂಟಾಗುತ್ತದೆಂದು ಹೇಗೆ ಹೇಳುತ್ತೀಯೆ? ಇಂಥ ಪರಾರ್ಥ ಕರ್ಮವೊಂದೇ ಕರ್ಮಬಂಧನದ ಮೂಲೋತ್ಪಾಟನೆ ಮಾಡುವುದಕ್ಕೆ ಇರುವ ಮಾರ್ಗ. 'ನಾನ್ಯಃ ಪಂಥಾ ವಿದ್ಯತೇsಯನಾಯ' - ಬೇರೆ ದಾರಿಯೇ ಇಲ್ಲ.

ಶಿಷ್ಯ: ತಮ್ಮ ಮಾತಿನಿಂದ ಅನ್ನ ಸತ್ರ, ಸೇವಾಶ್ರಮ ಸಂಬಂಧವಾಗಿ ತಮ್ಮ ಮನೋಭಾವವನ್ನು ಹೆಚ್ಚಾಗಿ ತಿಳಿದುಕೊಳ್ಳಬೇಕೆಂದು ನನಗೆ ಬಹು ಕುತೂಹಲವುಂಟಾಗಿದೆ.

ಸ್ವಾಮಿಜಿ: ಬಡಬಗ್ಗರಿಗೋಸ್ಕರ ಚೆನ್ನಾಗಿ ಗಾಳಿಯಾಡುವ ಸಣ್ಣ ಮನೆಗಳನ್ನು ಕಟ್ಟಿಸಬೇಕು. ಒಂದೊಂದು ಮನೆಯಲ್ಲಿ ಅಂಥ ಇಬ್ಬಿಬ್ಬರು ಅಥವಾ ಮೂರು ಮೂರು ಜನರು ಮಾತ್ರ ಇರಬೇಕು. ಅವರಿಗೆ ಒಳ್ಳೆಯ ಹಾಸಿಗೆ, ಬಟ್ಟೆಬರೆಗಳನ್ನು ಕೊಡಬೇಕು. ಅವರಿಗಾಗಿ ಒಬ್ಬ ವೈದ್ಯನಿರಬೇಕು. ವಾರಕ್ಕೆ ಒಂದು ಸಾರಿ ಅಥವಾ ಎರಡು ಸಾರಿ ಅನುಕೂಲವಾದಾಗ ಬಂದು ಆತನು ಅವರನ್ನು ನೋಡಿಕೊಂಡು ಹೋಗಬೇಕು. ಸೇವಾಶ್ರಮವು ಅನ್ನಸತ್ರದ ವಿಭಾಗದ ಹಾಗೆ ಇರಬೇಕು. ಅದರಲ್ಲಿ ರೋಗಿಗಳ ಉಪಚಾರ ನಡೆಯಬೇಕು. ಕ್ರಮವಾಗಿ, ಹಣ ಬಂದು ಸೇರಿದಾಗ, ಒಂದು ದೊಡ್ಡ ಅಡಿಗೆ ಮನೆಯನ್ನು ಕಟ್ಟಿಸಬೇಕು. ಅನ್ನ ಸತ್ರದಲ್ಲಿ ಕೇವಲ ‘ದೀಯತಾಂ ನೀಯತಾಂ ಭುಜ್ಯತಾಂ’ -ಕೊಡಬೇಕು ತರಬೇಕು - ಈ ಧ್ವನಿಯೇ ಬರುತ್ತಿರಬೇಕು. ಅನ್ನದ ಗಂಜಿ ಗಂಗೆಯಲ್ಲಿ ಹೋಗಿ ಬಿದ್ದು ಗಂಗೆಯ ನೀರೆಲ್ಲಾ ಬೆಳ್ಳಗೆ ಆಗಿಹೋಗಬೇಕು. ಹೀಗೆ ಅನ್ನಸತ್ರವಿರುವುದನ್ನು ನೋಡಿದರೆ, ಆಗ ನನ್ನ ಜೀವಕ್ಕೆ ಶಾಂತಿ.

ಶಿಷ್ಯ: ತಮಗೆ ಯಾವಾಗ ಹೀಗೆ ಇಚ್ಛೆಯುಂಟಾಯಿತೊ, ಕಾಲಕ್ರಮದಲ್ಲಿ ಇದು ನಿಜವಾಗಿ ಆಗುತ್ತದೆಂದು ತೋರುತ್ತದೆ.

ಶಿಷ್ಯನ ಮಾತನ್ನು ಕೇಳಿ ಸ್ವಾಮೀಜಿ ಸ್ವಲ್ಪ ನೀರನ್ನು ಕುಡಿದು ಸ್ವಲ್ಪ ಹೊತ್ತು ಸುಮ್ಮನಿದ್ದರು. ಆಮೇಲೆ ನಗುಮುಖದಿಂದ ವಿಶ್ವಾಸಪೂರ್ವಕವಾಗಿ ಶಿಷ್ಯನಿಗೆ ಹೇಳಿದ್ದೇನೆಂದರೆ - ನಿಮ್ಮಲ್ಲಿ ಯಾರೊಳಗೆ ಯಾವಾಗ ಸಿಂಹವು ಎಚ್ಚರಗೊಂಡು ಎದ್ದು ಬಿಡುತ್ತದೆಯೋ ಅದನ್ನು ಯಾರು ಬಲ್ಲರು? ನಿಮ್ಮಪೈಕಿ ಒಬ್ಬರಲ್ಲಿ ಜಗನ್ಮಾತೆ ಶಕ್ತಿಯನ್ನು ಎಚ್ಚರಗೊಳಿಸಿ ಎಬ್ಬಿಸಿದಳೆಂದರೆ ಭೂಮಂಡಲದ ತುಂಬಾ ಇಂಥ ಎಷ್ಟೋ ಅನ್ನಸತ್ರಗಳಾಗುತ್ತವೆ. ಏನು ತಿಳಿದುಕೊಂಡಿದ್ದೀಯೆ, ಜ್ಞಾನ, ಶಕ್ತಿ, ಭಕ್ತಿ ಇವೆಲ್ಲಾ ಎಲ್ಲರಲ್ಲಿಯೂ ಪೂರ್ಣಭಾವದಲ್ಲಿ ಇದ್ದುಕೊಂಡಿವೆ. ಅವುಗಳ ವಿಕಾಸದ ತಾರತಮ್ಯವನ್ನೇ ನಾವು ನೋಡುತ್ತ ಅವನು ದೊಡ್ಡವನು ಇವನು ಚಿಕ್ಕವನು ಎಂದುಕೊಳ್ಳುತ್ತೇವೆ. ಜೀವರ ಮನಸ್ಸಿನಲ್ಲಿ ಮಧ್ಯೆ ಒಂದು ಪರದೆ ಬಿಟ್ಟಂತಾಗಿ ಪೂರ್ಣ ವಿಕಾಸಕ್ಕೆ ಅಡ್ಡಿಯಾಗಿ ನಿಂತಿದೆ. ಅದು ಸರಿದುಹೋದರೆ ಸಾಕು; ಎಲ್ಲಾ ಆಗಿಹೋಯಿತು, ಏನು ಬೇಕು ಏನು ಕೇಳುತ್ತೀಯೆ ಅದು ಆಗುತ್ತದೆ.

ಸ್ವಾಮಿಗಳ ಮಾತನ್ನು ಕೇಳಿ ಶಿಷ್ಯನು ತನ್ನ ಮನಸ್ಸಿನೊಳಗಿನ ಪರದೆ ಯಾವಾಗ ಸರಿದುಹೋಗಿ ಈಶ್ವರ ದರ್ಶನವಾಗುವುದೋ ಎಂದು ಯೋಚಿಸುವುದಕ್ಕೆ ಮೊದಲುಮಾಡಿದನು. ಸ್ವಾಮಿಜಿ ಮತ್ತೆ ಹೇಳಿದ್ದೇನೆಂದರೆ - “ಈಶ್ವರನ ಸಂಕಲ್ಪವಿದ್ದರೆ ಈ ಮಠವನ್ನು ಮಹಾ ಸಮನ್ವಯ ಕ್ಷೇತ್ರವನ್ನಾಗಿ ಮಾಡಬಹುದು. ಪರಮಹಂಸರು ನಮ್ಮ ಸರ್ವಭಾವಗಳ ಸಾಕ್ಷಾತ್ ಸಮನ್ವಯ ಮೂರ್ತಿ. ಈ ಸಮನ್ವಯದ ಭಾವವನ್ನು ಇಲ್ಲಿ ಜಾಗ್ರತಗೊಳಿಸಿಬಿಟ್ಟರೆ ಪರಮಹಂಸರು ಜಗತ್ತಿನಲ್ಲಿ ಪ್ರತಿಷ್ಠಿತರಾಗಿ ನಿಲ್ಲುವರು. ಸರ್ವ ಮತ ಸರ್ವ ಪಥ ಬ್ರಾಹ್ಮಣ ಚಂಡಾಲ ಎಲ್ಲರೂ ಹೇಗಾದರೂ ಇಲ್ಲಿಗೆ ಬಂದು ತಮ್ಮ ತಮ್ಮ ಆದರ್ಶಗಳನ್ನು ಕಂಡುಕೊಳ್ಳುವ ಹಾಗೆ ಮಾಡಬೇಕು. ಆವೊತ್ತು ಮಠದ ನಿವೇಶನದಲ್ಲಿ ಪರಮಹಂಸರನ್ನು ಪ್ರತಿಷ್ಠೆ ಮಾಡಿದಾಗ - ಇಲ್ಲಿಂದ ಅವರ ಭಾವ ವಿಕಾಸಗೊಂಡು ಚರಾಚರ ಪ್ರಪಂಚವೆಲ್ಲವನ್ನೂ ಆವರಿಸಿಬಿಟ್ಟಂತೆ ನನ್ನ ಮನಸ್ಸಿಗೆ ಭಾವನೆಯುಂಟಾಯಿತು. ನಾನೇನೋ ಕೈಲಾದಷ್ಟು ಮಟ್ಟಿಗೆ ಮಾಡುತ್ತಿದ್ದೇನೆ. ಮುಂದೆಯೂ ಮಾಡುತ್ತೇನೆ; ನೀವೂ ಪರಮಹಂಸರ ಉದಾರಭಾವವನ್ನು ಜನರಿಗೆ ತಿಳಿಸಿಕೊಡಿ; ಸುಮ್ಮನೆ ವೇದಾಂತವನ್ನು ಓದಿ ಆಗುವುದೇನು? ಜೀವನದಲ್ಲಿ ಶುದ್ಧಾದ್ವೈತವಾದದ ಸತ್ಯತೆಯನ್ನು ಸಪ್ರಮಾಣವಾಗಿ ತೋರಿಸಬೇಕು. ಶಂಕರಾಚಾರ್ಯರು ಈ ಅದ್ವೈತವಾದವನ್ನು ಕಾಡುಮೇಡುಗಳಲ್ಲಿಯೂ ಬೆಟ್ಟಗುಡ್ಡಗಳಲ್ಲಿಯೂ ಇಟ್ಟುಹೋಗಿದ್ದಾರೆ. ನಾನು ಈಗ ಅದನ್ನು ಅಲ್ಲಿಂದ ತಂದು ಸಂಸಾರದ ಮತ್ತು ಸಮಾಜದ ಎಲ್ಲಾ ಕಡೆಗಳಲ್ಲಿಯೂ ಇಟ್ಟು ಹೋಗೋಣವೆಂದು ಬಂದಿದ್ದೇನೆ. ಮನೆ, ಮಠ, ಮೈದಾನ, ಬೆಟ್ಟ ಗುಡ್ಡಕಾಡುಮೇಡು ಎಲ್ಲಾ ಕಡೆಯಲ್ಲಿಯೂ ಈ ಅದ್ವೈತವಾದದ ದುಂದುಭಿ ನಾದವನ್ನು ತುಂಬಬೇಕು. ನೀವು ನನಗೆ ಸಹಾಯಕರಾಗಿ ಹೊರಡಿ.”

ಶಿಷ್ಯ: ಮಹಾಶಯರೆ, ಧ್ಯಾನದ ಸಹಾಯದಿಂದ ಈ ಭಾವವನ್ನು ಅನುಭವ ಮಾಡಿಕೊಂಡರೇ ನಮಗೆ ಒಳ್ಳೆಯದೆಂದು ತೋರುತ್ತದೆ. ನೆಗೆದಾಡುವುದರಲ್ಲಿ ಇಷ್ಟವಿಲ್ಲ.

ಸ್ವಾಮೀಜಿ: ಅದು ಮತ್ತಿನಿಂದ ಅಚೇತನರಾಗಿದ್ದು ಬಿಡುವುದರಂತೆ; ಸುಮ್ಮನೆ ಹೀಗಿದ್ದು ಬಿಟ್ಟರೆ ಏನಾದ ಹಾಗಾಯಿತು? ಅದ್ವೈತದ ಪ್ರೇರಣೆಯಿಂದ ಕೆಲವು ವೇಳೆ ತಾಂಡವ ನೃತ್ಯ ಮಾಡುವೆ ಮತ್ತೆ ಕೆಲವು ವೇಳೆ ಬಾಹ್ಯ ಜ್ಞಾನವಿಲ್ಲದೆ ಇರುವೆ. ಒಳ್ಳೆಯ ಪದಾರ್ಥ ಸಿಕ್ಕಿದರೆ, ಏನು ಒಬ್ಬನೇ ತಿಂದರೆ ಸುಖವಾಗುತ್ತದೆಯೆ? ಹತ್ತು ಜನಕ್ಕೆ ಕೊಡಬೇಕು, ತಾನೂ ತಿನ್ನಬೇಕು. ಆತ್ಮಾನುಭವ ಮಾಡಿಕೊಂಡು ನೀನು ಮುಕ್ತನಾಗಿ ಬಿಡುತ್ತೀಯೆ - ಅದರಿಂದ ಜಗತ್ತಿಗೆ ಆದದ್ದೇನು? ಹೋದದ್ದೇನು? ಮೂರು ಲೋಕಗಳಿಗೂ ಮುಕ್ತಿ ಕೊಟ್ಟು ಕರೆದುಕೊಂಡು ಹೋಗಬೇಕು. ಮಹಾಮಾಯೆ ರಾಜ್ಯಕ್ಕೆ ಬೆಂಕಿ ಹಾಕಿ ಹೊತ್ತಿಸಿ ಬಿಡಬೇಕು. ಆಗತಾನೇ ನಿತ್ಯವಾಗಿ ಸತ್ಯದಲ್ಲಿ ಪ್ರತಿಷ್ಠಿತನಾಗುವೆ. ಆ ಆನಂದಕ್ಕೆ ಸಮವುಂಟೇ! ‘ನಿರವಧಿ ಗಗನಾಭಂ’ - ಆಕಾಶ ಸಮಾನವಾದ ಮಹಾನಂದದಲ್ಲಿ ಪ್ರತಿಷ್ಠನಾಗುವೆ. ಜೀವಜಗತ್ತಿನ ಎಲ್ಲಾ ಕಡೆಯಲ್ಲಿಯೂ ನಿನ್ನ ನಿಜ ಸತ್ತೆಯನ್ನು ನೋಡಿ ಬೆರಗಾಗಿ ನಿಲ್ಲುವೆ! ಸ್ಥಾವರ ಜಂಗಮಗಳೆಲ್ಲವೂ ನಿನಗೆ ನಿನ್ನ ಸತ್ತಾಬಲದಲ್ಲಿ ಬೋಧೆಯಾಗುತ್ತವೆ. ಆಗ ಮಿಕ್ಕವರನ್ನೂ ನಿನ್ನ ಹಾಗೆ ಯತ್ನ ಮಾಡಗೊಡಿಸದೇ ಇರುವುದಕ್ಕಾಗುವುದಿಲ್ಲ. ಈ ಸ್ಥಿತಿಯೇ ಅನುಷ್ಠಾನ ವೇದಾಂತ, ತಿಳಿಯಿತೆ? ಬ್ರಹ್ಮ ಒಂದೇ ಆದರೂ ವ್ಯಾವಹಾರಿಕ ಸ್ಥಿತಿಯಲ್ಲಿ ನಾನಾ ರೂಪವಾಗಿ ಮುಂದೆ ನಿಂತಿದೆ. ನಾಮ ಮತ್ತು ರೂಪಗಳು ಈ ವ್ಯವಹಾರಕ್ಕೆ ಮೂಲವಾಗಿವೆ. ಹೇಗೆಂದರೆ, ಒಂದು ಗಡಿಗೆಯ ನಾಮರೂಪಗಳನ್ನು ಬಿಟ್ಟುಬಿಟ್ಟರೆ ಆಮೇಲೆ ಏನನ್ನು ನೋಡಬಲ್ಲೆ? ಅದರ ಪ್ರಕೃತ ಸತ್ತೆಯಾದ ಮಣ್ಣು ಒಂದೇ; ಅದರಂತೆ ಭ್ರಮದಿಂದ ಘಟ, ಪಟ, ಮಠ ಎಲ್ಲವನ್ನೂ ಭಾವಿಸುತ್ತಲೂ ನೋಡುತ್ತಲೂ ಇರುವೆ. ನಿಜವಾದ ಯಾವ ಸತ್ತೆಯೂ ಇಲ್ಲದೆ ಜ್ಞಾನಕ್ಕೆ ಪ್ರತಿಬಂಧಕವಾಗಿರುವ ಈ ಅಜ್ಞಾನವಿದೆಯಲ್ಲಾ, ಇದನ್ನಿಟ್ಟುಕೊಂಡೇ ವ್ಯವಹಾರ ನಡೆಯುತ್ತಿರುವುದು. ಹೆಂಡತಿ ಮಕ್ಕಳು ದೇಹ ಮನಸ್ಸುಗಳೆಂಬ ನಾಮರೂಪಗಳು ಅಜ್ಞಾನದಿಂದಾಗಿ ಸೃಷ್ಟಿಯಲ್ಲಿ ಕಂಡುಬರುತ್ತಿವೆ. ಅಜ್ಞಾನ ಸರಿದು ಹೋದರೆ, ಆಗ ಬ್ರಹ್ಮಸತ್ತೆಯು ಅನುಭವಕ್ಕೆ ಬಂದುಬಿಡುತ್ತದೆ.

ಶಿಷ್ಯ: ಈ ಅಜ್ಞಾನ ಎಲ್ಲಿಂದ ಬಂತು?

ಸ್ವಾಮಿಜಿ: ಎಲ್ಲಿಂದ ಬಂತೋ ಅದನ್ನು ಆಮೇಲೆ ಹೇಳುತ್ತೇನೆ. ನೀನು ಹಗ್ಗವನ್ನು ಹಾವೆಂದುಕೊಂಡು ಹೆದರಿಕೆಯಿಂದ ಓಡುವುದಕ್ಕೆ ತೊಡಗಿದಾಗ ಹಗ್ಗವೇನು ಹಾವಾಗಿಬಿಟ್ಟಿತೊ? ಇಲ್ಲ; ಅಜ್ಞಾನವೇ ನಿನ್ನಿಂದ ಹಾಗೆ ಮಾಡಿಸಿತೊ?

ಶಿಷ್ಯ: ಅಜ್ಞಾನದಿಂದಲೇ ಹೀಗೆ ಮಾಡಿದೆ.

ಸ್ವಾಮೀಜಿ: ಹಾಗಾದರೆ ಯೋಚಿಸಿ ನೋಡು - ನೀನು ಹಗ್ಗವನ್ನು ಹಗ್ಗವೆಂದು ತಿಳಿದುಕೊಂಡಾಗ ನಿನ್ನ ಹಿಂದಿನ ಅಜ್ಞಾನವನ್ನು ಜ್ಞಾಪಿಸಿಕೊಂಡು ನಗುವೆಯೊ ಇಲ್ಲವೊ? ಆಗ ನಾಮರೂಪಗಳು ಮಿಥ್ಯೆಯೆಂದು ತಿಳಿಯುವುದೊ ಇಲ್ಲವೊ?

ಶಿಷ್ಯ: ಹೌದು

ಸ್ವಾಮಿಜಿ: ಹಾಗಾದರೆ, ನಾಮರೂಪಗಳು ಮಿಥ್ಯೆಯಾಯಿತು. ಹೀಗೆ ಬ್ರಹ್ಮವೊಂದು ಮಾತ್ರ ಸತ್ಯವಾಗಿ ನಿಲ್ಲುತ್ತದೆ. ಈ ಅನಂತ ಸೃಷ್ಟಿ ವೈಚಿತ್ರ್ಯದಲ್ಲಿಯೂ ಅದರ ಸ್ವರೂಪವು ಸ್ವಲ್ಪವೂ ಬದಲಾಯಿಸುವುದಿಲ್ಲ. ನೀನು ಮಾತ್ರ ಈ ಅಜ್ಞಾನದ ಗಾಡಾಂಧಕಾರದಲ್ಲಿ ಇವಳು ಹೆಂಡತಿ, ಇದು ಮಗು, ಇದು ನಾನು, ಇದು ಅನ್ಯ ಎಂದು ಭಾವಿಸಿಕೊಂಡು ಸರ್ವ ವಿಭಾಸಕವಾದ ಆತ್ಮದ ಸತ್ತೆಯನ್ನು ತಿಳಿದುಕೊಳ್ಳಲಾರೆ. ಗುರೂಪದೇಶದಿಂದಲೂ ನಿನ್ನ ನಂಬಿಕೆಯಿಂದಲೂ ಈ ನಾಮರೂಪಾತ್ಮಕವಾದ ಜಗತ್ತನ್ನು ನೋಡದೆ ಇದರ ಮೂಲವಾದ ಸತ್ತೆಯನ್ನು ಮಾತ್ರವೇ ಅನುಭವ ಮಾಡುವಾಗ ಆಬ್ರಹ್ಮಸಂಭ ಪರ್ಯನ್ತವಾದ ಸಕಲ ಪದಾರ್ಥದಲ್ಲಿಯೂ ನಿನಗೆ ಆತ್ಮಾನುಭವವುಂಟಾಗುತ್ತದೆ - ಆಗ ‘ಭಿದ್ಯತೇ ಹೃದಯ ಗ್ರಂಥಿಶ್ಛಿದ್ಯಂತೇ ಸರ್ವಸಂಶಯಾಃ’ ಹೃದಯದ ಗಂಟುಗಳೆಲ್ಲ ಕತ್ತರಿಸಲ್ಪಟ್ಟು ಎಲ್ಲ ಸಂಶಯಗಳೂ ನಾಶವಾಗುತ್ತವೆ.

ಶಿಷ್ಯ: ಮಹಾಶಯರೆ, ಈ ಅಜ್ಞಾನದ ಆದ್ಯಂತಗಳ ವಿಷಯವನ್ನು ತಿಳಿದುಕೊಳ್ಳಬೇಕೆಂದು ಆಶೆಯಾಗಿದೆ.

ಸ್ವಾಮಿಜಿ: ಯಾವ ವಸ್ತು ಆಮೇಲೆ ಇರುವುದಿಲ್ಲವೋ - ಯಾವ ವಸ್ತು ಮಿಥ್ಯೆಯೊ ಅದನ್ನು ಹೇಗೆ ತಿಳಿದುಕೊಳ್ಳಬಲ್ಲೆ? ಯಾರು ಯಥಾರ್ಥ ಬ್ರಹ್ಮಜ್ಞರೊ ಅವರು ‘ಅಜ್ಞಾನವೆಲ್ಲಿದೆ?’ ಎಂದು ಕೇಳುವರು. ಅವರು ಹಗ್ಗವನ್ನು ಹಗ್ಗವನ್ನಾಗಿಯೆ ನೋಡುವರು, ಹಾವೆಂದು ತಿಳಿಯಲಾರರು. ಯಾರು ಹಗ್ಗವನ್ನು ಹಾವೆಂದು ತಿಳಿಯುವರೊ, ಅವರ ಭಯಭೀತಿಗಳನ್ನು ನೋಡಿ ಅವರಿಗೆ ನಗು ಬರುವುದು! ಆದ್ದರಿಂದ ಅಜ್ಞಾನವನ್ನು ಸತ್ ಎಂದು ಹೇಳುವುದಕ್ಕಾಗುವುದಿಲ್ಲ - ಅಸತ್ ಎಂದೂ ಹೇಳುವುದಕ್ಕಾಗುವುದಿಲ್ಲ. ‘ಸನ್ನಾಪ್ಯಸನ್ನಾಪ್ಯುಭಯಾತ್ಮಿಕಾನೋ’ ಅದು ಸತ್ತೂ ಅಲ್ಲ, ಅಸತ್ತೂ ಅಲ್ಲ, ಎರಡರ ಮಿಶ್ರಣವೂ ಅಲ್ಲ. ಯಾವ ವಸ್ತು ಹೀಗೆ ಮಿಥ್ಯೆಯೆಂದು ಗೊತ್ತಾಗಿದೆಯೋ ಅದರ ವಿಷಯವಾಗಿ ಪ್ರಶ್ನೆ ಮಾಡುವುದೇನು, ಉತ್ತರ ತಾನೇ ಹೇಳುವುದೇನು? ಈ ವಿಷಯವಾಗಿ ಪ್ರಶ್ನೆ ಮಾಡುವುದೂ ಯುಕ್ತವಾಗುವುದಿಲ್ಲ. ಏಕೆಂದರೆ, ಈ ಪ್ರಶ್ನೋತ್ತರಗಳೂ ನಾಮರೂಪ ಅಥವಾ ದೇಶಕಾಲಗಳನ್ನು ಅವಲಂಬಿಸಿಕೊಂಡೇ ನಡೆಯುತ್ತವೆ. ಯಾವ ಬ್ರಹ್ಮವಸ್ತುವು ನಾಮ ರೂಪ ದೇಶ ಕಾಲಗಳನ್ನು ಮೀರಿದ್ದೋ ಅದನ್ನು ಪ್ರಶ್ನೋತ್ತರಗಳಿಂದ ತಿಳಿಸಲೇನು ಆಗುವುದೆ? ಆದ್ದರಿಂದ ಶಾಸ್ತ್ರಮಂತ್ರಾದಿಗಳು ವ್ಯಾವಹಾರಿಕ ಭಾವದಲ್ಲಿ ಸತ್ಯ-ಪಾರಮಾರ್ಥಿಕ ರೂಪದಲ್ಲಿ ಸತ್ಯವಲ್ಲ. ಅಜ್ಞಾನಕ್ಕೆ ಸ್ವಂತ ರೂಪವೇ ಇಲ್ಲ. ಅದನ್ನು ಹೇಗೆ ತಿಳಿಯುವೆ? ಬ್ರಹ್ಮದ ಪ್ರಕಾಶವಾದಾಗಂತೂ ಇಂಥ ಪ್ರಶ್ನೆಯನ್ನು ಹಾಕುವ ಆವಶ್ಯಕತೆಯೇ ಇರುವುದಿಲ್ಲ. ಪರಮಹಂಸರ 'ಮೋಚಿ ಮತ್ತು ಬ್ರಾಹ್ಮಣನ ಕಥೆಯನ್ನು\footnote{ಒಮ್ಮೆ ಬ್ರಾಹ್ಮಣನೊಬ್ಬನಿಗೆ ತನ್ನ ಶಿಷ್ಯನ ಮನೆಗೆ ಹೋಗಬೇಕಾಗಿ ಬಂತು. ತನ್ನ ಸಾಮಾನನ್ನು ಹೊರಲು ಒಬ್ಬ ಕೂಲಿ ಬೇಕಾಯಿತು. ಯಾರೂ ಸಿಕ್ಕಲಿಲ್ಲ. ಕೊನೆಗೆ ಒಬ್ಬ ಮೋಚಿಯನ್ನು ಆ ಕೆಲಸಕ್ಕೆ ಕರೆದ. ಆತ 'ನಾನು ಅಸ್ಪೃಶ್ಯ, ಬರುವುದಿಲ್ಲ' ಎಂದ. ಬ್ರಾಹ್ಮಣ 'ಪರವಾಗಿಲ್ಲ, ಬಾ. ಒಂದು ಮಾತೂ ಆಡದೆ ಸುಮ್ಮನಿದ್ದು ಬಿಡು. ಯಾರೂ ನಿನ್ನನ್ನು ಗುರುತಿಸುವುದಿಲ್ಲ' ಎಂದು ಹೇಳಿ ಅವನನ್ನು ಒಪ್ಪಿಸಿದ. ಮೋಚಿಯೊಡನೆ ಬ್ರಾಹ್ಮಣನು ಶಿಷ್ಯನ ಊರನ್ನು ತಲುಪಿದ. ಅಲ್ಲಿ ಒಬ್ಬ ವ್ಯಕ್ತಿ ಮೋಚಿಯನ್ನು ಕುರಿತು “ಅಯ್ಯಾ, ಈ ಜೋಡನ್ನು ತೆಗೆದು ಆ ಕಡೆ ಇಡು” ಎಂದ. ಮೋಚಿ ಏನೂ ಉತ್ತರ ಕೊಡಲಿಲ್ಲ. ನಾಲ್ಕೂ ಸಲ ಕೇಳಿದರು, ಇವನು ಹು, ಉಹುಂ ಅನ್ನಲಿಲ್ಲ. ಕೊನೆಗೆ ಆ ವ್ಯಕ್ತಿ ರೇಗಿ “ಯಾಕೋ ಮಾತಾಡೊಲ್ಲ, ನೀನೇನು ಮೋಚಿಯೇನೋ?” ಎಂದು ಕೇಳಿದ. ಹೆದರಿದ ಮೋಚಿ ತನ್ನನ್ನು ಕರೆದು ತಂದಿದ್ದ ಬ್ರಾಹ್ಮಣನ ಕಡೆ ತಿರುಗಿ 'ಓ, ಸ್ವಾಮಿ, ಸರ್ವನಾಶವಾಯಿತು. ನಾನು ಯಾರು ಎಂದು ಕಂಡುಹಿಡಿದು ಬಿಟ್ಟರು. ಇನ್ನು ಇಲ್ಲಿರಲಾರೆ' ಎಂದು ಹೇಳುತ್ತಾ ದೌಡಾಯಿಸಿದ.} ಕೇಳಿಲ್ಲವೆ? ಅದರ ಹಾಗೆಯೇ, ಅಜ್ಞಾನವನ್ನು ಯಾರಾದರೂ ಗುರುತಿಸಿದರೆ, ಆಗ ಅದು ಓಡಿಹೋಗುತ್ತದೆ.

ಶಿಷ್ಯ: ಆದರೆ, ಮಹಾಶಯರೆ, ಅಜ್ಞಾನವೂ ಬಂದದ್ದು ಎಲ್ಲಿಂದ?

ಸ್ವಾಮಿಜಿ: ಯಾವ ಪದಾರ್ಥ ಇಲ್ಲವೊ ಅದು ಬರುವುದು ಹೇಗೆ? ಇದ್ದರೆ ತಾನೇ ಬರುವುದು.

ಶಿಷ್ಯ: ಹಾಗಾದರೆ ಈ ಜೀವಜಗತ್ತಿನ ಉತ್ಪತ್ತಿ ಹೇಗೆ ಆಯಿತು?

ಸ್ವಾಮೀಜಿ: ಬ್ರಹ್ಮಸತ್ತೆಯೊಂದೇ ಇರುವುದು; ನೀನು ಮಿಥ್ಯಾನಾಮರೂಪಗಳಿಂದ ಅದನ್ನು ಬೇರೆ ರೂಪದಲ್ಲಿಯೂ ಬೇರೆ ಹೆಸರಲ್ಲಿಯೂ ನೋಡುವೆ.

ಶಿಷ್ಯ: ಮಿಥ್ಯಾನಾಮರೂಪಗಳು ತಾನೇ ಏಕೆ? ಎಲ್ಲಿಂದ ಬಂದುವು?

ಸ್ವಾಮೀಜಿ: ಶಾಸ್ತ್ರದಲ್ಲಿ ಈ ನಾಮರೂಪಾತ್ಮಕವಾದ ಸಂಸಾರ ಅಥವಾ ಅಜ್ಞತೆಯು ಪ್ರವಾಹರೂಪವಾಗಿ ನಿತ್ಯಪ್ರಾಯವಾಗಿರುವಂತೆ ಹೇಳಿದೆ. ಆದರೆ ಅದು ಸಾಂತ. ಬ್ರಹ್ಮ ಸತ್ತೆ ಮಾತ್ರ ಯಾವಾಗಲೂ ಹಗ್ಗದ ಹಾಗೆ ಸ್ವಸ್ವರೂಪದಲ್ಲಿಯೇ ಇರುತ್ತದೆ. ಆದ್ದರಿಂದ ವೇದಾಂತ ಶಾಸ್ತ್ರದ ಸಿದ್ಧಾಂತವೇನೆಂದರೆ, ಬ್ರಹ್ಮಾಂಡವೆಲ್ಲಾ ಬ್ರಹ್ಮದಲ್ಲಿ ಅಧ್ಯಸ್ತವಾಗಿದೆ. ಇಂದ್ರಜಾಲದ ಹಾಗೆ ಕಾಣುತ್ತಿದೆ. ಅದರಿಂದ ಬ್ರಹ್ಮದ ಸ್ವರೂಪಕ್ಕೆ ಸ್ವಲ್ಪವೂ ವೈಲಕ್ಷಣ್ಯ ಬರುವುದಿಲ್ಲ. ತಿಳಿಯಿತೆ?

ಶಿಷ್ಯ: ಒಂದು ವಿಷಯವನ್ನು ಈಗಲೂ ತಿಳಿದುಕೊಳ್ಳುವುದಕ್ಕಾಗುವುದಿಲ್ಲ.

ಸ್ವಾಮಿಜಿ: ಏನು ಹೇಳು.

ಶಿಷ್ಯ: ಈ ಸೃಷ್ಟಿ ಸ್ಥಿತಿ ಲಯ ಮುಂತಾದುವುಗಳೆಲ್ಲಾ ಬ್ರಹ್ಮದಲ್ಲಿ ಅಧ್ಯಸ್ತವಾಗಿವೆ, ಅವುಗಳಿಗೆ ಯಾವ ವಿಧವಾದ ಸ್ವರೂಪ ಸತ್ತೆಯೂ ಇಲ್ಲ ಎಂದು ತಾವು ಹೇಳಿದಿರಲ್ಲಾ, ಅದು ಹೇಗೆ ಸಾಧ್ಯ? ಯಾರು ಯಾವುದನ್ನು ಮೊದಲು ನೋಡಿಲ್ಲವೋ ಆ ಪದಾರ್ಥದ ಭ್ರಮವು ಅದರಿಂದಲೇ ಉಂಟಾಗುವುದಿಲ್ಲ. ಯಾರು ಹಾವನ್ನು ಎಂದೂ ನೋಡಿಲ್ಲವೊ ಅವನಿಗೆ ಹಗ್ಗದಲ್ಲಿ ಹಾವು ಎಂಬ ಭ್ರಮೆ ಹೇಗೆ ಉಂಟಾಗುವುದಿಲ್ಲವೋ ಹಾಗೆ ಯಾರು ಈ ಸೃಷ್ಟಿಯನ್ನು ನೋಡಿಲ್ಲವೋ ಅವನಿಗೆ ಬ್ರಹ್ಮದಲ್ಲಿ ಸೃಷ್ಟಿ ಭ್ರಮೆ ಹೇಗಾಗುತ್ತದೆ? ಆದ್ದರಿಂದ ಸೃಷ್ಟಿಯ ಭ್ರಮೆ ಉಂಟಾಗುವುದಕ್ಕೆ ಸೃಷ್ಟಿಯು ಮೊದಲೇ ಇದ್ದಿರಬೇಕು ಅಥವಾ ಇದೆ. ಆದರೆ ಇದು ದ್ವೈತ ಭಾವವನ್ನು ತಂದುಬಿಡುತ್ತದೆ.

ಸ್ವಾಮೀಜಿ: ಬ್ರಹ್ಮಜ್ಞಾನಿ ನಿನ್ನ ಪ್ರಶ್ನೆಯನ್ನು ಮೊದಲೇ ಹೀಗೆ ನಿವಾರಣೆ ಮಾಡಿರುವನು: ಹೇಗೆಂದರೆ, ಅವನ ದೃಷ್ಟಿಗೆ ಸೃಷ್ಟಿ ಮುಂತಾದವು ಒಟ್ಟಿಗೆ ಕಾಣುವುದೇ ಇಲ್ಲ; ಅವನು ಬ್ರಹ್ಮಸತ್ತೆಯೊಂದನ್ನು ಮಾತ್ರ ನೋಡುವನು; ಹಗ್ಗವನ್ನೇ ನೋಡುವನು ಹಾವನ್ನು ನೋಡುವುದಿಲ್ಲ. ‘ನಾನು ಮಾತ್ರ ಈ ಸೃಷ್ಟಿ ಅಥವಾ ಹಾವನ್ನು ನೋಡುತ್ತೇನೆ’ ಎಂದು ನೀನು ಹೇಳಿದರೆ - ಆಗ, ನಿನ್ನ ದೃಷ್ಟಿದೋಷವನ್ನು ಹೋಗಲಾಡಿಸುವುದಕ್ಕೆ ಆತನು ನಿನಗೆ ಹಗ್ಗದ ಸ್ವರೂಪವನ್ನು ತಿಳಿಸಿಕೊಡುವುದಕ್ಕೆ ಪ್ರಯತ್ನ ಮಾಡುವನು. ಆತನ ಉಪದೇಶ ಮತ್ತು ವಿಚಾರಗಳ ಬಲದಿಂದ ನೀನು ರಜ್ಜುಸತ್ತೆ ಅಥವಾ ಬ್ರಹ್ಮಸತ್ತೆಯನ್ನು ತಿಳಿದುಕೊಳ್ಳುವ ಹಾಗಾದಾಗ, ಈ ಸರ್ಪಜ್ಞಾನ ಅಥವಾ ಸೃಷ್ಟಿಜ್ಞಾನವು ನಾಶವಾಗಿ ಹೋಗುತ್ತದೆ. ಆಗ ಈ ಸೃಷ್ಟಿ ಸ್ಥಿತಿ ಲಯ ರೂಪವಾದ ಬಾಹ್ಯ ಜ್ಞಾನವನ್ನು ಬ್ರಹ್ಮದಲ್ಲಿ ಆರೋಪಿತವೆಂದಲ್ಲದೆ ಮತ್ತೇನು ಹೇಳುವೆ? ಅನಾದಿ ಪ್ರವಾಹದಂತೆ ಈ ಸೃಷ್ಟಿ ಮುಂದುವರಿದರೆ ಅದರಿಂದ ಏನೂ ಬಾಧೆಯಿಲ್ಲ. ಬ್ರಹ್ಮತತ್ತ್ವವು ಕರತಲಾಮಲಕದ ಹಾಗೆ ಪ್ರತ್ಯಕ್ಷವಾಗದೆ ಈ ಪ್ರಶ್ನೆಯು ಪೂರ್ಣವಾಗಿ ಸಿದ್ಧಾಂತವಾಗುವುದು ಸಾಧ್ಯವಿಲ್ಲ; ಮತ್ತು ಆಗ ಪ್ರಶ್ನೆಯೂ ಹುಟ್ಟುವುದಿಲ್ಲ; ಉತ್ತರದ ಆವಶ್ಯಕತೆಯೂ ಇರುವುದಿಲ್ಲ. ಬ್ರಹ್ಮತತ್ತ್ವಾಸ್ವಾದವು ಆಗ ‘ಮೂಕಾಸ್ವಾದನವತ್’ ಮೂಕನು ಸವಿದಂತೆ ಆಗುತ್ತದೆ.

ಶಿಷ್ಯ: ಹಾಗಾದರೆ ಇಷ್ಟು ವಿಚಾರ ಮಾಡಿ ಆಗುವುದೇನು?

ಸ್ವಾಮಿಜಿ: ಈ ವಿಷಯವನ್ನು ತಿಳಿಸುವುದಕ್ಕೋಸ್ಕರ ವಿಚಾರ. ಸತ್ಯವನ್ನು ಮಾತ್ರ ವಿಚಾರಕ್ಕಿಂತ ಹೊರಗೆ - ‘ನೈಷಾ ತರ್ಕೇಣ ಮತಿರಾಪನೇಯಾ’ ಈ ತೀರ್ಮಾನವನ್ನು ತರ್ಕದ ಮೂಲಕ ತಲುಪಲು ಆಗುವುದಿಲ್ಲ. ಹೀಗೆ ಮಾತನಾಡುತ್ತ ಆಡುತ್ತ ಶಿಷ್ಯನು ಸ್ವಾಮೀಜಿಯೊಡನೆ ಮಠಕ್ಕೆ ಬಂದನು. ಮಠಕ್ಕೆ ಬಂದು ಸ್ವಾಮೀಜಿ ಮಠದ ಸಂನ್ಯಾಸಿ ಮತ್ತು ಬ್ರಹ್ಮಚಾರಿಗಳಿಗೆ ಇಂದಿನ ಬ್ರಹ್ಮವಿಚಾರದ ಮರ್ಮವನ್ನು ಸಂಕ್ಷೇಪವಾಗಿ ತಿಳಿಸಿದರು.

ಮೇಲಕ್ಕೆ ಹೋಗುವಾಗ ಶಿಷ್ಯನನ್ನು ಕುರಿತು ‘ನಾಯಮತ್ಮಾ ಬಲಹೀನೇನ ಲಭ್ಯಃ’ ದುರ್ಬಲರು ಆತ್ಮನನ್ನು ಸಾಧಿಸಲಾರರು ಎಂದು ಹೇಳುತ್ತಾ ಹೋದರು.

\newpage

\chapter[ಅಧ್ಯಾಯ ೨೨]{ಅಧ್ಯಾಯ ೨೨\protect\footnote{\engfoot{Complete Works of Swami Vivekananda, Volume VI, Page 445}}}

\begin{center}
ಸ್ಥಳ: ಬೇಲೂರು ಮಠವನ್ನು ಕಟ್ಟುತ್ತಿದ್ದಾಗ, ವರ್ಷ: ೧೮೯೮.
\end{center}

ಶಿಷ್ಯ: ಸ್ವಾಮೀಜಿ, ನೀವೇಕೆ ಈ ದೇಶದಲ್ಲಿ ಭಾಷಣ ಕೊಡುವುದಿಲ್ಲ? ಯೂರೋಪ್ ಮತ್ತು ಅಮೆರಿಕಾ ದೇಶಗಳನ್ನು ನಿಮ್ಮ ವಾಗ್ಧಾರೆಯಿಂದ ಎಚ್ಚರಿಸಿದಿರಿ, ಇಲ್ಲಿಗೆ ಹಿಂತಿರುಗಿದ ಮೇಲೆ ಮೌನವನ್ನವಲಂಬಿಸಿರುವಿರಿ.

ಸ್ವಾಮೀಜಿ: ಈ ದೇಶದಲ್ಲಿ ಮೊದಲು ಭೂಮಿ ಹಸನಾಗಬೇಕು. ನಂತರ ಬೀಜ ಬಿತ್ತಿದರೆ ಗಿಡ ಸೊಂಪಾಗಿ ಬೆಳೆಯುವುದು. ಪಾಶ್ಚಾತ್ಯ ದೇಶದಲ್ಲಿ, ಯೂರೋಪ್ ಮತ್ತು ಅಮೆರಿಕಾ ದೇಶಗಳಲ್ಲಿ ನೆಲ ತುಂಬಾ ಫಲವತ್ತಾಗಿದೆ. ಬೀಜ ಬಿತ್ತಲು ಯೋಗ್ಯವಾಗಿದೆ. ಅಲ್ಲಿ ಜನರು ಭೋಗದ ಪರಾಕಾಷ್ಠೆಯನ್ನು ಮುಟ್ಟಿದ್ದಾರೆ. ಭೋಗದಲ್ಲಿ ಕಂಠಪೂರ್ತಿ ಸಂಪೂರ್ಣವಾಗಿ ತೃಪ್ತಿ ಹೊಂದಿ ಈಗ ಅವರ ಮನಸ್ಸಿಗೆ ಆ ಭೋಗದಿಂದ ಕೂಡ ಶಾಂತಿ ಸಿಗದೆ ಬೇರೇನೋ ಬೇಕೆಂದು ಅನ್ನಿಸಿದೆ. ಈ ದೇಶದಲ್ಲಿ ಭೋಗವೂ ಇಲ್ಲ, ಯೋಗವೂ ಇಲ್ಲ. ಯಾರು ಸಂಪೂರ್ಣ ಭೋಗದಿಂದ ತೃಪ್ತರೋ ಅವರಿಗೆ ಮಾತ್ರ ಯೋಗದ ವಿಚಾರ ಹೇಳಲು ತಿಳಿದುಕೊಳ್ಳಲು ಸಾಧ್ಯ. ನಮ್ಮ ಹಿಂದೂ ದೇಶವಾದರೊ ದುಃಖ ಬೇನೆಗಳ ತೌರೂರಾಗಿ ಉಪವಾಸದಿಂದ ಕೊರಗಿ ದುರ್ಬಲ ಮನಸ್ಕರಾದ ಜನರಿಂದ ತುಂಬಿದೆ. ಇಂತಹ ಕಡೆ ಉಪನ್ಯಾಸದಿಂದಾಗುವ ಪ್ರಯೋಜನವೇನು?

ಶಿಷ್ಯ: ಅದು ಹೇಗೆ? ನಮ್ಮದು ಧರ್ಮದ ತೌರೂರು. ನಮ್ಮ ದೇಶೀಯರು ಧರ್ಮವನ್ನು ಅರ್ಥಮಾಡಿಕೊಳ್ಳುವಷ್ಟು ಮತ್ತಾವ ಭಾಗದ ಜನರೂ ಅರ್ಥಮಾಡಿಕೊಳ್ಳುವುದಿಲ್ಲವೆಂದು ನೀವೇ ಹೇಳುತ್ತೀರಿ? ಹಾಗಾದರೆ ನಮ್ಮ ದೇಶವೇಕೆ ನಿಮ್ಮ ಪ್ರಚಂಡ ವಾಗ್ವೈಖರಿಯಿಂದ ಸಚೇತನವಾಗಿ ಫಲಕಾರಿಯಾಗಬಾರದು?

ಸ್ವಾಮೀಜಿ: ಈಗ ಧರ್ಮವೇನೆಂಬುದನ್ನು ತಿಳಿದುಕೊ. ಮೊದಲನೆಯದಾಗಿ ಕೂರ್ಮ (ಆಮೆ) ಅವತಾರದ ಪೂಜೆ ಆವಶ್ಯಕ, ಎಂದರೆ ‘ಹೊಟ್ಟೆಯ ಪೂಜೆ!’ ನೀನು ಅದನ್ನು ಸಮಾಧಾನಪಡಿಸುವವರೆಗೆ ಧರ್ಮದ ವಿಚಾರವಾಗಿ ನಿನ್ನ ಮಾತನ್ನಾರೂ ಕೇಳುವುದಿಲ್ಲ. ಭರತಖಂಡವೀಗ ಈ ಹಸಿವೆಂಬ ಭೂತವನ್ನು ಹೇಗೆ ಎದುರಿಸುವುದೆಂದು ತಳಮಳಗೊಂಡಿದೆ. ಅನ್ಯ ದೇಶೀಯರಿಂದ ದೇಶದ ಅಮೂಲ್ಯ ಸಂಪತ್ತೆಲ್ಲಾ ಸೂರೆಯಾಗಿರುವುದು, ಸರಕುಗಳನ್ನು ತಡೆಯಿಲ್ಲದೆ ರಫ್ತು ಮಾಡುತ್ತಿರುವುದು, ಎಲ್ಲಕ್ಕಿಂತ ಹೆಚ್ಚಾಗಿ ದಾಸ್ಯ ಸ್ವಭಾವದಿಂದ ಪ್ರಾಪ್ತವಾದ ಹೊಲಸು, ಅಸೂಯೆ ಇವು ಭರತಖಂಡದ ಜೀವನವನ್ನೆಲ್ಲ ಸೂರೆಗೊಳ್ಳುತ್ತಿವೆ. ನೀನು ಯಾರಿಗೆ ಬೋಧಿಸಲು ಹೊರಟಿರುವೆಯೋ ಅವರನ್ನು ಈ ಹಸಿವು ಉಪವಾಸದಿಂದ, ಕೇವಲ ಜೀವಿಸುವುದು ಹೇಗೆಂಬ ನಿರಂತರ ಯೋಚನೆಯಿಂದ ಪಾರುಮಾಡಬೇಕು. ಇಲ್ಲದಿದ್ದಲ್ಲಿ ನಿನ್ನ ಉಪನ್ಯಾಸಗಳಿಂದ ಏನೊಂದೂ ಪ್ರಯೋಜನವಿಲ್ಲ.

ಶಿಷ್ಯ: ಈ ಕೆಡುಕನ್ನು ಹೋಗಲಾಡಿಸಲು ನಾವೇನು ಮಾಡಬೇಕು?

ಸ್ವಾಮೀಜಿ: ಮೊದಲು ಪೂರ್ತಿ ತ್ಯಾಗಬುದ್ಧಿಯಿಂದ ಕೂಡಿದ, ತಮ್ಮ ಸ್ವಾರ್ಥ ಸುಖವನ್ನು ಬಯಸದೆ ಇತರರಿಗಾಗಿ ತಮ್ಮ ಪ್ರಾಣವನ್ನೇ ಅರ್ಪಿಸಲು ಸಿದ್ಧರಾಗಿರುವ ಕೆಲವು ಯುವಕರು ಅಗತ್ಯ. ಈ ಉದ್ದೇಶವನ್ನು ಮುಂದಿಟ್ಟುಕೊಂಡು ತರುಣ ಸಂನ್ಯಾಸಿಗಳನ್ನು ತರಬೇತು ಮಾಡುವುದಕ್ಕಾಗಿ ನಾನು ಈ ಮಠವನ್ನು ಸ್ಥಾಪಿಸುತ್ತೇನೆ. ಅವರು ಮನೆಮನೆಗೂ ಹೋಗಿ ಜನರಿಗೆ ಸಕಾರಣವಾಗಿ ಅವರ ದುಃಸ್ಥಿತಿಯ ಅರಿವಾಗುವಂತೆ ಮಾಡಿ ಅವರ ಮೇಲ್ಮೆಗಾಗಿ ಶಿಕ್ಷಣ ಕೊಡಬೇಕು. ಅಲ್ಲದೆ ಸ್ಪಷ್ಟವಾದ ಸರಳ ಮತ್ತು ಸುಲಭ ಭಾಷೆಯಲ್ಲಿ ಧರ್ಮದ ಅಮೋಘ ಸತ್ಯಗಳನ್ನು ವಿವರಿಸಬೇಕು. ನಮ್ಮ ದೇಶದ ಜನಸಾಮಾನ್ಯರು, ಮಲಗಿರುವ ರಾಕ್ಷಸನಂತೆ. ಈಗಿನ ಕಾಲದ ವಿಶ್ವವಿದ್ಯಾನಿಲಯದಲ್ಲಿ ಕೊಡುವ ಶಿಕ್ಷಣ ಕೇವಲ ಶೇಕಡ ಒಂದು ಅಥವಾ ಎರಡರಷ್ಟು ಜನರಿಗೆ ಸಾಧ್ಯ. ಯಾರು ಇದನ್ನು ಪಡೆದಿರುವರೊ ಅವರು ಕೂಡ ದೇಶಕ್ಕೆ ಕಲ್ಯಾಣಕರವಾದ ಕೆಲಸಗಳನ್ನು ಮಾಡುವ ಪ್ರಯತ್ನಗಳೆಲ್ಲಾ ನಿಷ್ಪಲವಾಗಿದೆ. ಆದರೆ ಇದು ಪಾಪ ಅವರ ತಪ್ಪಲ್ಲ! ಅವರು ಕಾಲೇಜು, ಶಾಲೆಯನ್ನು ಬಿಟ್ಟು ಬರುವುದಕ್ಕಿಲ್ಲ, ಆಗಲೇ ಅವರು ಹಲವಾರು ಮಕ್ಕಳ ತಂದೆಯಾಗಿರುತ್ತಾರೆ. ಹೇಗೋ ಕಷ್ಟಪಟ್ಟು ಒಬ್ಬ ಗುಮಾಸ್ತ ಅಥವಾ ಹೆಚ್ಚೆಂದರೆ ಒಬ್ಬ ಉಪನ್ಯಾಯಾಧೀಶನ ಸ್ಥಾನವನ್ನು ಸಂಪಾದಿಸುತ್ತಾರೆ. ಇದೇ ಅವರ ವಿದ್ಯಾಭ್ಯಾಸದ ಚರಮ ಗುರಿ. ಸಂಸಾರದ ಭಾರವನ್ನು ಹೆಗಲ ಮೇಲೆ ಹೊತ್ತು ಅವರಿಗೆ ಏನಾದರೊಂದು ದೊಡ್ಡ ಕೆಲಸವನ್ನು ಮಾಡುವುದಕ್ಕಾಗಲಿ, ಯೋಚಿಸುವುದಕ್ಕಾಗಲಿ ಪುರಸತ್ತೇ ಇರುವುದಿಲ್ಲ. ತಮ್ಮ ಸ್ವಂತ ಇಚ್ಛೆ ಆಸಕ್ತಿಗಳನ್ನು ಪೂರೈಸಿಕೊಳ್ಳುವುದಕ್ಕೆ ಅವರಿಗೆ ದಾರಿಯಿಲ್ಲ - ಇಂತಹವರಿಂದ ಇತರರಿಗಾಗಿ ಏನಾದರೂ ಸಹಾಯವಾಗುವುದೆಂದು ಹೇಗೆ ನಂಬುವುದು?

ಶಿಷ್ಯ: ಹಾಗಾದರೆ ಇದರಿಂದ ಪಾರಾಗುವ ಹಾದಿಯೇ ಇಲ್ಲವೆ?

ಸ್ವಾಮೀಜಿ: ಖಂಡಿತ ಇದೆ. ನಮ್ಮದು ನಿತ್ಯ ಪುಣ್ಯಭೂಮಿ. ನಮ್ಮ ದೇಶವೇನೊ ಅಧೋಗತಿಗಿಳಿದಿದೆ ನಿಜ. ಆದರೆ ಅದು ಖಂಡಿತವಾಗಿಯೂ ಅಷ್ಟೇ ಮೇಲಕ್ಕೇಳುತ್ತದೆ. ಆಗ ಅದರ ರಭಸ ಇಡೀ ಭೂಮಂಡಲವನ್ನೇ ದಿಗ್ಭ್ರಮೆಗೊಳಿಸುವುದು. ಸಮುದ್ರದಲ್ಲಿ ಅಲೆಗಳು ಎಷ್ಟು ರಭಸದಿಂದ ಇಳಿಯುವುವೋ ಅಷ್ಟೇ ರಭಸದಿಂದ ಮೇಲಕ್ಕೇಳುವುವು.

ಶಿಷ್ಯ: ಭರತಖಂಡ ಹೇಗೆ ಮೇಲಕ್ಕೇಳುವುದು?

ಸ್ವಾಮೀಜಿ: ನೀನು ನೋಡುತ್ತಿಲ್ಲವೆ? ಪೂರ್ವ ದಿಗಂತದಲ್ಲಿ ಆಗಲೇ ಸೂರ್ಯೋದಯವಾಗಿದೆ. ಇನ್ನೇನು ಸೂರ್ಯನು ಹುಟ್ಟುತ್ತಾನೆ. ನೀವೆಲ್ಲಾ ಚಕ್ರಕ್ಕೆ ಹೆಗಲು ಕೊಡಿ. ಪ್ರಪಂಚವೇ ನನ್ನ ಸರ್ವಸ್ವವೆಂದು ‘ನನ್ನ ಸಂಸಾರ, ನನ್ನ ಸಂಸಾರ’ ಎಂದು ಯೋಚಿಸುತ್ತಿದ್ದರೆ ಏನಾದಹಾಗಾಯಿತು? ದೇಶದ ಒಂದು ಭಾಗದಿಂದ ಮತ್ತೊಂದು ಭಾಗಕ್ಕೆ, ಗ್ರಾಮ ಗ್ರಾಮಗಳಿಗೆ ಹೋಗಿ. ಸುಮ್ಮನೆ ಸೋಮಾರಿಗಳಂತೆ ಕುಳಿತುಕೊಂಡರೆ ಏನೂ ಫಲವಿಲ್ಲವೆಂದು ಜನರಿಗೆ ಬೋಧಿಸುವುದೇ ನಿಮ್ಮ ಈಗಿನ ಕರ್ತವ್ಯ. ಅವರಿಗೆ ತಮ್ಮ ನೈಜ ಸ್ಥಿತಿಯ ಅರಿವುಂಟಾಗುವಂತೆ ಮಾಡಿ. ‘ಓ! ನನ್ನ ಸಹೋದರರೆ, ಎಲ್ಲರೂ ಎದ್ದೇಳಿ, ಎಚ್ಚರಗೊಳ್ಳಿ, ಇನ್ನೆಷ್ಟು ಕಾಲ ನಿದ್ರಿಸುವಿರಿ!’ ಎಂದು ಬೋಧಿಸಿ. ಅವರ ಸ್ಥಿತಿ ಉತ್ತಮಗೊಳ್ಳಲು ಏನು ಮಾಡಬೇಕೆಂಬುದನ್ನು ತಿಳಿಯ ಹೇಳಿ. ಸ್ಪಷ್ಟವಾದ ಮತ್ತು ಅರ್ಥವಾಗುವ ಭಾಷೆಯಲ್ಲಿ ಅವರಿಗೆ ಶಾಸ್ತ್ರಗಳಲ್ಲಿರುವ ಮಹತ್ತಾದ ಸತ್ಯಗಳ ಪರಿಚಯ ಮಾಡಿಕೊಡಿ. ಇಲ್ಲಿಯವರೆಗೂ ಧರ್ಮವು ಬ್ರಾಹ್ಮಣ ಜಾತಿಗೆ ಮೀಸಲಾಗಿತ್ತು. ಆದರೆ ಪ್ರಬಲ ಕಾಲಪ್ರವಾಹದಲ್ಲಿ ಅವರು ಇಂದು ಅದೇ ಸ್ಥಳದಲ್ಲಿ ಇರಲಾರರು. ದೇಶದ ಪ್ರತಿಯೊಬ್ಬರೂ ಧರ್ಮವನ್ನರಿಯುವಂತೆ ಮಾಡಲು ತೊಡಗಿರಿ, ಬ್ರಾಹ್ಮಣರಷ್ಟೇ ತಮಗೂ ಹಕ್ಕಿದೆ ಎಂಬ ಭಾವನೆ ಅವರ ಮನಸ್ಸಿನಲ್ಲಿ ನಾಟುವಂತೆ ಮಾಡಿ. ಪ್ರತಿಯೊಬ್ಬರಿಗೂ, ಚಂಡಾಲರಿಗೂ ಈ ಪ್ರಜ್ವಲಿಸುವ ಮಂತ್ರದೀಕ್ಷೆ ಕೊಡಿ, ಸರಳ ಮಾತಿನಲ್ಲಿ ಅವರಿಗೆ ಜೀವನ, ವ್ಯಾಪಾರ, ವ್ಯವಹಾರ, ವ್ಯವಸಾಯ ಮುಂತಾದುವುಗಳ ಬಗ್ಗೆ ಶಿಕ್ಷಣ ಕೊಡಿ. ಇದನ್ನು ನೀನು ಮಾಡಲಶಕ್ತನಾದಲ್ಲಿ ನಿನ್ನ ವಿದ್ಯಾಭ್ಯಾಸ ಮತ್ತು ಸಂಸ್ಕೃತಿಗೆ ಧಿಕ್ಕಾರ! ನಿನ್ನ ವೇದ, ವೇದಾಂತಗಳ ವ್ಯಾಸಂಗಕ್ಕೆ ಧಿಕ್ಕಾರ!

ಶಿಷ್ಯ: ಆದರೆ ನಮಗೆ ಅಂತಹ ಶಕ್ತಿ ಎಲ್ಲಿದೆ? ನಿಮಗಿರುವುದರಲ್ಲಿ ನೂರರಲ್ಲಿ ಒಂದು ಪಾಲು ಶಕ್ತಿ ನನಗಿದ್ದರೆ ನಾನು ಧನ್ಯನಾಗುತ್ತಿದ್ದೆ.

ಸ್ವಾಮೀಜಿ: ಎಂತಹ ತಿಳಿಗೇಡಿತನ! ಶಕ್ತಿ ಮತ್ತು ವಸ್ತುಗಳು ತಮ್ಮಷ್ಟಕ್ಕೆ ತಾವೇ ಬರುವುವು. ನೀನು ಕೆಲಸಕ್ಕೆ ತೊಡಗು, ನಿನ್ನಲ್ಲಿ ಅದ್ಭುತ ಶಕ್ತಿ ಪ್ರವಹಿಸಿ ನಿನಗೆ ಅದನ್ನು ಸಹಿಸಿಕೊಳ್ಳುವುದಕ್ಕೆ ಆಗುವುದಿಲ್ಲವೆನ್ನಿಸುವುದು. ಇನ್ನೊಬ್ಬರಿಗಾಗಿ ಮಾಡುವ ಕೆಲಸ ಅದೆಷ್ಟೇ ಕ್ಷುದ್ರವಾಗಿರಲಿ, ನಿನ್ನ ಸುಪ್ತ ಶಕ್ತಿಯನ್ನು ಜಾಗೃತಗೊಳಿಸುವುದು. ಇತರರ ಕಲ್ಯಾಣಕ್ಕೆ ಕಿಂಚಿತ್ತು ನೀನು ಯೋಚಿಸಿದರೂ ಕೂಡ ಕ್ರಮೇಣ ಅದು ನಿನ್ನ ಹೃದಯದಲ್ಲಿ ಸಿಂಹಸದೃಶ ಶಕ್ತಿಯನ್ನುಂಟುಮಾಡುವುದು. ನಾನು ನಿಮ್ಮೆಲ್ಲರನ್ನೂ ನಿರಂತರ ಎಷ್ಟೊಂದು ಪ್ರೀತಿಸುವೆ. ಆದರೂ ನೀವೆಲ್ಲರೂ ಮತ್ತೊಬ್ಬರ ಹಿತಕ್ಕಾಗಿ ಸಾಯುವುದನ್ನು ಇಷ್ಟಪಡುವೆ. ನೀವು ಹಾಗೆ ಮಾಡುವುದರಿಂದ ನನಗೆ ಹೆಚ್ಚು ಸಂತೋಷವಾಗುತ್ತದೆ.

ಶಿಷ್ಯ: ಹಾಗಾದರೆ ನನ್ನನ್ನೇ ನೆಚ್ಚಿಕೊಂಡಿರುವವರ ಗತಿ ಏನು?

ಸ್ವಾಮೀಜಿ: ನೀನು ಇತರರಿಗಾಗಿ ನಿನ್ನ ಪ್ರಾಣವನ್ನರ್ಪಿಸಲು ಸಿದ್ಧನಾಗಿದ್ದರೆ ದೇವರು ಅವರಿಗೂ ಯಾವುದಾದರೊಂದು ಹಾದಿಯನ್ನು ತೋರಿಸುವನು. ನೀನು ಗೀತೆಯಲ್ಲಿ ಶ‍್ರೀಕೃಷ್ಣನ ಮಾತನ್ನು ಓದಿಲ್ಲವೆ? ‘ನನ್ನ ಪ್ರಿಯತಮ, ಒಳ್ಳೆಯ ಕೆಲಸ ಮಾಡಿದವನು ಎಂದಿಗೂ ದುಃಖಪಡುವುದಿಲ್ಲ.’

ಶಿಷ್ಯ: ಸರಿ, ಸ್ವಾಮಿಜಿ.

ಸ್ವಾಮೀಜಿ: ಮುಖ್ಯವಾದುದು ತ್ಯಾಗ, ತ್ಯಾಗಿಯಲ್ಲದವನು ಮತ್ತೊಬ್ಬರಿಗಾಗಿ ಹೃತ್ಪೂರ್ವಕ ಸೇವೆ ಮಾಡಲಾಗುವುದಿಲ್ಲ. ತ್ಯಾಗಿ ಎಲ್ಲರನ್ನೂ ಒಂದೇ ಸಮನಾಗಿ ನೋಡುವನು. ಎಲ್ಲರ ಸೇವೆಗೂ ಸಿದ್ಧನಾಗಿರುವನು. ನಮ್ಮ ವೇದಾಂತವೂ ಎಲ್ಲರನ್ನೂ ಸಮದೃಷ್ಟಿಯಿಂದ ಕಾಣಬೇಕೆಂದು ಹೇಳುವುದಿಲ್ಲವೆ? ಹಾಗಾದ ಮೇಲೆ ನೀನೇಕೆ ನಿನ್ನ ಹೆಂಡತಿ ಮಕ್ಕಳು ಇತರರೆಲ್ಲರಿಗಿಂತ ಹೆಚ್ಚಾಗಿ ನಿನಗೆ ಸೇರಿದವರು ಎಂಬ ಭಾವನೆಯನ್ನು ಇನ್ನೂ ಹೊಂದಿದ್ದೀಯೇ? ನಿನ್ನ ಮನೆ ಅಂಗಳದಲ್ಲೇ ನಾರಾಯಣನು ಬಡ ಭಿಕ್ಷುಕನ ರೂಪದಲ್ಲಿ ಉಪವಾಸದಿಂದ ಸಾಯುತ್ತಿದ್ದಾನೆ. ಅವನಿಗೇನನ್ನಾದರೂ ನೀಡುವ ಬದಲು ನಿನ್ನ ಹೆಂಡತಿ ಮಕ್ಕಳನ್ನು ಮಾತ್ರ ಮೃಷ್ಟಾನ್ನದಿಂದ ತೃಪ್ತಿಪಡಿಸುತ್ತೀಯೇನು? ಅದು ಮೃಗೀಯ ರೀತಿ!

ಶಿಷ್ಯ: ಇತರರಿಗಾಗಿ ಕೆಲಸ ಮಾಡಲು ಕೆಲವು ವೇಳೆ ಬಹಳ ಹಣ ಬೇಕಾಗುವುದು. ಅದನ್ನು ನಾನು ಪಡೆಯುವುದು ಹೇಗೆ?

ಸ್ವಾಮೀಜಿ: ನಿನ್ನ ಕೈಯಲ್ಲಿ ಎಷ್ಟಾಗುವುದೊ ಅಷ್ಟನ್ನಾದರೂ ಮಾಡಬಾರದೇಕೆ? ದುಡ್ಡಿಲ್ಲದೆ ಅವರಿಗೆ ನೀನೇನನ್ನೂ ಕೊಡಲಾಗದೆ ಹೋದಲ್ಲಿ ಅವರಿಗೆ ಸ್ವಲ್ಪ ಒಳ್ಳೆಯ ಮಾತನ್ನು ಒಳ್ಳೆಯ ಬೋಧನೆಯನ್ನು ಹೇಳಬಾರದೇಕೆ? ದುಡ್ಡಿಲ್ಲದೆ ಅವರಿಗೆ ನೀನೇನನ್ನೂ ಹೇಳಲಾಗದೆ ಅಥವಾ ಅದಕ್ಕೂ ದುಡ್ಡು ಬೇಕೇನು?

ಶಿಷ್ಯ: ಹೌದು ಸ್ವಾಮೀಜಿ, ಅದನ್ನು ನಾನು ಮಾಡಬಲ್ಲೆ.

ಸ್ವಾಮೀಜಿ: ಆದರೆ ಕೇವಲ ‘ನಾನು ಮಾಡಬಲ್ಲೆ’ ಎನ್ನುವುದರಿಂದ ಪ್ರಯೋಜನವಿಲ್ಲ. ನೀನೇನು ಮಾಡಬಲ್ಲೆ ಎಂಬುದನ್ನು ಕಾರ್ಯತಃ ತೋರಿಸು. ಆಗ ಮಾತ್ರ ನೀನು ನನ್ನಲ್ಲಿಗೆ ಬಂದುದಕ್ಕೆ ಸಾರ್ಥಕವೆಂದು ತಿಳಿಯುವೆ. ಎದ್ದೇಳು, ಚಕ್ರಕ್ಕೆ ನಿನ್ನ ಹೆಗಲನ್ನು ಕೊಡು - ಈ ಜೀವನ ಎಷ್ಟು ದಿನ ಇರಬಲ್ಲದು? ನೀನೀ ಜಗತ್ತಿಗೆ ಬಂದುದಕ್ಕೆ ಒಂದಾದರೂ ಗುರುತನ್ನು ಬಿಟ್ಟುಹೋಗು. ಇಲ್ಲದಿದ್ದಲ್ಲಿ ನಿನಗೂ ಮರ ಕಲ್ಲುಗಳಿಗೂ ಏನು ವ್ಯತ್ಯಾಸ? ಅವೂ ಹುಟ್ಟುತ್ತವೆ, ಬೆಳೆಯುತ್ತವೆ, ಸಾಯುತ್ತವೆ. ನೀನೂ ಅದೇ ರೀತಿ ಹುಟ್ಟಿ ಸಾಯಬೇಕೆಂದು ಬಯಸಿದಲ್ಲಿ ನೀನೂ ಹಾಗೇ ಮಾಡಬಹುದು. ವೇದಾಂತವನ್ನೋದಿದುದರಿಂದ ಒಳ್ಳೆಯ ಫಲ ದೊರಕಿತೆಂದು ನಿನ್ನ ಕಾರ್ಯದಿಂದ ಮಾಡಿ ತೋರಿಸು. ಹೋಗು, ಎಲ್ಲರಿಗೂ ಹೇಳು, ‘ನಿಮ್ಮೆಲ್ಲರಲ್ಲೂ ಆ ಅನಂತ ಶಕ್ತಿ ಹುದುಗಿದೆ’ ಎಂದು. ಅದನ್ನು ಜಾಗೃತಗೊಳಿಸಲು ಯತ್ನಿಸು. ಕೇವಲ ಒಬ್ಬ ವ್ಯಕ್ತಿಯ ಮೋಕ್ಷದಿಂದೇನಾಗುವುದು? ಅದು ಕೇವಲ ಸ್ವಾರ್ಥ. ನಿನ್ನ ಧ್ಯಾನವನ್ನು ಒಂದು ಕಡೆ ಕಟ್ಟಿಡು. ನಿನ್ನ ಮೋಕ್ಷವೇ ಮುಂತಾದುವನ್ನು ದೂರ ಬಿಸುಡು. ನಾನು ದೀಕ್ಷೆಗೊಂಡಿರುವ ಈ ಕೆಲಸಕ್ಕೆ ಮನಃಪೂರ್ವಕ ತೊಡಗು.

ಈ ಸ್ಫೂರ್ತಿಯುತ ನುಡಿಗಳನ್ನು ಉಸಿರುಬಿಡದೆ ಶಿಷ್ಯ ಆಲಿಸಿದನು. ಸ್ವಾಮೀಜಿಯವರು ತಮ್ಮ ಸ್ವಾಭಾವಿಕವಾದ, ನಿರರ್ಗಳ, ಸ್ಫೂರ್ತಿದಾಯಕ ಬೋಧೆಯನ್ನು ಮುಂದುವರಿಸಿದರು.

ಸ್ವಾಮೀಜಿ: ಮೊದಲು ಭೂಮಿಯನ್ನು ಹದಗೊಳಿಸು, ಸಾವಿರಾರು ವಿವೇಕಾನಂದರು ಧರ್ಮ ಬೋಧಿಸಲು ಈ ಪ್ರಪಂಚದಲ್ಲಿ ಹುಟ್ಟುವರು. ನೀನೇನೂ ಆ ವಿಚಾರಕ್ಕೆ ಯೋಚಿಸಬೇಕಾದ್ದಿಲ್ಲ! ಅನಾಥಾಲಯ, ಕ್ಷಾಮ ನಿವಾರಣಾ ಕೆಲಸಗಳು ಮುಂತಾದುವನ್ನು ನಾನೇಕೆ ಪ್ರಾರಂಭಿಸುತ್ತಿರುವೆನೆಂಬುದನ್ನು ನೀನು ನೋಡುತ್ತಿಲ್ಲವೆ? ಆಂಗ್ಲ ಮಹಿಳೆ ಸಹೋದರಿ ನಿವೇದಿತಾ ಹಿಂದೂಗಳ ಸೇವೆ ಮಾಡುವುದನ್ನು ಹಾಗೂ ಎಂತಹ ಕೀಳು ಕೆಲಸವನ್ನಾಗಲಿ ಮಾಡಲು ಎಷ್ಟರಮಟ್ಟಿಗೆ ಕಲಿತಿದ್ದಾಳೆ! ಹಿಂದೂದೇಶೀಯರಾದ ನೀವೇ ನಿಮ್ಮ ಜನರಿಗೆ ಸೇವೆ ಮಾಡಲಾಗುವುದಿಲ್ಲವೆ? ಎಲ್ಲೆಲ್ಲಿ ಪ್ಲೇಗ್ ಅಥವಾ ಕ್ಷಾಮವಿದೆಯೊ, ಎಲ್ಲೆಲ್ಲಿ ಜನರು ಕೊಲೆಗೆ ಗುರಿಯಾಗಿದ್ದಾರೆ ಅಲ್ಲಿಗೆ ನೀವೆಲ್ಲಾ ಹೋಗಿ ಅವರ ನೋವನ್ನು ಕಡಿಮೆ ಮಾಡಲು ಯತ್ನಿಸಿ. ಹೆಚ್ಚೆಂದರೆ ನೀನೀ ಕೆಲಸದಲ್ಲಿ ಸಾಯಬಹುದು - ಸತ್ತರೇನಂತೆ? ಪ್ರತಿದಿನವೂ ನಿಮ್ಮಲ್ಲಿ ಎಷ್ಟು ಜನ ಹುಟ್ಟಿ ಹುಳುಗಳಂತೆ ಸಾಯುತ್ತಿಲ್ಲ! ಪ್ರಪಂಚಕ್ಕೆ ಇದರಿಂದಾಗುವ ನಷ್ಟವೇನು? ನೀನು ಸತ್ತೇ ತೀರಬೇಕು. ಆದರೆ ಸಾಯಲು ಒಂದು ಘನ ಉದ್ದೇಶವಿರಲಿ. ಜೀವನದಲ್ಲಿ ಒಂದು ಘನ ಉದ್ದೇಶಕ್ಕಾಗಿ ಸಾಯುವುದೊಳ್ಳೆಯದು. ಪ್ರತಿ ಮನೆ ಮನೆಯ ಬಾಗಿಲಿಗೂ ಹೋಗಿ ಇದನ್ನು ಬೋಧಿಸಿ. ಇದರಿಂದ ನಿಮಗೂ ಒಳ್ಳೆಯದಾಗುವುದಲ್ಲದೆ ದೇಶ ಸೇವೆಯನ್ನೂ ಮಾಡಿದಂತಾಗುವುದು. ನಮ್ಮ ದೇಶದ ಭವಿಷ್ಯದ ಆಸೆ ನಿಮ್ಮನ್ನೇ ಅವಲಂಬಿಸಿದೆ. ನೀವು ಈ ರೀತಿ ಜೀವನ ನಡೆಸುವುದನ್ನು ನೋಡಿ ನನಗೆ ತುಂಬಾ ದುಃಖವಾಗುತ್ತದೆ - ಕೆಲಸಕ್ಕೆ ಹೊರಡಿ, ಹೊರಡಿ! ಹಿಂದೆ ನಿಲ್ಲಬೇಡಿ! ದಿನ ಕಳೆದಂತೆ ಸಾವು ಸಮೀಪಿಸುತ್ತಿದೆ. ಸೋಮಾರಿತನದಿಂದ ಎಲ್ಲವೂ ಕಾಲಕ್ರಮೇಣ ಸರಿಹೋಗುವುದೆಂದು ಸುಮ್ಮನೆ ಕುಳಿತಿರಬೇಡಿ! ನೀವು ಹೀಗೆ ಮಾಡುವುದರಿಂದ ಯಾವ ಕೆಲಸವೂ ಆಗುವುದಿಲ್ಲ.

\newpage

\chapter[ಅಧ್ಯಾಯ ೨೩]{ಅಧ್ಯಾಯ ೨೩\protect\footnote{ಒಮ್ಮೆ ರಾಮ ಮತ್ತು ಶಿವ ಜಗಳವಾಡಿದರು. ಶಿವನು ರಾಮನ ಗುರು, ರಾಮನು ಶಿವನ ಗುರು. ಜಗಳದ ನಂತರ ಅವರು ಮತ್ತೆ ಮೊದಲಿನಂತೆ ಮಿತ್ರರಾದರು. ಆದರೆ ಶಿವನ ಗಣಗಳಿಗೂ ರಾಮನ ಕಪಿಗಳಿಗೂ ಹತ್ತಿದ ಜಗಳ ಹರಿಯಲೇ ಇಲ್ಲ!}}

\begin{center}
ಸ್ಥಳ: ಬೇಲೂರು ಮಠ ಕಟ್ಟುತ್ತಿದ್ದಾಗ, ವರ್ಷ: ಕ್ರಿ.ಶ. ೧೮೯೮.
\end{center}

ಶಿಷ್ಯ: ಸ್ವಾಮಿಜಿ, ಜ್ಞಾನ ಮತ್ತು ಭಕ್ತಿಯನ್ನು ಒಂದುಗೂಡಿಸಲು ಹೇಗೆ ಸಾಧ್ಯ? ಭಕ್ತಿಮಾರ್ಗಾನುಯಾಯಿಗಳು ಶಂಕರನ ನಾಮವನ್ನು ಕೇಳಿದ ಕೂಡಲೇ ತಮ್ಮ ಕಿವಿಗಳನ್ನು ಮುಚ್ಚಿಕೊಳ್ಳುತ್ತಾರೆ. ಜ್ಞಾನಮಾರ್ಗಾವಲಂಬಿಗಳು ಭಗವನ್ನಾಮವನ್ನುಚ್ಚರಿಸುತ್ತ ಪ್ರವಾಹದೋಪಾದಿ ಕಣ್ಣೀರು ಸುರಿಸಿಕೊಂಡು ಮೈಮರೆತು ಹಾಡುತ್ತ ಕುಣಿಯುವ ಭಕ್ತರನ್ನು ಧರ್ಮಾಂಧರೆಂದು ಕರೆಯುವರು.

ಸ್ವಾಮೀಜಿ: ಈ ವೈಮನಸ್ಯವೆಲ್ಲಾ ಜ್ಞಾನ ಮತ್ತು ಭಕ್ತಿಯ ಪ್ರಥಮ ಘಟ್ಟದಲ್ಲಿ ಮಾತ್ರ ಇರುವುದು. ನೀನು ಶ‍್ರೀರಾಮಕೃಷ್ಣರು ಹೇಳುತ್ತಿದ್ದ ಶಿವನ ಭೂತಗಣ ಮತ್ತು ರಾಮನ ಕಪಿಗಳ ತಂಡದ ಕಥೆಯನ್ನು ಕೇಳಿಲ್ಲವೆ?**

ಶಿಷ್ಯ: ಕೇಳಿದ್ದೇನೆ ಸ್ವಾಮಿಜಿ.

ಸ್ವಾಮೀಜಿ: ಪೂರ್ಣ ಜ್ಞಾನಕ್ಕೂ, ಪೂರ್ಣ ಭಕ್ತಿಗೂ ಏನೂ ವ್ಯತ್ಯಾಸವಿಲ್ಲ. ಪರಾಭಕ್ತಿಯೇ ದೇವರನ್ನು ಪ್ರೇಮಸ್ವರೂಪನೆಂದು ಸಾಕ್ಷಾತ್ಕರಿಸಿಕೊಳ್ಳುವುದು. ಎಲ್ಲೆಲ್ಲಿಯೂ ಎಲ್ಲದರಲ್ಲಿಯೂ ಆ ಪ್ರೇಮರೂಪಿ ಭಗವಂತನ ಆವಿರ್ಭಾವವನ್ನು ನೀನು ನೋಡಿದರೆ ಇತರರನ್ನು ನೋಯಿಸಲು ಹೇಗೆ ಸಾಧ್ಯ? ಮನಸ್ಸಿನಲ್ಲಿ ಕಿಂಚಿತ್ತು ಆಸೆ ಇರುವವರೆಗೂ, ಶ‍್ರೀರಾಮಕೃಷ್ಣರು ಹೇಳುತ್ತಿದ್ದ ‘ಕಾಮಿನಿ ಕಾಂಚನ’ದ ಮೋಹ ಕಿಂಚಿತ್ತಿದ್ದರೂ ಪ್ರೇಮದ ಸಾಕ್ಷಾತ್ಕಾರವೆಂದಿಗೂ ಲಭಿಸುವುದಿಲ್ಲ. ಪೂರ್ಣ ಪ್ರೇಮದರ್ಶನದಲ್ಲಿ ನಮ್ಮ ದೇಹದ ಅರಿವೇ ಇರುವುದಿಲ್ಲ. ಎಲ್ಲೆಲ್ಲಿಯೂ ಏಕತ್ವವನ್ನು ನೋಡುವುದು, ಎಲ್ಲದರಲ್ಲಿಯೂ ನಿನ್ನ ಆತ್ಮವನ್ನು ನೋಡುವುದು, ಇದೇ ಪೂರ್ಣ ಜ್ಞಾನ. ಅದೂ ಕೂಡ, ಅಹಂಭಾವ ಕಿಂಚಿತ್ತು ಇರುವವರೆಗೂ ಬರುವುದಿಲ್ಲ.

ಶಿಷ್ಯ: ಹಾಗಾದರೆ ಪ್ರೇಮವೆಂದು ಹೇಳುವುದೂ, ಪೂರ್ಣಜ್ಞಾನ ಎರಡೂ ಒಂದೆಯೇ?

ಸ್ವಾಮೀಜಿ: ಹೌದು, ಸಂಪೂರ್ಣವಾಗಿ ಹೌದು. ಪೂರ್ಣಜ್ಞಾನಿಯಾಗುವವರೆಗೂ ಪ್ರೇಮ ಸಾಕ್ಷಾತ್ಕಾರವಾಗುವುದಿಲ್ಲ. ಬ್ರಹ್ಮವು ಸತ್-ಚಿತ್-ಆನಂದ, ಎಂದು ವೇದಾಂತ ಸಾರುವುದಿಲ್ಲವೇ?

ಶಿಷ್ಯ: ಹೌದು ಸ್ವಾಮಿಜಿ.

ಸ್ವಾಮೀಜಿ: ಸತ್-ಚಿತ್-ಆನಂದ ಇದೂ ಪ್ರೇಮವೆ. ಬ್ರಹ್ಮನ ಸತ್ ಸ್ವಭಾವದಲ್ಲಿ ಭಕ್ತನಿಗೂ, ಜ್ಞಾನಿಗೂ ಏನೊಂದೂ ಭೇದವಿಲ್ಲ; ಕೇವಲ ಜ್ಞಾನಿಯು ಚಿತ್ ಸ್ವಭಾವಕ್ಕೆ ಹೆಚ್ಚು ಪ್ರಾಮುಖ್ಯ ಕೊಡುತ್ತಾನೆ. ಭಕ್ತರು ಆನಂದ ಪ್ರೇಮಕ್ಕೆ ಹೆಚ್ಚು ಪ್ರಾಮುಖ್ಯ ಕೊಡುವರು. ಆದರೆ ಯಾವಾಗ ಚಿತ್ ಸಾಕ್ಷಾತ್ಕಾರವಾಗುವುದೊ ಆಗ ಆನಂದದ ಸಾಕ್ಷಾತ್ಕಾರವೂ ಆಗುವುದು. ಏಕೆಂದರೆ ಚಿತ್ ಮತ್ತು ಆನಂದ ಎರಡೂ ಒಂದೇ.

ಶಿಷ್ಯ: ಹಾಗಾದರೆ ಭರತಖಂಡದಲ್ಲೇಕೆ ಇಷ್ಟೊಂದು ಪಂಗಡಗಳು ಹರಡಿವೆ? ಶಾಸ್ತ್ರದಲ್ಲಿ ಭಕ್ತಿ, ಜ್ಞಾನದ ಬಗ್ಗೆ ಏಕೆ ಇಷ್ಟೊಂದು ಭೇದವಿದೆ?

ಸ್ವಾಮೀಜಿ: ನಿಜವಾದ ಭಕ್ತಿ ಮತ್ತು ಜ್ಞಾನವನ್ನು ಹೊಂದಲು ಜನರಿಗಿರಬೇಕಾದ ಪ್ರಥಮ ಸೋಪಾನಗಳ ವಿಚಾರವಾಗಿ ಇಷ್ಟೊಂದು ವಾಗ್ವಾದ ಜಗಳಗಳಿವೆ. ನೀನು ಇದರಲ್ಲಿ ಯಾವುದನ್ನು ದೊಡ್ಡದೆಂದು ಯೋಚಿಸುವೆ? ಅದೇ ಗುರಿಯೆ ಅಥವಾ ಅದನ್ನು ಮುಟ್ಟಲು ಅದು ಒಂದು ಸಾಧನವೆ? ಏಕೆಂದರೆ ಒಂದೇ ಗುರಿಗೆ ಪ್ರತಿಯೊಬ್ಬ ವ್ಯಕ್ತಿಯ ಮಾನಸಿಕ ಸಾಮರ್ಥ್ಯಕ್ಕನುಸಾರವಾಗಿ ಹಲವಾರು ಮಾರ್ಗಗಳಿವೆ. ಜಪಮಾಲೆಯನ್ನೆಣಿಸುವುದು, ಧ್ಯಾನ, ಪೂಜೆ, ಹೋಮಾಗ್ನಿಯಲ್ಲಿ ಅರ್ಪಿಸುವ ನೈವೇದ್ಯ ಮುಂತಾದುವೆಲ್ಲ ಧರ್ಮ ಸೋಪಾನಗಳು. ಅವೆಲ್ಲಾ ಕೇವಲ ಸಾಧನ ಮಾತ್ರ. ಪರಾಭಕ್ತಿ ಅಥವಾ ಪರಬ್ರಹ್ಮನ ಸಾಕ್ಷಾತ್ಕಾರವೇ ಸರ್ವೋತ್ಕೃಷ್ಟವಾದ ಗುರಿ. ನೀನಿನ್ನೂ ಗಾಢವಾಗಿ ಯೋಚಿಸಿದರೆ ಅವರು ಯಾವುದಕ್ಕೆ ಹೊಡೆದಾಡುತ್ತಿರುವರೆಂಬುದು ನಿನಗೇ ಅರ್ಥವಾಗುವುದು. ಒಬ್ಬ ಹೇಳುತ್ತಾನೆ ‘ನೀನು ಪೂರ್ವಾಭಿಮುಖವಾಗಿ ನಿಂತು ದೇವರನ್ನು ಪ್ರಾರ್ಥಿಸಿದರೆ ದೇವರನ್ನು ನೋಡುವೆ.’ ಇನೊಬ್ಬ ಹೇಳುತ್ತಾನೆ ‘ಇಲ್ಲ, ನೀನು ಪಶ್ಚಿಮದ ಕಡೆ ತಿರುಗಿ ಕುಳಿತುಕೊಳ್ಳಬೇಕು. ಆಗ ಮಾತ್ರ ನೀನು ದೇವರನ್ನು ನೋಡುವೆ.’ ಬಹುಶಃ ಯಾರೋ ಯುಗಾಂತರಗಳ ಹಿಂದೆ ಪೂರ್ವಾಭಿಮುಖವಾಗಿ ಕುಳಿತು ಧ್ಯಾನ ಮಾಡುತ್ತಾ ಸಾಕ್ಷಾತ್ಕಾರ ಹೊಂದಿರಬೇಕು. ಆತನ ಶಿಷ್ಯರು ಅಂದಿನಿಂದಲೂ ಯಾರು ಈ ಸ್ಥಿತಿಯಲ್ಲಿ ಕುಳಿತುಕೊಳ್ಳುವುದಿಲ್ಲವೋ ಅವರಿಗೆ ದೇವರ ದರ್ಶನವಾಗುವುದಿಲ್ಲವೆಂದು ಸಾರುತ್ತಾ ಹೊರಟಿರಬೇಕು. ಇನ್ನೊಂದು ಪಂಗಡ ಬಂದು ವಿಚಾರಿಸುತ್ತದೆ, ‘ಅದು ಹೇಗೆ ಸಾಧ್ಯ? ಪಶ್ಚಿಮಾಭಿಮುಖವಾಗಿ ಕುಳಿತು ಇಂಥವನೊಬ್ಬನು ದೇವರನ್ನು ಸಾಕ್ಷಾತ್ಕಾರ ಮಾಡಿಕೊಂಡಿದ್ದನ್ನು ನಾವೇ ಕಣ್ಣಾರೆ ನೋಡಿದ್ದೇವೆ’ ಎಂದು. ಹೀಗೆ ಈ ಪಂಗಡಗಳೆಲ್ಲಾ ಹುಟ್ಟಿಕೊಂಡಿವೆ. ಯಾರೋ ಹರಿನಾಮೋಚ್ಛಾರಣೆ ಮಾಡಿದುದರಿಂದ ಪರಮಭಕ್ತಿಯನ್ನು ಪಡೆದಿರಬಹುದು. ತಕ್ಷಣವೇ ಶಾಸ್ತ್ರದಲ್ಲಿ ಹೀಗೆ ಬಂತು:

\begin{verse}
ಹರೇರ್ನಾಮ ಹರೇರ್ನಾಮ ಹರೇರ್ನಾಮೈವ ಕೇವಲಮ್\\ಕಲೌ ನಾಸ್ತ್ಯೇವ ನಾಸ್ತ್ಯೇವ ನಾಸ್ತ್ಯೇವ ಗತಿರನ್ಯಥಾ~॥
\end{verse}

“ಹರಿನಾಮ, ಹರಿನಾಮ, ಕೇವಲ ಹರಿನಾಮ ಮಾತ್ರ ಈ ಕಲಿಯುಗದಲ್ಲಿ ಫಲಕಾರಿ. ಬೇರಾವ ಹಾದಿಯೂ ಇಲ್ಲವೇ ಇಲ್ಲ."

“ಪುನಃ ಮತ್ತೊಬ್ಬರು ಅಲ್ಲಾನ ಹೆಸರನ್ನುಚ್ಚರಿಸುತ್ತಾ ಪರಮ ಪದವಿಯನ್ನೈದಿರಬಹುದು. ತಕ್ಷಣವೇ ಅವನಿಂದ ಪ್ರಾರಂಭವಾದ ಮತ್ತೊಂದು ಪಂಗಡ ಹರಡಿತು. ಆದರೆ ನಾವು ಈ ಪೂಜಾ ವಿಧಾನಗಳು, ಧರ್ಮಸಾಧನೆಗಳು ಎಲ್ಲಿಗೆ ಒಯ್ಯುತ್ತವೆ, ಅವುಗಳ ಗುರಿ ಯಾವುದು ಎಂಬುದರ ಕಡೆ ಹೆಚ್ಚು ಗಮನ ಕೊಡಬೇಕು. ಗುರಿಯೇ ಶ್ರದ್ಧೆ. ಸಂಸ್ಕೃತದ ಶ್ರದ್ಧೆ ಎಂಬ ಪದಕ್ಕೆ ಬಂಗಾಳಿಯಲ್ಲಿ ಸರಿಯಾದ ಪರ್ಯಾಯ ಪದವೇ ಇಲ್ಲ. ನಚಿಕೇತನ ಹೃದಯಕ್ಕೆ ಶ್ರದ್ಧೆ ಹೊಕ್ಕಿತು ಎಂದು ಉಪನಿಷತ್ತುಗಳು ಹೇಳುತ್ತವೆ. ಏಕಾಗ್ರತೆ ಎಂಬ ಪದವೂ ಶ್ರದ್ಧೆ ಎಂಬ ಪದದ ಅರ್ಥವನ್ನು ಕೊಡುವುದಿಲ್ಲ. ಏಕಾಗ್ರನಿಷ್ಠ ಎಂದರೆ ಸ್ವಲ್ಪ ಮಟ್ಟಿಗೆ ಶ್ರದ್ಧೆ ಎಂಬ ಪದದ ಅರ್ಥ ಬರುತ್ತದೆ. ಯಾವ ಸತ್ಯವನ್ನಾದರೂ ಅಚಲಭಕ್ತಿ ಮತ್ತು ಚಿತ್ತೈಕಾಗ್ರತೆಯಿಂದ ಧ್ಯಾನಿಸಿದರೆ ಮನಸ್ಸು ಏಕಾಗ್ರತೆಯ ಕಡೆ ಹೆಚ್ಚು ವಾಲುತ್ತದೆ. ಅಖಂಡ ಸಚ್ಚಿದಾನಂದದ ಸಾಕ್ಷಾತ್ಕಾರವಾಗುತ್ತದೆ. ಭಕ್ತಿ ಮತ್ತು ಜ್ಞಾನದ ಮೇಲಿನ ಶಾಸ್ತ್ರಗಳು ಅಂತಹ ನಿಷ್ಠೆಯಿಂದ ಯಾವುದಾದರೊಂದು ದಾರಿಯನ್ನು ಹಿಡಿಯುವಂತೆ ಪ್ರತ್ಯೇಕ ಸಲಹೆಗಳನ್ನು ಕೊಡುವುವು. ಕಾಲ ಕಳೆದಂತೆಲ್ಲಾ ಈ ಮಹಾ ಸತ್ಯಗಳು ವಿರೂಪ ಹೊಂದಿ ಕ್ರಮೇಣ ದೇಶಾಚಾರಗಳಾಗಿ ಮಾರ್ಪಟ್ಟಿವೆ. ಇದು ನಮ್ಮ ಭರತಖಂಡದಲ್ಲಿ ಮಾತ್ರವೇ ಅಲ್ಲ, ಪ್ರಪಂಚದ ಪ್ರತಿಯೊಂದು ಸಮಾಜದಲ್ಲೂ ನಡೆದಿದೆ. ಸಾಮಾನ್ಯ ಜನರು ವಿಚಾರ ಮಾಡದೆ, ಇದನ್ನೇ ಹೋರಾಟದ ವಸ್ತುವಾಗಿ ಮಾಡಿಕೊಂಡು ತಮ್ಮ ತಮ್ಮಲ್ಲೇ ಹೊಡೆದಾಡುತ್ತಾರೆ. ಅವರಿಗೆ ಗುರಿಯೇ ಮರೆತುಹೋಗಿದೆ. ಅದಕ್ಕೇ ಈ ಮತಭೇದ, ಜಗಳ, ಹೊಡೆದಾಟ.”

ಶಿಷ್ಯ: ಇದರಿಂದ ಪಾರಾಗುವ ಬಗೆ, ಸ್ವಾಮೀಜಿ?

ಸ್ವಾಮೀಜಿ: ಆ ನಿಜವಾದ ಶ್ರದ್ಧೆ ಹಿಂದಿನಂತೆ ಈಗ ನಮಗೆ ಬರಬೇಕು. ಕಳೆಯನ್ನೆಲ್ಲಾ ಬೇರುಸಹಿತ ಕೀಳಬೇಕು. ಪ್ರತಿಯೊಂದು ಮತ, ಪ್ರತಿಯೊಂದು ಪಂಥದಲ್ಲೂ ಕಾಲ ದೇಶಗಳನ್ನು ಮೀರಿದ ಸತ್ಯಗಳಿವೆ ಎಂಬುದೇನೋ ಖಂಡಿತವಾಗಿಯೂ ನಿಜ. ಆದರೆ ಅದರ ಮೇಲೆ ಕಲ್ಮಷ ತುಂಬಿಕೊಂಡಿದೆ. ಅವುಗಳನ್ನು ತೊಳೆದು ಶುಚಿ ಮಾಡಬೇಕು. ಜನರ ಮುಂದೆ ಸತ್ಯವಾದ ಶಾಶ್ವತವಾದ ಸತ್ಯಗಳನ್ನು ಇಡಬೇಕು. ಆಗ ಮಾತ್ರ ನಮ್ಮ ದೇಶ ನಮ್ಮ ಧರ್ಮ ಉದ್ಧಾರವಾಗುವುದು.

ಶಿಷ್ಯ: ಅದನ್ನು ಹೇಗೆ ಕಾರ್ಯರೂಪಕ್ಕೆ ತರುವುದು?

ಸ್ವಾಮಿಜಿ: ಏಕೆ? ನಾವು ಮೊದಲು ಮಹಾತ್ಮರಾದ ಸಾಧುಗಳ ಪೂಜೆಯನ್ನು ಆರಂಭಿಸಬೇಕು. ಅವರು ಭರತಖಂಡದಲ್ಲಿ ಶಾಶ್ವತವಾಗಿರುವ ಸತ್ಯವನ್ನು ಪ್ರತ್ಯಕ್ಷ ಮಾಡಿಕೊಂಡರು. ಶ‍್ರೀರಾಮಚಂದ್ರ, ಶ‍್ರೀಕೃಷ್ಣ, ಮಹಾವೀರ, ಶ‍್ರೀರಾಮಕೃಷ್ಣ ಮುಂತಾದ ಪ್ರಖ್ಯಾತ ಮಹಾತ್ಮರೇ ನಮ್ಮ ಗುರಿಯಾಗಿರಬೇಕೆಂದು ಜನರಿಗೆ ತಿಳಿಸಬೇಕು. ಶ‍್ರೀರಾಮಚಂದ್ರನಂತಹ ಮಹಾವೀರರ ಪೂಜೆಯನ್ನು ಈ ದೇಶದಲ್ಲಿ ಜಾರಿಗೆ ತರಬಲ್ಲೆಯಾ? ಶ‍್ರೀಕೃಷ್ಣನ ಬೃಂದಾವನದ ಲೀಲೆಯನ್ನು ಈಗ ಕಟ್ಟಿಡು ಸಿಂಹವಾಣಿಯಲ್ಲಿ ಗೀತೆಯನ್ನು ಘೋಷಿಸಿದ ಆ ಶ‍್ರೀಕೃಷ್ಣನ ಪೂಜೆಯನ್ನು ಎಲ್ಲೆಡೆಯಲ್ಲೂ ಹರಡು. ಪ್ರತಿದಿನ ಶಕ್ತಿ ಪೂಜೆಯನ್ನು ಶಕ್ತಿ ಸ್ವರೂಪಿಣಿಯಾದ ಮಹಾಮಾತೆಯ ಪೂಜೆಯನ್ನು ಅನುಷ್ಠಾನಕ್ಕೆ ತನ್ನಿ.

ಶಿಷ್ಯ: ಹಾಗಾದರೆ ಬೃಂದಾವನದ ಗೋಪಿಗಳೊಡನಿದ್ದ ಶ‍್ರೀಕೃಷ್ಣನ ದಿವ್ಯಲೀಲೆ ಒಳ್ಳೆಯದಲ್ಲವೆ?

ಸ್ವಾಮೀಜಿ: ಈಗಿನ ಸಂದರ್ಭದಲ್ಲಿ ನಿಮಗೆ ಆ ಪೂಜೆಯು ಒಳ್ಳೆಯದಲ್ಲ. ಕೊಳಲೂದುವುದರಿಂದ ದೇಶವನ್ನು ಪುನರುಜ್ಜೀವನಗೊಳಿಸಲಾಗುವುದಿಲ್ಲ. ಯಾರ ನಾಡಿ ನಾಡಿಗಳಲ್ಲಿ ಅದ್ಭುತ ರಾಜಸ ಸ್ವಭಾವ ತುಂಬಿ ತುಳುಕಾಡುತ್ತಿದೆಯೊ, ಯಾರು ಸತ್ಯ ಸಾಕ್ಷಾತ್ಕಾರಕ್ಕಾಗಿ ತಮ್ಮ ಪ್ರಾಣವನ್ನೇ ಅರ್ಪಿಸಲು ಸಿದ್ಧರಾಗಿರುವರೊ, ಯಾರಿಗೆ ತ್ಯಾಗವೇ ಆಯುಧವಾಗಿದೆಯೊ, ಧರ್ಮವೆ ಖಡ್ಗವಾಗಿದೆಯೋ ಅಂತಹವರು ದೇಶಕ್ಕೆ ಬೇಕಾಗಿದೆ. ಜೀವನ ಸಂಗ್ರಾಮದಲ್ಲಿ ಹೋರಾಡಬಲ್ಲ ವೀರಸ್ವಭಾವದ ಮನುಷ್ಯ ನಮಗೀಗ ಬೇಕಾಗಿದೆ. ಜೀವನವನ್ನು ನಂದನವನದಂತೆ ನೋಡುವ ಪ್ರೇಮಿ ಬೇಕಾಗಿಲ್ಲ.

ಶಿಷ್ಯ: ಹಾಗಾದರೆ ಗೋಪಿಯರ ಧ್ಯೇಯವಾಗಿದ್ದ ಆ ಪ್ರೇಮದ ಹಾದಿ ತಪ್ಪೇ?

ಸ್ವಾಮೀಜಿ: ಹಾಗೆಂದವರಾರು? ಖಂಡಿತ ಹಾಗಲ್ಲ! ಅದು ಅತ್ಯಂತ ಶ್ರೇಷ್ಠವಾದ ಸಾಧನ. ವಿಷಯ ಸುಖ ಭೋಗಗಳಿಗೆ ಹೆಚ್ಚು ಮೋಹಗೊಂಡಿರುವ ಈ ಯುಗದಲ್ಲಿ ಆ ಮಹತ್ತಾದ ಗುರಿಯನ್ನು ಅರಿಯತಕ್ಕವರು ಅತಿ ವಿರಳ.

ಶಿಷ್ಯ: ಹಾಗಾದರೆ ಪ್ರಿಯತಮನಂತೆ, ಪತಿಯಂತೆ ದೇವರನ್ನು ಪೂಜಿಸುತ್ತಿರುವವರೆಲ್ಲ ಸರಿಯಾದ ಹಾದಿಯಲ್ಲಿ ಹೋಗುತ್ತಿಲ್ಲವೆ!

ಸ್ವಾಮಾಜಿ: ನಾನು ಇಲ್ಲವೆನ್ನಲಾರೆ. ಅವರಲ್ಲಿ ಕೊಂಚಮಂದಿ ಅಂತಹ ಗೌರವಕ್ಕೆ ಅರ್ಹರಾಗಿದ್ದಾರೆಯೇ ಹೊರತು ಬಹು ಮಂದಿ ಜಡ ತಾಮಸಿಕ ಸ್ವಭಾವದಿಂದ ತುಂಬಿರುವರು. ಅವರಲ್ಲಿ ಮುಕ್ಕಾಲುಪಾಲೆಲ್ಲಾ ಈ ಜಾಡ್ಯದಿಂದ ತುಂಬಿ, ಅತಿ ದುರ್ಬಲರಾಗಿದ್ದಾರೆ! ದೇಶವು ಉದ್ರಿಕ್ತವಾಗಬೇಕು! ಮಹಾವೀರನ ಪೂಜೆಯನ್ನು ತರಬೇಕು. ನಮ್ಮ ನಿತ್ಯ ಸಾಧನೆಯಲ್ಲಿ ಶಕ್ತಿ ಪೂಜೆಯೂ ಒಂದು ಅಂಗವಾಗಿರಬೇಕು. ಪ್ರತಿಯೊಂದು ಮನೆಯಲ್ಲೂ ಶ‍್ರೀರಾಮನ ಪೂಜೆ ನಡೆಯಬೇಕು. ಅಲ್ಲೇ ನಿನ್ನ ಏಳ್ಗೆ ಇರುವುದು, ದೇಶದ ಏಳ್ಗೆ ಇರುವುದು - ಅದಿಲ್ಲದೆ ಇನ್ನು ಯಾವ ಹಾದಿಯೂ ಇಲ್ಲ.

ಶಿಷ್ಯ: ಭಗವಾನ್ ಶ‍್ರೀರಾಮಕೃಷ್ಣರು ದೇವರ ನಾಮೋಚ್ಛಾರಣೆಯನ್ನು ಹೆಚ್ಚಾಗಿ ಮಾಡುತ್ತಿದ್ದರೆಂದು ಕೇಳಿರುವೆ.

ಸ್ವಾಮೀಜಿ: ಹೌದು, ಆದರೆ ಅವರ ಸ್ಥಿತಿಯೇ ಬೇರೆ - ಸಾಧಾರಣ ಮನುಷ್ಯರಿಗೂ ಅವರಿಗೂ ಯಾವ ಸಾದೃಶ್ಯವಿದೆ? ಅವರು ತಮ್ಮ ಜೀವನದಲ್ಲೇ ಎಲ್ಲಾ ಧರ್ಮಗಳ ಗುರಿಯನ್ನು ಸಾಧನೆ ಮಾಡಿ ಎಲ್ಲ ಮತಗಳು ಒಂದೇ ಸಾಕ್ಷಾತ್ಕಾರಕ್ಕೆ ಒಯ್ಯುತ್ತವೆ ಎಂಬುದನ್ನು ತೋರಿಸಿದರು. ಅವರು ಮಾಡಿದುದನ್ನು ನಾನಾಗಲಿ, ನೀನಾಗಲಿ ಎಂದಾದರೂ ಮಾಡುವುದಕ್ಕೆ ಸಾಧ್ಯವೆ? ನಮ್ಮಲ್ಲಿ ಯಾರೂ ಅವರನ್ನು ಸಂಪೂರ್ಣವಾಗಿ ಅರ್ಥಮಾಡಿಕೊಂಡಿಲ್ಲ. ಅದಕ್ಕೇ ನಾನೂ ಎಲ್ಲೆಂದರಲ್ಲಿ ಅವರ ವಿಚಾರ ಮಾತನಾಡಲು ಹಿಂಜರಿಯುವೆನು. ಕೇವಲ ಅವರಿಗೆ ಮಾತ್ರ ಅವರ ನಿಜಸ್ಥಿತಿ ಗೊತ್ತು. ಅವರ ದೇಹ ಕೇವಲ ಹೊರಗಡೆ ನಮ್ಮಂತಿತ್ತು. ಉಳಿದುದೆಲ್ಲಾ ನಮಗಿಂತ ಪೂರ್ತಿ ಬೇರೆ.

ಶಿಷ್ಯ: ಅವರು ಅವತಾರಪುರುಷರೆಂದು ನಂಬುವಿರಾ?

ಸ್ವಾಮೀಜಿ: ಅವತಾರ ಎಂದರೆ ನೀನು ಏನು ಅರ್ಥಮಾಡಿಕೊಂಡಿರುವೆ?

ಶಿಷ್ಯ: ಶ‍್ರೀರಾಮಚಂದ್ರ, ಶ‍್ರೀಕೃಷ್ಣ, ಶ‍್ರೀಗೌರಾಂಗ, ಬುದ್ಧ ಅಥವಾ ಏಸು ಇವರ ಹಾಗೆ.

ಸ್ವಾಮೀಜಿ: ನೀನೀಗ ಹೇಳಿದ ಎಲ್ಲರಿಗಿಂತ ಶ‍್ರೀರಾಮಕೃಷ್ಣರು ಇನ್ನೂ ದೊಡ್ಡವರು. ಎಂದಮೇಲೆ ಇನ್ನು ನಾನು ನಂಬುವಂತಹ ಕ್ಷುದ್ರ ವಿಷಯವನ್ನೇನು ಹೇಳುವುದು? ಸದ್ಯಕ್ಕೆ ಈ ಮಾತನ್ನು ಇಲ್ಲಿಗೆ ನಿಲ್ಲಿಸೋಣ, ಮತ್ತೊಮ್ಮೆ ಮಾತನಾಡೋಣ ಎಂದು ಸ್ವಲ್ಪ ಕಾಲದ ನಂತರ ಸ್ವಾಮೀಜಿ ಮುಂದುವರಿಸಿದರು: “ಧರ್ಮವನ್ನು ಪುನರುಜ್ಜೀವನಗೊಳಿಸಲು ಮಹಾಪುರುಷರು ಕಾಲ, ಸಮಾಜಕ್ಕನುಗುಣವಾಗಿ ಬರುವರು. ಅವರನ್ನು ಮಹಾಪುರುಷರು ಅಥವಾ ಅವತಾರ ಪುರುಷರು ಎಂದು ಹೇಗೆ ಬೇಕಾದರೂ ಕರೆಯಿರಿ. ಅವರಲ್ಲಿ ಪ್ರತಿಯೊಬ್ಬರೂ ತಮ್ಮ ಜೀವನದಲ್ಲಿ ಆ ಮಹಾಧ್ಯೇಯವನ್ನು ತೋರಿಸಿಕೊಡುವರು. ನಂತರ ಕ್ರಮೇಣ ಅವರ ಅಚ್ಚಿನಲ್ಲಿ ಮನುಷ್ಯರು ರೂಪುಗೊಳ್ಳುವಂತೆ ಮಾಡುತ್ತಾರೆ. ಕ್ರಮೇಣ ಪಂಥಗಳು ಬೆಳೆದು ಹರಡುತ್ತವೆ. ಕಾಲಕಳೆದಂತೆಲ್ಲಾ ಈ ಪಂಥಗಳು ಜೀರ್ಣಹೊಂದಿ ಪುನಃ ಉಜೀವನಗೊಳಿಸಲು ಸುಧಾರಕರು ಬರುವರು. ಹೀಗೆ ಯುಗಯುಗಾಂತರಗಳಿಂದ ನಿರರ್ಗಳವಾಗಿ ಪ್ರವಾಹ ಹರಿದು ಬರುತ್ತಿದೆ."

ಶಿಷ್ಯ: ಶ‍್ರೀರಾಮಕೃಷ್ಣರು ಅವತಾರವೆಂದು ನೀವೇಕೆ ಬೋಧಿಸುವುದಿಲ್ಲ - ನಿಮಗೆ ಶಕ್ತಿ, ವಾಕ್ಚಾತುರ್ಯ ಎಲ್ಲಾ ಇದೆಯಲ್ಲ.

ಸ್ವಾಮಿಜಿ: ನಿಜವಾಗಿ ನಾನು ಅವರನ್ನು ಬಹು ಸ್ವಲ್ಪ ಅರ್ಥಮಾಡಿಕೊಂಡಿರುವೆ. ಅವರು ನನಗೆ ಎಷ್ಟು ದೊಡ್ಡವರಾಗಿ ಕಾಣಿಸುವರೆಂದರೆ ನಾನು ಯಾವಾಗಲಾದರೂ ಅವರ ವಿಚಾರ ಮಾತನಾಡುವಾಗ ನನಗೆ ಹೆದರಿಕೆಯಾಗುತ್ತದೆ. ನಾನೆಲ್ಲಿ ನಿಜಸ್ಥಿತಿಯನ್ನು ಬಿಟ್ಟು ಅಲಕ್ಷ್ಯದಿಂದ ಮಾತನಾಡುವೆನೊ, ನನ್ನ ಈ ಕ್ಷುದ್ರ ಶಕ್ತಿಗೆ ಅದೆಲ್ಲಿ ನಿಲುಕುವುದಿಲ್ಲವೊ, ಅವರನ್ನು ಹೊಗಳುವಾಗ ಅವರ ಚಿತ್ರವನ್ನು ನನ್ನಿಷ್ಟಾನುಸಾರ ಚಿತ್ರಿಸಿ ಅವರನ್ನು ಎಲ್ಲಿ ಅಲ್ಪವಾಗಿ ಮಾಡುತ್ತೇನೆ ಎಂದು ಹೆದರುವೆ.

ಶಿಷ್ಯ: ಆದರೆ ಎಷ್ಟೋ ಮಂದಿ ಈಗ ಅವರನ್ನು ಅವತಾರ ಪುರುಷರೆಂದು ಪ್ರಚಾರ ಮಾಡುತ್ತಿದ್ದಾರೆ.

ಸ್ವಾಮೀಜಿ: ಅವರಿಷ್ಟಬಂದಂತೆ ಮಾಡಲಿ. ಅವರು ತಾವು ಹೇಗೆ ಅವರನ್ನು ಅರ್ಥಮಾಡಿಕೊಂಡಿದ್ದಾರೊ ಹಾಗೆ ಪ್ರಚಾರ ಮಾಡುತ್ತಿದ್ದಾರೆ. ನೀನೂ ಅವರನ್ನು ಅರ್ಥಮಾಡಿಕೊಂಡಿದ್ದಲ್ಲಿ ಹಾಗೆಯೇ ಹೋಗಿ ಅವರಂತೆಯೇ ಮಾಡಬಹುದು.

ಶಿಷ್ಯ: ನನಗೆ ನಿಮ್ಮನ್ನೆ ಅರ್ಥಮಾಡಿಕೊಳ್ಳಲಾಗುತ್ತಿಲ್ಲ. ಇನ್ನು ಶ‍್ರೀರಾಮಕೃಷ್ಣರ ಮಾತೇಕೆ? ನಿಮ್ಮ ಕೃಪೆ ಕಿಂಚಿತ್ತಾದರೂ ನನ್ನ ಮೇಲೆ ಬಿದ್ದರೆ ಈ ಜನ್ಮದಲ್ಲಿ ನಾನು ಧನ್ಯನೆಂದು ಭಾವಿಸುವೆ.

\newpage

\chapter[ಅಧ್ಯಾಯ ೨೪]{ಅಧ್ಯಾಯ ೨೪\protect\footnote{\engfoot{C.W, Vol. V, P. 390}}}

\begin{center}
ಸ್ಥಳ: ಬೇಲೂರು ಮಠವನ್ನು ಕಟ್ಟುತ್ತಿದ್ದಾಗ, ವರ್ಷ: ಕ್ರಿ.ಶ. ೧೮೯೮.
\end{center}

ಶಿಷ್ಯ: ಬ್ರಹ್ಮನು ಏಕಮಾತ್ರ ಸತ್ಯವಾಗಿದ್ದರೆ ಪ್ರಪಂಚದಲ್ಲೇಕೆ ಇಷ್ಟೊಂದು ಭಿನ್ನತೆಗಳಿವೆ.

ಸ್ವಾಮೀಜಿ: ಬಾಹ್ಯ ವಸ್ತುಗಳನ್ನು ನೋಡಿ ಈ ಪ್ರಶ್ನೆಯನ್ನು ಹಾಕುತ್ತಿದ್ದೀಯಲ್ಲವೆ? ಬಾಹ್ಯರೂಪದಲ್ಲಿ ವಸ್ತುಗಳನ್ನು ವಿಚಾರ ಮತ್ತು ಕಾರಣಗಳ ದೃಷ್ಟಿಯಿಂದ ವಿಮರ್ಶಿಸಿದರೆ ಕ್ರಮೇಣ ಪೂರ್ಣತ್ವದ ಕಡೆ ಬರುತ್ತೇವೆ. ನೀನು ಆ ಪೂರ್ಣತ್ವದಲ್ಲಿ ಲಯವಾಗಿ ಹೋಗಿದ್ದರೆ ಈ ಭಿನ್ನತೆ ನಿನಗೆ ಹೇಗೆ ಕಾಣಲು ಸಾಧ್ಯ?

ಶಿಷ್ಯ: ನಿಜ, ನಾನು ಪೂರ್ಣತ್ವದೊಂದಿಗೆ ಐಕ್ಯನಾಗಿದ್ದರೆ ‘ಏಕೆ?’ ಎನ್ನುವ ಪ್ರಶ್ನೆಯನ್ನು ಎತ್ತಲು ಹೇಗೆ ಸಾಧ್ಯವಾಗುತ್ತಿತ್ತು? ನಾನು ಈಗ ಆ ಭಿನ್ನತೆಯನ್ನು ನೋಡಿ ಅದನ್ನು ಒಪ್ಪಿಕೊಂಡು ಪ್ರಶ್ನೆಯನ್ನು ಕೇಳುತ್ತಿದ್ದೇನಲ್ಲ.

ಸ್ವಾಮೀಜಿ: ಸರಿ, ಏಕತ್ವವನ್ನು ಈ ಬಾಹ್ಯರೂಪದ ಭಿನ್ನ ದೃಷ್ಟಿಯಿಂದ ನೋಡುವುದನ್ನು, ಶಾಸ್ತ್ರಗಳು ವ್ಯತಿರೇಕೀಕರಣ ಅಥವಾ ಆರೋಪ ಎನ್ನುತ್ತವೆ. ಎಂದರೆ ಯಾವುದನ್ನಾದರೂ ಅದು ಇಲ್ಲದೆ ಇದ್ದರೂ ಇದೆ ಎಂದು ಭಾವಿಸಿ ನಂತರ ವಿಮರ್ಶೆ ಮಾಡಿ ಅದೂ ಇಲ್ಲವೆಂದು ಹೇಳುವುದು. ನೀನು ಮಾಡುತ್ತಿರುವುದು ಯಾವುದೋ ಒಂದನ್ನು ಸತ್ಯವಲ್ಲದಿದ್ದರೂ ಸತ್ಯವೆಂದು ಭಾವಿಸಿ ಅನಂತರ ಸತ್ಯಕ್ಕೆ ಬರುವ ಮಾರ್ಗವನ್ನು ಕೇಳುತ್ತಿದ್ದೀಯೆ ಅಲ್ಲವೆ?

ಶಿಷ್ಯ: ನನ್ನ ಮನಸ್ಸಿಗೆ ನಾನೀಗ ನೋಡುತ್ತಿರುವುದೆಲ್ಲಾ ನಿಜವೆಂದು ಅನ್ನಿಸುತ್ತದೆ. ಆದಕಾರಣ ಅದು ಸತ್ಯವೆಂದು ಹೇಳುವೆ. ಇದಕ್ಕೆ ವಿರುದ್ಧವಾಗಿರುವುದು ನಿಜವಲ್ಲವೆಂದೂ ಅನ್ನಿಸುತ್ತದೆ.

ಸ್ವಾಮೀಜಿ: ಆದರೆ ವೇದಗಳು ‘ಏಕಮೇವಾದ್ವಿತೀಯ’ ಎನ್ನುತ್ತವೆ. ಸತ್ಯವಾಗಿ ಒಂದೇ ಇದ್ದಿದ್ದರೆ - ಬ್ರಹ್ಮನು ಮಾತ್ರ ಇದ್ದಿದ್ದರೆ - ನಿನ್ನ ಭಿನ್ನಭಾವ ತಪ್ಪು. ನೀನು ವೇದಗಳನ್ನು ನಂಬುವೆ, ಅಲ್ಲವೆ?

ಶಿಷ್ಯ: ಓ, ಖಂಡಿತವಾಗಿ, ನನಗೇನೊ ವೇದಗಳೇ ಪರಮಶ್ರೇಷ್ಠ ಪ್ರಮಾಣಗಳು. ಆದರೆ ವಾಗ್ವಾದ ಮಾಡುವಾಗ ಅದನ್ನು ಒಪ್ಪಿಕೊಳ್ಳದಿದ್ದಲ್ಲಿ ಇತರ ಕಾರಣಗಳನ್ನು ಕೊಟ್ಟಾದರೂ ಅದನ್ನು ಖಂಡಿಸಬೇಕು.

ಸ್ವಾಮೀಜಿ: ಅದೂ ಕೂಡ ಆಗಬಹುದು. ನೋಡು ಮಗು, ಒಂದು ದಿನ ನೀನು ಈ ಭೇದಗಳೆಲ್ಲಾ ಮಾಯವಾಗುವುದನ್ನು ನೋಡುವೆ. ಅದನ್ನು ನಾವು ನೋಡಲಿಕ್ಕೆ ಕೂಡ ಆಗುವುದಿಲ್ಲ. ನನ್ನೀ ಜೀವನದಲ್ಲೇ ನಾನು ಅಂತಹ ಸ್ಥಿತಿಯನ್ನು ಅನುಭವಿಸಿರುವೆ.

ಶಿಷ್ಯ: ನಿಮಗೆ ಯಾವಾಗ ಹಾಗೆ ಆಗಿತ್ತು?

ಸ್ವಾಮೀಜಿ: ಒಂದು ದಿನ ದಕ್ಷಿಣೇಶ್ವರದ ದೇವಾಲಯದ ತೋಟದಲ್ಲಿ ಶ‍್ರೀರಾಮಕೃಷ್ಣರು ನನ್ನ ಎದೆಯನ್ನು ಮುಟ್ಟಿದರು. ಮೊದಲು ಮನೆ, ಕೊಠಡಿ, ಬಾಗಿಲು, ಕಿಟಕಿ, ವರಾಂಡ, ಗಿಡ, ಸೂರ್ಯ - ಚಂದ್ರ ಎಲ್ಲವೂ ಚೂರು ಚೂರಾಗಿ ಒಡೆದು ಅಣುಮಾತ್ರವಾಗಿ ಹಾರಿಹೋಗಿ ಕೊನೆಗೆ ಆಕಾಶದಲ್ಲಿ ಲೀನವಾದುದನ್ನು ನಾನು ನೋಡಿದೆ. ಕ್ರಮೇಣ ಆಕಾಶವೂ ಮಾಯವಾಯಿತು. ನಂತರ ನನ್ನ ಚೇತನದ ಭಾವವೂ ಹೊರಟುಹೋಯಿತು. ಆಮೇಲೆ ನನಗೇನಾಯಿತೆಂಬ ಅರಿವಿಲ್ಲ. ನನಗೆ ಮೊದಲು ತುಂಬಾ ಹೆದರಿಕೆಯಾಯಿತು. ಆ ಸ್ಥಿತಿಯಿಂದ ಕೆಳಗಿಳಿದು ಬಂದಮೇಲೆ ಪುನಃ ಮೊದಲಿನಂತೆ ಮನೆ, ಕಿಟಕಿ, ಬಾಗಿಲು, ವರಾಂಡ ಮುಂತಾದ್ದೆಲ್ಲವನ್ನೂ ನೋಡತೊಡಗಿದೆ. ಮತ್ತೊಮ್ಮೆ ಅಮೆರಿಕಾ ದೇಶದಲ್ಲಿ ಒಂದು ಸರೋವರದ ಪಕ್ಕದಲ್ಲಿ ನನಗೆ ಪುನಃ ಅದೇ ಅನುಭವವಾಯಿತು.

ಶಿಷ್ಯ: ಆ ಸ್ಥಿತಿಯನ್ನು ಕೂಡ ಮೆದುಳಿನ ನರಗಳ ದುರ್ಬಲತೆಯಿಂದಾದುದೆಂದು ಹೇಳಲಾಗುವುದಿಲ್ಲವೆ? ಅಂತಹ ಸ್ಥಿತಿಯನ್ನು ಪಡೆದುದರಿಂದ ಯಾವ ಆನಂದವುಂಟಾಗುವುದೋ ನನಗರ್ಥವಾಗುವುದಿಲ್ಲ.

ಸ್ವಾಮಿಜಿ: ಮಿದುಳಿನ ಅವ್ಯವಸ್ಥೆ! ಅದು ಯಾವ ಖಾಯಿಲೆಯೂ ಉಲ್ಬಣವಾಗಿ ಅದರಿಂದ ಭ್ರಮೆಯುಂಟಾಗಿ ಬಂದುದಲ್ಲ. ಕುಡಿತದ ಅಮಲೇರಿ ಬಂದುದಲ್ಲ. ವಿಲಕ್ಷಣವಾದ ಪ್ರಾಣಾಯಾಮದ ಅಭ್ಯಾಸಗಳಿಂದುಂಟಾದ ಕಣ್ಣು ಕಟ್ಟಿನಿಂದಲೂ ಬಂದುದಲ್ಲ - ಕೇವಲ ಒಬ್ಬ ಆರೋಗ್ಯವಂತನಾದ, ದೃಢಕಾಯನಾದ, ಬುದ್ಧಿ ನೆಟ್ಟಗಿರುವ ಮನುಷ್ಯನಿಗೆ ಈ ಸ್ಥಿತಿಯುಂಟಾದರೆ, ಅದನ್ನು ಹೇಗೆ ಮಿದುಳಿನ ಅವ್ಯವಸ್ಥೆಯಿಂದ ಉಂಟಾದುದೆಂದು ಹೇಳುವೆ? ವೇದಗಳಲ್ಲೂ ಇದೇ ಭಾವವನ್ನು ವಿವರಿಸಿದೆ. ಹಿಂದಿನ ಕಾಲದ ದೇವರ ಸಾಕ್ಷಾತ್ಕಾರ ಪಡೆದ ಋಷಿಗಳ, ಆಚಾರ್ಯರುಗಳ ಹೇಳಿಕೆಗೂ ಸರಿಹೊಂದುತ್ತದೆ. ನನ್ನನ್ನು ಒಬ್ಬ ಮಿದುಳು ಕೆಟ್ಟಿರುವ ಮನುಷ್ಯನೆಂದು ಹೇಳುವೆಯೇನು? (ನಗುತ್ತಾರೆ)

ಶಿಷ್ಯ: ಇಲ್ಲ, ಇಲ್ಲ, ನಾನು ಅದನ್ನು ಈ ಉದ್ದೇಶದಿಂದ ಹೇಳಲಿಲ್ಲ - ಶಾಸ್ತ್ರಗಳಲ್ಲಿ ಇಂತಹ ನೂರಾರು ಏಕತ್ವದ ಸಾಕ್ಷಾತ್ಕಾರ ನಿದರ್ಶನಗಳನ್ನು ಕಾಣಬಹುದು. ಅಂಗೈಮೇಲಿನ ನೆಲ್ಲಿಕಾಯಂತೆ ನೀವು ಅದನ್ನು ಸಾಕ್ಷಾತ್ಕರಿಸಿಕೊಂಡಿರುವಿರಿ. ಇದು ನಿಮ್ಮ ಜೀವನದಲ್ಲೇ ಸ್ವಂತ ಅನುಭವವಾಗಿದೆ. ಇತರ ಶಾಸ್ತ್ರಗಳು ವೇದದಲ್ಲಿ ಹೇಳಿರುವುದಕ್ಕೆ ಸಂಪೂರ್ಣ ಹೊಂದಾಣಿಕೆ ಇದೆ - ಅಂದ ಮೇಲೆ ಇದು ಸುಳ್ಳೆಂದು ಹೇಳಲು ನನಗಾವ ಧೈರ್ಯ? ಶ‍್ರೀಶಂಕರಾಚಾರ್ಯರೂ ಈ ಸ್ಥಿತಿಯನ್ನು ಸಾಕ್ಷಾತ್ಕರಿಸಿಕೊಂಡ ಮೇಲೆ, ಭೂಮಂಡಲವೆಲ್ಲಾ ಎಲ್ಲಿಗೆ ಹೋಯಿತು? ಎನ್ನುತ್ತಾರೆ.

ಸ್ವಾಮೀಜಿ: ಇದನ್ನು ತಿಳಿ - ಇದೇ, ಈ ಏಕತ್ವದ ಜ್ಞಾನವೇ, ಶಾಸ್ತ್ರಗಳು ಹೇಳುವ ಬ್ರಹ್ಮಸಾಕ್ಷಾತ್ಕಾರ. ಅದನ್ನು ತಿಳಿಯುವುದರಿಂದ ಅಂಜಿಕೆಯಿಂದ ಪಾರಾಗುವೆವು, ಹುಟ್ಟು ಸಾವುಗಳ ಸಂಕೋಲೆಯಿಂದ ಶಾಶ್ವತವಾಗಿ ಬಿಡುಗಡೆ ಹೊಂದುವೆವು. ಈ ಪರಮಾನಂದದ ಸಾಕ್ಷಾತ್ಕಾರವಾದ ಮೇಲೆ ಈ ಪ್ರಪಂಚದ ದುಃಖ ಸುಖಗಳು ತೃಣಮಾತ್ರವೂ ತಾಕುವುದಿಲ್ಲ. “ಕ್ಷುದ್ರ ಕಾಮ ಕಾಂಚನದ ಬಂಧನದಲ್ಲಿ ಮುಳುಗಿರುವ ಜನರಿಗೆ ಈ ಬ್ರಹ್ಮಾನಂದವನ್ನು ಸವಿಯಲಾಗುವುದಿಲ್ಲ."

ಶಿಷ್ಯ: ಇದು ಹೀಗಿದ್ದ ಪಕ್ಷಕ್ಕೆ, ಆ ಪರಬ್ರಹ್ಮನ ಅಂಶವೇ ನಮ್ಮಲ್ಲೂ ಇದ್ದ ಪಕ್ಷಕ್ಕೆ ನಾವೇಕೆ ಆ ಆನಂದವನ್ನು ಪಡೆಯಲು ಹೋರಾಡುವುದಿಲ್ಲ? ನಾವೇಕೆ ಈ ಕ್ಷುದ್ರ ಕಾಮಿನಿ ಕಾಂಚನದ ಬಲೆಗೆ ಮೋಹಗೊಂಡು ಮೃತ್ಯುವಿನ ದವಡೆಗೆ ಪುನಃ ಪುನಃ ಓಡಿಹೋಗಿ ಬೀಳುತ್ತೇವೆ?

ಸ್ವಾಮಿಜಿ: ನೀನೀಗ ಹೇಳುತ್ತಿರುವುದು ಮನುಷ್ಯನು ಆ ಸಚ್ಚಿದಾನಂದದ ಸವಿಗೆ ಆಸೆಪಡುತ್ತಿಲ್ಲ ಎಂದು! ಗಾಢವಾಗಿ ಯೋಚಿಸು. ಯಾರಾದರಾಗಲಿ, ಅವನೇನನ್ನು ಮಾಡುತ್ತಿದ್ದಾನೆಯೋ ಅದನ್ನು ಸಚ್ಚಿದಾನಂದವನ್ನು ಪಡೆಯುವ ಆಸೆಯಿಂದಲೇ ಮಾಡುತ್ತಿದ್ದಾನೆಂದು ನಿನಗೆ ತಿಳಿಯುವುದು. ಆದರೆ ಎಲ್ಲರೂ ಅದನ್ನು ಅರಿತು ಮಾಡುತ್ತಿಲ್ಲ. ಅದಕ್ಕೇ ಅದನ್ನು ಅರ್ಥಮಾಡಿಕೊಳ್ಳಲಾರರು. ಬ್ರಹ್ಮನಿಂದ ಹಿಡಿದು ಹುಲ್ಲುಕಡ್ಡಿಯವರೆಗೆ ಎಲ್ಲದರಲ್ಲೂ ಸಚ್ಚಿದಾನಂದವು ಪೂರ್ಣವಾಗಿ ಇದ್ದೇ ಇದೆ. ನೀನೂ ಕೂಡ ಆ ಅಖಂಡ ಬ್ರಹ್ಮ. ನೀನು ನಿಜವಾಗಿಯೂ ಪ್ರಮಾಣವಾಗಿ ಹೀಗೆಂದು ತಿಳಿದುಕೊಂಡಲ್ಲಿ, ಈ ಕ್ಷಣದಲ್ಲಿಯೇ ಅದನ್ನು ಸಾಕ್ಷಾತ್ಕರಿಸಿಕೊಳ್ಳಬಹುದು. ಕೇವಲ ಸ್ಪಷ್ಟವಾದ ಅನುಭವ ಬೇಕಾಗಿದೆ. ನಿನ್ನ ಸತಿಸುತರಿಗಾಗಿ ಇಷ್ಟೊಂದು ಕಷ್ಟಪಟ್ಟು ಕೆಲಸ ಮಾಡುತ್ತಿರುವುದು ಕೂಡ ಅಂತ್ಯದಲ್ಲಿ ಆ ಸಚ್ಚಿದಾನಂದ ಬ್ರಹ್ಮನನ್ನು ಪಡೆಯುವುದಕ್ಕಾಗಿಯೆ. ಬಾರಿಬಾರಿಗೂ ಮಾಯೆಯ ಜಟಿಲ ತೊಡಕಿನಲ್ಲಿ ಸಿಕ್ಕಿ ದುಃಖ ಕೋಟಲೆಗಳ ಪೆಟ್ಟನ್ನು ತಿಂದಾಗ ತನ್ನಷ್ಟಕ್ಕೆ ತಾನೇ ಮನಸ್ಸು ತನ್ನ ನಿಜ ಸ್ವರೂಪವಾದ ಆತ್ಮನ ಕಡೆ ತಿರುಗುತ್ತದೆ. ನಮ್ಮ ಹೃದಯದಲ್ಲಿರುವ ಈ ಸಚ್ಚಿದಾನಂದವನ್ನು ಪಡೆಯುವ ಆಸೆ ಇರುವುದರಿಂದಲೇ ಮಾನವನು ಒಂದಾದಮೇಲೊಂದು ಪೆಟ್ಟನ್ನು ತಿಂದು ತನ್ನ ಕಣ್ಣನ್ನು ಅಂತರಂಗದ ಕಡೆ ತಿರುಗಿಸುವನು. ಪ್ರತಿಯೊಬ್ಬರಿಗೂ ತಪ್ಪದೆ ಒಂದಲ್ಲ ಒಂದು ಸಾರಿ ಈ ಸ್ಥಿತಿ ಬಂದೇ ಬರುವುದು. ಆಗ ಅವನು ಇದನ್ನು ಮಾಡುವನು. ಒಬ್ಬನಿಗೆ ಈ ಜನ್ಮದಲ್ಲೇ ಬರಬಹುದು. ಇನ್ನೊಬ್ಬನಿಗೆ ಸಾವಿರಾರು ಜನ್ಮಗಳಾದನಂತರ ಬರಬಹುದು.

ಶಿಷ್ಯ: ಎಲ್ಲವೂ ಗುರು ಮತ್ತು ಭಗವಂತನ ಕೃಪಾದೃಷ್ಟಿಯ ಮೇಲೆ ನಿಂತಿದೆ.

ಸ್ವಾಮೀಜಿ: ಭಗವಂತನ ಕೃಪೆಯೆಂಬ ಗಾಳಿ ನಿರಂತರವಾಗಿ ಬೀಸುತ್ತಿದೆ. ನೀನು ನಿನ್ನ ಹಾಯಿಪಟವನ್ನು ಎತ್ತಿ ಕಟ್ಟಿರುವೆಯಾ? ನೀನು ಯಾವ ಕೆಲಸವನ್ನು ಮಾಡಬೇಕಾದರೂ ಅದನ್ನು ಹೃತ್ಪೂರ್ವಕವಾಗಿ ಮಾಡು. ಹಗಲು ರಾತ್ರಿ ‘ನಾನು ಅಖಂಡ ಸಚ್ಚಿದಾನಂದ ಪರಬ್ರಹ್ಮನ ಅಂಶ; ನನಗಾವ ಹೆದರಿಕೆ, ಕಳವಳ? ಈ ದೇಹ, ಮನಸ್ಸು, ಬುದ್ಧಿ ಎಲ್ಲಾ ಕ್ಷಣಿಕ, ಇವೆಲ್ಲವನ್ನೂ ಮೀರಿದುದು ನನ್ನ ಆತ್ಮ’ ಎಂದು ಯೋಚಿಸು.

ಶಿಷ್ಯ: ಈ ಬಗೆಯ ಆಲೋಚನೆಗಳು ಒಮ್ಮೊಮ್ಮೆ ಬರುತ್ತವೆ. ಆದರೆ ತಕ್ಷಣ ಮಾಯವಾಗಿ ಏನೇನೋ ಕೆಲಸಕ್ಕೆ ಬಾರದ ಅರ್ಥವಿಲ್ಲದ ಯೋಚನೆಗಳು ಬರುತ್ತವೆ.

ಸ್ವಾಮೀಜಿ: ಪ್ರಾರಂಭದಲ್ಲಿ ಹೀಗೆ ಆಗುತ್ತದೆ. ಆದರೆ ಕ್ರಮೇಣ ನೀನು ಇದನ್ನು ಮೀರಿಹೋಗುವೆ. ಆದರೆ ಮೊದಲಿನಿಂದಲೂ ಮನಸ್ಸಿನಲ್ಲಿ ಅದರ ಆಸೆ ತೀವ್ರವಾಗಿರಬೇಕು. ಯಾವಾಗಲೂ ಹೀಗೆ ಯೋಚಿಸು: ‘ನಾನು ನಿತ್ಯ ಪವಿತ್ರ, ನಿತ್ಯಜ್ಞಾನಿ, ನಿತ್ಯಮುಕ್ತ. ನಾನು ಕೆಡುಕನ್ನು ಮಾಡಲು ಹೇಗೆ ಸಾಧ್ಯ? ಸಾಮಾನ್ಯ ಜನರಂತೆ ಕ್ಷುದ್ರ ಕಾಮಿನಿ ಕಾಂಚನದ ಮೋಹಕ್ಕೆ ನಾನೆಲ್ಲಾದರೂ ಸಿಕ್ಕಿಬೀಳುವೆನೆ?’ ನಿನ್ನ ಮನಸ್ಸನ್ನು ಈ ಬಗೆಯ ಆಲೋಚನೆಗಳಿಂದ ದೃಢಪಡಿಸು. ಇದರಿಂದ ಖಂಡಿತ ಒಳ್ಳೆಯದಾಗುವುದು.

ಶಿಷ್ಯ: ಯಾವಾಗಲಾದರೊಮ್ಮೆ ದೃಢ ಮನಸ್ಸು ಬರುವುದು. ಆದರೆ ಪುನಃ ನಾನು ಆ ಉಪನ್ಯಾಯಾಧಿಪತಿಯ ಪರೀಕ್ಷೆಗೆ ಕುಳಿತುಕೊಂಡರೆ ಎಂದು ಯೋಚಿಸಿದಾಗ ನನಗೆ ಐಶ್ವರ್ಯ, ಹೆಸರು, ಕೀರ್ತಿ ಎಲ್ಲ ಬಂದು ನಾನು ಸುಖವಾಗಿ ಬದುಕುತ್ತೇನೆ ಅನ್ನಿಸುತ್ತದೆ.

ಸ್ವಾಮಿಜಿ: ನಿನಗೆ ಈ ರೀತಿ ಯೋಚನೆಗಳು ಬಂದಾಗಲೆಲ್ಲ ನಿತ್ಯಾನಿತ್ಯ ವಸ್ತುವಿವೇಕ ಜ್ಞಾನದ ವಿಚಾರ ಮಾಡು. ನೀನು ವೇದಾಂತವನ್ನು ಓದಿಲ್ಲವೆ? ನೀನು ನಿದ್ರಿಸುತ್ತಿರುವಾಗಲೂ ವಿಚಾರದ ಖಡ್ಗವು ನಿನ್ನ ಶಿರದ ಮೇಲಿರಲಿ. ಆಗ ನಿನ್ನ ಸ್ವಪ್ನದಲ್ಲಿಯೂ ದುರಾಶೆ ನಿನ್ನ ಹತ್ತಿರ ಸುಳಿಯಲಾರದು. ಈ ಬಗೆಯ ಶಕ್ತಿಯನ್ನು ನೀನು ವೃದ್ಧಿ ಪಡಿಸಿಕೊಂಡರೆ ತ್ಯಾಗ ಬುದ್ಧಿ ಕ್ರಮೇಣ ನಿನಗೆ ಬರುವುದು. ಆಗ ಸ್ವರ್ಗದ್ವಾರ ನಿನಗಾಗಿ ತೆರೆಯುವುದನ್ನು ನೋಡುವೆ.

ಶಿಷ್ಯ: ಅದು ಹೀಗಿದ್ದ ಪಕ್ಷಕ್ಕೆ ಏಕೆ ಭಕ್ತಿ ಗ್ರಂಥಗಳು ಅತಿ ತ್ಯಾಗ ಬುದ್ಧಿಯು ಮೃದು ಸ್ವಭಾವವನ್ನೇ ನಿರ್ಮೂಲ ಮಾಡುತ್ತದೆ ಎಂದು ಹೇಳುತ್ತವೆ?

ಸ್ವಾಮೀಜಿ: ಈ ರೀತಿ ಹೇಳುವ ಪುಸ್ತಕಗಳನ್ನು ಬಿಸಾಡು! ತ್ಯಾಗವಿಲ್ಲದೆ, ವಿಷಯ ಸುಖಗಳಲ್ಲಿ ಜುಗುಪ್ಸೆಯಿಲ್ಲದೆ, ಕಾಮ ಕಾಂಚನ ಹೇಸಿಗೆ ಸಮಾನವೆಂದು ಭಾವಿಸದೆ ಇದ್ದರೆ ‘ನ ಸಿದ್ಧ್ಯತಿ ಬ್ರಹ್ಮಶತಾಂತರೇಪಿ’ - ನೂರಾರು ಕಲ್ಪಗಳಲ್ಲಿಯೂ ಮುಕ್ತಿ ಪಡೆಯಲಾರರು. ಭಗವನ್ನಾಮಸ್ಮರಣೆ, ಧ್ಯಾನ, ಪೂಜೆ, ಹೋಮಾಗ್ನಿಯಲ್ಲಿ ತರ್ಪಣ ಕೊಡುವುದು ಇವೆಲ್ಲ ತ್ಯಾಗ ಬುದ್ಧಿಯನ್ನು ಹೊಂದುವುದಕ್ಕೆ. ಯಾರಿಗೆ ಈ ತ್ಯಾಗ ಭಾವನೆಯಿಲ್ಲವೊ ಅವರ ಸ್ಥಿತಿಯು ಲಂಗರು ಹಾಕಿರುವ ದೋಣಿಯನ್ನು ಹುಟ್ಟು ಹಾಕಲೆತ್ನಿಸುತ್ತಿರುವ ಮನುಷ್ಯನಂತಾಗುತ್ತದೆ. ‘ನ ಪ್ರಜಯಾ ಧನೇನ ತ್ಯಾಗೇನೈಕೇ ಅಮೃತತ್ವಮಾನುಶುಃ’ - ಸಂತತಿಯಿಂದಲ್ಲ, ಐಶ್ವರ್ಯದಿಂದಲ್ಲ, ಕೇವಲ ತ್ಯಾಗದಿಂದ ಮಾತ್ರ ಕೆಲವರು ಅಮರತ್ವವನ್ನೈದಿದರು.

ಶಿಷ್ಯ: ಕೇವಲ ಕಾಮಕಾಂಚನ ತ್ಯಾಗ ಮಾಡಿಬಿಟ್ಟರೆ ಎಲ್ಲವೂ ಮುಗಿದಂತಾಗುತ್ತದೆಯೇ?

ಸ್ವಾಮೀಜಿ: ಅವೆರಡನ್ನೂ ತ್ಯಜಿಸಿದ ಮೇಲೂ ಹಾದಿಯಲ್ಲಿ ಇತರ ಆತಂಕಗಳಿವೆ. ಉದಾಹರಣೆಗೆ ಹೆಸರು ಕೀರ್ತಿಗಳ ಆಸೆ, ಅಪೂರ್ವವಾದ ಶಕ್ತಿಯುಳ್ಳ ಕೆಲವರು ಹೊರತು ಮಿಕ್ಕವರು ಈ ಮಟ್ಟದಲ್ಲಿ ತಮ್ಮನ್ನು ಹಿಡಿತದಲ್ಲಿಟ್ಟುಕೊಳ್ಳಲು ಅಸಾಧ್ಯ. ಜನರು ಅವರ ಮೇಲೆ ಗೌರವದ ಮಳೆ ಸುರಿಸುವರು - ಮಟ್ಟಿಲು ಮೆಟ್ಟಿಲಾಗಿ ಇತರ ಭೋಗಲಾಲಸೆಗಳೂ ಕಾಲಿಡುವುವು. ಇದರ ಫಲವಾಗಿ ಮುಕ್ಕಾಲುಪಾಲು ಮಂದಿ ತ್ಯಾಗಿಗಳೆಲ್ಲಾ ತಮ್ಮ ಮುಂದಿನ ಪ್ರಗತಿಗೆ ಧಕ್ಕೆ ತಂದುಕೊಂಡಿರುವರು. ಈ ಮಠ ಮುಂತಾದುವುಗಳನ್ನು ಸ್ಥಾಪಿಸಿದುದರಿಂದ ನಾನೇ ಪುನಃ ಬರಬೇಕಾಗುವುದೊ ಏನೋ ಯಾರಿಗೆ ಗೊತ್ತು!

ಶಿಷ್ಯ: ನೀವೇ ಈ ರೀತಿ ಮಾತನಾಡಿದರೆ ನಮ್ಮ ಸ್ಥಿತಿ ಮುಗಿದಂತೆಯೇ!

ಸ್ವಾಮೀಜಿ: ಅಂಜಿಕೆಯೇನು? ಧೈರ್ಯದಿಂದಿರು. ಧೈರ್ಯದಿಂದಿರು. ನೀನು ನಾಗಮಹಾಶಯರನ್ನು ನೋಡಿರುವೆಯಲ್ಲವೆ? ಗೃಹಸ್ಥ ಜೀವನದ ಮಧ್ಯೆ ಇದ್ದರೂ ಕೂಡ ಅವರು ಸಂನ್ಯಾಸಿಗಳನ್ನು ಮೀರಿಸಿರುವರು. ಇದು ಅತಿ ವಿರಳ. ನಾನು ಇವರಂತಹ ಮನುಷ್ಯರನ್ನು ನೋಡೇ ಇಲ್ಲ. ಯಾರಾದರೂ ಗೃಹಸ್ಥರಾಗಬೇಕೆಂದು ಇಚ್ಛಿಸಿದರೆ ಅವರು ನಾಗಮಹಾಶಯರಂತಿರಲಿ. ಪೂರ್ವಬಂಗಾಳದ ಆಧ್ಯಾತ್ಮಿಕ ಆಗಸದಲ್ಲಿ ಥಳಥಳಿಸುತ್ತಿರುವ ಜ್ಯೋತಿರ್ಮಯ ಪ್ರದೀಪದಂತಿರುವರು. ಆ ಭಾಗದಲ್ಲಿ ವಾಸಿಸುವ ಜನರಿಗೆ ಅವರ ದರ್ಶನ ಮಾಡಲು ಹೇಳು. ಅದರಿಂದ ಅವರಿಗೆ ಶುಭವಾಗುವುದು.

ಶಿಷ್ಯ: ಶ‍್ರೀರಾಮಕೃಷ್ಣ ದೇವರ ಲೀಲಾನಾಟಕದಲ್ಲಿ ನಾಗಮಹಾಶಯರು ನಮ್ರತೆಯ ಅವತಾರವೆಂಬಂತಿದ್ದರೆಂದು ಹೇಳುವರು.

ಸ್ವಾಮಿಜಿ: ಖಂಡಿತವಾಗಿ, ಅದರಲ್ಲಿ ಕಿಂಚಿತ್ತೂ ಅನುಮಾನವಿಲ್ಲ. ಒಮ್ಮೆ ಅವರನ್ನು ಹೋಗಿ ನೋಡಬೇಕೆನ್ನಿಸಿದೆ. ನೀನೂ ನನ್ನೊಡನೆ ಬರುವೆಯಾ? ಮಳೆಯ ನೀರು ಬಂದು ಗದ್ದೆಯಲ್ಲೆಲ್ಲಾ ಹಾಯುತ್ತಿರುವುದನ್ನು ನೋಡಬೇಕೆಂದು ಆಸೆಯಾಗಿದೆ. ನೀನು ಅವರಿಗೆ ಬರೆಯುವೆಯಾ?

ಶಿಷ್ಯ: ಖಂಡಿತವಾಗಿ ಬರೆಯುವೆ. ನಿಮ್ಮ ವಿಷಯ ಕೇಳಿದಾಗಲೆಲ್ಲ ಅವರು ಸಂತೋಷದಿಂದ ಉನ್ಮತ್ತರಾಗುವರು. ಪೂರ್ವಬಂಗಾಳ ತಮ್ಮ ಪಾದಧೂಳಿಯಿಂದ ಪವಿತ್ರವಾಗುವುದೆಂದು ಹೇಳುವರು.

ಸ್ವಾಮೀಜಿ: ನಿನಗೆ ಗೊತ್ತೆ? ಶ‍್ರೀರಾಮಕೃಷ್ಣರು ನಾಗಮಹಾಶಯರನ್ನು ‘ಅಗ್ನಿಜ್ವಾಲೆ’ ಎನ್ನುತ್ತಿದ್ದರು.

ಶಿಷ್ಯ: ಹೌದು, ಹಾಗೆಂದು ನಾನು ಕೇಳಿರುವೆ.

ಸ್ವಾಮಿಜಿಯವರ ಅಪ್ಪಣೆಯಂತೆ ಶಿಷ್ಯ ಸ್ವಲ್ಪ ಪ್ರಸಾದ ಸ್ವೀಕರಿಸಿ, ಸಂಜೆ ಹೊತ್ತಾಗಿ ಕಲ್ಕತ್ತೆಗೆ ಹಿಂದಿರುಗಿದನು. ಸ್ಫೂರ್ತಿಯುತ ಗುರುಮುಖದಿಂದ ಕೇಳಿದ “ನಾನು ಮುಕ್ತ, ನಾನು ಮುಕ್ತ" ಎಂಬ ಅಭಯವಾಣಿಯನ್ನೇ ಕುರಿತು ಗಾಢವಾಗಿ ಚಿಂತಿಸುತ್ತಿದ್ದನು.

\newpage

\chapter[ಅಧ್ಯಾಯ ೨೫]{ಅಧ್ಯಾಯ ೨೫\protect\footnote{\engfoot{C.W, Vol. V, P. 396}}}

\begin{center}
ಸ್ಥಳ: ಬೇಲೂರು ಮಠ (ಬಾಡಿಗೆ ಕಟ್ಟಡ), ವರ್ಷ: ಕ್ರಿ.ಶ. ೧೮೯೮.
\end{center}

ಶಿಷ್ಯ: ಸ್ವಾಮೀಜಿ, ಕಾಮಕಾಂಚನವನ್ನು ಸಂಪೂರ್ಣವಾಗಿ ವರ್ಜಿಸಿದ ಹೊರತು ವ್ಯಕ್ತಿಯು ಆಧ್ಯಾತ್ಮಿಕ ಸಾಧನೆಯಲ್ಲಿ ಬಹುದೂರ ಮುಂದುವರಿಯಲಾರ ಎಂದು ಶ‍್ರೀರಾಮಕೃಷ್ಣರು ಹೇಳುತ್ತಿದ್ದರು. ಹಾಗಿದ್ದರೆ ಗೃಹಸ್ಥರ ಗತಿಯೇನು? ಅವರ ಮನಸ್ಸು ಈ ಎರಡು ವಸ್ತುಗಳ ಮೇಲೆ ನಿಂತಿದೆ.

ಸ್ವಾಮೀಜಿ: ಗೃಹಸ್ಥನಾಗಲಿ ಸಂನ್ಯಾಸಿಯಾಗಲಿ ಕಾಮಕಾಂಚನದ ಮೇಲಿನ ವ್ಯಾಮೋಹ ಸಂಪೂರ್ಣವಾಗಿ ತೊಲಗಿಹೋಗುವವರೆಗೂ ಮನಸ್ಸು ಖಂಡಿತವಾಗಿಯೂ ದೇವರ ಕಡೆ ತಿರುಗುವುದಿಲ್ಲ. ಇದನ್ನು ಖಂಡಿತವಾಗಿ ತಿಳಿ. ಎಲ್ಲಿಯವರೆಗೆ ಮನಸ್ಸು ಇವುಗಳಲ್ಲೇ ಮುಳುಗಿರುತ್ತದೆಯೋ ಅಲ್ಲಿಯವರೆಗೆ ನಿಜವಾದ ಭಕ್ತಿ, ಸ್ಥಿರಚಿತ್ತ ಮತ್ತು ಶ್ರದ್ಧೆ ಎಂದಿಗೂ ಬರುವುದಿಲ್ಲ.

ಶಿಷ್ಯ: ಹಾಗಾದರೆ ಗೃಹಸ್ಥರ ಗತಿಯೇನು? ಅವರು ಯಾವ ಮಾರ್ಗವನ್ನನುಸರಿಸಬೇಕು?

ಸ್ವಾಮೀಜಿ: ಸಣ್ಣಪುಟ್ಟ ಆಸೆಗಳನ್ನು ತೃಪ್ತಿಪಡಿಸಿ ಮತ್ತೆ ಅವುಗಳ ತಂಟೆಗೆ ಹೋಗದಿರುವುದು ಮತ್ತು ದೊಡ್ಡ ಆಸೆಗಳನ್ನು ವಿಚಾರದ ಮೂಲಕ ವರ್ಜಿಸುವುದು - ಇದೇ ಸರಿಯಾದ ಮಾರ್ಗ. ತ್ಯಾಗವಿಲ್ಲದೆ ದೇವರ ಸಾಕ್ಷಾತ್ಕಾರ ಎಂದಿಗೂ ಸಾಧ್ಯವಿಲ್ಲ. ಬ್ರಹ್ಮನೇ ಬಂದು ಬೇರೆ ರೀತಿ ಅಪ್ಪಣೆ ಮಾಡಿದರೂ ಅಸಾಧ್ಯ!

- ಶಿಷ್ಯ: ಸಂನ್ಯಾಸಿಯಾದ ಕೂಡಲೇ ಎಲ್ಲಾ ವಸ್ತುಗಳ ಮೇಲಿನ ಮೋಹವೂ ಹೋಗುತ್ತದೆಯೆ?

ಸ್ವಾಮೀಜಿ: ಸಂನ್ಯಾಸಿಗಳು ಕಡೆಗೆ ತ್ಯಾಗಜೀವನಕ್ಕಾದರೂ ಸಿದ್ಧತೆ ಮಾಡಿಕೊಳ್ಳಲು ಹೆಣಗಾಡುತ್ತಿದ್ದಾರೆ. ಗೃಹಸ್ಥರಾದರೊ ಈ ವಿಚಾರದಲ್ಲಿ ಲಂಗರು ಕಟ್ಟಿದ ದೋಣಿಯನ್ನು ಹುಟ್ಟುಹೊಡೆಯುವವರ ಗುಂಪಿಗೆ ಸೇರಿದವರು. ಭೋಗಲಾಲಸೆಯನ್ನು ಎಂದಿಗಾದರೂ ಶಾಂತಗೊಳಿಸಲು ಸಾಧ್ಯವೆ? ಅದು ನಿರಂತರ ಹೆಚ್ಚುತ್ತಾ ಹೋಗುವುದು.

ಶಿಷ್ಯ: ಹಾಗೇಕೆ? ಬಹುಕಾಲ ವಿಷಯ ಸುಖವನ್ನು ಅನೇಕ ಬಾರಿ ಅನುಭವಿಸಿದ ನಂತರ ಪ್ರಪಂಚದ ಮೇಲೆ ಜುಗುಪ್ಸೆ ಹುಟ್ಟಬಹುದಲ್ಲವೆ?

ಸ್ವಾಮೀಜಿ: ಆದರೆ ಎಷ್ಟು ಮಂದಿಗೆ ಈ ಭಾವನೆ ಬರುವುದು? ಬಹುಕಾಲ ವಿಷಯ ವಸ್ತುಗಳ ಸಂಪರ್ಕದಿಂದ ಮನಸ್ಸು ಕಲುಷಿತವಾಗಿ ಅದರ ಮುದ್ರೆ ಚಿರಮುದ್ರಿತವಾಗುವುದು. ತ್ಯಾಗ, ಸಂಪೂರ್ಣ ತ್ಯಾಗವೇ ಎಲ್ಲಾ ಸಾಕ್ಷಾತ್ಕಾರಗಳ ಮೂಲಮಂತ್ರದ ನಿಜವಾದ ರಹಸ್ಯ.

ಶಿಷ್ಯ: ಆದರೆ ಶಾಸ್ತ್ರಗಳಲ್ಲಿ ಈ ಕೆಲವು ನಿಯಮಗಳಿವೆಯಲ್ಲ? ‘ಗೃಹೇಷು ಪಂಚೇಂದ್ರಿಯನಿಗ್ರಹಸ್ತಪಃ’ - ಹೆಂಡತಿ ಮಕ್ಕಳೊಡನೆ ಇರುವಾಗ ಪಂಚೇಂದ್ರಿಯಗಳನ್ನು ಸಂಯಮ ಮಾಡುವುದೇ ತಪಸ್ಸು. ನಿವೃತ್ತರಾಗಸ್ಯ ಗೃಹಂ ತಪೋವನಂ' - ಯಾರು ಇಂದ್ರಿಯ ನಿಗ್ರಹ ಮಾಡಿರುವರೋ ಅವರಿಗೆ ಸಂಸಾರ ಮಧ್ಯದಲ್ಲಿರುವುದೂ, ಕಾಡಿನಲ್ಲಿ ತಪಸ್ಸನ್ನಾಚರಿಸುವುದೂ ಎರಡೂ ಒಂದೇ.

ಸ್ವಾಮಿಜಿ: ಯಾರು ಸಂಸಾರ ಮಧ್ಯದಲ್ಲಿದ್ದುಕೊಂಡು ಕಾಮಿನಿಕಾಂಚನ ತ್ಯಾಗ ಮಾಡಲು ಸಮರ್ಥರಾಗಿರುವರೋ ಅವರು ನಿಜವಾಗಿಯೂ ಧನ್ಯರು. ಆದರೆ ಎಷ್ಟು ಮಂದಿ ಅದನ್ನು ಮಾಡಬಲ್ಲರು?

ಶಿಷ್ಯ: ಆದರೆ ಸಂನ್ಯಾಸಿಗಳ ವಿಷಯವೇನು? ಅವರು ಸಂಪೂರ್ಣವಾಗಿ ಕಾಮಕಾಂಚನ ತ್ಯಾಗ ಮಾಡಲು ಸಮರ್ಥರಾಗಿರುವರೇ?

ಸ್ವಾಮೀಜಿ: ನಾನು ಈಗತಾನೆ ಹೇಳಿದಂತೆ ಸಂನ್ಯಾಸಿಗಳು ತ್ಯಾಗ ಜೀವನವನ್ನಪ್ಪಿದ್ದಾರೆ. ಗುರಿಗಾಗಿ ಹೋರಾಡಲು ಅವರು ಆ ರಂಗಕ್ಕೆ ಬಂದಿದ್ದಾರೆ. ಆದರೆ ಗೃಹಸ್ಥರಿಗೆ ಕಾಮಿನಿಕಾಂಚನದಿಂದುಟಾಗುವ ಅಪಾಯ ಪರಿಜ್ಞಾನ ಇನ್ನೂ ಬಂದಿಲ್ಲ. ಆತ್ಮಸಾಕ್ಷಾತ್ಕಾರದ ಸಾಧನೆಗೆ ಕೂಡ ತೊಡಗುವುದಿಲ್ಲ. ಇವುಗಳ ತ್ಯಾಗಕ್ಕಾಗಿ ತಾವು ಹೋರಾಡಬೇಕಾಗುವುದೆಂಬ ಭಾವನೆ ಅವರ ಮನಸ್ಸಿಗೆ ಇನ್ನೂ ಬಂದಿಲ್ಲ.

ಶಿಷ್ಯ: ಆದರೆ ಅದಕ್ಕಾಗಿ ಬಹುಮಂದಿ ಹೆಣಗುತ್ತಿದ್ದಾರೆ.

ಸ್ವಾಮೀಜಿ: ಅದು ನಿಜ. ಯಾರು ಆ ರೀತಿ ಮಾಡುತ್ತಿರುವರೋ ಅವರು ಖಂಡಿತವಾಗಿ ಕ್ರಮೇಣ ತ್ಯಜಿಸುವರು. ಅವರ ಕಾಮಿನಿಕಾಂಚನದ ತೀವ್ರ ಮೋಹ ಕ್ರಮೇಣ ಕಡಿಮೆಯಾಗುವುದು. ಆದರೆ ಯಾರು ‘ಇಷ್ಟು ಬೇಗ ಬೇಡ; ಕಾಲ ಬಂದಾಗ ಅದನ್ನು ಮಾಡಿದರಾಯಿತು’ ಎಂದು ವಿಳಂಬ ಮಾಡುವರೋ ಅಂತಹವರಿಗೂ ಆತ್ಮಸಾಕ್ಷಾತ್ಕಾರಕ್ಕೂ ಬಹು ದೂರ. ‘ಈ ಕ್ಷಣ ನನಗೆ ಸತ್ಯ ಸಾಕ್ಷಾತ್ಕಾರವಾಗಬೇಕು. ಈ ಜನ್ಮದಲ್ಲಿ ಆಗಬೇಕು’ ಎಂಬುದು ವೀರರ ಮಾತುಗಳು. ಅಂತಹ ವೀರರು ಯಾವ ಗಳಿಗೆಯಲ್ಲಿ ಬೇಕಾದರೂ ತ್ಯಜಿಸಲು ಸಿದ್ಧರಾಗಿರುವರು. ಅಂತಹವರಿಗೆ ಶಾಸ್ತ್ರ ‘ನಿನಗೆ ಯಾವ ಕ್ಷಣ ವೈರಾಗ್ಯವುಂಟಾಗುವುದೋ ಆ ಕ್ಷಣ ಪ್ರಪಂಚವನ್ನು ಪರಿತ್ಯಜಿಸು’ ಎನ್ನುವುದು.

ಶಿಷ್ಯ: ಆದರೆ ಶ‍್ರೀರಾಮಕೃಷ್ಣರು ‘ದೇವರನ್ನು ಪ್ರಾರ್ಥಿಸಿದರೆ ಭಗವತ್ಕೃಪೆಯಿಂದ ಈ ಮೋಹವೆಲ್ಲಾ ಮಾಯವಾಗುವುವು’ ಎಂದು ಹೇಳುತ್ತಿದ್ದರಲ್ಲವೇ?

ಸ್ವಾಮಿಜಿ: ಹೌದು, ಅವನ ಕೃಪೆಯಿಂದ ಖಂಡಿತ ಸಾಧ್ಯ. ಆತನ ಕೃಪೆಗೆ ಪಾತ್ರರಾಗಬೇಕಾದರೆ ಮೊದಲು ನಾವೂ ಪವಿತ್ರರಾಗಬೇಕು. ಮನೋವಾಕ್ಕಾಯವಾಗಿ ಪವಿತ್ರರಾಗಿದ್ದರೆ ಮಾತ್ರ ಆತನ ಕೃಪಾಕಟಾಕ್ಷ ನಮ್ಮ ಮೇಲೆ ಬೀಳುವುದು.

ಶಿಷ್ಯ: ಆದರೆ ಯಾರು ಮನೋವಾಕ್ಕಾಯವಾಗಿ ತನ್ನ ಮನಸ್ಸನ್ನು ನಿಗ್ರಹ ಮಾಡುವನೋ ಅವನಿಗೆ ದೇವರ ಕೃಪೆಯ ಆವಶ್ಯಕತೆ ಏನಿದೆ? ಕೇವಲ ತನ್ನ ಸಾಧನೆಯಿಂದಲೇ ಅವನು ಆಧ್ಯಾತ್ಮಿಕ ಪ್ರಗತಿ ಹೊಂದಬಹುದು.

ಸ್ವಾಮೀಜಿ: ಯಾರು ಹೃತ್ಪೂರ್ವಕವಾಗಿ ತನ್ನೆಡೆಗೆ ಬರಲು ಹೋರಾಡುತ್ತಿರುವನೋ ಅವನ ಮೇಲೆ ದೇವರು ಅಪಾರ ಕರುಣೆ ಬೀರುವನು. ನೀನು ಸೋಮಾರಿಯಾಗಿ ಏನನ್ನೂ ಮಾಡದೇ ಇದ್ದರೆ ಅವನು ಎಂದಿಗೂ ಬರುವುದಿಲ್ಲವೆಂದು ನಿನಗೇ ತಿಳಿಯುವುದು.

ಶಿಷ್ಯ: ಎಲ್ಲರೂ ಒಳ್ಳೆಯವರಾಗಬೇಕೆಂದಿಚ್ಛಿಸುವರು. ಆದರೆ ಯಾವುದೋ ಒಂದು ಗಹನವಾದ ಕಾರಣದಿಂದ ಕೆಟ್ಟಹಾದಿ ಹಿಡಿಯುತ್ತಾರೆ. ಎಲ್ಲರೂ ಒಳ್ಳೆಯವರಾಗಲು, ಪೂರ್ಣತ್ವ ಪಡೆಯಲು, ದೇವರ ದರ್ಶನ ಹೊಂದಲು ಇಚ್ಛಿಸುತ್ತಾರಲ್ಲವೇ?

ಸ್ವಾಮೀಜಿ: ಯಾರು ಇದನ್ನು ಇಚ್ಛಿಸುವರೋ ಅವರಾಗಲೇ ಹೋರಾಡುತ್ತಿರುವರೆಂದು ಭಾವಿಸು. ಈ ಹೋರಾಟ ಮುಂದುವರಿಸುತ್ತಿದ್ದರೆ ದೇವರು ತನ್ನ ಕೃಪಾ ದೃಷ್ಟಿಯನ್ನು ಬೀರುವನು.

ಶಿಷ್ಯ: ಅವತಾರಪುರುಷರ ಚರಿತ್ರೆಯಲ್ಲಿ ಹಲವಾರು ಮಂದಿ ವ್ಯರ್ಥ ಜೀವನ ನಡೆಸಿಯೂ ಹೆಚ್ಚು ಶ್ರಮವಿಲ್ಲದೆ, ಹೆಚ್ಚು ಸಾಧನೆಯನ್ನು ಮಾಡದೆ ದೇವರ ದರ್ಶನ ಪಡೆದಿರುವುದನ್ನು ನೋಡುತ್ತೇವೆ. ಇದಕ್ಕೆ ಏನು ಕಾರಣ?

ಸ್ವಾಮೀಜಿ: ಹೌದು, ಆದರೆ ಅವರಲ್ಲಿ ಆಗಲೇ ಘೋರ ವ್ಯಾಕುಲ ಪ್ರಾಪ್ತವಾಗಿರಬೇಕು. ಬಹುಕಾಲ ಇಂದ್ರಿಯ ಭೋಗದಲ್ಲಿ ಮುಳುಗಿದ್ದ ಅವರಿಗೆ ಅವುಗಳ ಮೇಲೆ ತೀವ್ರ ಜುಗುಪ್ಸೆ ಹುಟ್ಟಿರಬೇಕು, ಶಾಂತಿಗಾಗಿ ಅವರು ತವಕಿಸುತ್ತಿರಬೇಕು. ದೈವಕಟಾಕ್ಷದಿಂದ ದೊರಕುವ ಆ ಶಾಂತಿ ಇಲ್ಲದೆ ಇನ್ನು ಅವರಿಗೆ ಒಂದು ಕ್ಷಣ ಈ ಜೀವನದಲ್ಲಿರುವುದೂ ಅಸಾಧ್ಯವೆಂದು ಮನವರಿಕೆಯಾಗಬೇಕು. ಆದ್ದರಿಂದ ದೇವರು ಅಂಥವರ ವಿಷಯದಲ್ಲಿ ದಯೆ ತೋರುವನು. ಅಂಥವರಲ್ಲಿ ತಾಮಸ ಸ್ವಭಾವ ನೇರವಾಗಿ ಸತ್ಯಕ್ಕೆ ತಿರುಗುವುದು.

ಶಿಷ್ಯ: ಯಾವ ಹಾದಿಯೇ ಆಗಲಿ ಆ ಮಾರ್ಗದಲ್ಲಿಯೇ ಅವರು ಸಾಕ್ಷಾತ್ಕಾರ ಪಡೆಯುವರೆಂದಾಯಿತಲ್ಲವೆ?

ಸ್ವಾಮೀಜಿ: ಹೌದು, ಏಕಾಗಬಾರದು? ಆದರೆ ಅರಮನೆಯನ್ನು ಹೀನ ಹಿಂಬಾಗಿಲಿನಿಂದ ಪ್ರವೇಶಿಸುವುದಕ್ಕಿಂತ ಮುಂದಿನ ಯೋಗ್ಯವಾದ ಹೆಬ್ಬಾಗಿಲಿನಿಂದ ಪ್ರವೇಶಿಸುವುದು ಒಳ್ಳೆಯದಲ್ಲವೆ?

ಶಿಷ್ಯ: ಅದೇನೋ ನಿಜ. ಆದರೂ ಕೇವಲ ದೇವರ ಕೃಪೆಯಿಂದ ಮಾತ್ರವೇ ಅವನನ್ನು ಪಡೆಯಲು ಸಾಧ್ಯ ಎಂಬುದು ಸಿದ್ಧಾಂತವಾಗಿದೆ.

ಸ್ವಾಮೀಜಿ: ಹೌದು, ಹಾಗೂ ಆಗಬಹುದು. ಆದರೆ ಅಂತಹವರು ಬಹಳ ವಿರಳ.

ಶಿಷ್ಯ: ಯಾರು ಇಂದ್ರಿಯ ನಿಗ್ರಹ, ಕಾಮ ಕಾಂಚನ ತ್ಯಾಗ ಮಾಡಿ ದೇವರ ಸಾಕ್ಷಾತ್ಕಾರ ಪಡೆಯಲು ಹೊರಟಿರುವರೋ ಅವರು ಹೆಚ್ಚಾಗಿ ಸ್ವಂತ ಶ್ರಮ, ಸ್ವಪ್ರಯತ್ನಕ್ಕೆ ಹೆಚ್ಚು ಪ್ರಾಶಸ್ತ್ಯ ಕೊಟ್ಟಿರುವವರ ಗುಂಪಿಗೆ ಸೇರಿದವರೆಂದು ನನಗೆ ಅನ್ನಿಸುವುದು. ಯಾರು ಭಗವನ್ನಾಮಸ್ಮರಣೆ ಮಾಡಿ ಆತನನ್ನೇ ಶರಣು ಹೋಗಿರುವರೋ ಅವರನ್ನು ದೇವರೇ ಈ ಪ್ರಾಪಂಚಿಕ ಮೋಹದಿಂದ ಬಿಡುಗಡೆ ಮಾಡಿ ಆ ಪರಮ ಗುರಿಯಾದ ಸಾಕ್ಷಾತ್ಕಾರದ ಕಡೆಗೆ ಕರೆದೊಯ್ಯುವನು.

ಸ್ವಾಮೀಜಿ: ನಿಜ. ಇವೆರಡೂ ಭಿನ್ನ ದೃಷ್ಟಿಗಳು. ಮೊದಲನೆಯದು ಜ್ಞಾನಿಯ ದೃಷ್ಟಿ, ಎರಡನೆಯದು ಭಕ್ತನ ದೃಷ್ಟಿ. ಎರಡರಲ್ಲೂ ತ್ಯಾಗಕ್ಕೆ ಹೆಚ್ಚು ಪ್ರಾಮುಖ್ಯ.

ಶಿಷ್ಯ: ಅದರಲ್ಲಿ ಸಂದೇಹವೇ ಇಲ್ಲ. ಆದರೆ ಒಮ್ಮೆ ಶ‍್ರೀಯುತ ಗಿರೀಶಚಂದ್ರ ಘೋಷರು ದೇವರ ಕರುಣೆಗೆ ಯಾವ ನಿರ್ಬಂಧವೂ ಇಲ್ಲವೆಂದು ಹೇಳಿದರು. ಅದಕ್ಕೊಂದು ಕಾಯಿದೆ ಇಲ್ಲ! ಇದ್ದಿದ್ದರೆ ಅದನ್ನು ‘ಕರುಣೆ’ ಎನ್ನಲಾಗುತ್ತಿರಲಿಲ್ಲ. ಕರುಣೆ ಅಥವಾ ದಯೆಯ ರಾಜ್ಯ ಎಲ್ಲಾ ಕಟ್ಟುಗಳನ್ನೂ ಮೀರಿದುದು.

ಸ್ವಾಮಿಜಿ: ಆದರೆ ಗಿರೀಶರು ಹೇಳಿದ ಆ ರಾಜ್ಯದಲ್ಲೂ ನಮಗೆ ತಿಳಿಯದಿರುವ ಬೇರೆ ಮಹತ್ತಾದ ಕಾಯಿದೆಗಳಿದ್ದೇ ಇವೆ. ಅದು ಚರಮ ಸೀಮೆಗೆ ಸೇರಿದುದು. ಕಾಲ ದೇಶ ನಿಮಿತ್ತಾತೀತವಾದುದು. ನಾವು ಅಲ್ಲಿಗೆ ಹೋದಾಗ ಕಾರಣಗಳೇ ಇಲ್ಲದ ಸ್ಥಳದಲ್ಲಿ ಯಾರು ಕರುಣಾಮಯ? ಕರುಣೆ ತೋರುವುದು ಯಾರಿಗೆ? ಅಲ್ಲಿ ಭಗವಂತ ಭಕ್ತ; ಧ್ಯಾನಿ, ಧ್ಯೇಯ; ಜ್ಞಾನಿ, ಜ್ಞೇಯ ಎಲ್ಲಾ ಒಂದಾಗಿರುವುದು. ಅದನ್ನು ಬ್ರಹ್ಮ ಅಥವಾ ಕೃಪೆ, ಹೇಗೆ ಬೇಕಾದರೂ ಕರೆ. ಅದೆಲ್ಲ ಸಮರಸವಾಗಿರುವುದು.

ಶಿಷ್ಯ: ನಿಮ್ಮಿಂದ ಈ ಮಾತನ್ನು ಕೇಳಿದ ಮೇಲೆ ಸಕಲ ವೇದ ವೇದಾಂತಗಳ ಸಾರವನ್ನು ತಿಳಿದಂತಾಯಿತು. ಇಂದಿನವರೆಗೂ ನಾನು ಯಾವುದೋ ಅರ್ಥವಿಲ್ಲದ ಶಬ್ದ ಜಾಲದಲ್ಲಿ ಜೀವಿಸುತ್ತಿದ್ದೆ ಎನ್ನಿಸುತ್ತದೆ.

\newpage

\chapter[ಅಧ್ಯಾಯ ೨೬]{ಅಧ್ಯಾಯ ೨೬\protect\footnote{\engfoot{Complete Works of Swami Vivekananda, Volume VI, Page 445}}}

\begin{center}
ಸ್ಥಳ: ಬೇಲೂರು ಮಠ (ಬಾಡಿಗೆ ಕಟ್ಟಡ), ವರ್ಷ: ಕ್ರಿ.ಶ. ೧೮೯೮.
\end{center}

ಶಿಷ್ಯ: ಸ್ವಾಮಿಜಿ, ದಯವಿಟ್ಟು ನನಗೆ ಹೇಳಿ, ನಾವು ಸೇವಿಸುವ ಆಹಾರಕ್ಕೂ ಮಾನವನ ಆಧ್ಯಾತ್ಮಿಕ ಬೆಳವಣಿಗೆಗೂ ಏನಾದರೂ ಸಂಬಂಧವಿದೆಯೇ?

ಸ್ವಾಮೀಜಿ: ಹೌದು, ಹೆಚ್ಚು ಕಡಿಮೆ ಇದೆ ಎನ್ನಬಹುದು.

ಶಿಷ್ಯ: ಮೀನು, ಮಾಂಸವನ್ನು ತೆಗೆದುಕೊಳ್ಳುವುದು ಸರಿಯೆ? ಆವಶ್ಯಕವೆ?

ಸ್ವಾಮಿಜಿ: ಓಹೊ! ಇದಕ್ಕೇನಂತೆ, ತೆಗೆದುಕೊ ಮಗು! ಅದರಿಂದೇನಾದರೂ ಹಾನಿಯುಂಟಾದರೆ ನಾನು ಅದರ ಜವಾಬ್ದಾರಿಯನ್ನು ವಹಿಸಿಕೊಳ್ಳುತ್ತೇನೆ. ನಮ್ಮ ದೇಶದ ಜನಸಾಮಾನ್ಯರ ಕಡೆ ನೋಡು. ಎಂತಹ ಒಣಮುಖ; ಹೃದಯ ಸತ್ತ್ವಹೀನ, ಉತ್ಸಾಹಶೂನ್ಯ, ಹೊಟ್ಟೆ ಮಾತ್ರ ದೊಡ್ಡದು, ಕೈಕಾಲುಗಳೆಲ್ಲಾ ಶಕ್ತಿಹೀನ - ಕ್ಷುದ್ರವಾದೊಂದು ಸನ್ನಿವೇಶವಾದರೂ ಸರಿಯೇ, ಭಯಪಟ್ಟು ಓಡಿಹೋಗುವ ಹೇಡಿಗಳ ಗುಂಪು!

ಶಿಷ್ಯ: ಮೀನು, ಮಾಂಸ ತಿನ್ನುವುದರಿಂದ ಶಕ್ತಿ ಬರುವುದೇ ಬೌದ್ಧಮತ ವೈಷ್ಣವ ಮತಗಳೇಕೆ “ಅಹಿಂಸೆಯೇ ಪರಮಧರ್ಮ" ಎಂದು ಸಾರುತ್ತವೆ?

ಸ್ವಾಮೀಜಿ: ಬೌದ್ಧಮತ ವೈಷ್ಣವ ಮತಗಳೆರಡೂ ಬೇರೆಬೇರೆಯಲ್ಲ. ಬೌದ್ಧ ಮತ ಹೀನಸ್ಥಿತಿಯಲ್ಲಿದ್ದಾಗ ಹಿಂದೂಮತ ಅದರಿಂದ ಕೆಲವು ಮುಖ್ಯ ನೀತಿಯ ತತ್ತ್ವಗಳನ್ನು ತೆಗೆದುಕೊಂಡು ಅದನ್ನು ತನ್ನದನ್ನಾಗಿ ಮಾಡಿಕೊಂಡು ವೈಷ್ಣವ ಧರ್ಮ ಎಂಬ ಹೆಸರು ಪಡೆಯಿತು. ಬೌದ್ಧರ ತತ್ತ್ವ “ಅಹಿಂಸೆಯೆ ಶ್ರೇಷ್ಠಮಾರ್ಗ" ಎಂಬುದು ಬಹಳ ಒಳ್ಳೆಯದೆ. ಆದರೆ ಬೌದ್ಧಮತ ಎಲ್ಲರಮೇಲೂ ಯೋಗ್ಯಾಯೋಗ್ಯತೆಯನ್ನು ವಿಮರ್ಶಿಸದೆ ಇದನ್ನು ಬಲಾತ್ಕರಿಸಿ ಭರತವರ್ಷವನ್ನೇ ಹಾಳುಮಾಡಿತು! ನಾನೂ ಹಿಂದೂ ದೇಶದಲ್ಲಿ ಅನೇಕ ಮಂದಿ ಈ ಬಗೆಯ ಧಾರ್ಮಿಕ ಆಷಾಢಭೂತಿಗಳನ್ನು ನೋಡಿದ್ದೇನೆ. ಅವರು ಇರುವೆಗಳಿಗೆ ಸಕ್ಕರೆಯನ್ನು ಹಾಕುತ್ತಾರೆ. ಆದರೆ ತುಚ್ಛವಾದ ಹಣದಾಸೆಗಾಗಿ ತಮ್ಮ ಸಹೋದರರ ರಕ್ತ ಹೀರಲು ಹಿಂದೆ ಮುಂದೆ ನೋಡುವುದಿಲ್ಲ.

ಶಿಷ್ಯ: ಆದರೆ ವೇದ ಮತ್ತು ಮನುಧರ್ಮಶಾಸ್ತ್ರದಲ್ಲಿ ಮೀನು, ಮಾಂಸ ತಿನ್ನಬೇಕೆಂದು ಹಲವು ಕಟ್ಟಳೆಗಳಿವೆಯಲ್ಲ?

ಸ್ವಾಮೀಜಿ: ಅದೂ ಇದೆ. ಕೊಲ್ಲಕೂಡದೆಂಬುದೂ ಇದೆ. ವೇದಗಳಲ್ಲಿ ‘ಯಾವ ಜೀವಿಗೂ ನೋವುಂಟುಮಾಡಬೇಡ’ ಎಂದು ಹೇಳಿದೆ. ಮನು ಕೂಡ, ‘ಆಸೆಗಳನ್ನು ತೊರೆಯುವುದರಿಂದ ಮಹತ್ತಾದ ಪರಿಣಾಮವುಂಟಾಗುವುದು’ ಎಂದಿದ್ದಾನೆ. ಹಿಂಸೆ ಅಹಿಂಸೆ ಎರಡೂ ವ್ಯಕ್ತಿಯು ಅವಲಂಬಿಸುವ ಸಾಧನೆಗನುಗುಣವಾಗಿ ಆತನ ಯೋಗ್ಯತೆ, ಸಾಮರ್ಥ್ಯ, ಹೊಂದಾಣಿಕೆಗನುಸಾರವಾಗಿ ವಿಧಿಸಲ್ಪಟ್ಟಿವೆ.

ಶಿಷ್ಯ: ಈಗಿನ ಕಾಲದಲ್ಲಿ ಧರ್ಮವನ್ನವಲಂಬಿಸಿದ ಕೂಡಲೇ ಮೀನು ಮಾಂಸವನ್ನು ಬಿಡುವುದೊಂದು ಸಂಪ್ರದಾಯವಾಗಿದೆ. ಕೆಲವರಿಗಂತೂ ಇದನ್ನು ಅನುಸರಿಸದಿದ್ದಲ್ಲಿ ವ್ಯಭಿಚಾರದಂತಹ ಘೋರ ಪಾತಕಕ್ಕಿಂತ ಇದು ದೊಡ್ಡಪಾತಕ ಎನ್ನುವಷ್ಟರಮಟ್ಟಿಗಾಗಿದೆ.

ಸ್ವಾಮೀಜಿ: ನಮ್ಮ ದೇಶ, ನಮ್ಮ ಸಮಾಜ, ಈ ಬಗೆಯ ನಂಬಿಕೆಗಳಿಂದ ಹಾಳಾಗಿ ಹೋಗುತ್ತಿರುವುದನ್ನು ನೋಡುತ್ತಿಲ್ಲವೆ? ಈಗ ಅವು ಹೇಗೆ ಬಂತೆಂದು ತಿಳಿಯುವುದರಿಂದ ಪ್ರಯೋಜನವೇನು? ನೋಡು, ಪೂರ್ವಬಂಗಾಳಿಗಳು ಮೀನು ಮಾಂಸ ಆಮೆ ಮುಂತಾದುವನ್ನು ತಿನ್ನುತ್ತಾರೆ. ಅವರು ಈ ಭಾಗದ ಬಂಗಾಳಿಗಳಿಗಿಂತ ಹೆಚ್ಚು ದೃಢಕಾಯರಾಗಿರುವರು. ಪೂರ್ವಬಂಗಾಳದ ದೊಡ್ಡ ದೊಡ್ಡ ಬಾಬುಗಳೂ ಕೂಡ ರಾತ್ರಿಯೂಟಕ್ಕೆ ಪೂರಿ ಚಪಾತಿಗಳನ್ನು ಉಪಯೋಗಿಸುತ್ತಿಲ್ಲ. ಅವರು ನಮ್ಮಂತೆ ಅಗ್ನಿಮಾಂದ್ಯ ಮುಂತಾದ ರೋಗಗಳಿಂದ ನರಳುತ್ತಿಲ್ಲ. ಪೂರ್ವಬಂಗಾಳದ ಹಳ್ಳಿಗಳಲ್ಲಿ ಈ ಅಜೀರ್ಣ ರೋಗದ ಹೆಸರೇ ಗೊತ್ತಿಲ್ಲವೆಂದು ಕೇಳಿದ್ದೇನೆ.

ಶಿಷ್ಯ: ಹೌದು ಸ್ವಾಮೀಜಿ, ನಮ್ಮ ಕಡೆ ಈ ಅಗ್ನಿಮಾಂದ್ಯ ರೋಗವೇ ಇಲ್ಲ. ಈ ಭಾಗಕ್ಕೆ ಬಂದಮೇಲೆಯೇ ನಾನು ಅದರ ವಿಚಾರ ಕೇಳಿದ್ದು. ನಾವು ಅನ್ನದೊಂದಿಗೆ ಮೀನನ್ನು ಬೆಳಿಗ್ಗೆ ಸಂಜೆ ಎರಡು ಹೊತ್ತೂ ತೆಗೆದುಕೊಳ್ಳುತ್ತೇವೆ.

ಸ್ವಾಮೀಜಿ: ಟೀಕೆಗಳಿಗೆ ಗಮನಕೊಡದೆ ನಿನಗಿಷ್ಟಬಂದದ್ದನ್ನು ತೆಗೆದುಕೊ. ದೇಶವೆಲ್ಲಾ ಒಂದೇ ತರಕಾರಿಯಲ್ಲಿ ಜೀವಿಸುವ, ಅಜೀರ್ಣದಿಂದ ನರಳುವ ಬಾಬಾಜಿಗಳಿಂದ ತುಂಬಿಹೋಗಿದೆ. ಸತ್ತ್ವಗುಣದ ಚಿಹ್ನೆಯೇ ಇಲ್ಲ. ಬರೇ ತಮಸ್ಸು ಮೃತ್ಯುವಿನ ಛಾಯೆ. ನಗುಮುಖ, ಅಭಯ, ಉತ್ಸಾಹ, ತೃಪ್ತಿ, ಉತ್ಕಟ ಚಟುವಟಿಕೆ ಇವೆಲ್ಲಾ ಸತ್ತ್ವಗುಣದ ಪರಿಣಾಮ. ಸೋಮಾರಿತನ, ಆಲಸ್ಯ, ಉತ್ಕಟ ಮೋಹ, ನಿದ್ರೆ ಇವೆಲ್ಲಾ ತಾಮಸದ ಚಿಹ್ನೆ.

ಶಿಷ್ಯ: ಆದರೆ ಮಾಂಸ ಮೀನು ಮನುಷ್ಯನ ರಾಜಸ ಸ್ವಭಾವವನ್ನು ಉದ್ರೇಕಗೊಳಿಸುವುದಿಲ್ಲವೆ?

ಸ್ವಾಮೀಜಿ: ಆ ಸ್ವಭಾವ ನಿಮಗೆಲ್ಲಾ ಬರಲೆಂದೇ ನಾನು ಆಶಿಸುವುದು. ಈಗ ರಾಜಸ ಸ್ವಭಾವ ಬಹಳ ಬೇಕಾಗಿದೆ. ಸತ್ತ್ವಗುಣಿಗಳೆಂದು ತಿಳಿದಿರುವವರಲ್ಲಿ ಶೇಕಡ ತೊಂಬತ್ತು ಮಂದಿ ಘೋರ ತಾಮಸದಿಂದ ಆವೃತರಾಗಿದ್ದಾರೆ! ಅವರಲ್ಲಿ ಹದಿನಾರರಲ್ಲಿ ಒಂದು ಪಾಲು ಸತ್ತ್ವಗುಣಿಗಳು ಸಿಕ್ಕಿದರೆ ಹೆಚ್ಚು. ತೀವ್ರವಾದ ರಾಜಸಿಕ ಶಕ್ತಿಯ ಉದ್ಧೀಪನವಾಗಬೇಕು. ಇಡೀ ದೇಶವೆಲ್ಲ ತಮೋಗುಣದಿಂದ ಆಚ್ಛಾದಿತವಾಗಿದೆ. ದೇಶಾದ್ಯಂತ ಜನರಿಗೆಲ್ಲಾ ಆಹಾರ, ಬಟ್ಟೆ ಬೇಕು - ದೇಶ ಜಾಗೃತಗೊಳ್ಳಬೇಕು. ಚಟುವಟಿಕೆಯಿಂದ ಕೆಲಸ ಮಾಡುವಂತೆ ಪ್ರೇರೇಪಿಸಬೇಕು. ಇಲ್ಲದಿದ್ದರೆ ಅವರು ಮತ್ತಷ್ಟು ಜಡರಾಗಿ ಮರ ಕಲ್ಲುಗಳಾಗುತ್ತಾರೆ. ಆದ್ದರಿಂದಲೇ ಹೆಚ್ಚು ಮೀನು ಮಾಂಸ ತಿನ್ನಬೇಕೆಂದು ನಾನು ಹೇಳುವುದು.

ಶಿಷ್ಯ: ಸತ್ತ್ವಗುಣ ವೃದ್ಧಿಗೊಳಿಸಿಕೊಂಡ ಮೇಲೂ ಈ ಮೀನು ಮಾಂಸದ ಬಯಕೆ ಇದ್ದೇ ಇರುತ್ತದೆಯೆ?

ಸ್ವಾಮೀಜಿ: ಇಲ್ಲ, ಇರುವುದಿಲ್ಲ. ಪವಿತ್ರವಾಗಿ ಪೂರ್ಣ ಸತ್ತ್ವಶಾಲಿಯಾದಾಗ ಈ ಮೀನು ಮಾಂಸದ ಬಯಕೆ ಎಲ್ಲಾ ಮಾಯವಾಗುವುದು. ಅದು ಒಂದು ವ್ಯಕ್ತಿಯಲ್ಲಿ ಆವಿರ್ಭವಿಸಿದಾಗ ಅವನಲ್ಲಿ: ಇತರರಿಗಾಗಿ ಸರ್ವಸ್ವ ತ್ಯಾಗ, ಕಾಮಿನಿ ಕಾಂಚನದಲ್ಲಿ ಸಂಪೂರ್ಣ ವಿರಕ್ತಿ, ಹೆಮ್ಮೆ ಅಹಂಭಾವಗಳ ಅಭಾವ ಈ ಗುಣಗಳೆಲ್ಲಾ ಇರುವುವು. ಯಾವಾಗ ಈ ಗುಣಗಳೆಲ್ಲಾ ಮನುಷ್ಯನಲ್ಲಿ ಕಾಣಬರುವುದೋ ಆಗ ಮಾಂಸಾಹಾರದ ಆಸೆ ಹೋಗುವುದು. ಈ ಚಿಹ್ನೆಗಳೊಂದೂ ಇಲ್ಲದೆ ಅಹಿಂಸೆಯ ಪರವಾಗಿ ಜನರು ಮಾತನಾಡುತ್ತಿದ್ದಾರೆ. ಆಗ ಅದೆಲ್ಲಾ ಕೇವಲ ಕಾಪಟ್ಯ, ಧರ್ಮದ ಸೋಗೆಂದು ತಿಳಿ. ನೀನು ಶುದ್ಧ ಸಾತ್ತ್ವಿಕನಾದ ಮೇಲೆ ಮೀನು ಮಾಂಸವನ್ನು ಖಂಡಿತವಾಗಿಯೂ ಬಿಡು.

ಶಿಷ್ಯ: ಆದರೆ ಛಾಂದೋಗ್ಯ ಉಪನಿಷತ್ತಿನಲ್ಲಿ ಪವಿತ್ರವಾದ ಆಹಾರದಿಂದ ಮನುಷ್ಯನಲ್ಲಿರುವ ಸತ್ತ್ವಗುಣ ವೃದ್ಧಿಯಾಗುವುದು ಎಂಬ ಹೇಳಿಕೆ ಇದೆಯಲ್ಲ.

ಸ್ವಾಮೀಜಿ: ಹೌದು; ನನಗೆ ಅದು ಗೊತ್ತಿದೆ. ಶಂಕರಾಚಾರ್ಯರು ಆಹಾರವೆಂಬ ಮಾತಿಗೆ “ಇಂದ್ರಿಯ ವಿಷಯಗಳು" ಎಂದು ಅರ್ಥ ಮಾಡುತ್ತಾರೆ. ರಾಮಾನುಜಾಚಾರ್ಯರು ಆಹಾರವೆಂಬ ಪದಕ್ಕೆ ಆಹಾರವೆಂದೇ ತೆಗೆದುಕೊಳ್ಳುತ್ತಾರೆ. ನನಗೆ ತೋರುವುದೇನೆಂದರೆ ನಾವು ಇವೆರಡೂ ಅರ್ಥಗಳನ್ನು ಹೊಂದಿರುವ ಪದ ತೆಗೆದುಕೊಳ್ಳಬೇಕೆಂದು. ಒಬ್ಬನು ಜೀವಮಾನವನ್ನೆಲ್ಲಾ ಒಳ್ಳೆಯ ಆಹಾರ ಕೆಟ್ಟ ಆಹಾರ ಇವುಗಳ ವಾದ ವಿವಾದಗಳಲ್ಲೇ ಕಳೆಯಬೇಕೊ? ಅಥವಾ ಇಂದ್ರಿಯನಿಗ್ರಹವನ್ನು ಮಾಡುತ್ತಿರಬೇಕೊ? ಖಂಡಿತವಾಗಿ ಇಂದ್ರಿಯ ನಿಗ್ರಹ ಮುಖ್ಯ. ಒಳ್ಳೆಯ ಅಥವಾ ಕೆಟ್ಟ, ಪವಿತ್ರ ಅಥವಾ ಅಪವಿತ್ರವಾದ ಆಹಾರ ಇವುಗಳ ತಾರತಮ್ಯವು ಗುರಿಯನ್ನು ಸೇರುವುದಕ್ಕೆ ಸ್ವಲ್ಪಮಟ್ಟಿಗೆ ಮಾತ್ರ ಸಹಾಯ ಮಾಡುತ್ತವೆ. ಶಾಸ್ತ್ರಗಳ ಪ್ರಕಾರ ಮೂರು ವಿಷಯಗಳು ನಮ್ಮ ಮನಸ್ಸನ್ನು ಕಲುಷಿತಗೊಳಿಸುತ್ತವೆ: (೧) ಜಾತಿದೋಷ - ಈರುಳ್ಳಿ, ಬೆಳುಳ್ಳಿಯಂತಹ ಉದ್ರೇಕಗೊಳಿಸುವ ದುರ್ವಾಸನೆಯಿಂದ ಕೂಡಿದುದು, ಇದರಿಂದ ಆಹಾರ ಕುಲುಷಿತವಾಗುವುದು. (೨) ನಿಮಿತ್ತದೋಷ - ಕ್ರಿಮಿಕೀಟ ಧೂಳು ಕೊಳೆ ಕಳಿತ ಆಹಾರದಿಂದುಂಟಾಗುವ ದೋಷ. ಉದಾಹರಣೆಗೆ - ಪೇಟೆಯಿಂದ ತಂದ ಮಿಠಾಯಿ ಮುಂತಾದುವು. (೩) ಆಶ್ರಯ ದೋಷ - ಆಹಾರವು ಕೆಟ್ಟವರಿಂದ ಬಂದರೆ ಅದರಿಂದುಂಟಾಗುವ ದೋಷ. ಉದಾಹರಣೆಗೆ, ದುಷ್ಟ ಅಥವಾ ದುರಾತ್ಮನು ಮುಟ್ಟಿದ ಪದಾರ್ಥ. ಒಂದು ಎರಡನೆ ಕಾರಣಗಳಿಂದ ದೂರವಾಗಲು ಪ್ರತ್ಯೇಕ ಗಮನ ಕೊಡಬೇಕು. ಆದರೆ ಈ ದೇಶದಲ್ಲಿ ಅವೆರಡಕ್ಕೂ ಹೆಚ್ಚಾಗಿ ಗಮನ ಕೊಡುವುದಿಲ್ಲ. ಮೂರನೆಯದಕ್ಕೇ ಹೊಡೆದಾಡುತ್ತೇವೆ. ಕೇವಲ ಯೋಗಿಯೊಬ್ಬನಿಗೆ ಮಾತ್ರ ಮೂರನೆಯದನ್ನು ವಿಚಾರಿಸಲು ಸಾಧ್ಯ. ಇಡೀ ದೇಶವೆಲ್ಲಾ ಕೇವಲ ‘ಮುಟ್ಟಬೇಡ, ಮುಟ್ಟಬೇಡ’ ಎಂಬ ಮಡಿಯವರ ಕ್ರಂದನ ಅಟಾಟೋಪದಿಂದ ತುಂಬಿಹೋಗಿದೆ. ಅವರ ಆ ಪ್ರತ್ಯೇಕ ಆವರಣದಲ್ಲಿ ಕೂಡ ಒಳ್ಳೆಯ ಕೆಟ್ಟ ಮನುಷ್ಯರ ತಾರತಮ್ಯ ಇಲ್ಲ. ಏಕೆಂದರೆ ಅವರು ಯಾರಾದರೂ ಆಗಲಿ ಕತ್ತಿನಲ್ಲಿ ಜನಿವಾರವಿದ್ದು ಬ್ರಾಹ್ಮಣನೆಂದರೆ ಸಾಕು ಅವರ ಕೈಯಲ್ಲಿ ತಾರತಮ್ಯವಿಲ್ಲದೆ ಆಹಾರ ಸ್ವೀಕರಿಸುತ್ತಿದ್ದರು. ಶ‍್ರೀರಾಮಕೃಷ್ಣ ಪರಮಹಂಸರು ಅವರಿಂದ ಆಹಾರವನ್ನು ಸ್ವೀಕರಿಸುತ್ತಿರಲಿಲ್ಲ. ಅನೇಕ ವೇಳೆ ಕೆಲವರು ಆಹಾರ ತಂದರೆಂದರೆ ಸ್ವೀಕರಿಸುತ್ತಲೇ ಇರಲಿಲ್ಲ - ಸರಿಯಾಗಿ ವಿಚಾರಿಸಿದಾಗ ಆ ತಂದಂತಹ ಮನುಷ್ಯ ಕಲುಷಿತನಾಗೇ ಇರುತ್ತಿದ್ದ. ಈಗ ವರ್ತಮಾನಕಾಲದಲ್ಲಿ ಧರ್ಮವು ಅಡಿಗೆಯ ಪಾತ್ರೆಯಲ್ಲಿ ಮಾತ್ರ ಇದೆ - ಧರ್ಮದ ಮಹತ್ತಾದ ಸತ್ಯಗಳನ್ನು ಒಂದೆಡೆಯಿಟ್ಟು ನೀವು ಹೋರಾಡುತ್ತೀರಿ - ಕೇವಲ ಸಿಪ್ಪೆಗೆ, ತಿರುಳಿಗಲ್ಲ.

ಶಿಷ್ಯ: ಹಾಗಾದರೆ ನಾವು ಯಾರ ಹತ್ತಿರವೆಂದರೆ ಅವರ ಹತ್ತಿರ ಆಹಾರ ತಿನ್ನಬೇಕೇನು?

ಸ್ವಾಮೀಜಿ: ಏನೆಂದೆ? ಇಲ್ಲಿ ನೋಡು. ಉದಾಹರಣೆಗೆ ನೀವು ಪುರೋಹಿತ ವರ್ಗಕ್ಕೆ ಸೇರಿದವರು, ಎಲ್ಲಾ ವರ್ಗದ ಬ್ರಾಹ್ಮಣರು ಮಾಡಿದ ಅನ್ನವನ್ನೇಕೆ ಊಟ ಮಾಡುವುದಿಲ್ಲ? ರಾಹ್ರೀ ಪಂಥಕ್ಕೆ ಸೇರಿದ ನೀವು ವಾರೇಂದ್ರ ಪಂಥಕ್ಕೆ ಸೇರಿರುವವರಲ್ಲಿ ಏಕೆ ಊಟ ಮಾಡಬಾರದು? ಅಥವಾ ವಾರೇಂದ್ರರು ತಾನೇ ಏಕೆ ನಿಮ್ಮಲ್ಲಿ ಉಣ್ಣಲು ಆಕ್ಷೇಪಿಸಬೇಕು? ಪಶ್ಚಿಮ ಮತ್ತು ದಕ್ಷಿಣ ಇಂಡಿಯಾದಲ್ಲಿರುವ ಒಳಪಂಗಡಗಳು ಉದಾಹರಣೆಗೆ, ಮರಾಠಿ ತೆಲುಗರು ಕನೂಜಿಗಳು ಉಳಿದವರೂ ಏಕೆ ಹೀಗೆ ಮಾಡಬಾರದು? ಬಂಗಾಳದಲ್ಲಿ ನೂರಾರು ಮಂದಿ ಬ್ರಾಹ್ಮಣರು, ಕಾಯಸ್ಥರು ಎಂದು ಹೇಳಿಕೊಳ್ಳುವವರು, ಸಾರ್ವಜನಿಕ ಫಲಾಹಾರ ಮಂದಿರಗಳಿಗೆ ಗೋಪ್ಯವಾಗಿ ಹೋಗಿ ರುಚಿ ರುಚಿಯಾದ ತಿಂಡಿ ತಿಂದ ನಂತರ ಹೊರಕ್ಕೆ ಬಂದು ತಾವು ಸಮಾಜದ ನಾಯಕರೆಂದು ಹೇಳಿಕೊಂಡು ‘ಹೊಲೆಯನನ್ನು ಮುಟ್ಟಬಾರದು’ ಎಂಬ ತತ್ತ್ವಕ್ಕೆ ಉತ್ತೇಜನ ಕೊಡುತ್ತಿರುವುದನ್ನು ನೀನು ನೋಡುತ್ತಿಲ್ಲವೆ? ಇಂತಹ ಕಪಟಿಗಳು ಆಜ್ಞಾಪಿಸಿದ ಕಾಯಿದೆಗಳು ನಮ್ಮ ಸಮಾಜಕ್ಕೆ ಮಾರ್ಗದರ್ಶಕ ವಾಗಬೇಕೇನು? ಅದಕ್ಕೆ ಬದಲಾಗಿ ನಾವು ಅವರನ್ನು ಹೊಡೆದಟ್ಟಬೇಕು. ನಮ್ಮ ಮಹಾ ಋಷಿಗಳು ಮಾಡಿದ ಕಾನೂನುಗಳನ್ನು ಪುನಃ ತಂದು ಅವಕ್ಕೆ ಹೆಚ್ಚು ಪ್ರಾಮುಖ್ಯ ಕೊಡಬೇಕು. ಹೀಗೆ ಮಾಡಿದರೆ ಮಾತ್ರ ನಮ್ಮ ಜನಾಂಗ ಉದ್ಧಾರವಾಗುವುದು.

ಶಿಷ್ಯ: ಹಾಗಾದರೆ ಹಿಂದಿನ ಋಷಿಗಳು ಮಾಡಿದ ಕಾನೂನುಗಳೇ ಈಗ ನಮ್ಮ ಸಮಾಜಕ್ಕೆ ಮಾರ್ಗದರ್ಶಕಗಳಾಗಿಲ್ಲವೆ?

ಸ್ವಾಮೀಜಿ: ಎಂತಹ ಭ್ರಾಂತಿ! ಈಗಿನ ಕಾಲದಲ್ಲಿ ಹಾಗೆಲ್ಲಿ ಇದೆ? ನಾನು ಭರತಖಂಡವನ್ನೆಲ್ಲಾ ಸಂಚರಿಸುತ್ತಿರುವಾಗ ಎಷ್ಟೊಂದು ಎಚ್ಚರಿಕೆಯಿಂದ ಸಂಪೂರ್ಣವಾಗಿ ಹುಡುಕಿದರೂ ಎಲ್ಲಿಯೂ ಋಷಿಗಳ ನ್ಯಾಯಸೂತ್ರಗಳು ಬಳಕೆಯಲ್ಲಿದ್ದುದು ಕಾಣಲಿಲ್ಲ. ಮುಕ್ಕಾಲುಪಾಲು ವಾಡಿಕೆಯಲ್ಲಿರುವ ಅರ್ಥವಿಲ್ಲದ ಅಂಧ ಸಂಪ್ರದಾಯಗಳು, ಸ್ಥಳದ ಪಕ್ಷಪಾತ ಭಾವನೆಗಳು, ಹೆಂಗಸರಲ್ಲಿ ಹೆಚ್ಚು ವಾಡಿಕೆಯಲ್ಲಿರುವ ಪದ್ಧತಿಗಳು, ಸಂಸ್ಕಾರಗಳು, ಇವೇ ಸಮಾಜವನ್ನು ಎಲ್ಲೆಡೆಯಲ್ಲೂ ಆಳುತ್ತಿರುವುವು. ಅವರಲ್ಲಿ ಶಾಸ್ತ್ರಗಳನ್ನು ಓದಲು ಗಮನ ಕೊಡುವವರೆಷ್ಟು ಮಂದಿ? ಅಥವಾ ಎಚ್ಚರಿಕೆಯಿಂದ ಓದಿದ ಮೇಲೆ ಕಾಯಿದೆಗನುಸಾರ ಸಮಾಜವನ್ನು ನಡೆಸಬೇಕೆಂದು ಲಕ್ಷಿಸುವವರಾರು?

ಶಿಷ್ಯ: ಹಾಗಾದರೆ ನಾವೇನು ಮಾಡಬೇಕು?

ಸ್ವಾಮೀಜಿ: ನಾವು ಪೂರ್ವದ ಋಷಿಗಳ ಕಾಯಿದೆಗಳು ಪುನರುಜ್ಜೀವನಗೊಳ್ಳುವಂತೆ ಮಾಡಬೇಕು. ಮನು ಯಾಜ್ಞವಲ್ಕ್ಯರ ಧರ್ಮ ಶಾಸ್ತ್ರಗಳನ್ನು ಕಾಲಪರಿಸ್ಥಿತಿಗೆ ತಕ್ಕಂತೆ ಸ್ವಲ್ಪ ಬದಲಾವಣೆಮಾಡಿ ಜನಾಂಗವೆಲ್ಲಾ ಅದನ್ನು ಅನುಸರಿಸುವಂತೆ ಮಾಡಬೇಕು. ಭರತಖಂಡದಲ್ಲಿ ಎಲ್ಲಿಯೂ ಮೂಲ ಚಾತುರ್ವರ್ಣಗಳು ಕಾಣಬರುವುದಿಲ್ಲ. ನಾವು ಪುನಃ ಭರತಖಂಡದ ಜನರನ್ನೆಲ್ಲಾ ಹಿಂದಿನಂತೆ ಬ್ರಾಹ್ಮಣ, ಕ್ಷತ್ರಿಯ, ವೈಶ್ಯ, ಶೂದ್ರ ಎಂಬ ನಾಲ್ಕು ಮುಖ್ಯ ಪಂಗಡಗಳಾಗಿ ವಿಂಗಡಿಸಬೇಕು. ಇಂದಿನ ಬ್ರಾಹ್ಮಣರಲ್ಲಿರುವ ಲೆಕ್ಕವಿಲ್ಲದಷ್ಟು ಒಳಪಂಗಡಗಳೆಲ್ಲಾ ಅಳಿಸಿಹೋಗಿ ಬ್ರಾಹ್ಮಣರೆಲ್ಲಾ ಸೇರಿ ಒಂದೇ ಪಂಗಡವಾಗಬೇಕು. ಇತರ ಉಳಿದಿರುವ ಮೂರು ಪಂಗಡಗಳೂ ಹಾಗೇ ವೇದಗಳ ಕಾಲದಲ್ಲಿದ್ದಂತೆ ಒಂದೊಂದೇ ಪಂಗಡಗಳಾಗಿ ಮಾರ್ಪಡಬೇಕು. ಹೀಗೆ ಮಾಡಿದಲ್ಲದೆ ಕೇವಲ ‘ನಾವು ನಿಮ್ಮನ್ನು ಮುಟ್ಟುವುದಿಲ್ಲ’ ‘ಅವನನ್ನು ನಮ್ಮ ಜಾತಿಗೆ ಸೇರಿಸಿಕೊಳ್ಳುವುದಿಲ್ಲ’ ಎಂದು ಕೂಗಿಕೊಳ್ಳುವುದರಿಂದ ನಿನ್ನ ಮಾತೃಭೂಮಿಗೆ ಉಪಕಾರಮಾಡಿದಂತಾಗುವುದೇನು? ಎಂದಿಗೂ ಇಲ್ಲ.

\newpage

\chapter[ಅಧ್ಯಾಯ ೨೭]{ಅಧ್ಯಾಯ ೨೭\protect\footnote{\engfoot{C.W, Vol. VII, P. 172}}}

\begin{center}
ಸ್ಥಳ: ಬೇಲೂರು ಮಠ (ಬಾಡಿಗೆ ಕಟ್ಟಡ), ವರ್ಷ: ಕ್ರಿ.ಶ. ೧೮೯೯.
\end{center}

ಆಲಂಬಜಾರಿನಿಂದ ಬೇಲೂರಿನ ನೀಲಾಂಬರ ತೋಟಕ್ಕೆ ಮಠ ಬಂದ ಸ್ವಲ್ಪ ದಿನಕ್ಕೆ ಸ್ವಾಮೀಜಿ ಪರಮಹಂಸರ ಭಾವವನ್ನು ಜನಸಾಧಾರಣರಲ್ಲಿ ಪ್ರಚಾರ ಮಾಡುವುದಕ್ಕಾಗಿ ಬಂಗಾಳಿ ಭಾಷೆಯಲ್ಲಿ ಒಂದು ಪತ್ರಿಕೆಯನ್ನು ಹೊರಡಿಸಬೇಕೆಂದು ತಮ್ಮ ಗುರುಭ್ರಾತೃಗಳೊಡನೆ ಪ್ರಸ್ತಾವ ಮಾಡಿದರು. ಸ್ವಾಮೀಜಿ ಮೊದಲು ಒಂದು ದೈನಿಕ ಸಮಾಚಾರ ಪತ್ರದ ಪ್ರಸ್ತಾವವನ್ನೆತ್ತಿದರು. ಆದರೆ ಅದಕ್ಕೆ ಹೆಚ್ಚು ಹಣ ಬೇಕಾಗಿದ್ದರಿಂದ ಪಾಕ್ಷಿಕ ಪತ್ರಿಕೆಯನ್ನು ಹೊರಡಿಸುವುದೇ ಎಲ್ಲರಿಗೂ ಒಪ್ಪಿಗೆಯಾಗಿ, ತ್ರಿಗುಣಾತೀತ ಸ್ವಾಮಿಗಳ ಮೇಲೆ ಅದನ್ನು ನಡೆಸುವ ಭಾರವು ಬಿತ್ತು. ಸ್ವಾಮಿಗಳ ಹತ್ತಿರ ಒಂದು ಸಾವಿರ ರೂಪಾಯಿ ಇತ್ತು; ಪರಮಹಂಸರ ಗೃಹಸ್ಥ ಭಕ್ತರೊಬ್ಬರು ಮತ್ತೊಂದು ಸಾವಿರವನ್ನು ಸಾಲವಾಗಿ ಕೊಟ್ಟರು. ಈ ಹಣದಿಂದ ಕಾರ್ಯಾರಂಭವಾಯಿತು. ಒಂದು ಮುದ್ರಣ ಯಂತ್ರವನ್ನು ಕೊಂಡುಕೊಂಡು ಶ್ಯಾಮ ಬಜಾರಿನ ರಾಮಚಂದ್ರ ಮೈತ್ರ ಗಲ್ಲಿಯಲ್ಲಿರುವ ಶ‍್ರೀಯುತ ಗಿರೀಂದ್ರನಾಥ ಬಸಾಕರ ಮನೆಯಲ್ಲಿ ಮುದ್ರಣಾಲಯವನ್ನು ಇಟ್ಟಿದ್ದಾಯಿತು. ತ್ರಿಗುಣಾತೀತ ಸ್ವಾಮಿಜಿ ಹೀಗೆ ಕಾರ್ಯಭಾರವನ್ನು ಸ್ವೀಕರಿಸಿ ಕ್ರಿ.ಶ.೧೮೯೮ರ ಮಾಘಶುದ್ಧ ಪಾಡ್ಯಮಿಯ ದಿವಸ ಈ ಪತ್ರಿಕೆಯನ್ನು ಪ್ರಕಾಶಪಡಿಸಿದರು. ಸ್ವಾಮೀಜಿ ಈ ಪತ್ರಿಕೆಗೆ ‘ಉದ್ಭೋಧನ’ ಎಂಬ ಹೆಸರನ್ನು ಕೊಟ್ಟು ಅದರ ಅಭಿವೃದ್ಧಿಗೆ ತ್ರಿಗುಣಾತೀತರನ್ನು ಬಹುವಾಗಿ ಆಶೀರ್ವದಿಸಿದರು. ಸರ್ವಶ್ರಮಸಹಿಷ್ಣುಗಳಾದ ತ್ರಿಗುಣಾತೀತ ಸ್ವಾಮಿಗಳು ಸ್ವಾಮೀಜಿಯವರ ಅಪ್ಪಣೆ ಮೇರೆಗೆ ಅದರ ಮುದ್ರಣ ಮತ್ತು ಪ್ರಚಾರಕ್ಕಾಗಿ ಎಷ್ಟು ಪರಿಶ್ರಮಪಟ್ಟರೆಂಬುದನ್ನು ತಿಳಿಸುವುದಕ್ಕೆ ಮತ್ತೊಂದು ದೃಷ್ಟಾಂತವು ಹುಡುಕಿದರೂ ಸಿಕ್ಕುವುದು ಕಷ್ಟ. ಕೆಲವು ವೇಳೆ ಭಕ್ತ ಗೃಹಸ್ಥರಲ್ಲಿ ಭಿಕ್ಷಾನ್ನವನ್ನು ಪಡೆಯುತ್ತಲೂ, ಮತ್ತೆ ಕೆಲವು ವೇಳೆ ಕಾಲುನಡಿಗೆಯಲ್ಲಿ ಹತ್ತು ಹನ್ನೆರಡು ಮೈಲಿಗಳನ್ನು ನಡೆಯುತ್ತಲೂ ತ್ರಿಗುಣಾತೀತ ಸ್ವಾಮಿಗಳು ಈ ಪತ್ರಿಕೆಯ ಪ್ರಚಾರಕ್ಕಾಗಿ ಪ್ರಾಣವನ್ನಾದರೂ ಕೊಡುವುದಕ್ಕೆ ಹಿಂತೆಗೆಯದೆ ಕೆಲಸ ಮಾಡುತ್ತಿದ್ದರು. ಅಲ್ಲದೆ ಆಗ ಕೆಲಸ ಮಾಡುವವರನ್ನು ದುಡ್ಡು ಕೊಟ್ಟು ಇಟ್ಟುಕೊಳ್ಳುವುದಕ್ಕೆ ಅನುಕೂಲವಿರಲಿಲ್ಲ. ಪತ್ರಿಕೆಗಾಗಿ ಗೊತ್ತುಮಾಡಿ ಇಟ್ಟಿದ್ದ ಹಣದಲ್ಲಿ ಪತ್ರಿಕೆಗೆ ಹೊರತು ಮತ್ತಾವುದಕ್ಕೂ ಒಂದು ಕಾಸನ್ನೂ ಉಪಯೋಗಿಸಕೂಡದೆಂದು ಸ್ವಾಮಿಜಿ ಆಜ್ಞೆ ಮಾಡಿದ್ದರು. ಆದ್ದರಿಂದ ತ್ರಿಗುಣಾತೀತ ಸ್ವಾಮಿಗಳು ಭಕ್ತಾದಿಗಳ ಮನೆಯಲ್ಲಿ ಭಿಕ್ಷೆ ಗಿಕ್ಷೆಗಳನ್ನು ಮಾಡಿಕೊಂಡು ತಮ್ಮ ಹೊಟ್ಟೆಯ ಪಾಡನ್ನು ಹೇಗೊ ಕಳೆದುಕೊಳ್ಳುತ್ತ ಈ ಅಪ್ಪಣೆಯನ್ನು ಅಕ್ಷರಶಃ ನಡೆಸಿದರು.

ಪತ್ರಿಕೆಯ ಪ್ರಸ್ತಾವನೆಯನ್ನು ಸ್ವಾಮಿಜಿಯವರೇ ಸ್ವಂತವಾಗಿ ಬರೆದು ಕೊಟ್ಟರು; ಮತ್ತು ಸ್ವಾಮೀಜಿಯವರು ಪರಮಹಂಸರ ಸಂನ್ಯಾಸಿ ಮತ್ತು ಗೃಹಸ್ಥಭಕ್ತರೇ ಈ ಪತ್ರಿಕೆಗೆ ಲೇಖನಗಳನ್ನು ಬರೆಯಬೇಕೆಂದೂ, ಅಶ್ಲೀಲ ವ್ಯಂಜಕವಾದ ಯಾವ ಜಾಹಿರಾತು ಮುಂತಾದುವುಗಳನ್ನು ಈ ಪತ್ರಿಕೆಯಲ್ಲಿ ಹಾಕಕೂಡದೆಂದೂ ತಿಳಿಸಿದರು. ಸ್ವಾಮೀಜಿ ಸಂಘರೂಪವಾಗಿ ಪರಿಣಮಿಸಿದ ರಾಮಕೃಷ್ಣ ಮಿಷನ್‌ನ ಸದಸ್ಯರನ್ನು ಈ ಪತ್ರಿಕೆಗೆ ಲೇಖನಗಳನ್ನು ಬರೆಯುವಂತೆಯೂ, ಪತ್ರಿಕೆಯ ಸಹಾಯದಿಂದ ಪರಮಹಂಸರ ಧರ್ಮಸಂಬಂಧವಾದ ಅಭಿಪ್ರಾಯಗಳನ್ನು ಜನಸಾಮಾನ್ಯರಲ್ಲಿ ಹರಡುವಂತೆಯೂ ಹೇಳಿದರು. ಪತ್ರಿಕೆಯ ಮೊದಲನೆಯ ಸಂಚಿಕೆಯು ಪ್ರಕಟವಾದಾಗ ಶಿಷ್ಯನು ಒಂದು ದಿನ ಮಠಕ್ಕೆ ಹೋದನು. ಸ್ವಾಮೀಜಿಗೆ ನಮಸ್ಕಾರ ಮಾಡಿ ಕುಳಿತುಕೊಂಡಾಗ ಅವರು ಆತನೊಡನೆ ಉದ್ಭೋಧನದ ಸಂಬಂಧವಾಗಿ ಹೀಗೆ ಮಾತುಕತೆಗಳಿಗೆ ಆರಂಭ ಮಾಡಿದರು:

ಸ್ವಾಮೀಜಿ: (ಪತ್ರಿಕೆಯ ಹೆಸರನ್ನು ವಿಕಾರಪಡಿಸಿ, ಹಾಸ್ಯಕ್ಕೋಸ್ಕರ) “ಉದ್ಬಂಧನ"ವನ್ನು ನೋಡಿದೆಯೊ?

ಶಿಷ್ಯ: ನೋಡಿದೆ; ಚೆನ್ನಾಗಿದೆ.

ಸ್ವಾಮಿಜಿ: ಈ ಪತ್ರಿಕೆಯ ಭಾವ ಭಾಷೆ ಎಲ್ಲಾ ಹೊಸ ಮಾದರಿಯಲ್ಲಿ ಬರಬೇಕು.

ಶಿಷ್ಯ: ಹೇಗೆ?

ಸ್ವಾಮೀಜಿ: ಪರಮಹಂಸರ ಭಾವವನ್ನು ಎಲ್ಲರಿಗೂ ಕೊಡಲೇಬೇಕು; ಅಲ್ಲದೆ ಬಂಗಾಳಿ ಭಾಷೆಗೆ ನೂತನವಾದ ಓಜಸ್ಸನ್ನು ತಂದುಕೊಡಬೇಕು. ಹೇಗೆಂದರೆ ಹೆಚ್ಚಾಗಿ ಕ್ರಿಯಾಪದವನ್ನು ಉಪಯೋಗ ಮಾಡಿದರೆ ಭಾಷೆಯ ಸತ್ತ್ವ ಕಡಿಮೆಯಾಗಿಬಿಡುತ್ತದೆ. ವಿಶೇಷಣ ಬಳಸಿ ಕ್ರಿಯಾಪದದ ಉಪಯೋಗವನ್ನು ಕಡಿಮೆ ಮಾಡಬೇಕು. ನಾನು ಮೊದಲು ನೋಡಿ ಆಮೇಲೆ ಅಚ್ಚಿಗೆ ಕೊಡುತ್ತೇನೆ.

ಶಿಷ್ಯ: ಮಹಾಶಯರೇ, ತ್ರಿಗುಣಾತೀತ ಸ್ವಾಮಿಗಳು ಈ ಪತ್ರಿಕೆಗಾಗಿ ಪಡುತ್ತಿರುವ ಶ್ರಮ ಅನ್ಯರಿಗೆ ಸಾಧ್ಯವಿಲ್ಲ.

ಸ್ವಾಮೀಜಿ: ಪರಮಹಂಸರ ಈ ಸಂನ್ಯಾಸೀ ಸಂತಾನರೆಲ್ಲಾ ಸುಮ್ಮನೆ ಮರದ ಕೆಳಗೆ ಬೆಂಕಿಯನ್ನು ಹೊತ್ತಿಸಿಕೊಂಡು ಕುಳಿತುಕೊಳ್ಳುವುದಕ್ಕಾಗಿ ಹುಟ್ಟಿದ್ದಾರೆಂದು ನೀನು ತಿಳಿದುಕೊಂಡಿರುವ ಹಾಗಿದೆ. ಇವರಲ್ಲಿ ಯಾರು ಯಾವ ಕ್ಷೇತ್ರಕ್ಕೆ ಹೋದರೂ ಅಲ್ಲಿ ಅದರ ಉದ್ಯಮಶೀಲತೆಯನ್ನು ನೋಡಿ ಜನರು ಬೆರಗಾಗುವರು. ಕಾರ್ಯವನ್ನು ಹೇಗೆ ಮಾಡಬೇಕೆಂಬುದನ್ನು ಇವರಿಂದ ಕಲಿತುಕೊ. ನೋಡು, ನನ್ನ ಮಾತನ್ನು ನಡೆಸುವುದಕ್ಕಾಗಿ ತ್ರಿಗುಣಾತೀತನು ಸಾಧನಭಜನೆಗಳನ್ನೂ ಧ್ಯಾನ ಧಾರಣೆಗಳನ್ನೂ ಸಹ ಬಿಟ್ಟು ಕಾರ್ಯವನ್ನು ಕೈಗೊಂಡಿದ್ದಾನೆ. ಇದೇನು ಕಡಿಮೆ ತ್ಯಾಗದ ಮಾತೆ? ನನ್ನ ಮೇಲೆ ಎಷ್ಟು ವಿಶ್ವಾಸವಿಟ್ಟು ಈ ಕರ್ಮ ಪ್ರವೃತ್ತಿಯನ್ನು ತಂದುಕೊಂಡಿದ್ದಾನೆ ಬಲ್ಲೆಯಾ? ಕಾರ್ಯಸಿದ್ಧ ಮಾಡಿ ಆಮೇಲೆ ಬಿಡುತ್ತಾನೆ! ನಿನಗೆ ಇಷ್ಟೊಂದು ದಾರ್ಢ್ಯವಿದೆಯೆ?

ಶಿಷ್ಯ: ಆದರೆ ಮಹಾಶಯರೆ, ಕಾವಿಯನ್ನುಟ್ಟ ಸಂನ್ಯಾಸಿಗಳು ಹೀಗೆ ಗೃಹಸ್ಥರ ಮನೆಯ ಬಾಗಿಲು ಬಾಗಿಲಿಗೂ ಅಲೆಯುವುದು ನನಗೆ ಹೇಗೆ ಹೇಗೋ ಕಾಣುತ್ತದೆ.

ಸ್ವಾಮಿಜಿ: ಏಕೆ? ಪತ್ರಿಕೆಯ ಪ್ರಚಾರ ಗೃಹಿಗಳ ಕಲ್ಯಾಣಕ್ಕೋಸ್ಕರವೇ. ದೇಶದಲ್ಲಿ ಹೊಸ ಭಾವನೆಗಳನ್ನು ಪ್ರಚಾರಗೊಳಿಸುವುದರ ಮೂಲಕ ಜನಸಾಮಾನ್ಯರ ಕಲ್ಯಾಣ ಸಾಧಿತವಾಗಬೇಕು. ಫಲಾಕಾಂಕ್ಷೆಯಿಲ್ಲದೆ ಮಾಡುವ ಈ ಕರ್ಮವು ಸಾಧನಭಜನೆಗಳಿಗಿಂತ ಕಡಿಮೆಯೆಂದು ಯೋಚಿಸಿಕೊಂಡಿದ್ದೀಯೊ? ನಮ್ಮ ಉದ್ದೇಶವು ಜೀವಹಿತ ಸಾಧನ; ನಮಗೆ ಈ ಪತ್ರಿಕೆಯ ಆದಾಯದಿಂದ ದುಡ್ಡು ಕೂಡಿಹಾಕುವ ಯೋಚನೆ ಇಲ್ಲ. ನಾವು ಸರ್ವವನ್ನೂ ತ್ಯಾಗ ಮಾಡಿದ ಸಂನ್ಯಾಸಿಗಳು - ನಮಗೇನು ಹೆಂಡತಿ ಮಕ್ಕಳಿಲ್ಲ; ಅವರಿಗೋಸ್ಕರ ಏನಾದರೂ ಇಟ್ಟಿರಬೇಕೆನ್ನುವುದಕ್ಕೆ. ಕಾರ್ಯಸಾಧನೆ ಮತ್ತು ಸಂಪಾದನೆ. ಆದರೆ ಇದರ ಆದಾಯವೆಲ್ಲವೂ ಪ್ರಾಣಿಸೇವೆಗಾಗಿ ಉಪಯೋಗಿಸಲ್ಪಡುತ್ತದೆ. ಅಲ್ಲಲ್ಲಿ ಸಂಘಗಳ ಏರ್ಪಾಟು, ಸೇವಾಶ್ರಮ ಸ್ಥಾಪನೆ ಮತ್ತು ಇನ್ನೂ ನಾನಾ ಹಿತಕರವಾದ ಕಾರ್ಯಗಳಲ್ಲಿ ಹಣವು ಸದ್ವಿನಿಯೋಗವಾಗುತ್ತದೆ. ನಾವೇನು ಗೃಹಸ್ಥರ ಹಾಗೆ ಸ್ವಂತ ಸಂಪಾದನೆಯ ಉದ್ದೇಶವನ್ನಿಟ್ಟುಕೊಂಡು ಈ ಕೆಲಸ ಮಾಡುತ್ತಿಲ್ಲ. ಕೇವಲ ಪರಹಿತಕ್ಕಾಗಿಯೇ ನಮ್ಮ ಕಾರ್ಯ ಎಲ್ಲಾ - ಇದನ್ನು ತಿಳಿದುಕೊ.

ಶಿಷ್ಯ: ಹಾಗಿದ್ದರೂ ಎಲ್ಲರೂ ಈ ಭಾವವನ್ನು ಗ್ರಹಿಸಲಾರರು.

ಸ್ವಾಮೀಜಿ: ಗ್ರಹಿಸದಿದ್ದರೆ ಬಿಡಲಿ; ನಮಗೆ ಅದರಿಂದ ಬರುವುದೇನು, ಹೋಗುವುದೇನು? ನಾವು ನಿಂದೆ ಸ್ತುತಿ ಇವನ್ನು ಲೆಕ್ಕಮಾಡಿಕೊಂಡು ಕೆಲಸ ಮಾಡುವುದಿಲ್ಲ.

ಶಿಷ್ಯ: ಮಹಾಶಯರೆ, ಈ ಪತ್ರಿಕೆ ಹದಿನೈದು ದಿನಕ್ಕೊಂದು ಸಾರಿ ಬರುತ್ತದೆ; ಇದು ವಾರಪತ್ರಿಕೆಯಾಗಬೇಕೆಂದು ನನ್ನ ಇಚ್ಛೆ.

ಸ್ವಾಮೀಜಿ: ಅದೇನೋ ನಿಜ; ಆದರೆ ದುಡ್ಡು ಎಲ್ಲಿದೆ? ಪರಮಹಂಸರ ಇಚ್ಛೆಯಿಂದ ದುಡ್ಡು ಜೊತೆಯಾದರೆ ಇದನ್ನು ಆಮೇಲೆ ದೈನಿಕವನ್ನಾಗಿ ಬೇಕಾದರೆ ಮಾಡಬಹುದು. ದಿನವೂ ಲಕ್ಷ ಪ್ರತಿಗಳನ್ನು ಅಚ್ಚು ಮಾಡಿಸಿ ಕಲ್ಕತ್ತೆಯ ಗಲ್ಲಿ ಗಲ್ಲಿಗಳಲ್ಲಿಯೂ ದುಡ್ಡಿಲ್ಲದೆ ಹಂಚಬಹುದು.

ಶಿಷ್ಯ: ತಮ್ಮ ಈ ಯೋಚನೆ ಬಹಳ ಚೆನ್ನಾಗಿದೆ.

ಸ್ವಾಮೀಜಿ: ಕಾಗದವನ್ನು ಮಗ್ಗುಲಲ್ಲಿ ರಾಶಿಮಾಡಿ ನಿಲ್ಲಿಸಿ, ನಿನ್ನನ್ನು ಸಂಪಾದಕನನ್ನಾಗಿ ಮಾಡಿಬಿಡಬೇಕೆಂದು ನನ್ನ ಇಚ್ಛೆ. ಯಾವುದಾದರೂ ಒಂದು ಸ್ವಂತ ಕೆಲಸವನ್ನು ಮೂಲೆಗೆ ಒತ್ತರಿಸುವುದಕ್ಕೆ ನಿಮಗೆ ಇನ್ನೂ ಸಾಮರ್ಥ್ಯವಿಲ್ಲ. ಅದನ್ನು ಮಾಡುವುದಕ್ಕೆ ಈ ಸರ್ವತ್ಯಾಗಿಗಳಾದ ಸಾಧುಗಳೇ ಸಮರ್ಥರು. ಇವರು ಕೆಲಸ ಮಾಡಿ ಸಾಯುವರೇ ಹೊರತು ಹಿಂಜರಿಯುವ ಜನರಲ್ಲ. ಸ್ವಲ್ಪ ಅಡ್ಡಿ ಅಥವಾ ಸ್ವಲ್ಪ ನಿಂದೆ ಬಂದರೆ ಅಷ್ಟಕ್ಕೇ ನೀವು ಪ್ರಪಂಚವೇ ಅಂಧಕಾರಮಯವಾದಂತೆ ಭಾವಿಸುವಿರಿ.

ಶಿಷ್ಯ: ಆ ದಿವಸ ನೋಡಿದೆ. ಸ್ವಾಮಿ ತ್ರಿಗುಣಾತೀತರು ಪರಮಹಂಸರ ಚಿತ್ರಪಟವನ್ನು ಮುದ್ರಾಲಯದಲ್ಲಿ ಪೂಜಿಸಿ, ಆಮೇಲೆ ಕಾರ್ಯವನ್ನು ಆರಂಭಿಸಿದರು, ಮತ್ತು ಕಾರ್ಯ ಸಫಲವಾಗುವುದಕ್ಕೋಸ್ಕರ ತಮ್ಮ ಕೃಪೆಯನ್ನು ಯಾಚಿಸಿದರು.

ಸ್ವಾಮೀಜಿ: ನಮ್ಮ ಕೇಂದ್ರ ಪರಮಹಂಸರೆ. ನಾವು ಪ್ರತಿಯೊಬ್ಬರೂ ಆ ಕೇಂದ್ರ ಜ್ಯೋತಿಯ ಒಂದೊಂದು ಕಿರಣ. ಪರಮಹಂಸರ ಪೂಜೆ ಮಾಡಿ ಕಾರ್ಯವನ್ನು ಆರಂಭಿಸಿದನೋ, ಒಳ್ಳೆಯದು ಮಾಡಿದ. ನನಗೆ ಮಾತ್ರ ಯಾರೂ ಪೂಜೆಯ ಸಂಗತಿಯನ್ನು ಕುರಿತು ಏನೂ ಹೇಳಲಿಲ್ಲ.

ಶಿಷ್ಯ: ಮಹಾಶಯರೆ, ಅವರು ತಮಗೆ ಹೆದರುವರು. ತ್ರಿಗುಣಾತೀತ ಸ್ವಾಮಿಗಳು “ನೀನು ಮೊದಲು ಸ್ವಾಮಿಗಳ ಹತ್ತಿರ ಹೋಗಿ ತಿಳಿದುಕೊಂಡು ಬಾ, ಮೊದಲನೆಯ ಸಂಚಿಕೆಯ ವಿಚಾರವಾಗಿ ಅವರು ಏನು ಅಭಿಪ್ರಾಯಪಡುತ್ತಾರೆಂದು; ಆಮೇಲೆ ನಾನು ಅವರನ್ನು ನೋಡಲು ಹೋಗುತ್ತೇನೆ" ಎಂದು ಈವೊತ್ತು ಬೆಳಿಗ್ಗೆ ನನಗೆ ಹೇಳಿದರು.

ಸ್ವಾಮೀಜಿ: ಹೋಗಿ ಹೇಳು, ನಾನು ಅವನ ಕಾರ್ಯದಿಂದ ಬಹಳ ಸಂತೋಷ ಪಟ್ಟಿದ್ದೇನೆಂದು; ಅವನಿಗೆ ನನ್ನ ಸ್ನೇಹಾಶೀರ್ವಾದಗಳನ್ನು ತಿಳಿಸು; ಅಲ್ಲದೆ ನಿಮ್ಮಲ್ಲಿ ಪ್ರತಿಯೊಬ್ಬರೂ ಸಾಧ್ಯವಾದಷ್ಟು ಅವನಿಗೆ ಸಹಾಯ ಮಾಡಿ, ಅದರಿಂದ ದೇವರ ಕೆಲಸವೇ ಆಗುತ್ತದೆ. ಈ ಮಾತುಗಳನ್ನು ಹೇಳಿ ಸ್ವಾಮೀಜಿ, ಬ್ರಹ್ಮಾನಂದ ಸ್ವಾಮಿಗಳನ್ನು ಹತ್ತಿರಕ್ಕೆ ಕರೆದು ಆವಶ್ಯಕತೆ ಬಿದ್ದಂತೆ ಮುಂದೆ ‘ಉದ್ಭೋಧನ’ಕ್ಕಾಗಿ ತ್ರಿಗುಣಾತೀತರಿಗೆ ಮತ್ತಷ್ಟು ದುಡ್ಡನ್ನು ಕೊಡುವಂತೆ ಹೇಳಿದರು. ಆ ದಿವಸ ರಾತ್ರಿ ಊಟವಾದ ಮೇಲೆ ಸ್ವಾಮೀಜಿ ಪುನಃ ಶಿಷ್ಯನೊಡನೆ ‘ಉದ್ಭೋಧನ’ ಪತ್ರಿಕೆಯ ಸಂಬಂಧವಾಗಿ ಆಲೋಚನೆ ಮಾಡಿದರು. ಈ ಸಂದರ್ಭದಲ್ಲಿ ಅದನ್ನೂ ನಾವು ಪಾಠಕರಿಗೆ ಹೇಳುತ್ತೇವೆ.

ಸ್ವಾಮೀಜಿ: ಉದ್ಭೋಧನವು ಸಾಧಾರಣರಿಗೆ ಸಕಲ ವಿಚಾರಗಳಲ್ಲಿಯೂ ಜೀವನ ಪೋಷಕ ಆದರ್ಶಗಳನ್ನು ಕೊಡಬೇಕು. ಅಲ್ಲ, ಇಲ್ಲ ಎಂಬ ಭಾವ ಮನುಷ್ಯನನ್ನು ನಿರ್ಜಿವ ಮಾಡಿಬಿಡುತ್ತದೆ. ನೋಡುತ್ತಿಲ್ಲವೆ, ತಾಯಿ ತಂದೆಗಳು ಹುಡುಗರನ್ನು ಹಗಲೂ ರಾತ್ರಿ ಓದಿ ಬರೆಯಬೇಕೆಂದು ಹೊಡೆಯುತ್ತಾ, ‘ಇವನಿಂದ ಏನೂ ಆಗುವುದಿಲ್ಲ’ ‘ಶುದ್ಧ ಕತ್ತೆ’ ಎಂದು ಬೈಯುತ್ತಾರೆ. ಅವರ ಹುಡುಗರು ಅನೇಕ ಕಡೆಗಳಲ್ಲಿ ಹಾಗೇ ಆಗಿಬಿಡುತ್ತಾರೆ. ಹುಡುಗರಿಗೆ ಒಳ್ಳೆಯದನ್ನು ಹೇಳಿದರೆ – ಉತ್ಸಾಹವನ್ನು ಕೊಟ್ಟರೆ, ಕಾಲಕ್ರಮದಲ್ಲಿ ನಿಜವಾಗಿಯೂ ಒಳ್ಳೆಯದಾಗುತ್ತದೆ. ಮಕ್ಕಳಿಗೆ ಯಾವುದು ಒಳ್ಳೆಯದೋ, ಅದೇ ಉನ್ನತ ಭಾವನೆಗಳ ಸ್ಥರದಲ್ಲಿರುವ ಮಕ್ಕಳಿಗೂ ಅನ್ವಯಿಸುತ್ತದೆ. ಒಳ್ಳೆಯ ವಿಚಾರಗಳನ್ನು ಅವರಿಗೆ ನೀನು ಕೊಡಲು ಸಾಧ್ಯವಾದರೆ, ಮನುಷ್ಯರಾಗುವ ಮಟ್ಟಕ್ಕೆ ಜನರು ಬೆಳೆಯುತ್ತಾರೆ ಮತ್ತು ಅವರ ಸ್ವಂತ ಕಾಲುಗಳ ಮೇಲೆ ನಿಲ್ಲುವುದನ್ನು ಕಲಿಯುತ್ತಾರೆ. ಭಾಷೆ, ಸಾಹಿತ್ಯ, ದರ್ಶನ, ಕವಿತೆ, ಶಿಲ್ಪ ಇವುಗಳಲ್ಲಿಯೂ ಮನುಷ್ಯನು ಯೋಚಿಸುವ ಅಥವಾ ಮಾಡುವ ಸಮಸ್ತ ವಿಚಾರಗಳಲ್ಲಿಯೂ ತಪ್ಪನ್ನು ತೋರಿಸದೆ ಇವುಗಳನ್ನು ಹೇಗೆ ಕ್ರಮಕ್ರಮವಾಗಿ ಮತ್ತೂ ಚೆನ್ನಾಗಿ ಮಾಡಬಹುದೋ ಅದನ್ನೇ ಹೇಳಿಕೊಡಬೇಕು. ಭ್ರಮೆ ಪ್ರಮಾದಗಳನ್ನು ತೋರಿಸಿದರೆ ಮನುಷ್ಯನ ಮನಸ್ಸಿಗೆ ನೋವು ಆಗುತ್ತದೆ. ಪರಮಹಂಸರಲ್ಲಿ ನಾನು ಇದನ್ನು ನೋಡಿದ್ದೆ - ಯಾರನ್ನು ನಾವು ಹೇಯರೆಂದು ತಿಳಿದಿದ್ದೇವೋ ಅಂಥವರಿಗೂ ಅವರು ಪ್ರೋತ್ಸಾಹ ಕೊಟ್ಟು ಜೀವನದಲ್ಲಿ ಅವರ ಮತಿಗತಿಗಳನ್ನು ತಿರುಗಿಸಿ ಬಿಡುತ್ತಿದ್ದರು. ಅವರು ಶಿಕ್ಷಣ ಕೊಡುತ್ತಿದ್ದ ರೀತಿಯೇ ಒಂದು ಅದ್ಭುತ ವಿಚಾರ!

ಈ ಮಾತುಗಳನ್ನು ಹೇಳಿ ಸ್ವಾಮೀಜಿ ಸ್ವಲ್ಪ ಹೊತ್ತು ಸುಮ್ಮನಾದರು. ಆಮೇಲೆ ಪುನಃ ಹೇಳತೊಡಗಿದರು: “ಧರ್ಮ ಪ್ರಚಾರವೆಂದರೆ ಕೇವಲ ಯಾವುದೆಂದರೆ ಅದನ್ನು ತೆಗೆದುಕೊಂಡು ಯಾರೆಂದರೆ ಅವರ ಮೇಲೆ ಹೇರಿ ಮುಸುಡಿಯನ್ನು ತಿರುಗಿಸುವ ಕೆಲಸವೆಂದು ತಿಳಿದುಕೊಳ್ಳಬೇಡ. ದೈಹಿಕ, ಮಾನಸಿಕ ಮತ್ತು ಆಧ್ಯಾತ್ಮಿಕ ಹೀಗೆ ಸಕಲ ಕಾರ್ಯಗಳಲ್ಲಿಯೂ ಮನುಷ್ಯನಿಗೆ ಪೋಷಕ ಭಾವಗಳನ್ನು ಕೊಡಬೇಕು; ಹಾಗಲ್ಲದೆ ದ್ವೇಷಿಸಬಾರದು. ಒಬ್ಬರನ್ನೊಬ್ಬರು ದ್ವೇಷಿಸಿ ದ್ವೇಷಿಸಿಯೇ ನಿಮಗೆ ಅಧಃಪತನ ಉಂಟಾಗಿದೆ. ಈಗ ಕೇವಲ ಬಲಪ್ರದವಾಗುವ ಮತ್ತು ಜೀವಪೋಷಕವಾಗುವ ಭಾವವನ್ನೇ ಕೊಟ್ಟು ಜನರನ್ನು ಮೇಲಕ್ಕೆ ಎತ್ತಬೇಕು; ಮೊದಲು ಸಮಸ್ತ ಹಿಂದೂ ಜನರನ್ನು ಹೀಗೆ ಎತ್ತಬೇಕು. ಆಮೇಲೆ ಜಗತ್ತನ್ನು ಎತ್ತಬೇಕು. ಪರಮಹಂಸರು ಅವತಾರ ಮಾಡುವುದಕ್ಕೆ ಕಾರಣವೇ ಇದು. ಅವರು ಜಗತ್ತಿನಲ್ಲಿ ಯಾರ ಭಾವವನ್ನೂ ನಾಶಪಡಿಸಲಿಲ್ಲ. ಮಹಾ ಅಧಃಪತಿತನಾದ ಮನುಷ್ಯನಿಗೂ ಅವರು ಅಭಯವನ್ನು ಕೊಟ್ಟು ಉತ್ಸಾಹವನ್ನು ಕೊಟ್ಟು ಮೇಲೆತ್ತುತ್ತಿದ್ದರು. ನಾವೂ ಅವರ ಪದಾನುಸರಣ ಮಾಡಿಕೊಂಡು ಎಲ್ಲರನ್ನೂ ಮೇಲಕ್ಕೆತ್ತಬೇಕು, ಎಚ್ಚರಗೊಳಿಸಬೇಕು ತಿಳಿಯಿತೆ?"

ನಿಮ್ಮ ಚರಿತ್ರೆ, ಸಾಹಿತ್ಯ, ಪುರಾಣ ಮುಂತಾದ ಶಾಸ್ತ್ರಗ್ರಂಥಗಳೆಲ್ಲ ಮನುಷ್ಯನಿಗೆ ಕೇವಲ ಭಯವನ್ನೇ ತೋರಿಸುತ್ತವೆ! ಅವನನ್ನು ಕುರಿತು ನೀನು ನರಕಕ್ಕೆ ಹೋಗುವೆ, ನಿನಗೆ ಇನ್ನು ಮಾರ್ಗವಿಲ್ಲ ಎಂದು ಮಾತ್ರ ಹೇಳುತ್ತವೆ. ಅದರಿಂದಲೇ ಈ ಉತ್ಸಾಹರಾಹಿತ್ಯವೂ ದೌರ್ಬಲ್ಯವೂ ಭರತಖಂಡಕ್ಕೆ ಅಸ್ಥಿಗತವಾಗಿಬಿಟ್ಟಿವೆ. ಆದಕಾರಣ ವೇದ ವೇದಾಂತಗಳ ಉಚ್ಚ ಉಚ್ಚ ಭಾವಗಳನ್ನು ಸುಲಭವಾದ ಮಾತುಗಳಲ್ಲಿ ಜನರಿಗೆ ತಿಳಿಸಿಕೊಡಬೇಕಾಗಿದೆ. ಸದಾಚಾರ ಸದ್ವ್ಯವಹಾರ ವಿದ್ಯೆ ಶಿಕ್ಷಣ ಇವುಗಳನ್ನು ಕೊಟ್ಟು ಬ್ರಾಹ್ಮಣನನ್ನೂ ಚಂಡಾಲನನ್ನೂ ಒಂದು ಸಮಕ್ಕೆ ತರಬೇಕು. ‘ಉದ್ಭೋಧನ’ ಪತ್ರಿಕೆಯಲ್ಲಿ ಇದನ್ನೆಲ್ಲ ಬರೆದು ಹುಡುಗರು ಮಕ್ಕಳು ಹೆಂಗಸರು ಎಲ್ಲರನ್ನೂ ಉದ್ಧಾರಮಾಡಿ ಮತ್ತೆ, ನೋಡೋಣ. ಹಾಗಾದರೆ, ನಿಮ್ಮ ವೇದ ವೇದಾಂತವ್ಯಾಸಂಗ, ಸಾರ್ಥಕವಾಯಿತೆಂದು ತಿಳಿಯುತ್ತೇನೆ. ಏನೆನ್ನುತ್ತೀಯೆ - ಕೈಲಾಗುತ್ತದೆಯೊ?"

ಶಿಷ್ಯ: ತಮ್ಮ ಆಶೀರ್ವಾದ ಮತ್ತು ಆದೇಶಗಳಾದರೆ ಎಲ್ಲ ಸಂಗತಿಗಳಲ್ಲಿಯೂ ಜಯಶೀಲನಾಗುವೆನೆಂದು ತೋರುತ್ತದೆ.

ಸ್ವಾಮೀಜಿ: ಮತ್ತೊಂದು ಮಾತು - ಶರೀರವನ್ನು ಬಲಯುತವಾಗಿ ಮಾಡಿಕೊಳ್ಳುವುದನ್ನು ನೀನು ಕಲಿಯಬೇಕು ಮತ್ತು ಇತರರಿಗೆ ಕಲಿಸಬೇಕು. ನೋಡಿಲ್ಲವೇ ಈಗಲೂ ನಾನು ನಿತ್ಯವೂ ಡಂಬೆಲ್ ಸಾಧನೆ ಮಾಡುತ್ತಿರುವುದನ್ನು? ನಿತ್ಯವೂ ಬೆಳಿಗ್ಗೆ ಸಾಯಂಕಾಲ ಸಂಚಾರ ಮಾಡು; ದೇಹಶ್ರಮಪಡು, ದೇಹವೂ ಮನಸ್ಸೂ ಸರಿಸಮವಾಗಿ ಅಭಿವೃದ್ಧಿಯಾಗುತ್ತ ಹೋಗಬೇಕು. ಎಲ್ಲದಕ್ಕೂ ಮತ್ತೊಬ್ಬರನ್ನು ನೆಚ್ಚಿಕೊಂಡು ಏಕೆ ಹೋಗುತ್ತಿರಬೇಕು? ಶರೀರವನ್ನು ಬಲಪಡಿಸಿಕೊಳ್ಳುವುದರ ಉಪಯೋಗವನ್ನು ತಿಳಿದುಕೊಂಡರೆ ಆಮೇಲೆ ತಾವೇ ಇದರಲ್ಲಿ ಪ್ರವರ್ತಿಸುತ್ತಾರೆ. ಉಪಯೋಗವನ್ನು ತಿಳಿದುಕೊಳ್ಳುವುದಕ್ಕೆ ಆ ಶಿಕ್ಷಣ ಬೇಕಾಗಿರುವುದು.

\newpage

\chapter[ಅಧ್ಯಾಯ ೨೮]{ಅಧ್ಯಾಯ ೨೮\protect\footnote{\engfoot{C.W, Vol. VII, P. 172}}}

\begin{center}
ಸ್ಥಳ: ಬೇಲೂರು ಮಠ, ವರ್ಷ: ಕ್ರಿ.ಶ. ೧೮೯೯.
\end{center}

ಶಿಷ್ಯ: ಸ್ವಾಮೀಜಿ, ನಮ್ಮ ಸಮಾಜ ಮತ್ತು ನಮ್ಮ ದೇಶ ಇಷ್ಟೊಂದು ಅಧೋಗತಿಗಿಳಿಯಲು ಕಾರಣವೇನು?

ಸ್ವಾಮೀಜಿ: ಅದಕ್ಕೆ ನೀವೇ ಕಾರಣರು.

ಶಿಷ್ಯ: ಅದು ಹೇಗೆ ಸ್ವಾಮೀಜಿ? ನನಗೆ ಆಶ್ಚರ್ಯವಾಗುತ್ತಿದೆ.

ಸ್ವಾಮೀಜಿ: ಬಹುಕಾಲದಿಂದ ನೀವು ನಮ್ಮ ದೇಶದ ಕೆಳಗಿನ ಶ್ರೇಣಿಯ ಜನರನ್ನು ಹೀನಾಯವಾಗಿ ಕಾಣುತ್ತಿದ್ದೀರಿ. ಅದಕ್ಕೆ ನೀವು ಜಗತ್ತಿನ ದೃಷ್ಟಿಯಲ್ಲಿ ತಿರಸ್ಕೃತರಾಗಿದ್ದೀರಿ.

ಶಿಷ್ಯ: ನಾವು ಹಾಗೆ ಕೀಳಾಗಿ ಕಂಡದ್ದನ್ನು ನೀವು ಯಾವಾಗ ನೋಡಿದಿರಿ?

ಸ್ವಾಮೀಜಿ: ಏಕೆ ಪುರೋಹಿತ ವರ್ಗದವರು ಬ್ರಾಹ್ಮಣೇತರರಿಗೆ ವೇದ ಉಪನಿಷತ್ ಮುಂತಾದ ಶಾಸ್ತ್ರಗಳನ್ನು ಓದಲು ಅವಕಾಶ ಕೊಡಲಿಲ್ಲ? ಅವುಗಳನ್ನು ಮುಟ್ಟಲೂ ಕೂಡ ಬಿಡಲಿಲ್ಲ. ಅವರನ್ನು ಕೆಳಕ್ಕೆ ತುಳಿದಿದ್ದೀರಿ. ಕೇವಲ ಸ್ವಾರ್ಥದಿಂದ ನೀವೇ ಈ ರೀತಿ ಮಾಡಿದ್ದೀರಿ. ಪುರೋಹಿತ ವರ್ಗದವರೇ ಎಲ್ಲಾ ಧರ್ಮ ಗ್ರಂಥಗಳ ಏಕಮಾತ್ರ ಅಧಿಕಾರಿಗಳಾಗಿ ವಿಧಿ ನಿಷೇಧಗಳನ್ನೆಲ್ಲಾ ತಮ್ಮ ಹಿಡಿತದಲ್ಲೇ ಇಟ್ಟುಕೊಂಡಿದ್ದಾರೆ. ಬಾರಿಬಾರಿಗೂ ಇತರ ಜಾತಿಯವರನ್ನು ಕೀಳೆಂದೂ, ನೀಚರೆಂದೂ ಕರೆದು ಅವರ ತಲೆಯಲ್ಲಿ ತಾವು ನಿಜವಾಗಿಯೂ ಅಂತಹವರೆಂಬ ಭಾವನೆ ಬರುವಂತೆ ಮಾಡಿದ್ದೀರಿ. ನೀನು ಒಬ್ಬ ಮನುಷ್ಯನನ್ನು ಯಾವಾಗಲೂ ನೀನು ತುಚ್ಛ, ನೀಚ ಎಂದು ಹೇಳುತ್ತಿದ್ದರೆ ಅವನು ಕ್ರಮೇಣ ತಾನು ನೀಚನೆಂದೇ ಭಾವಿಸುವ ಹಾಗೆ ಆಗುತ್ತದೆ. ಇದೇ ಆ ಸಮ್ಮೋಹಿನಿ ವಿದ್ಯೆ \enginline{(hypnotism).} ಈಗ ಬ್ರಾಹ್ಮಣೇತರ ವರ್ಗದವರು ನಿಧಾನವಾಗಿ ಎಚ್ಚೆತ್ತುಕೊಳ್ಳುತ್ತಿದ್ದಾರೆ. ಅವರಿಗೆ ಬ್ರಾಹ್ಮಣ, ಪುರಾಣ, ಮಂತ್ರಗಳ ಮೇಲಿನ ಶ್ರದ್ಧೆ ಕಡಿಮೆಯಾಗುತ್ತಿದೆ. ಪಾಶ್ಚಾತ್ಯ ವಿದ್ಯಾಭ್ಯಾಸದ ಹರಡುವಿಕೆಯಿಂದ ಪುರೋಹಿತ ವರ್ಗದವರ ತಂತ್ರವೆಲ್ಲಾ ಕುಸಿದು ಬೀಳುತ್ತಿದೆ - ಮಳೆಗಾಲದಲ್ಲಿರುವ ಪದ್ಮಾನದಿಯ ದಡದಂತೆ, ನೀನದನ್ನು ನೋಡುತ್ತಿಲ್ಲವೆ?

ಶಿಷ್ಯ: ಸನಾತನಿಗಳ ದೋಷಾರೋಪಣೆಗಳೆಲ್ಲಾ ಕ್ರಮೇಣ ಈಗ ಕಡಿಮೆಯಾಗುತ್ತಿದೆ.

ಸ್ವಾಮೀಜಿ: ಅದು ಹಾಗೇ ಆಗಬೇಕು. ಬ್ರಾಹ್ಮಣರು ತೀವ್ರ ದುರ್ಮಾರ್ಗವರ್ತಿಗಳಾದರು. ಸ್ವಾರ್ಥ ದೃಷ್ಟಿಯಿಂದ ಅನೇಕ ವಿಚಿತ್ರವಾದ, ವೇದಗಳಲ್ಲಿಲ್ಲದ ದುರಾಚಾರ, ಅಧರ್ಮ ಮತ್ತು ಕುಯುಕ್ತಿಗಳನ್ನು ತಮ್ಮ ಪ್ರತಿಷ್ಠೆ ಕಾಪಾಡಿಕೊಳ್ಳುವುದಕ್ಕಾಗಿ ಪ್ರಚಾರಕ್ಕೆ ತಂದರು. ಈಗ ಅದರ ಫಲವನ್ನುಣ್ಣುತ್ತಿದ್ದಾರೆ.

ಶಿಷ್ಯ: ಆ ಫಲಗಳು ಯಾವುವು ಸ್ವಾಮೀಜಿ?

ಸ್ವಾಮೀಜಿ: ನಿನಗೆ ಕಾಣಿಸುತ್ತಿಲ್ಲವೆ! ಹಿಂದೂದೇಶದ ಸಾಧಾರಣ ಜನರನ್ನು ಕೀಳುದೃಷ್ಟಿಯಿಂದ ಕಂಡುದರಿಂದಲೇ ಸಾವಿರಾರು ವರ್ಷಗಳಿಂದಲೂ ಈ ಗುಲಾಮಗಿರಿಯ ಜೀವನವನ್ನು ನಡೆಸುತ್ತಿರುವಿರಿ. ಅದಕ್ಕೆ ವಿದೇಶೀಯರ ಕಣ್ಣಲ್ಲಿ ನೀವು ದೋಷಕ್ಕೆ ಪಾತ್ರರಾಗಿರುವಿರಿ. ನಿಮ್ಮ ಸ್ವದೇಶೀಯರೇ ನಿಮ್ಮನ್ನು ಉಪೇಕ್ಷಿಸುತ್ತಿದ್ದಾರೆ.

ಶಿಷ್ಯ: ಆದರೆ ಈಗಲೂ ಬ್ರಾಹ್ಮಣರೇ ಎಲ್ಲಾ ಬಗೆಯ ವಿಧಿನಿಯಮ ವ್ರತಗಳನ್ನು ನಡೆಸುತ್ತಿರುವರು. ಬ್ರಾಹ್ಮಣರ ಅಭಿಪ್ರಾಯದಂತೆಯೇ ಜನರು ಅವುಗಳನ್ನು ಆಚರಿಸುತ್ತಿರುವರು. ನೀವೇಕೆ ಹೀಗೆ ಮಾತನಾಡುವಿರಿ?

ಸ್ವಾಮೀಜಿ: ನನಗೇನೋ ಅದು ಕಾಣಿಸುತ್ತಿಲ್ಲ. ಶಾಸ್ತ್ರಗಳಲ್ಲಿ ಹೇಳಿರುವ ಆ ಹತ್ತು ಬಗೆಯ ಸಂಸ್ಕಾರಗಳು ಈಗೆಲ್ಲಿ ಬಳಕೆಯಲ್ಲಿವೆ? ನಾನು ಇದೇ ಭರತಖಂಡವನ್ನೆಲ್ಲಾ ಸುತ್ತಿದ್ದೇನೆ. ಎಲ್ಲಾ ಕಡೆಯೂ ಶ್ರುತಿ, ಸ್ಮೃತಿಗಳಲ್ಲಿ ನಿಷೇಧಿಸಲ್ಪಟ್ಟಿರುವ ಸ್ಥಳೀಯ ಸಂಪ್ರದಾಯಗಳೇ ಸಮಾಜಕ್ಕೆ ಮಾರ್ಗದರ್ಶಿಗಳಾಗಿವೆ. ಬಳಕೆಯಲ್ಲಿರುವ ಪದ್ಧತಿಗಳು, ಸ್ಥಳೀಯ ಸಂಪ್ರದಾಯಗಳು, ಹೆಂಗಸರಲ್ಲಿ ವಾಡಿಕೆಯಲ್ಲಿರುವ ಪದ್ಧತಿಗಳು ಇವೇ ಎಲ್ಲಾ ಕಡೆಯೂ ಸ್ಮೃತಿಗಳ ಸ್ಥಾನವನ್ನು ಆಕ್ರಮಿಸಿಕೊಂಡಿವೆ. ಅವುಗಳನ್ನು ಆಚರಿಸುವವರು ಯಾರು? ನೀನು ಸಾಕಷ್ಟು ಹಣವನ್ನು ಖರ್ಚುಮಾಡುವ ಹಾಗಿದ್ದರೆ ಈ ಪುರೋಹಿತ ವರ್ಗದವರು ನೀನಿಷ್ಟಪಡುವ ಯಾವ ಕಾನೂನು, ನಿಷೇಧ ನಿಯಮಗಳನ್ನಾದರೂ ಬರೆಯಲು ಸಿದ್ಧರಾಗಿರುವರು. ಅವರಲ್ಲಿ ಎಷ್ಟು ಮಂದಿ ವೈದಿಕ ಕಲ್ಪ, ಗೃಹ್ಯ, ಶ್ರೌತ ಸೂತ್ರಗಳನ್ನು ಓದಿದ್ದಾರೆ? ಅದಲ್ಲದೆ ಬಂಗಾಳದಲ್ಲಿ ರಘುನಂದನನ ಶಾಸ್ತ್ರವನ್ನು ಅನುಸರಿಸುವರು. ಇನ್ನು ಕೊಂಚ ದೂರದಲ್ಲಿ ಮಿತಾಕ್ಷರನ ಕಾಯಿದೆ ಪ್ರಚಾರದಲ್ಲಿದೆ. ದೇಶದ ಮತ್ತೊಂದು ಭಾಗದಲ್ಲಿ ಮನುಧರ್ಮ ಶಾಸ್ತ್ರ ರೂಢಿಯಲ್ಲಿದೆ. ಒಂದೇ ಬಗೆಯ ಕಾಯಿದೆಗಳು ಎಲ್ಲಾ ಕಡೆಗೂ ಅನ್ವಯಿಸುವುದೆಂದು ನೀವು ಯೋಚಿಸುವಿರಿ! ಆದ್ದರಿಂದ ನನಗೆ ಬೇಕಾದ್ದೇನೆಂದರೆ, ಜನರ ಮನಸ್ಸಿನಲ್ಲಿ ವೇದಗಳ ಮೇಲೆ ಹೆಚ್ಚು ಗೌರವ ಹುಟ್ಟುವಂತೆ ಮಾಡಿ ವೇದಾಭ್ಯಾಸ ಪ್ರಚಾರ ಮಾಡುವುದು. ವೇದಗಳ ಕಾಯಿದೆಗಳನ್ನು ಎಲ್ಲೆಡೆಯಲ್ಲೂ ಪ್ರಚಾರ ಮಾಡಬೇಕು.

ಶಿಷ್ಯ: ಈ ವರ್ತಮಾನ ಕಾಲದಲ್ಲಿ ಅವುಗಳನ್ನು ರೂಢಿಗೆ ತರಲು ಸಾಧ್ಯವೆ?

ಸ್ವಾಮೀಜಿ: ಪುರಾತನ ವೇದಗಳ ಎಲ್ಲಾ ಕಾಯಿದೆಗಳನ್ನೂ ಪ್ರಚಾರಕ್ಕೆ ತರಲಾಗುವುದಿಲ್ಲವೆಂಬುದು ನಿಜ. ಆದರೆ ಅವುಗಳನ್ನು ಕಾಲಧರ್ಮಕ್ಕೆ ತಕ್ಕಂತೆ ಬದಲಾಯಿಸಿ ಸಮಾಜಕ್ಕೆ ಹೊಸ ವಿಧಾನವನ್ನು ಎತ್ತಿ ಹಿಡಿದರೆ ಅದನ್ನು ಬಳಕೆಗೆ ತರಬಹುದು.

ಶಿಷ್ಯ: ಸ್ವಾಮೀಜಿ, ಮನುಧರ್ಮಶಾಸ್ತ್ರವಾದರೂ ಭರತಖಂಡದ ಎಲ್ಲೆಡೆಯಲ್ಲಿಯೂ ಆಚರಣೆಯದಲ್ಲಿದೆ ಎಂದು ತಿಳಿದಿದ್ದೆ.

ಸ್ವಾಮೀಜಿ: ಖಂಡಿತ ಇಲ್ಲ. ನಿನ್ನ ಸೀಮೆಯ ಕಡೆಯೇ ನೋಡು, ತಂತ್ರದ ವಾಮಾಚಾರ ಪದ್ಧತಿ ನಿಮ್ಮ ನಾಡಿನಾಡಿಗಳಲ್ಲೂ ಹಬ್ಬಿದೆ. ನಶಿಸಿ ಹೋದ ಬೌದ್ಧ ಧರ್ಮದ ಅವಶೇಷವಾದ ವರ್ತಮಾನಕಾಲದ ವೈಷ್ಣವ ಪದ್ಧತಿ ಕೂಡ ವಾಮಾಚಾರದಿಂದ ಕಲುಷಿತವಾಗಿದೆ. ವೇದಗಳಿಗೆ ವಿರೋಧವಾದ ಈ ವಾಮಾಚಾರ ಪದ್ದತಿಯನ್ನು ನಿರ್ಮೂಲ ಮಾಡಬೇಕು.

ಶಿಷ್ಯ: ಅಷ್ಟು ದಿನಗಳಿಂದ ಬೇರುಬಿಟ್ಟಿರುವುದನ್ನು ಪೂರ್ತಿ ತೆಗೆದು ಹಾಕಲು ಸಾಧ್ಯವೆ?

ಸ್ವಾಮೀಜಿ: ಹೇಡಿ! ಏನು ಅಸಂಬದ್ಧವಾಗಿ ಮಾತನಾಡುತ್ತಿರುವೆ! ‘ಇದು ಅಸಾಧ್ಯ, ಇದು ಅಸಾಧ್ಯ’ ಎಂಬ ಕೂಗಿನಿಂದ ನೀವು ಇಡೀ ದೇಶವನ್ನೇ ಮುಕ್ಕಾಲು ಪಾಲು ಹಾಳು ಮಾಡಿರುವಿರಿ. ಮನುಷ್ಯ ಪ್ರಯತ್ನಕ್ಕೆ ಅಸಾಧ್ಯವಾದುದು ಯಾವುದು?

ಶಿಷ್ಯ: ದೇಶದಲ್ಲಿ ಮನು ಯಾಜ್ಞವಲ್ಕ್ಯ ಋಷಿಗಳು ಪುನಃ ಹುಟ್ಟಿ ಬರುವವರೆಗೂ ಅಂತಹ ಸ್ಥಿತಿ ಅಸಾಧ್ಯವೆಂದೆನಿಸುತ್ತದೆ.

ಸ್ವಾಮಿಜಿ: ಅಯ್ಯೋ ದೇವರೆ! ಮನು, ಯಾಜ್ಞವಲ್ಕ್ಯರಾದುದು ಅವರ ಪಾವಿತ್ರ್ಯ, ನಿಷ್ಕಾಮಕರ್ಮದಿಂದಲ್ಲವೆ? ಅಥವಾ ಬೇರೆಯೊ? ನಾವೇ ಪ್ರಯತ್ನಪಟ್ಟರೆ ಮನು, ಯಾಜ್ಞವಲ್ಕ್ಯರಿಗಿಂತ ಹೆಚ್ಚು ಪ್ರಖ್ಯಾತರಾಗಬಹುದು. ಆಗ ನಮ್ಮ ಧ್ಯೇಯಗಳೇ ರೂಢಿಗೆ ಬರಬಹುದು.

ಶಿಷ್ಯ: ಪುರಾತನ ಪದ್ಧತಿ ಸಂಪ್ರದಾಯಗಳನ್ನು ನಮ್ಮ ದೇಶದಲ್ಲಿ ಪುನರುಜ್ಜೀವನಗೊಳಿಸಬೇಕೆಂದು ನೀವು ಈಗತಾನೆ ಹೇಳಿದಿರಲ್ಲ. ಹಾಗಿದ್ದ ಮೇಲೆ ಮನು ಮುಂತಾದ ಋಷಿಗಳನ್ನು ಗೌರವ ದೃಷ್ಟಿಯಿಂದ ನೋಡಬೇಕಲ್ಲವೆ?

ಸ್ವಾಮೀಜಿ: ಎಂತಹ ಹುಚ್ಚು ತರ್ಕ! ನಾನು ಹೇಳುವುದನ್ನು ನೀನು ಅರ್ಥಮಾಡಿಕೊಳ್ಳಲೇ ಇಲ್ಲ. ನಾನು ಹೇಳಿದ್ದು ಪುರಾತನ ವೇದದ ಸಂಪ್ರದಾಯಗಳನ್ನು ಸಮಾಜದ ಆವಶ್ಯಕತೆಗೆ ತಕ್ಕಂತೆ ಸರಿಪಡಿಸಿ ಹೊಸರೀತಿಯಲ್ಲಿ ಜಾರಿಗೆ ಬರುವಂತೆ ಮಾಡಬೇಕು ಎಂದಲ್ಲವೆ?

ಶಿಷ್ಯ: ಹೌದು.

ಸ್ವಾಮಿಜಿ: ಹಾಗಾದರೆ ನೀನು ಮಾತಾಡುತ್ತಿದ್ದುದೇನು? ನೀನು ಶಾಸ್ತ್ರಾಭ್ಯಾಸ ಮಾಡಿರುವೆ. ನನ್ನ ಆಸೆ ಶ್ರದ್ಧೆಗಳೆಲ್ಲ ನಿನ್ನಂತಿರುವ ಜನರನ್ನು ಅವಲಂಬಿಸಿವೆ. ನನ್ನ ಮಾತುಗಳನ್ನು ಸರಿಯಾಗಿ ತಿಳಿದುಕೊಂಡು ಆ ಹಾದಿಯಲ್ಲೇ ಕೆಲಸಕ್ಕೆ ತೊಡಗು.

ಶಿಷ್ಯ: ಆದರೆ ನಮ್ಮ ಮಾತಿಗೆ ಯಾರು ಕಿವಿಕೊಡುತ್ತಾರೆ? ನಮ್ಮ ದೇಶೀಯರು ಅವನ್ನೇಕೆ ಒಪ್ಪಿಕೊಳ್ಳಬೇಕು?

ಸ್ವಾಮೀಜಿ: ನೀನು ಅವರಿಗೆ ಇದನ್ನು ಮನಮುಟ್ಟುವಂತೆ ತಿಳಿಸಿ ನೀನೂ ನಿನ್ನ ಮಾತಿನಂತೆಯೇ ನಡೆದರೆ ಅವರು ಒಪ್ಪಿಕೊಂಡೇ ತೀರುವರು. ಅದನ್ನು ಬಿಟ್ಟು ಹೇಡಿಯಂತೆ ಕೇವಲ ಶ್ಲೋಕಗಳನ್ನು ಗಿಳಿ ಪಾಠದಂತೆ ಪಠಿಸಿದರೆ, ಕೇವಲ ಬರಿಯ ಮಾತುಗಾರನಾಗಿ ಸ್ವಲ್ಪವೂ ಕೆಲಸದ ಮೂಲಕ ತೋರಿಸದೆ ನಿಯಮಗಳನ್ನು ಬಾಯಿಯಿಂದ ಮಾತ್ರ ಪಠಿಸಿದರೆ ಯಾರು ತಾನೇ ನಿನ್ನ ಮಾತಿಗೆ ಕಿವಿಗೊಡುವರು?

ಶಿಷ್ಯ: ದಯವಿಟ್ಟು ಸಮಾಜ ಸುಧಾರಣೆ ವಿಷಯವಾಗಿ ನನಗೆ ಕೆಲವು ಸಲಹೆಗಳನ್ನು ಕೊಡಿ.

ಸ್ವಾಮೀಜಿ: ಏಕೆ, ನಾನು ನಿನಗೆ ಆಗಲೇ ಸಾಕಷ್ಟು ಸಲಹೆ ನೀಡಿರುವೆ. ಕಡೆಯ ಪಕ್ಷ ಒಂದನ್ನಾದರೂ ಕಾರ್ಯರೂಪಕ್ಕೆ ತಾ, ನಿನ್ನ ಶಾಸ್ತ್ರ ವ್ಯಾಸಂಗಗಳು ನನ್ನ ಮಾತುಗಳಿಗೆ ಕಿವಿಗೊಟ್ಟಿದ್ದರ ಫಲವಾಗಿ ಸಾರ್ಥಕವಾಯಿತೆಂಬುದನ್ನು ಜಗಕ್ಕೆ ತೋರು. ನೀನು ಓದಿರುವ ಮನುಧರ್ಮಶಾಸ್ತ್ರ ಮತ್ತು ಇತರ ಪುಸ್ತಕಗಳ ತಳಹದಿ ಮತ್ತು ಅವುಗಳಲ್ಲಡಗಿರುವ ಉದ್ದೇಶವೇನು? ಆ ತಳಹದಿಯನ್ನು ಭದ್ರವಾಗಿಟ್ಟು ಪೂರ್ವಿಕ ಋಷಿಗಳಂತೆ ಅವುಗಳ ಮುಖ್ಯ ಸತ್ಯಗಳನ್ನು ಸಂಗ್ರಹಿಸು. ಆಧುನಿಕ ಕಾಲಕ್ಕೆ ಹೊಂದಿಕೊಳ್ಳುವ ಭಾವನೆಗಳನ್ನೂ ಅದರೊಡನೆ ಸೇರಿಸು. ಆದರೆ ಒಂದನ್ನು ಎಚ್ಚರಿಕೆಯಿಂದ ನೆನಪಿನಲ್ಲಿಡು. ಆ ನಿಯಮಗಳನ್ನು ಪಾಲಿಸುವುದರಿಂದ ಇಡೀ ಭರತವರ್ಷದ ಎಲ್ಲಾ ಪಂಗಡಗಳಿಗೂ ಉಪಕಾರವಾಗಬೇಕು. ಅಂತಹುದೊಂದು ಸ್ಮೃತಿಯನ್ನು ಬರೆ. ನಾನದನ್ನು ತಿದ್ದಿಕೊಡುತ್ತೇನೆ.

ಶಿಷ್ಯ: ಸ್ವಾಮೀಜಿ ಅದೇನು ಸುಲಭದ ಕೆಲಸವಲ್ಲ. ಒಂದು ವೇಳೆ ಅಂತಹ ಸ್ಮೃತಿಯನ್ನು ಬರೆದರೂ ಜನ ಅದನ್ನು ಸ್ವೀಕರಿಸುವರೆ?

ಸ್ವಾಮೀಜಿ: ಸ್ವೀಕರಿಸದೆ ಏನು? ನೀನು ಸುಮ್ಮನೆ ಬರೆ. ‘ಕಾಲೋಹ್ಯಯಂ ನಿರವಧಿರ್ವಿಪುಲಾ ಚ ಪೃಥ್ವೀ’- ಕಾಲ ಅನಂತ, ಪ್ರಪಂಚ ವಿಶಾಲವಾಗಿದೆ. ನೀನು ಅದನ್ನು ಸರಿಯಾಗಿ ಬರೆದರೆ ಒಂದು ದಿನ ಬಂದೇ ಬರುವುದು. ಆಗ ಅದು ಸ್ವೀಕೃತವಾಗುವುದು. ನಿನಗೆ ಆತ್ಮವಿಶ್ವಾಸವಿರಲಿ. ಒಮ್ಮೆ ನೀವೆಲ್ಲ ವೇದಗಳ ಋಷಿಗಳಾಗಿದ್ದಿರಿ. ಈಗ ಕೇವಲ ಬಾಹ್ಯದಲ್ಲಿ ಬದಲಾವಣೆ ಹೊಂದಿ ಬಂದಿರುವಿರಿ. ನನಗೆ ಸೂರ್ಯನ ಬೆಳಕಿನಂತೆ ಅದು ಸ್ಪಷ್ಟವಾಗಿ ಕಾಣಿಸುತ್ತಿದೆ. ನಿಮ್ಮಲ್ಲಿ ಅನಂತ ಶಕ್ತಿ ಇದೆ. ಅದನ್ನು ಜಾಗೃತಗೊಳಿಸಿ, ಏಳಿ, ಹೃತ್ಪೂರ್ವಕವಾಗಿ ಸೊಂಟಕಟ್ಟಿ ಸಿದ್ಧರಾಗಿ. ಕ್ಷಣಿಕವಾದ ಸಿರಿ ಕೀರ್ತಿಯಿಂದೇನು ಮಾಡುವಿರಿ? ನಾನು ಏನನ್ನು ಯೋಚಿಸುತ್ತಿರುವೆನೆಂದು ಬಲ್ಲೆಯ? ನಾನು ಮುಕ್ತಿ ಮುಂತಾದುವೊಂದಕ್ಕೂ ಗಮನ ಕೊಡುವುದಿಲ್ಲ. ನಿಮ್ಮಲ್ಲೆಲ್ಲಾ ಈ ಭಾವನೆಯನ್ನು ಜಾಗೃತಗೊಳಿಸುವುದೇ ನನ್ನ ಧ್ಯೇಯ. ಕೇವಲ ಒಬ್ಬ ಮನುಷ್ಯನನ್ನು ಆ ರೀತಿ ತರಬೇತು ಮಾಡಲು ನಾನು ಸಾವಿರಾರು ಜನ್ಮಗಳನ್ನು ಎತ್ತಿಬರಲು ಸಿದ್ಧನಿದ್ದೇನೆ.

ಶಿಷ್ಯ: ಆದರೆ ಇಂತಹ ಕೆಲಸಗಳನ್ನು ಕೈಗೊಳ್ಳುವುದರಿಂದ ಪ್ರಯೋಜನವೇನು? ಮೃತ್ಯು ಅವುಗಳ ಹಿಂದೆಯೇ ಹೊಂಚಿ ನಿಂತಿಲ್ಲವೇ?

ಸ್ವಾಮೀಜಿ: ನಿನಗೆ ಧಿಕ್ಕಾರ! ನೀನು ಸಾಯುವೆಯಾದರೆ ಅದು ಒಂದೇ ಬಾರಿ. ಹೇಡಿಯಂತೆ ಸಾವಿನ ಯೋಚನೆಯನ್ನೇ ಮೆಲುಕು ಹಾಕುತ್ತಾ ನಿನ್ನ ಜೀವನದ ಪ್ರತಿ ಕ್ಷಣವೂ ಸಾಯುವೆಯೇಕೆ?

ಶಿಷ್ಯ: ಹಾಗೇ ಆಗಲಿ ಸ್ವಾಮೀಜಿ, ನಾನು ಮೃತ್ಯುವಿನ ಯೋಚನೆ ಮಾಡದಿರಬಹುದು. ಆದರೆ ಈ ನಶ್ವರ ಪ್ರಪಂಚದಲ್ಲಿ ಯಾವ ಬಗೆಯ ಕೆಲಸ ಮಾಡಿದರೂ ಆಗುವ ಪ್ರಯೋಜನವೇನು?

ಸ್ವಾಮಿಜಿ: ನನ್ನ ಮಗು! ಸಾವು ಅನಿವಾರ್ಯವಾದರೆ ಸಸ್ಯ, ಪ್ರಾಣಿಗಳಂತೆ ಸಾಯುವುದಕ್ಕಿಂತ ವೀರನಂತೆ ಮಡಿಯುವುದು ಲೇಸಲ್ಲವೆ? ಈ ಕ್ಷಣಭಂಗುರವಾದ ಪ್ರಪಂಚದಲ್ಲಿ ಒಂದೆರಡು ದಿನ ಜಾಸ್ತಿ ಬದುಕಿದರೆ ತಾನೆ ಏನು? ಸಾಯುವವರೆಗೆ ಕೊಳೆಯುವುದಕ್ಕಿಂತ ಇತರರಿಗೆ ಕೈಲಾದಷ್ಟು ಸಹಾಯಮಾಡಿ ಬೇಗ ಸಾಯುವುದು ಮೇಲು.

ಶಿಷ್ಯ: ನಿಜ ಸ್ವಾಮಿಜಿ, ನಿಮಗೆ ಇಷ್ಟು ತೊಂದರೆ ಕೊಟ್ಟಿದ್ದಕ್ಕಾಗಿ ಕ್ಷಮೆ ಬೇಡುವೆ.

ಸ್ವಾಮೀಜಿ: ಯಾರು ಇಚ್ಛೆಪಟ್ಟು ಕೇಳುವರೊ ಅವರೊಡನೆ ಇಡೀ ಎರಡು ರಾತ್ರಿ ಮಾತನಾಡಿದರೂ ನನಗೆ ದಣಿವಾಗುವುದಿಲ್ಲ. ಬೇಕಾದರೆ ಅನ್ನ ನಿದ್ರೆಗಳನ್ನೆಲ್ಲಾ ಬಿಟ್ಟು ಮಾತನಾಡುವೆ. ನಾನಿಚ್ಛೆಪಟ್ಟರೆ ಹಿಮಾಲಯದ ಗುಹೆಯಲ್ಲಿ ಸಮಾಧಿಸ್ಥಿತಿಯಲ್ಲಿ ಕುಳಿತಿರಬಲ್ಲೆ, ನೀನೇ ನೋಡುತ್ತಿರುವೆ. ಈಗೀಗಂತೂ ನನಗೆ ಆಹಾರದ ವಿಷಯವಾಗಿಯೂ ಯೊಚನೆಯಿಲ್ಲ. ತಾಯಿಯ ಕೃಪೆಯಿಂದ ಹೇಗೋ ಬರುತ್ತದೆ. ನಾನೇಕೆ ಹಾಗೆ ಮಾಡುವುದಿಲ್ಲ? ನಾನೇಕೆ ಇಲ್ಲಿರುವೆ? ದೇಶದ ದುಃಸ್ಥಿತಿ, ಅದರ ಭವಿಷ್ಯದ ಯೋಚನೆ ಇವು ನನ್ನನ್ನು ಸುಮ್ಮನೆ ಇರಲು ಬಿಡುತ್ತಿಲ್ಲ. ಸಮಾಧಿ ಮುಂತಾದುವು ಕೂಡ ಕ್ಷುದ್ರ ಎನ್ನಿಸುತ್ತದೆ; ಬ್ರಹ್ಮಜ್ಞಾನ ಬ್ರಹ್ಮಾನಂದವೂ ಕೂಡ ಸಪ್ಪೆಯಾಗಿ ತೋರುವುದು. ನನ್ನ ಜೀವನದ ಏಕಮಾತ್ರ ಧ್ಯೇಯವೇ ನಿಮ್ಮ ಪುರೋಭಿವೃದ್ಧಿ. ಎಂದು ನನ್ನ ವ್ರತ ಸಫಲವಾಗುವುದೋ ಅಂದೇ ನಾನು ಈ ದೇಹವನ್ನು ತ್ಯಜಿಸಿ ಒಂದೇ ಸಲ ಪರಾರಿಯಾಗುವೆನು.

ಸ್ವಾಮೀಜಿಯ ಮಾತುಗಳನ್ನು ಕೇಳುತ್ತಾ ಶಿಷ್ಯ ಸ್ವಲ್ಪ ಹೊತ್ತು ಅವರನ್ನೇ ದಿಟ್ಟಿಸಿ ನೋಡುತ್ತಾ ಆಶ್ಚರ್ಯದಿಂದ ಮೂಕನಂತೆ ಕುಳಿತಿದ್ದ. ನಂತರ ಭಕ್ತಿಯಿಂದ ಅವರಿಗೆ ನಮಸ್ಕಾರ ಮಾಡಿ, ಹೋಗಲು ಅಪ್ಪಣೆ ಬೇಡಿದ.

ಸ್ವಾಮೀಜಿ: ನೀನೇಕೆ ಹೋಗಲಿಚ್ಛಿಸುವೆ? ಮಠದಲ್ಲೇ ಏಕೆ ಇರಬಾರದು? ಪ್ರಾಪಂಚಿಕ ಮನಸ್ಸುಳ್ಳವರ ಹತ್ತಿರ ಹೋಗುವುದರಿಂದ ನಿನ್ನ ಮನಸ್ಸು ಪುನಃ ಕಲುಷಿತವಾಗುವುದು. ಇಲ್ಲಿ ನೋಡು, ಗಾಳಿ ಎಷ್ಟು ನಿರ್ಮಲವಾಗಿದೆ. ಅಲ್ಲಿ ಗಂಗಾನದಿಯಿದೆ - ಸಾಧುಗಳು ಧ್ಯಾನ ಸಾಧನೆ ಮಾಡುತ್ತಾ ಸತ್ಕಥೆಗಳಲ್ಲಿ ನಿರತರಾಗಿರುವರು. ನೀನು ಕಲ್ಕತ್ತೆಗೆ ಹೋದ ಮರುಗಳಿಗೆಯಲ್ಲೇ ಕೆಟ್ಟ ಯೋಚನೆಗಳು ನಿನ್ನ ಮನಸ್ಸಿಗೆ ಬರುತ್ತವೆ.

ಶಿಷ್ಯ: (ಸಂತೋಷದಿಂದ) ಆಗಲಿ ಗುರುಗಳೆ, ಇಂದು ನಾನು ಮಠದಲ್ಲೇ ಉಳಿಯುವೆ.

ಸ್ಯಾಮಾಜಿ: ಇಂದು ಮಾತ್ರವೇ ಏಕೆ? ಎಂದೆಂದೂ ನೀನು ಇಲ್ಲಿಯೇ ಇರಲಾರೆಯೇನು? ಪ್ರಪಂಚಕ್ಕೆ ಹಿಂತಿರುಗುವುದರಿಂದಾಗುವ ಪ್ರಯೋಜನವೇನು?

ಸ್ವಾಮಿಗಳ ಮಾತನ್ನು ಕೇಳಿ ಶಿಷ್ಯ ತಲೆ ತಗ್ಗಿಸಿದ. ಅನೇಕ ಆಲೋಚನೆಗಳು ಅವನ ತಲೆಯಲ್ಲಿ ತುಂಬಿ ಬಂದು ಅವನು ಮೂಕನಂತಾದ.

\newpage

\chapter[ಅಧ್ಯಾಯ ೨೯]{ಅಧ್ಯಾಯ ೨೯\protect\footnote{\engfoot{C.W, Vol. VII, P177}}}

\begin{center}
ಸ್ಥಳ: ಬೇಲೂರು ಮಠ, ವರ್ಷ: ಕ್ರಿ.ಶ. ೧೮೯೯ರ ಪ್ರಾರಂಭದಲ್ಲಿ.
\end{center}

ಇಂದು ಮಧ್ಯಾಹ್ನ ಸ್ವಾಮೀಜಿ ಶಿಷ್ಯನೊಡನೆ ಹೊಸ ಮಠದ ಸುತ್ತಲೂ ಸುತ್ತಾಡುತ್ತಿದ್ದಾರೆ. ಬಿಲ್ವ ವೃಕ್ಷಕ್ಕೆ ಕೊಂಚ ದೂರದಲ್ಲಿ ನಿಂತು ಸ್ವಾಮಿಗಳು ನಿಧಾನವಾಗಿ ಒಂದು ಬಂಗಾಳಿ ಹಾಡನ್ನು ಹಾಡಲಾರಂಭಿಸಿದರು. “ಓ ಹಿಮಾಲಯ, ಗಣೇಶನು ನನಗೆ ಮಂಗಳಮಯನಾಗಿರುವನು" ಎಂದು ಹಾಡಿದರು. ಕೊನೆಗೆ “ಅನೇಕ ಜಡೆಗಳನ್ನು ಬಿಟ್ಟ ಯೋಗಿಗಳೂ, ಜೋಗಿಗಳೂ ಬರುವರು" ಎಂದು ಹಾಡಿ ಸ್ವಾಮಿಗಳು ಶಿಷ್ಯನ ಕಡೆ ತಿರುಗಿ “ನಿನಗೆ ಅರ್ಥವಾಯಿತೆ? ಕಾಲಕ್ರಮದಲ್ಲಿ ಇಲ್ಲಿಗೂ ಅನೇಕ ಸಂನ್ಯಾಸಿಗಳೂ, ಯೋಗಿಗಳೂ ಬರುವರು" ಎಂದರು. ಹೀಗೆ ಹೇಳುತ್ತಾ ಸ್ವಾಮಿಗಳು ಅಲ್ಲಿ ಒಂದು ಮರದ ಕೆಳಗೆ ಕುಳಿತುಕೊಂಡು “ಬಿಲ್ವವೃಕ್ಷದ ಕೆಳಗಿರುವ ಸ್ಥಳ ಪವಿತ್ರವಾದುದು. ಇದರ ಕೆಳಗೆ ಕುಳಿತು ಧ್ಯಾನ ಮಾಡಿದರೆ ಬಹು ಬೇಗ ಆಧ್ಯಾತ್ಮಿಕ ಪ್ರವೃತ್ತಿ ಜಾಗೃತಗೊಳ್ಳುವುದು. ಶ‍್ರೀರಾಮಕೃಷ್ಣರು ಈ ರೀತಿ ಹೇಳುತ್ತಿದ್ದರು" ಎಂದರು.

ಶಿಷ್ಯ: ಸ್ವಾಮೀಜಿ, ಆತ್ಮ ಅನಾತ್ಮಗಳ ವಿವೇಚನಾಜ್ಞಾನವುಳ್ಳವರಿಗೂ ದೇಶಕಾಲ ಮುಂತಾದುವುಗಳ ಪವಿತ್ರತೆಯನ್ನೆಣಿಸುವ ಆವಶ್ಯಕತೆಯಿದೆಯೇ?

ಸ್ವಾಮೀಜಿ: ಆತ್ಮಪರಿಜ್ಞಾನವುಳ್ಳವರಿಗೆ ಈ ಬಗೆಯ ವಿವೇಚನೆಯ ಆವಶ್ಯಕತೆಯಿಲ್ಲ. ಆದರೆ ಆ ಸ್ಥಿತಿ ಪೂರ್ವಸಿದ್ಧತೆಯಿಲ್ಲದೆ ಬರುವುದಿಲ್ಲ - ದೀರ್ಘಕಾಲದ ಸಾಧನೆಯಿಂದ ಬರುವುದು. ಆದ್ದರಿಂದ ಪ್ರಾರಂಭದಲ್ಲಿ ಬಾಹ್ಯಸಹಾಯವನ್ನು ಹೊಂದಿ ನಮ್ಮ ಕಾಲಮೇಲೆ ನಾವು ನಿಲ್ಲುವಂತಾಗಬೇಕು. ಆಮೇಲೆ ಆತ್ಮಪರಿಜ್ಞಾನದಲ್ಲಿ ನುರಿತ ಮೇಲೆ ಹೊರಗಣ ಸಹಾಯದ ಆವಶ್ಯಕತೆ ಇರುವುದಿಲ್ಲ.

“ಶಾಸ್ತ್ರಗಳಲ್ಲಿ ಹೇಳಿರುವ ಅನೇಕ ಬಗೆಯ ಆಧ್ಯಾತ್ಮಿಕ ಸಾಧನೆಗಳೆಲ್ಲ ಬ್ರಹ್ಮಜ್ಞಾನ ಸಿದ್ಧಿಗಾಗಿ. ಸಹಜವಾಗಿ ಸಾಧನೆಗಳು ಆಯಾ ಸಾಧಕರಿಗೆ ತಕ್ಕಂತೆ ವೈವಿಧ್ಯ ಹೊಂದಿರುತ್ತವೆ. ಆದರೆ ಅವೂ ಒಂದು ಬಗೆಯ ಕರ್ಮಗಳು. ಎಲ್ಲಿಯವರೆಗೆ ಕರ್ಮವಿರುವುದೊ ಅಲ್ಲಿಯವರೆಗೂ ಆತ್ಮಸಾಕ್ಷಾತ್ಕಾರವಾಗುವುದಿಲ್ಲ. ಶಾಸ್ತ್ರಗಳಲ್ಲಿ ವಿಧಿಸಿರುವ ಸಾಧನೆಗಳಿಂದ ಆತ್ಮಪ್ರಕಾಶಕ್ಕಿರುವ ವಿಘ್ನಗಳನ್ನು ಜಯಿಸಲು ಸಾಧ್ಯ. ಆದರೆ ಕರ್ಮಕ್ಕೆ ನೇರವಾಗಿ ಆತ್ಮಸಾಕ್ಷಾತ್ಕಾರ ಪಡೆಯುವ ಶಕ್ತಿಯಿಲ್ಲ. ಕೇವಲ ಜ್ಞಾನವನ್ನು ಮುಸುಕಿರುವ ತೆರೆಯನ್ನು ಸ್ವಲ್ಪಮಟ್ಟಿಗೆ ತೆಗೆದುಹಾಕಲು ಮಾತ್ರ ಸಾಧ್ಯ. ನಂತರ ಆತ್ಮನು ತನ್ನ ಸ್ವಯಂ ಪ್ರಕಾಶದಿಂದ ಪ್ರತಿಷ್ಟಿತನಾಗುತ್ತಾನೆ. ಅರ್ಥವಾಯಿತೆ? ಅದಕ್ಕೆ ನಿನ್ನ ಭಾಷ್ಯಕಾರರು (ಶಂಕರಾಚಾರ್ಯರು) ‘ಬ್ರಹ್ಮಜ್ಞಾನದಲ್ಲಿ ಕರ್ಮದ ಸೋಂಕೇ ಇರುವುದಿಲ್ಲ’ ಎಂದು ಹೇಳುತ್ತಾರೆ."

ಶಿಷ್ಯ: ಆದರೆ ಸ್ವಾಮೀಜಿ, ಯಾವುದಾದರೊಂದು ರೂಪದಲ್ಲಿ ಕರ್ಮ ಮಾಡಿದ ಹೊರತು ಆತ್ಮಸಾಕ್ಷಾತ್ಕಾರಕ್ಕೆ ಇರುವ ವಿಘ್ನಗಳನ್ನು ದಾಟಲು ಅಸಾಧ್ಯ ಎಂದಮೇಲೆ ಪರೋಕ್ಷವಾಗಿ ಕರ್ಮವು ಬ್ರಹ್ಮಜ್ಞಾನಕ್ಕೆ ಸಾಧನವಾಗುವುದಲ್ಲವೆ?

ಸ್ವಾಮೀಜಿ: ಆಕಸ್ಮಿಕವಾಗಿ ನೋಡಿದಾಗ ಪ್ರಥಮಬಾರಿ ಹಾಗೆ ಕಾಣಿಸುವುದು. ಪೂರ್ವ ಮಿಮಾಂಸೆಯ ಈ ದೃಷ್ಟಿಯನ್ನು ತೆಗೆದುಕೊಂಡು ಕರ್ಮವನ್ನು ಒಂದು ನಿರ್ದಿಷ್ಟ ಗುರಿಗಾಗಿ ಮಾಡಿದಾಗ ಅದರಿಂದ ಒಂದೇ ನಿಶ್ಚಿತ ಫಲ ಉಂಟಾಗುವುದು ಎಂದು ಹೇಳಿದೆ. ಆದರೆ ಅಖಂಡ ಆತ್ಮಸಾಕ್ಷಾತ್ಕಾರವನ್ನು ಕರ್ಮದಿಂದ ಗಳಿಸಲಾಗುವುದಿಲ್ಲ. ಆತ್ಮಸಾಕ್ಷಾತ್ಕಾರಾಕಾಂಕ್ಷಿಯಾದವನು ಆಧ್ಯಾತ್ಮಿಕ ಸಾಧನೆಗಳನ್ನು ಮಾಡಬೇಕು, ಆದರೆ ಅದರ ಫಲದ ಮೇಲೆ ಆಸಕ್ತಿ ಇರಕೂಡದು ಎಂಬ ನಿಯಮವಿದೆ. ಆದ್ದರಿಂದ ಈ ಸಾಧನೆಗಳು ಸಾಧಕನ ಮನಸ್ಸನ್ನು ಪರಿಶುದ್ಧಗೊಳಿಸಲು ಇರುವ ಕೆಲವು ಸಾಧನಗಳು ಮಾತ್ರ. ಏಕೆಂದರೆ ಈ ಸಾಧನಗಳ ಫಲವಾಗಿ ಒಡನೆಯೇ ಆತ್ಮಸಾಕ್ಷಾತ್ಕಾರವಾಗುವ ಹಾಗಿದ್ದರೆ ಶಾಸ್ತ್ರಗಳು ಸಾಧಕನಿಗೆ ಕರ್ಮ ಫಲದ ಮೇಲೆ ದೃಷ್ಟಿಯಿರಬಾರದೆಂದು ಸಾರುತ್ತಿರಲಿಲ್ಲ. ಪೂರ್ವಮೀಮಾಂಸೆಯಲ್ಲಿರುವ ಫಲೋದ್ದೇಶಿತ ಕರ್ಮಗಳ ಸಿದ್ಧಾಂತಗಳಿಗೆ ವಿರೋಧವಾಗಿ ಗೀತೆಯಲ್ಲಿ ನಿಷ್ಕಾಮ ಕರ್ಮದ ಸಿದ್ಧಾಂತವನ್ನು ಹೇಳಿದೆ. ಅರ್ಥವಾಯಿತೆ?

ಶಿಷ್ಯ: ಆದರೆ ಸ್ವಾಮೀಜಿ, ಒಬ್ಬನು ಕರ್ಮತ್ಯಾಗ ಮಾಡಿದ ಮೇಲೆ, ಯಾವಾಗಲೂ ತೊಂದರೆಗೆ ಒಳಪಡಿಸುವ ಕರ್ಮಗಳನ್ನು ಪುನಃ ಮಾಡುವಂತೆ ಏಕೆ ಬಲಾತ್ಕರಿಸಬೇಕು?

ಸ್ವಾಮೀಜಿ: ಈ ಮಾನವ ಜನ್ಮದಲ್ಲಿ ಯಾವುದಾದರೊಂದು ಬಗೆಯ ಕರ್ಮವನ್ನು ಮಾಡುತ್ತಲೇ ಇರಬೇಕು. ಬಲಾತ್ಕಾರವಾಗಿಯಾದರೂ ಕರ್ಮ ಮಾಡಲೇಬೇಕಾದರೆ ಕರ್ಮಯೋಗ ಆ ಕರ್ಮವನ್ನು ಅನಾಸಕ್ತನಾಗಿ ಆತ್ಮಸಾಕ್ಷಾತ್ಕಾರಕ್ಕಾಗಿ ಮಾಡೆಂದು ಬೋಧಿಸುತ್ತದೆ. ಯಾರಿಗೂ ಕೆಲಸ ಮಾಡಲು ಮನಸ್ಸಾಗುವುದಿಲ್ಲವೆಂದು ನೀನು ಹೇಳಿದೆಯಲ್ಲವೆ? ನೀನು ಯಾವ ಕೆಲಸವನ್ನಾದರೂ ಮಾಡು, ಅದರ ಹಿಂದೆ ಉದ್ದೇಶವಿದ್ದೇ ಇರುತ್ತದೆ. ಬಹುಕಾಲ ಕೆಲಸ ಮಾಡಿದ ಮೇಲೆ ಒಂದು ಕರ್ಮ ಮತ್ತೊಂದು ಕರ್ಮಕ್ಕೆ ಕಾರಣವಾಗುವುದು ಎಂದು ನೀನು ಅರಿಯುವೆ. ನಂತರ ನಿರಂತರವಾದ ಈ ಕರ್ಮದ ಸರಪಳಿಗೆ ಕೊನೆ ಯಾವುದು? ಎಂಬ ಪ್ರಶ್ನೆ ಬರುವುದು. ಆಗ ಗೀತೆಯಲ್ಲಿ ಭಗವಂತ ಹೇಳಿರುವ ಮಾತುಗಳ ಸಂಪೂರ್ಣ ಅರ್ಥ ವೇದ್ಯವಾಗುವುದು. ಕರ್ಮದ ಹಾದಿ ಅಗಮ್ಯ. ಯಾವಾಗ ಸಾಧಕನಿಗೆ ಕಾಮ್ಯ ಕರ್ಮವನ್ನು ಫಲೋದ್ದೇಶದಿಂದ ಮಾಡಿದರೆ ಯಾವ ಸಂತೋಷವೂ ಇರುವುದಿಲ್ಲವೆಂದು ತಿಳಿಯುವುದೋ, ಆಗ ಅವನು ಕರ್ಮ ತ್ಯಾಗ ಮಾಡುವನು. ಆದರೆ ಮನುಷ್ಯ ದೇಹವು ಕರ್ಮ ಮಾಡೇ ತೀರಬೇಕೆಂಬ ಸ್ಥಿತಿಯಲ್ಲಿರುವುದರಿಂದ ಅವನು ಎಂತಹ ಕರ್ಮ ಮಾಡಬೇಕು? ಅವನು ಯಾವುದಾದರೊಂದು ಕೆಲಸವನ್ನು ನಿಃಸ್ವಾರ್ಥವಾಗಿ, ಸ್ವಲ್ಪವೂ ಅದರ ಫಲದ ಮೇಲೆ ದೃಷ್ಟಿಯಿಲ್ಲದೆ ಮಾಡುವನು. ಏಕೆಂದರೆ ಅವನಿಗೆ ಗೊತ್ತು ಈ ಕರ್ಮಫಲದಲ್ಲಿ ಅಸಂಖ್ಯಾತ ಹುಟ್ಟು ಸಾವುಗಳ ಭವಿಷ್ಯ ಬೀಜ ಅಡಗಿದೆ ಎಂದು. ಅದಕ್ಕೆ ಬ್ರಹ್ಮಜ್ಞಾನಿಯು ಎಲ್ಲಾ ಕರ್ಮಗಳನ್ನೂ ತ್ಯಜಿಸುವನು. ಹೊರಗೆ ನೋಡಲು ಯಾವುದಾದರೊಂದು ಕರ್ಮದಲ್ಲಿ ನಿರತನಾಗಿದ್ದರೂ ಅವನಿಗೆ ಯಾವುದರಲ್ಲೂ ಮೋಹವಿಲ್ಲ. ಅಂತಹವರನ್ನು ಕರ್ಮಯೋಗಿಗಳೆಂದು ಶಾಸ್ತ್ರಗಳು ಕರೆಯುವುವು.

ಶಿಷ್ಯ: ಹಾಗಾದರೆ ಬ್ರಹ್ಮಜ್ಞಾನಿ ಮಾಡುವ ನಿಷ್ಕಾಮಕರ್ಮವೆಲ್ಲಾ ಒಬ್ಬ ಹುಚ್ಚನು ಮಾಡುವ ಕೆಲಸದಂತೆ ಇರುವುದೇ?

ಸ್ವಾಮೀಜಿ: ಅದೇಕೆ, ಪ್ರತಿಫಲಾಪೇಕ್ಷೆ ಬಿಡುವುದೆಂದರೆ, ಕೇವಲ ಸ್ವಾರ್ಥ ದೃಷ್ಟಿಯಿಂದ ಕರ್ಮ ಮಾಡಬಾರದು ಎಂದು ಅರ್ಥ. ಬ್ರಹ್ಮಜ್ಞಾನಿಯು ಎಂದೂ ಸ್ವಸುಖವನ್ನೆಣಿಸುವುದಿಲ್ಲ. ಆದರೆ ಮತ್ತೊಬ್ಬರ ಸುಖಕ್ಕಾಗಿ ಕೆಲಸ ಮಾಡುವುದಕ್ಕೆ ಅಡ್ಡಿಯೇನೂ ಇಲ್ಲ. ಪ್ರತಿಫಲದ ಆಸಕ್ತಿ ಇಲ್ಲದೆ ಅವನೇನು ಕೆಲಸಮಾಡಿದರೂ ಅದರಿಂದ ಜಗತ್ತಿಗೇ ಒಳ್ಳೆಯದಾಗುವುದು. ಅದು ಕೇವಲ ‘ಬಹುಜನರ ಹಿತಕ್ಕೆ ಬಹುಜನರ ಸುಖಕ್ಕೆ’. ಶ‍್ರೀರಾಮಕೃಷ್ಣರು ‘ಅವರೆಂದಿಗೂ ಒಂದು ತಪ್ಪು ಹೆಜ್ಜೆಯನ್ನೂ ಇಡುವುದಿಲ್ಲ’ ಎಂದು ಹೇಳುತ್ತಿದ್ದರು. ನೀನು ಉತ್ತರರಾಮಚರಿತವನ್ನೋದಿಲ್ಲವೆ? - ‘ಪುರಾತನ ಮಹರ್ಷಿಗಳ ಮಾತಿನಲ್ಲಿ ಯಾವಾಗಲೂ ಕೊಂಚ ಅರ್ಥವಿದ್ದೇ ಇರುತ್ತದೆ. ಅದೆಂದೂ ಸುಳ್ಳಲ್ಲ’. ಎಲ್ಲಾ ಬಗೆಯ ರೂಪಾಂತರಗಳನ್ನೂ ನಿಗ್ರಹಿಸಿ ಆತ್ಮನಲ್ಲಿ ಮನಸ್ಸು ತನ್ಮಯವಾದಾಗ ಅದರಿಂದ ‘ಇಹ ಅಥವಾ ಪರಲೋಕದಲ್ಲಿ ಕರ್ಮಫಲಾಪೇಕ್ಷೆಯಲ್ಲಿ ನಿರಾಸಕ್ತಿ’ ಉಂಟಾಗುವುದು. ಎಂದರೆ ಈ ಲೋಕದಲ್ಲಾಗಲಿ ಅಥವಾ ಸತ್ತ ಮೇಲೆ ಸ್ವರ್ಗದಲ್ಲಾಗಲಿ ಸುಖಾನುಭೋಗದ ಅಭಿಲಾಷೆ ಇರುವುದಿಲ್ಲ. ಮನಸ್ಸಿನಲ್ಲಿ ಆಸೆಯ ಹೋರಾಟವಿರುವುದಿಲ್ಲ. ಆದರೆ ಮನಸ್ಸು ಯಾವಾಗ ಜ್ಞಾನಾತೀತವಾದ ಅವಸ್ಥೆಯಿಂದ ‘ನಾನು, ನನ್ನದು’ ಎಂಬ ಈ ಪ್ರಪಂಚಕ್ಕೆ ಇಳಿದು ಬರುವುದೋ ಆಗ ಹಿಂದಿನ ಕರ್ಮ, ಅಭ್ಯಾಸ, ಸಂಸ್ಕಾರಗಳ ಪರಿಮಾಣಕ್ಕನುಸಾರ ದೇಹದ ಕ್ರಿಯೆಗಳು ಸಾಗುವುವು. ಮನಸ್ಸು ಆಗ ಸಾಧಾರಣವಾಗಿ ಜಾಗೃತಾತೀತಾವಸ್ಥೆಯಲ್ಲಿ ಇರುವುದು; ತಿನ್ನುವುದೇ ಮುಂತಾದ ಶರೀರ ಕ್ರಿಯೆಗಳು ಅನಿವಾರ್ಯವಾಗಿ ಮಾಡಲ್ಪಟ್ಟು ದೇಹಭಾವನೆ ಬಹುಮಟ್ಟಿಗೆ ಕಡಿಮೆಯಾಗಿರುತ್ತದೆ. ಈ ಅತೀಂದ್ರಿಯಾವಸ್ಥೆ ಪಡೆದ ಬಳಿಕ ಮಾಡುವ ಕೆಲಸಗಳೆಲ್ಲಾ ಯೋಗ್ಯವಾಗಿರುತ್ತವೆ. ಜನರಿಗೆ ಮತ್ತು ಜಗತ್ತಿಗೆ ಕಲ್ಯಾಣಕರವಾಗಿರುತ್ತವೆ. ಏಕೆಂದರೆ ಆಗ ಕರ್ತನ ಮನಸ್ಸು ಸ್ವಾರ್ಥದಿಂದ, ಸ್ವಂತ ಲಾಭ ನಷ್ಟಗಳ ಎಣಿಕೆಯಿಂದ ಕಲುಷಿತವಾಗಿರುವುದಿಲ್ಲ. ದೇವರು ತಾನು ಮಾತ್ರ ಯಾವಾಗಲೂ ಅತೀಂದ್ರಿಯಾವಸ್ಥೆಯಲ್ಲಿದ್ದು ಈ ಅದ್ಭುತವಾದ ಭೂಮಂಡಲವನ್ನು ಸೃಷ್ಟಿಸಿದ್ದಾನೆ. ಆದ್ದರಿಂದ ಈ ಪ್ರಪಂಚದಲ್ಲಿ ಯಾವುದೂ ಅಪೂರ್ಣವಾಗಿಲ್ಲ. ಆದ್ದರಿಂದ ಆತ್ಮಪರಿಜ್ಞಾನವುಳ್ಳವನು ಕರ್ಮಫಲಾಪೇಕ್ಷೆ ಇಲ್ಲದೆ ಮಾಡಿದ ಕರ್ಮಗಳೆಲ್ಲಾ ನ್ಯೂನಾತೀತವಾದುವು, ಜಗತ್ಕಲ್ಯಾಣಕರವಾದುವು ಎಂದು ಹೇಳಿದ್ದು.

ಶಿಷ್ಯ: ಸ್ವಾಮೀಜಿ, ಜ್ಞಾನ, ಕರ್ಮ ಎರಡೂ ಪರಸ್ಪರ ವಿರೋಧಾತ್ಮಕಗಳೆಂದು ಈಗ ತಾನೇ ತಾವು ಹೇಳಿದಿರಿ. ಪರಮಾರ್ಥಜ್ಞಾನದಲ್ಲಿ ಕರ್ಮಕ್ಕೆ ಸ್ಥಳವೇ ಇಲ್ಲ. ಬೇರೆ ರೀತಿಯಲ್ಲಿ ಹೇಳುವುದಾದರೆ ಕರ್ಮದ ಮೂಲಕ ಬ್ರಹ್ಮ ಸಾಕ್ಷಾತ್ಕಾರವೆಂದಿಗೂ ಲಭಿಸುವುದಿಲ್ಲ. ನೀವೇಕೆ ಒಮ್ಮೊಮ್ಮೆ ಈ ರೀತಿ ರಾಜಸ ಸ್ವಭಾವವನ್ನು ಉದ್ರೇಕಗೊಳಿಸುವ ಮಾತನ್ನಾಡುವಿರಿ? ಅಂದು ನೀವೇ ನನಗೆ ಹೇಳಿದಿರಿ ‘ಕೆಲಸ, ಕೆಲಸ - ಅದರ ಹೊರತು ಮತ್ತಾವ ಹಾದಿಯೂ ಇಲ್ಲ’ ಎಂದು.

ಸ್ವಾಮೀಜಿ: ಪ್ರಪಂಚವನ್ನೆಲ್ಲಾ ಸಂಚರಿಸಿ ಬಂದಮೇಲೆ ಇತರ ದೇಶಗಳೊಡನೆ ಹೋಲಿಸಿ ನೋಡಿದರೆ ನಮ್ಮ ದೇಶೀಯರು ಸಂಪೂರ್ಣ ತಾಮಸದಲ್ಲಿ ಮುಳುಗಿರುವುದನ್ನು ನೋಡಿದೆ. ಹೊರಗಡೆ ನೋಡಿದರೆ ತೋರಿಕೆಗೆ ಸಾತ್ತ್ವಿಕ ಸ್ವಭಾವವಿರುವಂತೆ ತೋರುವುದು. ಆದರೆ ಒಳಗಡೆ ಸಂಪೂರ್ಣ ಕಲ್ಲು ಮರಗಳಂತೆ ಜಡರಾಗಿರುವರು. ಇಂತಹ ವಿಷಯಸುಖದಲ್ಲಿ ತಲ್ಲೀನರಾದ ಸೋಮಾರಿಗಳು, ಜಡವ್ಯಕ್ತಿಗಳು ಈ ಜಗತ್ತಿನಲ್ಲಿರಬಲ್ಲರೆ? ಮೊದಲು ಪಾಶ್ಚಾತ್ಯ ದೇಶದಲ್ಲಿ ಸಂಚರಿಸು. ನಂತರ ನನ್ನ ಮಾತಿಗೆ ಬದಲು ಹೇಳು. ಪಾಶ್ಚಾತ್ಯರಲ್ಲಿ ಕೆಲಸ ಮಾಡಲು ಎಷ್ಟೊಂದು ಕೆಚ್ಚು ಮತ್ತು ಉತ್ಸಾಹ, ಎಂತಹ ಧೈರ್ಯ ಮತ್ತು ರಾಜಸ ಕಳೆಯಿದೆ! ನಿಮ್ಮ ದೇಶದಲ್ಲಿ ಜನರ ರಕ್ತ ಅವರ ಎದೆಯಲ್ಲೇ ಹೆಪ್ಪುಗಟ್ಟಿ ರಕ್ತನಾಳಗಳಲ್ಲಿ ಹರಿಯಲಾರದೋ ಎಂಬಂತೆ ಪಾರ್ಶ್ವವಾಯು ಬಂದು ನಿಸ್ತೇಜವಾಗಿದೆಯೋ ಎಂಬಂತಿದೆ. ಆದ್ದರಿಂದ ನನ್ನ ಉದ್ದೇಶವೇನೆಂದರೆ ಮೊದಲು ಜನರ ರಾಜಸ ಸ್ವಭಾವವನ್ನು ವೃದ್ಧಿಗೊಳಿಸಿ ಉತ್ಸಾಹ ಪಟುಗಳಾಗಿ ಜೀವಿಸಲು ಅರ್ಹರಾಗುವಂತೆ ಮಾಡಬೇಕು. ದೇಹದಲ್ಲಿ ಶಕ್ತಿ ಇಲ್ಲದೆ, ಎದೆಯಲ್ಲಿ ಉತ್ಸಾಹವಿಲ್ಲದೆ, ಮಿದುಳಿನಲ್ಲಿ ಸ್ವತಂತ್ರಾಲೋಚನೆಯಿಲ್ಲದೆ ಅವರೇನು ಮಾಡಬಲ್ಲರು? ಕೇವಲ ಜಡತ್ವದ ಮುದ್ದೆಗಳು!! ಅವರನ್ನು ಜಾಗೃತಗೊಳಿಸುವುದರ ಮೂಲಕ ಅವರಲ್ಲಿ ಜೀವಕಳೆಯನ್ನು ತುಂಬಲಿಚ್ಛಿಸುವೆನು. ಇದಕ್ಕಾಗಿ ನನ್ನ ಜೀವವನ್ನೇ ಮೀಸಲಾಗಿಟ್ಟಿರುವೆನು. ವೇದ ಮಂತ್ರಗಳ ಅಮೋಘ ಶಕ್ತಿಯಿಂದ ಅವರನ್ನು ಹುರಿದುಂಬಿಸುವೆನು. ‘ಉತ್ತಿಷ್ಠತ ಜಾಗ್ರತ ಪ್ರಾಪ್ಯವರಾನ್ನಿ ಬೋಧತ...’ ಎಂಬ ಅಭಯವಾಣಿಯನ್ನುಸಿರುವುದಕ್ಕಾಗಿಯೇ ನಾನು ಜನ್ಮವೆತ್ತಿದ್ದೇನೆ. ನೀವೆಲ್ಲಾ ನನ್ನ ಕಾರ್ಯಕ್ಕೆ ಸಹಾಯಕರಾಗಿ ಗ್ರಾಮಗ್ರಾಮಗಳಿಗೂ ಹೋಗಿ, ದೇಶದ ಮೂಲೆಮೂಲೆಗೂ ಹೋಗಿ ಈ ಅಭಯವಾಣಿಯನ್ನು ಬ್ರಾಹ್ಮಣರು, ಚಂಡಾಲರು ಎಲ್ಲರಿಗೂ ಬೋಧಿಸಿ. ಪ್ರತಿಯೊಬ್ಬರಿಗೂ ಅವರಲ್ಲಿ ಅನಂತಶಕ್ತಿ ಹುದುಗಿದೆ ಎಂದೂ, ಅಮೃತಾನುಭವಕ್ಕೆ ಅವರೆಲ್ಲಾ ಭಾಗಿಗಳೆಂದೂ ಬೋಧಿಸಿ. ಹೀಗೆ ಅವರ ರಾಜಸ ಸ್ವಭಾವವನ್ನು ಕೆರಳಿಸಿ. ಅವರನ್ನು ಜೀವನದ ಹೋರಾಟಕ್ಕೆ ಅರ್ಹರಾಗುವಂತೆ ಮಾಡಿ. ನಂತರ ಅವರೊಡನೆ ಮೋಕ್ಷದ ವಿಷಯ ಮಾತನಾಡಿ. ಮೊದಲು ದೇಶದ ಜನರು ಅವರ ಅಂತಃಶಕ್ತಿಯಿಂದ ತಮ್ಮ ಕಾಲಮೇಲೆ ತಾವು ನಿಲ್ಲುವಂತೆ ಮಾಡಿ. ಮೊದಲು ಅವರಿಗೆ ಒಳ್ಳೆಯ ಆಹಾರ, ಉಡುಪು, ಬೇಕಾದಷ್ಟು ಸುಖ ದೊರಕಿಸಿಕೊಳ್ಳುವುದನ್ನು ಕಲಿಯುವಂತೆ ಮಾಡಿ. ನಂತರ ಈ ಸುಖದ ಬಂಧನದಿಂದ ಪಾರಾಗುವುದನ್ನು ಬೋಧಿಸಿ. ದೇಶದಲ್ಲಿ ಆಮೂಲಾಗ್ರವಾಗಿ ಕೇವಲ ಸೋಮಾರಿತನ ಮತ್ತು ಕಪಟ ತುಂಬಿದೆ. ಯಾವ ಬುದ್ಧಿಯುಳ್ಳ ಮನುಷ್ಯ ತಾನೆ ಇದನ್ನು ನೋಡಿ ಸುಮ್ಮನಿರಲು ಸಾಧ್ಯ? ಕಣ್ಣೀರು ತುಂಬಿ ಬರುವುದಿಲ್ಲವೆ? ಮದ್ರಾಸು, ಬೊಂಬಾಯಿ, ಪಂಜಾಬು, ಬಂಗಾಳ, - ಎಲ್ಲಿ ನೋಡಿದರೂ ಜೀವಕಳೆಯೇ ಇಲ್ಲ. ನೀವೆಲ್ಲಾ ತುಂಬಾ ವಿದ್ಯಾವಂತರೆಂದು ಹೆಮ್ಮೆಪಡುವಿರಿ. ಯಾವ ಅಸಂಬದ್ಧ ವಿಷಯವನ್ನು ನೀವು ಕಲಿತಿರಿ? ಪರ ಭಾಷೆಯಿಂದ ಬಂದ ಯೋಚನೆಗಳನ್ನು ಗಿಳಿಪಾಠ ಮಾಡಿ ನಿಮ್ಮ ಮಿದುಳನ್ನು ಅದರಿಂದ ತುಂಬಿಕೊಂಡು ವಿಶ್ವವಿದ್ಯಾನಿಲಯದ ಪದವಿಯೊಂದನ್ನು ಗಳಿಸಿಬಿಟ್ಟು ನೀವು ಮಹಾವಿದ್ಯಾವಂತರೆಂದು ಗರ್ವಪಡುವಿರಲ್ಲವೆ? ನಿಮಗೆ ಧಿಕ್ಕಾರ! ಇದು ವಿದ್ಯಾಭ್ಯಾಸವೆ? ನಿಮ್ಮ ವಿದ್ಯಾಭ್ಯಾಸದ ಗುರಿ ಏನು? ಒಬ್ಬ ಗುಮಾಸ್ತ ಅಥವಾ ಮೋಸಗಾರನಾದ ವಕೀಲ, ಹೆಚ್ಚೆಂದರೆ, ಉಪನ್ಯಾಯಾಧಿಪತಿ; ಅದೂ ಕೂಡ ಮತ್ತೊಂದು ಬಗೆಯ ಗುಮಾಸ್ತಗಿರಿ - ಇಷ್ಟೇ ಅಲ್ಲವೆ? ನಿನಗಾಗಲಿ, ಇಡೀ ದೇಶಕ್ಕೇ ಆಗಲಿ ಇದರಿಂದ ಏನು ಉಪಯೋಗ? ಭರತನ ದೇಶ, ಐಶ್ವರ್ಯಕ್ಕೆ ಹೆಸರುವಾಸಿಯಾದ ದೇಶದಲ್ಲಿಂದು ಅನ್ನಕ್ಕಾಗಿ ಎಂತಹ ಹೃದಯ ವಿದ್ರಾವಕ ಧ್ವನಿ ಕೇಳಿಬರುತ್ತಿದೆ! ನಿನ್ನ ವಿದ್ಯೆಯಿಂದ ಈ ಹಸಿವನ್ನಡಗಿಸಲು ಸಾಧ್ಯವೇ? ಎಂದಿಗೂ ಇಲ್ಲ. ಪಾಶ್ಚಾತ್ಯ ವಿಜ್ಞಾನದ ಸಹಾಯದಿಂದ ಭೂಮಿಯನ್ನು ಅಗೆಯಲು ತೊಡಗಿ - ಆಹಾರ ಪದಾರ್ಥಗಳನ್ನು ಉತ್ಪತ್ತಿ ಮಾಡಿ. ಆ ನಿಮ್ಮ ತುಚ್ಚ ಗುಲಾಮಗಿರಿಯಿಂದಲ್ಲ - ಪಾಶ್ಚಾತ್ಯ ವಿಜ್ಞಾನದ ಸಲಹೆ ಪಡೆದು ನಿಮ್ಮ ಸ್ವಪ್ರಯತ್ನವನ್ನೂ ಸೇರಿಸಿರಿ. ಆಹಾರ ಪದಾರ್ಥಗಳನ್ನೊದಗಿಸಲು ಹೊಸ ಹಾದಿಯನ್ನು ಕಂಡುಹಿಡಿಯಿರಿ. ಅದಕ್ಕೇ ನಾನು ಈ ದೇಶದ ಜನರಿಗೆ ಮೊದಲು ಉತ್ಸಾಹಪೂರಿತವಾಗಿ ತಮ್ಮ ಅನ್ನ ಬಟ್ಟೆಯನ್ನು ತಾವೇ ಒದಗಿಸಿಕೊಳ್ಳುವಂತೆ ಬೋಧಿಸುತ್ತೇನೆ. ಈ ಆಹಾರ ವಸ್ತ್ರಗಳ ಅಭಾವದಿಂದ ಇಡೀ ದೇಶವೇ ಹಾಹಾಕಾರದಿಂದ ತುಂಬಿದೆ. ದೇಶ ಅಧೋಗತಿಗಿಳಿದಿದೆ - ಇದನ್ನು ಪರಿಹರಿಸಲು ನೀವೇನು ಮಾಡಿದ್ದೀರಿ? ಗಂಗೆಯಲ್ಲಿ ನಿಮ್ಮ ಶಾಸ್ತ್ರಗಳನ್ನು ಬಿಸುಟು ಜನರಿಗೆ ಮೊದಲು ಅವರ ಅನ್ನವಸ್ತ್ರ ಸಂಪಾದಿಸುವ ಹಾದಿಯನ್ನು ಬೋಧಿಸಿ. ನಂತರ ಅವರಿಗೆ ನಿಮ್ಮ ಶಾಸ್ತ್ರಗಳನ್ನು ಓದಿ ಹೇಳಿ. ತೀವ್ರ ಚಟುವಟಿಕೆ ಅವರಲ್ಲಿ ಉದ್ರೇಕಿಸುವಂತೆ ಮಾಡಿ. ಈ ಪ್ರಾಪಂಚಿಕ ಸುಖಾಭಿಲಾಷೆ ತೀರುವಂತೆ ಮಾಡದಿದ್ದಲ್ಲಿ ಯಾರೂ ಆಧ್ಯಾತ್ಮಿಕ ಭಾವನೆಗೆ ಕಿವಿಗೊಡುವುದಿಲ್ಲ. ಅದಕ್ಕೆ ಮೊದಲು ನಿಮ್ಮ ಹೃದಯಾಂತರಾಳದಲ್ಲಿ ಅಂತರ್ಗತವಾಗಿರುವ ಆ ಶಕ್ತಿಯಲ್ಲಿ ಸಾಧ್ಯವಾದಷ್ಟು ಶ್ರದ್ಧೆ ಇರಲಿ. ಮೊದಲು ಅವರು ಆಹಾರಕ್ಕೆ ಬೇಕಾದ ಏರ್ಪಾಡನ್ನು ಮಾಡುವಂತೆ ಬೋಧಿಸಿ, ನಂತರ ಧರ್ಮಬೋಧೆ ಮಾಡಿ. ಸೋಮಾರಿಯಂತೆ ಕುಳಿತುಕೊಳ್ಳಲು ಸ್ವಲ್ಪವೂ ಕಾಲವಿಲ್ಲ - ಮೃತ್ಯುವಿನ ಕರೆ ಯಾವಾಗ ಬರುವುದೋ ಯಾರಿಗೆ ಗೊತ್ತು?

ಹೀಗೆ ಹೇಳುತ್ತಿರುವಾಗ ಅವರ ಮುಖವು ಮರುಕ, ದುಃಖ, ಸಹಾನುಭೂತಿ, ಶಕ್ತಿಗಳಿಂದ ಮಿಳಿತವಾದೊಂದು ಕಾಂತಿಯಿಂದ ಹೊಳೆಯುತ್ತಿತ್ತು. ಆ ಗಂಭೀರ ರೂಪ ಶಿಷ್ಯನನ್ನು ಮೂಕನನ್ನಾಗಿ ಮಾಡಿತು. ಸ್ವಲ್ಪ ಕಾಲದ ನಂತರ ಸ್ವಾಮಿಗಳು ಪುನಃ ಹೇಳಿದರು: “ಸಕಾಲದಲ್ಲಿ ಆ ಉತ್ಸಾಹ, ಮತ್ತು ಸ್ವಾವಲಂಬನ ಈ ದೇಶದ ಜನರಿಗೆ ಬಂದೇ ತೀರುವುದು - ನಾನೀಗ ಅದನ್ನು ಸ್ಪಷ್ಟವಾಗಿ ಕಾಣುತ್ತಿರುವೆ - ಅದರಿಂದ ತಪ್ಪಿಸಿಕೊಳ್ಳುವಂತಿಲ್ಲ. ವಿಚಾರ ಶೀಲನಾದ ವ್ಯಕ್ತಿ ತನ್ನ ಮುಂದಿನ ಮೂರು ಯುಗಗಳ ಭವಿಷ್ಯದ ದೃಶ್ಯವನ್ನು ಸ್ಪಷ್ಟವಾಗಿ ಕಾಣಬಲ್ಲನು. ಶ‍್ರೀರಾಮಕೃಷ್ಣರ ಅವತಾರವಾದಂದಿನಿಂದ ಪೂರ್ವದಿಗಂತ ಪ್ರಾತಃಕಾಲದ ಸೂರ್ಯನ ಹೊಂಬಣ್ಣದ ಕಾಂತಿಯಿಂದ ಹೊಳೆಯುತ್ತಿದೆ. ಕ್ರಮೇಣ ದೇಶವೆಲ್ಲಾ ಸೂರ್ಯನ ಅಪರಾಹ್ನದ ಕಾಂತಿಯಿಂದ ಪ್ರಜ್ವಲಿಸುವುದು."

\newpage

\chapter[ಅಧ್ಯಾಯ ೩೦]{ಅಧ್ಯಾಯ ೩೦\protect\footnote{\engfoot{Complete Works of Swami Vivekananda, Volume VI, Page 445}}}

\begin{center}
ಸ್ಥಳ: ಬೇಲೂರು ಮಠ, ವರ್ಷ: ಕ್ರಿ.ಶ. ೧೮೯೯.
\end{center}

ಈಗ ಇರುವ ಬೇಲೂರು ಮಠದ ಕಟ್ಟಡವು ಸುಮಾರಾಗಿ ಮುಕ್ಕಾಲು ಪಾಲು ಪೂರೈಸಿತ್ತು. ಸ್ವಾಮಿಗಳಿಗೆ ಆರೋಗ್ಯ ಚೆನ್ನಾಗಿಲ್ಲ. ವೈದ್ಯರು ಅವರನ್ನು ಬೆಳಿಗ್ಗೆ ಸಂಜೆ ಎರಡು ಹೊತ್ತೂ ಗಂಗಾನದಿಯಲ್ಲಿ ದೋಣಿಯಲ್ಲಿ ಹೋಗುವ ಹಾಗೆ ಹೇಳಿದ್ದರು.

ಇಂದು ಆದಿತ್ಯವಾರ ಶಿಷ್ಯನು ಸ್ವಾಮೀಜಿಯವರ ಕೋಣೆಯಲ್ಲಿ ಕುಳಿತು ಮಾತನಾಡುತ್ತಿದ್ದಾನೆ. ಸ್ವಾಮೀಜಿಯವರು ಮಠದ ಸಂನ್ಯಾಸಿಗಳ, ಬ್ರಹ್ಮಚಾರಿಗಳ ತರಬೇತಿಗಾಗಿ ಕೆಲವು ನಿಯಮಗಳನ್ನು ನಿಯೋಜಿಸಿದ್ದರು. ಪ್ರಾಪಂಚಿಕ ಜನರೊಡನೆ ವಿವೇಚನೆಯಿಲ್ಲದೆ ಅವರು ಮಿಳಿತವಾಗುವುದನ್ನು ತಪ್ಪಿಸಲು ಇದನ್ನು ತಯಾರಿಸಿದ್ದರು.

ಸ್ವಾಮಿಜಿ: ಈ ದಿನಗಳಲ್ಲಿ ಪ್ರಾಪಂಚಿಕ ಜನರ ಉಡುಪಿನಲ್ಲಿ ಒಂದು ಬಗೆಯ ಪ್ರತ್ಯೇಕವಾದ ಆತ್ಮಸಂಯಮದ ಅಸಡ್ಡೆಯ ವಾಸನೆಯಿರುವುದನ್ನು ನೋಡುತ್ತಿದ್ದೇನೆ. ಅದಕ್ಕೆ ನಾನು ಮಠದಲ್ಲಿ ಒಂದು ನಿಯಮವನ್ನು ತಂದಿದ್ದೇನೆ. ಗೃಹಸ್ಥರು ಸಾಧುಗಳ ಹಾಸಿಗೆಯಲ್ಲಿ ಕುಳಿತುಕೊಳ್ಳುವುದಾಗಲಿ, ಮಲಗುವುದಾಗಲಿ, ನಿಷಿದ್ಧ. ಈ ವಾಸನೆ ಬರುವುದು ನಿಜವೆಂದೂ, ಅದಕ್ಕೇ ಸಂನ್ಯಾಸಿಗಳಿಗೆ ಗೃಹಸ್ಥರ ವಾಸನೆಯನ್ನು ಸಹಿಸಲಾಗುವುದಿಲ್ಲವೆಂದೂ ಹಿಂದೆ ನಾನು ಶಾಸ್ತ್ರವನ್ನೋದುತ್ತಿದ್ದಾಗ ಓದಿದ್ದೆ. ಈಗ ಅದು ನಿಜವೆಂದು ಅನ್ನಿಸುತ್ತದೆ. ನಿಯೋಜಿಸಿರುವ ಈ ನಿಯಮಗಳನ್ನು ಕಠಿಣವಾಗಿ ಆಚರಿಸುವುದರಿಂದ ಬ್ರಹ್ಮಚಾರಿಗಳು ಸಕಾಲದಲ್ಲಿ ಯೋಗ್ಯ ಸಂನ್ಯಾಸಿಗಳಾಗುವರು. ಅವರು ಸಂನ್ಯಾಸದಲ್ಲಿ ಊರ್ಜಿತವಾದ ಮೇಲೆ ಅವರು ಗೃಹಸ್ಥರ ಜೊತೆಜೊತೆಗೆ ಬೆರೆತರೂ ಯಾವ ಹಾನಿಯೂ ಇಲ್ಲ. ಆದರೆ ಈಗ ಅವರನ್ನು ಈ ನಿಯಮಗಳ ಎಲ್ಲೆಯಲ್ಲಿಡದೆ ಇದ್ದಲ್ಲಿ ಅವರೆಲ್ಲಾ ತಪ್ಪು ಹಾದಿಗೆ ತಿರುಗುವರು. ಬ್ರಹ್ಮಚರ್ಯ ದೀಕ್ಷೆ ಪಡೆಯಲು ನಡತೆಯ ವಿಚಾರವಾಗಿ ಬಹಳ ಎಚ್ಚರಿಕೆಯಿಂದಿರಬೇಕು. ಸ್ತ್ರೀಯರೊಡನೆ ಯಾವ ಬಗೆಯ ಸಂಪರ್ಕವನ್ನೂ ಇಟ್ಟುಕೊಳ್ಳದೆ ದೂರವಿರಬೇಕು. ಅಲ್ಲದೆ ಮದುವೆಯಾದವರ ಸಂಗದಿಂದಲೂ ದೂರವಿರಬೇಕು.

ಗೃಹಸ್ಥನಾಗಿದ್ದ ಶಿಷ್ಯ ಸ್ವಾಮಿಗಳ ಈ ಮಾತನ್ನು ಕೇಳಿ ಸ್ತಂಭೀಭೂತನಾದ. ಮಠದ ಸಾಧುಗಳೊಡನೆ ಇನ್ನು ಮುಂದೆ ಇಷ್ಟೊಂದು ಸದರದಿಂದ ಇರಲಾಗುವುದಿಲ್ಲವೆಂದು ಖಿನ್ನನಾಗಿ ಹೇಳಿದ: “ಸ್ವಾಮಿಜಿ, ನನ್ನ ಸಂಸಾರದವರೊಂದಿಗಿಂತ ನನಗೆ ಮಠ ಮತ್ತು ಅದರ ನಿವಾಸಿಗಳೊಡನೆ ನಿಕಟಸಂಬಂಧವಿದೆ. ದೀರ್ಘಕಾಲದಿಂದ ಅವರಿಗೆ ನನ್ನ ಪರಿಚಯ ಎಂಬಂತಿವೆ. ಮಠದಲ್ಲಿ ನನಗೆ ಇರುವ ಅಪಾರ ಸಲಿಗೆ ಜಗತ್ತಿನಲ್ಲಿ ನನಗೆ ಮತ್ತೆಲ್ಲಿಯೂ ದೊರಕುವುದಿಲ್ಲವೆನ್ನಿಸುತ್ತದೆ."

ಸ್ವಾಮೀಜಿ: ಯಾರು ಪರಿಶುದ್ಧಾತ್ಮರಾಗಿರುವರೊ ಅವರಿಗೆ ಇಲ್ಲಿಗೆ ಬಂದಾಗ ಹಾಗೇ ಅನ್ನಿಸುವುದು. ಯಾರಿಗೆ ಆ ರೀತಿ ಅನ್ನಿಸುವುದಿಲ್ಲವೋ ಅವರು ಈ ಮಠದ ಧ್ಯೇಯಕ್ಕೆ ಸೇರಿದವರಲ್ಲ. ಅದಕ್ಕೆ ಅನೇಕ ಮಂದಿ ಇಲ್ಲಿಗೆ ತತ್ಕಾಲೀನ ಭಾವೋದ್ರೇಕದಿಂದ ಬಂದು ನಂತರ ಪಲಾಯನ ಮಾಡುತ್ತಾರೆ. ಯಾರು ಸಂಯಮಿಗಳಲ್ಲವೊ ಹಗಲು ರಾತ್ರಿ ಹಣವನ್ನರಸುತ್ತಿರುವರೋ ಅವರಿಗೆ ಮಠದ ಧ್ಯೇಯಗಳಾವುವೂ ಗಣ್ಯವಾಗುವುದಿಲ್ಲ. ಮಠದ ನಿವಾಸಿಗಳೂ ತಮ್ಮ ಆಪ್ತರೆಂದು ಅನ್ನಿಸುವುದಿಲ್ಲ. ಈ ಮಠದ ನಿವಾಸಿಗಳು ಪೂರ್ವದ ಸಂನ್ಯಾಸಿಗಳಂತೆ ಮೈಗೆಲ್ಲಾ ಬೂದಿ ಬಳಿದುಕೊಂಡು ಜಡೆಗಟ್ಟಿ, ಜಾಗಟೆ ಬಾರಿಸುತ್ತಾ ಹರಕು ಮುರುಕು ಔಷಧಿಗಳಿಂದ ಖಾಯಿಲೆ ಗುಣ ಮಾಡುತ್ತಿರುವಂತೆ ಇರುವುದಿಲ್ಲ. ಈ ವ್ಯತ್ಯಾಸವನ್ನು ನೋಡಿ ಜನರು ಅವರನ್ನು ಮೆಚ್ಚುವುದಿಲ್ಲ. ಗುರುಮಹಾರಾಜರು ನಡೆಯುತ್ತಿದ್ದ ರೀತಿ, ನಡೆವಳಿಕೆ ಮತ್ತು ಧ್ಯೇಯಗಳೆಲ್ಲ ಹೊಸ ಎರಕದಲ್ಲಿವೆ. ಆದ್ದರಿಂದ ನಾವೂ ಒಂದು ನೂತನ ವರ್ಗಕ್ಕೆ ಸೇರಿದವರು. ಕೆಲವು ವೇಳೆ ನಾವು ಸಭ್ಯರಂತೆ ಒಳ್ಳೆಯ ಉಡುಪನ್ನು ಧರಿಸಿ ಉಪನ್ಯಾಸ ಕೊಡುವೆವು. ಮತ್ತೆ ಕೆಲವು ವೇಳೆ ಎಲ್ಲವನ್ನೂ ಬಿಸುಟು ಹರ! ಹರ! ಓಂ! ಓಂ! ಎಂದು ಕೂಗುತ್ತಾ ಭಸ್ಮಲೇಪಿತರಾಗಿ ಪರ್ವತಗಳಲ್ಲಿ ಧ್ಯಾನ ತಪಶ್ಚರ್ಯದಲ್ಲಿ ನಿರತರಾಗಿರುವೆವು.

“ಈಗ ಕೇವಲ ಪುರಾತನ ಕಾಲದ ಗ್ರಂಥಗಳನ್ನು ಪಠಿಸುವುದರಿಂದೇನೂ ಪ್ರಯೋಜನವಿಲ್ಲ. ಪಾಶ್ಚಾತ್ಯ ನಾಗರಿಕತೆಯ ಪ್ರವಾಹ ರಭಸದಿಂದ ದೇಶಾದ್ಯಂತ ಪ್ರವಹಿಸುತ್ತಿದೆ. ಅದರ ಉಪಯೋಗವನ್ನು ತಿಳಿದುಕೊಳ್ಳದೆ ಗಿರಿಶಿಖರಗಳಲ್ಲಿ ಕೇವಲ ಧ್ಯಾನದಲ್ಲಿ ಕುಳಿತಿದ್ದರೆ ಪ್ರಯೋಜನವಿಲ್ಲ. ಗೀತೆಯಲ್ಲಿ ಶ‍್ರೀಕೃಷ್ಣನು ಹೇಳಿರುವ ಉಗ್ರ ಕರ್ಮಯೋಗ ಈಗ ಬೇಕಾಗಿದೆ. ಅಪಾರ ಧೈರ್ಯ, ಅದಮ್ಯ ಉತ್ಸಾಹದಿಂದ ಕೂಡಿದ ಕರ್ಮಯೋಗ ಬೇಕಾಗಿದೆ. ಆಗ ಮಾತ್ರ ದೇಶದ ಜನ ಜಾಗೃತಿಗೊಳ್ಳುವರು. ಇಲ್ಲದಿದ್ದಲ್ಲಿ ನೀನೆಷ್ಟು ಅಜ್ಞಾನದಲ್ಲಿರುವೆಯೋ ಅದೇ ಅಜ್ಞಾನಕೂಪದಲ್ಲಿ ಅವರು ಮುಳುಗಿರುವರು."

ಸುಮಾರು ಹಗಲೆಲ್ಲಾ ಕಳೆಯಿತು. ಸ್ವಾಮೀಜಿ ಗಂಗೆಯ ಮೇಲೆ ದೋಣಿ ಸಂಚಾರಕ್ಕೆ ಹೊರಡಲು ಉಡುಪನ್ನು ಧರಿಸಿ ಕೆಳಗೆ ಬಂದರು. ಈ ಶಿಷ್ಯನಲ್ಲದೆ ಮತ್ತಿಬ್ಬರು ಶಿಷ್ಯರನ್ನೂ ಜೊತೆಯಲ್ಲಿ ಕರೆದುಕೊಂಡು ದೋಣಿಯೊಳಕ್ಕೆ ಪ್ರವೇಶಿಸಿದರು. ದೋಣಿ ದಕ್ಷಿಣೇಶ್ವರ ದೇವಾಲಯವನ್ನು ಹಾದು ಪಣಿಹಾಟಿಯಲ್ಲಿ ಬಾಬು ಗೋವಿಂದಕುಮಾರ ಚೌಧರಿಯವರ ತೋಟದ ಮನೆಯ ಕೆಳಗೆ ಲಂಗರು ಹಾಕಿತು. ಒಮ್ಮೆ ಆ ಮನೆಯನ್ನು ಮಠದ ಉಪಯೋಗಕ್ಕಾಗಿ ಬಾಡಿಗೆಗೆ ತೆಗೆದುಕೊಳ್ಳುವ ಪ್ರಸ್ತಾವ ಇತ್ತು. ಸ್ವಾಮೀಜಿ ದೋಣಿಯಿಂದಿಳಿದು ಮನೆ ಮತ್ತು ತೋಟದ ಸುತ್ತಲೂ ಒಮ್ಮೆ ಸುತ್ತಾಡಿ ಬಂದು ಅದನ್ನು ಸೂಕ್ಷ್ಮವಾಗಿ ನೋಡುತ್ತಾ ಹೀಗೆಂದರು: “ತೋಟವೇನೋ ಅಂದವಾಗಿದೆ. ಆದರೆ ಕಲ್ಕತ್ತೆಗೆ ಬಹುದೂರ. ಕಲ್ಕತ್ತೆಯಿಂದ ಇಲ್ಲಿಗೆ ಕಾಲ್ನಡಿಗೆಯಲ್ಲಿ ಬರಲು ಶ‍್ರೀರಾಮಕೃಷ್ಣರ ಭಕ್ತವರ್ಗಕ್ಕೆ ತುಂಬಾ ತೊಂದರೆಯಾಗುತ್ತಿತ್ತು. ಮಠ ಇಲ್ಲಿ ಸ್ಥಾಪಿತವಾಗದಿದ್ದುದು ಒಂದು ಸುಯೋಗ." ನಂತರ ದೋಣಿಯು ಮುಸುಕಿದ ಅಂಧಕಾರದ ಮಧ್ಯೆ ಬೇಲೂರು ಮಠಕ್ಕೆ ಹಿಂತಿರುಗಿತು.

\newpage

\chapter[ಅಧ್ಯಾಯ ೩೧]{ಅಧ್ಯಾಯ ೩೧\protect\footnote{\engfoot{C.W. Vol. VII, P. 186}}}

\begin{center}
* - ಸ್ಥಳ: ಬೇಲೂರು ಮಠ, ವರ್ಷ: ಕ್ರಿ.ಶ. ೧೮೯೯.
\end{center}

ನಾಗಮಹಾಶಯರೊಡನೆ ಶಿಷ್ಯ ಇಂದು ಮಠಕ್ಕೆ ಬಂದಿದ್ದಾನೆ. ಸ್ವಾಮೀಜಿ ನಾಗಮಹಾಶಯರಿಗೆ ಕೈಮುಗಿದು “ಕ್ಷೇಮವೆಂದು ಭಾವಿಸಿದ್ದೇನೆ" ಎಂದರು.

ನಾಗಮಹಾಶಯ: ನಾನಿಂದು ನಿಮ್ಮ ದರ್ಶನಕ್ಕೆ ಬಂದಿರುವೆ. ಜಯಶಂಕರ! ಜಯಶಂಕರ! ಇಂದು ಶಿವದರ್ಶನದಿಂದ ನಾನು ಪುನೀತನಾದೆ ಎಂದು ಹೇಳುತ್ತಾ ನಾಗಮಹಾಶಯರು ಭಕ್ತಿಯಿಂದ ಕೈಜೋಡಿಸಿ ನಿಂತರು.

ಸ್ವಾಮೀಜಿ: ನಿಮ್ಮ ಆರೋಗ್ಯ ಈಗ ಹೇಗಿದೆ?

ನಾಗಮಹಾಶಯ: ಈ ಕ್ಷುದ್ರದೇಹದ ವಿಷಯವನ್ನೇಕೆ ಎತ್ತುವಿರಿ? ಈ ಮಾಂಸ ಮೂಳೆಯಿಂದ ಕೂಡಿದ ಪಂಜರ! ಸತ್ಯವಾಗಿ ಇಂದು ನಿಮ್ಮ ದರ್ಶನದಿಂದ ನಾನು ಧನ್ಯನಾದೆ.

ಹೀಗೆ ಹೇಳುತ್ತಾ ನಾಗಮಹಾಶಯರು ಸ್ವಾಮಿಗಳಿಗೆ ದೀರ್ಘದಂಡ ಪ್ರಣಾಮ ಮಾಡಿದರು.

ಸ್ವಾಮಿಜಿ: (ಅವರನ್ನು ಎಬ್ಬಿಸುತ್ತಾ) ನೀವೇಕೆ ನನಗೆ ಹೀಗೆ ಮಾಡುವಿರಿ?

ನಾಗಮಹಾಶಯ: ನನ್ನ ಅಂತರ್‌ದೃಷ್ಟಿಯಿಂದ ಇಂದು ಸಾಕ್ಷಾತ್ ಶಿವದರ್ಶನವನ್ನು ನೋಡುತ್ತಿರುವೆ – ನಾನು ಧನ್ಯ! ಜಯ ರಾಮಕೃಷ್ಣ!

ಸ್ವಾಮೀಜಿ: (ಶಿಷ್ಯನನ್ನುದ್ದೇಶಿಸಿ) ನೋಡುತ್ತಿರುವೆಯೇನು? ಇಲ್ಲಿ ನೋಡು! ನಿಜವಾದ ಭಕ್ತಿ ಮಾನವನ ಸ್ವಭಾವವನ್ನು ಎಷ್ಟರಮಟ್ಟಿಗೆ ಬದಲಾಯಿಸುತ್ತದೆಂಬುದನ್ನು. ನಾಗಮಹಾಶಯರು ದೈವೀಭಾವದಲ್ಲಿ ತನ್ಮಯರಾಗಿದ್ದಾರೆ. ಅವರ ದೇಹಭಾವನೆ ಸಂಪೂರ್ಣವಾಗಿ ಅಳಿಸಿಹೋಗಿದೆ. (ಸ್ವಾಮಿ ಪ್ರೇಮಾನಂದರಿಗೆ) ನಾಗಮಹಾಶಯರಿಗೆ ಕೊಂಚ ಪ್ರಸಾದವನ್ನು ತಾ.

ನಾಗಮಹಾಶಯ: ಪ್ರಸಾದ! (ಕೈಜೋಡಿಸಿ) ನಿಮ್ಮನ್ನು ನೋಡಿ ಇಂದು ನನ್ನ ಪ್ರಾಪಂಚಿಕ ಕ್ಷುಧೆ ಅಡಗಿಹೋಗಿದೆ.

ಮಠದ ಬ್ರಹ್ಮಚಾರಿಗಳು, ಸಂನ್ಯಾಸಿಗಳು ಉಪನಿಷತ್ತನ್ನು ವ್ಯಾಸಂಗ ಮಾಡುತ್ತಿದ್ದರು. ಸ್ವಾಮೀಜಿ ಅವರಿಗೆ, “ಇಂದು ಶ‍್ರೀರಾಮಕೃಷ್ಣರ ಹಿರಿಯ ಭಕ್ತರ ಆಗಮನವಾಗಿದೆ. ಮಠಕ್ಕೆ ನಾಗಮಹಾಶಯರು ಭೇಟಿ ಇತ್ತ ನಿಮಿತ್ತ ಇಂದು ವಿಶ್ರಾಂತಿ ದಿನವಾಗಲಿ" ಎಂದರು. ಎಲ್ಲರೂ ಪುಸ್ತಕಗಳನ್ನು ಮುಚ್ಚಿ ನಾಗಮಹಾಶಯರ ಸುತ್ತಲೂ ಕುಳಿತುಕೊಂಡರು.

ಸ್ವಾಮೀಜಿ: (ಎಲ್ಲರನ್ನೂ ಉದ್ದೇಶಿಸಿ) ನೀವು ನೋಡುತ್ತಿಲ್ಲವೆ? ನಾಗಮಹಾಶಯರನ್ನು ನೋಡಿ! ಅವರು ಒಬ್ಬ ಗೃಹಸ್ಥರು. ಆದರೂ ಸಂಸಾರದ ಪರಿವೆ ಅವರಿಗೆ ಕೊಂಚವೂ ಇಲ್ಲ. ಯಾವಾಗಲೂ ದೈವೀಭಾವದಲ್ಲಿ ಮಗ್ನರಾಗಿರುವರು. (ನಾಗಮಹಾಶಯರಿಗೆ) ದಯವಿಟ್ಟು ನಮಗೂ ಈ ಬ್ರಹ್ಮಚಾರಿಗಳಿಗೂ ಶ‍್ರೀರಾಮಕೃಷ್ಣರ ವಿಷಯವಾಗಿ ಏನಾದರೂ ಕೊಂಚ ಹೇಳಿ.

ನಾಗಮಹಾಶಯ: (ಪೂಜ್ಯಭಾವದಿಂದ) ನೀವೇನು ಹೇಳುತ್ತಿರುವುದು? ನಾನೇನು ಹೇಳಲಿ? ನಾನು ನಿಮ್ಮನ್ನು ನೋಡಲು, ಈ ವೀರರನ್ನು, ಶ‍್ರೀರಾಮಕೃಷ್ಣರ ದೈವೀಲೀಲೆಯ ಸಹಾಯಕರಾದ ನಿಮ್ಮನ್ನು ನೋಡಲು ಬಂದೆ. ಈಗ ಅವರ ಸಂದೇಶ ಬೋಧನೆಗಳನ್ನು ಜನರೆಲ್ಲಾ ಮೆಚ್ಚುವರು. ಜಯ ರಾಮಕೃಷ್ಣ.

ಸ್ವಾಮೀಜಿ: ಶ‍್ರೀರಾಮಕೃಷ್ಣರನ್ನು ಸರಿಯಾಗಿ ಅರ್ಥಮಾಡಿಕೊಂಡು ಮೆಚ್ಚಿದವರೆಂದರೆ ನೀವೊಬ್ಬರೇ. ನಾವೆಲ್ಲಾ ಕೇವಲ ನಿರರ್ಥಕವಾದ ಅಲೆದಾಟದಲ್ಲಿ ಕಳೆದವರು.

ನಾಗಮಹಾಶಯ: ಏನು ಮಾತನಾಡುತ್ತಿರುವಿರಿ? ನೀವು ಶ‍್ರೀರಾಮಕೃಷ್ಣರ ಪ್ರತಿಬಿಂಬ. ನಾಣ್ಯದ ಎರಡು ಮುಖಗಳಂತೆ. ಯಾರಿಗೆ ಕಣ್ಣಿದೆಯೋ ಅವರು ನೋಡಲಿ.

ಸ್ವಾಮೀಜಿ: ಈ ಮಠ ಮತ್ತು ಆಶ್ರಮಗಳ ಸ್ಥಾಪನೆ ಮುಂತಾದುವು ಸರಿಯಾದ ಹಾದಿಯಲ್ಲಿ ಇಟ್ಟಿರುವ ಹೆಜ್ಜೆಯೆ?

ನಾಗಮಹಾಶಯ: ನಾನೊಬ್ಬ ಕ್ಷುದ್ರ ಮನುಜ. ನನಗೇನು ಅರ್ಥವಾಗುತ್ತದೆ? ನೀವೇನು ಮಾಡಿದರೂ ಅದರಿಂದ ಜಗಕ್ಕೆ ಕಲ್ಯಾಣವಾಗುತ್ತದೆ. ಖಂಡಿತವಾಗಿ ಪ್ರಪಂಚಕ್ಕೆ ಒಳ್ಳೆಯದಾಗುತ್ತದೆ.

ಅನೇಕರು ಪೂಜ್ಯಭಾವದಿಂದ ನಾಗಮಹಾಶಯರ ಪಾದಧೂಳಿಯನ್ನು ಧರಿಸಲು ಅವರ ಹತ್ತಿರ ಬಂದರು. ಇದರಿಂದ ಅವರು ಮತ್ತಷ್ಟು ಕಳವಳಗೊಂಡರು. ಸ್ವಾಮೀಜಿ ಅವರನ್ನುದ್ದೇಶಿಸಿ “ನಾಗಮಹಾಶಯರಿಗೆ ನೋವುಂಟುಮಾಡಬೇಡಿ. ಅವರು ವ್ಯಥಿತರಾಗುವಂತೆ ಮಾಡಬೇಡಿ." ಇದನ್ನು ಕೇಳಿದ ಮೇಲೆ ಎಲ್ಲರೂ ಮೌನವಾದರು.

ಸ್ವಾಮಿಜಿ: ದಯವಿಟ್ಟು ಮಠಕ್ಕೆ ಆಗಾಗ್ಗೆ ಬಂದು ಹೋಗುತ್ತಾ ಇರಿ. ಇಲ್ಲಿನ ಹುಡುಗರಿಗೆಲ್ಲಾ ಮೇಲ್ಪಂಕ್ತಿಯಾಗಿರುವಿರಿ.

ನಾಗಮಹಾಶಯ: ನಾನು ಒಮ್ಮೆ ಶ‍್ರೀರಾಮಕೃಷ್ಣರನ್ನು ಈ ವಿಷಯವಾಗಿ ಕೇಳಿದೆ. ಅವರು ‘ನೀನು ಈಗ ಇರುವಂತೆಯೇ ಗೃಹಸ್ಥನಾಗಿರು’ ಎಂದರು. ಅಂತೆಯೇ ನಾನು ಆ ಜೀವನವನ್ನೇ ಅನುಸರಿಸುತ್ತಿರುವೆ. ನಾನು ಆಗಾಗ್ಗೆ ಇಲ್ಲಿಗೆ ಬಂದು ಧನ್ಯನಾಗುವೆ.

ಸ್ವಾಮೀಜಿ: ನಾನು ನಿಮ್ಮ ಸ್ಥಳಕ್ಕೆ ಒಮ್ಮೆ ಬರಬೇಕೆಂದಿರುವೆ.

ನಾಗಮಹಾಶಯ: (ಆನಂದದಿಂದ ಉನ್ಮತ್ತರಾಗಿ) ಅಂತಹ ಸುದಿನ ನಿಜವಾಗಿ ಉದಿಸುವುದೆ? ನನ್ನ ಊರು ಕಾಶಿಯಂತೆ ನಿಮ್ಮ ಪಾದಸ್ಪರ್ಶದಿಂದ ಪುನೀತವಾಗುವುದು. ನಾನು ಅಷ್ಟೊಂದು ಅದೃಷ್ಟಶಾಲಿಯೇ?

ಸ್ವಾಮೀಜಿ: ನನಗೇನೋ ಆಸೆಯಿದೆ. ಇನ್ನು ನಾನಲ್ಲಿಗೆ ಬರುವುದು ‘ಮಾತೆ’ಯ ಇಚ್ಛೆ.

ನಾಗಮಹಾಶಯ: ಯಾರಿಗೆ ನಿಮ್ಮನ್ನು ತಿಳಿಯಲು ಸಾಧ್ಯ? ಅಂತರ್‌ದೃಷ್ಟಿ ತೆರೆದ ಹೊರತು ಯಾರಿಗೂ ನಿಮ್ಮನ್ನು ಅರ್ಥಮಾಡಿಕೊಳ್ಳಲು ಅಸಾಧ್ಯ! ಕೇವಲ ಶ‍್ರೀರಾಮಕೃಷ್ಣರು ಮಾತ್ರ ನಿಮ್ಮನ್ನು ಅರಿತಿದ್ದರು; ಉಳಿದವರೆಲ್ಲ ಅವರ ಮಾತುಗಳಲ್ಲಿ ಶ್ರದ್ಧೆ ಇಟ್ಟಿದ್ದರು. ಆದರೆ ಯಾರೂ ನಿಮ್ಮನ್ನು ಸರಿಯಾಗಿ ಅರ್ಥಮಾಡಿಕೊಂಡಿಲ್ಲ.

ಸ್ವಾಮೀಜಿ: ಈಗ ನನಗಿರುವ ಒಂದು ಆಸೆಯೆಂದರೆ, ಇಡೀ ದೇಶವನ್ನೆಲ್ಲ - ನಿದ್ರಿಸುತ್ತಿರುವ ಈ ರಾಕ್ಷಸಾಕಾರವನ್ನು - ತನ್ನ ಶಕ್ತಿಯಲ್ಲಿರುವ ನಂಬಿಕೆಯನ್ನೆಲ್ಲ ಕಳೆದುಕೊಂಡು, ಪ್ರತಿಕ್ರಿಯೆಯೇ ಇಲ್ಲದೆ ಬಿದ್ದಿರುವ ಇದನ್ನು ಎಬ್ಬಿಸಬೇಕೆಂಬುದು. ನಾನು ಅದನ್ನು ಶಾಶ್ವತ ಧರ್ಮದ ಭಾವನೆಗೆ ಎಬ್ಬಿಸಬಲ್ಲೆನಾದರೆ ನನ್ನ ಜನ್ಮ ಮತ್ತು ಶ‍್ರೀರಾಮಕೃಷ್ಣರ ಅವತಾರ ಸಾರ್ಥಕವೆಂದು ಭಾವಿಸುವೆ; ಇದೊಂದೇ ನನ್ನ ಹೃದಯದ ತೀವ್ರ ಆಕಾಂಕ್ಷೆ. ಮುಕ್ತಿ ಮುಂತಾದುವುಗಳಿಂದ ಯಾವ ಪ್ರಯೋಜನವೂ ನನಗೆ ಕಾಣುವುದಿಲ್ಲ. ದಯವಿಟ್ಟು ನಾನು ಇದರಲ್ಲಿ ಜಯಶಾಲಿಯಾಗುವಂತೆ ಹರಸಿ.

ನಾಗಮಹಾಶಯ: ಶ‍್ರೀರಾಮಕೃಷ್ಣರು ಆಶೀರ್ವದಿಸುವರು! ನಿಮ್ಮ ಇಚ್ಛೆಯನ್ನು ತಿರುಗಿಸಬಲ್ಲವರಾರು? ನೀವಿಚ್ಛೆಪಟ್ಟಿದ್ದೆಲ್ಲಾ ಕಾರ್ಯರೂಪಕ್ಕೆ ಬಂದೇ ಬರುವುದು.

ಸ್ವಾಮೀಜಿ: ಆತನ ಇಚ್ಛೆ ಇಲ್ಲದಿದ್ದಲ್ಲಿ ಯಾವುದೂ ಸಂಭವಿಸುವುದಿಲ್ಲ.

ನಾಗಮಹಾಶಯ: ಆತನ ಇಚ್ಛೆ ಮತ್ತು ನಿಮ್ಮ ಇಚ್ಛೆ ಎರಡೂ ಒಂದೇ ಆಗಿದೆ. ನಿಮ್ಮ ಇಚ್ಛೆ ಯಾವುದೋ ಅದೇ ಆತನ ಇಚ್ಛೆ. ಜಯ ಶ‍್ರೀರಾಮಕೃಷ್ಣ!

ಸ್ವಾಮೀಜಿ: ಕೆಲಸ ಮಾಡಲು ದೃಢಕಾಯವಾದ ದೇಹ ಆವಶ್ಯಕ. ಈ ದೇಶಕ್ಕೆ ಬಂದಾಗಿನಿಂದಲೂ ನನಗೆ ಆರೋಗ್ಯ ಸರಿಯಿಲ್ಲ. ಪಶ್ಚಿಮ ದೇಶದಲ್ಲಿ ನನ್ನ ಆರೋಗ್ಯ ಬಹಳ ಚೆನ್ನಾಗಿತ್ತು.

ನಾಗಮಹಾಶಯ: ಈ ಶರೀರಧಾರಣೆ ಮಾಡಿದವರೆಲ್ಲರೂ ಅದರ ಬಾಡಿಗೆಯ ಸುಂಕವನ್ನು ತೆತ್ತೇ ತೀರಬೇಕು. ರೋಗ ಮತ್ತು ದುಃಖವೇ ಅದರ ಸುಂಕ. ಆದರೆ ನಿಮ್ಮ ದೇಹ ಚಿನ್ನದ ಮೊಹರಿರುವ ಪೆಟ್ಟಿಗೆ, ಅದರ ವಿಷಯದಲ್ಲಿ ಬಹಳ ಜೋಪಾನವಾಗಿರಬೇಕು. ಆದರೆ ಹಾಗೆ ನೋಡಿಕೊಳ್ಳುವವರು ಯಾರು? ಯಾರಿಗೆ ಅರ್ಥಮಾಡಿಕೊಳ್ಳಲು ಸಾಧ್ಯ? ಶ‍್ರೀರಾಮಕೃಷ್ಣರು ಮಾತ್ರ ಅರ್ಥಮಾಡಿಕೊಳ್ಳುತ್ತಿದ್ದರು; ಜೈ ಶ‍್ರೀರಾಮಕೃಷ್ಣ!

ಸ್ವಾಮೀಜಿ: ಮಠದಲ್ಲಿರುವವರೆಲ್ಲ ನನ್ನ ವಿಷಯದಲ್ಲಿ ಬಹಳ ಎಚ್ಚರಿಕೆ ತೆಗೆದುಕೊಳ್ಳುತ್ತಾರೆ.

ನಾಗಮಹಾಶಯ: ಅವರು ಅದನ್ನು ಮಾಡಿದರೆ ಅವರು ತಿಳಿಯಲಿ ತಿಳಿಯದಿರಲಿ ಅದು ಅವರ ಒಳ್ಳೆಯದಕ್ಕೇ. ನಿಮ್ಮ ದೇಹಕ್ಕೆ ತಕ್ಕ ಗಮನ ಕೊಡದಿದ್ದಲ್ಲಿ ಅದು ಬಹುಬೇಗ ಬಿದ್ದು ಹೋಗಬಹುದು.

ಸ್ವಾಮೀಜಿ: ನಾಗಮಹಾಶಯರೆ!ನಾನೀಗ ಮಾಡುತ್ತಿರುವುದು ಸರಿಯೋ ತಪ್ಪೋ ಎಂದು ಸಂಪೂರ್ಣ ನನಗೆ ಗೊತ್ತಾಗುವುದಿಲ್ಲ. ಕೆಲವು ನಿರ್ದಿಷ್ಟ ಸಮಯಗಳಲ್ಲಿ ನನಗೆ ಯಾವುದೋ ಒಂದು ಮಾರ್ಗದಲ್ಲಿ ಹೋಗುವ ತೀವ್ರ ಇಚ್ಛೆಯುಂಟಾಗುತ್ತದೆ. ಅದರಂತೆಯೇ ನಾನು ಮಾಡುತ್ತೇನೆ. ಅದು ಒಳ್ಳೆಯದಕ್ಕೊ, ಕೆಟ್ಟದಕ್ಕೊ ನನಗೆ ಅರ್ಥವಾಗುವುದಿಲ್ಲ.

ನಾಗಮಹಾಶಯ: ಶ‍್ರೀರಾಮಕೃಷ್ಣರು ‘ನಿಧಿಗೆ ಈಗ ಬೀಗಮುದ್ರೆ ಹಾಕಲ್ಪಟ್ಟಿದೆ’ ಎಂದು ಹೇಳಿದ್ದರು. ಅದಕ್ಕೆ ಅವರು ನೀವು ಪೂರ್ತಿ ತಿಳಿದುಕೊಳ್ಳಲು ಬಿಡುವುದಿಲ್ಲ. ಯಾವ ಘಳಿಗೆ ನೀವು ಅದನ್ನು ತಿಳಿಯುವಿರೋ ಆಗ ನಿಮ್ಮ ಮಾನವ ಜನ್ಮದ ಲೀಲೆ ಕೊನೆಗೊಳ್ಳುವುದು.

ಸ್ವಾಮೀಜಿ ಏನನ್ನೊ ಎವೆಯಿಕ್ಕದೆ ಯೋಚಿಸುತ್ತಿದ್ದರು. ಸ್ವಾಮಿ ಪ್ರೇಮಾನಂದರು ಸ್ವಲ್ಪ ಪ್ರಸಾದವನ್ನು ಪ್ರೇಮೋನ್ಮಾದದಲ್ಲಿ ಮುಳುಗಿದ್ದ ನಾಗಮಹಾಶಯರಿಗೆ ನೀಡಿದರು. ಸ್ವಲ್ಪ ಕಾಲಾನಂತರ ನಾಗಮಹಾಶಯರು ಸ್ವಾಮಿಗಳು ಕೊಳದ ಹತ್ತಿರ ನೆಲವನ್ನು ಗುದ್ದಲಿಯಿಂದ ನಿಧಾನವಾಗಿ ಅಗೆಯುತ್ತಿದ್ದುದನ್ನು ನೋಡಿ ಅವರ ಕೈಹಿಡಿದು “ನಾವಿರುವಾಗ ನೀವೇಕೆ ಇದನ್ನೆಲ್ಲಾ ಮಾಡಬೇಕು" ಎಂದು ತಡೆದರು. ಸ್ವಾಮೀಜಿ ಗುದ್ದಲಿಯನ್ನು ಅಲ್ಲೇ ಬಿಟ್ಟು ತೋಟದಲ್ಲೇ ಸ್ವಲ್ಪ ಹೊತ್ತು ಸುತ್ತಾಡುತ್ತಾ ಶಿಷ್ಯನಿಗೆ ಇದನ್ನು ಹೇಳಿದರು: “ಶ‍್ರೀರಾಮಕೃಷ್ಣರು ಕಣ್ಮರೆಯಾದ ಮೇಲೆ ನಾಗಮಹಾಶಯರು ಕಲ್ಕತ್ತೆಯ ತಮ್ಮ ಬಡ ಕುಟೀರದಲ್ಲಿ ಉಪವಾಸ ಮಲಗಿರುವರೆಂದು ಒಂದು ದಿನ ಕೇಳಿದೆವು. ನಾನು ಸ್ವಾಮಿ ತುರೀಯಾನಂದ ಇತರರೂ ಎಲ್ಲರೂ ಒಟ್ಟಿಗೆ ಅವರ ಕುಟೀರಕ್ಕೆ ಹೋದೆವು. ನಮ್ಮನ್ನು ನೋಡಿದೊಡನೆಯೇ ಅವರು ಹಾಸಿಗೆಯಿಂದ ಎದ್ದರು. ನಾವಿಂದು ನಮ್ಮ ಭಿಕ್ಷೆಯನ್ನು (ಆಹಾರ) ಇಲ್ಲೇ ತೆಗೆದುಕೊಳ್ಳುತ್ತೇವೆ ಎಂದೆವು. ತಕ್ಷಣ ನಾಗಮಹಾಶಯ ಅಕ್ಕಿ, ತಪ್ಪಲೆ, ಸೌದೆ ಮುಂತಾದುದನ್ನೆಲ್ಲಾ ಅಂಗಡಿಯಿಂದ ಕೊಂಡುತಂದು ಅಡಿಗೆ ಮಾಡಲಾರಂಭಿಸಿದರು. ನಾವು ಯೋಚಿಸಿದ್ದು ಏನೆಂದರೆ ನಾವು ಊಟ ಮಾಡೋಣ, ನಾಗಮಹಾಶಯರಿಗೂ ಮಾಡಿಸೋಣ ಎಂದು. ಆದರೆ ಅಡಿಗೆ ಮಾಡಿ ಅವರು ಅನ್ನವನ್ನು ನಮಗೇ ಬಡಿಸಿದರು. ನಾವು ಸ್ವಲ್ಪ ಭಾಗವನ್ನು ಬೇರೆ ತೆಗೆದಿರಿಸಿ ಊಟ ಮಾಡಲು ಕುಳಿತೆವು. ನಂತರ ಅವರನ್ನೂ ಊಟ ಮಾಡುವಂತೆ ಕೇಳಿದೆವು. ಅವರು ತಕ್ಷಣ ಆ ಅನ್ನದ ಮಡಕೆಯನ್ನು ಒಡೆದು ತಲೆಯನ್ನು ಚಚ್ಚಿಕೊಳ್ಳುತ್ತಾ "ದೈವಸಾಕ್ಷಾತ್ಕಾರವಾಗದ ಶರೀರಕ್ಕೆ ಆಹಾರವನ್ನು ಕೊಡಲೆ?" ಇದನ್ನು ನೋಡಿ ನಾವು ಸ್ತಂಭೀಭೂತರಾದೆವು. ಸ್ವಲ್ಪ ಕಾಲದ ನಂತರ ತುಂಬಾ ಬಲವಂತ ಮಾಡಿ ಅವರಿಗೆ ಸ್ವಲ್ಪ ಊಟ ಮಾಡಿಸಿ ಹಿಂತಿರುಗಿದೆವು.

ಸ್ವಾಮೀಜಿ: ನಾಗಮಹಾಶಯರು ಈ ರಾತ್ರಿ ಮಠದಲ್ಲೇ ಉಳಿಯುವರೇ.

ಶಿಷ್ಯ: ಇಲ್ಲ ಅವರಿಗೆ ಸ್ವಲ್ಪ ಕೆಲಸವಿದೆ. ಅವರು ಇಂದೇ ಹಿಂತಿರುಗಬೇಕು.

ಸ್ವಾಮಿಜಿ: ಹಾಗಾದರೆ ದೋಣಿಯನ್ನು ನೋಡು. ಆಗಲೇ ಕತ್ತಲಾಗುತ್ತಲಿದೆ.

ದೋಣಿಯು ಬಂದೊಡನೆ ಶಿಷ್ಯನೂ ನಾಗಮಹಾಶಯರೂ ಸ್ವಾಮಿಗಳಿಗೆ ನಮಸ್ಕಾರ ಮಾಡಿ ಕಲ್ಕತ್ತೆಗೆ ಹೊರಟರು.

\newpage

\chapter[ಅಧ್ಯಾಯ ೩೨]{ಅಧ್ಯಾಯ ೩೨\protect\footnote{\engfoot{C.W, Vol. VII, P190}}}

\begin{center}
ಸ್ಥಳ: ಬೇಲೂರು ಮಠ, ವರ್ಷ: ಕ್ರಿ.ಶ. ೧೮೯೯.
\end{center}

ಸ್ವಾಮೀಜಿಯವರ ಆರೋಗ್ಯ ಈಗ ಚೆನ್ನಾಗಿದೆ. ಶಿಷ್ಯ ಮಠಕ್ಕೆ ಒಂದು ಶನಿವಾರ ಬೆಳಿಗ್ಗೆ ಬಂದಿದ್ದಾನೆ. ಸ್ವಾಮೀಜಿಯವರನ್ನು ಸಂದರ್ಶಿಸಿ ಕೆಳಗಿಳಿದು ಬಂದು ಸ್ವಾಮಿ ನಿರ್ಮಲಾನಂದರೊಡನೆ ವೇದಾಂತಶಾಸ್ತ್ರ ವಿಚಾರ ಚರ್ಚೆ ನಡೆಸುತ್ತಿದ್ದಾನೆ. ಆ ಸಮಯದಲ್ಲಿ ಸ್ವಾಮೀಜಿ ಕೆಳಗಿಳಿದು ಬಂದು ಶಿಷ್ಯನಿಗೆ “ನೀನು ನಿರ್ಮಲಾನಂದರೊಡನೆ ಏನು ಚರ್ಚಿಸುತ್ತಿದ್ದೆ?”

ಶಿಷ್ಯ: ಸ್ವಾಮೀಜಿ, ಅವರು ಹೇಳುತ್ತಿದ್ದರು; ವೇದಾಂತದಲ್ಲಿರುವ ಬ್ರಹ್ಮನು ನಿನಗೆ ಮತ್ತು ನಿಮ್ಮ ಸ್ವಾಮೀಜಿಗೆ ಮಾತ್ರ ಗೊತ್ತು. ನಾವು ಮಾತ್ರ ಶ‍್ರೀಕೃಷ್ಣನನ್ನು ಅವತಾರಪುರುಷನೆಂದು ತಿಳಿದಿದ್ದೇವೆ.

ಸ್ವಾಮೀಜಿ: ನೀನೇನು ಹೇಳಿದೆ?

ಶಿಷ್ಯ: ನಾನು ಆತ್ಮನೊಬ್ಬನೇ ಸತ್ಯ ಎಂದೆ. ಶ‍್ರೀಕೃಷ್ಣನು ಆತ್ಮಸಾಕ್ಷಾತ್ಕಾರ ಹೊಂದಿದ ಒಬ್ಬ ಮನುಷ್ಯ ಅಷ್ಟೆ. ಸ್ವಾಮಿ ನಿರ್ಮಲಾನಂದರಿಗೆ ಆಂತರ್ಯದಲ್ಲಿ ನಂಬಿಕೆ ಇದೆ - ಆದರೆ ಹೊರಗಡೆ ಮಾತ್ರ ಅವರು ದ್ವೈತ ಭಾವದ ಚರ್ಚೆಯಲ್ಲಿ ಭಾಗವಹಿಸುವರು. ಮೊದಲು ಅವರು ಈಶ್ವರನ ಸಾಕಾರ ಭಾವನೆಯನ್ನು ಚರ್ಚಿಸುತ್ತಿದ್ದಂತಿತ್ತು. ನಂತರ ನಿಧಾನವಾಗಿ ತರ್ಕದ ಹಾದಿಯಿಂದ ವೇದಾಂತದ ತಳಹದಿಯನ್ನು ಬಲಪಡಿಸಲು ಯತ್ನಿಸಿದಂತಿತ್ತು. ಆದರೆ ಯಾವಾಗ ಅವರು ನನ್ನನ್ನು ವೈಷ್ಣವನೆಂದು ಕರೆದರೋ ಆಗ ನನಗೆ ಅವರ ನಿಜವಾದ ಉದ್ದೇಶವೇ ಮರೆತುಹೋಗಿ ಅವರೊಡನೆ ಬಿಸಿ ಬಿಸಿ ಚರ್ಚೆ ಪ್ರಾರಂಭಿಸಿದೆನು.

ಸ್ವಾಮೀಜಿ: ಅವರು ನಿನ್ನನ್ನು ಪ್ರೀತಿಸುತ್ತಾರೆ. ಅದಕ್ಕೆ ನಿನ್ನನ್ನು ಪೀಡಿಸುವುದರ ಮೂಲಕ ಸಂತೋಷಿಸುತ್ತಾರೆ. ಆದರೆ ನೀನೇಕೆ ಅವರ ಮಾತುಗಳಿಂದ ಅಷ್ಟೊಂದು ಕ್ಷೋಭೆಗೊಳ್ಳಬೇಕು? ನೀನೂ ಉತ್ತರ ಕೊಡು ‘ನೀನೊಬ್ಬ ನಾಸ್ತಿಕ, ಶೂನ್ಯ ವಾದವನ್ನು ನಂಬುವವನು ಎಂದು.’

ಶಿಷ್ಯ: ಸ್ವಾಮಿಜಿ, ಉಪನಿಷತ್ತಿನಲ್ಲಿ ಈಶ್ವರನು ಸರ್ವಶಕ್ತನಾದ ವ್ಯಕ್ತಿ ಎಂದು ಎಲ್ಲಾದರೂ ಹೇಳಿದೆಯೇ? ಆದರೆ ಜನರು ಸಾಧಾರಣವಾಗಿ ಅಂತಹ ಈಶ್ವರನನ್ನು ನಂಬುವರು.

ಸ್ವಾಮೀಜಿ: ವಿಶ್ವೇಶ್ವರನೆಂಬ ಪರಮ ಸಿದ್ಧಾಂತ ಕೇವಲ ಒಬ್ಬ ವ್ಯಕ್ತಿಯಾಗಲಾರದು. ಜೀವನು ಒಬ್ಬ ವ್ಯಕ್ತಿ. ಈ ಜೀವಿಗಳ ಮೊತ್ತ ಈಶ್ವರ. ಜೀವನಲ್ಲಿ ಅವಿದ್ಯೆ ಪ್ರಧಾನವಾಗಿರುವುದು. ಈಶ್ವರನು ವಿದ್ಯೆ ಅವಿದ್ಯೆಗಳಿಂದ ಕೂಡಿದ ಮಾಯೆಯನ್ನು ತನ್ನ ಹಿಡಿತದಲ್ಲಿಟ್ಟುಕೊಂಡಿದ್ದಾನೆ. ಸ್ವತಂತ್ರವಾಗಿ ತನ್ನಿಂದಲೇ ಚರಾಚರ ವಸ್ತುಗಳುಳ್ಳ ಈ ಪ್ರಪಂಚವನ್ನು ಸೃಷ್ಟಿಸಿದ್ದಾನೆ. ಆದರೆ ಬ್ರಹ್ಮನು ವ್ಯಕ್ತಿ ಸಮಷ್ಟಿಯನ್ನು, ಈಶ್ವರನನ್ನು ಮೀರಿದ್ದಾನೆ. ಬ್ರಹ್ಮದಲ್ಲಿ ವಿಭಾಗವಿಲ್ಲ. ಸುಲಭವಾಗಿ ಗ್ರಹಿಸಲು ಸಾಧ್ಯವಾಗುವಂತೆ ಈ ವಿಭಾಗಗಳನ್ನು ಕಲ್ಪಿಸಿಕೊಂಡಿದ್ದೇವೆ. ವಿಶ್ವದ ಸೃಷ್ಟಿ, ಸ್ಥಿತಿ, ಲಯಕರ್ತನೆಂದು ಹೇಳುವ ಆ ಬ್ರಹ್ಮನ ಈ ಅಂಶವನ್ನು ಈಶ್ವರನೆಂದು ಶಾಸ್ತ್ರ ಕರೆಯುವುದು. ವಿಭಿನ್ನವಾದ, ದ್ವೈತಾತೀತವಾದ ಇನ್ನೊಂದು ಅಂಶವನ್ನು ಬ್ರಹ್ಮನೆಂದು ಹೇಳುವರು. ಆದರೆ ಇದರಿಂದಾಗಿ ಬ್ರಹ್ಮವು ಈ ಜೀವ ಮತ್ತು ಜಗತ್ತುಗಳಿಂದ ಬೇರೆಯಾದ ಪ್ರತ್ಯೇಕವಾದ ವಸ್ತುವೆಂದೆಣಿಸಬೇಡ. ವಿಶಿಷ್ಟಾದ್ವೈತಿಗಳು ಬ್ರಹ್ಮನೇ ಜೀವ ಮತ್ತು ಜಗತ್ತಾಗಿ ಮಾರ್ಪಟ್ಟಿರುವನೆಂದು ಅಭಿಪ್ರಾಯಪಡುವರು. ಅದಕ್ಕೆ ವಿರುದ್ಧ ಅದ್ವೈತಿಗಳು ಬ್ರಹ್ಮನಲ್ಲಿ ಜೀವ, ವಿಶ್ವಗಳೆರಡೂ ಆರೋಪವಾಗಿವೆ ಎಂದು ಅಭಿಪ್ರಾಯಪಡುವರು. ಆದರೆ ವಾಸ್ತವಿಕವಾಗಿ ಬ್ರಹ್ಮನಲ್ಲಿ ಯಾವ ಬದಲಾವಣೆಯೂ ಇಲ್ಲ. ಅದ್ವೈತಿಗಳ ಪ್ರಕಾರ ಈ ಭೂಮಂಡಲ ಕೇವಲ ನಾಮರೂಪಗಳನ್ನೊಳಗೊಂಡಿದೆ. ಎಲ್ಲಿಯವರೆಗೆ ನಾಮರೂಪಗಳಿವೆಯೋ ಅಲ್ಲಿಯವರೆಗೆ ಅದು ಇರುವುದು. ಯಾವಾಗ ಧ್ಯಾನ ಸಾಧನೆಗಳಿಂದ ನಾಮರೂಪಗಳು ವಿಲೀನವಾಗುವುವೋ ಆಗ ಜ್ಞಾನಗಮ್ಯನಾದ ಬ್ರಹ್ಮನು ಮಾತ್ರ ಉಳಿಯುವನು. ಆಗ ಈ ಜೀವ, ವಿಶ್ವಗಳ ಪ್ರತ್ಯೇಕ ಸತ್ಯದ ಅರಿವೇ ಇರುವುದಿಲ್ಲ. ಆಗ ನಾನೇ ನಿತ್ಯ ಸಚ್ಚಿದಾನಂದ ಪರಬ್ರಹ್ಮನೆಂದು ಸತ್ಯಾನುಭವ ಉಂಟಾಗುವುದು. ಜೀವಿಯ ವಾಸ್ತವಿಕ ಸ್ವಭಾವವೇ ಬ್ರಹ್ಮ. ಧ್ಯಾನ ಮುಂತಾದುವುಗಳ ಮೂಲಕ ನಾಮರೂಪಗಳ ತೆರೆ ಕಳಚಿದ ಕೂಡಲೇ ಈ ಭಾವನೆಯ ಸಾಕ್ಷಾತ್ಕಾರವಾಗುತ್ತದೆ. ಇದೇ ಅದ್ವೈತದ ಸಾರ. ವೇದ ವೇದಾಂತ ಮತ್ತಿತರ ಶಾಸ್ತ್ರಗಳೆಲ್ಲಾ ಈ ವಿಚಾರಗಳನ್ನೇ ಹಲವು ವಿಧದಲ್ಲಿ ವಿವರಿಸುವುವು.

ಶಿಷ್ಯ: ಈಶ್ವರನೇ ಸರ್ವಶಕ್ತಿ ನಿಯಾಮಕನೆಂದು ಹೇಗೆ ಹೇಳುವುದು?

ಸ್ವಾಮೀಜಿ: ಮಾನಸಿಕ ಎಲ್ಲೆಯಿಂದ ಕೂಡಿದ ಮಾನವ ಎಷ್ಟಾದರೂ ಮಾನವನೇ. ಮನಸ್ಸಿನ ಮೂಲಕವೇ ಅವನು ಎಲ್ಲವನ್ನೂ ಅರಿಯಬೇಕು, ಗ್ರಹಿಸಬೇಕು. ಆದ್ದರಿಂದ ಅವನು ಯೋಚಿಸುವುದೆಲ್ಲಾ ಮನಸ್ಸಿನ ಅಂಕೆಗೊಳಪಟ್ಟಿದೆ. ಆದ್ದರಿಂದ ಮನುಷ್ಯನ ಸಹಜಸ್ವಭಾವವೇ ಈಶ್ವರನ ವ್ಯಕ್ತಿತ್ವವನ್ನು ತನ್ನ ವ್ಯಕ್ತಿತ್ವದ ಸಾಮ್ಯದಿಂದ ಚರ್ಚಿಸುವುದಾಗಿದೆ. ಮಾನವನು ತನ್ನ ಇಷ್ಟ ದೈವವನ್ನು ಮಾನವ ರೂಪದಲ್ಲಿಯೇ ಯೋಚಿಸಲು ಸಾಧ್ಯ. ವ್ಯಾಧಿ ಮೃತ್ಯುವಿನಿಂದ ಆವೃತವಾದ ಈ ಜಗತ್ತಿನಲ್ಲಿ ಶೋಕಪ್ರಹಾರಗಳು ಬಿದ್ದಾಗ, ನಿರಾಶೆ ನಿಸ್ಸಹಾಯಕತೆಯಿಂದ ಕುಗ್ಗಿ ಯಾವುದನ್ನು ಆಶ್ರಯಿಸುವುದರಿಂದ ನೆಮ್ಮದಿ ಸಿಕ್ಕುವುದೋ ಅಂತಹದನ್ನು ಅರಸಿಕೊಂಡು ಹೊರಡುವನು. ಆದರೆ ಅಂತಹ ಆಶ್ರಯವೆಲ್ಲಿ ಸಿಗುವುದು? ಅನ್ಯ ಆಶ್ರಯವಿಲ್ಲದೆ ಇರುವ ಸರ್ವವ್ಯಾಪಿ ಭಗವಂತನೊಬ್ಬನೇ ನಿಜವಾದ ಆಶ್ರಯದಾತ. ಮೊದಮೊದಲು ಮಾನವನಿಗೆ ಇದು ಗೊತ್ತಾಗುವುದಿಲ್ಲ. ಯಾವಾಗ ಧ್ಯಾನ ಸಾಧನೆಗಳ ಪರಿಣಾಮವಾಗಿ ವಿಚಾರ ವೈರಾಗ್ಯಗಳು ಉದಿಸುವುವೋ ಆಗ ಅವರಿಗೆ ಇದರ ಅರಿವುಂಟಾಗುವುದು. ಆಧ್ಯಾತ್ಮಿಕ ಮಾರ್ಗದಲ್ಲಿ ಅವನು ಯಾವ ಪಥದಲ್ಲೇ ಮುಂದುವರಿಯಲಿ ಅವನ ಅರಿವಿಲ್ಲದೆ ಆತನಲ್ಲಿರುವ ಬ್ರಹ್ಮನು ಜಾಗೃತಿಗೊಳ್ಳುವನು. ಇದು ಒಬ್ಬೊಬ್ಬರ ವಿಷಯದಲ್ಲಿ ಒಂದೊಂದು ಬಗೆಯಾಗಿರಬಹುದು. ಯಾರಿಗೆ ಸಾಕಾರ ಬ್ರಹ್ಮನಲ್ಲಿ ಶ್ರದ್ಧೆ ಇದೆಯೋ, ಆ ಭಾವನೆಯನ್ನೇ ಗುರಿಯಾಗಿಟ್ಟುಕೊಂಡು ಅವರು ಸಾಧನೆಯಲ್ಲಿ ನಿರತರಾಗಬೇಕು. ಎಲ್ಲಿ ನಿಜವಾದ ಶ್ರದ್ಧೆ ಇದೆಯೋ ಅಂತಹ ಹೃದಯದಲ್ಲಿ ಬ್ರಹ್ಮನು ಜಾಗೃತನಾಗುವನು. ಬ್ರಹ್ಮಜ್ಞಾನವೇ ಎಲ್ಲಾ ಜೀವಿಗಳ ಏಕಮಾತ್ರ ಗುರಿ. ಆದರೆ ಭಾವಕ್ಕೆ ತಕ್ಕ ಪಥಗಳಿವೆ. ಜೀವಿಯ ನೈಜ ಸ್ವಭಾವವೇ ಬ್ರಹ್ಮನಾಗಿದ್ದರೂ ಮಾನಸಿಕ ಎಲ್ಲೆಯೊಳಗೆ ಅವನು ಸ್ಥಾಪಿತವಾಗಿರುವುದರಿಂದ ಅನೇಕ ಬಗೆಯ ಸಂಶಯ ಕ್ಲೇಶಗಳಿಂದ, ಆನಂದ ನೋವುಗಳಿಂದ ವ್ಯಥಿತನಾಗುವನು. ಆದರೆ ಬ್ರಹ್ಮ ನಿಂದ ಹಿಡಿದು ಹುಲ್ಲಿನೆಸಳಿನವರೆಗೆ ಎಲ್ಲವೂ ಪರಮಾತ್ಮನ ಸತ್ಯ ಸಾಕ್ಷಾತ್ಕಾರದ ಕಡೆ ಮುಂದುವರಿಯುತ್ತಿವೆ. ಬ್ರಹ್ಮನಲ್ಲಿ ಐಕ್ಯಗೊಳ್ಳುವವರೆಗೂ ಜನನ ಮರಣಗಳ ಚಕ್ರದಿಂದ ಪಾರಾಗಲಾರ. ಮಾನವ ಜನ್ಮಧಾರಣೆಮಾಡಿ ಮುಕ್ತಿಹೊಂದುವ ಆಸೆಯ ಜೊತೆಗೆ ಬ್ರಹ್ಮಸಾಕ್ಷಾತ್ಕಾರ ಪಡೆದ ಮನುಷ್ಯನ ಕೃಪೆಯೂ ದೊರೆತರೆ ಮನುಷ್ಯನ ಆತ್ಮಜ್ಞಾನ ಪಡೆಯುವ ಆಕಾಂಕ್ಷೆ ತೀವ್ರವಾಗುವುದು. ಇಲ್ಲದಿದ್ದಲ್ಲಿ ಭೋಗ ಐಶ್ವರ್ಯಗಳಲ್ಲಿ ಮುಳುಗಿರುವ ಮಾನವನ ಮನಸ್ಸು ಎಂದಿಗೂ ಆ ಕಡೆ ವಾಲುವುದಿಲ್ಲ. ಯಾವಾತನ ಮನಸ್ಸು ಇನ್ನೂ ಸಂಸಾರ ಜೀವನದ ಆಸೆ, ಐಶ್ವರ್ಯ, ಕೀರ್ತಿಲಾಲಸೆಗೆ ಆತುರಪಡುತ್ತಿದೆಯೋ ಅಂಥವನ ಮನಸ್ಸಿನಲ್ಲಿ ಬ್ರಹ್ಮಜ್ಞಾನ ಪಡೆಯುವ ಬಯಕೆಯಾದರೂ ಉದಿಸುವುದೆ? ಯಾರು ಎಲ್ಲವನ್ನೂ ತ್ಯಾಗಮಾಡಲು ಸಿದ್ಧನಾಗಿರುವನೋ, ಒಳ್ಳೆಯದು ಕೆಟ್ಟದ್ದು ಸುಖದುಃಖಗಳೇ ಮುಂತಾದ ದ್ವೈತ ಪ್ರವಾಹದ ಮಧ್ಯದಲ್ಲಿಯೂ ಶಾಂತವಾಗಿ ಸ್ಥಿರವಾಗಿ ಸಮಭಾವದಲ್ಲಿ ಗುರಿಯೆಡೆಗೇ ದಿಟ್ಟಿಯಿಟ್ಟಿರುವನೋ, ಅಂಥವನು ಮಾತ್ರ ಆತ್ಮಜ್ಞಾನವನ್ನು ಹೊಂದಲು ಪ್ರಯತ್ನಿಸುವನು. ಅಂತಹವನು ತನ್ನ ಸ್ವಂತ ಶಕ್ತಿಯಿಂದ ಪ್ರಪಂಚದ ತೆರೆಯನ್ನು ಕಿತ್ತೊಗೆದು ಮಾಯಾ ಶೃಂಖಲೆಗಳನ್ನು ಕಡಿದು ಹಾಕಿ ಅಸದೃಶ ಸಿಂಹದಂತೆ ಹೊರಹೊಮ್ಮುವನು. ‘ನಿರ್ಗಚ್ಛತಿ ಜಗಜ್ಜಾಲಾತ್ ಪಿಂಜರಾದಿವ ಕೇಸರೀ’

ಶಿಷ್ಯ: ಹಾಗಾದರೆ ಸಂನ್ಯಾಸವಿಲ್ಲದೆ ಬ್ರಹ್ಮಜ್ಞಾನವನ್ನು ಪಡೆಯಲಾಗುವುದೇ ಇಲ್ಲವೆ?

ಸ್ವಾಮೀಜಿ: ಖಂಡಿತವಾಗಿ ಸಾವಿರಬಾರಿಗೂ ಸತ್ಯ. ಮಾನಸಿಕ ಮತ್ತು ಬಾಹ್ಯ ಸಂನ್ಯಾಸ ಹೊಂದಲೇಬೇಕು. ಸಾಂಪ್ರದಾಯಿಕ ಹಾಗೂ ಆಂತರಿಕ ತ್ಯಾಗ ಅತ್ಯವಶ್ಯಕ. ಶಂಕರಾಚಾರ್ಯರು ‘ತಪಸೋ ವಾಪ್ಯಲಿಂಗಾತ್’ - ‘ಬಾಹ್ಯ ಸಂನ್ಯಾಸ ಚಿಹ್ನೆಯಿಲ್ಲದೆ ತಪಸ್ಸೂ ನಿರರ್ಥಕ’ - ಎಂಬ ವಾಕ್ಯದ ಮೇಲೆ ಭಾಷ್ಯ ಬರೆಯುತ್ತಿದ್ದಾಗ ಹೇಳಿದ್ದಾರೆ: 'ಸಂನ್ಯಾಸದ ಬಾಹ್ಯ ಚಿಹ್ನೆ (ಕಾವಿಬಟ್ಟೆ ದಂಡ ಕಮಂಡಲು ಮುಂತಾದುವು) ಇಲ್ಲದೆ ಸಾಧನೆ ಮಾಡಿದರೆ ಬಹು ಕಷ್ಟಸಾಧ್ಯವಾದ ಬ್ರಹ್ಮ ಸಾಕ್ಷಾತ್ಕಾರವಾಗುವುದಿಲ್ಲ. ಪ್ರಪಂಚದಲ್ಲಿ ವಿರಕ್ತಿಯಿಲ್ಲದೆ, ತ್ಯಾಗಬುದ್ಧಿಯಿಲ್ಲದೆ ಭೋಗೇಚ್ಛೆಯನ್ನು ತ್ಯಜಿಸದೆ ಆಧ್ಯಾತ್ಮಿಕ ಜೀವನದಲ್ಲಿ ಏನನ್ನೂ ಸಾಧಿಸಲಾಗುವುದಿಲ್ಲ. ಅದು ಚಿಕ್ಕ ಮಗುವಿನ ಕೈಯಲ್ಲಿರುವ ಮಿಠಾಯಿಯನ್ನು ಉಪಾಯದಿಂದ ಅಪಹರಿಸುವಷ್ಟು ಸುಲಭವಲ್ಲ.

ಶಿಷ್ಯ: ಆಧ್ಯಾತ್ಮಿಕ ಸಾಧನೆಯಲ್ಲಿ ನಿರತರಾಗಿದ್ದಾಗ ತ್ಯಾಗ ತಾನಾಗಿಯೇ ಬರಬಹುದು.

ಸ್ವಾಮೀಜಿ: ಯಾರಿಗೆ ಈ ರೀತಿ ಕ್ರಮೇಣ ಬರಲೆಂದು ಇಚ್ಛೆ ಇದೆಯೋ ಅವರು ಹಾಗೆ ಹೊಂದಲಿ, ಆದರೆ ನೀನೇಕೆ ಅದಕ್ಕೆ ಕಾದು ಕುಳಿತಿರಬೇಕು? ಈ ಕ್ಷಣವೇ ಕಾಲುವೆಯನ್ನು ತೋಡಿ ನಿನ್ನ ಜೀವನಕ್ಕೆ ಆಧ್ಯಾತ್ಮಿಕ ಪ್ರವಾಹ ಹರಿಯುವಂತೆ ಮಾಡು. ಶ‍್ರೀರಾಮಕೃಷ್ಣರು ಆಧ್ಯಾತ್ಮಿಕ ಸಾಧನೆಯಲ್ಲಿ ಈ ನಿರುತ್ಸಾಹವನ್ನು ಖಂಡಿಸುತ್ತಿದ್ದರು. ಉದಾಹರಣೆಗೆ ಧರ್ಮ ನಿಧಾನವಾಗಿ ಬರುತ್ತದೆ, ಅದಕ್ಕೆ ಅವಸರವೇನೂ ಪಡಬೇಕಾಗಿಲ್ಲ ಎಂಬುದು. ತೃಷಾಪೀಡಿತನಾಗಿದ್ದಾಗ ಸುಮ್ಮನೇ ಸೋಮಾರಿಯಂತೆ ಕುಳಿತುಕೊಳ್ಳಲು ಸಾಧ್ಯವೆ? ನೀರು ಸಿಗುವ ಸ್ಥಳಕ್ಕೆ ಓಡಿಹೋಗುವುದಿಲ್ಲವೆ? ನಿನಗೆ ಇನ್ನೂ ಆಧ್ಯಾತ್ಮಿಕ ತೃಷೆಯಿಲ್ಲದ ಕಾರಣ ನೀನು ಸೋಮಾರಿಯಂತೆ ಕುಳಿತಿರುವೆ. ಜ್ಞಾನತೃಷೆ ನಿನ್ನಲ್ಲಿನ್ನೂ ಬಲವಾಗಿಲ್ಲ. ಅದಕ್ಕೇ ನಿನ್ನ ಸಾಂಸಾರಿಕ ಜೀವನದ ಕ್ಷುದ್ರ ಬಯಕೆಗಳಲ್ಲೇ ನೀನು ತೃಪ್ತನಾಗಿರುವಿ.

ಶಿಷ್ಯ: ನಿಜವಾಗಿ ನನಗೆ ಸರ್ವವನ್ನೂ ತ್ಯಾಗಮಾಡುವ ಬುದ್ಧಿ ಏಕೆ ಹುಟ್ಟುತ್ತಿಲ್ಲವೋ ಅರಿಯೆ. ದಯವಿಟ್ಟು ನನಗೊಂದು ದಾರಿ ತೋರಿಸಿಕೊಡಿ.

ಸ್ವಾಮೀಜಿ: ನಿನ್ನ ಕೈಯಲ್ಲೇ ಗುರಿ ಮತ್ತು ಸಾಧನೆ ಎರಡೂ ಇದೆ. ನಾನು ಕೇವಲ ಅದನ್ನು ಉದ್ರೇಕಿಸಬಲ್ಲೆ. ನೀನಿಷ್ಟೊಂದು ಶಾಸ್ತ್ರಪಾರಂಗತನಾಗಿರುವೆ - ಬ್ರಹ್ಮಜ್ಞಾನಿಗಳೊಡನೆ ಸಂಪರ್ಕ ಹೊಂದಿರುವೆ - ಇದೊಂದೂ ನಿನ್ನಲ್ಲಿ ತ್ಯಾಗಬುದ್ಧಿಯನ್ನು ಹುಟ್ಟಿಸದೆ ಹೋದಲ್ಲಿ ನಿನ್ನ ಬಾಳು ವ್ಯರ್ಥ. ಆದರೆ ಅದು ಸಂಪೂರ್ಣ ನಿರರ್ಥಕವಲ್ಲ - ಇದರ ಪರಿಣಾಮ ಯೋಗ್ಯಕಾಲದಲ್ಲಿ ಯಾವುದಾದರೊಂದು ಬಗೆಯಿಂದ ವಿಕಾಸಗೊಳ್ಳುವುದು.

ಶಿಷ್ಯ ಬಹು ಖಿನ್ನನಾಗಿ ಪುನಃ ಸ್ವಾಮಿಗಳನ್ನು ಕುರಿತು, ಸ್ವಾಮಾಜಿ, ನಾನು ತಮ್ಮ ಅಡಿದಾವರೆಗೆ ಬಂದಿರುವೆ. ಮುಕ್ತಿಯ ಹಾದಿಯನ್ನು ನನಗೆ ತೆರೆಯಿರಿ, ನಾನೀ ದೇಹದಲ್ಲೇ ಸತ್ಯಸಾಕ್ಷಾತ್ಕಾರ ಹೊಂದುವಂತೆ ಆಶೀರ್ವದಿಸಿ ಎಂದು ಕೇಳಿಕೊಂಡನು.

ಸ್ವಾಮೀಜಿ: ನಿನಗಾವ ಭಯವಿದೆ? ಯಾವಾಗಲೂ ವಿಚಾರಮಾಡು - ನಿನ್ನ ದೇಹ, ಮನೆ, ಜೀವಿಗಳು, ಪ್ರಪಂಚ ಎಲ್ಲ ಸಂಪೂರ್ಣವಾಗಿ ಸ್ವಪ್ನದಂತೆ ಮಿಥ್ಯ. ಯಾವಾಗಲೂ ಈ ಶರೀರ ಒಂದು ಜಡಯಂತ್ರವೆಂದು ಭಾವಿಸು. ನಿನ್ನ ಆಂತರದಲ್ಲಿರುವ ನಿತ್ಯಪೂರ್ಣ ಪುರುಷನೇ ನಿನ್ನ ವಾಸ್ತವಿಕ ಸ್ವಭಾವ. ಮನಸ್ಸೇ ಅವನ ಮೊದಲ ಸೂಕ್ಷ್ಮಹೊದಿಕೆ. ನಂತರ, ಸ್ಥೂಲವಾದ ಈ ಶರೀರ ಬಾಹ್ಯ ಹೊದಿಕೆ, ಅಖಂಡ ಅನಂತ ಸ್ವಯಂಪ್ರಕಾಶಮಾನ ಪುರುಷನು ಈ ಮಾಯೆಯ ಬಲೆಯಲ್ಲಿ ಹುದುಗಿದ್ದಾನೆ. ಅದಕ್ಕೇ ನಿನ್ನ ನಿಜಸ್ವಭಾವವನ್ನು ನೀನು ಅರಿಯದಿರುವೆ. ಯಾವಾಗಲೂ ವಿಷಯ ಸುಖದೆಡೆಗೆ ಓಡುತ್ತಿರುವ ಮನಸ್ಸು ಅಂತರ್ಮುಖವಾಗಬೇಕು. ಮನಸ್ಸನ್ನು ಕೊಲ್ಲಬೇಕು. ಶರೀರ ಕೇವಲ ಸ್ಥೂಲ. ಅದು ಸತ್ತು ಪಂಚಭೂತಗಳಲ್ಲಿ ಲಯವಾಗುವುದು. ಆದರೆ ಮನಸ್ಸಿನಲ್ಲಿ ಮುದ್ರಿತವಾದ ಆ ಮಾನಸಿಕ ಅನುಭವಗಳ ಕಂತೆ ಅಷ್ಟು ಬೇಗ ನಾಶವಾಗುವುದಿಲ್ಲ. ಅದು ಬೀಜರೂಪದಲ್ಲಿ ಕೆಲವು ಕಾಲ ಸ್ತಬ್ಧವಾಗಿರುವುದು. ನಂತರ ಚಿಗುರಿ ಕವಲೊಡೆದು ಮರವಾಗುವುದು. ಬೇರೊಂದು ಭೌತಿಕ ಶರೀರಧಾರಣೆಮಾಡಿ ಆತ್ಮಪರಿಜ್ಞಾನ ಬರುವವರೆಗೂ ಪುನಃ ಪುನಃ ಜನನ ಮರಣಗಳ ಚಕ್ರಕ್ಕೆ ಬೀಳುವುದು. ಆದ್ದರಿಂದ ಧ್ಯಾನ ಚಿತ್ತೈಕಾಗ್ರತೆಗಳಿಂದ, ನಿತ್ಯಾನಿತ್ಯವಸ್ತು ಪರಿಜ್ಞಾನದಿಂದ ಈ ಮನಸ್ಸನ್ನು ಅಖಂಡ ಸಚ್ಚಿದಾನಂದ ಸಾಗರದಲ್ಲಿ ಮುಳುಗಿಸು. ಮನಸ್ಸು ನಾಶವಾದಾಗ ಅದರ ಅಧೀನತೆಯೂ ಮಾಯವಾಗಿ ನೀನು ಬ್ರಹ್ಮದಲ್ಲಿ ಐಕ್ಯವಾಗುವೆ.

ಶಿಷ್ಯ: ಈ ಚಂಚಲ ಮನಸ್ಸನ್ನು ಬ್ರಹ್ಮನೆಡೆಗೆ ತಿರುಗಿಸುವುದು ಬಹಳ ಕಷ್ಟ.

ಸ್ವಾಮೀಜಿ: ವೀರನಿಗೆ ಕಷ್ಟವಾದುದಾವುದಿದೆ? ಕೇವಲ ದುರ್ಬಲ ಮನುಷ್ಯ ಮಾತ್ರ ಹೀಗೆ ಮಾತಾಡುವನು, ‘ಧೀರನಿಗೆ ಮಾತ್ರ ಮುಕ್ತಿಗಳಿಸುವುದು ಸುಲಭ, ಹೇಡಿಗಲ್ಲ’ - ‘ವೀರಾಣಾಮೇವ ಕರತಲಗತಾ ಮುಕ್ತಿಃ ನ ಪುನಃ ಕಾಪುರುಷಾಣಾಮ್’ ‘ತ್ಯಾಗ ಮತ್ತು ಅಭ್ಯಾಸದಿಂದ ಮನಸ್ಸನ್ನು ಹಿಡಿತದಲ್ಲಿಟ್ಟುಕೊಳ್ಳಬಹುದು, ಓ ಅರ್ಜುನಾ’ ‘ಅಭ್ಯಾಸೇನ ತು ಕೌಂತೇಯ ವೈರಾಗ್ಯೇಣ ಚ ಗೃಹ್ಯತೇ’ ಎಂದು ಗೀತೆ ಸಾರುವುದು. ಚಿತ್ತವು ಸ್ವಚ್ಛವಾದೊಂದು ಸರೋವರದಂತೆ. ಅದರಲ್ಲಿ ಇಂದ್ರಿಯ ಭಾವನೆಗಳಿಂದೇಳುವ ಅಲೆಗಳ ಹೊಡೆತವೇ ಮನಸ್ಸು. ಆದ್ದರಿಂದ ಮನಸ್ಸು ಈ ಯೋಚನಾತರಂಗಗಳ ಶ್ರೇಣಿ. ಈ ಮಾನಸಿಕ ತರಂಗಗಳಿಂದ ಆಸೆಗಳುದಿಸುವುವು. ನಂತರ ಆ ಆಸೆಯೇ ಸಂಕಲ್ಪವಾಗಿ ಮಾರ್ಪಟ್ಟು ಸ್ಥೂಲ ಯಂತ್ರವಾದ ಈ ದೇಹದ ಮೂಲಕ ಕೆಲಸ ಮಾಡುವುದು. ಅದಕ್ಕೇ ಮನಸ್ಸು ಸರ್ವ ಕಾಲದಲ್ಲಿಯೂ ಈ ಕೆಲಸದ ಫಲದಿಂದುಂಟಾದ ಅಸಂಖ್ಯಾತ ತರಂಗಗಳಿಂದ ಅಲ್ಲೋಲಕಲ್ಲೋಲವಾಗಿರುವುದು. ಮನಸ್ಸು ಈ ವೃತ್ತಿಗಳೆಲ್ಲದರಿಂದ ಬಿಡುಗಡೆ ಹೊಂದಿ ಮೊದಲಿನಂತೆ ಸ್ವಚ್ಛ ಸರೋವರವಾಗಿ ಮಾರ್ಪಡಬೇಕು. ಚಿತ್ತಫಲಕದ ಮೇಲೆ ಯಾವೊಂದು ವೃತ್ತಿಯ ತರಂಗವೂ ಇರಬಾರದು. ಆಗ ತಾನೇ ಬ್ರಹ್ಮನು ಅಲ್ಲಿ ವಿಕಾಸವಾಗುವನು. ‘ಆಗ ಹೃದಯದ ಎಲ್ಲಾ ಬಂಧನಗಳೂ ಕಡಿದುಹಾಕಲ್ಪಡುವುವು’ ಎಂದು ಶಾಸ್ತ್ರಗಳು ಈ ಅವಸ್ಥೆಯ ಒಂದು ಕ್ಷಣಿಕ ನೋಟವನ್ನು ಮಾತ್ರ ವ್ಯಕ್ತಗೊಳಿಸುವುವು. ಅರ್ಥವಾಯಿತೆ?

ಶಿಷ್ಯ: ಹೌದು, ಆದರೆ ಧ್ಯಾನಕ್ಕೆ ಒಂದು ನಿರ್ದಿಷ್ಟ ವಸ್ತು ಬೇಕಲ್ಲವೆ?

ಸ್ವಾಮೀಜಿ: ನೀನೇ ನಿನ್ನ ಧ್ಯಾನದ ವಸ್ತುವಾಗುವೆ. ನೀನೇ ಆ ಸರ್ವವ್ಯಾಪಿಯಾದ ಆತ್ಮನೆಂದು ಭಾವಿಸಿ ಧ್ಯಾನಿಸು. ‘ನಾನು ದೇಹವೂ ಅಲ್ಲ, ಮನಸ್ಸೂ ಅಲ್ಲ, ಬುದ್ಧಿಯೂ ಅಲ್ಲ. ಸ್ಥೂಲ ಮತ್ತು ಸೂಕ್ಷ್ಮ ಶರೀರವೂ ಅಲ್ಲ.’ ಈ ರೀತಿ ಎಲ್ಲವನ್ನೂ ವಿಸರ್ಜಿಸುತ್ತಾ ನಿನ್ನ ಮನಸ್ಸನ್ನು ನಿನ್ನ ಸಹಜ ಸ್ವಭಾವವಾದ ಪರಮಜ್ಞಾನದಲ್ಲಿ ಮುಳುಗಿಸಬೇಕು. ಇದರಲ್ಲಿ ಬಾರಿಬಾರಿಗೂ ಮುಳುಗುತ್ತಾ ನಿನ್ನ ಮನಸ್ಸನ್ನು ನಿಗ್ರಹಿಸು. ಆಗ ಮಾತ್ರ ನೀನು ಜ್ಞಾನದ ತಿರುಳನ್ನು ಸಾಕ್ಷಾತ್ಕರಿಸಿಕೊಳ್ಳುವೆ - ನಿನ್ನ ಸಹಜ ಪ್ರಕೃತಿಯನ್ನರಿಯುವೆ. ಜ್ಞಾತೃ ಜ್ಞೇಯ, ಧ್ಯಾತೃ ಧ್ಯೇಯ ಎಲ್ಲ ಏಕವಾಗಿ, ಅಧ್ಯಾಸವೆಲ್ಲ ಮಾಯವಾಗುವುವು. ಶಾಸ್ತ್ರಗಳಲ್ಲಿ ಇದನ್ನು ತ್ರಿಪುಟಿ ಭೇದಾತೀತವೆನ್ನುವರು. ಈ ಅವಸ್ಥೆಯಲ್ಲಿ ಜ್ಞಾನಕ್ಕೆ ಎಲ್ಲೆಯಿಲ್ಲ. ಆತ್ಮನೊಬ್ಬನೇ ಸರ್ವಜ್ಞ, ಆತನನ್ನಾವ ಹಾದಿಯಿಂದ ಅರಿಯಬಲ್ಲೆ? ಆತ್ಮನೇ ಜ್ಞಾನ, ಆತ್ಮನೇ ಬುದ್ಧಿ, ಆತ್ಮನೇ ಸಚ್ಚಿದಾನಂದ. ಮಾಯೆಯ ನಿಗೂಢವಾದ ಶಕ್ತಿಯಿಂದ ಯಾವುದು ಸತ್ಯ ಯಾವುದು ಮಿಥ್ಯೆ ಎಂದು ಅರಿಯಲಾಗದಿರುವ ಈ ಸಾಂಬಂಧಿಕ ದೇಹಭಾವನೆ ಜೀವನಿಗೆ ಉಂಟಾಗಿದೆ. ಅವನು ಬ್ರಹ್ಮನೇ ಹೊರತು ಮತ್ತಾವುದೂ ಅಲ್ಲ. ಇದಕ್ಕೆ ಸಾಧಾರಣವಾಗಿ ಜಾಗೃತಾವಸ್ಥೆ ಎಂದು ಹೆಸರು. ದ್ವೈತ ಸಾಂಬಂಧಿಕ ದೇಹ ಭಾವನೆಯು ಪವಿತ್ರ ಬ್ರಹ್ಮನಲ್ಲಿ ಐಕ್ಯವಾಗುವುದನ್ನೆ ಶಾಸ್ತ್ರಗಳು ಅತೀಂದ್ರಿಯಾವಸ್ಥೆ ಎನ್ನುವುವು. ಅದನ್ನು ಹೀಗೆ ವಿವರಿಸಲಾಗಿದೆ: ‘ಸ್ತಿಮಿತಸಲಿಲರಾಶಿಪ್ರಖ್ಯಮಾಖ್ಯಾವಿಹೀನಂ’, ಇದು ಹೆಸರಿಲ್ಲದ ಒಂದು ಶಾಂತವಾದ ಸ್ತಬ್ಧ ಸಾಗರದಂತೆ.

ಸ್ವಾಮಿಗಳು ಬ್ರಹ್ಮಸಾಕ್ಷಾತ್ಕಾರ ಪಡೆದ ಗಂಭೀರ ವಾಣಿಯಿಂದ ಮುಂದುವರಿಸಿದರು.

ಸ್ವಾಮೀಜಿ: ಎಲ್ಲಾ ತತ್ತ್ವಜ್ಞಾನಗಳೂ, ಧರ್ಮ ಗ್ರಂಥಗಳೂ ದೃಗ್ ದೃಶ್ಯ ಸಂಬಂಧದಿಂದ ಬಂದಿವೆ. ಆದರೆ ಯಾವ ಮಾನಸಿಕ ಯೋಚನೆಯಾಗಲೀ ಅಥವಾ ಭಾಷೆಯಾಗಲೀ ಕಾರ್ಯಕಾರಣ ಸಂಬಂಧಗಳ ಪ್ರಪಂಚವನ್ನು ಮೀರಿರುವ ಸತ್ಯವನ್ನು ಪೂರ್ಣವಾಗಿ ವಿವರಿಸುವುದು ಅಸಾಧ್ಯ. ವಿಜ್ಞಾನ, ತತ್ತ್ವಶಾಸ್ತ್ರ ಎಲ್ಲವೂ ಅಪೂರ್ಣ ಸತ್ಯಗಳು. ಅತೀಂದ್ರಿಯ ಸತ್ಯವನ್ನು ವಿವರಿಸಲು ಇವಾವುವೂ ಯೋಗ್ಯವಲ್ಲ. ಅಂತೆಯೇ ಅತೀಂದ್ರಿಯಾವಸ್ಥೆಯ ಪಕ್ಷದಿಂದ ಹೇಳುವುದಾದರೆ ಎಲ್ಲಾ ವಸ್ತುಗಳೂ ಮಿಥ್ಯ ಎನ್ನಿಸುವುದು - ಧಾರ್ಮಿಕ ಪಂಥಗಳು, ಕೆಲಸಗಳು, ನಾನು ಮತ್ತು ನೀನು ಈ ವಿಶ್ವ - ಪ್ರತಿಯೊಂದೂ ಮಿಥ್ಯ. ಈಗ ಕೇವಲ ನಾನು ಮಾತ್ರ ಸತ್ಯವೆಂಬುದರ ಅರಿವಾಗುವುದು. ನಾನೇ ಸರ್ವ ವ್ಯಾಪಕನಾದ ಆತ್ಮ, ನನ್ನ ಇರುವಿಕೆಗೆ ನಾನೇ ಪ್ರಮಾಣ. ನನ್ನ ಇರುವಿಕೆಯ ಸತ್ಯವನ್ನು ಸ್ಥಾಪಿಸಲು ಬೇರೊಂದು ಪ್ರತ್ಯೇಕ ಪ್ರಮಾಣವೇಕೆ? ಧರ್ಮಶಾಸ್ತ್ರಗಳು ಹೇಳುವಂತೆ ಯಾವಾಗಲೂ ನಾನು ನಿತ್ಯ ಸತ್ಯವೆಂಬುದು ನನಗೆ ಗೊತ್ತಿದೆ 'ನಿತ್ಯಮಸ್ಮತ್ ಪ್ರಸಿದ್ಧಮ್ (ವಿವೇಕಚೂಡಾಮಣಿ 409). ನಾನು ಈ ಸ್ಥಿತಿಯನ್ನು ಕಣ್ಣಾರೆ ಕಂಡಿರುವೆ, ಅನುಭವಿಸಿರುವೆ. ನೀನೂ ಕೂಡ ಅದನ್ನು ನೋಡಬಹುದು. ಎಲ್ಲರಿಗೂ ಈ ಬ್ರಹ್ಮಸತ್ಯವನ್ನು ಬೋಧಿಸಬಹುದು. ಆಗ ಮಾತ್ರ ನಿನಗೆ ಶಾಂತಿ ದೊರಕುವುದು.

ಈ ಮಾತನ್ನು ಹೇಳುತ್ತಿರುವಾಗ ಸ್ವಾಮಿಗಳು ಗಾಢವಾದ ಯೋಚನೆಯಲ್ಲಿ ತನ್ಮಯವಾಗಿದ್ದರು. ಸ್ವಲ್ಪಕಾಲದ ಮೇಲೆ ಅವರು ಮುಂದುವರಿಸಿದರು: “ಯಾವುದು ಎಲ್ಲಾ ತತ್ತ್ವಗಳನ್ನೂ ಒಳಗೊಂಡಿದೆಯೋ, ಎಲ್ಲ ಸತ್ಯಗಳ ತತ್ತ್ವಾಧಾರವೋ ಅಂತಹ ಬ್ರಹ್ಮಜ್ಞಾನವನ್ನು ನಿನ್ನ ಸ್ವಂತ ಜೀವನದಲ್ಲೇ ಸಾಕ್ಷಾತ್ಕರಿಸಿಕೊ. ಅದನ್ನು ಜಗತ್ತಿಗೆ ಬೋಧಿಸು. ಇದು ನಿನಗೂ ಮತ್ತು ಇತರರೆಲ್ಲರಿಗೂ ಒಳ್ಳೆಯದು. ಇಂದು ನಿನಗೆ ಎಲ್ಲಾ ಸತ್ಯಗಳ ತಿರುಳನ್ನೂ ಹೇಳಿರುವೆ, ಇದಕ್ಕಿಂತ ಮಹತ್ವವಾದುದಾವುದೂ ಇಲ್ಲ."

ಶಿಷ್ಯ: ಸ್ವಾಮಿಜಿ, ಈಗ ನೀವು ಜ್ಞಾನದ ವಿಚಾರ ಮಾತನಾಡುತ್ತಿರುವಿರಿ. ಆದರೆ ಕೆಲವು ವೇಳೆ ಭಕ್ತಿಗೆ, ಕೆಲವು ವೇಳೆ ಕರ್ಮಕ್ಕೆ, ಮತ್ತೆ ಕೆಲವು ವೇಳೆ ಯೋಗಕ್ಕೂ ಹೆಚ್ಚು ಪ್ರಾಧಾನ್ಯ ಕೊಟ್ಟು ಮಾತನಾಡುವಿರಿ. ಇದರಿಂದ ನಮ್ಮ ಮನಸ್ಸು ಹೆಚ್ಚು ಗಲಿಬಿಲಿಗೊಳ್ಳುವುದು.

ಸ್ವಾಮೀಜಿ: ಸತ್ಯ ಹೀಗಿದೆ, ಬ್ರಹ್ಮಜ್ಞಾನವೇ ಚರಮಗುರಿ, ಮಾನವನ ಧನ್ಯಾವಸ್ಥೆ. ಎಲ್ಲಾ ಕಾಲಗಳಲ್ಲಿಯೂ ಮಾನವನು ಬ್ರಹ್ಮನಲ್ಲಿಯೇ ಮುಳುಗಿರಲಾರ. ಅದರಿಂದ ಹೊರಗೆ ಬಂದಾಗ ಮಾಡಲು ಏನಾದರೂ ಇರಬೇಕು. ಅಂತಹ ಸಮಯದಲ್ಲಿ ಅವರು ಮಾಡುವ ಕೆಲಸ ಲೋಕಕಲ್ಯಾಣಕಾರಿಯಾಗಿರಬೇಕು. ಅದಕ್ಕೇ ನಾನು ನಿಮ್ಮನ್ನು ಜೀವಿಗಳೆಲ್ಲಾ ಒಂದೇ ಎಂದು ಭಾವಿಸಿ ಸೇವೆ ಮಾಡುವಂತೆ ಪ್ರೇರೇಪಿಸುತ್ತೇನೆ. ಆದರೆ ಮಗು! ಕರ್ಮದ ತೊಡಗು ಹೇಗಿದೆಯೆಂದರೆ ಎಂತಹ ದೊಡ್ಡ ಮಹಾತ್ಮರೂ ಕೂಡ ಅದರಲ್ಲಿ ಸಿಲುಕಿ ಅನುರಕ್ತರಾಗುವರು. ಆದ್ದರಿಂದ ಕರ್ಮವನ್ನು ಪ್ರತಿಫಲಾಪೇಕ್ಷೆಯಿಲ್ಲದೆ ಮಾಡಬೇಕು. ಇದೇ ಗೀತೆಯ ಉಪದೇಶ. ಆದರೆ ಬ್ರಹ್ಮಜ್ಞಾನದಲ್ಲಿ ಕರ್ಮಸಂಬಂಧದ ಸಂಪರ್ಕ ಲವಲೇಶವೂ ಇರುವುದಿಲ್ಲವೆಂಬುದನ್ನು ತಿಳಿ. ಸತ್ಕಾರ್ಯಗಳು ಹೆಚ್ಚೆಂದರೆ ಮನಸ್ಸನ್ನು ಪರಿಶುದ್ಧಗೊಳಿಸುವುವು. ಅದಕ್ಕೇ ಭಾಷ್ಯಕಾರ ಶಂಕರಾಚಾರ್ಯರು ಜ್ಞಾನ ಕರ್ಮಗಳ ಸಂಯೋಜನೆಯನ್ನು ಅಷ್ಟೊಂದು ತೀವ್ರವಾಗಿ ಖಂಡಿಸಿದ್ದಾರೆ. ಕೆಲವರು ನಿಷ್ಕಾಮ ಕರ್ಮದ ಮೂಲಕ ಬ್ರಹ್ಮಜ್ಞಾನ ಹೊಂದುವರು. ಇದೂ ಒಂದು ಮಾರ್ಗ ಮಾತ್ರ. ಆದರೆ ಅದರ ಗುರಿ ಬ್ರಹ್ಮಸಾಕ್ಷಾತ್ಕಾರ. ಆದ್ದರಿಂದ ಇದನ್ನು ಖಂಡಿತವಾಗಿ ತಿಳಿ - ಯುಕ್ತಾಯುಕ್ತ ವಿವೇಚನಜ್ಞಾನದ ಗುರಿಯೂ ಮತ್ತು ಇತರ ಎಲ್ಲಾ ಬಗೆಯ ಸಾಧನೆಗಳ ಗುರಿಯೂ ಒಂದೇ ಬ್ರಹ್ಮಸಾಕ್ಷಾತ್ಕಾರ.

ಶಿಷ್ಯ: ಸ್ವಾಮೀಜಿ, ನನಗೆ ದಯವಿಟ್ಟು ಭಕ್ತಿಯೋಗ, ರಾಜಯೋಗಗಳ ಉಪಯುಕ್ತತೆಯನ್ನು ತಿಳಿಸುವಿರಾ?

ಸ್ವಾಮಿಜಿ: ಈ ಹಾದಿಯಲ್ಲಿ ಹೋರಾಡಿಯೂ ಕೆಲವರು ಬ್ರಹ್ಮಜ್ಞಾನ ಪಡೆಯುವರು. ಭಕ್ತಿಮಾರ್ಗದಲ್ಲಿ ಬಹಳ ನಿಧಾನವಾದ ಪ್ರಗತಿ. ಆದರೆ ಸಾಧನೆಗೆ ಸುಲಭ ಮಾರ್ಗ. ಯೋಗಮಾರ್ಗದಲ್ಲಿ ಬಹಳ ಎಡರುಗಳಿವೆ. ಮನಸ್ಸು ಸಹಜವಾಗಿ ಮಾನಸಿಕ ಶಕ್ತಿಗೆ ಮರುಳಾಗಿ ಅದರಿಂದ ನಿನ್ನ ಸ್ವಭಾವಕ್ಕೆ ದೂರನಾಗುವೆ. ಜ್ಞಾನಮಾರ್ಗ ಬೇಗ ಫಲಪ್ರದವಾಗುವುದು. ಇದು ಇತರ ಪಂಥಗಳ ತತ್ತ್ವಾಧಾರ. ಆದ್ದರಿಂದ ಇದು ಸರ್ವ ಕಾಲಗಳಲ್ಲಿಯೂ ಸರ್ವ ದೇಶಗಳಲ್ಲಿಯೂ ಸಮಾನವಾಗಿ ಪ್ರಶಂಸಿಸಲ್ಪಡುವುದು. ಆದರೆ ವಿಚಾರದ ಹಾದಿಯಲ್ಲಿ ಕೂಡ ಮನಸು ಒಣ ತರ್ಕದ ಕೊನೆಯಿಲ್ಲದ ಬಲೆಯಲ್ಲಿ ಸಿಲುಕಬಹುದು. ಅದರ ಜೊತೆಯಲ್ಲೇ ಧ್ಯಾನವನ್ನು ಅಭ್ಯಾಸ ಮಾಡಬೇಕು. ಯುಕ್ತಾಯುಕ್ತ ವಿಮರ್ಶೆ, ಧ್ಯಾನ ಇವುಗಳಿಂದ ಬ್ರಹ್ಮನನ್ನು ಸೇರಬೇಕು. ಈ ಬಗೆಯ ಸಾಧನೆಯಿಂದ ಖಂಡಿತವಾಗಿಯೂ ಆತನನ್ನು ಹೊಂದಬಹುದು. ನನ್ನ ಅಭಿಪ್ರಾಯದಲ್ಲಿ ಇದೇ ಶೀಘ್ರ ಜಯ ದೊರಕಲು ಸುಲಭದ ಹಾದಿ.

ಶಿಷ್ಯ: ದಯವಿಟ್ಟು ಭಗವಂತನ ಅವತಾರದ ವಿಷಯವಾಗಿ ಏನನ್ನಾದರೂ ಹೇಳಿ.

ಸ್ವಾಮಿಜಿ: ಒಂದೇ ದಿನದಲ್ಲಿ ಸರ್ವಪಾರಂಗತನಾಗಬೇಕೆಂದು ತೋರುವುದು!

ಶಿಷ್ಯ: ಒಂದೇ ದಿನದಲ್ಲಿ ಮನಸ್ಸಿನ ಸಂಶಯ, ಕಷ್ಟ ಪರಿಹಾರವಾಗುವಹಾಗಿದ್ದರೆ, ನಿಮಗೆ ಮತ್ತೆ ಮತ್ತೆ ತೊಂದರೆ ಕೊಡಬೇಕಾದ ಆವಶ್ಯಕತೆಯೇ ಇರುವುದಿಲ್ಲ.

ಸ್ವಾಮಿಜಿ: ಯಾರ ಕೃಪೆಯಿಂದ ಧರ್ಮಗ್ರಂಥಗಳು ಅಷ್ಟೊಂದು ಹೊಗಳುವ ಆತ್ಮಪರಿಜ್ಞಾನವನ್ನು ಒಂದೇ ಒಂದು ನಿಮಿಷದಲ್ಲಿ ಸಾಧಿಸಬಹುದೋ ಅಂತಹ ಸಚಲ ತೀರ್ಥಗಳೇ ಅವತಾರಪುರುಷರು. ಹುಟ್ಟಿದಾಗಲೇ ಅವರು ಬ್ರಹ್ಮಜ್ಞಾನಿಗಳು. ಬ್ರಹ್ಮನಿಗೂ ಬ್ರಹ್ಮಜ್ಞಾನಿಗೂ ಯಾವೊಂದು ವ್ಯತ್ಯಾಸವೂ ಇಲ್ಲ. ‘ಯಾರು ಬ್ರಹ್ಮನನ್ನು ಅರಿತಿದ್ದಾನೋ ಅವನು ಬ್ರಹ್ಮನೇ ಆಗಿದ್ದಾನೆ’, ‘ಬ್ರಹ್ಮವೇದ ಬ್ರಹ್ಮೈವ ಭವತಿ’ - (ಮುಂಡಕ \enginline{III, ii.9)} ಮನಸ್ಸಿನಿಂದ ಆತ್ಮನನ್ನು ಗ್ರಹಿಸಲಾಗುವುದಿಲ್ಲ. ಏಕೆಂದರೆ ಆತ್ಮವೆ ಜ್ಞಾನವಸ್ತು. ಇದನ್ನು ನಾನು ಮೊದಲೇ ಹೇಳಿದ್ದೇನೆ. ಆದ್ದರಿಂದ ಮಾನವನ ಕಾರ್ಯಕಾರಣ ಸಂಬಂಧ ಜ್ಞಾನವು ಆತ್ಮಪ್ರತಿಷ್ಠಿತರಾದ ಅವತಾರಿಗಳವರೆಗೆ ಮಾತ್ರ ಹೋಗಲು ಸಾಧ್ಯ. ಮಾನವನ ಮನಸ್ಸು ಗ್ರಹಿಸಬಲ್ಲ ಭಗವಂತನ ಶ್ರೇಷ್ಠರೂಪವೇ ಅವತಾರ. ಅದರಿಂದಾಚೆ ಕಾರ್ಯಕಾರಣಸಂಬಂಧ ಜ್ಞಾನವಿಲ್ಲ. ಅಂತಹ ಬ್ರಹ್ಮಜ್ಞಾನಿಗಳು ಈ ಪ್ರಪಂಚದಲ್ಲಿ ಅತಿ ವಿರಳ. ಅಂಥವರನ್ನು ಅರ್ಥ ಮಾಡಿಕೊಳ್ಳುವವರು ಕೇವಲ ಕೆಲವು ಮಂದಿ ಮಾತ್ರ. ಅವರೇ ಧರ್ಮ ಗ್ರಂಥಗಳಲ್ಲಿರುವ ಸತ್ಯಕ್ಕೆ ಪ್ರಮಾಣ. ವಿಶ್ವಸಾಗರದಲ್ಲಿ ಇರುವ ಕೆಲವು ಜ್ಯೋತಿರ್ಮಯ ಸ್ತಂಭಗಳು ಅವರು. ಇಂತಹ ಅವತಾರ ಪುರುಷರ ಸಂಗದಿಂದ, ಅವರ ಕೃಪೆಯಿಂದ ನಿಮಿಷಮಾತ್ರದಲ್ಲಿ ಮನಸ್ಸಿನ ಅಂಧಕಾರ ತೊಲಗಿ ಹೃದಯದಲ್ಲಿ ಸಾಕ್ಷಾತ್ಕಾರದ ಮಿಂಚು ಹೊಳೆಯುವುದು. ಏಕೆ ಮತ್ತು ಯಾವ ವಿಧಾನದಿಂದ ಅದು ಬರುತ್ತದೆಂಬುದನ್ನು ನಿಶ್ಚಿತವಾಗಿ ಹೇಳಲಾಗುವುದಿಲ್ಲ. ಆದರೆ ಅದು ಬಂದೇ ಬರುತ್ತದೆ. ಅದು ಹಾಗೆ ಸಂಭವಿಸಿದ್ದನ್ನು ನಾನು ನೋಡಿರುವೆ. ಶ‍್ರೀಕೃಷ್ಣ ಆತ್ಮಸ್ಥಿತನಾಗಿ ಗೀತೆಯನ್ನು ಬೋಧಿಸಿದನು. ಗೀತೆಯ ಸೂತ್ರಗಳಲ್ಲಿ ಎಲ್ಲೆಲ್ಲಿ ‘ನಾನು’ ಎಂಬುದು ಬರುತ್ತದೆಯೋ ಅಲ್ಲೆಲ್ಲ ಯಥಾವಿಧಿಯಾಗಿ ಆತ್ಮನನ್ನು ಉದ್ದೇಶಿಸಿರುತ್ತದೆ. ‘ಕೇವಲ ನನ್ನನ್ನು ಶರಣುಹೊಂದಿ’ ಎಂದರೆ ‘ಆತ್ಮನಲ್ಲಿ ಲೀನವಾಗಿ’ ಎಂದರ್ಥ. ಈ ಆತ್ಮಪರಿಜ್ಞಾನವೇ ಗೀತೆಯ ಚರಮಗುರಿ. ಯೋಗದ ವಿಚಾರಗಳೆಲ್ಲಾ ಈ ಆತ್ಮಸಾಕ್ಷಾತ್ಕಾರಕ್ಕೆ ಪ್ರಾಸಂಗಿಕ ಮಾತ್ರ. ಯಾರಿಗೆ ಈ ಆತ್ಮವಿಜ್ಞಾನವಿಲ್ಲವೋ ಅವರೆಲ್ಲಾ ಆತ್ಮಘಾತಕರು. ‘ಮಾಯೆಯನ್ನಪ್ಪಿಕೊಂಡು ತಮ್ಮನ್ನೇ ನಾಶಪಡಿಸಿಕೊಳ್ಳುತ್ತಿದ್ದಾರೆ.’ - ವಿಷಯ ಸುಖದ ಕುಣಿಕೆಯಲ್ಲಿ ಅವರು ತಮ್ಮ ಪ್ರಾಣಹಾನಿ ಮಾಡಿಕೊಳ್ಳುವರು. ನೀವೂ ಮನುಷ್ಯರು. ಎರಡು ದಿನ ಬಾಳುವ ಈ ಕ್ಷುದ್ರ ಇಂದ್ರಿಯಸುಖದಾಸೆಯನ್ನು ನಿರ್ಲಕ್ಷಿಸಲಾರಿರಾ? ಕೇವಲ ಅಜ್ಞಾನದಲ್ಲಿ ಹುಟ್ಟಿ ಸಾಯುವವರ ಗುಂಪನ್ನೇ ನೀವು ಹೆಚ್ಚಿಸುವಿರಾ? ಯಾವುದು ಶ್ರೇಯಸ್ಕರವೋ ಅದನ್ನು ಸ್ವೀಕರಿಸಿ. ಯಾವುದು ಪ್ರೇಯಸ್ಸೋ ಅದನ್ನು ವಿಸರ್ಜಿಸಿ. ಈ ಆತ್ಮ ವಿಚಾರವನ್ನು ಎಲ್ಲರಿಗೂ, ಅತ್ಯಂತ ಕೀಳು ಜಾತಿಯವರಿಗೂ ಬೋಧಿಸಿ. ಈ ಮಾತನ್ನೇ ಮುಂದುವರಿಸುತ್ತಿದ್ದರೆ ನಿಮ್ಮ ತಿಳಿವಳಿಕೆಯೂ ಸ್ಪಷ್ಟವಾಗುವುದು. ಯಾವಾಗಲೂ ಈ ಮಹಾ ಮಂತ್ರವನ್ನು ಪಠಿಸುತ್ತಿರಿ “ಸೋಽಹಮಸ್ಮಿ - ನಾನು ಅದೇ ಆಗಿದ್ದೇನೆ", “ತತ್ತ್ವಮಸಿ - ನೀನು ಅದೇ ಆಗಿರುವೆ", “ಸರ್ವಂ ಖಲ್ವಿದಂ ಬ್ರಹ್ಮ - ಸರ್ವವು ಬ್ರಹ್ಮವೇ ಆಗಿದೆ". ಸಿಂಹಸಮನಾದ ಕೆಚ್ಚು ನಿಮ್ಮಲ್ಲಿರಲಿ. ಅಂಜಿಕೆಗೆ ಕಾರಣವೇನಿದೆ? ಅಂಜಿಕೆಯೇ ಸಾವು - ಅಂಜಿಕೆಯೇ ಮಹಾ ಪಾಪ! ಮಾನವನ ಪ್ರತಿನಿಧಿಯಂತೆ ಇರುವ ಅರ್ಜುನ ಭಯದಿಂದ ಆವೃತನಾದ. ಅದಕ್ಕೆ ಆತ್ಮ ಪ್ರತಿಷ್ಠಾಪಿತನಾದ ಭಗವಾನ್ ಶ‍್ರೀಕೃಷ್ಣನು ಗೀತೆಯ ಬೋಧನೆಗಳನ್ನು ಉಪದೇಶಿಸಿದನು. ಆದರೂ ಅವನ ಅಂಜಿಕೆ ದೂರವಾಗಲಿಲ್ಲ. ಭಗವಂತನ ವಿಶ್ವರೂಪದರ್ಶನ ಮಾಡಿ ಅವನು ಆತ್ಮಪ್ರತಿಷ್ಠಾಪಿತನಾದಾಗ ಕರ್ಮ ಬಂಧನಗಳೆಲ್ಲಾ ಜ್ಞಾನಾಗ್ನಿಯಿಂದ ಭಸ್ಮೀಭೂತವಾದವು. ಅನಂತರ ಅವನು ಯುದ್ಧ ಮಾಡಿದನು.

ಶಿಷ್ಯ: ಸಾಕ್ಷಾತ್ಕಾರವಾದ ಮೇಲೂ ಮನುಷ್ಯನು ಕೆಲಸ ಮಾಡಬೇಕಾಗಿ ಬಂದರೆ?

ಸ್ವಾಮೀಜಿ: ಸಾಕ್ಷಾತ್ಕಾರವಾದ ಮೇಲೆ ನಾವು ಸಾಧಾರಣವಾಗಿ ಮಾಡಬೇಕಾದ ಕರ್ಮ ಇರುವುದಿಲ್ಲ. ಅದರ ಸ್ವಭಾವ ವ್ಯತ್ಯಾಸವಾಗುತ್ತದೆ. ಜ್ಞಾನಿ ಮಾಡುವ ಕರ್ಮಗಳೆಲ್ಲಾ ಜಗದ ಕಲ್ಯಾಣಕ್ಕೆ; ಬ್ರಹ್ಮಸಾಕ್ಷಾತ್ಕಾರವಾದ ಮನುಷ್ಯನ ಮಾತು ಕೆಲಸಗಳೆಲ್ಲ ಸರ್ವರ ಹಿತಕ್ಕಾಗಿ. ನಾವು ಶ‍್ರೀರಾಮಕೃಷ್ಣರಲ್ಲಿ ಇದನ್ನು ನೋಡುತ್ತಿದ್ದೆವು. ಅವರು ಶರೀರದಲ್ಲಿದ್ದರು - ಅದಕ್ಕೆ ಸೇರಿರಲಿಲ್ಲ. ದೇಹಸ್ಥೋಽಪಿ ನ ದೇಹಸ್ಥಃ‘ ಅಂತಹ ಮಹಾತ್ಮರ ಕೆಲಸದ ಉದ್ದೇಶವಾಗಿ ಇಷ್ಟನ್ನು ಮಾತ್ರ ಹೇಳಬಹುದು: “ಅವರು ಮಾನವರಂತೆ ಮಾಡುವ ಕೆಲಸಗಳೆಲ್ಲ ಕೇವಲ ಲೀಲೆ ಮಾತ್ರ." - ’ಲೋಕವತ್ತು ಲೀಲಾ ಕೈವಲ್ಯಂ' (ಬ್ರಹ್ಮ ಸೂತ್ರಗಳು, \enginline{II.i.33).}

\newpage

\chapter[ಅಧ್ಯಾಯ ೩೩]{ಅಧ್ಯಾಯ ೩೩\protect\footnote{\engfoot{C.W, Vol. VII, P. 200}}}

\begin{center}
ಸ್ಥಳ: ಬೇಲೂರು ಮಠ (ಬಾಡಿಗೆ ಕಟ್ಟಡ), ವರ್ಷ: ಕ್ರಿ.ಶ. ೧೯೦೧.
\end{center}

ಇಂದು ಶಿಷ್ಯ ಜೂಬಿಲಿ ಕಲಾಶಾಲೆಯ ಸ್ಥಾಪಕ ಮತ್ತು ಅಧ್ಯಾಪಕರಾದ ರನದ ಪ್ರಸಾದ್ ದಾಸಗುಪ್ತರೊಡನೆ ಮಠಕ್ಕೆ ಬಂದಿದ್ದ. ರನದ ಬಾಬುಗಳು ಪ್ರವೀಣರು, ಘನವಿದ್ವಾಂಸರು, ಸ್ವಾಮಿಗಳನ್ನು ಮೆಚ್ಚಿದವರು. ಕುಶಲ ಪ್ರಶ್ನೆಯಾದ ನಂತರ ಸ್ವಾಮಿಗಳು ರನದ ಬಾಬುಗಳೊಡನೆ ಕಲೆಯ ಮೇಲೆ ಅನೇಕ ಸಂಗತಿಗಳನ್ನು ಮಾತನಾಡಲಾರಂಭಿಸಿದರು.

ಸ್ವಾಮೀಜಿ: ಪ್ರಪಂಚದಲ್ಲಿರುವ ಎಲ್ಲಾ ನಾಗರಿಕ ದೇಶಗಳ ಕಲಾ ಸೌಂದರ್ಯಗಳನ್ನು ನೋಡುವ ಸುಯೋಗ ನನಗೆ ದೊರಕಿತು. ಆದರೆ ನಮ್ಮ ದೇಶದಲ್ಲಿ ಬೌದ್ಧರ ಕಾಲದಲ್ಲಿ ಕಲೆಯು ವಿಕಾಸವಾದಷ್ಟು ಮತ್ತೆಲ್ಲಿಯೂ ಕಾಣಬರಲಿಲ್ಲ. ಮೊಗಲರ ಕಾಲದಲ್ಲಿಯೂ ಕಲೆಯ ಬೆಳವಣಿಗೆ ತಕ್ಕಷ್ಟು ಇತ್ತು. ತಾಜ್, ಜುಮ್ಮಾ ಮಸೀದಿ ಮುಂತಾದುವು ಆ ಸಂಸ್ಕೃತಿಯ ಚಿರಸ್ಮರಣೀಯ ಸ್ಮಾರಕಗಳು.

"ಮಾನವನು ಯಾವುದನ್ನೇ ನಿರ್ಮಿಸಲಿ ಆ ಕಲೆಯ ಮೂಲ ಯಾವುದೊ ಒಂದು ಭಾವವನ್ನು ಸ್ಪಷ್ಟಪಡಿಸುವುದಾಗಿದೆ. ಎಲ್ಲಿ ಆ ಉದ್ದೇಶದ ಅಭಾವವಿದೆಯೋ ಅಲ್ಲಿ ಎಷ್ಟೇ ಬಣ್ಣ ಮುಂತಾದುವುಗಳ ಪ್ರದರ್ಶನವಿದ್ದರೂ ಅದನ್ನು ಸತ್ಯವಾದ ಕಲೆ ಎನ್ನಲಾಗುವುದಿಲ್ಲ. ನಮ್ಮ ನಿತ್ಯ ಜೀವನದಲ್ಲಿ ಉಪಯೋಗಿಸುವ ನೀರಿನ ಚೊಂಬು ತಟ್ಟೆ ಬಟ್ಟಲುಗಳೂ ಕೂಡ ಒಂದು ಭಾವದ ಪ್ರತಿಬಿಂಬವಾಗಿರಬೇಕು. ಪ್ಯಾರಿಸ್ಸಿನ ವಸ್ತುಸಂಗ್ರಹಶಾಲೆಯಲ್ಲಿ ಅಮೃತಶಿಲೆಯಲ್ಲಿ ಕೆತ್ತಿದ ಒಂದು ಸುಂದರ ವಿಗ್ರಹವನ್ನು ನೋಡಿದೆ. ಆ ವಿಗ್ರಹದ ಕೆಳಗಡೆ ವಿವರಣೆಯಾಗಿ ಈ ಮಾತುಗಳು ಕೆತ್ತಲ್ಪಟ್ಟಿದ್ದವು: ‘ಕಲೆಯು ಪ್ರಕೃತಿಯನ್ನು ಅನಾವರಣ ಮಾಡುವುದು.’ ಅಂದರೆ ಕಲೆಯು ಪ್ರಕೃತಿ ಮೇಲೆ ಮುಸುಕಿರುವ ತೆರೆಯನ್ನು ತನ್ನ ಕೈಗಳಿಂದ ದೂರ ಸರಿಸಿ ಆಂತರಿಕ ಸೌಂದರ್ಯವನ್ನು ಕಾಣಿಸುತ್ತದೆ ಎಂದು. ಆ ಕೆಲಸ ಯಾವ ಭಾವನೆಯನ್ನು ವ್ಯಕ್ತಪಡಿಸುತ್ತಿತ್ತೆಂದರೆ ಪ್ರಕೃತಿಯ ಸೌಂದರ್ಯದ ಮೇಲೆ ಮುಸುಕಿರುವ ತೆರೆಯನ್ನಿನ್ನೂ ಪೂರ್ತಿ ಎಳೆದಿಲ್ಲ, ಆದರೆ ಕಲಾವಿದ ಎಷ್ಟು ಎಳೆದಿದ್ದಾನೋ ಅಷ್ಟರಿಂದಲೇ ಮುಗ್ಧನಾಗಿದ್ದಾನೆ ಎಂಬುದನ್ನು. ಈ ಮನೋಹರ ಭಾವನೆಯನ್ನು ವ್ಯಕ್ತಪಡಿಸಲು ಹೊರಟಿದ್ದ ಆ ಶಿಲ್ಪಿಯನ್ನು ಯಾರೂ ಹೊಗಳದಿರಲಾರರು. ನೀನೂ ಹಾಗೆಯೇ ಏನಾದರೂ ಮಂತ್ರತಂತ್ರ ಕಲ್ಪನೆಯಿಂದ ಮಾಡಬೇಕು."

ರನದಬಾಬು: ನನಗೂ ಹಾಗೆಯೇ ಸ್ವತಂತ್ರವಾಗಿ ವಿರಾಮವಾಗಿರುವಾಗ ಚಿತ್ರ ರಚಿಸಬೇಕೆಂಬ ಆಸೆಯಿದೆ. ಆದರೆ ಅದಕ್ಕೆ ಈ ದೇಶದಲ್ಲಿ ಸ್ವಲ್ಪವೂ ಪ್ರೋತ್ಸಾಹವೇ ಇಲ್ಲ. ಇದು ಬಡದೇಶ - ಪ್ರಶಂಸೆಯೇ ವಿರಳ.

ಸ್ವಾಮೀಜಿ: ನೀನು ಹೃದಯವನ್ನೆಲ್ಲಾ ಧಾರೆ ಎರೆದು ಒಂದು ಸ್ವತಂತ್ರ ಕೃತಿಯನ್ನು ರಚಿಸಬಲ್ಲೆಯಾದರೆ, ಕಲೆಯ ಒಂದು ಭಾಗಕ್ಕೆ ಪೂರ್ತಿ ಸೊಬಗು ಕೊಡಬಲ್ಲೆಯಾದರೆ, ಒಂದಲ್ಲ ಒಂದು ದಿನ ಅದು ಜನರ ಮೆಚ್ಚಿಗೆ ಪಡೆಯುವುದು. ಈ ಜಗತ್ತಿನಲ್ಲಿ ಸತ್ತ್ವವುಳ್ಳ ಯಾವುದೇ ಆಗಲಿ ಪ್ರಕಾಶಕ್ಕೆ ಬಂದೇ ತೀರುವುದು. ಎಷ್ಟೋ ಮಂದಿ ಕಲಾವಿದರ ಕೃತಿಗಳು ಅವರು ಸತ್ತು ಸಾವಿರಾರು ವರ್ಷಗಳಾದ ಮೇಲೆ ಬೆಳಕಿಗೆ ಬಂದುವೆಂದು ಕೇಳಿದ್ದೇವೆ.

ರನದಬಾಬು: ಅದು ನಿಜ. ಆದರೆ ನಾವು ನಿಷ್ಪ್ರಯೋಜಕರಾಗಿದ್ದೇವೆ. ವ್ಯರ್ಥವಾಗಿ ಅಷ್ಟೊಂದು ಶಕ್ತಿ ವ್ಯಯ ಮಾಡಲು ಧೈರ್ಯವಿಲ್ಲದವರಾಗಿದ್ದೇವೆ. ಐದು ವರ್ಷಗಳ ಪ್ರಯತ್ನದಿಂದ ನಾನು ಸ್ವಲ್ಪ ಜಯಶೀಲನಾಗಿರುವೆನು, ನನ್ನ ಪರಿಶ್ರಮ ಸಾರ್ಥಕವಾಗುವಂತೆ ಹರಸಿ.

ಸ್ವಾಮೀಜಿ: ನೀನು ಉತ್ಸಾಹಪೂರಿತನಾಗಿ ಕೆಲಸ ಮಾಡಿದರೆ ಖಂಡಿತ ಜಯ ಹೊಂದುವೆ. ಯಾರು ತ್ರಿಕರಣಪೂರ್ವಕ ಕೆಲಸ ಮಾಡುವರೋ ಅವರು ಜಯವೊಂದನ್ನೇ ಅಲ್ಲ, ಅದರಲ್ಲಿಟ್ಟಿರುವ ಏಕಾಗ್ರತೆಯಿಂದಾಗಿ, ಪರಮಾತ್ಮನನ್ನು ಕೂಡ ಸಾಕ್ಷಾತ್ಕರಿಸಿಕೊಳ್ಳಬಲ್ಲರು. ಯಾರು ಹೃತ್ಪೂರ್ವಕವಾಗಿ ತಮ್ಮ ಕೆಲಸ ಮಾಡುವರೋ ಅವರಿಗೆ ದೇವರು ಸಹಾಯ ಮಾಡುವನು.

ಶಿಷ್ಯ: ಹಿಂದೂದೇಶದ ಮತ್ತು ಪಾಶ್ಚಾತ್ಯ ದೇಶಗಳ ಕಲೆಗಳಲ್ಲಿ ನೀವು ಯಾವ ವ್ಯತ್ಯಾಸ ನೋಡಿದಿರಿ?

ಸ್ವಾಮೀಜಿ: ಎಲ್ಲಾ ಕಡೆಯೂ ಅದು ಸುಮಾರಾಗಿ ಒಂದೇ ತರಹ ಇದೆ. ಸ್ವಕಲ್ಪಿತ ಕೃತಿಗಳು ಬಹು ವಿರಳ. ಆ ದೇಶಗಳಲ್ಲಿ ಬಗೆಬಗೆಯ ವಸ್ತುಗಳ ಛಾಯಾ ಚಿತ್ರಗಳ ಪ್ರತಿಕೃತಿಯನ್ನು ಮುಂದಿಟ್ಟುಕೊಂಡು ಚಿತ್ರಗಳನ್ನು ಬರೆಯುವರು. ಯಾವಾಗ ಯಂತ್ರದ ಸಹಾಯವನ್ನು ತೆಗೆದುಕೊಳ್ಳುವರೋ ಆಗ ಕಲ್ಪನಾ ಶಕ್ತಿಯೇ ಮಾಯವಾಗುತ್ತದೆ. ಭಾವ ವ್ಯಕ್ತವಾಗಲು ಅವಕಾಶವೇ ಇರುವುದಿಲ್ಲ. ಪ್ರಾಚೀನ ಕಾಲದ ಕಲಾವಿದರು ಸ್ವತಂತ್ರ ಭಾವನೆಗಳನ್ನು ಚಿತ್ರದ ಮೂಲಕ ವ್ಯಕ್ತಪಡಿಸುತ್ತಿದ್ದರು. ಈಗ ವರ್ಣಚಿತ್ರಗಳು ಛಾಯಾ ಚಿತ್ರಗಳಂತೆಯೇ ಇದ್ದು, ಸ್ವಂತ ಕಲ್ಪನಾಶಕ್ತಿ, ಹೆಚ್ಚು ಪ್ರಯತ್ನ ಎಲ್ಲಾ ಕಡಿಮೆಯಾಗುತ್ತಿವೆ. ಆದರೆ ಪ್ರತಿಯೊಂದು ಜನಾಂಗವೂ ತನ್ನದೇ ಆದ ವೈಶಿಷ್ಟ್ಯವನ್ನು ಹೊಂದಿದೆ. ಅದರ ನಡೆ, ನುಡಿ, ಜೀವನದ ರೀತಿಗಳು ಈ ತೈಲ ಚಿತ್ರ ಮತ್ತು ಶಿಲ್ಪಕಲೆಗಳಲ್ಲಿ ತನ್ನದೇ ಆದೊಂದು ವಿಶೇಷ ವೈಶಿಷ್ಟ್ಯದೊಂದಿಗೆ ವ್ಯಕ್ತವಾಗುತ್ತವೆ. ಉದಾಹರಣೆಗೆ ಪಾಶ್ಚಾತ್ಯ ದೇಶದಲ್ಲಿ ಸಂಗೀತ ಮತ್ತು ನಾಟ್ಯಕಲೆಯ ಭಾವಗಳು ಬಹು ಸೂಕ್ಷ್ಮವಾಗಿರುತ್ತವೆ. ನಾಟ್ಯದಲ್ಲಿ ಅವರ ಅಂಗಾಂಗಗಳು ಮುರಿಯುವಂತೆ ಕಾಣುತ್ತವೆ. ವಾದ್ಯ ಸಂಗೀತದಲ್ಲಿ ಅದರ ಧ್ವನಿ ಭರ್ಜಿಯಂತೆ ನಮ್ಮ ಕಿವಿಯನ್ನಿರಿಯುತ್ತದೆ. ಹಾಗೇ ಹಾಡುಗಾರಿಕೆಯೂ ಕೂಡ. ಅದಕ್ಕೆ ವಿರುದ್ಧವಾಗಿ ನಮ್ಮ ದೇಶದಲ್ಲಿ ನಾಟ್ಯಕಲೆ ಅಲೆಅಲೆಯಾಗಿ ತೇಲುತ್ತಾ ಉರುಳಿ ಬಂದಂತೆ ಇರುವುದು. ಹಾಗೆಯೇ ಸಂಗೀತದ ವಿಧವಿಧದ ಉಚ್ಚ ಸ್ವರಗಳಲ್ಲಿ ಮತ್ತು ವಾದ್ಯ ಸಂಗೀತದಲ್ಲಿಯೂ ಕೂಡ. ಆದ್ದರಿಂದ ಕಲೆಯ ವಿಚಾರದಲ್ಲಿಯೂ ವಿವಿಧ ಜನರಲ್ಲಿ ವಿವಿಧ ಭಾವಗಳು ಸ್ಪುಟಗೊಳ್ಳುವುವು. ಯಾರು ತೀವ್ರ ಭೌತಿಕವಾದಿಗಳೊ ಅವರು ಪ್ರಕೃತಿಯನ್ನು ಗುರಿಯಾಗಿಟ್ಟುಕೊಂಡು ಅದಕ್ಕೆ ಸಂಬಂಧಪಟ್ಟಂತೆ ಕಲೆಯ ಭಾವವನ್ನು ವ್ಯಕ್ತಪಡಿಸುವರು. ಯಾರು ಪ್ರಕೃತಿಯನ್ನು ಮೀರಿದ ಅತೀಂದ್ರಿಯ ಸತ್ಯವನ್ನು ತಮ್ಮ ಗುರಿಯಾಗಿಟ್ಟುಕೊಂಡಿರುವರೋ ಅವರು ಪ್ರಕೃತಿ ಶಕ್ತಿಯ ಮೂಲಕ ಕಲೆಯಲ್ಲಿ ತಮ್ಮ ಭಾವವನ್ನು ವ್ಯಕ್ತಪಡಿಸುವರು. ಮೊದಲಿನ ವರ್ಗದ ಜನರಿಗೆ ಪ್ರಕೃತಿಯೇ ಕಲೆಯ ಮೂಲಾಧಾರ. ಎರಡನೆಯ ವರ್ಗದವರಿಗೆ ಭಾವನೆಯೇ ಕಲೆಯ ಬೆಳವಣಿಗೆಗೆ ಮುಖ್ಯ ಆಧಾರ. ಆದ್ದರಿಂದ ಕಲೆಯಲ್ಲಿ ಎರಡೂ ತಮ್ಮದೇ ಆದ ಹಾದಿಯಲ್ಲಿ ಮುಂದುವರಿದಿವೆ. ಪಾಶ್ಚಾತ್ಯರ ಕೇವಲ ತೈಲಚಿತ್ರಗಳನ್ನು ನೋಡಿದರೆ, ಅವು ಸಂಪೂರ್ಣ ಪ್ರಕೃತಿಯ ವಸ್ತುಗಳೆಂದು ಭ್ರಾಂತಿಪಡುವಂತಾಗುವುದು. ಈ ದೇಶದಲ್ಲಿಯೂ ನಮ್ಮ ಪೂರ್ವಿಕರು ಶಿಲ್ಪಕಲೆಯಲ್ಲಿ ಉಚ್ಚ ಶಿಖರವನ್ನೇರಿದರು. ಆಗಿನ ಕಾಲದ ಯಾವುದಾದರೊಂದು ವಿಗ್ರಹವನ್ನು ನೋಡಿದರೆ ಅದು ನಮ್ಮನ್ನು ಈ ಭೌತಿಕ ಪ್ರಪಂಚವನ್ನು ಮರೆಯುವಂತೆ ಮಾಡಿ ಒಂದು ಹೊಸ ಆದರ್ಶ ಪ್ರಪಂಚಕ್ಕೊಯ್ಯುವುದು. ಈಗ ಪಶ್ಚಿಮ ದೇಶಗಳಲ್ಲಿ ಹಿಂದಿನ ಕಾಲದ ತೈಲ ಚಿತ್ರಗಳನ್ನು ಚಿತ್ರಿಸಲಾಗುವುದಿಲ್ಲ. ಅದರಂತೆ ಈ ದೇಶದಲ್ಲಿಯೂ ತಮ್ಮದೇ ಆದ ಸ್ವತಂತ್ರ ಭಾವಗಳನ್ನು ವ್ಯಕ್ತಪಡಿಸುವುದೂ ಕಂಡುಬರುವುದಿಲ್ಲ. ಉದಾಹರಣೆಗೆ ನಿಮ್ಮ ಕಲಾಶಾಲೆಯ ತೈಲಚಿತ್ರಗಳು ಭಾವಶೂನ್ಯ. ಅದಕ್ಕೆ ಬದಲು ನೀವು ಹಿಂದೂಗಳ ನಿತ್ಯ ಜೀವನದ ವಸ್ತುಗಳನ್ನು ಅವಕ್ಕೆ ಪೂರ್ವಿಕರ ಆದರ್ಶವನ್ನೆರೆದು ಚಿತ್ರಿಸಲು ಪ್ರಯತ್ನ ಪಡಿ.

ರನದಬಾಬು: ನಿಮ್ಮ ಮಾತುಗಳಿಂದ ನನಗೆ ಬಹಳ ಉತ್ತೇಜನ ದೊರಕಿದಂತಾಗಿದೆ. ನಿಮ್ಮ ಸಲಹೆಯಂತೆ ನಡೆಯಲು ಪ್ರಯತ್ನ ಪಡುತ್ತೇನೆ.

ಸ್ವಾಮೀಜಿ: ಉದಾಹರಣೆಗೆ ಕಾಳಿಕಾಮಾತೆಯ ಮೂರ್ತಿಯನ್ನೇ ತೆಗೆದುಕೊಳ್ಳಿ. ಅದರಲ್ಲಿ ಆನಂದ ಹಾಗೂ ಭಯಂಕರಭಾವಗಳ ಮಿಲನವಿದೆ. ಆದರೆ ಯಾವ ತೈಲಚಿತ್ರದಲ್ಲೂ ಈ ಭಾವಗಳ ಸ್ಪಷ್ಟ ಪ್ರದರ್ಶನವಿಲ್ಲ. ಅದೊಂದೇ ಅಲ್ಲ, ಯಾವ ಒಂದು ಭಾವವನ್ನೂ ಸರಿಯಾಗಿ ನಿರೂಪಿಸುವುದಿಲ್ಲ. ನಾನು ನನ್ನ “ಕಾಳಿಮಾತೆ" ಎಂಬ ಆಂಗ್ಲ ಪದ್ಯದಲ್ಲಿ ಕಾಳಿಮಾತೆಯ ಕೆಲವು ಉಗ್ರಭಾವವನ್ನು ಕೊಡಲು ಯತ್ನಿಸಿದ್ದೇನೆ. ನೀವು ಆ ಭಾವನೆಗಳನ್ನು ಒಂದು ಚಿತ್ರದಲ್ಲಿ ವ್ಯಕ್ತಪಡಿಸಬಲ್ಲಿರಾ?

ರನದಬಾಬು: ದಯವಿಟ್ಟು ನನಗೆ ಅದನ್ನು ತಿಳಿಸಿ.

ಸ್ವಾಮಿಗಳು ಪುಸ್ತಕ ಸಂಗ್ರಹಾಲಯದಿಂದ ಆ ಪದ್ಯವನ್ನು ತಂದು ರನದಬಾಬುಗಳ ಮನಸ್ಸಿಗೆ ನಾಟುವಂತೆ ಓದತೊಡಗಿದರು. ರನದಬಾಬು ಮೌನವಾಗಿ ಅದನ್ನು ಕೇಳಿದರು. ಸ್ವಲ್ಪಕಾಲದ ನಂತರ ಆ ರೂಪವನ್ನು ತಮ್ಮ ಮಾನಸ ಚಕ್ಷುಗಳಿಂದ ದರ್ಶಿಸಿದರೋ ಎಂಬಂತೆ ಅವರು ಸ್ವಾಮಿಗಳ ಕಡೆ ಭಯಚಕಿತ ಕಣ್ಣುಗಳಿಂದ ನೋಡಿದರು.

ಸ್ವಾಮೀಜಿ: ನೀವೀ ಭಾವವನ್ನು ಚಿತ್ರದಲ್ಲಿ ವ್ಯಕ್ತಪಡಿಸಲು ಸಾಧ್ಯವೆ.

ರನದಬಾಬು: ಆಗಲಿ, ನಾನು ಪ್ರಯತ್ನಿಸುತ್ತೇನೆ. ಆ ಭಾವವನ್ನು ಕೇವಲ ಕಲ್ಪಿಸಿಕೊಂಡ ಮಾತ್ರದಿಂದಲೇ ತಲೆ ತಿರುಗಿಹೋಗುವುದು ಅನ್ನಿಸುತ್ತದೆ.

ಸ್ವಾಮೀಜಿ: ಚಿತ್ರ ಪೂರೈಸಿದ ಮೇಲೆ ದಯವಿಟ್ಟು ನನಗೆ ತೋರಿಸಿ, ಅದನ್ನು ಪೂರ್ತಿಗೊಳಿಸಲು ಬೇಕಾಗುವ ಕೆಲವು ಸಲಹೆಗಳನ್ನು ನಾನು ಹೇಳುವೆ.

ನಂತರ ಸ್ವಾಮೀಜಿ, ತಾವು ಗುರುತುಹಾಕಿಟ್ಟಿದ್ದ ಶ‍್ರೀರಾಮಕೃಷ್ಣ ಸಂಸ್ಥೆಯ ಚಿಹ್ನೆಯ ನಕಾಶೆಯನ್ನು ತರಿಸಿ ರನದಬಾಬುವಿಗೆ ತೋರಿಸಿ ಅವರ ಅಭಿಪ್ರಾಯವನ್ನು ಕೇಳಿದರು. ಅದರಲ್ಲಿ ತಾವರೆ ಅರಳಿರುವ ಸರೋವರವೂ ಒಂದು ಹಂಸವೂ ಅದರ ಸುತ್ತು ಬಳಸಿ ಹೊರಗೆ ಒಂದು ಸರ್ಪವೂ ಚಿತ್ರಿತವಾಗಿತ್ತು. ರನದಬಾಬುವಿಗೆ ಮೊದಲು ಅದರ ಭಾವನೆ ಅರ್ಥವಾಗಲಿಲ್ಲ. ಸ್ವಾಮಿಗಳನ್ನು ವಿವರಿಸಲು ಕೇಳಿದರು. ಸ್ವಾಮಿಗಳು ಹೇಳಿದರು: “ಸರೋವರದ ತರಂಗ ಮಾಲೆಗಳು ಕರ್ಮವನ್ನೂ, ತಾವರೆ ಭಕ್ತಿಯನ್ನೂ, ಮೂಡುತ್ತಿರುವ ಸೂರ್ಯ ಜ್ಞಾನವನ್ನೂ ಸೂಚಿಸುತ್ತವೆ. ಸುತ್ತಲೂ ಬಳಸಿರುವ ಸರ್ಪವು ಜಾಗೃತವಾದ ಕುಂಡಲಿನಿ ಶಕ್ತಿಯನ್ನೂ ಹಂಸವು ಪರಮಾತ್ಮನನ್ನೂ ಸೂಚಿಸುವುವು. ಯಾವಾಗ ಯೋಗ ಜ್ಞಾನ ಭಕ್ತಿ ಕರ್ಮಗಳು ಒಂದಾಗುವುವೋ ಆಗ ಪರಮಾತ್ಮನ ಸಾಕ್ಷಾತ್ಕಾರವಾಗುವುದೆಂಬ ಭಾವನೆಯನ್ನು ಈ ಚಿತ್ರ ಸೂಚಿಸುವುದು.”

ರನದಬಾಬುಗಳು ಚಿತ್ರದ ವ್ಯಾಖ್ಯಾನವನ್ನು ಕೇಳಿ ಕೃತಜ್ಞತೆಯಿಂದ ಕೊಂಚ ಹೊತ್ತು ಮೌನವಾಗಿದ್ದರು. ನಂತರ ಅವರು “ನಿಮ್ಮಿಂದ ಕಲೆಯ ವಿಚಾರವನ್ನು ತಿಳಿಯಬೇಕೆಂಬ ಆಸೆಯಾಗುವುದು" ಎಂದರು.

ನಂತರ ಸ್ವಾಮೀಜಿ ರನದಬಾಬುಗಳಿಗೆ ತಾವು ಕಟ್ಟಿಸಲಿರುವ ಶ‍್ರೀರಾಮಕೃಷ್ಣರ ದೇವಾಲಯ ಮತ್ತು ಮಠದ ಕಟ್ಟಡದ ನಕಾಶೆಯನ್ನು ತೋರಿಸಿದರು. ನಂತರ ಅವರು ಹೀಗೆ ಹೇಳತೊಡಗಿದರು: “ಈ ಭಾವೀ ದೇವಾಲಯ ಮತ್ತು ಮಠದ ಕಟ್ಟಡಗಳಲ್ಲಿ ನಾನು ಪೂರ್ವ ಮತ್ತು ಪಶ್ಚಿಮದ ವಾಸ್ತುಶಿಲ್ಪ ಕಲೆಗಳ ಅತ್ಯುತ್ತಮವಾದ ಅಂಶಗಳನ್ನು ಸೇರಿಸಬೇಕೆಂದಿದ್ದೇನೆ. ನಾನು ಇಡೀ ಜಗತ್ತನ್ನು ಸುತ್ತಿದುದರ ಪರಿಣಾಮವಾಗಿ ನನಗೆ ಗೊತ್ತಾಗಿರುವ ಶಿಲ್ಪಕಲೆಯ ಎಲ್ಲಾ ಭಾವನೆಗಳನ್ನೂ ಅದರ ರಚನೆಯಲ್ಲಿ ರೂಪಿಸಬೇಕೆಂದಿದ್ದೇನೆ. ಅಸಂಖ್ಯಾತ ಕಂಬಗಳ ಸಮುದಾಯದ ಆಧಾರದ ಮೇಲೆ ನಿಂತಿರುವ ಒಂದು ದೊಡ್ಡ ಪ್ರಾರ್ಥನಾ ಮಂದಿರ ಕಟ್ಟಲ್ಪಡುವುದು. ಅದರ ಗೋಡೆಗಳ ಮೇಲೆ ನೂರಾರು ಅರಳಿರುವ ತಾವರೆಗಳು ಕೊರೆಯಲ್ಪಡುವುವು. ಆ ಮಂದಿರವು ಸಾವಿರ ಮಂದಿ ಕುಳಿತು ಧ್ಯಾನಮಾಡಲು ಸಾಕಾಗುವಂತಿರಬೇಕು. ಈ ರಾಮಕೃಷ್ಣ ದೇವಾಲಯ ಮತ್ತು ಪ್ರಾರ್ಥನಾ ಮಂದಿರವು ಯಾವ ರೀತಿ ಕಟ್ಟಲ್ಪಡಬೇಕೆಂದರೆ ದೂರದಿಂದ ಅದನ್ನು ನೋಡಿದರೆ ಅದು "ಓಂ“ ಎಂಬ ಚಿಹ್ನೆಯ ಪ್ರತಿಬಿಂಬವಾಗಿರುವಂತೆ ಕಾಣಬೇಕು. ದೇವಾಲಯದೊಳಗಡೆ ಹಂಸದ ಮೇಲೆ ಕುಳಿತಿರುವ ಶ‍್ರೀರಾಮಕೃಷ್ಣರ ಪ್ರತಿಮೆ ಇರಬೇಕು. ಬಾಗಿಲಿನ ಇಕ್ಕೆಡೆಯಲ್ಲೂ ಒಂದು ಸಿಂಹ ಮತ್ತು ಕುರಿಮರಿ ಪ್ರೇಮದಿಂದ ಒಂದನ್ನೊಂದು ನೆಕ್ಕುತ್ತಿರುವಂತೆ ಚಿತ್ರಿಸಬೇಕು - ಮಹಾಶಕ್ತಿ ಮತ್ತು ಸಾಧುಸ್ವಭಾವ ಪ್ರೇಮದಲ್ಲಿ ಮಿಳಿತವಾಗಿರುವ ಭಾವ ಮೂಡಿರಬೇಕು. ನನ್ನ ಮನಸ್ಸಿನಲ್ಲಿ ಈ ಉದ್ದೇಶಗಳಿವೆ. ನಾನು ಸಾಕಷ್ಟು ದಿನ ಜೀವಿಸಿದ್ದರೆ ಇವುಗಳನ್ನು ಕಾರ್ಯರೂಪಕ್ಕೆ ತಂದೇ ತೀರುವೆ. ಇಲ್ಲದಿದ್ದಲ್ಲಿ ಮುಂದಿನವರು ಕ್ರಮೇಣ ಇವುಗಳನ್ನು ಕಾರ್ಯರೂಪಕ್ಕೆ ತರಲು ಪ್ರಯತ್ನಿಸುವರು. ನನ್ನ ಅಭಿಪ್ರಾಯವೇನೆಂದರೆ, ನಮ್ಮ ದೇಶದ ಕಲೆ ಮತ್ತು ಸಂಸ್ಕೃತಿಯ ಎಲ್ಲಾ ಶಾಖೆಗಳನ್ನೂ ಪುನರುಜ್ಜೀವನಗೊಳಿಸಲೆಂದು ಶ‍್ರೀರಾಮಕೃಷ್ಣರು ಅವತರಿಸಿದ್ದು. ಆದ್ದರಿಂದ ಧರ್ಮ, ಕರ್ಮ, ಶಿಕ್ಷಣ, ಜ್ಞಾನ, ಭಕ್ತಿ ಇವೆಲ್ಲಾ ಈ ಕೇಂದ್ರದಿಂದ ಜಗತ್ತಿಗೆಲ್ಲಾ ಹರಡುವಂತಹ ರೀತಿಯಲ್ಲಿ ಈ ಮಠ ಕಟ್ಟಲ್ಪಡಬೇಕು. ನೀವೆಲ್ಲಾ ಈ ಕೆಲಸಕ್ಕೆ ನನ್ನ ಸಹಾಯಕರಾಗಬೇಕು."

ರನದಬಾಬುಗಳು, ಅಲ್ಲಿ ಸೇರಿದ್ದ ಸಂನ್ಯಾಸಿಗಳು ಮತ್ತು ಬ್ರಹ್ಮಚಾರಿಗಳು, ಎಲ್ಲರೂ ಸ್ವಾಮಿಗಳ ಮಾತುಗಳನ್ನು ಮೌನಮುಗ್ಧರಾಗಿ ಕೇಳುತ್ತಿದ್ದರು. ಸ್ವಲ್ಪ ಕಾಲಾನಂತರ ಸ್ವಾಮೀಜಿ ಮುಂದುವರಿಸಿದರು: ನೀವು ಈ ಹಾದಿಯಲ್ಲಿ ಪ್ರವೀಣರಾದುದರಿಂದ ನಿಮ್ಮೊಡನೆ ಈ ವಿಷಯವನ್ನು ಅಷ್ಟು ದೀರ್ಘವಾಗಿ ಚರ್ಚಿಸುತ್ತಿದ್ದೇನೆ. ದಯವಿಟ್ಟು ಈಗ ನನಗೆ ಹೇಳಿ. ನೀವು ಇಷ್ಟು ದೀರ್ಘವಾಗಿ ಕಲೆಯನ್ನು ಅಭ್ಯಸಿಸಿರುವಿರಲ್ಲ, ಕಲೆಯ ಅತ್ಯುಚ್ಚ ಧ್ಯೇಯಗಳಾವುವೆಂಬುದರ ಬಗ್ಗೆ ಏನು ಕಲಿತಿರುವಿರಿ?

ರನದಬಾಬು: ನಿಮಗೆ ನಾನು ಆವ ಹೊಸ ವಿಷಯ ಹೇಳಲಿ? ಅದಕ್ಕೆ ಬದಲಾಗಿ ಈ ವಿಷಯದಲ್ಲಿ ನೀವೇ ನನ್ನ ಕಣ್ಣನ್ನು ತೆರೆಸಿರುವಿರಿ. ಕಲೆಯ ವಿಚಾರವಾಗಿ ಇಷ್ಟೊಂದು ಬೋಧಪ್ರದವಾದ ಮಾತುಗಳನ್ನು ನನ್ನ ಜೀವಮಾನದಲ್ಲೇ ಎಂದೂ ಕೇಳಿರಲಿಲ್ಲ. ಸ್ವಾಮೀಜಿ, ದಯವಿಟ್ಟು ನಿಮ್ಮಿಂದ ಪಡೆದ ಭಾವನೆಗಳನ್ನು ಕಾರ್ಯರೂಪಕ್ಕೆ ತರುವಂತೆ ನನ್ನನ್ನು ಹರಸಿ.

ನಂತರ ಸ್ವಾಮಿಗಳು ತಮ್ಮ ಪೀಠದಿಂದೆದ್ದು ಅಂಗಳದಲ್ಲಿ ಸುತ್ತುತ್ತಾ ಶಿಷ್ಯನಿಗೆ ಹೇಳಿದರು, “ಆ ಯುವಕ ಬಹಳ ಉತ್ಸಾಹಿ.”

ಶಿಷ್ಯ: ನಿಮ್ಮ ಮಾತುಗಳನ್ನು ಕೇಳಿ ಅವನಿಗೆ ಆಶ್ಚರ್ಯವಾಗಿದೆ.

ಸ್ವಾಮೀಜಿ ಶಿಷ್ಯನಿಗೆ ಉತ್ತರಕೊಡದೆ ಶ‍್ರೀರಾಮಕೃಷ್ಣರು ಹಾಡುತ್ತಿದ್ದ ಕೆಲವು ಹಾಡುಗಳನ್ನು ಮೆಲುಧ್ವನಿಯಲ್ಲಿ ಹಾಡತೊಡಗಿದರು: ಅಂಕೆಯಲ್ಲಿರುವ ಮನಸ್ಸು ಸ್ಪರ್ಶಮಣಿಯಂತೆ, ನೀನಾವುದನ್ನು ಇಚ್ಛಿಸುವೆಯೋ ಅದನ್ನೆಲ್ಲಾ ಕೊಡುವುದು.

ಕೊಂಚದೂರ ನಡೆದ ಮೇಲೆ ಸ್ವಾಮೀಜಿ ಮುಖ ತೊಳೆದು ಶಿಷ್ಯನೊಡನೆ ತಮ್ಮ ಕೊಠಡಿಯನ್ನು ಪ್ರವೇಶಿಸಿದರು. “ಎನ್‌ಸೈಕ್ಲೋಪೀಡಿಯಾ ಬ್ರಿಟಾನಿಕಾ" ಎಂಬ ಪುಸ್ತಕದಲ್ಲಿದ್ದ ಕಲೆಯ ಮೇಲಿನ ಒಂದು ಲೇಖನವನ್ನು ಓದಿದರು. ಅದನ್ನು ಮುಗಿಸಿದ ಮೇಲೆ ಶಿಷ್ಯನ ಪೂರ್ವ ಬಂಗಾಳದ ಪದಗಳ ಉಚ್ಚಾರಣೆಗಳನ್ನು ಕುರಿತು ಗೇಲಿ ಮಾಡತೊಡಗಿದರು.

\newpage

\chapter[ಅಧ್ಯಾಯ ೩೪]{ಅಧ್ಯಾಯ ೩೪\protect\footnote{\engfoot{C.W, Vol. VII, P. 206}}}

\begin{center}
ಸ್ಥಳ: ಬೇಲೂರು ಮಠ, ವರ್ಷ ಕ್ರಿ.ಶ. ೧೯೦೧.
\end{center}

ಸ್ವಾಮಿಜಿ ಪೂರ್ವ ಬಂಗಾಳ ಮತ್ತು ಅಸ್ಸಾಂನಿಂದ ಹಿಂತಿರುಗಿ ಕೆಲವು ದಿನಗಳಾಗಿತ್ತು. ಅವರಿಗೆ ಖಾಯಿಲೆಯಾಗಿ ಕಾಲುಗಳು ಊದಿದ್ದುವು. ಮಠಕ್ಕೆ ಬಂದ ಶಿಷ್ಯ ಸ್ವಾಮಿಗಳ ಪಾದಕ್ಕೆ ಸಾಷ್ಟಾಂಗ ಪ್ರಣಾಮ ಮಾಡಿದನು. ಖಾಯಿಲೆಯಲ್ಲಿದ್ದರೂ ಸ್ವಾಮೀಜಿ ತಮ್ಮ ಸಹಜವಾದ ನಗುಮುಖ ಕರುಣಾಪೂರಿತ ದೃಷ್ಟಿಯಿಂದಲೇ ಕೂಡಿದ್ದರು.

ಶಿಷ್ಯ: ಸ್ವಾಮಿಜಿ, ಈಗ ಹೇಗಿದ್ದೀರಿ?

ಸ್ವಾಮಿಜಿ: ನನ್ನ ಆರೋಗ್ಯದ ವಿಷಯ ಏನೆಂದು ಹೇಳಲಿ ಮಗು? ದಿನ ದಿನಕ್ಕೆ, ಈ ಶರೀರ ಕೆಲಸ ಮಾಡಲನರ್ಹವಾಗುತ್ತಿದೆ. ವಂಗಭೂಮಿಯಲ್ಲಿ ಹುಟ್ಟಿದ ದೇಹ. ಯಾವುದಾದರೊಂದು ರೋಗ ಯಾವಾಗಲೂ ಅದನ್ನು ಕಾಡುತ್ತಲೇ ಇರುತ್ತದೆ. ಶರೀರ ಪ್ರಕೃತಿ ಕೊಂಚವೂ ಚೆನ್ನಾಗಿಲ್ಲ, ನಾನಾವುದಾದರೂ ಕಷ್ಟವಾದ ಕೆಲಸ ಮಾಡಬೇಕಾದರೆ ಅದರ ಪ್ರಯಾಸವನ್ನು ಸಹಿಸಲಾಗುವುದಿಲ್ಲ. ಆದರೆ ಈ ದೇಹವಿರುವವರೆಗೂ ನಾನು ಕೆಲಸ ಮಾಡೇತೀರುವೆ. ಕರ್ಮ ಪ್ರಪಂಚದಲ್ಲಿ ದುಡಿಯುತ್ತಲೇ ಸಾಯುವೆ.

ಶಿಷ್ಯ: ನೀವು ಕೊಂಚಕಾಲ ಕೆಲಸ ಮಾಡುವುದನ್ನು ಬಿಟ್ಟು ವಿಶ್ರಾಂತಿ ತೆಗೆದು ಕೊಂಡರೆ ಆರೋಗ್ಯ ಹೊಂದುವಿರಿ. ನಿಮ್ಮ ಬದುಕಿನಿಂದ ಜಗತ್ಕಲ್ಯಾಣವಾಗುವುದು.

ಸ್ವಾಮಿಜಿ: ನಾನು ಸುಮ್ಮನೆ ಮೌನದಲ್ಲಿ ಕುಳಿತಿರಲು ಸಾಧ್ಯವೇ ಮಗು? ಶ‍್ರೀರಾಮಕೃಷ್ಣರು ನಿರ್ಯಾಣವಾಗಲು ಎರಡು ಮೂರು ದಿನ ಮೊದಲು ಅವರು ಯಾರನ್ನು “ಕಾಳಿ" ಎನ್ನುತ್ತಿದ್ದರೋ ಆ ದೇವಿ ಈ ದೇಹ ಪ್ರವೇಶ ಮಾಡಿದಳು. ಆಕೆಯೇ ನನ್ನನ್ನು ಸುಮ್ಮನಿರಿಸದೆ, ನನ್ನ ಸ್ವಂತ ಆರೋಗ್ಯದ ಕಡೆ ಕೂಡ ಗಮನ ಕೊಡದಂತೆ ಅಲ್ಲಿ ಇಲ್ಲಿ ಎಲ್ಲೆಡೆಗೂ ನನ್ನನ್ನು ಕರೆದೊಯ್ದು ಕೆಲಸ ಮಾಡಿಸುತ್ತಿರುವಳು.

ಶಿಷ್ಯ: ನೀವು ಇದನ್ನು ರೂಪಕ ದೃಷ್ಟಿಯಿಂದ ಹೇಳುತ್ತಿದ್ದೀರೇನು?

ಸ್ವಾಮೀಜಿ: ಹಾಗಲ್ಲ. ಮಹಾ ಸಮಾಧಿಗೆ ಮೂರು ದಿನ ಮೊದಲು ಅವರು ನನ್ನನ್ನು ತಮ್ಮ ಪಕ್ಕಕ್ಕೆ ಕರೆದು ತಮ್ಮೆದುರು ಕುಳ್ಳಿರಲು ಹೇಳಿ ಎವೆಯಿಕ್ಕದೆ ನನ್ನನ್ನೇ ನೋಡುತ್ತಾ ಸಮಾಧಿಸ್ಥರಾದರು. ಆಗ ನಿಜವಾಗಿಯೂ ನನ್ನ ದೇಹದಲ್ಲಿ ವಿದ್ಯುತ್ ಪ್ರವಾಹ ಸಂಚರಿಸಿದಂತಾಯ್ತು. ಕೊಂಚ ಹೊತ್ತಿನಲ್ಲಿಯೇ ನನಗೂ ಬಾಹ್ಯಪ್ರಜ್ಞೆ ತಪ್ಪಿದಂತಾಗಿ ಸ್ಥಿರವಾಗಿ ಕುಳಿತುಕೊಂಡೆ. ಹಾಗೇ ಎಷ್ಟು ಹೊತ್ತು ಕುಳಿತಿದ್ದೆನೋ ನನಗೆ ಗೊತ್ತಿಲ್ಲ. ನನಗೆ ಪ್ರಜ್ಞೆ ಬಂದಾಗ ಶ‍್ರೀರಾಮಕೃಷ್ಣರು ಕಂಬನಿಗರೆಯುತ್ತಿದ್ದುದನ್ನು ಕಂಡೆ. ಕಾರಣವೇನೆಂದು ಪ್ರಶ್ನಿಸಿದಾಗ ಅವರು ವಾತ್ಸಲ್ಯದಿಂದ ಹೇಳಿದರು, “ಇಂದು ನಿನಗೆ ನನ್ನ ಸರ್ವಸ್ವವನ್ನೂ ದಾನ ಮಾಡಿದ್ದೇನೆ. ನಾನಿಂದು ಒಬ್ಬ ದರಿದ್ರ ಫಕೀರ. ನೀನು ಹಿಂತಿರುಗುವ ಮುಂಚೆ ಈ ಶಕ್ತಿಯಿಂದ ಲೋಕಕ್ಕೆ ಮಹತ್ತಾದ ಉಪಕಾರವನ್ನು ಮಾಡಬೇಕು." ಆ ಶಕ್ತಿಯೇ ಈಗ ನನ್ನನ್ನು ಈ ಕೆಲಸವನ್ನೆಲ್ಲಾ ಮಾಡಲು ಪ್ರೇರೇಪಿಸುತ್ತಿದೆ ಎಂದು ಅನುಭವವಾಗುತ್ತಿದೆ. ಈ ದೇಹ ಕೇವಲ ಸೋಮಾರಿತನಕ್ಕಾಗಿ ಹುಟ್ಟಿ ಬರಲಿಲ್ಲ.

ಈ ಮಾತನ್ನು ಕೇಳಿ ವಿಸ್ಮಯಪಟ್ಟ ಶಿಷ್ಯ ಯೋಚಿಸಿದ: ‘ಸಾಧಾರಣ ಜನರು ಈ ವಿಷಯವನ್ನು ಹೇಗೆ ತೆಗೆದುಕೊಳ್ಳುವರೆಂಬುದು ಯಾರಿಗೆ ಗೊತ್ತು?’ ಆ ಕೂಡಲೇ ಅವನು ವಿಚಾರವನ್ನು ಬದಲಾಯಿಸಿ ಕೇಳಿದ, “ಸ್ವಾಮೀಜಿ, ನೀವು ಪೂರ್ವ ಬಂಗಾಳವನ್ನು ಇಷ್ಟಪಟ್ಟಿರಾ?”

ಸ್ವಾಮೀಜಿ: ಒಟ್ಟಿನಲ್ಲಿ ಇಷ್ಟಪಟ್ಟೆ. ನಾನು ಹೋದಾಗ ಹೊಲದಲ್ಲಿ ಬೆಳೆಗಳು ಹಚ್ಚನೆ ಹಸುರಾಗಿದ್ದುವು. ಹವಾ ಕೂಡ ಚೆನ್ನಾಗಿತ್ತು. ಬೆಟ್ಟದ ಮೇಲಿನ ಪ್ರಕೃತಿ ಸೌಂದರ್ಯ ಮನೋಹರವಾಗಿದೆ. ಬ್ರಹ್ಮಪುತ್ರಾ ನದಿಯ ಕಣಿವೆಯು ಸೌಂದರ್ಯದಲ್ಲಿ ಅಸದೃಶವಾದುದು. ಪೂರ್ವ ಬಂಗಾಳದ ಜನರು ಇಲ್ಲಿಯ ಜನರಿಗಿಂತ ಹೆಚ್ಚು ದೃಢಕಾಯರು, ಉತ್ಸಾಹಿಗಳು. ಬಹುಶಃ ಅವರು ಮೀನು ಮಾಂಸ ತಿನ್ನುವುದರಿಂದ ಹಾಗಿದ್ದಿರಬಹುದು. ಅವರು ಹಿಡಿದ ಕೆಲಸವನ್ನು ಮಾಡೇ ತೀರುವರು. ಅವರು ಆಹಾರದಲ್ಲಿ ಹೆಚ್ಚು ಎಣ್ಣೆ ಮತ್ತು ಕೊಬ್ಬನ್ನು ಉಪಯೋಗಿಸುವರು. ಇದು ಒಳ್ಳೆಯದಲ್ಲ. ಎಣ್ಣೆ ಮತ್ತು ಕೊಬ್ಬನ್ನು ಹೆಚ್ಚಾಗಿ ಸೇವಿಸುವುದರಿಂದ ದೇಹ ಸ್ಥೂಲವಾಗುವುದು.

ಶಿಷ್ಯ: ಅವರಲ್ಲಿ ಧಾರ್ಮಿಕ ಪ್ರವೃತ್ತಿ ಹೇಗಿದೆ?

ಸ್ವಾಮೀಜಿ: ಧಾರ್ಮಿಕ ಭಾವನೆಗಳ ವಿಚಾರವಾಗಿ ಹೇಳಬೇಕೆಂದರೆ ಅಲ್ಲಿನ ಜನರು ತೀವ್ರವಾದ ಪೂರ್ವಾಚಾರಪ್ರಿಯರು. ಧರ್ಮದಲ್ಲಿ ಉದಾರ ಭಾವ ಹೋಗಿ ಅನೇಕರು ಮತಭ್ರಾಂತರಾಗಿರುವರು. ಒಮ್ಮೆ ಡಾಕ್ಕಾದಲ್ಲಿ ಮೋಹಿನಿಬಾಬುಗಳ ಮನೆಯಲ್ಲಿದ್ದಾಗ ಒಬ್ಬ ತರುಣನು ಒಂದು ಭಾವಚಿತ್ರವನ್ನು ತಂದು ‘ಸ್ವಾಮಿ, ದಯವಿಟ್ಟು ಹೇಳಿ - ಈ ಚಿತ್ರದಲ್ಲಿರುವವನಾರು? ಅವನು ಅವತಾರಪುರುಷನೆ?’ ಎಂದ. ನಾನು ನಯವಾಗಿ ನನಗೆ ಅವರ ವಿಷಯ ಏನೂ ಗೊತ್ತಿಲ್ಲವೆಂದು ಹೇಳಿದೆ. ಮೂರು ನಾಲ್ಕು ಬಾರಿ ಹೇಳಿದರೂ ಆ ಮನುಷ್ಯ ಪದೇ ಪದೇ ನನ್ನನ್ನು ಪ್ರಶ್ನಿಸುವುದನ್ನು ಬಿಡಲಿಲ್ಲ. ಕಡೆಗೊಮ್ಮೆ ವಿಧಿಯಿಲ್ಲದೆ ನಾನು ಹೇಳಿದೆ ‘ಮಗು, ಇನ್ನು ಮುಂದೆ ಕೊಂಚ ಪುಷ್ಟಿಕರವಾದ ಆಹಾರವನ್ನು ತೆಗೆದುಕೊ, ನಿನ್ನ ಮೆದುಳು ನಂತರ ಬಲಿಷ್ಠವಾಗುವುದು. ಪುಷ್ಟಿಕರವಾದ ಆಹಾರದ ಅಭಾವದಿಂದ ನಿನ್ನ ಮೆದುಳು ಬತ್ತಿಹೋಗಿದೆ.’ ಈ ಮಾತುಗಳಿಂದ ಆ ತರುಣ ಅಸಂತುಷ್ಟನಾಗಿರಬೇಕು. ಆದರೆ ನಾನೇನು ಮಾಡಲಿ? ನಾನೀರೀತಿ ಯುವಕರಿಗೆ ಹೇಳದಿದ್ದರೆ ಅವರು ಸ್ವಲ್ಪದರಲ್ಲೇ ಹುಚ್ಚರಾಗಿಬಿಡುವರು.

ಶಿಷ್ಯ: ನಮ್ಮ ಪೂರ್ವ ಬಂಗಾಳದಲ್ಲಿ ಈಚೀಚೆಗೆ ಅನೇಕ ಅವತಾರಗಳು ತಲೆಯೆತ್ತಿಕೊಂಡಿವೆ.

ಸ್ವಾಮೀಜಿ: ಜನರು ತಮ್ಮ ಗುರುವನ್ನು ಅವತಾರಪುರುಷರೆಂದು ಕರೆಯಬಹುದು. ಅವರ ವಿಷಯದಲ್ಲಿ ಅವರಿಷ್ಟಬಂದ ಭಾವನೆ ಹೊಂದಿರಬಹುದು. ಆದರೆ ಎಲ್ಲೆಂದರಲ್ಲಿ, ಹೊತ್ತುಗೊತ್ತಿಲ್ಲದೆ ಭಗವಂತನ ಅವತಾರ ಆಗುವುದಿಲ್ಲ. ಡಾಕ್ಕಾದಲ್ಲೇ ಮೂರುನಾಲ್ಕು ಅವತಾರಗಳಿವೆ ಎಂದು ಕೇಳಿದೆ.

ಶಿಷ್ಯ: ಅಲ್ಲಿನ ಸ್ತ್ರೀಯರ ವಿಷಯದಲ್ಲಿ ನಿಮ್ಮ ಅಭಿಪ್ರಾಯವೇನು?

ಸ್ವಾಮೀಜಿ: ಸ್ತ್ರೀಯರು ಸಾಮಾನ್ಯವಾಗಿ ಎಲ್ಲಾ ಕಡೆಯೂ ಒಂದೇ ಬಗೆಯಾಗಿರುವರು. ಡಾಕ್ಕಾದಲ್ಲಿ ವೈಷ್ಣವಮತವು ಪ್ರಧಾನವಾಗಿರುವುದು ಕಂಡುಬಂತು. ಹ-ರ ಹೆಂಡತಿ ಬಹಳ ಜಾಣೆಯಂತೆ ಕಾಣಿಸಿದಳು. ಬಹು ಎಚ್ಚರಿಕೆಯಿಂದ ಆಕೆ ನನ್ನ ಆಹಾರವನ್ನು ಸಿದ್ಧಪಡಿಸಿ ಕಳುಹಿಸುತ್ತಿದ್ದಳು.

ಶಿಷ್ಯ: ನೀವು ನಾಗಮಹಾಶಯರ ಸ್ಥಳಕ್ಕೆ ಹೋಗಿದ್ದೀರೆಂದು ಕೇಳಿದೆ.

ಸ್ವಾಮೀಜಿ: ಹೌದು, ಅಷ್ಟು ದೂರ ಹೋದ ಮೇಲೆ ಅಂತಹ ಮಹಾಪುರುಷರ ಜನ್ಮಸ್ಥಳವನ್ನು ನೋಡದೆ ಬರುವುದುಂಟೆ? ಅವರ ಹೆಂಡತಿ ತಾನೇ ಮಾಡಿದ್ದ ಅನೇಕ ರುಚಿಕರವಾದ ಪದಾರ್ಥಗಳಿಂದ ಊಟ ಮಾಡಿಸಿದರು. ಆ ಕುಟೀರವು ಶಾಂತಿಮಯವೂ ಏಕಾಂತವೂ ಸುಂದರವೂ ಆಗಿದೆ. ನಾನು ಅಲ್ಲಿಯ ಕೆರೆಯೊಂದರಲ್ಲಿ ಈಜಾಡಿದೆ. ನಂತರ ನಾನು ಎಂತಹ ಸುಖನಿದ್ರೆಯಲ್ಲಿ ಮುಳುಗಿದೆನೆಂದರೆ ನಾನು ಎದ್ದಾಗ ಮಧ್ಯಾಹ್ನ ಎರಡು ಗಂಟೆಯಾಗಿತ್ತು. ನನ್ನ ಜೀವಮಾನದಲ್ಲಿ ಗಾಢನಿದ್ರೆಯಲ್ಲಿ ಕಳೆದ ದಿನಗಳು ಅತಿ ವಿರಳ. ನಾಗಮಹಾಶಯರ ಮನೆಯಲ್ಲಿ ಮಲಗಿದ್ದ ದಿನವೂ ಆ ದಿನಗಳಲ್ಲೊಂದು. ನಿದ್ರೆಯಿಂದೆಚ್ಚೆತ್ತ ಮೇಲೆ ನನಗೆ ಪುಷ್ಕಳ ಭೋಜನವಾಯಿತು. ನಾಗಮಹಾಶಯರ ಪತ್ನಿ ನನಗೊಂದು ಬಟ್ಟೆಯನ್ನು ಕೊಟ್ಟರು. ಅದನ್ನು ತಲೆಗೆ ರುಮಾಲಿನಂತೆ ಸುತ್ತಿಕೊಂಡು ನಾನು ಡಾಕ್ಕಾಕ್ಕೆ ಹೊರಟೆ. ಅಲ್ಲಿ ನಾಗಮಹಾಶಯರ ಭಾವಚಿತ್ರವನ್ನು ಪೂಜಿಸುವರು. ಅವರ ಅವಶೇಷವನ್ನಿಟ್ಟಿರುವ ಸ್ಥಳವನ್ನು ಚೆನ್ನಾಗಿಟ್ಟಿರಬೇಕು. ಈಗ ಕೂಡ ಅದು ಸರಿಯಾಗಿಲ್ಲ.

ಶಿಷ್ಯ: ಆ ಭಾಗದ ಜನರು ನಾಗಮಹಾಶಯರನ್ನು ಅಷ್ಟೊಂದು ಗೌರವಿಸುವುದಿಲ್ಲ.

ಸ್ವಾಮೀಜಿ: ಸಾಧಾರಣ ಮನುಷ್ಯರು ಹೇಗೆ ತಾನೇ ಅಂತಹ ಮಹಾಪುರುಷನನ್ನು ಗೌರವಿಸುವರು? ಅವರ ಸಹವಾಸ ಹೊಂದಿದವರು ನಿಜವಾಗಿ ಧನ್ಯರು.

ಶಿಷ್ಯ: ನೀವು ಕಾಮಾಖ್ಯದಲ್ಲೇನು ನೋಡಿದಿರಿ?

ಸ್ವಾಮೀಜಿ: ಷಿಲ್ಲಾಂಗ್ ಬೆಟ್ಟಗಳು ಬಹು ಸುಂದರವಾಗಿವೆ. ನಾನು ಅಲ್ಲಿ ಅಸ್ಸಾಮಿನ ಮುಖ್ಯ ಕಮಿಷನರ್ ಆದ ಸರ್ ಹೆನ್ರಿ ಕಾಟನ್‌ರವರನ್ನು ಭೇಟಿ ಮಾಡಿದೆ. ಅವರು ನನ್ನನ್ನು ಕೇಳಿದರು: ‘ಸ್ವಾಮೀಜಿ, ಯೂರೋಪ್, ಅಮೆರಿಕಾ ದೇಶಗಳ ಪ್ರವಾಸದ ನಂತರ ಈ ದೂರದ ಬೆಟ್ಟಗಳಲ್ಲೇನನ್ನು ನೋಡ ಬಂದಿರಿ?’ ಸರ್ ಹೆನ್ರಿ ಕಾಟನ್‌ರಂತಹ ಒಳ್ಳೆಯ ಮತ್ತು ದಯಾರ್ದ್ರಹೃದಯಿಗಳು ಬಹಳ ಅಪರೂಪ. ನನ್ನ ಖಾಯಿಲೆಯ ವಿಚಾರ ಕೇಳಿ ಅವರು ಸೀನಿಯರ್ ಸರ್ಜನ್ನರನ್ನು ಕಳುಹಿಸಿದರು. ಬೆಳಿಗ್ಗೆ ಸಂಜೆ ಎರಡು ಹೊತ್ತೂ ನನ್ನ ಆರೋಗ್ಯವನ್ನು ವಿಚಾರಿಸುತ್ತಿದ್ದರು. ನನಗೆ ಅಲ್ಲಿ ಹೆಚ್ಚು ಭಾಷಣ ಮಾಡಲಾಗಲಿಲ್ಲ. ನನ್ನ ಆರೋಗ್ಯ ಕೆಟ್ಟಿತು. ದಾರಿಯಲ್ಲಿ ನಿತಾಯ್ ನನ್ನನ್ನು ಚೆನ್ನಾಗಿ ನೋಡಿಕೊಂಡರು.

ಶಿಷ್ಯ: ಆ ಭಾಗದಲ್ಲಿ ಧಾರ್ಮಿಕ ಭಾವನೆ ಹೇಗಿದೆ?

ಸ್ವಾಮೀಜಿ: ಅದು ತಾಂತ್ರಿಕ ಪ್ರದೇಶ. ‘ಹಂಕರದೇವ’ ಎಂಬೊಬ್ಬನನ್ನು ಅವತಾರವೆಂದು ಪೂಜಿಸುವರೆಂದು ಕೇಳಿದೆ, ಆತನ ಪಂಥ ಬಹಳ ಹರಡಿದೆ ಎಂದು ಕೇಳಿದೆ. ನನಗೆ ಹಂಕರದೇವನೆಂಬುದು ಶಂಕರಾಚಾರ್ಯರ ಮತ್ತೊಂದು ರೂಪವೊ ಏನೊ ಎಂದು ವಿಚಾರಿಸಲಾಗಲಿಲ್ಲ. ಅವರು ಸಂನ್ಯಾಸಿಗಳಿರಬಹುದು – ಎಂದರೆ ತಾಂತ್ರಿಕ ಸಂನ್ಯಾಸಿಗಳು ಅಥವಾ ಶಂಕರ ಪಂಥದವರಾಗಿರಬಹುದು.

ಶಿಷ್ಯ: ಪೂರ್ವ ಬಂಗಾಳಿಗಳು ನಾಗಮಹಾಶಯರನ್ನು ಹೇಗೆ ಅರ್ಥಮಾಡಿಕೊಂಡಿಲ್ಲವೋ ಹಾಗೆಯೇ ನಿಮ್ಮನ್ನೂ ಅರಿತಿಲ್ಲ.

ಸ್ವಾಮೀಜಿ: ಅವರು ನನ್ನನ್ನು ಗೌರವಿಸಲಿ, ಬಿಡಲಿ, ಆ ಭಾಗದ ಜನರು ಇಲ್ಲಿನವರಿಗಿಂತ ಹೆಚ್ಚು ಉತ್ಸಾಹಿಗಳು, ಕಾರ್ಯಪಟುಗಳು. ಕಾಲ ಕ್ರಮೇಣ ಅವರು ಹೆಚ್ಚು ಮುಂದುವರಿಯುವರು. ಈ ವರ್ತಮಾನ ಕಾಲದಲ್ಲಿ ನಾವು ‘ನಾಗರಿಕತೆ’ ಎಂದು ಕರೆಯುವುದು ದೇಶದ ಆ ಭಾಗವನ್ನು ಇನ್ನೂ ಪೂರ್ತಿ ಒಳಹೊಕ್ಕಿಲ್ಲ. ಕ್ರಮೇಣ ಅದೂ ಆಗುತ್ತದೆ. ಎಲ್ಲಾ ಕಾಲದಲ್ಲಿಯೂ ರಾಜಧಾನಿಯಿಂದ ಹಳ್ಳಿಗಳಿಗೆ ಸಮಾಜ ಮರ್ಯಾದೆಗಳು, ಶೈಲಿಗಳು ಹರಡುವುವು. ಪೂರ್ವ ಬಂಗಾಳದಲ್ಲಿಯೂ ಇದೇ ರೀತಿಯಾಗುತ್ತಿದೆ. ನಾಗಮಹಾಶಯರಂಥ ಮಹಾತ್ಮರಿಗೆ ಜನನವಿತ್ತ ದೇಶ ಧನ್ಯ. ಅದರ ಭವಿಷ್ಯವೂ ಆಶಾದಾಯಕವಾಗಿದೆ. ಆ ಮಹಾತ್ಮನ ಪ್ರತಿಭೆಯಿಂದ ಪೂರ್ವ ಬಂಗಾಲ ರಂಜಿಸುತ್ತಿದೆ.

ಶಿಷ್ಯ: ಆದರೆ ಸ್ವಾಮಿಜಿ, ಜನಸಾಮಾನ್ಯರಿಗೆ ಅವರು ಅಂತಹ ದೊಡ್ಡ ವ್ಯಕ್ತಿಯೆಂದು ತಿಳಿದೇ ಇರಲಿಲ್ಲ. ಅವರು ತಾವೇ ಜನರಿಂದ ಬಹುಮಟ್ಟಿಗೆ ದೂರವಾಗಿದ್ದರು.

ಸ್ವಾಮೀಜಿ: ಅಲ್ಲಿ ಅವರೆಲ್ಲಾ ನನ್ನ ಆಹಾರದ ವಿಚಾರವಾಗಿ ಎಷ್ಟೊಂದು ಗೊಂದಲವನ್ನೆಬ್ಬಿಸುತ್ತಿದ್ದರು. ‘ನೀವೇಕೆ ಆ ಆಹಾರ ಸೇವಿಸುವಿರಿ? ಅವರ ಕೈಯಲ್ಲಿ ಏಕೆ ಊಟ ಮಾಡುವಿರಿ?’ ಮುಂತಾಗಿ, ಅದಕ್ಕೆ ನಾನು ಉತ್ತರ ಕೊಡಬೇಕಾಗಿತ್ತು, ‘ನಾನೊಬ್ಬ ಸಂನ್ಯಾಸಿ ಭಿಕ್ಷು, ಆಹಾರದ ವಿಷಯವಾಗಿ ನಾನೇಕೆ ಅಷ್ಟೊಂದು ಕಟ್ಟುನಿಟ್ಟನ್ನು ಅನುಸರಿಸಬೇಕಾಗಿದೆ?’ ನಿಮ್ಮ ಧರ್ಮ ಶಾಸ್ತ್ರಗಳೇ ಹೇಳುವುದಿಲ್ಲವೆ ‘ಮನೆ ಮನೆಗೂ ಹೋಗಿ ಭಿಕ್ಷೆ ಬೇಡಬೇಕು. ಚಂಡಾಲನ ಮನೆಗೂ ಹೋಗಿ ಭಿಕ್ಷೆ ಎತ್ತಬೇಕು’ ಎಂದು? ಆದರೆ ಪ್ರಾರಂಭದಲ್ಲಿ ಧರ್ಮವನ್ನು ಗಾಢವಾಗಿ ಅರಿಯಬೇಕಾದರೆ, ಧರ್ಮಗ್ರಂಥಗಳಲ್ಲಿರುವ ಸತ್ಯವನ್ನು ನಮ್ಮ ಜೀವನದಲ್ಲಿ ತರಲು ಬಾಹ್ಯ ನಡವಳಿಕೆಗೂ ನಾವು ಹೆಚ್ಚು ಗಮನ ಕೊಡಬೇಕು. ನೀನು ಶ‍್ರೀರಾಮಕೃಷ್ಣರು ಹೇಳುತ್ತಿದ್ದುದನ್ನು, ಪಂಚಾಂಗವನ್ನು ಹಿಂಡಿ ನೀರು ಬರಿಸುವುದರ ವಿಚಾರವನ್ನು ಕೇಳಿಲ್ಲವೆ? ಬಾಹ್ಯರೂಪ, ನಡವಳಿಕೆಗಳು ಮನುಷ್ಯನ ಆಂತರ್ಯದಲ್ಲಿರುವ ಮಹಾಶಕ್ತಿಯು ವಿಕಾಸಗೊಳ್ಳಲು ಸಹಾಯ ಮಾತ್ರ. ಎಲ್ಲಾ ಧರ್ಮಗ್ರಂಥಗಳ ಉದ್ದೇಶವೂ ಮನುಷ್ಯನ ಆಂತರಿಕ ಶಕ್ತಿಯನ್ನು ಜಾಗೃತಗೊಳಿಸುವುದು ಮತ್ತು ತನ್ನ ನೈಜ ಸ್ವಭಾವವನ್ನು ಸಾಕ್ಷಾತ್ಕರಿಸಿಕೊಳ್ಳುವುದಾಗಿದೆ. ಈ ವಿಧಿ ನಿಷೇಧಗಳೆಲ್ಲಾ ಅದರ ಸಾಧನಾಮಾರ್ಗಗಳು, ನೀನು ನಿನ್ನ ಗುರಿಯನ್ನೇ ಮರೆತು ಈ ಉಪಕರಣಗಳಿಗಾಗಿ ಕಾದಾಡಿದರೆ ಅದರಿಂದೇನು ಪ್ರಯೋಜನ? ಶ‍್ರೀರಾಮಕೃಷ್ಣರು ಈ ಸತ್ಯವನ್ನು ಪರಿಚಯ ಮಾಡಿಕೊಡಲು ಅವತರಿಸಿದರು.

ಸತ್ಯ ಸಾಕ್ಷಾತ್ಕಾರವೇ ಎಲ್ಲಕ್ಕಿಂತ ಮುಖ್ಯವಾದುದು. ನೀನು ಗಂಗಾನದಿಯಲ್ಲೇ ಸಾವಿರ ವರ್ಷ ಸ್ನಾನ ಮಾಡಿದರೂ, ದೀರ್ಘಕಾಲ ಸಸ್ಯಾಹಾರ ಸೇವಿಸಿದರೂ ಅದು ನಿನ್ನನ್ನು ಆತ್ಮವಿಕಾಸಕ್ಕೆ ಒಯ್ಯುವುದೆ? ಯಾವುದರಿಂದಲೂ ಏನೂ ಪ್ರಯೋಜನವಿಲ್ಲ. ಆದರೆ ಯಾರಾದರೂ ಈ ಬಾಹ್ಯಾಚರಣೆಗಳನ್ನಾಚರಿಸದೆ ಆತ್ಮಸಾಕ್ಷಾತ್ಕಾರ ಮಾಡಿಕೊಂಡಿದ್ದರೆ ಆ ಬಾಹ್ಯಾಚರಣೆಗಳನ್ನನುಸರಿಸದಿರುವ ಹಾದಿಯೇ ಶ್ರೇಷ್ಠ ಮಾರ್ಗ. ಆದರೆ ಬ್ರಹ್ಮಸಾಕ್ಷಾತ್ಕಾರವಾದ ಮೇಲೂ ಇತರರಿಗೆ ಮೇಲ್ಪಂಕ್ತಿಯಾಗಲೋಸುಗ ಕೊಂಚ ಬಾಹ್ಯವಿಧಿಗಳನ್ನನುಸರಿಸಬೇಕು. ಮುಖ್ಯವಾದುದೇನೆಂದರೆ ನೀನು ಮನಸ್ಸನ್ನು ಯಾವುದರ ಮೇಲಾದರೂ ಏಕಾಗ್ರ ಮಾಡಬೇಕು. ಅದು ಒಂದು ವಸ್ತುವಿನಲ್ಲಿ ಏಕಾಗ್ರತೆ ಪಡೆದರೆ ಅದರಿಂದ ಚಿತ್ತೈಕಾಗ್ರತೆ ಬರುವುದು. ಅದರ ಇತರ ಬದಲಾವಣೆಗಳೆಲ್ಲ ಮಾಯವಾಗಿ ಮನಸ್ಸು ಏಕಪ್ರಕಾರವಾಗಿ ಒಂದೇ ಕಡೆಗೆ ಹರಿಯುವುದು. ಅನೇಕರು ಸಂಪೂರ್ಣವಾಗಿ ಕೇವಲ ಬಾಹ್ಯರೂಪ, ಸಂಪ್ರದಾಯಗಳಲ್ಲೇ ಮಗ್ನರಾಗುವರು. ಅದರಿಂದ ತಮ್ಮ ಮನಸ್ಸನ್ನು ಆತ್ಮನ ಕಡೆ ತಿರುಗಿಸಲಸಮರ್ಥರಾಗುವರು. ನೀನು ಹಗಲು ರಾತ್ರಿ ಬರೀ ಉಪವಾಸ, ನಿಷೇಧಗಳ ಸಂಕುಚಿತ ಕಂದಾಚಾರಗಳಲ್ಲೇ ಮುಳುಗಿದ್ದರೆ ಆತ್ಮಪ್ರಕಾಶನಕ್ಕೆ ಅವಕಾಶವಿದೆ? ಯಾರು ಆತ್ಮ ಸಾಕ್ಷಾತ್ಕಾರದಲ್ಲಿ ಹೆಚ್ಚು ಹೆಚ್ಚು ಮುಂದುವರಿದಿರುವರೋ ಅವರು ಈ ಬಾಹ್ಯರೂಪ ಆಚರಣೆಗಳನ್ನು ಅಷ್ಟು ಕಡಿಮೆ ಅವಲಂಬಿಸುವರು. ಶಂಕರಾಚಾರ್ಯರೂ ಹೇಳಿದ್ದಾರೆ, ‘ಯಾರ ಮನಸ್ಸು ಯಾವಾಗಲೂ ಗುಣಾತೀತವಾಗಿದೆಯೋ ಅಂಥವನಿಗೆ ಈ ವಿಧಿ ನಿಷೇಧಗಳೆಲ್ಲಿವೆ?’ ಆದ್ದರಿಂದ ಆತ್ಮಸಾಕ್ಷಾತ್ಕಾರವೇ ಪ್ರಧಾನವಾದುದು. ಅದೇ ನಿನ್ನ ಗುರಿ ಎಂದು ತಿಳಿ. ಪ್ರತಿಯೊಂದು ಭಿನ್ನ ಪಂಥವೂ ಸತ್ಯದ ಒಂದು ಪಥ. ನೀನು ಎಷ್ಟು ಮಟ್ಟಿಗೆ ತ್ಯಾಗಿಯಾಗಿರುವೆಯೊ ಅದೇ ನೀನು ಮುಂದುವರಿದಿರುವುದಕ್ಕೆ ನಿದರ್ಶನ. ಯಾರಲ್ಲಿ ನೀನು ಕಾಮಿನಿ ಕಾಂಚನಗಳಲ್ಲಿ ಆಸಕ್ತಿಯು ಸಾಕಷ್ಟು ಕಡಿಮೆಯಾಗಿರುವುದನ್ನು ನೋಡುವೆಯೊ, ಅವನು ಯಾವ ಪಂಗಡಕ್ಕೇ ಸೇರಿರಲಿ, ಅವನ ಆತ್ಮಶಕ್ತಿ ಜಾಗೃತವಾಗಿದೆ ಎಂದು ತಿಳಿ. ಅಂಥವನ ಆತ್ಮಸಾಕ್ಷಾತ್ಕಾರದ ಹೆಬ್ಬಾಗಿಲು ತೆರೆದಿದೆ. ಅದಕ್ಕೆ ಬದಲಾಗಿ ಸಾವಿರಾರು ಶಿಷ್ಟಾಚಾರ ಸಂಪ್ರದಾಯವನ್ನು ಅನುಸರಿಸಿಕೊಂಡು ಧರ್ಮಗ್ರಂಥಗಳನ್ನೇ ಬಾಯಲ್ಲಿ ಪಠಿಸುತ್ತಿದ್ದರೂ ತ್ಯಾಗವು ನಿನ್ನಲ್ಲಿ ಸುಳಿಯದಿದ್ದರೆ ನಿನ್ನ ಜೀವನ ನಿಷ್ಪಲವೆಂದು ತಿಳಿ, ಆತ್ಮಸಾಕ್ಷಾತ್ಕಾರಕ್ಕೆ ಮನಃಪೂರ್ವಕವಾಗಿ ಪ್ರಯತ್ನಿಸು. ನೀನು ಸಾಕಷ್ಟು ಧರ್ಮಗ್ರಂಥಗಳ ವ್ಯಾಸಂಗ ಮಾಡಿರುವೆ. ಈಗ ಹೇಳು ಅವುಗಳಿಂದೆಷ್ಟು ಪ್ರಯೋಜನವಾಗಿದೆಯೆಂದು? ಅನೇಕರು ಹಣದ ವಿಷಯವನ್ನೇ ಯೋಚಿಸುತ್ತಾ ಕೋಟ್ಯಾಧೀಶ್ವರರಾದರೆ ನೀನು ಧರ್ಮಗ್ರಂಥಗಳನ್ನೇ ಯೋಚಿಸುತ್ತಾ ಪಂಡಿತನಾಗಿರುವೆ. ಆದರೆ ಎರಡೂ ಬಂಧನಗಳೇ. ವಿದ್ಯೆ ಅವಿದ್ಯೆ, ಜ್ಞಾನ ಅಜ್ಞಾನ ಇತ್ಯಾದಿಗಳನ್ನು ಮೀರಿ ಮುಂದೆ ಹೋಗಿ ಅನಂತ ಜ್ಞಾನವನ್ನು ಪಡೆದುಕೊ.

ಶಿಷ್ಯ: ನಿಮ್ಮ ಕೃಪೆಯಿಂದ ನಾನು ಎಲ್ಲವನ್ನೂ ಅರ್ಥಮಾಡಿಕೊಳ್ಳುವೆ. ಆದರೆ ನನ್ನ ಪೂರ್ವಾರ್ಜಿತ ಕರ್ಮವು ಈ ಬೋಧನೆಗಳನ್ನು ರಕ್ತಗತಮಾಡಿಕೊಳ್ಳಲು ಬಿಡುತ್ತಿಲ್ಲ.

ಸ್ವಾಮೀಜಿ: ನಿನ್ನ ಕರ್ಮ ಮುಂತಾದ ಮಾತನ್ನು ಆಚೆಗೆ ಬಿಸಾಡು. ನಿನ್ನ ಪೂರ್ವಾರ್ಜಿತ ಕರ್ಮದಿಂದ ನೀನು ಈ ಶರೀರವನ್ನು ಹೊಂದಿರುವುದು ನಿಜವಾದರೆ ಹೀನಕೆಲಸಗಳ ಪರಿಣಾಮವನ್ನು ಸತ್ಕಾರ್ಯಗಳ ಪುಣ್ಯದಿಂದ ತೊಡೆದುಹಾಕು. ನಿನ್ನ ಈ ದೇಹದಲ್ಲೇ ಜೀವನ್ಮುಕ್ತನಾಗಬಾರದೇಕೆ? ಆತ್ಮಜ್ಞಾನ ನಿನ್ನ ಕೈಯಲ್ಲೇ ಇದೆ ಎಂದು ತಿಳಿ, ಯಥಾರ್ಥ ಜ್ಞಾನದಲ್ಲಿ ಕೆಲಸದ ಸಂಪರ್ಕವೇ ಇರುವುದಿಲ್ಲ. ಜೀವನ್ಮುಕ್ತರಾದ ಮೇಲೂ ಕೆಲಸ ಮಾಡುವವರು ಇತರರ ಕಲ್ಯಾಣಕ್ಕಾಗಿ ಮಾಡುವರು. ಅವರಲ್ಲಿ ಕರ್ಮ ಫಲಾಪೇಕ್ಷೆ ಇರುವುದಿಲ್ಲ. ಆಸೆಯ ಬೀಜಾಂಕುರ ಅವರಲ್ಲಿ ಕೊಂಚವೂ ಇರುವುದಿಲ್ಲ. ಖಂಡಿತವಾದ ಮಾತಿನಲ್ಲಿ ಹೇಳಬೇಕಾದರೆ ಗೃಹಸ್ಥ ಜೀವನದಲ್ಲಿದ್ದು ವಿಶ್ವ ಕಲ್ಯಾಣಕ್ಕೋಸ್ಕರವಾಗಿ ಕೆಲಸ ಮಾಡುವುದು ಖಂಡಿತ ಅಸಾಧ್ಯ. ಇಡೀ ಹಿಂದೂ ಧಾರ್ಮಿಕ ಗ್ರಂಥಗಳಲ್ಲೆಲ್ಲಾ ಈ ರೀತಿ ಇರುವುದು ಜನಕರಾಜನ ನಿದರ್ಶನ ಒಂದೇ ಒಂದು. ಈಗಿನ ಕಾಲದಲ್ಲಿ ನೀವೆಲ್ಲಾ ವರ್ಷವರ್ಷಕ್ಕೂ ಮಕ್ಕಳನ್ನು ಹೆರುತ್ತ ಜನಕರಾಜನಂತೆ ಸೋಗು ಹಾಕಲು ಪ್ರಯತ್ನಿಸುವಿರಿ. ಅವನಿಗಾದರೋ ದೇಹದ ಪ್ರಜ್ಞೆ ಕೂಡ ಇರಲಿಲ್ಲ.

ಶಿಷ್ಯ: ದಯವಿಟ್ಟು ನಾನೀ ಜನ್ಮದಲ್ಲೇ ಆತ್ಮಸಾಕ್ಷಾತ್ಕಾರ ಹೊಂದುವಂತೆ ಹರಸಿ.

ಸ್ವಾಮೀಜಿ: ಭಯವೇನು? ನಿನ್ನಲ್ಲಿ ನಿಜವಾಗಿ ಪ್ರಾಮಾಣಿಕತನವಿದ್ದಲ್ಲಿ ಖಂಡಿತವಾಗಿಯೂ ನೀನು ಈ ಜನ್ಮದಲ್ಲೇ ಹೊಂದುವೆ. ಆದರೆ ಮನುಷ್ಯ ಪ್ರಯತ್ನ ಆವಶ್ಯಕ. ಅದೇನೆಂದು ನಿನಗೆ ಗೊತ್ತೆ? ‘ನಾನು ಖಂಡಿತ ಆತ್ಮಜ್ಞಾನ ಹೊಂದುವೆ. ಯಾವ ಅಡ್ಡಿ ಆತಂಕಗಳೇ ಬರಲಿ, ನಾನು ಅವುಗಳನ್ನೆಲ್ಲಾ ಜಯಿಸುವೆ’ - ಇಂತಹ ಅಚಲ ನಿರ್ಧಾರವೇ ಪುರುಷಕಾರ, ‘ನನ್ನ ತಂದೆ, ತಾಯಿ, ಅಣ್ಣ, ತಮ್ಮ, ಗೆಳೆಯ, ಹೆಂಡತಿ, ಮಕ್ಕಳು, ಯಾರಿಗೆ ಏನಾದರೂ ಆಗಲಿ, ಅವರು ಬದುಕಲಿ, ಸಾಯಲಿ, ನನಗೆ ಆತ್ಮಸಾಕ್ಷಾತ್ಕಾರವಾಗುವವರೆಗೂ ನಾನು ಎಂದಿಗೂ ಹಿಂತಿರುಗುವುದಿಲ್ಲ’. ಹೀಗೆ ಎಲ್ಲಾ ನಿಮಿತ್ತಗಳನ್ನೂ ಒತ್ತಟ್ಟಿಗಿಟ್ಟು ಗುರಿ ಸೇರಬೇಕೆಂದು ಏಕಪ್ರಕಾರವಾದ ಅಚಲನಿರ್ಧಾರದಿಂದ ಹೊರಡುವುದೇ ಪುರುಷ ಪ್ರಯತ್ನ. ಇಲ್ಲದಿದ್ದಲ್ಲಿ ಕೇವಲ ಮೃಗಪಕ್ಷಿಗಳೂ ತಮ್ಮ ಹೊಟ್ಟೆಪಾಡಿಗೆ ಪರಿಶ್ರಮಪಡುವುವು. ನೀನು ಪ್ರಪಂಚದ ಸಾಮಾನ್ಯ ಜನರಂತೆ ಜೀವಿಸುತ್ತಾ, ಜನಸಾಮಾನ್ಯರ ಪ್ರವಾಹದಲ್ಲೇ ತೇಲುತ್ತಿದ್ದರೆ ನಿನಗೆ ಪುರುಷ ಪ್ರಯತ್ನವೆಲ್ಲಿದೆ? ಜನಸಾಮಾನ್ಯರು ಮೃತ್ಯುವಿನ ದವಡೆಗೆ ನುಗ್ಗುತ್ತಿದ್ದಾರೆ. ಯಾವುದರಿಂದಲೂ ಹಿಂತೆಗೆಯಬೇಡ, ಈ ಸುಖದುಃಖದಿಂದ ಕೂಡಿದ ಶರೀರ ಎಷ್ಟುದಿನ ಬಾಳುವುದು? ನಿನಗೆ ಈ ಮಾನವ ಶರೀರ ಲಭಿಸಿರುವಾಗ ನಿನ್ನ ಆತ್ಮನನ್ನು ಜಾಗೃತಗೊಳಿಸು. ‘ನಾನು ನಿರ್ಭಯಾವಸ್ಥೆಯನ್ನು ಹೊಂದಿರುವೆ, ನಾನು ಆತ್ಮ, ನನ್ನ ಕೀಳು ಅಹಂಕಾರ ಶಾಶ್ವತವಾಗಿ ನಾಶಹೊಂದಿದೆ-’ ಈ ಭಾವನೆಯಲ್ಲಿ ಪರಿಪೂರ್ಣನಾಗು. ನಂತರ ಎಲ್ಲಿಯವರೆವಿಗೆ ಈ ದೇಹವಿರುವುದೋ ಅಲ್ಲಿಯವರೆಗೂ ಇತರರಿಗೆ ಈ ನಿರ್ಭಯತೆಯ ಸಂದೇಶವನ್ನು ಸಾರು, ‘ನೀನು ಅದೇ ಆಗಿರುವೆ, ಉತ್ತಿಷ್ಠತ ಜಾಗ್ರತ ಪ್ರಾಪ್ಯವರಾನ್ನಿ ಬೋಧತ?’ ಎದ್ದೇಳು ಎಚ್ಚರಗೊಳ್ಳು, ಗುರಿ ಮುಟ್ಟುವವರೆಗೂ ನಿಲ್ಲಬೇಡ. ನೀನಿದನ್ನು ಪಡೆಯಬಲ್ಲೆಯಾದರೆ ನೀನು ನಿಜವಾಗಿಯೂ ಸಮರ್ಥನಾದ ಪೂರ್ವ ಬಂಗಾಳಿ ಎಂದು ತಿಳಿಯುತ್ತೇನೆ.

\newpage

\chapter[ಅಧ್ಯಾಯ ೩೫]{ಅಧ್ಯಾಯ ೩೫\protect\footnote{\engfoot{C.W, Vol. VII, P. 213}}}

\begin{center}
ಸ್ಥಳ: ಬೇಲೂರು ಮಠ, ವರ್ಷ: ಕ್ರಿ.ಶ. ೧೯೦೧.
\end{center}

ಷಿಲ್ಲಾಂಗ್ ಬೆಟ್ಟಗಳಿಂದ ಹಿಂತಿರುಗಿ ಬಂದಾಗಿನಿಂದ ಸ್ವಾಮಿಗಳ ದೇಹ ಸ್ಥಿತಿ ಅಷ್ಟು ಸಮರ್ಪಕವಾಗಿಲ್ಲ. ಅವರ ಪಾದಗಳು ಊದಿಕೊಂಡಿವೆ. ಅವರ ಗುರುಭಾಯಿಗಳಿಗೆ ಇದರಿಂದ ಬಹಳ ಕಳವಳವಾಗಿದೆ. ಸ್ವಾಮಿ ನಿರಂಜನಾನಂದರ ಸಲಹೆಯಂತೆ ಸ್ವಾಮಿಗಳು ಕವಿರಾಜರಿಂದ ಔಷಧಿ ಸೇವಿಸಲು ಒಪ್ಪಿದ್ದಾರೆ. ಮುಂದಿನ ಮಂಗಳವಾರದಿಂದ ಆ ವೈದ್ಯರು ಸಂಪೂರ್ಣ ನೀರು, ಉಪ್ಪನ್ನು ತ್ಯಜಿಸುವಂತೆ ತಿಳಿಸಿ ತಮ್ಮ ಚಿಕಿತ್ಸೆಯನ್ನು ಪ್ರಾರಂಭ ಮಾಡುವವರಿದ್ದಾರೆ. ಇಂದು ಶನಿವಾರ, ಶಿಷ್ಯ ಅವರನ್ನು ಕೇಳಿದ: ಸ್ವಾಮೀಜಿ, ಈದಿನಗಳಲ್ಲಿ ಬಿಸಿಲಿನ ತಾಪ ಬಹಳ ತೀಕ್ಷ್ಣವಾಗಿದೆ. ನೀವು ನೀರನ್ನು ಪದೇಪದೇ ಕುಡಿಯುತ್ತಿದ್ದೀರಿ. ಈ ಚಿಕಿತ್ಸೆಗಾಗಿ ನೀರು ಕುಡಿಯುವುದನ್ನು ಸಂಪೂರ್ಣ ನಿಲ್ಲಿಸುವುದೆಂದರೆ ನಿಮಗೆ ಸಹಿಸಲಶಕ್ಯವಾಗಬಹುದು.

ಸ್ವಾಮೀಜಿ: ಏನು ಹೇಳುತ್ತಿರುವೆ? ನಾನು ಆ ಚಿಕಿತ್ಸೆಯ ದಿನ ಬೆಳಿಗ್ಗೆ ನೀರನ್ನು ಮುಟ್ಟುವುದಿಲ್ಲೆಂದು ದೃಢ ಪ್ರತಿಜ್ಞೆಮಾಡುವೆ. ಅಂದಿನಿಂದ ಒಂದು ತೊಟ್ಟು ನೀರನ್ನೂ ನನ್ನ ಗಂಟಲಿನೊಳಕ್ಕೆ ಹೋಗಲು ಬಿಡುವುದಿಲ್ಲ. ಮೂರು ವಾರಗಳವರೆಗೂ ತೊಟ್ಟು ನೀರೂ ಗಂಟಲೊಳಕ್ಕೆ ಇಳಿಯಲಾರದು. ಈ ದೇಹ ಕೇವಲ ಮನಸ್ಸಿನ ಬಾಹ್ಯ ಹೊದಿಕೆ. ಮನಸ್ಸು ಆಜ್ಞಾಪಿಸಿದ್ದನ್ನೆಲ್ಲಾ ಅದು ನಡಸೇ ತೀರಬೇಕು. ಆದ್ದರಿಂದ ಯಾವ ಹೆದರಿಕೆಯೂ ಇಲ್ಲ. ನಿರಂಜನನ ವಿನಂತಿಯಂತೆ ಈ ಚಿಕಿತ್ಸೆಗೆ ಒಪ್ಪಿಕೊಂಡಿರುವೆ. ನನ್ನ ಗುರುಭಾಯಿಗಳ ಪ್ರಾರ್ಥನೆಗೆ ನಾನು ಹೇಗೆ ಉದಾಸೀನನಾಗಿರಲಿ?

ಈಗ ಹತ್ತು ಗಂಟೆಯ ಸಮಯ. ಸ್ವಾಮಿಗಳು ಹಸನ್ಮುಖದಿಂದ ಕಟ್ಟಲಿರುವ ಸ್ತ್ರೀಯರ ಮಠದ ವಿಚಾರವಾಗಿ ಮಾತೆತ್ತಿದರು: ಶ‍್ರೀ ಮಹಾಮಾತೆಯನ್ನೇ ಸ್ಫೂರ್ತಿಯ ಕೇಂದ್ರವಾಗಿಟ್ಟುಕೊಂಡು ಗಂಗಾನದಿಯ ಪೂರ್ವ ತೀರದಲ್ಲಿ ಒಂದು ಮಠವನ್ನು ಸ್ಥಾಪಿಸಬೇಕು. ಈ ಮಠದಲ್ಲಿ ಬ್ರಹ್ಮಚಾರಿಗಳು ಸಾಧುಗಳು ತರಪೇತು ಹೊಂದುವಂತೆಯೇ ಆ ಮಠದಲ್ಲಿಯೂ ಬ್ರಹ್ಮಚಾರಿಣಿಯರು, ಸಾಧ್ವಿಯರು ತರಪೇತಾಗುವರು.

ಶಿಷ್ಯ: ಸ್ವಾಮೀಜಿ, ಚರಿತ್ರೆ ನಮಗೆ ಪೂರ್ವಕಾಲದಲ್ಲಿ ಹೆಂಗಸರಿಗಾಗಿ ಯಾವ ಮಠವೂ ಇದ್ದಂತೆ ಹೇಳುವುದಿಲ್ಲ. ಬುದ್ಧನ ಕಾಲದಲ್ಲಿ ಮಾತ್ರ ಸ್ತ್ರೀಯರಿಗಾಗಿ ಮಠವಿದ್ದುದನ್ನು ಕೇಳಿರುವೆವು. ಆದರೆ ಕ್ರಮೇಣ ಅದರಿಂದ ಅನೇಕ ಭ್ರಷ್ಟಾಚರಣೆಗಳು ಹುಟ್ಟಿ ಸಮಗ್ರದೇಶವೆಲ್ಲಾ ದುಷ್ಕೃತ್ಯದಲ್ಲಿ ಮುಳುಗಿತು.

ಸ್ವಾಮೀಜಿ: ಅದೇಕೆ ಈ ದೇಶದಲ್ಲಿ ಸ್ತ್ರೀಪುರುಷರಲ್ಲಿ ಇಷ್ಟೊಂದು ಭೇದವಿದೆಯೋ ದೇವರೇ ಬಲ್ಲ. ವೇದಗಳು ಆತ್ಮವು ಸರ್ವ ಪ್ರಾಣಿಗಳಲ್ಲೂ ಏಕಪ್ರಕಾರವಾಗಿದೆ ಎಂದು ಸಾರುವುವು. ನೀವು ಯಾವಾಗಲೂ ಸ್ತ್ರೀಯರನ್ನು ದೂರುವಿರಿ. ಆದರೆ ಹೇಳಿ, ನೀವು ಅವರ ಏಳಿಗೆಗಾಗಿ ಏನನ್ನು ಮಾಡಿರುವಿರಿ? ಸ್ಮೃತಿ ಮುಂತಾದುವುಗಳನ್ನು ಬರೆಯುವುದು, ಕಠಿಣವಾದ ನಿಯಮಗಳಿಂದ ಅವರನ್ನು ಬಂಧನಕ್ಕೀಡು ಮಾಡುವುದು. ಹೆಂಗಸರನ್ನು ಕೇವಲ ಮಕ್ಕಳನ್ನು ಹೆರುವ ಯಂತ್ರವನ್ನಾಗಿ ಗಂಡಸರು ಮಾರ್ಪಡಿಸಿದ್ದಾರೆ. ಜಗನ್ಮಾತೆಯ ಜೀವಂತ ಸ್ವರೂಪಿಣಿಯರಾದ ಹೆಂಗಸರನ್ನು ಉದ್ಧರಿಸಲು ಪ್ರಯತ್ನಿಸದಿದ್ದಲ್ಲಿ ನೀವು ಮುಂದುವರಿಯಲು ಬೇರಾವ ಹಾದಿಯೂ ಇಲ್ಲ.

ಶಿಷ್ಯ: ಹೆಂಗಸರು ಗಂಡಸರಿಗೆ ಬಂಧನ ಮತ್ತು ಪ್ರಲೋಭನಕಾರಿಗಳು. ತಮ್ಮ ಮೋಹದಿಂದ ಗಂಡಸರ ಜ್ಞಾನ ವೈರಾಗ್ಯವನ್ನು ಮುಚ್ಚುವರು. ಈ ಕಾರಣದಿಂದಲೇ ಧಾರ್ಮಿಕ ಗ್ರಂಥಗಳು ಭಕ್ತಿಜ್ಞಾನವನ್ನು ಅವರು ಹೊಂದಲು ಬಹಳ ಕಷ್ಟವೆಂದು ಹೇಳುವರು.

ಸ್ವಾಮೀಜಿ: ನೀನಾವ ಧರ್ಮಗ್ರಂಥದಲ್ಲಿ ಸ್ತ್ರೀಯರು ಜ್ಞಾನ ಭಕ್ತಿಗೆ ಅನರ್ಹರೆಂದು ಓದಿರುವೆ? ದೇಶ ಅವನತಿಯಲ್ಲಿದ್ದಾಗ ಈ ಪುರೋಹಿತ ವರ್ಗದವರು ಇತರ ಜಾತಿಯವರು ವೇದಗಳನ್ನೋದಲು ಅನರ್ಹರೆಂದು ತೀರ್ಮಾನಿಸಿದಾಗಲೇ ಹೆಂಗಸರ ಹಕ್ಕುಬಾಧ್ಯತೆಯನ್ನು ಕಿತ್ತುಕೊಂಡರು. ಇಲ್ಲದಿದ್ದಲ್ಲಿ ವೇದ ಉಪನಿಷತ್ತುಗಳ ಕಾಲದಲ್ಲಿ ಮೈತ್ರೇಯೀ ಗಾರ್ಗಿ ಮತ್ತು ಇತರ ಸ್ಮೃತಪುಣ್ಯರಾದ ವ್ಯಕ್ತಿಗಳು ಬ್ರಹ್ಮಜ್ಞಾನದ ವಿಚಾರ ಚರ್ಚಿಸುತ್ತಿದ್ದುದರಿಂದ ಋಷಿಗಳ ಸ್ಥಳವನ್ನು ಕೂಡ ಆಕ್ರಮಿಸಿದ್ದಾರೆ. ವೇದಪಾರಂಗತರಾದ ಸಾವಿರ ಜನರಿದ್ದ ಸಭೆಯೊಂದರಲ್ಲಿ ಗಾರ್ಗಿ, ಬ್ರಹ್ಮನ ವಿಷಯದಲ್ಲಿ ಚರ್ಚೆ ನಡೆಯುತ್ತಿದ್ದಾಗ ಯಾಜ್ಞವಲ್ಕ್ಯರನ್ನು ಧೈರ್ಯದಿಂದ ಪ್ರತಿಭಟಿಸಿದಳು. ಆಗ ಅಂತಹ ಆದರ್ಶ ಮಹಿಳೆಯರು ಆಧ್ಯಾತ್ಮಿಕ ಜ್ಞಾನಕ್ಕೆ ಯೋಗ್ಯರಾಗಿರುವಾಗ ಅದೇ ಹಕ್ಕುಬಾಧ್ಯತೆಗಳನ್ನು ಆಧುನಿಕ ಯುಗದ ಮಹಿಳೆಯರಿಗೆ ಏಕೆ ಕೊಡಬಾರದು? ಒಮ್ಮೆ ಯಾವುದು ನಡೆಯಿತೋ ಅದು ಈಗಲೂ ಆಗಬಹುದು. ಚರಿತ್ರೆ ಪುನರಾವೃತ್ತಿಯಾಗುತ್ತಲೇ ಇರುತ್ತದೆ. ಎಲ್ಲಾ ಜನಾಂಗಗಳೂ ಸ್ತ್ರೀಯರಿಗೆ ಯಥಾರ್ಥ ಗೌರವ ಕೊಡುವುದರ ಮೂಲಕ ಹೆಸರುವಾಸಿಯಾದುವು. ಯಾವ ದೇಶ ಅಥವಾ ಜನಾಂಗ ಸ್ತ್ರೀಯರಿಗೆ ಗೌರವ ಕೊಟ್ಟಿಲ್ಲವೊ ಅದು ಖಂಡಿತವಾಗಿ ಕೀರ್ತಿ ಪಡೆದಿಲ್ಲ: ಮುಂದೆ ಪಡೆಯುವುದೂ ಇಲ್ಲ. ಜೀವಂತ ದೇವಿ ಸ್ವರೂಪಿಣಿಯರಾದ ಸ್ತ್ರೀಯರಿಗೆ ಗೌರವ ಕೊಡದಿರುವುದೇ ನಿಮ್ಮ ಜನಾಂಗ ಇಷ್ಟೊಂದು ಅಧೋಗತಿಗಿಳಿದಿರುವುದಕ್ಕೆ ಮುಖ್ಯ ಕಾರಣ. ಮನು ಹೇಳುತ್ತಾನೆ: “ಎಲ್ಲಿ ಸ್ತ್ರೀಯರನ್ನು ಪೂಜ್ಯ ದೃಷ್ಟಿಯಿಂದ ನೋಡುವರೋ ಅಲ್ಲಿ ದೇವತೆಗಳು ಸಂತುಷ್ಟರಾಗಿರುತ್ತಾರೆ, ಎಲ್ಲಿ ಅಗೌರವದಿಂದ ಕಾಣುವರೋ ಅಲ್ಲಿ ಅವರ ಕ್ರಿಯೆಗಳೆಲ್ಲ ವ್ಯರ್ಥ." ಎಲ್ಲಿ ಹೆಂಗಸರಿಗೆ ತಕ್ಕ ಗೌರವವಿಲ್ಲವೊ ಎಲ್ಲಿ ಅವರು ದುಃಖಿಗಳೊ ಆ ದೇಶ ಅಥವಾ ಜನಾಂಗ ಖಂಡಿತವಾಗಿಯೂ ಏಳಿಗೆಗೆ ಬರುವುದಿಲ್ಲ. ಈ ಕಾರಣದಿಂದಲೇ ಮೊದಲು ಅವರನ್ನು ಎತ್ತಬೇಕು. ಆದರ್ಶವಾದ ಮಠವೊಂದು ಅವರಿಗಾಗಿ ಸ್ಥಾಪನೆಯಾಗಬೇಕು.

ಶಿಷ್ಯ: ಸ್ವಾಮಿಜಿ, ನೀವು ಮೊದಲು ಪಾಶ್ಚಾತ್ಯ ದೇಶಗಳಿಂದ ಇಲ್ಲಿಗೆ ಬಂದಾಗ ಸ್ಟಾರ್ ಥಿಯೇಟರಿನಲ್ಲಿ ಕೊಟ್ಟ ಉಪನ್ಯಾಸದಲ್ಲಿ ತಾಂತ್ರಿಕ ಸಾಧನೆಯನ್ನು ಕಟುವಾಗಿ ಟೀಕಿಸಿದ್ದೀರಿ. ಈಗ ತಂತ್ರದಲ್ಲಿ ಹೇಳಿರುವ ಸ್ತ್ರೀಪೂಜೆಯನ್ನು ಸಮರ್ಥಿಸುತ್ತಿರುವಿರಿ. ಒಂದಕ್ಕೊಂದು ವಿರೋಧವಾಗಿ ಹೇಳುವಿರಲ್ಲ?

ಸ್ವಾಮಿಜಿ: ಈಗಿನ ತಂತ್ರದಲ್ಲಿನ ವಾಮಾಚಾರದ ಪದ್ಧತಿಯ ಹೀನಾಚರಣೆಯನ್ನು ನಾನು ಖಂಡಿಸಿದೆ. ನಾನು ತಂತ್ರದಲ್ಲಿರುವ ನಿಜವಾದ ವಾಮಾಚಾರವನ್ನು, ಮಾತೃಪೂಜೆಯನ್ನು ಹಳಿಯಲಿಲ್ಲ. ತಂತ್ರದ ನಿಜವಾದ ಉದ್ದೇಶವೇನೆಂದರೆ ಸ್ತ್ರೀಯರನ್ನು ಜಗನ್ಮಾತೆಯ ಅಂಶವೆಂದು ಪೂಜಿಸುವುದು. ಬೌದ್ಧಧರ್ಮದ ಅವನತಿಯ ಕಾಲದಲ್ಲಿ ವಾಮಾಚಾರವೂ ಹೆಚ್ಚು ಭ್ರಷ್ಟವಾಯಿತು. ಆ ಭ್ರಷ್ಟತೆಯೇ ಇಂದಿಗೂ ಬೆಳದುಬಂದಿದೆ. ಈಗಲೂ ಹಿಂದೂದೇಶದ ತಾಂತ್ರಿಕ ಸಾಧನೆಯು ಅದೇ ಭಾವನೆಗಳಿಂದ ಆಚ್ಛಾದಿತವಾಗಿದೆ. ನಾನು ಆ ಭಯಂಕರ ನೀತಿಭ್ರಷ್ಟರ ಆಚರಣೆಗಳನ್ನು ಮಾತ್ರ ಬಹಿರಂಗವಾಗಿ ಖಂಡಿಸಿದೆ. ಈಗಲೂ ಖಂಡಿಸುವೆ. ಆದಿಶಕ್ತಿ ಜಗನ್ಮಾತೆಯ ಅಂಶರೂಪವಾದ ಸ್ತ್ರೀಯರ ಪೂಜೆಯನ್ನು ನಾನೆಂದೂ ದೂಷಿಸಿಲ್ಲ. ಅವರ ಬಾಹ್ಯರೂಪವು ಗಂಡಸರನ್ನು ಹುಚ್ಚರನ್ನಾಗಿ ಮಾಡಿರಬಹುದು. ಆದರೆ ಅವರ ಆಂತರಿಕ ವಿಕಾಸ, ಜ್ಞಾನ, ಭಕ್ತಿ, ವಿವೇಚನಾಜ್ಞಾನ, ವೈರಾಗ್ಯ ಮುಂತಾದುವು ಮಾನವರನ್ನು ಸರ್ವಜ್ಞನನ್ನಾಗಿ ಮಾಡಿ ಅನಂತ ಬ್ರಹ್ಮನನ್ನು ಅರಿಯುವಂತೆ ಮಾಡುತ್ತವೆ. “ಶಕ್ತಿಯು ಸಂತುಷ್ಟಳಾದಾಗ ಪ್ರಸನ್ನಳಾಗಿ ಮಾನವನ ಬಂಧನ ವಿಮೋಚನೆಗೆ ಕಾರಣಳಾಗುವಳು." ಪೂಜೆ ವ್ರತಗಳಿಂದ ಮಾತೆ ಪ್ರಸನ್ನಳಾಗದಿದ್ದಲ್ಲಿ ಬ್ರಹ್ಮವಿಷ್ಣುಗಳೂ ಕೂಡ ಆಕೆಯ ಹಿಡಿತದಿಂದ ತಪ್ಪಿಸಿಕೊಂಡು ಬಂಧನವಿಮುಕ್ತರಾಗಲು ಸಾಧ್ಯವಿಲ್ಲ. ಆದ್ದರಿಂದಲೇ ಈ ಗೃಹಲಕ್ಷ್ಮಿಯರನ್ನು ಪೂಜಿಸಲು, ಅವರಲ್ಲಿ ಬ್ರಹ್ಮನು ವಿಕಾಸಗೊಳ್ಳಲು, ನಾನು ಸ್ತ್ರೀಯರ ಮಠ ಸ್ಥಾಪಿಸುವೆನು.

ಶಿಷ್ಯ: ಅದೊಂದು ಒಳ್ಳೆಯ ಭಾವನೆಯೇ ಸರಿ. ಆದರೆ ಅದಕ್ಕೆ ಸೇರಲು ನೀವು ಹೆಂಗಸರನ್ನೆಲ್ಲಿಂದ ತರುವಿರಿ? ಈ ಸಮಾಜದ ತೀವ್ರ ಕಟ್ಟುನಿಟ್ಟಿನಲ್ಲಿ ನಿಮ್ಮ ಮಠವನ್ನು ಸೇರಲು ತಮ್ಮ ಮನೆಯಿಂದ ಯಾರನ್ನು ತಾನೇ ಕಳುಹಿಸಲಿಚ್ಛಿಸುವರು.

ಸ್ವಾಮಿಜಿ: ಅದೇಕೆ? ಈಗಲೂ ಶ‍್ರೀರಾಮಕೃಷ್ಣರ ಸ್ತ್ರೀ ಶಿಷ್ಯರಿದ್ದಾರೆ. ಅವರ ಸಹಾಯದಿಂದ ನಾನೇ ಮಠವನ್ನು ಸ್ಥಾಪಿಸುವೆ. ಶ‍್ರೀ ಶಾರದಾ ದೇವಿ ಅದರ ಕೇಂದ್ರ. ಶ‍್ರೀರಾಮಕೃಷ್ಣರ ಭಕ್ತರ ಹೆಂಡಿರು ಮಕ್ಕಳು ಅದರ ಪ್ರಥಮ ನಿವಾಸಿಗಳಾಗುವರು. ಏಕೆಂದರೆ ಅವರು ಬೇಗ ಇಂತಹ ಮಠದ ಉಪಯೋಗವನ್ನು ಗ್ರಹಿಸುವರು. ನಂತರ ಅವರ ಮೇಲ್ಪಂಕ್ತಿಯನ್ನನುಸರಿಸಿ ಅನೇಕ ಗೃಹಸ್ಥರು ಇಂತಹ ಒಳ್ಳೆಯ ಕೆಲಸಕ್ಕೆ ಸಹಾಯಮಾಡಲು ಮುಂದೆ ಬರುವರು.

ಶಿಷ್ಯ: ಶ‍್ರೀರಾಮಕೃಷ್ಣರ ಭಕ್ತರು ಖಂಡಿತವಾಗಿ ಇದಕ್ಕೆ ಸೇರುವರು. ಆದರೆ ಸಾಮಾನ್ಯ ಜನರು ಇದಕ್ಕೆ ಅಷ್ಟು ಸಹಾಯ ನೀಡುವರೆಂದು ನನಗೆ ಅನ್ನಿಸುವುದಿಲ್ಲ.

ಸ್ವಾಮೀಜಿ: ಈ ಜಗತ್ತಿನಲ್ಲಿ ತ್ಯಾಗವಿಲ್ಲದೆ ಯಾವ ಕೆಲಸವೂ ಸಾಧ್ಯವಿಲ್ಲ. ಅಷ್ಟೊಂದು ಚಿಕ್ಕ ಆಲದ ಬೀಜವನ್ನು ನೋಡಿದವರು ಅದು ಕ್ರಮೇಣ ಅಷ್ಟು ದೊಡ್ಡ ವೃಕ್ಷವಾಗಿ ಬೆಳೆಯುವುದೆಂದು ಊಹಿಸಬಲ್ಲರೆ? ಸದ್ಯಕ್ಕೆ ನಾನು ಮಠವನ್ನು ಹೀಗೆ ಆರಂಭಿಸುವೆ, ಮುಂದೆ ಒಂದೆರಡು ಪೀಳಿಗೆಯ ನಂತರ ನೀನೇ ನೋಡುವೆ, ದೇಶದ ಜನರು ಅಂತಹ ಮಠದ ಬೆಲೆಯನ್ನರಿತು ಇದನ್ನು ಮೆಚ್ಚುವರು. ನನ್ನ ಸ್ತ್ರೀ ಶಿಷ್ಯರು ಈ ಕೆಲಸಕ್ಕಾಗಿ ತಮ್ಮ ತನುಮನವನ್ನು ಅರ್ಪಿಸುವರು. ನೀವು ಭಯ, ಹೇಡಿತನಗಳನ್ನು ಬಿಸಾಡಿ. ಈ ಪವಿತ್ರ ಕೆಲಸಕ್ಕೆ ಸಹಾಯಕರಾಗಿ, ಈ ಉಚ್ಚತಮ ಆದರ್ಶವನ್ನು ಎಲ್ಲರ ಮುಂದೆಯೂ ಎತ್ತಿ ಹಿಡಿಯಿರಿ. ನೋಡುತ್ತಿರು, ಅದು ಇಡೀ ದೇಶದ ಮೇಲೆ ತನ್ನ ಕಾಂತಿಯನ್ನು ಬೀರುವುದು.

ಶಿಷ್ಯ: ಸ್ವಾಮೀಜಿ, ದಯವಿಟ್ಟು ನೀವು ಈ ಸ್ತ್ರೀ ಮಠಕ್ಕಾಗಿ ಯಾವ ಯೋಜನೆಗಳನ್ನು ತಯಾರಿಸಿರುವಿರಿ ಎಂಬುದನ್ನು ಹೇಳಿ.

ಸ್ವಾಮೀಜಿ: ಗಂಗಾನದಿಯ ಎದುರುಪಾರ್ಶ್ವದಲ್ಲಿ ಒಂದು ದೊಡ್ಡ ಪ್ರದೇಶವನ್ನು ಇದಕ್ಕಾಗಿ ಪಡೆಯಲಾಗುವುದು. ಅಲ್ಲಿ ಮದುವೆಯಾಗದ ಹುಡುಗಿಯರು ಅಥವಾ ಬ್ರಹ್ಮಚಾರಿಣಿಯರಾದ ವಿಧವೆಯರು ವಾಸಿಸುವರು. ನಿಷ್ಠರಾದ ಗೃಹಿಣಿಯರೂ ಆಗಾಗ್ಗೆ ಅಲ್ಲಿಗೆ ಬಂದಿರಲು ಅವಕಾಶವಿರುವುದು. ಗಂಡಸರಿಗೆ ಆ ಮಠದೊಡನೆ ಯಾವ ಸಂಪರ್ಕವೂ ಇರುವುದಿಲ್ಲ. ಮಠದ ಹಿರಿಯ ಸ್ವಾಮಿಗಳು ದೂರದಿಂದಲೇ ಜವಾಬ್ದಾರಿಯನ್ನು ನೋಡಿಕೊಳ್ಳುವರು. ಈ ಸ್ತ್ರೀ ಮಠಕ್ಕೆ ಸೇರಿದಂತೆ ಒಂದು ವಸತಿ ಶಾಲೆ ಇರುತ್ತದೆ - ಅದರಲ್ಲಿ ಧಾರ್ಮಿಕ ಗ್ರಂಥಗಳು, ಸಾಹಿತ್ಯ, ಸಂಸ್ಕೃತ, ವ್ಯಾಕರಣ, ಕೊಂಚ ಇಂಗ್ಲಿಷ್ ಭಾಷೆ ಕೂಡ ಕಲಿಸಲ್ಪಡುವುದು. ಹೊಲಿಗೆ, ಪಾಕಶಾಸ್ತ್ರ, ಗೃಹ ಕೆಲಸಗಳ ನಿಯಮ, ಮಕ್ಕಳ ಶಿಕ್ಷಣ ಮುಂತಾದ ಇತರ ವಿಷಯಗಳನ್ನು ಕಲಿಸುವರು. ಜಪ, ಧ್ಯಾನ, ಪೂಜೆ ಮುಂತಾದುವೂ ಈ ಬೋಧನೆಯ ಆವಶ್ಯಕವಾದ ಅಂಗವಾಗಿರುವುದು. ಯಾರು ಮನೆಮಠಗಳನ್ನೆಲ್ಲಾ ಬಿಟ್ಟುಬಂದು ಇಲ್ಲೇ ಚಿರಸ್ಥಾಯಿಯಾಗಿ ನಿಲ್ಲುವರೋ ಅವರಿಗೆ ಮಠವೇ ಅನ್ನ ವಸ್ತ್ರಗಳನ್ನೊದಗಿಸುವುದು. ಯಾರಿಗೆ ಹೀಗಿರಲಾಗುವುದಿಲ್ಲವೋ ಅವರು ಈ ಮಠದಲ್ಲೇ ಹಗಲಿನಲ್ಲಿ ವಿದ್ಯಾರ್ಥಿನಿಯರಂತೆ ಕಲಿಯಲು ಅವಕಾಶವಿರುತ್ತದೆ. ಮಠದ ಅಧ್ಯಕ್ಷಿಣಿಯ ಅನುಮತಿಯಿಂದ ಅಂತಹವರು ಆಗಾಗ್ಗೆ ಮಠದಲ್ಲೇ ಇರಲು ಅವಕಾಶ ಕಲ್ಪಿಸಲಾಗುವುದು. ಆಗ ಮಠದವರೇ ಅವರ ಖರ್ಚನ್ನು ನೋಡಿಕೊಳ್ಳುವರು. ಹಿರಿಯ ಬ್ರಹ್ಮಚಾರಿಣಿಯರು ಈ ವಿದ್ಯಾರ್ಥಿನಿಯರಿಗೆ ಬ್ರಹ್ಮಚರ್ಯ ದೀಕ್ಷೆ ಕೊಡುವ ಹೊಣೆ ವಹಿಸುವರು. ಈ ಮಠದಲ್ಲಿ ೪-೫ ವರ್ಷ ಶಿಕ್ಷಣ ಪಡೆದ ತರುವಾಯ ಹುಡುಗಿಯರ ಪೋಷಕರು ಅವರ ಮದುವೆ ಮಾಡಬಹುದು. ಯೋಗ ಮತ್ತು ಆಧ್ಯಾತ್ಮಿಕ ಜೀವನವನ್ನು ಇಚ್ಛಿಸುವುದಾದರೆ ಅವರ ಪೋಷಕರ ಅನುಮತಿಯಿಂದ ಅವರು ಮಠದಲ್ಲೇ ಇದ್ದು ಆಜನ್ಮ ಬ್ರಹ್ಮಚರ್ಯ ದೀಕ್ಷೆ ಪಡೆಯುವರು. ಕಾಲಕ್ರಮೇಣ ಈ ಅವಿವಾಹಿತ ಸಂನ್ಯಾಸಿನಿಯರೇ ಮುಂದೆ ಮಠದ ಬೋಧಕರು ಮತ್ತು ಅಧ್ಯಾಪಕರಾಗುವರು. ಪಟ್ಟಣ ಮತ್ತು ಹಳ್ಳಿಗಳಲ್ಲಿ ಅವರು ಕೇಂದ್ರಗಳನ್ನು ತೆರೆದು ಸ್ತ್ರೀ ವಿದ್ಯಾಭ್ಯಾಸ ಹರಡಲು ಪ್ರಯತ್ನಿಸುವರು. ಎಲ್ಲಿಯವರೆಗೆ ವಿದ್ಯಾರ್ಥಿನಿಯರು ಈ ಮಠದ ಸಹವಾಸದಲ್ಲಿರುವರೋ ಅಲ್ಲಿಯವರೆಗೂ ಅವರು ಮಠದ ಮೂಲತತ್ತ್ವವಾದ ಬ್ರಹ್ಮಚರ್ಯವನ್ನು ಆಚರಿಸಬೇಕು.

ಮಠದ ವಿದ್ಯಾರ್ಥಿನಿಯರಿಗೆ ಆಧ್ಯಾತ್ಮಿಕತೆ, ತ್ಯಾಗ ಮತ್ತು ನಿಗ್ರಹ ಇವೇ ಧ್ಯೇಯವಾಗಿರಬೇಕು. ಸೇವಾಧರ್ಮ ಅವರ ಜೀವನದ ವ್ರತವಾಗಿರಬೇಕು. ಇಂತಹ ಆದರ್ಶ ಧ್ಯೇಯವಿರುವಾಗ ಯಾರು ತಾನೇ ಅದಕ್ಕೆ ಗೌರವ ಕೊಟ್ಟು ಶ್ರದ್ಧೆ ಇಡುವುದಿಲ್ಲ! ಈ ದೇಶದ ಸ್ತ್ರೀಯರ ಜೀವನ ಈ ರೀತಿ ರೂಪುಗೊಳ್ಳುವುದಾದರೆ ಸೀತೆ, ಸಾವಿತ್ರಿ, ಗಾರ್ಗಿಯರಂತಹ ಆದರ್ಶ ವ್ಯಕ್ತಿಗಳನ್ನು ಮತ್ತೊಮ್ಮೆ ನಮ್ಮ ದೇಶದಲ್ಲಿ ಕಾಣಬಹುದು. ಸಮಾಜದ ನಿಯಮಗಳು ಎಂತಹ ಸ್ಥಿತಿಗೆ ನಮ್ಮ ದೇಶದ ಹೆಂಗಸರನ್ನು ಇಳಿಸಿವೆ, ಎಷ್ಟರಮಟ್ಟಿಗೆ ಅವರನ್ನು ಜೀವಚ್ಛವಗಳನ್ನಾಗಿ ಜಡರನ್ನಾಗಿ ಮಾಡಿವೆ ಎಂಬುದು ನೀನು ಪಾಶ್ಚಾತ್ಯ ದೇಶಗಳಿಗೆ ಹೋದರೆ ಗೊತ್ತಾಗುವುದು. ಅವರ ಇಂತಹ ಶೋಚನೀಯಾವಸ್ಥೆಗೆ ನೀವೇ ಕಾರಣ. ಅವರನ್ನು ಪುನಃ ಮೇಲೆತ್ತುವುದೂ ನಿಮ್ಮನ್ನೇ ಅವಲಂಬಿಸಿದೆ. ಆದ್ದರಿಂದಲೇ ನಾನು ಹೇಳುವುದು - ಕಾರ್ಯತತ್ಪರರಾಗಿ ಎಂದು. ವೇದ ಮುಂತಾದುವುಗಳನ್ನು ಗಿಣಿಪಾಠ ಮಾಡುವುದರಿಂದ ಏನಾಗುವುದು?

ಶಿಷ್ಯ: ಸ್ವಾಮೀಜಿ, ಈ ಮಠದಲ್ಲಿ ತರಬೇತಾದ ಹುಡುಗಿಯರು ಮದುವೆಯಾದರೆ ಆದರ್ಶಶೀಲವತಿಯರೆಂದು ಹೇಗೆ ಗೊತ್ತಾಗುವುದು? ಈ ಮಠದಲ್ಲಿ ವಿದ್ಯಾವತಿಯಾದವರೆಲ್ಲ ಮದುವೆಯಾಗಕೂಡದೆಂಬ ನಿಯಮವನ್ನೇ ತಂದರೆ ಒಳ್ಳೆಯದಲ್ಲವೆ?

ಸ್ವಾಮೀಜಿ: ಅದನ್ನು ತಕ್ಷಣವೇ ಕಾರ್ಯರೂಪಕ್ಕೆ ತರಲು ಸಾಧ್ಯವೆ? ಅವರಿಗೆ ಮೊದಲು ವಿದ್ಯಾಭ್ಯಾಸ ಕೊಟ್ಟು ಅವರ ಇಷ್ಟಕ್ಕೆ ಬಿಡಬೇಕು. ನಂತರ ಅವರು ತಮಗೆ ಸೂಕ್ತ ಕಂಡಂತೆ ಮಾಡಲಿ. ಮದುವೆಯಾಗಿ ಗೃಹಸ್ಥರಾದಮೇಲೂ ಹೀಗೆ ಶಿಕ್ಷಣ ಹೊಂದಿದ ಹುಡುಗಿಯರು ತಮ್ಮ ಗಂಡಂದಿರನ್ನು ಈ ಉಚ್ಚತಮ ಆದರ್ಶದಿಂದ ಸ್ಫೂರ್ತಿನೀಡಿ ವೀರಮಾತೆಯರಾಗುವರು. ಆದರೆ ವಿದ್ಯಾರ್ಥಿನಿಯರ ಪೋಷಕರು ಅವರಿಗೆ ೧೫ ವರ್ಷಗಳಿಗೆ ಮುಂಚೆ ಮದುವೆ ಮಾಡುವ ಯೋಚನೆಯನ್ನೇ ಮಾಡಬಾರದು ಎಂಬ ನಿಯಮ ಮಾತ್ರ ಇದ್ದೇ ತೀರಬೇಕು.

ಶಿಷ್ಯ: ಆದರೆ ಅಂತಹ ವಿದ್ಯಾರ್ಥಿನಿಯರನ್ನು ಸಮಾಜ ಗೌರವದಿಂದ ನೋಡುವುದಿಲ್ಲ. ಯಾರೂ ಅವರನ್ನು ಮದುವೆಯಾಗಲಿಚ್ಛಿಸುವುದಿಲ್ಲ.

ಸ್ವಾಮೀಜಿ: ಅವರನ್ನೇಕೆ ಮದುವೆಯಾಗಲಿಚ್ಛಿಸುವುದಿಲ್ಲ? ನೀನಿನ್ನೂ ಸಮಾಜದ ಪ್ರವೃತ್ತಿಯನ್ನೇ ಗ್ರಹಿಸಿಲ್ಲ. ವಿದ್ಯಾವಂತೆಯಾದ ಕುಶಲಿಯಾದ ಹುಡುಗಿಯರಿಗೆ ವರ ಸಿಗಲು ಖಂಡಿತ ಕಷ್ಟವಾಗುವುದಿಲ್ಲ. ಬಾಲ್ಯ ವಿವಾಹ ಪದ್ಧತಿ ಮುಂತಾದುವನ್ನು ಸಾರುತ್ತಿದ್ದ ಶಾಸ್ತ್ರಗಳನ್ನು ಸಮಾಜ ಈಗ ಖಂಡಿತ ಪಾಲಿಸುವುದಿಲ್ಲ. ಈಗಾಗಲೇ ನಿನಗಿದು ಕಾಣಿಸುತ್ತಿಲ್ಲವೆ?

ಶಿಷ್ಯ: ಆದರೆ ಖಂಡಿತವಾಗಿಯೂ ಪ್ರಾರಂಭದಲ್ಲಿ ಇವರಿಗೆ ತೀವ್ರ ವಿರೋಧ ಬರುವುದರಲ್ಲಿ ಸಂದೇಹವೇ ಇಲ್ಲ.

ಸ್ವಾಮೀಜಿ: ಬರಲಿ? ಅದಕ್ಕೆ ಹೆದರಬೇಕಾದ ಆವಶ್ಯಕತೆಯೇನಿದೆ? ನೈತಿಕ ಧೈರ್ಯದಿಂದ ಪ್ರೇರೇಪಿತರಾಗಿ ಮಾಡುವ ಒಳ್ಳೆಯ ಕೆಲಸಕ್ಕೆ ಎಷ್ಟೇ ಅಡ್ಡಿ ಆತಂಕಗಳು ಬಂದರೂ ಅದು ಅವರ ನೈತಿಕ ಶಕ್ತಿಯನ್ನು ಮತ್ತಷ್ಟು ಜಾಗೃತಗೊಳಿಸುವುದು. ಯಾವುದು ಯಾವ ಅಡ್ಡಿ ಆತಂಕಗಳಿಗೂ ವಿರೋಧಕ್ಕೂ ಸಿಗುವುದಿಲ್ಲವೋ ಅದು ಮಾನವನಿಗೆ ನೈತಿಕ ಮೃತ್ಯು. ಹೋರಾಟವೇ ಜೀವನದ ಚಿಹ್ನೆ.

ಶಿಷ್ಯ: ಹೌದು ಸ್ವಾಮಿಜಿ.

ಸ್ವಾಮೀಜಿ: ಪರಬ್ರಹ್ಮನ ಪರಾಕಾಷ್ಠೆಯಲ್ಲಿ ಲಿಂಗಭೇದವಿಲ್ಲ. ಕೇವಲ ಈ ಕಾರ್ಯಕಾರಣ ಸಂಬಂಧದ ಪ್ರಪಂಚದಲ್ಲಿ ನಾವದನ್ನು ನೋಡುವೆವು. ಮನಸ್ಸು ಅಂತರ್ಮುಖವಾದಷ್ಟೂ ಈ ಭೇದಭಾವನೆ ನಾಶವಾಗುವುದು. ಕಟ್ಟ ಕಡೆಗೆ ಯಾವಾಗ ಮನಸ್ಸು ಲಿಂಗಾತೀತವಾದ, ಅಭಿನ್ನವಾದ ಬ್ರಹ್ಮನಲ್ಲಿ ತನ್ಮಯವಾಗುವುದೋ ಆಗ ಈತ ಗಂಡಸು ಆಕೆ ಹೆಂಗಸು ಎಂಬ ಭಾವನೆಗಳಿಗೆ ಸ್ಥಳವೇ ಇರುವುದಿಲ್ಲ. ಶ‍್ರೀರಾಮಕೃಷ್ಣರ ಜೀವನದಲ್ಲಿ ಇದನ್ನು ಕಣ್ಣಾರೆ ಕಂಡಿರುವೆವು. ಪುರುಷನಾದವನು ಬ್ರಹ್ಮಜ್ಞಾನ ಪಡೆಯಬಲ್ಲವನಾದರೆ ಸ್ತ್ರೀಯು ಏಕೆ ಆ ಜ್ಞಾನ ಪಡೆಯಲಾಗದು? ಆದ್ದರಿಂದಲೇ ಸ್ತ್ರೀಯರಲ್ಲಿ ಓರ್ವಳು ಬ್ರಹ್ಮಜ್ಞಾನಿಯಾದರೂ ಆಕೆಯ ವ್ಯಕ್ತಿತ್ವದ ಪ್ರಭಾವದಿಂದ ಸಾವಿರಾರು ಮಂದಿ ಸ್ತ್ರೀಯರು ಸ್ಫೂರ್ತಿ ಹೊಂದಿ ಸತ್ಯವನ್ನರಿತು ಇಡೀ ದೇಶಕ್ಕೆ ಕಲ್ಯಾಣವಾಗಿ ಸಮಾಜವೂ ಭದ್ರವಾಗುವುದು. ಅರ್ಥವಾಯಿತೆ?

ಶಿಷ್ಯ: ಸ್ವಾಮೀಜಿ, ನಿಮ್ಮ ಮಾತುಗಳಿಂದ ಇಂದು ನನ್ನ ಕಣ್ತೆರೆದಂತಾಗಿದೆ.

ಸ್ವಾಮಿಜಿ: ಪೂರ್ತಿ ತೆರೆದಿಲ್ಲ. ಸಮಸ್ತವನ್ನೂ ಅರಿಯುವ ಆತ್ಮಜ್ಞಾನವನ್ನು ಎಂದು ಪಡೆಯುವೆಯೊ ಆಗ ನಿನಗೆ ಈ ಲಿಂಗಭಾವನೆ ಸಂಪೂರ್ಣವಾಗಿ ಅಳಿಸಿ ಹೋಗುವುದು. ಆಗ ಮಾತ್ರ ನೀನು ಎಲ್ಲಾ ಸ್ತ್ರೀಯರನ್ನೂ ಬ್ರಹ್ಮನ ಅವತಾರವೆಂದು ಭಾವಿಸಲು ಶಕ್ತನಾಗುವೆ. ಶ‍್ರೀರಾಮಕೃಷ್ಣರು ಹೇಗೆ ಪ್ರತಿಯೊಬ್ಬ ನಾರಿಯಲ್ಲೂ ಜಗನ್ಮಾತೆಯ ಅಂಶವನ್ನು, ಅವರು ಯಾವ ಜಾತಿಗೆ ಸೇರಿರಲಿ, ಎಂತಹ ಸ್ವಭಾವದವರೇ ಆಗಿರಲಿ, ಕಾಣುತ್ತಿದ್ದರೆಂಬುದನ್ನು ನಾವು ನೋಡಿದ್ದೇವೆ. ಇದನ್ನು ನೋಡಿರುವುದರಿಂದಲೇ ಇಷ್ಟೊಂದು ತೀವ್ರವಾಗಿ ನಿಮಗೆಲ್ಲಾ ಗ್ರಾಮ ಗ್ರಾಮಗಳಲ್ಲೂ ವಿದ್ಯಾರ್ಥಿನಿಯರಿಗೆ ಶಾಲೆಗಳನ್ನು ತೆರೆದು ಅವರನ್ನು ಉದ್ಧರಿಸಬೇಕೆಂದು ಹೇಳುವುದು. ಹೆಂಗಸರು ಮುಂದಕ್ಕೆ ಬಂದರೆ ಅವರ ಪುತ್ರರೂ ಅವರ ಸುಗುಣಗಳ ಪ್ರಭಾವದಿಂದ ದೇಶಕ್ಕೆ ಕೀರ್ತಿ ತರುವರು. ಆಗ ಸಂಸ್ಕೃತಿ ಜ್ಞಾನ ಶಕ್ತಿ ಭಕ್ತಿಗಳು ದೇಶಾದ್ಯಂತ ಅಭಿವೃದ್ಧಿಗೆ ಬರುವುವು.

ಶಿಷ್ಯ: ವರ್ತಮಾನಕಾಲದ ಹೆಂಗಸರ ವಿದ್ಯಾಭ್ಯಾಸದಿಂದ ವಿರುದ್ಧ ಫಲಿತಾಂಶಗಳನ್ನು ನೋಡುತ್ತಿದ್ದೇವೆ. ಈ ಅಲ್ಪಸ್ವಲ್ಪ ವಿದ್ಯೆಯಿಂದಲೇ ಅವರು ಪಾಶ್ಚಾತ್ಯ ನಡವಳಿಕೆಯನ್ನು ಅನುಕರಿಸಲು ಹೊರಡುವರು. ಆದರೆ ಎಷ್ಟರಮಟ್ಟಿಗೆ ತ್ಯಾಗಬುದ್ಧಿ, ಆತ್ಮನಿಗ್ರಹ, ಬ್ರಹ್ಮಚರ್ಯ, ತಪಸ್ಸು ಮುಂತಾದ ಗುಣಗಳು ಇವರಿಗೆ ಬರುವುದೊ ನೋಡಬೇಕು.

ಸ್ವಾಮಿಜಿ: ಪ್ರಾರಂಭದಲ್ಲಿ ಅಂತಹ ದೋಷಗಳು ಅನಿವಾರ್ಯ. ದೇಶದಲ್ಲಿ ಯಾವಾಗ ಒಂದು ಹೊಸ ಭಾವನೆಯನ್ನು ಪ್ರಚಾರಮಾಡಲು ಹೊರಡುವಿರೋ ಕೆಲವರು ಅದನ್ನು ಗ್ರಹಿಸಲಾರದೆ ವಿರೋಧ ಮಾರ್ಗ ಅನುಸರಿಸುವರು. ಆದರೆ ಇಡೀ ಸಮಾಜಕ್ಕೆ ಅದರಿಂದ ಆಗುವ ಬಾಧಕವೇನು? ದೇಶದಲ್ಲಿ ಈಗ ಕೊಂಚ ಪ್ರಚಾರದಲ್ಲಿರುವ ಸ್ತ್ರೀ ವಿದ್ಯಾಭ್ಯಾಸದ ಮೂಲಕರ್ತರು ನಿಸ್ಸಂಶಯವಾಗಿ ಉದಾರ ಹೃದಯರು. ಆದರೆ ನಿಜಕ್ಕೂ ಯಾವ ಶಿಕ್ಷಣ ಅಥವಾ ಸಂಸ್ಕೃತಿಯೇ ಆಗಲಿ ಅದಕ್ಕೆ ಆಧ್ಯಾತ್ಮಿಕ ತಳಹದಿ ಇಲ್ಲದೆ ಸ್ಥಾಪಿತವಾದಲ್ಲಿ ಏನಾದರೊಂದು ಕುಂದು ಅಲ್ಲಿ ಪ್ರವೇಶಿಸಿಯೇ ತೀರುವುದು. ಈಗಿನ ಸ್ತ್ರೀ ವಿದ್ಯಾಭ್ಯಾಸದಲ್ಲಿ ಧರ್ಮವೇ ಅದರ ಕೇಂದ್ರವಾಗಿರಬೇಕು. ಧಾರ್ಮಿಕ ಶಿಕ್ಷಣ, ಶೀಲ ನಿರ್ಮಾಣ, ಬ್ರಹ್ಮಚರ್ಯ ವ್ರತಾಚರಣೆ ಇವುಗಳಿಗೂ ಗೌರವ ಕೊಡಬೇಕು. ಇದುವರೆಗೆ ಇಂಡಿಯಾದಲ್ಲಿ ಸ್ತ್ರೀವಿದ್ಯಾಭ್ಯಾಸದಲ್ಲಿ ಧರ್ಮವು ಆನುಷಂಗಿಕವಾಗಿರುವುದರಿಂದ ಇತರ ಕುಂದುಕೊರತೆಗಳಿಗೆ ಕಾರಣವಾಗಿದೆ. ಆದರೆ ಇದರಲ್ಲಿ ಹೆಂಗಸರ ತಪ್ಪೇನೂ ಇಲ್ಲ. ಬ್ರಹ್ಮಚಾರಿಣಿಯರಿಲ್ಲದೆ ಈ ಸ್ತ್ರೀ ವಿದ್ಯಾಭ್ಯಾಸ ಆರಂಭಿಸಿದುದರಿಂದ ಈ ಸುಧಾರಕರು ಈ ರೀತಿ ಎಡವಿದ್ದಾರೆ. ಯಾವ ಒಳ್ಳೆಯ ಕಾರ್ಯ ಪ್ರವರ್ತಕರೇ ಆಗಲಿ ತಾವಿಚ್ಛಿಸಿದ ಕೆಲಸಕ್ಕೆ ಪ್ರಾರಂಭಿಸುವ ಮೊದಲು ಕಠಿಣ ತಪಶ್ಚರ್ಯೆಯಿಂದ ಆತ್ಮಜ್ಞಾನವನ್ನು ಹೊಂದಿರಬೇಕು. ಇಲ್ಲದಿದ್ದಲ್ಲಿ ಅವರ ಕೆಲಸಗಳಲ್ಲಿ ಕುಂದು ಇದ್ದೇ ತೀರುವುದು.

ಶಿಷ್ಯ: ಹೌದು ಸ್ವಾಮೀಜಿ, ಅನೇಕ ವಿದ್ಯಾವತಿಯರಾದ ಮಹಿಳೆಯರು ಕೇವಲ ಕಾದಂಬರಿ ಮುಂತಾದ ಪುಸ್ತಕಗಳನ್ನೋದುತ್ತಾ ಕಾಲಹರಣ ಮಾಡುತ್ತಿದ್ದಾರಂತೆ. ಆದರೆ ಪೂರ್ವ ಬಂಗಾಳದಲ್ಲಿ ವಿದ್ಯಾಭ್ಯಾಸದಿಂದ ಅವರೇನು ತಮ್ಮ ಧಾರ್ಮಿಕತೆ ಅನುಸರಿಸುವುದನ್ನು ಬಿಟ್ಟಿಲ್ಲ. ಈ ಭಾಗದಲ್ಲಿ ಹೇಗಿದೆ?

ಸ್ವಾಮಿಜಿ: ಎಲ್ಲಾ ದೇಶದ ಜನಾಂಗಗಳೂ ಒಳ್ಳೆಯದು ಮತ್ತು ಕೆಟ್ಟುದು ಎರಡನ್ನೂ ಹೊಂದಿರುವುವು. ನಮ್ಮ ಜೀವಮಾನದಲ್ಲಿ ನಾವು ಒಳ್ಳೆಯ ಕೆಲಸವನ್ನು ಮಾಡಬೇಕು; ಇತರರಿಗೆ ಮಾದರಿಯಾಗಬೇಕು. ಇದೇ ನಮ್ಮ ಕರ್ತವ್ಯ. ಬರೀ ದೋಷಾರೋಪಣೆಯಿಂದ ಯಾವ ಕಾರ್ಯವೂ ಸಾಗುವುದಿಲ್ಲ. ಜನರನ್ನು ಅದು ಕೇವಲ ಹಿಮ್ಮೆಟ್ಟುವಂತೆ ಮಾಡುವುದು. ಯಾರು ಏನು ಬೇಕಾದರೂ ಹೇಳಿಕೊಳ್ಳಲಿ, ಅದಕ್ಕೆ ಪ್ರತಿ ಹೇಳಬೇಡ. ಈ ಮಾಯಾ ಪ್ರಪಂಚದಲ್ಲಿ ನೀನಾವ ಕೆಲಸವನ್ನೇ ಪ್ರಾರಂಭಿಸು ಅದರಲ್ಲಿ ಯಾವುದಾದರೊಂದು ದೋಷ ಇದ್ದೇ ಇರುವುದು. ‘ಸರ್ವಾರಂಭಾ ಹಿ ದೋಷೇಣ ಧೂಮೇನಾಗ್ನಿರಿವಾವೃತಾಃ’ ಬೆಂಕಿಯಲ್ಲಿ ಹೊಗೆ ಇರುವಂತೆ ಎಲ್ಲಾ ಕೆಲಸಗಳಲ್ಲೂ ದೋಷಗಳಿರುವುವು. ಯಾವ ಬೆಂಕಿಯಲ್ಲೇ ಆಗಲಿ ಹೊಗೆ ಬರುವ ಸಂಭವವಿದೆ. ಆದರೆ ಅದಕ್ಕಾಗಿ ನೀನು ಕೈಕಟ್ಟಿ ಕುಳಿತುಕೊಳ್ಳುವೆಯೇನು? ನಿನ್ನ ಕೈಲಾದಷ್ಟೂ ನೀನು ಸತ್ಕಾರ್ಯನಿರತನಾಗು.

ಶಿಷ್ಯ: ಈ ಸತ್ಕಾರ್ಯ ಯಾವುದು?

ಸ್ವಾಮೀಜಿ: ಯಾವುದು ಬ್ರಹ್ಮನ ವಿಕಾಸಕ್ಕೆ ನೆರವಾಗುವುದೋ ಅದು ಸತ್ಕಾರ್ಯ. ಯಾವ ಕೆಲಸವನ್ನೇ ಆಗಲಿ ಪ್ರತ್ಯಕ್ಷವಾಗಿ ಅಲ್ಲದಿದ್ದರೆ ಪರೋಕ್ಷವಾಗಿಯಾದರೂ ಆತ್ಮನ ವಿಕಾಸಕ್ಕೆ ಸಹಾಯವಾಗುವಂತೆ ಮಾಡಬಹುದು. ಋಷಿಗಳು ಸಾರಿರುವ ಮಾರ್ಗವನ್ನನುಸರಿಸಿದರೆ ಆತ್ಮಜ್ಞಾನವು ಶೀಘ್ರವಾಗಿ ವಿಕಾಸಗೊಳ್ಳುವುದು. ಇಲ್ಲದಿದ್ದಲ್ಲಿ ಧಾರ್ಮಿಕ ಗ್ರಂಥಕರ್ತರು ಯಾವುದನ್ನು ತಪ್ಪೆಂದು ನಿರೂಪಿಸಿರುವರೊ ಅವುಗಳನ್ನು ಮಾಡುವುದು ಆತ್ಮದ ಬಂಧನಕ್ಕೆ ಕಾರಣವಾಗುತ್ತದೆ. ಅನೇಕ ವೇಳೆ ಈ ಮಾಯಾಬಂಧನಗಳಲ್ಲಿಯೂ ಜೀವನಿಗೆ ಮುಕ್ತಿ ಕೊನೆಗೆ ಸಿಕ್ಕೇಸಿಗುವುದು. ಏಕೆಂದರೆ ಜೀವನದ ಸತ್ಯಭಾವವೇ ಆತ್ಮ. ಯಾರಾದರೂ ತಮ್ಮ ಸ್ವಭಾವವನ್ನು ಬಿಡಲು ಸಾಧ್ಯವೆ? ನಿನ್ನ ನೆರಳಿನೊಡನೆ ನೀನು ಸಾವಿರ ವರ್ಷ ಯುದ್ಧ ಮಾಡಿದರೆ ತಾನೇ ಅದನ್ನು ಓಡಿಸಬಲ್ಲೆಯಾ? – ಅದು ಯಾವಾಗಲೂ ನಿನ್ನೊಡನಿದ್ದೇ ತೀರುವುದು.

ಶಿಷ್ಯ: ಆದರೆ ಸ್ವಾಮೀಜಿ, ಶಂಕರಾಚಾರ್ಯರ ಪ್ರಕಾರ ಕರ್ಮವು ಜ್ಞಾನಕ್ಕೆ ವಿರೋಧ. ಜ್ಞಾನ ಮತ್ತು ಕರ್ಮದ ಮಿಲನವನ್ನು ಅವರು ಅನೇಕ ರೀತಿ ಖಂಡಿಸಿದ್ದಾರೆ. ಜ್ಞಾನವಿಕಾಸಕ್ಕೆ ಕರ್ಮ ಹೇಗೆ ಸಹಕಾರಿಯಾಗುವುದು?

ಸ್ವಾಮೀಜಿ: ಶಂಕರರು ಹಾಗೆ ಹೇಳಿದ ಮೇಲೆ, ಪುನಃ, ಕರ್ಮವು ಜ್ಞಾನದ ಆವಿರ್ಭಾವನೆಗೆ ಪರೋಕ್ಷವಾಗಿ ಸಹಾಯಕಾರಿಯಾಗುವುದು, ಮನಸ್ಸನ್ನು ಪರಿಶುದ್ಧಗೊಳಿಸಲು ಸಹಾಯ ಮಾಡುವುದು ಎಂದು ವಿವರಿಸಿದ್ದಾರೆ. ಆದರೆ ಅತೀಂದ್ರಿಯ ಜ್ಞಾನದಲ್ಲಿ ಕರ್ಮದ ಸಂಪರ್ಕವು ಲೇಶಮಾತ್ರವೂ ಇರುವುದಿಲ್ಲ ಎಂದು ಅವರು ಹೇಳಿರುವುದಕ್ಕೆ ನಾನು ಪ್ರತಿ ಹೇಳುವುದಿಲ್ಲ. ಎಲ್ಲಿಯವರೆಗೆ ಮನುಷ್ಯನು ತಾನು ಮಾಡುತ್ತಿರುವ ಕರ್ಮದ ಕರ್ತೃವು ನಾನೇ ಎಂಬ ಮತ್ತು ಆ ಕರ್ಮಫಲ ತನ್ನದು ಎಂಬ ಪ್ರಜ್ಞೆ ಇರುತ್ತದೋ ಅಲ್ಲಿಯವರೆಗೂ ಅವನು ಸುಮ್ಮನೆ ಸೋಮಾರಿಯಾಗಿ ಕೆಲಸಮಾಡದೆ ಇರಲು ಸಾಧ್ಯವಿಲ್ಲ. ಮಾನವ ಸ್ವಭಾವದಲ್ಲಿ ಕರ್ಮವು ರಕ್ತಗತವಾಗಿದೆ. ಆದ್ದರಿಂದ ಆತ್ಮವಿಕಾಸಕ್ಕೆ ಸಹಕಾರಿಯಾದ ಸತ್ಕಾರ್ಯಗಳನ್ನು ನೀವೇಕೆ ಮಾಡಕೂಡದು? ಕೇವಲ ದೃಷ್ಟಿಯಿಂದ ನೋಡಿದರೆ ಎಲ್ಲಾ ಕರ್ಮವೂ ಅಜ್ಞಾನದ ಫಲವೆಂಬುದು ನಿಶ್ಚಯ. ಆದರೆ ಕಾರ್ಯಕಾರಣ ಸಂಬಂಧವಿರುವವರೆಗೆ ಅದರಿಂದ ಬಹಳ ಉಪಯೋಗವಿದೆ. ನೀನು ಆತ್ಮಸಾಕ್ಷಾತ್ಕಾರ ಮಾಡಿಕೊಂಡಾಗ ಕೆಲಸ ಮಾಡುವುದು ಮಾಡದೆ ಇರುವುದು ಎರಡೂ ನಿನ್ನ ಅಂಕೆಯಲ್ಲಿರುವುದು. ಆ ಅವಸ್ಥೆಯಲ್ಲಿ ನೀನು ಏನು ಮಾಡಿದರೂ ಅದು ಸತ್ಕಾರ್ಯವೇ ಆಗಿರುತ್ತದೆ. ಜೀವ ಜಗತ್ತಿಗೆ ಕಲ್ಯಾಣಕಾರಿಯಾಗುತ್ತದೆ. ಬ್ರಹ್ಮದ ಆವಿರ್ಭಾವನೆಯಾದ ಮೇಲೆ ನೀನು ಉಸಿರಾಡುವುದೂ ಕೂಡ ಜೀವರಿಗೆ ಒಳ್ಳೆಯದಾಗುವುದಕ್ಕೆ. ನಂತರ ನೀನು ಪ್ರಜ್ಞಾಪೂರ್ವಕವಾಗಿ ಯೋಜನೆಗಳ ಮೂಲಕ ಕೆಲಸಮಾಡಬೇಕಾದ ಪ್ರಮೇಯವೇ ಇರುವುದಿಲ್ಲ - ನಿನಗೆ ಅರ್ಥವಾಗುತ್ತಿದೆಯೆ?

ಶಿಷ್ಯ: ಆಗುತ್ತಿದೆ. ಇದು ವೇದಾಂತ ದೃಷ್ಟಿಯಿಂದ ಕರ್ಮ, ಜ್ಞಾನಗಳ ಸಮನ್ವಯದ ಸುಂದರವಾದ ಉಪಸಂಹಾರ.

ಈ ವೇಳೆಗೆ ಊಟದ ಗಂಟೆ ಹೊಡೆಯಿತು. ಶಿಷ್ಯನು ಅದಕ್ಕೆ ಹೋಗುವ ಮುನ್ನ ಕೈಜೋಡಿಸಿ ಪ್ರಾರ್ಥಿಸಿದ: “ಸ್ವಾಮೀಜಿ, ನಾನು ಈ ಜನ್ಮದಲ್ಲೇ ಬ್ರಹ್ಮಜ್ಞಾನ ಪಡೆಯುವಂತಾಗಲೆಂದು ನನ್ನನ್ನು ಆಶೀರ್ವದಿಸಿ." ಸ್ವಾಮಿಗಳು ತಮ್ಮ ಹಸ್ತವನ್ನು ಶಿಷ್ಯನ ಮಸ್ತಕದ ಮೇಲಿರಿಸಿ “ಏನೂ ಭಯಪಡದಿರು ಮಗು, ನೀನು ಇತರ ಪ್ರಾಪಂಚಿಕ ಮನುಷ್ಯರಂತಲ್ಲ - ಗೃಹಸ್ಥರಂತೆಯೂ ಅಲ್ಲ, ಪೂರ್ತಿ ಸಂನ್ಯಾಸಿಯಂತೆಯೂ ಅಲ್ಲ - ಹೊಸವರ್ಗದವನು" ಎಂದು ಹೇಳಿದರು.

\newpage

\chapter[ಅಧ್ಯಾಯ ೩೬]{ಅಧ್ಯಾಯ ೩೬\protect\footnote{\engfoot{C.W, Vol. VII, P. 222}}}

\begin{center}
ಸ್ಥಳ: ಬೇಲೂರು ಮಠ, ವರ್ಷ: ಕ್ರಿ.ಶ. ೧೯೦೧.
\end{center}

ಸ್ವಾಮೀಜಿಯವರ ಆರೋಗ್ಯ ಚೆನ್ನಾಗಿಲ್ಲ. ಸ್ವಾಮಿ ನಿರಂಜನಾನಂದರ ಒತ್ತಾಯದ ಪ್ರಾರ್ಥನೆಯಿಂದ ಆರೇಳು ದಿನಗಳಿಂದಲೂ ಕವಿರಾಜರ ಔಷಧಿಯನ್ನು ಸ್ವಾಮಿಗಳು ಸೇವಿಸುತ್ತಿದ್ದಾರೆ. ಈ ಚಿಕಿತ್ಸೆಯ ಪ್ರಕಾರ ನೀರು ಕುಡಿಯುವುದು ಪೂರ್ಣ ನಿಷಿದ್ಧ. ಹಾಲು ಕುಡಿದೇ ಅವರ ಬಾಯಾರಿಕೆ ಇಂಗಬೇಕು.

ಶಿಷ್ಯನು ಮಠಕ್ಕೆ ಇಂದು ಬಹುಬೇಗ ಬೆಳಿಗ್ಗೆ ಹೊತ್ತಿಗೆ ಮುಂಚೆ ಬಂದಿದ್ದ. ಅವನನ್ನು ನೋಡಿ ಸ್ವಾಮಿಗಳು ಪ್ರೇಮದಿಂದ “ಓ ನೀನು ಬಂದಿರುವೆಯಾ? ಒಳ್ಳೆಯದಾಯಿತು. ನಾನು ನಿನ್ನನ್ನೇ ಕುರಿತು ಆಲೋಚಿಸುತ್ತಿದ್ದೆ.”

ಶಿಷ್ಯ: ಸುಮಾರು ಆರೇಳು ದಿನಗಳಿಂದಲೂ ನೀವು ಹಾಲು ಮಾತ್ರ ಸೇವಿಸುತ್ತಿರುವಿರೆಂದು ಕೇಳಿದೆ.

ಸ್ವಾಮೀಜಿ: ಹೌದು, ನಿರಂಜನನ ತೀವ್ರ ಒತ್ತಾಯದಿಂದ ನಾನೀ ಚಿಕಿತ್ಸೆಗೆ ಒಪ್ಪಬೇಕಾಯಿತು. ನಾನು ಅವರ ಪ್ರಾರ್ಥನೆಗೆ ಕಿವಿಗೊಡದಿರಲಾರೆ.

ಶಿಷ್ಯ: ನೀವು ನೀರನ್ನು ಬಹಳ ಸಾರಿ ಕುಡಿಯುತ್ತಿದ್ದೀರಲ್ಲ, ನೀವು ಹೇಗೆ ಈಗ ಅದನ್ನು ಸಂಪೂರ್ಣ ಬಿಡಲು ಸಾಧ್ಯವಾಯಿತು?

ಸ್ವಾಮೀಜಿ: ನನಗೆ ಯಾವಾಗ ಈ ಚಿಕಿತ್ಸೆಯಿಂದ ನೀರು ಕುಡಿಯುವುದನ್ನು ಸಂಪೂರ್ಣ ಬಿಡಬೇಕೆಂದು ಗೊತ್ತಾಯಿತೋ ಆಗಲೇ ದೃಢ ಮನಸ್ಸಿನಿಂದ ಕುಡಿಯಬಾರದು ಎಂದು ನಿರ್ಧರಿಸಿದೆ. ಈಗ ನೀರನ್ನು ಕುಡಿಯಬೇಕೆಂಬ ಯೋಚನೆ ಒಮ್ಮೆಯೂ ಮನಸ್ಸಿಗೆ ಬರುವುದಿಲ್ಲ.

ಶಿಷ್ಯ: ಈ ಚಿಕಿತ್ಸೆಯಿಂದ ನಿಮಗೆ ಗುಣವಾಗುತ್ತಿರಬೇಕಲ್ಲವೇ?

ಸ್ವಾಮೀಜಿ: ನಾನರಿಯೆ; ಕೇವಲ ನನ್ನ ಗುರುಭಾಯಿಗಳ ಅಪ್ಪಣೆಯನ್ನು ನಾನು ಪಾಲಿಸುತ್ತಿದ್ದೇನೆ.

ಶಿಷ್ಯ: ಈ ವೈದ್ಯರು ಪ್ರಯೋಗಿಸುವ ಔಷಧಿಗಳು ನಮ್ಮ ದೇಹ ಪ್ರಕೃತಿಗೆ ಚೆನ್ನಾಗಿ ಒಗ್ಗುವುವೆಂದು ನಾನು ಕೇಳಿದ್ದೇನೆ.

ಸ್ವಾಮೀಜಿ: ಇದರಲ್ಲಿ ನನ್ನ ಭಾವನೆ ಏನೆಂದರೆ, ಕೇವಲ ಪ್ರವೀಣನಾದ ವೈದ್ಯನ ಕೈಯಲ್ಲಿ ಸತ್ತಾದರೂ ಸಾಯಬಹುದು. ಎಲ್ಲೋ ಕೆಲವು ರೋಗಗಳನ್ನು ವಾಸಿಮಾಡಿರುವ ಒಬ್ಬ ಪ್ರಾಪಂಚಿಕ ವ್ಯಕ್ತಿಯ ಕೈಯಲ್ಲಿ ರೋಗ ವಾಸಿಯಾಗುವುದನ್ನು ನಿರೀಕ್ಷಿಸುವುದಕ್ಕಿಂತ ಒಬ್ಬ ನಿಷ್ಣಾತ ವೈದ್ಯನ ಕೈಯಲ್ಲಿ ಸಾಯುವುದೇ ಮೇಲು.

ಸ್ವಾಮೀಜಿ ಕೆಲವು ಭಕ್ಷ್ಯಗಳನ್ನು ತಯಾರಿಸಿದ್ದರು. ಅದರಲ್ಲಿ ಒಂದು ಶಾವಿಗೆಯಿಂದ ಮಾಡಿದ್ದಾಗಿತ್ತು. ಅದನ್ನು ತಿಂದ ಶಿಷ್ಯನು ಅದೇನೆಂದು ಕೇಳಿದಾಗ “ಅವು ನಾನು ಲಂಡನ್ನಿನಿಂದ ತಂದ ಕೆಲವು ಒಣಗಿಸಿದ ಎರೆ ಹುಳುಗಳು" ಎಂದರು. ಎಲ್ಲರೂ ನಗತೊಡಗಿದರು. ಶಿಷ್ಯ ಕಕ್ಕಾಬಿಕ್ಕಿಯಾದ. ಸ್ವಾಮೀಜಿ ಅಷ್ಟು ಮಿತಾಹಾರಿಯಾಗಿ ಕಡಿಮೆ ನಿದ್ದೆ ಮಾಡುತ್ತಿದ್ದರೂ ಕೂಡ ಬಹಳ ಚಟುವಟಿಕೆಯಿಂದಿದ್ದಾರೆ. ಕೆಲವು ದಿನಗಳ ಹಿಂದೆ ಒಂದು ಹೊಸ ಎನ್‌ಸೈಕ್ಲೋಪೀಡಿಯಾ ಸಂಪುಟಗಳನ್ನು ಮಠಕ್ಕೆ ತರಿಸಿದ್ದರು. ಹೊಳೆಯುತ್ತಿದ್ದ ಆ ಪುಸ್ತಕಗಳನ್ನು ನೋಡಿ ಶಿಷ್ಯ “ಇಡೀ ಜೀವನವೆಲ್ಲ ಓದಿದರೂ ಈ ಸಂಪುಟಗಳು ಮುಗಿಯುವ ಮಟ್ಟಿಗೆ ಕಾಣೆ" ಎಂದು ಸ್ವಾಮೀಜಿಗೆ ಹೇಳಿದ. ಅವನಿಗೆ ಸ್ವಾಮೀಜಿ ಅಷ್ಟು ಹೊತ್ತಿಗಾಗಲೇ ಹತ್ತು ಸಂಪುಟಗಳನ್ನು ಮುಗಿಸಿ ಹನ್ನೊಂದನೆಯದನ್ನಾಗಲೇ ಓದಲು ಪ್ರಾರಂಭಿಸಿದ್ದಾರೆಂದು ಗೊತ್ತಿರಲಿಲ್ಲ.

ಸ್ವಾಮೀಜಿ: ನೀನು ಏನು ಹೇಳುವೆ? ಈ ಹತ್ತು ಸಂಪುಟಗಳಲ್ಲಿ ಯಾವುದನ್ನು ಬೇಕಾದರೂ ಕೇಳು. ನಾನೆಲ್ಲಕ್ಕೂ ಉತ್ತರ ಹೇಳುವೆ.

ಶಿಷ್ಯ: (ಆಶ್ಚರ್ಯದಿಂದ) ನೀವು ಈ ಪುಸ್ತಕಗಳನ್ನೆಲ್ಲಾ ಓದಿರುವಿರಾ?

ಸ್ವಾಮೀಜಿ: ಮತ್ತೆ, ಇಲ್ಲದಿದ್ದಲ್ಲಿ ನಿನಗೇಕೆ ಪ್ರಶ್ನೆ ಕೇಳೆಂದು ಹೇಳುತ್ತಿದ್ದೆ?

ಶಿಷ್ಯನು ಕೇಳಿದ್ದಕ್ಕೆಲ್ಲ ಸ್ವಾಮಿಜಿ ಉತ್ತರ ಹೇಳಿದ್ದಲ್ಲದೆ ಆ ಸಂಪುಟಗಳಲ್ಲಿ ಆರಿಸಿದ ಕೆಲವು ಕ್ಲಿಷ್ಟ ಶೈಲಿಯಲ್ಲಿದ್ದ ಅನೇಕ ವಿಷಯಗಳನ್ನೂ ಬಾಯಿಪಾಠದಂತೆ ಹೇಳಿದರು. ಅವಾಕ್ಕಾದ ಶಿಷ್ಯನು ಪುಸ್ತಕಗಳನ್ನು ಒತ್ತಟ್ಟಿಗಿಟ್ಟು ಹೇಳಿದ: ಇದು ಖಂಡಿತ ಮಾನುಷ ಶಕ್ತಿ ಅಲ್ಲ.

ಸ್ವಾಮೀಜಿ: ಗೊತ್ತಾಯಿತೇ ತೀವ್ರ ಬ್ರಹ್ಮಚರ್ಯಾಭ್ಯಾಸದಿಂದ ವಿದ್ಯೆಯನ್ನೆಲ್ಲಾ ಬಹು ಸ್ವಲ್ಪ ಕಾಲದಲ್ಲೇ ಕಲಿತು ಪೂರ್ಣಜ್ಞಾನ ಪಡೆಯಬಹುದು ಎಂದು. ಒಮ್ಮೆ ಓದಿದ ಅಥವಾ ಯಾರಿಂದಲಾದರೂ ಒಮ್ಮೆ ಕೇಳಿದುದನ್ನು ಮರೆಯದೆ ಇರುವಂಥಾ ಜ್ಞಾಪಕಶಕ್ತಿ ಅದರಿಂದ ಬರುವುದು. ನಮ್ಮ ದೇಶದಲ್ಲಿ ಬ್ರಹ್ಮಚರ್ಯದ ಅಭಾವವಿರುವುದರಿಂದಲೇ ಈ ದೇಶ ಇಷ್ಟೊಂದು ಅಧೋಗತಿಗಿಳಿದಿದೆ.

ಶಿಷ್ಯ: ನೀವೇನು ಬೇಕಾದರೂ ಹೇಳಿ, ಇಂತಹ ಅತಿಮಾನುಷ ಶಕ್ತಿ ಕೇವಲ ಬ್ರಹ್ಮಚರ್ಯಾಭ್ಯಾಸ ಮಾಡಿದಮಾತ್ರದಿಂದ ಬರುವ ಫಲವಲ್ಲ. ಬೇರೇನೋ ಇದ್ದೇ ಇರಬೇಕು.

ಸ್ವಾಮೀಜಿ ಇದಕ್ಕೆ ಉತ್ತರವನ್ನೇನೂ ಕೊಡಲಿಲ್ಲ.

ನಂತರ ಸ್ವಾಮಿಜಿ ಎಲ್ಲ ಬಗೆಯ ತತ್ತ್ವಗಳಲ್ಲೂ ಇರುವ ಬಗೆಬಗೆಯ ಕಷ್ಟಗಳನ್ನು ಅದರ ಪರಿಣಾಮಗಳನ್ನು ಕುರಿತು ದೀರ್ಘವಾಗಿ, ಉಜ್ವಲವಾಗಿ, ಚರ್ಚಿಸಿದರು. ಸಂಭಾಷಣೆಯಾಗುತ್ತಿರುವಾಗ ಸ್ವಾಮಿ ಬ್ರಹ್ಮಾನಂದರು ಆ ಕೊಠಡಿಗೆ ಪ್ರವೇಶಿಸಿ ಶಿಷ್ಯನಿಗೆ ಹೇಳಿದರು; “ನೀನೊಳ್ಳೆ ಹುಡುಗನಯ್ಯ! ಸ್ವಾಮೀಜಿಗೆ ಮೈಸರಿಯಿಲ್ಲ. ಏನಾದರೂ ತಮಾಷೆಯಾಗಿ ಮಾತನಾಡಿ ಅವರನ್ನು ನಗಿಸುವುದನ್ನು ಬಿಟ್ಟು ಅತ್ಯಂತ ಕಠಿಣವಾದ ಈ ವಿಷಯಗಳನ್ನು ಎಡೆಬಿಡದೆ ಅಷ್ಟು ಹೊತ್ತಿನಿಂದಲೂ ಮಾತಾಡುವಂತೆ ಮಾಡಿರುವೆ." ಶಿಷ್ಯನಿಗೆ ನಾಚಿಕೆಯಾಯಿತು. ಆದರೆ ಸ್ವಾಮೀಜಿ ಸ್ವಾಮಿ ಬ್ರಹ್ಮಾನಂದರಿಗೆ, “ನಿನ್ನ ಕವಿರಾಜರ ಚಿಕಿತ್ಸೆಯನ್ನು ಬದಿಗಿಡು. ಇವರೆಲ್ಲಾ ನನ್ನ ಮಕ್ಕಳು. ಇವರಿಗೆ ಬೋಧಿಸುವುದರಿಂದ ಈ ದೇಹ ಬಿದ್ದು ಹೋಗುವ ಹಾಗಿದ್ದರೆ ಹೋಗಲಿ. ಅದಕ್ಕೆ ನಾನು ಸ್ವಲ್ಪವೂ ಬೆಲೆಕೊಡುವುದಿಲ್ಲ" ಎಂದರು.

ನಂತರ ಏನೇನೋ ಲೋಕಾಭಿರಾಮವಾಗಿ ಮಾತನಾಡುತ್ತಿದ್ದೆವು. ನಂತರ ಬಂಗಾಳಿ ಸಾಹಿತಿ ಭರತಚಂದ್ರರ ವಿಚಾರ ಬಂದಿತು. ಮೊದಲಿನಿಂದಲೂ ಸ್ವಾಮೀಜಿ ಭರತಚಂದ್ರರನ್ನು ಅನೇಕ ಬಗೆಯಲ್ಲಿ ಟೀಕಿಸಿದ್ದರು. ಆತನ ಕಾಲದ ಜನಜೀವನ, ನಡವಳಿಕೆಗಳು, ವಿವಾಹ ಪದ್ದತಿಗಳು ಮತ್ತು ಸಮಾಜದ ಇತರ ಸಂಪ್ರದಾಯವನ್ನು ಕುರಿತು ವಿಡಂಬನೆ ಮಾಡಿದರು. ಭರತಚಂದ್ರ ಬಾಲ್ಯವಿವಾಹ ಪಕ್ಷವಾದಿ. ಆತನ ಪದ್ಯಗಳು ಬಹಳ ಕೀಳುಮಟ್ಟದ್ದು ಮತ್ತು ಅಶ್ಲೀಲತೆಯಿಂದ ತುಂಬಿವೆ. ಬಂಗಾಳವೊಂದನ್ನು ಬಿಟ್ಟರೆ ಮತ್ತಾವ ಸುಸಂಸ್ಕೃತ ಸಮಾಜವೂ ಅದನ್ನು ಇಷ್ಟಪಡುವುದಿಲ್ಲ ಎಂದು ಅಭಿಪ್ರಾಯಪಟ್ಟರು. ಅವರು ಹೇಳಿದ್ದೇನೆಂದರೆ “ಅಂತಹ ಪುಸ್ತಕಗಳು ಹುಡುಗರ ಕೈಗೆ ಸಿಗದಂತೆ ಎಚ್ಚರಿಕೆಯಿಂದಿರಬೇಕು." ನಂತರ ಮೈಕೇಲ್ ಮಧುಸೂದನದತ್ತರ ವಿಚಾರವಾಗಿ ಅವರು ಹೇಳಿದರು, “ನಿಮ್ಮ ಪ್ರಾಂತದಲ್ಲಿ ಅವನೊಬ್ಬ ಅದ್ಭುತ ಪ್ರತಿಭಾಶಾಲಿ, ಬಂಗಾಳಿ ಸಾಹಿತ್ಯದಲ್ಲಿ ಅವನ ‘ಮೇಘನಾದವಧ’ದಂತಹ ಮಹಾಕಾವ್ಯ ಮತ್ತೊಂದಿಲ್ಲ. ಈಗಿನ ಆಂಗ್ಲ ಸಾಹಿತ್ಯದಲ್ಲಿ ಕೂಡ ಎಲ್ಲಿಯೂ ಅಂತಹ ಪದ್ಯ ಕಾಣಬರುವುದು ಅಪರೂಪ."

ಶಿಷ್ಯ: ಆದರೆ ಸ್ವಾಮೀಜಿ ಮೈಕೆಲ್‌ನಿಗೆ ಬಹಳ ಆಡಂಬರದ ಶೈಲಿ ಇಷ್ಟ.

ಸ್ವಾಮೀಜಿ: ನಿಮ್ಮ ದೇಶದಲ್ಲಿ ಹೊಸದಾಗಿ ಯಾರು ಏನನ್ನು ಮಾಡಿದರೂ ಅದನ್ನು ಛೀಮಾರಿ ಮಾಡುವಿರಿ. ಅವನು ಏನು ಹೇಳುತ್ತಾನೆಂಬುದನ್ನು ಮೊದಲು ಪರೀಕ್ಷಿಸಿ. ಅದನ್ನು ಬಿಟ್ಟು ನಿಮ್ಮ ದೇಶೀಯರು ಯಾವುದು ಪುರಾತನವಾಗಿಲ್ಲವೋ ಅದನ್ನೆಲ್ಲಾ ತುಚ್ಛೀಕರಿಸುವರು. ಉದಾಹರಣೆಗೆ ಬಂಗಾಳಿ ಸಾಹಿತ್ಯದ ಶಿರೋರತ್ನವಾದ ಈ ಮೇಘನಾದವಧ ಕಾವ್ಯವನ್ನು ಹಳಿಯುವುದಕ್ಕೋಸ್ಕರವೇ ಚುಚುಂದರಿವಧ ಎಂಬ ವಿಡಂಬನ ಕಾವ್ಯ ಬರೆಯಲ್ಪಟ್ಟಿತು. ಅವರು ಎಷ್ಟೇ ವ್ಯಂಗ್ಯವಾಗಿ ಬರೆಯಲಿ ಅದರಿಂದೇನೂ ಫಲವಿಲ್ಲ. ಆ ಮೇಘನಾದವಧ ಕಾವ್ಯವು ಹಿಮಾಲಯದಂತೆ ಆಚಂದ್ರಾರ್ಕವಾದ ಕೀರ್ತಿಯನ್ನು ಹೊಂದಿದೆ. ಈ ಟೀಕೆಮಾಡುವ ವಿಮರ್ಶಕರ ಅಭಿಪ್ರಾಯಗಳೆಲ್ಲ ಮೂಲೆಗೆ ಬಿದ್ದಿವೆ. ಮೈಕೇಲನು ಸ್ವಂತ ಶೈಲಿಯಲ್ಲಿ ಎಂತಹ ವೀರವತ್ತಾದ ಪದಪ್ರಯೋಗ ಮಾಡಿ ಈ ಮಹಾಕಾವ್ಯ ಬರೆದಿದ್ದಾನೆಂಬುದು ಜನಸಾಮಾನ್ಯರಿಗೆ ಹೇಗೆತಾನೆ ಗೊತ್ತಾಗುವುದು? ಈಗತಾನೆ ಗಿರೀಶಬಾಬು ನಿಮ್ಮ ಪಂಡಿತರು ಅಷ್ಟೊಂದು ಅವಹೇಳನ ಮಾಡಿ ಟೀಕಿಸುತ್ತಿರುವ ಆ ನವಶೈಲಿಯಲ್ಲಿ ಅದ್ಭುತವಾದ ಗ್ರಂಥಗಳನ್ನು ಬರೆಯುತ್ತಿದ್ದಾರೆ. ಆದರೆ ಗಿರೀಶಬಾಬು ಆ ಟೀಕೆಗಳಿಗೆ ಕಿವಿಗೊಡುವರೇನು? ಜನರು ಕೆಲವು ಕಾಲದ ನಂತರ ಅವರ ಕೃತಿಗಳನ್ನು ಮೆಚ್ಚುವರು.

ಮೈಕೇಲನ ವಿಚಾರ ಹೀಗೆ ಕೆಲವು ಕಾಲ ಮಾತನಾಡುತ್ತಾ ಸ್ವಾಮಾಜಿ ಹೇಳಿದರು, “ಕೆಳಗಣ ಲೈಬ್ರರಿಯಿಂದ ಮೇಘನಾದವಧ ಕಾವ್ಯವನ್ನು ತೆಗೆದುಕೊಂಡು ಬಾ.” ಶಿಷ್ಯನು ತಂದಾಗ ಅವರು ಹೇಳಿದರು “ಈಗ ಓದು, ನೀನು ಹೇಗೆ ಓದುವೆ ನೋಡೋಣ.”

ಶಿಷ್ಯ ಒಂದು ಭಾಗವನ್ನು ಓದಿದ. ಅದು ಸ್ವಾಮೀಜಿಗೆ ಒಗ್ಗದೆ ಅವರೇ ಪುಸ್ತಕವನ್ನು ತೆಗೆದುಕೊಂಡು ಅದನ್ನು ಓದುವುದು ಹೀಗೆಂದು ತೋರಿಸಿಕೊಟ್ಟು ಪುನಃ ಓದುವಂತೆ ಹೇಳಿದರು. ನಂತರ ಅವರು ಈ ಕಾವ್ಯದ ಯಾವ ಭಾಗ ಅತ್ಯುತ್ತಮವಾದುದೆಂದು ನೀನು ಬಲ್ಲೆಯಾ? ಎಂದರು. ಶಿಷ್ಯ ನಿರುತ್ತರನಾದಾಗ ಸ್ವಾಮಿಗಳು “ಯಾವ ಭಾಗದಲ್ಲಿ ಇಂದ್ರಜಿತು ಯುದ್ಧದಲ್ಲಿ ಮಡಿದಿದ್ದಾನೆ, ಶೋಕಭರಿತಳಾಗಿದ್ದ ಮಂಡೋದರಿ ರಾವಣನನ್ನು ಯುದ್ಧ ವಿಮುಖನಾಗುವಂತೆ ಪ್ರಯತ್ನಿಸುತ್ತಿದ್ದಾಳೆ, ರಾವಣ ತನ್ನ ಮನಸ್ಸಿನಿಂದ ಬಲವಂತವಾಗಿ ಪುತ್ರಶೋಕವನ್ನು ತಳ್ಳಿ, ಮಹಾವೀರನಂತೆ ಯುದ್ಧಕ್ಕೆ ಹೋಗಲು ನಿರ್ಧರಿಸಿದ್ದಾನೆ, ತೀವ್ರ ಕೋಪ, ದ್ವೇಷಗಳಲ್ಲಿ ಪತ್ನಿಪುತ್ರರನ್ನು ಮರೆತು ಯುದ್ಧಕ್ಕೆ ನುಗ್ಗಲು ಸಿದ್ಧನಾಗಿದ್ದಾನೆ - ಅದೇ ಈ ಪುಸ್ತಕದಲ್ಲಿ ಅತ್ಯುತ್ತಮವಾದ ಭಾಗ. ಏನು ಬೇಕಾದರೂ ಬರಲಿ ನನ್ನ ಕರ್ತವ್ಯವನ್ನು ಮರೆಯುವುದಿಲ್ಲ, ಈ ಪ್ರಪಂಚ ಇದ್ದರೆಷ್ಟು ಹೋದರೆಷ್ಟು - ಇದು ಒಬ್ಬ ಮಹಾವೀರನಾದವನ ಬಾಯಲ್ಲಿ ಬರುವಂತಹ ಮಾತುಗಳು. ಇಂತಹ ಭಾವನೆಗಳಿಂದ ಸ್ಫೂರ್ತಿಗೊಂಡು ಮೈಕೇಲನು ಆ ಭಾಗವನ್ನು ಬರೆದಿದ್ದಾನೆ" ಎಂದರು.

ಹೀಗೆ ಹೇಳುತ್ತಾ ಸ್ವಾಮೀಜಿ ಆ ಭಾಗವನ್ನು ರಸವತ್ತಾಗಿ ಓದಲು ಮೊದಲುಮಾಡಿದರು.\footnote{\engfoot{C.W, Vol. VII, P. 226}}

\newpage

\chapter[ಅಧ್ಯಾಯ ೩೭]{ಅಧ್ಯಾಯ ೩೭\protect\footnote{\engfoot{C.W, Vol. VII, P. 226}}}

\begin{center}
ಸ್ಥಳ: ಬೇಲೂರು ಮಠ, ವರ್ಷ: ಕ್ರಿ.ಶ. ೧೯೦೧.
\end{center}

ಕವಿರಾಜರ ಚಿಕಿತ್ಸೆಯಲ್ಲಿ ಸ್ವಾಮೀಜಿಗೆ ಕೊಂಚ ಗುಣವಾಗಿದೆ. ಶಿಷ್ಯ ಈಗ ಮಠದಲ್ಲಿಯೇ ಇದ್ದಾನೆ. ಸ್ವಾಮೀಜಿಗೆ ಸೇವೆಮಾಡುತ್ತಿದ್ದಾಗ ಶಿಷ್ಯ “ಆತ್ಮ ಸರ್ವವ್ಯಾಪಿ, ಎಲ್ಲಾ ಜೀವಿಗಳ ಜೀವ, ತೀರ ಹತ್ತಿರದಲ್ಲಿದ್ದಾನೆ. ಆದರೂ ಅವನನ್ನೇಕೆ ನೋಡಲಾಗುವುದಿಲ್ಲ" ಎಂದು ಕೇಳಿದ.

ಸ್ವಾಮೀಜಿ: ನಿನಗೆ ಕಣ್ಣಿದೆಯೆಂಬುದನ್ನು ನೀನೇ ನೋಡುವೆಯೇನು? ಇತರರು ಕಣ್ಣುಗಳ ವಿಚಾರ ಮಾತನಾಡಿದಾಗ ನಿನಗೂ ಕಣ್ಣಿದೆ ಎಂಬ ನೆನಪಾಗುವುದು. ಅಲ್ಲದೆ ಧೂಳು ಮಣ್ಣೇನಾದರೂ ನಿನ್ನ ಕಣ್ಣಿನೊಳಗೆ ಬಿದ್ದು ವೇದನೆಯನ್ನುಂಟುಮಾಡಿದಾಗ ನಿನಗೂ ಕಣ್ಣಿದೆ ಎಂದು ಚೆನ್ನಾಗಿ ಗೊತ್ತಾಗುವುದು. ಹಾಗೆಯೇ ವಿಶ್ವಾತ್ಮನ ಸಾಕ್ಷಾತ್ಕಾರವೂ ಸುಲಭವಾಗಿ ಸಿಕ್ಕುವಂತಹುದಲ್ಲ. ಧರ್ಮಶಾಸ್ತ್ರಗಳನ್ನೋದುವುದರಿಂದ ಮತ್ತೊಬ್ಬರಿಂದ ಉಪದೇಶ ಕೇಳುವುದರಿಂದ ಇದರ ಸ್ವಭಾವ ಕೊಂಚಮಟ್ಟಿಗೆ ನಿನಗೆ ತಿಳಿಯುವುದು. ಆದರೆ ಯಾವಾಗ ಈ ಪ್ರಪಂಚದ ಕಹಿ ವೇದನೆ, ಕಷ್ಟಗಳ ಪೆಟ್ಟಿನಿಂದ ನಿನ್ನ ಹೃದಯ ಜರ್ಝರಿತವಾಗುವುದೋ, ನಮ್ಮ ಹತ್ತಿರದ ಪ್ರಿಯತಮ ಬಂಧುಗಳು ಮರಣವನ್ನೈದುವರೊ ಆಗ ಮಾನವನು ತಾನು ದಿಕ್ಕಿಲ್ಲದ ಅನಾಥನೆಂದು ತಿಳಿಯುತ್ತಾನೆ. ದಾಟಲಾಗದ ಅಭೇದ್ಯವಾದ ಶೂನ್ಯತೆ ಅವನ ಮನಸ್ಸನ್ನೆಲ್ಲಾ ಆವರಿಸಿದಾಗ ಜೀವನು ಆತ್ಮಸಾಕ್ಷಾತ್ಕಾರಕ್ಕೆ ಹಾತೊರೆಯುತ್ತಾನೆ. ಅದಕ್ಕೇ ದುಃಖ ಆತ್ಮಜ್ಞಾನಕ್ಕೆ ಸಹಕಾರಿ. ಆದರೆ ಈ ಅನುಭವದ ಕಹಿನೆನಪನ್ನು ನಾವು ಯಾವಾಗಲೂ ನೆನಪಿನಲ್ಲಿಟ್ಟಿರಬೇಕು. ಯಾರು ಕೇವಲ ನಾಯಿ ಬೆಕ್ಕುಗಳಂತೆ ಜೀವನದ ಅಳಲನ್ನು ಅನುಭವಿಸುತ್ತಾ ಸಾಯುವನೋ ಅವನು ಮನುಷ್ಯನೇನು? ಯಾವ ಮನುಷ್ಯ ಸುಖ ದುಃಖಗಳ ತೀಕ್ಷ್ಣ ಪ್ರಹಾರಗಳಿಂದ ಜರ್ಝರಿತನಾದಾಗಲೂ ಯುಕ್ತಾಯುಕ್ತ ವಿಚಾರ ಮಾಡುವನೋ ಅವನು ಮಾತ್ರ ಮನುಷ್ಯ. ಅಂಥವನು ಇವೆಲ್ಲಾ ಕ್ಷಣಭಂಗುರಗಳೆಂದು ತಿಳಿದು ಆತ್ಮನಲ್ಲಿ ಗಾಢವಾದ ಭಕ್ತಿಯುಳ್ಳವನಾಗುವನು. ಇದೇ ಮನುಷ್ಯರಿಗೂ ಪ್ರಾಣಿಗಳಿಗೂ ಇರುವ ವ್ಯತ್ಯಾಸ. ಯಾವುದು ಅತ್ಯಂತ ಹತ್ತಿರವಿದೆಯೋ ಅದು ಬೇಗ ದೃಷ್ಟಿಗೆ ಗೋಚರವಾಗುವುದಿಲ್ಲ. ಆತ್ಮನು ನಮಗೆ ಅತ್ಯಂತ ಹತ್ತಿರದವನು. ಅದಕ್ಕೇ ಉದಾಸೀನದಿಂದ ಅಸ್ಥಿರವಾದ ಮನಸ್ಸಿಗೆ ಅದರ ಅರಿವೇ ಆಗುವುದಿಲ್ಲ. ಯಾವ ಮನುಷ್ಯ ಚಟುವಟಿಕೆಯುಳ್ಳವನಾಗಿ, ಶಾಂತನಾಗಿ, ಆತ್ಮನಿಗ್ರಹ ಮತ್ತು ವಿಮರ್ಶಾಜ್ಞಾನವುಳ್ಳವನಾಗಿರುವನೋ ಅವನು ಈ ಬಾಹ್ಯ ಜಗತ್ತನ್ನು ನಿರ್ಲಕ್ಷಿಸಿ ಅಂತರ್‌ಜ್ಞಾನದಲ್ಲೇ ಹೆಚ್ಚು ಹೆಚ್ಚು ಮುಳುಗುತ್ತಾನೆ. ಆತ್ಮನ ಮಾಹಾತ್ಮ್ಯೆಯನ್ನರಿತು ತಾನೂ ಮಹಾತ್ಮನಾಗುವನು. ಆಗ ಮಾತ್ರ, ಅವನಿಗೆ ಆತ್ಮಪರಿಜ್ಞಾನ ಲಭಿಸಿ ಶಾಸ್ತ್ರದಲ್ಲಿ ಹೇಳುವಂತೆ ‘ನಾನೇ ಆತ್ಮ,’ ‘ನೀನೇ ಅದು ಆಗಿರುವೆ, ಓ! ಶ್ವೇತಕೇತು’ ಮುಂತಾದುವುಗಳಲ್ಲಿರುವ ಸತ್ಯವನ್ನು ಅವನು ಅರಿಯುತ್ತಾನೆ. ಅರ್ಥವಾಯಿತೆ?

ಶಿಷ್ಯ: ಆಯಿತು ಸ್ವಾಮೀಜಿ, ಹೀಗೆ ದುಃಖ ಸಂಕಟಗಳ ಹಾದಿಯಿಂದ ಆತ್ಮಜ್ಞಾನ ಪಡೆದುಕೊಳ್ಳುವ ಮಾರ್ಗವೇಕೆ? ಇದಕ್ಕೆ ಬದಲಾಗಿ ಸೃಷ್ಟಿಯೇ ಇಲ್ಲದಿದ್ದಲ್ಲಿ ಎಲ್ಲವೂ ಸರಿಹೋಗುತ್ತಿತ್ತು, ನಾವೆಲ್ಲ ಬ್ರಹ್ಮನೊಡನೆ ಐಕ್ಯವಾಗಿದ್ದೆವು. ಹೀಗಿದ್ದ ಮೇಲೆ ಬ್ರಹ್ಮನಿಗೇಕೆ ಈ ಸೃಷ್ಟಿಸುವ ಆಸೆ ಬಂತು? ಬ್ರಹ್ಮನೇ ಆದ ಜೀವನು ಹುಟ್ಟು ಸಾವುಗಳ ಪರಸ್ಪರ ದ್ವಂದ್ವಗಳಲ್ಲಿ ಹೋರಾಡುವುದು ಏಕೆ?

ಸ್ವಾಮೀಜಿ: ಯಾವಾಗ ಮನುಷ್ಯನಿಗೆ ಮತ್ತೇರಿದೆಯೋ ಅವನಿಗೆ ಅನೇಕ ಬಗೆಯ ಭ್ರಮೆಯುಂಟಾಗುವುದು. ಆದರೆ ಯಾವಾಗ ಮತ್ತಿಳಿಯುವುದೊ ಆಗ ಅವನಿಗೆ ಅವೆಲ್ಲ ಕಾವೇರಿದ ಮೆದುಳಿನ ಕಲ್ಪನೆಗಳೆಂದು ಗೊತ್ತಾಗುವುದು. ನೀನೀಗ ಆದಿಯೇ ಇಲ್ಲದ ಸೃಷ್ಟಿಯಲ್ಲಿ ಏನೇನು ನೋಡುವೆಯೋ, ಯಾವುದಕ್ಕೆ ಕೊನೆ ಇದೆ ಎನ್ನಿಸುವುದೋ, ಅವೆಲ್ಲಾ ನಿನ್ನ ಮತ್ತೇರಿದ ಸ್ಥಿತಿಯ ಪರಿಣಾಮ. ಯಾವಾಗ ಆ ಸ್ಥಿತಿಯನ್ನು ಮೀರಿ ಹೋಗುವೆಯೋ ಆಗ ಈ ಪ್ರಶ್ನೆಗಳಿಗೆ ಎಡೆಯೇ ಇರುವುದಿಲ್ಲ

ಶಿಷ್ಯ: ಹಾಗಾದರೆ ಈ ಭೂಮಂಡಲದ ಸೃಷ್ಟಿ ಸ್ಥಿತಿಗಳಲ್ಲಿ ಯಾವ ಸತ್ಯವೂ ಇಲ್ಲವೆ?~।

ಸ್ವಾಮೀಜಿ: ಏಕಿರಬಾರದು? ಎಲ್ಲಿಯವರೆಗೆ ನಿನ್ನಲ್ಲಿ ದೇಹಭಾವನೆ ಇದ್ದು ಅಹಂಭಾವವಿರುವುದೋ ಅಲ್ಲಿಯವರೆಗೂ ಇವೆಲ್ಲ ಇರುವುದು. ಆದರೆ ದೇಹಭಾವನೆ ಹೋಗಿ ನೀನು ಆತ್ಮನಲ್ಲಿ ಭಕ್ತಿಯುಳ್ಳವನಾಗಿ, ಆತ್ಮನಲ್ಲೇ ಜೀವಿಸುವೆಯೋ ಆಗ ನಿನಗೆ ಇವುಗಳೊಂದೂ ಇರುವುದಿಲ್ಲ. ಸೃಷ್ಟಿ, ಹುಟ್ಟು, ಸಾವು, ಇವೆಯೋ ಇಲ್ಲವೋ ಮುಂತಾದ ಪ್ರಶ್ನೆಗಳಿಗೆ ಎಡೆಯೇ ಇರುವುದಿಲ್ಲ. ಆಗ ನೀನು ಹೀಗೆ ಹೇಳಬೇಕಾಗುವುದು:

\begin{verse}
ಕ್ವಗತಂ ಕೇನ ವಾ ನೀತಂ ಕುತ್ರಲೀನಮಿದಂ ಜಗತ್~॥\\ಅಧುನೈವ ಮಯಾ ದೃಷ್ಟಂ ನಾಸ್ತಿ ಕಿಂ ಮಹದದ್ಭುತಂ~॥
\end{verse}

“ಪ್ರಪಂಚ ಎಲ್ಲಿ ಹೋಯಿತು? ಯಾರಿಂದ ಹೋಯಿತು? ಅದೆಲ್ಲಿ ಅವಿತುಕೊಂಡಿತು. ಈಗತಾನೇ ಅದನ್ನು ನೋಡಿದೆ, ಮರುಕ್ಷಣವೇ ಅದು ಇಲ್ಲವಾಗಿದೆ. ಎಂತಹ ಅದ್ಭುತ!"

ಶಿಷ್ಯ: ಪ್ರಪಂಚ ಇರುವುದನ್ನೇ ಅರಿಯದೆ ‘ಭೂಮಂಡಲವೆಲ್ಲಾ ಎಲ್ಲಿ ಲಯವಾಯಿತು?’ ಎಂದು ಹೇಗೆ ಕೇಳುವುದು?

ಸ್ವಾಮೀಜಿ: ಏಕೆಂದರೆ ನಮ್ಮ ಭಾವನೆಯನ್ನು ಭಾಷೆಯ ಮೂಲಕ ವಿವರಿಸಬೇಕು. ಅದಕ್ಕೇ ಹಾಗೆ ವಿವರಿಸುವುದು. ಕವಿಯು ಆ ಶ್ಲೋಕದ ಭಾವ, ಭಾಷೆಗೆ ಮೀರಿದ ಸ್ಥಿತಿಯನ್ನು ಭಾವ ಮತ್ತು ಭಾಷೆಯ ಮೂಲಕ ವಿವರಿಸತೊಡಗಿದ್ದಾನೆ. ಅದಕ್ಕೇ ಅವನು ಜಗತ್ತು ಮಾಯೆ, ಆಕಾಶದಂತೆ ಸಾಪೇಕ್ಷ ಎಂದು ವಿವರಿಸುತ್ತಾನೆ. ಜಗತ್ತು ನಿರಪೇಕ್ಷವಾಗಿ ಸತ್ಯವಲ್ಲ. ಮನಸ್ಸು, ಭಾಷೆಗಳಿಗೆ ಮೀರಿದ ಬ್ರಹ್ಮನು ಮಾತ್ರ ಸತ್ಯ. ನೀನೇನು ಕೇಳಬೇಕೆಂದಿದ್ದೀಯೊ ಕೇಳು, ಇಂದು ನಿನ್ನ ವಾಗ್ವಾದಗಳಿಗೆಲ್ಲಾ ಮಂಗಳ ಹಾಡುತ್ತೇನೆ.

ಪೂಜಾಮಂದಿರದಿಂದ ಸಂಜೆ ಆರತಿಯ ಗಂಟೆಯ ನಿನಾದ ಕೇಳಿಬಂತು. ಎಲ್ಲರೂ ಅಲ್ಲಿಗೆ ಹೋದರು. ಶಿಷ್ಯನು ಸ್ವಾಮೀಜಿಯವರ ಕೊಠಡಿಯಲ್ಲೇ ಉಳಿದ. ಅದನ್ನು ನೋಡಿ ಸ್ವಾಮಿಜಿ “ಪೂಜಾ ಮಂದಿರಕ್ಕೆ ನೀನು ಹೋಗುವುದಿಲ್ಲವೇ?" ಎಂದು ಕೇಳಿದರು.

ಶಿಷ್ಯ: ನನಗೆ ಇಲ್ಲಿಯೇ ಇರಲು ಇಚ್ಛೆ.

ಸ್ವಾಮೀಜಿ: ಹಾಗೇ ಮಾಡು.

ಸ್ವಲ್ಪ ಕಾಲಾನಂತರ ಶಿಷ್ಯ ಕೊಠಡಿಯ ಹೊರಗೆ ನೋಡಿ “ಇಂದು ಅಮಾವಾಸ್ಯೆ ರಾತ್ರಿ. ಎಲ್ಲಾ ಭಾಗಗಳೂ ಅಂಧಕಾರದಿಂದಾವೃತವಾಗಿವೆ. ಇಂದು ಕಾಳಿಮಾತೆಯನ್ನು ಪೂಜಿಸುವ ರಾತ್ರಿ" ಎಂದ.

ಸ್ವಾಮೀಜಿ ಮೌನವಾಗಿ ಕೊಂಚಹೊತ್ತು ಪೂರ್ವ ದಿಗಂತದ ಕಡೆ ದಿಟ್ಟಿಸಿ ನೋಡಿದರು. ನಂತರ ಹೇಳಿದರು “ಈ ಕತ್ತಲೆಯಲ್ಲಿ ಎಂತಹ ಗುಪ್ತವಾದ ಪ್ರಶಾಂತ ಸೌಂದರ್ಯವಿದೆ" - ಹೀಗೆ ಹೇಳುತ್ತಾ ಸ್ವಾಮಿಗಳು ದಟ್ಟವಾದ ಆ ಅಂಧಕಾರವನ್ನೇ ದಿಟ್ಟಿಸುತ್ತಾ ಅದರಲ್ಲೇ ಪರವಶರಾದರು. ಕೆಲವು ನಿಮಿಷಗಳಾದನಂತರ ಅವರು ನಿಧಾನವಾಗಿ ಒಂದು ಬಂಗಾಳಿ ಗೀತೆಯನ್ನು ಹಾಡಲು ಪ್ರಾರಂಭಿಸಿದರು. “ಓ! ಮಾತೆ, ತೀವ್ರ ಅಂಧಕಾರದಲ್ಲಿ ನಿನ್ನ ಸೌಂದರ್ಯ ಪ್ರಕಾಶಿಸುವುದು." ಹಾಡು ಮುಗಿದ ಮೇಲೆ ಸ್ವಾಮೀಜಿ ಕೊಠಡಿಯನ್ನು ಪ್ರವೇಶಿಸಿ ಬಾಯಲ್ಲಿ “ಮಾ, ಮಾ, ಕಾಳಿ, ಕಾಳಿ" ಎಂದು ಆಗಾಗ ಹೇಳುತ್ತಾ ಕುಳಿತುಕೊಂಡರು.

ಸ್ವಾಮೀಜಿಯ ಈ ಬಗೆಯ ಗಾಢ ಯೋಚನಾಮಗ್ನವಾದ ಸ್ಥಿತಿಯನ್ನು ನೋಡಿ ಕೊಂಚ ಕಳವಳಗೊಂಡು ಶಿಷ್ಯ ಕೇಳಿದ “ಸ್ವಾಮೀಜಿ, ದಯವಿಟ್ಟು ನನ್ನೊಡನೆ ಮಾತಾಡಿ.”

ಸ್ವಾಮೀಜಿ: (ನಗುಮುಖದಿಂದ ಹೇಳಿದರು) ಬಾಹ್ಯದಲ್ಲಿ ಇಷ್ಟು ಸುಂದರವಾಗಿ ಮಧುರವಾಗಿರುವ ಆತ್ಮನ ಅಗಾಧತೆ ಮತ್ತು ಸೌಂದರ್ಯದ ಆಳವನ್ನು ಅರಿಯ ಬಲ್ಲೆಯಾ? ಎಂದರು. ಶಿಷ್ಯ ಆ ಸಂಭಾಷಣೆಯನ್ನು ಬೇರೆ ಕಡೆ ತಿರುಗಿಸಲು ಇಚ್ಛಿಸಿದಾಗ ಅದನ್ನು ನೋಡಿ ಸ್ವಾಮೀಜಿ ಕಾಳಿಯ ಮೇಲೆ ಮತ್ತೊಂದು ಹಾಡನ್ನು ಎತ್ತಿದರು: “ಹೇ ಮಾತೆ, ಅಮೃತ ಪ್ರವಾಹ ನೀನು, ಎಷ್ಟು ಭಾವದಲ್ಲಿ ಎಷ್ಟು ಆಕಾರದಲ್ಲಿ ನೀನು ವ್ಯಕ್ತಳಾಗುತ್ತಿರುವೆ." ಹಾಡು ಮುಗಿದ ಮೇಲೆ ಸ್ವಾಮೀಜಿ: ಕಾಳಿಯು ಬ್ರಹ್ಮನ ವ್ಯಕ್ತಸ್ವರೂಪ, ಶ‍್ರೀರಾಮಕೃಷ್ಣರು ಹೇಳುತ್ತಿದ್ದ ಚಲಿಸುವ ಚಲಿಸದೆ ಇರುವ ಹಾವಿನ ಉಪಮಾನ ಗೊತ್ತಿಲ್ಲವೆ? (ಒಂದು ವ್ಯಕ್ತ ಮತ್ತೊಂದು ಅವ್ಯಕ್ತ ಸ್ವರೂಪ.)

ಶಿಷ್ಯ: ಹೌದು ಸ್ವಾಮೀಜಿ.

ಸ್ವಾಮೀಜಿ: ಈ ಬಾರಿ ನನಗೆ ಗುಣವಾದ ಮೇಲೆ ನಾನು ಮಾತೆಯನ್ನು ನನ್ನ ಹೃದಯದ ರಕ್ತದಿಂದ ಪೂಜಿಸುವೆ. ಆಗ ಮಾತ್ರ ಆಕೆಗೆ ಸಂತೋಷವಾಗುವುದು. ನಿನ್ನ ರಘುನಂದನ ಕೂಡ ಅದನ್ನು ಹೇಳುತ್ತಾನೆ. ಜಗನ್ಮಾತೆಯ ಮಗು ವೀರನಾಗಬೇಕು. ವ್ಯಥೆ, ದುಃಖ, ಸಾವಿಗೀಡಾದರೂ, ನಿರ್ಗತಿಕನಾದಾಗಲೂ ಜಗನ್ಮಾತೆಯ ಪುತ್ರ ನಿರ್ಭಯನಾಗಿರಬೇಕು.

\newpage

\chapter[ಅಧ್ಯಾಯ ೩೮]{ಅಧ್ಯಾಯ ೩೮\protect\footnote{\engfoot{C.W, Vol. VII, P. 230}}}

\begin{center}
ಸ್ಥಳ: ಬೇಲೂರು ಮಠ, ವರ್ಷ: ಕ್ರಿ.ಶ. ೧೯೦೧.
\end{center}

ಸ್ವಾಮೀಜಿ ಈಗ ಮಠದಲ್ಲೇ ಇದ್ದಾರೆ. ಅವರ ಆರೋಗ್ಯ ಅಷ್ಟೇನೂ ಗುಣಮುಖವಾಗಿಲ್ಲ. ಆದರೂ ಅವರು ಬೆಳಿಗ್ಗೆ ಮತ್ತು ಸಂಜೆ ಎರಡು ಹೊತ್ತೂ ಗಾಳಿಸಂಚಾರ ಹೋಗುತ್ತಾರೆ. ಶಿಷ್ಯನು ಸ್ವಾಮೀಜಿಗೆ ಪ್ರಣಾಮ ಮಾಡಿ ಅವರ ಆರೋಗ್ಯದ ವಿಷಯವಾಗಿ ಪ್ರಶ್ನೆ ಮಾಡಿದ.

ಸ್ವಾಮೀಜಿ: ಸರಿ, ಈ ದೇಹ ಅತ್ಯಂತ ಶೋಚನೀಯಾವಸ್ಥೆಯಲ್ಲಿದೆ. ಆದರೆ ನೀವಾರೂ ನನ್ನ ಕೆಲಸದಲ್ಲಿ ಸಹಾಯಮಾಡಲು ಒಂದು ಹೆಜ್ಜೆಯನ್ನೂ ಮುಂದಿಟ್ಟಿಲ್ಲ - ನಾನೊಬ್ಬ ಏನುತಾನೆ ಮಾಡಬಲ್ಲೆ? ಈ ಬಾರಿ ಈ ಶರೀರ ಬಂಗಾಳ ದೇಶದಲ್ಲಿ ಜನಿಸಿದುದರಿಂದ ಅದು ಹೇಗೆತಾನೆ ಹೆಚ್ಚು ಶ್ರಮವನ್ನು ಸಹಿಸಬಲ್ಲುದು? ಇಲ್ಲಿಗೆ ಬರುವವರೆಲ್ಲಾ ಪವಿತ್ರಾತ್ಮರು. ನನ್ನ ಕೆಲಸಕ್ಕೆ ನೀವು ಸಹಾಯಕರಾಗದಿದ್ದಲ್ಲಿ ನಾನೊಬ್ಬನೇ ಏನು ಮಾಡಬಲ್ಲೆ?

ಶಿಷ್ಯ: ಸ್ವಾಮೀಜಿ, ಈ ಆತ್ಮತ್ಯಾಗನಿರತರಾದ ಬ್ರಹ್ಮಚಾರಿಗಳು, ಸಂನ್ಯಾಸಿಗಳೆಲ್ಲಾ ನಿಮ್ಮ ಹಿಂದೆ ಇದ್ದಾರೆ. ಅವರಲ್ಲಿ ಪ್ರತಿಯೊಬ್ಬರೂ ತಮ್ಮ ಜೀವನವನ್ನು ನಿಮ್ಮ ಕೆಲಸಕ್ಕೆ ಅರ್ಪಿಸಲು ಸಿದ್ದರಾಗಿದ್ದಾರೆ-ಆದರೂ ನೀವೇಕೆ ಹೀಗೆ ಹೇಳುವಿರಿ?

ಸ್ವಾಮೀಜಿ: ನನಗೆ ಬಂಗಾಳಿ ಯುವಕರ ಒಂದು ತಂಡ ಬೇಕು. ಅವರೇ ದೇಶಕ್ಕೆ ಭರವಸೆ ನೀಡುವರು. ಒಳ್ಳೆಯ ಶೀಲವಂತರಾದ, ಬುದ್ಧಿವಂತರಾದ, ತಮ್ಮ ಸರ್ವಸ್ವವನ್ನೂ ತ್ಯಾಗಮಾಡುವಂತಹ, ವಿಧೇಯರಾಗಿರುವ, ನನ್ನ ಭಾವನೆಗಳನ್ನು ಕಾರ್ಯರೂಪಕ್ಕೆ ತರಲು ತಮ್ಮ ಪ್ರಾಣವನ್ನೇ ಅರ್ಪಿಸಬಲ್ಲ ಯುವಕರ ಮೇಲೆ ನನ್ನ ಭವಿಷ್ಯದ ಹಾರೈಕೆಯೆಲ್ಲಾ ನಿಂತಿದೆ. ಅದರಿಂದ ದೇಶಕ್ಕೂ ಅವರಿಗೂ ಕಲ್ಯಾಣವಾಗುವುದು. ಇಲ್ಲದಿದ್ದಲ್ಲಿ ಸಾಮಾನ್ಯ ಯುವಕರು ಬರುತ್ತಿದ್ದಾರೆ - ಬರುತ್ತಿರುತ್ತಾರೆ - ಅವರ ಮುಖಗಳಲ್ಲಿ ಗೋಳು ಸುರಿಯುತ್ತಿರುತ್ತದೆ. ಅವರ ಹೃದಯ ವೀರ್ಯಹೀನವಾಗಿದೆ - ಅವರ ದೇಹ ದುರ್ಬಲವಾಗಿದೆ, ಕೆಲಸಕ್ಕೆ ಅನರ್ಹವಾಗಿದೆ. ಅವರ ಮನಸ್ಸು ಧೈರ್ಯಹೀನವಾಗಿದೆ - ಅವರಿಂದ ಏನು ಕೆಲಸ ತಾನೇ ಸಾಧ್ಯ? ನಚಿಕೇತನಿಗಿದ್ದಂತಹ ಶ್ರದ್ಧೆಯುಳ್ಳ ೧೦-೧೨ ಜನ ಹುಡುಗರು ನನಗೆ ಸಿಕ್ಕಿದರೆ ದೇಶದ ಯೋಚನೆ ವೃತ್ತಿಯನ್ನೆಲ್ಲಾ ಹೊಸ ಮಾರ್ಗದಲ್ಲಿ ಹರಿಯುವಂತೆ ಮಾಡಬಲ್ಲೆ

ಶಿಷ್ಯ: ಅಷ್ಟೊಂದು ಮಂದಿ ಯುವಕರು ನಿಮ್ಮೆಡೆಗೆ ಬರುತ್ತಾರೆ. ಅವರಲ್ಲಿ ಯಾರೊಬ್ಬರಿಗೂ ನೀವು ಹೇಳಿದ ಗುಣಗಳಿಲ್ಲವೆ?

ಸ್ವಾಮೀಜಿ: ಅವರಲ್ಲಿ ಯಾರು ನನಗೆ ಒಳ್ಳೆಯವರೆಂದು ತೋರುವರೋ ಅವರಲ್ಲಿ ಕೆಲವರು ಮದುವೆಯ ಬಂಧನಕ್ಕೆ ಸಿಲುಕಿದ್ದಾರೆ, ಕೆಲವರು ಪ್ರಾಪಂಚಿಕ ಹೆಸರು ಕೀರ್ತಿ ಐಶ್ವರ್ಯಕ್ಕೆ ತಮ್ಮನ್ನು ಮಾರಿಕೊಂಡಿದ್ದಾರೆ - ಕೆಲವರು ನಿಶ್ಶಕ್ತರು. ಉಳಿದವರು, ಅವರೇ ಹೆಚ್ಚು ಮಂದಿ, ಯಾವ ಉತ್ತಮ ಆದರ್ಶಗಳನ್ನು ಪಡೆಯಲೂ ಯೋಗ್ಯರಲ್ಲ. ನೀನೇನೊ ನನ್ನ ಆದರ್ಶ ಭಾವನೆಗಳನ್ನು ಹೊಂದಲು ಅರ್ಹ. ಆದರೆ ಅದನ್ನು ವ್ಯವಹಾರದಲ್ಲಿ ತರಲು ನಿನಗೆ ಸಾಧ್ಯವಿಲ್ಲ. ಈ ಕಾರಣಗಳಿಂದ ಒಮೊಮ್ಮೆ ನನ್ನ ಮನಸ್ಸು ರೊಚ್ಚಿಗೇಳುವುದು. ಈ ಮಾನವ ದೇಹ ಧರಿಸಿರುವುದರಿಂದ ಇದು ಅಡ್ಡಿ ಬಂದು ನನಗೆ ಹೆಚ್ಚು ಕೆಲಸ ಮಾಡಲಾಗಲಿಲ್ಲ. ಆದಾಗ್ಯೂ ನಾನು ಸಂಪೂರ್ಣ ನಂಬಿಕೆ ಕಳೆದುಕೊಂಡಿಲ್ಲ. ಏಕೆಂದರೆ ದೇವರ ಇಚ್ಛೆಯಿಂದ ಈ ಕೆಲವು ಹುಡುಗರಿಂದಲೇ ಮುಂದೆ ಅನೇಕ ದೊಡ್ಡ ಕರ್ಮವೀರರು, ಆಧ್ಯಾತ್ಮಿಕ ವೀರರು ಹುಟ್ಟಿ ಬಂದು ನನ್ನ ಉದ್ದೇಶಗಳನ್ನು ಭವಿಷ್ಯದಲ್ಲಿ ಕಾರ್ಯರೂಪಕ್ಕೆ ತರಬಹುದು.

ಶಿಷ್ಯ: ನಿಮ್ಮ ವಿಶಾಲ ಮತ್ತು ಸರಳ ಆದರ್ಶಗಳನ್ನು ಒಂದಲ್ಲ ಒಂದು ದಿನ ಇಡೀ ವಿಶ್ವವೇ ಸ್ವೀಕರಿಸುವುದೆಂದು ನನಗೆ ದೃಢವಾದ ನಂಬಿಕೆಯಿದೆ. ಏಕೆಂದರೆ ಅವು ಸರ್ವ ಭಾವಗಳನ್ನೂ ಒಳಗೊಂಡಿವೆ, ಎಲ್ಲಾ ಅಭಿಪ್ರಾಯ ಕಾರ್ಯಗಳಿಗೆ ಶಕ್ತಿ ಒದಗಿಸುತ್ತವೆ. ದೇಶದ ಜನರೆಲ್ಲಾ ಪ್ರತ್ಯಕ್ಷವಾಗಿ ಆಗಲಿ, ಪರೋಕ್ಷವಾಗಿ ಆಗಲಿ ನಿಮ್ಮ ಅಭಿಪ್ರಾಯಗಳನ್ನು ಒಪ್ಪಿಕೊಂಡು ಜನರಿಗೆ ಬೋಧಿಸುತ್ತಿದ್ದಾರೆ.

ಸ್ವಾಮೀಜಿ: ಅವರು ನನ್ನ ಹೆಸರನ್ನು ಗೌರವಿಸಿದರೇನು ಬಿಟ್ಟರೇನು? ಅವರು ನನ್ನ ಅಭಿಪ್ರಾಯಗಳನ್ನು ಒಪ್ಪಿಕೊಂಡರೆ ಸಾಕು. ಕಾಮಿನಿ ಕಾಂಚನಗಳನ್ನು ತ್ಯಾಗ ಮಾಡಿದಮೇಲೂ ಶೇಕಡ ೯೯ ಮಂದಿ ಸಾಧುಗಳು ಹೆಸರು ಕೀರ್ತಿಗಳ ಬಲೆಗೆ ಸಿಕ್ಕಿಬೀಳುವರು. ‘ಕೀರ್ತಿ....ಉದಾತ್ತ ಮನಸ್ಸಿನ ಕಟ್ಟಕಡೆಯ ದೌರ್ಬಲ್ಯ’\footnote{\enginline{“Fame...that last infirmity of noble mind.”}} ನೀನಿದನ್ನು ಓದಿಲ್ಲವೆ? ಕರ್ಮಫಲದಾಸೆಯನ್ನೆಲ್ಲಾ ತ್ಯಜಿಸಿ ನಾವು ಕೆಲಸ ಮಾಡಬೇಕು. ಜನರು ನಿಮ್ಮನ್ನು ಒಳ್ಳೆಯವರು ಕೆಟ್ಟವರು ಎಂದು ಎರಡನ್ನೂ ಹೇಳುತ್ತಾರೆ. ಆದರೆ ನಾವು ನಮ್ಮ ಗುರಿಯನ್ನು ಮುಂದಿಟ್ಟುಕೊಂಡು ‘ಜನರು ನಮ್ಮನ್ನು ಹೊಗಳಲಿ ಬಿಡಲಿ’ ಯಾವುದಕ್ಕೂ ಸೊಪ್ಪು ಹಾಕದೆ ಸಿಂಹಸದೃಶರಾಗಿ ಕೆಲಸ ಮಾಡಬೇಕು.

ಶಿಷ್ಯ: ನಾವೀಗ ಯಾವ ಗುರಿಯನ್ನು ಅನುಸರಿಸಬೇಕು?

ಸ್ವಾಮಿಜಿ: ಮಹಾವೀರನನ್ನು ನಮ್ಮ ಗುರಿಯಾಗಿಟ್ಟುಕೊಳ್ಳಬೇಕು. ರಾಮಚಂದ್ರನ ಆಣತಿಯಂತೆ ಅವನು ಹೇಗೆ ಸಮುದ್ರವನ್ನೇ ದಾಟಿದ! ಅವನಿಗೆ ಜೀವನ ಮರಣ ಯಾವುದರ ಮೇಲೂ ಗಮನವಿರಲಿಲ್ಲ. ಅವನು ಇಂದ್ರಿಯಗಳನ್ನು ಸಂಪೂರ್ಣವಾಗಿ ತನ್ನ ವಶದಲ್ಲಿಟ್ಟುಕೊಂಡಿದ್ದ, ತೀಕ್ಷ್ಣಮತಿಯಾಗಿದ್ದ. ನೀವೀಗ ಈ ಸೇವೆಯ ಮಹೋದ್ದೇಶದ ಮೇಲೆ ನಿಮ್ಮ ಜೀವನವನ್ನು ಕಟ್ಟಬೇಕು. ಅದರ ಮೂಲಕ ಇತರ ಎಲ್ಲಾ ಉದ್ದೇಶಗಳೂ ಕ್ರಮೇಣ ವಿಕಸಿಸುವುವು. ಮರುಮಾತಿಲ್ಲದೆ ಗುರುಮಾತಿಗೆ ವಿಧೇಯರಾಗಿ ನಡೆಯುವುದು, ಮತ್ತು ಬ್ರಹ್ಮಚಾರಿ ಜೀವನ ನಡೆಸುವುದು, ಇದೇ ಜಯದ ರಹಸ್ಯ. ಹನುಮಂತ ಒಂದು ಕಡೆ ಸೇವೆಯ ಮಹೋದ್ದೇಶವನ್ನು ಇನ್ನೊಂದು ಕಡೆ ಇಡೀ ಜಗತ್ತೇ ಬೆರಗಾಗುವಂತಹ ಸಿಂಹಸದೃಶ ಧೈರ್ಯವನ್ನು ಪ್ರಕಟಿಸುತ್ತಾನೆ. ರಾಮನ ಕಲ್ಯಾಣಕ್ಕಾಗಿ ಪ್ರಾಣವನ್ನೇ ಅರ್ಪಿಸಲು ಅವನು ಸ್ವಲ್ಪವೂ ಅನುಮಾನಿಸಲಿಲ್ಲ. ರಾಮನ ಸೇವೆಯ ಹೊರತು ಉಳಿದುದರಲ್ಲೆಲ್ಲ ಅವನಿಗೆ ತಾತ್ಸಾರ. ಬ್ರಹ್ಮ ಶಿವ ಮುಂತಾದ ಈ ಜಗತ್ತಿನ ದೇವತೆಗಳ ಪದವಿಯೆಲ್ಲಾ ಅವನಿಗೆ ನಿಸ್ಸಾರ. ಶ‍್ರೀರಾಮನ ಆಣತಿಯಂತೆ ನಡೆಯುವುದೊಂದೇ ಅವನ ಜೀವನದ ವ್ರತ. ಅಂತಹ ಹೃತ್ಪೂರ್ವಕ ಭಕ್ತಿ, ಬೇಕಾಗಿದೆ. ಖೋಲ್ ಮತ್ತು ಕರ್ತಾಲ್‌ಗಳನ್ನು ನುಡಿಸುತ್ತಾ ತಾಳಕ್ಕೆ ಸರಿಯಾಗಿ ಕುಣಿಯುತ್ತಾ ಜನರೆಲ್ಲಾ ಅಧೋಗತಿಗಿಳಿದಿದ್ದಾರೆ. ಮೊಲದನೆಯದಾಗಿ ಅಜೀರ್ಣವ್ಯಾಧಿಯುಳ್ಳ ಜನಾಂಗ, ಜೊತೆಗೆ ಹೀಗೆ ಕುಣಿತಕ್ಕೂ ಪ್ರಾರಂಭಿಸಿದರೆ ಅವರು ಹೇಗೆ ತಾನೇ ಇಷ್ಟೊಂದು ಕಷ್ಟವನ್ನು ಸಹಿಸಬಲ್ಲರು? ಯಾವುದರ ಪ್ರಥಮ ಗುಣವೇ ಪವಿತ್ರತೆಯಾಗಿದೆಯೋ ಅಂತಹ ಉಚ್ಚ ಸಾಧನೆಯನ್ನು ಅನುಕರಿಸಹೊಗಿ ಸಂಪೂರ್ಣ ತಾಮಸದಲ್ಲಿ ಮುಳುಗಿ ಹೋಗಿರುವರು. ನೀನು ಪ್ರತಿಯೊಂದು ಗ್ರಾಮ ಮತ್ತು ಪಟ್ಟಣಕ್ಕೂ ಹೊಗಿನೋಡು, ಎಲ್ಲೆಲ್ಲೂ ನಿನಗೆ ಈ ಕುಣಿತವೇ ಕಂಡುಬರುವುದು. ದೇಶದಲ್ಲಿ ತಮಟೆಗಳನ್ನು ತಯಾರಿಸುವುದಿಲ್ಲವೆ? ಭರತಖಂಡದಲ್ಲಿ ತುತ್ತೂರಿ, ಡಮರುಗಳು ಸಿಗುವುದಿಲ್ಲವೆ? ಹುಡುಗರೆಲ್ಲಾ ಈ ತೂರ್ಯವಾಣಿಯನ್ನು ಕೇಳುವಂತೆ ಮಾಡಿ. ಬಾಲ್ಯದಿಂದ ಈ ಕೋಮಲಸ್ವಭಾವದ ಸಂಗೀತವನ್ನು ಕೇಳಿ ಕೇಳಿ, ಕೀರ್ತನೆಗಳನ್ನು ಕೇಳಿ ದೇಶವೆಲ್ಲಾ ನಾರಿಯರ ದೇಶವಾಗಿ ಪರಿಣಮಿಸಿದೆ. ಇದಕ್ಕಿಂತ ಬೇರೆ ಅಧೋಗತಿ ಯಾವುದಿದೆ? ಕವಿಯ ಕಲ್ಪನೆ ಕೂಡ ಈ ಚಿತ್ರವನ್ನು ಕಲ್ಪಿಸಲಾರದು. ತಮಟೆ ಕಹಳೆಯನ್ನು ಕೂಗಿಸಬೇಕು. ತಮಟೆಗಳನ್ನು ರಣರಂಗದಲ್ಲಿ ಬಾರಿಸಿದಂತೆ ಬಾರಿಸಬೇಕು. ‘ಮಹಾವೀರ, ಮಹಾವೀರ’ ಎಂಬ ಗರ್ಜನೆ ನಮ್ಮ ಬಾಯಿಂದ ಮೊರೆಯುತ್ತಿರಬೇಕು. ‘ಹರ, ಹರ, ಓಂ, ಓಂ!’ ಎಂಬ ನಿನಾದದಿಂದ ದಶ ದಿಕ್ಕುಗಳೂ ಪ್ರತಿಧ್ವನಿತವಾಗುವಂತೆ ಮಾಡಬೇಕು. ಮಾನವನ ಮೃದು ಹೃದಯವನ್ನು ಮಿಡಿಯುವಂತಹ ಸಂಗೀತ ಇಂದು ನಮಗೆ ಬೇಕಾಗಿಲ್ಲ - ಗೆಜ್ಜೆ ಧ್ವನಿಗಳ ನಾಟ್ಯ ಸಂಗೀತವನ್ನು ನಿಲ್ಲಿಸಿ, ದ್ರುಪದ ಸಂಗೀತಕ್ಕೆ ಕಿವಿಗೊಡುವಂತೆ ಮಾಡಬೇಕು. ಗಂಭೀರವಾದ ವೇದಘೋಷದ ಗರ್ಜನೆಯಿಂದ ದೇಶವನ್ನೆಲ್ಲ ತುಂಬಬೇಕು. ಪ್ರತಿಯೊಂದರಲ್ಲಿಯೂ ಪುರುಷಸಿಂಹನ ಕೆಚ್ಛೆದೆ ವ್ಯಕ್ತವಾಗುವಂತೆ ಮಾಡಬೇಕು. ಈ ಧ್ಯೇಯದ ಮೇಲೆ ನಿನ್ನ ಶೀಲವನ್ನು ರೂಪಿಸಬಲ್ಲೆಯಾದರೆ ಸಾವಿರಾರು ಜನ ನಿನ್ನನ್ನು ಹಿಂಬಾಲಿಸುವರು. ಆದರೆ ನಿನ್ನ ಧ್ಯೇಯದಿಂದ ಒಂದು ಅಂಗುಲವೂ ಹಿಮ್ಮೆಟ್ಟಬೇಡ, ಎದೆಗೆಡಬೇಡ. ಊಟ ಮಾಡುವಾಗ, ಉಡುಪು ಧರಿಸುವಾಗ, ಮಲಗಿರುವಾಗ, ಹಾಡುವಾಗ, ಆಟವಾಡುವಾಗ, ಸುಖದುಃಖ ಎಲ್ಲಾ ಅವಸ್ಥೆಯಲ್ಲೂ ಅತ್ಯುಚ್ಚ ಆಂತರಿಕ ಶಕ್ತಿ ನಿನ್ನಲ್ಲಿ ವಿಕಸಿಸಲಿ. ಆಗ ಮಾತ್ರ ನೀನು ಮಹಾಶಕ್ತಳಾದ ಜಗನ್ಮಾತೆಯ ಕೃಪೆಗೆ ಪಾತ್ರನಾಗುವೆ.

ಶಿಷ್ಯ: ಕೆಲವು ವೇಳೆ ಹೇಗೋ ನನಗೇ ಗೊತ್ತಿಲ್ಲದೆ ಮನಸ್ಸು ಅಧೋಗತಿಗಿಳಿಯುವುದು ಸ್ವಾಮೀಜಿ.

ಸ್ವಾಮೀಜಿ: ಹಾಗಿದ್ದಲ್ಲಿ ಹೀಗೆ ಯೋಚಿಸು: ‘ನಾನು ಯಾರ ಮಗು? ನಾನು ಅವನೊಡನೆ ಸೇರುವೆನು. ಹೀಗಿರುವಾಗ ನನ್ನಲ್ಲಿ ಅಂತಹ ದುರ್ಬಲ ಮನಸ್ಸು ಮತ್ತು ತುಚ್ಛ ಭಾವಗಳು ಬರುವುದು ಸರಿಯೇ?’ ಮನಸ್ಸಿನ ದುರ್ಬಲತೆಯನ್ನು ಮೆಟ್ಟಿ ಎದ್ದು ನಿಲ್ಲು. ‘ನನಗೆ ವೀರತವಿದೆ - ನನಗೆ ಬುದ್ಧಿ ಇದೆ - ನಾನು ಬ್ರಹ್ಮಜ್ಞಾನಿ, ಆತ ಸಾಕ್ಷಾತ್ಕಾರ ಪಡೆದವನು’ - ನಿನ್ನ ಸ್ಥಾನದ ಹಿರಿಮೆಯನ್ನು ಹೀಗೆ ಯೋಚಿಸು. ‘ಕಾಮಿನಿ ಕಾಂಚನ ತ್ಯಾಗಿಗಳಾದ ಶ‍್ರೀರಾಮಕೃಷ್ಣ ಪರಮಹಂಸರ ಒಡನಾಡಿಯಾಗಿದ್ದಂಥವರ ಶಿಷ್ಯ ನಾನು’ ಎಂದು ಚೆನ್ನಾಗಿ ಮನನ ಮಾಡಿಕೋ. ಇದರಿಂದ ಒಳ್ಳೆಯ ಪರಿಣಾಮ ಉಂಟಾಗುವುದು. ಯಾರಿಗೆ ಈ ಹೆಮ್ಮೆ ಇಲ್ಲವೋ ಅವನಲ್ಲಿ ಬ್ರಹ್ಮನ ಜಾಗೃತಿ ಕೊಂಚವೂ ಆಗಿಲ್ಲ. ರಾಮಪ್ರಸಾದನ ಹಾಡನ್ನು ಕೇಳಿಲ್ಲವೆ? ಅವನು ಹೇಳುತ್ತಿದ್ದ: ‘ಜಗನ್ಮಾತೆ ಆಳುತ್ತಿರುವ ಈ ಜಗತ್ತಿನಲ್ಲಿ ಆರನ್ನು ಕಂಡು ನಾನು ಭಯಪಡುವೆ?’ ಇಂತಹ ಹೆಮ್ಮೆಯನ್ನು ಯಾವಾಗಲೂ ಮನಸ್ಸಿನಲ್ಲಿ ಮೆಲುಕು ಹಾಕುತ್ತಿರು. ನಂತರ ಹೃದಯ ದೌರ್ಬಲ್ಯ ಯಾವುದೂ ನಿನ್ನ ಬಳಿ ಸುಳಿಯಲಾರವು. ನಿನ್ನ ಮನಸ್ಸು ಎಂದೂ ದುರ್ಬಲವಾಗಲು ಅವಕಾಶ ಕೊಡಬೇಡ - ಮಹಾವೀರನನ್ನು ನೆನೆ, ಜಗನ್ಮಾತೆಯನ್ನು ಸ್ಮರಿಸು. ಆಗ ನಿನ್ನಿಂದ ಎಲ್ಲಾ ದೌರ್ಬಲ್ಯವೂ ಹೇಡಿತನವೂ ತಕ್ಷಣವೇ ಮಾಯವಾಗುವುದು.

ಈ ಮಾತುಗಳನ್ನು ಹೇಳುತ್ತಾ ಸ್ವಾಮೀಜಿ ಮಹಡಿಯಿಂದ ಕೆಳಗಿಳಿದು ಬಂದು ಅಂಗಳದಲ್ಲಿ ತಾವು ದಿನವೂ ಕುಳಿತುಕೊಳ್ಳುತ್ತಿದ್ದ ಮಂಚದ ಮೇಲೆ ಕುಳಿತರು. ನಂತರ ಅಲ್ಲಿ ನೆರೆದಿದ್ದ ಸಂನ್ಯಾಸಿಗಳು, ಬ್ರಹ್ಮಚಾರಿಗಳನ್ನೆಲ್ಲಾ ಕುರಿತು ಈ ರೀತಿ ಹೇಳಿದರು: “ಇಲ್ಲಿ ತೆರೆ ಕಳಚಿದ ಬ್ರಹ್ಮನಿದ್ದಾನೆ. ಇದನ್ನು ನಂಬದೆ ಇತರ ವಸ್ತುಗಳ ಮೇಲೆ ಮನಸ್ಸಿಟ್ಟವರಿಗೆ ಧಿಕ್ಕಾರ! ಓ! ಇಲ್ಲೇ ಬ್ರಹ್ಮ ಅಂಗೈಮೇಲಣ ನೆಲ್ಲಿಕಾಯಂತೆ ಪ್ರತ್ಯಕ್ಷವಾಗಿ ಇದ್ದಾನೆ. ನಿಮಗೆ ಕಾಣುವುದಿಲ್ಲವೆ? ಇಲ್ಲೇ!"

ಈ ಮಾತನ್ನು ಹೇಗೆ ಮನಮುಟ್ಟುವಂತೆ ಹೇಳಿದರೆಂದರೆ ಅಲ್ಲಿದ್ದವರೆಲ್ಲಾ ಪರದೆಯ ಮೇಲೆ ಚಿತ್ರಿಸಿದ ಗೊಂಬೆಗಳಂತೆ ಅಲ್ಲಾಡದೆ ನಿಂತರು. ಎಲ್ಲರನ್ನೂ ಗಾಢಧ್ಯಾನ ಆವರಿಸಿದಂತೆ ಇತ್ತು. ಕೊಂಚ ಕಾಲದ ಮೇಲೆ ಆ ತೀವ್ರತೆ ಕಡಿಮೆಯಾಗಿ ಪ್ರಜ್ಞೆಗೆ ಬಂದರು.

ಮುಂದೆ ಕೊಂಚ ಕಾಲದ ನಂತರ ನಡೆಯುತ್ತಿದ್ದಾಗ ಸ್ವಾಮಿಗಳು ಶಿಷ್ಯನಿಗೆ ಹೇಳಿದರು, “ಇಂದು ನೋಡಿದೆಯಾ - ಅವರೆಲ್ಲ ಹೇಗೆ ಏಕಮನಸ್ಕರಾದರೆಂದು? ಅವರೆಲ್ಲಾ ಶ‍್ರೀರಾಮಕೃಷ್ಣರ ಮಕ್ಕಳು, ಕೇವಲ ಆ ಮಾತುಗಳನ್ನು ಕೇಳಿದುದರಿಂದಲೇ ಅವರಿಗೆ ಸತ್ಯದ ಅರಿವಾಯಿತು."

ಶಿಷ್ಯ: ಅವರ ವಿಷಯ ಹಾಗಿರಲಿ, ನನ್ನ ಹೃದಯವೂ ಅನಿರ್ವಚನೀಯವಾದ ಆನಂದದಿಂದ ಪುಲಕಿತವಾಯಿತು - ಈಗ ಅದೊಂದು ಮಾಯವಾದ ಸ್ವಪ್ನದಂತಿದೆ.

ಸ್ವಾಮೀಜಿ: ಎಲ್ಲವೂ ಸಕಾಲದಲ್ಲಿ ಬರುವುದು. ಈಗ ಕೆಲಸಮಾಡುತ್ತಾ ಹೋಗಿ. ಮೌಡ್ಯ, ಅಜ್ಞಾನಗಳಲ್ಲಿ ಮುಳುಗಿರುವವರನ್ನು ಎತ್ತಲು ಏನಾದರೂ ಕೊಂಚ ಕೆಲಸಮಾಡಿ. ಆಗ ನಿಮಗೇ ಅಂತಹ ಅನುಭವಗಳು ಬರುವುವು.

ಶಿಷ್ಯ: ಆ ಕರ್ಮದ ಕೋಟೆಯೊಳಗೆ ಹೋಗಲು ಹೆದರುತ್ತೇನೆ - ಅಷ್ಟು ಶಕ್ತಿಯೂ ಇಲ್ಲ - ಶಾಸ್ತ್ರಗಳೂ ಕರ್ಮದ ಹಾದಿ ಗಹನ ಎಂದು ಹೇಳುತ್ತವೆ.

ಸ್ವಾಮೀಜಿ: ಹಾಗಾದರೆ ನೀನೇನು ಮಾಡಬಯಸುವೆ?

ಶಿಷ್ಯ: ಎಲ್ಲಾ ಧರ್ಮಗ್ರಂಥಗಳ ಸತ್ಯವನ್ನೂ ಸಾಕ್ಷಾತ್ಕರಿಸಿಕೊಂಡಿರುವ ನಿಮ್ಮಂತಹ ಪೂಜ್ಯರೊಡನೆ ವಾಸಮಾಡುತ್ತಾ, ಚರ್ಚಿಸುತ್ತಾ, ಅವನ್ನು ಕೇಳುವುದರಿಂದ, ಅವುಗಳ ಮನನ, ಧ್ಯಾನದಿಂದ ಬ್ರಹ್ಮನನ್ನು ಈ ಜೀವನದಲ್ಲೇ ಸಾಕ್ಷಾತ್ಕಾರ ಮಾಡಿಕೊಳ್ಳಬೇಕು. ಇದಲ್ಲದೆ ಬೇರಾವುದಕ್ಕೂ ನನಗೆ ಉತ್ಸಾಹವೂ ಇಲ್ಲ, ಶಕ್ತಿಯೂ ಇಲ್ಲ.

ಸ್ವಾಮಿಜಿ: ನೀನದನ್ನು ಇಚ್ಛೆಪಟ್ಟಲ್ಲಿ ಹಾಗೆಯೇ ಮಾಡು. ಶಾಸ್ತ್ರಗಳ ಬಗ್ಗೆ ಇರುವ ನಿನ್ನ ಆಲೋಚನೆ ಮತ್ತು ನಿರ್ಧಾರಗಳನ್ನು ಇತರರಿಗೂ ಹೇಳು. ಅದರಿಂದ ಅವರಿಗೂ ಪ್ರಯೋಜನವಾಗುವುದು. ಎಲ್ಲಿಯವರೆಗೆ ದೇಹವಿರುವುದೋ ಅಲ್ಲಿಯವರೆಗೂ ಏನಾದರೊಂದು ಕೆಲಸ ಮಾಡುತ್ತಲೇ ಇರಬೇಕು. ಆದ್ದರಿಂದ ಇತರರಿಗೆ ಒಳ್ಳೆಯದಾಗುವಂಥ ಕೆಲಸಗಳನ್ನು ಮಾಡು. ಶಾಸ್ತ್ರಗಳ ಸತ್ಯಗಳ ಬಗ್ಗೆ ಹೊಂದಿರುವ ನಿನ್ನ ಸಾಕ್ಷಾತ್ಕಾರ ಮತ್ತು ನಿರ್ಧಾರಗಳು ಸತ್ಯವನ್ನು ಅರಸುತ್ತಿರುವ ಇನ್ನೊಬ್ಬನಿಗೆ ಉಪಯೋಗವಾಗಬಹುದು. ಅವುಗಳನ್ನು ಬರೆದಿಡುವುದರಿಂದ ಇತರರಿಗೆ ಸಹಾಯವಾಗುವುದು.

ಶಿಷ್ಯ: ಮೊದಲು ನನಗೆ ಸತ್ಯಸಾಕ್ಷಾತ್ಕಾರವಾಗಲಿ. ನಂತರ ನಾನು ಬರೆಯುವೆ. “ಅಧಿಕಾರ ಮುದ್ರೆ ಇಲ್ಲದಿದ್ದಲ್ಲಿ ಯಾರೂ ನಿನ್ನ ಮಾತಿಗೆ ಕಿವಿಗೊಡುವುದಿಲ್ಲ" ಎಂದು ಶ‍್ರೀರಾಮಕೃಷ್ಣರು ಹೇಳುತ್ತಿದ್ದರು.

ಸ್ವಾಮೀಜಿ: ಈ ಪ್ರಪಂಚದಲ್ಲಿ ನೀನಿರುವ ಈ ಅವಸ್ಥೆಯಲ್ಲೇ, ಆಧ್ಯಾತ್ಮಿಕ ಸಾಧನೆ ಮತ್ತು ತರ್ಕಾವಸ್ಥೆಯಲ್ಲೇ ಅನೇಕ ಮಂದಿ ಸಿಕ್ಕಿಕೊಂಡಿದ್ದಾರೆ. ಆ ಅವಸ್ಥೆಯನ್ನು ಅವರಿಗೆ ಮೀರಿಹೋಗಲಾಗುತ್ತಿಲ್ಲ. ನಿನ್ನ ಅನುಭವ ಮತ್ತು ಯೋಚನಾತರಂಗದಿಂದ ಕಡೆಯಪಕ್ಷ ಅವರಿಗಾದರೂ ಸಹಾಯವಾದೀತು. ನೀನು ಈ ಮಠದ ಸಾಧುಗಳೊಡನೆ ನಡೆಸಿದ ಸಂಭಾಷಣೆ ತರ್ಕಗಳ ಸಾರವನ್ನು ಸರಳ ಭಾಷೆಯಲ್ಲಿ ಬರೆದರೆ ಅನೇಕರಿಗೆ ಸಹಾಯವಾಗುವುದು.

ಶಿಷ್ಯ: ನೀವು ಹೀಗೆ ಇಚ್ಛೆಪಡುವುದರಿಂದ ಹಾಗೇ ಮಾಡುವೆ.

ಸ್ವಾಮೀಜಿ: ಇತರರಿಗೆ ಉಪಯೋಗವಾಗದ, ಅಜ್ಞಾನಕೂಪದಲ್ಲಿ ಮುಳುಗಿರುವ ಜನರ ಕಲ್ಯಾಣಕ್ಕೆ ನೆರವಾಗದ, ಕಾಮಿನಿ ಕಾಂಚನದ ಬಂಧನದಿಂದ ಪಾರಾಗಲು ಸಹಾಯ ಮಾಡದ ಆಧ್ಯಾತ್ಮಿಕ ಸಾಧನೆ ಮತ್ತು ಸಾಕ್ಷಾತ್ಕಾರದಿಂದಾಗುವ ಫಲವೇನು? ಎಲ್ಲಿಯವರೆಗೆ ಒಬ್ಬ ಜೀವಿಯು ಬಂಧನದಲ್ಲಿದ್ದಾನೋ ಅಲ್ಲಿಯವರೆಗೂ ನಿನಗೆ ಮೋಕ್ಷ ದೊರಕುವುದೆಂದು ತಿಳಿದಿರುವೆಯಾ? ಅವನಿಗೆ ಮುಕ್ತಿಯಾಗುವವರೆಗೂ - ಅದಕ್ಕೆ ಹಲವು ಜನ್ಮಗಳು ಬೇಕಾಗಬಹುದು - ನೀನು ಅವನಿಗೆ ಸಹಾಯಮಾಡಲು ಹುಟ್ಟುತ್ತಿರಬೇಕು. ಪ್ರತಿಯೊಬ್ಬ ಜೀವಿಯೂ ನಿನ್ನ ಒಂದು ಭಾಗ. ಇತರರಿಗಾಗಿ ಮಾಡುವ ಎಲ್ಲಾ ಕೆಲಸದ ತತ್ತ್ವಾಧಾರ ಇದು. ನಿನ್ನ ಹೆಂಡತಿ ಮಕ್ಕಳು ನಿನ್ನವರೆಂದು ತಿಳಿದು ಅವರಿಗೆ ಹೃತ್ಪೂರ್ವಕವಾಗಿ ಕಲ್ಯಾಣವನ್ನು ಬಯಸುವಂತೆಯೇ ಪ್ರತಿಯೊಬ್ಬ ಜೀವಿಯ ಮೇಲೆ ಅದೇ ಬಗೆಯ ಪ್ರೇಮ ಎಂದಿಗೆ ನಿನ್ನಲ್ಲಿ ಉದಿಸುವುದೋ ಅಂದು ನಿನ್ನಲ್ಲಿ ಬ್ರಹ್ಮ ಜಾಗೃತನಾಗಿರುವನೆಂದು ತಿಳಿ! ಅಲ್ಲಿಯವರೆಗೂ ಇಲ್ಲ. ಎಂದು ಯಾವ ಜಾತಿಗಳನ್ನೂ ಲೆಕ್ಕಿಸದೆ ಎಲ್ಲರಿಗೂ ಒಳ್ಳೆಯದಾಗಲೆಂದು ಬಯಸುವ ಭಾವನೆ ನಿನ್ನಲ್ಲಿ ಉದಿಸುವುದೋ ಅಂದು ನೀನು ನಿನ್ನ ಗುರಿಯೆಡೆಗೆ ಹೋಗುತ್ತಿರುವೆ ಎಂದು ಭಾವಿಸಿಕೊ

ಶಿಷ್ಯ: ಸ್ವಾಮೀಜಿ, ಇದೆಂತಹ ಪ್ರಚಂಡ ನಿರೂಪಣೆ! ಸರ್ವರಿಗೂ ಮುಕ್ತಿಯಾದಲ್ಲದೆ ವ್ಯಕ್ತಿಯೊಬ್ಬನಿಗೆ ಮುಕ್ತಿಯಿಲ್ಲ! - ನಾನೆಂದೂ ಇಂತಹ ಅದ್ಭುತ ಪ್ರಸ್ತಾವವನ್ನು ಕೇಳಿರಲಿಲ್ಲ.

ಸ್ವಾಮೀಜಿ: ಒಂದು ಬಗೆಯ ವೇದಾಂತವಾದಿಗಳು ಈ ಭಾವನೆ ಹೊಂದಿದ್ದಾರೆ. ಅವರ ಪ್ರಕಾರ ವ್ಯಷ್ಟಿಮೋಕ್ಷವು ನಿಜವಾದ ಪರಿಪೂರ್ಣವಾದ ಮೋಕ್ಷವಲ್ಲ, ಸಮಷ್ಟಿ ಮೋಕ್ಷವೇ ನಿಜವಾದ ಮುಕ್ತಿ. ಈ ಪಂಥದಲ್ಲಿ ಒಳ್ಳೆಯದು ಕೆಟ್ಟದ್ದು ಎರಡನ್ನೂ ತೋರಿಸಬಹುದು.

ಶಿಷ್ಯ: ವೇದಾಂತದ ಪ್ರಕಾರ ವ್ಯಕ್ತಿಯ ಪ್ರತ್ಯೇಕೀಕರಣವೇ ಎಲ್ಲಾ ಬಂಧನದ ಮೂಲ. ಆಸೆಗಳ, ಕರ್ಮಗಳ ಫಲವಾಗಿ, ಅನಂತವಾದ ಚಿಚ್ಛಕ್ತಿಯು ಸಾಂತವಾಗಿರುವಂತೆ ತೋರುತ್ತದೆ. ವಿವೇಚನಾಶಕ್ತಿಯಿಂದ ಆ ಸಾಂತತೆಯು ಮಾಯವಾದಾಗ ಜೀವವು ಎಲ್ಲಾ ಬಗೆಯ ಅಧೀನಾವಸ್ಥೆಯಿಂದಲೂ ಬಿಡುಗಡೆ ಹೊಂದುವುದು. ಅಂದ ಮೇಲೆ ಜ್ಞಾನಾತೀತನಾದ ಆತ್ಮನಿಗೆ ಬಂಧನವೆಂದರೇನು? ಜೀವ ಜಗತ್ತು ಯಾರಿಗೆ ಸತ್ಯವಾಗಿದೆಯೋ ಅವರು ಎಲ್ಲರಿಗೂ ಮುಕ್ತಿ ದೊರೆತ ಹೊರತು ತನಗೆ ಸಿಕ್ಕುವುದಿಲ್ಲವೆಂದು ಯೋಚಿಸಬಹುದು. ಆದರೆ ಯಾವಾಗ ಮನಸ್ಸು ಅಧೀನಾವಸ್ಥೆಯ ಎಲ್ಲೆಕಟ್ಟನ್ನೆಲ್ಲಾ ಮೀರಿ ಬ್ರಹ್ಮನಲ್ಲಿ ಲೀನವಾಗುವುದೋ ಆಗ ಅವನಿಗೆ ಭೇದ ಮಾಡಲು ಹೇಗೆ ಸಾಧ್ಯ? ಆದ್ದರಿಂದ ಯಾವುದೂ ಅವನ ಮುಕ್ತಿಗೆ ಅಡ್ಡಿ ಬರುವುದಿಲ್ಲ.

ಸ್ವಾಮೀಜಿ: ಹೌದು, ನೀನು ಹೇಳುವುದು ನಿಜ. ಮುಕ್ಕಾಲು ಪಾಲು, ವೇದಾಂತಿಗಳೆಲ್ಲಾ ಹಾಗೆಯೇ ಅಭಿಪ್ರಾಯಪಡುತ್ತಾರೆ. ಅದೂ ಒಳ್ಳೆಯದೆ - ಈ ಅಭಿಪ್ರಾಯದಲ್ಲಿ ವ್ಯಕ್ತಿಯ ಮುಕ್ತಿಗೆ ಅಡ್ಡಿಯಿಲ್ಲ. ಆದರೆ ಯಾವ ಮನುಷ್ಯ ತನ್ನ ಜೊತೆಯಲ್ಲೇ ಇಡೀ ವಿಶ್ವವನ್ನೇ ಮುಕ್ತಿಗೆ ಒಯ್ಯುವೆನೆಂದು ಯೋಚಿಸುತ್ತಾನೋ ಅವನೆಂತಹ ಮಹಾತ್ಮನಿರಬೇಕು.

ಶಿಷ್ಯ: ಸ್ವಾಮಿಜಿ, ಇದು ನಮ್ಮ ಧೈರ್ಯವನ್ನು ಸೂಚಿಸಬಹುದೇ ಹೊರತು ಇದಕ್ಕೆ ಶಾಸ್ತ್ರ ಸಮ್ಮತವಿಲ್ಲ.

ಸ್ವಾಮಿಗಳು ಬೇರಾವುದೋ ಯೋಚನೆಯಲ್ಲಿ ಮಗ್ನರಾದುದರಿಂದ ನನ್ನ ಮಾತುಗಳನ್ನು ಕೇಳಿಸಿಕೊಳ್ಳಲಿಲ್ಲ. ಅವರು ಕೊಂಚ ಹೊತ್ತಿನ ಮೇಲೆ ಹೇಳಿದರು; “ಹಗಲೂ ರಾತ್ರಿ ಬ್ರಹ್ಮನನ್ನು ಕುರಿತು ಯೋಚಿಸು, ಧ್ಯಾನಿಸು - ಒಂದೇ ಏಕಾಗ್ರತೆಯಿಂದ ಧ್ಯಾನಿಸು. ಬಾಹ್ಯ ಜೀವನದಲ್ಲಿ ಎಚ್ಚೆತ್ತಿರುವಾಗ ಇತರರಿಗಾಗಿ ಕೆಲಸ ಮಾಡು ಅಥವಾ ನಿನ್ನ ಮನಸ್ಸಿನಲ್ಲಿ ‘ಜಗತ್ತಿಗೂ ಜೀವನಿಗೂ ಕಲ್ಯಾಣವಾಗಲಿ’ ಎಂದು ಮರಳಿ ಮರಳಿ ಯೋಚಿಸು. ‘ಎಲ್ಲರ ಮನಸ್ಸು ಬ್ರಹ್ಮನೆಡೆಗೆ ಹರಿಯಲಿ’ ಎಂದೂ ಪುನಃ ಪುನಃ ಯೋಚಿಸು. ಇಂತಹ ನಿರರ್ಗಳ ಯೋಚನಾ ಪ್ರವಾಹದಿಂದ ಜಗತ್ತಿಗೂ ಉಪಯೋಗವಾಗುವುದು. ಪ್ರಪಂಚದಲ್ಲಿ ಯಾವುದೇ ಒಳ್ಳೆಯ ಕೆಲಸವಾಗಲಿ, ಯೋಚನೆಯಾಗಲಿ, ನಿರರ್ಥಕವಲ್ಲ. ನಿನ್ನ ಆಲೋಚನಾ ಪ್ರವಾಹ ಬಹುಶಃ ಅಮೆರಿಕಾದಲ್ಲಿರುವವನೊಬ್ಬನ ಧಾರ್ಮಿಕ ಭಾವನೆಯನ್ನು ಎಚ್ಚರಿಸಬಹುದು."

ಶಿಷ್ಯ: ಸ್ವಾಮೀಜಿ, ದಯವಿಟ್ಟು ನನ್ನನ್ನು ಹರಸಿ, ನನ್ನ ಮನಸ್ಸು ಸತ್ಯದಲ್ಲಿ ಕೇಂದ್ರೀಕೃತವಾಗುವಂತೆ ಹರಸಿ.

ಸ್ವಾಮೀಜಿ: ಹಾಗೇ ಆಗಲಿ, ನಿನಗೆ ಉತ್ಕಟ ಆಕಾಂಕ್ಷೆ ಇದ್ದರೆ ಖಂಡಿತ ಆಗುವುದು.

\newpage

\chapter[ಅಧ್ಯಾಯ ೩೯]{ಅಧ್ಯಾಯ ೩೯\protect\footnote{\engfoot{C.W, Vol. VII, P. 237}}}

\begin{center}
ಸ್ಥಳ: ಬೇಲೂರು ಮಠ, ವರ್ಷ: ಕ್ರಿ.ಶ. ೧೯೦೧.
\end{center}

ಬೇಲೂರು ಮಠ ಸ್ಥಾಪನೆಯಾದಾಗ ಅನೇಕ ಮಂದಿ ಸಂಪ್ರದಾಯಬದ್ಧ ಹಿಂದೂಗಳು ಮಠದ ಜೀವನದ ರೀತಿಯ ವಿಷಯದಲ್ಲಿ ಕಟುವಾಗಿ ಟೀಕಿಸುತ್ತಿದ್ದರು. ಶಿಷ್ಯನಿಂದ ಈ ಟೀಕೆಯ ವಿಷಯವನ್ನು ಕೇಳಿ ಸ್ವಾಮಿಗಳು ಹೇಳುತ್ತಿದ್ದರು: (ತುಲಸೀದಾಸನ ಶ್ಲೋಕದಲ್ಲಿರುವಂತೆ) “ಆನೆ ಅಂಗಡಿ ಬೀದಿಯಲ್ಲಿ ಹೋಗುತ್ತಿದ್ದರೆ ಸಾವಿರಾರು ನಾಯಿಗಳು ಬೊಗಳಲು ಮೊದಲುಮಾಡುವುವು." ಹೀಗೆ ಪ್ರಾಪಂಚಿಕ ಜನರು ಸಾಧುಗಳನ್ನು ಹಳಿದರೆ ಅದರಿಂದ ಸಾಧುಗಳಿಗೆ ಯಾವ ಕೆಟ್ಟ ಭಾವನೆಯೂ ಬರುವುದಿಲ್ಲ. ಅಥವಾ ಅವರು ಹೀಗೆ ಹೇಳುತ್ತಿದ್ದರು: ‘ಕಿರುಕುಳ ಕೊಡದೆ ಯಾವ ಉಪಯುಕ್ತವಾದ ಉದ್ದೇಶವೂ ಸಮಾಜದ ಹೃದಯವನ್ನು ಪ್ರವೇಶಿಸಲಾರದು.’ ಅಲ್ಲದೆ ಎಲ್ಲರಿಗೂ ‘ಪ್ರತಿಫಲದ ಮೇಲೆ ದೃಷ್ಟಿಯಿಡದೆ ನಿಮ್ಮ ನಿಮ್ಮ ಕರ್ಮದಲ್ಲಿ ನಿರತರಾಗಿ, ಒಂದಲ್ಲ ಒಂದು ದಿನ ಅದರ ಫಲವನ್ನು ಅನುಭವಿಸಿಯೇ ತೀರುವಿರಿ’ ಎಂದು ಬೋಧಿಸುತ್ತಿದ್ದರು. ಅಲ್ಲದೆ ಅನೇಕ ವೇಳೆ ಅವರ ಬಾಯಲ್ಲಿ ‘ಮಗು, ಸತ್ಕರ್ಮಿ ಎಂದೂ ದುಃಖಿಯಾಗುವುದಿಲ್ಲ’ ಎಂಬ ಮಾತು ಸದಾ ಇರುತ್ತಿತ್ತು.

೧೯೦೧ನೇ ಇಸವಿ ಮೇ ಅಥವಾ ಜೂನ್ ತಿಂಗಳಲ್ಲಿ ಶಿಷ್ಯನನ್ನು ನೋಡಿ ಸ್ವಾಮಾಜಿ “ಆದಷ್ಟು ಬೇಗ ನನಗೆ ರಘುನಂದನನ ಅಷ್ಟಾವಿಂಶತಿ ತತ್ತ್ವದ ಒಂದು ಪ್ರತಿಯನ್ನು ತಂದುಕೊಡು" ಎಂದರು.

ಶಿಷ್ಯ: ಹಾಗೇ ಆಗಲಿ. ಆದರೆ ಸ್ವಾಮೀಜಿ, ಈಗಿನ ಕಾಲದಲ್ಲಿ ಆ ಪುಸ್ತಕವನ್ನು ಒಂದು ಮೂಢಾಚಾರದ ರಾಶಿ ಎಂದು ಹೇಳುತ್ತಾರಲ್ಲ - ಅದನ್ನು ಕಟ್ಟಿಕೊಂಡೇನು ಮಾಡುವಿರಿ?

ಸ್ವಾಮೀಜಿ: ಏಕೆ, ರಘುನಂದನನು ಆಗಿನ ಕಾಲದ ಪ್ರಚಂಡ ಪಂಡಿತ. ಪುರಾತನ ಸ್ಮೃತಿಗಳನ್ನೆಲ್ಲಾ ಒಂದುಗೂಡಿಸಿ, ಹಿಂದೂಗಳ ಪದ್ಧತಿ ಸಂಪ್ರದಾಯಗಳನ್ನೆಲ್ಲಾ ಆಗಿನ ಕಾಲಕ್ಕೆ ತಕ್ಕಂತೆ ಹೊಸದಾಗಿ ಮತ್ತು ವಾತಾವರಣಕ್ಕೆ ಸರಿ ಹೊಂದುವಂತೆ ಮಾರ್ಪಡಿಸಿ, ಧರ್ಮಸೂತ್ರಗಳನ್ನು ಬರೆದನು. ಇಡೀ ಬಂಗಾಳ ಅವನಿಂದ ಬಂದ ಆ ನಿಯಮಗಳನ್ನೇ ಅನುಸರಿಸುತ್ತದೆ. ಆದರೆ ಹಿಂದೂಗಳನ್ನು ಮೃತ್ಯುಸಮವಾದ ಸಿದ್ಧಾಂತಕ್ಕೆ ಕಟ್ಟಲು ಹೋಗಿ ಸಮಾಜವನ್ನೇ ಅಧೋಗತಿಗೆ ತಂದನು. ಮುಖ್ಯ ವಿಷಯಗಳ ವಿಚಾರ ಹಾಗಿರಲಿ; ತಿನ್ನುವುದು, ಮಲಗುವುದು ಮುಂತಾದ ನಿತ್ಯಕ್ರಿಯೆಗಳ ವಿಷಯದಲ್ಲೂ ಅವನು ಜನರನ್ನು ನಿಯಮಗಳಿಗೆ ಒಳಪಡಿಸಲು ಯತ್ನಿಸಿದನು. ಈಗಿನ ಬದಲಾಯಿಸಿದ ಕಾಲದಲ್ಲಿ ಅದು ಬಹುಕಾಲ ನಿಲ್ಲಲಾರದು. ಎಲ್ಲಾ ದೇಶಗಳಲ್ಲೂ ಎಲ್ಲಾ ಕಾಲಗಳಲ್ಲೂ ಸಮಾಜದ ವಾಡಿಕೆ ಸಂಪ್ರದಾಯಗಳನ್ನೊಳಗೊಂಡ ಕರ್ಮಕಾಂಡವು ಅದರ ರೂಪವನ್ನು ಬದಲಾಯಿಸುವುದು. ಕೇವಲ ಜ್ಞಾನಕಾಂಡ ಮಾತ್ರ ಬದಲಾಯಿಸುವುದಿಲ್ಲ. ವೇದಗಳ ಕಾಲದಲ್ಲಿ ಕೂಡ ವಿಧಿನಿಯಮಗಳು ಕ್ರಮೇಣ ಬದಲಾಯಿಸಿದುದನ್ನು ನೀನು ನೋಡುವೆ. ಆದರೆ ಉಪನಿಷತ್ತಿನ ತತ್ತ್ವಭಾಗಗಳು ಮಾತ್ರ ಇಲ್ಲಿಯವರೆಗೂ ಬದಲಾಯಿಸದೆ ಹಾಗೇ ಇವೆ - ಅದರ ಬಗ್ಗೆ ಹಲವು ಭಾಷ್ಯಗಳಿವೆ ಅಷ್ಟೆ.

ಶಿಷ್ಯ: ರಘುನಂದನನ ಸ್ಮೃತಿಯಿಂದ ನೀವೇನು ಮಾಡುವಿರಿ?

ಸ್ವಾಮೀಜಿ: ನಾನು ಈ ಬಾರಿ ದುರ್ಗಾಪೂಜೆಯನ್ನು ಆಚರಿಸಬೇಕೆಂದಿರುವೆ. ಖರ್ಚೂ ಒದಗಿದರೆ ಮಹಾಮಾಯಾ ಪೂಜೆಯನ್ನೂ ಮಾಡುವೆ. ಅದಕ್ಕೆ ಆ ಪೂಜೆಯ ವಿಧಿವತ್ತಾದ ಕ್ರಮವನ್ನು ಓದಬೇಕೆಂದಿರುವೆ. ಮುಂದಿನ ಭಾನುವಾರ ನೀನು ಬರುವಾಗ ಅದರದೊಂದು ಪ್ರತಿಯನ್ನು ತಂದೇ ತೀರಬೇಕು.

ಶಿಷ್ಯ: ಹಾಗೇ ಆಗಲಿ, ಸ್ವಾಮಿಜಿ.

ಮುಂದಿನ ಶನಿವಾರ ಶಿಷ್ಯ ಆ ಪುಸ್ತಕದ ಪ್ರತಿಯೊಂದನ್ನು ತಂದ. ಸ್ವಾಮೀಜಿಗೆ ಅದನ್ನು ತೆಗೆದುಕೊಳ್ಳಲು ತುಂಬಾ ಸಂತೋಷವಾಯಿತು. ಇದಾದ ಒಂದು ವಾರದ ನಂತರ ಅವರು ಶಿಷ್ಯನಿಗೆ “ನೀನು ತಂದ ಆ ರಘುನಂದನನ ಸ್ಮೃತಿಯನ್ನು ನಾನು ಮುಗಿಸಿದೆ. ಸಾಧ್ಯವಿದ್ದಲ್ಲಿ ನಾನು ಜಗನ್ಮಾತೆಯ ಪೂಜೆಯನ್ನು ಆಚರಿಸುವೆ" ಎಂದು ಹೇಳಿದರು.

ತತ್ಕಾಲದಲ್ಲಿ ದುರ್ಗಾಪೂಜೆ ಬಹು ವಿಜೃಂಭಣೆಯಿಂದ ನಡೆಯಿತು.

\delimiter

ಇದಾದ ಕೊಂಚ ದಿನಗಳ ತರುವಾಯ ಸ್ವಾಮೀಜಿ ಕಾಳಿಘಾಟ್‌ನಲ್ಲಿ ಕಾಳಿ ಮಾತೆಯ ಮುಂದೆ ಹೋಮ ಮಾಡಿದರು. ಈ ಸಂದರ್ಭದ ವಿಚಾರವಾಗಿ ಮಾತನಾಡುತ್ತಾ ಒಮ್ಮೆ ಸ್ವಾಮೀಜಿ ಶಿಷ್ಯನಿಗೆ ಹೀಗೆ ಹೇಳಿದರು: "ಕಾಳಿಘಾಟ್‌ನಲ್ಲಿ ಆಚರಣೆಯ ಸ್ವಾತಂತ್ರ್ಯವು ಇನ್ನೂ ಉಳಿದಿದೆ: ನಾನು ಪಾಶ್ಚಾತ್ಯ ದೇಶದಿಂದ ಹಿಂತಿರುಗಿ ಬಂದವನೆಂದು ತಿಳಿದಿದ್ದರೂ ಆ ದೇವಸ್ಥಾನದ ಅಧಿಕಾರಿಗಳು ನಾನು ದೇವಸ್ಥಾನವನ್ನು ಪ್ರವೇಶಿಸಲು ಸ್ವಲ್ಪವೂ ಅಡ್ಡಿ ಮಾಡಲಿಲ್ಲ. ಅಷ್ಟೇ ಅಲ್ಲದೆ ಅವರು ಪವಿತ್ರ ಪೂಜಾಸ್ಥಳಕ್ಕೆ ನನ್ನನ್ನು ಕರೆದೊಯ್ದು ನಾನು ಮಾತೆಯನ್ನು ಮನದಣಿಯ ಪೂಜೆ ಮಾಡಲು ಸಹಾಯ ಮಾಡಿದರು."

\newpage

\chapter[ಅಧ್ಯಾಯ ೪೦]{ಅಧ್ಯಾಯ ೪೦\protect\footnote{\engfoot{C.W, Vol. VII, P. 239}}}

\begin{center}
ಸ್ಥಳ: ಬೇಲೂರು ಮಠ, ವರ್ಷ: ಕ್ರಿ.ಶ. ೧೯೦೨.
\end{center}

ಇಂದು ಶ‍್ರೀರಾಮಕೃಷ್ಣರ ವಾರ್ಷಿಕೋತ್ಸವ ಸ್ವಾಮಿಜಿಗೆ ಕೂಡ ಕಟ್ಟಕಡೆಯ ಉತ್ಸವ. ಶಿಷ್ಯನು ಶ‍್ರೀರಾಮಕೃಷ್ಣರ ಮೇಲೆ ಕಟ್ಟಿದ ಪ್ರಾರ್ಥನಾ ಶ್ಲೋಕವೊಂದನ್ನು ಸ್ವಾಮೀಜಿಗೆ ಒಪ್ಪಿಸಿದ. ನಂತರ ಸ್ವಾಮೀಜಿಯವರ ಪಾದಗಳನ್ನು ಒತ್ತಲು ಪ್ರಾರಂಭಿಸಿದ. ಪದ್ಯವನ್ನೋದುವ ಮೊದಲು ಸ್ವಾಮೀಜಿ ಶಿಷ್ಯನಿಗೆ ‘ಪಾದಗಳು ಬಹು ನೋಯುತ್ತಿರುವುದರಿಂದ ಮೃದುವಾಗಿ ಒತ್ತು’ ಎಂದು ಹೇಳಿದರು.

ಪದ್ಯವನ್ನೋದಿದ ಮೇಲೆ ಸ್ವಾಮೀಜಿ “ಚೆನ್ನಾಗಿ ಬರೆದಿರುವೆ" ಎಂದರು.

ಸ್ವಾಮೀಜಿಯ ಖಾಯಿಲೆ ಬಹು ದಾರುಣವಾದುದನ್ನು ನೋಡಿ ಶಿಷ್ಯನ ಹೃದಯ ಹಿಂಡಿದಂತಾಯಿತು. ಅವನ ಅಂತರಂಗದ ಭಾವನೆಗಳನ್ನು ತಿಳಿದು ಸ್ವಾಮಿಜಿ ‘ನೀನೇನು ಯೋಚಿಸುತ್ತಿರುವೆ? ದೇಹ ಹುಟ್ಟಿ ಆಗಿದೆ. ಅದು ಸತ್ತೇ ತೀರಬೇಕು. ನನ್ನ ಅಭಿಪ್ರಾಯಗಳಲ್ಲಿ ಕೆಲವನ್ನಾದರೂ ನಿಮ್ಮಲ್ಲಿ ಬೇರೂರುವ ಹಾಗೆ ಮಾಡಿದ್ದರೆ ನಾನು ಹುಟ್ಟಿದ್ದು ನಿರರ್ಥಕವಲ್ಲವೆಂದು ಭಾವಿಸುವೆ’ ಎಂದರು.

ಶಿಷ್ಯ: ನಾವು ನಿಮ್ಮ ಕೃಪೆಗೆ ಅರ್ಹರೆ? ನೀವು ನನ್ನ ಅರ್ಹತೆಯನ್ನು ಗಣನೆಗೆ ತಾರದೆ ನನ್ನನ್ನು ಆಶೀರ್ವದಿಸಿದರೆ ನಾನು ಧನ್ಯನೆಂದು ಭಾವಿಸುವೆ.

- ಸ್ವಾಮೀಜಿ: ಯಾವಾಗಲೂ ತ್ಯಾಗವೇ ತಳಹದಿ ಎಂಬುದನ್ನು ನೆನಪಿನಲ್ಲಿಡು. ಎಲ್ಲಿಯವರೆಗೆ ಈ ಭಾವನೆ ಬೇರೂರುವುದಿಲ್ಲವೋ ಅಲ್ಲಿಯವರೆಗೆ ಬ್ರಹ್ಮ ಅಥವಾ ಇತರ ಭೂಮಂಡಲದ ಯಾವ ದೇವರೂ ಮುಕ್ತಿ ಪಡೆಯಲು ಸಾಧ್ಯವಿಲ್ಲ.

ಶಿಷ್ಯ: ಪ್ರತಿದಿನವೂ ನಿಮ್ಮ ಬಾಯಿಂದ ಇದನ್ನು ಕೇಳುತ್ತಿದ್ದರೂ ನಾನು ಸಾಕ್ಷಾತ್ಕರಿಸಿಕೊಳ್ಳಲಾರದೆ ಹೋದೆನೆಂದು ನನಗೆ ತೀವ್ರ ವೇದನೆಯುಂಟಾಗಿದೆ.

ಸ್ವಾಮೀಜಿ: ತ್ಯಾಗ ಬಂದೇ ಬರುವುದು. ಆದರೆ ಸಕಾಲದಲ್ಲಿ ಬರುವುದು - “ಕಾಲೇನಾತ್ಮನಿ ವಿಂದತಿ," ತತ್ಕಾಲದಲ್ಲಿ ನಮ್ಮಲ್ಲಿಯೇ ನಮಗೆ ಮುಕ್ತಿ ದೊರಕುವುದು. ಹಿಂದಿನ ಜನ್ಮದ ಸಂಸ್ಕಾರಗಳು ತೀರಿದ ಮೇಲೆ ತ್ಯಾಗ ಎದೆಯಲ್ಲಿ ಚಿಗುರುವುದು.

ಸ್ವಲ್ಪ ಹೊತ್ತಿನ ನಂತರ, “ನೀನು ಹೊರಗೇಕೆ ಜನರ ಆ ದೊಡ್ಡ ದೊಂಬಿಗೆ ಹೋಗಬೇಕು? ನೀನು ನನ್ನೊಡನೆಯೆ ಇರು, ನಿರಂಜನನಿಗೆ ಬಾಗಿಲಲ್ಲಿಯೇ ಇರಲು ಹೇಳು. ಯಾರೂ ನನಗೆ ಇಂದು ತೊಂದರೆ ಕೊಡದೆ ಇರಲಿ" ಎಂದರು.

ನಂತರ ಕೆಳಗಣ ಸಂಭಾಷಣೆ ಶಿಷ್ಯನಿಗೂ ಸ್ವಾಮೀಜಿಗೂ ನಡೆಯಿತು.

ಸ್ವಾಮೀಜಿ: ಇನ್ನು ಮುಂದೆ ವಾರ್ಷಿಕೋತ್ಸವ ಬೇರೆ ವಿಧವಾಗಿ ನಡೆದರೆ ವಾಸಿ ಎಂದು ನನಗೆ ತೋರುತ್ತದೆ. ಒಂದು ದಿನಕ್ಕೆ ಬದಲು ಉತ್ಸವ ಐದಾರು ದಿನ ನಡೆದರೆ ವಾಸಿ. ಮೊದಲನೆ ದಿನ ಧರ್ಮಶಾಸ್ತ್ರಗಳನ್ನು ಓದುವುದು, ಅದಕ್ಕೆ ಅರ್ಥ ಹೇಳುವುದು. ಎರಡನೆಯ ದಿನ ವೇದಾಂತಗಳ ಮೇಲೆ, ಅವುಗಳ ಸಮಸ್ಯೆ ಮತ್ತು ಪರಿಹಾರಗಳ ಮೇಲೆ ಚರ್ಚಾಗೋಷ್ಠಿಯನ್ನು ಏರ್ಪಡಿಸುವುದು. ಮೂರನೆಯ ದಿನ ಉಪನ್ಯಾಸಗಳನ್ನಿಡುವುದು. ಕಡೆಯ ದಿನ ಈಗ ನಡೆಯುವಂತೆ ಹಬ್ಬವನ್ನಾಚರಿಸುವುದು. ದುರ್ಗಾಪೂಜೆಯಂತೆಯೆ ನಾಲ್ಕೈದು ದಿನ ನಡೆಯಬಹುದು. ಮೇಲೆ ಹೇಳಿದ ಹಾಗೆ ಮಾಡಿದರೆ ಶ‍್ರೀರಾಮಕೃಷ್ಣರ ಶಿಷ್ಯರ ಹೊರತು ಮತ್ತಾರೂ ಕಡೆಯ ದಿನ ಬಿಟ್ಟು ಉಳಿದ ದಿನಗಳಲ್ಲಿ ಅದಕ್ಕೆ ಸಹಕರಿಸಲಾರರು. ಆದರೆ ಅದರಿಂದೇನೂ ಬಾಧಕವಿಲ್ಲ. ಒಂದು ದೊಡ್ಡ ಜನದೊಂಬಿ ಸೇರಿದರೆ ಶ‍್ರೀರಾಮಕೃಷ್ಣರ ಸಂದೇಶ ಸಾರಿದಂತಾಗುವುದಿಲ್ಲ.

ಶಿಷ್ಯ: ಎಂತಹ ಸುಂದರವಾದ ಆಲೋಚನೆ. ಮುಂದಿನ ಬಾರಿ ತಮ್ಮ ಇಚ್ಛೆಯಂತೆಯೇ ಮಾಡಬಹುದು.

ಸ್ವಾಮೀಜಿ: ನೋಡು ಮಗು, ನೀವೆಲ್ಲಾ ಅದರಂತೆ ಮಾಡಿ, ನನಗೆ ಅದರ ಮೇಲೆ ಅಂತಹ ಮನಸ್ಸೇನೂ ಇಲ್ಲ.

ಶಿಷ್ಯ: ಸ್ವಾಮೀಜಿ, ಈ ಬಾರಿ ಅನೇಕ ಸಂಗೀತಗಾರರ ಗುಂಪು ಬಂದಿದೆ.

ಈ ಮಾತನ್ನು ಕೇಳಿ ಸ್ವಾಮೀಜಿ ಎದ್ದು ಕಿಟಕಿಯ ಕಂಬಿಗಳನ್ನು ಹಿಡಿದುಕೊಂಡು ಗುಂಪುಗೂಡಿದ ಭಕ್ತಗಣವನ್ನು ನೋಡಿದರು. ಕೊಂಚ ಹೊತ್ತಿನ ಮೇಲೆ ಕುಳಿತುಕೊಂಡರು.

ಸ್ವಾಮೀಜಿ: ಶ‍್ರೀರಾಮಕೃಷ್ಣರ ದೈವಲೀಲೆಯಲ್ಲಿ ನೀವೆಲ್ಲಾ ನಟರು. ಮುಂದೆ, ನಮ್ಮ ಮಾತೇಕೆ, ಜನರು ನಿಮ್ಮ ಹೆಸರನ್ನೂ ಪವಿತ್ರವಾಗಿ ಭಾವಿಸುವರು. ನೀನು ಬರೆಯುತ್ತಿರುವ ಈ ಶ್ಲೋಕಗಳು ಪ್ರೇಮ ಮತ್ತು ಜ್ಞಾನ ಸಂಪಾದನೆಗಾಗಿ ಜನರಿಂದ ಓದಲ್ಪಡುತ್ತವೆ. ಆತ್ಮಜ್ಞಾನ ಪಡೆಯುವುದೇ ಜೀವನದ ಮಹೋದ್ದೇಶವೆಂದು ತಿಳಿ. ಜಗದ್ಗುರುಗಳಾದ ಅವತಾರಪುರುಷರಲ್ಲಿ ನಿನಗೆ ಭಕ್ತಿಯಿದ್ದರೆ ಅದು ತನ್ನಷ್ಟಕ್ಕೆ ತಾನೇ ಸಕಾಲದಲ್ಲಿ ಆವಿರ್ಭಾವವಾಗುವುದು.

ಶಿಷ್ಯ: ಸ್ವಾಮಿಜಿ, ನನಗೆ ಜ್ಞಾನಲಾಭವಾಗುವುದೆ?

ಸ್ವಾಮೀಜಿ: ಶ‍್ರೀರಾಮಕೃಷ್ಣರ ಆಶೀರ್ವಾದದಿಂದ ನಿನಗೆ ಜ್ಞಾನಲಾಭವಾಗುವುದು. ನಿನಗೆ ಪ್ರಾಪಂಚಿಕ ಜೀವನದಲ್ಲಿ ಹೆಚ್ಚು ಸುಖ ಸಿಗುವುದಿಲ್ಲ.

ಶಿಷ್ಯ: ನೀವು ನನ್ನ ಮನಸ್ಸಿನ ದುರ್ಬಲತೆಯನ್ನು ನಾಶಮಾಡಲು ಅನುಗ್ರಹಿಸಿದರೆ ಮಾತ್ರ ನನಗೆ ಭರವಸೆಯುಂಟಾಗುವುದು.

ಸ್ವಾಮೀಜಿ: ಅಂಜಿಕೆಯೇಕೆ? ನೀನು ಅನಿರೀಕ್ಷಿತವಾಗಿ ಇಲ್ಲಿಗೆ ಬಂದಿರುವೆ. ನೀನು ಮುಕ್ತಿ ಹೊಂದೇ ತೀರುವೆ.

ಶಿಷ್ಯ: (ದೈನ್ಯದಿಂದ) ನೀವು ನನ್ನನ್ನು ಕಾಪಾಡಬೇಕು. ಈ ಜನ್ಮದಲ್ಲೇ ಅಜ್ಞಾನದಿಂದ ನನ್ನನ್ನು ಉದ್ಧರಿಸಬೇಕು.

ಸ್ವಾಮೀಜಿ: ಯಾರು ಯಾರನ್ನು ಕಾಪಾಡಲು ಸಾಧ್ಯ ಹೇಳು? ಗುರು ಕೆಲವು ತೆರೆಗಳನ್ನು ತೆಗೆದುಹಾಕಲು ಮಾತ್ರ ಸಾಧ್ಯ. ಯಾವಾಗ ಈ ಮುಸುಕುಗಳೆಲ್ಲಾ ಹೋಗುವುವೋ ಆಗ ನಮ್ಮ ಆತ್ಮ ಸ್ವಯಂಪ್ರಕಾಶಮಾನವಾಗಿ ಸೂರ್ಯನಂತೆ ಪ್ರಜ್ವಲಿಸುವುದು.

ಶಿಷ್ಯ: ಹಾಗಾದರೆ ಶಾಸ್ತ್ರದಲ್ಲಿ ನಾವೇಕೆ ಕೃಪೆ ಎಂಬುದನ್ನು ಓದುತ್ತೇವಲ್ಲ.

ಸ್ವಾಮೀಜಿ: ಕೃಪೆ ಎಂದರೆ ಇದು: ಯಾರಿಗೆ ಆತ್ಮಸಾಕ್ಷಾತ್ಕಾರವಾಗಿದೆಯೊ ಅವನು ದೊಡ್ಡ ಶಕ್ತಿಯ ಭಂಡಾರವಾಗಿರುವನು. ಆತನನ್ನು ಕೇಂದ್ರವಾಗಿಟ್ಟುಕೊಂಡು ಒಂದು ವೃತ್ತವನ್ನೆಳೆದರೆ ಯಾರು ಆ ವೃತ್ತದೊಳಗೆ ಬರುವರೋ ಅವರು ಆತನ ಅಭಿಪ್ರಾಯಗಳ ಆಕರ್ಷಣೆಗೊಳಗಾಗುವರು. ಹಾಗೇ ಹೆಚ್ಚು ತಪಸ್ಸಿಲ್ಲದೆ ಆತನ ಅದ್ಭುತ ಆಧ್ಯಾತ್ಮಿಕತೆಯ ಫಲವನ್ನು ಹೊಂದುವರು. ಇದನ್ನು ಕೃಪೆ ಎಂದು ನೀನು ಕರೆಯುವುದಾದರೆ ಹಾಗೇ ಕರೆ.

ಶಿಷ್ಯ: ಇದಕ್ಕಿಂತ ಹೆಚ್ಚಿನ ಪದವಿ ಇಲ್ಲವೆ?

ಸ್ವಾಮೀಜಿ: ಇದೆ. ಅವತಾರಪುರುಷರು ಬಂದಾಗ ಅವರೊಡನೆ ವಿಶ್ವ ನಾಟಕದ ಸಹಾಯಾರ್ಥವಾಗಿ ಅನೇಕ ಮುಕ್ತಾತ್ಮರೂ ಬರುವರು. ಲಕ್ಷಾಂತರ ಜೀವಿಗಳನ್ನು ಅಜ್ಞಾನದಿಂದ ಉದ್ಧಾರಮಾಡಿ ಅವರಿಗೆ ಅದೇ ಜನ್ಮದಲ್ಲೇ ಮುಕ್ತಿ ಕೊಡಲು ಅವತಾರಪುರುಷನಿಗೆ ಮಾತ್ರ ಸಾಧ್ಯ. ಇದಕ್ಕೆ ಕೃಪೆ ಎನ್ನುವರು. ನಿನಗೆ ಅರ್ಥವಾಯಿತೆ?

ಶಿಷ್ಯ: ಆಯಿತು ಸ್ವಾಮಿಜಿ. ಅಂತಹವರ ದರ್ಶನಲಾಭದಿಂದ ಧನ್ಯರಾಗದವರ ಪಾಡೇನು?

ಸ್ವಾಮೀಜಿ: ಅದಕ್ಕೆ ದಾರಿ ಆತನನ್ನು ಪ್ರಾರ್ಥಿಸುವುದೊಂದೆ - ಆತನನ್ನು ಪ್ರಾರ್ಥಿಸುವುದರಿಂದ ಅನೇಕ ಮಂದಿ ಆತನ ದರ್ಶನಲಾಭ ಪಡೆದಿದ್ದಾರೆ. ಮಾನವ ಶರೀರದಲ್ಲೇ ನೋಡಿ ಆತನ ಕೃಪೆಗೆ ಪಾತ್ರರಾಗಿದ್ದಾರೆ.

ಶಿಷ್ಯ: ಶ‍್ರೀರಾಮಕೃಷ್ಣರ ನಿರ್ಯಾಣವಾದ ಮೇಲೆ ನೀವೆಂದಾದರೂ ಅವರನ್ನು ನೋಡಿದ್ದೀರಾ?

ಸ್ವಾಮೀಜಿ: ಅವರು ದೇಹತ್ಯಾಗ ಮಾಡಿದಮೇಲೆ ನಾನು ಘಾಜಿಪುರದ ಪವಾಹಾರಿ ಬಾಬಾರೊಡನೆ ಕೆಲವು ಕಾಲ ಸಂಪರ್ಕವನ್ನಿಟ್ಟುಕೊಂಡಿದ್ದೆ. ಅವರ ಆಶ್ರಮಕ್ಕೆ ಕೊಂಚ ದೂರದಲ್ಲಿರುವ ತೋಟವೊಂದರಲ್ಲಿ ನಾನು ವಾಸಿಸುತ್ತಿದ್ದೆ. ಜನರು ಅದೊಂದು ದೆವ್ವದ ತೋಟವೆನ್ನುತ್ತಿದ್ದರು. ಆದರೆ ನನಗೇ ಗೊತ್ತಿರುವಂತೆ ನಾನೇ ಒಂದು ಭೂತವಾಗಿದ್ದುದರಿಂದ ನನಗೆ ದೆವ್ವದ ಭೀತಿಯೇನೂ ಇರಲಿಲ್ಲ. ಅದರಲ್ಲಿ ಅನೇಕ ನಿಂಬೆಹಣ್ಣಿನ ಗಿಡಗಳಿದ್ದು ಬೇಕಾದಷ್ಟು ಹಣ್ಣು ಬಿಟ್ಟಿತ್ತು. ಆಗ ನಾನು ಅತಿಸಾರದಿಂದ ನರಳುತ್ತಿದ್ದೆ. ಬ್ರೆಡ್ ಹೊರತು ಮತ್ತೇನೂ ಆಹಾರವಿರಲಿಲ್ಲ. ಆದ್ದರಿಂದ ಜೀರ್ಣಶಕ್ತಿಯನ್ನು ವೃದ್ಧಿ ಮಾಡಿಕೊಳ್ಳಲು ನಾನು ತುಂಬಾ ನಿಂಬೆಹಣ್ಣನ್ನು ತಿನ್ನುತ್ತಿದ್ದೆ. ಪವಾಹಾರಿ ಬಾಬಾ ಜೊತೆ ಇದ್ದು ನಾನು ಅವರನ್ನು ತುಂಬಾ ಇಷ್ಟಪಟ್ಟೆ, ಆತನನ್ನು ತುಂಬಾ ಪ್ರೀತಿಸತೊಡಗಿದೆ. ಶ‍್ರೀರಾಮಕೃಷ್ಣರ ಜೊತೆಯಲ್ಲಿ ಅಷ್ಟು ದಿನ ಇದ್ದರೂ ದೇಹವನ್ನು ಬಲಿಷ್ಠಮಾಡಲು ಯಾವ ಕಲೆಯನ್ನೂ ನಾನು ಕಲಿಯಲಿಲ್ಲ ಎಂದು ಒಂದು ದಿನ ಯೋಚಿಸಿದೆ. ನಾನು ಪವಾಹಾರಿಬಾಬರವರಿಗೆ ಹಠಯೋಗ ಗೊತ್ತಿತ್ತೆಂದು ಕೇಳಿದ್ದೆ. ನನಗೆ ತಿಳಿದಿರುವಂತೆ ನಾನು ದೃಢ ನಿರ್ಧಾರದಿಂದ ಯಾವುದರ ಮೇಲೆ ಮನಸ್ಸಿಡುವೆನೋ ಅದನ್ನು ಮಾಡೇ ತೀರುವೆ. ನಾನು ದೀಕ್ಷೆ ತೆಗೆದುಕೊಳ್ಳುವ ಹಿಂದಿನ ಸಂಜೆ ನಾನೊಂದು ಮಂಚದ ಮೇಲೆ ಮಲಗಿ ಯೋಚಿಸುತ್ತಿದ್ದೆ. ತಕ್ಷಣವೇ ನನ್ನ ಬಲಭಾಗದಲ್ಲಿ ಶ‍್ರೀರಾಮಕೃಷ್ಣರು ಬಹು ದುಃಖಿತರಾದಂತೆ ನಿಂತಿರುವುದನ್ನು ನೋಡಿದೆ. ನಾನು ಅವರಿಗೆ ನನ್ನನ್ನು ಸಂಪೂರ್ಣ ಅರ್ಪಿಸಿದ್ದೆ. ಈಗ ಮತ್ತೊಬ್ಬ ಗುರುವನ್ನು ಹೊಂದಲು ಯೋಚಿಸುತ್ತಿದ್ದ ನನ್ನ ಆಲೋಚನೆಯ ವಿಚಾರವಾಗಿ ನನಗೇ ನಾಚಿಕೆಯಾಗಿ ನಾನು ಅವರನ್ನು ನೋಡುತ್ತಾ ಇದ್ದೆ. ಹೀಗೇ ೨-೩ ಗಂಟೆಗಳ ಕಾಲ ಕಳೆದಿರಬಹುದು. ನನ್ನ ಬಾಯಿಂದ ಒಂದು ಮಾತೂ ಹೊರಡಲಿಲ್ಲ. ನಂತರ ಇದ್ದಕ್ಕಿದ್ದಂತೆ ಅವರು ಮಾಯವಾದರು. ಶ‍್ರೀರಾಮಕೃಷ್ಣರನ್ನು ನೋಡಿದುದರಿಂದ ಅಂದು ರಾತ್ರಿ ನನ್ನ ಮನಸ್ಸು ಅಲ್ಲೋಲ ಕಲ್ಲೋಲವಾಗಿ ಪವಾಹಾರಿಬಾಬಾ ಹತ್ತಿರ ದೀಕ್ಷೆ ತೆಗೆದುಕೊಳ್ಳುವುದನ್ನು ಮುಂದಕ್ಕೆ ಹಾಕಿದೆ. ಒಂದೆರಡು ದಿನಗಳಾದ ಮೇಲೆ ಪುನಃ ನನ್ನ ಮನಸ್ಸಿನಲ್ಲಿ ಪವಾಹಾರಿಬಾಬಾರವರಲ್ಲಿ ದೀಕ್ಷೆ ತೆಗೆದುಕೊಳ್ಳುವ ಯೋಚನೆ ಬಂತು. ಪುನಃ ರಾತ್ರಿ ಹಿಂದಿನಂತೆ ಶ‍್ರೀರಾಮಕೃಷ್ಣರು ಕಾಣಿಸಿಕೊಂಡರು. ಹೀಗೆ ಒಂದಾದಮೇಲೊಂದರಂತೆ ಅನೇಕ ದಿನಗಳು ನನಗೆ ಶ‍್ರೀರಾಮಕೃಷ್ಣರ ದರ್ಶನವಾದ ಮೇಲೆ ಪ್ರತಿ ಬಾರಿ ನಾನು ನಿರ್ಧಾರ ಮಾಡಿದಾಗಲೂ ಆ ರೀತಿ ದರ್ಶನವಾಗುತ್ತಿದ್ದರಿಂದ ಇದರಿಂದ ಕೆಡಕುಂಟಾಗುವುದಲ್ಲದೆ ಒಳ್ಳೆಯದೇನೂ ಆಗುವುದಿಲ್ಲವೆಂದು ನಾನು ದೀಕ್ಷೆ ತೆಗೆದುಕೊಳ್ಳುವ ಯೋಚನೆಯನ್ನು ಬಿಟ್ಟುಬಿಟ್ಟೆ.

ಕೊಂಚ ಕಾಲಾನಂತರ ಸ್ವಾಮೀಜಿಯವರು ಶಿಷ್ಯನನ್ನು ಕುರಿತು ಹೇಳಿದರು: “ಶ‍್ರೀರಾಮಕೃಷ್ಣರನ್ನು ನೋಡಿದವರೆಲ್ಲಾ ಧನ್ಯರು. ಅವರ ಜನ್ಮ ಅವರ ಕರ್ಮವೆಲ್ಲ ಪವಿತ್ರವಾಗಿವೆ. ನೀವೆಲ್ಲಾ ಅವರ ದರ್ಶನ ಪಡೆದೇ ಪಡೆಯುವಿರಿ. ನೀವು ಇಲ್ಲಿಗೆ ಬಂದಮೇಲೆ ಅವರಿಗೆ ಬಹಳ ಸಮೀಪದಲ್ಲಿರುವಿರಿ. ಶ‍್ರೀರಾಮಕೃಷ್ಣರಂತೆ ಈ ಭೂಮಿಗೆ ಯಾರು ಬಂದರೆಂದು ಯಾರಿಗೂ ಅರ್ಥವಾಗುವುದಿಲ್ಲ. ಅವರ ಅಂತರಂಗದ ಶಿಷ್ಯರೂ ಕೂಡ ಅದರ ಸತ್ಯಸಂಗತಿಯನ್ನರಿಯರು. ಕೇವಲ ಕೆಲವರಿಗೆ ಅವರ ವಿಚಾರ ಕೊಂಚ ತಿಳಿದಿದೆ. ಎಲ್ಲರಿಗೂ ಆಮೇಲೆ ಅರ್ಥವಾಗುವುದು.”

ಸಂಭಾಷಣೆ ನಡೆಯುತ್ತಿರುವಾಗ ಸ್ವಾಮಿ ನಿರಂಜನಾನಂದರು ಬಾಗಿಲನ್ನು ತಟ್ಟಲು ಶಿಷ್ಯನು ಎದ್ದು “ಯಾರು ಬಂದಿದ್ದಾರೆ?" ಎಂದು ವಿಚಾರಿಸಿದ. ನಿರಂಜನರು “ಸೋದರಿ ನಿವೇದಿತ ಮತ್ತು ಇತರ ಆಂಗ್ಲೇಯ ಸ್ತ್ರೀಯರು" ಎಂದಮೇಲೆ ಅವರನ್ನು ಕೊಠಡಿಯೊಳಕ್ಕೆ ಬಿಡಲಾಯಿತು. ಅವರು ನೆಲದಮೇಲೆ ಕುಳಿತುಕೊಂಡು ಸ್ವಾಮೀಜಿಯವರ ಆರೋಗ್ಯವನ್ನು ವಿಚಾರಿಸಿಕೊಂಡರು. ಸ್ವಲ್ಪ ಮಾತಾದ ಮೇಲೆ ಅವರು ಹೊರಟು ಹೋದರು. ನಂತರ ಸ್ವಾಮೀಜಿ ಶಿಷ್ಯನಿಗೆ ಹೇಳಿದರು: “ನೋಡು, ಅವರೆಷ್ಟು ಸುಸಂಸ್ಕೃತರು. ಅವರು ಬಂಗಾಳಿಗಳಾಗಿದ್ದರೆ ನಾನು ಆರೋಗ್ಯವಾಗಿಲ್ಲವೆಂದು ತಿಳಿದಿದ್ದರೂ ಅರ್ಧಗಂಟೆಯ ಹೊತ್ತಾದರೂ ಮಾತನಾಡದೆ ಹೋಗುತ್ತಿರಲಿಲ್ಲ."

ಈಗ ಎರಡೂವರೆ ಗಂಟೆಯಾಗಿರಬಹುದು. ಹೊರಗಡೆ ಬಹುಮಂದಿ ಸೇರಿದ್ದರು. ಸ್ವಾಮೀಜಿ ಶಿಷ್ಯನ ಮನೋಗತವನ್ನರಿತು “ಕೊಂಚ ಹೊರಗಡೆ ಹೋಗಿ ನೋಡಿಕೊಂಡು ಬಾ-ಆದರೆ ಬೇಗ ಬಂದುಬಿಡು" ಎಂದರು.

\newpage

\chapter[ಅಧ್ಯಾಯ ೪೧]{ಅಧ್ಯಾಯ ೪೧\protect\footnote{\engfoot{C.W, Vol. VII, P. 244}}}

\begin{center}
ಸ್ಥಳ: ಬೇಲೂರು ಮಠ, ವರ್ಷ: ಕ್ರಿ.ಶ. ೧೯೦೨.
\end{center}

ಪೂರ್ವ ಬಂಗಾಳದಿಂದ ಹಿಂತಿರುಗಿ ಬಂದಾಗಿನಿಂದಲೂ ಸ್ವಾಮೀಜಿ ಮಠದಲ್ಲಿಯೇ ಇದ್ದು ಮಗುವಿನಂತಹ ಜೀವನ ನಡೆಸುತ್ತಿದ್ದರು. ಪ್ರತಿ ವರುಷವೂ ಕೆಲವು ಸಂತಾಲ ಕೆಲಸಗಾರರು ಮಠದಲ್ಲಿ ಕೆಲಸ ಮಾಡುತ್ತಿದ್ದರು. ಸ್ವಾಮೀಜಿ ಅವರೊಂದಿಗೆ ಹಾಸ್ಯ ಮಾಡುತ್ತಾ ಅವರ ಸುಖ ದುಃಖಗಳನ್ನು ವಿಚಾರಿಸುವುದರಲ್ಲಿ ಸಂತೋಷಿಸುತ್ತಿದ್ದರು. ಒಂದು ದಿನ ಕೆಲವು ಮಂದಿ ದೊಡ್ಡ ಮನುಷ್ಯರು ಕಲ್ಕತ್ತೆಯಿಂದ ಸ್ವಾಮೀಜಿಯನ್ನು ಸಂದರ್ಶಿಸಲು ಬಂದರು. ಅಂದು ಸ್ವಾಮೀಜಿ ಸಂತಾಲರೊಡನೆ ಖುಷಿಯಾಗಿ ಮಾತನಾಡುತ್ತಿದ್ದಾಗ ಆ ದೊಡ್ಡ ಮನುಷ್ಯರು ಬಂದಿರುವ ಸಮಾಚಾರವನ್ನು ಕೇಳಿ, ‘ನನಗೆ ಈಗ ಹೋಗಲಾಗುವುದಿಲ್ಲ. ನಾನು ಈ ಜನರೊಡನೆ ಸಂತೋಷವಾಗಿದ್ದೇನೆ’ ಎಂದರು. ಅಂದು ನಿಜವಾಗಿಯೂ ಸ್ವಾಮೀಜಿ ಸಂತಾಲರನ್ನು ಬಿಟ್ಟು ಆ ಮನುಷ್ಯರನ್ನು ನೋಡಲು ಹೋಗಲೇ ಇಲ್ಲ.

ಆ ಸಂತಾಲರಲ್ಲಿ ಒಬ್ಬನ ಹೆಸರು ‘ಕಿಷ್ಟ’ ಎಂದಿತ್ತು. ಸ್ವಾಮೀಜಿ ಆ ಕಿಷ್ಟನನ್ನು ತುಂಬಾ ಪ್ರೀತಿಸುತ್ತಿದ್ದರು. ಸ್ವಾಮೀಜಿ ಅವರೊಡನೆ ಮಾತನಾಡಲು ಹೋದಾಗ ಕಿಷ್ಟನು ಸ್ವಾಮೀಜಿಗೆ ಹೇಳುತ್ತಿದ್ದ “ಓ ಸ್ವಾಮೀಜಿ, ನಾನು ಕೆಲಸಮಾಡುತ್ತಿದ್ದಾಗ ಬರಬೇಡಿ. ನಿಮ್ಮೊಡನೆ ಮಾತನಾಡುತ್ತಾ ಸಮಯವೆಲ್ಲಾ ಕಳೆದುಹೋಗಿ ಕೆಲಸವೇ ಸಾಗುವುದಿಲ್ಲ. ಕೊನೆಗೆ ಇದರ ಮೇಲ್ವಿಚಾರಕರಾದ ಸ್ವಾಮೀಜಿ ಬಂದಾಗ ನಮಗೆ ಚೆನ್ನಾಗಿ ಬೈಗುಳ ಸಿಗುತ್ತೆ." ಸ್ವಾಮಿಜಿ ಇದರಿಂದ ಎದೆಕರಗಿ ಹೇಳಿದರು: “ಇಲ್ಲ, ಇಲ್ಲ ಅವರೇನೂ ಬೈಯುವುದಿಲ್ಲ. ನಿಮ್ಮ ಊರಿನ ವಿಚಾರ ಕೊಂಚ ಹೇಳಿ." ಮಾತನಾಡುತ್ತಾ ಹಾಗೇ ಅವರ ಸಂಸಾರ ಜೀವನದ ವಿಚಾರವನ್ನೂ ಕೇಳುತ್ತಿದ್ದರು.

ಒಂದು ದಿನ ಸ್ವಾಮೀಜಿ ಕಿಷ್ಟನನ್ನು ಕೇಳಿದರು: “ನೀವು ಒಂದು ದಿನ ಇಲ್ಲಿ ಊಟಮಾಡುವಿರಾ?" ಕಿಷ್ಟ ಹೇಳಿದ: “ನೀವು ಮುಟ್ಟಿದ ಪದಾರ್ಥ ನಾವು ತಿನ್ನುವುದಿಲ್ಲ. ನೀವು ನಮ್ಮ ಆಹಾರಕ್ಕೆ ಉಪ್ಪನ್ನು ಹಾಕಿದರೆ ನಮ್ಮ ಜಾತಿ ಹೋಗುವುದು." ಸ್ವಾಮಾಜಿ “ನೀವು ಉಪ್ಪನ್ನೇಕೆ ತಿನ್ನಬೇಕು. ಉಪ್ಪು ಹಾಕದ ಪಲ್ಯ ಮಾಡಿದರೆ ನೀವು ಉಣ್ಣಬಹುದಲ್ಲವೆ?" ಎಂದು ಕೇಳಿದರು. ಕಿಷ್ಟ ಒಪ್ಪಿಕೊಂಡ. ನಂತರ ಸ್ವಾಮೀಜಿ ಅಪ್ಪಣೆಯಂತೆ ಬ್ರೆಡ್, ಪಲ್ಯ, ಮಿಠಾಯಿ, ಮೊಸರು ಮುಂತಾದುವನ್ನು ತಂದು ಸಂತಾಲರ ಊಟಕ್ಕಾಗಿ ಸಿದ್ದಪಡಿಸಲಾಯಿತು. ಸ್ವಾಮೀಜಿ ಅವರೆಲ್ಲಾ ತಮ್ಮ ಮುಂದೆಯೇ ಊಟ ಮಾಡುವಂತೆ ಮಾಡಿದರು. ಊಟ ಮಾಡುತ್ತಿದ್ದಾಗ ಕಿಷ್ಟ ಹೇಳಿದ: “ನಿಮಗೆ ಹೇಗೆ ಈ ಪದಾರ್ಥ ಸಿಕ್ಕಿತು? ಇದರಂತಹ ರುಚಿಯನ್ನು ನಾನೆಂದೂ ಕಂಡೇ ಇರಲಿಲ್ಲ." ಅವರಿಗೆ ಹೊಟ್ಟೆ ತುಂಬಾ ಊಟ ಮಾಡಿಸುತ್ತಾ ಸ್ವಾಮೀಜಿ ಹೇಳಿದರು: “ನೀವೆಲ್ಲಾ ನಾರಾಯಣರು, ದೇವರ ಆವಿರ್ಭಾವ. ಇಂದು ನಾನು ನಾರಾಯಣನಿಗೆ ಊಟವನ್ನರ್ಪಿಸಿದ್ದೇನೆ. ದರಿದ್ರ ನಾರಾಯಣ ಸೇವೆ." ಬಡವರಲ್ಲಿ ದೇವರನ್ನು ಕಾಣುವುದು ಎನ್ನುತ್ತಿದ್ದುದನ್ನು ಅಂದು ಸ್ವಾಮೀಜಿ ಕಾರ್ಯರೂಪದಲ್ಲಿ ನೆರವೇರಿಸಿದರು.

ಊಟವಾದ ನಂತರ ಸಂತಾಲರು ವಿಶ್ರಾಂತಿಗೆ ತೆರಳಿದರು. ಸ್ವಾಮೀಜಿ ಶಿಷ್ಯನನ್ನುದ್ದೇಶಿಸಿ ಹೇಳಿದರು: "ಅವರಲ್ಲಿ ಎಷ್ಟು ಸರಳತೆ ಇದೆ ನೋಡಿದೆಯಾ? ಅವರ ಕಷ್ಟವನ್ನು ಕೊಂಚವಾದರೂ ಕಡಿಮೆ ಮಾಡಬಲ್ಲಿರಾ? ಇಲ್ಲದಿದ್ದಲ್ಲಿ ಕಾವಿಬಟ್ಟೆ ಧರಿಸಿ ಪ್ರಯೋಜನವೇನು? ಇತರರ ಕಲ್ಯಾಣಕ್ಕೋಸುಗ ಎಲ್ಲವನ್ನೂ ತ್ಯಾಗಮಾಡುವುದೇ ನಿಜವಾದ ಸಂನ್ಯಾಸ. ಅವರು ಜೀವನದಲ್ಲಿ ಏನೊಂದು ಒಳ್ಳೆಯದನ್ನೂ ಅನುಭವಿಸಿಲ್ಲ. ಕೆಲವು ವೇಳೆ ನನಗೆ ಈ ಮಠವನ್ನೆಲ್ಲಾ ಮಾರಿ ಬಂದ ದುಡ್ಡನ್ನು ಬಡಬಗ್ಗರಿಗೆ ಹಂಚಿಬಿಡಬೇಕೆನ್ನಿಸುವುದು. ನಾವು ಮಠವನ್ನು ನಮ್ಮ ಆಶ್ರಯವನ್ನಾಗಿ ಮಾಡಿಕೊಂಡೆವು. ಅಯ್ಯೋ! ಈ ದೇಶದ ಜನರಿಗೆ ತಿನ್ನಲಿಕ್ಕೆ ಹಿಟ್ಟಿಲ್ಲ. ನಮಗೆ ತುತ್ತೆತ್ತಲಿಕ್ಕಾದರೂ ಹೇಗೆ ಮನಸ್ಸು ಬರುವುದು? ನಾನು ಪಾಶ್ಚಾತ್ಯ ದೇಶದಲ್ಲಿದ್ದಾಗ ಜಗನ್ಮಾತೆಗೆ ಪ್ರಾರ್ಥಿಸುತ್ತಿದ್ದೆ: ಇಲ್ಲಿಯ ಜನರು ಹೂವಿನ ಸುಪ್ಪತ್ತಿಗೆಯಲ್ಲಿ ಮಲಗುತ್ತಿದ್ದಾರೆ. ಎಲ್ಲಾ ಬಗೆಯ ರುಚಿಕರವಾದ ವಸ್ತುಗಳನ್ನೂ ತಿನ್ನುತ್ತಾರೆ. ಅವರು ಅನುಭವಿಸದ ಸುಖವೇನಿದೆ? ನಮ್ಮ ದೇಶದ ಜನರಾದರೂ ಹೊಟ್ಟೆಗಿಲ್ಲದೆ ಸಾಯುತ್ತಿದ್ದಾರೆ. ತಾಯಿ, ಅವರಿಗೆ ಬೇರೆ ಮಾರ್ಗವೇ ಇಲ್ಲವೇ? ನಾನು ಪರದೇಶಕ್ಕೆ ಧರ್ಮ ಬೋಧಿಸಲು ಹೋದುದಕ್ಕೆ ಒಂದು ಮುಖ್ಯ ಉದ್ದೇಶ ನಮ್ಮ ದೇಶದ ಜನರ ಹೊಟ್ಟೆ ತುಂಬಿಸಲು ಅಲ್ಲೇನಾದರೂ ಸಹಾಯವಾಗಬಹುದೆಂದು."

“ನನ್ನ ದೇಶದ ಬಡ ಜನರು ಉಪವಾಸದಿಂದ ನರಳುವುದನ್ನು ನೋಡಿದಾಗ ನನಗೆ ಈ ವಿಧಿಯುಕ್ತ ಪೂಜೆ, ಪಾಂಡಿತ್ಯವೆಲ್ಲವನ್ನೂ ಕಿತ್ತೊಗೆದು ಗ್ರಾಮ ಗ್ರಾಮಗಳಿಗೂ ಹೋಗಿ ಅಲ್ಲಿಯ ಹಣವಂತರಿಗೆ ನಮ್ಮ ಸಾಧನೆ ಮತ್ತು ಶೀಲದ ಬಲದಿಂದ ಮನದಟ್ಟು ಮಾಡಿಸಿ ಹಣವನ್ನು ಕೂಡಿಸಿ ಬಡವರಿಗೆ ಸೇವೆಮಾಡುತ್ತಾ ಇಡೀ ಜೀವನವೆಲ್ಲಾ ಕಳೆಯಬೇಕೆನ್ನಿಸುವುದು.”

“ಅಯ್ಯೋ! ಯಾರೂ ದೇಶದ ಬಡಬಗ್ಗರ ವಿಚಾರವಾಗಿ ಯೋಚಿಸುವುದಿಲ್ಲ. ಅವರೇ ಪಟ್ಟಣದ ಮೂಲಾಧಾರ. ಅವರ ಶ್ರಮದಿಂದ ಆಹಾರ ಉತ್ಪತ್ತಿಯಾಗುತ್ತಿದೆ. ಈ ಬಡಜನರು, ಈ ಗುಡಿಸುವವರು, ಕೂಲಿಕಾರರು ಒಂದು ದಿನ ತಮ್ಮ ಕೆಲಸವನ್ನು ನಿಲ್ಲಿಸಲಿ, ಇಡೀ ಪಟ್ಟಣವೇ ಹಾಹಾಕಾರವೇಳುವುದು. ಆದರೆ ಯಾರೂ ಅವರ ಬಗ್ಗೆ ಕನಿಕರಿಸುವುದಿಲ್ಲ. ಅವರ ಕಷ್ಟಗಳಲ್ಲಿ ಸಂತೈಸುವವರಿಲ್ಲ. ಸ್ವಲ್ಪ ನೋಡು, ಹಿಂದೂಗಳು ಸಹಾನುಭೂತಿ ತೋರದುದರ ಫಲವಾಗಿ ಮದ್ರಾಸಿನಲ್ಲಿ ಸಾವಿರಾರು ಜನ ಹೊಲೆಯರು ಕ್ರೈಸ್ತರಾಗುತ್ತಿದ್ದಾರೆ. ಇದಕ್ಕೆ ಕೇವಲ ಹಸಿವೇ ಮುಖ್ಯ ಕಾರಣವಲ್ಲ. ನಿಮ್ಮಿಂದ ಅವರಿಗೆ ಕೊಂಚವೂ ಸಹಾನುಭೂತಿ ದೊರಕದುದೇ ಇದಕ್ಕೆ ಕಾರಣ. ಹಗಲೂ ರಾತ್ರಿ ನಾವು ಅವರಿಗೆ ‘ಮುಟ್ಟಬೇಡಿ, ಮುಟ್ಟಬೇಡಿ’ ಎಂದು ಕೂಗುತ್ತಿದ್ದೇವೆ. ದೇಶದಲ್ಲಿ ಕೊಂಚವಾದರೂ ಕನಿಕರ, ಸಹಾನುಭೂತಿ ಉಳಿದಿರುವುದೇನು! ಎಲ್ಲೆಲ್ಲೂ ‘ಮುಟ್ಟಬೇಡಿ’ ಎಂಬ ಪಂಗಡದವರೇ! ಅಂತಹ ಪದ್ದತಿಯನ್ನು ಎಸೆಯಿರಿ! ಒಮೊಮ್ಮೆ ನನಗೆ ಈ ‘ಮುಟ್ಟಬೇಡಿ’ ಎಂಬ ಧರ್ಮದ ಎಲ್ಲೆಯನ್ನು ಕತ್ತರಿಸಿ ಹೊರಗೆ ನಿಂತು ಎಲ್ಲರಿಗೂ "ಯಾರು ಬಡವರೋ, ದುಃಖಿಗಳೊ, ದೀನರೋ, ದಲಿತರೋ, ಎಲ್ಲರೂ ಬನ್ನಿ“ ಎಂದು ಕರೆದು ಎಲ್ಲರನ್ನೂ ಶ‍್ರೀರಾಮಕೃಷ್ಣರ ಹೆಸರಿನಲ್ಲಿ ಒಂದುಗೂಡಿಸಬೇಕೆನ್ನಿಸುವುದು. ಅವರು ಮುಂದಕ್ಕೆ ಬಂದ ಹೊರತು ಮಾತೆಯೂ ಎಚ್ಚರಗೊಳ್ಳುವುದಿಲ್ಲ. ಇವರಿಗೆಲ್ಲಾ ಹೊಟ್ಟೆಬಟ್ಟೆಗಾಗುವಷ್ಟನ್ನು ನಾವು ಮಾಡಲಾರದೆ ಹೋದೆವು! ನಾವೇನು ಮಾಡಿದ್ದೇವೆ? ಅಯ್ಯೋ! ಅವರಿಗೆ ಪ್ರಪಂಚ ಏನೆಂಬುದೇ ಗೊತ್ತಿಲ್ಲ. ಅದಕ್ಕೆ ರಾತ್ರಿ ಹಗಲೂ ದುಡಿದರೂ ಸಾಕಾಗುವಷ್ಟು ಆಹಾರ ಬಟ್ಟೆ ಹೊಂದಲು ಶಕ್ತರಾಗಿಲ್ಲ. ನಾವು ಅವರು ಕರೆಯುವಂತೆ ಮಾಡೋಣ. ಈ ಹಗಲಿನಷ್ಟೇ ಸ್ಪಷ್ಟವಾಗಿ ನೋಡುತ್ತಿರುವೆ – ಎಲ್ಲರಲ್ಲೂ ಇರುವುದು ಒಬ್ಬನೇ ಬ್ರಹ್ಮ. ಅವರಲ್ಲಿ, ನನ್ನಲ್ಲಿ - ಎಲ್ಲರಲ್ಲೂ ಒಂದೇ ಶಕ್ತಿ ನೆಲಸಿದೆ. ವ್ಯತ್ಯಾಸ ಅದರ ಆವಿರ್ಭಾವದಲ್ಲಿದೆ. ಇಡೀ ದೇಹದಲ್ಲೆಲ್ಲಾ ರಕ್ತ ಸಂಚರಿಸಿದ ಹೊರತು ಯಾವ ದೇಶವು ಯಾವ ಕಾಲದಲ್ಲಿ ಮುಂದುವರಿದಿದೆ? ಒಂದು ಅಂಗಕ್ಕೆ ಪಾರ್ಶ್ವವಾಯು ಬಂದರೆ ಉಳಿದ ಎಲ್ಲಾ ಅಂಗಗಳು ಸರಿಯಾಗಿದ್ದರೂ ಆ ದೇಹದಿಂದ ಹೆಚ್ಚೇನೂ ಮಾಡಲಾಗುವುದಿಲ್ಲ. ಇದನ್ನು ಚೆನ್ನಾಗಿ ನೆನಪಿನಲ್ಲಿಡು."

ಶಿಷ್ಯ: ಸ್ವಾಮೀಜಿ, ದೇಶದಲ್ಲಿ ಇಷ್ಟೊಂದು ಬಗೆಯ ಧಾರ್ಮಿಕ ಪಂಗಡಗಳು ಅಭಿಪ್ರಾಯಗಳು ಇರುವುದರಿಂದಲೇ ಅವರನ್ನೆಲ್ಲಾ ಒಂದುಗೂಡಿಸಿ ಸಾಮರಸ್ಯ ಬರುವಂತೆ ಮಾಡುವುದು ಬಹುಕಷ್ಟದ ಕೆಲಸ.

ಸ್ವಾಮೀಜಿ: (ಕೋಪದಿಂದ) ನೀನು ಯಾವ ಕೆಲಸವನ್ನೇ ಆಗಲಿ ಕಷ್ಟವೆಂದು ತಿಳಿದಿದ್ದರೆ ಇಲ್ಲಿಗೆ ಬರಬೇಡ. ಭಗವತ್ಕೃಪೆಯಿಂದ ಎಲ್ಲಾ ಹಾದಿಯೂ ಸುಗಮವಾಗುವುದು. ನಿನ್ನ ಕೆಲಸ ಜಾತಿ ಮತ ವರ್ಣಗಳನ್ನು ಲೆಕ್ಕಿಸದೆ ಬಡಬಗ್ಗರಿಗೆ ಸೇವೆ ಮಾಡುವುದಾಗಿದೆ. ಅದರ ಪರಿಣಾಮವನ್ನು ಯೋಚಿಸುವ ಆವಶ್ಯಕತೆಯೇನಿಲ್ಲ. ಕೇವಲ ಕೆಲಸ ಮಾಡುತ್ತಾ ಹೋಗುವುದು ನಿನ್ನ ಕರ್ತವ್ಯ. ನಂತರ ಎಲ್ಲವೂ ತನ್ನಷ್ಟಕ್ಕೆ ತಾನೇ ಹಿಂಬಾಲಿಸುವುದು. ನನ್ನ ಕೆಲಸ ನಿರ್ಮಿಸುವುದು, ಧ್ವಂಸವಲ್ಲ. ವಿಶ್ವದ ಚರಿತ್ರೆಯನ್ನು ಓದಿ ನೋಡು, ಒಂದು ದೇಶದ ಸಂದಿಗ್ಧ ಕಾಲದಲ್ಲಿ ಒಬ್ಬ ಮಹಾತ್ಮನಾದ ವ್ಯಕ್ತಿ ಅದರ ಜೀವನಾಡಿಯಾಗಿರುತ್ತಾನೆ. ಆತನ ಉದ್ದೇಶಗಳಿಂದ ನೂರಾರು ಜನ ಜಾಗೃತರಾಗಿ ಜಗತ್ತಿಗೆ ಒಳ್ಳೆಯದನ್ನು ಮಾಡಿದ್ದಾರೆ. ನೀವೆಲ್ಲಾ ಬುದ್ಧಿವಂತರಾದ ಹುಡುಗರು - ಅನೇಕ ದಿನಗಳಿಂದ ಇಲ್ಲಿಗೆ ಬರುತ್ತಿರುವಿರಿ, ಹೇಳಿ? ಇತರರ ಸೇವೆಗಾಗಿ ಒಂದು ಜೀವನವನ್ನು ಅರ್ಪಿಸಲಾಗುವುದಿಲ್ಲವೆ? ಮುಂದಿನ ಜೀವನದಲ್ಲಿ ವೇದಾಂತ, ಧರ್ಮಶಾಸ್ತ್ರಗಳನ್ನೆಲ್ಲಾ ಓದಬಹುದು. ಈ ಜೀವನವನ್ನು ಇತರರ ಸೇವೆಗಾಗಿ ಅರ್ಪಿಸಿ. ಆಗ ನೀವು ಇಲ್ಲಿಗೆ ಬಂದುದು ವ್ಯರ್ಥವಾಗಲಿಕ್ಕಿಲ್ಲವೆಂದು ನನಗೆ ಗೊತ್ತಾಗುವುದು. ಈ ಮಾತುಗಳನ್ನಾಡುತ್ತಾ ಸ್ವಾಮೀಜಿ ಮೌನವಾಗಿ ಗಾಢ ಯೋಚನಾ ಮಗ್ನರಾದರು. ಕೊಂಚ ಹೊತ್ತಿನ ಮೇಲೆ ಅವರು ಹೇಳಿದರು, ಅಷ್ಟೊಂದು ಕಠಿಣ ಸಾಧನೆ ಮಾಡಿದಮೇಲೆ ನನಗೆ ತಿಳಿದ ಸತ್ಯ ಇದು - ದೇವರು ಎಲ್ಲಾ ಜೀವಿಗಳಲ್ಲೂ ಇದ್ದಾನೆ. ಅವನಿಗಿಂತ ಬೇರೆ ದೇವರಿಲ್ಲ. ಯಾರು ಜೀವರಿಗೆ ಸೇವೆ ಸಲ್ಲಿಸುತ್ತಾನೋ ಅವನು ದೇವರಿಗೆ ಸೇವೆಸಲ್ಲಿಸಿದಂತೆ ಎಂಬುದು ಮನದಟ್ಟಾಯಿತು. ಕೊಂಚ ಹೊತ್ತು ಸುಮ್ಮನಿದ್ದು ಶಿಷ್ಯನನ್ನುದ್ದೇಶಿಸಿ “ಇಂದು ನಾನು ಹೇಳಿದುದನ್ನು ನಿನ್ನ ಹೃದಯದಲ್ಲಿ ಚೆನ್ನಾಗಿ ಬರೆದಿಡು. ನೀನಿದನ್ನು ಮರೆಯದಂತೆ ನೋಡಿಕೊ" ಎಂದರು.

\newpage

\chapter[ಅಧ್ಯಾಯ ೪೨]{ಅಧ್ಯಾಯ ೪೨\protect\footnote{\engfoot{C.W, Vol. VII, P. 2A7}}}

\begin{center}
ಸ್ಥಳ: ಬೇಲೂರು ಮಠ, ವರ್ಷ: ಕ್ರಿ.ಶ. ೧೯೦೨.
\end{center}

ಅಂದು ಶನಿವಾರ, ಶಿಷ್ಯ ಮಠಕ್ಕೆ ಬಂದ. ಮಠದಲ್ಲಿ ಸಾಧುಗಳು ತೀವ್ರ ಸಾಧನೆಯಲ್ಲಿ ನಿರತರಾಗಿದ್ದರು. ಸ್ವಾಮಿಜಿಯವರು ಮಠದ ಎಲ್ಲಾ ಬ್ರಹ್ಮಚಾರಿಗಳಿಗೂ ಸಂನ್ಯಾಸಿಗಳಿಗೂ ಬೆಳಗಿನ ಜಾವಕ್ಕೆ ಬಹು ಮುಂಚೆಯೇ ಎದ್ದು ಪೂಜಾ ಮಂದಿರದಲ್ಲಿ ಜಪಧ್ಯಾನ ಮಾಡಬೇಕೆಂದು ಅಪ್ಪಣೆ ಹೊರಡಿಸಿದ್ದರು. ಸ್ವಾಮೀಜಿಯವರಿಗೆ ಆ ದಿನಗಳಲ್ಲಿ ಬಹು ಕೊಂಚ ನಿದ್ರೆ ಬರುತ್ತಿತ್ತು. ಬೆಳಗಿನ ಜಾವ ಮೂರು ಗಂಟೆಗೆಲ್ಲಾ ಏಳುತ್ತಿದ್ದರು.

ಮಠಕ್ಕೆ ಬಂದ ಕೂಡಲೇ ಶಿಷ್ಯನು ಸ್ವಾಮೀಜಿಗೆ ಪ್ರಣಾಮಮಾಡಿದ. ಸ್ವಾಮೀಜಿ ಹೇಳಿದರು: "ನೋಡು ಅವರೆಲ್ಲಾ ಈ ದಿನಗಳಲ್ಲಿ ಹೇಗೆ ಸಾಧನೆಯನ್ನು ಮಾಡುತ್ತಿದ್ದಾರೆ. ಪ್ರತಿಯೊಬ್ಬರೂ ಬೆಳಿಗ್ಗೆ ಮತ್ತು ಸಂಜೆ ಎರಡು ಹೊತ್ತೂ ಹೆಚ್ಚು ಕಾಲವನ್ನು ಜಪ ಮತ್ತು ಧ್ಯಾನದಲ್ಲಿ ಕಳೆಯುತ್ತಾರೆ. ಇಲ್ಲಿ ನೋಡು, ಇಲ್ಲೊಂದು ಜಾಗಟೆ ಇದೆ. ಎಲ್ಲರನ್ನೂ ನಿದ್ರೆಯಿಂದ ಎಬ್ಬಿಸಲು ಇಟ್ಟಿದ್ದೇವೆ. ಪ್ರತಿಯೊಬ್ಬರೂ ಸೂರ್ಯೋದಯಕ್ಕೆ ಮುಂಚೆ ಏಳಬೇಕು. ಶ‍್ರೀರಾಮಕೃಷ್ಣರು ಹೇಳುತ್ತಿದ್ದರು: ಬೆಳಿಗ್ಗೆ ಮತ್ತು ಸಂಜೆ ಮನಸ್ಸು ಸಾತ್ತ್ವಿಕ ಸ್ವಭಾವದಿಂದ ತುಂಬಿರುತ್ತದೆ. ಮನಸ್ಸಿಟ್ಟು ಧ್ಯಾನಮಾಡಬೇಕಾದರೆ ಇದೇ ಸರಿಯಾದ ಕಾಲ."

“ಶ‍್ರೀರಾಮಕೃಷ್ಣರ ನಿರ್ಯಾಣಾನಂತರ ನಾವು ಬಾರಾನಗರದ ಮಠದಲ್ಲಿ ಅನೇಕ ಕಠಿಣ ಸಾಧನೆಗಳನ್ನು ಮಾಡಿದೆವು. ಬೆಳಿಗ್ಗೆ ಮೂರು ಗಂಟೆಗೆ ಎದ್ದು ಮುಖ ತೊಳೆದು - ಕೆಲವರು ಸ್ನಾನಮಾಡಿ, ಕೆಲವರು ಹಾಗೆಯೇ ಪೂಜಾ ಗೃಹದಲ್ಲಿ ಕುಳಿತು ಜಪ ಮತ್ತು ಧ್ಯಾನದಲ್ಲಿ ಮಗ್ನರಾಗುತ್ತಿದ್ದೆವು. ಆ ದಿನಗಳಲ್ಲಿ ನಮಗೆ ಎಂಥಾ ತೀವ್ರವೈರಾಗ್ಯದ ಸ್ಫೂರ್ತಿ ಇತ್ತು? ಜಗತ್ತು ಇದೆಯೋ ಇಲ್ಲವೋ ಎಂಬುದರ ಯೋಚನೆಯೇ ನಮಗೆ ಇರಲಿಲ್ಲ. ರಾಮಕೃಷ್ಣಾನಂದರು ಮಾತ್ರ ಹಗಲು ರಾತ್ರಿ ಶ‍್ರೀರಾಮಕೃಷ್ಣರ ಪೂಜೆ ಸೇವೆಗಳ ಕೆಲಸವನ್ನು, ಮತ್ತು ಒಂದು ಸಂಸಾರದಲ್ಲಿ ಯಜಮಾನಿ ಯಾವ ಕೆಲಸ ಮಾಡುತ್ತಾಳೋ ಆ ಕೆಲಸವನ್ನೆಲ್ಲಾ ಮಠದಲ್ಲಿ ಮಾಡುತ್ತಿದ್ದರು. ಶ‍್ರೀರಾಮಕೃಷ್ಣರ ಪೂಜೆಗೆ ಮತ್ತು ನಮ್ಮ ನಿರ್ವಹಣೆಗೆ ಬೇಕಾದುದನ್ನೆಲ್ಲಾ ಭಿಕ್ಷೆ ಎತ್ತಿ ತರುತ್ತಿದ್ದರು. ಅನೇಕ ದಿನಗಳು ಬೆಳಗಿನ ಜಾವ ನಾಲ್ಕು ಗಂಟೆಯಿಂದ ಸಂಜೆ ಐದು ಗಂಟೆಯವರೆಗೂ ಧ್ಯಾನ ಜಪತಪಗಳಲ್ಲಿ ಕಳೆಯುತ್ತಿದ್ದೆವು. ರಾಮಕೃಷ್ಣಾನಂದರು ನಮ್ಮ ಆಹಾರವನ್ನು ಸಿದ್ದಪಡಿಸಿ ಕಾದಿದ್ದು ನಮ್ಮನ್ನು ಬಲವಂತದಿಂದ ಊಟಕ್ಕೆಬ್ಬಿಸುತ್ತಿದ್ದರು. ಓ! ಅವರಲ್ಲಿ ಎಂತಹ ನಿರರ್ಗಳ ಭಕ್ತಿ ವಿಶ್ವಾಸಗಳನ್ನು ನಾವು ನೋಡುತ್ತಿದ್ದೆವು."

ಶಿಷ್ಯ: ಆಗ ಮಠದ ಖರ್ಚನ್ನು ಹೇಗೆ ನಿರ್ವಹಿಸುತ್ತಿದ್ದೀರಿ?

ಸ್ವಾಮೀಜಿ: ಎಂತಹ ಪ್ರಶ್ನೆ! ನಾವು ಸಾಧುಗಳು, ಭಿಕ್ಷೆ ಎತ್ತಿ ಬಂದುದು ಮಠದ ಖರ್ಚಿಗಾಗಿ ಉಪಯೋಗಿಸಲ್ಪಡುತ್ತಿತ್ತು. ಇಂದು ಸುರೇಶ್ ಬಾಬು ಮತ್ತು ಬಲರಾಮಬಾಬು ಇವರಲ್ಲಿ ಯಾರೂ ಇಲ್ಲ - ಅವರು ಬದುಕಿದ್ದಿದ್ದರೆ ಈ ಮಠವನ್ನು ನೋಡಿ ತುಂಬಾ ಸಂತೋಷಪಡುತ್ತಿದ್ದರು. ನೀನು ಸುರೇಶಬಾಬುಗಳ ಹೆಸರನ್ನು ಖಂಡಿತವಾಗಿ ಕೇಳಿರುವೆಯಲ್ಲವೆ? ಒಂದು ಬಗೆಯಲ್ಲಿ ಅವನೇ ಈ ಮಠದ ಸ್ಥಾಪಕನೆಂದು ಹೇಳಬಹುದು. ಅವನೇ ಬಾರಾನಗರದ ಮಠದ ಖರ್ಚನ್ನೆಲ್ಲಾ ವಹಿಸಿಕೊಳ್ಳುತ್ತಿದ್ದನು. ಆ ದಿನಗಳಲ್ಲಿ ನಮಗಾಗಿ ಯೋಚಿಸುತ್ತಿದ್ದವನು ಸುರೇಶಮಿತ್ರ. ಅವನ ಭಕ್ತಿ ಮತ್ತು ಶ್ರದ್ಧೆಗೆ ಸರಿಸಮಾನರಾರೂ ಇಲ್ಲ.

ಶಿಷ್ಯ: ಸ್ವಾಮೀಜಿ, ಅವರು ಮೃತ್ಯುಶಯ್ಯೆಯಲ್ಲಿದ್ದಾಗ ನೀವು ಅವರಲ್ಲಿಗೆ ಹೆಚ್ಚಾಗಿ ಹೋಗುತ್ತಿರಲಿಲ್ಲವೆಂದು ಕೇಳಿದೆ.

ಸ್ವಾಮಿಜಿ: ಅವರ ನೆಂಟರು ಬಿಟ್ಟಿದ್ದರೆ ಖಂಡಿತವಾಗಿಯೂ ಹೋಗುತ್ತಿದ್ದೆವು. ಅದೊಂದು ದೊಡ್ಡ ಕಥೆ. ಆದರೆ ಇದನ್ನು ಚೆನ್ನಾಗಿ ನೆನಪಿನಲ್ಲಿಡು, ಪ್ರಾಪಂಚಿಕ ಜನರಿಗೆ ನೀನು ಬದುಕುವುದು ಸಾಯುವುದು ಇದೆಲ್ಲಾ ಅಷ್ಟು ಗಣನೀಯವಲ್ಲ. ನೀನು ಸ್ವಲ್ಪ ಆಸ್ತಿಯನ್ನು ಬಿಟ್ಟುಹೋಗುವಂತಿದ್ದರೆ, ನಿನ್ನ ಜೀವಿತಕಾಲದಲ್ಲೇ ನಿನ್ನ ಗೃಹದಲ್ಲಿ ಅದಕ್ಕಾಗಿ ಕಾದಾಟ ಶುರುವಾಗುವುದು. ನಿನ್ನ ಮರಣಶಯ್ಯೆಯಲ್ಲಿ ನಿನ್ನ ಹೆಂಡತಿ ಮಕ್ಕಳು ಕೂಡ ನಿನ್ನನ್ನು ಸಮಾಧಾನಪಡಿಸುವುದಿಲ್ಲ. ಇದೇ ಪ್ರಪಂಚದ ಸ್ವಭಾವ.

ಮಠದ ಹಿಂದಿನ ಸ್ಥಿತಿಯ ವಿಚಾರ ಮಾತಾಡುತ್ತಾ ಸ್ವಾಮೀಜಿ ಹೇಳಿದರು: “ಹಣವಿಲ್ಲದೆ ಅನೇಕ ವೇಳೆ ನಾನು ಮಠವನ್ನೇ ಮುಚ್ಚಿಬಿಡಬೇಕೆಂದು ಹೊಡೆದಾಡುತ್ತಿದ್ದೆ. ಆದರೆ ಏನು ಮಾಡಿದರೂ ರಾಮಕೃಷ್ಣಾನಂದರು ಈ ಅಭಿಪ್ರಾಯಕ್ಕೆ ಒಪ್ಪುತ್ತಿರಲಿಲ್ಲ. ರಾಮಕೃಷ್ಣಾನಂದರೇ ಮಠದ ಜೀವಾಳವಾಗಿದ್ದರು. ಅನೇಕ ದಿನಗಳು ಮಠದಲ್ಲಿ ಒಂದು ಕಾಳು ಆಹಾರವೂ ಇರುತ್ತಿರಲಿಲ್ಲ. ಭಿಕ್ಷೆಯಿಂದ ಕೊಂಚ ಅಕ್ಕಿ ಸಿಕ್ಕಿದರೆ ಅದರ ಜೊತೆಗೆ ತಿನ್ನಲು ಉಪ್ಪು ಕೂಡಾ ಇರುತ್ತಿರಲಿಲ್ಲ. ಕೆಲವು ದಿನಗಳು ಕೇವಲ ಉಪ್ಪು ಅನ್ನ ಇಷ್ಟೇ ಇರುತ್ತಿತ್ತು. ಆದರೆ ಯಾರಿಗೂ ತಾವೇನು ತಿನ್ನುತ್ತಿರುವೆವೆಂಬುದರ ಕಡೆ ಗಮನವೇ ಇರುತ್ತಿರಲಿಲ್ಲ. ನಾವೆಲ್ಲಾ ಆಗ ದೊಡ್ಡದೊಂದು ಆಧ್ಯಾತ್ಮಿಕ ಪ್ರವಾಹದಲ್ಲಿ ಕೊಚ್ಚಿಕೊಂಡು ಹೋಗುತ್ತಿದ್ದೆವು. ಬೇಯಿಸಿದ ಬಿಂಬಾಪತ್ರ, ಅನ್ನ, ಉಪ್ಪು ಇಷ್ಟೇ ಒಂದು ತಿಂಗಳವರೆಗೆ ನಮ್ಮ ಆಹಾರವಾಗಿತ್ತು. ಓ! ಅದೆಂತಹ ಅದ್ಭುತ ದಿನಗಳು! ಆ ಕಾಲದ ಸಾಧನೆಗಳು, ಮನುಷ್ಯರ ಮಾತೇಕೆ, ಭೂತಗಳಿಗೂ ಗಾಬರಿ ಹುಟ್ಟಿಸುವಂತಿದ್ದುವು. ನಿನ್ನಲ್ಲಿ ನಿಜವಾದ ಅರ್ಹತೆ ಇದ್ದರೆ ನೀನಿರುವ ವಾತಾವರಣ ನಿನಗೆ ವಿರುದ್ಧವಾಗಿದ್ದಷ್ಟೂ ನಿನ್ನ ಆಂತರಿಕ ಶಕ್ತಿ ಹೆಚ್ಚು ವಿಕಾಸಗೊಳ್ಳುವುದೆಂಬುದು ಅದ್ಭುತ ಸತ್ಯ. ನಾನೇಕೆ ಮಠದಲ್ಲಿರುವವರಿಗೆ ತಕ್ಕಮಟ್ಟಿಗೆ ಒಳ್ಳೆಯ ಜೀವನಕ್ಕೆ ಆಗುವಷ್ಟು ಹಾಸಿಗೆ ಮುಂತಾದುವನ್ನು ಒದಗಿಸಿದ್ದೇನೆಂದರೆ ಈಗಿನ ಕಾಲದಲ್ಲಿ ಸಂನ್ಯಾಸಿಗಳಾಗಲು ಬರುವವರು ನಮ್ಮಷ್ಟು ಕಷ್ಟ ಸಹಿಷ್ಣುಗಳಲ್ಲ. ಶ‍್ರೀರಾಮಕೃಷ್ಣರ ಜೀವನ ನಮ್ಮ ಮುಂದಿತ್ತು. ಅದಕ್ಕೇ ನಾವು ದಾರಿ ಕಷ್ಟಗಳಿಗೆ ಕೊಂಚವೂ ಗಮನ ಕೊಡಲಿಲ್ಲ. ಈಗಿನ ಕಾಲದ ಹುಡುಗರು ಅಷ್ಟೊಂದು ಕಷ್ಟವನ್ನು ಸಹಿಸಲಾರರು. ಅದಕ್ಕೆ ನಾನು ಅವರಿಗೆ ವಸತಿ ಮತ್ತು ಜೀವನಾಧಾರಕ್ಕೆ ತಕ್ಕಷ್ಟು ಒದಗಿಸಿಕೊಟ್ಟಿದ್ದೇನೆ. ಅವರಿಗೆ ತಕ್ಕಷ್ಟು ಆಹಾರ ಬಟ್ಟೆ ಸಿಕ್ಕಿದರೆ ಹುಡುಗರು ಧಾರ್ಮಿಕ ಸಾಧನೆಗೆ ಮನಗೊಟ್ಟು ವಿಶ್ವ ಜನತೆಯ ಸೇವೆಗೆ ತಮ್ಮ ಜೀವನವನ್ನು ಅರ್ಪಿಸಲು ಕಲಿಯುವರು.”

ಶಿಷ್ಯ: ಸ್ವಾಮೀಜಿ, ಹೊರಗಿನ ಜನರು ಈ ಬಗೆಯ ಹಾಸಿಗೆ ಮತ್ತು ಕುರ್ಚಿ ಮೇಜುಗಳ ವಿಚಾರವಾಗಿ ಬಹುವಾಗಿ ಟೀಕೆ ಮಾಡುವರು.

ಸ್ವಾಮೀಜಿ: ಮಾಡಿಕೊಳ್ಳಲಿ, ಹಾಸ್ಯಕ್ಕಾದರೂ ಅವರೊಮ್ಮೆ ಈ ಮಠದ ವಿಚಾರ ಯೋಚಿಸುವರು. ದ್ವೇಷಬುದ್ಧಿಯನ್ನು ಬೆಳೆಸುವುದರ ಮೂಲಕ ಅವರು ಮುಕ್ತಿ ಪಡೆಯಬಹುದೆಂದು ಯೋಚಿಸುವರು. ಶ‍್ರೀರಾಮಕೃಷ್ಣರು ‘ಮನುಷ್ಯರನ್ನು ಹುಳುಗಳೆಂದೆಣಿಸಬೇಕು’ ಎನ್ನುತ್ತಿದ್ದರು. ಇತರರು ಏನು ಹೇಳುವರೋ ಎಂದು ಅದಕ್ಕೆ ತಕ್ಕಂತೆ ನಾವು ನಡೆದುಕೊಳ್ಳಬೇಕೆಂದು ಹೇಳುವೆಯೇನು? ಧಿಕ್ಕಾರ!

ಶಿಷ್ಯ: ಸ್ವಾಮಿಜಿ, ನೀವು ಕೆಲವು ವೇಳೆ ಎಲ್ಲರೂ ನಾರಾಯಣರು, ದೀನರು ಆವಶ್ಯಕತೆಯುಳ್ಳವರು, ನನ್ನ ನಾರಾಯಣರೆಂದು ಹೇಳುವಿರಿ, ಪುನಃ ನೀವೇ ಮನುಷ್ಯರನ್ನು ಹುಳುಗಳೆಂದೆಣಿಸಬೇಕು ಎನ್ನುವಿರಿ. ನಿಜವಾಗಿಯೂ ನೀವು ಹೇಳುವುದೇನು?

ಸ್ವಾಮೀಜಿ: ಎಲ್ಲರೂ ನಾರಾಯಣರೆಂಬುದರಲ್ಲಿ ಕೊಂಚವೂ ಸಂದೇಹವಿಲ್ಲ. ಆದರೆ ಎಲ್ಲಾ ನಾರಾಯಣರೂ ಮಠದ ಸಾಮಾನುಗಳ ವಿಚಾರ ಚರ್ಚಿಸುವುದಿಲ್ಲ. ಯಾರು ಎಷ್ಟೇ ಟೀಕಿಸಲಿ ಅದೊಂದಕ್ಕೂ ನಾನು ಕಿವಿಕೊಡದೆ ಇತರರ ಕಲ್ಯಾಣಕ್ಕಾಗಿ ದುಡಿಯುತ್ತಾ ಹೋಗುವೆನೆಂಬ ಅರ್ಥದಲ್ಲಿ ‘ಮನುಷ್ಯರನ್ನು ಹುಳುಗಳಂತೆ ನೋಡಬೇಕು’ ಎಂದು ನಾನು ಹೇಳಿದೆ. ಯಾರಿಗೆ ಇಂತಹ ಪ್ರಚಂಡ ನಿರ್ಧಾರವಿದೆಯೊ ಅವರು ಎಲ್ಲವನ್ನೂ ಪಡೆಯುವರು. ಕೆಲವರು ಬೇಗ ಅದನ್ನು ಪಡೆಯಬಹುದು, ಇತರರು ಕೊಂಚ ನಿಧಾನವಾಗಿ ಪಡೆಯಬಹುದು ಅಷ್ಟೆ. ಆದರೆ ಖಂಡಿತವಾಗಿ ಎಲ್ಲರೂ ಗುರಿಯನ್ನು ಸೇರಿಯೇ ತೀರುವರು. ನಮಗೆ ಸ್ಥಿರವಾದ ನಿರ್ಧಾರವಿದ್ದುದರಿಂದ ನಾವೀಗ ಸಾಧಿಸಿರುವಷ್ಟು ಅಲ್ಪವನ್ನಾದರೂ ಸಾಧಿಸಲು ಸಾಧ್ಯವಾಯಿತು. ಇಲ್ಲದಿದ್ದಲ್ಲಿ ಎಷ್ಟೊಂದು ಘೋರವಾದ ದಾರಿದ್ರ್ಯದಲ್ಲಿ ನಾವು ನರಳಬೇಕಾಗಿತ್ತು! ಒಂದು ದಿನ ಆಹಾರದ ಅಭಾವದಿಂದ ನಾನು ಜ್ಞಾನತಪ್ಪಿ ರಸ್ತೆಯ ಪಕ್ಕದಲ್ಲಿದ್ದ ಒಂದು ಮನೆಯ ಅಂಗಳದಲ್ಲಿ ಬಿದ್ದು ಬಿಟ್ಟೆ. ನನಗೆ ಪ್ರಜ್ಞೆ ಬರುವ ಹೊತ್ತಿಗೆ ಮಳೆ ನನ್ನ ಮೇಲೆ ಸುರಿದಿತ್ತು. ಮತ್ತೊಂದು ದಿನ ನಾನು ಕಲ್ಕತ್ತೆಯಲ್ಲಿ ಆಹಾರವಿಲ್ಲದೆ ಒಂದು ಇಡೀ ದಿನ ಏನೇನೋ ಚಿಲ್ಲರೆ ಕೆಲಸಗಳನ್ನು ಮಾಡಬೇಕಾಯಿತು. ರಾತ್ರಿ ಹತ್ತು ಅಥವಾ ಹನ್ನೊಂದು ಘಂಟೆಗೆ ಮಠಕ್ಕೆ ಬಂದಮೇಲೆ ಊಟ ಮಾಡಿದೆ. ಅದೊಂದೇ ಅಲ್ಲ, ಇಂತಹ ಹಲವಾರು ನಿದರ್ಶನಗಳಿವೆ.

ಈ ಮಾತುಗಳನ್ನು ಹೇಳುತ್ತಿದ್ದಂತೆಯೇ ಸ್ವಾಮೀಜಿ ಸ್ವಲ್ಪ ಹೊತ್ತು ಯಾವುದೋ ಯೋಚನೆಯಲ್ಲಿ ಮಗ್ನರಾಗಿದ್ದರು. ಅನಂತರ ಮುಂದುವರಿಸಿದರು: ನಿಜವಾದ ಸಂನ್ಯಾಸವನ್ನು ಪಡೆಯುವುದು ಸುಲಭವಲ್ಲ. ಇಂತಹ ಜೀವನದಷ್ಟು ಕಷ್ಟತಮವಾದುದು ಯಾವುದೂ ಇಲ್ಲ. ನೀನು ಕೊಂಚ ಎಡವಿದರೆ ಸಾಕು. ದೊಡ್ಡ ಕಂದಕಕ್ಕೆ ಜಾರಿ ಬೀಳುವೆ - ಚೂರುಚೂರಾಗುವೆ. ಒಮ್ಮೆ ನಾನು ಆಗ್ರಾದಿಂದ ಪೂನಾಕ್ಕೆ ಕಾಲ್ನಡಿಗೆಯಲ್ಲಿ ಹೋಗುತ್ತಿದೆ. ನನ್ನಲ್ಲಿ ಒಂದು ಬಿಡಿಗಾಸೂ ಇರಲಿಲ್ಲ. ಬೃಂದಾವನಕ್ಕೆ ಇನ್ನೂ ಒಂದೆರಡು ಮೈಲಿ ದೂರವಿತ್ತು. ಹಾದಿಯ ಪಕ್ಕದಲ್ಲಿ ಒಬ್ಬ ಮನುಷ್ಯ ಗುಡಿಗುಡಿ ಸೇದುತ್ತಾ ಕುಳಿತಿದ್ದ. ಆ ಕ್ಷಣ ನನಗೂ ಸೇದಬೇಕೆಂಬ ಆಸೆಯಾಯಿತು. ನಾನು ಆ ಮನುಷ್ಯನಿಗೆ ‘ನಿನ್ನ ಚಿಲುಮೆಯಲ್ಲಿ ಒಮ್ಮೆ ಸೇದಲು ನನಗೆ ಅವಕಾಶ ಕೊಡುವೆಯಾ?’ ಎಂದೆ. ಆ ಮನುಷ್ಯ ನನಗೆ ಅದನ್ನು ಕೊಡಲು ತುಂಬಾ ಹಿಂಜರಿಯುತ್ತಾ ‘ನಾನು ಜಲಗಾರ’ ಎಂದ. ಇನ್ನೂ ನನ್ನಲ್ಲಿ ಹಿಂದಣ ಸಂಸ್ಕಾರದ ಪ್ರಭಾವವಿತ್ತು - ನಾನು ತಕ್ಷಣ ಹಿಂದೆ ಸರಿದು ಸೇದದೆ ಮುಂದಕ್ಕೆ ಹೊರಟೆ. ಕೊಂಚ ದೂರ ಹೋಗುವಷ್ಟರಲ್ಲಿಯೇ ನನಗೆ ನೆನಪಿಗೆ ಬಂತು, ನಾನೊಬ್ಬ ಸಂನ್ಯಾಸಿ, ಜಾತಿ, ಮತ, ಸಂಸಾರ, ಗೌರವ ಎಲ್ಲವನ್ನೂ ತ್ಯಜಿಸಿದವನು - ಆದರೂ ಆ ಮನುಷ್ಯ ಕಸ ಗುಡಿಸುವ ಜಾಡಮಾಲಿ ಎಂದ ಕೂಡಲೇ ಹಿಂದಕ್ಕೆ ಸರಿದೆ. ಅವನು ಮುಟ್ಟಿದ ಗುಡಿಗುಡಿಯನ್ನು ಸೇದದೆ ಹೊರಟುಬಂದೆ ಎಂಬುದು ನೆನಪಿಗೆ ಬಂದಿತು. ಈ ಯೋಚನೆ ನನ್ನ ಮನಸ್ಸನ್ನು ಕ್ಷುಬ್ಧಗೊಳಿಸಿತು. ನಾನಾಗಲೇ ಅರ್ಧ ಮೈಲಿ ದೂರ ಬಂದಿದ್ದೆ. ಪುನಃ ನಾನು, ಹಿಂತಿರುಗಿ ಆ ಜಲಗಾರನಿದ್ದಲ್ಲಿಗೆ ಬಂದೆ. ಅವನಿನ್ನೂ ಅಲ್ಲಿಯೇ ಕುಳಿತಿದ್ದ. ನಾನು ತಕ್ಷಣ ಅವನಿಗೆ ‘ಭಾಯಿ, ದಯವಿಟ್ಟು ನನಗೊಂದು ಗುಡಿಗುಡಿಯನ್ನು ಸಿದ್ಧಪಡಿಸು’ ಎಂದೆ. ಅವನು ಏನೇನು ಅಡ್ಡಿಗಳನ್ನು ಹೇಳಿದರೂ ಕೇಳದೆ ಅದನ್ನು ಕೊಟ್ಟೇ ತೀರಬೇಕೆಂದು ಹಠ ಹಿಡಿದೆ. ಆತ ನನಗೆ ಚಿಲುಮೆಯನ್ನು ಸಿದ್ಧಪಡಿಸಲೇಬೇಕಾಗಿ ಬಂತು. ಅನಂತರ ನಾನು ಸಂತೋಷದಿಂದ ಆ ಚಿಲುಮೆಯನ್ನು ಸೇದಿ ಬೃಂದಾವನಕ್ಕೆ ಹೊರಟೆ. ಯಾರು ಸಂನ್ಯಾಸಾಶ್ರಮ ಸ್ವೀಕರಿಸಿದ್ದಾರೋ ಅವರು ಈ ಜಾತಿ ಮತ ಗೌರವ ಮುಂತಾದುವನ್ನೆಲ್ಲಾ ಮೀರಿಹೋಗಿದ್ದೇವೋ ಇಲ್ಲವೋ ಎಂದು ಚೆನ್ನಾಗಿ ಪರೀಕ್ಷಿಸಿಕೊಳ್ಳಬೇಕು. ನಿಜವಾದ ಶ್ರದ್ಧೆಯಿಂದ ಸಂನ್ಯಾಸಾಶ್ರಮವನ್ನು ಅನುಸರಿಸುವುದು ಬಹು ಕಷ್ಟ. ನಮ್ಮ ನಡೆ ನುಡಿ ಎರಡಕ್ಕೂ ಕೊಂಚ ವ್ಯತ್ಯಾಸವೂ ಇರಕೂಡದು.

ಶಿಷ್ಯ: ನೀವು ಒಮ್ಮೆ ಸಂನ್ಯಾಸಾಶ್ರಮದ ಧ್ಯೇಯ, ಮತ್ತೊಮ್ಮೆ ಗೃಹಸ್ಥಾಶ್ರಮದ ಧ್ಯೇಯವನ್ನು ಎತ್ತಿಹಿಡಿಯುವಿರಿ. ನಾನಾವುದನ್ನು ಅನುಸರಿಸಬೇಕು?

ಸ್ವಾಮೀಜಿ: ಎಲ್ಲವನ್ನೂ ಕೇಳು. ಕೊನೆಗೆ ನಿನಗೆ ಯಾವುದು ಮನಸ್ಸಿಗೆ ಹೆಚ್ಚಾಗಿ ಹಿಡಿಸುವುದೋ ಅದನ್ನು ಅನುಸರಿಸು. ಪಟ್ಟು ಹಿಡಿದು ಅದನ್ನೇ ಅನುಸರಿಸು.

ಹೀಗೆ ಮಾತನಾಡುತ್ತಾ ಸ್ವಾಮಿಜಿ ಕೆಳಗಿಳಿದು ಬಂದರು. ಒಮ್ಮೊಮ್ಮೆ ‘ಶಿವ’ ಅಥವಾ ‘ಓ! ತಾಯಿ, ಅಮೃತತರಂಗಿಣಿ ನೀನು’ ಎನ್ನುವ ಜಗನ್ಮಾತೆಯ ಮೇಲಿನ ಹಾಡನ್ನು ಮೆಲ್ಲಗೆ ಹೇಳುತ್ತ ಸುತ್ತುತ್ತಿದ್ದರು.

\newpage

\chapter[ಅಧ್ಯಾಯ ೪೩]{ಅಧ್ಯಾಯ ೪೩\protect\footnote{\engfoot{C.W, Vol. VII, P. 252}}}

\begin{center}
ಸ್ಥಳ: ಬೇಲೂರು ಮಠ, ವರ್ಷ: ಕ್ರಿ.ಶ. ೧೯೦೨.
\end{center}

ಶಿಷ್ಯನು ಸ್ವಾಮೀಜಿಯ ಕೊಠಡಿಯಲ್ಲೇ ಹಿಂದಿನ ರಾತ್ರಿಯನ್ನು ಕಳೆದಿದ್ದ. ಬೆಳಗಿನ ಜಾವ ನಾಲ್ಕು ಘಂಟೆಗೆ ಸ್ವಾಮೀಜಿ ಶಿಷ್ಯನನ್ನೆಬ್ಬಿಸಿ, ‘ಗಂಟೆಯನ್ನು ಹೊಡೆದು ಬ್ರಹ್ಮಚಾರಿಗಳನ್ನೂ, ಸಾಧುಗಳನ್ನೂ ನಿದ್ರೆಯಿಂದ ಎಬ್ಬಿಸು’ ಎಂದರು. ಅಪ್ಪಣೆಯ ಮೇರೆಗೆ ಶಿಷ್ಯನು ನಿದ್ರಿಸುತ್ತಿದ್ದ ಸಾಧುಗಳ ಹತ್ತಿರ ಹೋಗಿ ಗಂಟೆ ಬಾರಿಸಿದ. ಆಶ್ರಮದ ನಿವಾಸಿಗಳೆಲ್ಲ ಪೂಜಾಮಂದಿರಕ್ಕೆ ತ್ವರೆಯಿಂದ ಧ್ಯಾನ ಮಾಡಲು ಹೊರಟರು.

ಸ್ವಾಮೀಜಿಯ ಅಪ್ಪಣೆಯಂತೆ ಶಿಷ್ಯನು ಸ್ವಾಮಿ ಬ್ರಹ್ಮಾನಂದರ ಹಾಸಿಗೆಯ ಹತ್ತಿರ ಜೋರಾಗಿ ಗಂಟೆ ಬಾರಿಸಿದ. ಅವರು ಗಟ್ಟಿಯಾಗಿ ಒದರಿದರು: “ಅಯ್ಯೋ ರಾಮ! ಈ ಬಂಗಾಳನ ದೆಸೆಯಿಂದ ನಮಗೆ ಈ ಮಠದಲ್ಲಿ ಉಳಿಗಾಲವಿಲ್ಲ” ಎಂದು. ಶಿಷ್ಯನಿಂದ ಇದನ್ನು ಕೇಳಿದ ಸ್ವಾಮಿಜಿ ಹೊಟ್ಟೆ ಹುಣ್ಣಾಗುವಂತೆ ನಕ್ಕರು.

ಸಂನ್ಯಾಸಿಗಳೆಲ್ಲಾ, ಸ್ವಾಮಿ ಬ್ರಹ್ಮಾನಂದರೂ ಕೂಡ, ಧ್ಯಾನಾರೂಢರಾಗಿದ್ದರು. ಸ್ವಾಮೀಜಿಗಾಗಿ ಒಂದು ಪ್ರತ್ಯೇಕ ಆಸನ ಏರ್ಪಡಿಸಲಾಗಿತ್ತು. ಸ್ವಾಮೀಜಿ ಪೂರ್ವಾಭಿಮುಖವಾಗಿ ಕುಳಿತುಕೊಂಡು ತಮ್ಮ ಮುಂದಿದ್ದ ಆಸನವನ್ನು ಶಿಷ್ಯನಿಗೆ ತೋರಿಸಿ “ಹೋಗು ಅಲ್ಲಿಯೇ ಕುಳಿತು ಧ್ಯಾನ ಮಾಡು" ಎಂದರು.

ಆಸನಾರೂಢರಾದ ಕೊಂಚ ಹೊತ್ತಿನಲ್ಲಿಯೇ ಸ್ವಾಮೀಜಿ ಸಂಪೂರ್ಣ ಶಾಂತ ಚಿತ್ತರಾಗಿ ಪ್ರತಿಮೆಯಂತೆ ಸ್ಥಿರವಾಗಿದ್ದರು - ಅವರ ಉಸಿರಾಡುವಿಕೆ ಬಹು ನಿಧಾನವಾಗುತ್ತಾ ಬಂದಿತು. ಎಲ್ಲರೂ ತಮ್ಮ ತಮ್ಮ ಸ್ಥಳಗಳಲ್ಲಿಯೇ ಕುಳಿತಿದ್ದರು.

ಸುಮಾರು ಒಂದೂವರೆ ಘಂಟೆಯಾದ ಮೇಲೆ ಸ್ವಾಮೀಜಿ “ಶಿವ, ಶಿವ" ಎಂದು ಹೇಳುತ್ತಾ ಧ್ಯಾನದಿಂದ ಮೇಲೆದ್ದರು. ಅವರ ಕಣ್ಣುಗಳು ಪ್ರಜ್ವಲಿಸುತ್ತಿದ್ದುವು. ಮುಖಭಾವ ಪ್ರಶಾಂತವಾಗಿ ಸೌಮ್ಯವಾಗಿ ಗಂಭೀರವಾಗಿದ್ದಿತು. ಶ‍್ರೀರಾಮಕೃಷ್ಣರಿಗೆ ಪ್ರಣಾಮ ಮಾಡಿ, ಕೆಳಕ್ಕೆ ಇಳಿದು ಬಂದು ಮಠದ ವರಾಂಡದಲ್ಲಿ ಶತಪಥ ತಿರುಗುತ್ತಿದ್ದರು. ಕೊಂಚ ಹೊತ್ತಿನ ತರುವಾಯ ಶಿಷ್ಯನಿಗೆ ಹೇಳಿದರು, “ನೋಡಿದೆಯಾ, ಈಗ ಮಠದ ಸಾಧುಗಳು ಧ್ಯಾನ ಮುಂತಾದುವನ್ನು ಹೇಗೆ ಮಾಡುತ್ತಿರುವರೆಂದು. ಧ್ಯಾನ ಗಾಢವಾಗುತ್ತಾ ಹೋದ ಹಾಗೆಲ್ಲಾ ವ್ಯಕ್ತಿಯು ಅನೇಕ ಅದ್ಭುತವಾದ ವಸ್ತುಗಳನ್ನು ಕಾಣುತ್ತಾನೆ. ಬಾರಾನಗರದ ಮಠದಲ್ಲಿ ಧ್ಯಾನಿಸುತ್ತಿದ್ದಾಗ ನಾನು ಇಡಾ ಮತ್ತು ಪಿಂಗಳಾ ಎಂಬ ನಾಡಿಗಳನ್ನು ನೋಡಿದೆ. ಕೊಂಚ ಪರಿಶ್ರಮದಿಂದಲೇ ನಾವದನ್ನು ನೋಡಬಹುದು. ಅನಂತರ ಸುಷುಮ್ನಾವನ್ನು ಯಾರು ನೋಡುವರೋ ಅವರು ತಮಗಿಷ್ಟಬಂದುದನ್ನು ನೋಡಬಹುದು. ಯಾರಿಗೆ ಗುರುವಿನಲ್ಲಿ ಅವಿಚ್ಛಿನ್ನವಾದ ಭಕ್ತಿ ಇದೆಯೋ ಅವನಿಗೆ ಜಪಧ್ಯಾನ ಮುಂತಾದುವು ತಾವಾಗಿಯೇ ಬರುವುವು. ಅದಕ್ಕಾಗಿ ಅವನು ಹೋರಾಡಬೇಕಾಗಿಲ್ಲ. ‘ಗುರುವೇ ಬ್ರಹ್ಮ, ಗುರುವೇ ವಿಷ್ಣು, ಗುರುವೇ ಮಹೇಶ್ವರ’."

ಅನಂತರ ಶಿಷ್ಯನು ಸ್ವಾಮಿಗಳಿಗೆ ತಂಬಾಕನ್ನು ಸಿದ್ಧಪಡಿಸಿದಾಗ ಸ್ವಾಮಿಗಳು ಅದನ್ನು ಸೇದುತ್ತಾ ಹೇಳಿದರು: “ಆಂತರ್ಯದಲ್ಲಿರುವುದು ಸಿಂಹ - ನಿರಂತರ ಪವಿತ್ರ ಜ್ಯೋತಿರ್ಮಯ ಮುಕ್ತಾತ್ಮ - ಅವನನ್ನು ನೇರವಾಗಿ ಧ್ಯಾನ ಮತ್ತು ಚಿತ್ತೈಕಾಗ್ರತೆಯಿಂದ ಯಾರು ಸಾಕ್ಷಾತ್ಕರಿಸಿಕೊಳ್ಳುವರೋ ಅವರಿಗೆ ಈ ಮಾಯಾಪ್ರಪಂಚ ಮಾಯವಾಗುವುದು. ಅವನು ಎಲ್ಲದರಲ್ಲಿಯೂ ಮೂರ್ತೀಭವಿಸಿದ್ದಾನೆ. ಹೆಚ್ಚು ಸಾಧನೆ ಮಾಡಿದಷ್ಟೂ ಬೇಗ ಕುಂಡಲಿನಿ ಜಾಗೃತಗೊಳ್ಳುವುದು. ಯಾವಾಗ ಈ ಶಕ್ತಿಯು ಶಿರವನ್ನು ಸೇರುವುದೋ ಆಗ ಅವನ ದೃಷ್ಟಿಗೆ ಯಾವುದೂ ಅಡ್ಡವಾಗುವುದಿಲ್ಲ - ಅವನಿಗೆ ಆತ್ಮಸಾಕ್ಷಾತ್ಕಾರವಾಗುವುದು.”

ಶಿಷ್ಯ: ಸ್ವಾಮೀಜಿ, ನಾನು ಇವುಗಳನ್ನೆಲ್ಲಾ ಧರ್ಮಗ್ರಂಥಗಳಲ್ಲಿ ಓದಿದ್ದೇನೆ ಅಷ್ಟೆ. ಯಾವುದೂ ಸಾಕ್ಷಾತ್ಕಾರವಾಗಿಲ್ಲ.

ಸ್ವಾಮೀಜಿ: ‘ಕಾಲೇನಾತ್ಮನಿ ವಿಂದತಿ’ ಯೋಗ್ಯಕಾಲದಲ್ಲಿ ಅದು ಬಂದೇ ಬರುವುದು. ಕೆಲವರು ಬೇಗ ಸಾಧಿಸುವರು. ಕೆಲವರು ಹೆಚ್ಚು ಕಾಲ ತೆಗೆದುಕೊಳ್ಳುವರು. ಎಂದಿಗೂ ಬಿಡುವುದಿಲ್ಲೆಂಬ ನಿರ್ಧಾರದಿಂದ ಅದನ್ನೇ ಗಟ್ಟಿಯಾಗಿ ಹಿಡಿದಿರಬೇಕು. ಇದೇ ನಿಜವಾದ ಪುರುಷಕಾರ. ತಡೆಯಿಲ್ಲದ ಎಣ್ಣೆಯ ಪ್ರವಾಹದಂತೆ ಮನಸ್ಸನ್ನು ಒಂದೇ ವಸ್ತುವಿನಲ್ಲಿ ಏಕಾಗ್ರ ಮಾಡಬೇಕು. ಸಾಧಾರಣ ಮನುಷ್ಯನ ಮನಸ್ಸು ನಾನಾ ವಸ್ತುಗಳ ಮೇಲೆ ಹರಿದು ಹಂಚಿಹೋಗಿರುತ್ತದೆ. ಧ್ಯಾನ ಕಾಲದ ಪ್ರಾರಂಭದಲ್ಲಿ ಮನಸ್ಸು ಚಂಚಲವಾಗಿಯೇ ಇರುವುದು. ಮನಸ್ಸಿನಲ್ಲಿ ಯಾವ ಆಸೆ ಬೇಕಾದರೂ ಬರಲಿ, ಶಾಂತನಾಗಿ ಕುಳಿತು ಯಾವ ಭಾವನೆಗಳು ಬರುತ್ತವೆಂಬುದನ್ನು ನೋಡು. ಹೀಗೆ ಪರೀಕ್ಷಿಸುತ್ತಾ ಹೋದ ಹಾಗೆ ಮನಸ್ಸು ಶಾಂತವಾಗಿ ಇತರ ಯೋಚನಾತರಂಗಗಳು ಬರುವುದಿಲ್ಲ. ಈ ತರಂಗಗಳು ಮನಸ್ಸಿನ ಯೋಚನಾಶಕ್ತಿಯ ಪ್ರತಿನಿಧಿಗಳು. ನೀನು ಹಿಂದೆ ಯಾವ ಯಾವ ಗಾಢ ಆಲೋಚನೆಗಳಲ್ಲಿ ಮುಳುಗಿದ್ದೆಯೋ ಅವೆಲ್ಲಾ ನಿನ್ನ ಸುಪ್ತಾವಸ್ಥೆಯ ತರಂಗಗಳಾಗಿ ಮಾರ್ಪಟ್ಟಿರುತ್ತವೆ - ಆದ್ದರಿಂದಲೇ ಧ್ಯಾನಾಸಕ್ತ ಮನಸ್ಸಿನಲ್ಲಿ ಬರುತ್ತವೆ. ಧ್ಯಾನದಲ್ಲಿರುವಾಗ ಮನಸ್ಸಿನಲ್ಲಿ ಈ ರೀತಿ ಅಲೆಗಳು ಅಥವಾ ಆಲೋಚನೆಗಳು ಏಳುವುದು ನಿನ್ನ ಮನಸ್ಸು ಏಕಾಗ್ರಗೊಳ್ಳುತ್ತಿದೆ ಎಂಬುದರ ಸೂಚಕ. ಕೆಲವೊಮ್ಮೆ ಮನಸ್ಸು ಒಂದೇ ರೀತಿಯ ಭಾವನೆಗಳ ಮೇಲೆ ಏಕಾಗ್ರಗೊಂಡಿರುತ್ತದೆ. ಇದಕ್ಕೆ ವಿಕಲ್ಪ ಧ್ಯಾನ ಎಂದು ಹೆಸರು. ಆದರೆ ಯಾವಾಗ ಮನಸ್ಸು ಎಲ್ಲಾ ಬಗೆಯ ಕಾರ್ಯಗಳಿಂದಲೂ ಬಿಡುಗಡೆ ಹೊಂದುವುದೋ ಆಗ ಅದು ಆಂತರಿಕ ಶಕ್ತಿಯಲ್ಲಿ ಲೀನವಾಗುವುದು. ಇದೇ ಅಖಂಡ, ಅನಂತಜ್ಞಾನ, ತನ್ನ ನೆಲೆ ತಾನೇ ಆಗಿರುವುದು. ಇದೇ ನಿರ್ವಿಕಲ್ಪ ಸಮಾಧಿ - ಎಲ್ಲಾ ಬಗೆಯ ಕರ್ಮಗಳಿಂದಲೂ ಮುಕ್ತವಾಗಿರುವುದು. ಶ‍್ರೀರಾಮಕೃಷ್ಣರಲ್ಲಿ ನಾವು ಅನೇಕ ಬಾರಿ ಈ ಎರಡೂ ಬಗೆಯ ಸಮಾಧಿಗಳನ್ನೂ ನೋಡುತ್ತಿದ್ದೆವು. ಅವರು ಈ ಸ್ಥಿತಿ ಹೊಂದಲು ಹೋರಾಡಬೇಕಾಗಿರಲಿಲ್ಲ. ಅವು ಸಹಜವಾಗಿಯೇ ಅವರಿಗೆ ಯಾವಾಗೆಂದರೆ ಆಗ ಬರುತ್ತಿದ್ದುವು. ಅದೊಂದು ಅದ್ಭುತ ಪ್ರಸಂಗ. ಅವರನ್ನು ಪ್ರತ್ಯಕ್ಷ ನೋಡಿದ ಮೇಲೆ ನಮಗೆ ಈ ವಿಷಯವನ್ನು ಗ್ರಹಿಸಲು ಸಾಧ್ಯವಾಯಿತು. ಪ್ರತಿದಿನವೂ ಏಕಾಂಗಿಯಾಗಿ ಧ್ಯಾನಮಾಡು. ಎಲ್ಲವೂ ತಮ್ಮಷ್ಟಕ್ಕೆ ತಾವೇ ಗೋಚರವಾಗುವುವು. ಈಗ ಜಗನ್ಮಾತೆ, ಜೋತಿರ್ಮಯ ಮೂರ್ತಿ ನಿನ್ನಲ್ಲಿ ನಿದ್ರಿಸುತ್ತಿರುವಳು. ಅದಕ್ಕೇ ನೀನೇನೂ ತಿಳಿದುಕೊಳ್ಳಲಾರೆ. ಅವಳೇ ಕುಂಡಲಿನಿ. ಧ್ಯಾನಕ್ಕೆ ಮುಂಚೆ ನೀನು ‘ನರಶುದ್ಧಿ’ ಮಾಡುವಾಗ ಮಾನಸಿಕವಾಗಿ ನಿನ್ನ ಮೂಲಾಧಾರದಲ್ಲಿರುವ ಕುಂಡಲಿನಿಯನ್ನು ಹೊಡೆದೆಬ್ಬಿಸು - ‘ಏಳು ತಾಯಿ, ಏಳು’ ಎನ್ನು. ಇದನ್ನು ನಿಧಾನವಾಗಿ ಮಾಡಬೇಕು. ಧ್ಯಾನದ ವೇಳೆಯಲ್ಲಿ ನಿನ್ನ ಉದ್ರೇಕ ಭಾವನೆಗಳನ್ನು ಆದಷ್ಟು ನಿಗ್ರಹಿಸು. ಅದು ಬಹಳ ಅಪಾಯಕಾರಿ. ಯಾರು ಹೆಚ್ಚು ಉದ್ವೇಗಪರರೋ ಅವರಲ್ಲಿ ಕುಂಡಲಿನಿ ಬಹು ಬೇಗ ಜಾಗೃತಗೊಳ್ಳುವುದು. ಹಾಗೇ ಅಷ್ಟೇ ಬೇಗ ಅದು ಕೆಳಕ್ಕೂ ಹೋಗುವುದು. ಅದು ಕೆಳಕ್ಕೆ ಬಂದಾಗ ಭಕ್ತನನ್ನು ತುಂಬಾ ಶೋಚನೀಯಾವಸ್ಥೆಯಲ್ಲಿ ಬಿಟ್ಟು ಹೋಗುವುದು. ಅದಕ್ಕೇ ಕೀರ್ತನೆ ಮುಂತಾದ ಈ ಉದ್ವೇಗಪರ ಭಾವಕ್ಕೆ ಸಹಕಾರಿಯಾಗಿರುವವುಗಳಲ್ಲಿ ಒಂದು ದೊಡ್ಡ ಕೊರತೆಯಿದೆ. ಕುಣಿತ ಮುಂತಾದುವುಗಳಿಂದ ಆ ಕ್ಷಣಕ್ಕೆ ಆ ಶಕ್ತಿ ಮೇಲಕ್ಕೇರುವುದೇನೋ ನಿಜ. ಆದರೆ ಅದೆಂದೂ ಸ್ಥಿರವಲ್ಲ. ಅದಕ್ಕೆ ಬದಲು ಅದು ತನ್ನ ಸ್ಥಾನಕ್ಕೆ ಹಿಂತಿರುಗಿ ಬಂದು ವ್ಯಕ್ತಿಯ ಕಾಮವನ್ನು ಉದ್ರೇಕಿಸುವುದು. ಅಮೆರಿಕಾದಲ್ಲಿ ನನ್ನ ಉಪನ್ಯಾಸಗಳನ್ನು ಕೇಳಿ ತಾತ್ಕಾಲಿಕ ಉದ್ವೇಗದಿಂದ ಪ್ರೇಕ್ಷಕರಲ್ಲಿ ಹಲವರು ಭಾವವಶರಾಗುತ್ತಿದ್ದರು. ಕೆಲವರು ವಿಗ್ರಹದಂತೆ ನಿಶ್ಚಲರಾಗಿಬಿಡುತ್ತಿದ್ದರು. ನಂತರ ವಿಚಾರಿಸಿದುದರಲ್ಲಿ ಅವರಲ್ಲಿ ಅನೇಕರಿಗೆ ಈ ಅವಸ್ಥೆ ಕಳೆದ ಕೂಡಲೆ ಕಾಮಾಸಕ್ತಿ ಹೆಚ್ಚಾಗುತ್ತಿತ್ತೆಂದು ತಿಳಿಯಿತು. ಆದರೆ ಇದು ಸರಿಯಾಗಿ ಜಪ ಧ್ಯಾನ ಮಾಡದ ಕಾರಣ ಹೀಗಾಗುವುದು.

ಶಿಷ್ಯ: ಸ್ವಾಮೀಜಿ, ಯಾವ ಧರ್ಮಗ್ರಂಥಗಳಲ್ಲೂ ನಾನು ಈ ಆಧ್ಯಾತ್ಮಿಕ ಸಾಧನೆಯ ರಹಸ್ಯಗಳನ್ನು ಓದಿರಲಿಲ್ಲ - ಇಂದು ಅನೇಕ ಹೊಸ ವಿಷಯಗಳನ್ನು ಕೇಳಿದೆ.

ಸ್ವಾಮೀಜಿ: ಧರ್ಮಶಾಸ್ತ್ರಗಳು ಆಧ್ಯಾತ್ಮಿಕ ಸಾಧನೆಯ ಎಲ್ಲಾ ರಹಸ್ಯವನ್ನು ಒಳಗೊಂಡಿರುವುವೆಂದು ತಿಳಿದೆಯಾ? ಇವೆಲ್ಲಾ ಗುರು ಶಿಷ್ಯ ಪೀಳಿಗೆಯಿಂದ ರಹಸ್ಯವಾಗಿ ಕೊಡಲ್ಪಡುತ್ತವೆ. ಬಹು ಎಚ್ಚರಿಕೆಯಿಂದ ಧ್ಯಾನ, ಜಪವನ್ನು ಮಾಡು. ಸುವಾಸನಾ ಪುಷ್ಪಗಳನ್ನು ಇಟ್ಟು ಊದಿನ ಕಡ್ಡಿ ಹಚ್ಚಿಸು. ಪ್ರಾರಂಭದಲ್ಲಿ ಮನಸ್ಸು ಪರಿಶುದ್ಧವಾಗಲು ಈ ಬಾಹ್ಯ ಸಹಾಯವನ್ನು ತೆಗೆದುಕೋ. ನೀನು ನಿನ್ನ ಗುರು ಮತ್ತು ಇಷ್ಟದೇವತೆಯ ಮಂತ್ರವನ್ನು ಜಪಿಸುತ್ತಿರುವಾಗ ಹೀಗೆ ಹೇಳು: ಸರ್ವಜೀವಿಗಳಿಗೂ ಶಾಂತಿ ಇರಲಿ, ವಿಶ್ವದಲ್ಲೆಲ್ಲಾ ಶಾಂತಿ ಇರಲಿ. ಮೊದಲು ಈ ಶುಭಾಶಯಗಳನ್ನು ಉತ್ತರ, ದಕ್ಷಿಣ, ಪೂರ್ವ, ಪಶ್ಚಿಮ ಮೇಲೆ ಕೆಳಗೆ ಎಲ್ಲಾ ಕಡೆಗೂ ಕಳುಹಿಸು, ನಂತರ ಧ್ಯಾನಕ್ಕೆ ಕುಳಿತುಕೊ. ಪ್ರಥಮ ಘಟ್ಟದಲ್ಲಿ ಎಲ್ಲರೂ ಹೀಗೆ ಮಾಡಬೇಕು. ನಂತರ ಸ್ಥಿರವಾಗಿ ಕುಳಿತು (ಯಾವ ದಿಕ್ಕಿಗೆ ತಿರುಗಿ ಬೇಕಾದರೂ ಕುಳಿತುಕೋ) ನಾನು ನಿನಗೆ ದೀಕ್ಷೆ ಕೊಟ್ಟ ರೀತಿ ಧ್ಯಾನಮಾಡು. ಒಂದು ದಿನವನ್ನೂ ಹಾಗೆಯೇ ಬಿಡಬೇಡ. ನಿನಗೆ ತುಂಬಾ ಜರೂರಾದ ಕೆಲಸವಿದ್ದಲ್ಲಿ ಕಾಲು ಗಂಟೆಯಾದರೂ ಆಧ್ಯಾತ್ಮಿಕ ಸಾಧನೆ ಮಾಡು. ಮಗು, ನಿನ್ನಲ್ಲಿ ನಿಶ್ಚಲ ಭಕ್ತಿಯಿಲ್ಲದಿದ್ದಲ್ಲಿ ನಿನ್ನ ಗುರಿಯನ್ನು ಹೇಗೆ ಸೇರಬಲ್ಲೆ?

ಸ್ವಾಮೀಜಿ ಮಹಡಿಗೆ ಹೋದರು. ಹೋಗುವಾಗ ಹೇಳಿದರು: “ನೀವೆಲ್ಲಾ ಹೆಚ್ಚು ಶ್ರಮವಿಲ್ಲದೆ ಆಧ್ಯಾತ್ಮಿಕ ಅಂತಃಶಕ್ತಿ ಹೊಂದಿದ್ದೀರಿ. ನೀವಿಲ್ಲಿಗೆ ಬರುವಷ್ಟು ಅದೃಷ್ಟಶಾಲಿಗಳಾದುದರಿಂದ ನಿಮಗೆ ಮುಕ್ತಿ ಸಮೀಪದಲ್ಲೇ ಇದೆ. ಈಗ ನಿಮ್ಮ ಧ್ಯಾನ ಮುಂತಾದುವುಗಳೊಂದಿಗೆ ಗೋಳಿನಿಂದ ತುಂಬಿರುವ ವಿಶ್ವ ಜನತೆಯ ಕಷ್ಟ ನಷ್ಟಗಳನ್ನು ಕಡಿಮೆಮಾಡಲು ಹೃತ್ಪೂರ್ವಕವಾಗಿ ಪ್ರಯತ್ನಿಸಿ. ಕಠಿಣ ತಪಶ್ಚರ್ಯೆಯಿಂದ ನಾನು ಈ ದೇಹವನ್ನು ಕೃಶಮಾಡಿದೆ. ನನ್ನೀ ಮೂಳೆಮಾಂಸಗಳ ಚೀಲದಲ್ಲಿ ಕೊಂಚ ಕೂಡ ಶಕ್ತಿ ಇಲ್ಲ. ಈಗ ನೀವು ಕೆಲಸಕ್ಕೆ ಹೊರಡಿ. ನನಗೆ ಕೊಂಚ ವಿಶ್ರಾಂತಿ ಕೊಡಿ. ನೀವು ಮತ್ತಾವುದನ್ನು ಮಾಡಿ ಜಯಶೀಲರಾಗದಿದ್ದರೂ ಚಿಂತೆಯಿಲ್ಲ. ಇಲ್ಲಿಯವರೆಗೆ ವ್ಯಾಸಂಗ ಮಾಡಿರುವ ಆಧ್ಯಾತ್ಮಿಕ ಸತ್ಯಗಳನ್ನು ಜನತೆಗೆ ಬೋಧಿಸಿ. ಇದಕ್ಕಿಂಥ ಹೆಚ್ಚಾದ ಧ್ಯಾನವಿಲ್ಲ. ಏಕೆಂದರೆ ಪ್ರಪಂಚದಲ್ಲಿ ಜ್ಞಾನ ದಾನಕ್ಕಿಂತ ಮಿಗಿಲಾದ ದಾನವಿಲ್ಲ."

\newpage

\chapter[ಅಧ್ಯಾಯ ೪೪]{ಅಧ್ಯಾಯ ೪೪\protect\footnote{\engfoot{C.W. Vol. VII, P. 256}}}

\begin{center}
ಸ್ಥಳ: ಬೇಲೂರು ಮಠ, ವರ್ಷ: ಕ್ರಿ.ಶ. ೧೯೦೨.
\end{center}

ಸ್ವಾಮೀಜಿ ಈಗ ಮಠದಲ್ಲಿಯೇ ಇರುತ್ತಿದ್ದರು. ಶಿಷ್ಯನು ಮಠಕ್ಕೆ ಬಂದು ಸ್ವಾಮೀಜಿ ಮತ್ತು ಸ್ವಾಮಿ ಪ್ರೇಮಾನಂದರೊಡನೆ ಸಂಜೆ ತಿರುಗಾಡಿಕೊಂಡು ಬರಲು ಹೋಗಿದ್ದನು. ಸ್ವಾಮೀಜಿ ಆಲೋಚನಾಮಗ್ನರಾಗಿದ್ದುದನ್ನು ನೋಡಿ ಶಿಷ್ಯನು ಸ್ವಾಮಿ ಪ್ರೇಮಾನಂದರೊಡನೆ ಶ‍್ರೀರಾಮಕೃಷ್ಣರು ಸ್ವಾಮೀಜಿಯ ಮಹಿಮೆಯ ವಿಚಾರವಾಗಿ ಏನು ಹೇಳುತ್ತಿದ್ದರೆಂಬುದರ ಬಗ್ಗೆ ಮಾತಾಡುತ್ತಿದ್ದನು. ಕೊಂಚ ದೂರ ಹೋದ ಮೇಲೆ ಸ್ವಾಮೀಜಿ ಮಠಕ್ಕೆ ಹಿಂತಿರುಗಲು ಹೊರಟರು. ತಮ್ಮ ಬಳಿ ಇದ್ದ ಸ್ವಾಮಿ ಪ್ರೇಮಾನಂದ ಮತ್ತು ಶಿಷ್ಯನನ್ನು ನೋಡಿ ಸ್ವಾಮೀಜಿ “ನೀವೇನು ಮಾತನಾಡುತ್ತಿದ್ದಿರಿ?" ಎಂದು ಕೇಳಿದರು. ಶಿಷ್ಯ “ನಾವು ಶ‍್ರೀರಾಮಕೃಷ್ಣರು ಮತ್ತು ಅವರಿಗೆ ಸಂಬಂಧಪಟ್ಟ ವಿಚಾರ ಮಾತನಾಡುತ್ತಿದ್ದೆವು" ಎಂದನು. ಸ್ವಾಮೀಜಿ ಉತ್ತರವನ್ನು ಕೇಳಿದರು ಮಾತ್ರ - ಪುನಃ ಯೋಚನಾಮಗ್ನರಾದರು. ರಸ್ತೆಯ ಮೂಲಕ ಮಠಕ್ಕೆ ಹಿಂತಿರುಗಿದರು. ಮಾವಿನ ಮರದ ಕೆಳಗೆ ಇಟ್ಟಿದ್ದ ಹಗ್ಗದ ಮಂಚದ ಮೇಲೆ ಕೊಂಚ ಹೊತ್ತು ವಿಶ್ರಮಿಸಿಕೊಂಡರು. ಮುಖ ತೊಳೆದುಕೊಂಡು ಮೇಲಣ ವರಾಂಡದಲ್ಲಿ ತಿರುಗಾಡುತ್ತಾ ಹೇಳಿದರು: “ನೀನಿರುವ ಪ್ರದೇಶದಲ್ಲೇಕೆ ವೇದಾಂತ ಪ್ರಚಾರಮಾಡಲು ಪ್ರಯತ್ನಿಸಬಾರದು? ಅಲ್ಲಿ ತಾಂತ್ರಿಕ ಸಂಪ್ರದಾಯ ಭೀಷಣವಾಗಿ ಹಬ್ಬಿದೆ. ಅದ್ವೈತವಾದದ ಸಿಂಹಗರ್ಜನೆಯಿಂದ ದೇಶವನ್ನು ಅಲ್ಲೋಲಕಲ್ಲೋಲ ಮಾಡಿ ಎಬ್ಬಿಸು. ಆಗ ನೀನೊಬ್ಬ ವೇದಾಂತಿ ಎನ್ನುತ್ತೇನೆ. ಮೊದಲು ಅಲ್ಲೊಂದು ಶಾಲೆ ತೆರೆದು ಉಪನಿಷತ್ತು ಮತ್ತು ಬ್ರಹ್ಮಸೂತ್ರಗಳನ್ನು ಬೋಧಿಸು. ಹುಡುಗರಿಗೆ ಬ್ರಹ್ಮಚರ್ಯೆಯ ವಿಚಾರವಾಗಿ ಬೋಧಿಸು. ನೀನಿರುವ ಪ್ರದೇಶದಲ್ಲೇ ನ್ಯಾಯ ಪ್ರಧಾನ ತರ್ಕ, ಮಾತಿಗೆ ಮಾತು ಜೋಡಿಸುವುದು ಹೆಚ್ಚೆಂದು ಕೇಳಿದ್ದೇನೆ. ಅದರಲ್ಲಿ ಏನಿದೆ? ಕೇವಲ ವ್ಯಾಪ್ತಿ ಮತ್ತು ಅನುಮಾನ ಅಷ್ಟೆ. ಈ ವಿಷಯಗಳ ಮೇಲೆ ನೈಯಾಯಿಕ ಪಂಡಿತರು ತಿಂಗಳುಗಟ್ಟಲೆ ಚರ್ಚಿಸುವರು. ಆತ್ಮಜ್ಞಾನಕ್ಕೆ ಇದು ಎಷ್ಟರಮಟ್ಟಿಗೆ ಸಹಾಯಕಾರಿ? ನಿಮ್ಮ ಹಳ್ಳಿ ಅಥವಾ ನಾಗಮಹಾಶಯರ ಹಳ್ಳಿಯಲ್ಲಿ ಒಂದು ಶಾಲೆ ತೆಗೆದು ಧರ್ಮಶಾಸ್ತ್ರವನ್ನು, ಶ‍್ರೀರಾಮಕೃಷ್ಣರ ಜೀವನ ಮತ್ತು ಉಪದೇಶಗಳನ್ನು ಓದುವಂತೆ ಮಾಡಿ. ಇದರಿಂದ ನಿಮಗೆ ಶ್ರೇಯಸ್ಸಾಗುವುದಲ್ಲದೆ ಜಗತ್ಕಲ್ಯಾಣವಾಗಿ ನಿಮ್ಮ ಕೀರ್ತಿ ಶಾಶ್ವತವಾಗಿ ಉಳಿಯುವುದು."

ಶಿಷ್ಯ: ಸ್ವಾಮೀಜಿ, ನನಗೆ ಹೆಸರು ಕೀರ್ತಿಗಳ ಆಸೆ ಇಲ್ಲ. ಕೇವಲ ಒಮ್ಮೊಮ್ಮೆ ಮಾತ್ರ ನೀವು ಹೇಳಿದಂತೆ ಮಾಡಬೇಕೆನ್ನಿಸುವುದು. ಆದರೆ ಮದುವೆಯಾಗಿರುವುದರಿಂದ ಸಂಸಾರತಾಪತ್ರಯಕ್ಕೆ ಸಿಕ್ಕಿ ಈ ಆಸೆ ಕೇವಲ ನನ್ನ ಮನಸ್ಸಿನಲ್ಲಿಯೇ ಉಳಿದು ಬಿಡುವುದೇನೋ ಎಂದು ಹೆದರಿಕೆಯಾಗುವುದು.

ಸ್ವಾಮೀಜಿ: ನೀನು ಮದುವೆಯಾಗಿದ್ದರೇನಂತೆ? ನಿನ್ನ ತಂದೆತಾಯಿಗಳಿಗೆ ಸಹೋದರರಿಗೆ ಹಿಟ್ಟುಬಟ್ಟೆ ಕೊಟ್ಟು ಸಾಕುತ್ತಿರುವಂತೆ ನಿನ್ನ ಹೆಂಡತಿಗೂ ಮಾಡು. ಆಕೆಗೂ ಆಧ್ಯಾತ್ಮಿಕ ಸಲಹೆಗಳನ್ನು ಕೊಟ್ಟು ನಿನ್ನ ದಾರಿಯೆಡೆಗೆ ತಿರುಗುವಂತೆ ಮಾಡು. ಆಕೆಯನ್ನು ನಿನ್ನ ಆಧ್ಯಾತ್ಮಿಕ ಜೀವನದ ಜೊತೆಗಾತಿ, ಸಹಾಯಕಳೆಂದು ತಿಳಿ. ಉಳಿದ ಕಾಲಗಳಲ್ಲಿ ಇತರರನ್ನು ನೋಡುವಂತೆಯೇ ಆಕೆಯನ್ನು ನೋಡು. ಈ ರೀತಿ ಯೋಚಿಸುವುದರಿಂದ ಮನಸ್ಸಿನ ಚಂಚಲತೆಯೆಲ್ಲಾ ಮಾಯವಾಗುವುದು. ಅಂಜಿಕೆಯೇಕೆ?

ಶಿಷ್ಯನಿಗೆ ಈ ಮಾತುಗಳಿಂದ ತುಂಬಾ ಸಮಾಧಾನವಾಯಿತು. ಊಟವಾದ ಮೇಲೆ ಸ್ವಾಮೀಜಿ ತಮ್ಮ ಹಾಸಿಗೆಯ ಮೇಲೆ ಕುಳಿತುಕೊಂಡರು. ಶಿಷ್ಯನಿಗೆ ಈಗ ಕೊಂಚ ಸ್ವಾಮೀಜಿಗೆ ಸೇವೆಮಾಡಲು ಅವಕಾಶ ದೊರಕಿತು.

ಸ್ವಾಮೀಜಿ ಶಿಷ್ಯನಿಗೆ ಮಠದ ನಿವಾಸಿಗಳ ವಿಷಯದಲ್ಲಿ ಎಷ್ಟು ಗೌರವವಿಡಬೇಕೆಂಬುದರ ವಿಚಾರವಾಗಿ ಮಾತನಾಡಿದರು. “ನೀನೀಗ ನೋಡುತ್ತಿರುವ ಶ‍್ರೀರಾಮಕೃಷ್ಣರ ಮಕ್ಕಳು ಅದ್ಭುತ ತ್ಯಾಗಿಗಳು, ಅವರಿಗೆ ಸೇವೆ ಸಲ್ಲಿಸುವುದರಿಂದ ನಿನ್ನ ಮನಶ್ಶುದ್ಧಿಯಾಗಿ ಆತ್ಮಸಾಕ್ಷಾತ್ಕಾರ ಪಡೆದು ಧನ್ಯನಾಗುವೆ. ನಿನಗೆ ‘ಮಹಾತ್ಮರ ಸೇವೆ ಮಾಡುವುದರ ಮೂಲಕ ಮತ್ತು ಪ್ರಶ್ನಿಸುವುದರಿಂದ’ ಎಂಬ ಗೀತೆಯ ವಾಕ್ಯ ನೆನಪಿದೆಯೇ? ಆದ್ದರಿಂದ ನೀನು ಅವರಿಗೆ ಖಂಡಿತ ಸೇವೆಸಲ್ಲಿಸಬೇಕು. ಇದರಿಂದ ನೀನು ನಿನ್ನ ಗುರಿಯನ್ನು ಮುಟ್ಟುವುದಲ್ಲದೆ, ಅವರು ನಿನ್ನನ್ನು ಎಷ್ಟು ಪ್ರೀತಿಸುವರೆಂಬುದನ್ನು ತಿಳಿಯುವೆ."

ಶಿಷ್ಯ: ಆದರೆ ಅವರನ್ನು ಅರ್ಥಮಾಡಿಕೊಳ್ಳಲು ನನಗೆ ಬಹು ಕಷ್ಟವಾಗುತ್ತದೆ. ಪ್ರತಿಯೊಬ್ಬರೂ ಬೇರೆ ಬೇರೆ ಗುಂಪಿಗೆ ಸೇರಿದವರೆಂದು ಅನ್ನಿಸುವುದು.

ಸ್ವಾಮೀಜಿ: ಶ‍್ರೀರಾಮಕೃಷ್ಣರು ಒಬ್ಬ ಅದ್ಭುತ ತೋಟಗಾರರು. ಬಗೆಬಗೆಯ ಹೂಗಳಿಂದ ಹೂಗೊಂಚಲನ್ನು ಮಾಡಿ ಸಂಸ್ಥೆಯನ್ನು ಸ್ಥಾಪಿಸಿದ್ದಾರೆ. ಬಗೆಬಗೆಯ ನಮೂನೆ ಮತ್ತು ಭಾವನೆಗಳು ಅಲ್ಲಿವೆ. ಮುಂದೆಯೂ ಬರುತ್ತ ಇರುವುವು. ಶ‍್ರೀರಾಮಕೃಷ್ಣರು ‘ಯಾರು ಹೃತ್ಪೂರ್ವಕವಾಗಿ ದೇವರನ್ನು ಒಂದು ದಿನವಾದರೂ ಪ್ರಾರ್ಥಿಸುವರೋ ಅವರು ಇಲ್ಲಿಗೆ ಬಂದೇ ಬರುವರು’ ಎನ್ನುತ್ತಿದ್ದರು. ಇಲ್ಲಿರುವ ಪ್ರತಿಯೊಬ್ಬರೂ ಒಂದು ದೊಡ್ಡ ಆಧ್ಯಾತ್ಮಿಕ ಶಕ್ತಿಯೆಂದು ತಿಳಿ. ಅವರು ನನ್ನ ಮುಂದೆ ಸಂಕುಚಿತವಾದಂತಿದ್ದರೂ ಅವರು ಕೇವಲ ಸಾಧಾರಣ ಮನುಷ್ಯರೆಂದು ತಿಳಿಯಬೇಡ. ಅವರು ಯಾವಾಗ ಹೊರಗೆ ಹೋಗುವರೋ ಆಗ ಜನರಲ್ಲಿ ಆಧ್ಯಾತ್ಮಿಕ ಜಾಗೃತಿಯನ್ನುಂಟುಮಾಡುವರು. ಧಾರ್ಮಿಕ ಭಾವನೆಗಳ ಮೂರ್ತಿಮತ್ತಾದ ಶ‍್ರೀರಾಮಕೃಷ್ಣರ ಆಧ್ಯಾತ್ಮಿಕ ದೇಹದ ಭಾಗಗಳು ಅವರೆಂದು ತಿಳಿ. ನಾನು ಅವರನ್ನೆಲ್ಲಾ ಈ ದೃಷ್ಟಿಯಿಂದ ನೋಡುತ್ತೇನೆ. ಉದಾಹರಣೆಗೆ ಇಲ್ಲಿರುವ ಸ್ವಾಮಿ ಬ್ರಹ್ಮಾನಂದರನ್ನು ನೋಡು - ಅವರಲ್ಲಿರುವಷ್ಟು ಆಧ್ಯಾತ್ಮಿಕತೆ ನನ್ನಲ್ಲಿಯೂ ಇಲ್ಲ. ಶ‍್ರೀರಾಮಕೃಷ್ಣರು ಅವರನ್ನು ತಮ್ಮ ಆಧ್ಯಾತ್ಮಿಕ ಪುತ್ರರೆಂದು ಭಾವಿಸಿ ಅವರೊಡನೆಯೆ ಎಲ್ಲ ವ್ಯವಹಾರದಲ್ಲಿಯೂ ಬೆರೆಯುತ್ತಿದ್ದರು. ಅವರು ನಮ್ಮ ಮಠದ ಆಭರಣ - ನಮ್ಮ ರಾಜ. ಹಾಗೆಯೇ ಪ್ರೇಮಾನಂದ, ತುರೀಯಾನಂದ, ತ್ರಿಗುಣಾತೀತ, ಅಖಂಡಾನಂದ, ಶಾರದಾನಂದ, ರಾಮಕೃಷ್ಣಾನಂದ, ಸುಬೋಧಾನಂದ, ಮತ್ತಿತರರು. ನೀನು ಪ್ರಪಂಚದ ಯಾವ ಕಡೆ ಬೇಕಾದರೂ ಹುಡುಕಿನೋಡು. ಇಷ್ಟೊಂದು ಆಧ್ಯಾತ್ಮಿಕತೆ, ದೇವರಲ್ಲಿ ಶ್ರದ್ಧೆ ಇರುವವರು ಸಿಗುವುದು ದುರ್ಲಭ. ಅವರಲ್ಲಿ ಒಬ್ಬೊಬ್ಬರೂ ಆಧ್ಯಾತ್ಮಿಕ ಶಕ್ತಿಯ ಕೇಂದ್ರ. ಯೋಗ್ಯ ಕಾಲದಲ್ಲಿ ಆ ಶಕ್ತಿ ವಿಕಾಸಗೊಳ್ಳುವುದು.

ಶಿಷ್ಯ ಆಶ್ಚರ್ಯದಿಂದ ಎಲ್ಲವನ್ನೂ ಕೇಳಿಕೊಂಡ. ಸ್ವಾಮೀಜಿ ಮತ್ತೂ ಹೇಳಿದರು: “ಆದರೆ ನಿಮ್ಮ ಊರಿನ ಕಡೆಯಿಂದ ನಾಗಮಹಾಶಯರು ಹೊರತು ಮತ್ತಾರೂ ಶ‍್ರೀರಾಮಕೃಷ್ಣರ ಬಳಿಗೆ ಬರಲಿಲ್ಲ. ಶ‍್ರೀರಾಮಕೃಷ್ಣರನ್ನು ಸಂದರ್ಶಿಸಿದ ಕೆಲವರು ಅವರ ಮಹತ್ವವನ್ನು ಅರಿಯಲಾರದವರಾಗಿದ್ದರು.” ನಾಗಮಹಾಶಯರ ಯೋಚನೆ ಬಂದೊಡನೆ ಸ್ವಾಮೀಜಿ ಕೊಂಚಹೊತ್ತು ಮೌನವಾಗಿದ್ದರು. ಅವರಿನ್ನೂ ಕಾಲವಾಗಿ ೪-೫ ತಿಂಗಳಾಗಿತ್ತು. ಒಮ್ಮೆ ನಾಗಮಹಾಶಯರ ಮನೆಯಲ್ಲಿ ಗಂಗೆಯುದ್ಭವವಾದ ವಿಷಯವನ್ನು ಸ್ವಾಮೀಜಿ ಕೇಳಿದ್ದರು. ಅದನ್ನು ಜ್ಞಾಪಿಸಿಕೊಂಡು ಶಿಷ್ಯನನ್ನು ಕುರಿತು “ಆ ಘಟನೆ ಹೇಗೆ ನಡೆಯಿತು? ಅದರ ವಿಚಾರ ಹೇಳು" ಎಂದರು.

ಶಿಷ್ಯ: ಕೇವಲ ಆ ವಿಷಯ ನಾನು ಕೇಳಿದ್ದೇನೆಯೇ ಹೊರತು ಕಣ್ಣಾರೆ ನೋಡಲಿಲ್ಲ. ಮಹಾವಾರುಣಿಯೋಗದಲ್ಲಿ ನಾಗಮಹಾಶಯರು ತಂದೆಯೊಡನೆ ಕಲ್ಕತ್ತೆಗೆ ಹೊರಟರಂತೆ. ಆದರೆ ಹೊಗೆಬಂಡಿಯಲ್ಲಿ ಸ್ಥಳ ಸಿಗದ ಕಾರಣ ನಾರಾಯಣಗಂಜಿನಲ್ಲಿ ನಾಲ್ಕೈದು ದಿನಗಳು ತಂಗಬೇಕಾಯಿತಂತೆ. ಆದರೂ ವಿಫಲವಾಗಿ ಮನೆಗೆ ಹಿಂತಿರುಗಿದರು. ಆಗ ನಾಗಮಹಾಶಯರು ತಮ್ಮ ತಂದೆಗೆ ‘ಮನಸ್ಸು ಪರಿಶುದ್ಧವಾಗಿದ್ದರೆ ಗಂಗೆಯೇ ಪ್ರತ್ಯಕ್ಷಳಾಗುವಳು’ ಎಂದರಂತೆ. ನಂತರ ಆ ಪವಿತ್ರ ಸ್ನಾನ ಮಾಡುವ ಗಳಿಗೆ ಸಮೀಪಿಸಿದೊಡನೆಯೆ ಒಂದು ನೀರಿನ ಚಿಲುಮೆ ಅವರ ಅಂಗಳದ ನೆಲವನ್ನು ಭೇದಿಸಿಕೊಂಡು ಬಂದಿತು. ಅದನ್ನು ನೋಡಿದ ಹಲವರು ಇನ್ನೂ ಇರುವರು. ಆದರೆ ನಾನವರನ್ನು ನೋಡಿದ್ದು ಈ ಘಟನೆ ನಡೆದ ಹಲವು ವರುಷಗಳ ಮೇಲೆ.

ಸ್ವಾಮೀಜಿ: ಇದರಲ್ಲಿ ಹೊಸದೇನೂ ಇಲ್ಲ. ಅವರೊಬ್ಬ ಸತ್ಯಸಂಕಲ್ಪ ಮಹಾತ್ಮರು. ಅವರ ವಿಷಯವಾಗಿ ಈ ಘಟನೆ ಸೋಜಿಗವೇನಲ್ಲ ಎಂದು ನನ್ನ ಭಾವನೆ.

ಸ್ವಾಮಿಜಿ ಇದನ್ನು ಹೇಳುತ್ತಿರುವಾಗ ನಿದ್ರೆ ಬರುವ ಹಾಗಾಗಲು ಮಲಗಿದರು. ಆಗ ಶಿಷ್ಯ ಭೋಜನ ಮಾಡಲು ಕೆಳಗಿಳಿದು ಬಂದ.

\newpage

\chapter[ಅಧ್ಯಾಯ ೪೫]{ಅಧ್ಯಾಯ ೪೫\protect\footnote{\engfoot{C.W, Vol. VII, P. 259}}}

\begin{center}
ಸ್ಥಳ: ಕಲ್ಕತ್ತೆಯಿಂದ ಮಠಕ್ಕೆ ದೋಣಿಯಲ್ಲಿ ಹೋಗುತ್ತಿದ್ದಾಗ, ವರ್ಷ: ಕ್ರಿ.ಶ. ೧೯೦೨.
\end{center}

ಒಂದು ದಿನ ಮಧ್ಯಾಹ್ನ ಕಲ್ಕತ್ತೆಯಲ್ಲಿ ಗಂಗಾತೀರದಲ್ಲಿ ಅಡ್ಡಾಡುತ್ತಿದ್ದಾಗ ಒಬ್ಬ ಸಂನ್ಯಾಸಿಯು ದೂರದಿಂದ ಅಹಿರಿಟೊಲಘಾಟಿನ ಕಡೆಗೆ ಬರುತ್ತಿರುವುದು ಶಿಷ್ಯನಿಗೆ ಕಾಣಿಸಿತು. ಹತ್ತಿರ ಬಂದಾಗ ಶಿಷ್ಯನಿಗೆ ಅವರು ಮತ್ತಾರೂ ಅಲ್ಲ, ತನ್ನ ಗುರು ಸ್ವಾಮಿ ವಿವೇಕಾನಂದರೆಂದು ತಿಳಿಯಿತು. ಅವರ ಎಡಗೈಯಲ್ಲಿದ್ದ ಎಲೆಯ ಪೊಟ್ಟಣದಲ್ಲಿ ಹುರಿಗಾಳು ಇತ್ತು. ಅದನ್ನು ಚಿಕ್ಕ ಹುಡುಗನಂತೆ ತಿನ್ನುತ್ತಾ ತುಂಬಾ ಹರ್ಷದಿಂದ ದಾರಿಯಲ್ಲಿ ನಡೆದುಕೊಂಡು ಬರುತ್ತಿದ್ದರು. ಶಿಷ್ಯನ ಮುಂದೆ ನಿಂತಾಗ ಶಿಷ್ಯ ಪ್ರಣಾಮಮಾಡಿ ಅವರು ಕಲ್ಕತ್ತೆಗೆ ಬಂದ ಕಾರಣವನ್ನು ಕೇಳಿದನು.

ಸ್ವಾಮೀಜಿ: ನಾನು ಕಾರ್ಯಾರ್ಥವಾಗಿ ಇಲ್ಲಿಗೆ ಬಂದಿದ್ದೆ. ನೀನೂ ಮಠಕ್ಕೆ ಬರುವೆಯೇನು? ಕೊಂಚ ಹುರಿಗಾಳನ್ನು ತಿನ್ನು. ಇದು ಉಪ್ಪು ಖಾರವಾಗಿ ರುಚಿಯಾಗಿದೆ.

ಶಿಷ್ಯನು ಸಂತೋಷದಿಂದ ಆ ತಿಂಡಿಯನ್ನು ತೆಗೆದುಕೊಂಡು ಸ್ವಾಮೀಜಿಯೊಡನೆ ಮಠಕ್ಕೆ ಹೋಗಲೊಪ್ಪಿದನು.

ಸ್ವಾಮೀಜಿ: ಹಾಗಾದರೆ ದೋಣಿಯನ್ನು ನೋಡು.

ಶಿಷ್ಯನು ದೋಣಿಯನ್ನು ಬಾಡಿಗೆಗೆ ಗೊತ್ತುಮಾಡಲು ಅವಸರದಿಂದ ಹೋದನು. ದೋಣಿಯವನು ಎಂಟು ಆಣೆಗಳನ್ನು ಕೇಳಿದಾಗ ದೋಣಿಯ ಬಾಡಿಗೆಯನ್ನು ಚೌಕಾಸಿ ಮಾಡುತ್ತಿದ್ದ ಶಿಷ್ಯನನ್ನು ಸ್ವಾಮೀಜಿ ನಿಲ್ಲಿಸಿ ತಾವೇ ಮುಂದೆ ಬಂದು “ನೀನೇಕೆ ಅವರೊಂದಿಗೆ ಚೌಕಾಸಿ ಮಾಡುತ್ತಿರುವೆ" ಎಂದು ಹೇಳಿ ದೋಣಿಯವನಿಗೆ “ಆಗಲಿ, ನಿನಗೆ ಎಂಟು ಆಣೆಯನ್ನೇ ಕೊಡುತ್ತೇನೆ" ಎಂದು ದೋಣಿಯನ್ನು ಹತ್ತಿದರು. ದೋಣಿ ಪ್ರವಾಹಕ್ಕೆದುರಾಗಿ ನಿಧಾನವಾಗಿ ಮುಂದುವರಿದು ಮಠವನ್ನು ಸೇರಲು ಸುಮಾರು ಒಂದೂವರೆ ಗಂಟೆ ತೆಗೆದುಕೊಂಡಿತು. ದೋಣಿಯಲ್ಲಿ ಸ್ವಾಮೀಜಿಯವರೊಬ್ಬರೇ ಇದ್ದುದರಿಂದ ಶಿಷ್ಯನಿಗೆ ಎಲ್ಲಾ ವಿಷಯಗಳನ್ನೂ ಯಾವ ಅಡ್ಡಿಯೂ ಇಲ್ಲದೆ ನಿರಾತಂಕವಾಗಿ ಸ್ವಾಮಿಗಳೊಡನೆ ಮಾತನಾಡುವ ಸದವಕಾಶ ದೊರೆಯಿತು. ಶ‍್ರೀರಾಮಕೃಷ್ಣರ ಶಿಷ್ಯರ ಘನತೆಯನ್ನು ಹಾಡುವ ಒಂದು ಗೌರವಯುತ ಹಾಡನ್ನು ಶಿಷ್ಯನು ಇತ್ತೀಚೆಗೆ ರಚಿಸಿದ್ದನು. ಅದರ ವಿಷಯವಾಗಿ ಮಾತನಾಡುತ್ತಾ ಸ್ವಾಮೀಜಿ “ನೀನು ಶ್ಲೋಕದಲ್ಲಿ ಹೇಳಿರುವವರೆಲ್ಲ ಶ‍್ರೀರಾಮಕೃಷ್ಣರ ಅಂತರಂಗ ಶಿಷ್ಯರೆಂದು ನೀನು ಹೇಗೆ ಬಲ್ಲೆ?" ಎಂದು ಕೇಳಿದರು.

ಶಿಷ್ಯ: ಸ್ವಾಮೀಜಿ, ನಾನು ಶ‍್ರೀರಾಮಕೃಷ್ಣರ ಸಂನ್ಯಾಸಿ ಭಕ್ತರು ಮತ್ತು ಗೃಹಸ್ಥ ಭಕ್ತರೊಡನೆ ಇಷ್ಟೊಂದು ವರುಷಗಳು ಕಲೆತಿದ್ದೇನೆ. ನಾನು ಅವರ ಬಾಯಿಂದಲೇ ಅವರೆಲ್ಲಾ ಶ‍್ರೀರಾಮಕೃಷ್ಣರ ಶಿಷ್ಯರೆಂದು ಕೇಳಿರುವೆ.

ಸ್ವಾಮೀಜಿ: ಹೌದು, ಅವರೆಲ್ಲಾ ಶ‍್ರೀರಾಮಕೃಷ್ಣರ ಭಕ್ತರು. ಆದರೆ ಎಲ್ಲರೂ ಶ‍್ರೀರಾಮಕೃಷ್ಣರ ಅತ್ಯಂತ ನಿಕಟವಾದ, ಅಂತರಂಗ ವರ್ಗಕ್ಕೆ ಸೇರಿದವರಲ್ಲ. ಕಾಶೀಪುರದ ತೋಟದಲ್ಲಿದ್ದಾಗ ಶ‍್ರೀರಾಮಕೃಷ್ಣರು ನಮಗೆ ‘ಇವರೆಲ್ಲಾ ನನ್ನ ಅಂತರಂಗ ಭಕ್ತರಲ್ಲವೆಂದು, ಜಗನ್ಮಾತೆ ನನಗೆ ತೋರಿಸಿದಳು’ ಎಂದು ಹೇಳಿದ್ದರು. ಸ್ತ್ರೀ ಮತ್ತು ಪುರುಷ ಭಕ್ತರೆಲ್ಲರ ವಿಷಯವಾಗಿ ಅವರು ಹೇಳಿದರು.

ಅನಂತರ ಶ‍್ರೀರಾಮಕೃಷ್ಣರು ತಮ್ಮ ಶಿಷ್ಯವರ್ಗದಲ್ಲಿ ಹೇಗೆ ದೊಡ್ಡವರು ಸಣ್ಣವರೆಂಬ ಮಟ್ಟವನ್ನು ತೋರಿಸಿಕೊಡುತ್ತಿದ್ದರು ಎಂಬ ವಿಷಯವಾಗಿ ಮಾತನಾಡುತ್ತಾ ಸ್ವಾಮಿಗಳು ಶಿಷ್ಯನಿಗೆ ಗೃಹಸ್ಥ ಮತ್ತು ಸಂನ್ಯಾಸಿ ಜೀವನದ ಮಹತ್ತರವಾದ ಭೇದವನ್ನು ವಿಸ್ತಾರವಾಗಿ ವಿವರಿಸಿದರು.

ಸ್ವಾಮೀಜಿ: ಯಾರು ಕಾಮಿನಿ ಕಾಂಚನಕ್ಕೆ ದಾಸರಾಗಿದ್ದಾರೆ ಅವರು ಶ‍್ರೀರಾಮಕೃಷ್ಣರನ್ನು ಸರಿಯಾಗಿ ಅರ್ಥಮಾಡಿಕೊಳ್ಳುವರೇನು? ಅದು ಎಂದಾದರೂ ಸಾಧ್ಯವೇನು? ಅಂತಹ ಮಾತುಗಳನ್ನು ಕೊಂಚವೂ ನಂಬಬೇಡಿ. ಶ‍್ರೀರಾಮಕೃಷ್ಣರ ಭಕ್ತರಲ್ಲಿ ಅನೇಕರು ತಾವು ಈಶ್ವರಕೋಟಿಗಳು, ಅಂತರಂಗ ಭಕ್ತರೆಂದು ಹೇಳಿಕೊಳ್ಳುತ್ತಿರುವರು. ಅವರು ಆ ಉಚ್ಚಮಟ್ಟದ ತ್ಯಾಗ ಅಥವಾ ವೈರಾಗ್ಯವನ್ನು ಗ್ರಹಿಸಲು ಕೂಡ ಅಶಕ್ತರು. ಅಂಥವರು ತಾವು ಅವರ ಅಂತರಂಗ ಭಕ್ತರೆಂದು ಹೇಳಿಕೊಳ್ಳುತ್ತಿರುವರು. ಶ‍್ರೀ ರಾಮಕೃಷ್ಣರು ತ್ಯಾಗ ಚಕ್ರವರ್ತಿ(ಶಿರೋಮಣಿ)ಗಳಾಗಿದ್ದರು. ಅವರ ಕೃಪಾಕಟಾಕ್ಷವು ಬಿದ್ದ ಯಾವನಾದರೂ ಕಾಮಿನಿ ಕಾಂಚನ ಭೋಗದಲ್ಲಿ ತನ್ನ ಜೀವನವನ್ನು ಕಳೆಯಲು ಸಾಧ್ಯವೆ?

ಶಿಷ್ಯ: ಹಾಗಾದರೆ ಸ್ವಮೀಜಿ, ದಕ್ಷಿಣೇಶ್ವರದಲ್ಲಿದ್ದಾಗ ಅವರ ಹತ್ತಿರ ಬಂದವರಾರೂ ಅವರ ಭಕ್ತರಲ್ಲವೆ?

ಸ್ವಾಮೀಜಿ: ಹಾಗೆಂದವರಾರು? ಯಾರು ಯಾರು ಶ‍್ರೀರಾಮಕೃಷ್ಣರ ಹತ್ತಿರ ಹೋಗಿದ್ದರೆ ಅವರೆಲ್ಲಾ ಆಧ್ಯಾತ್ಮಿಕತೆಯಲ್ಲಿ ಮುಂದುವರಿದಿರುವರು, ಮುಂದುವರಿಯುತ್ತಿರುವರು, ಮುಂದುವರಿಯುವರು. ಹಿಂದಿನ ಕಲ್ಪದಲ್ಲಿ ಪರಿಪೂರ್ಣತೆ ಪಡೆದ ಋಷಿಗಳು ಮನುಷ್ಯ ದೇಹಧಾರಣೆಮಾಡಿ ಅವತಾರಪುರುಷರೊಡನೆ ಈ ಭೂಮಿಗೆ ಬರುವರು. ಅವರೇ ದೇವನ ಒಡನಾಡಿಗಳು. ಭಗವಂತನು ಅವರ ಮೂಲಕ ಕೆಲಸ ಮಾಡಿ ತನ್ನ ಧರ್ಮವನ್ನು ಪ್ರಚಾರ ಮಾಡುವನು. ಯಾರು ಇತರರಿಗಾಗಿ ತಮ್ಮ ಸರ್ವಸ್ವವನ್ನೂ ತ್ಯಾಗ ಮಾಡುವರೊ, ಯಾರು ಭೋಗಲಾಲಸೆಯನ್ನೆಲ್ಲಾ ತಿರಸ್ಕರಿಸಿ ಬಿಟ್ಟು ತಮ್ಮ ಜೀವಮಾನವನ್ನೆಲ್ಲಾ ಜೀವಿಗಳ ಉದ್ದಾರಕ್ಕಾಗಿ, ಜಗತ್ಕಲ್ಯಾಣಕ್ಕಾಗಿ ಅರ್ಪಿಸುವರೊ, ಅವರು ಮಾತ್ರ ಅವತಾರಪುರುಷರ ಸಹಚರರೆಂದು ತಿಳಿ. ಏಸುವಿನ ಶಿಷ್ಯರೆಲ್ಲಾ ಸಂನ್ಯಾಸಿಗಳು, ಶಂಕರ, ರಾಮಾನುಜ, ಶ‍್ರೀಚೈತನ್ಯ ಮತ್ತು ಬುದ್ಧ - ಇವರ ಕೃಪೆಗೆ ನೇರವಾಗಿ ಪಾತ್ರರಾದವರು ಸರ್ವಪರಿತ್ಯಾಗಿಗಳಾದ ಸಂನ್ಯಾಸಿಗಳು. ಇಂತಹ ಮಹಾವ್ಯಕ್ತಿಗಳು ಜಗತ್ತಿಗೆಲ್ಲಾ ಬ್ರಹ್ಮವಿದ್ಯೆಯನ್ನು ತಮ್ಮ ಶಿಷ್ಯ ಪರಂಪರೆಯಿಂದ ಪ್ರಚಾರ ಮಾಡಿದವರು. ಕಾಮಿನೀಕಾಂಚನಕ್ಕೆ ದಾಸನಾಗಿರುವ ಮನುಷ್ಯ ಮತ್ತೊಬ್ಬನನ್ನು ಮುಕ್ತನನ್ನಾಗಿ ಮಾಡಿರುವುದನ್ನು ಅಥವಾ ದೇವರ ಹಾದಿಯನ್ನು ತೋರಿಸಬಲ್ಲವನಾಗಿರುವನೆಂಬುದನ್ನು ನೀನು ಎಲ್ಲಿಯಾದರೂ, ಯಾವಾಗಲಾದರೂ ಕೇಳಿರುವೆಯೇನು? ತಾನೇ ಬದ್ಧನಾಗಿರುವಾಗ ಇತರರನ್ನು ಮುಕ್ತರನ್ನಾಗಿ ಮಾಡಲು ಹೇಗೆ ಸಾಧ್ಯ? ವೇದ, ವೇದಾಂತ, ಇತಿಹಾಸ, ಪುರಾಣ ಯಾವುದರಲ್ಲಿ ನೋಡಿದರೂ ಕೇವಲ ಸಂನ್ಯಾಸಿಗಳೇ ಎಲ್ಲಾ ಯುಗಗಳಲ್ಲೂ ಧರ್ಮ ಬೋಧಕರಾಗಿದ್ದರೆಂಬುದನ್ನು ನೋಡುವೆ. ಇತಿಹಾಸ, ಪುನರಾವೃತ್ತಿಯಾಗುವುದು. ಈಗಲೂ ಹಾಗೆಯೇ ಆಗುವುದು. ಸರ್ವಧರ್ಮ ಸಮನ್ವಯಾಚಾರ್ಯರಾದ ಶ‍್ರೀರಾಮಕೃಷ್ಣರ ಸಮರ್ಥ ಸಂನ್ಯಾಸಿ ಶಿಷ್ಯರು ಎಲ್ಲೆಡೆಯಲ್ಲಿಯೂ ಮಾನವರ ಗುರುಗಳೆಂದು ಗೌರವಿಸಲ್ಪಡುವರು. ಇತರರ ಮಾತು ಪೊಳ್ಳು ಧ್ವನಿಯಂತೆ ಗಾಳಿಯಲ್ಲಿ ವ್ಯರ್ಥವಾಗುವುದು. ಮಠದ ಸ್ವಾರ್ಥ ತ್ಯಾಗಿಗಳಾದ ಸಂನ್ಯಾಸಿಗಳು ಧಾರ್ಮಿಕ ಭಾವನೆಗಳ ಕೇಂದ್ರವಾಗಿ, ಅವುಗಳನ್ನು ಹರಡುವರು. ನಿನಗೆ ಅರ್ಥವಾಯಿತೆ?

ಶಿಷ್ಯ: ಹಾಗಾದರೆ ಶ‍್ರೀರಾಮಕೃಷ್ಣರ ಗೃಹಸ್ಥ ಭಕ್ತರು ಅವರ ವಿಷಯವಾಗಿ ವಿಧವಿಧವಾಗಿ ಪ್ರಚಾರ ಮಾಡುತ್ತಿರುವುದು ನಿಜವಲ್ಲವೇನು?

ಸ್ವಾಮೀಜಿ: ಅವರು ಹೇಳುವುದೆಲ್ಲ ಸಂಪೂರ್ಣವಾಗಿ ಸುಳ್ಳೆಂದು ಹೇಳಲಾಗುವುದಿಲ್ಲ. ಆದರೆ ಅವರು ಶ‍್ರೀರಾಮಕೃಷ್ಣರ ವಿಷಯವಾಗಿ ಹೇಳುತ್ತಿರುವುದೆಲ್ಲಾ ಭಾಗಶಃ ಸತ್ಯ. ಪ್ರತಿಯೊಬ್ಬರೂ ಅವರವರ ಯೋಗ್ಯತೆಗೆ ತಕ್ಕಂತೆ ಶ‍್ರೀರಾಮಕೃಷ್ಣರನ್ನು ಅರ್ಥಮಾಡಿಕೊಂಡಿದ್ದಾರೆ. ಹಾಗೆಯೇ ಅವರ ವಿಷಯವಾಗಿ ಚರ್ಚಿಸುತ್ತಿದ್ದಾರೆ. ಹೀಗೆ ಮಾಡುವುದರಲ್ಲಿ ಕೆಟ್ಟದ್ದೇನೂ ಇಲ್ಲ. ಆದರೆ ಅವರ ಯಾವ ಭಕ್ತನೇ ಆಗಲಿ ತಾನು ಅವರನ್ನು ಅರ್ಥಮಾಡಿಕೊಂಡಿರುವುದೇ ಸತ್ಯ ಎಂದು ನಿಷ್ಕರ್ಷಿಸಿದರೆ ಅಂಥವನ ವಿಷಯವಾಗಿ ಕನಿಕರಪಡಬೇಕಷ್ಟೆ. ಕೆಲವರು ಶ‍್ರೀರಾಮಕೃಷ್ಣರು ಒಬ್ಬ ತಾಂತ್ರಿಕರೆಂದೂ, ಕೌಲರೆಂದೂ, ಮತ್ತೆ ಕೆಲವರು ನಾರದೀಯ ಭಕ್ತಿಯನ್ನು ಹರಡಲು ಅವತರಿಸಿದ ಶ‍್ರೀಚೈತನ್ಯರೆಂದೂ, ಇನ್ನೂ ಕೆಲವರು ಅವರು ಸಾಧನೆಗಳನ್ನು ಮಾಡಿದುದರಿಂದ ಅವತಾರಪುರುಷರಲ್ಲವೆಂದೂ, ಮತ್ತೂ ಕೆಲವರು ಸಂನ್ಯಾಸ ತೆಗೆದುಕೊಳ್ಳುವುದು ಅವರ ಉಪದೇಶಕ್ಕೆ ವಿರುದ್ದವೆಂದೂ ಅಭಿಪ್ರಾಯಪಡುವರು. ಈ ಮಾತುಗಳನ್ನು ನೀನು ಅವರ ಗೃಹಸ್ಥ ಭಕ್ತರಿಂದ ಕೇಳಿರಬಹುದು. ಈ ಪಕ್ಷಪಾತ ದೃಷ್ಟಿಯುಳ್ಳ ಅಭಿಪ್ರಾಯಕ್ಕೆ ಕೊಂಚವೂ ಗಮನ ಕೊಡಬೇಡ. ನಮ್ಮ ಇಡೀ ಜೀವಮಾನವೆಲ್ಲಾ ಸಾಧನೆಯಲ್ಲಿ ಕಳೆದರೂ ಅವರೇನೆಂಬುದನ್ನು, ಹಿಂದಿನ ಎಷ್ಟೊಂದು ಅವತಾರಗಳು ಅವರಲ್ಲಿ ಮೂರ್ತಿಮತ್ತಾಗಿದ್ದುವೆಂಬುದನ್ನು ಗ್ರಹಿಸಲಾರದೆ ಹೋದೆವು. ಅದಕ್ಕೇ ಅವರ ವಿಚಾರ ಮಾತಾಡುವಾಗ ಬಹು ಎಚ್ಚರಿಕೆಯಿಂದ, ಹಿಡಿತದಿಂದ ಮಾತಾಡಬೇಕು. ಅವರವರ ಯೋಗ್ಯತಾನುಸಾರ ಅವರ ಭಾವನೆಗಳನ್ನು ಗ್ರಹಿಸಿದ್ದಾರೆ. ಅವರ ಆಧ್ಯಾತ್ಮಿಕ ತುಂಬು ಸಾಗರದ ಒಂದು ತುಂತುರನ್ನು ಸಾಕ್ಷಾತ್ಕರಿಸಿಕೊಂಡರೆ ಸಾಕು, ಮಾನವನು ದೇವನಾಗುತ್ತಾನೆ. ಪ್ರಪಂಚದ ಇತಿಹಾಸದಲ್ಲಿ ಮತ್ತೊಮ್ಮೆ ಎಲ್ಲಿಯೂ ಇಂತಹ ವಿಶ್ವಭಾವನೆಗಳ ಸಂಯೋಗ ಲಭಿಸುವುದಿಲ್ಲ. ಇದರಿಂದಲೇ ನೀನು ಶ‍್ರೀರಾಮಕೃಷ್ಣರು ಎಂತಹ ಮನುಷ್ಯರಾಗಿ ಹುಟ್ಟಿದ್ದರೆಂಬುದನ್ನು ಅರ್ಥಮಾಡಿಕೊ. ಅವರು ತಮ್ಮ ಸಂನ್ಯಾಸಿ ಶಿಷ್ಯರಿಗೆ ಶಿಕ್ಷಣ ಕೊಡುತ್ತಿದ್ದಾಗ ತಮ್ಮ ಪೀಠದಿಂದೆದ್ದು ಯಾರಾದರೂ ಗೃಹಸ್ಥರು ಅಲ್ಲಿಗೆ ಬರುತ್ತಿರುವರೋ ಎಂದು ಸರಿಯಾಗಿ ನೋಡುತ್ತಿದ್ದರು. ಯಾರೂ ಕಾಣದಿದ್ದಲ್ಲಿ ಆವೇಶಪೂರಿತ ಮಾತುಗಳಿಂದ ನಮಗೆ ತ್ಯಾಗ ಮತ್ತು ತಪಸ್ಸಿನ ಮಹಿಮೆಯನ್ನು ವರ್ಣಿಸುತ್ತಿದ್ದರು. ಇದರ ಪರಿಣಾಮದಿಂದುಂಟಾದ ತೀಕ್ಷ್ಣ ವೈರಾಗ್ಯದ ಶಕ್ತಿಯಿಂದಲೇ ನಾವು ಪ್ರಪಂಚವನ್ನು ತ್ಯಜಿಸಿ ಪ್ರಾಪಂಚಿಕತೆಗೆ ವಿಮುಖರಾದೆವು.

ಶಿಷ್ಯ: ಅವರು ಸಂನ್ಯಾಸಿಗಳಿಗೂ ಮತ್ತು ಗೃಹಸ್ಥರಿಗೂ ಅಷ್ಟೊಂದು ಭೇದಮಾಡುತ್ತಿದ್ದರೆ?

ಸ್ವಾಮೀಜಿ: ಗೃಹಸ್ಥ ಭಕ್ತರನ್ನೇ ಆ ವಿಷಯವಾಗಿ ಕೇಳಿ ತಿಳಿದುಕೊ. ನೀನೇ ಯೋಚಿಸಿ ನೋಡು, ಯಾರು ಹೆಚ್ಚೆಂದು - ದೇವರ ಸಾಕ್ಷಾತ್ಕಾರಕ್ಕಾಗಿ ಪ್ರಾಪಂಚಿಕ ಜೀವನದ ಎಲ್ಲಾ ಭೋಗಗಳನ್ನೂ ತ್ಯಜಿಸಿ, ಬೆಟ್ಟಗಳಲ್ಲಿ, ಕಾಡುಗಳಲ್ಲಿ, ತೀರ್ಥ, ಆಶ್ರಮಗಳಲ್ಲಿ ಸಾಧನೆ ಮಾಡುತ್ತಾ ಜೀವನವನ್ನು ಕಳೆಯುತ್ತಿರುವವರು ಹೆಚ್ಚೋ, ಅಥವಾ ಕೇವಲ ದೇವರ ಹೆಸರು ಮತ್ತು ನೆನಪುಗಳನ್ನು ಕೊಂಡಾಡುತ್ತಾ ಪ್ರಪಂಚದ ಮಾಯೆಯಲ್ಲಿ ಮುಳುಗಿ ಏಳಲಾರದೆ ಇರುವವರು ಹೆಚ್ಚೋ? ಯಾರು ವಿಶ್ವ ಜನತೆಯ ಸೇವೆಗಾಗಿ, ಅವರನ್ನು ಆತ್ಮವೆಂದು ಭಾವಿಸಿ ಮುಂದೆ ಬರುವರೋ, ಯಾರು ಸಣ್ಣ ವಯಸ್ಸಿನಿಂದಲೂ ಬ್ರಹ್ಮಚರ್ಯೆಯಲ್ಲಿದ್ದು, ತ್ಯಾಗ ವೈರಾಗ್ಯಗಳೇ ಮೂರ್ತಿವೆತ್ತಂತಿರುವರೊ ಅಂಥವರು ಹೆಚ್ಚೋ ಅಥವಾ ನೊಣಗಳಂತೆ ಒಮ್ಮೆ ಪುಷ್ಪದಮೇಲೆ ಮರುಗಳಿಗೆಯೇ ಗೊಬ್ಬರದ ಗುಂಡಿಯ ಮೇಲೆ ಕುಳಿತುಕೊಳ್ಳುವರೋ ಅಂಥವರು ಹೆಚ್ಚೋ? ನೀನೇ ಇದನ್ನು ಯೋಚಿಸಿ ನಿರ್ಧರಿಸು.

ಶಿಷ್ಯ: ಆದರೆ ಸ್ವಾಮೀಜಿ, ಯಾರು ಭಗವತ್ಕೃಪೆಗೆ ಪಾತ್ರರಾಗಿದ್ದಾರೆ ಅವರು ಈ ಪ್ರಪಂಚವನ್ನು ಹೇಗೆ ನೋಡುವರು? ಅವರು ಗೃಹಸ್ಥರಾಗಿಯೆ ಉಳಿಯಲಿ ಅಥವಾ ಸಂನ್ಯಾಸಿಗಳಾಗಲಿ - ಅದರಿಂದೇನೂ ಬಾಧಕವಿಲ್ಲ ಎಂದು ನನಗನ್ನಿಸುವುದು.

ಸ್ವಾಮೀಜಿ: ಯಾರು ನಿಜವಾಗಿ ಆತನ ಕೃಪೆಯನ್ನು ಪಡೆದಿರುವರೊ ಅಂಥವರು ಪ್ರಾಪಂಚಿಕತೆಗೆ ಎಂದೂ ವಶರಾಗಿರುವುದಿಲ್ಲ. ಆತನ ಕೃಪೆಯ ನಿಜವಾದ ಒರೆಗಲ್ಲೆ - ಕಾಮಿನಿಕಾಂಚನ ವಿರಕ್ತಿ. ಯಾವ ಮನುಷ್ಯನ ಜೀವನದಲ್ಲಿ ಇದು ಬಂದಿಲ್ಲವೋ ಅಂಥವನಿಗೆ ಖಂಡಿತವಾಗಿಯೂ ಭಗವತ್ಕೃಪೆ ಲಭಿಸಿಲ್ಲ.

ಮೇಲಿನ ಸಂಭಾಷಣೆ ಹೀಗೆ ಕೊನೆಗೊಂಡ ನಂತರ ಬೇರೊಂದು ವಿಷಯವನ್ನೆತ್ತುತ್ತಾ ಶಿಷ್ಯನು ಸ್ವಾಮೀಜಿಯವರನ್ನು ಕುರಿತು, “ಸ್ವಾಮೀಜಿ, ನೀವು ಇಲ್ಲಿ ಮತ್ತು ಪರದೇಶಗಳಲ್ಲೆಲ್ಲಾ ಮಾಡಿದ ಕೆಲಸದ ಫಲಿತಾಂಶವೇನು?"

ಸ್ವಾಮೀಜಿ: ನಾನು ಮಾಡಿರುವುದರಲ್ಲಿ ಎಲ್ಲೋ ಕೊಂಚ ಭಾಗಮಾತ್ರ ಈಗ ಪ್ರಕಾಶಕ್ಕೆ ಬಂದಿದೆ, ಅಷ್ಟೆ. ಯೋಗ್ಯಕಾಲದಲ್ಲಿ ಇಡೀ ಪ್ರಪಂಚವೇ ಶ‍್ರೀರಾಮಕೃಷ್ಣರ ಸಾರ್ವತ್ರಿಕ ಉದಾರಭಾವಗಳನ್ನು ಅಂಗೀಕರಿಸುವುದು. ಈಗ ಆಗಿರುವುದು ಕೇವಲ ಆರಂಭ ಮಾತ್ರ. ಈ ಪ್ರವಾಹವು ಮುಂದೆ ಎಲ್ಲರನ್ನೂ ಕೊಚ್ಚಿಕೊಂಡು ಹೋಗುವುದು.

ಶಿಷ್ಯ: ದಯವಿಟ್ಟು ನನಗೆ ಶ‍್ರೀರಾಮಕೃಷ್ಣರ ವಿಷಯವಾಗಿ ಇನ್ನೂ ಹೇಳಿ, ನಿಮ್ಮ ಬಾಯಿಂದ ಅವರ ವಿಷಯ ಕೇಳಲು ನನಗೆ ತುಂಬಾ ಇಷ್ಟ.

ಸ್ವಾಮೀಜಿ: ನೀನು ಯಾವಾಗ ನೋಡಿದರೂ ಅವರ ವಿಷಯವಾಗಿ ಎಷ್ಟೊಂದು ವಿಚಾರವನ್ನು ಕೇಳುತ್ತಿರುವೆಯಲ್ಲವೆ? ಇನ್ನೇನು ಬೇಕು? ಅವರನ್ನು ಅವರೇ ಸರಿಗಟ್ಟಬೇಕು. ಅವರನ್ನು ಮೀರಿಸಿದವರು ಯಾರಾದರೂ ಇರುವರೆ?

ಶಿಷ್ಯ: ಅವರನ್ನು ನೋಡುವ ಭಾಗ್ಯವಿಲ್ಲದ ನಮಗೆ ಯಾವ ಹಾದಿ ಇದೆ?

ಸ್ವಾಮೀಜಿ: ಪ್ರತ್ಯಕ್ಷವಾಗಿ ಅವರ ಕೃಪೆಗೆ ಪಾತ್ರರಾಗಿರುವ ಈ ಸಾಧುಗಳ ಸಹವಾಸ ಭಾಗ್ಯ ನಿನಗೆ ಲಭಿಸಿದೆ. ಅಂದಮೇಲೆ ನೀನು ಹೇಗೆ ಅವರನ್ನು ನೋಡಿಲ್ಲವೆಂದು ಹೇಳುವೆ? ಅವರು ತಮ್ಮ ಸಂನ್ಯಾಸಿ ಶಿಷ್ಯರಲ್ಲೇ ಪ್ರತ್ಯಕ್ಷವಾಗಿದ್ದಾರೆ. ಅವರ ಸೇವೆಮಾಡುವುದರ ಮೂಲಕ, ಸತ್ಕಾಲದಲ್ಲಿ ನಿನ್ನ ಹೃದಯದಲ್ಲೇ ಅವರು ಪ್ರತ್ಯಕ್ಷರಾಗುವರು. ಯೋಗ್ಯಕಾಲದಲ್ಲಿ ನೀನು ಎಲ್ಲವನ್ನೂ ಸಾಧಿಸುವೆ.

ಶಿಷ್ಯ: ಆದರೆ ಸ್ವಾಮೀಜಿ, ನೀವು ಅವರ ಕೃಪೆಗೆ ಪಾತ್ರರಾದವರ ವಿಚಾರವನ್ನೇ ಮಾತನಾಡುತ್ತಿರುವಿರಿ. ಆದರೆ ಎಂದೂ ಅವರು ನಿಮ್ಮ ವಿಚಾರ ಏನು ಹೇಳುತ್ತಿದ್ದರೆಂಬುದನ್ನು ಹೇಳುವುದೇ ಇಲ್ಲ.

ಸ್ವಾಮೀಜಿ: ನನ್ನ ವಿಚಾರ ನಾನು ಏನು ಹೇಳಲಿ? ನೋಡು ಮಗು, ನಾನು ಅವರ ಭೂತಗಳಲ್ಲೊಂದಾಗಿರಬೇಕು. ಪ್ರತ್ಯಕ್ಷ ಅವರ ಎದುರಿನಲ್ಲಿಯೇ ನಾನದೆಷ್ಟೋ ಬಾರಿ ಅವರನ್ನು ನಿಂದಿಸುತ್ತಿದ್ದೆ. ಅದನ್ನು ಕೇಳಿ ಅವರು ನಗುತ್ತಿದ್ದರು.

ಹೀಗೆ ಹೇಳುತ್ತಾ ಸ್ವಾಮಿಗಳು ಗಂಭೀರ ಯೋಚನೆಯಲ್ಲಿ ಮಗ್ನರಾಗಿ, ನದಿಯ ಕಡೆ ನೋಡುತ್ತಾ ಕೊಂಚ ಹೊತ್ತು ಸ್ಥಿರವಾಗಿ ಕುಳಿತುಕೊಂಡಿದ್ದರು. ಕೊಂಚ ಹೊತ್ತಿನಲ್ಲಿಯೇ ಸಾಯಂಕಾಲವಾಗಿ ದೋಣಿಯೂ ಮಠವನ್ನು ತಲುಪಿತು. ಸ್ವಾಮೀಜಿ ಮೆಲ್ಲ ಮೆಲ್ಲಗೆ ಹಾಡನ್ನು ಹಾಡಿಕೊಳ್ಳುತ್ತಿದ್ದರು: “ಈಗ ಜೀವನದ ಸಂಜೆಯಲ್ಲಿ, ಮಗುವನ್ನು ಮನೆಗೆ ಹಿಂತಿರುಗಿ ಕರೆದುಕೊ."

ಹಾಡು ಮುಗಿದಮೇಲೆ ಸ್ವಾಮೀಜಿ “ನಿಮ್ಮ ಕಡೆಯಲ್ಲಿ (ಪೂರ್ವಬಂಗಾಳ) ಮಂಜುಳ ಧ್ವನಿಯಿಂದ ಹಾಡುವವರು ಹುಟ್ಟೇ ಇಲ್ಲ. ಗಂಗಾಮಾತೆಯ ಜಲವನ್ನು ಪಾನಮಾಡಿದಲ್ಲದೆ ಸುಮಧುರಕಂಠ ಬರುವುದಿಲ್ಲ" ಎಂದು ಹೇಳಿದರು. ಬಾಡಿಗೆಯನ್ನು ಕೊಟ್ಟಾದಮೇಲೆ ಸ್ವಾಮೀಜಿ ದೋಣಿಯಿಂದಿಳಿದು ತಮ್ಮ ಕೋಟನ್ನು ಬಿಚ್ಚಿ ಮಠದ ಪಶ್ಚಿಮ ವರಾಂಡದಲ್ಲಿ ಕುಳಿತುಕೊಂಡರು. ಅವರ ಗೌರವರ್ಣ ಜೊತೆಗೆ ಕಾವಿಬಟ್ಟೆ ಒಂದು ಅಪೂರ್ವ ಸೊಬಗನ್ನು ಬೀರುತ್ತಿತ್ತು.

\newpage

\chapter[ಅಧ್ಯಾಯ ೪೬]{ಅಧ್ಯಾಯ ೪೬\protect\footnote{\engfoot{C.W, Vol. VII, P. 265}}}

\begin{center}
ಸ್ಥಳ: ಬೇಲೂರು ಮಠ, ವರ್ಷ: ಕ್ರಿ.ಶ. ೧೯೦೨.
\end{center}

ಇಂದು ಆಷಾಢದ (ಜೂನ್-ಜುಲೈ) ಮೊದಲ ದಿನ. ಶಿಷ್ಯನು ಸಂಜೆಗೆ ಮುಂಚೆಯೇ ಉಡುಪನ್ನು ಬದಲಾಯಿಸಲು ವೇಳೆಯಿಲ್ಲದ್ದರಿಂದ ಬಾಲಿಯಿಂದ ಕಛೇರಿಯ ಉಡುಪಿನಲ್ಲಿಯೇ ಮಠಕ್ಕೆ ಬಂದಿದ್ದ. ಸ್ವಾಮೀಜಿಗೆ ಸಾಷ್ಟಾಂಗ ಪ್ರಣಾಮಮಾಡಿ ಅವರ ಆರೋಗ್ಯದ ವಿಷಯವಾಗಿ ಪ್ರಶ್ನಿಸಿದ. ಸ್ವಾಮೀಜಿ ತಾವು ಆರೋಗ್ಯವೆಂದು ಹೇಳಿ ಶಿಷ್ಯನ ಉಡುಪನ್ನು ನೋಡುತ್ತ “ನೀನು ಕೋಟು ಷರಾಯಿ ಹಾಕಿಕೊಂಡಿರುವೆ. ಕತ್ತಿನ ಪಟ್ಟಿಯನ್ನೇಕೆ ಹಾಕಿಕೊಂಡಿಲ್ಲ?” ಎಂದು ಕೇಳಿ ಹತ್ತಿರವಿದ್ದ ಶಾರದಾನಂದರನ್ನು ಕರೆದು “ನಾಳೆ ನನ್ನ ಪೆಟ್ಟಿಗೆಯಿಂದ ಎರಡು ಕತ್ತಿನ ಪಟ್ಟಿಯನ್ನು ಇವನಿಗೆ ತೆಗೆದುಕೊಡು" ಎಂದರು. ಅವರ ಅಪ್ಪಣೆಗೆ ಶಾರದಾನಂದರು ಒಪ್ಪಿಕೊಂಡರು.

ನಂತರ ಶಿಷ್ಯ ತನ್ನ ಉಡುಪನ್ನು ಬದಲಾಯಿಸಿಕೊಂಡು ಸ್ವಾಮಿಗಳ ಹತ್ತಿರ ಬಂದಾಗ ಸ್ವಾಮಿಗಳು ಅವನನ್ನುದ್ದೇಶಿಸಿ: “ನಮ್ಮ ದೇಶೀಯ ಉಡುಪು, ಆಹಾರ ಮತ್ತು ಜೀವನದ ಕ್ರಮವನ್ನು ಬಿಡುವುದರಿಂದ ನಾವು ನಮ್ಮ ಜನಾಂಗದ ಧರ್ಮವನ್ನೇ ಕಳೆದುಕೊಳ್ಳುವೆವು. ನಾವು ಎಲ್ಲರಿಂದಲೂ ಕಲಿತುಕೊಳ್ಳಬಹುದು. ಆದರೆ ಯಾವುದನ್ನು ಕಲಿಯುವುದರಿಂದ ನಮ್ಮ ರಾಷ್ಟ್ರದ ವೈಶಿಷ್ಟ್ಯವನ್ನೇ ಕಳೆದುಕೊಳ್ಳುವೆವೊ ಅಂತಹುದರಿಂದ ನಮ್ಮ ಏಳಿಗೆಗೇನೂ ಸಹಾಯವಾಗುವುದಿಲ್ಲ; ಅಲ್ಲದೆ ನಾವೂ ಅಧೋಗತಿಗಿಳಿಯುವೆವು.”

ಶಿಷ್ಯ: ಸ್ವಾಮಿಜಿ, ಮೇಲಿನ ಐರೋಪ್ಯ ಅಧಿಕಾರವರ್ಗ ಒಪ್ಪುವ ಉಡುಪನ್ನು ನಾವು ಧರಿಸದೆ ವಿಧಿಯಿಲ್ಲ.

ಸ್ವಾಮೀಜಿ: ಯಾರೂ ಅದಕ್ಕೆ ಅಡ್ಡಿ ಬರುವುದಿಲ್ಲ. ನಿನ್ನ ಕೆಲಸದ ಹಿತಕ್ಕಾಗಿ ನೀನು ಆಫೀಸಿನಲ್ಲಿ ಆ ಬಗೆಯ ಉಡುಪನ್ನೇ ಧರಿಸುವೆ. ಆದರೆ ಮನೆಗೆ ಹಿಂತಿರುಗಿದಾಗ, ಹರಿದಾಡುವ ವಸ್ತ್ರ, ದೇಶೀಯ ಜುಬ್ಬ, ಹೆಗಲಮೇಲಿನ ಉತ್ತರೀಯ ಇವುಗಳನ್ನು ಧರಿಸಿದ ಅಚ್ಚ ಬಂಗಾಳಿ ಬಾಬುವೇ ಆಗಿರಬೇಕು. ನಿನಗೆ ತಿಳಿಯಿತೆ?

ಶಿಷ್ಯ: ತಿಳಿಯಿತು ಸ್ವಾಮೀಜಿ.

ಸ್ವಾಮೀಜಿ: ನೀವು ಮನೆಯಿಂದ ಮನೆಗೆ ಷರ್ಟನ್ನು ಧರಿಸಿಹೋಗುವಿರಿ, ಪಾಶ್ಚಾತ್ಯ ದೇಶದಲ್ಲಿ ಹೀಗೆ ಬರೀ ಷರ್ಟು ಧರಿಸಿ ಜನರನ್ನು ಭೇಟಿಮಾಡಲು ಹೋದರೆ ಅದು ಸಭ್ಯ ಮನುಷ್ಯನ ವರ್ತನೆಯಲ್ಲ. ಹೀಗೆ ಹೋದರೆ ಅಲ್ಲಿ ನಗ್ನರಾಗಿ ಹೋದಂತೆ. ಷರ್ಟಿನ ಮೇಲೆ ಕೋಟನ್ನು ಹಾಕಿಕೊಂಡು ಹೋದ ಹೊರತು ನಿಮಗೆ ಯಾವ ಸಭ್ಯ ಗೃಹಸ್ಥನ ಮನೆಯಲ್ಲೂ ಸ್ವಾಗತ ದೊರಕಲಾರದು. ಈ ಉಡುಪಿನ ವಿಷಯ ಅನುಕರಿಸಹೋಗಿ ನೀವೆಂತಹ ಅಸಂಬದ್ಧತೆಯನ್ನು ಕಲಿತಿರುವಿರಿ! ಈಗಿನ ಕಾಲದ ಯುವಕರು ಧರಿಸುವ ಉಡುಪು ಎಷ್ಟು ವಿಚಿತ್ರವಾಗಿದೆಯೆಂದರೆ ಅದು ಪಾಶ್ಚಾತ್ಯರ ಉಡುಪನ್ನೂ ಹೋಲುವುದಿಲ್ಲ, ಹಿಂದೂ ದೇಶೀಯರ ಉಡುಪನ್ನೂ ಹೋಲುವುದಿಲ್ಲ - ಅದೊಂದು ವಿಲಕ್ಷಣ ಸಂಯೋಗವಾಗಿದೆ.

ಹೀಗೆ ಮಾತನಾಡಿದ ಮೇಲೆ ಸ್ವಾಮೀಜಿ ನದೀತೀರದಲ್ಲಿ ಅಡ್ಡಾಡತೊಡಗಿದರು. ಶಿಷ್ಯನೊಬ್ಬನೇ ಅವರ ಜೊತೆಯಲ್ಲಿದ್ದನು. ಸ್ವಾಮೀಜಿಗಳನ್ನು ಆಧ್ಯಾತ್ಮಿಕ ಸಾಧನೆಯ ವಿಷಯವಾಗಿ ಏನನ್ನೊ ಕೇಳಬೇಕೆಂದು ಅನುಮಾನಪಡುತ್ತಿದ್ದನು.

ಸ್ವಾಮೀಜಿ: ಏನನ್ನು ಯೋಚಿಸುತ್ತಿರುವೆ? ಅದನ್ನು ಹೊರಗೆಡಹು.

ಶಿಷ್ಯ ತುಂಬಾ ಸಂಕೋಚದಿಂದ ಹೇಳಿದ: “ಸ್ವಾಮೀಜಿ, ಮನಸ್ಸು ಕೊಂಚ ಹೊತ್ತಿನಲ್ಲಿಯೇ ಏಕಾಗ್ರತೆ ಹೊಂದಿ ಅದರಿಂದ ನಾನು ಬೇಗ ಧ್ಯಾನದಲ್ಲಿ ಮುಳುಗುವಂತಹ ಯಾವುದಾದರೂ ರೀತಿಯನ್ನು ನೀವು ನನಗೆ ಹೇಳಿಕೊಟ್ಟರೆ ನನಗೆ ತುಂಬಾ ಉಪಕಾರವಾಗುವುದು. ನನ್ನ ಪ್ರಾಪಂಚಿಕ ಕರ್ತವ್ಯಗಳ ಮಧ್ಯೆ ಮನಸ್ಸನ್ನು ಸಾಧನೆಯಲ್ಲಿ ಸ್ಥಿರಗೊಳಿಸಲು ನನಗೆ ತುಂಬ ಕಷ್ಟವಾಗುವುದು.”

ಸ್ವಾಮಿಗಳಿಗೆ ಶಿಷ್ಯನ ನಮ್ರಭಾವನೆಯನ್ನೂ ಆತನ ಆಸಕ್ತಿಯನ್ನೂ ನೋಡಿ ಹರ್ಷವುಂಟಾಯಿತು. ಅವರು ಪ್ರೇಮದಿಂದ ಹೇಳಿದರು: “ಕೊಂಚ ಹೊತ್ತಿನ ಮೇಲೆ ನಾನು ಮಹಡಿಯಮೇಲೆ ಒಬ್ಬನೇ ಇದ್ದಾಗ ಬಾ, ನಾನು ಈ ವಿಚಾರ ನಿನ್ನೊಡನೆ ಮಾತನಾಡುವೆ."

ಕೊಂಚ ಹೊತ್ತಿನ ಮೇಲೆ ಮಹಡಿಗೆ ಬಂದಾಗ ಸ್ವಾಮಿಗಳು ಪಶ್ಚಿಮ ಮುಖವಾಗಿ ಕುಳಿತು ಧ್ಯಾನನಿರತರಾಗಿದ್ದುದನ್ನು ಶಿಷ್ಯನು ನೋಡಿದನು. ಅವರ ಮುಖದಲ್ಲಿ ಒಂದು ಅದ್ಭುತ ತೇಜಸ್ಸು ಇತ್ತು. ಅವರ ಇಡೀ ದೇಹ ಸಂಪೂರ್ಣವಾಗಿ ನಿಶ್ಚಲವಾಗಿತ್ತು. ಶಿಷ್ಯನು ಧ್ಯಾನದಲ್ಲಿದ್ದ ಸ್ವಾಮಿಗಳ ಈ ಸ್ಥಿತಿಯನ್ನು ನೋಡಿ ಸ್ಥಂಭೀಭೂತನಾಗಿ ನಿಂತುಬಿಟ್ಟನು. ಬಹುಹೊತ್ತು ಈ ರೀತಿ ಇದ್ದರೂ ಅವರಲ್ಲಿ ಬಾಹ್ಯಪ್ರಜ್ಞೆ ಕಾಣದಿರಲು ಶಿಷ್ಯನೂ ಹತ್ತಿರ ನಿಶ್ಶಬ್ದವಾಗಿ ಕುಳಿತುಕೊಂಡ. ಅರ್ಧ ಘಂಟೆಯಾದ ಮೇಲೆ ಸ್ವಾಮಿಗಳು ಬಾಹ್ಯಪ್ರಜ್ಞೆತಾಳುವ ಲಕ್ಷಣಗಳು ಕಂಡುಬಂದವು. ಅವರ ಜೋಡಿಸಿದ ಕೈಗಳು ಸಡಿಲವಾಗಿ ಕೆಲವು ನಿಮಿಷಗಳಲ್ಲಿಯೇ ಅವರು ಕಣ್ತೆರೆದು ಶಿಷ್ಯನನ್ನು ನೋಡಿ “ನೀನು ಯಾವಾಗ ಬಂದೆ?" ಎಂದು ಕೇಳಿದರು.

ಶಿಷ್ಯ: ಈಗ ಕೊಂಚ ಹೊತ್ತಾಯಿತು.

ಸ್ವಾಮೀಜಿ: ಒಳ್ಳೆಯದು ನನಗೊಂದು ಲೋಟ ನೀರನ್ನು ತಾ.

ಶಿಷ್ಯ: ಬೇಗನೆ ನೀರನ್ನು ತಂದುಕೊಟ್ಟ. ಸ್ವಾಮಿಗಳು ಕೊಂಚ ಕುಡಿದು ಲೋಟವನ್ನು ಸರಿಯಾದ ಸ್ಥಳದಲ್ಲಿಡಲು ಶಿಷ್ಯನಿಗೆ ಹೇಳಿದರು. ಶಿಷ್ಯನು ಇಟ್ಟುಬಂದು ಸ್ವಾಮಿಗಳ ಬಳಿ ಕುಳಿತುಕೊಂಡ.

ಸ್ವಾಮಿಜಿ: ಈ ದಿನ ನಾನು ಗಾಢವಾದ ಧ್ಯಾನದಲ್ಲಿದ್ದೆ.

ಶಿಷ್ಯ: ದಯವಿಟ್ಟು ನನಗೂ ನನ್ನ ಮನಸ್ಸು ಧ್ಯಾನದಲ್ಲಿ ಮುಳುಗುವಂತೆ ಹೇಳಿಕೊಡಿ.

ಸ್ವಾಮೀಜಿ: ನಾನಾಗಲೇ ನಿನಗೆ ಎಲ್ಲಾ ರೀತಿಯನ್ನೂ ಹೇಳಿರುವೆ. ಅದರಂತೆ ನಿತ್ಯವೂ ಧ್ಯಾನಿಸು. ಯೋಗ್ಯ ಕಾಲದಲ್ಲಿ ನಿನಗೂ ಹಾಗೇ ಅನುಭವವಾಗುವುದು. ಈಗ ಹೇಳು ನಿನಗೆ ಯಾವ ಬಗೆಯ ಸಾಧನೆ ಇಷ್ಟ?

ಶಿಷ್ಯ: ಸ್ವಾಮೀಜಿ, ನಿತ್ಯವೂ ನೀವು ಹೇಳಿದಂತೆಯೇ ಸಾಧನೆ ಮಾಡುವೆ. ಆದರೂ ನನಗೆ ಗಾಢಧ್ಯಾನ ಬರುವುದಿಲ್ಲ. ಕೆಲವು ವೇಳೆ ನಾನು ಈ ರೀತಿ ಧ್ಯಾನ ಮಾಡುವುದು ನಿಷ್ಪಲವೆನಿಸುವುದು. ಆದ್ದರಿಂದ ನಾನು ಇನ್ನು ಅದರಲ್ಲಿ ಮುಂದುವರಿಯಲಾರೆ ಎನ್ನಿಸಿ ನಿಮ್ಮೊಡನೆ ನಿರಂತರ ಸಹವಾಸ ಮಾತ್ರ ಮಾಡಬೇಕೆಂಬ ಆಸೆ ಬಲವಾಗಿದೆ.

ಸ್ವಾಮೀಜಿ: ಇದು ಮನಸ್ಸಿನ ದುರ್ಬಲತೆ. ಯಾವಾಗಲೂ ಶಾಶ್ವತ ಸತ್ಯವಾದ ಆತ್ಮನಲ್ಲಿ ತಲ್ಲೀನವಾಗಲು ಯತ್ನಿಸು. ಒಮ್ಮೆ ನಿನಗೆ ಆತನ ದರ್ಶನವಾದರೆ ಎಲ್ಲವೂ ಲಭಿಸುವುದು. ಜನನಮರಣ ಬಂಧನದ ಕಟ್ಟು ಕಳಚಿಹೋಗುವುದು.

ಶಿಷ್ಯ: ನೀವು ನಾನದನ್ನು ಪಡೆಯುವಂತೆ ಆಶೀರ್ವದಿಸಿ. ನೀವು ನಾನಿಂದು ಒಬ್ಬನೇ ಬರುವಂತೆ ಹೇಳಿದಿರಿ. ಅದರಂತೆ ಬಂದಿರುವೆ. ಹೇಗಾದರೂ ಮಾಡಿ ನನ್ನ ಮನಸ್ಸು ಏಕಾಗ್ರತೆ ಹೊಂದುವಂತೆ ಮಾಡಿ.

ಸ್ವಾಮೀಜಿ: ನಿನಗೆ ಯಾವಾಗ ಕಾಲ ಸಿಗುವುದೋ ಆಗೆಲ್ಲಾ ಧ್ಯಾನ ಮಾಡು. ಮನಸ್ಸು ಒಮ್ಮೆ ಸುಷುಮ್ನಾ ಹಾದಿಯನ್ನು ಪ್ರವೇಶಿಸಿದರೆ ಸಾಕು. ಎಲ್ಲವೂ ಸರಿಯಾಗುವುದು. ಅದಾದನಂತರ ನೀನು ಮಾಡಬೇಕಾದ್ದು ಹೆಚ್ಚೇನು ಇರುವುದಿಲ್ಲ.

ಶಿಷ್ಯ: ನೀವು ಅನೇಕ ವಿಧವಾಗಿ ನನ್ನನ್ನು ಪ್ರೋತ್ಸಾಹಗೊಳಿಸುತ್ತೀರಿ. ನಿಜವಾಗಿಯೂ ನನಗೆ ಸತ್ಯಸಾಕ್ಷಾತ್ಕಾರವಾಗುವ ಲಭ್ಯವಿದೆಯೇ? ಸತ್ಯ - ಜ್ಞಾನ ಪಡೆದು ನಾನು ಮುಕ್ತನಾಗುವೆನೆ?~।

ಸ್ವಾಮೀಜಿ: ಹೌದು, ಖಂಡಿತವಾಗಿಯೂ ಹೌದು. ಕೀಟದಿಂದ ಹಿಡಿದು ಬ್ರಹ್ಮನವರೆಗೆ ಎಲ್ಲರೂ ಮುಕ್ತಿಯನ್ನು ಪಡೆಯುವರು. ನೀನೊಬ್ಬ ಆಗುವುದಿಲ್ಲವೇನು? ಇವೆಲ್ಲಾ ಮನಸ್ಸಿನ ದುರ್ಬಲತೆ. ಇಂಥ ವಿಷಯಗಳನ್ನು ಎಂದೂ ಯೋಚಿಸಬೇಡ.

ಅನಂತರ ಅವರು ಪುನಃ ಹೇಳಿದರು: ಶ್ರದ್ಧೆ, ಧೈರ್ಯವನ್ನು ಪಡೆ. ಆತ್ಮ ಜ್ಞಾನವನ್ನು ಹೊಂದು. ಇತರರ ಕಲ್ಯಾಣಕ್ಕಾಗಿ ನಿನ್ನ ಜೀವನವನ್ನು ಬಲಿದಾನಮಾಡು - ಇದೇ ನನ್ನ ಅಭೀಷ್ಟ ಮತ್ತು ಆಶೀರ್ವಾದ.

ಇದೇ ವೇಳೆಗೆ ಸರಿಯಾಗಿ ಊಟದ ಗಂಟೆ ಬಾರಿಸಲು ಸ್ವಾಮೀಜಿ ಶಿಷ್ಯನಿಗೆ ಅಲ್ಲೇ ಊಟ ಮಾಡಲು ಹೇಳಿದರು. ಸ್ವಾಮಿಗಳ ಪಾದಕ್ಕೆ ಸಾಷ್ಟಾಂಗ ಪ್ರಣಾಮ ಮಾಡುತ್ತಾ ಶಿಷ್ಯ ಅವರ ಆಶೀರ್ವಾದಕ್ಕಾಗಿ ಪ್ರಾರ್ಥಿಸಿದನು. ಶಿಷ್ಯನ ಶಿರದ ಮೇಲೆ ತಮ್ಮ ಹಸ್ತವನ್ನಿರಿಸಿ ಸ್ವಾಮೀಜಿ ಹರಸಿದರು. ನಂತರ ಹೇಳಿದರು: “ನನ್ನ ಆಶೀರ್ವಾದದಿಂದ ನಿನಗೇನಾದರೂ ಒಳ್ಳೆಯದಾಗುವ ಹಾಗಿದ್ದರೆ - ನಾನು ಹೇಳುವೆ - ಭಗವಾನ್ ಶ‍್ರೀರಾಮಕೃಷ್ಣರು ನಿನಗೆ ಕೃಪೆ ಮಾಡಲಿ. ಇದಕ್ಕಿಂತ ಹೆಚ್ಚಾದ ಆಶೀರ್ವಾದ ನನಗೆ ಗೊತಿಲ." ಊಟವಾದ ಮೇಲೆ ಸ್ವಾಮೀಜಿ ಬೇಗ ಮಲಗಿದುದರಿಂದ, ಶಿಷ್ಯನು ಮಹಡಿಗೆ ಪುನಃ ಹೋಗಲಿಲ್ಲ. ಮಾರನೆಯ ದಿನ ಬೆಳಿಗ್ಗೆ ಶಿಷ್ಯನು ಕಲ್ಕತ್ತೆಗೆ ತನ್ನ ಕೆಲಸದ ನಿಮಿತ್ತ ಹೋಗಬೇಕಾಗಿದ್ದುದರಿಂದ ಮಹಡಿಗೆ ಸ್ವಾಮಿಗಳನ್ನು ಕಾಣಲು ಹೊದ.

ಸ್ವಾಮೀಜಿ: ನೀನು ಈಗಲೇ ಹೊರಡುವೆಯಾ?

ಶಿಷ್ಯ: ಹೌದು ಸ್ವಾಮಿಜಿ.

ಸ್ವಾಮೀಜಿ: ಮುಂದಿನ ಭಾನುವಾರ ಮತ್ತೊಮ್ಮೆ ಬಾ, ಬರುವೆಯಾ?

ಶಿಷ್ಯ: ಖಂಡಿತ ಬರುವೆ ಸ್ವಾಮೀಜಿ.

ಸ್ವಾಮೀಜಿ: ಸರಿ ಹಾಗಾದರೆ, ಅದೋ ಒಂದು ದೋಣಿ ಬರುತ್ತಿದೆ.

ಶಿಷ್ಯನು ಸ್ವಾಮಿಗಳನ್ನು ಬೀಳ್ಕೊಟ್ಟ. ಇದೇ ತನ್ನ ಇಷ್ಟದೇವರ ಭೌತಿಕ ಶರೀರದ ಅಂತಿಮ ದರ್ಶನವೆಂದು ಶಿಷ್ಯನಿಗೆ ಆಗ ಗೊತ್ತಾಗಲಿಲ್ಲ. ಸ್ವಾಮಿಗಳು ಹರುಷಚಿತ್ತದಿಂದ ಶಿಷ್ಯನನ್ನು ಬೀಳ್ಕೊಟ್ಟರು. “ಭಾನುವಾರ ಬಾ" ಎಂದರು. ಶಿಷ್ಯನೂ “ಬರುತ್ತೇನೆ" ಎಂದು ಕೆಳಗೆ ಬಂದ.

ದೋಣಿಯ ನಾವಿಕರು ಶಿಷ್ಯನನ್ನು ಕೂಗುತ್ತಿದ್ದರು. ಆದ್ದರಿಂದ ಅವನು ದೋಣಿಗೆ ಓಡಿಹೋದ. ದೋಣಿಯನ್ನು ಹತ್ತುವಾಗ ಸ್ವಾಮಿಗಳು ಮೇಲಿನ ವರಾಂಡದಲ್ಲಿ ಶತಪಥ ಸುತ್ತುತ್ತಿರುವುದನ್ನು ಶಿಷ್ಯನು ನೋಡಿದ. ಅವರನ್ನು ನಮಸ್ಕರಿಸಿ ದೋಣಿಯನ್ನು ಪ್ರವೇಶಿಸಿದ.

ಇದಾದ ಒಂದು ವಾರದ ಮೇಲೆ ಸ್ವಾಮಿಗಳು ಮರ್ತ್ಯಜೀವನದಿಂದ ಪಾರಾದರು. ಶಿಷ್ಯನಿಗೆ ಈ ಮುಂಬರುವ ಆಘಾತದ ಸುಳಿವು ಕೊಂಚವೂ ಇರಲಿಲ್ಲ. ಸ್ವಾಮಿಗಳು ಕಾಲವಾದ ಮಾರನೆಯ ದಿನ ಈ ಸಂಗತಿ ಕೇಳಿದ ಶಿಷ್ಯ ಮಠಕ್ಕೆ ಬಂದ. ಪುನಃ ಸ್ವಾಮಿಗಳ ಭೌತಿಕ ಶರೀರವನ್ನು ಕಾಣುವ ಸುಯೋಗವೂ ಅವನ ಹಣೆಯಲ್ಲಿ ಬರೆದಿರಲಿಲ್ಲ.

