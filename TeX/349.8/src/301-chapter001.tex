
\chapter[ಅಧ್ಯಾಯ ೧]{ಅಧ್ಯಾಯ ೧\protect\footnote{\engfoot{C.W, Vol. V, P. 332}}}

\begin{center}
ವರ್ಷ: ಕ್ರಿ.ಶ. ೧೮೯೮ನೇ ಜನವರಿ ೨೨.
\end{center}

ಕಲ್ಕತ್ತೆಯ ರಮಾಕಾಂತ ಬೋಸ್ ಬೀದಿಯಲ್ಲಿ ೫೭ನೇ ನಂಬರ್‌ ಮನೆಯ ಬಲರಾಮಬಾಬುಗಳ ಮನೆಯಲ್ಲಿ ಸ್ವಾಮಿಗಳು ಇದ್ದಾಗ ನಾನು ಒಂದು ದಿನ ಬೆಳಿಗ್ಗೆ ಹೊತ್ತಿಗೆ ಮುಂಚೆಯೇ ಸ್ವಾಮಿಗಳಲ್ಲಿಗೆ ಹೋದೆ. ಕೊಠಡಿಯು ಕೇಳುವವರ ಸಂಖ್ಯೆಯಿಂದ ಭರ್ತಿಯಾಗಿತ್ತು. ಸ್ವಾಮಿಗಳು ಹೀಗೆ ಹೇಳುತ್ತಿದ್ದರು; “ನಮಗೆ ಶ್ರದ್ಧೆ ಬೇಕು, ನಮ್ಮಲ್ಲಿ ನಮಗೆ ನಂಬಿಕೆ ಇರಬೇಕು. ಶಕ್ತಿಯೇ ಜೀವನ, ದುರ್ಬಲತೆಯೇ ಮರಣ. ನಾವೆಲ್ಲಾ ಆತ್ಮ, ಅಜೇಯರು, ಮುಕ್ತರು, ಪವಿತ್ರರು, ಸ್ವಭಾವತಃ ಪರಿಶುದ್ಧರು. ನಾವು ಎಂದಾದರೂ ಪಾಪ ಮಾಡಲು ಸಾಧ್ಯವೆ? ಖಂಡಿತ ಇಲ್ಲ. ಇಂತಹ ಶ್ರದ್ಧೆ ಆವಶ್ಯಕ. ಇಂತಹ ಶ್ರದ್ಧೆ ನಮ್ಮನ್ನು ಮಾನವರನ್ನಾಗಿ ಮಾಡುವುದು. ನಮ್ಮನ್ನು ದೇವತೆಗಳನ್ನಾಗಿ ಮಾಡುವುದು. ಈ ಶ್ರದ್ಧೆಯ ಭಾವನೆಯನ್ನು ಕಳೆದುಕೊಂಡಿದ್ದರಿಂದಲೇ ದೇಶ ಅಧೋಗತಿಗಿಳಿದಿದೆ.”

ಪ್ರಶ್ನೆ: ನಾವು ಹೇಗೆ ಈ ಶ್ರದ್ಧೆಯನ್ನು ಕಳೆದುಕೊಂಡೆವು?

ಸ್ವಾಮೀಜಿ: ನಮ್ಮ ಬಾಲ್ಯದಿಂದಲೇ ನಿಷೇಧಮಯ ಶಿಕ್ಷಣ ಹೊಂದಿದ್ದೇವೆ. ನಾವು ಕೇವಲ ಅನಾಮಧೇಯರು ಎಂಬುದನ್ನು ಕಲಿತಿದ್ದೇವೆ. ನಮ್ಮ ದೇಶದಲ್ಲಿ ಮಹಾತ್ಮರು ಎಂದಾದರು ಉದಿಸಿದ್ದರೆಂದು ಆಲಿಸುವುದು ಕೂಡ ಅತಿ ವಿರಳ. ಯಾವ ವಾಸ್ತವಿಕತೆಯನ್ನೂ ನಮಗೆ ಕಲಿಸಿಲ್ಲ. ನಿಮ್ಮ ಕೈಕಾಲುಗಳನ್ನು ಹೇಗೆ ಉಪಯೋಗಿಸಬೇಕೆಂಬುದೂ ನಿಮಗೆ ಗೊತ್ತಿಲ್ಲ. ಆಂಗ್ಲೇಯ ಪೂರ್ವಿಕರ ಬಗ್ಗೆ ಅಂಕಿ ಅಂಶಗಳನ್ನು ತಿಳಿದು ನಾವು ಪ್ರವೀಣರಾಗುವೆವು. ಆದರೆ ನಮ್ಮ ಪೂರ್ವಿಕರದ್ದೇ ನಮಗೆ ತಿಳಿಯದಿರುವುದು ಶೋಚನೀಯವಾಗಿದೆ. ನಾವು ಕೇವಲ ದುರ್ಬಲತೆಯನ್ನು ಕಲಿತಿದ್ದೇವೆ. ನಾವು ಪರಾಜಿತರಾದುದರಿಂದ ನಾವು ದುರ್ಬಲರು, ನಮಗೆ ಯಾವುದರಲ್ಲಿಯೂ ಸ್ವಾತಂತ್ರ್ಯವಿಲ್ಲ ಎಂದು ನಾವು ನಂಬುವಂತೆ ಮಾಡಿಕೊಂಡಿದ್ದೇವೆ. ಹೀಗಾದಮೇಲೆ ಇನ್ನು ಶ್ರದ್ಧೆಯನ್ನು ಕಳೆದುಕೊಳ್ಳದೆ ಮತ್ತೇನು? ನಿಜವಾದ ಶ್ರದ್ಧೆಯ ಮನೋಭಾವನೆ ನಮಗೆ ಮತ್ತೊಮ್ಮೆ ಹಿಂತಿರುಗಿ ಬರುವಂತೆ ಮಾಡಬೇಕು. ನಮ್ಮ ಸ್ವಂತ ಆತ್ಮಶಕ್ತಿಯಲ್ಲಿ ನಮಗೆ ನಂಬಿಕೆ ಮತ್ತೊಮ್ಮೆ ಜಾಗೃತವಾಗಬೇಕು. ಆಗ ಮಾತ್ರ ದೇಶದೆದುರಿಗಿರುವ ಸಮಸ್ಯೆಗಳೆಲ್ಲಾ ಕ್ರಮೇಣ ನಮ್ಮಿಂದಲೇ ಪರಿಹಾರವಾಗುವುವು.

ಪ್ರಶ್ನೆ: ಅದು ಹೇಗೆ ಸಾಧ್ಯ? ನಮ್ಮ ಸಮಾಜವನ್ನು ಆವರಿಸಿರುವ ಅಸಂಖ್ಯಾತವಾದ ಹಾನಿಗಳೆಲ್ಲಾ ಹೇಗೆ ತಾನೆ ಕೇವಲ ಶ್ರದ್ಧೆಯಿಂದ ಪರಿಹಾರ ವಾಗುವುವು? ಅದೂ ಅಲ್ಲದೆ ದೇಶದಲ್ಲಿ ಅನೇಕ ಪ್ರಬಲವಾದ ದೋಷಗಳಿವೆ. ಅದನ್ನು ತಿಳಿಸಲು ಇಂಡಿಯನ್ ನ್ಯಾಷನಲ್ ಕಾಂಗ್ರೆಸ್ ಮತ್ತು ಇನ್ನೂ ಇತರ ಅನೇಕ ದೇಶಭಕ್ತ ಸಂಸ್ಥೆಗಳು ಬ್ರಿಟಿಷ್ ಸರಕಾರಕ್ಕೆ ಮನವಿಗಳನ್ನು ಕಳುಹಿಸಿ ಸತತವಾದ ಹೋರಾಟ ನಡೆಸುತ್ತಿವೆ. ಅವರ ಬೇಡಿಕೆಗಳನ್ನು ಪ್ರಕಟಿಸಲು ಯಾವುದು ಉತ್ತಮವಾದ ಹಾದಿ? ಇಲ್ಲಿ ಶ್ರದ್ಧೆಗೂ ಈ ವಿಷಯಕ್ಕೂ ಏನು ಸಂಬಂಧ?

ಸ್ವಾಮೀಜಿ: ಹೇಳು, ಅವು ಯಾರ ಬೇಡಿಕೆಗಳು? ನಿಮ್ಮವೋ ಅಥವಾ ನಿಮ್ಮ ಪ್ರಭುಗಳದ್ದೋ? ನಿಮ್ಮದಾಗಿದ್ದರೆ, ನಿಮ್ಮ ರಾಜರು ಅವನ್ನು ಒದಗಿಸುವರೋ ಅಥವಾ ನೀವೇ ಅದನ್ನು ನಿಮಗಾಗಿ ಮಾಡುವಿರೋ.

ಪ್ರಶ್ನೆ: ಆದರೆ ತನ್ನ ಪ್ರಜೆಗಳ ಇಷ್ಟಾರ್ಥವನ್ನು ನೆರವೇರಿಸುವುದು ರಾಜರ ಕರ್ತವ್ಯ. ರಾಜರಿಂದಲ್ಲದೆ ಮತ್ತಾರಿಂದ ನಾವು ಇದನ್ನು ನಿರೀಕ್ಷಿಸಬೇಕು?

ಸ್ವಾಮೀಜಿ: ಭಿಕ್ಷುಕನ ಬೇಡಿಕೆಗಳೊಂದೂ ಈಡೇರುವುದಿಲ್ಲ. ಸರ್ಕಾರ ನಿಮಗೆ ಬೇಕಾದುದನ್ನೆಲ್ಲಾ ಕೊಡುವುದೆಂದಿಟ್ಟುಕೊಳ್ಳೋಣ. ಜರೂರಾದ ಕೆಲಸಗಳನ್ನು ನೋಡಿಕೊಳ್ಳುವವರಾರು? ಆದ್ದರಿಂದ ಮನುಷ್ಯರನ್ನು ಮೊದಲು ನಿರ್ಮಾಣಮಾಡು. ಪೌರುಷವಂತರಾದ ಮನುಷ್ಯರು ನಮಗೀಗ ಬೇಕಾಗಿರುವುದು. ಶ್ರದ್ಧೆಯಿಲ್ಲದಿದ್ದಲ್ಲಿ ಅಂತಹ ಮನುಷ್ಯರನ್ನು ತಯಾರುಮಾಡಲು ಹೇಗೆ ಸಾಧ್ಯ?

ಪ್ರಶ್ನೆ: ಆದರೆ, ಮಹಾಶಯ, ಬಹುಮಂದಿಗೆ ಈ ಅಭಿಪ್ರಾಯವಿರುವುದಿಲ್ಲ.

ಸ್ವಾಮೀಜಿ: ನೀನು ಹೇಳುವ ‘ಬಹುಮಂದಿ’ಯಲ್ಲಿ ಮುಖ್ಯವಾಗಿ ದಡ್ಡರು, ಸಾಧಾರಣ ಬುದ್ಧಿಯುಳ್ಳವರು. ತಮ್ಮಷ್ಟಕ್ಕೆ ತಾವೇ ಯೋಚಿಸುವಂತಹ ಬುದ್ಧಿಯುಳ್ಳವರು ಎಲ್ಲಾ ಕಡೆಯಲ್ಲಿಯೂ ಎಲ್ಲೋ ಅತ್ಯಲ್ಪ ಮಂದಿ ಮಾತ್ರ. ಕೆಲಸದ ಎಲ್ಲಾ ಇಲಾಖೆಗಳಲ್ಲಿಯೂ, ಎಲ್ಲೆಡೆಯಲ್ಲಿಯೂ ಈ ಬುದ್ಧಿವಂತರಾದ ಮನುಷ್ಯರೇ ನಾಯಕರು, ಜನಸಾಧಾರಣರು ದಾರದಂತೆ ಈ ನಾಯಕರು ಎಳೆದ ಕಡೆ ಹೋಗುತ್ತಾರೆ. ಇದೂ ಒಳ್ಳೆಯದೆ. ಏಕೆಂದರೆ ಈ ನಾಯಕರ ಹೆಜ್ಜೆಯನ್ನೇ ಅನುಸರಿಸುವುದರಿಂದ ಎಲ್ಲವೂ ಸರಿಯಾಗಿ ಆಗುವುದು. ಯಾರು ಇನ್ನೊಬ್ಬರಿಗೆ ತಲೆಬಾಗುವುದಕ್ಕಿಂತ ತಾವು ಮೇಲ್ಮಟ್ಟದಲ್ಲಿರುವೆವೆಂದು ಯೋಚಿಸುವರೊ ಅಂಥವರು ಮೂಢರು. ತಮ್ಮ ಅಭಿಪ್ರಾಯದಂತೆಯೇ ನಡೆದು ತಮ್ಮ ಸರ್ವನಾಶಕ್ಕೆ ತಾವೇ ಕಾರಣರಾಗುವರು. ಸಮಾಜ ಸುಧಾರಣೆಯ ವಿಷಯ ಕುರಿತು ನೀನು ಮಾತನಾಡುವೆಯಲ್ಲವೆ! ಆದರೆ ನೀನೇನು ಮಾಡುತ್ತೀಯೆ? ನಿನ್ನ ಸಮಾಜ ಸುಧಾರಣೆ ಎಂಬುದರ ಅರ್ಥವು ವಿಧವಾ ವಿವಾಹ, ಇಲ್ಲವೇ ಸ್ತ್ರೀಯರ ದಾಸ್ಯವಿಮೋಚನೆ ಅಥವಾ ಅಂಥದೇ ಮತ್ತೇನೋ; ಅಷ್ಟೇ ಅಲ್ಲವೆ? ಅಲ್ಲದೆ ಇದೂ ಕೂಡ ಎಲ್ಲೊ ಕೆಲವು ಮತದವರಿಗೆ ಮಾತ್ರ ಅನ್ವಯಿಸುವಂಥದು. ಅಂತಹ ಸುಧಾರಣೆಯ ಯೋಜನೆಯಿಂದ ಕೆಲವರಿಗೆ ಖಂಡಿತವಾಗಿಯೂ ಒಳ್ಳೆಯದಾಗಬಹುದು. ಆದರೆ ಇಡೀ ಜನಾಂಗಕ್ಕೆ ಇದರಿಂದ ಯಾವ ಉಪಯೋಗವಿದೆ? ಇದು ಸುಧಾರಣೆ ಎನ್ನಿಸಿಕೊಳ್ಳುವುದೋ ಅಥವಾ ಸ್ವಾರ್ಥದ ಒಂದು ರೂಪವೊ? ಹೇಗಾದರೂ ಮಾಡಿ ನಿನ್ನ ಕೋಣೆಯನ್ನು ಶುಚಿ ಮಾಡಿದರಾಯಿತು. ಉಳಿದವರ ಪಾಡು ಇನ್ನೂ ಕೆಟ್ಟರೂ ಚಿಂತೆಯಿಲ್ಲ.

ಪ್ರಶ್ನೆ: ಹಾಗಾದರೆ ನೀವು ಹೇಳುವುದು ಸಮಾಜ ಸುಧಾರಣೆಯ ಅಗತ್ಯವೇ ಇಲ್ಲವೆಂದೆ?

ಸ್ವಾಮೀಜಿ: ಯಾರು ಹಾಗೆ ಹೇಳಿದ್ದು? ನಿಸ್ಸಂದೇಹವಾಗಿ ಅದರ ಆವಶ್ಯಕತೆಯಿದೆ. ನೀವು ಮಾತನಾಡುವ ಸಮಾಜ ಸುಧಾರಣೆಗಳಲ್ಲಿ ಮುಕ್ಕಾಲು ಭಾಗ ಬಡಜನ ಮತ್ತು ಸಾಮಾನ್ಯರಿಗೆ ಸಂಬಂಧಪಟ್ಟಿರುವುದಿಲ್ಲ. ಇಲ್ಲಿ ನೀವು ಕೂಗಾಡುತ್ತಿರುವ ವಿಧವಾ ವಿವಾಹ, ಸ್ತ್ರೀ ಸ್ವಾತಂತ್ರ್ಯ ಮುಂತಾದುವುಗಳೆಲ್ಲಾ ಈಗಾಗಲೇ ಅವರಲ್ಲಿ ಬಳಕೆಯಲ್ಲಿವೆ. ಈ ಕಾರಣದಿಂದ ಅವರು ಈ ವಿಷಯಗಳನ್ನೆಲ್ಲಾ ಸುಧಾರಣೆಗಳೆಂದು ತಿಳಿಯುವುದೇ ಇಲ್ಲ. ನಾನೀಗ ಹೇಳುವುದೇನೆಂದರೆ ಶ್ರದ್ಧೆಯ ಅಭಾವವೇ ನಮಗುಂಟಾಗಿರುವ ಹಾನಿಗೆಲ್ಲಾ ಕಾರಣ, ಹಾಗೂ ಇನ್ನೂ ಹಾನಿಯನ್ನುಂಟುಮಾಡುತ್ತಿದೆ ಎಂದು. ನನ್ನ ಚಿಕಿತ್ಸಾಕ್ರಮ ಏನೆಂದರೆ ರೋಗದ ಕಾರಣಗಳನ್ನೆಲ್ಲಾ ಬೇರುಸಹಿತ ಕಿತ್ತೊಗೆಯುವುದು. ಅವುಗಳನ್ನು ಹಾಗೇ ಅಡಗಿಸಿಟ್ಟರೆ ಪ್ರಯೋಜನವಿಲ್ಲ. ನಮಗೆ ಅನೇಕ ವಿಧವಾದ ಸುಧಾರಣೆಗಳು ಬೇಕು. ಇದನ್ನು ನಿರಾಕರಿಸುವಷ್ಟು ದಡ್ಡರಾರಿದ್ದಾರೆ? ಉದಾಹರಣೆಗೆ ಭರತಖಂಡಕ್ಕೆ ಅಂತರ್ಜಾತೀಯ ವಿವಾಹ ಪದ್ಧತಿ ಆವಶ್ಯಕ – ಇದಿಲ್ಲದೆ ಹಿಂದೂ ಜನಾಂಗ ದಿನದಿನಕ್ಕೆ ದುರ್ಬಲವಾಗುತ್ತಿದೆ.

ಅಂದು ಸೂರ್ಯಗ್ರಹಣವಾದುದರಿಂದ ಈ ಪ್ರಶ್ನೆಗಳನ್ನೆಲ್ಲಾ ಕೇಳುತ್ತಿದ್ದ ಸಭ್ಯಗೃಹಸ್ಥನು ಸ್ವಾಮೀಜಿಗೆ ಪ್ರಣಾಮ ಮಾಡಿ, “ನಾನು ಗಂಗಾ ನದಿಗೆ ಸ್ನಾನಕ್ಕೆ ತೆರಳಬೇಕು. ಖಂಡಿತ ನಾನು ಇನ್ನೊಂದು ದಿನ ಬರುವೆ” ಎಂದು ಹೇಳಿ ಹೋದನು.

\newpage

\chapter[ಅಧ್ಯಾಯ ೨]{ಅಧ್ಯಾಯ ೨\protect\footnote{\engfoot{C.W, Vol. V, P. 334}}}

\begin{center}
ವರ್ಷ: ಕ್ರಿ.ಶ. ೧೮೯೮ನೇ ಜನವರಿ ೨೩.
\end{center}

ಸಂಜೆಯಾಗಿತ್ತು. ಬಾಗ್‌ಬಜಾರಿನ ಬಲರಾಮ ಬಾಬುಗಳ ಮನೆಯಲ್ಲಿ ರಾಮಕೃಷ್ಣ ಸಂಸ್ಥೆಯ ವಾರದ ಸಭೆಯ ಸಂದರ್ಭ, ಸ್ವಾಮಿ ತುರೀಯಾನಂದ, ಸ್ವಾಮಿ ಯೋಗಾನಂದ, ಸ್ವಾಮಿ ಪ್ರೇಮಾನಂದ ಮತ್ತಿತರರು ಮಠದಿಂದ ಬಂದಿದ್ದರು. ಸ್ವಾಮೀಜಿ ವರಾಂಡದಲ್ಲಿ ಪೂರ್ವಾಭಿಮುಖವಾಗಿ ಆಸನಾರೂಢರಾಗಿದ್ದರು. ವರಾಂಡದ ಪೂರ್ವ ದಕ್ಷಿಣ ಮತ್ತು ಉತ್ತರ ಭಾಗಗಳೆಲ್ಲಾ ಜನರಿಂದ ಕಿಕ್ಕಿರಿದಿದ್ದಿತು. ಸ್ವಾಮೀಜಿ ಕಲ್ಕತ್ತೆಯಲ್ಲಿದ್ದಾಗ ಪ್ರತಿದಿನವೂ ಹೀಗೆ ಆಗುತ್ತಿತ್ತು.


ಸಭೆಗೆ ಬಂದಿದ್ದವರಲ್ಲಿ ಅನೇಕರಿಗೆ ಸ್ವಾಮೀಜಿ ಬಹು ಚೆನ್ನಾಗಿ ಹಾಡುವರೆಂದು ತಿಳಿದಿದ್ದಿತು. ಅವರಿಂದ ಹಾಡಿಸಬೇಕೆಂದು ಅವರೆಲ್ಲಾ ಅಪೇಕ್ಷಿಸುತ್ತಿದ್ದರು. ಅದನ್ನು ತಿಳಿದು ಮಾಸ್ಟರ್ ಮಹಾಶಯರು ತಮ್ಮ ಬಳಿಯಲ್ಲಿದ್ದ ಕೆಲವು ಸಭ್ಯ ಗೃಹಸ್ಥರಿಗೆ ಸ್ವಾಮೀಜಿಯನ್ನು ಹಾಡಲು ಕೇಳುವಂತೆ ಪಿಸುಗುಟ್ಟಿದರು. ಆದರೆ ಸ್ವಾಮೀಜಿ ಅವರ ಅಭಿಲಾಷೆಯನ್ನು ಅರಿತು ನಸುನಗುತ್ತಾ ಕೇಳಿದರು: “ಮಾಸ್ಟರ್ ಮಹಾಶಯ, ನಿಮ್ಮಲ್ಲೇ ಏನನ್ನು ಕುರಿತು ಹೀಗೆ ಪಿಸುಗುಟ್ಟಿಕೊಳ್ಳುತ್ತಿರುವಿರಿ? ದಯವಿಟ್ಟು ಹೇಳಿ.” ಮಾಸ್ಟರ್ ಮಹಾಶಯರ ಪ್ರಾರ್ಥನೆಯಂತೆ ಸ್ವಾಮೀಜಿ ತಮ್ಮ ಇಂಪಾದ ಕಂಠದಿಂದ ಹಾಡಲುಪಕ್ರಮಿಸಿದರು – “ಪ್ರೇಮ ಮಾತೆ ಶ್ಯಾಮಾಳನ್ನು ನಿನ್ನ ಹೃದಯದಲ್ಲಿ ಪ್ರೇಮಪೂರ್ವಕವಾಗಿ ಎಚ್ಚರಿಕೆಯಿಂದ ನೋಡಿಕೊ” – ವೀಣೆಯನ್ನಾರೋ ನುಡಿಸುತ್ತಿರುವರೋ ಎಂಬಂತಿತ್ತು! ಮುಕ್ತಾಯವಾದ ಮೇಲೆ, ಅವರು ಮಾಸ್ಟರ್ ಮಹಾಶಯರಿಗೆ “ಈಗ ನಿಮಗೆ ತೃಪ್ತಿಯಾಯಿತೇ? ಇನ್ನು ಸಾಕು, ಹೆಚ್ಚಿಗೆ ಹಾಡುವುದಿಲ್ಲ. ಇಲ್ಲದಿದ್ದರೆ ಅದರ ಪರವಶತೆಯಲ್ಲೇ ನಾನು ಸಂಪೂರ್ಣವಾಗಿ ಮೈಮರೆಯುವೆ. ಅಲ್ಲದೆ ಪಾಶ್ಚಾತ್ಯ ದೇಶದಲ್ಲಿ ಉಪನ್ಯಾಸ ಮಾಡಿ ಮಾಡಿ ನನ್ನ ಧ್ವನಿ ಒಡೆದು ಹೋಗಿದೆ. ನನ್ನ ಧ್ವನಿ ಕೊಂಚ ಚೆನ್ನಾಗಿಯೇ ಕಂಪಿಸುತ್ತದೆ” ಎಂದರು.

ನಂತರ ಸ್ವಾಮಿಜಿ ತಮ್ಮ ಬ್ರಹ್ಮಚಾರಿ ಶಿಷ್ಯನೊಬ್ಬನನ್ನು ಮುಕ್ತಿಯ ವಿಷಯವನ್ನು ಕುರಿತು ಮಾತನಾಡುವಂತೆ ಹೇಳಿದರು. ಹಾಗೆಯೇ ಬ್ರಹ್ಮಚಾರಿ ಎದ್ದು ನಿಂತು ಕೊಂಚ ಹೊತ್ತು ಮಾತನಾಡಿದ, ಇನ್ನೂ ಕೆಲವರು ಹಾಗೇ ಮಾಡಿದರು. ನಂತರ ಸ್ಯಾಮೀಜಿ ಆಗ ಮಾತನಾಡಿದ ವಿಷಯದ ಮೇಲೆ ಚರ್ಚಿಸಲು ಕರೆದರು. ತಮ್ಮ ಗೃಹಸ್ಥ ಭಕ್ತನೊಬ್ಬನನ್ನು ಅದನ್ನು ಮೊದಲು ಪ್ರಾರಂಭಿಸಲು ಹೇಳಿದರು. ಆದರೆ ಅವನು ಅದ್ವೈತ ಮತ್ತು ಜ್ಞಾನವನ್ನು ಸಮರ್ಥಿಸಹೋಗಿ ದ್ವೈತ ಮತ್ತು ಭಕ್ತಿಗೆ ಕೀಳು ಸ್ಥಳವನ್ನು ನಿರ್ದೇಶಿಸಹೋದದ್ದರಿಂದ ಸಭಿಕರಲ್ಲೊಬ್ಬರು ಅದನ್ನು ಪ್ರತಿಭಟಿಸಿದರು. ಇಬ್ಬರೂ ಎದುರಾಳಿಗಳೂ ತಮ್ಮ ತಮ್ಮ ಅಭಿಪ್ರಾಯವನ್ನೇ ಮುಂದಿಟ್ಟುಕೊಂಡು ಅದನ್ನು ಸ್ಥಿರಪಡಿಸಲು ಪ್ರಯತ್ನಿಸಿದಾಗ ಅಲ್ಲೊಂದು ನಿಜವಾದ ವಾಗ್ಯುದ್ಧವೇ ಪ್ರಾರಂಭವಾಯಿತು. ಸ್ವಾಮೀಜಿ ಕೊಂಚ ಹೊತ್ತು ಇದನ್ನು ವೀಕ್ಷಿಸುತ್ತಿದ್ದರು. ಅವರು ಬರುಬರುತ್ತಾ ಗಂಭೀರವಾಗಲಾರಂಭಿಸಿದಾಗ ಸ್ವಾಮೀಜಿಯವರು ಅವರಿಗೆ ಭಾಷಣ ಮಾಡುವಂತೆ ಹೇಳಿ ಸುಮ್ಮನಿರಿಸಿದರು.

ವಾದ ಮಾಡುವಾಗ ಏಕೆ ಉದ್ವಿಗ್ನರಾಗಿ ಎಲ್ಲವನ್ನೂ ಹಾಳುಮಾಡುವಿರಿ? ಇಲ್ಲಿ ಕೇಳಿ! ಶ‍್ರೀರಾಮಕೃಷ್ಣರು ನಿಜವಾದ ಜ್ಞಾನ, ನಿಜವಾದ ಭಕ್ತಿ ಎರಡೂ ಒಂದೇ ಎಂದು ಹೇಳುತ್ತಿದ್ದರು. ಭಕ್ತಿಸೂತ್ರದ ಪ್ರಕಾರ ದೇವರು ‘ಪ್ರೇಮಮಯನು.’ ನಾವು ಅವನನ್ನು ‘ನಾನು ದೇವರನ್ನು ಪ್ರೀತಿಸುತ್ತೇನೆ’ ಎಂದು ಕೂಡ ಹೇಳಲಾರೆವು. ಏಕೆಂದರೆ ಅವನು ಸಂಪೂರ್ಣ ಪ್ರೇಮಸ್ವರೂಪನು. ಅವನಿಂದ ಹೊರಗೆ ಪ್ರೇಮವೇ ಇಲ್ಲ. ನಾವು ಅವನನ್ನು ಪ್ರೀತಿಸುವುದಕ್ಕಾಗಿ ನಮ್ಮ ಹೃದಯದಲ್ಲಿ ಹೊಂದಿರುವ ಪ್ರೇಮವೂ ಕೂಡ ಅವನೇ. ಇದೇ ರೀತಿಯಲ್ಲಿ, ದೇವರೆಡೆಗೆ ನಮ್ಮನ್ನು ಸೆಳೆಯುವ ಯಾವ ಆಕರ್ಷಣೆಯೇ ಆಗಲಿ, ಮನೋಪ್ರವೃತ್ತಿಯೇ ಆಗಲಿ ಎಲ್ಲಾ ಅವನೇ ಆಗಿದ್ದಾನೆ. ಕಳ್ಳನು ಕದಿಯುವಾಗ, ಸೂಳೆ ತನ್ನ ದೇಹವನ್ನು ವ್ಯಭಿಚಾರಕ್ಕೆ ಮಾರಿಕೊಳ್ಳುವಾಗ, ತಾಯಿ ಮಗುವನ್ನು ಪ್ರೀತಿಸುವಾಗ – ಇವೆಲ್ಲದರಲ್ಲಿಯೂ ದೇವರಿರುವನು. ಅವನು ಸರ್ವವ್ಯಾಪಿ. ಜ್ಞಾನಸೂತ್ರದ ಪ್ರಕಾರ ಕೂಡ ಎಲ್ಲೆಡೆಯಲ್ಲಿಯೂ ದೇವರನ್ನು ನಾವು ಸಾಕ್ಷಾತ್ಕರಿಸಿಕೊಳ್ಳಬಹುದು. ಜ್ಞಾನ ಮತ್ತು ಭಕ್ತಿಯ ಸಮನ್ವಯತೆ ಇಲ್ಲಿದೆ. ಯಾವಾಗ ದೈವೀಭಾವದಲ್ಲಿ ಸಂಪೂರ್ಣವಾಗಿ ತನ್ಮಯರಾಗುವೆವೊ, ಎಂದರೆ ಸಮಾಧಿಸ್ಥಿತಿಯನ್ನು ಪಡೆಯುವೆವೊ ಆಗ ಮಾತ್ರ ಈ ದ್ವೈತಭಾವ ಅಳಿಸಿಹೋಗುವುದು. ದೇವರು, ಭಕ್ತ ಎಂಬ ಭಿನ್ನತೆ ಮಾಯವಾಗುವುದು. ಭಕ್ತಿಯನ್ನು ವಿವರಿಸುವ ಧರ್ಮಗ್ರಂಥದಲ್ಲಿ ಐದು ಬೇರೆ ಬೇರೆ ವಿಧವಾದ ಸಂಬಂಧಗಳನ್ನು ಹೇಳಿದೆ. ಇವುಗಳಲ್ಲಿ ಯಾವುದಾದರೊಂದು ಮಾರ್ಗವನ್ನನುಸರಿಸಿದರೂ ದೇವರನ್ನು ಪಡೆಯಬಹುದು. ಇದರ ಜೊತೆಗೆ ಇನ್ನೊಂದನ್ನೂ ಸೇರಿಸಬಹುದು – ಅದೇ ದೇವರಲ್ಲಿ ತಾನು ಐಕ್ಯ ಎಂದು ಧ್ಯಾನಮಾಡುವುದು. ಹೀಗೆ ಭಕ್ತರು ಅದ್ವೈತಿಗಳನ್ನು ಭಕ್ತರೆಂದೂ ಆದರೆ ದೇವರಲ್ಲಿ ಐಕ್ಯರಾದವರೆಂದೂ ಕರೆಯಬಹುದು. ಎಲ್ಲಿಯವರೆಗೆ ಮಾಯಾ ಜಗತ್ತಿನ ಎಲ್ಲೆಯೊಳಗೆ ಇರುವೆವೊ ಅಲ್ಲಿಯವರೆಗೂ ದ್ವೈತಭಾವ ನಿಸ್ಸಂದೇಹವಾಗಿ ಇದ್ದೇ ಇರುವುದು. ದೇಶ–ಕಾಲ–ನಿಮಿತ್ತ ಅಥವಾ ನಾಮರೂಪಗಳೇ ಮಾಯೆಯೆನ್ನುವುದು. ಯಾವಾಗ ಈ ಮಾಯೆಯನ್ನು ಮೀರಿಹೋಗುವೆವೋ ಆಗ ಮಾತ್ರ ಈ ಏಕತ್ವ ಸಾಕ್ಷಾತ್ಕಾರವಾಗುವುದು. ಆಗ ಮಾನವನು ದ್ವೈತಿಯೂ ಅಲ್ಲ ಅದ್ವೈತಿಯೂ ಅಲ್ಲ. ಅವನಿಗೆ ಎಲ್ಲವೂ ಒಂದೆ. ಭಕ್ತ ಮತ್ತು ಜ್ಞಾನಿಗಳ ಮಧ್ಯೆ ನಮಗೆ ಕಾಣಬರುವ ಈ ಭಿನ್ನತೆಯೆಲ್ಲ ಪ್ರಾರಂಭದಲ್ಲಿರುವ ಅವಸ್ಥೆ. ದೇವರನ್ನು ಒಬ್ಬ ಹೊರಗಡೆ ನೋಡುವನು, ಮತ್ತೊಬ್ಬ ಒಳಗಡೆ ಕಾಣುವನು. ಆದರೆ ಮತ್ತೊಂದು ವಿಷಯ – ಶ‍್ರೀರಾಮಕೃಷ್ಣರು ಹೇಳುತ್ತಿದ್ದರು – ಭಕ್ತಿಯ ಮತ್ತೊಂದು ಅವಸ್ಥೆ ಪರಾಭಕ್ತಿ. ಎಂದರೆ ಮುಕ್ತಿ ಗಳಿಸಿದಮೇಲೆ, ಅದ್ವೈತಾವಸ್ಥೆಯಲ್ಲಿ ಸ್ಥಿತವಾದಮೇಲೆ ದೇವರನ್ನು ಪ್ರೀತಿಸುವುದು. ಇದೊಂದು ಅಸಂಗತೋಕ್ತಿಯಂತೆ ಕಾಣಬಹುದು. ಮುಕ್ತಿಯನ್ನು ಗಳಿಸಿದ ಮೇಲೂ ಭಕ್ತಿ ಭಾವವನ್ನುಳಿಸಿಕೊಳ್ಳಬೇಕೆ, ಬಯಸಬೇಕೆ ಎಂಬ ಪ್ರಶ್ನೆ ಏಳಬಹುದು. ಇದಕ್ಕೆ ಉತ್ತರ ‘ಮುಕ್ತ ಅಥವಾ ಬಂಧನರಹಿತನು ಎಲ್ಲಾ ನಿಯಮಗಳಿಗೂ ಅತೀತನು.’ ಅವನ ವಿಷಯದಲ್ಲಿ ಯಾವ ನಿಯಮವನ್ನೂ ಅಳವಡಿಸಲಾಗುವುದಿಲ್ಲ. ಆದ್ದರಿಂದ ಅವನ ವಿಷಯವಾಗಿ ಯಾವ ಪ್ರಶ್ನೆಯೂ ಏಳುವುದಿಲ್ಲ. ಮುಕ್ತನಾದ ಮೇಲೂ ಕೆಲವರು ಕೇವಲ ತಮ್ಮ ಸ್ವಂತ ಇಚ್ಛೆಯಿಂದ ಅದರ ಸವಿಯನ್ನು ಅನುಭವಿಸಲು ಭಕ್ತರಾಗುಳಿಯುವರು.
=
ಸಭೆಗೆ ಬಂದಿದ್ದವರಲ್ಲಿ ಅನೇಕರಿಗೆ ಸ್ವಾಮೀಜಿ ಬಹು ಚೆನ್ನಾಗಿ ಹಾಡುವರೆಂದು ತಿಳಿದಿದ್ದಿತು. ಅವರಿಂದ ಹಾಡಿಸಬೇಕೆಂದು ಅವರೆಲ್ಲಾ ಅಪೇಕ್ಷಿಸುತ್ತಿದ್ದರು. ಅದನ್ನು ತಿಳಿದು ಮಾಸ್ಟರ್ ಮಹಾಶಯರು ತಮ್ಮ ಬಳಿಯಲ್ಲಿದ್ದ ಕೆಲವು ಸಭ್ಯ ಗೃಹಸ್ಥರಿಗೆ ಸ್ವಾಮೀಜಿಯನ್ನು ಹಾಡಲು ಕೇಳುವಂತೆ ಪಿಸುಗುಟ್ಟಿದರು. ಆದರೆ ಸ್ವಾಮೀಜಿ ಅವರ ಅಭಿಲಾಷೆಯನ್ನು ಅರಿತು ನಸುನಗುತ್ತಾ ಕೇಳಿದರು: “ಮಾಸ್ಟರ್ ಮಹಾಶಯ, ನಿಮ್ಮಲ್ಲೇ ಏನನ್ನು ಕುರಿತು ಹೀಗೆ ಪಿಸುಗುಟ್ಟಿಕೊಳ್ಳುತ್ತಿರುವಿರಿ? ದಯವಿಟ್ಟು ಹೇಳಿ.” ಮಾಸ್ಟರ್ ಮಹಾಶಯರ ಪ್ರಾರ್ಥನೆಯಂತೆ ಸ್ವಾಮೀಜಿ ತಮ್ಮ ಇಂಪಾದ ಕಂಠದಿಂದ ಹಾಡಲುಪಕ್ರಮಿಸಿದರು - “ಪ್ರೇಮ ಮಾತೆ ಶ್ಯಾಮಾಳನ್ನು ನಿನ್ನ ಹೃದಯದಲ್ಲಿ ಪ್ರೇಮಪೂರ್ವಕವಾಗಿ ಎಚ್ಚರಿಕೆಯಿಂದ ನೋಡಿಕೊ” - ವೀಣೆಯನ್ನಾರೋ ನುಡಿಸುತ್ತಿರುವರೋ ಎಂಬಂತಿತ್ತು! ಮುಕ್ತಾಯವಾದ ಮೇಲೆ, ಅವರು ಮಾಸ್ಟರ್ ಮಹಾಶಯರಿಗೆ “ಈಗ ನಿಮಗೆ ತೃಪ್ತಿಯಾಯಿತೇ? ಇನ್ನು ಸಾಕು, ಹೆಚ್ಚಿಗೆ ಹಾಡುವುದಿಲ್ಲ. ಇಲ್ಲದಿದ್ದರೆ ಅದರ ಪರವಶತೆಯಲ್ಲೇ ನಾನು ಸಂಪೂರ್ಣವಾಗಿ ಮೈಮರೆಯುವೆ. ಅಲ್ಲದೆ ಪಾಶ್ಚಾತ್ಯ ದೇಶದಲ್ಲಿ ಉಪನ್ಯಾಸ ಮಾಡಿ ಮಾಡಿ ನನ್ನ ಧ್ವನಿ ಒಡೆದು ಹೋಗಿದೆ. ನನ್ನ ಧ್ವನಿ ಕೊಂಚ ಚೆನ್ನಾಗಿಯೇ ಕಂಪಿಸುತ್ತದೆ” ಎಂದರು.

ನಂತರ ಸ್ವಾಮಿಜಿ ತಮ್ಮ ಬ್ರಹ್ಮಚಾರಿ ಶಿಷ್ಯನೊಬ್ಬನನ್ನು ಮುಕ್ತಿಯ ವಿಷಯವನ್ನು ಕುರಿತು ಮಾತನಾಡುವಂತೆ ಹೇಳಿದರು. ಹಾಗೆಯೇ ಬ್ರಹ್ಮಚಾರಿ ಎದ್ದು ನಿಂತು ಕೊಂಚ ಹೊತ್ತು ಮಾತನಾಡಿದ, ಇನ್ನೂ ಕೆಲವರು ಹಾಗೇ ಮಾಡಿದರು. ನಂತರ ಸ್ಯಾಮೀಜಿ ಆಗ ಮಾತನಾಡಿದ ವಿಷಯದ ಮೇಲೆ ಚರ್ಚಿಸಲು ಕರೆದರು. ತಮ್ಮ ಗೃಹಸ್ಥ ಭಕ್ತನೊಬ್ಬನನ್ನು ಅದನ್ನು ಮೊದಲು ಪ್ರಾರಂಭಿಸಲು ಹೇಳಿದರು. ಆದರೆ ಅವನು ಅದ್ವೈತ ಮತ್ತು ಜ್ಞಾನವನ್ನು ಸಮರ್ಥಿಸಹೋಗಿ ದ್ವೈತ ಮತ್ತು ಭಕ್ತಿಗೆ ಕೀಳು ಸ್ಥಳವನ್ನು ನಿರ್ದೇಶಿಸಹೋದದ್ದರಿಂದ ಸಭಿಕರಲ್ಲೊಬ್ಬರು ಅದನ್ನು ಪ್ರತಿಭಟಿಸಿದರು. ಇಬ್ಬರೂ ಎದುರಾಳಿಗಳೂ ತಮ್ಮ ತಮ್ಮ ಅಭಿಪ್ರಾಯವನ್ನೇ ಮುಂದಿಟ್ಟುಕೊಂಡು ಅದನ್ನು ಸ್ಥಿರಪಡಿಸಲು ಪ್ರಯತ್ನಿಸಿದಾಗ ಅಲ್ಲೊಂದು ನಿಜವಾದ ವಾಗ್ಯುದ್ಧವೇ ಪ್ರಾರಂಭವಾಯಿತು. ಸ್ವಾಮೀಜಿ ಕೊಂಚ ಹೊತ್ತು ಇದನ್ನು ವೀಕ್ಷಿಸುತ್ತಿದ್ದರು. ಅವರು ಬರುಬರುತ್ತಾ ಗಂಭೀರವಾಗಲಾರಂಭಿಸಿದಾಗ ಸ್ವಾಮೀಜಿಯವರು ಅವರಿಗೆ ಭಾಷಣ ಮಾಡುವಂತೆ ಹೇಳಿ ಸುಮ್ಮನಿರಿಸಿದರು.

ವಾದ ಮಾಡುವಾಗ ಏಕೆ ಉದ್ವಿಗ್ನರಾಗಿ ಎಲ್ಲವನ್ನೂ ಹಾಳುಮಾಡುವಿರಿ? ಇಲ್ಲಿ ಕೇಳಿ! ಶ‍್ರೀರಾಮಕೃಷ್ಣರು ನಿಜವಾದ ಜ್ಞಾನ, ನಿಜವಾದ ಭಕ್ತಿ ಎರಡೂ ಒಂದೇ ಎಂದು ಹೇಳುತ್ತಿದ್ದರು. ಭಕ್ತಿಸೂತ್ರದ ಪ್ರಕಾರ ದೇವರು ‘ಪ್ರೇಮಮಯನು.’ ನಾವು ಅವನನ್ನು ‘ನಾನು ದೇವರನ್ನು ಪ್ರೀತಿಸುತ್ತೇನೆ’ ಎಂದು ಕೂಡ ಹೇಳಲಾರೆವು. ಏಕೆಂದರೆ ಅವನು ಸಂಪೂರ್ಣ ಪ್ರೇಮಸ್ವರೂಪನು. ಅವನಿಂದ ಹೊರಗೆ ಪ್ರೇಮವೇ ಇಲ್ಲ. ನಾವು ಅವನನ್ನು ಪ್ರೀತಿಸುವುದಕ್ಕಾಗಿ ನಮ್ಮ ಹೃದಯದಲ್ಲಿ ಹೊಂದಿರುವ ಪ್ರೇಮವೂ ಕೂಡ ಅವನೇ. ಇದೇ ರೀತಿಯಲ್ಲಿ, ದೇವರೆಡೆಗೆ ನಮ್ಮನ್ನು ಸೆಳೆಯುವ ಯಾವ ಆಕರ್ಷಣೆಯೇ ಆಗಲಿ, ಮನೋಪ್ರವೃತ್ತಿಯೇ ಆಗಲಿ ಎಲ್ಲಾ ಅವನೇ ಆಗಿದ್ದಾನೆ. ಕಳ್ಳನು ಕದಿಯುವಾಗ, ಸೂಳೆ ತನ್ನ ದೇಹವನ್ನು ವ್ಯಭಿಚಾರಕ್ಕೆ ಮಾರಿಕೊಳ್ಳುವಾಗ, ತಾಯಿ ಮಗುವನ್ನು ಪ್ರೀತಿಸುವಾಗ - ಇವೆಲ್ಲದರಲ್ಲಿಯೂ ದೇವರಿರುವನು. ಅವನು ಸರ್ವವ್ಯಾಪಿ. ಜ್ಞಾನಸೂತ್ರದ ಪ್ರಕಾರ ಕೂಡ ಎಲ್ಲೆಡೆಯಲ್ಲಿಯೂ ದೇವರನ್ನು ನಾವು ಸಾಕ್ಷಾತ್ಕರಿಸಿಕೊಳ್ಳಬಹುದು. ಜ್ಞಾನ ಮತ್ತು ಭಕ್ತಿಯ ಸಮನ್ವಯತೆ ಇಲ್ಲಿದೆ. ಯಾವಾಗ ದೈವೀಭಾವದಲ್ಲಿ ಸಂಪೂರ್ಣವಾಗಿ ತನ್ಮಯರಾಗುವೆವೊ, ಎಂದರೆ ಸಮಾಧಿಸ್ಥಿತಿಯನ್ನು ಪಡೆಯುವೆವೊ ಆಗ ಮಾತ್ರ ಈ ದ್ವೈತಭಾವ ಅಳಿಸಿಹೋಗುವುದು. ದೇವರು, ಭಕ್ತ ಎಂಬ ಭಿನ್ನತೆ ಮಾಯವಾಗುವುದು. ಭಕ್ತಿಯನ್ನು ವಿವರಿಸುವ ಧರ್ಮಗ್ರಂಥದಲ್ಲಿ ಐದು ಬೇರೆ ಬೇರೆ ವಿಧವಾದ ಸಂಬಂಧಗಳನ್ನು ಹೇಳಿದೆ. ಇವುಗಳಲ್ಲಿ ಯಾವುದಾದರೊಂದು ಮಾರ್ಗವನ್ನನುಸರಿಸಿದರೂ ದೇವರನ್ನು ಪಡೆಯಬಹುದು. ಇದರ ಜೊತೆಗೆ ಇನ್ನೊಂದನ್ನೂ ಸೇರಿಸಬಹುದು - ಅದೇ ದೇವರಲ್ಲಿ ತಾನು ಐಕ್ಯ ಎಂದು ಧ್ಯಾನಮಾಡುವುದು. ಹೀಗೆ ಭಕ್ತರು ಅದ್ವೈತಿಗಳನ್ನು ಭಕ್ತರೆಂದೂ ಆದರೆ ದೇವರಲ್ಲಿ ಐಕ್ಯರಾದವರೆಂದೂ ಕರೆಯಬಹುದು. ಎಲ್ಲಿಯವರೆಗೆ ಮಾಯಾ ಜಗತ್ತಿನ ಎಲ್ಲೆಯೊಳಗೆ ಇರುವೆವೊ ಅಲ್ಲಿಯವರೆಗೂ ದ್ವೈತಭಾವ ನಿಸ್ಸಂದೇಹವಾಗಿ ಇದ್ದೇ ಇರುವುದು. ದೇಶ-ಕಾಲ-ನಿಮಿತ್ತ ಅಥವಾ ನಾಮರೂಪಗಳೇ ಮಾಯೆಯೆನ್ನುವುದು. ಯಾವಾಗ ಈ ಮಾಯೆಯನ್ನು ಮೀರಿಹೋಗುವೆವೋ ಆಗ ಮಾತ್ರ ಈ ಏಕತ್ವ ಸಾಕ್ಷಾತ್ಕಾರವಾಗುವುದು. ಆಗ ಮಾನವನು ದ್ವೈತಿಯೂ ಅಲ್ಲ ಅದ್ವೈತಿಯೂ ಅಲ್ಲ. ಅವನಿಗೆ ಎಲ್ಲವೂ ಒಂದೆ. ಭಕ್ತ ಮತ್ತು ಜ್ಞಾನಿಗಳ ಮಧ್ಯೆ ನಮಗೆ ಕಾಣಬರುವ ಈ ಭಿನ್ನತೆಯೆಲ್ಲ ಪ್ರಾರಂಭದಲ್ಲಿರುವ ಅವಸ್ಥೆ. ದೇವರನ್ನು ಒಬ್ಬ ಹೊರಗಡೆ ನೋಡುವನು, ಮತ್ತೊಬ್ಬ ಒಳಗಡೆ ಕಾಣುವನು. ಆದರೆ ಮತ್ತೊಂದು ವಿಷಯ - ಶ‍್ರೀರಾಮಕೃಷ್ಣರು ಹೇಳುತ್ತಿದ್ದರು - ಭಕ್ತಿಯ ಮತ್ತೊಂದು ಅವಸ್ಥೆ ಪರಾಭಕ್ತಿ. ಎಂದರೆ ಮುಕ್ತಿ ಗಳಿಸಿದಮೇಲೆ, ಅದ್ವೈತಾವಸ್ಥೆಯಲ್ಲಿ ಸ್ಥಿತವಾದಮೇಲೆ ದೇವರನ್ನು ಪ್ರೀತಿಸುವುದು. ಇದೊಂದು ಅಸಂಗತೋಕ್ತಿಯಂತೆ ಕಾಣಬಹುದು. ಮುಕ್ತಿಯನ್ನು ಗಳಿಸಿದ ಮೇಲೂ ಭಕ್ತಿ ಭಾವವನ್ನುಳಿಸಿಕೊಳ್ಳಬೇಕೆ, ಬಯಸಬೇಕೆ ಎಂಬ ಪ್ರಶ್ನೆ ಏಳಬಹುದು. ಇದಕ್ಕೆ ಉತ್ತರ ‘ಮುಕ್ತ ಅಥವಾ ಬಂಧನರಹಿತನು ಎಲ್ಲಾ ನಿಯಮಗಳಿಗೂ ಅತೀತನು.’ ಅವನ ವಿಷಯದಲ್ಲಿ ಯಾವ ನಿಯಮವನ್ನೂ ಅಳವಡಿಸಲಾಗುವುದಿಲ್ಲ. ಆದ್ದರಿಂದ ಅವನ ವಿಷಯವಾಗಿ ಯಾವ ಪ್ರಶ್ನೆಯೂ ಏಳುವುದಿಲ್ಲ. ಮುಕ್ತನಾದ ಮೇಲೂ ಕೆಲವರು ಕೇವಲ ತಮ್ಮ ಸ್ವಂತ ಇಚ್ಛೆಯಿಂದ ಅದರ ಸವಿಯನ್ನು ಅನುಭವಿಸಲು ಭಕ್ತರಾಗುಳಿಯುವರು.


ಪ್ರಶ್ನೆ: ತಾಯಿಗೆ ಮಗುವಿನಲ್ಲಿರುವ ಪ್ರೇಮದಲ್ಲಿ ದೇವರಿರಬಹುದು. ಆದರೆ, ಮಹಾಶಯರೆ, ಕಳ್ಳರಲ್ಲಿ, ಸೂಳೆಯಲ್ಲಿ ಅವರ ಸಹಜ ಪಾಪವೃತ್ತಿಗಳಲ್ಲೂ ದೇವರಿದ್ದಾನೆಂದರೆ ಈ ಅಭಿಪ್ರಾಯ ನಿಜವಾಗಿಯೂ ದಿಗ್ಭ್ರಮೆ ಹಿಡಿಸುವುದು. ಹೀಗಾದ ಪಕ್ಷಕ್ಕೆ, ದೇವರು ಈ ಜಗತ್ತಿನಲ್ಲಿ ಪುಣ್ಯಕ್ಕೆ ಎಷ್ಟು ಜವಾಬ್ದಾರನೋ ಪಾಪಕ್ಕೂ ಅಷ್ಟೇ ಜವಾಬ್ದಾರನೆಂದು ಅರ್ಥವಾಗುವುದು.

ಸ್ವಾಮೀಜಿ: ಪೂರ್ಣ ಸಾಕ್ಷಾತ್ಕಾರ ಸ್ಥಿತಿಯಲ್ಲಿ ಈ ಬಗೆಯ ಜ್ಞಾನ ಬರುವುದು. ಆಗ ನಾವು ಎಲ್ಲಿ ಪ್ರೇಮದ ಆವಿರ್ಭಾವವನ್ನು ನೋಡಿದರೂ, ಆಕರ್ಷಣೀಯವಾದುದನ್ನು ನೋಡಿದರೂ, ಅವೆಲ್ಲಾ ಪರಮಾತ್ಮನೆಂದು ಗೊತ್ತಾಗುವುದು. ಆದರೆ ಈ ಜನ್ಮದಲ್ಲೇ ಆ ಗುರಿಯನ್ನು ನೋಡಿ ಸಾಕ್ಷಾತ್ಕರಿಸಿಕೊಳ್ಳಬೇಕಾದರೆ ಮೊದಲು ಆ ಸ್ಥಿತಿಯನ್ನು ಪಡೆಯಬೇಕು.

ಪ್ರಶ್ನೆ: ಆದರೂ ದೇವರು ಪಾಪದಲ್ಲಿಯೂ ಇದ್ದಾನೆ ಎಂಬುದನ್ನು ಒಪ್ಪಿಕೊಳ್ಳಬೇಕು!

ಸ್ವಾಮೀಜಿ: ಇಲ್ಲಿ ನೋಡು, ವಾಸ್ತವಿಕವಾಗಿ ಒಳ್ಳೆಯದು ಕೆಟ್ಟುದು ಎಂಬ ಭಿನ್ನ ವಸ್ತುಗಳಿಲ್ಲವೇ ಇಲ್ಲ. ಅವು ಕೇವಲ ಸಾಂಪ್ರದಾಯಿಕ ಪದಗಳು. ಒಂದೇ ವಸ್ತುವನ್ನು ನಾವು ಒಮ್ಮೆ ಕೆಟ್ಟುದೆಂದೂ ಮತ್ತೊಮ್ಮೆ ಒಳ್ಳೆಯದೆಂದೂ ನಾವು ಅದನ್ನು ಉಪಯೋಗಿಸಿಕೊಳ್ಳುವ ರೀತಿಯಲ್ಲಿ ಹೇಳುತ್ತೇವೆ. ಉದಾಹರಣೆಗೆ ದೀಪದ ಬೆಳಕನ್ನು ತೆಗೆದುಕೊ, ಅದು ಉರಿಯುತ್ತಿರುವುದರಿಂದ ನಮಗೆ ನೋಡುವುದಕ್ಕೂ, ಅನೇಕ ಉಪಯುಕ್ತವಾದ ಕೆಲಸಗಳನ್ನು ಮಾಡುವುದಕ್ಕೂ ಸಾಧ್ಯ. ಇದು ದೀಪವನ್ನು ಉಪಯೋಗಿಸುವ ಹಲವು ವಿಧಾನಗಳಲ್ಲೊಂದು. ನಾವು ಬೆರಳನ್ನು ದೀಪಕ್ಕೆ ಹಿಡಿದರೆ ಬೆರಳು ಸುಟ್ಟುಹೋಗುವುದು. ಅದೇ ದೀಪವನ್ನು ಉಪಯೋಗಿಸುವ ಮತ್ತೊಂದು ವಿಧಾನ ಇದು. ಆದ್ದರಿಂದ ಒಂದು ವಸ್ತುವು ಒಳ್ಳೆಯದಾಗುವುದೂ, ಕೆಟ್ಟದಾಗುವುದೂ ಅದನ್ನು ಉಪಯೋಗಿಸುವ ರೀತಿಯಲ್ಲಿದೆ. ಹೀಗೆಯೇ ಸದ್ಗುಣ ದುರ್ಗುಣಗಳೂ ಕೂಡ. ವಿಸ್ತಾರವಾಗಿ ಹೇಳಬೇಕಾದರೆ ನಮ್ಮ ಮನಸ್ಸಿನ ಮತ್ತು ದೇಹದ ಸ್ವಾಭಾವಿಕ ಶಕ್ತಿಯನ್ನು ಯೋಗ್ಯರೀತಿಯಲ್ಲಿ ಉಪಯೋಗಿಸಿದರೆ ಅದು ಸದ್ಗುಣವೆನ್ನಿಸಿಕೊಳ್ಳುವುದು. ಅಯೋಗ್ಯ ರೀತಿಯಲ್ಲಿ ಉಪಯೋಗಿಸಿದರೆ, ವ್ಯರ್ಥಮಾಡಿದರೆ, ಅದೇ ದುರ್ಗುಣವೆನ್ನಿಸಿಕೊಳ್ಳುವುದು.

ಹೀಗೆ ಪ್ರಶ್ನೆಯಾದ ಮೇಲೆ ಪ್ರಶ್ನೆಗಳು ಕೇಳಲ್ಪಟ್ಟು ಉತ್ತರ ಕೊಡಲಾಯಿತು. ಯಾರೋ ಹೇಳಿದರು, “ಒಂದು ಸ್ವರ್ಗೀಯ ವಸ್ತು ಮತ್ತೊಂದನ್ನು ಆಕರ್ಷಿಸುವ ಕಡೆ ಕೂಡ ದೇವರಿದ್ದಾನೆಂಬ ವಾದ ನಿಜವಾದ ಸಂಗತಿಯಾಗಿರಬಹುದು, ಅಥವಾ ಸುಳ್ಳಾಗಿರಬಹುದು. ಆದರೆ ಈ ಉದ್ದೇಶವನ್ನು ತಿಳಿಸುವ ಮನೋಹರವಾದ ಭಾವನೆಯನ್ನು ಯಾರೂ ನಿರಾಕರಿಸಲಾರರು.”

ಸ್ವಾಮೀಜಿ: ಇಲ್ಲ, ನನ್ನ ಪ್ರಿಯ ಮಹಾಶಯರೆ, ಅದು ಕವಿತೆಯಲ್ಲ. ಯಾವಾಗ ಜ್ಞಾನವನ್ನು ಪಡೆಯುವೆವೋ ಆಗ ಅದರ ಸತ್ಯಾಂಶ ಅರಿವಾಗುವುದು.

ಈ ವಿಷಯವಾಗಿ ಸ್ವಾಮೀಜಿ ಮುಂದುವರಿಸಿ ಹೇಳಿದ್ದನ್ನೆಲ್ಲಾ ಕೇಳಿ ನಾನು ಈ ರೀತಿ ಅರ್ಥಮಾಡಿಕೊಂಡೆ: ದೇಹ ಮತ್ತು ಆತ್ಮ ಎರಡೂ ಬಾಹ್ಯದೃಷ್ಟಿಯಿಂದ ಭಿನ್ನವಸ್ತುಗಳೆಂದು ಅನ್ನಿಸಿದರೂ ವಾಸ್ತವಿಕವಾಗಿ ಅವೆರಡೂ ಒಂದೇ ವಸ್ತುವಿನ ಎರಡು ರೂಪಗಳು. ಹಾಗೆಯೇ ಈ ಪಂಚಭೂತಾತ್ಮಕ ಪ್ರಪಂಚದಲ್ಲಿ ನಮಗೆ ತಿಳಿದಿರುವ ಎಲ್ಲಾ ಬಗೆಯ ವಿಭಿನ್ನ ಶಕ್ತಿಗಳು ಒಂದೇ ಶಕ್ತಿಯ ವಿವಿಧ ಆವಿರ್ಭಾವಗಳು. ಕಡಿಮೆ ಆತ್ಮಶಕ್ತಿಯು ಆವಿರ್ಭಾವವಾಗಿರುವ ವಸ್ತುವನ್ನು ನಾವು ಜಡವಸ್ತುವೆನ್ನುತ್ತೇವೆ. ಹೆಚ್ಚು ವಿಕಾಸವಾಗಿದ್ದರೆ ಜೀವಂತವೆನ್ನುತ್ತೇವೆ. ಎಲ್ಲಾ ಕಾಲದಲ್ಲಿಯೂ ಎಲ್ಲಾ ಸ್ಥಿತಿಯಲ್ಲಿಯೂ ಯಾವೊಂದು ವಸ್ತುವೂ ಸಂಪೂರ್ಣವಾಗಿ ಜಡವಾಗೇ ಉಳಿಯುವುದಿಲ್ಲ. ಯಾವ ಶಕ್ತಿಯು ಈ ಭೌತಿಕ ಪ್ರಪಂಚದಲ್ಲಿ ಆಕರ್ಷಣ ಅಥವಾ ಗುರುತ್ವಾಕರ್ಷಣದಂತೆ ಪ್ರದರ್ಶಿತವಾಗಿದೆಯೋ ಅದೇ ಶಕ್ತಿಯು ಇನ್ನೂ ಶ್ರೇಷ್ಠವಾದ ಸೂಕ್ಷ್ಮಸ್ಥಿತಿಯಲ್ಲಿ, ಆತ್ಮಸಾಕ್ಷಾತ್ಕಾರದ ಅತ್ಯುಚ್ಚ ಆಧ್ಯಾತ್ಮಿಕ ಸ್ಥಿತಿಯಲ್ಲಿ ಪ್ರೇಮದ ಹಾಗೆ ಭಾಸವಾಗುವುದು.

ಪ್ರಶ್ನೆ: ಪ್ರತಿಯೊಬ್ಬ ವ್ಯಕ್ತಿಯ ಸಾಧನೆಯ ಸಂದರ್ಭದಲ್ಲಿ ಕೂಡ ಈ ಬಗೆಯ ವ್ಯತ್ಯಾಸ ಏಕೆ ತೋರಬೇಕು? ತನ್ನ ಸ್ವಾಭಾವಿಕ ಶಕ್ತಿಯನ್ನು ಈ ರೀತಿ ಅಯೋಗ್ಯ ರೀತಿಯಲ್ಲಿ ಉಪಯೋಗಿಸುವ ಪ್ರವೃತ್ತಿಯಾದರೂ ಮನುಷ್ಯನಿಗೇಕಿರಬೇಕು?

ಸ್ವಾಮೀಜಿ: ಪ್ರವೃತ್ತಿಯು ಅವನ ಪೂರ್ವಜನ್ಮದ ಕರ್ಮಫಲದಿಂದ ಪ್ರಾಪ್ತವಾಗುವುದು. ಪ್ರತಿಯೊಂದೂ ಅವನವನ ಕರ್ಮಫಲ. ಆದ್ದರಿಂದ ಪ್ರವೃತ್ತಿಗಳನ್ನು ಹಿಡಿತದಲ್ಲಿಟ್ಟುಕೊಂಡು ಅದನ್ನು ಯೋಗ್ಯರೀತಿಯಲ್ಲಿ ವ್ಯವಸ್ಥೆಗೊಳಿಸುವುದು ಪ್ರತಿಯೊಬ್ಬ ವ್ಯಕ್ತಿಯ ಕೈಯಲ್ಲೂ ಸಂಪೂರ್ಣವಾಗಿ ಇದೆ.

ಪ್ರಶ್ನೆ: ಪ್ರತಿಯೊಂದೂ ನಮ್ಮ ನಮ್ಮ ಕರ್ಮಫಲಗಳೇ ಆದರೂ, ಅದಕ್ಕೆ ಒಂದು ಆದಿ ಇರಬೇಕಲ್ಲವೆ? ಹಾಗಾದರೆ ಆದಿಯಲ್ಲಿ ನಮ್ಮ ಪ್ರವೃತ್ತಿಗಳು ಏಕೆ ಒಳ್ಳೆಯದಾಗಿ ಅಥವಾ ಕೆಟ್ಟುದಾಗಿ ಇರಬೇಕು?

ಸ್ವಾಮೀಜಿ: ಒಂದು ಆದಿ ಇದೆ ಎಂದು ನಿನಗೆ ಹೇಗೆ ಗೊತ್ತು? ಸೃಷ್ಟಿಗೆ ಆದಿಯಿಲ್ಲ – ಇದೇ ವೇದಗಳ ಸಿದ್ಧಾಂತ, ದೇವರಿರುವರೆಗೂ ಸೃಷ್ಟಿಯೂ ಇರುವುದು.

ಪ್ರಶ್ನೆ: ಸರಿ ಮಹಾಶಯರೆ, ಈ ಮಾಯೆ ಏಕೆ ಇಲ್ಲಿದೆ? ಅದು ಎಲ್ಲಿಂದ ಬಂತು?

ಸ್ವಾಮೀಜಿ: ದೇವರ ವಿಷಯವಾಗಿ ‘ಏಕೆ’ ಎಂದು ಕೇಳುವುದು ತಪ್ಪು. ಯಾರಿಗೆ ಅಪೇಕ್ಷೆ ಇದೆಯೊ ಯಾರು ಅಪೂರ್ಣರೊ ಅವರ ವಿಷಯವಾಗಿ ಹೀಗೆ ಕೇಳಬಹುದು. ಯಾರಿಗೆ ಅಪೇಕ್ಷೆಗಳೇ ಇಲ್ಲವೊ ಯಾರು ಪೂರ್ಣನೋ ಅವನ ವಿಚಾರವಾಗಿ ‘ಏಕೆ?’ ಎಂಬುದು ಇರಲು ಸಾಧ್ಯವೆ? ‘ಎಲ್ಲಿಂದ ಮಾಯೆ ಬಂದಿತು?’ ಎಂಬ ಪ್ರಶ್ನೆಯನ್ನೇ ಕೇಳಬಾರದು. ಕಾಲ ದೇಶ ನಿಮಿತ್ತವೇ ಮಾಯೆ. ನೀನು, ನಾನು, ಪ್ರತಿಯೊಬ್ಬರೂ ಈ ಮಾಯೆಯೊಳಗೆ ಇರುವೆವು. ನೀನು ಮಾಯೆಯ ಆಚೆ ಏನಿದೆ ಎಂದು ಕೇಳುತ್ತಿರುವೆ: ನೀನು ಮಾಯೆಯೊಳಗೆ ಇರಬೇಕಾದರೆ ಈ ಪ್ರಶ್ನೆಯನ್ನು ಹೇಗೆ ಕೇಳುವೆ?

ಪುನಃ ಅನೇಕ ಪ್ರಶ್ನೆಗಳು ಕೇಳಲ್ಪಟ್ಟವು. ನಂತರ ಸಂಭಾಷಣೆಯು ಮಿಲ್, ಹ್ಯಾಮಿಲ್ಟನ್, ಹರ್ಬರ್ಟ್ ಸ್ಪೆನ್ಸರ್ ಮುಂತಾದವರ ತತ್ತ್ವ, ಸಿದ್ಧಾಂತಗಳ ಮೇಲೆ ಜರುಗಿತು. ಸ್ವಾಮಿಗಳು ಎಲ್ಲರಿಗೂ ತೃಪ್ತಿಯಾಗುವಂತೆ ಅವುಗಳ ಮೇಲೆ ವಿಸ್ತಾರವಾಗಿ ಮಾತನಾಡಿದರು. ಅವರ ಪಾಶ್ಚಾತ್ಯ ತತ್ತ್ವಶಾಸ್ತ್ರದ ಪಾಂಡಿತ್ಯವನ್ನೂ, ಅವರು ತಕ್ಷಣವೇ ಕೊಡುತ್ತಿದ್ದ ಉತ್ತರವನ್ನೂ ನೋಡಿ ಪ್ರತಿಯೊಬ್ಬರೂ ವಿಸ್ಮಿತರಾದರು.

ಅನಂತರ ಸ್ವಲ್ಪ ಹೊತ್ತು ವಿವಿಧ ವಿಷಯಗಳ ಮೇಲೆ ಸಂಭಾಷಣೆ ಜರುಗಿ ಸಭೆ ಮುಕ್ತಾಯಗೊಂಡಿತು.

\newpage

\chapter[ಅಧ್ಯಾಯ ೩]{ಅಧ್ಯಾಯ ೩\protect\footnote{\engfoot{} C. W, Vol. V, P. 339}}

\begin{center}
ವರ್ಷ: ಕ್ರಿ.ಶ. ೧೮೯೮ನೇ ಜನವರಿ ೨೪.
\end{center}

ಶನಿವಾರ ಸ್ವಾಮೀಜಿಯನ್ನು ಪ್ರಶ್ನಿಸಿದ ಸಭ್ಯ ಮನುಷ್ಯನೇ ಪುನಃ ಇಂದೂ ಬಂದಿದ್ದನು. ಅವನು ಮತ್ತೆ ಅಂತರ್ಜಾತೀಯ ವಿವಾಹದ ವಿಷಯವನ್ನೇ ಎತ್ತಿ ಅಂತರ್ಜಾತೀಯ ವಿವಾಹ ಪದ್ಧತಿಯು ಜನಾಂಗಗಳಲ್ಲಿ ಜಾರಿಗೆ ಬರುವಂತೆ ಮಾಡುವುದು ಹೇಗೆ ಎಂದನು.

ಸ್ವಾಮೀಜಿ: ಪರಧರ್ಮವನ್ನವಲಂಬಿಸಿದ ಜನಾಂಗದೊಡನೆ ಅಂತರ್‌ಜಾತೀಯ ವಿವಾಹದ ಪದ್ದತಿಯನ್ನು ನಾನು ಸೂಕ್ತವೆಂದು ಹೇಳುವುದಿಲ್ಲ. ಏಕೆಂದರೆ ಸದ್ಯದ ಕಾಲದಲ್ಲಿ ಅದು ನಿಸ್ಸಂದೇಹವಾಗಿ ಸಮಾಜದ ಕಟ್ಟುಪಾಡುಗಳನ್ನು ಸಡಿಲ ಮಾಡುತ್ತದೆ ಮತ್ತು ನಾನಾ ಬಗೆಯ ಹಾನಿಗೂ ಕಾರಣವಾಗುತ್ತದೆ. ಆದ್ದರಿಂದ ಈಗ ಒಂದೇ ಧರ್ಮವನ್ನವಲಂಬಿಸಿರುವ ಜನರಲ್ಲಿ ನಾನು ಈ ಅಂತರ್ಜಾತೀಯ ವಿವಾಹ ಪದ್ದತಿಯಿರಬೇಕೆಂದು ಹೇಳುತ್ತೇನೆ.

ಪ್ರಶ್ನೆ: ಆಗಲೂ ಕೂಡಾ ಅದು ತುಂಬಾ ಗಲಿಬಿಲಿಗೆ ಕಾರಣವಾಗುವುದು. ನನಗೆ ಬಂಗಾಳದಲ್ಲಿ ಹುಟ್ಟಿ ಬೆಳೆದ ಒಬ್ಬ ಮಗಳಿದ್ದಾಳೆಂದೂ ಅವಳನ್ನು ನಾನು ಒಬ್ಬ ಮರಾಠಿ ಅಥವಾ ಮದ್ರಾಸಿನವನಿಗೆ ಮದುವೆ ಮಾಡುತ್ತೇನೆಂದೂ ಇಟ್ಟುಕೊಳ್ಳೋಣ. ಹುಡುಗಿಗೆ ಗಂಡನ ಭಾಷೆ ಅರ್ಥವಾಗುವುದಿಲ್ಲ. ಅಲ್ಲದೆ ಅವರವರ ನಡೆನುಡಿ ಸಂಪ್ರದಾಯಗಳಲ್ಲಿ ತುಂಬಾ ಭಿನ್ನತೆಯಿದೆ. ಮದುವೆಯಾದ ದಂಪತಿಗಳಿಗೆ ಇಷ್ಟು ತೊಂದರೆಗಳಿರುವಾಗ ಇನ್ನು ಸಮಾಜವನ್ನು ತೆಗೆದುಕೊಂಡರೆ ಈ ತೊಡಕು ಮತ್ತೂ ಜಟಿಲವಾಗುವುದು.

ಸ್ವಾಮೀಜಿ: ಈ ಬಗೆಯ ಮದುವೆಗಳು ಹರಡಲು ಇನ್ನೂ ಕೆಲವು ಕಾಲ ಬೇಕು. ಅದೂ ಅಲ್ಲದೆ, ಇದ್ದಕ್ಕಿದ್ದಂತೆಯೇ ಈಗ ಇದಕ್ಕೆ ತೊಡಗುವುದು ಅಷ್ಟೇನೂ ಹಿತವಲ್ಲ. ಕರ್ಮದ ಒಂದು ರಹಸ್ಯವೇನೆಂದರೆ ಎಲ್ಲಿ ತೀರ ಸ್ವಲ್ಪ ಪ್ರತಿಭಟನೆ ಇದೆಯೊ ಆ ಹಾದಿಯಿಂದ ಮುಂದೆ ಹೋಗಬೇಕು. ಆದ್ದರಿಂದ ಮೊದಲು, ತಮ್ಮ ತಮ್ಮ ಸ್ವಂತ ಜಾತಿಯವರ ಕ್ಷೇತ್ರದಲ್ಲೇ ಈ ಮದುವೆ ರೂಢಿಗೆ ಬರಲಿ. ಉದಾಹರಣೆಗೆ ಬಂಗಾಳದ ಕಾಯಸ್ಥರನ್ನು ತೆಗೆದುಕೊಳ್ಳೋಣ. ಅವರಲ್ಲಿ ಉತ್ತರರಾಹ್ರಿ, ದಕ್ಷಿಣರಾಹ್ರಿ, ಬಂಗಜ ಮುಂತಾದ ಹಲವು ಒಳಪಂಗಡಗಳಿವೆ. ಅವರು ತಮ್ಮ ತಮ್ಮಲ್ಲೇ ಪರಸ್ಪರ ವಿವಾಹ ಸಂಬಂಧ ಬೆಳೆಸುವುದಿಲ್ಲ. ಈಗ ಈ ಉತ್ತರರಾಹ್ರಿ, ದಕ್ಷಿಣರಾಹ್ರಿಗಳ ಮಧ್ಯೆ ಪರಸ್ಪರ ವಿವಾಹ ನಡೆಯಲಿ. ಸದ್ಯಕ್ಕೆ ಅದು ಸಾಧ್ಯವಿಲ್ಲದೇ ಹೋದರೆ ಬಂಗಾಳಿಗಳು, ದಕ್ಷಿಣರಾಹ್ರಿಗಳು ಇವರ ನಡುವೆ ಈ ಪದ್ದತಿ ಬರಲಿ. ಹೀಗೆ ಯಾವುದು ಆಗಲೇ ಪ್ರತಿಷ್ಠಿತವಾಗಿದೆಯೋ ಮತ್ತು ಯಾವುದನ್ನು ಜಾರಿಗೆ ತರುವುದು ನಮ್ಮ ಕೈಯಲ್ಲೇ ಇದೆಯೋ ಅದನ್ನು ನಾವು ಬೆಳೆಸಬೇಕು. ಸುಧಾರಣೆ ಎಂದರೆ ಸಂಪೂರ್ಣವಾಗಿ ಧ್ವಂಸಮಾಡುವುದು ಎಂದರ್ಥವಲ್ಲ.

ಪ್ರಶ್ನೆ: ಸರಿ, ನೀವು ಹೇಳಿದಂತೆಯೇ ಆಗಲಿ. ಆದರೆ ಅದರಿಂದ ಯಾವ ಹಿತವಾಗುವುದು?

ಸ್ವಾಮೀಜಿ: ನೀನು ನೋಡುತ್ತಿಲ್ಲವೆ – ನಮ್ಮ ಸಮಾಜದಲ್ಲಿ ನೂರಾರು ವರ್ಷಗಳಿಂದಲೂ ಮದುವೆ ಪ್ರತಿಯೊಂದು ಮತದ ಒಳಪಂಗಡಗಳಲ್ಲಿಯೇ ನಡೆದು ಎಂತಹ ಅವಸ್ಥೆಗೆ ಬಂದಿದೆಯೆಂದರೆ ಮದುವೆ ಎಂದರೆ ಭ್ರಾತೃವರ್ಗದವರು, ಬಂಧುವರ್ಗದವರಿಗೇ ಇದು ಮೀಸಲಾಗಿಹೋಗಿದೆ. ಈ ಕಾರಣದಿಂದಲೇ ಜನಾಂಗ ಶಾರೀರಿಕ ದುರ್ಬಲತೆಯನ್ನು ಹೊಂದುತ್ತಿದೆ. ಇದರ ಪರಿಣಾಮವಾಗಿ ಎಲ್ಲಾ ಬಗೆಯ ರೋಗರುಜಿನಗಳು ಮತ್ತು ಇತರ ಕೆಡಕುಗಳು ಸುಲಭವಾಗಿ ಪ್ರವೇಶ ಮಾಡುತ್ತಿವೆ. ಕೆಲವೇ ಮಂದಿ ವ್ಯಕ್ತಿಗಳ ಸಣ್ಣ ವೃತ್ತದೊಳಗೇ ರಕ್ತಚಲನೆಯಾಗಬೇಕಾಯಿತು. ಅದರಿಂದ ರಕ್ತ ಕಲುಷಿತವಾಗಿದೆ. ಆದ್ದರಿಂದ ಹೊಸದಾಗಿ ಹುಟ್ಟಿದ ಮಕ್ಕಳು ತಮ್ಮ ತಂದೆ ತಾಯಿಗಳಿಂದ ಹುಟ್ಟುತ್ತಲೇ ಅವರ ದೇಹ ಪ್ರಕೃತಿಯ ರೋಗಗಳನ್ನು ಪಿತ್ರಾರ್ಜಿತವಾಗಿ ಹೊಂದುತ್ತಾರೆ. ಹೀಗೆ ರಿಕ್ತ ರಕ್ತದಿಂದ ಹುಟ್ಟಿದ ಅವರ ದೇಹಗಳಲ್ಲಿ ಯಾವ ರೋಗದ ಸೂಕ್ಷ್ಮಕ್ರಿಮಿಗಳನ್ನೂ ತಡೆಗಟ್ಟುವ ಶಕ್ತಿ ಇರುವುದಿಲ್ಲ. ಈ ರೋಗಗಳಾದರೋ ಯಾವಾಗಲೂ ಅವರನ್ನು ಬಲಿ ತೆಗೆದುಕೊಳ್ಳಲು ಕಾದುಕೊಂಡಿರುತ್ತವೆ. ಈ ಮದುವೆಯ ವೃತ್ತವನ್ನು ವಿಶಾಲವಾಗಿ ಮಾಡುವುದರ ಮೂಲಕ ಮಾತ್ರ ನಾವು ಹೊಸ ಬಗೆಯ ಬೇರೆ ವಿಧವಾದ ರಕ್ತವನ್ನು ನಮ್ಮ ಸಂತಾನಕ್ಕೆ ತುಂಬಬಹುದು. ಅದರಿಂದ ಅವರು ಈಗಿನ ಕಾಲದ ನಾನಾ ಬಗೆಯ ರೋಗಗಳಿಂದ ಮತ್ತು ಅವುಗಳ ಕೆಟ್ಟ ಪರಿಣಾಮಗಳಿಂದ ಪಾರಾಗಬಹುದು.

ಪ್ರಶ್ನೆ: ಮಹಾಶಯರೆ, ಬಾಲ್ಯವಿವಾಹದ ವಿಚಾರವಾಗಿ ನಿಮ್ಮ ಅಭಿಪ್ರಾಯವೇನೆಂದು ಕೇಳಬಹುದೆ?

ಸ್ವಾಮೀಜಿ: ಬಂಗಾಳದ ವಿದ್ಯಾವಂತರಲ್ಲಿ ಈ ರೀತಿ ಹುಡುಗರಿಗೆ ಬಹುಬೇಗ ಮದುವೆ ಮಾಡಿಬಿಡುವ ಪದ್ಧತಿ ಕ್ರಮೇಣ ಕಡಿಮೆಯಾಗುತ್ತಿದೆ. ಹುಡುಗಿಯರನ್ನು ಕೂಡ ಮೊದಲಿಗಿಂತ ಒಂದೆರಡು ವರ್ಷ ದೊಡ್ಡವರಾದ ಮೇಲೆ ಮದುವೆ ಮಾಡುತ್ತಾರೆ. ಆದರೆ ಒತ್ತಡದಿಂದ, ಹಣದ ಅಭಾವದಿಂದ ಹೀಗೆ ಆಗುತ್ತಿದೆ. ಅದಕ್ಕೆ ಕಾರಣ ಏನು ಬೇಕಾದರೂ ಇರಲಿ, ಹುಡುಗಿಯ ಮದುವೆ ವಯಸ್ಸು ಈಗಿಗಿಂತ ತುಂಬಾ ಹೆಚ್ಚಿರಬೇಕು. ಆದರೆ ಬಡತಂದೆ ಏನು ಮಾಡುವನು? ಹುಡುಗಿ ಕೊಂಚ ಬೆಳೆದರೆ ಸಾಕು ತಾಯಿತಂದೆಗಳಿಂದ ಹಿಡಿದು ಎಲ್ಲಾ ಬಂಧುಬಳಗ ನೆರೆಹೊರೆಯ ಹೆಂಗಸರೂ ಒಬ್ಬ ವರನನ್ನು ಹುಡುಕಬೇಕೆಂದು ಯಜಮಾನನನ್ನು ಕಾಡಿ ಕೊನೆಗೂ ಅವನು ಹಾಗೆ ಮಾಡುವವರೆಗೆ ಅವನನ್ನು ಸುಮ್ಮನೆ ಬಿಡುವುದಿಲ್ಲ. ಇನ್ನು ಕಪಟ ಪುರೋಹಿತರ ವಿಷಯ ಎಷ್ಟು ಕಡಿಮೆ ಹೇಳಿದರೆ ಅಷ್ಟುವಾಸಿ. ಈಗಿನ ಕಾಲದಲ್ಲಿ ಅವರ ಮಾತನ್ನಾರೂ ಕೇಳುವುದಿಲ್ಲ. ಆದರೂ ಅವರು ತಾವೇ ಆ ನಾಯಕ ಪಟ್ಟವನ್ನು ವಹಿಸುವರು. ಹನ್ನೆರಡುವರ್ಷದ ಹುಡುಗಿಯೊಡನೆ ಬೆರೆತರೆ ಜುಲ್ಮಾನೆಯಾಗುವುದೆಂದು ಗಂಡಸನ್ನು ಪ್ರತಿಬಂಧಿಸುವ ‘ವಯಸ್ಸಿನ ಒಪ್ಪಿಗೆಯ ಆಜ್ಞೆ’ \enginline{(age of consent)} ಯನ್ನು ರಾಜರು ಹೊರಡಿಸಿದಾಗ ತಕ್ಷಣವೇ ಈ ನಿಮ್ಮ ಧರ್ಮದ ಮಾರ್ಗದರ್ಶಕರೆಲ್ಲಾ ಅದಕ್ಕೆ ವಿರೋಧವಾಗಿ ಅಬ್ಬರವನ್ನೆಬ್ಬಿಸಿ “ಅಯ್ಯೋ ನಮ್ಮ ಧರ್ಮ ಹಾಳಾಯಿತು!” ಎಂದು ಕೂಗಾಡಿದರು. ಹನ್ನೆರಡು ಅಥವಾ ಹದಿಮೂರು ವರುಷದ ಹುಡುಗಿಯನ್ನು ತಾಯಿಯಾಗಿ ಮಾಡುವುದರಲ್ಲೇ ಧರ್ಮದ ಸಾರವೆಲ್ಲ ಹುದುಗಿಕೊಂಡಿರುವುದಂತೆ! ಅದಕ್ಕೆ ನಮ್ಮನ್ನಾಳುವವರು ‘ಇದು ಎಂತಹ ವಿಚಿತ್ರ! ಇವರದು ಎಂತಹ ಧರ್ಮ!’ ಎಂದು ಸಹಜವಾಗಿ ಯೋಚಿಸುವರು, ಮತ್ತು ಇಂತಹ ಜನರು ರಾಜಕೀಯ ಚಳುವಳಿಯನ್ನು ನಡೆಸುವರು ಮತ್ತು ರಾಜಕೀಯ ಹಕ್ಕುಗಳಿಗಾಗಿ ತಗಾದೆ ಮಾಡುವರು!

ಪ್ರಶ್ನೆ: ಹಾಗಾದರೆ ನಿಮ್ಮ ಅಭಿಪ್ರಾಯವು ಗಂಡಸರು ವಯಸ್ಸಾದ ಮೇಲೆ ಮದುವೆಯಾಗಬೇಕೆಂದೆ?

ಸ್ವಾಮೀಜಿ: ನಿಸ್ಸಂದೇಹವಾಗಿ! ಆದರೆ ಅದರೊಡನೆಯೇ ವಿದ್ಯಾಭ್ಯಾಸವನ್ನೂ ಕೊಡಿಸಬೇಕು. ಇಲ್ಲದಿದ್ದಲ್ಲಿ ಅಕ್ರಮ ಭ್ರಷ್ಟತೆಗಳೆಲ್ಲಾ ತಲೆದೋರುವುವು. ವಿದ್ಯಾಭ್ಯಾಸವೆಂದರೆ ಈಗಿನ ಕಾಲದಲ್ಲಿರುವ ಪದ್ಧತಿಯಂತಲ್ಲ. ಅನುಷ್ಠಾನ ಯೋಗ್ಯವಾದ ಶಿಕ್ಷಣ ಕೊಡಿಸಬೇಕು. ಕೇವಲ ಪುಸ್ತಕ ಪಾಂಡಿತ್ಯ ಪ್ರಯೋಜನವಿಲ್ಲ. ಯಾವ ಬಗೆಯ ಶಿಕ್ಷಣದಿಂದ ಶೀಲ ರೂಪುಗೊಳ್ಳುವುದೊ, ಮನಶ್ಶಕ್ತಿ ವೃದ್ಧಿಯಾಗುವುದೊ, ಬುದ್ಧಿ ವಿಕಾಸಗೊಳ್ಳುವುದೊ ಮತ್ತು ಯಾವುದರಿಂದ ನಮ್ಮ ಸ್ವಂತ ಕಾಲ ಮೇಲೆ ನಾವು ನಿಂತುಕೊಳ್ಳಬಲ್ಲೆವೊ ಅಂತಹ ಶಿಕ್ಷಣ ನಮಗೆ ಆವಶ್ಯಕ.

ಪ್ರಶ್ನೆ: ನಮ್ಮ ಹೆಂಗಸರಿಗೆ ಅನೇಕ ಸುಧಾರಣೆಗಳು ಆವಶ್ಯಕ.

ಸ್ವಾಮೀಜಿ: ಈ ಬಗೆಯ ಶಿಕ್ಷಣದಿಂದ ಹೆಂಗಸರು ತಮ್ಮ ಸಮಸ್ಯೆಗಳನ್ನು ತಾವೇ ಬಗೆಹರಿಸಿಕೊಳ್ಳುವರು. ಅವರು ಮೊದಲಿನಿಂದ ಕಡೆಯವರೆಗೂ ಕೇವಲ ಅಸಹಾಯಕತೆ ತೋರುವುದರಲ್ಲಿ ಮತ್ತು ಇತರರನ್ನು ಯಾವಾಗಲೂ ನೆಚ್ಚಿಕೊಂಡಿರುವುದರಲ್ಲಿ ನಿಪುಣರು. ಆದ್ದರಿಂದ ಕೊಂಚ ಅಪಾಯ ಅಥವಾ ಒಂದು ದುರ್ಘಟನೆಯೊದಗಿದರೆ. ಸಾಕು, ಚೆನ್ನಾಗಿ ಅಳುವುದೊಂದು ಅವರಿಗೆ ಬರುತ್ತದೆ. ಇತರ ವಿಷಯಗಳ ಜೊತೆಗೆ ಅವರು ಪರಾಕ್ರಮ ಮತ್ತು ಶೌರ್ಯವನ್ನು ಕಲಿಯಬೇಕು. ಈ ವರ್ತಮಾನ ಕಾಲದಲ್ಲಿ ಅವರು ಆತ್ಮರಕ್ಷಣೆಯ ಪಾಠವನ್ನೂ ಕಲಿಯುವುದು ಅತ್ಯಾವಶ್ಯಕವಾಗಿದೆ. ಝಾನ್ಸಿರಾಣಿಯು ಒಂದು ಹೊಸ ಆದರ್ಶ, ನಮ್ಮ ಹೆಂಗಸರನ್ನು ಆ ರೀತಿ ತರಬೇತು ಮಾಡಲು ಬಹಳ ಕಾಲ ಹಿಡಿಯುವುದೆಂದೆನ್ನಿಸುವುದು.

“ಹೇಗಾದರಾಗಲಿ, ನಾವು ಶಕ್ತಿಮೀರಿ ಪ್ರಯತ್ನಿಸಬೇಕು. ನಾವು ಅವರಿಗೆ ಕಲಿಸುವುದೊಂದೇ ಅಲ್ಲ. ನಾವೂ ಕಲಿತುಕೊಳ್ಳಬೇಕು. ಕೇವಲ ಮಕ್ಕಳನ್ನು ಹುಟ್ಟಿಸಿದರೆ ತಂದೆಯಾಗುವುದಿಲ್ಲ. ನಮ್ಮ ಹೆಗಲ ಮೇಲೆ ಅತ್ಯಂತ ದೊಡ್ಡ ಜವಾಬ್ದಾರಿಗಳನ್ನು ತೆಗೆದುಕೊಳ್ಳಬೇಕಾಗುತ್ತದೆ. ಸ್ತ್ರೀ ವಿದ್ಯಾಭ್ಯಾಸ ಪ್ರಾರಂಭಿಸಬೇಕು. ನಮ್ಮ ಹಿಂದೂ ಸ್ತ್ರೀಯರಿಗೆ ಪಾತಿವ್ರತ್ಯ ಎಂದರೇನೆಂಬುದು ಚೆನ್ನಾಗಿ ಗೊತ್ತು. ಏಕೆಂದರೆ ಅದು ಅವರಿಗೆ ಪಿತ್ರಾರ್ಜಿತವಾಗಿ ಬಂದುದು. ಈಗ ಮೊಟ್ಟ ಮೊದಲು ಅವರಲ್ಲಿ ಎಲ್ಲಕ್ಕಿಂತ ಹೆಚ್ಚಾಗಿ ಈ ಧ್ಯೇಯವನ್ನು ಗಾಢವಾಗಿ ಬೇರೂರುವಂತೆ ಮಾಡಿ. ಇದರಿಂದ ಅವರು ಶುದ್ಧ ಚಾರಿತ್ರ್ಯವನ್ನು ರೂಢಿಸಿಕೊಳ್ಳಬಹುದು. ಇದರ ಶಕ್ತಿಯಿಂದ ಅವರ ಜೀವನದ ಎಲ್ಲಾ ಹಂತಗಳಲ್ಲಿಯೂ, ಅವರು ಮದುವೆಯಾಗಿರಲಿ ಅಥವಾ ಏಕಾಂಗಿಯಾಗಿರಲಿ, ಅವರು ತಮ್ಮ ಮಾನಕ್ಕೆ ಧಕ್ಕೆ ಬರುವ ಸಂದರ್ಭ ಬಂದಾಗ ತಮ್ಮ ಪ್ರಾಣವನ್ನೇ ತ್ಯಾಗ ಮಾಡಲು ಕೊಂಚವೂ ಹಿಂಜರಿಯುವುದಿಲ್ಲ. ತಮ್ಮ ಧ್ಯೇಯಕ್ಕಾಗಿ, ಅದು ಯಾವುದೇ ಆಗಿರಲಿ, ತಮ್ಮ ಸ್ವಂತ ಜೀವವನ್ನೇ ತ್ಯಾಗ ಮಾಡುವುದು ಅದೇನು ಕಡಿಮೆ ವೀರತ್ವವಾಯಿತೆ? ಈ ಯುಗದ ಆವಶ್ಯಕತೆಯನ್ನು ಸರಿಯಾಗಿ ವೀಕ್ಷಿಸಿದರೆ ಕೆಲವರನ್ನು ಈ ತ್ಯಾಗದ ಧ್ಯೇಯದಲ್ಲಿ ತರಬೇತು ಮಾಡುವುದು ಅನಿವಾರ್ಯವಾಗಿದೆ. ಅವರು ಆಜೀವಪರ್ಯಂತ ಕನ್ಯೆಯರಾಗೇ ಉಳಿವ ವ್ರತದಲ್ಲಿ ತೊಡಗುವರು. ಪ್ರಾಚೀನ ಸಂಪ್ರದಾಯದಿಂದ ಅವರು ತಮ್ಮಲ್ಲಿ ಅಂತರ್ಗತವಾಗಿರುವ ಸತೀತ್ವದ ಶಕ್ತಿಯಿಂದ ಪ್ರಜ್ವಲಿಸುತ್ತಿರುವರು. ಅದರ ಜೊತೆಗೆ ಅವರಿಗೆ ವಿಜ್ಞಾನ ಮತ್ತಿತರ ವಿಷಯಗಳನ್ನು ತಿಳಿಸಿಕೊಡಬೇಕು. ಇದರಿಂದ ಅವರಿಗೆ ಉಪಯೋಗವಾಗುವುದಲ್ಲದೆ ಇತರರಿಗೂ ಉಪಯೋಗವಾಗುವುದು. ಇದನ್ನು ತಿಳಿಯುವುದರಿಂದ ಅವರು ಅವುಗಳನ್ನು ಕಲಿಯುವರು, ಮತ್ತು ಅದನ್ನು ಅನುಸರಿಸುವುದರಲ್ಲಿ ಆನಂದಿಸುವರು. ಇಂತಹ ಪವಿತ್ರಾತ್ಮರಾದ ಹಲವು ಬ್ರಹ್ಮಚಾರಿಗಳು, ಬ್ರಹ್ಮಚಾರಿಣಿಯರು ನಮ್ಮ ತಾಯಿನಾಡಿನ ಕಲ್ಯಾಣಕ್ಕೆ ಈಗ ಅತ್ಯಾವಶ್ಯಕ.”

ಪ್ರಶ್ನೆ: ಇದು ಯಾವ ಬಗೆಯಿಂದ ದೇಶದ ಕಲ್ಯಾಣಕ್ಕೆ ಸಹಕಾರಿಯಾಗುವುದು?

ಸ್ವಾಮೀಜಿ: ಜನಾಂಗದ ಧ್ಯೇಯವನ್ನು ತಮ್ಮ ನಿದರ್ಶನದಿಂದಲೂ ಪರಿಶ್ರಮದಿಂದಲೂ ಜನರೆದುರಿಗೆ ಇಟ್ಟರೆ ಭಾವಕ್ರಾಂತಿಯುಂಟಾಗುವುದು. ಈಗ ವಸ್ತುಸ್ಥಿತಿ ಹೇಗಿದೆ? ತಾಯ್ತಂದೆಗಳು ಮಗಳಿಗೆ ಒಂಬತ್ತು ವರ್ಷವಾಗಿರಲಿ, ಹತ್ತಾಗಿರಲಿ, ಹೇಗಾದರೂ ಮದುವೆ ಮಾಡಿ ದಾಟಿಸಬೇಕು! ಅವಳಿಗೆ ಹದಿಮೂರನೇ ವರ್ಷದಲ್ಲೇ ಒಂದು ಮಗುವೂ ಆಗಿಬಿಟ್ಟರೆ ಇಡೀ ಸಂಸಾರದವರ ಆನಂದಕ್ಕೆ ಪಾರವೇ ಇಲ್ಲ. ಈ ಬಗೆಯ ಅಭಿಪ್ರಾಯಗಳೆಲ್ಲಾ ತಲೆಕೆಳಗಾದರೆ ನಮ್ಮ ಪೂರ್ವದ ಶ್ರದ್ಧೆ ಪುನಃ ಬರುವುದೆಂದು ಕೊಂಚ ಭರವಸೆ ಹುಟ್ಟುತ್ತದೆ. ಮೇಲೆ ಹೇಳಿದ ಬ್ರಹ್ಮಚರ್ಯ ಸಾಧನೆ ಮಾಡುವವರ ವಿಚಾರವನ್ನಂತೂ ಹೇಳಬೇಕಾದ್ದೇ ಇಲ್ಲ. ಅವರಿಗೆ ತಮ್ಮಲ್ಲೇ ಎಷ್ಟು ಆತ್ಮವಿಶ್ವಾಸವಿರುವುದೆಂಬುದನ್ನು ಯೋಚಿಸು. ಅವರು ಶ್ರೇಯಸ್ಸಿಗೆ ಎಂತಹ ಪ್ರೇರಕ ಶಕ್ತಿಯಾಗುವರು?

ಪ್ರಶ್ನಿಸಿದವನು ಸ್ವಾಮೀಜಿಗೆ ಪ್ರಣಾಮ ಮಾಡಿ ಹೊರಡಲು ಸಿದ್ಧನಾದನು. ಸ್ವಾಮೀಜಿ ಅವನನ್ನು ಆಗಾಗ್ಗೆ ಬರುವಂತೆ ಹೇಳಿದರು. “ನಿಸ್ಸಂದೇಹವಾಗಿ ಬರುವೆ ಮಹಾಶಯರೆ” ಎಂದು ಆತನು ಹೇಳಿ “ನನಗೆ ತುಂಬಾ ಸಹಾಯವಾಯಿತು. ಮತ್ತೆಲ್ಲಿಯೂ ಹಿಂದೆ ನಾನು ಕೇಳದೆ ಇದ್ದ ಅನೇಕ ಹೊಸ ವಿಷಯಗಳನ್ನೆಲ್ಲಾ ನಾನು ನಿಮ್ಮಿಂದ ಕೇಳಿದೆ” ಎಂದನು. ಆಗ ಊಟದ ವೇಳೆಯಾಗಿದ್ದುದರಿಂದ ನಾನೂ ಮನೆಗೆ ಹೋದೆ.

\newpage

\chapter[ಅಧ್ಯಾಯ ೪]{ಅಧ್ಯಾಯ ೪\protect\footnote{\engfoot{C.W, Vol. V, P 344}}}

\begin{center}
ವರ್ಷ: ಕ್ರಿ.ಶ. ೧೮೯೮ ಜನವರಿ ೨೪.
\end{center}

ಮಧ್ಯಾಹ್ನ ಸ್ವಾಮೀಜಿಯನ್ನು ನೋಡಲು ನಾನು ಪುನಃ ಹೋದೆ. ಅವರ ಸುತ್ತಲೂ ಸುಮಾರು ಮಂದಿ ನೆರೆದಿದ್ದರು. ಶ‍್ರೀಚೈತನ್ಯರ ಅನುಯಾಯಿಗಳಲ್ಲಿ ರೂಢಿಯಲ್ಲಿರುವ ಮಧುರಭಾವ ಅಥವಾ ಭಗವಂತನನ್ನು ಪತಿಯಂತೆ ಭಾವಿಸಿ ಪೂಜಿಸುವ ವಿಷಯವನ್ನು ಕುರಿತು ಮಾತನಾಡುತ್ತಿದ್ದರು. ಅವರು ಆಗಾಗ್ಗೆ ಮಾಡುತ್ತಿದ್ದ ಹಾಸ್ಯದಿಂದ ನಗುವಿನ ಲಹರಿ ಏಳುತ್ತಿತ್ತು. ಯಾರೋ ಹೀಗೆ ಟೀಕಿಸಿದರು: “ಚೈತನ್ಯನ ಕಾರ್ಯಗಳನ್ನು ಕುರಿತು ಇಷ್ಟೊಂದು ತಮಾಷೆ ಮಾಡಲು ಏನಿದೆ? ಹಾಗಾದರೆ ಅವನೊಬ್ಬ ದೊಡ ಮಹಾತ್ಮನಲ್ಲ, ಜಗತ್ಕಲ್ಯಾಣಕ್ಕೆ ಅವನೇನೂ ಮಾಡಲಿಲ್ಲವೆಂದು ತಿಳಿದಿರುವಿರೇನು?”

ಸ್ವಾಮೀಜಿ: ಯಾರದು! ಹಾಗಾದರೆ, ಪ್ರಿಯಮಹಾಶಯ ನಿಮ್ಮನ್ನು ತಮಾಷೆ ಮಾಡಲೇನು! ನೀವು ಕೇವಲ ತಮಾಷೆಯನ್ನು ನೋಡುವಿರಿ ಹೌದೇನು? ಮಹಾಶಯ, ಶ‍್ರೀಚೈತನ್ಯರ ಕಾಮಕಾಂಚನ ತ್ಯಾಗದ ಪ್ರಜ್ವಲಿಸುವ ಆದರ್ಶದಲ್ಲಿ ನನ್ನ ಜೀವನವನ್ನು ರೂಪಿಸಿಕೊಳ್ಳಬೇಕೆಂದು ನಾನು ಅನುಭವಿಸಿದ ನನ್ನ ಜೀವಾವಧಿಯ ಹೋರಾಟವನ್ನು ನೀವು ಗಮನಿಸುವುದಿಲ್ಲ. ಈ ಧ್ಯೇಯವನ್ನು ಜನಸಮುದಾಯದಲ್ಲಿ ಹರಡಲು ನಾನು ಪಟ್ಟ ಪರಿಶ್ರಮವನ್ನೂ ಎಣಿಸುವುದಿಲ್ಲ! ಶ‍್ರೀಚೈತನ್ಯರು ಪ್ರಚಂಡ ತ್ಯಾಗಮೂರ್ತಿಗಳು. ಹೆಂಗಸು ಮತ್ತು ಇಂದ್ರಿಯ ಸುಖದ ಅಪೇಕ್ಷೆ ಅವರಿಗೆ ಲವಲೇಶವೂ ಇರಲಿಲ್ಲ. ಆದರೆ ಕಾಲಕ್ರಮೇಣ ಅವರ ಶಿಷ್ಯರು ಸ್ತ್ರೀಯರನ್ನು ತಮ್ಮ ಪಂಥಕ್ಕೆ ಸೇರಲು ಎಡೆ ಕೊಟ್ಟು ತಾರತಮ್ಯ ಭಾವನೆಯಿಲ್ಲದೆ ದೇವರ ಹೆಸರಿನಲ್ಲಿ ಅವರೊಡನೆ ಕಲೆತು ಎಲ್ಲವನ್ನೂ ಸಂಪೂರ್ಣವಾಗಿ ಹೊಲಸು ಮಾಡಿ ಅವ್ಯವಸ್ಥೆ ಮಾಡಿದರು? ಶ‍್ರೀಚೈತನ್ಯನು ತನ್ನ ಜೀವನವನ್ನೇ ದೃಷ್ಟಾಂತವಾಗಿ ಮಾಡಿಟ್ಟು ತೋರಿದ ಆ ಆದರ್ಶಪ್ರೇಮ ಸಂಪೂರ್ಣವಾಗಿ ಸ್ವಾರ್ಥ ರಹಿತವಾದುದು. ಇಂದ್ರಿಯಾಸಕ್ತಿ ಲೇಶಮಾತ್ರವೂ ಇಲ್ಲದ್ದು. ಕಾಮರಹಿತವಾದ ಆ ಪ್ರೇಮ ಎಂದಿಗೂ ಜನಸಾಮಾನ್ಯರ ಸ್ವತ್ತಾಗಲಾರದು. ಅನಂತರ ಬಂದ ವೈಷ್ಣವ ಗುರುಗಳು ತಮ್ಮ ಗುರುದೇವನ ಜೀವನದ ತ್ಯಾಗ ದೃಷ್ಟಿಯನ್ನು ಎತ್ತಿ ಹಿಡಿದು ಮೊದಲು ಅದಕ್ಕೆ ಪ್ರತ್ಯೇಕವಾದ ಮಹತ್ವವನ್ನು ಕೊಡುವ ಬದಲು ಆ ಪ್ರೇಮಾದರ್ಶವನ್ನು ಜನಸಮುದಾಯದಲ್ಲಿ ತುಂಬಲು ತಮ್ಮ ಉತ್ಸಾಹವನ್ನೆಲ್ಲಾ ವಿನಿಯೋಗಿಸಿದರು. ಪರಿಣಾಮವಾಗಿ ಜನಸಾಮಾನ್ಯರು ಈ ದೈವೀಪ್ರೇಮದ ಉಚ್ಚ ಆದರ್ಶವನ್ನು ರಕ್ತಗತಮಾಡಿಕೊಂಡು ಗ್ರಹಿಸಲಾರದೇ ಹೋದರು. ಸ್ವಾಭಾವಿಕವಾಗಿ ಅದನ್ನು ಸ್ತ್ರೀಪುರುಷರ ಮಧ್ಯೆ ಇರುವ ಅತ್ಯಂತ ಕೀಳುತರದ ಪ್ರೇಮವಾಗಿ ಮಾರ್ಪಡಿಸಿದರು.

ಪ್ರಶ್ನೆ: ಆದರೆ ಮಹಾಶಯರೆ, ಶ‍್ರೀಚೈತನ್ಯರು ಹರಿನಾಮವನ್ನು ಎಲ್ಲರಿಗೂ ಚಂಡಾಲರಿಗೂ ಕೂಡ ಬೋಧಿಸಿದರು. ಹೀಗಿದ್ದ ಪಕ್ಷಕ್ಕೆ ಜನಸಾಮಾನ್ಯರಿಗೂ ಇದನ್ನು ಹೊಂದಲು ಏಕೆ ಹಕ್ಕಿರಬಾರದು?

ಸ್ವಾಮೀಜಿ: ನಾನು ಅವರ ಉಪದೇಶದ ವಿಚಾರ ಮಾತನಾಡುತ್ತಿಲ್ಲ. ಅವರ ಪ್ರೇಮದ ಉಚ್ಚ ಆದರ್ಶವನ್ನು – ಯಾವ ರಾಧಾಪ್ರೇಮದಲ್ಲಿ ಹಗಲೂ ರಾತ್ರಿ ತಮ್ಮ ವ್ಯಕ್ತಿತ್ವವನ್ನೆಲ್ಲಾ ಲೀನಮಾಡಿ ಸಂಪೂರ್ಣವಾಗಿ ತನ್ಮಯರಾಗುತ್ತಿದ್ದರೋ ಅಂತಹ ಉಚ್ಚವಾದ ಆದರ್ಶ ಪ್ರೇಮವನ್ನು ಕುರಿತು ಮಾತನಾಡುತ್ತಿದ್ದೇನೆ.

ಪ್ರಶ್ನೆ: ಏಕೆ ಅದನ್ನು ಎಲ್ಲರ ಸ್ವತ್ತನ್ನಾಗಿ ಮಾಡಬಾರದು?

ಸ್ವಾಮೀಜಿ: ನಮ್ಮ ಜನಾಂಗವನ್ನು ನೋಡು. ಆ ಪ್ರಯತ್ನದ ಪರಿಣಾಮ ಏನಾಗಿದೆಯೆಂಬುದನ್ನು ನೋಡು. ಆ ಪ್ರೇಮ ಪ್ರಸಾರದ ಬೋಧನೆಯಿಂದ ಇಡೀ ಜನಾಂಗವೇ ಸ್ತ್ರೀಲಂಪಟವಾಗಿದೆ – ಹೆಣ್ಣಿಗಜಾತಿಯಾಗಿದೆ! ಇಡೀ ಒರಿಸ್ಸಾ ದೇಶವೇ ಹೇಡಿಗಳ ದೇಶವಾಗಿ ಮಾರ್ಪಟ್ಟಿದೆ. ಬಂಗಾಳವು ರಾಧಾಪ್ರೇಮವನ್ನು ನಾನೂರು ವರ್ಷಗಳಿಂದಲೂ ಅನುಸರಿಸಿ ಈಗ ತನ್ನ ಪುರುಷತ್ವದ ಪರಿಜ್ಞಾನವನ್ನೇ ಸಂಪೂರ್ಣವಾಗಿ ಕಳೆದುಕೊಂಡಿದೆ! ಜನ ಕೇವಲ ಅಳುವುದರಲ್ಲಿ, ಪ್ರಲಾಪಿಸುವುದರಲ್ಲಿ ಬಹು ಜಾಣರಾಗಿದ್ದಾರೆ. ಅದೇ ಅವರ ಜನಾಂಗದ ವೈಶಿಷ್ಟ್ಯವಾಗಿಬಿಟ್ಟಿದೆ. ಅವರ ಸಾಹಿತ್ಯವನ್ನು ನೋಡಿ, ಜನಾಂಗದ ವಿಚಾರ ಶಕ್ತಿಯ, ಉದ್ದೇಶಗಳ ಖಂಡಿತವಾದ ಸಂಕೇತವದು. ಏಕೆ, ಈ ನಾನೂರು ವರ್ಷಗಳ ಬಂಗಾಳಿ ಸಾಹಿತ್ಯದ ಹಾಡಿನ ಪಲ್ಲವಿಗಳು ಅದೇ ಬಗೆಯ ರೋದನ, ಗೋಳಿನ ಧ್ವನಿಯಿಂದ ಉದ್ರಿಕ್ತವಾಗಿದೆ. ನಿಜವಾದ ವೀರಸ್ಫೂರ್ತಿಯನ್ನು ಪ್ರಚೋದಿಸುವ ಕಾವ್ಯಕ್ಕೆ ಜನ್ಮ ಕೊಡುವುದರಲ್ಲಿ ಅದು ಪೂರ್ಣವಾಗಿ ಸೋತಿದೆ.

ಪ್ರಶ್ನೆ: ಹಾಗಾದರೆ ನಿಜವಾಗಿ ಆ ಪ್ರೇಮವನ್ನು ಹೊಂದಲು ಅರ್ಹತೆಯುಳ್ಳವರಾರು?

ಸ್ವಾಮೀಜಿ: ಎಲ್ಲಿಯವರೆಗೆ ಹೃದಯದಲ್ಲಿ ಲವಲೇಶವಾದರೂ ಕಾಮ ವಿರುವದೊ, ಅಲ್ಲಿಯವರೆಗೂ ಈ ಪ್ರೇಮವಿರುವುದಿಲ್ಲ. ಯಾರು ಮಹಾತ್ಯಾಗಿಗಳೊ, ಮನುಷ್ಯರಲ್ಲಿ ಪ್ರಚಂಡಪುರುಷರೊ, ಅಂತಹವರಿಗೆ ಮಾತ್ರ ಈ ದೈವೀಪ್ರೇಮಕ್ಕೆ ಹಕ್ಕಿದೆ. ಜನಸಮುದಾಯಕ್ಕೆ ಈ ಉಚ್ಛ ಪ್ರೇಮದ ಆದರ್ಶವನ್ನು ತೋರಿಸಿದರೆ ಅದು ಪರೋಕ್ಷವಾಗಿ ಮನುಷ್ಯರ ಹೃದಯದಲ್ಲಿ ಪ್ರಬಲವಾದ ಪ್ರಾಪಂಚಿಕ ಆಸೆಗಳನ್ನು ಉದ್ದೀಪನಗೊಳಿಸುವುದು. ಏಕೆಂದರೆ ನಮ್ಮನ್ನು ಭಗವಂತನ ಪತ್ನಿ ಅಥವಾ ಪ್ರಿಯತಮಳೆಂದು ಯೋಚಿಸುತ್ತಾ ಆತನ ಪ್ರೇಮವನ್ನು ಕುರಿತು ಧ್ಯಾನ ಮಾಡುವಾಗ ಸಹಜವಾಗಿ ಮನಸ್ಸು ಮುಕ್ಕಾಲುಪಾಲು ಹೊತ್ತೆಲ್ಲಾ ತನ್ನ ಸ್ವಂತ ಹೆಂಡತಿಯ ವಿಚಾರ ಯೋಚಿಸುವಂತಾಗುವುದು – ಪರಿಣಾಮ ಹೇಳಬೇಕಾದ್ದೆ ಇಲ್ಲ.

ಪ್ರಶ್ನೆ: ಹಾಗಾದರೆ ಗೃಹಸ್ಥರು ದೇವರನ್ನು ಪ್ರೇಮದ ಹಾದಿಯಲ್ಲಿ ಪತಿ ಅಥವಾ ಪ್ರಿಯತಮನೆಂದು ಪೂಜಿಸಿ ತಮ್ಮನ್ನು ಆತನ ಸತಿಯಂತೆ ಭಾವಿಸಿ ಭಗವತ್ಸಾಕ್ಷಾತ್ಕಾರ ಪಡೆಯುವುದು ಅಸಾಧ್ಯವೆ?


ಸ್ವಾಮೀಜಿ: ಎಲ್ಲೋ ಕೆಲವರ ವಿನಃ ಸಾಮಾನ್ಯ ಗೃಹಸ್ಥರಿಗೆ ನಿಸ್ಸಂದೇಹವಾಗಿ ಇದು ಅಸಾಧ್ಯ. ಮತ್ತೆ ಎಲ್ಲಕ್ಕಿಂತ ಹೆಚ್ಚಾಗಿ ಈ ಸೂಕ್ಷ್ಮವಾದ ಹಾದಿಗೇಕೆ ಇಷ್ಟೊಂದು ಪ್ರಾಮುಖ್ಯತೆ ಕೊಡಬೇಕು? ಈ ಮಧುರ ಭಾವದ ಪ್ರೇಮಾದರ್ಶದ ಹೊರತು ದೇವರನ್ನು ಪೂಜಿಸಲು ಮತ್ತಾವ ಬಾಂಧವ್ಯವೂ ಇಲ್ಲವೇನು? ಇತರ ನಾಲ್ಕು ಮಾರ್ಗಗಳನ್ನೇಕೆ ಅವಲಂಬಿಸಿ ಹೃತ್ಪೂರ್ವಕವಾಗಿ ಭಗವನ್ನಾಮವನ್ನು ಸ್ಮರಿಸಬಾರದು? ಮೊದಲು ನಿಮ್ಮ ಹೃದಯ ತೆರೆಯಲಿ, ಮಿಕ್ಕೆಲ್ಲವೂ ತಮ್ಮಷ್ಟಕ್ಕೆ ತಾವೇ ಹಿಂಬಾಲಿಸುವುವು. ಆದರೆ ಇದೊಂದನ್ನು ಮಾತ್ರ ಖಂಡಿತವಾಗಿ ತಿಳಿ. ಎಲ್ಲಿಯವರೆಗೆ ಕಾಮವಿರುವುದೊ ಅಲ್ಲಿಯವರೆಗೆ ಪ್ರೇಮವಿಲ್ಲ. ಮೊದಲು ಏಕೆ ಕಾಮಾಸಕ್ತಿಗಳನ್ನು ತೊರೆಯಲು ಪ್ರಯತ್ನಿಸಬಾರದು? ನೀವು ‘ಅದು ಹೇಗೆ ಸಾಧ್ಯ – ನಾನೊಬ್ಬ ಗೃಹಸ್ಥ’ ಎಂದು ಹೇಳುವಿರಿ. ಅಸಂಬದ್ದತೆ! ಗೃಹಸ್ಥನಾದ ಮಾತ್ರಕ್ಕೆ ವಿಷಯ ಭೋಗದ ಮೂರ್ತಿಮತ್ತಾಗಿರಬೇಕೇನು ಅಥವಾ ಆಜೀವಪರ್ಯಂತ ದೇಹ ಸಂಬಂಧವನ್ನಿಟ್ಟು ಕೊಂಡು ಜೀವಿಸಬೇಕೇನು? ಕಟ್ಟಕಡೆಗೆ, ಗಂಡಸು ಈ ಮಧುರ ಭಾವಸಾಧನೆಗಾಗಿ ತಾನು ಒಬ್ಬ ಹೆಂಗಸಾಗುವುದು ಎಷ್ಟು ಅನುಚಿತ!
=
ಸ್ವಾಮೀಜಿ: ಎಲ್ಲೋ ಕೆಲವರ ವಿನಃ ಸಾಮಾನ್ಯ ಗೃಹಸ್ಥರಿಗೆ ನಿಸ್ಸಂದೇಹವಾಗಿ ಇದು ಅಸಾಧ್ಯ. ಮತ್ತೆ ಎಲ್ಲಕ್ಕಿಂತ ಹೆಚ್ಚಾಗಿ ಈ ಸೂಕ್ಷ್ಮವಾದ ಹಾದಿಗೇಕೆ ಇಷ್ಟೊಂದು ಪ್ರಾಮುಖ್ಯತೆ ಕೊಡಬೇಕು? ಈ ಮಧುರ ಭಾವದ ಪ್ರೇಮಾದರ್ಶದ ಹೊರತು ದೇವರನ್ನು ಪೂಜಿಸಲು ಮತ್ತಾವ ಬಾಂಧವ್ಯವೂ ಇಲ್ಲವೇನು? ಇತರ ನಾಲ್ಕು ಮಾರ್ಗಗಳನ್ನೇಕೆ ಅವಲಂಬಿಸಿ ಹೃತ್ಪೂರ್ವಕವಾಗಿ ಭಗವನ್ನಾಮವನ್ನು ಸ್ಮರಿಸಬಾರದು? ಮೊದಲು ನಿಮ್ಮ ಹೃದಯ ತೆರೆಯಲಿ, ಮಿಕ್ಕೆಲ್ಲವೂ ತಮ್ಮಷ್ಟಕ್ಕೆ ತಾವೇ ಹಿಂಬಾಲಿಸುವುವು. ಆದರೆ ಇದೊಂದನ್ನು ಮಾತ್ರ ಖಂಡಿತವಾಗಿ ತಿಳಿ. ಎಲ್ಲಿಯವರೆಗೆ ಕಾಮವಿರುವುದೊ ಅಲ್ಲಿಯವರೆಗೆ ಪ್ರೇಮವಿಲ್ಲ. ಮೊದಲು ಏಕೆ ಕಾಮಾಸಕ್ತಿಗಳನ್ನು ತೊರೆಯಲು ಪ್ರಯತ್ನಿಸಬಾರದು? ನೀವು ‘ಅದು ಹೇಗೆ ಸಾಧ್ಯ - ನಾನೊಬ್ಬ ಗೃಹಸ್ಥ’ ಎಂದು ಹೇಳುವಿರಿ. ಅಸಂಬದ್ದತೆ! ಗೃಹಸ್ಥನಾದ ಮಾತ್ರಕ್ಕೆ ವಿಷಯ ಭೋಗದ ಮೂರ್ತಿಮತ್ತಾಗಿರಬೇಕೇನು ಅಥವಾ ಆಜೀವಪರ್ಯಂತ ದೇಹ ಸಂಬಂಧವನ್ನಿಟ್ಟು ಕೊಂಡು ಜೀವಿಸಬೇಕೇನು? ಕಟ್ಟಕಡೆಗೆ, ಗಂಡಸು ಈ ಮಧುರ ಭಾವಸಾಧನೆಗಾಗಿ ತಾನು ಒಬ್ಬ ಹೆಂಗಸಾಗುವುದು ಎಷ್ಟು ಅನುಚಿತ!


ಪ್ರಶ್ನೆ: ನಿಜ, ಸ್ವಾಮೀಜಿ, ಭಗವಂತನ ನಾಮಸಂಕೀರ್ತನೆ ತುಂಬಾ ಸಹಾಯ ಮಾಡುತ್ತದೆ. ಮನಸ್ಸಿನಲ್ಲಿ ಹರ್ಷಭಾವನೆ ಮೂಡುತ್ತದೆ. ನಮ್ಮ ಪುರಾಣಗಳೂ ಹೀಗೆ ಹೇಳುತ್ತವೆ. ಶ‍್ರೀಚೈತನ್ಯದೇವನೂ ಜನಸಮುದಾಯಕ್ಕೆ ಇದನ್ನೇ ಬೋಧಿಸಿದ್ದು. ಮೃದಂಗ ಬಾರಿಸುತ್ತಿದ್ದಾಗ ಉಲ್ಲಾಸವಾಗಿ ಎಲ್ಲಿಗೋ ಹೋದಂತೆ ಭಾಸವಾಗಿ ಕುಣಿಯುವ ಹಾಗಾಗುತ್ತದೆ.

ಸ್ವಾಮೀಜಿ: ಅದು ಸರಿ. ಆದರೆ ಕೀರ್ತನೆ ಎಂದರೆ ಬರೀ ಕುಣಿತವೇ ಎಂದು ಯೋಚಿಸಬೇಡ. ನಿನಗಿಷ್ಟಬಂದ ರೀತಿಯಲ್ಲಿ ದೇವರ ಮಹಿಮೆಯನ್ನು ಹಾಡುವುದು ಎಂದರ್ಥ. ಈ ರೀತಿ ಪ್ರಚಂಡವಾಗಿ ಭಾವೋದ್ರೇಕವನ್ನುಂಟುಮಾಡುವ ಭಾವನೆ ಮತ್ತು ವೈಷ್ಣವರ ಕುಣಿತ ಇವೆಲ್ಲಾ ಒಳ್ಳೆಯದು ಮತ್ತು ಮನಸ್ಸಿಗೆ ಹಿಡಿಸುವಂಥದು ನಿಜ. ಆದರೆ ಇದರ ಸಾಧನೆಯಿಂದ ಉಂಟಾಗುವ ಅಪಾಯದಿಂದಲೂ ನೀವು ಪಾರಾಗಬೇಕು. ಅಪಾಯವು ಇದಕ್ಕೆ ತೋರುವ ಪ್ರತಿಕ್ರಿಯೆಯಲ್ಲಿದೆ. ಒಂದು ಕಡೆ ಮನಸ್ಸು ಇದ್ದಕ್ಕಿದ್ದಂತೆ ಅತ್ಯುಚ್ಚ ಮಟ್ಟಕ್ಕೆ ಏರುವುದು, ಕಣ್ಣಿನಿಂದ ಕಂಬನಿಯ ಪ್ರವಾಹವೇ ಹರಿಯುವುದು, ತನ್ಮಯಾವಸ್ಥೆಯಲ್ಲಿರುವಂತೆ ತಲೆ ತೂಗಾಡುವುದು. ಮತ್ತೊಂದು ಕಡೆ ಸಂಕೀರ್ತನೆ ಮುಕ್ತಾಯವಾದೊಡನೆಯೇ ಈ ಭಾವನೆಗಳ ಸಮುದಾಯ ಎಷ್ಟು ಮೇಲಕ್ಕೆ ಹೋಗಿತ್ತೋ ಅಷ್ಟೇ ರಭಸದಿಂದ ಬೀಳುವುದು. ಸಾಗರದಲ್ಲಿ ಅಲೆ ಎಷ್ಟು ಎತ್ತರಕ್ಕೆ ಹೋಗುವುದೋ ಅಷ್ಟೇ ವೇಗವಾಗಿ ಕೆಳಕ್ಕೂ ಬೀಳುವುದು. ಈ ಪ್ರತಿಕ್ರಿಯೆಯನ್ನು ಎದುರಿಸಿ ನಮ್ಮನ್ನು ನಾವು ತಡೆದುಕೊಳ್ಳುವುದು ಈ ಅವಸ್ಥೆಯಲ್ಲಿ ತುಂಬಾ ಕಷ್ಟ. ಯೋಗ್ಯವಾದ ವಿವೇಕಜ್ಞಾನವಿಲ್ಲದಿದ್ದಲ್ಲಿ ಸಹಜವಾಗಿ ಕಾಮ ಮೊದಲಾದ ಕೀಳು ಪ್ರವೃತ್ತಿಗಳಿಗೆ ವಶರಾಗುವೆವು. ನಾನು ಅಮೆರಿಕಾದಲ್ಲಿ ಕೂಡ ಈ ಸ್ಥಿತಿಯನ್ನು ನೋಡಿರುವೆನು. ಅನೇಕರು ಚರ್ಚಿಗೆ ಹೋಗಿ ತುಂಬಾ ಭಕ್ತಿಯಿಂದ ಪ್ರಾರ್ಥಿಸಿ ಉದಾತ್ತ ಭಾವದಿಂದ ಹಾಡುವರು. ಧರ್ಮೋಪದೇಶವನ್ನು ಕೇಳುವಾಗ ನಿರರ್ಗಳವಾಗಿ ಕಣ್ಣೀರನ್ನೂ ಸುರಿಸುವರು. ಆದರೆ ಚರ್ಚಿನಿಂದೀಚೆಗೆ ಬಂದಮೇಲೆ ಅವರಲ್ಲಿ ದೊಡ್ಡ ಪ್ರತಿಕ್ರಿಯೆಯುಂಟಾಗಿ ಕಾಮ ಪ್ರವೃತ್ತಿಗೆ ವಶರಾಗುವರು.

ಪ್ರಶ್ನೆ: ಹಾಗಾದರೆ, ಮಹಾಶಯರೆ, ಶ‍್ರೀಕೃಷ್ಣಚೈತನ್ಯರು ಉಪದೇಶಿಸಿದ ಯಾವ ಉಪದೇಶಗಳು ನಮಗೆ ಸೂಕ್ತವಾಗಿವೆ ಎಂದು ನಮಗೆ ಸಲಹೆ ಕೊಡಿ. ಇದರಿಂದ ನಾವು ತಪ್ಪು ಹೆಜ್ಜೆ ಇಡದಂತಾಗುವುದು.

ಸ್ವಾಮೀಜಿ: ಜ್ಞಾನದಿಂದ ಹದವಾದ ಭಕ್ತಿಯಿಂದ ದೇವರನ್ನು ಪೂಜಿಸಿ, ಭಕ್ತಿಯೊಡನೆ ವಿಚಾರಜ್ಞಾನವನ್ನೂ ಹೊಂದಿ. ಇದರ ಜೊತೆಗೆ ಶ‍್ರೀಚೈತನ್ಯರಿಂದ ಅವರ ಭಾವನೆ, ಎಲ್ಲಾ ಜೀವಿಗಳ ಮೇಲೂ ಅವರಿಗಿದ್ದ ಕರುಣೆ, ದೇವರಿಗಾಗಿ ಅವರಲ್ಲಿ ಪ್ರಜ್ವಲಿಸುತ್ತಿದ್ದ ಉತ್ಸಾಹ ಇವನ್ನು ತೆಗೆದುಕೊಳ್ಳಿ. ಅವರ ತ್ಯಾಗವೇ ನಿಮ್ಮ ಜೀವಿತದ ಧ್ಯೇಯವಾಗಿರಲಿ.

ಪ್ರಶ್ನಿಸಿದವನು ಈಗ ಸ್ವಾಮೀಜಿಗೆ ಕೈ ಜೋಡಿಸಿ ಹೇಳಿದನು: ಮಹಾಶಯರೆ, ನಾನು ನಿಮ್ಮ ಕ್ಷಮಾಪಣೆ ಕೇಳುತ್ತಿರುವೆ. ನೀವು ಹೇಳುವುದು ಸರಿಯೆಂದು ಈಗ ನನಗೆ ಅರ್ಥವಾಯಿತು. ವೈಷ್ಣವರ ಮಧುರಭಾವದ ಪ್ರೇಮವನ್ನು ತಮಾಷೆಯಾಗಿ ನೀವು ಟೀಕಿಸುತ್ತಿದ್ದುದನ್ನು ನೋಡಿ ನನಗೆ ಮೊದಲು ನಿಮ್ಮ ಹೇಳಿಕೆಯ ರೀತಿ ಅರ್ಥವಾಗಲಿಲ್ಲ. ಆದ್ದರಿಂದಲೇ ನಾನು ಆಕ್ಷೇಪಣೆ ತೆಗೆದೆ.

ಸ್ವಾಮೀಜಿ: ಸರಿ, ಇಲ್ಲಿ ನೋಡು, ನಾವು ಟೀಕಿಸಬೇಕಾಗಿಬಂದರೆ ದೇವರನ್ನು, ದೇವಮಾನವರನ್ನು ಟೀಕಿಸುವುದು ಒಳ್ಳೆಯದು. ನೀನು ನನ್ನನ್ನು ನಿಂದಿಸಿದರೆ ಸಹಜವಾಗಿ ನಾನು ನಿನ್ನ ಮೇಲೆ ಕೋಪಗೊಳ್ಳುವೆ. ನಾನು ನಿನ್ನನ್ನು ಬೈದರೆ ನೀನು ಪ್ರತೀಕಾರ ಮಾಡಲಿಚ್ಛಿಸುವೆ, ಅಲ್ಲವೆ? ಆದರೆ ದೇವರು ದೇವಮಾನವರು ಎಂದಿಗೂ ಕೇಡಿಗೆ ಕೇಡನ್ನು ಕೊಡುವುದಿಲ್ಲ.

ಆ ಸಭ್ಯಗೃಹಸ್ಥ ಸ್ವಾಮೀಜಿಯವರ ಪಾದಗಳಿಗೆ ಸಾಷ್ಟಾಂಗ ನಮಸ್ಕಾರ ಮಾಡಿ ಹೊರಟುಹೋದನು. ಸ್ವಾಮೀಜಿ ಕಲ್ಕತ್ತೆಯಲ್ಲಿ ವಾಸಿಸುತ್ತಿದ್ದಾಗ ಹೀಗೆ ಜನರ ಗುಂಪು ಸೇರುವುದು ನಿತ್ಯದ ರೂಢಿಯಾಗಿತ್ತು ಎಂದು ನಾನು ಮೊದಲೇ ಹೇಳಿದ್ದೆ. ಬೆಳಗಿನ ಜಾವದಿಂದ ಹಿಡಿದು ರಾತ್ರಿ ಎಂಟುಗಂಟೆಯವರೆಗೂ ದಿನದ ಪ್ರತಿ ಗಂಟೆಯೂ ಜನರು ಅವರನ್ನು ಮುತ್ತಿರುತ್ತಿದ್ದರು. ಇದರಿಂದ ಅವರ ಊಟದ ವೇಳೆ ಸಹಜವಾಗಿ ತುಂಬ ಅಸ್ತವ್ಯಸ್ತವಾಯಿತು. ಅನೇಕರು ಈ ಸ್ಥಿತಿಯನ್ನು ತಪ್ಪಿಸಲು ಇಚ್ಛಿಸಿ ಸ್ವಾಮೀಜಿಯನ್ನು ಕೆಲವು ನಿರ್ದಿಷ್ಟ ಗಂಟೆಗಳ ಹೊರತು ಇತರ ವೇಳೆಯಲ್ಲಿ ಮಾತನಾಡಬಾರದೆಂದು ಒತ್ತಿ ಹೇಳಿದರು. ಆದರೆ ಪ್ರೇಮಹೃದಯರಾದ ಸ್ವಾಮೀಜಿ ಇತರರ ಉದ್ಧಾರಕ್ಕೆ ಏನು ಬೇಕಾದರೂ ಮಾಡಲು ಸಿದ್ಧರಾಗಿದ್ದವರು. ಜನರಲ್ಲಿ ಧರ್ಮಕ್ಕಾಗಿ ಇಷ್ಟೊಂದು ತೃಷೆ ಇರುವುದನ್ನು ನೋಡಿ ದಯೆಯಿಂದ ಕರಗಿ ಹೋಗುತ್ತಿದ್ದರು. ಅವರು ಅನಾರೋಗ್ಯದಿಂದಿದ್ದರೂ ಕೂಡ ಈ ಬಗೆಯ ಯಾವ ಬೇಡಿಕೆಗಳಿಗೂ ಒಪ್ಪಲಿಲ್ಲ. “ಅವರು ತಮ್ಮ ಮನೆಯಿಂದ ಅಷ್ಟು ದೂರ ಇಲ್ಲಿಗೆ ಅಷ್ಟೊಂದು ಶ್ರಮಪಟ್ಟು ಬರುತ್ತಾರೆ. ನನ್ನ ದೇಹಸ್ಥಿತಿ ಕೊಂಚ ಕೆಡುವುದೆಂದು ನಾನು ಸುಮ್ಮನೆ ಇಲ್ಲಿ ಕುಳಿತು ಅವರೊಡನೆ ಕೊಂಚವೂ ಮಾತನಾಡದೆ ಇರಲೇನು?” ಎಂದು ಉತ್ತರ ಕೊಡುತ್ತಿದ್ದರು.

ನಾಲ್ಕು ಗಂಟೆ ಹೊತ್ತಿಗೆ ಮಾತುಕತೆಗಳು ಮುಕ್ತಾಯಗೊಂಡವು. ಎಲ್ಲೋ ಕೆಲವು ಮಂದಿ ಸಭ್ಯ ಗೃಹಸ್ಥರ ವಿನಃ ಜನಸಮೂಹವೆಲ್ಲಾ ಚದುರಿತು. ಸ್ವಾಮೀಜಿ ಅವರೊಡನೆ ಇಂಗ್ಲೆಂಡ್ ಅಮೆರಿಕಾ ಮುಂತಾದ ಅನೇಕ ವಿಷಯಗಳನ್ನು ಕುರಿತು ಮಾತನಾಡಿದರು. ಸಂಭಾಷಣೆಯ ನಡುವೆ ಸ್ವಾಮೀಜಿ "ನಾನು ಇಂಗ್ಲೇಂಡಿನಿಂದ ಕಡಲ ಪ್ರಯಾಣದಲ್ಲಿ ಹಿಂತಿರುಗಿ ಬರುವಾಗ ಒಂದು ವಿಸ್ಮಯಕರವಾದ ಸ್ವಪ್ನವನ್ನು ಕಂಡೆ. ನಮ್ಮ ಹಡಗು ಮೆಡಿಟರೇನಿಯನ್ ಸಮುದ್ರವನ್ನು ಹಾಯುತ್ತಿರುವಾಗ, ನನ್ನ ನಿದ್ರೆಯಲ್ಲಿ ಒಬ್ಬ ವೃದ್ಧ ಪೂಜ್ಯ, ನೋಡಲು ಋಷಿಯಂತೆ ತೋರುವ ವ್ಯಕ್ತಿಯೊಬ್ಬ ನನ್ನ ಮುಂದೆ ನಿಂತು ಹೇಳಿದನು: ‘ದಯವಿಟ್ಟು ಬಂದು ನಮ್ಮನ್ನು ಉದ್ಧಾರಮಾಡು, ಭರತಖಂಡದ ಋಷಿಗಳ ಬೋಧನೆಯೇ ಮೂಲವಾಗಿರುವ ಥೇರಾಪುಟ ಪಂಥದವರಲ್ಲಿ ನಾನೂ ಒಬ್ಬನು. ನಾವು ಬೋಧಿಸಿದ ಸತ್ಯ ಮಹೋದ್ದೇಶಗಳನ್ನು ಕ್ರೈಸ್ತರು ಏಸುವಿನಿಂದ ಕಲಿತುದೆಂದು ಹೇಳಿದ್ದಾರೆ. ಈ ಸಂಗತಿಯ ಹೊರತು ಜೀಸಸ್ ಎಂಬ ಹೆಸರಿನ ಯಾವ ವ್ಯಕ್ತಿಯೂ ಎಂದೂ ಜನಿಸಿಲ್ಲ. ಈ ವಿಷಯವನ್ನು ಒರೆ ಹಚ್ಚಿ ನೋಡಲು ಈ ಸ್ಥಳದಲ್ಲಿ ಭೂಶೋಧನೆಯನ್ನು ನಡೆಸಿದರೆ ಅನೇಕ ಪ್ರಮಾಣಗಳು ಬೆಳಕಿಗೆ ಬರುವುವು.’ ನಾನು ‘ಯಾವ ಜಾಗವನ್ನು ಶೋಧಿಸಿದರೆ ನೀವು ಹೇಳುವ ಪ್ರಮಾಣಗಳ ಅವಶೇಷಗಳು ಸಿಗುತ್ತವೆ?’ ಎಂದು ಕೇಳಿದೆ. ಆ ಪ್ರಾಚೀನ ಮುಪ್ಪಿನ ವೃದ್ಧ ಟರ್ಕಿಯ ಹತ್ತಿರದ ಒಂದು ಸ್ಥಳದ ಕಡೆ ಕೈ ತೋರಿಸುತ್ತಾ ‘ಅಲ್ಲಿ ನೋಡು’ ಎಂದನು. ಇದಾದ ಕೂಡಲೇ ನಾನು ನಿದ್ದೆಯಿಂದ ಎದ್ದು, ತಕ್ಷಣವೇ ಹಡಗಿನ ಮೇಲಣ ಅಂತಸ್ತಿಗೆ ಓಡಿ ಹೋಗಿ ಹಡಗಿನ ನಾಯಕನನ್ನು ‘ಈಗ ಹಡಗು ಯಾವ ದೇಶದ ಹತ್ತಿರ ಇದೆ?’ ಎಂದು ಕೇಳಿದೆ. ನಾಯಕನು ‘ಅಲ್ಲಿ ನೋಡಿ, ಅದೇ ಟರ್ಕಿ ಮತ್ತು ಕ್ರೀಟ್ ದ್ವೀಪ’ ಎಂದನು."

ಅದು ಕೇವಲ ಸ್ವಪ್ನವೋ ಅಥವಾ ಆ ದರ್ಶನದಲ್ಲಿ ಏನಾದರೂ ಅರ್ಥವಿದೆಯೋ? ಯಾರಿಗೆ ಗೊತ್ತು.

