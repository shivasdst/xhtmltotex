
\chapter[ಅಧ್ಯಾಯ ೧]{ಅಧ್ಯಾಯ ೧\protect\footnote{\engfoot{C.W, Vol. V, P 349}}}

ನಮ್ಮ ಮನೆ ಸ್ವಾಮೀಜಿ ಮನೆಗೆ ತುಂಬ ಸಮೀಪದಲ್ಲಿತ್ತು. ನಾವು ಪಟ್ಟಣದ ಒಂದೇ ಭಾಗದಲ್ಲಿದ್ದ ಹುಡುಗರಾಗಿದ್ದುದರಿಂದ ನಾನು ಅವರೊಡನೆ ಅನೇಕ ವೇಳೆ ಆಟವಾಡುತ್ತಿದ್ದೆ. ಬಾಲ್ಯಾರಭ್ಯ ಅವರನ್ನು ಕಂಡರೆ ನನಗೆ ಅನನ್ಯ ವಿಶ್ವಾಸವಿತ್ತು. ಅವರು ಒಬ್ಬ ಪ್ರಖ್ಯಾತ ವ್ಯಕ್ತಿಗಳಾಗುವರೆಂದು ನನಗೆ ದೃಢವಾದ ನಂಬುಗೆಯಿತ್ತು. ಅವರು ಸಂನ್ಯಾಸಿಗಳಾದಾಗ ಉಚ್ಚ ಪದವಿಗೇರುವರೆಂಬ ಭರವಸೆಯೆಲ್ಲಾ ವ್ಯರ್ಥವಾಯಿತು.

ಆ ನಂತರ ಅವರು ಅಮೆರಿಕಾಕ್ಕೆ ಹೋದಾಗ ನಾನು ವರ್ತಮಾನ ಪತ್ರಿಕೆಗಳಲ್ಲಿ ಅವರು ಚಿಕಾಗೊ ಸರ್ವಧರ್ಮ ಸಮ್ಮೇಳನದಲ್ಲಿ ಮತ್ತು ಅಮೆರಿಕಾದ ಹಲವು ಭಾಗಗಳಲ್ಲಿ ಮಾಡಿದ ಭಾಷಣಗಳ ಸುದ್ದಿಯನ್ನೋದಿದೆ. ಆಗ ಬಟ್ಟೆಯ ಕೆಳಗೆ ಬೆಂಕಿಯನ್ನು ಬಚ್ಚಿಡಲಾಗುವುದಿಲ್ಲ ಎಂದುಕೊಂಡೆ. ಸ್ವಾಮೀಜಿಯಲ್ಲಿ ಇಷ್ಟು ದಿನವೂ ಹುದುಗಿದ್ದ ತೇಜಸ್ಸು ಈಗ ಉಜ್ವಲವಾಗಿ ಹೊರಹೊಮ್ಮಿತು. ಇಷ್ಟು ವರ್ಷಗಳ ನಂತರ ಮೊಗ್ಗು ಅರಳಿತು, ಈಗ ಹೂವಾಯಿತು.

ಕೆಲವು ಕಾಲಾನಂತರ ಅವರು ಭರತಖಂಡಕ್ಕೆ ಹಿಂತಿರುಗಿರುವರೆಂದೂ ಮದ್ರಾಸಿನಲ್ಲಿ ಪ್ರಚಂಡ ಭಾಷಣಗಳನ್ನು ಕೊಡುತ್ತಿರುವರೆಂದೂ ನನಗೆ ತಿಳಿದುಬಂತು. ಅವುಗಳನ್ನೋದಿ ನನಗೆ ಹಿಂದೂಧರ್ಮದಲ್ಲಿ ಇಂತಹ ಮಹತ್ವ ಸತ್ಯಗಳೂ ಇವೆ, ಅವುಗಳನ್ನು ಇಷ್ಟೊಂದು ಸ್ಪಷ್ಟವಾಗಿ ವಿವರಿಸಲೂ ಸಾಧ್ಯ ಎಂದು ತಿಳಿದು ಆಶ್ಚರ್ಯವಾಯಿತು. ಅವರಿಗೆ ಎಂತಹ ಅಸಾಧಾರಣ ಶಕ್ತಿ ಇತ್ತು. ಅವರು ಮಾನವರೊ, ದೇವತೆಗಳೊ!

ಸ್ವಾಮೀಜಿ ಕಲ್ಕತ್ತೆಗೆ ಬಂದಾಗ ಪ್ರಚಂಡ ಉತ್ಸಾಹ ಹಬ್ಬಿತ್ತು. ನಾವು ಕಾಶೀಪುರದಲ್ಲಿ ಗಂಗಾನದಿಯ ಮೇಲಿದ್ದ ಶೀಲರ ತೋಟದ ಮನೆಗೆ ಅವರನ್ನು ಹಿಂಬಾಲಿಸಿ ಹೋದೆವು. ಕೆಲವು ದಿನಗಳ ನಂತರ ರಾಜಾ ರಾಧಾ ಕಾಂತದೇವ್ ಅವರ ನಿವಾಸದಲ್ಲಿ ‘ಕಲ್ಕತ್ತೆಯ ಹುಡುಗ’ ಒಂದು ಪ್ರಚಂಡ ಜನಸಮುದಾಯಕ್ಕೆ ತಮ್ಮ ಸ್ವಾಗತ ಬಿನ್ನವತ್ತಳೆಗೆ ಉತ್ತರ ಕೊಡುತ್ತ ಒಂದು ಸ್ಪೂರ್ತಿದಾಯಕವಾದ ಭಾಷಣವನ್ನು ಮಾಡಿದರು. ಕಲ್ಕತ್ತೆಯ ಜನತೆ ಮೊದಲ ಬಾರಿಗೆ ಅವರ ಭಾಷಣವನ್ನು ಕೇಳಿ ಅವರನ್ನು ಮೆಚ್ಚಿ ಕೊಂಡಾಡಿತು.

ಅವರು ಕಲ್ಕತ್ತೆಗೆ ಬಂದ ಮೇಲೆ ನಾನವರನ್ನೊಮ್ಮೆ ಏಕಾಂತದಲ್ಲಿ ಭೇಟಿ ಮಾಡಿ ನಾವು ಹುಡುಗರಾಗಿದ್ದಾಗ ಮಾತನಾಡುತ್ತಿದ್ದಂತೆ ಸದರವಾಗಿ ಮಾತನಾಡಬೇಕೆಂದು ತುಂಬಾ ಉತ್ಸುಕನಾಗಿದ್ದೆ. ಆದರೆ ಯಾವಾಗಲೂ ಅವರ ಸುತ್ತಲೂ ಕುತೂಹಲಿಗಳಾದ ಅನ್ವೇಷಕರು ಮುತ್ತಿರುತ್ತಿದ್ದುದರಿಂದ ಸಂಭಾಷಣೆ ನಿರರ್ಗಳವಾಗಿ ಸಾಗುತ್ತಿತ್ತು. ಆದ್ದರಿಂದ ಕೆಲವು ಕಾಲದವರೆಗೆ ನನಗೆ ಅವಕಾಶ ದೊರಕಲಿಲ್ಲ. ಒಮ್ಮೆ ನಾವಿಬ್ಬರೂ ಗಂಗಾತೀರದಲ್ಲಿ ತೋಟದಲ್ಲಿ ಅಡ್ಡಾಡಲು ಹೋದಾಗ ಈ ಅವಕಾಶ ದೊರಕಿತು. ಕೂಡಲೇ ಸ್ವಾಮೀಜಿ ಹಿಂದಿನ ನನ್ನ ಬಾಲ್ಯ ಸ್ನೇಹಿತನಂತೆ ನನ್ನೊಡನೆ ಮಾತನಾಡ ತೊಡಗಿದರು. ಕೊಂಚ ನಾವು ಮಾತನಾಡುವಷ್ಟರಲ್ಲೇ ಬಾರಿಬಾರಿಗೂ ಸ್ವಾಮೀಜಿಯವರಿಗೆ ಅನೇಕ ಸಭ್ಯ ಗೃಹಸ್ಥರು ಅವರ ಭೇಟಿಗಾಗಿ ಬಂದಿರುವರೆಂದು ಕರೆ ಬರತೊಡಗಿತು. ಕೊನೆಗೆ ಸ್ವಾಮೀಜಿ ಕೊಂಚ ರೇಗಿ, ಆ ಸುದ್ದಿ ತಂದವನಿಗೆ “ಮಗು, ನನಗೆ ವಿಶ್ರಾಂತಿ ಕೊಡು, ನನ್ನ ಬಾಲ್ಯ ಸ್ನೇಹಿತನಾದ ಇವನೊಡನೆ ಕೊಂಚ ಮಾತನಾಡಲು ಬಿಡು. ಕೊಂಚ ಹೊತ್ತು ನಾನು ಬಯಲಿನಲ್ಲಿರಲು ಬಿಡು, ಹೋಗು, ಬಂದವರನ್ನು ಸ್ವಾಗತಿಸಿ ಅವರಿಗೆ ಕುಳ್ಳಿರಲು ಹೇಳಿ ಹೊಗೆಸೊಪ್ಪನ್ನು ಕೊಟ್ಟು ಕೊಂಚ ಕಾದಿರಬೇಕೆಂದು ಅರಿಕೆ ಮಾಡಿಕೊ” ಎಂದರು.

ಪುನಃ ನಾವಿಬ್ಬರೇ ಆದಾಗ ನಾನವರನ್ನು ಕೇಳಿದೆ: “ಸ್ವಾಮೀಜಿ, ನೀವೊಬ್ಬ ಸಾಧುಗಳು, ನಿಮ್ಮ ಸ್ವಾಗತ ಸಮಾರಂಭಕ್ಕಾಗಿ ಚಂದಾ ಎತ್ತಿ ಹಣವನ್ನು ಕೂಡಿಸಿದ್ದರು. ಈಗ ಈ ದೇಶದಲ್ಲಿ ಬರಗಾಲ ಬಂದಿರುವುದರಿಂದ ನೀವು ಕಲ್ಕತ್ತೆಗೆ ಬರುವ ಮೊದಲೇ ಒಂದು ತಂತಿಯನ್ನು ‘ನನ್ನ ಸ್ವಾಗತಕ್ಕೆ ಒಂದು ಬಿಡಿಕಾಸನ್ನೂ ಖರ್ಚುಮಾಡಬೇಡಿ. ಬದಲಿಗೆ ಕ್ಷಾಮ ನಿವಾರಣಾ ನಿಧಿಗೆ ಹಣವನ್ನೆಲ್ಲಾ ಕೊಟ್ಟು ಸಹಾಯ ಮಾಡಿ’ ಎಂದು ಕಳುಹಿಸುವಿರೆಂದುಕೊಂಡಿದ್ದೆ. ಆದರೆ ನೀವು ಆ ರೀತಿ ಮಾಡಲಿಲ್ಲವೆಂದು ನನಗೆ ಗೊತ್ತಾಯಿತು. ಅದು ಹೇಗೆ?”

ಸ್ವಾಮೀಜಿ: ಏಕೆ? ನಾನು ಪ್ರಚಂಡ ಉತ್ಸಾಹವನ್ನು ಎಬ್ಬಿಸಬೇಕೆಂದು ಇಷ್ಟಪಟ್ಟೆ. ನೀನೇ ನೋಡು. ಹೀಗೆ ಮಾಡದೆ ಇದ್ದಿದ್ದರೆ ಶ‍್ರೀರಾಮಕೃಷ್ಣರೆಡೆಗೆ ಜನರು ಹೇಗೆ ಆಕರ್ಷಿತರಾಗಿ ಆತನ ಹೆಸರಿನಲ್ಲಿ ಹೇಗೆ ಉತ್ಸುಕರಾಗುತ್ತಿದ್ದರು? ಉತ್ಸಾಹಪೂರ್ಣ ಸ್ವಾಗತ ನನ್ನ ವ್ಯಕ್ತಿತ್ವಕ್ಕಾಗಿಯೋ ಅಥವಾ ಇದರಿಂದ ಅವರ ಹೆಸರಿಗೆ ಕೀರ್ತಿ ಬಂತೊ? ನೋಡು, ಜನರ ಮನಸ್ಸಿನಲ್ಲೆಲ್ಲಾ ಅವರ ವಿಷಯ ಹೆಚ್ಚು ತಿಳಿಯಬೇಕೆಂದು ಎಂತಹ ತೃಷೆ ಉತ್ಪನ್ನವಾಗಿದೆ? ಈಗ ಅವರು ಕ್ರಮೇಣ ಅವರ ವಿಷಯಗಳನ್ನು ತಿಳಿದುಕೊಳ್ಳುವರು. ಇದರಿಂದ ದೇಶದ ಏಳ್ಗೆಗೆ ಸಹಾಯವಾಗುವುದಿಲ್ಲವೆ? ಯಾರು ನಮ್ಮ ದೇಶದ ಕಲ್ಯಾಣಕ್ಕಾಗಿ ಬಂದರೆ ಅವರ ವಿಷಯವೇ ಜನರಿಗೆ ಗೊತ್ತಿಲ್ಲದೆ ಹೋದರೆ ಅವರಿಗೆ ಹೇಗೆ ಒಳ್ಳೆಯದಾಗುವುದು? ಅವರು ನಿಜವಾಗಿ ಯಾರೆಂಬುದು ತಿಳಿದಾಗ ಮನುಷ್ಯರು ನಿಜವಾದ ಮನುಷ್ಯರು ಆಗುವರು. ಯಾವಾಗ ಅಂತಹ ಮನುಷ್ಯರಿರುವರೋ ಆಗ ದೇಶದಿಂದ ಕ್ಷಾಮ ಮುಂತಾದುವನ್ನು ಹೊಡೆದೋಡಿಸಲು ಎಷ್ಟು ಸಮಯ ಬೇಕಾಗುವುದು? ಅದಕ್ಕೇ ನಾನು ಕಲ್ಕತ್ತೆಯಲ್ಲಿ ಕೊಂಚ ಉತ್ಸಾಹ, ಗಡಿಬಿಡಿ ಆಗಲಿ, ಇದರಿಂದ ಜನಸಾಮಾನ್ಯರು ಶ‍್ರೀರಾಮಕೃಷ್ಣ ಸಂಸ್ಥೆಯಲ್ಲಿ ನಂಬಿಕೆ ಇಟ್ಟು ಕೊಂಚವಾದರೂ ಅದಕ್ಕೆ ಮನಸ್ಸು ಕೊಡಲಿ, ಎಂದು ಇಚ್ಛೆಪಟ್ಟೆ. ಇಲ್ಲದಿದ್ದಲ್ಲಿ ಕೇವಲ ನನಗಾಗಿ ಇಷ್ಟೊಂದು ಗದ್ದಲದ ಆವಶ್ಯಕತೆ ಏನಿತ್ತು? ಇದನ್ನು ಕಟ್ಟಿಕೊಂಡು ನನಗೇನು? ನಿಮ್ಮ ಮನೆಯಲ್ಲಿ ನಿನ್ನೊಡನೆ ಆಟವಾಡುತ್ತಿದ್ದ ಕಾಲಕ್ಕಿಂತ ಈಗ ನಾನು ಹೆಚ್ಚು ಶ್ರೇಷ್ಠನಾಗಿರುವೆನೇನು? ನಾನು ಅಂದಿನಂತೆ ಇಂದೂ ಇರುವೆ, ಹೇಳು ನಿನಗೆ ನನ್ನಲ್ಲಿ ಏನಾದರೂ ಬದಲಾವಣೆ ಕಂಡುಬರುವುದೇನು?

ನಾನು, ಇಲ್ಲ, ಹೇಳುವಷ್ಟು ನಿಮ್ಮಲ್ಲಿ ಯಾವ ಬದಲಾವಣೆಯೂ ಆಗಿಲ್ಲ ಎಂದು ಬಾಯಲ್ಲಿ ಹೇಳಿದರೂ ಮನಸ್ಸಿನಲ್ಲಿ ‘ನಿಜವಾಗಿಯೂ ಈಗ ನೀನು ದೇವರಾಗಿರುವೆ’ ಎಂದುಕೊಂಡೆ. ಸ್ವಾಮೀಜಿ ಮುಂದುವರಿಸಿದರು: “ನಮ್ಮ ದೇಶದಲ್ಲಿ ಕ್ಷಾಮ ಒಂದು ನಿರಂತರ ವಿಪತ್ತಾಗಿತ್ತು. ಈಗ ಅದು ಒಂದು ರೋಗವಾಗಿ ಪರಿಣಮಿಸಿದೆ. ಇನ್ನಾವ ದೇಶದಲ್ಲಾದರೂ ಕ್ಷಾಮದ ಹಾವಳಿ ಮೇಲಿಂದ ಮೇಲೆ ಸಂಭವಿಸುವುದನ್ನು ನೋಡಿರುವೆಯಾ? ಇಲ್ಲ. ಏಕೆಂದರೆ ಇತರ ದೇಶಗಳಲ್ಲಿ ಮನುಷ್ಯರಿದ್ದಾರೆ. ನಮ್ಮ ದೇಶದಲ್ಲಾದರೋ ಜನರು ಜಡರಾಗಿ ನಿರ್ಜೀವ ವಸ್ತುಗಳಾಗಿದ್ದಾರೆ. ಮೊದಲು ಜನರು ಶ‍್ರೀರಾಮಕೃಷ್ಣರ ವಿಷಯವನ್ನು ಓದಿ ಅವರು ನಿಜವಾಗಿಯೂ ಯಾರೆಂಬುದನ್ನು ತಿಳಿದು ತಮ್ಮ ಸ್ವಾರ್ಥ ಬುದ್ಧಿಯನ್ನು ಬಿಡುವುದನ್ನು ಕಲಿಯಲಿ. ನಂತರ ಪದೇಪದೇ ಬರುತ್ತಿರುವ ಕ್ಷಾಮವನ್ನು ನಿಲ್ಲಿಸಲು ನಿಜವಾದ ಪ್ರಯತ್ನಗಳು ಉತ್ಪನ್ನವಾಗುವುವು. ಸದ್ಯದಲ್ಲೆ ನಾನು ಆ ಕಾರ್ಯಕ್ಕೂ ಕೊಂಚ ಪ್ರಯತ್ನ ಮಾಡಬೇಕೆಂದಿರುವೆ. ನೀನೇ ನೋಡುವೆ.”

ನಾನು: ಅದು ಒಳ್ಳೆಯದು. ಹಾಗಾದರೆ ನೀವಿಲ್ಲಿ ಅನೇಕ ಭಾಷಣಗಳನ್ನು ಕೊಡುವಿರೆಂದು ಭಾವಿಸಿರುವೆ. ಇಲ್ಲದಿದ್ದಲ್ಲಿ ಅವರ ಹೆಸರನ್ನು ತಿಳಿಸಲು ಹೇಗೆ ಸಾಧ್ಯ?

ಸ್ವಾಮೀಜಿ: ಕೆಲಸಕ್ಕೆ ಬಾರದುದು! ಆ ಬಗೆಯೇನೂ ಇಲ್ಲ! ಅವರ ಹೆಸರು ಪ್ರಕಾಶಕ್ಕೆ ಬರುವಂತಹ ಯಾವ ಕೆಲಸವಾದರೂ ಆಗದೆ ಹಾಗೇ ನಿಂತಿರುವುದೇನು? ಆ ಹಾದಿಯಲ್ಲಿ ಸಾಕಷ್ಟು ಮಾಡಿಯಾಗಿದೆ. ಈ ದೇಶದಲ್ಲಿ ಭಾಷಣಗಳಿಂದೇನೂ ಪ್ರಯೋಜನವಿಲ್ಲ. ನಮ್ಮ ದೇಶದ ವಿದ್ಯಾವಂತರು ಅದನ್ನು ಕೇಳುವರು. ಹೆಚ್ಚೆಂದರೆ ಆನಂದದಿಂದ ಬಹಳ ಚೆನ್ನಾಗಿದೆ ಎಂದು ಕೈಚಪ್ಪಾಳೆ ತಟ್ಟುವರು, ಅಷ್ಟೇ. ನಂತರ ಅವರು ಮನೆಗೆ ಹೋಗಿ ನಾವು ಹೇಳುವ ಹಾಗೆ, ಕೇಳಿದುದೆಲ್ಲವನ್ನೂ ತಮ್ಮ ಊಟದೊಂದಿಗೆ ಜೀರ್ಣಿಸಿಕೊಳ್ಳುವರು. ತುಕ್ಕು ಹಿಡಿಯಲ್ಪಟ್ಟ ಹಳೆಯ ಕಬ್ಬಿಣದ ತುಂಡನ್ನು ಸುತ್ತಿಗೆಯಿಂದ ಎಷ್ಟು ಕುಟ್ಟಿದರೆ ತಾನೇ ಏನು ಪ್ರಯೋಜನ? ಅದು ಕೇವಲ ಪುಡಿಪುಡಿಯಾಗುವುದಕ್ಕೆ ಮೊದಲು ಅದನ್ನು ಕೆಂಪಗೆ ಕೆಂಡದಂತೆ ಕಾಯಿಸಿ ನಂತರ ಸುತ್ತಿಗೆಯಿಂದ ಹೊಡೆದರೆ ನಮಗೆ ಬೇಕಾದ ಆಕಾರವನ್ನು ಕೊಡಬಹುದು. ನಮ್ಮ ದೇಶದಲ್ಲಿ ಉಜ್ವಲವಾದ ಜೀವಂತ ದೃಷ್ಟಾಂತವನ್ನು ಜನರೆದುರಿಗಿಟ್ಟ ಹೊರತು ಯಾವುದೂ ಪ್ರಯೋಜನವಿಲ್ಲ. ನಮಗೆ ಬೇಕಾಗಿರುವುದು ಎಲ್ಲವನ್ನೂ ತ್ಯಾಗ ಮಾಡಿ ದೇಶಕ್ಕಾಗಿ ತಮ್ಮ ಜೀವವನ್ನೇ ಅರ್ಪಿಸುವಂತಹ ಕೆಲವು ಮಂದಿ ಯುವಕರು. ಮೊದಲು ನಾವು ಅವರ ಜೀವನವನ್ನು ಮಾದರಿಯ ಜೀವನವನ್ನಾಗಿ ಮಾಡಬೇಕು. ನಂತರ ಅವರಿಂದ ಕೊಂಚ ನಿಧಾನವಾಗಿ ಕೆಲಸವನ್ನು ನಿರೀಕ್ಷಿಸಬಹುದು.

ನಾನು: ಸರಿ ಸ್ವಾಮೀಜಿ, ನನಗೆ ಯಾವಾಗಲೂ ಇದನ್ನು ಬಗೆಹರಿಸಲಾಗುವುದಿಲ್ಲ. ನಮ್ಮ ದೇಶೀಯರು ತಮ್ಮ ಸ್ವಧರ್ಮವನ್ನೇ ಅರ್ಥಮಾಡಿಕೊಳ್ಳಲಾರದೆ ಪರಧರ್ಮಗಳಾದ ಕ್ರೈಸ್ತ, ಮಹಮ್ಮದೀಯರ ಧರ್ಮವನ್ನವಲಂಬಿಸತೊಡಗಿದಾಗ, ನೀವು ಅವರಿಗೆ ಏನನ್ನಾದರೂ ಮಾಡುವುದನ್ನು ಬಿಟ್ಟು, ಇಂಗ್ಲೆಂಡ್ ಮತ್ತು ಅಮೆರಿಕಾ ದೇಶಕ್ಕೆ ಹಿಂದೂಧರ್ಮ ಪ್ರಚಾರ ಮಾಡಲು ಹೋದಿರಲ್ಲ?

ಸ್ವಾಮೀಜಿ: ಪರಿಸ್ಥಿತಿ ಬದಲಾಯಿಸಿರುವುದು ಕಾಣುವುದಿಲ್ಲವೆ? ನಮ್ಮ ದೇಶೀಯರಿಗೆ ಸತ್ಯವಾದ ಧರ್ಮವನ್ನು ಅವಲಂಬಿಸಿ ಅದನ್ನು ಸಾಧನೆಮಾಡುವಷ್ಟು ಶಕ್ತಿಯಾದರೂ ಉಳಿದಿರುವುದೇನು? ತಾವು ಸಾತ್ವಿಕರೆಂದು ತಮ್ಮಲ್ಲಿ ತಾವೇ ಹೆಮ್ಮೆ ಪಡುವುದೊಂದೇ ಅವರಿಗಿರುವುದು. ಒಂದು ಕಾಲದಲ್ಲಿ ಅವರು ಸಾತ್ವಿಕರಾಗಿದ್ದರೆಂಬುದು ಖಂಡಿತವಾಗಿಯೂ ನಿಜ. ಆದರೆ ಈಗ ಅವರು ತುಂಬಾ ಅಧೋಗತಿಗಿಳಿದಿರುವರು. ಸತ್ತ್ವದಿಂದ ಉರುಳಿದವನು ಒಂದೇ ಬಾರಿ ತಾಮಸಕ್ಕೆ ಬೀಳುವನು. ಅವರಿಗಾಗಿರುವುದು ಇದೇ. ಯಾವ ಮನುಷ್ಯ ಕೊಂಚವೂ ದೇಹಶ್ರಮ ಪಡದೆ ಕೇವಲ ಹರಿನಾಮವನ್ನು ಜಪಿಸುತ್ತಾ ಕೊಠಡಿಯ ಬಾಗಿಲನ್ನು ಮುಚ್ಚಿಕೊಂಡು ಕುಳಿತಿರುವನೊ, ಯಾರು ಸ್ವತಃ ತನ್ನ ಕಣ್ಣೆದುರಿಗೇ ಇತರರಿಗಾಗುತ್ತಿರುವ ಅನ್ಯಾಯ ಅತ್ಯಾಚಾರಗಳನ್ನು ನೋಡಿಯೂ ಮೌನವಾಗಿ ಉದಾಸೀನದಿಂದಿರುವನೊ ಅಂತಹ ಮನುಷ್ಯನಿಗೆ ಸಾತ್ತ್ವಿಕಗುಣವಿದೆಯೆನ್ನುವೆಯೇನು? ಖಂಡಿತವಾಗಿಯೂ ಇಲ್ಲ. ಅಂಥವನು ಸಂಪೂರ್ಣ ತಾಮಸದ ಆವರಣದಿಂದ ಆಚ್ಛಾದಿತನಾಗಿರುವನು. ಒಂದು ದೇಶದ ಜನರು ತಮ್ಮ ಹಸಿವನ್ನಿಂಗಿಸಲು ಸಾಕಷ್ಟು ಆಹಾರವೂ ಇಲ್ಲದೆ ಇದ್ದಾಗ ಹೇಗೆ ತಾನೇ ಧರ್ಮವನ್ನು ಸಾಧಿಸಬಲ್ಲರು? ಭೋಗದ ಆಸೆ ಆಕಾಂಕ್ಷೆ ತುಂಬಿದ ಮನಸ್ಸಿಗೆ ಕೊಂಚವೂ ತೃಪ್ತಿ ಸಿಗದೆ ಇರುವಾಗ ಈ ದೇಶದ ಜನರಿಗೆ ತ್ಯಾಗಬುದ್ಧಿ ಹೇಗೆ ತಾನೇ ಬರುವುದು? ಆದ್ದರಿಂದಲೇ ಮೊಟ್ಟಮೊದಲು ಜನರಿಗೆ ಸಾಕಷ್ಟು ತಿನ್ನಲು ಆಹಾರ ಮತ್ತು ಕೊಂಚ ಭೋಗಿಸಲು ಸಾಕಷ್ಟು ಬೊಗಸಾಮಗ್ರಿ, ಇಷ್ಟು ಸಿಗುವ ಮಾರ್ಗೊಪಾಯವನ್ನು ಕಂಡುಹಿಡಿ. ನಂತರ ಕ್ರಮೇಣ ನಿಜವಾದ ವೈರಾಗ್ಯ ಬರುವುದು. ಜನರು ಜೀವನದಲ್ಲಿ ಧರ್ಮಸಾಕ್ಷಾತ್ಕಾರ ಮಾಡಿಕೊಳ್ಳಲು ಅರ್ಹರಾಗಿ ಸಿದ್ಧರಾಗುವರು. ಇಂಗ್ಲೆಂಡ್ ಮತ್ತು ಅಮೆರಿಕಾ ದೇಶೀಯರು ಎಂತಹ ರಾಜಸಿಕ ಪ್ರವೃತ್ತಿಯಿಂದ ತುಂಬಿ ತುಳುಕಾಡುತ್ತಿರುವರು! ಅವರು ಎಲ್ಲಾ ಬಗೆಯ ಪ್ರಾಪಂಚಿಕ ಸುಖಭೋಗಗಳಿಂದಲೂ ಸಂತೃಪ್ತರಾಗಿದ್ದಾರೆ. ಅದೂ ಅಲ್ಲದೆ ಶ್ರದ್ಧೆ ಮತ್ತು ಮೂಢಭಕ್ತಿಯ ಧರ್ಮವಾದ ಕ್ರೈಸ್ತಧರ್ಮ ನಮ್ಮ ಪುರಾಣಗಳ ಧರ್ಮದ ಮಟ್ಟಕ್ಕೆ ಸೇರುವುದು. ಪಾಶ್ಚಾತ್ಯರಿಗೆ, ವಿದ್ಯೆ ಮತ್ತು ಸಂಸ್ಕೃತಿಯ ಹರಡುವಿಕೆಯಿಂದಾಗಿ ಇನ್ನು ಖಂಡಿತವಾಗಿಯೂ ಅದರಲ್ಲಿ ಶಾಂತಿ ಸಿಗುವುದಿಲ್ಲ. ಅವರ ಈಗಿನ ಸ್ಥಿತಿ ಹೇಗಿದೆಯೆಂದರೆ ಅವರನ್ನು ಕೊಂಚ ಎತ್ತಿದರೆ ಸಾಕು, ಅವರು ಸಾತ್ತ್ವಿಕರಾಗುವರು. ಅದೂ ಅಲ್ಲದೆ ಈಗಿನ ಕಾಲದಲ್ಲಿ ಒಬ್ಬ ಬಿಳಿಯನು (ಪಾಶ್ಚಾತ್ಯ) ಬಂದು ನಿಮ್ಮ ಧರ್ಮದಮೇಲೆ ಮಾತನಾಡಿದಾಗ ಅವನ ಮಾತನ್ನು ಒಪ್ಪಿಕೊಳ್ಳುವಂತೆ ಹರಕು ಚಿಂದಿ ಧರಿಸಿದ ಸಂನ್ಯಾಸಿಯೊಬ್ಬನ ಮಾತನ್ನು ಒಪ್ಪಿಕೊಳ್ಳುವಿರಾ?

ನಾನು: ನಿಜವಾಗಿ ಸರಿ! ಸ್ವಾಮೀಜಿ. ಶ‍್ರೀಮಾನ್ ಎನ್. ಘೋಷ್‌ರೂ ಕೂಡಾ ಇದೇ ಭಾವನೆಯಿಂದ ಮಾತನಾಡುತ್ತಿದ್ದರು.

ಸ್ವಾಮೀಜಿ: ಹೌದು, ಸರಿಯಾದ ತರಬೇತಿ ಮತ್ತು ಜ್ಞಾನವನ್ನು ಗಳಿಸಿ ನನ್ನ ಐರೋಪ್ಯ ಶಿಷ್ಯರು ಒಬ್ಬರಾಗುತ್ತಲೊಬ್ಬರು ಬಂದು ನಿಮ್ಮನ್ನು ‘ನೀವೆಲ್ಲ ಏನು ಮಾಡುತ್ತಿರುವಿರಿ? ನಿಮಗೇಕೆ ಇಷ್ಟೊಂದು ಅಲ್ಪಶ್ರದ್ಧೆ? ನಿಮ್ಮ ಆಚಾರ, ವ್ಯವಹಾರ, ಧರ್ಮ, ನಡೆನುಡಿ, ನೀತಿಗಳು ಯಾವ ಭಾಗದಲ್ಲಿ ಕಡಿಮೆಯಾಗಿದೆ? ನಾವು ನಿಮ್ಮ ಧರ್ಮವೇ ಅತ್ಯಂತ ಉಚ್ಚತಮವಾದುದೆಂದು ಭಾವಿಸುವೆವು’ ಎಂದು ಹೇಳಿದರೆ, ನಮ್ಮಲ್ಲಿಯೇ ಅನೇಕ ಮಂದಿ ದೊಡ್ಡವರ ಮತ್ತು ಪ್ರತಿಭಾವಂತರ ತಂಡವೇ ಅವರ ಮಾತನ್ನು ಕೇಳಲು ಬರುವುದು. ಹೀಗೆ ಅವರಿಂದ ನಮ್ಮ ದೇಶಕ್ಕೆ ಲಾಭವಾಗುವುದು. ಅವರು ನಿಮ್ಮ ಧರ್ಮಬೋಧಕರ ಸ್ಥಾನವನ್ನಾಕ್ರಮಿಸಲು ಬರುವರೆಂದು ಒಂದು ಗಳಿಗೆಯೂ ಯೋಚಿಸಬೇಡ. ನಿಸ್ಸಂದೇಹವಾಗಿ ಅವರು ನಿಮ್ಮ ಪ್ರಾಪಂಚಿಕ ಸ್ಥಿತಿಗಳ ಸುಧಾರಣೆಗಾಗಿ ಮಾಡುವ ವ್ಯಾವಹಾರಿಕ ವಿಜ್ಞಾನಗಳ ವಿಷಯದಲ್ಲಿ ನಿಮ್ಮ ಗುರುಗಳಾಗುವರು - ನಮ್ಮ ಜನರು ಧರ್ಮಕ್ಕೆ ಸಂಬಂಧಪಟ್ಟ ವಿಷಯವೆಲ್ಲಕ್ಕೂ ಅವರಿಗೆ ಗುರುಗಳಾಗುವರು. ಈ ಬಗೆಯ ಗುರು ಶಿಷ್ಯ ಬಾಂಧವ್ಯವು ಧರ್ಮದ ಕ್ಷೇತ್ರದಲ್ಲಿ ಭರತಖಂಡ ಮತ್ತು ಪ್ರಪಂಚದ ಇನ್ನುಳಿದ ಭಾಗಗಳ ನಡುವೆ ಎಂದೆಂದಿಗೂ ಉಳಿದಿರುವುದು.

ನಾನು: ಅದು ಹೇಗೆ ಸಾಧ್ಯ ಸ್ವಾಮೀಜಿ? ಅವರು ಎಂತಹ ದ್ವೇಷಭಾವನೆಯಿಂದ ನಮ್ಮನ್ನು ಕಾಣುತ್ತಿರುವರೆಂಬ ಬಗ್ಗೆ ವಿಚಾರಿಸಿ ನೋಡಿದರೆ ಅವರು ಸಂಪೂರ್ಣ ನಿಃಸ್ವಾರ್ಥ ಭಾವನೆಯಿಂದ ನಮಗೆ ಒಳ್ಳೆಯದನ್ನು ಮಾಡುವರೆಂಬುದು ಅಸಂಭವ ಅನ್ನಿಸುವುದು.

ಸ್ವಾಮೀಜಿ: ಅವರು ನಮ್ಮನ್ನು ದ್ವೇಷಿಸಲು ಅನೇಕ ಕಾರಣಗಳಿವೆ. ಅದಕ್ಕೇ ಅವರು ತಾವೀರೀತಿ ಮಾಡುವುದು ಸರಿಯೆಂದು ತಮ್ಮನ್ನು ಸಮರ್ಥಿಸಿಕೊಳ್ಳಬಹುದು. ಮೊದಲನೆಯದಾಗಿ ನಾವು ಅವರಿಂದ ಜಯಿಸಲ್ಪಟ್ಟ ಜನಾಂಗ. ಅಲ್ಲದೆ ಇಷ್ಟೊಂದು ಮಂದಿ ಭಿಕ್ಷುಕರಿರುವ ಜನಾಂಗ ಪ್ರಪಂಚದಲ್ಲಿ ಮತ್ತೆಲ್ಲಿಯೂ ಇಲ್ಲ. ನಮ್ಮಲ್ಲಿ ಅಧಿಕ ಸಂಖ್ಯೆಯಲ್ಲಿರುವ ಕೀಳುಜಾತಿಯ ಜನರು ಮೇಲು ಜಾತಿಯವರ ನಿರಂತರ ದಬ್ಬಾಳಿಕೆಯಿಂದಲೂ ಮತ್ತು ಅವರಿಂದ ಹೆಜ್ಜೆ ಹೆಜ್ಜೆಗೂ ಹೊಡೆತ ಒದೆತಗಳನ್ನು ತಿಂದು ಕಡೆಗೆ ತಮ್ಮ ಪುರುಷತ್ವವನ್ನೇ ಕಳೆದುಕೊಂಡು ವೃತ್ತಿಯಿಂದಲೇ ಭಿಕ್ಷುಕರಾಗಿಬಿಟ್ಟಿದ್ದಾರೆ. ಇದರಿಂದ ಒಂದು ಹೆಜ್ಜೆ ಮುಂದೆ ಹೋದವರು ಒಂದೆರಡು ಪುಟ ಇಂಗ್ಲಿಷನ್ನೋದಿ ಕೈಯಲ್ಲಿ ಅರ್ಜಿಯನ್ನು ಹಿಡಿದು ಸಾರ್ವಜನಿಕ ಕಛೇರಿಗಳ ಬಾಗಿಲಲ್ಲಿ ಬಿದ್ದಿದ್ದಾರೆ. ಇಪ್ಪತ್ತು ಅಥವಾ ಮೂವತ್ತು ರೂಪಾಯಿ ಸಂಬಳವಿರುವ ಕೆಲಸ ಖಾಲಿಬಿದ್ದರೆ ಐನೂರು ಮಂದಿ ಬಿ.ಎ. ಮತ್ತು ಎಂ.ಎ.ಗಳು ಅದಕ್ಕೆ ಅರ್ಜಿ ಹಾಕಿಕೊಳ್ಳುವರು. ಆಹಾ! ಆ ಅರ್ಜಿಗಳಾದರೂ ಇಂತಹ ವಿಚಿತ್ರ ಪದಗಳಿಂದ ಕೂಡಿರುತ್ತದೆ: “ನನಗೆ ಮನೆಯಲ್ಲಿ ತಿನ್ನಲಿಕ್ಕೆ ಏನೇನೂ ಇಲ್ಲ. ಹೆಂಡತಿ ಮಕ್ಕಳು ಉಪವಾಸ, ಸ್ವಾಮಿ, ನನಗೆ ಮತ್ತು ನನ್ನ ಸಂಸಾರಕ್ಕೆ ಒಂದು ಹಾದಿಯನ್ನು ಕಲ್ಪಿಸಿಕೊಡಬೇಕೆಂದು ನಿಮ್ಮನ್ನು ಸೆರಗೊಡ್ಡಿ ಬೇಡಿಕೊಳ್ಳುತ್ತಿದ್ದೇನೆ. ಇಲ್ಲದಿದ್ದಲ್ಲಿ ನಾವೆಲ್ಲ ಉಪವಾಸದಿಂದ ಸಾಯುವೆವು!?” ಅವರು ಕೆಲಸಕ್ಕೆ ಸೇರಿದಮೇಲೆ ಕೂಡಾ ತಮ್ಮ ಆತ್ಮಗೌರವವನ್ನೆಲ್ಲಾ ಗಾಳಿಗೆ ತೂರಿಬಿಡುವರು. ಅವರು ಸಾಧಿಸುವುದು ಅತ್ಯಂತ ಕೀಳು ದರ್ಜೆಯ ಗುಲಾಮಗಿರಿ, ನಮ್ಮ ದೇಶದ ಜನಸಾಮಾನ್ಯರ ಪರಿಸ್ಥಿತಿ ಹೀಗಿದೆ! ನಿಮ್ಮಲ್ಲಿ ಉಚ್ಚತಮ ವಿದ್ಯಾವಂತರಾಗಿ ಪ್ರಮುಖರಾಗಿರುವವರೆಲ್ಲಾ ತಾವೇ ಒಂದು ಸಮಾಜ ರಚಿಸಿಕೊಂಡು ಉಚ್ಚ ಕಂಠದಿಂದ “ಅಯ್ಯೋ! ಭರತಖಂಡವು ದಿನದಿನಕ್ಕೆ ಹಾಳಾಗಿ ಹೋಗುತ್ತಿದೆ. ಓ! ಆಂಗ್ಲೇಯ ಪ್ರಭುಗಳೇ, ನಮ್ಮ ದೇಶೀಯರಿಗೆ ದೊಡ್ಡ ದೊಡ್ಡ ಹುದ್ದೆಗಳು ದೊರಕುವಂತೆ ಮಾಡಿ. ಕ್ಷಾಮದಿಂದ ನಮ್ಮನ್ನು ಪಾರುಮಾಡಿ” ಎಂದು ಮೊರೆಯಿಡುತ್ತಾ ಹಗಲೂ ರಾತ್ರಿ ‘ಕೊಡಿ, ’ ‘ಕೊಡಿ’ ಎಂಬ ಅವಿಚ್ಛಿನ್ನ ರೋದನದಿಂದ ಆಕಾಶವನ್ನು ತುಂಬುತ್ತೀರಿ. ಅವರ ಭಾಷಣದ ಸಾರಾಂಶವೆಲ್ಲಾ “ಆಂಗ್ಲೇಯರೇ ನಮಗೆ ಕೊಡಿ, ಮತ್ತೂ ಹೆಚ್ಚಾಗಿ ಕೊಡಿ!?” ಅಯ್ಯೋ ಅವರು ನಿಮಗೆ ಮತ್ತಿನ್ನೇನು ಕೊಡಬೇಕು? ಅವರು ಹೊಗೆಬಂಡಿ, ತಂತಿಸಮಾಚಾರ, ದೇಶಕ್ಕೆ ಸುವ್ಯವಸ್ಥಿತವಾದ ರಾಜ್ಯಾಡಳಿತವನ್ನು ಕೊಟ್ಟಿದ್ದಾರೆ - ಕಳ್ಳರನ್ನು ಹೆಚ್ಚುಕಡಿಮೆ ಸಂಪೂರ್ಣವಾಗಿ ಅಡಗಿಸಿದ್ದಾರೆ. ವಿಜ್ಞಾನ ವ್ಯಾಸಂಗಕ್ಕೆ ಅವಕಾಶ ಕಲ್ಪಿಸಿದ್ದಾರೆ - ಅವರು ಇನ್ನೇನನ್ನು ಕೊಡಬೇಕು? ಸಂಪೂರ್ಣ ನಿಃಸ್ವಾರ್ಥತೆಯಿಂದ ಇನ್ನೊಬ್ಬರಿಗೆ ಏನನ್ನೇ ಆಗಲಿ ಯಾರು ಕೊಡುವರು? ಅವರು ನಿಮಗೆ ಎಷ್ಟೊಂದು ಕೊಟ್ಟಿದ್ದಾರೆ. ನಾನು ಕೇಳುವೆ, ನೀವು ಅದಕ್ಕೆ ಪ್ರತಿಯಾಗಿ ಅವರಿಗೆ ಏನನ್ನು ಕೊಟ್ಟಿರುವಿರಿ?

ನಾನು: ನಾವೇನನ್ನು ಕೊಡಬೇಕು ಸ್ವಾಮೀಜಿ? ನಾವು ತೆರಿಗೆಯನ್ನು ಕೊಡುವುದಿಲ್ಲವೆ?

ಸ್ವಾಮೀಜಿ: ಹೌದೇನು, ನಿಜವಾಗಿ? ನೀವು ಕೇವಲ ನಿಮ್ಮ ಸ್ವಂತ ಇಚ್ಛೆಯಿಂದ ತೆರಿಗೆಯನ್ನು ಕೊಡುವಿರೇನು? ಅಥವಾ ದೇಶ ಶಾಂತಿಯಿಂದಿರಲು ಅವರೇ ಕಡ್ಡಾಯವಾಗಿ ಅದನ್ನು ಸೆಳೆದುಕೊಳ್ಳುವರೆ? ನಿರ್ವಂಚನೆಯಿಂದ ಹೇಳು, ಅವರು ನಮಗೆ ಮಾಡಿರುವುದಕ್ಕೆಲ್ಲಾ ನೀವೇನು ಪ್ರತಿಫಲ ಕೊಟ್ಟಿದ್ದೀರಿ? ಅವರಿಗೆ ಯಾವುದಿಲ್ಲವೊ ಅದನ್ನು ನೀವು ಕೊಡಬೇಕು. ನೀವು ಇಂಗ್ಲೆಂಡಿಗೆ ಹೋಗುವಿರಿ. ಆದರೆ ಅದೂ ಕೂಡ ಭಿಕ್ಷುಕನ ತೊಡಿಗೆಯಲ್ಲಿ - ವಿದ್ಯಾರ್ಜನೆಗೆ ಕೆಲವರು ಹೋಗುವರು. ಅವರು ಬಹುಶಃ, ಕೆಲವು ಕಡೆ ಪಾಶ್ಚಾತ್ಯ ಧರ್ಮವನ್ನು ಪ್ರಶಂಸಿಸಿ ಭಾಷಣ ಮಾಡಿ ಹಿಂತಿರುಗಿ ಬರಬಹುದು, ಅಷ್ಟೆ. ಎಂತಹ ಮಹಾಕಾರ್ಯ ಸಾಧನೆ! ನಿಜವಾಗಿ! ಏಕೆ? ಅವರಿಗೆ ಕೊಡಲು ನಿಮ್ಮಲ್ಲೇನೂ ಇಲ್ಲವೆ? ನೀವು ಕೊಡಬೇಕಿದ್ದರೆ ನಿಮ್ಮಲ್ಲಿ ಅಮೂಲ್ಯ ಸಂಪತ್ತಿದೆ - ನಿಮ್ಮ ಧರ್ಮವನ್ನು ಕೊಡಿ. ನಿಮ್ಮ ತತ್ತ್ವಶಾಸ್ತ್ರವನ್ನು ಕೊಡಿ! ಇಡೀ ಜಗತ್ತಿನ ಇತಿಹಾಸವನ್ನೆಲ್ಲಾ ಓದಿನೋಡಿ, ಎಲ್ಲೆಲ್ಲಿ ಉದಾತ್ತ ಧ್ಯೇಯಗಳಿರುವುವೋ ಅಲ್ಲೆಲ್ಲ ಭರತಖಂಡವೇ ಅದಕ್ಕೆ ಮೂಲವೆಂದು ನೋಡುವೆ. ಪ್ರಾಚೀನ ಕಾಲದಿಂದಲೂ ಭರತಖಂಡ ಅಮೋಘವಾದ ಆದರ್ಶಗಳ ಗಣಿಯಾಗಿದೆ. ತಾನೇ ಮಹತ್ತಾದ ಧ್ಯೇಯಗಳ ತೌರೂರಾಗಿ ಅವುಗಳನ್ನು ಇಡೀ ಜಗತ್ತಿಗೇ ಉದಾರವಾಗಿ ಪ್ರಚಾರ ಮಾಡಲು ನೀಡಿದೆ. ಇಂದು ಆಂಗ್ಲೇಯರು ಭರತಖಂಡದಲ್ಲಿರುವುದು ಆ ಮಹತ್ತಾದ ಉದ್ದೇಶಗಳನ್ನು ಸಂಗ್ರಹಿಸಲು, ವೇದಾಂತಜ್ಞಾನವನ್ನಾರ್ಜಿಸಲು ಮತ್ತು ನಿಮ್ಮ ಶಾಶ್ವತ ಧರ್ಮದಲ್ಲಿರುವ ಗಾಢ ರಹಸ್ಯಗಳನ್ನು ಒಳಹೊಕ್ಕು ನೋಡಲು. ನೀವು ಅವರಿಂದ ಪಡೆದಿರುವುದಕ್ಕೆ ಪ್ರತಿಯಾಗಿ ನೀವು ಈ ಅಮೂಲ್ಯ ರತ್ನಗಳನ್ನವರಿಗೆ ನೀಡಿ. ನಮಗೆ ಅವರು ಮಾಡಿರುವ ಅಪಮಾನವನ್ನು ತೊಡೆದುಹಾಕುವುದಕ್ಕಾಗಿ ದೇವರು ನನ್ನನ್ನು ಅವರ ದೇಶಕ್ಕೆ ಹೋಗುವಂತೆ ಮಾಡಿದ. ಆದರೆ ಬರೇ ತಿರುಪೆಗಾಗಿಯೇ ಇಂಗ್ಲೆಂಡಿಗೆ ಹೋಗುವುದು ಸರಿಯಲ್ಲ. ಅವರೇ ಯಾವಾಗಲೂ ನಮಗೆ ಭಿಕ್ಷೆ ನೀಡಬೇಕೆ? ಎಂದೆಂದೂ ಯಾರಾದರೂ ಹಾಗೆ ಮಾಡುವರೆ? ನಿರಂತರವಾಗಿ ಭಿಕ್ಷುಕರಂತೆ ಕೈಚಾಚಿ ದಾನವನ್ನು ತೆಗೆದುಕೊಳ್ಳುವುದು ಪ್ರಕೃತಿ ನಿಯಮವಲ್ಲ. ಯಾವ ವ್ಯಕ್ತಿಯೇ ಆಗಲಿ, ಪಂಗಡ ಅಥವಾ ಜನಾಂಗವೇ ಆಗಲಿ ಈ ನಿಯಮಗಳನ್ನು ಪಾಲಿಸದಿದ್ದರೆ ಜೀವನದಲ್ಲಿ ಎಂದಿಗೂ ಏಳ್ಗೆ ಹೊಂದುವುದಿಲ್ಲ. ನಾವೂ ಕೂಡ ಆ ನಿಯಮವನ್ನನುಸರಿಸಬೇಕು. ಅದಕ್ಕೇ ನಾನು ಅಮೆರಿಕಾಕ್ಕೆ ಹೋದುದು. ಅಲ್ಲಿಯ ಜನರಿಗೆ ಧರ್ಮದ ತೃಷ್ಣೆ ಎಷ್ಟು ಉತ್ಕಟವಾಗಿದೆಯೆಂದರೆ ನನ್ನಂತಹ ಸಾವಿರಾರು ಮಂದಿ ಅಲ್ಲಿಗೆ ಹೋದರೂ ಅವರಿಗೆ ಸಾಕಷ್ಟು ಸ್ಥಳವಿದೆ. ದೀರ್ಘಕಾಲದಿಂದಲೂ ಅವರು ನಿಮಗೆ ತಮ್ಮಲ್ಲಿರುವ ಐಶ್ವರ್ಯವನ್ನು ದಾನ ಮಾಡುತ್ತ ಬಂದಿರುವರು. ಈಗ ನೀವು ನಿಮ್ಮ ಅಮೂಲ್ಯ ಸಂಪತ್ತನ್ನು ಹಂಚಿಕೊಳ್ಳುವ ಸಕಾಲ ಬಂದಿದೆ. ಎಷ್ಟು ಬೇಗ ಅವರ ದ್ವೇಷಭಾವನೆ ನಿಮ್ಮಲ್ಲಿರುವ ಶ್ರದ್ಧೆ, ಭಕ್ತಿ, ಗೌರವದಿಂದ ಮಾರ್ಪಟ್ಟು ನೀವು ಕೇಳದೆಯೇ ಎಷ್ಟೊಂದು ಕಲ್ಯಾಣವನ್ನು ಅವರು ನಿಮ್ಮ ದೇಶಕ್ಕೆ ಮಾಡುವರೆಂಬುದನ್ನು ನೀನೇ ನೋಡುವೆ. ಅವರೊಂದು ಶೂರ ಜನಾಂಗ. ಅವರೆಂದೂ ತಮಗಾದ ಕಲ್ಯಾಣವನ್ನು ಮರೆಯುವವರಲ್ಲ.

ನಾನು: ಸರಿ, ಸ್ವಾಮೀಜಿ, ನೀವು ಪಾಶ್ಚಾತ್ಯ ದೇಶಗಳಲ್ಲಿ ಕೊಟ್ಟ ಭಾಷಣಗಳಲ್ಲಿ ಪದೇಪದೇ ನೀವು ನಮ್ಮ ವೈಯಕ್ತಿಕವಾದ ಮೇಧಾಶಕ್ತಿ ಮತ್ತು ಸದ್ಗುಣಗಳ ವಿಷಯವಾಗಿ ಅಮೋಘವಾದ ಭಾಷಣ ಮಾಡಿದಿರಿ. ನಮಗೆ ಧರ್ಮದ ಮೇಲಿರುವ ಮನಃಪೂರ್ವಕವಾದ ಪ್ರೇಮವನ್ನು ಪುಷ್ಟೀಕರಿಸುವ ಅನೇಕ ಪ್ರಮಾಣಗಳನ್ನು ಅವರ ಮುಂದಿಟ್ಟಿದ್ದೀರಿ. ಆದರೆ ಈಗ ನಾವು ಸಂಪೂರ್ಣ ತಾಮಸದಲ್ಲಿ ಮುಳುಗಿದ್ದೇವೆಂದು ಹೇಳುತ್ತಿದ್ದೀರಿ. ಅದೇ ಸಮಯದಲ್ಲಿ ನಾವು ಋಷಿಗಳ ಶಾಶ್ವತ ಧರ್ಮದ ಆಚಾರ್ಯರೆಂದು ಜಗತ್ತೆಲ್ಲಾ ಅಂಗೀಕರಿಸುವಂತೆ ಮಾಡುತ್ತಿದ್ದೀರಿ! ಅದು ಹೇಗೆ?

ಸ್ವಾಮೀಜಿ: ನಾನು ಪರದೇಶಗಳಿಗೆ ಹೋಗಿ ಜನಸಾಮಾನ್ಯರೆದುರಿಗೆ ನಮ್ಮ ನ್ಯೂನತೆಗಳನ್ನು ಕುರಿತು ವಿಸ್ತಾರವಾಗಿ ಉಪನ್ಯಾಸ ಮಾಡಬೇಕೆಂದು ಹೇಳುತ್ತೀಯೇನು! ಅದಕ್ಕೆ ಬದಲು ನಮ್ಮನ್ನು ಒಂದು ಜನಾಂಗವಾಗಿ ಪ್ರತ್ಯೇಕಿಸುವ ವಿಶೇಷ ಸದ್ಗುಣಗಳನ್ನು ಅವರ ಮುಂದೆ ಎತ್ತಿ ಹಿಡಿಯಬೇಕಲ್ಲವೆ? ಒಬ್ಬ ಮನುಷ್ಯನಿಗೆ ನೇರವಾಗಿ ಅವನಲ್ಲಿರುವ ನ್ಯೂನತೆಯನ್ನು ಹೇಳು; ಸ್ನೇಹಭಾವನೆಯಿಂದ ಅವುಗಳನ್ನು ಅವನಿಗೆ ಮನದಟ್ಟು ಮಾಡಿಸಿ ಅವನು ತಿದ್ದಿಕೊಳ್ಳುವಂತೆ ಮಾಡುವುದು ಯಾವಾಗಲೂ ಒಳ್ಳೆಯದು. ಆದರೆ ಇತರರ ಮುಂದೆ ನೀನು ಅವನ ಸುಗುಣಗಳನ್ನು ಪ್ರಶಂಸೆ ಮಾಡಬೇಕು. ನಾವು ಒಬ್ಬ ಕೆಟ್ಟ ಮನುಷ್ಯನನ್ನು ಪದೇಪದೇ ಅವನು ಒಳ್ಳೆಯವನೆಂದು ಹೇಳುತ್ತಿದ್ದರೆ ಸಕಾಲದಲ್ಲಿ ಅವನು ಒಳ್ಳೆಯವನೇ ಆಗುತ್ತಾನೆ ಎಂದೂ ಹಾಗೆಯೇ ಒಬ್ಬ ಒಳ್ಳೆಯ ಮನುಷ್ಯನನ್ನು ಕೆಟ್ಟವನೆಂದು ಪದೇಪದೇ ಹೇಳುತ್ತಿದ್ದರೆ ಅವನು ಕೆಟ್ಟವನೇ ಆಗುವನು ಎಂದೂ ಶ‍್ರೀರಾಮಕೃಷ್ಣರು ಹೇಳುತ್ತಿದ್ದರು. ಪಾಶ್ಚಾತ್ಯ ದೇಶಗಳಲ್ಲಿ ಆ ಜನರಿಗೆ ನಾನು ಸಾಕಷ್ಟು ಅವರ ಕುಂದುಕೊರತೆಗಳನ್ನು ಹೇಳಿದ್ದೇನೆ. ನೆನಪಿನಲ್ಲಿದೆಯೆ! ನನ್ನ ಕಾಲದವರೆಗೂ ನಮ್ಮ ದೇಶದಿಂದ ಐರೋಪ್ಯ ದೇಶಗಳಿಗೆ ಹೋದವರೆಲ್ಲಾ ಅವರ ಉತ್ಕೃಷ್ಟತೆಯನ್ನು ಹೊಗಳಿ, ಅವರ ಮೇಲೆ ಜಯಘೋಷ ಗೀತೆಯನ್ನು ಹಾಡಿದ್ದಾರೆ. ಅವರ ಕಿವಿಗೆ ಕೇವಲ ನಮ್ಮ ನ್ಯೂನತೆಗಳ ಕಹಳೆಯನ್ನೂದಿದ್ದಾರೆ. ಪರಿಣಾಮವಾಗಿ ಅವರು ನಮ್ಮನ್ನು ದ್ವೇಷಿಸಲು ತೊಡಗಿರುವುದರಲ್ಲಿ ಆಶ್ಚರ್ಯವೇನೂ ಇಲ್ಲ. ಈ ಕಾರಣದಿಂದಲೇ ಈಗ ನಾನು ನಿಮಗೆ ನಿಮ್ಮ ನ್ಯೂನತೆಗಳನ್ನು ತೋರಿಸಿ ಅವರ ಒಳ್ಳೆಯ ಗುಣಗಳನ್ನು ಹೇಳುತ್ತಿರುವಂತೆಯೇ, ಅವರ ಮುಂದೆ ನಿಮ್ಮ ಸದ್ಗುಣಗಳನ್ನಿಟ್ಟು ಅವರ ಕುಂದುಕೊರತೆಗಳನ್ನು ತೋರಿಸಿಕೊಟ್ಟಿದ್ದೇನೆ. ನೀವೆಷ್ಟೇ ತಾಮಸದಿಂದ ತುಂಬಿದ್ದರೂ ಪುರಾತನ ಋಷಿಗಳ ಸ್ವಭಾವದ ಅಲ್ಪಾಂಶವಾದರೂ ನಿಸ್ಸಂದೇಹವಾಗಿ ಇನ್ನೂ ನಿಮ್ಮಲ್ಲಿದೆ. ಕಡೆಯಪಕ್ಷ ಅಸ್ತಿಭಾರವಾದರೂ ಇದೆ. ಆದರೆ ಇದರಿಂದ ನಾವು ಈ ತಕ್ಷಣವೇ ಆಚಾರ್ಯ ಪಟ್ಟ ವಹಿಸಿ ಐರೋಪ್ಯ ದೇಶಗಳಿಗೆ ಬೋಧಿಸಲು ಹೊರಡಬೇಕೆಂದರ್ಥವಲ್ಲ. ಮೊಟ್ಟಮೊದಲು ಏಕಾಂತದಲ್ಲಿ ನಮ್ಮ ಆಧ್ಯಾತ್ಮಿಕ ಜೀವನವನ್ನು ಸಂಪೂರ್ಣವಾಗಿ ರೂಪಿಸಿಕೊಳ್ಳಬೇಕು, ಸಂಪೂರ್ಣ ತ್ಯಾಗಜೀವಿಗಳಾಗಬೇಕು, ಬ್ರಹ್ಮಚರ್ಯವನ್ನು ನಿರಂತರವಾಗಿ ರಕ್ಷಿಸಬೇಕು. ತಾಮಸ ಪ್ರವೃತ್ತಿ ನಿಮ್ಮನ್ನು ಪ್ರವೇಶಿಸಿದೆ - ಆದರೇನಂತೆ? ತಾಮಸವನ್ನು ಧ್ವಂಸ ಮಾಡಲಿಕ್ಕಾಗುವುದಿಲ್ಲವೆ? ಒಂದು ಕ್ಷಣದಲ್ಲಿ ಅದನ್ನು ಮಾಡಬಹುದು. ಈ ತಾಮಸದ ನಾಶಕ್ಕಾಗಿಯೇ ಭಗವಾನ್ ಶ‍್ರೀರಾಮಕೃಷ್ಣರು ನಮ್ಮಲ್ಲಿಗೆ ಬಂದುದು.

ನಾನು: ಸ್ವಾಮೀಜಿ, ನಿಮ್ಮಂತೆ ಬೇರೆ ಯಾರು ಆಗಬಲ್ಲರು?

ಸ್ವಾಮೀಜಿ: ನಾನು ಸತ್ತಮೇಲೆ ವಿವೇಕಾನಂದರು ಇರುವುದಿಲ್ಲವೆಂದು ಯೋಚಿಸುವೆಯೇನು? ಈಗ ಸ್ವಲ್ಪ ಹೊತ್ತಿಗೆ ಮುಂಚೆ ನನ್ನ ಮುಂದೆ ಸಂಗೀತವನ್ನು ಹಾಡಿದರಲ್ಲ ಆ ಯುವಕರ ತಂಡದಲ್ಲಿ, ಯಾರನ್ನು ನೀವು ಕುಡುಕರು ದುರ್ವ್ಯಸನಿಗಳು, ಕೆಲಸಕ್ಕೆ ಬಾರದವರು ಎಂದು ತಿರಸ್ಕರಿಸುತ್ತಿದ್ದೀರೋ ಅವರಲ್ಲಿ, ದೇವೇಚ್ಛೆಯಿದ್ದರೆ, ಒಬ್ಬೊಬ್ಬನೂ ಒಬ್ಬ ವಿವೇಕಾನಂದನಾಗಬಹುದು. ಜಗತ್ತಿಗೆ ಆವಶ್ಯಕತೆಯಿದ್ದರೆ, ವಿವೇಕಾನಂದರಂಥವರಿಗೆ ಕೊರತೆಯಿರುವುದಿಲ್ಲ - ಸಾವಿರಾರು ಲಕ್ಷಾಂತರ ವಿವೇಕಾನಂದರು ಕಾಣಿಸಿಕೊಳ್ಳುವರು. ಎಲ್ಲಿಂದ ಯಾರಿಗೆ ಗೊತ್ತು! ನನ್ನಿಂದ ಮಾಡಲ್ಪಟ್ಟ ಕೆಲಸ ಖಂಡಿತವಾಗಿಯೂ ವಿವೇಕಾನಂದರ ಕೆಲಸವಲ್ಲ. ಅದು ಭಗವಂತನ ಕೆಲಸ - ಅವನ ಸ್ವಂತ ಕೆಲಸ! ಒಬ್ಬ ಗೌರ್ನರ್ ಜನರಲ್ ನಿವೃತ್ತನಾದರೆ ಚಕ್ರವರ್ತಿ ಮತ್ತೊಬ್ಬನನ್ನು ಆ ಸ್ಥಾನಕ್ಕೆ ಖಂಡಿತವಾಗಿಯೂ ಕಳುಹಿಸುವನು. ತಾಮಸದಿಂದ ನೀವು ಎಷ್ಟೇ ಆಚ್ಛಾದಿತರಾಗಿದ್ದರೂ ನೀವು ನಿರ್ವಂಚನೆಯಿಂದ ಅವನಲ್ಲಿ ಶರಣು ಹೊಂದಿದರೆ ಎಲ್ಲವೂ ತೊಳೆದು ಹೋಗುವುದು. ಭವರೋಗವೈದ್ಯನು ಬಂದಿರುವನು - ಇದು ಯುಕ್ತ ಕಾಲವಾಗಿದೆ. ನೀನು ಆತನ ಹೆಸರನ್ನು ತೆಗೆದುಕೊಂಡು ಕೆಲಸಕ್ಕೆ ತೊಡಗಿದರೆ ಅವನು ಎಲ್ಲವನ್ನೂ ನಿನ್ನ ಮೂಲಕ ಈಡೇರಿಸುವನು. ತಾಮಸವೇ ಉಚ್ಚತಮ ಸತ್ತ್ವವಾಗಿ ಮಾರ್ಪಡುವುದು.

ನಾನು: ನೀವೇನು ಬೇಕಾದರೂ ಹೇಳಿ, ನನಗೆ ಮಾತ್ರ ಈ ಮಾತುಗಳಲ್ಲಿ ನಂಬಿಕೆ ಬರುವುದಿಲ್ಲ. ನಿಮ್ಮಂತೆ ತತ್ತ್ವಶಾಸ್ತ್ರಗಳನ್ನು ವಿಶದವಾಗಿ ವಿವರಿಸುವಂತಹ ವಾಗ್ವೈಖರಿಯನ್ನು ಮತ್ತಾರು ಪಡೆಯಬಲ್ಲರು?

ಸ್ವಾಮೀಜಿ: ನಿನಗೆ ತಿಳಿಯದು. ಆ ಶಕ್ತಿ ಎಲ್ಲರಿಗೂ ಬರಬಹುದು. ಯಾರು ಅವಿಚ್ಛಿನ್ನವಾಗಿ ಹನ್ನೆರಡು ವರುಷಗಳ ಕಾಲ ಕೇವಲ ಭಗವತ್ಸಾಕ್ಷಾತ್ಕಾರ ಪಡೆಯುವ ಉದ್ದೇಶದಿಂದ ಬ್ರಹ್ಮಚರ್ಯವನ್ನು ಪಾಲಿಸುತ್ತಾರೋ ಅಂಥವರಿಗೆ ಈ ಶಕ್ತಿ ಬರುವುದು. ಈ ಬಗೆಯ ಬ್ರಹ್ಮಚರ್ಯವನ್ನು ನಾನೇ ಸಾಧನೆ ಮಾಡಿದ್ದೇನೆ. ಅಂತೆಯೇ ನನ್ನ ಮಿದುಳಿನಿಂದ ಒಂದು ತೆರೆಯನ್ನು ಎತ್ತಿದಂತಾಗಿದೆ. ಈ ಕಾರಣದಿಂದಲೇ ನಾನು ಯಾವ ಭಾಷಣಕ್ಕೇ ಆಗಲಿ ಮತ್ತು ಎಂತಹ ಗಹನವಾದ ಧರ್ಮತತ್ತ್ವಗಳ ವಿಚಾರಕ್ಕಾಗಲೀ ಅದಕ್ಕಾಗಿ ಏನೂ ಯೋಚಿಸಬೇಕಾಗಿಲ್ಲ, ಸಿದ್ಧತೆ ಮಾಡಿಕೊಳ್ಳಬೇಕಾಗಿಲ್ಲ. ನಾನು ನಾಳೆ ಒಂದು ಭಾಷಣ ಮಾಡಬೇಕಾಗಿದೆ ಎಂದಿಟ್ಟುಕೊಳ್ಳೋಣ. ನಾನು ನಾಳೆ ಏನು ಮಾತನಾಡಬೇಕೋ ಅದೆಲ್ಲಾ ಇಂದು ರಾತ್ರಿ ನನ್ನ ಕಣ್ಣ ಮುಂದೆ ಅನೇಕ ಚಿತ್ರಗಳಂತೆ ಸುಳಿದು ಹೋಗುವುವು. ಮರುದಿನ ನಾನು ಭಾಷಣ ಮಾಡುವಾಗ ನಾನು ನೋಡಿದ್ದ ವಿಷಯಗಳನ್ನೇ ನನ್ನ ಭಾಷಣದಲ್ಲಿ ಮಾತನಾಡುವೆ. ಆದ್ದರಿಂದ ಈ ಶಕ್ತಿ ಕೇವಲ ಪ್ರತ್ಯೇಕವಾಗಿ ನನ್ನದೇ ಎಂದು ನೀನು ಭಾವಿಸದಿರು. ಯಾರು ನಿರಾತಂಕವಾಗಿ ಹನ್ನೆರಡು ವರುಷಗಳ ಬ್ರಹ್ಮಚರ್ಯ ವ್ರತಾಚರಣೆ ಮಾಡುವರೊ ಅಂತಹವರಿಗೆ ಖಂಡಿತವಾಗಿಯೂ ಇದು ದೊರೆಯುವುದು. ನೀನೂ ಹಾಗೆ ಮಾಡಿದರೆ ನೀನೂ ಅದನ್ನು ಪಡೆಯುವಿ. ಕೇವಲ ಇಂಥವನು ಅದನ್ನು ಪಡೆಯುವನು, ಇತರರು ಪಡೆಯಲಾರರು ಎಂದು ನಮ್ಮ ಶಾಸ್ತ್ರಗಳೇನೂ ಹೇಳುವುದಿಲ್ಲ.

ನಾನು: ಸ್ವಾಮೀಜಿ, ನಿಮಗೆ ನೆನಪಿದೆಯೆ, ಒಂದು ದಿನ ನೀವು ಸಂನ್ಯಾಸಿಗಳಾಗುವ ಮುನ್ನ ಆ ಮನೆಯಲ್ಲಿ ನಾವೆಲ್ಲಾ ಕುಳಿತಿರುವಾಗ ನೀವು ನಮಗೆಲ್ಲಾ ಸಮಾಧಿಯ ರಹಸ್ಯವನ್ನು ವಿವರಿಸಲು ಯತ್ನಿಸಿದುದು? ನಾನು ಕಲಿಯುಗದಲ್ಲಿ ಸಮಾಧಿ ಅಸಾಧ್ಯವೆಂದು ನಿಮ್ಮ ಮಾತಿನ ಸತ್ಯಾಂಶವನ್ನು ಸಂದೇಹಿಸಿದಾಗ ನೀವು ತೀಕ್ಷ್ಣವಾಗಿ “ನೀನು ಸಮಾಧಿಯನ್ನು ನೋಡಲಿಚ್ಛಿಸುವೆಯೊ ಅಥವಾ ನೀನೇ ಆ ಸ್ಥಿತಿಯನ್ನು ಪಡೆಯಲಿಚ್ಛಿಸುವೆಯೊ? ನಾನು ಖಂಡಿತ ಸಮಾಧಿ ಹೊಂದಿರುವೆ, ನೀನೂ ಅದನ್ನು ಹೊಂದುವಂತೆ ಮಾಡಬಲ್ಲೆ” ಎಂದು ಹೇಳಿದ್ದಿರಿ. ನೀವು ಇನ್ನೂ ಆ ಮಾತನ್ನು ಪೂರ್ಣ ಮಾಡುವಷ್ಟರಲ್ಲೇ ಯಾವನೋ ಆಗಂತುಕನೊಬ್ಬನು ಬಂದುದರಿಂದ ನಾವು ಆ ವಿಷಯವನ್ನು ಮುಂದುವರಿಸಲಿಲ್ಲ.

ಸ್ವಾಮೀಜಿ: ಹೌದು ಆ ಸಂದರ್ಭ ನನಗೆ ನೆನಪಿದೆ.

ನಂತರ ನಾನು ಸಮಾಧಿ ಪಡೆಯುವಂತೆ ಮಾಡಬೇಕೆಂದು ಅವರನ್ನು ಒತ್ತಾಯಮಾಡಲು ಅವರು, “ನೋಡು, ಅನೇಕ ವರುಷಗಳು ನಿರಂತರ ಭಾಷಣ ಮಾಡಿ ಮಾಡಿ ತುಂಬಾ ಕಠಿಣವಾಗಿ ಕೆಲಸ ಮಾಡಿದುದರಿಂದ ನನ್ನಲ್ಲೀಗ ರಾಜಸಗುಣ ಪ್ರಬಲವಾಗಿದೆ. ಅದಕ್ಕೆ ಆ ಶಕ್ತಿ ಈಗ ನನ್ನಲ್ಲಿ ಸಪ್ತವಾಗಿ ಅಡಗಿದೆ. ನಾನೀಗ ಎಲ್ಲಾ ಕೆಲಸವನ್ನೂ ಬಿಟ್ಟು ಹಿಮಾಲಯಕ್ಕೆ ಹೋಗಿ ಏಕಾಂತವಾಗಿ ಧ್ಯಾನ ಮಾಡಿದರೆ ಆಗ ಆ ಶಕ್ತಿ ಪುನಃ ನನ್ನಲ್ಲಿ ವ್ಯಕ್ತವಾಗುತ್ತದೆ” ಎಂದರು.

\newpage

\chapter[ಅಧ್ಯಾಯ ೨]{ಅಧ್ಯಾಯ ೨\protect\footnote{\engfoot{C.W, Vol. V, P. 359}}}

ಇದಾಗಿ ಒಂದೆರಡು ದಿನಗಳ ನಂತರ ಸ್ವಾಮೀಜಿಯನ್ನು ನೋಡುವ ಸಲುವಾಗಿ ನಾನು ಮನೆಯಿಂದ ಹೊರಕ್ಕೆ ಬಂದಾಗ ನನ್ನ ಇಬ್ಬರು ಸ್ನೇಹಿತರು ಸಿಕ್ಕಿ ತಾವು ಸ್ವಾಮೀಜಿಯನ್ನು ಪ್ರಾಣಾಯಾಮದ ವಿಚಾರವಾಗಿ ಏನನ್ನೊ ಕೇಳಬೇಕೆಂದಿರುವೆವೆಂದೂ, ನನ್ನ ಜೊತೆಯಲ್ಲಿ ಬರಲು ಇಚ್ಛಿಸುವೆವೆಂದೂ ಹೇಳಿದರು. ದೇವಸ್ಥಾನಕ್ಕೇ ಆಗಲಿ, ಒಬ್ಬ ಸಾಧುವಿನ ಸಂದರ್ಶನಕ್ಕೇ ಆಗಲಿ ಯಾವಾಗಲೂ ಬರಿಗೈಯಲ್ಲಿ ಹೋಗಬಾರದೆಂದು ನಾನು ಕೇಳಿದ್ದುದರಿಂದ ನಾವು ಕೊಂಚ ಹಣ್ಣು ಮತ್ತು ಮಿಠಾಯಿಯನ್ನು ನಮ್ಮೊಡನೆ ತೆಗೆದುಕೊಂಡು ಹೋಗಿ ಅವರ ಮುಂದಿರಿಸಿದೆವು. ಸ್ವಾಮೀಜಿ ಅವನ್ನು ತಮ್ಮ ಕೈಯಲ್ಲಿ ತೆಗೆದುಕೊಂಡು ತಮ್ಮ ತಲೆಗೆ ಸೋಂಕಿಸಿ ನಾವು ಅವರಿಗೆ ಪ್ರಣಾಮಮಾಡುವ ಮೊದಲೇ ನಮಗೆ ತಲೆ ಬಾಗಿದರು. ನನ್ನ ಸ್ನೇಹಿತರಲ್ಲೊಬ್ಬರು ಸ್ವಾಮೀಜಿಯ ಸಹಪಾಠಿಗಳಾಗಿದ್ದರು. ಸ್ವಾಮೀಜಿ ತಕ್ಷಣವೇ ಅವರನ್ನು ಗುರುತಿಸಿ ಅವರ ಯೋಗಕ್ಷೇಮವನ್ನು ವಿಚಾರಿಸಿದರು. ನಂತರ ನಮ್ಮನ್ನು ತಮ್ಮ ಹತ್ತಿರ ಕುಳ್ಳಿರಿಸಿಕೊಂಡರು. ಇನ್ನೂ ಅನೇಕರು ಅವರನ್ನು ನೋಡಲು ಮತ್ತು ಮಾತನಾಡಲು ಅಲ್ಲಿಗೆ ಬಂದಿದ್ದರು. ಆ ಸಭ್ಯ ಗೃಹಸ್ಥರಲ್ಲಿ ಕೆಲವರು ಕೇಳಿದ ಪ್ರಶ್ನೆಗಳಿಗೆ ಉತ್ತರವಾಗಿ ಸ್ವಾಮೀಜಿ ಸಂಭಾಷಣೆಯ ನಡುವೆ ಪ್ರಾಣಾಯಾಮದ ವಿಚಾರವಾಗಿ ಮಾತನಾಡತೊಡಗಿದರು. ಮೊಟ್ಟಮೊದಲು ಅವರು ಆಧುನಿಕ ವಿಜ್ಞಾನದ ಮೂಲಕ ವಸ್ತುವಿನ ಮೂಲವನ್ನು ವಿವರಿಸಿದರು. ನಂತರ ಪ್ರಾಣಾಯಾಮ ಏನೆಂಬುದನ್ನು ತೋರಿಸಲು ಮುಂದುವರಿದರು. ನಾವು ಮೂವರೂ ಅವರ ರಾಜಯೋಗವೆಂಬ ಪುಸ್ತಕವನ್ನು ಮುಂಚೆಯೇ ಎಚ್ಚರಿಕೆಯಿಂದ ಓದಿದ್ದೆವು. ಆದರೆ ಅಂದು ನಾವು ಅವರ ಬಾಯಿಂದ ಪ್ರಾಣಾಯಾಮದ ವಿಚಾರವಾಗಿ ಕೇಳಿದುದರಲ್ಲಿ ಅವರಲ್ಲಿರುವ ಜ್ಞಾನದ ಎಲ್ಲೋ ಅತಿ ಸ್ವಲ್ಪ ಭಾಗ ಮಾತ್ರ ಆ ಪುಸ್ತಕದಲ್ಲಿ ನಿರೂಪಿಸಲ್ಪಟ್ಟಿದೆ ಎಂದೆನ್ನಿಸಿತು. ಅವರು ಹೇಳಿದುದೆಲ್ಲಾ ಕೇವಲ ಬರಿಯ ಪುಸ್ತಕ ಪಾಂಡಿತ್ಯವಲ್ಲ. ಏಕೆಂದರೆ ಯಾರಿಗೆ ತಾನೇ ಹಾಗೆ ತಾವೇ ಸತ್ಯ ಸಾಕ್ಷಾತ್ಕಾರವನ್ನು ಪಡೆಯದೆ ವಿಜ್ಞಾನದ ಸಹಾಯವನ್ನೂ ತೆಗೆದುಕೊಂಡು ಧರ್ಮದ ಎಲ್ಲಾ ಜಟಿಲ ಸಮಸ್ಯೆಗಳನ್ನೂ ಅಷ್ಟೊಂದು ಸ್ಪಷ್ಟವಾಗಿ ವಿಸ್ತಾರವಾಗಿ ವಿವರಿಸಲು ಸಾಧ್ಯ?

ಅವರ ಪ್ರಾಣಾಯಾಮದ ಮೇಲಿನ ಸಂಭಾಷಣೆ ಮಧ್ಯಾಹ್ನ ಮೂರು ಘಂಟೆಯಿಂದ ಸಂಜೆ ಏಳುವರೆ ಘಂಟೆಯವರೆಗೂ ನಡೆಯಿತು. ಆ ಭೇಟಿ ಮುಕ್ತಾಯವಾಗಿ ನಾವು ಹೊರಟು ಬಂದಮೇಲೆ ನನ್ನ ಸ್ನೇಹಿತರು ಸ್ವಾಮೀಜಿಯವರು ನಮ್ಮ ಮನಸ್ಸಿನಲ್ಲಿದ್ದ ಪ್ರಶ್ನೆಗಳನ್ನೆಲ್ಲಾ ಹೇಗೆ ತಿಳಿದುಕೊಂಡರೆಂದು ಆಶ್ಚರ್ಯಪಟ್ಟು, ನಾನೇನಾದರೂ ಅವರಿಗೆ ಆ ಪ್ರಶ್ನೆಗಳನ್ನೆಲ್ಲಾ ತಿಳಿಸಿಬಿಟ್ಟಿದ್ದೇನೊ ಎಂದು ನನ್ನನ್ನು ಕೇಳಿದರು.

ಈ ಪ್ರಸಂಗ ನಡೆದ ಕೆಲವು ದಿನಗಳ ನಂತರ ನಾನು ಸ್ವಾಮೀಜಿಯನ್ನು ಬಾಗಬಜಾರಿನಲ್ಲಿರುವ ದಿವಂಗತ ಪ್ರಿಯನಾಥ ಮುಖರ್ಜಿಗಳ ಮನೆಯಲ್ಲಿ ನೋಡಿದೆ. ಅಲ್ಲಿ ಸ್ವಾಮಿ ಬ್ರಹ್ಮಾನಂದರು, ಸ್ವಾಮಿ ಯೋಗಾನಂದರು, ಶ‍್ರೀಮಾನ್ ಜಿ.ಸಿ. ಘೋಷರು, ಅತುಲಬಾಬು ಮತ್ತು ಇನ್ನೂ ಒಬ್ಬಿಬ್ಬರು ಸ್ನೇಹಿತರೂ ಇದ್ದರು. ನಾನು ಅವರಿಗೆ ಹೇಳಿದೆ: “ಸ್ವಾಮೀಜಿ, ಅಂದು ನಿಮ್ಮನ್ನು ನೋಡಲು ಬಂದಿದ್ದ ಇಬ್ಬರು ಸಭ್ಯ ಗೃಹಸ್ಥರು ಪ್ರಾಣಾಯಾಮದ ವಿಚಾರವಾಗಿ ನಿಮ್ಮನ್ನು ಕೆಲವು ಪ್ರಶ್ನೆಗಳನ್ನು ಕೇಳಬೇಕೆಂದಿದ್ದರು. ನೀವು ಭರತಖಂಡಕ್ಕೆ ಹಿಂತಿರುಗುವುದಕ್ಕೆ ಕೊಂಚ ದಿನಗಳ ಮೊದಲೇ ನಿಮ್ಮ ರಾಜಯೋಗದ ಪುಸ್ತಕಗಳನ್ನು ಓದಿದಾಗ ಅವರ ಮನಸ್ಸಿನಲ್ಲಿದ್ದ ಕೆಲವು ಪ್ರಶ್ನೆಗಳನ್ನು ನನ್ನಲ್ಲಿ ಆಗ ಹೇಳಿದ್ದರು. ಆದರೆ ಅಂದು ಅವರು ನಿಮ್ಮನ್ನು ಏನನ್ನಾದರೂ ಕೇಳುವ ಮೊದಲೇ ಅವರಿಗೆ ಉಂಟಾಗಿದ್ದ ಅನುಮಾನಗಳನ್ನೇ ನೀವು ಎತ್ತಿ ಅವುಗಳನ್ನು ಪರಿಹರಿಸಿದಿರಿ! ಅವರಿಗೆ ತುಂಬಾ ಆಶ್ಚರ್ಯವಾಗಿ ಮೊದಲೇ ನಾನೇನಾದರೂ ನಿಮಗೆ ಅವರ ಸಂದೇಹಗಳನ್ನು ತಿಳಿಸಿರಬಹುದೇ ಎಂದು ವಿಚಾರಿಸಿದರು.” ಸ್ವಾಮೀಜಿ ಉತ್ತರಿಸಿದರು: “ಪಾಶ್ಚಾತ್ಯ ದೇಶಗಳಲ್ಲೂ ಇಂತಹ ಪ್ರಸಂಗಗಳು ಅನೇಕ ವೇಳೆ ನಡೆದಾಗ ಜನರು ನನ್ನನ್ನು ಪದೇಪದೇ ‘ನಿಮಗೆ ಹೇಗೆ ನಮ್ಮ ಮನಸ್ಸನ್ನು ಕಲಕುತ್ತಿದ್ದ ಪ್ರಶ್ನೆಗಳು ಗೊತ್ತಾಗುತ್ತವೆ?’ ಎಂದು ಕೇಳುತ್ತಿದ್ದರು. ಈ ಬಗೆಯ ಜ್ಞಾನ ನನಗೆ ಪದೇಪದೇ ಉಂಟಾಗುವುದಿಲ್ಲ. ಆದರೆ ಶ‍್ರೀರಾಮಕೃಷ್ಣರಿಗೆ ಮಾತ್ರ ಇದು ಯಾವಾಗಲೂ ಗೊತ್ತಾಗುತ್ತಿತ್ತು.”

ಈ ಸಂದರ್ಭದಲ್ಲಿ ಅತುಲಬಾಬುಗಳು ಸ್ವಾಮೀಜಿಯನ್ನು “ನೀವು ರಾಜಯೋಗದಲ್ಲಿ ನಮ್ಮ ಪೂರ್ವಜನ್ಮದ ವಿಷಯವನ್ನೆಲ್ಲಾ ತಿಳಿದುಕೊಳ್ಳಬಹುದೆಂದು ಹೇಳುವಿರಿ. ನಿಮಗೆ ಅವು ಗೊತ್ತಿದೆಯೆ?” ಎಂದು ಕೇಳಿದರು.

ಸ್ವಾಮೀಜಿ: ಹೌದು ಗೊತ್ತಿದೆ.

ಅತುಲಬಾಬು: ನಿಮಗೇನು ಗೊತ್ತಿದೆ? ಅದನ್ನು ಹೇಳಲು ಅಭ್ಯಂತರ ಉಂಟೆ?

ಸ್ವಾಮಾಜಿ: ನಾನು ಅದನ್ನು ತಿಳಿಯಬಲ್ಲೆ - ನನಗೆ ಅವು ತಿಳಿದಿವೆ - ಆದರೆ ಯಾವುದನ್ನೂ ನಾನು ವಿವರವಾಗಿ ಹೇಳಲು ಇಚ್ಛೆಪಡುವುದಿಲ್ಲ.

\newpage

\chapter[ಅಧ್ಯಾಯ ೩]{ಅಧ್ಯಾಯ ೩\protect\footnote{\engfoot{C.W, Vol. V, P. 361}}}

೧೮೯೮ ಜುಲೈ ತಿಂಗಳು. ಬೇಲೂರಿನ ನೀಲಾಂಬರ ಮುಖರ್ಜಿಗಳ ತೋಟದ ಮನೆಯಲ್ಲಿದ್ದ ಮಠದಲ್ಲಿ ಒಂದು ಸಂಜೆ ಸ್ವಾಮೀಜಿ ತಮ್ಮೆಲ್ಲ ಶಿಷ್ಯರೊಡನೆ ಧ್ಯಾನ ಮಾಡುತ್ತಿದ್ದರು. ಧ್ಯಾನ ಪೂರೈಸಿದ ಮೇಲೆ ಈಚೆಗೆ ಬಂದು ಕೊಠಡಿಯೊಂದರಲ್ಲಿ ಕುಳಿತುಕೊಂಡರು. ಅಂದು ತುಂಬಾ ಜೋರುಮಳೆ, ಶೀತಲವಾದ ಗಾಳಿ ಬೀಸುತ್ತಿದ್ದುದರಿಂದ ಸ್ವಾಮೀಜಿ ಕದವನ್ನು ಮುಚ್ಚಿ ತಂಬೂರಿ ಮೀಟುತ್ತ ಹಾಡಲುಪಕ್ರಮಿಸಿದರು. ಹಾಡು ಮುಗಿದ ಮೇಲೆ ದೀರ್ಘವಾಗಿ ಸಂಭಾಷಣೆ ಜರುಗಿತು. ಸ್ವಾಮಿ ಶಿವಾನಂದರು ಅವರನ್ನು ‘ಪಾಶ್ಚಾತ್ಯ ಸಂಗೀತ ಹೇಗಿರುತ್ತದೆ’ ಎಂದು ಕೇಳಿದರು.

ಸ್ವಾಮಾಜಿ: ಓ! ಅದು ತುಂಬಾ ಚೆನ್ನಾಗಿರುತ್ತದೆ. ನಮ್ಮಲ್ಲಿ ಇಲ್ಲದ ಸಾಮರಸ್ಯದ ಪರಾಕಾಷ್ಠೆ ಅದರಲ್ಲಿದೆ. ನುರಿತಿಲ್ಲದ ನಮ್ಮ ಕಿವಿಗೆ ಅದು ಇಂಪಾಗಿ ಕೇಳುವುದಿಲ್ಲ. ಅದಕ್ಕೆ ನಾವು ಅದನ್ನು ಇಷ್ಟಪಡುವುದಿಲ್ಲ. ಆ ಸಂಗೀತಗಾರರು ನರಿಗಳಂತೆ ಕೂಗುವರೆಂದೆನ್ನಿಸುವುದು. ನನಗೂ ಮೊದಲು ಇದೇ ಬಗೆಯ ಅಭಿಪ್ರಾಯವಿತ್ತು. ಆದರೆ ಯಾವಾಗ ನಾನು ಗಮನವಿಟ್ಟು ಆ ಸಂಗೀತವನ್ನು ಕೇಳತೊಡಗಿದೆನೊ ಮತ್ತು ಸೂಕ್ಷ್ಮವಾಗಿ ಅಭ್ಯಾಸ ಮಾಡ ತೊಡಗಿದೆನೊ, ಆಗ ನಾನು ಅದನ್ನು ಹೆಚ್ಚು ಹೆಚ್ಚು ಅರ್ಥಮಾಡಿಕೊಳ್ಳತೊಡಗಿದೆ. ಸಂಪೂರ್ಣವಾಗಿ ಅನುರಕ್ತನಾದೆ. ಹೀಗೆ ಎಲ್ಲಾ ಕಲೆಗಳೂ ಕೂಡ. ಮಹಾಪರಿಷ್ಕೃತವಾದ ತೈಲಚಿತ್ರವನ್ನು ಪ್ರಾಸಂಗಿಕವಾಗಿ ನೋಡಿದಾಗ ನಮಗೆ ಅದರ ಸೌಂದರ್ಯ ಎಲ್ಲಿದೆಯೆಂದು ಗ್ರಹಿಸಲಾಗುವುದಿಲ್ಲ. ಅದೂ ಅಲ್ಲದೆ ಒಂದು ನಿರ್ದಿಷ್ಟ ಗುರಿಯವರೆಗೆ ಕಣ್ಣು ಪಳಗಿದ ಹೊರತು ಒಂದು ಕಲಾಕೃತಿಯ ಆಳವಾದ ಪ್ರತಿಭೆಯನ್ನು ಅದರ ಸೂಕ್ಷ್ಮತಮ ರಚನೆ ಮತ್ತು ಸಂಯೋಜನೆಗಳನ್ನು ಮೆಚ್ಚಲಾಗುವುದಿಲ್ಲ. ನಮ್ಮ ನಿಜವಾದ ಸಂಗೀತವು ಕೀರ್ತನೆ ಮತ್ತು ಧ್ರುಪದ್‌ಗಳಲ್ಲಿದೆ. ಉಳಿದುವು ಮುಸಲ್ಮಾನ ಪದ್ಧತಿ ಪ್ರಕಾರ ಮಾರ್ಪಟ್ಟು ಹಾಳಾಗಿಹೋಗಿವೆ. ಚಿಕ್ಕದಾದ ಮತ್ತು ಲಘುವಾದ ಥಪ್ಪಾ ಹಾಡುಗಳನ್ನು ಮೂಗಿನಲ್ಲಿ ಹಾಡಿ ಹುಚ್ಚುಹುಚ್ಚಾಗಿ ಒಂದು ಸ್ವರದಿಂದ ಮತ್ತೊಂದು ಸ್ವರಕ್ಕೆ ಮಿಂಚಿನಂತೆ ತಾರಾಡಿದ ಮಾತ್ರಕ್ಕೆ ಅದು ಸಂಗೀತ ಪ್ರಪಂಚದಲ್ಲಿ ಮಹತ್ವಪೂರಿತವಾದುದೆಂದು ತಿಳಿದಿರುವಿಯೇನು? ಖಂಡಿತ ಅಲ್ಲ. ಪ್ರತಿಯೊಂದು ರಾಗವನ್ನೂ ಎಲ್ಲಾ ಸ್ವರದಲ್ಲಿ ಸಂಪೂರ್ಣ ಹೊರಡಿಸಿದ ಹೊರತು ಸಂಗೀತಶಾಸ್ತ್ರವೆಲ್ಲಾ ಹಾಳಾಗಿ ಹೋಗುವುದು. ಚಿತ್ರಕಲೆಯಲ್ಲಿ ಪ್ರಕೃತಿಯ ಸಂಪರ್ಕವನ್ನಿಟ್ಟುಕೊಂಡು ನೀನು ಅದನ್ನು ಎಷ್ಟೊಂದು ಕಲಾಸೌಂದರ್ಯಪೂರಿತವಾಗಿ ಬೇಕಾದರೂ ಮಾಡಬಹುದು. ಹೀಗೆ ಮಾಡುವುದರಿಂದ ಯಾವ ಹಾನಿಯೂ ಇಲ್ಲ. ಪರಿಣಾಮ ಕೂಡ ಒಳ್ಳೆಯದೆ ಆಗುವುದು. ಅದರಂತೆಯೆ ಸಂಗೀತ ಶಾಸ್ತ್ರವನ್ನನುಸರಿಸಿ, ನಿನ್ನ ಕೌಶಲವನ್ನು ಎಷ್ಟು ಬೇಕಾದರೂ ತೋರಿಸಬಹುದು. ಕಿವಿಗೂ ಅದು ಹಿತಕರವಾಗಿರುವುದು. ಮಹಮ್ಮದೀಯರು ಅನೇಕ ಬಗೆಯ ರಾಗ ರಾಗಿಣಿಗಳನ್ನು ಇಂಡಿಯಾಕ್ಕೆ ಬಂದಮೇಲೆ ತೆಗೆದುಕೊಂಡರು. ಆದರೆ ಅವರು ‘ಥಪ್ಪ’ ಸಂಗೀತದ ಮೇಲೆ ತಮ್ಮ ವೈಶಿಷ್ಟ್ಯದ ಮುದ್ರೆಯನ್ನೊತ್ತಿದರು. ಅದಕ್ಕೇ ಸಂಗೀತಶಾಸ್ತ್ರವೆಲ್ಲ ಹಾಳಾಗಿ ಹೋಯಿತು.

ಪ್ರಶ್ನೆ: ಏಕೆ ಮಹಾರಾಜ್, ಯಾರಿಗೆ ತಾನೆ ಥಪ್ಪ ಸಂಗೀತದ ಮೇಲೆ ಅಭಿಲಾಷೆಯಿಲ್ಲ?

ಸ್ವಾಮೀಜಿ: ಜೀರುಂಡೆಯ ಜೀರೆಂಬುವ ಶಬ್ದ ಕೂಡ ಕೆಲವರಿಗೆ ತುಂಬಾ ಇಷ್ಟ. ಸಂತಾಲರಿಗೆ ಅವರ ಸಂಗೀತವೇ ಎಲ್ಲಾ ಸಂಗೀತಕ್ಕಿಂತಲೂ ಉತ್ತಮವೆನ್ನಿಸುವುದು. ಒಂದಾದ ಮೇಲೊಂದು ಸ್ವರ ಅಷ್ಟು ಬೇಗನೆ ಬರುತ್ತಿದ್ದರೆ ಸಂಗೀತದ ಮಾಧುರ್ಯವನ್ನೇ ನಾಶಮಾಡುವುದಲ್ಲದೆ ಅಪಸ್ವರವನ್ನು ಕೂಡ ಉಂಟುಮಾಡುವುದೆಂಬುದು ನಿಮಗೆ ಅರ್ಥವಾಗುವುದಿಲ್ಲ ಎಂದು ಕಾಣುತ್ತದೆ. ಸಂಗೀತದಲ್ಲಿ ಏಳು ಸ್ವರಗಳ ಪರಿವರ್ತನೆ ಮತ್ತು ಸಂಯೋಗಗಳು ಯಾವುದಾದರೊಂದು ಬಗೆಯ ಅಥವಾ ಬಗೆಬಗೆಯ ರಾಗ ರಾಗಿಣಿಗಳನ್ನು ಉಂಟುಮಾಡುತ್ತದೆಂದು ನಿಮಗೆ ಗೊತ್ತಿಲ್ಲವೆ? ಈಗ ಥಪ್ಪದಲ್ಲಿ, ಒಂದು ಸಂಪೂರ್ಣ ರಾಗವನ್ನು ತೇಲಿಸಿ ಹೇಳಿ, ಹೊಸರಾಗವನ್ನು ಉತ್ಪತ್ತಿ ಮಾಡಿದಾಗ, ಎಲ್ಲಕ್ಕಿಂತ ಹೆಚ್ಚಾಗಿ ಧ್ವನಿಯನ್ನು ನಡುಗಿಸಿ ಸ್ವರದಿಂದ ಸ್ವರಕ್ಕೆ ಬದಲಾಯಿಸುತ್ತಾ ತೀವ್ರಮಟ್ಟಕ್ಕೆ ಎತ್ತಿದಾಗ, ರಾಗವು ಹೇಗೆ ತಾನೇ ಅಖಂಡವಾಗಿರುವುದು? ಜೊತೆಗೆ ಕೇವಲ ಪರಿಣಾಮವನ್ನುಂಟುಮಾಡಲು ಧಾರಾಳವಾಗಿ ತೇಲಿಸಿ ಮತ್ತು ಲಘುವಾಗಿ ಎಳೆಯುವುದರಿಂದ ಸಂಗೀತದ ಶೈಲಿಯೇ ಸಂಪೂರ್ಣವಾಗಿ ಹಾಳಾಗುವುದು. ಹಾಡನ್ನು ಭಾವಪೂರ್ವಕವಾಗಿ ಅದರ ಅರ್ಥ ತಿಳಿಯುವಂತೆ ಹಾಡುವುದು ಈ ಥಪ್ಪವು ರೂಢಿಗೆ ಬಂದಾಗಿನಿಂದ ಸಂಪೂರ್ಣವಾಗಿ ಕಣ್ಮರೆಯಾಗಿದೆ. ಈಗಿನ ಕಾಲದಲ್ಲಿ ನಾಟಕ ಮಂದಿರಗಳನ್ನು ಉತ್ತಮಗೊಳಿಸಿರುವುದರಿಂದ ಕೊಂಚಮಟ್ಟಿಗೆ ನಿಜವಾದ ಕಲೆ ಊರ್ಜಿತಗೊಳ್ಳುತ್ತಿದೆ. ಆದರೆ ಮತ್ತೊಂದು ಕಡೆ ಎಲ್ಲಾ ರಾಗ ಮತ್ತು ರಾಗಿಣಿಗಳ ಮೇಲಿನ ಗೌರವವನ್ನೂ ಹೆಚ್ಚು ಹೆಚ್ಚಾಗಿ ಗಾಳಿಗೆ ತೂರಿಬಿಟ್ಟಂತಾಗಿದೆ.

“ಅದರಂತೆಯೇ ಧ್ರುಪದ್ ಹಾಡುಗಾರಿಕೆಯಲ್ಲಿ ಪ್ರವೀಣರಾದವರಿಗೆ ಈ ಥಪ್ಪವನ್ನು ಕೇಳುವುದಕ್ಕೆ ತುಂಬಾ ಕಷ್ಟವಾಗುತ್ತದೆ. ಆದರೆ ನಮ್ಮ ಸಂಗೀತದಲ್ಲಿ ಆರೋಹಣ ಅವರೋಹಣ ಚೆನ್ನಾಗಿರುವುದು. ಫ್ರೆಂಚರು ಮೊದಲು ಈ ವೈಶಿಷ್ಟ್ಯವನ್ನು ಕಂಡುಹಿಡಿದು ಶ್ಲಾಘಿಸಿದರು. ಅದನ್ನು ತಮ್ಮ ಸಂಗೀತದಲ್ಲೂ ಅನುಸರಿಸಲು ಆಚರಣೆಗೆ ತರಲು ಪ್ರಯತ್ನಿಸಿದರು. ಅವರು ಈ ರೀತಿ ಮಾಡಿದ ಮೇಲೆ ಇಡೀ ಯೂರೋಪು ಅದರಲ್ಲಿ ಸಂಪೂರ್ಣ ಪ್ರಾವೀಣ್ಯ ಪಡೆದಿದೆ.”

ಪ್ರಶ್ನೆ: ಮಹಾರಾಜ್, ಅವರ ಸಂಗೀತ ಎಲ್ಲಕ್ಕಿಂತ ಹೆಚ್ಚಾಗಿ ಯುದ್ಧೋಚಿತವಾದುದಾಗಿದೆ. ಈ ಗುಣ ನಮ್ಮಲ್ಲಿ ಸಂಪೂರ್ಣವಾಗಿ ಲಯವಾಗಿದೆ.

ಸ್ವಾಮೀಜಿ: ಇಲ್ಲ, ಇಲ್ಲ, ನಮ್ಮಲ್ಲೂ ಇದು ಇದೆ. ವೀರರಸ ಪ್ರಧಾನ ಸಂಗೀತದಲ್ಲಿ ಸ್ವರಮೈತ್ರಿಯು ತುಂಬಾ ಆವಶ್ಯಕ. ನಮ್ಮಲ್ಲಿ ಸ್ವರಮೈತ್ರಿ ಇಲ್ಲದಿರುವುದು ಶೋಚನೀಯ. ಅದಕ್ಕೇ ಅದು ಹೆಚ್ಚಾಗಿ ಕಾಣುವುದಿಲ್ಲ. ನಮ್ಮ ಸಂಗೀತವು ಹಂತಹಂತವಾಗಿ ಅಭಿವೃದ್ಧಿ ಹೊಂದುತ್ತಿತ್ತು. ಆದರೆ ಮಹಮ್ಮದೀಯರು ಬಂದಾಗ ಅವರು ಅದನ್ನು ಎಷ್ಟರಮಟ್ಟಿಗೆ ತಮ್ಮ ವಶಪಡಿಸಿಕೊಂಡರೆಂದರೆ ಸಂಗೀತವೃಕ್ಷ ಇನ್ನು ಮುಂದಕ್ಕೆ ಬೆಳೆಯಲಾಗದೆ ಹೋಯಿತು. ಪಾಶ್ಚಾತ್ಯರ ಸಂಗೀತದಲ್ಲಿರಬೇಕಾದ ದುಃಖಭಾವ ಮತ್ತು ವೀರಭಾವ ಎರಡೂ ಅಲ್ಲಿದೆ. ಆದರೆ ನಮ್ಮ ಪೂರ್ವದ ಸೋರೆಬುರುಡೆಯ ಸಂಗೀತದ ಉಪಕರಣ ಕೊಂಚವೂ ಉತ್ತಮವಾಗಿಲ್ಲ.

ಪ್ರಶ್ನೆ: ರಾಗ ಮತ್ತು ರಾಗಿಣಿಗಳಲ್ಲಿ ಯಾವುದು ವೀರರಸಪ್ರಧಾನವಾದದ್ದು?

ಸ್ವಾಮೀಜಿ: ಪ್ರತಿಯೊಂದು ರಾಗದಲ್ಲೂ ಸರಿಯಾದ ಸ್ವರ ಕೂಡಿಸಿದರೆ ಮತ್ತು ವಾದ್ಯಗಳನ್ನು ತಕ್ಕಂತೆ ಶ್ರುತಿ ಮಾಡಿದರೆ ವೀರರಸವನ್ನು ತರಬಹುದು. ಕೆಲವು ರಾಗಿಣಿಗಳಲ್ಲಿ ಕೂಡ ವೀರಭಾವ ತರಬಹುದು.

ಊಟದ ವೇಳೆಯಾದ್ದರಿಂದ ಸಂಭಾಷಣೆ ಮುಕ್ತಾಯವಾಯಿತು. ಊಟವಾದ ಮೇಲೆ ಸ್ವಾಮೀಜಿ ಕಲ್ಕತ್ತೆಯಿಂದ ಮಠಕ್ಕೆ ರಾತ್ರಿ ತಂಗಲು ಬಂದಿದ್ದ ಅತಿಥಿಗಳಿಗೆ ಸರಿಯಾದ ವ್ಯವಸ್ಥೆ ಆಗಿದೆಯೇ ಎಂದು ವಿಚಾರಿಸಿ ನಂತರ ಮಲಗುವ ಕೋಣೆಗೆ ಹೋದರು.

\newpage

\chapter[ಅಧ್ಯಾಯ ೪]{ಅಧ್ಯಾಯ ೪\protect\footnote{\engfoot{C.W, Vol. V, P. 364}}}

ಹೊಸಮಠವನ್ನು ಕಟ್ಟಿ ಆಗಲೇ ಎರಡು ವರುಷಗಳು ಆಗಿದ್ದವು. ಎಲ್ಲಾ ಸ್ವಾಮಿಗಳ ಮಠದಲ್ಲೇ ವಾಸಿಸುತ್ತಿದ್ದರು. ಆಗ ಒಂದು ದಿನ ಬೆಳಿಗ್ಗೆ ನಾನು ನನ್ನ ಗುರುವನ್ನು ಸಂದರ್ಶಿಸಲು ಅಲ್ಲಿಗೆ ಹೋದೆ. ನನ್ನನ್ನು ನೋಡಿ ಸ್ವಾಮೀಜಿ ಮುಗುಳುನಗೆ ನಕ್ಕು ನನ್ನ ಯೋಗಕ್ಷೇಮವನ್ನು ವಿಚಾರಿಸಿದಮೇಲೆ “ಇಂದು ನೀನು ಇಲ್ಲೇ ಇರುವೆ ಅಲ್ಲವೆ?” ಎಂದು ಕೇಳಿದರು. ನಾನು “ನಿಸ್ಸಂದೇಹವಾಗಿ” ಎಂದೆ. ಇನ್ನು ಏನೇನೋ ವಿಚಾರಗಳನ್ನು ಕೇಳಿದ ಬಳಿಕ ನಾನು ಸ್ವಾಮೀಜಿಯನ್ನು “ನಮ್ಮ ಹುಡುಗರ ವಿದ್ಯಾಭ್ಯಾಸದ ವಿಚಾರವಾಗಿ ನಿಮ್ಮ ಅಭಿಪ್ರಾಯವೇನು?” ಎಂದು ಕೇಳಿದೆ.

ಸ್ವಾಮೀಜಿ: ಗುರುಕುಲವಾಸ, ಗುರುವಿನೊಡನೆ ವಾಸಿಸುವುದು.

ಪ್ರಶ್ನೆ: ಹೇಗೆ?

ಸ್ವಾಮೀಜಿ: ಹಿಂದಿನಂತೆಯೇ ಇರಬೇಕು. ಈ ವಿದ್ಯಾಭ್ಯಾಸದ ಜೊತೆಗೆ ಪಾಶ್ಚಾತ್ಯ ವಿಜ್ಞಾನ ಶಾಸ್ತ್ರವನ್ನೂ ಕಲಿಸಬೇಕು. ಇವೆರಡೂ ಆವಶ್ಯಕ.

ಪ್ರಶ್ನೆ: ಏಕೆ? ಈಗಿನ ವಿಶ್ವವಿದ್ಯಾನಿಲಯದ ಪದ್ಧತಿಯಲ್ಲಿ ಯಾವ ನ್ಯೂನತೆ ಇದೆ?

ಸ್ವಾಮೀಜಿ: ಅದೆಲ್ಲ ಲೋಪದೋಷಗಳ ಕಂತೆ. ಅದು ಗುಮಾಸ್ತರನ್ನು ಉತ್ಪತ್ತಿಮಾಡುವ ಯಂತ್ರವಲ್ಲದೆ ಮತ್ತೇನೂ ಅಲ್ಲ. ಇಷ್ಟೇ ಆಗಿದ್ದರೆ ಗ್ರಹಚಾರಕ್ಕೆ, ನಾನೆಷ್ಟೋ ಕೃತಜ್ಞನಾಗಿರುತ್ತಿದ್ದೆ. ಇಷ್ಟೇ ಅಲ್ಲ, ಜನರು ಹೇಗೆ ನಂಬಿಕೆಯಿಲ್ಲದೆ ಶ್ರದ್ಧಾಹೀನರಾಗುತ್ತಿದ್ದಾರೆಂಬುದನ್ನು ನೋಡು. ಅವರು ಗೀತೆಯು ಕೇವಲ ಒಂದು ಪ್ರಕ್ಷಿಪ್ತ ಭಾಗವೆಂದೂ, ವೇದಗಳೆಲ್ಲ ಗ್ರಾಮ್ಯ ಹಾಡುಗಳೆಂದೂ ಕಂಠೋಕ್ತವಾಗಿ ಹೇಳುತ್ತಿದ್ದಾರೆ. ಭರತಖಂಡದ ಹೊರಗಡೆ ಇರುವ ವಸ್ತುಗಳ, ರಾಷ್ಟ್ರಗಳ ವಿಚಾರಗಳಲ್ಲೆಲ್ಲಾ ಪಾರಂಗತರಾಗಲು ಇಚ್ಛಿಸುವರು. ಆದರೆ ಅವರ ಪೂರ್ವಿಕರ ಏಳು ಸಂತತಿಗಳ ಹೆಸರನ್ನು ಕೇಳು. ಅದೂ ಕೂಡ ಅವರಿಗೆ ಗೊತ್ತಿರುವುದಿಲ್ಲ. ಇನ್ನು ಹದಿನಾಲ್ಕು ಸಂತತಿ ಪಾಡೇನು!

ಪ್ರಶ್ನೆ: ಆದರೆ ಅದರಿಂದೇನಾಗುವುದು? ಅವರ ಪೂರ್ವಿಕರ ವಂಶಾವಳಿಯ ಹೆಸರು ಅವರಿಗೆ ಗೊತ್ತಿಲ್ಲದಿದ್ದರೇನಂತೆ?

ಸ್ವಾಮೀಜಿ: ಹಾಗೆ ಯೋಚಿಸಬೇಡ. ಯಾವ ಜನಾಂಗಕ್ಕೆ ಅದರದೇ ಆದ ಇತಿಹಾಸವಿಲ್ಲವೋ ಅದಕ್ಕೆ ಈ ಜಗತ್ತಿನಲ್ಲಿ ಯಾವ ಸ್ಥಾನವೂ ಇಲ್ಲ. ‘ನಾನು ಕುಲೀನ ಮನೆತನದವನು’ ಎಂದು ಭಾವಿಸುವಷ್ಟು ಹೆಮ್ಮೆ ಮತ್ತು ನಂಬಿಕೆ ಹೊಂದಿರುವವನು ಎಂದಾದರೂ ಕೆಟ್ಟವನಾಗಬಲ್ಲನೆಂದು ನಂಬುವೆಯೇನು? ಅದು ಹೇಗೆ ಸಾಧ್ಯ? ಅವನಲ್ಲಿರುವ ಶ್ರದ್ಧೆಯೇ ಅವನ ಕೆಲಸಗಳನ್ನು ಮತ್ತು ಯೋಚನೆಗಳನ್ನೆಲ್ಲಾ ಎಷ್ಟರಮಟ್ಟಿಗೆ ನಿಯಂತ್ರಿಸುತ್ತವೆಂದರೆ ಅವನು ಕೆಟ್ಟ ಕೆಲಸವನ್ನು ಮಾಡುವುದಕ್ಕೆ ಬದಲು ಪ್ರಾಣವನ್ನಾದರೂ ಬಿಡುವನು. ಆದ್ದರಿಂದ ರಾಷ್ಟ್ರದ ಇತಿಹಾಸ ಆ ರಾಷ್ಟ್ರವನ್ನು ಸರಿಯಾಗಿ ನಿಗ್ರಹದಲ್ಲಿಟ್ಟು ಅದು ಹೀನಗತಿಗೆ ಬರದಂತೆ ತಡೆಯುವುದು. ನನಗೆ ಗೊತ್ತು, ನೀನು ‘ಆದರೆ ನಮಗೆ ಅಂತಹ ಇತಿಹಾಸವಿಲ್ಲವಲ್ಲ!’ ಎಂದು ಹೇಳಬಹುದು. ಇಲ್ಲ, ಯಾರು ನಿನ್ನಂತೆ ಯೋಚಿಸುವರೋ ಅಂತಹವರಿಗೆ ಖಂಡಿತವಾಗಿಯೂ ಇಲ್ಲ. ನಿಮ್ಮ ವಿಶ್ವವಿದ್ಯಾನಿಲಯದ ದೊಡ್ಡ ಪಂಡಿತರ ದೃಷ್ಟಿಯಿಂದಲೂ ಏನೂ ಇಲ್ಲ. ಅಲ್ಲದೆ ಯಾರು ಒಮ್ಮೆ ಪಾಶ್ಚಾತ್ಯ ದೇಶಗಳಲ್ಲಿ ಅವಸರವಸರವಾಗಿ ಪ್ರಯಾಣ ಮಾಡಿ ಐರೋಪ್ಯಶೈಲಿಯ ಉಡುಪನ್ನು ಧರಿಸಿ ‘ನಮ್ಮಲ್ಲಿ ಏನೂ ಇಲ್ಲ, ನಾವು ಅನಾಗರಿಕರು’ ಎಂದು ಗಂಟಾಘೋಷವಾಗಿ ಹೇಳುವರೋ ಅವರೂ ಕೂಡ ಹೀಗೆಯೇ ಆಲೋಚಿಸುವರು. ಇತರ ದೇಶಗಳಿಗಿರುವಂತಹ ಇತಿಹಾಸ ನಮಗಿಲ್ಲವೆಂಬುದೇನೋ ನಿಜ, ನಾವು ಅಕ್ಕಿಯನ್ನು ತಿನ್ನುತ್ತೇವೆಂದುಕೊಳ್ಳೋಣ. ಆಂಗ್ಲೇಯರು ಅದನ್ನು ತಿನ್ನುವುದಿಲ್ಲ. ಈ ಕಾರಣದಿಂದ ಅವರೆಲ್ಲಾ ಉಪವಾಸದಿಂದ ಸಾಯುವರೆಂದೂ, ನಿರ್ನಾಮವಾಗುವುದೆಂದೂ ಊಹಿಸುವೆಯೇನು? ತಮ್ಮ ದೇಶದಲ್ಲಿ ಏನು ಸಿಗುವುದೊ, ಯಾವುದನ್ನು ಸುಲಭವಾಗಿ ಉತ್ಪತ್ತಿ ಮಾಡಬಹುದೊ, ಯಾವುದು ಅವರ ದೇಹಸ್ಥಿತಿಗೆ ಅನುಗುಣವಾಗಿರುವುದೋ ಅದನ್ನು ತಿಂದು ಸುಖವಾಗಿ ಜೀವಿಸುವರು. ಅದರಂತಯೆ ನಾವೂ ಕೂಡ ನಮಗೆ ಯಾವ ಬಗೆಯ ಇತಿಹಾಸವಿರಬೇಕೊ ಅದೇ ಬಗೆಯ ಇತಿಹಾಸವನ್ನು ಹೊಂದಿದ್ದೇವೆ. ನಾವು ಕಣ್ಣನ್ನು ಮುಚ್ಚಿಕೊಂಡು ‘ಅಯ್ಯೋ! ಇತಿಹಾಸವೇ ಇಲ್ಲ!’ ಎಂದು ಕೂಗಿದರೆ ಇತಿಹಾಸವು ನಿರ್ಮೂಲವಾಗುವುದೇನು? ಯಾರಿಗೆ ಕಣ್ಣುಗಳಿವೆಯೊ ಅವರು ಇಲ್ಲಿ ಉಜ್ವಲ ಇತಿಹಾಸವನ್ನು ನೋಡುವರು. ಈ ಶಕ್ತಿಯಿಂದಲೇ ರಾಷ್ಟ್ರವು ಇನ್ನೂ ಜೀವಂತವಾಗಿರುವುದೆಂದು ತಿಳಿದಿರುವರು. ಆದರೆ ಆ ಇತಿಹಾಸವನ್ನು ಪರಿಷ್ಕರಿಸಿ ಬರೆಯಬೇಕು. ಪಾಶ್ಚಾತ್ಯ ಶಿಕ್ಷಣದ ಮೂಲಕ ಆಧುನಿಕ ಯುಗದಲ್ಲಿ ನಮ್ಮವರು ಆರ್ಜಿಸಿರುವ ತಿಳಿವಳಿಕೆ ಮತ್ತು ಆಲೋಚನಾಮಾರ್ಗಕ್ಕೆ ತಕ್ಕಂತೆ ಅದನ್ನು ಹೊಸ ರೀತಿಯಲ್ಲಿ ಹೇಳಬೇಕು.

ಪ್ರಶ್ನೆ: ಅದನ್ನು ಹೇಗೆ ಮಾಡಬೇಕು?

ಸ್ವಾಮೀಜಿ: ಅಷ್ಟೊಂದು ದೊಡ್ಡ ವಿಷಯವನ್ನು ಈಗ ಮಾತನಾಡಲು ಸಮಯವಿಲ್ಲ. ಆದರೂ, ಅದನ್ನು ತರಬೇಕಾದರೆ ನಮ್ಮ ಪ್ರಾಚೀನ ಸಂಸ್ಥೆಗಳಾದ ‘ಗುರುಕುಲ’ ಮುಂತಾದ ವಿದ್ಯಾ ಕೇಂದ್ರಗಳನ್ನು ತೆರೆಯಬೇಕು. ವೇದಾಂತದೊಡನೆ ಕೂಡಿರುವ ಪಾಶ್ಚಾತ್ಯ ವಿಜ್ಞಾನ, ಬ್ರಹ್ಮಚರ್ಯವೇ ಆಸರೆಯಾದ ಧ್ಯೇಯ ಮತ್ತು ನಮ್ಮ ಆತ್ಮದಲ್ಲಿ ದೃಢವಾದ ಶ್ರದ್ಧೆ ಮತ್ತು ನಂಬಿಕೆ - ಇವು ನಮಗೆ ಬೇಕಾಗಿರುವುದು. ನಮಗೆ ಬೇಕಾದ ಮತ್ತೊಂದೇನೆಂದರೆ ಈಗಿನ ವಿದ್ಯಾಭ್ಯಾಸದ ಪದ್ಧತಿಯನ್ನು ತೊಡೆದುಹಾಕುವುದು. ಒಬ್ಬ ಮನುಷ್ಯನಿಗೆ ಯಾರೋ ಕತ್ತೆಯನ್ನು ಚೆನ್ನಾಗಿ ಸದೆಬಡಿದರೆ ಅದು ಕುದುರೆಯಾಗುವುದೆಂದು ಹೇಳಲು ಅವನು ಹಾಗೆಯೇ ಮಾಡಿದನು. ಇದೇ ನಮ್ಮ ಹುಡುಗರಿಗೆ ವಿದ್ಯೆ ಕಲಿಸುವ ಶಿಕ್ಷಣ ಪದ್ಧತಿಯ ಗುರಿಯೂ ಆಗಿದೆ. ಆದ್ದರಿಂದ ಈ ಶಿಕ್ಷಣಪದ್ಧತಿ ಅಳಿಸಿಹೋಗಬೇಕು.

ಪ್ರಶ್ನೆ: ನೀವು ಹೀಗೆ ಹೇಳಿದುದರ ಅರ್ಥವೇನು?

ಸ್ವಾಮೀಜಿ: ನೋಡು, ಯಾರೂ ಯಾರಿಗೂ ಬೋಧಿಸಲಾರರು. ಉಪಾಧ್ಯಾಯನು ತಾನು ಬೋಧಿಸುತ್ತಿರುವೆನೆಂದು ತಿಳಿದು ಎಲ್ಲವನ್ನೂ ಹಾಳುಮಾಡುವನು. ಅದಕ್ಕೇ ವೇದಾಂತ ಹೇಳುವುದು - ಮನುಷ್ಯನಲ್ಲಿಯೇ ಎಲ್ಲಾ ಜ್ಞಾನವೂ ಇದೆ - ಚಿಕ್ಕ ಹುಡುಗನಲ್ಲಿ ಕೂಡ ಇದು ಇದೆ - ಅದನ್ನು ಜಾಗೃತಗೊಳಿಸುವುದು ಮಾತ್ರ ಆವಶ್ಯಕ. ಇಷ್ಟು ಮಾತ್ರ ಉಪಾಧ್ಯಾಯನ ಕೆಲಸ. ನಮ್ಮ ಹುಡುಗರಿಗೆ ತಮ್ಮ ಕೈ ಕಾಲು ಕಿವಿ ಕಣ್ಣು ಮುಂತಾದುವುಗಳ ಸರಿಯಾದ ಉಪಯೋಗವನ್ನು ಕಲಿಸಿಕೊಡುವುದು ಮಾತ್ರ ನಮ್ಮ ಕೆಲಸ. ಉಳಿದವೆಲ್ಲಾ ಸುಲಭವಾಗುವುದು. ಆದರೆ ಧರ್ಮವೇ ಎಲ್ಲದರ ಮೂಲ. ಧರ್ಮವೇ ಅನ್ನ. ಉಳಿದುವೆಲ್ಲಾ ವ್ಯಂಜನದಂತೆ. ಕೇವಲ ಪಲ್ಯವನ್ನೇ ತಿನ್ನುತ್ತಿದ್ದರೆ ಅಜೀರ್ಣವಾಗುವುದು. ಬರೇ ಅನ್ನವನ್ನೇ ತಿಂದರೂ ಅಜೀರ್ಣವಾಗುವುದು. ನಮ್ಮ ಉಪಾಧ್ಯಾಯರುಗಳು ನಮ್ಮ ಹುಡುಗರನ್ನು ಗಿಳಿಗಳನ್ನಾಗಿ ಮಾಡುತ್ತಿದ್ದಾರೆ. ಅನೇಕ ವಿಷಯಗಳನ್ನು ಕಂಠಪಾಠ ಮಾಡುವಂತೆ ಮಾಡಿ ಅವರ ಮಿದುಳನ್ನು ನಾಶಗೊಳಿಸುತ್ತಿದ್ದಾರೆ. ಒಂದು ದೃಷ್ಟಿಯಿಂದ ನೋಡಿದರೆ, ಈ ವಿಶ್ವವಿದ್ಯಾನಿಲಯದ ಪದ್ದತಿಯ ಸುಧಾರಣೆಯಾಗಬೇಕೆಂದು ಸಲಹೆ ಮಾಡಿದ ವೈಸರಾಯರಿಗೆ ಕೃತಜ್ಞರಾಗಿರಬೇಕು. ಏಕೆಂದರೆ ಇದರಿಂದ ಉಚ್ಛ ಶಿಕ್ಷಣ ಪದ್ಧತಿಯು ಕಾರ್ಯತಃ ತೆಗೆದುಹಾಕಲ್ಪಟ್ಟಂತೆಯೆ - ದೇಶಕ್ಕೆ ಇದರಿಂದ ಕೊಂಚ ಉಸಿರಾಡಲವಕಾಶ ಸಿಕ್ಕಿ ನೆಮ್ಮದಿಯಾಗುವುದು. ಅಯ್ಯೋ ದೇವರೆ! ಈ ಪದವೀಧರ ಪಟ್ಟಕ್ಕಾಗಿ ಎಷ್ಟೊಂದು ರೋಷಾವೇಶ, ಗಲಿಬಿಲಿ! ಕೊಂಚ ದಿನಗಳಲ್ಲಿಯೇ ಎಲ್ಲಾ ತಣ್ಣಗಾಯಿತು! ಇಷ್ಟೆಲ್ಲಾ ಆದಮೇಲೆ ಅವರು ಏನನ್ನು ಕಲಿಯುವುದು - ನಮ್ಮ ಧರ್ಮ, ನಮ್ಮ ನಡೆನುಡಿಗಳೆಲ್ಲಾ ಕೆಟ್ಟದ್ದು, ಪಾಶ್ಚಾತ್ಯರು ಹೊಂದಿರುವುದೆಲ್ಲಾ ಒಳ್ಳೆಯದು ಎಂಬುದನ್ನು ಕಲಿಯುವರಲ್ಲವೆ! ಕೊನೆಗೆ ಅವರು ಉಪವಾಸದಿಂದ ಪಾರಾಗಲಾರರು. ಈ ಉಚ್ಛ ಶಿಕ್ಷಣ ಇದ್ದರೆಷ್ಟು ಹೋದರೆಷ್ಟು? ಜನರು ತಾವು ಕೆಲಸಮಾಡಿ ತಮ್ಮ ಅನ್ನವನ್ನು ಸಂಪಾದಿಸುವಂತಹ ಕೈಗಾರಿಕಾ ಕಲೆಗಳ ಶಿಕ್ಷಣವನ್ನು ಪಡೆದರೆ ಎಷ್ಟೋ ವಾಸಿ. ಕೆಲಸಕ್ಕಾಗಿ ಅಲೆಯುತ್ತಾ ವ್ಯರ್ಥವಾಗಿ ಕಾಲ ಕಳೆಯುವುದು ತಪ್ಪುವುದು.

ಪ್ರಶ್ನೆ: ಹೌದು, ಮಾರ್ವಾಡಿಗಳೂ ಹಾಗೆಯೇ! ಅವರು ಯಾರಿಂದಲೂ ಕೆಲಸ ಮಾಡಿಸಿಕೊಳ್ಳುವುದಿಲ್ಲ. ಮುಕ್ಕಾಲು ಪಾಲು ಮಂದಿ ಯಾವುದಾದರೊಂದು ವ್ಯಾಪಾರದಲ್ಲಿ ನಿರತರಾಗಿರುವರು.

ಸ್ವಾಮೀಜಿ: ಎಂತಹ ಹುಚ್ಚು ಮಾತು! ಅವರು ನಮ್ಮ ದೇಶವನ್ನು ಹಾಳುಮಾಡುವುದರಲ್ಲಿರುವರು. ಅವರಿಗೆ ತಮ್ಮ ಹಿತದ ಅರ್ಥ ಕೂಡ ಆಗಿಲ್ಲ. ನೀವು ಎಷ್ಟೋ ವಾಸಿ. ಏಕೆಂದರೆ ಕೈಗಾರಿಕೆಯ ಕಡೆಗೆ ಹೆಚ್ಚು ಗಮನ ಕೊಡುವಿರಿ. ಅವರು ತಮ್ಮ ವ್ಯಾಪಾರಕ್ಕಾಗಿ ಹಾಕುವ ಹಣ ಮತ್ತು ಅವರಿಗೆ ಬರುವ ಅಲ್ಪಬಡ್ಡಿಯ ಹಣವೆಲ್ಲವನ್ನೂ ಕಾರ್ಖಾನೆ ಕೈಗಾರಿಕಾ ಸ್ಥಳಗಳಿಗೆ ಸರಿಯಾಗಿ ಉಪಯೋಗಿಸಿದರೆ ಆ ಯೂರೋಪಿಯನ್ ಜನರು ಆ ಹಣವನ್ನು ಹೊಂದಿ ಆ ವ್ಯಾಪಾರದ ಮುಕ್ಕಾಲು ಭಾಗ ಲಾಭವನ್ನು ಪಡೆದು ತಮ್ಮ ಜೇಬನ್ನು ತುಂಬಿಸಿಕೊಳ್ಳುವಂತೆ ಮಾಡುವುದು ತಪ್ಪಿ ನಮ್ಮ ದೇಶದ ಅಭಿವೃದ್ಧಿಗೆ ಸಹಾಯವಾಗುವುದಲ್ಲದೆ ಅವರಿಗೆ ಅಧಿಕವಾದ ಲಾಭವುಂಟಾಗುವುದು. ಕಾಬೂಲಿಗಳು ಮಾತ್ರ ಕೆಲಸಕ್ಕಾಗಿ ಕೊಂಚವೂ ಗಮನ ಕೊಡುವುದಿಲ್ಲ - ಅವರಲ್ಲಿ ಸ್ವಾತಂತ್ರ್ಯದ ಕೆಚ್ಚು ಅಸ್ಥಿಗತವಾಗಿದೆ. ಅವರಲ್ಲಿ ಯಾರಿಗಾದರೂ ಕೆಲಸಕ್ಕೆ ಸೇರಲು ಹೇಳು, ಮುಂದೇನಾಗುವುದೆಂದು ಗೊತ್ತಾಗುವುದು.

ಪ್ರಶ್ನೆ: ಸರಿ, ಮಹಾರಾಜ್, ಉಚ್ಛ ಶಿಕ್ಷಣವನ್ನು ತೆಗೆದುಹಾಕಿದ ಪಕ್ಷಕ್ಕೆ ಜನರು ಹಿಂದಿನಷ್ಟೇ ಮೂಢರಾಗಿ ಉಳಿಯುವರಲ್ಲವೆ?

ಸ್ವಾಮೀಜಿ: ಎಂತಹ ತಿಳಿಗೇಡಿತನ! ಸಿಂಹವು ನರಿಯಾಗಲು ಸಾಧ್ಯವೆ? ನೀನೇನು ಹೇಳುತ್ತಿರುವೆ? ಅನಾದಿಕಾಲದಿಂದಲೂ ಯಾವ ನಾಡಿನ ಮಕ್ಕಳು ಇಡೀ ಪ್ರಪಂಚಕ್ಕೆ ಜ್ಞಾನವನ್ನು ಕೊಟ್ಟು ಪೋಷಿಸಿದ್ದಾರೊ ಅಂಥವರು, ಲಾರ್ಡ್ ಕರ್ಜನ್ ಉಚ್ಚ ಶಿಕ್ಷಣವನ್ನು ತೆಗೆದುಹಾಕಿದ್ದರಿಂದ ದಡ್ಡರಾಗಲು ಸಾಧ್ಯವೇನು?

ಪ್ರಶ್ನೆ: ಆದರೆ ಆಂಗ್ಲೇಯರು ನಮ್ಮ ದೇಶಕ್ಕೆ ಬರುವ ಮೊದಲು ನಮ್ಮ ದೇಶ ಹೇಗಿತ್ತು, ಈಗ ಹೇಗಿದೆ ಎಂದು ಯೋಚಿಸಿ.

ಸ್ವಾಮೀಜಿ: ಪ್ರೌಢ ವಿದ್ಯಾಭ್ಯಾಸ ಎಂದರೆ ಕೇವಲ ಭೌತಶಾಸ್ತ್ರಗಳ ವ್ಯಾಸಂಗ ಮತ್ತು ನಾವು ನಿತ್ಯ ಬಳಸುವ ವಸ್ತುಗಳನ್ನು ಯಂತ್ರದಿಂದ ತಯಾರಿಸುವುದು ಇಷ್ಟೇ ಏನು? ಜೀವನದ ಸಮಸ್ಯೆಗಳನ್ನು ಬಿಡಿಸುವುದು ಹೇಗೆ ಎಂಬುದನ್ನು ಕಂಡುಹಿಡಿಯುವದು ಪ್ರೌಢ ಶಿಕ್ಷಣದಿಂದಾಗುವ ಉಪಯೋಗ. ಈ ಯೋಚನೆಯೇ ಆಧುನಿಕ ನಾಗರಿಕ ಜಗತ್ತನ್ನು ಗಾಢವಾದ ಆಲೋಚನೆಯಲ್ಲಿ ಮುಳುಗಿಸಿದೆ. ಆದರೆ ನಮ್ಮ ದೇಶೀಯರು ಸಾವಿರಾರು ವರುಷಗಳ ಹಿಂದೆಯೇ ಈ ಸಮಸ್ಯೆಯನ್ನು ಬಗೆಹರಿಸಿದ್ದರು.

ಪ್ರಶ್ನೆ: ಆದರೆ ನಿಮ್ಮ ವೇದಾಂತವೂ ಇನ್ನೇನು ಕಣ್ಮರೆಯಾಗುವುದರಲ್ಲಿತ್ತು.

ಸ್ವಾಮೀಜಿ: ಹಾಗಿದ್ದಿರಬಹುದು. ಕಾಲದ ಬಹಿಃಪ್ರವಾಹದಲ್ಲಿ ವೇದಾಂತ ಜ್ಯೋತಿ ಆಗಾಗ್ಗೆ ಆರಿಹೋಗುವುದರಲ್ಲಿದೆ ಎಂಬಂತೆ ತೋರುವುದು. ಹೀಗೆ ಆದಾಗ ಭಗವಂತನೇ ಮಾನವ ಶರೀರದಲ್ಲಿ ಅವತಾರ ಮಾಡಬೇಕಾಗುವುದು. ಅವನು ಧರ್ಮಕ್ಕೆ ಎಷ್ಟರಮಟ್ಟಿಗೆ ಜೀವವನ್ನು, ಶಕ್ತಿಯನ್ನು ತುಂಬುವನೆಂದರೆ, ಅದು ಪುನಃ ಕೊಂಚಕಾಲದಲ್ಲಿಯೇ ತಡೆಯಲಸಾಧ್ಯವಾದ ಚೈತನ್ಯದಿಂದ ಕೂಡಿ ಪ್ರವಹಿಸುವುದು. ಆ ಜೀವಕಳೆ ಮತ್ತು ಸತ್ತ್ವ ಪುನಃ ಅದಕ್ಕೆ ಬರುವುವು.

ಪ್ರಶ್ನೆ: ಮಹಾರಾಜ್, ಭರತಖಂಡವು ಉಳಿದ ಜಗತ್ತಿಗೆಲ್ಲಾ ಉದಾರವಾಗಿ ಜ್ಞಾನದಾನ ಮಾಡಿರುವುದಕ್ಕೆ ಯಾವ ಆಧಾರವಿದೆ?

ಸ್ವಾಮೀಜಿ: ಇತಿಹಾಸವೇ ಇದಕ್ಕೆ ಸಾಕ್ಷಿ. ಪ್ರಪಂಚದಲ್ಲಿರುವ ಎಲ್ಲಾ ಆತ್ಮೋನ್ನತಿಯ ಭಾವನೆಗಳಿಗೂ, ಎಲ್ಲಾ ಬಗೆಯ ಜ್ಞಾನದ ಶಾಖೆಗಳಿಗೂ ಭರತಖಂಡವೇ ಮೂಲವೆಂದು ಸೂಕ್ತ ಅಧ್ಯಯನದಿಂದ ತಿಳಿದುಬಂದಿದೆ.

ಉತ್ಸಾಹದಿಂದ ಪ್ರಜ್ವಲಿಸುತ್ತಾ ಸ್ವಾಮಿಜಿಯವರು ಈ ವಿಚಾರವಾಗಿ ದೀರ್ಘವಾಗಿ ಮಾತನಾಡಿದರು. ಅವರ ಆರೋಗ್ಯ ಆಗ ತುಂಬಾ ಕೆಟ್ಟಿತ್ತು. ಅಲ್ಲದೆ ಬೇಸಿಗೆಯ ತೀವ್ರತೆಯಿಂದ ಅವರಿಗೆ ತುಂಬಾ ಬಾಯಾರಿಕೆಯಾಗಿ ನೀರನ್ನು ಆಗಾಗ್ಗೆ ಕುಡಿಯುತ್ತಿದ್ದರು. ಕಡೆಗೆ ಅವರು “ಪ್ರಿಯ ಸಿಂಗ್, ದಯವಿಟ್ಟು ಐಸ್ ಹಾಕಿದ ನೀರನ್ನು ತೆಗೆದುಕೊಂಡು ಬಾ, ಎಲ್ಲವನ್ನೂ ನಿನಗೆ ವಿಶದವಾಗಿ ತಿಳಿಸುವೆ” ಎಂದರು. ಶೀತಲವಾದ ನೀರನ್ನು ಕುಡಿದಾದ ಮೇಲೆ ಪುನಃ ಪ್ರಾರಂಭಿಸಿದರು.

ಸ್ವಾಮೀಜಿ: ನಮ್ಮದೇ ಆಗಿರುವ ಅನೇಕ ಬಗೆಯ ಜ್ಞಾನದ ಶಾಖೆಗಳನ್ನೆಲ್ಲಾ ಪರದೇಶೀಯರ ಹತೋಟಿಯಿಂದ ವಿಮುಕ್ತರಾಗಿ ಕಲಿಯಬೇಕು. ಜೊತೆಗೆ ಆಂಗ್ಲ ಭಾಷೆ ಮತ್ತು ಪಾಶ್ಚಾತ್ಯ ವಿಜ್ಞಾನವನ್ನೂ ಕಲಿಯಬೇಕು. ಇದೇ ನಮಗೆ ಈಗ ಬೇಕಾಗಿರುವುದು. ನಮಗೆ ಕೈಗಾರಿಕಾ ಶಿಕ್ಷಣ ಆವಶ್ಯಕ ಮತ್ತು ಯಾವುದರಿಂದ ಕೈಗಾರಿಕೆಗಳ ಬೆಳವಣಿಗೆಯಾಗುವುದೋ ಯಾವುದರಿಂದ ಜನರು ಉದ್ಯೋಗವನ್ನು ಹುಡುಕಿಕೊಂಡು ಅಲೆಯುವುದರ ಬದಲು ತಮಗೆ ಸಾಕಾಗುವಷ್ಟನ್ನು ಮತ್ತು ಕಷ್ಟಕಾಲಕ್ಕೆ ಕೊಂಚ ಉಳಿಸುವಷ್ಟನ್ನು ಸಂಪಾದಿಸಬಹುದೊ ಅಂತಹ ಶಿಕ್ಷಣ ಆವಶ್ಯಕ.

ಪ್ರಶ್ನೆ: ಅಂದು ತಾವು ಸಂಸ್ಕೃತ ಶಾಲೆಯ ವಿಚಾರವಾಗಿ ಏನನ್ನು ಹೇಳತೊಡಗಿದ್ದೀರಿ?

ಸ್ವಾಮೀಜಿ: ನೀನು ಉಪನಿಷತ್ತಿನ ಕಥೆಗಳನ್ನು ಓದಿಲ್ಲವೆ? ನಾನು ಸತ್ಯಕಾಮನ ವಿಚಾರವಾಗಿ ಹೇಳುವೆ. ಅವನು ತನ್ನ ಗುರುವಿನ ಬಳಿಗೆ ಬ್ರಹ್ಮಚಾರಿಯ ಜೀವನ ನಡೆಸಲು ಹೋದನು. ಗುರು ಅವನ ವಶಕ್ಕೆ ಕೆಲವು ಹಸುಗಳನ್ನು ಕೊಟ್ಟು ಅವನನ್ನು ಅವುಗಳೊಡನೆ ಕಾಡಿಗೆ ಕಳುಹಿಸಿದನು. ಅನೇಕ ತಿಂಗಳು ಕಳೆದವು. ಸತ್ಯಕಾಮ ಹಸುಗಳ ಸಂಖ್ಯೆ ಇಮ್ಮಡಿಯಾದುದನ್ನು ನೋಡಿ ಗುರುವಿನ ಬಳಿಗೆ ಹಿಂತಿರುಗಬೇಕೆಂದು ಯೋಚಿಸಿದನು. ದಾರಿಯಲ್ಲಿ ಹೋರಿಗಳಲ್ಲೊಂದು, ಅಗ್ನಿ ಮತ್ತು ಕೆಲವು ಇತರ ಪ್ರಾಣಿಗಳು ಅವನಿಗೆ ಪರಬ್ರಹ್ಮನ ವಿಚಾರವಾಗಿ ಕೆಲವು ಸಲಹೆಗಳನ್ನು ಕೊಟ್ಟುವು. ಶಿಷ್ಯ ಹಿಂತಿರುಗಿದಾಗ ಗುರು ಶಿಷ್ಯನ ಮುಖವನ್ನು ಒಮ್ಮೆ ನೋಡಿದ ಕ್ಷಣವೇ ಶಿಷ್ಯನಿಗೆ ಪರಬ್ರಹ್ಮನ ಜ್ಞಾನದ ಅರಿವಾಗಿದೆ ಎಂದು ತಿಳಿಯಿತು. ನಿಜವಾದ ವಿದ್ಯಾಭ್ಯಾಸವನ್ನು ಪ್ರಕೃತಿಯೊಡನೆ ಬೆರೆತು ಜೀವಿಸುವುದರಿಂದ ಪಡೆಯಬಹುದು ಎಂಬ ನೀತಿಯನ್ನು ಈ ಕಥೆಯು ಬೋಧಿಸುತ್ತದೆ.

ಈ ರೀತಿಯಾಗಿ ಜ್ಞಾನವನ್ನು ಪಡೆಯಬೇಕು. ಹಾಗಲ್ಲದೆ ಒಬ್ಬ ಪಂಡಿತನ ಶಾಲೆಯಲ್ಲಿ ಶಿಕ್ಷಣ ಪಡೆಯುತ್ತಾ ನೀನು ನಿನ್ನ ಇಡೀ ಜೀವನವನ್ನು ಕಳೆದರೂ ಒಬ್ಬ ಮಾನವ ಕಪಿಯಂತಾಗುವೆ. ಯಾರ ಶೀಲವು ಅಗ್ನಿಯಂತೆ ಉಜ್ವಲವಾಗಿದೆಯೋ ಅಂತಹವರೊಡನೆ ಬಾಲ್ಯದಿಂದಲೂ ಜೀವಿಸಬೇಕು. ಮಹತ್ವಪೂರ್ಣವಾದ ಉಪದೇಶದ ಜೀವಂತ ದೃಷ್ಟಾಂತವು ಯಾವಾಗಲೂ ನಮ್ಮ ಮುಂದಿರಬೇಕು. ಸುಳ್ಳು ಹೇಳುವುದು ಪಾಪಕರ ಎಂದು ಸುಮ್ಮನೆ ಓದಿದರೆ ಏನೂ ಪ್ರಯೋಜನವಿಲ್ಲ. ಪ್ರತಿಯೊಬ್ಬ ಹುಡುಗನೂ ಅಖಂಡ ಬ್ರಹ್ಮಚರ್ಯ ಸಾಧನೆಯ ಶಿಕ್ಷಣ ಹೊಂದಬೇಕು. ನಂತರವೇ ನಂಬಿಕೆ ಮತ್ತು ಶ್ರದ್ಧೆ ಬರುವುದು. ಇಲ್ಲದಿದ್ದರೆ ಯಾರು ನಂಬಿಕೆ ಮತ್ತು ಶ್ರದ್ಧೆಯನ್ನು ಹೊಂದಿಲ್ಲವೊ ಅವರು ಸುಳ್ಳನ್ನು ಹೇಳದೆ ಇರುತ್ತಾರೇನು? ನಮ್ಮ ದೇಶದಲ್ಲಿ ಜ್ಞಾನ ವಿನಿಯೋಗ ಯಾವಾಗಲೂ ತ್ಯಾಗಜೀವಿಗಳ ಮೂಲಕ ಆಗಿದೆ. ಆಮೇಲೆ ಬಂದ ಪಂಡಿತರು ಜ್ಞಾನವನ್ನೆಲ್ಲಾ ತಾವೇ ಸ್ವಾಧೀನಪಡಿಸಿಕೊಂಡು ಅವು ಕೇವಲ ಶಾಲೆಯಲ್ಲಿ ಮಾತ್ರ ಕೊನೆಗಾಣುವಂತೆ ಮಾಡಿ ದೇಶವನ್ನು ನಾಶದ ಸ್ಥಿತಿಗೆ ತಂದಿರುವರು. ಎಲ್ಲಿಯವರೆಗೆ ತ್ಯಾಗಿಗಳು ಜ್ಞಾನಪ್ರಸಾರ ಮಾಡುತ್ತಿದ್ದರೋ ಅಲ್ಲಿಯವರೆಗೂ ಭರತಖಂಡ ಎಲ್ಲಾ ಬಗೆಯ ಒಳ್ಳೆಯ ಭರವಸೆಗಳನ್ನೂ ಹೊಂದಿತ್ತು.

ಪ್ರಶ್ನೆ: ಮಹಾರಾಜ್, ನೀವು ಹೇಳಿದುದರ ಅರ್ಥವೇನು? ಇತರ ದೇಶಗಳಲ್ಲಿ ಸಂನ್ಯಾಸಿಗಳೇ ಇಲ್ಲ. ಆದರೆ ನೋಡಿ, ಅವರ ವಿದ್ಯಾಶಕ್ತಿಯಿಂದ ಭರತಖಂಡ ಅವರ ಪಾದದೆಡೆ ಬಿದ್ದಿದೆ.

ಸ್ವಾಮೀಜಿ: ಹುಚ್ಚುಹುಚ್ಚಾಗಿ ಮಾತನಾಡಬೇಡ. ಸ್ನೇಹಿತನೆ, ನಾನು ಹೇಳುವುದನ್ನು ಕೇಳು. ತನ್ನ ಮಕ್ಕಳಿಗೆ ಜ್ಞಾನಪ್ರಸಾರಮಾಡುವ ಕಾರ್ಯ ತ್ಯಾಗಿಗಳ ಹೆಗಲ ಮೇಲೆ ಪುನಃ ಬೀಳುವವರೆಗೂ ಭರತಖಂಡ ಇತರರ ಪಾದುಕೆಗಳನ್ನು ತನ್ನ ತಲೆಯ ಮೇಲೆ ಹೊರಬೇಕಾಗುವುದು. ಒಬ್ಬ ಅವಿದ್ಯಾವಂತನಾದ ಆದರೆ ತ್ಯಾಗಿಯಾದ ಹುಡುಗ ನಿಮ್ಮ ದೊಡ್ಡ ದೊಡ್ಡ ವಯಸ್ಸಾದ ಪಂಡಿತರ ತಲೆಯನ್ನೆಲ್ಲಾ ತಿರುಗಿಸಿಬಿಟ್ಟಿದ್ದನ್ನು ನೀನು ತಿಳಿದಿಲ್ಲವೆ? ದಕ್ಷಿಣೇಶ್ವರದಲ್ಲಿ ಒಮ್ಮೆ ವಿಷ್ಣು ದೇವಾಲಯದ ಅರ್ಚಕನಾಗಿದ್ದ ಬ್ರಾಹ್ಮಣನು ವಿಗ್ರಹದ ಕಾಲೊಂದನ್ನು ಮುರಿದುಬಿಟ್ಟನು. ಪಂಡಿತರ ಅಭಿಪ್ರಾಯಗಳನ್ನು ಕೇಳಲು ಸಭೆಯನ್ನು ಸೇರಿಸಲಾಗಿತ್ತು. ಅವರು ಹಳೆಯ ಗ್ರಂಥ ಮತ್ತು ಓಲೆಗರಿಯ ಪುಸ್ತಕಗಳನ್ನೆಲ್ಲಾ ತಿರುವಿಹಾಕಿ ಶಾಸ್ತ್ರ ಪ್ರಕಾರ ಭಿನ್ನವಾದ ವಿಗ್ರಹವು ಪೂಜಾರ್ಹವಲ್ಲವೆಂದು ನಿರ್ಧರಿಸಿ ಹೊಸ ವಿಗ್ರಹ ಪ್ರತಿಷ್ಠಾಪನೆಯಾಗಬೇಕೆಂದು ಹೇಳಿದರು. ಪರಿಣಾಮವಾಗಿ ದೊಡ್ಡ ಕೋಲಾಹಲವೆದ್ದಿತು. ಕಡೆಗೆ ಶ‍್ರೀರಾಮಕೃಷ್ಣರನ್ನು ಕರೆಸಿದರು. ಅವರು ಎಲ್ಲವನ್ನೂ ಕೇಳಿ ನಂತರ ‘ಗಂಡ ಹೆಳವನಾದನೆಂದು ಹೆಂಡತಿಯಾದವಳು ಅವನನ್ನು ತ್ಯಜಿಸುವಳೇನು?’ ಎಂದು ಕೇಳಿದರು. ಮುಂದೇನಾಯಿತು? ಪಂಡಿತರೆಲ್ಲಾ ಸ್ತಂಭೀಭೂತರಾದರು. ಅವರ ಶಾಸ್ತ್ರದ ಭಾಷ್ಯ ಟೀಕೆಗಳು ಎಲ್ಲ ಈ ಒಂದು ಸರಳವಾದ ಮಾತಿನ ಮುಂದೆ ನಿಲ್ಲಲಾರದೆ ಹೋದವು. ನಿನ್ನ ಮಾತು ನಿಜವಾಗಿದ್ದರೆ ಶ‍್ರೀರಾಮಕೃಷ್ಣರು ಭೂಮಿಗೇಕೆ ಬರಬೇಕಿತ್ತು? ಕೇವಲ ಪುಸ್ತಕ ಪಾಂಡಿತ್ಯಕ್ಕೆ ಅವರೇಕೆ ಪ್ರೋತ್ಸಾಹ ಕೊಡಲಿಲ್ಲ? ಅವರು ತಮ್ಮೊಡನೆ ತಂದ ಹೊಸ ಜೀವ ಚೈತನ್ಯವನ್ನು, ಪಾಂಡಿತ್ಯಕ್ಷೇತ್ರಕ್ಕೆ ತುಂಬಬೇಕು. ನಂತರವೇ ನಿಜವಾದ ಕೆಲಸ ಮಾಡಬಹುದು.

ಪ್ರಶ್ನೆ: ಆದರೆ ಅದನ್ನು ಮಾಡುವುದಕ್ಕಿಂತ ಹೇಳುವುದು ಸುಲಭ.

ಸ್ವಾಮೀಜಿ: ಅದು ಸುಲಭವಾಗಿದ್ದರೆ ಶ‍್ರೀರಾಮಕೃಷ್ಣರು ಬರುವ ಆವಶ್ಯಕತೆಯೇನಿತ್ತು? ಪ್ರತಿಯೊಂದು ಪಟ್ಟಣದಲ್ಲಿಯೂ ಹಳ್ಳಿಯಲ್ಲಿಯೂ ಮಠವನ್ನು ಸ್ಥಾಪಿಸುವುದೇ ನೀವೀಗ ಮಾಡಬೇಕಾದ ಕೆಲಸ. ನೀನು ಅದನ್ನು ಮಾಡಬಲ್ಲೆಯಾ? ಕಡೆಯಪಕ್ಷ ಕೊಂಚವನ್ನಾದರೂ ಮಾಡು. ಕಲ್ಕತ್ತೆಯ ಮಧ್ಯಭಾಗದಲ್ಲಿ ದೊಡ್ಡ ಮಠವನ್ನು ಪ್ರಾರಂಭಿಸು. ಒಬ್ಬ ವಿದ್ಯಾವಂತನಾದ ಸಾಧು ಅದರ ಮೇಲ್ವಿಚಾರಕನಾಗಲಿ. ಅವನ ಕೈಕೆಳಗೆ ಪ್ರಾಯೋಗಿಕ ವಿಜ್ಞಾನ, ಕಲೆ ಮುಂತಾದುವುಗಳನ್ನು ಬೋಧಿಸಲು ಶಾಖೆಗಳಿರಲಿ, ಪ್ರತಿಯೊಂದು ಶಾಖೆಗೂ ಒಬ್ಬ ಪ್ರತ್ಯೇಕ ಸಂನ್ಯಾಸಿ ಅದರ ನಿರ್ವಾಹಕನಾಗಿರಲಿ.

ಪ್ರಶ್ನೆ: ನಿಮಗೆ ಅಂತಹ ಸಾಧುಗಳೆಲ್ಲಿ ಸಿಗುವರು?

ಸ್ವಾಮೀಜಿ: ನಾವು ಅವರನ್ನು ಸಿದ್ಧಪಡಿಸಬೇಕು, ಸೃಷ್ಟಿಮಾಡಬೇಕು. ಆದ್ದರಿಂದಲೇ ಪ್ರಜ್ವಲಿಸುವ ದೇಶಭಕ್ತಿ ಮತ್ತು ತ್ಯಾಗದಿಂದ ಕೂಡಿದ ಯುವಕರು ಬೇಕಾಗಿದೆ ಎಂದು ನಾನು ಯಾವಾಗಲೂ ಹೇಳುವುದು. ಯಾರು ತ್ಯಾಗಿಗಳಷ್ಟು ಶೀಘ್ರವಾಗಿ ಪೂರ್ಣವಾಗಿ ಒಂದು ವಿಷಯವನ್ನು ತಿಳಿಯಬಲ್ಲರು?

ಕೊಂಚ ಹೊತ್ತು ಮೌನವಾಗಿದ್ದು ಸ್ವಾಮೀಜಿ “ಸಿಂಗ್‌ಜಿ, ನಮ್ಮ ದೇಶಕ್ಕೆ ಮಾಡಬೇಕಾದ ಕೆಲಸಗಳು ಹೇರಳವಾಗಿವೆ. ನಿಮ್ಮಂತಹ ಸಾವಿರಾರು ಮಂದಿ ಬೇಕಾಗಿದ್ದಾರೆ. ಕೇವಲ ಮಾತಿನಿಂದೇನು ಪ್ರಯೋಜನ? ದೇಶ ಎಂತಹ ಶೋಚನೀಯಾವಸ್ಥೆಗಿಳಿದಿದೆ ಎಂಬುದನ್ನು ನೋಡು, ಈಗ ಏನನ್ನಾದರೂ ಮಾಡು! ನಮ್ಮ ಚಿಕ್ಕ ಹುಡುಗರಿಗೆ ಯೋಗ್ಯವಾಗಿರುವ ಒಂದು ಪುಸ್ತಕ ಕೂಡ ಈಗ ನಮ್ಮಲ್ಲಿಲ್ಲ.”

ಪ್ರಶ್ನೆ: ಚಿಕ್ಕ ಹುಡುಗರಿಗೆ ಈಶ್ವರಚಂದ್ರ ವಿದ್ಯಾಸಾಗರರ ಹಲವಾರು ಪುಸ್ತಕಗಳಿವೆಯಲ್ಲ?

ನಾನು ಅದನ್ನು ಹೇಳಿದ ಕೂಡಲೇ ಸ್ವಾಮೀಜಿ ಗಹಗಹಿಸಿ ನಕ್ಕು “ಹೌದು, ಅಲ್ಲಿ ನೀನು ಈಶ್ವರ ನಿರಾಕಾರ ಚೈತನ್ಯಸ್ವರೂಪನು - ಸುಬಲನು ತುಂಬಾ ಒಳ್ಳೆಯ ಬುದ್ಧಿವಂತನಾದ ಹುಡುಗ, ಮುಂತಾಗಿ ಓದುವೆ. ಅದರಿಂದೇನೂ ಪ್ರಯೋಜನವಿಲ್ಲ. ರಾಮಾಯಣ ಮಹಾಭಾರತ ಉಪನಿಷತ್ತು ಮುಂತಾದವುಗಳಿಂದ ಚಿಕ್ಕ ಕಥೆಗಳನ್ನು ಆರಿಸಿ ಬಂಗಾಳಿ ಮತ್ತು ಇಂಗ್ಲೀಷಿನಲ್ಲಿ ಸಣ್ಣ ಪುಸ್ತಕಗಳನ್ನು ಮಾಡಬೇಕು. ಅದರಲ್ಲಿ ಸರಳವಾದ ಸುಲಭ ಶೈಲಿ ಇರಬೇಕು. ಇವುಗಳನ್ನು ನಮ್ಮ ಚಿಕ್ಕ ಹುಡುಗರಿಗೆ ಓದಲು ಕೊಡಬೇಕು.”

ಅಷ್ಟು ಹೊತ್ತಿಗೆ ಹನ್ನೊಂದು ಗಂಟೆಯ ವೇಳೆಯಾಗಿತ್ತು. ಆಕಾಶವು ಇದ್ದಕ್ಕಿದ್ದಂತೆಯೇ ಮೇಘದಿಂದಾವೃತವಾಗಿ ತಂಗಾಳಿ ಬೀಸತೊಡಗಿತು. ಸ್ವಾಮೀಜಿಗೆ ಮಳೆಯ ಸೂಚನೆಯನ್ನು ನೋಡಿ ತುಂಬಾ ಆನಂದವಾಯಿತು. ಅವರು ಎದ್ದು ನಿಂತು “ಸಿಂಗ್ ಜಿ, ಗಂಗಾತೀರದಲ್ಲಿ ನಾವೆಲ್ಲಾ ಸುತ್ತಾಡಿಕೊಂಡು ಬರುವ, ನಡೆ” ಎಂದರು. ನಾವು ಹಾಗೇ ಮಾಡಿದೆವು. ಕಾಳಿದಾಸನ ಮೇಘದೂತದ ಅನೇಕ ಶ್ಲೋಕಗಳನ್ನು ಅವರು ಹಾಡಿದರು. ಆದರೆ ಮೊದಲಿನಿಂದ ಕೊನೆಯವರೆಗೂ ಅವರ ಮನಸ್ಸಿನಲ್ಲಿ ಪ್ರವಹಿಸುತ್ತಿದ್ದ ಒಂದು ಆಲೋಚನೆಯ ಅಂತಃಪ್ರವಾಹವೆಂದರೆ ಭರತಖಂಡದ ಕಲ್ಯಾಣ. ಅವರು “ಇಲ್ಲಿ ನೋಡು ಸಿಂಗ್‌ಜಿ, ನೀನು ಒಂದು ಕೆಲಸವನ್ನು ಮಾಡಬಲ್ಲೆಯಾ! ನಮ್ಮ ಹುಡುಗರು ಮದುವೆ ಮಾಡಿಕೊಳ್ಳುವುದನ್ನು ಕೊಂಚ ಕಾಲದವರೆಗೆ ನಿಲ್ಲಿಸಬಲ್ಲೆಯಾ?” ಎಂದರು.

ನಾನು ಹೇಳಿದೆ: ಸರಿ ಮಹಾರಾಜ್, ಬಾಬುಗಳಂತಹವರೆ ಅದಕ್ಕೆ ವಿರೋಧವಾಗಿ ಮದುವೆಯನ್ನು ಆದಷ್ಟು ಸುಲಭವಾಗಿ ಮಾಡಲು ಅನೇಕ ರೀತಿ ಪ್ರಯತ್ನಿಸುತ್ತಿದ್ದಾರೆ - ಎಂದ ಮೇಲೆ ನಾವು ಮಾಡಲು ಹೇಗೆ ಸಾಧ್ಯ?

ಸ್ವಾಮೀಜಿ: ಆ ವಿಚಾರವಾಗಿ ನಿನ್ನ ತಲೆಯನ್ನು ಕೆಡಿಸಿಕೊಳ್ಳಬೇಡ. ಕಾಲಪ್ರವಾಹವನ್ನು ತಡೆಯಬಲ್ಲವರಾರು? ಈ ಬಗೆಯ ಚಳುವಳಿಗಳೆಲ್ಲಾ ಕೊನೆಗೆ ತಣ್ಣಗಾಗುವುವು ಅಷ್ಟೆ. ಮದುವೆಗಳು ದುಬಾರಿಯಾದಷ್ಟೂ ದೇಶಕ್ಕೆ ಒಳ್ಳೆಯದಾಗುವುದು. ಪರೀಕ್ಷೆಯನ್ನು ಎಷ್ಟು ಅವಸರದಲ್ಲಿ ಮುಗಿಸಿ ತಕ್ಷಣ ಮದುವೆ ಮಾಡಿಕೊಳ್ಳುವರು! ಯಾರೂ ಅವಿವಾಹಿತರು ಉಳಿಯುವುದೇ ಇಲ್ಲವೇನೋ ಎನ್ನಿಸುವುದು. ಪುನಃ ಮುಂದಿನ ವರುಷ ಅದೇ ಸ್ಥಿತಿ!

ಸ್ವಲ್ಪ ಹೊತ್ತು ಸುಮ್ಮನೆ ಇದ್ದು ಪುನಃ ಸ್ವಾಮೀಜಿ, ನನಗೆ ಕೆಲವು ಮಂದಿ ಅವಿವಾಹಿತ ಪದವೀಧರರು ಸಿಕ್ಕಿದರೆ ನಾನು ಅವರನ್ನು ಜಪಾನಿಗೆ ಕಳುಹಿಸಿ ಅವರು ಅಲ್ಲಿಯ ಕೈಗಾರಿಕಾ ಶಿಕ್ಷಣ ಪಡೆಯುವಂತೆ ವ್ಯವಸ್ಥೆ ಮಾಡುವೆ. ಅವರು ಹಿಂತಿರುಗಿ ಬಂದಾಗ ಭರತಖಂಡದ ಉನ್ನತಿಗಾಗಿ ಅವರು ತಮ್ಮ ಜ್ಞಾನವನ್ನೆಲ್ಲಾ ಉಪಯೋಗಿಸಬಹುದು. ಹೀಗೆ ಮಾಡಿದರೆ ಎಂತಹ ಹಿತವಾಗುವುದು!

ಪ್ರಶ್ನೆ: ಏಕೆ ಮಹಾರಾಜ್, ನಾವು ಇಂಗ್ಲೆಂಡಿಗೆ ಹೋಗುವುದಕ್ಕಿಂತ ಜಪಾನಿಗೆ ಹೋಗುವುದೊಳ್ಳೆಯದೇ?

ಸ್ವಾಮೀಜಿ: ಖಂಡಿತವಾಗಿ! ನನ್ನ ಅಭಿಪ್ರಾಯದಲ್ಲಿ ನಮ್ಮ ದೇಶದ ಎಲ್ಲಾ ಶ‍್ರೀಮಂತರೂ ಬುದ್ಧಿವಂತರೂ ಒಮ್ಮೆ ಜಪಾನಿಗೆ ಹೋಗಿಬಂದರೆ ಅವರೆಲ್ಲರ ಕಣ್ಣುಗಳೂ ತೆರೆಯುವುವು.

ಪ್ರಶ್ನೆ: ಹೇಗೆ?

ಸ್ವಾಮಿಜಿ: ಜಪಾನೀಯರು ಜ್ಞಾನವನ್ನು ಜೀರ್ಣಿಸಿಕೊಂಡಿರುವುದನ್ನು ನೋಡುತ್ತೇವೆ. ನಮ್ಮಲ್ಲಿರುವಂತೆ ಅದರ ಅಜೀರ್ಣತೆಯನ್ನಲ್ಲ. ಅವರು ಐರೋಪ್ಯರಿಂದ ಎಲ್ಲವನ್ನೂ ತೆಗೆದುಕೊಂಡಿದ್ದಾರೆ. ಆದರೆ ಅದೇ ಕಾಲದಲ್ಲಿ ಜಪಾನೀಯರಾಗಿಯೇ ಉಳಿದಿದ್ದಾರೆ, ಐರೋಪ್ಯರಾಗಿಲ್ಲ. ನಮ್ಮ ದೇಶದಲ್ಲಾದರೊ, ಪಾಶ್ಚಾತ್ಯನಾಗಿಬಿಡಬೇಕೆಂಬ ತೀವ್ರ ಹುಚ್ಚು ಪ್ಲೇಗಿನಂತೆ ನಮ್ಮನ್ನು ಹಿಡಿದುಕೊಂಡಿದೆ.

ನಾನು: ಮಹಾರಾಜ್ - ನಾನು ಕೇವಲ ಜಪಾನರ ತೈಲಚಿತ್ರಗಳನ್ನು ನೋಡಿರುವೆ. ಅವರ ಕಲೆಗೆ ಬೆರಗಾಗದೆ ಇರುವವರಾರೂ ಇಲ್ಲ. ಅದನ್ನು ನೋಡಿದಾಗ ತಕ್ಷಣ ಅವರದೇ ಆದದ್ದೇನೋ ಅಲ್ಲಿದೆ ಎಂದೆನ್ನಿಸುವುದು.

ಸ್ವಾಮೀಜಿ: ನಿಜ ನಿಜ. ಅವರ ಕಲೆಯಿಂದಲೇ ಅವರು ಅಷ್ಟೊಂದು ದೊಡ್ಡ ಜನಾಂಗವಾಗಿರುವುದು. ಅವರೂ ನಮ್ಮಂತೆಯೇ ಏಷಿಯಾಖಂಡದವರು ಎಂಬುದು ನಿನಗೆ ಗೊತ್ತಿಲ್ಲವೆ? ನಾವು ಬಹುಮಟ್ಟಿಗೆ ಎಲ್ಲವನ್ನೂ ಕಳೆದುಕೊಂಡಿದ್ದರೂ ಇನ್ನೂ ನಮ್ಮಲ್ಲಿ ಏನುಳಿದಿದೆಯೊ ಅದು ಕೂಡ ಅದ್ಭುತವಾದುದು. ಏಷ್ಯಾಖಂಡದವನ ಜೀವಾತ್ಮವು ಕಲೆಯೊಡನೆ ಹೆಣೆದುಕೊಂಡಿದೆ. ಯಾವ ವಸ್ತುವಿನಲ್ಲಿ ಕಲಾ ಕೌಶಲವಿಲ್ಲವೋ ಅಂತಹ ವಸ್ತುವನ್ನು ಏಷ್ಯಾದವನೆಂದಿಗೂ ಉಪಯೋಗಿಸುವುದೇ ಇಲ್ಲ. ನಮಗೆ ಕಲೆ ಧರ್ಮದ ಒಂದು ಅಂಗವಾಗಿದೆಯೆಂಬುದನ್ನು ನೀನು ತಿಳಿಯೆಯಾ? ಶುಭಕಾರ್ಯಗಳಾಗುವ ಸಂದರ್ಭಗಳಲ್ಲಿ ನೆಲ ಮತ್ತು ಗೋಡೆಗಳನ್ನು ಅಕ್ಕಿಹಿಟ್ಟನ್ನು ನಾದಿ ಅದರಿಂದ ಚಿತ್ರಗಳನ್ನು ಬರೆದು ಸಿಂಗರಿಸುವ ಸ್ತ್ರೀಯನ್ನು ನಾವು ಎಷ್ಟು ಶ್ಲಾಘಿಸುತ್ತೇವೆ? ಶ‍್ರೀರಾಮಕೃಷ್ಣರೆ ಎಂತಹ ದೊಡ್ಡ ಶಿಲ್ಪಿಗಳು!

ಪ್ರಶ್ನೆ: ಆಂಗ್ಲೇಯ ಕಲೆ ಕೂಡ ಚೆನ್ನಾಗಿದೆ. ಅಲ್ಲವೆ?

ಸ್ವಾಮೀಜಿ: ನೀನೆಂತಹ ಹುಚ್ಚ! ಆದರೆ ಈ ರೀತಿಯ ಅಭಿಪ್ರಾಯವೇ ನಮ್ಮಲ್ಲಿ ಎಲ್ಲೆಲ್ಲಿಯೂ ಕಂಡುಬರುವಾಗ ನಿನ್ನನ್ನು ದೂರಿ ಏನು ಪ್ರಯೋಜನ? ಅಯ್ಯೋ! ದೇಶವು ಎಂತಹ ಅವಸ್ಥೆಗಿಳಿದಿದೆ! ಜನರು ತಮ್ಮಲ್ಲಿರುವ ಚಿನ್ನವನ್ನೇ ಹಿತ್ತಾಳೆಯಂತೆ ನೋಡುವರು. ಪರದೇಶಿಯರ ಹಿತ್ತಾಳೆಯೇ ಚಿನ್ನ ಇವರಿಗೆ. ಆಧುನಿಕ ವಿದ್ಯಾಭ್ಯಾಸದಿಂದ ಉದ್ಭವಿಸಿದ ಯಕ್ಷಿಣಿ ಇದು. ಯೂರೋಪಿಯನ್ನರಿಗೆ ಏಷ್ಯಾದ ಸಂಪರ್ಕವಾದಂದಿನಿಂದ ಅವರು ತಮ್ಮ ಜೀವನದಲ್ಲಿಯೂ ಕಲೆಯನ್ನು ಬೆಳೆಸಿಕೊಳ್ಳತೊಡಗಿದ್ದಾರೆಂದು ತಿಳಿ.

ನಾನು: ಮಹಾರಾಜ್, ನೀವು ಹೀಗೆ ಮಾತನಾಡುತ್ತಿರುವುದನ್ನು ಇತರರು ನೋಡಿದರೆ ನಿಮ್ಮನ್ನು ನಿರಾಶಾವಾದಿಗಳೆಂದು ತಿಳಿದುಕೊಳ್ಳುವರು.

ಸ್ವಾಮಿಜಿ: ಸಹಜ! ರೂಢಿಯ ಜಾಡಿನಲ್ಲೇ ಹೋಗುವವರು ಮತ್ತೇನು ತಿಳಿಯಬೇಕು? ಪ್ರತಿಯೊಂದನ್ನೂ ನನ್ನ ಕಣ್ಣುಗಳ ಮೂಲಕ ನಿನಗೆ ತೋರಿಸ ಬಲ್ಲೆನಾದರೆ ಚೆನ್ನಾಗಿತ್ತು ಎನ್ನಿಸುತ್ತಿದೆ! ಅವರ ಕಟ್ಟಡಗಳನ್ನು ನೋಡು, ಅವೆಷ್ಟು ಸಾಧಾರಣ ಭಾವರಹಿತವಾಗಿವೆ! ಆ ಸರಕಾರಿ ಕಟ್ಟಡಗಳನ್ನು ನೋಡು. ಅವುಗಳನ್ನು ಸುಮ್ಮನೇ ಹೊರಗಡೆಯಿಂದ ನೋಡಿದರೆ ಅವುಗಳಲ್ಲಿ ಒಂದೊಂದೂ ಯಾವ ಭಾವವನ್ನು ಸೂಚಿಸುತ್ತವೆಂದೇನಾದರೂ ಹೇಳಬಲ್ಲೆಯಾ? ಇಲ್ಲ. ಏಕೆಂದರೆ ಅವೆಲ್ಲಾ ಸಂಕೇತರಹಿತವಾಗಿವೆ. ಮತ್ತೆ, ಪಾಶ್ಚಾತ್ಯರ ಉಡುಪನ್ನು ನೋಡು, ದೇಹಕ್ಕೆ ಬಹುಮಟ್ಟಿಗೆ ಬಿಗಿಯಾಗಿ ತೊಡಿಸುವ ಬಿಗಿಯಂಗಿ, ನೆಟ್ಟಗಿರುವ ಷರಾಯಿ ಇವೆಲ್ಲಾ ನಮ್ಮ ಅಭಿಪ್ರಾಯದ ಪ್ರಕಾರ ಕೊಂಚವೂ ಮರ್ಯಾದೆಯ ಚಿಹ್ನೆಯಲ್ಲ, ಅಲ್ಲವೆ? ಓ! ಅದರಲ್ಲಿ ಎಂತಹ ಸೌಂದರ್ಯ? ಈಗ ನಮ್ಮ ತಾಯ್ನಾಡಿನ ಮೂಲೆ ಮೂಲೆಗೂ ಒಮ್ಮೆ ಹೋಗಿ ನೋಡು, ನಮ್ಮ ಕಟ್ಟಡಗಳನ್ನು ನೋಡಿದ ಕೂಡಲೆ ಅವುಗಳ ಭಾವ ಮತ್ತು ಅದರಲ್ಲಿರುವ ಕಲೆ ಸ್ಪಷ್ಟವಾಗಿ ಕಾಣುವುದಿಲ್ಲವೇ? ಅವರು ನೀರು ಕುಡಿಯಲು ಉಪಯೋಗಿಸುವ ಪಾತ್ರೆ ಗಾಜಿನದು. ನಮ್ಮದಾದರೋ ಲೋಹದ ಹೂಜಿ. ಇವೆರಡರಲ್ಲಿ ಯಾವುದು ಕಲಾಕುಶಲತೆಯುಳ್ಳದ್ದು? ನಮ್ಮ ಹಳ್ಳಿಗಳಲ್ಲಿರುವ ರೈತರ ಮನೆಗಳನ್ನು ನೋಡಿರುವೆಯಾ?

ನಾನು: ಹೌದು ನೋಡಿದ್ದೇನೆ.

ಸ್ವಾಮೀಜಿ: ಅವುಗಳಲ್ಲಿ ನೀನು ಏನನ್ನು ನೋಡಿದೆ?

ನನಗೆ ಏನು ಹೇಳಬೇಕೆಂದು ತೋಚದಾಯಿತು. ಆದಾಗ್ಗೂ ನಾನು ಮಹಾರಾಜ್, ಅಂಗಳ ಮತ್ತು ನೆಲವನ್ನು ನಿತ್ಯವೂ ಸಾರಿಸುವುದರಿಂದ ಅವು ಯಾವ ನ್ಯೂನತೆಯೂ ಇಲ್ಲದೆ ಚೊಕ್ಕಟವಾಗಿವೆ ಎಂದೆ.

ಸ್ವಾಮೀಜಿ: ನೀನು ಅವರು ಬತ್ತವನ್ನು ಇಡುವ ಕಣಜವನ್ನು ನೋಡಿರುವೆಯಾ? ಅದರಲ್ಲಿ ಎಂತಹ ಕಲೆ ಇದೆ! ಅವರ ಮಣ್ಣಿನ ಗೋಡೆಗಳ ಮೇಲೂ ಕೂಡಾ ಎಂತಹ ವಿವಿಧ ತೈಲಚಿತ್ರಗಳಿವೆ! ನೀನು ಪಶ್ಚಿಮದ ಕಡೆ ಹೋಗಿ ಕೆಳವರ್ಗದವರು ಹೇಗೆ ಜೀವಿಸುವರೆಂಬುದನ್ನು ನೋಡು. ನಿನಗೆ ಕ್ಷಣದಲ್ಲಿಯೇ ವ್ಯತ್ಯಾಸ ಗೊತ್ತಾಗುವುದು. ಪಾಶ್ಚಾತ್ಯರು ಎಲ್ಲವನ್ನೂ ಉಪಯೋಗದ ದೃಷ್ಟಿಯಿಂದ ನೋಡುವರು. ಅವರ ಧ್ಯೇಯ ಅದು. ನಾವು ಎಲ್ಲೆಲ್ಲೂ ಕಲೆಯನ್ನು ಹುಡುಕುವೆವು. ಪಾಶ್ಚಾತ್ಯ ವಿದ್ಯಾಭ್ಯಾಸದಿಂದ ನಮ್ಮ ಸುಂದರ ಹೂಜಿಗಳು ತ್ಯಜಿಸಲ್ಪಟ್ಟು ಆ ಸ್ಥಳವನ್ನು ಗಾಜಿನ ಲೋಟಾಗಳು ಆಕ್ರಮಿಸಿವೆ. ಹೀಗೆ ಉಪಯುಕ್ತತೆಯ ಭಾವನೆ ನಮ್ಮನ್ನು ಎಷ್ಟರಮಟ್ಟಿಗೆ ವ್ಯಾಪಿಸಿದೆ ಎಂದರೆ ಅದು ಕುಚೋದ್ಯವಾಗಿ ಪರಿಣಮಿಸಿದೆ. ನಮಗೆ ಈಗ ಬೇಕಾಗಿರುವುದು ಕಲೆ ಮತ್ತು ಉಪಯುಕ್ತತೆ ಇವುಗಳ ಸಂಯೋಗ, ಜಪಾನು ಅದನ್ನು ಬಹುಬೇಗ ಮಾಡಿದೆ. ಅದಕ್ಕೇ ಅದು ಅಷ್ಟು ಪ್ರಚಂಡ ವೇಗದಲ್ಲಿ ಮುಂದುವರಿಯಿತು. ಈಗ ಪಾಶ್ಚಾತ್ಯರಿಗೆ ಕಲಿಸುವ ಸರದಿ ಜಪಾನೀಯರದು.

ಪ್ರಶ್ನೆ: ಮಹಾರಾಜ್, ಪ್ರಪಂಚದಲ್ಲಿ ಯಾವ ಜನಾಂಗ ತುಂಬಾ ಅಂದವಾಗಿ ಉಡುಪನ್ನು ಧರಿಸುವುದು?

ಸ್ವಾಮೀಜಿ: ಆರ್ಯರು, ಐರೋಪ್ಯರೂ ಕೂಡ ಇದನ್ನು ಒಪ್ಪಿಕೊಳ್ಳುವರು. ಅವರ ಉಡುಪು ಹೇಗೆ ಆಕರ್ಷಕ ಚಿತ್ರದಂತೆ ಸುರುಳಿ ಸುತ್ತಿಕೊಂಡಿರುವುದು. ಮುಕ್ಕಾಲುಪಾಲು ಜನಾಂಗಗಳ ರಾಜಯೋಗ್ಯವಾದ ಉಡುಪುಗಳು ಆರ್ಯರನ್ನು ಅನುಕರಣೆ ಮಾಡಿವೆ - ಅದೇ ರೀತಿ ಮಡಿಕೆ ಬರುವಂತೆ ಪ್ರಯತ್ನಿಸಿದ್ದಾರೆ. ಆ ಉಡುಪುಗಳು ತಮ್ಮ ಜನಾಂಗದ ಉಡುಪಿಗಿಂತ ಬೇರೆಯಾಗಿರುವುದು ಸ್ಪಷ್ಟವಾಗಿ ಕಂಡುಬರುತ್ತದೆ.

ಅದಿರಲಿ ಸಿಂಗ್ ಜಿ, ಆ ಐರೋಪ್ಯ ಶರಟುಗಳನ್ನು ಧರಿಸುವ ದರಿದ್ರ ಪದ್ಧತಿಯನ್ನು ಬಿಟ್ಟುಬಿಡು.

ಪ್ರಶ್ನೆ: ಏಕೆ ಮಹಾರಾಜ್?

ಸ್ವಾಮೀಜಿ: ಏಕೆಂದರೆ ಪಾಶ್ಚಾತ್ಯರು ಅವನ್ನು ಒಳ ಉಡುಪಿಗೆ ಉಪಯೋಗಿಸುವರು. ಅವುಗಳನ್ನು ಹೊರಗೆ ಹಾಕಿಕೊಳ್ಳುವುದನ್ನು ಸ್ವಲ್ಪವೂ ಇಚ್ಚಿಸುವುದಿಲ್ಲ. ಬಂಗಾಳಿಗಳು ಹೀಗೆ ಮಾಡುವುದು ಎಂತಹ ತಪ್ಪು! ಯಾರು ಯಾವುದನ್ನು ಬೇಕಾದರೂ ಹಾಕಿಕೊಳ್ಳಬಹುದು. ಉಡುಪಿಗೆ ಲಿಖಿತ ನಿಯಮವೇನೂ ಇಲ್ಲ, ಅನುಸರಿಸಲು ಪೂರ್ವಿಕರ ಶೈಲಿಯೂ ಇಲ್ಲವೇನೋ ಎಂಬಂತೆ! ನಮ್ಮ ಜನರಿಗೆ ಕೀಳು ಜಾತಿಯ ಜನರು ಮುಟ್ಟಿದ ಆಹಾರ ಸೇವಿಸಿದರೆ ಜಾತಿ ಕೆಟ್ಟುಹೋಗುವುದು. ಇದೇ ನಿಯಮವನ್ನೇ ಉಡುಪನ್ನು ಧರಿಸಿದಾಗಲೂ ಪರಿಪಾಲಿಸಿದರೆ ಸರಿ, ನಮ್ಮ ಮಾದರಿಯಂತೆಯೇ ಯಾವುದಾದರೊಂದು ರೀತಿಯಲ್ಲಿ ನಿನ್ನ ಉಡುಪನ್ನು ಮಾರ್ಪಡಿಸಿಕೊಳ್ಳುವುದಿಲ್ಲವೇಕೆ? ಐರೋಪ್ಯರ ಶರಟು, ಕೋಟುಗಳ ಮಾದರಿಯಲ್ಲಿ ಯಾವ ಪುರುಷಾರ್ಥವಿದೆ?

ಇಷ್ಟು ಹೊತ್ತಿಗೆ ಮಳೆ ಬರಲಾರಂಭವಾಯಿತು. ಊಟದ ಗಂಟೆಯೂ ಬಾರಿಸಿತು. ನಾವು ಇತರರೊಡನೆ ಪ್ರಸಾದ ಸ್ವೀಕಾರ ಮಾಡಲು ಹೋದೆವು. ಊಟ ಮಾಡುತ್ತಿರುವಾಗ ಮಧ್ಯೆ ಸ್ವಾಮಿಗಳು “ಬಲವರ್ಧಕವಾದ ಆಹಾರವನ್ನು ಸೇವಿಸಬೇಕು. ಹೊಟ್ಟೆಯನ್ನು ಬರೇ ಅನ್ನ ಊಟ ಮಾಡಿ ತುಂಬಿಸಿಕೊಳ್ಳುವುದು ಸೋಮಾರಿತನಕ್ಕೆ ಮೂಲ” ಎಂದರು. ಕೊಂಚ ಹೊತ್ತಿನ ಮೇಲೆ ಪುನಃ “ಜಪಾನೀಯರನ್ನು ನೋಡು, ಬೇಳೆ ಸಾರಿನೊಡನೆ ಅವರು ಅನ್ನವನ್ನು ದಿನಕ್ಕೆ ಎರಡು ಅಥವಾ ಮೂರು ಬಾರಿ ತೆಗೆದುಕೊಳ್ಳುವರು. ಆದರೆ ಅವರು! ಎಂಥ ಬಲಿಷ್ಠರೂ ಕೂಡ. ಹೆಚ್ಚು ಸಲ ಊಟ ಮಾಡಿದರೂ ಒಂದೊಂದು ಬಾರಿಯೂ ಕಡಿಮೆ ಪ್ರಮಾಣದಲ್ಲಿ ತೆಗೆದುಕೊಳ್ಳುವರು. ಅವರಲ್ಲಿ ಶ‍್ರೀಮಂತರು ನಿತ್ಯವೂ ಮಾಂಸ ಸೇವಿಸುವರು. ನಾವು ದಿನಕ್ಕೆರಡು ಬಾರಿ ಮೂಗಿಗೆ ಬರುವವರೆಗೂ ಊಟ ಮಾಡುವೆವು. ಅಷ್ಟೊಂದು ಅನ್ನವನ್ನು ಜೀರ್ಣಿಸಿಕೊಳ್ಳುವ ಹೊತ್ತಿಗೇ ನಮ್ಮಲ್ಲಿ ಶಕ್ತಿಯೆಲ್ಲಾ ಕರಗಿ ಹೋಗುವುದು.”

ಪ್ರಶ್ನೆ: ಬಡವರಾದ ನಾವು, ಬಂಗಾಳಿಗಳು, ಮಾಂಸ ತಿನ್ನುವುದು ಸಾಧ್ಯವೆ?

ಸ್ವಾಮೀಜಿ: ಏಕೆ ಸಾಧ್ಯವಿಲ್ಲ? ನೀವು ಕೊಂಚ ಕೊಂಚ ಖಂಡಿತವಾಗಿಯೂ ತಿನ್ನಲು ಸಾಧ್ಯ. ದಿನಕ್ಕೆ ಅರ್ಧಪೌಂಡು ಬೇಕಾದಷ್ಟು ಆಯಿತು. ನಿಜವಾದ ದೋಷವೇ ಸೋಮಾರಿತನ, ಇದೇ ನಮ್ಮ ಬಡತನಕ್ಕೆ ಮೂಲ. ಒಬ್ಬ ವಾಣಿಜ್ಯ ಸಂಸ್ಥೆಯ ಮೇಲಧಿಕಾರಿ ಯಾರೊಡನೆಯೊ ಅಸಮಾಧಾನಗೊಂಡು ಅವನ ಸಂಬಳವನ್ನು ಕಡಿಮೆ ಮಾಡಿದ ಎಂದಿಟ್ಟುಕೊಳ್ಳೋಣ, ಅಥವಾ ಒಂದು ಸಂಸಾರದಲ್ಲಿ ಸಂಪಾದಿಸುತ್ತಿದ್ದ ಮೂರು ನಾಲ್ಕು ಮಂದಿ ಮಕ್ಕಳಲ್ಲಿ ಒಬ್ಬ ಸತ್ತುಹೋದ ಎಂದಿಟ್ಟುಕೊಳ್ಳೋಣ. ಅವರೇನು ಮಾಡುವರು? ಅವರು ತಕ್ಷಣ ಮಕ್ಕಳಿಗೆ ಕೊಡುತ್ತಿದ್ದ ಹಾಲನ್ನು ಕೊಂಚ ಕಡಿಮೆ ಮಾಡುವರು. ಅಥವಾ, ಒಂದು ಹೊತ್ತು ಊಟ ಮಾಡಿ ರಾತ್ರಿ ಕೊಂಚ ಅರಳು ಹಿಟ್ಟನ್ನು ತಿಂದು ಮಲಗುವರು.

ಪ್ರಶ್ನೆ: ಈ ಸಂದರ್ಭದಲ್ಲಿ ಅವರು ಇನ್ನೇನು ಮಾಡಲು ಸಾಧ್ಯ?

ಸ್ವಾಮೀಜಿ: ಏಕೆ? ಅವರು ಇನ್ನೂ ಶ್ರಮಪಟ್ಟು, ತಮ್ಮ ಅಂತಸ್ತಿಗೆ ತಕ್ಕಂತೆ ಆಹಾರವನ್ನು ಸಂಪಾದಿಸಲಾರರೆ? ಹರಟೆ ಹೊಡೆಯುವ ಸ್ಥಳಕ್ಕೆ ಹೋಗಲೇಬೇಕು, ಸೋಮಾರಿಯಾಗಿ ಕಾಲ ಕಳೆಯಲೇಬೇಕು!ಓ! ತಾವೆಷ್ಟು ಕಾಲ ವ್ಯರ್ಥಮಾಡುತ್ತಿರುವೆವೆಂಬುದು ಅವರಿಗೆ ಗೊತ್ತಾಗಿದ್ದಿದ್ದರೆ!

\newpage

\chapter[ಅಧ್ಯಾಯ ೫]{ಅಧ್ಯಾಯ ೫\protect\footnote{\engfoot{C.W, Vol. V, P. 376}}}

ಒಮ್ಮೆ ಸ್ವಾಮೀಜಿ ಕಲ್ಕತ್ತೆಯಲ್ಲಿ ಬಲರಾಮಬಾಬುಗಳ ಮನೆಯಲ್ಲಿ ಇದ್ದಾಗ ನಾನು ಅವರನ್ನು ಕಾಣಲು ಹೋದೆ. ಜಪಾನ್ ಮತ್ತು ಅಮೆರಿಕಾ ದೇಶಗಳ ಮೇಲೆ ದೀರ್ಘ ಸಂಭಾಷಣೆಯಾದ ನಂತರ ನಾನು ಅವರನ್ನು ಸ್ವಾಮೀಜಿ, ನಿಮಗೆ ಪಾಶ್ಚಾತ್ಯರಲ್ಲಿ ಎಷ್ಟು ಮಂದಿ ಶಿಷ್ಯರಿದ್ದಾರೆ? ಎಂದು ಕೇಳಿದೆ.

ಸ್ವಾಮೀಜಿ: ಅನೇಕರು.

ಪ್ರಶ್ನೆ: ಎರಡು ಅಥವಾ ಮೂರು ಸಾವಿರವೆ?

ಸ್ವಾಮೀಜಿ: ಬಹುಶಃ ಅದಕ್ಕಿಂತ ಹೆಚ್ಚು ಇರಬಹುದು.

ಪ್ರಶ್ನೆ: ಅವರೆಲ್ಲಾ ನಿಮ್ಮಿಂದ ಮಂತ್ರದೀಕ್ಷೆ ಪಡೆದಿದ್ದಾರೆಯೇ?

ಸ್ವಾಮೀಜಿ: ಹೌದು.

ಪ್ರಶ್ನೆ: ನೀವು ಹೇಗೆ ಕೊಟ್ಟಿರಿ ಮಹಾರಾಜ್? ಶಾಸ್ತ್ರಗಳಲ್ಲಿ ಬ್ರಾಹ್ಮಣರ ಹೊರತು ಮತ್ತಾವ ಶೂದ್ರರಿಗೂ ಪ್ರಣವ ಮಂತ್ರೋಚ್ಚಾರಣೆಗೆ ಅಧಿಕಾರವಿಲ್ಲ ಎಂದು ಹೇಳಿದೆ. ಅಲ್ಲದೆ ಪಾಶ್ಚಾತ್ಯರು ಶೂದ್ರರು ಕೂಡ ಅಲ್ಲ, ಮ್ಲೇಚ್ಛರು.

ಸ್ವಾಮೀಜಿ: ನಾನು ಮಂತ್ರದೀಕ್ಷೆ ಕೊಟ್ಟಿರುವವರು ಬ್ರಾಹ್ಮಣರಲ್ಲವೆಂಬುದು ನಿನಗೆ ಹೇಗೆ ಗೊತ್ತು?

ಪ್ರಶ್ನೆ: ಭರತಖಂಡದ ಹೊರಗಡೆ ಯವನರ, ಮ್ಲೇಚ್ಛರ ನಾಡಿನಲ್ಲಿ ನಿಮಗೆ ಬ್ರಾಹ್ಮಣರು ಹೇಗೆ ಸಿಗುತ್ತಾರೆ?

ಸ್ವಾಮೀಜಿ: ನನ್ನ ಶಿಷ್ಯರೆಲ್ಲಾ ಬ್ರಾಹ್ಮಣರು! ಬ್ರಾಹ್ಮಣರ ಹೊರತು ಮತ್ತಾರಿಗೂ ಪ್ರಣವಕ್ಕೆ ಹಕ್ಕಿಲ್ಲವೆಂಬುದನ್ನು ನಾನು ಖಂಡಿತ ಒಪ್ಪುತ್ತೇನೆ. ಆದರೆ ಬ್ರಾಹ್ಮಣನ ಮಗ ಯಾವಾಗಲೂ ಬ್ರಾಹ್ಮಣನೇ ಆಗಿರಬೇಕಾದ ಆವಶ್ಯಕತೆಯಿಲ್ಲ. ಅವನು ಹಾಗೆ ಆಗುವುದು ಬಹುಮಟ್ಟಿಗೆ ಶಕ್ಯವಿದ್ದರೂ, ಅವನು ಬ್ರಾಹ್ಮಣನಾಗದೆ ಇರಬಹುದು. ಬಾಗ್‌ಬಜಾರಿನ ಅಘೋರ ಚಕ್ರವರ್ತಿಗಳ ಸೋದರಳಿಯ ಒಬ್ಬ ಜಾಡಮಾಲಿಯಾದನೆಂಬುದನ್ನೂ, ಅವನು ಅಂಗೀಕರಿಸಿದ ಜಾತಿ ಪ್ರಕಾರ ಎಲ್ಲಾ ಕೀಳ್ಗೆಲಸಗಳನ್ನು ಪ್ರತ್ಯಕ್ಷವಾಗಿ ಮಾಡುತ್ತಿದ್ದನೆಂಬುದನ್ನೂ ನೀನು ಕೇಳಿಲ್ಲವೆ? ಅವನು ಬ್ರಾಹ್ಮಣನ ಮಗನಲ್ಲವೇನು?

ಬ್ರಾಹ್ಮಣ ಜಾತಿ ಮತ್ತು ಬ್ರಾಹ್ಮಣ ಗುಣಗಳು ಎರಡೂ ಭಿನ್ನವಾದ ವಿಷಯಗಳು. ಭರತಖಂಡದಲ್ಲಿ ಜಾತಿಯಿಂದ ಬ್ರಾಹ್ಮಣರೆಂದು ನಿಷ್ಕರ್ಷಿಸುವರು. ಪಶ್ಚಿಮದಲ್ಲಿ ಬ್ರಾಹ್ಮಣ್ಯ ಗುಣಗಳಿಂದ ಬ್ರಾಹ್ಮಣರೆಂದು ಗುರುತಿಸಬೇಕು. ಸತ್ತ್ವರಜಸ್ಸುತಮಸ್ಸುಗಳೆಂಬ ಮೂರು ಗುಣಗಳಿರುವಂತೆಯೇ ಬ್ರಾಹ್ಮಣ ಕ್ಷತ್ರಿಯ ವೈಶ್ಯ ಶೂದ್ರರೆಂದು ತೋರಿಸಲು ಗುಣಗಳಿವೆ. ಬ್ರಾಹ್ಮಣ ಅಥವಾ ಕ್ಷತ್ರಿಯರಿಗೆ ಯೋಗ್ಯವಾದ ಗುಣಗಳು ದೇಶದಿಂದ ಕಣ್ಮರೆಯಾಗುತ್ತಲಿವೆ. ಆದರೆ ಪಶ್ಚಿಮದಲ್ಲಿ ಅವರು ಕ್ಷತ್ರಿಯ ಪದವಿ ಪಡೆದಿದ್ದಾರೆ. ಅವರ ಮುಂದಿನ ಮೆಟ್ಟಲೇ ಬ್ರಾಹ್ಮಣ ಪದವಿ. ಆ ಪದವಿಗೆ ಅರ್ಹತೆ ಪಡೆದಿರುವವರೂ ಅಲ್ಲಿ ಅನೇಕರಿದ್ದಾರೆ.

ಪ್ರಶ್ನೆ: ಹಾಗಾದರೆ ಯಾರು ಸಾತ್ತ್ವಿಕ ಸ್ವಭಾವವುಳ್ಳವರೋ ಅವರನ್ನು ಬ್ರಾಹ್ಮಣರೆಂದು ನೀವು ಕರೆಯುವಿರಿ.

ಸ್ವಾಮೀಜಿ: ಹಾಗೆಯೇ ಸರಿ. ಸಾತ್ತ್ವಿಕ ರಾಜಸ ತಾಮಸ ಗುಣಗಳಲ್ಲಿ ಯಾವುದಾದರೊಂದು ಕೊಂಚ ಹೆಚ್ಚಾಗಿ ಅಥವಾ ಕಡಿಮೆಯಾಗಿ ಪ್ರತಿಯೊಬ್ಬ ಮನುಷ್ಯನಲ್ಲಿಯೂ ಇರುವ ಹಾಗೆ ಬ್ರಾಹ್ಮಣ, ಕ್ಷತ್ರಿಯ, ವೈಶ್ಯ, ಶೂದ್ರ ಈ ಗುಣಗಳೂ ಕೂಡ ಪ್ರತಿಯೊಬ್ಬ ಮನುಷ್ಯನಲ್ಲಿಯೂ ಹೆಚ್ಚು ಕಡಿಮೆ ಪ್ರಮಾಣದಲ್ಲಿ ಅಂತರ್ಗತವಾಗಿರುವುವು. ಆದರೆ ಆಗಾಗ ಅವನಲ್ಲಿ ಈ ಗುಣಗಳಲ್ಲಿ ಯಾವುದಾದರೊಂದು ಬೇರೆ ಬೇರೆ ಪ್ರಮಾಣದಲ್ಲಿ ಪ್ರಬಲವಾಗಿರುವುದು ಮತ್ತು ಅದರಂತೆಯೇ ಆವಿರ್ಭಾವ ಹೊಂದುವುದು. ಒಬ್ಬ ಮನುಷ್ಯನ ವಿವಿಧ ಹವ್ಯಾಸಗಳನ್ನು ತೆಗೆದುಕೊಳ್ಳಿ. ಉದಾಹರಣೆಗೆ ಅವನು ಮತ್ತೊಬ್ಬನಲ್ಲಿ ಸಂಬಳಕ್ಕಾಗಿ ದುಡಿಯುತ್ತಿರುವುದು. ಅವನು ಈಗ ಶೂದ್ರ ಪದವಿಯಲ್ಲಿರುವನು. ಯಾರು ತನ್ನ ಸ್ವಂತ ಲಾಭಕ್ಕಾಗಿ ವ್ಯಾಪಾರದ ವ್ಯವಹಾರದಲ್ಲಿ ತೊಡಗಿದ್ದಾನೋ ಅವನು ವೈಶ್ಯ, ತಪ್ಪನ್ನು ಸರಿಪಡಿಸಲು ಯಾರು ಯುದ್ಧ ಮಾಡುವನೋ ಅವನಲ್ಲಿ ಕ್ಷತ್ರಿಯ ಗುಣಗಳು ಹೊರಬೀಳುವುವು. ಯಾವಾಗ ಅವನು ದೇವರಲ್ಲಿ ಧ್ಯಾನಾಸಕ್ತನಾಗುವನೋ ಅಥವಾ ದೇವರ ವಿಷಯಕ್ಕೆ ಸಂಬಂಧಪಟ್ಟ ಸಂಭಾಷಣೆಯಲ್ಲಿ ಕಾಲಕಳೆಯುವನೋ ಆಗ ಅವನು ಬ್ರಾಹ್ಮಣ. ಸಹಜವಾಗಿ ಒಂದು ಜಾತಿಯಿಂದ ಮತ್ತೊಂದು ಜಾತಿಗೆ ಬದಲಾಯಿಸುವುದು ಖಂಡಿತ ಶಕ್ಯ. ಇಲ್ಲದಿದ್ದಲ್ಲಿ ವಿಶ್ವಾಮಿತ್ರ ಹೇಗೆ ಬ್ರಾಹ್ಮಣನಾಗುತ್ತಿದ್ದ? ಪರಶುರಾಮ ಹೇಗೆ ಕ್ಷತ್ರಿಯನಾಗುತ್ತಿದ್ದ?

ಪ್ರಶ್ನೆ: ನೀವು ಹೇಳುವುದು ಯಥಾರ್ಥವೆನ್ನಿಸುವುದು. ಆದರೆ ನಮ್ಮ ಪಂಡಿತರು, ಕುಲಗುರುಗಳೇಕೆ ನಮಗೆ ಇವನ್ನೇ ಬೋಧಿಸುವುದಿಲ್ಲ?

ಸ್ವಾಮಿಜಿ: ನಮ್ಮ ದೇಶಕ್ಕುಂಟಾಗಿರುವ ಮಹಾ ಕೆಡುಕುಗಳಲ್ಲಿ ಇದೂ ಒಂದು. ಸದ್ಯಕ್ಕೆ ಈ ವಿಷಯವನ್ನು ಇಲ್ಲಿಗೇ ನಿಲ್ಲಿಸೋಣ.

ಇಲ್ಲಿ ಸ್ವಾಮೀಜಿ ಪಾಶ್ಚಾತ್ಯರ ಕಾರ್ಯಪ್ರವೃತ್ತಿಯ ಸ್ಫೂರ್ತಿಯನ್ನು ಅತಿಶಯವಾಗಿ ಬಣ್ಣಿಸಿ ಅವರು ಧರ್ಮವನ್ನು ತೆಗೆದುಕೊಂಡಾಗ ಅದೇ ಸ್ಫೂರ್ತಿಯನ್ನು ಹೇಗೆ ತೋರುವರೆಂಬುದನ್ನೂ ವಿವರಿಸಿದರು.

ನಾನು: ನಿಜ, ಮಹಾರಾಜ್, ಅವರು ಧರ್ಮವನ್ನು ಸಾಧನೆ ಮಾಡುವಾಗ ಬಹುಬೇಗ ಅವರ ಆಧ್ಯಾತ್ಮಿಕ ಮತ್ತು ಮಾನಸಿಕ ಶಕ್ತಿ ವಿಕಾಸಗೊಳ್ಳುವುದೆಂದು ಕೇಳಿದ್ದೇನೆ. ಅಂದು ಸ್ವಾಮಿ ಶಾರದಾನಂದರು ತಮ್ಮ ಪಾಶ್ಚಾತ್ಯ ಶಿಷ್ಯನೊಬ್ಬನು ಬರೆದಿದ್ದ ಪತ್ರವೊಂದನ್ನು ತೋರಿಸಿದರು. ಕೇವಲ ನಾಲ್ಕು ತಿಂಗಳಲ್ಲೇ ಆ ಬರಹಗಾರನು ಸಾಧನೆ ಮಾಡಿ ತನ್ನಲ್ಲುಂಟಾದ ವಿಶೇಷ ಆಧ್ಯಾತ್ಮಿಕ ಬೆಳವಣಿಗೆಗಳನ್ನು ವಿವರಿಸಿದ್ದನು.

ಸ್ವಾಮೀಜಿ: ಆದ್ದರಿಂದ ನೀನೇ ನೋಡು! ಪಶ್ಚಿಮದಲ್ಲಿ ಬ್ರಾಹ್ಮಣರಿದ್ದಾರೆಯೊ ಇಲ್ಲವೊ ಎಂದು ನಿನಗೆ ಈಗ ಅರ್ಥವಾಗಲಿದೆ. ನಮ್ಮಲ್ಲೂ ಬ್ರಾಹ್ಮಣರಿದ್ದಾರೆ. ಅವರು ತಮ್ಮ ತೀವ್ರ ದಬ್ಬಾಳಿಕೆಯಿಂದ ದೇಶವನ್ನು ತೀರ ಅಧೋಗತಿಗೆ ಇಳಿಸುವ ಪ್ರಯತ್ನ ಮಾಡುತ್ತಿದ್ದಾರೆ. ಸಹಜವಾಗಿ ಅವರಲ್ಲಿದ್ದುದೂ ಕ್ರಮೇಣ ಮಾಯವಾಗುತ್ತಿದೆ. ಗುರು ಶಿಷ್ಯನಿಗೆ ಮಂತ್ರದೀಕ್ಷೆ ಕೊಡುತ್ತಾನೆ. ಆದರೆ ಅವನಿಗೆ ಅದೇ ಒಂದು ವ್ಯಾಪಾರವಾಗಿ ಹೋಗಿದೆ. ಮತ್ತು ಈಗಿನ ಕಾಲದಲ್ಲಿ ಗುರು ಶಿಷ್ಯರ ಮಧ್ಯೆ ಇರುವ ಸಂಬಂಧ ಎಷ್ಟು ವಿಸ್ಮಯಕರವಾಗಿದೆ! ಪ್ರಾಯಶಃ ಗುರುವಿಗೆ ಮನೆಯಲ್ಲಿ ತಿನ್ನಲು ಏನೇನೂ ಇರುವುದಿಲ್ಲ. ಹೆಂಡತಿ ಇದನ್ನು ಅವನ ಲಕ್ಷ್ಯಕ್ಕೆ ತಂದು ದಯಮಾಡಿ, ನಿಮ್ಮ ಶಿಷ್ಯರ ಬಳಿಗೆ ಮತ್ತೊಮ್ಮೆ ಹೋಗಿ, ನೀವು ದಿನವೆಲ್ಲ ಪಗಡೆಯಾಡುತ್ತಿದ್ದರೆ ನಮ್ಮ ಹಸಿವು ಇಂಗುವುದೇನು? ಎನ್ನುವಳು. ಬ್ರಾಹ್ಮಣ ಉತ್ತರ ಕೊಡುತ್ತಾ ಹೇಳುವನು: ಸರಿ. ನಾಳೆ ಬೆಳಿಗ್ಗೆ ನನಗೆ ಜ್ಞಾಪಿಸು. ನನ್ನ ಶಿಷ್ಯನೊಬ್ಬನಿಗೆ ಅದೃಷ್ಟ ಕುಲಾಯಿಸಿದೆಯೆಂದು ಕೇಳಿದೆ. ಅದೂ ಅಲ್ಲದೆ ನಾನವನ ಬಳಿಗೆ ಬಹುಕಾಲದಿಂದಲೂ ಹೋಗಿಯೇ ಇಲ್ಲ. ಇದೇ ಬಂಗಾಳದ ಕುಲ ಗುರುಗಳಿಗೆ ಪ್ರಾಪ್ತವಾಗಿರುವ ದುರವಸ್ಥೆ! ಪಶ್ಚಿಮದಲ್ಲಿನ ಪೌರೋಹಿತ್ಯ ಇನ್ನೂ ಇಷ್ಟೊಂದು ಹೀನ ಸ್ಥಿತಿಗಿಳಿದಿಲ್ಲ. ಒಟ್ಟಿನಲ್ಲಿ ಅದು ನಿಮ್ಮ ಜಾತಿಗಿಂತ ಶ್ರೇಷ್ಠವಾದ ಸ್ಥಿತಿಯಲ್ಲಿದೆ.

\newpage

\chapter[ಅಧ್ಯಾಯ ೬]{ಅಧ್ಯಾಯ ೬\protect\footnote{\engfoot{C.W, Vol. VII, P. 268}}}

ನಾವು ಸಂಭಾಷಣೆ ಮಾಡುವಾಗ ಅಥವಾ ಸಂಗೀತ ಮೊದಲಾದ ಸಾಮೂಹಿಕ ಕ್ರಿಯೆಗಳಲ್ಲಿ ಭಾಗವಹಿಸುವಾಗ ಕೊಂಚವೂ ಹತೋಟಿಯಿಲ್ಲದೆ ಇರುತ್ತೇವೆ. ಪ್ರತಿಯೊಬ್ಬರೂ ತಾವು ಅಗ್ರಗಣ್ಯರೆಂದು ತೋರ್ಪಡಿಸಿಕೊಳ್ಳಲು ಯತ್ನಿಸುತ್ತಾರೆ. ಬೇಲೂರು ಮಠದಲ್ಲಿ ಒಮ್ಮೆ ಈ ವಿಷಯವಾಗಿ ಸ್ವಾಮೀಜಿಯ ಸ್ನೇಹಿತನೊಬ್ಬನು ಅವರೊಡನೆ ಮಾತನಾಡಿದನು. ಸ್ವಾಮೀಜಿ ಹೇಳಿದ್ದೇನೆಂದರೆ: ನೋಡಿ ನಮ್ಮಲ್ಲಿ ಒಂದು ಗಾದೆಯಿದೆ - ನಿನ್ನ ಮಗನಿಗೆ ವ್ಯಾಸಂಗದಲ್ಲಿ ಅಭಿರುಚಿ ಇಲ್ಲದಿದ್ದಲ್ಲಿ ಅವನನ್ನು (ಸಭೆ) ದರ್ಬಾರಿನಲ್ಲಿಡು ಎಂದು. ಇಲ್ಲಿ ಸಭೆ ಎಂದರೆ ಕೆಲವು ಸಂದರ್ಭಗಳಲ್ಲಿ ನಮ್ಮ ಮನೆಗಳಲ್ಲಿ ಸೇರುವ ಸಮಾಜದ ಸಭೆಯಲ್ಲ. ದೊರೆಗಳ ದರ್ಬಾರು. ಬಂಗಾಳದ ನಿರಂಕುಶಪ್ರಭುಗಳ ಕಾಲದಲ್ಲಿ ಬೆಳಿಗ್ಗೆ ಮತ್ತು ಸಂಜೆ ಎರಡು ಹೊತ್ತೂ ರಾಜಸಭೆ ಸೇರುತ್ತಿತ್ತು. ಅಲ್ಲಿ ದೇಶದ ಎಲ್ಲಾ ಸ್ಥಿತಿಗಳೂ ಬೆಳಿಗ್ಗೆ ಚರ್ಚಿಸಲ್ಪಡುತ್ತಿದ್ದವು. ಆಗಿನ ಕಾಲದಲ್ಲಿ ವರ್ತಮಾನ ಪತ್ರಿಕೆಗಳಿಲ್ಲದ ಕಾರಣ ರಾಜನಾದವನು ರಾಜಧಾನಿಯ ಪ್ರಮುಖ ಸಭ್ಯರೊಡನೆ ಜನರ ಮತ್ತು ರಾಜ್ಯದ ವಿಷಯವಾಗಿ ಸಂಭಾಷಿಸುತ್ತಿದ್ದನು. ಈ ಸಭ್ಯ ಮನುಷ್ಯರು ಆ ಸಭೆಗಳಿಗೆ ಹೋಗಲೇಬೇಕಾಗಿತ್ತು. ಅವರು ಹಾಗೆ ಮಾಡದಿದ್ದಲ್ಲಿ ರಾಜನು ಅವರ ಗೈರುಹಾಜರಿಯನ್ನು ಪ್ರಶ್ನಿಸುತ್ತಿದ್ದನು. ಇಂತಹ ದರ್ಬಾರುಗಳು ನಮ್ಮ ದೇಶದಲ್ಲಿ ಮಾತ್ರವೇ ಅಲ್ಲ, ಎಲ್ಲಾ ದೇಶಗಳಲ್ಲಿಯೂ ನಾಗರಿಕತೆಯ ಕೇಂದ್ರವಾಗಿದ್ದುವು. ಈಗಿನ ಕಾಲದಲ್ಲಿ ಪಶ್ಚಿಮ ಇಂಡಿಯಾ ಅದರಲ್ಲೂ ರಾಜಪುಟಾಣವು ಈ ವಿಷಯದಲ್ಲಿ ಬಂಗಾಳಕ್ಕಿಂತಲೂ ಹೆಚ್ಚು ಉತ್ತಮ ಸ್ಥಿತಿಯಲ್ಲಿದೆ. ಈಗಲೂ ಅಲ್ಲಿ ಇಂತಹ ದರ್ಬಾರುಗಳನ್ನು ಕಾಣಬಹುದು.

ಪ್ರಶ್ನೆ: ಹಾಗಾದರೆ, ಸ್ವಾಮೀಜಿ, ನಮ್ಮ ಜನಗಳು ದೊರೆಗಳಿಲ್ಲದ ಕಾರಣ ಒಳ್ಳೆಯ ನಡವಳಿಕೆಯನ್ನು ಕಳೆದುಕೊಂಡಿರುವರೆ?

ಸ್ವಾಮೀಜಿ: ಇದೆಲ್ಲಾ ಅವನತ ಸ್ಥಿತಿ. ಸ್ವಾರ್ಥವೇ ಇದರ ಮೂಲ. ಹಡಗನ್ನು ಹತ್ತುವಾಗ ನಾವು ‘ಮಾವ, ನಿನ್ನ ಪ್ರಾಣ ರಕ್ಷಿಸಿಕೊ’ ಎಂಬ ಅಸಭ್ಯ ಗಾದೆಯನ್ನನು ಸರಿಸುವೆವು, ಸಂಗೀತ ಮತ್ತು ವಿನೋದಗಳಲ್ಲಿ ನಮ್ಮನ್ನೇ ಪ್ರದರ್ಶಿಸಿಕೊಳ್ಳಲು ಪ್ರಯತ್ನಿಸುವೆವು. ಇದೇ ನಮ್ಮ ವಿಶೇಷ ಲಕ್ಷಣ. ಕೊಂಚ ಆತ್ಮತ್ಯಾಗದ ತರಬೇತಿ ಕೊಟ್ಟರೆ ಅದೆಲ್ಲಾ ಮಾಯವಾಗುವುದು. ಇವೆಲ್ಲಾ ತಮ್ಮ ಮಕ್ಕಳಿಗೆ ಒಳ್ಳೆಯ ನಡತೆಯನ್ನು ಕೂಡ ಕಲಿಸದ ತಂದೆತಾಯಿಗಳ ತಪ್ಪು. ಸ್ವಾರ್ಥತ್ಯಾಗವೇ ನಿಜವಾಗಿ ಎಲ್ಲಾ ನಾಗರಿಕತೆಯ ತಳಹದಿ.

ಮತ್ತೊಂದು ದೃಷ್ಟಿಯಿಂದ ನೋಡಿದರೆ ಈ ತಂದೆತಾಯಿಗಳು ತಮ್ಮ ಮಕ್ಕಳ ಮೇಲೆ ಚಲಾಯಿಸುವ ಮಿತಿಮೀರಿದ ಅಧಿಕಾರದಿಂದ, ನಮ್ಮ ಹುಡುಗರ ಅನಿರ್ಬಂಧವಾದ ಬೆಳವಣಿಗೆಗೆ ಅವಕಾಶವಿಲ್ಲ. ಸಂಗೀತ ಅನುಚಿತವೆಂದು ತಂದೆತಾಯಿಗಳ ಅಭಿಪ್ರಾಯ. ಯಾವಾಗ ಮಗ ಒಳ್ಳೆಯ ಸಂಗೀತವನ್ನು ಕೇಳುವನೋ ಆಗ ಸಂಪೂರ್ಣವಾಗಿ ಅವನ ಮನಸ್ಸೆಲ್ಲಾ ಹೇಗೆ ಅದನ್ನು ಕಲಿಯುವುದೆಂಬುದರಲ್ಲಿ ತೊಡಗುವುದು. ಆಗ ಸಹಜವಾಗಿ ಅದನ್ನು ಕಲಿಯಲು ಅಯೋಗ್ಯರ ಸಹವಾಸಕ್ಕೆ ತಿರುಗುವರು. ಪುನಃ ‘ಧೂಮಪಾನ ಮಾಡುವುದು ಪಾಪಕರ.’ ಆದ್ದರಿಂದ ಆ ಯುವಕ ಮನೆಯ ಆಳುಕಾಳುಗಳೊಡನೆ ಸೇರಿ ಗೋಪ್ಯವಾಗಿ ಆ ಚಟವನ್ನು ತೃಪ್ತಿಪಡಿಸಿಕೊಳ್ಳದೆ ಮತ್ತೇನು ಮಾಡಬಲ್ಲ? ಪ್ರತಿಯೊಬ್ಬರಲ್ಲಿಯೂ ಅಸಂಖ್ಯಾತ ಪ್ರವೃತ್ತಿಗಳಿವೆ. ಅವು ತೃಪ್ತಿ ಹೊಂದಲು ಸರಿಯಾದ ಅವಕಾಶವಿರಬೇಕು. ಆದರೆ ನಮ್ಮ ದೇಶದಲ್ಲಿ ಇವನ್ನು ಅನುಮೋದಿಸುವುದಿಲ್ಲ. ಇದನ್ನು ಬೇರೆ ಸ್ಥಿತಿಗೆ ತಿರುಗಿಸಬೇಕಾದರೆ ತಂದೆತಾಯಿಗಳಿಗೆ ಹೊಸ ಬಗೆಯ ಶಿಕ್ಷಣ ಕೊಡಬೇಕು. ಇದೇ ಈಗಿನ ಅವಸ್ಥೆ! ಎಂತಹ ಅನ್ಯಾಯ! ನಾವಿನ್ನೂ ನಾಗರಿಕತೆಯ ಮೇಲ್ಮಟ್ಟಕ್ಕೆ ಬಂದಿಲ್ಲ. ಆದರೂ ನಮ್ಮ ವಿದ್ಯಾವಂತ ಬಾಬುಗಳು ಆಂಗ್ಲೇಯರು ನಮಗೆ ಸರಕಾರವನ್ನು ವಹಿಸಿಕೊಡಬೇಕೆಂದು ಇಚ್ಛಿಸುವರು. ಇದನ್ನು ನೋಡಿ ನನಗೆ ನಗು ಅಳು ಎರಡೂ ಒಟ್ಟಿಗೆ ಬರುತ್ತವೆ. ಆಗಲಿ, ನಮ್ಮಲ್ಲಿ ಶೂರತ್ವದ ಸ್ಫೂರ್ತಿ ಎಲ್ಲಿದೆ? ಅದಕ್ಕೆ ಪ್ರಾರಂಭದಲ್ಲೇ ಆವಶ್ಯಕವಾಗಿರುವ ಆತ್ಮನಿಗ್ರಹ, ಸೇವೆ, ವಿಧೇಯತೆ ಎಲ್ಲಿವೆ? ನಾವು ಇತರರ ಹೃದಯವನ್ನು ಆಳಬೇಕಾದರೆ ಮೊದಲು ನಾವು ಆಜ್ಞೆಯ ಮಾತನ್ನು ಕೇಳಿದ ಕೂಡಲೇ ನಮ್ಮ ಜೀವವನ್ನು ಒತ್ತೆಯಿಟ್ಟು ಮುಂದೆ ಬರಬೇಕು. ನಮ್ಮನ್ನು ನಾವು ಮೊದಲು ಅರ್ಪಿಸಬೇಕು.

ಒಮ್ಮೆ ಶ‍್ರೀರಾಮಕೃಷ್ಣರ ಭಕ್ತನೊಬ್ಬನು ತನ್ನ ಒಂದು ಪುಸ್ತಕದಲ್ಲಿ ಯಾರು ಯಾರು ಶ‍್ರೀರಾಮಕೃಷ್ಣರನ್ನು ಅವತಾರಪುರುಷರೆಂದು ನಂಬಲಿಲ್ಲವೋ ಅವರ ಮೇಲೆಲ್ಲಾ ತೀವ್ರವಾಗಿ ಟೀಕಿಸಿ ಬರೆದಿದ್ದ. ಸ್ವಾಮೀಜಿ ಆ ಗ್ರಂಥಕರ್ತನನ್ನು ಬರಹೇಳಿ ಅವನಿಗೆ ಈ ರೀತಿ ಆವೇಶದಿಂದ ಹೇಳಿದರು:

ಇತರರನ್ನು ಈ ರೀತಿ ನಿಂದಿಸಲು ನಿನಗೇನು ಅಧಿಕಾರವಿದೆ? ನಿನ್ನ ಸ್ವಾಮಿಯಲ್ಲಿ ಅವರು ನಂಬಿಕೆಯಿಡದಿದ್ದರೆ ಬಾಧಕವೇನು? ನಾವೊಂದು ಮತವನ್ನು ಸೃಷ್ಟಿಸಿದ್ದೇವೆಯೆ? ಯಾರು ಅವರನ್ನು ಪೂಜಿಸುವುದಿಲ್ಲವೋ ಅವರೆಲ್ಲಾ ನಮ್ಮ ಶತ್ರುಗಳೆಂದು ಹೇಳಲು ನಾವು ರಾಮಕೃಷ್ಣ ಮತೀಯರೇನು? ನಿನ್ನ ಈ ಮತಭ್ರಾಂತಿಯಿಂದ ನೀನು ಶ‍್ರೀರಾಮಕೃಷ್ಣರನ್ನು ಕೆಳ ಅಂತಸ್ತಿಗೆ ತಂದುದೂ ಅಲ್ಲದೆ ಅವರನ್ನು ಸಣ್ಣವರನ್ನಾಗಿ ಮಾಡಿದೆ. ನಿನ್ನ ಸ್ವಾಮಿಯು ದೇವರೇ ಆಗಿದ್ದರೆ, ಆತನನ್ನು ಯಾವ ಹೆಸರಿನಿಂದ ಕರೆದರೂ ಆತನನ್ನು ಪೂಜಿಸಿದ ಹಾಗೆ ಎಂದು ನಿನಗೆ ತಿಳಿದಿರಬೇಕು. ಇತರರನ್ನು ನಿಂದಿಸಲು ನೀನಾರು? ನೀನು ಅವರ ವಿರುದ್ಧವಾಗಿ ದೂರಿದ ಮಾತ್ರಕ್ಕೆ ಅವರು ನಿನ್ನ ಮಾತನ್ನು ಕೇಳುವರೇನು? ಎಂತಹ ತಿಳಿಗೇಡಿತನ! ನೀನು ಮತ್ತೊಬ್ಬರಿಗಾಗಿ ತ್ಯಾಗ ಮಾಡಿದರೆ ಮಾತ್ರ ನೀನು ಇತರರ ಹೃದಯವನ್ನು ಒಲಿಸಿಕೊಳ್ಳಬಲ್ಲೆ. ಇಲ್ಲದಿದ್ದರೆ ಅವರೇಕೆ ನಿನ್ನ ಮಾತನ್ನು ಕೇಳುವರು?

ಕೊಂಚ ಹೊತ್ತಿನಲ್ಲಿಯೇ ಸ್ವಾಮೀಜಿ ಮೊದಲಿನಂತೆ ಶಾಂತಚಿತ್ತರಾಗಿ ಶೋಕಪೂರಿತ ಧ್ವನಿಯಲ್ಲಿ ಹೀಗೆ ಹೇಳಿದರು: ನನ್ನ ಪ್ರಿಯ ಗೆಳೆಯನೇ, ಯಾರಾದರೂ ತಾನೇ ಒಬ್ಬ ವೀರನಲ್ಲದಿದ್ದಲ್ಲಿ, ದೇವರಲ್ಲಿ ಶ್ರದ್ಧೆ ಇಡಲು ಅಥವಾ ಆತ್ಮಾರ್ಪಣೆ ಮಾಡಿಕೊಳ್ಳಲು ಸಾಧ್ಯವೆ? ನಾವು ಧೀರರಾಗುವವರೆಗೂ ನಮ್ಮ ಹೃದಯದಿಂದ ದ್ವೇಷ ಅಸೂಯೆಗಳು ಮಾಯವಾಗುವುದಿಲ್ಲ. ಇವುಗಳಿಂದ ಪಾರಾಗದಿದ್ದಲ್ಲಿ ನಾವು ನಾಗರಿಕರಾಗಲು ಹೇಗೆ ಸಾಧ್ಯ? ನಮ್ಮ ದೇಶದಲ್ಲಿ ಇಂತಹ ದೃಢಕಾಯರಾದ ಪೌರುಷವುಳ್ಳ ಗಂಡಸರು, ಇಂತಹ ವೀರಸ್ಫೂರ್ತಿಯುಳ್ಳವರು ಎಲ್ಲಿದ್ದಾರೆ? ಎಲ್ಲಿಯೂ ಇಲ್ಲ. ಇದಕ್ಕಾಗಿ ನಾನು ಅನೇಕ ವೇಳೆ ಹುಡುಕಿದ್ದೇನೆ. ಕೇವಲ ಒಂದೇ ಒಂದು ನಿದರ್ಶನ ಮಾತ್ರ ನಮಗೆ ಸಿಕ್ಕಿದೆ.

ಪ್ರಶ್ನೆ: ನೀವು ಯಾರಲ್ಲಿ ಇದನ್ನು ನೋಡಿದಿರಿ, ಸ್ವಾಮೀಜಿ?

ಸ್ವಾಮೀಜಿ: ಗಿರೀಶಚಂದ್ರಘೋಷರಲ್ಲಿ ಮಾತ್ರ ನಾನು ಆ ರೀತಿಯ ಸತ್ಯವಾದ ಆತ್ಮಾರ್ಪಣೆಯನ್ನು - ಭಗವಂತನ ಸೇವಕ ತಾನೆಂಬ ನಮ್ರಭಾವವನ್ನು ನೋಡಿದೆ. ಅವರು ನಿರಂತರ ತ್ಯಾಗಕ್ಕೆ ಸಿದ್ಧರಾಗಿದ್ದುದರಿಂದಲೇ ಅಲ್ಲವೆ ಶ‍್ರೀರಾಮಕೃಷ್ಣರು ಅವರ ಜವಾಬ್ದಾರಿಯನ್ನೆಲ್ಲ ತಾವೇ ವಹಿಸಿಕೊಂಡುದು? ದೇವರಲ್ಲಿ ಎಂತಹ ಅಪೂರ್ವ ಶರಣಾಗತ ಭಾವನೆ! ಅವರಿಗೆ ಸರಿಸಮಾನರಾದವರನ್ನು ನಾನಾರನ್ನೂ ನೋಡಿಲ್ಲ. ಅವರಿಂದ ನಾವೀ ಆತ್ಮ ಸಮರ್ಪಣೆಯ ಪಾಠವನ್ನು ಕಲಿತೆವು.

ಹೀಗೆ ಹೇಳುತ್ತಾ ಸ್ವಾಮಿಗಳು ಅವರನ್ನು ಗೌರವಿಸಲು ಕೈಗಳನ್ನೆತ್ತಿ ನಮಸ್ಕರಿಸಿದರು.

\newpage

\chapter[ಅಧ್ಯಾಯ ೭]{ಅಧ್ಯಾಯ ೭\protect\footnote{\engfoot{C.W, Vol. VII, P. 271}}}

ಸ್ವಾಮಿಗಳು ಎರಡನೆಯ ಬಾರಿ ಇಂಡಿಯಾದಿಂದ ಅಮೆರಿಕಾಕ್ಕೆ ಹೋಗಲು (೧೮೯೯ರಲ್ಲಿ) ಸಿದ್ಧತೆಗಳಾಗುತ್ತಿದ್ದವು. ಅವರು ಕಲ್ಕತ್ತೆಗೆ ತಮ್ಮ ಸ್ನೇಹಿತರೊಬ್ಬರನ್ನು ನೋಡಲು ಹೋಗಿದ್ದರು. ಅವರು ಹಿಂತಿರುಗಿ ಬರುವಾಗ ಬಾಗ್‌ಬಜಾರಿನಲ್ಲಿ ಬಲರಾಮಬಾಬುಗಳ ಮನೆಯಲ್ಲಿ ಕೆಲವು ನಿಮಿಷಗಳಿದ್ದರು. ಮಠಕ್ಕೆ ತಮ್ಮ ಜೊತೆಯಲ್ಲಿ ಬರುವ ಮತ್ತೊಬ್ಬ ಗೆಳೆಯನಿಗೆ ಹೇಳಿ ಕಳುಹಿಸಿದರು. ಆ ಸ್ನೇಹಿತನು ಬಂದಾಗ ಅವನಿಗೂ ಸ್ವಾಮಿಗಳಿಗೂ ಈ ರೀತಿ ಸಂಭಾಷಣೆ ಜರುಗಿತು.

ಸ್ವಾಮೀಜಿ: ಇಂದು ಒಂದು ಕೌತುಕದ ಸಂಗತಿ ಜರುಗಿತು. ನಾನೊಬ್ಬ ಸ್ನೇಹಿತನ ಮನೆಗೆ ಹೋಗಿದ್ದೆ. ಅವನ ಬಳಿ ಒಂದು ತೈಲ ಚಿತ್ರವಿತ್ತು. ಶ‍್ರೀಕೃಷ್ಣನು ಕುರುಕ್ಷೇತ್ರದ ಭೂಮಿಯಲ್ಲಿ ಅರ್ಜುನನಿಗೆ ಭಗವದ್ಗೀತೆಯನ್ನು ಬೋಧಿಸುತ್ತಿದ್ದ ವಿಷಯ ಚಿತ್ರಿತವಾಗಿತ್ತು. ಶ‍್ರೀಕೃಷ್ಣನು ರಥದಲ್ಲಿ ನಿಂತುಕೊಂಡು ಲಗಾಮನ್ನು ಕೈಯಲ್ಲಿ ಹಿಡಿದುಕೊಂಡು, ಅರ್ಜುನನಿಗೆ ಗೀತೆಯನ್ನು ಉಪದೇಶಿಸುತ್ತಿದ್ದನು. ಅವನು ನನಗೆ ಆ ಪಟವನ್ನು ತೋರಿಸಿ ಅದನ್ನು ನಾನು ಹೇಗೆ ಇಷ್ಟಪಡುವೆನೆಂದು ಕೇಳಿದ. ‘ಸುಮಾರಾಗಿ ಚೆನ್ನಾಗಿದೆ’ ಎಂದೆ. ಆದರೆ ಅವನು ನನ್ನ ವಿಮರ್ಶೆಯನ್ನು ಹೇಳಲೇಬೇಕೆಂದು ಒತ್ತಿ ಹೇಳಿದುದರಿಂದ ನಾನು ಸತ್ಯವಾದ ಅಭಿಪ್ರಾಯವನ್ನೇ ಕೊಟ್ಟೆ. ಇದರಲ್ಲಿ ಪ್ರಶಂಸನೀಯವಾದುದೇನೂ ನನಗೆ ತೋರುವುದಿಲ್ಲ. ಏಕೆಂದರೆ ಮೊದಲನೆಯದಾಗಿ ಶ‍್ರೀಕೃಷ್ಣನ ಕಾಲದಲ್ಲಿದ್ದ ರಥ ಇದರಲ್ಲಿರುವಂತೆ ಗೋಪುರಾಕಾರದ ವಾಹನದಂತೆ ಇರಲಿಲ್ಲ. ಅಲ್ಲದೆ ಶ‍್ರೀಕೃಷ್ಣನ ಆಕೃತಿ ಭಾವಶೂನ್ಯವಾಗಿದೆ.

ಪ್ರಶ್ನೆ: ಆಗ ಗೋಪುರಾಕೃತಿಯ ರಥಗಳಿರಲಿಲ್ಲವೆ?

ಸ್ವಾಮೀಜಿ: ಬೌದ್ಧರ ಕಾಲದಿಂದೀಚೆಗೆ ನಮ್ಮ ದೇಶದಲ್ಲಿ ಎಲ್ಲದರಲ್ಲಿಯೂ ಬಹಳ ಗಲಿಬಿಲಿಯಾಗಿದೆ ಎಂದು ನಿನಗೆ ಗೊತ್ತಿಲ್ಲವೆ? ದೊರೆಗಳು ಗೋಪುರಾಕೃತಿಯ ರಥದಲ್ಲಿ ಎಂದೂ ಯುದ್ಧ ಮಾಡುತ್ತಿರಲಿಲ್ಲ. ಈಗಲೂ ರಾಜಪುಟಾಣದಲ್ಲಿರುವ ಕೆಲವು ರಥಗಳು ಪೂರ್ವದ ರಥಗಳನ್ನೇ ಹೋಲುತ್ತವೆ. ಗ್ರೀಕರ ಪುರಾಣಗಳಲ್ಲಿರುವ ರಥಗಳ ಚಿತ್ರಗಳನ್ನು ನೀನು ನೋಡಿಲ್ಲವೆ? ಅದಕ್ಕೆ ಎರಡು ಚಕ್ರಗಳಿರುತ್ತವೆ, ಅದನ್ನು ಹಿಂದುಗಡೆಯಿಂದ ಹತ್ತಬೇಕು. ನಮ್ಮಲ್ಲಿ ಅಂತಹ ರಥಗಳಿದ್ದವು. ವಿವರಗಳೆಲ್ಲಾ ತಪ್ಪುತಪ್ಪಾಗಿ ಇರುವ ತೈಲಚಿತ್ರ ಬರೆಯುವುದರಿಂದ ಪ್ರಯೋಜನವೇನು? ಸರಿಯಾಗಿ ವ್ಯಾಸಂಗ ಮತ್ತು ಸಂಶೋಧನೆಗಳನ್ನು ಮಾಡಿ ಆಗಿನ ಕಾಲದಲ್ಲಿ ವಸ್ತುಗಳು ಹೇಗಿದ್ದವೋ ಹಾಗೆಯೇ ಚಾಚೂ ತಪ್ಪದೆ ಚಿತ್ರಿಸಿದರೆ, ಅಂತಹ ಚಿತ್ರವೀಗ ಅತಿ ಶ್ರೇಷ್ಠವಾದ ತರಗತಿಗೆ ಸೇರಿದುದೆಂದು ಹೇಳಬಹುದು. ವಾಸ್ತವವಾಗಿರಬೇಕು. ಇಲ್ಲದಿದ್ದಲ್ಲಿ ಆ ಚಿತ್ರದಿಂದೇನು ಪ್ರಯೋಜನ? ಈಗಿನ ಕಾಲದಲ್ಲಿ ಯಾರು ವರ್ಣಚಿತ್ರಕಾರರಾಗಲು ಇಷ್ಟಪಡುವರೋ ಅವರೆಲ್ಲಾ ಸಾಧಾರಣವಾಗಿ ಶಾಲೆಯಲ್ಲಿ ಅನುತ್ತೀರ್ಣರಾದವರು ಮತ್ತು ಮನೆಯಲ್ಲಿ ಯಾವ ಕೆಲಸಕ್ಕೂ ಬಾರದವರು. ಇಂಥವರಿಂದ ಎಂತಹ ಕಲಾಕೃತಿಯನ್ನು ನಿರೀಕ್ಷಿಸಬಹುದು? ಉತ್ತಮ ಮಟ್ಟದ ಕಲಾಕೃತಿಯನ್ನು ನಿರೀಕ್ಷಿಸಬೇಕಾದರೆ ಒಂದು ಮೇಲ್ಮಟ್ಟದ ನಾಟಕವನ್ನು ಬರೆಯಲು ಎಂತಹ ಪ್ರತಿಭೆ ಬೇಕೊ ಅಂತಹ ಸಾಮರ್ಥ್ಯ ಆವಶ್ಯಕ.

ಪ್ರಶ್ನೆ: ಹಾಗಾದರೆ, ಈಗ ಪ್ರಶ್ನೆಯಲ್ಲಿರುವ ಚಿತ್ರದಲ್ಲಿ ಶ‍್ರೀಕೃಷ್ಣನನ್ನು ಹೇಗೆ ಚಿತ್ರಿಸಬೇಕು?

ಸ್ವಾಮೀಜಿ: ಗೀತೆಯ ಮೂರ್ತಿಮತ್ತಾದ ಶ‍್ರೀಕೃಷ್ಣನನ್ನು, ಅವನು ನಿಜವಾಗಿ ಇದ್ದಂತೆಯೇ ಚಿತ್ರಿಸಬೇಕು. ಭ್ರಾಂತಿ ಮತ್ತು ಹೇಡಿತನಕ್ಕೆ ಒಳಗಾಗಿದ್ದ ಅರ್ಜುನನಿಗೆ ಧರ್ಮಮಾರ್ಗವನ್ನು ಬೋಧಿಸುತ್ತಿದ್ದುದರಿಂದ ಗೀತೆಯ ಪ್ರಧಾನಭಾವ ಕೃಷ್ಣನ ಇಡೀ ದೇಹದಿಂದ ಹೊರ ಹೊಮ್ಮುವಂತಿರಬೇಕು.

ಹೀಗೆ ಹೇಳುತ್ತಾ ಸ್ವಾಮಿಗಳು ಶ‍್ರೀಕೃಷ್ಣನು ಯಾವ ರೀತಿಯಲ್ಲಿ ಚಿತ್ರಿತವಾಗಬೇಕೆಂದು ತಾವೇ ಆ ಭಂಗಿಯನ್ನು ತೋರಿಸಿದರು. ನಂತರ ಮುಂದುವರಿಸಿ ಈ ರೀತಿ ಹೇಳಿದರು: ಇಲ್ಲಿ ನೋಡು, ಹೀಗೆ ಅವನು ಆ ಕುದುರೆಗಳ ಲಗಾಮನ್ನು ಹಿಡಿದಿರಬೇಕು - ಎಷ್ಟು ಬಿಗಿಯಾಗಿ ಹಿಡಿದಿರಬೇಕೆಂದರೆ, ಕುದುರೆಗಳು ತಮ್ಮ ಮುಂಗಾಲುಗಳಿಂದ ಗಾಳಿಯನ್ನು ಬಡಿಯುತ್ತಾ ಇರಬೇಕು. ಬಾಯಿತೆರೆದು ನಿಂತಿರಬೇಕು. ಇದು ಶ‍್ರೀಕೃಷ್ಣನ ಆಕೃತಿಯಲ್ಲಿ ಒಂದು ಪ್ರಚಂಡ ಕರ್ಮದ ಅಭಿವ್ಯಕ್ತಿಯನ್ನು ತೋರಿಸುವುದು. ಅವನ ಸ್ನೇಹಿತ ಜಗದ್ವಿಖ್ಯಾತ ವೀರ ಎರಡು ಸೈನ್ಯಗಳ ಮಧ್ಯೆ ತನ್ನ ಬಿಲ್ಲು ಬಾಣಗಳನ್ನು ಬಿಸುಟು ಹೇಡಿಯಂತೆ ರಥದಲ್ಲಿ ಕುಗ್ಗಿ ಕುಳಿತಿದ್ದಾನೆ. ಆಗ ಶ‍್ರೀಕೃಷ್ಣ ಒಂದು ಕೈಯಲ್ಲಿ ಚಾವಟಿ, ಮತ್ತೊಂದು ಕೈಯಲ್ಲಿ ಬಿಗಿಹಿಡಿದ ಲಗಾಮು ಹಿಡಿದು ಅರ್ಜುನನೆಡೆಗೆ ತಿರುಗಿದ್ದಾನೆ - ಆತನ ಶಿಶುಸಹಜವಾದ ಮುಖದಾವರೆ ಅಲೌಕಿಕವಾದ ಪ್ರೇಮ ಮತ್ತು ದಯೆಯಿಂದ ತುಂಬಿ ತುಳುಕಾಡುತ್ತಿದೆ; ಪ್ರಶಾಂತವಾದ ಸ್ಥಿರವಾದ ಮುಖಮುದ್ರೆಯಿಂದ ಕೂಡಿದ ಶ‍್ರೀಕೃಷ್ಣ ತನ್ನ ಪ್ರಿಯ ಗೆಳೆಯನಿಗೆ ಗೀತಾ ಸಂದೇಶವನ್ನು ಬೋಧಿಸುತ್ತಿದ್ದಾನೆ. ಈಗ ಹೇಳು ಗೀತಾ ಸಂದೇಶಕನ ಈ ಚಿತ್ರ ನಿನಗಾವ ಭಾವವನ್ನು ತಿಳಿಸುತ್ತದೆಂದು.

ಸ್ನೇಹಿತ: ದೃಢತೆ ಮತ್ತು ಗಂಭೀರ ಭಾವದಿಂದ ಕೂಡಿದ ಕ್ರಿಯೆಯನ್ನು ತಿಳಿಸುವುದು.

ಸ್ವಾಮೀಜಿ: ಹೌದು! ಅದೇ! ಇಡೀ ದೇಹದಲ್ಲೆಲ್ಲಾ ತೀವ್ರಚಲನೆ, ಜೊತೆಗೆ ಮುಖದಲ್ಲಿ ನೀಲಿಮ ಆಕಾಶದಂತೆ ಪ್ರಶಾಂತತೆ ಮತ್ತು ನಿಶ್ಚಲತೆ, ಇದೇ ಗೀತೆಯ ಜೀವಾಳ ಎಲ್ಲಾ ಪರಿಸ್ಥಿತಿಯಲ್ಲೂ ಪ್ರಶಾಂತವಾಗಿ ದೃಢವಾಗಿದ್ದು ತನುಮನ ಆತ್ಮವೆಲ್ಲಾ ಆತನ ಪಾದಪದ್ಮಗಳಲ್ಲಿ ಕೇಂದ್ರೀಕೃತವಾಗಬೇಕು.

\begin{verse}
ಕರ್ಮಣ್ಯಕರ್ಮ ಯಃ ಪಶ್ಯೇದಕರ್ಮಣಿ ಚ ಕರ್ಮ ಯಃ~।\\ಸ ಬುದ್ಧಿ ಮಾನ್ಮನುಷ್ಯೇಷು ಸ ಯುಕ್ತಃ ಕೃತ್ಸ್ನಕರ್ಮಕೃತ್~॥
\end{verse}

\versenum{(ಗೀತೆ, \general{\enginline{ IV,}} ೧೮)}

“ಯಾರು ಕರ್ಮದಲ್ಲಿ ನಿರತರಾಗಿದ್ದರೂ, ಪ್ರಶಾಂತಚಿತ್ತರೊ, ಕರ್ಮ ಮಾಡದಿರುವಾಗಲೂ ಬ್ರಹ್ಮಧ್ಯಾನವೆಂಬ ಕರ್ಮದಲ್ಲಿ ನಿರತರಾಗಿರುವರೊ ಅವರೇ ಬುದ್ಧಿಮಾನ್, ಯೋಗಿ, ಕರ್ಮಕುಶಲಿ.”

ಆ ಸಮಯಕ್ಕೆ ಸರಿಯಾಗಿ ದೋಣಿಯನ್ನು ಗೊತ್ತುಮಾಡಿಕೊಂಡು ಬರಲು ಹೋಗಿದ್ದ ಮನುಷ್ಯ ಹಿಂತಿರುಗಿ ಅದು ಸಿದ್ಧವಾಗಿದೆ ಎಂದು ತಿಳಿಸಿದ. ಸ್ವಾಮೀಜಿ ಸ್ನೇಹಿತನಿಗೆ: ನಡೆ ಮಠಕ್ಕೆ ಹೋಗೋಣ. ನೀನು ಮನೆಯಲ್ಲಿ ಹೇಳಿ ಬಂದಿರಬೇಕಲ್ಲವೆ, ನನ್ನ ಜೊತೆಯಲ್ಲಿ ಹೋಗುವೆನೆಂದು?

ಹೀಗೆ ಮಾತನಾಡುತ್ತಲೇ ಅವರು ದೋಣಿಯೆಡೆಗೆ ಹೋದರು.

ಸ್ವಾಮೀಜಿ: ಕೆಲಸ-ಕೆಲಸ-ನಿರಂತರ ಕೆಲಸ - ಈ ಉದ್ದೇಶವನ್ನು ಎಲ್ಲರಿಗೂ ಬೋಧಿಸಬೇಕು. ಪ್ರತಿಫಲಾಪೇಕ್ಷೆಗೆ ಗಮನಕೊಡದೆ ಇಡೀ ಮನಸ್ಸು, ಆತ್ಮವನ್ನೆಲ್ಲಾ ಭಗವಂತನ ಅಡಿದಾವರೆಯಲ್ಲಿ ನಿಶ್ಚಲವಾಗಿ ನೆಲಸುವಂತೆ ಮಾಡಬೇಕು.

ಪ್ರಶ್ನೆ: ಆದರೆ ಇದು ಕರ್ಮಯೋಗವಲ್ಲವೆ?

ಸ್ವಾಮೀಜಿ: ಹೌದು ಇದು ಕರ್ಮಯೋಗ, ಆದರೆ ಆಧ್ಯಾತ್ಮಿಕ ಸಾಧನೆ ಮಾಡದೆ ಕರ್ಮಯೋಗವನ್ನು ನೀನು ಮಾಡಲಾರೆ. ನಾಲ್ಕು ವಿಧವಾದ ಯೋಗಗಳನ್ನೂ ನೀನು ಸಮರಸ ಮಾಡಬೇಕು. ಇಲ್ಲದಿದ್ದಲ್ಲಿ ಸಂಪೂರ್ಣವಾಗಿ ಮನಸ್ಸನ್ನೆಲ್ಲಾ ಭಗವಂತನಲ್ಲಿಡಲು ಹೇಗೆ ಸಾಧ್ಯ?

ಪ್ರಶ್ನೆ: ಗೀತೆಯ ಪ್ರಕಾರ ವೇದಗಳಲ್ಲಿ ಹೇಳುವ ಹೋಮ ಮತ್ತು ಇತರ ಧಾರ್ಮಿಕ ವಿಧಿಗಳನ್ನು ಮಾಡುವುದೇ ಕರ್ಮ ಎಂದು ಸಾಧಾರಣವಾಗಿ ಹೇಳುವರಲ್ಲ, ಇತರ ಎಲ್ಲ ಬಗೆಯ ಕೆಲಸವೂ ನಿಷ್ಪಲವೆ?

ಸ್ವಾಮೀಜಿ: ಹಾಗೆಯೇ ಇಟ್ಟುಕೊಳ್ಳೋಣ. ನೀನು ಅದನ್ನು ಇನ್ನೂ ಹೆಚ್ಚು ವಿವರವಾಗಿ ಹೇಳಬೇಕು. ನೀನು ಮಾಡುವ ಪ್ರತಿಯೊಂದು ಕೆಲಸಕ್ಕೂ ನೀನು ಉಸಿರಾಡುವುದಕ್ಕೂ ನೀನು ಯೋಚಿಸುವ ಪ್ರತಿಯೊಂದು ಆಲೋಚನೆಗೂ ಯಾರು ಹೊಣೆ? ನೀನೇ ಅದಕ್ಕೆ ಹೊಣೆಯಲ್ಲವೆ?

ಸ್ನೇಹಿತ: ಹೌದು ಮತ್ತು ಅಲ್ಲ. ನಾನಿದನ್ನು ಸ್ಪಷ್ಟವಾಗಿ ವಿವರಿಸಲಾರೆ. ವಾಸ್ತವಾಂಶವೇನೆಂದರೆ ಮಾನವ ಒಂದು ಯಂತ್ರ, ಭಗವಂತ ಕರ್ತ. ಆದ್ದರಿಂದ ನಾನು ಯಾವಾಗ ಆತನ ಇಚ್ಛೆಯಂತೆ ನಡೆಯುವೆನೊ ಆಗ ನಾನು ಹೇಗೆ ನಾನು ಮಾಡುವ ಕರ್ಮಕ್ಕೆ ಜವಾಬ್ದಾರನಾಗುವೆ?

- ಸ್ವಾಮೀಜಿ: ಸರಿ, ಇದನ್ನು ಭಗವತ್ಸಾಕ್ಷಾತ್ಕಾರದ ಅತ್ಯಂತ ಉನ್ನತಮಟ್ಟಕ್ಕೆ ಹೋದಾಗ ಹೇಳಬಹುದು. ಯಾವಾಗ ಮನಸ್ಸು ಕರ್ಮದಿಂದ ಪರಿಶುದ್ಧವಾಗುವುದೋ ಆಗ ನೀನು ಖಂಡಿತ ನೋಡುವೆ, ಆತನೆ ಎಲ್ಲಾ ಕರ್ಮಕ್ಕೂ ಕಾರಣನೆಂದು. ಆಗ ಮಾತ್ರ ನೀನು ಈ ರೀತಿ ಮಾತನಾಡಲು ಅರ್ಹನಾಗುವೆ. ಇಲ್ಲದೆ ಇದ್ದರೆ ಇದೆಲ್ಲಾ ಬರಿಯ ಹುಸಿಮಾತು; ಕಪಟ, ಅಷ್ಟೆ.

ಪ್ರಶ್ನೆ: ಹಾಗೇಕೆ? ಭಗವಂತನೇ ನಮ್ಮೆಲ್ಲಾ ಕೆಲಸಗಳಿಗೂ ಕಾರಣನೆಂದು ತರ್ಕದಿಂದ ನಮಗೆ ಸ್ಪಷ್ಟವಾಗಿ ಗೊತ್ತಾದರೆ?

ಸ್ವಾಮೀಜಿ: ನಮಗೆ ಹಾಗೆ ಗೊತ್ತಾದರೆ ಅದೇನೋ ಒಳ್ಳೆಯದಾಗಬಹುದು. ಆದರೆ ಅದು ಕೇವಲ ಒಂದು ಕ್ಷಣಮಾತ್ರ. ಆಮೇಲೆ ಲೇಶಮಾತ್ರವೂ ಇರುವುದಿಲ್ಲ. ಆಗಲಿ, ನೀನು ಪ್ರತಿನಿತ್ಯವೂ ಮಾಡುವುದಕ್ಕೆಲ್ಲಾ ನೀನು ಕರ್ತನಲ್ಲವೆಂಬ ಅಹಂಭಾವನೆಯನ್ನು ಬಿಟ್ಟು ಮಾಡುವೆಯೇನು? ಸರಿಯಾಗಿ ಯೋಚಿಸಿ ನೋಡು. ದೇವರೇ ನಿನ್ನನ್ನು ಈ ರೀತಿ ಕೆಲಸ ಮಾಡಿಸುತ್ತಿರುವನೆಂದು ನೀನು ಎಷ್ಟು ಹೊತ್ತು ನೆನಪಿನಲ್ಲಿಡಬಲ್ಲೆ? ಆದರೆ ನೀನು ಹೀಗೆ ಬಾರಿಬಾರಿಗೂ ಪರೀಕ್ಷಿಸುತ್ತಾ ಹೋದ ಹಾಗೆಲ್ಲಾ, ಅಹಂಭಾವ ಮಾಯವಾಗುವ ಮಟ್ಟಕ್ಕೆ ಬರುವೆ. ಆ ಸ್ಥಳದಲ್ಲಿ ಭಗವಂತನು ಬರುವನು. ಆಗ ನೀನು ನ್ಯಾಯವಾಗಿ ಹೇಳಬಹುದು: 'ಓ ದೇವರೇ! ನೀನೇ ನನ್ನೆಲ್ಲಾ ಕೆಲಸಗಳನ್ನು ಆಂತರ್ಯದೊಳಗಿನಿಂದ ಮಾಡಿಸುತ್ತಿರುವೆ. ಆದರೆ ಸ್ನೇಹಿತನೆ, ಅಹಂಭಾವ ನಿನ್ನ ಹೃದಯವನ್ನೆಲ್ಲಾ ಸಂಪೂರ್ಣವಾಗಿ ಆವರಿಸಿದ್ದರೆ, ಭಗವಂತ ಅಲ್ಲಿಗೆ ಪ್ರವೇಶಿಸಲು ನಿಜವಾಗಿ ಎಲ್ಲಿ ಸ್ಥಳವಿದೆ? ಆಗ ಭಗವಂತ ಸತ್ಯವಾಗಿಯೂ ಅಲ್ಲಿರುವುದಿಲ್ಲ.

ಪ್ರಶ್ನೆ: ಆದರೆ ಅವನೇ ನನಗೆ ಕೆಟ್ಟ ಪ್ರೇರೇಪಣೆಗಳನ್ನೂ ಕೊಡುವವನು.

ಸ್ವಾಮೀಜಿ: ಇಲ್ಲ, ಖಂಡಿತವಾಗಿಯೂ ಇಲ್ಲ. ಈ ರೀತಿ ಯೋಚಿಸುವುದು ದೇವರನ್ನು ನಿಂದಿಸಿದಂತೆ, ಅವನು ನಿನ್ನನ್ನು ಕೆಟ್ಟ ಕಾರ್ಯಗಳಿಗೆ ಪ್ರೇರೇಪಿಸುತ್ತಿಲ್ಲ. ಇದೆಲ್ಲಾ ನಿನ್ನ ಭೋಗಲಾಲಸೆ. ದೇವರೇ ಪ್ರತಿಯೊಂದಕ್ಕೂ ಕಾರಣಕರ್ತನೆಂದು ಹೇಳಿ ಮನಃಪೂರ್ವಕವಾಗಿ ಕೆಟ್ಟ ಕೆಲಸಗಳಲ್ಲಿ ಒಬ್ಬನು ಛಲದಿಂದ ಪ್ರಯತ್ನಿಸಿದರೆ ಅವನು ನಿಜವಾಗಿ ಹಾಳಾಗುವನು. ಇದೇ ಆತ್ಮವಂಚನೆಯ ಮೂಲ. ನೀನೊಂದು ಒಳ್ಳೆಯ ಕೆಲಸವನ್ನು ಮಾಡಿದ ಕೂಡಲೇ ನಿನಗೊಂದು ಬಗೆಯ ಉಲ್ಲಾಸವಾಗುವುದಿಲ್ಲವೇ? ಆಗ ನೀನೇನೊ ಒಂದು ಒಳ್ಳೆಯ ಕೆಲಸ ಮಾಡಿದೆ ಎಂದು ಹೆಮ್ಮೆಪಡುವೆ. ಇದು ಮನುಷ್ಯ ಸಹಜವಾದುದು. ನೀನೇನೂ ಮಾಡಲಾರೆ. ಆದರೆ ಒಳ್ಳೆಯ ಕೆಲಸ ಮಾಡಿದಾಗ ಅದರ ಕೀರ್ತಿಯನ್ನು ನೀನು ತೆಗೆದುಕೊಂಡು, ಕೆಟ್ಟ ಕೆಲಸ ಮಾಡಿದಾಗ ಮಾತ್ರ ಅದನ್ನು ದೇವರ ಮೇಲೆ ಹೊರಿಸುವುದು ಎಂತಹ ತಿಳಿಗೇಡಿತನ! ಇದೊಂದು ಅಪಾಯಕರವಾದ ಭಾವನೆ. ಗೀತೆ ವೇದಾಂತಗಳನ್ನು ಅಪಾರ್ಥಮಾಡಿಕೊಂಡುದರ ಫಲ. ಈ ದೃಷ್ಟಿಯನ್ನೆಂದಿಗೂ ಇಟ್ಟುಕೊಳ್ಳಬೇಡ. ಅದಕ್ಕೆ ಬದಲು ಒಳ್ಳೆಯ ಕೆಲಸಗಳನ್ನೆಲ್ಲಾ ಅವನು ಮಾಡಿಸುತ್ತಿದ್ದಾನೆ, ಕೆಟ್ಟ ಕೆಲಸಗಳಿಗೆಲ್ಲಾ ನೀನು ಹೊಣೆಯೆಂಬ ಭಾವನೆ ಇಟ್ಟುಕೋ. ಇದರಿಂದ ನಿನ್ನಲ್ಲೇ ಶ್ರದ್ಧೆ ಮತ್ತು ಭಕ್ತಿಯುಂಟಾಗುವುದು. ಹೆಜ್ಜೆ ಹೆಜ್ಜೆಗೂ ನೀನು ಅವನ ಕೃಪೆಯ ಆವಿರ್ಭಾವವನ್ನು ನೋಡುವೆ. ವಾಸ್ತವಾಂಶವೇನೆಂದರೆ ಯಾರೂ ನಿನ್ನನ್ನು ಸೃಷ್ಟಿಸಿಲ್ಲ. ನೀನೇ ನಿನ್ನನ್ನು ಸೃಷ್ಟಿಸಿಕೊಂಡಿರುವೆ. ಇದೇ ವಿವೇಚನೆ. ಇದೇ ವೇದಾಂತ, ಭಗವತ್ಸಾಕ್ಷಾತ್ಕಾರಕ್ಕೆ ಮುಂಚೆ ಇದು ಯಾರಿಗೂ ಅರ್ಥವಾಗುವುದಿಲ್ಲ. ಆದ್ದರಿಂದ ಸಾಧಕನು ಯಾವಾಗಲೂ ದೇವರು ಒಳ್ಳೆಯ ಕೆಲಸಗಳಿಗೆ ಕಾರಣ, ತಾನು ಕೆಟ್ಟ ಕಾರ್ಯಗಳಿಗೆ ಕರ್ತ ಎಂಬ ದ್ವೈತಭಾವನೆಯಿಂದ ಪ್ರಾರಂಭಿಸಬೇಕು. ಇದೇ ಮನಸ್ಸನ್ನು ಶುದ್ಧಿ ಮಾಡಲು ಸುಲಭ ಮಾರ್ಗ. ಅದಕ್ಕೇ ವೈಷ್ಣವರಲ್ಲಿ ಈ ದ್ವೈತಭಾವನೆ ಅತ್ಯಂತ ಪ್ರಬಲವಾಗಿರುವುದನ್ನು ನೋಡುವೆವು. ಪ್ರಾರಂಭದಲ್ಲೇ ಅದ್ವೈತಭಾವನೆಗೆ ಎಡೆಗೊಡುವುದು ಬಹು ಕಷ್ಟ. ದ್ವೈತ ಭಾವನೆಯೇ ಕ್ರಮೇಣ ಅದ್ವೈತದ ಸಾಕ್ಷಾತ್ಕಾರಕ್ಕೆ ಒಯ್ಯುವುದು.

"ಕಪಟಾಚರಣೆ ಯಾವಾಗಲೂ ತುಂಬಾ ಅಪಾಯಕರ. ತಿಳಿಯದೆ ಮಾಡುವ ಆತ್ಮವಂಚನೆ, ಎಂದರೆ ಅತ್ಯಂತ ಕೆಟ್ಟ ಪ್ರೇರೇಪಣೆ ಕೂಡ ಭಗವಂತನಿಂದಲೇ ಪ್ರೇರೇಪಿತವಾದದ್ದೆಂದು ಯಾರು ಮನಃಪೂರ್ವಕವಾಗಿ ನಂಬುವನೋ ಅಂತಹವನು ಬಹುಕಾಲ ಈ ಕೀಳು ಕೆಲಸಗಳನ್ನು ಮಾಡಬೇಕಾಗುವುದಿಲ್ಲವೆಂದು ದೃಢವಾಗಿ ನಂಬು. ಆತನ ಮನಸ್ಸಿನ ಕಲ್ಮಷವೆಲ್ಲಾ ಬಹುಬೇಗ ನಾಶವಾಗುವುದು. ನಮ್ಮ ಪೂರ್ವ ಧರ್ಮಗ್ರಂಥಕಾರರು ಇದನ್ನು ಚೆನ್ನಾಗಿ ಗ್ರಹಿಸಿದ್ದರು. ಬೌದ್ಧ ಮತದ ಅವನತಿಯ ಕಾಲದಲ್ಲಿ ತಾಂತ್ರಿಕ ಪೂಜಾವಿಧಾನವು ಪ್ರಾರಂಭವಾಗಿರಬೇಕೆಂದು ನನಗನ್ನಿಸುವುದು. ಬೌದ್ಧರ ದಬ್ಬಾಳಿಕೆಯಿಂದ ಜನರು ವೇದ ಕಾಲದ ಹೋಮಗಳನ್ನು ರಹಸ್ಯವಾಗಿ ಮಾಡಲುಪಕ್ರಮಿಸಿದರು. ಅವುಗಳನ್ನು ಎರಡು ತಿಂಗಳವರೆಗೆ ಅನಿರ್ಬಂಧವಾಗಿ ಮಾಡಲು ಅವಕಾಶ ಸಿಗದೆಹೋದ ಕಾರಣ ಅವರು ಮಣ್ಣಿನ ವಿಗ್ರಹಗಳನ್ನು ಮಾಡಿ, ಒಂದೇ ರಾತ್ರಿಯಲ್ಲಿ ಎಲ್ಲವನ್ನೂ ಪೂರೈಸಿ ಯಾವ ಗುರುತೂ ಇಲ್ಲದಂತೆ ನೀರಿನ ಪಾಲು ಮಾಡಿಬಿಡುತ್ತಿದ್ದರು. ಮಾನವ ಸಾಕಾರ ಸಂಕೇತವನ್ನು ಆಶಿಸುವನು. ಇಲ್ಲದಿದ್ದಲ್ಲಿ ಅವನ ಹೃದಯಕ್ಕೆ ತೃಪ್ತಿಯಾಗುವುದಿಲ್ಲ. ಆದ್ದರಿಂದ ಪ್ರತಿಯೊಂದು ಮನೆಯಲ್ಲಿಯೂ ‘ಒಂದು ರಾತ್ರಿ ಹೋಮ’ ಮಾಡಲು ಪ್ರಾರಂಭವಾಯಿತು. ಆದರೆ ಜನರ ಪ್ರವೃತ್ತಿಗಳು ಕ್ರಮೇಣ ಇಂದ್ರಿಯಾಸಕ್ತವಾದವು. ಶ‍್ರೀರಾಮಕೃಷ್ಣರು ಹೇಳುತ್ತಿದ್ದರು: ‘ಕೆಲವರು ಮನೆಯನ್ನು ಜಾಡಮಾಲಿ ಪ್ರವೇಶಿಸುವ ಬಾಗಿಲಿನಿಂದ ಪ್ರವೇಶಿಸುವರು’ ಎಂದು. ಅದಕ್ಕೆ ಆಗಿನ ಆಧ್ಯಾತ್ಮಿಕ ಗುರುಗಳು ಯಾರು ತಮ್ಮ ಕೆಟ್ಟ ಪ್ರವೃತ್ತಿಗಳ ಪರಿಣಾಮವಾಗಿ ಯಾವೊಂದು ಧಾರ್ಮಿಕ ವಿಧಿಗಳನ್ನೂ ಮಾಡಲಾರದವರಾಗಿರುವರೋ ಅಂಥವರೂ ಯಾವುದಾದರೊಂದು ಹಾದಿಯಿಂದ ಕ್ರಮೇಣ ಒಳ್ಳೆಯ ದಾರಿಗೆ ಬರುವುದು ಆವಶ್ಯಕವೆಂದು ತಿಳಿಸಿದರು. ಅವರಿಗಾಗಿ ಕೆಲವು ವಿಲಕ್ಷಣ ತಾಂತ್ರಿಕ ವಿಧಿಗಳನ್ನು ನಿರ್ಮಿಸಿದರು."

ಪ್ರಶ್ನೆ: ಅವರು ಕೆಟ್ಟ ಕಾರ್ಯಗಳನ್ನೆಲಾ ಅವು ಒಳ್ಳೆಯ ಕೆಲಸಗಳೆಂದು ತಿಳಿದು ಮಾಡುತ್ತಾ ಹೋದರೆ ಅವು ಅವರ ಕೆಟ್ಟ ಪ್ರವೃತ್ತಿಗಳನ್ನು ಹೇಗೆ ನಿವಾರಿಸುವುವು?

ಸ್ವಾಮೀಜಿ: ಏಕೆ, ಅವರು ತಮ್ಮ ಪ್ರವೃತ್ತಿಗಳನ್ನು ಬೇರೆ ಮಾರ್ಗಕ್ಕೆ ತಿರುಗಿಸುತ್ತಾರೆ - ದೇವರ ಸಾಕ್ಷಾತ್ಕಾರಕ್ಕಾಗಿಯೇ ಅವರು ಅವುಗಳನ್ನು ಮಾಡುತ್ತಾರೆ.

ಪ್ರಶ್ನೆ: ಇದು ನಿಜವಾಗಿಯೂ ಸಾಧ್ಯವೆ?

ಸ್ವಾಮೀಜಿ: ಅದರ ಗುರಿಯೂ ಒಂದೇ. ಉದ್ದೇಶ ಮಾತ್ರ ಸರಿಯಾಗಿರಬೇಕು. ಅವರು ಜಯಶಾಲಿಗಳಾಗಲು ಯಾವ ಆತಂಕವಿದೆ?

ಪ್ರಶ್ನೆ: ಅಂದರೆ ಎಷ್ಟೋ ಮಂದಿ ಈ ಸಾಧನದ ಮೂಲಕ ಹೋಗಬೇಕೆಂದು ಪ್ರಯತ್ನಿಸಿ, ಕುಡಿತ ಮಾಂಸಭಕ್ಷಣ ಮುಂತಾದುವುಗಳ ಬಲೆಗೆ ಬಿದ್ದಿರುವರು.

ಸ್ವಾಮೀಜಿ: ಅದಕ್ಕೆ ಶ‍್ರೀರಾಮಕೃಷ್ಣರು ಬಂದುದು. ಆ ರೀತಿ ತಂತ್ರ ಸಾಧನೆ ಮಾಡುತ್ತಿದ್ದ ಕಾಲ ಕಳೆದುಹೋಯಿತು. ಅವರೂ ತಾಂತ್ರಿಕ ಸಾಧನೆ ಮಾಡಿದರು, ಈ ದೃಷ್ಟಿಯಿಂದಲ್ಲ. ಎಲ್ಲಿ ಮದ್ಯ ಕುಡಿಯಬೇಕೆಂಬ ಕಟ್ಟಳೆ ಇದೆಯೋ ಅಲ್ಲಿ ಅವರು ಒಂದು ತೊಟ್ಟನ್ನು ಸುಮ್ಮನೆ ತಲೆಗೆ ಸೋಕಿಸಿಕೊಳ್ಳುತ್ತಿದ್ದರು. ತಾಂತ್ರಿಕ ಸಾಧನೆಯ ಪೂಜೆ ಬಹಳ ಕೃತ್ರಿಮದ ಹಾದಿ. ಅದಕ್ಕೇ ನಾನು ಹೇಳಿದ್ದು, ಈ ಪ್ರದೇಶದಲ್ಲಿ ಸಾಕಷ್ಟು ತಾಂತ್ರಿಕ ಸಾಧನೆಯಾಗಿದೆ. ಈಗ ಅದನ್ನು ಮೀರಿ ಹೋಗಬೇಕು, ವೇದಗಳನ್ನು ವ್ಯಾಸಂಗ ಮಾಡಬೇಕು. ನಾಲ್ಕು ಬಗೆಯ ಯೋಗಗಳ ಸಾಮರಸ್ಯವನ್ನು ಸಾಧನೆ ಮಾಡಬೇಕು. ಸಂಪೂರ್ಣ ಜಿತೇಂದ್ರಿಯತ್ವವನ್ನು ಉಳಿಸಿಕೊಳ್ಳಬೇಕು.

ಪ್ರಶ್ನೆ: ನಾಲ್ಕು ಯೋಗಗಳ ಸಾಮರಸ್ಯ ಎಂದರೆ ನಿಮ್ಮ ಅಭಿಪ್ರಾಯವೇನು?

ಸ್ವಾಮೀಜಿ: ಸತ್ಯ ಮಿಥ್ಯೆಗಳ ವಿವೇಚನಾ ಜ್ಞಾನ, ವೈರಾಗ್ಯ, ಭಕ್ತಿ, ಕರ್ಮ, ಧ್ಯಾನ ಮತ್ತು ಸ್ತ್ರೀಯರ ಬಗ್ಗೆ ಪೂಜ್ಯಭಾವನೆ ಇರಬೇಕು.

ಪ್ರಶ್ನೆ: ಸ್ತ್ರೀಯರನ್ನು ಹೇಗೆ ಪೂಜ್ಯ ದೃಷ್ಟಿಯಿಂದ ನೋಡುವುದು?

ಸ್ವಾಮೀಜಿ: ಅವರು ಜಗನ್ಮಾತೆಯ ಪ್ರತಿನಿಧಿ ಸ್ವರೂಪರು. ಎಂದಿನಿಂದ ಜಗನ್ಮಾತೆಯ ನಿಜವಾದ ಪೂಜೆ ಪ್ರಾರಂಭವಾಗುವುದೊ, ಎಂದಿನಿಂದ ಪ್ರತಿಯೊಬ್ಬ ಮನುಷ್ಯನೂ ಮಾತೆಯ ಸಾನ್ನಿಧ್ಯದಲ್ಲಿ ತನ್ನನ್ನು ತಾನು ಅರ್ಪಿಸಿಕೊಳ್ಳುವನೊ, ಅಂದಿನಿಂದ ಭರತಖಂಡದ ನಿಜವಾದ ಏಳ್ಗೆ ಪ್ರಾರಂಭವಾಗುವುದು.

ಪ್ರಶ್ನೆ: ಸ್ವಾಮೀಜಿ, ನೀವು ಹುಡುಗರಾಗಿದ್ದಾಗ, ಮದುವೆಮಾಡಿಕೊಳ್ಳಿ ಎಂದು ನಿಮಗೆ ಹೇಳುತ್ತಿದ್ದಾಗ ನೀವು ‘ಮಾಡಿಕೊಳ್ಳುವುದಿಲ್ಲ, ನಾನೇನಾಗುವೆನೆಂಬುದನ್ನು ನೀವು ನೋಡುತ್ತೀರಿ’ ಎಂದು ಉತ್ತರ ಕೊಡುತ್ತಿದ್ದಿರಿ. ನಿಮ್ಮ ಮಾತುಗಳನ್ನು ಅಕ್ಷರಶಃ ಸಮರ್ಥಿಸಿದಿರಿ.

ಸ್ವಾಮೀಜಿ: ಹೌದು, ನನ್ನ ಪ್ರಿಯ ಸಹೋದರ! ನೀನೇ ನೋಡಿದ್ದೆ, ನನಗೆಷ್ಟು ಆಹಾರದ ಅಭಾವವಿತ್ತು ಎಂಬುದನ್ನು. ಜೊತೆಗೆ ನಾನೆಷ್ಟು ಕಷ್ಟಪಟ್ಟು ಕೆಲಸ ಮಾಡಬೇಕಾಗಿತ್ತು! ಓಹ್! ಅದೆಂತಹ ಕಠೋರವಾದ ದುಡಿತ! ಇಂದು ಅಮೆರಿಕನ್ನರು ಪ್ರೀತಿಯಿಂದ ಈ ಒಳ್ಳೆಯ ಹಾಸಿಗೆಯನ್ನು ಕೊಟ್ಟಿರುವರು. ನನಗೆ ಉಣ್ಣಲೂ ಕೊಂಚ ಇದೆ. ಆದರೆ ಅಯ್ಯಾ, ನನಗೆ ಶಾರೀರಿಕವಾಗಿ ಸುಖಪಡಲು ಹಣೆಯಲ್ಲಿ ಬರೆದಿಲ್ಲ. ಈ ಸುಪ್ಪತ್ತಿಗೆಯಲ್ಲಿ ಮಲಗುವುದರಿಂದ ನನ್ನ ಖಾಯಿಲೆ ಉಲ್ಬಣವಾಗಿ ನನಗೆ ಉಸಿರು ಕಟ್ಟಿದಂತಾಗುವುದು. ಅದರಿಂದ ಬಿಡುಗಡೆ ಹೊಂದಲು ನಾನು ಇಳಿದು ಬಂದು ನೆಲದ ಮೇಲೆ ಮಲಗಬೇಕು.

