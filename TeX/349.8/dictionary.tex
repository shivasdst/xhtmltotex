\sethyphenation{kannada}{
ಅಂಕಿ
ಅಂಕುರಿ-ಸಿತು
ಅಂಕುಶ-ವನಿಡುವ
ಅಂಕೆಗೊಳ-ಪಟ್ಟಿದೆ
ಅಂಕೆ-ಯಲ್ಲಿ-ರುವ
ಅಂಕೆ-ಯಲ್ಲಿ-ರುವುದು
ಅಂಗ
ಅಂಗಕ್ಕೆ
ಅಂಗ-ಗಳಿವೆ
ಅಂಗ-ಗಳು
ಅಂಗಡಿ
ಅಂಗಡಿ-ಯಿಂದ
ಅಂಗಳ
ಅಂಗಳಕ್ಕೆ
ಅಂಗಳದ
ಅಂಗಳ-ದಲ್ಲಿ
ಅಂಗಳ-ದಲ್ಲೇ
ಅಂಗ-ವಾಗಲಿ
ಅಂಗ-ವಾಗಿದೆ-ಯೆಂಬು-ದನ್ನು
ಅಂಗ-ವಾಗಿರ-ಬೇಕು
ಅಂಗ-ವಾಗಿರು-ವುದು
ಅಂಗ-ವೆಂದೇ
ಅಂಗ-ಸಂಗ
ಅಂಗಾಂಗ
ಅಂಗಾಂಗ-ಗಳು
ಅಂಗಾತ
ಅಂಗೀಕರಿಸಿದ
ಅಂಗೀಕರಿಸು-ವಂತೆ
ಅಂಗೀಕರಿಸು-ವುದು
ಅಂಗುಲವೂ
ಅಂಗೈನ
ಅಂಗೈ-ಮೇಲಣ
ಅಂಗೈ-ಮೇಲಿನ
ಅಂಜ-ದಿರುವೆ
ಅಂಜ-ಬೇಕು
ಅಂಜಲೇಕೆ
ಅಂಜಿಕೆ
ಅಂಜಿಕೆಗೆ
ಅಂಜಿಕೆಯದೆನ-ಗೆಲ್ಲಿ
ಅಂಜಿಕೆ-ಯಿಂದ
ಅಂಜಿಕೆಯೇ
ಅಂಜಿಕೆ-ಯೇಕೆ
ಅಂಜಿಕೆ-ಯೇನು
ಅಂಜಿಸುವ
ಅಂಜುತ್ತಿರು-ವಿರಿ
ಅಂಜುಬುರುಕ-ರಂತೆ
ಅಂಜುಬುರುಕರು
ಅಂಜುಬುರುಕರೊ
ಅಂಜುವಿರಿ
ಅಂಜು-ವುದು
ಅಂಜು-ವುದೇ
ಅಂಟಿ-ಕೊಂಡಿರುವ
ಅಂಟಿಸ-ಲಾಗಿ-ರುವ
ಅಂಟು
ಅಂಟು-ವಂತೆ
ಅಂತ
ಅಂತಃಕರಣ
ಅಂತಃಕರಣ-ದಿಂದ
ಅಂತಃಕರಣ-ಪೂರ್ವಕ-ವಾದ
ಅಂತಃಪ್ರವಾಹ-ವೆಂದರೆ
ಅಂತಃಶಕ್ತಿ
ಅಂತಃಶಕ್ತಿ-ಯಿಂದ
ಅಂತಃಸ್ಫೂರ್ತಿ
ಅಂತರಂಗ
ಅಂತ-ರಂಗಕ್ಕೆ
ಅಂತ-ರಂಗದ
ಅಂತರಂಗ-ಶಕ್ತಿ
ಅಂತರ-ದಲ್ಲಿ
ಅಂತರವು
ಅಂತರಾತ್ಮ
ಅಂತರಾತ್ಮ-ನನ್ನು
ಅಂತರಾತ್ಮ-ನೊಂದಿಗಿನ
ಅಂತರಾಳ-ದಲ್ಲಿ
ಅಂತರಾಳ-ದಲ್ಲಿ-ರುವ
ಅಂತರಿಕ
ಅಂತರೆ
ಅಂತರ್ಗತ-ವಾಗಿ-ರುವ
ಅಂತರ್ಗತ-ವಾಗಿ-ರು-ವುವು
ಅಂತರ್ಜಾತೀಯ
ಅಂತರ್ಮುಖ-ವಾಗ-ಬೇಕು
ಅಂತರ್ಮುಖ-ವಾದಷ್ಟೂ
ಅಂತರ್ಮುಖಿ-ಗಳಾದಂತಾಗಿ
ಅಂತರ್-ಜಾತೀಯ
ಅಂತರ್ಜ್ಞಾನ-ದಲ್ಲೇ
ಅಂತರ್ಜ್ಞಾನ-ದಿಂದ
ಅಂತರ್-ದೃಷ್ಟಿ
ಅಂತರ್-ದೃಷ್ಟಿ-ಯಿಂದ
ಅಂತವರು
ಅಂತಸ್ತಿಗೆ
ಅಂತಸ್ತಿನ
ಅಂತಸ್ಸತ್ವ
ಅಂತಸ್ಸತ್ವ-ವಿರು-ವುದು
ಅಂತಹ
ಅಂತಹ-ದನ್ನು
ಅಂತಹ-ವನ
ಅಂತಹ-ವನು
ಅಂತಹ-ವ-ನೊಬ್ಬ
ಅಂತಹ-ವರ
ಅಂತಹ-ವ-ರನ್ನು
ಅಂತಹ-ವ-ರಿಗೂ
ಅಂತಹ-ವ-ರಿಗೆ
ಅಂತಹ-ವರು
ಅಂತಹ-ವ-ರೆಂಬ
ಅಂತಹ-ವರೇ
ಅಂತಹ-ವ-ರೊಂದಿಗೆ
ಅಂತಹ-ವ-ರೊಡನೆ
ಅಂತಹವಳಿ-ಗಾಗಿ
ಅಂತಹ-ವಳಿದ್ದಾಳೆಂದಾಗಲೀ
ಅಂತಹು-ದರಿಂದ
ಅಂತಹುದು
ಅಂತಹು-ದೊಂದು
ಅಂತಿಮ
ಅಂತಿಮ-ವಾಗಿ
ಅಂತು
ಅಂತೂ
ಅಂತೆಯೆ
ಅಂತೆಯೇ
ಅಂತ್ಯ
ಅಂತ್ಯಕ್ರಿಯೆ-ಯಲ್ಲಿ
ಅಂತ್ಯಜ
ಅಂತ್ಯಜ-ನೊಬ್ಬ
ಅಂತ್ಯಜ-ರನ್ನು
ಅಂತ್ಯಜರು
ಅಂತ್ಯ-ದಲ್ಲಿ
ಅಂತ್ಯ-ವಾಗು-ವುದೂ
ಅಂಥ
ಅಂಥದೇ
ಅಂಥಲ್ಲಿ
ಅಂಥವನ
ಅಂಥವ-ನಿಗೆ
ಅಂಥವನು
ಅಂಥವರ
ಅಂಥವ-ರನ್ನು
ಅಂಥವ-ರಲ್ಲಿ
ಅಂಥವ-ರಿಗೂ
ಅಂಥವ-ರಿಗೆ
ಅಂಥವರು
ಅಂಥವರೂ
ಅಂಥವಳು
ಅಂದ
ಅಂದಂದಿನ
ಅಂದ-ಮೇಲೆ
ಅಂದರೆ
ಅಂದ-ವನ್ನು
ಅಂದ-ವಾಗಿ
ಅಂದ-ವಾಗಿದೆ
ಅಂದಿನ
ಅಂದಿನಂತೆ
ಅಂದಿನಿಂದ
ಅಂದಿ-ನಿಂದಲು
ಅಂದಿ-ನಿಂದಲೂ
ಅಂದು
ಅಂದೆ
ಅಂದೇ
ಅಂಧ
ಅಂಧಕಾರ
ಅಂಧಕಾರದ
ಅಂಧಕಾರ-ದಲ್ಲಿ
ಅಂಧಕಾರ-ದಿಂದಾವೃತ-ವಾಗಿವೆ
ಅಂಧಕಾರ-ಮಯ-ವಾದಂತೆ
ಅಂಧಕಾರ-ವನ್ನೇ
ಅಂಧಕಾರವು
ಅಂಧಕಾರ-ಸ್ವರೂಪ
ಅಂಧಕಾರ್
ಅಂಧನ
ಅಂಧಾರೆ
ಅಂಧೇನೈವ
ಅಂಬಾ
ಅಂಬು-ರಾಶಿ
ಅಂಶ
ಅಂಶ-ಗಳನ್ನು
ಅಂಶ-ಗಳಿವೆ
ಅಂಶ-ಗಳೆಂದು
ಅಂಶ-ರೂಪ-ವಾದ
ಅಂಶ-ವನ್ನು
ಅಂಶ-ವನ್ನೇ
ಅಂಶ-ವನ್ನೇನೊ
ಅಂಶ-ವೆಂದರೆ
ಅಂಶ-ವೆಂದು
ಅಂಶವೇ
ಅಂಶಾವತಾರ
ಅಕಕ್ಷಣ
ಅಕಲಿ-ತಮ-ಹಿ-ಮಾನಃ
ಅಕಾಂಕ್ಷೆ-ಯುಳ್ಳವ-ರಾಗಿ-ರುವರೋ
ಅಕಾಲ-ದಲ್ಲಿ
ಅಕಿಂಚನತೆ
ಅಕಿಂಚಿನ
ಅಕುಲ
ಅಕೌಂಟೆಂಟ್
ಅಕ್ಕಿ
ಅಕ್ಕಿಯ
ಅಕ್ಕಿ-ಯನ್ನು
ಅಕ್ಕಿ-ಹಿಟ್ಟನ್ನು
ಅಕ್ರಮ
ಅಕ್ಷಯ
ಅಕ್ಷಯ-ವಾಗಿ-ರುವು-ದಾದರೂ
ಅಕ್ಷರಕ್ಷರ-ದಲ್ಲಿಯೂ
ಅಕ್ಷರ-ದಿಂದ
ಅಕ್ಷರ-ವನ್ನು
ಅಕ್ಷರಶಃ
ಅಖಂಡ
ಅಖಂಡದ
ಅಖಂಡ-ವನ್ನು
ಅಖಂಡ-ವರ್ಗ
ಅಖಂಡ-ವಲ್ಲ
ಅಖಂಡ-ವಾಗಿ-ರುವುದು
ಅಖಂಡಾ-ನಂದ
ಅಖಿಲ
ಅಗಣನ-ಬಹು-ರೂಪೋ
ಅಗಣಿತ-ವಾದ
ಅಗತ್ಯ
ಅಗತ್ಯ-ವಾಗಿ
ಅಗತ್ಯ-ವಾದ
ಅಗತ್ಯವೇ
ಅಗಮ್ಯ
ಅಗಸಗಿತ್ತಿ-ಯನ್ನು
ಅಗಾಧ
ಅಗಾಧತೆ
ಅಗಾಧ-ವಾದ
ಅಗಾಮಿ
ಅಗುವಾನ್
ಅಗೆಜಾಯ್
ಅಗೆಯಲು
ಅಗೆಯು-ತ್ತಿದ್ದುದನ್ನು
ಅಗೋ
ಅಗೋಚರ
ಅಗೋಚರಕ
ಅಗೋಚರ-ವಾ
ಅಗೌರವಕ್ಕೆ
ಅಗೌರವ-ದಿಂದ
ಅಗ್ನಿ
ಅಗ್ನಿ-ಕುಂಡ-ದಿಂದ
ಅಗ್ನಿ-ಕುಂಡ-ವನ್ನು
ಅಗ್ನಿಜ್ವಾಲೆ
ಅಗ್ನಿಮಾಂದ್ಯ
ಅಗ್ನಿಯ
ಅಗ್ನಿ-ಯಂತೆ
ಅಗ್ನಿ-ಯನ್ನು
ಅಗ್ನಿ-ಶಿಖಾ
ಅಗ್ನಿ-ಶಿಖೆ-ಯನು
ಅಗ್ರಗಣ್ಯ-ರೆಂದು
ಅಗ್ರಸರ
ಅಘಟನ
ಅಘಟನ-ಘಟನ-ಗಳು
ಅಘದೂಷಣ
ಅಘೋರ
ಅಚಲ
ಅಚಲ-ನಿರ್ಧಾರ-ದಿಂದ
ಅಚಲನು
ಅಚಲ-ಭಕ್ತಿ
ಅಚಲ-ವಾದ
ಅಚಿರ
ಅಚೇತನ-ರಾಗಿದ್ದು
ಅಚ್ಚ
ಅಚ್ಚಳಿ-ಯದೆ
ಅಚ್ಚಾಗುವುದಕ್ಕೆ
ಅಚ್ಚಿಗೆ
ಅಚ್ಚಿ-ನಲ್ಲಿ
ಅಚ್ಚು
ಅಜರಾಮರ
ಅಜಸ್ರ
ಅಜೀರ್ಣ
ಅಜೀರ್ಣತೆ-ಯನ್ನಲ್ಲ
ಅಜೀರ್ಣ-ದಿಂದ
ಅಜೀರ್ಣ-ವಾಗು-ವುದು
ಅಜೀರ್ಣವ್ಯಾಧಿ-ಯುಳ್ಳ
ಅಜೇಯತಾವಾದಿ-ಗಳೆಂದೆ-ನಿಸಿ-ಕೊಳ್ಳುವ
ಅಜೇಯತಾವಾದಿ-ಯಾಗಿದ್ದನು
ಅಜೇಯರು
ಅಜೇಯ-ವನ್ನು
ಅಜ್ಜ
ಅಜ್ಞತೆಯು
ಅಜ್ಞತೆಯೇ
ಅಜ್ಞರಾದ
ಅಜ್ಞರಿಗೆ
ಅಜ್ಞಾನ
ಅಜ್ಞಾನ-ಕೂಪ-ದಲ್ಲಿ
ಅಜ್ಞಾನಕ್ಕೆ
ಅಜ್ಞಾನ-ಗಳಲ್ಲಿ
ಅಜ್ಞಾನದ
ಅಜ್ಞಾನ-ದಲ್ಲಿ
ಅಜ್ಞಾನ-ದಲ್ಲಿ-ರುವೆಯೋ
ಅಜ್ಞಾನ-ದಿಂದ
ಅಜ್ಞಾನ-ದಿಂದಲೆ
ಅಜ್ಞಾನ-ದಿಂದಲೇ
ಅಜ್ಞಾನ-ದಿಂದಾಗಿ
ಅಜ್ಞಾನ-ವನ್ನು
ಅಜ್ಞಾನ-ವಿದೆ-ಯಲ್ಲಾ
ಅಜ್ಞಾನವೂ
ಅಜ್ಞಾನ-ವೆಲ್ಲಿದೆ
ಅಜ್ಞಾನವೇ
ಅಜ್ಞೇಯ
ಅಜ್ಞೇಯ-ವಾದು-ದನ್ನು
ಅಜ್ಞೇಯ-ವೆಂದರೆ
ಅಟಾಟೋಪ-ದಿಂದ
ಅಟಾರ್ನಿ
ಅಟ್ಟ-ಹಾಸ
ಅಟ್ಟಿದ-ನೆಂದುಕೋ
ಅಡಗಿದೆ
ಅಡಗಿರು-ತ್ತವೆ
ಅಡಗಿಸಿಟ್ಟರೆ
ಅಡಗಿಸಿ-ದ್ದಾರೆ
ಅಡಗಿ-ಹೋಗಿದೆ
ಅಡಗಿ-ಹೋಯಿತು
ಅಡಗು-ತ್ತದೆ
ಅಡಚಣೆ-ಗಳನ್ನು
ಅಡಿ
ಅಡಿಗಳು
ಅಡಿಗೆ
ಅಡಿಗೆಗೆ
ಅಡಿಗೆಯ
ಅಡಿಗೆ-ಯಾಗು-ತ್ತಿದೆ
ಅಡಿ-ದಾವರೆ-ಗಳನ್ನು
ಅಡಿ-ದಾವರೆಗೆ
ಅಡಿ-ದಾವರೆ-ಯಲ್ಲಿ
ಅಡಿ-ಮುಡಿ-ಗಳ-ದಕಿಲ್ಲ
ಅಡಿಯ
ಅಡುಗೆ
ಅಡುಗೆ-ಗಳನ್ನೆಲ್ಲಾ
ಅಡುಗೆಗೆ
ಅಡುಗೆಯ
ಅಡುಗೆ-ಯನ್ನು
ಅಡುಗೆ-ಯಲ್ಲಿ
ಅಡುಗೆ-ಯಾಗುವು-ದಿಲ್ಲ
ಅಡುಗೆ-ಯಾಗುವುದು
ಅಡ್ಡ-ದಾರಿ-ಗಳ
ಅಡ್ಡ-ದಾರಿ-ಯಲ್ಲಿರ-ಲಾರರು
ಅಡ್ಡ-ವಾಗುವು-ದಿಲ್ಲ
ಅಡ್ಡ-ಹಾದಿಗೆ
ಅಡ್ಡಾಡ-ತೊಡಗಿ-ದರು
ಅಡ್ಡಾಡಲು
ಅಡ್ಡಾಡುತ್ತಿದ್ದಾಗ
ಅಡ್ಡಿ
ಅಡ್ಡಿ-ಗಳನ್ನು
ಅಡ್ಡಿ-ಯನ್ನೂ
ಅಡ್ಡಿ-ಯಾಗ-ಬೇಡಿ
ಅಡ್ಡಿ-ಯಾಗಿ
ಅಡ್ಡಿ-ಯಿಲ್ಲ
ಅಡ್ಡಿಯೂ
ಅಡ್ಡಿ-ಯೇನೂ
ಅಣಕ
ಅಣಿ-ಮಾಡಿ
ಅಣಿ-ಯಾಗುವುದು
ಅಣಿ-ಯಾಗುವುವು
ಅಣಿ-ಯಾಗುವೆವು
ಅಣು
ಅಣು-ಮಾತ್ರ-ವಾಗಿ
ಅಣು-ವನ್ನು
ಅಣು-ವಿನಲಿ
ಅಣು-ವಿನಲ್ಲಿ
ಅಣ್ಣ
ಅಣ್ಣ-ತಮ್ಮಂದಿರು
ಅತ
ಅತರ್ಕ್ಯ
ಅತಿ
ಅತಿಕ್ರಮಿಸಿ
ಅತಿ-ಗಳಿಂದ
ಅತಿಗೆ
ಅತಿಥಿ
ಅತಿಥಿ-ಗಳಿಗೆ
ಅತಿಥಿ-ಯಾಗಿದ್ದರು
ಅತಿಥಿ-ಯಾಗಿದ್ದಾಗ
ಅತಿಥಿ-ಯಾಗಿದ್ದೆ
ಅತಿ-ದೂರ
ಅತಿ-ನಿಂದಿತ
ಅತಿನೀಚ
ಅತಿ-ಪಕ್ಷ-ಪಾತಿ-ಗಳಾದ
ಅತಿಪ್ರಜ್ಞೆಯ
ಅತಿ-ಮರುಳ-ನಾಗಿರಲು
ಅತಿಮಾನುಷ
ಅತಿಯಾಗಿ
ಅತಿರೇಕ-ಗಳೆರಡೂ
ಅತಿರೇಕ-ವಾದ
ಅತಿರೇಕ-ವಾದುದು
ಅತಿ-ವಿಕಲಿತ-ರೂಪಂ
ಅತಿಶಯ-ವಾಗಿ
ಅತಿಶಯ-ವಾದ
ಅತಿಸಾರ-ದಿಂದ
ಅತೀಂದ್ರಿಯ
ಅತೀಂದ್ರಿಯ-ದರ್ಶಿಗಳು
ಅತೀಂದ್ರಿಯಾವಸ್ಥೆ
ಅತೀಂದ್ರಿಯಾವಸ್ಥೆಯ
ಅತೀಂದ್ರಿಯಾವಸ್ಥೆ-ಯಲ್ಲಿದ್ದು
ಅತೀತ
ಅತೀತನು
ಅತೀತ-ವಾಗಿ
ಅತೀತ-ವಾಗಿರು-ವುದನ್ನು
ಅತೀವ
ಅತುಲ-ಬಾಬು
ಅತುಲ-ಬಾಬು-ಗಳು
ಅತ್ತ
ಅತ್ತ-ಲಾಗಿ
ಅತ್ತೆ
ಅತ್ತೆಯರು
ಅತ್ಯಂತ
ಅತ್ಯಂತ-ವಾಗಿ
ಅತ್ಯಂತ-ವಾದ
ಅತ್ಯಧಿಕ-ವಾದ
ಅತ್ಯಲ್ಪ
ಅತ್ಯಲ್ಪ-ಮತಿ-ಯಾದ
ಅತ್ಯವಶ್ಯಕ
ಅತ್ಯಾಚಾರ
ಅತ್ಯಾಚಾರ-ಗಳನ್ನು
ಅತ್ಯಾಚಾರ-ವನ್ನು
ಅತ್ಯಾನಂದ-ದಲ್ಲಿ
ಅತ್ಯಾನಂದ-ದಿಂದ
ಅತ್ಯಾವಶ್ಯಕ
ಅತ್ಯಾವಶ್ಯಕ-ವಾಗಿ
ಅತ್ಯಾವಶ್ಯಕ-ವಾಗಿದೆ
ಅತ್ಯುಚ್ಚ
ಅತ್ಯುಚ್ಚ-ವಾದುದು
ಅತ್ಯುತ್ತಮ
ಅತ್ಯುತ್ತಮ-ವಾದ
ಅತ್ಯುತ್ತಮ-ವಾದು-ದೆಂದು
ಅತ್ಯುನ್ನತ
ಅಥವಾ
ಅದ
ಅದಕೆ
ಅದಕೋಸ್ಕರವೇ
ಅದಕ್ಕ-ವನು
ಅದಕ್ಕಾಗಿ
ಅದಕ್ಕಾಗಿಯೇ
ಅದಕ್ಕಾತ
ಅದಕ್ಕಿಂತ
ಅದಕ್ಕಿಂತಲೂ
ಅದಕ್ಕುತ್ತರ-ವಾಗಿ
ಅದಕ್ಕೂ
ಅದಕ್ಕೆ
ಅದಕ್ಕೇ
ಅದಕ್ಕೊಂದು
ಅದಕ್ಕೋಸ್ಕರ
ಅದಕ್ಕೋಸ್ಕರವೆ
ಅದಕ್ಕೋಸ್ಕರವೇ
ಅದನಾರು
ಅದನು
ಅದನೊಂದು
ಅದನ್ನು
ಅದನ್ನೂ
ಅದನ್ನೆ
ಅದನ್ನೆಲ್ಲಾ
ಅದನ್ನೇ
ಅದನ್ನೊಂದು
ಅದಮ್ಯ
ಅದಮ್ಯ-ವಾದ
ಅದರ
ಅದರಂತಯೆ
ಅದ-ರಂತೆ
ಅದ-ರಂತೆಯೆ
ಅದ-ರಂತೆಯೇ
ಅದರದೇ
ಅದರ-ದೊಂದು
ಅದರಲ್ಲಿ
ಅದರಲ್ಲಿಟ್ಟಿ-ರುವ
ಅದರಲ್ಲಿದೆ
ಅದರಲ್ಲಿಯೂ
ಅದರಲ್ಲಿಯೆ
ಅದರಲ್ಲಿಯೇ
ಅದರಲ್ಲಿ-ರುವ
ಅದರಲ್ಲಿರೂ
ಅದರಲ್ಲೂ
ಅದರಲ್ಲೇ
ಅದರಾಚೆ
ಅದರಾಚೆಗೆ
ಅದ-ರಿಂದ
ಅದ-ರಿಂದಲೂ
ಅದ-ರಿಂದಲೆ
ಅದ-ರಿಂದಲೇ
ಅದ-ರಿಂದಾಚೆ
ಅದ-ರಿಂದಾಚೆಗೆ
ಅದ-ರಿಂದುಂಟಾಗುವ
ಅದರಿಂದೇ-ನಾಗು-ವುದು
ಅದರಿಂದೇ-ನಾದರೂ
ಅದ-ರಿಂದೇನು
ಅದ-ರಿಂದೇನೂ
ಅದ-ರೊಡನೆ
ಅದ-ರೊಡನೆಯೇ
ಅದ-ರೊಳಗೆ
ಅದಲ್ಲದೆ
ಅದಾದ
ಅದಾದ-ನಂತರ
ಅದಾದ-ಮೇಲೆ
ಅದಿರಲಿ
ಅದಿರು-ತ್ತಿರಲು
ಅದಿಲ್ಲ
ಅದಿಲ್ಲದೆ
ಅದು
ಅದುಮಿ-ಕೊಂಡು
ಅದುಮಿಡಲು
ಅದುವೆ
ಅದೂ
ಅದೃಷ್ಟ
ಅದೃಷ್ಟವಶಾತ್
ಅದೃಷ್ಟಶಾಲಿ-ಗಳಾದು-ದರಿಂದ
ಅದೃಷ್ಟಶಾಲಿಯೇ
ಅದೆ
ಅದೆಂತಹ
ಅದೆಂದರೆ
ಅದೆಂದಿಗೂ
ಅದೆಂದೂ
ಅದೆಲ್ಲ
ಅದೆಲ್ಲ-ದರ
ಅದೆಲ್ಲ-ವನ್ನೂ
ಅದೆಲ್ಲವೂ
ಅದೆಲ್ಲಾ
ಅದೆಲ್ಲಿ
ಅದೆಷ್ಟೇ
ಅದೇ
ಅದೇಕೆ
ಅದೇ-ನಾದರೂ
ಅದೇ-ನಿದ್ದರೂ
ಅದೇನು
ಅದೇನೂ
ಅದೇ-ನೆಂದರೆ
ಅದೇ-ನೆಂದು
ಅದೇ-ನೇನೂ
ಅದೇನೊ
ಅದೇನೋ
ಅದೇ-ರೀತಿ
ಅದೇ-ರೀತಿಯ
ಅದೊ
ಅದೊಂದಕ್ಕೂ
ಅದೊಂದು
ಅದೊಂದೇ
ಅದೋ
ಅದ್ಭುತ
ಅದ್ಭುತದ
ಅದ್ಭುತ-ವನ್ನು
ಅದ್ಭುತ-ವಾಗಿ
ಅದ್ಭುತ-ವಾದ
ಅದ್ಭುತ-ವಾದದ್ದು
ಅದ್ಭುತ-ವಾದುದು
ಅದ್ಭುತ-ವಾದುವು
ಅದ್ಭುತ-ವಿರು-ವುದಿಲ್ಲ
ಅದ್ಭುತ-ವೆಂಬ
ಅದ್ಭುತವೇ
ಅದ್ಭುತ-ಶಕ್ತಿ
ಅದ್ಯೈವ
ಅದ್ವಯ
ಅದ್ವಯ-ತತ್ತ್ವ-ಮಾಹಿತ-ಚಿತ್ತಂ
ಅದ್ವಿತೀಯರು
ಅದ್ವೈತ
ಅದ್ವೈತದ
ಅದ್ವೈತ-ದಲ್ಲಿ
ಅದ್ವೈತ-ಭಾವನೆಗೆ
ಅದ್ವೈತ-ಮತವು
ಅದ್ವೈತ-ವನ್ನು
ಅದ್ವೈತ-ವಾದದ
ಅದ್ವೈತ-ವಾದ-ವನ್ನು
ಅದ್ವೈತ-ವಿದೆ
ಅದ್ವೈತಾವಸ್ಥೆ-ಯಲ್ಲಿ
ಅದ್ವೈತಿ-ಗಳ
ಅದ್ವೈತಿ-ಗಳನ್ನು
ಅದ್ವೈತಿ-ಗಳು
ಅದ್ವೈತಿಯೂ
ಅಧಃಪತನ
ಅಧಃಪತನ-ಕ್ಕಾಗಿ
ಅಧಃಪತನಕ್ಕೆ
ಅಧಃಪತನದ
ಅಧಃಪತನವೇ
ಅಧಃಪತಿತ-ನಾದ
ಅಧಃಪತಿತ-ರಾದ
ಅಧಮ-ನಿಗೆ
ಅಧರ
ಅಧರ್ಮ
ಅಧಿಕ
ಅಧಿಕ-ವಾಗಿಯೇ
ಅಧಿಕ-ವಾದ
ಅಧಿಕಾಂಶ
ಅಧಿಕಾರ
ಅಧಿಕಾರ-ಕ್ಕಾಗಿ
ಅಧಿಕಾರದ
ಅಧಿಕಾರ-ದಿಂದ
ಅಧಿಕಾರ-ಭೇದದ
ಅಧಿಕಾರ-ವನ್ನು
ಅಧಿಕಾರ-ವನ್ನೂ
ಅಧಿಕಾರ-ವರ್ಗ
ಅಧಿಕಾರ-ವಿದೆ
ಅಧಿಕಾರ-ವಿರ-ಲಿಲ್ಲ
ಅಧಿಕಾರ-ವಿರು-ವುದಿಲ್ಲ
ಅಧಿಕಾರ-ವಿರು-ವುದೋ
ಅಧಿಕಾರ-ವಿಲ್ಲ
ಅಧಿಕಾರವು
ಅಧಿಕಾರವೂ
ಅಧಿಕಾರ-ಶಾಹಿಯೇ
ಅಧಿಕಾರಿ
ಅಧಿಕಾರಿ-ಗಳಾಗಿ
ಅಧಿಕಾರಿ-ಗಳಾಗಿರು-ವವರಿಗೆ
ಅಧಿಕಾರಿ-ಗಳು
ಅಧಿಕಾರಿ-ಯಾಗಿ-ರುವೆ
ಅಧಿಕಾರಿ-ಯಾಗು-ವುದಿಲ್ಲ
ಅಧಿಕಾರಿ-ಯೆಂದು
ಅಧಿಕಾರಿಯೇ
ಅಧಿಪತ್ಯದಡಿ
ಅಧಿವಾಸ-ಕ್ರಿಯೆ-ಯೆಂದು
ಅಧಿವೇಶನವು
ಅಧಿಷ್ಠಾನ
ಅಧಿಷ್ಠಾನ-ವನ್ನು
ಅಧೀನ
ಅಧೀನಕ್ಕೆ
ಅಧೀನತೆಯೂ
ಅಧೀನ-ದಲ್ಲಿ
ಅಧೀನ-ದಲ್ಲಿಟ್ಟು-ಕೊಳ್ಳಿ
ಅಧೀನ-ದಲ್ಲಿದ್ದಾಗ
ಅಧೀನ-ದಲ್ಲಿದ್ದುವು
ಅಧೀನ-ದಲ್ಲಿ-ರುವ
ಅಧೀನ-ದಲ್ಲಿ-ರುವುವು
ಅಧೀನ-ವಾಗಿ-ಬಿಡು-ವುದೆಂಬು-ದನ್ನು
ಅಧೀನಾವಸ್ಥೆಯ
ಅಧೀನಾವಸ್ಥೆ-ಯಿಂದಲೂ
ಅಧುನೈವ
ಅಧೋಗತಿ
ಅಧೋಗತಿ-ಗಿಳಿ-ದಿದೆ
ಅಧೋಗತಿ-ಗಿಳಿ-ದಿದ್ದಾರೆ
ಅಧೋಗತಿ-ಗಿಳಿ-ದಿರು-ವರು
ಅಧೋಗತಿ-ಗಿಳಿ-ದಿರು-ವುದಕ್ಕೆ
ಅಧೋಗತಿ-ಗಿಳಿಯಲು
ಅಧೋಗತಿ-ಗಿಳಿ-ಯುವುದು
ಅಧೋಗತಿ-ಗಿಳಿ-ಯುವೆವು
ಅಧೋಗತಿಗೆ
ಅಧೋಮುಖ-ವಾಗಿ
ಅಧ್ಯಕ್ಷರಿ-ಗೊಂದು
ಅಧ್ಯಕ್ಷರೂ
ಅಧ್ಯಕ್ಷಿಣಿಯ
ಅಧ್ಯಯನ
ಅಧ್ಯಯನಕ್ಕೆ
ಅಧ್ಯಯನ-ದಲ್ಲಿ
ಅಧ್ಯಯನ-ದಿಂದ
ಅಧ್ಯಯನ-ವನ್ನು
ಅಧ್ಯವಸಾಯ-ವುಳ್ಳ
ಅಧ್ಯಸ್ತ-ವಾಗಿದೆ
ಅಧ್ಯಸ್ತ-ವಾಗಿವೆ
ಅಧ್ಯಾತ್ಮ
ಅಧ್ಯಾತ್ಮದ
ಅಧ್ಯಾತ್ಮ-ದಲ್ಲಿ
ಅಧ್ಯಾಪಕ-ರಾಗುವರು
ಅಧ್ಯಾಪಕ-ರಾದ
ಅಧ್ಯಾಯ
ಅಧ್ಯಾಯದ
ಅಧ್ಯಾಯ-ವನ್ನು
ಅಧ್ಯಾಸ-ವೆಲ್ಲ
ಅನಂತ
ಅನಂತ-ಕಾಲದ-ವರೆಗಿನ
ಅನಂತಜ್ಞಾನ
ಅನಂತತೆ-ಯಲ್ಲಿ-ರುವೆವು
ಅನಂತ-ದೊಂದಿಗೆ
ಅನಂತ-ನಾದ
ಅನಂತಪ್ರೇಮದ
ಅನಂತ-ಭಾವ-ಮಯ-ರಾದ
ಅನಂತ-ಭಾವ-ಮಯ-ರಾದ-ವರು
ಅನಂತರ
ಅನಂತರ-ದಲ್ಲಿ
ಅನಂತರದ್ದೆ
ಅನಂತರವೇ
ಅನಂತ-ವನ್ನೂ
ಅನಂತ-ವಾಗಿ
ಅನಂತ-ವಾಗಿರು-ವುದೋ
ಅನಂತ-ವಾದ
ಅನಂತ-ವೃತ್ತ-ದಂತೆ
ಅನಂತ-ಶಕ್ತಿ
ಅನಂತ-ಸಾಗರಕ್ಕೂ
ಅನಂತ-ಸಾಗರದ
ಅನಂತೇರ್
ಅನನ
ಅನನ್ಯ
ಅನರ್ಥ
ಅನರ್ಥ-ವಾದದ್ದು
ಅನರ್ಹ-ರೆಂದು
ಅನರ್ಹ-ವಾಗಿದೆ
ಅನಲ
ಅನವರತ
ಅನವರತವೂ
ಅನಾಗರಿಕರು
ಅನಾಚಾರಿ-ಗಳು
ಅನಾತ್ಮ-ಗಳ
ಅನಾಥ
ಅನಾಥ-ನೆಂದು
ಅನಾಥಾಲಯ
ಅನಾದಿ
ಅನಾದಿ-ಕಾಲ-ದಿಂದಲೂ
ಅನಾದಿತ್ವ-ವನ್ನು
ಅನಾದಿ-ನಾದ-ವೆಂಬುದು
ಅನಾದಿ-ಯಾದ
ಅನಾಮಧೇಯರು
ಅನಾರೋಗ್ಯ-ದಿಂದಿದ್ದರೂ
ಅನಾವರಣ
ಅನಾವೃತ್ತಿಃ
ಅನಾಸಕ್ತ-ನಾಗಿ
ಅನಾಹತ
ಅನಾಹತ-ನಾದ
ಅನಾಹುತ
ಅನಿಂದಿತ-ವಾದ
ಅನಿತನಿತು
ಅನಿತ್ಯ
ಅನಿತ್ಯಂ
ಅನಿತ್ಯ-ವಾದು-ದರಿಂದ
ಅನಿಬಾರ
ಅನಿರೀಕ್ಷಿತ-ವಾಗಿ
ಅನಿರ್ಬಂಧ-ವಾಗಿ
ಅನಿರ್ಬಂಧ-ವಾದ
ಅನಿರ್ವಚನೀಯ-ವಾದ
ಅನಿವಾರ್ಯ
ಅನಿವಾರ್ಯ-ವಾಗಿ
ಅನಿವಾರ್ಯ-ವಾಗಿದೆ
ಅನಿವಾರ್ಯ-ವಾಗುವ
ಅನಿವಾರ್ಯ-ವಾದ
ಅನಿವಾರ್ಯ-ವಾದರೆ
ಅನಿವಾರ್ಯ-ವಾಯಿತು
ಅನಿಶ್ಚಿತ
ಅನು
ಅನುಕ
ಅನುಕಂಪ
ಅನುಕಂಪ-ದಿಂದ
ಅನುಕಂಪ-ವನ್ನು
ಅನು-ಕರಣ
ಅನು-ಕರಣೆ
ಅನು-ಕರಿಸಲು
ಅನು-ಕರಿಸಹೊಗಿ
ಅನು-ಕರಿಸ-ಹೋಗಿ
ಅನು-ಕರಿಸು-ವಿರಿ
ಅನು-ಕೀಟ
ಅನು-ಕೂಲ
ಅನು-ಕೂಲ-ಗಳನ್ನು
ಅನು-ಕೂಲ-ಗಳನ್ನೂ
ಅನು-ಕೂಲ-ವಾಗಿತ್ತು
ಅನು-ಕೂಲ-ವಾದಾಗ
ಅನು-ಕೂಲ-ವಿದ್ದ-ವರು
ಅನು-ಕೂಲ-ವಿರ-ಲಿಲ್ಲ
ಅನು-ಕೂಲ-ವೆಂದಾದರೆ
ಅನು-ಕೂಲಿ-ಸುವು-ದಿಲ್ಲ-ವೆಂಬು-ದನ್ನು
ಅನುಕ್ಷಣ
ಅನು-ಗುಣ-ವಾಗಿ
ಅನು-ಗುಣ-ವಾಗಿಯೇ
ಅನು-ಗುಣ-ವಾಗಿ-ರು-ವುದೋ
ಅನುಗ್ರಹ-ದಿಂದ
ಅನುಗ್ರಹಿಸಿ-ದರೆ
ಅನುಗ್ರಹಿಸು-ವವಳಾಗಲಿ
ಅನು-ಚರರು
ಅನು-ಚಿತ
ಅನು-ಚಿತ-ವೆಂದು
ಅನುತ್ತೀರ್ಣ-ರಾದ-ವರು
ಅನು-ದಿನವೂ
ಅನುಪಮ
ಅನುಭವ
ಅನುಭವಕ್ಕೆ
ಅನುಭವ-ಕ್ಕೋಸ್ಕರ
ಅನು-ಭವ-ಗಳ
ಅನು-ಭವ-ಗಳನ್ನು
ಅನು-ಭವ-ಗಳನ್ನೂ
ಅನು-ಭವ-ಗಳು
ಅನು-ಭವದ
ಅನು-ಭವ-ದಿಂದ
ಅನು-ಭವ-ದಿಂದಲೇ
ಅನು-ಭವ-ಮಾಡಿ-ಕೊಳ್ಳ-ಬೇಕಾದ
ಅನು-ಭವ-ವನ್ನು
ಅನು-ಭವ-ವಾಗಿದೆ
ಅನು-ಭವ-ವಾಗುತ್ತಿದೆ
ಅನು-ಭವ-ವಾಗುತ್ತಿರು-ವುದಿಲ್ಲ-ವೇಕೆ
ಅನು-ಭವ-ವಾಗು-ವುದು
ಅನು-ಭವ-ವಾದರೆ
ಅನು-ಭವ-ವಾಯಿತು
ಅನು-ಭವ-ವಿರು-ತ್ತದೆಯೋ
ಅನು-ಭವ-ವೆಂಥಾದ್ದು
ಅನು-ಭವವೇ
ಅನು-ಭವಿಸದ
ಅನು-ಭವಿಸದಿಹ
ಅನು-ಭವಿಸ-ಬಲ್ಲ
ಅನು-ಭವಿಸ-ಬಹು-ದಾದ
ಅನು-ಭವಿಸ-ಬಹುದು
ಅನು-ಭವಿಸ-ಬೇಕು
ಅನು-ಭವಿಸಲು
ಅನು-ಭವಿಸಲೇ
ಅನು-ಭವಿಸಿ
ಅನು-ಭವಿಸಿದ
ಅನು-ಭವಿಸಿಯೇ
ಅನು-ಭವಿಸಿರು-ವರು
ಅನು-ಭವಿಸಿ-ರುವೆ
ಅನು-ಭವಿಸಿಲ್ಲ
ಅನು-ಭವಿಸುತಿದ್ದ-ನವನು
ಅನು-ಭವಿಸುತ್ತಾ
ಅನು-ಭವಿಸು-ತ್ತಿರು-ವಂತೆ
ಅನು-ಭವಿಸುತ್ತಿ-ರುವ-ವ-ರಲ್ಲಿಯೂ
ಅನು-ಭವಿಸುತ್ತೇವೆ
ಅನು-ಭವಿಸುವ
ಅನು-ಭವಿಸು-ವರು
ಅನು-ಭವಿಸು-ವರೋ
ಅನು-ಭವಿಸು-ವವನು
ಅನು-ಭವಿಸು-ವುದರ
ಅನು-ಭಾವ-ದನು-ಭವವು
ಅನು-ಭೂತಿ-ಯನ್ನು
ಅನು-ಭೂತಿ-ಯಾಗುತ್ತದೆ
ಅನು-ಮತಿ
ಅನು-ಮತಿ-ಯನ್ನು
ಅನು-ಮತಿ-ಯಿಂದ
ಅನು-ಮಾನ
ಅನು-ಮಾನ-ಗಳನ್ನೇ
ಅನು-ಮಾನ-ಪಡುತ್ತಿದ್ದನು
ಅನು-ಮಾನ-ವಿಲ್ಲ
ಅನು-ಮಾನವೇ
ಅನು-ಮಾನಾಸ್ಪದ
ಅನು-ಮಾನಿಸ-ಲಿಲ್ಲ
ಅನು-ಮೋದಿ-ಸಲು
ಅನು-ಮೋದಿಸಿದ್ದಾರೆ-ಯೇನು
ಅನು-ಮೋದಿಸು-ವುದಿಲ್ಲ
ಅನುಯಾಯಿ-ಗಳಲ್ಲಿ
ಅನುಯಾಯಿ-ಗಳಲ್ಲಿಯೂ
ಅನುಯಾಯಿ-ಗಳಷ್ಟು
ಅನುಯಾಯಿ-ಗಳಿಂದಲೂ
ಅನುಯಾಯಿ-ಗಳು
ಅನು-ರಕ್ತ-ನಾದೆ
ಅನು-ರಕ್ತ-ರಾಗುವರು
ಅನು-ರಣಿತ-ವಾಗುತ್ತಿದೆ
ಅನು-ರಾಗ
ಅನು-ರಾಗದ
ಅನು-ರಾಗ-ವನ್ನು
ಅನು-ರಾಗವು
ಅನು-ರಾಗಿ-ಯಾದ
ಅನು-ರೂಪ-ವಾದ
ಅನು-ವಾಗು-ವಂತೆ
ಅನು-ವಾದ
ಅನು-ವಾದಿ-ಸಿದ್ದಾರೆ
ಅನುವು
ಅನು-ಶಾಸನದ
ಅನು-ಶಾಸಿಸಲ್ಪಟ್ಟಿ-ದೆಯೋ
ಅನುಷ್ಠಾನ
ಅನುಷ್ಠಾನ-ಕರ್ಮ
ಅನುಷ್ಠಾನಕ್ಕೆ
ಅನುಷ್ಠಾನ-ಗಳಿಗೂ
ಅನುಷ್ಠಾನ-ಗಳೇ
ಅನುಷ್ಠಾನ-ಗೊಳಿಸ-ಬೇಕು
ಅನುಷ್ಠಾನದ
ಅನುಷ್ಠಾನ-ದಲ್ಲಿ
ಅನುಷ್ಠಾನ-ಮಾಡಿ
ಅನುಷ್ಠಾನ-ರಂಗಕ್ಕೆ
ಅನು-ಸರಣ
ಅನು-ಸರಿಸ-ದಿದ್ದಲ್ಲಿ
ಅನು-ಸರಿಸ-ಬ-ಹುದು
ಅನು-ಸರಿಸ-ಬೇಕಾಗಿದೆ
ಅನು-ಸರಿಸ-ಬೇಕು
ಅನು-ಸರಿಸಲು
ಅನು-ಸರಿಸಲೆ
ಅನು-ಸ-ರಿಸಿ
ಅನು-ಸರಿಸಿ-ಕೊಂಡಿದೆ
ಅನು-ಸರಿಸಿ-ಕೊಂಡು
ಅನು-ಸರಿಸಿ-ದರು
ಅನು-ಸರಿಸಿ-ದರೆ
ಅನು-ಸರಿಸಿಯೇ
ಅನು-ಸರಿಸು
ಅನು-ಸರಿಸುತ್ತದೆ
ಅನು-ಸರಿಸುತ್ತಿರುವೆ
ಅನು-ಸರಿಸುವ
ಅನು-ಸರಿಸು-ವಂತೆ
ಅನು-ಸರಿಸು-ವರು
ಅನು-ಸರಿಸು-ವುದಕ್ಕಲ್ಲ
ಅನು-ಸರಿಸು-ವುದಕ್ಕೆ
ಅನು-ಸರಿಸು-ವು-ದನ್ನು
ಅನು-ಸರಿಸು-ವುದ-ರಲ್ಲಿ
ಅನು-ಸರಿಸು-ವು-ದ-ರಿಂದ
ಅನು-ಸರಿಸು-ವುದು
ಅನು-ಸಾರ-ವಾಗಿ
ಅನುಸ್ಯೂತ-ನಾಗಿ-ರು-ವ-ವನು
ಅನೇಕ
ಅನೇಕ-ತೆ-ಯಲ್ಲಿ
ಅನೇ-ಕದ
ಅನೇಕ-ರದಕ್ಕಿಂತ
ಅನೇಕ-ರಿಗೆ
ಅನೇಕ-ರಿದ್ದಾರೆ
ಅನೇ-ಕರು
ಅನೇಕವು
ಅನೇ-ಕವೇ
ಅನೇಕ-ವೇಳೆ
ಅನೇಕ-ಸಲ
ಅನೇಕ-ಸಾರಿ
ಅನೈತಿಕ
ಅನ್ನ
ಅನ್ನಕ್ಕಾಗಿ
ಅನ್ನಕ್ಕೋಸ್ಕರ
ಅನ್ನ-ಗಳನ್ನು
ಅನ್ನದ
ಅನ್ನ-ದಾನ
ಅನ್ನ-ದಾನ-ದಿಂದ
ಅನ್ನ-ದಿಂದ
ಅನ್ನ-ದೊಂದಿಗೆ
ಅನ್ನ-ಪೂರ್ಣೆಯ
ಅನ್ನ-ವನ್ನು
ಅನ್ನ-ವನ್ನೂ
ಅನ್ನ-ವನ್ನೇ
ಅನ್ನ-ವನ್ನೇಕೆ
ಅನ್ನ-ವಸ್ತ್ರ
ಅನ್ನ-ವಿಲ್ಲ
ಅನ್ನವು
ಅನ್ನವೂ
ಅನ್ನ-ಸತ್ರ
ಅನ್ನ-ಸತ್ರ-ಗಳಾಗುತ್ತವೆ
ಅನ್ನ-ಸತ್ರದ
ಅನ್ನ-ಸತ್ರ-ವಾಗು-ವುದು
ಅನ್ನ-ಸತ್ರ-ವಿರು-ವು-ದನ್ನು
ಅನ್ನ-ಸತ್ರವು
ಅನ್ನಿಸಿ-ದರೂ
ಅನ್ನಿಸಿದೆ
ಅನ್ನಿ-ಸುತ್ತದೆ
ಅನ್ನಿ-ಸು-ವು-ದಿಲ್ಲ
ಅನ್ನಿ-ಸು-ವು-ದಿಲ್ಲವೋ
ಅನ್ನಿ-ಸು-ವುದು
ಅನ್ನು
ಅನ್ಯ
ಅನ್ಯಕಿ
ಅನ್ಯ-ದೇಶ-ಗಳ
ಅನ್ಯ-ದೇಶದ
ಅನ್ಯ-ಮನಸ್ಕ-ರಾಗಿ
ಅನ್ಯರ
ಅನ್ಯ-ರಿ-ಗಾಗಿ
ಅನ್ಯ-ರಿಗೆ
ಅನ್ಯ-ರಿ-ಗೋಸ್ಕರ
ಅನ್ಯ-ವೀರ
ಅನ್ಯಾಯ
ಅನ್ಯಾಯ-ವನ್ನು
ಅನ್ಯಾಯ-ವಾದರೆ
ಅನ್ವಯಿ-ಸಲು
ಅನ್ವಯಿ-ಸುತ್ತದೆ
ಅನ್ವಯಿ-ಸುತ್ತ-ದೆಯೆ
ಅನ್ವಯಿ-ಸುತ್ತ-ದೆಯೋ
ಅನ್ವಯಿ-ಸುತ್ತೀರಿ
ಅನ್ವಯಿ-ಸುವಂಥದು
ಅನ್ವಯಿಸು-ವು-ದೆಂದು
ಅನ್ವೇಷ-ಕರು
ಅನ್ವೇಷಣೆ
ಅಪ-ಕಾರ-ವಾ-ಗು-ವು-ದಿಲ್ಲ
ಅಪ-ಜ-ಯದ
ಅಪ-ನಂಬಿಕೆ
ಅಪನ-ಬುಲಾವೊ
ಅಪನಾಯ
ಅಪನಾರ್
ಅಪ-ಮಾನ-ಗಳನ್ನೂ
ಅಪ-ಮಾನ-ವನ್ನು
ಅಪ-ಯಶಸ್ಸು
ಅಪರಾವ-ತಾ-ರವೇ
ಅಪರಾಹ್ನದ
ಅಪರಾಹ್ನ-ವಾಗಿತ್ತು
ಅಪರಿ-ಚಿತ-ರಾದ
ಅಪರಿಮಿತ-ವಾದ
ಅಪ-ರೂಪ
ಅಪ-ರೂಪಳು
ಅಪರೋಕ್ಷಾನು-ಭೂತಿ
ಅಪರೋಕ್ಷಾನು-ಭೂತಿ-ಯನ್ನು
ಅಪರೋಕ್ಷಾನು-ಭೂತಿ-ಯೆಂದು
ಅಪ-ವಾದ-ಗಳನ್ನು
ಅಪ-ವಾದ-ವನ್ನು
ಅಪವಿತ್ರ-ವಾದ
ಅಪಸ್ವರ-ವನ್ನು
ಅಪಹರಿ-ಸು-ವಷ್ಟು
ಅಪ-ಹಾಸ್ಯ
ಅಪ-ಹಾಸ್ಯಕ್ಕೆ
ಅಪಾಯ
ಅಪಾಯ-ಕರ
ಅಪಾಯ-ಕರ-ವಾದ
ಅಪಾಯ-ಕಾರಿ
ಅಪಾ-ಯಕ್ಕೆ
ಅಪಾಯ-ಗಳಿವೆ
ಅಪಾ-ಯದ
ಅಪಾಯ-ದಿಂದಲೂ
ಅಪಾಯ-ವಿಲ್ಲ
ಅಪಾ-ಯವು
ಅಪಾರ
ಅಪಾರ-ವಾದ
ಅಪಾರ್ಥ-ಮಾಡಿ-ಕೊಂಡು-ದರ
ಅಪೂರ್ಣ
ಅಪೂರ್ಣತೆ
ಅಪೂರ್ಣರೊ
ಅಪೂರ್ಣ-ವಾಗಿಲ್ಲ
ಅಪೂರ್ವ
ಅಪೂರ್ವ-ದರ್ಶನ
ಅಪೂರ್ವ-ವಾಗಿದೆ
ಅಪೂರ್ವ-ವಾದ
ಅಪೇಕ್ಷಿ-ಸಲು
ಅಪೇಕ್ಷಿ-ಸಿ-ದನು
ಅಪೇಕ್ಷಿ-ಸುತ್ತಾರೆ
ಅಪೇಕ್ಷಿ-ಸುತ್ತಿದ್ದರು
ಅಪೇಕ್ಷಿ-ಸುತ್ತಿ-ರು-ವ-ರೆಂದು
ಅಪೇಕ್ಷಿ-ಸುತ್ತೀ-ಯಷ್ಟೆ
ಅಪೇಕ್ಷಿ-ಸುವರು
ಅಪೇಕ್ಷೆ
ಅಪೇಕ್ಷೆ-ಗಳೇ
ಅಪೇಕ್ಷೆ-ಯಾಗಿತ್ತು
ಅಪ್ಪ
ಅಪ್ಪಟ
ಅಪ್ಪಣೆ
ಅಪ್ಪ-ಣೆಗೆ
ಅಪ್ಪ-ಣೆಯ
ಅಪ್ಪ-ಣೆ-ಯಂತೆ
ಅಪ್ಪ-ಣೆ-ಯನ್ನನು-ಸ-ರಿಸಿ
ಅಪ್ಪ-ಣೆ-ಯನ್ನು
ಅಪ್ಪ-ಣೆ-ಯಾ-ಗಿದೆ
ಅಪ್ಪ-ಣೆ-ಯಿಂದ
ಅಪ್ಪ-ಣೆ-ಯೆಂದ-ರಾಗ-ಲಿಲ್ಲ
ಅಪ್ರಜ್ಞೆ
ಅಪ್ರಜ್ಞೆಯ
ಅಪ್ರತಿ-ಭಟನೆ
ಅಪ್ರತಿ-ಭ-ನಾಗಿ
ಅಪ್ರತಿ-ಮ-ನಾದ
ಅಪ್ರಯೋಜಕ
ಅಬಾಧಿತ-ವಾದ
ಅಬ್ಬ-ರದ
ಅಬ್ಬರ-ವನ್ನೆಬ್ಬಿಸಿ
ಅಬ್ರಾಹ್ಮ-ಣರು
ಅಭ
ಅಭಕ್ಷ್ಯ-ವಸ್ತು-ಗಳನ್ನು
ಅಭಯ
ಅಭಯಂ
ಅಭ-ಯಕ್ಕೆ
ಅಭ-ಯದ
ಅಭಯ-ದಲಿ
ಅಭಯ-ವನ್ನು
ಅಭಯ-ವಾಣಿ-ಯನ್ನು
ಅಭಯ-ವಾಣಿ-ಯನ್ನು-ಸಿರು-ವುದಕ್ಕಾಗಿಯೇ
ಅಭಯ-ವಾಣಿ-ಯನ್ನೇ
ಅಭಾವ
ಅಭಾವ-ಗಳನ್ನೂ
ಅಭಾವದ
ಅಭಾವ-ದಲ್ಲಿಯೂ
ಅಭಾವ-ದಿಂದ
ಅಭಾವ-ರೂಪಿ-ಯಾದ
ಅಭಾವ-ವನ್ನು
ಅಭಾವ-ವನ್ನೂ
ಅಭಾವ-ವಿತ್ತು
ಅಭಾವ-ವಿ-ದೆಯೋ
ಅಭಾವ-ವಿರು-ವು-ದ-ರಿಂದಲೇ
ಅಭಾವವೇ
ಅಭಿಜ್ಞ-ರಾದ
ಅಭಿ-ನಂದಿಸಿ
ಅಭಿನ್ನ
ಅಭಿನ್ನ-ವಾದ
ಅಭಿಪ್ರಾಯ
ಅಭಿಪ್ರಾ-ಯಕ್ಕೆ
ಅಭಿಪ್ರಾಯ-ಗಳ
ಅಭಿಪ್ರಾಯ-ಗಳನ್ನು
ಅಭಿಪ್ರಾಯ-ಗಳಲ್ಲಿ
ಅಭಿಪ್ರಾಯ-ಗಳು
ಅಭಿಪ್ರಾಯ-ಗಳೆಲ್ಲ
ಅಭಿಪ್ರಾಯ-ಗಳೆಲ್ಲಾ
ಅಭಿಪ್ರಾಯದ
ಅಭಿಪ್ರಾಯ-ದಂತೆಯೇ
ಅಭಿಪ್ರಾಯ-ದಲ್ಲಿ
ಅಭಿಪ್ರಾಯ-ಪಟ್ಟರು
ಅಭಿಪ್ರಾಯ-ಪಡುತ್ತಾರೆ
ಅಭಿಪ್ರಾಯ-ಪಡುತ್ತಾ-ರೆಂದು
ಅಭಿಪ್ರಾಯ-ಪ-ಡು-ವರು
ಅಭಿಪ್ರಾಯ-ವನ್ನು
ಅಭಿಪ್ರಾಯ-ವನ್ನೂ
ಅಭಿಪ್ರಾಯ-ವನ್ನೇ
ಅಭಿಪ್ರಾಯ-ವಾ-ಗಿ-ರಲಿಲ್ಲ-ವೆಂದು
ಅಭಿಪ್ರಾಯ-ವಿತ್ತು
ಅಭಿಪ್ರಾಯ-ವಿದೆ
ಅಭಿಪ್ರಾಯ-ವಿರು-ವು-ದಿಲ್ಲ
ಅಭಿಪ್ರಾಯವು
ಅಭಿಪ್ರಾಯವೆ
ಅಭಿಪ್ರಾಯವೇ
ಅಭಿಪ್ರಾಯ-ವೇನು
ಅಭಿಪ್ರಾಯ-ವೇ-ನೆಂದರೆ
ಅಭಿಪ್ರಾಯ-ವೇ-ನೆಂದು
ಅಭಿ-ಮಾನ
ಅಭಿ-ಮಾನ-ವನ್ನಿಟ್ಟು-ಕೊಂಡು
ಅಭಿ-ಮಾನ-ವನ್ನು
ಅಭಿ-ಮಾನ-ವಷ್ಟೆ
ಅಭಿ-ಮಾನವೂ
ಅಭಿ-ಮಾನ-ಶಾಲಿ-ಗಳೂ
ಅಭಿಯಾ-ನದ
ಅಭಿ-ರುಚಿ
ಅಭಿಲಾಷೆ
ಅಭಿಲಾಷೆ-ಗಳನ್ನೂ
ಅಭಿಲಾಷೆ-ಯನ್ನು
ಅಭಿಲಾಷೆ-ಯಿಂದ
ಅಭಿಲಾಷೆ-ಯಿಲ್ಲ
ಅಭಿ-ವಂದಿಸಿ
ಅಭಿ-ವಾದನ
ಅಭಿವೃದ್ದಿ-ಯಾಗಿತ್ತು
ಅಭಿ-ವೃದ್ಧಿ
ಅಭಿ-ವೃದ್ಧಿಗೆ
ಅಭಿ-ವೃದ್ಧಿ-ಯಾಗುತ್ತ
ಅಭಿ-ವೃದ್ಧಿ-ಯಿಂದ
ಅಭಿವ್ಯಕ್ತಿಗೂ
ಅಭಿವ್ಯಕ್ತಿಗೆ
ಅಭಿವ್ಯಕ್ತಿಯ
ಅಭಿವ್ಯಕ್ತಿ-ಯನ್ನು
ಅಭಿವ್ಯಕ್ತಿ-ಯಷ್ಟೇ
ಅಭಿವ್ಯಕ್ತಿ-ಯಾ-ಗಿಯೂ
ಅಭಿವ್ಯಕ್ತಿಯು
ಅಭಿವ್ಯಕ್ತಿಯೂ
ಅಭಿವ್ಯಕ್ತಿಯೇ
ಅಭಿ-ಶಾ-ಪವೂ
ಅಭಿ-ಶಾಪ-ವೆಂದರೆ
ಅಭೀಃ
ಅಭೀಷ್ಟ
ಅಭೀಷ್ಟ-ಗಳನ್ನು
ಅಭೇದ
ಅಭೇದ್ಯ-ವಾದ
ಅಭ್ಯಂತರ
ಅಭ್ಯಸಿ-ಸಿರು-ವಿ-ರಲ್ಲ
ಅಭ್ಯಾಸ
ಅಭ್ಯಾಸ-ಗಳಿಂದುಂಟಾದ
ಅಭ್ಯಾಸ-ದಿಂದ
ಅಭ್ಯಾಸ-ದಿಂದಲೆ
ಅಭ್ಯಾಸ-ವಾಗಿ
ಅಭ್ಯಾಸೇನ
ಅಭ್ಯುತ್ಥಾನ
ಅಭ್ಯುತ್ಥಾನ-ದೊಂದಿಗೆ
ಅಭ್ಯುತ್ಥಾನ-ವನ್ನು
ಅಭ್ರ-ಭೇದಿ
ಅಮನಿ
ಅಮರತ್ವ-ವನ್ನೈದಿ-ದರು
ಅಮರತ್ವವೂ
ಅಮರ-ನಾಥ
ಅಮರ-ನಾಥಕ್ಕೆ
ಅಮರ-ನಾಥ-ದಲ್ಲಿ
ಅಮರ-ನಾಥ-ದಲ್ಲಿಯೂ
ಅಮರ-ನಾಥ-ನನ್ನು
ಅಮರ-ನಾ-ದೆನು
ಅಮಲೇರಿ
ಅಮಾ-ವಾಸ್ಯೆ
ಅಮಿ
ಅಮಿ-ತ-ವಾದ
ಅಮೂಲ್ಯ
ಅಮೂಲ್ಯ-ವಾದ
ಅಮೂಲ್ಯ-ವಾ-ದದ್ದು
ಅಮೃತ
ಅಮೃತ-ತರಂಗಿಣಿ
ಅಮೃತತ್ವ
ಅಮೃತತ್ವ-ದಾಕಾಂಕ್ಷೆ
ಅಮೃತತ್ವ-ಮಾನುಶುಃ
ಅಮೃತತ್ವ-ವನ್ನು
ಅಮೃತತ್ವವು
ಅಮೃ-ತದಿ
ಅಮೃತ-ವನ್ನು
ಅಮೃತ-ವಾಗಿ-ರು-ವುವು
ಅಮೃ-ತವು
ಅಮೃತ-ಶಿಲೆಯ
ಅಮೃತ-ಶಿಲೆಯಲ್ಲಿ
ಅಮೃತ-ಶಿಲೆಯಿಂದ
ಅಮೃತಾತ್ಮ
ಅಮೃತಾತ್ಮ-ನಾಗಿ-ರು-ವನು
ಅಮೃತಾನು-ಭ-ವಕ್ಕೆ
ಅಮೃತೆ-ಗರಳ
ಅಮೆ-ರಿಕದ
ಅಮೆ-ರಿಕ-ದಿಂದ
ಅಮೆರಿ-ಕನ್
ಅಮೆರಿ-ಕನ್ನರು
ಅಮೆ-ರಿಕಾ
ಅಮೆ-ರಿಕಾಕ್ಕಿಂತಲೂ
ಅಮೆ-ರಿಕಾಕ್ಕೆ
ಅಮೆ-ರಿಕಾದ
ಅಮೆ-ರಿಕಾ-ದಲ್ಲಿ
ಅಮೆ-ರಿಕಾ-ದಲ್ಲಿದ್ದಾಗ
ಅಮೆ-ರಿಕಾ-ದಲ್ಲಿಯೂ
ಅಮೆ-ರಿಕಾ-ದಲ್ಲಿ-ರುವ-ವ-ನೊಬ್ಬನ
ಅಮೆ-ರಿಕಾ-ದ-ವ-ರಿಗೆ
ಅಮೆ-ರಿಕಾ-ದಿಂದ
ಅಮೆ-ರಿಕಾ-ದೇಶೀಯ-ನೊಬ್ಬ
ಅಮೆ-ರಿಕಾ-ವಾಸಿ-ಗಳಂಥ
ಅಮೇರಿ-ಕನ್
ಅಮೋಘ
ಅಮೋಘ-ವಾಗಿ-ರು-ವುದೋ
ಅಮೋಘ-ವಾದ
ಅಮೋಘ-ವಾದುದೇನೋ
ಅಮ್ಮ
ಅಮ್ಮಮ್ಮ
ಅಯನಾಯ
ಅಯನೆ
ಅಯಮಾತ್ಮಾ
ಅಯಮೇವ
ಅಯಾ-ಚಿತ
ಅಯಾ-ಚಿತ-ವಾಗಿ
ಅಯಿ
ಅಯುತ
ಅಯೋಗ್ಯ
ಅಯೋಗ್ಯರ
ಅಯೋಗ್ಯ-ವಾಗಿ-ರು-ವುದು
ಅಯೋಗ್ಯ-ವಾದ
ಅಯ್ಯಾ
ಅಯ್ಯೊ
ಅಯ್ಯೋ
ಅರ-ಗಿನ
ಅರ-ಗಿಳಿ-ಯಂತೆ
ಅರಗಿಸಿ-ಕೊಂಡಿದ್ದೇನೆ
ಅರಚಾ-ಟವಷ್ಟೇ
ಅರಚಿ-ಕೊಳ್ಳುವುದು
ಅರಚಿ-ದರು
ಅರಚುತ್ತ
ಅರಚುತ್ತಾರೆ
ಅರ-ಮನೆ-ಗಳ
ಅರ-ಮನೆ-ಯನ್ನು
ಅರ-ಮನೆ-ಯಲ್ಲಿ
ಅರಳಿತು
ಅರ-ಳಿದ
ಅರ-ಳಿದಂತಾಯಿತು
ಅರಳಿ-ಬಿಟ್ಟರೆ
ಅರಳಿ-ರುವ
ಅರಳು
ಅರಳುವ
ಅರಸ-ಬಹುದೆ
ಅರಸ-ಬೇಕೆ
ಅರಸಿ-ಕೊಂಡು
ಅರ-ಸುತ್ತಿ-ರುವ
ಅರ-ಸುವ
ಅರ-ಸು-ವಂತೆ
ಅರಿಕೆ
ಅರಿ-ಗಳ
ಅರಿತ
ಅರಿತಂತೆ
ಅರಿ-ತರೆ
ಅರಿ-ತಾಗ
ಅರಿ-ತಾದ
ಅರಿ-ತಿದ್ದರು
ಅರಿತಿದ್ದಾನೋ
ಅರಿತಿಲ್ಲ
ಅರಿತು
ಅರಿತು-ಕೊಳ್ಳಲಿ
ಅರಿತೂ
ಅರಿಯ
ಅರಿಯ-ತಕ್ಕ-ವರು
ಅರಿಯ-ದಿರುವೆ
ಅರಿ-ಯದೆ
ಅರಿಯ-ನ-ವನು
ಅರಿಯ-ಬಲ್ಲೆ
ಅರಿಯ-ಬ-ಹುದು
ಅರಿಯ-ಬೇಕಾಗಿತ್ತು
ಅರಿಯ-ಬೇ-ಕಾದರೆ
ಅರಿಯ-ಬೇಕು
ಅರಿ-ಯರು
ಅರಿಯ-ಲಾಗ-ದ-ವರು
ಅರಿಯ-ಲಾಗ-ದಿರುವ
ಅರಿಯ-ಲಾರ-ದವ-ರಾಗಿದ್ದರು
ಅರಿಯ-ಲಾರದೆ
ಅರಿಯ-ಲಾರಿರಿ
ಅರಿ-ಯಲು
ಅರಿ-ಯಿರಿ
ಅರಿಯುತ್ತಾನೆ
ಅರಿಯುತ್ತೇವೆ
ಅರಿಯುವ
ಅರಿ-ಯು-ವಂತೆ
ಅರಿ-ಯುವನು
ಅರಿಯು-ವಾಗಲೇ
ಅರಿ-ಯು-ವುದಕ್ಕೆ
ಅರಿ-ಯುವೆ
ಅರಿಯೆ
ಅರಿ-ಯೆಯಾ
ಅರಿ-ವರಿ-ದನು
ಅರಿ-ವ-ವನು
ಅರಿ-ವಾಗಿದೆ
ಅರಿ-ವಾಗುತ್ತದೆ
ಅರಿ-ವಾ-ಗು-ವಂತೆ
ಅರಿ-ವಾಗು-ವುದು
ಅರಿ-ವಾ-ಯಿತು
ಅರಿ-ವಿಗೆ
ಅರಿ-ವಿನ
ಅರಿ-ವಿಲ್ಲ
ಅರಿ-ವಿಲ್ಲದೆ
ಅರಿವು
ಅರಿವುಂಟಾ-ಗು-ವಂತೆ
ಅರಿ-ವುಂಟಾಗು-ವುದು
ಅರಿವೆ-ಯನ್ನು
ಅರಿವೇ
ಅರುಂಧತಿ-ಯ-ರಂತೆ
ಅರುಹಿದ
ಅರೂಪ
ಅರೂಪ-ನಾಮ-ವ-ರಣ
ಅರೆ-ಮನುಷ್ಯ
ಅರೆ-ಮುಚ್ಚಿದ
ಅರೆಯು
ಅರ್ಚ-ಕ-ನಾಗಿದ್ದ
ಅರ್ಚನೆ
ಅರ್ಜಿ
ಅರ್ಜಿ-ಗ-ಳಾದರೂ
ಅರ್ಜಿ-ಯನ್ನು
ಅರ್ಜಿಸಿ
ಅರ್ಜಿ-ಸುವುದಕ್ಕಿ-ರುವ
ಅರ್ಜುನ
ಅರ್ಜುನ-ನಿಗೆ
ಅರ್ಜು-ನನೂ
ಅರ್ಜುನ-ನೆ-ಡೆಗೆ
ಅರ್ಜುನಾ
ಅರ್ಥ
ಅರ್ಥ-ಗಳನ್ನು
ಅರ್ಥ-ದಲ್ಲಿ
ಅರ್ಥ-ಮಾಡಿಕೊ
ಅರ್ಥ-ಮಾಡಿ-ಕೊಂಡಿದ್ದಲ್ಲಿ
ಅರ್ಥ-ಮಾಡಿ-ಕೊಂಡಿದ್ದಾರೆ
ಅರ್ಥ-ಮಾಡಿ-ಕೊಂಡಿದ್ದಾರೊ
ಅರ್ಥ-ಮಾಡಿ-ಕೊಂಡಿದ್ದೇನೆ
ಅರ್ಥ-ಮಾಡಿ-ಕೊಂಡಿರು-ವುದೇ
ಅರ್ಥ-ಮಾಡಿ-ಕೊಂಡಿರುವೆ
ಅರ್ಥ-ಮಾಡಿ-ಕೊಂಡಿಲ್ಲ
ಅರ್ಥ-ಮಾಡಿ-ಕೊಂಡಿಲ್ಲವೋ
ಅರ್ಥ-ಮಾಡಿ-ಕೊಂಡು
ಅರ್ಥ-ಮಾಡಿ-ಕೊಂಡೆ
ಅರ್ಥ-ಮಾಡಿ-ಕೊಳ್ಳ-ತೊಡಗಿದೆ
ಅರ್ಥ-ಮಾಡಿ-ಕೊಳ್ಳ-ಬಲ್ಲರು
ಅರ್ಥ-ಮಾಡಿ-ಕೊಳ್ಳಲಾಗುತ್ತಿಲ್ಲ
ಅರ್ಥ-ಮಾಡಿ-ಕೊಳ್ಳ-ಲಾರದೆ
ಅರ್ಥ-ಮಾಡಿ-ಕೊಳ್ಳ-ಲಾ-ರರು
ಅರ್ಥ-ಮಾಡಿ-ಕೊಳ್ಳ-ಲಿಲ್ಲ
ಅರ್ಥ-ಮಾಡಿ-ಕೊಳ್ಳಲು
ಅರ್ಥ-ಮಾಡಿ-ಕೊಳ್ಳಲೇ
ಅರ್ಥ-ಮಾಡಿ-ಕೊಳ್ಳುತ್ತಿದ್ದರು
ಅರ್ಥ-ಮಾಡಿ-ಕೊಳ್ಳುವ-ರೇನು
ಅರ್ಥ-ಮಾಡಿ-ಕೊಳ್ಳು-ವಷ್ಟು
ಅರ್ಥ-ಮಾಡಿ-ಕೊಳ್ಳು-ವುದಕ್ಕಾ-ಗಲಿಲ್ಲ
ಅರ್ಥ-ಮಾಡಿ-ಕೊಳ್ಳು-ವು-ದಿಲ್ಲ-ವೆಂದು
ಅರ್ಥ-ಮಾಡಿ-ಕೊಳ್ಳುವುದು
ಅರ್ಥ-ಮಾಡಿ-ಕೊಳ್ಳುವೆ
ಅರ್ಥ-ಮಾಡಿದ್ದಾರೆ
ಅರ್ಥ-ವನ್ನು
ಅರ್ಥ-ವಾಗದೆ
ಅರ್ಥ-ವಾಗ-ಲಿದೆ
ಅರ್ಥ-ವಾಗ-ಲಿಲ್ಲ
ಅರ್ಥ-ವಾಗ-ಲಿಲ್ಲವೆ
ಅರ್ಥ-ವಾಗಿಲ್ಲ
ಅರ್ಥ-ವಾಗುತ್ತದೆ
ಅರ್ಥ-ವಾ-ಗುತ್ತಿದೆಯೆ
ಅರ್ಥ-ವಾಗುವ
ಅರ್ಥ-ವಾ-ಗು-ವಂತೆ
ಅರ್ಥ-ವಾ-ಗು-ವು-ದಿಲ್ಲ
ಅರ್ಥ-ವಾಗು-ವುದು
ಅರ್ಥ-ವಾ-ಯಿತು
ಅರ್ಥ-ವಾ-ಯಿತೆ
ಅರ್ಥ-ವಿದೆ
ಅರ್ಥ-ವಿ-ದೆಯೋ
ಅರ್ಥ-ವಿದ್ದೇ
ಅರ್ಥ-ವಿಲ್ಲದ
ಅರ್ಥವು
ಅರ್ಥ-ವೇನು
ಅರ್ಥ-ವೇ-ನೆಂದು
ಅರ್ಥ-ಶೂನ್ಯ-ವಾ-ದದ್ದು
ಅರ್ಥೈ-ಸಿ-ದರು
ಅರ್ಧ
ಅರ್ಧ-ಗಂಟೆ
ಅರ್ಧ-ಗಂಟೆಯ
ಅರ್ಧ-ಪೌಂಡು
ಅರ್ಧೇಕ್
ಅರ್ಪಣ
ಅರ್ಪಿಸ-ಬಲ್ಲ
ಅರ್ಪಿಸ-ಬೇಕು
ಅರ್ಪಿಸ-ಲಾ-ಗು-ವು-ದಿಲ್ಲವೆ
ಅರ್ಪಿ-ಸಲು
ಅರ್ಪಿಸಿ
ಅರ್ಪಿಸಿ-ಕೊಳ್ಳುವನೊ
ಅರ್ಪಿಸಿ-ಕೊಳ್ಳುವುದ-ರಲ್ಲಿದ್ದ
ಅರ್ಪಿಸಿ-ದರು
ಅರ್ಪಿ-ಸಿದ್ದೆ
ಅರ್ಪಿ-ಸುವ
ಅರ್ಪಿಸು-ವಂತಹ
ಅರ್ಪಿ-ಸುವರು
ಅರ್ಪಿ-ಸುವರೊ
ಅರ್ಪಿ-ಸು-ವೆನು
ಅರ್ಪಿಸೆಲ್ಲ-ವನ-ವನ
ಅರ್ಹ
ಅರ್ಹತೆ
ಅರ್ಹ-ತೆ-ಯನ್ನು
ಅರ್ಹ-ತೆ-ಯುಳ್ಳ-ವ-ರಾರು
ಅರ್ಹ-ನಾಗಿದ್ದನು
ಅರ್ಹ-ನಾಗು-ವ-ನೆಂದು
ಅರ್ಹ-ನಾಗುವೆ
ಅರ್ಹ-ರಾಗಿ
ಅರ್ಹ-ರಾಗಿದ್ದಾರೆಯೇ
ಅರ್ಹ-ರಾ-ಗು-ವಂತೆ
ಅರ್ಹರೆ
ಅಲಂಕರಿ-ಸ-ಬೇಕೆಂದು
ಅಲಂಕಾರ
ಅಲಕ್ಷ್ಯ-ದಿಂದ
ಅಲಕ್ಷ್ಯ-ಮಾಡಿ-ಬಿಟ್ಟರು
ಅಲಿ-ಕುಲ
ಅಲುಗ-ದಂತೆ
ಅಲುಗಾಡ-ದಿರುವ
ಅಲೆ
ಅಲೆ-ಅಲೆ-ಯಾಗಿ
ಅಲೆಕ್ಸಾಂಡರನ
ಅಲೆಕ್ಸಾಂಡ್ರಿಯಾ-ದಲ್ಲಿ
ಅಲೆಕ್ಸಾಂಡ್ರಿಯಾ-ದ-ವ-ರೆಗೂ
ಅಲೆ-ಗಳ
ಅಲೆ-ಗಳನ್ನು
ಅಲೆ-ಗಳಲಿ
ಅಲೆ-ಗಳು
ಅಲೆ-ಗಳೇ-ಳುತಿ-ಹವು
ಅಲೆ-ಗಳೊಂದಿಗೆ
ಅಲೆಗೆ
ಅಲೆ-ದಲೆದು
ಅಲೆ-ದಾಟ-ದಲ್ಲಿ
ಅಲೆದು
ಅಲೆ-ಮಾರಿ
ಅಲೆಯ
ಅಲೆ-ಯನ್ನು
ಅಲೆ-ಯಲ್ಲ
ಅಲೆ-ಯಲ್ಲಿ-ರುವ
ಅಲೆ-ಯಾ-ದು-ದ-ರಿಂದ
ಅಲೆಯು
ಅಲೆ-ಯು-ತಿದ್ದೆ
ಅಲೆ-ಯುತ್ತಾ
ಅಲೆ-ಯುತ್ತಿದ್ದೀರಿ
ಅಲೆ-ಯು-ವು-ದರ
ಅಲೆ-ಯು-ವುದು
ಅಲೆ-ಯು-ವೆನು
ಅಲೆಯೂ
ಅಲೆ-ಯೆಂದು
ಅಲೆ-ಯೆದುರು
ಅಲೆ-ಯೆದ್ದು
ಅಲೌಕಿ-ಕತೆ
ಅಲೌಕಿಕ-ವಾದ
ಅಲ್ಪ
ಅಲ್ಪ-ಬಡ್ಡಿಯ
ಅಲ್ಪರು
ಅಲ್ಪ-ವನ್ನಾದರೂ
ಅಲ್ಪ-ವಾಗಿ
ಅಲ್ಪ-ವಾದರೂ
ಅಲ್ಪಶ್ರದ್ಧೆ
ಅಲ್ಪಸ್ವಲ್ಪ
ಅಲ್ಪಸ್ವಲ್ಪ-ವಾದರೂ
ಅಲ್ಪಾಂಶ-ವಾದರೂ
ಅಲ್ಬರ್ಟಾ
ಅಲ್ಲ
ಅಲ್ಲ-ಗಳೆ-ಯಿರಿ
ಅಲ್ಲ-ಗಳೆ-ಯುತ್ತೀರಿ
ಅಲ್ಲ-ಗಳೆ-ಯುವನು
ಅಲ್ಲ-ದಿದ್ದರೂ
ಅಲ್ಲ-ದಿದ್ದರೆ
ಅಲ್ಲದೆ
ಅಲ್ಲಲ್ಲಿ
ಅಲ್ಲಲ್ಲೆ
ಅಲ್ಲವೆ
ಅಲ್ಲವೇ
ಅಲ್ಲ-ವೇ-ನಯ್ಯಾ
ಅಲ್ಲವೋ
ಅಲ್ಲಾ-ಡದೆ
ಅಲ್ಲಾಡುತ್ತಿತ್ತು
ಅಲ್ಲಾನ
ಅಲ್ಲಿ
ಅಲ್ಲಿಂದ
ಅಲ್ಲಿಂದಲೇ
ಅಲ್ಲಿಂದಿಲ್ಲಿಗೆ
ಅಲ್ಲಿ-ಇಲ್ಲಿ
ಅಲ್ಲಿಗೂ
ಅಲ್ಲಿಗೆ
ಅಲ್ಲಿ-ಗೆಲ್ಲಾ
ಅಲ್ಲಿತ್ತು
ಅಲ್ಲಿದೆ
ಅಲ್ಲಿದ್ದ
ಅಲ್ಲಿದ್ದರು
ಅಲ್ಲಿದ್ದ-ವ-ರಿ-ಗೆಲ್ಲಾ
ಅಲ್ಲಿದ್ದ-ವರೂ
ಅಲ್ಲಿದ್ದ-ವ-ರೆಲ್ಲ-ರನ್ನೂ
ಅಲ್ಲಿದ್ದ-ವ-ರೆಲ್ಲಾ
ಅಲ್ಲಿದ್ದು-ಕೊಂಡು
ಅಲ್ಲಿನ
ಅಲ್ಲಿಯ
ಅಲ್ಲಿ-ಯ-ವ-ರೆಗೂ
ಅಲ್ಲಿ-ಯ-ವರೆಗೆ
ಅಲ್ಲಿ-ಯ-ವರೆ-ವಿಗೂ
ಅಲ್ಲಿಯೂ
ಅಲ್ಲಿಯೇ
ಅಲ್ಲಿ-ರುವ
ಅಲ್ಲಿ-ರು-ವು-ದಿಲ್ಲ
ಅಲ್ಲಿಲ್ಲ
ಅಲ್ಲಿವೆ
ಅಲ್ಲೆಲ್ಲ
ಅಲ್ಲೇ
ಅಲ್ಲೇ-ನಾದರೂ
ಅಲ್ಲೊಂದು
ಅಲ್ಲೋಲ
ಅಲ್ಲೋಲ-ಕಲ್ಲೋಲ
ಅಲ್ಲೋಲ-ಕಲ್ಲೋಲ-ವಾಗಿ-ರು-ವುದು
ಅಳದ
ಅಳ-ಬೇಕು
ಅಳ-ಲನ್ನು
ಅಳಲಿ-ಗಳು-ಕದೆ
ಅಳ-ಲಿನಲು
ಅಳಲು
ಅಳಲು-ಗಳ
ಅಳಲು-ಗಳನು
ಅಳವಡಿ-ಸ-ಲಾ-ಗು-ವು-ದಿಲ್ಲ
ಅಳ-ವಿನ
ಅಳಸಿಂಗನೆ
ಅಳಸಿಂಗಾ-ಚಾರ್ಯರು
ಅಳಿದರೂ
ಅಳಿದು-ಹೋಗಿ-ರುವಾಗ
ಅಳಿದು-ಹೋಗು-ವುದು
ಅಳಿದು-ಹೋಗು-ವುದೋ
ಅಳಿ-ಯದ
ಅಳಿಯ-ಬೇಕು
ಅಳಿ-ಯು-ವು-ದಿಲ್ಲ
ಅಳಿವಿಲ್ಲವು
ಅಳಿ-ಸಲಾ-ಗದ
ಅಳಿಸಿ
ಅಳಿ-ಸಿಯೇ
ಅಳಿಸಿ-ಹೋಗ-ಬೇಕು
ಅಳಿಸಿ-ಹೋಗಿ
ಅಳಿಸಿ-ಹೋಗಿದೆ
ಅಳಿಸಿ-ಹೋಗು-ವುದು
ಅಳು
ಅಳು-ತಲಿದ್ದೆ
ಅಳುತ್ತ
ಅಳು-ವುದಕ್ಕೆ
ಅಳು-ವುದ-ರಲ್ಲಿ
ಅಳು-ವುದು
ಅಳು-ವು-ದೊಂದು
ಅಳು-ವು-ನ-ಗೆಯಲಿ
ಅಳೆದಳೆದು
ಅಳೆ-ಯು-ವುದಕ್ಕೆ
ಅವ
ಅವಕಾಶ
ಅವಕಾಶ-ಕೊಡುತ್ತಿ-ರ-ಲಿಲ್ಲ
ಅವಕಾಶ-ಗಳೂ
ಅವಕಾಶ-ವನ್ನು
ಅವಕಾಶ-ವಾಗಿತ್ತು
ಅವಕಾಶ-ವಿದೆ
ಅವಕಾಶ-ವಿರ-ಬೇಕು
ಅವಕಾಶ-ವಿ-ರುತ್ತದೆ
ಅವಕಾಶ-ವಿರು-ವು-ದಿಲ್ಲ
ಅವಕಾಶ-ವಿರು-ವುದು
ಅವಕಾಶ-ವಿಲ್ಲ
ಅವಕಾಶವೂ
ಅವಕಾಶವೇ
ಅವಕ್ಕೆ
ಅವ-ಗುಣ-ಗಳು
ಅವಚ್ಛಿನ್ನ-ವಾಗಿ
ಅವತ-ರ-ಗಳೊ-ಡನಿ-ರುವ
ಅವ-ತ-ರಿಸಿ
ಅವ-ತ-ರಿಸಿದ
ಅವ-ತ-ರಿಸಿ-ದರು
ಅವ-ತ-ರಿಸಿ-ದರೆ
ಅವ-ತ-ರಿಸಿ-ದಾಗ
ಅವ-ತ-ರಿಸಿದ್ದಾ-ನೆಂದು
ಅವ-ತ-ರಿಸಿದ್ದಾರೆ
ಅವ-ತ-ರಿಸಿದ್ದು
ಅವ-ತರಿ-ಸುವುದು
ಅವ-ತಾರ
ಅವ-ತಾರ-ಗಳನ್ನು
ಅವ-ತಾರ-ಗಳಿವೆ
ಅವ-ತಾರ-ಗಳು
ಅವ-ತಾರದ
ಅವ-ತಾರ-ಪುರುಷ-ನಿಗೆ
ಅವ-ತಾರ-ಪುರುಷನೆ
ಅವ-ತಾರ-ಪುರುಷ-ನೆಂದು
ಅವ-ತಾರ-ಪುರುಷರ
ಅವ-ತಾರ-ಪುರುಷ-ರಲ್ಲ-ವೆಂದೂ
ಅವ-ತಾರ-ಪುರುಷ-ರಲ್ಲಿ
ಅವ-ತಾರ-ಪುರುಷರು
ಅವ-ತಾರ-ಪುರುಷ-ರೆಂದು
ಅವ-ತಾರ-ಪುರುಷ-ರೊಡನೆ
ಅವ-ತಾರ-ಮಾಡಿ-ದರು
ಅವ-ತಾರ-ವರಿಷ್ಠಾಯ
ಅವ-ತಾರ-ವಾದ
ಅವ-ತಾರ-ವಾದಂದಿ-ನಿಂದ
ಅವ-ತಾರ-ವೆಂದು
ಅವ-ತಾರ-ವೆಂಬ
ಅವ-ತಾರ-ವೆಂಬಂತಿದ್ದ-ರೆಂದು
ಅವ-ತಾರ-ವೆತ್ತಿ
ಅವ-ತಾರ-ವೆನ್ನುತ್ತಾರೆ
ಅವ-ತಾರವ್ಯಕ್ತಿ
ಅವ-ತಾರಶ್ರೇಷ್ಠನೂ
ಅವ-ತಾರ-ಸ-ಮಾನ-ರಾದ
ಅವ-ತಾರ-ಸಾರ
ಅವ-ತಾರಿ-ಗಳ-ವರೆಗೆ
ಅವಧಿ-ಯಲ್ಲಿ
ಅವನ
ಅವ-ನಂತಹ
ಅವ-ನಂತೆ
ಅವನತ
ಅವನತಿ
ಅವನ-ತಿಗೆ
ಅವನ-ತಿಯ
ಅವನ-ತಿ-ಯಲ್ಲಿದ್ದಾಗ
ಅವನ-ದಾಗಿತ್ತು
ಅವನದೇ
ಅವನ-ನಾ-ರಾದಿ-ಸುತ
ಅವನ-ನಾರಾಧಿ-ಸುತ
ಅವ-ನನ್ನು
ಅವ-ನನ್ನುದ್ದೇಶಿಸಿ
ಅವ-ನನ್ನೂ
ಅವ-ನನ್ನೆಂದಿಗೂ
ಅವ-ನನ್ನೇಕೆ
ಅವನಲ್ಲ-ದೇ-ನನೂ
ಅವ-ನಲ್ಲಿ
ಅವನಲ್ಲಿ-ರುವ
ಅವನ-ವನ
ಅವ-ನಾಗಿದ್ದನು
ಅವ-ನಿಂದ
ಅವ-ನಿಂದಲೇ
ಅವನಿ-ಗಲ್ಲ
ಅವ-ನಿಗಾ-ದರೋ
ಅವ-ನಿ-ಗಿಂತ
ಅವ-ನಿಗೂ
ಅವ-ನಿಗೆ
ಅವನಿಗೇ
ಅವನಿಗೇ-ನನ್ನಾದರೂ
ಅವ-ನಿನ್ನೂ
ಅವನಿ-ರವಿ-ನಂಗಾಂಗ-ವೆಲ್ಲ-ವನು
ಅವನೀ
ಅವನು
ಅವನೆ
ಅವನೆಂತಹ
ಅವ-ನೆಂದನು
ಅವನೆಂದಿಗೂ
ಅವನೆಂದೂ
ಅವ-ನೆಂಬ
ಅವನೆ-ದು-ರಿಗೆ
ಅವನೆಷ್ಟೇ
ಅವನೇ
ಅವನೇ-ನನ್ನು
ಅವ-ನೇ-ನಾದರೂ
ಅವನೇ-ನಿದ್ದರೂ
ಅವ-ನೇನು
ಅವ-ನೇನೂ
ಅವ-ನೊಂದಿಗೆ
ಅವನೊಂದು
ಅವ-ನೊ-ಡನೆ
ಅವ-ನೊಬ್ಬ
ಅವ-ನೊಬ್ಬ-ನಿಗೆ
ಅವ-ನೊಬ್ಬನೇ
ಅವನ್ನು
ಅವನ್ನೆಲ್ಲಾ
ಅವನ್ನೇಕೆ
ಅವನ್ಯಾ-ವುದೇ
ಅವರ
ಅವ-ರಂತೆಯೇ
ಅವ-ರಂಥ
ಅವರ-ದಕ್ಕಿಂತಲೂ
ಅವ-ರದು
ಅವ-ರದೇ
ಅವ-ರನ್ನು
ಅವರನ್ನುದ್ದೇಶಿಸಿ
ಅವ-ರನ್ನೂ
ಅವರನ್ನೆಲ್ಲಾ
ಅವ-ರನ್ನೇ
ಅವರನ್ನೇಕೆ
ಅವರಲ್ಲದೆ
ಅವ-ರಲ್ಲಿ
ಅವ-ರಲ್ಲಿಗೆ
ಅವ-ರಲ್ಲಿದ್ದುದೂ
ಅವ-ರಲ್ಲಿಯೂ
ಅವ-ರಲ್ಲಿಯೇ
ಅವ-ರಲ್ಲಿ-ರ-ಲಿಲ್ಲ
ಅವ-ರಲ್ಲಿ-ರುವ
ಅವ-ರಲ್ಲಿ-ರು-ವಷ್ಟು
ಅವರ-ವರ
ಅವರ-ವರು
ಅವ-ರಾಗಲೇ
ಅವ-ರಾರೂ
ಅವರಾವ
ಅವ-ರಿಂದ
ಅವ-ರಿಂದಲೇ
ಅವ-ರಿ-ಗಾಗಿ
ಅವ-ರಿ-ಗಾಗಿ-ರು-ವುದು
ಅವರಿಗಾದ
ಅವರಿ-ಗಾ-ದರೂ
ಅವರಿಗಿದ್ದ
ಅವರಿ-ಗಿ-ರುವ
ಅವರಿಗಿ-ರು-ವುದು
ಅವ-ರಿಗೂ
ಅವ-ರಿಗೆ
ಅವ-ರಿ-ಗೆಲ್ಲ
ಅವ-ರಿ-ಗೆಲ್ಲಾ
ಅವ-ರಿ-ಗೋಸ್ಕರ
ಅವರಿನ್ನೂ
ಅವರಿಬ್ಬ-ರಿಗೂ
ಅವರಿ-ರುವ
ಅವರಿಷ್ಟ-ಬಂದ
ಅವರಿಷ್ಟ-ಬಂದಂತೆ
ಅವರು
ಅವರೂ
ಅವರೆ
ಅವರೆಂದಿಗೂ
ಅವ-ರೆಂದು
ಅವ-ರೆಂದೂ
ಅವ-ರೆಲ್ಲ
ಅವರೆಲ್ಲರ
ಅವರೆಲ್ಲ-ರಿಗೂ
ಅವರೆಲ್ಲರೂ
ಅವ-ರೆಲ್ಲಾ
ಅವ-ರೆಲ್ಲಿ
ಅವರೆಲ್ಲಿ-ಯ-ವರೆ-ವಿಗೂ
ಅವ-ರೆಲ್ಲೊ
ಅವ-ರೆಷ್ಟು
ಅವರೇ
ಅವರೇಕೆ
ಅವ-ರೇನು
ಅವರೇನೂ
ಅವರೇ-ನೆಂಬು-ದನ್ನು
ಅವ-ರೊಂದಿಗೆ
ಅವ-ರೊಂದು
ಅವ-ರೊಡನೆ
ಅವ-ರೊಡನೆಯೆ
ಅವರೊಬ್ಬ
ಅವರೊಬ್ಬರು
ಅವರೊಬ್ಬರೇ
ಅವರೊಮ್ಮೆ
ಅವರೊ-ಳಗೂ
ಅವರೋ-ಹಣ
ಅವಲಂಬನ
ಅವಲಂಬನೆ
ಅವಲಂಬಿಸ-ಬಹು-ದಾದ
ಅವಲಂಬಿಸ-ಬಾ-ರದೆ
ಅವಲಂಬಿ-ಸಲಿ
ಅವಲಂಬಿಸಿ
ಅವಲಂಬಿಸಿ-ಕೊಂಡು
ಅವಲಂಬಿಸಿ-ಕೊಂಡೇ
ಅವಲಂಬಿಸಿದೆ
ಅವಲಂಬಿಸಿವೆ
ಅವಲಂಬಿ-ಸುವ
ಅವಲಂಬಿ-ಸು-ವಂತೆ
ಅವಲಂಬಿ-ಸುವರು
ಅವಳ
ಅವ-ಳನ್ನು
ಅವಳನ್ನೆಷ್ಟು
ಅವಳಿಗೆ
ಅವಳು
ಅವಳೇ
ಅವಶೇಷ-ಗಳಲ್ಲಿವೆ
ಅವಶೇಷ-ಗಳು
ಅವಶೇಷ-ವನ್ನಿಟ್ಟಿ-ರುವ
ಅವಶೇಷ-ವಾಗಿತ್ತು
ಅವಶೇಷ-ವಾದ
ಅವಶ್ಯ-ಕತೆ-ಯೇನೂ
ಅವಶ್ಯಕ-ವಾ-ದದ್ದು
ಅವಶ್ಯ-ವಾಗಿ
ಅವಶ್ಯ-ವಾಗಿ-ರು-ವುದು
ಅವಸ-ರ-ದಲ್ಲಿ
ಅವ-ಸರ-ದಿಂದ
ಅವಸ-ರವ-ಸರ-ವಾಗಿ
ಅವ-ಸರ-ವೇನೂ
ಅವಸಾ-ನಕ್ಕೆ
ಅವಸ್ಥಿತ
ಅವಸ್ಥೆ
ಅವಸ್ಥೆ-ಗಿಳಿ-ದಿದೆ
ಅವಸ್ಥೆಗೆ
ಅವಸ್ಥೆಗೇ
ಅವಸ್ಥೆಯ
ಅವಸ್ಥೆ-ಯನ್ನು
ಅವಸ್ಥೆ-ಯಲ್ಲಿ
ಅವಸ್ಥೆ-ಯಲ್ಲೂ
ಅವಸ್ಥೆ-ಯಲ್ಲೇ
ಅವಸ್ಥೆ-ಯಾದರೆ
ಅವಸ್ಥೆ-ಯಿಂದ
ಅವಸ್ಥೆಯು
ಅವಸ್ಥೆಯೇ
ಅವಸ್ಥೆ-ಯೊಂದು
ಅವಹೇ-ಳನ
ಅವಾಂಗ್
ಅವಾಂತ-ರ-ದಲ್ಲಿ
ಅವಾಕ್ಕಾದ
ಅವಾಙ್ಮಾನ-ಸ-ಗೋ-ಚರಂ
ಅವಿ-ಚಲಶ್ರದ್ಧೆ
ಅವಿ-ಚಾರ-ಗಳ
ಅವಿಚ್ಛಿನ್ನ
ಅವಿಚ್ಛಿನ್ನ-ವಾಗಿ
ಅವಿಚ್ಛಿನ್ನ-ವಾದ
ಅವಿತಿಟ್ಟ
ಅವಿತು-ಕೊಂಡಿತು
ಅವಿತು-ಕೊಳ್ಳುವನು
ಅವಿದ್ಯಾರ
ಅವಿದ್ಯಾ-ವಂತ-ನಾದ
ಅವಿದ್ಯೆ
ಅವಿದ್ಯೆ-ಗಳಿಂದ
ಅವಿದ್ಯೆ-ಗಳು
ಅವಿವಾಹಿತ
ಅವಿವಾಹಿತ-ನಾಗಿದ್ದರೆ
ಅವಿವಾ-ಹಿತನು
ಅವಿವಾಹಿತ-ರಾದ
ಅವಿವಾ-ಹಿತರು
ಅವು
ಅವು-ಗಳ
ಅವು-ಗಳನ್ನು
ಅವು-ಗಳನ್ನೆ
ಅವು-ಗಳನ್ನೆಲ್ಲ
ಅವು-ಗಳನ್ನೆಲ್ಲಾ
ಅವು-ಗಳನ್ನೊಂದಿಷ್ಟು
ಅವು-ಗಳನ್ನೋದಿ
ಅವು-ಗಳಲ್ಲ-ಡ-ಗಿ-ರುವ
ಅವು-ಗಳಲ್ಲಿ
ಅವು-ಗ-ಳಲ್ಲೇ
ಅವು-ಗಳಿಂದ
ಅವು-ಗಳಿಂದೆಷ್ಟು
ಅವು-ಗಳಿ-ಗಿಂತ
ಅವು-ಗಳಿಗೆ
ಅವು-ಗಳಿಗೆಲ್ಲ
ಅವು-ಗಳು
ಅವು-ಗಳೂ
ಅವು-ಗಳೆಲ್ಲ
ಅವು-ಗಳೆಲ್ಲ-ವು-ಗಳ
ಅವು-ಗಳೇ
ಅವು-ಗಳೊ-ಡನೆ
ಅವೂ
ಅವೆ-ರಡಕ್ಕೂ
ಅವೆರಡನ್ನೂ
ಅವೆ-ರಡೂ
ಅವೆಲ್ಲ
ಅವೆಲ್ಲ-ವನ್ನೂ
ಅವೆಲ್ಲವೂ
ಅವೆಲ್ಲಾ
ಅವೆಷ್ಟು
ಅವೇ
ಅವೇನು
ಅವೈಜ್ಞಾನಿ-ಕ-ವೆನ್ನುತ್ತಾರೆ
ಅವೈದಿಕ
ಅವೈದಿ-ಕವೂ
ಅವ್ಯಕ್ತ
ಅವ್ಯಕ್ತ-ದಿಂದ
ಅವ್ಯಕ್ತ-ಶಕ್ತಿ-ಯಲಿ
ಅವ್ಯವಸ್ಥೆ
ಅವ್ಯವಸ್ಥೆ-ಯಿಂದ
ಅಶಕ್ತರು
ಅಶನಿ
ಅಶರೀರ-ವಾಣಿ-ಯನ್ನು
ಅಶಾಸ್ತ್ರೀಯ
ಅಶಾಸ್ತ್ರೀ-ಯವೂ
ಅಶಿಥಿಲ-ಪರಿ-ರಂಭಃ
ಅಶುದ್ಧ
ಅಶುದ್ಧ-ವಾದು-ದೆಲ್ಲ-ದರ
ಅಶೋಕ
ಅಶೋ-ಕನ
ಅಶೋಕ-ನಾದರೋ
ಅಶೋಕ-ನಿಗೆ
ಅಶೋ-ಕನು
ಅಶೋಕ-ಮಂತಃ
ಅಶ್ರದ್ಧೆಯೇ
ಅಶ್ರು-ಜಲ-ದಿಂದ
ಅಶ್ರುಜಲ-ಪಾನ
ಅಶ್ರುರಾಷಿ
ಅಶ್ಲೀಲ
ಅಶ್ಲೀಲ-ತೆ-ಯಿಂದ
ಅಶ್ಲೀಲ-ವಾದ
ಅಶ್ವಬಲ
ಅಶ್ವಾರೂಢರು
ಅಷ್ಟಕ್ಕೇ
ಅಷ್ಟನ್ನಾದರೂ
ಅಷ್ಟನ್ನು
ಅಷ್ಟರ-ಮಟ್ಟಿಗೆ
ಅಷ್ಟ-ರಿಂದಲೇ
ಅಷ್ಟ-ರೊ-ಳಗೆ
ಅಷ್ಟಷಟ್ಟದಿ-ಭಾವ-ಗೀತೆ-ಗಳ-ಕಬ್ಬ-ಗಳ
ಅಷ್ಟಷ್ಟು
ಅಷ್ಟಾಧ್ಯಾಯಿ-ಯನ್ನು
ಅಷ್ಟಾವಿಂಶತಿ
ಅಷ್ಟಿಲ್ಲ
ಅಷ್ಟು
ಅಷ್ಟು-ಮಾತ್ರ
ಅಷ್ಟು-ವಾಸಿ
ಅಷ್ಟೂ
ಅಷ್ಟೆ
ಅಷ್ಟೇ
ಅಷ್ಟೇಕೆ
ಅಷ್ಟೇನು
ಅಷ್ಟೇನೂ
ಅಷ್ಟೊಂದು
ಅಷ್ಟೊಂದೆಲ್ಲಾ
ಅಸಂಖ್ಯ
ಅಸಂಖ್ಯ-ವಾದ
ಅಸಂಖ್ಯಾತ
ಅಸಂಖ್ಯಾತ-ವಾದ
ಅಸಂಗ-ತವೇಕಾ-ಗುತ್ತದೆ
ಅಸಂಗ-ತೋಕ್ತಿ-ಯಂತೆ
ಅಸಂತುಷ್ಟ-ನಾಗಿ-ರ-ಬೇಕು
ಅಸಂಬದ್ದತೆ
ಅಸಂಬದ್ಧ
ಅಸಂಬದ್ಧ-ತೆ-ಯನ್ನು
ಅಸಂಬದ್ಧ-ವಾಗ-ಲಾರದು
ಅಸಂಬದ್ಧ-ವಾಗಿ
ಅಸಂಭವ
ಅಸಂಭ-ವ-ವಲ್ಲ
ಅಸಂಭವ್
ಅಸಡ್ಡೆಯ
ಅಸತ್
ಅಸತ್ತೂ
ಅಸತ್ಯ
ಅಸತ್ಯ-ದಿಂದ
ಅಸತ್ಯ-ವಾಗಿ-ರು-ವುದು
ಅಸತ್ಯ-ವಾದರೆ
ಅಸದಳ
ಅಸದೃಶ
ಅಸದೃಶ-ವಾ-ದುದು
ಅಸಭ್ಯ
ಅಸಮಾಧಾನ
ಅಸಮಾಧಾನ-ಗಳನ್ನು
ಅಸಮಾಧಾನ-ಗೊಂಡು
ಅಸಮಾಧಾನ-ಪಟ್ಟು-ಕೊಳ್ಳುತ್ತಿದ್ದರು
ಅಸಲು
ಅಸಹನೀ-ಯ-ವೆಂದು
ಅಸಹಾಯ
ಅಸಹಾಯ-ಕತೆ
ಅಸಹಾಯ-ಕನು
ಅಸಹ್ಯ-ಕರ-ವಾದ
ಅಸಹ್ಯ-ದಿಂದ
ಅಸಾ-ಧಾರಣ
ಅಸಾಧ್ಯ
ಅಸಾಧ್ಯ-ವಾ-ದುದು
ಅಸಾಧ್ಯವೆ
ಅಸಾಧ್ಯ-ವೆಂದು
ಅಸಾಧ್ಯ-ವೆಂದೆ-ನಿ-ಸುತ್ತದೆ
ಅಸಾ-ಮಾನ್ಯ-ವಾದ
ಅಸಿಲೆ
ಅಸೂಯೆ
ಅಸೂಯೆ-ಗಳು
ಅಸೌ-ಜನ್ಯ-ವೆಂದು
ಅಸ್ತವ್ಯಸ್ತ-ವಾ-ಯಿತು
ಅಸ್ತಿ
ಅಸ್ತಿತ್ವ
ಅಸ್ತಿತ್ವಕ್ಕೆ
ಅಸ್ತಿತ್ವದ
ಅಸ್ತಿತ್ವ-ದಲ್ಲಿತ್ತು
ಅಸ್ತಿತ್ವ-ದಲ್ಲಿತ್ತೇ
ಅಸ್ತಿತ್ವ-ದಲ್ಲಿತ್ತೋ
ಅಸ್ತಿತ್ವ-ದಲ್ಲಿದೆ
ಅಸ್ತಿತ್ವ-ದಲ್ಲಿ-ರುವು-ದೆಂದೂ
ಅಸ್ತಿತ್ವ-ವನ್ನು
ಅಸ್ತಿ-ಭಾರ-ವಾದರೂ
ಅಸ್ತಿಯೂ
ಅಸ್ತ್ರ-ಗಳವು
ಅಸ್ತ್ರ-ವನ್ನು
ಅಸ್ತ್ರ-ಶಕ್ತಿಯ
ಅಸ್ಥಿ
ಅಸ್ಥಿ-ಗತ-ವಾಗಿದೆ
ಅಸ್ಥಿ-ಗ-ತ-ವಾಗಿ-ಬಿಟ್ಟಿವೆ
ಅಸ್ಥಿ-ಪಂಜರ-ದಲ್ಲಿ
ಅಸ್ಥಿ-ಯನ್ನು
ಅಸ್ಥಿ-ಯನ್ನೇ
ಅಸ್ಥಿರ-ನಾಗಿದ್ದರೂ
ಅಸ್ಥಿರ-ನಾಗಿದ್ದೆ
ಅಸ್ಥಿರ-ನಾದೆ
ಅಸ್ಥಿರ-ವಾದ
ಅಸ್ಪಷ್ಟ
ಅಸ್ಪಷ್ಟ-ವಾದ
ಅಸ್ಪುಟ
ಅಸ್ಪುಟದ
ಅಸ್ವಾ-ಭಾವಿಕ
ಅಸ್ವಾ-ಭಾವಿಕ-ವಾದ
ಅಸ್ವಾಲನ
ಅಸ್ಸಾಂನಿಂದ
ಅಸ್ಸಾಮಿನ
ಅಹಂ
ಅಹಂಕಾರ
ಅಹಂಕಾರ-ವನ್ನು
ಅಹಂನ
ಅಹಂಭಾವ
ಅಹಂಭಾವ-ಗಳ
ಅಹಂಭಾವದ
ಅಹಂಭಾವ-ನೆ-ಯನ್ನು
ಅಹಂಭಾವ-ವಿರು-ವುದೋ
ಅಹಂಸ್ರೋತೇ
ಅಹಮಹಮಿತಿ
ಅಹಮಿ-ಕೆಯೆ
ಅಹಮಿಹು-ದಿಲ್ಲಿ
ಅಹಹ
ಅಹಿಂಸೆ
ಅಹಿಂಸೆಯ
ಅಹಿಂಸೆ-ಯನ್ನು
ಅಹಿಂಸೆಯೆ
ಅಹಿಂಸೆಯೇ
ಅಹಿರಿ-ಟೊಲಘಾಟಿನ
ಅಹೇತು-ಕ-ವಾದ
ಆ
ಆಂಗ್ಲ
ಆಂಗ್ಲೇಯ
ಆಂಗ್ಲೇ-ಯರು
ಆಂಗ್ಲೇ-ಯರೇ
ಆಂಜನೇಯನ
ಆಂತ-ರ-ದಲ್ಲಿ
ಆಂತ-ರದಲ್ಲಿ-ರುವ
ಆಂತ-ರಿಕ
ಆಂತರ್ಯ-ದಲ್ಲಿ
ಆಂತರ್ಯ-ದಲ್ಲಿ-ರುವ
ಆಂತರ್ಯ-ದಲ್ಲಿ-ರು-ವುದು
ಆಂತರ್ಯ-ದಿಂದ
ಆಂತರ್ಯದೊ-ಳಗಿ-ನಿಂದ
ಆಂದೋಲನ-ವನ್ನೇ
ಆಂದೋ-ಳನ-ವನ್ನೇ
ಆಂದೋ-ಳನ-ವಾದ
ಆಕರ
ಆಕರ್ಷಕ
ಆಕರ್ಷಣ
ಆಕರ್ಷಣ-ದಲ್ಲಿ
ಆಕರ್ಷಣೀಯ-ನಾಗುತ್ತಿದ್ದ
ಆಕರ್ಷಣೀಯ-ವಾಗಿದೆ
ಆಕರ್ಷಣೀಯ-ವಾ-ದು-ದನ್ನು
ಆಕರ್ಷಣೆ
ಆಕರ್ಷಣೆ-ಗಳಿಗೆ
ಆಕರ್ಷಣೆ-ಗೊಳಗಾಗು-ವರು
ಆಕರ್ಷ-ಣೆಯೇ
ಆಕರ್ಷಿತ-ರಾಗಿ
ಆಕರ್ಷಿತ-ರಾದರೆ
ಆಕರ್ಷಿಸ-ಬಲ್ಲೆಯೋ
ಆಕರ್ಷಿ-ಸಲ್ಪ-ಡು-ವರು
ಆಕರ್ಷಿ-ಸುವ
ಆಕರ್ಷಿ-ಸು-ವುದು
ಆಕಸ್ಮಿ-ಕ-ವಾಗಿ
ಆಕಾಂಕ್ಷೆ
ಆಕಾರ
ಆಕಾರ-ಗ-ಣನ್ನು
ಆಕಾರ-ಗಳನ್ನು
ಆಕಾರದ
ಆಕಾರ-ದಲ್ಲಿ
ಆಕಾರ-ವನ್ನು
ಆಕಾಶ
ಆಕಾ-ಶದ
ಆಕಾಶ-ದಂತೆ
ಆಕಾಶ-ದಲ್ಲಿ
ಆಕಾಶ-ದಿಂದ
ಆಕಾಶ-ವನ್ನು
ಆಕಾಶವು
ಆಕಾಶವೂ
ಆಕಾಶೆ
ಆಕಾಶೇ
ಆಕುಲ
ಆಕೃತಿ
ಆಕೃತಿ-ಯಲ್ಲಿ
ಆಕೆ
ಆಕೆ-ಗವ
ಆಕೆಗೂ
ಆಕೆಗೆ
ಆಕೆಯ
ಆಕೆ-ಯನ್ನು
ಆಕೆಯು
ಆಕೆಯೂ
ಆಕೆಯೇ
ಆಕ್ರಮಿಸಿ-ಕೊಂಡಿವೆ
ಆಕ್ರಮಿ-ಸಿದ್ದಾರೆ
ಆಕ್ರಮಿಸಿವೆ
ಆಕ್ಷೇ-ಪಣೆ
ಆಕ್ಷೇ-ಪಣೆ-ಮಾಡ-ಲಿಲ್ಲ
ಆಕ್ಷೇಪಿಸ-ಬೇಕು
ಆಕ್ಷೇಪಿ-ಸಿದಾಗ
ಆಕ್ಷೇಪಿ-ಸಿದ್ದರು
ಆಖಿ
ಆಗ
ಆಗಂತುಕ-ನೊಬ್ಬನು
ಆಗ-ತಕ್ಕದ್ದೆ
ಆಗ-ತಾನೇ
ಆಗ-ದಂತೆ
ಆಗದ-ವ-ರಿಂದ
ಆಗ-ದಿದ್ದರೆ
ಆಗ-ದಿ-ರಲಿ
ಆಗದೆ
ಆಗ-ಬಲ್ಲರು
ಆಗ-ಬಲ್ಲ-ರೆಂದು
ಆಗ-ಬಲ್ಲಿರಿ
ಆಗ-ಬ-ಹುದು
ಆಗ-ಬೇಕಾಗಿದೆ
ಆಗ-ಬೇ-ಕಾದ್ದೇನು
ಆಗ-ಬೇಕು
ಆಗಮನ
ಆಗಮ-ನಕ್ಕೆ
ಆಗ-ಮ-ನದಿಂದ
ಆಗಮನ-ವಾಗಿದೆ
ಆಗಮನ-ವಾಯಿ-ತೆಂದರೆ
ಆಗಮಾದಿ-ಗಳನ್ನು
ಆಗ-ಲಾರ-ದವ-ರಾಗಿದ್ದಾರೆ
ಆಗ-ಲಾರದು
ಆಗಲಿ
ಆಗ-ಲಿಲ್ಲ
ಆಗಲೀ
ಆಗಲೂ
ಆಗಲೆ
ಆಗಲೇ
ಆಗಸ-ದಲಿ
ಆಗಸ-ದಲ್ಲಿ
ಆಗ-ಸದಿ
ಆಗಸವ
ಆಗಸ-ವನು
ಆಗಸ್ಟ್
ಆಗಾಗ
ಆಗಾಗೆ
ಆಗಾಗ್ಗೆ
ಆಗಾಗ್ಯೆ
ಆಗಿ
ಆಗಿತ್ತು
ಆಗಿದೆ
ಆಗಿ-ದೆಯೇ
ಆಗಿ-ದೆಯೊ
ಆಗಿ-ದೆಯೋ
ಆಗಿದ್ದ
ಆಗಿದ್ದನು
ಆಗಿದ್ದರು
ಆಗಿದ್ದರೆ
ಆಗಿದ್ದವು
ಆಗಿದ್ದಾನೆ
ಆಗಿದ್ದಾರೆ
ಆಗಿದ್ದಾಳೆ
ಆಗಿದ್ದೀಯೆ
ಆಗಿದ್ದೀರಿ
ಆಗಿದ್ದುವು
ಆಗಿದ್ದೇನೆ
ಆಗಿನ
ಆಗಿ-ನ-ವರು
ಆಗಿ-ನಿಂದ
ಆಗಿ-ಬಿಟ್ಟರೆ
ಆಗಿ-ಬಿಟ್ಟಿದೆ
ಆಗಿ-ಬಿಡುತ್ತಾರೆ
ಆಗಿ-ಬಿಡು-ವು-ದೆಂದು
ಆಗಿಯೇ
ಆಗಿ-ರ-ಬೇ-ಕಾದ
ಆಗಿ-ರ-ಬೇಕು
ಆಗಿ-ರಲಿ
ಆಗಿ-ರುತ್ತದೆ
ಆಗಿ-ರುವ
ಆಗಿ-ರುವನು
ಆಗಿ-ರುವರು
ಆಗಿ-ರುವ-ವ-ರನ್ನೂ
ಆಗಿ-ರು-ವಿರಿ
ಆಗಿ-ರು-ವುದು
ಆಗಿ-ರು-ವುವೋ
ಆಗಿ-ರುವೆ
ಆಗಿಲ್ಲ
ಆಗಿವೆ
ಆಗಿ-ಹೊ-ಯಿತು
ಆಗಿ-ಹೋಗ-ಬೇಕು
ಆಗಿ-ಹೋ-ಯಿತು
ಆಗಿ-ಹೋ-ಯಿತೆಂಬು-ದನ್ನು
ಆಗುತ್ತದೆ
ಆಗುತ್ತದೆಂದು
ಆಗುತ್ತದೆ-ಯೇನು
ಆಗುತ್ತವೆ
ಆಗುತ್ತ-ವೆಂದು
ಆಗುತ್ತಾನೆ
ಆಗುತ್ತಾರೆ
ಆಗುತ್ತಿತ್ತು
ಆಗುತ್ತಿದೆ
ಆಗುತ್ತಿದ್ದ
ಆಗುತ್ತಿದ್ದರೆ
ಆಗುತ್ತಿರ-ಬೇಕು
ಆಗುತ್ತಿ-ರ-ಲಿಲ್ಲ
ಆಗುತ್ತಿ-ರ-ಲಿಲ್ಲವೋ
ಆಗುತ್ತಿಲ್ಲ
ಆಗುತ್ತೇನೆ
ಆಗುವ
ಆಗು-ವನು
ಆಗು-ವರು
ಆಗು-ವಷ್ಟು
ಆಗುವುದದೆಷ್ಟೋ
ಆಗು-ವು-ದಾದರೆ
ಆಗು-ವು-ದಿಲ್ಲ
ಆಗು-ವು-ದಿಲ್ಲ-ವೆಂದು
ಆಗು-ವು-ದಿಲ್ಲ-ವೆನ್ನಿ-ಸು-ವುದು
ಆಗು-ವು-ದಿಲ್ಲವೇ
ಆಗು-ವು-ದಿಲ್ಲ-ವೇನು
ಆಗು-ವುದು
ಆಗು-ವುದುಂಟು
ಆಗು-ವುದೆ
ಆಗು-ವುದೆಂಬು-ದಲ್ಲ
ಆಗು-ವುದೇ
ಆಗು-ವು-ದೇನು
ಆಗುವೆ-ನನೇಕ
ಆಗೆ
ಆಗೆಲ್ಲಾ
ಆಗ್ರಾ-ದಿಂದ
ಆಘತ
ಆಘಾತದ
ಆಘಾತ-ವಾಗ-ಬಹು-ದೆಂದೂ
ಆಘಾತ-ವಾ-ಯಿತು
ಆಘಾತ್
ಆಘೂರ್ಣಿತಂ
ಆಚಂಡಾಲಾಪ್ರತಿ-ಹ-ತರಯೋ
ಆಚಂದ್ರಾರ್ಕ-ವಾದ
ಆಚಮನಾ-ನಂತರ
ಆಚ-ರಣೆ-ಗಳನ್ನು
ಆಚರ-ಣೆಗೆ
ಆಚ-ರಣೆಯ
ಆಚ-ರಣೆ-ಯ-ದಲ್ಲಿದೆ
ಆಚ-ರಣೆ-ಯನ್ನು
ಆಚ-ರಣೆ-ಯಲ್ಲಿ
ಆಚ-ರಣೆಯೆ
ಆಚರಿಸ-ದಂತೆ
ಆಚರಿಸಬಯ-ಸಿ-ದರು
ಆಚರಿಸ-ಬೇ-ಕಾದ
ಆಚರಿಸ-ಬೇಕು
ಆಚರಿಸ-ಬೇಕೆಂದಿರುವೆ
ಆಚರಿ-ಸುತ್ತ
ಆಚರಿ-ಸುತ್ತಿರು-ವರು
ಆಚರಿ-ಸುವ
ಆಚರಿ-ಸುವ-ವರು
ಆಚರಿ-ಸುವಾಗ
ಆಚರಿ-ಸು-ವು-ದ-ರಿಂದ
ಆಚರಿ-ಸುವೆ
ಆಚಾರ
ಆಚಾ-ರಕ್ಕೆ
ಆಚಾರ-ಗಳನ್ನು
ಆಚಾರ-ಗಳಲ್ಲಿ
ಆಚಾರ-ಗಳೆಲ್ಲ-ವನ್ನೂ
ಆಚಾರ-ಗಳೇ
ಆಚಾರ-ದಲ್ಲಿ-ರುವ
ಆಚಾರ-ನಿಯ-ಮ-ಗಳನ್ನು
ಆಚಾರ-ವಂತ-ರಾದ
ಆಚಾರ-ವಿ-ಚಾರ-ಗಳೂ
ಆಚಾರವ್ಯವ-ಹಾರ-ಗಳ
ಆಚಾರವ್ಯವ-ಹಾರ-ಗಳು
ಆಚಾರವ್ಯವ-ಹಾರ-ಗಳೊಂದಿಗೆ
ಆಚಾರ-ಶೀಲ
ಆಚಾರ-ಶೀಲ-ನಾದ-ವನು
ಆಚಾರ-ಶೀಲನೂ
ಆಚಾರ್ಯ
ಆಚಾರ್ಯನೇ
ಆಚಾರ್ಯ-ರು-ಗಳ
ಆಚಾರ್ಯ-ರೆಂದು
ಆಚಾರ್ಯ-ರೆಲ್ಲ
ಆಚೆ
ಆಚೆಗೆ
ಆಚೆಯ
ಆಚೆಯೇ
ಆಚ್ಛಾದನ
ಆಚ್ಛಾದಿತ-ನಾಗಿ-ರು-ವನು
ಆಚ್ಛಾದಿತ-ರಾಗಿದ್ದರೂ
ಆಚ್ಛಾದಿತ-ವಾಗಿದೆ
ಆಛೆ
ಆಜನ್ಮ
ಆಜನ್ಮ-ಸಿದ್ಧ
ಆಜೀವ-ಪರ್ಯಂತ
ಆಜು
ಆಜ್ಞಾ
ಆಜ್ಞಾ-ನುವರ್ತಿ-ಗಳು
ಆಜ್ಞಾ-ನು-ಸಾರ-ವಾಗಿ
ಆಜ್ಞಾ-ಪಿ-ತ-ನಾಗಿ
ಆಜ್ಞಾಪಿ-ಸಿದ
ಆಜ್ಞಾಪಿಸಿ-ದರು
ಆಜ್ಞಾಪಿಸಿದ್ದನ್ನೆಲ್ಲಾ
ಆಜ್ಞೆ
ಆಜ್ಞೆಗೆ
ಆಜ್ಞೆಯ
ಆಜ್ಞೆ-ಯನ್ನು
ಆಜ್ಞೆ-ಯಿಂದ
ಆಜ್ಞೆ-ಯಿಂದಲೇ
ಆಟ
ಆಟಕೆ
ಆಟದ
ಆಟದಾ
ಆಟ-ದೊಳು
ಆಟ-ಪಾಠ-ಗಳನ್ನು
ಆಟ-ವನ್ನಾಡಿ
ಆಟವಾಡಿ
ಆಟವಾಡುತ್ತಿದ್ದ
ಆಟವಾಡುತ್ತಿದ್ದೆ
ಆಟವಾಡು-ವಾಗ
ಆಟವಾಡುವೆ
ಆಟವು
ಆಡಂಬ-ರದ
ಆಡಂಬರ-ದಿಂದ
ಆಡ-ಬೇಕಾಗಿದೆ
ಆಡ-ಲಿಲ್ಲ
ಆಡಳಿತ-ವನ್ನಿಟ್ಟು-ಕೊಂಡಿದ್ದ
ಆಡಿ
ಆಡಿ-ಕೊಳ್ಳುತ್ತಿದ್ದರೂ
ಆಡಿದ
ಆಡಿನ
ಆಡಿಲ್ಲ
ಆಡುತ್ತ
ಆಡುತ್ತಿದ್ದ
ಆಡು-ಭಾಷೆ-ಯಾಗಿ
ಆಡು-ಭಾಷೆ-ಯೊಂದಿ-ಗಿನ
ಆಡುವ
ಆಡುವಾ-ಟವೆ
ಆಡು-ವುದಕ್ಕೆ
ಆಡು-ವುದೂ
ಆಣತಿ-ಯಂತೆ
ಆಣೆ-ಗಳನ್ನು
ಆಣೆ-ಯನ್ನೇ
ಆಣೆಯಿಡುವೆ
ಆತ
ಆತಂಕ-ಗಳನ್ನು
ಆತಂಕ-ಗಳಿಗೂ
ಆತಂಕ-ಗಳಿವೆ
ಆತಂಕ-ಗಳು
ಆತಂಕ-ಗಳೇ
ಆತಂಕ-ವಿದೆ
ಆತನ
ಆತ-ನಂಥ
ಆತ-ನನ್ನಾವ
ಆತ-ನನ್ನು
ಆತ-ನನ್ನೇ
ಆತ-ನಲ್ಲಿ
ಆತನಲ್ಲಿದ್ದ
ಆತನಲ್ಲಿ-ರುವ
ಆತ-ನಾದರೋ
ಆತ-ನಿಂದ
ಆತನಿಗಿ-ರ-ಲಿಲ್ಲ
ಆತ-ನಿಗೆ
ಆತನಿಗೊಬ್ಬ
ಆತನಿ-ರುವ
ಆತನು
ಆತನೂ
ಆತನೆ
ಆತನೇ
ಆತ-ನೊ-ಡನೆ
ಆತ-ನೊಬ್ಬ
ಆತಾರಕ
ಆತಿಥ್ಯ-ವನ್ನು
ಆತುರ
ಆತುರ-ದಿಂದ
ಆತುರ-ಪಡುತ್ತಿದೆಯೋ
ಆತ್ಮ
ಆತ್ಮಕ್ಕೆ
ಆತ್ಮಗೆ
ಆತ್ಮ-ಗೌ-ರವ
ಆತ್ಮ-ಗೌ-ರವ-ವನ್ನೆಲ್ಲಾ
ಆತ್ಮ-ಘಾತ-ಕರು
ಆತ್ಮ-ಚರಿತ್ರಾತ್ಮಕ
ಆತ್ಮಜ್ಞಾನ
ಆತ್ಮಜ್ಞಾನಕ್ಕಿಂತ
ಆತ್ಮಜ್ಞಾನಕ್ಕೆ
ಆತ್ಮಜ್ಞಾನಕ್ಕೋಸ್ಕರ
ಆತ್ಮಜ್ಞಾನ-ಗಳನ್ನು
ಆತ್ಮಜ್ಞಾನ-ದಿಂದ
ಆತ್ಮಜ್ಞಾನ-ವನ್ನು
ಆತ್ಮಜ್ಞಾನ-ವಾದರೆ
ಆತ್ಮಜ್ಞಾನವು
ಆತ್ಮಜ್ಞಾನ-ವುಂಟಾಗು-ವುದು
ಆತ್ಮಜ್ಞಾನವೆ
ಆತ್ಮಜ್ಞಾನಿಗೆ
ಆತ್ಮ-ತತ್ತ್ವ-ವನ್ನು
ಆತ್ಮ-ತತ್ತ್ವ-ವಿ-ಚಾರ
ಆತ್ಮತ್ಯಾಗದ
ಆತ್ಮತ್ಯಾಗ-ನಿರತ-ರಾದ
ಆತ್ಮದ
ಆತ್ಮ-ದರ್ಶನಕ್ಕಾಗಿ
ಆತ್ಮ-ದರ್ಶನ-ದಲ್ಲಿ
ಆತ್ಮ-ದರ್ಶನವೆ
ಆತ್ಮ-ದರ್ಶನ-ವೆಂದ-ರೇನು
ಆತ್ಮ-ದಲ್ಲಿ
ಆತ್ಮ-ದೊಳಿ-ಳಿದು
ಆತ್ಮನ
ಆತ್ಮ-ನನ್ನು
ಆತ್ಮ-ನಲ್ಲದೆ
ಆತ್ಮ-ನಲ್ಲಿ
ಆತ್ಮ-ನಲ್ಲೇ
ಆತ್ಮ-ನಿ-ಗಾಗಿ
ಆತ್ಮ-ನಿಗೆ
ಆತ್ಮ-ನಿಗ್ರಹ
ಆತ್ಮ-ನಿ-ರು-ವುದು
ಆತ್ಮ-ನಿಷ್ಠೆಯೂ
ಆತ್ಮನು
ಆತ್ಮ-ನೆಂದು
ಆತ್ಮನೇ
ಆತ್ಮ-ನೊಬ್ಬನೇ
ಆತ್ಮನೋ
ಆತ್ಮ-ಪರಿಜ್ಞಾನ
ಆತ್ಮ-ಪರಿಜ್ಞಾನ-ದಲ್ಲಿ
ಆತ್ಮ-ಪರಿಜ್ಞಾನ-ವನ್ನು
ಆತ್ಮ-ಪರಿಜ್ಞಾನ-ವುಳ್ಳ-ವನು
ಆತ್ಮ-ಪರಿಜ್ಞಾನ-ವುಳ್ಳ-ವ-ರಿಗೆ
ಆತ್ಮ-ಪರಿಜ್ಞಾನವೇ
ಆತ್ಮಪ್ರಕಾಶಕ್ಕಿ-ರುವ
ಆತ್ಮಪ್ರಕಾಶ-ನಕ್ಕೆ
ಆತ್ಮಪ್ರತಿಷ್ಠಾ-ಪಿತ-ನಾದಾಗ
ಆತ್ಮಪ್ರತಿಷ್ಠಿತ-ರಾದ
ಆತ್ಮ-ಬಲ-ವನ್ನು
ಆತ್ಮ-ರಕ್ಷಣೆ
ಆತ್ಮ-ರಕ್ಷಣೆ-ಗಾಗಿ
ಆತ್ಮ-ರಕ್ಷಣೆಯ
ಆತ್ಮ-ರೂಪ-ದಲ್ಲಿ
ಆತ್ಮ-ರೆಲ್ಲ
ಆತ್ಮ-ವಂಚನೆ
ಆತ್ಮ-ವಂಚನೆಯ
ಆತ್ಮ-ವನ್ನರಿತ-ವನ
ಆತ್ಮ-ವನ್ನರಿಯ-ದ-ವನ
ಆತ್ಮ-ವನ್ನಾಗಲೀ
ಆತ್ಮ-ವನ್ನು
ಆತ್ಮ-ವನ್ನೆಲ್ಲಾ
ಆತ್ಮ-ವನ್ನೇ
ಆತ್ಮ-ವಲ್ಲ
ಆತ್ಮ-ವಸ್ತುವು
ಆತ್ಮ-ವಾ-ರಿಗೆ
ಆತ್ಮ-ವಿಕಾಸ
ಆತ್ಮ-ವಿಕಾಸಕ್ಕೆ
ಆತ್ಮ-ವಿಕಾಸ-ವಾಗುತ್ತ-ದೆಯೋ
ಆತ್ಮ-ವಿಕಾಸವೆ
ಆತ್ಮ-ವಿಜ್ಞಾನ-ವಿಲ್ಲವೋ
ಆತ್ಮ-ವಿದೆ
ಆತ್ಮ-ವಿಶ್ವಾಸ-ವಿ-ರಲಿ
ಆತ್ಮ-ವಿಶ್ವಾಸ-ವಿ-ರುವು-ದೆಂಬು-ದನ್ನು
ಆತ್ಮ-ವಿಶ್ವಾಸ-ವಿಲ್ಲ
ಆತ್ಮವು
ಆತ್ಮವೂ
ಆತ್ಮವೆ
ಆತ್ಮ-ವೆಂದ-ರೇನು
ಆತ್ಮ-ವೆಂದು
ಆತ್ಮ-ವೆನ್ನುತ್ತೇವೆ
ಆತ್ಮ-ವೆಲ್ಲಾ
ಆತ್ಮ-ವೆಲ್ಲಿಯು
ಆತ್ಮವೇ
ಆತ್ಮ-ವೊಂದಾಗಿ-ರು-ವುದು
ಆತ್ಮ-ವೊಂದೇ
ಆತ್ಮ-ಶಕ್ತಿ
ಆತ್ಮ-ಶಕ್ತಿ-ಯನ್ನು
ಆತ್ಮ-ಶಕ್ತಿ-ಯಲ್ಲಿ
ಆತ್ಮ-ಶಕ್ತಿಯು
ಆತ್ಮ-ಸಂಯ-ಮದ
ಆತ್ಮ-ಸಾಕ್ಷಾತ್ಕಾರ
ಆತ್ಮ-ಸಾಕ್ಷಾತ್ಕಾರಕ್ಕಾಗಿ
ಆತ್ಮ-ಸಾಕ್ಷಾತ್ಕಾ-ರಕ್ಕೂ
ಆತ್ಮ-ಸಾಕ್ಷಾತ್ಕಾ-ರಕ್ಕೆ
ಆತ್ಮ-ಸಾಕ್ಷಾತ್ಕಾರದ
ಆತ್ಮ-ಸಾಕ್ಷಾತ್ಕಾರ-ವನ್ನು
ಆತ್ಮ-ಸಾಕ್ಷಾತ್ಕಾರ-ವಾಗಿ-ದೆಯೊ
ಆತ್ಮ-ಸಾಕ್ಷಾತ್ಕಾರ-ವಾಗುವ
ಆತ್ಮ-ಸಾಕ್ಷಾತ್ಕಾರ-ವಾಗು-ವ-ವ-ರೆಗೂ
ಆತ್ಮ-ಸಾಕ್ಷಾತ್ಕಾರ-ವಾ-ಗು-ವು-ದಿಲ್ಲ
ಆತ್ಮ-ಸಾಕ್ಷಾತ್ಕಾರ-ವಾಗು-ವುದು
ಆತ್ಮ-ಸಾಕ್ಷಾತ್ಕಾ-ರವೇ
ಆತ್ಮ-ಸಾಕ್ಷಾತ್ಕಾರಾಕಾಂಕ್ಷಿಯಾ-ದ-ವನು
ಆತ್ಮಸ್ಥಿ-ತ-ನಾಗಿ
ಆತ್ಮ-ಹತ್ಯೆ
ಆತ್ಮಹತ್ಯೆಯೇ
ಆತ್ಮಾ
ಆತ್ಮಾನಂ
ಆತ್ಮಾ-ನು-ಭವ
ಆತ್ಮಾ-ನು-ಭವ-ವಾಗುವ
ಆತ್ಮಾ-ನು-ಭವ-ವುಂಟಾ-ಗುತ್ತದೆ
ಆತ್ಮಾ-ರಾಮ-ಪುರುಷರ
ಆತ್ಮಾರ್ಪಣೆ
ಆತ್ಮಾರ್ಪಣೆ-ಯನ್ನು
ಆತ್ಮೀಯತೆ-ಯನ್ನು
ಆತ್ಮೋದ್ಧಾರಕ್ಕೋಸ್ಕರ
ಆತ್ಮೋನ್ನ-ತಿಯ
ಆತ್ಮೋನ್ನತಿ-ಯನ್ನು
ಆತ್ಯಂತಿಕ
ಆದ
ಆದ-ಕಾರಣ
ಆದ-ಕಾರ-ಣ-ದಿಂದಲೇ
ಆದ-ಕಾರ-ಣವೆ
ಆದ-ಕಾರ-ಣವೇ
ಆದದ್ದಿಲ್ಲ
ಆದದ್ದೇನು
ಆದದ್ದೇನೋ
ಆದ-ಮೇಲೆ
ಆದರ
ಆದರ-ದಿಂದ
ಆದ-ರದು
ಆದರಾತಿಥ್ಯ-ಗಳನ್ನು
ಆದರಿದು
ಆದರಿ-ಸುತ್ತಾರೆ
ಆದರು
ಆದರೂ
ಆದರೆ
ಆದರೆಂದೆಂದಿಗೂ
ಆದ-ರೇ-ನಂತೆ
ಆದ-ರೇನು
ಆದರ್ಶ
ಆದರ್ಶಕ್ಕೆ
ಆದರ್ಶ-ಗಳ
ಆದರ್ಶ-ಗಳನ್ನು
ಆದರ್ಶ-ಗಳಲ್ಲಿ
ಆದರ್ಶ-ಗಳಿಗೆ
ಆದರ್ಶ-ಗಳೆಲ್ಲಾ
ಆದರ್ಶ-ಜೀವನ-ವನ್ನು
ಆದರ್ಶದ
ಆದರ್ಶ-ದಲ್ಲಿ
ಆದರ್ಶ-ದಿಂದ
ಆದರ್ಶ-ನಾರಿ-ಯರ
ಆದರ್ಶಪ್ರೇಮ
ಆದರ್ಶ-ವನ್ನಾಗಿಟ್ಟು-ಕೊಂಡು
ಆದರ್ಶ-ವನ್ನು
ಆದರ್ಶ-ವನ್ನೆ-ರೆದು
ಆದರ್ಶ-ವಾಗಿ
ಆದರ್ಶ-ವಾದ
ಆದರ್ಶ-ವಾದರೊ
ಆದರ್ಶ-ವೆಂದರೆ
ಆದರ್ಶ-ವೆಂದು
ಆದರ್ಶ-ಶೀಲ-ವ-ತಿಯ-ರೆಂದು
ಆದರ್ಶಸ್ವ-ರೂಪ-ವಾದ
ಆದವು
ಆದಷ್ಟು
ಆದಾಗ
ಆದಾಗ್ಗೂ
ಆದಾಗ್ಯೂ
ಆದಾನ
ಆದಾ-ಯ-ದಿಂದ
ಆದಾಯ-ವಿಲ್ಲ
ಆದಾ-ಯವೆಲ್ಲವೂ
ಆದಿ
ಆದಿ-ಕವಿ
ಆದಿ-ಗುರು
ಆದಿತ್ಯ-ವಾರ
ಆದಿ-ಯನ್ನು
ಆದಿ-ಯಲ್ಲಿ
ಆದಿ-ಯಲ್ಲಿದ್ದ
ಆದಿ-ಯಲ್ಲಿದ್ದುದು
ಆದಿ-ಯಲ್ಲಿಯೂ
ಆದಿ-ಯಿಂದಲು
ಆದಿ-ಯಿಲ್ಲ
ಆದಿಯೇ
ಆದಿ-ವಾಣಿ
ಆದಿ-ವಾಣಿಯು
ಆದಿ-ವಾಸಿ-ಗಳ
ಆದಿ-ಶಕ್ತಿ
ಆದು-ದ-ರಿಂದ
ಆದೇಶ-ಗ-ಳಾದರೆ
ಆದೇ-ಶಿ-ಸಿ-ದನು
ಆದೊಂದು
ಆದೊಡೇನಂತಾತ್ಮ-ವೆಂಬುದು
ಆದ್ದ-ರಿಂದ
ಆದ್ದ-ರಿಂದ-ನಿಮಗೆ
ಆದ್ದ-ರಿಂದಲೆ
ಆದ್ದ-ರಿಂದಲೇ
ಆದ್ಯಂತ-ಗಳ
ಆದ್ಯಂತ-ವಾಗಿ
ಆಧ-ರಿಸಿ
ಆಧರಿ-ಸಿ-ರು-ವುದು
ಆಧಾರ
ಆಧಾರಕ್ಕಾಗಿ
ಆಧಾರ-ಗಳೆಲ್ಲ
ಆಧಾರದ
ಆಧಾರ-ವಾದ
ಆಧಾರ-ವಿದೆ
ಆಧಾರವು
ಆಧಾರವೆ
ಆಧಿಪತ್ಯ
ಆಧಿಪತ್ಯ-ದಲ್ಲಿ
ಆಧಿಪತ್ಯ-ವನ್ನು
ಆಧಿವ್ಯಾಧಿ-ಗಳನ್ನೂ
ಆಧುನಿಕ
ಆಧ್ಯಾತ್ಮ-ದಲ್ಲಿ
ಆಧ್ಯಾತ್ಮಿಕ
ಆಧ್ಯಾತ್ಮಿ-ಕತೆ
ಆಧ್ಯಾತ್ಮಿಕ-ತೆಯ
ಆಧ್ಯಾತ್ಮಿಕ-ತೆ-ಯನ್ನು
ಆಧ್ಯಾತ್ಮಿಕ-ತೆ-ಯಲ್ಲಿ
ಆನಂದ
ಆನಂದಕ್ಕೆ
ಆನಂದದ
ಆನಂದ-ದಲ್ಲಿದ್ದರು
ಆನಂದ-ದಿಂದ
ಆನಂದ-ನನ್ನುದ್ದೇಶಿಸಿ
ಆನಂದ-ನರ್ತನ
ಆನಂದ-ಮತ್ತ-ತೆ-ಯಲ್ಲಿ
ಆನಂದ-ವನ್ನು
ಆನಂದ-ವಾಗುತ್ತದೆ
ಆನಂದ-ವಾ-ಯಿತು
ಆನಂದ-ವುಂಟಾಗು-ವುದೋ
ಆನಂದಿತ-ಚಿತ್ತ-ನಾಗಿ
ಆನಂದಿ-ಸುತ್ತಿದ್ದೇವೆ
ಆನಂದಿ-ಸುವರು
ಆನಂದಿಸು-ವವರು
ಆನು-ವಂಶಿ-ಕತೆ
ಆನುಷಂಗಿ-ಕ-ವಾಗಿ-ರು-ವು-ದ-ರಿಂದ
ಆನೆ
ಆನೆ-ಕುದುರೆ
ಆನೆ-ಚಿನೆ
ಆನೆಯು
ಆಪತ್ಕಾಲದ
ಆಪತ್ತಿ-ನಲ್ಲಿಯೂ
ಆಪತ್ತು-ಗಳಿಗೆಂದು
ಆಪ-ವನ-ರೂಪ
ಆಪೂ-ರಣ-ದಿಂದ
ಆಪೋಶನ-ವಾಗಿ
ಆಪೋಶಿಸಿದ್ದ-ರಿಂದ
ಆಪ್ತ
ಆಪ್ತ-ಪುರುಷರ
ಆಪ್ತರೂ
ಆಪ್ತ-ರೆಂದು
ಆಪ್ತ-ವಾಕ್ಯ
ಆಪ್ತ-ವಾಕ್ಯ-ಗಳಿಂದ
ಆಪ್ಯಾಯ-ಮಾನ-ವಾಗಿವೆ
ಆಫೀ-ಸಿಗೆ
ಆಫೀ-ಸಿ-ನಲ್ಲಿ
ಆಫ್
ಆಬಾರ
ಆಬ್ರಹ್ಮ-ಸಂಭ
ಆಬ್ರಹ್ಮಸ್ತಂಭ
ಆಭ-ರಣ
ಆಭಾಸ
ಆಭಾಸ-ವನ್ನು
ಆಭಾಸವೂ
ಆಮಂತ್ರ-ಣದ
ಆಮಿ
ಆಮೂಲಾಗ್ರ-ವಾಗಿ
ಆಮೆ
ಆಮೆ-ಯನ್ನು
ಆಮೆ-ಯಾಗಿ
ಆಮೇ-ಲಿನ
ಆಮೇಲೂ
ಆಮೇಲೆ
ಆಮೋದ
ಆಮೋ-ದವು
ಆಯಾ
ಆಯಾ-ಸ-ಗೊಂಡಿದೆ
ಆಯಿತು
ಆಯಿತೆಂದು
ಆಯಿತೆನ್ನ-ಬ-ಹುದು
ಆಯುಕ್ಷಯ
ಆಯುಧ-ವಾಗಿ-ದೆಯೊ
ಆಯುರ್ವೇದದ
ಆಯುಷ್ಯವರ್ಧ-ವನೆ
ಆಯ್ಕೆ
ಆರ
ಆರಂಬ-ಗಾ-ರರು
ಆರಂಭ
ಆರಂಭ-ದಲ್ಲಿ
ಆರಂಭ-ದಲ್ಲಿ-ಯೇನೋ
ಆರಂಭ-ಮಾಡಿ
ಆರಂಭ-ಮಾಡಿ-ಕೊಡುತ್ತೇನೆ
ಆರಂಭ-ಮಾಡಿತು
ಆರಂಭ-ವಾಗಿ
ಆರಂಭ-ವಾಗುತ್ತ-ದೆಂಬ
ಆರಂಭ-ವಾ-ಯಿತು
ಆರಂಭಿಸ-ಬೇಕು
ಆರಂಭಿ-ಸಲು
ಆರಂಭಿ-ಸಲ್ಪಡು-ವುದು
ಆರಂಭಿಸಿ
ಆರಂಭಿಸಿ-ದನೋ
ಆರಂಭಿಸಿ-ದರು
ಆರಂಭಿಸಿ-ದರೆ
ಆರಂಭಿಸಿ-ದಾಗ
ಆರಂಭಿಸಿ-ದು-ದ-ರಿಂದ
ಆರಂಭಿ-ಸುತ್ತೀರಿ
ಆರಂಭಿ-ಸುತ್ತೇನೆ
ಆರಂಭಿ-ಸುವೆ
ಆರತಿ
ಆರತಿ-ಗೀತೆ
ಆರ-ತಿಯ
ಆರತಿ-ಯಿದೊ
ಆರದು
ಆರನ್ನು
ಆರ-ರೆಂದು
ಆರಾತ್ರಿಕ
ಆರಾ-ಧನೆ
ಆರಾ-ಧನೆ-ಗಾಗಿ
ಆರಾಧ-ನೆಯ
ಆರಾಧ-ನೆಯು
ಆರಾಧ-ನೆಯೇ
ಆರಾಧಿಸ-ಬಲ್ಲರು
ಆರಾಧಿ-ಸಲು
ಆರಾಧಿಸಿ
ಆರಾಧಿಸು
ಆರಾಧಿಸುತ್ತಿತ್ತು
ಆರಾಧಿಸುತ್ತೇನೆ
ಆರಾಧಿ-ಸುವ
ಆರಾಧಿಸು-ವರು
ಆರಾಧಿಸು-ವು-ದನ್ನು
ಆರಾಧಿ-ಸೋಣ
ಆರಾಮ-ಗೃಹ-ದಲ್ಲಿ
ಆರಾಮ-ಗೃಹ-ವನ್ನು
ಆರಿಗೆ
ಆರಿಸಿ
ಆರಿಸಿ-ಕೊಳ್ಳ-ಬೇಕಾಗಿದೆ
ಆರಿಸಿ-ಕೊಳ್ಳಿ
ಆರಿಸಿ-ಕೊಳ್ಳುವನು
ಆರಿಸಿ-ಕೊಳ್ಳುವುದ-ರಲ್ಲಿ
ಆರಿಸಿದ
ಆರಿ-ಹೋಗು-ವುದ-ರಲ್ಲಿದೆ
ಆರು
ಆರೂಢ-ಳಾಗಿದ್ದಂತಿತ್ತು
ಆರೇಳು
ಆರೋಗ್ಯ
ಆರೋಗ್ಯದ
ಆರೋಗ್ಯ-ಭಾಗ್ಯ-ವೆಂದು
ಆರೋಗ್ಯ-ವಂತ-ನಾದ
ಆರೋಗ್ಯ-ವಂತರೂ
ಆರೋಗ್ಯ-ವಂತ-ವಾದ
ಆರೋಗ್ಯ-ವನ್ನು
ಆರೋಗ್ಯ-ವಾಗಿಲ್ಲ-ವೆಂದು
ಆರೋಗ್ಯವೂ
ಆರೋಗ್ಯ-ವೆಂದು
ಆರೋಪ
ಆರೋ-ಪಣೆ
ಆರೋಪ-ವಾಗಿವೆ
ಆರೋಪ-ವಿದೆ
ಆರೋಪಿ-ತವೆಂದಲ್ಲದೆ
ಆರೋಪಿಸಿ
ಆರೋಹಣ
ಆರ್
ಆರ್ಜಿಸಿ-ರುವ
ಆರ್ನಾಲ್ಡ್
ಆರ್ಭಟಿಸ-ದಿದೆ
ಆರ್ಯ
ಆರ್ಯ-ಜನಾಂಗ
ಆರ್ಯ-ಜನಾಂಗ-ದ-ವನು
ಆರ್ಯ-ಜನಾಂಗ-ದೊಳಗೇ
ಆರ್ಯ-ನಲ್ಲ-ದವ-ನಿಗಾಗಲೀ
ಆರ್ಯ-ನಿಂದ
ಆರ್ಯ-ನಿಗಾಗಲೀ
ಆರ್ಯರ
ಆರ್ಯ-ರನ್ನು
ಆರ್ಯ-ರಲ್ಲದ
ಆರ್ಯ-ರಲ್ಲ-ದ-ವರ
ಆರ್ಯ-ರಲ್ಲ-ದ-ವರು
ಆರ್ಯ-ರಿಗಿ-ರು-ವಂತೆಯೇ
ಆರ್ಯ-ರಿಗೆ
ಆರ್ಯರು
ಆರ್ಯರೇ
ಆರ್ಯ-ರೊಡನೆ
ಆರ್ಯೇತ-ರರ
ಆರ್ಷ
ಆರ್ಷೇಯ
ಆಲಂಬ-ಜಾರಿನ
ಆಲಂಬ-ಜಾರಿ-ನ-ವ-ರೆಗೂ
ಆಲಂಬ-ಜಾರಿ-ನಿಂದ
ಆಲಂಬ-ಜಾರ್
ಆಲಂಬ-ಜಾರ್
ಆಲದ
ಆಲಯ
ಆಲ-ಸರೂ
ಆಲಸ್ಯ
ಆಲಾಪ
ಆಲಾಪ-ನವೆ
ಆಲಾಪ-ನೆಯೂ
ಆಲಾಯ
ಆಲಿಂಗನ
ಆಲಿಂಗ-ನವು
ಆಲಿಂಗಿಸು
ಆಲಿ-ಪುರದ
ಆಲಿ-ಸಿ-ದನು
ಆಲಿ-ಸು-ವುದು
ಆಲೋಕ
ಆಲೋಕರೇ
ಆಲೋಚನಾ
ಆಲೋಚನಾ-ಪರ-ವಾದ
ಆಲೋಚನಾ-ಮಗ್ನ-ರಾಗಿದ್ದು-ದನ್ನು
ಆಲೋಚನಾ-ಮಾರ್ಗಕ್ಕೆ
ಆಲೋಚನೆ
ಆಲೋಚನೆ-ಗಳನ್ನು
ಆಲೋಚನೆ-ಗಳಲ್ಲಿ
ಆಲೋಚನೆ-ಗಳಿಂದ
ಆಲೋಚನೆ-ಗಳು
ಆಲೋಚನೆಗೂ
ಆಲೋಚ-ನೆಯ
ಆಲೋಚನೆ-ಯನ್ನೇ
ಆಲೋಚನೆ-ಯಲ್ಲಿ
ಆಲೋಚನೆ-ಯಲ್ಲೇ
ಆಲೋಚ-ನೆಯೇ
ಆಲೋಚಿಸ-ಬ-ಹುದು
ಆಲೋಚಿಸ-ಲಾರೆವು
ಆಲೋಚಿ-ಸುತ್ತಲೇ
ಆಲೋಚಿ-ಸುತ್ತಿದ್ದೆ
ಆಲೋಚಿ-ಸುತ್ತಿ-ರುವರೋ
ಆಲೋಚಿ-ಸುವರು
ಆಲೋಚಿ-ಸು-ವುದಕ್ಕೆ
ಆಲೋಚಿ-ಸು-ವು-ದಿಲ್ಲ
ಆಲೋಚಿ-ಸು-ವುದು
ಆಲ್ಬರ್ಟಾಗೆ
ಆಲ್ಮೋರ-ದಲ್ಲಿದ್ದಾಗ
ಆಲ್ಮೋರಾಕ್ಕೆ
ಆಲ್ಮೋರಾ-ದಲ್ಲಿದ್ದಾಗ
ಆಳಕ್ಕೆ
ಆಳ-ದಾ-ಳದ
ಆಳದಿಂ
ಆಳ-ದೊಂದಿಗೆ
ಆಳ-ಬೇ-ಕಾದರೆ
ಆಳ-ವನು
ಆಳ-ವನ್ನು
ಆಳ-ವನ್ನೂ
ಆಳ-ವಾದ
ಆಳು-ಕಾಳು-ಗಳೊ-ಡನೆ
ಆಳುತ್ತಿರುವ
ಆಳುತ್ತಿರು-ವುವು
ಆಳು-ವುದಕ್ಕೆ
ಆಳ್ವಿಕೆ
ಆಳ್ವಿ-ಕೆಗೆ
ಆವ
ಆವ-ರಣ-ದಲ್ಲಿ
ಆವ-ರಣ-ದಿಂದ
ಆವ-ರಣ-ವನ್ನು
ಆವ-ರಿಸಿ-ಕೊಂಡಿದೆ
ಆವರಿ-ಸಿ-ಕೊಂಡಿರು-ವು-ದ-ರಿಂದ
ಆವರಿ-ಸಿತ್ತು
ಆವರಿ-ಸಿ-ದಂತೆ
ಆವ-ರಿಸಿ-ದಾಗ
ಆವರಿ-ಸಿದ್ದರೆ
ಆವ-ರಿಸಿ-ಬಿಟ್ಟಂತೆ
ಆವ-ರಿಸಿ-ರುವ
ಆವ-ರಿಸಿವೆ
ಆವರ್ತ
ಆವರ್ತನ
ಆವಶ್ಯ
ಆವಶ್ಯಕ
ಆವಶ್ಯ-ಕತೆ
ಆವಶ್ಯ-ಕ-ತೆಗೆ
ಆವಶ್ಯ-ಕ-ತೆಯ
ಆವಶ್ಯ-ಕ-ತೆ-ಯನ್ನು
ಆವಶ್ಯ-ಕ-ತೆ-ಯನ್ನೂ
ಆವಶ್ಯ-ಕ-ತೆ-ಯಿಂದ
ಆವಶ್ಯ-ಕ-ತೆ-ಯಿದೆ
ಆವಶ್ಯ-ಕ-ತೆ-ಯಿ-ದೆಯೇ
ಆವಶ್ಯ-ಕ-ತೆ-ಯಿದ್ದರೆ
ಆವಶ್ಯ-ಕ-ತೆ-ಯಿ-ರು-ವು-ದಿಲ್ಲ
ಆವಶ್ಯ-ಕ-ತೆ-ಯಿಲ್ಲ
ಆವಶ್ಯ-ಕ-ತೆಯು
ಆವಶ್ಯ-ಕ-ತೆ-ಯುಂಟಾಗುವುದು
ಆವಶ್ಯ-ಕ-ತೆ-ಯುಂಟಾ-ಯಿತು
ಆವಶ್ಯ-ಕ-ತೆ-ಯುಳ್ಳ-ವರು
ಆವಶ್ಯ-ಕ-ತೆಯೂ
ಆವಶ್ಯ-ಕ-ತೆಯೇ
ಆವಶ್ಯ-ಕ-ತೆ-ಯೇ-ನಿತ್ತು
ಆವಶ್ಯ-ಕ-ತೆ-ಯೇ-ನಿದೆ
ಆವಶ್ಯ-ಕ-ತೆ-ಯೇ-ನಿಲ್ಲ
ಆವಶ್ಯ-ಕ-ವಾಗಿ
ಆವಶ್ಯ-ಕ-ವಾಗಿ-ರುವ
ಆವಶ್ಯ-ಕ-ವಾದ
ಆವಶ್ಯ-ಕವೂ
ಆವಶ್ಯ-ಕವೆ
ಆವಶ್ಯ-ಕ-ವೆಂದು
ಆವಾಗ
ಆವಾಗಲೂ
ಆವಾಸ
ಆವಾಹನೆ
ಆವಿರ್ಭವಿಸಿ-ದಾಗ
ಆವಿರ್ಭಾವ
ಆವಿರ್ಭಾ-ವಕ್ಕೆ
ಆವಿರ್ಭಾವ-ಗಳು
ಆವಿರ್ಭಾವ-ದಲ್ಲಿದೆ
ಆವಿರ್ಭಾವ-ನೆಗೆ
ಆವಿರ್ಭಾವ-ನೆ-ಯಾದ
ಆವಿರ್ಭಾವ-ವನ್ನು
ಆವಿರ್ಭಾವ-ವಾಗಿ-ರುವ
ಆವಿರ್ಭಾವ-ವಾಗು-ವುದು
ಆವಿರ್ಭಾವ-ವಾ-ಯಿತು
ಆವಿಷ್ಕ-ರಿಸಿ
ಆವಿಷ್ಕರಿ-ಸುತ್ತಲೇ
ಆವೃತ-ನಾದ
ಆವೃತ-ರಾಗಿದ್ದಾರೆ
ಆವೃತ-ವಾದ
ಆವೃತ್ತ-ವಾಗಿ-ದೆಯೋ
ಆವೇಶ
ಆವೇಶ-ದಿಂದ
ಆವೇಶ-ಪೂರಿತ
ಆವೊತ್ತು
ಆಶ
ಆಶಾ
ಆಶಾ-ದಾಯ-ಕ-ವಾಗಿದೆ
ಆಶಾ-ವಾ-ಸನೆ-ಗಳ
ಆಶಿ-ಸುತ್ತೇನೆ
ಆಶಿ-ಸುವ
ಆಶಿ-ಸುವನು
ಆಶಿ-ಸುವುದನ್ನೆಲ್ಲ
ಆಶಿ-ಸು-ವುದು
ಆಶಿ-ಸು-ವೆನು
ಆಶಿಸು-ವೆವೋ
ಆಶೀರ್ವದಿಸಿ
ಆಶೀರ್ವದಿಸಿ-ದರು
ಆಶೀರ್ವದಿಸಿ-ದರೆ
ಆಶೀರ್ವದಿ-ಸುವರು
ಆಶೀರ್ವಾದ
ಆಶೀರ್ವಾದಕ್ಕಾಗಿ
ಆಶೀರ್ವಾದ-ದಿಂದ
ಆಶೀರ್ವಾದ-ಮಾಡಿ
ಆಶೀರ್ವಾದ-ವನ್ನು
ಆಶೆ
ಆಶೆಗೆ
ಆಶೆ-ಯ-ನಾ-ವನು
ಆಶೆ-ಯನ್ನಿಟ್ಟು-ಕೊಂಡು
ಆಶೆ-ಯನ್ನು
ಆಶೆ-ಯಾ-ಗಿದೆ
ಆಶೆ-ಯಾ-ಯಿತು
ಆಶೆ-ಯಿಂದ
ಆಶೆ-ಯಿತ್ತು
ಆಶೆ-ಯಿದೆ
ಆಶೆ-ಯಿಲ್ಲದೆ
ಆಶೆಯು
ಆಶೆ-ರಾಶಿ-ಯನು
ಆಶೇ
ಆಶ್ಚರ್ಯ-ಕರ-ವಾದ
ಆಶ್ಚರ್ಯ-ದಿಂದ
ಆಶ್ಚರ್ಯ-ಪಟ್ಟು
ಆಶ್ಚರ್ಯ-ಯುಕ್ತ-ರಾಗಿ
ಆಶ್ಚರ್ಯ-ವಲ್ಲವೇ
ಆಶ್ಚರ್ಯ-ವಾಗ-ಬ-ಹುದು
ಆಶ್ಚರ್ಯ-ವಾಗಿ
ಆಶ್ಚರ್ಯ-ವಾಗಿದೆ
ಆಶ್ಚರ್ಯ-ವಾ-ಗುತ್ತಿದೆ
ಆಶ್ಚರ್ಯ-ವಾಗು-ವುದು
ಆಶ್ಚರ್ಯ-ವಾ-ಯಿತು
ಆಶ್ಚರ್ಯ-ವೆಂದರೆ
ಆಶ್ಚರ್ಯ-ವೆನಿ-ಸುತ್ತದೆ
ಆಶ್ಚರ್ಯ-ವೇ-ನಾಯಿ-ತೆಂದರೆ
ಆಶ್ಚರ್ಯ-ವೇನೂ
ಆಶ್ಚರ್ಯ-ವೇ-ನೆಂದರೆ
ಆಶ್ರಮ
ಆಶ್ರಮಕ್ಕೆ
ಆಶ್ರಮ-ಗಳ
ಆಶ್ರಮ-ಗಳನ್ನು
ಆಶ್ರಮ-ಗಳಲ್ಲಿ
ಆಶ್ರಮ-ಗಳಿಗೆ
ಆಶ್ರಮ-ಗಳು
ಆಶ್ರಮ-ಗಳೊ-ಡನೆ
ಆಶ್ರಮದ
ಆಶ್ರಮ-ದಲ್ಲೂ
ಆಶ್ರಮ-ವನ್ನು
ಆಶ್ರಮವೂ
ಆಶ್ರಯ
ಆಶ್ರಯ-ದಲ್ಲಿದ್ದು
ಆಶ್ರಯ-ದಾತ
ಆಶ್ರಯ-ವನ್ನಾಗಿ
ಆಶ್ರಯ-ವನ್ನು
ಆಶ್ರಯ-ವಿರುತ್ತಿತ್ತು
ಆಶ್ರಯ-ವಿಲ್ಲದೆ
ಆಶ್ರ-ಯವೂ
ಆಶ್ರಯ-ವೆಲ್ಲಿ
ಆಶ್ರಯಿ-ಸದೆ
ಆಶ್ರಯಿಸಿ
ಆಶ್ರಯಿಸಿ-ಕೊಂಡಿದ್ದರೆ
ಆಶ್ರಯಿಸಿ-ಕೊಂಡು
ಆಶ್ರಯಿ-ಸು-ವು-ದ-ರಿಂದ
ಆಷಾಢದ
ಆಷಾಢ-ಭೂತಿ-ಗಳನ್ನು
ಆಷಾಢ-ಭೂತಿ-ಗಳಾಗ-ಬೇಡಿ
ಆಸಂಸಿದ್ಧೇಃ
ಆಸಕ್ತ-ರಾಗಿದ್ದಾರೆಂದು
ಆಸಕ್ತ-ರಾದರು
ಆಸಕ್ತರೋ
ಆಸಕ್ತಿ
ಆಸಕ್ತಿ-ಕರ-ವಾಗಿವೆ
ಆಸಕ್ತಿ-ಗಳನ್ನು
ಆಸಕ್ತಿ-ಯಂತೆ
ಆಸಕ್ತಿ-ಯನ್ನು
ಆಸಕ್ತಿ-ಯನ್ನೂ
ಆಸಕ್ತಿ-ಯಲ್ಲ
ಆಸಕ್ತಿ-ಯಿಂದ
ಆಸಕ್ತಿ-ಯಿಲ್ಲದೆ
ಆಸಕ್ತಿಯು
ಆಸನ
ಆಸ-ನದ
ಆಸನ-ದಲ್ಲಿ
ಆಸನ-ವನ್ನು
ಆಸನಾರೂಢ-ರಾಗಿದ್ದರು
ಆಸನಾರೂಢ-ರಾದ
ಆಸಮುದ್ರ
ಆಸರೆ-ಯಾದ
ಆಸೂರ್ಯ
ಆಸೆ
ಆಸೆ-ಗಳ
ಆಸೆ-ಗಳನ್ನು
ಆಸೆ-ಗಳು-ದಿ-ಸು-ವುವು
ಆಸೆಗೂ
ಆಸೆಗೆ
ಆಸೆ-ಪಡುತ್ತಿಲ್ಲ
ಆಸೆಯ
ಆಸೆ-ಯ-ಗಲಿತು
ಆಸೆ-ಯನ್ನು
ಆಸೆ-ಯಾ-ಗಿದೆ
ಆಸೆ-ಯಾಗುವುದು
ಆಸೆ-ಯಾ-ಯಿತು
ಆಸೆ-ಯಿಂದ
ಆಸೆ-ಯಿಂದಲೇ
ಆಸೆ-ಯಿದೆ
ಆಸೆ-ಯೆಂದರೆ
ಆಸೆಯೇ
ಆಸೋ-ಯಾರ
ಆಸ್ತಿ
ಆಸ್ತಿಕ್ಯಂತ್ವಿದಂತು
ಆಸ್ತಿ-ಯನ್ನು
ಆಸ್ತಿ-ಯನ್ನೂ
ಆಸ್ಪತ್ರೆ
ಆಸ್ಪತ್ರೆ-ಗಳನ್ನು
ಆಸ್ಪತ್ರೆ-ಗಳಲ್ಲಿ
ಆಸ್ಪತ್ರೆ-ಗಳಿಗೆ
ಆಸ್ಪತ್ರೆ-ಗಳು
ಆಸ್ಪತ್ರೆಗೆ
ಆಸ್ವಾಲನ
ಆಹಾ
ಆಹಾರ
ಆಹಾ-ರಕ್ಕಾಗಿ
ಆಹಾ-ರಕ್ಕಾಗಿ-ಯಾದರೂ
ಆಹಾ-ರಕ್ಕೂ
ಆಹಾ-ರಕ್ಕೆ
ಆಹಾ-ರದ
ಆಹಾ-ರ-ದಲ್ಲಿ
ಆಹಾ-ರ-ದಿಂದ
ಆಹಾ-ರ-ದಿಂದುಂಟಾಗುವ
ಆಹಾ-ರ-ವನ್ನಾದರೂ
ಆಹಾ-ರ-ವನ್ನು
ಆಹಾ-ರ-ವಾಗಿತ್ತು
ಆಹಾ-ರ-ವಿ-ರ-ಲಿಲ್ಲ
ಆಹಾ-ರ-ವಿಲ್ಲದೆ
ಆಹಾ-ರವು
ಆಹಾ-ರವೂ
ಆಹಾ-ರ-ವೆಂದೇ
ಆಹಾ-ರ-ವೆಂಬ
ಆಹಾ-ರ-ಸೇ-ವನೆ
ಆಹುತಿ
ಆಹ್ವಾನ
ಆಹ್ವಾನ-ಮಾಡಿ
ಆಹ್ವಾನ-ವಿದೆ
ಆಹ್ವಾನಿ-ಸಲ್ಪಟ್ಟಿದ್ದರು
ಆಹ್ವಾನಿ-ಸಿದಾಗಲೇ
ಇಂಗ-ಬೇಕು
ಇಂಗಿತ-ವೇನೋ
ಇಂಗು-ವು-ದೇನು
ಇಂಗ್ಲಿಷನ್ನೋದಿ
ಇಂಗ್ಲಿಷರ
ಇಂಗ್ಲಿಷ-ರನ್ನು
ಇಂಗ್ಲಿಷಿಗೆ
ಇಂಗ್ಲಿಷಿ-ನಲ್ಲಿ
ಇಂಗ್ಲಿಷಿ-ನಲ್ಲಿ-ರುವ
ಇಂಗ್ಲಿಷಿ-ನವ-ರಂಥ
ಇಂಗ್ಲಿಷ್
ಇಂಗ್ಲೀಷರು
ಇಂಗ್ಲೀಷಿ-ನಲ್ಲಿ
ಇಂಗ್ಲೀಷ್
ಇಂಗ್ಲೆಂಡಿಗೆ
ಇಂಗ್ಲೆಂಡಿನ
ಇಂಗ್ಲೆಂಡಿನಲ್ಲಿಯೇ
ಇಂಗ್ಲೆಂಡಿ-ನಿಂದ
ಇಂಗ್ಲೆಂಡ್
ಇಂಗ್ಲೇಂಡಿ-ನಿಂದ
ಇಂಚರ-ದಲಿ
ಇಂಚಿಗೆ
ಇಂಜಿನ್
ಇಂಡಿಯನ್
ಇಂಡಿಯನ್-ರೊಡನೆ
ಇಂಡಿಯಾ
ಇಂಡಿಯಾಕ್ಕೆ
ಇಂಡಿಯಾ-ದಲ್ಲಿ
ಇಂಡಿಯಾ-ದಲ್ಲಿ-ರುವ
ಇಂಡಿಯಾ-ದಿಂದ
ಇಂಡಿಯಾ-ದೇಶ
ಇಂಡಿಯಾ-ದೇಶದ
ಇಂಡಿಯಾ-ದೇಶ-ದಲ್ಲಿ
ಇಂತಹ
ಇಂತಹ-ವ-ರಿಂದ
ಇಂತ-ಹುದು
ಇಂತಾ-ದರೂ
ಇಂತಿ-ರಲು
ಇಂತಿಷ್ಟೇ
ಇಂತು
ಇಂಥ
ಇಂಥ-ವನು
ಇಂಥ-ವ-ನೊಬ್ಬನು
ಇಂಥ-ವರ
ಇಂಥ-ವ-ರಿಂದ
ಇಂಥ-ವರು
ಇಂಥಾದ್ದನ್ನು
ಇಂಥಿಂಥ
ಇಂದಿಗೂ
ಇಂದಿಗೆ
ಇಂದಿನ
ಇಂದಿನ-ವ-ರೆಗೂ
ಇಂದಿನಾ-ಕಾರ-ಗಳು
ಇಂದಿ-ನಿಂದ
ಇಂದಿಲ್ಲಿ
ಇಂದು
ಇಂದೂ
ಇಂದೇ
ಇಂದ್ರ
ಇಂದ್ರ-ಜಾಲ
ಇಂದ್ರ-ಜಾಲದ
ಇಂದ್ರ-ಜಾಲ-ದಂತೆ
ಇಂದ್ರ-ಜಾಲ-ವನ್ನು
ಇಂದ್ರ-ಜಾಲ-ವೆಂದು
ಇಂದ್ರ-ಜಿತು
ಇಂದ್ರಿಯ
ಇಂದ್ರಿಯ-ಕೆಟು-ಕದೆ
ಇಂದ್ರಿ-ಯಕೊ
ಇಂದ್ರಿಯ-ಗಳ
ಇಂದ್ರಿಯ-ಗಳನ್ನು
ಇಂದ್ರಿಯ-ಗಳನ್ನೇ
ಇಂದ್ರಿಯ-ಗಳಿ-ರುವ
ಇಂದ್ರಿಯ-ಗಳಿವೆ
ಇಂದ್ರಿಯ-ಗಳೂ
ಇಂದ್ರಿಯಗ್ರಹಣ
ಇಂದ್ರಿಯಗ್ರಾಹ್ಯ
ಇಂದ್ರಿಯ-ಜಿ-ತರೊ
ಇಂದ್ರಿ-ಯದ
ಇಂದ್ರಿಯ-ದಲ್ಲಿ
ಇಂದ್ರಿಯ-ನಿಗ್ರಹ-ವನ್ನು
ಇಂದ್ರಿಯ-ರಾಗ-ವಿ-ದೂರ
ಇಂದ್ರಿಯ-ಸುಖ-ದಾಸೆ-ಯನ್ನು
ಇಂದ್ರಿಯಾಸಕ್ತ-ವಾದವು
ಇಂದ್ರಿಯಾಸಕ್ತಿ
ಇಂದ್ರೀಯ
ಇಂಪಾಗಿ
ಇಂಪಾ-ಗಿಯೂ
ಇಂಪಾದ
ಇಂಪು-ದನಿ
ಇಕ್ಕಟ್ಟಾದ
ಇಕ್ಕೆಡೆ-ಯಲ್ಲೂ
ಇಗೋ
ಇಚ್ಚಾ
ಇಚ್ಚಾ-ಶಕ್ತಿ
ಇಚ್ಚಿ-ಸು-ವು-ದಿಲ್ಲ
ಇಚ್ಚೆ-ಯಿಂದಲೆ
ಇಚ್ಛಾ
ಇಚ್ಛಾ-ಧೀನ
ಇಚ್ಛಾ-ನು-ಸಾರ-ವಾಗಿ
ಇಚ್ಛಾ-ನು-ಸಾರ-ವಾಗಿಯೆ
ಇಚ್ಛಾ-ಪಾಶ-ಗಳಿಂದ
ಇಚ್ಛಾ-ಪಾಶೈರ್ನಿಯಮಿತಾ
ಇಚ್ಛಾ-ಮತಿ-ಮಾನ್
ಇಚ್ಛಾ-ಶಕಿ
ಇಚ್ಛಾ-ಶಕ್ತಿ-ಯನ್ನು
ಇಚ್ಛಾ-ಶಕ್ತಿ-ಯಿದೆ
ಇಚ್ಛಿಸಿ
ಇಚ್ಛಿಸಿ-ದರೆ
ಇಚ್ಛಿಸಿ-ದ-ವ-ರಲ್ಲ
ಇಚ್ಛಿಸಿ-ದಾಗ
ಇಚ್ಛಿ-ಸುತ್ತಾ-ರಲ್ಲವೇ
ಇಚ್ಛಿ-ಸುವ
ಇಚ್ಛಿ-ಸುವರು
ಇಚ್ಛಿ-ಸುವರೋ
ಇಚ್ಛಿ-ಸು-ವು-ದಾದರೆ
ಇಚ್ಛಿ-ಸು-ವುದು
ಇಚ್ಛಿ-ಸುವೆಯೋ
ಇಚ್ಛಿ-ಸುವೆ-ವೆಂದೂ
ಇಚ್ಛೆ
ಇಚ್ಛೆಗೆ
ಇಚ್ಛೆ-ಪಟ್ಟಲ್ಲಿ
ಇಚ್ಛೆ-ಪಟ್ಟು
ಇಚ್ಛೆ-ಪಟ್ಟೆ
ಇಚ್ಛೆ-ಪಡು-ವು-ದ-ರಿಂದ
ಇಚ್ಛೆ-ಪಡು-ವು-ದಿಲ್ಲ
ಇಚ್ಛೆ-ಬಂದಂತೆ
ಇಚ್ಛೆಯ
ಇಚ್ಛೆ-ಯಂತೆ
ಇಚ್ಛೆ-ಯಂತೆಯೇ
ಇಚ್ಛೆ-ಯನ್ನು
ಇಚ್ಛೆ-ಯಿಂದ
ಇಚ್ಛೆ-ಯಿಲ್ಲ
ಇಚ್ಛೆ-ಯಿಲ್ಲದೆ
ಇಚ್ಛೆ-ಯುಂಟಾ-ಗುತ್ತದೆ
ಇಚ್ಛೆ-ಯುಂಟಾ-ಯಿತು
ಇಚ್ಛೆ-ಯುಂಟಾ-ಯಿತೊ
ಇಚ್ಛೆಯೇ
ಇಜಿಪ್ಟಾಲಿಸ್ಟನು
ಇಟಲಿ-ಯಲ್ಲಿ
ಇಟಲಿ-ಯವರ
ಇಟಾಲಿಯನ್
ಇಟುಕೊ
ಇಟ್ಟ
ಇಟ್ಟಂತೆ
ಇಟ್ಟರು
ಇಟ್ಟರೆ
ಇಟ್ಟಾಗ
ಇಟ್ಟಿದೆ
ಇಟ್ಟಿದ್ದ
ಇಟ್ಟಿದ್ದರು
ಇಟ್ಟಿದ್ದಾ-ಯಿತು
ಇಟ್ಟಿದ್ದೇವೆ
ಇಟ್ಟಿರ-ಬೇಕು
ಇಟ್ಟಿರ-ಬೇಕೆನ್ನು-ವುದಕ್ಕೆ
ಇಟ್ಟಿ-ರುವ
ಇಟ್ಟು
ಇಟ್ಟು-ಕೊಂಡಿದ್ದಾರೆ
ಇಟ್ಟು-ಕೊಂಡಿದ್ದು-ದನ್ನು
ಇಟ್ಟು-ಕೊಂಡಿ-ರದೆ
ಇಟ್ಟು-ಕೊಂಡಿರು
ಇಟ್ಟು-ಕೊಂಡು
ಇಟ್ಟು-ಕೊಂಡೆ
ಇಟ್ಟು-ಕೊಳ್ಳದೆ
ಇಟ್ಟು-ಕೊಳ್ಳ-ಬಾ-ರದು
ಇಟ್ಟು-ಕೊಳ್ಳ-ಬೇಡ
ಇಟ್ಟು-ಕೊಳ್ಳ-ಲಾರಿರಿ
ಇಟ್ಟು-ಕೊಳ್ಳಿ
ಇಟ್ಟು-ಕೊಳ್ಳುತ್ತಾರೆ
ಇಟ್ಟು-ಕೊಳ್ಳುತ್ತಿದ್ದರು
ಇಟ್ಟು-ಕೊಳ್ಳುವ
ಇಟ್ಟು-ಕೊಳ್ಳು-ವುದಕ್ಕೆ
ಇಟ್ಟು-ಕೊಳ್ಳು-ವು-ದಿಲ್ಲ
ಇಟ್ಟು-ಕೊಳ್ಳೋಣ
ಇಟ್ಟುಕೋ
ಇಟ್ಟು-ನೋಡ-ಬಹು-ದಾಗಿದೆ
ಇಟ್ಟು-ಬಂದು
ಇಟ್ಟು-ಹೋಗಿದ್ದಾರೆ
ಇಟ್ಟೇ
ಇಡ-ದಂತಾಗು-ವುದು
ಇಡದೆ
ಇಡ-ಬೇ-ಕಾದ
ಇಡ-ಬೇಕು
ಇಡಲು
ಇಡಾ
ಇಡಿ
ಇಡಿಯ
ಇಡೀ
ಇಡೀಯ
ಇಡುತ್ತಿದ್ದರು
ಇಡುವ
ಇಡು-ವು-ದರ
ಇಡು-ವು-ದಿಲ್ಲ
ಇಡು-ವುವು
ಇಣುಕಿತು
ಇಣುಕು
ಇತರ
ಇತ-ರರ
ಇತರ-ರನ್ನು
ಇತರ-ರಿ-ಗಾಗಿ
ಇತರ-ರಿಗಾಗುತ್ತಿರುವ
ಇತರ-ರಿಗೂ
ಇತರ-ರಿಗೆ
ಇತ-ರರು
ಇತ-ರರೂ
ಇತರ-ರೆಲ್ಲ-ರಿ-ಗಿಂತ
ಇತರ-ರೆಲ್ಲ-ರಿಗೂ
ಇತರ-ರೊಡನೆ
ಇತರೆ
ಇತರೇ
ಇತಿ
ಇತಿ-ಹಾಸ
ಇತಿ-ಹಾಸ-ಕಾರ-ರೆಲ್ಲರೂ
ಇತಿ-ಹಾಸ-ದಲ್ಲಿ
ಇತಿ-ಹಾಸ-ದುದ್ದಕ್ಕೂ
ಇತಿ-ಹಾಸ-ವನ್ನು
ಇತಿ-ಹಾಸ-ವನ್ನೆಲ್ಲಾ
ಇತಿ-ಹಾಸ-ವಿರ-ಬೇಕೊ
ಇತಿ-ಹಾಸ-ವಿಲ್ಲ-ವಲ್ಲ
ಇತಿ-ಹಾಸ-ವಿಲ್ಲವೋ
ಇತಿ-ಹಾಸವು
ಇತಿ-ಹಾಸವೇ
ಇತ್ತ
ಇತ್ತ-ಕ-ಡೆಗೆ
ಇತ್ತೀಚೆಗೆ
ಇತ್ತು
ಇತ್ತೆಂದರೆ
ಇತ್ತೊ
ಇತ್ಯರ್ಥ
ಇತ್ಯಾದಿ-ಗಳನ್ನು
ಇತ್ಯಾದಿ-ಯಾಗಿ
ಇದಂತೂ
ಇದಕ್ಕಾಗಿ
ಇದಕ್ಕಿಂತ
ಇದಕ್ಕಿಂಥ
ಇದಕ್ಕೂ
ಇದಕ್ಕೆ
ಇದಕ್ಕೇ-ನಂತೆ
ಇದಕ್ಕೊಂದು
ಇದಕ್ಕೋಸ್ಕರ
ಇದಕ್ಕೋಸ್ಕರವೆ
ಇದಕ್ಕೋಸ್ಕರವೇ
ಇದ-ನರಿ-ಯ-ಲಾರೆ
ಇದ-ನೆಂದು
ಇದನ್ನಿಟ್ಟು-ಕೊಂಡೇ
ಇದನ್ನು
ಇದನ್ನೆಲ್ಲ
ಇದನ್ನೆಲ್ಲಾ
ಇದನ್ನೇ
ಇದರ
ಇದರಂತಹ
ಇದ-ರಂತೆಯೇ
ಇದ-ರಲ್ಲಿ
ಇದ-ರಲ್ಲಿದೆ
ಇದ-ರಲ್ಲಿ-ರು-ವಂತೆ
ಇದ-ರಿಂದ
ಇದ-ರಿಂದಲೆ
ಇದ-ರಿಂದಲೇ
ಇದ-ರಿಂದಾಗಿ
ಇದ-ರಿಂದಾಗುವ
ಇದ-ರೊಡನೆ
ಇದಲ್ಲದೆ
ಇದಾಗಿ
ಇದಾದ
ಇದಾದ-ಮೇಲೆ
ಇದಾ-ವುದೂ
ಇದಿಲ್ಲ
ಇದಿಲ್ಲ-ದಿದ್ದರೆ
ಇದಿಲ್ಲದೆ
ಇದಿಷ್ಟನ್ನು
ಇದಿಷ್ಟು
ಇದು
ಇದು-ಮಾತ್ರ
ಇದು-ವ-ರೆಗೂ
ಇದು-ವರೆಗೆ
ಇದು-ವರೆ-ವಿಗೂ
ಇದುವೆ
ಇದೂ
ಇದೆ
ಇದೆಂತಹ
ಇದೆ-ಯಲ್ಲ
ಇದೆ-ಯಷ್ಟೆ
ಇದೆ-ಯೆಂದು
ಇದೆ-ಯೇನು
ಇದೆಯೊ
ಇದೆಯೋ
ಇದೆಲ್ಲ
ಇದೆಲ್ಲ-ವನ್ನೂ
ಇದೆಲ್ಲಾ
ಇದೇ
ಇದೇನು
ಇದೇನೋ
ಇದೊ
ಇದೊಂದನ್ನು
ಇದೊಂದರ
ಇದೊಂದು
ಇದೊಂದೂ
ಇದೊಂದೇ
ಇದೋ
ಇದ್ದ
ಇದ್ದಂತೆ
ಇದ್ದಂತೆಯೇ
ಇದ್ದ-ಕಿದ್ದೊಲೆ
ಇದ್ದಕ್ಕಿದ್ದ
ಇದ್ದಕ್ಕಿದ್ದಂತೆ
ಇದ್ದಕ್ಕಿದ್ದಂತೆಯೇ
ಇದ್ದಕ್ಕಿದ್ದೊಲೆ
ಇದ್ದದ್ದನ್ನು
ಇದ್ದದ್ದ-ರಿಂದ
ಇದ್ದದ್ದ-ರಿಂದಲೆ
ಇದ್ದದ್ದು
ಇದ್ದದ್ದೇನು
ಇದ್ದ-ನೆಂಬುದಂತೂ
ಇದ್ದರು
ಇದ್ದರೂ
ಇದ್ದರೆ
ಇದ್ದ-ರೆಂದು
ಇದ್ದ-ರೆ-ದೀನ
ಇದ್ದ-ರೆಷ್ಟು
ಇದ್ದರೇ
ಇದ್ದಲ್ಲಿ
ಇದ್ದವು
ಇದ್ದಾಗ
ಇದ್ದಾ-ಗಲೇ
ಇದ್ದಾಗ್ಯೂ
ಇದ್ದಾ-ನಲ್ಲ
ಇದ್ದಾನೆ
ಇದ್ದಾ-ರಲ್ಲಾ
ಇದ್ದಾರೆ
ಇದ್ದಾರೆಂದೂ
ಇದ್ದಾರೆಯೊ
ಇದ್ದಾಳೆ
ಇದ್ದಿತಲ್ಲಿ
ಇದ್ದಿದ್ದರೆ
ಇದ್ದಿರ-ಬೇಕು
ಇದ್ದೀಯೆ
ಇದ್ದೀರಿ
ಇದ್ದು
ಇದ್ದು-ಕೊಂಡಿವೆ
ಇದ್ದು-ಕೊಂಡು
ಇದ್ದು-ದನ್ನು
ಇದ್ದು-ದ-ರಿಂದ
ಇದ್ದುದು
ಇದ್ದು-ಬಿಟ್ಟರೆ
ಇದ್ದುವು
ಇದ್ದು-ವೆಂಬು-ದಕ್ಕೆ
ಇದ್ದೆ
ಇದ್ದೆನೊ
ಇದ್ದೇ
ಇದ್ದೇನೆ
ಇನಿ-ತನೂ
ಇನಿತೆ
ಇನಿತೆಲ್ಲ
ಇನೊಬ್ಬ
ಇನ್ನಾದರೂ
ಇನ್ನಾ-ರನ್ನೂ
ಇನ್ನಾವ
ಇನ್ನಾ-ವುದೋ
ಇನ್ನಿಬ್ಬರು
ಇನ್ನಿಲ್ಲ
ಇನ್ನು
ಇನ್ನು-ಳಿದ
ಇನ್ನೂ
ಇನ್ನೂರ-ರಲ್ಲಿ
ಇನ್ನೂರು
ಇನ್ನೆಲ್ಲ
ಇನ್ನೆಷ್ಟು
ಇನ್ನೇ-ನನ್ನು
ಇನ್ನೇನು
ಇನ್ನೊಂದನ್ನೂ
ಇನ್ನೊಂದ-ರೊಡನೆ
ಇನ್ನೊಂದು
ಇನ್ನೊಂದೆಡೆ
ಇನ್ನೊಬ್ಬನ
ಇನ್ನೊಬ್ಬ-ನಿಗೆ
ಇನ್ನೊಬ್ಬನೂ
ಇನ್ನೊಬ್ಬರ
ಇನ್ನೊಬ್ಬ-ರನ್ನು
ಇನ್ನೊಬ್ಬ-ರಿ-ಗಾಗಿ
ಇನ್ನೊಬ್ಬ-ರಿಗೆ
ಇನ್ನೊಬ್ಬರು
ಇನ್ಯಾರು
ಇಪ್ಪತ್ತ-ನಾಲ್ಕು
ಇಪ್ಪತ್ತು
ಇಪ್ಪತ್ತು-ನಾಲ್ಕು
ಇಪ್ಪತ್ತೇಳು
ಇಪ್ಪತ್ತೊಂದು
ಇಬ್ಬರ
ಇಬ್ಬ-ರಿಗೂ
ಇಬ್ಬರು
ಇಬ್ಬರೂ
ಇಬ್ಬರೇ
ಇಬ್ಬಿಬ್ಬರು
ಇಮಿಟೇಷನ್
ಇಮ್ಮಡಿ-ಯಾ-ದು-ದನ್ನು
ಇರ-ಕೂ-ಡದು
ಇರ-ತಕ್ಕ
ಇರ-ತಕ್ಕದ್ದು
ಇರ-ದಿದ್ದರೆ
ಇರದು
ಇರ-ಬಲ್ಲದು
ಇರ-ಬಲ್ಲ-ನಾದರೆ
ಇರ-ಬಲ್ಲರು
ಇರ-ಬಲ್ಲ-ವ-ನಾದರೆ
ಇರ-ಬಲ್ಲಿರಿ
ಇರ-ಬ-ಹುದು
ಇರ-ಬಹುದೋ
ಇರ-ಬಾ-ರದು
ಇರ-ಬೇ-ಕಲ್ಲವೆ
ಇರ-ಬೇಕಾಗಿಲ್ಲ
ಇರ-ಬೇಕಾಗಿವೆ
ಇರ-ಬೇ-ಕಾದ
ಇರ-ಬೇ-ಕಾದರೆ
ಇರ-ಬೇಕು
ಇರ-ಬೇಕೆಂದರು
ಇರ-ಬೇಕೆಂದು
ಇರ-ಲಾ-ಗು-ವು-ದಿಲ್ಲ-ವೆಂದು
ಇರ-ಲಾ-ರರು
ಇರ-ಲಾ-ರವು
ಇರ-ಲಾರೆ
ಇರ-ಲಾರೆ-ಯೇನು
ಇರಲಿ
ಇರ-ಲಿಲ್ಲ
ಇರಲಿಲ್ಲವೆ
ಇರಲು
ಇರಲೇ
ಇರ-ಲೇನು
ಇರ-ವಿನ
ಇರವಿ-ನಾಳ-ವನು
ಇರವಿ-ನೊಳಗೊಂದಾಗಿ
ಇರವು
ಇರಿ
ಇರಿಸಿ
ಇರಿ-ಸು-ವುದಕ್ಕೋಸ್ಕರ
ಇರು
ಇರು-ತಿ-ರಲು
ಇರು-ತಿ-ರುವೆ
ಇರುತ್ತ
ಇರುತ್ತದೆ
ಇರುತ್ತ-ದೆಯೆ
ಇರುತ್ತ-ದೆಯೊ
ಇರುತ್ತ-ದೆಯೋ
ಇರುತ್ತದೋ
ಇರುತ್ತವೆ
ಇರುತ್ತಾನೆ
ಇರುತ್ತಾರೆ
ಇರುತ್ತಾ-ರೆಂದು
ಇರುತ್ತಾರೆಯೊ
ಇರುತ್ತಾ-ರೇನು
ಇರುತ್ತಾಳೆಯೋ
ಇರುತ್ತಿತ್ತು
ಇರುತ್ತಿದ್ದ
ಇರುತ್ತಿದ್ದರು
ಇರುತ್ತಿದ್ದ-ಳೆಂದು
ಇರುತ್ತಿದ್ದಾರೆ
ಇರುತ್ತಿ-ರ-ಲಿಲ್ಲ
ಇರುತ್ತಿ-ರ-ಲಿಲ್ಲ-ವಂತೆ
ಇರುತ್ತೀರಿ
ಇರುತ್ತೇನೆ
ಇರುತ್ತೇವೆ
ಇರುಳ
ಇರು-ಳಿನ
ಇರುಳೂ
ಇರುವ
ಇರು-ವಂತೆ
ಇರು-ವಂತೆಯೇ
ಇರು-ವಂಥಾ
ಇರು-ವನು
ಇರು-ವರು
ಇರು-ವ-ರುಜ್ಞಾನಿ-ಗಳು
ಇರು-ವರೆ
ಇರು-ವಳೋ
ಇರು-ವ-ವನು
ಇರು-ವ-ವರ
ಇರು-ವ-ವ-ರಾರೂ
ಇರು-ವವರು
ಇರು-ವ-ವ-ರೆಗೂ
ಇರು-ವಾಗ
ಇರು-ವಿ-ಕೆಗೆ
ಇರು-ವಿ-ಕೆಯ
ಇರು-ವಿ-ಕೆಯೆಲ್ಲ
ಇರು-ವುದಕ್ಕಾ-ಗು-ವು-ದಿಲ್ಲ
ಇರು-ವುದಕ್ಕೆ
ಇರು-ವು-ದನ್ನು
ಇರು-ವುದನ್ನೇ
ಇರು-ವು-ದ-ರಿಂದ
ಇರು-ವು-ದ-ರಿಂದಲೆ
ಇರು-ವು-ದ-ರಿಂದಲೇ
ಇರು-ವು-ದಿಲ್ಲ
ಇರು-ವು-ದಿಲ್ಲ-ವೆಂದು
ಇರು-ವು-ದಿಲ್ಲ-ವೆಂಬು-ದನ್ನು
ಇರು-ವು-ದಿಲ್ಲವೋ
ಇರು-ವುದು
ಇರು-ವುದೂ
ಇರು-ವು-ದೆಲ್ಲಿ
ಇರು-ವುದೇ
ಇರು-ವುದೇ-ನಿದ್ದರೂ
ಇರು-ವು-ದೇನು
ಇರು-ವುದೊ
ಇರು-ವು-ದೊಂದೇ
ಇರು-ವುದೋ
ಇರು-ವುವು
ಇರುವೆ
ಇರು-ವೆ-ಗಳಿಗೆ
ಇರು-ವೆ-ಯನ್ನು
ಇರು-ವೆವು
ಇರು-ವೆವೊ
ಇರ್ವರೆ-ದೆಯೊಳು
ಇಲಾಖೆ-ಗಳಲ್ಲಿಯೂ
ಇಲಿ
ಇಲಿ-ಗಳ
ಇಲಿ-ಗಳೆಲ್ಲ
ಇಲ್ಲ
ಇಲ್ಲದ
ಇಲ್ಲ-ದಂತೆ
ಇಲ್ಲ-ದಾಗ
ಇಲ್ಲ-ದಿದ್ದರೂ
ಇಲ್ಲ-ದಿದ್ದರೆ
ಇಲ್ಲ-ದಿದ್ದಲ್ಲಿ
ಇಲ್ಲ-ದಿದ್ದಿದ್ದರೆ
ಇಲ್ಲ-ದಿರ-ಬ-ಹುದು
ಇಲ್ಲ-ದಿ-ರಲಿ
ಇಲ್ಲ-ದಿ-ರುವಾಗ
ಇಲ್ಲ-ದಿ-ರು-ವು-ದ-ರಿಂದ
ಇಲ್ಲ-ದಿ-ರು-ವುದು
ಇಲ್ಲ-ದಿ-ರು-ವುದೂ
ಇಲ್ಲ-ದಿರೆ
ಇಲ್ಲದೆ
ಇಲ್ಲದೇ
ಇಲ್ಲದ್ದು
ಇಲ್ಲ-ವಾಗಿದೆ
ಇಲ್ಲ-ವಾದರೆ
ಇಲ್ಲವೆ
ಇಲ್ಲ-ವೆಂದನು
ಇಲ್ಲ-ವೆಂದು
ಇಲ್ಲ-ವೆಂದೆ
ಇಲ್ಲ-ವೆಂಬು-ದನ್ನು
ಇಲ್ಲ-ವೆ-ನಗೆ
ಇಲ್ಲ-ವೆನ್ನ-ಲಾರೆ
ಇಲ್ಲ-ವೆನ್ನಿ-ಸುತ್ತದೆ
ಇಲ್ಲವೇ
ಇಲ್ಲ-ವೇನು
ಇಲ್ಲ-ವೇನೋ
ಇಲ್ಲವೊ
ಇಲ್ಲವೋ
ಇಲ್ಲಿ
ಇಲ್ಲಿಂದ
ಇಲ್ಲಿಂದಲೂ
ಇಲ್ಲಿಂದಾಚೆಗೆ
ಇಲ್ಲಿಗೂ
ಇಲ್ಲಿಗೆ
ಇಲ್ಲಿಗೇ
ಇಲ್ಲಿದೆ
ಇಲ್ಲಿದ್ದೇವೆ
ಇಲ್ಲಿನ
ಇಲ್ಲಿ-ನ-ವ-ರಿ-ಗಿಂತ
ಇಲ್ಲಿಯ
ಇಲ್ಲಿ-ಯ-ವ-ರೆಗೂ
ಇಲ್ಲಿ-ಯ-ವರೆಗೆ
ಇಲ್ಲಿಯೂ
ಇಲ್ಲಿಯೇ
ಇಲ್ಲಿ-ರಲು
ಇಲ್ಲಿ-ರುವ
ಇಲ್ಲಿ-ರುವ-ವನ
ಇಲ್ಲಿ-ರುವೆ
ಇಲ್ಲಿವೆ
ಇಲ್ಲೆ
ಇಲ್ಲೇ
ಇಲ್ಲೊಂದು
ಇಳಿದರು
ಇಳಿದು
ಇಳಿದು-ಕೊಂಡಿದ್ದರು
ಇಳಿದು-ಬಂದಾಗ
ಇಳಿದು-ಬಂದು
ಇಳಿದು-ಬ-ರ-ಲಾ-ರರು
ಇಳಿದು-ಬರು-ವಂತಿಲ್ಲ
ಇಳಿ-ದೊಡ-ನೆಯೇ
ಇಳಿ-ಯ-ಲಾರದು
ಇಳಿ-ಯಿತು
ಇಳಿ-ಯುತ್ತಿದ್ದ
ಇಳಿಯು-ವುವೋ
ಇಳಿವು-ದನು
ಇಳಿಸ-ಬಲ್ಲರೋ
ಇಳಿ-ಸಲು
ಇಳಿಸಿ
ಇಳಿಸಿವೆ
ಇಳಿ-ಸುವ
ಇಳೆಗೆ
ಇಳೆಯೊಳಾಡು-ತಲಿ-ಹೆನು
ಇವ
ಇವಕ್ಕಾಗಿ
ಇವತ್ತಿಗೂ
ಇವತ್ತು
ಇವನ
ಇವ-ನನ್ನು
ಇವನಾಚೆಗಟ್ಟಿತ್ತೊ
ಇವ-ನಿಂದ
ಇವನಿ-ಗಾಯ್ತು
ಇವ-ನಿಗೆ
ಇವ-ನಿಲ್ಲಿ
ಇವನು
ಇವ-ನೇನು
ಇವ-ನೊ-ಡನೆ
ಇವನ್ನು
ಇವನ್ನೆಲ್ಲ
ಇವನ್ನೇ
ಇವರ
ಇವರಂತಹ
ಇವ-ರದು
ಇವ-ರನ್ನು
ಇವ-ರಲ್ಲಿ
ಇವ-ರಿಂದ
ಇವ-ರಿಗೆ
ಇವ-ರಿ-ಗೆಲ್ಲಾ
ಇವರಿಬ್ಬರ
ಇವ-ರಿಬ್ಬರೂ
ಇವರು
ಇವರೆಯೇ
ಇವ-ರೆಲ್ಲ
ಇವ-ರೆಲ್ಲರ
ಇವ-ರೆಲ್ಲ-ರಲ್ಲೂ
ಇವ-ರೆಲ್ಲ-ರಿಗೂ
ಇವ-ರೆಲ್ಲಾ
ಇವ-ರೆಷ್ಟು
ಇವರೇ
ಇವಳು
ಇವಾ-ವತಸ್ಥೇ
ಇವಾ-ವು-ದಕ್ಕೂ
ಇವಾವುವೂ
ಇವಿಷ್ಟೇ
ಇವು
ಇವು-ಗಳ
ಇವು-ಗಳನ್ನು
ಇವು-ಗಳನ್ನೆಲ್ಲ
ಇವು-ಗಳನ್ನೆಲ್ಲಾ
ಇವು-ಗಳನ್ನೇ
ಇವು-ಗಳಲ್ಲಿ
ಇವು-ಗಳಲ್ಲಿಯೂ
ಇವು-ಗಳಲ್ಲಿಯೇ
ಇವು-ಗ-ಳಲ್ಲೇ
ಇವು-ಗಳಿಂದ
ಇವು-ಗಳಿ-ಗಾಗಿ
ಇವು-ಗಳಿಗೂ
ಇವು-ಗಳಿಗೆ
ಇವು-ಗಳು
ಇವು-ಗಳೂ
ಇವು-ಗಳೆಲ್ಲ
ಇವು-ಗಳೆಲ್ಲ-ದರ
ಇವು-ಗಳೆಲ್ಲ-ದ-ರಿಂದ
ಇವು-ಗಳೆಲ್ಲ-ವನ್ನೂ
ಇವು-ಗಳೆಲ್ಲಾ
ಇವು-ಗಳೊಂದನ್ನೂ
ಇವು-ಗಳೊಂದೂ
ಇವೆ
ಇವೆಯೆ
ಇವೆ-ಯೆಂದು
ಇವೆಯೇ
ಇವೆಯೋ
ಇವೆ-ರಡಕ್ಕೂ
ಇವೆ-ರಡರ
ಇವೆ-ರಡ-ರಲ್ಲಿ
ಇವೆ-ರಡು
ಇವೆ-ರಡೂ
ಇವೆಲ್ಲ
ಇವೆಲ್ಲಕ್ಕಿಂತಲೂ
ಇವೆಲ್ಲ-ದ-ರಲ್ಲಿಯೂ
ಇವೆಲ್ಲ-ವನ್ನೂ
ಇವೆಲ್ಲವೂ
ಇವೆಲ್ಲಾ
ಇವೇ
ಇವೊಂದೂ
ಇವೊತ್ತಿಗೂ
ಇವೊತ್ತು
ಇಷ್ಟ
ಇಷ್ಟಕ್ಕೆ
ಇಷ್ಟ-ದಂತೆ
ಇಷ್ಟ-ದೇವತೆ
ಇಷ್ಟ-ದೇವ-ತೆಯ
ಇಷ್ಟ-ದೇವರ
ಇಷ್ಟನ್ನು
ಇಷ್ಟ-ಪಟ್ಟರೆ
ಇಷ್ಟ-ಪಟ್ಟರೊ
ಇಷ್ಟ-ಪಟ್ಟಿರಾ
ಇಷ್ಟ-ಪಟ್ಟೆ
ಇಷ್ಟ-ಪ-ಡದೆ
ಇಷ್ಟ-ಪಡುತ್ತೀರಿ
ಇಷ್ಟ-ಪಡು-ವರೊ
ಇಷ್ಟ-ಪಡು-ವರೋ
ಇಷ್ಟ-ಪಡು-ವು-ದಿಲ್ಲ
ಇಷ್ಟ-ಪಡುವೆ
ಇಷ್ಟ-ಪಡು-ವೆ-ನೆಂದು
ಇಷ್ಟ-ಬಂದದ್ದನ್ನು
ಇಷ್ಟ-ಮಿತ್ರ-ರೊಡನೆ
ಇಷ್ಟ-ರಲ್ಲಿ
ಇಷ್ಟಲ್ಲದೆ
ಇಷ್ಟ-ವನ್ನು
ಇಷ್ಟ-ವಿದ್ದರೆ
ಇಷ್ಟ-ವಿದ್ದು
ಇಷ್ಟ-ವಿ-ರ-ಲಿಲ್ಲ
ಇಷ್ಟ-ವಿಲ್ಲ
ಇಷ್ಟ-ವಿಲ್ಲ-ದಿದ್ದರೂ
ಇಷ್ಟ-ವೇ-ನೆಂದರೆ
ಇಷ್ಟವೋ
ಇಷ್ಟಾ-ದರೂ
ಇಷ್ಟಾರ್ಥ-ವನ್ನು
ಇಷ್ಟು
ಇಷ್ಟು-ದಿನ
ಇಷ್ಟೆ
ಇಷ್ಟೆಲ್ಲಾ
ಇಷ್ಟೇ
ಇಷ್ಟೊಂದು
ಇಸವಿ
ಇಹ
ಇಹ-ಜೀವನ-ದಲ್ಲಿ
ಇಹನು
ಇಹ-ಲೋಕದ
ಇಹವು
ಇಹವೂ
ಇಹಾಮಿತ-ಶಕ್ತಿ-ಪಾಲಾಃ
ಇಹಾಸನೇ
ಇಹುದು
ಈ
ಈಕೆಯು
ಈಕ್ಷಿ-ಸು-ವ-ವನು
ಈಗ
ಈಗಂತೂ
ಈಗ-ತಾನೆ
ಈಗ-ತಾನೇ
ಈಗಲೂ
ಈಗಲೆ
ಈಗಲೇ
ಈಗಾ-ಗಲೇ
ಈಗಿ-ಗಿಂತ
ಈಗಿನ
ಈಗಿನ್ನೂ
ಈಗಿ-ರುವ
ಈಗಿ-ರು-ವಂತೆಯೇ
ಈಗೀಗ
ಈಗೀಗಂತೂ
ಈಗೆಲ್ಲಿ
ಈಗ್ಗೆ
ಈಚಿನ
ಈಚೀಚೆಗೆ
ಈಚೆ
ಈಚೆಗೆ
ಈಜಾ-ಡಿದೆ
ಈಜಿಪ್ಟ್
ಈಡು
ಈಡೇರಿ-ದೆಯೇ
ಈಡೇರಿ-ಸುವನು
ಈಡೇ-ರುತ್ತದೆ-ಯೆಂದು
ಈಡೇ-ರು-ವು-ದಿಲ್ಲ
ಈತ
ಈತನ
ಈತ-ನಲ್ಲಿ
ಈತ-ನಿಗೂ
ಈತ-ನಿಗೆ
ಈದಿನ-ಗಳಲ್ಲಿ
ಈರುಳ್ಳಿ
ಈರ್ಷ್ಯೆಗೊಳಗಾ-ಗುತ್ತದೆ
ಈವ-ರೆಗೂ
ಈವಾಗ
ಈವೊತ್ತು
ಈಶತ್ವ
ಈಶನ
ಈಶ-ಪೂಜೆ
ಈಶಸಂಸ್ಥೇಪ್ಯನೀಶೇ
ಈಶಾನ್ಯ-ದಲ್ಲಿ-ರುವ
ಈಶಾನ್ವೇಷಣೆ
ಈಶ್ವರ
ಈಶ್ವರ-ಕೋಟಿ-ಗಳು
ಈಶ್ವರ-ಚಂದ್ರ
ಈಶ್ವರನ
ಈಶ್ವರ-ನನ್ನು
ಈಶ್ವರ-ನಿಂದೆ
ಈಶ್ವರನು
ಈಶ್ವರ-ನೆಂದು
ಈಶ್ವರನೇ
ಈಶ್ವರಾರ್ಥ-ವಾಗಿ
ಈಶ್ವರೋದ್ದೀಪ-ನವೂ
ಈಸಾರಿಗೆ
ಈಸುತ್ತ
ಈಸ್ಟ್
ಉಂಗುರ-ವನ್ನು
ಉಂಟಾಗ-ಬೇಕು
ಉಂಟಾಗ-ಲಿಲ್ಲ
ಉಂಟಾ-ಗಿದೆ
ಉಂಟಾಗಿ-ದೆಯೊ
ಉಂಟಾಗಿದ್ದ
ಉಂಟಾ-ಗುತ್ತದೆ
ಉಂಟಾಗುತ್ತಿತ್ತು
ಉಂಟಾಗುತ್ತಿ-ರ-ಲಿಲ್ಲ
ಉಂಟಾಗುವ
ಉಂಟಾಗು-ವುದಕ್ಕೆ
ಉಂಟಾ-ಗು-ವು-ದಿಲ್ಲ
ಉಂಟಾ-ಗು-ವು-ದಿಲ್ಲವೋ
ಉಂಟಾಗು-ವುದು
ಉಂಟಾದ
ಉಂಟಾ-ದರೆ
ಉಂಟಾ-ದೀತು
ಉಂಟಾ-ದು-ದೆಂದು
ಉಂಟಾ-ಯಿತು
ಉಂಟು
ಉಂಟು-ಮಾಡಿ
ಉಂಟು-ಮಾಡಿದೆ
ಉಂಟು-ಮಾಡುತ್ತ-ದೆಂದು
ಉಂಟು-ಮಾಡುತ್ತವೆ
ಉಂಟು-ಮಾಡುತ್ತಾರೆ
ಉಂಟು-ಮಾಡುವ
ಉಂಟು-ಮಾಡು-ವ-ರೆಂದು
ಉಂಟು-ಮಾಡು-ವುದಕ್ಕೆ
ಉಂಟು-ಮಾಡು-ವು-ದಿಲ್ಲ
ಉಂಟು-ಮಾಡು-ವುದು
ಉಂಟು-ಮಾಡು-ವುದೆಂಬುದು
ಉಂಟೆ
ಉಕ್ಕಿ
ಉಕ್ಕಿ-ಬಂದಂತಾಗಿದೆ
ಉಕ್ಕಿ-ಸುತ್ತ-ವೆಯೋ
ಉಕ್ತ-ವಾಗಿ-ರು-ವುದು
ಉಗರೆ
ಉಗ್ರ
ಉಗ್ರ-ಭಾವ-ವನ್ನು
ಉಗ್ರ-ವಾಗಿದ್ದರು
ಉಗ್ರಾ-ಣಕ್ಕೆ
ಉಗ್ರಾಣ-ದಿಂದ
ಉಚಿತ
ಉಚಿತ-ವಲ್ಲ
ಉಚ್ಚ
ಉಚ್ಚ-ಜಾತಿ-ಯ-ವನು
ಉಚ್ಚ-ತಮ
ಉಚ್ಚ-ತಮ-ವಾ-ದು-ದೆಂದು
ಉಚ್ಚ-ತರ
ಉಚ್ಚ-ತರ-ವಾದ
ಉಚ್ಚ-ಭಾವ-ದಿಂದ
ಉಚ್ಚ-ಮಟ್ಟದ
ಉಚ್ಚ-ರಿಸ-ಬೇಕು
ಉಚ್ಚ-ರಿಸಿ
ಉಚ್ಚ-ರಿಸಿ-ದರು
ಉಚ್ಚ-ಲಕ್ಷ್ಯ-ವನ್ನು
ಉಚ್ಚ-ವಾಗಿತ್ತೊ
ಉಚ್ಚ-ವಾಗಿ-ರುತ್ತದೋ
ಉಚ್ಚ-ವಾದ
ಉಚ್ಚಾ-ದರ್ಶ
ಉಚ್ಚಾ-ದರ್ಶ-ಗಳನ್ನು
ಉಚ್ಚಾರಣೆ
ಉಚ್ಚಾರ-ಣೆ-ಗಳನ್ನು
ಉಚ್ಚಾರ-ಣೆಯ
ಉಚ್ಛ
ಉಚ್ಛ-ಧರ್ಮ-ಭಾವ-ಗಳಿಗೆ
ಉಚ್ಛೃಂಖಲತೆ-ಯಾ-ಗಲಿ
ಉಚ್ಛ್ವಾಸ
ಉಚ್ವಾಸ
ಉಜಲಾ
ಉಜೀವನ-ಗೊಳಿ-ಸಲು
ಉಜ್ಜಿ
ಉಜ್ಜು-ವಾಗ
ಉಜ್ಜ್ವಲ
ಉಜ್ವಲ
ಉಜ್ವಲ-ವಾಗಿ
ಉಜ್ವಲ-ವಾಗಿ-ದೆಯೋ
ಉಜ್ವಲ-ವಾದ
ಉಟೆಚೆ
ಉಟ್ಟು-ಕೊಂಡು
ಉಠೆ
ಉಠೇ
ಉಡಾಯಿ-ಸುತ್ತಿದ್ದರು
ಉಡಿಸಿದ್ದಾ-ಯಿತು
ಉಡು-ಪನ್ನು
ಉಡು-ಪನ್ನೂ
ಉಡು-ಪನ್ನೇ
ಉಡುಪಿ-ಗಿಂತ
ಉಡುಪಿಗೆ
ಉಡುಪಿನ
ಉಡುಪಿ-ನಲ್ಲಿ
ಉಡುಪಿನಲ್ಲಿಯೂ
ಉಡುಪಿನಲ್ಲಿಯೇ
ಉಡುಪು
ಉಡುಪು-ಗಳು
ಉಡೇ
ಉಣ್ಣ-ಬಹು-ದಲ್ಲವೆ
ಉಣ್ಣಲು
ಉಣ್ಣಲೂ
ಉಣ್ಣುತ್ತಿದ್ದರೆ
ಉತ-ಸಾರಿಕ
ಉತ್ಕಟ
ಉತ್ಕಟ-ವಾಗಿ
ಉತ್ಕಟ-ವಾಗಿ-ದೆ-ಯೆಂದರೆ
ಉತ್ಕಟ-ವಾದ
ಉತ್ಕೃಷ್ಟ
ಉತ್ಕೃಷ್ಟತೆ
ಉತ್ಕೃಷ್ಟ-ತೆ-ಯನ್ನು
ಉತ್ಕೃಷ್ಟ-ವಾದ
ಉತ್ತಮ
ಉತ್ತಮ-ಗೊಳಿಸ-ಬಹುದೊ
ಉತ್ತಮ-ಗೊಳಿ-ಸಲು
ಉತ್ತಮ-ಗೊಳಿ-ಸಿ-ರು-ವು-ದ-ರಿಂದ
ಉತ್ತಮ-ಗೊಳ್ಳಲು
ಉತ್ತಮ-ನಾಗಲು
ಉತ್ತಮ-ಪಡಿ-ಸ-ಲಾ-ಗು-ವು-ದಿಲ್ಲ
ಉತ್ತಮ-ಪಡಿ-ಸಿ-ಕೊಂಡೇ
ಉತ್ತಮ-ರಾಗುತ್ತೇವೆ
ಉತ್ತ-ಮರು
ಉತ್ತಮ-ವಾಗಿಲ್ಲ
ಉತ್ತಮ-ವಾಗುತ್ತಿತ್ತು
ಉತ್ತಮ-ವಾದ
ಉತ್ತಮ-ವಾದುದ-ರೊಂದಿಗೆ
ಉತ್ತಮವೆ
ಉತ್ತಮ-ವೆನ್ನಿ-ಸು-ವುದು
ಉತ್ತರ
ಉತ್ತರ-ಕೊ-ಡದೆ
ಉತ್ತ-ರಕ್ಕೆ
ಉತ್ತರದ
ಉತ್ತರ-ದಲ್ಲಿ
ಉತ್ತರ-ದಲ್ಲಿದ್ದ
ಉತ್ತರಪ್ರ-ದೇಶದ
ಉತ್ತರ-ಮೀ-ಮಾಂಸಾ
ಉತ್ತರ-ರಾಮ-ಚರಿತ-ವನ್ನೋ-ದಿಲ್ಲವೆ
ಉತ್ತರ-ರಾಹ್ರಿ
ಉತ್ತರ-ವನ್ನು
ಉತ್ತರ-ವನ್ನೂ
ಉತ್ತರ-ವನ್ನೇನೂ
ಉತ್ತರ-ವಾಗಿ
ಉತ್ತರವು
ಉತ್ತರಾ-ಧಿ-ಕಾರಿ-ಗಳು
ಉತ್ತರಾಯಣ-ರವಿಯ
ಉತ್ತರಾರ್ಧ
ಉತ್ತ-ರಿ-ಸಿ-ದರು
ಉತ್ತರಿ-ಸುತ್ತಾರೆ
ಉತ್ತರೀಯ
ಉತ್ತರೋತ್ತರ
ಉತ್ತಿಷ್ಠತ
ಉತ್ತೇ-ಜನ
ಉತ್ತೇಜಿತ-ರಾದದ್ದನ್ನು
ಉತ್ಪತ್ತಿ
ಉತ್ಪತ್ತಿ-ಮಾಡುವ
ಉತ್ಪತ್ತಿ-ಯಾಗುತ್ತವೆ
ಉತ್ಪತ್ತಿ-ಯಾ-ಗುತ್ತಿದೆ
ಉತ್ಪತ್ತಿ-ಯೆಂದು
ಉತ್ಪನ್ನ-ವಾಗಿದೆ
ಉತ್ಪನ್ನ-ವಾಗು-ವುವು
ಉತ್ಪಾತ-ವಿಲ್ಲ
ಉತ್ಫುಲ್ಲ
ಉತ್ಸವ
ಉತ್ಸವಕ್ಕಾಗಿ
ಉತ್ಸ-ವಕ್ಕೆ
ಉತ್ಸವದ
ಉತ್ಸವಾಮೋದ-ಗಳಿಂದ
ಉತ್ಸವಾಮೋದ-ಗಳು
ಉತ್ಸಾಹ
ಉತ್ಸಾಹ-ಗಳು
ಉತ್ಸಾಹ-ಗೊಳಿಸಿ
ಉತ್ಸಾಹ-ಗೊಳಿ-ಸುತ್ತಲೂ
ಉತ್ಸಾಹದ
ಉತ್ಸಾಹ-ದಿಂದ
ಉತ್ಸಾಹ-ದಿಂದಲೂ
ಉತ್ಸಾಹ-ಪೂರಿ-ತ-ನಾಗಿ
ಉತ್ಸಾಹ-ಪೂರಿ-ತ-ವಾಗಿ
ಉತ್ಸಾಹ-ಪೂರ್ಣ
ಉತ್ಸಾಹ-ರಾ-ಹಿತ್ಯವೂ
ಉತ್ಸಾಹ-ವನ್ನು
ಉತ್ಸಾಹ-ವನ್ನುಂಟು-ಮಾಡಿ
ಉತ್ಸಾಹ-ವನ್ನೆಲ್ಲಾ
ಉತ್ಸಾಹ-ವಿದ್ದರೆ
ಉತ್ಸಾಹ-ವಿಲ್ಲದೆ
ಉತ್ಸಾಹವೂ
ಉತ್ಸಾಹ-ವೆಲ್ಲಾ
ಉತ್ಸಾಹ-ವೇನೂ
ಉತ್ಸಾಹ-ಶಾಲಿ-ಗಳಾದ
ಉತ್ಸಾಹ-ಶೂನ್ಯ
ಉತ್ಸಾಹಿ
ಉತ್ಸಾಹಿ-ಗಳಾಗಿದ್ದಾರೆ
ಉತ್ಸಾಹಿ-ಗ-ಳಾದ
ಉತ್ಸಾಹಿ-ಗಳು
ಉತ್ಸಾಹಿ-ಗಳೂ
ಉತ್ಸಾಹಿ-ತ-ರಾಗಿ
ಉತ್ಸಾಹಿ-ತ-ರಾಗಿದ್ದೀರಿ
ಉತ್ಸುಕ-ನಾಗಿದ್ದೆ
ಉತ್ಸುಕ-ರಾಗಿ
ಉತ್ಸುಕ-ರಾಗಿದ್ದ
ಉತ್ಸುಕ-ರಾಗಿದ್ದರು
ಉತ್ಸುಕ-ರಾಗಿದ್ದಾರೆ
ಉತ್ಸುಕ-ರಾಗುತ್ತಿದ್ದ
ಉತ್ಸುಕ-ರಾಗುತ್ತಿದ್ದರು
ಉತ್ಸು-ಕರೂ
ಉತ್ಸು-ಕ-ವಾಗಿ
ಉದಯ-ವಾಗುವ
ಉದಯಿಸ-ಲಾರದು
ಉದಯಿ-ಸುವ
ಉದಯಿ-ಸುವ-ವರೆಗೆ
ಉದರ-ಪೂರನ್
ಉದಾತ್ತ
ಉದಾರ
ಉದಾರ-ಚಿತ್ತರೂ
ಉದಾರತೆ
ಉದಾರ-ತೆಯು
ಉದಾರ-ಭಾವ-ಗಳ
ಉದಾರ-ಭಾವ-ಗಳನ್ನು
ಉದಾರ-ಭಾವ-ವನ್ನು
ಉದಾರ-ವಾಗಿ
ಉದಾರ-ವಾಗಿತ್ತೋ
ಉದಾರ-ವಾದ
ಉದಾಸೀ-ನದಿಂದ
ಉದಾಸೀ-ನ-ದಿಂದಿರು-ವನೊ
ಉದಾಸೀನ-ನಾ-ಗಿ-ರಲಿ
ಉದಾಹರ-ಣೆಗೆ
ಉದಾ-ಹ-ರಣೆ-ಯನ್ನೇ
ಉದಾ-ಹ-ರಣೆ-ಯಾ-ಗಿದೆ
ಉದಾಹರಿ-ಸಿದರೆ
ಉದಿಸಿತಲ್ಲಿ
ಉದಿ-ಸಿದ
ಉದಿ-ಸಿದ್ದ-ರೆಂದು
ಉದಿಸು-ವುದೆ
ಉದಿ-ಸು-ವುದೋ
ಉದಿಸು-ವುವೋ
ಉದುರಿ
ಉದು-ರು-ವಂತೆ
ಉದ್ಗಾರ
ಉದ್ಘೋಷ-ಗೈದ-ನವ
ಉದ್ಘೋಷಿಸಿ
ಉದ್ಘೋಷಿಸು
ಉದ್ಘೋಷಿಸುತ್ತ
ಉದ್ದ
ಉದ್ದಕ್ಕೆ
ಉದ್ದ-ವಲ್ಲದ
ಉದ್ದ-ವಾದ
ಉದ್ದಾಮ
ಉದ್ದಾರಕ್ಕಾಗಿ
ಉದ್ದಾರ-ವಾ-ಗು-ವು-ದಿಲ್ಲ
ಉದ್ದೀ-ಪನ
ಉದ್ದೀ-ಪನ-ಗೊಳಿಸಿ
ಉದ್ದೀ-ಪನ-ಗೊಳಿಸು-ವುದು
ಉದ್ದೀ-ಪನ-ಗೊಳ್ಳುವ
ಉದ್ದೀಪ್ತ-ವಾದ
ಉದ್ದೇಶ
ಉದ್ದೇಶಕ್ಕಾಗಿ
ಉದ್ದೇಶ-ಗಳ
ಉದ್ದೇಶ-ಗಳನ್ನು
ಉದ್ದೇಶ-ಗಳಿಂದ
ಉದ್ದೇಶ-ಗಳಿವೆ
ಉದ್ದೇಶ-ಗಳೂ
ಉದ್ದೇಶದ
ಉದ್ದೇಶ-ದಿಂದ
ಉದ್ದೇಶ-ವನ್ನಿಟ್ಟು-ಕೊಂಡು
ಉದ್ದೇಶ-ವನ್ನು
ಉದ್ದೇಶ-ವಾಗಲಿ
ಉದ್ದೇಶ-ವಾಗಿ
ಉದ್ದೇಶ-ವಾದರೂ
ಉದ್ದೇಶ-ವಿದ್ದೇ
ಉದ್ದೇಶ-ವಿ-ರಲಿ
ಉದ್ದೇಶ-ವಿಲ್ಲದೆ
ಉದ್ದೇಶವು
ಉದ್ದೇಶವೂ
ಉದ್ದೇಶ-ವೆಂದು
ಉದ್ದೇಶ-ವೆಂದೂ
ಉದ್ದೇಶ-ವೆಂಬು-ದಾಗಿ
ಉದ್ದೇಶವೇ
ಉದ್ದೇಶ-ವೇನು
ಉದ್ದೇಶ-ವೇ-ನೆಂದರೆ
ಉದ್ದೇಶಿಸಿ
ಉದ್ದೇಶಿಸಿ-ರುತ್ತದೆ
ಉದ್ಧರಿ-ಸ-ಬೇಕು
ಉದ್ಧರಿ-ಸ-ಬೇಕೆಂದು
ಉದ್ಧರಿ-ಸಲು
ಉದ್ಧ-ರಿಸಿ
ಉದ್ಧರಿ-ಸು-ವುದು
ಉದ್ಧಾಮ
ಉದ್ಧಾರ
ಉದ್ಧಾ-ರಕ್ಕೆ
ಉದ್ಧಾರದ
ಉದ್ಧಾರ-ಮಾಡಿ
ಉದ್ಧಾರ-ಮಾಡು
ಉದ್ಧಾರ-ವಾಗು-ವುದಕ್ಕೆ
ಉದ್ಧಾರ-ವಾಗು-ವುದು
ಉದ್ಧೀ-ಪನ-ವಾಗ-ಬೇಕು
ಉದ್ಧೃತಗೊಳಿಸ-ಲಾಗಿ-ದೆ-ಯೆಂದು
ಉದ್ಬಂಧ-ನ-ವನ್ನು
ಉದ್ಭವ-ವಾ-ಯಿತು
ಉದ್ಭವಿಸಿದ
ಉದ್ಭಾವನ
ಉದ್ಭುದ್ಧ-ವಾಗುತ್ತದೆ
ಉದ್ಭೂತ-ವಾಗುತ್ತದೆ
ಉದ್ಭೋ-ಧನ
ಉದ್ಭೋ-ಧನಕ್ಕಾಗಿ
ಉದ್ಭೋ-ಧನದ
ಉದ್ಭೋ-ಧನ-ದಲ್ಲಿ
ಉದ್ಭೋಧ-ನವು
ಉದ್ಯಮ
ಉದ್ಯ-ಮ-ಗಳನ್ನು
ಉದ್ಯಮ-ವಾಗಿತ್ತು
ಉದ್ಯಮ-ಶೀಲ-ತೆ-ಯನ್ನು
ಉದ್ಯಮ-ಶೀಲ-ನಾಗಿ
ಉದ್ಯಮ-ಶೀಲರೂ
ಉದ್ಯಮ-ಹೀನರೂ
ಉದ್ಯಾನ
ಉದ್ಯಾನ-ಗೃಹಕ್ಕೆ
ಉದ್ಯುಕ್ತ-ನಾಗುತ್ತಾನೆ
ಉದ್ಯುಕ್ತ-ನಾದನು
ಉದ್ಯುಕ್ತ-ರಾ-ಗಿ-ರುವ
ಉದ್ಯುಕ್ತ-ವಾ-ದಂತಿದೆ
ಉದ್ಯೋಗ
ಉದ್ಯೋಗ-ಗಳನ್ನೂ
ಉದ್ಯೋಗ-ವನ್ನು
ಉದ್ರಿಕ್ತ-ವಾಗ-ಬೇಕು
ಉದ್ರಿಕ್ತ-ವಾಗಿದೆ
ಉದ್ರೇಕ
ಉದ್ರೇಕ-ಗೊಳಿಸುವ
ಉದ್ರೇಕ-ಗೊಳಿ-ಸು-ವು-ದಿಲ್ಲವೆ
ಉದ್ರೇಕಿಸ-ಬಲ್ಲೆ
ಉದ್ರೇಕಿ-ಸು-ವಂತೆ
ಉದ್ರೇಕಿ-ಸು-ವುದು
ಉದ್ವಿಗ್ನ-ರಾಗಿ
ಉದ್ವೇಗ
ಉದ್ವೇಗದ
ಉದ್ವೇಗ-ದಿಂದ
ಉದ್ವೇಗ-ಪರ
ಉದ್ವೇಗ-ಪರರೋ
ಉದ್ವೇಗ-ಪರ-ವಶತೆ
ಉದ್ವೇಗ-ವಶ-ರಾಗ-ಕೂ-ಡದು
ಉದ್ವೇಗ-ವಶ-ರಾಗಿ
ಉದ್ವೇಗ-ವಿಲ್ಲದ
ಉನ್ನತ
ಉನ್ನತ-ಮಟ್ಟಕ್ಕೆ
ಉನ್ನತ-ರನ್ನಾಗಿ
ಉನ್ನ-ತರು
ಉನ್ನತಿ
ಉನ್ನತಿ-ಗಳಲ್ಲಿ
ಉನ್ನತಿ-ಗಾಗಿ
ಉನ್ನ-ತಿಗೆ
ಉನ್ನತಿ-ಗೋಸ್ಕರ
ಉನ್ನ-ತಿಯ
ಉನ್ನತಿ-ಯನ್ನು
ಉನ್ನ-ತಿಯು
ಉನ್ನತಿ-ಯುಂಟಾಗುವ
ಉನ್ನತಿ-ಯುಂಟು-ಮಾಡಿ-ಕೊಡು-ವುದಕ್ಕೆ
ಉನ್ನತೋನ್ನತ
ಉನ್ಮತ್ತ-ತೆ-ಯುಂಟಾಗಿತ್ತೊ
ಉನ್ಮತ್ತ-ನಾಗಿದ್ದ-ನ-ವ-ನೊಬ್ಬನು
ಉನ್ಮತ್ತ-ನಾಗಿ-ಬಿಡು-ವನೋ
ಉನ್ಮತ್ತನೊ
ಉನ್ಮತ್ತರ
ಉನ್ಮತ್ತ-ರಂತಾಗಿ-ಬಿಡುತ್ತಿದ್ದರು
ಉನ್ಮತ್ತ-ರಾಗಿ
ಉನ್ಮತ್ತ-ರಾಗುವರು
ಉನ್ಮದ
ಉನ್ಮಾದ
ಉನ್ಮಾದೇರ
ಉನ್ಮಾದ್
ಉಪ-ಕರಣ
ಉಪ-ಕರ-ಣ-ಗಳಿ-ಗಾಗಿ
ಉಪ-ಕರ-ಣ-ಗಳು
ಉಪ-ಕಾರ
ಉಪ-ಕಾರಕ್ಕೋಸ್ಕರ
ಉಪ-ಕಾರ-ದರ್ಶಿ-ಗಳಾಗಿಯೂ
ಉಪ-ಕಾರ-ಮಾಡಿ-ದಂತಾಗು-ವು-ದೇನು
ಉಪ-ಕಾರ-ವನ್ನು
ಉಪ-ಕಾರ-ವಾಗ-ಬೇಕು
ಉಪ-ಕಾರ-ವಾಗುತ್ತ-ದೆಂದು
ಉಪ-ಕಾರ-ವಾಗುತ್ತ-ದೆ-ಯೇನು
ಉಪ-ಕಾರ-ವಾಗು-ವುದು
ಉಪ-ಕಾರ-ವೆಸಗಿ-ದ-ವರಲ್ಲೆಲ್ಲಾ
ಉಪ-ಕಾ-ರವೇ
ಉಪ-ಚಾರ
ಉಪ-ಚಾರ-ಗಳನ್ನು
ಉಪ-ಚಾರಾದಿ-ಗಳ
ಉಪ-ಚಾರಾದಿ-ಗಳನ್ನು
ಉಪಟಳ-ದಲ್ಲಿ-ರುವ-ವ-ರಿಗೆ
ಉಪ-ದೇಶ
ಉಪದೇ-ಶ-ಕರು
ಉಪ-ದೇಶಕ್ಕೆ
ಉಪ-ದೇಶ-ಗಳನ್ನು
ಉಪ-ದೇಶ-ಗಳಿಂದ
ಉಪದೇ-ಶ-ಗಳು
ಉಪದೇ-ಶದ
ಉಪದೇ-ಶ-ಮಾಡುತ್ತಿದ್ದರೆ
ಉಪ-ದೇಶ-ವನ್ನು
ಉಪದೇ-ಶ-ವನ್ನೂ
ಉಪದೇ-ಶ-ವೇನು
ಉಪದೇ-ಶಿ-ಸಿದ
ಉಪದೇ-ಶಿ-ಸಿ-ದನು
ಉಪದೇ-ಶಿ-ಸಿ-ದರು
ಉಪದೇ-ಶಿಸಿ-ಬಿಟ್ಟಿದ್ದಾರೆ
ಉಪದೇ-ಶಿ-ಸುತ್ತವೆ
ಉಪದೇ-ಶಿ-ಸುತ್ತಿದ್ದನು
ಉಪದೇ-ಶಿ-ಸುವರು
ಉಪದ್ರವ
ಉಪ-ನಯನ
ಉಪ-ನಯನ-ವೆಂದು
ಉಪನಿಷತ್
ಉಪನಿಷತ್ತನ್ನು
ಉಪನಿಷತ್ತಲ್ಲದೆ
ಉಪನಿಷತ್ತಿನ
ಉಪನಿಷತ್ತಿ-ನಲ್ಲಿ
ಉಪನಿಷತ್ತು
ಉಪನಿಷತ್ತು-ಗಳ
ಉಪನಿಷತ್ತು-ಗಳನ್ನು
ಉಪನಿಷತ್ತು-ಗಳಲ್ಲಿ
ಉಪನಿಷತ್ತು-ಗಳಲ್ಲಿಯೂ
ಉಪನಿಷತ್ತು-ಗಳು
ಉಪನಿಷತ್
ಉಪನಿಷದ್ಧರ್ಮ-ವನ್ನೆ
ಉಪನ್ಯಾಯಾಧಿ-ಪತಿ
ಉಪನ್ಯಾಯಾಧಿ-ಪ-ತಿಯ
ಉಪನ್ಯಾಯಾಧೀಶನ
ಉಪನ್ಯಾಸ
ಉಪನ್ಯಾ-ಸಕ್ಕೆ
ಉಪನ್ಯಾಸ-ಗಳ
ಉಪನ್ಯಾಸ-ಗಳನ್ನಿಡು-ವುದು
ಉಪನ್ಯಾಸ-ಗಳನ್ನು
ಉಪನ್ಯಾಸ-ಗಳಿಂದ
ಉಪನ್ಯಾ-ಸದ
ಉಪನ್ಯಾಸ-ದಲ್ಲಿ
ಉಪನ್ಯಾಸ-ದಲ್ಲಿಯೂ
ಉಪನ್ಯಾಸ-ದಿಂದಾಗುವ
ಉಪನ್ಯಾಸ-ವನ್ನು
ಉಪನ್ಯಾಸ-ವಾದ
ಉಪನ್ಯಾಸವು
ಉಪ-ಮಾನ
ಉಪ-ಮಾನದ
ಉಪ-ಯುಕ್ತತೆ
ಉಪ-ಯುಕ್ತ-ತೆಯ
ಉಪ-ಯುಕ್ತ-ತೆ-ಯನ್ನು
ಉಪ-ಯುಕ್ತ-ವಾಗು-ವಿರೋ
ಉಪ-ಯುಕ್ತ-ವಾದ
ಉಪ-ಯೋಗ
ಉಪ-ಯೋಗಕ್ಕಾಗಿ
ಉಪ-ಯೋಗದ
ಉಪ-ಯೋಗ-ಮಾಡುತ್ತಾರೆ
ಉಪ-ಯೋಗ-ವನ್ನು
ಉಪ-ಯೋಗ-ವಾಗದ
ಉಪ-ಯೋಗ-ವಾಗ-ಬ-ಹುದು
ಉಪ-ಯೋಗ-ವಾ-ಗು-ವಂತೆ
ಉಪ-ಯೋಗ-ವಾಗು-ವು-ದಲ್ಲದೆ
ಉಪ-ಯೋಗ-ವಾಗು-ವುದು
ಉಪ-ಯೋಗ-ವಿದೆ
ಉಪ-ಯೋಗ-ವುಳ್ಳದ್ದಾಗ-ಲಾರದು
ಉಪ-ಯೋಗಿ-ಸ-ಕೂಡ-ದೆಂದು
ಉಪ-ಯೋಗಿ-ಸ-ಬ-ಹುದು
ಉಪ-ಯೋಗಿ-ಸ-ಬೇಕು
ಉಪ-ಯೋಗಿ-ಸ-ಬೇಕೆಂಬು-ದನ್ನು
ಉಪ-ಯೋಗಿ-ಸ-ಬೇಕೆಂಬುದೂ
ಉಪ-ಯೋಗಿ-ಸಲೇ-ಬೇಕಾಗಿದೆ
ಉಪ-ಯೋಗಿ-ಸಲ್ಪಡುತ್ತದೆ
ಉಪ-ಯೋಗಿ-ಸಲ್ಪಡುತ್ತಿತ್ತು
ಉಪ-ಯೋಗಿಸಿ
ಉಪ-ಯೋಗಿ-ಸಿ-ಕೊಳ್ಳ-ಬೇ-ಕಾದರೆ
ಉಪ-ಯೋಗಿ-ಸಿ-ಕೊಳ್ಳುವ
ಉಪ-ಯೋಗಿ-ಸಿ-ದರೆ
ಉಪ-ಯೋಗಿ-ಸುತ್ತಿಲ್ಲ
ಉಪ-ಯೋಗಿ-ಸುತ್ತೇನೆ
ಉಪ-ಯೋಗಿ-ಸುವ
ಉಪ-ಯೋಗಿ-ಸುವರು
ಉಪ-ಯೋಗಿ-ಸುವುದು
ಉಪ-ಯೋಗಿ-ಸುವುದೇ
ಉಪರ
ಉಪರ್
ಉಪಲಬ್ಧಿ-ಯನ್ನು
ಉಪ-ವಾಸ
ಉಪ-ವಾಸ-ದಿಂದ
ಉಪಸಂಹಾರ
ಉಪಾಖ್ಯಾನ-ಗಳು
ಉಪಾಖ್ಯಾ-ನದಿಂದಾಗಲೀ
ಉಪಾಧ್ಯಾಯನ
ಉಪಾಧ್ಯಾಯನು
ಉಪಾಧ್ಯಾಯರ
ಉಪಾಧ್ಯಾಯ-ರನ್ನು
ಉಪಾಧ್ಯಾಯರು
ಉಪಾಧ್ಯಾಯ-ರು-ಗಳು
ಉಪಾಯ
ಉಪಾಯ-ದಿಂದ
ಉಪಾಯ-ವನ್ನು
ಉಪಾಯ-ವಿದೆ
ಉಪಾರ್ಜುನ
ಉಪಾಸಕ-ರಾಗಿ
ಉಪಾಸತೇ
ಉಪಾಸನ
ಉಪಾ-ಸನೆ
ಉಪಾ-ಸನೆ-ಗಳಿಂದ
ಉಪಾ-ಸನೆ-ಗಳಿವೆ
ಉಪಾ-ಸನೆ-ಯನ್ನು
ಉಪಾ-ಸನೆ-ಯಾಗುವುದು
ಉಪಾ-ಸನೆ-ಯಿಂದ
ಉಪೇಕ್ಷಿ-ಸುತ್ತಿದ್ದಾರೆ
ಉಪೇಕ್ಷೆ
ಉಪ್ಪನ್ನು
ಉಪ್ಪನ್ನೇಕೆ
ಉಪ್ಪಿನ
ಉಪ್ಪು
ಉಫ್
ಉಬ್ಬಸ
ಉಬ್ಬಿ-ರುವ-ವ-ರಿಗೆ
ಉಮೆ
ಉಮೆ-ಯನ್ನು
ಉಯ್ಯಾಲೆ-ಯಂತೆ
ಉರಿದು
ಉರಿ-ಯಿಂದ
ಉರಿ-ಯುತ್ತಿದೆ
ಉರಿ-ಯುತ್ತಿರ-ಬೇಕು
ಉರಿ-ಯುತ್ತಿರುವ
ಉರಿಯುತ್ತಿ-ರು-ವು-ದ-ರಿಂದ
ಉರಿವ
ಉರುಳಿ
ಉರುಳಿ-ಕೊಂಡ
ಉರುಳಿ-ದನಿ-ವನು
ಉರುಳಿ-ದ-ವನು
ಉರುಳಿ-ದುವು
ಉರುಳಿ-ಸ-ಲಾ-ಯಿತು
ಉರುಳಿ-ಸುತ್ತಿರೆ
ಉರುಳಿ-ಹೋ-ಗಲಿ
ಉರುಳುತುರುಳುತ
ಉರ್ಧ್ವ
ಉಲಿಯೈ
ಉಲು
ಉಲ್ಬಣ-ವಾಗಿ
ಉಲ್ಲ-ಘಿಸಿ
ಉಲ್ಲಾಸ-ದಿಂದ
ಉಲ್ಲಾಸ-ವಾಗಿ
ಉಲ್ಲಾಸ-ವಾ-ಗು-ವು-ದಿಲ್ಲವೇ
ಉಲ್ಲೇಖ-ವಿದೆ
ಉಲ್ಲೇಖಿಸಿ
ಉಲ್ಲೇಖಿ-ಸಿದ್ದು
ಉಳಿ
ಉಳಿ-ಗಾಲ-ವಿಲ್ಲ
ಉಳಿದ
ಉಳಿ-ದದ್ದು
ಉಳಿ-ದರು
ಉಳಿ-ದ-ವರ
ಉಳಿ-ದ-ವರು
ಉಳಿ-ದ-ವರೂ
ಉಳಿ-ದ-ವ-ರೆಲ್ಲ
ಉಳಿ-ದ-ವು-ಗಳಿ-ಗಿಂತ
ಉಳಿ-ದ-ವು-ಗಳೆಲ್ಲ
ಉಳಿ-ದ-ವೆಲ್ಲ
ಉಳಿ-ದ-ವೆಲ್ಲಾ
ಉಳಿ-ದಿದೆ
ಉಳಿ-ದಿದ್ದಾರೆ
ಉಳಿ-ದಿರುವ
ಉಳಿ-ದಿ-ರು-ವುದು
ಉಳಿ-ದಿ-ರು-ವುದೇನು
ಉಳಿದು
ಉಳಿ-ದು-ಕೊಂಡಿದೆ
ಉಳಿ-ದು-ಕೊಂಡಿದ್ದಳು
ಉಳಿ-ದು-ಕೊಳ್ಳುತ್ತದೆ
ಉಳಿ-ದು-ದರಲ್ಲೆಲ್ಲ
ಉಳಿ-ದು-ದೆಲ್ಲ
ಉಳಿ-ದು-ದೆಲ್ಲವು
ಉಳಿ-ದು-ದೆಲ್ಲಾ
ಉಳಿ-ದು-ಬಿ-ಡ-ಲಿಲ್ಲ
ಉಳಿ-ದುವು
ಉಳಿ-ದು-ವು-ಗಳೆಲ್ಲ
ಉಳಿ-ದು-ವೆಲ್ಲ
ಉಳಿ-ದು-ವೆಲ್ಲಾ
ಉಳಿ-ದೆರಡನ್ನು
ಉಳಿ-ಯ-ಲಾರದು
ಉಳಿ-ಯಲಿ
ಉಳಿ-ಯು-ವಂತೆ
ಉಳಿ-ಯುವನು
ಉಳಿ-ಯುವ-ರಲ್ಲವೆ
ಉಳಿ-ಯುವರೇ
ಉಳಿ-ಯು-ವು-ದಿಲ್ಲ
ಉಳಿ-ಯು-ವುದು
ಉಳಿ-ಯು-ವುದೇ
ಉಳಿ-ಯುವೆ
ಉಳಿವ
ಉಳಿ-ವು-ದೇನು
ಉಳಿ-ಸಿ-ಕೊಂಡು
ಉಳಿ-ಸಿ-ಕೊಳ್ಳ-ಬೇಕು
ಉಳಿ-ಸಿ-ಕೊಳ್ಳು-ವುದಕ್ಕೆ
ಉಳಿ-ಸುವಷ್ಟನ್ನು
ಉಳಿ-ಸು-ವುದಕ್ಕಾಗಿ
ಉಳ್ಳವು
ಉಷ್ಣಂಭಾರ
ಉಸಿರ-ನಾಡುವ
ಉಸಿರಾಡ-ಬೇಕು
ಉಸಿರಾಡಲವಕಾಶ
ಉಸಿರಾಡುವಿಕೆ
ಉಸಿರಾಡು-ವು-ದಕ್ಕೂ
ಉಸಿರಾಡು-ವುದೂ
ಉಸಿರು
ಉಸಿರು-ಬಿ-ಡದೆ
ಉಸಿರೇ
ಊಟ
ಊಟಕ್ಕಾಗಿ
ಊಟಕ್ಕೆ
ಊಟಕ್ಕೆಬ್ಬಿ-ಸುತ್ತಿದ್ದರು
ಊಟದ
ಊಟ-ದೊಂದಿಗೆ
ಊಟ-ಮಾಡಿ
ಊಟ-ಮಾಡುತ್ತ
ಊಟ-ಮಾಡು-ವಿರಾ
ಊಟ-ಮಾಡು-ವು-ದನ್ನು
ಊಟ-ವನ್ನರ್ಪಿಸಿದ್ದೇನೆ
ಊಟ-ವನ್ನು
ಊಟ-ವಾದ
ಊಟವೇ
ಊಟಿ-ಯಲ್ಲಿ
ಊಠ-ಭಾಸೆ
ಊದಿ-ಕೊಂಡಿವೆ
ಊದಿ-ದರೆ
ಊದಿದ್ದುವು
ಊದಿನ
ಊರಿಂದೂ-ರಿಗೆ
ಊರಿಗೆ
ಊರಿನ
ಊರಿನಾಚೆಯೇ
ಊರಿ-ನೊ-ಳಗೆ
ಊರು
ಊರು-ಗಳಲ್ಲಿಯೂ
ಊರು-ಗೋಲನ್ನಾ-ಗಿಟ್ಟು-ಕೊಳ್ಳ-ಬೇಡಿ
ಊರು-ಗೋ-ಲಿಲ್ಲ-ದಾಗ
ಊರೇ
ಊರ್ಜಿತ-ಗೊಳ್ಳುತ್ತಿದೆ
ಊರ್ಜಿತ-ವಾದ
ಊರ್ಮಿದಾಯ
ಊರ್ಮಿ-ಮಾಲಾ
ಊಹಿ-ಸದ
ಊಹಿಸ-ಬಲ್ಲರೆ
ಊಹಿ-ಸುವೆ-ಯೇನು
ಊಹೆ
ಊಹೆ-ಯಲ್ಲ
ಋಗ್ವೇದ
ಋಗ್ವೇದದ
ಋಗ್ವೇದ-ದಲ್ಲಿ
ಋಗ್ವೇದ-ವನ್ನು
ಋಣಿ-ಯಾಗಿ-ರ-ಬೇಕಿಲ್ಲ
ಋಣಿ-ಯಾಗಿ-ರಬೇಕೇ
ಋತಂ
ಋತ-ಪಥೇ
ಋದ್ಧಿ-ಗಳೆಲ್ಲಾ
ಋಷಿ
ಋಷಿ-ಗಳ
ಋಷಿ-ಗಳಂತೆ
ಋಷಿ-ಗಳನ್ನು
ಋಷಿ-ಗಳಾಗಿದ್ದಿರಿ
ಋಷಿ-ಗಳಿಗೆ
ಋಷಿ-ಗಳು
ಋಷಿಯ
ಋಷಿ-ಯಂತೆ
ಋಷಿಯು
ಋಷಿ-ರೂಪ-ಗಳನ್ನು
ಎ
ಎಂಎ
ಎಂಎ-ಗಳು
ಎಂಜ-ಲನ್ನು
ಎಂಟು
ಎಂಟು-ಗಂಟೆ-ಯ-ವ-ರೆಗೂ
ಎಂತಹ
ಎಂತ-ಹುದು
ಎಂತು
ಎಂಥ
ಎಂಥ-ವರು
ಎಂಥ-ವು-ಗಳೆಂದರೆ
ಎಂಥಾ
ಎಂಥಾದ್ದಯ್ಯಾ
ಎಂದ
ಎಂದನು
ಎಂದ-ಮೇಲೆ
ಎಂದ-ರಂತೆ
ಎಂದ-ರಲ್ಲದೆ
ಎಂದ-ರಹುತ್ತಾನೆ
ಎಂದ-ರಿತು
ಎಂದರು
ಎಂದರೆ
ಎಂದ-ರೇನು
ಎಂದ-ರೇ-ನೆಂಬುದು
ಎಂದರ್ಥ
ಎಂದರ್ಥ-ವಲ್ಲ
ಎಂದಲ್ಲವೆ
ಎಂದಷ್ಟೇ
ಎಂದಾಗ
ಎಂದಾ-ಗಲಿ
ಎಂದಾದರು
ಎಂದಾದರೂ
ಎಂದಾದ-ರೊಂದು
ಎಂದಿ-ಗಾ-ದರೂ
ಎಂದಿ-ಗಿಂತಲೂ
ಎಂದಿಗೂ
ಎಂದಿಗೆ
ಎಂದಿಟ್ಟು-ಕೊಳ್ಳೋಣ
ಎಂದಿತ್ತು
ಎಂದಿದ್ದಾನೆ
ಎಂದಿ-ನಂತೆಯೇ
ಎಂದಿನ-ವರೆಗೆ
ಎಂದಿ-ನಿಂದ
ಎಂದಿರಿ
ಎಂದು
ಎಂದುಕೊ
ಎಂದು-ಕೊಂಡನು
ಎಂದು-ಕೊಂಡು
ಎಂದು-ಕೊಂಡು-ಬಿಡುತ್ತಾರೆ
ಎಂದು-ಕೊಂಡೆ
ಎಂದು-ಕೊಳ್ಳುತ್ತಾರೆಯೋ
ಎಂದು-ಕೊಳ್ಳುತ್ತಿದ್ದರು
ಎಂದು-ಕೊಳ್ಳುತ್ತಿ-ರು-ವುದು
ಎಂದು-ಕೊಳ್ಳುತ್ತೇನೆ
ಎಂದು-ಕೊಳ್ಳುತ್ತೇವೆ
ಎಂದುತ್ತ-ರಿಸಿ-ದರು
ಎಂದೂ
ಎಂದೆ
ಎಂದೆಂದಿಗು
ಎಂದೆಂದಿಗೂ
ಎಂದೆಂದು
ಎಂದೆಂದೂ
ಎಂದೆ-ನಿ-ಸುತ್ತದೆ
ಎಂದೆ-ನಿ-ಸು-ವು-ದಿಲ್ಲವೆ
ಎಂದೆನ್ನಿಸಿತು
ಎಂದೆನ್ನಿ-ಸು-ವುದು
ಎಂದೆನ್ನು-ವುದ-ರೊಂದಿಗೆ
ಎಂದೆಲ್ಲಾ
ಎಂದೆವು
ಎಂದೇ
ಎಂದೋ
ಎಂಬ
ಎಂಬಂತಿತ್ತು
ಎಂಬಂತಿದೆ
ಎಂಬಂತಿದ್ದಾನೆ
ಎಂಬಂತಿವೆ
ಎಂಬಂತೆ
ಎಂಬಲ್ಲಿ
ಎಂಬಲ್ಲಿಗೆ
ಎಂಬಲ್ಲಿದ್ದಾಗ
ಎಂಬಷ್ಟೇ
ಎಂಬಾ-ಕೆಗೆ
ಎಂಬಿವು-ಗಳನ್ನೇ
ಎಂಬೀ
ಎಂಬು-ದಕ್ಕೆ
ಎಂಬುದ-ನೊಬ್ಬರೂ
ಎಂಬು-ದನ್ನು
ಎಂಬುದನ್ನೆಲ್ಲಾ
ಎಂಬುದನ್ನೇ
ಎಂಬು-ದರ
ಎಂಬು-ದ-ರಲ್ಲಿ
ಎಂಬು-ದ-ರಿಂದಲೇ
ಎಂಬು-ದಾಗಿ
ಎಂಬು-ದಾಗಿತ್ತು
ಎಂಬು-ದಿದೆ
ಎಂಬು-ದಿಲ್ಲ
ಎಂಬುದಿಷ್ಟು
ಎಂಬುದು
ಎಂಬುದುನ್ನು
ಎಂಬುದೂ
ಎಂಬುದೆ
ಎಂಬುದೇ
ಎಂಬು-ದೇನೋ
ಎಂಬು-ದೊಂದು
ಎಂಬು-ವನು
ಎಂಬು-ವುದು
ಎಂಬುವೇ
ಎಂಬೊಬ್ಬ-ನನ್ನು
ಎಕ್ಕಡ-ವನ್ನು
ಎಚ್ಎನ್ಮುರಳೀ-ಧರ
ಎಚ್ಚತ್ತವ-ನನ್ನು
ಎಚ್ಚರ-ಗೊಂಡು
ಎಚ್ಚರ-ಗೊಳಿಸ-ಬೇಕು
ಎಚ್ಚರ-ಗೊಳಿಸಿ
ಎಚ್ಚರ-ಗೊಳ್ಳಿ
ಎಚ್ಚರ-ಗೊಳ್ಳು
ಎಚ್ಚರ-ಗೊಳ್ಳು-ವು-ದಿಲ್ಲ
ಎಚ್ಚರ-ವಾಗು
ಎಚ್ಚ-ರಿಕೆ
ಎಚ್ಚ-ರಿಕೆ-ಯಿಂದ
ಎಚ್ಚ-ರಿಕೆ-ಯಿಂದ-ವರನು
ಎಚ್ಚ-ರಿಕೆ-ಯಿಂದಿರ-ಬೇಕು
ಎಚ್ಚರಿಸ-ಬ-ಹುದು
ಎಚ್ಚ-ರಿಸಿ-ದಿರಿ
ಎಚ್ಚರಿ-ಸುವನು
ಎಚ್ಚೆತ್ತಿರು-ವಾಗ
ಎಚ್ಚೆತ್ತು
ಎಚ್ಚೆತ್ತು-ಕೊಂಡ-ರೆಂದರೆ
ಎಚ್ಚೆತ್ತು-ಕೊಂಡು
ಎಚ್ಚೆತ್ತು-ಕೊಳ್ಳುತ್ತಿದ್ದಾರೆ
ಎಚ್ಚೆತ್ತು-ಕೊಳ್ಳುವರು
ಎಚ್ಚೆತ್ತು-ಬಿಡು-ವರೊ
ಎಟುಕ-ದಿ-ಹುದು
ಎಡ-ಗಣ್ಣಿನ
ಎಡ-ಗೈ-ಯಲ್ಲಿ
ಎಡಗೈ-ಯಲ್ಲಿದ್ದ
ಎಡ-ಭಾಗ-ದಲ್ಲಿ
ಎಡರು-ಗಳಿವೆ
ಎಡ-ವಿದರೆ
ಎಡವಿದ್ದಾರೆ
ಎಡವು-ತಲಿ
ಎಡೆ
ಎಡೆ-ಎಡೆ-ಯೊಳು
ಎಡೆಗೆ
ಎಡೆ-ಗೊಡ-ಬೇಕು
ಎಡೆ-ಗೊಡ-ಬೇಡ
ಎಡೆ-ಗೊಡು-ವುದು
ಎಡೆ-ಬಿಡ-ದಲೆ
ಎಡೆ-ಬಿ-ಡದಿ-ರದ
ಎಡೆ-ಬಿ-ಡದೆ
ಎಡೆ-ಯಿದೆ
ಎಡೆ-ಯಿಲ್ಲ
ಎಡೆಯೇ
ಎಡ್ವಿನ್
ಎಣಿಕೆ-ಯಿಂದ
ಎಣಿ-ಸ-ಲಾಗಿತ್ತು
ಎಣಿ-ಸಿಲ್ಲ
ಎಣಿ-ಸುತ್ತಾ
ಎಣಿ-ಸು-ವು-ದಿಲ್ಲ
ಎಣ್ಣೆ
ಎಣ್ಣೆಗೆ
ಎಣ್ಣೆಯ
ಎತ್ತನ್ನೋ
ಎತ್ತ-ಬೇಕು
ಎತ್ತ-ರಕ್ಕೆ
ಎತ್ತ-ರದಿ
ಎತ್ತರ-ದೊಂದಿಗೆ
ಎತ್ತರ-ವಾಗಿ
ಎತ್ತಲು
ಎತ್ತಿ
ಎತ್ತಿ-ಕಟ್ಟು-ವು-ದಿಲ್ಲ
ಎತ್ತಿ-ಕೊಂಡು
ಎತ್ತಿ-ತೋರು-ವಳು
ಎತ್ತಿ-ತೋರು-ವುದು
ಎತ್ತಿ-ದಂತಾಗಿದೆ
ಎತ್ತಿ-ದನು
ಎತ್ತಿ-ದರು
ಎತ್ತಿ-ದರೆ
ಎತ್ತಿ-ದಾಗ
ಎತ್ತಿದ್ದರು
ಎತ್ತಿದ್ದೇನೆ
ಎತ್ತಿ-ಬ-ರಲು
ಎತ್ತಿ-ಯಾ-ಡಿದರೆ
ಎತ್ತಿ-ಹಿಡಿ-ಯುವಿರಿ
ಎತ್ತು-ವಿರಿ
ಎತ್ತು-ವೆನು
ಎದುರಾಳಿ-ಗಳೂ
ಎದುರಿಗಿದ್ದು-ಕೊಂಡು
ಎದುರಿ-ಗಿ-ರುವ
ಎದು-ರಿಗೆ
ಎದುರಿನಲ್ಲಿಯೇ
ಎದುರಿಸ-ಬೇಕಾಗಿಲ್ಲ
ಎದುರಿ-ಸ-ಲಾ-ರರು
ಎದು-ರಿಸಿ
ಎದುರಿ-ಸುವ
ಎದುರಿ-ಸು-ವುದು
ಎದುರಿಸು-ವು-ದೆಂದು
ಎದುರು
ಎದುರು-ಗೊಂಡು
ಎದುರು-ನೋಡುತ್ತಿದ್ದಾರೆ
ಎದುರು-ಪಾರ್ಶ್ವ-ದಲ್ಲಿ
ಎದುರು-ಬೀಳುವಿಕೆ
ಎದೆ
ಎದೆ-ಕರಗಿ
ಎದೆ-ಗಾ-ರಿಕೆಯ
ಎದೆ-ಗಾ-ರಿಕೆಯಿಂದ
ಎದೆ-ಗಾ-ರಿಕೆಯಿಲ್ಲ
ಎದೆ-ಗುಂದದೆ
ಎದೆ-ಗುಂದು-ವು-ದಿಲ್ಲವೆ
ಎದೆಗೆ
ಎದೆ-ಗೆ-ಡದೆ
ಎದೆ-ಗೆ-ಡ-ಬೇಡ
ಎದೆಯ
ಎದೆ-ಯನ್ನು
ಎದೆ-ಯಲ್ಲಿ
ಎದೆ-ಯಲ್ಲೇ
ಎದೆ-ಯಾ-ಳದ
ಎದೆಯು
ಎದೆ-ಯೊಡೆ-ದಂತಾಗಿ
ಎದೆ-ಯೊಳು-ಳಿ-ದಿಹು-ದಿನ್ನು
ಎದ್ದರು
ಎದ್ದಲಾ-ಗಾಯ್ತು
ಎದ್ದಾಗ
ಎದ್ದು
ಎದ್ದೇಳಿ
ಎದ್ದೇಳು
ಎನಗೆ
ಎನಲು
ಎನಿತೆನಿತು
ಎನಿತೊ
ಎನುಲು
ಎನುವೆ
ಎನ್
ಎನ್ನ
ಎನ್ನ-ತೊಡಗಿ-ದರು
ಎನ್ನದೆ
ಎನ್ನನು
ಎನ್ನ-ಬ-ಹುದು
ಎನ್ನ-ಲಾಗುತ್ತಿ-ರ-ಲಿಲ್ಲ
ಎನ್ನ-ಲಾ-ಗು-ವು-ದಿಲ್ಲ
ಎನ್ನಲು
ಎನ್ನಿಂದೆಲ್ಲ
ಎನ್ನಿಸಿ
ಎನ್ನಿಸಿ-ಕೊಳ್ಳುವ
ಎನ್ನಿಸಿ-ಕೊಳ್ಳುವುದೋ
ಎನ್ನಿಸಿತು
ಎನ್ನಿಸಿದೆ
ಎನ್ನಿ-ಸುತ್ತದೆ
ಎನ್ನಿ-ಸುತ್ತಿದೆ
ಎನ್ನಿ-ಸು-ವುದು
ಎನ್ನಿ-ಸು-ವುದೋ
ಎನ್ನು
ಎನ್ನುತ್ತ
ಎನ್ನುತ್ತದೆ
ಎನ್ನುತ್ತವೆ
ಎನ್ನುತ್ತಾ
ಎನ್ನುತ್ತಾನೆ
ಎನ್ನುತ್ತಾರೆ
ಎನ್ನುತ್ತಿದ್ದ
ಎನ್ನುತ್ತಿದ್ದರು
ಎನ್ನುತ್ತಿದ್ದರೋ
ಎನ್ನುತ್ತಿದ್ದಾರೆಯೋ
ಎನ್ನುತ್ತಿದ್ದು-ದನ್ನು
ಎನ್ನುತ್ತೇನೆ
ಎನ್ನುತ್ತೇವೆ
ಎನ್ನುತ್ತೇವೆಯೋ
ಎನ್ನುವ
ಎನ್ನು-ವನು
ಎನ್ನು-ವರು
ಎನ್ನು-ವಳು
ಎನ್ನು-ವವರು
ಎನ್ನು-ವಷ್ಟರಮಟ್ಟಿಗಾ-ಗಿದೆ
ಎನ್ನು-ವ-ಹಾಗೆ
ಎನ್ನು-ವಾಗ
ಎನ್ನು-ವಿರಾ
ಎನ್ನು-ವಿರಿ
ಎನ್ನು-ವಿರೋ
ಎನ್ನು-ವುದಂತೂ
ಎನ್ನು-ವುದಕ್ಕೆ
ಎನ್ನು-ವು-ದನ್ನು
ಎನ್ನು-ವು-ದ-ರಿಂದ
ಎನ್ನು-ವು-ದಾದರೂ
ಎನ್ನು-ವು-ದಿಲ್ಲ
ಎನ್ನು-ವುದು
ಎನ್ನು-ವುದೇ
ಎನ್ನು-ವು-ದೊಂದು
ಎನ್ನು-ವುವು
ಎನ್ನು-ವೆವು
ಎನ್ಸೈಕ್ಲೋಪೀಡಿಯಾ
ಎಬ್ಬಿಸ-ಬಲ್ಲೆ-ನಾದರೆ
ಎಬ್ಬಿಸ-ಬೇಕೆಂದು
ಎಬ್ಬಿ-ಸ-ಬೇಕೆಂಬುದು
ಎಬ್ಬಿಸ-ಲಾರೆ-ನೇನು
ಎಬ್ಬಿ-ಸಲು
ಎಬ್ಬಿಸಿ-ದಳೆಂದರೆ
ಎಬ್ಬಿ-ಸಿದ್ದು
ಎಬ್ಬಿಸು
ಎಬ್ಬಿ-ಸುತ್ತಾ
ಎಬ್ಬಿಸು-ವಂತೆ
ಎಬ್ಬಿಸು-ವುದಕ್ಕೆ
ಎಯಿರಬ
ಎರಕ-ದಲ್ಲಿವೆ
ಎರಗು-ತಿ-ರಲು
ಎರಗುತಿ-ಹೆನು
ಎರಡಂತಸ್ತಿನ
ಎರಡಕ್ಕೂ
ಎರ-ಡನೆ
ಎರ-ಡನೆಯ
ಎರ-ಡನೆ-ಯ-ದ-ರಲ್ಲಿ
ಎರ-ಡನೆ-ಯ-ದಾಗಿ
ಎರ-ಡನೆ-ಯ-ದಿಲ್ಲದ
ಎರ-ಡನೆ-ಯದು
ಎರ-ಡನೆ-ಯದೇ
ಎರಡನೇ
ಎರಡನ್ನೂ
ಎರಡರ
ಎರಡ-ರಂದು
ಎರಡ-ರಲ್ಲೂ
ಎರಡ-ರಷ್ಟು
ಎರಡಲ್ಲವು
ಎರಡಾಣೆ
ಎರಡಿಲ್ಲದ
ಎರಡು
ಎರಡೂ
ಎರಡೂ-ವರೆ
ಎರಡೇ
ಎರೆ
ಎರೆದು
ಎಲೆ
ಎಲೆಯ
ಎಲೆ-ಯೊ-ಳಗೆ
ಎಲೆ-ವನೆ
ಎಲೈ
ಎಲ್ಲ
ಎಲ್ಲಕ್ಕಿಂತ
ಎಲ್ಲಕ್ಕಿಂತಲೂ
ಎಲ್ಲಕ್ಕೂ
ಎಲ್ಲ-ದಕು
ಎಲ್ಲ-ದಕ್ಕೂ
ಎಲ್ಲ-ದರ
ಎಲ್ಲ-ದ-ರಲಿ
ಎಲ್ಲ-ದರಲ್ಲಿಯೂ
ಎಲ್ಲ-ದ-ರಲ್ಲೂ
ಎಲ್ಲರ
ಎಲ್ಲ-ರನ್ನು
ಎಲ್ಲ-ರನ್ನೂ
ಎಲ್ಲ-ರ-ಮೇಲೂ
ಎಲ್ಲ-ರಲ್ಲಿಯೂ
ಎಲ್ಲ-ರಲ್ಲಿ-ರುವ
ಎಲ್ಲ-ರಲ್ಲೂ
ಎಲ್ಲ-ರ-ಳಿದು
ಎಲ್ಲ-ರಿಂದಲೂ
ಎಲ್ಲ-ರಿ-ಗಿಂತ
ಎಲ್ಲ-ರಿ-ಗಿಂತಲೂ
ಎಲ್ಲ-ರಿಗೂ
ಎಲ್ಲರೂ
ಎಲ್ಲ-ರೆದು-ರಿಗೂ
ಎಲ್ಲ-ರೊಂದಿಗೂ
ಎಲ್ಲ-ರೊಡ-ನೆಯೂ
ಎಲ್ಲ-ರೊಡ-ಲಾಗಿ-ರುವನೊ
ಎಲ್ಲ-ಲೆಯು-ತಿದ್ದರೂ
ಎಲ್ಲವ
ಎಲ್ಲ-ವನು
ಎಲ್ಲ-ವನ್ನು
ಎಲ್ಲ-ವನ್ನೂ
ಎಲ್ಲವು
ಎಲ್ಲವೂ
ಎಲ್ಲಾ
ಎಲ್ಲಾ-ದರೂ
ಎಲ್ಲಿ
ಎಲ್ಲಿಂದ
ಎಲ್ಲಿಂದಲೋ
ಎಲ್ಲಿಗೆ
ಎಲ್ಲಿಗೊ
ಎಲ್ಲಿಗೋ
ಎಲ್ಲಿದೆ
ಎಲ್ಲಿ-ದೆ-ಯೆಂದು
ಎಲ್ಲಿದ್ದರೂ
ಎಲ್ಲಿದ್ದಾನೆ
ಎಲ್ಲಿದ್ದಾರೆ
ಎಲ್ಲಿ-ಯದು
ಎಲ್ಲಿ-ಯ-ವರೆಗೆ
ಎಲ್ಲಿ-ಯ-ವರೆ-ವಿಗೂ
ಎಲ್ಲಿ-ಯವ-ರೆ-ವಿಗೆ
ಎಲ್ಲಿ-ಯಾದರೂ
ಎಲ್ಲಿಯೂ
ಎಲ್ಲಿಯೋ
ಎಲ್ಲಿ-ರುತ್ತೀರಿ
ಎಲ್ಲಿ-ರು-ವು-ದೆಂದು
ಎಲ್ಲಿವೆ
ಎಲ್ಲೆಂದ-ರಲ್ಲಿ
ಎಲ್ಲೆ-ಕಟ್ಟನ್ನೆಲ್ಲಾ
ಎಲ್ಲೆ-ಗಳಲ್ಲಿ
ಎಲ್ಲೆಡೆ-ಗಳಲೂ
ಎಲ್ಲೆಡೆಗೂ
ಎಲ್ಲೆಡೆ-ಯಲ್ಲಿಯೂ
ಎಲ್ಲೆಡೆ-ಯಲ್ಲೂ
ಎಲ್ಲೆ-ಯನ್ನು
ಎಲ್ಲೆ-ಯಲ್ಲಿ-ಡದೆ
ಎಲ್ಲೆ-ಯಿಂದ
ಎಲ್ಲೆ-ಯಿಲ್ಲ
ಎಲ್ಲೆಯು
ಎಲ್ಲೆಯುಂಟು
ಎಲ್ಲೆಯೇ
ಎಲ್ಲೆ-ಯೊ-ಳಗೆ
ಎಲ್ಲೆಲ್ಲಿ
ಎಲ್ಲೆಲ್ಲಿಯೂ
ಎಲ್ಲೆಲ್ಲು
ಎಲ್ಲೆಲ್ಲೂ
ಎಲ್ಲೊ
ಎಲ್ಲೋ
ಎಳ-ಸಿದೆ
ಎಳೆಎಳೆ-ಗಳಲ್ಲಿಯೂ
ಎಳೆದ
ಎಳೆದಾಡು-ವು-ದ-ರಿಂದ
ಎಳೆದಿದ್ದಾನೋ
ಎಳೆ-ದಿಲ್ಲ
ಎಳೆದು
ಎಳೆ-ಮಕ್ಕ-ಳನ್ನು
ಎಳೆ-ಯು-ವು-ದ-ರಿಂದ
ಎಳ್ಳನಿತು
ಎಳ್ಳು
ಎವೆಯಿಕ್ಕದ
ಎವೆಯಿಕ್ಕದೆ
ಎಷ್ಟರ
ಎಷ್ಟರ-ಮಟ್ಟಿಗಿ-ದೆಯೋ
ಎಷ್ಟರ-ಮಟ್ಟಿಗೆ
ಎಷ್ಟಾಗು-ವುದೊ
ಎಷ್ಟಾ-ದರೂ
ಎಷ್ಟಿ-ವೆಯೋ
ಎಷ್ಟು
ಎಷ್ಟು-ದಿನ
ಎಷ್ಟೆಷ್ಟನ್ನು
ಎಷ್ಟೆಷ್ಟಿವೆ
ಎಷ್ಟೆಷ್ಟು
ಎಷ್ಟೆಷ್ಟೋ
ಎಷ್ಟೇ
ಎಷ್ಟೊ
ಎಷ್ಟೊಂದು
ಎಷ್ಟೋ
ಎಸೆ-ದರೆ
ಎಸೆ-ದಿ-ರಲಿ
ಎಸೆಯ-ಬೇಕಾಗಿದೆ
ಎಸೆ-ಯಿತು
ಎಸೆ-ಯಿರಿ
ಎಸ್
ಎಸ್ಸಿನ್ಸ್
ಏ
ಏಕ
ಏಕ-ಕಾಲ-ದಲ್ಲಿ
ಏಕ-ಕಾಲ-ದಲ್ಲಿಯೇ
ಏಕ-ಕಾಲ-ದಲ್ಲೇ
ಏಕ-ತಾ-ಭಾವ-ನೆ-ಯನ್ನು
ಏಕತೆ
ಏಕತೆಯ
ಏಕತೆ-ಯನ್ನು
ಏಕತೆ-ಯಲ್ಲಿ
ಏಕತೆ-ಯಿದೆ
ಏಕ-ತೆಯೇ
ಏಕತ್ವ
ಏಕತ್ವದ
ಏಕತ್ವ-ಮನು-ಪಶ್ಯತಃ
ಏಕತ್ವ-ವನ್ನು
ಏಕ-ದರ್ಶಿನಿ
ಏಕಪ್ರ-ಕಾರ-ವಾಗಿ
ಏಕಪ್ರ-ಕಾರ-ವಾಗಿದೆ
ಏಕಪ್ರ-ಕಾರ-ವಾದ
ಏಕ-ಮನಸ್ಕ-ರಾದ-ರೆಂದು
ಏಕ-ಮಾತ್ರ
ಏಕ-ಮೇ-ವಾದ್ವಯಂ
ಏಕ-ಮೇ-ವಾದ್ವಿ-ತೀಯ
ಏಕ-ಮೇ-ವಾದ್ವಿ-ತೀಯಂ
ಏಕ-ರೂಪ
ಏಕ-ವನು
ಏಕ-ವಾಗಿ
ಏಕ-ವಾಗಿಹ
ಏಕ-ವಾಗು-ವುವೋ
ಏಕ-ವಾದ
ಏಕವೇ
ಏಕಾ
ಏಕಾಂಗಿ
ಏಕಾಂಗಿ-ಯಾಗಿ
ಏಕಾಂಗಿ-ಯಾಗಿ-ರಲಿ
ಏಕಾಂತ-ದಲ್ಲಿ
ಏಕಾಂತ-ವಾಗಿ
ಏಕಾಂತವೂ
ಏಕಾ-ಕಾರ
ಏಕಾಕಿ
ಏಕಾ-ಗ-ಬಾ-ರದು
ಏಕಾಗ್ರ
ಏಕಾಗ್ರ-ಗೊಂಡಿ-ರುತ್ತದೆ
ಏಕಾಗ್ರ-ಗೊಳ್ಳುತ್ತಿದೆ
ಏಕಾಗ್ರತೆ
ಏಕಾಗ್ರ-ತೆಯ
ಏಕಾಗ್ರ-ತೆ-ಯಿಂದ
ಏಕಾಗ್ರ-ತೆ-ಯಿಂದಾಗಿ
ಏಕಾಗ್ರ-ತೆ-ಯಿದ್ದರೆ
ಏಕಾಗ್ರ-ತೆ-ಯುಂಟಾ-ದರೂ
ಏಕಾಗ್ರ-ನಿಷ್ಠ
ಏಕಾಗ್ರ-ಮಾಡು-ವುದಕ್ಕೆ
ಏಕಾಗ್ರ-ವಾಗಲು
ಏಕಾ-ಸನ-ದಲ್ಲಿ
ಏಕಿರ-ಬಾ-ರದು
ಏಕೀ
ಏಕೆ
ಏಕೆಂದರೆ
ಏಕೇಂದ್ರಿಯ-ವುಳ್ಳವು
ಏಕೈಕ
ಏಕೊ
ಏಕೋ
ಏಕ್
ಏಟಿಗೆ
ಏಟು
ಏತಕ್ಕೆ
ಏತಕ್ಕೋಸ್ಕರ
ಏತ-ರಿಂದ
ಏನನೆಸಗುವ-ನಿಲ್ಲಿ
ಏನನ್ನಾದರೂ
ಏನನ್ನು
ಏನನ್ನೂ
ಏನನ್ನೇ
ಏನನ್ನೊ
ಏನನ್ನೋ
ಏನಯ್ಯಾ
ಏನಾಗಿದೆ
ಏನಾಗಿ-ದೆ-ಯೆಂಬು-ದನ್ನು
ಏನಾಗಿ-ದೆಯೋ
ಏನಾಗಿದ್ದಾ-ನೆಂದು
ಏನಾಗಿದ್ದೇನೋ
ಏನಾಗಿ-ರು-ವೆನೋ
ಏನಾ-ಗುತ್ತದೆ
ಏನಾ-ಗು-ವು-ದಿಲ್ಲ
ಏನಾಗು-ವುದು
ಏನಾಗು-ವುದೊ
ಏನಾದ
ಏನಾದರೂ
ಏನಾದ-ರೊಂದು
ಏನಾದ-ಹಾ-ಗಾ-ಯಿತು
ಏನಿತ್ತು
ಏನಿದೆ
ಏನಿದ್ದರೂ
ಏನು
ಏನು-ತಾನೆ
ಏನು-ಮಾಡ-ಬಹು-ದೆಂದರೆ
ಏನು-ಮಾಡಲಿ
ಏನು-ಮಾಡಲೂ
ಏನು-ಳಿದಿ-ದೆಯೊ
ಏನೂ
ಏನೆಂದರೆ
ಏನೆಂದು
ಏನೆಂದು-ಕೊಂಡಿದ್ದೀಯೆ
ಏನೆಂದೆ
ಏನೆಂಬು-ದನ್ನು
ಏನೆಂಬುದೇ
ಏನೆನ್ನ-ಬೇಕು
ಏನೆನ್ನುತ್ತೀಯೆ
ಏನೇ
ಏನೇನು
ಏನೇನೂ
ಏನೇನೋ
ಏನೊ
ಏನೊಂದು
ಏನೊಂದೂ
ಏನೋ
ಏನ್ರೀ
ಏಪ್ರಿಲ್
ಏಬಾರ
ಏಯಿ
ಏಯಿ-ಜ-ಗತ
ಏರಿದ
ಏರಿಳಿತ-ಗಳಿಂದ
ಏರಿಸಿದ
ಏರಿ-ಸಿ-ದರು
ಏರಿ-ಸಿದ್ದರೆ
ಏರು-ವುದು
ಏರ್ಪಡಿ-ಸ-ಲಾಗಿತ್ತು
ಏರ್ಪಡಿ-ಸಿದ
ಏರ್ಪಡಿ-ಸು-ವುದು
ಏರ್ಪಾಟು
ಏರ್ಪಾಡನ್ನು
ಏರ್ಪಾಡಾ-ಗಿದೆ
ಏರ್ಪಾಡಾದ
ಏರ್ಪಾಡು
ಏಳ-ದಿ-ರಲು
ಏಳ-ಬ-ಹುದು
ಏಳ-ಬೇಕು
ಏಳ-ಲಾರದೆ
ಏಳಲು
ಏಳಿ
ಏಳಿ-ಗೆ-ಗಾಗಿ
ಏಳಿ-ಗೆಗೆ
ಏಳಿ-ಗೆಗೇನೂ
ಏಳು
ಏಳುತ್ತಿತ್ತು
ಏಳುತ್ತಿದ್ದರು
ಏಳುವ
ಏಳು-ವರೆ
ಏಳು-ವು-ದಿಲ್ಲ
ಏಳು-ವುದು
ಏಳ್ಗೆ
ಏಳ್ಗೆಗೆ
ಏಷಿಯಾ-ಖಂಡ-ದ-ವರು
ಏಷ್ಯಾ
ಏಷ್ಯಾ-ಖಂಡ-ದ-ವನ
ಏಷ್ಯಾ-ಖಂಡವೇ
ಏಷ್ಯಾದ
ಏಷ್ಯಾ-ದಲ್ಲಿ
ಏಷ್ಯಾ-ದ-ವನೆಂದಿಗೂ
ಏಸು
ಏಸು-ವಿನ
ಏಸು-ವಿ-ನಿಂದ
ಏಸೇಛೆ
ಐಕ-ಮತ್ಯ-ವೆಂಬ
ಐಕ್ಯ
ಐಕ್ಯ-ಗೊಳ್ಳುವ-ವ-ರೆಗೂ
ಐಕ್ಯ-ನಾಗಿದ್ದರೆ
ಐಕ್ಯ-ರಾದ-ವ-ರೆಂದೂ
ಐಕ್ಯ-ವಾಗಿದ್ದೆವು
ಐಕ್ಯ-ವಾಗು-ವುದನ್ನೆ
ಐಕ್ಯ-ವಾಗು-ವುದು
ಐಕ್ಯ-ವಾಗು-ವುವು
ಐಕ್ಯ-ವಾಗುವೆ
ಐತಿ-ಹಾಸಿಕ
ಐತಿ-ಹಾಸಿ-ಕ-ತೆ-ಯಿದೆ
ಐತಿ-ಹಾಸಿ-ಕ-ವಾಗಿ
ಐದಾರು
ಐದು
ಐದು-ಜನ
ಐನೂರು
ಐಪಡೆ
ಐರೋಪ್ಯ
ಐರೋಪ್ಯರ
ಐರೋಪ್ಯ-ರಾಗಿಲ್ಲ
ಐರೋಪ್ಯ-ರಿಂದ
ಐರೋಪ್ಯರು
ಐರೋಪ್ಯರೂ
ಐರೋಪ್ಯ-ಶೈಲಿಯ
ಐಲೆಂಡ್
ಐಲೋಕೇಶೀ
ಐವತ್ತು
ಐವತ್ತೆರ-ಡ-ನೆಯ
ಐವ-ರಲ್ಲಿ
ಐಶ್ವರ್ಯ
ಐಶ್ವರ್ಯ-ಅಧಿಕಾರ
ಐಶ್ವರ್ಯ-ಅಧಿಕಾರ-ಗಳ
ಐಶ್ವರ್ಯಕ್ಕೆ
ಐಶ್ವರ್ಯ-ಗಳಲ್ಲಿ
ಐಶ್ವರ್ಯ-ದಿಂದಲ್ಲ
ಐಶ್ವರ್ಯ-ವನ್ನು
ಐಶ್ವರ್ಯವು
ಐಸ್
ಐಹಿಕ
ಒಂಟಿ
ಒಂದಂತಸ್ತಿನ
ಒಂದಕ್ಕೆ
ಒಂದಕ್ಕೊಂದು
ಒಂದಕ್ಷ-ರವೂ
ಒಂದನ್ನಾದರೂ
ಒಂದನ್ನು
ಒಂದನ್ನೊಂದು
ಒಂದ-ರಂತೆ
ಒಂದ-ರಲ್ಲಿ
ಒಂದರೊ-ಡ-ನೊಂದೆಂದು
ಒಂದಲ್ಲ
ಒಂದಲ್ಲ-ದಿದ್ದರೆ
ಒಂದಾಗದೇ
ಒಂದಾಗಿ
ಒಂದಾಗಿ-ಬಿಡುತ್ತದೆ
ಒಂದಾಗಿ-ರು-ವುದು
ಒಂದಾಗಿವೆ
ಒಂದಾಗು
ಒಂದಾಗು-ವುವೋ
ಒಂದಾದ
ಒಂದಾದ-ಮೇಲೆ
ಒಂದಾದ-ಮೇಲೊಂದ-ರಂತೆ
ಒಂದಾದ-ಮೇಲೊಂದು
ಒಂದಾದರೂ
ಒಂದಾನೊಂದು
ಒಂದಾ-ಯಿತು
ಒಂದಿನಿತಿಲ್ಲ
ಒಂದಿ-ರ-ಲಾರು
ಒಂದಿರುಳು
ಒಂದಿಷ್ಟು
ಒಂದು
ಒಂದು-ಕಡೆ
ಒಂದು-ಕಾಲ
ಒಂದು-ಗೂ-ಡಲು
ಒಂದು-ಗೂ-ಡಿದಾಗ
ಒಂದು-ಗೂಡಿಸ-ಬೇಕೆನ್ನಿ-ಸು-ವುದು
ಒಂದು-ಗೂಡಿ-ಸಲು
ಒಂದು-ಗೂಡಿಸಿ
ಒಂದು-ಗೂಡು-ವುವು
ಒಂದು-ದಿನ
ಒಂದು-ಭಾಗ
ಒಂದು-ಮಾಡಿ
ಒಂದು-ವೇಳೆ
ಒಂದೂ-ಕಾಲು
ಒಂದೂ-ವರೆ
ಒಂದೆ
ಒಂದೆಡೆ
ಒಂದೆ-ಡೆ-ಯಿಟ್ಟು
ಒಂದೆ-ಯಾ-ಗಿ-ರುವಾ-ತನು
ಒಂದೆಯೇ
ಒಂದೆ-ರಡು
ಒಂದೇ
ಒಂದೇ-ಏಕ
ಒಂದೊಂದನಪ್ಪಳಿಸಿ
ಒಂದೊಂದಾಗಿ
ಒಂದೊಂದು
ಒಂದೊಂದು-ಸಾರಿ
ಒಂದೊಂದೂ
ಒಂದೊಂದೇ
ಒಂಬತ್ತು
ಒಗೆದು
ಒಗೆಯಲು
ಒಗ್ಗದೆ
ಒಗ್ಗಿ-ದರೆ
ಒಗ್ಗುತ್ತದೆ
ಒಗ್ಗು-ವಂತಹ
ಒಗ್ಗುವು-ವೆಂದು
ಒಟ್ಟಾರೆ
ಒಟ್ಟಿಗೆ
ಒಟ್ಟಿ-ಗೆಯೇ
ಒಟ್ಟಿಗೇ-ನಾದರೂ
ಒಟ್ಟಿನ
ಒಟ್ಟಿ-ನಲ್ಲಿ
ಒಟ್ಟು
ಒಟ್ಟು-ಗೂಡಿ-ಸು-ವುದು
ಒಡಂಬಡಿ-ಕೆಗೂ
ಒಡಂಬಡಿಕೆ-ಯಂತೆ
ಒಡ-ನಾಡಿ-ಗಳು
ಒಡ-ನಾಡಿ-ಯಾಗಿದ್ದಂಥ-ವರ
ಒಡ-ನೆಯೇ
ಒಡ-ಲಿನೇ-ಕ-ವನು
ಒಡೆದು
ಒಡೆದು-ಕೊಂಡು
ಒಡೆದು-ಹಾಕಿ
ಒಡೆಯ-ನಾದರೂ
ಒಡೆಯ-ನಿಲ್ಲದ
ಒಣ
ಒಣ-ಕಲಾ-ಗಿ-ರುವ
ಒಣಗಿ
ಒಣಗಿ-ಕೊಂಡೆ
ಒಣಗಿ-ಸಿದ
ಒಣ-ಮುಖ
ಒತ್ತಟ್ಟಿ-ಗಿಟ್ಟು
ಒತ್ತಡ-ದಿಂದ
ಒತ್ತರಿ-ಸು-ವುದಕ್ಕೆ
ಒತ್ತಲು
ಒತ್ತಾಯದ
ಒತ್ತಾಯ-ದಿಂದ
ಒತ್ತಾಯ-ಮಾ-ಡಲು
ಒತ್ತಿ
ಒತ್ತು
ಒತ್ತುತ್ತಿದ್ದನು
ಒತ್ತುತ್ತಿದ್ದರು
ಒತ್ತೆ-ಯಿಟ್ಟು
ಒದಗಿತು
ಒದಗಿದ
ಒದಗಿ-ದರೆ
ಒದಗಿಸ-ಬೇಕೆಂದು
ಒದಗಿ-ಸ-ಲಾರದೆ
ಒದಗಿಸಿ-ಕೊಟ್ಟಿದ್ದೇನೆ
ಒದಗಿಸಿ-ಕೊಡ-ಬೇಕು
ಒದಗಿಸಿ-ಕೊಡುತ್ತ
ಒದಗಿಸಿ-ಕೊಳ್ಳುವಂತೆ
ಒದಗಿಸಿ-ಕೊಳ್ಳುವುದ-ರಲ್ಲಿ
ಒದಗಿಸಿದ್ದೇ-ನೆಂದರೆ
ಒದಗಿ-ಸುತ್ತವೆ
ಒದಗಿ-ಸುವರೋ
ಒದಗಿ-ಸು-ವು-ದಿಲ್ಲವೋ
ಒದ-ಗು-ವು-ದಿಲ್ಲ
ಒದರಿ-ದರು
ಒದೆತ-ಗಳನ್ನು
ಒದೆದಾಟ
ಒದ್ದಾಡುತ್ತಾ
ಒದ್ದಾಡುತ್ತಿದೆ
ಒದ್ದಾಡುತ್ತಿದ್ದೇನೆ
ಒದ್ದಾಡು-ವುದು
ಒದ್ದಾಡು-ವುದೇಕೆ
ಒಪ್ಪ-ದಿದ್ದರೆ
ಒಪ್ಪದೆ
ಒಪ್ಪಬೇ-ಕಾ-ಯಿತು
ಒಪ್ಪ-ಲಿಲ್ಲ
ಒಪ್ಪಿ
ಒಪ್ಪಿಕೊ
ಒಪ್ಪಿ-ಕೊಂಡ
ಒಪ್ಪಿ-ಕೊಂಡರು
ಒಪ್ಪಿ-ಕೊಂಡರೂ
ಒಪ್ಪಿ-ಕೊಂಡರೆ
ಒಪ್ಪಿ-ಕೊಂಡಾಗ
ಒಪ್ಪಿ-ಕೊಂಡಿರುವೆ
ಒಪ್ಪಿ-ಕೊಂಡು
ಒಪ್ಪಿ-ಕೊಂಡೆನು
ಒಪ್ಪಿ-ಕೊಂಡೇ
ಒಪ್ಪಿ-ಕೊಳ್ಳ
ಒಪ್ಪಿ-ಕೊಳ್ಳ-ದಿದ್ದಲ್ಲಿ
ಒಪ್ಪಿ-ಕೊಳ್ಳ-ಬೇ-ಕಲ್ಲದೆ
ಒಪ್ಪಿ-ಕೊಳ್ಳ-ಬೇಕು
ಒಪ್ಪಿ-ಕೊಳ್ಳ-ಲಿಲ್ಲ
ಒಪ್ಪಿ-ಕೊಳ್ಳ-ಲೇಬೇ-ಕಾ-ಯಿತು
ಒಪ್ಪಿ-ಕೊಳ್ಳುತ್ತದೆ
ಒಪ್ಪಿ-ಕೊಳ್ಳುವಂತೆ
ಒಪ್ಪಿ-ಕೊಳ್ಳುವರು
ಒಪ್ಪಿ-ಕೊಳ್ಳುವಿರಾ
ಒಪ್ಪಿ-ಕೊಳ್ಳುವುದು
ಒಪ್ಪಿ-ಕೊಳ್ಳೋಣ
ಒಪ್ಪಿ-ಗೆಯ
ಒಪ್ಪಿ-ಗೆ-ಯಾಗಿ
ಒಪ್ಪಿ-ಗೆ-ಯಾಗುವ
ಒಪ್ಪಿ-ದರೆ
ಒಪ್ಪಿದ್ದರೆ
ಒಪ್ಪಿದ್ದಾರೆ
ಒಪ್ಪಿ-ಸಿದ
ಒಪ್ಪಿ-ಸಿ-ದಳು
ಒಪ್ಪಿಸು
ಒಪ್ಪುತ್ತಾರೆ
ಒಪ್ಪುತ್ತಿದ್ದರು
ಒಪ್ಪುತ್ತಿ-ರ-ಲಿಲ್ಲ
ಒಪ್ಪುತ್ತೇನೆ
ಒಪ್ಪುವ
ಒಪ್ಪು-ವು-ದಿಲ್ಲ
ಒಬ್ಬ
ಒಬ್ಬನ
ಒಬ್ಬ-ನನ್ನು
ಒಬ್ಬ-ನಲ್ಲದೇ
ಒಬ್ಬ-ನಿಗೆ
ಒಬ್ಬನು
ಒಬ್ಬನೆ
ಒಬ್ಬ-ನೆಂದನು
ಒಬ್ಬನೇ
ಒಬ್ಬರ
ಒಬ್ಬ-ರನ್ನೊಬ್ಬರ
ಒಬ್ಬ-ರನ್ನೊಬ್ಬರು
ಒಬ್ಬ-ರಲ್ಲಿ
ಒಬ್ಬ-ರಾಗುತ್ತಲೊಬ್ಬರು
ಒಬ್ಬ-ರಾದರೂ
ಒಬ್ಬ-ರಿಗೆ
ಒಬ್ಬರು
ಒಬ್ಬರೇ
ಒಬ್ಬ-ರೊಡ-ನೊಬ್ಬರು
ಒಬ್ಬ-ಳನ್ನು
ಒಬ್ಬಾತ
ಒಬ್ಬಿಬ್ಬರು
ಒಬ್ಬೊಬ್ಬನೂ
ಒಬ್ಬೊಬ್ಬರ
ಒಬ್ಬೊಬ್ಬರಲ್ಲಿಯೂ
ಒಬ್ಬೊಬ್ಬ-ರಿಗೆ
ಒಬ್ಬೊಬ್ಬರೂ
ಒಮೊಮ್ಮೆ
ಒಮ್ಮತ-ವಿರು-ವುದು
ಒಮ್ಮೆ
ಒಮ್ಮೆಯೂ
ಒಮ್ಮೊಮ್ಮೆ
ಒಯ್
ಒಯ್ದರೂ
ಒಯ್ಯುತ್ತಲೂ
ಒಯ್ಯುತ್ತವೆ
ಒಯ್ಯುವ
ಒಯ್ಯುವನು
ಒಯ್ಯುವರು
ಒಯ್ಯು-ವುದು
ಒಯ್ಯು-ವುದೆ
ಒಯ್ಯುವೆ-ನೆಂದು
ಒಯ್ವ
ಒರಗಿ
ಒರಗಿ-ಕೊಂಡು
ಒರಲಿ-ಡುತ್ತಿದ್ದಾರೋ
ಒರಿಸ್ಸಾ
ಒರೆ
ಒರೆ-ಗಲ್ಲಿ-ನಲ್ಲಿ
ಒರೆ-ಗಲ್ಲೆ
ಒರ್ವರೊರ್ವರ
ಒಲ-ವಿನೆದೆ-ಗಳು
ಒಲವೀ-ವಳು
ಒಲವು
ಒಲವೇ
ಒಲಿದಿ-ರುವಳೋ
ಒಲಿ-ಯುವ
ಒಲಿ-ಯುವ-ವ-ನಲ್ಲ
ಒಲಿಸಿ-ಕೊಳ್ಳ-ಬಲ್ಲೆ
ಒಲುಮೆ
ಒಲುಮೆಯ
ಒಲುಮೆ-ಯಾಳವ
ಒಲುಮೆ-ಸೆರೆಯಲ್ಲೀಗ
ಒಲೆ
ಒಲೆಯ
ಒಳ
ಒಳಕ್ಕೆ
ಒಳ-ಗಡೆ
ಒಳ-ಗಾಗಿ
ಒಳ-ಗಾಗಿದ್ದ
ಒಳ-ಗಿ-ನದೆ
ಒಳಗಿ-ನಿಂದ
ಒಳ-ಗಿ-ರಲಿ
ಒಳ-ಗಿ-ರುವ
ಒಳ-ಗಿ-ರುವೆ
ಒಳಗು
ಒಳಗೂ
ಒಳಗೆ
ಒಳಗೇ
ಒಳ-ಗೊಂಡಿತು
ಒಳ-ಗೊಂಡಿದೆಯೋ
ಒಳ-ಗೊಂಡಿ-ರುವು-ವೆಂದು
ಒಳ-ಗೊಂಡಿವೆ
ಒಳಗೊಳಗೇ
ಒಳ-ಪಂಗಡ-ಗಳಲ್ಲಿಯೇ
ಒಳ-ಪಂಗಡ-ಗಳಿವೆ
ಒಳ-ಪಂಗಡ-ಗಳು
ಒಳ-ಪಂಗಡ-ಗಳೆಲ್ಲಾ
ಒಳ-ಪಟ್ಟಿದೆ
ಒಳ-ಪಟ್ಟಿ-ರುವರು
ಒಳ-ಪಟ್ಟಿ-ರು-ವುದೋ
ಒಳ-ಪಡಿ-ಸಲು
ಒಳ-ಪಡಿ-ಸುವ
ಒಳ-ಭಾಗ
ಒಳ-ಭಾಗ-ದಲ್ಲಿ
ಒಳ-ಭಾಗ-ವೇಕೆ
ಒಳಸಂಚು
ಒಳಹೊಕ್ಕಿಲ್ಲ
ಒಳಹೊಕ್ಕು
ಒಳಿ-ತನೆ
ಒಳಿತಾ-ಗಲಿ
ಒಳಿ-ತಿನ
ಒಳಿತು
ಒಳಿತು-ಕೆಡು-ಕಿನ
ಒಳಿತು-ಗಳು-ಕೆಡುಕು-ಗಳು
ಒಳಿತು-ಗಳೆಲ್ಲ
ಒಳಿತೆಂದಿಗು
ಒಳ್ಳಿತಹ
ಒಳ್ಳೆಯ
ಒಳ್ಳೆಯ-ದಕ್ಕೂ
ಒಳ್ಳೆಯ-ದಕ್ಕೇ
ಒಳ್ಳೆಯ-ದಕ್ಕೊ
ಒಳ್ಳೆಯ-ದನ್ನು
ಒಳ್ಳೆಯ-ದನ್ನೂ
ಒಳ್ಳೆಯ-ದಲ್ಲ
ಒಳ್ಳೆಯ-ದಲ್ಲವೆ
ಒಳ್ಳೆಯ-ದಾಗ-ಬ-ಹುದು
ಒಳ್ಳೆಯ-ದಾ-ಗಲೆಂದು
ಒಳ್ಳೆಯ-ದಾ-ಗಲೆಂಬ
ಒಳ್ಳೆಯ-ದಾಗಿ
ಒಳ್ಳೆಯ-ದಾ-ಗುತ್ತದೆ
ಒಳ್ಳೆಯ-ದಾಗುವ
ಒಳ್ಳೆಯ-ದಾಗುವಂಥ
ಒಳ್ಳೆಯ-ದಾಗು-ವುದಕ್ಕೆ
ಒಳ್ಳೆಯ-ದಾಗು-ವು-ದಲ್ಲದೆ
ಒಳ್ಳೆಯ-ದಾ-ಗು-ವು-ದಿಲ್ಲ
ಒಳ್ಳೆಯ-ದಾಗುವುದು
ಒಳ್ಳೆಯ-ದಾಗುವುದೂ
ಒಳ್ಳೆಯ-ದಾ-ಯಿತು
ಒಳ್ಳೆ-ಯದು
ಒಳ್ಳೆ-ಯದೆ
ಒಳ್ಳೆಯ-ದೆಂದು
ಒಳ್ಳೆಯ-ದೆಂದೂ
ಒಳ್ಳೆಯ-ದೆಂಬ
ಒಳ್ಳೆಯ-ದೆಂಬು-ದನ್ನು
ಒಳ್ಳೆ-ಯದೇ
ಒಳ್ಳೆಯ-ದೇನೂ
ಒಳ್ಳೆಯ-ದೇನೋ
ಒಳ್ಳೆ-ಯದೋ
ಒಳ್ಳೆಯ-ವ-ನಾಗಿದ್ದರೆ
ಒಳ್ಳೆಯ-ವ-ನಾ-ಗಿ-ರಲಿಲ್ಲ
ಒಳ್ಳೆಯ-ವನು
ಒಳ್ಳೆಯ-ವ-ನೆಂದು
ಒಳ್ಳೆಯ-ವನೇ
ಒಳ್ಳೆಯ-ವ-ರಾಗ-ಬೇಕೆಂದಿಚ್ಛಿ-ಸುವರು
ಒಳ್ಳೆಯ-ವ-ರಾಗಲು
ಒಳ್ಳೆಯ-ವ-ರಾಗಿ
ಒಳ್ಳೆಯ-ವ-ರಾಗೋಣ
ಒಳ್ಳೆಯ-ವರು
ಒಳ್ಳೆಯ-ವ-ರೆಂದು
ಓ
ಓಂ
ಓಂಕಾರ
ಓಂಕಾರದ
ಓಂಕಾರ-ದಿಂದ
ಓಂಕಾ-ರಾತ್ಮಕ
ಓಜಃಪೂರ್ಣ-ವಾದ
ಓಜಸ್ವಿ-ಯಾದ
ಓಜಸ್ಸನ್ನು
ಓಡ-ಬ-ಹುದು
ಓಡಿ
ಓಡಿದೆ
ಓಡಿ-ಬಂದು
ಓಡಿ-ಯಾಡುತ್ತ
ಓಡಿ-ಯಾಡುತ್ತಿದೆ
ಓಡಿ-ಸ-ಬಲ್ಲೆಯಾ
ಓಡಿ-ಸಲ್ಪಟ್ಟಿತು
ಓಡಿ-ಸುತ್ತಿದ್ದಂಥ-ವನು
ಓಡಿ-ಸುವರು
ಓಡಿ-ಹೋಗಿ
ಓಡಿ-ಹೋಗಿ-ಬಿಡುತ್ತವೆ
ಓಡಿ-ಹೋಗುತ್ತದೆ
ಓಡಿ-ಹೋಗುವ
ಓಡಿ-ಹೋಗು-ವು-ದಿಲ್ಲವೆ
ಓಡಿ-ಹೋದ
ಓಡಿ-ಹೋದೆ
ಓಡಿ-ಹೋ-ಯಿತು
ಓಡು-ತಿಹರು
ಓಡುತ್ತಿರುವ
ಓಡುತ್ತಿವೆ
ಓಡು-ವುದಕ್ಕೆ
ಓತಪ್ರೋತ-ವಾಗಿದೆ
ಓದ-ತೊಡಗಿ-ದರು
ಓದದೇ
ಓದ-ಬ-ಹುದು
ಓದ-ಬೇಕಾಗಿದೆ
ಓದ-ಬೇಕು
ಓದ-ಬೇಕೆಂದಿರುವೆ
ಓದ-ಬೇಕೆಂಬ
ಓದಲಾ-ಗದ
ಓದ-ಲಾಗಿದೆ
ಓದ-ಲಾಗಿ-ರ-ಲಿಲ್ಲ
ಓದ-ಲಿಲ್ಲ
ಓದಲು
ಓದಲ್ಪಡುತ್ತವೆ
ಓದಿ
ಓದಿದ
ಓದಿ-ದನು
ಓದಿ-ದರು
ಓದಿ-ದರೂ
ಓದಿ-ದರೆ
ಓದಿ-ದ-ವ-ರಿಗೆ
ಓದಿ-ದಾಗ
ಓದಿದೆ
ಓದಿದ್ದಾರೆ
ಓದಿದ್ದೀ-ಯಷ್ಟೆ
ಓದಿದ್ದೀರಾ
ಓದಿದ್ದೀರಿ
ಓದಿದ್ದೆ
ಓದಿದ್ದೆವು
ಓದಿದ್ದೇನೆ
ಓದಿ-ನೋಡಿ
ಓದಿ-ಬಿಟ್ಟರೆ
ಓದಿ-ರ-ಬ-ಹುದು
ಓದಿ-ರ-ಲಿಲ್ಲ
ಓದಿ-ರುವ
ಓದಿ-ರು-ವಿರಾ
ಓದಿ-ರುವೆ
ಓದಿಲ್ಲ
ಓದಿಲ್ಲದೆ
ಓದಿಲ್ಲವೆ
ಓದಿ-ಸಿ-ದರು
ಓದು
ಓದುತ್ತ
ಓದುತ್ತಾ-ನಂತೆ
ಓದುತ್ತಿದ್ದಾನೆ
ಓದುತ್ತೇ-ವಲ್ಲ
ಓದು-ಬರಹ-ಗಳನ್ನು
ಓದುವ
ಓದು-ವಂತೆ
ಓದು-ವರು
ಓದು-ವಾಗ
ಓದು-ವುದಕ್ಕೆ
ಓದು-ವು-ದ-ರಿಂದ
ಓದು-ವುದು
ಓದು-ವುದೂ
ಓದುವೆ
ಓದೋಣವಣ್ಣ
ಓರ್ವ
ಓರ್ವಳು
ಓಲೆ-ಗರಿಯ
ಓಲೆ-ಗಾಗಿ
ಓಲೈ-ಸಿದ
ಓವೊ
ಓಹೊ
ಓಹೋ
ಓಹ್
ಔನ್ನತ್ಯದ
ಔಷಧಿ
ಔಷಧಿ-ಗಳಿಂದ
ಔಷಧಿ-ಗಳು
ಔಷಧಿ-ಯನ್ನು
ಕ
ಕಂಗಳ
ಕಂಗಳಿಂ
ಕಂಗಳಿಗೆ
ಕಂಗಳು
ಕಂಗಾ-ಲಾಗಿ
ಕಂಠ-ದಲ್ಲಿ
ಕಂಠದಿ
ಕಂಠ-ದಿಂದ
ಕಂಠ-ಪಾಠ
ಕಂಠ-ಪಾಠ-ಮಾಡಿದ್ದೀಯಾ
ಕಂಠ-ಪೂರ್ತಿ
ಕಂಠ-ಮೋರ
ಕಂಠೆ
ಕಂಠೋಕ್ತ-ವಾಗಿ
ಕಂಡ
ಕಂಡಂತೆ
ಕಂಡ-ದನು
ಕಂಡದ್ದನ್ನು
ಕಂಡದ್ದು
ಕಂಡ-ನ-ವನು
ಕಂಡರು
ಕಂಡರೂ
ಕಂಡರೆ
ಕಂಡ-ವನು
ಕಂಡಾಗ
ಕಂಡಿತು
ಕಂಡಿದ್ದಿಲ್ಲ
ಕಂಡಿದ್ದೇನೆ
ಕಂಡಿ-ರದ
ಕಂಡಿರ-ಬ-ಹುದು
ಕಂಡಿರ-ಬೇಕು
ಕಂಡಿ-ರಲು
ಕಂಡಿ-ರುವೆ
ಕಂಡಿ-ರು-ವೆವು
ಕಂಡಿಲ್ಲ
ಕಂಡಿಲ್ಲವೇ
ಕಂಡು
ಕಂಡು-ಕೊಂಡನು
ಕಂಡು-ಕೊಂಡರು
ಕಂಡು-ಕೊಂಡಿದ್ದೇನೆ
ಕಂಡು-ಕೊಂಡು
ಕಂಡು-ಕೊಳ್ಳುತ್ತಾನೆ
ಕಂಡು-ಕೊಳ್ಳುವ
ಕಂಡು-ದ-ರಿಂದಲೇ
ಕಂಡು-ದೇ-ನನ್ನು
ಕಂಡು-ದೇನು
ಕಂಡು-ಬಂತು
ಕಂಡು-ಬಂದರೆ
ಕಂಡು-ಬಂದವು
ಕಂಡು-ಬಂದಿತು
ಕಂಡು-ಬಂದಿದೆ
ಕಂಡು-ಬಂದಿವೆ
ಕಂಡು-ಬರ-ದಿ-ರುವಾಗ
ಕಂಡು-ಬ-ರ-ಲಾ-ರರು
ಕಂಡು-ಬ-ರ-ಲಿಲ್ಲ
ಕಂಡು-ಬ-ರುತ್ತದೆ
ಕಂಡು-ಬರುತ್ತ-ದೆ-ಯಲ್ಲಾ
ಕಂಡು-ಬರುತ್ತ-ದೆಯೇ
ಕಂಡು-ಬರುತ್ತ-ದೆಯೋ
ಕಂಡು-ಬರುತ್ತ-ವೆ-ಯೆಂದು
ಕಂಡು-ಬರುತ್ತಿದೆ
ಕಂಡು-ಬರುತ್ತಿವೆ
ಕಂಡು-ಬ-ರುವ
ಕಂಡು-ಬ-ರುವಾಗ
ಕಂಡು-ಬ-ರು-ವು-ದಿಲ್ಲ
ಕಂಡು-ಬ-ರು-ವುದು
ಕಂಡು-ಬ-ರು-ವುದೇನು
ಕಂಡು-ಬ-ರು-ವುದೇನೋ
ಕಂಡು-ಹಿಡಿ
ಕಂಡು-ಹಿಡಿ-ದನು
ಕಂಡು-ಹಿಡಿ-ದರು
ಕಂಡು-ಹಿಡಿ-ದರೆ
ಕಂಡು-ಹಿಡಿ-ದ-ವರೇ
ಕಂಡು-ಹಿಡಿ-ದಿಲ್ಲ
ಕಂಡು-ಹಿಡಿದು
ಕಂಡು-ಹಿಡಿ-ಯಲು
ಕಂಡು-ಹಿಡಿ-ಯಿರಿ
ಕಂಡು-ಹಿಡಿ-ಯುವದು
ಕಂಡೂ
ಕಂಡೆ
ಕಂಡೇ
ಕಂತೆ
ಕಂತೆ-ಗಳು
ಕಂದ
ಕಂದ-ಕಕ್ಕೆ
ಕಂದ-ನನು
ಕಂದಾ-ಚಾರ-ಗ-ಳಲ್ಲೇ
ಕಂಪ-ನ-ಗಳವು
ಕಂಪ-ನದಿಂ
ಕಂಪ-ನನ
ಕಂಪ-ನಿ-ಯವರು
ಕಂಪಿಸಿ
ಕಂಪಿ-ಸುತ್ತದೆ
ಕಂಪಿ-ಸುತ್ತಿದೆ
ಕಂಪೆನಿಯ
ಕಂಬ-ಗಳ
ಕಂಬನಿ
ಕಂಬ-ನಿ-ಗರೆ-ಯುತ್ತಿದ್ದು-ದನ್ನು
ಕಂಬ-ನಿ-ಧಾರೆ-ಯನೆ
ಕಂಬ-ನಿ-ಧಾರೆ-ಯಲಿ
ಕಂಬ-ನಿಯ
ಕಂಬ-ನಿ-ಯೊ-ಡನೆ
ಕಂಬಿ-ಗಳನ್ನು
ಕಃ
ಕಕ್ಕಾಬಿಕ್ಕಿ-ಯಾದ
ಕಕ್ಷಿ
ಕಕ್ಷಿ-ಗಳ
ಕಗ್ಗವ
ಕಚ್ಚಾ
ಕಚ್ಚಾ-ಪದಾರ್ಥ-ಗಳಿಂದ
ಕಚ್ಚಿತ್ತು
ಕಚ್ಚಿ-ದಂತೆ
ಕಚ್ಚಿ-ದರೂ
ಕಛೇರಿ-ಗಳ
ಕಛೇರಿಯ
ಕಟಿಯ
ಕಟುಕ-ನೊಬ್ಬನು
ಕಟು-ಕರ
ಕಟುಕ-ರಿಂದ
ಕಟು-ಮಧುರ
ಕಟು-ವಾಗಿ
ಕಟ್ಟ
ಕಟ್ಟ-ಕ-ಡೆಗೆ
ಕಟ್ಟ-ಕಡೆಯ
ಕಟ್ಟ-ಕಡೆ-ಯಲ್ಲಿ
ಕಟ್ಟಡ
ಕಟ್ಟ-ಡ-ಗಳನ್ನು
ಕಟ್ಟ-ಡ-ಗಳಲ್ಲಿ
ಕಟ್ಟ-ಡದ
ಕಟ್ಟ-ಡವು
ಕಟ್ಟ-ಡ-ವೊಂದ-ರಲ್ಲಿ
ಕಟ್ಟ-ಡ-ವೊಂದಿತ್ತು
ಕಟ್ಟನೆ
ಕಟ್ಟನ್ನು
ಕಟ್ಟ-ಬೇಕು
ಕಟ್ಟ-ಲಿ-ರುವ
ಕಟ್ಟಲು
ಕಟ್ಟಲ್ಪಟ್ಟಿವೆ
ಕಟ್ಟಲ್ಪಡ-ಬೇಕು
ಕಟ್ಟಲ್ಪಡ-ಬೇಕೆಂದರೆ
ಕಟ್ಟಲ್ಪಡು-ವುದು
ಕಟ್ಟಳೆ
ಕಟ್ಟ-ಳೆ-ಗಳನ್ನೆಲ್ಲಾ
ಕಟ್ಟ-ಳೆ-ಗಳಿವೆ-ಯಲ್ಲ
ಕಟ್ಟ-ಳೆ-ಗಳೂ
ಕಟ್ಟಾ
ಕಟ್ಟಿ
ಕಟ್ಟಿ-ಕೊಂಡು
ಕಟ್ಟಿ-ಕೊಂಡೇನು
ಕಟ್ಟಿಗೆ
ಕಟ್ಟಿಡು
ಕಟ್ಟಿದ
ಕಟ್ಟಿ-ದಂತಾಗು-ವುದು
ಕಟ್ಟಿ-ದಂತಿದೆ
ಕಟ್ಟಿ-ನಿಂದಲೂ
ಕಟ್ಟಿ-ಬಿಗಿ-ದಿಹ
ಕಟ್ಟಿ-ರುವರು
ಕಟ್ಟಿ-ರು-ವೆಯಾ
ಕಟ್ಟಿ-ಸ-ಬೇಕು
ಕಟ್ಟಿ-ಸ-ಲಿ-ರುವ
ಕಟ್ಟಿ-ಸಿ-ದನು
ಕಟ್ಟಿ-ಸಿದ್ದ
ಕಟ್ಟಿ-ಸಿದ್ದರು
ಕಟ್ಟಿ-ಹಾ-ಕಲಿಚ್ಚಿಸು-ವಿ-ರೆಂದು
ಕಟ್ಟಿ-ಹಾಕಿ
ಕಟ್ಟು
ಕಟ್ಟು-ಕತೆ-ಗಳು
ಕಟ್ಟು-ಕತೆ-ಯಿಂದಾಗಲೀ
ಕಟ್ಟು-ಗಳನ್ನೂ
ಕಟ್ಟುತ್ತಿದ್ದನು
ಕಟ್ಟುತ್ತಿದ್ದಾಗ
ಕಟ್ಟು-ನಿಟ್ಟನ್ನು
ಕಟ್ಟು-ನಿಟ್ಟಾದ
ಕಟ್ಟು-ನಿಟ್ಟಿನ
ಕಟ್ಟು-ನಿಟ್ಟಿನಲ್ಲಿ
ಕಟ್ಟು-ಪಾಡಿ-ನಲ್ಲಿಟ್ಟಿದ್ದಾರೆ
ಕಟ್ಟು-ಪಾಡು
ಕಟ್ಟು-ಪಾಡು-ಗಳನ್ನು
ಕಟ್ಟು-ಪಾಡು-ಗಳನ್ನೆಲ್ಲಾ
ಕಟ್ಟು-ಪಾಡು-ಗ-ಳಿದ್ದಾಗ್ಯೂ
ಕಟ್ಟು-ವ-ವ-ರಾರು
ಕಟ್ಟು-ವುದ-ರಲ್ಲಿ
ಕಟ್ಟು-ವುದು
ಕಟ್ಟು-ವೆನು
ಕಟ್ಟು-ಹಾಕಿದ್ದ
ಕಠಿಣ
ಕಠಿಣ-ತಪ-ಗಳ-ನೆಲ್ಲ
ಕಠಿಣ-ವಾಗಿ
ಕಠಿಣ-ವಾದ
ಕಠಿಣವೊ
ಕಠಿನ-ವಾದ
ಕಠೋ-ಪನಿಷತ್ತನ್ನು
ಕಠೋ-ಪನಿಷತ್ತಿ-ನಲ್ಲಿ
ಕಠೋರ
ಕಠೋರ-ತೆ-ಯನ್ನು
ಕಠೋರ-ವಾದ
ಕಡಕಡ
ಕಡಮೆ
ಕಡಮೆ-ಯಾಗಿ-ರ-ಲಿಲ್ಲ
ಕಡಮೆಯೇ
ಕಡಲ
ಕಡ-ಲಿನ-ಲೆ-ಗಳ
ಕಡ-ಲಿನ-ಲೆ-ಗಳಲ್ಲಿ
ಕಡ-ಲಿನ-ಲೆ-ಯನ್ನೇ-ಳಿ-ಸುತ್ತಿತ್ತು
ಕಡ-ಲಿ-ನಲ್ಲಿ
ಕಡ-ಲಿನೆದೆಯ
ಕಡಲು-ಗಳ
ಕಡಿ
ಕಡಿ-ಕ-ಡಿದು
ಕಡಿ-ತ-ಗ-ಳಿದ್ದವು
ಕಡಿ-ದಾದ
ಕಡಿದು
ಕಡಿ-ದು-ಹಾಕಲ್ಪ-ಡು-ವುವು
ಕಡಿದೇ
ಕಡಿಮೆ
ಕಡಿ-ಮೆ-ಮಾ-ಡಲು
ಕಡಿ-ಮೆ-ಮಾಡು
ಕಡಿ-ಮೆ-ಯಾಗ-ದಂತೆ
ಕಡಿ-ಮೆ-ಯಾಗಿ
ಕಡಿ-ಮೆ-ಯಾ-ಗಿದೆ
ಕಡಿ-ಮೆ-ಯಾಗಿ-ಬಿಡುತ್ತದೆ
ಕಡಿ-ಮೆ-ಯಾಗಿ-ರುತ್ತದೆ
ಕಡಿ-ಮೆ-ಯಾಗಿ-ರು-ವು-ದನ್ನು
ಕಡಿ-ಮೆ-ಯಾಗಿ-ಹೋಗಿದೆ
ಕಡಿ-ಮೆ-ಯಾಗುತ್ತ
ಕಡಿ-ಮೆ-ಯಾ-ಗುತ್ತಿದೆ
ಕಡಿ-ಮೆ-ಯಾಗುತ್ತಿವೆ
ಕಡಿ-ಮೆ-ಯಾಗುವುದು
ಕಡಿ-ಮೆ-ಯೆಂದು
ಕಡು
ಕಡು-ಗೆಚ್ಚಿ-ನಿಂದ
ಕಡು-ಬಡವ-ನಾದ
ಕಡೆ
ಕಡೆ-ಗಣಿ-ಸಲ್ಪಡು-ವು-ದಿಲ್ಲ
ಕಡೆ-ಗಲ್ಲಿ
ಕಡೆ-ಗಳಲ್ಲಿ
ಕಡೆ-ಗಳಲ್ಲಿಯೂ
ಕಡೆಗೂ
ಕಡೆಗೆ
ಕಡೆಗೇ
ಕಡೆ-ಗೊಂದು
ಕಡೆ-ಗೊಮ್ಮೆ
ಕಡೆಯ
ಕಡೆ-ಯ-ಪಕ್ಷ
ಕಡೆ-ಯಲ್ಲಿ
ಕಡೆ-ಯಲ್ಲಿಯೂ
ಕಡೆ-ಯ-ವರ
ಕಡೆ-ಯ-ವ-ರೆಗೂ
ಕಡೆ-ಯಾ-ಗಿದೆ
ಕಡೆ-ಯಿಂದ
ಕಡೆಯೂ
ಕಡೆಯೇ
ಕಡೆ-ಯೊಳು
ಕಡೆ-ಹಾಯ್ದು
ಕಡ್ಡಾಯ-ವಾಗಿ
ಕಡ್ಡಿ
ಕಣಕ-ಣದಿ
ಕಣಜ-ದಲ್ಲಿ
ಕಣಜ-ವನ್ನು
ಕಣ-ವಾದರೂ
ಕಣಿವೆ-ಗಳ-ನಿಳಿ-ದೆವು
ಕಣಿ-ವೆಯು
ಕಣ್
ಕಣ್ಗೆ
ಕಣ್ಗೊಂಬೆ-ಯಲ್ಲಿ
ಕಣ್ಣ
ಕಣ್ಣಂಚಿ-ನಲ್ಲಿ
ಕಣ್ಣನು
ಕಣ್ಣನ್ನು
ಕಣ್ಣಲ್ಲಿ
ಕಣ್ಣಲ್ಲಿ-ವನು
ಕಣ್ಣಾರೆ
ಕಣ್ಣಿ
ಕಣ್ಣಿಗೆ
ಕಣ್ಣಿದೆ
ಕಣ್ಣಿ-ದೆ-ಯೆಂಬು-ದನ್ನು
ಕಣ್ಣಿ-ದೆಯೋ
ಕಣ್ಣಿನ
ಕಣ್ಣಿ-ನಲ್ಲಿ
ಕಣ್ಣಿ-ನಿಂದ
ಕಣ್ಣಿ-ನಿಂದಲೆ
ಕಣ್ಣಿ-ನೊ-ಳಗೆ
ಕಣ್ಣಿಯೆ
ಕಣ್ಣಿಲ್ಲ
ಕಣ್ಣೀ-ರನ್ನು
ಕಣ್ಣೀ-ರನ್ನೂ
ಕಣ್ಣೀರು
ಕಣ್ಣು
ಕಣ್ಣು-ಗಳ
ಕಣ್ಣು-ಗ-ಳನೆ
ಕಣ್ಣು-ಗಳನ್ನು
ಕಣ್ಣು-ಗಳಲ್ಲಿ
ಕಣ್ಣು-ಗಳಿಂದ
ಕಣ್ಣು-ಗಳಿಗೆ
ಕಣ್ಣು-ಗಳಿವೆಯೊ
ಕಣ್ಣು-ಗಳು
ಕಣ್ಣು-ಗಳೂ
ಕಣ್ಣು-ರೆಪ್ಪೆ-ಗಳೂ
ಕಣ್ಣೆ-ದು-ರಿಗೆ
ಕಣ್ಣೆದು-ರಿಗೇ
ಕಣ್ತೆರೆ-ದಂತಾಗಿದೆ
ಕಣ್ತೆರೆದು
ಕಣ್ತೆರೆ-ಸಿ-ಹಿರಿ
ಕಣ್ದೆರೆದು
ಕಣ್ಮರೆ-ಯಾ-ಗಿದೆ
ಕಣ್ಮರೆ-ಯಾಗುತ್ತ-ಲಿವೆ
ಕಣ್ಮರೆ-ಯಾಗುವುದ-ರಲ್ಲಿತ್ತು
ಕಣ್ಮರೆ-ಯಾದ
ಕಣ್ಮುಂದೆ
ಕಣ್ಮುಚ್ಚಿ
ಕತ-ಕಾಲ್
ಕತ-ಮತ
ಕತವಾ
ಕತೆ
ಕತ್ತರಿ-ಸಲ್ಪಟ್ಟು
ಕತ್ತ-ರಿಸಿ
ಕತ್ತ-ರಿಸಿ-ಹಾಕು-ವುದಕ್ಕೋಸ್ಕರ
ಕತ್ತಲ
ಕತ್ತಲ-ನಿರಾಸೆ-ಗಳ
ಕತ್ತಲನು
ಕತ್ತಲ-ನು-ಗುಳಿ-ಹುದು
ಕತ್ತಲಲಿ
ಕತ್ತಲ-ಳಿ-ಯಲಿ
ಕತ್ತಲಾಗುತ್ತಲಿದೆ
ಕತ್ತಲೆ
ಕತ್ತಲೆಯ
ಕತ್ತಲೆ-ಯಲ್ಲಿ
ಕತ್ತಲೆ-ಯಾದದ್ದನ್ನು
ಕತ್ತಲೆ-ಯಾ-ಯಿತು
ಕತ್ತಲೆಯೆ
ಕತ್ತಲೊಳ-ಗಿನ
ಕತ್ತಿನ
ಕತ್ತಿ-ನಲ್ಲಿ
ಕತ್ತಿನಲ್ಲಿದ್ದ
ಕತ್ತಿಯ
ಕತ್ತಿಯನ್ನೆಂದೂ
ಕತ್ತೆ
ಕತ್ತೆ-ಗಳಂತೆ
ಕತ್ತೆ-ಯನ್ನು
ಕತ್ತೆ-ಯೊಂದಿಗೆ
ಕಥ
ಕಥಾ
ಕಥಾ-ನಕ-ದಲ್ಲಿ
ಕಥೆ
ಕಥೆ-ಗಳನ್ನಾ-ಡದೆ
ಕಥೆ-ಗಳನ್ನು
ಕಥೆ-ಗಳನ್ನೂ
ಕಥೆ-ಗಳಲ್ಲಿಯೇ
ಕಥೆ-ಗಿಂತ
ಕಥೆ-ಯನ್ನು
ಕಥೆಯು
ಕಥೋಪಕಥ-ನ-ಗಳನ್ನು
ಕಥೋಪಕಥನ-ಗಳು
ಕದ-ನಕೆ
ಕದ-ನದ
ಕದ-ನವು
ಕದಲಿ-ಸಿ-ದರೂ
ಕದವ
ಕದ-ವನ್ನು
ಕದಿಯ-ಬೇ-ಕಾ-ಯಿತು
ಕದಿ-ಯುವಾಗ
ಕನಸ
ಕನ-ಸನ್ನಾದರೂ
ಕನ-ಸಾ-ಯಿತು
ಕನ-ಸಿಗೆ
ಕನ-ಸಿದು
ಕನ-ಸಿನ
ಕನ-ಸಿ-ನಲು
ಕನಸು
ಕನ-ಸು-ಗಳ
ಕನ-ಸು-ಗಳನು
ಕನ-ಸು-ಣಿ-ಗಳ
ಕನ-ಸು-ಣಿ-ಗ-ಳಿದ್ದರೆ
ಕನ-ಸು-ಣಿಗೂ
ಕನಸೆ
ಕನ-ಸೆ-ನುವ
ಕನ-ಸೆಲ್ಲ
ಕನಿ-ಕರ
ಕನಿ-ಕರ-ಪಡಬೇ-ಕಷ್ಟೆ
ಕನಿ-ಕ-ರಿಸಿ-ದರೂ
ಕನಿ-ಕರಿ-ಸು-ವು-ದಿಲ್ಲ
ಕನೂಜಿ-ಗಳು
ಕನ್ನಡ
ಕನ್ನಡಿ-ಗಳಿವು
ಕನ್ಯೆಯ-ರಾಗೇ
ಕಪಟ
ಕಪಟ-ರ-ಹಿತ-ನಾಗಿ
ಕಪಟಾಚ-ರಣೆ
ಕಪಟಿ-ಗಳು
ಕಪಾಲ
ಕಪಾಲ-ಲೇಖಃ
ಕಪಿ-ಗಳ
ಕಪಿಯಂತಾಗುವೆ
ಕಪ್ಪಾಗಿ-ಬಿಟ್ಟಿತ್ತು
ಕಪ್ಪು
ಕಬಳಿ-ಸಿ-ಹುದು
ಕಬಳಿ-ಸುತ್ತಿತ್ತು
ಕಬು
ಕಬೆ
ಕಬ್ಬಿ-ಣದ
ಕಬ್ಬಿ-ಣವು
ಕಬ್ಬಿಣವೊ
ಕಮಂಡಲು
ಕಮಲ
ಕಮಲ-ದಂತೆ
ಕಮಲದಿ
ಕಮಲ-ನಯನ-ಗಳಿಂದ
ಕಮಲ-ನೇತ್ರ-ಗಳು
ಕಮಲವ
ಕಮಿಷ-ನರ್
ಕಮ್ರಕಲ್ಹಾರ
ಕಯಾಲಿ
ಕರ
ಕರ-ಕಮಲ-ವನ್ನು
ಕರ-ಕರಗಿ
ಕರ-ಗದೆ
ಕರಗಿ
ಕರ-ಗಿ-ದರು
ಕರ-ಗಿ-ಹೋ-ಗಲಿ
ಕರ-ಗಿ-ಹೋಗುವ
ಕರ-ಗುತ
ಕರ-ಗುತಿವೆ
ಕರ-ಗು-ತಿಹ
ಕರ-ಗು-ವರು
ಕರ-ಗು-ವುದು
ಕರಣ
ಕರ-ಣದ
ಕರ-ಣ-ದಾರ್ಭಟ
ಕರ-ಣವೂ
ಕರ-ಣಸ್ಥಾನ-ದಲ್ಲಿದೆ
ಕರ-ಣಾದಿ
ಕರಣೇ
ಕರ-ತಲ-ಗತಾ
ಕರ-ತಲಾಮಲ-ಕದ
ಕರ-ದಲಿ
ಕರ-ಧಾರಾ
ಕರ-ಫಲಾಯತೇ
ಕರಾಲ
ಕರಾಳಿ
ಕರಾಳಿನ
ಕರಾಳಿನಿ
ಕರಿ
ಕರಿ-ಕರಾಳ
ಕರಿ-ಖೇಲ
ಕರಿನು
ಕರಿ-ಯರು
ಕರುಣಾ
ಕರು-ಣಾ-ಘನ
ಕರು-ಣಾ-ನಿಧಿಯೆ
ಕರು-ಣಾ-ಪೂರಿತ
ಕರು-ಣಾ-ಮಯ
ಕರು-ಣಾ-ಮಯಿ
ಕರು-ಣಾಳು
ಕರುಣಿ
ಕರುಣೆ
ಕರು-ಣೆಗೆ
ಕರು-ಣೆ-ಯನ್ನು
ಕರು-ಣೆ-ಯಲಿ
ಕರು-ಣೆ-ಯಿಂದ
ಕರೆ
ಕರೆ-ಕರೆದು
ಕರೆಗೆ
ಕರೆ-ದರು
ಕರೆ-ದರೂ
ಕರೆ-ದರೆ
ಕರೆ-ದರೋ
ಕರೆ-ದಾಗ
ಕರೆ-ದಿದ್ದಾರೆ
ಕರೆ-ದಿದ್ದಾರೆಂದು
ಕರೆ-ದಿ-ರು-ವಂತೆ
ಕರೆದು
ಕರೆ-ದುಕೊ
ಕರೆ-ದು-ಕೊಂಡು
ಕರೆ-ದೊಯ್ದು
ಕರೆ-ದೊಯ್ಯಲು
ಕರೆ-ದೊಯ್ಯುವನು
ಕರೆ-ಯ-ಬ-ಹುದು
ಕರೆ-ಯಲಿ
ಕರೆ-ಯಲ್ಪಡುವ
ಕರೆ-ಯಿತ್ತನು
ಕರೆ-ಯಿರಿ
ಕರೆ-ಯು-ತಿಹೆ-ವಿಂದು
ಕರೆ-ಯುತ್ತಿದ್ದರು
ಕರೆ-ಯುತ್ತಿರು-ವಂತೆ
ಕರೆ-ಯುತ್ತೇವೋ
ಕರೆ-ಯುವ
ಕರೆ-ಯು-ವಂತೆ
ಕರೆ-ಯುವರು
ಕರೆ-ಯುವಿರಿ
ಕರೆ-ಯು-ವು-ದಾದರೆ
ಕರೆ-ಯು-ವುದು
ಕರೆ-ಯು-ವುವು
ಕರೆ-ವಂತೆನ್ನ
ಕರೆ-ವುದೀ
ಕರೆ-ಸಿ-ಕೊಂಡ
ಕರೆ-ಸಿ-ಕೊಂಡು
ಕರೆ-ಸಿ-ದರು
ಕರೇಛಿ
ಕರೊ
ಕರೋತಿ
ಕರೋಮಿ
ಕರ್ಜನ್
ಕರ್ಣಾ-ಮೃ-ತದ
ಕರ್ತ
ಕರ್ತನ
ಕರ್ತ-ನಲ್ಲ-ವೆಂಬ
ಕರ್ತ-ಭಾಜ
ಕರ್ತವ್ಯ
ಕರ್ತವ್ಯ-ಗಳ
ಕರ್ತವ್ಯ-ಗಳನ್ನು
ಕರ್ತವ್ಯದ
ಕರ್ತವ್ಯ-ದಿಂದ
ಕರ್ತವ್ಯ-ಪರಾಯಣ-ತೆ-ಯನ್ನು
ಕರ್ತವ್ಯ-ಭಾರ-ದಿಂದ
ಕರ್ತವ್ಯ-ವನ್ನು
ಕರ್ತವ್ಯ-ವೆಂದು
ಕರ್ತವ್ಯವೇ
ಕರ್ತಾಲ್-ಗಳನ್ನು
ಕರ್ತೃ-ವಾದ
ಕರ್ತೃವು
ಕರ್ಪೂರಧೂ-ಸರಿ-ತ-ವಾದ
ಕರ್ಮ
ಕರ್ಮ-ಇವು
ಕರ್ಮ-ಇವು-ಗಳ
ಕರ್ಮ-ಕಠೋರ
ಕರ್ಮ-ಕಠೋರನೆ
ಕರ್ಮ-ಕಲೇ-ವರ-ಮದ್ಭುತ
ಕರ್ಮ-ಕಾಂಡದ
ಕರ್ಮ-ಕಾಂಡ-ದೊಂದಿಗೆ
ಕರ್ಮ-ಕಾಂಡವು
ಕರ್ಮ-ಕುಶಲಿ
ಕರ್ಮಕ್ಕೂ
ಕರ್ಮಕ್ಕೆ
ಕರ್ಮಕ್ಷೇತ್ರ-ದಲ್ಲಿ
ಕರ್ಮ-ಗಳ
ಕರ್ಮ-ಗಳನ್ನು
ಕರ್ಮ-ಗಳನ್ನೂ
ಕರ್ಮ-ಗಳಲ್ಲೆಲ್ಲಾ
ಕರ್ಮ-ಗಳಾಗಿ-ರು-ವುವು
ಕರ್ಮ-ಗಳಿಂದ
ಕರ್ಮ-ಗಳಿಂದಲೂ
ಕರ್ಮ-ಗಳಿಗೆ
ಕರ್ಮ-ಗಳು
ಕರ್ಮ-ಗಳೆಲ್ಲಾ
ಕರ್ಮಣಾ
ಕರ್ಮಣಾಂ
ಕರ್ಮಣ್ಯ-ಕರ್ಮ
ಕರ್ಮ-ತತ್ಪರ-ತೆ-ಯನ್ನೂ
ಕರ್ಮ-ತತ್ಪರ-ತೆಯೂ
ಕರ್ಮ-ತತ್ಪರ-ರಾಗ-ಬೇಕೊ
ಕರ್ಮ-ತತ್ಪರ-ರಾಗುತ್ತಾರೆ
ಕರ್ಮತ್ಯಾಗ
ಕರ್ಮದ
ಕರ್ಮ-ದ-ಲೆ-ಗಳ
ಕರ್ಮ-ದಲ್ಲಿ
ಕರ್ಮ-ದಿಂದ
ಕರ್ಮ-ದಿಂದಾ-ಗಲಿ
ಕರ್ಮ-ನದಿ-ಯಲಿ
ಕರ್ಮ-ನಿರತ
ಕರ್ಮ-ಪಾಶ
ಕರ್ಮಪ್ರಪಂಚ-ದಲ್ಲಿ
ಕರ್ಮಪ್ರಾಣ-ರನ್ನಾಗಿ
ಕರ್ಮ-ಫಲ
ಕರ್ಮ-ಫಲ-ಗಳೇ
ಕರ್ಮ-ಫಲ-ದಲ್ಲಿ
ಕರ್ಮ-ಫಲ-ದಾಸೆ-ಯನ್ನೆಲ್ಲಾ
ಕರ್ಮ-ಫಲ-ದಿಂದ
ಕರ್ಮ-ಫಲ-ದಿಂದಲೇ
ಕರ್ಮ-ಫಲಾಪೇಕ್ಷೆ
ಕರ್ಮ-ಫಲಾಪೇಕ್ಷೆ-ಯಲ್ಲಿ
ಕರ್ಮ-ಬಂಧ-ನದ
ಕರ್ಮ-ಬಂಧ-ನ-ವನ್ನು
ಕರ್ಮ-ಯೋಗ
ಕರ್ಮ-ಯೋಗ-ವನ್ನು
ಕರ್ಮ-ಯೋಗ-ವಲ್ಲವೆ
ಕರ್ಮ-ಯೋಗ-ವೆಂದು
ಕರ್ಮ-ಯೋಗಿ
ಕರ್ಮ-ಯೋಗಿ-ಗ-ಳೆಂದು
ಕರ್ಮ-ವದು
ಕರ್ಮ-ವನ್ನು
ಕರ್ಮ-ವನ್ನೂ
ಕರ್ಮ-ವಾದವು
ಕರ್ಮ-ವಿ-ರುವುದೊ
ಕರ್ಮ-ವೀರ
ಕರ್ಮ-ವೀರರು
ಕರ್ಮವು
ಕರ್ಮವೂ
ಕರ್ಮ-ವೆಲ್ಲ
ಕರ್ಮವೇ
ಕರ್ಮ-ವೊಂದೇ
ಕರ್ಮ-ಶಕ್ತಿಯು
ಕರ್ಮ-ಶೀಲರೂ
ಕರ್ಮ-ಸಂಬಂಧದ
ಕರ್ಮ-ಹೀನ-ನಾಗಿದ್ದ
ಕರ್ಮ-ಹೀನ-ರಲ್ಲ
ಕರ್ಮಾಂಗ
ಕರ್ಮಾಚ-ರಣೆಯ
ಕರ್ಮಾನುಷ್ಠಾನ
ಕರ್ಮಾನುಷ್ಠಾನ-ಗಳ
ಕರ್ಮಾನುಷ್ಠಾನ-ಗಳನ್ನು
ಕರ್ಮಾನುಷ್ಠಾನ-ಗಳು
ಕರ್ಮಾನುಷ್ಠಾನ-ವಲ್ಲವೇ
ಕರ್ಮಾನುಷ್ಠಾನ-ವೆಂದರೆ
ಕರ್ಮಿ
ಕರ್ಮಿ-ಗಳು
ಕಲಂಕ-ವನ್ನು
ಕಲಕಲು
ಕಲಕಿರೆ
ಕಲಕಿ-ಹೋಗಿತ್ತು
ಕಲಕು-ತಲಿ
ಕಲಕುತ್ತಿದ್ದ
ಕಲಹದ
ಕಲಾ
ಕಲಾ-ಕುಶಲ-ತೆ-ಯುಳ್ಳದ್ದು
ಕಲಾ-ಕೃ-ತಿಯ
ಕಲಾ-ಕೃತಿ-ಯನ್ನು
ಕಲಾಪಂ
ಕಲಾ-ವಿದ
ಕಲಾ-ವಿದರ
ಕಲಾ-ವಿದರು
ಕಲಾ-ಶಾಲೆಯ
ಕಲಾ-ಸೌಂದರ್ಯ-ಪೂರಿ-ತ-ವಾಗಿ
ಕಲಿ
ಕಲಿ-ಡೋರ
ಕಲಿ-ತ-ಕಲಿ-ಕಲಂಕಂ
ಕಲಿ-ತರು
ಕಲಿ-ತಿದ್ದಾಳೆ
ಕಲಿ-ತಿದ್ದೇವೆ
ಕಲಿ-ತಿರಿ
ಕಲಿ-ತಿ-ರು-ವಿರಿ
ಕಲಿತು
ಕಲಿ-ತುಕೊ
ಕಲಿ-ತು-ಕೊಂಡಿದ್ದಾರೆ
ಕಲಿ-ತು-ಕೊಂಡಿದ್ದೀ-ರಷ್ಟೆ
ಕಲಿ-ತು-ಕೊಂಡಿದ್ದೇವೆ
ಕಲಿ-ತು-ಕೊಂಡಿಲ್ಲದೆ
ಕಲಿ-ತು-ಕೊಂಡು
ಕಲಿ-ತು-ಕೊಳ್ಳ-ಬ-ಹುದು
ಕಲಿ-ತು-ಕೊಳ್ಳ-ಬೇಕು
ಕಲಿ-ತು-ಕೊಳ್ಳ-ಬೇ-ಕೇನು
ಕಲಿ-ತು-ಕೊಳ್ಳಿ
ಕಲಿ-ತು-ಕೊಳ್ಳುವ
ಕಲಿ-ತು-ಕೊಳ್ಳು-ವುದಕ್ಕೆ
ಕಲಿ-ತು-ಕೊಳ್ಳು-ವು-ದಿಲ್ಲ-ವೇಕೆ
ಕಲಿ-ತು-ದೆಂದು
ಕಲಿ-ತೆವು
ಕಲಿ-ಯ-ಬೇ-ಕಾದ್ದು
ಕಲಿ-ಯಬೇ-ಕಾ-ಯಿತು
ಕಲಿ-ಯ-ಬೇಕು
ಕಲಿ-ಯಲಿ
ಕಲಿ-ಯ-ಲಿಲ್ಲ
ಕಲಿ-ಯಲು
ಕಲಿ-ಯಿರಿ
ಕಲಿ-ಯುಗ-ದಲ್ಲಿ
ಕಲಿ-ಯುತ್ತಾರೆ
ಕಲಿ-ಯುತ್ತಾರೆಯೋ
ಕಲಿ-ಯುತ್ತಿದ್ದೀರಿ
ಕಲಿ-ಯು-ವಂತೆ
ಕಲಿ-ಯುವ-ರಲ್ಲವೆ
ಕಲಿ-ಯುವರು
ಕಲಿ-ಯು-ವುದಕ್ಕೆ
ಕಲಿ-ಯು-ವು-ದ-ರಿಂದ
ಕಲಿ-ಯು-ವುದು
ಕಲಿ-ಯು-ವುದೆಂಬುದ-ರಲ್ಲಿ
ಕಲಿ-ವಿಧ್ವಂಸನ
ಕಲಿ-ಸದ
ಕಲಿ-ಸ-ಬೇಕು
ಕಲಿ-ಸಲ್ಪಡು-ವುದು
ಕಲಿಸಿ
ಕಲಿ-ಸಿ-ಕೊಡ-ಬಲ್ಲೆ
ಕಲಿ-ಸಿ-ಕೊಡು-ವುದು
ಕಲಿ-ಸಿಲ್ಲ
ಕಲಿ-ಸುವ
ಕಲಿ-ಸುವರು
ಕಲಿ-ಸು-ವುದಕ್ಕಾಗಿ
ಕಲಿ-ಸು-ವುದಕ್ಕೆ
ಕಲಿ-ಸುವುದಯ್ಯ
ಕಲಿ-ಸು-ವು-ದೊಂದೇ
ಕಲುಷಂ
ಕಲುಷಿತಗೊಳಿ-ಸುತ್ತವೆ
ಕಲುಷಿ-ತನಾಗೇ
ಕಲುಷಿ-ತ-ವಾಗಿ
ಕಲುಷಿತ-ವಾಗಿದೆ
ಕಲುಷಿ-ತ-ವಾಗಿ-ರು-ವು-ದಿಲ್ಲ
ಕಲುಷಿತ-ವಾಗು-ವುದು
ಕಲೆ
ಕಲೆ-ಗಳ
ಕಲೆ-ಗಳನ್ನು
ಕಲೆ-ಗಳಲ್ಲಿ
ಕಲೆ-ಗಳಲ್ಲಿಯೂ
ಕಲೆ-ಗಳೂ
ಕಲೆಗೆ
ಕಲೆ-ತಾಗ
ಕಲೆ-ತಿದ್ದೇನೆ
ಕಲೆತು
ಕಲೆಯ
ಕಲೆ-ಯನ್ನಾಗಿ
ಕಲೆ-ಯನ್ನು
ಕಲೆ-ಯನ್ನೂ
ಕಲೆ-ಯಲ್ಲಿ
ಕಲೆ-ಯಿಂದಲೇ
ಕಲೆಯು
ಕಲೆ-ಯೊ-ಡನೆ
ಕಲೌ
ಕಲ್ಕತ್ತ
ಕಲ್ಕತ್ತಕ್ಕೆ
ಕಲ್ಕತ್ತಾ
ಕಲ್ಕತ್ತೆಗೆ
ಕಲ್ಕತ್ತೆಯ
ಕಲ್ಕತ್ತೆ-ಯನ್ನು
ಕಲ್ಕತ್ತೆ-ಯಲ್ಲಿ
ಕಲ್ಕತ್ತೆಯಲ್ಲಿದ್ದಾಗ
ಕಲ್ಕತ್ತೆಯಲ್ಲಿಲ್ಲ
ಕಲ್ಕತ್ತೆ-ಯಲ್ಲೇ
ಕಲ್ಕತ್ತೆ-ಯಿಂದ
ಕಲ್ಕತ್ತೆಯೆ
ಕಲ್ಪ
ಕಲ್ಪ-ಗಳಲ್ಲಿಯೂ
ಕಲ್ಪದ
ಕಲ್ಪ-ದಲ್ಲಿ
ಕಲ್ಪನಾ
ಕಲ್ಪ-ನಾ-ಶಕ್ತಿ
ಕಲ್ಪ-ನಾ-ಶಕ್ತಿ-ಯನ್ನೆಲ್ಲಾ
ಕಲ್ಪನೆ
ಕಲ್ಪ-ನೆ-ಗಳು
ಕಲ್ಪ-ನೆ-ಗ-ಳೆಂದು
ಕಲ್ಪ-ನೆಗು
ಕಲ್ಪ-ನೆ-ಯಿಂದ
ಕಲ್ಪ-ನೆಯು
ಕಲ್ಪ-ನೆ-ಯೇ-ನಿದೆ-ಯಯ್ಯಾ
ಕಲ್ಪಿತ-ವಾಗಿದೆ
ಕಲ್ಪಿತಾ
ಕಲ್ಪಿ-ಸಲಾಗು-ವುದು
ಕಲ್ಪಿಸ-ಲಾರದು
ಕಲ್ಪಿ-ಸಿ-ಕೊಂಡ
ಕಲ್ಪಿಸಿ-ಕೊಂಡಂತೆ
ಕಲ್ಪಿ-ಸಿ-ಕೊಂಡಿದ್ದೇವೆ
ಕಲ್ಪಿ-ಸಿ-ಕೊಂಡು
ಕಲ್ಪಿಸಿ-ಕೊಟ್ಟರು
ಕಲ್ಪಿ-ಸಿಕೊ-ಡ-ಬೇಕೆಂದು
ಕಲ್ಪಿ-ಸಿ-ಕೊಳ್ಳಲಿ
ಕಲ್ಪಿ-ಸಿ-ಕೊಳ್ಳ-ಲಿಕ್ಕೆಯೇ
ಕಲ್ಪಿ-ಸಿ-ಕೊಳ್ಳಲು
ಕಲ್ಪಿಸಿ-ಕೊಳ್ಳುವೆವೋ
ಕಲ್ಪಿ-ಸಿದ್ದಾರೆ
ಕಲ್ಪಿ-ಸು-ವುದಕ್ಕೆ
ಕಲ್ಪಿ-ಸು-ವುದು
ಕಲ್ಮಷ
ಕಲ್ಮಷ-ವೆಲ್ಲಾ
ಕಲ್ಯಾಣ
ಕಲ್ಯಾಣ-ಕರ-ವಾಗಿ-ರುತ್ತವೆ
ಕಲ್ಯಾಣ-ಕರ-ವಾದ
ಕಲ್ಯಾಣ-ಕಾರಿ-ಯಾಗುತ್ತದೆ
ಕಲ್ಯಾಣ-ಕೃತ್
ಕಲ್ಯಾಣಕ್ಕಾಗಿ
ಕಲ್ಯಾ-ಣಕ್ಕೆ
ಕಲ್ಯಾಣಕ್ಕೋಸುಗ
ಕಲ್ಯಾಣಕ್ಕೋಸ್ಕರ-ವಾಗಿ
ಕಲ್ಯಾಣಕ್ಕೋಸ್ಕರವೇ
ಕಲ್ಯಾಣ-ಗುಣ-ಗಳ
ಕಲ್ಯಾಣಪ್ರದ-ವಾದ
ಕಲ್ಯಾಣ-ವನ್ನು
ಕಲ್ಯಾಣ-ವನ್ನುಂಟು
ಕಲ್ಯಾಣ-ವಾಗಲಿ
ಕಲ್ಯಾಣ-ವಾಗಿ
ಕಲ್ಯಾಣ-ವಾಗುತ್ತದೆ
ಕಲ್ಯಾಣ-ವಾಗು-ವುದು
ಕಲ್ಯಾಣ-ವಾ-ಯಿತು
ಕಲ್ಲ
ಕಲ್ಲಾರ-ಪುಷ್ಪ-ದಂತೆ
ಕಲ್ಲಿನ
ಕಲ್ಲಿ-ನಂತಿ-ರ-ಲಿಲ್ಲ
ಕಲ್ಲು
ಕಲ್ಲು-ಗಳ
ಕಲ್ಲು-ಗಳನ್ನು
ಕಲ್ಲು-ಗಳಾಗುತ್ತಾರೆ
ಕಲ್ಲು-ಗಳಿಗೂ
ಕಲ್ಲು-ಗಳೆಲ್ಲಾ
ಕಲ್ಲು-ಮರ-ಮಣ್ಣು-ಗಳ
ಕಲ್ಲೆದೆ-ಯವ-ರಾಗಿ-ರ-ಲಿಲ್ಲ
ಕಲ್ಲೆ-ದೆಯು
ಕಲ್ಲೊಂದು
ಕಲ್ಲೋಲ-ವಾಗಿ
ಕಳಂಕವಿ-ದೆ-ಯೆಲ್ಲಾ
ಕಳಂಕವಿಲ್ಲದೆ
ಕಳಂಕ-ವೆಂದರೆ
ಕಳ-ಕೊಂಡ
ಕಳಚಿ-ಕೊಂಡ
ಕಳಚಿದ
ಕಳಚಿ-ಹೋಗು-ವುದು
ಕಳವಳ
ಕಳವಳ-ಕೊಂಡು
ಕಳವಳ-ಗೊಂಡರು
ಕಳವಳ-ಗೊಂಡು
ಕಳವಳ-ವಾಗಿದೆ
ಕಳಿತ
ಕಳಿಸಿ-ಕೊಡುತ್ತಾರೆ
ಕಳಿ-ಸಿದ್ದರು
ಕಳಿ-ಸಿದ್ದು
ಕಳಿ-ಸಿ-ರುವ
ಕಳಿ-ಸುತ್ತಾಳೆ
ಕಳುಹಿದೆ
ಕಳುಹಿ-ಸಲಾಗುತ್ತಿತ್ತು
ಕಳುಹಿ-ಸಲಿ
ಕಳುಹಿ-ಸಲಿಚ್ಛಿ-ಸುವರು
ಕಳುಹಿಸಿ
ಕಳುಹಿಸಿ-ಕೊಟ್ಟು
ಕಳುಹಿಸಿ-ದನು
ಕಳುಹಿಸಿ-ದರು
ಕಳುಹಿಸಿ-ದರೂ
ಕಳುಹಿಸಿ-ದು-ದಾಗಿ
ಕಳುಹಿಸಿದ್ದ-ರಿಂದ
ಕಳುಹಿಸಿದ್ದಾರೆ
ಕಳುಹಿಸು
ಕಳುಹಿ-ಸುತ್ತಾ
ಕಳುಹಿ-ಸುತ್ತಿದ್ದಳು
ಕಳುಹಿಸು-ವನು
ಕಳುಹಿಸು-ವಿ-ರೆಂದು-ಕೊಂಡಿದ್ದೆ
ಕಳುಹಿಸು-ವುದಕ್ಕೆ
ಕಳುಹಿಸು-ವು-ದನ್ನು
ಕಳೆ
ಕಳೆ-ಕಳೆದು
ಕಳೆದ
ಕಳೆ-ದಂತೆ
ಕಳೆ-ದಂತೆಲ್ಲಾ
ಕಳೆ-ದದ್ದಾ-ಯಿತು
ಕಳೆ-ದರೂ
ಕಳೆ-ದರೆ
ಕಳೆ-ದ-ವರು
ಕಳೆ-ದವು
ಕಳೆ-ದವೊ
ಕಳೆ-ದಿದ್ದ
ಕಳೆ-ದಿರ-ಬ-ಹುದು
ಕಳೆದು
ಕಳೆ-ದು-ಕೊಂಡ
ಕಳೆ-ದು-ಕೊಂಡಿದೆ
ಕಳೆ-ದು-ಕೊಂಡಿದ್ದ-ರಿಂದಲೇ
ಕಳೆ-ದು-ಕೊಂಡಿದ್ದರು
ಕಳೆ-ದು-ಕೊಂಡಿದ್ದರೂ
ಕಳೆ-ದು-ಕೊಂಡಿರು-ವರೆ
ಕಳೆ-ದು-ಕೊಂಡಿಲ್ಲ
ಕಳೆ-ದು-ಕೊಂಡು
ಕಳೆ-ದು-ಕೊಂಡೆವು
ಕಳೆ-ದು-ಕೊಂಡೋ
ಕಳೆ-ದು-ಕೊಳ್ಳದೆ
ಕಳೆ-ದು-ಕೊಳ್ಳುತ್ತ
ಕಳೆ-ದು-ಕೊಳ್ಳುವು-ದಕ್ಕೂ
ಕಳೆ-ದು-ಕೊಳ್ಳು-ವುದಕ್ಕೆ
ಕಳೆ-ದು-ಕೊಳ್ಳು-ವುದಕ್ಕೋಸ್ಕರವೂ
ಕಳೆ-ದು-ಕೊಳ್ಳುವೆವು
ಕಳೆ-ದು-ಕೊಳ್ಳುವೆವೊ
ಕಳೆ-ದು-ಬಿಟ್ಟರು
ಕಳೆ-ದು-ಬಿಟ್ಟಿದ್ದೇ-ವೆಂದು
ಕಳೆ-ದು-ಬಿಡುತ್ತಾರೆ
ಕಳೆ-ದು-ಹೋಗಿ
ಕಳೆ-ದು-ಹೋ-ಯಿತು
ಕಳೆ-ಯ-ಗಲಿ-ದರು
ಕಳೆ-ಯನ್ನೆಲ್ಲಾ
ಕಳೆ-ಯ-ಬೇಕಾಗಿ
ಕಳೆ-ಯ-ಬೇಕು
ಕಳೆ-ಯ-ಬೇಕೆನ್ನಿ-ಸು-ವುದು
ಕಳೆ-ಯ-ಬೇಕೊ
ಕಳೆ-ಯಲಿ
ಕಳೆ-ಯಲು
ಕಳೆ-ಯಲೇ-ಬೇಕು
ಕಳೆ-ಯ-ವರನು
ಕಳೆ-ಯಿತು
ಕಳೆ-ಯಿತೊ
ಕಳೆ-ಯಿದೆ
ಕಳೆ-ಯುತ್ತಾ
ಕಳೆ-ಯುತ್ತಾರೆ
ಕಳೆ-ಯುತ್ತಿದ್ದೆವು
ಕಳೆ-ಯುತ್ತಿರು-ವವರು
ಕಳೆ-ಯುತ್ತಿ-ರು-ವುದು
ಕಳೆ-ಯು-ವುದು
ಕಳ್ಳ
ಕಳ್ಳ-ತನಕೆ
ಕಳ್ಳ-ತನ-ವಿಲ್ಲ-ದಿ-ರುವಿಕೆ
ಕಳ್ಳನು
ಕಳ್ಳ-ರನ್ನು
ಕಳ್ಳ-ರಲ್ಲಿ
ಕವಡೆಯೂ
ಕವನ
ಕವನಕ್ಕಿದೆ
ಕವನ-ಗಳ
ಕವನ-ಗಳನ್ನು
ಕವನ-ಗಳು
ಕವನ-ಗಳೂ
ಕವನ-ಗಳೊಂದಿಗೂ
ಕವನದ
ಕವನ-ದಲ್ಲಿ
ಕವನ-ದಲ್ಲಿದೆ
ಕವನ-ದಲ್ಲಿಯೂ
ಕವನ-ರೂಪದ
ಕವನ-ವನ್ನು
ಕವನ-ವಾ-ಯಿತು
ಕವಯೋ
ಕವಲೊಡೆದ
ಕವಲೊ-ಡೆದು
ಕವಳ
ಕವಿ
ಕವಿತೆ
ಕವಿ-ತೆ-ಯಲ್ಲ
ಕವಿ-ತೆ-ಯಲ್ಲೂ
ಕವಿದ
ಕವಿ-ದಿದ್ದರೂ
ಕವಿ-ದು-ಕೊಂಡಿತು
ಕವಿಯ
ಕವಿ-ಯನ್ನು
ಕವಿ-ಯಾ-ಗಲೆ-ಳ-ಸುತ್ತ
ಕವಿಯು
ಕವಿ-ಯೊಬ್ಬ
ಕವಿ-ರಾಜರ
ಕವಿ-ರಾಜ-ರಿಂದ
ಕಶ್ಚಿತ್
ಕಷ್ಟ
ಕಷ್ಟ-ಕಾಲಕ್ಕೆ
ಕಷ್ಟಕ್ಕಾಗಿ
ಕಷ್ಟಕ್ಕೆ
ಕಷ್ಟ-ಗಳ
ಕಷ್ಟ-ಗಳನ್ನು
ಕಷ್ಟ-ಗಳಲ್ಲಿ
ಕಷ್ಟ-ಗಳಿಗೆ
ಕಷ್ಟ-ಜೀವಿ-ಗಳಾ-ದ-ವರು
ಕಷ್ಟ-ತಮ-ವಾ-ದುದು
ಕಷ್ಟದ
ಕಷ್ಟ-ದಲ್ಲಿ-ರುವ-ವ-ರಿ-ಗಾಗಿ
ಕಷ್ಟ-ನಿಷ್ಟುರ-ಗಳ
ಕಷ್ಟ-ಪಟ್ಟು
ಕಷ್ಟ-ಪಟ್ಟು-ಕೊಂಡು
ಕಷ್ಟ-ವನ್ನು
ಕಷ್ಟ-ವಲ್ಲ
ಕಷ್ಟ-ವಾಗಿತ್ತು
ಕಷ್ಟ-ವಾಗುತ್ತದೆ
ಕಷ್ಟ-ವಾ-ಗು-ವು-ದಿಲ್ಲ
ಕಷ್ಟ-ವಾಗು-ವುದು
ಕಷ್ಟ-ವಾದ
ಕಷ್ಟ-ವಾದುದಾವು-ದಿದೆ
ಕಷ್ಟ-ವಿರು-ವುದು
ಕಷ್ಟವೂ
ಕಷ್ಟ-ವೆಂದು
ಕಷ್ಟ-ಸಹಿಷ್ಣು-ಗಳು
ಕಷ್ಟ-ಸಾಧ್ಯ-ವಾದ
ಕಸ
ಕಸುಬನ್ನು
ಕಸುಬಿನ
ಕಸುಬು
ಕಸುಬು-ಗಳಲ್ಲಿಯೂ
ಕಸೂತಿ
ಕಸ್ಯಾದ್ಯ
ಕಹತ
ಕಹಳೆ-ಯನ್ನು
ಕಹಳೆಯನ್ನೂ-ದಿದ್ದಾರೆ
ಕಹಳೆಯು
ಕಹಿ
ಕಹಿ-ಗೋ-ಳಿನಲಿ
ಕಹಿ-ನೆನ-ಪನ್ನು
ಕಹಿಯ
ಕಹಿಯಾ-ದಾಗಲೂ
ಕಾ
ಕಾಂಗ್ರೆಸ್
ಕಾಂಚನ
ಕಾಂಚನಕ್ಕೆ
ಕಾಂಚನ-ಗಳನ್ನು
ಕಾಂಚನ-ಗಳಲ್ಲಿ
ಕಾಂಚನದ
ಕಾಂಚನ-ದಲ್ಲಿ
ಕಾಂಚನ-ದಾಸೆ
ಕಾಂಚನ-ವೆಂಬುವಾಸೆ-ಗಳಿಂದ
ಕಾಂಚನವೊ
ಕಾಂತ-ದೇವ್
ಕಾಂತಮ್
ಕಾಂತಿ
ಕಾಂತಿ-ಯನ್ನು
ಕಾಂತಿ-ಯಿಂದ
ಕಾಂಪೆ
ಕಾಂಬಾತ
ಕಾಕತಾಳೀಯ
ಕಾಕೆ
ಕಾಗದ
ಕಾಗದ-ಪತ್ರ-ಗಳನ್ನು
ಕಾಗದ-ವನ್ನು
ಕಾಜ
ಕಾಟನ್ರಂತಹ
ಕಾಟನ್ರ-ವ-ರನ್ನು
ಕಾಟ-ವಾ-ಯಿತು
ಕಾಡಿ
ಕಾಡಿಗೆ
ಕಾಡಿ-ಗೆ-ಯಂತಹ
ಕಾಡಿ-ನಲ್ಲಿ
ಕಾಡು
ಕಾಡು-ಗಳಲ್ಲಿ
ಕಾಡು-ತಿ-ರು-ವುದು
ಕಾಡುತ್ತಲೇ
ಕಾಡುತ್ತಿದೆ
ಕಾಡು-ಮನುಷ್ಯರು
ಕಾಡು-ಮೇಡು-ಗಳಲ್ಲಿಯೂ
ಕಾಣ
ಕಾಣದ
ಕಾಣ-ದಿದೆ
ಕಾಣ-ದಿದ್ದಲ್ಲಿ
ಕಾಣ-ದಿ-ರಲು
ಕಾಣ-ದಿರು
ಕಾಣದು
ಕಾಣ-ದುವೆ
ಕಾಣದೆ
ಕಾಣದೇ
ಕಾಣ-ದೇನು
ಕಾಣ-ಬ-ರ-ಲಿಲ್ಲ
ಕಾಣ-ಬ-ರುತ್ತದೆ
ಕಾಣ-ಬರುತ್ತಿದೆ
ಕಾಣ-ಬ-ರುವ
ಕಾಣ-ಬ-ರು-ವು-ದಿಲ್ಲ
ಕಾಣ-ಬ-ರು-ವು-ದಿಲ್ಲ-ವೆಂದೂ
ಕಾಣ-ಬ-ರು-ವುದು
ಕಾಣ-ಬ-ರು-ವುದೋ
ಕಾಣ-ಬಲ್ಲನು
ಕಾಣ-ಬಹು-ದಾಗಿದೆ
ಕಾಣ-ಬ-ಹುದು
ಕಾಣ-ಬೇಕು
ಕಾಣ-ಬೇಕೆಂದು
ಕಾಣ-ಲಾ-ಗುತ್ತದೆ
ಕಾಣ-ಲಾರಿರಿ
ಕಾಣ-ಲಿಲ್ಲ
ಕಾಣಲು
ಕಾಣ-ಲು-ಬ-ಹುದು
ಕಾಣ-ಸಿಗುತ್ತಿರುವ
ಕಾಣ-ಸಿಗು-ವುದು
ಕಾಣಿ-ಸ-ಲಿಲ್ಲ
ಕಾಣಿ-ಸಿ-ಕೊಂಡರು
ಕಾಣಿ-ಸಿ-ಕೊಳ್ಳಲು
ಕಾಣಿಸಿ-ಕೊಳ್ಳುತ್ತಾರೆ
ಕಾಣಿ-ಸಿ-ಕೊಳ್ಳುತ್ತಿದ್ದರು
ಕಾಣಿಸಿ-ಕೊಳ್ಳುವನು
ಕಾಣಿಸಿ-ಕೊಳ್ಳುವರು
ಕಾಣಿ-ಸಿತು
ಕಾಣಿ-ಸಿ-ದಳು
ಕಾಣಿ-ಸುತ್ತದೆ
ಕಾಣಿ-ಸುತ್ತಿದೆ
ಕಾಣಿ-ಸುತ್ತಿಲ್ಲ
ಕಾಣಿ-ಸುತ್ತಿಲ್ಲವೆ
ಕಾಣಿ-ಸುವ
ಕಾಣಿ-ಸುವನು
ಕಾಣಿ-ಸುವ-ರೆಂದರೆ
ಕಾಣಿ-ಸು-ವುದು
ಕಾಣು
ಕಾಣು-ತಲ-ವರು
ಕಾಣು-ತಿದೆ
ಕಾಣು-ತಿ-ರುವೆ
ಕಾಣು-ತಿಹ
ಕಾಣು-ತಿ-ಹುದು
ಕಾಣು-ತಿಹೆ
ಕಾಣುತ್ತದೆ
ಕಾಣುತ್ತದೊ
ಕಾಣುತ್ತವೆ
ಕಾಣುತ್ತಾನೆ
ಕಾಣುತ್ತಿತ್ತು
ಕಾಣುತ್ತಿದೆ
ಕಾಣುತ್ತಿದ್ದರು
ಕಾಣುತ್ತಿದ್ದ-ರೆಂಬು-ದನ್ನು
ಕಾಣುತ್ತಿದ್ದೀರಿ
ಕಾಣುತ್ತಿರು-ವ-ರೆಂಬ
ಕಾಣುತ್ತಿರುವೆ
ಕಾಣುತ್ತೀರಿ
ಕಾಣುತ್ತೇವೆ
ಕಾಣುವ
ಕಾಣು-ವಂತೆ
ಕಾಣು-ವ-ನಾ-ತನು
ಕಾಣು-ವನು
ಕಾಣು-ವರು
ಕಾಣು-ವರೋ
ಕಾಣು-ವು-ದ-ರಲ್ಲೇ
ಕಾಣು-ವುದ-ರೊಂದಿಗೆ
ಕಾಣು-ವು-ದಿಲ್ಲ
ಕಾಣು-ವು-ದಿಲ್ಲವೆ
ಕಾಣು-ವು-ದಿಲ್ಲವೇ
ಕಾಣು-ವುದು
ಕಾಣು-ವುದೇ
ಕಾಣು-ವುವು
ಕಾಣುವೆ
ಕಾಣು-ವೆ-ಡೆ-ಯೊಳು
ಕಾಣು-ವೆನು
ಕಾಣು-ವೆ-ಳೆ-ಯರ
ಕಾಣು-ವೆವೆ
ಕಾಣೆ
ಕಾಣೆನು
ಕಾಣ್ಕೆಯ-ನೆಂದು
ಕಾತರ
ಕಾತರ-ಗೊಂಡಿತು
ಕಾತರ-ಗೊಂಡಿದೆ
ಕಾತರ-ನಾಗಿದ್ದ
ಕಾತರನು
ಕಾತರ-ರಾ-ಗಿ-ರುವರು
ಕಾತರ-ರೆಲ್ಲರು
ಕಾತರಿ-ಸುತ್ತಿದ್ದಾಗ
ಕಾದ
ಕಾದಂಬರಿ
ಕಾದಂಬರಿ-ಗಳನ್ನು
ಕಾದನು
ಕಾದಾಟ
ಕಾದಾಡಿ
ಕಾದಾಡಿ-ದರೆ
ಕಾದಾಡುತ್ತಾ
ಕಾದಾಡುತ್ತಿರು-ವರು
ಕಾದಿದ್ದು
ಕಾದಿರ-ಬೇಕೆಂದು
ಕಾದಿಹೆ
ಕಾದು
ಕಾದು-ಕೊಂಡಿದ್ದನು
ಕಾದು-ಕೊಂಡಿ-ರ-ಲಾರೆ
ಕಾದು-ಕೊಂಡಿರುತ್ತವೆ
ಕಾನ-ನವ
ಕಾನೂ-ನನ್ನು
ಕಾನೂ-ನಿನ
ಕಾನೂನು
ಕಾನೂನು-ಗಳನ್ನು
ಕಾನೂನು-ಗಳೇ
ಕಾಪಟ್ಯ
ಕಾಪಾಡ-ಬೇಕು
ಕಾಪಾ-ಡಲು
ಕಾಪಾಡಿಕೊಳ್ಳು-ವುದಕ್ಕಾಗಿ
ಕಾಪಾಡುತ್ತಿದ್ದೇವೆ
ಕಾಪಾಡು-ವುದಕ್ಕಾಗಿ
ಕಾಪುರುಷ-ತೆಯ
ಕಾಪುರುಷಾಣಾಮ್
ಕಾಪುರುಷ್
ಕಾಪೇ-ಧರಾ
ಕಾಬೂಲಿ-ಗಳು
ಕಾಮ
ಕಾಮ-ಕಾಂಚನ
ಕಾಮ-ಕಾಂಚನ-ಗಳಲ್ಲಿ
ಕಾಮ-ಕಾಂಚನ-ಗಳಿಂದ
ಕಾಮ-ಕಾಂಚನ-ಗಳಿಗೆ
ಕಾಮ-ಕಾಂಚನದ
ಕಾಮ-ಕಾಂಚನ-ಬದ್ಧ-ರಾದ
ಕಾಮ-ಕಾಂಚನ-ವನ್ನು
ಕಾಮಕ್ಕೆ
ಕಾಮಕ್ರೋಧ
ಕಾಮಕ್ರೋಧ-ಗಳ
ಕಾಮಕ್ರೋಧ-ಗಳಿಗೆ
ಕಾಮನೆ
ಕಾಮ-ಮೋ-ಹಿತ-ರಾಗಿ-ಬಿಡುತ್ತಾರೆ
ಕಾಮ-ರ-ಹಿತ-ವಾದ
ಕಾಮ-ವನ್ನು
ಕಾಮ-ವಿ-ರುವುದೊ
ಕಾಮ-ವಿಲ್ಲ
ಕಾಮವು
ಕಾಮ-ವೆಲ್ಲ
ಕಾಮಾಖ್ಯ-ದಲ್ಲೇನು
ಕಾಮಾದಿ
ಕಾಮಾಸಕ್ತಿ
ಕಾಮಾಸಕ್ತಿ-ಗಳನ್ನು
ಕಾಮಿನಿ
ಕಾಮಿನಿ-ಕಾಂಚನ
ಕಾಮಿನಿ-ಕಾಂಚನದ
ಕಾಮಿನಿ-ಕಾಂಚನ-ದಿಂದುಟಾಗುವ
ಕಾಮಿನಿಯ
ಕಾಮಿನಿ-ಯಲ್ಲಿ
ಕಾಮಿನೀ-ಕಾಂಚನಕ್ಕೆ
ಕಾಮ್ಯ
ಕಾಮ್ಯಾನಾಂ
ಕಾಯ
ಕಾಯ-ಬಲ್ಲನು
ಕಾಯ-ಬೇಕು
ಕಾಯಸ್ಥ
ಕಾಯಸ್ಥ-ರನ್ನು
ಕಾಯಸ್ಥರು
ಕಾಯಾ
ಕಾಯಿದೆ
ಕಾಯಿದೆ-ಗನು-ಸಾರ
ಕಾಯಿದೆ-ಗಳನ್ನು
ಕಾಯಿದೆ-ಗಳನ್ನೂ
ಕಾಯಿದೆ-ಗ-ಳಿದ್ದೇ
ಕಾಯಿದೆ-ಗಳು
ಕಾಯಿ-ಪಲ್ಯ-ಗಳನ್ನು
ಕಾಯಿಲೆ-ಯಲ್ಲಿ
ಕಾಯಿಲೆ-ಯವರ
ಕಾಯಿಸಿ
ಕಾಯಿ-ಸುತ್ತಿದ್ದುದು
ಕಾಯುತ-ಲಿ-ರುವೆ
ಕಾಯುತಿತ್ತು
ಕಾಯು-ತಿದ್ದರು
ಕಾಯು-ತಿ-ಹುದು
ಕಾಯುತ್ತಿದ್ದರು
ಕಾಯುತ್ತಿದ್ದಳು
ಕಾಯುತ್ತಿರು-ವುದೇ
ಕಾಯುವ
ಕಾಯ್ದಿ-ರಿಸ-ಲಾಗಿದೆ
ಕಾರಣ
ಕಾರ-ಣ-ಕರ್ತ-ನೆಂದು
ಕಾರ-ಣಕೆ
ಕಾರ-ಣಕ್ಕಾಗಿ
ಕಾರ-ಣಕ್ಕಾಗಿಯೇ
ಕಾರ-ಣ-ಗಳ
ಕಾರ-ಣ-ಗಳನ್ನು
ಕಾರ-ಣ-ಗಳನ್ನೆಲ್ಲಾ
ಕಾರ-ಣ-ಗಳಲ್ಲೊಂದಾ-ಯಿತು
ಕಾರ-ಣ-ಗಳಿಂದ
ಕಾರ-ಣ-ಗಳಿವೆ
ಕಾರ-ಣ-ಗಳೂ
ಕಾರ-ಣ-ಗಳೇ
ಕಾರ-ಣದ
ಕಾರ-ಣ-ದಲ್ಲಿ
ಕಾರ-ಣ-ದಿಂದ
ಕಾರ-ಣ-ದಿಂದಲೇ
ಕಾರ-ಣ-ದಿಂದಲೋ
ಕಾರ-ಣ-ಧಾರ
ಕಾರ-ಣ-ನೆಂದು
ಕಾರ-ಣ-ರಾಗುವರು
ಕಾರ-ಣ-ರಾದ-ವರು
ಕಾರ-ಣರು
ಕಾರ-ಣ-ರೂಪ-ವಾಗಿ
ಕಾರ-ಣ-ರೂಪ-ವಾಗಿ-ರುತ್ತವೆ
ಕಾರ-ಣ-ಳಾಗು-ವಳು
ಕಾರ-ಣ-ವನ್ನು
ಕಾರ-ಣ-ವಲ್ಲ
ಕಾರ-ಣ-ವಸ್ತು-ವಿ-ನಿಂದ
ಕಾರ-ಣ-ವಾಗಿದೆ
ಕಾರ-ಣ-ವಾಗಿ-ರುವ
ಕಾರ-ಣ-ವಾಗುತ್ತದೆ
ಕಾರ-ಣ-ವಾಗು-ವುದು
ಕಾರ-ಣ-ವಾದ
ಕಾರ-ಣ-ವಿಲ್ಲ
ಕಾರ-ಣವೂ
ಕಾರ-ಣವೆ
ಕಾರ-ಣ-ವೆಂದು
ಕಾರ-ಣವೇ
ಕಾರ-ಣ-ವೇ-ನಿದೆ
ಕಾರ-ಣ-ವೇನು
ಕಾರ-ಣ-ವೇ-ನೆಂದು
ಕಾರವು
ಕಾರಿ-ರು-ಳಿನ
ಕಾರೆ
ಕಾರ್ಖಾನೆ
ಕಾರ್ಖಾನೆ-ಗಳನ್ನು
ಕಾರ್ನವಾಲೀಸ್
ಕಾರ್ಮೋಡ
ಕಾರ್ಯ
ಕಾರ್ಯ-ಕಾರಣ
ಕಾರ್ಯ-ಕಾರ-ಣ-ಗಳ
ಕಾರ್ಯ-ಕಾರ-ಣದ
ಕಾರ್ಯ-ಕಾರ-ಣ-ಸಂಬಂಧ
ಕಾರ್ಯಕ್ಕೂ
ಕಾರ್ಯಕ್ಕೆ
ಕಾರ್ಯಕ್ರಮ
ಕಾರ್ಯಕ್ರ-ಮ-ಗಳನ್ನು
ಕಾರ್ಯಕ್ರಮ-ಗಳೆಲ್ಲಾ
ಕಾರ್ಯಕ್ಷೇತ್ರ
ಕಾರ್ಯ-ಗತ
ಕಾರ್ಯ-ಗಳ
ಕಾರ್ಯ-ಗಳನ್ನು
ಕಾರ್ಯ-ಗಳನ್ನೆಲಾ
ಕಾರ್ಯ-ಗಳಲ್ಲಿ
ಕಾರ್ಯ-ಗಳಲ್ಲಿಯೂ
ಕಾರ್ಯ-ಗಳಿಂದಲೂ
ಕಾರ್ಯ-ಗಳಿಗೆ
ಕಾರ್ಯ-ಗಳು
ಕಾರ್ಯತಃ
ಕಾರ್ಯ-ತತ್ಪರನ
ಕಾರ್ಯ-ತತ್ಪರ-ರನ್ನಾಗಿ
ಕಾರ್ಯ-ತತ್ಪರ-ರಾಗಿ
ಕಾರ್ಯ-ತತ್ಪರ-ರಾಗುತ್ತಾರೆ
ಕಾರ್ಯದ
ಕಾರ್ಯ-ದರ್ಶಿ-ಗಳಾಗಿಯೂ
ಕಾರ್ಯ-ದಲ್ಲಿ
ಕಾರ್ಯ-ದಲ್ಲಿಯೂ
ಕಾರ್ಯ-ದಿಂದ
ಕಾರ್ಯ-ದಿಂದಲೂ
ಕಾರ್ಯ-ನಿರತ-ನೆಂದರೆ
ಕಾರ್ಯ-ಪಟು-ಗಳು
ಕಾರ್ಯಪ್ರ-ವೃತ್ತಿಯ
ಕಾರ್ಯ-ಭಾರ-ವನ್ನು
ಕಾರ್ಯ-ಮಾಡಿ-ಕೊಂಡು
ಕಾರ್ಯ-ರಭ-ಸಕ್ಕೆ
ಕಾರ್ಯ-ರೂಪಕ್ಕೆ
ಕಾರ್ಯ-ರೂಪ-ದಲ್ಲಿ
ಕಾರ್ಯ-ರೂಪ-ವಾಗಿ
ಕಾರ್ಯ-ವನ್ನು
ಕಾರ್ಯ-ವಾಗಲಿ
ಕಾರ್ಯ-ವಿ-ಭಾಗ
ಕಾರ್ಯವು
ಕಾರ್ಯವೂ
ಕಾರ್ಯ-ವೆಂದು
ಕಾರ್ಯ-ವೇನೊ
ಕಾರ್ಯ-ವೈಖರಿ
ಕಾರ್ಯ-ವೈಖರಿ-ಯನ್ನಂತೂ
ಕಾರ್ಯ-ವೈಖರಿ-ಯಲ್ಲಿ
ಕಾರ್ಯ-ಶೀಲ-ವಾಗು-ವು-ದನ್ನು
ಕಾರ್ಯ-ಸಾ-ಧನೆ
ಕಾರ್ಯ-ಸಿದ್ಧ
ಕಾರ್ಯಾ
ಕಾರ್ಯಾ-ಚರ-ಣೆಗೆ
ಕಾರ್ಯಾ-ರಂಭ
ಕಾರ್ಯಾ-ರಂಭ-ವಾ-ಯಿತು
ಕಾರ್ಯಾರ್ಥ-ವಾಗಿ
ಕಾರ್ಯೋನ್ಮುಖ-ರಾಗಿ
ಕಾರ್ಯೋನ್ಮುಖ-ರಾಗುವಿರಿ
ಕಾಲ
ಕಾಲ-ಕಳೆ-ದಂತೆಲ್ಲಾ
ಕಾಲ-ಕಳೆ-ದು-ಹೋಗುತ್ತಿತ್ತು
ಕಾಲ-ಕಳೆ-ಯುವನೋ
ಕಾಲ-ಕಾಲಕ್ಕೆ
ಕಾಲಕ್ಕಿಂತ
ಕಾಲಕ್ಕೂ
ಕಾಲಕ್ಕೆ
ಕಾಲಕ್ರಮ-ದಲ್ಲಿ
ಕಾಲಕ್ರಮ-ದಿಂದ
ಕಾಲಕ್ರಮೇಣ
ಕಾಲ-ಗಳನ್ನು
ಕಾಲ-ಗಳಲ್ಲಿ
ಕಾಲ-ಗಳಲ್ಲಿಯೂ
ಕಾಲ-ಗ-ಳಲ್ಲೂ
ಕಾಲ-ಗಳೇ
ಕಾಲ-ಚಕ್ರಕ್ಕೆ
ಕಾಲ-ಚಕ್ರದ
ಕಾಲ-ಡಿ-ಯಲಿ
ಕಾಲದ
ಕಾಲ-ದಂತೆ
ಕಾಲ-ದ-ಲೆ-ಗಳ
ಕಾಲ-ದಲ್ಲಿ
ಕಾಲ-ದಲ್ಲಿದ್ದ
ಕಾಲ-ದಲ್ಲಿದ್ದಂತೆ
ಕಾಲ-ದಲ್ಲಿನ
ಕಾಲ-ದಲ್ಲಿಯೂ
ಕಾಲ-ದಲ್ಲಿಯೇ
ಕಾಲ-ದಲ್ಲಿ-ರುವ
ಕಾಲ-ದಲ್ಲಿ-ರು-ವಿರಿ
ಕಾಲ-ದಲ್ಲೂ
ಕಾಲ-ದಲ್ಲೇ
ಕಾಲ-ದ-ವ-ರೆಗೂ
ಕಾಲ-ದ-ವರೆಗೆ
ಕಾಲ-ದ-ವಿರತ
ಕಾಲ-ದವು
ಕಾಲ-ದಿಂದ
ಕಾಲ-ದಿಂದಲೂ
ಕಾಲ-ದಿಂದೀಚೆಗೆ
ಕಾಲ-ದೇಶ-ಗಳನ್ನು
ಕಾಲ-ದೇಶ-ಗಳಿಲ್ಲ
ಕಾಲ-ದೇಶ-ಗಳು
ಕಾಲ-ದೇಶವ
ಕಾಲ-ಧರ್ಮಕ್ಕೆ
ಕಾಲನ
ಕಾಲನ್ನು
ಕಾಲ-ಪರಿಸ್ಥಿತಿಗೆ
ಕಾಲಪ್ರವಾಹ-ದಲ್ಲಿ
ಕಾಲಪ್ರವಾಹ-ವನ್ನು
ಕಾಲ-ಮಾನ
ಕಾಲ-ಮೇಲೆ
ಕಾಲ-ಮೇಲೇ
ಕಾಲರಾ
ಕಾಲ-ರೂಪಿ
ಕಾಲ-ವನ್ನು
ಕಾಲ-ವಾಗಿ
ಕಾಲ-ವಾಗಿದೆ
ಕಾಲ-ವಾಗು-ವರು
ಕಾಲ-ವಾಗು-ವಿರಿ
ಕಾಲ-ವಾಗು-ವುದು
ಕಾಲ-ವಾದ
ಕಾಲ-ವಾದರು
ಕಾಲ-ವಿದ್ದರೆ
ಕಾಲ-ವಿಲ್ಲ
ಕಾಲ-ವಿಳಂಬ
ಕಾಲವು
ಕಾಲ-ಹ-ರಣ
ಕಾಲ-ಹೀನ
ಕಾಲಾಂತ-ರ-ದಲ್ಲಿ
ಕಾಲಾ-ಕಾಲ-ಗಳಾ-ಗಲಿ
ಕಾಲಾ-ಕಾಲ-ವಿಲ್ಲ
ಕಾಲಾಗಿ-ರುವನೊ
ಕಾಲಾ-ನಂತರ
ಕಾಲಾ-ನಂತರ-ದಲ್ಲಿ
ಕಾಲಾ-ನಂತರವೂ
ಕಾಲಿ
ಕಾಲಿಂದೊದೆದುನೂ-ಕಲಿ
ಕಾಲಿಗೆ
ಕಾಲಿ-ಡು-ವುವು
ಕಾಲಿನ
ಕಾಲಿ-ನಡಿ
ಕಾಲಿ-ನಲ್ಲಿ
ಕಾಲಿ-ನಲ್ಲಿಯೂ
ಕಾಲು
ಕಾಲು-ಕೆರೆ-ದ-ವನು
ಕಾಲು-ಕೆರೆ-ದ-ವ-ರಲ್ಲ
ಕಾಲು-ಗಳ
ಕಾಲು-ಗಳನ್ನು
ಕಾಲು-ಗಳು
ಕಾಲು-ತನಕ
ಕಾಲು-ನ-ಡಿಗೆ-ಯಲ್ಲಿ
ಕಾಲು-ಭಾಗ
ಕಾಲು-ವೆಯ
ಕಾಲು-ವೆ-ಯನ್ನು
ಕಾಲೇಜಿನ
ಕಾಲೇಜು
ಕಾಲೇಜು-ಗಳಲ್ಲೆಲ್ಲಾ
ಕಾಲೇಜು-ಗಳೂ
ಕಾಲೇನಾತ್ಮನಿ
ಕಾಲೊಂದನ್ನು
ಕಾಲೋಹ್ಯಯಂ
ಕಾಲ್ಕಿತ್ತಿವೆ
ಕಾಲ್ಗಳು
ಕಾಲ್ನ-ಡಿಗೆ-ಯಲ್ಲಿ
ಕಾಳ-ಕತ್ತ-ಲಿ-ನಲ್ಲಿ
ಕಾಳಜಿ-ಯಿ-ರ-ಲಿಲ್ಲ
ಕಾಳ-ನಾಗರ
ಕಾಳ-ಮೇಘ-ಗಳೆಲ್ಲ
ಕಾಳಿ
ಕಾಳಿ-ಕಾ-ದೇವಿ
ಕಾಳಿ-ಕಾ-ಮಾತೆಯ
ಕಾಳಿ-ಕಾ-ಮಾತೆ-ಯನ್ನು
ಕಾಳಿ-ಘಾಟ್-ನಲ್ಲಿ
ಕಾಳಿ-ದಾಸನ
ಕಾಳಿ-ದೇವಸ್ಥಾನ-ದಲ್ಲಿ
ಕಾಳಿ-ಮಾತೆ
ಕಾಳಿ-ಮಾತೆಯ
ಕಾಳಿ-ಮಾತೆ-ಯನ್ನು
ಕಾಳಿ-ಮೆ-ಯಲಿ
ಕಾಳಿಯ
ಕಾಳಿ-ಯನ್ನು
ಕಾಳಿಯು
ಕಾಳಿ-ಯೆಂದೆ-ನಿ-ಸು-ವುದು
ಕಾಳೀ-ಮಂದಿ-ರದ
ಕಾಳೀ-ಮಾತೆಯ
ಕಾಳು
ಕಾವಲಿ-ನ-ವರು
ಕಾವಿ
ಕಾವಿ-ಬಟ್ಟೆ
ಕಾವಿಯ
ಕಾವಿ-ಯನ್ನುಟ್ಟ
ಕಾವಿ-ಯ-ಬಟ್ಟೆ-ಯನ್ನು
ಕಾವೇರಿದ
ಕಾವ್ಯ
ಕಾವ್ಯ-ಕಲಾ-ಪ-ಗಳು
ಕಾವ್ಯಕ್ಕೆ
ಕಾವ್ಯದ
ಕಾವ್ಯ-ದೇವ-ತೆ-ಯನ್ನು
ಕಾವ್ಯಪ್ರತಿ-ಭೆಯ
ಕಾವ್ಯ-ಮಯ-ವಾದ
ಕಾವ್ಯ-ಮಯ-ವಾ-ಯಿತು
ಕಾವ್ಯ-ರಚ-ನೆಯ
ಕಾವ್ಯ-ವನ್ನು
ಕಾವ್ಯ-ವನ್ನೇ
ಕಾವ್ಯವು
ಕಾಶಿ-ಪುರದ
ಕಾಶಿ-ಯಂತೆ
ಕಾಶೀ-ಪುರ
ಕಾಶೀಪು-ರಕ್ಕೆ
ಕಾಶೀ-ಪುರದ
ಕಾಶೀ-ಪುರ-ದಲ್ಲಿ
ಕಾಶ್ಮೀ-ರ-ದಲ್ಲಿ
ಕಾಶ್ಮೀ-ರ-ದಲ್ಲಿದ್ದಾಗ
ಕಾಶ್ಮೀರ-ದಿಂದ
ಕಾಸನ್ನು
ಕಾಸನ್ನೂ
ಕಾಸಿಲ್ಲದೆ
ಕಿಂ
ಕಿಂಕರನು
ಕಿಂಕರರೊ
ಕಿಂಚನ
ಕಿಂಚಿತ್
ಕಿಂಚಿತ್ತಾ-ದರೂ
ಕಿಂಚಿತ್ತಿದ್ದರೂ
ಕಿಂಚಿತ್ತು
ಕಿಂಚಿತ್ತೂ
ಕಿಂತು
ಕಿಂವಾ-ದೃಷ್ಟಂ
ಕಿಕ್ಕಿರಿ-ದಿದ್ದಿತು
ಕಿಟಕಿ
ಕಿಟಕಿಯ
ಕಿಡಿ
ಕಿಡಿ-ಗರೆ-ಯುತಿವೆ
ಕಿಡಿ-ಗಳು
ಕಿಡಿ-ಯಂತೆ
ಕಿತ್ತಡಿ
ಕಿತ್ತಡಿ-ಯಿಡುತ
ಕಿತ್ತಾಡುತ್ತಾ
ಕಿತ್ತು
ಕಿತ್ತು-ಕೊಂಡರು
ಕಿತ್ತು-ಹಾಕಿ
ಕಿತ್ತು-ಹಾಕುವುದು
ಕಿತ್ತೆಸೆದು
ಕಿತ್ತೊಗೆದ
ಕಿತ್ತೊಗೆದು
ಕಿತ್ತೊಗೆದೆ
ಕಿತ್ತೊಗೆ-ಯಿರಿ
ಕಿತ್ತೊಗೆ-ಯು-ವುದು
ಕಿದನ
ಕಿನ್ತು
ಕಿನ್ನರ
ಕಿನ್ನರ-ಕಂಠ-ದಲ್ಲಿ
ಕಿಮ-ಕೃತಂ
ಕಿರಣ
ಕಿರಣ-ಗಳ
ಕಿರಣ-ಗಳಿಂದ
ಕಿರಣ-ಗಳಿಗೆ
ಕಿರಣ-ಗಳು
ಕಿರಣ-ವೊಂದು
ಕಿರೀಟ-ವನ್ನು
ಕಿರುಕುಳ
ಕಿರುಗ-ವನ
ಕಿವಾ
ಕಿವಿ
ಕಿವಿ-ಕೊ-ಡದೆ
ಕಿವಿ-ಕೊಡುತ್ತಾರೆ
ಕಿವಿ-ಗಳನ್ನು
ಕಿವಿ-ಗಳಿಂದಲೇ
ಕಿವಿ-ಗಳು
ಕಿವಿಗೂ
ಕಿವಿಗೆ
ಕಿವಿ-ಗೊಟ್ಟರೆ
ಕಿವಿ-ಗೊಟ್ಟಿದ್ದರ
ಕಿವಿ-ಗೊ-ಡದಿ-ರ-ಲಾರೆ
ಕಿವಿ-ಗೊ-ಡಲು
ಕಿವಿ-ಗೊಡು-ವಂತೆ
ಕಿವಿ-ಗೊ-ಡು-ವರು
ಕಿವಿ-ಗೊಡುವ-ರೇನು
ಕಿವಿ-ಗೊಡು-ವು-ದಿಲ್ಲ
ಕಿವಿ-ಯನು
ಕಿವಿ-ಯನ್ನಿರಿ-ಯುತ್ತದೆ
ಕಿವಿ-ಯಲ್ಲಿ
ಕಿವಿ-ಯಾರೆ
ಕಿವಿ-ಯಿಂದ
ಕಿವುಡ-ನಾದರೂ
ಕಿವು-ಡನೆ-ನಿಸ-ವರ
ಕಿಷ್ಕಂಧ-ವಾದ
ಕಿಷ್ಟ
ಕಿಷ್ಟ-ನನ್ನು
ಕಿಷ್ಟನು
ಕೀಟ
ಕೀಟ-ಗಳನ್ನು
ಕೀಟ-ದಲಿ
ಕೀಟ-ದಲ್ಲಿಯು
ಕೀಟ-ದಿಂದ
ಕೀಟ-ವನ್ನು
ಕೀಟಾಧಮ
ಕೀರ್ತನ-ವನ್ನು
ಕೀರ್ತನೆ
ಕೀರ್ತನೆ-ಗಳನ್ನು
ಕೀರ್ತನೆ-ಗಳೇ
ಕೀರ್ತನೆ-ಯನ್ನು
ಕೀರ್ತಿ
ಕೀರ್ತಿ-ಉ-ದಾತ್ತ
ಕೀರ್ತಿ-ಗಳ
ಕೀರ್ತಿಯ
ಕೀರ್ತಿ-ಯನು
ಕೀರ್ತಿ-ಯನ್ನು
ಕೀರ್ತಿ-ಯಿಂದೇನು
ಕೀರ್ತಿ-ಲಾಲ-ಸೆಗೆ
ಕೀಳ-ಬೇಕು
ಕೀಳಾಗಿ
ಕೀಳು
ಕೀಳು-ಜಾ-ತಿಯ
ಕೀಳು-ಜಾತಿ-ಯ-ವ-ರಿ-ಗೆಲ್ಲಾ
ಕೀಳು-ತ-ರದ
ಕೀಳು-ದರ್ಜೆಯ
ಕೀಳು-ದೃಷ್ಟಿ-ಯಿಂದ
ಕೀಳು-ಮಟ್ಟದ್ದು
ಕೀಳು-ಮಾಡಿ
ಕೀಳೆಂದೂ
ಕೀಳ್ಗೆಲಸ-ಗಳನ್ನು
ಕುಂಚವು
ಕುಂಟ-ರನ್ನು
ಕುಂಡಲ
ಕುಂಡಲ-ವನ್ನೂ
ಕುಂಡ-ಲಿನಿ
ಕುಂಡ-ಲಿನಿ-ಯನ್ನು
ಕುಂಡೆ
ಕುಂದು
ಕುಂದು-ಕೊರ-ತೆ-ಗಳನ್ನು
ಕುಂದು-ಕೊರತೆ-ಗಳಿಗೆ
ಕುಂದು-ಕೊರ-ತೆ-ಗಳು
ಕುಂದು-ಗಳಿವೆ
ಕುಂಬಾರ
ಕುಂಬಾರ-ನಾಗಿಯೆ
ಕುಂಭ-ಕರ್ಣ
ಕುಕ್ಕಿದೆ
ಕುಗ್ಗಿ
ಕುಗ್ಗಿ-ರುವ-ವ-ರಿಗೆ
ಕುಚೋದ್ಯ-ವಾಗಿ
ಕುಟಿರ
ಕುಟೀ-ರಕ್ಕೆ
ಕುಟೀ-ರ-ದಲ್ಲಿ
ಕುಟೀ-ರವು
ಕುಟ್ಟಿ
ಕುಟ್ಟಿ-ದರೆ
ಕುಡಿ
ಕುಡಿ-ಗಳನ್ನು
ಕುಡಿತ
ಕುಡಿ-ತದ
ಕುಡಿ-ದದ್ದು
ಕುಡಿ-ದರು
ಕುಡಿ-ದರೆ
ಕುಡಿ-ದಳು
ಕುಡಿ-ದಾದ
ಕುಡಿದು
ಕುಡಿ-ದು-ಕೊಟ್ಟಿದ್ದ-ರಿಂದ
ಕುಡಿದೆ
ಕುಡಿದೇ
ಕುಡಿ-ಯ-ಬಾ-ರದು
ಕುಡಿ-ಯ-ಬೇಕು
ಕುಡಿ-ಯ-ಬೇಕೆ
ಕುಡಿ-ಯ-ಬೇಕೆಂಬ
ಕುಡಿ-ಯಲು
ಕುಡಿ-ಯುತ್ತಿದ್ದರು
ಕುಡಿ-ಯುತ್ತಿದ್ದೀ-ರಲ್ಲ
ಕುಡಿ-ಯುತ್ತಿದ್ದೀರಿ
ಕುಡಿ-ಯುವಾಗ
ಕುಡಿ-ಯು-ವುದಕ್ಕೆ
ಕುಡಿ-ಯು-ವು-ದನ್ನು
ಕುಡಿ-ಯುವುದ-ರಲ್ಲಿ
ಕುಡಿ-ಯು-ವು-ದಿಲ್ಲ
ಕುಡಿ-ಯು-ವುದು
ಕುಡುಕ
ಕುಡುಕ-ನಿಗೂ
ಕುಡು-ಕರು
ಕುಣಿಕುಣಿ-ಯುತ
ಕುಣಿಕುಣಿವ
ಕುಣಿ-ಕೆ-ಯಲ್ಲಿ
ಕುಣಿತ
ಕುಣಿ-ತಕ್ಕೂ
ಕುಣಿ-ತವೇ
ಕುಣಿ-ಯುತಿ-ಹನು
ಕುಣಿ-ಯುತ್ತಾ
ಕುಣಿ-ಯುವ
ಕುತಃ
ಕುತರ್ಕ
ಕುತೂಹಲ
ಕುತೂ-ಹಲ-ವನ್ನು
ಕುತೂಹಲ-ವುಂಟಾ-ಗಿದೆ
ಕುತೂಹಲಿ-ಗಳಾಗಿ-ರುವರು
ಕುತೂಹಲಿ-ಗ-ಳಾದ
ಕುತ್ತಿಗೆಯ
ಕುತ್ತೋ
ಕುತ್ರಲೀ-ನ-ಮಿದಂ
ಕುಥಾ-ಜಾಯ
ಕುದಿ-ಯುತ್ತಿರುವ
ಕುದಿವ
ಕುದಿಸಿ
ಕುದಿ-ಸಿದ
ಕುದಿಸಿ-ದಾಗ
ಕುದುರೆ-ಗಳ
ಕುದುರೆ-ಗಳು
ಕುದುರೆ-ಯಾಗು-ವು-ದೆಂದು
ಕುದುರೆಯು
ಕುಮಾರಿ
ಕುಮಾರಿ-ಗಳಲ್ಲಿ
ಕುಯುಕ್ತಿ
ಕುಯುಕ್ತಿ-ಗಳನ್ನು
ಕುರಾನಿ-ನಲಿ
ಕುರಾನು-ಗಳು
ಕುರಿತದ್ದಾದರೂ
ಕುರಿ-ತದ್ದು
ಕುರಿತು
ಕುರಿ-ಮರಿ
ಕುರಿ-ಮರಿ-ಗಾಗಿ
ಕುರುಕ್ಷೇತ್ರದ
ಕುರುಡ
ಕುರುಡ-ನಿಗೆ
ಕುರುಡನು
ಕುರುಡ-ರಾಗುತ
ಕುರುಡ-ರೇ-ನನು
ಕುರುಡು
ಕುರುಡು-ಕೀಟವು
ಕುರುಡು-ಗತ್ತಲೆ-ಯಾದರೂ
ಕುರ್ಚಿ
ಕುರ್ಮಸ್ತಾರ-ಕ-ಚರ್ವಣಂ
ಕುಲ
ಕುಲಂ
ಕುಲಕೆ
ಕುಲ-ಗುರು
ಕುಲ-ಗುರು-ಗಳೇಕೆ
ಕುಲ-ಗೆಟ್ಟು
ಕುಲ-ದಲ್ಲಿ
ಕುಲ-ಧರ್ಮ-ಗಳನ್ನು
ಕುಲಪ್ರತಿಷ್ಠೆ-ಯಲ್ಲಿ
ಕುಲತ್ರೀ-ಯನ್ನು
ಕುಲಾಯಿ-ಸಿದೆ-ಯೆಂದು
ಕುಲಾವಿ
ಕುಲಿಶ
ಕುಲೀನ
ಕುಲುಷಿತ-ವಾಗು-ವುದು
ಕುಳಿತ
ಕುಳಿ-ತನು
ಕುಳಿ-ತರು
ಕುಳಿ-ತರೆ
ಕುಳಿ-ತಾಗ
ಕುಳಿತಿದ್ದ
ಕುಳಿ-ತಿದ್ದರು
ಕುಳಿತಿದ್ದರೆ
ಕುಳಿತಿದ್ದ-ವರೇ
ಕುಳಿತಿದ್ದಾನೆ
ಕುಳಿತಿದ್ದಾರೆ
ಕುಳಿತಿದ್ದೀಯೊ
ಕುಳಿ-ತಿದ್ದೆನೋ
ಕುಳಿತಿರ-ಬಲ್ಲೆ
ಕುಳಿತಿರ-ಬೇಕು
ಕುಳಿತಿರ-ಬೇಡಿ
ಕುಳಿ-ತಿ-ರಲು
ಕುಳಿತಿರುತ್ತಾ-ರೆಂದೂ
ಕುಳಿತಿ-ರುವ
ಕುಳಿತಿ-ರುವನು
ಕುಳಿತಿ-ರುವನೊ
ಕುಳಿತಿ-ರುವ-ವರ
ಕುಳಿತಿ-ರುವಾಗ
ಕುಳಿ-ತಿ-ರುವೆ
ಕುಳಿತು
ಕುಳಿ-ತುಕೊ
ಕುಳಿತು-ಕೊಂಡ
ಕುಳಿತು-ಕೊಂಡನು
ಕುಳಿತು-ಕೊಂಡರು
ಕುಳಿತು-ಕೊಂಡರೂ
ಕುಳಿತು-ಕೊಂಡರೆ
ಕುಳಿತು-ಕೊಂಡಾಗ
ಕುಳಿತು-ಕೊಂಡಿದ್ದ
ಕುಳಿತು-ಕೊಂಡಿದ್ದನು
ಕುಳಿತು-ಕೊಂಡಿದ್ದರು
ಕುಳಿತು-ಕೊಂಡಿದ್ದೆ
ಕುಳಿತು-ಕೊಂಡಿ-ರುತ್ತಿದ್ದರು
ಕುಳಿತು-ಕೊಂಡಿರು-ವೆವು
ಕುಳಿತು-ಕೊಂಡು
ಕುಳಿತು-ಕೊಂಡು-ಬಿಟ್ಟರು
ಕುಳಿತು-ಕೊಂಡು-ಬಿಟ್ಟು
ಕುಳಿತು-ಕೊಂಡೂ
ಕುಳಿತು-ಕೊಂಡೆ
ಕುಳಿತು-ಕೊಂಡೆವು
ಕುಳಿತು-ಕೊಂಡೊ-ಡ-ನೆಯೆ
ಕುಳಿತು-ಕೊಳ್ಳ-ಬೇಕು
ಕುಳಿತು-ಕೊಳ್ಳ-ಬೇಕೆಂದು
ಕುಳಿತು-ಕೊಳ್ಳ-ಲಾರೆ
ಕುಳಿತು-ಕೊಳ್ಳಲು
ಕುಳಿತು-ಕೊಳ್ಳಲೂ
ಕುಳಿತು-ಕೊಳ್ಳಿ
ಕುಳಿತು-ಕೊಳ್ಳುತ್ತದೆ
ಕುಳಿತು-ಕೊಳ್ಳುತ್ತಿದ್ದ
ಕುಳಿತು-ಕೊಳ್ಳುತ್ತಿದ್ದರು
ಕುಳಿತು-ಕೊಳ್ಳುವ
ಕುಳಿತು-ಕೊಳ್ಳುವರೊ
ಕುಳಿತು-ಕೊಳ್ಳುವರೋ
ಕುಳಿತು-ಕೊಳ್ಳು-ವುದಕ್ಕಾಗಿ
ಕುಳಿತು-ಕೊಳ್ಳು-ವುದಕ್ಕೆ
ಕುಳಿತು-ಕೊಳ್ಳುವು-ದಾ-ಗಲಿ
ಕುಳಿತು-ಕೊಳ್ಳು-ವು-ದಿಲ್ಲವೋ
ಕುಳಿತು-ಕೊಳ್ಳುವುದು
ಕುಳಿತು-ಕೊಳ್ಳು-ವು-ದೆಂದರೆ
ಕುಳಿತು-ಕೊಳ್ಳುವೆ-ಯೇನು
ಕುಳಿತುಕೋ
ಕುಳಿ-ತೆವು
ಕುಳಿತೇ
ಕುಳ್ಳಿ-ರಲು
ಕುಳ್ಳಿರಿ-ಸಿ-ಕೊಂಡರು
ಕುವ-ಲಯ
ಕುವೆಂಪು
ಕುಶಲ
ಕುಶಲ-ತೆ-ಯಲ್ಲಿ
ಕುಶಲ-ವಾಗಿದ್ದಾಳೆಂದೂ
ಕುಶಲ-ವೆಂದು
ಕುಶಲಿ-ಯಾದ
ಕುಸಂಸ್ಕಾರ-ಗಳಿಂದ
ಕುಸಿದ
ಕುಸಿ-ದರು
ಕುಸಿದು
ಕುಸಿದು-ಬಿತ್ತು
ಕುಸಿ-ಯುತ್ತದೆ
ಕುಸಿ-ಯುವಾಗ
ಕುಸಿ-ಯುವೆ
ಕುಸಿ-ವುದು
ಕುಸುಮ-ಗಳ
ಕುಸುಮ-ಗಳನ್ನಾಗಿ-ಸುತ್ತದೆ
ಕುಸುಮ-ಗಳ-ರಳಿರೆ
ಕುಸುಮ-ರಾಶಿ-ಗಳಂತೆ
ಕುಹಕ-ಕಾವ್ಯ-ವನ್ನು
ಕುಹ-ಕಾಂತ-ಕಾರಿ
ಕೂಗಾಡಿ
ಕೂಗಾಡಿ-ದರು
ಕೂಗಾಡುತ್ತಿರುವ
ಕೂಗಿ
ಕೂಗಿ-ಕೊಂಡ
ಕೂಗಿ-ಕೊಂಡರೆ
ಕೂಗಿ-ಕೊಳ್ಳುವು-ದರಿಂದ
ಕೂಗಿಟ್ಟರೆ
ಕೂಗಿ-ದರೂ
ಕೂಗಿ-ದರೆ
ಕೂಗಿ-ನಿಂದ
ಕೂಗಿ-ಸ-ಬೇಕು
ಕೂಗು
ಕೂಗು-ತಲಿ
ಕೂಗುತ್ತಾ
ಕೂಗುತ್ತಿದ್ದರು
ಕೂಗುತ್ತಿದ್ದೇವೆ
ಕೂಗು-ವರೆಂದೆನ್ನಿ-ಸು-ವುದು
ಕೂಜಂತಂ
ಕೂಟಕ್ಕೆ
ಕೂಟಪ್ರಶ್ನೆ-ಗಳನ್ನು
ಕೂಡ
ಕೂಡಲು
ಕೂಡಲೆ
ಕೂಡಲೇ
ಕೂಡಾ
ಕೂಡಿ
ಕೂಡಿದ
ಕೂಡಿ-ದ-ವ-ರಾಗಿದ್ದರು
ಕೂಡಿ-ದುದು
ಕೂಡಿದೆ
ಕೂಡಿದ್ದ
ಕೂಡಿದ್ದರು
ಕೂಡಿದ್ದು
ಕೂಡಿ-ರುತ್ತದೆ
ಕೂಡಿ-ರುವ
ಕೂಡಿ-ರುವ-ವ-ನೆಂದು
ಕೂಡಿ-ರು-ವುದು
ಕೂಡಿರೆ
ಕೂಡಿಸಿ
ಕೂಡಿ-ಸಿ-ದರೆ
ಕೂಡಿ-ಸಿದ್ದರು
ಕೂಡಿಸು
ಕೂಡಿ-ಹಾಕಿ-ಕೊಂಡು
ಕೂಡಿ-ಹಾಕುವ
ಕೂಡು
ಕೂಡುತ
ಕೂಡೆ
ಕೂತುಕೊ
ಕೂಥಾ
ಕೂದಲಾ-ಗಲಿ
ಕೂದ-ಲಿನ
ಕೂಪದಾಳ-ದಲ್ಲಿ
ಕೂರಿಸಿ-ಕೊಂಡು
ಕೂರಿಸಿ-ಕೊಳ್ಳು-ವು-ದಿಲ್ಲ
ಕೂರಿಸಿದ್ದೇನೆ
ಕೂರಿಸಿದ್ದೇವೆ
ಕೂರ್ಮ
ಕೂಲಂಕಷ-ವಾಗಿ
ಕೂಲಿ-ಕಾ-ರರು
ಕೂಲಿ-ಯಾಗಿ
ಕೃಂತನ
ಕೃತಕ
ಕೃತ-ಕೃತ್ಯ-ನಾಗಿದ್ದೀಯೆ
ಕೃತ-ಕೃತ್ಯ-ರಾಗಿದ್ದಾರೆ
ಕೃತಜ್ಞತೆ-ಗಳೊಂದಿಗೆ
ಕೃತಜ್ಞ-ತೆ-ಯಿಂದ
ಕೃತಜ್ಞ-ನಾಗಿ-ರುತ್ತಿದ್ದೆ
ಕೃತಜ್ಞ-ರಾಗಿದ್ದಿರ-ಬ-ಹುದು
ಕೃತಜ್ಞ-ರಾಗಿ-ರ-ದಿದ್ದರೆ
ಕೃತಜ್ಞ-ರಾಗಿ-ರ-ಬೇಕು
ಕೃತಜ್ಞ-ರಾಗಿ-ರುತ್ತಾರೆ
ಕೃತ-ಫಲಂ
ಕೃತ-ಸಂಕಲ್ಪ-ನಾಗಿ
ಕೃತಾರ್ಥ-ರಾದರು
ಕೃತಾರ್ಥಾ
ಕೃತಿ-ಗಳನ್ನು
ಕೃತಿ-ಗಳು
ಕೃತಿ-ಗಿಳಿ-ಸಲು
ಕೃತಿ-ಗಿಳಿ-ಸುವು-ದಿದೆ-ಯಲ್ಲ
ಕೃತಿ-ಯನ್ನು
ಕೃತಿಶ್ರೇಣಿ
ಕೃತೀ
ಕೃತ್ಯಂ
ಕೃತ್ರಿ-ಮದ
ಕೃತ್ಸ್ನ-ಕರ್ಮ-ಕೃತ್
ಕೃಪಾ
ಕೃಪಾ-ಕಟಾಕ್ಷ
ಕೃಪಾ-ಕಟಾಕ್ಷ-ದಿಂದ
ಕೃಪಾ-ಕಟಾಕ್ಷವು
ಕೃಪಾ-ಗುಣ-ದಿಂದ
ಕೃಪಾ-ದೃಷ್ಟಿ
ಕೃಪಾ-ದೃಷ್ಟಿಯ
ಕೃಪಾಪ್ರಸಾದ-ಗಳೇ
ಕೃಪಾ-ಬಲ-ದಿಂದ
ಕೃಪಾ-ಮಾರುತವು
ಕೃಪಾಯ
ಕೃಪೆ
ಕೃಪೆಗೆ
ಕೃಪೆ-ದೋರ
ಕೃಪೆ-ದೋರೈ
ಕೃಪೆ-ಮಾಡ-ಬೇಕು
ಕೃಪೆ-ಮಾಡಿ
ಕೃಪೆ-ಮಾಡುತ್ತಾನೆ
ಕೃಪೆಯ
ಕೃಪೆ-ಯನು
ಕೃಪೆ-ಯನ್ನು
ಕೃಪೆ-ಯಲ್ಲದೆ
ಕೃಪೆ-ಯಾ-ಗಲಿ
ಕೃಪೆಯಿಂ
ಕೃಪೆ-ಯಿಂದ
ಕೃಪೆ-ಯಿಂದಲಿ
ಕೃಪೆ-ಯಿಟ್ಟು
ಕೃಪೆ-ಯಿ-ದೆಯೋ
ಕೃಪೆಯು
ಕೃಪೆ-ಯುಂಟಾ-ಗುತ್ತದೆ
ಕೃಪೆಯೂ
ಕೃಪೆ-ಯೆಂಬ
ಕೃಶ-ಮಾಡಿದೆ
ಕೃಷ್ಣ
ಕೃಷ್ಣ-ಗೀತೆ
ಕೃಷ್ಣನ
ಕೃಷ್ಣ-ನನ್ನು
ಕೃಷ್ಣ-ನನ್ನೇ
ಕೃಷ್ಣ-ನಲ್ಲಿ
ಕೃಷ್ಣ-ನಿ-ಗಿಂತ
ಕೃಷ್ಣನು
ಕೃಷ್ಣನೊ
ಕೃಷ್ಣ-ರೂಪದಿ
ಕೆಂಡ-ಗಳಂತೆ
ಕೆಂಡದ
ಕೆಂಡ-ದಂತೆ
ಕೆಂಪ-ಗಾ-ಯಿತು
ಕೆಂಪಗೆ
ಕೆಂಪು
ಕೆಂಪು-ಸೂರ್ಯನ
ಕೆಂಪೇರಿದಂತಾಯಿತು
ಕೆಂಪೇರಿದೆ
ಕೆಂಪೇರಿದ್ದದ್ದನ್ನು
ಕೆಂಬಣ್ಣ
ಕೆಚ್ಚು
ಕೆಚ್ಚೆದೆ-ಯಿಂದ
ಕೆಚ್ಛೆದೆ
ಕೆಟ್ಟ
ಕೆಟ್ಟ-ಕಾಲ-ದಲ್ಲಿ
ಕೆಟ್ಟ-ದಕ್ಕೊ
ಕೆಟ್ಟ-ದಾಗುವುದೂ
ಕೆಟ್ಟದ್ದಕ್ಕಿಂತಲೂ
ಕೆಟ್ಟದ್ದಕ್ಕೂ
ಕೆಟ್ಟದ್ದನ್ನಾದರೂ
ಕೆಟ್ಟದ್ದನ್ನು
ಕೆಟ್ಟದ್ದು
ಕೆಟ್ಟದ್ದೇನೂ
ಕೆಟ್ಟರೂ
ಕೆಟ್ಟ-ವನಾಗ-ಬಲ್ಲ-ನೆಂದು
ಕೆಟ್ಟ-ವ-ನೆಂದು
ಕೆಟ್ಟ-ವನೇ
ಕೆಟ್ಟ-ವ-ರಿಂದ
ಕೆಟ್ಟ-ವರು
ಕೆಟ್ಟ-ಹಾದಿ
ಕೆಟ್ಟಿತು
ಕೆಟ್ಟಿತ್ತು
ಕೆಟ್ಟಿ-ರುವ
ಕೆಟ್ಟು
ಕೆಟ್ಟು-ದಾಗಿ
ಕೆಟ್ಟುದು
ಕೆಟ್ಟು-ದೆಂದೂ
ಕೆಟ್ಟು-ಹೋಗು-ವುದು
ಕೆಡಕುಂಟಾಗು-ವು-ದಲ್ಲದೆ
ಕೆಡಕು-ಗಳು
ಕೆಡಹು-ವೆನು
ಕೆಡಿಸ-ದಂತೆ
ಕೆಡಿಸಿ
ಕೆಡಿಸಿ-ಕೊಳ್ಳ-ಬೇಡ
ಕೆಡಿಸಿ-ಕೊಳ್ಳುವ-ವ-ರಲ್ಲ
ಕೆಡಿಸಿ-ಬಿಟ್ಟಿರಿ
ಕೆಡು-ಕನ್ನು
ಕೆಡುಕಾ-ಗಲಿ
ಕೆಡುಕಿಗೆ
ಕೆಡು-ಕಿನಲಿ
ಕೆಡುಕು
ಕೆಡುಕು-ಗಳಲ್ಲಿ
ಕೆಡುಕು-ಗಳು
ಕೆಡು-ಕೊಂದಿದೆ
ಕೆಡು-ವು-ದೆಂದು
ಕೆಣಕಲು
ಕೆತ್ತಲ್ಪಟ್ಟಿದ್ದವು
ಕೆತ್ತಿದ
ಕೆತ್ತಿರುವ
ಕೆದಕಿದ
ಕೆನೊ
ಕೆನ್ನೆ-ಯನ್ನು
ಕೆರಳಿತು
ಕೆರ-ಳಿದ
ಕೆರಳಿ-ದಾಸೆಯ
ಕೆರಳಿಸಿ
ಕೆರಳಿ-ಸಿದೆ
ಕೆರಳು-ತಲಿ
ಕೆರೆಗೆ
ಕೆರೆಯೊಂದ-ರಲ್ಲಿ
ಕೆಲ-ಕೆಲ-ವ-ರಿಗೆ
ಕೆಲ-ಕೆಲ-ವರು
ಕೆಲ-ಕೆಲವು
ಕೆಲರು
ಕೆಲ-ವನ್ನಾದರೂ
ಕೆಲ-ವನ್ನು
ಕೆಲ-ವರ
ಕೆಲವ-ರಂತೂ
ಕೆಲ-ವ-ರನ್ನು
ಕೆಲ-ವ-ರಲ್ಲಿ
ಕೆಲವ-ರಾದರೂ
ಕೆಲ-ವರಿಗಂತೂ
ಕೆಲ-ವ-ರಿಗೆ
ಕೆಲ-ವರು
ಕೆಲವು
ಕೆಲವೆಡೆ
ಕೆಲವೆಡೆ-ಗಳಲ್ಲಿ
ಕೆಲವೇ
ಕೆಲವೊಂದ-ರಲ್ಲಿ
ಕೆಲವೊಮ್ಮ
ಕೆಲವೊಮ್ಮೆ
ಕೆಲವೊಮ್ಮೆ-ಯಂತೂ
ಕೆಲಸ
ಕೆಲಸ-ಕಾರ್ಯ-ಗಳೊಂದನ್ನೂ
ಕೆಲಸ-ಕೆಲಸ-ನಿರಂತರ
ಕೆಲಸಕ್ಕಾಗಿ
ಕೆಲಸಕ್ಕೂ
ಕೆಲ-ಸಕ್ಕೆ
ಕೆಲಸ-ಗಳ
ಕೆಲಸ-ಗಳನ್ನು
ಕೆಲಸ-ಗಳನ್ನೆಲ್ಲಾ
ಕೆಲಸ-ಗಳಲ್ಲಿ
ಕೆಲಸ-ಗ-ಳಲ್ಲೂ
ಕೆಲಸ-ಗಳಿಗೂ
ಕೆಲಸ-ಗಳಿಗೆ
ಕೆಲಸ-ಗಳಿಗೆಲ್ಲಾ
ಕೆಲಸ-ಗಳು
ಕೆಲಸ-ಗ-ಳೆಂದು
ಕೆಲಸ-ಗಳೆಲ್ಲ
ಕೆಲಸ-ಗಳೆಲ್ಲಾ
ಕೆಲಸ-ಗಾ-ರರು
ಕೆಲ-ಸದ
ಕೆಲಸ-ದಂತೆ
ಕೆಲಸ-ದಲ್ಲಿ
ಕೆಲಸ-ದಲ್ಲಿಯೂ
ಕೆಲಸ-ದಲ್ಲೂ
ಕೆಲಸ-ಮಾ-ಡದೆ
ಕೆಲಸ-ಮಾಡ-ಬಲ್ಲರು
ಕೆಲಸ-ಮಾಡ-ಬೇ-ಕಾದ
ಕೆಲಸ-ಮಾ-ಡಲು
ಕೆಲಸ-ಮಾಡಿ
ಕೆಲಸ-ಮಾಡಿ-ದರೂ
ಕೆಲಸ-ಮಾಡಿದ್ದು
ಕೆಲಸ-ಮಾಡಿ-ಸ-ಬಲ್ಲ-ವ-ರಾದರು
ಕೆಲಸ-ಮಾಡುತ್ತಾ
ಕೆಲಸ-ಮಾಡುತ್ತಿದ್ದ
ಕೆಲಸ-ಮಾಡುತ್ತಿದ್ದಾಗ
ಕೆಲಸ-ಮಾಡುವ
ಕೆಲಸ-ಮಾಡು-ವುದಕ್ಕೆ
ಕೆಲಸ-ಮಾಡು-ವುದೇ
ಕೆಲಸ-ವನ್ನಾ-ಗಲಿ
ಕೆಲಸ-ವನ್ನಾದರೂ
ಕೆಲಸ-ವನ್ನು
ಕೆಲಸ-ವನ್ನೂ
ಕೆಲಸ-ವನ್ನೆಲ್ಲಾ
ಕೆಲಸ-ವನ್ನೇ
ಕೆಲಸ-ವಲ್ಲ
ಕೆಲಸ-ವಾಗಲಿ
ಕೆಲಸ-ವಾದರೂ
ಕೆಲ-ಸ-ವಿದೆ
ಕೆಲ-ಸ-ವಿದೆ-ಯೆಂದು
ಕೆಲಸ-ವಿದ್ದಲ್ಲಿ
ಕೆಲಸವು
ಕೆಲಸವೂ
ಕೆಲಸ-ವೆಂದು
ಕೆಲಸವೇ
ಕೆಲಸ-ವೇ-ನಿದ್ದರೂ
ಕೆಲಸ-ವೇ-ನೆಂದರೆ
ಕೆಲಸವೊ
ಕೆಳ
ಕೆಳಕ್ಕೂ
ಕೆಳಕ್ಕೆ
ಕೆಳ-ಗಡೆ
ಕೆಳ-ಗಣ
ಕೆಳ-ಗಿಟ್ಟು
ಕೆಳ-ಗಿಡ-ದಿರು
ಕೆಳ-ಗಿನ
ಕೆಳ-ಗಿ-ರುವ
ಕೆಳ-ಗಿ-ರು-ವುದೇ
ಕೆಳ-ಗಿಳಿ-ದಂತಾಯಿತು
ಕೆಳ-ಗಿ-ಳಿದು
ಕೆಳ-ಗು-ಗಳೆಲ್ಲ
ಕೆಳಗೆ
ಕೆಳಗೇ
ಕೆಳ-ಜಾತಿ-ಯ-ವನು
ಕೆಳ-ವರ್ಗ-ದ-ವರು
ಕೆಳೆಯು
ಕೇ
ಕೇಂದ್ರ
ಕೇಂದ್ರಕ್ಕೆ
ಕೇಂದ್ರ-ಗಳನ್ನು
ಕೇಂದ್ರ-ಗಳಲ್ಲಿ
ಕೇಂದ್ರ-ಗಳಾಗಿದ್ದುವೆಂಬು-ದೇನೋ
ಕೇಂದ್ರ-ಗಳೇ
ಕೇಂದ್ರದ
ಕೇಂದ್ರ-ದಿಂದ
ಕೇಂದ್ರ-ದೆಡೆ
ಕೇಂದ್ರ-ದೆಡೆಗೆ
ಕೇಂದ್ರ-ವನ್ನಾಗಿ
ಕೇಂದ್ರ-ವನ್ನು
ಕೇಂದ್ರ-ವಲ್ಲ
ಕೇಂದ್ರ-ವಸ್ತು-ವಾಗಿದೆ
ಕೇಂದ್ರ-ವಾಗಿ
ಕೇಂದ್ರ-ವಾಗಿಟ್ಟು-ಕೊಂಡು
ಕೇಂದ್ರ-ವಾಗಿದ್ದುವು
ಕೇಂದ್ರ-ವಾಗಿ-ರ-ಬೇಕು
ಕೇಂದ್ರ-ವಾ-ಯಿತು
ಕೇಂದ್ರಸ್ಥಾನ-ವಾಗುತ್ತದೆ
ಕೇಂದ್ರಸ್ಥಾನ-ವಾಗು-ವುದು
ಕೇಂದ್ರೀ-ಕರಿ-ಸು-ವು-ದರ
ಕೇಂದ್ರೀಕೃತ-ವಾಗ-ಬೇಕು
ಕೇಂದ್ರೀಕೃ-ತ-ವಾ-ಗು-ವಂತೆ
ಕೇಡನ್ನು
ಕೇಡಿಗೆ
ಕೇಡಿ-ನಿಂದ
ಕೇತೊರೆ
ಕೇನ
ಕೇಬಲ
ಕೇರೇ
ಕೇಳ-ತಕ್ಕದ್ದೇ-ನಾದರೂ
ಕೇಳ-ತೊಡಗಿ-ದೆನೊ
ಕೇಳ-ದಿದ್ದರು
ಕೇಳ-ದಿರಿ
ಕೇಳದೆ
ಕೇಳ-ದೆಯೇ
ಕೇಳ-ಬ-ಹುದು
ಕೇಳ-ಬಹುದೆ
ಕೇಳ-ಬಾ-ರದು
ಕೇಳ-ಬೇಕೆ
ಕೇಳ-ಬೇಕೆಂದಿದ್ದರು
ಕೇಳ-ಬೇಕೆಂದಿದ್ದೀಯೊ
ಕೇಳ-ಬೇಕೆಂದಿದ್ದೇನೆ
ಕೇಳ-ಬೇಕೆಂದಿರು-ವೆ-ವೆಂದೂ
ಕೇಳ-ಬೇಕೆಂದು
ಕೇಳ-ಲಾರದೆ
ಕೇಳಲಿಲ್ಲ-ವೇನು
ಕೇಳಲು
ಕೇಳಲ್ಪಟ್ಟವು
ಕೇಳಲ್ಪಟ್ಟು
ಕೇಳಿ
ಕೇಳಿ-ಕೊಂಡ
ಕೇಳಿ-ಕೊಂಡದ್ದ-ರಿಂದಲೂ
ಕೇಳಿ-ಕೊಂಡನು
ಕೇಳಿ-ಕೊಂಡು
ಕೇಳಿ-ಕೊಳ್ಳ-ಕೂ-ಡದು
ಕೇಳಿ-ಕೊಳ್ಳಿ
ಕೇಳಿ-ಕೊಳ್ಳುತ್ತೇನೆ
ಕೇಳಿದ
ಕೇಳಿ-ದನು
ಕೇಳಿ-ದ-ಮೇಲೆ
ಕೇಳಿ-ದರು
ಕೇಳಿ-ದರೂ
ಕೇಳಿ-ದರೆ
ಕೇಳಿ-ದಾಗ
ಕೇಳಿ-ದಾಗ-ಲೆಲ್ಲ
ಕೇಳಿ-ದಾರಭ್ಯ
ಕೇಳಿ-ದಿ-ರಷ್ಟೇ
ಕೇಳಿ-ದು-ದನ್ನು
ಕೇಳಿ-ದುದ-ರಲ್ಲಿ
ಕೇಳಿ-ದು-ದ-ರಿಂದಲೇ
ಕೇಳಿ-ದು-ದೆಲ್ಲ-ವನ್ನೂ
ಕೇಳಿದೆ
ಕೇಳಿ-ದೆಯೊ
ಕೇಳಿ-ದೆವು
ಕೇಳಿ-ದೊಡ-ನೆಯೇ
ಕೇಳಿದ್ದಕ್ಕೆಲ್ಲ
ಕೇಳಿದ್ದರು
ಕೇಳಿದ್ದರೂ
ಕೇಳಿದ್ದರೆ
ಕೇಳಿದ್ದಿರಾ
ಕೇಳಿದ್ದೀ-ಯಷ್ಟೆ
ಕೇಳಿದ್ದೀಯೇನು
ಕೇಳಿದ್ದು
ಕೇಳಿದ್ದು-ದ-ರಿಂದ
ಕೇಳಿದ್ದೆ
ಕೇಳಿದ್ದೇನೆ
ಕೇಳಿದ್ದೇನೆಯೇ
ಕೇಳಿದ್ದೇವೆ
ಕೇಳಿದ್ದೇವೆ-ಯಲ್ಲಾ
ಕೇಳಿ-ಬಂತು
ಕೇಳಿ-ಬ-ರುತ್ತದೆ
ಕೇಳಿ-ಬರುತ್ತಿದೆ
ಕೇಳಿ-ಬ-ರು-ವು-ದಿಲ್ಲ
ಕೇಳಿ-ಬಲ್ಲರೆ
ಕೇಳಿ-ಬಿಟ್ಟನು
ಕೇಳಿ-ಬಿಟ್ಟರೆ
ಕೇಳಿಯೇ
ಕೇಳಿ-ರ-ಬ-ಹುದು
ಕೇಳಿ-ರ-ಲಿಲ್ಲ
ಕೇಳಿ-ರು-ವು-ದೇ-ನೆಂದರೆ
ಕೇಳಿ-ರುವೆ
ಕೇಳಿ-ರುವೆ-ಯಲ್ಲವೆ
ಕೇಳಿ-ರುವೆ-ಯೇನು
ಕೇಳಿ-ರು-ವೆವು
ಕೇಳಿಲ್ಲ
ಕೇಳಿಲ್ಲವೆ
ಕೇಳಿಲ್ಲವೇ
ಕೇಳಿಲ್ಲವೋ
ಕೇಳಿ-ಸಿ-ಕೊಳ್ಳ-ಲಿಲ್ಲ
ಕೇಳಿ-ಸಿತು
ಕೇಳಿ-ಸಿತ್ತು
ಕೇಳಿ-ಸುತ್ತದೆ
ಕೇಳು
ಕೇಳು-ತಿ-ಹುದು
ಕೇಳುತ್ತ
ಕೇಳುತ್ತಾ
ಕೇಳುತ್ತಿದ್ದ
ಕೇಳುತ್ತಿದ್ದನು
ಕೇಳುತ್ತಿದ್ದರು
ಕೇಳುತ್ತಿದ್ದರೂ
ಕೇಳುತ್ತಿದ್ದರೆ
ಕೇಳುತ್ತಿದ್ದ-ವರು
ಕೇಳುತ್ತಿದ್ದೀಯೆ
ಕೇಳುತ್ತಿದ್ದೇ-ನಲ್ಲ
ಕೇಳುತ್ತಿರ-ಬೇಕು
ಕೇಳುತ್ತಿ-ರ-ಲಿಲ್ಲ
ಕೇಳುತ್ತಿರು-ವಾಗ
ಕೇಳುತ್ತಿರುವೆ
ಕೇಳುತ್ತಿರು-ವೆನು
ಕೇಳುತ್ತಿರು-ವೆ-ಯಲ್ಲವೆ
ಕೇಳುತ್ತೀಯೆ
ಕೇಳುವ
ಕೇಳು-ವಂತೆ
ಕೇಳು-ವನೋ
ಕೇಳು-ವರು
ಕೇಳು-ವ-ರೇನು
ಕೇಳು-ವರೊ
ಕೇಳು-ವ-ವರ
ಕೇಳು-ವ-ವರೇ
ಕೇಳು-ವಾಗ
ಕೇಳು-ವುದಕ್ಕೆ
ಕೇಳು-ವು-ದ-ರಿಂದ
ಕೇಳು-ವು-ದಿಲ್ಲ
ಕೇಳು-ವುದು
ಕೇಳು-ವು-ದೇ-ನೆಂದರೆ
ಕೇಳುವೆ
ಕೇಳೆಂದು
ಕೇವಲ
ಕೇವಲಮ್
ಕೇವಾ
ಕೇವಾ-ಕೀವಾ
ಕೇವಾಸೆ
ಕೇಶ
ಕೇಶ-ರಾಶಿ-ಯಂತೆ
ಕೇಶವ
ಕೇಸರಿ
ಕೇಸರೀ
ಕೇಹ-ನಹಿ
ಕೈ
ಕೈಕಟ್ಟಿ
ಕೈಕಾಲು
ಕೈಕಾಲು-ಗಳನ್ನು
ಕೈಕಾಲು-ಗಳನ್ನೆಲ್ಲಾ
ಕೈಕಾಲು-ಗಳೆಲ್ಲಾ
ಕೈಕುಲುಕುತಿ-ರುವಾಗ
ಕೈಕೆ-ಳಗೆ
ಕೈಕೊಂಡರು
ಕೈಕೊಂಡಿದ್ದಾರೆ
ಕೈಕೊಂಡು
ಕೈಕೊಳ್ಳು-ವುದಕ್ಕೆ
ಕೈಗಳ
ಕೈಗಳನ್ನೆತ್ತಿ
ಕೈಗಳಿಂದ
ಕೈಗಳಿಂದಲೂ
ಕೈಗಳು
ಕೈಗಳೊ-ಳಗೆ
ಕೈಗಾ-ರಿಕಾ
ಕೈಗಾ-ರಿಕೆ-ಗಳ
ಕೈಗಾ-ರಿಕೆಯ
ಕೈಗೂಡಿತು
ಕೈಗೂಡುತ್ತ-ದೆಂಬು-ದನ್ನು
ಕೈಗೆ
ಕೈಗೊಂಡರೂ
ಕೈಗೊಂಡಿದ್ದಾನೆ
ಕೈಗೊಂಡು
ಕೈಗೊಳ್ಳ-ದಿದ್ದರೆ
ಕೈಗೊಳ್ಳದೆ
ಕೈಗೊಳ್ಳ-ಬೇಕೆಂದು
ಕೈಗೊಳ್ಳ-ಬೇಕೆಂಬ
ಕೈಗೊಳ್ಳಿರಿ
ಕೈಗೊಳ್ಳುವುದ-ರಲ್ಲಿ
ಕೈಗೊಳ್ಳುವು-ದರಿಂದ
ಕೈಚಪ್ಪಾಳೆ
ಕೈಚಾಚಿ
ಕೈಜೋಡಿಸಿ
ಕೈತೊಳೆದು-ಕೊಂಡು
ಕೈಬಿ-ಡ-ಲಿಲ್ಲ
ಕೈಬಿಡು
ಕೈಬಿಡುತ್ತಾರೆ
ಕೈಬೀಸಿ
ಕೈಮು-ಗಿದು
ಕೈಮು-ಗಿದು-ಕೊಂಡು
ಕೈಯಲ್ಲಾ-ಗುತ್ತಿದ್ದರೂ
ಕೈಯಲ್ಲಿ
ಕೈಯಲ್ಲಿಟ್ಟು-ಕೊಂಡು
ಕೈಯಲ್ಲಿತ್ತು
ಕೈಯಲ್ಲಿನ
ಕೈಯಲ್ಲಿಯೇ
ಕೈಯಲ್ಲಿ-ರುವ
ಕೈಯಲ್ಲೂ
ಕೈಯಲ್ಲೇ
ಕೈಯಾಗಿ-ರುವನೊ
ಕೈಯಾರ
ಕೈಯಿಂದ
ಕೈಯಿಂದಲೆ
ಕೈಯಿಂದಲೇ
ಕೈಯಿಟ್ಟು
ಕೈಯಿನ
ಕೈಯುರಿ-ಯಿಂದಲೂ
ಕೈಲಾ-ಗದ
ಕೈಲಾ-ಗದ-ವರು
ಕೈಲಾ-ಗುತ್ತದೆಯೊ
ಕೈಲಾ-ದಷ್ಟು
ಕೈಲಾ-ದಷ್ಟೂ
ಕೈಲಾಸ-ಪರ್ವ-ತದ
ಕೈಲಿದೆ
ಕೈವಲ್ಯಂ
ಕೈಹಿಡಿದು
ಕೊಂಚ
ಕೊಂಚ-ಕಾಲ
ಕೊಂಚ-ಕಾಲ-ದಲ್ಲಿಯೇ
ಕೊಂಚ-ದೂರ
ಕೊಂಚ-ಮಂದಿ
ಕೊಂಚ-ಮಟ್ಟಿಗೆ
ಕೊಂಚ-ವನ್ನಾದರೂ
ಕೊಂಚ-ವಾದರೂ
ಕೊಂಚವೂ
ಕೊಂಚ-ಹೊತ್ತು
ಕೊಂಡನು
ಕೊಂಡರೆ
ಕೊಂಡ-ರೆನ್ನನು
ಕೊಂಡಾಡಿತು
ಕೊಂಡಾಡುತ್ತಾ
ಕೊಂಡಾಡುತ್ತಿದ್ದಾನೆ
ಕೊಂಡಿ-ಗಳನ್ನು
ಕೊಂಡಿದ್ದಂತೆ
ಕೊಂಡಿದ್ದರು
ಕೊಂಡಿದ್ದೀಯಪ್ಪಾ
ಕೊಂಡಿರು-ವರು
ಕೊಂಡು
ಕೊಂಡು-ಕೊಂಡಳು
ಕೊಂಡು-ಕೊಂಡಿದ್ದ
ಕೊಂಡು-ಕೊಂಡು
ಕೊಂಡು-ಕೊಳ್ಳ-ಬೇಕು
ಕೊಂಡು-ಕೊಳ್ಳುವಾಗ
ಕೊಂಡು-ತಂದ
ಕೊಂಡು-ತಂದು
ಕೊಂಡೊಯ್ಯ-ಲಾರ-ದೆಂದು
ಕೊಂಡೊಯ್ಯುತ್ತಿರು-ವಳೋ
ಕೊಂದ
ಕೊಂದು-ಹಾಕಿ-ದರು
ಕೊಚ್ಚಿ-ಕೊಂಡು
ಕೊಚ್ಚಿ-ಹೋಗಿ
ಕೊಚ್ಚಿ-ಹೋಗಿತ್ತು
ಕೊಚ್ಚಿ-ಹೋಗುತ್ತವೆ
ಕೊಚ್ಚುತ್ತಿದೆ
ಕೊಟ್ಟ
ಕೊಟ್ಟನು
ಕೊಟ್ಟನೋ
ಕೊಟ್ಟ-ರಲ್ಲಾ
ಕೊಟ್ಟ-ರಷ್ಟೆ
ಕೊಟ್ಟರು
ಕೊಟ್ಟರೆ
ಕೊಟ್ಟ-ರೆಂದು
ಕೊಟ್ಟ-ವನೂ
ಕೊಟ್ಟಾಗ
ಕೊಟ್ಟಾದ-ಮೇಲೆ
ಕೊಟ್ಟಾ-ದರೂ
ಕೊಟ್ಟಾ-ಹಾರ-ವನ್ನವು
ಕೊಟ್ಟಿದೆ
ಕೊಟ್ಟಿದ್ದಕ್ಕಾಗಿ
ಕೊಟ್ಟಿದ್ದರು
ಕೊಟ್ಟಿದ್ದ-ವನ
ಕೊಟ್ಟಿದ್ದಾನೆ
ಕೊಟ್ಟಿದ್ದಾರೆ
ಕೊಟ್ಟಿದ್ದೀರಿ
ಕೊಟ್ಟಿದ್ದೇವೆ
ಕೊಟ್ಟಿರಿ
ಕೊಟ್ಟಿರುತ್ತಾರೆ
ಕೊಟ್ಟಿ-ರುವರು
ಕೊಟ್ಟಿ-ರುವ-ವರ
ಕೊಟ್ಟಿ-ರುವ-ವರು
ಕೊಟ್ಟಿ-ರುವಿರಿ
ಕೊಟ್ಟಿಲ್ಲ
ಕೊಟ್ಟಿಲ್ಲವೊ
ಕೊಟ್ಟಿ-ಹನು
ಕೊಟ್ಟೀಯೆ
ಕೊಟ್ಟು
ಕೊಟ್ಟು-ಕೊಂಡು
ಕೊಟ್ಟುವು
ಕೊಟ್ಟೆ
ಕೊಟ್ಟೇ
ಕೊಠಡಿ
ಕೊಠ-ಡಿಗೆ
ಕೊಠಡಿಯ
ಕೊಠಡಿ-ಯನ್ನು
ಕೊಠಡಿ-ಯಲ್ಲಿ
ಕೊಠಡಿ-ಯಲ್ಲೇ
ಕೊಠಡಿಯು
ಕೊಠಡಿ-ಯೊಂದ-ರಲ್ಲಿ
ಕೊಠಡಿ-ಯೊ-ಳಕ್ಕೆ
ಕೊಡ-ತಕ್ಕ-ವನು
ಕೊಡ-ದಂತೆ
ಕೊಡ-ದಿದ್ದಲ್ಲಿ
ಕೊಡದಿ-ರು-ವುದೇ
ಕೊಡದೆ
ಕೊಡ-ಬಲ್ಲೆ
ಕೊಡ-ಬಲ್ಲೆ-ಯಾದರೆ
ಕೊಡ-ಬ-ಹುದು
ಕೊಡ-ಬಾ-ರದು
ಕೊಡ-ಬೇಕಾಗಿತ್ತು
ಕೊಡ-ಬೇ-ಕಾದ
ಕೊಡಬೇಕಿದ್ದರೆ
ಕೊಡ-ಬೇಕು
ಕೊಡ-ಬೇಡ
ಕೊಡ-ಬೇಡಿರಿ
ಕೊಡ-ಲಾ-ಗದೆ
ಕೊಡ-ಲಾಗಿತ್ತು
ಕೊಡ-ಲಾ-ಯಿತು
ಕೊಡ-ಲಾರ-ನೇನು
ಕೊಡಲಿ
ಕೊಡಲಿಯ
ಕೊಡಲಿ-ಯನ್ನು
ಕೊಡಲಿ-ಯಿಂದ
ಕೊಡಲಿ-ಯೇನೊ
ಕೊಡ-ಲಿಲ್ಲ
ಕೊಡಲು
ಕೊಡಲೆ
ಕೊಡಲೇ-ಬೇಕು
ಕೊಡಲ್ಪಡುತ್ತವೆ
ಕೊಡಲ್ಪಡುತ್ತಿತ್ತೆಂಬು-ದನ್ನು
ಕೊಡಹಿ-ಕೊಳ್ಳ-ಬೇಕೆಂದು
ಕೊಡಹು
ಕೊಡಹು-ವೆನು
ಕೊಡಿ
ಕೊಡಿ-ಸ-ಬೇಕು
ಕೊಡಿ-ಸುತ್ತೇನೆ
ಕೊಡಿ-ಸುವ
ಕೊಡು
ಕೊಡು-ಗೆ-ಗಳಿಂದ
ಕೊಡು-ಗೆ-ಗಳೇ
ಕೊಡು-ಗೆ-ಯೆಂದೇ
ಕೊಡುತ್ತ
ಕೊಡುತ್ತದೆ
ಕೊಡುತ್ತಾ
ಕೊಡುತ್ತಾನೆ
ಕೊಡುತ್ತಾರೆ
ಕೊಡುತ್ತಾ-ರೆ-ಯಲ್ಲ
ಕೊಡುತ್ತಾಳೆ
ಕೊಡುತ್ತಿದೆ
ಕೊಡುತ್ತಿದ್ದ
ಕೊಡುತ್ತಿದ್ದನು
ಕೊಡುತ್ತಿದ್ದರು
ಕೊಡುತ್ತಿದ್ದ-ರೆಂದೂ
ಕೊಡುತ್ತಿದ್ದಾಗ
ಕೊಡುತ್ತಿದ್ದಾರೆಯೊ
ಕೊಡುತ್ತಿದ್ದಿರಿ
ಕೊಡುತ್ತಿದ್ದೆನು
ಕೊಡುತ್ತಿ-ರುವ-ರೆಂದೂ
ಕೊಡುತ್ತಿ-ರು-ವು-ದನ್ನು
ಕೊಡುತ್ತಿ-ರು-ವು-ದ-ರಿಂದಲೇ
ಕೊಡುತ್ತಿರು-ವೆವು
ಕೊಡುತ್ತೀಯೋ
ಕೊಡುತ್ತೇನೆ
ಕೊಡುತ್ತೇನೆಂದು
ಕೊಡುತ್ತೇವೆಂದು
ಕೊಡುವ
ಕೊಡು-ವಂತಾಗು-ವಷ್ಟು
ಕೊಡು-ವಂತೆ
ಕೊಡು-ವರು
ಕೊಡು-ವ-ವನು
ಕೊಡು-ವ-ವ-ರನ್ನು
ಕೊಡು-ವವರು
ಕೊಡು-ವ-ವ-ರೆಷ್ಟು
ಕೊಡು-ವಾಗಲೂ
ಕೊಡು-ವಿರಿ
ಕೊಡು-ವಿ-ರೆಂದು
ಕೊಡು-ವಿ-ರೇನು
ಕೊಡು-ವುದಕ್ಕಾ-ಗುತ್ತದೆ-ಯೇನು
ಕೊಡು-ವುದಕ್ಕಾ-ಗು-ವು-ದಿಲ್ಲ
ಕೊಡು-ವು-ದಕ್ಕೂ
ಕೊಡು-ವುದಕ್ಕೆ
ಕೊಡು-ವುದಕ್ಕೋಸ್ಕರ
ಕೊಡು-ವು-ದನ್ನು
ಕೊಡು-ವು-ದರ
ಕೊಡು-ವು-ದ-ರಲ್ಲಿ
ಕೊಡು-ವು-ದಾಗಿತ್ತು
ಕೊಡು-ವು-ದಿಲ್ಲ
ಕೊಡು-ವು-ದಿಲ್ಲವೆ
ಕೊಡು-ವು-ದಿಲ್ಲ-ವೆಂದು
ಕೊಡು-ವುದು
ಕೊಡು-ವುದೆಂದಿಟ್ಟು-ಕೊಳ್ಳೋಣ
ಕೊಡು-ವುವು
ಕೊಡು-ವೆನು
ಕೊಡು-ವೆಯಾ
ಕೊಡು-ವೆವು
ಕೊಣಪಣ
ಕೊತೂ
ಕೊತೊ-ಜಡ-ಜೀವ
ಕೊತೊ-ಮತೋ
ಕೊತೊಯಿ
ಕೊತೊಯಿ-ರೂಪ
ಕೊಥಾ
ಕೊನೆ
ಕೊನೆ-ಗಾಣು-ವಂತೆ
ಕೊನೆ-ಗಾಣು-ವುದು
ಕೊನೆ-ಗಾಣು-ವುದೋ
ಕೊನೆಗೂ
ಕೊನೆಗೆ
ಕೊನೆ-ಗೊಂಡ
ಕೊನೆ-ಗೊಂಡಾಗ
ಕೊನೆ-ಗೊಂದು
ಕೊನೆ-ಗೊಳ್ಳು-ವು-ದನ್ನು
ಕೊನೆ-ಗೊಳ್ಳುವುದು
ಕೊನೆಯ
ಕೊನೆ-ಯ-ತನಕ
ಕೊನೆ-ಯ-ದಾಗಿ
ಕೊನೆ-ಯ-ಪಕ್ಷ
ಕೊನೆ-ಯಲ್ಲಿ
ಕೊನೆ-ಯ-ವ-ರೆಗೂ
ಕೊನೆ-ಯಿಲ್ಲದ
ಕೊನೆ-ಯಿಲ್ಲ-ದಾಕಾಶ
ಕೊನೆ-ಯು-ಸಿರೆಳೆ-ಯುತ್ತಾನೆ
ಕೊಬ್ಬನ್ನು
ಕೊಬ್ಬು
ಕೊಯ್ದು
ಕೊಯ್ವನು
ಕೊರಗಿ
ಕೊರ-ತೆ-ಯಿದೆ
ಕೊರತೆ-ಯಿ-ರ-ಲಿಲ್ಲ
ಕೊರತೆ-ಯಿ-ರು-ವು-ದಿಲ್ಲ
ಕೊರ-ತೆಯೆ
ಕೊರ-ತೆಯೇ
ಕೊರಳ
ಕೊರಳಿಗೆ
ಕೊರ-ಳಿನಿಂ
ಕೊರಿ
ಕೊರಿಛೆ
ಕೊರೆತ-ವೆಂದರೆ
ಕೊರೆಯಲ್ಪ-ಡು-ವುವು
ಕೊರೆವ
ಕೊರೇ
ಕೊರೊ
ಕೊಲೆಗೆ
ಕೊಲೇ
ಕೊಲೊಂಬೋವಿ-ನಿಂದ
ಕೊಲ್ಲ-ಕೂಡ-ದೆಂಬುದೂ
ಕೊಲ್ಲ-ಬೇಕಾ-ಗುತ್ತದೆ
ಕೊಲ್ಲ-ಬೇಕು
ಕೊಲ್ಲ-ಲಾಗ-ಲಿಲ್ಲ
ಕೊಲ್ಲ-ಲಾ-ಗು-ವು-ದಿಲ್ಲ
ಕೊಲ್ಲಲು
ಕೊಲ್ಲಲ್ಪಟ್ಟರೂ
ಕೊಲ್ಲಲ್ಪಡುತ್ತವೆ
ಕೊಲ್ಲಿ
ಕೊಲ್ಲಿ-ಸಿ-ದರು
ಕೊಲ್ಲುತ್ತಾರೋ
ಕೊಲ್ಲುತ್ತಿ-ರ-ಲಿಲ್ಲ
ಕೊಲ್ಲು-ವುದಕ್ಕೆ
ಕೊಲ್ಲು-ವುದು
ಕೊಲ್ಲು-ವುವು
ಕೊಲ್ವ
ಕೊಳಕ
ಕೊಳಕ್ಕೂ
ಕೊಳದ
ಕೊಳ-ದಿಂದ
ಕೊಳಲನ್ನಿ-ಡು-ವರು
ಕೊಳಲನ್ನೂದುವ
ಕೊಳಲೂದು-ವು-ದ-ರಿಂದ
ಕೊಳು
ಕೊಳೆ
ಕೊಳೆತ
ಕೊಳೆತು
ಕೊಳೆ-ಯುವುದಕ್ಕಿಂತ
ಕೊಳ್ಳ-ಬಲ್ಲುದೋ
ಕೋ
ಕೋಟನ್ನು
ಕೋಟ-ಲೆ-ಗಳ
ಕೋಟಿ
ಕೋಟಿಗೆ
ಕೋಟಿಯ
ಕೋಟಿ-ಯನ್ನು
ಕೋಟು
ಕೋಟು-ಗಳ
ಕೋಟೆ-ಯೊ-ಳಗೆ
ಕೋಟ್ಯಧಿ-ಕ-ವಾಗಿ
ಕೋಟ್ಯಾಧೀಶ್ವರ-ರಾದರೆ
ಕೋಡು-ಬಂಡಿ-ಯಿಲ್ಲ
ಕೋಣ
ಕೋಣೆಗೆ
ಕೋಣೆಯ
ಕೋಣೆ-ಯನ್ನು
ಕೋಣೆ-ಯಲ್ಲಿ
ಕೋಣೆಯೂ
ಕೋಣೆ-ಯೊ-ಳಗೆ
ಕೋಪ
ಕೋಪ-ಗೊಳ್ಳುವೆ
ಕೋಪ-ದಿಂದ
ಕೋಪಿಸಿ-ಕೊಳ್ಳುವ-ನೆಂದು
ಕೋಮಲತೆ
ಕೋಮಲಸ್ವ-ಭಾವದ
ಕೋರಲಿ
ಕೋರಿಕೆಯ
ಕೋರಿಕೆ-ಯಿದೆ
ಕೋರು-ವು-ದಿಲ್ಲ
ಕೋರುವೆ-ನಿಂದು
ಕೋರೈಸಿ
ಕೋರೈಸಿವೆ
ಕೋರೈ-ಸುವ
ಕೋಲಾಹಲ-ದಲ್ಲಿಯೂ
ಕೋಲಾಹಲವೆದ್ದಿತು
ಕೋಲು
ಕೋಲ್ಮಿಂಚಿ-ನೊಲು
ಕೋಳಿ
ಕೋವಿ-ಯುಗುಳು-ತಿದೆ
ಕೋಶವೈ-ದಲ್ಲ
ಕೌಂತೇಯ
ಕೌತು-ಕದ
ಕೌತುಕ-ವಾಗಿ-ರು-ವು-ದನ್ನು
ಕೌತು-ಕವೇ
ಕೌಪೀನ
ಕೌಪೀನ-ವಂತಃ
ಕೌಪೀನ-ವನ್ನು
ಕೌಲ-ರೆಂದೂ
ಕೌಶಲ-ವನ್ನು
ಕೌಶಲ-ವಿಲ್ಲವೋ
ಕೌಶಲ್ಯ
ಕೌಶಲ್ಯ-ವನ್ನು
ಕ್ಯಾಂಡ-ಲನ್ನು
ಕ್ಯಾಥೋಲಿಕ್
ಕ್ಯಾಲಿಫೋರ್ನಿಯಾ-ದಲ್ಲಿ
ಕ್ಯಾಲಿಫೋರ್ನಿಯಾ-ದಲ್ಲಿದ್ದಾಗ
ಕ್ರಂದನ
ಕ್ರಮ
ಕ್ರಮಕ್ರಮ-ವಾಗಿ
ಕ್ರಮ-ಗಳನ್ನ-ನು-ಸರಿಸುತ್ತಿದ್ದ
ಕ್ರಮ-ಗಳನ್ನು
ಕ್ರಮ-ದಲ್ಲಿ
ಕ್ರಮ-ದಿಂದ
ಕ್ರಮ-ವನ್ನು
ಕ್ರಮ-ವಾಗಿ
ಕ್ರಮ-ವಿಕಾಸ
ಕ್ರಮ-ವಿಕಾಸದ
ಕ್ರಮ-ವಿಕಾಸ-ವಾದದ
ಕ್ರಮವು
ಕ್ರಮೇಣ
ಕ್ರಮೋನ್ನ-ತಿಯಾಗು-ವು-ದಾದರೆ
ಕ್ರಿ
ಕ್ರಿಪೂ
ಕ್ರಿಮಿ-ಕೀಟ
ಕ್ರಿಮಿ-ಗಳ
ಕ್ರಿಯಾ-ಕಲಾ-ಪ-ಗಳನ್ನು
ಕ್ರಿಯಾತ್ಮಕ-ನಾಗುತ್ತಾನೆ
ಕ್ರಿಯಾತ್ಮ-ಕ-ವಾಗಿ
ಕ್ರಿಯಾತ್ಮಕ-ವಾದ
ಕ್ರಿಯಾದಿ-ಗಳನ್ನು
ಕ್ರಿಯಾ-ಪದ
ಕ್ರಿಯಾ-ಪದ-ಗಳ
ಕ್ರಿಯಾ-ಪದ-ಗಳನ್ನು
ಕ್ರಿಯಾ-ಪದದ
ಕ್ರಿಯಾ-ಪದ-ವನ್ನು
ಕ್ರಿಯಾ-ಶೀಲ-ವಾಗಿ
ಕ್ರಿಯಾ-ಶೀಲ-ವಾಗಿದ್ದು
ಕ್ರಿಯೆ
ಕ್ರಿಯೆ-ಗಳಲ್ಲಿ
ಕ್ರಿಯೆ-ಗಳು
ಕ್ರಿಯೆ-ಗಳೆಲ್ಲ
ಕ್ರಿಯೆಗೆ
ಕ್ರಿಯೆ-ಯನ್ನು
ಕ್ರಿಶ
ಕ್ರಿಶರ
ಕ್ರಿಶ್ಚಿಯನ್
ಕ್ರಿಶ್ಚಿಯನ್ನರು
ಕ್ರಿಸ್ಟೀ-ನೆಗೆ
ಕ್ರಿಸ್ತ
ಕ್ರಿಸ್ತನ
ಕ್ರಿಸ್ತ-ನಂತೆ
ಕ್ರಿಸ್ತ-ನನ್ನು
ಕ್ರಿಸ್ತ-ನಿ-ಗಿಂತ
ಕ್ರಿಸ್ತ-ನಿಗೆ
ಕ್ರಿಸ್ತನು
ಕ್ರಿಸ್ತನೂ
ಕ್ರೀಟ್
ಕ್ರೀಡಾ-ವಸ್ತು-ಗಳು
ಕ್ರೂರಿ
ಕ್ರೂರಿ-ಯಲ್ಲ
ಕ್ರೈಸ್ಟ್
ಕ್ರೈಸ್ತ
ಕ್ರೈಸ್ತ-ಧರ್ಮ
ಕ್ರೈಸ್ತ-ಧರ್ಮವೂ
ಕ್ರೈಸ್ತ-ಮತ
ಕ್ರೈಸ್ತ-ಮತಕ್ಕೊದಗಿದ
ಕ್ರೈಸ್ತ-ಮ-ತದ
ಕ್ರೈಸ್ತ-ಮತ-ದಿಂದ
ಕ್ರೈಸ್ತರ
ಕ್ರೈಸ್ತ-ರಾಗುತ್ತಿದ್ದಾರೆ
ಕ್ರೈಸ್ತ-ರಿಗೆ
ಕ್ರೈಸ್ತರು
ಕ್ರೈಸ್ತ-ರೆಂತಲೂ
ಕ್ರೋಧ
ಕ್ರೋಧ-ಗಳ
ಕ್ರೋಧ-ದಿಂದ
ಕ್ರೋಧ-ವನ್ನು
ಕ್ರೋಧ-ವೆಂದೂ
ಕ್ರೋಧ-ವೆಲ್ಲ
ಕ್ರೌರ್ಯ
ಕ್ರೌರ್ಯ-ಸಾ-ಸಿರವ
ಕ್ಲಬ್
ಕ್ಲಿಷ್ಟ
ಕ್ಲೇಶ-ಗಳಿಂದ
ಕ್ವ
ಕ್ವಗತಂ
ಕ್ವಾಂಬಾ
ಕ್ಷಂತವ್ಯ
ಕ್ಷಣ
ಕ್ಷಣಕ್ಕೆ
ಕ್ಷಣ-ಗಳನ್ನು
ಕ್ಷಣ-ಗಳಲ್ಲಿ
ಕ್ಷಣ-ದಲ್ಲಿ
ಕ್ಷಣ-ದಲ್ಲಿಯೇ
ಕ್ಷಣ-ದಲ್ಲೂ
ಕ್ಷಣ-ದ-ವ-ರೆಗೂ
ಕ್ಷಣ-ದಷ್ಟೇ
ಕ್ಷಣದಿ
ಕ್ಷಣ-ಭಂಗುರ-ಗ-ಳೆಂದು
ಕ್ಷಣ-ಭಂಗುರ-ವಾದ
ಕ್ಷಣ-ಮಾತ್ರ
ಕ್ಷಣ-ಮಾತ್ರ-ದಲ್ಲಿ
ಕ್ಷಣ-ಮಾತ್ರ-ವಿರ-ತಕ್ಕವು
ಕ್ಷಣ-ವನ್ನೇ
ಕ್ಷಣವೂ
ಕ್ಷಣವೆ
ಕ್ಷಣವೇ
ಕ್ಷಣ-ವೊಂದು
ಕ್ಷಣಿಕ
ಕ್ಷಣಿಕ-ವಾದ
ಕ್ಷಣಿಕ-ವೆಂಬುದು
ಕ್ಷಣೇ
ಕ್ಷತ್ರಿಯ
ಕ್ಷತ್ರಿಯ-ಕುಲಕ್ಕೇ
ಕ್ಷತ್ರಿಯ-ನಾಗುತ್ತಿದ್ದ
ಕ್ಷತ್ರಿ-ಯರ
ಕ್ಷತ್ರಿಯ-ರದು
ಕ್ಷತ್ರಿಯ-ರಿಗೆ
ಕ್ಷತ್ರಿ-ಯರು
ಕ್ಷತ್ರಿ-ಯರೇ
ಕ್ಷತ್ರಿ-ಯಾದಿ
ಕ್ಷಮಾ
ಕ್ಷಮಾ-ಪಣೆ
ಕ್ಷಮಿಸ-ಬೇಕು
ಕ್ಷಮಿಸು
ಕ್ಷಮೆ
ಕ್ಷಮೆಯ
ಕ್ಷಮೆ-ಯನ್ನು
ಕ್ಷಮೆ-ಯನ್ನೇಕೆ
ಕ್ಷಾತ್ರ
ಕ್ಷಾಮ
ಕ್ಷಾಮದ
ಕ್ಷಾಮ-ದಿಂದ
ಕ್ಷಾಮ-ವನ್ನು
ಕ್ಷಾಮ-ವಿ-ದೆಯೊ
ಕ್ಷೀಣ
ಕ್ಷೀಣ-ವಾಗಿದೆ
ಕ್ಷೀಣ-ವಾದ
ಕ್ಷೀಣಾಃ
ಕ್ಷೀರ-ಭವಾನಿ
ಕ್ಷೀರ-ಭವಾನಿಯ
ಕ್ಷೀರ-ಸಾಗರ-ದಲ್ಲಿ
ಕ್ಷೀರೇ
ಕ್ಷುದ್ರ
ಕ್ಷುದ್ರ-ತನ
ಕ್ಷುದ್ರ-ದೇಹದ
ಕ್ಷುದ್ರರು
ಕ್ಷುದ್ರ-ವಾ-ಗಿ-ರಲಿ
ಕ್ಷುದ್ರ-ವಾ-ದೊಂದು
ಕ್ಷುಧೆ
ಕ್ಷುಬ್ಧ-ಗೊಳಿಸಿತು
ಕ್ಷೇತ್ರ
ಕ್ಷೇತ್ರಕ್ಕೆ
ಕ್ಷೇತ್ರದ
ಕ್ಷೇತ್ರ-ದಲ್ಲಿ
ಕ್ಷೇತ್ರ-ದಲ್ಲೇ
ಕ್ಷೇತ್ರ-ವನ್ನಾಗಿ
ಕ್ಷೇಮ
ಕ್ಷೇಮ-ವೆಂದು
ಕ್ಷೋಭ-ವನ್ನು
ಕ್ಷೋಭೆ
ಕ್ಷೋಭೆ-ಗೊಳ್ಳ-ಬೇಕು
ಕ್ಷೋಭೆ-ಯಿಂದಲೂ
ಕ್ಷೌರಿಕನ
ಕ್ಷೌರಿಕ-ನಾದ
ಕ್ಷೌರಿಕ-ನಿಗೂ
ಕ್ಷೌರಿಕ-ನಿಗೆ
ಖಂಡ
ಖಂಡ-ಗಳ
ಖಂಡ-ಗಳನ್ನಾಗಿಯೂ
ಖಂಡ-ದಲ್ಲಿ
ಖಂಡನ
ಖಂಡನೆ
ಖಂಡವೂ
ಖಂಡಿತ
ಖಂಡಿತ-ವಾಗಿ
ಖಂಡಿತ-ವಾಗಿಯೂ
ಖಂಡಿತ-ವಾದ
ಖಂಡಿಸ-ಬೇಕು
ಖಂಡಿಸಿ
ಖಂಡಿ-ಸಿದೆ
ಖಂಡಿಸಿದ್ದಾರೆ
ಖಂಡಿ-ಸುತ್ತಿದ್ದರು
ಖಂಡಿ-ಸುವಾ-ತನೆ
ಖಂಡಿ-ಸುವೆ
ಖಗ-ಕುಲ
ಖಗೋಲ
ಖಚಿತ
ಖಚಿತ-ವಾಗಿ
ಖಡ್ಗ
ಖಡ್ಗ-ವನೊಲ್ಲದೆ
ಖಡ್ಗ-ವನ್ನು
ಖಡ್ಗ-ವಾಗಿ-ದೆಯೋ
ಖಡ್ಗವು
ಖನಾ
ಖನಿ-ಗಳು
ಖರೀ-ದಿಗೆ
ಖರೀದಿ-ಯಾ-ಯಿತು
ಖರ್ಚನ್ನು
ಖರ್ಚನ್ನೆಲ್ಲಾ
ಖರ್ಚಿ-ಗಾಗಿ
ಖರ್ಚು-ಪಟ್ಟಿ-ಯನ್ನು
ಖರ್ಚು-ಮಾಡ-ಬೇಡಿ
ಖರ್ಚು-ಮಾಡುವ
ಖರ್ಚು-ಮಾಡು-ವುದು
ಖರ್ಚೂ
ಖರ್ಚೆಲ್ಲವೂ
ಖಲು
ಖಲ್ವಿದಂ
ಖಸಾಯಿಯೆ
ಖಸೆ
ಖಾನಿ
ಖಾನೆಗೆ
ಖಾನೆ-ಯಲ್ಲಿ
ಖಾಯಿಲೆ
ಖಾಯಿಲೆಯ
ಖಾಯಿಲೆ-ಯಲ್ಲಿದ್ದರೂ
ಖಾಯಿಲೆ-ಯಾಗಿ
ಖಾಯಿಲೆಯೂ
ಖಾರ-ವಾಗಿ
ಖಾಲಿ-ಬಿದ್ದರೆ
ಖಿನ್ನ-ನಾಗಿ
ಖುಂಜಿಛೊ
ಖುಲೆ
ಖುಷಿ-ಯಾಗಿ
ಖೇತ್ರಿಯ
ಖೋಲ್
ಖ್ಯಾತಿ
ಗಂಗಾ
ಗಂಗಾ-ತೀರ
ಗಂಗಾ-ತೀರ-ದಲ್ಲಿ
ಗಂಗಾ-ತೀರ-ದಲ್ಲಿ-ರುವ
ಗಂಗಾ-ದರ್ಶನ-ವಾಗಲು
ಗಂಗಾ-ನದಿ
ಗಂಗಾ-ನ-ದಿಯ
ಗಂಗಾ-ನದಿ-ಯಲ್ಲಿ
ಗಂಗಾ-ನದಿ-ಯಲ್ಲೇ
ಗಂಗಾ-ನದಿ-ಯಿದೆ
ಗಂಗಾ-ನ-ದಿಯು
ಗಂಗಾ-ಮಾತೆಯ
ಗಂಗಾಸ್ನಾನ
ಗಂಗಾಸ್ನಾನಕ್ಕಾಗಿ
ಗಂಗಾಸ್ನಾನ-ಮಾಡಿ
ಗಂಗಾಸ್ನಾನ-ವಾದ
ಗಂಗೆ
ಗಂಗೆಗೆ
ಗಂಗೆಗೇಂ
ಗಂಗೆಯ
ಗಂಗೆ-ಯಲ್ಲಿ
ಗಂಗೆಯು
ಗಂಗೆ-ಯುದ್ಭವ-ವಾದ
ಗಂಗೆಯೇ
ಗಂಜಿ
ಗಂಜಿ-ಯನ್ನು
ಗಂಟ-ಲಿನೊ-ಳಕ್ಕೆ
ಗಂಟಲೊ-ಳಕ್ಕೆ
ಗಂಟಾ-ಘೋಷ-ವಾಗಿ
ಗಂಟು
ಗಂಟು-ಗಳೆಲ್ಲ
ಗಂಟೆ
ಗಂಟೆ-ಗಳ
ಗಂಟೆ-ಗಿಂತ
ಗಂಟೆಗೆ
ಗಂಟೆ-ಗೆಲ್ಲಾ
ಗಂಟೆಯ
ಗಂಟೆ-ಯನ್ನು
ಗಂಟೆ-ಯಲ್ಲಿ
ಗಂಟೆ-ಯ-ವ-ರೆಗೂ
ಗಂಟೆ-ಯಾಗಿ
ಗಂಟೆ-ಯಾಗಿತ್ತು
ಗಂಟೆ-ಯಾ-ಗಿದೆ
ಗಂಟೆ-ಯಾಗಿ-ರ-ಬ-ಹುದು
ಗಂಟೆ-ಯಾದ
ಗಂಟೆ-ಯಾದರೂ
ಗಂಟೆ-ಯಾ-ಯಿತು
ಗಂಟೆ-ಯಿಂದ
ಗಂಟೆಯೂ
ಗಂಡ
ಗಂಡಂದಿ-ರನ್ನು
ಗಂಡನ
ಗಂಡ-ನನ್ನು
ಗಂಡ-ನಲ್ಲ-ದ-ವ-ರೆಲ್ಲಾ
ಗಂಡ-ನಿ-ಗಾಗಿ
ಗಂಡನು
ಗಂಡ-ಸನ್ನು
ಗಂಡ-ಸರ
ಗಂಡ-ಸ-ರನ್ನು
ಗಂಡ-ಸ-ರಿ-ಗಿಂತಲೂ
ಗಂಡ-ಸ-ರಿಗೂ
ಗಂಡ-ಸ-ರಿಗೆ
ಗಂಡ-ಸ-ರಿ-ಗೋಸ್ಕರ
ಗಂಡ-ಸರು
ಗಂಡ-ಸಿಗೆ
ಗಂಡಸು
ಗಂಭೀರ
ಗಂಭೀರ-ತೆ-ಯನ್ನು
ಗಂಭೀರ-ಭಾವ-ದಿಂದ
ಗಂಭೀರ-ರಾಗ-ಬೇಕೆಂದು
ಗಂಭೀರ-ವಾಗ-ಲಾ-ರಂಭಿ-ಸಿದಾಗ
ಗಂಭೀರ-ವಾಗಿ
ಗಂಭೀರ-ವಾಗಿದ್ದಿತು
ಗಂಭೀರ-ವಾದ
ಗಗನ
ಗಗನ-ದೊಳು
ಗಗನ-ಮಂಡಲ-ವೆಲ್ಲ
ಗಗನ-ವನ್ನು
ಗಗ-ನವೇ
ಗಗನಾಭಂ
ಗಗನಾಭೇ
ಗಚ್ಛಂತ್ಯಲಂ
ಗಚ್ಛತಿ
ಗಚ್ಛತು
ಗಟ್ಟಿ-ಮಾಡಿ-ರ-ಬೇಕೆಂದು
ಗಟ್ಟಿ-ಯಾಗಿ
ಗಟ್ಟಿ-ಯಾದ
ಗಠನ್
ಗಡಿಗೆ-ಗಳಿವೆಯೋ
ಗಡಿಗೆಯ
ಗಡಿ-ಬಿಡಿ
ಗಡಿ-ಬಿಡಿ-ಯಿಂದ
ಗಡಿಯ
ಗಡಿ-ಯಾರದ
ಗಡಿ-ಯೇಕೆ
ಗಡೇ
ಗಣ
ಗಣ-ಗಳ
ಗಣನ
ಗಣನಂ
ಗಣ-ನೀಯ-ವಲ್ಲ
ಗಣ-ನೆಗೆ
ಗಣ-ನೆಗೇ
ಗಣಿ
ಗಣಿ-ತವೂ
ಗಣಿ-ಯಾ-ಗಿದೆ
ಗಣಿ-ಸದ-ವರನು
ಗಣಿ-ಸಿದ-ವ-ರಾರು
ಗಣಿ-ಸೆನು
ಗಣೇಶನು
ಗಣ್ಯ-ರಾಗಿ-ರುತ್ತಾರೆ
ಗಣ್ಯರು
ಗಣ್ಯ-ವಾ-ಗು-ವು-ದಿಲ್ಲ
ಗತಪ್ರಾಣ-ಳಾದ
ಗತ-ಭಯಾ
ಗತ-ಸಂಶಯ
ಗತಿ
ಗತಿಂ
ಗತಿಃ
ಗತಿ-ಗೆಟ್ಟ
ಗತಿಯೇ
ಗತಿ-ಯೇನು
ಗತಿ-ರನ್ಯಥಾ
ಗತಿರ್ಭವೇತ್
ಗತಿಸ್ಥಿತಿ
ಗದ್ಗದ
ಗದ್ದಲ
ಗದ್ದಲದ
ಗದ್ದಲ-ದಲ್ಲಿ
ಗದ್ದಲ-ಮಾಡಿ
ಗದ್ದೆ-ಯಲ್ಲೆಲ್ಲಾ
ಗಭೀರ್
ಗಮನ
ಗಮನ-ಕೊಟ್ಟ-ವ-ರಲ್ಲ
ಗಮನ-ಕೊ-ಡದೆ
ಗಮನ-ವಿಟ್ಟು
ಗಮನ-ವಿ-ರ-ಲಿಲ್ಲ
ಗಮ-ನವೇ
ಗಮನಾಯ
ಗಮನಿ-ಸ-ತಕ್ಕ
ಗಮನಿ-ಸ-ಬಹು-ದಾ-ದಂತೆ
ಗಮನಿ-ಸ-ಬೇ-ಕಾದ
ಗಮನಿ-ಸ-ಬೇ-ಕಾ-ದುದು
ಗಮನಿ-ಸ-ಬೇಕು
ಗಮನಿ-ಸ-ಬೇಡ
ಗಮ-ನಿಸಿ
ಗಮ-ನಿಸಿ-ದನು
ಗಮ-ನಿಸಿ-ದರೆ
ಗಮನಿ-ಸು-ವು-ದಿಲ್ಲ
ಗಮ-ನಿ-ಸು-ವು-ದಿಲ್ಲವೆ
ಗರಜಿ
ಗರಜೇ
ಗರಡಿಯ
ಗರಳ
ಗರ್ಗರಿ-ಸುತ
ಗರ್ಜನೆ
ಗರ್ಜನೆ-ಯನ್ನು
ಗರ್ಜನೆ-ಯಿಂದ
ಗರ್ಜನ್
ಗರ್ಜರಿ-ಸುತ
ಗರ್ಜಿ
ಗರ್ಜಿಸಿ
ಗರ್ಜಿ-ಸಿದೆ
ಗರ್ಜಿ-ಸು-ವು-ದನ್ನೂ
ಗರ್ಜಿಸೆ
ಗರ್ಜೆ
ಗರ್ಭ-ಗುಡಿಯ
ಗರ್ಭ-ದಲ್ಲಿ
ಗರ್ಭಾತ್ಮ-ಕವೂ
ಗರ್ವ-ಪಡು-ವಿ-ರಲ್ಲವೆ
ಗಲಿ-ತತಿಮಿರ-ಮಾಲಃ
ಗಲಿ-ಬಿಲಿ
ಗಲಿ-ಬಿ-ಲಿಗೆ
ಗಲಿ-ಬಿಲಿ-ಗೊಳ್ಳುವುದು
ಗಲಿ-ಬಿಲಿ-ಯಾ-ಗಿದೆ
ಗಲೆ
ಗಲ್ಲಿ
ಗಲ್ಲಿ-ಗಳಲ್ಲಿಯೂ
ಗಲ್ಲಿ-ಯಲ್ಲಿ-ರುವ
ಗಳ
ಗಳಾಗಿದ್ದಾ-ರಲ್ಲಾ
ಗಳಿ-ಗಿಂತಲು
ಗಳಿಗೆ
ಗಳಿಗೆ-ಯಲ್ಲಿ
ಗಳಿಗೆ-ಯಲ್ಲೂ
ಗಳಿಗೆಯೂ
ಗಳಿ-ಸ-ಲಾ-ಗು-ವು-ದಿಲ್ಲ
ಗಳಿ-ಸಲು
ಗಳಿಸಿ
ಗಳಿ-ಸಿದ
ಗಳಿಸಿ-ದ-ಮೇಲೆ
ಗಳಿಸಿ-ಬಿಟ್ಟು
ಗಳಿ-ಸುವೆ
ಗಹಗಹಿಸಿ
ಗಹನ
ಗಹನ-ವಾದ
ಗಹನಾಲೋಚನೆ
ಗಹ್ವರ
ಗಾಂಭೀರ್ಯ
ಗಾಂಭೀರ್ಯವೇ
ಗಾಗರೀಯ
ಗಾಜನ್ನು
ಗಾಜಿನ
ಗಾಜಿ-ನದು
ಗಾಡಾಂಧಕಾ-ರ-ದಲ್ಲಿ
ಗಾಡಿ
ಗಾಡಿಯ
ಗಾಡಿ-ಯನ್ನು
ಗಾಡಿ-ಯಲ್ಲಿ
ಗಾಡಿ-ಯ-ವ-ನಿಗೆ
ಗಾಡಿ-ಯಿಂದ
ಗಾಡಿಯು
ಗಾಡಿ-ಯೊ-ಳಕ್ಕೆ
ಗಾಡಿ-ಯೊ-ಳಗೆ
ಗಾಢ
ಗಾಢಧ್ಯಾನ
ಗಾಢ-ನಿದ್ರೆ-ಯಲ್ಲಿ
ಗಾಢ-ವಾಗಿ
ಗಾಢ-ವಾಗುತ್ತ
ಗಾಢ-ವಾಗುತ್ತಾ
ಗಾಢ-ವಾದ
ಗಾಢಾಂಧ-ಕಾರದ
ಗಾತ್ರನು
ಗಾಥೆ-ಯನ್ನೂ
ಗಾದೆಯನ್ನನು
ಗಾದೆ-ಯಿದೆ
ಗಾನ
ಗಾನ-ಗಳಲ್ಲಿ
ಗಾನ-ದಲಿ
ಗಾನ-ದಲ್ಲಿ
ಗಾನ-ಮಾಡು-ವಂತೆ
ಗಾನ-ಮಾಡು-ವುದಕ್ಕೆ
ಗಾನ-ವದು
ಗಾನ-ವನು
ಗಾನ-ವನ್ನು
ಗಾನ-ವಾದರೂ
ಗಾನ-ಸುಧೆಗೆ
ಗಾಬರಿ
ಗಾಬರಿ-ಗೊಂಡು
ಗಾಯಕ್ಕೆ
ಗಾಯತ್ರಿ
ಗಾಯತ್ರಿ-ಯನ್ನು
ಗಾಯಿ
ಗಾಯಿಛೇ
ಗಾಯ್
ಗಾರಹಸ್ಥ-ಸಂನ್ಯಾಸ
ಗಾರೆಯ
ಗಾರ್ಗಿ
ಗಾರ್ಗಿ-ಯರಂತಹ
ಗಾರ್ಹಸ್ಥ್ಯ
ಗಾಲ
ಗಾಲ್
ಗಾಳಿ
ಗಾಳಿಗೆ
ಗಾಳಿ-ಯದು
ಗಾಳಿ-ಯನ್ನು
ಗಾಳಿ-ಯಲ್ಲಿ
ಗಾಳಿ-ಯಾಡುವ
ಗಾಳಿ-ಯಿಲ್ಲದ
ಗಾಳಿ-ಸಂಚಾರ
ಗಿಕ್ಷೆ-ಗಳನ್ನು
ಗಿಡ
ಗಿಡ-ಗ-ಳಿದ್ದು
ಗಿಡದ
ಗಿಡ-ವಾಗು-ವುದು
ಗಿಡ್ಡ
ಗಿಣಿ-ಪಾಠ
ಗಿರಗಿರನೆ
ಗಿರವರ
ಗಿರಿ-ಗಳ
ಗಿರಿ-ಗುಹೆ
ಗಿರಿ-ಗುಹೆ-ಕಂದ-ರದ
ಗಿರಿ-ತರಂಗ-ವೆಬ್ಬಿ-ಸುತ್ತ
ಗಿರಿ-ಶಿಖರ-ಗಳಲ್ಲಿ
ಗಿರೀಂದ್ರ-ನಾಥ
ಗಿರೀಶ
ಗಿರೀಶ-ಘೋಷ
ಗಿರೀಶ-ಚಂದ್ರ
ಗಿರೀಶ-ಚಂದ್ರ-ಘೋಷ-ರಲ್ಲಿ
ಗಿರೀಶ-ಚಂದ್ರ-ಘೋಷ್
ಗಿರೀಶ-ಬಾಬು
ಗಿರೀಶ-ಬಾಬು-ಗಳ
ಗಿರೀಶ-ಬಾಬು-ಗಳಂತೆ
ಗಿರೀಶ-ಬಾಬು-ಗಳನ್ನು
ಗಿರೀಶ-ಬಾಬು-ಗಳಿಗೆ
ಗಿರೀಶ-ಬಾಬು-ಗಳು
ಗಿರೀಶ-ಬಾಬು-ಗಳೂ
ಗಿರೀಶರು
ಗಿಲ್ಲದೆ
ಗಿಳಿ
ಗಿಳಿ-ಗಳನ್ನಾಗಿ
ಗಿಳಿ-ಪಾಠ
ಗೀತ
ಗೀತಂ
ಗೀತ-ಗೋವಿಂದದ
ಗೀತಾ
ಗೀತೆ
ಗೀತೆಯ
ಗೀತೆ-ಯಂತೆ
ಗೀತೆ-ಯನ್ನು
ಗೀತೆ-ಯಲ್ಲಿ
ಗೀತೆ-ಯಲ್ಲಿಯೂ
ಗೀತೆ-ಯಲ್ಲೆಲ್ಲ
ಗೀತೆ-ಯಾಗಿದ್ದು
ಗೀತೆಯು
ಗೀತೋ
ಗೀಳು
ಗೀಹೆ
ಗುಂಗುರು
ಗುಂಡಿಗೆ
ಗುಂಡಿಗೆ-ಯವರು
ಗುಂಡಿಟ್ಟು
ಗುಂಡಿಯ
ಗುಂಡಿ-ಯಲ್ಲಿ
ಗುಂಡು-ಗಳ
ಗುಂಪನ್ನೇ
ಗುಂಪಾಗಿ
ಗುಂಪಾ-ಗಿದೆ
ಗುಂಪಿಗೆ
ಗುಂಪಿನ
ಗುಂಪಿ-ನಿಂದ
ಗುಂಪು
ಗುಂಪು-ಗಳಾ-ದುವೆಂದು
ಗುಂಪು-ಗಳು
ಗುಂಪು-ಗೂ-ಡಿದ
ಗುಂಪು-ಗೂಡಿದ್ದಾರೆ
ಗುಜುಗುಜು
ಗುಟ್ಟ-ನರಿ-ತಿ-ರುವೆ
ಗುಟ್ಟಾಗಿ
ಗುಟ್ಟು-ಗಳಿವು
ಗುಟ್ಟೆಂದರೆ
ಗುಡಿ
ಗುಡಿ-ಗುಡಿ
ಗುಡಿ-ಗುಡಿ-ಯನ್ನು
ಗುಡಿ-ಗೋಪು-ರಾದಿ-ಗಳನ್ನು
ಗುಡಿ-ಯಲ್ಲಿ
ಗುಡಿ-ಯೊ-ಳದು
ಗುಡಿ-ಸ-ಲನ್ನು
ಗುಡಿ-ಸ-ಲಲ್ಲಿ
ಗುಡಿ-ಸ-ಲಿಗೆ
ಗುಡಿ-ಸಲೇ
ಗುಡಿ-ಸಿ-ಲಿ-ನಲ್ಲಿ
ಗುಡಿ-ಸುವ
ಗುಡಿ-ಸುವ-ವರು
ಗುಡು-ಗದು
ಗುಡುಗಿರೆ
ಗುಡುಗು-ವುದು
ಗುಡ್ಡ-ಕಾಡು-ಮೇಡು
ಗುಡ್ವಿನ್
ಗುಡ್ವಿನ್ನನ
ಗುಣ
ಗುಣಕ್ಕೆ
ಗುಣ-ಗಳನ್ನು
ಗುಣ-ಗಳನ್ನೆಲ್ಲ
ಗುಣ-ಗಳಲ್ಲಿ
ಗುಣ-ಗಳಷ್ಟೇ
ಗುಣ-ಗಳಿಂದ
ಗುಣ-ಗಳಿಂದಾಗಿ-ರುವನು
ಗುಣ-ಗಳಿಗೆ
ಗುಣ-ಗಳಿ-ರು-ವಂತೆಯೇ
ಗುಣ-ಗಳಿಲ್ಲವೆ
ಗುಣ-ಗಳಿವೆ
ಗುಣ-ಗಳು
ಗುಣ-ಗಳೂ
ಗುಣ-ಗಳೆಲ್ಲಾ
ಗುಣ-ಗಳೇ
ಗುಣ-ಗಾನ-ದಲ್ಲಿಯೂ
ಗುಣ-ಜಿದ್ಗುಣೇಡ್ಯೋ
ಗುಣ-ಮ-ಯನೆ
ಗುಣ-ಮುಖ-ವಾಗಿಲ್ಲ
ಗುಣ-ವ-ತಿಯ-ರಾದ
ಗುಣ-ವನ್ನು
ಗುಣ-ವಾಗಿದೆ
ಗುಣ-ವಾಗುತ್ತಿರಬೇ-ಕಲ್ಲವೇ
ಗುಣ-ವಾದ
ಗುಣವೇ
ಗುಣಾತೀತ-ವಾಗಿ-ದೆಯೋ
ಗುಣಾವ-ಗುಣ
ಗುದ್ದಲಿ-ಯನ್ನು
ಗುದ್ದಲಿ-ಯಿಂದ
ಗುದ್ದಾಡುತಿರೆ
ಗುದ್ದಾಡು-ವನು
ಗುಪ್ತರು
ಗುಪ್ತ-ವಾದ
ಗುಪ್ತ-ಶಕ್ತಿ
ಗುಮಾಸ್ತ
ಗುಮಾಸ್ತ-ಗಿರಿ
ಗುಮಾಸ್ತ-ರನ್ನು
ಗುರಿ
ಗುರಿ-ಗಾಗಿ
ಗುರಿಗೆ
ಗುರಿಯ
ಗುರಿ-ಯನು
ಗುರಿ-ಯ-ನೆಂದೆಂದು
ಗುರಿ-ಯನ್ನು
ಗುರಿ-ಯನ್ನೇ
ಗುರಿ-ಯ-ವರೆಗೆ
ಗುರಿ-ಯಾಗಿ
ಗುರಿ-ಯಾಗಿಟ್ಟು-ಕೊಂಡಿರು-ವರೋ
ಗುರಿ-ಯಾಗಿಟ್ಟು-ಕೊಂಡು
ಗುರಿ-ಯಾಗಿಟ್ಟು-ಕೊಳ್ಳ-ಬೇಕು
ಗುರಿ-ಯಾಗಿತ್ತು
ಗುರಿ-ಯಾ-ಗಿದೆ
ಗುರಿ-ಯಾಗಿದ್ದಾರೆ
ಗುರಿ-ಯಾಗಿ-ರ-ಬೇಕು
ಗುರಿ-ಯಾಗಿ-ರ-ಬೇಕೆಂದು
ಗುರಿ-ಯಾಗುತ್ತದೆ
ಗುರಿ-ಯಾಗುಳ್ಳ
ಗುರಿ-ಯಾದ
ಗುರಿಯು
ಗುರಿಯೂ
ಗುರಿಯೆ
ಗುರಿ-ಯೆ-ಡೆಗೆ
ಗುರಿ-ಯೆ-ಡೆಗೇ
ಗುರಿಯೇ
ಗುರು
ಗುರುಂ
ಗುರು-ಕುಲ
ಗುರು-ಕುಲ-ವಾಸ
ಗುರು-ಗಳ
ಗುರು-ಗಳನ್ನು
ಗುರು-ಗಳಾಗಿ-ರುವರು
ಗುರು-ಗಳಾಗುವರು
ಗುರು-ಗಳಾಗು-ವುದಕ್ಕೆ
ಗುರು-ಗಳಿಗೆ
ಗುರು-ಗಳು
ಗುರು-ಗಳೆ
ಗುರು-ಗ-ಳೆಂದು
ಗುರು-ಗೋವಿಂದರ
ಗುರು-ಗೋವಿಂದರು
ಗುರು-ಜನ-ರೆಲ್ಲಾ
ಗುರು-ತನ್ನು
ಗುರು-ತಾಯಿತು
ಗುರು-ತಿನ
ಗುರು-ತಿಸ-ಬ-ಹುದು
ಗುರು-ತಿಸ-ಬೇಕು
ಗುರು-ತಿಸಿ
ಗುರು-ತಿಸಿ-ದರೆ
ಗುರು-ತಿ-ಸುತ್ತಾರೆ
ಗುರು-ತಿ-ಸು-ವುದೇ
ಗುರುತು
ಗುರು-ತು-ಹಾಕಿಟ್ಟಿದ್ದ
ಗುರು-ತು-ಹಾಕುತ್ತಿದ್ದನು
ಗುರುತೂ
ಗುರುತ್ವ-ವೆಂಬ
ಗುರುತ್ವಾ-ಕರ್ಷಣ-ದಂತೆ
ಗುರು-ದಕ್ಷಿಣೆ-ಯನ್ನು
ಗುರು-ದೇವ
ಗುರು-ದೇವನ
ಗುರು-ದೇವ-ವರೇಣ್ಯ
ಗುರು-ಪದ
ಗುರು-ಪದಕೆ
ಗುರು-ಪರಂಪರೆ-ಯನ್ನು
ಗುರು-ಭಕ್ತಿ
ಗುರು-ಭಕ್ತಿ-ಯಿದ್ದರೆ
ಗುರು-ಭಾಯಿ-ಗಳ
ಗುರು-ಭಾಯಿ-ಗಳಿಗೆ
ಗುರುಭ್ರಾತೃ-ಗಳ
ಗುರುಭ್ರಾತೃ-ಗಳಲ್ಲಿ
ಗುರುಭ್ರಾತೃ-ಗಳಿಗೆ
ಗುರುಭ್ರಾತೃ-ಗಳು
ಗುರುಭ್ರಾತೃ-ಗಳೂ
ಗುರುಭ್ರಾತೃ-ಗಳೊ-ಡನೆ
ಗುರುಭ್ರಾತೃ-ವಾದ
ಗುರು-ಮಹಾ-ರಾಜರು
ಗುರು-ಮಾ-ತಿಗೆ
ಗುರು-ಮುಖ-ದಿಂದ
ಗುರು-ವನ್ನು
ಗುರು-ವರ-ಪದಂ
ಗುರು-ವಾಗಿ
ಗುರು-ವಾಗು-ವುದಕ್ಕೆ
ಗುರು-ವಿ-ಗಾಗಿ
ಗುರು-ವಿಗೆ
ಗುರು-ವಿನ
ಗುರು-ವಿ-ನಂತೆ
ಗುರು-ವಿ-ನಲ್ಲಿ
ಗುರು-ವಿ-ನೊ-ಡನೆ
ಗುರು-ವಿಲ್ಲ
ಗುರುವೆ
ಗುರು-ವೆಂದು
ಗುರುವೇ
ಗುರು-ಶಿಷ್ಯ-ರಾ-ರಿಲ್ಲ
ಗುರುಸ್ಥಾನ-ದಲ್ಲಿ-ರು-ವೆವು
ಗುರೂಪ-ದೇಶ-ದಿಂದಲೂ
ಗುಲಾಬಿಯ
ಗುಲಾಮ-ಗಿರಿ
ಗುಲಾಮ-ಗಿರಿಯ
ಗುಲಾಮ-ಗಿರಿ-ಯನ್ನು
ಗುಲಾಮ-ಗಿರಿ-ಯಲ್ಲಿ
ಗುಲಾಮ-ಗಿರಿ-ಯಿಂದಲ್ಲ
ಗುಲಾಮ-ನನ್ನಾಗಿ
ಗುಲಾಮ-ನಾಗಿ-ರುತ್ತದೆ
ಗುಲಾಮ-ರನ್ನೂ
ಗುಲಿಜರಿಛೆ
ಗುಲಿಸ್ವನಸ್ವನ
ಗುಳ್ಳೆ-ಗಳೆಲ್ಲವು
ಗುಹಾ
ಗುಹೆ-ಗಳಲಿ
ಗುಹೆ-ಯಲ್ಲಿ
ಗುಹೆ-ಯಲ್ಲಿ-ರುವವೊ
ಗುಹೆ-ಯಿಂದ
ಗುಹೆ-ಯೊ-ಳಕ್ಕೆ
ಗೂಡಿಗೆ
ಗೂಡು
ಗೂಢ
ಗೂಢ-ತತ್ತ್ವ-ಗಳನ್ನು
ಗೂಢಾರ್ಥ
ಗೂಢಾರ್ಥ-ವನ್ನು
ಗೃಹ
ಗೃಹಂ
ಗೃಹ-ಕಾರ್ಯ
ಗೃಹ-ಕೃತ್ಯ
ಗೃಹ-ಕೃತ್ಯ-ಗಳಲ್ಲಿ
ಗೃಹ-ಕೃತ್ಯ-ದಲ್ಲಿ
ಗೃಹ-ಕೃತ್ಯ-ನಿಯಮ-ಗಳು
ಗೃಹಕ್ಕೆ
ಗೃಹ-ದಲ್ಲಿ
ಗೃಹ-ಲಕ್ಷ್ಮಿ-ಯ-ರನ್ನು
ಗೃಹಸ್ಥ
ಗೃಹಸ್ಥನ
ಗೃಹಸ್ಥ-ನನ್ನು
ಗೃಹಸ್ಥ-ನಾ-ಗಲಿ
ಗೃಹಸ್ಥ-ನಾಗಿದ್ದ
ಗೃಹಸ್ಥ-ನಾಗಿರು
ಗೃಹಸ್ಥ-ನಾದ
ಗೃಹಸ್ಥ-ನಿಗೂ
ಗೃಹಸ್ಥ-ನಿಗೆ
ಗೃಹಸ್ಥನು
ಗೃಹಸ್ಥ-ಭಕ್ತರು
ಗೃಹಸ್ಥ-ಭಕ್ತರೇ
ಗೃಹಸ್ಥರ
ಗೃಹಸ್ಥ-ರಂತೆಯೂ
ಗೃಹಸ್ಥ-ರಲ್ಲಿ
ಗೃಹಸ್ಥ-ರಾಗ-ಬೇಕೆಂದು
ಗೃಹಸ್ಥ-ರಾಗಿಯೆ
ಗೃಹಸ್ಥ-ರಾದ-ಮೇಲೂ
ಗೃಹಸ್ಥ-ರಾದರೊ
ಗೃಹಸ್ಥ-ರಿಗೂ
ಗೃಹಸ್ಥ-ರಿಗೆ
ಗೃಹಸ್ಥರು
ಗೃಹಸ್ಥರೇ
ಗೃಹಸ್ಥಾಶ್ರಮದ
ಗೃಹಿ-ಗಳ
ಗೃಹಿ-ಣಿಯ-ರಾಗುವರೊ
ಗೃಹಿ-ಣಿಯರೂ
ಗೃಹಿ-ಣಿಯೊಬ್ಬಳಿಂದ
ಗೃಹೀ-ಶಿಷ್ಯ-ನಾದ
ಗೃಹೇಷು
ಗೃಹ್ಯ
ಗೃಹ್ಯತೇ
ಗೃಹ್ಯ-ಸೂತ್ರ-ದಲ್ಲಿಯೂ
ಗೆ
ಗೆಜ್ಜೆ
ಗೆದ್ದ
ಗೆದ್ದು-ದ-ರಿಂದ
ಗೆರೆ
ಗೆಲ್ಲಲು
ಗೆಲ್ಲು-ವನು
ಗೆಳೆತ-ನವು
ಗೆಳೆಯ
ಗೆಳೆಯ-ನಿಗೆ
ಗೆಳೆ-ಯನೆ
ಗೆಳೆಯನೇ
ಗೆಳೆಯ-ರೆಂಬುವ-ರರಿ-ಯರು
ಗೇಲಿ
ಗೈಯುತ
ಗೈರಿಕ-ವಸನ-ಧಾರಿ
ಗೈರಿಕ-ವಸನ-ಧಾರಿ-ಯಾದ
ಗೈರುಹಾಜರಿ-ಯನ್ನು
ಗೊಂಚ-ಲಿನ
ಗೊಂದಲ-ದಲ್ಲಿ
ಗೊಂದಲ-ವನ್ನೆಬ್ಬಿ-ಸುತ್ತಿದ್ದರು
ಗೊಂಬೆ-ಗಳಂತೆ
ಗೊಡ-ವೆಯೂ
ಗೊತಿಲ
ಗೊತ್ತಾಗ-ದಿ-ರು-ವುದು
ಗೊತ್ತಾಗ-ಲಿಲ್ಲ
ಗೊತ್ತಾ-ಗಲಿಲ್ಲವೆ
ಗೊತ್ತಾಗಿತ್ತು
ಗೊತ್ತಾಗಿ-ದೆಯೋ
ಗೊತ್ತಾಗಿದ್ದಿದ್ದರೆ
ಗೊತ್ತಾ-ಗಿ-ರುವ
ಗೊತ್ತಾ-ಗುತ್ತದೆ
ಗೊತ್ತಾಗುತ್ತವೆ
ಗೊತ್ತಾಗುತ್ತಿತ್ತು
ಗೊತ್ತಾ-ಗು-ವಂತೆ
ಗೊತ್ತಾ-ಗು-ವು-ದಿಲ್ಲ
ಗೊತ್ತಾ-ಗು-ವು-ದಿಲ್ಲವೆ
ಗೊತ್ತಾಗು-ವುದು
ಗೊತ್ತಾ-ದರೆ
ಗೊತ್ತಾ-ದೊಡ-ನೆಯೇ
ಗೊತ್ತಾಯಿತು
ಗೊತ್ತಾಯಿತೆ
ಗೊತ್ತಾಯಿತೇ
ಗೊತ್ತಾಯಿತೋ
ಗೊತ್ತಿತ್ತು
ಗೊತ್ತಿತ್ತೆಂದು
ಗೊತ್ತಿದೆ
ಗೊತ್ತಿದೆಯೆ
ಗೊತ್ತಿದೆಯೊ
ಗೊತ್ತಿದೆಯೋ
ಗೊತ್ತಿದ್ದುದು
ಗೊತ್ತಿದ್ದುವು
ಗೊತ್ತಿರ-ಬ-ಹುದು
ಗೊತ್ತಿರ-ಬೇಕು
ಗೊತ್ತಿ-ರ-ಲಿಲ್ಲ
ಗೊತ್ತಿರುವ
ಗೊತ್ತಿರು-ವಂತೆ
ಗೊತ್ತಿ-ರು-ವು-ದಿಲ್ಲ
ಗೊತ್ತಿ-ರು-ವುದು
ಗೊತ್ತಿಲ್ಲ
ಗೊತ್ತಿಲ್ಲದ
ಗೊತ್ತಿಲ್ಲ-ದಿದ್ದ-ರೇ-ನಂತೆ
ಗೊತ್ತಿಲ್ಲ-ದಿ-ರುವಾಗ
ಗೊತ್ತಿಲ್ಲದೆ
ಗೊತ್ತಿಲ್ಲವೆ
ಗೊತ್ತಿಲ್ಲ-ವೆಂದು
ಗೊತ್ತಿಲ್ಲವೋ
ಗೊತ್ತಿವೆ
ಗೊತ್ತು
ಗೊತ್ತು-ಮಾಡ-ಲಾರ-ದವ-ನಾಗಿದ್ದೇನೆ
ಗೊತ್ತು-ಮಾ-ಡಲು
ಗೊತ್ತು-ಮಾಡಿ
ಗೊತ್ತು-ಮಾಡಿ-ಕೊಂಡು
ಗೊತ್ತು-ಹಿಡಿ-ಯ-ಲಾರದೆ
ಗೊತ್ತೆ
ಗೊತ್ತೇ
ಗೊತ್ತೋ
ಗೊಬ್ಬ-ರದ
ಗೊಬ್ಬು
ಗೊಳಿಸ-ಬ-ಹುದು
ಗೋಕುಲದ
ಗೋಚರಂ
ಗೋಚರ-ವಲ್ಲ
ಗೋಚರ-ವಾ-ಗಿ-ರಲಿಲ್ಲ
ಗೋಚರ-ವಾಗುತ್ತದೆ
ಗೋಚರ-ವಾ-ಗು-ವು-ದಿಲ್ಲ
ಗೋಚರ-ವಾಗು-ವುವು
ಗೋಚರಿ-ಸುತ್ತದೆ
ಗೋಚರಿ-ಸುವನು
ಗೋಜು
ಗೋಡೆ-ಗಳ
ಗೋಡೆ-ಗಳನ್ನು
ಗೋಡೆಗೆ
ಗೋಡೆಯ
ಗೋಡೆ-ಯನ್ನು
ಗೋಡೆ-ಯಲ್ಲಿ
ಗೋಢಾಹಾತಿ
ಗೋಣನೊಡ್ಡ-ದಿರು
ಗೋಧಿ
ಗೋಪಾಲ-ಲಾಲ
ಗೋಪಿ
ಗೋಪಿ-ಕೆ-ಯೊಬ್ಬ-ಳನ್ನು
ಗೋಪಿ-ಗಳೊ-ಡ-ನಿದ್ದ
ಗೋಪಿ-ಯರ
ಗೋಪಿ-ಯ-ರಿಗೆ
ಗೋಪೀ-ಲೀಲೆ
ಗೋಪುರ
ಗೋಪುರಾ-ಕಾರದ
ಗೋಪುರಾ-ಕೃ-ತಿಯ
ಗೋಪ್ಯ-ವಾಗಿ
ಗೋಪ್ಯ-ವಾದ
ಗೋಭಿಲ
ಗೋಮಾಂಸ-ವನ್ನು
ಗೋಮಾತೆ-ಗಳು
ಗೋಮಾತೆಯ
ಗೋಮಾತೆ-ಯನ್ನು
ಗೋಮಾತೆ-ಯ-ರನ್ನು
ಗೋರಕ್ಷಣಾ
ಗೋರಕ್ಷಾ
ಗೋಲಾ-ಬರಿ-ಶಣ
ಗೋಳಾಟ-ವನ್ನು
ಗೋಳಾಡುತ್ತ
ಗೋಳಾಡುತ್ತೀರಿ
ಗೋಳಿಡುವುದದೆ
ಗೋಳಿನ
ಗೋಳಿ-ನಿಂದ
ಗೋಳು
ಗೋವನ್ನು
ಗೋವನ್ನೋ
ಗೋವ-ಳರ
ಗೋವಿಂದ
ಗೋವಿಂದ-ಕುಮಾರ
ಗೋವಿಂದರ
ಗೋವಿಂದರು
ಗೋವು
ಗೋಶ್ಪದ
ಗೋಷ್ಟದದ
ಗೋಷ್ಪದ
ಗೋಸುಂಬೆ-ಯಂತೆ
ಗೋಹತ್ಯಾ-ಪಾ-ಪದ
ಗೋಹತ್ಯಾ-ಪಾಪ-ವನ್ನು
ಗೌಣ
ಗೌರವ
ಗೌರ-ವಕ್ಕೆ
ಗೌರವದ
ಗೌರವ-ದಿಂದ
ಗೌರವ-ಯುತ
ಗೌರ-ವರ್ಣ
ಗೌರವ-ವನ್ನು
ಗೌರವ-ವನ್ನೂ
ಗೌರವ-ವಿಡ-ಬೇಕೆಂಬು-ದರ
ಗೌರವ-ವಿಲ್ಲವೊ
ಗೌರವವೂ
ಗೌರವಾಧ್ಯಕ್ಷ-ರಾದರು
ಗೌರವಿಸ-ಬೇಕು
ಗೌರವಿ-ಸಲಿ
ಗೌರವಿ-ಸಲು
ಗೌರವಿ-ಸಲ್ಪ-ಡು-ವರು
ಗೌರವಿ-ಸಿದ-ರೇನು
ಗೌರವಿ-ಸುತ್ತಿದ್ದರು
ಗೌರವಿ-ಸುವರು
ಗೌರವಿ-ಸು-ವು-ದಿಲ್ಲ
ಗೌರವಿ-ಸು-ವು-ದಿಲ್ಲ-ವೆಂದು
ಗೌರವಿ-ಸು-ವುದು
ಗೌರ್ನರ್
ಗ್ರಂಥ
ಗ್ರಂಥ-ಕರ್ತ-ನನ್ನು
ಗ್ರಂಥ-ಕರ್ತರು
ಗ್ರಂಥಕ್ಕೆ
ಗ್ರಂಥ-ಗಳ
ಗ್ರಂಥ-ಗಳನ್ನ
ಗ್ರಂಥ-ಗಳನ್ನು
ಗ್ರಂಥ-ಗಳನ್ನೇ
ಗ್ರಂಥ-ಗಳಲ್ಲಿ
ಗ್ರಂಥ-ಗಳಲ್ಲಿ-ದೆಯೇ
ಗ್ರಂಥ-ಗಳಲ್ಲಿಯೂ
ಗ್ರಂಥ-ಗಳಲ್ಲಿ-ರುವ
ಗ್ರಂಥ-ಗಳಲ್ಲಿವೆ
ಗ್ರಂಥ-ಗಳಲ್ಲೆಲ್ಲಾ
ಗ್ರಂಥ-ಗ-ಳಾದ
ಗ್ರಂಥ-ಗಳಿಗೆ
ಗ್ರಂಥ-ಗಳಿವೆ
ಗ್ರಂಥ-ಗಳು
ಗ್ರಂಥ-ಗಳೂ
ಗ್ರಂಥ-ಗಳೆಲ್ಲ
ಗ್ರಂಥ-ಗಳೆಲ್ಲ-ವನ್ನೂ
ಗ್ರಂಥ-ತಾನೇ
ಗ್ರಂಥ-ದಂತಿ-ರ-ಲಿಲ್ಲ
ಗ್ರಂಥ-ದಲ್ಲಿ
ಗ್ರಂಥ-ವನ್ನು
ಗ್ರಂಥ-ವನ್ನೂ
ಗ್ರಂಥ-ವಾದ
ಗ್ರಂಥವು
ಗ್ರಂಥ-ವೊಂದನ್ನು
ಗ್ರಂಥ-ವೊಂದನ್ನೂ
ಗ್ರಂಥಿ
ಗ್ರಂಥಿಶ್ಛಿದ್ಯಂತೇ
ಗ್ರಹ
ಗ್ರಹ-ಚಾರ
ಗ್ರಹ-ಚಾ-ರಕ್ಕೆ
ಗ್ರಹ-ಚಾರ-ವನ್ನು
ಗ್ರಹಣ
ಗ್ರಹ-ಣದ
ಗ್ರಹ-ಣ-ಮಾ-ಡಲು
ಗ್ರಹ-ಣ-ವನ್ನು
ಗ್ರಹ-ಣ-ವಾಗುವ
ಗ್ರಹ-ಣ-ವಾಗು-ವುದಕ್ಕೆ
ಗ್ರಹ-ಣ-ಶಕ್ತಿ
ಗ್ರಹ-ತಾರಾ
ಗ್ರಹ-ತಾರೆ
ಗ್ರಹಿ-ಸ-ದಿದ್ದರೆ
ಗ್ರಹಿಸ-ಬಲ್ಲ
ಗ್ರಹಿಸ-ಬ-ಹುದು
ಗ್ರಹಿಸ-ಬೇಕು
ಗ್ರಹಿ-ಸ-ಲಾ-ಗು-ವು-ದಿಲ್ಲ
ಗ್ರಹಿ-ಸ-ಲಾರದೆ
ಗ್ರಹಿಸ-ಲಾರದೇ
ಗ್ರಹಿ-ಸ-ಲಾ-ರರು
ಗ್ರಹಿ-ಸಲು
ಗ್ರಹಿಸಿ-ಕೊಳ್ಳ-ಬೇಕೆಂದು
ಗ್ರಹಿ-ಸಿದ್ದರು
ಗ್ರಹಿ-ಸಿದ್ದಾರೆ
ಗ್ರಹಿ-ಸಿಯೇ
ಗ್ರಹಿ-ಸಿಲ್ಲ
ಗ್ರಹಿ-ಸುತ್ತಿದ್ದರೋ
ಗ್ರಹಿ-ಸುವರು
ಗ್ರಹಿ-ಸುವುದ-ರಲ್ಲಿ
ಗ್ರಾಮ
ಗ್ರಾಮ-ಗಳಲ್ಲಿಯೂ
ಗ್ರಾಮ-ಗ-ಳಲ್ಲೂ
ಗ್ರಾಮ-ಗಳಿಗೂ
ಗ್ರಾಮ-ಗಳಿಗೆ
ಗ್ರಾಮಗ್ರಾಮ-ಗಳಿಗೂ
ಗ್ರಾಮದ
ಗ್ರಾಮ-ದಲ್ಲಿ
ಗ್ರಾಮ್ಯ
ಗ್ರಾಸ-ವಾಗಿ
ಗ್ರಾಸಿ
ಗ್ರೀಕರ
ಗ್ರೀಕರು
ಗ್ರೀಸ್-ನೊಂದಿಗೆ
ಗ್ಲಾನಿ-ಕರ-ವಾದ
ಗ್ಲಾನಿಯನ್ನೆಲ್ಲಾ
ಗ್ಲಾಸಿನ
ಗ್ಲಾಸಿ-ನಂತೆ
ಘಂಟೆ
ಘಂಟೆಗೆ
ಘಂಟೆಯ
ಘಂಟೆ-ಯ-ವ-ರೆಗೂ
ಘಂಟೆ-ಯಾದ
ಘಂಟೆ-ಯಿಂದ
ಘಂಟೆ-ಯಿತ್ತು
ಘಂಟೆ-ಯಿರ-ಬ-ಹುದು
ಘಟ
ಘಟ-ಕ-ವನ್ನೇ
ಘಟ-ಕವೇ
ಘಟತ್ವ
ಘಟದ
ಘಟನ
ಘಟನೆ
ಘಟನೆ-ಗಳ
ಘಟನೆ-ಗಳಿಂದಲೂ
ಘಟನೆ-ಗಳು
ಘಟನೆ-ಗಿಂತ
ಘಟನೆ-ಯನ್ನು
ಘಟ-ಸರ್ಪ-ವನ್ನು
ಘಟಸ್ಮೃತಿ
ಘಟಾಕಾಶ
ಘಟೆ
ಘಟ್ಟ-ದಲ್ಲಿ
ಘಟ್ಟಿಸೆ
ಘನ
ಘನ-ಗಂಭೀರ-ವಾದ
ಘನ-ತಮ-ಹಾರಿ
ಘನತೆ
ಘನ-ತೆಗೆ
ಘನ-ತೆ-ಯನ್ನು
ಘನ-ಮೂರ್ತಿ
ಘನ-ವಾದ
ಘನ-ವಿದ್ವಾಂಸರು
ಘನೀ-ಭೂತ-ವಾ-ಗಿ-ರು-ವುದೇ
ಘರ
ಘರೆ
ಘರ್ಷಣೆ-ಯಿಂದಲೇ
ಘಳಿಗೆ
ಘಾಜಿ-ಪುರ
ಘಾಜಿ-ಪುರದ
ಘಾಟಿಗೆ
ಘಾಟಿ-ನಲ್ಲಿ
ಘಾಟಿ-ನಿಂದ
ಘಾಟಿಯೂ
ಘಾತ
ಘಾತದಿ
ಘಾಸಿಗೊಳ್ಳು-ವು-ದಿಲ್ಲ
ಘೂಚಾಯೆ
ಘೋರ
ಘೋರ-ತಮ
ಘೋರ-ವಾದ
ಘೋರಾಂಧಕಾ-ರ-ದಲ್ಲಿ
ಘೋಶ
ಘೋಶೆತೋಪ
ಘೋಷ
ಘೋಷರ
ಘೋಷ-ರಿಗೂ
ಘೋಷರು
ಘೋಷರೆ
ಘೋಷಿಸಿದ
ಘೋಷಿಸಿ-ದರು
ಘೋಷಿ-ಸುತ್ತಲೇ
ಘೋಷೈ
ಘೋಷ್
ಘೋಷ್ರೂ
ಚ
ಚಂಚಲ
ಚಂಚಲ-ತೆ-ಯೆಲ್ಲಾ
ಚಂಚಲ-ನಾಗಿದ್ದರೂ
ಚಂಚಲ-ವಾಗಿಯೇ
ಚಂಚಲ-ವಾಗುತ್ತಾರೋ
ಚಂಡ
ಚಂಡ-ಕಾಂತಿಯ
ಚಂಡನ
ಚಂಡ-ಮಾರುತ
ಚಂಡ-ಮಾರುತ-ಗಳಲಿ
ಚಂಡಾಲ
ಚಂಡಾಲನ
ಚಂಡಾಲ-ನನ್ನೂ
ಚಂಡಾಲ-ರಿಗೂ
ಚಂಡಾಲರು
ಚಂದಾ
ಚಂದಿರನ
ಚಂದ್ರ
ಚಂದ್ರ-ಚಲ
ಚಂದ್ರ-ತಾರಗೆ
ಚಂದ್ರನ
ಚಂದ್ರ-ಮ-ನಿಲ್ಲ
ಚಂದ್ರಮಾ
ಚಂದ್ರ-ರೆದ್ದಿಹ-ರಲ್ಲಿ
ಚಂದ್ರ-ಲೋಕ
ಚಂದ್ರ-ಸೂರ್ಯ
ಚಂದ್ರ-ಸೇ-ನನು
ಚಕಿತ-ರಾದರು
ಚಕ್ರಕ್ಕೆ
ಚಕ್ರ-ಗಳಿ-ರುತ್ತವೆ
ಚಕ್ರ-ಗಳು
ಚಕ್ರದಿಂದ
ಚಕ್ರವ
ಚಕ್ರವರ್ತಿ
ಚಕ್ರವರ್ತಿ-ಗಳ
ಚಕ್ರವರ್ತಿ-ಗಳಲ್ಲೆಲ್ಲಾ
ಚಕ್ರವರ್ತಿ-ಗಳು
ಚಕ್ರವರ್ತಿ-ಶಿರೋ-ಮಣಿ-ಗಳಾಗಿದ್ದರು
ಚಕ್ರಾ-ಕಾರ-ವಾಗಿ
ಚಕ್ರಾಧಿಪತ್ಯದ
ಚಕ್ಷು
ಚಕ್ಷು-ಗಳಿಂದ
ಚಚ್ಚಿ-ಕೊಳ್ಳ-ಬೇ-ಕಾದ್ದಿಲ್ಲ
ಚಚ್ಚಿ-ಕೊಳ್ಳುತ್ತಾ
ಚಟ-ವನ್ನು
ಚಟುವಟಿಕೆ
ಚಟುವಟಿಕೆ-ಯಿಂದ
ಚಟುವಟಿಕೆ-ಯಿಂದಿದ್ದಾರೆ
ಚಟುವಟಿಕೆ-ಯುಳ್ಳ-ವ-ನಾಗಿ
ಚಟುವಟಿಕೆ-ಸಂಭ್ರಮ-ಗಳಿಂದ
ಚಡಾಉ
ಚಣ
ಚಣ-ಕಾಲ-ವಾದರೂ
ಚತುರ-ತೆಯು
ಚತುರ-ರಾಗಲಿ
ಚತು-ರರೂ
ಚತುರ್ಥಸ್ತ-ರದಲ್ಲಿ-ರುವ
ಚತುರ್ಭುಜ
ಚತುಷ್ಟಯ
ಚದು-ರಿತು
ಚನ್ನಾಗಿಲ್ಲ-ದಿದ್ದರೆ
ಚಪಲತೆ
ಚಪಾತಿ-ಗಳನ್ನು
ಚಪ್ಪಾಳೆ
ಚಮತ್ಕಾರ-ವಾದ
ಚಮತ್ಕಾ-ರವೇ
ಚರಂಡಿಯ
ಚರಂತಃ
ಚರ-ಣ-ಗಳನ್ನು
ಚರ-ಣ-ಗಳು
ಚರ-ಣ-ಯುಗಲ
ಚರಮ
ಚರ-ಮ-ಗುರಿ
ಚರ-ಮ-ಲಕ್ಷಣ-ವೆಂಬುದು
ಚರಾ-ಚರ
ಚರಿತ್ರೆ
ಚರಿತ್ರೆ-ಗಳ
ಚರಿತ್ರೆ-ಗಳನ್ನು
ಚರಿತ್ರೆ-ಯನ್ನು
ಚರಿತ್ರೆ-ಯಲ್ಲಿ
ಚರ್ಚಾಗೋಷ್ಠಿ-ಯನ್ನು
ಚರ್ಚಿಗೆ
ಚರ್ಚಿನ
ಚರ್ಚಿ-ನಿಂದೀಚೆಗೆ
ಚರ್ಚಿಸ-ಬೇಕೆಂದಿರುವ
ಚರ್ಚಿ-ಸಲು
ಚರ್ಚಿ-ಸಲ್ಪಡುತ್ತಿದ್ದವು
ಚರ್ಚಿ-ಸಿ-ದರು
ಚರ್ಚಿ-ಸುತ್ತಾ
ಚರ್ಚಿ-ಸುತ್ತಿದ್ದಂತಿತ್ತು
ಚರ್ಚಿ-ಸುತ್ತಿದ್ದಾರೆ
ಚರ್ಚಿ-ಸುತ್ತಿದ್ದು-ದ-ರಿಂದ
ಚರ್ಚಿ-ಸುತ್ತಿದ್ದೆ
ಚರ್ಚಿ-ಸುತ್ತಿದ್ದೇನೆ
ಚರ್ಚಿ-ಸುವರು
ಚರ್ಚಿ-ಸು-ವು-ದಾಗಿದೆ
ಚರ್ಚಿ-ಸು-ವು-ದಿಲ್ಲ
ಚರ್ಚು-ಗಳಲಿ
ಚರ್ಚೆ
ಚರ್ಚೆಗೆ
ಚರ್ಚೆ-ಯಲ್ಲಿ
ಚರ್ಚೆ-ಯಿಂದಲೂ
ಚರ್ಚ್
ಚರ್ಯ
ಚಲ-ನವ-ಲನ-ಗಳ
ಚಲ-ನೆ-ಯೆಲ್ಲ
ಚಲಾಯಿ-ಸುವ
ಚಲಾವಣೆ
ಚಲಿ-ಪನಾರು
ಚಲಿ-ಸದೆ
ಚಲಿ-ಸುತ
ಚಲಿ-ಸುತ್ತಿ-ರು-ವುದು
ಚಲಿ-ಸುವ
ಚಲಿ-ಸುವನು
ಚಲೆ
ಚಲೇ
ಚಳಿ
ಚಳಿ-ಗಾಲ-ದಲ್ಲೇ
ಚಳಿ-ಯಂಗಿ-ಯನೆ
ಚಳಿ-ಯಿಂದ
ಚಳುವಳಿ-ಗಳೆಲ್ಲಾ
ಚಳುವಳಿ-ಯನ್ನು
ಚಳುವಳಿಯೇ
ಚಾಂತ-ರಿಕ್ಷಮಥೋ
ಚಾಕರಿ
ಚಾಗವು
ಚಾಗಿಗೆ
ಚಾಗಿಯ
ಚಾಚಿ
ಚಾಚಿದ
ಚಾಚು-ತಿದ್ದರು
ಚಾಚೂ
ಚಾತಕ-ಪಕ್ಷಿ
ಚಾತುರ್ವರ್ಣ-ಗಳು
ಚಾನೊ
ಚಾಪಲ್ಯ-ಗಳನ್ನೆಲ್ಲ
ಚಾಪಲ್ಯ-ಗಳಲ್ಲೆಲ್ಲಾ
ಚಾಪಲ್ಯ-ವನ್ನು
ಚಾಪಲ್ಯ-ವಿ-ರುವ
ಚಾಪಲ್ಯವೂ
ಚಾಯ
ಚಾಯಿ
ಚಾಯು
ಚಾಯ್
ಚಾರಿತ್ರಬಲ
ಚಾರಿತ್ರ್ಯ
ಚಾರಿರ್ಯ-ಗಳೇ
ಚಾರಿರ್ಯದ
ಚಾರಿರ್ಯ-ವನ್ನು
ಚಾರಿರ್ಯ-ವಾಗಿ-ರುತ್ತದೆ
ಚಾರಿರ್ಯ-ವೊಂದನ್ನೇ
ಚಾರು
ಚಾಲ-ನೆಯಲ್ಲಿತ್ತು
ಚಾಲಾಯ್
ಚಾಲ್ತಿ-ಯಲ್ಲಿ-ರುವ
ಚಾಳಿ-ಗಳನ್ನು
ಚಾವಟಿ
ಚಾವೊ
ಚಾಹಾ
ಚಾಹೆ
ಚಿಂತ-ಕರೂ
ಚಿಂತ-ನದಿ
ಚಿಂತನ-ಶೀಲ-ರನ್ನು
ಚಿಂತನ-ಶೀಲ-ರಾಗಿದ್ದರು
ಚಿಂತನೆ
ಚಿಂತನೆ-ಗಳಲ್ಲಿ
ಚಿಂತ-ನೆಗು
ಚಿಂತ-ನೆಯ
ಚಿಂತನೆ-ಯಲ್ಲಿ
ಚಿಂತನೆ-ಯಿಂದಲೇ
ಚಿಂತನೆ-ಯೆಳೆ-ಗಳ
ಚಿಂತನೆಸ್ವರಕೆ
ಚಿಂತಾ
ಚಿಂತಾಪ್ರಣಾಲಿ-ಯನ್ನಾಗಲೀ
ಚಿಂತಾ-ಭಾರ
ಚಿಂತಿ-ಸುತ್ತಿದ್ದನು
ಚಿಂತಿ-ಸುತ್ತಿದ್ದರೇನೇ
ಚಿಂತಿ-ಸುತ್ತಿದ್ದಳು
ಚಿಂತಿ-ಸುತ್ತಿ-ರ-ಬೇಕಾಗಿಲ್ಲ
ಚಿಂತೆ
ಚಿಂತೆ-ಯನ್ನು
ಚಿಂತೆ-ಯನ್ನೇ
ಚಿಂತೆ-ಯಿಂದಲೂ
ಚಿಂತೆ-ಯಿಲ್ಲ
ಚಿಂದಿ
ಚಿಂದಿ-ಗಳ
ಚಿಂದಿ-ಯನುಟ್ಟು
ಚಿಕಾಗೊ
ಚಿಕಾಗೊ-ವಿನ
ಚಿಕಾಗೋ
ಚಿಕಿತ್ಸಾಕ್ರಮ
ಚಿಕಿತ್ಸೆ-ಗಾಗಿ
ಚಿಕಿತ್ಸೆಗೆ
ಚಿಕಿತ್ಸೆಯ
ಚಿಕಿತ್ಸೆ-ಯನ್ನು
ಚಿಕಿತ್ಸೆ-ಯಲ್ಲಿ
ಚಿಕಿತ್ಸೆ-ಯಿಂದ
ಚಿಕ್ಕ
ಚಿಕ್ಕ-ದಾದ
ಚಿಕ್ಕ-ಪುಟ್ಟ
ಚಿಕ್ಕಪ್ಪ
ಚಿಕ್ಕ-ಮ-ನೆಗೆ
ಚಿಕ್ಕ-ವನು
ಚಿಗಿತು
ಚಿಗುರಿ
ಚಿಗುರು-ವುದು
ಚಿಚ್ಛಕ್ತಿಯು
ಚಿಚ್ಛಾಯಾ-ವೇಶತಃ
ಚಿಟ್ಟು
ಚಿಟ್ಟೆ
ಚಿತಾಗ್ನಿ
ಚಿತಾಮಾಝೆ
ಚಿತ್
ಚಿತ್ತ
ಚಿತ್ತಂ
ಚಿತ್ತಕ್ಷೋಭ-ದಿಂದಲೂ
ಚಿತ್ತ-ದಾಕಾಶ-ದಲಿ
ಚಿತ್ತನು
ಚಿತ್ತ-ಫಲ-ಕದ
ಚಿತ್ತಭ್ರಮಣೆ
ಚಿತ್ತ-ರಾಗಿ
ಚಿತ್ತರು
ಚಿತ್ತ-ವನಿಡೆ
ಚಿತ್ತ-ವಿಹಂಗಂ
ಚಿತ್ತವು
ಚಿತ್ತ-ವೃತ್ತಿ-ಯನ್ನು
ಚಿತ್ತ-ವೃತ್ತೇರ್ನಿರೋಧಮ್
ಚಿತ್ತ-ಶುದ್ದಿ-ಯನ್ನೂ
ಚಿತ್ತ-ಶುದ್ಧಿ
ಚಿತ್ತ-ಶುದ್ಧಿ-ಯನ್ನು
ಚಿತ್ತ-ಶುದ್ಧಿ-ಯುಂಟಾ-ದಾಗ
ಚಿತ್ತ-ಶುದ್ಧಿಯೂ
ಚಿತ್ತ-ಸಂಯಮ-ದಿಂದಲೇ
ಚಿತ್ತೈಕಾಗ್ರತೆ
ಚಿತ್ತೈಕಾಗ್ರತೆ-ಗಳಿಂದ
ಚಿತ್ತೈಕಾಗ್ರತೆ-ಯಿಂದ
ಚಿತ್ತೈಕಾಗ್ರತೆ-ಯಿಂದಲೋ
ಚಿತ್ಪುರದ
ಚಿತ್ಪ್ರಕಾಶ-ವುಂಟಾಗಿ-ಬಿಡುತ್ತದೆ
ಚಿತ್ರ
ಚಿತ್ರ-ಕರ
ಚಿತ್ರ-ಕಲೆ-ಯನ್ನು
ಚಿತ್ರ-ಕಲೆ-ಯಲ್ಲಿ
ಚಿತ್ರ-ಗಳ
ಚಿತ್ರ-ಗಳಂತೆ
ಚಿತ್ರ-ಗಳಂತೆಯೇ
ಚಿತ್ರ-ಗಳನ್ನು
ಚಿತ್ರ-ಗಳನ್ನೋ
ಚಿತ್ರ-ಗಳೆಲ್ಲ
ಚಿತ್ರಣ
ಚಿತ್ರ-ಣ-ವನ್ನೇ
ಚಿತ್ರದ
ಚಿತ್ರ-ದಂತೆ
ಚಿತ್ರ-ದಲ್ಲಿ
ಚಿತ್ರ-ದಲ್ಲಿ-ರುವ-ವನಾರು
ಚಿತ್ರ-ದಿಂದೇನು
ಚಿತ್ರ-ಪಟದ
ಚಿತ್ರ-ಪಟ-ವನ್ನು
ಚಿತ್ರ-ವನ್ನು
ಚಿತ್ರ-ವಿ-ಚಿತ್ರ
ಚಿತ್ರ-ವಿ-ಚಿತ್ರದ
ಚಿತ್ರ-ವಿ-ಚಿತ್ರ-ವಾದ
ಚಿತ್ರ-ವಿತ್ತು
ಚಿತ್ರ-ವೀಗ
ಚಿತ್ರ-ಹಂಕಾರ
ಚಿತ್ರ-ಹಂಕಾರ-ಗಳು
ಚಿತ್ರಾರ್ಪಿ-ತಾ-ರಂಭ
ಚಿತ್ರಿತ-ವಾಗ-ಬೇಕೆಂದು
ಚಿತ್ರಿತ-ವಾಗಿತ್ತು
ಚಿತ್ರಿಸ-ಬೇಕು
ಚಿತ್ರಿಸ-ಲಾಗಿದೆ
ಚಿತ್ರಿ-ಸ-ಲಾ-ಗು-ವು-ದಿಲ್ಲ
ಚಿತ್ರಿ-ಸಲಿ
ಚಿತ್ರಿ-ಸಲು
ಚಿತ್ರಿಸಿ
ಚಿತ್ರಿಸಿದ
ಚಿತ್ರಿಸಿ-ದರೆ
ಚಿತ್ರಿಸಿದೆ
ಚಿತ್ರಿ-ಸುವರು
ಚಿತ್ಸೂರ್ಯನು
ಚಿತ್ಸ್ವ-ರೂಪ-ವಾದ
ಚಿದ್ಘನ-ಕಾಯ
ಚಿದ್ರವಿಯು
ಚಿನುಮಃ
ಚಿನ್ನ
ಚಿನ್ನದ
ಚಿನ್ನ-ವನ್ನೇ
ಚಿನ್ಮಯ
ಚಿನ್ಮಾತ್ರ
ಚಿಮ್ಮಿ
ಚಿಮ್ಮಿ-ಚೆಲ್ಲಿದೆ
ಚಿಮ್ಮಿದ
ಚಿಮ್ಮಿದೆ
ಚಿಮ್ಮು-ತಿದೆ
ಚಿಮ್ಮುತಿ-ರಲಿ
ಚಿಮ್ಮುತಿವೆ
ಚಿಮ್ಮು-ತಿಹ
ಚಿಮ್ಮು-ವಂತೆ
ಚಿರ
ಚಿರಂತನ-ವಾಗಿ
ಚಿರ-ಕಾಲ
ಚಿರದ
ಚಿರ-ಪುತ್ರ
ಚಿರ-ಮ-ರಣ
ಚಿರ-ಮಾಧುರ್ಯ
ಚಿರ-ಮುದ್ರಿತ-ವಾಗು-ವುದು
ಚಿರ-ಮುದ್ರೆ-ಯನ್ನೊತ್ತಿ
ಚಿರಸ್ಥಾಯಿ-ಯಾಗಿ
ಚಿರಸ್ಥಾಯಿ-ಯಾಗಿ-ರು-ವು-ದರ
ಚಿರಸ್ಮ-ರಣೀಯ
ಚಿರಸ್ಮ-ರಣೀಯ-ವಾದ
ಚಿಲ-ಕ-ವನ್ನು
ಚಿಲುಮೆ
ಚಿಲುಮೆ-ಯನ್ನು
ಚಿಲುಮೆ-ಯಲ್ಲಿ
ಚಿಲ್ಲರೆ
ಚಿಹ್ನೆ
ಚಿಹ್ನೆ-ಗಳೊಂದೂ
ಚಿಹ್ನೆಯ
ಚಿಹ್ನೆ-ಯಲ್ಲ
ಚಿಹ್ನೆ-ಯಾಗಿ
ಚಿಹ್ನೆ-ಯಿಲ್ಲದೆ
ಚಿಹ್ನೆಯೆ
ಚಿಹ್ನೆಯೇ
ಚೀಟಿ
ಚೀಟಿಯ
ಚೀಟಿ-ಯನ್ನು
ಚೀನಾ
ಚೀಲ-ದಲ್ಲಿ
ಚುಕ್ಕೆ
ಚುಕ್ಕೆ-ಯಂತೆ
ಚುಚುಂದರಿ-ವಧ
ಚುಚ್ಚಿದಂತಾ-ಗುತ್ತದೆ
ಚುಚ್ಚು
ಚುನಾಯಿ-ಸಲ್ಪಟ್ಟರು
ಚುರು-ಕಿನ
ಚುರು-ಕಿ-ನಿಂದ
ಚೂಡಾ
ಚೂಡಾ-ತಾರ
ಚೂಡಾ-ಮಣಿ
ಚೂಡಾ-ಮ-ಣಿಯ
ಚೂರಾಗಿ
ಚೂರಾಗಿತ್ತು
ಚೂರು
ಚೂರು-ಚೂ-ರಾಗುವೆ
ಚೂರೊಂದನ್ನು
ಚೂರ್ಣ
ಚೆಂಡು
ಚೆನ್ನಾಗಿ
ಚೆನ್ನಾಗಿಟ್ಟಿರ-ಬೇಕು
ಚೆನ್ನಾಗಿತ್ತು
ಚೆನ್ನಾಗಿದೆ
ಚೆನ್ನಾಗಿಯೇ
ಚೆನ್ನಾಗಿ-ರ-ಲಿಲ್ಲ
ಚೆನ್ನಾಗಿ-ರುತ್ತದೆ
ಚೆನ್ನಾಗಿ-ರುತ್ತಿತ್ತೆ
ಚೆನ್ನಾಗಿ-ರು-ವುದು
ಚೆನ್ನಾಗಿಲ್ಲ
ಚೆನ್ನಾಗಿಲ್ಲ-ದಿದ್ದರೂ
ಚೆನ್ನಾಗಿಲ್ಲ-ದಿದ್ದರೆ
ಚೆಲ್ಲಿ
ಚೆಲ್ಲಿದ
ಚೆಲ್ಲಿದೆ
ಚೇತನ
ಚೇತನದ
ಚೇತನ-ದಲ್ಲಿ
ಚೇತನ-ದೇವ-ನೆ-ನಿ-ಪನೊ
ಚೇತನ-ವನೊ-ಳ-ಗೊಂಡ
ಚೇಷ್ಟಂ
ಚೈತನ್ಯ
ಚೈತನ್ಯದ
ಚೈತನ್ಯ-ದಿಂದ
ಚೈತನ್ಯ-ದೇವನೂ
ಚೈತನ್ಯನ
ಚೈತನ್ಯ-ಮಯಿ-ಯೆಂದು
ಚೈತನ್ಯ-ವನ್ನು
ಚೈತನ್ಯಸ್ವ-ರೂಪನು
ಚೊಂಬು
ಚೊಕ್ಕಟ-ವಾಗಿವೆ
ಚೋಟಿ
ಚೋರಬಾ-ಗಾ-ನಿನ
ಚೌಕ
ಚೌಕಾಸಿ
ಚೌಗು-ನೆಲದ
ಚೌಧರಿ-ಯವರ
ಚ್ಹಡಾಯ್
ಛ
ಛಂದ
ಛಂದಸ್ಸಿ-ಗಾಗಿ
ಛಂದಸ್ಸಿನ
ಛಂದೋ-ಭಂಗ
ಛಡೊ
ಛಲ
ಛಲ-ದಿಂದ
ಛಲ-ವನ್ನು
ಛವಿ
ಛಾಂದೋಗ್ಯ
ಛಾಗ-ಕಂಠ
ಛಾಡಿ
ಛಾಡಿತೇ
ಛಾಡೆ
ಛಾಯ
ಛಾಯ-ಸಮ
ಛಾಯಾ
ಛಾಯಾ-ದಲ
ಛಾಯಾಯ
ಛಾಯಾ-ರೂಪ-ವಾಗಿ
ಛಾಯಿಲ
ಛಾಯೆ
ಛಾಯೆ-ಯನ್ನು
ಛಾಯೆ-ಯನ್ನೂ
ಛಾವಣಿ-ಯಿಂದ
ಛಿ
ಛಿದ್ರ
ಛಿದ್ರ-ಗೊಳಿಸಿ-ದರೋ
ಛಿದ್ರ-ಛಿದ್ರ-ವಾ-ಯಿತು
ಛೀಮಾರಿ
ಛೆಲೆ
ಛೋಡೇ
ಛೋಯಾ
ಜಂಗಮ-ಗಳೆಲ್ಲವೂ
ಜಂಗುಳಿ
ಜಂಜಡ
ಜಂಜಡ-ಗಳನ್ನು
ಜಂಝೆ
ಜಗ
ಜಗಕೆ
ಜಗಕ್ಕೆ
ಜಗ-ಜನ
ಜಗಜ್ಜಾಲಾತ್
ಜಗತ
ಜಗತ್
ಜಗತ್ಕಲ್ಯಾಣ-ಕರ-ವಾ-ದುವು
ಜಗತ್ಕಲ್ಯಾಣಕ್ಕಾಗಿ
ಜಗತ್ಕಲ್ಯಾ-ಣಕ್ಕೆ
ಜಗತ್ಕಲ್ಯಾಣ-ವಾಗಿ
ಜಗತ್ಕಲ್ಯಾಣ-ವಾಗು-ವುದು
ಜಗತ್ತನ್ನು
ಜಗತ್ತನ್ನೇ
ಜಗತ್ತಾಗಿ
ಜಗತ್ತಾಗಿ-ದೆಯೋ
ಜಗತ್ತಿಗೂ
ಜಗತ್ತಿಗೆ
ಜಗತ್ತಿ-ಗೆಲ್ಲಾ
ಜಗತ್ತಿಗೇ
ಜಗತ್ತಿನ
ಜಗತ್ತಿನಲ್ಲಾದ
ಜಗತ್ತಿ-ನಲ್ಲಿ
ಜಗತ್ತಿನಲ್ಲಿಯೆ
ಜಗತ್ತಿನಲ್ಲಿ-ರ-ಬಲ್ಲರೆ
ಜಗತ್ತಿನಲ್ಲಿ-ರು-ವುದಕ್ಕೆ
ಜಗತ್ತಿನಲ್ಲಿಲ್ಲ
ಜಗತ್ತಿನಲ್ಲೆಲ್ಲಾ
ಜಗತ್ತಿ-ನಿಂದ
ಜಗತ್ತು
ಜಗತ್ತು-ಗಳಿಂದ
ಜಗತ್ತೂ
ಜಗತ್ತೆಂಬ
ಜಗತ್ತೆಲ್ಲಾ
ಜಗತ್ತೇ
ಜಗತ್ಪಾಲಕ-ನಾದ
ಜಗತ್ಸೃಷ್ಟಿ
ಜಗದ
ಜಗ-ದಲಿ
ಜಗದಲ್ಲಿಯೆ
ಜಗ-ದಾಚೆಗೂ
ಜಗದೀಶ್ವರ
ಜಗದುದ್ಧಾರಣ
ಜಗದೇಕಗಮ್ಯ
ಜಗ-ದೊಡ-ಲಿನಾಳ-ದಲಿ
ಜಗ-ದೊ-ಳಗೆ
ಜಗ-ದೊಳು-ಪೊಳ್ಳು
ಜಗದ್ಗುರು-ಗ-ಳಾದ
ಜಗದ್ಧಿ-ತಾಯ
ಜಗದ್ಭೋಧಾತ್ಮಕ-ವಾದ
ಜಗದ್ವಿಕಾಸ-ವಾಗು-ವುದಕ್ಕೆ
ಜಗದ್ವಿಖ್ಯಾತ
ಜಗನ್ನಾಥ
ಜಗನ್ನಾಥಕ್ಷೇತ್ರ-ವಾಗಿತ್ತು
ಜಗನ್ನಾಥನ
ಜಗನ್ಮಾತೆ
ಜಗನ್ಮಾ-ತೆಗೆ
ಜಗನ್ಮಾತೆಯ
ಜಗನ್ಮಾತೆ-ಯನ್ನು
ಜಗನ್ಮಾತೆಯೂ
ಜಗ-ಭೂಷಣ
ಜಗರ್ಜ
ಜಗಲಿಯ
ಜಗಳ
ಜಗಳ-ಗಳಿವೆ
ಜಗಳ-ವನ್ನು
ಜಗವ
ಜಗವಂದನ
ಜಗವ-ನೆಚ್ಚರಿಸು-ತಿದೆ
ಜಗವ-ನೊಂದಾಗಿ-ಪಳು
ಜಗವು
ಜಗ-ವೆಂಬುದಲ್ಲಿಲ್ಲ
ಜಗವೆನಿತು
ಜಗ-ವೆಲ್ಲ
ಜಗ-ವೆಲ್ಲ-ವನು
ಜಜ್ಜಿ
ಜಟಾ
ಜಟಾ-ಜಾಲ
ಜಟಿಲ
ಜಟಿಲ-ವಾಗು-ವುದು
ಜಟೆ-ಗಳು
ಜಟೆ-ಯನ್ನೂ
ಜಡ
ಜಡ-ಜೀವ
ಜಡ-ಜೀವ-ವಿದೆ
ಜಡತ್ವದ
ಜಡತ್ವ-ವನ್ನು
ಜಡದ
ಜಡ-ನಾಗಿ
ಜಡ-ಪದಾರ್ಥದ
ಜಡ-ಪದಾರ್ಥ-ವನ್ನು
ಜಡಪ್ರಾಯ
ಜಡ-ಮನಸ್ಸು
ಜಡ-ಯಂತ್ರ-ವೆಂದು
ಜಡರಂತಾಗಿ
ಜಡ-ರನ್ನಾಗಿ
ಜಡ-ರಾಗಿ
ಜಡ-ರಾಗಿ-ರುವರು
ಜಡ-ರೂಪ-ವಾದ
ಜಡವ
ಜಡ-ವಸ್ತು-ವಿನ
ಜಡ-ವಸ್ತು-ವೆನ್ನುತ್ತೇವೆ
ಜಡ-ವಾಗೇ
ಜಡ-ವಾದ-ವಷ್ಟೆ
ಜಡವ್ಯಕ್ತಿ-ಗಳು
ಜಡ-ಶರೀ-ರದ
ಜಡೆಗಟ್ಟಿ
ಜಡೆ-ಗಳನ್ನು
ಜಡೆಯ
ಜತ
ಜತುರಾಯೀ
ಜತೆ-ಯಲ್ಲಿ
ಜನ
ಜನಕ
ಜನ-ಕ-ಜನಿತ-ಭಾವೋ
ಜನ-ಕ-ನಂತೆ
ಜನ-ಕ-ನನ್ನು
ಜನ-ಕನು
ಜನ-ಕ-ರಾಜ
ಜನ-ಕ-ರಾಜನ
ಜನ-ಕ-ರಾಜ-ನಂತೆ
ಜನ-ಕ-ರಾಜ-ನೊ-ಡನೆ
ಜನ-ಕ-ವಲ್ಲ
ಜನಕ್ಕಿಂತ
ಜನಕ್ಕೆ
ಜನ-ಗಳ
ಜನ-ಗಳನ್ನು
ಜನ-ಗಳು
ಜನ-ಜೀವನ
ಜನ-ಜೀವನ-ವೆಂಬ
ಜನತೆ
ಜನ-ತೆಗೆ
ಜನ-ತೆಯ
ಜನ-ತೆ-ಯೆಂದೂ
ಜನದ
ಜನದೊಂಬಿ
ಜನನ
ಜನ-ನ-ಕಾಲ-ದಲ್ಲಿ
ಜನ-ನ-ದೆಡೆಯಿಂ
ಜನ-ನ-ಮ-ರಣ
ಜನ-ನ-ಮ-ರಣ-ಗಳಿಲ್ಲವೊ
ಜನ-ನ-ಮ-ರಣ-ಗಳೆಂಬ
ಜನ-ನ-ಮ-ರಣದ
ಜನ-ನ-ವಿತ್ತ
ಜನ-ನ-ವು-ಮರ-ಣವು
ಜನ-ನಾರಭ್ಯ
ಜನನೀ
ಜನಮ
ಜನರ
ಜನ-ರಂತೆ
ಜನ-ರ-ನೇ-ಕರು
ಜನ-ರನ್ನು
ಜನ-ರನ್ನೂ
ಜನ-ರನ್ನೆಲ್ಲಾ
ಜನ-ರಲ್
ಜನ-ರಲ್ಲ
ಜನ-ರಲ್ಲಿ
ಜನ-ರಲ್ಲಿಯೂ
ಜನ-ರಲ್ಲಿ-ರುವುದಕ್ಕಿಂತಲೂ
ಜನ-ರಲ್ಲೆಲ್ಲಾ
ಜನ-ರಾಗಿ-ಬಿಟ್ಟಿದ್ದೀರಿ
ಜನ-ರಾದರೂ
ಜನ-ರಿಂದ
ಜನ-ರಿ-ಗಾಗಿ
ಜನ-ರಿ-ಗಿಂತ
ಜನ-ರಿಗಿರ-ಬೇ-ಕಾದ
ಜನ-ರಿಗೂ
ಜನ-ರಿಗೆ
ಜನ-ರಿ-ಗೆಲ್ಲಾ
ಜನ-ರಿದ್ದ
ಜನ-ರಿದ್ದಾರೆ
ಜನರು
ಜನರೂ
ಜನ-ರೆಂದು
ಜನ-ರೆದುರಿಗಿಟ್ಟ
ಜನ-ರೆ-ದು-ರಿಗೆ
ಜನ-ರೆಲ್ಲ
ಜನ-ರೆಲ್ಲಾ
ಜನರೊ
ಜನ-ರೊಡನೆ
ಜನ-ರೊ-ಡ-ನೆಯೂ
ಜನವರಿ
ಜನವ-ರಿ-ಯಲ್ಲಿ
ಜನವೂ
ಜನ-ವೆಲ್ಲ
ಜನ-ಸಂಘ
ಜನ-ಸಂಘ-ದೊ-ಡನೆ
ಜನ-ಸಮಾಜ-ದಲ್ಲಿ
ಜನ-ಸಮುದಾ-ಯಕ್ಕೆ
ಜನ-ಸಮುದಾಯ-ದಲ್ಲಿ
ಜನ-ಸಮೂಹ
ಜನ-ಸಮೂಹ-ವೆಲ್ಲಾ
ಜನ-ಸಾ-ಧಾರ-ಣ-ರಲ್ಲಿ
ಜನ-ಸಾ-ಧಾರ-ಣ-ರಲ್ಲಿಯೂ
ಜನ-ಸಾ-ಧಾರ-ಣ-ರಿಗೆ
ಜನ-ಸಾ-ಧಾರ-ಣರು
ಜನ-ಸಾ-ಮಾನ್ಯರ
ಜನ-ಸಾ-ಮಾನ್ಯ-ರಲ್ಲಿ
ಜನ-ಸಾ-ಮಾನ್ಯ-ರಿಗೂ
ಜನ-ಸಾ-ಮಾನ್ಯ-ರಿಗೆ
ಜನ-ಸಾ-ಮಾನ್ಯರು
ಜನ-ಸಾ-ಮಾನ್ಯ-ರೆ-ದು-ರಿಗೆ
ಜನಸ್ತೋಮ
ಜನಸ್ತೋಮವು
ಜನಾಂಗ
ಜನಾಂಗಕ್ಕೆ
ಜನಾಂಗ-ಗಳ
ಜನಾಂಗ-ಗಳಲ್ಲಿ
ಜನಾಂಗ-ಗ-ಳಾದ
ಜನಾಂಗ-ಗಳು
ಜನಾಂಗ-ಗಳೂ
ಜನಾಂಗದ
ಜನಾಂಗ-ದಲ್ಲಿ
ಜನಾಂಗ-ದಲ್ಲಿಯೇ
ಜನಾಂಗ-ದ-ವ-ರನ್ನು
ಜನಾಂಗ-ದ-ವ-ರಿಂದ
ಜನಾಂಗ-ದ-ವ-ರಿಗೆ
ಜನಾಂಗ-ದ-ವ-ರೊಡನೆ
ಜನಾಂಗ-ದೊಂದಿಗೆ
ಜನಾಂಗ-ದೊ-ಡನೆ
ಜನಾಂಗಳು
ಜನಾಂಗ-ವನ್ನು
ಜನಾಂಗ-ವನ್ನೇ
ಜನಾಂಗ-ವಾಗಿ
ಜನಾಂಗ-ವಾಗಿ-ರ-ಬೇಕು
ಜನಾಂಗ-ವಾಗಿ-ರಲು
ಜನಾಂಗ-ವಾಗಿ-ರು-ವುದು
ಜನಾಂಗ-ವಿದ್ದದ್ದು
ಜನಾಂಗವು
ಜನಾಂಗವೂ
ಜನಾಂಗ-ವೆಲ್ಲಾ
ಜನಾಂಗವೇ
ಜನಾಂಗ-ವೊಂದು
ಜನಾಂಗೀಯ
ಜನಾಃ
ಜನಿತ-ಭಾವ
ಜನಿ-ಮೃತ್ಯು-ಜಾಲಂ
ಜನಿ-ವಾರ
ಜನಿ-ವಾರ-ವನ್ನು
ಜನಿ-ವಾರ-ವಿದ್ದು
ಜನಿಸಿದ
ಜನಿಸಿ-ದು-ದ-ರಿಂದ
ಜನಿಸಿ-ದುದು
ಜನಿ-ಸಿ-ರುವರು
ಜನಿ-ಸಿಲ್ಲ
ಜನಿ-ಸುವ
ಜನುಮ
ಜನುಮ-ಗ-ಳಲ್ಲು
ಜನುಮ-ಜನುಮ-ದಲ್ಲು
ಜನುಮ-ಜನು-ಮದಿ
ಜನುಮೆ
ಜನೆ
ಜನ್ಮ
ಜನ್ಮಕು
ಜನ್ಮಕ್ಕೆ
ಜನ್ಮ-ಗಳ
ಜನ್ಮ-ಗಳನ್ನು
ಜನ್ಮ-ಗ-ಳಾದ-ನಂತರ
ಜನ್ಮ-ಗಳು
ಜನ್ಮಗ್ರಹ-ಣ-ವಾದಾಗ
ಜನ್ಮ-ಜನ್ಮ-ದಲ್ಲಿಯೂ
ಜನ್ಮ-ಜನ್ಮಾಂತ-ರದಿ
ಜನ್ಮತಃ
ಜನ್ಮ-ತಳೆ-ದಾಗ
ಜನ್ಮ-ತಾಳಿ-ರುವರೋ
ಜನ್ಮ-ತಾಳುತ್ತಲೇ
ಜನ್ಮ-ತಿಥಿ
ಜನ್ಮ-ತಿಥಿಯ
ಜನ್ಮದ
ಜನ್ಮ-ದತ್ತ-ವಾಗಿ
ಜನ್ಮ-ದಲ್ಲಿ
ಜನ್ಮ-ದಲ್ಲಿಯೂ
ಜನ್ಮ-ದಲ್ಲಿಯೆ
ಜನ್ಮ-ದಲ್ಲೇ
ಜನ್ಮ-ದಾ-ತೆಯೇ
ಜನ್ಮದಿ
ಜನ್ಮ-ದಿನ
ಜನ್ಮ-ಧಾರ-ಣೆ-ಮಾಡಿ
ಜನ್ಮ-ವನ್ನು
ಜನ್ಮ-ವನ್ನೆ
ಜನ್ಮ-ವಿರು-ವುದು
ಜನ್ಮ-ವಿಲ್ಲ
ಜನ್ಮ-ವಿಶೇಷ-ದಿಂದ
ಜನ್ಮವು
ಜನ್ಮ-ವೆತ್ತಿದ್ದೇನೆ
ಜನ್ಮ-ಸಂಸಿದ್ಧಸ್ತತೋ
ಜನ್ಮಸ್ಥಳ-ವನ್ನು
ಜನ್ಮಸ್ಥಾನ
ಜನ್ಮಸ್ಥೇಮ-ಭಂಗ
ಜನ್ಮಾಂತ-ರದ
ಜನ್ಮೋತ್ಸವ
ಜನ್ಮೋತ್ಸವಕ್ಕಾಗಿ
ಜನ್ಮೋತ್ಸವಕ್ಕೆ
ಜನ್ಯ-ವಾದ
ಜಪ
ಜಪತಪ
ಜಪತಪ-ಗಳನ್ನು
ಜಪತಪ-ಗಳಲ್ಲಿ
ಜಪಧ್ಯಾನ
ಜಪ-ಮಾಡುತ್ತಿದ್ದುದು
ಜಪ-ಮಾಲೆ-ಯನ್ನೆಣಿ-ಸುವುದು
ಜಪಯಜ್ಞ
ಜಪ-ವನ್ನು
ಜಪ-ವೆಂದು
ಜಪಾನರ
ಜಪಾ-ನಿಗೆ
ಜಪಾನೀಯ-ರದು
ಜಪಾನೀಯ-ರನ್ನು
ಜಪಾನೀಯ-ರಾಗಿಯೇ
ಜಪಾನೀ-ಯರು
ಜಪಾನು
ಜಪಾನ್
ಜಪಾನ್-ವ-ರೆಗೂ
ಜಪಿ-ಸುತ್ತಾ
ಜಪಿ-ಸುತ್ತಿ-ರುವಾಗ
ಜಬ್
ಜಮಖಾ-ನದ
ಜಮೀ-ನೆಲ್ಲವೂ
ಜಯ
ಜಯ-ಘೋಷ
ಜಯಜಯ
ಜಯಜಯ-ವೆಂದಿವೆ
ಜಯದ
ಜಯ-ದೇವನು
ಜಯ-ದೇವನೇ
ಜಯ-ಪುರ-ದಲ್ಲಿದ್ದಾಗ
ಜಯಪ್ರದ-ವಾಗಿವೆ
ಜಯ-ರಾಮ-ಕೃಷ್ಣ
ಜಯ-ರಾಮ್
ಜಯ-ವದು
ಜಯ-ವೊಂದನ್ನೇ
ಜಯ-ಶಂಕರ
ಜಯಶಾಲಿ-ಗಳಾಗಲು
ಜಯಶಾಲಿ-ಯಾ-ಗು-ವಂತೆ
ಜಯ-ಶೀಲ-ನಾಗಿ-ರು-ವೆನು
ಜಯ-ಶೀಲ-ನಾಗು-ವೆ-ನೆಂದು
ಜಯ-ಶೀಲ-ರಾಗ-ದಿದ್ದರೂ
ಜಯ-ಶೀಲ-ರಾಗುವರು
ಜಯಿಸ-ಬೇಕು
ಜಯಿ-ಸ-ಲಾ-ರರು
ಜಯಿ-ಸಲು
ಜಯಿ-ಸಲ್ಪಟ್ಟ
ಜಯಿಸಿ
ಜಯಿ-ಸಿದ
ಜಯಿಸಿ-ದ-ವನು
ಜಯಿ-ಸುವೆ
ಜರಾ
ಜರಿ-ಯುವರು
ಜರುಗಿ
ಜರುಗಿತು
ಜರೂ-ರಾದ
ಜರೇ
ಜರ್ಝರಿ-ತ-ನಾದಾಗಲೂ
ಜರ್ಝರಿ-ತ-ವಾಗು-ವುದೋ
ಜರ್ಮನ್
ಜಲ
ಜಲಗಾರ
ಜಲಗಾರ-ನಿದ್ದಲ್ಲಿಗೆ
ಜಲದ
ಜಲ-ದಂತೆ
ಜಲಧಿ
ಜಲಧಿಂ
ಜಲ-ಮತಿ-ತರಲಂ
ಜಲ-ಮಯ್
ಜಲ-ರಾಶಿ
ಜಲ-ರಾಶಿ-ಯಲ್ಲೆಲ್ಲಾ
ಜಲ-ವನ್ನು
ಜಲೆಜಿಲ
ಜಲ್ಪಂತಿ
ಜವದಿ
ಜವಾಬ್ದಾರ-ನಾಗುವೆ
ಜವಾಬ್ದಾರ-ನೆಂದು
ಜವಾಬ್ದಾರನೋ
ಜವಾಬ್ದಾರಿ
ಜವಾಬ್ದಾರಿ-ಗಳನ್ನು
ಜವಾಬ್ದಾರಿಯ
ಜವಾಬ್ದಾರಿ-ಯನ್ನು
ಜವಾಬ್ದಾರಿ-ಯನ್ನೆಲ್ಲ
ಜವ್ವನ
ಜಹಜಿ-ನಲ್ಲಿ
ಜಹಜಿನಲ್ಲಿಯೇ
ಜಹಜು
ಜಹೌ
ಜಾಗಟೆ
ಜಾಗತಿಕ
ಜಾಗರಿ-ತ-ವಾಗಿದೆ
ಜಾಗರೂಕ-ನಾಗಿ-ರುತ್ತಾ-ನೆಯೋ
ಜಾಗ-ವನ್ನು
ಜಾಗವೂ
ಜಾಗೃತ-ಗೊಂಡಿ-ರುವುದು
ಜಾಗೃತ-ಗೊಳಿ-ಸಲು
ಜಾಗೃತ-ಗೊಳಿಸಿ
ಜಾಗೃತ-ಗೊಳಿಸು
ಜಾಗೃತ-ಗೊಳಿಸು-ವು-ದರ
ಜಾಗೃತ-ಗೊಳಿಸು-ವುದು
ಜಾಗೃತ-ಗೊಳಿಸು-ವುದೇ
ಜಾಗೃತಗೊಳ್ಳ-ಬೇಕು
ಜಾಗೃತ-ಗೊಳ್ಳುವುದು
ಜಾಗೃ-ತ-ನಾಗಿ-ರು-ವ-ನೆಂದು
ಜಾಗೃತ-ನಾಗು-ವನು
ಜಾಗೃತ-ರನ್ನಾಗಿ
ಜಾಗೃತ-ರಾಗಿ
ಜಾಗೃತ-ವಾಗ-ಬೇಕು
ಜಾಗೃತ-ವಾಗಿದೆ
ಜಾಗೃತ-ವಾಗುತ್ತದೆ
ಜಾಗೃತ-ವಾದ
ಜಾಗೃ-ತಾತೀ-ತಾವಸ್ಥೆ-ಯಲ್ಲಿ
ಜಾಗೃ-ತಾವಸ್ಥೆ
ಜಾಗೃತಿ
ಜಾಗೃತಿ-ಗೊಳ್ಳುವನು
ಜಾಗೃತಿ-ಗೊಳ್ಳುವರು
ಜಾಗೃತಿ-ಯನ್ನುಂಟು-ಮಾಡು-ವರು
ಜಾಗೊ
ಜಾಗ್ರತ
ಜಾಗ್ರತ-ಗೊಳಿಸಿ-ಬಿಟ್ಟರೆ
ಜಾಟಾಜೂಟ
ಜಾಡ-ಮಾಲಿ
ಜಾಡ-ಮಾಲಿ-ಯಾದ-ನೆಂಬು-ದನ್ನೂ
ಜಾಡಿನಲ್ಲಿಯೇ
ಜಾಡಿ-ನಲ್ಲಿ-ರ-ಲಾ-ರರು
ಜಾಡಿ-ನಲ್ಲೇ
ಜಾಡ್ಯ-ದಿಂದ
ಜಾಣನೇ
ಜಾಣ-ರಾಗಿದ್ದಾರೆ
ಜಾಣೆ-ಯಂತೆ
ಜಾತಃ
ಜಾತಕ-ಪಕ್ಷಿ
ಜಾತಿ
ಜಾತಿ-ಗತ
ಜಾತಿ-ಗಳ
ಜಾತಿ-ಗಳನ್ನು
ಜಾತಿ-ಗಳನ್ನೂ
ಜಾತಿ-ಗಳಾಗಿ
ಜಾತಿ-ಗಿಂತ
ಜಾತಿ-ಗೀತೆ-ಗಳ
ಜಾತಿಗೂ
ಜಾತಿಗೆ
ಜಾತಿ-ದೋಷ
ಜಾತಿ-ನಿಯಮ-ಗಳು
ಜಾತಿ-ನಿರ್ಮೂಲ-ನಕ್ಕೋಸ್ಕರ
ಜಾತಿ-ಪದ್ಧತಿ
ಜಾತಿಪ್ರತಿಷ್ಠೆ
ಜಾತಿ-ಬಂಧ-ನ-ದಿಂದಲೂ
ಜಾತಿ-ಬದ್ದ-ನಾಗಿ-ರು-ವುದ-ರಲ್ಲಿ
ಜಾತಿ-ಬಾಹಿರ
ಜಾತಿಯ
ಜಾತಿ-ಯನ್ನಾಗಿ
ಜಾತಿ-ಯನ್ನು
ಜಾತಿ-ಯ-ವರ
ಜಾತಿ-ಯ-ವ-ರನ್ನು
ಜಾತಿ-ಯ-ವ-ರಲ್ಲಿ
ಜಾತಿ-ಯ-ವ-ರಿಗೂ
ಜಾತಿ-ಯ-ವ-ರಿಗೆ
ಜಾತಿ-ಯ-ವರು
ಜಾತಿ-ಯ-ವರೂ
ಜಾತಿ-ಯಾಗಿ
ಜಾತಿ-ಯಾಗುತ್ತದೆ
ಜಾತಿ-ಯಿಂದ
ಜಾತಿ-ಯಿ-ರಲಿ
ಜಾತಿ-ವರ್ಣ-ಗಳನ್ನು
ಜಾತಿ-ವಿ-ಭ-ಜನೆ-ಯಿತ್ತು
ಜಾತಿ-ವಿ-ಭಾಗವು
ಜಾತಿ-ವೈಶಿಷ್ಟ್ಯವೇ
ಜಾತಿ-ಹೀನ-ನಾಗಿ-ರಲಿ
ಜಾತೀಯ
ಜಾತ್ಯಂತರ
ಜಾನಕೀಪ್ರಾಣ-ಬಂಧೋ
ಜಾನಕೊ
ಜಾನಕೋ
ಜಾನ-ಜಾನಿ
ಜಾನಾತಿ
ಜಾನಾಮ್ಯಹಂ
ಜಾನಿ
ಜಾನಿ-ವಾರೆ
ಜಾನೆಕೊ
ಜಾನ್
ಜಾಯ
ಜಾಯ-ಕೇಯಜಾನೆ
ಜಾಯಸ್ವ
ಜಾಯಿತ
ಜಾಯ್
ಜಾರ
ಜಾರಿ
ಜಾರಿಗೆ
ಜಾರಿ-ಬಿದ್ದನು
ಜಾರಿ-ಬಿದ್ದಾಗ
ಜಾರಿ-ಬಿದ್ದು
ಜಾರಿ-ಹೋಗಿ-ರು-ವುದೋ
ಜಾರು-ತಲಿ-ಹುದು
ಜಾರು-ತಿ-ಹುದು
ಜಾಲ-ಗಳನ್ನು
ಜಾಲ-ದಲ್ಲಿ
ಜಾವ
ಜಾವಕ್ಕೆ
ಜಾವ-ದಿಂದ
ಜಾವೂ
ಜಾಸ್ತಿ
ಜಾಸ್ತಿ-ಯಾ-ಯಿತು
ಜಾಹ
ಜಾಹಿರಾತು
ಜಿ
ಜಿಜ್ಞಾಸು-ಗಳಾಗಿ
ಜಿತೇಂದ್ರಿಯತ್ವ-ವನ್ನು
ಜಿತೇಂದ್ರಿಯನ
ಜಿತ್ವಾ
ಜಿದ್ದಿ-ನಿಂದ
ಜಿನುಗು-ವುವು
ಜಿಯೀ
ಜಿಲ್ಲೆಯ
ಜಿಸಿ
ಜಿಸಿಯ
ಜೀತದಾಳಿವ-ನೆಲ್ಲಿ
ಜೀರುಂಡೆಯ
ಜೀರೆಂಬುವ
ಜೀರ್ಣ-ಶಕ್ತಿ-ಯನ್ನು
ಜೀರ್ಣ-ಹೊಂದಿ
ಜೀರ್ಣಾಗ್ನಿ-ಯಾತ-ನದು
ಜೀರ್ಣಿಸಿ
ಜೀರ್ಣಿಸಿ-ಕೊಂಡಿ-ರು-ವು-ದನ್ನು
ಜೀರ್ಣಿಸಿ-ಕೊಳ್ಳುವ
ಜೀರ್ಣಿಸಿ-ಕೊಳ್ಳುವರು
ಜೀವ
ಜೀವಂತ
ಜೀವಂತ-ವಾಗಿ
ಜೀವಂತ-ವಾಗಿ-ರ-ಬೇ-ಕಾದರೆ
ಜೀವಂತ-ವಾಗಿ-ರುತ್ತದೆ
ಜೀವಂತ-ವಾಗಿ-ರು-ವು-ದೆಂದು
ಜೀವಂತ-ವೆನ್ನುತ್ತೇವೆ
ಜೀವಂತಿ-ಕೆಯ
ಜೀವ-ಕಳೆ
ಜೀವ-ಕಳೆ-ಯನ್ನು
ಜೀವ-ಕಳೆಯೇ
ಜೀವ-ಕೋಶವೂ
ಜೀವಕ್ಕೆ
ಜೀವ-ಗಳ
ಜೀವ-ಗಳಿಗೆ
ಜೀವ-ಗಳು
ಜೀವಚ್ಛವ-ಗಳನ್ನಾಗಿ
ಜೀವ-ಜಂತು-ಗಳನ್ನು
ಜೀವ-ಜಂತು-ಗಳಾಳಿಗೆ
ಜೀವ-ಜಂತು-ಗಳೂ
ಜೀವ-ಜ-ಗತ್ತಿನ
ಜೀವದ
ಜೀವ-ದಲಿ
ಜೀವ-ದಾನ
ಜೀವ-ದಾಳ-ವನು
ಜೀವ-ದಾಸೆಯ
ಜೀವನ
ಜೀವನಂ
ಜೀವನ-ಕಥೆಯ
ಜೀವನ-ಕೊಂದು
ಜೀವನಕ್ಕೆ
ಜೀವನಕ್ಕೇ
ಜೀವನ-ಗಳನ್ನು
ಜೀವನ-ಗಳನ್ನೂ
ಜೀವನ-ಚರಿತ್ರೆ-ಗಳನ್ನು
ಜೀವನ-ಚಿತ್ರ-ಣ-ಗಳನ್ನು
ಜೀವನದ
ಜೀವನ-ದಲ್ಲಿ
ಜೀವನ-ದಲ್ಲಿದ್ದು
ಜೀವನ-ದಲ್ಲಿಯೂ
ಜೀವನ-ದಲ್ಲಿಯೇ
ಜೀವನ-ದಲ್ಲಿ-ರುವ
ಜೀವನ-ದಲ್ಲಿ-ರು-ವುದೂ
ಜೀವನ-ದಲ್ಲೇ
ಜೀವನ-ದಷ್ಟು
ಜೀವನ-ದಾದ್ಯಂತ
ಜೀವನ-ದಿಂದ
ಜೀವನ-ದೆಸೆ-ಯಲ್ಲಿ
ಜೀವನಲಿ
ಜೀವನಲ್ಲಿ
ಜೀವನ-ವನ್ನಪ್ಪಿದ್ದಾರೆ
ಜೀವನ-ವನ್ನಾಗಿ
ಜೀವನ-ವನ್ನು
ಜೀವನ-ವನ್ನೆಲ್ಲ
ಜೀವನ-ವನ್ನೆಲ್ಲಾ
ಜೀವನ-ವನ್ನೇ
ಜೀವನ-ವಾದರೋ
ಜೀವನ-ವಿದರ
ಜೀವನ-ವಿದ್ದು
ಜೀವನ-ವಿರು-ವುದು
ಜೀವನವೂ
ಜೀವನ-ವೆಲ್ಲ
ಜೀವನ-ವೆಲ್ಲಾ
ಜೀವನವೇ
ಜೀವನ-ವೇನೋ
ಜೀವನಸ್ಪರ್ಧೆ
ಜೀವನಾಡಿ-ಯಾಗಿ-ರುತ್ತಾನೆ
ಜೀವನಾ-ಧಾ-ರಕ್ಕೆ
ಜೀವನಾನು-ಭವದ
ಜೀವನಿಗೂ
ಜೀವನಿಗೆ
ಜೀವನು
ಜೀವನೆಂಬ
ಜೀವನೋತ್ಸಾಹದ
ಜೀವನೋ-ಪಾ-ಯಕ್ಕೆ
ಜೀವನ್ಮುಕ್ತ
ಜೀವನ್ಮುಕ್ತ-ನನ್ನು
ಜೀವನ್ಮುಕ್ತ-ನಲ್ಲಿ
ಜೀವನ್ಮುಕ್ತ-ನಾಗ-ಬಾ-ರದೇಕೆ
ಜೀವನ್ಮುಕ್ತ-ನಿಗೆ
ಜೀವನ್ಮುಕ್ತ-ರಾದ
ಜೀವನ್ಮುಕ್ತಿ
ಜೀವನ್ಮುಕ್ತಿಯ
ಜೀವ-ಪೋಷಕ-ವಾಗುವ
ಜೀವ-ಮಾನ-ದಲ್ಲಿ
ಜೀವ-ಮಾನ-ದಲ್ಲೇ
ಜೀವ-ಮಾನ-ವನ್ನೆಲ್ಲಾ
ಜೀವ-ಮಾನ-ವೆಂಬ
ಜೀವ-ಮಾನ-ವೆಲ್ಲಾ
ಜೀವರ
ಜೀವ-ರಾಶಿ-ಯನ್ನು
ಜೀವ-ರಾಶಿ-ಯಲ್ಲಿಯೂ
ಜೀವ-ರಿಗೆ
ಜೀವ-ವನ್ನು
ಜೀವ-ವನ್ನೂ
ಜೀವ-ವನ್ನೇ
ಜೀವ-ವಿಕಾಸದ
ಜೀವ-ವಿಲ್ಲದ
ಜೀವವು
ಜೀವ-ವೆಂದೆ-ಣಿ-ಸಿದಾಗ
ಜೀವ-ವೆಲ್ಲವ
ಜೀವ-ಹಿತ
ಜೀವ-ಹಿ-ತಕ್ಕೋಸ್ಕರ-ವಾಗಿ
ಜೀವ-ಹಿತೇಚ್ಛೆ-ಯಿಂದ
ಜೀವಾಣು-ಗಳಲ್ಲವೇ
ಜೀವಾಣುವೂ
ಜೀವಾತ್ಮ
ಜೀವಾತ್ಮ-ಗಳ
ಜೀವಾತ್ಮ-ಗಳನ್ನು
ಜೀವಾತ್ಮನ
ಜೀವಾತ್ಮ-ನಿ-ರುವನು
ಜೀವಾತ್ಮ-ನಿಲ್ಲ
ಜೀವಾತ್ಮನು
ಜೀವಾತ್ಮವು
ಜೀವಾಳ
ಜೀವಾಳ-ವಾಗಿದ್ದರು
ಜೀವಾ-ವಧಿಯ
ಜೀವಿ
ಜೀವಿ-ಗಳ
ಜೀವಿ-ಗಳನ್ನಾಗಿ
ಜೀವಿ-ಗಳನ್ನು
ಜೀವಿ-ಗಳನ್ನೂ
ಜೀವಿ-ಗ-ಳಲ್ಲೂ
ಜೀವಿ-ಗಳಾಗಿ
ಜೀವಿ-ಗಳಿಗೂ
ಜೀವಿ-ಗಳಿಗೆ
ಜೀವಿ-ಗಳು
ಜೀವಿ-ಗಳೆಲ್ಲಾ
ಜೀವಿ-ಗಳೇ
ಜೀವಿಗೂ
ಜೀವಿತಂ
ಜೀವಿ-ತ-ಕಾಲ-ದಲ್ಲೇ
ಜೀವಿ-ತದ
ಜೀವಿ-ಪನು
ಜೀವಿಯ
ಜೀವಿಯು
ಜೀವಿಯೂ
ಜೀವಿ-ಸ-ಬೇಕು
ಜೀವಿ-ಸ-ಬೇ-ಕೇನು
ಜೀವಿ-ಸಲು
ಜೀವಿ-ಸಿ-ಕೊಂಡಿದ್ದು
ಜೀವಿ-ಸಿದ್ದರೆ
ಜೀವಿ-ಸಿ-ರು-ವು-ದ-ರಿಂದ
ಜೀವಿ-ಸಿ-ರು-ವುದು
ಜೀವಿ-ಸುತ್ತಾ
ಜೀವಿ-ಸುತ್ತಿದ್ದೆ
ಜೀವಿ-ಸುವ
ಜೀವಿ-ಸುವರು
ಜೀವಿ-ಸುವ-ರೆಂಬು-ದನ್ನು
ಜೀವಿ-ಸು-ವು-ದ-ರಿಂದ
ಜೀವಿ-ಸು-ವುದು
ಜೀವಿ-ಸು-ವೆಯೋ
ಜೀವೆ
ಜೀಸಸ್
ಜುಗುಪ್ಸೆ
ಜುಗುಪ್ಸೆ-ಯಿಲ್ಲದೆ
ಜುಬ್ಬ
ಜುಮ್ಮಾ
ಜುಲೈ
ಜುಲ್ಮಾನೆಯಾಗು-ವು-ದೆಂದು
ಜೂನ್
ಜೂನ್ಜುಲೈ
ಜೂಬಿಲಿ
ಜೆ
ಜೆಂದವಸ್ತ-ಗಳು
ಜೆಜೆ
ಜೆನೆಛಿ
ಜೇಗೆ
ಜೇತೆ
ಜೇನಹನಿ
ಜೇನೆಚಿ
ಜೇಬನ್ನು
ಜೇಯಿ
ಜೈ
ಜೈನ
ಜೈನ-ಮ-ತದ
ಜೈನರ
ಜೈನ-ರನ್ನು
ಜೈನ-ರಿದ್ದರು
ಜೈನರು
ಜೈನರೂ
ಜೈನಶ್ರಮಣ
ಜೈನಶ್ರಮ-ಣರು
ಜೊ
ಜೊತೆ
ಜೊತೆ-ಗಾತಿ
ಜೊತೆಗೂ
ಜೊತೆಗೆ
ಜೊತೆ-ಜೊ-ತೆಗೆ
ಜೊತೆ-ಜೊತೆ-ಯಲ್ಲಿ
ಜೊತೆ-ಜೊತೆ-ಯಲ್ಲಿಯೆ
ಜೊತೆ-ಯಲಿ
ಜೊತೆ-ಯಲ್ಲಿ
ಜೊತೆ-ಯಲ್ಲಿದ್ದ
ಜೊತೆ-ಯಲ್ಲಿದ್ದನು
ಜೊತೆ-ಯಲ್ಲಿದ್ದಾರೆ
ಜೊತೆ-ಯಲ್ಲಿಯೂ
ಜೊತೆ-ಯಲ್ಲಿಯೇ
ಜೊತೆ-ಯಲ್ಲೇ
ಜೊತೆ-ಯಾದರೆ
ಜೊತೆ-ಯಿ-ರಲು
ಜೋಂಪಿ-ನಿಂದ
ಜೋಂಪು
ಜೋಕೆ
ಜೋಗಿ-ಗಳೂ
ಜೋಗುಳ-ದಲ್ಲು
ಜೋಡಿ
ಜೋಡಿ-ಯ-ದಲ್ಲಿ
ಜೋಡಿಸಿ
ಜೋಡಿ-ಸಿಟ್ಟಿ-ರ-ಲಿಲ್ಲ
ಜೋಡಿ-ಸಿದ
ಜೋಡಿ-ಸಿ-ದಂತೆ
ಜೋಡಿ-ಸುವಾಗ
ಜೋಡಿ-ಸು-ವುದು
ಜೋಡು
ಜೋತಿ
ಜೋತಿಃ
ಜೋತಿರ್ಮಯ
ಜೋಪಾನ-ವಾಗಿ-ರ-ಬೇಕು
ಜೋಪಾನ-ವಾಗಿರಿ
ಜೋರಾಗಿ
ಜೋರು-ಮಳೆ
ಜೋಲಾಡುವ
ಜೋಲು
ಜೋಸೆಫೈನ್
ಜ್ಞಾತಾ-ಲಯ
ಜ್ಞಾತೃ
ಜ್ಞಾನ
ಜ್ಞಾನಂ
ಜ್ಞಾನ-ಕಾಂಡ
ಜ್ಞಾನ-ಕಾಂಡದ
ಜ್ಞಾನಕ್ಕೂ
ಜ್ಞಾನಕ್ಕೆ
ಜ್ಞಾನಕ್ಷುಧೆ-ಯನು
ಜ್ಞಾನ-ಗಮ್ಯ-ನಾದ
ಜ್ಞಾನ-ಗಳ
ಜ್ಞಾನ-ಗಳೆಂಬ
ಜ್ಞಾನಜ್ಞೇಯ
ಜ್ಞಾನ-ತಪ್ಪಿ
ಜ್ಞಾನ-ತೃಷೆ
ಜ್ಞಾನ-ತೃಷ್ಣೆ
ಜ್ಞಾನದ
ಜ್ಞಾನ-ದಲ್ಲಿ
ಜ್ಞಾನ-ದಾನ
ಜ್ಞಾನ-ದಾನ-ಗಳನ್ನು
ಜ್ಞಾನ-ದಾನದ
ಜ್ಞಾನ-ದಾನ-ದಿಂದ
ಜ್ಞಾನ-ದಾ-ನವೇ
ಜ್ಞಾನ-ದಿಂದ
ಜ್ಞಾನ-ದೇಹದಿ
ಜ್ಞಾನ-ದೊಂದಿಗೆ
ಜ್ಞಾನ-ಪುಂಜಾಟ್ಟ-ಹಾಸಃ
ಜ್ಞಾನಪ್ರ-ಸಾರ
ಜ್ಞಾನಪ್ರ-ಸಾರ-ಮಾಡುವ
ಜ್ಞಾನ-ಬೋ-ಧನೆ
ಜ್ಞಾನ-ಭಂಡಾರ
ಜ್ಞಾನ-ಮಾರ್ಗ
ಜ್ಞಾನ-ಮಾರ್ಗಾವಲಂಬಿ-ಗಳು
ಜ್ಞಾನ-ಯಜ್ಞ-ವಿದೆ
ಜ್ಞಾನ-ರಾಶಿ-ಯಾಗಿ
ಜ್ಞಾನ-ಲಾಭ-ವಾಗು-ವುದು
ಜ್ಞಾನ-ಲಾಭ-ವಾಗು-ವುದೆ
ಜ್ಞಾನವ
ಜ್ಞಾನ-ವನ್ನು
ಜ್ಞಾನ-ವನ್ನೂ
ಜ್ಞಾನ-ವನ್ನೆಲ್ಲಾ
ಜ್ಞಾನ-ವಸ್ತು
ಜ್ಞಾನ-ವಿಕಾಸ
ಜ್ಞಾನ-ವಿಕಾಸಕ್ಕೆ
ಜ್ಞಾನ-ವಿಜ್ಞಾನ-ಗಳ
ಜ್ಞಾನ-ವಿ-ರುತ್ತದೆ
ಜ್ಞಾನ-ವಿ-ರುತ್ತದೆಯೊ
ಜ್ಞಾನ-ವಿ-ರುತ್ತದೆಯೋ
ಜ್ಞಾನ-ವಿರು-ವು-ದಿಲ್ಲ
ಜ್ಞಾನ-ವಿಲ್ಲ
ಜ್ಞಾನ-ವಿಲ್ಲ-ದಿದ್ದರೆ
ಜ್ಞಾನ-ವಿಲ್ಲದೆ
ಜ್ಞಾನವು
ಜ್ಞಾನವೂ
ಜ್ಞಾನ-ವೃದ್ಧ-ರಾಗುವಿರಿ
ಜ್ಞಾನ-ವೆಂದು
ಜ್ಞಾನ-ವೆಂದೂ
ಜ್ಞಾನ-ವೆಲ್ಲ
ಜ್ಞಾನವೇ
ಜ್ಞಾನ-ಸರ್ವಜ್ಞಾನ-ಗಳೂ
ಜ್ಞಾನ-ಸೂತ್ರದ
ಜ್ಞಾನಾಂಜನ
ಜ್ಞಾನಾಗ್ನಿ-ಯಿಂದ
ಜ್ಞಾನಾತೀತ
ಜ್ಞಾನಾತೀತ-ನಾದ
ಜ್ಞಾನಾತೀತ-ವಾದ
ಜ್ಞಾನಾರ್ಜನೆ-ಗಳನ್ನು
ಜ್ಞಾನಾರ್ಜ-ನೆಯ
ಜ್ಞಾನಾ-ಲೋಕ-ದಿಂದ
ಜ್ಞಾನಾ-ಲೋ-ಕ-ವನ್ನು
ಜ್ಞಾನಿ
ಜ್ಞಾನಿ-ಗಳ
ಜ್ಞಾನಿ-ಗಳು
ಜ್ಞಾನಿಗೂ
ಜ್ಞಾನಿಗೆ
ಜ್ಞಾನಿಯ
ಜ್ಞಾನಿ-ಯಾದ
ಜ್ಞಾನಿಯು
ಜ್ಞಾನಿಯೂ
ಜ್ಞಾನೈಶ್ವರ್ಯಾ-ದಿ-ಗಳು
ಜ್ಞಾನೋದಯ
ಜ್ಞಾನೋದಯ-ವನ್ನು
ಜ್ಞಾನೋದಯ-ವಾಗಿಲ್ಲ
ಜ್ಞಾನೋದಯ-ವಾಗುತ್ತ-ದೆಯೋ
ಜ್ಞಾನೋದಯ-ವಾದರೂ
ಜ್ಞಾನೋಪಾ-ಸನೆ
ಜ್ಞಾಪಕ
ಜ್ಞಾಪ-ಕಕ್ಕೆ
ಜ್ಞಾಪಕ-ದಲ್ಲಿಡಿ
ಜ್ಞಾಪಕ-ದಲ್ಲಿವೆ
ಜ್ಞಾಪಕ-ವಿದೆ
ಜ್ಞಾಪಕ-ವಿಲ್ಲ
ಜ್ಞಾಪಕ-ಶಕ್ತಿ
ಜ್ಞಾಪಿಸಿ
ಜ್ಞಾಪಿಸಿ-ಕೊಂಡು
ಜ್ಞಾಪಿಸು
ಜ್ಞೇಯ
ಜ್ಯೋತಿ
ಜ್ಯೋತಿಃ
ಜ್ಯೋತಿಯ
ಜ್ಯೋತಿ-ಯನ್ನು
ಜ್ಯೋತಿ-ಯಿಲ್ಲ
ಜ್ಯೋತಿರ
ಜ್ಯೋತಿರ್ಮಯ
ಜ್ಯೋತಿರ್ಜ್ಯೋ-ತಿಯೆ
ಜ್ಯೋತಿಷ್ಯರು
ಜ್ಯೋತಿಸ್ತಂಭ-ರೂಪ
ಜ್ವಲತ
ಜ್ವಲಿ-ಸಿದೆ
ಜ್ವಲಿ-ಸುತ್ತಿ-ರುವ
ಜ್ವಲೆ
ಜ್ವಲೇ
ಜ್ವಾಲಾ-ಮುಖಿ
ಜ್ವಾಲೆ
ಜ್ವಾಲೆ-ಗಳ-ನಾಡಿ-ಸುತ
ಜ್ವಾಲೆ-ಯಂತಿ-ರುವ
ಜ್ವಾಲೆಯು
ಝಂಕಾರ
ಝಂಫ
ಝಗ-ಮುಖಿ
ಝರಿ
ಝರೆ
ಝರ್
ಝಲಕಿ
ಝಳಪಿ-ನಿಂದ
ಝಳಪಿ-ಸುವ
ಝಾಕಿ-ರಣೆ
ಝಾಡ-ಮಾಲಿ-ಗಳನ್ನು
ಝಾನ್ಸಿ-ರಾ-ಣಿಯು
ಝೃಂಬಿತ
ಝೇಲಮಿನ
ಟರ್ಕಿ
ಟರ್ಕಿಯ
ಟಲಮಲ
ಟೀ
ಟೀಕಿಸ-ಬೇಕಾಗಿ-ಬಂದರೆ
ಟೀಕಿ-ಸಲಿ
ಟೀಕಿಸಿ
ಟೀಕಿಸಿ-ದರು
ಟೀಕಿಸಿದ್ದರು
ಟೀಕಿಸಿದ್ದೀರಿ
ಟೀಕಿ-ಸುತ್ತಿದ್ದರು
ಟೀಕಿ-ಸುತ್ತಿದ್ದು-ದನ್ನು
ಟೀಕಿ-ಸುತ್ತಿದ್ದೆ
ಟೀಕಿ-ಸುತ್ತಿ-ರುವ
ಟೀಕಿ-ಸುತ್ತೀರಿ
ಟೀಕಿ-ಸು-ವುದು
ಟೀಕೆ
ಟೀಕೆ-ಗಳಿಗೆ
ಟೀಕೆ-ಗಳು
ಟೀಕೆ-ಮಾಡುವ
ಟೀಕೆಯ
ಟೀಕೆ-ಯ-ನೆಲ್ಲ
ಟೀಯನ್ನು
ಟೊಲೆ
ಟ್ರಾಂನಲ್ಲಿ
ಟ್ರಾಮ್
ಠಕ್ಕ-ತನ
ಡಂಬೆಲ್
ಡಮರು
ಡಮರು-ಗಳು
ಡಮರು-ನಾದ
ಡರಾಕ
ಡವಡ-ವನೆ
ಡಾ
ಡಾಕೇ
ಡಾಕ್ಕಾಕ್ಕೆ
ಡಾಕ್ಕಾ-ದಲ್ಲಿ
ಡಾಕ್ಕಾ-ದಲ್ಲೇ
ಡಾಕ್ಟರ್
ಡಾಜಿ-ಎಸ್ಶಿವ-ರುದ್ರಪ್ಪ
ಡಾರ್ಜಿಲಿಂಗಿಗೆ
ಡಾರ್ಜಿಲಿಂಗ್-ನಿಂದ
ಡಾರ್ವಿನ್
ಡಾರ್ವಿನ್ನ
ಡಾಲರ್
ಡಿಮಿ
ಡಿಸೆಂಬರ್
ಡೈರಿ-ಯಿಂದ
ಡೊಂಕುಪಂಕು-ಗಳೆಲ್ಲಾ
ಡೊರಾಯಿ
ಡೋಬೆ
ಡೋರಾಯ
ಢಾಕಾ-ಪಟ್ಟ-ಣ-ದಲ್ಲಿ
ಢಾಲೆ
ಢೇರ
ತಂ
ತಂಗಬೇ-ಕಾ-ಯಿತಂತೆ
ತಂಗಲು
ತಂಗಾಳಿ
ತಂಟೆಗೆ
ತಂಡ
ತಂಡ-ಗಳು
ತಂಡದ
ತಂಡ-ದಲ್ಲಿ
ತಂಡ-ವಿಲ್ಲ
ತಂಡವೇ
ತಂತಿ
ತಂತಿಯ
ತಂತಿ-ಯನ್ನು
ತಂತಿ-ಸಮಾ-ಚಾರ
ತಂತ್ರ
ತಂತ್ರ-ಗಳೊ
ತಂತ್ರ-ಗಾರನೇ
ತಂತ್ರದ
ತಂತ್ರ-ದಲ್ಲಿ
ತಂತ್ರ-ದಲ್ಲಿನ
ತಂತ್ರ-ದಲ್ಲಿಯೂ
ತಂತ್ರ-ದಲ್ಲಿ-ರುವ
ತಂತ್ರ-ವೆಲ್ಲಾ
ತಂತ್ರಿ
ತಂದ
ತಂದಂತಹ
ತಂದನು
ತಂದ-ರಳಿ-ಸಿದೆ
ತಂದ-ರೀತಿ
ತಂದರು
ತಂದರೆ
ತಂದ-ರೆಂದರೆ
ತಂದ-ವ-ನಿಗೆ
ತಂದ-ವನು
ತಂದ-ವರೇ
ತಂದಾಗ
ತಂದಿಡುತ್ತಾರೆ
ತಂದಿಡು-ವುದು
ತಂದಿತು
ತಂದಿದ್ದ-ವ-ರೊಡನೆ
ತಂದಿದ್ದಾರೆ
ತಂದಿದ್ದೀ-ಯಷ್ಟೇ
ತಂದಿದ್ದೇನೆ
ತಂದಿ-ರುವರು
ತಂದು
ತಂದುಕೊ
ತಂದು-ಕೊಂಡಿದ್ದಾನೆ
ತಂದು-ಕೊಂಡಿರು-ವರು
ತಂದು-ಕೊಂಡು
ತಂದು-ಕೊಟ್ಟ
ತಂದು-ಕೊಟ್ಟನು
ತಂದು-ಕೊಟ್ಟಿದ್ದಾರೆ
ತಂದು-ಕೊಡ-ಬೇಕು
ತಂದು-ಕೊ-ಡಲ್ಲು-ದೆಂಬು-ದರ
ತಂದು-ಕೊಡು
ತಂದು-ಕೊಳ್ಳ-ಬೇಕು
ತಂದು-ಕೊಳ್ಳಲು
ತಂದು-ಕೊಳ್ಳಿ
ತಂದು-ಕೊಳ್ಳುತ್ತೇನೆ
ತಂದು-ಕೊಳ್ಳು-ವುದಕ್ಕೋಸ್ಕರ
ತಂದು-ಕೊಳ್ಳುವುದು
ತಂದುದೂ
ತಂದು-ಬಿಡುತ್ತದೆ
ತಂದು-ಹಾಕಿ-ಬಿಡು
ತಂದೆ
ತಂದೆ-ಗಳು
ತಂದೆಗೆ
ತಂದೆ-ತಾಯಿ-ಗಳ
ತಂದೆ-ತಾಯಿ-ಗಳಿಗೆ
ತಂದೆ-ತಾಯಿ-ಗಳು
ತಂದೆ-ಯನ್ನು
ತಂದೆ-ಯಾಗಲೀ
ತಂದೆ-ಯಾಗಿ-ರುತ್ತಾರೆ
ತಂದೆ-ಯಾ-ಗು-ವು-ದಿಲ್ಲ
ತಂದೆಯು
ತಂದೆಯೂ
ತಂದೆಯೇ
ತಂದೆ-ಯೊ-ಡನೆ
ತಂದೇ
ತಂಪನು
ತಂಪನೆರೆಯು-ತಿ-ರಲು
ತಂಪಾಗಿ-ಸುವಳು
ತಂಪೆದೆಯ-ವಳು
ತಂಬಾ-ಕನ್ನು
ತಂಬೂರಿ
ತಂಬೂರಿಯ
ತಂಬೆಳ-ಕಿನ
ತಂಬೆಳಕು
ತಕ್ಕ
ತಕ್ಕಂತಿರ-ಬೇಕು
ತಕ್ಕಂತೆ
ತಕ್ಕ-ಡಿಯಲ್ಲಿಟ್ಟು
ತಕ್ಕ-ಮಟ್ಟಿಗೆ
ತಕ್ಕ-ವ-ರಾಗುವರು
ತಕ್ಕಷ್ಟು
ತಕ್ಷಣ
ತಕ್ಷಣವೆ
ತಕ್ಷಣವೇ
ತಗಲಿ
ತಗಾದೆ
ತಗ್ಗಿಸಿ-ಕೊಂಡು
ತಗ್ಗಿ-ಸಿದ
ತಗ್ಗಿಸೆಂದು
ತಗ್ಗಿ-ಹುದು
ತಚನಾ
ತಟಪ್ರ-ದೇಶವು
ತಟಾಯಿ-ಸಿದ
ತಟೆ
ತಟ್ಟದೊ
ತಟ್ಟನೆ
ತಟ್ಟಲು
ತಟ್ಟಿ-ದಾಗ
ತಟ್ಟು-ವರು
ತಟ್ಟು-ವು-ದರ
ತಟ್ಟು-ವುದು
ತಟ್ಟೆ
ತಟ್ಟೆ-ಯಲ್ಲಿ
ತಠಂಗೆ
ತಡ
ತಡಕಾಟ-ವಲ್ಲ
ತಡ-ಕಾಡು-ವವರು
ತಡ-ದಿರು
ತಡ-ವಾದರೆ
ತಡೆ
ತಡೆ-ಗಟ್ಟಿವೆ
ತಡೆ-ಗಟ್ಟುವ
ತಡೆ-ಗಟ್ಟುವುದೇ
ತಡೆ-ದರು
ತಡೆ-ದಾಗ
ತಡೆದು
ತಡೆ-ದು-ಕೊಳ್ಳುವುದು
ತಡೆ-ದುಕೋ
ತಡೆ-ಯ-ದಲೆ
ತಡೆ-ಯ-ದ-ವನು
ತಡೆ-ಯ-ಬಲ್ಲ-ವ-ರಾರು
ತಡೆ-ಯ-ಬೇಡ
ತಡೆ-ಯಲ-ಸಾಧ್ಯ-ವಾದ
ತಡೆ-ಯಾಗುತ್ತಿತ್ತು
ತಡೆ-ಯಿಲ್ಲದ
ತಡೆ-ಯಿಲ್ಲದೆ
ತಡೆ-ಯುವತಿ
ತಡೆ-ಯು-ವುದಕ್ಕಾ-ಗು-ವು-ದಿಲ್ಲ
ತಡೆ-ಯು-ವುದು
ತಡೆವ
ತಣಿ-ಸಲು
ತಣಿಸೆ
ತಣ್ಣಗಾಗು-ವುವು
ತಣ್ಣ-ಗಾ-ಯಿತು
ತತ
ತತ್
ತತ್ಕಾಲ-ದಲ್ಲಿ
ತತ್ಕಾಲೀನ
ತತ್ತರಿ-ಸುತ-ಲಿರೆ
ತತ್ತರಿ-ಸುವ
ತತ್ತ್ವ
ತತ್ತ್ವಕ್ಕೆ
ತತ್ತ್ವ-ಗಳನ್ನು
ತತ್ತ್ವ-ಗಳನ್ನೂ
ತತ್ತ್ವ-ಗ-ಳಲ್ಲೂ
ತತ್ತ್ವ-ಗಳಿಗೆ
ತತ್ತ್ವ-ಗಳು
ತತ್ತ್ವಜ್ಞಾನ-ಗಳೂ
ತತ್ತ್ವಜ್ಞೇರ
ತತ್ತ್ವದ
ತತ್ತ್ವ-ದರ್ಶಿ-ಗಳೆಂಬರು
ತತ್ತ್ವ-ದಲ್ಲಿ
ತತ್ತ್ವದಿ
ತತ್ತ್ವ-ದಿಂದ
ತತ್ತ್ವ-ಭಾಗ-ಗಳು
ತತ್ತ್ವ-ಮಸಿ
ತತ್ತ್ವಮ್
ತತ್ತ್ವ-ವನ್ನು
ತತ್ತ್ವ-ವಿದು
ತತ್ತ್ವವು
ತತ್ತ್ವ-ವೊಂದನ್ನೇ
ತತ್ತ್ವ-ಶಾಸ್ತ್ರ
ತತ್ತ್ವ-ಶಾಸ್ತ್ರ-ಗಳನ್ನು
ತತ್ತ್ವ-ಶಾಸ್ತ್ರದ
ತತ್ತ್ವ-ಶಾಸ್ತ್ರ-ವನ್ನು
ತತ್ತ್ವಾ-ಧಾರ
ತತ್ತ್ವಾ-ಧಾರವೋ
ತತ್ತ್ವಾನ್ವೇಷಣ
ತತ್ತ್ವಾನ್ವೇಷಾತ್
ತತ್ಪರ-ರಾಗಿ
ತತ್ಪರ-ರಾಗಿದ್ದಾರೆ
ತತ್ಪೂರ್ಣ-ಪಾತ್ರ-ಮಿದಂ
ತತ್ಯಾನ್ವೇಷಿ-ಗಳಾಗಿ
ತತ್ಯಾನ್ವೇಷಿ-ಗಳಾಗಿಯೂ
ತತ್ರ
ತತ್ಕ್ಷಣ
ತಥ-ದುಃಖ
ತಥಾ
ತದ-ನಂತರ
ತದ-ನಂತರದ
ತದಸ್ಯಾರ್
ತದ-ಹರೇವ
ತದ-ಹಹರೇವ
ತದ್ವಜ್ಜೀವಿ-ತ-ಮತಿ-ಶಯ-ಚಪಲಂ
ತನಕ
ತನ-ಕವೂ
ತನ-ಗಾಗಿ
ತನ-ಗಿಷ್ಟ-ಬಂದದ್ದನ್ನೆಲ್ಲಾ
ತನಗೆ
ತನ-ವನ್ನೆಂದು
ತನು-ಮನ
ತನು-ಮನ-ವನ್ನು
ತನುವು
ತನ್ನ
ತನ್ನ-ದನ್ನಾಗಿ
ತನ್ನದು
ತನ್ನದೇ
ತನ್ನನ್ನು
ತನ್ನನ್ನೇ
ತನ್ನಲ್ಲಿ
ತನ್ನಲ್ಲುಂಟಾದ
ತನ್ನಲ್ಲೇ
ತನ್ನಷ್ಟಕ್ಕೆ
ತನ್ನಷ್ಟಕ್ಕೇ
ತನ್ನಿ
ತನ್ನಿಂದಲೇ
ತನ್ನೆಡೆ
ತನ್ನೆ-ಡೆಗೆ
ತನ್ನೆಲ್ಲ
ತನ್ನೊಡ-ನಿದ್ದ
ತನ್ಮಯ-ನಾಗಿ
ತನ್ಮಯ-ನಾಗಿ-ಬಿಡು
ತನ್ಮಯ-ನಾಗಿ-ಹೋ-ಗದೆ
ತನ್ಮಯ-ರಾಗಿ
ತನ್ಮಯ-ರಾಗಿದ್ದಾರೆ
ತನ್ಮಯ-ರಾಗುತ್ತಿದ್ದರೋ
ತನ್ಮಯ-ರಾಗುವೆವೊ
ತನ್ಮಯ-ವಾಗಿದ್ದರು
ತನ್ಮಯ-ವಾಗಿ-ಬಿಟ್ಟಿತು
ತನ್ಮಯ-ವಾಗು-ವುದೋ
ತನ್ಮಯ-ವಾದಾಗ
ತನ್ಮ-ಯಾವಸ್ಥೆ-ಯಲ್ಲಿ-ರುವಂತೆ
ತನ್ಮಾತ್ರ-ದಂತೆಯೂ
ತಪ
ತಪತಿ
ತಪದ
ತಪನ
ತಪನಜ್ವಾಲ
ತಪಶ್ಚರ್ಯ-ದಲ್ಲಿ
ತಪಶ್ಚರ್ಯೆಯಿಂದ
ತಪಸೋ
ತಪಸ್ಯ-ಕಠೋರ
ತಪಸ್ವಿನೀ-ಮಾತೆ
ತಪಸ್ಸನ್ನಾ-ಚರಿ-ಸು-ವುದೂ
ತಪಸ್ಸನ್ನು
ತಪಸ್ಸಲ್ಲ-ವೆಂಬುದು
ತಪಸ್ಸಾ-ಗುತ್ತದೆ
ತಪಸ್ಸಿನ
ತಪಸ್ಸಿನಿಂದ
ತಪಸ್ಸಿಲ್ಲದೆ
ತಪಸ್ಸು
ತಪಸ್ಸೂ
ತಪೋ-ವನಂ
ತಪ್ತ
ತಪ್ತ-ರಾ-ದ-ವ-ರಿಗೆ
ತಪ್ತಶ್ವಾಸ
ತಪ್ಪ-ದಿಹ
ತಪ್ಪದೆ
ತಪ್ಪನು
ತಪ್ಪನ್ನು
ತಪ್ಪಲಲಿ
ತಪ್ಪಲೆ
ತಪ್ಪಲೆ-ಯಿಟ್ಟಿಲ್ಲ
ತಪ್ಪಲ್ಲ
ತಪ್ಪಿ
ತಪ್ಪಿ-ಗಾಗಿ
ತಪ್ಪಿಗೆ
ತಪ್ಪಿ-ದಂತಾಗಿ
ತಪ್ಪಿ-ರ-ಬ-ಹುದು
ತಪ್ಪಿ-ಸಲು
ತಪ್ಪಿಸಿ
ತಪ್ಪಿ-ಸಿ-ಕೊಂಡು
ತಪ್ಪಿ-ಸಿ-ಕೊಳ್ಳ-ಬೇಕು
ತಪ್ಪಿ-ಸಿ-ಕೊಳ್ಳು-ವಂತಿಲ್ಲ
ತಪ್ಪಿಸು
ತಪ್ಪಿ-ಸು-ವುದಕ್ಕೆ
ತಪ್ಪಿ-ಹೋಗು-ವು-ದೆಂದು
ತಪ್ಪಿ-ಹೋಗು-ವುದೋ
ತಪ್ಪು
ತಪ್ಪು-ತಪ್ಪಾಗಿ
ತಪ್ಪು-ತಿಳಿವೆ
ತಪ್ಪು-ವು-ದಿಲ್ಲ
ತಪ್ಪು-ವುದು
ತಪ್ಪೆಂದು
ತಪ್ಪೆನ್ನಲು
ತಪ್ಪೇ
ತಪ್ಪೇನೂ
ತಪ್ಪೋ
ತಬು-ನಹಿ
ತಬ್ಬಲೆಂದು
ತಬ್ಬು-ತಿಹ
ತಬ್ಬುವ
ತಬ್ಬು-ವಂತಹ
ತಬ್ಬುವ-ರೆಗೂ
ತಮ-ಗಾದ
ತಮ-ಗಿದ್ದ
ತಮ-ಗಿಷ್ಟ-ಬಂದು-ದನ್ನು
ತಮ-ಗಿಷ್ಟ-ವಿ-ರಲಿ
ತಮಗೂ
ತಮಗೆ
ತಮಟೆ
ತಮ-ಟೆ-ಗಳನ್ನು
ತಮಸ್ಸು
ತಮ-ಹಹ
ತಮಾಷೆ
ತಮಾಷೆ-ಯನ್ನು
ತಮಾಷೆ-ಯಾಗಿ
ತಮೇವ
ತಮೋ-ಗುಣ-ದಲ್ಲಿ
ತಮೋ-ಗುಣ-ದಿಂದ
ತಮ್ಮ
ತಮ್ಮಂಥ-ವರು
ತಮ್ಮಂದಿ-ರಾದ
ತಮ್ಮಂದಿರು
ತಮ್ಮ-ದೆಲ್ಲಾ
ತಮ್ಮದೇ
ತಮ್ಮನ್ನು
ತಮ್ಮನ್ನೇ
ತಮ್ಮಲ್ಲಿ
ತಮ್ಮಲ್ಲಿಟ್ಟು-ಕೊಂಡು
ತಮ್ಮಲ್ಲಿ-ರುವ
ತಮ್ಮಲ್ಲೇ
ತಮ್ಮಷ್ಟಕ್ಕೆ
ತಮ್ಮಿಚ್ಛೆ
ತಮ್ಮೆದುರು
ತಮ್ಮೆಲ್ಲ
ತಮ್ಮೊ-ಡನೆ
ತಮ್ಮೊಲುಮೆ
ತಯಾರಾ-ಗ-ಬಲ್ಲರು
ತಯಾರಾ-ಗ-ಬೇಕಾಗಿದೆ
ತಯಾರಾ-ಗು-ವು-ದಿಲ್ಲ
ತಯಾರಾ-ದರೆ
ತಯಾರಿ-ಸಲ್ಪಟ್ಟ
ತಯಾರಿ-ಸಿದ್ದರು
ತಯಾರಿ-ಸಿ-ರು-ವಿರಿ
ತಯಾರಿ-ಸು-ವು-ದಿಲ್ಲವೆ
ತಯಾರಿ-ಸು-ವುದು
ತಯಾರು
ತಯಾರು-ಮಾಡ-ಬೇಕು
ತಯಾರು-ಮಾ-ಡಲು
ತಯಾರು-ಮಾಡಿ
ತಯಾರು-ಮಾಡು-ವುದು
ತರಂಗ
ತರಂಗಕೆ
ತರಂಗ-ಗಳಲ್ಲಿ
ತರಂಗ-ಗಳಾಗಿ
ತರಂಗ-ಗಳಿಂದ
ತರಂಗ-ಗಳು
ತರಂಗ-ಗಳೇ-ಳುತ್ತವೆ
ತರಂಗ-ಮಾಲೆ-ಯಂತೆ
ತರಂಗ-ಲೀ-ಲಾರ
ತರಂಗವು
ತರಂಗವೂ
ತರ-ಕಾರಿ
ತರ-ಕಾರಿ-ಗಳನ್ನು
ತರ-ಕಾರಿ-ಯಲ್ಲಿ
ತರ-ಗತಿ-ಗಳನ್ನೂ
ತರ-ಗತಿ-ಗಳಿಗೂ
ತರ-ಗ-ತಿಗೂ
ತರ-ಗ-ತಿಗೆ
ತರ-ಗ-ತಿಯ
ತರ-ಗತಿ-ಯನ್ನು
ತರ-ಗೆಲೆಯ
ತರ-ಗೆಲೆ-ಯಾಗಿ
ತರ-ತಮ-ದಲ್ಲಿ-ರ-ಬೇಕು
ತರದ
ತರದಿ-ರ-ಬ-ಹುದು
ತರ-ಪೇ-ತಾಗು-ವರು
ತರ-ಪೇತು
ತರ-ಬಲ್ಲೆಯಾ
ತರ-ಬ-ಹುದು
ತರ-ಬೇ-ಕಾದರೆ
ತರ-ಬೇಕು
ತರ-ಬೇಕೆಂದಿದ್ದೀರಿ
ತರ-ಬೇ-ತಾದ
ತರ-ಬೇತಿ
ತರ-ಬೇತಿ-ಗಾಗಿ
ತರ-ಬೇತಿ-ಯನ್ನು
ತರ-ಬೇತು
ತರ-ಲಾಗಿತ್ತು
ತರ-ಲಾ-ಗು-ವು-ದಿಲ್ಲ-ವೆಂಬುದು
ತರ-ಲಾರಿರಿ
ತರಲಿ
ತರ-ಲಿಲ್ಲ
ತರಲು
ತರ-ವಲ್ಲ
ತರಸಾ
ತರಹ
ತರಿಸಿ
ತರಿಸಿದ್ದರು
ತರಿ-ಸು-ವುದಕ್ಕೋಸ್ಕರ
ತರುಣ
ತರು-ಣ-ತರು-ಣಿ-ಯರು
ತರು-ಣನ
ತರು-ಣನು
ತರು-ಣ-ಭಾಸ್ಕರ
ತರುಣಿ
ತರುತ್ತಿತ್ತಂತೆ
ತರುತ್ತಿದ್ದರು
ತರುತ್ತೀರೋ
ತರು-ಲತಾ
ತರು-ಲ-ತೆ-ಗಳನ್ನುಟ್ಟು
ತರುವ
ತರು-ವಂತೆ
ತರುವರು
ತರುವಾಯ
ತರು-ವಿರಿ
ತರು-ವುದಕ್ಕೆ
ತರುವುದನೆ
ತರು-ವು-ದಿಲ್ಲ
ತರುವುದಿ-ವ-ರಿಗೆಂದೂ
ತರು-ವುದು
ತರು-ವುದೇ
ತರೋಣ
ತರ್ಕ
ತರ್ಕ-ಗಳ
ತರ್ಕ-ಗಳಿಗೆ
ತರ್ಕ-ಗಳೆಲ್ಲ
ತರ್ಕದ
ತರ್ಕ-ದಲ್ಲಿ
ತರ್ಕ-ದಿಂದ
ತರ್ಕ-ವದು
ತರ್ಕ-ಶಾಸ್ತ್ರ
ತರ್ಕಾವಸ್ಥೆ-ಯಲ್ಲೇ
ತರ್ಕೇಣ
ತರ್ಪಣ
ತಲ-ತಲಾಂತರ-ದಿಂದ
ತಲಾ-ತಲ
ತಲುಪ-ದಿ-ರು-ವುದು
ತಲುಪಲು
ತಲುಪಿತು
ತಲುಪಿ-ದರು
ತಲುಪಿ-ದರೂ
ತಲುಪಿ-ದೆವು
ತಲುಪಿ-ದೊಡ-ನೆಯೇ
ತಲುಪಿಸು
ತಲುಪುತ್ತಾನೆ
ತಲುಪು-ವುದು
ತಲೆ
ತಲೆ-ಕೂದ-ಲಿನಲ್ಲಿಯೂ
ತಲೆ-ಕೆಡಿಸಿ-ಕೊಳ್ಳುವರು
ತಲೆ-ಕೆಡಿಸಿ-ಕೊಳ್ಳು-ವು-ದಿಲ್ಲ
ತಲೆ-ಕೆಳ-ಗಾ-ದರೆ
ತಲೆ-ಕೆಳಗು
ತಲೆ-ಗಳಿಂದ
ತಲೆ-ಗಳು
ತಲೆಗೆ
ತಲೆ-ತಗ್ಗಿಸಿ-ಕೊಂಡು
ತಲೆ-ತಗ್ಗಿ-ಸಿರಲು
ತಲೆ-ದೂಗು-ವೆನು
ತಲೆ-ದೋ-ರು-ವುದುಂಟು
ತಲೆ-ದೋ-ರು-ವುವು
ತಲೆ-ಬಾಗಿ-ದ-ವ-ನಲ್ಲ
ತಲೆ-ಬಾಗುವುದಕ್ಕಿಂತ
ತಲೆಯ
ತಲೆ-ಯನ್ನು
ತಲೆ-ಯನ್ನೆಲ್ಲಾ
ತಲೆ-ಯ-ಮೇಲೆ
ತಲೆ-ಯಲ್ಲಿ
ತಲೆ-ಯ-ವರೆ-ಗಿನ
ತಲೆಯು
ತಲೆ-ಯು-ರುಳಿ-ಸಲು
ತಲೆ-ಯುಳ್ಳ
ತಲೆ-ಯೆತ್ತಿ
ತಲೆ-ಯೆತ್ತಿ-ಕೊಂಡಿವೆ
ತಲೆ-ಯೆತ್ತಿ-ದುವು
ತಲೆ-ಯೆತ್ತುತ್ತಿದ್ದವು
ತಲೆ-ಯೆತ್ತುತ್ತಿದ್ದು-ದನ್ನು
ತಲ್ಲ-ಣಿಸಿ
ತಲ್ಲೀನ-ರಾದ
ತಲ್ಲೀನ-ವಾಗಲು
ತಳಕೊತ್ತುತ್ತ
ತಳಮಳ-ಗೊಂಡಿದೆ
ತಳಹದಿ
ತಳಹದಿ-ಯನ್ನು
ತಳಿರು
ತಳೆದ
ತಳೆದು
ತಳ್ಳಿ
ತಳ್ಳಿ-ಬಿಟ್ಟರೆ
ತಳ್ಳು
ತಳ್ಳುತ
ತಳ್ಳು-ವನು
ತಳ್ಳು-ವಾ-ಸೆಯೆ
ತವ
ತವ-ಕ-ದಲ್ಲೇ
ತವ-ಕ-ಪಡುತ್ತಿದ್ದರು
ತವ-ಕಿ-ಸಿ-ರುವೆ
ತವ-ಕಿ-ಸುತ್ತಿ-ರ-ಬೇಕು
ತವ-ಗತಿ
ತವ-ದಯಾ
ತವ-ನಾಮ
ತವ-ಪದಂ
ತವ-ಪಾದ-ಪದ್ಮಂ
ತವ-ಪಾವೆ
ತವರು-ಮನೆ-ಯಲ್ಲಿ
ತವ-ರೂಪ
ತವಸ್ಥಾನ
ತವು
ತಸ್ಮಾತ್
ತಸ್ಮಿನ್
ತಸ್ಯ
ತಹನಾ
ತಹೆ
ತಾ
ತಾಂ
ತಾಂಡವ
ತಾಂತ್ರಿಕ
ತಾಂತ್ರಿಕ-ರೆಂದೂ
ತಾಕಿರೆ
ತಾಕಿಸಯ
ತಾಕುವ
ತಾಕು-ವು-ದಿಲ್ಲ
ತಾಕೆ
ತಾಗ-ಮಾಡಿದ್ದೇವೆ
ತಾಗುತಿವೆ
ತಾಜ್
ತಾಜ್ಮಹ-ಲನ್ನು
ತಾಟಸ್ಥ್ಯವೆನ್ನು-ವುದು
ತಾಡಾಯ
ತಾಣ
ತಾಣ-ದೆಡೆಗೆ
ತಾಣ-ವಾಗಿಯೇ
ತಾತ
ತಾತ-ನಾಗಲೀ
ತಾತನು
ತಾತ್ಕಾಲಿಕ
ತಾತ್ತ್ವಿ-ಕ-ರಿಗೆ
ತಾತ್ತ್ವಿ-ಕ-ವಾಗಿ
ತಾತ್ಪರ್ಯ-ವನ್ನು
ತಾತ್ಸಾರ
ತಾಥೈಯಾ
ತಾನ
ತಾನ-ತರಂಗಿಣಿ
ತಾನ-ದುವೆ
ತಾನಲ್ಲ
ತಾನಲ್ಲಿಲ್ಲ
ತಾನ-ಳಿ-ಯಲಿ
ತಾನಾಗಿ
ತಾನಾಗಿಯೇ
ತಾನಿನ್ನು
ತಾನು
ತಾನು-ಮತ್ತೇನು
ತಾನೂ
ತಾನೆ
ತಾನೆಂಬ
ತಾನೆನ್ನ
ತಾನೆಲ್ಲ-ವೊಂದೆ
ತಾನೆಲ್ಲಿಯೂ
ತಾನೆ-ಳ-ಸುವನು
ತಾನೇ
ತಾನೊಂದೆ
ತಾನೊಬ್ಬಳೆ
ತಾನ್
ತಾಪ
ತಾಪತ್ರ-ಯ-ವನ್ನು
ತಾಪ-ದಲ್ಲಿ
ತಾಪ-ಸಿ-ಗಳ
ತಾಪಿ
ತಾಮಸ
ತಾಮ-ಸಕ್ಕೆ
ತಾಮಸ-ಗು-ಣಕ್ಕೆ
ತಾಮ-ಸದ
ತಾಮಸ-ದಲ್ಲಿ
ತಾಮಸ-ದಿಂದ
ತಾಮಸ-ವನ್ನು
ತಾಮಸವೇ
ತಾಮಸಿಕ
ತಾಮ್ರ
ತಾಯ
ತಾಯಂದಿ-ರಂತೆ
ತಾಯಾಗಿ
ತಾಯಿ
ತಾಯಿ-ಗಳಿಂದ
ತಾಯಿಗೆ
ತಾಯಿ-ತಂದೆ-ಗಳ
ತಾಯಿ-ತಂದೆ-ಗಳಿಂದ
ತಾಯಿ-ನಾಡಿನ
ತಾಯಿಯ
ತಾಯಿ-ಯಂತೆ
ತಾಯಿ-ಯಾಗಿ
ತಾಯಿ-ಯಾದ-ವಳು
ತಾಯಿಯು
ತಾಯಿಯೆ
ತಾಯಿ-ಯೆಂದು
ತಾಯಿ-ಯೆಂಬು-ದನ್ನು
ತಾಯಿಲ್ಲ
ತಾಯೆ
ತಾಯೇ
ತಾಯೊಲುಮೆ
ತಾಯ್ತಂದೆ
ತಾಯ್ತಂದೆ-ಗಳು
ತಾಯ್ನಾಡನು
ತಾಯ್ನಾಡಿನ
ತಾಯ್ನೆಲ-ದಿಂದ
ತಾರ
ತಾರಕೆ
ತಾರಣ
ತಾರ-ತಮ್ಯ
ತಾರ-ತಮ್ಯ-ದಿಂದಲೇ
ತಾರ-ತಮ್ಯ-ವನ್ನೇ
ತಾರ-ತಮ್ಯ-ವಿದೆ
ತಾರ-ತಮ್ಯ-ವಿ-ರುತ್ತದೆ
ತಾರ-ತಮ್ಯ-ವಿಲ್ಲದೆ
ತಾರ-ತಮ್ಯವು
ತಾರದೆ
ತಾರಲ್ಯೇರ್
ತಾರಾ
ತಾರಾ-ಡಿದ
ತಾರಿ
ತಾರಿಧ್ವಜ
ತಾರೀ
ತಾರುಣ್ಯ-ದಲ್ಲಿದೆ
ತಾರುಣ್ಯಾವಸ್ಥೆಯಲ್ಲಿದೆ
ತಾರೆ-ಗಳ
ತಾರೆ-ಗಳವು
ತಾರೆ-ಗಳು
ತಾರ್ಕಿಕ
ತಾರ್ಕಿಕ-ವಾಗಿ
ತಾಲ
ತಾಳಕ್ಕೆ
ತಾಳಗೊಳಲು
ತಾಳಲಹುದೆ
ತಾಳ-ಲಾರೆನು
ತಾಳಿ
ತಾಳಿ-ಬ-ರುವನು
ತಾಳು
ತಾಳು-ವುದು
ತಾಳ್ಮೆ
ತಾಳ್ಮೆ-ಯಿಂದ
ತಾಳ್ಮೆ-ಯಿಂದಿಡುವ
ತಾವ-ರಿಯ-ಬ-ಹುದು
ತಾವ-ರಿ-ಯರು
ತಾವರೆ
ತಾವ-ರೆ-ಗಳು
ತಾವ-ರೆಯ
ತಾವಾಗಿಯೇ
ತಾವಾರು
ತಾವಿಚ್ಛಿ-ಸಿದ
ತಾವಿದ್ದ
ತಾವಿಲ್ಲಿ
ತಾವೀ-ರೀತಿ
ತಾವು
ತಾವೂ
ತಾವೆ
ತಾವೆನ್ನ
ತಾವೆಲ್ಲ
ತಾವೆಲ್ಲರೂ
ತಾವೆಲ್ಲವೂ
ತಾವೆಷ್ಟು
ತಾವೇ
ತಾವೇನು
ತಾವೇನೊ
ತಾಹತೆ
ತಾಹಾತೆ
ತಾಹಾತೇ
ತಾಹಾರ
ತಾಹೇ
ತಿಂಗ-ಳಲ್ಲಿ
ತಿಂಗ-ಳಲ್ಲೇ
ತಿಂಗಳ-ವರೆಗೆ
ತಿಂಗ-ಳಾಗಿತ್ತು
ತಿಂಗಳಿ-ಗಿಂತ
ತಿಂಗಳಿಗೆ
ತಿಂಗ-ಳಿನ
ತಿಂಗಳಿ-ನಲ್ಲಿ
ತಿಂಗಳು
ತಿಂಗಳು-ಗಟ್ಟಲೆ
ತಿಂಗಳು-ಗಳು
ತಿಂಡಿ
ತಿಂಡಿ-ಯನ್ನು
ತಿಂದ
ತಿಂದನು
ತಿಂದರೂ
ತಿಂದರೆ
ತಿಂದರೇ
ತಿಂದರೊ
ತಿಂದ-ಹಾ-ಗಾ-ಯಿತು
ತಿಂದಾಗ
ತಿಂದಿ-ರ-ಲಿಲ್ಲ
ತಿಂದು
ತಿಂದು-ತೇಗುವ
ತಿಂದು-ಬಿಟ್ಟು
ತಿಂದೆ
ತಿಂದೊ
ತಿಕ್ಕಾಟ-ವಿತ್ತು
ತಿಟ್ಟನ್ನು
ತಿದ್ದಿ-ಕೊಡುತ್ತೇನೆ
ತಿದ್ದಿ-ಕೊಳ್ಳುವಂತೆ
ತಿದ್ದು
ತಿನಿ
ತಿನಿ-ಸಿ-ನಲ್ಲಿ
ತಿನ್ನ-ಬಲ್ಲರು
ತಿನ್ನ-ಬಲ್ಲೆ
ತಿನ್ನ-ಬೇಕಾಗಿಲ್ಲ
ತಿನ್ನ-ಬೇ-ಕಿತ್ತು
ತಿನ್ನ-ಬೇಕು
ತಿನ್ನ-ಬೇಕೆಂದು
ತಿನ್ನ-ಬೇ-ಕೇನು
ತಿನ್ನ-ಲಿಕ್ಕೆ
ತಿನ್ನ-ಲಿಲ್ಲ-ವೆಂದು
ತಿನ್ನಲು
ತಿನ್ನಿ-ಸಿ-ದರು
ತಿನ್ನಿ-ಸು-ವುದಕ್ಕೆ
ತಿನ್ನು
ತಿನ್ನುತ್ತ
ತಿನ್ನುತ್ತಾ
ತಿನ್ನುತ್ತಾರೆ
ತಿನ್ನುತ್ತಿದ್ದರೆ
ತಿನ್ನುತ್ತಿದ್ದೆ
ತಿನ್ನುತ್ತಿ-ರ-ಲಿಲ್ಲ
ತಿನ್ನುತ್ತಿರು-ವೆ-ವೆಂಬು-ದರ
ತಿನ್ನುತ್ತೀ-ರಲ್ಲವೇ
ತಿನ್ನುತ್ತೇವೆಂದು-ಕೊಳ್ಳೋಣ
ತಿನ್ನುವ
ತಿನ್ನು-ವುದಕ್ಕೆ
ತಿನ್ನು-ವು-ದನ್ನೂ
ತಿನ್ನು-ವುದ-ರಲ್ಲಿ
ತಿನ್ನು-ವು-ದ-ರಿಂದ
ತಿನ್ನು-ವು-ದಿ-ರಲಿ
ತಿನ್ನು-ವು-ದಿಲ್ಲ
ತಿನ್ನು-ವು-ದಿಲ್ಲವೋ
ತಿನ್ನು-ವುದು
ತಿನ್ನು-ವುದೇ
ತಿಬೇಟ್
ತಿಮಿರ-ಮಾಲೆ
ತಿರಸ್ಕರಿ-ಸ-ಬ-ಹುದು
ತಿರಸ್ಕರಿ-ಸಲಾ-ಗದ
ತಿರಸ್ಕ-ರಿಸಿ
ತಿರಸ್ಕರಿ-ಸುತ್ತದೆ
ತಿರಸ್ಕರಿ-ಸುತ್ತಿದ್ದೀರೋ
ತಿರಸ್ಕೃತ-ರಾಗಿದ್ದೀರಿ
ತಿರುಕ-ನನ್ನೂ
ತಿರು-ಕರ
ತಿರುಗ
ತಿರುಗ-ಲಾರೆ-ನೆಂದು
ತಿರುಗಾಡಿ-ಕೊಂಡು
ತಿರುಗಾಡುತ್ತ
ತಿರುಗಾಡುತ್ತಾ
ತಿರುಗಾಡುತ್ತಿದ್ದಾರೆ
ತಿರುಗಾಡುತ್ತಿದ್ದು
ತಿರುಗಾಡು-ವುದಕ್ಕೆ
ತಿರುಗಿ
ತಿರುಗಿ-ಕೊಂಡು
ತಿರುಗಿದ
ತಿರುಗಿ-ದರು
ತಿರುಗಿದ್ದಾನೆ
ತಿರುಗಿ-ಸ-ಬಲ್ಲ-ವ-ರಾರು
ತಿರುಗಿ-ಸ-ಬೇ-ಕಾದರೆ
ತಿರುಗಿ-ಸಲ-ಸಮರ್ಥ-ರಾಗುವರು
ತಿರುಗಿ-ಸಲು
ತಿರುಗಿಸಿ
ತಿರುಗಿ-ಸಿ-ಕೊಂಡು
ತಿರುಗಿ-ಸಿ-ದಂತೆ
ತಿರುಗಿ-ಸಿ-ಬಿಟ್ಟಿದ್ದನ್ನು
ತಿರುಗಿ-ಸುತ್ತಾರೆ
ತಿರುಗಿ-ಸುತ್ತಾಳೆ
ತಿರುಗಿ-ಸುತ್ತಿ-ರು-ವುದು
ತಿರುಗಿ-ಸುವ
ತಿರುಗಿ-ಸುವನು
ತಿರುಗಿ-ಸು-ವುದು
ತಿರುಗಿ-ಹೋಗು-ವುದು
ತಿರುಗುತ್ತ
ತಿರು-ಗುತ್ತದೆ
ತಿರುಗುತ್ತ-ದೆಯೋ
ತಿರು-ಗುತ್ತಾರೆ
ತಿರುಗುತ್ತಿದ್ದರು
ತಿರುಗುತ್ತಿದ್ದಾರೆ
ತಿರುಗುತ್ತಿವೆ
ತಿರು-ಗು-ವಂತೆ
ತಿರುಗು-ವರು
ತಿರು-ಗು-ವು-ದಿಲ್ಲ
ತಿರುಗು-ವುದು
ತಿರು-ತಿ-ರುಗಿ
ತಿರುಪೆ-ಗಾಗಿಯೇ
ತಿರು-ಳನ್ನು
ತಿರು-ಳನ್ನೂ
ತಿರುಳಿ-ಗಲ್ಲ
ತಿರುಳು
ತಿರುವಿ-ಹಾಕಿ
ತಿರೆ-ಗೆರೆ-ದಿಹರು
ತಿಳಿ
ತಿಳಿ-ಗೇಡಿ-ತನ
ತಿಳಿದ
ತಿಳಿ-ದಂತಾಯಿತು
ತಿಳಿ-ದಂತೆ
ತಿಳಿ-ದರೆ
ತಿಳಿ-ದ-ವರೆನ್ನಿ-ಸಿ-ಕೊಂಡ-ವರು
ತಿಳಿ-ದಷ್ಟನ್ನು
ತಿಳಿ-ದಾಗ
ತಿಳಿ-ದಿದೆ
ತಿಳಿ-ದಿದ್ದ-ರಿಂದ
ತಿಳಿ-ದಿದ್ದರು
ತಿಳಿ-ದಿದ್ದರೂ
ತಿಳಿ-ದಿದ್ದರೆ
ತಿಳಿ-ದಿದ್ದಿತು
ತಿಳಿ-ದಿದ್ದೀರಿ
ತಿಳಿ-ದಿದ್ದೆ
ತಿಳಿ-ದಿದ್ದೇನೆ
ತಿಳಿ-ದಿದ್ದೇವೆ
ತಿಳಿ-ದಿದ್ದೇವೋ
ತಿಳಿ-ದಿರ-ಬೇಕು
ತಿಳಿ-ದಿರುವ
ತಿಳಿ-ದಿ-ರು-ವಂತೆ
ತಿಳಿ-ದಿ-ರುವರು
ತಿಳಿ-ದಿರು-ವ-ವ-ರಲ್ಲಿ
ತಿಳಿ-ದಿರು-ವಿ-ಯೇನು
ತಿಳಿ-ದಿ-ರು-ವಿರಿ
ತಿಳಿ-ದಿರು-ವಿ-ರೇನು
ತಿಳಿ-ದಿ-ರು-ವೆಯಾ
ತಿಳಿ-ದಿಲ್ಲ
ತಿಳಿ-ದಿಲ್ಲವೆ
ತಿಳಿ-ದಿವೆ
ತಿಳಿದು
ತಿಳಿ-ದುಕೊ
ತಿಳಿ-ದು-ಕೊಂಡ
ತಿಳಿ-ದು-ಕೊಂಡಂತೆ
ತಿಳಿ-ದು-ಕೊಂಡದ್ದು
ತಿಳಿ-ದು-ಕೊಂಡರು
ತಿಳಿ-ದು-ಕೊಂಡರೆ
ತಿಳಿ-ದು-ಕೊಂಡ-ರೆಂದು
ತಿಳಿ-ದು-ಕೊಂಡಲ್ಲಿ
ತಿಳಿ-ದು-ಕೊಂಡ-ವನು
ತಿಳಿ-ದು-ಕೊಂಡಾಗ
ತಿಳಿ-ದು-ಕೊಂಡಾರು
ತಿಳಿ-ದು-ಕೊಂಡಿದ್ದರೂ
ತಿಳಿ-ದು-ಕೊಂಡಿದ್ದರೆ
ತಿಳಿ-ದು-ಕೊಂಡಿದ್ದ-ರೆಂದೂ
ತಿಳಿ-ದು-ಕೊಂಡಿದ್ದಾ-ನೆಯೋ
ತಿಳಿ-ದು-ಕೊಂಡಿದ್ದಾರೆ
ತಿಳಿ-ದು-ಕೊಂಡಿದ್ದಾರೆಯೊ
ತಿಳಿ-ದು-ಕೊಂಡಿದ್ದಾರೆಯೋ
ತಿಳಿ-ದು-ಕೊಂಡಿದ್ದೀಯೆ
ತಿಳಿ-ದು-ಕೊಂಡಿದ್ದೀಯೊ
ತಿಳಿ-ದು-ಕೊಂಡಿದ್ದೀಯೋ
ತಿಳಿ-ದು-ಕೊಂಡಿದ್ದೀರಿ
ತಿಳಿ-ದು-ಕೊಂಡಿದ್ದೆವು
ತಿಳಿ-ದು-ಕೊಂಡಿದ್ದೇನೆ
ತಿಳಿ-ದು-ಕೊಂಡಿದ್ದೇನೆಯೊ
ತಿಳಿ-ದು-ಕೊಂಡಿದ್ದೇ-ವೆಯೊ
ತಿಳಿ-ದು-ಕೊಂಡಿರಿ
ತಿಳಿ-ದು-ಕೊಂಡಿರುವ
ತಿಳಿ-ದು-ಕೊಂಡಿರು-ವಂತೆ
ತಿಳಿ-ದು-ಕೊಂಡಿರು-ವರೋ
ತಿಳಿ-ದು-ಕೊಂಡಿಲ್ಲ
ತಿಳಿ-ದು-ಕೊಂಡು
ತಿಳಿ-ದು-ಕೊಂಡು-ಬಿಟ್ಟರೆ
ತಿಳಿ-ದು-ಕೊಂಡೆ
ತಿಳಿ-ದು-ಕೊಂಡೆಯೊ
ತಿಳಿ-ದು-ಕೊಂಡೆಯೋ
ತಿಳಿ-ದು-ಕೊಳ್ಳದೆ
ತಿಳಿ-ದು-ಕೊಳ್ಳ-ಬಲ್ಲ-ವ-ನಾಗಿದ್ದೆ
ತಿಳಿ-ದು-ಕೊಳ್ಳ-ಬಲ್ಲೆ
ತಿಳಿ-ದು-ಕೊಳ್ಳ-ಬಲ್ಲೆ-ಯಷ್ಟೆ
ತಿಳಿ-ದು-ಕೊಳ್ಳ-ಬಹು-ದೆಂದು
ತಿಳಿ-ದು-ಕೊಳ್ಳ-ಬೇ-ಕಾ-ದು-ದಿಲ್ಲ
ತಿಳಿ-ದು-ಕೊಳ್ಳ-ಬೇಕು
ತಿಳಿ-ದು-ಕೊಳ್ಳ-ಬೇಕೆಂದು
ತಿಳಿ-ದು-ಕೊಳ್ಳ-ಬೇಕೊ
ತಿಳಿ-ದು-ಕೊಳ್ಳ-ಬೇಡ
ತಿಳಿ-ದು-ಕೊಳ್ಳ-ಲಾಗ-ಲಿಲ್ಲ
ತಿಳಿ-ದು-ಕೊಳ್ಳ-ಲಾ-ಗು-ವು-ದಿಲ್ಲ-ವೆಂಬ
ತಿಳಿ-ದು-ಕೊಳ್ಳ-ಲಾರ-ದವ-ರಾಗಿದ್ದರು
ತಿಳಿ-ದು-ಕೊಳ್ಳ-ಲಾರದೆ
ತಿಳಿ-ದು-ಕೊಳ್ಳ-ಲಾರನು
ತಿಳಿ-ದು-ಕೊಳ್ಳ-ಲಾ-ರರು
ತಿಳಿ-ದು-ಕೊಳ್ಳ-ಲಾರರೊ
ತಿಳಿ-ದು-ಕೊಳ್ಳ-ಲಾರೆ
ತಿಳಿ-ದು-ಕೊಳ್ಳಲು
ತಿಳಿ-ದು-ಕೊಳ್ಳಿ
ತಿಳಿ-ದು-ಕೊಳ್ಳುತ್ತಲೂ
ತಿಳಿ-ದು-ಕೊಳ್ಳುತ್ತಾ
ತಿಳಿ-ದು-ಕೊಳ್ಳುತ್ತಾರೆ
ತಿಳಿ-ದು-ಕೊಳ್ಳುವ
ತಿಳಿ-ದು-ಕೊಳ್ಳುವರು
ತಿಳಿ-ದು-ಕೊಳ್ಳು-ವುದಕ್ಕಾ-ಗಲಿಲ್ಲ
ತಿಳಿ-ದು-ಕೊಳ್ಳು-ವುದಕ್ಕಾ-ಗು-ವು-ದಿಲ್ಲ
ತಿಳಿ-ದು-ಕೊಳ್ಳು-ವುದಕ್ಕೆ
ತಿಳಿ-ದು-ಕೊಳ್ಳು-ವುದಕ್ಕೋಸ್ಕರ
ತಿಳಿ-ದು-ಕೊಳ್ಳು-ವು-ದನ್ನು
ತಿಳಿ-ದು-ಕೊಳ್ಳುವುದು
ತಿಳಿ-ದು-ಕೊಳ್ಳೋಣ-ವಾ-ಯಿತು
ತಿಳಿ-ದು-ಬಂತು
ತಿಳಿ-ದು-ಬಂದಿದೆ
ತಿಳಿ-ದು-ಬ-ರುತ್ತದೆ
ತಿಳಿ-ದು-ಬರುತ್ತ-ವೆಯೋ
ತಿಳಿ-ದು-ಬ-ರುವ
ತಿಳಿ-ದು-ಹೋ-ಯಿತು
ತಿಳಿದೂ
ತಿಳಿ-ದೆಯಾ
ತಿಳಿದೇ
ತಿಳಿದೊ
ತಿಳಿಯ
ತಿಳಿ-ಯ-ಗೊಡಿ-ಸು-ವು-ದಿಲ್ಲ
ತಿಳಿ-ಯದ
ತಿಳಿ-ಯ-ದ-ವರು
ತಿಳಿ-ಯ-ದಿದ್ದರೆ
ತಿಳಿ-ಯ-ದಿ-ರಲಿ
ತಿಳಿ-ಯ-ದಿರುವ
ತಿಳಿ-ಯ-ದಿ-ರು-ವುದು
ತಿಳಿ-ಯ-ದಿ-ರು-ವೆನು
ತಿಳಿ-ಯ-ದಿಹ
ತಿಳಿ-ಯದು
ತಿಳಿ-ಯದೆ
ತಿಳಿ-ಯ-ದೆಯೊ
ತಿಳಿ-ಯದೊ
ತಿಳಿ-ಯ-ಪಡಿಸಿ
ತಿಳಿ-ಯ-ಬಲ್ಲ
ತಿಳಿ-ಯ-ಬಲ್ಲರು
ತಿಳಿ-ಯ-ಬಲ್ಲೆ
ತಿಳಿ-ಯ-ಬಹು-ದೆಂದರೆ
ತಿಳಿ-ಯ-ಬೇಕು
ತಿಳಿ-ಯ-ಬೇಕೆಂದು
ತಿಳಿ-ಯ-ಬೇಕೆಂಬ
ತಿಳಿ-ಯ-ಬೇಡ
ತಿಳಿ-ಯ-ಬೇಡಿ
ತಿಳಿ-ಯ-ಲ-ಹುದು
ತಿಳಿ-ಯ-ಲಾ-ರರು
ತಿಳಿ-ಯಲಿ
ತಿಳಿ-ಯಲು
ತಿಳಿ-ಯಾದ
ತಿಳಿ-ಯಿತು
ತಿಳಿ-ಯಿತೆ
ತಿಳಿ-ಯಿತೇ
ತಿಳಿ-ಯಿರಿ
ತಿಳಿ-ಯುತ್ತ
ತಿಳಿ-ಯುತ್ತದೆ
ತಿಳಿ-ಯುತ್ತ-ದೆಯೋ
ತಿಳಿ-ಯುತ್ತಾನೆ
ತಿಳಿ-ಯುತ್ತಿದ್ದರು
ತಿಳಿ-ಯುತ್ತಿದ್ದರೋ
ತಿಳಿ-ಯುತ್ತೇನೆ
ತಿಳಿ-ಯುವ
ತಿಳಿ-ಯು-ವಂತೆ
ತಿಳಿ-ಯುವರೊ
ತಿಳಿ-ಯುವ-ವ-ರೆಗೂ
ತಿಳಿ-ಯುವಾ-ಸೆಯು
ತಿಳಿ-ಯುವಿ
ತಿಳಿ-ಯುವಿರೋ
ತಿಳಿ-ಯು-ವುದಕ್ಕೆ
ತಿಳಿ-ಯು-ವು-ದ-ರಿಂದ
ತಿಳಿ-ಯು-ವು-ದಿಲ್ಲ
ತಿಳಿ-ಯು-ವುದು
ತಿಳಿ-ಯು-ವುದೇ
ತಿಳಿ-ಯು-ವುದೊ
ತಿಳಿ-ಯು-ವುದೋ
ತಿಳಿ-ಯು-ವುವು
ತಿಳಿ-ಯುವೆ
ತಿಳಿ-ಯು-ವೆವು
ತಿಳಿ-ಯು-ವೆವೊ
ತಿಳಿ-ಯೆಯಾ
ತಿಳಿ-ವ-ಳಿಕೆ
ತಿಳಿ-ವ-ಳಿಕೆ-ಯುಂಟಾಗುತ್ತದೋ
ತಿಳಿ-ವ-ಳಿಕೆಯೂ
ತಿಳಿ-ವಾಗಿ-ರು-ವನೊ
ತಿಳಿ-ವಿಗೆ-ಳ-ಸುವ
ತಿಳಿವು
ತಿಳಿ-ವೆಲ್ಲ-ವನು
ತಿಳಿ-ಸ-ಬಾ-ರದೆಂದಿದ್ದರೊ
ತಿಳಿ-ಸ-ಬಾ-ರದೇಕೆ
ತಿಳಿ-ಸ-ಬೇಕು
ತಿಳಿ-ಸ-ಬೇಕೆಂದು
ತಿಳಿ-ಸ-ಲಾರದೆ
ತಿಳಿ-ಸಲು
ತಿಳಿ-ಸಲೂ
ತಿಳಿ-ಸ-ಲೇನು
ತಿಳಿಸಿ
ತಿಳಿ-ಸಿ-ಕೊಟ್ಟರು
ತಿಳಿ-ಸಿ-ಕೊಟ್ಟು
ತಿಳಿ-ಸಿ-ಕೊಟ್ಟೆ-ಯೆಂದರೆ
ತಿಳಿ-ಸಿ-ಕೊಡ-ದಿದ್ದರೆ
ತಿಳಿ-ಸಿ-ಕೊಡ-ಬೇಕಾಗಿದೆ
ತಿಳಿ-ಸಿ-ಕೊಡ-ಬೇಕು
ತಿಳಿ-ಸಿ-ಕೊಡಿ
ತಿಳಿ-ಸಿ-ಕೊಡುತ್ತಾನೆ
ತಿಳಿ-ಸಿ-ಕೊಡುತ್ತಿದ್ದರು
ತಿಳಿ-ಸಿ-ಕೊಡುತ್ತೇನೆ
ತಿಳಿ-ಸಿ-ಕೊಡು-ವಿರಾ
ತಿಳಿ-ಸಿ-ಕೊಡು-ವುದಕ್ಕೆ
ತಿಳಿ-ಸಿದ
ತಿಳಿ-ಸಿ-ದನು
ತಿಳಿ-ಸಿ-ದರು
ತಿಳಿ-ಸಿದೆ
ತಿಳಿ-ಸಿದ್ದನು
ತಿಳಿ-ಸಿ-ಬಿಟ್ಟರೆ
ತಿಳಿ-ಸಿ-ಬಿಟ್ಟಿದ್ದೇನೊ
ತಿಳಿ-ಸಿ-ಬಿಟ್ಟೆ
ತಿಳಿ-ಸಿ-ರ-ಬಹುದೇ
ತಿಳಿ-ಸಿ-ಹೇಳು-ತ್ತಿದ್ದರು
ತಿಳಿಸು
ತಿಳಿ-ಸುತ್ತ
ತಿಳಿ-ಸುತ್ತ-ದೆಂದು
ತಿಳಿ-ಸುತ್ತಾ
ತಿಳಿ-ಸುತ್ತಾರೆ
ತಿಳಿ-ಸುತ್ತಿದ್ದರು
ತಿಳಿ-ಸುತ್ತೇನೆ
ತಿಳಿ-ಸುವ
ತಿಳಿ-ಸು-ವ-ಹಾಗೆ
ತಿಳಿ-ಸು-ವಿರಾ
ತಿಳಿ-ಸು-ವುದಕ್ಕೆ
ತಿಳಿ-ಸು-ವುದಕ್ಕೋಸ್ಕರ
ತಿಳಿ-ಸು-ವುದು
ತಿಳಿ-ಸು-ವುದೂ
ತಿಳಿ-ಸುವೆ
ತಿಳಿ-ಸು-ವೆ-ಯಂತೆ
ತಿಳು-ವ-ಳಿಕೆ
ತಿವಿದರು
ತಿವಿ-ಯುತಿವೆ
ತಿವಿ-ಯುತ್ತಿದ್ದರು
ತೀಕ್ಷ್ಣ
ತೀಕ್ಷ್ಣ-ಮತಿ-ಯಾಗಿದ್ದ
ತೀಕ್ಷ್ಣ-ವಾಗಿ
ತೀಕ್ಷ್ಣ-ವಾಗಿದೆ
ತೀಡುತ್ತ
ತೀರ
ತೀರಕೆ
ತೀರಕ್ಕೆ
ತೀರದ
ತೀರ-ದಲ್ಲಿ
ತೀರ-ದಲ್ಲಿದ್ದ
ತೀರ-ದ-ವರೆಗೆ
ತೀರ-ದೊಳೆಸೆ-ದಿಹ
ತೀರ-ಬೇಕು
ತೀರ-ಬೇಕೆಂದು
ತೀರ-ಬೇಕೆಂಬ
ತೀರಾ
ತೀರಿದ
ತೀರಿ-ಸಿ-ಕೊಳ್ಳುವ
ತೀರಿ-ಸಿ-ಕೊಳ್ಳುವರು
ತೀರು-ವಂತೆ
ತೀರುವರು
ತೀರು-ವಿರಿ
ತೀರು-ವುದು
ತೀರುವೆ
ತೀರ್ಥ
ತೀರ್ಥಂಕರ-ರಂತೆ
ತೀರ್ಥಕ್ಷೇತ್ರಕ್ಕೆ
ತೀರ್ಥಕ್ಷೇತ್ರಾದಿ-ಗಳನ್ನು
ತೀರ್ಥ-ಗಳನ್ನು
ತೀರ್ಥ-ಗ-ಳಲ್ಲ
ತೀರ್ಥ-ಗಳೇ
ತೀರ್ಥಸ್ಥಳ
ತೀರ್ಥಾದಿ
ತೀರ್ಮಾನ-ವನ್ನು
ತೀರ್ಮಾ-ನಿಸಿ
ತೀರ್ಮಾ-ನಿಸಿ-ದಾ-ಗಲೇ
ತೀವ್ರ
ತೀವ್ರ-ಚಲನೆ
ತೀವ್ರ-ತರ-ವಾಗಿತ್ತೆಂದರೆ
ತೀವ್ರತೆ
ತೀವ್ರ-ತೆ-ಯಿಂದ
ತೀವ್ರ-ಮಟ್ಟಕ್ಕೆ
ತೀವ್ರ-ವಾಗಿ
ತೀವ್ರ-ವಾಗಿಯೇ
ತೀವ್ರ-ವಾಗಿ-ರ-ಬೇಕು
ತೀವ್ರ-ವಾಗಿ-ರ-ವುದು
ತೀವ್ರ-ವಾಗಿ-ರುವ
ತೀವ್ರ-ವಾಗು-ವುದು
ತೀವ್ರ-ವಾದ
ತೀವ್ರ-ವೈ-ರಾಗ್ಯದ
ತು
ತುಂಡನ್ನು
ತುಂಡರಿಸು
ತುಂಡರಿಸುತ್ತಲಿ
ತುಂಡಾ-ದವು
ತುಂಡಿನ
ತುಂತು-ರನ್ನು
ತುಂಬ
ತುಂಬ-ಬಲ್ಲುದು
ತುಂಬ-ಬ-ಹುದು
ತುಂಬ-ಬೇಕು
ತುಂಬ-ಲಿಚ್ಛಿ-ಸುವೆನು
ತುಂಬ-ಲಿಲ್ಲವೋ
ತುಂಬಲು
ತುಂಬಾ
ತುಂಬಿ
ತುಂಬಿ-ಕೊಂಡಿತು
ತುಂಬಿ-ಕೊಂಡಿದೆ
ತುಂಬಿ-ಕೊಂಡಿರು-ವಂತಿದೆ
ತುಂಬಿ-ಕೊಂಡು
ತುಂಬಿ-ಕೊಳ್ಳುತ್ತಿತ್ತೆಂದರೆ
ತುಂಬಿ-ಗಳು
ತುಂಬಿಟ್ಟು
ತುಂಬಿತು
ತುಂಬಿ-ತುಳು-ಕಾಡುತ್ತಿದೆ
ತುಂಬಿತೋ
ತುಂಬಿತ್ತು
ತುಂಬಿದ
ತುಂಬಿ-ದರು
ತುಂಬಿ-ದ-ವಳು
ತುಂಬಿ-ದವು
ತುಂಬಿದೆ
ತುಂಬಿದ್ದರೂ
ತುಂಬಿ-ರುತ್ತದೆ
ತುಂಬಿ-ರುವ
ತುಂಬಿ-ರುವರು
ತುಂಬಿ-ರು-ವುದು
ತುಂಬಿ-ರು-ವುವೋ
ತುಂಬಿ-ರುವೆ
ತುಂಬಿವೆ
ತುಂಬಿ-ಸಲು
ತುಂಬಿಸಿ
ತುಂಬಿ-ಸಿ-ಕೊಳ್ಳುವಂತೆ
ತುಂಬಿ-ಸಿ-ಕೊಳ್ಳುವುದು
ತುಂಬಿಹ
ತುಂಬಿ-ಹ-ನ-ವನು
ತುಂಬಿ-ಹುದು
ತುಂಬಿ-ಹೋಗಿ
ತುಂಬಿ-ಹೋಗಿತ್ತು
ತುಂಬಿ-ಹೋಗಿದೆ
ತುಂಬಿ-ಹೋಗಿ-ರುವ
ತುಂಬಿ-ಹೋಗುತ್ತದೆ
ತುಂಬು
ತುಂಬುತ
ತುಂಬು-ತಿದೆ
ತುಂಬು-ತಿ-ರುವೆ
ತುಂಬುತ್ತೀರಿ
ತುಂಬು-ವ-ನೆಂದರೆ
ತುಂಬು-ವುದು
ತುಂಬು-ಹಂಬಲ-ಗೊಂಡಿಹೆ
ತುಕ್ಕಿದು
ತುಕ್ಕು
ತುಚ್ಚ
ತುಚ್ಛ
ತುಚ್ಛ-ತನದ
ತುಚ್ಛ-ವಾದ
ತುಚ್ಛ-ವಾ-ದುವು-ಗಳು
ತುಚ್ಛೀ-ಕರಿ-ಸುವರು
ತುಟಿ-ಗಳು
ತುಡಿಯುತ್ತಿರುತ್ತಿವೆ
ತುಡಿ-ಯುತ್ತಿ-ರು-ವುದು
ತುಡಿವ
ತುಣುಕು
ತುಣು-ಕೊಂದು
ತುತ್ತಾಗಿ
ತುತ್ತಾಗು-ವರು
ತುತ್ತಿ-ನೊ-ಡನೆ
ತುತ್ತು
ತುತ್ತೂರಿ
ತುತ್ತೆತ್ತಲಿಕ್ಕಾ-ದರೂ
ತುದಿ
ತುದಿ-ಗಾ-ಲಿ-ನಲ್ಲಿ
ತುದಿ-ಯನು
ತುದಿ-ಯ-ವ-ರೆಗೂ
ತುದಿ-ಯಿಂದ
ತುಮಾ
ತುಮಾಯ
ತುಮಾರ
ತುಮಾರ್
ತುಮಿ
ತುಮಿ-ತಮ
ತುರೀಯಾ-ನಂದ
ತುಲಸೀ-ದಾಸನ
ತುಲಾ-ರಾಶಿ
ತುಳಸೀ-ದಾಸರ
ತುಳಿ-ತಕ್ಕೊಳಗಾದ
ತುಳಿ-ತಕ್ಕೊಳಗಾದ-ವ-ರಿಂದ
ತುಳಿ-ದಿದ್ದೀರಿ
ತುಳಿ-ಯುತ
ತುಳಿ-ಯುವಾಗ
ತುಳು-ಕಾಡುತ್ತಿದೆ
ತುಳು-ಕಾಡುತ್ತಿದೆಯೊ
ತುಳು-ಕಾಡುತ್ತಿರು-ವರು
ತುಳು-ಕಾಡು-ವು-ದನ್ನು
ತುಳುಕಿದ
ತುಳುಕಿಹುದ-ದುವೆ
ತುಳುಕುತ್ತಿತ್ತು
ತುಳುಕುತ್ತಿದ್ದ
ತುಳುಕುತ್ತಿವೆ
ತುಷ್ಟಿ-ಮಂತಃ
ತುಸು
ತುಹಾರಿ
ತೂಕ-ವುಳ್ಳ
ತೂಗಾಡು-ವುದು
ತೂಗಿದೆ
ತೂಗಿ-ಸು-ವುವು
ತೂಗುತ್ತಿದ್ದಾಗ
ತೂಗುತ್ತಿವೆ
ತೂರಾಡು-ವಂತಿತ್ತು
ತೂರಿ
ತೂರಿ-ಬಿಟ್ಟಂತಾ-ಗಿದೆ
ತೂರಿ-ಬಿಡು-ವರು
ತೂರ್ಯ-ವಾಣಿ-ಯನ್ನು
ತೃಣ-ಮಾತ್ರವೂ
ತೃಣಾದಪಿ
ತೃಪ್ತತೃಷ್ಣಾಃ
ತೃಪ್ತ-ನಾಗಿ-ರುವಿ
ತೃಪ್ತರೋ
ತೃಪ್ತಿ
ತೃಪ್ತಿ-ಕರ-ವಾದ
ತೃಪ್ತಿ-ಪಡಿಸಿ
ತೃಪ್ತಿ-ಪಡಿ-ಸಿ-ಕೊಳ್ಳದೆ
ತೃಪ್ತಿ-ಪಡಿ-ಸುತ್ತೀ-ಯೇನು
ತೃಪ್ತಿಯ
ತೃಪ್ತಿ-ಯ-ವರೆಗೆ
ತೃಪ್ತಿ-ಯಾಗುತ್ತ-ದೆಯೆ
ತೃಪ್ತಿ-ಯಾ-ಗು-ವಂತೆ
ತೃಪ್ತಿ-ಯಾ-ಗು-ವು-ದಿಲ್ಲ
ತೃಪ್ತಿ-ಯಾಗುವು-ದೆಂಬು-ದಕ್ಕೆ
ತೃಪ್ತಿ-ಯಾ-ಯಿತೇ
ತೃಷಾಪೀಡಿತ-ನಾಗಿದ್ದಾಗ
ತೃಷೆ
ತೃಷೆ-ಯಿಲ್ಲದ
ತೃಷ್ಣೆ
ತೃಷ್ಣೆ-ಯಿಂದಿಲ್ಲೀಗ
ತೃಷ್ಣೆಯು
ತೆಕ್ಕೆಗೆ
ತೆಕ್ಕೆ-ಯೊ-ಳಗೆ
ತೆಗ-ಳಿಕೆ-ಗಳೆ-ರಡೂ
ತೆಗಳಿ-ಕೆಗೆ
ತೆಗಳಿ-ಕೆಯೊ
ತೆಗೆದ
ತೆಗೆ-ದಿಟ್ಟಿದ್ದೇನೆ
ತೆಗೆ-ದಿಡು
ತೆಗೆ-ದಿರಿಸಿ
ತೆಗೆದು
ತೆಗೆ-ದು-ಕ-ಕೊಂಡು
ತೆಗೆ-ದುಕೊ
ತೆಗೆ-ದು-ಕೊಂಡ
ತೆಗೆ-ದು-ಕೊಂಡದ್ದ-ರಿಂದ
ತೆಗೆ-ದು-ಕೊಂಡ-ರಂತೆ
ತೆಗೆ-ದು-ಕೊಂಡರು
ತೆಗೆ-ದು-ಕೊಂಡರೂ
ತೆಗೆ-ದು-ಕೊಂಡರೆ
ತೆಗೆ-ದು-ಕೊಂಡಾಗ
ತೆಗೆ-ದು-ಕೊಂಡಿತು
ತೆಗೆ-ದು-ಕೊಂಡಿದ್ದಾರೆ
ತೆಗೆ-ದು-ಕೊಂಡು
ತೆಗೆ-ದು-ಕೊಂಡು-ಹೋಗಿ
ತೆಗೆ-ದು-ಕೊಂಡೊ-ಡ-ನೆಯೇ
ತೆಗೆ-ದು-ಕೊಡು
ತೆಗೆ-ದು-ಕೊಳ್ಳ-ದಿದ್ದರೆ
ತೆಗೆ-ದು-ಕೊಳ್ಳ-ಬ-ಹುದು
ತೆಗೆ-ದು-ಕೊಳ್ಳ-ಬೇಕಾಗಿದ್ದ
ತೆಗೆ-ದು-ಕೊಳ್ಳ-ಬೇಕಾ-ಗುತ್ತದೆ
ತೆಗೆ-ದು-ಕೊಳ್ಳ-ಬೇಕೆಂದು
ತೆಗೆ-ದು-ಕೊಳ್ಳ-ಬೇಕೆಂಬ
ತೆಗೆ-ದು-ಕೊಳ್ಳ-ಲಾಗಿತ್ತು
ತೆಗೆ-ದು-ಕೊಳ್ಳಲಿ
ತೆಗೆ-ದು-ಕೊಳ್ಳ-ಲಿಲ್ಲ
ತೆಗೆ-ದು-ಕೊಳ್ಳಲು
ತೆಗೆ-ದು-ಕೊಳ್ಳಿ
ತೆಗೆ-ದು-ಕೊಳ್ಳುತ್ತ
ತೆಗೆ-ದು-ಕೊಳ್ಳುತ್ತಾರೆ
ತೆಗೆ-ದು-ಕೊಳ್ಳುತ್ತಾ-ರೆಯೋ
ತೆಗೆ-ದು-ಕೊಳ್ಳುತ್ತಿದ್ದರು
ತೆಗೆ-ದು-ಕೊಳ್ಳುತ್ತೀರಿ
ತೆಗೆ-ದು-ಕೊಳ್ಳುತ್ತೇವೆ
ತೆಗೆ-ದು-ಕೊಳ್ಳುವ
ತೆಗೆ-ದು-ಕೊಳ್ಳುವಂತೆ
ತೆಗೆ-ದು-ಕೊಳ್ಳುವ-ರಲ್ಲವೆ
ತೆಗೆ-ದು-ಕೊಳ್ಳುವರು
ತೆಗೆ-ದು-ಕೊಳ್ಳುವ-ರೆಂಬುದು
ತೆಗೆ-ದು-ಕೊಳ್ಳುವರೋ
ತೆಗೆ-ದು-ಕೊಳ್ಳುವ-ವರು
ತೆಗೆ-ದು-ಕೊಳ್ಳು-ವುದಕ್ಕೆ
ತೆಗೆ-ದು-ಕೊಳ್ಳು-ವುದಕ್ಕೆಂದು
ತೆಗೆ-ದು-ಕೊಳ್ಳು-ವು-ದನ್ನು
ತೆಗೆ-ದು-ಕೊಳ್ಳುವು-ದರ
ತೆಗೆ-ದು-ಕೊಳ್ಳುವು-ದಾಗಿ
ತೆಗೆ-ದು-ಕೊಳ್ಳು-ವು-ದಿಲ್ಲ
ತೆಗೆ-ದು-ಕೊಳ್ಳುವುದು
ತೆಗೆ-ದು-ಕೊಳ್ಳೋಣ
ತೆಗೆ-ದುಕೋ
ತೆಗೆ-ದು-ಹಾಕಲು
ತೆಗೆ-ದು-ಹಾಕಲ್ಪಟ್ಟಂತೆಯೆ
ತೆಗೆ-ದು-ಹಾಕಿ
ತೆಗೆ-ದು-ಹಾಕಿದ
ತೆಗೆ-ದು-ಹಾಕಿದ್ದ-ರಿಂದ
ತೆಗೆ-ದು-ಹಾಕಿ-ಬಿಟ್ಟ
ತೆಗೆ-ದು-ಹಾಕುವನು
ತೆಗೆದೆ
ತೆಗೆ-ಯ-ಬೇಕಾಗಿದೆ
ತೆಗೆ-ಯು-ವುದು
ತೆತ್ತು
ತೆತ್ತು-ಕೊಂಡು
ತೆತ್ತೇ
ತೆಪ್ಪಗಾದ
ತೆಯ್ಯು-ವುದಕ್ಕೆ
ತೆರ
ತೆರ-ದಲಿ
ತೆರ-ದಲೀ-ತನ
ತೆರ-ದಲ್ಲಿ
ತೆರದಿ
ತೆರ-ನಾಗೆ
ತೆರ-ಬೇ-ಕಿತ್ತು
ತೆರ-ಬೇಕು
ತೆರ-ಳ-ಬೇಕು
ತೆರ-ಳಿ-ದಂತೆಯೇ
ತೆರ-ಳಿದರು
ತೆರಿಗೆ-ಯನ್ನು
ತೆರು-ವುದಕ್ಕೆ
ತೆರೆ
ತೆರೆ-ಗಳ
ತೆರೆ-ಗಳನ್ನು
ತೆರೆದ
ತೆರೆ-ದಿದೆ
ತೆರೆ-ದಿ-ದೆಯೋ
ತೆರೆ-ದಿಲ್ಲ
ತೆರೆದು
ತೆರೆ-ಮರೆಯೊಳಡ-ಗದೆ
ತೆರೆಯ
ತೆರೆ-ಯಂತೆ
ತೆರೆ-ಯ-ದಿ-ರು-ವುದು
ತೆರೆ-ಯದು
ತೆರೆ-ಯನ್ನಿನ್ನೂ
ತೆರೆ-ಯನ್ನಿಳಿಸು
ತೆರೆ-ಯನ್ನು
ತೆರೆ-ಯ-ಬೇಕು
ತೆರೆ-ಯಲಿ
ತೆರೆ-ಯ-ಲೇಬೇ-ಕಿಂದಿಗೆ
ತೆರೆ-ಯಿತು
ತೆರೆ-ಯಿರಿ
ತೆರೆ-ಯಿರೆ-ನಗೆ
ತೆರೆ-ಯುತ್ತಿವೆ
ತೆರೆ-ಯು-ವಂತೆ
ತೆರೆ-ಯು-ವು-ದನ್ನು
ತೆರೆ-ಯು-ವುವು
ತೆರೆಯೌ
ತೆರೆ-ವಳು
ತೆರೆವೀ
ತೆರೆ-ವುದು
ತೆರೆ-ಸಿ-ರು-ವಿರಿ
ತೆಲುಗರು
ತೆಳು-ವಾದ
ತೆಳೆದು
ತೆವಳುತ್ತಿರುವ
ತೇ
ತೇಖಿಲೆ
ತೇಜಸ್ತರಂತಿ
ತೇಜಸ್ವಿ-ಯಾದ
ತೇಜಸ್ಸನ್ನು
ತೇಜಸ್ಸಿಗೆ
ತೇಜಸ್ಸಿನ
ತೇಜಸ್ಸಿನಿಂದಷ್ಟೇ
ತೇಜಸ್ಸು
ತೇಜೋ-ಬಲ-ದಿಂದಿರುವ
ತೇಜೋ-ಹೀನ-ತೆ-ಯಿಂದಲೇ
ತೇದ-ರ-ವರು
ತೇದೆ
ತೇಯು-ವಂತೆ
ತೇಲಲಿ
ತೇಲಿದೆ
ತೇಲಿ-ಬಿಟ್ಟು
ತೇಲಿಸಿ
ತೇಲುತ
ತೇಲುತ-ಲಿ-ರುವೆ
ತೇಲು-ತಿದೆ
ತೇಲು-ತಿ-ರುವೆ
ತೇಲುತ್ತ
ತೇಲುತ್ತಾ
ತೇಲುತ್ತಿದ್ದರೆ
ತೈಲ
ತೈಲ-ಚಿತ್ರ
ತೈಲ-ಚಿತ್ರ-ಗಳನ್ನು
ತೈಲ-ಚಿತ್ರ-ಗಳಿವೆ
ತೈಲ-ಚಿತ್ರ-ಗಳು
ತೈಲ-ಚಿತ್ರ-ದಲ್ಲೂ
ತೈಲ-ಚಿತ್ರ-ವನ್ನು
ತೊಂದರೆ
ತೊಂದರೆ-ಗಳಿ-ರುವಾಗ
ತೊಂದರೆಗೆ
ತೊಂದರೆ-ಯನ್ನು
ತೊಂದರೆ-ಯಾಗುತ್ತಿತ್ತು
ತೊಂದರೆ-ಯಾಗುವುದು
ತೊಂದರೆ-ಯೆಲ್ಲಾ
ತೊಂಬತ್ತ-ರಷ್ಟು
ತೊಂಬತ್ತು
ತೊಂಬತ್ತೊಂಬತ್ತ-ರಷ್ಟು
ತೊಟ್ಟನ್ನು
ತೊಟ್ಟರೂ
ತೊಟ್ಟಿಕ್ಕು-ವಂತಿತ್ತು
ತೊಟ್ಟಿಗೆ
ತೊಟ್ಟಿ-ಯಲ್ಲಿ
ತೊಟ್ಟಿಲಲಿ
ತೊಟ್ಟು
ತೊಡ-ಕಿ-ನಲ್ಲಿ
ತೊಡಕು
ತೊಡಕು-ಗಳಿಲ್ಲದ
ತೊಡಗಿ
ತೊಡಗಿತು
ತೊಡಗಿ-ದರು
ತೊಡಗಿ-ದರೆ
ತೊಡಗಿ-ದಾಗ
ತೊಡಗಿ-ದೆನೊ
ತೊಡಗಿದ್ದಾನೋ
ತೊಡ-ಗಿರಿ
ತೊಡಗಿ-ರುತ್ತವೆ
ತೊಡಗಿ-ರುವ
ತೊಡಗಿ-ರುವುದ-ರಲ್ಲಿ
ತೊಡಗಿ-ಸಿ-ಕೊಳ್ಳಲಿ
ತೊಡಗು
ತೊಡಗು-ವರು
ತೊಡಗು-ವವು
ತೊಡಗು-ವು-ದಿಲ್ಲ
ತೊಡಗು-ವುದು
ತೊಡಿಗೆ-ಯಲ್ಲಿ
ತೊಡಿ-ಸುತ
ತೊಡಿ-ಸುವ
ತೊಡಿ-ಸು-ವುದಕ್ಕೆ
ತೊಡೆದು-ಹಾಕು
ತೊಡೆದು-ಹಾಕು-ವುದಕ್ಕಾಗಿ
ತೊಡೆದು-ಹಾಕು-ವುದು
ತೊಡೆ-ಯು-ವುದಕ್ಕೆ
ತೊಮಾ
ತೊಮಾಯ
ತೊಮಾರ
ತೊಮಾರಿ
ತೊರೆ
ತೊರೆ-ದರು
ತೊರೆ-ದಿ-ರು-ವೆನು
ತೊರೆದು
ತೊರೆ-ದೊಂದೆ-ಯೆಂಬುದ-ನರಿಯೆ
ತೊರೆಯ
ತೊರೆ-ಯಲು
ತೊರೆ-ಯುವಳು
ತೊರೆ-ಯು-ವು-ದ-ರಿಂದ
ತೊರೆಯೊ
ತೊರೆವೆ
ತೊಲಗಿ
ತೊಲಗಿ-ದರೆ
ತೊಲಗಿ-ಹೋಗು-ವ-ವ-ರೆಗೂ
ತೊಲೆ
ತೊಳಲಾಟ-ವನ್ನೂ
ತೊಳಲು-ತಿದ್ದೆ
ತೊಳಲುತ್ತಿರುವ
ತೊಳೆದು
ತೊಳೆದು-ಕೊಂಡು
ತೊವ್ವೆ
ತೋಚ-ದಾ-ಯಿತು
ತೋಚದೆ
ತೋಚಿದ
ತೋಚಿ-ದು-ದನ್ನು
ತೋಚು-ವುದಿಷ್ಟೆ
ತೋಟ
ತೋಟಕ್ಕೆ
ತೋಟ-ಗಾ-ರರು
ತೋಟದ
ತೋಟ-ದಲ್ಲಿ
ತೋಟ-ದಲ್ಲಿದ್ದ
ತೋಟ-ದಲ್ಲಿದ್ದಾಗ
ತೋಟ-ದಲ್ಲಿಯೇ
ತೋಟ-ದಲ್ಲಿ-ರುವ
ತೋಟ-ದಲ್ಲಿ-ರುವ-ನೆಂಬುದು
ತೋಟ-ದಲ್ಲೇ
ತೋಟ-ವೆನ್ನುತ್ತಿದ್ದರು
ತೋಟ-ವೇನೋ
ತೋಟ-ವೊಂದ-ರಲ್ಲಿ
ತೋಡಿ
ತೋಪನು
ತೋಪು
ತೋಮಾ
ತೋಮಾಮ್
ತೋಮಾರ್
ತೋರ
ತೋರಣ
ತೋರ-ಣ-ಗಳಿಂದ
ತೋರದು
ತೋರ-ದು-ದರ
ತೋರದೆ
ತೋರ-ಬೇಕು
ತೋರ-ಲಿಲ್ಲ
ತೋರ-ಲಿಲ್ಲವೋ
ತೋರಿ
ತೋರಿ-ಕೆ-ಗಷ್ಟೇ
ತೋರಿ-ಕೆಗೆ
ತೋರಿ-ಕೆಯ
ತೋರಿ-ಕೆಯು
ತೋರಿತು
ತೋರಿದ
ತೋರಿ-ದರು
ತೋರಿ-ದರೂ
ತೋರಿಸ
ತೋರಿ-ಸದೆ
ತೋರಿ-ಸ-ಬಲ್ಲ-ವ-ನಾಗಿ-ರು-ವ-ನೆಂಬು-ದನ್ನು
ತೋರಿ-ಸ-ಬ-ಹುದು
ತೋರಿ-ಸ-ಬೇಕು
ತೋರಿ-ಸಲು
ತೋರಿಸಿ
ತೋರಿ-ಸಿ-ಕೊಟ್ಟನು
ತೋರಿ-ಸಿ-ಕೊಟ್ಟರು
ತೋರಿ-ಸಿ-ಕೊಟ್ಟಿದ್ದೇನೆ
ತೋರಿ-ಸಿ-ಕೊಟ್ಟಿ-ರುವ
ತೋರಿ-ಸಿ-ಕೊಟ್ಟು
ತೋರಿ-ಸಿ-ಕೊಡಿ
ತೋರಿ-ಸಿ-ಕೊಡು
ತೋರಿ-ಸಿ-ಕೊಡುತ್ತಿದ್ದರು
ತೋರಿ-ಸಿ-ಕೊಡುತ್ತೇನೆ
ತೋರಿ-ಸಿ-ಕೊಡು-ವರು
ತೋರಿ-ಸಿ-ಕೊಡು-ವುದಕ್ಕಾಗಿ
ತೋರಿ-ಸಿ-ಕೊಡು-ವುದಕ್ಕೆ
ತೋರಿ-ಸಿದ
ತೋರಿ-ಸಿ-ದಂತೆ
ತೋರಿ-ಸಿ-ದರು
ತೋರಿ-ಸಿ-ದರೆ
ತೋರಿ-ಸಿ-ದಳು
ತೋರಿ-ಸಿದ್ದಾನೆ
ತೋರಿ-ಸಿದ್ದಾರೆ
ತೋರಿ-ಸಿದ್ದಾರೆಯೊ
ತೋರಿ-ಸಿಲ್ಲ
ತೋರಿಸು
ತೋರಿ-ಸುತ್ತ
ತೋರಿ-ಸುತ್ತಲೂ
ತೋರಿ-ಸುತ್ತವೆ
ತೋರಿ-ಸುತ್ತಾ
ತೋರಿ-ಸುತ್ತಿದ್ದರು
ತೋರಿ-ಸುವ
ತೋರಿ-ಸು-ವಂತೆ
ತೋರಿ-ಸು-ವನು
ತೋರಿ-ಸು-ವುದಕ್ಕೆ
ತೋರಿ-ಸು-ವು-ದರ
ತೋರಿ-ಸು-ವುದು
ತೋರು
ತೋರು-ತಿದ್ದರು
ತೋರು-ತಿ-ರುವೆ
ತೋರು-ತಿ-ಹುದು
ತೋರುತ್ತದೆ
ತೋರುತ್ತಲೂ
ತೋರುತ್ತವೆ
ತೋರುತ್ತಿತ್ತು
ತೋರುತ್ತಿದ್ದರು
ತೋರುವ
ತೋರು-ವ-ನಾ-ತನು
ತೋರು-ವನು
ತೋರು-ವ-ರೆಂಬು-ದನ್ನೂ
ತೋರು-ವರೋ
ತೋರು-ವಳು
ತೋರು-ವು-ದನ್ನು
ತೋರು-ವುದ-ರಲ್ಲಿ
ತೋರು-ವು-ದಿಲ್ಲ
ತೋರು-ವುದು
ತೋರು-ವು-ದೇ-ನೆಂದರೆ
ತೋರು-ವುದೋ
ತೋರ್ದ
ತೋರ್ಪಡಿ-ಸ-ಬೇಕು
ತೋರ್ಪಡಿ-ಸಿ-ಕೊಳ್ಳಲು
ತೋಲೆ
ತೋಳಪ್ಪು-ಗೆಯ
ತೋಳಿನ
ತೋಳು-ಗಳಲ್ಲಿ
ತೋಳು-ಗಳಲ್ಲಿಯೂ
ತೌರೂ-ರಾಗಿ
ತೌರೂರು
ತ್ತಬೇ-ಕಾ-ದರೂ
ತ್ಯಕ್ತ
ತ್ಯಜಿ
ತ್ಯಜಿ-ಸದೆ
ತ್ಯಜಿ-ಸ-ಬೇಕು
ತ್ಯಜಿ-ಸಲು
ತ್ಯಜಿ-ಸಲ್ಪಟ್ಟು
ತ್ಯಜಿಸಿ
ತ್ಯಜಿ-ಸಿದ
ತ್ಯಜಿ-ಸಿ-ದ-ವನು
ತ್ಯಜಿ-ಸಿ-ಬಿಟ್ಟಿದ್ದೇನೆ
ತ್ಯಜಿ-ಸುತ
ತ್ಯಜಿ-ಸು-ವಂತೆ
ತ್ಯಜಿ-ಸುವನು
ತ್ಯಜಿ-ಸುವರು
ತ್ಯಜಿ-ಸುವರೊ
ತ್ಯಜಿ-ಸುವ-ಳೇನು
ತ್ಯಜಿ-ಸು-ವುದೇ
ತ್ಯಜೇತ್
ತ್ಯಾಗ
ತ್ಯಾಗಕ್ಕಾಗಿ
ತ್ಯಾಗಕ್ಕೆ
ತ್ಯಾಗ-ಗಳೆ-ರಡನ್ನೂ
ತ್ಯಾಗ-ಜೀವನಕ್ಕಾ-ದರೂ
ತ್ಯಾಗ-ಜೀವನದ
ತ್ಯಾಗ-ಜೀವಿ-ಗಳ
ತ್ಯಾಗ-ಜೀವಿ-ಗಳಾಗ-ಬೇಕು
ತ್ಯಾಗತ್ಯಾಗತ್ಯಾಗ
ತ್ಯಾಗದ
ತ್ಯಾಗ-ದಿಂದ
ತ್ಯಾಗ-ದಿಂದಲೇ
ತ್ಯಾಗ-ಬುದ್ಧಿ
ತ್ಯಾಗ-ಬುದ್ಧಿ-ಯನ್ನು
ತ್ಯಾಗ-ಬುದ್ಧಿ-ಯಿಂದ
ತ್ಯಾಗ-ಬುದ್ಧಿ-ಯಿಲ್ಲದೆ
ತ್ಯಾಗ-ಭೋಗ-ಬುದ್ಧಿ-ರ್
ತ್ಯಾಗ-ಮಾಡ-ಬೇಕೆಂಬ
ತ್ಯಾಗ-ಮಾ-ಡಲು
ತ್ಯಾಗ-ಮಾಡಿ
ತ್ಯಾಗ-ಮಾಡುವ
ತ್ಯಾಗ-ಮಾಡು-ವಂತಹ
ತ್ಯಾಗ-ಮಾಡು-ವುದೇ
ತ್ಯಾಗ-ಮೂರ್ತಿ-ಗಳು
ತ್ಯಾಗ-ರೂಪ-ವಾದ
ತ್ಯಾಗ-ವನ್ನು
ತ್ಯಾಗ-ವಲ್ಲ
ತ್ಯಾಗ-ವಿಲ್ಲದ
ತ್ಯಾಗ-ವಿಲ್ಲದೆ
ತ್ಯಾಗವು
ತ್ಯಾಗವೇ
ತ್ಯಾಗ-ವೊಂದ-ರಿಂದಲೇ
ತ್ಯಾಗವೋ
ತ್ಯಾಗವ್ರತ
ತ್ಯಾಗ-ಸಂಘರ್ಷ-ಗಳ
ತ್ಯಾಗಿ
ತ್ಯಾಗಿ-ಗಳ
ತ್ಯಾಗಿ-ಗಳಷ್ಟು
ತ್ಯಾಗಿ-ಗ-ಳಾದ
ತ್ಯಾಗಿ-ಗಳು
ತ್ಯಾಗಿ-ಗಳೆಲ್ಲಾ
ತ್ಯಾಗಿ-ಯಲ್ಲ-ದ-ವನು
ತ್ಯಾಗಿ-ಯಾ-ಗಿ-ರುವೆಯೊ
ತ್ಯಾಗಿ-ಯಾದ
ತ್ಯಾಗೀಶ್ವರ
ತ್ಯಾಗೇನೈಕೇ
ತ್ಯಾಜೆ
ತ್ರಿಕರ-ಣ-ಪೂರ್ವಕ
ತ್ರಿಗುಣ-ಗಳನ್ನು
ತ್ರಿಗುಣಾತೀತ
ತ್ರಿಗುಣಾತೀ-ತನು
ತ್ರಿಗುಣಾತೀತ-ರನ್ನು
ತ್ರಿಗುಣಾತೀತ-ರಿಗೆ
ತ್ರಿಗುಣಾತೀ-ತರು
ತ್ರಿಪುಟಿ
ತ್ರಿಭು-ವನ-ಮುತ್ಪಾಟಯಾಮೋ
ತ್ರಿವರ್ಣ-ದ-ವರೆಗೆ
ತ್ರಿಶೂನ್ಯ
ತ್ರಿಶೂಲ
ತ್ರೈಲೋಕ್ಯೇಽಪ್ಯಪ್ರತಿ-ಮಮ-ಹಿಮಾ
ತ್ವಂ
ತ್ವಕೃ-ತಸ್ಯ
ತ್ವನಾದಿ-ನಿ-ಧನಂ
ತ್ವಮ-ಚಲೋ
ತ್ವಮಸಿ
ತ್ವಮೃತಂಚ
ತ್ವಮೇವ
ತ್ವಯಿ
ತ್ವರೆ-ಯಿಂದ
ತ್ವವಿತಥಂ
ತ್ವವಿ-ರಾಮ-ವೃತ್ತಾ
ತ್ವವಿಷಮಂ
ತ್ವಶ-ರಣೋ
ತ್ಸರ್ಯ
ಥಟ್ಟನೆ
ಥಪ್ಪ
ಥಪ್ಪ-ದಲ್ಲಿ
ಥಪ್ಪ-ವನ್ನು
ಥಪ್ಪವು
ಥಪ್ಪಾ
ಥರಥರ
ಥಳಥಳಿ-ಸುತ್ತಿ-ರುವ
ಥಳಿಸಿ
ಥಾಕೆ
ಥಾಯ
ಥಿಯೇಟರಿ-ನಲ್ಲಿ
ಥೆರಾ-ಪುಟೇ
ಥೇರಾ-ಪುಟ
ಥೇಲ್
ಥೌಸೆಂಡ್
ದಂಗೆ
ದಂಗೆ-ಯೆದ್ದ
ದಂಡ
ದಂಡಕ್ಕಿಂತಲೂ
ದಂಡ-ನ-ರೂಪ-ವಾದ
ದಂಡ-ವೆಂದಾದರೆ
ದಂಡಿಸ-ಬೇಕು
ದಂಡಿಸಿ-ಕೊಳ್ಳುವು-ದರಿಂದ
ದಂಡೆ
ದಂತ-ಕತೆ-ಗಳನ್ನು
ದಂತ-ಕತೆ-ಗಳು
ದಂತ-ಕಥೆ
ದಂತ-ಕಥೆ-ಗಳಿಗೆ
ದಂತ-ಕಥೆ-ಯಲ್ಲಾಗಲೀ
ದಂತಮಾರ್ಜನ
ದಂಪತಿ-ಗಳ
ದಂಪತಿ-ಗಳಿಗೆ
ದಕ್ಷ
ದಕ್ಷ-ಜಾ-ದತ್ತ-ದೋಷಂ
ದಕ್ಷ-ತೆ-ಯನ್ನುಂಟು-ಮಾಡು-ವುದಕ್ಕೆ
ದಕ್ಷನೋ
ದಕ್ಷಿಣ
ದಕ್ಷಿಣಕ್ಕಿ-ರುವ
ದಕ್ಷಿ-ಣಕ್ಕೆ
ದಕ್ಷಿ-ಣದ
ದಕ್ಷಿಣ-ರಾಹ್ರಿ
ದಕ್ಷಿಣ-ರಾಹ್ರಿ-ಗಳ
ದಕ್ಷಿಣ-ರಾಹ್ರಿ-ಗಳು
ದಕ್ಷಿಣಾಭಿ-ಮುಖ-ವಾಗಿ
ದಕ್ಷಿಣೆ
ದಕ್ಷಿಣೇಶ್ವರ
ದಕ್ಷಿಣೇಶ್ವ-ರದ
ದಕ್ಷಿಣೇಶ್ವರ-ದಲ್ಲಿ
ದಕ್ಷಿಣೇಶ್ವರ-ದಲ್ಲಿದ್ದಾಗ
ದಕ್ಷಿಣೇಶ್ವರ-ದಲ್ಲಿ-ರುವ
ದಗ್ಧ
ದಟ್ಟಡಿ-ಯಿಡುತಲಿ
ದಟ್ಟ-ವಾಗಿದೆ
ದಟ್ಟ-ವಾದ
ದಟ್ಟಿಸಿ
ದಡ
ದಡಕ್ಕೆ
ದಡದ
ದಡ-ದಂತೆ
ದಡ-ದಲ್ಲಿ
ದಡದಲ್ಲಿದ್ದ
ದಡ-ವನ್ನು
ದಡ್ಡ
ದಡ್ಡನ
ದಡ್ಡ-ನಂತೆ
ದಡ್ಡ-ರಾಗಲು
ದಡ್ಡ-ರಾ-ರಿದ್ದಾರೆ
ದಡ್ಡ-ರಿಗೂ
ದಡ್ಡರು
ದಣಿ-ವರಿ-ಯ-ದಿಹ
ದಣಿ-ವಾ-ಗು-ವು-ದಿಲ್ಲ
ದತ್ತಂ
ದಧೀಚಿ
ದನ-ಗಳೆಲ್ಲಾ
ದನಗಾಹಿ
ದನಗಾಹಿಗೆ
ದನದ
ದನಿ
ದನಿಗೆ
ದನಿ-ಗೈ-ವುದು
ದನಿ-ಯಲಿ
ದನಿಯು
ದಬ್ಬಾ-ಳಿಕೆ
ದಬ್ಬಾಳಿ-ಕೆಗೆ
ದಬ್ಬಾಳಿ-ಕೆ-ಯಿಂದ
ದಬ್ಬಾಳಿ-ಕೆ-ಯಿಂದಲೂ
ದಮಯಂತಿ
ದಮಾನು
ದಯ
ದಯ-ಮಾಡಿ
ದಯ-ವಿಟ್ಟು
ದಯಾ
ದಯಾ-ನಿಧಿ
ದಯಾ-ನಿಧೆ
ದಯಾ-ಪರ-ರಾದ
ದಯಾ-ಪೂರಿತ
ದಯಾ-ಮಯ-ನಾದ
ದಯಾ-ಮಯ-ರಾದ
ದಯಾ-ಮಯಿ
ದಯಾರಾ-ಥಾರ್
ದಯಾರ್ದ್ರ-ಹೃದಯಿ-ಗಳು
ದಯೆ
ದಯೆಗೆ
ದಯೆಯ
ದಯೆ-ಯಿಂದ
ದಯೆ-ಯಿಟ್ಟು
ದಯೆ-ಯಿದ್ದಲ್ಲಿ
ದಯೆ-ಯೆನ್ನುವೆ-ಯೇನು
ದರಿದ್ರ
ದರಿದ್ರ-ನೊಬ್ಬನ
ದರಿದ್ರ-ರಿಗೆ
ದರಿದ್ರರೂ
ದರು-ಶನ-ದಾಸೆಯ
ದರೇಚಿ
ದರ್ಜಿಯ
ದರ್ಜೆಯ
ದರ್ಬಾರಿ-ನಲ್ಲಿಡು
ದರ್ಬಾರು
ದರ್ಬಾರು-ಗಳನ್ನು
ದರ್ಬಾರು-ಗಳು
ದರ್ಶಕರ
ದರ್ಶನ
ದರ್ಶನಕ್ಕಾಗಿ
ದರ್ಶ-ನಕ್ಕೆ
ದರ್ಶನಕ್ಕೋಸ್ಕರ
ದರ್ಶನ-ಗಳ
ದರ್ಶನ-ಗಳಲ್ಲಿ
ದರ್ಶನ-ಗಳಾ-ದುವೆ
ದರ್ಶನ-ಗಳಿಗೆ
ದರ್ಶನ-ಗಳೆಲ್ಲಾ
ದರ್ಶನದ
ದರ್ಶನ-ದಲ್ಲಿ
ದರ್ಶನ-ದಾರಭ್ಯ
ದರ್ಶನ-ದಿಂದ
ದರ್ಶನ-ಮಾಡ-ಬೇಕೆಂದು
ದರ್ಶನ-ಮಾಡಿ
ದರ್ಶನ-ಲಾಭ
ದರ್ಶನ-ಲಾಭ-ದಿಂದ
ದರ್ಶನ-ಲಾಭ-ವಾಗುತ್ತದೆ
ದರ್ಶನ-ವಾಗಿ-ಬಿಟ್ಟರೆ
ದರ್ಶನ-ವಾಗುತ್ತದೆ
ದರ್ಶನ-ವಾಗುತ್ತಿದ್ದ-ರಿಂದ
ದರ್ಶನ-ವಾ-ಗು-ವು-ದಿಲ್ಲ-ವೆಂದು
ದರ್ಶನ-ವಾಗು-ವುದೋ
ದರ್ಶನ-ವಾದ
ದರ್ಶನ-ವಾದರೆ
ದರ್ಶನವು
ದರ್ಶನ-ವುಂಟಾ-ಗುತ್ತದೆ
ದರ್ಶನವೂ
ದರ್ಶನ-ವೆಂದು
ದರ್ಶನವೇ
ದರ್ಶನ-ವೊಂದನ್ನು
ದರ್ಶನ-ಶಾಸ್ತ್ರ-ಗಳು
ದರ್ಶನಾಕಾಂಕ್ಷಿ-ಗಳಾಗಿ
ದರ್ಶನಾರ್ಥ-ವಾಗಿ
ದರ್ಶನೇಂದ್ರಿಯ-ಗಳಿವೆ
ದರ್ಶನೋತ್ಸು-ಕ-ರಾದ
ದರ್ಶನ್ವಿಜ್ಞಾನ
ದರ್ಶಿತಪ್ರೇಮ-ವಿಜೃಂಭಿ-ತರಂಗಂ
ದರ್ಶಿ-ಸಿದರೋ
ದಲಿ-ತರೋ
ದಲ್ಲಿ-ರುವಿರೋ
ದಳ-ವೆಲ್ಲ
ದವ-ಡೆಗೆ
ದವ-ರೂಪನೆ
ದವಸಧಾನ್ಯ-ಗಳು
ದಶ
ದಶದಿಕ್
ದಷ್ಟೇ
ದಾಂಟುವ
ದಾಂಪತ್ಯವಿಚ್ಛೇದ-ನಕ್ಕೆ
ದಾಕ್ಷಿಣಾತ್ಯ
ದಾಕ್ಷಿಣಾತ್ಯರ
ದಾಖಲಿ-ಸಿದ್ದಾನೆ
ದಾಖಲೆ-ಯಿಲ್ಲ
ದಾಟ-ಬೇಕು
ದಾಟಲಾ-ಗದ
ದಾಟಲು
ದಾಟಿ
ದಾಟಿದ
ದಾಟಿ-ದರೋ
ದಾಟಿ-ದೆವು
ದಾಟಿದ್ದೇನೆ
ದಾಟಿ-ರುತ್ತಾನೋ
ದಾಟಿ-ರುತ್ತಾರೆ
ದಾಟಿ-ಸ-ಬೇಕು
ದಾಟುತ್ತಿತ್ತು
ದಾಟು-ವರು
ದಾಟುವಷ್ಟರೊಳ-ಗಾಗಿ
ದಾಟು-ವುದಕ್ಕೆ
ದಾಡಾಯಿಯೆ
ದಾಡಾಯೆ
ದಾಢ್ಯ-ವಿ-ರ-ಲಿಲ್ಲ
ದಾನ
ದಾನಕ್ಕಿಂತ
ದಾನ-ದಿಂದ
ದಾನ-ಪರತೆ
ದಾನ-ಮಾಡು
ದಾನ-ಮಾಡುತ್ತ-ವೆಯೋ
ದಾನ-ಮಾಡು-ವ-ವರೆ-ವಿಗೂ
ದಾನವ
ದಾನ-ವನ್ನು
ದಾನ-ವ-ರಾಗ-ಲೀ-ಯಾರೂ
ದಾನ-ವರು
ದಾನ-ವಿಲ್ಲ
ದಾನಾದಿ-ಗಳ
ದಾನಿಯಾಗ-ಬ-ಹುದು
ದಾನಿ-ಯಾಗಿಯೇ
ದಾಯ-ರ-ವರು
ದಾರದ
ದಾರ-ದಂತೆ
ದಾರಿ
ದಾರಿ-ಗಳಲ್ಲಿ
ದಾರಿಗೆ
ದಾರಿ-ಗೆನ್ನನು
ದಾರಿ-ತಪ್ಪಿ
ದಾರಿ-ತಪ್ಪಿದ
ದಾರಿ-ತೋ-ರೆಂದು
ದಾರಿದ್ರ್ಯ
ದಾರಿರ್ಯ-ಗಳಲ್ಲಿ
ದಾರಿರ್ಯ-ದಲ್ಲಿ
ದಾರಿರ್ಯ-ದಲ್ಲಿಯೂ
ದಾರಿರ್ಯ-ದಲ್ಲಿಯೇ
ದಾರಿರ್ಯವೂ
ದಾರಿಯ
ದಾರಿ-ಯನ್ನು
ದಾರಿ-ಯಲ್ಲ
ದಾರಿ-ಯಲ್ಲಿ
ದಾರಿ-ಯಲ್ಲಿಯೇ
ದಾರಿ-ಯಿದು
ದಾರಿ-ಯಿ-ದುವೆ
ದಾರಿ-ಯಿ-ದೆಯೆ
ದಾರಿ-ಯಿಲ್ಲ
ದಾರಿ-ಯಿಲ್ಲವು
ದಾರಿ-ಯೆ-ಡೆಗೆ
ದಾರಿಯೇ
ದಾರು-ಣ-ವಾ-ದು-ದನ್ನು
ದಾರ್ಢ್ಯ-ವನ್ನು
ದಾರ್ಢ್ಯವಿ-ದೆಯೆ
ದಾರ್ಶನಿಕ
ದಾರ್ಶನಿ-ಕರ
ದಾರ್ಶನಿಕ-ರಲ್ಲಿ
ದಾರ್ಶನಿಕ-ರಾಗಿ
ದಾರ್ಶನಿಕ-ರಿಗೆ
ದಾರ್ಶನಿ-ಕರು
ದಾರ್ಶನಿ-ಕರೂ
ದಾವ್
ದಾವ್ಯೇವಾ
ದಾಸ
ದಾಸ-ಗುಪ್ತ
ದಾಸ-ಗುಪ್ತ-ರೊಡನೆ
ದಾಸತ್ವ
ದಾಸತ್ವ-ದಿಂದ
ದಾಸ-ನಾ-ಗದೆ
ದಾಸ-ನಾಗಿ-ರುವ
ದಾಸ-ನಾದರೆ
ದಾಸ-ನಿದೊ
ದಾಸನು
ದಾಸ-ನೆಂಬುದೆ
ದಾಸ-ರಾಗಿ
ದಾಸ-ರಾಗಿದ್ದಾರೆ
ದಾಸಾ-ನು-ದಾಸ-ರಾಗಿ-ಬಿಡು-ವುವು
ದಾಸಿ-ಯಾ-ಗಿ-ರುವೆ
ದಾಸೋಽಹಂ
ದಾಸ್ಯ
ದಾಸ್ಯ-ದಲ್ಲಿ-ರುವಿರಿ
ದಾಸ್ಯ-ವಿ-ಮೋಚನೆ
ದಿ
ದಿಂಡರು
ದಿಂಡರೂ
ದಿಂಬು
ದಿಕ್ಕನ್ನು
ದಿಕ್ಕಿಗೆ
ದಿಕ್ಕಿನ
ದಿಕ್ಕಿಲ್ಲದ
ದಿಕ್ಕು
ದಿಕ್ಕು-ಗಳಲ್ಲಿಯೂ
ದಿಕ್ಕು-ಗಳಲ್ಲೆಲ್ಲಾ
ದಿಕ್ಕು-ಗಳೂ
ದಿಕ್ಕು-ಗಳೆಲ್ಲವೂ
ದಿಕ್ಕು-ದಿಕ್ಕಿಗೆ
ದಿಗಂತದ
ದಿಗಂತ-ದಲ್ಲಿ
ದಿಗಂಬರ
ದಿಗಿಲು
ದಿಗಿಲು-ಪಟ್ಟು
ದಿಗ್ದಿಗಂತಕೆ
ದಿಗ್ದಿ-ಸೆಯ
ದಿಗ್ಭ್ರಮೆ
ದಿಗ್ಭ್ರಮೆ-ಗೊಳಿ-ಸುತ್ತಾರೆ
ದಿಗ್ಭ್ರಮೆ-ಗೊಳಿಸು-ವುದು
ದಿಗ್ವಸ-ನವ
ದಿಗ್ವಿ-ಜ-ಯದ
ದಿಟ
ದಿಟಕೆ
ದಿಟದ
ದಿಟ-ದಾಳಕಿಳಿ-ದ-ವರು
ದಿಟ-ದಿ-ರವು
ದಿಟರು
ದಿಟ-ವೇನೋ
ದಿಟ್ಟ
ದಿಟ್ಟ-ತನ
ದಿಟ್ಟಿಯಿಟ್ಟಿ-ರುವನೋ
ದಿಟ್ಟಿಸಿ
ದಿಟ್ಟಿ-ಸುತ್ತಾ
ದಿನ
ದಿನಕ್ಕೆ
ದಿನಕ್ಕೆ-ರಡು
ದಿನಕ್ಕೊಂದು
ದಿನ-ಗಳ
ದಿನ-ಗಳನ್ನು
ದಿನ-ಗಳಲ್ಲಿ
ದಿನ-ಗಳಲ್ಲಿಯೆ
ದಿನ-ಗಳಲ್ಲಿಯೇ
ದಿನ-ಗಳಲ್ಲೊಂದು
ದಿನ-ಗಳ-ವರೆಗೆ
ದಿನ-ಗಳಾಗಿತ್ತು
ದಿನ-ಗಳಾಗಿದ್ದವು
ದಿನ-ಗಳಾಗಿವೆ
ದಿನ-ಗ-ಳಾದ
ದಿನ-ಗಳಾ-ದುವು
ದಿನ-ಗಳಿಂದ
ದಿನ-ಗಳಿಂದಲೂ
ದಿನ-ಗಳು
ದಿನ-ಗಳೆ-ನಿತೊ
ದಿನ-ಚರಿ-ಯಲ್ಲಿ
ದಿನದ
ದಿನ-ದಲ್ಲಿ
ದಿನ-ದ-ವರೆಗೆ
ದಿನ-ದಿಂದ
ದಿನ-ದಿನ
ದಿನ-ದಿ-ನಕ್ಕೆ
ದಿನ-ದಿ-ನವು
ದಿನ-ವನ್ನು
ದಿನ-ವನ್ನೂ
ದಿನ-ವಾಗಲಿ
ದಿನ-ವಾದರೂ
ದಿನವೂ
ದಿನ-ವೆಲ್ಲ
ದಿನ-ವೆಲ್ಲಾ
ದಿನಾಂಕ-ಗಳ
ದಿನೇ
ದಿನೇ-ದಿನೇ
ದಿಯೇಛಿ
ದಿವಂ
ದಿವಂಗತ
ದಿವದ
ದಿವಸ
ದಿವ-ಸದ
ದಿವಸ-ದಿಂದಲೂ
ದಿವಾ-ಕರ
ದಿವ್ಯ
ದಿವ್ಯ-ದರ್ಶನ-ಗಳನ್ನು
ದಿವ್ಯ-ಭಾವದ
ದಿವ್ಯ-ಲೀಲೆ
ದಿವ್ಯ-ವಾಗಿ
ದಿವ್ಯಾತ್ಮ-ದಲಿ
ದಿವ್ಯಾತ್ಮನು
ದಿವ್ಯಾ-ನು-ಭವ
ದೀಕ್ಷಾಗ್ರ-ಹಣ-ದಲ್ಲಿ
ದೀಕ್ಷೆ
ದೀಕ್ಷೆ-ಗೊಂಡಿ-ರುವ
ದೀಕ್ಷೆ-ಯನ್ನು
ದೀನ
ದೀನ-ತೆಯ
ದೀನ-ದುಃಖಿ-ಗಳ
ದೀನ-ಬಂಧು
ದೀನ-ಬಂಧೋ
ದೀನರ
ದೀನ-ರಿಗೆ
ದೀನರು
ದೀನರೋ
ದೀನಾಃ
ದೀಪ
ದೀಪಕ್ಕೆ
ದೀಪ-ಗಳನು
ದೀಪದ
ದೀಪ-ವನ್ನು
ದೀಪಸ್ತಂಭ-ದಲ್ಲಿ
ದೀಪ್ತ-ವಾದ
ದೀಯತಾಂ
ದೀರ್ಘ
ದೀರ್ಘ-ಕಾಲ
ದೀರ್ಘ-ಕಾಲದ
ದೀರ್ಘ-ಕಾಲ-ದಿಂದ
ದೀರ್ಘ-ಕಾಲ-ದಿಂದಲೂ
ದೀರ್ಘ-ದಂಡ
ದೀರ್ಘಧ್ಯಾನ-ದಿಂದ
ದೀರ್ಘ-ವಾಗಿ
ದೀರ್ಘ-ವಾದ
ದುಂದುಭಿ
ದುಂಬಿಯೇ
ದುಃಖ
ದುಃಖ-ಕರ-ವಾಗಿ-ರು-ವು-ದನ್ನೂ
ದುಃಖ-ಕರ-ವಾದ
ದುಃಖ-ಕಷ್ಟ-ಗಳ
ದುಃಖ-ಕಷ್ಟ-ಗಳಿವೆ
ದುಃಖಕ್ಕಾಗಿ
ದುಃಖ-ಗಂಜನ
ದುಃಖ-ಗಳ
ದುಃಖ-ಗಳನ್ನು
ದುಃಖ-ಗಳನ್ನೂ
ದುಃಖ-ಚಾಯಿ
ದುಃಖ-ಚಾವೋ
ದುಃಖದ
ದುಃಖ-ದಲ್ಲಿ
ದುಃಖ-ದಾಚೆಗೆ
ದುಃಖ-ದಾ-ಳದ
ದುಃಖ-ದಿಂದಲೂ
ದುಃಖ-ಧರ್ಮಾ-ಧರ್ಮ
ದುಃಖ-ಪಡು-ವು-ದಿಲ್ಲ
ದುಃಖ-ಭಾರ
ದುಃಖ-ಭಾರ-ದಲಿ
ದುಃಖ-ಭಾವ
ದುಃಖವ
ದುಃಖ-ವದು
ದುಃಖ-ವನು
ದುಃಖ-ವನ್ನರ-ಸುವನು
ದುಃಖ-ವನ್ನು
ದುಃಖ-ವನ್ನೂ
ದುಃಖ-ವನ್ನೇ
ದುಃಖ-ವಲ್ಲ
ದುಃಖ-ವಾಗುತ್ತದೆ
ದುಃಖ-ವಿದೆ
ದುಃಖ-ವಿರ-ಕೂ-ಡದು
ದುಃಖವೇ
ದುಃಖ-ಸಂಕಟ-ಗಳಲು
ದುಃಖ-ಸಂತಪ್ತ
ದುಃಖ-ಸಂತಾಪ-ಗಳಲ್ಲಿ
ದುಃಖ-ಹೋಗು-ವು-ದಾದರೂ
ದುಃಖಾ-ಟವಿ
ದುಃಖಿ-ಗಳೂ
ದುಃಖಿ-ಗಳೊ
ದುಃಖಿ-ಗಳೊ-ಡನೆ
ದುಃಖಿ-ತರಾ-ದಂತೆ
ದುಃಖಿನೀ
ದುಃಖಿ-ಯಾ-ಗು-ವು-ದಿಲ್ಲ
ದುಃಖಿ-ಸದ
ದುಃಖೆ
ದುಃಖೇರ್-ಪಾರ
ದುಃಸ್ಥಿತಿ
ದುಃಸ್ಥಿತಿಯ
ದುಟಿ
ದುಟಿ-ಕರ-ಬಂಚಾ
ದುಡಿ
ದುಡಿತ
ದುಡಿ-ದರೂ
ದುಡಿ-ಯ-ಬೇಕಾಗಿದೆ
ದುಡಿ-ಯಿತೊ
ದುಡಿ-ಯುತ್ತ
ದುಡಿ-ಯುತ್ತಲೇ
ದುಡಿ-ಯುತ್ತಾ
ದುಡಿ-ಯುತ್ತಾರೆ
ದುಡಿ-ಯುತ್ತಿ-ರು-ವುದು
ದುಡಿ-ವೆಲ್ಲ
ದುಡುಕುವೆ
ದುಡ್ಡನ್ನು
ದುಡ್ಡಿಲ್ಲದೆ
ದುಡ್ಡು
ದುಬಾರಿ-ಯಾ-ದಷ್ಟೂ
ದುರಂತ-ದಲ್ಲಿ
ದುರವಸ್ಥೆ
ದುರವಸ್ಥೆ-ಯನ್ನು
ದುರವಸ್ಥೆ-ಯುಂಟಾಗಿ
ದುರಾಚ-ರಣೆಯ
ದುರಾ-ಚಾರ
ದುರಾತ್ಮನು
ದುರಾಶೆ
ದುರಿತ-ಗಳನ್ನು
ದುರಿ-ತದ-ಲನ-ದಕ್ಷಂ
ದುರುಗುಟ್ಟಿ-ಕೊಂಡು
ದುರೇ
ದುರ್ಗತಿ
ದುರ್ಗತಿಂ
ದುರ್ಗಾ-ಪೂಜೆ
ದುರ್ಗಾ-ಪೂಜೆ-ಯಂತೆಯೆ
ದುರ್ಗಾ-ಪೂಜೆ-ಯನ್ನು
ದುರ್ಗುಣ-ಗಳೂ
ದುರ್ಗುಣ-ವೆನ್ನಿ-ಸಿ-ಕೊಳ್ಳುವುದು
ದುರ್ಘಟನೆ-ಯೊದಗಿ-ದರೆ
ದುರ್ಜನ-ರಾ-ದು-ದ-ರಿಂದ
ದುರ್ಜನರು
ದುರ್ಜ್ಞೇಯ-ವಾದ
ದುರ್ದಶೆ-ಯನ್ನು
ದುರ್ದಶೆಯೆ
ದುರ್ದೆಶೆಯು
ದುರ್ದೆಶೆಯುಂಟಾಗುತ್ತಿತ್ತೆ
ದುರ್ನಿ-ತಿಯ
ದುರ್ಬಲ
ದುರ್ಬಲತೆ
ದುರ್ಬಲ-ತೆಯ
ದುರ್ಬಲ-ತೆ-ಯನ್ನು
ದುರ್ಬಲ-ತೆ-ಯಿಂದಾ-ದು-ದೆಂದು
ದುರ್ಬಲ-ತೆಯೇ
ದುರ್ಬಲ-ನಾಗಿ-ರು-ವು-ದನ್ನು
ದುರ್ಬಲ-ನಾದ
ದುರ್ಬಲ-ರಾಗಿದ್ದಾರೆ
ದುರ್ಬಲ-ರಾದ
ದುರ್ಬಲರು
ದುರ್ಬಲ-ವಾಗಲು
ದುರ್ಬಲ-ವಾಗಿ
ದುರ್ಬಲ-ವಾಗಿದೆ
ದುರ್ಬಲ-ವಾಗಿ-ರು-ವುದು
ದುರ್ಬಲ-ವಾ-ಗುತ್ತಿದೆ
ದುರ್ಬಲ-ವಾದ
ದುರ್ಬಲ-ವಾದದ್ದೆಂದರೆ
ದುರ್ಭಾಗ್ಯ-ವೆನ್ನ
ದುರ್ಭಾವ
ದುರ್ಭಿಕ್ಷ
ದುರ್ಭಿಕ್ಷ-ಕಾಲ-ದಲ್ಲಿ
ದುರ್ಮಾರ್ಗ-ವರ್ತಿ-ಗ-ಳಾದರು
ದುರ್ಯೋಧ-ನನು
ದುರ್ಯೋ-ಧನನೂ
ದುರ್ಯೋ-ಧನನೆ
ದುರ್ಲಭ
ದುರ್ಲಭ-ವಾದ
ದುರ್ವಾ-ಸನೆ-ಯಿಂದ
ದುರ್ವ್ಯಸನಿ-ಗಳು
ದುಲಿಛೆ
ದುಶ್ಚರಿತ್ರ-ವಾಗಲಿ
ದುಶ್ಚಿಂತೆ-ಗಳು
ದುಷ್ಕಾಲ-ದಲ್ಲಿ
ದುಷ್ಕೃತ್ಯ-ದಲ್ಲಿ
ದುಷ್ಕೃತ್ಯ-ವನ್ನು
ದುಷ್ಕೃತ್ಯವು
ದುಷ್ಟ
ದುಸ್ಥೇ
ದುಸ್ಸಾಹಸ-ವನ್ನು
ದುಸ್ಸುಹರೆ
ದೂಡಿ
ದೂಡುವೆ
ದೂರ
ದೂರ-ಕರ
ದೂರಕೆ
ದೂರದ
ದೂರ-ದ-ಡವಿಯೊಳೆಲ್ಲಿ
ದೂರ-ದರ್ಶಿ-ಗಳು
ದೂರ-ದಲಿ
ದೂರ-ದಲ್ಲಿ
ದೂರ-ದಲ್ಲಿ-ರುವ
ದೂರ-ದಲ್ಲಿ-ರು-ವುದು
ದೂರ-ದಿಂದ
ದೂರ-ದಿಂದಲೇ
ದೂರ-ದೂರಕೆ
ದೂರ-ದೇಶ-ದಿಂದ
ದೂರ-ನಾಗುವೆ
ದೂರ-ನಿಲ್ಲಿ
ದೂರ-ಮಾಡು-ವಂತೆ
ದೂರ-ಮಾಡು-ವುದಕ್ಕೆ
ದೂರ-ಲಾರಿಹ-ರಿಲ್ಲಿ
ದೂರಲಿ
ದೂರ-ವಾಗ-ಲಿಲ್ಲ
ದೂರ-ವಾಗಲು
ದೂರ-ವಾಗಿದ್ದರು
ದೂರ-ವಾಗಿದ್ದಿತು
ದೂರ-ವಾ-ಗು-ವು-ದಿಲ್ಲ
ದೂರ-ವಾಯ್ತು
ದೂರ-ವಿತ್ತು
ದೂರ-ವಿ-ರ-ಬೇಕು
ದೂರ-ಸನಿಹ-ಗಳಲ್ಲಿ
ದೂರಾಗಿ
ದೂರಿ
ದೂರಿದ
ದೂರಿ-ದ-ವ-ರಲ್ಲ
ದೂರುವ
ದೂರು-ವಂತಿಲ್ಲ
ದೂರು-ವಿರಿ
ದೂಷಣೆ
ದೂಷಿ-ಸಿಲ್ಲ
ದೃಗ್
ದೃಢ
ದೃಢ-ಕಾಯ-ನಾದ
ದೃಢ-ಕಾ-ಯನು
ದೃಢ-ಕಾಯ-ರಾ-ಗಿ-ರುವರು
ದೃಢ-ಕಾಯ-ರಾದ
ದೃಢ-ಕಾ-ಯರು
ದೃಢ-ಕಾ-ಯರೂ
ದೃಢ-ಕಾಯ-ವಾದ
ದೃಢ-ಚಿತ್ತರು
ದೃಢತೆ
ದೃಢ-ನಿಶ್ಚಯ
ದೃಢ-ಪಡಿಸಿ
ದೃಢ-ಪಡಿ-ಸಿ-ದನು
ದೃಢ-ಪಡಿಸು
ದೃಢ-ಬದ್ಧ-ವಾದ
ದೃಢರು
ದೃಢ-ವಾಗಿ
ದೃಢ-ವಾಗಿದೆ
ದೃಢ-ವಾಗಿದ್ದು
ದೃಢ-ವಾಗುತ್ತದೆ
ದೃಢ-ವಾಗುತ್ತವೆ
ದೃಢ-ವಾದ
ದೃಢ-ವಾದಂತಾಯಿತು
ದೃಢೀ-ಕರಿ-ಸುತ್ತದೆ
ದೃಶ್ಯ
ದೃಶ್ಯ-ಕೋಟಿಯು
ದೃಶ್ಯ-ಗಳನ್ನು
ದೃಶ್ಯ-ಗಳು
ದೃಶ್ಯ-ಚಿತ್ರ-ಗಳ
ದೃಶ್ಯ-ರೂಪ-ವನ್ನು
ದೃಶ್ಯವ
ದೃಶ್ಯ-ವನ್ನು
ದೃಶ್ಯ-ವಸ್ತು-ವೆಂಬಂತೆ
ದೃಶ್ಯ-ವೆಲ್ಲವ
ದೃಷ್ಟಂ
ದೃಷ್ಟಾಂತ
ದೃಷ್ಟಾಂತ-ವನ್ನು
ದೃಷ್ಟಾಂತ-ವಾಗಿ
ದೃಷ್ಟಾಂತವು
ದೃಷ್ಟಾಂತ-ವೆಂದರೆ
ದೃಷ್ಟಿ
ದೃಷ್ಟಿ-ಕೋನ-ಗಳಿಂದ
ದೃಷ್ಟಿ-ಕೋನ-ವನ್ನು
ದೃಷ್ಟಿ-ಗಳಿಂದ
ದೃಷ್ಟಿ-ಗಳು
ದೃಷ್ಟಿಗೆ
ದೃಷ್ಟಿ-ದೋಷ-ವನ್ನು
ದೃಷ್ಟಿ-ಯನ್ನು
ದೃಷ್ಟಿ-ಯನ್ನೆಂದಿಗೂ
ದೃಷ್ಟಿ-ಯನ್ನೇ
ದೃಷ್ಟಿ-ಯಲೀಗ
ದೃಷ್ಟಿ-ಯಲ್ಲಿ
ದೃಷ್ಟಿ-ಯಿಂದ
ದೃಷ್ಟಿ-ಯಿಂದಲೂ
ದೃಷ್ಟಿ-ಯಿಂದಲೇ
ದೃಷ್ಟಿ-ಯಿಂದಲ್ಲ
ದೃಷ್ಟಿ-ಯಿಂದಾಚೆಗೆ
ದೃಷ್ಟಿ-ಯಿ-ಡದೆ
ದೃಷ್ಟಿ-ಯಿರ-ಬಾ-ರದೆಂದು
ದೃಷ್ಟಿ-ಯಿಲ್ಲ
ದೃಷ್ಟಿ-ಯಿಲ್ಲ-ದಿದ್ದರೆ
ದೃಷ್ಟಿ-ಯಿಲ್ಲದೆ
ದೃಷ್ಟಿಯು
ದೃಷ್ಟಿ-ಯುಳ್ಳ
ದೃಷ್ಟಿಯೂ
ದೃಷ್ಟ್ವಾ
ದೆವ್ವ
ದೆವ್ವದ
ದೆವ್ವ-ವನ್ನು
ದೆಶೆಯೆ
ದೆಸೆಗೆ
ದೆಸೆ-ಯಿಂದ
ದೇ
ದೇಖ
ದೇಖಾಯ್
ದೇಖಿ
ದೇಖಿತೆ
ದೇಖೀಬೆ
ದೇಖೆ
ದೇಖೇ
ದೇಖೇಚಿ
ದೇಖೊ
ದೇಖ್
ದೇಯ
ದೇಯ್
ದೇರೆ
ದೇವ
ದೇವ-ಗಣ
ದೇವ-ಗ-ಣದಿ
ದೇವ-ಗರ್ಪಿತ-ವಾದ
ದೇವ-ಘಡ-ದಲ್ಲಿದ್ದಾಗ
ದೇವ-ತರು
ದೇವ-ತೆ-ಗಳ
ದೇವ-ತೆ-ಗಳನ್ನಾಗಿ
ದೇವ-ತೆ-ಗಳನ್ನು
ದೇವ-ತೆ-ಗಳಾಗಲೀ
ದೇವ-ತೆ-ಗಳಾಗಿದ್ದರು
ದೇವ-ತೆ-ಗಳಿಗೂ
ದೇವ-ತೆ-ಗಳಿಗೆ
ದೇವ-ತೆ-ಗಳು
ದೇವ-ತೆ-ಗಳೂ
ದೇವ-ತೆ-ಗ-ಳೆಂದು-ಕೊಂಡರು
ದೇವ-ತೆ-ಗಳೆಲ್ಲ
ದೇವ-ತೆ-ಗಳೆಲ್ಲಾ
ದೇವ-ತೆ-ಗಳೊ
ದೇವ-ದಾರು
ದೇವ-ದೂತ-ರಾಗಲೀ
ದೇವ-ದೇವ-ರಿ-ಗೆಲ್ಲ
ದೇವ-ದೇವಿ-ಯ-ರನ್ನು
ದೇವ-ದೇವಿ-ಯರು
ದೇವ-ದೇವಿ-ಯ-ರೆಲ್ಲಾ
ದೇವನ
ದೇವ-ನದಿ-ಯ-ಮರ-ಗಾನದ
ದೇವ-ನನ್ನಾಗಿ
ದೇವ-ನಲಿ
ದೇವ-ನಲ್ಲಿಯು
ದೇವ-ನಾಗುತ್ತಾನೆ
ದೇವ-ನಾ-ವನು
ದೇವನು
ದೇವ-ನು-ಮನು-ಜ-ನುಪ್ರಾಣಿ-ಯಲ್ಲಿ
ದೇವನೆ
ದೇವ-ಭಾಷೆ
ದೇವ-ಭಾಷೆ-ಯಲ್ಲಿ
ದೇವ-ಮಾನ-ವ-ರನ್ನು
ದೇವ-ಮಾನ-ವರು
ದೇವರ
ದೇವ-ರಂತೆ
ದೇವ-ರ-ಗುಣ-ಗಳೇ
ದೇವ-ರ-ಡಿಗೆ
ದೇವ-ರನ್ನಾಗಲೀ
ದೇವ-ರನ್ನು
ದೇವ-ರ-ಮ-ನೆಗೆ
ದೇವ-ರ-ಮನೆ-ಯನ್ನು
ದೇವ-ರಲ್ಲದೆ
ದೇವ-ರಲ್ಲಿ
ದೇವ-ರಾಗ-ಲಾರ
ದೇವ-ರಾಗಲೀ
ದೇವ-ರಾ-ಗಿ-ರುವೆ
ದೇವ-ರಾಗುವನು
ದೇವ-ರಾದರೆ
ದೇವ-ರಿಂದ
ದೇವ-ರಿ-ಗಾಗಿ
ದೇವ-ರಿ-ಗಿಂತ
ದೇವ-ರಿಗೆ
ದೇವ-ರಿ-ಗೋಸ್ಕರ
ದೇವ-ರಿದ್ದಾ-ನೆಂದರೆ
ದೇವ-ರಿದ್ದಾ-ನೆಂದು
ದೇವ-ರಿದ್ದಾ-ನೆಂಬ
ದೇವ-ರಿರ-ಬ-ಹುದು
ದೇವ-ರಿ-ರುವನು
ದೇವ-ರಿ-ರುವನೇ
ದೇವ-ರಿರು-ವ-ರೆಗೂ
ದೇವ-ರಿಲ್ಲ
ದೇವ-ರಿಲ್ಲದೆ
ದೇವರು
ದೇವ-ರು-ಗಳನ್ನು
ದೇವ-ರು-ಗಳನ್ನೂ
ದೇವ-ರು-ಗಳು
ದೇವ-ರು-ಗಳೂ
ದೇವ-ರು-ಗಳೆಲ್ಲರೂ
ದೇವ-ರು-ಜಾಗೃತ-ವಾಗಿ-ರುವ
ದೇವರೂ
ದೇವರೆ
ದೇವ-ರೆಂದರೆ
ದೇವ-ರೆಂದು
ದೇವ-ರೆಂಬುವವ-ನನ್ನು
ದೇವ-ರೆ-ಡೆಗೆ
ದೇವ-ರೆನ್ನುತ
ದೇವರೇ
ದೇವ-ರೇನು
ದೇವ-ರೊಂದಿಗೆ
ದೇವ-ರೊಡನೆ
ದೇವ-ರೊಬ್ಬನು
ದೇವ-ರೊಬ್ಬನೇ
ದೇವ-ವಾಣಿ-ಗಳೆಲ್ಲಾ
ದೇವ-ವಾಣಿ-ಯನ್ನು
ದೇವ-ವಾಣಿ-ಯಿ-ರಲು
ದೇವಸ್ಥಾನ
ದೇವಸ್ಥಾನಕ್ಕೇ
ದೇವಸ್ಥಾನ-ಗಳನ್ನೊಳ-ಗೊಂಡ
ದೇವಸ್ಥಾನದ
ದೇವಸ್ಥಾನ-ದಲ್ಲಿ
ದೇವಸ್ಥಾನ-ವನ್ನು
ದೇವಾ-ಲಯ
ದೇವಾ-ಲಯದ
ದೇವಾ-ಲಯ-ದಲ್ಲಿ
ದೇವಾ-ಲಯ-ದೊಳ-ಗಡೆ
ದೇವಾ-ಲಯ-ವನ್ನು
ದೇವಿ
ದೇವಿ-ಯನ್ನು
ದೇವಿ-ಯರ
ದೇವೀ
ದೇವೇಚ್ಛೆ-ಯಿದ್ದರೆ
ದೇವೈರ್ಬಲಮ್
ದೇಶ
ದೇಶ-ಕಾಲ
ದೇಶ-ಕಾಲ-ಕಾರ್ಯ-ಕಾರ-ಣ-ಗಳೆಲ್ಲ-ದರ
ದೇಶ-ಕಾಲ-ಗಳನ್ನು
ದೇಶ-ಕಾಲ-ನಿಮಿತ್ತ
ದೇಶ-ಕಾಲಾತೀತ
ದೇಶಕ್ಕಾಗಿ
ದೇಶಕ್ಕುಂಟಾ-ಗಿ-ರುವ
ದೇಶಕ್ಕೂ
ದೇಶಕ್ಕೆ
ದೇಶಕ್ಕೇ
ದೇಶಕ್ಕೋಸ್ಕರ
ದೇಶ-ಗಳ
ದೇಶ-ಗಳನ್ನು
ದೇಶ-ಗಳಲ್ಲಿ
ದೇಶ-ಗಳಲ್ಲಿನ
ದೇಶ-ಗಳಲ್ಲಿಯೂ
ದೇಶ-ಗ-ಳಲ್ಲೂ
ದೇಶ-ಗಳಲ್ಲೆಲ್ಲಾ
ದೇಶ-ಗಳಿಂದ
ದೇಶ-ಗಳಿ-ಗಿಂತ
ದೇಶ-ಗಳಿ-ಗಿ-ರುವಂತಹ
ದೇಶ-ಗಳಿಗೆ
ದೇಶ-ಗಳೂ
ದೇಶ-ಗಳೆಲ್ಲ
ದೇಶ-ಗಳೊ-ಡನೆ
ದೇಶದ
ದೇಶ-ದಲ್ಲಾಗಲೀ
ದೇಶ-ದಲ್ಲಾದರೂ
ದೇಶ-ದಲ್ಲಾದರೊ
ದೇಶ-ದಲ್ಲಾದರೋ
ದೇಶ-ದಲ್ಲಿ
ದೇಶ-ದಲ್ಲಿಂದು
ದೇಶ-ದಲ್ಲಿದ್ದಾಗ
ದೇಶ-ದಲ್ಲಿ-ಯಾದರೋ
ದೇಶ-ದಲ್ಲಿಯೂ
ದೇಶ-ದಲ್ಲಿ-ರುವ
ದೇಶ-ದಲ್ಲೂ
ದೇಶ-ದಲ್ಲೆಲ್ಲಾ
ದೇಶ-ದವ-ರಂತೆಯೆ
ದೇಶ-ದವ-ರೊಡನೆ
ದೇಶ-ದಿಂದ
ದೇಶ-ದೆದುರಿಗಿರುವ
ದೇಶ-ದೊಡನೆ
ದೇಶ-ಭಕ್ತ
ದೇಶ-ಭಕ್ತಿ
ದೇಶ-ವನ್ನು
ದೇಶ-ವನ್ನೆಲ್ಲ
ದೇಶ-ವನ್ನೇ
ದೇಶ-ವಾಗಿ
ದೇಶ-ವಾಗಿದ್ದರೆ
ದೇಶ-ವಾದರೊ
ದೇಶ-ವಿದೇಶ-ಗಳ
ದೇಶವು
ದೇಶವೂ
ದೇಶ-ವೆಲ್ಲ
ದೇಶ-ವೆಲ್ಲಾ
ದೇಶವೇ
ದೇಶ-ವೇಕೆ
ದೇಶ-ವೇನೊ
ದೇಶ-ಹೀನ
ದೇಶಾಂತರ-ಗಳಿಗೆ
ದೇಶಾಚಾರ
ದೇಶಾಚಾರ-ಗಳಾಗಿ
ದೇಶಾತೀತ
ದೇಶಾದ್ಯಂತ
ದೇಶಾನು-ಸಾರ-ವಾಗಿ
ದೇಶಿ-ಯರ
ದೇಶೀಯ
ದೇಶೀ-ಯರ
ದೇಶೀಯ-ರಿಂದ
ದೇಶೀಯ-ರಿಗೆ
ದೇಶೀ-ಯರು
ದೇಹ
ದೇಹಕ್ಕಿಂತ
ದೇಹಕ್ಕಿ-ರ-ಲಿಲ್ಲ
ದೇಹಕ್ಕೆ
ದೇಹ-ಗಳಲ್ಲಿ
ದೇಹ-ಗಳಿ-ಗಿಂತ
ದೇಹತ್ಯಾಗ
ದೇಹತ್ಯಾಗ-ವಾ-ಯಿತು
ದೇಹದ
ದೇಹ-ದಂಡನಾ
ದೇಹ-ದಲ್ಲಿ
ದೇಹ-ದಲ್ಲಿ-ರುವ
ದೇಹ-ದಲ್ಲಿ-ರುವಾಗ
ದೇಹ-ದಲ್ಲೆಲ್ಲಾ
ದೇಹ-ದಲ್ಲೇ
ದೇಹ-ದಿಂದ
ದೇಹ-ಧಾರಣೆ
ದೇಹ-ಧಾರ-ಣೆ-ಮಾಡಿ
ದೇಹ-ಧಾರ-ಣೆ-ಯೊಂದು
ದೇಹ-ಪದೆ
ದೇಹ-ಬದ್ಧ-ವೆಂದು
ದೇಹ-ಭಾವನೆ
ದೇಹ-ಮನಸು-ಹೆಣ್ಗಂಡಲ್ಲ
ದೇಹ-ರೂಪ-ವಾದ
ದೇಹ-ವನೆ
ದೇಹ-ವನ್ನು
ದೇಹ-ವನ್ನೇ
ದೇಹ-ವಲ್ಲ
ದೇಹ-ವಾಗಿ
ದೇಹ-ವಾ-ದು-ದ-ರಿಂದ
ದೇಹ-ವಿದೆ
ದೇಹ-ವಿರು-ವ-ವ-ರೆಗೂ
ದೇಹ-ವಿರು-ವುದೋ
ದೇಹವು
ದೇಹ-ವುಳ್ಳ
ದೇಹವೂ
ದೇಹ-ವೆಂದು
ದೇಹ-ವೆಂಬ
ದೇಹವೇ
ದೇಹಶ್ರಮ
ದೇಹಶ್ರಮ-ಪಡು
ದೇಹ-ಸುಖವ
ದೇಹಸ್ಥಃ
ದೇಹಸ್ಥಿತಿ
ದೇಹಸ್ಥಿತಿಗೆ
ದೇಹಸ್ಥೋಽಪಿ
ದೇಹಾತೀ-ತ-ವಾಗು-ವುದಕ್ಕೆ
ದೇಹಾತ್ಮ
ದೇಹಾತ್ಮ-ಭಾವ-ದಲಿ
ದೇಹಾ-ರೋಗ್ಯದ
ದೇಹಾವಸಾನ-ವಾದ
ದೈತ್ಯ-ತರಂಗ-ಗಳೇ-ಳುತ
ದೈನಂದಿನ
ದೈನಿಕ
ದೈನಿಕ-ವನ್ನಾಗಿ
ದೈನ್ಯ
ದೈನ್ಯ-ದಿಂದ
ದೈನ್ಯ-ವನ್ನು
ದೈವ
ದೈವ-ಕಟಾಕ್ಷ-ದಿಂದ
ದೈವಜ್ಞರೂ
ದೈವದ
ದೈವ-ದತ್ತ-ವೆಂದು
ದೈವ-ದಿಂದಲೂ
ದೈವ-ನವ
ದೈವ-ನಿರ್ಭರತೆ
ದೈವರು
ದೈವ-ಲೀಲೆ-ಯಲ್ಲಿ
ದೈವ-ವನ್ನು
ದೈವ-ವಾಣಿಯು
ದೈವವು
ದೈವ-ಸಾಕ್ಷಾತ್ಕಾರ-ವಾಗದ
ದೈವೀ-ಗುಣ-ಸಂಪನ್ನ-ನಾದ
ದೈವೀಪ್ರೇಮಕ್ಕೆ
ದೈವೀಪ್ರೇಮದ
ದೈವೀ-ಭಾವ-ದಲ್ಲಿ
ದೈವೀ-ಮಾಯೆಯ
ದೈವೀ-ಲೀಲೆಯ
ದೈವೋನ್ಮಾದದಲ್ಲಿದ್ದ
ದೈಹಿಕ
ದೊಂಬಿಗೆ
ದೊಡ
ದೊಡ್ಡ
ದೊಡ್ಡ-ದಾಗಿ
ದೊಡ್ಡ-ದಾಗಿದೆ
ದೊಡ್ಡ-ದಾದ
ದೊಡ್ಡದು
ದೊಡ್ಡ-ದು-ಮಾಡಿ
ದೊಡ್ಡ-ದು-ಮಾಡಿ-ಕೊಂಡು
ದೊಡ್ಡ-ದೆಂದು
ದೊಡ್ಡ-ದೊಂದು
ದೊಡ್ಡ-ಪಾತಕ
ದೊಡ್ಡ-ವ-ನಾದರೆ
ದೊಡ್ಡ-ವನು
ದೊಡ್ಡ-ವರ
ದೊಡ್ಡ-ವ-ರಾಗಿ
ದೊಡ್ಡ-ವ-ರಾಗಿದ್ದ-ರಲ್ಲಾ
ದೊಡ್ಡ-ವ-ರಾಗಿದ್ದಾರೆ
ದೊಡ್ಡ-ವ-ರಾಗಿ-ರುತ್ತಾರೆಯೋ
ದೊಡ್ಡ-ವ-ರಾಗುತ್ತೀರಿ
ದೊಡ್ಡ-ವ-ರಾದ
ದೊಡ್ಡ-ವ-ರಾದರು
ದೊಡ್ಡ-ವರು
ದೊಡ್ಡ-ವ-ರೆಂದು
ದೊತ್ತ-ಡದ
ದೊರಕ-ದಿದ್ದರೆ
ದೊರ-ಕದುದೇ
ದೊರ-ಕಲಾರದು
ದೊರ-ಕ-ಲಿಲ್ಲ
ದೊರಕಲು
ದೊರಕಿತು
ದೊರಕಿತೆಂದು
ದೊರಕಿದ
ದೊರಕಿ-ದಂತಾಗಿದೆ
ದೊರಕಿ-ದರೆ
ದೊರಕಿ-ದೆಯೇ
ದೊರಕಿಸಿ-ಕೊಳ್ಳು-ವು-ದನ್ನು
ದೊರಕುತ್ತದೆ
ದೊರಕುವ
ದೊರಕು-ವಂತೆ
ದೊರಕುವ-ವರೆಗೆ
ದೊರಕುವುದ-ರಲ್ಲಿ
ದೊರಕು-ವು-ದಿಲ್ಲ
ದೊರಕು-ವು-ದಿಲ್ಲ-ವೆನ್ನಿ-ಸುತ್ತದೆ
ದೊರಕು-ವುದು
ದೊರಕು-ವು-ದೆಂದು
ದೊರೆ-ಗಳ
ದೊರೆ-ಗಳಿಲ್ಲದ
ದೊರೆ-ಗಳು
ದೊರೆತ
ದೊರೆ-ತರೂ
ದೊರೆ-ತರೆ
ದೊರೆತಿಲ್ಲ
ದೊರೆತೊಡ-ನೆಯೇ
ದೊರೆ-ಯದಿದ್ದರೆ
ದೊರೆ-ಯಲಿಲ್ಲ-ವೆಂಬುದು
ದೊರೆ-ಯಿತು
ದೊರೆ-ಯು-ವು-ದಿಲ್ಲ
ದೊರೆ-ಯು-ವುದು
ದೊರೆ-ಯು-ವುವು
ದೊಲೆ
ದೋಚುವ
ದೋಣಿ
ದೋಣಿ-ಗಳನ್ನು
ದೋಣಿಗೆ
ದೋಣಿಯ
ದೋಣಿ-ಯನ್ನು
ದೋಣಿ-ಯಲ್ಲಿ
ದೋಣಿ-ಯ-ವ-ನಿಗೆ
ದೋಣಿ-ಯ-ವನು
ದೋಣಿ-ಯಿಂದಿ-ಳಿದು
ದೋಣಿಯು
ದೋಣಿಯೂ
ದೋಣಿ-ಯೆ-ಡೆಗೆ
ದೋಣಿ-ಯೊ-ಳಕ್ಕೆ
ದೋರ್ಭ್ಯಾಮಿವ
ದೋಷ
ದೋಷಕ್ಕೆ
ದೋಷ-ಗಳನ್ನೆಲ್ಲಾ
ದೋಷ-ಗ-ಳಲ್ಲ
ದೋಷ-ಗಳಿ-ರು-ವುವು
ದೋಷ-ಗಳಿವೆ
ದೋಷ-ಗಳು
ದೋಷ-ಗಳೇ-ನಾದರೂ
ದೋಷ-ದಿಂದ
ದೋಷ-ದಿಂದಲೇ
ದೋಷ-ಪೂರ್ಣ-ವಾದದ್ದಾದರೂ
ದೋಷ-ವನೆ
ದೋಷ-ವನ್ನು
ದೋಷ-ವಿದ್ದರೂ
ದೋಷ-ವಿದ್ದೇ
ದೋಷವೂ
ದೋಷ-ವೆಂದರೆ
ದೋಷ-ವೆಂದು
ದೋಷ-ವೆಂಬುದು
ದೋಷವೇ
ದೋಷಾರೋ-ಪಣೆ-ಗಳ
ದೋಷಾರೋ-ಪಣೆ-ಗಳೆಲ್ಲಾ
ದೋಷಾರೋ-ಪಣೆ-ಯಿಂದ
ದೋಷೇಣ
ದೋಹಾ-ಕಾರ
ದೋಹಾಶ್ಲೋ-ಕ-ವನ್ನು
ದೋಹೆ-ಯಲ್ಲಿ-ರು-ವು-ದನ್ನು
ದೌರ್ಜನ್ಯ-ವನ್ನು
ದೌರ್ಬಲ್ಯ
ದೌರ್ಬಲ್ಯಕ್ಕೂ
ದೌರ್ಬಲ್ಯಕ್ಕೆ
ದೌರ್ಬಲ್ಯ-ಗಳಿ-ಗಿಂತಲೂ
ದೌರ್ಬಲ್ಯದ
ದೌರ್ಬಲ್ಯ-ದಿಂದಲೆ
ದೌರ್ಬಲ್ಯ-ದಿಂದೆ-ದೆಯು
ದೌರ್ಬಲ್ಯ-ದೊಂದಿಗೆ
ದೌರ್ಬಲ್ಯವೂ
ದೌರ್ಬಲ್ಯವೇ
ದ್ರವ್ಯ
ದ್ರವ್ಯ-ಕಣ-ಗಳು
ದ್ರವ್ಯಕ್ಕೆ
ದ್ರವ್ಯ-ಗಳೆ-ರಡೂ
ದ್ರವ್ಯದ
ದ್ರವ್ಯ-ದಂತೆಯೂ
ದ್ರವ್ಯ-ವನ್ನು
ದ್ರವ್ಯ-ವನ್ನೇ
ದ್ರವ್ಯವು
ದ್ರವ್ಯವೇ
ದ್ರವ್ಯ-ಸಂಚಯನ-ವನ್ನೂ
ದ್ರವ್ಯ-ಸಂಪಾದನೆ
ದ್ರವ್ಯ-ಸಹಾಯ-ವನ್ನೂ
ದ್ರವ್ಯಾರ್ಜನೆ
ದ್ರಷ್ಟಾರನಿರ-ಬ-ಹುದು
ದ್ರಾಕ್ಷಾ-ಫಲ
ದ್ರುಪದ
ದ್ವಂದ್ವ
ದ್ವಂದ್ವ-ಗಳನ್ನು
ದ್ವಂದ್ವ-ಗಳಲ್ಲಿ
ದ್ವಂದ್ವ-ಗಳು
ದ್ವಂದ್ವದ
ದ್ವಂದ್ವ-ಭಾವ-ಗಳಿಗೆಲ್ಲಾ
ದ್ವಂದ್ವ-ಭಾವ-ಗಳೆಲ್ಲಾ
ದ್ವಂದ್ವ-ಯುದ್ಧ
ದ್ವಂದ್ವ-ರಾಜ್ಯದ
ದ್ವನುಕ
ದ್ವಾರ-ಕೆ-ಯಲ್ಲಿ
ದ್ವಾರೆ
ದ್ವಿಗುಣ-ವಾಗುತ್ತಾ
ದ್ವಿಜರ
ದ್ವಿಜರಾ-ದಿರಿ
ದ್ವಿಜ-ರಿಗೆ
ದ್ವಿಜರು
ದ್ವೀಪ
ದ್ವೀಪ-ಪುಂಜ-ಗಳಲ್ಲಿ
ದ್ವೇಷ
ದ್ವೇಷ-ಗಳಲ್ಲಿ
ದ್ವೇಷ-ದಿಂದ
ದ್ವೇಷ-ಪರಾಯಣ-ರಲ್ಲ
ದ್ವೇಷ-ಬುದ್ಧಿ-ಯನ್ನು
ದ್ವೇಷ-ಭಾವನೆ
ದ್ವೇಷ-ಭಾವ-ನೆ-ಯಿಂದ
ದ್ವೇಷ-ರಾಗ-ಗಳಿಲ್ಲ
ದ್ವೇಷ-ವೆಂದೂ
ದ್ವೇಷ-ವೆಲ್ಲ
ದ್ವೇಷಿಸ-ಬಾ-ರದು
ದ್ವೇಷಿ-ಸಲು
ದ್ವೇಷಿಸಿ
ದ್ವೇಷಿ-ಸಿಯೇ
ದ್ವೇಷಿ-ಸುತ್ತಿದ್ದೆ
ದ್ವೇಷಿ-ಸು-ವು-ದಿಲ್ಲ
ದ್ವೈತ
ದ್ವೈತ-ದಿಂದ
ದ್ವೈತ-ಭಾವ
ದ್ವೈತ-ಭಾವ-ಗಳಿಲ್ಲ
ದ್ವೈತ-ಭಾವನೆ
ದ್ವೈತ-ಭಾವ-ನೆ-ಯಿಂದ
ದ್ವೈತ-ಭಾವ-ವಿ-ರುತ್ತದೆ
ದ್ವೈತ-ಭೂಮಿಗೆ
ದ್ವೈತ-ಭೂಮಿ-ಯಲ್ಲಿ
ದ್ವೈತ-ವಿ-ಹೀನ-ವಾದ
ದ್ವೈತಾತೀತ-ವಾದ
ದ್ವೈತಿ-ಗಳು
ದ್ವೈತಿಯೂ
ಧಕ್
ಧಕ್ಕೆ
ಧಗಧ-ಗನೆ
ಧಗೆಗೆ-ಳ-ಸುವ-ನಾರು
ಧತ್ತೇ
ಧನ
ಧನ-ಉ-ಪಾರ್ಜನ
ಧನದಿಂದ
ಧನ-ದಿಂದಾ-ಗಲಿ
ಧನ-ಧಾನ್ಯ-ಗಳನ್ನು
ಧನ-ಧಾನ್ಯ-ಗಳು
ಧನ-ಧಾನ್ಯ-ವನ್ನು
ಧನ-ಪಿಶಾಚಿ-ಗಳಾಗಿ
ಧನ-ಲಕ್ಷ್ಮಿ-ಯನ್ನು
ಧನ-ವರ್ಜನೆ
ಧನಾರ್ಜನೆ-ಯನ್ನು
ಧನಿಕ-ನಾಗಲು
ಧನುರ್ಧಾರಿ-ಯಾದ
ಧನೇನ
ಧನ್ಯ
ಧನ್ಯ-ತಾಸ್ವ-ರೂಪದ್ದಾಗಿ-ರುತ್ತದೆ
ಧನ್ಯ-ತೆಯ
ಧನ್ಯ-ತೆ-ಯನ್ನು
ಧನ್ಯ-ನಾಗಿದ್ದೇನೆ
ಧನ್ಯ-ನಾಗುತ್ತಿದ್ದೆ
ಧನ್ಯ-ನಾಗುವೆ
ಧನ್ಯ-ನಾದೆ
ಧನ್ಯ-ನಾದೆ-ನೆಂದು
ಧನ್ಯ-ನೆಂದು
ಧನ್ಯ-ರನ್ನಾಗಿ
ಧನ್ಯ-ರಾಗ-ದ-ವರ
ಧನ್ಯ-ರಾಗೋಣ
ಧನ್ಯ-ರಾದ
ಧನ್ಯ-ರಾದೆ-ವೆಂದು
ಧನ್ಯ-ರಾದೇವು
ಧನ್ಯರು
ಧನ್ಯಳು
ಧನ್ಯ-ವಾದ-ಗಳೇ
ಧನ್ಯವ್ಯವ-ಹಾರ
ಧನ್ಯಾವಸ್ಥೆ
ಧಮನಿ-ಗಳಲ್ಲಿ
ಧರಣ್ಯಾಂ
ಧರತ
ಧರ-ವಾಸೆ
ಧರಾ
ಧರಾ-ತಲ
ಧರಾ-ಪಟೆ
ಧರಾ-ಬಪು
ಧರಿ
ಧರಿಯೆ
ಧರಿ-ಸದೆ
ಧರಿ-ಸ-ಬೇಕು
ಧರಿ-ಸಲು
ಧರಿಸಿ
ಧರಿ-ಸಿ-ಕೊಂಡಿರು-ವ-ವ-ಳಾದ
ಧರಿ-ಸಿ-ಕೊಂಡಿ-ರು-ವುದಕ್ಕೆ
ಧರಿ-ಸಿ-ಕೊಂಡು
ಧರಿ-ಸಿ-ಕೊಳ್ಳಲು
ಧರಿ-ಸಿದ
ಧರಿ-ಸಿ-ದಾಗಲೂ
ಧರಿ-ಸಿ-ರುವ
ಧರಿ-ಸಿ-ರುವ-ವಳೇ
ಧರಿ-ಸಿ-ರು-ವು-ದ-ರಿಂದ
ಧರಿ-ಸಿ-ಹೋಗು-ವಿರಿ
ಧರಿ-ಸುತ
ಧರಿ-ಸುವ
ಧರಿ-ಸುವನು
ಧರಿ-ಸುವಾಗ
ಧರಿ-ಸು-ವುದು
ಧರಿ-ಸುವೆ
ಧರೆ
ಧರೆ-ಗಿಳಿ-ಯು-ವುದು
ಧರೆ-ಯಲ್ಲ
ಧರೆ-ಯಾಗಸ-ಗಳ
ಧರೆ-ಯಾಗಸ-ಗಳು
ಧರ್ತ್ಥುಂ
ಧರ್ಮ
ಧರ್ಮಃ
ಧರ್ಮ-ಅ-ಧರ್ಮ-ಗಳ
ಧರ್ಮಕ್ಕಾಗಿ
ಧರ್ಮಕ್ಕೂ
ಧರ್ಮಕ್ಕೆ
ಧರ್ಮ-ಗಳ
ಧರ್ಮ-ಗಳನ್ನೂ
ಧರ್ಮ-ಗಳಲ್ಲಿ
ಧರ್ಮ-ಗಳಲ್ಲೆಲ್ಲಾ
ಧರ್ಮ-ಗಳಷ್ಟೇ
ಧರ್ಮ-ಗಳು
ಧರ್ಮ-ಗಳೂ
ಧರ್ಮ-ಗಳೆ-ರಡೂ
ಧರ್ಮ-ಗಳೆಲ್ಲ
ಧರ್ಮ-ಗಳೇ
ಧರ್ಮ-ಗಳೇನೋ
ಧರ್ಮ-ಗಳೊ-ಳಗೆ
ಧರ್ಮ-ಗುರು
ಧರ್ಮ-ಗುರು-ಗಳಲ್ಲೆಲ್ಲ
ಧರ್ಮಗ್ರಂಥ-ಕಾ-ರರು
ಧರ್ಮಗ್ರಂಥ-ಗಳ
ಧರ್ಮಗ್ರಂಥ-ಗಳನ್ನೇ
ಧರ್ಮಗ್ರಂಥ-ಗಳಲ್ಲಿ
ಧರ್ಮಗ್ರಂಥ-ಗಳಲ್ಲಿ-ರುವ
ಧರ್ಮಗ್ರಂಥ-ಗ-ಳಲ್ಲೂ
ಧರ್ಮಗ್ರಂಥ-ಗ-ಳಾದ
ಧರ್ಮಗ್ರಂಥ-ಗಳು
ಧರ್ಮಗ್ರಂಥ-ದಲ್ಲಿ
ಧರ್ಮ-ತತ್ತ್ವ-ಗಳ
ಧರ್ಮ-ತರ
ಧರ್ಮ-ತರು-ವಿ-ನಲ್ಲಿ
ಧರ್ಮದ
ಧರ್ಮ-ದಂತಹ
ಧರ್ಮ-ದ-ಮೇಲೆ
ಧರ್ಮ-ದಲ್ಲಿ
ಧರ್ಮ-ದಲ್ಲಿಯೇ
ಧರ್ಮ-ದಲ್ಲಿ-ರುವ
ಧರ್ಮ-ದಾ-ನ-ಗಳನ್ನು
ಧರ್ಮ-ದಿಂದ
ಧರ್ಮ-ನಿಷ್ಠರು
ಧರ್ಮ-ಪರಾಯಣ-ರನ್ನಾಗಿಯೂ
ಧರ್ಮ-ಪರಾಯಣ-ರಾ-ದಷ್ಟೂ
ಧರ್ಮ-ಪಿಪಾಸು-ಗ-ಳಾದ
ಧರ್ಮ-ಪಿಪಾಸೆ
ಧರ್ಮಪ್ರ-ಚಾರ
ಧರ್ಮಪ್ರ-ಚಾರ-ಕ-ರನ್ನು
ಧರ್ಮಪ್ರ-ಚಾರ-ಕರು
ಧರ್ಮಪ್ರ-ಚಾರದ
ಧರ್ಮಪ್ರ-ಭಾವಿಭಾಸಿತ-ವಾದ
ಧರ್ಮಪ್ರಾಣವು
ಧರ್ಮ-ಬೋಧ-ಕರ
ಧರ್ಮ-ಬೋಧೆ
ಧರ್ಮ-ಭಾವದ
ಧರ್ಮ-ಭಾ-ವವು
ಧರ್ಮ-ಮತ-ಗಳೂ
ಧರ್ಮ-ಮಾರ್ಗಕ್ಕೆ
ಧರ್ಮ-ಮಾರ್ಗ-ವನ್ನು
ಧರ್ಮ-ಯುದ್ಧ
ಧರ್ಮ-ಲಾಭದ
ಧರ್ಮ-ಲಾಭ-ದಲ್ಲಿ
ಧರ್ಮ-ವನ್ನರ-ಸುತಲಿ
ಧರ್ಮ-ವನ್ನರಿ-ಯು-ವಂತೆ
ಧರ್ಮ-ವನ್ನವ-ಲಂಬಿಸ-ತೊಡಗಿ-ದಾಗ
ಧರ್ಮ-ವನ್ನವ-ಲಂಬಿಸಿದ
ಧರ್ಮ-ವನ್ನವ-ಲಂಬಿ-ಸಿದ್ದ
ಧರ್ಮ-ವನ್ನವ-ಲಂಬಿಸಿ-ರುವ
ಧರ್ಮ-ವನ್ನು
ಧರ್ಮ-ವನ್ನೂ
ಧರ್ಮ-ವನ್ನೇ
ಧರ್ಮ-ವಾಗಿತ್ತು
ಧರ್ಮ-ವಾಗು-ವುದು
ಧರ್ಮ-ವಾದ
ಧರ್ಮ-ವಾಹಿನಿ
ಧರ್ಮ-ವಿಲ್ಲ
ಧರ್ಮ-ವೀರರು
ಧರ್ಮವು
ಧರ್ಮವೂ
ಧರ್ಮವೆ
ಧರ್ಮ-ವೆಂದ-ರೇನು
ಧರ್ಮ-ವೆಂದು
ಧರ್ಮ-ವೆಂಬ
ಧರ್ಮ-ವೆಂಬುದು
ಧರ್ಮವೇ
ಧರ್ಮ-ವೇ-ನೆಂಬು-ದನ್ನು
ಧರ್ಮ-ವೊಂದನ್ನು
ಧರ್ಮ-ವೊಂದು
ಧರ್ಮ-ವೊಂದೇ
ಧರ್ಮ-ಶಾಲೆ-ಯನ್ನು
ಧರ್ಮ-ಶಾಸ್ತ್ರ-ಗಳನ್ನು
ಧರ್ಮ-ಶಾಸ್ತ್ರ-ಗಳನ್ನೆಲ್ಲಾ
ಧರ್ಮ-ಶಾಸ್ತ್ರ-ಗಳನ್ನೋದು-ವು-ದ-ರಿಂದ
ಧರ್ಮ-ಶಾಸ್ತ್ರ-ಗಳು
ಧರ್ಮ-ಶಾಸ್ತ್ರ-ವನ್ನು
ಧರ್ಮ-ಶಾಸ್ತ್ರಾನು-ಸಾರ-ವಾಗಿ
ಧರ್ಮ-ಶಿಕ್ಷಣಕ್ಕಾಗಿ
ಧರ್ಮ-ಶೀಲಃ
ಧರ್ಮ-ಶೀಲ-ನಾಗ-ಬೇಕು
ಧರ್ಮ-ಶೀಲ-ರೆಲ್ಲಾ
ಧರ್ಮ-ಸಂಪ್ರದಾಯ-ದಲ್ಲಿಯೂ
ಧರ್ಮ-ಸಂಬಂಧ-ವಾದ
ಧರ್ಮ-ಸಂಸ್ಥಾಪ-ಕನೂ
ಧರ್ಮ-ಸಂಸ್ಥೆ-ಯೊಂದನ್ನು
ಧರ್ಮ-ಸಾಕ್ಷಾತ್ಕಾರ
ಧರ್ಮ-ಸಾಧ-ನಕ್ಕೆ
ಧರ್ಮ-ಸಾ-ಧನೆ-ಗಳು
ಧರ್ಮ-ಸೂತ್ರ-ಗಳನ್ನು
ಧರ್ಮಸ್ಯ
ಧರ್ಮಾಂಧತೆ
ಧರ್ಮಾಂಧ-ರೆಂದು
ಧರ್ಮಾತ್ಮ-ರಾಗು-ವುದಕ್ಕೆ
ಧರ್ಮಾತ್ಮರು
ಧರ್ಮಾ-ಧರ್ಮ
ಧರ್ಮಾ-ಧರ್ಮ-ಗಳನ್ನು
ಧರ್ಮಾನು-ಯಾಯಿ-ಗಳಲ್ಲಿಯೂ
ಧರ್ಮಾರ್ಥ
ಧರ್ಮಾಲಾ-ಪಾದಿ-ಗಳನ್ನು
ಧರ್ಮೋಪ-ದೇಶ-ವನ್ನು
ಧವಲ-ಕಮಲ-ಶೋಭಃ
ಧವಳ
ಧಾಟಿ-ಯಲ್ಲೇ
ಧಾತಾ
ಧಾತು
ಧಾಯ
ಧಾರ
ಧಾರಣ
ಧಾರ-ಣ-ಮಾಡಿ
ಧಾರ-ಣೆ-ಗಳನ್ನೂ
ಧಾರಾ
ಧಾರಾಒ
ಧಾರಾ-ಳ-ವಾಗಿ
ಧಾರಿಣಿ
ಧಾರೆ
ಧಾರೆಯ
ಧಾರೆ-ಯಂಥ
ಧಾರೆ-ಯದು
ಧಾರೆ-ಯ-ದೊಂದೆ
ಧಾರೆ-ಯೆರೆ-ಯು-ವು-ದಕ್ಕೂ
ಧಾರ್ಮಿಕ
ಧಾರ್ಮಿಕ-ಜೀವಿ
ಧಾರ್ಮಿ-ಕತೆ
ಧಾರ್ಮಿಕ-ನಾಗ-ಲಾರ
ಧಾರ್ಮಿಕ-ರನ್ನಾಗಿ
ಧಾರ್ಮಿಕ-ರಾಗ-ಬೇಕೆಂದು
ಧಾವಿ-ಸುತ್ತಿದೆ
ಧಾವಿ-ಸು-ವಂತೆ
ಧಿಃಕರಿ-ಸುತ್ತೀರಿ
ಧಿಕ್
ಧಿಕ್ಕಾರ
ಧೀರ
ಧೀರ-ಚಿಂತ-ಕನು
ಧೀರ-ತನ-ದಲಿ
ಧೀರ-ನ-ವ-ನಿಗೆ
ಧೀರ-ನಿಗೆ
ಧೀರ-ರ-ವರು
ಧೀರ-ರಾಗು-ವ-ವ-ರೆಗೂ
ಧೀರ-ರಾದ
ಧೀರರು
ಧೀರಾ
ಧೀರಾಃ
ಧೀರಾತ್ಮ
ಧೀರಾತ್ಮ-ನಲ್ಲಿ-ರುವ
ಧೀರಾತ್ಮ-ನಿಗೆ
ಧೀರಾತ್ಮನೆ
ಧೀರೆ
ಧುಮುಕಿ
ಧುಮುಕಿ-ದರೆ
ಧೂಮ
ಧೂಮ-ಕೇತು
ಧೂಮ-ಕೇತುವೊ
ಧೂಮ-ಪಾನ
ಧೂಮ-ವಲ-ಯ-ವನ್ನು
ಧೂಮಾವೃತ-ದಲಿ
ಧೂಮೆ
ಧೂಮೇ-ನಾಗ್ನಿರಿವಾವೃತಾಃ
ಧೂರ್ತಚೋ-ರರು
ಧೂಳೀಪ-ಟ-ವಾಗಿದೆ
ಧೂಳು
ಧೃತ-ಕರ್ಮ-ಪಾಶಾ
ಧೇ
ಧೈರ್ಯ
ಧೈರ್ಯ-ದಂತೆ
ಧೈರ್ಯ-ದಲಿ
ಧೈರ್ಯ-ದಿಂದ
ಧೈರ್ಯ-ದಿಂದಿದ-ನೆಲ್ಲ-ರಾಲಿಸೆ
ಧೈರ್ಯ-ದಿಂದಿರು
ಧೈರ್ಯ-ವನ್ನು
ಧೈರ್ಯ-ವಾಗಿ
ಧೈರ್ಯ-ವಿದ್ದುದು
ಧೈರ್ಯ-ವಿ-ರ-ಲಿಲ್ಲ
ಧೈರ್ಯ-ವಿಲ್ಲ
ಧೈರ್ಯ-ವಿಲ್ಲ-ದವ-ರಾಗಿದ್ದೇವೆ
ಧೈರ್ಯ-ಶಾಲಿ-ಗಳಾಗುವುವು
ಧೈರ್ಯ-ಹೀನ-ವಾಗಿದೆ
ಧೋತ್ರ
ಧ್ಯಾತೃ
ಧ್ಯಾತ್ವಾ
ಧ್ಯಾನ
ಧ್ಯಾನಕ್ಕೆ
ಧ್ಯಾನದ
ಧ್ಯಾನ-ದಲ್ಲಿ
ಧ್ಯಾನ-ದಲ್ಲಿದ್ದ
ಧ್ಯಾನ-ದಲ್ಲಿದ್ದೆ
ಧ್ಯಾನ-ದಲ್ಲಿ-ರುವಾಗ
ಧ್ಯಾನ-ದಿಂದ
ಧ್ಯಾನ-ಧಾರಣ
ಧ್ಯಾನ-ಧಾರ-ಣ-ಗಳಲ್ಲಿ
ಧ್ಯಾನ-ನಿರತ-ರಾಗಿದ್ದು-ದನ್ನು
ಧ್ಯಾನ-ಮಗ್ನ-ರಾಗಿದ್ದಂತೆ
ಧ್ಯಾನ-ಮಗ್ನ-ರಾದರು
ಧ್ಯಾನ-ಮಾಡ-ಬೇ-ಕಾದರೆ
ಧ್ಯಾನ-ಮಾ-ಡಲು
ಧ್ಯಾನ-ಮಾಡು
ಧ್ಯಾನ-ಮಾಡು-ವುದಕ್ಕೆ
ಧ್ಯಾನ-ಮಾಡು-ವುದು
ಧ್ಯಾನ-ವನ್ನು
ಧ್ಯಾನ-ವಿಲ್ಲ
ಧ್ಯಾನ-ಸಿದ್ಧ-ನಾಗುತ್ತಾ-ನೆಯೋ
ಧ್ಯಾನಸ್ಥ-ರಾದರು
ಧ್ಯಾನಸ್ಥ-ವಾಗುತ್ತ-ದೆಂದು
ಧ್ಯಾನಾರೂಢ-ರಾಗಿದ್ದರು
ಧ್ಯಾನಾಸಕ್ತ
ಧ್ಯಾನಾಸಕ್ತ-ನಾಗು-ವನೋ
ಧ್ಯಾನಿ
ಧ್ಯಾನಿ-ಸಲ್ಪಡುತ್ತಿರುವ
ಧ್ಯಾನಿ-ಸಿದರೆ
ಧ್ಯಾನಿಸು
ಧ್ಯಾನಿ-ಸುತ್ತಿದ್ದಾಗ
ಧ್ಯಾಯ-ಮಾನಃ
ಧ್ಯೂಲೋಕ
ಧ್ಯೇಯ
ಧ್ಯೇಯಕ್ಕಾಗಿ
ಧ್ಯೇಯಕ್ಕೆ
ಧ್ಯೇಯ-ಗಳ
ಧ್ಯೇಯ-ಗಳಾವುವೂ
ಧ್ಯೇಯ-ಗಳಾವು-ವೆಂಬು-ದರ
ಧ್ಯೇಯ-ಗಳಿ-ರು-ವುವೋ
ಧ್ಯೇಯ-ಗಳೆಲ್ಲ
ಧ್ಯೇಯ-ಗಳೇ
ಧ್ಯೇಯದ
ಧ್ಯೇಯ-ದಲ್ಲಿ
ಧ್ಯೇಯ-ದಿಂದ
ಧ್ಯೇಯ-ವನ್ನು
ಧ್ಯೇಯ-ವಾಗಿದ್ದ
ಧ್ಯೇಯ-ವಾಗಿ-ರ-ಬೇಕು
ಧ್ಯೇಯ-ವಾ-ಗಿ-ರಲಿ
ಧ್ಯೇಯ-ವಿ-ರುವಾಗ
ಧ್ಯೇಯವೇ
ಧ್ಯೇಯ-ಸಾ-ಧನೆಗೆ
ಧ್ಯೇಯೋದ್ದೇಶ-ಗಳನ್ನು
ಧ್ರುಪದ್
ಧ್ರುಪದ್-ಗಳಲ್ಲಿದೆ
ಧ್ರುವ-ದಲ್ಲಿ-ರುತ್ತಾರೆ
ಧ್ರುವ-ಮಿದಂ
ಧ್ವಂಸ
ಧ್ವಂಸದ
ಧ್ವಂಸ-ಪಡಿ-ಸ-ತಕ್ಕದ್ದು
ಧ್ವಂಸ-ಮಾಡಿ
ಧ್ವಂಸ-ಮಾಡು-ವುದು
ಧ್ವಂಸ-ವಲ್ಲ
ಧ್ವಂಸ-ವಾಗಿ
ಧ್ವಂಸ-ವಾದರೂ
ಧ್ವಜ
ಧ್ವಜ-ಧಾರಿಯ
ಧ್ವನಿ
ಧ್ವನಿ-ಗಳ
ಧ್ವನಿ-ತ-ವಾ-ದದ್ದು
ಧ್ವನಿ-ಯಂತೆ
ಧ್ವನಿ-ಯನ್ನು
ಧ್ವನಿ-ಯಲ್ಲಿ
ಧ್ವನಿ-ಯಿಂದ
ಧ್ವನಿ-ಯಿಂದಲೂ
ಧ್ವನಿಯು
ಧ್ವನಿ-ಯೆದ್ದಿತು
ಧ್ವನಿಯೇ
ಧ್ವನಿ-ಯೊಂದು
ನ
ನಂಜು
ನಂತರ
ನಂತರವೇ
ನಂದ-ನ-ವನ-ದಂತೆ
ನಂದಿ
ನಂಬದೆ
ನಂಬದೇ
ನಂಬ-ಬೇಕೆಂದು
ನಂಬ-ಬೇಡ
ನಂಬ-ಬೇಡಿ
ನಂಬರ್
ನಂಬಲ-ಸಾಧ್ಯ-ವಾ-ದದ್ದು
ನಂಬಲಿಲ್ಲವೋ
ನಂಬಿ
ನಂಬಿಕೆ
ನಂಬಿ-ಕೆ-ಗಳಿಂದ
ನಂಬಿ-ಕೆ-ಗಳೆಂದರೆ
ನಂಬಿ-ಕೆಗೆ
ನಂಬಿ-ಕೆಯ
ನಂಬಿ-ಕೆ-ಯನ್ನಿಟ್ಟು-ಕೊಳ್ಳ-ಬೇ-ಕೇನು
ನಂಬಿ-ಕೆ-ಯನ್ನು
ನಂಬಿ-ಕೆ-ಯನ್ನೆಲ್ಲ
ನಂಬಿ-ಕೆ-ಯಿಂದಲೂ
ನಂಬಿ-ಕೆ-ಯಿಂದಲೆ
ನಂಬಿ-ಕೆ-ಯಿಟ್ಟು
ನಂಬಿ-ಕೆ-ಯಿಡ-ದಿದ್ದರೆ
ನಂಬಿ-ಕೆ-ಯಿದೆ
ನಂಬಿ-ಕೆ-ಯಿಲ್ಲದೆ
ನಂಬಿ-ಕೆ-ಯುಂಟಾಗಿ
ನಂಬಿ-ಕೆಯೂ
ನಂಬಿ-ಕೊಂಡಿ-ರು-ವು-ದೇ-ನೆಂದರೆ
ನಂಬಿ-ಕೊಂಡು
ನಂಬಿ-ಕೊಂಡೆ
ನಂಬಿದ್ದೆ
ನಂಬಿ-ರುವ
ನಂಬಿ-ಹೆನು
ನಂಬು
ನಂಬುಗೆ
ನಂಬು-ಗೆ-ಯನ್ನಿಟ್ಟು-ಕೊಂಡಿದ್ದರು
ನಂಬು-ಗೆ-ಯನ್ನು
ನಂಬು-ಗೆ-ಯಾ-ಗು-ವು-ದಿಲ್ಲ
ನಂಬು-ಗೆ-ಯಿತ್ತು
ನಂಬು-ಗೆ-ಯುಂಟಾ-ಗುತ್ತದೆ
ನಂಬುತ್ತಾರೆ
ನಂಬುತ್ತಿದ್ದ-ನಯ್ಯ
ನಂಬುತ್ತಿದ್ದರು
ನಂಬುತ್ತಿ-ರ-ಲಿಲ್ಲ
ನಂಬುತ್ತೇನೆ
ನಂಬುತ್ತೇವೆ
ನಂಬು-ವಂತಹ
ನಂಬು-ವಂತೆ
ನಂಬು-ವನೋ
ನಂಬು-ವರು
ನಂಬು-ವ-ವನು
ನಂಬು-ವವ-ನೇನು
ನಂಬು-ವಿರಾ
ನಂಬು-ವುದಕ್ಕೆ
ನಂಬು-ವು-ದಿಲ್ಲ
ನಂಬು-ವುದು
ನಂಬು-ವುದೂ
ನಂಬು-ವು-ದೇನು
ನಂಬುವೆ
ನಂಬು-ವೆನು
ನಂಬು-ವೆ-ಯೇನು
ನಕಲು
ನಕಾಶೆ-ಯನ್ನು
ನಕಾಸೆ
ನಕ್ಕರು
ನಕ್ಕಾಡ್
ನಕ್ಕು
ನಕ್ತಂ
ನಕ್ಷತ್ರ-ಗಳ
ನಕ್ಷತ್ರ-ಮುಕುಟ-ಗಳ
ನಗ-ತೊಡಗಿ-ದರು
ನಗ-ದಾಗಿ
ನಗ-ಬೇಕು
ನಗರ
ನಗರ-ಗಳಲ್ಲಿ
ನಗರ-ದಲ್ಲಿ
ನಗರಿ-ಯಾದ
ನಗ-ಲಾ-ರಂಭಿ-ಸಿ-ದರು
ನಗಾರಿ
ನಗಿ-ಸು-ವು-ದನ್ನು
ನಗು
ನಗು-ತಿದ್ದರೂ
ನಗುತ್ತ
ನಗುತ್ತಾರೆ
ನಗುತ್ತಿದ್ದರು
ನಗು-ನಗುತ್ತ
ನಗು-ಮುಖ
ನಗು-ಮುಖ-ದಿಂದ
ನಗು-ವರು
ನಗು-ವರು-ಳಿದ-ವ-ರೆಲ್ಲರೂ
ನಗು-ವಿನ
ನಗು-ವಿಲ್ಲ
ನಗು-ವುದಕ್ಕೆ
ನಗು-ವೆಯೊ
ನಗೆಯ
ನಗ್ನದಿಕ್-ವಾಸ
ನಗ್ನ-ರಾಗಿ
ನಚಿಕೇ-ತನ
ನಚಿಕೇ-ತನಿಗಿದ್ದಂತಹ
ನಟರು
ನಟಿ-ಸುವೆ
ನಡ-ತೆಯ
ನಡ-ತೆ-ಯನ್ನು
ನಡನಡುಗಿ
ನಡ-ವ-ಳಿಕೆ-ಗಳು
ನಡ-ವ-ಳಿಕೆಗೂ
ನಡ-ವ-ಳಿಕೆ-ಯನ್ನು
ನಡ-ವ-ಳಿಕೆ-ಯನ್ನೂ
ನಡಸೇ
ನಡುಗ-ತೊಡಗಿತು
ನಡುಗಿಪುದು
ನಡುಗಿಸಿ
ನಡುಗಿ-ಸು-ವು-ದಕ್ಕೂ
ನಡುಗಿ-ಸು-ವುದು
ನಡು-ಬಿಸಿ-ಲಿನ
ನಡು-ಮನೆ-ಯಲ್ಲಿ
ನಡುವಣ
ನಡು-ವ-ಳಿಕೆಯೇ
ನಡು-ವಿನದು
ನಡು-ವಿನಿಂ
ನಡುವೆ
ನಡು-ಹಗಲು
ನಡೆ
ನಡೆದ
ನಡೆ-ದ-ರ-ವರು
ನಡೆ-ದರೆ
ನಡೆ-ದಾಗ
ನಡೆ-ದಿದೆ
ನಡೆ-ದಿ-ರು-ವೆವು
ನಡೆ-ದಿರೆ
ನಡೆ-ದಿಲ್ಲವೋ
ನಡೆ-ದಿಹ
ನಡೆ-ದಿಹ-ನೊಬ್ಬ
ನಡೆ-ದೀತು
ನಡೆ-ದೀತೆ
ನಡೆದು
ನಡೆ-ದು-ಕೊಂಡು
ನಡೆ-ದು-ಕೊಂಡೆ
ನಡೆ-ದು-ಕೊಳ್ಳ-ದಿದ್ದರೆ
ನಡೆ-ದು-ಕೊಳ್ಳ-ಬೇಕು
ನಡೆ-ದು-ಕೊಳ್ಳ-ಬೇಕೆಂದು
ನಡೆ-ದು-ಬರೆ
ನಡೆ-ದುವು
ನಡೆ-ದುವೋ
ನಡೆ-ದೇನೋ
ನಡೆ-ನುಡಿ
ನಡೆ-ನುಡಿ-ಗಳಲ್ಲಿ
ನಡೆ-ನುಡಿ-ಗಳೆಲ್ಲಾ
ನಡೆ-ಯದಿದ್ದರೆ
ನಡೆ-ಯ-ಬಲ್ಲದು
ನಡೆ-ಯ-ಬ-ಹುದು
ನಡೆ-ಯ-ಬೇ-ಕಾದ
ನಡೆ-ಯ-ಬೇಕು
ನಡೆ-ಯ-ಲಾರದು
ನಡೆ-ಯಲಿ
ನಡೆ-ಯಲು
ನಡೆ-ಯಿತಷ್ಟೆ
ನಡೆ-ಯಿತು
ನಡೆ-ಯಿತೊ
ನಡೆ-ಯಿತೋ
ನಡೆ-ಯುತ್ತ
ನಡೆ-ಯುತ್ತದೆ
ನಡೆ-ಯುತ್ತಲೂ
ನಡೆ-ಯುತ್ತಲೇ
ನಡೆ-ಯುತ್ತವೆ
ನಡೆ-ಯುತ್ತಾ
ನಡೆ-ಯುತ್ತಾನೆ
ನಡೆ-ಯುತ್ತಾರೆ
ನಡೆ-ಯುತ್ತಿತ್ತು
ನಡೆ-ಯುತ್ತಿತ್ತೆಂದರೆ
ನಡೆ-ಯುತ್ತಿದೆ
ನಡೆ-ಯುತ್ತಿದೆಯೋ
ನಡೆ-ಯುತ್ತಿದ್ದ
ನಡೆ-ಯುತ್ತಿದ್ದದ್ದನ್ನು
ನಡೆ-ಯುತ್ತಿದ್ದವು
ನಡೆ-ಯುತ್ತಿದ್ದಾಗ
ನಡೆ-ಯುತ್ತಿದ್ದುವು
ನಡೆ-ಯುತ್ತಿರು-ವಾಗ
ನಡೆ-ಯುತ್ತಿ-ರು-ವುದು
ನಡೆ-ಯುತ್ತಿಲ್ಲ-ವಲ್ಲಾ
ನಡೆ-ಯುತ್ತಿವೆ
ನಡೆ-ಯುತ್ತೇನೆ
ನಡೆ-ಯುವ
ನಡೆ-ಯು-ವಂತೆ
ನಡೆ-ಯುವಾಗ
ನಡೆ-ಯು-ವುದಕ್ಕಾ-ಗಲಿಲ್ಲ
ನಡೆ-ಯು-ವುದಕ್ಕೆ
ನಡೆ-ಯು-ವು-ದಿಲ್ಲ
ನಡೆ-ಯು-ವುದು
ನಡೆ-ಯು-ವು-ದೊಂದೇ
ನಡೆ-ಯು-ವೆನೊ
ನಡೆ-ಯೆಲೆ
ನಡೆಯೊ
ನಡೆವ
ನಡೆ-ವ-ಳಿಕೆ
ನಡೆ-ವೆಲ್ಲ
ನಡೆ-ಸ-ತೊಡಗಿದ
ನಡೆ-ಸ-ಬೇಕೆಂದು
ನಡೆ-ಸಲು
ನಡೆ-ಸಲ್ಪಟ್ಟ
ನಡೆ-ಸಲ್ಪಡುತ್ತಿದ್ದಾರೆ
ನಡೆಸಿ
ನಡೆ-ಸಿ-ಕೊಂಡು
ನಡೆ-ಸಿದ
ನಡೆ-ಸಿ-ದರು
ನಡೆ-ಸಿ-ದರೂ
ನಡೆ-ಸಿ-ದರೆ
ನಡೆ-ಸಿಯೂ
ನಡೆ-ಸುತ್ತಾರೆ
ನಡೆ-ಸುತ್ತಿದ್ದರು
ನಡೆ-ಸುತ್ತಿದ್ದಾನೆ
ನಡೆ-ಸುತ್ತಿ-ರುವರು
ನಡೆ-ಸುತ್ತಿ-ರು-ವಿರಿ
ನಡೆ-ಸುತ್ತಿ-ರು-ವು-ದ-ರಿಂದ
ನಡೆ-ಸುತ್ತಿವೆ
ನಡೆ-ಸುವ
ನಡೆ-ಸುವರು
ನಡೆ-ಸು-ವುದಕ್ಕಾಗಿ
ನಡೆ-ಸು-ವು-ದಕ್ಕೂ
ನಡೆ-ಸು-ವುದಕ್ಕೆ
ನಡೆ-ಸು-ವು-ದನ್ನು
ನಡೆ-ಸು-ವುದು
ನಡೆ-ಸು-ವುದೆಂದೇ
ನಡೆಸೆ
ನತ-ನಯನ-ನಿ-ಯುಕ್ತಂ
ನತ್ವಾ
ನದ
ನದ-ನದಿ-ಸ-ರಸಿ-ಗಳವು
ನದಿ
ನದಿಗೆ
ನದಿಯ
ನದಿ-ಯೊ-ಡನೆ
ನದೀ
ನದೀ-ತೀರ-ದಲ್ಲಿ
ನನ-ಗಂತೂ
ನನ-ಗದಕ್ಕಿಂತಲೂ
ನನ-ಗನ್ನಿಸಿ-ದರೂ
ನನ-ಗನ್ನಿ-ಸುತ್ತಿದೆ
ನನ-ಗನ್ನಿ-ಸು-ವು-ದಿಲ್ಲ
ನನ-ಗನ್ನಿ-ಸು-ವುದು
ನನಗರ್ಥ-ವಾಗುತ್ತದೆ
ನನಗರ್ಥ-ವಾ-ಗು-ವು-ದಿಲ್ಲ
ನನ-ಗಾ-ಗಲಿ
ನನ-ಗಾಗಿ
ನನ-ಗಾವ
ನನ-ಗಿಂತ
ನನ-ಗಿದೆ
ನನ-ಗಿದ್ದರೆ
ನನ-ಗಿ-ರುವ
ನನ-ಗಿಲ್ಲ
ನನ-ಗಿಷ್ಟ-ವಿಲ್ಲ-ದಿದ್ದರೂ
ನನಗೂ
ನನಗೆ
ನನಗೆ-ರಡು
ನನಗೆಷ್ಟು
ನನಗೇ
ನನ-ಗೇ-ನಂತೆ
ನನ-ಗೇ-ನಾಯಿ-ತೆಂಬ
ನನ-ಗೇನು
ನನ-ಗೇನೂ
ನನ-ಗೇನೊ
ನನ-ಗೇನೋ
ನನ-ಗೊಂದು
ನನ-ಗೋಸ್ಕರ
ನನಸು
ನನ್ನ
ನನ್ನಂತರಾತ್ಮ-ನಿಗೆ
ನನ್ನಂತಹ
ನನ್ನಂತಹ-ನಿಗೆ
ನನ್ನ-ದಾಗಿ-ರಲು
ನನ್ನ-ದಿದೊ
ನನ್ನದು
ನನ್ನದೇ
ನನ್ನ-ನಿನ್ನಯ
ನನ್ನನು
ನನ್ನ-ನು-ಳಿದು
ನನ್ನನ್ನು
ನನ್ನನ್ನೆ
ನನ್ನನ್ನೇ
ನನ್ನಯ
ನನ್ನ-ರಿ-ವಿನೆಚ್ಚರಕೆ
ನನ್ನಲ್ಲಿ
ನನ್ನಲ್ಲಿಗೆ
ನನ್ನಲ್ಲಿದ್ದಿದ್ದರೆ
ನನ್ನಲ್ಲಿಯೂ
ನನ್ನಲ್ಲೀಗ
ನನ್ನಲ್ಲೂ
ನನ್ನಲ್ಲೇ
ನನ್ನ-ವ-ನೆಂದು
ನನ್ನಷ್ಟಕ್ಕೆ
ನನ್ನಷ್ಟೇ
ನನ್ನಿಂದ
ನನ್ನಿ-ದಿರು
ನನ್ನಿ-ಯರಿವಾ-ನಂದ-ವಾಹಿನಿ-ಯಲ್ಲಿ
ನನ್ನಿಷ್ಟಾನು-ಸಾರ
ನನ್ನೀ
ನನ್ನೆಡೆ
ನನ್ನೆದೆಗೆ
ನನ್ನೆ-ದೆಯೊ-ಳಡಗಿದ್ದ
ನನ್ನೆಲ್ಲಾ
ನನ್ನೊಡನಿ-ರು-ವಿರಿ
ನನ್ನೊ-ಡನೆ
ನನ್ನೊ-ಡನೆಯೆ
ನನ್ನೊ-ಡನೆಯೇ
ನನ್ನೊಲವೆ
ನನ್ನೊಳ-ವನು
ನಪ್ಪ-ಬ-ಹುದು
ನಭ-ವಲ್ಲ
ನಮಃ
ನಮ-ಗಾಗಿ
ನಮಗಾಗುವ
ನಮ-ಗಿಂತ
ನಮಗಿಲ್ಲ-ವೆಂಬು-ದೇನೋ
ನಮಗೀಗ
ನಮಗುಂಟಾ-ಗಿ-ರುವ
ನಮಗೂ
ನಮಗೆ
ನಮ-ಗೆಲ್ಲಾ
ನಮಗೇ
ನಮಗೇನು
ನಮಗೇನೂ
ನಮಸ್ಕ-ರಿಸಿ
ನಮಸ್ಕರಿ-ಸಿ-ದರು
ನಮಸ್ಕರಿ-ಸಿದೆ
ನಮಸ್ಕರಿ-ಸುತ್ತೇನೆ
ನಮಸ್ಕಾರ
ನಮಸ್ಕಾರ-ಗಳು
ನಮಸ್ಕಾರ-ಮಾಡಿ
ನಮಸ್ಕಾರ-ಮಾಡುವ
ನಮಾಮಃ
ನಮಿ
ನಮಿ-ಸು-ವೆನು
ನಮೂನೆ
ನಮೋ
ನಮ್ಮ
ನಮ್ಮಂತಿತ್ತು
ನಮ್ಮಂತೆ
ನಮ್ಮಂತೆಯೇ
ನಮ್ಮಂಥ
ನಮ್ಮ-ಗಳೆ-ದು-ರಿಗೆ
ನಮ್ಮ-ದ-ರಂಥ
ನಮ್ಮ-ದಾದರೋ
ನಮ್ಮದು
ನಮ್ಮದೇ
ನಮ್ಮ-ದೊಂದು
ನಮ್ಮ-ದೊಂದೇ
ನಮ್ಮನ್ನ-ಗಲಿ
ನಮ್ಮನ್ನಾಳು-ವವರು
ನಮ್ಮನ್ನು
ನಮ್ಮನ್ನೆಲ್ಲಾ
ನಮ್ಮನ್ನೇ
ನಮ್ಮಲ್ಲಿ
ನಮ್ಮಲ್ಲಿಗೆ
ನಮ್ಮಲ್ಲಿನ
ನಮ್ಮಲ್ಲಿಯೇ
ನಮ್ಮಲ್ಲಿ-ರುವ
ನಮ್ಮಲ್ಲಿ-ರು-ವಂತೆ
ನಮ್ಮಲ್ಲಿಲ್ಲ
ನಮ್ಮಲ್ಲೂ
ನಮ್ಮ-ವರು
ನಮ್ಮ-ವು-ಗಳಾಗಿ-ಬಿಡು-ವುವು
ನಮ್ಮಷ್ಟು
ನಮ್ಮಷ್ಟೇ
ನಮ್ಮಾತ್ಮ-ನಾಶಕೆ
ನಮ್ಮಿಂದ
ನಮ್ಮಿಂದಲೇ
ನಮ್ಮಿಬ್ಬರ
ನಮ್ಮಿರ್ವ-ರೊ-ಳಗೆ
ನಮ್ಮೆಲ್ಲರ
ನಮ್ಮೆಲ್ಲ-ರಿಗೂ
ನಮ್ಮೆಲ್ಲಾ
ನಮ್ಮೊ-ಡನೆ
ನಮ್ರ-ತೆಯ
ನಮ್ರ-ತೆ-ಯನ್ನು
ನಮ್ರ-ಭಾವ-ನೆ-ಯನ್ನೂ
ನಮ್ರ-ಭಾವ-ವನ್ನು
ನಯ-ಗೊಳಿಸಿ-ಕೊಳ್ಳುವುದು
ನಯ-ಗೊಳಿಸು
ನಯನ
ನಯನ-ಕುಸುಮ-ಗಳ
ನಯನ-ಗಳಲ್ಲಿ
ನಯನ-ದಿಂದ
ನಯ-ವಾಗಿ
ನರ
ನರಕ
ನರ-ಕಕ್ಕೆ
ನರ-ಕ-ಗಳಾಸೆಭಯ-ಗಳ-ನೆಲ್ಲ
ನರ-ಗಳ
ನರ-ಗಳಿಗೂ
ನರ-ಗಳೂ
ನರ-ಜನ್ಮ-ವನ್ನು
ನರ-ದೇವ
ನರನ
ನರ-ನಾರಿ-ಯರಿ-ರಲು
ನರ-ನಾರಿ-ಯರು
ನರ-ನಾರಿಯೂ
ನರ-ನಾರೀ
ನರನು
ನರ-ಬಲಿ
ನರ-ಭಕ್ಷಣೆ
ನರ-ಮಾಂಸ-ವನ್ನು
ನರ-ರೂಪ-ಧರ
ನರ-ಳ-ಬೇಕಾಗಿತ್ತು
ನರ-ಳಾಟ-ಕಂಬ-ನಿಯ
ನರ-ಳುತ್ತ
ನರ-ಳುತ್ತಿದ್ದರು
ನರ-ಳುತ್ತಿದ್ದೆ
ನರ-ಳುತ್ತಿಲ್ಲ
ನರ-ಳುವ
ನರ-ಳು-ವು-ದನ್ನು
ನರವರ
ನರ-ಶರೀರ-ದಲ್ಲಿಲ್ಲ
ನರ-ಶುದ್ಧಿ
ನರಿ-ಗಳಂತೆ
ನರಿ-ಯನ್ನೂ
ನರಿ-ಯಾಗಲು
ನರೇಂದ್ರ-ನಾಥ
ನರೇಂದ್ರ-ನಾಥ-ಸೇ-ನರು
ನರೇಂದ್ರ-ಬಾಬು
ನರೇಂದ್ರ-ಬಾಬು-ಗಳು
ನರೇನ-ನನ್ನು
ನರೇನ-ನಲ್ಲಿ
ನರೇ-ನನು
ನರೇನ್
ನರ್ತನ
ನರ್ತನ-ವನು
ನರ್ತಿ-ಸಲಿ
ನರ್ತಿ-ಸುವನು
ನರ್ತಿ-ಸುವುದ
ನರ್ಮದಾ
ನರ್ಮದಾ-ನ-ದಿಯ
ನಲ-ವತ್ತು
ನಲಿ-ದಾ-ಡಿದರು
ನಲಿನ-ಲಿ-ಯುತ
ನಲಿನೀ-ದಲ-ಗತ
ನಲಿ-ಯುತ್ತಿರು-ವ-ವ-ರೊಡನೆ
ನಲಿ-ವಾಗ
ನಲಿ-ವಿನಲಿ
ನಲಿ-ವಿಲ್ಲ
ನಲಿವು
ನಲಿ-ವುದೊ
ನಲಿ-ವೆಲ್ಲ
ನಲು-ಗಿದ್ದರೂ
ನಲ್ವತ್ತು
ನವ-ಗೋಪಾಲ
ನವ-ಚೇ-ತನವ
ನವ-ಚೈ-ತನ್ಯ
ನವ-ಜಾತ
ನವಪ್ರಾಣ
ನವ-ಶಕ್ತಿ
ನವ-ಶೈಲಿ-ಯಲ್ಲಿ
ನವೀನ-ವಾಗಿ-ರು-ವು-ದನ್ನು
ನವೆಂಬರ್
ನವೆಂಬರ್-ನಲ್ಲಿ
ನಶಿಸಿ
ನಶಿಸಿ-ಹೋ-ಯಿತು
ನಶಿ-ಸುತಿಹ
ನಶ್ವರ
ನಷ್ಟ-ಗಳ
ನಷ್ಟ-ಗಳನ್ನು
ನಷ್ಟ-ಪಡಿ-ಸ-ಲಿಲ್ಲ
ನಷ್ಟ-ವಾದ
ನಷ್ಟ-ವಾದರೂ
ನಷ್ಟ-ವೇನು
ನಸು-ನ-ಗುತ್ತಾ
ನಹಿ
ನಹಿ-ದೇಯಿ
ನಹೆ
ನಹೇ
ನಾ
ನಾಂತರ್ಬಹಿಶ್ಚ
ನಾಕ
ನಾಕ-ನರ-ಕ-ಗಳು
ನಾಕ-ನರ-ಕ-ಗಳೆಲ್ಲ
ನಾಗ-ಮಹಾ-ಶಯ
ನಾಗ-ಮಹಾ-ಶಯರ
ನಾಗ-ಮಹಾ-ಶಯ-ರಂತಿ-ರಲಿ
ನಾಗ-ಮಹಾ-ಶಯ-ರಂಥ
ನಾಗ-ಮಹಾ-ಶಯ-ರನ್ನು
ನಾಗ-ಮಹಾ-ಶಯ-ರಲ್ಲಿ
ನಾಗ-ಮಹಾ-ಶಯ-ರಿಗೂ
ನಾಗ-ಮಹಾ-ಶಯ-ರಿಗೆ
ನಾಗ-ಮಹಾ-ಶಯರು
ನಾಗ-ಮಹಾ-ಶಯರೂ
ನಾಗ-ಮಹಾ-ಶಯರೆ
ನಾಗ-ಮಹಾ-ಶಯ-ರೊಡನೆ
ನಾಗ-ಮಹಾ-ಶಯ-ರೊಬ್ಬ-ರಲ್ಲಿ
ನಾಗರ
ನಾಗರಿಕ
ನಾಗರಿ-ಕತೆ
ನಾಗರಿಕ-ತೆಗೂ
ನಾಗರಿಕ-ತೆಯ
ನಾಗರಿಕ-ತೆ-ಯನ್ನು
ನಾಗರಿಕ-ತೆ-ಯನ್ನೂ
ನಾಗರಿಕ-ತೆ-ಯಲ್ಲಿ
ನಾಗರಿಕ-ತೆ-ಯಲ್ಲಿಯೂ
ನಾಗರಿಕ-ತೆ-ಯೊ-ಡನೆ
ನಾಗರಿಕ-ರಾಗಲು
ನಾಗರಿಕ-ವಾ-ಯಿತೆಂದು
ನಾಚತ
ನಾಚಿ
ನಾಚಿಕೆ
ನಾಚಿ-ಕೆ-ಯಾಗಿ
ನಾಚಿ-ಕೆ-ಯಾ-ಯಿತು
ನಾಚಿ-ಕೆ-ಯಿಂದ
ನಾಚುಕ್
ನಾಚೇ
ನಾಜರೆತ್ನ
ನಾಟಕ
ನಾಟಕ-ಕಾರ
ನಾಟ-ಕದ
ನಾಟಕ-ವನ್ನು
ನಾಟ-ಕೀಯ-ವಾಗಿ
ನಾಟು-ವಂತೆ
ನಾಟ್ಯ
ನಾಟ್ಯ-ಕಲೆ
ನಾಟ್ಯ-ಕಲೆಯ
ನಾಟ್ಯ-ದಲ್ಲಿ
ನಾಡ-ದೆಲ್ಲವು
ನಾಡಿ
ನಾಡಿ-ಗಳನ್ನು
ನಾಡಿ-ಗಳಲ್ಲಿ
ನಾಡಿನ
ನಾಡಿ-ನಲ್ಲಿ
ನಾಡಿ-ನಾಡಿ-ಗ-ಳಲ್ಲೂ
ನಾಡಿಯ
ನಾಡು-ಗಳಲ್ಲಿ
ನಾಣ್ಯ-ಗಳನ್ನು
ನಾಣ್ಯದ
ನಾತ್ರ
ನಾಥ
ನಾಥನೊ
ನಾದ
ನಾದದ
ನಾದ-ದಿಂದ
ನಾದ-ವನ್ನು
ನಾದವು
ನಾದಾತ್ಮ-ಕ-ವಾಗಿ-ಬಿಡುತ್ತದೆ
ನಾದಿ
ನಾನಂತೂ
ನಾನ-ಕರು
ನಾನ-ದನ್ನು
ನಾನದೆಷ್ಟೋ
ನಾನರಿಯೆ
ನಾನಲ್ಲ
ನಾನಲ್ಲಿಗೆ
ನಾನ-ವನ
ನಾನ-ವ-ರನ್ನು
ನಾನವ-ರನ್ನೊಮ್ಮೆ
ನಾನಹೆ
ನಾನಾ
ನಾನಾ-ಗಲಿ
ನಾನಾ-ಗಲೇ
ನಾನಾ-ಗಿ-ರುವೆ
ನಾನಾ-ದರೋ
ನಾನಾ-ರನ್ನೂ
ನಾನಾ-ವು-ದನ್ನು
ನಾನಾ-ವು-ದಾದರೂ
ನಾನಾಸ್ತಿ
ನಾನಿಂದು
ನಾನಿಚ್ಛೆ-ಪಟ್ಟರೆ
ನಾನಿದ
ನಾನಿ-ದನ್ನು
ನಾನಿದು-ವರೆ-ವಿಗೂ
ನಾನಿರುತಿ-ಹೆನು
ನಾನಿ-ರುವೆ
ನಾನಿ-ರುವೆ-ನಲ್ಲಿಯೂ
ನಾನಿ-ರು-ವೆನು
ನಾನೀ
ನಾನೀಗ
ನಾನೀ-ರೀತಿ
ನಾನು
ನಾನು-ಗಳು
ನಾನೂ
ನಾನೂರು
ನಾನೆ
ನಾನೆಂದು
ನಾನೆಂದೂ
ನಾನೆಂದೆಂಬ
ನಾನೆಂಬ
ನಾನೆಂಬು-ದ-ರಿಂದ
ನಾನೆಂಬು-ದಿಲ್ಲದೇ
ನಾನೆನ್ನುತ್ತ
ನಾನೆಲ್ಲಕ್ಕೂ
ನಾನೆಲ್ಲಾದರೂ
ನಾನೆಲ್ಲಿ
ನಾನೆಷ್ಟು
ನಾನೆಷ್ಟೋ
ನಾನೇ
ನಾನೇ-ಕಾಂಗಿ
ನಾನೇಕೆ
ನಾನೇನ
ನಾನೇ-ನಾಗು-ವೆ-ನೆಂಬು-ದನ್ನು
ನಾನೇ-ನಾದರೂ
ನಾನೇನು
ನಾನೇನೂ
ನಾನೇನೋ
ನಾನೊಂದು
ನಾನೊಪ್ಪು-ವುದೇ
ನಾನೊಬ್ಬ
ನಾನೊಬ್ಬನೇ
ನಾನ್ಯಃ
ನಾಮ
ನಾಮ-ಕೀರ್ತನೆ-ಗಳು
ನಾಮ-ಜ-ಪ-ವನ್ನು
ನಾಮ-ದೇಯ
ನಾಮ-ರೂಪ
ನಾಮ-ರೂಪ-ಗಳ
ನಾಮ-ರೂಪ-ಗಳನ್ನು
ನಾಮ-ರೂಪ-ಗಳನ್ನೊಳ-ಗೊಂಡಿದೆ
ನಾಮ-ರೂಪ-ಗಳಿಂದ
ನಾಮ-ರೂಪ-ಗಳಿಗೆ
ನಾಮ-ರೂಪ-ಗಳಿಲ್ಲ
ನಾಮ-ರೂಪ-ಗಳಿಲ್ಲದೇ
ನಾಮ-ರೂಪ-ಗಳಿವೆಯೋ
ನಾಮ-ರೂಪ-ಗಳು
ನಾಮ-ರೂಪ-ಗಳೇ
ನಾಮ-ರೂಪವ
ನಾಮ-ರೂಪಾ-ತೀತ-ನಾ-ತನು
ನಾಮ-ರೂಪಾ-ತೀ-ತವು
ನಾಮ-ರೂಪಾತ್ಮಕ-ವಾದ
ನಾಮ-ವನ್ನು
ನಾಮ-ವ-ರಣ
ನಾಮವು
ನಾಮ-ಸಂಕೀರ್ತನೆ
ನಾಮಾ-ಮೃತ-ವನ್ನು
ನಾಮೋಚ್ಛಾ-ರಣೆ
ನಾಮೋಚ್ಛಾ-ರಣೆ-ಯನ್ನು
ನಾಯಕ
ನಾಯಕನ
ನಾಯಕ-ನನ್ನು
ನಾಯಕ-ನಾಗು-ವುದಕ್ಕೆ
ನಾಯ-ಕನು
ನಾಯ-ಕರ
ನಾಯ-ಕರು
ನಾಯಕ-ರೆಂದು
ನಾಯ-ಮತ್ಮಾ
ನಾಯಮಾತ್ಮಾ
ನಾಯಿ
ನಾಯಿ-ಗಳು
ನಾಯಿ-ಗಿಂತಲೂ
ನಾರದೀಯ
ನಾರಾಯಣ
ನಾರಾಯಣ-ಗಂಜಿ-ನಲ್ಲಿ
ನಾರಾಯಣ-ನಿಗೆ
ನಾರಾಯ-ಣನು
ನಾರಾಯಣ-ನೆಂಬ
ನಾರಾಯ-ಣರು
ನಾರಾಯಣರೂ
ನಾರಾಯಣ-ರೆಂದು
ನಾರಾಯಣ-ರೆಂಬುದ-ರಲ್ಲಿ
ನಾರಿ-ಯರ
ನಾರಿ-ಯಲ್ಲೂ
ನಾರೀ
ನಾರುತ್ತಿರುವ
ನಾರುವ
ನಾಲಕೈದು
ನಾಲ-ಗೆಯ
ನಾಲಗೆ-ಯನ್ನು
ನಾಲಗೆ-ಯಲ್ಲಿ
ನಾಲ-ಗೆಯುಮಲ್ಲ
ನಾಲ್ಕ-ನೆಯ
ನಾಲ್ಕನೇ
ನಾಲ್ಕು
ನಾಲ್ಕು-ಜನ
ನಾಲ್ಕೂ
ನಾಲ್ಕೂ-ಕಡೆ
ನಾಲ್ಕೂ-ವರೆ
ನಾಲ್ಕೈದು
ನಾಲ್ವರ
ನಾಲ್ವ-ರಿಗೆ
ನಾಳಿ-ನಿಂದಲೇ
ನಾಳೆ
ನಾಳೆ-ಯಿಂದ
ನಾವಗಲು-ವುದೆ
ನಾವ-ದನ್ನು
ನಾವಾದರೊ
ನಾವಿಂದು
ನಾವಿ-ಕರು
ನಾವಿಟ್ಟಿ-ರುವ
ನಾವಿದ್ದ
ನಾವಿನ್ನಾ-ರನ್ನೂ
ನಾವಿನ್ನೂ
ನಾವಿಬ್ಬರೂ
ನಾವಿಬ್ಬರೇ
ನಾವಿ-ರುವಾಗ
ನಾವಿ-ರುವೆಡೆ-ಯಲ್ಲಿಯೇ
ನಾವಿಲ್ಲಿ
ನಾವೀ
ನಾವೀಗ
ನಾವೀನ್ಯತೆ-ಯೆಂದರೆ
ನಾವು
ನಾವೂ
ನಾವೆಂದಿಗೂ
ನಾವೆಂದೂ
ನಾವೆ-ಯಿ-ಹುದು
ನಾವೆಲ್ಲ
ನಾವೆಲ್ಲರೂ
ನಾವೆಲ್ಲಾ
ನಾವೆಷ್ಟು
ನಾವೇ
ನಾವೇಕೆ
ನಾವೇ-ನನ್ನು
ನಾವೇ-ನಾದರೂ
ನಾವೇನು
ನಾವೊಂದು
ನಾಶ
ನಾಶ-ಕವೂ
ನಾಶಕ್ಕಾಗಿಯೇ
ನಾಶ-ಗೊಳಿ-ಸುತ್ತಿದ್ದಾರೆ
ನಾಶದ
ನಾಶ-ದಿಂದ
ನಾಶ-ಪಡಿ-ಸ-ಬಲ್ಲದೊ
ನಾಶ-ಪಡಿ-ಸ-ಲಿಲ್ಲ
ನಾಶ-ಪಡಿ-ಸಲು
ನಾಶ-ಪಡಿಸಿ
ನಾಶ-ಪಡಿ-ಸಿ-ಕೊಳ್ಳುತ್ತಿದ್ದಾರೆ
ನಾಶ-ಪಡಿ-ಸುತ್ತವೆ
ನಾಶ-ಪಡಿ-ಸು-ವು-ದಕ್ಕೂ
ನಾಶ-ಮಹಾತ್ರವು
ನಾಶ-ಮಾ-ಡಲು
ನಾಶ-ಮಾಡಿ
ನಾಶ-ಮಾಡುತ್ತಿದ್ದುವೋ
ನಾಶ-ಮಾಡುವ
ನಾಶ-ಮಾಡು-ವು-ದಲ್ಲದೆ
ನಾಶ-ಮಾಡು-ವುದು
ನಾಶ-ಮಾಡು-ವುದೂ
ನಾಶ-ವಾಗ-ಬೇಕು
ನಾಶ-ವಾಗಿ
ನಾಶ-ವಾಗಿದ್ದೇವೆ
ನಾಶ-ವಾಗುತ್ತವೆ
ನಾಶ-ವಾಗು-ವ-ವರೆಗೆ
ನಾಶ-ವಾ-ಗು-ವು-ದಿಲ್ಲ
ನಾಶ-ವಾಗು-ವುದು
ನಾಶ-ವಾಗು-ವುವು
ನಾಶ-ವಾದ
ನಾಶ-ವಾದರೂ
ನಾಶ-ವಾದಾಗ
ನಾಶ-ಹೊಂದಿದೆ
ನಾಶಿನಿ-ಯೆನ್ನುವರು
ನಾಶ್ರಿತ್ಯ
ನಾಷ್ಟಿ-ಕರು
ನಾಸದೀಯ-ಸೂಕ್ತ
ನಾಸ್ತಿ
ನಾಸ್ತಿಕ
ನಾಸ್ತಿ-ಕ-ತೆ-ಯೆಂದು
ನಾಸ್ತಿಕ್ಯಂತ್ವಿದಂತು
ನಾಸ್ತಿ-ಗಳನ್ನು
ನಾಸ್ತಿಯೂ
ನಾಸ್ತ್ಯೇವ
ನಾಹಿ
ನಿಂತನು
ನಿಂತರು
ನಿಂತರೂ
ನಿಂತರೆ
ನಿಂತವ-ನ-ವನು
ನಿಂತಾಗ
ನಿಂತಿ-ತದುವೂ
ನಿಂತಿತ್ತು
ನಿಂತಿದೆ
ನಿಂತಿದ್ದರು
ನಿಂತಿದ್ದರೆ
ನಿಂತಿದ್ದೇವೆ
ನಿಂತಿರ-ಬ-ಹುದು
ನಿಂತಿರ-ಬೇಕು
ನಿಂತಿ-ರುತ್ತದೆ
ನಿಂತಿರುತ್ತಾರೆ
ನಿಂತಿ-ರುವ
ನಿಂತಿ-ರು-ವು-ದನ್ನು
ನಿಂತಿ-ರು-ವುದು
ನಿಂತಿ-ರು-ವುದೇನು
ನಿಂತಿಲ್ಲವೇ
ನಿಂತಿ-ವರು
ನಿಂತಿ-ಹನು
ನಿಂತು
ನಿಂತು-ಕೊಂಡ
ನಿಂತು-ಕೊಂಡರು
ನಿಂತು-ಕೊಂಡಿತು
ನಿಂತು-ಕೊಂಡಿತ್ತು
ನಿಂತು-ಕೊಂಡಿದೆ
ನಿಂತು-ಕೊಂಡಿದೆ-ಯಲ್ಲಾ
ನಿಂತು-ಕೊಂಡಿದ್ದಂತಿತ್ತು
ನಿಂತು-ಕೊಂಡಿದ್ದನು
ನಿಂತು-ಕೊಂಡಿದ್ದರು
ನಿಂತು-ಕೊಂಡಿದ್ದೀರಿ
ನಿಂತು-ಕೊಂಡು
ನಿಂತು-ಕೊಂಡೆ
ನಿಂತು-ಕೊಳ್ಳ-ಬಲ್ಲೆವೊ
ನಿಂತು-ಕೊಳ್ಳ-ಬಹುದೋ
ನಿಂತು-ಕೊಳ್ಳುತ್ತಿದ್ದ
ನಿಂತು-ಕೊಳ್ಳುವನು
ನಿಂತು-ಕೊಳ್ಳುವುದು
ನಿಂತು-ಬಿಟ್ಟನು
ನಿಂತು-ಬಿಟ್ಟರೆ
ನಿಂತು-ಬಿಡಲಿ
ನಿಂತು-ಸಿರ
ನಿಂತು-ಹೋಗಿತ್ತು
ನಿಂತು-ಹೋದರೆ
ನಿಂತು-ಹೋ-ದೊ-ಡನೆ
ನಿಂತು-ಹೋ-ಯಿತು
ನಿಂತೇ
ನಿಂದ
ನಿಂದಂತು
ನಿಂದ-ನ-ವನು
ನಿಂದ-ನೆಗೆ
ನಿಂದನೊ
ನಿಂದಿ-ಪ-ರೆಲ್ಲ
ನಿಂದಿ-ರಲು
ನಿಂದಿ-ಸಲಿ
ನಿಂದಿ-ಸಲು
ನಿಂದಿ-ಸಿ-ದಂತೆ
ನಿಂದಿ-ಸಿದರೆ
ನಿಂದಿ-ಸುತ್ತಾರೆ
ನಿಂದಿ-ಸುತ್ತಿದ್ದೆ
ನಿಂದಿಹವು
ನಿಂದಿಹುದ
ನಿಂದೆ
ನಿಂದೆ-ಯನುಂಬರು
ನಿಂದೇ
ನಿಂಬೆ-ಹಣ್ಣನ್ನು
ನಿಂಬೆಹಣ್ಣಿನ
ನಿಃಶ್ವಾಸಿ-ಸಲ್ಪಟ್ಟು
ನಿಃಸೃತ-ವಾದ
ನಿಃಸೃತ-ವಾ-ಯಿತು
ನಿಃಸ್ವಾರ್ಥ
ನಿಃಸ್ವಾರ್ಥತೆ
ನಿಃಸ್ವಾರ್ಥ-ತೆ-ಯಿಂದ
ನಿಃಸ್ವಾರ್ಥಪ್ರೇಮಿಗೀ
ನಿಃಸ್ವಾರ್ಥ-ವಾಗಿ
ನಿಃಸ್ವಾರ್ಥಿ
ನಿಕಟ
ನಿಕಟ-ವಾಗಿ
ನಿಕಟ-ವಾದ
ನಿಕಟ-ಸಂಬಂಧ-ವಿದೆ
ನಿಕೃಷ್ಟ
ನಿಕೃಷ್ಟ-ವಾದ
ನಿಖರ-ವಾಗಿ
ನಿಖಿಲ
ನಿಖಿಲ-ಭು-ವನ
ನಿಗೂ-ಢ-ವಾಗಿ-ರುವ
ನಿಗೂ-ಢ-ವಾದ
ನಿಗ್ರಹ
ನಿಗ್ರಹ-ದಲ್ಲಿಟ್ಟು
ನಿಗ್ರಹ-ವನ್ನು
ನಿಗ್ರಹಿ-ಸ-ಲಾ-ರರು
ನಿಗ್ರಹಿ-ಸಲ್ಪಟ್ಟಿದೆ
ನಿಗ್ರಹಿಸಿ
ನಿಗ್ರಹಿಸಿ-ಕೊಂಡು
ನಿಗ್ರಹಿಸು
ನಿಗ್ರಹಿಸು-ವು-ದನ್ನು
ನಿಚ್ಚ-ವಾಗಿಹ
ನಿಚ್ವಾಸ್
ನಿಜ
ನಿಜಕು
ನಿಜಕ್ಕೂ
ನಿಜದ
ನಿಜ-ದಲಿ
ನಿಜದಿ
ನಿಜ-ದೊಲು-ಮೆಗೊಲಿ-ದ-ವನೆ
ನಿಜ-ರೂಪ
ನಿಜವ
ನಿಜ-ವ-ನರಿತ-ವ-ರೆಲ್ಲೊ
ನಿಜ-ವಲ್ಲ
ನಿಜ-ವಲ್ಲದೆ
ನಿಜ-ವಲ್ಲ-ವೆಂದೂ
ನಿಜ-ವಲ್ಲ-ವೇನು
ನಿಜ-ವಾಗಲೂ
ನಿಜ-ವಾಗಿ
ನಿಜ-ವಾಗಿದೆ
ನಿಜ-ವಾಗಿದ್ದರೆ
ನಿಜ-ವಾಗಿಯೂ
ನಿಜ-ವಾಗುತ್ತಿ-ರ-ಲಿಲ್ಲ-ವಂತೆ
ನಿಜ-ವಾದ
ನಿಜ-ವಾದರೂ
ನಿಜ-ವಾದರೆ
ನಿಜ-ವಿದೆ
ನಿಜ-ವೆಂದು
ನಿಜ-ವೆಂದೂ
ನಿಜವೇ
ನಿಜಸ್ಥಿತಿ
ನಿಜಸ್ಥಿತಿ-ಯನ್ನು
ನಿಜಸ್ವ-ಭಾವ-ವನ್ನು
ನಿಟ್ಟು-ಸಿರು-ಗಳನ್ನು
ನಿಟ್ಟು-ಸಿರೆನಿತೊ
ನಿತಾಯ್
ನಿತ್ಯ
ನಿತ್ಯಕ್ರಿ-ಯೆ-ಗಳ
ನಿತ್ಯ-ಜೀವನದ
ನಿತ್ಯಜ್ಞಾನಿ
ನಿತ್ಯದ
ನಿತ್ಯ-ದೊಲು-ಮೆಯ
ನಿತ್ಯ-ನೈಮಿತ್ತಿಕ
ನಿತ್ಯ-ನೈಮಿತ್ತಿಕ-ಗಳಾಗಿ-ಬಿಟ್ಟಿವೆಯೊ
ನಿತ್ಯ-ಪೂರ್ಣ
ನಿತ್ಯಪ್ರಾಯ-ವಾಗಿ-ರು-ವಂತೆ
ನಿತ್ಯ-ಮಸ್ಮತ್
ನಿತ್ಯ-ಮುಕ್ತ
ನಿತ್ಯ-ಮುಕ್ತನು
ನಿತ್ಯ-ವಸ್ತು-ವನ್ನು
ನಿತ್ಯ-ವಾಗಿ
ನಿತ್ಯ-ವಾಗಿ-ರು-ವುದು
ನಿತ್ಯ-ವಾದ
ನಿತ್ಯ-ವಾ-ದು-ದನ್ನು
ನಿತ್ಯ-ವಾದುದನ್ನೇ
ನಿತ್ಯ-ವಾದುದ-ರಲ್ಲಿ
ನಿತ್ಯ-ವಾಹಿನಿ
ನಿತ್ಯವು
ನಿತ್ಯವೂ
ನಿತ್ಯ-ಶುದ್ಧ
ನಿತ್ಯ-ಸತ್ಯ-ವನ್ನು
ನಿತ್ಯಾತ್ಮ-ನಾದ
ನಿತ್ಯಾ-ನಂದ
ನಿತ್ಯಾ-ನಂದಸ್ವಾಮಿ-ಗಳು
ನಿತ್ಯಾ-ನಿತ್ಯ
ನಿತ್ಯಾ-ನಿತ್ಯ-ವಸ್ತು
ನಿತ್ಯೋತ್ಸವ-ವಿದ್ದಂತಿದೆ
ನಿದರ್ಶನ
ನಿದರ್ಶ-ನಕ್ಕೆ
ನಿದರ್ಶನ-ಗಳ
ನಿದರ್ಶನ-ಗಳನ್ನು
ನಿದರ್ಶನ-ಗಳಿವೆ
ನಿದರ್ಶನ-ದಿಂದಲೂ
ನಿದರ್ಶನ-ವನ್ನು
ನಿದರ್ಶನ-ವಾಗಿ-ರುತ್ತಾರೆ
ನಿದರ್ಶನವೋ
ನಿದ್ದೆ
ನಿದ್ದೆ-ಯನ್ನು
ನಿದ್ದೆ-ಯಾಗಿ-ರ-ಬ-ಹುದು
ನಿದ್ದೆ-ಯಿಂದ
ನಿದ್ರ
ನಿದ್ರಾ-ಭಂಗ-ವಾಗಿ
ನಿದ್ರಿಸ-ಬ-ಹುದು
ನಿದ್ರಿ-ಸಲೇ-ಬೇಕು
ನಿದ್ರಿ-ಸುತ್ತಿದ್ದ
ನಿದ್ರಿ-ಸುತ್ತಿದ್ದಾರೆ
ನಿದ್ರಿ-ಸುತ್ತಿ-ರುವ
ನಿದ್ರಿ-ಸುತ್ತಿ-ರುವಳು
ನಿದ್ರಿ-ಸುತ್ತಿ-ರುವಾಗಲೂ
ನಿದ್ರಿಸು-ವಿರಿ
ನಿದ್ರೆ
ನಿದ್ರೆ-ಗಳನ್ನು
ನಿದ್ರೆ-ಗಳನ್ನೆಲ್ಲಾ
ನಿದ್ರೆಗೆ
ನಿದ್ರೆ-ಮಾಡ-ತೊಡಗಿ-ದರು
ನಿದ್ರೆ-ಯನ್ನು
ನಿದ್ರೆ-ಯಲ್ಲಿ
ನಿದ್ರೆ-ಯಿಂದ
ನಿದ್ರೆ-ಯಿಂದೆಚ್ಚೆತ್ತ
ನಿದ್ರೆ-ಯಿಂದೆದ್ದು
ನಿಧನ-ನಾದ
ನಿಧಾನ-ವಾಗಿ
ನಿಧಾನ-ವಾಗುತ್ತಾ
ನಿಧಾನ-ವಾದ
ನಿಧಿ-ಗಳನ್ನು
ನಿಧಿಗೆ
ನಿಧಿಯ
ನಿಧಿ-ಯ-ನಂತಕೆ
ನಿಧಿಯು
ನಿನ-ಗಾ-ಗಲಿ
ನಿನ-ಗಾಗಿ
ನಿನ-ಗಾ-ಗಿದೆ
ನಿನ-ಗಾದ
ನಿನ-ಗಾವ
ನಿನ-ಗಿದು
ನಿನ-ಗಿಲ್ಲ
ನಿನ-ಗಿಲ್ಲಿ
ನಿನ-ಗಿಷ್ಟ-ಬಂದ
ನಿನ-ಗಿಷ್ಟ-ಬಂದದ್ದನ್ನು
ನಿನಗೂ
ನಿನಗೆ
ನಿನಗೇ
ನಿನ-ಗೇಕೆ
ನಿನ-ಗೇ-ನಾದರೂ
ನಿನ-ಗೇನು
ನಿನ-ಗೊಂದು
ನಿನದಿ-ರಲಿ
ನಿನಾದ
ನಿನಾದ-ದಿಂದ
ನಿನಾದೆ
ನಿನ್ನ
ನಿನ್ನಂತಹ
ನಿನ್ನಂತಿ-ರುವ
ನಿನ್ನಂತೆ
ನಿನ್ನಂತೆಯೇ
ನಿನ್ನ-ಡಿ-ಗಳ
ನಿನ್ನ-ಡೆಗೆ
ನಿನ್ನ-ದಾ-ಗಲಿ
ನಿನ್ನ-ದಿದು
ನಿನ್ನ-ದಿದೊ
ನಿನ್ನ-ನ-ರಸಿ
ನಿನ್ನ-ನರ-ಸುತ
ನಿನ್ನನು
ನಿನ್ನನ್ನು
ನಿನ್ನನ್ನೆ
ನಿನ್ನನ್ನೇ
ನಿನ್ನಯ
ನಿನ್ನಲ್ಲಾಶ್ರಯ
ನಿನ್ನಲ್ಲಿ
ನಿನ್ನಲ್ಲಿನ್ನೂ
ನಿನ್ನಲ್ಲಿ-ರುವ
ನಿನ್ನಲ್ಲೇ
ನಿನ್ನ-ವನು
ನಿನ್ನ-ವ-ನೆಂದು
ನಿನ್ನ-ವ-ರಂತೆ
ನಿನ್ನ-ವ-ರೆಂದು
ನಿನ್ನಾಜ್ಞೆ-ಯಂತೆಯೇ
ನಿನ್ನಾಣ-ತಿಯ
ನಿನ್ನಿಂದ
ನಿನ್ನಿಂದಲೇ
ನಿನ್ನುದ್ಧಾರ-ಕನು-ದಿ-ಸುವನು
ನಿನ್ನು-ಸಿರು
ನಿನ್ನೆ
ನಿನ್ನೆಚ್ಚ-ರಿಕೆ
ನಿನ್ನೆ-ಡೆಗೆ
ನಿನ್ನೆ-ದೆ-ಯಲಿ
ನಿನ್ನೆ-ದೆಯು
ನಿನ್ನೊಡ-ನಿದ್ದೇ
ನಿನ್ನೊಡ-ನಿ-ಹನು
ನಿನ್ನೊ-ಡನೆ
ನಿನ್ನೊಲವ
ನಿನ್ನೊಲ-ವಿನೆದೆ
ನಿನ್ನೊಲುಮೆ
ನಿನ್ನೊಳಗಿ-ಹುದು
ನಿನ್ನೊಳಗಿಹುದೊ
ನಿನ್ನೊಳಗೂ
ನಿನ್ನೊ-ಳಗೆ
ನಿನ್ನೊಳ-ಹೊರಗ
ನಿಪುಣತೆ
ನಿಪುಣ-ರಾಗ-ಬ-ಹುದು
ನಿಪು-ಣರು
ನಿಬಂಧ-ನೆ-ಗಳಿಂದ
ನಿಬಂಧ-ನೆ-ಗಳೆಲ್ಲಾ
ನಿಬಂಧಿ-ಸಲ್ಪಟ್ಟ
ನಿಬೋಧತ
ನಿಮಗ-ದನ್ನು
ನಿಮಗಾಗ
ನಿಮಗಾಗಲೀ
ನಿಮಗಾ-ಗಲೇ
ನಿಮ-ಗಾಗಿ
ನಿಮಗಾಗು-ವು-ದೆಂದು
ನಿಮ-ಗಿಂತ
ನಿಮಗಿ-ದ-ರಿಂದ
ನಿಮಗಿರ-ಬೇ-ಕಾದ
ನಿಮ-ಗಿ-ರುವುದ-ರಲ್ಲಿ
ನಿಮಗಿಷ್ಟ
ನಿಮಗೂ
ನಿಮಗೆ
ನಿಮ-ಗೆಲ್ಲ
ನಿಮ-ಗೆಲ್ಲಾ
ನಿಮಗೇ
ನಿಮ-ಗೇಕೆ
ನಿಮಗೇನು
ನಿಮಗ್ನ-ರಾಗಿದ್ದು-ದನ್ನು
ನಿಮಗ್ಯಾವ
ನಿಮಿತ್ತ
ನಿಮಿತ್ತ-ಗಳ
ನಿಮಿತ್ತ-ಗಳನ್ನೂ
ನಿಮಿತ್ತ-ದೋಷ
ನಿಮಿತ್ತವೇ
ನಿಮಿತ್ತಾತೀತ-ವಾ-ದುದು
ನಿಮಿಷ
ನಿಮಿಷ-ಗಳಲ್ಲಿ
ನಿಮಿಷ-ಗಳಲ್ಲಿಯೇ
ನಿಮಿಷ-ಗ-ಳಾದ-ನಂತರ
ನಿಮಿಷ-ಗ-ಳಿದ್ದರು
ನಿಮಿಷ-ಗಳೊ-ಳಗೇ
ನಿಮಿಷ-ದಲ್ಲಿ
ನಿಮಿಷ-ಮಾತ್ರ-ದಲ್ಲಿ
ನಿಮಿಷ-ವಿದೆ
ನಿಮಿಷವೂ
ನಿಮ್ನ
ನಿಮ್ಮ
ನಿಮ್ಮಂತಹ
ನಿಮ್ಮಂತ-ಹ-ವನೇ
ನಿಮ್ಮಂತೆ
ನಿಮ್ಮಂಥ
ನಿಮ್ಮಂಥ-ವರು
ನಿಮ್ಮ-ಗಳಲ್ಲಿ
ನಿಮ್ಮ-ದಾಗಿದ್ದರೆ
ನಿಮ್ಮನ್ನು
ನಿಮ್ಮನ್ನೂ
ನಿಮ್ಮನ್ನೆ
ನಿಮ್ಮನ್ನೇ
ನಿಮ್ಮನ್ನೇ-ನನ್ನ-ಬೇಕು
ನಿಮ್ಮ-ಪೈಕಿ
ನಿಮ್ಮಲ್ಲಿ
ನಿಮ್ಮಲ್ಲಿದೆ
ನಿಮ್ಮಲ್ಲಿನ
ನಿಮ್ಮಲ್ಲಿಯೂ
ನಿಮ್ಮಲ್ಲಿಯೇ
ನಿಮ್ಮಲ್ಲಿ-ರಲಿ
ನಿಮ್ಮಲ್ಲಿ-ರುವ
ನಿಮ್ಮಲ್ಲಿ-ರು-ವು-ದನ್ನು
ನಿಮ್ಮಲ್ಲೆಲ್ಲಾ
ನಿಮ್ಮಲ್ಲೇ
ನಿಮ್ಮಲ್ಲೇನೂ
ನಿಮ್ಮಲ್ಲೊಬ್ಬರು
ನಿಮ್ಮವೋ
ನಿಮ್ಮಷ್ಟು
ನಿಮ್ಮಿಂದ
ನಿಮ್ಮೆ-ಡೆಗೆ
ನಿಮ್ಮೆಲ್ಲ-ರನ್ನೂ
ನಿಮ್ಮೆಲ್ಲ-ರಲ್ಲೂ
ನಿಮ್ಮೆಲ್ಲ-ರಿಗೂ
ನಿಮ್ಮೊಡನಿ-ರುವಾಗ
ನಿಮ್ಮೊ-ಡನೆ
ನಿಮ್ಮೊ-ಳಗೆ
ನಿಯಂತೃ-ಗಳು
ನಿಯಂತ್ರಿ-ಸುತ್ತ-ವೆಂದರೆ
ನಿಯಮ
ನಿಯಮ-ಗಳ
ನಿಯಮ-ಗಳನ್ನಾದರೂ
ನಿಯ-ಮ-ಗಳನ್ನು
ನಿಯಮ-ಗಳನ್ನೂ
ನಿಯಮ-ಗಳನ್ನೇ
ನಿಯಮ-ಗಳಿಂದ
ನಿಯಮ-ಗಳಿಂದಲೂ
ನಿಯಮ-ಗಳಿಗೂ
ನಿಯ-ಮ-ಗಳಿಗೆ
ನಿಯಮ-ಗಳಿ-ರುತ್ತವೆ
ನಿಯಮ-ಗಳಿವೆ
ನಿಯಮ-ಗಳಿವೆ-ಯಲ್ಲ
ನಿಯಮ-ಗಳಿವೆಯೋ
ನಿಯಮ-ಗಳು
ನಿಯಮ-ಗಳೂ
ನಿಯಮದ
ನಿಯಮ-ದಂತೆ
ನಿಯಮನ
ನಿಯಮ-ಬದ್ಧ
ನಿಯಮ-ಮಿತಿ-ಗಳ-ನೆಲ್ಲ
ನಿಯಮ-ವನ್ನನು-ಸರಿಸ-ಬೇಕು
ನಿಯಮ-ವನ್ನು
ನಿಯಮ-ವನ್ನೂ
ನಿಯಮ-ವನ್ನೇ
ನಿಯಮ-ವಲ್ಲ
ನಿಯಮ-ವಾಗಲೀ
ನಿಯಮ-ವಿದು
ನಿಯಮ-ವಿದೆ
ನಿಯಮ-ವಿದೆ-ಯೇನು
ನಿಯಮ-ವಿರುತ್ತದೋ
ನಿಯಮವು
ನಿಯಮವೂ
ನಿಯಮ-ವೆಂದರೆ
ನಿಯಮ-ವೇನೂ
ನಿಯಮ-ಶೃಂಖಲೆ-ಯಲ್ಲಿ
ನಿಯಮಾಃ
ನಿಯಮಾವಳಿ-ಗಳ
ನಿಯಮಾವಳಿ-ಯನ್ನು
ನಿಯಮಿತ
ನಿಯಮಿಸಬೇ-ಕಾ-ಯಿತು
ನಿಯಮಿ-ಸುತ್ತವೆ
ನಿಯಾ-ಮಕ-ನೆಂದು
ನಿಯುಕ್ತೋಽಸ್ಮಿ
ನಿಯೋ-ಜಿಸಿದ್ದರು
ನಿಯೋ-ಜಿಸಿ-ರುವ
ನಿರಂಕುಶ
ನಿರಂಕುಶಪ್ರಭು-ಗಳ
ನಿರಂಜನ
ನಿರಂಜನನ
ನಿರಂಜನ-ನಿಗೆ
ನಿರಂಜನರು
ನಿರಂಜನಾ-ನಂದ
ನಿರಂಜನಾ-ನಂದರ
ನಿರಂಜನಾ-ನಂದರು
ನಿರಂತರ
ನಿರಂತರಧ್ಯಾನ-ದಿಂದ
ನಿರಂತರ-ವಾಗಿ
ನಿರಂತರ-ವಾದ
ನಿರಕಿ-ಸುತ
ನಿರಖಿ
ನಿರತ-ನಾಗಿದ್ದರೂ
ನಿರತ-ನಾಗಿ-ರುತ್ತಾನೆ
ನಿರತ-ನಾದ
ನಿರತ-ರಾಗ-ಬೇಕು
ನಿರತ-ರಾಗಿ
ನಿರತ-ರಾಗಿದ್ದರು
ನಿರತ-ರಾಗಿದ್ದರೂ
ನಿರತ-ರಾಗಿದ್ದಾಗ
ನಿರತ-ರಾಗಿ-ರುವರು
ನಿರತ-ರಾಗಿ-ರುವರೊ
ನಿರತ-ರಾಗಿ-ರು-ವೆವು
ನಿರಪೇಕ್ಷ
ನಿರಪೇಕ್ಷ-ನಾಗಿ-ರು-ವನು
ನಿರಪೇಕ್ಷ-ವಾಗಿ
ನಿರಪೇಕ್ಷ-ವಾದ
ನಿರಪೇಕ್ಷ-ವಾ-ದದ್ದು
ನಿರಪೇಕ್ಷ-ವಾದು-ದಲ್ಲ
ನಿರಭ್ರ
ನಿರರ್ಗಳ
ನಿರರ್ಗಳ-ವಾಗಿ
ನಿರರ್ಥಕ
ನಿರರ್ಥಕ-ವಲ್ಲ
ನಿರರ್ಥಕ-ವಲ್ಲ-ವೆಂದು
ನಿರರ್ಥಕ-ವಾದ
ನಿರ-ವಧಿ
ನಿರವ-ಧಿರ್ವಿ-ಪುಲಾ
ನಿರಶ-ನದ
ನಿರಾ-ಕರಿ-ಸ-ಲಾ-ರರು
ನಿರಾ-ಕರಿ-ಸಿ-ದನು
ನಿರಾ-ಕರಿ-ಸುತ್ತ
ನಿರಾ-ಕರಿ-ಸುತ್ತಾರೆ
ನಿರಾ-ಕರಿ-ಸು-ವಷ್ಟು
ನಿರಾ-ಕಾರ
ನಿರಾ-ಕಾರ-ಗಳೆ-ರಡೂ
ನಿರಾ-ಕಾರ-ದೊಂದಿಗೆ
ನಿರಾ-ಕಾರ-ವನ್ನು
ನಿರಾತಂಕ-ವಾಗಿ
ನಿರಾಶನಾಗ-ದಿರು
ನಿರಾಶಾ-ವಾದಿ-ಗ-ಳೆಂದು
ನಿರಾಶೆ
ನಿರಾಶೆ-ಯಲ್ಲಿ
ನಿರಾಶೆ-ಯಿಂದ
ನಿರಾಸಕ್ತಿ
ನಿರೀಕ್ಷಿಸ-ಬ-ಹುದು
ನಿರೀಕ್ಷಿಸ-ಬೇ-ಕಾದರೆ
ನಿರೀಕ್ಷಿಸ-ಬೇಕು
ನಿರೀಕ್ಷಿ-ಸುತ್ತಿದ್ದೆ
ನಿರೀಕ್ಷಿ-ಸುವುದಕ್ಕಿಂತ
ನಿರುತ್ತರ-ನಾದಾಗ
ನಿರುತ್ಸಾಹ-ವನ್ನು
ನಿರುದ್ಧ-ವಾದರೆ
ನಿರು-ಪಮ
ನಿರುವೆ-ನೀಗ
ನಿರೂಪಣೆ
ನಿರೂಪ-ಣೆ-ಯಿದೆ
ನಿರೂಪಿ-ತ-ವಾಗಿದೆ
ನಿರೂಪಿ-ಸ-ಲಾಗಿದೆ
ನಿರೂಪಿ-ಸಲ್ಪಟ್ಟಿದೆ
ನಿರೂಪಿ-ಸಿದ್ದಾರೆ
ನಿರೂಪಿ-ಸಿ-ರುವರೊ
ನಿರೂಪಿ-ಸು-ವು-ದಿಲ್ಲ
ನಿರೋಧ
ನಿರೋಧ-ದಿಂದ
ನಿರೋಧನ
ನಿರ್ಗಚ್ಛತಿ
ನಿರ್ಗತಿ-ಕ-ನಾದಾಗಲೂ
ನಿರ್ಗಮಿಸ-ಬ-ಹುದು
ನಿರ್ಗಮಿ-ಸಿದ-ವ-ರನ್ನು
ನಿರ್ಗಮಿ-ಸಿದ್ದಾರೆ
ನಿರ್ಗಮಿ-ಸುತ್ತಿದ್ದೇನೆ
ನಿರ್ಗುಣ
ನಿರ್ಗುಣ-ಗುಣ-ಮಯ
ನಿರ್ಗುಣವೋ
ನಿರ್ಘೋಷಕ್ಕೆ
ನಿರ್ಜರಿಣಿ
ನಿರ್ಜಿವ
ನಿರ್ಜೀವ
ನಿರ್ಜೀವ-ವಾದ
ನಿರ್ಣಯ
ನಿರ್ದಯ-ತೆಯೇ
ನಿರ್ದಾಕ್ಷಿಣ್ಯ-ವಾಗಿ
ನಿರ್ದಿಶಾಮಿ
ನಿರ್ದಿಷ್ಟ
ನಿರ್ದೆಶ
ನಿರ್ದೆಶಿಸಿದ್ದಾರೆಂದಲ್ಲವೇ
ನಿರ್ದೇಶ-ನ-ಗಳನ್ನು
ನಿರ್ದೇಶಿಸ-ಹೋದದ್ದ-ರಿಂದ
ನಿರ್ಧ-ರಿಸಿ
ನಿರ್ಧರಿ-ಸಿ-ಕೊಂಡಿದ್ದ-ರೆಂದೂ
ನಿರ್ಧರಿ-ಸಿದೆ
ನಿರ್ಧರಿ-ಸಿದ್ದಾನೆ
ನಿರ್ಧರಿಸು
ನಿರ್ಧರಿ-ಸುತ್ತೇವೆ
ನಿರ್ಧರಿ-ಸು-ವು-ದನ್ನು
ನಿರ್ಧರಿ-ಸು-ವುದು
ನಿರ್ಧಾರ
ನಿರ್ಧಾರ-ಗಳನ್ನು
ನಿರ್ಧಾರ-ಗಳು
ನಿರ್ಧಾರ-ದಿಂದ
ನಿರ್ಧಾರ-ವನ್ನು
ನಿರ್ಧಾರ-ವಿ-ದೆಯೊ
ನಿರ್ಧಾರ-ವಿದ್ದು-ದ-ರಿಂದ
ನಿರ್ಧಾರವೇ
ನಿರ್ಧೇಶಿಸಿ
ನಿರ್ನಾಮ-ವಾಗಿ
ನಿರ್ನಾಮ-ವಾಗಿ-ಬಿಡುತ್ತೀರಿ
ನಿರ್ನಾಮ-ವಾಗು-ವುದೆಂದೂ
ನಿರ್ನಾಮ-ವಾಗು-ವುವು
ನಿರ್ನಾಮ-ವಾದ
ನಿರ್ನಾಮವು
ನಿರ್ಬಂಧ-ವಾದ
ನಿರ್ಬಂಧವೂ
ನಿರ್ಭಯ
ನಿರ್ಭಯ-ತೆಯ
ನಿರ್ಭಯ-ತೆ-ಯನ್ನು
ನಿರ್ಭಯ-ನಾಗಿ-ರ-ಬೇಕು
ನಿರ್ಭಯ-ನಾದನು
ನಿರ್ಭಯ-ಮೂರ್ತಿ-ಯನ್ನು
ನಿರ್ಭಯಾ-ನಂದ
ನಿರ್ಭ-ಯಾವಸ್ಥೆ-ಯನ್ನು
ನಿರ್ಭರತೆ
ನಿರ್ಭರ-ತೆಯು
ನಿರ್ಭರಸ್ಥಿತಿ
ನಿರ್ಭಿತ-ರಾಗಿ
ನಿರ್ಭೀತ
ನಿರ್ಭೀತ-ರಾಗಿ
ನಿರ್ಮಲ-ವಾಗಿ
ನಿರ್ಮಲ-ವಾಗಿತ್ತೆಂದರೆ
ನಿರ್ಮಲ-ವಾಗಿದೆ
ನಿರ್ಮಲಾ-ನಂದ
ನಿರ್ಮಲಾ-ನಂದ-ರಿಗೆ
ನಿರ್ಮಲಾ-ನಂದ-ರೊಡನೆ
ನಿರ್ಮಾಣ
ನಿರ್ಮಾಣ-ಮಾಡು
ನಿರ್ಮಿತ-ವಾಗಿದೆ
ನಿರ್ಮಿತ-ವಾಗುತ್ತ
ನಿರ್ಮಿತ-ವಾಗುತ್ತದೆ
ನಿರ್ಮಿತ-ವಾದ
ನಿರ್ಮಿಸ-ಬ-ಹುದು
ನಿರ್ಮಿಸ-ಬೇಕು
ನಿರ್ಮಿ-ಸಲಿ
ನಿರ್ಮಿ-ಸಿ-ದರು
ನಿರ್ಮಿ-ಸು-ವುದು
ನಿರ್ಮೂಲ
ನಿರ್ಮೂಲ-ವಾಗು-ವು-ದೇನು
ನಿರ್ಯಾಣ-ವಾಗಲು
ನಿರ್ಯಾಣ-ವಾದ
ನಿರ್ಯಾಣಾ-ನಂತರ
ನಿರ್ಲಕ್ಷಿಸ-ಲಾರಿರಾ
ನಿರ್ಲಕ್ಷಿಸಿ
ನಿರ್ಲಿಪ್ತ
ನಿರ್ಲೇಪ
ನಿರ್ಲೇಪ-ನಾದ
ನಿರ್ವಂಚನೆ-ಯಿಂದ
ನಿರ್ವಹ-ಣೆಗೆ
ನಿರ್ವ-ಹಣೆ-ಯಲ್ಲಿ
ನಿರ್ವಹಿ-ಸಲ್ಪಡುತ್ತಿದ್ದುವು
ನಿರ್ವಹಿ-ಸುತ್ತಿದ್ದೀರಿ
ನಿರ್ವಾಕ
ನಿರ್ವಾಣ
ನಿರ್ವಾಣ-ದತ್ತ
ನಿರ್ವಾಣ-ಷಟ್ಕಂ
ನಿರ್ವಾಣಾ-ನಂತರ
ನಿರ್ವಾಹಕ-ನಾ-ಗಿ-ರಲಿ
ನಿರ್ವಾಹ-ವಿಲ್ಲದೆ
ನಿರ್ವಿ-ಕಲ್ಪ
ನಿರ್ವಿ-ಕಲ್ಪಕ್ಕೆ
ನಿರ್ವಿ-ಕಲ್ಪನು
ನಿರ್ವಿ-ಕಾರ-ವಾದ
ನಿರ್ವಿಘ್ನ-ವಾಗಿ
ನಿರ್ವಿಷಯ
ನಿರ್ವೀರ್ಯ-ರಾಗಿ
ನಿಲಲಿ
ನಿಲು-ಕದ
ನಿಲುಕ-ದದು
ನಿಲುಕ-ದಿಹ
ನಿಲು-ಕದೆ
ನಿಲುಕಲು-ಬ-ಹುದು
ನಿಲುಕು-ವು-ದಿಲ್ಲವೊ
ನಿಲುಕು-ವುದು
ನಿಲುಗಂಬದ
ನಿಲು-ಗಡೆ
ನಿಲುವ
ನಿಲು-ವನ್ನು
ನಿಲು-ವರು
ನಿಲು-ವಿದ್ದ
ನಿಲು-ವಿನ
ನಿಲು-ವಿ-ನಿಂದ
ನಿಲುವು
ನಿಲು-ವುದೆ
ನಿಲುವೆ
ನಿಲುವೇ
ನಿಲ್ಲ-ದಲ್ಲ
ನಿಲ್ಲದೆ
ನಿಲ್ಲ-ಬೇಕೆಂದು
ನಿಲ್ಲ-ಬೇಕೆಂಬ
ನಿಲ್ಲ-ಬೇಡ
ನಿಲ್ಲ-ಬೇಡಿ
ನಿಲ್ಲ-ಲಾರದು
ನಿಲ್ಲ-ಲಾರದೆ
ನಿಲ್ಲಲು
ನಿಲ್ಲಿ-ಸ-ಬಲ್ಲೆಯಾ
ನಿಲ್ಲಿ-ಸಲಿ
ನಿಲ್ಲಿ-ಸಲು
ನಿಲ್ಲಿಸಿ
ನಿಲ್ಲಿ-ಸಿಕೊ
ನಿಲ್ಲಿ-ಸಿ-ಕೊಳ್ಳು-ವಷ್ಟು
ನಿಲ್ಲಿ-ಸಿ-ಬಿಡು-ವು-ದಿಲ್ಲ
ನಿಲ್ಲಿ-ಸುತ್ತಿದ್ದೆನು
ನಿಲ್ಲಿ-ಸು-ವು-ದೆಂದರೆ
ನಿಲ್ಲಿ-ಸೋಣ
ನಿಲ್ಲು
ನಿಲ್ಲುತ್ತದೆ
ನಿಲ್ಲುತ್ತದೆಯೋ
ನಿಲ್ಲುತ್ತದೋ
ನಿಲ್ಲುತ್ತವೆ
ನಿಲ್ಲುತ್ತಿ-ರ-ಲಿಲ್ಲ
ನಿಲ್ಲುತ್ತೇವೆ
ನಿಲ್ಲುವ
ನಿಲ್ಲು-ವಂತಾಗ-ಬೇಕು
ನಿಲ್ಲು-ವಂತೆ
ನಿಲ್ಲು-ವರು
ನಿಲ್ಲು-ವರೋ
ನಿಲ್ಲು-ವು-ದನ್ನು
ನಿಲ್ಲು-ವು-ದಿಲ್ಲ
ನಿಲ್ಲು-ವುದು
ನಿಲ್ಲು-ವುದೆ
ನಿಲ್ಲುವೆ
ನಿವಾರಣಾ
ನಿವಾ-ರಣೆ
ನಿವಾ-ರಣೆಯ
ನಿವಾರಿ-ಸ-ಬ-ಹುದು
ನಿವಾರಿ-ಸ-ಬೇಕಾಗಿದೆ
ನಿವಾರಿ-ಸು-ವುವು
ನಿವಾಸ
ನಿವಾಸ-ದಲ್ಲಿ
ನಿವಾ-ಸಿ-ಗಳ
ನಿವಾಸಿ-ಗಳಾಗುವರು
ನಿವಾ-ಸಿ-ಗಳು
ನಿವಾ-ಸಿ-ಗಳೂ
ನಿವಾಸಿ-ಗಳೆಲ್ಲ
ನಿವಾಸಿ-ಗಳೊ-ಡನೆ
ನಿವೃತ್ತ-ನಾದರೆ
ನಿವೃತ್ತ-ರಾಗಸ್ಯ
ನಿವೇದಿತ
ನಿವೇದಿತ-ಗಾಗಿ
ನಿವೇದಿತಾ
ನಿವೇದಿ-ತಾಗೆ
ನಿವೇದಿತಾ-ಳನ್ನು
ನಿವೇದಿತಾ-ಳನ್ನೂ
ನಿವೇದಿತಾಳೂ
ನಿವೇದಿ-ತೆಗೆ
ನಿವೇದಿ-ತೆಯ
ನಿವೇದಿ-ತೆಯೂ
ನಿವೇ-ಶ-ನಕ್ಕೆ
ನಿವೇಶ-ನದ
ನಿವೇಶ-ನ-ದಲ್ಲಿ
ನಿವೇಶ-ನ-ವನ್ನು
ನಿಶೆಯಡ್ಡ-ನಿಂದರೂ
ನಿಶ್ಚಯ
ನಿಶ್ಚಯ-ಮಾಡಿ-ಕೊಂಡು
ನಿಶ್ಚಯಿಸ-ಬ-ಹುದು
ನಿಶ್ಚಯಿ-ಸಲು
ನಿಶ್ಚಯಿ-ಸಿ-ಕೊಂಡೆನೋ
ನಿಶ್ಚಲ
ನಿಶ್ಚಲತೆ
ನಿಶ್ಚಲ-ತೆಯು
ನಿಶ್ಚಲ-ರಾಗಿ-ಬಿಡುತ್ತಿದ್ದರು
ನಿಶ್ಚಲ-ವಾಗಿ
ನಿಶ್ಚಲ-ವಾಗಿತ್ತು
ನಿಶ್ಚಲ-ವಾಗುತ್ತ-ದೆಯೋ
ನಿಶ್ಚಿಂತೆ-ಯಾಗಿ
ನಿಶ್ಚಿಂತೆ-ಯಿಂದ
ನಿಶ್ಚಿತ
ನಿಶ್ಚಿತ-ವಾಗಿ
ನಿಶ್ಚಿತ-ವಾದುದ-ರಲ್ಲಿ
ನಿಶ್ಚಿತವು
ನಿಶ್ಶಕ್ತರು
ನಿಶ್ಶಬ್ದ-ವಾಗಿ
ನಿಶ್ಶೇಷ
ನಿಶ್ಶೇಷ-ನಿರೋಧ
ನಿಷಿದ್ಧ
ನಿಷೇಧ
ನಿಷೇಧ-ಗಳ
ನಿಷೇಧ-ಗಳನ್ನೆಲ್ಲಾ
ನಿಷೇಧ-ಗಳೆಲ್ಲಾ
ನಿಷೇಧ-ಗಳೆಲ್ಲಿವೆ
ನಿಷೇಧ-ಮಯ
ನಿಷೇಧಿ-ಸಲ್ಪಟ್ಟಿ-ರುವ
ನಿಷೇಧಿ-ಸುವ
ನಿಷ್ಕಪಟಿ-ಗಳಾಗಿ-ರ-ಬ-ಹುದು
ನಿಷ್ಕರ್ಷಿ-ಸಲು
ನಿಷ್ಕರ್ಷಿ-ಸಿದರೆ
ನಿಷ್ಕರ್ಷಿ-ಸುವರು
ನಿಷ್ಕರ್ಷೆ
ನಿಷ್ಕಲಂ
ನಿಷ್ಕಾಮ
ನಿಷ್ಕಾಮ-ಕರ್ಮ-ದಿಂದಲ್ಲವೆ
ನಿಷ್ಕಾಮ-ಕರ್ಮ-ವೆಲ್ಲಾ
ನಿಷ್ಕಾರಣ
ನಿಷ್ಕ್ರಿ-ಯ-ತೆ-ಯಲ್ಲಿ
ನಿಷ್ಟುರ
ನಿಷ್ಟುರ-ದಿಂದಿದ್ದರೂ
ನಿಷ್ಟುರ-ನಾದ
ನಿಷ್ಟುರನು
ನಿಷ್ಠ-ರಾ-ಗಿ-ರುವರೋ
ನಿಷ್ಠ-ರಾದ
ನಿಷ್ಠಾ-ವಂತನೂ
ನಿಷ್ಠೆ-ಯಿಂದ
ನಿಷ್ಠೆ-ಯಿ-ರ-ಲಿಲ್ಲ
ನಿಷ್ಣಾತ
ನಿಷ್ಪಲ-ವಾಗಿದೆ
ನಿಷ್ಪಲವೆ
ನಿಷ್ಪಲ-ವೆಂದು
ನಿಷ್ಪಲವೆ-ನಿ-ಸು-ವುದು
ನಿಷ್ಪ್ರಯೋಜಕ-ರಾಗಿದ್ದೇವೆ
ನಿಷ್ಪ್ರಯೋಜಕ-ವಾಗಲಿ
ನಿಷ್ಪ್ರಯೋಜಕ-ವಾ-ಗು-ವು-ದಿಲ್ಲ
ನಿಷ್ಪ್ರಯೋಜಕ-ವೆಂದು
ನಿಸರ್ಗ
ನಿಸರ್ಗದ
ನಿಸರ್ಗ-ಮಾತೆಯ
ನಿಸರ್ಗ-ವನ್ನು
ನಿಸ್ತೇಜಗೊಳಿಸ-ಬೇಕು
ನಿಸ್ತೇಜ-ವಾಗಿ-ದೆಯೋ
ನಿಸ್ವನ
ನಿಸ್ವಾರ್ಥ-ದಗ್ನಿ-ಯಲ್ಲಿ
ನಿಸ್ವಾರ್ಥಪ್ರೇಮಿಕ್
ನಿಸ್ಸಂದೇ-ಹ-ವಾಗಿ
ನಿಸ್ಸಂದೇ-ಹವಾ-ಗಿಯೂ
ನಿಸ್ಸಂಶಯ-ವಾಗಿ
ನಿಸ್ಸಂಶಯ-ವಾದುದೂ
ನಿಸ್ಸಹಾಯ-ಕತೆ-ಯಿಂದ
ನಿಸ್ಸಾರ
ನಿಸ್ಸೀಮ
ನಿಸ್ಸೀ-ಮರು
ನಿಹತ-ನಿಖಿಲ-ಮೋಹೇಽಧೀಶತಾ
ನಿಹಿತ-ವಾಗಿ-ರುವ
ನಿಹಿತ-ವಾಗಿ-ರು-ವು-ದನ್ನು
ನೀ
ನೀಗಿ-ರುವೆ
ನೀಗಿ-ಸುವ
ನೀಗ್ರೋ
ನೀಗ್ರೋ-ಗಳೊಂದಿಗೆ
ನೀಗ್ರೋ-ಗಳೊ-ಡನೆ
ನೀಚ
ನೀಚ-ತನ
ನೀಚ-ನಾದ
ನೀಚ-ನೆಂದೇ
ನೀಚಪ್ರ-ವೃತ್ತಿಯ
ನೀಚ-ರೆಂದೂ
ನೀಚ-ರೆದೆಗೂ
ನೀಚೇ
ನೀಡ-ಬಯ-ಸು-ವೆನು
ನೀಡ-ಬೇಕೆ
ನೀಡಿ
ನೀಡಿದ
ನೀಡಿ-ದರು
ನೀಡಿ-ದ-ವರು
ನೀಡಿದೆ
ನೀಡಿದ್ದಳು
ನೀಡಿದ್ದಾರೆ
ನೀಡಿದ್ದು
ನೀಡಿ-ರುವೆ
ನೀಡು
ನೀಡುತ್ತಲೂ
ನೀಡುತ್ತಾ
ನೀಡುತ್ತಾರೆ
ನೀಡುತ್ತಿದ್ದರು
ನೀಡುತ್ತಿದ್ದಾರೆ
ನೀಡುವ
ನೀಡು-ವನು
ನೀಡು-ವರು
ನೀಡು-ವ-ರೆಂದು
ನೀಡು-ವುದು
ನೀಡು-ವೊಲು
ನೀತಂ
ನೀತಿ
ನೀತಿ-ಗಡಿಯು
ನೀತಿ-ಗಳಿಗೂ
ನೀತಿ-ಗಳು
ನೀತಿ-ಗೆಟ್ಟ
ನೀತಿ-ನಿ-ಪುಣಾ
ನೀತಿ-ಪರಾಯಣ-ರನ್ನಾಗಿಯೂ
ನೀತಿ-ಪರಾಯಣ-ಳಾ-ಗಿಯೂ
ನೀತಿಭ್ರಷ್ಟರ
ನೀತಿಯ
ನೀತಿ-ಯನ್ನು
ನೀತಿಯೇ
ನೀತಿ-ಶತಕ
ನೀತಿ-ಸಂಹಿ-ತೆಗೆ
ನೀನದ
ನೀನ-ದನ್ನು
ನೀನಲ್ಲವೇ
ನೀನಾ-ಗಲಿ
ನೀನಾಗಿ-ರುವೆ
ನೀನಾದರೋ
ನೀನಾರು
ನೀನಾವ
ನೀನಾ-ವಳು
ನೀನಾ-ವು-ದನ್ನು
ನೀನಿಂದು
ನೀನಿ-ದನ್ನು
ನೀನಿದೇ-ನನು
ನೀನಿನ್ನೂ
ನೀನಿ-ರುವ
ನೀನಿಲ್ಲಿಗೆ
ನೀನಿಲ್ಲಿಯೇ
ನೀನಿಲ್ಲೆ
ನೀನಿಷ್ಟ-ಪಡುವ
ನೀನಿಷ್ಟೊಂದು
ನೀನೀ
ನೀನೀಗ
ನೀನು
ನೀನು-ಗಳ-ಳಿದು
ನೀನು-ಸಿರು-ತಿ-ರುವೆ
ನೀನೂ
ನೀನೆ
ನೀನೆಂತಹ
ನೀನೆಂದೆಂದು
ನೀನೆಂಬುದನ-ರಿಯುತ
ನೀನೆತ್ತ
ನೀನೆನ್ನ
ನೀನೆನ್ನುವರು
ನೀನೆಲ್ಲಿ
ನೀನೆಷ್ಟು
ನೀನೇ
ನೀನೇಕೆ
ನೀನೇ-ನನ್ನೂ
ನೀನೇ-ನಾದರೂ
ನೀನೇನು
ನೀನೇನೂ
ನೀನೇನೊ
ನೀನೇನೋ
ನೀನೊಂದು
ನೀನೊಬ್ಬ
ನೀನೊಳ್ಳೆ
ನೀಯತಾಂ
ನೀಯ-ಮಾನಾ
ನೀರ
ನೀರ-ಗುಳ್ಳೆ-ಯಂತೆ
ನೀರ-ಡಿಕೆ-ಯಿಂದ
ನೀರನು
ನೀರನ್ನು
ನೀರನ್ನೂ
ನೀರನ್ನೇ
ನೀರ-ವತ್
ನೀರಾಗಿತ್ತು
ನೀರಿನ
ನೀರಿ-ನಂತೆ
ನೀರಿ-ನಲ್ಲಿ
ನೀರಿ-ನಿಂದ
ನೀರಿ-ನೊ-ಡನೆ
ನೀರು
ನೀರೂ
ನೀರೆಲ್ಲಾ
ನೀರೇ
ನೀರೊಳು
ನೀಲ
ನೀಲ-ಕಂಠಂ
ನೀಲ-ಕಂಠ-ನಿಗೆ
ನೀಲ-ಕುಸು-ಮಕೆ
ನೀಲ-ಕೋಮಲಸುಮವೆ
ನೀಲ-ನಯನ-ಗಳೊ
ನೀಲಾಂಬರ
ನೀಲಾಕಾಶ
ನೀಲಿ
ನೀಲಿಮ
ನೀಲಿ-ಯಾಗಸ-ದಲ್ಲಿ
ನೀಲಿ-ಯಾಗಿ
ನೀಲಿ-ಯಾಗಿಯೇ
ನೀಲಿ-ಯಾಗಿ-ರು-ವುದು
ನೀಲಿಯು
ನೀಲೋತ್ಪಲ
ನೀಲೋತ್ಪಲ-ಸದೃಶ-ವಾದ
ನೀವಲ್ಲ
ನೀವಾದರೋ
ನೀವಾರು
ನೀವಾರೂ
ನೀವಿಚ್ಛೆ-ಪಟ್ಟಿದ್ದೆಲ್ಲಾ
ನೀವಿದ್ದೀರಿ
ನೀವಿಲ್ಲಿ
ನೀವಿಲ್ಲಿಗೆ
ನೀವೀ
ನೀವೀಗ
ನೀವು
ನೀವೂ
ನೀವೆ
ನೀವೆಂತಹ
ನೀವೆಂದಾದರೂ
ನೀವೆಲ್ಲ
ನೀವೆಲ್ಲರೂ
ನೀವೆಲ್ಲಾ
ನೀವೆಲ್ಲಿದ್ದರೂ
ನೀವೆಲ್ಲೇ
ನೀವೆಷ್ಟೇ
ನೀವೇ
ನೀವೇಕೆ
ನೀವೇ-ನನ್ನೂ
ನೀವೇ-ನಾದರೂ
ನೀವೇನು
ನೀವೇನೂ
ನೀವೇ-ಶ-ನಕ್ಕೆ
ನೀವೊ
ನೀವೊಂದು
ನೀವೊಬ್ಬ
ನೀವೊಬ್ಬರು
ನೀವೊಬ್ಬರೇ
ನೀವೋ
ನೀವ್ಯಾ-ವು-ದನ್ನೂ
ನೀಹಾರಿ-ಕೆಯು
ನುಂಗಿ-ಕೊಳ್ಳ-ಬೇಕು
ನುಂಗಿ-ರು-ವುದು
ನುಂಗಿ-ಹಾಕಿ-ಬಿಡುತ್ತಾನೆ
ನುಂಗು-ತಲಿದೆ
ನುಂಗು-ವುದು
ನುಂಗುವೆ
ನುಗ್ಗಲು
ನುಗ್ಗಿ
ನುಗ್ಗಿದ
ನುಗ್ಗಿದ್ದರು
ನುಗ್ಗು-ತಿದೆ
ನುಗ್ಗು-ತಿ-ಹುದು
ನುಗ್ಗುತ್ತಿದ್ದಾರೆ
ನುಗ್ಗುವ
ನುಗ್ಗು-ವುದು
ನುಚ್ಚುನೂ-ರಾಗುವರು
ನುಚ್ಚುನೂರೆ
ನುಡಿ
ನುಡಿ-ಗಳನ್ನು
ನುಡಿ-ಗಳವೆ
ನುಡಿ-ಗಳಿಂದ
ನುಡಿ-ಗಳಿಗೆ
ನುಡಿ-ಗಳೊಂದಿಗೆ
ನುಡಿಗೆ
ನುಡಿದ
ನುಡಿ-ದನು
ನುಡಿ-ದರು
ನುಡಿ-ದ-ವ-ರಲ್ಲಿ
ನುಡಿ-ದಿದೆ
ನುಡಿ-ದಿಹವು
ನುಡಿದು
ನುಡಿ-ಮುತ್ತು-ಗಳು
ನುಡಿ-ಯಲಾ-ಗದ
ನುಡಿ-ಯಲೆಂದು
ನುಡಿಯು
ನುಡಿ-ಯುತ್ತಿದ್ದರು
ನುಡಿವೆ
ನುಡಿ-ಸುತ್ತಾ
ನುಡಿ-ಸುತ್ತಿ-ರುವರೋ
ನುತಿ-ಸುತಿ-ರು-ವುವು
ನುರಿತ
ನುರಿ-ತಿಲ್ಲದ
ನೂಕಲ್ಪಡುವ
ನೂಕುನುಗ್ಗುತ
ನೂತನ
ನೂತನತ್ವ-ವಿಲ್ಲದೆ
ನೂತನ-ವಾದ
ನೂರ-ರಲ್ಲಿ
ನೂರ-ರಷ್ಟು
ನೂರಾರು
ನೂರು
ನೂರು-ಭಾಗಕ್ಕಿಷ್ಟು
ನೂರು-ಸಾ-ಸಿರ-ದ-ಳದ
ನೂರ್ಮಡಿ-ಯಾದ
ನೂಲು-ಗಳ
ನೃಣಾಂ
ನೃತ್ಯ
ನೃತ್ಯ-ದಿಂದಲೂ
ನೆಂಟರಿಷ್ಟ-ರನ್ನು
ನೆಂಟರು
ನೆಂದೆಂದಿಗು
ನೆಕ್ಕುತ್ತಿರು-ವಂತೆ
ನೆಗೆ-ದಾಡಲಿ
ನೆಗೆ-ದಾಡುವುದ-ರಲ್ಲಿ
ನೆಗೆ-ಯು-ವುದು
ನೆಚ್ಚಲೇ
ನೆಚ್ಚಿ-ಕೊಂಡಿರು-ವ-ವರ
ನೆಚ್ಚಿ-ಕೊಂಡಿರು-ವುದ-ರಲ್ಲಿ
ನೆಚ್ಚಿ-ಕೊಂಡು
ನೆಚ್ಚಿಗೆ
ನೆಟ್ಟ-ಗಾಗಿ
ನೆಟ್ಟ-ಗಿ-ರುವ
ನೆಟ್ಟಿ-ರುವ
ನೆಟ್ಟು
ನೆಡೆಗೆ
ನೆತ್ತರ
ನೆತ್ತರು
ನೆತ್ತಿಯೇ-ನಾದರೂ
ನೆನ-ಪಾಗು-ವುದು
ನೆನಪಿ-ಗಾಗಿ
ನೆನಪಿಗೆ
ನೆನಪಿದೆ
ನೆನಪಿ-ದೆಯೆ
ನೆನಪಿ-ದೆಯೇ
ನೆನಪಿ-ನಲ್ಲಿಟ್ಟಿರ-ಬೇಕು
ನೆನಪಿ-ನಲ್ಲಿ-ಡ-ಬಲ್ಲೆ
ನೆನಪಿನಲ್ಲಿಡಿ
ನೆನಪಿ-ನಲ್ಲಿಡು
ನೆನಪಿನಲ್ಲಿದೆ
ನೆನಪಿನಲ್ಲಿ-ದೆಯೆ
ನೆನಪಿರ-ಬೇಕು
ನೆನಪು-ಗಳನ್ನು
ನೆನೆ
ನೆನೆ-ದಾಗ-ಲಂತೂ
ನೆನೆ-ದು-ಕೊಂಡು
ನೆನೆ-ಯ-ಬ-ಹುದು
ನೆನೆ-ಸಿ-ಕೊಂಡರೂ
ನೆನೆ-ಸಿ-ಕೊಂಡಿದ್ದದ್ದೇ-ನೆಂದರೆ
ನೆಮ್ಮದಿ
ನೆಮ್ಮದಿ-ಯಾಗುವುದು
ನೆಮ್ಮಿ
ನೆಯ
ನೆರಳ
ನೆರ-ಳದು
ನೆರಳಿ-ನೊ-ಡನೆ
ನೆರಳಿ-ನೊಲು
ನೆರಳು
ನೆರಳು-ಗಳ
ನೆರಳು-ಗಳನ್ನೆ
ನೆರಳು-ಗಳಿಗಂಜುವೆನೇ
ನೆರವ
ನೆರ-ವಾಗದ
ನೆರ-ವಾಗಲಿ
ನೆರ-ವಾಗುವ
ನೆರ-ವಾಗು-ವುದೋ
ನೆರ-ವಿಗೆ
ನೆರವು
ನೆರವೇ-ರಿ-ಸಿ-ದರು
ನೆರವೇ-ರಿ-ಸು-ವುದು
ನೆರೆ-ದಿದ್ದ
ನೆರೆ-ದಿದ್ದರು
ನೆರೆ-ಮನೆ-ಯ-ವರು
ನೆರೆಯ-ಬೇಕು
ನೆರೆಯ-ವ-ರನ್ನು
ನೆರೆ-ಯವರು
ನೆರೆ-ಹೊರೆಯ
ನೆಲ
ನೆಲಕ್ಕುರು-ಳಿದರು
ನೆಲಕ್ಕೆ
ನೆಲದ
ನೆಲ-ದ-ಮೇಲೆ
ನೆಲದಿ
ನೆಲ-ದೊಳೋಡುವ
ನೆಲ-ಮುಟ್ಟಿ
ನೆಲವ
ನೆಲ-ವನ್ನು
ನೆಲವು
ನೆಲ-ಸದೊ
ನೆಲ-ಸಲಿ
ನೆಲಸಿ
ನೆಲ-ಸಿತ್ತು
ನೆಲ-ಸಿದ
ನೆಲ-ಸಿ-ದಂತಿತ್ತು
ನೆಲ-ಸಿದೆ
ನೆಲ-ಸಿ-ರುವ
ನೆಲ-ಸಿ-ರು-ವುದು
ನೆಲ-ಸು-ವಂತೆ
ನೆಲೆ
ನೆಲೆ-ಗೊಂಡ
ನೆಲೆ-ಯದಾ-ವನೊ
ನೆಲೆ-ಯನೆಂದಿಗು
ನೆಲೆ-ಯಲಿ-ಹವು
ನೆಲೆ-ಯಲ್ಲಿ
ನೆಲೆ-ಯಾ-ಗಿದೆ
ನೆಲೆ-ಯಾ-ಗಿ-ರುವುದೊ
ನೆಲೆಯು
ನೆಲೆ-ಸಿ-ರು-ವು-ದನ್ನು
ನೆಲೆ-ಸಿ-ರು-ವುದು
ನೆಲ್ಲ
ನೆಲ್ಲಿ-ಕಾ-ಯಂತೆ
ನೆಲ್ಲಿ-ಕಾಯಾ-ಗುತ್ತದೆ
ನೆವ-ವನ್ನು
ನೇ
ನೇತಿ
ನೇತಿ-ನೇತಿ
ನೇತಿ-ಯೆಂಬುದು
ನೇತೃ-ಗಳು
ನೇತ್ರೀ
ನೇನಾದರೂ
ನೇಯು-ತಿದೆ
ನೇರ-ವಾಗಿ
ನೇಹ
ನೇಹ-ದಲಿ
ನೇಹ-ದೊಳು-ಸುರಿದ
ನೇಹಿ
ನೇಹಿ-ಗನ
ನೇಹಿ-ಗರನುಳಿ-ದ-ವರು
ನೇಹಿ-ಗ-ರೆಲ್ಲ
ನೈಜ
ನೈಜಸ್ಥಿತಿ-ಯನ್ನು
ನೈತಿಕ
ನೈತಿಕ-ತೆಯ
ನೈಯಾಯಿಕ
ನೈವೇದ್ಯ
ನೈವೇದ್ಯಕ್ಕೆ
ನೈಷಾ
ನೈಷ್ಠಿಕ
ನೊಂದಿಹ
ನೊಂದು-ಕೊಂಡು
ನೊಂದೆ
ನೊಗ-ವಿದನು
ನೊಡಗೂಡಿ
ನೊಣ-ಗಳಂತೆ
ನೊಣದ
ನೊರೆ-ನೊರೆಯ-ಲೆ-ಗಳ
ನೊರೆಯ
ನೊರೆ-ಯಿಂದ
ನೋಟ
ನೋಟ-ಕ-ರನ್ನು
ನೋಟ-ಕರು
ನೋಟಕ್ಕೆ
ನೋಟ-ದಲಿ
ನೋಟ-ದಾಚೆಯ
ನೋಟ-ವನ್ನು
ನೋಡ
ನೋಡ-ತೊಡಗಿದೆ
ನೋಡ-ದಿದ್ದರೆ
ನೋಡ-ದಿರು
ನೋಡದೆ
ನೋಡ-ಬಲ್ಲರೋ
ನೋಡ-ಬಲ್ಲ-ವ-ನಾಗಿದ್ದೆನು
ನೋಡ-ಬಲ್ಲಿರಿ
ನೋಡ-ಬಲ್ಲೆ
ನೋಡ-ಬಲ್ಲೆ-ಯಾದರೆ
ನೋಡ-ಬಹು-ದಾಗಿತ್ತು
ನೋಡ-ಬಹು-ದಿತ್ತು
ನೋಡ-ಬ-ಹುದು
ನೋಡ-ಬಹುದೆ
ನೋಡ-ಬೇ-ಕಲ್ಲವೆ
ನೋಡ-ಬೇ-ಕಾದ
ನೋಡ-ಬೇ-ಕಾದರೆ
ನೋಡ-ಬೇಕು
ನೋಡ-ಬೇಕೆ
ನೋಡ-ಬೇಕೆಂದರೆ
ನೋಡ-ಬೇಕೆಂದು
ನೋಡ-ಬೇಕೆಂಬ
ನೋಡ-ಬೇಕೆನ್ನಿಸಿದೆ
ನೋಡ-ಲಾಗಿ
ನೋಡ-ಲಾ-ಗು-ವು-ದಿಲ್ಲ
ನೋಡ-ಲಾರದೆ
ನೋಡ-ಲಾರದ್ದ-ರಿಂದ
ನೋಡ-ಲಾರೆ
ನೋಡಲಿ
ನೋಡ-ಲಿಕ್ಕೆ
ನೋಡ-ಲಿಚ್ಛಿಸ-ಬ-ಹುದು
ನೋಡ-ಲಿಚ್ಛಿ-ಸುವೆಯೊ
ನೋಡ-ಲಿಲ್ಲ
ನೋಡ-ಲಿಲ್ಲ-ವಲ್ಲ
ನೋಡ-ಲಿಲ್ಲವೇ
ನೋಡಲು
ನೋಡಿ
ನೋಡಿಕೊ
ನೋಡಿ-ಕೊಂಡರು
ನೋಡಿ-ಕೊಂಡಿರುತ್ತಿದ್ದೆ-ನೆ-ಹೀಗೆಂದು
ನೋಡಿ-ಕೊಂಡು
ನೋಡಿ-ಕೊಂಡೇ
ನೋಡಿ-ಕೊಂಡೊ
ನೋಡಿ-ಕೊಳ್ಳ-ಬೇಕು
ನೋಡಿ-ಕೊಳ್ಳಲಿ
ನೋಡಿ-ಕೊಳ್ಳುತ್ತಿದ್ದು
ನೋಡಿ-ಕೊಳ್ಳುತ್ತೇನೆ
ನೋಡಿ-ಕೊಳ್ಳುವರು
ನೋಡಿ-ಕೊಳ್ಳು-ವ-ವ-ರಾರು
ನೋಡಿ-ಕೊಳ್ಳುವ-ವರು
ನೋಡಿ-ಕೊಳ್ಳುವುದು
ನೋಡಿದ
ನೋಡಿ-ದನು
ನೋಡಿ-ದರು
ನೋಡಿ-ದರೂ
ನೋಡಿ-ದರೆ
ನೋಡಿ-ದ-ವರು
ನೋಡಿ-ದ-ವ-ರೆಲ್ಲಾ
ನೋಡಿ-ದಾಗ
ನೋಡಿ-ದಿಯೊ
ನೋಡಿ-ದಿರಿ
ನೋಡಿ-ದಿ-ರೇನು
ನೋಡಿ-ದು-ದ-ರಿಂದ
ನೋಡಿದೆ
ನೋಡಿ-ದೆನು
ನೋಡಿ-ದೆ-ಯಷ್ಟೆ
ನೋಡಿ-ದೆಯಾ
ನೋಡಿ-ದೆ-ಯೇ-ನಯ್ಯ
ನೋಡಿ-ದೆಯೊ
ನೋಡಿ-ದೆ-ವಲ್ಲ
ನೋಡಿ-ದೊಡ-ನೆಯೆ
ನೋಡಿ-ದೊಡ-ನೆಯೇ
ನೋಡಿದ್ದ
ನೋಡಿದ್ದ-ರೆಂದು
ನೋಡಿದ್ದಾ-ರೇನು
ನೋಡಿದ್ದೀಯಾ
ನೋಡಿದ್ದೀಯೋ
ನೋಡಿದ್ದೀರಾ
ನೋಡಿದ್ದೀರಿ
ನೋಡಿದ್ದು
ನೋಡಿದ್ದೆ
ನೋಡಿದ್ದೇನೆ
ನೋಡಿದ್ದೇ-ನೆಂದರೆ
ನೋಡಿದ್ದೇವೆ
ನೋಡಿ-ಬಿಟ್ಟಿದ್ದೇನೆ
ನೋಡಿ-ಬಿಡು
ನೋಡಿಯೂ
ನೋಡಿಯೇ
ನೋಡಿ-ರ-ಬೇಕೆಂದು
ನೋಡಿ-ರ-ಲಿಲ್ಲವೆ
ನೋಡಿ-ರುವನೋ
ನೋಡಿ-ರುವರು
ನೋಡಿ-ರು-ವು-ದ-ರಿಂದಲೇ
ನೋಡಿ-ರುವೆ
ನೋಡಿ-ರು-ವೆನು
ನೋಡಿ-ರುವೆ-ಯಲ್ಲವೆ
ನೋಡಿ-ರು-ವೆಯಾ
ನೋಡಿ-ರು-ವೆವು
ನೋಡಿಲ್ಲ
ನೋಡಿಲ್ಲವೆ
ನೋಡಿಲ್ಲ-ವೆಂದು
ನೋಡಿಲ್ಲವೇ
ನೋಡಿಲ್ಲ-ವೇನು
ನೋಡಿಲ್ಲವೊ
ನೋಡಿಲ್ಲವೋ
ನೋಡು
ನೋಡುತ
ನೋಡು-ತಿ-ರುವೆ
ನೋಡು-ತಿ-ರುವೆನು
ನೋಡುತ್ತ
ನೋಡುತ್ತದೆ
ನೋಡುತ್ತಲೂ
ನೋಡುತ್ತಾ
ನೋಡುತ್ತಿತ್ತು
ನೋಡುತ್ತಿದ್ದ
ನೋಡುತ್ತಿದ್ದಂತೆ
ನೋಡುತ್ತಿದ್ದನು
ನೋಡುತ್ತಿದ್ದರು
ನೋಡುತ್ತಿದ್ದೆನು
ನೋಡುತ್ತಿದ್ದೆವು
ನೋಡುತ್ತಿದ್ದೇನೆ
ನೋಡುತ್ತಿದ್ದೇವೆ
ನೋಡುತ್ತಿರು
ನೋಡುತ್ತಿರುವ
ನೋಡುತ್ತಿರು-ವರು
ನೋಡುತ್ತಿರು-ವು-ದನ್ನು
ನೋಡುತ್ತಿರು-ವು-ದೆಲ್ಲಾ
ನೋಡುತ್ತಿರುವೆ
ನೋಡುತ್ತಿರು-ವೆ-ಯೇನು
ನೋಡುತ್ತಿಲ್ಲವೆ
ನೋಡುತ್ತೀರಿ
ನೋಡುತ್ತೇನೆ
ನೋಡುತ್ತೇವೆ
ನೋಡುವ
ನೋಡು-ವಂತೆಯೇ
ನೋಡು-ವನು
ನೋಡು-ವರು
ನೋಡು-ವರೊ
ನೋಡು-ವರೋ
ನೋಡು-ವಿ-ಯಂತೆ
ನೋಡು-ವಿರಿ
ನೋಡು-ವು-ದಕ್ಕೂ
ನೋಡು-ವುದಕ್ಕೆ
ನೋಡು-ವುದಕ್ಕೆಂದು
ನೋಡು-ವು-ದನ್ನು
ನೋಡು-ವು-ದ-ರಿಂದ
ನೋಡು-ವು-ದ-ರಿಂದಲೇ
ನೋಡು-ವು-ದಲ್ಲ
ನೋಡು-ವು-ದಾದರೆ
ನೋಡು-ವು-ದಿಲ್ಲ
ನೋಡು-ವುದು
ನೋಡು-ವುದೇ
ನೋಡುವೆ
ನೋಡು-ವೆ-ಯಂತೆ
ನೋಡು-ವೆ-ಯೇನು
ನೋಡು-ವೆಯೊ
ನೋಡು-ವೆಯೋ
ನೋಡು-ವೆವು
ನೋಡೆ
ನೋಡೇ
ನೋಡೋಣ
ನೋಯಿ-ಸಲು
ನೋಯಿ-ಸು-ವುದು
ನೋಯುತ್ತಿ-ರು-ವು-ದ-ರಿಂದ
ನೋವನ್ನು
ನೋವಾಗಲು
ನೋವಾಗೆ
ನೋವಾ-ಯಿತು
ನೋವಿ-ಗಾಗಿ
ನೋವಿನ
ನೋವಿ-ನಲ್ಲಿ
ನೋವಿ-ರದ
ನೋವು
ನೋವುಂಟು-ಮಾಡ-ಬೇಡ
ನೋವುಂಟು-ಮಾಡ-ಬೇಡಿ
ನೋವು-ಕಷ್ಟ-ಗಳನ್ನನು-ಭವಿ-ಸಲೇ
ನೋವು-ಗಳಿಂದ
ನೋವು-ನಲಿ-ವಿನ
ನೋವು-ಸಂಕಟ-ಗಳ
ನೋವು-ಸಂಕಟ-ಗಳನ್ನು
ನೋವು-ಸಂಕಟ-ಗಳಿಂದ
ನೋವು-ಸಂಕಟದ
ನೌಕರಿ
ನೌಕರಿ-ಯಲ್ಲಿ
ನೌಪತ್ತಿನ
ನೌಮಿ
ನ್ನೊಳ-ಗಿನ
ನ್ಯಾಯ
ನ್ಯಾಯ-ದಂತೆ
ನ್ಯಾಯ-ವಾಗಿ
ನ್ಯಾಯ-ವಾದ
ನ್ಯಾಯವೇ
ನ್ಯಾಯ-ಸೂತ್ರ-ಗಳು
ನ್ಯಾಯಾಧಿ-ಪತಿ
ನ್ಯಾಯಾಧಿ-ಪತಿ-ಯಂತೆ
ನ್ಯಾಯಾನ್ಯಾಯ-ವನ್ನು
ನ್ಯಾಯ್ಯಾತ್
ನ್ಯಾಷನಲ್
ನ್ಯಾಸಂ
ನ್ಯೂಟನ್ನಿನ
ನ್ಯೂನತೆ
ನ್ಯೂನತೆ-ಗಳ
ನ್ಯೂನತೆ-ಗಳನ್ನು
ನ್ಯೂನತೆ-ಯನ್ನು
ನ್ಯೂನ-ತೆಯೂ
ನ್ಯೂನಾತೀತ-ವಾ-ದುವು
ನ್ಯೂಯಾರ್ಕಿನ
ನ್ಯೂಯಾರ್ಕಿ-ನಲ್ಲಿ
ನ್ಯೂಯಾರ್ಕಿನಲ್ಲಿದ್ದಾಗ
ನ್ಯೂಯಾರ್ಕಿ-ನಿಂದ
ನ್ಯೂಯಾರ್ಕ್
ಪಂಕ್ತಿ-ಗಳನ್ನು
ಪಂಕ್ತಿಯು
ಪಂಗಡ
ಪಂಗಡಕ್ಕೂ
ಪಂಗಡಕ್ಕೇ
ಪಂಗಡ-ಗಳ
ಪಂಗಡ-ಗಳನ್ನಾಗಲೀ
ಪಂಗಡ-ಗಳನ್ನು
ಪಂಗಡ-ಗಳನ್ನೆಲ್ಲ
ಪಂಗಡ-ಗಳಾಗಿ
ಪಂಗಡ-ಗಳಿಗೂ
ಪಂಗಡ-ಗಳಿಗೆ
ಪಂಗಡ-ಗಳು
ಪಂಗಡ-ಗಳೂ
ಪಂಗಡ-ಗಳೆಲ್ಲಾ
ಪಂಗ-ಡದ
ಪಂಗಡ-ದ-ವರೇ
ಪಂಗಡ-ವನ್ನು
ಪಂಗಡ-ವಾಗ-ಬೇಕು
ಪಂಗಡ-ವಾಗಿತ್ತು
ಪಂಗಡ-ವಾಗು-ವರು
ಪಂಚದಶೀ-ಕಾ-ರರು
ಪಂಚ-ಭೂತ-ಗಳಲ್ಲಿ
ಪಂಚ-ಭೂತ-ಗಳೂ
ಪಂಚ-ಭೂತದ
ಪಂಚಭೂ-ತಾತ್ಮಕ
ಪಂಚಮಃ
ಪಂಚವಟಿ
ಪಂಚವಟಿಯ
ಪಂಚ-ವಾಯು-ಗ-ಳಲ್ಲ
ಪಂಚಾಂಗ
ಪಂಚಾಂಗ-ವನ್ನು
ಪಂಚೇಂದ್ರಿಯ-ಗಳ
ಪಂಚೇಂದ್ರಿಯ-ಗಳನ್ನು
ಪಂಚೇಂದ್ರಿಯ-ಗಳಿ-ರುತ್ತವೆ
ಪಂಚೇಂದ್ರಿಯ-ನಿಗ್ರಹಸ್ತಪಃ
ಪಂಜರ
ಪಂಜಾಬು
ಪಂಜೊಂದು
ಪಂಡಿತ
ಪಂಡಿತನ
ಪಂಡಿತ-ನಾಗಿ-ರುವೆ
ಪಂಡಿತ-ನಾಗು-ವನು
ಪಂಡಿತನೊ
ಪಂಡಿತರ
ಪಂಡಿತ-ರನ್ನು
ಪಂಡಿತ-ರಾಗಿದ್ದರು
ಪಂಡಿತ-ರಾಗಿದ್ದರೂ
ಪಂಡಿತ-ರಿ-ಗಿಂತ
ಪಂಡಿತ-ರಿಗೆ
ಪಂಡಿತರು
ಪಂಡಿತ-ರು-ಗಳಿಗೆ
ಪಂಡಿತರೂ
ಪಂಡಿತ-ರೆಂಬರು
ಪಂಡಿತ-ರೆಲ್ಲರೂ
ಪಂಡಿತ-ರೆಲ್ಲಾ
ಪಂಡಿತ-ರೊಡನೆ
ಪಂಡಿತಾನಾಂ
ಪಂಡಿತೆಯ-ರಾಗಿ-ಬಿಟ್ಟಿದ್ದಾರೆ
ಪಂಡಿತೆ-ಯರು
ಪಂತವ
ಪಂಥ
ಪಂಥಕ್ಕೆ
ಪಂಥ-ಗಳ
ಪಂಥ-ಗಳನ್ನು
ಪಂಥ-ಗಳು
ಪಂಥದ
ಪಂಥ-ದಲ್ಲಿ
ಪಂಥ-ದಲ್ಲೂ
ಪಂಥ-ದ-ವ-ರಲ್ಲಿ
ಪಂಥ-ದ-ವ-ರಾಗಿ-ರ-ಬ-ಹುದು
ಪಂಥ-ವಾಗಿತ್ತು
ಪಂಥ-ವಾಗಿದೆ
ಪಂಥವೂ
ಪಂಥಾ
ಪಕ್ಕಕ್ಕೆ
ಪಕ್ಕ-ದಲ್ಲಿ
ಪಕ್ಕದಲ್ಲಿದ್ದ
ಪಕ್ಕ-ದಲ್ಲಿದ್ದ-ವರೊಬ್ಬರು
ಪಕ್ವಾನ್ನ
ಪಕ್ಷ
ಪಕ್ಷಕ್ಕೆ
ಪಕ್ಷ-ಗಳನ್ನೂ
ಪಕ್ಷದ
ಪಕ್ಷ-ದಲ್ಲಿ
ಪಕ್ಷ-ದ-ವ-ರಲ್ಲಿಯೂ
ಪಕ್ಷ-ದ-ವರು
ಪಕ್ಷ-ದಿಂದ
ಪಕ್ಷ-ಪಾತ
ಪಕ್ಷ-ಪಾತದ
ಪಕ್ಷ-ಪಾತ-ವನ್ನು
ಪಕ್ಷ-ಪಾತ-ವಿ-ರುವ
ಪಕ್ಷ-ಪಾತಿ
ಪಕ್ಷ-ಪಾತಿ-ಗಳಾಗಿದ್ದರೆ
ಪಕ್ಷ-ವನ್ನು
ಪಕ್ಷ-ವಾದಿ
ಪಕ್ಷ-ಹೀನ
ಪಕ್ಷಿ
ಪಕ್ಷಿ-ಗಳ
ಪಗಡೆ-ಯಾಡುತ್ತಿದ್ದರೆ
ಪಟ
ಪಟ-ವನು
ಪಟ-ವನ್ನು
ಪಟೀಯಸಿ-ಯಾದ
ಪಟು-ಗಳಾಗಿ
ಪಟ್ಟ
ಪಟ್ಟಕ್ಕಾಗಿ
ಪಟ್ಟಣ
ಪಟ್ಟ-ಣಕ್ಕೂ
ಪಟ್ಟ-ಣಕ್ಕೆ
ಪಟ್ಟ-ಣ-ಗಳಲ್ಲಿ
ಪಟ್ಟ-ಣದ
ಪಟ್ಟ-ಣ-ದಲ್ಲಿಯೂ
ಪಟ್ಟ-ಣವೇ
ಪಟ್ಟ-ರಲ್ಲದೆ
ಪಟ್ಟರೂ
ಪಟ್ಟ-ವನ್ನು
ಪಟ್ಟಿ
ಪಟ್ಟಿದ್ದರೆ
ಪಟ್ಟಿದ್ದೇನೆಂದು
ಪಟ್ಟಿ-ಯನ್ನು
ಪಟ್ಟಿ-ಯನ್ನೇಕೆ
ಪಟ್ಟು
ಪಟ್ಟು-ಕೊಂಡು
ಪಟ್ಟು-ಕೊಳ್ಳು-ವು-ದಿಲ್ಲ-ವಷ್ಟೆ
ಪಠಿ-ಸಿದರೆ
ಪಠಿ-ಸುತ್ತಿದ್ದರೂ
ಪಠಿ-ಸುತ್ತಿರಿ
ಪಠಿ-ಸು-ವು-ದ-ರಿಂದೇನೂ
ಪಡದೆ
ಪಡ-ಬೇಕಾಗಿಲ್ಲ
ಪಡಿ
ಪಡಿ-ಸಿ-ಕೊಂಡರೆ
ಪಡು-ತಿದ್ದ
ಪಡುತ್ತಿರುವ
ಪಡುತ್ತಿರು-ವ-ವರ
ಪಡುತ್ತೇನೆ
ಪಡುವ
ಪಡು-ವ-ಣದ
ಪಡು-ವ-ಣದಿ
ಪಡು-ವು-ದೊಂದೇ
ಪಡೆ
ಪಡೆದ
ಪಡೆ-ದಂತಿತ್ತು
ಪಡೆ-ದದ್ದನ್ನು
ಪಡೆ-ದದ್ದ-ರಿಂದ
ಪಡೆ-ದನು
ಪಡೆ-ದ-ಬಳಿಕ
ಪಡೆ-ದರು
ಪಡೆ-ದರೂ
ಪಡೆ-ದರೆ
ಪಡೆ-ದ-ವನು
ಪಡೆ-ದ-ವರು
ಪಡೆ-ದ-ವ-ರೆಂದು
ಪಡೆ-ದಾಗ
ಪಡೆ-ದಿದೆ
ಪಡೆ-ದಿದ್ದಕ್ಕಾಗಿ
ಪಡೆ-ದಿದ್ದರೆ
ಪಡೆ-ದಿದ್ದಾನೆ
ಪಡೆ-ದಿದ್ದಾರೆ
ಪಡೆ-ದಿದ್ದಾರೆಯೇ
ಪಡೆ-ದಿದ್ದಾರೆಯೊ
ಪಡೆ-ದಿದ್ದೇನೆ
ಪಡೆ-ದಿರ-ಬ-ಹುದು
ಪಡೆ-ದಿ-ರ-ಲಿಲ್ಲ
ಪಡೆ-ದಿರಿ
ಪಡೆ-ದಿರುತ್ತಾರೊ
ಪಡೆ-ದಿರುವ
ಪಡೆ-ದಿ-ರುವರೊ
ಪಡೆ-ದಿರು-ವ-ವರೂ
ಪಡೆ-ದಿ-ರು-ವುದಕ್ಕೆ
ಪಡೆ-ದಿ-ರು-ವು-ದನ್ನು
ಪಡೆ-ದಿ-ರು-ವು-ದ-ರಿಂದ
ಪಡೆ-ದಿಲ್ಲ
ಪಡೆದು
ಪಡೆ-ದುಕೊ
ಪಡೆ-ದು-ಕೊಂಡಿದ್ದೀಯೆ
ಪಡೆ-ದು-ಕೊಂಡಿರುತ್ತಾರೆ
ಪಡೆ-ದು-ಕೊಂಡು
ಪಡೆ-ದು-ಕೊಳ್ಳುವ
ಪಡೆ-ದು-ಕೊಳ್ಳುವ-ನೆಂಬು-ದನ್ನು
ಪಡೆ-ದು-ಕೊಳ್ಳು-ವುದಕ್ಕೋಸ್ಕರ
ಪಡೆ-ದು-ದ-ರಿಂದ
ಪಡೆದೆ
ಪಡೆ-ದೆ-ವೆಂದು-ಕೊಳ್ಳುತ್ತಾ-ರಲ್ಲಾ
ಪಡೆದೇ
ಪಡೆ-ಯ-ದ-ವನು
ಪಡೆ-ಯದಿದ್ದರೆ
ಪಡೆ-ಯದೆ
ಪಡೆ-ಯ-ಬಲ್ಲರು
ಪಡೆ-ಯ-ಬಲ್ಲ-ವ-ನಾದರೆ
ಪಡೆ-ಯ-ಬಲ್ಲೆ-ಯಾ-ದರೆ
ಪಡೆ-ಯ-ಬ-ಹುದು
ಪಡೆ-ಯ-ಬಹು-ದೆಂದು
ಪಡೆ-ಯ-ಬಹುದೊ
ಪಡೆ-ಯ-ಬೇಕಾಗಿ-ರು-ವು-ದ-ರಿಂದ
ಪಡೆ-ಯ-ಬೇ-ಕಾದರೆ
ಪಡೆ-ಯ-ಬೇಕು
ಪಡೆ-ಯ-ಬೇಕೆಂದು
ಪಡೆ-ಯ-ಬೇಕೆಂಬು-ದನ್ನು
ಪಡೆ-ಯ-ಲ-ಹುದು
ಪಡೆ-ಯಲಾ-ಗದು
ಪಡೆ-ಯ-ಲಾ-ಗು-ವು-ದಿಲ್ಲ
ಪಡೆ-ಯಲಾಗು-ವುದು
ಪಡೆ-ಯಲಾಗು-ವುದೇ
ಪಡೆ-ಯ-ಲಾರ
ಪಡೆ-ಯ-ಲಾರದೆ
ಪಡೆ-ಯ-ಲಾ-ರರು
ಪಡೆ-ಯ-ಲಾರಿರಿ
ಪಡೆ-ಯ-ಲಾರೆ
ಪಡೆ-ಯಲಿ
ಪಡೆ-ಯಲಿಚ್ಛಿ-ಸುವೆಯೊ
ಪಡೆ-ಯ-ಲಿಲ್ಲ
ಪಡೆ-ಯಲು
ಪಡೆ-ಯಲೂ
ಪಡೆ-ಯಿತಲ್ಲಿ
ಪಡೆ-ಯಿತು
ಪಡೆ-ಯುತ್ತ
ಪಡೆ-ಯುತ್ತಲೂ
ಪಡೆ-ಯುತ್ತವೆ
ಪಡೆ-ಯುತ್ತಾ
ಪಡೆ-ಯುತ್ತಾನೆ
ಪಡೆ-ಯುತ್ತಾನೋ
ಪಡೆ-ಯುತ್ತಾರೆ
ಪಡೆ-ಯುತ್ತಾ-ರೆಂದು
ಪಡೆ-ಯುತ್ತೀರಿ
ಪಡೆ-ಯುತ್ತೇನೆ
ಪಡೆ-ಯುವ
ಪಡೆ-ಯುವಂತಾ-ಗಲೆಂದು
ಪಡೆ-ಯು-ವಂತೆ
ಪಡೆ-ಯುವನು
ಪಡೆ-ಯುವರು
ಪಡೆ-ಯುವ-ರೆಂದಾಯಿತಲ್ಲವೆ
ಪಡೆ-ಯುವ-ವರು
ಪಡೆ-ಯುವಿ
ಪಡೆ-ಯುವಿರಿ
ಪಡೆ-ಯು-ವುದಕ್ಕಾಗಿ
ಪಡೆ-ಯು-ವುದಕ್ಕಾಗಿಯೆ
ಪಡೆ-ಯು-ವುದಕ್ಕಾ-ಗು-ವು-ದಿಲ್ಲ
ಪಡೆ-ಯುವುದಕ್ಕಿಂತ
ಪಡೆ-ಯು-ವುದಕ್ಕೆ
ಪಡೆ-ಯು-ವುದಕ್ಕೋಸ್ಕರ
ಪಡೆ-ಯುವುದ-ರಲ್ಲಿ
ಪಡೆ-ಯುವು-ದಾ-ಗಲಿ
ಪಡೆ-ಯು-ವುದು
ಪಡೆ-ಯು-ವುದೂ
ಪಡೆ-ಯು-ವುದೆಂದ-ರೇನು
ಪಡೆ-ಯು-ವುದೇ
ಪಡೆ-ಯು-ವು-ದೊಂದೇ
ಪಡೆ-ಯುವೆ
ಪಡೆ-ಯು-ವೆಯೊ
ಪಡೆ-ಯು-ವೆವೊ
ಪಡೆ-ಯು-ವೆವೋ
ಪಡೆ-ವಂತೆ
ಪಡೆ-ವ-ವ-ರೆಗೂ
ಪಣಕಿಡು
ಪಣ-ತೊಟ್ಟು
ಪಣಿಹಾಟಿ-ಯಲ್ಲಿ
ಪತಂಗಮ
ಪತಂಗವ
ಪತಂಜಲಿ
ಪತಂಜಲಿ-ಗಳ
ಪತಂಜಲಿಯೇ
ಪತತಿ
ಪತತ್ರಿನಿಚಯ
ಪತತ್ರೇ
ಪತನ
ಪತಾ-ಕನಿಚಯ
ಪತಿ
ಪತಿ-ಗಾಗಿ
ಪತಿತ
ಪತಿ-ತ-ರಾಗಿ
ಪತಿ-ತ-ರಾದ-ವರೂ
ಪತಿ-ಯಂತೆ
ಪತಿ-ಯನ್ನು
ಪತಿ-ಯಲ್ಲಿ-ರುವ
ಪತಿಯು
ಪತ್ನಿ
ಪತ್ನಿ-ಪುತ್ರ-ರನ್ನು
ಪತ್ರ
ಪತ್ರದ
ಪತ್ರ-ದಲ್ಲಿ
ಪತ್ರ-ದಲ್ಲಿದೆ
ಪತ್ರ-ದಲ್ಲಿಯೂ
ಪತ್ರ-ವನ್ನು
ಪತ್ರ-ವೊಂದನ್ನು
ಪತ್ರವ್ಯವ-ಹಾರ
ಪತ್ರಿಕೆ
ಪತ್ರಿಕೆ-ಗಳಲ್ಲಿ
ಪತ್ರಿಕೆ-ಗಳಿಲ್ಲದ
ಪತ್ರಿಕೆ-ಗಾಗಿ
ಪತ್ರಿ-ಕೆಗೆ
ಪತ್ರಿ-ಕೆಯ
ಪತ್ರಿಕೆ-ಯನ್ನು
ಪತ್ರಿಕೆ-ಯಲ್ಲಿ
ಪತ್ರಿಕೆ-ಯಿಂದ
ಪತ್ರೆ-ಗ-ಳಿದ್ದುವೊ
ಪಥ
ಪಥಃ
ಪಥ-ಗಳಿವೆ
ಪಥ-ದಲ್ಲೇ
ಪಥದಿ
ಪಥ-ನಾದ
ಪಥ-ವನ್ನು
ಪಥೆ
ಪದ
ಪದಂ
ಪದಕ್ಕೆ
ಪದ-ಗಳ
ಪದ-ಗ-ಳಲ್ಲ
ಪದ-ಗಳಿಂದ
ಪದ-ಗಳಿ-ಗಾಗಿ
ಪದ-ಗಳು
ಪದ-ತಲ-ದಲಿ
ಪದ-ತಲ-ದಲ್ಲಿ
ಪದ-ತಲ-ದಲ್ಲಿದ್ದಾಗಲೂ
ಪದ-ತಲೆ
ಪದದ
ಪದ-ದಡಿ
ಪದ-ದಲ್ಲಿದೆ
ಪದಪ್ರ-ಯೋಗ
ಪದರ
ಪದ-ರ-ಗಳ
ಪದ-ವನ್ನು
ಪದವಿ
ಪದ-ವಿಗೆ
ಪದ-ವಿ-ಗೇರು-ವ-ರೆಂಬ
ಪದ-ವಿ-ಯನ್ನೈದಿರ-ಬ-ಹುದು
ಪದ-ವಿ-ಯಲ್ಲಿ-ರುವನು
ಪದ-ವಿ-ಯೆಲ್ಲಾ
ಪದ-ವಿ-ಯೊಂದನ್ನು
ಪದ-ವೀ-ಧರ
ಪದ-ವೀಧ-ರರು
ಪದವೂ
ಪದವೇ
ಪದಸ್ಪರ್ಶ-ದಿಂದ
ಪದಾತಿ-ದಲ
ಪದಾನು-ಸ-ರಣ
ಪದಾರ್ಥ
ಪದಾರ್ಥ-ಗಳ
ಪದಾರ್ಥ-ಗಳನ್ನು
ಪದಾರ್ಥ-ಗಳನ್ನೂ
ಪದಾರ್ಥ-ಗಳನ್ನೊದಗಿ-ಸಲು
ಪದಾರ್ಥ-ಗಳಲ್ಲಿ
ಪದಾರ್ಥ-ಗಳಿಂದ
ಪದಾರ್ಥ-ಗಳಿಗೆ
ಪದಾರ್ಥ-ಗಳು
ಪದಾರ್ಥ-ಗಳೆಲ್ಲಾ
ಪದಾರ್ಥದ
ಪದಾರ್ಥ-ದಲ್ಲಿಯೂ
ಪದಾರ್ಥ-ವನ್ನು
ಪದಾರ್ಥ-ವಾಗಿ
ಪದಾರ್ಥವೂ
ಪದಾರ್ಥ-ವೆಂದು
ಪದೆ
ಪದೇ
ಪದೇ-ಪದೇ
ಪದ್ದತಿ
ಪದ್ದತಿ-ಗಳು
ಪದ್ದ-ತಿಯ
ಪದ್ದತಿ-ಯನ್ನು
ಪದ್ದತಿ-ಯಿರ-ಬೇಕೆಂದು
ಪದ್ಧತಿ
ಪದ್ಧತಿ-ಗಳು
ಪದ್ಧತಿ-ಗಳೊಂದಿಗೆ
ಪದ್ಧತಿಯ
ಪದ್ಧತಿ-ಯಂತಲ್ಲ
ಪದ್ಧತಿ-ಯನ್ನು
ಪದ್ಧತಿ-ಯಲ್ಲಿ
ಪದ್ಧತಿಯು
ಪದ್ಮ
ಪದ್ಮಕ್ಕೆ
ಪದ್ಮ-ನೇತ್ರಂ
ಪದ್ಮಾ-ನ-ದಿಯ
ಪದ್ಮಾಸನ
ಪದ್ಮಾಸನ-ದಲ್ಲಿ
ಪದ್ಯ
ಪದ್ಯ-ಗಳು
ಪದ್ಯ-ದಲ್ಲಿ
ಪದ್ಯ-ರೂಪ-ದಲ್ಲಿ-ರುವ
ಪದ್ಯ-ವನ್ನು
ಪದ್ಯ-ವನ್ನೋದಿದ
ಪದ್ಯ-ವನ್ನೋದುವ
ಪಯಣ
ಪಯಣ-ದಾದಿ
ಪಯಣ-ವ-ನಿಲ್ಲೆ
ಪಯ-ತಹೆ
ಪಯಸಾಮರ್ಣವ
ಪರ
ಪರಂಧಾಮ-ದಲ್ಲಿ
ಪರಂಧಾಮ-ವನ್ನು
ಪರಂಪರೆ-ಗಳ
ಪರಂಪರೆ-ಗಳುಳ್ಳ
ಪರಂಪರೆಯ
ಪರಂಪರೆ-ಯಿಂದ
ಪರಕ್ಷಣೇ
ಪರ-ತತ್ತ್ವದ
ಪರ-ದಾ-ಡಿದ
ಪರದೆ
ಪರ-ದೆಯ
ಪರ-ದೇಶಕ್ಕೆ
ಪರ-ದೇಶ-ಗಳಲ್ಲೆಲ್ಲಾ
ಪರ-ದೇಶ-ಗಳಿಗೆ
ಪರ-ದೇಶಿ-ಯರ
ಪರ-ದೇಶೀ-ಯರ
ಪರ-ಧರ್ಮ-ಗ-ಳಾದ
ಪರ-ಧರ್ಮ-ವನ್ನವ-ಲಂಬಿಸಿದ
ಪರ-ಪುರುಷನ
ಪರಪ್ರ-ಯೋ-ಜನಕ್ಕಾಗಿಯೂ
ಪರಬ್ರಹ್ಮನ
ಪರಬ್ರಹ್ಮ-ನೆಂದು
ಪರಮ
ಪರ-ಮ-ಗುರಿ
ಪರ-ಮ-ಗುರಿ-ಯೆಂದು
ಪರ-ಮಜ್ಞಾನ-ದಲ್ಲಿ
ಪರ-ಮ-ಧರ್ಮ
ಪರ-ಮಧ್ವನಿ-ಯದು
ಪರ-ಮ-ಪುರುಷಾರ್ಥ-ವೆಂಬು-ದನ್ನು
ಪರ-ಮ-ಪೂ-ರವು
ಪರ-ಮಪ್ರಾರ್ಥನೆ
ಪರ-ಮಪ್ರೇಮಕ್ಕೆ
ಪರ-ಮಪ್ರೇಮ-ದಿಂದ
ಪರ-ಮ-ಭಕ್ತ-ರಾದ
ಪರ-ಮ-ಭಕ್ತಿ-ಯನ್ನು
ಪರ-ಮ-ಮ-ಮೃತಂ
ಪರ-ಮ-ಶಾಂತಿ-ಯಲ್ಲಿ
ಪರ-ಮ-ಶಿ-ವನು
ಪರ-ಮ-ಶಿವ-ನೆಂಬ
ಪರ-ಮಶ್ರೇಷ್ಠ
ಪರ-ಮ-ಸತ್ಯ
ಪರ-ಮ-ಸತ್ಯದ
ಪರ-ಮ-ಸತ್ಯವೆ
ಪರ-ಮ-ಸುಖದ
ಪರ-ಮ-ಹಂಸ
ಪರ-ಮ-ಹಂಸರ
ಪರ-ಮ-ಹಂಸ-ರಂಥ
ಪರ-ಮ-ಹಂಸ-ರನ್ನು
ಪರ-ಮ-ಹಂಸ-ರಲ್ಲಿ
ಪರ-ಮ-ಹಂಸ-ರಾದರೊ
ಪರ-ಮ-ಹಂಸ-ರಿಗೆ
ಪರ-ಮ-ಹಂಸರು
ಪರ-ಮ-ಹಂಸರೆ
ಪರ-ಮಾಣು
ಪರ-ಮಾಣು-ಕಾಯ
ಪರ-ಮಾತ್ಮ
ಪರ-ಮಾತ್ಮನ
ಪರ-ಮಾತ್ಮ-ನನ್ನು
ಪರ-ಮಾತ್ಮ-ನನ್ನೂ
ಪರ-ಮಾತ್ಮ-ನಲ್ಲಿ
ಪರ-ಮಾತ್ಮ-ನಾದ
ಪರ-ಮಾತ್ಮ-ನಿ-ಗಿಂತ
ಪರ-ಮಾತ್ಮ-ನಿ-ಗೋಸ್ಕರ
ಪರ-ಮಾತ್ಮ-ನೆಂದು
ಪರ-ಮಾತ್ಮ-ನೊ-ಡನೆ
ಪರ-ಮಾದ್ವೈತ-ವನ್ನು
ಪರ-ಮಾ-ನಂದದ
ಪರ-ಮಾನ್ನ-ವನ್ನು
ಪರ-ಮಾ-ಮೃ-ತವ
ಪರ-ಮಾರ್ಥ
ಪರ-ಮಾರ್ಥಜ್ಞಾನ-ದಲ್ಲಿ
ಪರ-ಮಾರ್ಥ-ದಿಂದ
ಪರ-ಮಾಸಿ
ಪರಮೇ
ಪರ-ಮೇಶ್ವರ-ನಾಗಿ-ರುವ-ವ-ನೊಬ್ಬ-ನಿಗೇ
ಪರರ
ಪರ-ರನ್ನು
ಪರ-ರಿ-ಗೋಸ್ಕರ
ಪರರು
ಪರ-ಲೋಕ-ದಲ್ಲಿ
ಪರ-ವಶತೆ-ಯಲ್ಲೇ
ಪರ-ವಶ-ರಾಗ-ಬೇಡಿ
ಪರ-ವಶ-ರಾಗಿಯೂ
ಪರ-ವಶ-ರಾದರು
ಪರ-ವಾಗಿ
ಪರ-ವಾಗಿಯೂ
ಪರವು
ಪರವೂ
ಪರ-ಶನ
ಪರ-ಶು-ರಾಮ
ಪರ-ಸ-ತಿಯ
ಪರಸೇವಾ-ತತ್ಪರತೆ
ಪರಸೇ-ವಾರ್ಥ-ವಾಗಿ
ಪರಸ್ಪರ
ಪರ-ಹಿತ
ಪರ-ಹಿತ-ಕರ-ಣಾಯ
ಪರ-ಹಿ-ತಕ್ಕಾಗಿ
ಪರ-ಹಿ-ತಕ್ಕಾಗಿಯೇ
ಪರ-ಹಿ-ತಾರ್ಥ-ವಾಗಿ
ಪರ-ಹಿತೇಚ್ಛೆಯು
ಪರಾಂ
ಪರಾಂಚಿ
ಪರಾಕಾಷ್ಠೆ
ಪರಾಕಾಷ್ಠೆ-ಯನ್ನು
ಪರಾಕಾಷ್ಠೆ-ಯಲ್ಲಿ
ಪರಾಕಾಷ್ಠೆ-ಯ-ವರೆಗೆ
ಪರಾಕ್ರಮ
ಪರಾಕ್ರಮ-ಶಾಲಿ-ಯಾದ
ಪರಾಜಿ-ತರಾ-ದಂತಾಗಿ
ಪರಾಜಿ-ತರಾ-ದು-ದ-ರಿಂದ
ಪರಾ-ಧೀನ-ನಾಗಿ
ಪರಾ-ಭಕ್ತಿ
ಪರಾ-ಭಕ್ತಿಯ
ಪರಾ-ಭಕ್ತಿಯೇ
ಪರಾಯಣತೆ
ಪರಾಯಣ-ತೆ-ಯನ್ನು
ಪರಾಯಣ-ತೆಯೇ
ಪರಾಯಣರೂ
ಪರಾಯೆ
ಪರಾರಿ-ಯಾಗುವೆನು
ಪರಾರ್ಥ
ಪರಾರ್ಥ-ವನ್ನು
ಪರಾರ್ಥ-ವಾಗಿ
ಪರಾರ್ಥ-ವಾದ
ಪರಿಕಿ-ಸಲು
ಪರಿ-ಗಣಿ-ಸಲ್ಪಡುತ್ತದೆ
ಪರಿಚಯ
ಪರಿಚಯ-ಮಾಡಿ-ಸಿ-ದರು
ಪರಿಚಯ-ವನ್ನು
ಪರಿಚಯ-ವಾಗಿದ್ದ-ರಿಂದ
ಪರಿಚಯ-ವಾ-ಗಿ-ರಲಿಲ್ಲ
ಪರಿಚಯ-ವಿಲ್ಲ-ದಿದ್ದರೂ
ಪರಿಚಯವು
ಪರಿಚಾಲ-ನವೆ
ಪರಿಚಿ-ತ-ನಾಗಿ
ಪರಿ-ಚಿತ-ರಾದ
ಪರಿ-ಚಿತರೂ
ಪರಿಜ್ಞಾನ
ಪರಿಜ್ಞಾನ-ದಿಂದ
ಪರಿಜ್ಞಾನ-ವನ್ನೇ
ಪರಿಜ್ಞಾನ-ವೆಲ್ಲ
ಪರಿಣ-ತರು
ಪರಿಣತ-ವಾಗದ
ಪರಿಣ-ತ-ವಾಗಿ
ಪರಿಣತಿ-ಯುಂಟಾಗುವು-ದೆಂಬು-ದನ್ನು
ಪರಿಣಮಿ-ಸಿತು
ಪರಿಣಮಿ-ಸಿದ
ಪರಿಣಮಿ-ಸಿದಂತಾ-ದರು
ಪರಿಣಮಿ-ಸಿದೆ
ಪರಿಣಮಿಸಿ-ಬಿಡು-ವನು
ಪರಿಣಮಿ-ಸುತ್ತದೆ
ಪರಿಣಮಿ-ಸುತ್ತಾರೆ
ಪರಿಣಮಿ-ಸುತ್ತಾ-ರೆಯೆ
ಪರಿಣಾಮ
ಪರಿಣಾ-ಮ-ಗಳನ್ನು
ಪರಿಣಾಮ-ಗಳಿಂದ
ಪರಿಣಾಮ-ಗಳೇ
ಪರಿಣಾಮ-ಗೊಳಿ-ಸು-ವುದಕ್ಕಾಗಿಯೇ
ಪರಿಣಾ-ಮದ
ಪರಿಣಾಮ-ದಲ್ಲಿ-ರು-ವು-ದೆಲ್ಲ
ಪರಿಣಾಮ-ದಿಂದ
ಪರಿಣಾಮ-ದಿಂದುಂಟಾದ
ಪರಿಣಾಮ-ವನ್ನು
ಪರಿಣಾಮ-ವನ್ನುಂಟು-ಮಾ-ಡಲು
ಪರಿಣಾಮ-ವನ್ನೂ
ಪರಿಣಾಮ-ವಾಗಿ
ಪರಿಣಾಮ-ವಾಗಿಯೇ
ಪರಿಣಾಮ-ವಾ-ದಕ್ಕೆ
ಪರಿಣಾಮವು
ಪರಿಣಾಮ-ವುಂಟಾಗು-ವುದು
ಪರಿಣಾಮ-ವೆಲ್ಲವೂ
ಪರಿಣಾಮವೇ
ಪರಿಣಿ-ತರು
ಪರಿತಪಿ-ಸಿದೆ
ಪರಿ-ತಾಪ-ಪಟ್ಟು-ಕೊಂಡು
ಪರಿ-ತಾಪಿ-ಯಾದೆನ್ನ
ಪರಿತ್ಯ-ಜಿಸಿ
ಪರಿತ್ಯಜಿಸು
ಪರಿತ್ಯಾಗ
ಪರಿಧಿ
ಪರಿಧಿಯ
ಪರಿ-ಪಾಲ-ನೆಯೇ
ಪರಿ-ಪಾಲಿ-ಸಿದರೆ
ಪರಿ-ಪೂರ್ಣ
ಪರಿ-ಪೂರ್ಣತೆ
ಪರಿ-ಪೂರ್ಣ-ತೆಗೆ
ಪರಿ-ಪೂರ್ಣ-ತೆ-ಯನ್ನು
ಪರಿ-ಪೂರ್ಣ-ತೆಯೂ
ಪರಿ-ಪೂರ್ಣ-ನಾಗು
ಪರಿ-ಪೂರ್ಣ-ನಾ-ದಾತನು
ಪರಿ-ಪೂರ್ಣನು
ಪರಿ-ಪೂರ್ಣರು
ಪರಿ-ಪೂರ್ಣ-ವಾಗಿ
ಪರಿ-ಪೂರ್ಣ-ವಾದ
ಪರಿ-ಭಾವಿಸಿ
ಪರಿ-ಭಾಷೆ-ಯಲ್ಲಿ
ಪರಿಮಳ
ಪರಿಮಳ-ದೊಲವ
ಪರಿಮಾಣಕ್ಕನು-ಸಾರ
ಪರಿಮಾ-ಣವು
ಪರಿಯ-ದೆಂತು
ಪರಿ-ವರ್ತ-ನದಿಂದ
ಪರಿ-ವರ್ತನ-ಶೀಲ-ವಾಗಿ-ರುವ
ಪರಿ-ವರ್ತನೆ
ಪರಿ-ವರ್ತನೆ-ಯನ್ನು
ಪರಿ-ವರ್ತಿತ-ನಾಗಿದ್ದನು
ಪರಿ-ವಾರ-ದಲ್ಲಿದ್ದ-ವ-ರೆಲ್ಲರೂ
ಪರಿ-ವಾರ-ವನ್ನು
ಪರಿ-ವಿಲ್ಲ
ಪರಿವೆ
ಪರಿವ್ರ-ಜ-ನದ
ಪರಿವ್ರಾಜಕ
ಪರಿ-ಶೀಲಿಸು
ಪರಿಶುದ್ದ
ಪರಿ-ಶುದ್ಧ
ಪರಿ-ಶುದ್ಧ-ಗೊಳಿ-ಸಲು
ಪರಿ-ಶುದ್ಧ-ಗೊಳಿಸು-ವುವು
ಪರಿ-ಶುದ್ಧ-ರಾಗಿ
ಪರಿ-ಶುದ್ಧ-ರಾಗುವರು
ಪರಿ-ಶುದ್ಧರು
ಪರಿ-ಶುದ್ಧ-ಳೆಂದು
ಪರಿ-ಶುದ್ಧ-ವಾಗಲು
ಪರಿ-ಶುದ್ಧ-ವಾಗಿ
ಪರಿ-ಶುದ್ಧ-ವಾಗಿದೆ
ಪರಿ-ಶುದ್ಧ-ವಾಗಿದ್ದರೆ
ಪರಿ-ಶುದ್ಧ-ವಾಗಿ-ರ-ಬಲ್ಲವು
ಪರಿ-ಶುದ್ಧ-ವಾಗು-ವುದೋ
ಪರಿ-ಶುದ್ಧ-ವಾದ
ಪರಿ-ಶುದ್ಧ-ವಾದಾಗ
ಪರಿ-ಶುದ್ಧಾತ್ಮ-ರಾ-ಗಿ-ರುವರೊ
ಪರಿಶೋಭಿತ-ವಾಗಿದೆ
ಪರಿಶ್ರಮ
ಪರಿಶ್ರಮ-ಗಳಿಂದಾಗಿ
ಪರಿಶ್ರಮದ
ಪರಿಶ್ರಮ-ದಿಂದಲೂ
ಪರಿಶ್ರಮ-ದಿಂದಲೇ
ಪರಿಶ್ರಮ-ಪಟ್ಟ-ರೆಂಬು-ದನ್ನು
ಪರಿಶ್ರಮ-ಪ-ಡು-ವುವು
ಪರಿಶ್ರಮ-ವನ್ನೂ
ಪರಿಷ್ಕ-ರಿಸಿ
ಪರಿಷ್ಕಾರ-ವಾಗುತ್ತದೆ
ಪರಿಸ-ರಕ್ಕೂ
ಪರಿಸ್ಥಿತಿ
ಪರಿಸ್ಥಿತಿ-ಗಳನ್ನು
ಪರಿಸ್ಥಿತಿ-ಗಳಲ್ಲಿಯೂ
ಪರಿಸ್ಥಿತಿ-ಗ-ಳಲ್ಲೂ
ಪರಿಸ್ಥಿತಿಯ
ಪರಿಸ್ಥಿತಿ-ಯನ್ನು
ಪರಿಸ್ಥಿತಿ-ಯಲ್ಲಿಯೂ
ಪರಿಸ್ಥಿತಿ-ಯಲ್ಲೂ
ಪರಿಹರಿಸ-ಬೇಕಾಗಿದೆ
ಪರಿಹ-ರಿ-ಸಲು
ಪರಿಹ-ರಿಸಿ-ದಿರಿ
ಪರಿಹರಿ-ಸುವುದ-ರಲ್ಲಿ
ಪರಿ-ಹಾರ
ಪರಿ-ಹಾರ-ಗಳ
ಪರಿ-ಹಾರ-ವನ್ನು
ಪರಿ-ಹಾರ-ವಾಗುತ್ತದೆ
ಪರಿ-ಹಾರ-ವಾಗು-ವ-ಹಾಗಿದ್ದರೆ
ಪರಿ-ಹಾರ-ವಾಗು-ವುದು
ಪರಿ-ಹಾರ-ವಾಗು-ವುವು
ಪರಿ-ಹಾರವೇ
ಪರಿ-ಹಾಸ್ಯ
ಪರಿ-ಹಾಸ್ಯ-ಗಳನ್ನು
ಪರೀಕ್ಷಿಸ-ಬೇಕೆಂದು
ಪರೀಕ್ಷಿಸಿ
ಪರೀಕ್ಷಿಸಿ-ಕೊಳ್ಳ-ಬೇಕು
ಪರೀಕ್ಷಿ-ಸಿದ
ಪರೀಕ್ಷಿಸಿ-ದರೆ
ಪರೀಕ್ಷಿ-ಸುತ್ತಾ
ಪರೀಕ್ಷಿ-ಸು-ವುದಕ್ಕೆ
ಪರೀಕ್ಷೆ
ಪರೀಕ್ಷೆಗೆ
ಪರೀಕ್ಷೆ-ಯನ್ನು
ಪರೀಕ್ಷೆ-ಯೆಂಬ
ಪರೆ
ಪರೇರ್
ಪರೋಕ್ಷ-ವಾಗಿ
ಪರೋಕ್ಷ-ವಾಗಿ-ಯಾದರೂ
ಪರೋಪ-ಕಾರದ
ಪರೋಪ-ಕಾರ-ವೆಂದೇ
ಪರೋಪ-ಕಾರ-ವೆಂಬ
ಪರ್ಯಂತ
ಪರ್ಯಂತ-ವಾಗಿ
ಪರ್ಯನ್ತ-ವಾದ
ಪರ್ಯಾಯ
ಪರ್ಯಾಲೋಚನೆ
ಪರ್ವತ
ಪರ್ವ-ತಕ್ಕೂ
ಪರ್ವತ-ಗಳಲ್ಲಿ
ಪರ್ವತ-ಗಳಿಗೆ
ಪರ್ವತ-ಗಹ್ವರ
ಪರ್ವತ-ಚೂಡಾ
ಪರ್ವ-ತದ
ಪರ್ವತ-ದಲ್ಲಿ-ರುವವೋ
ಪರ್ವತ-ವಾಸಿ-ಯನ್ನು
ಪಲ
ಪಲಾಯನ
ಪಲಾಯಿಯೆ
ಪಲಾ-ವನ್ನು
ಪಲ್ಯ
ಪಲ್ಯ-ವನ್ನು
ಪಲ್ಯ-ವನ್ನೇ
ಪಲ್ಲ-ವಿ-ಗಳು
ಪಳಗಿದ
ಪವನ
ಪವನ-ವದು
ಪವಾಡ-ಗಳು
ಪವಾಡ-ಗಳೆಲ್ಲವೂ
ಪವಾ-ಹಾರಿ
ಪವಾ-ಹಾರಿ-ಬಾಬರ-ವ-ರಿಗೆ
ಪವಾ-ಹಾರಿ-ಬಾಬಾ
ಪವಾ-ಹಾರಿ-ಬಾಬಾ-ರವ-ರಲ್ಲಿ
ಪವಿತ್ರ
ಪವಿತ್ರಂ
ಪವಿತ್ರ-ಜೀವನ-ದಿಯ
ಪವಿತ್ರ-ತಮ-ರೆಂದು
ಪವಿತ್ರತೆ
ಪವಿತ್ರ-ತೆಯ
ಪವಿತ್ರ-ತೆ-ಯನ್ನು
ಪವಿತ್ರ-ತೆ-ಯನ್ನೆಣಿ-ಸುವ
ಪವಿತ್ರ-ತೆ-ಯಾ-ಗಿದೆಯೋ
ಪವಿತ್ರ-ಭಾವ
ಪವಿತ್ರ-ಭಾವ-ಗಳನ್ನೆಲ್ಲಾ
ಪವಿತ್ರ-ರಾಗ-ಬೇಕು
ಪವಿತ್ರ-ರಾಗಿದ್ದರೆ
ಪವಿತ್ರ-ರಾಗಿ-ರುತ್ತಾರೆಯೋ
ಪವಿತ್ರ-ರಾದ
ಪವಿತ್ರರು
ಪವಿತ್ರ-ವಾಗಿ
ಪವಿತ್ರ-ವಾಗಿವೆ
ಪವಿತ್ರ-ವಾಗು-ವುದೆಂದು
ಪವಿತ್ರ-ವಾದ
ಪವಿತ್ರ-ವಾ-ದುದು
ಪವಿತ್ರ-ವಾದುದೇ
ಪವಿತ್ರಾತ್ಮ
ಪವಿತ್ರಾತ್ಮ-ನಾಗಲು
ಪವಿತ್ರಾತ್ಮ-ರಾದ
ಪವಿತ್ರಾತ್ಮರು
ಪಶು
ಪಶು-ಗಳ
ಪಶು-ಗಳಿ-ಗಾಗಿ
ಪಶು-ಗಳಿಗೆ
ಪಶು-ಪಕ್ಷಿ
ಪಶು-ಪಕ್ಷಿ-ಗಳ
ಪಶು-ಪಕ್ಷಿ-ಗಳೂ
ಪಶು-ಪತಿ
ಪಶು-ಬಲಿ
ಪಶು-ರಕ್ಷಣೆಯ
ಪಶು-ವಿ-ಗೋಸ್ಕರ
ಪಶು-ಶಾಲೆ-ಗಳ
ಪಶ್ಚಾತ್ತಾಪ-ವನ್ನೂ
ಪಶ್ಚಿಮ
ಪಶ್ಚಿ-ಮದ
ಪಶ್ಚಿಮ-ದಲ್ಲಿ
ಪಶ್ಚಿಮ-ದಲ್ಲಿನ
ಪಶ್ಚಿಮಾಭಿ-ಮುಖ-ವಾಗಿ
ಪಶ್ಯೇದ-ಕರ್ಮಣಿ
ಪಸರಿಸಿ-ರುವೆ
ಪಸರಿಸುವ
ಪಸರಿಸುವನು
ಪಾಂಚ-ಭೌತಿಕ
ಪಾಂಡಿತ್ಯ
ಪಾಂಡಿತ್ಯಕ್ಕೆ
ಪಾಂಡಿತ್ಯಕ್ಷೇತ್ರಕ್ಕೆ
ಪಾಂಡಿತ್ಯ-ವನ್ನು
ಪಾಂಡಿತ್ಯ-ವನ್ನೂ
ಪಾಂಡಿತ್ಯ-ವಲ್ಲ
ಪಾಂಡಿತ್ಯ-ವೆಲ್ಲ-ವನ್ನೂ
ಪಾಕ-ಶಾಲೆ-ಯಲ್ಲಿ
ಪಾಕ-ಶಾಸ್ತ್ರ
ಪಾಕ್ಷಿಕ
ಪಾಗಲ್
ಪಾಗು
ಪಾಛೆ
ಪಾಠ
ಪಾಠ-ಕ-ರಿಗೆ
ಪಾಠ-ಗಳನ್ನು
ಪಾಠ-ದಂತೆ
ಪಾಠ-ವನ್ನು
ಪಾಠ-ವನ್ನೂ
ಪಾಠ-ಶಾಲೆ
ಪಾಠ-ಶಾಲೆ-ಗಳಲ್ಲಿ
ಪಾಠ-ಶಾಲೆಯ
ಪಾಠ-ಶಾಲೆ-ಯನ್ನಿಟ್ಟು
ಪಾಠ-ಶಾಲೆ-ಯನ್ನು
ಪಾಠ-ಶಾಲೆ-ಯಲ್ಲಿ
ಪಾಡನ್ನು
ಪಾಡಿನಲಿ
ಪಾಡು
ಪಾಡು-ಗಳು
ಪಾಡೇನು
ಪಾಡ್ಯಮಿಯ
ಪಾಣಿ-ನಿಯ
ಪಾತ-ಕಕ್ಕಿಂತ
ಪಾತಾಯ್
ಪಾತಾಲ
ಪಾತಾಳ
ಪಾತಾಳಕೆ
ಪಾತಿವ್ರತ್ಯ
ಪಾತಿವ್ರತ್ಯವೇ
ಪಾತುಮಾ-ಮಂತ್ರ-ಯಾಮಃ
ಪಾತ್ರ
ಪಾತ್ರ-ಗಳ-ನಭಿ-ನಯಿ-ಸ-ಬೇಕು
ಪಾತ್ರ-ನಾಗಿದ್ದನು
ಪಾತ್ರ-ನಾಗುವೆ
ಪಾತ್ರ-ನಾದ
ಪಾತ್ರ-ಭೇದ-ಗಳಿಂದ
ಪಾತ್ರ-ರಾಗ-ಬೇ-ಕಾದರೆ
ಪಾತ್ರ-ರಾಗಿದ್ದಾರೆ
ಪಾತ್ರ-ರಾ-ಗಿ-ರುವ
ಪಾತ್ರ-ರಾ-ಗಿ-ರುವಿರಿ
ಪಾತ್ರ-ರಾದ-ವರ
ಪಾತ್ರ-ರಾದ-ವರು
ಪಾತ್ರರೂ
ಪಾತ್ರ-ವನ್ನು
ಪಾತ್ರೆ
ಪಾತ್ರೆ-ಗಳಲ್ಲಿ
ಪಾತ್ರೆಗೆ
ಪಾತ್ರೆ-ಯಲ್ಲಿ
ಪಾಥಾರ
ಪಾದ
ಪಾದ-ಕಮಲ-ಗಳನ್ನು
ಪಾದಕ್ಕೆ
ಪಾದ-ಗಳ
ಪಾದ-ಗಳನ್ನು
ಪಾದ-ಗಳಲ್ಲಿದೆ
ಪಾದ-ಗಳಿಗೆ
ಪಾದ-ಗಳು
ಪಾದ-ದಡಿ
ಪಾದ-ದೆಡೆ
ಪಾದ-ಧೂಳಿ-ಯನ್ನು
ಪಾದ-ಧೂಳಿ-ಯಿಂದ
ಪಾದ-ಪದ್ಮ-ಗಳನ್ನು
ಪಾದ-ಪದ್ಮ-ಗಳಲ್ಲಿ
ಪಾದ-ಪದ್ಮ-ಗಳಿಗೆ
ಪಾದ-ಪದ್ಮ-ದಲ್ಲಿ
ಪಾದ-ಪದ್ಮ-ವನ್ನು
ಪಾದ-ಪಾದ್ಮ-ದಲ್ಲಿ
ಪಾದ-ವನ್ನು
ಪಾದ-ಸೇವೆ
ಪಾದ-ಸೇವೆ-ಯನ್ನು
ಪಾದಸ್ಪರ್ಶ-ದಿಂದ
ಪಾದಾ-ಘಾತದಿ
ಪಾದುಕೆ-ಗಳನ್ನು
ಪಾದು-ಕೆಯ
ಪಾದ್ರಿ
ಪಾದ್ರಿ-ಗಳು
ಪಾನ
ಪಾನ-ಮಾಡಿ
ಪಾನ-ಮಾಡಿದ
ಪಾನ-ಮಾಡಿ-ದಲ್ಲದೆ
ಪಾನೀಯ
ಪಾನೆ
ಪಾಪ
ಪಾಪ-ಕರ
ಪಾಪಕೆ
ಪಾಪಕ್ಕೂ
ಪಾಪದ
ಪಾಪ-ದಲ್ಲಿಯೂ
ಪಾಪ-ದಿಂದ
ಪಾಪ-ಪುಣ್ಯ-ಗಳ
ಪಾಪ-ಪುಣ್ಯ-ಗಳನ್ನು
ಪಾಪ-ಪುಣ್ಯ-ಗಳಾವುವೂ
ಪಾಪ-ಪುಣ್ಯ-ಗಳೆಂಬ
ಪಾಪ-ಪುಣ್ಯಾತೀ-ತನು
ಪಾಪ-ಭಾರದಿ
ಪಾಪ-ವನ್ನು
ಪಾಪ-ವಿದ್ದರೆ
ಪಾಪವು
ಪಾಪ-ವೃತ್ತಿ-ಗ-ಳಲ್ಲೂ
ಪಾಪ-ವೆಂದು
ಪಾಪ-ವೆಂಬ
ಪಾಪ-ವೇನು
ಪಾಪಿ
ಪಾಪಿ-ಗಳನ್ನು
ಪಾಪಿ-ಗಳಿಗೆ
ಪಾಪಿಗೆ
ಪಾಪಿ-ಯಲ್ಲಿಯು
ಪಾಮರ
ಪಾಯ
ಪಾಯಿಚೊ
ಪಾಯೆ
ಪಾರ
ಪಾರಂ
ಪಾರಂಗತ-ರಾಗಲು
ಪಾರಂಪರ್ಯ-ವಾಗಿ
ಪಾರಕೆ
ಪಾರ-ಮಾರ್ಥಿಕ
ಪಾರ-ಮಾರ್ಥಿಕ-ವಾದ
ಪಾರ-ವನು
ಪಾರ-ವಿಲ್ಲದ
ಪಾರ-ವಿಲ್ಲ-ದಲೆ
ಪಾರವೇ
ಪಾರಾಗ-ದಿದ್ದಲ್ಲಿ
ಪಾರಾಗ-ಬಲ್ಲಿರಿ
ಪಾರಾಗ-ಬ-ಹುದು
ಪಾರಾಗ-ಬೇಕು
ಪಾರಾಗ-ಲಾರ
ಪಾರಾಗ-ಲಾ-ರರು
ಪಾರಾಗ-ಲಾರೆ
ಪಾರಾಗಲು
ಪಾರಾಗಿ
ಪಾರಾಗಿದ್ದಿದ್ದರೆ
ಪಾರಾಗುತ್ತೇವೆ
ಪಾರಾಗುವ
ಪಾರಾಗು-ವುದಕ್ಕೆ
ಪಾರಾಗು-ವು-ದನ್ನು
ಪಾರಾಗುವುದೇ
ಪಾರಾಗುವೆವು
ಪಾರಾದರು
ಪಾರಾದರೆ
ಪಾರಿಭಾಷಿಕ
ಪಾರಿವಾಳ
ಪಾರಿವಾಳ-ಗಳನ್ನು
ಪಾರಿವಾ-ಳದ
ಪಾರುಪತ್ಯವೆ
ಪಾರು-ಮಾಡ-ಬೇಕು
ಪಾರು-ಮಾಡಿ
ಪಾರು-ಮಾಡಿಸು
ಪಾರೆ
ಪಾರ್ಕ್
ಪಾರ್ಥಿವ
ಪಾರ್ಶ್ವ-ವಾಯು
ಪಾರ್ಸಿ-ಗಳಲ್ಲಿ
ಪಾಲನೆ
ಪಾಲಾ-ದರೂ
ಪಾಲಾಬಾರ
ಪಾಲಿ
ಪಾಲಿಗೆ
ಪಾಲಿ-ನಷ್ಟು
ಪಾಲಿ-ಭಾಷೆ-ಯಲ್ಲಿ
ಪಾಲಿ-ಸ-ದಿದ್ದರೆ
ಪಾಲಿ-ಸ-ಬ-ಹುದು
ಪಾಲಿ-ಸ-ಬೇಕಿಲ್ಲ
ಪಾಲಿ-ಸುತ್ತಾರೋ
ಪಾಲಿ-ಸುತ್ತಿತ್ತು
ಪಾಲಿ-ಸುತ್ತಿದ್ದೇನೆ
ಪಾಲಿ-ಸು-ವುದಕ್ಕೆ
ಪಾಲಿ-ಸು-ವು-ದ-ರಿಂದ
ಪಾಲಿ-ಸು-ವು-ದಿಲ್ಲ
ಪಾಲೀ
ಪಾಲು
ಪಾಲ್ಗರೆವ
ಪಾಲ್ಗೊಳ್ಳಲು
ಪಾಲ್-ಬಾಬು-ಗಳು
ಪಾಳು
ಪಾವನ-ಗೊಳಿಸಿ
ಪಾವಲಿಯ
ಪಾವಿತ್ರ್ಯ
ಪಾವಿತ್ರ್ಯತೆ
ಪಾಶ-ಗಳ
ಪಾಶವ
ಪಾಶ-ವದು
ಪಾಶ-ವನ್ನು
ಪಾಶೆ
ಪಾಶ್ಚಾತ್ಯ
ಪಾಶ್ಚಾತ್ಯ-ದೇಶ
ಪಾಶ್ಚಾತ್ಯ-ನಾಗಿ-ಬಿಡ-ಬೇಕೆಂಬ
ಪಾಶ್ಚಾತ್ಯರ
ಪಾಶ್ಚಾತ್ಯ-ರಲ್ಲಿ
ಪಾಶ್ಚಾತ್ಯ-ರಲ್ಲಿ-ರುವ
ಪಾಶ್ಚಾತ್ಯ-ರಾದ
ಪಾಶ್ಚಾತ್ಯ-ರಿಂದ
ಪಾಶ್ಚಾತ್ಯ-ರಿಗೆ
ಪಾಶ್ಚಾತ್ಯರು
ಪಾಷಾಣ-ಗಳನ್ನು
ಪಿಂಗಳಾ
ಪಿಂಜರ
ಪಿಂಜ-ರಾದಿವ
ಪಿಂಜರಾಪೋಲು-ಗಳು
ಪಿಂಡ
ಪಿಂಡ-ಗಳನ್ನು
ಪಿಂಡ-ವನ್ನು
ಪಿಚೆ
ಪಿಛೆ
ಪಿತನ
ಪಿತ-ನಂತೆ
ಪಿತಾ-ಪುತ್ರೆ
ಪಿತ್ರಾರ್ಜಿ-ತ-ವಾಗಿ
ಪಿನಾಕ-ಪಾಣಿ
ಪಿನಾಕ-ಪಾಣಿಯು
ಪಿಯರ್
ಪಿಶಾಚ
ಪಿಶಾಚ-ಸಿದ್ಧ-ನಾದ
ಪಿಶಾಚಿ-ಗಳನ್ನು
ಪಿಶಾಚಿ-ಗಳಲ್ಲಿನ
ಪಿಶಾಚಿ-ಯಾಗಿ
ಪಿಸುಗುಟ್ಟಿ-ಕೊಳ್ಳುತ್ತಿ-ರು-ವಿರಿ
ಪಿಸುಗುಟ್ಟಿ-ದರು
ಪಿಸು-ದನಿ-ಯನೆಚ್ಚ-ರಿಸಿ
ಪಿಸು-ದನಿ-ಯಲ್ಲಿ
ಪೀಠದ
ಪೀಠ-ದಿಂದೆದ್ದು
ಪೀಠ-ವನು
ಪೀಠವಿರ-ಬೇಕು
ಪೀಡಿ-ಸಲಿ
ಪೀಡಿ-ಸಲೇ-ಕಿನ್ನು
ಪೀಡಿ-ಸು-ವು-ದನ್ನು
ಪೀಡಿ-ಸು-ವು-ದರ
ಪೀತ
ಪೀತ್ವಾ
ಪೀಳಿಗೆ-ಗಳೆಲ್ಲಾ
ಪೀಳಿ-ಗೆಯ
ಪೀಳಿ-ಗೆ-ಯಿಂದ
ಪೀಳಿ-ಗೆಯೆನಿ-ಸುವ
ಪುಂಜದ
ಪುಂಡರ
ಪುಂಡರು
ಪುಟ
ಪುಟಿ-ಪುಟಿವ
ಪುಟಿವುದು
ಪುಟ್ಟ
ಪುಡಿ
ಪುಡಿ-ಪುಡಿ
ಪುಡಿ-ಪುಡಿ-ಮಾಡಿ-ಹಾಕುತ್ತಿದ್ದರು
ಪುಡಿ-ಪುಡಿ-ಮಾಡು
ಪುಡಿ-ಪುಡಿ-ಯಾ-ಗಲಿ
ಪುಡಿ-ಪುಡಿ-ಯಾಗಿ
ಪುಡಿ-ಪುಡಿ-ಯಾಗಿ-ಹುದು
ಪುಡಿ-ಪುಡಿ-ಯಾಗು-ವುದಕ್ಕೆ
ಪುಡಿ-ಮಾಡ-ಬ-ಹುದು
ಪುಣ್ಯ
ಪುಣ್ಯ-ಕಾಲ-ದಲ್ಲಿ
ಪುಣ್ಯಕೆ
ಪುಣ್ಯಕ್ಕೆ
ಪುಣ್ಯಕ್ಷೇತ್ರ-ದಲ್ಲಿ
ಪುಣ್ಯಕ್ಷೇತ್ರ-ವಾದ
ಪುಣ್ಯ-ಗ-ಳಲ್ಲ
ಪುಣ್ಯ-ಗಳೆಂಬವು
ಪುಣ್ಯ-ಗಳೆ-ರಡನ್ನೂ
ಪುಣ್ಯದ
ಪುಣ್ಯ-ದರ್ಶನ-ವನ್ನು
ಪುಣ್ಯ-ದಿಂದ
ಪುಣ್ಯ-ನದಿ-ಗಳ
ಪುಣ್ಯ-ನಾಮವ
ಪುಣ್ಯ-ನಾಮವೂ
ಪುಣ್ಯ-ಭೂಮಿ
ಪುಣ್ಯ-ಭೂಮಿ-ಯಾದ
ಪುಣ್ಯ-ವನ್ನಾಗಿ
ಪುಣ್ಯ-ವನ್ನೇ
ಪುಣ್ಯ-ವಶಾತ್
ಪುಣ್ಯ-ವಾಗಿ-ರ-ಬೇಕು
ಪುಣ್ಯ-ವಿತ್ತು
ಪುಣ್ಯ-ವಿದ್ದರೆ
ಪುಣ್ಯ-ವೇನು
ಪುಣ್ಯಾತ್ಮರು
ಪುತ್ತಳಿ
ಪುತ್ರ
ಪುತ್ರ-ತರೆ
ಪುತ್ರ-ತವ
ಪುತ್ರ-ರನ್ನೆಲ್ಲಾ
ಪುತ್ರರೂ
ಪುತ್ರ-ರೆಂದು
ಪುತ್ರ-ವಿ-ಯೋಗ-ದಿಂದ
ಪುತ್ರ-ಶೋಕ-ವನ್ನು
ಪುನಃ
ಪುನರಾ-ವೃತ್ತಿ-ಯಾಗುತ್ತಲೇ
ಪುನರಾ-ವೃತ್ತಿ-ಯಾಗುವುದು
ಪುನರುಜ್ಜೀವನ-ಗೊಳಿಸ-ಬೇಕೆಂದು
ಪುನರುಜ್ಜೀವನ-ಗೊಳಿ-ಸ-ಲಾ-ಗು-ವು-ದಿಲ್ಲ
ಪುನರುಜ್ಜೀವನ-ಗೊಳಿ-ಸಲು
ಪುನರುಜ್ಜೀವನ-ಗೊಳಿ-ಸಲೆಂದು
ಪುನರುಜ್ಜೀವನ-ಗೊಳಿಸಿ-ದನು
ಪುನರುಜ್ಜೀವನ-ಗೊಳ್ಳುವಂತೆ
ಪುನರುಜ್ಜೀವ-ವಾ-ಯಿತು
ಪುನರ್ಜನ್ಮ
ಪುನರ್ಜನ್ಮ-ವಿಲ್ಲ
ಪುನರ್ಜನ್ಮಾದಿ-ಗಳಲ್ಲಿ
ಪುನೀ-ತ-ನಾಗಿ-ಸಿ-ಕೊಳ್ಳುತ್ತಾನೋ
ಪುನೀತ-ನಾದೆ
ಪುನೀತ-ವಾಗು-ವುದು
ಪುರದ
ಪುರ-ಸತ್ತೇ
ಪುರಸ್ಕಾರ
ಪುರಾಣ
ಪುರಾಣ-ಗಳ
ಪುರಾಣ-ಗಳಲ್ಲಿ-ರುವ
ಪುರಾಣ-ಗಳೂ
ಪುರಾ-ಣದ
ಪುರಾಣ-ದಲ್ಲಿ
ಪುರಾಣಾದಿ-ಗಳಲ್ಲಿ
ಪುರಾಣಾ-ದಿ-ಗಳು
ಪುರಾ-ತನ
ಪುರಾ-ತನ-ವಾಗಿಲ್ಲವೋ
ಪುರಾವೆ-ಗಳಿ-ರುವ
ಪುರುಷ
ಪುರುಷ-ಕಾರ
ಪುರುಷ-ಕಾರದ
ಪುರುಷ-ಕಾರ-ವೆಂಬುದು
ಪುರುಷತ್ವದ
ಪುರುಷತ್ವ-ವನ್ನೇ
ಪುರುಷನ
ಪುರುಷ-ನಂತೆ
ಪುರುಷ-ನಾದ-ವನು
ಪುರುಷ-ನಿಗೂ
ಪುರುಷನು
ಪುರುಷ-ನೆಂದು
ಪುರುಷನೇ
ಪುರುಷರ
ಪುರುಷ-ರಲ್ಲೆಲ್ಲಾ
ಪುರುಷ-ರಾಗುವುದು
ಪುರುಷರು
ಪುರುಷ-ರೆಂದು
ಪುರುಷರೇ
ಪುರುಷ-ಸಿಂಹನ
ಪುರುಷಾರ್ಥ
ಪುರುಷಾರ್ಥ-ವಾ-ಯಿತು
ಪುರುಷಾರ್ಥ-ವಿದೆ
ಪುರುಷೋತ್ತಮ
ಪುರೋಭಿ-ವೃದ್ಧಿ
ಪುರೋ-ಹಿತ
ಪುರೋ-ಹಿತನ
ಪುರೋ-ಹಿತ-ನಿರಲೇ-ಬೇಕು
ಪುರೋ-ಹಿತನು
ಪುರೋ-ಹಿತ-ನೊಬ್ಬನೇ
ಪುರೋ-ಹಿತರ
ಪುರೋ-ಹಿತ-ರಿಂದ
ಪುರೋ-ಹಿತ-ರಿ-ಗಿಂತಲೂ
ಪುರೋ-ಹಿತ-ರಿಗೆ
ಪುರೋ-ಹಿತರು
ಪುರೋ-ಹಿತರೇ
ಪುರೋ-ಹಿತ-ಶಾಹಿ
ಪುಲ
ಪುಲ-ಕಿತ-ವಾ-ಯಿತು
ಪುಲ್ಲ
ಪುಷ್ಕಳ
ಪುಷ್ಟಿ-ಕರ-ವಾದ
ಪುಷ್ಟೀ-ಕರಿ-ಸುವ
ಪುಷ್ಪ
ಪುಷ್ಪ-ಗಳ
ಪುಷ್ಪ-ಗಳನ್ನು
ಪುಷ್ಪ-ದ-ಮೇಲೆ
ಪುಷ್ಪೆ
ಪುಸಲಾಯಿ-ಸಲ್ಪಟ್ಟ
ಪುಸ್ತಕ
ಪುಸ್ತಕ-ಗಳ
ಪುಸ್ತಕ-ಗಳನ್ನು
ಪುಸ್ತಕ-ಗಳನ್ನೆಲ್ಲಾ
ಪುಸ್ತಕ-ಗಳನ್ನೋದಿ
ಪುಸ್ತಕ-ಗಳನ್ನೋದುತ್ತಾ
ಪುಸ್ತಕ-ಗಳಲ್ಲಿ
ಪುಸ್ತಕ-ಗಳಿವೆ-ಯಲ್ಲ
ಪುಸ್ತಕ-ಗಳು
ಪುಸ್ತ-ಕದ
ಪುಸ್ತಕ-ದಲ್ಲಿ
ಪುಸ್ತಕ-ದಲ್ಲಿದ್ದ
ಪುಸ್ತಕ-ವನ್ನು
ಪುಸ್ತ-ಕವೂ
ಪೂ
ಪೂಜ-ಕನ
ಪೂಜನೀ-ಯವೂ
ಪೂಜಾ
ಪೂಜಾ-ಗೃಹದ
ಪೂಜಾ-ಗೃಹ-ದಲ್ಲಿ
ಪೂಜಾ-ಗೃಹ-ವನ್ನು
ಪೂಜಾ-ಗೃ-ಹವು
ಪೂಜಾ-ಮಂದಿ-ರಕ್ಕೆ
ಪೂಜಾ-ಮಂದಿ-ರ-ದಿಂದ
ಪೂಜಾರಿ
ಪೂಜಾ-ರಿಯು
ಪೂಜಾರ್ಹವಲ್ಲ-ವೆಂದು
ಪೂಜಾ-ವಿಧಾ-ನವು
ಪೂಜಾಶ್ಚಲೆ
ಪೂಜಾ-ಸನ-ದಲ್ಲಿ
ಪೂಜಾಸ್ಥ-ಳಕ್ಕೆ
ಪೂಜಿಸ-ಬಲ್ಲೆವು
ಪೂಜಿಸ-ಬ-ಹುದು
ಪೂಜಿಸ-ಬಾ-ರದು
ಪೂಜಿಸ-ಬೇಕು
ಪೂಜಿಸ-ಲಾರೆ-ವಾದರೆ
ಪೂಜಿ-ಸಲು
ಪೂಜಿಸಿ
ಪೂಜಿಸಿದ
ಪೂಜಿಸಿ-ದನು
ಪೂಜಿಸಿ-ಮನುಷ್ಯರು
ಪೂಜಿ-ಸುತ್ತಾರೆ
ಪೂಜಿ-ಸುತ್ತಿ-ರುವ-ವ-ರೆಲ್ಲ
ಪೂಜಿ-ಸುತ್ತಿ-ರುವೆ
ಪೂಜಿ-ಸುವ
ಪೂಜಿ-ಸುವಂತಾ-ದಾಗ
ಪೂಜಿ-ಸುವರು
ಪೂಜಿ-ಸುವ-ರೆಂದು
ಪೂಜಿ-ಸುವಾಗ
ಪೂಜಿ-ಸು-ವು-ದಿಲ್ಲ
ಪೂಜಿ-ಸು-ವು-ದಿಲ್ಲವೋ
ಪೂಜಿ-ಸು-ವುದು
ಪೂಜಿ-ಸುವೆ
ಪೂಜೆ
ಪೂಜೆ-ಗಳನ್ನು
ಪೂಜೆ-ಗಳಲ್ಲೆಲ್ಲ
ಪೂಜೆ-ಗ-ಳಾದ
ಪೂಜೆ-ಗಳೆಲ್ಲ
ಪೂಜೆ-ಗಳೊ
ಪೂಜೆ-ಗಾಗಿ
ಪೂಜೆಗೆ
ಪೂಜೆ-ಗೋಸ್ಕರ
ಪೂಜೆಯ
ಪೂಜೆ-ಯನ್ನು
ಪೂಜೆ-ಯನ್ನೂ
ಪೂಜೆ-ಯಾದ
ಪೂಜೆ-ಯಿಂದ
ಪೂಜೆಯು
ಪೂಜೆಯೂ
ಪೂಜೆ-ಯೆಂದರೆ
ಪೂಜೆಯೇ
ಪೂಜೆ-ಯೇನೋ
ಪೂಜೋಪ-ಚಾರ-ಗಳು
ಪೂಜೋಪ-ಚಾರ-ಗಳೂ
ಪೂಜ್ಯ
ಪೂಜ್ಯ-ದೃಷ್ಟಿ-ಯಿಂದ
ಪೂಜ್ಯ-ಭಾವ-ದಿಂದ
ಪೂಜ್ಯ-ಭಾವನೆ
ಪೂಜ್ಯ-ಭಾವ-ನೆ-ಯಿದೆ
ಪೂಜ್ಯ-ರೊಡನೆ
ಪೂತಾತ್ಮ-ರಾಗಿ
ಪೂನಾಕ್ಕೆ
ಪೂರಕ-ವಾದರೆ
ಪೂರಿ
ಪೂರಿ-ಗಿಂತಲೂ
ಪೂರೈಸಿ
ಪೂರೈಸಿ-ಕೊಳ್ಳು-ವುದಕ್ಕೆ
ಪೂರೈ-ಸಿತ್ತು
ಪೂರೈ-ಸಿದ
ಪೂರ್ಣ
ಪೂರ್ಣ-ಕಾಮ-ರಾಗಿ
ಪೂರ್ಣ-ಕುಂಭ
ಪೂರ್ಣಗ್ರಹಣ
ಪೂರ್ಣಜ್ಞಾನ
ಪೂರ್ಣಜ್ಞಾನಿ-ಯಾಗು-ವ-ವ-ರೆಗೂ
ಪೂರ್ಣ-ತೆ-ಯನ್ನು
ಪೂರ್ಣ-ತೆ-ಯುಂಟಾಗುವುದ-ರಲ್ಲಿ
ಪೂರ್ಣ-ತೆಯೂ
ಪೂರ್ಣತ್ವ
ಪೂರ್ಣತ್ವದ
ಪೂರ್ಣತ್ವ-ದಲ್ಲಿ
ಪೂರ್ಣತ್ವ-ದೊಂದಿಗೆ
ಪೂರ್ಣನೋ
ಪೂರ್ಣಬ್ರಹ್ಮ
ಪೂರ್ಣ-ಭಾವ-ದಲ್ಲಿ
ಪೂರ್ಣ-ಮಾಡುತ್ತಿದ್ದರು
ಪೂರ್ಣ-ರೂಪ-ದಲ್ಲಿ
ಪೂರ್ಣ-ವಾಗಿ
ಪೂರ್ಣ-ವಾಗಿದೆ
ಪೂರ್ಣ-ವಾಗಿಲ್ಲ
ಪೂರ್ಣ-ವಾದ
ಪೂರ್ಣ-ವಾಯ್ತು
ಪೂರ್ಣ-ವಿಕಾಸ
ಪೂರ್ಣ-ವಿಕಾಸ-ವಾಗುತ್ತದೆ
ಪೂರ್ಣಾತ್ಮ-ನನ್ನಾಗಿ
ಪೂರ್ಣಾತ್ಮ-ರನ್ನಾಗಿ
ಪೂರ್ಣಾಯುಷ್ಯ
ಪೂರ್ಣಿತೇವೋರ್ಮಿ-ಮಾಲಾ
ಪೂರ್ಣೇ
ಪೂರ್ತ
ಪೂರ್ತಿ
ಪೂರ್ತಿ-ಗೊಳಿ-ಸಲು
ಪೂರ್ತಿ-ಯಾಗಿ
ಪೂರ್ತಿ-ಯಾಗು-ವ-ವ-ರೆಗೂ
ಪೂರ್ವ
ಪೂರ್ವಕ
ಪೂರ್ವ-ಕಾಲದ
ಪೂರ್ವ-ಕಾಲ-ದಲ್ಲಿ
ಪೂರ್ವ-ಜನ್ಮ
ಪೂರ್ವ-ಜನ್ಮದ
ಪೂರ್ವ-ಜನ್ಮ-ದಲ್ಲಿ
ಪೂರ್ವ-ಜರಿಗಾಗಲೀ
ಪೂರ್ವ-ಜರು
ಪೂರ್ವದ
ಪೂರ್ವ-ದಲ್ಲಿದ್ದ
ಪೂರ್ವ-ದಿಗಂತ
ಪೂರ್ವ-ನಿಶ್ಚಿತ-ವಾಗಿದೆ
ಪೂರ್ವ-ಪಕ್ಷ
ಪೂರ್ವ-ಬಂಗಾಳ
ಪೂರ್ವ-ಬಂಗಾಳದ
ಪೂರ್ವ-ಬಂಗಾಳಿ-ಗಳು
ಪೂರ್ವ-ಮ-ಕಲ್ಪ-ಯತ್
ಪೂರ್ವ-ಮೀ-ಮಾಂಸಾ
ಪೂರ್ವ-ಮೀ-ಮಾಂಸೆ-ಯಲ್ಲಿ-ರುವ
ಪೂರ್ವ-ಮುಖ-ರಾಗಿ
ಪೂರ್ವ-ಮುಖ-ವಾಗಿ
ಪೂರ್ವ-ಸಂಪ್ರದಾಯ-ದ-ವರು
ಪೂರ್ವ-ಸಂಸ್ಕಾರ
ಪೂರ್ವ-ಸಂಸ್ಕಾರ-ವೆಂಬ
ಪೂರ್ವ-ಸಿದ್ಧ-ತೆ-ಯಿಲ್ಲದೆ
ಪೂರ್ವಾಕೃತಿ
ಪೂರ್ವಾ-ಚಾರ
ಪೂರ್ವಾ-ಚಾರ-ಪರಾಯಣ-ತೆ-ಯಂತೂ
ಪೂರ್ವಾ-ಚಾರಪ್ರಿ-ಯರು
ಪೂರ್ವಾಭಿ-ಮುಖ-ವಾಗಿ
ಪೂರ್ವಾರ್ಜಿತ
ಪೂರ್ವಿಕ
ಪೂರ್ವಿ-ಕರ
ಪೂರ್ವಿಕ-ರದ್ದೇ
ಪೂರ್ವಿ-ಕರು
ಪೂರ್ವಿಕ-ರೇನು
ಪೂರ್ವೋಕ್ತ-ವಾದ
ಪೂಲ್
ಪೃಥಕ್ಕಾಗಿಯೇ
ಪೃಥಿವೀಂ
ಪೃಥ್ವಿ
ಪೃಥ್ವಿಛ್ಛೇದಿ
ಪೃಥ್ವಿ-ತಲ
ಪೃಥ್ವಿ-ಮನುಷ್ಯ
ಪೃಥ್ವಿ-ಯಲ್ಲಿ
ಪೃಥ್ವಿ-ಯಲ್ಲಿಯೂ
ಪೃಥ್ವಿಯೂ
ಪೃಥ್ವೀ
ಪೆಂಡುಲಂ
ಪೆಚ್ಚಾ-ದಾಗ
ಪೆಟ್ಟನ್ನು
ಪೆಟ್ಟಿಗೆ
ಪೆಟ್ಟಿ-ಗೆಯ
ಪೆಟ್ಟಿಗೆ-ಯನ್ನು
ಪೆಟ್ಟಿಗೆ-ಯಲ್ಲಿ
ಪೆಟ್ಟಿಗೆ-ಯಲ್ಲಿಟ್ಟು
ಪೆಟ್ಟಿಗೆ-ಯಿಂದ
ಪೆಟ್ಟಿ-ನಿಂದ
ಪೆಟ್ಟು
ಪೆಟ್ಟು-ಗಳ
ಪೇಟೆ-ಯಿಂದ
ಪೇಳುವೆ-ನಿದ
ಪೇಳ್ದ
ಪೈಪೋಟಿಯ
ಪೈಪೋಟಿಯಂತ
ಪೈರನ್ನು
ಪೈಲ್ವಾನ
ಪೈಶಾಚಿಕ
ಪೊಟ್ಟಣ-ದಲ್ಲಿ
ಪೊರಕೆ
ಪೊರೆಯಿವು
ಪೊರ್ಸಿಲೇ-ನಿನ
ಪೊಲೀ-ಸರೂ
ಪೊಳ್ಳು
ಪೋಪ್
ಪೋಷಕ
ಪೋಷ-ಕರ
ಪೋಷಕ-ರಾಗಿದ್ದಾರೆ
ಪೋಷ-ಕರು
ಪೋಷಾಕೆಲ್ಲಾ
ಪೋಷಿ-ತರು
ಪೋಷಿ-ಸಿದ್ದಾರೊ
ಪೋಷಿ-ಸುತ್ತಿದ್ದು-ದರ
ಪೌರಾಣಿಕ
ಪೌರಾಣಿ-ಕತೆ
ಪೌರುಷ
ಪೌರುಷ-ದಲ್ಲಿ
ಪೌರುಷದಿ
ಪೌರುಷ-ವಂತ-ರಾದ
ಪೌರುಷ-ವುಳ್ಳ
ಪೌರೋ-ಹಿತ್ಯ
ಪೌರೋ-ಹಿತ್ಯ-ವಂತೂ
ಪೌರ್ವಾತ್ಯ
ಪ್ಯಾರಿ-ಸಿನಲ್ಲಿದ್ದಾಗ
ಪ್ಯಾರಿ-ಸಿ-ನಿಂದ
ಪ್ಯಾರಿಸ್ಸಿನ
ಪ್ಯಾಸಡೇನಾ
ಪ್ಯಾಸು
ಪ್ರಕಟ-ವಾಗುತ್ತಿದ್ದ
ಪ್ರಕಟ-ವಾದ
ಪ್ರಕಟ-ವಾದಾಗ
ಪ್ರಕಟ-ವಾ-ಯಿತು
ಪ್ರಕಟಿತ-ಪರಪ್ರೇಮ್ಣಾ
ಪ್ರಕಟಿ-ಸಲು
ಪ್ರಕಟಿ-ಸುತ್ತಾನೆ
ಪ್ರಕರಣೇ
ಪ್ರಕಾರ
ಪ್ರಕಾಶ
ಪ್ರಕಾಶಕ್ಕೆ
ಪ್ರಕಾಶ-ಗೊಳಿಸು-ವಂತೆ
ಪ್ರಕಾ-ಶದ
ಪ್ರಕಾಶ-ದಿಂದ
ಪ್ರಕಾಶ-ನಾಗಿ
ಪ್ರಕಾಶ-ನಾಗುತ್ತಾನೆ
ಪ್ರಕಾಶ-ಪಡಿ-ಸಿ-ದರು
ಪ್ರಕಾಶ-ವಾದರೆ
ಪ್ರಕಾಶ-ವಾದಾಗಂತೂ
ಪ್ರಕಾಶ-ವಾ-ಯಿತು
ಪ್ರಕಾಶಾ-ನಂದ
ಪ್ರಕಾಶಿ-ಕೆ-ಯಲ್ಲಿ
ಪ್ರಕಾಶಿತ-ವಾಗಿದೆ
ಪ್ರಕಾಶಿತ-ವಾಗುತ್ತದೆ
ಪ್ರಕಾಶಿತ-ವಾಗು-ವುದು
ಪ್ರಕಾಶಿ-ಸು-ವುದು
ಪ್ರಕೃತ
ಪ್ರಕೃತಿ
ಪ್ರಕೃತಿ-ಗಾಗಿ
ಪ್ರಕೃ-ತಿಗೆ
ಪ್ರಕೃತಿ-ಜೀವರ
ಪ್ರಕೃ-ತಿಯ
ಪ್ರಕೃತಿ-ಯ-ದೆಲ್ಲವು
ಪ್ರಕೃತಿ-ಯನ್ನರಿ-ಯುವೆ
ಪ್ರಕೃತಿ-ಯನ್ನು
ಪ್ರಕೃತಿ-ಯಲ್ಲಿ
ಪ್ರಕೃತಿ-ಯಲ್ಲೂ
ಪ್ರಕೃತಿ-ಯ-ವರನ್ನೆಲ್ಲಾ
ಪ್ರಕೃತಿ-ಯೆಲ್ಲವು
ಪ್ರಕೃ-ತಿಯೇ
ಪ್ರಕೃತಿ-ಯೊ-ಡನೆ
ಪ್ರಕೃತಿ-ಸಹಜಾಮನ್ಧತಾ-ಮಿಸ್ರಮಿಶ್ರಾಮ್
ಪ್ರಕೃತಿ-ಸೌಂದರ್ಯ-ದಲಿ
ಪ್ರಕ್ಷಾಲನ
ಪ್ರಕ್ಷಿಪ್ತ
ಪ್ರಕ್ಷುಬ್ಧ
ಪ್ರಕ್ಷೇಪ
ಪ್ರಕ್ಷೇಪ-ವಾದ
ಪ್ರಕ್ಷೇಪ-ವಾ-ಯಿತು
ಪ್ರಖ್ಯಾತ
ಪ್ರಖ್ಯಾತ-ರಾಗ-ಬ-ಹುದು
ಪ್ರಖ್ಯಾತಿ
ಪ್ರಗತಿ
ಪ್ರಗ-ತಿಗೂ
ಪ್ರಗ-ತಿಗೆ
ಪ್ರಗತಿ-ಪರ-ರಾದರು
ಪ್ರಗ-ತಿಯ
ಪ್ರಗತಿ-ಯುಂಟಾಗುವ
ಪ್ರಗತಿ-ಯೆಲ್ಲ
ಪ್ರಗಲ್ಭತಾ
ಪ್ರಚಂಡ
ಪ್ರಚಂಡ-ಪುರುಷರೊ
ಪ್ರಚಂಡ-ವಾಗಿ
ಪ್ರಚಂಡ-ವಾದ
ಪ್ರಚಂಡ-ಶಕ್ತಿ
ಪ್ರಚಂಡ-ಶಕ್ತಿ-ನಿ-ಹಿತ-ವಾದ
ಪ್ರಚಲತಿ
ಪ್ರಚಲಿತ-ವಾಗಿದ್ದುವು
ಪ್ರಚಾರ
ಪ್ರಚಾರಕ
ಪ್ರಚಾರ-ಕನ
ಪ್ರಚಾರ-ಕನು
ಪ್ರಚಾರ-ಕ-ನೊಬ್ಬನು
ಪ್ರಚಾರ-ಕ-ರಷ್ಟು
ಪ್ರಚಾರ-ಕರು
ಪ್ರಚಾರ-ಕಾರ್ಯ
ಪ್ರಚಾರ-ಕಾರ್ಯಕ್ಕೋಸ್ಕರ
ಪ್ರಚಾರಕ್ಕಾಗಿ
ಪ್ರಚಾ-ರಕ್ಕೆ
ಪ್ರಚಾರ-ಗೊಳಿಸು-ವು-ದರ
ಪ್ರಚಾರ-ದಲ್ಲಿ
ಪ್ರಚಾರ-ದಲ್ಲಿದೆ
ಪ್ರಚಾರ-ದಲ್ಲಿ-ರುವ
ಪ್ರಚಾರ-ದಿಂದ
ಪ್ರಚಾರ-ದಿಂದಲೂ
ಪ್ರಚಾರ-ಪಡಿ-ಸಿತೋ
ಪ್ರಚಾರ-ಮಾಡ-ಬೇಕು
ಪ್ರಚಾರ-ಮಾಡ-ಲಿಲ್ಲ
ಪ್ರಚಾರ-ಮಾ-ಡಲು
ಪ್ರಚಾರ-ಮಾಡಿ
ಪ್ರಚಾರ-ಮಾಡುತ್ತಲೂ
ಪ್ರಚಾರವೂ
ಪ್ರಚಾರ-ವೆಂದರೆ
ಪ್ರಚಾರಾದಿ-ಗಳನ್ನು
ಪ್ರಚಾರಿತ-ವಾದ
ಪ್ರಚೋದಿ-ತ-ವಾಗಿ
ಪ್ರಚೋದಿ-ಸುತ್ತಿ-ರುವಳೋ
ಪ್ರಚೋದಿ-ಸುವ
ಪ್ರಚೋದಿ-ಸು-ವುದು
ಪ್ರಚೋದಿ-ಸು-ವುದೋ
ಪ್ರಚ್ಛನ್ನ
ಪ್ರಜಯಾ
ಪ್ರಜಾಭಿಪ್ರಾಯ-ವನ್ನು
ಪ್ರಜೆ-ಗಳ
ಪ್ರಜ್ಞಾ-ಪೂರ್ವ-ಕ-ವಾಗಿ
ಪ್ರಜ್ಞೆ
ಪ್ರಜ್ಞೆಗೆ
ಪ್ರಜ್ಞೆಯ
ಪ್ರಜ್ಞೆ-ಯೆಂದರೆ
ಪ್ರಜ್ಞೆಯೇ
ಪ್ರಜ್ವಲ-ಮಾಡಿ
ಪ್ರಜ್ವಲಿ-ಸಲು
ಪ್ರಜ್ವಲಿ-ಸಿ-ದರು
ಪ್ರಜ್ವಲಿ-ಸುತ್ತಾ
ಪ್ರಜ್ವಲಿ-ಸುತ್ತಿದೆ
ಪ್ರಜ್ವಲಿ-ಸುತ್ತಿದ್ದ
ಪ್ರಜ್ವಲಿ-ಸುತ್ತಿದ್ದುವು
ಪ್ರಜ್ವಲಿ-ಸುತ್ತಿ-ರುವರು
ಪ್ರಜ್ವಲಿ-ಸುತ್ತಿ-ರು-ವು-ದನ್ನು
ಪ್ರಜ್ವಲಿ-ಸುತ್ತೇವೆ
ಪ್ರಜ್ವಲಿ-ಸುವ
ಪ್ರಜ್ವಲಿ-ಸು-ವುದು
ಪ್ರಜ್ವಾಸ್
ಪ್ರಣ-ತ-ನಾದ
ಪ್ರಣ-ತಮ-ವತು
ಪ್ರಣ-ಯತಿ
ಪ್ರಣಯಿ
ಪ್ರಣವ
ಪ್ರಣ-ವಕ್ಕೆ
ಪ್ರಣಾಮ
ಪ್ರಣಾಮ-ಮಾಡಿ
ಪ್ರಣಾಮ-ಮಾಡಿದ
ಪ್ರಣಾಮ-ಮಾಡಿ-ದರು
ಪ್ರಣಾಮ-ಮಾಡುವ
ಪ್ರತಿ
ಪ್ರತಿ-ಕೂಲ
ಪ್ರತಿ-ಕೃತಿ-ಯನ್ನು
ಪ್ರತಿಕ್ರಿ-ಯೆ-ಗಳಿಂದ
ಪ್ರತಿಕ್ರಿ-ಯೆಗೆ
ಪ್ರತಿಕ್ರಿ-ಯೆಯ
ಪ್ರತಿಕ್ರಿ-ಯೆ-ಯನ್ನು
ಪ್ರತಿಕ್ರಿ-ಯೆ-ಯನ್ನೂ
ಪ್ರತಿಕ್ರಿ-ಯೆ-ಯಲ್ಲಿದೆ
ಪ್ರತಿಕ್ರಿ-ಯೆ-ಯುಂಟಾಗಿ
ಪ್ರತಿಕ್ರಿ-ಯೆಯೇ
ಪ್ರತಿಕ್ರಿ-ಯೆ-ಯೊಂದು
ಪ್ರತಿಕ್ಷಣ-ದಲ್ಲಿಯೂ
ಪ್ರತಿ-ಗಳನ್ನು
ಪ್ರತಿ-ಘಾತ
ಪ್ರತಿ-ಘಾತ-ದಿಂದೆದ್ದು
ಪ್ರತಿಜ್ಞತೆ
ಪ್ರತಿಜ್ಞೆ-ಮಾಡುವೆ
ಪ್ರತಿಜ್ಞೆ-ಯನ್ನು
ಪ್ರತಿ-ದಿನ
ಪ್ರತಿ-ದಿ-ನವೂ
ಪ್ರತಿಧ್ವನಿ
ಪ್ರತಿಧ್ವನಿ-ಗೊಳಿಸಿ-ದರು
ಪ್ರತಿಧ್ವನಿ-ತ-ವಾಗುತ್ತಿತ್ತು
ಪ್ರತಿಧ್ವನಿ-ತ-ವಾಗುತ್ತಿವೆ
ಪ್ರತಿಧ್ವನಿ-ತ-ವಾ-ಗು-ವಂತೆ
ಪ್ರತಿ-ನಗರ-ದಲ್ಲಿಯೂ
ಪ್ರತಿ-ನಮಸ್ಕಾರ
ಪ್ರತಿ-ನಿತ್ಯವೂ
ಪ್ರತಿ-ನಿಧಿ
ಪ್ರತಿ-ನಿಧಿ-ಗಳು
ಪ್ರತಿ-ನಿಧಿ-ಯಂತೆ
ಪ್ರತಿ-ನಿಧಿ-ಯಾಗಿ-ರುವರು
ಪ್ರತಿ-ನಿಧಿ-ಸಭೆ-ಯಲ್ಲಿ
ಪ್ರತಿ-ಪಕ್ಷ-ದ-ವರ
ಪ್ರತಿ-ಪಾದ-ಕ-ನಾಗಿದ್ದ
ಪ್ರತಿ-ಪಾದನ
ಪ್ರತಿ-ಪಾದ-ನೆ-ಯನ್ನು
ಪ್ರತಿ-ಪಾದ-ನೆ-ಯೆಂದರೆ
ಪ್ರತಿ-ಪಾದಿ-ಸಿ-ದಲ್ಲಿ
ಪ್ರತಿ-ಪಾದಿ-ಸುತ್ತಿದ್ದನು
ಪ್ರತಿ-ಪಾದಿ-ಸುವ
ಪ್ರತಿ-ಫಲ
ಪ್ರತಿ-ಫಲದ
ಪ್ರತಿ-ಫಲವ
ಪ್ರತಿ-ಫಲ-ವಾಗಿ
ಪ್ರತಿ-ಫಲಾಪೇಕ್ಷೆ
ಪ್ರತಿ-ಫಲಾಪೇಕ್ಷೆಗೆ
ಪ್ರತಿ-ಫಲಾಪೇಕ್ಷೆ-ಯಿಲ್ಲದೆ
ಪ್ರತಿ-ಫಲಿತ-ವಾಗುತ್ತದೆ
ಪ್ರತಿ-ಫಲಿಸು-ವ-ರೆಗೂ
ಪ್ರತಿ-ಬಂಧ-ಕ-ಗಳನ್ನು
ಪ್ರತಿ-ಬಂಧ-ಕ-ಗಳು
ಪ್ರತಿ-ಬಂಧ-ಕ-ರೂಪ-ವಾದ
ಪ್ರತಿ-ಬಂಧ-ಕ-ವಾಗಿ
ಪ್ರತಿ-ಬಂಧ-ಕ-ವಾಗಿದೆ
ಪ್ರತಿ-ಬಂಧ-ಕ-ವಾಗಿ-ರುವ
ಪ್ರತಿ-ಬಂಧ-ಕ-ವಾದ
ಪ್ರತಿ-ಬಂಧ-ಕ-ವೆಂಬು-ದನ್ನು
ಪ್ರತಿ-ಬಂಧ-ನೆ-ಯಿಂದಲೇ
ಪ್ರತಿ-ಬಂಧಿ-ಸುವ
ಪ್ರತಿ-ಬಿಂಬ
ಪ್ರತಿ-ಬಿಂಬ-ಗಳು
ಪ್ರತಿ-ಬಿಂಬದ
ಪ್ರತಿ-ಬಿಂಬ-ವಾಗಿ-ರ-ಬೇಕು
ಪ್ರತಿ-ಬಿಂಬ-ವಾಗಿ-ರು-ವಂತೆ
ಪ್ರತಿ-ಭಟನೆ
ಪ್ರತಿ-ಭಟಿ-ಸದೆ
ಪ್ರತಿ-ಭಟಿ-ಸಿ-ದರು
ಪ್ರತಿ-ಭಟಿ-ಸಿ-ದಳು
ಪ್ರತಿ-ಭಟಿ-ಸುತ್ತ-ದೆಯೋ
ಪ್ರತಿ-ಭಾ-ಬಲ-ದಿಂದ
ಪ್ರತಿ-ಭಾವಂತರ
ಪ್ರತಿ-ಭಾಶಾಲಿ
ಪ್ರತಿಭೆ
ಪ್ರತಿ-ಭೆಯ
ಪ್ರತಿ-ಭೆ-ಯನ್ನು
ಪ್ರತಿ-ಭೆ-ಯಿಂದ
ಪ್ರತಿಮೆ
ಪ್ರತಿ-ಮೆ-ಯಂತೆ
ಪ್ರತಿ-ಮೆ-ಯನ್ನು
ಪ್ರತಿ-ಯನ್ನು
ಪ್ರತಿ-ಯಾಗಿ
ಪ್ರತಿ-ಯೊಂದಕ್ಕೂ
ಪ್ರತಿ-ಯೊಂದನ್ನು
ಪ್ರತಿ-ಯೊಂದನ್ನೂ
ಪ್ರತಿ-ಯೊಂದರ
ಪ್ರತಿ-ಯೊಂದ-ರಲ್ಲಿ
ಪ್ರತಿ-ಯೊಂದರಲ್ಲಿಯೂ
ಪ್ರತಿ-ಯೊಂದ-ರಲ್ಲೂ
ಪ್ರತಿ-ಯೊಂದು
ಪ್ರತಿ-ಯೊಂದೂ
ಪ್ರತಿ-ಯೊಬ್ಬ
ಪ್ರತಿ-ಯೊಬ್ಬ-ನಲ್ಲಿಯೂ
ಪ್ರತಿ-ಯೊಬ್ಬನು
ಪ್ರತಿ-ಯೊಬ್ಬನೂ
ಪ್ರತಿ-ಯೊಬ್ಬರ
ಪ್ರತಿ-ಯೊಬ್ಬ-ರನ್ನೂ
ಪ್ರತಿ-ಯೊಬ್ಬ-ರಲ್ಲಿಯೂ
ಪ್ರತಿ-ಯೊಬ್ಬ-ರಲ್ಲೂ
ಪ್ರತಿ-ಯೊಬ್ಬ-ರಿ-ಗಾಗಿ
ಪ್ರತಿ-ಯೊಬ್ಬ-ರಿಗೂ
ಪ್ರತಿ-ಯೊಬ್ಬರೂ
ಪ್ರತಿ-ರೋಧ-ವಿಲ್ಲ-ದಿ-ರುವಿಕೆ
ಪ್ರತಿ-ವಾದ
ಪ್ರತಿ-ವಾದಿ-ಸಿಯೂ
ಪ್ರತಿಷ್ಟಿತ-ನಾಗುತ್ತಾನೆ
ಪ್ರತಿಷ್ಠಂ
ಪ್ರತಿಷ್ಠ-ನಾಗುವೆ
ಪ್ರತಿಷ್ಠಾ
ಪ್ರತಿಷ್ಠಾ-ಪ-ನೆಯಾಗ-ಬೇಕೆಂದು
ಪ್ರತಿಷ್ಠಾ-ಪಿತ-ನಾದ
ಪ್ರತಿಷ್ಠಾ-ಪಿತ-ವಾಗಿತ್ತು
ಪ್ರತಿಷ್ಠಿತ-ನಾಗುವೆ
ಪ್ರತಿಷ್ಠಿತ-ರಾಗಿ
ಪ್ರತಿಷ್ಠಿತ-ವಾಗಿ-ದೆಯೋ
ಪ್ರತಿಷ್ಠಿತ-ವಾಗು-ವು-ದಾಗಿದೆ
ಪ್ರತಿಷ್ಠೆ
ಪ್ರತಿಷ್ಠೆ-ಯನ್ನು
ಪ್ರತಿಷ್ಠೆ-ಯಾ-ಯಿತು
ಪ್ರತೀಕ
ಪ್ರತೀ-ಕಾರ
ಪ್ರತೀ-ಕಾರ-ಗೊಳಿ-ಸಲು
ಪ್ರತೀತಂ
ಪ್ರತ್ಯಕ್ಷ
ಪ್ರತ್ಯಕ್ಷ-ನಾಗುತ್ತಾನೆ
ಪ್ರತ್ಯಕ್ಷಪ್ರಮಾಣ
ಪ್ರತ್ಯಕ್ಷ-ರಾಗುವರು
ಪ್ರತ್ಯಕ್ಷ-ಳಾಗು-ವಳು
ಪ್ರತ್ಯಕ್ಷ-ವಾಗದೆ
ಪ್ರತ್ಯಕ್ಷ-ವಾಗಿ
ಪ್ರತ್ಯಕ್ಷ-ವಾಗಿದೆ
ಪ್ರತ್ಯಕ್ಷ-ವಾಗಿದ್ದಾರೆ
ಪ್ರತ್ಯಕ್ಷ-ವಾಗಿ-ಬಿಡುತ್ತಿತ್ತು
ಪ್ರತ್ಯಕ್ಷ-ವಾಗಿಲ್ಲ-ವಲ್ಲಾ
ಪ್ರತ್ಯಕ್ಷ-ವಾಗುತ್ತದೆ
ಪ್ರತ್ಯಕ್ಷ-ವಾಗುತ್ತವೆ
ಪ್ರತ್ಯಕ್ಷ-ವಾ-ಗು-ವು-ದಿಲ್ಲ
ಪ್ರತ್ಯಕ್ಷ-ವಾಗು-ವುದು
ಪ್ರತ್ಯಕ್ಷ-ವಾಗು-ವುದೋ
ಪ್ರತ್ಯಕ್ಷ-ವಾ-ದದ್ದು
ಪ್ರತ್ಯಕ್ಷ-ವಾದರೆ
ಪ್ರತ್ಯಕ್ಷಾ-ನು-ಭವ
ಪ್ರತ್ಯಕ್ಷಾ-ನು-ಭವಕ್ಕೋಸ್ಕರ
ಪ್ರತ್ಯಕ್ಷಾನು-ಭೂತಿ
ಪ್ರತ್ಯಗಾತ್ಮ-ನೊಬ್ಬನೇ
ಪ್ರತ್ಯಾಶೆಯುಂಟಾ-ಗುತ್ತಿದೆ
ಪ್ರತ್ಯಾಶೆಯುಂಟೆ
ಪ್ರತ್ಯೇಕ
ಪ್ರತ್ಯೇಕ-ವಾಗಿ
ಪ್ರತ್ಯೇಕ-ವಾದ
ಪ್ರತ್ಯೇಕಿ-ಸುವ
ಪ್ರತ್ಯೇಕೀ-ಕರ-ಣವೇ
ಪ್ರಥಮ
ಪ್ರಥಮ-ಬಾರಿ
ಪ್ರಥಮಾವ-ತಾರ-ದಿಂದ
ಪ್ರಥಿತ-ಪುರುಷೋ
ಪ್ರದರ್ಶನ-ವಿದ್ದರೂ
ಪ್ರದರ್ಶನ-ವಿಲ್ಲ
ಪ್ರದರ್ಶಿತ-ವಾಗಿ-ದೆಯೋ
ಪ್ರದರ್ಶಿ-ಸಿ-ಕೊಳ್ಳಲು
ಪ್ರದಾನ-ಗಳಿಂದ
ಪ್ರದಾನ-ಗಳಿಲ್ಲ-ದಿದ್ದರೆ
ಪ್ರದಿ-ದೂರ
ಪ್ರದೀಪ-ದಂತಿ-ರುವರು
ಪ್ರದೀಪ್ತ
ಪ್ರದೀಪ್ತ-ನಾಗಿದ್ದ-ವ-ನೆಂದು
ಪ್ರದೀಪ್ತ-ರಾದ
ಪ್ರದೀಪ್ತ-ವಾಗಿ-ಬಿಟ್ಟಿತು
ಪ್ರದೇಶ
ಪ್ರದೇಶಕ್ಕೆ
ಪ್ರದೇಶ-ಗಳಲ್ಲಿ
ಪ್ರದೇಶ-ದಲ್ಲಿ
ಪ್ರದೇಶ-ದಲ್ಲೇ
ಪ್ರದೇಶ-ದಲ್ಲೇಕೆ
ಪ್ರದೇಶ-ದಿಂದ
ಪ್ರದೇಶ-ವನ್ನು
ಪ್ರಧಾನ
ಪ್ರಧಾನ-ಭಾವ
ಪ್ರಧಾನ-ವಾಗಿ
ಪ್ರಧಾನ-ವಾಗಿದೆ
ಪ್ರಧಾನ-ವಾಗಿ-ರುವ
ಪ್ರಧಾನ-ವಾಗಿ-ರು-ವುದು
ಪ್ರಧಾನ-ವಾಗುಳ್ಳ
ಪ್ರಧಾನ-ವಾದ
ಪ್ರಧಾನ-ವಾ-ದುದು
ಪ್ರಪಂಚ
ಪ್ರಪಂಚಕ್ಕೆ
ಪ್ರಪಂಚಕ್ಕೊಯ್ಯು-ವುದು
ಪ್ರಪಂಚ-ಗಳೆಲ್ಲ
ಪ್ರಪಂಚದ
ಪ್ರಪಂಚ-ದಲ್ಲಿ
ಪ್ರಪಂಚ-ದಲ್ಲಿಯೂ
ಪ್ರಪಂಚ-ದಲ್ಲಿ-ರುವ
ಪ್ರಪಂಚ-ದಲ್ಲಿ-ರುವ-ವರು
ಪ್ರಪಂಚ-ದಲ್ಲಿ-ರುವ-ವ-ರೆಲ್ಲ
ಪ್ರಪಂಚ-ದಲ್ಲೂ
ಪ್ರಪಂಚ-ದಲ್ಲೆಲ್ಲಾ
ಪ್ರಪಂಚ-ದಲ್ಲೇ
ಪ್ರಪಂಚ-ದಲ್ಲೇಕೆ
ಪ್ರಪಂಚ-ವನ್ನು
ಪ್ರಪಂಚ-ವನ್ನೆಲ್ಲಾ
ಪ್ರಪಂಚವು
ಪ್ರಪಂಚ-ವೆಲ್ಲ-ವನ್ನೂ
ಪ್ರಪಂಚ-ವೆಲ್ಲಾ
ಪ್ರಪಂಚವೇ
ಪ್ರಪದ್ಯೇ
ಪ್ರಪ್ರಥಮ
ಪ್ರಫುಲ್ಲ-ವಾದ
ಪ್ರಬಂಧ-ಗಳನ್ನು
ಪ್ರಬಂಧ-ವನ್ನು
ಪ್ರಬಲ
ಪ್ರಬಲ-ವಾಗಿ
ಪ್ರಬಲ-ವಾಗಿದೆ
ಪ್ರಬಲ-ವಾಗಿ-ರುತ್ತದೆಯೋ
ಪ್ರಬಲ-ವಾಗಿ-ರು-ವು-ದನ್ನು
ಪ್ರಬಲ-ವಾಗಿ-ರು-ವುದು
ಪ್ರಬಲ-ವಾಗು-ವುದು
ಪ್ರಬಲ-ವಾದ
ಪ್ರಬಲ-ವಾ-ದದ್ದು
ಪ್ರಬಲ-ವಿಲ್ಲಿ
ಪ್ರಬಲೋರ್ಮಿ-ಭಂಗೈಃ
ಪ್ರಬುದ್ಧ
ಪ್ರಬುದ್ಧ-ನಾಗಿ
ಪ್ರಬುದ್ಧನೆ
ಪ್ರಭಂಜನ್
ಪ್ರಭಾತ
ಪ್ರಭಾವ
ಪ್ರಭಾ-ವಕ್ಕೆ
ಪ್ರಭಾವ-ಗಳು
ಪ್ರಭಾವ-ಗೊಂಡಿದೆ
ಪ್ರಭಾವ-ದಿಂದ
ಪ್ರಭಾವ-ವನ್ನು
ಪ್ರಭಾವ-ವಿತ್ತು
ಪ್ರಭಾವ-ಶಾಲಿ-ಗಳಾದ-ವರು
ಪ್ರಭಾವಿತ-ರಾಗಿಯೇ
ಪ್ರಭು
ಪ್ರಭು-ಗಳದ್ದೋ
ಪ್ರಭು-ಗಳೇ
ಪ್ರಭುತ್ವವೂ
ಪ್ರಭು-ದೇವನ
ಪ್ರಭು-ವಿನ
ಪ್ರಭುವು
ಪ್ರಭುವೆ
ಪ್ರಭುವೇ
ಪ್ರಭೆ-ಯಲ್ಲಿ
ಪ್ರಭೆ-ಯಿಂದ
ಪ್ರಭೇದ-ಗಳಾಗಲೀ
ಪ್ರಮತ್ತ-ತೆ-ಯಲ್ಲಿಯೇ
ಪ್ರಮಥತಿ
ಪ್ರಮಾಣ
ಪ್ರಮಾಣ-ಗಳ
ಪ್ರಮಾಣ-ಗಳನ್ನು
ಪ್ರಮಾಣ-ಗ-ಳಲ್ಲ
ಪ್ರಮಾಣ-ಗಳು
ಪ್ರಮಾಣ-ದಲ್ಲಿ
ಪ್ರಮಾಣ-ದಲ್ಲಿ-ರುವ
ಪ್ರಮಾಣ-ವನ್ನೂ
ಪ್ರಮಾಣ-ವಾಗ-ಬಾ-ರದು
ಪ್ರಮಾಣ-ವಾಗಿ
ಪ್ರಮಾಣ-ವಾಗಿದೆ
ಪ್ರಮಾಣ-ವಾಗಿವೆ
ಪ್ರಮಾಣ-ವಾದರೆ
ಪ್ರಮಾಣವೂ
ಪ್ರಮಾಣ-ವೆಂದು
ಪ್ರಮಾಣವೇ
ಪ್ರಮಾಣ-ವೇಕೆ
ಪ್ರಮಾಣ-ವೇ-ನೆಂದರೆ
ಪ್ರಮಾಣ-ಸಮ್ಮತ-ವಲ್ಲ-ವೆಂದು
ಪ್ರಮಾಣ್ಯ-ವನ್ನು
ಪ್ರಮಾದ-ಗಳನ್ನು
ಪ್ರಮುಖ
ಪ್ರಮುಖ-ನಾದ-ವನು
ಪ್ರಮುಖ-ರಾಗಿ-ರುವ-ವ-ರೆಲ್ಲಾ
ಪ್ರಮುಖ-ವಾದ
ಪ್ರಮುಖ-ವಾ-ದದ್ದು
ಪ್ರಮುಖ-ವಾ-ದುದು
ಪ್ರಮೇ-ಯವೇ
ಪ್ರಮೋದ-ಗಳನ್ನು
ಪ್ರಯತಿ-ಸೋಣ
ಪ್ರಯತ್ನ
ಪ್ರಯತ್ನಕ್ಕೆ
ಪ್ರಯತ್ನ-ಗಳು
ಪ್ರಯತ್ನ-ಗಳೆಲ್ಲಾ
ಪ್ರಯತ್ನದ
ಪ್ರಯತ್ನ-ದಲ್ಲಿ
ಪ್ರಯತ್ನ-ದಿಂದ
ಪ್ರಯತ್ನ-ದಿಂದಲೂ
ಪ್ರಯತ್ನ-ಪಟ್ಟನು
ಪ್ರಯತ್ನ-ಪಟ್ಟರೂ
ಪ್ರಯತ್ನ-ಪಟ್ಟರೆ
ಪ್ರಯತ್ನ-ಪೂರ್ವ-ಕ-ವಾಗಿ
ಪ್ರಯತ್ನ-ವನ್ನು
ಪ್ರಯತ್ನ-ವಾಗಲಿ
ಪ್ರಯತ್ನ-ವಿದೆ
ಪ್ರಯತ್ನವೂ
ಪ್ರಯತ್ನ-ವೆಲ್ಲಿದೆ
ಪ್ರಯತ್ನಿ-ಸ-ದಿದ್ದಲ್ಲಿ
ಪ್ರಯತ್ನಿ-ಸದೇ
ಪ್ರಯತ್ನಿಸ-ಬಾ-ರದು
ಪ್ರಯತ್ನಿಸ-ಬೇಕು
ಪ್ರಯತ್ನಿಸಿ
ಪ್ರಯತ್ನಿಸಿ-ದರು
ಪ್ರಯತ್ನಿಸಿ-ದರೆ
ಪ್ರಯತ್ನಿಸಿ-ದಾಗ
ಪ್ರಯತ್ನಿಸಿದ್ದಾರೆ
ಪ್ರಯತ್ನಿಸು
ಪ್ರಯತ್ನಿಸುತ್ತಿದ್ದರು
ಪ್ರಯತ್ನಿಸುತ್ತಿದ್ದಾರೆ
ಪ್ರಯತ್ನಿಸುತ್ತಿದ್ದಾಳೆ
ಪ್ರಯತ್ನಿಸುತ್ತಿರು-ವೆಯಾ
ಪ್ರಯತ್ನಿಸುತ್ತೇನೆ
ಪ್ರಯತ್ನಿಸುತ್ತೇವೆ
ಪ್ರಯತ್ನಿಸು-ವನು
ಪ್ರಯತ್ನಿಸು-ವರು
ಪ್ರಯತ್ನಿಸು-ವಿರಿ
ಪ್ರಯತ್ನಿಸು-ವು-ದಿಲ್ಲ
ಪ್ರಯತ್ನಿಸು-ವುದು
ಪ್ರಯತ್ನಿಸು-ವೆವು
ಪ್ರಯಾಣ
ಪ್ರಯಾಣ-ದಲ್ಲಿ
ಪ್ರಯಾಣ-ಮಾಡಿ-ದಾಗ
ಪ್ರಯಾಣ-ವನ್ನು
ಪ್ರಯಾಣಿ-ಸುತ್ತಿದ್ದಾಗ
ಪ್ರಯಾಸ
ಪ್ರಯಾಸ-ವನ್ನು
ಪ್ರಯೋಗ
ಪ್ರಯೋಗ-ಗಳು
ಪ್ರಯೋಗ-ದಲ್ಲಿ
ಪ್ರಯೋಗಿ-ಸಲ್ಪಡುತ್ತದೆ
ಪ್ರಯೋಗಿ-ಸಿಯೇ
ಪ್ರಯೋಗಿ-ಸುವ
ಪ್ರಯೋಗಿ-ಸುವುದು
ಪ್ರಯೋಜಕ-ವಾಗಲಿ
ಪ್ರಯೋ-ಜನ
ಪ್ರಯೋ-ಜನ-ವಾಗ-ಲಾರದು
ಪ್ರಯೋ-ಜನ-ವಾಗ-ಲಿಲ್ಲ
ಪ್ರಯೋ-ಜನ-ವಾಗಿ-ದೆ-ಯೆಂದು
ಪ್ರಯೋ-ಜನ-ವಾ-ಗು-ವು-ದಿಲ್ಲ-ವೆಂದು
ಪ್ರಯೋ-ಜನ-ವಾಗು-ವುದು
ಪ್ರಯೋ-ಜನ-ವಿದೆ
ಪ್ರಯೋ-ಜನ-ವಿಲ್ಲ
ಪ್ರಯೋಜ-ನವೂ
ಪ್ರಯೋ-ಜನ-ವೇನು
ಪ್ರರೋಹಾಃ
ಪ್ರಲಯ-ಕಲಿತಂ
ಪ್ರಲಯ-ವಾಯು
ಪ್ರಲಯೇರ್-ಕಾಲೆ
ಪ್ರಲಯ್
ಪ್ರಲಾಪ
ಪ್ರಲಾಪಿ-ಸುತ್ತಾ
ಪ್ರಲಾಪಿ-ಸುತ್ತಿದ್ದಾನೋ
ಪ್ರಲಾಪಿ-ಸುತ್ತಿ-ರುವ
ಪ್ರಲಾಪಿ-ಸುವುದ-ರಲ್ಲಿ
ಪ್ರಲೋಭನ-ಕಾರಿ-ಗಳು
ಪ್ರಳಯ-ಗಳ
ಪ್ರಳಯ-ಗಳೆ-ರಡೂ
ಪ್ರಳಯ-ದಲಿ
ಪ್ರಳ-ಯ-ದಲ್ಲಿ
ಪ್ರಳಯ-ನಾಟ್ಯ-ಗೈವ
ಪ್ರಳಯಪ್ರ-ಭಂಜನ
ಪ್ರಳಯ-ವಾಗು-ವಾಗ
ಪ್ರಳಯ-ವಾದ
ಪ್ರಳ-ಯ-ವೆಂದರೆ
ಪ್ರಳಯ-ಶಬ್ದವ
ಪ್ರವ-ಚನ
ಪ್ರವರ್ತಕ-ನಾದರೂ
ಪ್ರವರ್ತ-ಕರು
ಪ್ರವರ್ತ-ಕರೇ
ಪ್ರವರ್ತಿ-ಸುತ್ತಾರೆ
ಪ್ರವರ್ತಿ-ಸು-ವಂತೆ
ಪ್ರವರ್ತಿ-ಸು-ವು-ದನ್ನು
ಪ್ರವಹಿಸಿ
ಪ್ರವಹಿ-ಸುತ
ಪ್ರವಹಿ-ಸುತ್ತಿದೆ
ಪ್ರವಹಿ-ಸುತ್ತಿದ್ದ
ಪ್ರವಹಿ-ಸು-ವುದು
ಪ್ರವಾದಿ-ಗಳಲ್ಲಿ
ಪ್ರವಾದಿ-ಗಳಿಗಂತೂ
ಪ್ರವಾಸದ
ಪ್ರವಾಹ
ಪ್ರವಾಹಕ್ಕೆದು-ರಾಗಿ
ಪ್ರವಾಹದ
ಪ್ರವಾಹ-ದಂತೆ
ಪ್ರವಾಹ-ದಲ್ಲಿ
ಪ್ರವಾಹ-ದಲ್ಲೇ
ಪ್ರವಾಹ-ದಿಂದ
ಪ್ರವಾಹ-ದೋ-ಪಾದಿ
ಪ್ರವಾಹ-ರೂಪ-ವಾಗಿ
ಪ್ರವಾ-ಹವು
ಪ್ರವಾಹ-ವೆಂದು
ಪ್ರವಾಹವೇ
ಪ್ರವಿ-ಚಲಂತಿ
ಪ್ರವೀಣ-ನಾದ
ಪ್ರವೀಣ-ರಾಗುವೆವು
ಪ್ರವೀಣ-ರಾ-ದ-ವ-ರಿಗೆ
ಪ್ರವೀಣ-ರಾ-ದು-ದ-ರಿಂದ
ಪ್ರವೀ-ಣರು
ಪ್ರವೃತ್ತ-ರಾದರು
ಪ್ರವೃತ್ತಿ
ಪ್ರವೃತ್ತಿ-ಗಳ
ಪ್ರವೃತ್ತಿ-ಗಳನ್ನು
ಪ್ರವೃತ್ತಿ-ಗಳಿಗೆ
ಪ್ರವೃತ್ತಿ-ಗಳಿವೆ
ಪ್ರವೃತ್ತಿ-ಗಳು
ಪ್ರವೃತ್ತಿ-ಗಳೆಂಬ
ಪ್ರವೃತ್ತಿಗೆ
ಪ್ರವೃತ್ತಿ-ಯನ್ನು
ಪ್ರವೃತ್ತಿ-ಯನ್ನೇ
ಪ್ರವೃತ್ತಿ-ಯಾದರೂ
ಪ್ರವೃತ್ತಿ-ಯಿಂದ
ಪ್ರವೃತ್ತಿಯು
ಪ್ರವೃತ್ತಿ-ಯುಂಟಾ-ಗುತ್ತದೆ
ಪ್ರವೃತ್ತಿಯೂ
ಪ್ರವೇಶ
ಪ್ರವೇ-ಶ-ವಿಲ್ಲ
ಪ್ರವೇ-ಶಿಲ
ಪ್ರವೇ-ಶಿ-ಸದೆ
ಪ್ರವೇ-ಶಿಸ-ಲಾರದು
ಪ್ರವೇ-ಶಿ-ಸಲು
ಪ್ರವೇ-ಶಿಸಿ
ಪ್ರವೇ-ಶಿ-ಸಿತ್ತು
ಪ್ರವೇ-ಶಿ-ಸಿದ
ಪ್ರವೇ-ಶಿಸಿ-ದರು
ಪ್ರವೇ-ಶಿಸಿ-ದರೆ
ಪ್ರವೇ-ಶಿ-ಸಿದೆ
ಪ್ರವೇ-ಶಿಸಿ-ದೊಡ-ನೆಯೇ
ಪ್ರವೇ-ಶಿಸಿದ್ದು-ದ-ರಿಂದ
ಪ್ರವೇ-ಶಿ-ಸಿಯೇ
ಪ್ರವೇ-ಶಿ-ಸುತ್ತ
ಪ್ರವೇ-ಶಿ-ಸುತ್ತದೆ
ಪ್ರವೇ-ಶಿ-ಸುವ
ಪ್ರವೇ-ಶಿ-ಸು-ವಂತೆ
ಪ್ರವೇ-ಶಿ-ಸುವರು
ಪ್ರವೇ-ಶಿ-ಸು-ವಷ್ಟು
ಪ್ರವೇ-ಶಿ-ಸುವುದಕ್ಕಿಂತ
ಪ್ರವೇ-ಶಿ-ಸು-ವುದು
ಪ್ರವ್ರಜೇತ್
ಪ್ರಶಂಸನೀ-ಯನು
ಪ್ರಶಂಸನೀಯ-ವಾದುದೇನೂ
ಪ್ರಶಂಸಿ-ಸಲ್ಪಡು-ವುದು
ಪ್ರಶಂಸಿಸಿ
ಪ್ರಶಂಸೆ
ಪ್ರಶಂಸೆ-ಯನ್ನೂ
ಪ್ರಶಂಸೆಯೇ
ಪ್ರಶಾಂತ
ಪ್ರಶಾಂತ-ಚಿತ್ತರೊ
ಪ್ರಶಾಂತತೆ
ಪ್ರಶಾಂತ-ವಾಗಿ
ಪ್ರಶಾಂತ-ವಾದ
ಪ್ರಶಾಂತ-ವಾದರೆ
ಪ್ರಶಿಷ್ಯರು
ಪ್ರಶೋತ್ತರ-ಗಳಲ್ಲಿ
ಪ್ರಶ್ನಾದಿ-ಗಳನ್ನು
ಪ್ರಶ್ನಿ-ಸಲು
ಪ್ರಶ್ನಿಸಿದ
ಪ್ರಶ್ನಿಸಿ-ದ-ವನು
ಪ್ರಶ್ನಿ-ಸಿದಾಗ
ಪ್ರಶ್ನಿ-ಸುತ್ತಿದ್ದನು
ಪ್ರಶ್ನಿ-ಸು-ವು-ದನ್ನು
ಪ್ರಶ್ನಿ-ಸು-ವು-ದ-ರಿಂದ
ಪ್ರಶ್ನೆ
ಪ್ರಶ್ನೆ-ಗಳನ್ನು
ಪ್ರಶ್ನೆ-ಗಳನ್ನೆಲ್ಲಾ
ಪ್ರಶ್ನೆ-ಗಳಿಗೆ
ಪ್ರಶ್ನೆ-ಗಳು
ಪ್ರಶ್ನೆ-ಗ-ಳೆಂದು
ಪ್ರಶ್ನೆಗೂ
ಪ್ರಶ್ನೆಗೆ
ಪ್ರಶ್ನೆ-ಯನ್ನು
ಪ್ರಶ್ನೆ-ಯನ್ನೂ
ಪ್ರಶ್ನೆ-ಯನ್ನೇ
ಪ್ರಶ್ನೆ-ಯಲ್ಲಿ-ರುವ
ಪ್ರಶ್ನೆ-ಯಾದ
ಪ್ರಶ್ನೆಯು
ಪ್ರಶ್ನೆಯೂ
ಪ್ರಶ್ನೋತ್ತರ-ಗಳಿಂದ
ಪ್ರಶ್ನೋತ್ತರ-ಗಳೂ
ಪ್ರಸಂಗ
ಪ್ರಸಂಗ-ಗಳು
ಪ್ರಸಂಗ-ಗಳೂ
ಪ್ರಸಂಗ-ದಲ್ಲಿಯೂ
ಪ್ರಸಕ್ತಿ-ಯನ್ನು
ಪ್ರಸನ್ನಳಾಗ-ದಿದ್ದಲ್ಲಿ
ಪ್ರಸನ್ನ-ಳಾಗಿ
ಪ್ರಸ-ರಿಸಿ
ಪ್ರಸಾದ
ಪ್ರಸಾದ-ಗಳಿಗೆ
ಪ್ರಸಾದ-ದಂತೆ
ಪ್ರಸಾದ-ರೂಪ-ವಾದ
ಪ್ರಸಾದ-ವನ್ನು
ಪ್ರಸಾದ-ವೆಂದು
ಪ್ರಸಾದ್
ಪ್ರಸಾರದ
ಪ್ರಸಾರ-ವಾ-ಯಿತೋ
ಪ್ರಸಿದ್ಧ
ಪ್ರಸಿದ್ಧಮ್
ಪ್ರಸಿದ್ಧ-ರಾಗಿದ್ದಾರೆಯೋ
ಪ್ರಸಿದ್ಧ-ವಾಗಿ-ರುವ
ಪ್ರಸಿದ್ಧ-ವಾದ
ಪ್ರಸ್ತಾಪ
ಪ್ರಸ್ತಾಪ-ಗಳನ್ನು
ಪ್ರಸ್ತಾಪ-ಗಳು
ಪ್ರಸ್ತಾಪ-ವನ್ನು
ಪ್ರಸ್ತಾಪ-ವನ್ನೆತ್ತಿ-ಕೊಂಡು
ಪ್ರಸ್ತಾಪ-ವಿಲ್ಲ
ಪ್ರಸ್ತಾ-ಪವು
ಪ್ರಸ್ತಾಪಿ-ಸಿದಾಗ
ಪ್ರಸ್ತಾವ
ಪ್ರಸ್ತಾ-ವಕ್ಕೆ
ಪ್ರಸ್ತಾವ-ದಲ್ಲಿ
ಪ್ರಸ್ತಾವ-ನೆ-ಯನ್ನು
ಪ್ರಸ್ತಾವ-ವನ್ನು
ಪ್ರಸ್ತಾವ-ವನ್ನೆತ್ತಿ
ಪ್ರಸ್ತಾವ-ವನ್ನೆತ್ತಿ-ದರು
ಪ್ರಸ್ತಾವವು
ಪ್ರಸ್ತುತ
ಪ್ರಹಾರ
ಪ್ರಹಾರ-ಗಳ
ಪ್ರಹಾರ-ಗಳಿಂದ
ಪ್ರಹ್ಲಾದನ
ಪ್ರಹ್ಲಾದ-ನಿಗೆ
ಪ್ರಾಂತ-ದಲ್ಲಿ
ಪ್ರಾಂತ-ದ-ವ-ರೆಗೂ
ಪ್ರಾಂತ-ದಿಂದ
ಪ್ರಾಂತವು
ಪ್ರಾಕೃತಿಕ
ಪ್ರಾಚೀನ
ಪ್ರಾಚೀನ-ತಮ-ವಾದ
ಪ್ರಾಚೀನ-ರೀತಿ
ಪ್ರಾಚೀನ-ವಾದ
ಪ್ರಾಚ್ಯ
ಪ್ರಾಚ್ಯರ
ಪ್ರಾಜ್ಞ-ನಾದ-ವನು
ಪ್ರಾಜ್ಞರೂ
ಪ್ರಾಣ
ಪ್ರಾಣ-ಕಾಫೆ
ಪ್ರಾಣ-ಜಾರ
ಪ್ರಾಣ-ತೊರೆ-ಯು-ವುದೇನೂ
ಪ್ರಾಣತ್ಯಾಗ
ಪ್ರಾಣದ
ಪ್ರಾಣ-ಪಕ್ಷಿ-ಯನು
ಪ್ರಾಣ-ಪನ
ಪ್ರಾಣ-ಪಾಖೀ
ಪ್ರಾಣ-ವದು
ಪ್ರಾಣ-ವನೆ
ಪ್ರಾಣ-ವನ್ನರ್ಪಿ-ಸಲು
ಪ್ರಾಣ-ವನ್ನಾದರೂ
ಪ್ರಾಣ-ವನ್ನು
ಪ್ರಾಣ-ವನ್ನೇ
ಪ್ರಾಣ-ವಿಚ್ಛೇದ-ಸೂತ್ಕಂ
ಪ್ರಾಣವು
ಪ್ರಾಣವೂ
ಪ್ರಾಣ-ಶಕ್ತಿ-ಯನ್ನು
ಪ್ರಾಣ-ಸಖ
ಪ್ರಾಣ-ಸಖಾ
ಪ್ರಾಣ-ಸದಾ-ಲೋಲ
ಪ್ರಾಣ-ಸಾಕ್ಷಿ
ಪ್ರಾಣ-ಸಾರೈರ್ಭೌಮ-ನಾರಾಯಣಾನಾಂ
ಪ್ರಾಣ-ಹಾನಿ
ಪ್ರಾಣ-ಹೀನ
ಪ್ರಾಣ-ಹೀನತೆ
ಪ್ರಾಣಾಯಾಮ
ಪ್ರಾಣಾಯಾ-ಮದ
ಪ್ರಾಣಾರ್ಪಣ
ಪ್ರಾಣಾರ್ಪಣ-ಜ-ಗತ
ಪ್ರಾಣಿ
ಪ್ರಾಣಿ-ಗಳ
ಪ್ರಾಣಿ-ಗಳಂತೆ
ಪ್ರಾಣಿ-ಗಳನ್ನು
ಪ್ರಾಣಿ-ಗಳನ್ನೆಂದೂ
ಪ್ರಾಣಿ-ಗಳನ್ನೆಲ್ಲಾ
ಪ್ರಾಣಿ-ಗ-ಳಲ್ಲೂ
ಪ್ರಾಣಿ-ಗಳಾಗಿದ್ದೆವು
ಪ್ರಾಣಿ-ಗಳಿಗೂ
ಪ್ರಾಣಿ-ಗಳಿಗೆ
ಪ್ರಾಣಿ-ಗಳು
ಪ್ರಾಣಿ-ಗಳೂ
ಪ್ರಾಣಿ-ಗಳೆಲ್ಲ
ಪ್ರಾಣಿ-ಗಳೆಲ್ಲ-ವನ್ನು
ಪ್ರಾಣಿಗೂ
ಪ್ರಾಣಿ-ಜ-ಗತ್ತಿ-ನಲ್ಲಿ
ಪ್ರಾಣಿ-ದಯೆ
ಪ್ರಾಣಿ-ಬಲಿ
ಪ್ರಾಣಿ-ಯನ್ನಾಗಲೀ
ಪ್ರಾಣಿ-ಯನ್ನು
ಪ್ರಾಣಿ-ಯಲ್ಲಿಯೂ
ಪ್ರಾಣಿಯೂ
ಪ್ರಾಣಿಯೋ
ಪ್ರಾಣಿ-ವಧೆ
ಪ್ರಾಣಿ-ಸಂಕುಲ-ವನ್ನೂ
ಪ್ರಾಣಿ-ಸೇವೆ-ಗಾಗಿ
ಪ್ರಾಣಿ-ಹತ್ಯೆ
ಪ್ರಾಣಿ-ಹತ್ಯೆ-ಯನ್ನು
ಪ್ರಾಣೀ
ಪ್ರಾತಃಕಾಲದ
ಪ್ರಾಧಾನ್ಯ
ಪ್ರಾಧಾನ್ಯ-ಕೊಡು-ವರು
ಪ್ರಾಧಾನ್ಯ-ವನ್ನು
ಪ್ರಾಪಂಚಿಕ
ಪ್ರಾಪಂಚಿ-ಕತೆ
ಪ್ರಾಪಂಚಿಕ-ತೆಗೆ
ಪ್ರಾಪ್ತಂ
ಪ್ರಾಪ್ತ-ವಾಗಲಿ
ಪ್ರಾಪ್ತ-ವಾಗಿದೆ
ಪ್ರಾಪ್ತ-ವಾಗಿ-ರ-ಬೇಕು
ಪ್ರಾಪ್ತ-ವಾಗಿ-ರುವ
ಪ್ರಾಪ್ತ-ವಾಗು-ವುದು
ಪ್ರಾಪ್ತ-ವಾದ
ಪ್ರಾಪ್ತ-ವಾದಾಗ
ಪ್ರಾಪ್ತಾ
ಪ್ರಾಪ್ತಿ-ಯಾ-ಗದೆ
ಪ್ರಾಪ್ತಿ-ಯಾಗುತ್ತದೆ
ಪ್ರಾಪ್ಯ
ಪ್ರಾಪ್ಯ-ವರಾನ್
ಪ್ರಾಪ್ಯ-ವರಾನ್ನಿ
ಪ್ರಾಪ್ಯ-ವರಾನ್ನಿ-ಬೋಧತ
ಪ್ರಾಪ್ಸ್ಯಸಿ
ಪ್ರಾಬಲ್ಯ
ಪ್ರಾಬಲ್ಯಕ್ಕಾಗಿ
ಪ್ರಾಮಾಣಿಕ-ತನ-ವಿದ್ದಲ್ಲಿ
ಪ್ರಾಮಾಣಿ-ಕತೆ-ಯಿ-ರಲಿ
ಪ್ರಾಮಾಣಿಕ-ರಾಗಿದ್ದರೆ
ಪ್ರಾಮಾಣ್ಯ-ದಿಂದ
ಪ್ರಾಮುಖ್ಯ
ಪ್ರಾಮುಖ್ಯತೆ
ಪ್ರಾಯ
ಪ್ರಾಯಶಃ
ಪ್ರಾಯಶ್ಚಿತ್ತ
ಪ್ರಾಯಶ್ಚಿತ್ತ-ಗಳನ್ನು
ಪ್ರಾಯೋಗಿಕ
ಪ್ರಾರಂಭ
ಪ್ರಾರಂಭ-ದಲ್ಲಿ
ಪ್ರಾರಂಭ-ದಲ್ಲಿ-ರುವ
ಪ್ರಾರಂಭ-ದಲ್ಲೇ
ಪ್ರಾರಂಭ-ಮಾಡ-ಬೇಕಾಗಿ
ಪ್ರಾರಂಭ-ವಾಗಿ-ರ-ಬೇಕೆಂದು
ಪ್ರಾರಂಭ-ವಾಗುವ
ಪ್ರಾರಂಭ-ವಾಗು-ವುದಕ್ಕೆ
ಪ್ರಾರಂಭ-ವಾಗು-ವುದು
ಪ್ರಾರಂಭ-ವಾಗು-ವುದೊ
ಪ್ರಾರಂಭ-ವಾದ
ಪ್ರಾರಂಭ-ವಾ-ದಂತೆ
ಪ್ರಾರಂಭ-ವಾ-ಯಿತು
ಪ್ರಾರಂಭಿಕ
ಪ್ರಾರಂಭಿಸ-ಬೇಕು
ಪ್ರಾರಂಭಿ-ಸಲು
ಪ್ರಾರಂಭಿ-ಸಿದ
ಪ್ರಾರಂಭಿ-ಸಿ-ದರು
ಪ್ರಾರಂಭಿ-ಸಿದರೆ
ಪ್ರಾರಂಭಿಸಿ-ದೆನು
ಪ್ರಾರಂಭಿಸಿದ್ದಾರೆಂದು
ಪ್ರಾರಂಭಿಸು
ಪ್ರಾರಂಭಿಸುತ್ತಿರು-ವೆ-ನೆಂಬು-ದನ್ನು
ಪ್ರಾರಂಭಿ-ಸುವ
ಪ್ರಾರಂಭಿಸು-ವುದಕ್ಕೆ
ಪ್ರಾರಬ್ಧ
ಪ್ರಾರಬ್ಧ-ದಿಂದ
ಪ್ರಾರ್ಥನಾ
ಪ್ರಾರ್ಥನೆ
ಪ್ರಾರ್ಥನೆಗೆ
ಪ್ರಾರ್ಥನೆ-ಯಂತೆ
ಪ್ರಾರ್ಥನೆ-ಯನ್ನು
ಪ್ರಾರ್ಥನೆ-ಯಿಂದ
ಪ್ರಾರ್ಥಿಸಿ
ಪ್ರಾರ್ಥಿ-ಸಿದ
ಪ್ರಾರ್ಥಿಸಿ-ದನು
ಪ್ರಾರ್ಥಿಸಿ-ದರೆ
ಪ್ರಾರ್ಥಿ-ಸುತ್ತಿದ್ದೆ
ಪ್ರಾರ್ಥಿ-ಸುತ್ತಿ-ರುವ
ಪ್ರಾರ್ಥಿ-ಸುವರೋ
ಪ್ರಾರ್ಥಿ-ಸು-ವು-ದ-ರಿಂದ
ಪ್ರಾರ್ಥಿ-ಸುವು-ದೊಂದೆ
ಪ್ರಾವೀಣ್ಯ
ಪ್ರಾಶಸ್ತ್ಯ
ಪ್ರಾಸಂಗಿಕ
ಪ್ರಾಸಂಗಿಕ-ವಾಗಿ
ಪ್ರಿಯ
ಪ್ರಿಯ-ತಮ
ಪ್ರಿಯ-ತಮ-ನಂತೆ
ಪ್ರಿಯ-ತಮ-ನೆಂದು
ಪ್ರಿಯ-ತಮ-ಳೆಂದು
ಪ್ರಿಯ-ನಾಥ
ಪ್ರಿಯ-ನಾಥನ
ಪ್ರಿಯ-ಮಹಾ-ಶಯ
ಪ್ರಿಯ-ಸಖನೆ
ಪ್ರಿಸ್ಬಿಟೇರಿಯನ್ನ-ರಾಗು-ವುದಕ್ಕೆ
ಪ್ರಿಸ್ಬಿಟೇರಿಯನ್ನರು
ಪ್ರೀತಿ
ಪ್ರೀತಿಗೂ
ಪ್ರೀತಿಗೆ
ಪ್ರೀತಿ-ಜಲ-ವನ್ನ-ರಸಿ
ಪ್ರೀತಿ-ಪೂರ್ವಕ
ಪ್ರೀತಿಯ
ಪ್ರೀತಿ-ಯಂತೆ
ಪ್ರೀತಿ-ಯದು
ಪ್ರೀತಿ-ಯನ್ನು
ಪ್ರೀತಿ-ಯಲ್ಲಿ
ಪ್ರೀತಿ-ಯಿಂದ
ಪ್ರೀತಿಯೇ
ಪ್ರೀತಿ-ಸ-ತೊಡಗಿದೆ
ಪ್ರೀತಿ-ಸ-ಬ-ಹುದು
ಪ್ರೀತಿ-ಸ-ಬೇಕು
ಪ್ರೀತಿ-ಸ-ಬೇಕೆಂದು
ಪ್ರೀತಿ-ಸಲಿ
ಪ್ರೀತಿ-ಸ-ಲಿಲ್ಲ
ಪ್ರೀತಿಸಿ
ಪ್ರೀತಿ-ಸಿದೆ
ಪ್ರೀತಿ-ಸಿ-ರ-ಬ-ಹುದು
ಪ್ರೀತಿಸು
ಪ್ರೀತಿ-ಸುತ್ತಾ-ನೆಂದು
ಪ್ರೀತಿ-ಸುತ್ತಾರೆ
ಪ್ರೀತಿ-ಸುತ್ತಿದ್ದ
ಪ್ರೀತಿ-ಸುತ್ತಿದ್ದರು
ಪ್ರೀತಿ-ಸುತ್ತಿದ್ದೆ
ಪ್ರೀತಿ-ಸುತ್ತೇನೆ
ಪ್ರೀತಿ-ಸುತ್ತೇವೆ
ಪ್ರೀತಿ-ಸುವ
ಪ್ರೀತಿ-ಸು-ವಂತೆ
ಪ್ರೀತಿ-ಸು-ವನು
ಪ್ರೀತಿ-ಸು-ವ-ರೆಂಬು-ದನ್ನು
ಪ್ರೀತಿ-ಸು-ವ-ವನು
ಪ್ರೀತಿ-ಸು-ವಾಗ
ಪ್ರೀತಿ-ಸು-ವುದಕ್ಕಾಗಿ
ಪ್ರೀತಿ-ಸು-ವು-ದನ್ನು
ಪ್ರೀತಿ-ಸು-ವುದು
ಪ್ರೀತಿ-ಸುವೆ
ಪ್ರೇಕ್ಷಕ-ರಲ್ಲಿ
ಪ್ರೇತ-ಗಳ
ಪ್ರೇತ-ಗಳನ್ನು
ಪ್ರೇತ-ಗಳೇನೊ
ಪ್ರೇತ-ಭೂಮಿ
ಪ್ರೇತ-ವನ್ನಾದರೂ
ಪ್ರೇತಾತ್ಮಕ್ಕೆ
ಪ್ರೇತಾತ್ಮ-ಗಳ
ಪ್ರೇತಾತ್ಮ-ಗಳಿಗೆ
ಪ್ರೇಮ
ಪ್ರೇಮ-ಏಯಿ-ಮಾತ್ರ
ಪ್ರೇಮಕ್ಕೆ
ಪ್ರೇಮ-ಗಳಲ್ಲಿ
ಪ್ರೇಮ-ಗಳೆಂಬವು
ಪ್ರೇಮದ
ಪ್ರೇಮ-ದರ್ಶನ-ದಲ್ಲಿ
ಪ್ರೇಮ-ದಲ್ಲಿ
ಪ್ರೇಮ-ದಾರಾಧ-ನೆಯೆ
ಪ್ರೇಮ-ದಿಂದ
ಪ್ರೇಮ-ಪಂಜ-ರದಿ
ಪ್ರೇಮ-ಪಾನ-ದಲ್ಲಿ
ಪ್ರೇಮ-ಪೂರ್ವ-ಕ-ವಾಗಿ
ಪ್ರೇಮಪ್ರವಾಹಃ
ಪ್ರೇಮ-ಮತ್ತ
ಪ್ರೇಮ-ಮಯ
ಪ್ರೇಮ-ಮ-ಯನು
ಪ್ರೇಮ-ಮಯಿ-ಯಾಗಿ-ರ-ಬೇಕು
ಪ್ರೇಮಯಿ
ಪ್ರೇಮ-ರೂಪನು
ಪ್ರೇಮ-ರೂಪಸ್ಯ
ಪ್ರೇಮ-ರೂಪಿ
ಪ್ರೇಮ-ವನ್ನು
ಪ್ರೇಮ-ವಲ್ಲ
ಪ್ರೇಮ-ವಾಗಿ
ಪ್ರೇಮ-ವಾರಿ-ಧಿ-ಯಿ-ಹುದು
ಪ್ರೇಮ-ವಿದ್ದರೆ
ಪ್ರೇಮ-ವಿರು-ವು-ದಿಲ್ಲ
ಪ್ರೇಮ-ವಿಲ್ಲ
ಪ್ರೇಮವು
ಪ್ರೇಮವೂ
ಪ್ರೇಮವೆ
ಪ್ರೇಮ-ವೆಂದು
ಪ್ರೇಮವೇ
ಪ್ರೇಮ-ವೇನೋ
ಪ್ರೇಮ-ವೊಂದೇ
ಪ್ರೇಮ-ಸಿಂಧು
ಪ್ರೇಮ-ಸುಧಾ-ಪಾನ
ಪ್ರೇಮ-ಸುಧಾ-ಪಾನದ
ಪ್ರೇಮಸ್ವ-ರೂಪನು
ಪ್ರೇಮಸ್ವ-ರೂಪ-ನೆಂದು
ಪ್ರೇಮ-ಹೃದಯ-ರಾದ
ಪ್ರೇಮ-ಹೇತು
ಪ್ರೇಮಾ-ದರ್ಶದ
ಪ್ರೇಮಾ-ದರ್ಶ-ವನ್ನು
ಪ್ರೇಮಾ-ನಂದ
ಪ್ರೇಮಾ-ನಂದ-ರಿಗೆ
ಪ್ರೇಮಾ-ನಂದರು
ಪ್ರೇಮಾ-ನಂದ-ರೊಡನೆ
ಪ್ರೇಮಾರ್ಪಣ
ಪ್ರೇಮಾಶ್ರು-ಗಳಿಂದ
ಪ್ರೇಮಿ
ಪ್ರೇಮಿಕ
ಪ್ರೇಮಿ-ಗಳು
ಪ್ರೇಮಿ-ಸಿದ
ಪ್ರೇಮೇರ
ಪ್ರೇಮೇರ್
ಪ್ರೇಮೇಶ್ವ-ರನ್ನೂ
ಪ್ರೇಮೋನ್ಮಾದ-ದಲ್ಲಿ
ಪ್ರೇಯಸಿ
ಪ್ರೇಯಸ್ಸೋ
ಪ್ರೇರಕ
ಪ್ರೇರಣ
ಪ್ರೇರಣೆ-ಯಿಂದ
ಪ್ರೇರಿತ-ರಾಗಿ
ಪ್ರೇರೇ-ಪಣೆ
ಪ್ರೇರೇ-ಪಣೆ-ಗಳನ್ನೂ
ಪ್ರೇರೇಪಿತ-ರಾಗಿ
ಪ್ರೇರೇಪಿತ-ವಾದದ್ದೆಂದು
ಪ್ರೇರೇಪಿಸ-ಬೇಕು
ಪ್ರೇರೇಪಿಸಿ
ಪ್ರೇರೇಪಿ-ಸುತ್ತಿದೆ
ಪ್ರೇರೇಪಿ-ಸುತ್ತಿ-ರುವ-ವಳು
ಪ್ರೇರೇಪಿ-ಸುತ್ತಿಲ್ಲ
ಪ್ರೇರೇಪಿ-ಸುತ್ತೇನೆ
ಪ್ರೇರೇಪಿ-ಸು-ವುದು
ಪ್ರೊ
ಪ್ರೋಜ್ವಲ-ಭಕ್ತಿ-ಪಟಾವೃತ-ವೃತ್ತಂ
ಪ್ರೋತ್ಸಾಹ
ಪ್ರೋತ್ಸಾಹ-ಗೊಳಿ-ಸುತ್ತೀರಿ
ಪ್ರೋತ್ಸಾಹ-ವನ್ನು
ಪ್ರೋತ್ಸಾಹವೇ
ಪ್ರೋತ್ಸಾಹಿಸಿ
ಪ್ರೌಢ
ಪ್ಲೇಗಿ-ನಂತೆ
ಪ್ಲೇಗ್
ಪ್ಲೇಟೋ-ವಿನ
ಫಕೀರ
ಫಬ್ರವರಿ
ಫಲ
ಫಲ-ಕಾರಿ
ಫಲ-ಕಾರಿ-ಯಾಗ-ಬಾ-ರದು
ಫಲ-ಗಳು
ಫಲದ
ಫಲ-ದಲ್ಲಿ
ಫಲ-ದಿಂದ
ಫಲ-ದಿಂದುಂಟಾದ
ಫಲಪ್ರದ-ವಾಗು-ವುದು
ಫಲ-ಮ-ಹಾಸ್ತಿ
ಫಲ-ರ-ಹಿತ-ನಾ-ಗಲಿ
ಫಲ-ವತ್ತಾಗಿದೆ
ಫಲ-ವನ್ನು
ಫಲ-ವನ್ನುಣ್ಣುತ್ತಿದ್ದಾರೆ
ಫಲ-ವಲ್ಲ
ಫಲ-ವಾಗಿ
ಫಲ-ವಾಗಿದೆ
ಫಲ-ವಾಗು-ವುದು
ಫಲ-ವಾಗು-ವು-ದೇನು
ಫಲ-ವಿಲ್ಲ
ಫಲ-ವಿಲ್ಲ-ವೆಂದು
ಫಲವು
ಫಲ-ವೆಂಬುದು
ಫಲ-ವೇ-ನಪ್ಪಾ
ಫಲ-ವೇನು
ಫಲ-ಸ-ಹಿತ-ನಾ-ಗಲಿ
ಫಲಾಕಾಂಕ್ಷೆ-ಯಿಲ್ಲದೆ
ಫಲಾಪೇಕ್ಷೆ
ಫಲಾ-ಫಲ-ಗಳನ್ನು
ಫಲಾ-ಹಾರ
ಫಲಾಹಾ-ರಕ್ಕೆ
ಫಲಾ-ಹಾರದ
ಫಲಿ-ತಾಂಶ-ಗಳನ್ನು
ಫಲಿ-ತಾಂಶ-ವೇನು
ಫಲಿ-ಸು-ವಂತೆ
ಫಲೋದ್ದೇಶ-ದಿಂದ
ಫಲೋದ್ದೇಶಿತ
ಫಾಟಿಗೋಲಾ
ಫಾದರ್
ಫಾಲ್ಗುಣ
ಫಿರಿ
ಫಿರೆ
ಫಿರೆ-ಚಾಯ
ಫಿರೆ-ನಯಿ
ಫಿರೆ-ಫಿರೆ
ಫಿಲಿಫೈನ್ಸ್
ಫೆಬ್ರವರಿ
ಫೇನ-ಮಯ
ಫೇನ-ಮಯಿ
ಫೇನ-ಶುಭ್ರ-ಶಿರ
ಫೇರೆ
ಫೋಟೋ
ಫ್ರಾನ್ಸಿಸ್
ಫ್ರಾನ್ಸಿಸ್ಕನ್
ಫ್ರೆಂಚರು
ಫ್ರೆಂಚ್
ಬಂ
ಬಂಗಜ
ಬಂಗಾ-ರದ
ಬಂಗಾರ-ವನ್ನು
ಬಂಗಾಲ
ಬಂಗಾಲಿ
ಬಂಗಾಲ್
ಬಂಗಾಳ
ಬಂಗಾಳಕ್ಕಿಂತಲೂ
ಬಂಗಾಳದ
ಬಂಗಾಳ-ದಲ್ಲಿ
ಬಂಗಾಳ-ದಲ್ಲಿಯೂ
ಬಂಗಾಳ-ದ-ವನು
ಬಂಗಾಳ-ದಿಂದ
ಬಂಗಾಳ-ದೇಶ
ಬಂಗಾಳ-ದೇಶ-ದಲ್ಲಿ
ಬಂಗಾಳನ
ಬಂಗಾಳ-ನಿಗೆ
ಬಂಗಾಳನು
ಬಂಗಾಳ-ವನ್ನು
ಬಂಗಾಳವು
ಬಂಗಾಳ-ವೊಂದನ್ನು
ಬಂಗಾಳಿ
ಬಂಗಾಳಿ-ಗಳಾಗಿದ್ದರೆ
ಬಂಗಾಳಿ-ಗಳಿ-ಗಿಂತ
ಬಂಗಾಳಿ-ಗಳು
ಬಂಗಾಳಿ-ಗೀತೆ-ಯನ್ನು
ಬಂಗಾಳಿಯ
ಬಂಗಾಳಿ-ಯಲ್ಲಿ
ಬಂಡವಾಳ
ಬಂಡಿ-ಯನ್ನು
ಬಂಡೆ
ಬಂಡೆಗೆ
ಬಂತು
ಬಂತೆಂದು
ಬಂತೊ
ಬಂತೋ
ಬಂದ
ಬಂದಂತೆ
ಬಂದ-ಕೂಡಲೆ
ಬಂದ-ಕೂಡಲೇ
ಬಂದದ್ದನ್ನು
ಬಂದದ್ದಾಯಿ-ತಾ-ದರೂ
ಬಂದದ್ದು
ಬಂದದ್ದೆಲ್ಲ-ವನ್ನೂ
ಬಂದದ್ದೇನು
ಬಂದದ್ದೇ-ನೆಂದರೆ
ಬಂದ-ನಂತರ
ಬಂದನು
ಬಂದ-ಮೇಲೆ
ಬಂದ-ಮೇಲೆಯೇ
ಬಂದರು
ಬಂದರೂ
ಬಂದರೆ
ಬಂದ-ರೆಂದು
ಬಂದಳು
ಬಂದ-ವನ
ಬಂದ-ವನು
ಬಂದ-ವ-ನೆಂದು
ಬಂದ-ವ-ರನ್ನು
ಬಂದ-ವ-ರಾರೂ
ಬಂದವು
ಬಂದ-ವು-ಗ-ಳಲ್ಲ
ಬಂದಾಗ
ಬಂದಾಗ-ಲೆಲ್ಲ
ಬಂದಾಗಿ-ನಿಂದ
ಬಂದಾಗಿ-ನಿಂದಲೂ
ಬಂದಾ-ಯಿತು
ಬಂದಾರಭ್ಯ
ಬಂದಿ-ತದು
ಬಂದಿತು
ಬಂದಿ-ತೊಂದು
ಬಂದಿತ್ತು
ಬಂದಿತ್ತೊ
ಬಂದಿದೆ
ಬಂದಿದೆ-ಯೆಂದರೆ
ಬಂದಿದೆ-ಯೆಂದು
ಬಂದಿದೆ-ಯೆಂಬು-ದಕ್ಕೆ
ಬಂದಿದ್ದ
ಬಂದಿದ್ದನು
ಬಂದಿದ್ದನ್ನು
ಬಂದಿದ್ದರು
ಬಂದಿದ್ದರೆ
ಬಂದಿದ್ದ-ವ-ರಲ್ಲಿ
ಬಂದಿದ್ದ-ವ-ರಿಗೆ
ಬಂದಿದ್ದಾನೆ
ಬಂದಿದ್ದಾರೆ
ಬಂದಿದ್ದಾರೆಂದು
ಬಂದಿದ್ದೀರಿ
ಬಂದಿದ್ದು
ಬಂದಿದ್ದೆ
ಬಂದಿದ್ದೇನೆ
ಬಂದಿರ-ಬೇ-ಕಲ್ಲವೆ
ಬಂದಿ-ರ-ಲಿಲ್ಲ
ಬಂದಿ-ರಲು
ಬಂದಿರಿ
ಬಂದಿರುತ್ತಾರೆ
ಬಂದಿರುವ
ಬಂದಿ-ರು-ವಂತೆ
ಬಂದಿ-ರುವನು
ಬಂದಿ-ರುವರು
ಬಂದಿರು-ವ-ರೆಂದು
ಬಂದಿ-ರು-ವಿರಿ
ಬಂದಿ-ರು-ವು-ದ-ರಿಂದ
ಬಂದಿ-ರು-ವುದು
ಬಂದಿರುವೆ
ಬಂದಿ-ರು-ವೆಯಾ
ಬಂದಿ-ರು-ವೆವು
ಬಂದಿಲ್ಲ
ಬಂದಿಲ್ಲ-ವಂತೆ
ಬಂದಿಲ್ಲವೋ
ಬಂದಿವೆ
ಬಂದೀ-ಖಾ-ನೆಯಲ್ಲಿ-ರ-ಬೇಕಾಗು-ವುದು
ಬಂದು
ಬಂದು-ದಕ್ಕೆ
ಬಂದು-ದದು
ಬಂದು-ದ-ರಿಂದ
ಬಂದು-ದಲ್ಲ
ಬಂದು-ದಾಗಿ-ರ-ಬೇಕು
ಬಂದು-ದಾ-ವಾಗ
ಬಂದುದು
ಬಂದು-ನೋಡ-ಬೇಕೆಂದು
ಬಂದು-ಬಿಟ್ಟರೆ
ಬಂದು-ಬಿಟ್ಟಿತು
ಬಂದು-ಬಿಟ್ಟಿದ್ದಂತಿತ್ತು
ಬಂದು-ಬಿಟ್ಟೆ
ಬಂದು-ಬಿಡಿ
ಬಂದು-ಬಿಡು
ಬಂದು-ಬಿಡುತ್ತದೆ
ಬಂದು-ಬಿಡುತ್ತಾರೆ
ಬಂದು-ಬಿಡುತ್ತೇನೆ
ಬಂದುವು
ಬಂದು-ವೆಂದು
ಬಂದೂಕ
ಬಂದೂ-ಕೇರ್
ಬಂದೆ
ಬಂದೆಯೊ
ಬಂದೆವು
ಬಂದೇ
ಬಂದೇನು
ಬಂದೊ-ಡನೆ
ಬಂದೊಡ-ನೆಯೇ
ಬಂದೊದಗುವು-ದೆಂಬು-ದನ್ನು
ಬಂಧ
ಬಂಧಃ
ಬಂಧ-ಗಳ
ಬಂಧನ
ಬಂಧ-ನಕ್ಕೀಡು
ಬಂಧ-ನಕ್ಕೆ
ಬಂಧ-ನಕ್ಕೊಳ-ಗಾಗಿ
ಬಂಧ-ನ-ಗಳನ್ನೆಲ್ಲಾ
ಬಂಧ-ನ-ಗಳೂ
ಬಂಧ-ನ-ಗಳೆಲ್ಲಾ
ಬಂಧ-ನ-ಗಳೇ
ಬಂಧ-ನದ
ಬಂಧ-ನ-ದಲ್ಲಿ
ಬಂಧ-ನ-ದಲ್ಲಿದ್ದಾನೋ
ಬಂಧ-ನ-ದಲ್ಲಿ-ರುವ
ಬಂಧ-ನದಿಂದ
ಬಂಧ-ನ-ರ-ಹಿತನು
ಬಂಧ-ನ-ವಿ-ಮುಕ್ತ-ರಾಗಲು
ಬಂಧ-ನವು
ಬಂಧ-ನ-ವುಂಟಾ-ಗುತ್ತದೆ
ಬಂಧ-ನ-ವುಂಟಾ-ಗುತ್ತದೆಂದು
ಬಂಧ-ನ-ವೆಂದ-ರೇನು
ಬಂಧ-ನವೇ
ಬಂಧಾ-ಜಾರ್ಕೃತ-ದಾಸ
ಬಂಧಿ-ಯಾದ
ಬಂಧಿಸ-ತಕ್ಕ
ಬಂಧಿ-ಸಲಾ-ಗದಷ್ಟು
ಬಂಧಿಸ-ಲಾ-ರವು
ಬಂಧಿ-ಸುವ
ಬಂಧು
ಬಂಧು-ಗಳು
ಬಂಧು-ಬಳಗ
ಬಂಧು-ವರ್ಗಕ್ಕೆ
ಬಂಧು-ವರ್ಗ-ದ-ವ-ರಿಗೇ
ಬಗೆ
ಬಗೆ-ಗಣ್ಣು
ಬಗೆ-ಗಳೂ
ಬಗೆ-ಗಿನ
ಬಗೆ-ಗಿ-ರುವ
ಬಗೆಗೆ
ಬಗೆ-ದಾಗ
ಬಗೆದು
ಬಗೆ-ಬಗೆ
ಬಗೆ-ಬ-ಗೆಯ
ಬಗೆ-ಬಗೆ-ಯೊಳೆಸೆ-ದಿ-ಹುದು
ಬಗೆಯ
ಬಗೆ-ಯ-ನರಿಯೆ
ಬಗೆ-ಯನ್ನು
ಬಗೆ-ಯಲ್ಲಿ
ಬಗೆ-ಯಾಗಿ
ಬಗೆ-ಯಾಗಿ-ರ-ಬ-ಹುದು
ಬಗೆ-ಯಾಗಿ-ರುವರು
ಬಗೆ-ಯಿಂದ
ಬಗೆ-ಯೇನೂ
ಬಗೆ-ಯೊ-ಳಗೆ
ಬಗೆ-ಯೊಳೊ-ಲುಮೆಯ
ಬಗೆ-ಹ-ರಿ-ಸ-ಲಾ-ಗು-ವು-ದಿಲ್ಲ
ಬಗೆ-ಹ-ರಿಸಿ-ಕೊಳ್ಳುವರು
ಬಗೆ-ಹರಿ-ಸಿದ್ದರು
ಬಗ್ಗಿ-ದುವು
ಬಗ್ಗಿಸಿ
ಬಗ್ಗಿಸಿ-ಕೊಂಡು
ಬಗ್ಗೆ
ಬಗ್ಗೆ-ಯಾಗಲೀ
ಬಗ್ಗೆ-ಯಾದರೋ
ಬಗ್ಗೆಯೂ
ಬಚಾಯಿ-ಸಿ-ಕೊಂಡರೆ
ಬಚ್ಚಿಡ-ಲಾ-ಗು-ವು-ದಿಲ್ಲ
ಬಜಾರಿನ
ಬಟ್ಟಲ
ಬಟ್ಟಲನು
ಬಟ್ಟಲಲಿ
ಬಟ್ಟಲಿದೊ
ಬಟ್ಟಲು
ಬಟ್ಟಲು-ಗಳೂ
ಬಟ್ಟಿ-ನಲ್ಲಿ
ಬಟ್ಟು
ಬಟ್ಟೆ
ಬಟ್ಟೆ-ಗಳನ್ನು
ಬಟ್ಟೆ-ಗಳನ್ನೆಲ್ಲಿಂದ
ಬಟ್ಟೆ-ಗಳಿಗೆ
ಬಟ್ಟೆ-ಗಳು
ಬಟ್ಟೆ-ಗಳೆಲ್ಲವೂ
ಬಟ್ಟೆ-ಬರೆ-ಗಳನ್ನು
ಬಟ್ಟೆ-ಬರೆ-ಗಳು
ಬಟ್ಟೆಯ
ಬಟ್ಟೆ-ಯಂತೆ
ಬಟ್ಟೆ-ಯನ್ನು
ಬಟ್ಟೆ-ಯನ್ನುಟ್ಟು
ಬಟ್ಟೆ-ಯಿಂದ
ಬಟ್ಟೆ-ಯಿಲ್ಲ
ಬಟ್ಟೆ-ಯಿಲ್ಲದೆ
ಬಟ್ಟೆಯೆ
ಬಡ
ಬಡಗಿ-ಮಾತ್ರವೇ
ಬಡಗಿ-ಯಾಗಿ
ಬಡಗಿ-ಯಾಗಿದ್ದೀರಾ
ಬಡಗಿಯೇ
ಬಡ-ಜನ
ಬಡ-ಜನರು
ಬಡ-ತಂದೆ
ಬಡ-ತ-ನಕ್ಕೆ
ಬಡ-ತನದ
ಬಡ-ದೇಶ
ಬಡಬಗ್ಗರ
ಬಡಬಗ್ಗ-ರಿಂದ
ಬಡಬಗ್ಗ-ರಿಗೆ
ಬಡಬಗ್ಗ-ರಿ-ಗೋಸ್ಕರ
ಬಡವ
ಬಡ-ವನ
ಬಡವ-ರಲ್ಲಿ
ಬಡವ-ರಾದ
ಬಡವ-ರಿಗೆ
ಬಡ-ವರು
ಬಡ-ವರೋ
ಬಡಾಯಿ
ಬಡಾಯಿ-ಕೊಚ್ಚಿ-ಕೊಳ್ಳುತ್ತೀರಿ
ಬಡಾಯಿ-ಯನ್ನು
ಬಡಿದಾಟ-ಗಳಾಗುತ್ತಿದ್ದುವೊ
ಬಡಿದಾಡಿ-ಕೊಳ್ಳುವುದಕ್ಕಿಂತಲೂ
ಬಡಿದು-ಕೊಳ್ಳುತ್ತಿತ್ತು
ಬಡಿದು-ಹಾಕಲು
ಬಡಿ-ಯುತ್ತಾ
ಬಡಿಸ-ಬೇಕು
ಬಡಿ-ಸಿದ
ಬಡಿ-ಸಿ-ದನು
ಬಡಿ-ಸಿ-ದರು
ಬಡಿ-ಸುತ್ತಾನೆ
ಬಡಿ-ಸು-ವುದಕ್ಕೆ
ಬಡೊಬಾ-ಜಾರಿ-ನಲ್ಲಿ
ಬಣಗಿ-ಗಳಾಗುತ್ತಾರೆ
ಬಣಗಿ-ಗಳೆ
ಬಣಗಿತ-ನವೇ
ಬಣ್ಣ
ಬಣ್ಣಕ್ಕಾಗಿದ್ದ
ಬಣ್ಣ-ಗಳ
ಬಣ್ಣ-ಗಳಿಂದ
ಬಣ್ಣ-ಗಳು
ಬಣ್ಣದ
ಬಣ್ಣ-ದೊಂದಿಗೆ
ಬಣ್ಣ-ವನ್ನು
ಬಣ್ಣವೂ
ಬಣ್ಣಿ-ಸಲು
ಬಣ್ಣಿಸಿ
ಬಣ್ಣಿ-ಸುತ್ತಿದ್ದ-ವರು
ಬತ್ತಲಿ
ಬತ್ತ-ವನ್ನು
ಬತ್ತಿ-ರು-ವುದು
ಬತ್ತಿ-ಹೋಗಿದೆ
ಬತ್ತು-ವುದು
ಬದಲಾ-ಗದೇ
ಬದಲಾಗ-ಬೇಕು
ಬದ-ಲಾಗಿ
ಬದ-ಲಾಗಿ-ಬಿಟ್ಟಿವೆ
ಬದ-ಲಾಗಿ-ಬಿಡುತ್ತದೆ
ಬದ-ಲಾ-ಗುತ್ತದೆ
ಬದಲಾ-ಗುತ್ತಿದೆ
ಬದಲಾಗುತ್ತಿ-ರುತ್ತದೆ
ಬದಲಾಗುತ್ತಿರುವ
ಬದಲಾಗುತ್ತಿ-ರು-ವುದು
ಬದಲಾಗು-ವುದು
ಬದಲಾ-ದರೆ
ಬದ-ಲಾ-ಯಿತು
ಬದಲಾಯಿ-ಸದೆ
ಬದಲಾಯಿಸ-ಬಲ್ಲದೆ
ಬದಲಾಯಿ-ಸಲು
ಬದಲಾಯಿಸಿ
ಬದಲಾಯಿಸಿ-ಕೊಂಡು
ಬದಲಾಯಿಸಿ-ಕೊಂಡೇ
ಬದಲಾಯಿ-ಸಿದ
ಬದಲಾಯಿಸಿ-ದು-ದನ್ನು
ಬದಲಾಯಿಸಿ-ರು-ವುದು
ಬದಲಾಯಿ-ಸುತ್ತ-ದೆಂಬು-ದನ್ನು
ಬದಲಾಯಿ-ಸುತ್ತಾ
ಬದಲಾಯಿ-ಸು-ವು-ದ-ರಿಂದ
ಬದಲಾಯಿ-ಸು-ವು-ದಿಲ್ಲ
ಬದಲಾಯಿ-ಸು-ವುದು
ಬದಲಾವಣೆ
ಬದಲಾವಣೆ-ಗಳನು
ಬದಲಾವಣೆ-ಗಳನ್ನು
ಬದಲಾವಣೆ-ಗಳೆಲ್ಲ
ಬದಲಾವಣೆ-ಮಾಡಿ
ಬದಲಾವ-ಣೆಯ
ಬದಲಾವಣೆ-ಯಿಂದ
ಬದಲಾವ-ಣೆಯೂ
ಬದ-ಲಿಗೆ
ಬದಲಿ-ಸ-ಬ-ಹುದು
ಬದಲಿ-ಸುತ
ಬದಲು
ಬದಿಗಿಟ್ಟರೂ
ಬದಿ-ಗಿಡು
ಬದಿಗೊತ್ತಿ
ಬದಿ-ಯಲ್ಲೇ
ಬದು-ಕಂದು
ಬದು-ಕನ್ನು
ಬದುಕ-ಬೇ-ಕಾದರೆ
ಬದುಕ-ಬೇಕು
ಬದುಕ-ಬೇಕೆಂದು
ಬದುಕ-ಬೇಕೆಂಬ
ಬದು-ಕಲಿ
ಬದುಕಿ
ಬದುಕಿ-ಕೊಂಡು-ಬಿಟ್ಟರೆ
ಬದುಕಿಗೆ
ಬದುಕಿದ
ಬದುಕಿ-ದರೆ
ಬದುಕಿದೆ
ಬದುಕಿದ್ದರೆ
ಬದುಕಿದ್ದಿದ್ದರೆ
ಬದು-ಕಿನ
ಬದುಕಿ-ನಲ್ಲಿ
ಬದುಕಿ-ನಿಂದ
ಬದುಕಿ-ರುತ್ತಾರೆ
ಬದುಕಿ-ಸಿ-ಕೊಳ್ಳ-ಬೇಕು
ಬದುಕು
ಬದುಕುತ್ತೇನೆ
ಬದುಕುವ
ಬದುಕು-ವುದು
ಬದು-ಕೋಣ
ಬದ್ದೋ
ಬದ್ಧ
ಬದ್ಧ-ನಲ್ಲ
ಬದ್ಧ-ನಾಗಿದ್ದೆ
ಬದ್ಧ-ನಾಗಿ-ರು-ವಾಗ
ಬದ್ಧ-ನಾಗಿ-ರು-ವು-ದ-ರಿಂದ
ಬದ್ಧ-ನಾಗಿ-ರು-ವು-ದಿಲ್ಲ
ಬದ್ಧ-ನೆಂದು
ಬದ್ಧಭ್ರುಕುಟಿಯ
ಬದ್ಧಭ್ರುಕುಟಿಯಲಿ
ಬದ್ಧ-ರಿಗೆ
ಬದ್ಧ-ಳಲ್ಲ
ಬದ್ಧ-ವಾದ
ಬದ್ಧಾಭಿ-ಮಾನ್ಯಪಿ
ಬನಾ-ರಸ್
ಬನೊ-ಯಾರಿ
ಬನೋ-ಯಾರಿ
ಬನ್ನಿ
ಬನ್ನಿರಿ
ಬಬ
ಬಬಬಬಬಮ್
ಬಯಕೆ
ಬಯಕೆ-ಗಳನ್ನು
ಬಯಕೆ-ಗ-ಳಲ್ಲೇ
ಬಯಕೆ-ಗಾಲಿಯು
ಬಯ-ಕೆಯ
ಬಯಕೆ-ಯಾದರೂ
ಬಯ-ಕೆಯು
ಬಯಕೆ-ವೇಷವ
ಬಯಲ
ಬಯಲ-ಗಾಳಿ
ಬಯ-ಲಿ-ನಲ್ಲಿ-ರಲು
ಬಯಲು
ಬಯ-ಸದಲೆ
ಬಯ-ಸದೆ
ಬಯಸ-ಬೇಕೆ
ಬಯ-ಸರು
ಬಯಸಿ-ದಲ್ಲಿ
ಬಯಸಿ-ದೊಡೆ
ಬಯ-ಸುತ್ತಾನೆ
ಬಯ-ಸುತ್ತಿತ್ತು
ಬಯ-ಸುವ
ಬಯಸು-ವಂತೆಯೇ
ಬಯ-ಸುವನು
ಬಯೆ
ಬರ-ಗಾಲ
ಬರ-ತಕ್ಕದ್ದು
ಬರ-ತೊಡಗಿತು
ಬರದ
ಬರ-ದಂತೆ
ಬರದೆ
ಬರ-ಬಲ್ಲರು
ಬರ-ಬಲ್ಲೆಯಾ
ಬರ-ಬ-ಹುದು
ಬರ-ಬಹುದೆ
ಬರ-ಬೇಕಾಗಿದೆ
ಬರ-ಬೇಕಾಗು-ವುದೊ
ಬರ-ಬೇ-ಕಾದದ್ದಿ-ದೆಯೋ
ಬರ-ಬೇ-ಕಾ-ದದ್ದು
ಬರ-ಬೇ-ಕಿತ್ತು
ಬರ-ಬೇಕು
ಬರ-ಬೇಕೆಂದಿರುವೆ
ಬರ-ಬೇಕೆಂದು
ಬರ-ಬೇಡ
ಬರ-ಬೇಡಿ
ಬರ-ಮಾಡಿ-ಕೊಂಡರು
ಬರ-ಮಾಡಿ-ಕೊಂಡು
ಬರ-ಮಾಡಿ-ಕೊಳ್ಳುವನೋ
ಬರ-ಮಾಡಿ-ಕೊಳ್ಳು-ವುದಕ್ಕೋಸ್ಕರ
ಬರ-ಲಾರಂಭ-ವಾ-ಯಿತು
ಬರ-ಲಾರದು
ಬರ-ಲಾರರೆ
ಬರಲಿ
ಬರ-ಲಿಲ್ಲ
ಬರಲು
ಬರಲೆಂದು
ಬರಲೆಂದೇ
ಬರಲೆ-ಬೇಕು
ಬರಲೇ
ಬರಲೇ-ಬೇಕು
ಬರಹ-ಗಳು
ಬರಹಗಾರನು
ಬರಹ-ದಲ್ಲಿ
ಬರಹ-ವನ್ನು
ಬರ-ಹೇಳಿ
ಬರಿ
ಬರಿ-ಗಾ-ಲಿನ
ಬರಿ-ಗೈ-ಯಲ್ಲಿ
ಬರಿದು
ಬರಿದೆ
ಬರಿದೇ
ಬರಿಯ
ಬರಿ-ಯೋ-ಟವು
ಬರಿ-ಸು-ವು-ದರ
ಬರೀ
ಬರು-ತಿದ್ದರು
ಬರು-ತಿಹ
ಬರುತ್ತ
ಬರುತ್ತದೆ
ಬರುತ್ತ-ದೆಂಬು-ದನ್ನು
ಬರುತ್ತ-ದೆಯೋ
ಬರುತ್ತವೆ
ಬರುತ್ತ-ವೆಂಬು-ದನ್ನು
ಬರುತ್ತಾನೆ
ಬರುತ್ತಾರೆ
ಬರುತ್ತಾ-ರೆಂದು
ಬರುತ್ತಾ-ರೆಯೋ
ಬರುತ್ತಿತ್ತು
ಬರುತ್ತಿದೆ
ಬರುತ್ತಿದ್ದ
ಬರುತ್ತಿದ್ದಂತೆ
ಬರುತ್ತಿದ್ದನು
ಬರುತ್ತಿದ್ದ-ನೆಂಬುದೂ
ಬರುತ್ತಿದ್ದರು
ಬರುತ್ತಿದ್ದರೆ
ಬರುತ್ತಿದ್ದರೊ
ಬರುತ್ತಿದ್ದಾರೆ
ಬರುತ್ತಿದ್ದುವು
ಬರುತ್ತಿದ್ದೆಯೋ
ಬರುತ್ತಿರ-ಬೇಕು
ಬರುತ್ತಿ-ರ-ಲಿಲ್ಲ
ಬರುತ್ತಿ-ರುತ್ತದೆ
ಬರುತ್ತಿ-ರುತ್ತದೆ-ಯೆಂದೂ
ಬರುತ್ತಿರುತ್ತಾರೆ
ಬರುತ್ತಿರುವ
ಬರುತ್ತಿರು-ವರೋ
ಬರುತ್ತಿ-ರು-ವಿರಿ
ಬರುತ್ತಿ-ರು-ವುದು
ಬರುತ್ತೀ-ಯಷ್ಟೆ
ಬರುತ್ತೀರೆಂದೇ
ಬರುತ್ತೇನೆ
ಬರುತ್ತೇವೆ
ಬರು-ಬರುತ್ತಾ
ಬರುವ
ಬರುವಂತಹ
ಬರು-ವಂತಿತ್ತು
ಬರು-ವಂತಿಲ್ಲ
ಬರು-ವಂತೆ
ಬರುವ-ನ-ವನು
ಬರುವನು
ಬರುವರು
ಬರುವ-ರೆಂದು
ಬರುವರೋ
ಬರುವಳು
ಬರುವ-ವರು
ಬರುವ-ವ-ರೆಗೂ
ಬರುವ-ವರೆಗೆ
ಬರುವ-ವ-ರೆಲ್ಲಾ
ಬರು-ವಷ್ಟು
ಬರುವ-ಹಾ-ಗಿದೆ
ಬರುವಾಗ
ಬರುವುದಕ್ಕಾಗುತ್ತಿತ್ತೇ-ನಯ್ಯಾ
ಬರು-ವುದಕ್ಕಾ-ಗು-ವು-ದಿಲ್ಲ
ಬರುವುದಕ್ಕಿಲ್ಲ
ಬರು-ವುದಕ್ಕೆ
ಬರು-ವುದಕ್ಕೆಂದು
ಬರು-ವು-ದನ್ನು
ಬರುವುದ-ರಲ್ಲಿ
ಬರು-ವು-ದಲ್ಲ
ಬರು-ವು-ದಿಲ್ಲ
ಬರು-ವು-ದಿಲ್ಲವೆ
ಬರು-ವು-ದಿಲ್ಲ-ವೆಂದು
ಬರು-ವುದು
ಬರು-ವುದುಂಟೆ
ಬರು-ವುದೂ
ಬರು-ವು-ದೆಂದು
ಬರು-ವುದೇ
ಬರು-ವುದೇನು
ಬರು-ವುದೊ
ಬರು-ವುದೋ
ಬರು-ವುವು
ಬರುವೆ
ಬರುವೆ-ಯಂತೆ
ಬರು-ವೆಯಾ
ಬರುವೆ-ಯೇನು
ಬರೆ
ಬರೆದ
ಬರೆ-ದದ್ದು
ಬರೆ-ದನು
ಬರೆ-ದರು
ಬರೆ-ದರೂ
ಬರೆ-ದರೆ
ಬರೆ-ದ-ವ-ರಂತೆ
ಬರೆ-ದಿಟ್ಟರು
ಬರೆ-ದಿಡು
ಬರೆ-ದಿಡು-ವು-ದ-ರಿಂದ
ಬರೆ-ದಿದ್ದ
ಬರೆ-ದಿದ್ದರು
ಬರೆ-ದಿದ್ದಾನೆ
ಬರೆ-ದಿದ್ದಾನೆಂಬುದು
ಬರೆ-ದಿದ್ದಾರೆ
ಬರೆ-ದಿ-ರ-ಲಿಲ್ಲ
ಬರೆ-ದಿ-ರು-ವು-ದ-ರಿಂದ
ಬರೆ-ದಿ-ರು-ವುದು
ಬರೆ-ದಿರುವೆ
ಬರೆ-ದಿಲ್ಲ
ಬರೆದು
ಬರೆ-ದು-ಕಳಿ-ಸಿದ್ದರು
ಬರೆ-ದು-ಕೊಂಡ-ವನು
ಬರೆದೆ
ಬರೆ-ಯ-ಬ-ಹುದು
ಬರೆ-ಯ-ಬೇಕು
ಬರೆ-ಯ-ಬೇಕೆಂದು
ಬರೆ-ಯ-ಬೇಕೆಂದೂ
ಬರೆ-ಯಲಿ
ಬರೆ-ಯಲು
ಬರೆ-ಯಲ್ಪಟ್ಟಿತು
ಬರೆ-ಯಲ್ಪಟ್ಟಿದೆ
ಬರೆ-ಯಲ್ಪಟ್ಟಿದ್ದು
ಬರೆ-ಯಲ್ಪಟ್ಟಿವೆ
ಬರೆ-ಯುತ್ತ
ಬರೆ-ಯುತ್ತಾ
ಬರೆ-ಯುತ್ತಾನೆ
ಬರೆ-ಯುತ್ತಾರೆ
ಬರೆ-ಯುತ್ತಿದ್ದಾಗ
ಬರೆ-ಯುತ್ತಿದ್ದಾರೆ
ಬರೆ-ಯುತ್ತಿ-ರ-ಲಿಲ್ಲ
ಬರೆ-ಯುತ್ತಿರುವ
ಬರೆ-ಯುತ್ತಿರು-ವಾಗ
ಬರೆ-ಯುವ
ಬರೆ-ಯು-ವಂತೆಯೂ
ಬರೆ-ಯುವರು
ಬರೆ-ಯುವಷ್ಟ-ರಲ್ಲಿ
ಬರೆ-ಯುವಾಗ
ಬರೆ-ಯು-ವು-ದಕ್ಕೂ
ಬರೆ-ಯು-ವುದಕ್ಕೆ
ಬರೆ-ಯು-ವು-ದ-ರಿಂದ
ಬರೆ-ಯು-ವುದು
ಬರೆ-ಯುವೆ
ಬರೆ-ಯು-ವೆಯಾ
ಬರೆ-ಯೋಣ-ವೆಂದು
ಬರೇ
ಬರ್ಬರ
ಬರ್ಬರ-ರಿಗೆ
ಬರ್ಬ-ರರು
ಬಲ
ಬಲಕಿಂ
ಬಲದ
ಬಲ-ದಿಂದ
ಬಲ-ನಿಧಿಯು
ಬಲ-ಪಡಿ-ಸಲು
ಬಲ-ಪಡಿ-ಸಿಕೊಳ್ಳುವು-ದರ
ಬಲಪ್ರದ-ವಾಗುವ
ಬಲಪ್ರ-ಯೋಗ-ದಿಂದ
ಬಲ-ಭಾಗ-ದಲ್ಲಿ
ಬಲ-ಭುಜದ
ಬಲಯು-ತ-ವಾಗಿ
ಬಲ-ಯುತ-ವಾದುದೊ
ಬಲ-ರಾಮ
ಬಲ-ರಾಮ-ಬಸು
ಬಲ-ರಾಮ-ಬಾಬು
ಬಲ-ರಾಮ-ಬಾಬು-ಗಳ
ಬಲ-ರಾಮ-ವಸು
ಬಲ-ವಂತ
ಬಲ-ವಂತ-ದಿಂದ
ಬಲ-ವಂತ-ವಾಗಿ
ಬಲವತ್ತರ-ವಾಗುತ್ತ
ಬಲ-ವನ್ನು
ಬಲ-ವನ್ನೂ
ಬಲ-ವರ್ಧಕ-ವಾದ
ಬಲ-ವಾಗಿ
ಬಲ-ವಾಗಿದೆ
ಬಲ-ವಾಗಿ-ರಲಿ
ಬಲ-ವಾಗಿಲ್ಲ
ಬಲ-ವಾದ
ಬಲ-ವಿ-ರುವ
ಬಲ-ವಿಲ್ಲ-ದಿದ್ದರೆ
ಬಲವು
ಬಲ-ವೃಂದಂ
ಬಲವೊಂದನ್ನೆ
ಬಲ-ಶಾಲಿ-ಗಳು
ಬಲಶಾಲಿಗೊಳೊ-ಡನೆ
ಬಲಶಾಲಿ-ಯಾಗಿದ್ದರೆ
ಬಲಶಾಲಿ-ಯಾಗಿ-ರ-ಬೇಕು
ಬಲಶಾಲಿ-ಯಾ-ಗು-ವು-ದಿಲ್ಲ
ಬಲಶಾಲಿ-ಯಾಗುವುದು
ಬಲ-ಶಾಲಿ-ಯಾದ
ಬಲಹೀ-ನ-ರನ್ನು
ಬಲ-ಹೀನೇನ
ಬಲಾಢ್ಯನ
ಬಲಾಢ್ಯ-ರನ್ನು
ಬಲಾಢ್ಯರು
ಬಲಾತ್
ಬಲಾತ್ಕರಿ-ಸ-ಬೇಕು
ಬಲಾತ್ಕ-ರಿಸಿ
ಬಲಾತ್ಕರಿ-ಸುತ್ತಾರೆ
ಬಲಾತ್ಕಾರ-ವಾಗಿ
ಬಲಾತ್ಕಾರ-ವಾಗಿ-ಯಾದರೂ
ಬಲಿ
ಬಲಿ-ಕೊಟ್ಟು
ಬಲಿ-ಕೊಡ-ಬೇಕಾಗಿತ್ತು
ಬಲಿ-ಕೊ-ಡಲು
ಬಲಿ-ದಾನ-ಮಾಡು
ಬಲಿಯ
ಬಲಿಷ್ಠ-ಮಾ-ಡಲು
ಬಲಿಷ್ಠ-ರಾಗು-ವುದಕ್ಕೆ
ಬಲಿಷ್ಠರೂ
ಬಲಿಷ್ಠ-ವಾಗು-ವುದು
ಬಲಿಷ್ಠ-ವಾದ
ಬಲು
ಬಲೆ
ಬಲೆಗೆ
ಬಲೆಯ
ಬಲೆ-ಯಲಿ
ಬಲೆ-ಯಲ್ಲಿ
ಬಲೆ-ಯೊ-ಳಗೆ
ಬಲ್ಲ
ಬಲ್ಲರು
ಬಲ್ಲ-ವನೆ
ಬಲ್ಲ-ವರು
ಬಲ್ಲಿರಾ
ಬಲ್ಲೆ
ಬಲ್ಲೆ-ನಾದರೆ
ಬಲ್ಲೆಯ
ಬಲ್ಲೆಯಾ
ಬಲ್ಲೆವೋ
ಬಳಕೆ
ಬಳ-ಕೆಗೆ
ಬಳಕೆ-ಯಲ್ಲಿದ್ದುದು
ಬಳಕೆ-ಯಲ್ಲಿ-ರುವ
ಬಳಕೆ-ಯಲ್ಲಿವೆ
ಬಳಲಿ
ಬಳಲಿದ್ದ
ಬಳಲಿಹೆ
ಬಳಲು-ತಲಿದ್ದೆ
ಬಳ-ಸಲ್ಪಡುತ್ತದೆ
ಬಳಸಿ
ಬಳಸಿ-ಕೊಂಡೇ
ಬಳಸಿ-ರುವ
ಬಳ-ಸುತ್ತೇನೆ
ಬಳ-ಸುವ
ಬಳಿ
ಬಳಿಕ
ಬಳಿಗೆ
ಬಳಿದು
ಬಳಿ-ದು-ಕೊಂಡು
ಬಳಿ-ಯಲ್ಲಿ
ಬಳಿ-ಯಲ್ಲಿದ್ದ
ಬಳಿ-ಯೆಂದಿಗೂ
ಬಳಿ-ಯೊಳೀ
ಬಳುವಳಿ-ಯಾಗಿ
ಬವಣೆ
ಬವಣೆ-ಯರ್ಥವ-ದೇನು
ಬವಣೆ-ಯಲ್ಲದೆ
ಬಸಾ-ಕರ
ಬಸಿದು
ಬಸಿ-ಯುವ
ಬಸು
ಬಸು-ಗಳ
ಬಹ
ಬಹಳ
ಬಹಳ-ಕಾಲ
ಬಹಳ-ಮಟ್ಟಿಗೆ
ಬಹಳ-ವಾಗಿ
ಬಹಳಷ್ಟು
ಬಹಿಃಪ್ರಕಾಶವೇ
ಬಹಿಃಪ್ರವಾಹ-ದಲ್ಲಿ
ಬಹಿರಂಗದ
ಬಹಿರಂಗ-ವಾಗಿ
ಬಹಿರಾ-ವ-ರಣ-ವೆಂಬುದೂ
ಬಹಿರ್ಗತ
ಬಹಿರ್ಗತ-ವಾಗಲು
ಬಹಿರ್ದೃಷ್ಟಿ-ಯಾ-ದಂತೆ
ಬಹಿರ್ಮುಖ
ಬಹಿಷ್ಕರಿ-ಸಿ-ದರು
ಬಹು
ಬಹು-ಕಷ್ಟದ
ಬಹು-ಕಾಲ
ಬಹು-ಕಾಲದ
ಬಹು-ಕಾಲ-ದ-ವರೆಗೆ
ಬಹು-ಕಾಲ-ದಿಂದ
ಬಹು-ಕಾಲ-ದಿಂದಲೂ
ಬಹು-ಕಾಲ-ವಾದ
ಬಹು-ಕೃತಂ
ಬಹು-ಜನ
ಬಹು-ಜನರ
ಬಹು-ಜನ-ರೊಡನೆ
ಬಹು-ತೇ-ಕರು
ಬಹುತ್ವ
ಬಹುತ್ವ-ಭಾವ-ದಿಂದಲೇ
ಬಹು-ದಿನ-ದಿಂದಲೂ
ಬಹು-ದಿವಸ-ದಿಂದಲೂ
ಬಹುದು
ಬಹು-ದೂರ
ಬಹು-ದೂ-ರಕ್ಕೆ
ಬಹು-ದೂರ-ದಿಂದ
ಬಹುಧಾ
ಬಹು-ಪಾಲು
ಬಹುಪ್ರ-ಯತ್ನ
ಬಹು-ಬೇಗ
ಬಹು-ಭಾಗ
ಬಹು-ಮಂದಿ
ಬಹು-ಮಂದಿಗೆ
ಬಹು-ಮಂದಿ-ಯಲ್ಲಿ
ಬಹು-ಮಟ್ಟಿಗೆ
ಬಹು-ರೂಪೆ
ಬಹು-ವಾಗಿ
ಬಹುಶಃ
ಬಹು-ಸಂಖ್ಯಾತ-ರಾಗಿದ್ದ
ಬಹು-ಸಂಖ್ಯಾ-ತರು
ಬಹು-ಹೊತ್ತು
ಬಹೇ
ಬಾ
ಬಾಂದಿ
ಬಾಂಧವ್ಯವು
ಬಾಂಧವ್ಯವೂ
ಬಾಂಶಿ
ಬಾಆ-ದರೆ
ಬಾಗಬ-ಜಾ-ರಿಗೆ
ಬಾಗ-ಬ-ಜಾರಿನ
ಬಾಗ-ಬ-ಜಾರಿ-ನಲ್ಲಿ
ಬಾಗ-ಬ-ಜಾರಿ-ನಲ್ಲಿ-ರುವ
ಬಾಗಲಿ
ಬಾಗಿ
ಬಾಗಿ-ದರು
ಬಾಗಿಲ
ಬಾಗಿ-ಲ-ನಲ್ಲಿ-ಹುದು
ಬಾಗಿ-ಲನ್ನು
ಬಾಗಿ-ಲಲ್ಲಿ
ಬಾಗಿ-ಲಲ್ಲಿಯೇ
ಬಾಗಿ-ಲಿಗೂ
ಬಾಗಿ-ಲಿಗೆ
ಬಾಗಿ-ಲಿನ
ಬಾಗಿ-ಲಿ-ನಲ್ಲಿ
ಬಾಗಿ-ಲಿ-ನಿಂದ
ಬಾಗಿ-ಲಿರ-ಬ-ಹುದು
ಬಾಗಿಲು
ಬಾಗಿ-ಲು-ಗಳೇ
ಬಾಗುವ
ಬಾಗ್-ಬ-ಜಾರಿನ
ಬಾಗ್-ಬ-ಜಾರಿ-ನಲ್ಲಿ
ಬಾಗ್ಬ-ಜಾರ್
ಬಾಚಿ
ಬಾಜಾರಮೇ
ಬಾಜೆ
ಬಾಜೇ
ಬಾಡಿಗೆ
ಬಾಡಿಗೆಗೆ
ಬಾಡಿಗೆಯ
ಬಾಡಿಗೆ-ಯನ್ನು
ಬಾಣ-ಗಳನ್ನು
ಬಾತು-ಕೊಂಡರೆ
ಬಾದಾ
ಬಾಧ-ಕ-ವಾಗಿದೆ
ಬಾಧ-ಕ-ವಿಲ್ಲ
ಬಾಧಕ-ವೇನು
ಬಾಧೆ-ಯಿಲ್ಲ
ಬಾನಾಡಿಯೇ
ಬಾನೆತ್ತರ
ಬಾನೊಳಾಡುವ
ಬಾನ್
ಬಾಪ್ಪಾ
ಬಾಬ-ನಿಂದ
ಬಾಬಾ
ಬಾಬಾ-ಜಿ-ಗಳಿಂದ
ಬಾಬಾ-ರೊಡನೆ
ಬಾಬು
ಬಾಬು-ಗಳ
ಬಾಬು-ಗಳಂತಹ-ವರೆ
ಬಾಬು-ಗಳಿಗೆ
ಬಾಬು-ಗಳು
ಬಾಬು-ಗಳೂ
ಬಾಬು-ಗಳೊ-ಡನೆ
ಬಾಬು-ವಿನ
ಬಾಬುವೇ
ಬಾಬೂ
ಬಾಬೂ-ರಾಮ
ಬಾಯಲಿ
ಬಾಯಲ್ಲಿ
ಬಾಯಲ್ಲಿಯೂ
ಬಾಯಲ್ಲಿಯೇ
ಬಾಯಾರಿ
ಬಾಯಾರಿಕೆ
ಬಾಯಾರಿ-ಕೆ-ಯನ್ನು
ಬಾಯಾರಿ-ಕೆ-ಯಾಗಲು
ಬಾಯಾರಿ-ಕೆ-ಯಾಗಿ
ಬಾಯಾರಿ-ಕೆ-ಯಿಂದಲೇ
ಬಾಯಿಂದ
ಬಾಯಿಂದಲೆ
ಬಾಯಿಂದಲೇ
ಬಾಯಿಗೆ
ಬಾಯಿ-ತೆರೆದು
ಬಾಯಿ-ಪಾಠ-ದಂತೆ
ಬಾಯಿ-ಬಿಟ್ಟು
ಬಾಯಿ-ಯಿಂದ
ಬಾಯಿ-ಯಿಂದಲೆ
ಬಾಯ್ದೆರೆದಾ
ಬಾರಂಬಾರ
ಬಾರದ
ಬಾರ-ದಂತೆ
ಬಾರದ-ವರು
ಬಾರ-ದಿದ್ದರೆ
ಬಾರ-ದಿ-ರಲು
ಬಾರ-ದಿ-ಹುದು
ಬಾರದು
ಬಾರದುದು
ಬಾರದ್ದು
ಬಾರಯ್ಯ
ಬಾರಾ-ನಗರ
ಬಾರಾ-ನಗರದ
ಬಾರಿ
ಬಾರಿಗೆ
ಬಾರಿ-ಬಾರಿಗು
ಬಾರಿ-ಬಾರಿಗೂ
ಬಾರಿಯೂ
ಬಾರಿ-ಸ-ಬೇಕು
ಬಾರಿ-ಸಲು
ಬಾರಿ-ಸಿತು
ಬಾರಿ-ಸಿದ
ಬಾರಿ-ಸಿ-ದಂತೆ
ಬಾರಿ-ಸಿ-ದರು
ಬಾರಿ-ಸುತ್ತ
ಬಾರಿ-ಸುತ್ತಾ
ಬಾರಿ-ಸುತ್ತಿದ್ದಾಗ
ಬಾರಿ-ಸು-ವ-ವನು
ಬಾರೊ
ಬಾಲ
ಬಾಲ-ಬುದ್ಧಿ-ಯ-ವರು
ಬಾಲ-ಭಾಸಾ
ಬಾಲ-ಸಂನ್ಯಾಸಿ-ಗಳ
ಬಾಲಿ-ಕೆಯರ
ಬಾಲಿಕೆ-ಯ-ರಿಗೆ
ಬಾಲಿ-ಯಿಂದ
ಬಾಲ್ಯ
ಬಾಲ್ಯ-ಕಾಲದ
ಬಾಲ್ಯ-ಕಾಲ-ದಲ್ಲಿಯೇ
ಬಾಲ್ಯ-ದಲ್ಲಿ
ಬಾಲ್ಯ-ದಿಂದ
ಬಾಲ್ಯ-ದಿಂದಲೂ
ಬಾಲ್ಯ-ದಿಂದಲೇ
ಬಾಲ್ಯ-ವಿವಾಹ
ಬಾಲ್ಯ-ವಿವಾಹದ
ಬಾಲ್ಯ-ವಿವಾಹ-ದಿಂದ
ಬಾಲ್ಯ-ವಿವಾಹ-ವನ್ನು
ಬಾಲ್ಯ-ವಿವಾಹವು
ಬಾಲ್ಯಾರಭ್ಯ
ಬಾಳ
ಬಾಳ-ದಿಹ
ಬಾಳನ್ನು
ಬಾಳ-ಬೇಕು
ಬಾಳ-ಬೇಕೆನ್ನುವ-ವರಿ-ಗಷ್ಟೇ
ಬಾಳ-ಮೋ-ಹವು
ಬಾಳಲಿ
ಬಾಳಿ
ಬಾಳಿ-ದರು
ಬಾಳಿನ
ಬಾಳಿ-ನುದ್ದಕು
ಬಾಳಿ-ನುರಿ
ಬಾಳು
ಬಾಳುಗೆ
ಬಾಳುವ
ಬಾಳು-ವರು
ಬಾಳು-ವು-ದ-ರಲ್ಲೇ
ಬಾಳು-ವುದು
ಬಾಳೆಯ
ಬಾಳೆಯ-ಕಂಬ
ಬಾಳ್ಗೆ
ಬಾಸ್ಟನ್
ಬಾಹಿರಂತರ
ಬಾಹ್ಯ
ಬಾಹ್ಯದ
ಬಾಹ್ಯ-ದಲ್ಲಿ
ಬಾಹ್ಯ-ದೃಷ್ಟಿ-ಯಿಂದ
ಬಾಹ್ಯಪ್ರಜ್ಞೆ
ಬಾಹ್ಯಪ್ರಜ್ಞೆ-ತಾಳುವ
ಬಾಹ್ಯ-ರೂಪ
ಬಾಹ್ಯ-ರೂಪದ
ಬಾಹ್ಯ-ರೂಪ-ದಲ್ಲಿ
ಬಾಹ್ಯ-ರೂಪವು
ಬಾಹ್ಯ-ವಿಧಿ-ಗಳನ್ನ-ನು-ಸರಿಸ-ಬೇಕು
ಬಾಹ್ಯ-ಸಹಾಯ-ವನ್ನು
ಬಾಹ್ಯಾಚ-ರಣೆ-ಗಳ
ಬಾಹ್ಯಾಚ-ರಣೆ-ಗಳನ್ನನು-ಸರಿಸ-ದಿರುವ
ಬಾಹ್ಯಾಚ-ರಣೆ-ಗಳನ್ನಾಚ-ರಿ-ಸದೆ
ಬಾಹ್ಯಾಚ-ರಣೆಯ
ಬಾಹ್ಯಾವ-ಲಂಬನ-ವನ್ನೇ
ಬಾಹ್ಯಾವ-ಲಂಬ-ನವೆ
ಬಿಂದು
ಬಿಂದು-ವಿ-ನಲಿ
ಬಿಂದು-ವಿ-ನಿಂದಲೇ
ಬಿಂಬ
ಬಿಂಬ-ಸಾರನು
ಬಿಂಬಾ-ಧರ-ಗಳೊ
ಬಿಂಬಾ-ಪತ್ರ
ಬಿಎ
ಬಿಗಿ
ಬಿಗಿ-ದ-ನೆಂದುಕೋ
ಬಿಗಿ-ದಿಹ
ಬಿಗಿ-ದೆಳೆ-ಗಳವು
ಬಿಗಿ-ಯಂಗಿ
ಬಿಗಿ-ಯಲು
ಬಿಗಿ-ಯಾಗಿ
ಬಿಗಿ-ಯಾದ
ಬಿಗಿ-ಯಿಲ್ಲದೆ
ಬಿಗಿ-ಯುತಿರೆ
ಬಿಗಿ-ವರಿರಿ-ವರು
ಬಿಗಿ-ಹಿಡಿದ
ಬಿಚ್ಚಿ
ಬಿಚ್ಚಿ-ಕೊಳ್ಳುತ್ತಾ
ಬಿಚ್ಚಿ-ಡು-ತಿ-ರುವೆ
ಬಿಚ್ಚಿಹ
ಬಿಚ್ಚಿ-ಹಾಕಿ-ದರು
ಬಿಜಲಿ
ಬಿಜಲಿಜ್ವಾಲಾ
ಬಿಟ್ಟ
ಬಿಟ್ಟಂತಾಗಿ
ಬಿಟ್ಟಂತಾಗುತ್ತಿತ್ತು
ಬಿಟ್ಟನು
ಬಿಟ್ಟರು
ಬಿಟ್ಟರೆ
ಬಿಟ್ಟ-ರೇನು
ಬಿಟ್ಟಿತು
ಬಿಟ್ಟಿತ್ತು
ಬಿಟ್ಟಿದ್ದರು
ಬಿಟ್ಟಿದ್ದರೆ
ಬಿಟ್ಟಿದ್ದಾರೆ
ಬಿಟ್ಟಿ-ರುವುದು
ಬಿಟ್ಟಿಲ್ಲ
ಬಿಟ್ಟಿವೆ
ಬಿಟ್ಟು
ಬಿಟ್ಟು-ಕೊಂಡಿದ್ದಾರೆ
ಬಿಟ್ಟು-ಕೊಂಡಿರ-ಬೇಕು
ಬಿಟ್ಟು-ಕೊಂಡು
ಬಿಟ್ಟು-ಕೊಟ್ಟರು
ಬಿಟ್ಟು-ಕೊಡುತ್ತೇನೆ
ಬಿಟ್ಟು-ದರ
ಬಿಟ್ಟು-ಬಂದು
ಬಿಟ್ಟು-ಬಿಟ್ಟರೆ
ಬಿಟ್ಟು-ಬಿಟ್ಟೆ
ಬಿಟ್ಟು-ಬಿಡ-ದಿದ್ದರೆ
ಬಿಟ್ಟು-ಬಿಡಿ
ಬಿಟ್ಟು-ಬಿಡು
ಬಿಟ್ಟು-ಹೋಗು
ಬಿಟ್ಟು-ಹೋಗು-ವಂತಿದ್ದರೆ
ಬಿಟ್ಟು-ಹೋದನು
ಬಿಟ್ಟೆ
ಬಿಟ್ಟೆದ್ದನು
ಬಿಡದ
ಬಿಡ-ದಿದ್ದರೆ
ಬಿಡ-ದಿರು
ಬಿಡದೆ
ಬಿಡ-ಬಲ್ಲೆ-ಯಾ-ದರೆ
ಬಿಡ-ಬೇಕಾ-ಗುತ್ತದೆ
ಬಿಡ-ಬೇಕು
ಬಿಡ-ಬೇಕೆಂದು
ಬಿಡ-ಬೇಡ
ಬಿಡ-ಬೇಡಿ
ಬಿಡ-ಲಾ-ಯಿತು
ಬಿಡ-ಲಾರೆಯೊ
ಬಿಡಲಿ
ಬಿಡ-ಲಿಲ್ಲ
ಬಿಡಲು
ಬಿಡಿ
ಬಿಡಿ-ಕಾಸನ್ನೂ
ಬಿಡಿ-ಗಾಸೂ
ಬಿಡಿ-ಸಿ-ಕೊಡ-ಬೇಕು
ಬಿಡಿ-ಸಿ-ದಷ್ಟೂ
ಬಿಡಿಸು
ಬಿಡಿ-ಸು-ವುದಕ್ಕೆ
ಬಿಡಿ-ಸು-ವುದು
ಬಿಡು
ಬಿಡು-ಗಡೆ
ಬಿಡು-ಗಡೆ-ಗಿದು
ಬಿಡು-ಗಡೆಯ
ಬಿಡು-ಗಣ್ಣಿ-ನಿಂದ
ಬಿಡುತ್ತದೆಯೋ
ಬಿಡುತ್ತಾನೆ
ಬಿಡುತ್ತಾರೆ
ಬಿಡುತ್ತಾರೆಯೋ
ಬಿಡುತ್ತಿದ್ದ-ರಂತೆ
ಬಿಡುತ್ತಿದ್ದರು
ಬಿಡುತ್ತಿದ್ದವು
ಬಿಡುತ್ತಿದ್ದೆ
ಬಿಡುತ್ತಿ-ರ-ಲಿಲ್ಲ
ಬಿಡುತ್ತಿಲ್ಲ
ಬಿಡುತ್ತೀಯೆ
ಬಿಡುತ್ತೇನೆ
ಬಿಡು-ಮುಡಿಗೂ-ದಲ
ಬಿಡು-ವಂಥ-ವ-ರಲ್ಲ
ಬಿಡು-ವನು
ಬಿಡು-ವರು
ಬಿಡು-ವ-ಹಾಗೆ
ಬಿಡು-ವಾಗಿತ್ತು
ಬಿಡು-ವಿರಿ
ಬಿಡು-ವಿಲ್ಲ
ಬಿಡುವು
ಬಿಡು-ವುದಕ್ಕೆ
ಬಿಡು-ವು-ದನ್ನು
ಬಿಡು-ವು-ದ-ರಂತೆ
ಬಿಡು-ವು-ದ-ರಿಂದ
ಬಿಡು-ವು-ದಿಲ್ಲ
ಬಿಡು-ವು-ದಿಲ್ಲೆಂಬ
ಬಿಡು-ವುದೆ
ಬಿಡು-ವು-ದೆಂದರೆ
ಬಿಡು-ವು-ದೆಂದೂ
ಬಿಡು-ವು-ದೇನೋ
ಬಿಡು-ವು-ದೊಂದು
ಬಿಡೋಣ
ಬಿತ್ತ-ರದ
ಬಿತ್ತ-ರಿಸಿ
ಬಿತ್ತಲು
ಬಿತ್ತಿ
ಬಿತ್ತಿ-ದರೆ
ಬಿತ್ತಿ-ದಾತನು
ಬಿತ್ತು
ಬಿದಿಗೆ-ಯಂದು
ಬಿದಿಯು
ಬಿದ್ದ
ಬಿದ್ದಂತೆ
ಬಿದ್ದರೆ
ಬಿದ್ದಾಗ
ಬಿದ್ದಿದೆ
ಬಿದ್ದಿದ್ದಾರೆ
ಬಿದ್ದಿದ್ದೀರಿ
ಬಿದ್ದಿದ್ದು
ಬಿದ್ದಿ-ರ-ಲಿಲ್ಲ
ಬಿದ್ದಿರುತ್ತಿದ್ದೆನೊ
ಬಿದ್ದಿ-ರುವ
ಬಿದ್ದಿ-ರುವರು
ಬಿದ್ದಿ-ರು-ವು-ದೆಂದು
ಬಿದ್ದಿವೆ
ಬಿದ್ದಿ-ಹುದು
ಬಿದ್ದು
ಬಿದ್ದೊಡ-ನೆಯೇ
ಬಿನ್ನ-ವತ್ತಳೆಗೆ
ಬಿಮ್ಮೆನ್ನುತ್ತಿದೆ
ಬಿರಿಯು-ತಿದೆ
ಬಿರು-ಗಾಳಿ
ಬಿರು-ಗಾಳಿಗೆ
ಬಿಲ್ಲು
ಬಿಲ್ವ
ಬಿಲ್ವ-ಮಂಗಲನ
ಬಿಲ್ವ-ಮೂಲ-ಗಳನ್ನು
ಬಿಲ್ವ-ವೃಕ್ಷದ
ಬಿಳಿಯ
ಬಿಳಿ-ಯನು
ಬಿಳಿ-ಯರು
ಬಿಸಾಕಿದ
ಬಿಸಾಡಿ
ಬಿಸಾಡು
ಬಿಸಿ
ಬಿಸಿದು
ಬಿಸಿಯೋ
ಬಿಸಿ-ಲಿನ
ಬಿಸಿಲು
ಬಿಸುಟು
ಬಿಸುಡು
ಬಿಸುಡೈ
ಬಿಸ್ಕತ್ತು
ಬೀಗ-ಮುದ್ರೆ
ಬೀಜ
ಬೀಜ-ಗಳು
ಬೀಜದ
ಬೀಜ-ದಂತೆಯೆ
ಬೀಜ-ರೂಪಕ್ಕೆ
ಬೀಜ-ರೂಪ-ದಲ್ಲಿ
ಬೀಜ-ವನ್ನು
ಬೀಜವು
ಬೀಜಾಂಕುರ
ಬೀಡನ್
ಬೀದಿಧೂಳೂ
ಬೀದಿ-ಯಲ್ಲಿ
ಬೀರಿ
ಬೀರಿದ್ದಕ್ಕೆ
ಬೀರುತ
ಬೀರುತ್ತದೆ
ಬೀರುತ್ತವೆ
ಬೀರುತ್ತಿತ್ತು
ಬೀರುತ್ತಿದ್ದರು
ಬೀರುತ್ತಿರುವ
ಬೀರುವನು
ಬೀರು-ವುದು
ಬೀಳದಿರ-ಬಹುದೋ
ಬೀಳ-ಬ-ಹುದು
ಬೀಳ-ಬೇಕು
ಬೀಳ-ಲಾರೆಯೊ
ಬೀಳಲಿ
ಬೀಳಿಸಿ
ಬೀಳುತ
ಬೀಳು-ತಿದೆ
ಬೀಳುತೇಳುತ
ಬೀಳುತ್ತದೆ
ಬೀಳುತ್ತಿತ್ತು
ಬೀಳುತ್ತಿದೆ
ಬೀಳುತ್ತಿ-ರಲು
ಬೀಳುತ್ತೇವೆ
ಬೀಳುವ
ಬೀಳು-ವನು
ಬೀಳು-ವರು
ಬೀಳುವ-ವ-ರೆಗೂ
ಬೀಳು-ವಿರಿ
ಬೀಳು-ವುದು
ಬೀಳು-ವುವೋ
ಬೀಳುವೆ
ಬೀಳ್ಕೊಟ್ಟ
ಬೀಳ್ಕೊಟ್ಟರು
ಬೀಸ-ಣಿಗೆ-ಯನ್ನು
ಬೀಸ-ತೊಡಗಿತು
ಬೀಸಿ
ಬೀಸುತ್ತಲೇ
ಬೀಸುತ್ತಿದೆ
ಬೀಸುತ್ತಿದ್ದು-ದ-ರಿಂದ
ಬೀಸು-ವುದು
ಬುಗುರಿ-ಯೊಲು
ಬುಗ್ಗೆ-ಯಂತೆ
ಬುಗ್ಗೆ-ಯಾಳವನ್ನಳೆ-ಯುವರು
ಬುಟ್ಟಿ-ಯಲ್ಲಿ
ಬುಡದಲ್ಲಿಯೆ
ಬುದ್ದ
ಬುದ್ದಿ-ವಂತ
ಬುದ್ಧ
ಬುದ್ಧತ್ವ
ಬುದ್ಧ-ದೇವ
ಬುದ್ಧ-ದೇವನ
ಬುದ್ಧ-ದೇವ-ನಂತೂ
ಬುದ್ಧ-ದೇವ-ನನ್ನು
ಬುದ್ಧ-ದೇವ-ನಿಂದ
ಬುದ್ಧ-ದೇವ-ನಿಂದಲೇ
ಬುದ್ಧ-ದೇವ-ನಿಗಾ-ದರೋ
ಬುದ್ಧ-ದೇವನು
ಬುದ್ಧ-ದೇವನೇ
ಬುದ್ಧ-ದೇವ-ನೊಬ್ಬನೆ
ಬುದ್ಧನ
ಬುದ್ಧ-ನಂಥ
ಬುದ್ಧ-ನನ್ನು
ಬುದ್ಧ-ನಲ್ಲಿ
ಬುದ್ಧ-ನಿಂದಾದ
ಬುದ್ಧ-ನಿ-ಗಿಂತ
ಬುದ್ಧ-ನಿ-ಗಿಂತಲೂ
ಬುದ್ಧ-ನಿಗೆ
ಬುದ್ಧನು
ಬುದ್ಧ-ಪೂರ್ವದ
ಬುದ್ಧಿ
ಬುದ್ಧಿಂ
ಬುದ್ಧಿ-ಗದೇಕೊ
ಬುದ್ಧಿ-ಪೂರ್ವ-ಕ-ವಾದ
ಬುದ್ಧಿ-ಮಾನ್
ಬುದ್ಧಿಯ
ಬುದ್ಧಿ-ಯನ್ನಿಟ್ಟು-ಕೊಂಡು
ಬುದ್ಧಿ-ಯನ್ನು
ಬುದ್ಧಿ-ಯಲ್ಲ
ಬುದ್ಧಿ-ಯಿಂದ
ಬುದ್ಧಿ-ಯಿಂದಲೇ
ಬುದ್ಧಿ-ಯಿಂದಲೊ
ಬುದ್ಧಿ-ಯಿಟ್ಟು-ಕೊಂಡಿದ್ದನು
ಬುದ್ಧಿ-ಯಿಲ್ಲ-ದ-ವರು
ಬುದ್ಧಿಯು
ಬುದ್ಧಿ-ಯುಳ್ಳ
ಬುದ್ಧಿ-ಯುಳ್ಳ-ವರು
ಬುದ್ಧಿಯೂ
ಬುದ್ಧಿಯೇ
ಬುದ್ಧಿ-ವಂತ-ನಾದ
ಬುದ್ಧಿ-ವಂತ-ನೆನಿ-ಸಿ-ಕೊಂಡ-ವ-ನಲ್ಲ
ಬುದ್ಧಿ-ವಂತ-ರಾದ
ಬುದ್ಧಿ-ವಂತ-ರಿಗೂ
ಬುದ್ಧಿ-ವಂತರೂ
ಬುದ್ಧಿ-ವಾದ-ವನ್ನು
ಬುದ್ಧಿ-ಶಕ್ತಿ-ಯಲ್ಲಿ
ಬುದ್ಧಿ-ಸಂಪನ್ನ-ರಾಗಿ
ಬುದ್ಧಿಸ್ವಾ-ಧೀನ-ತೆ-ಯನ್ನು
ಬುದ್ಧಿ-ಹೀನ-ರಾದ
ಬುಸುಗುಟ್ಟಿದೆ
ಬೂಕೆ
ಬೂದಿ
ಬೃಂದಾ-ವನ
ಬೃಂದಾವ-ನಕ್ಕೆ
ಬೃಂದಾ-ವನದ
ಬೃಂದಾ-ವನ-ದಲ್ಲಿ
ಬೃಉ
ಬೃಹತ್
ಬೆಂಕಿ
ಬೆಂಕಿಯ
ಬೆಂಕಿ-ಯಂತಿರುತ್ತಾರೆ
ಬೆಂಕಿ-ಯಂತೆ
ಬೆಂಕಿ-ಯನು
ಬೆಂಕಿ-ಯನ್ನು
ಬೆಂಕಿ-ಯಲ್ಲಿ
ಬೆಂಕಿ-ಯಲ್ಲೇ
ಬೆಂಕಿ-ಯಾಗಿ-ರದೆ
ಬೆಂಕಿಯು
ಬೆಂಗಾಡು
ಬೆಂಚಿನ
ಬೆಂಡಾಗಿ
ಬೆಂಡಾದ
ಬೆಂದ
ಬೆಂದು
ಬೆಂಬಿ-ಡದೆ
ಬೆಕ್ಕಿಗೆ
ಬೆಕ್ಕು
ಬೆಕ್ಕು-ಗಳಂತೆ
ಬೆಚ್ಚ-ಗಿ-ರುವ
ಬೆಟ್ಟ
ಬೆಟ್ಟ-ಗಳ
ಬೆಟ್ಟ-ಗಳಲ್ಲಿ
ಬೆಟ್ಟ-ಗಳಲ್ಲೇ-ನನ್ನು
ಬೆಟ್ಟ-ಗಳಿಂದ
ಬೆಟ್ಟ-ಗಳು
ಬೆಟ್ಟ-ಗುಡ್ಡ-ಗಳಲ್ಲಿ
ಬೆಟ್ಟ-ಗುಡ್ಡ-ಗಳಲ್ಲಿಯೂ
ಬೆಟ್ಟದ
ಬೆಡಗಿ
ಬೆಣ್ಣೆ
ಬೆತ್ತಲೆ-ಯಾಗಿ
ಬೆನ್ನ
ಬೆನ್ನಟ್ಟಿ
ಬೆನ್ನಟ್ಟಿ-ದರು
ಬೆನ್ನಿನ
ಬೆನ್ನುಳ್ಳದ್ದಾಗಿದೆ
ಬೆನ್ನೆಡೆ
ಬೆರಗಾ-ಗದೆ
ಬೆರ-ಗಾಗಿ
ಬೆರ-ಗಾಗಿದ್ದನು
ಬೆರ-ಗಾಗಿ-ಬಿಟ್ಟೆ
ಬೆರಗಾಗುತ್ತಾರೆ
ಬೆರಗಾಗುತ್ತಿದ್ದರು
ಬೆರಗಾಗುತ್ತಿದ್ದರೊ
ಬೆರಗಾಗು-ವಂತಹ
ಬೆರಗಾಗು-ವರು
ಬೆರಗಾಗು-ವುದು
ಬೆರಗಾ-ದನು
ಬೆರ-ಗಾ-ಯಿತು
ಬೆರ-ಳನ್ನು
ಬೆರಳು
ಬೆರಳು-ಮಾಡಿ
ಬೆರೆತ
ಬೆರೆ-ತರೂ
ಬೆರೆ-ತರೆ
ಬೆರೆ-ತವು
ಬೆರೆ-ತಾಗ
ಬೆರೆತು
ಬೆರೆ-ಯದಿ-ರು-ವಂತೆ
ಬೆರೆಯ-ಬಾ-ರದೋ
ಬೆರೆಯ-ಬೇಕು
ಬೆರೆ-ಯ-ಬೇಕೆಂಬುದು
ಬೆರೆ-ಯಿರಿ
ಬೆರೆ-ಯುತ್ತಿದ್ದರು
ಬೆರೆ-ಸಿ-ಕೊಂಡಿದ್ದಲ್ಲದೆ
ಬೆಲೆ
ಬೆಲೆ-ಕೊಡು-ವು-ದಿಲ್ಲ
ಬೆಲೆ-ಯನ್ನ-ರಿತು
ಬೆಲೆ-ಯರಿ-ವುದೆಂದಿಗೋ
ಬೆಲೆ-ಯಿದೆ-ಯೆಂಬುದು
ಬೆಲೆಯೂ
ಬೆಳಕ
ಬೆಳ-ಕದು
ಬೆಳ-ಕನ್ನು
ಬೆಳ-ಕನ್ನೂ
ಬೆಳಕಿಗೆ
ಬೆಳಕಿದು
ಬೆಳ-ಕಿನ
ಬೆಳಕಿ-ನಂತೆ
ಬೆಳ-ಕಿ-ನಲ್ಲಿ
ಬೆಳ-ಕಿನಲ್ಲಿಯೆ
ಬೆಳ-ಕಿನಾ
ಬೆಳ-ಕಿ-ನಿಂದ
ಬೆಳಕೀ-ಯುವರು
ಬೆಳಕೀ-ವುದು
ಬೆಳಕು
ಬೆಳಕೆಂದು
ಬೆಳಕೆಲ್ಲ
ಬೆಳಗಿ
ಬೆಳ-ಗಿನ
ಬೆಳಗಿ-ನಲಿ
ಬೆಳಗು-ತಲಿ-ರು-ವುವು
ಬೆಳಗು-ತಿ-ರಲು
ಬೆಳಗು-ತಿಹ
ಬೆಳಗು-ತಿ-ಹುದು
ಬೆಳಗುತ್ತಿ-ರು-ವುದು
ಬೆಳಗುತ್ತಿವೆ
ಬೆಳಗುವ
ಬೆಳಗು-ವನು
ಬೆಳಗ್ಗೆ
ಬೆಳಗ್ಗೆ-ಯಾ-ಗಲಿ
ಬೆಳದು-ಬಂದಿದೆ
ಬೆಳಯ
ಬೆಳವ-ಣಿಗೆ
ಬೆಳವ-ಣಿಗೆ-ಗಳನ್ನು
ಬೆಳವ-ಣಿಗೆ-ಗಳಷ್ಟೆ
ಬೆಳವ-ಣಿಗೆಗೂ
ಬೆಳವ-ಣಿಗೆಗೆ
ಬೆಳವ-ಣಿಗೆಯ
ಬೆಳವ-ಣಿಗೆ-ಯಾಗುವುದೋ
ಬೆಳ-ಸಿದ್ದು-ವಲ್ಲಿ
ಬೆಳಿಗ್ಗೆ
ಬೆಳುಳ್ಳಿ-ಯಂತಹ
ಬೆಳೆ-ಗಳು
ಬೆಳೆದ
ಬೆಳೆ-ದರೆ
ಬೆಳೆದ-ವ-ರಾಗಿ-ರುತ್ತಾರೆ
ಬೆಳೆದು
ಬೆಳೆದು-ಕೊಂಡಿತ್ತು
ಬೆಳೆ-ಬೆಳೆದು
ಬೆಳೆಯ-ಬೇಕಾಗಿದೆ
ಬೆಳೆಯ-ಬೇಕಾ-ಗುತ್ತದೆ
ಬೆಳೆಯ-ಬೇಕು
ಬೆಳೆಯ-ಲಾ-ಗದೆ
ಬೆಳೆ-ಯುತಿತ್ತು
ಬೆಳೆಯು-ತಿ-ರಲು
ಬೆಳೆ-ಯುತ್ತ
ಬೆಳೆ-ಯುತ್ತವೆ
ಬೆಳೆ-ಯುತ್ತಾರೆ
ಬೆಳೆ-ಯುವ
ಬೆಳೆ-ಯು-ವಂತೆ
ಬೆಳೆ-ಯು-ವುದು
ಬೆಳೆಯು-ವು-ದೆಂದು
ಬೆಳೆಸ-ಬ-ಹುದು
ಬೆಳೆಸ-ಬೇ-ಕಾದರೆ
ಬೆಳೆಸ-ಬೇಕು
ಬೆಳೆಸಿ-ಕೊಳ್ಳ-ತೊಡಗಿದ್ದಾರೆಂದು
ಬೆಳೆ-ಸಿ-ದಂತೆ
ಬೆಳೆ-ಸಿ-ದನು
ಬೆಳೆ-ಸಿದರೆ
ಬೆಳೆ-ಸು-ವುದಕ್ಕೆ
ಬೆಳೆ-ಸು-ವು-ದರ
ಬೆಳೆ-ಸು-ವು-ದಿಲ್ಲ
ಬೆಳ್ಳ-ಗಿ-ರುವ
ಬೆಳ್ಳಗೆ
ಬೆಳ್ಳಿಯ
ಬೆಸುಗೆ-ಗೊಂಡು
ಬೆಸು-ಗೆಯ
ಬೆಸೆ-ದಿ-ರು-ವುದು
ಬೆಸೆ-ದುಕೊಂಡಿಹು-ದಿಲ್ಲಿ
ಬೆಸ್ತ
ಬೆಸ್ತ-ನಾಗಿಯೆ
ಬೇಕಂತಲೇ
ಬೇಕಲ್ಲವೆ
ಬೇಕಾಗ-ಬ-ಹುದು
ಬೇಕಾಗಿದೆ
ಬೇಕಾಗಿದ್ದ-ರಿಂದ
ಬೇಕಾಗಿದ್ದರೆ
ಬೇಕಾಗಿದ್ದಾರೆ
ಬೇಕಾಗಿ-ರುವರು
ಬೇಕಾಗಿ-ರು-ವುದು
ಬೇಕಾಗಿಲ್ಲ
ಬೇಕಾ-ಗುತ್ತದೆ
ಬೇಕಾಗುವ
ಬೇಕಾಗು-ವುದು
ಬೇಕಾದ
ಬೇಕಾ-ದದ್ದು
ಬೇಕಾ-ದರೂ
ಬೇಕಾದರೆ
ಬೇಕಾದ-ವನು
ಬೇಕಾ-ದಷ್ಟು
ಬೇಕಾ-ದು-ದನ್ನು
ಬೇಕಾದುದನ್ನೆಲ್ಲಾ
ಬೇಕಾದುದ-ರಲ್ಲಿ
ಬೇಕಾದ್ದೇ-ನೆಂದರೆ
ಬೇಕಿತ್ತು
ಬೇಕಿಲ್ಲ
ಬೇಕಿಲ್ಲದೆ
ಬೇಕು
ಬೇಕೆಂದು
ಬೇಕೆಂದೇ
ಬೇಕೆ-ನಿ-ಸಿದರೆ
ಬೇಕೇನು
ಬೇಕೊ
ಬೇಕೋ
ಬೇಗ
ಬೇಗನೆ
ಬೇಡ
ಬೇಡ-ಬೇಕು
ಬೇಡ-ವೆಂದು
ಬೇಡ-ವೆಂದೆ-ನಿ-ಸಿದರೆ
ಬೇಡ-ವೆಂದೆ-ನಿಸಿ-ದೊಡ-ನೆಯೇ
ಬೇಡಿ-ಕೆ-ಗಳನ್ನು
ಬೇಡಿ-ಕೆ-ಗಳಿಗೂ
ಬೇಡಿ-ಕೆ-ಗಳು
ಬೇಡಿ-ಕೆ-ಗಳೊಂದೂ
ಬೇಡಿ-ಕೊಳ್ಳುತ್ತಿದ್ದೇನೆ
ಬೇಡಿ-ತದು
ಬೇಡಿದ
ಬೇಡಿ-ದನು
ಬೇಡು-ತಿದ್ದೆ
ಬೇಡುತ್ತಿದ್ದರು
ಬೇಡುವು-ದಾಗಿ
ಬೇಡುವೆ
ಬೇನೆ-ಗಳ
ಬೇಯಿಸಿ
ಬೇಯಿ-ಸಿದ
ಬೇರಾವ
ಬೇರಾ-ವು-ದಕ್ಕೂ
ಬೇರಾ-ವುದೂ
ಬೇರಾ-ವುದೋ
ಬೇರಿನ್ನಾವ
ಬೇರು-ಬಿಟ್ಟಿ-ರು-ವು-ದನ್ನು
ಬೇರುಸ-ಹಿತ
ಬೇರೂ-ರುವ
ಬೇರೂ-ರು-ವಂತೆ
ಬೇರೂ-ರು-ವು-ದಿಲ್ಲವೋ
ಬೇರೆ
ಬೇರೆ-ಬೇರೆ-ಯಲ್ಲ
ಬೇರೆ-ಬೇರೆ-ಯೆಂಬು-ದನ್ನು
ಬೇರೆ-ಬೇರೆಯೇ
ಬೇರೆ-ಬೇರೆ-ಯೇನು
ಬೇರೆಯ
ಬೇರೆ-ಯಲ್ಲ
ಬೇರೆ-ಯ-ವರ
ಬೇರೆ-ಯಾಗಿ
ಬೇರೆ-ಯಾಗಿಟ್ಟು-ಕೊಳ್ಳಲು
ಬೇರೆ-ಯಾ-ಗಿದೆ
ಬೇರೆ-ಯಾಗಿದ್ದರು
ಬೇರೆ-ಯಾಗಿ-ರು-ವು-ದನ್ನು
ಬೇರೆ-ಯಾಗಿ-ರು-ವುದು
ಬೇರೆ-ಯಾದ
ಬೇರೆ-ಯಾದವು
ಬೇರೆಯೊ
ಬೇರೆಲ್ಲಿಯೂ
ಬೇರೇ-ನನ್ನೂ
ಬೇರೇ-ನಲ್ಲ
ಬೇರೇನೂ
ಬೇರೇನೋ
ಬೇರೊಂದು
ಬೇರೊಬ್ಬ
ಬೇರ್ಪಡಿ-ಸಲಾ-ಗದಂತೆ
ಬೇರ್ಪಡಿ-ಸ-ಲಾ-ಗು-ವು-ದಿಲ್ಲ
ಬೇರ್ಪಡಿ-ಸಲು
ಬೇಲೂರಿನ
ಬೇಲೂರಿ-ನಲ್ಲಿ
ಬೇಲೂರು
ಬೇಲೂರು-ಮಠ-ದಿಂದ
ಬೇಳೆ
ಬೇವಾಗು-ವವು
ಬೇಸತ್ತು
ಬೇಸರ-ಗೊಂಡಿದೆ
ಬೇಸರ-ಪಟ್ಟು-ಕೊಂಡು
ಬೇಸಿಗೆಯ
ಬೈಗುಳ
ಬೈಠಕ್
ಬೈಠಕ್-ಖಾ-ನೆ-ಯಲ್ಲಿ
ಬೈದರೆ
ಬೈಬಲು
ಬೈಬಲ್
ಬೈಬಲ್ಲನ್ನು
ಬೈಬಲ್ಲಿನ
ಬೈಬಲ್ಲಿ-ನಲ್ಲಿ
ಬೈಬಲ್ಲೇ
ಬೈಬಲ್-ಗಳಿಂದ
ಬೈಯುತ್ತಾರೆ
ಬೈಯು-ವು-ದಿಲ್ಲ
ಬೊಂಬಾಯಿ
ಬೊಂಬೆ
ಬೊಂಬೆ-ಗಳಂತೆ
ಬೊಕ್ಕಸ
ಬೊಗಳಲು
ಬೊಗಳುತ್ತವೆ
ಬೊಗಸಾಮಗ್ರಿ
ಬೊಗಸೆ
ಬೊಝೆ
ಬೊಲೊ
ಬೋಧ-ಕ-ನಾಗಿದ್ದ-ನೆಂಬುದು
ಬೋಧ-ಕ-ರನ್ನು
ಬೋಧಕ-ರಾಗಲು
ಬೋಧಕ-ರಾಗಿದ್ದ-ರೆಂಬು-ದನ್ನು
ಬೋಧ-ಕರು
ಬೋಧತ
ಬೋಧನೆ
ಬೋಧನೆ-ಗಳನ್ನು
ಬೋಧನೆಯ
ಬೋಧನೆ-ಯನ್ನು
ಬೋಧನೆ-ಯಲ್ಲಿ
ಬೋಧನೆ-ಯಿಂದ
ಬೋಧನೆಯೂ
ಬೋಧನೆಯೇ
ಬೋಧಪ್ರದ-ವಾದ
ಬೋಧಿ-ಸದೆ
ಬೋಧಿ-ಸದೇ
ಬೋಧಿಸ-ಬ-ಹುದು
ಬೋಧಿಸ-ಬೇಕು
ಬೋಧಿಸ-ಲಾಗಿದೆ
ಬೋಧಿ-ಸ-ಲಾ-ರರು
ಬೋಧಿ-ಸಲು
ಬೋಧಿ-ಸಲ್ಪ-ಡು-ವುವು
ಬೋಧಿಸಿ
ಬೋಧಿ-ಸಿದ
ಬೋಧಿಸಿ-ದಂತೆ
ಬೋಧಿಸಿ-ದನು
ಬೋಧಿಸಿ-ದನೇ
ಬೋಧಿಸಿ-ದನ್ನೆನ್ನು-ವಿರಾ
ಬೋಧಿಸಿ-ದರು
ಬೋಧಿಸಿ-ದರೆ
ಬೋಧಿಸಿದ್ದಾ-ನೆಯೆ
ಬೋಧಿಸಿದ್ದಾರೆ
ಬೋಧಿ-ಸಿದ್ದು
ಬೋಧಿಸಿ-ರುವನು
ಬೋಧಿಸಿ-ರುವನೆ
ಬೋಧಿಸಿ-ರು-ವುದು
ಬೋಧಿಸಿ-ರು-ವೆನು
ಬೋಧಿ-ಸಿಲ್ಲ
ಬೋಧಿಸು
ಬೋಧಿಸುತ್ತದೆ
ಬೋಧಿಸುತ್ತಾ-ನೆಯೇ
ಬೋಧಿ-ಸುತ್ತಿದ್ದ
ಬೋಧಿ-ಸುತ್ತಿದ್ದನು
ಬೋಧಿ-ಸುತ್ತಿದ್ದರು
ಬೋಧಿ-ಸುತ್ತಿದ್ದಾನೆ
ಬೋಧಿ-ಸುತ್ತಿದ್ದಾರೆ
ಬೋಧಿ-ಸುತ್ತಿದ್ದು-ದ-ರಿಂದ
ಬೋಧಿ-ಸುತ್ತಿ-ರು-ವುದು
ಬೋಧಿ-ಸುತ್ತಿ-ರು-ವುದೂ
ಬೋಧಿ-ಸುತ್ತಿ-ರುವೆ-ನೆಂದು
ಬೋಧಿಸುತ್ತೇನೆ
ಬೋಧಿ-ಸುವ
ಬೋಧಿಸು-ವನು
ಬೋಧಿಸು-ವರು
ಬೋಧಿಸು-ವರೋ
ಬೋಧಿಸು-ವು-ದ-ರಿಂದ
ಬೋಧಿಸು-ವು-ದಿಲ್ಲ
ಬೋಧಿಸು-ವುದು
ಬೋಧಿಸು-ವುದೇ
ಬೋಧಿಸು-ವೆನು
ಬೋಧೆ-ಯನ್ನು
ಬೋಧೆ-ಯಾಗುತ್ತವೆ
ಬೋಧೆ-ಯುಂಟಾಗುವುದು
ಬೋಲೆ
ಬೋಲೊ
ಬೋಳು
ಬೋಸ್
ಬೌದ್ಧ
ಬೌದ್ಧ-ಧರ್ಮ
ಬೌದ್ಧ-ಧರ್ಮದ
ಬೌದ್ಧ-ಮತ
ಬೌದ್ಧ-ಮ-ತದ
ಬೌದ್ಧ-ಮತ-ದಿಂದ
ಬೌದ್ಧ-ಮತ-ವನ್ನು
ಬೌದ್ಧ-ಮತ-ವೊಂದೇ
ಬೌದ್ಧ-ಮ-ತಾವ-ಲಂಬಿ-ಗಳು
ಬೌದ್ಧ-ಯುಗಕ್ಕಿಂತ
ಬೌದ್ಧರ
ಬೌದ್ಧ-ರನ್ನು
ಬೌದ್ಧ-ರಾಗಿ
ಬೌದ್ಧ-ರಿಗೇ
ಬೌದ್ಧರು
ಬೌದ್ಧರೂ
ಬೌದ್ಧ-ರೆಂದು
ಬೌದ್ಧಶ್ರಮ-ಣರು
ಬೌದ್ಧಿಕ
ಬ್ಯಾಪ್ಟಿಸ್ಟ್
ಬ್ರಜೇರ್
ಬ್ರಹ್ಮ
ಬ್ರಹ್ಮ-ಗಳ
ಬ್ರಹ್ಮ-ಚರ್ಯ
ಬ್ರಹ್ಮ-ಚರ್ಯದ
ಬ್ರಹ್ಮ-ಚರ್ಯ-ದಿಂದ
ಬ್ರಹ್ಮ-ಚರ್ಯ-ವನ್ನು
ಬ್ರಹ್ಮ-ಚರ್ಯವೇ
ಬ್ರಹ್ಮ-ಚರ್ಯಾಭ್ಯಾಸ
ಬ್ರಹ್ಮ-ಚರ್ಯಾಭ್ಯಾಸ-ದಿಂದ
ಬ್ರಹ್ಮ-ಚರ್ಯಾಶ್ರಮದ
ಬ್ರಹ್ಮ-ಚರ್ಯಾಶ್ರಮ-ವನ್ನು
ಬ್ರಹ್ಮ-ಚರ್ಯೆಯ
ಬ್ರಹ್ಮ-ಚರ್ಯೆಯಲ್ಲಿದ್ದು
ಬ್ರಹ್ಮ-ಚಾರಿ
ಬ್ರಹ್ಮ-ಚಾರಿ-ಗಳ
ಬ್ರಹ್ಮ-ಚಾರಿ-ಗಳನ್ನೂ
ಬ್ರಹ್ಮ-ಚಾರಿ-ಗಳನ್ನೆಲ್ಲಾ
ಬ್ರಹ್ಮ-ಚಾರಿ-ಗಳಲ್ಲಿ
ಬ್ರಹ್ಮ-ಚಾರಿ-ಗಳಿಂದ
ಬ್ರಹ್ಮ-ಚಾರಿ-ಗಳಿಗೂ
ಬ್ರಹ್ಮ-ಚಾರಿ-ಗಳಿಗೆ
ಬ್ರಹ್ಮ-ಚಾರಿ-ಗಳು
ಬ್ರಹ್ಮ-ಚಾರಿ-ಗಳೂ
ಬ್ರಹ್ಮ-ಚಾರಿ-ಗಳೆಲ್ಲಾ
ಬ್ರಹ್ಮ-ಚಾರಿ-ಗಳೇ
ಬ್ರಹ್ಮ-ಚಾರಿ-ಣಿ-ಯ-ರನ್ನು
ಬ್ರಹ್ಮ-ಚಾರಿ-ಣಿಯ-ರಾದ
ಬ್ರಹ್ಮ-ಚಾರಿ-ಣಿಯ-ರಿಲ್ಲದೆ
ಬ್ರಹ್ಮ-ಚಾರಿ-ಣಿಯರು
ಬ್ರಹ್ಮ-ಚಾರಿ-ಣಿಯರೂ
ಬ್ರಹ್ಮ-ಚಾರಿಯ
ಬ್ರಹ್ಮ-ಚಾರಿ-ಯಾಗಿ
ಬ್ರಹ್ಮ-ಜಾ-ಗರಿ-ತ-ವಾಗುತ್ತದೆ
ಬ್ರಹ್ಮಜ್ಞ
ಬ್ರಹ್ಮಜ್ಞ-ನಾಗು-ವುದು
ಬ್ರಹ್ಮಜ್ಞ-ನಾದ-ನೆಂದು
ಬ್ರಹ್ಮಜ್ಞ-ನಿಗೆ
ಬ್ರಹ್ಮಜ್ಞನೂ
ಬ್ರಹ್ಮಜ್ಞನೊ
ಬ್ರಹ್ಮಜ್ಞ-ರಾಗ-ಲಾ-ರರು
ಬ್ರಹ್ಮಜ್ಞ-ರಾಗಿ-ರು-ವು-ದನ್ನು
ಬ್ರಹ್ಮಜ್ಞ-ರಾಗುವರು
ಬ್ರಹ್ಮಜ್ಞರು
ಬ್ರಹ್ಮಜ್ಞ-ರೆಂದಾದರೂ
ಬ್ರಹ್ಮಜ್ಞರೊ
ಬ್ರಹ್ಮಜ್ಞಾನ
ಬ್ರಹ್ಮಜ್ಞಾನಕ್ಕಾ-ದರೂ
ಬ್ರಹ್ಮಜ್ಞಾನಕ್ಕೆ
ಬ್ರಹ್ಮಜ್ಞಾನ-ಗಳ
ಬ್ರಹ್ಮಜ್ಞಾನದ
ಬ್ರಹ್ಮಜ್ಞಾನ-ದಲ್ಲಿ
ಬ್ರಹ್ಮಜ್ಞಾನ-ವನ್ನು
ಬ್ರಹ್ಮಜ್ಞಾನ-ವಾಗುವ
ಬ್ರಹ್ಮಜ್ಞಾನ-ವಾಗು-ವು-ದನ್ನು
ಬ್ರಹ್ಮಜ್ಞಾನ-ವಾ-ಗು-ವು-ದಿಲ್ಲ
ಬ್ರಹ್ಮಜ್ಞಾನ-ವಾದರೆ
ಬ್ರಹ್ಮಜ್ಞಾನವೆ
ಬ್ರಹ್ಮಜ್ಞಾನವೇ
ಬ್ರಹ್ಮಜ್ಞಾನಿ
ಬ್ರಹ್ಮಜ್ಞಾನಿ-ಗಳು
ಬ್ರಹ್ಮಜ್ಞಾನಿ-ಗಳೊ-ಡನೆ
ಬ್ರಹ್ಮಜ್ಞಾನಿಗೂ
ಬ್ರಹ್ಮಜ್ಞಾನಿ-ಯಾದರೂ
ಬ್ರಹ್ಮಜ್ಞಾನಿಯು
ಬ್ರಹ್ಮ-ತತ್ತ್ವವು
ಬ್ರಹ್ಮ-ತತ್ತ್ವಾಸ್ವಾದವು
ಬ್ರಹ್ಮದ
ಬ್ರಹ್ಮ-ದಲಿ
ಬ್ರಹ್ಮ-ದಲ್ಲಿ
ಬ್ರಹ್ಮ-ದಲ್ಲಿಯೇ
ಬ್ರಹ್ಮ-ದಿಂದ
ಬ್ರಹ್ಮಧ್ಯಾನ-ವೆಂಬ
ಬ್ರಹ್ಮನ
ಬ್ರಹ್ಮ-ನನ್ನು
ಬ್ರಹ್ಮ-ನಲ್ಲಿ
ಬ್ರಹ್ಮ-ನಲ್ಲಿಯೇ
ಬ್ರಹ್ಮ-ನ-ವರೆಗೆ
ಬ್ರಹ್ಮ-ನಾಗ-ಬಲ್ಲ
ಬ್ರಹ್ಮ-ನಾಗಿದ್ದರೂ
ಬ್ರಹ್ಮ-ನಿಂದ
ಬ್ರಹ್ಮ-ನಿಗೂ
ಬ್ರಹ್ಮ-ನಿ-ಗೇಕೆ
ಬ್ರಹ್ಮ-ನಿದ್ದಾನೆ
ಬ್ರಹ್ಮನು
ಬ್ರಹ್ಮ-ನೆಂದು
ಬ್ರಹ್ಮ-ನೆ-ಡೆಗೆ
ಬ್ರಹ್ಮನೇ
ಬ್ರಹ್ಮ-ನೊ-ಡನೆ
ಬ್ರಹ್ಮ-ಪುತ್ರಾ
ಬ್ರಹ್ಮಪ್ರಕಾ-ಶದ
ಬ್ರಹ್ಮ-ಲೋಕ
ಬ್ರಹ್ಮ-ವನ್ನು
ಬ್ರಹ್ಮ-ವಸ್ತು-ವನ್ನು
ಬ್ರಹ್ಮ-ವಸ್ತುವು
ಬ್ರಹ್ಮ-ವಾಗಿಯೆ
ಬ್ರಹ್ಮ-ವಾದರೆ
ಬ್ರಹ್ಮ-ವಿಕಾಸ-ವಾದರೆ
ಬ್ರಹ್ಮ-ವಿ-ಚಾರದ
ಬ್ರಹ್ಮ-ವಿದ್ಯಾ
ಬ್ರಹ್ಮ-ವಿದ್ಯಾ-ಲಾಭಕ್ಕೆ
ಬ್ರಹ್ಮ-ವಿದ್ಯೆ-ಯನ್ನು
ಬ್ರಹ್ಮ-ವಿಷ್ಣು-ಗಳೂ
ಬ್ರಹ್ಮ-ವೀರ್ಯ-ದಲ್ಲಿ
ಬ್ರಹ್ಮವು
ಬ್ರಹ್ಮವೇ
ಬ್ರಹ್ಮ-ವೇದ
ಬ್ರಹ್ಮ-ವೊಂದು
ಬ್ರಹ್ಮ-ವೊಂದೇ
ಬ್ರಹ್ಮ-ಶ-ತಾಂತರೇಪಿ
ಬ್ರಹ್ಮ-ಸತ್ತೆ-ಯನ್ನು
ಬ್ರಹ್ಮ-ಸತ್ತೆಯು
ಬ್ರಹ್ಮ-ಸತ್ತೆ-ಯೊಂದನ್ನು
ಬ್ರಹ್ಮ-ಸತ್ತೆ-ಯೊಂದೇ
ಬ್ರಹ್ಮ-ಸತ್ಯ-ದಲ್ಲಿ
ಬ್ರಹ್ಮ-ಸತ್ಯ-ವನ್ನು
ಬ್ರಹ್ಮ-ಸಮಾಜ-ವನ್ನು
ಬ್ರಹ್ಮ-ಸಾಕ್ಷಾತ್ಕಾರ
ಬ್ರಹ್ಮ-ಸಾಕ್ಷಾತ್ಕಾರ-ವಾದ
ಬ್ರಹ್ಮ-ಸಾಧ-ನಕ್ಕೆ
ಬ್ರಹ್ಮ-ಸಿಂಹ-ವನ್ನು
ಬ್ರಹ್ಮ-ಸೂತ್ರ-ಗಳನ್ನು
ಬ್ರಹ್ಮಸ್ವ-ರೂಪ
ಬ್ರಹ್ಮಾಂಡ
ಬ್ರಹ್ಮಾಂಡ-ಗಳು
ಬ್ರಹ್ಮಾಂಡದಿ
ಬ್ರಹ್ಮಾಂಡ-ವನ್ನು
ಬ್ರಹ್ಮಾಂಡವೂ
ಬ್ರಹ್ಮಾಂಡ-ವೆಲ್ಲವೂ
ಬ್ರಹ್ಮಾಂಡ-ವೆಲ್ಲಾ
ಬ್ರಹ್ಮಾ-ನಂದ
ಬ್ರಹ್ಮಾ-ನಂದದ
ಬ್ರಹ್ಮಾ-ನಂದರ
ಬ್ರಹ್ಮಾ-ನಂದ-ರನ್ನು
ಬ್ರಹ್ಮಾ-ನಂದ-ರಿಗೆ
ಬ್ರಹ್ಮಾ-ನಂದರು
ಬ್ರಹ್ಮಾ-ನಂದರೂ
ಬ್ರಹ್ಮಾ-ನಂದ-ವನ್ನು
ಬ್ರಹ್ಮಾ-ನಂದವೂ
ಬ್ರಹ್ಮಾಭಾ-ಸದ
ಬ್ರಹ್ಮೈವ
ಬ್ರಾಹ್ಮಣ
ಬ್ರಾಹ್ಮಣತ್ವ-ವನ್ನು
ಬ್ರಾಹ್ಮಣನ
ಬ್ರಾಹ್ಮಣ-ನನ್ನೂ
ಬ್ರಾಹ್ಮಣ-ನಲ್ಲ
ಬ್ರಾಹ್ಮಣ-ನಾ-ಗದೆ
ಬ್ರಾಹ್ಮಣ-ನಾಗುತ್ತಿದ್ದ
ಬ್ರಾಹ್ಮ-ಣನು
ಬ್ರಾಹ್ಮಣನೂ
ಬ್ರಾಹ್ಮಣ-ನೆಂದರೆ
ಬ್ರಾಹ್ಮಣ-ನೆಂದು
ಬ್ರಾಹ್ಮಣನೇ
ಬ್ರಾಹ್ಮಣರ
ಬ್ರಾಹ್ಮಣ-ರಲ್ಲ-ವೆಂಬುದು
ಬ್ರಾಹ್ಮಣ-ರಲ್ಲಿ-ರುವ
ಬ್ರಾಹ್ಮಣ-ರಷ್ಟೇ
ಬ್ರಾಹ್ಮಣ-ರಿಂದ
ಬ್ರಾಹ್ಮಣ-ರಿಗೆ
ಬ್ರಾಹ್ಮಣ-ರಿದ್ದಾರೆ
ಬ್ರಾಹ್ಮಣ-ರಿದ್ದಾರೆಯೊ
ಬ್ರಾಹ್ಮ-ಣರು
ಬ್ರಾಹ್ಮಣರೂ
ಬ್ರಾಹ್ಮಣರೆ
ಬ್ರಾಹ್ಮಣ-ರೆಂದು
ಬ್ರಾಹ್ಮಣ-ರೆಲ್ಲಾ
ಬ್ರಾಹ್ಮಣರೇ
ಬ್ರಾಹ್ಮಣ-ವರ್ಗ-ದಲ್ಲಲ್ಲ
ಬ್ರಾಹ್ಮಣ-ಸಂಜಾತ-ರಾದ
ಬ್ರಾಹ್ಮಣಾದಿ
ಬ್ರಾಹ್ಮಣೀ
ಬ್ರಾಹ್ಮಣೇ-ತರ
ಬ್ರಾಹ್ಮಣೇ-ತರ-ರಿಗೆ
ಬ್ರಾಹ್ಮಣ್ಯ
ಬ್ರಾಹ್ಮಣ್ಯ-ವಿದ್ದು-ದ-ರಿಂದ
ಬ್ರಾಹ್ಮೋ-ಗಳ
ಬ್ರಿಟಾನಿ
ಬ್ರಿಟಾನಿಕಾ
ಬ್ರಿಟಿಷ್
ಬ್ರೆಡ್
ಬ್ರೌನಿಂಗನ
ಭಂಗ
ಭಂಗ-ತಾರದೆ
ಭಂಗ-ವೀಣ
ಭಂಗಿ
ಭಂಗಿ-ಯನ್ನು
ಭಂಗಿಸಿ
ಭಂಜನ
ಭಂಜನ-ಹಾರ
ಭಂಡಾರ-ವಾಗಿ-ರು-ವನು
ಭಕತ-ವೃಂದ
ಭಕತ-ಶ-ರಣ
ಭಕುತಿ-ಯಿಂದಲಿ
ಭಕ್ತ
ಭಕ್ತ-ಗಣ-ವನ್ನು
ಭಕ್ತ-ಜನ
ಭಕ್ತ-ಜನರು
ಭಕ್ತನ
ಭಕ್ತ-ನನ್ನು
ಭಕ್ತ-ನಿಗೂ
ಭಕ್ತನು
ಭಕ್ತನೂ
ಭಕ್ತನೇ
ಭಕ್ತ-ನೊಬ್ಬ-ನನ್ನು
ಭಕ್ತ-ನೊಬ್ಬನು
ಭಕ್ತ-ಮಂಡಲಿ-ಯನ್ನು
ಭಕ್ತ-ಮಂಡಲಿಯು
ಭಕ್ತರ
ಭಕ್ತ-ರನ್ನು
ಭಕ್ತ-ರನ್ನೇ
ಭಕ್ತ-ರಲ್ಲವೆ
ಭಕ್ತ-ರಲ್ಲ-ವೆಂದು
ಭಕ್ತ-ರಲ್ಲಿ
ಭಕ್ತ-ರಲ್ಲಿಯೂ
ಭಕ್ತ-ರಾಗುಳಿ-ಯುವರು
ಭಕ್ತ-ರಾದ
ಭಕ್ತ-ರಿಂದ
ಭಕ್ತ-ರಿಗೆ
ಭಕ್ತ-ರಿ-ಗೆಲ್ಲ
ಭಕ್ತ-ರಿ-ಗೆಲ್ಲಾ
ಭಕ್ತರು
ಭಕ್ತರೂ
ಭಕ್ತ-ರೆಂದು
ಭಕ್ತ-ರೆಂದೂ
ಭಕ್ತ-ರೆಲ್ಲರ
ಭಕ್ತ-ರೆಲ್ಲರೂ
ಭಕ್ತರೊ
ಭಕ್ತ-ರೊಡನೆ
ಭಕ್ತ-ರೊಬ್ಬರು
ಭಕ್ತ-ವರ್ಗಕ್ಕೆ
ಭಕ್ತಶ್ರೇಷ್ಠ-ರಾದ
ಭಕ್ತಾಗ್ರಣಿ-ಗಳಿಂದ
ಭಕ್ತಾದಿ-ಗಳ
ಭಕ್ತಾದಿ-ಗಳಿ-ಗೋಸ್ಕರ
ಭಕ್ತಾ-ದಿ-ಗಳು
ಭಕ್ತಾರ್ಜನ
ಭಕ್ತಿ
ಭಕ್ತಿ-ಗಳು
ಭಕ್ತಿಗೂ
ಭಕ್ತಿಗೆ
ಭಕ್ತಿಜ್ಞಾನ-ವನ್ನು
ಭಕ್ತಿ-ಪೂಜೆಯು
ಭಕ್ತಿ-ಭಾವ-ದಲ್ಲಿ
ಭಕ್ತಿ-ಭಾವ-ವನ್ನು
ಭಕ್ತಿ-ಮಾರ್ಗ-ದಲ್ಲಿ
ಭಕ್ತಿ-ಮಾರ್ಗಾನು-ಯಾಯಿ-ಗಳು
ಭಕ್ತಿಯ
ಭಕ್ತಿ-ಯನ್ನು
ಭಕ್ತಿ-ಯನ್ನೂ
ಭಕ್ತಿ-ಯಲ್ಲಿ
ಭಕ್ತಿ-ಯಿಂದ
ಭಕ್ತಿ-ಯಿಂದಾವೃತ
ಭಕ್ತಿ-ಯಿತ್ತು
ಭಕ್ತಿ-ಯಿದ್ದರೆ
ಭಕ್ತಿ-ಯಿಲ್ಲ-ದಿದ್ದಲ್ಲಿ
ಭಕ್ತಿ-ಯುಂಟಾಗುವುದು
ಭಕ್ತಿ-ಯುಳ್ಳ-ವ-ನಾಗಿ
ಭಕ್ತಿ-ಯುಳ್ಳ-ವ-ನಾಗು-ವನು
ಭಕ್ತಿ-ಯೊ-ಡನೆ
ಭಕ್ತಿ-ಯೋಗ
ಭಕ್ತಿರ್ಭಗಶ್ಚ
ಭಕ್ತಿ-ವಿಶ್ವಾಸ-ಗಳು
ಭಕ್ತಿ-ಸೂತ್ರದ
ಭಕ್ತ್ಯಾ
ಭಕ್ಷ್ಯ-ಗಳನ್ನು
ಭಗತ್ಸಾಕ್ಷಾತ್ಕಾರ-ವನ್ನು
ಭಗ-ವಂತ
ಭಗ-ವಂತನ
ಭಗ-ವಂತ-ನನ್ನು
ಭಗ-ವಂತ-ನನ್ನು-ಳಿದು
ಭಗ-ವಂತ-ನಲ್ಲಿ
ಭಗ-ವಂತ-ನಲ್ಲಿ-ಡಲು
ಭಗ-ವಂತ-ನಿಂದಲೇ
ಭಗವಂತನು
ಭಗ-ವಂತನೆ
ಭಗ-ವಂತ-ನೆಂಬ
ಭಗ-ವಂತ-ನೆ-ಡೆಗೆ
ಭಗ-ವಂತನೇ
ಭಗ-ವಂತ-ನೊಟ್ಟಿಗೇ
ಭಗ-ವಂತ-ನೊ-ಡನೆ
ಭಗ-ವಂತ-ನೊಬ್ಬನು
ಭಗ-ವಂತ-ನೊಬ್ಬನೇ
ಭಗ-ವತೀ
ಭಗ-ವತ್ಕೃಪೆ
ಭಗ-ವತ್ಕೃಪೆಗೆ
ಭಗ-ವತ್ಕೃಪೆ-ಯಿಂದ
ಭಗ-ವತ್ಸಾಕ್ಷಾತ್ಕಾರ
ಭಗ-ವತ್ಸಾಕ್ಷಾತ್ಕಾ-ರಕ್ಕೆ
ಭಗ-ವತ್ಸಾಕ್ಷಾತ್ಕಾರದ
ಭಗವ-ದನು-ರಾಗ
ಭಗವ-ದಿಚ್ಛೆಯಿದ್ದಂತೆ
ಭಗವದ್ಗೀತೆ-ಯನ್ನು
ಭಗವದ್ಜ್ಞಾನ-ದಾಚೆ
ಭಗ-ವನ್ನಾಮ-ವನ್ನು
ಭಗ-ವನ್ನಾಮ-ವನ್ನುಚ್ಚರಿ-ಸುತ್ತ
ಭಗ-ವನ್ನಾಮಸ್ಮ-ರಣೆ
ಭಗವಾ-ನರ
ಭಗ-ವಾನ್
ಭಗ್ನ-ದೇಹ
ಭಜನ
ಭಜನಂ
ಭಜ-ನ-ಗಳನ್ನು
ಭಜನೆ
ಭಜ-ನೆಗೆ
ಭಜನೆ-ಯನ್ನು
ಭಜಿ-ಸ-ಲಿಲ್ಲ
ಭಜೇ
ಭಟ್ಟಾ-ಚಾರ್ಯ
ಭಟ್ಟಾ-ಚಾರ್ಯನ
ಭಟ್ಟಾ-ಚಾರ್ಯ-ನನ್ನು
ಭಟ್ಟಾ-ಚಾರ್ಯರು
ಭದ್ರವಾ-ಗಿಟ್ಟು
ಭದ್ರ-ವಾಗು-ವುದು
ಭಯ
ಭಯಂಕರ
ಭಯಂಕರ-ಭಾವ-ಗಳ
ಭಯಂಕರ-ವಾಗಿ
ಭಯಂಕರ-ವಾದ
ಭಯಕಿ
ಭಯಕ್ಕೋ
ಭಯ-ಗೌ-ರವ-ಗಳಿತ್ತು
ಭಯಚಕಿತ
ಭಯದ
ಭಯ-ದಿಂದ
ಭಯ-ದಿಂದಲಿ
ಭಯ-ಪಟ್ಟು
ಭಯಪಡ-ದಿರು
ಭಯಪಡ-ಬೇಡ
ಭಯ-ಪಡುವೆ
ಭಯಭೀತ-ರೆಲ್ಲರೂ
ಭಯ-ಭೀತಿ-ಗಳನ್ನು
ಭಯರ-ಹಿತ-ನಾಗು
ಭಯರ-ಹಿತ-ರಾಗುವರೋ
ಭಯ-ರ-ಹಿತ-ವಾ-ದುದು
ಭಯ-ರೂಪಿ
ಭಯವ
ಭಯ-ವನ್ನೇ
ಭಯ-ವಾಗು-ವುದು
ಭಯ-ವಿದೆ
ಭಯ-ವಿಲ್ಲ
ಭಯ-ವುಂಟಾ-ಗಿದೆ
ಭಯವೇ
ಭಯವೇ-ತಕೆ
ಭಯ-ವೇನು
ಭಯಾತ್
ಭಯಾದಸ್ಯಾಗ್ನಿಸ್ತ-ಪತಿ
ಭಯಾ-ದಿಂದ್ರಶ್ಚ
ಭಯಾನಕ-ವಾಗಿ-ರು-ವು-ದನ್ನೂ
ಭಯಾನಕ-ವಾದ
ಭಯಾನ್ವಿತಂ
ಭಯೆರ್
ಭರತ
ಭರತ-ಖಂಡ
ಭರತ-ಖಂಡಕ್ಕೆ
ಭರತ-ಖಂಡದ
ಭರತ-ಖಂಡ-ದಲ್ಲಿ
ಭರತ-ಖಂಡ-ದಲ್ಲಿದ್ದಾಗ
ಭರತ-ಖಂಡ-ದಲ್ಲಿನ
ಭರತ-ಖಂಡ-ದಲ್ಲಿಯೂ
ಭರತ-ಖಂಡ-ದಲ್ಲಿ-ರುವರು
ಭರತ-ಖಂಡ-ದಲ್ಲಿ-ರು-ವುದು
ಭರತ-ಖಂಡ-ದಲ್ಲೇಕೆ
ಭರತ-ಖಂಡ-ದಿಂದ
ಭರತ-ಖಂಡ-ವನ್ನು
ಭರತ-ಖಂಡ-ವನ್ನೆಲ್ಲಾ
ಭರತ-ಖಂಡ-ವೀಗ
ಭರತ-ಖಂಡವು
ಭರತ-ಖಂಡವೇ
ಭರತ-ಚಂದ್ರ
ಭರತ-ಚಂದ್ರರ
ಭರತ-ಚಂದ್ರ-ರನ್ನು
ಭರ-ತನ
ಭರತ-ವರ್ಷ
ಭರತ-ವರ್ಷದ
ಭರತ-ವರ್ಷ-ದಲ್ಲಿ
ಭರತ-ವರ್ಷ-ದಲ್ಲಿದ್ದ
ಭರತ-ವರ್ಷ-ದಲ್ಲೆಲ್ಲಾ
ಭರತ-ವರ್ಷ-ದಲ್ಲೇ
ಭರತ-ವರ್ಷ-ವನ್ನೇ
ಭರತ-ವರ್ಷ-ವಾದರೋ
ಭರ-ದಲಿ
ಭರವಸೆ
ಭರವ-ಸೆ-ಗಳನ್ನೂ
ಭರವ-ಸೆಯ
ಭರವ-ಸೆ-ಯ-ಳಿದು
ಭರವ-ಸೆ-ಯಿಂದ
ಭರವ-ಸೆಯು
ಭರವ-ಸೆ-ಯುಂಟಾಗುವುದು
ಭರವ-ಸೆ-ಯೆಲ್ಲಾ
ಭರಾ
ಭರೀ
ಭರ್ಜಿ-ಯಂತೆ
ಭರ್ಜಿ-ಯಿಂದ
ಭರ್ತಿ-ಯಾಗಿತ್ತು
ಭರ್ತೃಹರಿ
ಭವ
ಭವ-ಗೋಷ್ಪದ
ಭವ-ಘೋರ್
ಭವ-ಜಲಂ
ಭವತಿ
ಭವತು
ಭವತ್ವನು-ದಿನಂ
ಭವದ
ಭವ-ನದೊಳಲ್ಲಿ
ಭವ-ಪಾರ
ಭವ-ಬಂಧನ
ಭವ-ಬಂಧ-ನ-ವನು
ಭವ-ಭೂತಿ-ಯಲ್ಲಿ
ಭವ-ಭೇದ-ಕಾರಿ
ಭವ-ರೋ-ಗವು
ಭವ-ರೋಗ-ವೈದ್ಯನು
ಭವ-ವೈದ್ಯ
ಭವ-ವೈದ್ಯಂ
ಭವ-ವೈದ್ಯ-ನಾಗ-ಬೇಕು
ಭವ-ಸಾಗರ
ಭವ-ಸಾಗರ-ದಲ್ಲಿ
ಭವ-ಸಾಗರ-ವನ್ನು
ಭವ-ಸಾಗರವೂ
ಭವಾನಿ-ಯನ್ನು
ಭವಾನಿಯು
ಭವಾನಿಯೇ
ಭವಾನೀ
ಭವಾನೀ-ಮಾತೆ
ಭವಿ-ತವ್ಯ
ಭವಿಷ್ಯ
ಭವಿಷ್ಯತಿ
ಭವಿಷ್ಯತ್ಕಾಲ-ಗಳ
ಭವಿಷ್ಯತ್ಕಾಲ-ಗಳಲ್ಲಿ
ಭವಿಷ್ಯತ್ತು
ಭವಿಷ್ಯದ
ಭವಿಷ್ಯ-ದಲ್ಲಿ
ಭವಿಷ್ಯ-ದಾಚೆಗೆ
ಭವಿಷ್ಯ-ವನ್ನು
ಭವಿಷ್ಯ-ವಾಣಿ
ಭವಿಷ್ಯವೂ
ಭವಿಷ್ಯ-ವೇನು
ಭವೇಸ್ಮಿನ್
ಭವ್ಯ-ಜೀವನ-ವನ್ನು
ಭವ್ಯ-ತಮ
ಭವ್ಯ-ತರ-ವಾದ
ಭವ್ಯ-ಭಾವ-ನೋದ್ದೀಪ-ಕ-ವಲ್ಲ
ಭವ್ಯ-ವಾಗಿಯೇ
ಭವ್ಯ-ವಾದ
ಭವ್ಯ-ವಾದುದೇ
ಭಸ್ಮ-ಗಳ
ಭಸ್ಮ-ಲೇಪಿತ-ರಾಗಿ
ಭಸ್ಮಾಸ್ಥಿ-ಗಳನ್ನು
ಭಸ್ಮೀ-ಭೂತ-ವಾದವು
ಭಾಂಗೇ
ಭಾಗ
ಭಾಗಕ್ಕೆ
ಭಾಗ-ಗಳ
ಭಾಗ-ಗಳನ್ನಾಗಿ
ಭಾಗ-ಗಳಲ್ಲಂತೂ
ಭಾಗ-ಗಳಲ್ಲಿ
ಭಾಗ-ಗಳಲ್ಲೆಲ್ಲಾ
ಭಾಗ-ಗಳಾಗಿ
ಭಾಗ-ಗಳಿಗೆ
ಭಾಗ-ಗಳಿವೆ
ಭಾಗ-ಗಳು
ಭಾಗ-ಗಳೂ
ಭಾಗ-ಗಳೆಲ್ಲಾ
ಭಾಗದ
ಭಾಗ-ದಲ್ಲಿ
ಭಾಗ-ದಲ್ಲಿದೆ
ಭಾಗ-ದಲ್ಲಿದ್ದ
ಭಾಗ-ದಲ್ಲಿಯೂ
ಭಾಗ-ದಷ್ಟು
ಭಾಗ-ದಿಂದ
ಭಾಗ-ಮಾತ್ರ
ಭಾಗ-ವತ
ಭಾಗ-ವತ-ದಲ್ಲಿ
ಭಾಗ-ವನ್ನು
ಭಾಗ-ವನ್ನೇ
ಭಾಗ-ವಹಿ-ಸುವರು
ಭಾಗ-ವಹಿ-ಸುವಾಗ
ಭಾಗ-ವಾಗಿದೆ
ಭಾಗ-ವೆಂದರೆ
ಭಾಗ-ವೆಂದೂ
ಭಾಗಶಃ
ಭಾಗಿ-ಗಳು
ಭಾಗಿ-ಗಳೆಂದೂ
ಭಾಗಿ-ಗಳೊ
ಭಾಗೀರಥಿಯ
ಭಾಗ್ಯ
ಭಾಗ್ಯ-ಲಕ್ಷ್ಮಿ
ಭಾಗ್ಯ-ವಂತಃ
ಭಾಗ್ಯ-ವಶ-ದಿಂದ
ಭಾಗ್ಯ-ವಿಲ್ಲದ
ಭಾಗ್ಯ-ವೇನಿ-ದ-ರಿಂದೆ
ಭಾಗ್ಯ-ಶಾಲಿ-ಗಳಾದ
ಭಾತಿ
ಭಾನು
ಭಾನು-ವಾರ
ಭಾನು-ವಾ-ರವೂ
ಭಾಯ
ಭಾಯಿ
ಭಾರ
ಭಾರಕೆ
ಭಾರತ
ಭಾರ-ತಕ್ಕೆ
ಭಾರ-ತದ
ಭಾರ-ತ-ದಲ್ಲಂತೂ
ಭಾರ-ತ-ದಲ್ಲಿ
ಭಾರ-ತ-ದಲ್ಲಿನ
ಭಾರ-ತ-ದಲ್ಲಿಯೂ
ಭಾರ-ತ-ದಿಂದ
ಭಾರ-ತ-ವನ್ನು
ಭಾರ-ತ-ವರ್ಷ
ಭಾರ-ತ-ವರ್ಷ-ದಲ್ಲಿ
ಭಾರ-ತವು
ಭಾರ-ತೀಯ
ಭಾರ-ತೀಯರ
ಭಾರ-ತೀಯ-ರಂತೂ
ಭಾರ-ತೀಯ-ರಲ್ಲಿ
ಭಾರ-ತೀಯ-ರಾದರೋ
ಭಾರ-ತೀಯರು
ಭಾರತ್
ಭಾರ-ದಲಿ
ಭಾರವ
ಭಾರ-ವನ್ನು
ಭಾರ-ವಾಗಿ
ಭಾರವು
ಭಾರ-ಹಾಕುವುದು
ಭಾರೆ
ಭಾಲ
ಭಾಲೊ
ಭಾವ
ಭಾವಕ್ಕಿಂತಲೂ
ಭಾವಕ್ಕೆ
ಭಾವಕ್ರಾಂತಿ-ಯುಂಟಾಗುವುದು
ಭಾವ-ಗಳ
ಭಾವ-ಗಳನ್ನು
ಭಾವ-ಗಳನ್ನೂ
ಭಾವ-ಗಳನ್ನೆಲ್ಲ
ಭಾವ-ಗಳಲ್ಲಿ
ಭಾವ-ಗಳು
ಭಾವ-ಗಳೂ
ಭಾವ-ಗಳೆ-ರಡೂ
ಭಾವ-ಗಳೆಲ್ಲ
ಭಾವ-ಚಿತ್ರ-ವನ್ನು
ಭಾವ-ತೀವ್ರತೆ
ಭಾವದ
ಭಾವ-ದ-ಲೆ-ಗಳ
ಭಾವ-ದಲ್ಲಿ
ಭಾವ-ದಲ್ಲೇ
ಭಾವ-ದಾ-ಳನು
ಭಾವ-ದಾಳ-ವನು
ಭಾವ-ದಿಂದ
ಭಾವ-ದಿಂದಲೂ
ಭಾವ-ದಿಂದಲೊ
ಭಾವ-ದುನ್ಮಾದ-ದಲಿ
ಭಾವ-ನಾ-ಸಿದ್ಧಾಂತ-ವನ್ನು
ಭಾವನೆ
ಭಾವ-ನೆ-ಗಳ
ಭಾವ-ನೆ-ಗಳನ್ನು
ಭಾವ-ನೆ-ಗಳನ್ನೂ
ಭಾವ-ನೆ-ಗಳಿಂದ
ಭಾವ-ನೆ-ಗಳಿಂದೇಳುವ
ಭಾವ-ನೆ-ಗಳಿಗೂ
ಭಾವ-ನೆ-ಗಳಿಗೆ
ಭಾವ-ನೆ-ಗಳು
ಭಾವ-ನೆ-ಗಳೆಲ್ಲ
ಭಾವ-ನೆ-ಗಳೆಲ್ಲಾ
ಭಾವ-ನೆ-ಗಿಂತ
ಭಾವ-ನೆಗೆ
ಭಾವ-ನೆಯ
ಭಾವ-ನೆ-ಯನ್ನು
ಭಾವ-ನೆ-ಯನ್ನೂ
ಭಾವ-ನೆ-ಯನ್ನೇ
ಭಾವ-ನೆ-ಯಲ್ಲ
ಭಾವ-ನೆ-ಯಲ್ಲಿ
ಭಾವ-ನೆ-ಯಾದರೋ
ಭಾವ-ನೆ-ಯಿಂದ
ಭಾವ-ನೆ-ಯಿಲ್ಲದೆ
ಭಾವ-ನೆ-ಯಿಲ್ಲವೊ
ಭಾವ-ನೆಯು
ಭಾವ-ನೆ-ಯುಂಟಾ-ಯಿತು
ಭಾವ-ನೆಯೂ
ಭಾವ-ನೆಯೇ
ಭಾವ-ಪೂರ್ವ-ಕ-ವಾಗಿ
ಭಾವಪ್ರಧಾನ
ಭಾವ-ಬಂಧಃ
ಭಾವ-ಭಂಗಿ-ಗಳನ್ನು
ಭಾವ-ಮಯ
ಭಾವ-ಮುಖ-ದಲ್ಲಿ
ಭಾವ-ಮೆನಗಿಲ್ಲ
ಭಾವ-ರ-ಹಿತ-ವಾಗಿವೆ
ಭಾವ-ರಾಜ್ಯದ
ಭಾವ-ರಾಶಿ
ಭಾವ-ರಾಶಿಯ
ಭಾವ-ವನ್ನು
ಭಾವ-ವನ್ನು-ಳಿಸಿ-ಕೊಳ್ಳ-ಬೇಕೆ
ಭಾವ-ವನ್ನೂ
ಭಾವ-ವನ್ನೆ
ಭಾವ-ವನ್ನೇ
ಭಾವ-ವಶ-ರಾಗುತ್ತಿದ್ದರು
ಭಾವವು
ಭಾವ-ವುಂಟಾ-ಗಿದೆ
ಭಾವವೂ
ಭಾವವೇ
ಭಾವ-ವೇನು
ಭಾವ-ಶೂನ್ಯ
ಭಾವ-ಶೂನ್ಯ-ವಾಗಿದೆ
ಭಾವ-ಸಮಾಧಿ
ಭಾವ-ಸಮಾಧಿಯ
ಭಾವ-ಸಾಗರ
ಭಾವ-ಸಾಗರವ
ಭಾವ-ಸಾಗರ-ವ-ನುಕ್ಕಿ-ಸಿದೆ
ಭಾವ-ಸಾ-ಧನೆ-ಗಾಗಿ
ಭಾವಾತ್ಮಕ
ಭಾವಾ-ವೇಶಕ್ಕೆ
ಭಾವಿಸ-ಕೂ-ಡದು
ಭಾವಿಸ-ದಿರು
ಭಾವಿ-ಸದೆ
ಭಾವಿಸ-ಬ-ಹುದು
ಭಾವಿಸ-ಬೇಕು
ಭಾವಿ-ಸಲು
ಭಾವಿಸಿ
ಭಾವಿಸಿಕೊ
ಭಾವಿಸಿ-ಕೊಂಡನು
ಭಾವಿಸಿ-ಕೊಂಡಿರು-ವಿರಿ
ಭಾವಿಸಿ-ಕೊಂಡಿಲ್ಲ
ಭಾವಿಸಿ-ಕೊಂಡು
ಭಾವಿಸಿ-ಕೊಳ್ಳ-ತೊಡಗಿ-ದನು
ಭಾವಿಸಿ-ಕೊಳ್ಳುತ್ತ
ಭಾವಿಸಿ-ಕೊಳ್ಳುತ್ತಾ
ಭಾವಿಸಿ-ಕೊಳ್ಳುತ್ತಾ-ನೆಯೋ
ಭಾವಿಸಿ-ಕೊಳ್ಳುತ್ತಿದ್ದನು
ಭಾವಿಸಿ-ಕೊಳ್ಳುತ್ತೇನೆ
ಭಾವಿಸಿ-ಕೊಳ್ಳುವನು
ಭಾವಿಸಿ-ಕೊಳ್ಳುವಾಗ
ಭಾವಿಸಿ-ಕೊಳ್ಳು-ವುದಕ್ಕಾ-ಗಲಿ
ಭಾವಿಸಿ-ಕೊಳ್ಳು-ವುದಕ್ಕೆ
ಭಾವಿಸಿ-ದನು
ಭಾವಿಸಿ-ದರು
ಭಾವಿಸಿ-ದರೆ
ಭಾವಿಸಿ-ದಾಗ
ಭಾವಿಸಿ-ದಿ-ರೇನು
ಭಾವಿಸಿದ್ದೆ
ಭಾವಿಸಿದ್ದೇನೆ
ಭಾವಿಸಿ-ರು-ವಿರಾ
ಭಾವಿಸಿ-ರುವೆ
ಭಾವಿಸು
ಭಾವಿ-ಸುತ್ತಲೂ
ಭಾವಿ-ಸುತ್ತಾ
ಭಾವಿ-ಸುತ್ತಿದ್ದರು
ಭಾವಿಸುತ್ತೀಯೋ
ಭಾವಿಸುತ್ತೀರಿ
ಭಾವಿಸುತ್ತೇನೆ
ಭಾವಿಸುತ್ತೇವೆ
ಭಾವಿ-ಸುವ
ಭಾವಿಸು-ವನು
ಭಾವಿಸು-ವರು
ಭಾವಿಸು-ವ-ವರೆಗೆ
ಭಾವಿಸು-ವಷ್ಟು
ಭಾವಿಸು-ವಿರಿ
ಭಾವಿಸು-ವಿ-ರೇನು
ಭಾವಿಸು-ವು-ದರ
ಭಾವಿಸು-ವುದು
ಭಾವಿಸು-ವುದೇ
ಭಾವಿ-ಸುವೆ
ಭಾವಿಸು-ವೆವು
ಭಾವೀ
ಭಾವುಕ
ಭಾವೆ
ಭಾವೇರ-ಸಾಗರ
ಭಾವೋದ್ರೇಕ-ದಿಂದ
ಭಾವೋದ್ರೇ-ಕ-ವನ್ನುಂಟು-ಮಾಡುವ
ಭಾವೋದ್ವೇಗ-ಗಳು
ಭಾವ್
ಭಾಷಣ
ಭಾಷಣಕ್ಕೇ
ಭಾಷಣ-ಗಳ
ಭಾಷಣ-ಗಳನ್ನು
ಭಾಷಣ-ಗಳಲ್ಲಿ
ಭಾಷಣ-ಗಳಿಂದೇನೂ
ಭಾಷ-ಣದ
ಭಾಷಣ-ದಲ್ಲಿ
ಭಾಷಣ-ವನ್ನು
ಭಾಷಾಂತರ-ಕಾರನ
ಭಾಷಾಂತ-ರಿಸಿದ್ದಾರೆ
ಭಾಷೆ
ಭಾಷೆ-ಗಳಲ್ಲಿ
ಭಾಷೆ-ಗಳಿಗೆ
ಭಾಷೆಗೆ
ಭಾಷೆ-ಚಿಂತನೆ-ಗಳಲಿ
ಭಾಷೆಯ
ಭಾಷೆ-ಯನ್ನು
ಭಾಷೆ-ಯಲ್ಲಿ
ಭಾಷೆ-ಯಲ್ಲಿದೆ
ಭಾಷೆ-ಯಲ್ಲೇ
ಭಾಷೆ-ಯಾಗಲೀ
ಭಾಷೆ-ಯಿಂದ
ಭಾಷೆಯೂ
ಭಾಷೆ-ಯೆಲ್ಲ
ಭಾಷೆ-ಯೊಂದು
ಭಾಷೆ-ಯೊ-ಡನೆ
ಭಾಷ್ಯ
ಭಾಷ್ಯ-ಕಾರ
ಭಾಷ್ಯ-ಕಾರ-ರನ್ನು
ಭಾಷ್ಯ-ಕಾರ-ರಾದ
ಭಾಷ್ಯ-ಕಾ-ರರು
ಭಾಷ್ಯ-ಕಾರರೊ
ಭಾಷ್ಯ-ಗಳಿವೆ
ಭಾಷ್ಯದ
ಭಾಷ್ಯ-ದಲ್ಲಿ
ಭಾಷ್ಯ-ವನ್ನು
ಭಾಷ್ಯ-ವೆಂದೂ
ಭಾಸ-ವಾಗಿ
ಭಾಸ-ವಾಗುತ್ತಿತ್ತು
ಭಾಸ-ವಾಗು-ವುದು
ಭಾಸುರೋ
ಭಾಸೆ
ಭಾಸೇ
ಭಾಸ್ಕರ
ಭಾಸ್ಕರನ
ಭಾಸ್ವರ
ಭಾಸ್ವರನೆ
ಭಿಕಾರಿ-ಯಂತೆ
ಭಿಕಾರಿ-ಯಾದ
ಭಿಕ್ಷ
ಭಿಕ್ಷಾನ್ನ
ಭಿಕ್ಷಾನ್ನ-ವನ್ನು
ಭಿಕ್ಷಾರ್ಥಿ-ಯಾದೆ
ಭಿಕ್ಷಾ-ಸನೆ
ಭಿಕ್ಷು
ಭಿಕ್ಷು-ಕನ
ಭಿಕ್ಷು-ಕನಿಗುಂಟೆ
ಭಿಕ್ಷು-ಕರ
ಭಿಕ್ಷು-ಕ-ರಂತೆ
ಭಿಕ್ಷು-ಕ-ರಾಗಿ-ಬಿಟ್ಟಿದ್ದಾರೆ
ಭಿಕ್ಷು-ಕರಿ-ರುವ
ಭಿಕ್ಷು-ಕೇರ
ಭಿಕ್ಷು-ಗಳನ್ನು
ಭಿಕ್ಷು-ಗ-ಳಿದ್ದ
ಭಿಕ್ಷು-ಗಳು
ಭಿಕ್ಷು-ವಿಗೆ
ಭಿಕ್ಷು-ವಿನ
ಭಿಕ್ಷು-ವಿ-ನಲ್ಲಿ
ಭಿಕ್ಷೆ
ಭಿಕ್ಷೆ-ಗಿಕ್ಷೆ-ಗಳನ್ನು
ಭಿಕ್ಷೆಗೆ
ಭಿಕ್ಷೆ-ಯನ್ನು
ಭಿಕ್ಷೆ-ಯಿಂದ
ಭಿದ್ಯತೇ
ಭಿನ್ನ
ಭಿನ್ನತೆ
ಭಿನ್ನ-ತೆ-ಗಳಿವೆ
ಭಿನ್ನ-ತೆ-ಯನ್ನು
ಭಿನ್ನ-ತೆ-ಯಿದೆ
ಭಿನ್ನ-ತೆ-ಯೆಲ್ಲ
ಭಿನ್ನ-ಭಾವ
ಭಿನ್ನ-ಭಿನ್ನ
ಭಿನ್ನ-ರಾಗಿದ್ದರು
ಭಿನ್ನ-ವಸ್ತು-ಗ-ಳೆಂದು
ಭಿನ್ನ-ವಾದ
ಭಿನ್ನ-ವಾಸ
ಭಿನ್ನಾಭಿಪ್ರಾಯ-ಗಳಿವೆ
ಭಿನ್ನಾಭಿಪ್ರಾಯ-ವಿದ್ದರೂ
ಭೀಕರ-ತೆ-ಯನ್ನು
ಭೀಕರದ
ಭೀತಿ
ಭೀತಿ-ಯಲ್ಲಿ-ರುವ-ವ-ರಿಗೆ
ಭೀತಿ-ಯೇನೂ
ಭೀಮ
ಭೀಮ-ಗರ್ಜನೆ
ಭೀಮ-ಗರ್ಜನೆ-ಯಿಂದ
ಭೀಮ-ಘೋಷ-ದಲಿ
ಭೀಮ-ತರ-ವಾದ
ಭೀಮ-ರಣಸ್ಥಲ
ಭೀಮರುದ್ರ-ಮುಖ-ದಟ್ಟ-ಹಾಸ-ದಲಿ
ಭೀಮಾ-ಕಾರ-ವಾದ
ಭೀರುತ್ವ
ಭೀಷಣ
ಭೀಷಣ-ವಾಗಿ
ಭೀಷಣವೂ
ಭುಕೇ
ಭುಕ್ತಶೇಷ-ವನ್ನು
ಭುಕ್ತಿ
ಭುಕ್ತಿ-ಮುಕ್ತಿ-ಗಳನ್ನು
ಭುಜದ
ಭುಜದ-ಮೇಲೆ
ಭುಜದ್ವಯ
ಭುಜವ
ಭುಜ್ಯತಾಂ
ಭುವಃ
ಭುವಿ
ಭುವಿಯೆ
ಭೂಃ
ಭೂಗೋಳ
ಭೂತ
ಭೂತ-ಗಣ
ಭೂತ-ಗಳ
ಭೂತ-ಗಳಲ್ಲೊಂದಾಗಿ-ರ-ಬೇಕು
ಭೂತ-ಗಳಿಗೂ
ಭೂತ-ಗಳು
ಭೂತದ
ಭೂತ-ದಾಳ-ಗಳಲ್ಲಿ
ಭೂತ-ನಾಥ
ಭೂತಪ್ರೇತ
ಭೂತಪ್ರೇತ-ಗಳನ್ನು
ಭೂತಪ್ರೇತ-ಗಳಲ್ಲಿ
ಭೂತಪ್ರೇತ-ಗಳು
ಭೂತಲ
ಭೂತವ
ಭೂತ-ವದು
ಭೂತ-ವನ್ನು
ಭೂತ-ವಾಗಿದ್ದು-ದ-ರಿಂದ
ಭೂತ-ವಾಗಿ-ಬಿಡುತ್ತಾರೆ
ಭೂಮಂಡಲ
ಭೂಮಂಡಲದ
ಭೂಮಂಡಲ-ದಲ್ಲಿ
ಭೂಮಂಡಲ-ವನ್ನು
ಭೂಮಂಡಲ-ವನ್ನೇ
ಭೂಮಂಡಲ-ವೆಲ್ಲಾ
ಭೂಮಿ
ಭೂಮಿ-ಕೆಗೆ
ಭೂಮಿ-ಕೆಯ
ಭೂಮಿ-ಕೆ-ಯಲ್ಲಿ
ಭೂಮಿ-ಕೆಯೂ
ಭೂಮಿಗೆ
ಭೂಮಿ-ಗೇಕೆ
ಭೂಮಿಯ
ಭೂಮಿ-ಯನ್ನು
ಭೂಮಿ-ಯಲ್ಲಿ
ಭೂಮಿ-ಯಿಂದ
ಭೂಮಿಯು
ಭೂರ-ಮಣಿ
ಭೂಲೋಕ
ಭೂಲೋಕ-ದಲ್ಲಿ
ಭೂಶೋ-ಧನೆ-ಯನ್ನು
ಭೂಸಂಚಾರ
ಭೇಟಿ
ಭೇಟಿ-ಗಾಗಿ
ಭೇಟಿ-ನೀಡಿ-ದಾಗ
ಭೇಟಿ-ಮಾ-ಡಲು
ಭೇದ
ಭೇದ-ಗಳಿಲ್ಲ
ಭೇದ-ಗಳು
ಭೇದ-ಗಳೆಲ್ಲಾ
ಭೇದ-ಭಾವನೆ
ಭೇದ-ಮಾಡುತ್ತಿದ್ದರೆ
ಭೇದ-ಮಾಡು-ವನು
ಭೇದ-ವನು
ಭೇದ-ವನ್ನಿಟ್ಟು-ಕೊಂಡು
ಭೇದ-ವನ್ನು
ಭೇದ-ವಿತ್ತು
ಭೇದ-ವಿದೆ
ಭೇದ-ವಿದೆ-ಯೆಂದು
ಭೇದ-ವಿ-ದೆಯೋ
ಭೇದ-ವಿಲ್ಲ
ಭೇದ-ವಿಲ್ಲದೆ
ಭೇದ-ವೆಂಬುವ
ಭೇದಾತೀ-ತವೆನ್ನುವರು
ಭೇದಿ
ಭೇದಿಸಿ
ಭೇದಿ-ಸಿ-ಕೊಂಡು
ಭೇದಿ-ಸು-ವುದಕ್ಕಾ-ಗು-ವು-ದಿಲ್ಲ
ಭೇದಿಸೆ
ಭೇರಿಯು
ಭೇರೆ
ಭೇಸೆ
ಭೈರವ
ಭೈರ-ವನ
ಭೈರ-ವನೆ
ಭೈಷ್ವ
ಭೋಃ
ಭೋಕ್ತ
ಭೋಕ್ಷ್ಯಸೇ
ಭೋಗ
ಭೋಗಕ್ಕಾಗಿ
ಭೋಗ-ಗಳನ್ನೂ
ಭೋಗ-ಗಳಿಗೆ
ಭೋಗ-ಗಳು
ಭೋಗ-ತೈಲದಿ
ಭೋಗದ
ಭೋಗ-ದಲ್ಲಿ
ಭೋಗ-ದಲ್ಲಿ-ರು-ವುದೇನು
ಭೋಗ-ದಿಂದ
ಭೋಗ-ದುನ್ಮಾದ-ದಲ್ಲಿ
ಭೋಗ-ಲಾಲಸೆ
ಭೋಗ-ಲಾಲ-ಸೆ-ಗಳೂ
ಭೋಗ-ಲಾಲ-ಸೆ-ಯನ್ನು
ಭೋಗ-ಲಾಲ-ಸೆ-ಯನ್ನೆಲ್ಲಾ
ಭೋಗ-ವನ್ನು
ಭೋಗವು
ಭೋಗವೂ
ಭೋಗವೋ
ಭೋಗ-ಸುಖ-ಗಳ
ಭೋಗಿ-ಗಳ
ಭೋಗಿ-ಸಲು
ಭೋಗೇಚ್ಛೆ
ಭೋಗೇಚ್ಛೆ-ಯನ್ನು
ಭೋಜನ
ಭೋಜ-ನಕ್ಕೆ
ಭೋಜನ-ವಲ್ಲ
ಭೋಜನ-ವಾ-ಯಿತು
ಭೋಜ್ಯ
ಭೋನ
ಭೋರ್ಗರೆ-ದರೂ
ಭೋಲಾ
ಭೌತ-ವಾದ
ಭೌತ-ಶಾಸ್ತ್ರ-ಗಳ
ಭೌತಿಕ
ಭೌತಿಕ-ವಾದದ್ದೆಲ್ಲವೂ
ಭೌತಿಕ-ವಾದಿ-ಗಳೊ
ಭೌತಿಕ-ಶಕ್ತಿ-ಯಂತೆಯೂ
ಭೌತಿಕಸ್ತ-ರಕ್ಕೆ
ಭ್ರಮಣ-ಕೆಳ-ಸುವವು
ಭ್ರಮ-ದಿಂದ
ಭ್ರಮರ
ಭ್ರಮರಾಳಿ-ಗಳವು
ಭ್ರಮವು
ಭ್ರಮಿ-ಸುವ
ಭ್ರಮಿ-ಸುವನು
ಭ್ರಮೆ
ಭ್ರಮೆಯ
ಭ್ರಮೆ-ಯನು
ಭ್ರಮೆ-ಯುಂಟಾಗಿ
ಭ್ರಮೆ-ಯುಂಟಾಗುವುದು
ಭ್ರಷ್ಟ-ತೆ-ಗಳೆಲ್ಲಾ
ಭ್ರಷ್ಟ-ತೆಯೇ
ಭ್ರಷ್ಟ-ರಾಗುತ್ತಾರೆ
ಭ್ರಷ್ಟ-ವಾ-ಯಿತು
ಭ್ರಷ್ಟಾಚ-ರಣೆ-ಗಳು
ಭ್ರಾಂತಿ
ಭ್ರಾಂತಿ-ಪಡು-ವಂತಾಗು-ವುದು
ಭ್ರಾಂತಿಭ್ರಮೆ-ಕಿಂಚಿತ್ತೂ
ಭ್ರಾಂತಿಯ
ಭ್ರಾಂತಿ-ಯಷ್ಟೆ
ಭ್ರಾಂತಿ-ಯಿಂದ
ಭ್ರಾಂತಿ-ಯಿಂದಲೂ
ಭ್ರಾತೃ-ಗಳ
ಭ್ರಾತೃ-ಗಳು
ಭ್ರಾತೃ-ಗಳೊ-ಡನೆ
ಭ್ರಾತೃ-ವರ್ಗ-ದ-ವರು
ಭ್ರಾನ್ತ
ಮಂಕುಬ-ಡಿದಂತಾಯಿತು
ಮಂಕುಬ-ಡಿದು
ಮಂಗ-ಮಾಯ
ಮಂಗಳ
ಮಂಗಳ-ಕರ-ವಾದ
ಮಂಗಳ-ಗೌರಿಯ
ಮಂಗಳ-ಮಯ-ನಾಗಿ-ರು-ವನು
ಮಂಗಳ-ರವ
ಮಂಗಳ-ವನ್ನುಂಟು-ಮಾಡು-ವುದಕ್ಕೆ
ಮಂಗಳ-ವಾಗುತ್ತದೆ
ಮಂಗಳ-ವಾಗು-ವುದೆಂದು
ಮಂಗಳ-ವಾರ-ದಿಂದ
ಮಂಗಳವೂ
ಮಂಗಳಾರತಿ-ಗಾಗಿ
ಮಂಗಳಾರ-ತಿಯ
ಮಂಗಳಾರ-ತಿಯಾದ
ಮಂಗಾಲಾಪ
ಮಂಚದ
ಮಂಚದ-ಮೇಲೆ
ಮಂಜಾ-ದರು
ಮಂಜು
ಮಂಜು-ಕವಿ-ಯಿತು
ಮಂಜುಳ
ಮಂಟಪ-ದಲ್ಲಿ
ಮಂಡಲ
ಮಂಡಲ-ರೂಪೆ
ಮಂಡಲಿ
ಮಂಡಲಿ-ಯಲ್ಲಿ
ಮಂಡಲಿ-ಯೊಂದನ್ನು
ಮಂಡಿ-ಯ-ವರೆಗೆ
ಮಂಡೋ-ದರಿ
ಮಂತ್ರ
ಮಂತ್ರ-ಗಳ
ಮಂತ್ರ-ಗಳನ್ನು
ಮಂತ್ರ-ಗಳಿವೆ
ಮಂತ್ರ-ಗಳೊ
ಮಂತ್ರ-ತಂತ್ರ
ಮಂತ್ರ-ದಲ್ಲಿಯೂ
ಮಂತ್ರ-ದೀಕ್ಷೆ
ಮಂತ್ರ-ಮಂತ್ರ
ಮಂತ್ರ-ಮುಗ್ಧ-ನಾಗಿದ್ದಾನೆ
ಮಂತ್ರ-ಮುಗ್ಧ-ರನ್ನಾಗಿ
ಮಂತ್ರ-ವನ್ನು
ಮಂತ್ರ-ವಾದಿ-ಯಾಗುತ್ತದೆ
ಮಂತ್ರ-ಶಕ್ತಿ-ಯಿತ್ತು
ಮಂತ್ರಸ್ಪರ್ಶ
ಮಂತ್ರಾರ್ಥ-ಗಳನ್ನು
ಮಂತ್ರಾರ್ಥದ್ರಷ್ಟೃ
ಮಂತ್ರೋಚ್ಚಾರ-ಣೆಗೆ
ಮಂತ್ರೋಚ್ಛಾ-ರಣೆ
ಮಂತ್ರೋಚ್ಛಾ-ರಣೆ-ಯಿಂದ
ಮಂತ್ರೋಚ್ಛಾ-ರಣೆ-ಯಿಂದಲೇ
ಮಂತ್ರೋಪ-ದೇಶದ
ಮಂದ
ಮಂದ-ಬುದ್ಧಿ-ಯ-ವ-ರಿಗೆ
ಮಂದ-ಬುದ್ಧಿಯೂ
ಮಂದ-ವಾಯ್ತು
ಮಂದ-ಹಾಸ-ವನು
ಮಂದಿ
ಮಂದಿಗೆ
ಮಂದಿಯ
ಮಂದಿಯು
ಮಂದಿರ
ಮಂದಿ-ರಕ್ಕೆ
ಮಂದಿ-ರ-ಗಳನ್ನು
ಮಂದಿ-ರ-ಗಳಿಗೆ
ಮಂದಿ-ರ-ಗಳು
ಮಂದಿ-ರದ
ಮಂದಿ-ರ-ದಲ್ಲಿ
ಮಂದಿ-ರ-ದಲ್ಲಿಯೂ
ಮಂದಿ-ರ-ವನ್ನು
ಮಂದಿ-ರವು
ಮಕುಟ-ದಲ್ಲಿ-ರುತ್ತಾರೆ
ಮಕ್ಕಳ
ಮಕ್ಕಳಂತೆ
ಮಕ್ಕ-ಳನ್ನು
ಮಕ್ಕಳಲ್ಲಿ
ಮಕ್ಕಳಾಟ
ಮಕ್ಕಳಿಗೂ
ಮಕ್ಕಳಿಗೆ
ಮಕ್ಕಳಿಲ್ಲ
ಮಕ್ಕಳು
ಮಕ್ಕಳು-ಮರಿ-ಗಳಾಗಲೀ
ಮಕ್ಕಳು-ಮರಿ-ಗಳಿ-ಗಾಗಿ
ಮಕ್ಕಳೆಲ್ಲ
ಮಕ್ಕಳೆಲ್ಲಾ
ಮಕ್ಕಳೊ-ಡನೆ
ಮಖೋದ್ಭಿನ್ನ-ಮೋಹಾಂಧ-ಕಾರೈಃ
ಮಗ
ಮಗ-ದೊಂದು
ಮಗನ
ಮಗ-ನನ್ನು
ಮಗನಲ್ಲ-ವೇನು
ಮಗ-ನಿಗೆ
ಮಗ-ನಿಗೆ-ದು-ರಾಗಿ
ಮಗನೂ
ಮಗಳನ್ನು
ಮಗ-ಳಲ್ಲಿ
ಮಗಳಿಗೆ
ಮಗ-ಳಿದ್ದಾಳೆಂದೂ
ಮಗು
ಮಗು-ವನ್ನು
ಮಗು-ವಿಗೆ
ಮಗು-ವಿನ
ಮಗು-ವಿ-ನಂತಹ
ಮಗು-ವಿ-ನಂತೆ
ಮಗು-ವಿನಲ್ಲಿ-ರುವ
ಮಗುವು
ಮಗುವೂ
ಮಗುವೆ
ಮಗೂ
ಮಗ್ಗು-ಲಲ್ಲಿ
ಮಗ್ನನಾ-ಗು-ವಂತೆ
ಮಗ್ನ-ರಾಗಲು
ಮಗ್ನ-ರಾಗಿ
ಮಗ್ನ-ರಾಗಿದ್ದರು
ಮಗ್ನ-ರಾಗಿ-ರುವರು
ಮಗ್ನ-ರಾಗುತ್ತಿದ್ದೆವು
ಮಗ್ನ-ರಾಗುವರು
ಮಗ್ನ-ರಾದರು
ಮಗ್ನರಾ-ದು-ದ-ರಿಂದ
ಮಜಲು-ಗಳು
ಮಟ್ಟಕ್ಕೆ
ಮಟ್ಟದ
ಮಟ್ಟ-ದಲ್ಲಿ
ಮಟ್ಟ-ವನ್ನು
ಮಟ್ಟಿಗೆ
ಮಟ್ಟಿಲು
ಮಠ
ಮಠಕ್ಕಾಗಿ
ಮಠಕ್ಕೂ
ಮಠಕ್ಕೆ
ಮಠಕ್ಕೋಸ್ಕರ
ಮಠ-ಗಳ
ಮಠ-ಗಳನ್ನು
ಮಠ-ಗಳನ್ನೇ
ಮಠ-ಗಳಲ್ಲಿ
ಮಠ-ಗಳಲ್ಲೆಲ್ಲಾ
ಮಠ-ಗಳು
ಮಠ-ಗಿಠ-ಗಳನ್ನು
ಮಠದ
ಮಠ-ದಲ್ಲಿ
ಮಠ-ದಲ್ಲಿದ್ದಾನೆ
ಮಠ-ದಲ್ಲಿಯೂ
ಮಠ-ದಲ್ಲಿಯೆ
ಮಠ-ದಲ್ಲಿಯೇ
ಮಠ-ದಲ್ಲಿ-ರುವ-ವ-ರಿಗೆ
ಮಠ-ದಲ್ಲಿ-ರುವ-ವ-ರೆಲ್ಲ
ಮಠ-ದಲ್ಲೇ
ಮಠ-ದ-ವರೇ
ಮಠ-ದಿಂದ
ಮಠ-ದೊ-ಡನೆ
ಮಠ-ದೊ-ಳಕ್ಕೆ
ಮಠಪ್ರ-ದೇಶ-ದಲ್ಲಿ
ಮಠ-ವನ್ನು
ಮಠ-ವನ್ನೆಲ್ಲಾ
ಮಠ-ವನ್ನೇ
ಮಠ-ವಿ-ದೆ-ಯಲ್ಲಾ
ಮಠ-ವಿದ್ದು-ದನ್ನು
ಮಠವೂ
ಮಠವೇ
ಮಠ-ವೊಂದು
ಮಠಾಧಿ-ಪತಿ-ಗಳು
ಮಠಾಧ್ಯಕ್ಷರು
ಮಡಕೆ-ಯನ್ನು
ಮಡಿ
ಮಡಿಕೆ
ಮಡಿ-ದಿದ್ದಾನೆ
ಮಡಿ-ಯವರ
ಮಡಿ-ಯು-ವುದು
ಮಡಿ-ಲಿಗೆ
ಮಡಿ-ಲಿನಲಿ
ಮಢಿತ್ವಾ
ಮಣ
ಮಣಿದು
ಮಣಿಯದ
ಮಣಿ-ಯ-ಲಿಲ್ಲ
ಮಣಿ-ಯುತ್ತಿ-ರ-ಲಿಲ್ಲ
ಮಣಿವ
ಮಣಿ-ಸು-ವುದು
ಮಣ್ಣಿನ
ಮಣ್ಣಿ-ನಲ್ಲಿ
ಮಣ್ಣಿ-ನಿಂದಲೇ
ಮಣ್ಣು
ಮಣ್ಣೇ-ನಾದರೂ
ಮಣ್ದ
ಮತ
ಮತಕ್ಕೂ
ಮತಕ್ಕೆ
ಮತ-ಗಳ
ಮತ-ಗಳನ್ನು
ಮತ-ಗಳಿಗೆಲ್ಲ
ಮತ-ಗಳು
ಮತ-ಗಳೆ-ರಡೂ
ಮತ-ಗಳೇಕೆ
ಮತ-ಗಿತ-ಗಳ
ಮತ-ತತ್ತ್ವ
ಮತದ
ಮತ-ದಲ್ಲಿ
ಮತದ-ವ-ರಿಗೆ
ಮತ-ಪಥ-ಗಳೆಲ್ಲ-ದರ
ಮತಪ್ರ-ಚಾರ-ಕರ
ಮತ-ಭೇದ
ಮತಭ್ರಾಂತ-ರಾ-ಗಿ-ರುವರು
ಮತಭ್ರಾಂತಿ
ಮತಭ್ರಾಂತಿ-ಯಿಂದ
ಮತಭ್ರಾಂತಿಯೇ
ಮತ-ವನ್ನು
ಮತ-ವನ್ನೂ
ಮತ-ವನ್ನೇ
ಮತ-ವಾದ
ಮತ-ವಾದ-ಗಳೂ
ಮತ-ವಾದಿಯೂ
ಮತವು
ಮತ-ವೇ-ನೆಂದರೆ
ಮತಾಂತರ-ಗೊಂಡ-ನೆಂಬುದೆಲ್ಲಾ
ಮತಾಂತರ-ಗೊಂಡಿದ್ದನ್ನು
ಮತಾಂತರ-ಗೊಳಿಸಿ
ಮತಾಂತರ-ವನ್ನೇ
ಮತಾಚ-ರಣೆ-ಗಳಿ-ಗಿಂತಲೂ
ಮತಾನು-ಸಾರ-ವಾಗಿ
ಮತಾ-ಮತ್
ಮತಾವ-ಲಂಬಿ-ಗಳು
ಮತಾವ-ಲಂಬಿ-ಯಾಗಿ-ರ-ಲಿಲ್ಲ
ಮತಿ
ಮತಿ-ಗತಿ-ಗಳನ್ನು
ಮತಿ-ರಾ-ಪನೇಯಾ
ಮತಿರ್ಜ-ಗದೇಕಧಾತ್ರೀಂ
ಮತೀಯ
ಮತೀಯ-ರಿಂದ
ಮತೀಯ-ರೇನು
ಮತು
ಮತೊಂದು
ಮತೊಮ್ಮೆ
ಮತ್
ಮತ್ತ
ಮತ್ತ-ದುವೆ
ಮತ್ತ-ರಾಗಿ
ಮತ್ತ-ರಾಗಿ-ಬಿಟ್ಟಿದ್ದರು
ಮತ್ತಷ್ಟು
ಮತ್ತಾ-ರಿಂದ
ಮತ್ತಾರಿ-ಗಾ-ದರೂ
ಮತ್ತಾ-ರಿಗೂ
ಮತ್ತಾರು
ಮತ್ತಾರೂ
ಮತ್ತಾವ
ಮತ್ತಾ-ವು-ದಕ್ಕೂ
ಮತ್ತಾ-ವು-ದನ್ನು
ಮತ್ತಾ-ವು-ದ-ರಿಂದಲೂ
ಮತ್ತಾ-ವು-ದಾದರೂ
ಮತ್ತಾವು-ದಾದ-ರೊಂದರ
ಮತ್ತಾ-ವುದು
ಮತ್ತಾ-ವುದೂ
ಮತ್ತಾ-ವುದೋ
ಮತ್ತಿ-ತರ
ಮತ್ತಿ-ತರರು
ಮತ್ತಿ-ನಿಂದ
ಮತ್ತಿನ್ನೇನು
ಮತ್ತಿಬ್ಬರು
ಮತ್ತಿಳಿ-ಯು-ವುದೊ
ಮತ್ತು
ಮತ್ತೂ
ಮತ್ತೆ
ಮತ್ತೆಂದೂ
ಮತ್ತೆಲ್ಲಿ
ಮತ್ತೆಲ್ಲಿಗೊ
ಮತ್ತೆಲ್ಲಿಯೂ
ಮತ್ತೇ-ನನ್ನೂ
ಮತ್ತೇ-ನಾದ-ರೇ-ನಂತೆ
ಮತ್ತೇ-ನಿದೆ
ಮತ್ತೇನು
ಮತ್ತೇನೂ
ಮತ್ತೇನೆನ್ನ-ಬ-ಹುದು
ಮತ್ತೇನೋ
ಮತ್ತೇರಿದ
ಮತ್ತೇರಿದೆಯೋ
ಮತ್ತೊಂದನ್ನು
ಮತ್ತೊಂದರ
ಮತ್ತೊಂದ-ರಲ್ಲಿ
ಮತ್ತೊಂದಿತ್ತು
ಮತ್ತೊಂದಿಲ್ಲ
ಮತ್ತೊಂದು
ಮತ್ತೊಂದು-ಕಡೆ
ಮತ್ತೊಂದೇ-ನೆಂದರೆ
ಮತ್ತೊಬ್ಬ
ಮತ್ತೊಬ್ಬನ
ಮತ್ತೊಬ್ಬ-ನನ್ನು
ಮತ್ತೊಬ್ಬ-ನಲ್ಲಿ
ಮತ್ತೊಬ್ಬ-ನಿಗೆ
ಮತ್ತೊಬ್ಬ-ನಿದ್ದ-ನವ
ಮತ್ತೊಬ್ಬನೇ
ಮತ್ತೊಬ್ಬರ
ಮತ್ತೊಬ್ಬ-ರನ್ನು
ಮತ್ತೊಬ್ಬ-ರಾರೋ
ಮತ್ತೊಬ್ಬ-ರಿಂದ
ಮತ್ತೊಬ್ಬ-ರಿ-ಗಾಗಿ
ಮತ್ತೊಬ್ಬ-ರಿಗೆ
ಮತ್ತೊಬ್ಬ-ರಿದ್ದಾರೆಯೆ
ಮತ್ತೊಬ್ಬರು
ಮತ್ತೊಬ್ಬಾತ
ಮತ್ತೊಮ್ಮೆ
ಮತ್ಯಾರಿಗೇ
ಮತ್ಯಾವ
ಮತ್ಯಾ-ವಾಗಲೂ
ಮತ್ಸ್ಯಾವ-ತಾರ-ದಿಂದ
ಮಥುರ
ಮದ
ಮದರ್
ಮದಿರೆ-ಯನು
ಮದಿರೆ-ಯುಕ್ಕು-ತಿದೆ
ಮದುವೆ
ಮದುವೆ-ಗಳು
ಮದುವೆ-ಮಾ-ಡದೆ
ಮದುವೆ-ಮಾಡಿ-ಕೊಳ್ಳಿ
ಮದುವೆ-ಮಾಡು-ವುದು
ಮದುವೆಯ
ಮದುವೆ-ಯನ್ನು
ಮದುವೆ-ಯಾಗ-ಕೂಡ-ದೆಂಬ
ಮದುವೆ-ಯಾ-ಗದ
ಮದುವೆ-ಯಾಗ-ದಿರು-ವ-ವ-ರನ್ನು
ಮದುವೆ-ಯಾಗ-ಬೇಕೆಂದೆ
ಮದುವೆ-ಯಾ-ಗಲಿಚ್ಛಿ-ಸು-ವು-ದಿಲ್ಲ
ಮದುವೆ-ಯಾ-ಗಲೇ-ಬೇಕು
ಮದುವೆ-ಯಾಗಿ
ಮದುವೆ-ಯಾಗಿದ್ದ-ರೇ-ನಂತೆ
ಮದುವೆ-ಯಾಗಿ-ರಲಿ
ಮದುವೆ-ಯಾಗಿ-ರು-ವು-ದ-ರಿಂದ
ಮದುವೆ-ಯಾಗುವ
ಮದುವೆ-ಯಾಗು-ವುದಕ್ಕೆ
ಮದುವೆ-ಯಾದ
ಮದುವೆ-ಯಾದರೆ
ಮದುವೆ-ಯಾದ-ವರ
ಮದುವೆಯೇ
ಮದ್ದಲೆಯ
ಮದ್ದಿಗೆ
ಮದ್ಯ
ಮದ್ರಾಸಿನ
ಮದ್ರಾ-ಸಿ-ನಲ್ಲಿ
ಮದ್ರಾಸಿನಲ್ಲಿದ್ದಾಗ
ಮದ್ರಾಸಿನ-ವ-ನಿಗೆ
ಮದ್ರಾಸಿ-ನಿಂದ
ಮದ್ರಾಸು
ಮಧು
ಮಧು-ಕರ
ಮಧು-ತರ
ಮಧುರ
ಮಧು-ರ-ಗಾನ-ವ-ಹುದು
ಮಧು-ರ-ತರ
ಮಧು-ರ-ಭಾವ
ಮಧು-ರ-ಭಾವದ
ಮಧು-ರ-ಮಧು-ರ-ವಾಗಿ
ಮಧು-ರ-ಮಪಿ
ಮಧು-ರ-ವಾಗಿ-ರುವ
ಮಧು-ರ-ವೆನಿ-ಸಿ-ದರು
ಮಧು-ವನು
ಮಧು-ವನ್ನು
ಮಧು-ಸೂ-ದ-ನದತ್ತರ
ಮಧ್ಯ
ಮಧ್ಯದ
ಮಧ್ಯ-ದಲಿ
ಮಧ್ಯ-ದಲ್ಲಿ
ಮಧ್ಯ-ದಲ್ಲಿದ್ದ
ಮಧ್ಯ-ದಲ್ಲಿದ್ದು-ಕೊಂಡು
ಮಧ್ಯ-ದಲ್ಲಿಯೂ
ಮಧ್ಯ-ದಲ್ಲಿಯೆ
ಮಧ್ಯ-ದಲ್ಲಿ-ರುವ
ಮಧ್ಯ-ದಲ್ಲಿ-ರು-ವುದೂ
ಮಧ್ಯದಿ
ಮಧ್ಯ-ಭಾಗ-ದಲ್ಲಿ
ಮಧ್ಯಾಹ್ನ
ಮಧ್ಯಾಹ್ನದ
ಮಧ್ಯಾಹ್ನ-ವಾಗಲಿ
ಮಧ್ಯೆ
ಮಧ್ಯೆ-ಮಧ್ಯೆ
ಮಧ್ಯೇ
ಮನ
ಮನಃಕಲ್ಪಿ-ತವೇ
ಮನಃಪೂರ್ವಕ
ಮನಃಪೂರ್ವ-ಕ-ವಾಗಿ
ಮನಃಪೂರ್ವ-ಕ-ವಾದ
ಮನಃಶಕ್ತಿ-ಗಳೆಲ್ಲಾ
ಮನಃಶಕ್ತಿ-ಯಂತೆಯೂ
ಮನ-ಗಂಡ
ಮನ-ಗಳ
ಮನಗಾಣ-ಬ-ಹುದು
ಮನಗೊಟ್ಟು
ಮನದ
ಮನದಟ್ಟಾ-ಯಿತು
ಮನದಟ್ಟು
ಮನದ-ಣಿಯ
ಮನದ-ಲೆ-ಗಳೆಲ್ಲವೂ
ಮನದಿ
ಮನದಿಂ
ಮನನ
ಮನ-ಬುದ್ಧಿ
ಮನಮುಟ್ಟು-ವಂತೆ
ಮನವ
ಮನವ-ರಿಕೆ-ಯಾಗ-ಬೇಕು
ಮನವಿ-ಗಳನ್ನು
ಮನ-ವಿದು
ಮನವು
ಮನ-ವೆಂಬ
ಮನಶ್ಚಕ್ಷು-ವಿನ
ಮನಶ್ಶಕ್ತಿ
ಮನಶ್ಶುದ್ಧಿ-ಯಾಗಿ
ಮನಸ
ಮನಸಾ
ಮನಸಿಗು
ಮನ-ಸಿಗೆ
ಮನಸು
ಮನಸ್ಕ-ರಾದ
ಮನಸ್ತಾಪ
ಮನಸ್ತಾಪಕ್ಕೆ
ಮನಸ್ಸನ್ನು
ಮನಸ್ಸನ್ನೂ
ಮನಸ್ಸನ್ನೆಲ್ಲಾ
ಮನಸ್ಸನ್ನೆಳೆದು
ಮನಸ್ಸಲ್ಲ
ಮನಸ್ಸಾ-ಗು-ವು-ದಿಲ್ಲ-ವೆಂದು
ಮನಸ್ಸಿಗೆ
ಮನಸ್ಸಿಟ್ಟರೆ
ಮನಸ್ಸಿಟ್ಟ-ವ-ರಿಗೆ
ಮನಸ್ಸಿಟ್ಟು
ಮನಸ್ಸಿಡು-ವಂತೆ
ಮನಸ್ಸಿಡುವೆನೋ
ಮನಸ್ಸಿದ್ದರೆ
ಮನಸ್ಸಿನ
ಮನಸ್ಸಿನಂತೆ
ಮನಸ್ಸಿ-ನಲ್ಲಿ
ಮನಸ್ಸಿನಲ್ಲಿಡಿ
ಮನಸ್ಸಿನಲ್ಲಿದ್ದ
ಮನಸ್ಸಿನಲ್ಲಿಯೂ
ಮನಸ್ಸಿನಲ್ಲಿಯೆ
ಮನಸ್ಸಿನಲ್ಲಿಯೇ
ಮನಸ್ಸಿನಲ್ಲಿ-ರುವ
ಮನಸ್ಸಿನಲ್ಲಿ-ರು-ವು-ದನ್ನು
ಮನಸ್ಸಿನಲ್ಲೆಲ್ಲಾ
ಮನಸ್ಸಿನಿಂದ
ಮನಸ್ಸಿನೊ-ಡನೆ
ಮನಸ್ಸಿನೊಳ-ಗಿನ
ಮನಸ್ಸಿನೊಳ-ಗಿ-ರುವುದನ್ನೆಲ್ಲಾ
ಮನಸ್ಸು
ಮನಸ್ಸು-ಕೊಟ್ಟು
ಮನಸ್ಸು-ಗಳಲ್ಲಿ
ಮನಸ್ಸು-ಗಳೆಂಬ
ಮನಸ್ಸುಳ್ಳ-ವರ
ಮನಸ್ಸೂ
ಮನಸ್ಸೆ
ಮನಸ್ಸೆಂಬ
ಮನಸ್ಸೆನ್ನು-ವುದು
ಮನಸ್ಸೆಲ್ಲಾ
ಮನಸ್ಸೇ
ಮನಸ್ಸೇನೂ
ಮನಿ
ಮನು
ಮನುಜ
ಮನು-ಜ-ಕುಲದಿ
ಮನು-ಜಕ್ರಿಮಿ
ಮನು-ಜ-ನಲಿ
ಮನು-ಜ-ನೊಬ್ಬನೆ
ಮನು-ಜ-ರೂಪವ
ಮನು-ಧರ್ಮ
ಮನು-ಧರ್ಮ-ಶಾಸ್ತ್ರ
ಮನು-ಧರ್ಮ-ಶಾಸ್ತ್ರ-ದಲ್ಲಿ
ಮನು-ಧರ್ಮ-ಶಾಸ್ತ್ರ-ವಾದರೂ
ಮನುಷ್ಯ
ಮನುಷ್ಯ-ಜನಾಂಗದ
ಮನುಷ್ಯ-ಜಾ-ತಿಯ
ಮನುಷ್ಯನ
ಮನುಷ್ಯ-ನಂತಾ-ಗುತ್ತದೆ
ಮನುಷ್ಯ-ನಂತೆಯೇ
ಮನುಷ್ಯ-ನನ್ನು
ಮನುಷ್ಯ-ನಲ್ಲವೇ
ಮನುಷ್ಯ-ನಲ್ಲಿ
ಮನುಷ್ಯ-ನಲ್ಲಿಯೂ
ಮನುಷ್ಯ-ನಲ್ಲಿಯೇ
ಮನುಷ್ಯ-ನಲ್ಲಿ-ರುವ
ಮನುಷ್ಯ-ನಷ್ಟೇ
ಮನುಷ್ಯ-ನಾಗಿದ್ದ
ಮನುಷ್ಯ-ನಾಗಿದ್ದನು
ಮನುಷ್ಯ-ನಾಗುತ್ತಾ-ನೆಂಬ
ಮನುಷ್ಯ-ನಾದರೋ
ಮನುಷ್ಯ-ನಿಂದ
ಮನುಷ್ಯ-ನಿ-ಗಿಂತಲೂ
ಮನುಷ್ಯ-ನಿಗೂ
ಮನುಷ್ಯ-ನಿಗೆ
ಮನುಷ್ಯ-ನಿಗೇಕಿರ-ಬೇಕು
ಮನುಷ್ಯ-ನಿ-ಗೋಸ್ಕರ
ಮನುಷ್ಯ-ನಿದ್ದರೆ
ಮನುಷ್ಯನು
ಮನುಷ್ಯನೂ
ಮನುಷ್ಯ-ನೆಂದು
ಮನುಷ್ಯನೇ
ಮನುಷ್ಯ-ನೇನು
ಮನುಷ್ಯರ
ಮನುಷ್ಯ-ರಂತಲ್ಲ
ಮನುಷ್ಯ-ರನ್ನು
ಮನುಷ್ಯ-ರಲ್ಲಿ
ಮನುಷ್ಯ-ರಾಗಿ
ಮನುಷ್ಯ-ರಾಗಿದ್ದೀರಾ
ಮನುಷ್ಯ-ರಾಗುವ
ಮನುಷ್ಯ-ರಾಗು-ವು-ದನ್ನು
ಮನುಷ್ಯ-ರಿಗೂ
ಮನುಷ್ಯ-ರಿದ್ದಾರೆ
ಮನುಷ್ಯ-ರಿ-ರುವರೋ
ಮನುಷ್ಯರು
ಮನುಷ್ಯರೆ
ಮನುಷ್ಯ-ರೆಂದು
ಮನುಷ್ಯ-ರೆಂದು-ಕೊಂಡಿದ್ದೀರಾ
ಮನುಷ್ಯರೇ
ಮನುಷ್ಯೇ-ತರ
ಮನೆ
ಮನೆ-ಕಡೆ
ಮನೆ-ಗಳನ್ನು
ಮನೆ-ಗಳಲ್ಲಿ
ಮನೆ-ಗಳು
ಮನೆ-ಗಾಗಿ
ಮನೆ-ಗಾ-ದರೂ
ಮನೆಗೂ
ಮನೆಗೆ
ಮನೆ-ತನ-ದ-ವನು
ಮನೆ-ಬಾಗಿ-ಲಿಗೆ
ಮನೆ-ಮಠ-ಗಳನ್ನೆಲ್ಲಾ
ಮನೆ-ಮನೆಗೂ
ಮನೆ-ಮನೆಯ
ಮನೆ-ಮಾಡಿದೆ
ಮನೆಯ
ಮನೆ-ಯ-ಕಡೆ
ಮನೆ-ಯನ್ನು
ಮನೆ-ಯನ್ನೆಲ್ಲಾ
ಮನೆ-ಯಲ್ಲಿ
ಮನೆ-ಯಲ್ಲಿತ್ತು
ಮನೆ-ಯಲ್ಲಿದ್ದ
ಮನೆ-ಯಲ್ಲಿದ್ದಾಗ
ಮನೆ-ಯಲ್ಲಿದ್ದಾರೆ
ಮನೆ-ಯಲ್ಲಿಯೂ
ಮನೆ-ಯಲ್ಲಿಯೇ
ಮನೆ-ಯಲ್ಲಿ-ರುತ್ತಿದ್ದಾರೆ
ಮನೆ-ಯಲ್ಲೂ
ಮನೆ-ಯ-ವನ
ಮನೆ-ಯ-ವ-ನನ್ನು
ಮನೆ-ಯ-ವ-ರಿಗೆ
ಮನೆ-ಯಾಗ-ಬೇಕು
ಮನೆ-ಯಾಗುವುದು
ಮನೆ-ಯಿಂದ
ಮನೆಯು
ಮನೆ-ಯೆಂದು
ಮನೆ-ಯೆ-ಡೆಗೆ
ಮನೆ-ಯೊ-ಳಕ್ಕೆ
ಮನೆ-ಯೊ-ಳಗೆ
ಮನೇ
ಮನೇರ
ಮನೋಗ-ತವನ್ನ-ರಿತು
ಮನೋಜ್ಞ-ವಾಗಿ
ಮನೋ-ಧರ್ಮ
ಮನೋ-ಧರ್ಮ-ದವ-ರಿ-ಗೆಲ್ಲ
ಮನೋ-ಧರ್ಮ-ವನ್ನು
ಮನೋಪ್ರಪಂಚ-ವನ್ನು
ಮನೋಪ್ರ-ವೃತ್ತಿಯೇ
ಮನೋ-ಭಾವಕ್ಕನು-ಗುಣ-ವಾದ
ಮನೋ-ಭಾವ-ದ-ವರು
ಮನೋ-ಭಾವನೆ
ಮನೋ-ಭಾವ-ನೆ-ಯನ್ನು
ಮನೋ-ಭಾವ-ವನ್ನು
ಮನೋ-ರೂಪ-ವಾದ
ಮನೋ-ವಚನೈಕಾ-ಧಾರ
ಮನೋವಾಕ್ಕಾಯ
ಮನೋವಾಕ್ಕಾಯ-ಗಳಲ್ಲಿ
ಮನೋವಾಕ್ಕಾಯ-ಗಳಿಂದ
ಮನೋವಾಕ್ಕಾಯ-ವಾಗಿ
ಮನೋ-ವೃತ್ತಿಯು
ಮನೋಹರ
ಮನೋಹರ-ವಾಗಿದೆ
ಮನೋಹರ-ವಾಗಿದ್ದಾನೆ
ಮನೋಹರ-ವಾದ
ಮನ್ನಣೆ
ಮನ್ನಿಸಿ
ಮನ್ನಿ-ಸುವು-ದಾಗಿ
ಮನ್ಮಥ
ಮನ್ಮಥ-ನಾಥ
ಮನ್ಮಥ-ಬಾಬು
ಮನ್ಮಥ-ಬಾಬು-ಗಳ
ಮನ್ಮಥ-ಬಾಬು-ಗಳು
ಮಬ್ಬನು
ಮಬ್ಬು-ಕವಿ-ಸುವ
ಮಮ
ಮಮ-ಕಾರ
ಮಮ-ಗತಿ-ತಹವೊ
ಮಮತೆ
ಮಮ-ಶಕ್ತಿ
ಮಮಾದ್ಯಾ
ಮಯಾ
ಮರ
ಮರ-ಗಳ
ಮರ-ಗಳಂತೆ
ಮರ-ಗಳೆಲ್ಲಾ
ಮರ-ಗಿಡ-ಗಳಲ್ಲಿ
ಮರಣ
ಮರಣ-ಗಳ
ಮರಣ-ದಿಂದ
ಮರಣ-ದೆಡೆ-ಗಾಗೆಳೆ-ವುದೆಮ್ಮನು
ಮರಣ-ಮಸ್ತು
ಮರಣ-ವನ್ನೈದು-ವರೊ
ಮರಣ-ವಲ್ಲ-ವಿದು
ಮರಣ-ವಾಗಲಿ
ಮರಣ-ವಿ-ಹುದು
ಮರ-ಣವು
ಮರಣ-ವೆಂಬ
ಮರಣ-ಶಯ್ಯೆ-ಯಲ್ಲಿ
ಮರಣಾ-ಪನ್ನ-ನಾಗಿದ್ದ
ಮರಣೋರ್ಮಿ-ನಾಶಂ
ಮರತೇ-ಬಿಟ್ಟಿತ್ತು
ಮರದ
ಮರದ-ಡಿ-ಯಲ್ಲಿ
ಮರ-ಮಣ್ಣು-ಗಳು
ಮರ-ಮರ-ಗಳ
ಮರ-ಮರಳಿ
ಮರಮೇರ್
ಮರಳಲು
ಮರಳಿ
ಮರಳು-ವುದೆಮ್ಮ
ಮರ-ವಾಗು-ವುದು
ಮರವು-ಮಣ್ಣು
ಮರಾಠಿ
ಮರಿ
ಮರಿ-ಗಳಿ-ಗಾಗಿ
ಮರಿ-ಗಳೂ
ಮರಿಗೆ
ಮರಿ-ಯನ್ನು
ಮರಿ-ಯಾಗಿ
ಮರೀಚಿಕೆ
ಮರೀಚಿಕೆ-ಯನ್ನು
ಮರು
ಮರುಕ
ಮರುಕ್ಷಣ-ದಲ್ಲಿಯೇ
ಮರುಕ್ಷಣವೇ
ಮರು-ಗಳಿಗೆ-ಯಲ್ಲೇ
ಮರು-ಗಳಿಗೆಯೇ
ಮರುಗು
ಮರು-ಗುತ್ತಾ
ಮರು-ಚಣ-ದಲಿ
ಮರು-ಜೀವ
ಮರು-ದನಿಯ
ಮರು-ದಿನ
ಮರು-ದಿ-ನವೇ
ಮರು-ಧರೆ
ಮರು-ಮರೀಚಿಕೆ
ಮರು-ಮರೀಚಿಕೆ-ಯಂತೆ
ಮರು-ಮಾತಿಲ್ಲದೆ
ಮರು-ಮಾತಿಲ್ಲವು
ಮರುಳ
ಮರು-ಳನ
ಮರು-ಳ-ನಾ-ವನು
ಮರು-ಳನೆ
ಮರು-ಳ-ನೊಲು
ಮರು-ಳರ
ಮರು-ಳ-ರಾಟಿ-ಗೆಯಿ-ದನು
ಮರು-ಳಾಗಿ
ಮರು-ಳಾದ
ಮರುಳು
ಮರು-ಳು-ಲೀಲೆಯೆ
ಮರೆತಂತೆ
ಮರೆ-ತರು
ಮರೆ-ತಾ-ನಂದ-ಜಲದಿ
ಮರೆತಿದ್ದ
ಮರೆತು
ಮರೆತು-ಬಿಡುತ್ತಾರೆಯೊ
ಮರೆತು-ಹೋಗಿ
ಮರೆತು-ಹೋಗಿದೆ
ಮರೆತೇ-ಹೋಗಿ-ರು-ವಂತೆ
ಮರೆಯ-ದಂತೆ
ಮರೆ-ಯದಿ-ರೆಂದಿಗು
ಮರೆ-ಯದಿ-ರೋಣ
ಮರೆ-ಯದೆ
ಮರೆಯ-ಬೇಕಾ-ಗುತ್ತದೆ
ಮರೆಯ-ಬೇಡ
ಮರೆಯ-ಬೇಡಿ
ಮರೆ-ಯಾಗಿದ್ದರು
ಮರೆಯಿಂ
ಮರೆ-ಯಿರಿ
ಮರೆ-ಯುತ್ತ
ಮರೆ-ಯು-ವಂತೆ
ಮರೆ-ಯುವನು
ಮರೆ-ಯುವ-ವ-ರಲ್ಲ
ಮರೆ-ಯು-ವು-ದಿಲ್ಲ
ಮರೆ-ಯು-ವುದು
ಮರೆಸಿ-ಬಿಡುತ್ತಿದ್ದೆ
ಮರ್ಕಟ
ಮರ್ತ್ಯ-ಜೀವನ-ದಿಂದ
ಮರ್ತ್ಯ-ಲೋಕ-ದಲ್ಲಿ
ಮರ್ತ್ಯಾ-ಮೃತಂ
ಮರ್ಮ
ಮರ್ಮಚ್ಛೇದ
ಮರ್ಮರ
ಮರ್ಮ-ವನ್ನು
ಮರ್ಮಾ-ಘಾತದಿ
ಮರ್ಯಾ-ದಾಸಂಪನ್ನ
ಮರ್ಯಾದೆ-ಗಳಿಗೆ
ಮರ್ಯಾದೆ-ಗಳು
ಮರ್ಯಾದೆಯ
ಮರ್ಯಾದೆ-ಯಷ್ಟೇ
ಮರ್ಯಾದೆ-ಯಿಂದ
ಮರ್ಯಾದೆಯೂ
ಮಲಗ-ಬೇಕು
ಮಲಗಲು
ಮಲಗಿ
ಮಲಗಿಕೊ
ಮಲಗಿ-ಕೊಂಡಿದ್ದ
ಮಲಗಿ-ಕೊಂಡು
ಮಲಗಿ-ಕೊಂಡೂ
ಮಲಗಿ-ಕೊಂಡೇ
ಮಲಗಿ-ಕೊಳ್ಳುತ್ತೇನೆ
ಮಲಗಿತ್ತು
ಮಲಗಿ-ದರು
ಮಲಗಿ-ದರೆ
ಮಲಗಿ-ದು-ದ-ರಿಂದ
ಮಲಗಿದ್ದ
ಮಲಗಿದ್ದಾನೆ
ಮಲಗಿ-ಬಿಟ್ಟೆ
ಮಲಗಿ-ರ-ಬ-ಹುದು
ಮಲಗಿ-ರುವ
ಮಲಗಿ-ರುವ-ರೆಂದು
ಮಲಗಿ-ರುವಾಗ
ಮಲಗಿವೆ
ಮಲಗಿ-ಸುವ
ಮಲಗಿಹ
ಮಲಗುತ್ತಾನೆ
ಮಲಗುತ್ತಿದ್ದಾರೆ
ಮಲಗುವ
ಮಲಗು-ವರು
ಮಲಗುವ-ವ-ರೆಗೂ
ಮಲಗು-ವು-ದ-ರಿಂದ
ಮಲಗುವು-ದಾ-ಗಲಿ
ಮಲಗು-ವುದು
ಮಲ-ಬದ್ಧ-ತೆಯ-ವ-ನಿಗೆ
ಮಲಯ
ಮಲಯ-ಮಾರುತನ
ಮಳೆ
ಮಳೆ-ಗರಿ
ಮಳೆ-ಗರೆ-ಯುತ-ಲಿದೆ
ಮಳೆ-ಗಳಿಗೆಲ್ಲ
ಮಳೆ-ಗಾಲ-ದಲ್ಲಿ-ರುವ
ಮಳೆ-ನೀ-ರಿ-ಗಾಗಿ
ಮಳೆಯ
ಮಸ-ಕಾ-ದರೂ
ಮಸಕು
ಮಸಕು-ನುಡಿ-ಯೊಳೊ
ಮಸಣ
ಮಸಣ-ದಲಿ
ಮಸ-ಣವೆ
ಮಸೀದಿ
ಮಸೀದಿ-ಗಳಲ್ಲಿ
ಮಸುಳ-ದಲ್ಲಿ
ಮಸ್ತ-ಕದ
ಮಸ್ತಿಷ್ಕದ
ಮಸ್ತಿಷ್ಕ-ವಿತ್ತು
ಮಹ-ಡಿಗೆ
ಮಹಡಿಯ
ಮಹಡಿಯ-ಮೇಲೆ
ಮಹಡಿ-ಯಿಂದ
ಮಹ-ತಾಯಿ
ಮಹತಿ-ನಲಿ
ಮಹ-ತಿಯನ್ನರ-ಸುವ-ವ-ರಿಗೆ
ಮಹ-ತಿಯಿ-ರು-ವುದು
ಮಹತ್
ಮಹತ್ಕಾರ್ಯ-ಗಳ
ಮಹತ್ಕಾರ್ಯ-ವನ್ನೂ
ಮಹತ್ಕಾರ್ಯವೂ
ಮಹತ್ತರ-ವಾದ
ಮಹತ್ತರ-ವಾ-ದುದು
ಮಹತ್ತಾದ
ಮಹತ್ತಿಗೆ
ಮಹತ್ತ್ವದ
ಮಹತ್ತ್ವ-ವನ್ನು
ಮಹತ್ತ್ವ-ವುಳ್ಳ
ಮಹತ್ವ
ಮಹತ್ವ-ಪೂರಿ-ತ-ವಾ-ದು-ದೆಂದು
ಮಹತ್ವ-ಪೂರ್ಣ-ವಾದ
ಮಹತ್ವ-ವನ್ನು
ಮಹತ್ವ-ವಾದುದಾ-ವುದೂ
ಮಹತ್ವವೂ
ಮಹತ್ವ-ವೇ-ನಾದರೂ
ಮಹದದ್ಭುತಂ
ಮಹದುದಾರ-ಭಾ-ವವೂ
ಮಹದುಪ-ಕಾರಿ
ಮಹಮ್ಮದನ
ಮಹಮ್ಮದೀ-ಯರ
ಮಹಮ್ಮದೀ-ಯರು
ಮಹಮ್ಮದ್
ಮಹರ್ಷಿ-ಗಳ
ಮಹಾ
ಮಹಾ-ಕವಿ
ಮಹಾ-ಕವಿಪ್ರ-ಯೋಗ
ಮಹಾ-ಕಾಯ
ಮಹಾ-ಕಾರ್ಯ
ಮಹಾ-ಕಾಳಿ
ಮಹಾ-ಕಾಳಿಯೇ
ಮಹಾ-ಕಾವ್ಯ
ಮಹಾ-ಕಾಶ-ವಿದ್ದ
ಮಹಾ-ಗರ್ಜ-ನೆಯು
ಮಹಾ-ಘೋರ
ಮಹಾ-ಚಲ
ಮಹಾ-ತತ್ತ್ವ-ವನ್ನು
ಮಹಾತ್ಮನ
ಮಹಾತ್ಮನನ್ನಾಗಿ
ಮಹಾತ್ಮನಲ್ಲ
ಮಹಾತ್ಮ-ನಾಗು-ವನು
ಮಹಾತ್ಮನಾದ
ಮಹಾತ್ಮನಿರ-ಬೇಕು
ಮಹಾತ್ಮನೆ
ಮಹಾತ್ಮ-ನೆಂದು
ಮಹಾತ್ಮನೇ
ಮಹಾತ್ಮರ
ಮಹಾತ್ಮ-ರನ್ನು
ಮಹಾತ್ಮರಾದ
ಮಹಾತ್ಮರಿಗೆ
ಮಹಾತ್ಮರು
ಮಹಾತ್ಮರೂ
ಮಹಾತ್ಮರೇ
ಮಹಾಮ್ಯ-ವನ್ನು
ಮಹಾಮ್ಯ-ವಿ-ರುವುದ-ರಲ್ಲಿ
ಮಹಾತ್ಯಾಗ
ಮಹಾತ್ಯಾಗಿ-ಗಳೊ
ಮಹಾ-ದೇವ
ಮಹಾ-ದೇವನ
ಮಹಾ-ದೇವ-ನೆಂಬ
ಮಹಾ-ದೇವ-ಸಂಜ್ಞಃ
ಮಹಾ-ದೈತ್ಯ
ಮಹಾಧ್ಯೇಯ-ವನ್ನು
ಮಹಾ-ನಂದ-ದಲ್ಲಿ
ಮಹಾ-ನಾಗ್
ಮಹಾ-ನಾದ
ಮಹಾ-ನಿರ್ವಾಣ
ಮಹಾನ್
ಮಹಾನ್ತಂ
ಮಹಾ-ಪಂಡಿತ
ಮಹಾ-ಪರಿಷ್ಕೃತ-ವಾದ
ಮಹಾ-ಪಾಪ-ಗಳು
ಮಹಾ-ಪಾಪಿ
ಮಹಾ-ಪಾಪಿಗೂ
ಮಹಾ-ಪುರುಷ
ಮಹಾ-ಪುರುಷನ
ಮಹಾ-ಪುರುಷ-ನನ್ನು
ಮಹಾ-ಪುರುಷನು
ಮಹಾ-ಪುರುಷ-ನೊಬ್ಬ
ಮಹಾ-ಪುರುಷರ
ಮಹಾ-ಪುರುಷ-ರಲ್ಲಿ
ಮಹಾ-ಪುರುಷ-ರಿದ್ದರು
ಮಹಾ-ಪುರುಷರು
ಮಹಾ-ಪುರುಷ-ರೆಂದು
ಮಹಾ-ಪುರುಷ-ರೆಲ್ಲಾ
ಮಹಾಪ್ರತಿ-ಭೆ-ಯಿಂದ
ಮಹಾಪ್ರಭು-ವಿನ
ಮಹಾಪ್ರಳಯ
ಮಹಾಪ್ರ-ವಾದಿ
ಮಹಾಪ್ರೇಮ-ದಿಂದ
ಮಹಾ-ಬೀಜ
ಮಹಾ-ಭಾರತ
ಮಹಾ-ಭಾರ-ತದ
ಮಹಾ-ಭಾವ-ನೆ-ಗಳ
ಮಹಾ-ಭೂತ
ಮಹಾ-ಮನೀ-ಷಿ-ಯಾದ
ಮಹಾ-ಮ-ಹಿಮ-ರಾಗಿದ್ದರೋ
ಮಹಾ-ಮಾತೆಯ
ಮಹಾ-ಮಾತೆ-ಯನ್ನೇ
ಮಹಾ-ಮಾಯಾ
ಮಹಾ-ಮಾಯೆ
ಮಹಾ-ಮಾಯೆಯ
ಮಹಾ-ಮಾರಿ
ಮಹಾ-ಮಾರಿ-ಯಿಂದ
ಮಹಾ-ಯಾಗ-ವನ್ನು
ಮಹಾ-ರಣ
ಮಹಾ-ರಾಜ-ನಿಗೆ
ಮಹಾ-ರಾಜ-ರನ್ನು
ಮಹಾ-ರಾಜ್
ಮಹಾ-ಲಯ
ಮಹಾ-ಲಯ-ವನು
ಮಹಾ-ವಾರುಣಿ-ಯೋಗ-ದಲ್ಲಿ
ಮಹಾ-ವಿದ್ಯಾ-ವಂತ-ರೆಂದು
ಮಹಾ-ವೀರ
ಮಹಾ-ವೀರನ
ಮಹಾ-ವೀರ-ನಂತೆ
ಮಹಾ-ವೀರ-ನನ್ನು
ಮಹಾ-ವೀರ-ನಾದ-ವನ
ಮಹಾ-ವೀರರ
ಮಹಾವ್ಯಕ್ತಿ-ಗಳು
ಮಹಾವ್ಯಕ್ತಿಯ
ಮಹಾ-ಶಕ್ತ-ಳಾದ
ಮಹಾ-ಶಕ್ತಿ
ಮಹಾ-ಶಕ್ತಿಯ
ಮಹಾ-ಶಕ್ತಿಯು
ಮಹಾ-ಶಯ
ಮಹಾ-ಶ-ಯನು
ಮಹಾ-ಶಯರ
ಮಹಾ-ಶಯ-ರನ್ನು
ಮಹಾ-ಶಯ-ರಿಗೆ
ಮಹಾ-ಶಯ-ರಿದ್ದೆಡೆಗೆ
ಮಹಾ-ಶಯರು
ಮಹಾ-ಶಯರೆ
ಮಹಾ-ಶಯರೇ
ಮಹಾ-ಶಾಂತ
ಮಹಾ-ಶಿ-ವನ
ಮಹಾ-ಶೂನ್ಯ
ಮಹಾ-ಶೂನ್ಯ-ದಲ್ಲಿತ್ತೊ
ಮಹಾ-ಶೂನ್ಯ-ದೊಳು
ಮಹಾ-ಶೂನ್ಯ-ವನ್ನು
ಮಹಾ-ಶೂ-ರರು
ಮಹಾ-ಸಂಘದ
ಮಹಾ-ಸಂದಿಗ್ಧ
ಮಹಾ-ಸಮಾಧಿ-ಯಲ್ಲಿ
ಮಹಾ-ಸಮುದ್ರದ
ಮಹಾ-ಸಾಗರದ
ಮಹಾ-ಸಾಗರದಿ
ಮಹಾ-ಸಾಮ್ರಾಟ
ಮಹಾಸ್ವಾಮಿ
ಮಹಿಮರ
ಮಹಿಮೆ
ಮಹಿಮೆ-ಗಳನು
ಮಹಿಮೆಯ
ಮಹಿಮೆ-ಯನ್ನು
ಮಹಿಮೆ-ಯಲ್ಲಿ
ಮಹಿಮೆ-ಯಿಂದಲೇ
ಮಹಿಮೆ-ಯಿ-ರು-ವಂತೆ
ಮಹಿಳೆ
ಮಹಿಳೆ-ಯ-ರಷ್ಟೇ
ಮಹಿಳೆ-ಯ-ರಿಗೆ
ಮಹಿಳೆ-ಯರು
ಮಹಿಳೆ-ಯೊಬ್ಬಳಿದ್ದಳು
ಮಹೀಂ
ಮಹೀಯಾನ್
ಮಹೆ
ಮಹೇಂದ್ರ
ಮಹೇಂದ್ರ-ಜಾಲ
ಮಹೇಂದ್ರ-ನಾಥ
ಮಹೇಶ್ವರ
ಮಹೋದ್ದೇಶ-ಗಳನ್ನು
ಮಹೋದ್ದೇಶದ
ಮಹೋದ್ದೇಶ-ವನ್ನು
ಮಹೋದ್ದೇಶ-ವೆಂದು
ಮಹೋನ್ನತ
ಮಾ
ಮಾಂ
ಮಾಂಗಿಛ
ಮಾಂತ್ರಿಕ
ಮಾಂತ್ರಿಕ-ತೆಗೆ
ಮಾಂತ್ರಿಕ-ನಿ-ಹನು
ಮಾಂಸ
ಮಾಂಸ-ಖಂಡ-ಗಳಲ್ಲಿ
ಮಾಂಸ-ಖಂಡ-ಗಳು
ಮಾಂಸ-ಗಳ
ಮಾಂಸ-ಗಳನ್ನು
ಮಾಂಸದ
ಮಾಂಸ-ಪಂಜರ-ವಾದ
ಮಾಂಸ-ಭಕ್ಷಣ
ಮಾಂಸ-ವನ್ನು
ಮಾಂಸ-ವನ್ನೂ
ಮಾಂಸಾ-ಹಾರದ
ಮಾಕ್ಸ್-ಮುಲ್ಲರ್
ಮಾಗಿ
ಮಾಘ-ಶುದ್ಧ
ಮಾಝೇ
ಮಾಡ
ಮಾಡ-ಕೂ-ಡದು
ಮಾಡ-ಗೊಡಿ-ಸದೇ
ಮಾಡ-ತಕ್ಕ
ಮಾಡ-ತೊಡಗಿ-ದನು
ಮಾಡ-ತೊಡಗಿ-ದರು
ಮಾಡದ
ಮಾಡ-ದಿದ್ದರೆ
ಮಾಡ-ದಿದ್ದಲ್ಲಿ
ಮಾಡ-ದಿರ-ಬ-ಹುದು
ಮಾಡ-ದಿ-ರುವಾಗಲೂ
ಮಾಡ-ದಿ-ರು-ವುದು
ಮಾಡದೆ
ಮಾಡದೇ
ಮಾಡ-ಬಯ-ಸುತ್ತೇನೆ
ಮಾಡ-ಬಯ-ಸುವೆ
ಮಾಡ-ಬಲ್ಲ
ಮಾಡ-ಬಲ್ಲದು
ಮಾಡ-ಬಲ್ಲ-ನಿದ
ಮಾಡ-ಬಲ್ಲ-ರಾದರೆ
ಮಾಡ-ಬಲ್ಲರು
ಮಾಡ-ಬಲ್ಲರೋ
ಮಾಡ-ಬಲ್ಲಿರಾ
ಮಾಡ-ಬಲ್ಲಿರಿ
ಮಾಡ-ಬಲ್ಲೆ
ಮಾಡ-ಬಲ್ಲೆಯಾ
ಮಾಡ-ಬಲ್ಲೆಯೋ
ಮಾಡ-ಬ-ಹುದು
ಮಾಡ-ಬಹುದೊ
ಮಾಡ-ಬಹುದೋ
ಮಾಡ-ಬಾ-ರದು
ಮಾಡ-ಬಾ-ರದೇಕೆ
ಮಾಡ-ಬೇಕಾಗಿ
ಮಾಡ-ಬೇಕಾಗಿತ್ತು
ಮಾಡ-ಬೇಕಾಗಿದೆ
ಮಾಡ-ಬೇಕಾಗಿ-ದೆ-ಯೆಂದು
ಮಾಡ-ಬೇಕಾಗಿ-ರ-ಲಿಲ್ಲ
ಮಾಡ-ಬೇಕಾಗಿಲ್ಲ
ಮಾಡ-ಬೇಕಾ-ಗು-ವು-ದಿಲ್ಲ-ವೆಂದು
ಮಾಡ-ಬೇಕಾಗು-ವುದು
ಮಾಡ-ಬೇ-ಕಾದ
ಮಾಡ-ಬೇ-ಕಾ-ದರೂ
ಮಾಡ-ಬೇ-ಕಾದರೆ
ಮಾಡ-ಬೇ-ಕಾ-ದುದು
ಮಾಡ-ಬೇ-ಕಾದ್ದು
ಮಾಡ-ಬೇ-ಕಾ-ಯಿತು
ಮಾಡ-ಬೇಕಿಲ್ಲ
ಮಾಡ-ಬೇಕು
ಮಾಡ-ಬೇಕೆಂದಿದ್ದರೊ
ಮಾಡ-ಬೇಕೆಂದಿರುವ
ಮಾಡ-ಬೇಕೆಂದಿರು-ವುದೇ
ಮಾಡ-ಬೇಕೆಂದಿರುವೆ
ಮಾಡ-ಬೇಕೆಂದು
ಮಾಡ-ಬೇಕೆಂಬ
ಮಾಡ-ಬೇಕೆಂಬು-ದನ್ನು
ಮಾಡ-ಬೇಕೆನ್ನಿ-ಸು-ವುದು
ಮಾಡ-ಬೇಕೊ
ಮಾಡ-ಬೇಡ
ಮಾಡ-ಬೇಡಿ
ಮಾಡ-ಲ-ನರ್ಹವಾ-ಗುತ್ತಿದೆ
ಮಾಡ-ಲ-ಶಕ್ತ-ನಾ-ದಲ್ಲಿ
ಮಾಡ-ಲಾಗ-ಲಿಲ್ಲ
ಮಾಡ-ಲಾಗಿತ್ತು
ಮಾಡ-ಲಾ-ಗು-ವು-ದಿಲ್ಲ
ಮಾಡ-ಲಾ-ಗು-ವು-ದಿಲ್ಲವೆ
ಮಾಡ-ಲಾ-ಯಿತು
ಮಾಡ-ಲಾರ
ಮಾಡ-ಲಾ-ರಂಭಿ-ಸಿ-ದನು
ಮಾಡ-ಲಾ-ರಂಭಿ-ಸಿ-ದರು
ಮಾಡ-ಲಾರ-ದವ-ರಾ-ಗಿ-ರುವರೋ
ಮಾಡ-ಲಾರದು
ಮಾಡ-ಲಾರದೆ
ಮಾಡ-ಲಾ-ರರು
ಮಾಡ-ಲಾರರೊ
ಮಾಡ-ಲಾರೆ
ಮಾಡ-ಲಾರೆವು
ಮಾಡಲಿ
ಮಾಡ-ಲಿಕ್ಕಾ-ಗು-ವು-ದಿಲ್ಲವೆ
ಮಾಡ-ಲಿಚ್ಛಿ-ಸುವೆ
ಮಾಡ-ಲಿಚ್ಛಿ-ಸುವೆನು
ಮಾಡ-ಲಿಲ್ಲ
ಮಾಡ-ಲಿಲ್ಲ-ವೆಂದು
ಮಾಡಲು
ಮಾಡ-ಲು-ಪಕ್ರಮಿಸಿ-ದರು
ಮಾಡ-ಲೇನು
ಮಾಡ-ಲೇ-ಬೇಕಾಗಿದೆ
ಮಾಡ-ಲೇ-ಬೇ-ಕಾದರೆ
ಮಾಡ-ಲೇ-ಬೇಕು
ಮಾಡಲ್ಪಟ್ಟ
ಮಾಡಲ್ಪಟ್ಟಿತು
ಮಾಡಲ್ಪಟ್ಟಿದೆ
ಮಾಡಲ್ಪಟ್ಟು
ಮಾಡ-ಹೊರ-ಟಿ-ರು-ವುದೂ
ಮಾಡಿ
ಮಾಡಿಕೊ
ಮಾಡಿ-ಕೊಂಡ
ಮಾಡಿ-ಕೊಂಡಂತೆ
ಮಾಡಿ-ಕೊಂಡನು
ಮಾಡಿ-ಕೊಂಡರು
ಮಾಡಿ-ಕೊಂಡರೆ
ಮಾಡಿ-ಕೊಂಡರೇ
ಮಾಡಿ-ಕೊಂಡ-ವ-ನಾ-ಗಿ-ರಲಿಲ್ಲ
ಮಾಡಿ-ಕೊಂಡಾಗ
ಮಾಡಿ-ಕೊಂಡಿದ್ದ
ಮಾಡಿ-ಕೊಂಡಿದ್ದನ್ನು
ಮಾಡಿ-ಕೊಂಡಿದ್ದರೆ
ಮಾಡಿ-ಕೊಂಡಿದ್ದಾರೆಯೊ
ಮಾಡಿ-ಕೊಂಡಿದ್ದೀರಿ
ಮಾಡಿ-ಕೊಂಡಿದ್ದೆ
ಮಾಡಿ-ಕೊಂಡಿದ್ದೇನೆ
ಮಾಡಿ-ಕೊಂಡಿದ್ದೇವೆ
ಮಾಡಿ-ಕೊಂಡಿರು-ವರೋ
ಮಾಡಿ-ಕೊಂಡಿಲ್ಲ
ಮಾಡಿ-ಕೊಂಡು
ಮಾಡಿ-ಕೊಂಡು-ಬಿಟ್ಟ
ಮಾಡಿ-ಕೊಂಡೆವು
ಮಾಡಿ-ಕೊಟ್ಟರು
ಮಾಡಿ-ಕೊಟ್ಟಿದ್ದ-ರಷ್ಟೆ
ಮಾಡಿ-ಕೊಡ-ಬ-ಹುದು
ಮಾಡಿ-ಕೊಡ-ಬೇಕು
ಮಾಡಿ-ಕೊ-ಡಲು
ಮಾಡಿ-ಕೊಡಿ
ಮಾಡಿ-ಕೊಡುತ್ತಾಳೆ
ಮಾಡಿ-ಕೊಡುತ್ತೇನೆ
ಮಾಡಿ-ಕೊಡು-ವುದಕ್ಕೆ
ಮಾಡಿ-ಕೊಳ್ಳ-ಬಲ್ಲೆ-ಯಾ-ದರೆ
ಮಾಡಿ-ಕೊಳ್ಳ-ಬೇಕಾಗಿದೆ
ಮಾಡಿ-ಕೊಳ್ಳ-ಬೇಕಾಗಿಲ್ಲ
ಮಾಡಿ-ಕೊಳ್ಳ-ಬೇಕು
ಮಾಡಿ-ಕೊಳ್ಳಲಿ
ಮಾಡಿ-ಕೊಳ್ಳಲು
ಮಾಡಿ-ಕೊಳ್ಳುತ್ತಾನೆ
ಮಾಡಿ-ಕೊಳ್ಳುತ್ತಾರೆ
ಮಾಡಿ-ಕೊಳ್ಳುತ್ತಿದ್ದರು
ಮಾಡಿ-ಕೊಳ್ಳುವ
ಮಾಡಿ-ಕೊಳ್ಳುವರು
ಮಾಡಿ-ಕೊಳ್ಳುವ-ವರು
ಮಾಡಿ-ಕೊಳ್ಳು-ವುದಕ್ಕೆ
ಮಾಡಿ-ಕೊಳ್ಳು-ವು-ದನ್ನು
ಮಾಡಿ-ಕೊಳ್ಳು-ವು-ದಿಲ್ಲ
ಮಾಡಿ-ಕೊಳ್ಳುವುದು
ಮಾಡಿ-ಕೊಳ್ಳು-ವು-ದೆಂದು
ಮಾಡಿಕೋ
ಮಾಡಿಟ್ಟು
ಮಾಡಿಟ್ಟು-ಕೊಂಡಿದ್ದಾರೆ
ಮಾಡಿ-ಡಲು
ಮಾಡಿತು
ಮಾಡಿದ
ಮಾಡಿ-ದಂತಾಗು-ವುದು
ಮಾಡಿ-ದಂತೆ
ಮಾಡಿ-ದನು
ಮಾಡಿ-ದ-ನೆಂದು
ಮಾಡಿ-ದನೊ
ಮಾಡಿ-ದ-ಮಾತ್ರ-ದಿಂದ
ಮಾಡಿ-ದ-ಮೇಲೂ
ಮಾಡಿ-ದ-ಮೇಲೆ
ಮಾಡಿ-ದ-ರಾ-ಯಿತು
ಮಾಡಿ-ದರು
ಮಾಡಿ-ದರೂ
ಮಾಡಿ-ದರೆ
ಮಾಡಿ-ದರೊ
ಮಾಡಿ-ದಲ್ಲದೆ
ಮಾಡಿ-ದಳು
ಮಾಡಿ-ದ-ವ-ನಲ್ಲ
ಮಾಡಿ-ದ-ವನು
ಮಾಡಿ-ದ-ವರು
ಮಾಡಿ-ದ-ವರೂ
ಮಾಡಿ-ದ-ವ-ರೆಲ್ಲರೂ
ಮಾಡಿ-ದಷ್ಟೂ
ಮಾಡಿ-ದಾಗ
ಮಾಡಿ-ದಾಗಲೂ
ಮಾಡಿ-ದಾಗ-ಲೆಲ್ಲ
ಮಾಡಿ-ದಾ-ಗಲೇ
ಮಾಡಿ-ದಾರಭ್ಯ
ಮಾಡಿ-ದಿರಿ
ಮಾಡಿ-ದಿ-ರೆಂದೂ
ಮಾಡಿ-ದು-ದನ್ನು
ಮಾಡಿ-ದು-ದ-ರಿಂದ
ಮಾಡಿ-ದುವು
ಮಾಡಿದೆ
ಮಾಡಿ-ದೆನೊ
ಮಾಡಿ-ದೆ-ಯೇನು
ಮಾಡಿ-ದೆವು
ಮಾಡಿ-ದೊಡ-ನೆಯೆ
ಮಾಡಿದ್ದ
ಮಾಡಿದ್ದಕ್ಕೆ
ಮಾಡಿದ್ದನ್ನು
ಮಾಡಿದ್ದ-ರಿಂದ
ಮಾಡಿದ್ದರು
ಮಾಡಿದ್ದರೆ
ಮಾಡಿದ್ದ-ರೆಂದು
ಮಾಡಿದ್ದಾಗ
ಮಾಡಿದ್ದಾಗಿತ್ತು
ಮಾಡಿದ್ದಾನೆ
ಮಾಡಿದ್ದಾರೆ
ಮಾಡಿದ್ದಾರೆಯೆ
ಮಾಡಿದ್ದಾಳೆ
ಮಾಡಿದ್ದಿಲ್ಲ
ಮಾಡಿದ್ದೀರಿ
ಮಾಡಿದ್ದು
ಮಾಡಿದ್ದೇನೆ
ಮಾಡಿದ್ದೇವೆ
ಮಾಡಿ-ನೋಡಿ
ಮಾಡಿ-ನೋಡು
ಮಾಡಿ-ಬಿಟ್ಟನು
ಮಾಡಿ-ಬಿಟ್ಟರು
ಮಾಡಿ-ಬಿಟ್ಟರೆ
ಮಾಡಿ-ಬಿಟ್ಟಿದ್ದಾನೆ
ಮಾಡಿ-ಬಿಟ್ಟಿದ್ದಾರೆ
ಮಾಡಿ-ಬಿಟ್ಟೇನು
ಮಾಡಿ-ಬಿಡ-ಬೇಕೆಂದು
ಮಾಡಿ-ಬಿಡುತ್ತದೆ
ಮಾಡಿ-ಬಿಡುತ್ತಾರೆ
ಮಾಡಿ-ಬಿಡುತ್ತಿದ್ದರು
ಮಾಡಿ-ಬಿಡುತ್ತೇನೆ
ಮಾಡಿ-ಬಿಡುವ
ಮಾಡಿ-ಯಾ-ಗಿದೆ
ಮಾಡಿಯೆ
ಮಾಡಿ-ಯೇನು
ಮಾಡಿಯೊ
ಮಾಡಿ-ರ-ಬ-ಹುದು
ಮಾಡಿ-ರ-ಲಿಲ್ಲ
ಮಾಡಿ-ರುವ
ಮಾಡಿ-ರುವನು
ಮಾಡಿ-ರುವನೊ
ಮಾಡಿ-ರುವರೋ
ಮಾಡಿ-ರು-ವಿರಿ
ಮಾಡಿ-ರು-ವುದಕ್ಕೆ
ಮಾಡಿ-ರು-ವುದಕ್ಕೆಲ್ಲಾ
ಮಾಡಿ-ರು-ವು-ದನ್ನು
ಮಾಡಿ-ರುವುದ-ರಲ್ಲಿ
ಮಾಡಿ-ರು-ವುದು
ಮಾಡಿ-ರುವೆ
ಮಾಡಿ-ರೆಂದೂ
ಮಾಡಿವೆ
ಮಾಡಿ-ಸ-ಬೇಕು
ಮಾಡಿ-ಸ-ಬೇಕೆಂಬು-ದೊಂದು
ಮಾಡಿ-ಸಲು
ಮಾಡಿಸಿ
ಮಾಡಿ-ಸಿ-ಕೊಂಡು
ಮಾಡಿ-ಸಿ-ಕೊಳ್ಳ-ಬೇಕೆಂದು
ಮಾಡಿ-ಸಿ-ಕೊಳ್ಳು-ವು-ದಿಲ್ಲ
ಮಾಡಿ-ಸಿತೊ
ಮಾಡಿ-ಸಿ-ದರು
ಮಾಡಿ-ಸಿದ್ದಾರೆ
ಮಾಡಿಸು
ಮಾಡಿ-ಸುತ್ತ
ಮಾಡಿ-ಸುತ್ತಾ
ಮಾಡಿ-ಸುತ್ತಾನೆ
ಮಾಡಿ-ಸುತ್ತಾ-ರೆಂಬು-ದನ್ನು
ಮಾಡಿ-ಸುತ್ತಾ-ರೆಯೋ
ಮಾಡಿ-ಸುತ್ತಿದ್ದರು
ಮಾಡಿ-ಸುತ್ತಿದ್ದಾನೆ
ಮಾಡಿ-ಸುತ್ತಿದ್ದಾರೆ
ಮಾಡಿ-ಸುತ್ತಿ-ರು-ವ-ನೆಂದು
ಮಾಡಿ-ಸುತ್ತಿರು-ವಳು
ಮಾಡಿ-ಸುತ್ತಿರುವೆ
ಮಾಡಿ-ಸುತ್ತೇನೆ
ಮಾಡಿ-ಸು-ವ-ವನು
ಮಾಡಿ-ಸು-ವುದಕ್ಕೆ
ಮಾಡಿ-ಸು-ವು-ದಿಲ್ಲ
ಮಾಡಿ-ಸೋಣ
ಮಾಡಿ-ಹ-ವಿಲ್ಲಿ
ಮಾಡಿ-ಹೋದನು
ಮಾಡು
ಮಾಡುತ
ಮಾಡು-ತಿ-ರುವೆ
ಮಾಡುತ್ತ
ಮಾಡುತ್ತದೆ
ಮಾಡುತ್ತಲೂ
ಮಾಡುತ್ತಲೇ
ಮಾಡುತ್ತವೆ
ಮಾಡುತ್ತ-ವೆಯೋ
ಮಾಡುತ್ತಾ
ಮಾಡುತ್ತಾನೆ
ಮಾಡುತ್ತಾ-ನೆಯೋ
ಮಾಡುತ್ತಾನೋ
ಮಾಡುತ್ತಾರೆ
ಮಾಡುತ್ತಾ-ರೆಂದೂ
ಮಾಡುತ್ತಾ-ರೆಯೋ
ಮಾಡುತ್ತಾರೋ
ಮಾಡುತ್ತಾಳೋ
ಮಾಡುತ್ತಿದೆ
ಮಾಡುತ್ತಿದ್ದ
ಮಾಡುತ್ತಿದ್ದದ್ದನ್ನು
ಮಾಡುತ್ತಿದ್ದನು
ಮಾಡುತ್ತಿದ್ದ-ನೆಂಬು-ದನ್ನೂ
ಮಾಡುತ್ತಿದ್ದರು
ಮಾಡುತ್ತಿದ್ದರೂ
ಮಾಡುತ್ತಿದ್ದ-ರೆಂದು
ಮಾಡುತ್ತಿದ್ದ-ರೆಂಬು-ದನ್ನು
ಮಾಡುತ್ತಿದ್ದರೋ
ಮಾಡುತ್ತಿದ್ದಳು
ಮಾಡುತ್ತಿದ್ದಾಗ
ಮಾಡುತ್ತಿದ್ದಾನೆಂದು
ಮಾಡುತ್ತಿದ್ದಾನೆಂದುಕೊ
ಮಾಡುತ್ತಿದ್ದಾ-ನೆಯೋ
ಮಾಡುತ್ತಿದ್ದಾ-ರಂತೆ
ಮಾಡುತ್ತಿದ್ದಾರೆ
ಮಾಡುತ್ತಿದ್ದಿರಿ
ಮಾಡುತ್ತಿದ್ದೀರಿ
ಮಾಡುತ್ತಿದ್ದೆ
ಮಾಡುತ್ತಿದ್ದೇನೆ
ಮಾಡುತ್ತಿ-ಯೇನು
ಮಾಡುತ್ತಿಯೋ
ಮಾಡುತ್ತಿರ-ಬೇಕು
ಮಾಡುತ್ತಿರ-ಬೇಕೊ
ಮಾಡುತ್ತಿ-ರ-ಲಿಲ್ಲ
ಮಾಡುತ್ತಿ-ರಲು
ಮಾಡುತ್ತಿರಿ
ಮಾಡುತ್ತಿರುತ್ತಾರೆ
ಮಾಡುತ್ತಿರುವ
ಮಾಡುತ್ತಿರು-ವಂತೆ
ಮಾಡುತ್ತಿರು-ವನು
ಮಾಡುತ್ತಿರು-ವ-ರೆಂದು
ಮಾಡುತ್ತಿರು-ವರೋ
ಮಾಡುತ್ತಿರು-ವ-ವನೂ
ಮಾಡುತ್ತಿರು-ವಾಗ
ಮಾಡುತ್ತಿ-ರು-ವಿರಿ
ಮಾಡುತ್ತಿ-ರು-ವು-ದನ್ನು
ಮಾಡುತ್ತಿರು-ವುದನ್ನೆಲ್ಲಾ
ಮಾಡುತ್ತಿ-ರು-ವು-ದ-ರಿಂದ
ಮಾಡುತ್ತಿ-ರು-ವುದು
ಮಾಡುತ್ತಿರುವೆ
ಮಾಡುತ್ತಿರು-ವೆವು
ಮಾಡುತ್ತಿಲ್ಲ
ಮಾಡುತ್ತಿವೆ
ಮಾಡುತ್ತೀಯೆ
ಮಾಡುತ್ತೀರಿ
ಮಾಡುತ್ತೇನೆ
ಮಾಡುತ್ತೇನೆಂದೂ
ಮಾಡುತ್ತೇವೆಯೋ
ಮಾಡುವ
ಮಾಡು-ವಂತಿತ್ತು
ಮಾಡು-ವಂತಿಲ್ಲ
ಮಾಡು-ವಂತೆ
ಮಾಡು-ವನು
ಮಾಡು-ವ-ನೇನು
ಮಾಡು-ವನೋ
ಮಾಡು-ವರು
ಮಾಡು-ವರೆ
ಮಾಡು-ವ-ರೆಂಬು-ದನ್ನು
ಮಾಡು-ವ-ರೆಂಬುದು
ಮಾಡು-ವರೊ
ಮಾಡು-ವರೋ
ಮಾಡು-ವಳು
ಮಾಡು-ವ-ವ-ನಿಗೆ
ಮಾಡು-ವ-ವನು
ಮಾಡು-ವ-ವರ
ಮಾಡು-ವ-ವ-ರನ್ನು
ಮಾಡು-ವ-ವ-ರಿದ್ದಾರೆ
ಮಾಡು-ವವರು
ಮಾಡು-ವ-ವರೆಗೆ
ಮಾಡು-ವಷ್ಟ-ರಲ್ಲೇ
ಮಾಡು-ವಾಗ
ಮಾಡು-ವಿಕೆ-ಯನ್ನು
ಮಾಡು-ವಿರಿ
ಮಾಡು-ವಿರೋ
ಮಾಡು-ವುದಕ್ಕಾ-ಗಲಿ
ಮಾಡು-ವುದಕ್ಕಾಗಿ
ಮಾಡು-ವುದಕ್ಕಾಗಿಯೂ
ಮಾಡು-ವುದಕ್ಕಾ-ಗು-ವು-ದಿಲ್ಲ
ಮಾಡು-ವುದಕ್ಕಾ-ಗು-ವು-ದಿಲ್ಲ-ವೆಂದು
ಮಾಡು-ವುದಕ್ಕಾ-ಗು-ವು-ದಿಲ್ಲವೋ
ಮಾಡು-ವುದಕ್ಕಾ-ಯಿತೆ
ಮಾಡು-ವುದಕ್ಕಿಂತ
ಮಾಡು-ವು-ದಕ್ಕೂ
ಮಾಡು-ವುದಕ್ಕೆ
ಮಾಡು-ವುದಕ್ಕೆಂದು
ಮಾಡು-ವುದಕ್ಕೆಲ್ಲಾ
ಮಾಡು-ವುದಕ್ಕೋಸ್ಕರ
ಮಾಡು-ವು-ದನ್ನು
ಮಾಡು-ವು-ದರ
ಮಾಡು-ವು-ದ-ರಲ್ಲಿ
ಮಾಡು-ವು-ದ-ರಲ್ಲೇ
ಮಾಡು-ವು-ದ-ರಿಂದ
ಮಾಡು-ವು-ದ-ರಿಂದಲೂ
ಮಾಡು-ವು-ದಾಗಿದೆ
ಮಾಡು-ವು-ದಿಲ್ಲ
ಮಾಡು-ವು-ದಿಲ್ಲ-ದಿದ್ದರೆ
ಮಾಡು-ವುದು
ಮಾಡು-ವುದೂ
ಮಾಡು-ವು-ದೆಲ್ಲಾ
ಮಾಡು-ವುದೇ
ಮಾಡು-ವು-ದೇನು
ಮಾಡು-ವುದೋ
ಮಾಡು-ವುವು
ಮಾಡುವೆ
ಮಾಡು-ವೆನು
ಮಾಡು-ವೆ-ಯೇನು
ಮಾಡು-ವೆವು
ಮಾಡೆಂದು
ಮಾಡೇ
ಮಾಡೇ-ತೀ-ರುವೆ
ಮಾಡೋಣ
ಮಾತಂತೂ
ಮಾತಃ
ಮಾತನಾಡ
ಮಾತನಾಡ-ತೊಡಗಿ-ದರು
ಮಾತನಾ-ಡದ
ಮಾತನಾ-ಡದೆ
ಮಾತನಾಡ-ಬಾ-ರದೆಂದು
ಮಾತನಾಡ-ಬೇಕು
ಮಾತನಾಡ-ಬೇಕೆಂದು
ಮಾತನಾಡ-ಬೇಕೋ
ಮಾತನಾಡ-ಬೇಡ
ಮಾತನಾಡ-ಲಾ-ರಂಭಿ-ಸಿ-ದರು
ಮಾತನಾಡ-ಲಾರದೆ
ಮಾತನಾಡ-ಲಿಲ್ಲ
ಮಾತನಾ-ಡಲು
ಮಾತ-ನಾಡಿ
ಮಾತ-ನಾಡಿ-ಕೊಳ್ಳು-ವುದಕ್ಕೆ
ಮಾತ-ನಾಡಿದ
ಮಾತ-ನಾಡಿ-ದನು
ಮಾತ-ನಾಡಿ-ದರು
ಮಾತ-ನಾಡಿ-ದರೂ
ಮಾತ-ನಾಡಿ-ದರೆ
ಮಾತ-ನಾಡಿ-ದಾಗ
ಮಾತ-ನಾಡಿ-ದಾಗ-ಲೆಲ್ಲ
ಮಾತ-ನಾಡಿ-ದು-ದಿಲ್ಲ
ಮಾತ-ನಾಡಿದ್ದು
ಮಾತ-ನಾಡಿಸಿ
ಮಾತನಾಡುತ್ತ
ಮಾತನಾಡುತ್ತಲೂ
ಮಾತನಾಡುತ್ತಲೇ
ಮಾತನಾಡುತ್ತಾ
ಮಾತನಾಡುತ್ತಾರೆ
ಮಾತನಾಡುತ್ತಿದ್ದಂತೆ
ಮಾತನಾಡುತ್ತಿದ್ದನು
ಮಾತನಾಡುತ್ತಿದ್ದರು
ಮಾತನಾಡುತ್ತಿದ್ದಾಗ
ಮಾತನಾಡುತ್ತಿದ್ದಾನೆ
ಮಾತನಾಡುತ್ತಿದ್ದಾರೆ
ಮಾತನಾಡುತ್ತಿದ್ದಿರಿ
ಮಾತನಾಡುತ್ತಿದ್ದೆವು
ಮಾತನಾಡುತ್ತಿದ್ದೇನೆ
ಮಾತನಾಡುತ್ತಿ-ರಲು
ಮಾತನಾಡುತ್ತಿ-ರು-ವಿರಿ
ಮಾತನಾಡುತ್ತಿ-ರು-ವು-ದನ್ನು
ಮಾತನಾಡುತ್ತಿರುವೆ
ಮಾತನಾಡುತ್ತಿಲ್ಲ
ಮಾತನಾಡುತ್ತೀಯೆ
ಮಾತನಾಡುತ್ತೇನೆ
ಮಾತನಾಡುವ
ಮಾತನಾಡು-ವಂತೆ
ಮಾತನಾಡುವ-ವರು
ಮಾತನಾಡುವಷ್ಟ-ರಲ್ಲೇ
ಮಾತನಾಡು-ವಾಗ
ಮಾತನಾಡು-ವಿರಿ
ಮಾತನಾಡು-ವುದಕ್ಕೆ
ಮಾತನಾಡು-ವು-ದನ್ನು
ಮಾತನಾಡುವುದ-ರಲ್ಲಿಯೂ
ಮಾತನಾಡು-ವು-ದಿಲ್ಲ
ಮಾತನಾಡುವೆ
ಮಾತನಾಡು-ವೆನೊ
ಮಾತನಾಡುವೆ-ಯಲ್ಲವೆ
ಮಾತನಾಡೋಣ
ಮಾತನ್ನಾಡು-ವಿರಿ
ಮಾತನ್ನಾರೂ
ಮಾತನ್ನು
ಮಾತನ್ನೂ
ಮಾತನ್ನೆಲ್ಲಾ
ಮಾತನ್ನೇ
ಮಾತಲ್ಲ
ಮಾತಾಗಿ
ಮಾತಾಜಿ
ಮಾತಾ-ಜಿಗೆ
ಮಾತಾ-ಜಿಯ
ಮಾತಾಜಿ-ಯ-ವರ
ಮಾತಾ-ಜಿಯು
ಮಾತಾಡ-ಬೇಕು
ಮಾತಾಡಲೆ
ಮಾತಾಡಿ
ಮಾತಾಡಿ-ದಂತೆ
ಮಾತಾಡುತ್ತಾ
ಮಾತಾಡುತ್ತಿದ್ದನು
ಮಾತಾಡುತ್ತಿದ್ದು-ದೇನು
ಮಾತಾಡುತ್ತಿ-ರು-ವು-ದನ್ನು
ಮಾತಾಡು-ವಂತಿತ್ತು
ಮಾತಾಡು-ವಂತೆ
ಮಾತಾಡು-ವನು
ಮಾತಾಡು-ವಾಗ
ಮಾತಾಡು-ವುದಕ್ಕೆ
ಮಾತಾದ
ಮಾತಿ-ಗಿಂತಲೂ
ಮಾತಿಗೂ
ಮಾತಿಗೆ
ಮಾತಿನ
ಮಾತಿ-ನಂತೆಯೇ
ಮಾತಿ-ನಲಿ
ಮಾತಿ-ನಲ್ಲಿ
ಮಾತಿನಲ್ಲಿ-ಯಾದರೂ
ಮಾತಿನಲ್ಲಿಯೂ
ಮಾತಿ-ನಿಂದ
ಮಾತಿ-ನಿಂದಲೂ
ಮಾತಿ-ನಿಂದೇನು
ಮಾತಿ-ರಲಿ
ಮಾತಿಲ್ಲ
ಮಾತಿಲ್ಲ-ದಿದ್ದ-ರಿಂದ
ಮಾತಿಲ್ಲದೆ
ಮಾತು
ಮಾತು-ಕತೆ
ಮಾತು-ಕತೆ-ಗಳ
ಮಾತು-ಕತೆ-ಗಳನ್ನು
ಮಾತು-ಕತೆ-ಗಳನ್ನೇ
ಮಾತು-ಕತೆ-ಗಳಲ್ಲಿ
ಮಾತು-ಕತೆ-ಗಳಿಗೆ
ಮಾತು-ಕತೆ-ಗಳು
ಮಾತು-ಕತೆ-ಯಾಗುತ್ತಿತ್ತು
ಮಾತು-ಕತೆ-ಯಾದ
ಮಾತು-ಕಥೆ-ಗಳನ್ನಾಡುತ್ತಿದ್ದರು
ಮಾತು-ಕಥೆ-ಗಳನ್ನಾಡುತ್ತಿದ್ದೇ-ವಲ್ಲಾ
ಮಾತು-ಕಥೆ-ಗಳನ್ನು
ಮಾತು-ಕಥೆ-ಗಳು
ಮಾತು-ಕಥೆ-ಯನ್ನೂ
ಮಾತು-ಕೊಟ್ಟರು
ಮಾತು-ಗಳ
ಮಾತು-ಗಳಂತೂ
ಮಾತು-ಗಳನ್ನಾಡುತ್ತ
ಮಾತು-ಗಳನ್ನಾಡುತ್ತಾ
ಮಾತು-ಗಳನ್ನು
ಮಾತು-ಗಳನ್ನೆಲ್ಲಾ
ಮಾತು-ಗಳಲ್ಲಿ
ಮಾತು-ಗಳಿಂದ
ಮಾತು-ಗಳಿಗೆ
ಮಾತು-ಗಳಿಗೆಲ್ಲಾ
ಮಾತು-ಗಳು
ಮಾತು-ಗಳೆಂದರೆ
ಮಾತು-ಗಳೆಲ್ಲಾ
ಮಾತು-ಗಳೇ
ಮಾತು-ಗಾರ-ನಾಗಿ
ಮಾತು-ಮನ-ಗಳ-ನೆಲ್ಲ
ಮಾತೂ
ಮಾತೃ-ಪೂಜೆ-ಯನ್ನು
ಮಾತೃ-ಭಾವ
ಮಾತೃ-ಭೂಮಿಗೆ
ಮಾತೆ
ಮಾತೆಂದರೆ
ಮಾತೆಗೆ
ಮಾತೆತ್ತಿ-ದರು
ಮಾತೆಯ
ಮಾತೆ-ಯನ್ನು
ಮಾತೆಯು
ಮಾತೆಯೂ
ಮಾತೆ-ಯೆದೆ
ಮಾತೆಯೇ
ಮಾತೆಲ್ಲ
ಮಾತೇ
ಮಾತೇಕೆ
ಮಾತೇನು
ಮಾತೇ-ನೆಂದರೆ
ಮಾತ್ರ
ಮಾತ್ರಕ್ಕೆ
ಮಾತ್ರ-ದಲ್ಲಿ
ಮಾತ್ರ-ದಿಂದ
ಮಾತ್ರ-ದಿಂದಲೆ
ಮಾತ್ರ-ದಿಂದಲೇ
ಮಾತ್ರ-ವನೆ
ಮಾತ್ರ-ವಲ್ಲ
ಮಾತ್ರ-ವೆಂದು
ಮಾತ್ರವೇ
ಮಾತ್ರೇಣ
ಮಾತ್ಸರ್ಯ
ಮಾದರಿಯ
ಮಾದರಿ-ಯಂತೆಯೇ
ಮಾದರಿ-ಯಲ್ಲಿ
ಮಾದರಿ-ಯಾಗ-ಬೇಕು
ಮಾದರಿ-ಯಾಗಿ
ಮಾಧುರ್ಯಕೆ
ಮಾಧುರ್ಯದ
ಮಾಧುರ್ಯ-ವನ್ನೇ
ಮಾಧುರ್ಯವೇ
ಮಾನ
ಮಾನಕ್ಕೆ
ಮಾನದ
ಮಾನವ
ಮಾನ-ವ-ಕೋಟಿಯ
ಮಾನ-ವ-ಕೋಟಿ-ಯನ್ನು
ಮಾನ-ವ-ಕೋಟಿ-ಯಿಂದ
ಮಾನ-ವ-ಗುಣ
ಮಾನ-ವನ
ಮಾನ-ವ-ನನ್ನಾಗಿ
ಮಾನ-ವ-ನನ್ನು
ಮಾನ-ವ-ನಾಗ-ಬೇಕು
ಮಾನ-ವ-ನಾಗಿ-ರು-ವನು
ಮಾನ-ವ-ನಾಗಿ-ರು-ವ-ವನು
ಮಾನ-ವ-ನಾದ
ಮಾನ-ವ-ನಿಗೆ
ಮಾನ-ವನು
ಮಾನ-ವ-ನೆ-ದು-ರಿಗೆ
ಮಾನ-ವನೇ
ಮಾನ-ವರ
ಮಾನ-ವ-ರಂತೆ
ಮಾನ-ವ-ರನ್ನಾಗಿ
ಮಾನ-ವ-ರನ್ನು
ಮಾನ-ವ-ರಲ್ಲಿ
ಮಾನ-ವ-ರಿಗೂ
ಮಾನ-ವ-ರಿಗೆ
ಮಾನ-ವರು
ಮಾನ-ವರೊ
ಮಾನ-ವ-ವಿಧ್ಯೆಯೂ
ಮಾನ-ವ-ಸಾ-ಧನೆಯ
ಮಾನ-ವಸ್ವ-ಭಾವದ
ಮಾನ-ವ-ಹಿ-ತಾರ್ಥ-ವಾಗಿ
ಮಾನ-ವೇ-ತರ
ಮಾನಶುಃ
ಮಾನಸ
ಮಾನ-ಸ-ವಾನ್
ಮಾನ-ಸಿಕ
ಮಾನ-ಸಿಕ-ವಾಗಿ
ಮಾನಸೋ
ಮಾನುಷ
ಮಾನ್ಮನುಷ್ಯೇಷು
ಮಾನ್ಯರು
ಮಾಭೈಃ
ಮಾಮಾ-ಜನ್ಮ
ಮಾಯ
ಮಾಯ-ವಾಗಿ
ಮಾಯ-ವಾಗಿದೆ
ಮಾಯ-ವಾಗುತ್ತದೆ
ಮಾಯ-ವಾ-ಗುತ್ತಿದೆ
ಮಾಯ-ವಾಗುವ
ಮಾಯ-ವಾಗು-ವು-ದನ್ನು
ಮಾಯ-ವಾ-ಗು-ವು-ದಿಲ್ಲ
ಮಾಯ-ವಾಗು-ವುದು
ಮಾಯ-ವಾಗು-ವುವು
ಮಾಯ-ವಾದ
ಮಾಯ-ವಾದರು
ಮಾಯ-ವಾದಾಗ
ಮಾಯ-ವಾ-ಯಿತು
ಮಾಯಾ
ಮಾಯಾ-ಜಾಲ-ದಿಂದ
ಮಾಯಾಪ್ರಪಂಚ
ಮಾಯಾ-ಬಂಧ-ನ-ಗಳಲ್ಲಿಯೂ
ಮಾಯಾ-ಮಾತ್ರ
ಮಾಯಾ-ಮಾಯಾ
ಮಾಯಾ-ಮೋಹ-ಗಳ
ಮಾಯಾರ
ಮಾಯಾ-ರಾ-ಹಿತ್ಯ
ಮಾಯಾ-ಶಕ್ತಿ-ಯಿಂದ
ಮಾಯಾ-ಸನೆ
ಮಾಯೆ
ಮಾಯೆಯ
ಮಾಯೆ-ಯನ್ನಪ್ಪಿ-ಕೊಂಡು
ಮಾಯೆ-ಯನ್ನು
ಮಾಯೆ-ಯಲ್ಲಿ
ಮಾಯೆ-ಯಿಂದ
ಮಾಯೆಯು
ಮಾಯೆ-ಯೆಂದು
ಮಾಯೆ-ಯೆನ್ನು-ವುದು
ಮಾಯೆ-ಯೊ-ಳಗೆ
ಮಾರ-ಕ-ವಾಗಿದೆ
ಮಾರಕ-ವಾಗುತ್ತ-ದೆ-ಒಬ್ಬ
ಮಾರನು
ಮಾರ-ನೆಂದರೆ
ಮಾರ-ನೆಯ
ಮಾರವಾಡಿ
ಮಾರಾಟ
ಮಾರಾ-ಮಾರಿ-ಯಾದ್ದನ್ನು
ಮಾರಿ
ಮಾರಿ-ಕೊಂಡಿದ್ದಾರೆ
ಮಾರಿ-ಕೊಂಡೆನು
ಮಾರಿ-ಕೊಳ್ಳುವಾಗ
ಮಾರಿಯ
ಮಾರುತ
ಮಾರುತ್ತಾ
ಮಾರ್ಗ
ಮಾರ್ಗಂ
ಮಾರ್ಗಕ್ಕೆ
ಮಾರ್ಗ-ಗಳ
ಮಾರ್ಗ-ಗಳನ್ನು
ಮಾರ್ಗ-ಗಳನ್ನೇಕೆ
ಮಾರ್ಗ-ಗಳಿವೆ
ಮಾರ್ಗ-ಗಳು
ಮಾರ್ಗದ
ಮಾರ್ಗ-ದರ್ಶಕ
ಮಾರ್ಗ-ದರ್ಶಕ-ಗಳಾಗಿಲ್ಲವೆ
ಮಾರ್ಗ-ದರ್ಶಕ-ರಾದ
ಮಾರ್ಗ-ದರ್ಶಕ-ರೆಲ್ಲಾ
ಮಾರ್ಗ-ದರ್ಶಿ-ಗಳಾಗಿವೆ
ಮಾರ್ಗ-ದರ್ಶಿ-ಯಾದ
ಮಾರ್ಗ-ದಲ್ಲಿ
ಮಾರ್ಗ-ದಲ್ಲಿದೆ
ಮಾರ್ಗ-ದಲ್ಲಿಯೂ
ಮಾರ್ಗ-ದಲ್ಲಿಯೇ
ಮಾರ್ಗ-ದಲ್ಲಿ-ರುವ
ಮಾರ್ಗ-ವನ್ನನು-ಸರಿಸ-ಬೇಕು
ಮಾರ್ಗ-ವನ್ನನು-ಸ-ರಿಸಿ
ಮಾರ್ಗ-ವನ್ನನು-ಸರಿಸಿ-ದರೂ
ಮಾರ್ಗ-ವನ್ನನು-ಸರಿಸಿ-ದರೆ
ಮಾರ್ಗ-ವನ್ನು
ಮಾರ್ಗ-ವನ್ನೇ
ಮಾರ್ಗ-ವಲ್ಲ
ಮಾರ್ಗ-ವಲ್ಲವೆ
ಮಾರ್ಗ-ವಾ-ವುದು
ಮಾರ್ಗ-ವಿರು-ವುದು
ಮಾರ್ಗ-ವಿಲ್ಲ
ಮಾರ್ಗವೂ
ಮಾರ್ಗವೆ
ಮಾರ್ಗ-ವೆಂದು
ಮಾರ್ಗವೇ
ಮಾರ್ಗ-ವೇಕೆ
ಮಾರ್ಗ-ಶೀರ್ಷಮಾಸ
ಮಾರ್ಗೊ-ಪಾಯ-ವನ್ನು
ಮಾರ್ಚಿ
ಮಾರ್ಚ್
ಮಾರ್ಜಿತ
ಮಾರ್ದನಿ-ಗುಡು-ತಲಿದೆ
ಮಾರ್ದನಿ-ಸು-ವುದು
ಮಾರ್ನುಡಿ-ದನು
ಮಾರ್ಪಟ್ಟಿದೆ
ಮಾರ್ಪಟ್ಟಿ-ರುತ್ತವೆ
ಮಾರ್ಪಟ್ಟಿ-ರು-ವ-ನೆಂದು
ಮಾರ್ಪಟ್ಟಿ-ರು-ವುದು
ಮಾರ್ಪಟ್ಟಿವೆ
ಮಾರ್ಪಟ್ಟು
ಮಾರ್ಪಡ-ಬೇಕು
ಮಾರ್ಪಡಿಸಿ
ಮಾರ್ಪಡಿ-ಸಿ-ಕೊಳ್ಳು-ವು-ದಿಲ್ಲ-ವೇಕೆ
ಮಾರ್ಪಡಿ-ಸಿ-ದರು
ಮಾರ್ಪಡಿ-ಸಿದ್ದಾರೆ
ಮಾರ್ಪಡಿ-ಸು-ವುದು
ಮಾರ್ಪಡು-ವುದು
ಮಾರ್ವಾ-ಡಿ-ಗಳೂ
ಮಾಲ
ಮಾಲೆ
ಮಾಲೆ-ಗಳು
ಮಾಲೆ-ಯನು
ಮಾವ
ಮಾವಂದಿರ
ಮಾವನ
ಮಾವಿನ
ಮಾವಿನ-ಹಣ್ಣು
ಮಾಸ-ಪತ್ರಿಕೆ-ಯಲ್ಲಿ
ಮಾಸಾಶ-ನ-ಗಳನ್ನು
ಮಾಸ್ಟರ್
ಮಾಹಾತ್ಮ್ಯೆ-ಯನ್ನ-ರಿತು
ಮಾಹಾ-ಪುರುಷರು
ಮಾಹಿತಿ-ಯೇನೂ
ಮಿಂಚದು
ಮಿಂಚಿ
ಮಿಂಚಿನ
ಮಿಂಚಿ-ನಂತೆ
ಮಿಂಚು
ಮಿಂಚು-ಗಳ
ಮಿಂಚು-ಗಳಂತೆ
ಮಿಂಚು-ಗಳಿ-ಗದು
ಮಿಂಚು-ಗುಡುಗು-ಗಳು
ಮಿಂಚು-ತಿ-ರಲು
ಮಿಂಚುವ
ಮಿಂಚೆ
ಮಿಂಚೊ
ಮಿಂದೆ
ಮಿಕ್ಕ
ಮಿಕ್ಕದ್ದನ್ನು
ಮಿಕ್ಕದ್ದೆಲ್ಲವೂ
ಮಿಕ್ಕದ್ದೆಲ್ಲಾ
ಮಿಕ್ಕರೆ
ಮಿಕ್ಕ-ವ-ರನ್ನೂ
ಮಿಕ್ಕ-ವರು
ಮಿಕ್ಕ-ವರೂ
ಮಿಕ್ಕ-ವ-ರೆಲ್ಲರೂ
ಮಿಕ್ಕ-ವು-ಗಳು
ಮಿಕ್ಕಿತ್ತು
ಮಿಕ್ಕಿ-ರುವುದು
ಮಿಕ್ಕೆಲ್ಲ
ಮಿಕ್ಕೆಲ್ಲ-ರಿಗೂ
ಮಿಕ್ಕೆಲ್ಲವೂ
ಮಿಗಿಲಾಗ-ಬಲ್ಲದು
ಮಿಗಿ-ಲಾಗಿ
ಮಿಗಿಲಾದ
ಮಿಗಿಲು
ಮಿಟುಕಿಸಿ
ಮಿಠಾಯಿ
ಮಿಠಾಯಿ-ಯನ್ನು
ಮಿಠಾಯಿ-ವನ್ನು
ಮಿಡಿತ
ಮಿಡಿ-ದಿ-ರಲು
ಮಿಡಿ-ಯುತಿರೆ
ಮಿಡಿ-ಯುತ್ತಿದೆ
ಮಿಡಿಯು-ವಂತಹ
ಮಿತಾಕ್ಷರನ
ಮಿತಾ-ಹಾರಿ-ಯಾಗಿ
ಮಿತಿ-ಮೀರಿದ
ಮಿತಿಯಲಿ
ಮಿತ್ರನು
ಮಿತ್ರನೇ
ಮಿತ್ರರಂತಾಗು-ವಿರಿ
ಮಿತ್ರ-ರಲ್ಲಿ
ಮಿತ್ರ-ರಲ್ಲಿಯೂ
ಮಿತ್ರರು
ಮಿತ್ರರೂ
ಮಿತ್ರರೇ
ಮಿತ್ರೇ
ಮಿಥ್ಯ
ಮಿಥ್ಯಾ
ಮಿಥ್ಯಾ-ನಾಮ-ರೂಪ-ಗಳಿಂದ
ಮಿಥ್ಯಾ-ನಾಮ-ರೂಪ-ಗಳು
ಮಿಥ್ಯೆ
ಮಿಥ್ಯೆ-ಗಳ
ಮಿಥ್ಯೆ-ಯಾದರೆ
ಮಿಥ್ಯೆ-ಯಾ-ಯಿತು
ಮಿಥ್ಯೆ-ಯೆಂದು
ಮಿಥ್ಯೆ-ಯೆಂಬು-ದಕ್ಕೆ
ಮಿಥ್ಯೆಯೊ
ಮಿಥ್ಯೆ-ಯೊಂದಿಗೆ
ಮಿದು-ಳನ್ನು
ಮಿದು-ಳಿನ
ಮಿದುಳಿ-ನಲ್ಲಿ
ಮಿದು-ಳಿನಲ್ಲಿ-ರುವ
ಮಿದುಳಿ-ನಿಂದ
ಮಿದುಳು
ಮಿಮಾಂಸೆಯ
ಮಿರರ್
ಮಿರು-ಗಿದೆ
ಮಿಲನ-ವನ್ನು
ಮಿಲನ-ವಿದೆ
ಮಿಲಮಿಲನೆ
ಮಿಲರ್
ಮಿಲಾಯಿಲ
ಮಿಲಾಯ್
ಮಿಲೆ
ಮಿಲ್
ಮಿಲ್ಟನ್ನಿ-ನಂತೆ
ಮಿಳಿತ-ವಾಗಿ-ರುತ್ತಾರೆ
ಮಿಳಿತ-ವಾಗಿ-ರುವ
ಮಿಳಿತ-ವಾಗು-ವು-ದನ್ನು
ಮಿಳಿತ-ವಾ-ದೊಂದು
ಮಿಶ್ರಣ
ಮಿಶ್ರಣ-ದಿಂದ
ಮಿಶ್ರಣ-ವುಂಟಾ-ಯಿತು
ಮಿಶ್ರಣವೂ
ಮಿಷನ್
ಮಿಷನ್ನ
ಮಿಷನ್ನಿನ
ಮಿಷನ್ನ
ಮಿಸ್
ಮೀಟುತ್ತ
ಮೀನನ್ನು
ಮೀನು
ಮೀರಾ
ಮೀರಿ
ಮೀರಿದ
ಮೀರಿ-ದ-ವರು
ಮೀರಿ-ದುದು
ಮೀರಿದ್ದಾನೆ
ಮೀರಿದ್ದು
ಮೀರಿದ್ದೋ
ಮೀರಿ-ರುವ
ಮೀರಿ-ರುವನು
ಮೀರಿ-ರು-ವು-ದನ್ನು
ಮೀರಿ-ಸಿ-ದ-ವರು
ಮೀರಿ-ಸಿ-ರುವರು
ಮೀರಿ-ಸಿ-ರು-ವು-ದಿಲ್ಲ
ಮೀರಿ-ಸು-ವಂತೆ
ಮೀರಿ-ಹನಾವ-ನಿ-ರುವನು
ಮೀರಿ-ಹೋಗಲಾಗುತ್ತಿಲ್ಲ
ಮೀರಿ-ಹೋಗಲು
ಮೀರಿ-ಹೋಗಿದೆ
ಮೀರಿ-ಹೋಗಿದ್ದೇವೋ
ಮೀರಿ-ಹೋಗುತ್ತಾರೆ
ಮೀರಿ-ಹೋಗು-ವಂತೆ
ಮೀರಿ-ಹೋಗುವೆ
ಮೀರಿ-ಹೋಗು-ವೆವೋ
ಮೀರು-ವಷ್ಟು
ಮೀರ್ದ
ಮೀರ್ವ-ರಾರು
ಮೀಸ-ಲಾಗಿಟ್ಟಿ-ರುವೆನು
ಮೀಸಲಾ-ಗಿಡಿ
ಮೀಸ-ಲಾಗಿತ್ತು
ಮೀಸ-ಲಾಗಿ-ಹೋಗಿದೆ
ಮೀಸಲಿ-ಹುದು
ಮೀಸಲೆ
ಮುಂಗಾಲು-ಗಳಿಂದ
ಮುಂಗುಸಿ-ಯನ್ನು
ಮುಂಚಂತು
ಮುಂಚಿನ
ಮುಂಚೆ
ಮುಂಚೆ-ಯಿದ್ದ-ವ-ನೆಂದು
ಮುಂಚೆಯೂ
ಮುಂಚೆಯೇ
ಮುಂಜಾನೆ
ಮುಂಡಕ
ಮುಂಡನ
ಮುಂಡನ-ವಾದ
ಮುಂಡ-ಮಾಲ
ಮುಂಡ-ಮಾಲೆ-ಯನು
ಮುಂಡಿತ
ಮುಂತಾಗಿ
ಮುಂತಾದ
ಮುಂತಾದ-ವಕ್ಕೆ
ಮುಂತಾದ-ವರ
ಮುಂತಾದ-ವ-ರಿಗೆ
ಮುಂತಾದ-ವರು
ಮುಂತಾದ-ವರೂ
ಮುಂತಾ-ದವು
ಮುಂತಾದ-ವು-ಗಳನ್ನು
ಮುಂತಾದ-ವು-ಗಳಿಂದ
ಮುಂತಾದುದನ್ನೆಲ್ಲಾ
ಮುಂತಾದು-ವನ್ನು
ಮುಂತಾದು-ವನ್ನೆಲ್ಲ
ಮುಂತಾದು-ವನ್ನೆಲ್ಲಾ
ಮುಂತಾ-ದುವು
ಮುಂತಾ-ದುವು-ಗಳ
ಮುಂತಾ-ದುವು-ಗಳನ್ನು
ಮುಂತಾ-ದುವು-ಗಳನ್ನೆಲ್ಲಾ
ಮುಂತಾ-ದುವು-ಗಳಲ್ಲಿ-ರುವ
ಮುಂತಾ-ದುವು-ಗಳಿಂದ
ಮುಂತಾ-ದುವು-ಗಳು
ಮುಂತಾ-ದುವು-ಗಳೆಲ್ಲಾ
ಮುಂತಾ-ದುವು-ಗಳೊಂದಿಗೆ
ಮುಂತಾದುವೂ
ಮುಂತಾದು-ವೆಲ್ಲ
ಮುಂತಾದುವೊಂದಕ್ಕೂ
ಮುಂತಾದ್ದನ್ನು
ಮುಂತಾದ್ದನ್ನೆಲ್ಲಾ
ಮುಂತಾದ್ದು
ಮುಂತಾದ್ದೆಲ್ಲ-ವನ್ನೂ
ಮುಂದಕೆ
ಮುಂದಕ್ಕೂ
ಮುಂದಕ್ಕೆ
ಮುಂದನು
ಮುಂದಾಗಲು
ಮುಂದಾಗಿ
ಮುಂದಾಗಿ-ದೆಯೊ
ಮುಂದಾಗಿರಿ
ಮುಂದಾಗು
ಮುಂದಾಲೋಚನೆ
ಮುಂದಾಳು-ಗಳಾಗಿ
ಮುಂದಾಳು-ಗಳಾಗಿದ್ದರು
ಮುಂದಾಳು-ತನ
ಮುಂದಿಟ್ಟಿದೆ-ಯೆಂದೇ
ಮುಂದಿಟ್ಟಿದ್ದೀರಿ
ಮುಂದಿಟ್ಟಿಲ್ಲ
ಮುಂದಿಟ್ಟು-ಕೊಂಡು
ಮುಂದಿತ್ತು
ಮುಂದಿದ್ದ
ಮುಂದಿನ
ಮುಂದಿನ-ವರು
ಮುಂದಿರ-ಬೇಕು
ಮುಂದಿ-ರಿಸಿ-ದೆವು
ಮುಂದಿರು-ತಿ-ರುವೆ
ಮುಂದಿರುವ
ಮುಂದು
ಮುಂದು-ಗಡೆ-ಯಲ್ಲಿ-ರುವ
ಮುಂದು-ಗಳಾ-ರ-ಲಿಲ್ಲವೊ
ಮುಂದು-ಮುಂದಕೆ
ಮುಂದು-ವರಿ-ದಂತೆ
ಮುಂದು-ವರಿ-ದರು
ಮುಂದು-ವರಿ-ದರೆ
ಮುಂದು-ವರಿ-ದಿದೆ
ಮುಂದು-ವರಿ-ದಿ-ರುವರು
ಮುಂದು-ವರಿದಿ-ರುವರೋ
ಮುಂದು-ವರಿ-ದಿ-ರು-ವುದಕ್ಕೆ
ಮುಂದು-ವರಿ-ದಿವೆ
ಮುಂದು-ವರಿದು
ಮುಂದು-ವರಿಯ-ಬೇ-ಕಾದರೆ
ಮುಂದು-ವರಿ-ಯ-ಬೇಕೆಂಬ
ಮುಂದು-ವರಿ-ಯ-ಲಾರ
ಮುಂದು-ವರಿ-ಯ-ಲಾರದು
ಮುಂದು-ವರಿ-ಯ-ಲಾರೆ
ಮುಂದು-ವರಿ-ಯಲಿ
ಮುಂದು-ವರಿ-ಯಲು
ಮುಂದು-ವರಿ-ಯಿತು
ಮುಂದು-ವರಿ-ಯಿರಿ
ಮುಂದು-ವ-ರಿಯುತ್ತಾ-ನೆಯೋ
ಮುಂದು-ವ-ರಿಯುತ್ತಾರೆ
ಮುಂದು-ವ-ರಿಯುತ್ತಿರು-ವರು
ಮುಂದು-ವರಿ-ಯುತ್ತಿವೆ
ಮುಂದು-ವ-ರಿಯುವ
ಮುಂದು-ವ-ರಿಯು-ವರು
ಮುಂದು-ವ-ರಿಯು-ವಿರಿ
ಮುಂದು-ವರಿ-ಯು-ವುದಕ್ಕೆ
ಮುಂದು-ವ-ರಿಯು-ವುದು
ಮುಂದು-ವರಿ-ಸ-ಬೇಕೆಂಬ
ಮುಂದು-ವರಿ-ಸ-ಲಿಲ್ಲ
ಮುಂದು-ವ-ರಿ-ಸಲು
ಮುಂದು-ವ-ರಿಸಿ
ಮುಂದು-ವ-ರಿಸಿ-ದರು
ಮುಂದು-ವರಿಸು
ಮುಂದು-ವರಿಸುತ್ತಿದ್ದರೆ
ಮುಂದೆ
ಮುಂದೆಯೂ
ಮುಂದೆಯೇ
ಮುಂದೇ-ನಾಗು-ವು-ದೆಂದು
ಮುಂದೇ-ನಾ-ಯಿತು
ಮುಂಬರಿ-ದಿ-ಹುದು
ಮುಂಬ-ರುವ
ಮುಂಬಾಗಿ-ಲಿ-ನಲ್ಲಿ
ಮುಕುಟ
ಮುಕ್ಕಾಲು
ಮುಕ್ಕಾಲು-ಪಾಲು
ಮುಕ್ಕಾಲು-ಪಾ-ಲೆಲ್ಲಾ
ಮುಕ್ತ
ಮುಕ್ತ-ಚೇತ-ವದು
ಮುಕ್ತದ್ವಾರ-ದಿಂದ
ಮುಕ್ತನ
ಮುಕ್ತ-ನನ್ನಾಗಿ
ಮುಕ್ತ-ನಾಗ-ಬೇಕೆಂದಿರು-ವ-ವನು
ಮುಕ್ತ-ನಾಗಿ
ಮುಕ್ತ-ನಾಗಿಯೇ
ಮುಕ್ತ-ನಾಗುತ್ತಿದ್ದನು
ಮುಕ್ತ-ನಾಗು-ವೆನೆ
ಮುಕ್ತ-ನಾದ
ಮುಕ್ತನು
ಮುಕ್ತನೆ
ಮುಕ್ತನೇ
ಮುಕ್ತ-ರನ್ನಾಗಿ
ಮುಕ್ತ-ರಾಗಿ
ಮುಕ್ತ-ರಾಗುತ್ತೀರಿ
ಮುಕ್ತ-ರಿಗೆ
ಮುಕ್ತರು
ಮುಕ್ತ-ವಾಗಿ-ರು-ವುದು
ಮುಕ್ತ-ವಾದ
ಮುಕ್ತಸ್ವ-ರೂಪ-ನಾದ
ಮುಕ್ತಾತ್ಮ
ಮುಕ್ತಾತ್ಮ-ರಿಗೆ
ಮುಕ್ತಾತ್ಮರೂ
ಮುಕ್ತಾಭಿ-ಮಾನೀ
ಮುಕ್ತಾಯ-ಗೊಂಡವು
ಮುಕ್ತಾಯ-ಗೊಂಡಿತು
ಮುಕ್ತಾಯ-ವಾಗಿ
ಮುಕ್ತಾಯ-ವಾದ
ಮುಕ್ತಾಯ-ವಾದೊಡ-ನೆಯೇ
ಮುಕ್ತಾಯ-ವಾ-ಯಿತು
ಮುಕ್ತಿ
ಮುಕ್ತಿಃ
ಮುಕ್ತಿ-ಗಳಿ-ಸು-ವುದು
ಮುಕ್ತಿ-ಗಾಗಿ
ಮುಕ್ತಿ-ಗೀತೆ
ಮುಕ್ತಿಗೆ
ಮುಕ್ತಿಪ್ರಭೆ-ಯನು
ಮುಕ್ತಿ-ಬಂಧ-ಗಳಿಲ್ಲ-ದಾತ್ಮವು
ಮುಕ್ತಿ-ಮಾರ್ಗ-ದಲ್ಲಿ
ಮುಕ್ತಿಯ
ಮುಕ್ತಿ-ಯನ್ನು
ಮುಕ್ತಿ-ಯ-ಹುದು
ಮುಕ್ತಿ-ಯಾಗುತ್ತಿತ್ತು
ಮುಕ್ತಿ-ಯಾಗುವ
ಮುಕ್ತಿ-ಯಾಗುವ-ವ-ರೆಗೂ
ಮುಕ್ತಿ-ಯಾದಲ್ಲದೆ
ಮುಕ್ತಿ-ಯಿಲ್ಲ
ಮುಕ್ತಿ-ಯಿ-ಹುದು
ಮುಕ್ತಿಯು
ಮುಕ್ತಿ-ಯೆ-ಡೆಗೆ
ಮುಕ್ತಿಯೇ
ಮುಕ್ತಿ-ಹೊಂದುವ
ಮುಕ್ತೋ
ಮುಖ
ಮುಖಂಡರು
ಮುಖ-ಗಳಂತೆ
ಮುಖ-ಗಳನ್ನು
ಮುಖ-ಗಳಲ್ಲಿ
ಮುಖದ
ಮುಖ-ದಲ್ಲಿ
ಮುಖ-ದಲ್ಲಿಯೆ
ಮುಖ-ದಲ್ಲೊಂದು
ಮುಖ-ದಾ-ವರೆ
ಮುಖದಿ
ಮುಖ-ಭಾವ
ಮುಖ-ಮಂಡಲವು
ಮುಖ-ಮುದ್ರೆ
ಮುಖ-ಮುದ್ರೆ-ಯನ್ನು
ಮುಖ-ಮುದ್ರೆ-ಯಿಂದ
ಮುಖ-ಮುದ್ರೆ-ಯಿತ್ತು
ಮುಖ-ರಿ-ತ-ವಾಗುತ್ತಿತ್ತು
ಮುಖರ್ಜಿ-ಗಳ
ಮುಖ-ವನ್ನು
ಮುಖ-ವನ್ನೇಕೆ
ಮುಖ-ವಾಗಿ
ಮುಖ-ವಾಗು
ಮುಖ-ವಾಡ-ವನು
ಮುಖ-ವಾಣಿ-ಯಾಗ-ಬೇಕೆಂದು
ಮುಖವು
ಮುಖಾ-ಮುಖಿ-ಯಾಗಿ
ಮುಖಾ-ರವಿಂದವು
ಮುಖೆ
ಮುಖೆ-ಬೋಲೆ
ಮುಖ್ಯ
ಮುಖ್ಯನ
ಮುಖ್ಯ-ವಲ್ಲ
ಮುಖ್ಯ-ವಾಗಿ
ಮುಖ್ಯ-ವಾದ
ಮುಖ್ಯ-ವಾ-ದದ್ದು
ಮುಖ್ಯ-ವಾ-ದುದು
ಮುಖ್ಯ-ವಾದು-ದೇ-ನೆಂದರೆ
ಮುಖ್ಯೋಪಾಧ್ಯಾಯರ
ಮುಗಿದ
ಮುಗಿದಂತಾ-ಗುತ್ತದೆಯೇ
ಮುಗಿದಂತಾಯಿತು
ಮುಗಿ-ದಂತೆಯೇ
ಮುಗಿದ-ನಂತರ
ಮುಗಿದ-ಮೇಲೆ
ಮುಗಿ-ದರೆ
ಮುಗಿದು-ವಂದೇ
ಮುಗಿ-ದುವು
ಮುಗಿದು-ಹೋಗಿ
ಮುಗಿದು-ಹೋಗಿ-ರ-ಬೇಕು
ಮುಗಿದು-ಹೋ-ಯಿತು
ಮುಗಿ-ಯದಾ-ಟವ
ಮುಗಿ-ಯ-ಲಿಲ್ಲವೊ
ಮುಗಿ-ಯಲು
ಮುಗಿಯಿ-ತಿಂದಿಗೆ
ಮುಗಿ-ಯಿತು
ಮುಗಿ-ಯಿತೊ
ಮುಗಿ-ಯುವ
ಮುಗಿ-ಯು-ವುದು
ಮುಗಿ-ಯುವೆ
ಮುಗಿಲ
ಮುಗಿಲಲಿ
ಮುಗಿಲು
ಮುಗಿ-ವ-ವ-ರೆಗೂ
ಮುಗಿಸಿ
ಮುಗಿಸಿ-ಕೊಂಡು
ಮುಗಿ-ಸಿದ
ಮುಗಿಸಿ-ದರು
ಮುಗಿ-ಸಿದೆ
ಮುಗಿಸು
ಮುಗುಳು-ನಗೆ
ಮುಗ್ಗರಿ-ಸುತ
ಮುಗ್ಧ
ಮುಗ್ಧ-ನಾಗಿ
ಮುಗ್ಧ-ನಾಗಿದ್ದಾನೆ
ಮುಚ್ಚ-ದಂತೆ
ಮುಚ್ಚಿ
ಮುಚ್ಚಿ-ಕೊಂಡಿದೆ
ಮುಚ್ಚಿ-ಕೊಂಡಿರುತ್ತವೆಂಬುದೂ
ಮುಚ್ಚಿ-ಕೊಂಡು
ಮುಚ್ಚಿ-ಕೊಳ್ಳುತ್ತಾರೆ
ಮುಚ್ಚಿಟ್ಟು
ಮುಚ್ಚಿದೆ
ಮುಚ್ಚಿ-ಬಿಡ-ಬೇಕೆಂದು
ಮುಚ್ಚಿವೆ
ಮುಚ್ಚಿ-ಹುದು
ಮುಚ್ಚುಮರೆ-ಯಲ್ಲಿ
ಮುಚ್ಚು-ವರು
ಮುಟ್ಟದೊ
ಮುಟ್ಟ-ಬಾ-ರದು
ಮುಟ್ಟ-ಬೇಡ
ಮುಟ್ಟ-ಬೇಡಿ
ಮುಟ್ಟ-ಲಾರದೆ
ಮುಟ್ಟಲು
ಮುಟ್ಟಲೂ
ಮುಟ್ಟಿ
ಮುಟ್ಟಿದ
ಮುಟ್ಟಿ-ದರು
ಮುಟ್ಟಿದೆ
ಮುಟ್ಟಿದ್ದಾರೆ
ಮುಟ್ಟುತ್ತಿರುವ
ಮುಟ್ಟು-ವ-ವ-ರೆಗೂ
ಮುಟ್ಟು-ವುದಕ್ಕೆ
ಮುಟ್ಟು-ವು-ದಲ್ಲದೆ
ಮುಟ್ಟು-ವು-ದಿಲ್ಲ
ಮುಟ್ಟು-ವು-ದಿಲ್ಲೆಂದು
ಮುಠ್ಠಾ-ಳನಲ್ಲದೆ
ಮುಠ್ಠಾ-ಳನೇ
ಮುತು-ವರ್ಜಿ-ಯಿಂದ
ಮುತ್ತ-ಬಾ-ರದೆ
ಮುತ್ತಿಗೆ
ಮುತ್ತಿಗೆ-ಹಾಕಿ
ಮುತ್ತಿ-ರುತ್ತಿದ್ದರು
ಮುತ್ತಿರುತ್ತಿದ್ದು-ದ-ರಿಂದ
ಮುತ್ತಿಹರು
ಮುತ್ತಿ-ಹವು
ಮುತ್ತಿ-ಹುದು
ಮುದಿ
ಮುದಿ-ತನ
ಮುದಿ-ಯಾದ
ಮುದುಕ
ಮುದುಕಿ-ಯ-ರನ್ನು
ಮುದುಡ-ದಿ-ರಲಿ
ಮುದುಡಿ
ಮುದುರಿ
ಮುದ್ದಿ-ಸಲಿ
ಮುದ್ದು
ಮುದ್ದೆ-ಗಳನ್ನು
ಮುದ್ದೆ-ಗಳು
ಮುದ್ರಣ
ಮುದ್ರಣಾಲ-ಯ-ವನ್ನು
ಮುದ್ರಾಲ-ಯ-ದಲ್ಲಿ
ಮುದ್ರಿತ
ಮುದ್ರಿತ-ವಾದ
ಮುದ್ರೆ
ಮುದ್ರೆ-ಯನ್ನು
ಮುದ್ರೆ-ಯನ್ನೊತ್ತಿ-ದರು
ಮುದ್ರೆ-ಯಿ-ರುವ
ಮುನಿ-ಗಳ
ಮುನಿಯ
ಮುನಿಸಿ-ಕೊಳುವೆ
ಮುನಿಸಿನ
ಮುನ್ನ
ಮುನ್ನಡೆ
ಮುನ್ನ-ಡೆದು
ಮುನ್ನವೆ
ಮುನ್ನವೇ
ಮುನ್ನುಗ್ಗ-ಬೇಡ
ಮುನ್ನುಗ್ಗಿ
ಮುನ್ನೂರು
ಮುಪ್ಪಿನ
ಮುಮೂರ್ಷು
ಮುಯ್ಯಿ
ಮುರಿ-ದಿದ್ದೇನೆ
ಮುರಿದು-ಬಿಟ್ಟನು
ಮುರಿದೆಸೆ
ಮುರಿಮು-ರಿಯುತ
ಮುರಿ-ಯು-ವಂತೆ
ಮುರಿಯು-ವವೆ
ಮುರುಕು
ಮುರ್ಷಿದಾಬಾದಿ-ನಿಂದ
ಮುಲ್ಲ-ರನ್ನು
ಮುಲ್ಲರ್
ಮುಳುಗಿ
ಮುಳುಗಿತು
ಮುಳುಗಿ-ದರು
ಮುಳುಗಿ-ದೆ-ನೆಂದರೆ
ಮುಳುಗಿದ್ದ
ಮುಳುಗಿದ್ದರೆ
ಮುಳುಗಿದ್ದೆಯೋ
ಮುಳುಗಿದ್ದೇ-ವೆಂದು
ಮುಳುಗಿ-ರ-ಲಾರ
ಮುಳುಗಿ-ರಲು
ಮುಳುಗಿ-ರುತ್ತದೆಯೋ
ಮುಳುಗಿ-ರುತ್ತಾರೆ
ಮುಳುಗಿ-ರುತ್ತಿದ್ದರು
ಮುಳುಗಿ-ರುವ
ಮುಳುಗಿ-ರುವರು
ಮುಳುಗಿ-ರುವ-ವ-ರನ್ನು
ಮುಳುಗಿ-ರು-ವು-ದನ್ನು
ಮುಳುಗಿ-ಸ-ಬೇಕು
ಮುಳುಗಿ-ಸಿದೆ
ಮುಳುಗಿಸು
ಮುಳುಗಿ-ಸು-ವರು
ಮುಳುಗಿ-ಹೋಗಿದೆ
ಮುಳುಗಿ-ಹೋಗಿ-ರು-ವು-ದ-ರಿಂದ
ಮುಳುಗಿ-ಹೋಗು-ವುದು
ಮುಳುಗು-ತಿದ್ದರು
ಮುಳು-ಗುತ್ತಾ
ಮುಳು-ಗುತ್ತಾನೆ
ಮುಳುಗುತ್ತಿಲ್ಲ
ಮುಳುಗು-ವಂತಹ
ಮುಳು-ಗು-ವಂತೆ
ಮುಳುಗು-ವುವು
ಮುಳ್ಳನ್ನು
ಮುಳ್ಳಿ-ನಿಂದ
ಮುಳ್ಳು
ಮುಳ್ಳು-ಕಂಟಿ-ಯೊಲು
ಮುಳ್ಳು-ಗಳ
ಮುಳ್ಳು-ಗಳನ್ನೂ
ಮುಷ್ಕ-ರ-ವಿದೆ
ಮುಷ್ಟಿ-ಯಲ್ಲಿಟ್ಟು-ಕೊಳ್ಳಲು
ಮುಸಲ್ಮಾನ
ಮುಸಲ್ಮಾನ-ರಿಬ್ಬ-ರನ್ನೂ
ಮುಸಲ್ಮಾನರು
ಮುಸಲ್ಮಾನ-ರೆಲ್ಲರೂ
ಮುಸುಕಲು
ಮುಸುಕಿತ್ತೆ-ನಗೆ
ಮುಸುಕಿದ
ಮುಸುಕಿ-ದಾಗ
ಮುಸುಕಿ-ರುವ
ಮುಸುಕಿ-ರುವಾಗ
ಮುಸುಕು-ಗಳೆಲ್ಲಾ
ಮುಸುಡಿ-ಯನ್ನು
ಮುಹೂರ್ತ-ದಲ್ಲಿ
ಮುಹೂರ್ತದಲ್ಲಿಯೆ
ಮುಹೂರ್ತವೂ
ಮೂಕನಂತಾದ
ಮೂಕ-ನಂತೆ
ಮೂಕ-ನನ್ನಾಗಿ
ಮೂಕನು
ಮೂಕ-ವಾಗಿ-ಹುದು
ಮೂಕ-ವಾ-ದುದು
ಮೂಕಾಸ್ವಾದ-ನ-ವತ್
ಮೂಗಲ್ಲ
ಮೂಗಿಗೆ
ಮೂಗಿ-ನಲ್ಲಿ
ಮೂಜೆ
ಮೂಟೆ
ಮೂಡಾರಾಧ-ನೆಯ
ಮೂಡಿ
ಮೂಡಿತು
ಮೂಡಿದ
ಮೂಡಿದೆ
ಮೂಡಿ-ಬಂದಿದೆ
ಮೂಡಿ-ರ-ಬೇಕು
ಮೂಡುತ್ತದೆ
ಮೂಡುತ್ತಿರುವ
ಮೂಡು-ವು-ದಕ್ಕೂ
ಮೂಡು-ವುದು
ಮೂಢತ-ನವೆಂದೂರೆ-ದರು
ಮೂಢ-ನಂಬಿಕೆ
ಮೂಢ-ನಂಬಿ-ಕೆ-ಗಳನ್ನು
ಮೂಢ-ನಂಬಿ-ಕೆ-ಗಳು
ಮೂಢ-ನಂಬಿ-ಕೆಯ
ಮೂಢ-ನಂಬಿ-ಕೆ-ಯಿಂದಾಗಲೀ
ಮೂಢನು
ಮೂಢ-ಭಕ್ತಿಯ
ಮೂಢ-ರಾಗಿ
ಮೂಢರು
ಮೂಢಾ
ಮೂಢಾ-ಚಾರದ
ಮೂದಲಿ-ಸಿ-ದರೂ
ಮೂರತಿ
ಮೂರ-ನೆಯ
ಮೂರ-ನೆಯ-ದಕ್ಕೇ
ಮೂರ-ನೆಯ-ದನ್ನು
ಮೂರ-ನೆ-ಯದೇ
ಮೂರ-ನೆಯ-ದೊಂದು
ಮೂರ-ನೆಯ-ವನು
ಮೂರನೇ
ಮೂರು
ಮೂರು-ಜನ
ಮೂರು-ನಾಲ್ಕು
ಮೂರೂ
ಮೂರ್ಖ-ತನ
ಮೂರ್ಖ-ತನ-ವಿನ್ನೇ-ತಕೆ
ಮೂರ್ಖತೆ-ಯನ್ನು
ಮೂರ್ಖನ
ಮೂರ್ಖನೇ
ಮೂರ್ಖರ
ಮೂರ್ಖರಂತಿದ್ದರೆ
ಮೂರ್ಖರು
ಮೂರ್ಚಿತ
ಮೂರ್ಛನಾ
ಮೂರ್ತ-ರೂಪ
ಮೂರ್ತ-ರೂಪವೇ
ಮೂರ್ತ-ವಾಗಿ
ಮೂರ್ತಿ
ಮೂರ್ತಿ-ಗಳ
ಮೂರ್ತಿ-ಗಳನ್ನು
ಮೂರ್ತಿ-ಗಳಿಗೆ
ಮೂರ್ತಿ-ಗಳೆಲ್ಲವ
ಮೂರ್ತಿ-ಗಳೋ
ಮೂರ್ತಿ-ಪೂಜೆ
ಮೂರ್ತಿ-ಮಂತ
ಮೂರ್ತಿ-ಮತ್ತಾಗಿ
ಮೂರ್ತಿ-ಮತ್ತಾಗಿದ್ದು-ವೆಂಬು-ದನ್ನು
ಮೂರ್ತಿ-ಮತ್ತಾಗಿ-ರ-ಬೇ-ಕೇನು
ಮೂರ್ತಿ-ಮತ್ತಾದ
ಮೂರ್ತಿಯ
ಮೂರ್ತಿ-ಯನ್ನು
ಮೂರ್ತಿ-ಯನ್ನೇ
ಮೂರ್ತಿ-ಯಲ್ಲಿ
ಮೂರ್ತಿ-ಯಾಗಿ
ಮೂರ್ತಿಯೇ
ಮೂರ್ತಿ-ಯೊಂದು
ಮೂರ್ತಿ-ವೆತ್ತ
ಮೂರ್ತಿ-ವೆತ್ತಂತಿ-ರುವರೊ
ಮೂರ್ತಿ-ವೆತ್ತಂತೆ
ಮೂರ್ತೀ-ಭವಿಸಿದ್ದಾನೆ
ಮೂಲ
ಮೂಲಕ
ಮೂಲ-ಕರ್ತರು
ಮೂಲ-ಕ-ವಲ್ಲದೆ
ಮೂಲ-ಕ-ವಾಗಿ
ಮೂಲ-ಕವೆ
ಮೂಲ-ಕವೇ
ಮೂಲಕ್ಕೆ
ಮೂಲತಃ
ಮೂಲ-ತತ್ತ್ವ-ವಾದ
ಮೂಲದ
ಮೂಲ-ದಲ್ಲಿ
ಮೂಲ-ದಲ್ಲಿ-ರು-ವುದು
ಮೂಲ-ದಿಂದ
ಮೂಲ-ಧನ
ಮೂಲ-ಧನ-ವೆಲ್ಲಿಂದ
ಮೂಲ-ನಿ-ವಾಸಿ-ಗಳನ್ನು
ಮೂಲ-ಪುರುಷರು
ಮೂಲ-ಬೀಜ-ವಾದ
ಮೂಲ-ಭೂತ-ಗಳೆಲ್ಲ
ಮೂಲ-ಮಂತ್ರದ
ಮೂಲ-ಮಂತ್ರ-ವಾ-ಗಿ-ರಲಿ
ಮೂಲ-ರಹಸ್ಯ
ಮೂಲ-ವನ್ನು
ಮೂಲ-ವಾಗಿ-ರುವ
ಮೂಲ-ವಾಗಿವೆ
ಮೂಲ-ವಾಗುತ್ತದೆ
ಮೂಲ-ವಾದ
ಮೂಲ-ವೆಂದು
ಮೂಲ-ವೆಂಬುದು
ಮೂಲ-ಸಿದ್ಧಾಂತ-ಗಳ
ಮೂಲಾ-ಧಾರ
ಮೂಲಾ-ಧಾರ-ದಲ್ಲಿ-ರುವ
ಮೂಲೆ
ಮೂಲೆಗೂ
ಮೂಲೆಗೆ
ಮೂಲೆ-ಮೂಲೆಗೂ
ಮೂಲೆ-ಯಲ್ಲಿ
ಮೂಲೋತ್ಪಾಟನೆ
ಮೂಳೆ
ಮೂಳೆ-ಮಾಂಸ-ಗಳ
ಮೂಳೆ-ಯಿಂದ
ಮೂವತ್ತು
ಮೂವತ್ತೆ-ರಡು
ಮೂವ-ರನ್ನು
ಮೂವರೂ
ಮೃಗ-ಗುಣ
ಮೃಗಜಲವು
ಮೃಗ-ಪಕ್ಷಿ-ಗಳೂ
ಮೃಗ-ವಾಗು-ವುದು
ಮೃಗ-ಶಾಲೆಗೆ
ಮೃಗ-ಶಾಲೆ-ಯನ್ನು
ಮೃಗ-ಶಾಲೆ-ಯಲ್ಲಿ
ಮೃಗ-ಶಾಲೆ-ಯೊ-ಳಕ್ಕೆ
ಮೃಗ-ಸದೃಶ-ನಾದ-ವ-ನನ್ನು
ಮೃಗೀಯ
ಮೃತ
ಮೃತ-ವಾದ
ಮೃತ-ವೀರ-ಕಾಯ
ಮೃತ್ತಿಕೆ-ಯನ್ನು
ಮೃತ್ತಿ-ಕೆಯನ್ನೆಲ್ಲ
ಮೃತ್ಯು
ಮೃತ್ಯು-ಕಾಲೋ
ಮೃತ್ಯು-ಗಳಿಲ್ಲ
ಮೃತ್ಯುಚ್ಛಾಯಾ
ಮೃತ್ಯುಚ್ಛಾಯೆ-ಯಲ್ಲಿ
ಮೃತ್ಯು-ಛಾಯೆಯ
ಮೃತ್ಯು-ತಮಿ
ಮೃತ್ಯು-ಭಯ-ವಿಲ್ಲ
ಮೃತ್ಯು-ಭ-ಯವು
ಮೃತ್ಯು-ಮಾಂಗೆ
ಮೃತ್ಯು-ರೂಪವ
ಮೃತ್ಯು-ರೂಪಾ
ಮೃತ್ಯು-ರೂಪಾ-ಕಾಳಿ
ಮೃತ್ಯು-ರೂಪಿ
ಮೃತ್ಯು-ರೂಪಿ-ಯನು
ಮೃತ್ಯು-ರೂಪಿ-ಯಹ
ಮೃತ್ಯುರ್ಧಾ-ವತಿ
ಮೃತ್ಯು-ವದು
ಮೃತ್ಯು-ವನಾಲಿಂಗಿಸಿ
ಮೃತ್ಯು-ವನ್ನು
ಮೃತ್ಯು-ವರ್ಣ
ಮೃತ್ಯು-ವಶ-ರಾಗುವರು
ಮೃತ್ಯು-ವಾಗಿದೆ
ಮೃತ್ಯು-ವಾಗಿಯೇ
ಮೃತ್ಯು-ವಿಗೆ
ಮೃತ್ಯು-ವಿನ
ಮೃತ್ಯು-ವಿ-ನಿಂದ
ಮೃತ್ಯು-ವಿಲ್ಲ
ಮೃತ್ಯುವೂ
ಮೃತ್ಯುವೆ
ಮೃತ್ಯು-ವೆಂಬುದು
ಮೃತ್ಯು-ಶಯ್ಯೆ-ಯಲ್ಲಿದ್ದಾಗ
ಮೃತ್ಯು-ಸಮ-ವಾದ
ಮೃದ
ಮೃದಂಗ
ಮೃದಂಗದ
ಮೃದಂಗ-ವನ್ನು
ಮೃದು
ಮೃದು-ಗೊಳಿಸು
ಮೃದು-ನುಡಿ-ಯೊಂದು
ಮೃದು-ಪ-ವನ
ಮೃದು-ಮಂದ
ಮೃದು-ಮಧು-ರ-ಗಂಭೀರ
ಮೃದು-ಮೃದು-ವಾಣಿ
ಮೃದು-ವಾಗಿ
ಮೃಷ್ಟಾನ್ನ
ಮೃಷ್ಟಾನ್ನ-ದಿಂದ
ಮೆಚ್ಚ-ಬಲ್ಲ
ಮೆಚ್ಚ-ಬಹು-ದಾದ
ಮೆಚ್ಚ-ಲಾ-ಗು-ವು-ದಿಲ್ಲ
ಮೆಚ್ಚಿ
ಮೆಚ್ಚಿಗೆ
ಮೆಚ್ಚಿ-ದ-ವರು
ಮೆಚ್ಚಿ-ದವ-ರೆಂದರೆ
ಮೆಚ್ಚುಗೆ-ಯಾಗಿದ್ದ
ಮೆಚ್ಚುತ್ತಾರೆಂಬ
ಮೆಚ್ಚುತ್ತೀರಿ
ಮೆಚ್ಚುತ್ತೇನೆ
ಮೆಚ್ಚುವ
ಮೆಚ್ಚು-ವಂತೆಯೇ
ಮೆಚ್ಚು-ವರು
ಮೆಚ್ಚು-ವು-ದಿಲ್ಲ
ಮೆಚ್ಚು-ವು-ದಿಲ್ಲ-ವೆಂದು
ಮೆಚ್ಚು-ವುದು
ಮೆಚ್ಚುವೆ-ಯೆಂಬ
ಮೆಟ್ಟಲಾದ
ಮೆಟ್ಟ-ಲಿನ
ಮೆಟ್ಟಲು-ಗಳಲ್ಲಿ
ಮೆಟ್ಟಲೇ
ಮೆಟ್ಟಿ
ಮೆಟ್ಟಿ-ಕೊಂಡಿ-ರು-ವು-ದನ್ನು
ಮೆಟ್ಟಿದ್ದರೂ
ಮೆಟ್ಟಿ-ನಿಲ್ಲು-ವುದು
ಮೆಟ್ಟಿ-ಲಾಗಿ
ಮೆಡಿಟರೇನಿಯನ್
ಮೆಡಿಟರೇನಿಯನ್
ಮೆಥಾಡಿಸ್ಟ-ರಾಗು-ವುದಕ್ಕೆ
ಮೆಥಾಡಿಸ್ಟರು
ಮೆದು-ಳಿನ
ಮೆದು-ಳಿನ-ವರು
ಮೆದುಳು
ಮೆರೆ-ವರು
ಮೆಲುಕು
ಮೆಲುಧ್ವನಿ-ಯಲ್ಲಿ
ಮೆಲು-ನುಡಿ-ಯೊಳೇ
ಮೆಲ್ಲ
ಮೆಲ್ಲಗೆ
ಮೆಲ್ಲ-ಡಿಯನಿಟ್ಟರೂ
ಮೆಲ್ಲನೆ
ಮೇ
ಮೇಕೆ
ಮೇಕೆ-ಗಳ
ಮೇಕೆ-ಯನ್ನೇ
ಮೇಖಲೆಗೆ
ಮೇಖಲೆ-ಯನ್ನು
ಮೇಗಡೆ
ಮೇಘ
ಮೇಘ-ಕುಲ
ಮೇಘ-ಗಳಂತೆ
ಮೇಘ-ಗಳು
ಮೇಘ-ದಿಂದಾವೃ-ತ-ವಾಗಿ
ಮೇಘ-ದೂ-ತದ
ಮೇಘ-ನಾದ-ವಧ
ಮೇಘ-ನಾದ-ವಧ-ದಂತಹ
ಮೇಘ-ಮಂದ್ರ
ಮೇಘ-ಮಾಲೆ
ಮೇಘ-ಮಾಲೆಯ
ಮೇಘ-ಮಾಲೆ-ಯ-ನಿರಿದು
ಮೇಜಿನ
ಮೇಜು-ಗಳ
ಮೇಣ್
ಮೇಧಾವಿ-ಗಳೂ
ಮೇಧಾ-ಶಕ್ತಿ
ಮೇಧಾ-ಶಕ್ತಿಯ
ಮೇನರ್
ಮೇನರ್-ನಲ್ಲಿ
ಮೇರಾ
ಮೇರಿ
ಮೇರೀ
ಮೇರು
ಮೇರು-ದಂಡ
ಮೇರು-ವಿನ
ಮೇರುವ್ಯಕ್ತಿತ್ವ
ಮೇರೆ
ಮೇರೆ-ಗಳ
ಮೇರೆಗೆ
ಮೇರೆ-ಯಿಲ್ಲ
ಮೇರೆ-ಯಿಲ್ಲ-ದಂತಿದ್ದವು
ಮೇಲ-ಕೇಳುತ
ಮೇಲಕ್ಕೆ
ಮೇಲಕ್ಕೆತ್ತ-ಬೇಕು
ಮೇಲಕ್ಕೇ-ರು-ವುದೇನೋ
ಮೇಲಕ್ಕೇಳುತ್ತದೆ
ಮೇಲಕ್ಕೇಳು-ವುದು
ಮೇಲಕ್ಕೇಳು-ವುವು
ಮೇಲಣ
ಮೇಲ-ಧಿ-ಕಾರಿ
ಮೇಲಾಗ-ಬ-ಹುದು
ಮೇಲಾಗಿ-ರುವರೊ
ಮೇಲಾ-ಗು-ವು-ದಿಲ್ಲ
ಮೇಲಾಗು-ವೆವು
ಮೇಲಾ-ದರೂ
ಮೇಲಿಂದ
ಮೇಲಿದ್ದ
ಮೇಲಿನ
ಮೇಲಿ-ನಿಂದ
ಮೇಲಿ-ರಲಿ
ಮೇಲಿ-ರಿಸಿ
ಮೇಲಿ-ರುವ
ಮೇಲಿ-ರು-ವುದು
ಮೇಲಿಹ
ಮೇಲು
ಮೇಲು-ಜಾತಿ-ಯ-ವ-ರನ್ನು
ಮೇಲು-ಜಾತಿ-ಯ-ವ-ರಿಗೆ
ಮೇಲು-ಜಾತಿ-ಯ-ವರು
ಮೇಲು-ಬ-ರಲು
ಮೇಲು-ಮೇಲೆ
ಮೇಲೂ
ಮೇಲೆ
ಮೇಲೆಂದು
ಮೇಲೆತ್ತ
ಮೇಲೆತ್ತಿ
ಮೇಲೆತ್ತಿ-ದರೆ
ಮೇಲೆತ್ತುತ್ತಿದ್ದರು
ಮೇಲೆತ್ತು-ವುದೂ
ಮೇಲೆದ್ದರು
ಮೇಲೆದ್ದು
ಮೇಲೆಯೂ
ಮೇಲೆಯೆ
ಮೇಲೆಯೇ
ಮೇಲೆಲ್ಲಾ
ಮೇಲೇ-ರು-ತಿದೆ
ಮೇಲೇ-ರುವುದ-ರಲ್ಲಿ
ಮೇಲೇ-ಳು-ವುದು
ಮೇಲೇ-ಳೇಳು
ಮೇಲೊಂದು
ಮೇಲೋ
ಮೇಲ್ಪಂಕ್ತಿಯನ್ನನು-ಸ-ರಿಸಿ
ಮೇಲ್ಪಂಕ್ತಿಯಾಗಲೋಸುಗ
ಮೇಲ್ಪಂಕ್ತಿ-ಯಾಗಿ-ರು-ವಿರಿ
ಮೇಲ್ಮಟ್ಟಕ್ಕೆ
ಮೇಲ್ಮಟ್ಟದ
ಮೇಲ್ಮಟ್ಟದಲ್ಲಿ-ರುವೆ-ವೆಂದು
ಮೇಲ್ಮೆ-ಗಾಗಿ
ಮೇಲ್ಮೆ-ಯನ್ನು
ಮೇಲ್ವರ್ಗದ
ಮೇಲ್ವಿ-ಚಾರ-ಕ-ನಾ-ಗಲಿ
ಮೇಲ್ವಿ-ಚಾರ-ಕ-ರಾದ
ಮೇಷ್ಟರ
ಮೈ
ಮೈಕೆಲ್-ನಿಗೆ
ಮೈಕೇಲನ
ಮೈಕೇಲನು
ಮೈಕೇಲ್
ಮೈಗೆಲ್ಲಾ
ಮೈಚಾಚಿದ
ಮೈತ್ರ
ಮೈತ್ರಿ
ಮೈತ್ರೇಯೀ
ಮೈದಾನ
ಮೈದಾನ-ದಲ್ಲಿ
ಮೈದಾ-ಳಿದೆ
ಮೈದಾಳಿ-ರುವ
ಮೈದೋರದು
ಮೈನರ್
ಮೈಮರೆತ
ಮೈಮರೆತ-ವ-ರಿಗೆ
ಮೈಮರೆತು
ಮೈಮರೆ-ಯುವೆ
ಮೈಮೇಲೆ
ಮೈಯ
ಮೈಯ-ನೊಡ್ಡುತ
ಮೈಯನ್ನು
ಮೈಯಲ್ಲಿ
ಮೈಲಿ
ಮೈಲಿ-ಗಳನ್ನು
ಮೈಲು-ಗಳಷ್ಟು
ಮೈವಡೆ-ದಾತ್ಮ
ಮೈಸರಿ-ಯಿಲ್ಲ
ಮೈಸೂರು
ಮೊಗಲರ
ಮೊಗಲ್
ಮೊಗವ
ಮೊಗ-ವನು
ಮೊಗ್ಗು
ಮೊಟ್ಟ
ಮೊಟ್ಟ-ಮೊ-ದಲ
ಮೊಟ್ಟ-ಮೊ-ದಲು
ಮೊಟ್ಟೆಗೆ
ಮೊಟ್ಟೆ-ಯಿಂದ
ಮೊಟ್ಟೆಯೆ
ಮೊತ್ತ
ಮೊತ್ತವೇ
ಮೊದ-ಮೊ-ದಲು
ಮೊದಲ
ಮೊದಲನೆ
ಮೊದಲ-ನೆಯ
ಮೊದಲ-ನೆ-ಯ-ದಂತಿದೆ
ಮೊದಲ-ನೆ-ಯ-ದಕ್ಕಿಂತ
ಮೊದಲ-ನೆ-ಯ-ದ-ರಲ್ಲಿ
ಮೊದಲ-ನೆ-ಯ-ದಾಗಿ
ಮೊದಲ-ನೆ-ಯ-ದಾಗಿಯೇ
ಮೊದಲ-ನೆ-ಯದು
ಮೊದಲನೇ
ಮೊದಲ-ಬಾರಿ
ಮೊದಲ-ಬಾರಿಗೆ
ಮೊದಲ-ಸಾರಿ
ಮೊದಲಾದ
ಮೊದಲಾದ-ವು-ಗಳೂ
ಮೊದಲಾದು-ವನ್ನು
ಮೊದಲಾದು-ವು-ಗಳ
ಮೊದಲಾ-ದುವು-ಗಳನ್ನು
ಮೊದಲಾ-ದುವು-ಗಳಲ್ಲಿ
ಮೊದಲಾದ್ದನ್ನು
ಮೊದ-ಲಾ-ಯಿತು
ಮೊದಲಾ-ಯಿತೊ
ಮೊದಲಿ-ಗನು
ಮೊದಲಿಗ-ರ-ವರು
ಮೊದಲಿಗರು
ಮೊದಲಿ-ಗಿಂತ
ಮೊದ-ಲಿಗೆ
ಮೊದಲಿದ್ದೆನು
ಮೊದಲಿನ
ಮೊದಲಿ-ನಂತೆ
ಮೊದ-ಲಿ-ನಿಂದ
ಮೊದ-ಲಿ-ನಿಂದಲೂ
ಮೊದ-ಲಿಲ್ಲದ
ಮೊದಲು
ಮೊದಲು-ಮಾಡಿ
ಮೊದಲು-ಮಾಡಿತು
ಮೊದಲು-ಮಾಡಿ-ದನು
ಮೊದಲು-ಮಾಡಿ-ದರು
ಮೊದಲು-ಮಾಡಿ-ದಳು
ಮೊದಲು-ಮಾಡಿದೆ
ಮೊದಲು-ಮಾಡು-ವುವು
ಮೊದಲೆ
ಮೊದಲೇ
ಮೊನ್ನೆ
ಮೊನ್ನೆಯ
ಮೊಮ್ಮಕ್ಕಳು
ಮೊರ
ಮೊರಬ್ಬ
ಮೊರೆ-ತದ
ಮೊರೆ-ದಿವೆ
ಮೊರೆದು
ಮೊರೆ-ಯಿಡುತ್ತಾ
ಮೊರೆ-ಯುತ-ಲಿರೆ
ಮೊರೆಯು-ತಿ-ರಲು
ಮೊರೆ-ಯುತ್ತಿರ-ಬೇಕು
ಮೊರೆ-ಯುವ-ಲೆ-ಗಳ
ಮೊರೆಹೊಕ್ಕರು
ಮೊಲದನೆ-ಯ-ದಾಗಿ
ಮೊಳ-ಗಿದೆ
ಮೊಳಗಿರೆ
ಮೊಳಗು
ಮೊಳಗುತ
ಮೊಳಗುತ್ತಿದೆ
ಮೊಸರು
ಮೊಹಮ್ಮದ್ನ
ಮೊಹರಿ-ರುವ
ಮೊಹಲ್ಲ-ದಲ್ಲಿದ್ದ
ಮೊಹಲ್ಲೆ-ಯಲ್ಲಿದ್ದ
ಮೋಕ್ಷ
ಮೋಕ್ಷಕ್ಕೆ
ಮೋಕ್ಷಕ್ಕೋಸ್ಕರವೂ
ಮೋಕ್ಷದ
ಮೋಕ್ಷ-ದಿಂದೇ-ನಾಗು-ವುದು
ಮೋಕ್ಷ-ನಾ-ನಲ್ಲ
ಮೋಕ್ಷಪ್ರಾಪ್ತಿ-ಯ-ವ-ರೆಗೂ
ಮೋಕ್ಷ-ವನ್ನೂ
ಮೋಕ್ಷ-ವಲ್ಲ
ಮೋಕ್ಷವೇ
ಮೋಕ್ಷಾರ್ಥಂ
ಮೋಚನ
ಮೋಚಿ
ಮೋಡ
ಮೋಡ-ಗಳ
ಮೋಡ-ಗಳೊ-ಡಲನು
ಮೋಡ-ದೊಡ್ಯಾಣ-ವನು
ಮೋಡ-ಮೋಡ-ಗಳಂಚಲಿ
ಮೋಡ-ವದು
ಮೋಡ-ವನ್ನು
ಮೋಡ-ವೆಲ್ಲ-ವ-ನೀಗ
ಮೋದದಿ
ಮೋರ
ಮೋಸ
ಮೋಸ-ಗಾರ-ನಾದ
ಮೋಸ-ಮಾಡಿ
ಮೋಸವು
ಮೋಹ
ಮೋಹಂಕಷಂ
ಮೋಹಃ
ಮೋಹಕ
ಮೋಹಕ್ಕೆ
ಮೋಹ-ಗಳಂತು
ಮೋಹ-ಗೊಂಡಿ-ರುವ
ಮೋಹ-ಗೊಂಡು
ಮೋಹ-ಗೊಳಿ-ಪರು
ಮೋಹ-ಗೊಳಿಸಿ-ಬಿಟ್ಟ-ರೆಂದರೆ
ಮೋಹ-ಜಾಯ
ಮೋಹ-ತಿಮಿ-ರವ-ನಿಲ್ಲ
ಮೋಹ-ದಲ್ಲಿ
ಮೋಹ-ದಿಂದ
ಮೋಹ-ಮುಗ್ಧ-ಕರ
ಮೋಹ-ವನು
ಮೋಹ-ವನ್ನು
ಮೋಹ-ವಿಲ್ಲ
ಮೋಹವು
ಮೋಹವೂ
ಮೋಹವೆ
ಮೋಹ-ವೆಲ್ಲಾ
ಮೋಹಿತ-ರನ್ನಾಗಿ
ಮೋಹಿನಿ-ಬಾಬು-ಗಳ
ಮೌಂಜಿ
ಮೌಡ್ಯ
ಮೌಡ್ಯ-ವನ್ನೆಲ್ಲ
ಮೌಡ್ಯವೇ
ಮೌಢ್ಯ
ಮೌಢ್ಯಕ್ಕೆ
ಮೌಢ್ಯ-ತೆಗೆ
ಮೌಢ್ಯ-ವೆಂದರೂ
ಮೌಢ್ಯವೇ
ಮೌನ
ಮೌನ-ದಲ್ಲಿ
ಮೌನ-ಮುಗ್ಧ-ರಾಗಿ
ಮೌನ-ವನ್ನವ-ಲಂಬಿಸಿ-ರು-ವಿರಿ
ಮೌನ-ವಾಗಿ
ಮೌನ-ವಾಗಿದ್ದರು
ಮೌನ-ವಾಗಿದ್ದು
ಮೌನ-ವಾದರು
ಮೌನಾಶ್ರು-ಧಾರೆ-ಯಲಿ
ಮೌಲಿ
ಮ್ಯಾಕ್ಲಾಡಳಿಗೆ
ಮ್ಯಾಕ್ಸ್
ಮ್ಯಾಕ್ಸ್-ಮುಲ್ಲರ್
ಮ್ಯಾಕ್ಸ್-ಮುಲ್ಲರ್ನ
ಮ್ಯಾಕ್ಸ್-ಮುಲ್ಲರ್-ನಿಂದ
ಮ್ರಿಯಸ್ವ
ಮ್ಲೇಚ್ಛ
ಮ್ಲೇಚ್ಛರ
ಮ್ಲೇಚ್ಛ-ರಾಗಿ
ಮ್ಲೇಚ್ಛರು
ಮ್ಲೇಚ್ಛ-ರೆಂದು
ಯ
ಯಂ
ಯಂತ್ರ
ಯಂತ್ರ-ಗಳ
ಯಂತ್ರ-ಗಳನ್ನಾಗಿ
ಯಂತ್ರದ
ಯಂತ್ರ-ದಿಂದ
ಯಂತ್ರ-ವನ್ನಾಗಿ
ಯಂತ್ರ-ವನ್ನು
ಯಂತ್ರ-ವಲ್ಲದೆ
ಯಂತ್ರ-ವಾದ
ಯಃ
ಯಃಕಶ್ಚಿತ್
ಯಕ್ಷ
ಯಕ್ಷನು
ಯಕ್ಷಿಣಿ
ಯಜ-ಮಾನ
ಯಜ-ಮಾನ-ನನ್ನು
ಯಜ-ಮಾ-ನನು
ಯಜ-ಮಾನನೆ
ಯಜ-ಮಾನರ
ಯಜ-ಮಾನಿ
ಯಜ-ಮಾನಿಯು
ಯಜ್ಞಕುಂಡದಿ
ಯಜ್ಞ-ಗಳ
ಯಜ್ಞ-ಗಳಿಗೆ
ಯಜ್ಞ-ಜಪತಪ-ಗಳನು
ಯಜ್ಞಧ್ವಾನಧ್ವನಿ-ತ-ಗ-ಗನೈರ್ಬ್ರಾಹ್ಮಣೈರ್ಜ್ಞಾತ
ಯಜ್ಞಯಾಗಾದಿ
ಯಜ್ಞಯಾಗಾದಿ-ಗಳ
ಯಜ್ಞಯಾಗಾದಿ-ಗಳಿಂದಲ್ಲ
ಯಜ್ಞಯಾಗಾದಿ-ಗಳು
ಯಜ್ಞಯಾಗಾದಿ-ಗಳೆಲ್ಲ
ಯಜ್ಞ-ವನ್ನು
ಯಜ್ಞ-ಸೂತ್ರ
ಯಜ್ಞ-ಸೂತ್ರ-ವನ್ನು
ಯಜ್ಞಸ್ಥಳ-ದಲ್ಲಿ
ಯಜ್ಞಾಗ್ನಿಗೆ
ಯಜ್ಞೋಪವೀತ
ಯಜ್ಞೋಪವೀತ-ಗಳನ್ನು
ಯಜ್ಞೋಪವೀ-ತದ
ಯಜ್ಞೋಪವೀತ-ವನ್ನು
ಯಜ್ಞೋಪವೀತ-ಸೂತ್ರದ
ಯತ
ಯತಯೋಽಸ್ಯ
ಯತಿ-ಜೀವನ-ವನ್ನು
ಯತಿ-ರಾಜ-ನಿಗೆ
ಯತೋಽಹಂ
ಯತ್ತು
ಯತ್ನ
ಯತ್ನ-ದಲಿ
ಯತ್ನ-ಶೀಲ-ರಾಗಿ
ಯತ್ನಿಸಿ
ಯತ್ನಿಸಿದ
ಯತ್ನಿಸಿ-ದಂತಿತ್ತು
ಯತ್ನಿಸಿ-ದನು
ಯತ್ನಿಸಿ-ದರೆ
ಯತ್ನಿಸಿ-ದಾಗ
ಯತ್ನಿಸಿ-ದುದು
ಯತ್ನಿಸಿದ್ದೇನೆ
ಯತ್ನಿಸು
ಯತ್ನಿ-ಸುತ
ಯತ್ನಿಸುತ್ತಾರೆ
ಯತ್ನಿಸುತ್ತಿದ್ದರು
ಯತ್ನಿಸುತ್ತಿರು-ವೆವು
ಯತ್ನಿಸುತ್ತೇವೆ
ಯತ್ನಿ-ಸುವ
ಯತ್ನಿಸು-ವಂತೆ
ಯತ್ನಿಸು-ವರು
ಯತ್ನಿಸು-ವವರು
ಯತ್ನಿಸು-ವುದು
ಯತ್ರ
ಯಥಾ
ಯಥಾಂಧಾಃ
ಯಥಾಯ
ಯಥಾಯ್
ಯಥಾರ್ಥ
ಯಥಾರ್ಥಃ
ಯಥಾರ್ಥ-ವಾಗಿ
ಯಥಾರ್ಥ-ವಾದ
ಯಥಾರ್ಥ-ವೆನ್ನಿ-ಸು-ವುದು
ಯಥಾ-ವತ್ತಾಗಿ
ಯಥಾ-ವಿಧಿ-ಯಾಗಿ
ಯಥೇಚ್ಛ-ವಾಗಿ
ಯಥೇಷ್ಟಮ್
ಯಥೇಷ್ಟ-ವಾಗಿ
ಯದ-ಹರೇವ
ಯದಾ
ಯದಿ
ಯದ್ವೈ
ಯನಾಯ
ಯನ್ನು
ಯಬೆ
ಯಮತ್ವ
ಯಮನ
ಯಮ-ನಿಯಮ
ಯಮನು
ಯಮ-ಪಾಶವು
ಯಮ-ಯಾತ-ನೆಗೆ
ಯಮ-ಲೋ-ಕಕ್ಕೆ
ಯಮಿ-ಜನ
ಯಮು-ನೆಗೆ
ಯಮು-ನೆಯ
ಯವನರ
ಯವ-ನರು
ಯವೆ
ಯಶ
ಯಶಕೆ
ಯಶಸ್ವಿ-ಯಾಗಿಲ್ಲ
ಯಶಸ್ವಿ-ಯಾದ
ಯಶಸ್ವಿ-ಯಾದ-ರೆಂದು-ಕೊಳ್ಳೋಣ
ಯಶಸ್ಸಿನ
ಯಶಸ್ಸು
ಯಶಸ್ಸೇ
ಯಸೀನಿಯ-ವನು
ಯಸ್ಮಾದಹಂ
ಯಸ್ಯ
ಯಸ್ಯಾ
ಯಹೂದಿ
ಯಹೂದಿ-ಗಳ
ಯಹೂದಿ-ಗಳಲ್ಲಿ
ಯಹೂದಿ-ಗಳಿಂದ
ಯಹೂದಿ-ಗಳು
ಯಾ
ಯಾಂ
ಯಾಂತ್ರಿಕ-ವಾಗು-ವುದು
ಯಾಕೆ
ಯಾಕ್
ಯಾಖನ
ಯಾಗ-ದಲ್ಲಿ
ಯಾಗ-ಬೇಕು
ಯಾಗ-ವನ್ನು
ಯಾಚಿ-ಸಿ-ದರು
ಯಾಜ್ಞವಲ್ಕ್ಯ
ಯಾಜ್ಞವಲ್ಕ್ಯರ
ಯಾಜ್ಞವಲ್ಕ್ಯ-ರನ್ನು
ಯಾಜ್ಞವಲ್ಕ್ಯ-ರಾ-ದುದು
ಯಾಜ್ಞವಲ್ಕ್ಯ-ರಿ-ಗಿಂತ
ಯಾತಕ್ಕೆ
ಯಾತನ
ಯಾತ-ನಾಮ-ಯ-ವಾದ
ಯಾತನೆ
ಯಾತ-ರಿಂದ
ಯಾತ-ರಿಂದಲೂ
ಯಾತಾ
ಯಾತಿ
ಯಾತ್ರಾಸ್ಥಳ-ವಾಗಿ
ಯಾತ್ರಿ-ಕರು
ಯಾತ್ರಿ-ಕ-ರೆಲ್ಲಾ
ಯಾತ್ರೆಗೆ
ಯಾಥಾ
ಯಾಥಾರ್ಥ-ವಾದ
ಯಾದ-ವ-ಗಿರಿ
ಯಾಮಿ
ಯಾಯ
ಯಾರ
ಯಾರದು
ಯಾರನ್ನಾದರೂ
ಯಾರನ್ನು
ಯಾರನ್ನೂ
ಯಾರನ್ನೋ
ಯಾರ-ಮೇಲೂ
ಯಾರ-ಲೊಂದಾ-ಗಿ-ರುವುದೊ
ಯಾರಲ್ಲಿ
ಯಾರಲ್ಲಿ-ಯಾದರೂ
ಯಾರ-ವರು
ಯಾರಾ
ಯಾರಾ-ದ-ರಾಗಲಿ
ಯಾರಾ-ದರು
ಯಾರಾ-ದರೂ
ಯಾರಾರು
ಯಾರಿಂದ
ಯಾರಿಂದ-ಲಾ-ದರೂ
ಯಾರಿಂದಲೂ
ಯಾರಿ-ಗಾಗಿ
ಯಾರಿ-ಗಾ-ದರೂ
ಯಾರಿಗೂ
ಯಾರಿಗೆ
ಯಾರು
ಯಾರು-ಯಾರಿಗೆ
ಯಾರೂ
ಯಾರೂ-ಕೂಡಾ
ಯಾರೆಂದರೆ
ಯಾರೆಂಬು-ದನು
ಯಾರೆಂಬು-ದನ್ನು
ಯಾರೆಂಬುದು
ಯಾರೇ
ಯಾರೊ
ಯಾರೊ-ಡನೆ
ಯಾರೊ-ಡ-ನೆಯೂ
ಯಾರೊ-ಡ-ನೆಯೊ
ಯಾರೊಬ್ಬ
ಯಾರೊಬ್ಬ-ರಿಗೂ
ಯಾರೊಬ್ಬರೂ
ಯಾರೊ-ಳಗೆ
ಯಾರೋ
ಯಾರ್
ಯಾರ್ಯಾ-ರೊಂದಿಗೆ
ಯಾವ
ಯಾವಜ್ಜೀವವೂ
ಯಾವತ್ತೂ
ಯಾವ-ದೊಂದು
ಯಾವ-ನಪ್ರತಿ-ಮ-ಹಿಮನೊ
ಯಾವ-ನಾದರೂ
ಯಾವ-ನಿ-ಗಾ-ದರೂ
ಯಾವ-ನಿಗೆ
ಯಾವನು
ಯಾವನೋ
ಯಾವ-ರೀತಿ
ಯಾವಳು
ಯಾವಾಗ
ಯಾವಾಗ-ಲಾ-ದರು
ಯಾವಾಗ-ಲಾ-ದರೂ
ಯಾವಾಗ-ಲಾ-ದರೊಮ್ಮೆ
ಯಾವಾಗಲು
ಯಾವಾಗಲೂ
ಯಾವಾಗಲೋ
ಯಾವಾಗೆಂದರೆ
ಯಾವಾ-ಗೆಲ್ಲಾ
ಯಾವಾ-ತನ
ಯಾವುದಕ್ಕಾಗಿ
ಯಾವು-ದಕ್ಕೂ
ಯಾವುದಕ್ಕೆ
ಯಾವು-ದಕ್ಕೊ
ಯಾವು-ದದು
ಯಾವುದನ್ನಾದರೂ
ಯಾವು-ದನ್ನು
ಯಾವು-ದನ್ನೂ
ಯಾವುದನ್ನೇ
ಯಾವು-ದರ
ಯಾವು-ದ-ರಲ್ಲಿ
ಯಾವು-ದರಲ್ಲಿಯೂ
ಯಾವು-ದ-ರಲ್ಲೂ
ಯಾವು-ದ-ರಿಂದ
ಯಾವು-ದ-ರಿಂದಲೂ
ಯಾವು-ದಾದರೂ
ಯಾವು-ದಾದ-ರೊಂದನ್ನು
ಯಾವು-ದಾದ-ರೊಂದು
ಯಾವು-ದಿದೆ
ಯಾವು-ದಿಲ್ಲವೊ
ಯಾವುದು
ಯಾವುದೂ
ಯಾವು-ದೆಂದರೆ
ಯಾವುದೇ
ಯಾವುದೊ
ಯಾವುದೊಂದೂ
ಯಾವುದೋ
ಯಾವುವು
ಯಾವೊಂದು
ಯುಕ್ತ
ಯುಕ್ತಃ
ಯುಕ್ತ-ವಾ-ಗು-ವು-ದಿಲ್ಲ
ಯುಕ್ತ-ವೆಂಬುದು
ಯುಕ್ತಾ-ಯುಕ್ತ
ಯುಕ್ತಿ
ಯುಕ್ತಿ-ಗಳಿಂದ
ಯುಕ್ತಿ-ಗಳಿವೆ
ಯುಕ್ತಿಗೆ
ಯುಕ್ತಿ-ತರ್ಕ-ಗಳ
ಯುಕ್ತಿಯ
ಯುಕ್ತಿ-ಯನ್ನು
ಯುಕ್ತಿ-ಯುಕ್ತ-ವಲ್ಲ
ಯುಕ್ತಿಯೂ
ಯುಗ
ಯುಗ-ಈಶ್ವರ
ಯುಗ-ಗಳ
ಯುಗ-ಗ-ಳಲ್ಲೂ
ಯುಗ-ಗ-ಳಾದರೂ
ಯುಗದ
ಯುಗ-ದಲ್ಲಿ
ಯುಗ-ಯುಗ-ಗಳ
ಯುಗ-ಯುಗ-ಗಳಿಂದಲೂ
ಯುಗ-ಯುಗಾಂತರ-ಗಳಿಂದ
ಯುಗ-ಲ-ಚರಣ
ಯುಗಲ್
ಯುಗ-ವಾಗಿ
ಯುಗಾಂತರ-ಗಳ
ಯುಗಾಂತರೇ
ಯುಗಾವ-ತಾರ
ಯುದ್ಧ
ಯುದ್ಧಕ್ಕೆ
ಯುದ್ಧದ
ಯುದ್ಧ-ದಲ್ಲಿ
ಯುದ್ಧ-ರಂಗ-ದಲ್ಲಿಯೇ
ಯುದ್ಧ-ರಂಗದಿ
ಯುದ್ಧ-ವನ್ನು
ಯುದ್ಧೋ-ಚಿತ-ವಾದು-ದಾಗಿದೆ
ಯುನ್ಮಂ
ಯುಮು-ನಾಕಿ
ಯುರೋಪಿ-ನಲ್ಲಿ
ಯುವಕ
ಯುವ-ಕರ
ಯುವ-ಕ-ರನ್ನು
ಯುವ-ಕ-ರಿಗೆ
ಯುವ-ಕರು
ಯುವ-ಕರೂ
ಯುವತಿ
ಯುವ-ಶಿಷ್ಯನ
ಯುವೈವ
ಯುಷ್ಮದಸ್ಮತ್
ಯೂನಿಟೇರಿಯನ್ನ-ರಾಗು-ವುದಕ್ಕೆ
ಯೂನಿಟೇರಿಯನ್ನರು
ಯೂರೋಪಿ-ನಲ್ಲಿಯೂ
ಯೂರೋಪಿ-ನಿಂದ
ಯೂರೋಪಿಯನ್
ಯೂರೋಪಿಯನ್ನ-ರಿಗೆ
ಯೂರೋಪು
ಯೂರೋಪ್
ಯೆ
ಯೆನ
ಯೆಹೂದ್ಯರ
ಯೇ
ಯೇನೈವ
ಯೇಯಿ
ಯೇವಾ
ಯೇಸುವು
ಯೊಚ-ನೆ-ಯಿಲ್ಲ
ಯೋ
ಯೋಗ
ಯೋಗಕ್ಕೂ
ಯೋಗಕ್ಷೇಮ-ವನ್ನು
ಯೋಗಕ್ಷೇಮಾದಿ-ಗಳನ್ನು
ಯೋಗ-ಗಳ
ಯೋಗ-ಗಳನ್ನೂ
ಯೋಗ-ಗಳು
ಯೋಗದ
ಯೋಗ-ದೃಷ್ಟಿಯ
ಯೋಗ-ಭೋಗ
ಯೋಗ-ಮಾರ್ಗ-ದಲ್ಲಿ
ಯೋಗ-ವಲ್ಲ
ಯೋಗ-ವಾಸಿಷ್ಠ-ದಲ್ಲಿಯೂ
ಯೋಗವು
ಯೋಗವೂ
ಯೋಗ-ಸಹಾಯ
ಯೋಗ-ಸಹಾಯನೆ
ಯೋಗಾ-ನಂದ
ಯೋಗಾ-ನಂದ-ರಿಗೂ
ಯೋಗಾ-ನಂದರು
ಯೋಗಾ-ನಂದರೂ
ಯೋಗಾ-ನಂದಸ್ವಾಮಿ-ಗಳನ್ನು
ಯೋಗಾ-ನಂದಸ್ವಾಮಿ-ಗಳು
ಯೋಗಾಭ್ಯಾ-ಸಕ್ಕೆ
ಯೋಗಾಭ್ಯಾಸದ
ಯೋಗಿ
ಯೋಗಿ-ಗಳ
ಯೋಗಿ-ಗಳಿಗೆ
ಯೋಗಿ-ಗಳು
ಯೋಗಿ-ಗಳೂ
ಯೋಗಿಗೆ
ಯೋಗಿಯೂ
ಯೋಗಿ-ಯೊಬ್ಬ-ನಿಗೆ
ಯೋಗೀನ್
ಯೋಗೀಶ್ವರ
ಯೋಗ್ಯ
ಯೋಗ್ಯ-ಕಾಲ-ದಲ್ಲಿ
ಯೋಗ್ಯ-ತಮ-ವಾ-ದದ್ದು
ಯೋಗ್ಯ-ತಮವೇ
ಯೋಗ್ಯ-ತರ-ವಾ-ದದ್ದು
ಯೋಗ್ಯ-ತಾನು-ಸಾರ
ಯೋಗ್ಯತೆ
ಯೋಗ್ಯ-ತೆಗೆ
ಯೋಗ್ಯ-ತೆ-ಯನ್ನು
ಯೋಗ್ಯ-ತೆ-ಯನ್ನೂ
ಯೋಗ್ಯ-ನಲ್ಲ
ಯೋಗ್ಯ-ನಾದ
ಯೋಗ್ಯನೋ
ಯೋಗ್ಯ-ರಲ್ಲ
ಯೋಗ್ಯ-ರಾ-ಗಿ-ರುವಾಗ
ಯೋಗ್ಯ-ರಾ-ಗು-ವು-ದಿಲ್ಲ
ಯೋಗ್ಯ-ರೀತಿ-ಯಲ್ಲಿ
ಯೋಗ್ಯರು
ಯೋಗ್ಯ-ವಲ್ಲ
ಯೋಗ್ಯ-ವಲ್ಲ-ವೆಂದೇ
ಯೋಗ್ಯ-ವಾಗಿದೆ
ಯೋಗ್ಯ-ವಾಗಿ-ರುತ್ತವೆ
ಯೋಗ್ಯ-ವಾಗಿ-ರುವ
ಯೋಗ್ಯ-ವಾದ
ಯೋಗ್ಯವೋ
ಯೋಗ್ಯಾ-ಯೋಗ್ಯ-ತೆ-ಯನ್ನು
ಯೋಚನಾ
ಯೋಚನಾ-ತರಂಗ-ಗಳ
ಯೋಚನಾ-ತರಂಗ-ಗಳು
ಯೋಚನಾ-ತರಂಗ-ದಿಂದ
ಯೋಚನಾ-ಮಗ್ನ-ರಾದರು
ಯೋಚನಾ-ಮಗ್ನ-ವಾದ
ಯೋಚನಾ-ಶಕ್ತಿಯ
ಯೋಚನೆ
ಯೋಚನೆ-ಗಳನ್ನು
ಯೋಚನೆ-ಗಳನ್ನೆಲ್ಲಾ
ಯೋಚನೆ-ಗಳು
ಯೋಚನೆ-ಯನ್ನು
ಯೋಚನೆ-ಯನ್ನೇ
ಯೋಚನೆ-ಯಲ್ಲಿ
ಯೋಚನೆ-ಯಾ-ಗಲಿ
ಯೋಚನೆ-ಯಾಗಲೀ
ಯೋಚನೆ-ಯಿಂದ
ಯೋಚ-ನೆಯೇ
ಯೋಚಿ-ಸದೆ
ಯೋಚಿಸ-ಬಲ್ಲ
ಯೋಚಿಸ-ಬ-ಹುದು
ಯೋಚಿಸ-ಬೇಕಾಗಿಲ್ಲ
ಯೋಚಿಸ-ಬೇ-ಕಾದ್ದಿಲ್ಲ
ಯೋಚಿಸ-ಬೇಕು
ಯೋಚಿಸ-ಬೇಡ
ಯೋಚಿ-ಸಲಾ-ರಂಭಿಸಿ-ದೆವು
ಯೋಚಿ-ಸ-ಲಿಲ್ಲವೆ
ಯೋಚಿ-ಸಲು
ಯೋಚಿಸಿ
ಯೋಚಿಸಿ-ಕೊಂಡಿದ್ದೀಯೊ
ಯೋಚಿಸಿ-ಕೊಂಡು
ಯೋಚಿ-ಸಿದ
ಯೋಚಿಸಿ-ದನು
ಯೋಚಿಸಿ-ದರೂ
ಯೋಚಿಸಿ-ದರೆ
ಯೋಚಿಸಿ-ದಾಗ
ಯೋಚಿ-ಸಿದೆ
ಯೋಚಿ-ಸಿದ್ದು
ಯೋಚಿಸಿ-ನೋಡಿ
ಯೋಚಿಸಿ-ನೋಡು
ಯೋಚಿಸು
ಯೋಚಿ-ಸುತ್ತ
ಯೋಚಿ-ಸುತ್ತಾ
ಯೋಚಿಸುತ್ತಾನೋ
ಯೋಚಿಸುತ್ತಿದ್ದ
ಯೋಚಿಸುತ್ತಿದ್ದರು
ಯೋಚಿಸುತ್ತಿದ್ದರೆ
ಯೋಚಿಸುತ್ತಿದ್ದ-ವನು
ಯೋಚಿಸುತ್ತಿದ್ದೆ
ಯೋಚಿಸುತ್ತಿ-ರ-ಲಿಲ್ಲ
ಯೋಚಿಸುತ್ತಿರುವೆ
ಯೋಚಿಸುತ್ತಿರು-ವೆ-ನೆಂದು
ಯೋಚಿಸುತ್ತೇನೆ
ಯೋಚಿ-ಸುವ
ಯೋಚಿಸು-ವಂತಹ
ಯೋಚಿಸು-ವಂತಾಗು-ವುದು
ಯೋಚಿಸು-ವಂತೆ
ಯೋಚಿಸು-ವರು
ಯೋಚಿಸು-ವರೊ
ಯೋಚಿಸು-ವರೋ
ಯೋಚಿಸು-ವಿರಿ
ಯೋಚಿಸು-ವುದಕ್ಕಾ-ಗಲಿ
ಯೋಚಿಸು-ವುದಕ್ಕೆ
ಯೋಚಿಸು-ವು-ದ-ರಿಂದ
ಯೋಚಿಸು-ವು-ದಿಲ್ಲ
ಯೋಚಿಸು-ವುದು
ಯೋಚಿಸು-ವುದೂ
ಯೋಚಿಸು-ವು-ದೆಲ್ಲಾ
ಯೋಚಿ-ಸುವೆ
ಯೋಚಿಸು-ವೆ-ಯೇನು
ಯೋಜನ
ಯೋಜನೆ
ಯೋಜನೆ-ಗಳ
ಯೋಜನೆ-ಗಳನ್ನು
ಯೋಜನೆ-ಯನ್ನು
ಯೋಜನೆ-ಯಿಂದ
ಯೋಜನೆಯೂ
ಯೋಧ-ದಲ
ಯೋಧ-ನಲ್ಲವೆ
ಯೋಧರ
ಯೋಧ-ರಾಗಿದ್ದರೆ
ಯೌವನ-ದಲ್ಲಿ
ಯೌವನ-ದಲ್ಲಿಯೇ
ರ
ರಂಗಕ್ಕೆ
ರಂಗ-ದಲ್ಲಿ
ರಂಗಿನ
ರಂಗಿನೋಕುಳಿ
ರಂಜಿ-ಸುತ್ತಿದೆ
ರಂದು
ರಕ್ತ
ರಕ್ತ-ಕಾಯ
ರಕ್ತ-ಗತ-ಮಾಡಿ-ಕೊಂಡು
ರಕ್ತ-ಗತ-ಮಾಡಿ-ಕೊಳ್ಳಲು
ರಕ್ತ-ಗತ-ವಾಗಿದೆ
ರಕ್ತ-ಗಳು
ರಕ್ತ-ಚಲ-ನೆಯಾಗಬೇ-ಕಾ-ಯಿತು
ರಕ್ತ-ದಿಂದ
ರಕ್ತ-ಧಾರಾ
ರಕ್ತ-ನಾಳ-ಗಳಲ್ಲಿ
ರಕ್ತ-ನಾಳ-ದಲ್ಲಿಯೂ
ರಕ್ತ-ಪಾತ-ವಿಲ್ಲದೆ
ರಕ್ತ-ಪಾವಿತ್ರ್ಯ-ತೆ-ಯನ್ನು
ರಕ್ತ-ವನ್ನು
ರಕ್ತವು
ರಕ್ತ-ಸಂಪರ್ಕ-ವನ್ನು
ರಕ್ತ-ಸಂಬಂಧ-ವನ್ನು
ರಕ್ಷಕಳಾ-ಗಲಿ
ರಕ್ಷಣ-ವಿಲ್ಲ
ರಕ್ಷಣೆ-ಗಾಗಿ
ರಕ್ಷಣೆಗೆ
ರಕ್ಷಣೆ-ಯನ್ನು
ರಕ್ಷಣೆ-ಯಲ್ಲಿ
ರಕ್ಷಿಸ-ಬೇಕು
ರಕ್ಷಿ-ಸಲಿ
ರಕ್ಷಿ-ಸಲು
ರಕ್ಷಿ-ಸಲ್ಪಟ್ಟ
ರಕ್ಷಿ-ಸಿಕೊ
ರಕ್ಷಿಸು
ರಕ್ಷಿಸುತ್ತೇವೆ
ರಕ್ಷಿ-ಸುವ
ರಕ್ಷಿಸು-ವೆನೊ
ರಕ್ಷಿಸು-ವೆಯೊ
ರಕ್ಷೆಯೇ
ರಘು-ನಂದನ
ರಘು-ನಂದ-ನನ
ರಘು-ನಂದ-ನನು
ರಘುರಾಈ
ರಘು-ವಂಶದ
ರಚನ
ರಚನಾ
ರಚನಾ-ಶಾಸ್ತ್ರ
ರಚನಾ-ಶಾಸ್ತ್ರವೂ
ರಚನೆ
ರಚ-ನೆಯ
ರಚನೆ-ಯಲ್ಲಿ
ರಚ-ನೆಯೂ
ರಚಿತ-ವಾಗಿತ್ತು
ರಚಿತ-ವಾಗಿದ್ದಿತು
ರಚಿತ-ವಾಗಿದ್ದು
ರಚಿತ-ವಾ-ದುವು
ರಚಿಸ-ಬಲ್ಲೆ-ಯಾ-ದರೆ
ರಚಿಸ-ಬೇಕೆಂದು
ರಚಿ-ಸ-ಬೇಕೆಂಬ
ರಚಿ-ಸಲ್ಪಟ್ಟ
ರಚಿ-ಸಲ್ಪಟ್ಟಿದ್ದಲ್ಲ
ರಚಿ-ಸಲ್ಪಟ್ಟಿದ್ದು
ರಚಿಸಿ
ರಚಿಸಿ-ಕೊಂಡು
ರಚಿ-ಸಿದ
ರಚಿಸಿ-ದರು
ರಚಿಸಿದ್ದನು
ರಚಿಸಿದ್ದರು
ರಚಿ-ಸಿದ್ದು
ರಚಿಸಿ-ರುವ
ರಜಸ್ಸು
ರಜೋ
ರಜೋ-ಗುಣ
ರಜೋ-ಗುಣದ
ರಜೋ-ಗುಣ-ವನ್ನು
ರಜೋ-ಗುಣವು
ರಜೋ-ಗುಣವೂ
ರಜೋ-ಭಾವ-ವನ್ನು
ರಜ್ಜು-ಸತ್ತೆ
ರಣ-ರಂಗ
ರಣ-ರಂಗದ
ರಣ-ರಂಗ-ದಲ್ಲಿ
ರಣಸ್ಥಲ
ರಣ-ಹದ್ದು-ಗಳಂತೆ
ರತ್ನ-ಗಳನ್ನ-ವ-ರಿಗೆ
ರತ್ನ-ಗಳು
ರಥ
ರಥ-ಗಳ
ರಥ-ಗಳನ್ನೇ
ರಥ-ಗ-ಳಿದ್ದವು
ರಥ-ಗಳಿ-ರ-ಲಿಲ್ಲವೆ
ರಥ-ಗಳು
ರಥದ
ರಥ-ದಲ್ಲಿ
ರಥ-ವನ್ನೇರಿ
ರಥ-ವಿದೆ-ಯಷ್ಟೆ
ರಥಿನಂ
ರಥೆ
ರಥೇ
ರದುವೆ
ರನದ
ರನದ-ಬಾಬು
ರನದ-ಬಾಬು-ಗಳ
ರನದ-ಬಾಬು-ಗಳಿಗೆ
ರನದ-ಬಾಬು-ಗಳು
ರನದ-ಬಾಬು-ವಿಗೆ
ರಫ್ತು
ರಭಸ
ರಭಸ-ದಲಿ
ರಭಸ-ದಿಂದ
ರಭಸ-ವನು
ರಭಸ-ವನ್ನು
ರಮಂತಃ
ರಮಣೀಯ
ರಮಣೀಯತೆ
ರಮಣೀಯ-ತೆ-ಯನ್ನು
ರಮಾ-ಕಾಂತ
ರಲ್ಲಿ
ರವಿ
ರವಿ-ಚಂದ್ರ-ತಾರೆ-ಗಳು
ರವಿ-ಯಿಳಿಯೆ
ರವಿಯು
ರವಿ-ಯೆ-ಡೆಗೆ
ರವಿ-ಶಶಿ-ತಾರ-ಕಗ್ರಹ-ನೀ-ಹಾರಿಕೆ
ರಸ-ಗವ-ಳವೂ
ರಸ-ಭಂಗ
ರಸ-ಭಂಗ-ವಾ-ದಂತೆ
ರಸ-ವತ್ತಾಗಿ
ರಸ-ವತ್ತಾಗಿ-ರ-ಬಲ್ಲದು
ರಸ-ವನ್ನು
ರಸಾ-ತಲ
ರಸಾಯನ
ರಸಾಯನ-ಶಾಸ್ತ್ರ
ರಸ್ತೆ
ರಸ್ತೆ-ಗಳಲ್ಲಿ
ರಸ್ತೆ-ಗಿಳಿ-ದರೆ
ರಸ್ತೆಗೆ
ರಸ್ತೆಯ
ರಸ್ತೆ-ಯಲ್ಲಿ
ರಸ್ತೆ-ಯಿಂದಲೇ
ರಸ್ತೆಯೂ
ರಹಸ್ಯ
ರಹಸ್ಯ-ಗಳನ್ನು
ರಹಸ್ಯ-ಗಳು
ರಹಸ್ಯ-ಗಳೂ
ರಹಸ್ಯ-ವನ್ನು
ರಹಸ್ಯ-ವಲ್ಲ
ರಹಸ್ಯ-ವಾಗಿ
ರಹಸ್ಯ-ವಿದೆ
ರಹಸ್ಯ-ವೇ-ನೆಂದರೆ
ರಹಿತ-ವಾ-ದುದು
ರಾಕ್ಷಸ
ರಾಕ್ಷಸ-ನಂತೆ
ರಾಕ್ಷ-ಸರು
ರಾಕ್ಷ-ಸಾ-ಕಾರ-ವನ್ನು
ರಾಗ
ರಾಗ-ಗಳನ್ನು
ರಾಗ-ಗಳು
ರಾಗಚ್ಛಟಾ
ರಾಗ-ತಾಳಲ-ಯ-ದಲೆ-ಯಲೆ-ಯಲ್ಲಿ
ರಾಗ-ದಲ್ಲೂ
ರಾಗ-ದಾ-ವೇಶ-ದಲಿ
ರಾಗ-ದಿಂದ
ರಾಗ-ಪರಿಚಯ
ರಾಗ-ವನ್ನು
ರಾಗ-ವನ್ನೂ
ರಾಗ-ವಾಗಿ
ರಾಗವು
ರಾಗವೆ
ರಾಗ-ಸಂಯೋ-ಜನೆ
ರಾಗಾನುಗಾ
ರಾಗಿ-ಣಿ-ಗಳ
ರಾಗಿ-ಣಿ-ಗಳನ್ನು
ರಾಗಿ-ಣಿ-ಗಳಲ್ಲಿ
ರಾಗೇ-ಕೃತೇ
ರಾಜ
ರಾಜ-ಕೀಯ
ರಾಜ-ಕೀಯ-ದಲ್ಲಿ
ರಾಜ-ಕುಮಾರ-ನಿಗೂ
ರಾಜ-ಗೃಹ-ದಲ್ಲಿ
ರಾಜ-ಧಾನಿಯ
ರಾಜ-ಧಾನಿಯಿಂದ
ರಾಜನ
ರಾಜ-ನಾದ-ವನು
ರಾಜ-ನೀತಿ
ರಾಜನು
ರಾಜ-ಪೀಠದಿ
ರಾಜ-ಪುಟಾಣ-ದಲ್ಲಿ-ರುವ
ರಾಜ-ಪುಟಾ-ಣವು
ರಾಜ-ಪು-ತಾನ-ದಲ್ಲಿ
ರಾಜ-ಯೋಗ-ಗಳ
ರಾಜ-ಯೋಗದ
ರಾಜ-ಯೋಗ-ದಲ್ಲಿ
ರಾಜ-ಯೋಗ-ವೆಂಬ
ರಾಜ-ಯೋಗ್ಯ-ವಾದ
ರಾಜರ
ರಾಜ-ರದು
ರಾಜ-ರಾಗಿದ್ದರು
ರಾಜ-ರಾದ
ರಾಜ-ರಿಂದಲ್ಲದೆ
ರಾಜರು
ರಾಜ-ವಂಶಕ್ಕೆ
ರಾಜ-ವರ್ಗದ
ರಾಜ-ವಲ್ಲಭ
ರಾಜಸ
ರಾಜ-ಸಕ್ಕೆ
ರಾಜ-ಸ-ಗುಣ
ರಾಜ-ಸಭೆ
ರಾಜ-ಸಿಂಹಾಸ-ನ-ದಿಂದಲೇ
ರಾಜ-ಸಿಕ
ರಾಜ-ಹಂಸಃ
ರಾಜ-ಹಂಸವು
ರಾಜಾ
ರಾಜಾ-ಧಿ-ರಾಜ-ನಿ-ಗಿಂತಲೂ
ರಾಜಿ
ರಾಜಿಗೆ
ರಾಜಿ-ಮಾಡಿ-ಕೊಂಡಿದ್ದರೆ
ರಾಜಿ-ಯನ್ನು
ರಾಜಿ-ಸುವಳೋ
ರಾಜೇ
ರಾಜೇಂದ್ರಮಲ್ಲಿಕ
ರಾಜ್ಯ
ರಾಜ್ಯಕ್ಕೆ
ರಾಜ್ಯದ
ರಾಜ್ಯ-ದಲ್ಲಿ
ರಾಜ್ಯ-ದಲ್ಲೂ
ರಾಜ್ಯವು
ರಾಜ್ಯಾಡಳಿತ-ವನ್ನು
ರಾಣಿ
ರಾಣಿ-ರಾಸ-ಮ-ಣಿಯ
ರಾತ್ರಿ
ರಾತ್ರಿಂ
ರಾತ್ರಿ-ಗ-ಳಲ್ಲ
ರಾತ್ರಿಯ
ರಾತ್ರಿ-ಯನ್ನು
ರಾತ್ರಿ-ಯಲ್ಲಿ
ರಾತ್ರಿ-ಯಾ-ಯಿತು
ರಾತ್ರಿ-ಯಿಂದ
ರಾತ್ರಿಯೂ
ರಾತ್ರಿ-ಯೂ-ಟಕ್ಕೆ
ರಾತ್ರೆಯೂ
ರಾಧಾ
ರಾಧಾ-ಕಾಂತ
ರಾಧಾ-ಕೃಷ್ಣರ
ರಾಧಾಪ್ರೇಮ-ದಲ್ಲಿ
ರಾಧಾಪ್ರೇಮ-ವನ್ನು
ರಾಧೆ
ರಾಧೆ-ಯನ್ನು
ರಾಧೆ-ಯಾಗಿ
ರಾಮ
ರಾಮಃ
ರಾಮ-ಕೃಷ್ಣ
ರಾಮ-ಕೃಷ್ಣ-ಗತಪ್ರಾಣರೇ
ರಾಮ-ಕೃಷ್ಣ-ದಾನೀಮ್
ರಾಮ-ಕೃಷ್ಣ-ದಾಸಾ
ರಾಮ-ಕೃಷ್ಣನ
ರಾಮ-ಕೃಷ್ಣ-ಪುರ
ರಾಮ-ಕೃಷ್ಣ-ಪುರದ
ರಾಮ-ಕೃಷ್ಣ-ಪುರ-ದಲ್ಲಿ
ರಾಮ-ಕೃಷ್ಣ-ಪುರ-ವೆಂದು
ರಾಮ-ಕೃಷ್ಣಮ್
ರಾಮ-ಕೃಷ್ಣರ
ರಾಮ-ಕೃಷ್ಣರು
ರಾಮ-ಕೃಷ್ಣಸ್ತನುಂ
ರಾಮ-ಕೃಷ್ಣಾ-ನಂದ
ರಾಮ-ಕೃಷ್ಣಾ-ನಂದ-ರಿಗೆ
ರಾಮ-ಕೃಷ್ಣಾ-ನಂದರು
ರಾಮ-ಕೃಷ್ಣಾ-ನಂದರೇ
ರಾಮ-ಕೃಷ್ಣಾಯ
ರಾಮ-ಕೃಷ್ಣೇ
ರಾಮ-ಚಂದ್ರ
ರಾಮ-ಚಂದ್ರನ
ರಾಮನ
ರಾಮ-ನನ್ನು
ರಾಮ-ನನ್ನೇ
ರಾಮ-ನಾಡಿ-ನಲ್ಲಿದ್ದಾಗ
ರಾಮ-ನಾಮ
ರಾಮ-ನಾಮ-ವನ್ನು
ರಾಮನೊ
ರಾಮ-ನೊಂದಿಗೆ
ರಾಮಪ್ರಸಾದನ
ರಾಮಬ್ರಹ್ಮ
ರಾಮಬ್ರಹ್ಮ-ಬಾಬು
ರಾಮಬ್ರಹ್ಮ-ಬಾಬು-ಗಳ
ರಾಮಬ್ರಹ್ಮ-ಬಾಬು-ಗಳು
ರಾಮಬ್ರಹ್ಮ-ಬಾಬು-ಗಳೊ-ಡನೆ
ರಾಮ-ರಾಮ
ರಾಮ-ರಾಮೇತಿ
ರಾಮ-ಲಾಲ
ರಾಮ-ಸೀತೆ-ಯ-ರಲ್ಲಿ
ರಾಮ-ಸೀತೆ-ಯರು
ರಾಮಾನುಜ
ರಾಮಾನುಜರ
ರಾಮಾನುಜಾ-ಚಾರ್ಯರು
ರಾಮಾಯಣ
ರಾಮಾಯಣ-ವನ್ನು
ರಾಯಬಹದ್ದೂರ-ರನ್ನು
ರಾವಣ
ರಾವಣ-ನನ್ನು
ರಾಶಿ
ರಾಶಿ-ಮಾಡಿ
ರಾಶಿ-ಯನ್ನು
ರಾಶಿ-ಯಲ್ಲಿ
ರಾಶಿ-ರಾಶಿ-ಗಳೆ
ರಾಶಿ-ರಾಶಿ-ಯಾಗಿ
ರಾಷ್ಟ್ರ
ರಾಷ್ಟ್ರ-ಗಳ
ರಾಷ್ಟ್ರದ
ರಾಷ್ಟ್ರ-ವನ್ನು
ರಾಷ್ಟ್ರವು
ರಾಷ್ಟ್ರೀ-ಕರ-ಣ-ಗೊಂಡಿತ್ತು
ರಾಷ್ಟ್ರೀಯ
ರಾಷ್ಟ್ರೀಯತ್ವವು
ರಾಸ-ಮ-ಣಿಯ
ರಾಹ್ರೀ
ರಿಕ್ತ
ರಿಜ್ಲಿ
ರೀತಿ
ರೀತಿ-ಗಳು
ರೀತಿ-ನೀತಿ-ಗಳ
ರೀತಿ-ನೀತಿ-ಗಳನ್ನು
ರೀತಿ-ನೀತಿ-ಗಳಲ್ಲಿ
ರೀತಿಯ
ರೀತಿ-ಯನ್ನನು-ಸ-ರಿಸಿ
ರೀತಿ-ಯನ್ನು
ರೀತಿ-ಯನ್ನೂ
ರೀತಿ-ಯನ್ನೆಲ್ಲಾ
ರೀತಿ-ಯಲ್ಲಿ
ರೀತಿ-ಯಲ್ಲಿದೆ
ರೀತಿ-ಯಲ್ಲೇ
ರೀತಿ-ಯಾಗಿ
ರೀತಿ-ಯಾ-ಗುತ್ತಿದೆ
ರೀತಿ-ಯಿಂದಲೂ
ರೀತಿ-ಯಿದೆ
ರೀತಿಯೂ
ರೀತಿಯೇ
ರುಂಡ-ಮಾಲೆ
ರುಗ್ಣಾವಸ್ಥೆ-ಯಲ್ಲಿದ್ದ
ರುಚಿ
ರುಚಿ-ಕರ-ವಾದ
ರುಚಿ-ಯನ್ನು
ರುಚಿ-ಯಾಗಿ
ರುಚಿ-ಯಾ-ಗಿದೆ
ರುಚಿ-ಯಾದ
ರುಚಿ-ಯಿಲ್ಲ
ರುಚಿ-ಸದು
ರುಚಿ-ಸ-ಲಿಲ್ಲ
ರುಚಿ-ಸಿದ್ದು
ರುಜು-ಹಾಕಿ-ಕೊಂಡು
ರುದ್ರ-ಮುಖದಿ
ರುದ್ರ-ಮುಖ-ವನು
ರುದ್ರ-ಮುಖೆ
ರುದ್ರಾಕ್ಷಿ
ರುದ್ರಾಕ್ಷಿಯ
ರುದ್ರಾಕ್ಷಿ-ಯನ್ನೂ
ರುದ್ರಾಕ್ಷಿ-ವ-ಲಯ
ರುಧಿ-ರಾರು-ಣ-ವಹ
ರುಧೀರ
ರುಧೀರೇರ
ರುಮಾಲಿ-ನಂತೆ
ರುಮಾಲು
ರೂಢಾ
ರೂಢಿಗೆ
ರೂಢಿಯ
ರೂಢಿಯಲ್ಲಿದೆ
ರೂಢಿಯಲ್ಲಿ-ರುವ
ರೂಢಿ-ಯಾಗಿತ್ತು
ರೂಢಿಸ-ಬೇಕು
ರೂಢಿಸಿ-ಕೊಂಡು
ರೂಢಿಸಿ-ಕೊಳ್ಳ-ಬ-ಹುದು
ರೂಢಿ-ಸಿ-ರುವರು
ರೂಪ
ರೂಪಃ
ರೂಪಕ
ರೂಪ-ಕ-ವಾಗಿ
ರೂಪಕ್ಕೆ
ರೂಪ-ಗಳ
ರೂಪ-ಗಳು
ರೂಪ-ಗಳೆಂದರೂ
ರೂಪ-ಗಳೆ-ನಿತೊ
ರೂಪ-ಗೊಂಡಿದೆ
ರೂಪ-ತಾಳಿದ
ರೂಪ-ತಾಳು-ವುವು
ರೂಪದ
ರೂಪ-ದಲಿ
ರೂಪ-ದಲ್ಲಿ
ರೂಪ-ದಲ್ಲಿಯೂ
ರೂಪ-ದಲ್ಲಿಯೇ
ರೂಪ-ದಿಂದ
ರೂಪ-ಮುಗ್ಧ
ರೂಪ-ರಾಗ
ರೂಪ-ವನ್ನು
ರೂಪ-ವನ್ನೂ
ರೂಪ-ವಾಗಿ
ರೂಪ-ವಾಗಿತ್ತೇ
ರೂಪ-ವಾಗಿದ್ದ
ರೂಪ-ವಾದ
ರೂಪವು
ರೂಪವೇ
ರೂಪವೊ
ರೂಪಾಂತರ
ರೂಪಾಂತರ-ಗಳನ್ನೂ
ರೂಪಾಂತರ-ವನ್ನು
ರೂಪಾಯಿ
ರೂಪಾ-ಯಿ-ಗಳನ್ನು
ರೂಪಾ-ಯಿ-ಯಿಂದ
ರೂಪಿ-ಸ-ಬಲ್ಲೆ-ಯಾ-ದರೆ
ರೂಪಿ-ಸ-ಬೇಕೆಂದಿದ್ದೇನೆ
ರೂಪಿ-ಸಿ-ಕೊಂಡಿದ್ದಷ್ಟೇ
ರೂಪಿ-ಸಿ-ಕೊಳ್ಳ-ಬೇಕು
ರೂಪಿ-ಸಿ-ಕೊಳ್ಳ-ಬೇಕೆಂದು
ರೂಪಿ-ಸಿ-ಕೊಳ್ಳಿ
ರೂಪಿ-ಸಿ-ಕೊಳ್ಳುತ್ತಾನೆ
ರೂಪಿ-ಸಿ-ಕೊಳ್ಳು-ವುದಕ್ಕೆ
ರೂಪಿ-ಸುವ
ರೂಪಿ-ಸು-ವುದು
ರೂಪು-ಗೊಳ್ಳುತ್ತದೆ
ರೂಪು-ಗೊಳ್ಳುವಂತೆ
ರೂಪುಗೊಳ್ಳು-ವು-ದಾದರೆ
ರೂಪು-ಗೊಳ್ಳುವುದೊ
ರೂಪುದಾ-ಳಿದಂತಿ-ರುವ
ರೂಪುದಾಳಿ-ರು-ವು-ದನ್ನು
ರೂಪೆ
ರೂವಾರಿ-ಯಾಗಿದ್ದಾನೆ
ರೆಕ್ಕೆ-ಗಳ
ರೇ
ರೇಖೆ
ರೇಖೆ-ಗಳೆ-ರಡೂ
ರೇಗಿ
ರೇಗಿ-ಬಿಡುತ್ತಾರೆ
ರೇತೆ
ರೈಟ್
ರೈತ
ರೈತ-ನಾಗಿಯೆ
ರೈತರ
ರೈಲಿ-ನಲ್ಲಿ
ರೈಲು
ರೈಲ್ವೆ
ರೊಚ್ಚಿಗೇಳು-ವುದು
ರೊಟ್ಟಿ
ರೊಟ್ಟಿ-ಗಾಗಿ
ರೊವೆ
ರೋಗ
ರೋಗ-ಗಳನ್ನು
ರೋಗ-ಗ-ಳಾದರೋ
ರೋಗ-ಗಳಿಂದ
ರೋಗಗ್ರಸ್ತ
ರೋಗದ
ರೋಗ-ದಿಂದ
ರೋಗ-ರುಜಿನ
ರೋಗ-ರುಜಿನ-ಗಳು
ರೋಗ-ವನ್ನು
ರೋಗ-ವಾಗಿ
ರೋಗವೇ
ರೋಗಿ-ಗಳ
ರೋಗಿ-ಗಳಿಗೆ
ರೋಗಿ-ಗಳು
ರೋದನ
ರೋದನ-ದಿಂದ
ರೋದಿಸ-ಬೇ-ಕಾದರೆ
ರೋದಿ-ಸಿರೆ
ರೋಮ
ರೋಮ-ಗಳಿಂದಲೂ
ರೋಮ-ನರ
ರೋಮನ್
ರೋಮನ್ನರ
ರೋಮನ್ನ-ರಿಗೆ
ರೋಲೆ
ರೋಷ
ರೋಷಾ-ವೇಶ
ರೌದ್ರಕ್ಕಾಗಿ
ರೌದ್ರ-ತೆ-ಗಳ
ರೌದ್ರತೆ-ಯನ್ನು
ರೌದ್ರದ
ರೌದ್ರ-ವನ್ನು
ರೌದ್ರ-ವಾಗಿಯೇ
ರೌದ್ರ-ವೆಂದು
ಲಂಕಾ-ನಗರ-ದಲ್ಲಿ
ಲಂಕೆ
ಲಂಕೆ-ಯಲ್ಲಿ-ರುವ
ಲಂಗ
ಲಂಗ-ರಂಗ-ಭಂಗ
ಲಂಗರು
ಲಂಘಿತೆ
ಲಂಡನ್ನಿ-ನಿಂದ
ಲಂಪ
ಲಕ್ಕೋಪ-ಲಕ್ಷ
ಲಕ್ಷ
ಲಕ್ಷ-ಗಟ್ಟಲೆ
ಲಕ್ಷಣ
ಲಕ್ಷ-ಣ-ಗಳು
ಲಕ್ಷ-ಣ-ಗ-ಳೆಂದು
ಲಕ್ಷ-ಣವು
ಲಕ್ಷ-ಣ-ವೇ-ನೆಂದು
ಲಕ್ಷ-ಲಕ್ಷ
ಲಕ್ಷಾಂತರ
ಲಕ್ಷಿಸು-ವ-ವ-ರಾರು
ಲಕ್ಷ್ಮಿಯ
ಲಕ್ಷ್ಯ
ಲಕ್ಷ್ಯ-ಕೊಟ್ಟು
ಲಕ್ಷ್ಯಕ್ಕೆ
ಲಕ್ಷ್ಯ-ಮಾ-ಡದೆ
ಲಕ್ಷ್ಯ-ವನ್ನು
ಲಕ್ಷ್ಯ-ವನ್ನೇ
ಲಕ್ಷ್ಯ-ವಿಟ್ಟು
ಲಕ್ಷ್ಯವೇ
ಲಗಾಮನ್ನು
ಲಗಾಮು
ಲಘು-ವಾಗಿ
ಲಘು-ವಾದ
ಲಜ್ಜೆ
ಲಭಿಸಿ
ಲಭಿ-ಸಿದೆ
ಲಭಿಸಿ-ರುವಾಗ
ಲಭಿ-ಸಿಲ್ಲ
ಲಭಿ-ಸುತ್ತದೆ
ಲಭಿ-ಸು-ವು-ದಿಲ್ಲ
ಲಭಿ-ಸು-ವುದು
ಲಭ್ಯಃ
ಲಭ್ಯ-ವಾಗಿಲ್ಲ
ಲಭ್ಯ-ವಿ-ದೆಯೇ
ಲಭ್ಯವಿದ್ದಿದ್ದರೆ
ಲಯ
ಲಯ-ಕರ್ತ-ನೆಂದು
ಲಯ-ಕಾಲ-ದಲ್ಲಿ
ಲಯ-ಗಳು
ಲಯ-ವಾಗಲು
ಲಯ-ವಾಗಿ
ಲಯ-ವಾಗಿದೆ
ಲಯ-ವಾಗು-ವುದು
ಲಯ-ವಾ-ದುದು
ಲಯ-ವಾ-ಯಿತು
ಲಯವೂ
ಲಯೆ
ಲರ್ಕ್ಷೀಃ
ಲಲಾಟ-ದಲ್ಲಿ
ಲಲಿತ
ಲಲಿ-ತವಃ
ಲಲಿತ-ವಿಲಾಸ-ಗಳಿಂದ
ಲವ-ಲೇಶ-ವಾದರೂ
ಲವ-ಲೇಶವೂ
ಲಹರಿ
ಲಾಖ್
ಲಾಗೆ
ಲಾಭ
ಲಾಭಕ್ಕಾಗಿ
ಲಾಭ-ವನ್ನು
ಲಾಭ-ವಾಗು-ವುದು
ಲಾಭ-ವಾಗು-ವುದೆಂದು
ಲಾಭ-ವುಂಟಾಗು-ವುದು
ಲಾಭವೆ
ಲಾಭ-ವೇನು
ಲಾರ್ಡ್
ಲಾಳಿಯಾಡು-ತಿದೆ-ವಸ-ನ-ವನು
ಲಿಂಗ-ಭಾವನೆ
ಲಿಂಗ-ಭೇದ
ಲಿಂಗ-ಭೇದ-ವನ್ನು
ಲಿಂಗ-ಭೇದ-ವಿಲ್ಲ
ಲಿಂಗವ-ರಿ-ಯದ
ಲಿಂಗಾತೀತ-ವಾದ
ಲಿಖಿತ
ಲಿಖಿತ-ವಾಗಿ-ರುವ
ಲಿಚಿ
ಲಿಪಿ-ಗಳನ್ನೊಳ-ಗೊಂಡ
ಲಿಪಿ-ಬದ್ಧ-ವಲ್ಲದ
ಲೀನ-ಗಳನ್ನಾಗಿ
ಲೀನಪ್ರಾಯ-ವಾಗಿ-ರುತ್ತದೆ
ಲೀನ-ಮಾಡಿ
ಲೀನ-ರಾಗಿ
ಲೀನ-ವಾಗಿ
ಲೀನ-ವಾಗುವ
ಲೀನ-ವಾಗು-ವುದು
ಲೀನ-ವಾಗು-ವುದೋ
ಲೀನ-ವಾ-ದು-ದನ್ನು
ಲೀಲಾ
ಲೀಲಾ-ದರ್ಶನ
ಲೀಲಾ-ನಾಟಕ-ದಲ್ಲಿ
ಲೀಲಾ-ರಂಗವು
ಲೀಲಾ-ವತಿ
ಲೀಲೆ
ಲೀಲೆ-ಗಳ
ಲೀಲೆ-ಗಾಗಿ
ಲೀಲೆಯ
ಲೀಲೆ-ಯನ್ನು
ಲೀಲೆ-ಯಲ್ಲಿ
ಲೀಲೆ-ಯಲ್ಲಿಯೇ
ಲೀಲೆ-ಯಾಗಿ
ಲುಕಾಯೆ
ಲುಪ್ತ-ವಾಗಿ
ಲುಪ್ತ-ವಾಗಿ-ದೆಯೆಂದರೂ
ಲುಪ್ತ-ವಾಗಿ-ಬಿಡುತ್ತದೆ
ಲುಪ್ತ-ವಾಗುತ್ತದೆ
ಲುಪ್ತ-ವಾದರೆ
ಲೆಕ್ಕ
ಲೆಕ್ಕದ
ಲೆಕ್ಕ-ಮಾಡಿ-ಕೊಂಡು
ಲೆಕ್ಕ-ವಿಲ್ಲ
ಲೆಕ್ಕ-ವಿಲ್ಲದ
ಲೆಕ್ಕ-ವಿಲ್ಲ-ದಷ್ಟು
ಲೆಕ್ಕವೇ
ಲೆಕ್ಕಿ-ಸದಲೆ
ಲೆಕ್ಕಿ-ಸದೆ
ಲೆಕ್ಕಿ-ಸ-ಲಾರ-ದಷ್ಟು
ಲೆಕ್ಕಿ-ಸೆನು
ಲೆಗೆಟ್
ಲೇಖ-ಕರು
ಲೇಖ-ನ-ಗಳನ್ನು
ಲೇಖನ-ವನ್ನು
ಲೇಖನಿ
ಲೇಪವಿರಲೇ-ಬೇಕು
ಲೇಶ-ಮಾತ್ರವೂ
ಲೇಸಲ್ಲವೆ
ಲೇಸೆಂದೂ
ಲೈಬ್ರರಿ-ಯಿಂದ
ಲೋಕ
ಲೋಕಂ
ಲೋಕ-ಕಲ್ಯಾಣ-ಕಾರಿ-ಯಾಗಿ-ರ-ಬೇಕು
ಲೋಕ-ಕಲ್ಯಾಣಕ್ಕಾಗಿಯೂ
ಲೋಕ-ಕಲ್ಯಾಣಕ್ಕೋಸ್ಕರ
ಲೋಕ-ಕಲ್ಯಾಣ-ತತ್ಪರ-ರಾದ
ಲೋಕ-ಕಲ್ಯಾಣ-ಮಾರ್ಗಮ್
ಲೋಕಕೆ
ಲೋಕಕ್ಕೆ
ಲೋಕ-ಗಳಿಗೂ
ಲೋಕದ
ಲೋಕ-ದಲ್ಲಾ-ಗಲಿ
ಲೋಕ-ದಲ್ಲಿ
ಲೋಕ-ದಷ್ಟೇ
ಲೋಕ-ವತ್ತು
ಲೋಕ-ವನೆ
ಲೋಕ-ವನ್ನು
ಲೋಕ-ವಿದು
ಲೋಕ-ವಿದೆ
ಲೋಕವು
ಲೋಕವೆ
ಲೋಕ-ವೆಲ್ಲವು
ಲೋಕ-ಸಂಗ್ರಹಾರ್ಥ-ವಾಗಿ
ಲೋಕ-ಹಿ-ತಾರ್ಥ-ವಾಗಿ
ಲೋಕಾ-ಚಾರ
ಲೋಕಾತೀ-ತಮ-ಹಿಮನ
ಲೋಕಾತೀತೋಪ್ಯಹಹ
ಲೋಕಾಭಿ-ರಾಮ-ವಾಗಿ
ಲೋಕಾ-ರೂಢಿ-ಯಾದ
ಲೋಟ
ಲೋಟ-ವನ್ನು
ಲೋಟಾ-ಗಳು
ಲೋಪ-ದೋಷ-ಗಳ
ಲೋಪವೂ
ಲೋಭ-ಮೆನಗಿಲ್ಲ
ಲೋಭ-ಮೋಹ
ಲೋಭಿಯೆ
ಲೋಲುಪ್ತಿ-ಯಲ್ಲಿ
ಲೋಹದ
ಲೋಹ-ಪಿಂಡ
ಲೌಕಿಕ
ಲೌಕಿಕ-ವಿಷಯ-ವಾಸನೆ
ಲೌಕಿಕ-ವೆಂದು
ಲ್ಯಾಟಿನ್
ಲ್ಯಾಪ್ಲ್ಯಾಂಡಿನ
ವಂಗ-ಭೂಮಿ-ಯಲ್ಲಿ
ವಂಚನ
ವಂಚನೆ-ಯಿಲ್ಲದೆ
ವಂದನೆ
ವಂದಿತೊಮಾಯ
ವಂದಿತೋಮಾಯ
ವಂದಿಸಿ
ವಂದಿಸು-ತ್ತೇವೆ
ವಂದಿಸು-ವಾತನೆ
ವಂದಿಸು-ವೆವು
ವಂಶ
ವಂಶ-ಗಳು
ವಂಶ-ದಲ್ಲಿ
ವಂಶ-ದಲ್ಲಿ-ರುವ
ವಂಶ-ದವರಿಗೆ
ವಂಶ-ದವರು
ವಂಶ-ಪಾರಂಪರ್ಯ-ವಾಗಿಯೇ
ವಂಶ-ಪಾರಂಪರ್ಯ-ವಾದ
ವಂಶ-ವೃದ್ಧಿ
ವಂಶಾವಳಿಯ
ವಂಶೀಯರು
ವಕೀಲ
ವಕೀಲನ
ವಕ್ತ್ರೋದ್ಧೃತೋಪಿ
ವಕ್ಷದಲಿ
ವಕ್ಷೇ
ವಚನ
ವಚನ-ವೇದ
ವಜ್ರ-ದಂಡದ
ವಜ್ರಸಮ
ವಟತಲ-ದಲ್ಲಿ
ವದ
ವದಂತೀಹ
ವದನ-ವನ್ನು
ವದನಾರವಿಂದ-ವನು
ವನಸ್ಪತಿ
ವಯಂ
ವಯಮ್
ವಯಲೆಟ್
ವಯಸಿಗೆ
ವಯಸ್ಕರಿಗೆ
ವಯಸ್ಸಾಗಿತ್ತು
ವಯಸ್ಸಾಗಿದ್ದ
ವಯಸ್ಸಾಗಿದ್ದರೂ
ವಯಸ್ಸಾದ
ವಯಸ್ಸಾದಂತೆಲ್ಲಾ
ವಯಸ್ಸಾದ-ಮೇಲೆ
ವಯಸ್ಸಾದಾಗ
ವಯಸ್ಸಿಗೆ
ವಯಸ್ಸಿನ
ವಯಸ್ಸಿ-ನಲ್ಲಿ
ವಯಸ್ಸಿನಿಂದ
ವಯಸ್ಸಿ-ನಿಂದಲೂ
ವಯಸ್ಸು
ವರ
ವರಣ್
ವರದಾ
ವರ-ನನ್ನು
ವರ-ಮಾನ-ವನ್ನು
ವರವೂ
ವರಾಂಡ
ವರಾಂಡದ
ವರಾಂಡ-ದಲ್ಲಿ
ವರಾನ್
ವರುಣರು
ವರುಷ
ವರುಷ-ಗಳ
ವರುಷ-ಗಳ-ವರೆಗೆ
ವರುಷ-ಗಳಾಗಿದ್ದುವು
ವರುಷ-ಗಳು
ವರುಷ-ಗಳೆ
ವರುಷದ
ವರುಷವೂ
ವರುಷ-ವೆನಿ-ತನೊ
ವರುಷ-ವೆನಿತೋ
ವರ್ಗಕ್ಕೆ
ವರ್ಗ-ಗಳನ್ನು
ವರ್ಗದ
ವರ್ಗ-ದವರ
ವರ್ಗ-ದವರಿಗೆ
ವರ್ಗ-ದವರು
ವರ್ಗ-ದವರೇ
ವರ್ಗಾಯಿಸಿ-ದರು
ವರ್ಗಾಯಿಸಿದೆ
ವರ್ಗಾವಣೆ-ಯಾದ
ವರ್ಚಸ್ಸುಳ್ಳ-ವರೂ
ವರ್ಜಿಸಿದ
ವರ್ಜಿಸು-ವುದು
ವರ್ಣ
ವರ್ಣ-ಖೇಲಾ
ವರ್ಣ-ಗಳ
ವರ್ಣ-ಗಳನ್ನು
ವರ್ಣ-ಗಳು
ವರ್ಣ-ಚಿತ್ರಕಾರ-ರಾಗಲು
ವರ್ಣ-ಚಿತ್ರ-ಗಳು
ವರ್ಣ-ದವರ
ವರ್ಣ-ದವರೂ
ವರ್ಣ-ದವರೇ
ವರ್ಣನೆ
ವರ್ಣನೆ-ಮಾಡಲು
ವರ್ಣನೆ-ಮಾಡಿ
ವರ್ಣನೆ-ಯಿದೆ
ವರ್ಣ-ವಿಭಜನೆ
ವರ್ಣವೇ
ವರ್ಣಾಶ್ರಮ
ವರ್ಣಾಶ್ರಮ-ಧರ್ಮ-ವನ್ನು
ವರ್ಣಾಶ್ರಮಾಚಾರ-ಗಳನ್ನು
ವರ್ಣಿಸ-ತೊಡಗಿ-ದರು
ವರ್ಣಿಸ-ಲಾಗಿದೆ
ವರ್ಣಿಸಿ-ದ್ದಾರೆ
ವರ್ಣಿಸು-ತ್ತಿದ್ದರು
ವರ್ತಕರ
ವರ್ತಕರು
ವರ್ತನೆ
ವರ್ತನೆಯ
ವರ್ತನೆ-ಯಲ್ಲ
ವರ್ತಮಾನ
ವರ್ತಮಾನ-ಕಾಲದ
ವರ್ತಮಾನ-ಕಾಲ-ದಲ್ಲಿ
ವರ್ತಮಾನಕೆ
ವರ್ತಮಾನ-ಗಳನ್ನೂ
ವರ್ತಮಾನ-ದಿಂದಲೂ
ವರ್ತಮಾನ-ವನ್ನು
ವರ್ತಮಾನವೂ
ವರ್ತಿಸು-ತ್ತಿದ್ದುವು
ವರ್ಷ
ವರ್ಷಕಾಲ
ವರ್ಷಕಾಲ-ದಲ್ಲಿ
ವರ್ಷಕ್ಕೊಂದು
ವರ್ಷ-ಗಳ
ವರ್ಷ-ಗಳಲ್ಲಿ
ವರ್ಷ-ಗಳ-ವರೆಗೂ
ವರ್ಷ-ಗಳ-ವರೆಗೆ
ವರ್ಷ-ಗಳಾದ
ವರ್ಷ-ಗಳಿಂದಲೂ
ವರ್ಷ-ಗಳಿಂದಷ್ಟೇ
ವರ್ಷ-ಗಳಿಂದೀಚೆಗೆ
ವರ್ಷ-ಗಳಿ-ಗಿಂತಲೂ
ವರ್ಷ-ಗಳಿಗೆ
ವರ್ಷ-ಗಳು
ವರ್ಷ-ಗಳೇನೋ
ವರ್ಷ-ತಾಳಿ
ವರ್ಷದ
ವರ್ಷ-ದಲ್ಲಿ
ವರ್ಷ-ದಲ್ಲೇ
ವರ್ಷ-ದಿಂದ
ವರ್ಷ-ವರ್ಷಕ್ಕೂ
ವರ್ಷ-ವರ್ಷವೂ
ವರ್ಷ-ವಾಗಿರಲಿ
ವರ್ಷವೂ
ವಲಯ-ವನ್ನೂ
ವಲ್ಲಿ
ವಶಕ್ಕೆ
ವಶದಲ್ಲಿಟ್ಟು-ಕೊಂಡಿದ್ದ
ವಶಪಡಿಸಿ-ಕೊಂಡ-ರೆಂದರೆ
ವಶರಾಗಿರು-ವುದಿಲ್ಲ
ವಶರಾಗು-ವರು
ವಶರಾಗು-ವೆವು
ವಶೀಭೂತ-ನಲ್ಲ
ವಸಂತ
ವಸಂತಕ್ಕೆ
ವಸಂತ-ದಲ್ಲಿ
ವಸತಿ
ವಸಾಹತು-ಗಳನ್ನು
ವಸಿಷ್ಠ
ವಸಿಷ್ಠ-ನಲ್ಲಿ
ವಸುಂಧರಾ
ವಸುಂಧರೆ-ಯನ್ನು
ವಸೇ
ವಸ್ತು
ವಸ್ತು-ಗಳ
ವಸ್ತು-ಗಳನ್ನು
ವಸ್ತು-ಗಳನ್ನೂ
ವಸ್ತು-ಗಳಲ್ಲಿಯೂ
ವಸ್ತು-ಗಳಾಗಿದ್ದಾರೆ
ವಸ್ತು-ಗಳಿಂದ
ವಸ್ತು-ಗಳಿಗೆ
ವಸ್ತು-ಗಳಿಲ್ಲವೇ
ವಸ್ತು-ಗಳು
ವಸ್ತು-ಗಳುಳ್ಳ
ವಸ್ತು-ಗಳೂ
ವಸ್ತು-ಗಳೆಂದು
ವಸ್ತು-ಗಳೆಲ್ಲ
ವಸ್ತುತಃ
ವಸ್ತು-ಪೂಜೆ
ವಸ್ತು-ಪೂಜೆ-ಯನ್ನು
ವಸ್ತು-ವನ್ನು
ವಸ್ತು-ವನ್ನೂ
ವಸ್ತು-ವಾಗಿ
ವಸ್ತು-ವಾಗುವೆ
ವಸ್ತು-ವಾದ
ವಸ್ತು-ವಿನ
ವಸ್ತು-ವಿನಲ್ಲಿ
ವಸ್ತು-ವಿನಲ್ಲಿ-ರುವ
ವಸ್ತು-ವಿನಿಂದ
ವಸ್ತು-ವಿವೇಕ
ವಸ್ತುವು
ವಸ್ತುವೂ
ವಸ್ತು-ವೆಂದು
ವಸ್ತು-ವೆಂದೆಣಿಸ-ಬೇಡ
ವಸ್ತುವೇ
ವಸ್ತು-ಸಂಗ್ರಹ-ಶಾಲೆ-ಯಲ್ಲಿ
ವಸ್ತುಸ್ಥಿತಿ
ವಸ್ತ್ರ
ವಸ್ತ್ರ-ಗಳ
ವಸ್ತ್ರ-ಗಳನ್ನು
ವಸ್ತ್ರ-ಗಳನ್ನೊದಗಿಸು-ವುದು
ವಸ್ತ್ರ-ಗಳಿಗೆ
ವಸ್ತ್ರಾದಿ-ಗಳನ್ನೆಲ್ಲ
ವಹತಿ
ವಹ-ಮಾತ್ರ
ವಹಿಸ-ಬೇಕು
ವಹಿಸಿ
ವಹಿಸಿ-ಕೊಂಡಿದ್ದಾರೆ
ವಹಿಸಿ-ಕೊಂಡುದು
ವಹಿಸಿ-ಕೊಡ-ಬೇಕೆಂದು
ವಹಿಸಿ-ಕೊಳ್ಳ-ಬೇಕು
ವಹಿಸಿ-ಕೊಳ್ಳು-ತ್ತಿದ್ದನು
ವಹಿಸಿ-ಕೊಳ್ಳು-ತ್ತೇನೆ
ವಹಿಸಿದ್ದರು
ವಹಿಸು-ವರು
ವಹೇ
ವಾ
ವಾಕ್ಕನ್ನೂ
ವಾಕ್ಚಾತುರ್ಯ
ವಾಕ್ಯ
ವಾಕ್ಯ-ಗಳನ್ನು
ವಾಕ್ಯ-ಗಳಿಗೆ
ವಾಕ್ಯದ
ವಾಕ್ಯ-ಮನಾತೀತ
ವಾಕ್ಯ-ವನ್ನು
ವಾಕ್ಯ-ವಿನ್ಯಾಸ-ವನ್ನು
ವಾಕ್ಯಾರ್ಥ
ವಾಕ್-ಸಾಮರ್ಥ್ಯ
ವಾಗ-ಬೇಕೇನು
ವಾಗು-ವುವು
ವಾಗ್ಧಾರೆ-ಯಿಂದ
ವಾಗ್ಯುದ್ಧವೇ
ವಾಗ್ವಾದ
ವಾಗ್ವಾದ-ಗಳಿಗೆಲ್ಲಾ
ವಾಗ್ವೈಖರಿ-ಯನ್ನು
ವಾಗ್ವೈಖರಿ-ಯಿಂದ
ವಾಜ್ಯ
ವಾಜ್ಯ-ಭೂಮಿ
ವಾಡಿಕೆ
ವಾಡಿಕೆ-ಯಲ್ಲಿ-ರುವ
ವಾಣಿ
ವಾಣಿಜ್ಯ
ವಾಣಿ-ಯನ್ನಾಲಿ-ಸುತ
ವಾಣಿ-ಯನ್ನು
ವಾಣಿ-ಯಿಂದ
ವಾತಾವರಣ
ವಾತಾವರಣಕ್ಕಿಂತ
ವಾತಾವರಣಕ್ಕೆ
ವಾತಾವರಣ-ದಲ್ಲಿ
ವಾತಾವರಣ-ದೊಡನೆ
ವಾತ್ಸಲ್ಯ
ವಾತ್ಸಲ್ಯ-ದಿಂದ
ವಾದ
ವಾದದ್ದಾಗಿರ-ಲಿಲ್ಲ
ವಾದ-ಮಾಡು-ವಾಗ
ವಾದ-ವನ್ನು
ವಾದ-ವಾದ
ವಾದಾತುರಾಃ
ವಾದಿ-ಸಿದರೆ
ವಾದ್ಯ
ವಾದ್ಯ-ಗಳನ್ನು
ವಾದ್ಯ-ಗಾರ್ತಿಯೂ
ವಾದ್ಯಧ್ವನಿ-ಯಿಂದ
ವಾಪ್ಯಲಿಂಗಾತ್
ವಾಮನಂ
ವಾಮನ-ನನ್ನು
ವಾಮನಮಾಸೀನಂ
ವಾಮನ-ರೂಪಿ-ಯಾದ
ವಾಮಾಚಾರ
ವಾಮಾಚಾರದ
ವಾಮಾಚಾರ-ದಿಂದ
ವಾಮಾಚಾರ-ವನ್ನು
ವಾಮಾಚಾರವೂ
ವಾಮಿ
ವಾಯು
ವಾಯು-ವಿನ
ವಾಯುಶ್ಚ
ವಾರಕ್ಕೆ
ವಾರ-ಗಳ
ವಾರ-ಗಳ-ವರೆಗೂ
ವಾರದ
ವಾರ-ಪತ್ರಿಕೆ-ಯಾಗ-ಬೇಕೆಂದು
ವಾರಾಂಗ-ನೆಯ
ವಾರಿ
ವಾರಿಧಿ-ಗಾನ
ವಾರೇಂದ್ರ
ವಾರೇಂದ್ರರು
ವಾರ್ಷಿಕೋತ್ಸವ
ವಾಲುತ್ತದೆ
ವಾಲುವು-ದಿಲ್ಲ
ವಾಸ-ಗೃಹಕ್ಕೆ
ವಾಸನಾ
ವಾಸನಾದಿ-ಗಳು
ವಾಸನಾವೇಶ
ವಾಸನೆ
ವಾಸನೆ-ಗಳ
ವಾಸನೆ-ಯನ್ನು
ವಾಸನೆ-ಯಿರು-ವುದನ್ನು
ವಾಸ-ಮಾಡ-ಬೇಕೆಂದು
ವಾಸ-ಮಾಡಿ-ಕೊಂಡು
ವಾಸ-ಮಾಡಿರ-ಲಿಲ್ಲ
ವಾಸ-ಮಾಡುತ್ತಾ
ವಾಸ-ಮಾಡುತ್ತಿದ್ದ-ರೆಂಬು-ದನ್ನು
ವಾಸ-ಮಾಡು-ವಂತೆ
ವಾಸ-ಮಾಡು-ವುದಕ್ಕೆ
ವಾಸ-ವಾಗಿದ್ದ
ವಾಸ-ವಾಗಿದ್ದಾಳೆ
ವಾಸ-ವಾಗಿದ್ದು
ವಾಸಸ್ಥಳಕ್ಕೆ
ವಾಸಿ
ವಾಸಿ-ಗಳಿಗೆ
ವಾಸಿ-ಮಾಡಿ-ರುವ
ವಾಸಿ-ಯಾಗು-ವುದನ್ನು
ವಾಸಿಸಿ-ರುವೆ
ವಾಸಿಸು-ತ್ತಿದ್ದರು
ವಾಸಿಸು-ತ್ತಿದ್ದಾಗ
ವಾಸಿಸು-ತ್ತಿದ್ದೆ
ವಾಸಿಸುವ
ವಾಸಿಸು-ವರು
ವಾಸಿಸು-ವುದು
ವಾಸ್ತವಕ್ಕಿಳಿಯೋಣ
ವಾಸ್ತವ-ದಲ್ಲಿ
ವಾಸ್ತವ-ವಾಗಿ
ವಾಸ್ತವ-ವಾಗಿರ-ಬೇಕು
ವಾಸ್ತವಾಂಶ-ವೇನೆಂದರೆ
ವಾಸ್ತವಿಕ
ವಾಸ್ತವಿಕತೆ-ಯನ್ನು
ವಾಸ್ತವಿಕತೆ-ಯನ್ನೂ
ವಾಸ್ತವಿಕತೆ-ಯಲ್ಲಿ
ವಾಸ್ತವಿಕ-ವಾಗಿ
ವಾಸ್ತು-ಕಲೆ-ಯನ್ನು
ವಾಸ್ತು-ಶಾಸ್ತ್ರದ
ವಾಸ್ತು-ಶಿಲ್ಪ
ವಾಹನ-ದಂತೆ
ವಾಹವೋತ್ಥಂ
ವಿಂಗಡಿಸ-ಬೇಕು
ವಿಂಗಡಿಸಿ-ಕೊಂಡಿದೆ
ವಿಂಗಡಿಸಿ-ಕೊಳ್ಳ-ಲಾಗಿದೆ
ವಿಂಗಡಿಸು-ತ್ತಾರೆ
ವಿಂದತಿ
ವಿಕಟ
ವಿಕಟಾಕಾರ-ವುಳ್ಳ
ವಿಕಟಾಟ್ಟ-ಹಾಸ-ದಲಿ
ವಿಕಲಿತ
ವಿಕಲ್ಪ
ವಿಕಸನದ
ವಿಕಸಿತ-ವಾದಾಗ
ವಿಕಸಿ-ಸಲಿ
ವಿಕಸಿ-ಸಿದೆ
ವಿಕಸಿಸಿವೆ
ವಿಕಸಿಸು-ವುವು
ವಿಕಸಿಸುವೆ
ವಿಕಾರ
ವಿಕಾರ-ಪಡಿಸಿ
ವಿಕಾರ-ವುಳ್ಳದ್ದಾಗಿಯೂ
ವಿಕಾಶ
ವಿಕಾಶೇ
ವಿಕಾಸ
ವಿಕಾಸ-ಕ್ಕಾಗಿಯೂ
ವಿಕಾಸಕ್ಕೂ
ವಿಕಾಸಕ್ಕೆ
ವಿಕಾಸ-ಗೊಂಡು
ವಿಕಾಸ-ಗೊಳಿಸಿ-ಕೊಳ್ಳುವ
ವಿಕಾಸ-ಗೊಳಿಸು-ವುದಕ್ಕೆ
ವಿಕಾಸ-ಗೊಳ್ಳಲು
ವಿಕಾಸ-ಗೊಳ್ಳುವುದು
ವಿಕಾಸ-ಗೊಳ್ಳು-ವುದೆಂದು
ವಿಕಾಸ-ಗೊಳ್ಳು-ವುದೆಂಬುದು
ವಿಕಾಸ-ಗೊಳ್ಳುವುದೊ
ವಿಕಾಸದ
ವಿಕಾಸ-ದಲ್ಲಿ
ವಿಕಾಸ-ದಿಂದ
ವಿಕಾಸ-ವನ್ನು
ವಿಕಾಸ-ವಾಗ-ತೊಡಗಿತು
ವಿಕಾಸ-ವಾಗಿ
ವಿಕಾಸ-ವಾಗಿದೆ
ವಿಕಾಸ-ವಾಗಿದ್ದರೆ
ವಿಕಾಸ-ವಾಗು-ವನು
ವಿಕಾಸ-ವಾಗು-ವುದು
ವಿಕಾಸ-ವಾದದ
ವಿಕಾಸ-ವಾದಷ್ಟು
ವಿಕಾಸ-ವಾಯಿತೆಂಬೀ
ವಿಕಾಸ-ವಿದೆಯೊ
ವಿಕಾಸ-ವಿರುತ್ತದೆ
ವಿಕಾಸ-ವಿಲ್ಲ
ವಿಕಾಸವು
ವಿಕಾಸ-ವುಂಟಾಗು-ವುದು
ವಿಕಾಸ-ವುಳ್ಳ
ವಿಕೃತ
ವಿಕೃತಂ
ವಿಕೃತ-ಗೊಳಿ-ಸುತ್ತವೆ
ವಿಕೃತ-ವಾಗಿದೆ
ವಿಕೃತಿವಾತೇ
ವಿಕ್ರಮ್
ವಿಕ್ಷಿಪ್ತ
ವಿಕ್ಷುಬ್ಧ-ವಾದ
ವಿಕ್ಷೇಪದ
ವಿಖ್ಯಾತ್
ವಿಗ್ರಹ
ವಿಗ್ರಹ-ಗಳ
ವಿಗ್ರಹ-ಗಳನ್ನು
ವಿಗ್ರಹ-ಗಳೂ
ವಿಗ್ರಹ-ಗಳೆಲ್ಲವೂ
ವಿಗ್ರಹದ
ವಿಗ್ರಹ-ದಂತೆ
ವಿಗ್ರಹ-ದಲ್ಲಿ
ವಿಗ್ರಹ-ವನ್ನು
ವಿಗ್ರಹವು
ವಿಗ್ರಹವೊ
ವಿಗ್ರಹಾರಾಧನೆಗೆ
ವಿಘ್ನಕಾರಿ
ವಿಘ್ನ-ಗಳನ್ನು
ವಿಚಲಿತ-ನಾಗ-ದವನು
ವಿಚಾರ
ವಿಚಾರಕರು
ವಿಚಾರ-ಕ್ಕಾಗಲೀ
ವಿಚಾರ-ಕ್ಕಿಂತ
ವಿಚಾರಕ್ಕೆ
ವಿಚಾರ-ಕ್ಕೋಸ್ಕರ
ವಿಚಾರ-ಗಳ
ವಿಚಾರ-ಗಳನ್ನು
ವಿಚಾರ-ಗಳನ್ನೂ
ವಿಚಾರ-ಗಳನ್ನೇ
ವಿಚಾರ-ಗಳಲ್ಲಿ
ವಿಚಾರ-ಗಳಲ್ಲಿಯೂ
ವಿಚಾರ-ಗಳಲ್ಲೆಲ್ಲಾ
ವಿಚಾರ-ಗಳಿಗೆ
ವಿಚಾರ-ಗಳು
ವಿಚಾರ-ಗಳೆಲ್ಲ
ವಿಚಾರ-ಗಳೆಲ್ಲಾ
ವಿಚಾರ-ಗಳೊಂದಿಗೆ
ವಿಚಾರಜ್ಞಾನ-ವನ್ನೂ
ವಿಚಾರಣಾ
ವಿಚಾರದ
ವಿಚಾರ-ದಲ್ಲಿ
ವಿಚಾರ-ದಲ್ಲಿಯೂ
ವಿಚಾರ-ದಿಂದ
ವಿಚಾರ-ಪೂರ್ವಕ-ವಾಗಿ
ವಿಚಾರ-ಪೂರ್ವಕ-ವಾದ
ವಿಚಾರ-ಮಾಡ-ಬೇಕೆಂದು
ವಿಚಾರ-ಮಾಡ-ಲಾಗಿ
ವಿಚಾರ-ಮಾಡಿ
ವಿಚಾರ-ಮಾಡಿದ್ದು
ವಿಚಾರ-ಮಾಡು
ವಿಚಾರ-ಮಾಡುತ್ತ
ವಿಚಾರ-ಮಾಡುತ್ತಿರ-ಬೇಕು
ವಿಚಾರ-ವನ್ನಂತೂ
ವಿಚಾರ-ವನ್ನು
ವಿಚಾರ-ವನ್ನೂ
ವಿಚಾರ-ವನ್ನೆ
ವಿಚಾರ-ವನ್ನೇ
ವಿಚಾರ-ವನ್ನೇನೊ
ವಿಚಾರ-ವಾಗಿ
ವಿಚಾರ-ವಾಗಿಯೂ
ವಿಚಾರ-ವೆಲ್ಲ
ವಿಚಾರವೇ
ವಿಚಾರ-ಶೂನ್ಯ-ರಾದ
ವಿಚಾರಿಸ-ಬೇಕಾಗಿಲ್ಲ
ವಿಚಾರಿಸ-ಲಾಗಲಿಲ್ಲ
ವಿಚಾರಿಸಲು
ವಿಚಾರಿಸಿ
ವಿಚಾರಿಸಿ-ಕೊಂಡರು
ವಿಚಾರಿಸಿದ
ವಿಚಾರಿಸಿದ-ಮೇಲೆ
ವಿಚಾರಿಸಿ-ದರು
ವಿಚಾರಿಸಿ-ದಾಗ
ವಿಚಾರಿಸಿ-ದುದ-ರಲ್ಲಿ
ವಿಚಾರಿಸು-ತ್ತದೆ
ವಿಚಾರಿಸು-ತ್ತಿದ್ದರು
ವಿಚಾರಿಸು-ವುದ-ರಲ್ಲಿ
ವಿಚಿತ್ರ
ವಿಚಿತ್ರ-ವಾಗಿದೆ-ಯೆಂದರೆ
ವಿಚಿತ್ರ-ವಾದ
ವಿಚ್ಛಿನ್ನ-ವಾಗಿತ್ತು
ವಿಜಯ
ವಿಜಯಧ್ವಜ-ವದೊ
ವಿಜಯ-ವಾಗು-ವುದಕ್ಕಾಗಿಯೆ
ವಿಜಾನಾಸ್ಯಸ್ಮಯಾನ್
ವಿಜಾನೀಯಾತ್
ವಿಜೃಂಭಣೆ-ಯಿಂದ
ವಿಜ್ಞಾತಾರ-ಮರೇ
ವಿಜ್ಞಾತಾರ-ವರೇ
ವಿಜ್ಞಾನ
ವಿಜ್ಞಾನ-ಗಳ
ವಿಜ್ಞಾನ-ಗಳನ್ನು
ವಿಜ್ಞಾನದ
ವಿಜ್ಞಾನ-ವನ್ನೂ
ವಿಜ್ಞಾನವೂ
ವಿಜ್ಞಾನ-ವೇದಾಂತ-ಷಟ್-ಶಾಸ್ತ್ರವೊ
ವಿಜ್ಞಾನ-ಶಾಸ್ತ್ರದ
ವಿಜ್ಞಾಪನ
ವಿಜ್ಞಾಪಿಸಿ-ಕೊಂಡರು
ವಿಡಂಬನ
ವಿಡಂಬನೆ
ವಿತರಿಛ
ವಿತರಿಛೆ
ವಿತೃಷ್ಣೆ-ಗಳೆ
ವಿದಾಯ
ವಿದಾಯ್
ವಿದುಃ
ವಿದುರನ
ವಿದೇಶಕ್ಕೆ
ವಿದೇಶದ
ವಿದೇಶಿ-ಗಳು
ವಿದೇಶೀ
ವಿದೇಶೀಯ
ವಿದೇಶೀಯರ
ವಿದೇಶೀಯ-ರಾದ
ವಿದೇಹ
ವಿದೇಹ-ಬುದ್ಧಿ
ವಿದ್ಧಿ
ವಿದ್ಯತೇ
ವಿದ್ಯತೇಯನಾಯ
ವಿದ್ಯತೇಯನಾಯ
ವಿದ್ಯಮಾನ
ವಿದ್ಯಾ
ವಿದ್ಯಾದಾನ
ವಿದ್ಯಾದಾನಕ್ಕೆ
ವಿದ್ಯಾದಾನದ
ವಿದ್ಯಾಬುದ್ಧಿ
ವಿದ್ಯಾಬುದ್ಧಿ-ಗಳಿಂದಲೂ
ವಿದ್ಯಾಭ್ಯಾಸ
ವಿದ್ಯಾಭ್ಯಾಸದ
ವಿದ್ಯಾಭ್ಯಾಸ-ದಲ್ಲಿ
ವಿದ್ಯಾಭ್ಯಾಸ-ದಿಂದ
ವಿದ್ಯಾಭ್ಯಾಸ-ವನ್ನು
ವಿದ್ಯಾಭ್ಯಾಸ-ವನ್ನೂ
ವಿದ್ಯಾಭ್ಯಾಸವೆ
ವಿದ್ಯಾಭ್ಯಾಸ-ವೆಂದರೆ
ವಿದ್ಯಾಭ್ಯಾಸ-ವೆಲ್ಲ
ವಿದ್ಯಾ-ಮಂದಿರವು
ವಿದ್ಯಾರ್ಜನೆ
ವಿದ್ಯಾರ್ಜನೆಗೆ
ವಿದ್ಯಾರ್ಥಿ-ಗಳನ್ನು
ವಿದ್ಯಾರ್ಥಿ-ಗಳು
ವಿದ್ಯಾರ್ಥಿದೆಶೆ-ಯಲ್ಲಿ
ವಿದ್ಯಾರ್ಥಿನಿ
ವಿದ್ಯಾರ್ಥಿನಿಯರ
ವಿದ್ಯಾರ್ಥಿನಿಯ-ರಂತೆ
ವಿದ್ಯಾರ್ಥಿನಿಯ-ರನ್ನು
ವಿದ್ಯಾರ್ಥಿನಿಯ-ರಿಗೆ
ವಿದ್ಯಾರ್ಥಿನಿಯರು
ವಿದ್ಯಾಲಯ
ವಿದ್ಯಾಲಯ-ದಲ್ಲಿ
ವಿದ್ಯಾವಂತ
ವಿದ್ಯಾವಂತ-ನಾದ
ವಿದ್ಯಾವಂತ-ರನ್ನಾಗಿ
ವಿದ್ಯಾವಂತ-ರಲ್ಲ
ವಿದ್ಯಾವಂತ-ರಲ್ಲಿ
ವಿದ್ಯಾವಂತ-ರಾಗ-ಬೇಕು
ವಿದ್ಯಾವಂತ-ರಾಗಿ
ವಿದ್ಯಾವಂತ-ರಾದ
ವಿದ್ಯಾವಂತರು
ವಿದ್ಯಾವಂತ-ರೆಂದು
ವಿದ್ಯಾವಂತ-ರೆಲ್ಲಾ
ವಿದ್ಯಾವಂತೆ-ಯಾದ
ವಿದ್ಯಾವತಿ-ಯರಾದ
ವಿದ್ಯಾವತಿ-ಯಾದವ-ರೆಲ್ಲ
ವಿದ್ಯಾಶಕ್ತಿ-ಯಿಂದ
ವಿದ್ಯಾಶ್ರಮ-ದಲ್ಲಿ
ವಿದ್ಯಾ-ಸಾಗರರ
ವಿದ್ಯಾ-ಹೇತು
ವಿದ್ಯುತ್
ವಿದ್ಯುತ್ವಂತಂ
ವಿದ್ಯುತ್
ವಿದ್ಯೆ
ವಿದ್ಯೆಗೆ
ವಿದ್ಯೆಯ
ವಿದ್ಯೆ-ಯನು
ವಿದ್ಯೆ-ಯನ್ನು
ವಿದ್ಯೆ-ಯನ್ನೂ
ವಿದ್ಯೆ-ಯನ್ನೆಲ್ಲಾ
ವಿದ್ಯೆ-ಯಲ್ಲಿ
ವಿದ್ಯೆ-ಯಿಂದ
ವಿದ್ಯೆ-ಯಿಂದಲೇ
ವಿದ್ಯೆ-ಯಿನ್ನಾವು-ದದು
ವಿದ್ಯೆಯು
ವಿದ್ಯೆ-ಯೇನು
ವಿದ್ರಾವಕ
ವಿದ್ವಂಸ್ತವನಾಸ್ತ್ಯಪಾಯಃ
ವಿದ್ವತ್
ವಿದ್ವಾಂಸನೇ
ವಿದ್ವಾಂಸರೆಂದು
ವಿಧ-ಗಳಲ್ಲಿ
ವಿಧ-ಗಳಲ್ಲಿಯೂ
ವಿಧ-ಗಳಿವೆ
ವಿಧದ
ವಿಧ-ದಲ್ಲಿ
ವಿಧದಲ್ಲಿಯೂ
ವಿಧ-ದಿಂದ
ವಿಧರ್ಮಿ-ಗಳನ್ನು
ವಿಧವಾ
ವಿಧ-ವಾಗಿ
ವಿಧವಾಗಿದ್ದು-ದನ್ನು
ವಿಧವಾಗಿ-ಬಿಟ್ಟಿದೆ
ವಿಧವಾಗಿಯೂ
ವಿಧ-ವಾದ
ವಿಧವಿಧ
ವಿಧವಿಧದ
ವಿಧ-ವಿಧ-ವಾಗಿ
ವಿಧವೂ
ವಿಧವೆ
ವಿಧವೆ-ಯರ
ವಿಧವೆ-ಯರಿಗೆ
ವಿಧವೆ-ಯರು
ವಿಧಾತುಮಿಹ
ವಿಧಾನ
ವಿಧಾನ-ಗಳಲ್ಲೊಂದು
ವಿಧಾನ-ಗಳು
ವಿಧಾನ-ಗಳೆಲ್ಲ-ವನ್ನೂ
ವಿಧಾನ-ದಲ್ಲೇ
ವಿಧಾನ-ದಿಂದ
ವಿಧಾನ-ವನ್ನು
ವಿಧಿ
ವಿಧಿ-ಗಳನ್ನು
ವಿಧಿ-ಗಳನ್ನೂ
ವಿಧಿ-ಗಳು
ವಿಧಿ-ನಿಬಂಧನೆ-ಗಳನ್ನು
ವಿಧಿ-ನಿಬಂಧನೆ-ಗಳಾಚೆಯೇ
ವಿಧಿ-ನಿಬಂಧನೆ-ಗಳಿಂದ
ವಿಧಿ-ನಿಬಂಧನೆ-ಗಳಿಗೆ
ವಿಧಿ-ನಿಯಮ
ವಿಧಿ-ನಿಯಮ-ಗಳಿಂದಾಚೆಗೆ
ವಿಧಿ-ನಿಯಮ-ಗಳಿತ್ತು
ವಿಧಿ-ನಿಯಮ-ಗಳು
ವಿಧಿ-ನಿಯಮ-ಗಳೆಲ್ಲ
ವಿಧಿ-ನಿಯಮ-ಗಳೆಲ್ಲ-ವನ್ನೂ
ವಿಧಿ-ನಿಷೇಧ
ವಿಧಿ-ನಿಷೇಧ-ಗಳಿವೆ
ವಿಧಿ-ಪದ್ಧತಿ-ಗಳನ್ನು
ವಿಧಿ-ಪೂರ್ವಕ
ವಿಧಿಯ
ವಿಧಿ-ಯಟ್ಟಿ-ದರೂ
ವಿಧಿ-ಯಿದೆ
ವಿಧಿ-ಯಿಲ್ಲ
ವಿಧಿ-ಯಿಲ್ಲದೆ
ವಿಧಿ-ಯುಕ್ತ
ವಿಧಿಯೆ
ವಿಧಿಯೇ
ವಿಧಿ-ಯೊಲುಮೆ
ವಿಧಿ-ವತ್ತಾದ
ವಿಧಿ-ವಿಧಾನ-ಗಳೆಲ್ಲ-ವನ್ನೂ
ವಿಧಿ-ವಿ-ಹಿತ-ವಾದ
ವಿಧಿ-ಸಲ್ಪಟ್ಟಿದೆ
ವಿಧಿ-ಸಲ್ಪಟ್ಟಿವೆ
ವಿಧಿ-ಸಿರುವ
ವಿಧಿಸುವು-ದಿಲ್ಲವಷ್ಟೆ
ವಿಧೃತ-ವಾಗಿರು-ವುದು
ವಿಧೇಯತೆ
ವಿಧೇಯ-ರಾಗಿ
ವಿಧೇಯ-ರಾಗಿ-ರುವ
ವಿಧ್ಯುಕ್ತ-ವಾದ
ವಿನಂತಿ
ವಿನಂತಿ-ಯಂತೆ
ವಿನಃ
ವಿನಯ
ವಿನಯತ್ಯತಿ-ದುಃಖ-ಮಾರ್ಗೈಃ
ವಿನಾ
ವಿನಾಯಿತಿ-ಯುಂಟು
ವಿನಾಶ-ಕಾರಿ
ವಿನಿ-ಯೋಗ
ವಿನಿ-ಯೋಗಿಸಿ-ದರು
ವಿನೀತ-ಭಾವ-ದಿಂದ
ವಿನೋದ-ಗಳಲ್ಲಿ
ವಿನೋದ-ವಾಗಿ
ವಿನ್ನಾವು-ದಿದೆ
ವಿನ್ಯಾಸ-ಗಳ
ವಿನ್ಯಾಸ-ವೆಲ್ಲ-ವನ್ನೂ
ವಿಪತ್ತಾಗಿತ್ತು
ವಿಪತ್ತಿ-ನಲ್ಲಿಯೂ
ವಿಪತ್ತು-ಗಳ
ವಿಪರೀತಕ್ಕೊಯ್ದದ್ದ-ರಿಂದ
ವಿಪರೀತ-ವಾಗಿ
ವಿಪಿನ
ವಿಪಿನದಲಿ
ವಿಪುಲ-ವಾಗಿದೆ
ವಿಪುಲ-ವಾತಃ
ವಿಪುಲ-ವಾದ
ವಿಪ್ರ-ರೆಂದರೆ
ವಿಪ್ಲವ
ವಿಪ್ಲವ-ದೊಂದಿಗೆ
ವಿಪ್ಲವ-ವನ್ನು
ವಿಫಲತೆ-ಯಲಿ
ವಿಫಲ-ಯತ್ನದಿ
ವಿಫಲ-ವಾಗಿ
ವಿಭಗ್ನಮ್
ವಿಭಗ್ನಾಂ
ವಿಭಜನಾತ್ಮಕ-ವಾದುದು
ವಿಭಜನೆ
ವಿಭಜನೆ-ಗೊಂಡಿತ್ತು
ವಿಭಾಗ
ವಿಭಾಗ-ಗಳನ್ನು
ವಿಭಾಗ-ಗಳಿವೆ
ವಿಭಾಗದ
ವಿಭಾಗ-ವಿಲ್ಲ
ವಿಭಾತಿ
ವಿಭಾಸಕ-ವಾದ
ವಿಭಿನ್ನ
ವಿಭಿನ್ನ-ವಾದ
ವಿಭೀಷಣ
ವಿಭುತ್ವಮ್
ವಿಭುವೆ
ವಿಭೂತಯ
ವಿಭೂತಿ
ವಿಭೂತಿ-ಗಳನ್ನು
ವಿಭೂತಿ-ಗಳು
ವಿಭೂತಿ-ಭೂಷಿ-ತಾಂಗ-ರಾದ
ವಿಭೂತಿ-ಯನ್ನು
ವಿಭ್ರಮ
ವಿಭ್ರಮೆ-ಗಳವು
ವಿಭ್ರಾಂತ-ನಾಗಿ-ಹನು
ವಿಮಗ್ನ-ನಾಗಿದ್ದಾನೆ
ವಿಮರ್ಶಕರ
ವಿಮರ್ಶಾಜ್ಞಾನ-ವುಳ್ಳವ-ನಾಗಿ-ರು-ವನೋ
ವಿಮರ್ಶಿಸದೆ
ವಿಮರ್ಶಿಸಿ
ವಿಮರ್ಶಿಸಿ-ದರೆ
ವಿಮರ್ಶೆ
ವಿಮರ್ಶೆ-ಯನ್ನು
ವಿಮಲ
ವಿಮಲವೂ
ವಿಮುಕ್ತ-ರಾಗಿ
ವಿಮುಖ-ನಾಗು-ವಂತೆ
ವಿಮುಖ-ರಾದೆವು
ವಿಮುಖ-ವಾಗ-ದಿರಲಿ
ವಿಮೋಚನೆ
ವಿಮೋಚನೆಗೆ
ವಿರಕ್ತನೆ
ವಿರಕ್ತ-ರಾಗು-ತ್ತಾರೆಯೋ
ವಿರಕ್ತಿ
ವಿರಕ್ತಿ-ಯಿಲ್ಲದೆ
ವಿರಜಾನಂದ
ವಿರಜೇತ್
ವಿರಮಿಸಲಾರೆ
ವಿರಮಿಸಿಹು-ದಲ್ಲಿ
ವಿರಳ
ವಿರಾಜಿ-ಸುತ್ತದೆ
ವಿರಾಟ್
ವಿರಾಮ
ವಿರಾಮ-ವಾಗಿರು-ವಾಗ
ವಿರಾಮ-ವಿಲ್ಲ
ವಿರಾಮ-ವಿಲ್ಲದ
ವಿರಾಮ-ವಿಲ್ಲದೆ
ವಿರುದ್ದ
ವಿರುದ್ದ-ವೆಂದೂ
ವಿರುದ್ಧ
ವಿರುದ್ಧ-ವಾಗಿ
ವಿರುದ್ಧ-ವಾಗಿದ್ದಷ್ಟೂ
ವಿರುದ್ಧ-ವಾಗಿರು-ವುದು
ವಿರುದ್ಧ-ವಾದು-ದಲ್ಲ
ವಿರುವದೊ
ವಿರೂಪ
ವಿರೋಚನನ
ವಿರೋಧ
ವಿರೋಧಕ್ಕೂ
ವಿರೋಧ-ವಾಗಿ
ವಿರೋಧ-ವಾಗಿದೆ
ವಿರೋಧ-ವಾಗಿರ-ಲಿಲ್ಲ
ವಿರೋಧ-ವಾಗಿರು-ವುದೋ
ವಿರೋಧ-ವಾದ
ವಿರೋಧ-ವಿತ್ತು
ವಿರೋಧವೇ
ವಿರೋಧಾತ್ಮಕ-ಗಳೆಂದು
ವಿರೋಧಾಭಾಸದ
ವಿರೋಧಾಭಾಸ-ವಾಗಿದೆ
ವಿರೋಧಾಭಾಸ-ವಿದೆ
ವಿರೋಧಿ-ಗಳಾ-ಗಲಿಲ್ಲವೆ
ವಿರೋಧಿ-ಗಳಾದ-ವರನ್ನು
ವಿರೋಧಿ-ಯಾದ
ವಿರೋಧಿಸ-ಬಹುದು
ವಿರೋಧಿಸ-ಬೇಕಾಗಿದೆ
ವಿರೋಧಿಸ-ಬೇಕು
ವಿರೋಧಿಸ-ಬೇಕೆಂದು
ವಿರೋಧಿಸುವ
ವಿಲಕ್ಷಣ
ವಿಲಕ್ಷಣತೆ-ಗಳಲ್ಲಿ
ವಿಲಕ್ಷಣ-ವಾದ
ವಿಲವಿಲನೆ
ವಿಲಾಯತಿಗೆ
ವಿಲಾಯತಿಯ
ವಿಲಾಯಿತಿಯ
ವಿಲಾಯಿತಿ-ಯಿಂದ
ವಿಲಾಸ-ಗಳು
ವಿಲೀನ-ವಾಗು-ವುವೋ
ವಿಲ್ಲ-ದಿದ್ದರೆ
ವಿಳಂಬ
ವಿವರ-ಗಳಿಗೆ
ವಿವರ-ಗಳೆಲ್ಲಾ
ವಿವರ-ಗಳೊಂದಿಗೆ
ವಿವರಣೆ
ವಿವರಣೆ-ಗಳ
ವಿವರಣೆ-ಯಾಗಿ
ವಿವರವಾಗಿ
ವಿವರಿಸ-ತೊಡಗಿದ್ದಾನೆ
ವಿವರಿಸ-ಬಹುದು
ವಿವರಿಸ-ಬೇಕು
ವಿವರಿಸ-ಲಾಗದ
ವಿವರಿಸ-ಲಾಗಿದೆ
ವಿವರಿಸ-ಲಾಯಿತು
ವಿವರಿಸ-ಲಾರೆ
ವಿವರಿಸ-ಲಿಲ್ಲ
ವಿವರಿಸಲು
ವಿವರಿಸಲೂ
ವಿವರಿಸಲ್ಪಟ್ಟುವೊ
ವಿವರಿಸಿ
ವಿವರಿಸಿ-ದರು
ವಿವರಿಸಿ-ದರೆ
ವಿವರಿಸಿದೆ
ವಿವರಿಸಿ-ದ್ದನು
ವಿವರಿಸಿ-ದ್ದಾರೆ
ವಿವರಿಸು
ವಿವರಿಸುತ್ತಾನೆ
ವಿವರಿಸು-ತ್ತಿರು-ವರು
ವಿವರಿಸುವ
ವಿವರಿಸು-ವಂತಹ
ವಿವರಿಸು-ವನು
ವಿವರಿಸು-ವರು
ವಿವರಿಸು-ವುದಕ್ಕೆ
ವಿವರಿಸು-ವುದು
ವಿವರಿಸು-ವುವು
ವಿವಸ್ವನ
ವಿವಾದ-ಗಳಲ್ಲೇ
ವಿವಾಹ
ವಿವಾಹ-ಗಳು
ವಿವಾಹದ
ವಿವಾಹ-ವನ್ನು
ವಿವಾಹ-ವಾಗ
ವಿವಾಹ-ವಾಗದೆ
ವಿವಾಹ-ವಾಗಲು
ವಿವಾಹ-ವಾಗಲೇ
ವಿವಾಹ-ವಾಗಿ
ವಿವಾಹಾದಿ-ಗಳನ್ನು
ವಿವಿದಿಷಾ
ವಿವಿಧ
ವಿವೇಕ
ವಿವೇಕ-ಚೂಡಾಮಣಿ
ವಿವೇಕ-ಚೂಡಾಮಣಿಯ
ವಿವೇಕಜ್ಞಾನ-ವಿಲ್ಲ-ದಿದ್ದಲ್ಲಿ
ವಿವೇಕ-ದಲ್ಲಿ
ವಿವೇಕಾನಂದ
ವಿವೇಕಾನಂದನ
ವಿವೇಕಾನಂದ-ನಾಗ-ಬಹುದು
ವಿವೇಕಾನಂದರ
ವಿವೇಕಾನಂದ-ರಂಥ-ವರಿಗೆ
ವಿವೇಕಾನಂದ-ರಿಗೆ
ವಿವೇಕಾನಂದರು
ವಿವೇಕಾನಂದ-ರೆಂದು
ವಿವೇಚನಜ್ಞಾನದ
ವಿವೇಚನಾ
ವಿವೇಚನಾಜ್ಞಾನ
ವಿವೇಚನಾಜ್ಞಾನ-ವುಳ್ಳ-ವರಿಗೂ
ವಿವೇಚನಾ-ಶಕ್ತಿ-ಯಿಂದ
ವಿವೇಚನಾ-ಶೀಲ-ನಾದ
ವಿವೇಚನೆ
ವಿವೇಚನೆಯ
ವಿವೇಚನೆ-ಯಿಲ್ಲದೆ
ವಿಶಕುಂಬ
ವಿಶದಪಡಿಸಿ
ವಿಶದವಾಗಿ
ವಿಶಾಲ
ವಿಶಾಲ-ವಾಗಿ
ವಿಶಾಲ-ವಾಗಿದೆ
ವಿಶಾಲ-ವಾಗಿದ್ದು
ವಿಶಾಲ-ವಾದ
ವಿಶಾಲ-ಹೃದಯದ
ವಿಶಿಷ್ಟ-ವಾದ
ವಿಶಿಷ್ಟಾದ್ವೈತ-ತತ್ತ್ವ
ವಿಶಿಷ್ಟಾದ್ವೈತದ
ವಿಶಿಷ್ಟಾದ್ವೈತ-ವನ್ನು
ವಿಶಿಷ್ಟಾದ್ವೈತ-ವೆಂದರೆ
ವಿಶಿಷ್ಟಾದ್ವೈತಿ-ಗಳು
ವಿಶೇಷ
ವಿಶೇಷ-ಗಳನ್ನು
ವಿಶೇಷ-ಗಳು
ವಿಶೇಷಣ
ವಿಶೇಷಣ-ಗಳಿಂದ
ವಿಶೇಷಣದ
ವಿಶೇಷ-ದಲ್ಲಿ
ವಿಶೇಷ-ವಾಗಿ
ವಿಶೇಷ-ವಾಗಿದೆ
ವಿಶೇಷ-ವಾದ
ವಿಶೇಷ-ವಿತ್ತು
ವಿಶೇಷ-ವೇನೂ
ವಿಶ್ರಮಿಸಿ-ಕೊಂಡ
ವಿಶ್ರಮಿಸಿ-ಕೊಂಡರು
ವಿಶ್ರಮಿಸಿ-ಕೊಂಡು
ವಿಶ್ರಮಿಸಿ-ಕೊಳ್ಳು-ತ್ತಿದ್ದರು
ವಿಶ್ರಮಿಸಿ-ಕೊಳ್ಳು-ವುದಕ್ಕೆ
ವಿಶ್ರಾಂತಿ
ವಿಶ್ರಾಂತಿಗೆ
ವಿಶ್ರಾಂತಿ-ಯನ್ನು
ವಿಶ್ರಾಂತಿ-ಯಲ್ಲಿದ್ದರೆ
ವಿಶ್ರಾಂತಿ-ಯಾದ
ವಿಶ್ಲೇಷಿಸಿ
ವಿಶ್ವ
ವಿಶ್ವಂ
ವಿಶ್ವಕ್ಕೆಲ್ಲ
ವಿಶ್ವ-ಗಳ
ವಿಶ್ವ-ಗಳೆರಡೂ
ವಿಶ್ವದ
ವಿಶ್ವ-ದಲ್ಲಿ
ವಿಶ್ವ-ದಲ್ಲೆಲ್ಲಾ
ವಿಶ್ವ-ದಾದ್ಯಂತ
ವಿಶ್ವ-ಧರ್ಮ
ವಿಶ್ವ-ಧರ್ಮದ
ವಿಶ್ವಪ್ರೇಮ
ವಿಶ್ವ-ಭಾವನೆ-ಗಳ
ವಿಶ್ವಭ್ರಾತೃತ್ವದ
ವಿಶ್ವ-ಮಾನವ
ವಿಶ್ವ-ಮಾಯಾಧೀಶ-ನಾತನು
ವಿಶ್ವ-ರೂಪ-ದರ್ಶನ
ವಿಶ್ವ-ರೂಪ-ವನ್ನು
ವಿಶ್ವ-ವಂದ್ಯ-ನಾದ
ವಿಶ್ವ-ವನ್ನು
ವಿಶ್ವ-ವನ್ನೇ
ವಿಶ್ವ-ವರಿಯಲಿ
ವಿಶ್ವ-ವಿಜಯಿ-ಗಳಾದ
ವಿಶ್ವ-ವಿಜೇತ
ವಿಶ್ವ-ವಿದ್ಯಾನಿಲಯದ
ವಿಶ್ವ-ವಿದ್ಯಾನಿಲಯ-ದಲ್ಲಿ
ವಿಶ್ವ-ವಿಭು
ವಿಶ್ವವೂ
ವಿಶ್ವವೇ
ವಿಶ್ವ-ಸಾಕ್ಷಿ
ವಿಶ್ವ-ಸಾಗರ-ದಲ್ಲಿ
ವಿಶ್ವ-ಸ್ವರೂಪಳೂ
ವಿಶ್ವಾಕೃತಿ
ವಿಶ್ವಾತ್ಮ
ವಿಶ್ವಾತ್ಮನ
ವಿಶ್ವಾತ್ಮ-ನನ್ನು
ವಿಶ್ವಾಮಿತ್ರ
ವಿಶ್ವಾಮಿತ್ರನ
ವಿಶ್ವಾಮಿತ್ರ-ನನ್ನು
ವಿಶ್ವಾಸ
ವಿಶ್ವಾಸ-ಗಳನ್ನು
ವಿಶ್ವಾಸ-ಗಳಿವೆಯೋ
ವಿಶ್ವಾಸ-ಪೂರ್ವಕ-ವಾಗಿ
ವಿಶ್ವಾಸ-ವನ್ನಿಟ್ಟು
ವಿಶ್ವಾಸ-ವನ್ನೂ
ವಿಶ್ವಾಸ-ವಿಟ್ಟು
ವಿಶ್ವಾಸ-ವಿತ್ತು
ವಿಶ್ವಾಸ-ವಿಲ್ಲ
ವಿಶ್ವಾಸವೇ
ವಿಶ್ವೇ
ವಿಶ್ವೇದೇವಾ
ವಿಶ್ವೇಶ್ವರ-ನೆಂಬ
ವಿಷ
ವಿಷದ
ವಿಷಪಾನ
ವಿಷಯ
ವಿಷಯಕ-ವಾದ
ವಿಷಯಕ್ಕೂ
ವಿಷಯಕ್ಕೆ
ವಿಷಯ-ಗಳ
ವಿಷಯ-ಗಳನ್ನು
ವಿಷಯ-ಗಳನ್ನೂ
ವಿಷಯ-ಗಳನ್ನೆಲ್ಲಾ
ವಿಷಯ-ಗಳನ್ನೇ
ವಿಷಯ-ಗಳಲ್ಲವೇ
ವಿಷಯ-ಗಳಲ್ಲಿ
ವಿಷಯ-ಗಳಲ್ಲಿಯೂ
ವಿಷಯ-ಗಳಲ್ಲಿ-ರುವ
ವಿಷಯ-ಗಳಿಂದ
ವಿಷಯ-ಗಳಿಗೆ
ವಿಷಯ-ಗಳಿವೆ
ವಿಷಯ-ಗಳು
ವಿಷಯದ
ವಿಷಯ-ದಲ್ಲಂತೂ
ವಿಷಯ-ದಲ್ಲಿ
ವಿಷಯ-ದಲ್ಲಿಯೂ
ವಿಷಯ-ದಲ್ಲೂ
ವಿಷಯ-ವನ್ನು
ವಿಷಯ-ವನ್ನೂ
ವಿಷಯ-ವನ್ನೆತ್ತುತ್ತಾ
ವಿಷಯ-ವನ್ನೆಲ್ಲಾ
ವಿಷಯ-ವನ್ನೇ
ವಿಷಯ-ವನ್ನೇಕೆ
ವಿಷಯ-ವನ್ನೇನು
ವಿಷಯ-ವಲ್ಲ-ದಿದ್ದರೆ
ವಿಷಯ-ವಾಗಿ
ವಿಷಯ-ವಾಗಿದೆ
ವಿಷಯ-ವಾಗಿಯೂ
ವಿಷಯ-ವಾಗಿಯೆ
ವಿಷಯ-ವಿದೆ
ವಿಷಯವು
ವಿಷಯ-ವೆಲ್ಲಕ್ಕೂ
ವಿಷಯ-ವೆಲ್ಲಾ
ವಿಷಯವೇ
ವಿಷಯ-ವೇನು
ವಿಷಯ-ವೇನೆಂದರೆ
ವಿಷಯ-ವೊಂದಿದೆ
ವಿಷಯ-ಸುಖ-ದಲ್ಲಿ
ವಿಷಯಾಕಾರ-ವನ್ನು
ವಿಷಯಾಕಾರ-ವಾಗಿರು-ವುದೇನೋ
ವಿಷಯೀಕರಣವೇ
ವಿಷಯೇಂದ್ರಿಯಂಗಳಾ-ನಲ್ಲ
ವಿಷ್ಣು
ವಿಷ್ಣು-ವಿನ
ವಿಸರ್ಜನ
ವಿಸರ್ಜಿಸಿ
ವಿಸರ್ಜಿ-ಸುತ್ತಾ
ವಿಸ್ತರ
ವಿಸ್ತರಣೆಯ
ವಿಸ್ತರಿಸಿ
ವಿಸ್ತರಿಸಿ-ರಲು
ವಿಸ್ತರಿಸು-ವಲ್ಲಿ
ವಿಸ್ತಾರ
ವಿಸ್ತಾರ-ಗಳ
ವಿಸ್ತಾರ-ವಾಗಿ
ವಿಸ್ತಾರ-ವಾದ
ವಿಸ್ತೃತ-ಗೊಳಿಸಿತ್ತು
ವಿಸ್ತೃತ-ವಾಗ-ಬೇಕು
ವಿಸ್ತೃತ-ವಾಗಿ
ವಿಸ್ತೃತ-ವಾದ
ವಿಸ್ಮಯಕ
ವಿಸ್ಮಯಕರ-ವಾಗಿದೆ
ವಿಸ್ಮಯಕರ-ವಾದ
ವಿಸ್ಮಯ-ಪಟ್ಟ
ವಿಸ್ಮಯ-ವನ್ನೂ
ವಿಸ್ಮಿತ-ರಾಗಿ
ವಿಸ್ಮಿತ-ರಾದರು
ವಿಹಂಗಮ
ವಿಹರಿಸು-ತ್ತಿದ್ದನು
ವಿಹರಿಸು-ತ್ತಿರುವ
ವಿಹಾರ
ವಿಹಾರ-ಗಳಲ್ಲಿ
ವಿಹಾರ-ಗಳು
ವಿಹಾರ-ದಿಂದ
ವಿಹಾರಿ
ವಿಹಿತ-ವಾದ
ವೀಕ್ಷಣೆ
ವೀಕ್ಷಿಸಿ-ದರೆ
ವೀಕ್ಷಿಸು-ತ್ತಿದ್ದರು
ವೀಣಾಪಾಣಿ
ವೀಣಾಪಾಣೀ
ವೀಣಾರವ
ವೀಣಾರವ-ವದು
ವೀಣೆ
ವೀಣೆ-ಯನ್ನಾರೋ
ವೀಣೆ-ಯಲ್ಲ
ವೀತ-ಸಂಸಾರ-ರಾಗಾಃ
ವೀರ
ವೀರತ-ವಿದೆ
ವೀರತ್ವ
ವೀರತ್ವ-ವಾಯಿತೆ
ವೀರದಾಪೆ
ವೀರಧ್ವಜ-ದಾರಿ
ವೀರಧ್ವಜ-ವನ್ನು
ವೀರನ
ವೀರ-ನಂತೆ
ವೀರ-ನಲ್ಲ-ದಿದ್ದಲ್ಲಿ
ವೀರ-ನಾಗ-ಬೇಕು
ವೀರ-ನಾಗು
ವೀರ-ನಾದ-ವನು
ವೀರ-ನಿಗೆ
ವೀರನೇ
ವೀರ-ಭಾವ
ವೀರ-ಭೋಗ್ಯಾ
ವೀರ-ಮದ
ವೀರ-ಮಾತೆಯ-ರಾಗುವರು
ವೀರರ
ವೀರ-ರನ್ನು
ವೀರ-ರಸ
ವೀರ-ರಸ-ಪ್ರಧಾನ-ವಾದದ್ದು
ವೀರ-ರಸ-ವನ್ನು
ವೀರ-ರಸ-ವಿಲ್ಲ
ವೀರ-ರಾಗಿ-ಹೆವು
ವೀರರು
ವೀರವತ್ತಾದ
ವೀರಸ್ಫೂರ್ತಿ-ಯನ್ನು
ವೀರಸ್ಫೂರ್ತಿ-ಯುಳ್ಳ-ವರು
ವೀರ-ಸ್ವಭಾವದ
ವೀರಾ
ವೀರಾಣಾಮೇವ
ವೀರಾತ್ಮನೆ
ವೀರೋನ್ಮಾದದಿ
ವೀರ್ಯಪರಿಚಯ
ವೀರ್ಯವಂತ-ರೆಲ್ಲಾ
ವೀರ್ಯಹೀನ-ವಾಗಿದೆ
ವುಕಿಗಾಯ
ವೃಂದ-ದಲ್ಲೇ
ವೃಂದಾವನ-ದಲ್ಲಿ-ರುವರು
ವೃಂದಾವನ-ವನ್ನು
ವೃಕ್ಷ
ವೃಕ್ಷ-ಕಾರ್ಯಕೆ
ವೃಕ್ಷಕ್ಕೆ
ವೃಕ್ಷ-ಗಳನ್ನು
ವೃಕ್ಷಚ್ಛಾಯೆ
ವೃಕ್ಷದ
ವೃಕ್ಷ-ದಿಂದ
ವೃಕ್ಷ-ವಾಗಿ
ವೃತವರವಪುಃ
ವೃತ್ತ
ವೃತ್ತಕ್ಕೆ
ವೃತ್ತ-ಗಳು
ವೃತ್ತದ
ವೃತ್ತ-ದಲ್ಲಿ
ವೃತ್ತ-ದೊಳಗೆ
ವೃತ್ತ-ದೊಳಗೇ
ವೃತ್ತ-ಪಥ
ವೃತ್ತಯಃ
ವೃತ್ತ-ವನ್ನು
ವೃತ್ತ-ವನ್ನೆಳೆದರೆ
ವೃತ್ತಾಂತ-ವನ್ನು
ವೃತ್ತಾಂತವು
ವೃತ್ತಿ
ವೃತ್ತಿ-ಗಳನ್ನು
ವೃತ್ತಿ-ಗಳು
ವೃತ್ತಿ-ಗಳೂ
ವೃತ್ತಿ-ಗಳೆಲ್ಲ-ದರಿಂದ
ವೃತ್ತಿ-ಗಳೆಲ್ಲವೂ
ವೃತ್ತಿಗೆ
ವೃತ್ತಿಯ
ವೃತ್ತಿ-ಯನ್ನು
ವೃತ್ತಿ-ಯನ್ನೆಲ್ಲಾ
ವೃತ್ತಿ-ಯಾಗಿ-ರುತ್ತದೋ
ವೃತ್ತಿ-ಯಿಂದಲೇ
ವೃತ್ತಿ-ಶೂನ್ಯ
ವೃತ್ತಿ-ಶೂನ್ಯ-ವಾಗುತ್ತ-ದೆಯೋ
ವೃತ್ತಿ-ಹೀನ-ವಾಗುತ್ತ-ದೆಯೋ
ವೃಥಾ
ವೃಥಾಯ
ವೃಥಾ-ಯತ್ನಿಸಿ
ವೃಥೈವ
ವೃದ್ಧ
ವೃದ್ಧನ
ವೃದ್ಧರು
ವೃದ್ಧಿ
ವೃದ್ಧಿ-ಗೊಳಿಸಿ
ವೃದ್ಧಿ-ಗೊಳಿಸಿ-ಕೊಂಡ
ವೃದ್ಧಿ-ಯಾಗುವುದು
ವೃದ್ಧಿ-ಯಾಗುವುದೊ
ವೃದ್ಧೆ-ಯಾದ
ವೃಷಭ-ದೇವ
ವೆ
ವೆಚ್ಚ
ವೇಗ
ವೇಗ-ದಲ್ಲಿ
ವೇಗ-ವನ್ನು
ವೇಗ-ವಾಗಿ
ವೇಗೆ
ವೇದ
ವೇದ-ಉಪನಿಷತ್ತು-ಗಳ
ವೇದ-ಕಾಲದ
ವೇದಕ್ಕೆ
ವೇದ-ಗಳ
ವೇದ-ಗಳನ್ನು
ವೇದ-ಗಳನ್ನೆಲ್ಲಾ
ವೇದ-ಗಳನ್ನೋ-ದಲು
ವೇದ-ಗಳಲ್ಲಿ
ವೇದ-ಗಳಲ್ಲಿಲ್ಲದ
ವೇದ-ಗಳಲ್ಲೂ
ವೇದ-ಗಳಿಂದ
ವೇದ-ಗಳಿಗಿಂತ
ವೇದ-ಗಳಿಗೆ
ವೇದ-ಗಳು
ವೇದ-ಗಳೂ
ವೇದ-ಗಳೆಂದು
ವೇದ-ಗಳೆಂಬ
ವೇದ-ಗಳೆಂಬುದು
ವೇದ-ಗಳೆಲ್ಲ
ವೇದ-ಗಳೆಲ್ಲಾ
ವೇದ-ಗಳೇ
ವೇದ-ಗ್ರಂಥ-ಗಳ
ವೇದ-ಘೋಷದ
ವೇದಜ್ಞಾನ
ವೇದ-ತತ್ತ್ವಜ್ಞರೂ
ವೇದದ
ವೇದ-ದಲ್ಲಿ
ವೇದ-ದಲ್ಲಿದ್ದ
ವೇದ-ದಲ್ಲಿಯೆ
ವೇದ-ದಲ್ಲಿರು-ವುದ-ರಿಂದಲೇ
ವೇದ-ದಲ್ಲಿರು-ವುದು
ವೇದ-ದಲ್ಲೇ
ವೇದ-ದಿಂದಲೇ
ವೇದನೆ
ವೇದನೆಯನ್ನುಂಟು-ಮಾಡಿ-ದಾಗ
ವೇದನೆ-ಯುಂಟಾಗಿದೆ
ವೇದ-ಪಂಡಿತ-ನೆಂದು
ವೇದ-ಪಾಠ
ವೇದ-ಪಾಠವು
ವೇದ-ಪಾರಂಗತ-ರಾದ
ವೇದ-ಮಖ-ವಲ್ಲ
ವೇದ-ರೂಪೀ
ವೇದ-ವನ್ನು
ವೇದ-ವಾಙ್ಮ-ಯವು
ವೇದ-ವಾರಿ-ಧಿಮಥ-ನದಿಂ
ವೇದವೂ
ವೇದವೇ
ವೇದ-ವೇದಾಂತ-ಗಳ
ವೇದ-ವೇದಾಂತ-ಗಳಿಗೆ
ವೇದ-ವೇದಾಂತ-ಗಳು
ವೇದ-ವೇದಾಂತ-ವನ್ನೆಲ್ಲಾ
ವೇದ-ವೇದಾಂತ-ವೆಲ್ಲಾ
ವೇದ-ವೇದಾಂತ-ಸಿದ್ಧ
ವೇದ-ವೇನೊ
ವೇದ-ಶಾಸ್ತ್ರಜ್ಞ-ರಾದ
ವೇದ-ಸಮೂಹ-ವೆಂಬುದು
ವೇದಾಂತ
ವೇದಾಂತಕ್ಕೆ
ವೇದಾಂತ-ಗಳ
ವೇದಾಂತ-ಗಳನ್ನು
ವೇದಾಂತ-ಜ್ಞಾನ-ವನ್ನಾರ್ಜಿಸಲು
ವೇದಾಂತದ
ವೇದಾಂತ-ದಲ್ಲಿ-ರುವ
ವೇದಾಂತ-ದೊಡನೆ
ವೇದಾಂತ-ಧರ್ಮ
ವೇದಾಂತ-ವನ್ನು
ವೇದಾಂತ-ವನ್ನೇನೊ
ವೇದಾಂತ-ವನ್ನೋದಿದು-ದರಿಂದ
ವೇದಾಂತ-ವಾಕ್ಯ-ಗಳ
ವೇದಾಂತ-ವಾಕ್ಯೇಷು
ವೇದಾಂತ-ವಾದಿ-ಗಳು
ವೇದಾಂತ-ವಿದೆ-ಯೆಲ್ಲಾ
ವೇದಾಂತವೂ
ವೇದಾಂತ-ವ್ಯಾಸಂಗ
ವೇದಾಂತ-ಶಾಸ್ತ್ರ
ವೇದಾಂತಿ
ವೇದಾಂತಿ-ಗಳೆಲ್ಲಾ
ವೇದಾಂತಿಗೆ
ವೇದಾಂತೋಕ್ತ
ವೇದಾದಿ
ವೇದಾಧ್ಯಯನವೂ
ವೇದಾನ್ತಾಖ್ಯೈಃ
ವೇದಾಭ್ಯಾಸ
ವೇದೈಃ
ವೇದೋಕ್ತ
ವೇದೋಚ್ಚಾರಣೆಯ
ವೇದೋದಧಿಂ
ವೇದೋದ್ಧಾರ-ವಾಗಿರು-ವುದು
ವೇದ್ಯ-ವಾಗು-ವುದು
ವೇಳುತ
ವೇಳೆ
ವೇಳೆಗೆ
ವೇಳೆ-ಯಲ್ಲಿ
ವೇಳೆ-ಯಾಗಿತ್ತು
ವೇಳೆ-ಯಾಗಿದ್ದು-ದರಿಂದ
ವೇಳೆ-ಯಾದ್ದ-ರಿಂದ
ವೇಳೆ-ಯಿಲ್ಲದ್ದ-ರಿಂದ
ವೇಳೆಯೇ
ವೇಷದ
ವೇಷ-ದಲ್ಲಿ
ವೇಷ-ದಿಂದ
ವೇಷ-ಭೂಷಣ-ಗಳನ್ನು
ವೇಷ-ಭೂಷಣ-ಗಳನ್ನೂ
ವೇಷ-ಭೂಷಣ-ಗಳನ್ನೆಲ್ಲಾ
ವೇಷ-ಭೂಷಣ-ಗಳಿಂದ
ವೇಷ-ಭೂಷಣ-ಗಳಿಂದಲಂಕೃತ-ರಾಗಿ
ವೇಷ-ಭೂಷಾದಿ-ಗಳು
ವೈ
ವೈಖರಿಯೇ
ವೈಚಿತ್ರ್ಯ-ದಲ್ಲಿಯೂ
ವೈಚಿತ್ರ್ಯ-ವೇನು
ವೈಜ್ಞಾನಿಕ
ವೈದಿಕ
ವೈದಿಕ-ಭಾಷೆ
ವೈದಿಕರಿ-ಗಿಂತ
ವೈದ್ಯನ
ವೈದ್ಯ-ನಾಥ
ವೈದ್ಯನಿರ-ಬೇಕು
ವೈದ್ಯರು
ವೈಧೀ
ವೈಭವ
ವೈಭವದ
ವೈಭವದಿ
ವೈಭವ-ವನ್ನು
ವೈಮನಸ್ಯ
ವೈಮನಸ್ಯ-ವೆಲ್ಲಾ
ವೈಯಕ್ತಿಕ
ವೈಯಕ್ತಿಕತೆ-ಯನ್ನು
ವೈಯಕ್ತಿಕತೆಯೇ
ವೈಯಕ್ತಿಕ-ವಾಗಿ
ವೈಯಕ್ತಿಕ-ವಾದ
ವೈಯಾಕರಣರ
ವೈರಾಗ್ಯ
ವೈರಾಗ್ಯ-ಗಳನ್ನು
ವೈರಾಗ್ಯ-ಗಳು
ವೈರಾಗ್ಯ-ಗಳೇ
ವೈರಾಗ್ಯದ
ವೈರಾಗ್ಯ-ದಲ್ಲಿ
ವೈರಾಗ್ಯ-ಮೇವಾಭಯಂ
ವೈರಾಗ್ಯ-ವನ್ನು
ವೈರಾಗ್ಯ-ವುಂಟಾಗು-ತ್ತದೆ
ವೈರಾಗ್ಯ-ವುಂಟಾಗು-ವುದೋ
ವೈರಾಗ್ಯವೂ
ವೈರಾಗ್ಯವೇ
ವೈರಾಗ್ಯ-ವೊಂದೇ
ವೈರಾಗ್ಯ-ಶತಕ
ವೈರಾಗ್ಯೇಣ
ವೈರಿ-ಗಳನ್ನು
ವೈರಿ-ಗಳಲ್ಲಿ
ವೈರಿ-ಗಳಿಂದ
ವೈರಿ-ಗಳಿಬ್ಬರೂ
ವೈರಿ-ಗಳೂ
ವೈಲಕ್ಷಣ್ಯ
ವೈವಿಧ್ಯ
ವೈವಿಧ್ಯಕ್ಕೆ
ವೈವಿಧ್ಯದ
ವೈವಿಧ್ಯ-ದಲ್ಲಿ
ವೈವಿಧ್ಯ-ಮಯ-ವಾದ
ವೈವಿಧ್ಯ-ವನ್ನು
ವೈಶಾಖ
ವೈಶಿಷ್ಟ್ಯ
ವೈಶಿಷ್ಟ್ಯದ
ವೈಶಿಷ್ಟ್ಯ-ದೊಂದಿಗೆ
ವೈಶಿಷ್ಟ್ಯ-ವನ್ನು
ವೈಶಿಷ್ಟ್ಯ-ವನ್ನೇ
ವೈಶಿಷ್ಟ್ಯ-ವಾಗಿ-ಬಿಟ್ಟಿದೆ
ವೈಶ್ಯ
ವೈಷಮ್ಯ-ವಿಲ್ಲದೆ
ವೈಷ್ಣವ
ವೈಷ್ಣವ-ನೆಂದು
ವೈಷ್ಣವ-ಮ-ತವು
ವೈಷ್ಣವರ
ವೈಷ್ಣವ-ರಲ್ಲಿ
ವೈಷ್ಣವರು
ವೈಷ್ಣವ-ರೆಂತಲೂ
ವೈಸರಾಯ-ರಿಗೆ
ವೈಸಾದೃಶ್ಯ-ವನ್ನು
ವ್ಯಂಗ್ಯ-ವಾಗಿ
ವ್ಯಂಜಕ-ವಾದ
ವ್ಯಂಜನ-ದಂತೆ
ವ್ಯಂಜ-ನಾದಿ-ಗಳನ್ನು
ವ್ಯಕ್ತ
ವ್ಯಕ್ತ-ಗೊಂಡಿರು-ವುದನ್ನು
ವ್ಯಕ್ತ-ಗೊಳಿಸಿ
ವ್ಯಕ್ತ-ಗೊಳಿಸು-ತ್ತೀರಿ
ವ್ಯಕ್ತ-ಗೊಳಿಸುವ
ವ್ಯಕ್ತ-ಗೊಳಿಸು-ವುದಿಲ್ಲ
ವ್ಯಕ್ತ-ಗೊಳಿಸು-ವುವು
ವ್ಯಕ್ತತೆಯ
ವ್ಯಕ್ತ-ಪಡಿಸ-ಬಲ್ಲಿರಾ
ವ್ಯಕ್ತ-ಪಡಿಸಲು
ವ್ಯಕ್ತ-ಪಡಿಸಿದ
ವ್ಯಕ್ತ-ಪಡಿಸಿ-ದ್ದಾರೆ
ವ್ಯಕ್ತ-ಪಡಿಸು-ತ್ತಿತ್ತೆಂದರೆ
ವ್ಯಕ್ತ-ಪಡಿಸು-ತ್ತಿದ್ದರು
ವ್ಯಕ್ತ-ಪಡಿಸುವ
ವ್ಯಕ್ತ-ಪಡಿಸು-ವರು
ವ್ಯಕ್ತ-ಪಡಿಸು-ವುದಾದರೆ
ವ್ಯಕ್ತ-ಪಡಿಸು-ವುದೂ
ವ್ಯಕ್ತಳಾಗು-ತ್ತಿರುವೆ
ವ್ಯಕ್ತ-ವಾಗಲು
ವ್ಯಕ್ತ-ವಾಗುತ್ತದೆ
ವ್ಯಕ್ತ-ವಾಗುತ್ತವೆ
ವ್ಯಕ್ತ-ವಾಗು-ವಂತೆ
ವ್ಯಕ್ತ-ವಾಗು-ವುದು
ವ್ಯಕ್ತ-ವಾದಾಗ
ವ್ಯಕ್ತ-ವಾಯಿತು
ವ್ಯಕ್ತ-ಸ್ವರೂಪ
ವ್ಯಕ್ತಿ
ವ್ಯಕ್ತಿ-ಗಳ
ವ್ಯಕ್ತಿ-ಗಳನ್ನು
ವ್ಯಕ್ತಿ-ಗಳಾಗಿದ್ದರು
ವ್ಯಕ್ತಿ-ಗಳಾಗುವ-ರೆಂದು
ವ್ಯಕ್ತಿ-ಗಳಾದರೂ
ವ್ಯಕ್ತಿ-ಗಳಿದ್ದ-ರೆಂಬುದಂತೂ
ವ್ಯಕ್ತಿ-ಗಳು
ವ್ಯಕ್ತಿಗೆ
ವ್ಯಕ್ತಿತ್ವ
ವ್ಯಕ್ತಿತ್ವಕ್ಕಾಗಿಯೋ
ವ್ಯಕ್ತಿತ್ವಕ್ಕೆ
ವ್ಯಕ್ತಿತ್ವ-ಗಳ
ವ್ಯಕ್ತಿತ್ವದ
ವ್ಯಕ್ತಿತ್ವ-ವನ್ನು
ವ್ಯಕ್ತಿತ್ವ-ವನ್ನೆಲ್ಲಾ
ವ್ಯಕ್ತಿ-ಪೂಜೆ-ಯಲ್ಲಿ
ವ್ಯಕ್ತಿಯ
ವ್ಯಕ್ತಿ-ಯಂತೆ
ವ್ಯಕ್ತಿ-ಯನ್ನು
ವ್ಯಕ್ತಿ-ಯಲ್ಲ
ವ್ಯಕ್ತಿ-ಯಲ್ಲಿ
ವ್ಯಕ್ತಿ-ಯಲ್ಲಿದ್ದ
ವ್ಯಕ್ತಿ-ಯಾಗ-ಲಾರದು
ವ್ಯಕ್ತಿ-ಯಾಗಿದ್ದನು
ವ್ಯಕ್ತಿ-ಯಿಂದ
ವ್ಯಕ್ತಿ-ಯಿ-ರ-ಲಿಲ್ಲ
ವ್ಯಕ್ತಿಯು
ವ್ಯಕ್ತಿಯೂ
ವ್ಯಕ್ತಿ-ಯೆಂದು
ವ್ಯಕ್ತಿಯೇ
ವ್ಯಕ್ತಿ-ಯೊಬ್ಬ
ವ್ಯಕ್ತಿ-ಯೊಬ್ಬ-ನಿಗೆ
ವ್ಯಕ್ತಿ-ಯೊಬ್ಬನು
ವ್ಯಕ್ತಿ-ವೈಶಿಷ್ಟ್ಯತೆ
ವ್ಯಕ್ತಿ-ವೈಶಿಷ್ಟ್ಯ-ತೆಯ
ವ್ಯಕ್ತಿ-ವೈಶಿಷ್ಟ್ಯ-ತೆ-ಯುಳ್ಳ
ವ್ಯಕ್ತಿ-ವೈಶಿಷ್ಟ್ಯ-ವುಳ್ಳ
ವ್ಯಕ್ತಿಸ್ವಾ-ತಂತ್ರ್ಯಕ್ಕೆ
ವ್ಯಕ್ತಿಸ್ವಾಮ್ಯತೆ
ವ್ಯಕ್ತಿಸ್ವಾಮ್ಯತೆ-ಯನ್ನು
ವ್ಯಕ್ತಿಸ್ವಾಮ್ಯತೆ-ಯಲ್ಲಿ
ವ್ಯಕ್ತಿಸ್ವಾಮ್ಯತೆ-ಯುಳ್ಳ
ವ್ಯಕ್ತಿಸ್ವಾಮ್ಯತೆ-ಯುಳ್ಳ-ವ-ರಾಗಿದ್ದರು
ವ್ಯಕ್ತಿಸ್ವಾಮ್ಯತೆ-ಯುಳ್ಳ-ವರಾದ್ದ-ರಿಂದ
ವ್ಯತಿಕ್ರಮ-ವಿಲ್ಲದೆ
ವ್ಯತಿಕ್ರಮವೂ
ವ್ಯತಿ-ರಿಕ್ತ-ವಾದ
ವ್ಯತಿರೇಕೀ-ಕರಣ
ವ್ಯತೃಣತ್
ವ್ಯತ್ಯಸ್ತ-ವಾಗಿ-ರುತ್ತದೆ
ವ್ಯತ್ಯಾಸ
ವ್ಯತ್ಯಾಸ-ಗಳಿ-ರುತ್ತವೆ
ವ್ಯತ್ಯಾಸ-ವನ್ನು
ವ್ಯತ್ಯಾಸ-ವಾಗುತ್ತದೆ
ವ್ಯತ್ಯಾಸ-ವಾ-ಗು-ವು-ದಿಲ್ಲ
ವ್ಯತ್ಯಾಸ-ವಿದೆ
ವ್ಯತ್ಯಾಸ-ವಿ-ದೆಯೊ
ವ್ಯತ್ಯಾಸ-ವಿ-ದೆಯೋ
ವ್ಯತ್ಯಾಸ-ವಿಲ್ಲ
ವ್ಯತ್ಯಾಸವೂ
ವ್ಯತ್ಯಾಸ-ವೆನಗೀಗ
ವ್ಯತ್ಯಾಸವೇ
ವ್ಯಥಿತ-ನಾಗು-ವನು
ವ್ಯಥಿತ-ರಾ-ಗು-ವಂತೆ
ವ್ಯಥೆ
ವ್ಯಥೆ-ಗೊಳ್ಳುತ್ತಿತ್ತು
ವ್ಯಥೆ-ಪಡು-ವುದು
ವ್ಯಥೆ-ಯಾ-ಗು-ವು-ದಿಲ್ಲ
ವ್ಯಥೆ-ಯಾಗುವುದೇ
ವ್ಯಭಿ-ಚಾರ
ವ್ಯಭಿ-ಚಾ-ರಕ್ಕೆ
ವ್ಯಭಿ-ಚಾರ-ದಂತಹ
ವ್ಯಭಿ-ಚಾರವು
ವ್ಯಯ
ವ್ಯಯ-ಮಾಡ-ಬೇಕು
ವ್ಯರ್ಥ
ವ್ಯರ್ಥ-ಮಾಡಿ-ದರೆ
ವ್ಯರ್ಥ-ಮಾಡುತ್ತಿರು-ವೆ-ವೆಂಬುದು
ವ್ಯರ್ಥ-ಯತ್ನ-ವನೇಕೆ
ವ್ಯರ್ಥ-ವಲ್ಲ-ವೆಂದೆಂದಿಗೂ
ವ್ಯರ್ಥ-ವಾಗ-ಲಿಕ್ಕಿಲ್ಲ-ವೆಂದು
ವ್ಯರ್ಥ-ವಾಗಿ
ವ್ಯರ್ಥ-ವಾಗು-ವುದು
ವ್ಯರ್ಥ-ವಾ-ಯಿತು
ವ್ಯರ್ಥವೇ
ವ್ಯವ-ಸಾಯ
ವ್ಯವ-ಸಾಯ-ಗಳು
ವ್ಯವಸ್ಥೆ
ವ್ಯವಸ್ಥೆ-ಗಾಗಿ
ವ್ಯವಸ್ಥೆ-ಗೊಳಿಸು-ವುದು
ವ್ಯವಸ್ಥೆಯ
ವ್ಯವಸ್ಥೆ-ಯನ್ನು
ವ್ಯವಸ್ಥೆ-ಯಲ್ಲಿ
ವ್ಯವಸ್ಥೆಯೇ
ವ್ಯವಹ-ರಿ-ಸಲ್ಪಡು-ವು-ದಿಲ್ಲ
ವ್ಯವಹ-ರಿ-ಸಲ್ಪಡು-ವುದು
ವ್ಯವಹ-ರಿ-ಸಲ್ಪಡು-ವುದೋ
ವ್ಯವ-ಹಾರ
ವ್ಯವ-ಹಾರಕ್ಕೆ
ವ್ಯವ-ಹಾರ-ಗಳಲ್ಲಿ
ವ್ಯವ-ಹಾರ-ಗಳಾಗಲೀ
ವ್ಯವ-ಹಾರ-ಗಳಿಗೆ
ವ್ಯವ-ಹಾರ-ಗಳು
ವ್ಯವ-ಹಾರ-ಗಳೆಲ್ಲವೂ
ವ್ಯವ-ಹಾರದ
ವ್ಯವ-ಹಾರ-ದಲ್ಲಿ
ವ್ಯವ-ಹಾರ-ದಲ್ಲಿಯೂ
ವ್ಯಷ್ಟಿ-ಮೋಕ್ಷವು
ವ್ಯಷ್ಟಿಯೂ
ವ್ಯಷ್ಟಿಯೇ
ವ್ಯಾಕರಣ
ವ್ಯಾಕರ-ಣದ
ವ್ಯಾಕರ-ಣ-ದಂತಹ
ವ್ಯಾಕರ-ಣ-ದಲ್ಲಿ
ವ್ಯಾಕರ-ಣ-ವನ್ನು
ವ್ಯಾಕುಲ
ವ್ಯಾಕುಲ-ಗೊಂಡು
ವ್ಯಾಕುಲ-ಗೊಳ್ಳುತ್ತದೆ
ವ್ಯಾಕುಲತೆ
ವ್ಯಾಕುಲ-ತೆಯ
ವ್ಯಾಕುಲ-ತೆ-ಯನ್ನು
ವ್ಯಾಕುಲ-ತೆ-ಯುಂಟಾ-ಯಿತು
ವ್ಯಾಕುಲ-ತೆ-ಯೆಂದು
ವ್ಯಾಕುಲ-ದಿಂದ
ವ್ಯಾಕುಲ-ನಾಗಿ
ವ್ಯಾಕುಲ-ರಾಗುತ್ತಾರೆ
ವ್ಯಾಕುಲಾಗ್ರಹ-ದಿಂದ
ವ್ಯಾಕುಲಿತ-ವಾ-ಯಿತು
ವ್ಯಾಖ್ಯಾನ
ವ್ಯಾಖ್ಯಾ-ನದ
ವ್ಯಾಖ್ಯಾನ-ವನ್ನು
ವ್ಯಾಜ
ವ್ಯಾಜ-ಮಾತ್ರಂ
ವ್ಯಾಧಿ
ವ್ಯಾಧಿ-ಯಾಗಿ
ವ್ಯಾಪಕ-ನಾದ
ವ್ಯಾಪ-ಕ-ವಾಗಿ
ವ್ಯಾಪಕ-ವಾದ
ವ್ಯಾಪಾರ
ವ್ಯಾಪಾರಕ್ಕಾಗಿ
ವ್ಯಾಪಾರದ
ವ್ಯಾಪಾರ-ದಲ್ಲಿ
ವ್ಯಾಪಾರ-ವಾಗಿ
ವ್ಯಾಪಾರಿ-ಗಳು
ವ್ಯಾಪಿ
ವ್ಯಾಪಿ-ಸ-ಬಲ್ಲ
ವ್ಯಾಪಿಸಿ
ವ್ಯಾಪಿ-ಸಿತು
ವ್ಯಾಪಿ-ಸಿದೆ
ವ್ಯಾಪಿ-ಸು-ವಂತೆ
ವ್ಯಾಪ್ತನೊ
ವ್ಯಾಪ್ತ-ವಾದ
ವ್ಯಾಪ್ತಿ
ವ್ಯಾಪ್ತಿ-ಯನ್ನು
ವ್ಯಾಪ್ತಿ-ಯೊಳಗೇ
ವ್ಯಾಮೋಹ
ವ್ಯಾಮೋಹ-ದಿಂದ
ವ್ಯಾವ-ಹಾರಿಕ
ವ್ಯಾವ-ಹಾರಿ-ಕದ
ವ್ಯಾಸಂಗ
ವ್ಯಾಸಂಗಕ್ಕೆ
ವ್ಯಾಸಂಗ-ಗಳು
ವ್ಯಾಸಂಗ-ದಲ್ಲಿ
ವ್ಯಾಸಂಗ-ಮಾಡಿ
ವ್ಯಾಸಂಗ-ವನ್ನು
ವ್ಯುತ್ಥಾನ
ವ್ಯುತ್ಥಾನ-ಗಳೂ
ವ್ಯುತ್ಪತ್ತಿ
ವ್ಯೋಮ
ವ್ಯೋಮ-ಕೇಶ
ವ್ಯೋಮ-ದಲಿ
ವ್ಯೋಮ-ದೊಳಗೊಂದಾಗಿವೆ
ವ್ರಜಾಮಃ
ವ್ರತ
ವ್ರತಕ್ಕೆ
ವ್ರತ-ಗಳನ್ನು
ವ್ರತ-ಗಳಷ್ಟೇ
ವ್ರತ-ಗಳಿಂದ
ವ್ರತತ್ಯಾಗ
ವ್ರತ-ದಲ್ಲಿ
ವ್ರತ-ಧಾರಿ-ಗಳಾಗುವಂತೆ
ವ್ರತ-ವನ್ನವ-ಲಂಬಿಸಿ-ಕೊಂಡು
ವ್ರತ-ವನ್ನು
ವ್ರತ-ವಾಗಿ-ರ-ಬೇಕು
ವ್ರತ-ವೆಂದು
ವ್ರತಾಚ-ರಣೆ
ವ್ರಾತ್ಯ-ರಾಗಿದ್ದೀರಿ
ಶಂಕರ
ಶಂಕರನ
ಶಂಕರ-ಭಾಷ್ಯ-ದೊ-ಡನೆ
ಶಂಕರರು
ಶಂಕರಾ-ಚಾರ್ಯ
ಶಂಕರಾ-ಚಾರ್ಯರ
ಶಂಕರಾ-ಚಾರ್ಯ-ರಿಂದ
ಶಂಕರಾ-ಚಾರ್ಯ-ರಿಗೆ
ಶಂಕರಾ-ಚಾರ್ಯರು
ಶಂಕರಾ-ಚಾರ್ಯರೂ
ಶಂಕ-ರಾದಿ-ಗಳಲ್ಲಾದರೋ
ಶಂಖ
ಶಂಖದ
ಶಂಖಧ್ವನಿಯೂ
ಶಕತಿ
ಶಕ್ತ-ನಾಗುವೆ
ಶಕ್ತನು
ಶಕ್ತ-ರಲ್ಲ
ಶಕ್ತ-ರಾಗಿಲ್ಲ
ಶಕ್ತಿ
ಶಕ್ತಿ-ಗಳನ್ನು
ಶಕ್ತಿ-ಗಳಿಂದ
ಶಕ್ತಿ-ಗಳು
ಶಕ್ತಿ-ಗಳೊ-ಡನೆ
ಶಕ್ತಿ-ಗುಂದಿ-ದರು
ಶಕ್ತಿಗೆ
ಶಕ್ತಿ-ಗೆಲ್ಲಾ
ಶಕ್ತಿಪ್ರ-ಯೋಗ
ಶಕ್ತಿ-ಮೀರಿ
ಶಕ್ತಿಯ
ಶಕ್ತಿ-ಯಂತೆ
ಶಕ್ತಿ-ಯದು
ಶಕ್ತಿ-ಯನ್ನು
ಶಕ್ತಿ-ಯನ್ನುಂಟು-ಮಾಡು-ವುದು
ಶಕ್ತಿ-ಯನ್ನೂ
ಶಕ್ತಿ-ಯನ್ನೆಲ್ಲಾ
ಶಕ್ತಿ-ಯಲ್ಲದ
ಶಕ್ತಿ-ಯಲ್ಲಿ
ಶಕ್ತಿ-ಯಲ್ಲಿ-ರುವ
ಶಕ್ತಿ-ಯಾಗಿ
ಶಕ್ತಿ-ಯಾಗುವರು
ಶಕ್ತಿ-ಯಾದರೂ
ಶಕ್ತಿ-ಯಿಂದ
ಶಕ್ತಿ-ಯಿಂದಲೂ
ಶಕ್ತಿ-ಯಿಂದಲೇ
ಶಕ್ತಿ-ಯಿದೆ
ಶಕ್ತಿ-ಯಿಲ್ಲ
ಶಕ್ತಿ-ಯಿಲ್ಲ-ದಿ-ರು-ವು-ದ-ರಿಂದ
ಶಕ್ತಿಯು
ಶಕ್ತಿ-ಯು-ತ-ವಾದ
ಶಕ್ತಿ-ಯುಳ್ಳ
ಶಕ್ತಿಯೂ
ಶಕ್ತಿ-ಯೆಂದು
ಶಕ್ತಿ-ಯೆನಿತೋ
ಶಕ್ತಿ-ಯೆಲ್ಲಾ
ಶಕ್ತಿಯೇ
ಶಕ್ತಿ-ರೂಪ
ಶಕ್ತಿ-ವಂತ-ರಾಗಿ
ಶಕ್ತಿ-ಶಾಲಿ-ಗಳ
ಶಕ್ತಿ-ಶಾ-ಲಿನಿ
ಶಕ್ತಿ-ಶಾಲಿ-ಯಾದದ್ದು
ಶಕ್ತಿಶ್ಚೇ-ತನೇವ
ಶಕ್ತಿ-ಸಂಪನ್ನ-ನಾದ
ಶಕ್ತಿ-ಸಮುದ್ರಸ-ಮುತ್ಥ-ತರಂಗಂ
ಶಕ್ತಿ-ಸಾಧಕ-ನಾಗಿದ್ದನು
ಶಕ್ತಿ-ಸಾಮರ್ಥ್ಯ-ದಿಂದ
ಶಕ್ತಿ-ಹೀನ
ಶಕ್ತಿ-ಹೀನರು
ಶಕ್ಯ
ಶಕ್ಯ-ವಿದ್ದರೂ
ಶತ
ಶತ-ಕೋಟಿ
ಶತ-ಪಥ
ಶತ-ಮಾನ-ಗಳಲ್ಲಿ
ಶತ-ಮಾನ-ಗಳಿಂದ
ಶತ-ಮಾನ-ದಲ್ಲಿ
ಶತ-ವಂದನೆ
ಶತಶತ
ಶತ-ಸಹಸ್ರ
ಶತೃತೋಪ್
ಶತ್ರು-ಗಳಲ್ಲಿಯೂ
ಶತ್ರು-ಗ-ಳೆಂದು
ಶತ್ರುವು
ಶತ್ರೌ
ಶನಿ-ವಾರ
ಶಬ್ದ
ಶಬ್ದಕ್ಕೆ
ಶಬ್ದ-ಗಳ
ಶಬ್ದ-ಗಳು
ಶಬ್ದ-ದಿಂದ
ಶಬ್ದ-ಮಯ-ವಾದ
ಶಬ್ದ-ವನ್ನು
ಶಬ್ದ-ವನ್ನೇ
ಶಬ್ದ-ವಿ-ದೆಯೋ
ಶಬ್ದ-ವಿರ-ಬಲ್ಲ-ವೆಂದು
ಶಬ್ದವು
ಶಬ್ದವೂ
ಶಬ್ದವೆ
ಶಬ್ದವೇ
ಶಬ್ದ-ವೇ-ನಿದ್ದರೂ
ಶಬ್ದ-ವೊಂದೇ
ಶಬ್ದಾತ್
ಶಬ್ದಾತ್ಮಕ
ಶಬ್ದಾತ್ಮಕ-ವಾಗಿ-ರುತ್ತದೆ
ಶಬ್ದಾವಸ್ಥೆ
ಶಬ್ದಾವಸ್ಥೆ-ಯಲ್ಲಿ-ರುವ
ಶಬ್ದಾವಸ್ಥೆಯೆ
ಶಭಾ-ಶುಭ
ಶಮನ
ಶಮನ-ಗೊಳಿಸಿ-ದಂತಾ-ಗು-ವು-ದಿಲ್ಲ
ಶಮನ-ವಾಗಲಿ
ಶಮನ-ವಾ-ಗು-ವು-ದಿಲ್ಲ
ಶಮಿತ
ಶರಚ್ಚಂದ್ರ
ಶರಟು
ಶರಟು-ಗಳನ್ನು
ಶರಣ
ಶರಣಂ
ಶರಣ-ರಾದ
ಶರಣಾ-ಗತ
ಶರಣಾ-ಗತಿ
ಶರಣಾಗಿ
ಶರಣಾಗುತ್ತಾನೆ
ಶರಣಾಗುತ್ತೇನೆ
ಶರಣಾಗು-ವು-ದನ್ನು
ಶರಣಾದ
ಶರಣು
ಶರಣು-ಹೊಂದಿ
ಶರತ್-ಚಂದ್ರ
ಶರೀರ
ಶರೀರಂ
ಶರೀ-ರಕ್ಕೆ
ಶರೀರಜ್ಞಾನ
ಶರೀರತ್ಯಾಗ
ಶರೀ-ರದ
ಶರೀರ-ದಲ್ಲಿ
ಶರೀರ-ದಲ್ಲಿದ್ದರು
ಶರೀರ-ದಲ್ಲಿಯೂ
ಶರೀರ-ದಲ್ಲೇ
ಶರೀರ-ದಿಂದ
ಶರೀರ-ಧಾರಣ
ಶರೀರ-ಧಾರಣೆ
ಶರೀರ-ಧಾರ-ಣೆ-ಮಾಡಿ
ಶರೀರ-ಪಾಲನ
ಶರೀರ-ವನ್ನು
ಶರೀರ-ವಾದ
ಶರೀ-ರವು
ಶರೀರ-ವುಳ್ಳ-ವ-ರಿಗೆ
ಶರೀ-ರವೂ
ಶರೀರ-ವೆಂದು
ಶರೀರ-ವೆಂದೇ
ಶವ-ಗಳ
ಶವ-ಗಳನ್ನು
ಶವ-ಸಂಸ್ಕಾರ-ವನ್ನು
ಶಶಕ್ತಿ
ಶಶ-ಧರ
ಶಶಾಂಕ
ಶಶಾಂಕ-ಸುಂದರ
ಶಶಿ
ಶಶಿ-ಕಳೆಯೂ
ಶಶಿ-ಭೂಷಣ
ಶಶಿಯ
ಶಶಿಯು
ಶಾಂತ
ಶಾಂತ-ಗೀತೆಯ
ಶಾಂತ-ಗೊಳಿ-ಸಲು
ಶಾಂತ-ಚಿತ್ತ-ದಿಂದ
ಶಾಂತ-ಚಿತ್ತ-ರಾಗಿ
ಶಾಂತ-ತೆಯು
ಶಾಂತ-ನಾಗಿ
ಶಾಂತ-ಭಾವ-ವನ್ನು
ಶಾಂತ-ಮನಸ್ಕ-ನಾಗಿ
ಶಾಂತ-ಮರ್ಮರ
ಶಾಂತ-ವಾಗಿ
ಶಾಂತ-ವಾಗು-ವುವೋ
ಶಾಂತ-ವಾದ
ಶಾಂತಿ
ಶಾಂತಿಂ
ಶಾಂತಿ-ಗಾಗಿ
ಶಾಂತಿಗೆ
ಶಾಂತಿ-ಧಾ-ಮದ
ಶಾಂತಿ-ಮ-ಯವೂ
ಶಾಂತಿಯ
ಶಾಂತಿ-ಯದು
ಶಾಂತಿ-ಯನು
ಶಾಂತಿ-ಯನ್ನು
ಶಾಂತಿ-ಯಿಂದ
ಶಾಂತಿ-ಯಿಂದಿ-ರಲು
ಶಾಂತಿ-ರಸ್ತು
ಶಾಂತಿ-ರಾಮ
ಶಾಂತಿ-ವಿಶ್ರಾಂತಿ-ಗಳ
ಶಾಂತಿ-ಶುಭ-ವೆಂದೆಂದು
ಶಾಂತಿ-ಸುಖ-ವನ್ನು
ಶಾಂತೀರ್
ಶಾಖೆ-ಗಳನ್ನು
ಶಾಖೆ-ಗಳನ್ನೂ
ಶಾಖೆ-ಗಳನ್ನೆಲ್ಲಾ
ಶಾಖೆ-ಗಳಲ್ಲಿ
ಶಾಖೆ-ಗಳಿಗೂ
ಶಾಖೆ-ಗಳಿ-ರಲಿ
ಶಾಖೆ-ಗಳು
ಶಾಖೆಗೂ
ಶಾಖೆಯೇ
ಶಾನ್ತಂ
ಶಾಪ
ಶಾಪವೂ
ಶಾಬ್ದಿಕ
ಶಾರದಾ
ಶಾರದಾ-ನಂದ
ಶಾರದಾ-ನಂದ-ರನ್ನು
ಶಾರದಾ-ನಂದರು
ಶಾರೀ-ರಿಕ
ಶಾರೀ-ರಿಕ-ವಾಗಿ
ಶಾಲಾ-ಕಾಲೇಜು-ಗಳನ್ನು
ಶಾಲೆ
ಶಾಲೆ-ಗಳನ್ನು
ಶಾಲೆಯ
ಶಾಲೆ-ಯನ್ನು
ಶಾಲೆ-ಯಲ್ಲಿ
ಶಾವಿ-ಗೆ-ಯಿಂದ
ಶಾಶ್ವತ
ಶಾಶ್ವ-ತದ
ಶಾಶ್ವತ-ವಾಗಿ
ಶಾಶ್ವತ-ವಾಗಿ-ರುವ
ಶಾಶ್ವತ-ವಾಗಿ-ರು-ವುದು
ಶಾಶ್ವತ-ವಾದ
ಶಾಸನ
ಶಾಸನ-ಕಾರ-ರೆಲ್ಲರ
ಶಾಸನ-ವನ್ನು
ಶಾಸ್ತಿ
ಶಾಸ್ತ್ರ
ಶಾಸ್ತ್ರ-ಕಾ-ರರು
ಶಾಸ್ತ್ರಕ್ಕೆ
ಶಾಸ್ತ್ರ-ಗಳ
ಶಾಸ್ತ್ರ-ಗಳನ್ನು
ಶಾಸ್ತ್ರ-ಗಳಲ್ಲಿ
ಶಾಸ್ತ್ರ-ಗಳಲ್ಲಿ-ರುವ
ಶಾಸ್ತ್ರ-ಗಳಲ್ಲೆಲ್ಲಾ
ಶಾಸ್ತ್ರ-ಗಳು
ಶಾಸ್ತ್ರ-ಗಳೂ
ಶಾಸ್ತ್ರ-ಗ-ಳೆಂದು
ಶಾಸ್ತ್ರ-ಗಳೆಲ್ಲಾ
ಶಾಸ್ತ್ರ-ಗಳೇ
ಶಾಸ್ತ್ರ-ಗಳೇನೂ
ಶಾಸ್ತ್ರಗ್ರಂಥ-ಗಳ
ಶಾಸ್ತ್ರಗ್ರಂಥ-ಗಳೆಲ್ಲ
ಶಾಸ್ತ್ರಗ್ರಂಥ-ಗಳೇ
ಶಾಸ್ತ್ರಜ್ಞಾನಾಡಂಬರ-ಗಳಿಂದ
ಶಾಸ್ತ್ರದ
ಶಾಸ್ತ್ರ-ದಲ್ಲಿ
ಶಾಸ್ತ್ರ-ದಲ್ಲಿಯೂ
ಶಾಸ್ತ್ರ-ದಲ್ಲೆಲ್ಲ
ಶಾಸ್ತ್ರ-ದಿಂದ
ಶಾಸ್ತ್ರ-ಪಾಠ
ಶಾಸ್ತ್ರ-ಪಾಠ-ಕ-ನಾಗಿಯೂ
ಶಾಸ್ತ್ರ-ಪಾರಂಗತ-ನಾಗಿ-ರುವೆ
ಶಾಸ್ತ್ರಪ್ರ-ಸಂಗ-ಗಳು
ಶಾಸ್ತ್ರ-ಮಂತ್ರಾ-ದಿ-ಗಳು
ಶಾಸ್ತ್ರ-ಮುಖ-ದಿಂದ
ಶಾಸ್ತ್ರ-ರೀತಿ
ಶಾಸ್ತ್ರ-ವನ್ನನು-ಸ-ರಿಸಿ
ಶಾಸ್ತ್ರ-ವನ್ನು
ಶಾಸ್ತ್ರ-ವನ್ನೂ
ಶಾಸ್ತ್ರ-ವನ್ನೋದ-ಬೇ-ಕಾದದ್ದಿಲ್ಲ
ಶಾಸ್ತ್ರ-ವನ್ನೋದುತ್ತಿದ್ದಾಗ
ಶಾಸ್ತ್ರವು
ಶಾಸ್ತ್ರ-ವೆಲ್ಲ
ಶಾಸ್ತ್ರವೇ
ಶಾಸ್ತ್ರ-ವೇನು
ಶಾಸ್ತ್ರಾದಿ
ಶಾಸ್ತ್ರಾದಿ-ಗಳನ್ನು
ಶಾಸ್ತ್ರಾನು-ಸಾರ-ವಾಗಿ
ಶಾಸ್ತ್ರಾಭ್ಯಾಸ
ಶಾಸ್ತ್ರಾರ್ಥವು
ಶಾಸ್ತ್ರೀಯ
ಶಾಸ್ತ್ರೋಕ್ತ-ವಾಗಿ
ಶಾಸ್ತ್ರೋಕ್ತ-ವಾದ
ಶಾಸ್ತ್ರೋಪ-ದೇಶ-ವನ್ನು
ಶಾಸ್ರೋಕ್ತ-ಮಾರ್ಗ-ವನ್ನು
ಶಾಸ್ರೋಕ್ತ-ವಾದ
ಶಿಕಾಗೋದ
ಶಿಕ್ಷಕ
ಶಿಕ್ಷಣ
ಶಿಕ್ಷಣದ
ಶಿಕ್ಷಣ-ದಿಂದ
ಶಿಕ್ಷಣ-ದಿಂದಾಗುವ
ಶಿಕ್ಷಣ-ಪದ್ಧತಿ
ಶಿಕ್ಷಣಪ್ರ-ಚಾರ-ವಾ-ಗ-ದಿದ್ದರೆ
ಶಿಕ್ಷಣ-ವನ್ನು
ಶಿಕ್ಷಣ-ವಾದ
ಶಿಕ್ಷಣ-ವುಳ್ಳ-ವನಿರ-ಬೇಕು
ಶಿಕ್ಷಣವೂ
ಶಿಕ್ಷಾ
ಶಿಕ್ಷಿತ
ಶಿಕ್ಷಿತ-ರನ್ನಾಗಿ
ಶಿಕ್ಷಿತ-ರಾಗದೇ
ಶಿಕ್ಷಿತ-ರಾದ
ಶಿಕ್ಷಿತ-ರಾದಂತಾಯಿತು
ಶಿಕ್ಷಿತ-ರಾದ-ವರು
ಶಿಕ್ಷಿತ-ರೆಂದು
ಶಿಕ್ಷಿತೆ-ಯರೂ
ಶಿಕ್ಷಿಸಬೇ-ಕಾ-ಯಿತು
ಶಿಕ್ಷಿಸ-ಬೇಕು
ಶಿಕ್ಷಿಸ-ಲಾ-ಗುತ್ತದೆ
ಶಿಕ್ಷೆಗೊಳಗಾಗುವ
ಶಿಕ್ಷೆ-ಯನ್ನು
ಶಿಖರ
ಶಿಖರ-ವನ್ನೇ-ರಿ-ದರು
ಶಿಖಿ-ವಾಯು-ವಲ್ಲ
ಶಿಯರೆ
ಶಿಯಾರೆ
ಶಿರ
ಶಿರದ
ಶಿರ-ವನ್ನು
ಶಿರ-ಸಾ-ವಹಿಸಿ
ಶಿರ-ಸಾ-ವಹಿಸಿ-ಕೊಂಡನು
ಶಿರ-ಸಾ-ವಹಿಸಿ-ದನು
ಶಿರಸ್ಥಾನ-ದಲ್ಲಿ
ಶಿರೋರತ್ನಪ್ರಾಯ-ವಾದದ್ದೆಂಬ
ಶಿರೋರತ್ನ-ವಾದ
ಶಿಲಾಲಿಪಿ-ಗಳು
ಶಿಲಾ-ಶಾಸನ
ಶಿಲಾ-ಶಾಸನ-ಗಳು
ಶಿಲಾ-ಶಾಸನದ
ಶಿಲುಬೆ
ಶಿಲುಬೆಗೆ
ಶಿಲೆಯ
ಶಿಲ್ಪ
ಶಿಲ್ಪ-ಕಲೆ-ಗಳಲ್ಲಿ
ಶಿಲ್ಪ-ಕಲೆಯ
ಶಿಲ್ಪ-ಕಲೆ-ಯನ್ನು
ಶಿಲ್ಪ-ಕಲೆ-ಯಲ್ಲಿ
ಶಿಲ್ಪ-ಗಳನ್ನೋ
ಶಿಲ್ಪಿ-ಗಳು
ಶಿಲ್ಪಿ-ಯನ್ನು
ಶಿವ
ಶಿವಂ
ಶಿವ-ಕರೇ
ಶಿವ-ದರ್ಶನ-ದಿಂದ
ಶಿವ-ದರ್ಶನ-ವನ್ನು
ಶಿವ-ದರ್ಶನ-ವಾ-ಯಿತೆಂದು
ಶಿವನ
ಶಿವ-ನರ್ತನ
ಶಿವ-ನಲ್ಲಿ
ಶಿವ-ನಾಮ-ವನ್ನು
ಶಿವನು
ಶಿವ-ಭಾವ-ದಲ್ಲಿ
ಶಿವ-ಮ-ಹಿಮ್ನ
ಶಿವ-ರಾತ್ರಿಯ
ಶಿವ-ರುದ್ರಪ್ಪ
ಶಿವ-ಲಿಂಗ-ಗಳನ್ನು
ಶಿವ-ಶಕ್ತಿ-ಯರೆ
ಶಿವ-ಸಂಗೀತ
ಶಿವಸ್ತೋತ್ರ
ಶಿವಸ್ತೋತ್ರಂ
ಶಿವಸ್ತೋತ್ರ-ವನ್ನು
ಶಿವಸ್ಥಮ್
ಶಿವಾ-ನಂದ
ಶಿವಾ-ನಂದರು
ಶಿಶಿರ
ಶಿಶಿರದ
ಶಿಶು
ಶಿಶು-ರಕ್ರಂದನ
ಶಿಶು-ವಾಗಿ
ಶಿಶು-ವಿಗೋಸುಗ
ಶಿಶು-ವಿನ
ಶಿಶು-ವಿ-ನಂತೆ
ಶಿಶು-ಸಹಜ-ವಾದ
ಶಿಶು-ಹತ್ಯ
ಶಿಶು-ಹತ್ಯ-ವಾ-ಯಿತು
ಶಿಷ್ಟಾ-ಚಾರ
ಶಿಷ್ಯ
ಶಿಷ್ಯನ
ಶಿಷ್ಯ-ನನ್ನು
ಶಿಷ್ಯ-ನನ್ನುದ್ದೇಶಿಸಿ
ಶಿಷ್ಯ-ನನ್ನೂ
ಶಿಷ್ಯ-ನನ್ನೆಬ್ಬಿಸಿ
ಶಿಷ್ಯ-ನಲ್ಲದೆ
ಶಿಷ್ಯ-ನಾಗು-ವನು
ಶಿಷ್ಯ-ನಾದ
ಶಿಷ್ಯ-ನಿಂದ
ಶಿಷ್ಯ-ನಿಗೂ
ಶಿಷ್ಯ-ನಿಗೆ
ಶಿಷ್ಯನು
ಶಿಷ್ಯನೂ
ಶಿಷ್ಯ-ನೆನ್ನುತ್ತಾನೆ
ಶಿಷ್ಯನೇ
ಶಿಷ್ಯ-ನೊ-ಡನೆ
ಶಿಷ್ಯ-ನೊಬ್ಬ
ಶಿಷ್ಯ-ನೊಬ್ಬ-ನನ್ನು
ಶಿಷ್ಯ-ನೊಬ್ಬ-ನಿಗೆ
ಶಿಷ್ಯ-ನೊಬ್ಬನು
ಶಿಷ್ಯ-ನೊಬ್ಬನೇ
ಶಿಷ್ಯ-ನೊಳಗೂ
ಶಿಷ್ಯರ
ಶಿಷ್ಯ-ರ-ಚಿತ-ವಾದ
ಶಿಷ್ಯ-ರನ್ನು
ಶಿಷ್ಯ-ರನ್ನೂ
ಶಿಷ್ಯ-ರಲ್ಲಿ
ಶಿಷ್ಯ-ರಲ್ಲೇ
ಶಿಷ್ಯ-ರಾಗಿ-ಬಿಡುತ್ತಿದ್ದರು
ಶಿಷ್ಯ-ರಾಗು-ವುದಕ್ಕೆ
ಶಿಷ್ಯ-ರಾದ
ಶಿಷ್ಯ-ರಿಂದಲೇ
ಶಿಷ್ಯ-ರಿಗೆ
ಶಿಷ್ಯ-ರಿದ್ದಾರೆ
ಶಿಷ್ಯರು
ಶಿಷ್ಯರೂ
ಶಿಷ್ಯ-ರೆಂದು
ಶಿಷ್ಯ-ರೆಲ್ಲರೂ
ಶಿಷ್ಯ-ರೆಲ್ಲಾ
ಶಿಷ್ಯ-ರೊಂದಿಗೆ
ಶಿಷ್ಯ-ರೊಡಗೂಡಿ
ಶಿಷ್ಯ-ರೊಡನೆ
ಶಿಷ್ಯ-ರೊಬ್ಬ-ರನ್ನು
ಶಿಷ್ಯರೋ
ಶಿಷ್ಯ-ವರ್ಗ-ದಲ್ಲಿ
ಶೀಘ್ರ
ಶೀಘ್ರ-ದಲ್ಲಿ
ಶೀಘ್ರ-ದಲ್ಲಿಯೇ
ಶೀಘ್ರ-ದಲ್ಲೇ
ಶೀಘ್ರ-ಲಿಪಿ-ಯಲ್ಲಿ
ಶೀಘ್ರ-ವಾಗಿ
ಶೀಘ್ರ-ವಾಗಿಯೆ
ಶೀಘ್ರ-ವಾಗಿಯೇ
ಶೀತಲ
ಶೀತಲ-ಮಾರುತನ
ಶೀತಲ-ವಾದ
ಶೀರ್ಷಿಕೆ
ಶೀರ್ಷಿಕೆ-ಯದು
ಶೀರ್ಷಿಕೆ-ಯಿಂದಿರುವ
ಶೀಲ
ಶೀಲದ
ಶೀಲ-ನಾದ
ಶೀಲರ
ಶೀಲ-ವಂತ-ರಾಗಿ-ರುತ್ತಾರೋ
ಶೀಲ-ವಂತ-ರಾದ
ಶೀಲ-ವನ್ನು
ಶೀಲ-ವನ್ನೆಲ್ಲಾ
ಶೀಲವು
ಶೀಲ-ಸಂಪನ್ನ-ನಾಗಿದ್ದರೂ
ಶುಕ
ಶುಕ-ದೇವ
ಶುಕ-ದೇವನ
ಶುಕ-ದೇವ-ನಲ್ಲಿ
ಶುಚಿ
ಶುದ್ಧ
ಶುದ್ಧ-ವಹ
ಶುದ್ಧ-ವಾಗಿಲ್ಲ-ದಿದ್ದರೆ
ಶುದ್ಧ-ವಾದರೆ
ಶುದ್ಧ-ಸತ್ತ್ವ-ರಾದ
ಶುದ್ಧಾದ್ವೈತ-ವಾದದ
ಶುದ್ಧಾ-ನಂದ
ಶುದ್ಧಾ-ನಂದರ
ಶುದ್ಧಾ-ನಂದ-ರಿಗೆ
ಶುದ್ಧಾ-ನಂದ-ರೊಡನೆ
ಶುದ್ಧಿ
ಶುದ್ಧಿಗೆ
ಶುದ್ಧಿ-ಯಲ್ಲಿ
ಶುನಾತೆ
ಶುನಾತೇ
ಶುನಾಯ್
ಶುನೆ
ಶುಭ
ಶುಭ-ಕಾರ್ಯ-ಗಳಾಗುವ
ಶುಭಕೆ
ಶುಭ-ಜನ್ಮೋತ್ಸವ
ಶುಭ-ದೃಷ್ಟಯಸ್ತೇ
ಶುಭ-ವಾಗು-ವುದು
ಶುಭಾ-ಶಯ-ಗಳನ್ನು
ಶುಭಾ-ಶುಭ-ಗಳು
ಶುಭೇ
ಶುಭ್ರ
ಶುಭ್ರ-ತೆ-ಯಲ್ಲಿ
ಶುಭ್ರ-ತೇಜಃ
ಶುಭ್ರ-ತೇಜಃಪ್ರಕಾಶಃ
ಶುಭ್ರ-ವಸ್ತ್ರ-ವನ್ನುಟ್ಟು
ಶುಯೇಛೆ
ಶುರು-ವಾಗು-ವುದು
ಶುಲ್ಕ-ವನ್ನು
ಶುಷ್ಯತು
ಶೂದ್ರ
ಶೂದ್ರ-ನಿಗೆ
ಶೂದ್ರ-ನೊಂದಿಗೆ
ಶೂದ್ರ-ರನ್ನು
ಶೂದ್ರ-ರಿಗೂ
ಶೂದ್ರರು
ಶೂದ್ರ-ರೆಂದು
ಶೂದ್ರ-ರೆಂಬ
ಶೂನ್ಯ
ಶೂನ್ಯತೆ
ಶೂನ್ಯ-ತೆಯು
ಶೂನ್ಯ-ದಲಿ
ಶೂನ್ಯ-ಪಥ-ದಲಿ
ಶೂನ್ಯವ
ಶೂನ್ಯ-ವಲ್ಲ
ಶೂರ
ಶೂರತ್ವದ
ಶೂಲ-ಕಾಂತಿ
ಶೃಂಖಲೆ-ಗಳನ್ನು
ಶೃಂಖಲೆ-ಗಳಲ್ಲಿ
ಶೃಂಖಲೆ-ಗಳೆಲ್ಲ-ವನ್ನೂ
ಶೃಂಖಲೆಗೆ
ಶೃಂಗೆ
ಶೇ
ಶೇಕಡ
ಶೇಕಡಾ
ಶೇಖರ-ವಾಗು-ವು-ದನ್ನು
ಶೇಖರ-ವಾ-ಯಿತು
ಶೇಖ-ರಿಸಿ-ಕೊಂಡು
ಶೇಖರಿ-ಸು-ವು-ದ-ರಿಂದ
ಶೇಖರಿ-ಸುವೆ
ಶೇಷೆ
ಶೈಲಿ
ಶೈಲಿ-ಗಳು
ಶೈಲಿ-ಯಲ್ಲಿ
ಶೈಲಿ-ಯಲ್ಲಿದ್ದ
ಶೈಲಿಯೂ
ಶೈಲಿಯೇ
ಶೊನೊ
ಶೊನೋ
ಶೋಕ
ಶೋಕಃ
ಶೋಕ-ದಾರಿರ್ಯ-ಗಳು
ಶೋಕ-ಪೂರಿತ
ಶೋಕಪ್ರ-ಹಾರ-ಗಳು
ಶೋಕ-ಭರಿತ-ಳಾಗಿದ್ದ
ಶೋಕ-ಮೋಹ-ಗಳು
ಶೋಕಿಪುದೇಕೆ
ಶೋಕಿಸ-ಬೇಡ
ಶೋಚನೀಯ
ಶೋಚನೀಯ-ವಾಗಿದೆ
ಶೋಚನೀ-ಯಾವಸ್ಥೆ-ಗಿಳಿ-ದಿದೆ
ಶೋಚನೀ-ಯಾವಸ್ಥೆಗೆ
ಶೋಚನೀ-ಯಾವಸ್ಥೆ-ಯಲ್ಲಿ
ಶೋಚನೀ-ಯಾವಸ್ಥೆಯಲ್ಲಿದೆ
ಶೋತೃ-ಗಳನ್ನು
ಶೋಧಿ-ಸಿದರೆ
ಶೋಭನ-ವಾದ
ಶೋಭಾ-ಮಯ
ಶೋಭಿ-ಸುತ್ತಿ-ರುವನೋ
ಶೋಭಿ-ಸುವ
ಶೌರ್ಯ-ವನ್ನು
ಶ್ಯಾಮ
ಶ್ಯಾಮಾ
ಶ್ಯಾಮಾ-ಳನ್ನು
ಶ್ಯಾಮೆಯು
ಶ್ರದ್ದೆಗೂ
ಶ್ರದ್ಧಾ-ಭಕ್ತಿ-ಗಳು
ಶ್ರದ್ಧಾ-ಭಕ್ತಿ-ಯುಳ್ಳ-ವ-ರಾಗಿ
ಶ್ರದ್ಧಾ-ಭಕ್ತಿ-ಯುಳ್ಳ-ವಳು
ಶ್ರದ್ಧಾ-ವಂತ
ಶ್ರದ್ಧಾ-ವಂತ-ರಾಗಿ
ಶ್ರದ್ಧಾ-ಹೀನ-ರಾಗುತ್ತಿದ್ದಾರೆಂಬು-ದನ್ನು
ಶ್ರದ್ಧೆ
ಶ್ರದ್ಧೆ-ಗಳನ್ನಿಟ್ಟು-ಕೊಂಡಿದ್ದರು
ಶ್ರದ್ಧೆ-ಗಳೆಲ್ಲ
ಶ್ರದ್ಧೆಗೂ
ಶ್ರದ್ಧೆಗೆ
ಶ್ರದ್ಧೆಯ
ಶ್ರದ್ಧೆ-ಯನ್ನಿಟ್ಟಿದ್ದೇವೆ
ಶ್ರದ್ಧೆ-ಯನ್ನು
ಶ್ರದ್ಧೆ-ಯಲ್ಲ
ಶ್ರದ್ಧೆ-ಯಿಂದ
ಶ್ರದ್ಧೆ-ಯಿಲ್ಲ
ಶ್ರದ್ಧೆ-ಯಿಲ್ಲ-ದಿದ್ದಲ್ಲಿ
ಶ್ರದ್ಧೆ-ಯುಳ್ಳ
ಶ್ರದ್ಧೆಯೇ
ಶ್ರಮ
ಶ್ರಮಣ
ಶ್ರಮ-ಣ-ನಿ-ಗಾ-ಯಿತೇ
ಶ್ರಮ-ಣ-ರನ್ನು
ಶ್ರಮ-ಣ-ರೆಲ್ಲರೂ
ಶ್ರಮ-ದಿಂದ
ಶ್ರಮ-ಪಟ್ಟು
ಶ್ರಮ-ವನ್ನು
ಶ್ರಮ-ವಿಲ್ಲದೆ
ಶ್ರಮ-ಶೀಲತೆ
ಶ್ರವಣ
ಶ್ರವ-ಣ-ಮ-ನನ-ಗಳನ್ನು
ಶ್ರವ-ಣೇಂದ್ರಿಯ-ಗಳಿವೆ
ಶ್ರಾದ್ಧ
ಶ್ರಾದ್ಧಕ್ಕೆ
ಶ್ರಾದ್ಧ-ಗಳನ್ನು
ಶ್ರಾದ್ಧದ
ಶ್ರಾದ್ಧ-ವನ್ನು
ಶ್ರಾದ್ಧಾದಿ-ಗಳನ್ನು
ಶ್ರಾದ್ಧಾದಿ-ಗಳಿಂದ
ಶ್ರೀ
ಶ್ರೀಕೃಷ್ಣ
ಶ್ರೀಕೃಷ್ಣ-ಚೈ-ತನ್ಯರು
ಶ್ರೀಕೃಷ್ಣನ
ಶ್ರೀಕೃಷ್ಣ-ನಂತೂ
ಶ್ರೀಕೃಷ್ಣ-ನನ್ನು
ಶ್ರೀಕೃಷ್ಣ-ನನ್ನೇ
ಶ್ರೀಕೃಷ್ಣ-ನಿಗೆ
ಶ್ರೀಕೃಷ್ಣ-ನಿರ-ಬ-ಹುದು
ಶ್ರೀಕೃಷ್ಣನು
ಶ್ರೀಕೃಷ್ಣನೂ
ಶ್ರೀಕೃಷ್ಣನೆ
ಶ್ರೀಗುರು-ದೇವ
ಶ್ರೀಗುರು-ವಿಗೆ
ಶ್ರೀಗುರು-ವಿನ
ಶ್ರೀಗೌರಾಂಗ
ಶ್ರೀಚೈ-ತನ್ಯ
ಶ್ರೀಚೈ-ತನ್ಯ-ದೇವನೂ
ಶ್ರೀಚೈ-ತನ್ಯನು
ಶ್ರೀಚೈ-ತನ್ಯರ
ಶ್ರೀಚೈ-ತನ್ಯ-ರಿಂದ
ಶ್ರೀಚೈ-ತನ್ಯರು
ಶ್ರೀಚೈ-ತನ್ಯ-ರೆಂದೂ
ಶ್ರೀದೇವಿ-ಯಿಂದ
ಶ್ರೀನಗ-ರಕ್ಕೆ
ಶ್ರೀನಗರ-ದಲ್ಲಿ
ಶ್ರೀಪದ
ಶ್ರೀಮಂತರ
ಶ್ರೀಮಂತರು
ಶ್ರೀಮಂತರೂ
ಶ್ರೀಮಂತ-ವಾಗಿತ್ತು
ಶ್ರೀಮದ್-ಶಂಕರಾ-ಚಾರ್ಯರ
ಶ್ರೀಮಾತೆ
ಶ್ರೀಮಾನ್
ಶ್ರೀಮುಖ-ದಲ್ಲಿ
ಶ್ರೀಮುಖ-ದಿಂದ
ಶ್ರೀಮುಖ-ದಿಂದಲೇ
ಶ್ರೀಯುತ
ಶ್ರೀರಾಮ
ಶ್ರೀರಾಮ-ಕೃಷ್ಟ
ಶ್ರೀರಾಮ-ಕೃಷ್ಣ
ಶ್ರೀರಾಮ-ಕೃಷ್ಣ-ದಾಸ-ರದು
ಶ್ರೀರಾಮ-ಕೃಷ್ಣನ
ಶ್ರೀರಾಮ-ಕೃಷ್ಣ-ನಿಗೆ
ಶ್ರೀರಾಮ-ಕೃಷ್ಣರ
ಶ್ರೀರಾಮ-ಕೃಷ್ಣ-ರಂತೆ
ಶ್ರೀರಾಮ-ಕೃಷ್ಣ-ರನ್ನು
ಶ್ರೀರಾಮ-ಕೃಷ್ಣ-ರಲಿ
ಶ್ರೀರಾಮ-ಕೃಷ್ಣ-ರಲ್ಲಿ
ಶ್ರೀರಾಮ-ಕೃಷ್ಣ-ರಲ್ಲಿನ
ಶ್ರೀರಾಮ-ಕೃಷ್ಣ-ರಿಗೆ
ಶ್ರೀರಾಮ-ಕೃಷ್ಣರು
ಶ್ರೀರಾಮ-ಕೃಷ್ಣರೆ
ಶ್ರೀರಾಮ-ಕೃಷ್ಣ-ರೆಂಬ
ಶ್ರೀರಾಮ-ಕೃಷ್ಣ-ರೆ-ಡೆಗೆ
ಶ್ರೀರಾಮ-ಚಂದ್ರ
ಶ್ರೀರಾಮ-ಚಂದ್ರ-ನಂತಹ
ಶ್ರೀರಾಮನ
ಶ್ರೀರಾಮನೂ
ಶ್ರೀರಾಮ-ರಾಮ
ಶ್ರೀಲಂಕೆ
ಶ್ರೀವಚ-ನವು
ಶ್ರೀಶಂಕರಾ-ಚಾರ್ಯರೂ
ಶ್ರೀಶು-ಕನು
ಶ್ರೀಸಂಚಿಂತ್ಯಂ
ಶ್ರುತಿ
ಶ್ರುತಿಗೆ
ಶ್ರುತಿ-ಗೊಡುತ್ತಿದೆ
ಶ್ರುತಿಯು
ಶ್ರುತಿ-ಯೊ-ಡನೆ
ಶ್ರೇಣಿ
ಶ್ರೇಣಿ-ಗಳಾಚೆ
ಶ್ರೇಣಿಯ
ಶ್ರೇಣಿ-ಯಲ್ಲಿ
ಶ್ರೇಣಿ-ಯೊ-ಳಕ್ಕೆ
ಶ್ರೇಯಸ್ಕರವೋ
ಶ್ರೇಯಸ್ಸಾಗು-ವು-ದಲ್ಲದೆ
ಶ್ರೇಯಸ್ಸಿ-ಗಾಗಿ
ಶ್ರೇಯಸ್ಸಿಗೆ
ಶ್ರೇಷ್ಠ
ಶ್ರೇಷ್ಠ-ತಮ
ಶ್ರೇಷ್ಠ-ತೆ-ಯನ್ನು
ಶ್ರೇಷ್ಠತ್ವ-ವನ್ನು
ಶ್ರೇಷ್ಠ-ನಾಗಿ-ರು-ವೆ-ನೇನು
ಶ್ರೇಷ್ಠ-ಮಾರ್ಗ
ಶ್ರೇಷ್ಠ-ರಾಗುವರು
ಶ್ರೇಷ್ಠ-ರಾದ
ಶ್ರೇಷ್ಠ-ರೂಪವೇ
ಶ್ರೇಷ್ಠ-ವಲ್ಲ
ಶ್ರೇಷ್ಠ-ವಾದ
ಶ್ರೇಷ್ಠ-ವಾದದ್ದಾದರೆ
ಶ್ರೇಷ್ಠ-ವಾ-ದದ್ದು
ಶ್ರೇಷ್ಠ-ವಾದ-ವ-ರನ್ನು
ಶ್ರೇಷ್ಠ-ವಾದುದೊಂದು
ಶ್ರೇಷ್ಠ-ವೆಂಬು-ದಾಗಿ
ಶ್ರೇಷ್ಠಾಂಶ-ವೆಂದರೆ
ಶ್ರೋತೃ-ಗಳ
ಶ್ರೋತೃ-ಗಳು
ಶ್ರೌತ
ಶ್ಲತ
ಶ್ಲಾಘ-ನೆಗೆ
ಶ್ಲಾಘಿ-ಸಿ-ದರು
ಶ್ಲಾಘಿ-ಸುತ್ತಿದ್ದರು
ಶ್ಲಾಘಿ-ಸುತ್ತೇವೆ
ಶ್ಲಾಘಿ-ಸುವನು
ಶ್ಲೋಕಕ್ಕೆ
ಶ್ಲೋಕ-ಗಳನ್ನು
ಶ್ಲೋಕ-ಗಳು
ಶ್ಲೋಕದ
ಶ್ಲೋಕ-ದಲ್ಲಿ
ಶ್ಲೋಕ-ದಲ್ಲಿ-ರು-ವಂತೆ
ಶ್ಲೋಕ-ವನ್ನು
ಶ್ಲೋಕ-ವೊಂದನ್ನು
ಶ್ವಾಸ-ದಲ್ಲಿಯೂ
ಶ್ವೇತ
ಶ್ವೇತ-ಕೇತು
ಶ್ವೇತಶ್ಯಾಮಲ
ಷಡ್ರಿಪು-ಗಳೆಲ್ಲವೂ
ಷರಾಯಿ
ಷರ್ಟನ್ನು
ಷರ್ಟಿನ
ಷರ್ಟು
ಷಷ್ಠಿಯ-ಪೂಜೆ
ಷಾರಾ-ಪಾರ
ಷಿಯಾ
ಷಿಲ್ಲಾಂಗ್
ಷೇಕ್ಸ್
ಷೋಕಿ
ಷ್ಣ
ಷ್ಣಾಂತಂ
ಸ
ಸಂ
ಸಂಕಟ
ಸಂಕಟಕೆ
ಸಂಕಟ-ಗಳ
ಸಂಕಟ-ಗಳ-ನೀಡಾಡುತ
ಸಂಕಟದ
ಸಂಕಟ-ದಲ್ಲಿಯೇ
ಸಂಕ-ಟವ
ಸಂಕಟ-ವಿಲ್ಲಿ
ಸಂಕ-ಟವೆ
ಸಂಕಟ-ಸಮುದ್ರ-ವನ್ನು
ಸಂಕಲ್ಪ
ಸಂಕಲ್ಪದ್ವಾರಾ
ಸಂಕಲ್ಪ-ವನ್ನು
ಸಂಕಲ್ಪ-ವನ್ನೂ
ಸಂಕಲ್ಪ-ವನ್ನೆಲ್ಲಾ
ಸಂಕಲ್ಪ-ವಾಗಿ
ಸಂಕಲ್ಪ-ವಾದ
ಸಂಕಲ್ಪ-ವಿದ್ದರೆ
ಸಂಕಲ್ಪ-ಶಕ್ತಿ-ಯನ್ನು
ಸಂಕಲ್ಪ-ಸಾಗರದ
ಸಂಕಲ್ಪಿ-ಸಿದ್ದರು
ಸಂಕೀರ್ತನೆ
ಸಂಕೀರ್ತನೆಯು
ಸಂಕು-ಚಿತ
ಸಂಕು-ಚಿತ-ತೆಯ
ಸಂಕು-ಚಿತ-ವಾದಂತಿದ್ದರೂ
ಸಂಕು-ಚಿತ-ವೆನಿ-ಸುತ್ತದೆ
ಸಂಕುಲ
ಸಂಕೇ-ತರ-ಹಿತ-ವಾಗಿವೆ
ಸಂಕೇತ-ವದು
ಸಂಕೇತ-ವನ್ನು
ಸಂಕೇತ-ವಾಗಿ-ರುವ
ಸಂಕೇತ-ವಾಗುತ್ತದೆ
ಸಂಕೋಚ
ಸಂಕೋಚ-ಗಳಿಲ್ಲದೆ
ಸಂಕೋಚ-ದಿಂದ
ಸಂಕೋಚ-ಪ-ಡದೆ
ಸಂಕೋಚ-ಪಡುತ್ತಿದ್ದು-ದನ್ನು
ಸಂಕೋ-ಲೆ-ಗಳ
ಸಂಕೋಲೆ-ಯಿಂದ
ಸಂಕ್ಷೇಪ-ವಾಗಿ
ಸಂಖ್ಯೆ
ಸಂಖ್ಯೆಯ
ಸಂಖ್ಯೆ-ಯನ್ನು
ಸಂಖ್ಯೆ-ಯಲ್ಲಿ
ಸಂಖ್ಯೆ-ಯಲ್ಲಿ-ರುವ
ಸಂಖ್ಯೆ-ಯಿಂದ
ಸಂಖ್ಯೆಯೂ
ಸಂಗಡ
ಸಂಗತಿ
ಸಂಗತಿ-ಗಳನ್ನು
ಸಂಗತಿ-ಗಳನ್ನೂ
ಸಂಗತಿ-ಗಳಲ್ಲಿಯೂ
ಸಂಗತಿ-ಗಳಿವೆ
ಸಂಗ-ತಿಗೆ
ಸಂಗ-ತಿಯ
ಸಂಗತಿ-ಯನ್ನು
ಸಂಗತಿ-ಯಾಗಿ-ರ-ಬ-ಹುದು
ಸಂಗತಿ-ಯೆಲ್ಲಾ
ಸಂಗ-ದಿಂದ
ಸಂಗ-ದಿಂದಲೂ
ಸಂಗ-ಮ-ವಾಗಿ
ಸಂಗಮ್
ಸಂಗರ
ಸಂಗ-ಲಾಭ
ಸಂಗ-ವನ್ನು
ಸಂಗ-ವೆಲ್ಲಾ
ಸಂಗ-ಸುಖ-ವನ್ನು
ಸಂಗಾತಿ
ಸಂಗೀತ
ಸಂಗೀತಕ್ಕಿಂತಲೂ
ಸಂಗೀ-ತಕ್ಕೆ
ಸಂಗೀತ-ಗಾ-ರರ
ಸಂಗೀತ-ಗಾ-ರರು
ಸಂಗೀತದ
ಸಂಗೀತ-ದಲ್ಲಿ
ಸಂಗೀತ-ದಲ್ಲಿಯೂ
ಸಂಗೀತ-ದಲ್ಲಿ-ರ-ಬೇ-ಕಾದ
ಸಂಗೀತ-ದಲ್ಲೂ
ಸಂಗೀತ-ವನ್ನು
ಸಂಗೀತವು
ಸಂಗೀತ-ವೃಕ್ಷ
ಸಂಗೀತವೇ
ಸಂಗೀತ-ಶಾಸ್ತ್ರ-ವೆಲ್ಲ
ಸಂಗೀತ-ಶಾಸ್ತ್ರ-ವೆಲ್ಲಾ
ಸಂಗೀತ-ಸು-ಧಾರ
ಸಂಗೆ
ಸಂಗ್ರಹ
ಸಂಗ್ರಹ-ವಾದ
ಸಂಗ್ರಹಾಲ-ಯ-ದಿಂದ
ಸಂಗ್ರಹಿಸ-ಬ-ಹುದು
ಸಂಗ್ರಹಿ-ಸಲು
ಸಂಗ್ರಹಿಸಿ
ಸಂಗ್ರಹಿಸು
ಸಂಗ್ರಹಿಸುತ್ತಾರೆ
ಸಂಗ್ರಹಿಸು-ವೆನು
ಸಂಗ್ರಾಮ
ಸಂಗ್ರಾಮ-ದಲ್ಲಿ
ಸಂಗ್ರಾಮ-ದಲ್ಲಿಯೇ
ಸಂಗ್ರಾಮವು
ಸಂಗ್ರಾಮ-ವೆಂಬ
ಸಂಗ್ರಾಮೋಪ-ಯೋಗಿ-ಯಾದ
ಸಂಘ
ಸಂಘಕ್ಕೆ
ಸಂಘ-ಗಳ
ಸಂಘ-ಗಳನ್ನೂ
ಸಂಘ-ಟನಾ
ಸಂಘ-ಟನಾ-ಶಕ್ತಿ-ಯನ್ನು
ಸಂಘ-ಟನೆ-ಗೊಳ್ಳ-ಲಿಲ್ಲ
ಸಂಘ-ಟಿಸಿ
ಸಂಘದ
ಸಂಘ-ದಲ್ಲಿ
ಸಂಘ-ದಲ್ಲಿನ
ಸಂಘ-ದೊ-ಡನೆ
ಸಂಘ-ರೂಪ-ವಾಗಿ
ಸಂಘರ್ಷ-ಸಾಗರದ
ಸಂಘ-ವನ್ನು
ಸಂಘ-ವಿಲ್ಲದೆ
ಸಂಘವೇ
ಸಂಘ-ವೊಂದನ್ನು
ಸಂಘಸ್ಥಾ-ಪನೆ
ಸಂಚಕಾ-ರ-ವಾಗಿದೆ
ಸಂಚನ್ನು
ಸಂಚರಿಸ-ಬ-ಹುದು
ಸಂಚ-ರಿಸಿ
ಸಂಚ-ರಿಸಿ-ಕೊಂಡು
ಸಂಚ-ರಿಸಿದ
ಸಂಚ-ರಿಸಿ-ದಂತಾಯ್ತು
ಸಂಚ-ರಿಸಿ-ದಂತೆ
ಸಂಚರಿಸು
ಸಂಚರಿಸುತ್ತಿತ್ತು
ಸಂಚರಿಸುತ್ತಿದ್ದರೂ
ಸಂಚರಿಸುತ್ತಿರುವ
ಸಂಚರಿಸುತ್ತಿರು-ವಾಗ
ಸಂಚರಿ-ಸುತ್ತಿವೆ
ಸಂಚರಿಸು-ವುದಕ್ಕೆ
ಸಂಚರಿಸು-ವುದು
ಸಂಚಲನೆ
ಸಂಚಾತ
ಸಂಚಾರ
ಸಂಚಾ-ರಕ್ಕೆ
ಸಂಚಾರದ
ಸಂಚಾರಿಗೆ
ಸಂಚಿಕೆ
ಸಂಚಿ-ಕೆಯ
ಸಂಚಿ-ಕೆಯು
ಸಂಜಾ-ತನೇ
ಸಂಜೆ
ಸಂಜೆಗೆ
ಸಂಜೆ-ಯಲಿ
ಸಂಜೆ-ಯಲ್ಲಿ
ಸಂಜೆ-ಯಾಗಲು
ಸಂಜೆ-ಯಾಗಿತ್ತು
ಸಂಜೆ-ಯಾದ
ಸಂತ
ಸಂತತ
ಸಂತ-ತವು
ಸಂತತಿ
ಸಂತ-ತಿ-ಗಳ
ಸಂತ-ತಿ-ಗಳು
ಸಂತ-ತಿ-ಯವರು
ಸಂತ-ತಿ-ಯಿಂದಲ್ಲ
ಸಂತ-ತಿ-ಯಿಂದಾ-ಗಲಿ
ಸಂತ-ನಲ್ಲಿಯು
ಸಂತ-ರೆಂದಾದರೂ
ಸಂತಸ
ಸಂತ-ಸಕೆ
ಸಂತ-ಸದ
ಸಂತ-ಸದಿ
ಸಂತ-ಸ-ವನೀವ
ಸಂತ-ಸ-ವನೀವುದಕೆ
ಸಂತಾ-ನಕ್ಕೆ
ಸಂತಾನ-ದ-ವರು
ಸಂತಾನಯಂತಿ
ಸಂತಾನ-ರೆಲ್ಲಾ
ಸಂತಾನ-ವನ್ನು
ಸಂತಾಲ
ಸಂತಾಲರ
ಸಂತಾಲ-ರನ್ನು
ಸಂತಾಲ-ರಲ್ಲಿ
ಸಂತಾಲ-ರಿಗೆ
ಸಂತಾಲರು
ಸಂತಾಲ-ರೊಡನೆ
ಸಂತುಷ್ಟ-ರಾಗಿ-ರುತ್ತಾರೆ
ಸಂತುಷ್ಟ-ರಾದರು
ಸಂತುಷ್ಟ-ಳಾದಾಗ
ಸಂತೃಪ್ತ-ರಾಗಿದ್ದಾರೆ
ಸಂತೈಸಿ
ಸಂತೈ-ಸುವ-ವ-ರಿಲ್ಲ
ಸಂತೋಷ
ಸಂತೋಷ-ಗೊಳಿಸಿ
ಸಂತೋಷ-ಗೊಳಿಸಿ-ದರು
ಸಂತೋಷ-ದಲ್ಲಿ
ಸಂತೋಷ-ದಿಂದ
ಸಂತೋಷ-ಪಟ್ಟು
ಸಂತೋಷ-ಪಡುತ್ತಿದ್ದರು
ಸಂತೋಷ-ವನ್ನು
ಸಂತೋಷ-ವಾಗಿದ್ದಾರೆ
ಸಂತೋಷ-ವಾಗಿದ್ದೇನೆ
ಸಂತೋಷ-ವಾಗುತ್ತದೆ
ಸಂತೋಷ-ವಾಗು-ವುದು
ಸಂತೋಷ-ವಾ-ಯಿತು
ಸಂತೋಷವೂ
ಸಂತೋಷವೆ
ಸಂತೋಷಿ-ಸುತ್ತಾರೆ
ಸಂತೋಷಿ-ಸುತ್ತಿದ್ದರು
ಸಂದರ್ಭ
ಸಂದರ್ಭಕ್ಕೆ
ಸಂದರ್ಭ-ಗಳಲ್ಲಿ
ಸಂದರ್ಭ-ಗಳಲ್ಲಿಯೂ
ಸಂದರ್ಭದ
ಸಂದರ್ಭ-ದಲ್ಲಿ
ಸಂದರ್ಭ-ದಲ್ಲಿಯೇ
ಸಂದರ್ಭ-ವನ್ನು
ಸಂದರ್ಶನ
ಸಂದರ್ಶನಕ್ಕೇ
ಸಂದರ್ಶನ-ದಿಂದ
ಸಂದರ್ಶಿ-ಸಲು
ಸಂದರ್ಶಿಸಿ
ಸಂದರ್ಶಿ-ಸಿದ
ಸಂದರ್ಶಿ-ಸಿದೆ
ಸಂದಿಗ್ಧ
ಸಂದೇಶ
ಸಂದೇಶ-ಕನ
ಸಂದೇಶ-ವನ್ನು
ಸಂದೇಶ-ವನ್ನೂ
ಸಂದೇಶ-ವನ್ನೆಲ್ಲ
ಸಂದೇಹ
ಸಂದೇಹ-ಗಳನ್ನು
ಸಂದೇಹ-ದಲ್ಲಿ
ಸಂದೇಹ-ವಾದ
ಸಂದೇಹ-ವಿಲ್ಲ
ಸಂದೇಹವು
ಸಂದೇಹವೇ
ಸಂದೇಹಿ-ಸಿದಾಗ
ಸಂಧಾನ-ದಲ್ಲಿಯೇ
ಸಂಧ್ಯಾ-ವಂದನೆಯ
ಸಂಧ್ಯೆ-ಯಲ್ಲಿ
ಸಂನ್ಯಾಸ
ಸಂನ್ಯಾಸಂ
ಸಂನ್ಯಾಸಕ್ಕೆ
ಸಂನ್ಯಾಸಗ್ರಹ-ಣವೆ
ಸಂನ್ಯಾಸದ
ಸಂನ್ಯಾಸ-ದಂತೆ
ಸಂನ್ಯಾಸ-ದಲ್ಲಿ
ಸಂನ್ಯಾಸ-ಧರ್ಮದ
ಸಂನ್ಯಾಸ-ವನ್ನು
ಸಂನ್ಯಾಸ-ವಿದೆ
ಸಂನ್ಯಾಸ-ವಿಲ್ಲದೆ
ಸಂನ್ಯಾಸ-ವೆಂದು
ಸಂನ್ಯಾಸವ್ರತ-ವನ್ನು
ಸಂನ್ಯಾಸ-ಸಮುದಾ-ಯದ
ಸಂನ್ಯಾಸಾಶ್ರಮ
ಸಂನ್ಯಾಸಾಶ್ರಮದ
ಸಂನ್ಯಾಸಾಶ್ರಮ-ದಲ್ಲಿ
ಸಂನ್ಯಾಸಾಶ್ರಮ-ವನ್ನು
ಸಂನ್ಯಾಸಿ
ಸಂನ್ಯಾಸಿ-ಗಳ
ಸಂನ್ಯಾಸಿ-ಗಳಂತೆ
ಸಂನ್ಯಾಸಿ-ಗಳಂತೆಯೇ
ಸಂನ್ಯಾಸಿ-ಗಳನ್ನು
ಸಂನ್ಯಾಸಿ-ಗಳನ್ನೆಲ್ಲಾ
ಸಂನ್ಯಾಸಿ-ಗಳಲ್ಲಿ
ಸಂನ್ಯಾಸಿ-ಗಳಾ-ಗಲಿ
ಸಂನ್ಯಾಸಿ-ಗಳಾಗಲು
ಸಂನ್ಯಾಸಿ-ಗಳಾಗುವ
ಸಂನ್ಯಾಸಿ-ಗಳಾಗುವರು
ಸಂನ್ಯಾಸಿ-ಗಳಾಗು-ವುದಕ್ಕೆ
ಸಂನ್ಯಾಸಿ-ಗ-ಳಾದ
ಸಂನ್ಯಾಸಿ-ಗ-ಳಾದಾಗ
ಸಂನ್ಯಾಸಿ-ಗಳಿಂದ
ಸಂನ್ಯಾಸಿ-ಗಳಿ-ಗಿಂತಲೂ
ಸಂನ್ಯಾಸಿ-ಗಳಿಗೂ
ಸಂನ್ಯಾಸಿ-ಗಳಿಗೆ
ಸಂನ್ಯಾಸಿ-ಗಳಿರ-ಬ-ಹುದು
ಸಂನ್ಯಾಸಿ-ಗಳು
ಸಂನ್ಯಾಸಿ-ಗಳೂ
ಸಂನ್ಯಾಸಿ-ಗ-ಳೆಂದು
ಸಂನ್ಯಾಸಿ-ಗಳೆಲ್ಲಾ
ಸಂನ್ಯಾಸಿ-ಗಳೇ
ಸಂನ್ಯಾಸಿ-ಗಳೊ-ಡನೆ
ಸಂನ್ಯಾಸಿ-ಗೀತ
ಸಂನ್ಯಾಸಿ-ಗೀತೆ
ಸಂನ್ಯಾಸಿಗೂ
ಸಂನ್ಯಾಸಿಗೆ
ಸಂನ್ಯಾಸಿ-ನಿ-ಯರೇ
ಸಂನ್ಯಾಸಿಯ
ಸಂನ್ಯಾಸಿ-ಯಂತೆ
ಸಂನ್ಯಾಸಿ-ಯಂತೆಯೂ
ಸಂನ್ಯಾಸಿ-ಯಾ-ಗಲಿ
ಸಂನ್ಯಾಸಿ-ಯಾಗಿ
ಸಂನ್ಯಾಸಿ-ಯಾಗಿದ್ದ
ಸಂನ್ಯಾಸಿ-ಯಾಗಿದ್ದೇ-ನೆಯೊ
ಸಂನ್ಯಾಸಿ-ಯಾತ
ಸಂನ್ಯಾಸಿ-ಯಾದ
ಸಂನ್ಯಾಸಿಯು
ಸಂನ್ಯಾಸಿಯೇ
ಸಂನ್ಯಾಸಿ-ಯೊಬ್ಬನ
ಸಂನ್ಯಾಸಿ-ಯೋರ್ವನ
ಸಂನ್ಯಾಸಿ-ವೃಂದ
ಸಂನ್ಯಾಸಿ-ವೃಂದ-ದಿಂದಲೇ
ಸಂನ್ಯಾಸೀ
ಸಂಪತ್ತನ್ನು
ಸಂಪತ್ತಿದೆ
ಸಂಪತ್ತು
ಸಂಪತ್ತೆಲ್ಲಾ
ಸಂಪದ
ಸಂಪದಕೀ
ಸಂಪದದ
ಸಂಪದ-ವೀವ
ಸಂಪನ್ನ-ನಾಗಿ-ರದ
ಸಂಪನ್ನ-ರಾಗಿದ್ದರೆ
ಸಂಪನ್ನ-ರಾದ
ಸಂಪರ್ಕ
ಸಂಪರ್ಕ-ದಲ್ಲಿ-ರುವರೊ
ಸಂಪರ್ಕ-ದಿಂದ
ಸಂಪರ್ಕ-ದಿಂದಾಗಿ
ಸಂಪರ್ಕ-ವನ್ನಿಟ್ಟು-ಕೊಂಡಿದ್ದೆ
ಸಂಪರ್ಕ-ವನ್ನಿಟ್ಟು-ಕೊಂಡು
ಸಂಪರ್ಕ-ವನ್ನೂ
ಸಂಪರ್ಕ-ವಾದಂದಿ-ನಿಂದ
ಸಂಪರ್ಕವು
ಸಂಪರ್ಕವೂ
ಸಂಪರ್ಕವೇ
ಸಂಪಾದ-ಕ-ನನ್ನಾಗಿ
ಸಂಪಾದ-ಕ-ರಾದ
ಸಂಪಾದನೆ
ಸಂಪಾದ-ನೆ-ಗಳ
ಸಂಪಾದ-ನೆ-ಗಾಗಿ
ಸಂಪಾದ-ನೆಗೆ
ಸಂಪಾದ-ನೆಯ
ಸಂಪಾದ-ನೆ-ಯಾಗುತ್ತದೆ
ಸಂಪಾದ-ಯತ್ಯ-ವಿರತಂ
ಸಂಪಾದಿ-ಸ-ಬಹುದೊ
ಸಂಪಾದಿ-ಸ-ಬೇಕು
ಸಂಪಾದಿ-ಸ-ಲಾರದೆ
ಸಂಪಾದಿ-ಸಲಾ-ರರೆ
ಸಂಪಾದಿ-ಸಿ-ಕೊಂಡ
ಸಂಪಾದಿ-ಸಿ-ಕೊಳ್ಳುವಂತಹ
ಸಂಪಾದಿ-ಸಿದ
ಸಂಪಾದಿ-ಸುತ್ತಾರೆ
ಸಂಪಾದಿ-ಸುತ್ತಿದ್ದ
ಸಂಪಾದಿ-ಸುವ
ಸಂಪಾದಿ-ಸುವಂತಹ
ಸಂಪಾದಿ-ಸು-ವುದಕ್ಕೋಸ್ಕರ
ಸಂಪಾದಿ-ಸುವುದು
ಸಂಪುಟ
ಸಂಪುಟ-ಗಳನ್ನು
ಸಂಪುಟ-ಗಳಲ್ಲಿ
ಸಂಪುಟ-ಗಳು
ಸಂಪುಟ-ಗಳುಳ್ಳ
ಸಂಪುಟ-ವನ್ನು
ಸಂಪೂರ್ಣ
ಸಂಪೂರ್ಣ-ವಾಗಿ
ಸಂಪೂರ್ಣ-ವಾದ
ಸಂಪ್ರದಾಯ
ಸಂಪ್ರದಾಯ-ಗಳನ್ನು
ಸಂಪ್ರದಾಯ-ಗಳನ್ನೆಲ್ಲಾ
ಸಂಪ್ರದಾಯ-ಗಳನ್ನೊಳ-ಗೊಂಡ
ಸಂಪ್ರದಾಯ-ಗಳಲ್ಲಿ
ಸಂಪ್ರದಾಯ-ಗ-ಳಲ್ಲೇ
ಸಂಪ್ರದಾಯ-ಗಳಾಗಿ
ಸಂಪ್ರದಾಯ-ಗಳಿಂದ
ಸಂಪ್ರದಾಯ-ಗಳಿವೆ
ಸಂಪ್ರದಾಯ-ಗಳು
ಸಂಪ್ರದಾಯ-ಗಳೂ
ಸಂಪ್ರದಾಯ-ಗಳೇ
ಸಂಪ್ರದಾ-ಯದ
ಸಂಪ್ರದಾಯ-ದ-ವರು
ಸಂಪ್ರದಾಯ-ದಿಂದ
ಸಂಪ್ರದಾಯ-ಬದ್ಧ
ಸಂಪ್ರದಾಯ-ಬದ್ಧತೆ
ಸಂಪ್ರದಾಯ-ಬದ್ಧ-ವಾಗಿತ್ತು
ಸಂಪ್ರದಾಯ-ವನ್ನು
ಸಂಪ್ರದಾಯ-ವಾಗಿ
ಸಂಪ್ರದಾಯ-ವಾಗಿದೆ
ಸಂಪ್ರದಾಯ-ವಾದಿ-ಗಳಾಗಿ-ರುವುದ-ರಲ್ಲಿ
ಸಂಪ್ರದಾಯ-ವಾ-ದಿ-ಗಳು
ಸಂಪ್ರದಾಯ-ವಿ-ಹೀನತೆ
ಸಂಪ್ರದಾ-ಯವು
ಸಂಪ್ರದಾಯ-ವೊಂದಿತ್ತು
ಸಂಪ್ರದಾಯ-ಶೂನ್ಯ-ನಾ-ಗಿ-ರಲಿ
ಸಂಪ್ರದಾಯಸ್ಥ-ನಾ-ಗಿ-ರಲಿ
ಸಂಬಂಧ
ಸಂಬಂಧಕ್ಕೆ
ಸಂಬಂಧ-ಗಳ
ಸಂಬಂಧ-ಗಳನ್ನು
ಸಂಬಂಧ-ಗಳನ್ನೆಲ್ಲಾ
ಸಂಬಂಧದ
ಸಂಬಂಧ-ದಂತಿತ್ತು
ಸಂಬಂಧ-ದಲ್ಲಿ
ಸಂಬಂಧ-ದಲ್ಲಿಯೂ
ಸಂಬಂಧ-ದಿಂದ
ಸಂಬಂಧ-ಪಟ್ಟ
ಸಂಬಂಧ-ಪಟ್ಟಂತೆ
ಸಂಬಂಧ-ಪಟ್ಟಿ-ರು-ವು-ದಿಲ್ಲ
ಸಂಬಂಧ-ರೂಪ-ವಾದ
ಸಂಬಂಧ-ವನ್ನಿಟ್ಟು
ಸಂಬಂಧ-ವನ್ನು
ಸಂಬಂಧ-ವಾಗಿ
ಸಂಬಂಧ-ವಾದ
ಸಂಬಂಧ-ವಿದೆ
ಸಂಬಂಧ-ವಿ-ದೆಯೇ
ಸಂಬಂಧ-ವಿದ್ದಂತೆ
ಸಂಬಂಧ-ವಿ-ರುವ-ವರೆಗೆ
ಸಂಬಂಧವೂ
ಸಂಬಂಧಾದಿ-ಗಳನ್ನು
ಸಂಬಂಧಿ-ಸಿದ
ಸಂಬಂಧಿಸಿ-ದುದು
ಸಂಬಲ
ಸಂಬಳಕ್ಕಾಗಿ
ಸಂಬಳ-ವನ್ನು
ಸಂಬಳ-ವಿ-ರುವ
ಸಂಬೋಧಿಸಿ
ಸಂಭವ
ಸಂಭ-ವ-ವಿದೆ
ಸಂಭ-ವ-ವಿಲ್ಲ
ಸಂಭ-ವ-ವುಂಟೆ
ಸಂಭ-ವ-ವೆಂದು
ಸಂಭವಿಸಿದ್ದನ್ನು
ಸಂಭ-ವಿ-ಸು-ವು-ದನ್ನು
ಸಂಭ-ವಿ-ಸು-ವು-ದಿಲ್ಲ
ಸಂಭಾ-ವಯಂತ್ಯ-ವಿ-ಕೃತಂ
ಸಂಭಾವಿತ
ಸಂಭಾಷಣ-ಗಳಿಂದ
ಸಂಭಾಷಣೆ
ಸಂಭಾಷ-ಣೆಯ
ಸಂಭಾಷಣೆ-ಯನ್ನು
ಸಂಭಾಷಣೆ-ಯಲ್ಲಿ
ಸಂಭಾಷಣೆ-ಯಾಗುತ್ತಿರು-ವಾಗ
ಸಂಭಾಷಣೆ-ಯಾದ
ಸಂಭಾಷ-ಣೆಯು
ಸಂಭಾಷಿ-ಸುತ್ತಿದ್ದನು
ಸಂಭ್ರ-ಮದಿ
ಸಂಮುಖೆ
ಸಂಯಮ
ಸಂಯಮ-ವುಳ್ಳ-ವ-ನಾಗಿದ್ದು
ಸಂಯಮಿ-ಗಳಲ್ಲವೊ
ಸಂಯಿಯಾ
ಸಂಯೋಗ
ಸಂಯೋಗ-ಗಳು
ಸಂಯೋಗ-ವಾಗಿದೆ
ಸಂಯೋ-ಜನಾತ್ಮಕ-ವಾ-ದುದು
ಸಂಯೋ-ಜನೆ-ಗಳನ್ನು
ಸಂಯೋ-ಜನೆ-ಯನ್ನು
ಸಂರಕ್ಷಣೆ
ಸಂರಕ್ಷಣೆ-ಯಲ್ಲಿ
ಸಂರಕ್ಷಣೆ-ಯಲ್ಲಿ-ರ-ಬೇಕೆಂದು
ಸಂರಕ್ಷಿಸು-ವುದೆಂದಾಗುತ್ತದೋ
ಸಂವತ್ಸ-ರದ
ಸಂವಿ-ಭಾತಿ
ಸಂಶಯ
ಸಂಶಯ-ಗಳೂ
ಸಂಶಯ-ಗಳೆಲ್ಲ
ಸಂಶಯ-ರಾಕ್ಷಸ-ನಾಶ-ಮ-ಹಾಸ್ತ್ರಂ
ಸಂಶಯ-ವನ್ನು
ಸಂಶಯವು
ಸಂಶಯವೇ
ಸಂಶೋಧ-ನೆ-ಗಳನ್ನು
ಸಂಸರ್ಗ-ದಿಂದ
ಸಂಸಾರ
ಸಂಸಾ-ರಕ್ಕೆ
ಸಂಸಾರ-ತಾ-ಪತ್ರ-ಯಕ್ಕೆ
ಸಂಸಾರದ
ಸಂಸಾರ-ದ-ಲೆ-ಗಳವು
ಸಂಸಾರ-ದಲ್ಲಿ
ಸಂಸಾರ-ದಲ್ಲೇ
ಸಂಸಾರ-ದ-ವರ
ಸಂಸಾರ-ದ-ವರೊಂದಿ-ಗಿಂತ
ಸಂಸಾರ-ದಿಂದ
ಸಂಸಾರ-ನಾಶ-ಕ-ವಾದ
ಸಂಸಾರ-ಮಾಯೆ-ಯನು
ಸಂಸಾರ-ವನ್ನು
ಸಂಸಾರ-ಸಾಗರದ
ಸಂಸಾರ-ಸಾಗರವೆ
ಸಂಸಾರಾ
ಸಂಸಾರಾಶ್ರಮ-ದಲ್ಲಿ
ಸಂಸಾರಾಶ್ರಮ-ವನ್ನು
ಸಂಸಾರಿ-ಕರು
ಸಂಸಾರಿ-ಗರು
ಸಂಸಾರಿ-ಗಳ
ಸಂಸಾರಿ-ಗಳಾಗ-ಬ-ಹುದು
ಸಂಸಾರಿ-ಗ-ಳಾದ
ಸಂಸಾರಿ-ಗಳಿಗೆ
ಸಂಸಾರಿ-ಗಳು
ಸಂಸಾರಿ-ಯಾಗಿ
ಸಂಸಾರಿ-ಯಾಗಿ-ರುವಿಕೆ
ಸಂಸಾರೇ
ಸಂಸಾರ್
ಸಂಸ್ಕರ-ಣ-ವನ್ನು
ಸಂಸ್ಕಾರ
ಸಂಸ್ಕಾ-ರಕ್ಕೆ
ಸಂಸ್ಕಾರ-ಗಳ
ಸಂಸ್ಕಾರ-ಗಳನ್ನು
ಸಂಸ್ಕಾರ-ಗಳನ್ನೂ
ಸಂಸ್ಕಾರ-ಗಳು
ಸಂಸ್ಕಾರ-ಗೊಳಿಸಿದೆ
ಸಂಸ್ಕಾರದ
ಸಂಸ್ಕಾರ-ದಿಂದ
ಸಂಸ್ಕಾರ-ವನ್ನೇ
ಸಂಸ್ಕಾರ-ವಿಲ್ಲ-ದಿದ್ದರೆ
ಸಂಸ್ಕಾರವೆ
ಸಂಸ್ಕೃತ
ಸಂಸ್ಕೃ-ತದ
ಸಂಸ್ಕೃತ-ದಲ್ಲಿ
ಸಂಸ್ಕೃತ-ದಲ್ಲಿಯೆ
ಸಂಸ್ಕೃತ-ದಲ್ಲಿಯೇ
ಸಂಸ್ಕೃತ-ವನ್ನು
ಸಂಸ್ಕೃ-ತವೇ
ಸಂಸ್ಕೃತ-ವೊಂದೇ
ಸಂಸ್ಕೃತಾಶ್ಚ
ಸಂಸ್ಕೃತಿ
ಸಂಸ್ಕೃ-ತಿಗೆ
ಸಂಸ್ಕೃ-ತಿಯ
ಸಂಸ್ಕೃತಿ-ಯನ್ನು
ಸಂಸ್ಕೃ-ತಿಯೇ
ಸಂಸ್ಥಾ
ಸಂಸ್ಥೆ
ಸಂಸ್ಥೆ-ಗಳ
ಸಂಸ್ಥೆ-ಗಳನ್ನು
ಸಂಸ್ಥೆ-ಗಳಲ್ಲೆಲ್ಲಾ
ಸಂಸ್ಥೆ-ಗ-ಳಾದ
ಸಂಸ್ಥೆ-ಗಳಿವೆ
ಸಂಸ್ಥೆ-ಗಳು
ಸಂಸ್ಥೆ-ಗಳೆಲ್ಲ
ಸಂಸ್ಥೆಯ
ಸಂಸ್ಥೆ-ಯನ್ನು
ಸಂಸ್ಥೆ-ಯಲ್ಲಿ
ಸಂಸ್ಥೆ-ಯಲ್ಲಿದ್ದರೂ
ಸಂಹಾರ-ಮಾಡಿ
ಸಂಹಿತಾ
ಸಂಹಿತಾ-ಕಾ-ರರು
ಸಂಹಿತೆ
ಸಂಹಿತೆ-ಗಳು
ಸಂಹಿತೆ-ಯನ್ನು
ಸಃ
ಸಕರುಣಂ
ಸಕರುಣಾ
ಸಕಲ
ಸಕಲ-ಕಲಹಪ್ರಾಪಿಣೀಂ
ಸಕಲ-ಕಾಲ-ದಲ್ಲಿಯೂ
ಸಕಲ-ಭು-ವನಂ
ಸಕಲ-ಮ-ತದ
ಸಕಲರೂ
ಸಕಲ-ವನ್ನೂ
ಸಕಲಸ್ಥೂಲ
ಸಕಾರ-ಣ-ವಾಗಿ
ಸಕಾಲ
ಸಕಾಲ-ದಲ್ಲಿ
ಸಕ್ಕರೆ
ಸಕ್ಕರೆ-ಯನ್ನು
ಸಖ
ಸಖನೆ
ಸಖನೇ
ಸಖ-ರಿಲ್ಲ
ಸಖಾರ
ಸಖಾರ್
ಸಖೆ
ಸಖ್ಯ
ಸಗಟು
ಸಗ್ಗ-ವನೂ
ಸಗ್ಗಿಗ-ರದೊ
ಸಚರಾಚವ
ಸಚಲ
ಸಚೇ-ತನ-ವಾಗಿ
ಸಚೇ-ತನ-ವಾಗಿದ್ದುವು
ಸಚ್ಚರಿತ್ರೆ-ಯರೂ
ಸಚ್ಚಿ-ದಾನಂದ
ಸಚ್ಚಿ-ದಾನಂದದ
ಸಚ್ಚಿ-ದಾನಂದ-ವನ್ನು
ಸಚ್ಚಿ-ದಾನಂದವು
ಸಚ್ಚಿ-ದಾನಂದಸ್ವ-ರೂಪ
ಸಚ್ಚಿ-ದಾನಂದಾತ್ಮ
ಸಜಲ
ಸಜೀವ
ಸಜೀವ-ಗಳಾಗಿ
ಸಜೀವ-ವಾಗಿ
ಸಜ್ಜ-ನನು
ಸಜ್ಜನರು
ಸಜ್ಜನಿಕೆ-ಯನ್ನು
ಸಜ್ಜು-ಗೊಳಿಸಿ-ಹುದು
ಸಡಗರ-ದಿಂದ
ಸಡಗರ-ಪಡುತ್ತಿದ್ದರು
ಸಡಿಲ
ಸಡಿಲ-ವಾಗಿ
ಸಡಿಲ-ವಾಗು-ವುದು
ಸಡಿಲಿಸ-ದಂತೆ
ಸಣ್ಣ
ಸಣ್ಣ-ದಿದು
ಸಣ್ಣ-ಪುಟ್ಟ
ಸಣ್ಣ-ವ-ರನ್ನಾಗಿ
ಸಣ್ಣ-ವ-ರೆಂಬ
ಸತತ
ಸತತಂ
ಸತತ-ವಾಗಿ
ಸತತ-ವಾದ
ಸತ-ತವು
ಸತ-ತವೂ
ಸತಿ
ಸತಿ-ಗಾಗಿ
ಸತಿ-ಯಂತೆ
ಸತಿ-ಯನ್ನು
ಸತಿ-ಯಲ್ಲಿ-ರುವ
ಸತಿಯು
ಸತಿ-ಸುತ-ರಿ-ಗಾಗಿ
ಸತೀತ್ವದ
ಸತ್
ಸತ್ಕಥೆ-ಗಳಲ್ಲಿ
ಸತ್ಕರಿ-ಸು-ವುದು
ಸತ್ಕರ್ಮ
ಸತ್ಕರ್ಮ-ವನ್ನು
ಸತ್ಕರ್ಮವೂ
ಸತ್ಕರ್ಮಿ
ಸತ್ಕಾರ
ಸತ್ಕಾರ್ಯ
ಸತ್ಕಾರ್ಯಕ್ಕಾಗಿ
ಸತ್ಕಾರ್ಯಕ್ಕೆ
ಸತ್ಕಾರ್ಯಕ್ಕೋಸ್ಕರ
ಸತ್ಕಾರ್ಯ-ಗಳ
ಸತ್ಕಾರ್ಯ-ಗಳನ್ನು
ಸತ್ಕಾರ್ಯ-ಗಳು
ಸತ್ಕಾರ್ಯ-ಗಳೆಲ್ಲಾ
ಸತ್ಕಾರ್ಯ-ನಿರತ-ನಾಗು
ಸತ್ಕಾರ್ಯ-ವನ್ನು
ಸತ್ಕಾರ್ಯವೇ
ಸತ್ಕಾಲ-ದಲ್ಲಿ
ಸತ್ಚಿತ್ಆ-ನಂದ
ಸತ್ತ
ಸತ್ತಂತಾಗು-ವುದು
ಸತ್ತಂತೆಯೇ
ಸತ್ತ-ನೆಂದು
ಸತ್ತಪ್ರಾಣಿ-ಗಳ
ಸತ್ತ-ಮೇಲೆ
ಸತ್ತರೂ
ಸತ್ತರೆ
ಸತ್ತ-ರೇ-ನಂತೆ
ಸತ್ತ-ವರಂತಾದರೋ
ಸತ್ತ-ಹಾಗೆ
ಸತ್ತಾ-ದರೂ
ಸತ್ತಾ-ಬಲ-ದಲ್ಲಿ
ಸತ್ತಿ-ರ-ಲಿಲ್ಲ
ಸತ್ತು
ಸತ್ತು-ಹೋಗಿದ್ದಂತೆ
ಸತ್ತು-ಹೋಗುತ್ತಾರೆ
ಸತ್ತು-ಹೋದ
ಸತ್ತು-ಹೋದನು
ಸತ್ತು-ಹೋದರೆ
ಸತ್ತು-ಹೋದ-ರೆಂದು
ಸತ್ತೂ
ಸತ್ತೆ
ಸತ್ತೆ-ಯನ್ನು
ಸತ್ತೆ-ಯಾದ
ಸತ್ತೆಯೂ
ಸತ್ತೇ
ಸತ್ತ್ವ
ಸತ್ತ್ವ-ಗುಣ
ಸತ್ತ್ವ-ಗು-ಣಕ್ಕೆ
ಸತ್ತ್ವ-ಗುಣದ
ಸತ್ತ್ವ-ಗುಣವೂ
ಸತ್ತ್ವ-ಗು-ಣಿ-ಗಳು
ಸತ್ತ್ವ-ಗುಣಿ-ಗ-ಳೆಂದು
ಸತ್ತ್ವ-ದಿಂದ
ಸತ್ತ್ವ-ಪೂರ್ಣ-ವಾಗಿದೆ
ಸತ್ತ್ವ-ರಜಸ್ಸು-ತಮಸ್ಸು-ಗಳೆಂಬ
ಸತ್ತ್ವ-ವನ್ನು
ಸತ್ತ್ವ-ವಾಗಿ
ಸತ್ತ್ವವು
ಸತ್ತ್ವ-ವುಳ್ಳ
ಸತ್ತ್ವ-ಶಾಲಿ-ಯಾದಾಗ
ಸತ್ತ್ವ-ಹೀನ
ಸತ್ತ್ವ-ಹೀನ-ವಾ-ದುವು
ಸತ್ಯ
ಸತ್ಯ-ಕಾಮ
ಸತ್ಯ-ಕಾಮನ
ಸತ್ಯಕ್ಕಾಗಿ
ಸತ್ಯಕ್ಕಿ-ರುವ
ಸತ್ಯಕ್ಕೆ
ಸತ್ಯ-ಗಳ
ಸತ್ಯ-ಗಳನ್ನು
ಸತ್ಯ-ಗಳನ್ನೂ
ಸತ್ಯ-ಗಳಿವೆ
ಸತ್ಯ-ಗಳು
ಸತ್ಯ-ಗಳೂ
ಸತ್ಯ-ತುಮಿ
ಸತ್ಯ-ತೆ-ಯನ್ನು
ಸತ್ಯದ
ಸತ್ಯ-ದಂತೆ
ಸತ್ಯ-ದಲಿ
ಸತ್ಯ-ದಲ್ಲಿ
ಸತ್ಯ-ದಿಂದ
ಸತ್ಯ-ದು-ಸಿರದು
ಸತ್ಯ-ದೆಡೆಗೆ
ಸತ್ಯ-ನಿಷ್ಠೆ-ಯನ್ನು
ಸತ್ಯ-ಪಾರ-ಮಾರ್ಥಿಕ
ಸತ್ಯ-ಭಾಮೆಯ
ಸತ್ಯ-ಭಾವ
ಸತ್ಯ-ಭಾವವೇ
ಸತ್ಯವ
ಸತ್ಯ-ವ-ನ-ವರು
ಸತ್ಯ-ವನ್ನ-ರಿತು
ಸತ್ಯ-ವನ್ನಾದರೂ
ಸತ್ಯ-ವನ್ನು
ಸತ್ಯ-ವನ್ನೂ
ಸತ್ಯ-ವನ್ನೇ
ಸತ್ಯ-ವಲ್ಲ
ಸತ್ಯ-ವಲ್ಲ-ದಿದ್ದರೂ
ಸತ್ಯ-ವಲ್ಲದೇ
ಸತ್ಯ-ವಸ್ತುವು
ಸತ್ಯ-ವಾಗಿ
ಸತ್ಯ-ವಾಗಿ-ದೆಯೋ
ಸತ್ಯ-ವಾಗಿದ್ದರೆ
ಸತ್ಯ-ವಾಗಿಯೂ
ಸತ್ಯ-ವಾಗಿ-ರ-ಬ-ಹುದು
ಸತ್ಯ-ವಾಗಿ-ರ-ಲಾ-ರವು
ಸತ್ಯ-ವಾಗಿ-ರುವ-ವನೂ
ಸತ್ಯ-ವಾಗಿ-ರು-ವುದು
ಸತ್ಯ-ವಾದ
ಸತ್ಯ-ವಾದರೂ
ಸತ್ಯ-ವಾದರೆ
ಸತ್ಯ-ವಾದುದ್ದಲ್ಲವೆ
ಸತ್ಯ-ವಿದು
ಸತ್ಯ-ವಿದೆ
ಸತ್ಯ-ವಿರ-ಬ-ಹುದು
ಸತ್ಯವು
ಸತ್ಯವೂ
ಸತ್ಯವೆ
ಸತ್ಯ-ವೆಂದು
ಸತ್ಯ-ವೆಂಬಂತೆ
ಸತ್ಯ-ವೆಂಬು-ದರ
ಸತ್ಯ-ವೆಂಬುದು
ಸತ್ಯ-ವೆಂಬುವು-ದಲ್ಲಿ
ಸತ್ಯವೇ
ಸತ್ಯ-ವೊಂದೇ
ಸತ್ಯ-ಸಂಕಲ್ಪ
ಸತ್ಯ-ಸಂಕಲ್ಪಾವಸ್ಥೆ-ಯುಂಟಾ-ದರೂ
ಸತ್ಯ-ಸಂಗತಿ-ಯನ್ನರಿ-ಯರು
ಸತ್ಯ-ಸಾಕ್ಷಾತ್ಕಾರ
ಸತ್ಯ-ಸಾಕ್ಷಾತ್ಕಾರ-ವಾಗಲಿ
ಸತ್ಯ-ಸಾಕ್ಷಾತ್ಕಾರ-ವಾಗುವ
ಸತ್ಯ-ಸಾರ
ಸತ್ಯಸ್ವ-ರೂಪನು
ಸತ್ಯ-ಹೀನ
ಸತ್ಯಾಂಶ
ಸತ್ಯಾಂಶ-ವನ್ನು
ಸತ್ಯಾ-ನು-ಭವ
ಸತ್ಯಾನ್ವೇಷಣೆ
ಸತ್ಯಾನ್ವೇಷಣೆ-ಯಲ್ಲಿನ
ಸತ್ಯೋಽಯಂ
ಸತ್ರ
ಸತ್ರ-ದಲ್ಲಿ
ಸತ್ಸಾಹ-ಸಿಯೂ
ಸದ-ರ-ದಿಂದ
ಸದ-ರ-ವಾಗಿ
ಸದ-ವಕಾಶ
ಸದ-ಸದ್ವಿ-ಚಾರ-ವಂತರೊ
ಸದ-ಸದ್ವಿ-ಚಾರ-ವನ್ನು
ಸದಸ್ಯ-ರನ್ನು
ಸದಸ್ಯರು
ಸದಾ
ಸದಾ-ಚಾರ
ಸದಾನಂದ
ಸದಾ-ಪ-ರಾಜಯ
ಸದೃಢ-ವಾದ
ಸದೆ-ಬ-ಡಿದರೆ
ಸದೆ-ಬಡಿ-ಯಿರಿ
ಸದೈವ
ಸದೋಷ-ಮಪಿ
ಸದ್
ಸದ್ಗುಣ
ಸದ್ಗುಣ-ಗಳ
ಸದ್ಗುಣ-ಗಳನ್ನಿಟ್ಟು
ಸದ್ಗುಣ-ಗಳನ್ನು
ಸದ್ಗುಣ-ವೆನ್ನಿ-ಸಿ-ಕೊಳ್ಳುವುದು
ಸದ್ಗುಣ-ಸಂಪನ್ನ-ರಾದ
ಸದ್ಗ್ರಂಥವೂ
ಸದ್ದಿಲ್ಲದೆ
ಸದ್ದು
ಸದ್ದು-ಗದ್ದಲ-ವಡಗಿ
ಸದ್ದು-ಗದ್ದಲ-ವಿಲ್ಲದೆ
ಸದ್ದೇ
ಸದ್ಯಕ್ಕಂತೂ
ಸದ್ಯಕ್ಕೆ
ಸದ್ಯದ
ಸದ್ಯ-ದಲ್ಲಿ
ಸದ್ಯ-ದಲ್ಲೆ
ಸದ್ಯುಕ್ತಿ
ಸದ್ವಿನಿ-ಯೋಗ-ವಾಗುತ್ತದೆ
ಸದ್ವ್ಯವ-ಹಾರ
ಸನಾ-ತನ
ಸನಾ-ತನ-ವಾದ
ಸನಾ-ತನಿ-ಗಳ
ಸನಿಹ
ಸನಿಹ-ಗ-ಳಲ್ಲು
ಸನಿಹ-ದಲ್ಲಿಯೂ
ಸನೆ
ಸನ್ನಾಪ್ಯಸನ್ನಾಪ್ಯು-ಭ-ಯಾತ್ಮಿಕಾನೋ
ಸನ್ನಿಧಿ-ಯಲ್ಲಿ
ಸನ್ನಿ-ವೇಶ
ಸನ್ನಿ-ವೇಶ-ವನ್ನು
ಸನ್ನಿ-ವೇಶ-ವಾದರೂ
ಸನ್ನಿ-ಹಿತ-ವಾಗಿದೆ
ಸನ್ನಿ-ಹಿತ-ವಾಗಿ-ರು-ವುದು
ಸನ್ಮಾನ
ಸನ್ಮಾ-ನಕ್ಕೆ
ಸನ್ಮಾನದ
ಸನ್ಮಾನ-ವನ್ನೂ
ಸನ್ಮಾರ್ಗಾವಲಂಬಿ-ಗ-ಳಾದರೆ
ಸನ್ಯಾಲ
ಸಪ್ಟೆಂಬರ್
ಸಪ್ತ-ಧಾತು-ಗ-ಳಲ್ಲ
ಸಪ್ತ-ಭು-ವನ
ಸಪ್ತ-ವಾಗಿ
ಸಪ್ತಸ್ವರ-ಗಳಿಂದ
ಸಪ್ಪೆ
ಸಪ್ಪೆ-ತನ
ಸಪ್ಪೆ-ಯಾಗಿ
ಸಪ್ರಮಾಣ-ವಾಗಿ
ಸಫಲ
ಸಫಲ-ಮಾಡಿ
ಸಫಲ-ವಾಗು-ವುದಕ್ಕೋಸ್ಕರ
ಸಫಲ-ವಾಗು-ವುದೋ
ಸಫಲ-ವಾದದ್ದನ್ನು
ಸಫಲ-ವಾದರೆ
ಸಫಲೇ-ಫಲೇ
ಸಬ
ಸಬರೇ
ಸಬಾಈ
ಸಬಾಯಿ
ಸಬಾರ
ಸಬ್
ಸಭಾ
ಸಭಿಕ-ರಲ್ಲೊಬ್ಬರು
ಸಭಿ-ಕರೆಲ್ಲರೂ
ಸಭೆ
ಸಭೆ-ಗಳಿಗೆ
ಸಭೆ-ಗಾಗಿ
ಸಭೆಗೆ
ಸಭೆಯ
ಸಭೆ-ಯನ್ನು
ಸಭೆ-ಯಲ್ಲ
ಸಭೆ-ಯಲ್ಲಿ
ಸಭೆಯು
ಸಭೆ-ಯೊಂದ-ರಲ್ಲಿ
ಸಭ್ಯ
ಸಭ್ಯ-ಗೃಹಸ್ಥ
ಸಭ್ಯ-ಗೃಹಸ್ಥನು
ಸಭ್ಯತೆ
ಸಭ್ಯ-ತೆಗೇ
ಸಭ್ಯ-ತೆ-ಯನ್ನು
ಸಭ್ಯ-ರಂತಿದ್ದರು
ಸಭ್ಯ-ರಂತೆ
ಸಭ್ಯ-ರಾದಾಗ
ಸಭ್ಯ-ರೊಡನೆ
ಸಮ
ಸಮಂಜಸ-ವಾಗಿ-ರು-ವುದೋ
ಸಮ-ಕಾಲೀನ
ಸಮ-ಕಾಲೀನ-ನಾಗಿದ್ದನು
ಸಮಕ್ಕೆ
ಸಮಕ್ಷಮ-ದಲ್ಲಿ
ಸಮಕ್ಷೇತ್ರೆ
ಸಮಗ್ರ
ಸಮಗ್ರ-ದೇಶ-ವೆಲ್ಲಾ
ಸಮತಾ-ವಾದ
ಸಮತಾ-ವಾದ-ದಿಂದ
ಸಮದ-ರಶನ
ಸಮದರ್ಶಿತ್ವದ
ಸಮ-ದೃಷ್ಟಿ-ಯಿಂದ
ಸಮ-ದೃಷ್ಟಿಯೆ
ಸಮ-ನಾಗಿ
ಸಮ-ನಾಗಿದ್ದ
ಸಮ-ನಾಗಿ-ರುವ
ಸಮ-ನಾಗಿ-ರು-ವು-ದಿಲ್ಲ
ಸಮನ್ವಯ
ಸಮನ್ವ-ಯಕ್ಕೆ
ಸಮನ್ವಯತೆ
ಸಮನ್ವ-ಯದ
ಸಮನ್ವಯ-ವನ್ನು
ಸಮನ್ವ-ಯವು
ಸಮನ್ವಯಾ-ಚಾರ್ಯ-ರಾದ
ಸಮನ್ವಯಾ-ಚಾರ್ಯರು
ಸಮ-ಭಾವ-ದಲ್ಲಿ
ಸಮಯ
ಸಮಯ-ಕೊಥಾ
ಸಮ-ಯಕ್ಕೆ
ಸಮಯ-ಗಳಲ್ಲಿ
ಸಮಯ-ದಲ್ಲಿ
ಸಮಯ-ದಲ್ಲಿಯೂ
ಸಮಯ-ದಲ್ಲಿಯೇ
ಸಮಯ-ದಲ್ಲೇ
ಸಮ-ಯದಿ
ಸಮ-ಯ-ದಿಂದಲೇ
ಸಮಯ-ವಿಲ್ಲ
ಸಮ-ಯ-ವೆಲ್ಲಾ
ಸಮ-ರಸ
ಸಮ-ರಸ-ವಾಗಿ-ರು-ವುದು
ಸಮ-ರಾಂಗ-ಣದ
ಸಮರ್ಥ
ಸಮರ್ಥ-ನಲ್ಲ
ಸಮರ್ಥ-ನಾಗುತ್ತಿದ್ದನು
ಸಮರ್ಥ-ನಾಗುವೆ
ಸಮರ್ಥ-ನಾಗು-ವೆಯೋ
ಸಮರ್ಥ-ನಾದ
ಸಮರ್ಥ-ನಾದರೆ
ಸಮರ್ಥ-ನಾ-ದಾನು
ಸಮರ್ಥ-ನಾದ್ದ-ರಿಂದ
ಸಮರ್ಥನೆ
ಸಮರ್ಥನೊ
ಸಮರ್ಥ-ರನ್ನಾಗಿ
ಸಮರ್ಥ-ರಾಗ-ಬಲ್ಲ-ರೆಂದು
ಸಮರ್ಥ-ರಾಗಿದ್ದಾರೆ
ಸಮರ್ಥ-ರಾ-ಗಿ-ರುವರೇ
ಸಮರ್ಥ-ರಾ-ಗಿ-ರುವರೋ
ಸಮರ್ಥ-ರಾಗುತ್ತಾರೆ
ಸಮರ್ಥ-ರಾಗುವರು
ಸಮರ್ಥ-ರಾದ
ಸಮರ್ಥ-ರಾದರೆ
ಸಮರ್ಥರು
ಸಮರ್ಥಿಸ-ಹೋಗಿ
ಸಮರ್ಥಿಸಿ-ಕೊಂಡಿದ್ದರು
ಸಮರ್ಥಿಸಿ-ಕೊಳ್ಳ-ಬ-ಹುದು
ಸಮರ್ಥಿಸಿ-ದಿರಿ
ಸಮರ್ಥಿ-ಸುತ್ತಿ-ರು-ವಿರಿ
ಸಮರ್ಥಿ-ಸುವ
ಸಮರ್ಥಿ-ಸು-ವುದಕ್ಕೆ
ಸಮರ್ಪಕ-ವಾಗಿಲ್ಲ
ಸಮರ್ಪಕ-ವಾ-ಯಿತೆಂದು
ಸಮರ್ಪಣೆಯ
ಸಮರ್ಪಿಸ-ಬ-ಹುದು
ಸಮರ್ಪಿ-ಸಿ-ದನು
ಸಮ-ವಾಗಿ
ಸಮ-ವಾಗಿ-ರ-ಲಿಲ್ಲ
ಸಮ-ವುಂಟೇ
ಸಮಶ್ರುತಿಯ
ಸಮಷ್ಟಿ
ಸಮಷ್ಟಿಪ್ರಕಾಶ-ವೆಂದು
ಸಮಷ್ಟಿಯ
ಸಮಷ್ಟಿ-ಯಂತೆ
ಸಮಷ್ಟಿ-ಯನ್ನು
ಸಮಷ್ಟಿ-ಯಲ್ಲಿ
ಸಮಷ್ಟಿ-ಯಷ್ಟೆ
ಸಮಷ್ಟಿ-ಯಿಂದ
ಸಮಷ್ಟಿಯೂ
ಸಮಸ್ತ
ಸಮಸ್ತ-ವನ್ನೂ
ಸಮಸ್ಯೆ
ಸಮಸ್ಯೆ-ಗಳನ್ನು
ಸಮಸ್ಯೆ-ಗಳನ್ನೂ
ಸಮಸ್ಯೆ-ಗಳೆಲ್ಲಾ
ಸಮಸ್ಯೆಗೆ
ಸಮಸ್ಯೆಯ
ಸಮಸ್ಯೆ-ಯನ್ನು
ಸಮಸ್ವರ-ದಲ್ಲಿ
ಸಮಾಗಮ
ಸಮಾ-ಚಾರ
ಸಮಾ-ಚಾರ-ವನ್ನು
ಸಮಾಜ
ಸಮಾಜಕ್ಕನು-ಗುಣ-ವಾಗಿ
ಸಮಾಜಕ್ಕೆ
ಸಮಾಜ-ತಂತ್ರ-ದಲ್ಲಿ
ಸಮಾಜದ
ಸಮಾಜ-ದಂಡ-ನೆಗೆ
ಸಮಾಜ-ದಲ್ಲಾಗಲೀ
ಸಮಾಜ-ದಲ್ಲಿ
ಸಮಾಜ-ದಲ್ಲಿದ್ದರೆ
ಸಮಾಜ-ದಲ್ಲಿ-ರಲು
ಸಮಾಜ-ದಲ್ಲೂ
ಸಮಾಜ-ದಿಂದ
ಸಮಾಜ-ದೊಳಗಿಂದಲೇ
ಸಮಾಜಪ್ರ-ಗತಿ
ಸಮಾಜ-ವನ್ನು
ಸಮಾಜ-ವನ್ನೇ
ಸಮಾಜ-ವಾದ-ವೆಂದು
ಸಮಾಜ-ವಾ-ದಿ-ಗಳು
ಸಮಾಜವು
ಸಮಾಜವೂ
ಸಮಾಜ-ವೆಲ್ಲಾ
ಸಮಾಜವೇ
ಸಮಾಜ-ವೇ-ನಾದರೂ
ಸಮಾಧಾನ
ಸಮಾಧಾನ-ಗೊಂಡನು
ಸಮಾಧಾನ-ಪಡಿ-ಸುವ-ವರೆಗೆ
ಸಮಾಧಾನ-ಪಡಿ-ಸು-ವು-ದಿಲ್ಲ
ಸಮಾಧಾನ-ವಾ-ಯಿತು
ಸಮಾಧಿ
ಸಮಾಧಿ-ಕಾಲ-ದಲ್ಲಿಯೆ
ಸಮಾಧಿ-ಗಳ
ಸಮಾಧಿ-ಗಳನ್ನೂ
ಸಮಾಧಿಗೆ
ಸಮಾಧಿ-ಮನು-ತಿಷ್ಠಸಿ
ಸಮಾಧಿಯ
ಸಮಾಧಿ-ಯನ್ನು
ಸಮಾಧಿ-ಯಲ್ಲ
ಸಮಾಧಿ-ಯಲ್ಲಿ
ಸಮಾಧಿ-ಯಾದರೆ
ಸಮಾಧಿ-ಯಾದಾಗ
ಸಮಾಧಿ-ಯಿಂದ
ಸಮಾಧಿ-ಯುಂಟಾ-ಗುತ್ತದೆ
ಸಮಾಧಿ-ಯೆಂದು
ಸಮಾಧಿಸ್ಥ-ನಾಗಲು
ಸಮಾಧಿಸ್ಥ-ರಾದರು
ಸಮಾಧಿಸ್ಥಿತಿಗೆ
ಸಮಾಧಿಸ್ಥಿತಿ-ಯನ್ನು
ಸಮಾಧಿಸ್ಥಿತಿ-ಯಲ್ಲಿ
ಸಮಾಧಿಸ್ಥಿತಿ-ಯಿಂದ
ಸಮಾನ
ಸಮಾನ-ದೃಷ್ಟಿ-ಯನ್ನು
ಸಮಾನ-ರಲ್ಲ
ಸಮಾನ-ರಾದ
ಸಮಾನರು
ಸಮಾನ-ವಾಗಿ
ಸಮಾನ-ವಾಗಿಯೇ
ಸಮಾನ-ವಾದ
ಸಮಾನ-ವಾ-ದದ್ದು
ಸಮಾನ-ವೆಂದು
ಸಮಾನಾಂತರ
ಸಮಾ-ರಂಭಕ್ಕಾಗಿ
ಸಮಾರು
ಸಮಾಲೋಚನೆ
ಸಮಾಲೋಚ-ನೆಯು
ಸಮಾಲೋಚಿ-ಸಿ-ದರು
ಸಮಾವಿಶತು
ಸಮಾ-ಹಿತ
ಸಮಿತಿ
ಸಮಿತಿ-ಗಳು
ಸಮಿ-ತಿಯ
ಸಮಿತ್ತನ್ನು
ಸಮೀಪಕ್ಕೆ
ಸಮೀ-ಪದಲ್ಲಿತ್ತು
ಸಮೀಪ-ದಲ್ಲಿ-ರುವ
ಸಮೀಪ-ದಲ್ಲಿ-ರುವ-ವ-ರಿಗೆ
ಸಮೀಪ-ದಲ್ಲಿ-ರುವಿರಿ
ಸಮೀಪ-ದಲ್ಲೇ
ಸಮೀಪಿಸಿ-ದೊಡ-ನೆಯೆ
ಸಮೀಪಿ-ಸುತ್ತಿತ್ತು
ಸಮೀಪಿ-ಸುತ್ತಿದೆ
ಸಮುದಾಯ
ಸಮುದಾ-ಯಕ್ಕೆ
ಸಮುದಾ-ಯದ
ಸಮುದಾ-ಯವೂ
ಸಮುದ್ರದ
ಸಮುದ್ರದಲ್ಲಿ
ಸಮುದ್ರದಾಚೆ-ಯಿಂದ
ಸಮುದ್ರದೊಳೆದ್ದ
ಸಮುದ್ರಯಾನ-ವನ್ನು
ಸಮುದ್ರ-ವನ್ನು
ಸಮುದ್ರ-ವನ್ನೇ
ಸಮುದ್ರವು
ಸಮುದ್ರ-ವೆಂದರೂ
ಸಮೂಹಕ್ಕೆಲ್ಲ
ಸಮೂಹದ
ಸಮೂಹ-ದೊ-ಡನೆ
ಸಮೂಹ-ವನ್ನೇ
ಸಮೂಹವೇ
ಸಮೃದ್ಧಿ
ಸಮ್
ಸಮ್ಮತ-ವಿಲ್ಲ
ಸಮ್ಮತಿ-ಯನ್ನು
ಸಮ್ಮತಿ-ಸಲು
ಸಮ್ಮ-ತಿಸಿ
ಸಮ್ಮತಿ-ಸುತ್ತೀರಾ
ಸಮ್ಮಿಶ್ರ-ಗೊಳಿ-ಸಲು
ಸಮ್ಮಿಶ್ರ-ಣದ
ಸಮ್ಮಿಶ್ರಣ-ದಿಂದ
ಸಮ್ಮೇ-ಳನ-ದಲ್ಲಿ
ಸಮ್ಮೋಹಿನಿ
ಸರಂಜಾಮು-ಗಳನ್ನೆಲ್ಲಾ
ಸರ-ಕಾ-ರಕ್ಕೆ
ಸರ-ಕಾ-ರರು
ಸರ-ಕಾರ-ವನ್ನು
ಸರ-ಕಾರಿ
ಸರ-ಕಿನ
ಸರ-ಕು-ಗಳನ್ನು
ಸರದಿ
ಸರ-ಪಳಿ
ಸರ-ಪಳಿಗೆ
ಸರ-ಪಳಿ-ಯಂತೆ
ಸರ-ಪಳಿ-ಯನ್ನು
ಸರ-ಪಳಿ-ಯನ್ನೂ
ಸರ-ಪಳಿ-ಯಲ್ಲಿ
ಸರಳ
ಸರ-ಳತೆ
ಸರ-ಳ-ವಾಗಿಯೇ
ಸರ-ಳ-ವಾದ
ಸರಸಿ
ಸರಸ್ವತಿ
ಸರಸ್ವ-ತಿಯ
ಸರಸ್ವ-ತಿಯೆ
ಸರಸ್ವ-ತಿಯೇ
ಸರಾಗ-ವಾಗಿ
ಸರಿ
ಸರಿ-ಗಟ್ಟ-ಬೇಕು
ಸರಿ-ತಾನೆ
ಸರಿದು
ಸರಿದು-ಹೋಗಿ
ಸರಿದು-ಹೋದರೆ
ಸರಿದೆ
ಸರಿಪಡಿಸಲು
ಸರಿಪಡಿಸಿ
ಸರಿಮಿಗಿಲು
ಸರಿಯ-ಕೂಡದು
ಸರಿ-ಯಲ್ಲ
ಸರಿ-ಯಾಗಿ
ಸರಿ-ಯಾಗಿದ್ದರೂ
ಸರಿ-ಯಾಗಿರ-ಬೇಕು
ಸರಿ-ಯಾಗಿರು-ತ್ತದೆ
ಸರಿ-ಯಾಗಿರು-ವುದು
ಸರಿ-ಯಾಗಿಲ್ಲ
ಸರಿ-ಯಾಗಿಲ್ಲದ್ದ-ರಿಂದ
ಸರಿ-ಯಾಗಿಲ್ಲ-ವೆಂದು
ಸರಿ-ಯಾಗುವುದು
ಸರಿ-ಯಾದ
ಸರಿ-ಯಿರು-ವುದೆಂದು
ಸರಿ-ಯಿಲ್ಲ
ಸರಿ-ಯುವರು
ಸರಿಯೆ
ಸರಿ-ಯೆಂದು
ಸರಿ-ಯೆಂದೇ
ಸರಿಯೇ
ಸರಿಯೋ
ಸರಿಸಮ-ನಾದ
ಸರಿಸಮ-ವಾಗಿ
ಸರಿಸ-ಮಾನ-ರಾದ-ವರನ್ನು
ಸರಿಸ-ಮಾನ-ರಾರೂ
ಸರಿಸಿ
ಸರಿಸುತ
ಸರಿಸು-ಮಾರಿ-ನಲ್ಲೇ
ಸರಿಸು-ವೆವು
ಸರಿ-ಹೊಂದುತ್ತದೆ
ಸರಿ-ಹೋಗುತ್ತಿತ್ತು
ಸರಿ-ಹೋಗು-ವುದೆಂದು
ಸರೋವರ
ಸರೋವರದ
ಸರೋವರ-ದಂತೆ
ಸರೋವರ-ದಲ್ಲಿ
ಸರೋವರ-ವಾಗಿ
ಸರೋವರವೂ
ಸರ್
ಸರ್ಕಾರ
ಸರ್ಕಾರದ
ಸರ್ಕಾರ-ದವರು
ಸರ್ಕಾರ-ವಾಗಲೀ
ಸರ್ಕಾರ್
ಸರ್ಗ-ದಲ್ಲಿ
ಸರ್ಜನ್ನ-ರನ್ನು
ಸರ್ಪ
ಸರ್ಪಜ್ಞಾನ
ಸರ್ಪರಾಜಿ
ಸರ್ಪವು
ಸರ್ಪವೂ
ಸರ್ವ
ಸರ್ವಂ
ಸರ್ವ-ಕಲ್ಯಾಣ-ರೂಪಂ
ಸರ್ವ-ಕಾಲ-ದಲ್ಲಿಯೂ
ಸರ್ವಕ್ಷಣ
ಸರ್ವ-ಗತ
ಸರ್ವ-ಗತ-ನಾದ
ಸರ್ವ-ಗತ-ಸರ್ವಾಂತರಾತ್ಮಾ
ಸರ್ವ-ಗುಣ
ಸರ್ವಗ್ರಾಸಿನಿ
ಸರ್ವ-ಚಿತ್ಸಂಚಾರಿ
ಸರ್ವ-ಜೀವಿ-ಗಳಿಗೂ
ಸರ್ವಜ್ಞ
ಸರ್ವಜ್ಞ-ನನ್ನಾಗಿ
ಸರ್ವಜ್ಞಾನಿಯು
ಸರ್ವತೋ-ಮುಖ-ವಾದ
ಸರ್ವತ್ಯಾಗಿ-ಗಳಾದ
ಸರ್ವತ್ರ
ಸರ್ವಥಾ
ಸರ್ವ-ದರ್ಶನ-ಗಳಿಗೂ
ಸರ್ವದಾ
ಸರ್ವ-ಧರ್ಮ
ಸರ್ವ-ಧರ್ಮ-ಗಳನ್ನೂ
ಸರ್ವ-ಧರ್ಮದ
ಸರ್ವ-ಧರ್ಮ-ಸ್ವರೂಪಿಣೇ
ಸರ್ವ-ಧರ್ಮ-ಸ್ವರೂಪಿಯೂ
ಸರ್ವ-ನಾಶಕ್ಕೆ
ಸರ್ವ-ನಾಶ-ವಾದೀತು
ಸರ್ವ-ನಿಯಮಾತೀತವು
ಸರ್ವ-ಪರಿತ್ಯಾಗಿ-ಗಳಾದ
ಸರ್ವ-ಪಾಪವ
ಸರ್ವ-ಪಾರಂಗ-ತನಾಗ-ಬೇಕೆಂದು
ಸರ್ವಪ್ರಾಣಿ-ಗಳಲ್ಲೂ
ಸರ್ವಪ್ರಾಣಿ-ಗಳಿಗೂ
ಸರ್ವ-ಭಾವ-ಗಳ
ಸರ್ವ-ಭಾವಾತೀತ-ನಾದ
ಸರ್ವ-ಭೂತಸ್ಥ
ಸರ್ವ-ಭೂತೆ
ಸರ್ವರ
ಸರ್ವ-ರಾತ್ಮನು
ಸರ್ವ-ರಿಗಿರಲಿ
ಸರ್ವ-ರಿಗೂ
ಸರ್ವ-ವನ್ನೂ
ಸರ್ವವು
ಸರ್ವವೂ
ಸರ್ವ-ವೃತ್ತಿ
ಸರ್ವ-ವೃತ್ತಿ-ಗಳ
ಸರ್ವ-ವೆನ್ನುತ
ಸರ್ವವ್ಯಾಪಿ
ಸರ್ವವ್ಯಾಪಿ-ಯಾದ
ಸರ್ವ-ಶಕ್ತ
ಸರ್ವ-ಶಕ್ತ-ನಾದ
ಸರ್ವ-ಶಕ್ತಿ
ಸರ್ವ-ಶಾಸ್ತ್ರ
ಸರ್ವಶ್ರಮ-ಸಹಿಷ್ಣು-ಗ-ಳಾದ
ಸರ್ವಶ್ರೇಷ್ಠ
ಸರ್ವಶ್ರೇಷ್ಠ-ನಾದ
ಸರ್ವಶ್ರೇಷ್ಠ-ರಾದ
ಸರ್ವಶ್ರೇಷ್ಠ-ವಾದ
ಸರ್ವ-ಸಂಶಯಾಃ
ಸರ್ವ-ಸತ್ಯವೂ
ಸರ್ವ-ಸಾ-ಧಾರಣ
ಸರ್ವ-ಸಾ-ಧಾರ-ಣ-ರಿಗೂ
ಸರ್ವಸ್ಥಾನ-ದಲ್ಲಿಯೂ
ಸರ್ವಸ್ವ
ಸರ್ವಸ್ವ-ವನ್ನೂ
ಸರ್ವಸ್ವ-ವೆಂದು
ಸರ್ವ-ಹೀನ
ಸರ್ವಾ
ಸರ್ವಾಂಗ
ಸರ್ವಾ-ಧಿ-ಕಾರಿ
ಸರ್ವಾ-ರಂಭಾ
ಸರ್ವೇಂದ್ರಿ-ಯಾತೀತ
ಸರ್ವೊಚ್ಚ
ಸರ್ವೋಚ್ಚಾಸನ-ವನ್ನು
ಸರ್ವೋತ್ಕೃಷ್ಟ-ವಾದ
ಸಲ
ಸಲ-ಕ-ರಣೆ
ಸಲದ
ಸಲಹೆ
ಸಲ-ಹೆ-ಗಳನ್ನು
ಸಲ-ಹೆ-ಯಂತೆ
ಸಲಿಗೆ
ಸಲು-ವಾಗಿ
ಸಲ್ಲ-ಬೇ-ಕಾದ
ಸಲ್ಲ-ಬೇಕು
ಸಲ್ಲಿ-ಸುತ್ತಾ
ಸಲ್ಲಿ-ಸುತ್ತಾನೋ
ಸಲ್ಲಿ-ಸು-ವು-ದ-ರಿಂದ
ಸಲ್ಲು-ವುದು
ಸಲ್ಲು-ವುದೇ-ನಿದ್ದರೂ
ಸವಲತ್ತು-ಗಳನ್ನೂ
ಸವಾ-ಲನ್ನು
ಸವಿಗಂಪ
ಸವಿ-ಗನಸೆ
ಸವಿಗೆ
ಸವಿ-ದಂತೆ
ಸವಿದು
ಸವಿದೆ
ಸವಿ-ಧಾರೆ-ಯನು
ಸವಿ-ನುಡಿಯ
ಸವಿಯ
ಸವಿ-ಯನ್ನು
ಸವಿಯ-ಲಾ-ಗು-ವು-ದಿಲ್ಲ
ಸವಿಯ-ಲಾರನು
ಸವಿ-ಯು-ವಂತೆ
ಸವಿ-ಯೆಂದು
ಸವೆದು-ಹೋ-ಯಿತು
ಸವೆಯಿಸು
ಸವೆ-ಸುತ್ತಾನೆ
ಸವೆ-ಸುತ್ತಿ-ಹನು
ಸವೈ
ಸಸಿ-ಯನ್ನು
ಸಸ್ಯ
ಸಸ್ಯಾ-ಹಾರ
ಸಸ್ಯಾ-ಹಾರಿ-ಗಳ
ಸಹ
ಸಹ-ಕರಿ-ಸ-ಲಾ-ರರು
ಸಹ-ಕಾರಿ
ಸಹ-ಕಾರಿ-ಗ-ಳಾದ
ಸಹ-ಕಾರಿ-ಗಳೂ
ಸಹ-ಕಾರಿ-ಗ-ಳೆಂದು
ಸಹ-ಕಾರಿ-ಯಾ-ಗಿ-ರುವ-ವು-ಗಳಲ್ಲಿ
ಸಹ-ಕಾರಿ-ಯಾಗುವುದು
ಸಹ-ಕಾರಿ-ಯಾದ
ಸಹ-ಚರನು
ಸಹ-ಚರ-ನೊ-ಡನೆ
ಸಹ-ಚರ-ರೆಂದು
ಸಹಜ
ಸಹಜಂ
ಸಹಜ-ವಾಗಿ
ಸಹಜ-ವಾಗಿಯೇ
ಸಹಜ-ವಾದ
ಸಹಜ-ವಾ-ದುದು
ಸಹಜಸ್ವ-ಭಾವವೇ
ಸಹಜೆ
ಸಹನೆ
ಸಹನೆ-ಯಿಂದಲೆ
ಸಹ-ಪಾಠಿ-ಗಳಾಗಿದ್ದರು
ಸಹ-ಭಾವಿ-ಯಾಗಿದ್ದುವು
ಸಹ-ಮತ-ವಿಲ್ಲ
ಸಹ-ವಾಸ
ಸಹವಾ-ಸಕ್ಕೆ
ಸಹವಾ-ಸ-ದಲ್ಲಿ-ರುವರೋ
ಸಹವಾ-ಸ-ದಿಂದ
ಸಹವಾ-ಸ-ವನ್ನು
ಸಹವಾ-ಸಿ-ಯೊಬ್ಬರು
ಸಹಸ
ಸಹಸಾ
ಸಹಸ್ರ
ಸಹಸ್ರಾರು
ಸಹಾನು-ಭೂತಿ
ಸಹಾನು-ಭೂತಿ-ಯನ್ನು
ಸಹಾನು-ಭೂತಿ-ಯನ್ನೂ
ಸಹಾನು-ಭೂತಿ-ಯಿಲ್ಲ
ಸಹಾನು-ಭೂತಿಯೂ
ಸಹಾಯ
ಸಹಾಯ-ಕ-ರಾಗ-ದಿದ್ದಲ್ಲಿ
ಸಹಾಯ-ಕ-ರಾಗ-ಬೇಕು
ಸಹಾಯ-ಕ-ರಾಗಿ
ಸಹಾಯ-ಕ-ರಾದ
ಸಹಾಯ-ಕ-ಳೆಂದು
ಸಹಾಯ-ಕಾರಿ
ಸಹಾಯ-ಕಾರಿ-ಯಾಗುವುದು
ಸಹಾ-ಯಕ್ಕೆ
ಸಹಾಯ-ಗಳನ್ನು
ಸಹಾಯ-ಗಳಿಂದ
ಸಹಾಯದ
ಸಹಾಯ-ದಿಂದ
ಸಹಾಯ-ಮಾ-ಡಲು
ಸಹಾಯ-ಮಾಡಿ
ಸಹಾಯ-ವನ್ನು
ಸಹಾಯ-ವನ್ನೂ
ಸಹಾಯ-ವಾಗ-ಬಹು-ದೆಂದು
ಸಹಾಯ-ವಾಗುತ್ತ-ದೆ-ಯೇನು
ಸಹಾಯ-ವಾ-ಗು-ವಂತೆ
ಸಹಾಯ-ವಾಗು-ವು-ದಲ್ಲದೆ
ಸಹಾಯ-ವಾ-ಗು-ವು-ದಿಲ್ಲ
ಸಹಾಯ-ವಾ-ಗು-ವು-ದಿಲ್ಲವೆ
ಸಹಾಯ-ವಾಗು-ವುದು
ಸಹಾಯ-ವಾಗು-ವುದೆಂದು
ಸಹಾಯ-ವಾ-ದೀತು
ಸಹಾಯ-ವಾ-ಯಿತು
ಸಹಾಯವೂ
ಸಹಾಯ-ಹಸ್ತ
ಸಹಾಯಾರ್ಥ-ವಾಗಿ
ಸಹಿಬೆ
ಸಹಿಷ್ಣು-ಗ-ಳಲ್ಲ
ಸಹಿಷ್ಣು-ಗ-ಳಾದ
ಸಹಿಷ್ಣುತೆ
ಸಹಿಸ-ಬಲ್ಲರು
ಸಹಿಸ-ಬಲ್ಲುದು
ಸಹಿ-ಸಲ-ಶಕ್ಯ-ವಾಗ-ಬ-ಹುದು
ಸಹಿ-ಸ-ಲಾ-ಗು-ವು-ದಿಲ್ಲ
ಸಹಿ-ಸ-ಲಾ-ಗು-ವು-ದಿಲ್ಲ-ವೆಂದೂ
ಸಹಿ-ಸ-ಲಾ-ರರು
ಸಹಿಸಿ
ಸಹಿ-ಸಿಕೊ
ಸಹಿಸಿ-ಕೊಳ್ಳಬೇ-ಕಾ-ಯಿತು
ಸಹಿಸಿ-ಕೊಳ್ಳು-ವುದಕ್ಕೆ
ಸಹಿ-ಸಿಯೂ
ಸಹಿಸು-ತಿ-ರುವೆ
ಸಹಿ-ಸುತ್ತಿ-ರ-ಲಿಲ್ಲ
ಸಹಿ-ಸು-ವುದು
ಸಹಿ-ಸುವೆ
ಸಹೃದ-ಯರೂ
ಸಹೋದರ
ಸಹೋದ-ರತ್ವದ
ಸಹೋದ-ರತ್ವ-ವನ್ನು
ಸಹೋದ-ರ-ನಿಂದಲೇ
ಸಹೋದ-ರ-ನಿದ್ದ
ಸಹೋದ-ರರ
ಸಹೋದ-ರ-ರಂತೆ
ಸಹೋದ-ರ-ರಿ-ಗಾಗಿ
ಸಹೋದ-ರ-ರಿಗೆ
ಸಹೋದ-ರ-ರಿಬ್ಬರೂ
ಸಹೋದ-ರರು
ಸಹೋದ-ರರೆ
ಸಹೋದರಿ
ಸಾ
ಸಾಂಖ್ಯ-ದರ್ಶನ-ದಲ್ಲಿ
ಸಾಂತ
ಸಾಂತ-ತೆಯು
ಸಾಂತ-ವಾಗದೆ
ಸಾಂತ-ವಾಗಿ
ಸಾಂತ-ವಾಗಿ-ರು-ವಂತೆ
ಸಾಂತ್ವನ-ವಾ-ಗಲೆಂದು
ಸಾಂಪ್ರದಾಯಿಕ
ಸಾಂಬಂಧಿಕ
ಸಾಂಬಾ
ಸಾಂಸಾರಿಕ
ಸಾಕಲಾ-ಗ-ದಿದ್ದರೆ
ಸಾಕಷ್ಟು
ಸಾಕಾಗು-ವಂತಿರ-ಬೇಕು
ಸಾಕಾಗುವಷ್ಟನ್ನು
ಸಾಕಾಗು-ವಷ್ಟು
ಸಾಕಾ-ದಷ್ಟು
ಸಾಕಾರ
ಸಾಕಾರ-ಗೊಂಡಿ-ಹುದು
ಸಾಕಾರ-ದೊಂದಿಗೆ
ಸಾಕಾರ-ಮೂರ್ತಿ-ಯಂತಿದ್ದ
ಸಾಕಾರ-ವನ್ನು
ಸಾಕಾರ-ವಾದ
ಸಾಕಾ-ರವೇ
ಸಾಕು
ಸಾಕುತ್ತಿರು-ವಂತೆ
ಸಾಕು-ವುದು
ಸಾಕ್ಷಾತ್
ಸಾಕ್ಷಾತ್ಕರಿ-ಸಿಕೊ
ಸಾಕ್ಷಾತ್ಕರಿ-ಸಿ-ಕೊಂಡ
ಸಾಕ್ಷಾತ್ಕರಿ-ಸಿ-ಕೊಂಡರೆ
ಸಾಕ್ಷಾತ್ಕರಿ-ಸಿ-ಕೊಂಡಿರುವ
ಸಾಕ್ಷಾತ್ಕರಿ-ಸಿ-ಕೊಂಡಿರು-ವಿರಿ
ಸಾಕ್ಷಾತ್ಕರಿ-ಸಿಕೊಳ್ಳ-ಬಲ್ಲರು
ಸಾಕ್ಷಾತ್ಕರಿ-ಸಿಕೊಳ್ಳ-ಬ-ಹುದು
ಸಾಕ್ಷಾತ್ಕರಿ-ಸಿಕೊಳ್ಳ-ಬೇ-ಕಾದರೆ
ಸಾಕ್ಷಾತ್ಕರಿ-ಸಿಕೊಳ್ಳ-ಲಾರದೆ
ಸಾಕ್ಷಾತ್ಕರಿ-ಸಿ-ಕೊಳ್ಳಿ
ಸಾಕ್ಷಾತ್ಕರಿ-ಸಿಕೊಳ್ಳುವರೋ
ಸಾಕ್ಷಾತ್ಕರಿ-ಸಿಕೊಳ್ಳುವು-ದಾಗಿದೆ
ಸಾಕ್ಷಾತ್ಕರಿ-ಸಿಕೊಳ್ಳುವುದು
ಸಾಕ್ಷಾತ್ಕರಿ-ಸಿಕೊಳ್ಳುವೆ
ಸಾಕ್ಷಾತ್ಕಾರ
ಸಾಕ್ಷಾತ್ಕಾರಕ್ಕಾಗಿ
ಸಾಕ್ಷಾತ್ಕಾರಕ್ಕಾಗಿಯೂ
ಸಾಕ್ಷಾತ್ಕಾರಕ್ಕಾಗಿಯೇ
ಸಾಕ್ಷಾತ್ಕಾ-ರಕ್ಕೆ
ಸಾಕ್ಷಾತ್ಕಾರ-ಗಳ
ಸಾಕ್ಷಾತ್ಕಾರದ
ಸಾಕ್ಷಾತ್ಕಾರ-ದಲ್ಲಿ
ಸಾಕ್ಷಾತ್ಕಾರ-ದಿಂದಾಗುವ
ಸಾಕ್ಷಾತ್ಕಾರ-ವನ್ನು
ಸಾಕ್ಷಾತ್ಕಾರ-ವಾಗ-ಬೇಕು
ಸಾಕ್ಷಾತ್ಕಾರ-ವಾಗಿಲ್ಲ
ಸಾಕ್ಷಾತ್ಕಾರ-ವಾಗುತ್ತದೆ
ಸಾಕ್ಷಾತ್ಕಾರ-ವಾ-ಗು-ವು-ದಿಲ್ಲ
ಸಾಕ್ಷಾತ್ಕಾರ-ವಾಗು-ವುದು
ಸಾಕ್ಷಾತ್ಕಾರ-ವಾಗು-ವುದೆಂಬ
ಸಾಕ್ಷಾತ್ಕಾರ-ವಾಗು-ವುದೊ
ಸಾಕ್ಷಾತ್ಕಾರ-ವಾದ
ಸಾಕ್ಷಾತ್ಕಾ-ರವೂ
ಸಾಕ್ಷಾತ್ಕಾರ-ವೆಂದಿಗೂ
ಸಾಕ್ಷಾತ್ಕಾ-ರವೇ
ಸಾಕ್ಷಾತ್ತಾಗಿ
ಸಾಕ್ಷಿ
ಸಾಕ್ಷಿ-ಯಾ-ಗಿದೆ
ಸಾಕ್ಷಿ-ಯಾಗಿ-ರುವರು
ಸಾಕ್ಷಿ-ಯಾತನು
ಸಾಕ್ಷಿಸ್ವ-ರೂಪ-ವಾಗಿ
ಸಾಕ್ಷೀ-ಭೂತ-ರಾಗಿ
ಸಾಗರ
ಸಾಗರದ
ಸಾಗರ-ದಂತಿರುತ್ತಿತ್ತು
ಸಾಗರ-ದಂತೆ
ಸಾಗರ-ದಲ್ಲಿ
ಸಾಗರ-ದೊಲು
ಸಾಗರ-ಭೈರ-ವನು
ಸಾಗರವ
ಸಾಗರ-ವನ್ನು
ಸಾಗರ-ವಾಗು-ವುದು
ಸಾಗರ-ವಿ-ದ-ರಲಿ
ಸಾಗರವು
ಸಾಗರವೆ
ಸಾಗರೇ
ಸಾಗಲೇ-ಬೇಕು
ಸಾಗಿ-ದನು
ಸಾಗಿ-ರಲು
ಸಾಗಿ-ಸ-ಲಾಗಿತ್ತು
ಸಾಗಿ-ಹನು
ಸಾಗಿಹೆ
ಸಾಗುತ
ಸಾಗುತಿತ್ತು
ಸಾಗು-ತಿದೆ
ಸಾಗುತಿದ್ದನು
ಸಾಗುತ್ತಿತ್ತು
ಸಾಗುತ್ತಿತ್ತೋ
ಸಾಗುತ್ತಿದ್ದೇನೆ
ಸಾಗುತ್ತಿ-ರು-ವುದು
ಸಾಗು-ವವು
ಸಾಗು-ವು-ದಿಲ್ಲ
ಸಾಗು-ವುವು
ಸಾಗೆಲೈ
ಸಾಜೆ
ಸಾಟಿ-ಯಾದ
ಸಾತ್ತ್ವಿಕ
ಸಾತ್ತ್ವಿಕ-ಗುಣ-ವಿದೆ-ಯೆನ್ನುವೆ-ಯೇನು
ಸಾತ್ತ್ವಿಕ-ನಾದ
ಸಾತ್ತ್ವಿಕ-ರಾಗುವರು
ಸಾತ್ವಿಕ-ರಾಗಿದ್ದ-ರೆಂಬುದು
ಸಾತ್ವಿಕ-ರಾಗುತ್ತಾರೆಯೆ
ಸಾತ್ವಿ-ಕ-ರೆಂದು
ಸಾದರ
ಸಾದೃಶ್ಯ-ವಿದೆ
ಸಾದ್ವಿ-ಶಿರೋ-ಮಣಿ
ಸಾಧಕ
ಸಾಧ-ಕನ
ಸಾಧಕ-ನನ್ನು
ಸಾಧಕ-ನಿಗೆ
ಸಾಧ-ಕನು
ಸಾಧಕ-ರಿಗೆ
ಸಾಧ-ಕರು
ಸಾಧಕ-ವಾದ
ಸಾಧನ
ಸಾಧನ-ಗಳ
ಸಾಧನ-ಗಳಿಗೆ
ಸಾಧನ-ಗಳು
ಸಾಧನದ
ಸಾಧನ-ಭ-ಜನೆ-ಗಳನ್ನೂ
ಸಾಧನ-ಭ-ಜನೆ-ಗಳಿ-ಗಿಂತ
ಸಾಧನ-ವನ್ನು
ಸಾಧನ-ವಾಗು-ವು-ದಲ್ಲವೆ
ಸಾಧ-ನವೆ
ಸಾಧ-ನಾದಿ-ಗಳ
ಸಾಧನಾ-ಮಾರ್ಗ-ಗಳು
ಸಾಧನೆ
ಸಾಧನೆ-ಗನು-ಗುಣ-ವಾಗಿ
ಸಾಧನೆ-ಗಳ
ಸಾಧನೆ-ಗಳನ್ನು
ಸಾಧನೆ-ಗಳಿಂದ
ಸಾಧನೆ-ಗಳು
ಸಾಧನೆ-ಗಳೆಲ್ಲ
ಸಾಧನೆಗೆ
ಸಾಧನೆ-ಮಾಡು-ವಷ್ಟು
ಸಾಧನೆಯ
ಸಾಧನೆ-ಯನ್ನು
ಸಾಧನೆ-ಯಲ್ಲಿ
ಸಾಧನೆ-ಯಾ-ಗಿದೆ
ಸಾಧನೆ-ಯಿಂದ
ಸಾಧನೆ-ಯಿಂದಲೇ
ಸಾಧನೆಯು
ಸಾಧ-ಮಾನ
ಸಾಧಾರಣ
ಸಾಧಾರ-ಣ-ರನ್ನು
ಸಾಧಾರ-ಣ-ರಿಗೆ
ಸಾಧಾರ-ಣ-ವಾಗಿ
ಸಾಧಾರ-ಣ-ವಾಗಿದ್ದರೆ
ಸಾಧಾರ-ಣೀ-ಕ-ರಿಸಿ
ಸಾಧಿತ-ವಾಗ-ಬೇಕು
ಸಾಧಿತ-ವಾಗುತ್ತದೆ
ಸಾಧಿತ-ವಾ-ಯಿತು
ಸಾಧಿತೆ
ಸಾಧಿಸ-ಬಲ್ಲರು
ಸಾಧಿಸ-ಬಹುದೋ
ಸಾಧಿ-ಸ-ಲಾ-ಗು-ವು-ದಿಲ್ಲ
ಸಾಧಿ-ಸ-ಲಾ-ರರು
ಸಾಧಿ-ಸಲು
ಸಾಧಿ-ಸಲ್ಪಟ್ಟಿದೆ
ಸಾಧಿಸಿ
ಸಾಧಿ-ಸಿಯೇ
ಸಾಧಿಸಿ-ರು-ವಷ್ಟು
ಸಾಧಿ-ಸುವರು
ಸಾಧಿ-ಸು-ವುದಕ್ಕಾ-ಗು-ವು-ದಿಲ್ಲ
ಸಾಧಿ-ಸು-ವು-ದಕ್ಕೂ
ಸಾಧಿ-ಸು-ವುದು
ಸಾಧಿ-ಸುವೆ
ಸಾಧು
ಸಾಧು-ಗಳ
ಸಾಧು-ಗಳನ್ನು
ಸಾಧು-ಗಳನ್ನೂ
ಸಾಧು-ಗಳಲ್ಲಿಯೂ
ಸಾಧು-ಗಳಾಗ-ಲಾ-ರರು
ಸಾಧು-ಗಳಿಗೆ
ಸಾಧು-ಗಳು
ಸಾಧು-ಗಳೆಲ್ಲಿ
ಸಾಧು-ಗಳೇ
ಸಾಧು-ಗಳೊ-ಡನೆ
ಸಾಧುನ್
ಸಾಧು-ವಾಗ-ಬೇಕೆಂದು
ಸಾಧು-ವಾದ
ಸಾಧು-ವಿಗೆ
ಸಾಧು-ವಿನ
ಸಾಧುವೆ
ಸಾಧು-ವೊಬ್ಬನೇ
ಸಾಧು-ಸಂಗ
ಸಾಧು-ಸಂತರ
ಸಾಧು-ಸಂನ್ಯಾಸಿ-ಗಳನ್ನು
ಸಾಧು-ಸಂನ್ಯಾಸಿ-ಗಳು
ಸಾಧುಸ್ವ-ಭಾವ
ಸಾಧುಸ್ವ-ಭಾವ-ದಲ್ಲಿ
ಸಾಧ್ಯ
ಸಾಧ್ಯ-ತೆ-ಗಳನ್ನು
ಸಾಧ್ಯ-ವಲ್ಲ
ಸಾಧ್ಯ-ವಾಗ-ಲಿಲ್ಲ
ಸಾಧ್ಯ-ವಾಗಿತ್ತು
ಸಾಧ್ಯ-ವಾಗಿ-ರು-ವಾಗ
ಸಾಧ್ಯ-ವಾಗುತ್ತದೆ
ಸಾಧ್ಯ-ವಾಗುತ್ತಿತ್ತು
ಸಾಧ್ಯ-ವಾ-ಗು-ವಂತೆ
ಸಾಧ್ಯ-ವಾ-ಗು-ವು-ದಿಲ್ಲ
ಸಾಧ್ಯ-ವಾಗು-ವುದು
ಸಾಧ್ಯ-ವಾಗು-ವುದೋ
ಸಾಧ್ಯ-ವಾದ
ಸಾಧ್ಯ-ವಾದರೆ
ಸಾಧ್ಯ-ವಾ-ದಷ್ಟು
ಸಾಧ್ಯ-ವಾ-ಯಿತು
ಸಾಧ್ಯ-ವಿದ್ದರೆ
ಸಾಧ್ಯ-ವಿದ್ದಲ್ಲಿ
ಸಾಧ್ಯ-ವಿದ್ದಷ್ಟೂ
ಸಾಧ್ಯ-ವಿಲ್ಲ
ಸಾಧ್ಯ-ವಿಲ್ಲದೆ
ಸಾಧ್ಯ-ವಿಲ್ಲದೇ
ಸಾಧ್ಯವೆ
ಸಾಧ್ಯವೇ
ಸಾಧ್ಯ-ವೇ-ನಪ್ಪಾ
ಸಾಧ್ಯ-ವೇನು
ಸಾಧ್ಯವೋ
ಸಾಧ್ವಿ-ಯರು
ಸಾನ್ನಿಧ್ಯ-ದಲ್ಲಿ
ಸಾಪೇಕ್ಷ
ಸಾಪೇಕ್ಷ-ತೆಯ
ಸಾಪೇಕ್ಷ-ತೆ-ಯಲ್ಲಿ
ಸಾಪೇಕ್ಷ-ವಾದ
ಸಾಪೇಕ್ಷ-ವಾ-ದುದು
ಸಾಮ-ಗಾ-ನವೇ
ಸಾಮಗ್ರಿ-ಗಳಿಂದ
ಸಾಮ-ರಸ್ಯ
ಸಾಮ-ರಸ್ಯದ
ಸಾಮ-ರಸ್ಯ-ವನ್ನು
ಸಾಮ-ರಸ್ಯ-ವಿರು-ವುದು
ಸಾಮ-ರಸ್ಯವೇ
ಸಾಮರ್ಥ್ಯ
ಸಾಮರ್ಥ್ಯಕ್ಕನು-ಸಾರ-ವಾಗಿ
ಸಾಮರ್ಥ್ಯದ
ಸಾಮರ್ಥ್ಯ-ದಲ್ಲಿಯೂ
ಸಾಮರ್ಥ್ಯ-ದಿಂದ
ಸಾಮರ್ಥ್ಯ-ದಿಂದಲ್ಲ
ಸಾಮರ್ಥ್ಯ-ವನ್ನೂ
ಸಾಮರ್ಥ್ಯ-ವಿ-ರ-ಲಿಲ್ಲ
ಸಾಮರ್ಥ್ಯ-ವಿಲ್ಲ
ಸಾಮರ್ಥ್ಯ-ವುಳ್ಳ-ವ-ರಾಗಿದ್ದರು
ಸಾಮರ್ಥ್ಯವೂ
ಸಾಮರ್ಥ್ಯವೇ
ಸಾಮಾಖ್ಯಾದ್ಯೈರ್ಗೀತಿ
ಸಾಮಾಜಿಕ
ಸಾಮಾಜಿಕ-ತೆಯೂ
ಸಾಮಾದಿ
ಸಾಮಾನು
ಸಾಮಾನು-ಗಳ
ಸಾಮಾನು-ಗಳನ್ನು
ಸಾಮಾನ್ಯ
ಸಾಮಾನ್ಯ-ಕೋಟಿ
ಸಾಮಾನ್ಯ-ನಾಗಿ-ರು-ವನು
ಸಾಮಾನ್ಯ-ರಿಗೆ
ಸಾಮಾನ್ಯ-ವಾಗಿ
ಸಾಮಾನ್ಯ-ವಾದ
ಸಾಮಿಜಿ
ಸಾಮೂಹಿಕ
ಸಾಮ್ಯ-ದಿಂದ
ಸಾಮ್ರಾಜ್ಯ-ಗಳೇ
ಸಾಮ್ರಾಟ-ನಿಗೆ
ಸಾಮ್ರಾಟನು
ಸಾಮ್ರಾಟನೇ
ಸಾಮ್ರಾಟ್
ಸಾಯಂಕಾಲ
ಸಾಯಂಕಾಲಕ್ಕೆ
ಸಾಯಂಕಾಲದ
ಸಾಯಂಕಾಲ-ವಾಗಲಿ
ಸಾಯಂಕಾಲ-ವಾಗಲು
ಸಾಯಂಕಾಲ-ವಾಗಿ
ಸಾಯಂಕಾಲ-ವಾಗು-ವುದಕ್ಕೆ
ಸಾಯಣ-ಭಾಷ್ಯ-ವನ್ನು
ಸಾಯಣರ
ಸಾಯಣರೆ
ಸಾಯಣ-ರೆಂದು
ಸಾಯಣಾ-ಚಾರ್ಯ-ರನ್ನು
ಸಾಯಣಾ-ಚಾರ್ಯರು
ಸಾಯ-ಬ-ಹುದು
ಸಾಯ-ಬೇಕಾಗಿದ್ದ
ಸಾಯ-ಬೇಕಾಗಿ-ರುವಾಗ
ಸಾಯಬೇ-ಕಾ-ಯಿತು
ಸಾಯ-ಬೇಕು
ಸಾಯ-ಬೇಕೆಂದು
ಸಾಯಲಿ
ಸಾಯ-ಲಿಕ್ಕೆ
ಸಾಯಲಿಚ್ಛಿ-ಸುವೆನು
ಸಾಯಲು
ಸಾಯಿ-ಸಲ್ಪಟ್ಟರೂ
ಸಾಯಿಸಿ
ಸಾಯಿಸಿ-ದರೂ
ಸಾಯಿ-ಸುತ್ತಾರೆ
ಸಾಯಿ-ಸು-ವುದು
ಸಾಯುತ್ತವೆ
ಸಾಯುತ್ತಾರೆ
ಸಾಯುತ್ತಿದ್ದಾನೆ
ಸಾಯುತ್ತಿದ್ದಾರೆ
ಸಾಯುತ್ತಿದ್ದಾರೆಂದು
ಸಾಯುತ್ತಿರುವ
ಸಾಯುತ್ತಿಲ್ಲ
ಸಾಯುತ್ತೀರಿ
ಸಾಯುತ್ತೇವೆ
ಸಾಯುವ
ಸಾಯುವ-ತನಕ
ಸಾಯುವನು
ಸಾಯುವನೋ
ಸಾಯುವ-ರೆಂದೂ
ಸಾಯುವರೇ
ಸಾಯುವ-ವರ
ಸಾಯುವ-ವರೆಗೆ
ಸಾಯುವುದಕ್ಕಿಂತ
ಸಾಯು-ವುದಕ್ಕೆ
ಸಾಯು-ವು-ದನ್ನು
ಸಾಯು-ವು-ದಿಲ್ಲ
ಸಾಯು-ವುದು
ಸಾಯು-ವುದೇ
ಸಾಯು-ವುದೊಳ್ಳೆ-ಯದು
ಸಾಯುವೆ
ಸಾಯುವೆ-ಯಾದರೆ
ಸಾಯುವೆ-ಯೇಕೆ
ಸಾಯು-ವೆವು
ಸಾರ
ಸಾರ-ತ-ರವೂ
ಸಾರನ್ನು
ಸಾರ-ಬೇಕಾಗಿದ್ದ
ಸಾರ-ಭೂತ-ವಾದದ್ದೆಂದರೆ
ಸಾರ-ಲಾ-ರರು
ಸಾರ-ವನ್ನು
ಸಾರ-ವನ್ನೆಲ್ಲಾ
ಸಾರ-ವಾದ
ಸಾರ-ವಿಲ್ಲ-ದಿದ್ದರೆ
ಸಾರ-ವೆಂದು
ಸಾರ-ವೆಲ್ಲ
ಸಾರಾಂಶ-ವೆಲ್ಲಾ
ಸಾರಿ
ಸಾರಿ-ದಂತಾ-ಗು-ವು-ದಿಲ್ಲ
ಸಾರಿ-ದರು
ಸಾರಿ-ದರೆ
ಸಾರಿ-ನೊ-ಡನೆ
ಸಾರಿ-ಯಾದರೂ
ಸಾರಿ-ರುವ
ಸಾರಿ-ಸು-ವು-ದ-ರಿಂದ
ಸಾರು
ಸಾರುತ್ತವೆ
ಸಾರುತ್ತಾ
ಸಾರುತ್ತಿದ್ದ
ಸಾರುತ್ತಿ-ರ-ಲಿಲ್ಲ
ಸಾರು-ವು-ದಿಲ್ಲವೇ
ಸಾರು-ವುದು
ಸಾರು-ವುವು
ಸಾರೈ
ಸಾರ್ಥಕ-ಗೊಳಿಸಿ
ಸಾರ್ಥ-ಕ-ತೆಗೆ
ಸಾರ್ಥ-ಕತೆ-ಯನು
ಸಾರ್ಥಕ-ವಾ-ಗು-ವಂತೆ
ಸಾರ್ಥಕ-ವಾದದ್ದನ್ನೂ
ಸಾರ್ಥಕ-ವಾ-ಯಿತೆಂದು
ಸಾರ್ಥಕ-ವಾ-ಯಿತೆಂಬು-ದನ್ನು
ಸಾರ್ಥಕ-ವೆಂದು
ಸಾರ್ದೆ
ಸಾರ್ವ-ಜನಿಕ
ಸಾರ್ವ-ಜನಿ-ಕರ
ಸಾರ್ವತ್ರಿಕ
ಸಾರ್ವತ್ರಿಕ-ತೆ-ಯಲ್ಲಿ
ಸಾರ್ವಭೌಮ-ನಾಗಿದ್ದನು
ಸಾರ್ವಭೌಮಿಕ
ಸಾಲದು
ಸಾಲದೆ
ಸಾಲದ್ದಕ್ಕೆ
ಸಾಲ-ವಾಗಿ
ಸಾಲು
ಸಾಲು-ಗಳಿವು
ಸಾಲು-ಗಳು
ಸಾಲುತ್ತಿ-ರ-ಲಿಲ್ಲ
ಸಾಲೊಮನ್ನಿನ
ಸಾಲೋಮನ್
ಸಾಲೋಮನ್ನನ
ಸಾಲ್ವುದೆ
ಸಾವಧಾ-ನದಿಂದ
ಸಾವನ್ನಪ್ಪಿದ-ವ-ರೆಲ್ಲ
ಸಾವವು
ಸಾವಿಗೀಡಾ-ದರೂ
ಸಾವಿಗು
ಸಾವಿಗೆ
ಸಾವಿತ್ರಿ
ಸಾವಿತ್ರಿ-ಯರ
ಸಾವಿದೆ
ಸಾವಿನ
ಸಾವಿರ
ಸಾವಿರ-ದಷ್ಟು
ಸಾವಿರ-ಬಾರಿಗೂ
ಸಾವಿರ-ವನ್ನು
ಸಾವಿ-ರವೆ
ಸಾವಿ-ರಾರು
ಸಾವು
ಸಾವು-ಗಳ
ಸಾವು-ಗಳೆ-ರಡು
ಸಾವು-ಬದು-ಕಿನ
ಸಾಷ್ಟಾಂಗ
ಸಾಸಿರ
ಸಾಸಿರದ
ಸಾಸಿವೆ-ಕಾಳಿಗೂ
ಸಾಹಸ
ಸಾಹ-ಸದ
ಸಾಹಸ-ಮಯ
ಸಾಹಿತಿ
ಸಾಹಿತ್ಯ
ಸಾಹಿತ್ಯದ
ಸಾಹಿತ್ಯ-ದಲ್ಲಿ
ಸಾಹಿತ್ಯ-ವನ್ನು
ಸಾಹೇಬನು
ಸಾಹೇಬರು
ಸಾಹೇಬರೂ
ಸಿಂಗರಿ-ಸ-ದನು
ಸಿಂಗರಿ-ಸುವ
ಸಿಂಗ್
ಸಿಂಗ್ಜಿ
ಸಿಂಚನ
ಸಿಂಧು
ಸಿಂಧು-ರೋಲೆ
ಸಿಂಧುವೂ
ಸಿಂಧೋಸ್ತ-ರಣೇಽಸ್ತ್ಯು-ಪಾಯಃ
ಸಿಂಹ
ಸಿಂಹ-ಗರ್ಜನೆ
ಸಿಂಹ-ಗರ್ಜನೆ-ಯಂತೆ
ಸಿಂಹ-ಗರ್ಜನೆ-ಯಿಂದ
ಸಿಂಹ-ಗಳನ್ನು
ಸಿಂಹ-ಗಳಿಗೆ
ಸಿಂಹದ
ಸಿಂಹ-ದಂತೆ
ಸಿಂಹ-ನಾದ
ಸಿಂಹ-ನಾದಂ
ಸಿಂಹ-ವದು
ಸಿಂಹ-ವನ್ನೂ
ಸಿಂಹ-ವಾಣಿ-ಯಲ್ಲಿ
ಸಿಂಹವು
ಸಿಂಹ-ಸದೃಶ
ಸಿಂಹ-ಸದೃಶ-ರಾಗಿ
ಸಿಂಹ-ಸಮ-ನಾದ
ಸಿಂಹ-ಸಾಹಸಿ-ಕತೆ
ಸಿಂಹಾಸನ
ಸಿಂಹಾಸ-ನದ
ಸಿಂಹಾಸ-ನದಿ
ಸಿಕ್ಕ
ಸಿಕ್ಕದೆ
ಸಿಕ್ಕ-ಬ-ಹುದು
ಸಿಕ್ಕರ
ಸಿಕ್ಕ-ಲಾ-ರರು
ಸಿಕ್ಕ-ಲಿಲ್ಲ-ವೆಂದು
ಸಿಕ್ಕ-ವ-ರಿ-ಗೆಲ್ಲಾ
ಸಿಕ್ಕಿ
ಸಿಕ್ಕಿ-ಕೊಂಡಿದ್ದಾರೆ
ಸಿಕ್ಕಿ-ಕೊಂಡಿರುತ್ತಾರೆಂಬುದೇ
ಸಿಕ್ಕಿ-ಕೊಂಡಿರುವಿ
ಸಿಕ್ಕಿತು
ಸಿಕ್ಕಿತೋ
ಸಿಕ್ಕಿ-ದಂತೆ
ಸಿಕ್ಕಿ-ದರೂ
ಸಿಕ್ಕಿ-ದರೆ
ಸಿಕ್ಕಿದೆ
ಸಿಕ್ಕಿ-ಬೀಳು-ವರು
ಸಿಕ್ಕಿ-ಬೀಳುವೆನೆ
ಸಿಕ್ಕಿ-ಯಾರು
ಸಿಕ್ಕಿ-ರುತ್ತಾರೆ
ಸಿಕ್ಕುತ್ತದೆ
ಸಿಕ್ಕುತ್ತದೆಯೊ
ಸಿಕ್ಕುತ್ತಾರೆ
ಸಿಕ್ಕುವ
ಸಿಕ್ಕು-ವಂತ-ಹು-ದಲ್ಲ
ಸಿಕ್ಕು-ವು-ದಿಲ್ಲ
ಸಿಕ್ಕು-ವು-ದಿಲ್ಲ-ವೆಂದು
ಸಿಕ್ಕು-ವುದು
ಸಿಕ್ಕು-ವುದೋ
ಸಿಕ್ಕೇ-ಸಿಗು-ವುದು
ಸಿಗದ
ಸಿಗ-ದಂತೆ
ಸಿಗದ-ವೊಲು
ಸಿಗ-ದಿದ್ದರೂ
ಸಿಗದೆ
ಸಿಗದೆ-ಹೋದ
ಸಿಗ-ಲಿಲ್ಲ
ಸಿಗಲು
ಸಿಗುತ್ತವೆ
ಸಿಗುತ್ತಾರೆ
ಸಿಗುತ್ತೆ
ಸಿಗುವ
ಸಿಗು-ವರು
ಸಿಗು-ವು-ದಿಲ್ಲ
ಸಿಗು-ವು-ದಿಲ್ಲವೆ
ಸಿಗು-ವು-ದಿಲ್ಲವೋ
ಸಿಗು-ವುದು
ಸಿಗು-ವುದೊ
ಸಿಗು-ವುದೋ
ಸಿಡಿ-ದಿಹ
ಸಿಡಿದು
ಸಿಡಿದೆದ್ದ
ಸಿಡಿ-ಯುತಿರೆ
ಸಿಡಿಲ
ಸಿಡಿಲೈ
ಸಿಡಿ-ವಂತಿತ್ತು
ಸಿಡುಕು-ಕತ್ತಲು
ಸಿದ್ದ-ನಾಗು
ಸಿದ್ದ-ಪಡಿ-ಸ-ಲಾ-ಯಿತು
ಸಿದ್ದ-ಪಡಿಸಿ
ಸಿದ್ದ-ಮಾಡಿ
ಸಿದ್ದ-ರಾಗಿದ್ದಾರೆ-ಆ-ದರೂ
ಸಿದ್ಧ
ಸಿದ್ಧತೆ
ಸಿದ್ಧ-ತೆ-ಗಳಾಗುತ್ತಿದ್ದವು
ಸಿದ್ಧ-ನಾಗ-ಬೇ-ಕಾದರೆ
ಸಿದ್ಧ-ನಾಗ-ಲಾರನು
ಸಿದ್ಧ-ನಾಗಿದ್ದ
ಸಿದ್ಧ-ನಾಗಿದ್ದನೋ
ಸಿದ್ಧ-ನಾಗಿದ್ದರೆ
ಸಿದ್ಧ-ನಾಗಿದ್ದಾನೆ
ಸಿದ್ಧ-ನಾಗಿ-ರ-ಬೇಕು
ಸಿದ್ಧ-ನಾಗಿ-ರುತ್ತಿದ್ದನು
ಸಿದ್ಧ-ನಾಗಿ-ರು-ವನು
ಸಿದ್ಧ-ನಾಗಿ-ರು-ವನೋ
ಸಿದ್ಧ-ನಾಗಿ-ರು-ವೆಯಾ
ಸಿದ್ಧ-ನಾಗು-ವುದಕ್ಕೆ
ಸಿದ್ಧ-ನಾದನು
ಸಿದ್ಧ-ನಾದ-ವನು
ಸಿದ್ಧ-ನಿದ್ದೇನೆ
ಸಿದ್ಧ-ನಿ-ಹನು
ಸಿದ್ಧನು
ಸಿದ್ಧನೆ
ಸಿದ್ಧ-ಪಡಿ-ಸ-ಬೇಕು
ಸಿದ್ಧ-ಪಡಿ-ಸಲೇ-ಬೇಕಾಗಿ
ಸಿದ್ಧ-ಪಡಿಸಿ
ಸಿದ್ಧ-ಪಡಿ-ಸಿ-ಕೊಂಡದ್ದು
ಸಿದ್ಧ-ಪಡಿ-ಸಿ-ಕೊಂಡು
ಸಿದ್ಧ-ಪಡಿ-ಸಿ-ಕೊಟ್ಟನು
ಸಿದ್ಧ-ಪಡಿ-ಸಿ-ದಾಗ
ಸಿದ್ಧ-ಪಡಿಸು
ಸಿದ್ಧ-ಪಡಿ-ಸುತ್ತಾ-ರೆಂಬುದು
ಸಿದ್ಧ-ಪಡಿ-ಸುತ್ತಿದ್ದ-ನಂತೆ
ಸಿದ್ಧ-ಪಡಿ-ಸುತ್ತೇನೆ
ಸಿದ್ಧ-ಪುರುಷ-ರಾದರೊ
ಸಿದ್ಧ-ಮಾಡಲ್ಪಟ್ಟಿದ್ದುವು
ಸಿದ್ಧ-ಮಾಡಿ-ಕೊಂಡು
ಸಿದ್ಧ-ರಾಗಲು
ಸಿದ್ಧ-ರಾಗಿ
ಸಿದ್ಧ-ರಾಗಿದ್ದರು
ಸಿದ್ಧ-ರಾಗಿದ್ದ-ವರು
ಸಿದ್ಧ-ರಾಗಿದ್ದಾರೆ
ಸಿದ್ಧ-ರಾಗಿದ್ದು-ದ-ರಿಂದಲೇ
ಸಿದ್ಧ-ರಾಗಿ-ರುವ
ಸಿದ್ಧ-ರಾಗಿ-ರುವರು
ಸಿದ್ಧ-ರಾಗಿ-ರುವರೊ
ಸಿದ್ಧ-ರಾಗುವರು
ಸಿದ್ಧ-ರಾದರು
ಸಿದ್ಧರು
ಸಿದ್ಧರೆ
ಸಿದ್ಧ-ವಾಗಿದೆ
ಸಿದ್ಧ-ವಾಗಿ-ದೆ-ಯೆಂದು
ಸಿದ್ಧ-ವಾಗಿದ್ದುವು
ಸಿದ್ಧ-ವಾಗಿ-ರುತ್ತದೆ
ಸಿದ್ಧ-ವಾಗಿ-ರುತ್ತವೆ
ಸಿದ್ಧ-ವಾಗುತ್ತದೆ
ಸಿದ್ಧ-ವಾ-ಯಿತು
ಸಿದ್ಧ-ಸಂಕಲ್ಪ
ಸಿದ್ಧಾಂತ
ಸಿದ್ಧಾಂತಕ್ಕಿಂತಲೂ
ಸಿದ್ಧಾಂತಕ್ಕೆ
ಸಿದ್ಧಾಂತ-ಗಳ
ಸಿದ್ಧಾಂತ-ಗಳನ್ನಾಗಲೀ
ಸಿದ್ಧಾಂತ-ಗಳನ್ನು
ಸಿದ್ಧಾಂತ-ಗಳನ್ನೂ
ಸಿದ್ಧಾಂತ-ಗಳಲ್ಲಿ
ಸಿದ್ಧಾಂತ-ಗಳಿಂದ
ಸಿದ್ಧಾಂತ-ಗಳಿಗೆ
ಸಿದ್ಧಾಂತ-ಗಳು
ಸಿದ್ಧಾಂತ-ಗಳೂ
ಸಿದ್ಧಾಂತ-ಗಳೆಲ್ಲ-ವನ್ನೂ
ಸಿದ್ಧಾಂತ-ಗಳೇ
ಸಿದ್ಧಾಂತ-ಗಳೊಂದಿಗೆ
ಸಿದ್ಧಾಂತ-ಗಿದ್ದಾಂತ
ಸಿದ್ಧಾಂತದ
ಸಿದ್ಧಾಂತ-ದಂತೆ
ಸಿದ್ಧಾಂತ-ದಲ್ಲಿ
ಸಿದ್ಧಾಂತ-ವಂತೂ
ಸಿದ್ಧಾಂತ-ವನ್ನು
ಸಿದ್ಧಾಂತ-ವನ್ನೊಪ್ಪಿ-ಕೊಂಡು
ಸಿದ್ಧಾಂತ-ವಾಗಿ
ಸಿದ್ಧಾಂತ-ವಾಗಿದೆ
ಸಿದ್ಧಾಂತ-ವಾಗು-ವುದು
ಸಿದ್ಧಾಂತ-ವಿದೆ
ಸಿದ್ಧಾಂತ-ವಿದ್ದಂತೆ
ಸಿದ್ಧಾಂತವು
ಸಿದ್ಧಾಂತವೂ
ಸಿದ್ಧಾಂತ-ವೆಂದರೆ
ಸಿದ್ಧಾಂತ-ವೆಂದು
ಸಿದ್ಧಾಂತ-ವೆಂಬುದು
ಸಿದ್ಧಾಂತವೇ
ಸಿದ್ಧಾಂತ-ವೇ-ನೆಂದರೆ
ಸಿದ್ಧಿ
ಸಿದ್ಧಿ-ಗಳನ್ನು
ಸಿದ್ಧಿ-ಗಾಗಿ
ಸಿದ್ಧಿಗೆ
ಸಿದ್ಧಿಯ
ಸಿದ್ಧಿ-ಯನ್ನು
ಸಿದ್ಧಿ-ಯಾಗುತ್ತ-ದೆಯೇ
ಸಿದ್ಧಿ-ಯಾಗುತ್ತ-ದೆಯೊ
ಸಿದ್ಧಿ-ಯೆಂದು
ಸಿದ್ಧಿ-ಸು-ವುದೆಂಬ
ಸಿದ್ಧ್ಯತಿ
ಸಿನ್ಹ
ಸಿಪಾಯಿ-ಗಳು
ಸಿಪಾಯಿ-ದಂಗೆ-ಯಲ್ಲಿ
ಸಿಪ್ಪೆಗೆ
ಸಿರಿ
ಸಿರಿಯು
ಸಿಲು-ಕ-ದ-ವನು
ಸಿಲುಕ-ಬ-ಹುದು
ಸಿಲುಕಿ
ಸಿಲುಕಿ-ಕೊಂಡು
ಸಿಲುಕಿದ್ದಾರೆ
ಸಿಲುಕಿ-ರುವ
ಸಿಲುಕಿವೆ
ಸಿಲುಕಿ-ಹವು
ಸಿಸ್ಟರ್
ಸಿಹಿ
ಸಿಹಿ-ಮುತ್ತಿ-ನಲಿ
ಸೀತಯಾ
ಸೀತಾ
ಸೀತಾ-ಪತಿ
ಸೀತಾ-ಮಾತೆಯು
ಸೀತಾ-ರಾಮರು
ಸೀತೆ
ಸೀತೆಗೆ
ಸೀತೆಯ
ಸೀತೆ-ಯಂತೆ
ಸೀನಿ-ಯರ್
ಸೀಮಾ
ಸೀಮಿತ-ವಾಗಿದ್ದವು
ಸೀಮೆ
ಸೀಮೆಗೆ
ಸೀಮೆಯ
ಸೀಮೆ-ಯನ್ನು
ಸೀರೆಯ
ಸೀರೆ-ಯಾಗುತ್ತ-ದೆ-ಯಷ್ಟೆ
ಸೀಳಿ
ಸೀಳುತ
ಸೀಸದ-ಕಡ್ಡಿ-ಯನ್ನು
ಸುಂಕ
ಸುಂಕ-ವನ್ನು
ಸುಂಟರ-ಗಾಳಿ-ಗಳನ್ನು
ಸುಂದರ
ಸುಂದರ-ನಾಗಿ-ರು-ವನೋ
ಸುಂದರ-ವಾಗಿ
ಸುಂದರ-ವಾಗಿ-ರು-ವುದು
ಸುಂದರ-ವಾಗಿವೆ
ಸುಂದರ-ವಾದ
ಸುಂದರ-ವಾ-ದುದು
ಸುಂದರವೂ
ಸುಂದರಿ
ಸುಕಂಠವೂ
ಸುಕ್ಷೇಮ
ಸುಖ
ಸುಖ-ಕ-ರವೂ
ಸುಖಕ್ಕಾಗಿ
ಸುಖಕ್ಕೆ
ಸುಖಕ್ಕೋಸ್ಕರ
ಸುಖ-ಗಳಲ್ಲಿ
ಸುಖ-ಗಳು
ಸುಖ-ತರ
ಸುಖದ
ಸುಖ-ದ-ದುಃಖದ
ಸುಖ-ದಲ್ಲಿ
ಸುಖ-ದಾಗರ
ಸುಖ-ದುಃಖ
ಸುಖ-ದುಃಖ-ಗಳ
ಸುಖ-ದುಃಖ-ಗಳನ್ನು
ಸುಖ-ದುಃಖ-ಗಳೇ
ಸುಖ-ದುಃಖ-ದಿಂದ
ಸುಖ-ದುಃಖ-ವಿದೆ
ಸುಖ-ದುಃಖ-ಸುಳಿ
ಸುಖ-ದುಃಖ-ಹಸ್ತೇ
ಸುಖ-ದುಃಖೆ
ಸುಖ-ದೆಡೆಗೆ
ಸುಖ-ನಿದ್ರೆ-ಯಲ್ಲಿ
ಸುಖ-ಪ-ಡಲು
ಸುಖ-ಭೋಗ-ಗಳಿಂದಲೂ
ಸುಖ-ಭೋಗ-ಗಳಿಗೆ
ಸುಖವ
ಸುಖ-ವ-ದೆಂದಿಗು
ಸುಖ-ವ-ನ-ಮಾಲಿ
ಸುಖ-ವ-ನ-ಮಾಲಿಯು
ಸುಖ-ವನು
ಸುಖ-ವ-ನೆಳ-ಸುವ-ನ-ವನು
ಸುಖ-ವನ್ನು
ಸುಖ-ವನ್ನೂ
ಸುಖ-ವಾಗಿ
ಸುಖ-ವಾಗುತ್ತ-ದೆಯೆ
ಸುಖ-ವಿಲ್ಲ
ಸುಖವು
ಸುಖ-ವು-ದುಃಖ-ವುಹರುಷ-ಕಂಬ-ನಿ-ಯಾಚೆಗೆನ್ನನು
ಸುಖವೆ
ಸುಖವೇ
ಸುಖ-ವೇ-ನಿದೆ
ಸುಖ-ಸಂತೋಷ-ಗಳೆಲ್ಲ
ಸುಖಸ್ವಪ್ನ
ಸುಖಾ-ನು-ಭವ
ಸುಖಾನು-ಭೋಗದ
ಸುಖಾಭಿಲಾಷೆ
ಸುಖಾಯ
ಸುಖೆ
ಸುಖೇರ-ನಹಿ-ಲೇಶ
ಸುಖೇರ್
ಸುಗಂಧ
ಸುಗಮ-ಗೊಳಿಸಿಹೆ
ಸುಗಮ-ವಾಗಿ-ರ-ಬೇಕೆಂದೆನ್ನುತ್ತಿದ್ದನು
ಸುಗಮ-ವಾಗು-ವುದು
ಸುಗಮ-ವಾದ
ಸುಗುಣ-ಗಳ
ಸುಗುಣ-ಗಳನ್ನು
ಸುಚರಣಂ
ಸುಟ್ಟು-ಬಿಡು
ಸುಟ್ಟುಹೋ-ಗದಿದ್ದು-ದನ್ನು
ಸುಟ್ಟು-ಹೋಗುತ್ತಿ-ರ-ಲಿಲ್ಲ
ಸುಟ್ಟು-ಹೋಗು-ವುದು
ಸುಟ್ಟು-ಹೋ-ಯಿತು
ಸುಡದ
ಸುಡು-ತಿದೆ
ಸುಡುವ
ಸುತನ
ಸುತರಾಂ
ಸುತಾರಾಂ
ಸುತ್ತ
ಸುತ್ತ-ಮುತ್ತ-ಲಿ-ರುವ
ಸುತ್ತ-ಮುತ್ತಲು
ಸುತ್ತ-ಮುತ್ತಲೂ
ಸುತ್ತ-ಲಿ-ರುವ
ಸುತ್ತಲೂ
ಸುತ್ತಾಡಿ
ಸುತ್ತಾ-ಡಿ-ಕೊಂಡು
ಸುತ್ತಾ-ಡುತ್ತಾ
ಸುತ್ತಾ-ಡುತ್ತಿದ್ದಾರೆ
ಸುತ್ತಿ
ಸುತ್ತಿ-ಕೊಂಡಿದ್ದ
ಸುತ್ತಿ-ಕೊಂಡಿರು-ವುದು
ಸುತ್ತಿ-ಕೊಂಡು
ಸುತ್ತಿ-ಗೆ-ಯಿಂದ
ಸುತ್ತಿದ
ಸುತ್ತಿ-ದು-ದರ
ಸುತ್ತಿದ್ದೇನೆ
ಸುತ್ತಿವೆ
ಸುತ್ತು
ಸುತ್ತುತ
ಸುತ್ತುತ್ತ
ಸುತ್ತುತ್ತಾ
ಸುತ್ತುತ್ತಿದ್ದರು
ಸುತ್ತುತ್ತಿರು-ವಾಗ
ಸುತ್ತುತ್ತಿ-ರು-ವು-ದನ್ನು
ಸುತ್ತು-ವರಿಯಲ್ಪಟ್ಟಿದ್ದ
ಸುದಿನ
ಸುದೀರ್ಘ-ವಾದ
ಸುದ್ದಿ
ಸುದ್ದಿ-ಯನ್ನ-ರಿತು
ಸುದ್ದಿ-ಯನ್ನೋದಿದೆ
ಸುಧಾರಕ
ಸುಧಾರ-ಕರು
ಸುಧಾರಣೆ
ಸುಧಾರ-ಣೆ-ಗಳಲ್ಲಿ
ಸುಧಾರ-ಣೆ-ಗಳು
ಸುಧಾರ-ಣೆ-ಗ-ಳೆಂದು
ಸುಧಾರ-ಣೆ-ಗಾಗಿ
ಸುಧಾರ-ಣೆಗೆ
ಸುಧಾರ-ಣೆಯ
ಸುಧಾರ-ಣೆ-ಯಾಗ-ಬೇಕೆಂದು
ಸುಧಾರಿತ
ಸುಧಾರಿ-ಸುತ್ತದೆ
ಸುಧೆ
ಸುಧೆಯು
ಸುನಾಉ
ಸುನಿ
ಸುನೀಚೇನ
ಸುನ್ನಿ-ಗಳಿಗೆ
ಸುಪರಿ-ಚಿತ-ರಾಗಿದ್ದಾರೆಯೋ
ಸುಪವಿತ್ರ
ಸುಪ್ತ
ಸುಪ್ತಪ್ರಜ್ಞೆ-ಯಲ್ಲಿ-ರುವ
ಸುಪ್ತ-ರೂಪ-ದಿಂದ
ಸುಪ್ತಾವಸ್ಥೆಯ
ಸುಪ್ಪತ್ತಿಗೆ-ಯಲ್ಲಿ
ಸುಪ್ರತಿಷ್ಠಿತ-ವಾಗಿತ್ತು
ಸುಬಲನು
ಸುಬೋಧಾ-ನಂದ
ಸುಮಧುರ
ಸುಮಧು-ರ-ಕಂಠ
ಸುಮಧು-ರೈರ್ಮೇಘ
ಸುಮಾತ್ರ
ಸುಮಾ-ರಾಗಿ
ಸುಮಾರು
ಸುಮೇರು-ವನ್ನು
ಸುಮ್ಮ-ನಾದನು
ಸುಮ್ಮ-ನಾದರು
ಸುಮ್ಮನಿದ್ದರು
ಸುಮ್ಮ-ನಿದ್ದು
ಸುಮ್ಮನಿ-ರಲು
ಸುಮ್ಮನಿ-ರಿ-ಸದೆ
ಸುಮ್ಮನಿ-ರಿ-ಸಿ-ದರು
ಸುಮ್ಮನೆ
ಸುಮ್ಮನೇ
ಸುಯೋಗ
ಸುಯೋಗ-ವಿತ್ತು
ಸುಯೋಗವೂ
ಸುರಕ್ಷಿತ-ಳಾಗಿ-ರುವಳು
ಸುರಕ್ಷಿತ-ವಾದ
ಸುರರ
ಸುರಿ-ದರು
ಸುರಿ-ದಿತ್ತು
ಸುರಿದು
ಸುರಿದೆ
ಸುರಿ-ಮಳೆ
ಸುರಿ-ಮಳೆ-ಯನ್ನೂ
ಸುರಿ-ಮಳೆ-ಯನ್ನೇ
ಸುರಿ-ಮಳೆಯು
ಸುರಿ-ಯುತಿರೆ
ಸುರಿಯುತ್ತಿ-ರುತ್ತದೆ
ಸುರಿ-ಯು-ವುದು
ಸುರಿವ
ಸುರಿ-ವುದು
ಸುರಿ-ವುದೊ
ಸುರಿಸಿ-ಕೊಂಡು
ಸುರಿಸಿದ
ಸುರಿ-ಸುತ್ತಿದ್ದ
ಸುರಿ-ಸುವರು
ಸುರುಳಿ
ಸುರೇಂದ್ರ-ನಾಥ
ಸುರೇಂದ್ರ-ನಾಥ-ಸೇನ್
ಸುರೇಶ-ಬಾಬು-ಗಳ
ಸುರೇ-ಶಮಿತ್ರ
ಸುರೇಶ್
ಸುಲಭ
ಸುಲಭದ
ಸುಲಭ-ದಲಿ
ಸುಲಭ-ವಲ್ಲ
ಸುಲಭ-ವಾಗಿ
ಸುಲಭ-ವಾಗಿದ್ದರೆ
ಸುಲಭ-ವಾಗಿಯೇ
ಸುಲಭ-ವಾಗು-ವುದು
ಸುಲಭ-ವಾದ
ಸುಲ-ಭವೆ
ಸುಲಭ-ವೆಂದು
ಸುಲಭ-ವೇ-ನಪ್ಪಾ
ಸುಲ-ಭವೋ
ಸುಲಲಿತ-ವಾಗಿಯೂ
ಸುಲಲಿತ-ವಾದ
ಸುಳಿ-ಗಳಿಂದ
ಸುಳಿಗೆ
ಸುಳಿ-ದಿ-ರಲು
ಸುಳಿದು
ಸುಳಿ-ಯದಿದ್ದರೆ
ಸುಳಿ-ಯದು
ಸುಳಿ-ಯದೊ
ಸುಳಿ-ಯ-ಲಾರದು
ಸುಳಿ-ಯ-ಲಾ-ರವು
ಸುಳಿ-ಯಲೂ
ಸುಳಿ-ಯುತ್ತಿ-ರ-ಲಿಲ್ಲ
ಸುಳಿ-ಯು-ವು-ದಿಲ್ಲ
ಸುಳಿವ
ಸುಳಿ-ವನ್ನು
ಸುಳಿವು
ಸುಳಿ-ವುದೊ
ಸುಳಿವೆ
ಸುಳಿವೇ
ಸುಳ್ಳನ್ನು
ಸುಳ್ಳಲ್ಲ
ಸುಳ್ಳಾಗಿ
ಸುಳ್ಳಾಗಿ-ರ-ಬ-ಹುದು
ಸುಳ್ಳು
ಸುಳ್ಳೆಂದು
ಸುವಾಸನಾ
ಸುವಿದಧೇ
ಸುವಿದಾಯ
ಸುವಿ-ಪುಲಂ
ಸುವಿಮಲ
ಸುವಿಮಲಂ
ಸುವಿಸ್ತ್ರತ
ಸುವಿ-ಹಿತ
ಸುವ್ಯವಸ್ಥಿತ-ವಾದ
ಸುಶಿಕ್ಷಿತ-ರಾಗಿದ್ದಾರೆ
ಸುಶಿಕ್ಷಿತಳೂ
ಸುಶ್ರುತ
ಸುಷುಮ್ನಾ
ಸುಷುಮ್ನಾ-ವನ್ನು
ಸುಸಂಸ್ಕೃತ
ಸುಸಂಸ್ಕೃ-ತರ
ಸುಸಂಸ್ಕೃ-ತರು
ಸುಸಭ್ಯ-ವಾದ
ಸುಸ್ಪಷ್ಟ-ವಾಗಿ
ಸುಸ್ವರ
ಸುಸ್ವಾಗ-ತವ
ಸೂಕ್ತ
ಸೂಕ್ತ-ವಾಗಿತ್ತು
ಸೂಕ್ತ-ವಾಗಿವೆ
ಸೂಕ್ತ-ವಾದ
ಸೂಕ್ತ-ವೆಂದು
ಸೂಕ್ಷ್ಮ
ಸೂಕ್ಷ್ಮಕ್ರಿ-ಮಿ-ಗಳನ್ನೂ
ಸೂಕ್ಷ್ಮ-ಜೀವಿ-ಗಳನ್ನು
ಸೂಕ್ಷ್ಮ-ತಮ
ಸೂಕ್ಷ್ಮ-ಭಾವ
ಸೂಕ್ಷ್ಮ-ರೂಪ
ಸೂಕ್ಷ್ಮ-ರೂಪವ
ಸೂಕ್ಷ್ಮ-ರೂಪವು
ಸೂಕ್ಷ್ಮ-ವಾಗಿ
ಸೂಕ್ಷ್ಮ-ವಾಗಿ-ರುತ್ತವೆ
ಸೂಕ್ಷ್ಮ-ವಾಗುತ್ತದೆ
ಸೂಕ್ಷ್ಮ-ವಾದ
ಸೂಕ್ಷ್ಮ-ಶರೀರ-ದಲ್ಲಿ
ಸೂಕ್ಷ್ಮಸ್ಥಿತಿ-ಯಲ್ಲಿ
ಸೂಕ್ಷ್ಮಸ್ವ-ರೂಪ-ಗಳ
ಸೂಕ್ಷ್ಮ-ಹೊದಿಕೆ
ಸೂಕ್ಷ್ಮಾಂಶ
ಸೂಕ್ಷ್ಮಾಣು
ಸೂಕ್ಷ್ಮಾವಸ್ಥೆ
ಸೂಗದ
ಸೂಚಕ
ಸೂಚನೆ
ಸೂಚನೆ-ಯನ್ನಾಗಲೀ
ಸೂಚನೆ-ಯನ್ನು
ಸೂಚಿ-ತ-ವಾಗಿ-ರು-ವು-ದ-ರಿಂದಲೇ
ಸೂಚಿಸ-ಬಹುದೇ
ಸೂಚಿ-ಸಲು
ಸೂಚಿಸಿ
ಸೂಚಿ-ಸಿದ
ಸೂಚಿಸಿ-ದನು
ಸೂಚಿಸಿ-ದ-ರೇನು
ಸೂಚಿ-ಸಿಲ್ಲ
ಸೂಚಿ-ಸುತ್ತವೆ
ಸೂಚಿ-ಸುತ್ತ-ವೆಂದೇ-ನಾದರೂ
ಸೂಚಿ-ಸುವ
ಸೂಚಿ-ಸು-ವುದು
ಸೂಚಿ-ಸು-ವುವು
ಸೂಜಿ
ಸೂತ್ರ
ಸೂತ್ರ-ಗಳನ್ನು
ಸೂತ್ರ-ಗಳಲ್ಲಿ
ಸೂತ್ರ-ಗಳು
ಸೂತ್ರದ
ಸೂತ್ರ-ದಿಂದ
ಸೂತ್ರ-ಧಾರ
ಸೂತ್ರ-ಪಾತ-ವಾ-ಯಿತು
ಸೂತ್ರ-ವಿಲ್ಲ-ದಲೆ
ಸೂತ್ರವು
ಸೂಪರಿಂಟೆಂಡೆಂಟ-ರಾದ
ಸೂರೆಗೊಳ್ಳುತ್ತಿವೆ
ಸೂರೆ-ಗೊಳ್ಳುವ
ಸೂರೆ-ಯಾಗಿ-ರು-ವುದು
ಸೂರ್ಯ
ಸೂರ್ಯಃ
ಸೂರ್ಯಗ್ರಹಣ
ಸೂರ್ಯಗ್ರಹ-ಣ-ವಾ-ದು-ದ-ರಿಂದ
ಸೂರ್ಯ-ಚಂದ್ರರು
ಸೂರ್ಯ-ಚಂದ್ರರೆ
ಸೂರ್ಯತ್ವ
ಸೂರ್ಯನ
ಸೂರ್ಯ-ನಂತಿದ್ದಾರೆ
ಸೂರ್ಯ-ನಂತೆ
ಸೂರ್ಯ-ನನ್ನೇ
ಸೂರ್ಯ-ನಿಗೂ
ಸೂರ್ಯ-ನಿಗೆ
ಸೂರ್ಯನು
ಸೂರ್ಯರ
ಸೂರ್ಯ-ಲೋಕ
ಸೂರ್ಯ-ಲೋ-ಕ-ವನ್ನು
ಸೂರ್ಯಾ-ಚಂದ್ರ-ಮಸೌ
ಸೂರ್ಯೋದ-ಯಕ್ಕೆ
ಸೂರ್ಯೋದಯ-ವಾಗಿದೆ
ಸೂಳೆ
ಸೂಳೆ-ಯಲ್ಲಿ
ಸೃಷ್ಟಿ
ಸೃಷ್ಟಿಗೆ
ಸೃಷ್ಟಿಜ್ಞಾನವು
ಸೃಷ್ಟಿ-ಮಾಡ-ಬೇಕು
ಸೃಷ್ಟಿಯ
ಸೃಷ್ಟಿ-ಯನ್ನು
ಸೃಷ್ಟಿ-ಯಲ್ಲಿ
ಸೃಷ್ಟಿ-ಯಾಗಿ
ಸೃಷ್ಟಿ-ಯಾ-ಗಿದೆ
ಸೃಷ್ಟಿ-ಯಾಗಿದ್ದು
ಸೃಷ್ಟಿ-ಯಾಗುವುದು
ಸೃಷ್ಟಿ-ಯಾದದ್ದೆಂದರು
ಸೃಷ್ಟಿ-ಯಾ-ಯಿತು
ಸೃಷ್ಟಿಯು
ಸೃಷ್ಟಿಯೂ
ಸೃಷ್ಟಿಯೇ
ಸೃಷ್ಟಿ-ರೂಪ
ಸೃಷ್ಟಿ-ಸ-ಬಲ್ಲ
ಸೃಷ್ಟಿ-ಸ-ಲಾ-ರರು
ಸೃಷ್ಟಿ-ಸ-ಲಿಲ್ಲ
ಸೃಷ್ಟಿ-ಸಲ್ಪಟ್ಟಿದೆ
ಸೃಷ್ಟಿ-ಸಿ-ಕೊಂಡಿರುವೆ
ಸೃಷ್ಟಿ-ಸಿ-ಕೊಳ್ಳುವುದು
ಸೃಷ್ಟಿ-ಸಿದ
ಸೃಷ್ಟಿ-ಸಿ-ದನೊ
ಸೃಷ್ಟಿ-ಸಿದ-ಳೆಂದು
ಸೃಷ್ಟಿ-ಸಿದ್ದಾನೆ
ಸೃಷ್ಟಿ-ಸಿದ್ದೇವೆಯೆ
ಸೃಷ್ಟಿ-ಸಿ-ರುವ-ನೆಂದೂ
ಸೃಷ್ಟಿ-ಸಿಲ್ಲ
ಸೃಷ್ಟಿ-ಸುತ್ತಿದ್ದೀರಿ
ಸೃಷ್ಟಿ-ಸುವ
ಸೃಷ್ಟಿ-ಸು-ವು-ದಕ್ಕೂ
ಸೆ
ಸೆಕೆ
ಸೆಣ-ಸಲಾ-ರಂಭಿ-ಸುತ್ತಾನೆ
ಸೆಣಸಾಡ-ಬೇಕಾ-ಗುತ್ತದೆ
ಸೆಣಸಿ
ಸೆಪ್ಟೆಂಬರ್
ಸೆಪ್ಟೆಂಬರ್ರಂದು
ಸೆರಗೊಡ್ಡಿ
ಸೆರೆ-ಮ-ನೆಯಿಂ
ಸೆರೆ-ಯೊ-ಳಗೆ
ಸೆರೆ-ಯೊಳು
ಸೆಳೆದು-ಕೊಳ್ಳ-ಬೇಕು
ಸೆಳೆದು-ಕೊಳ್ಳುವರೆ
ಸೆಳೆ-ದುಕೋ
ಸೆಳೆ-ಯುತಿ-ಹುದೆನ್ನನು
ಸೆಳೆ-ಯುವ
ಸೆಳೆ-ಯುವೆ
ಸೆಳೆವ
ಸೇ
ಸೇಜನ
ಸೇತುವು
ಸೇತುವೆ
ಸೇತು-ವೆಯ
ಸೇತುವೆ-ಯನ್ನು
ಸೇಥಾ
ಸೇದದೆ
ಸೇದ-ಬೇಕೆಂಬ
ಸೇದಲು
ಸೇದಿ
ಸೇದುತ್ತಾ
ಸೇಯಿ
ಸೇರ-ದಂತೆ
ಸೇರ-ದಿ-ರು-ವುದು
ಸೇರದು
ಸೇರಬಯ-ಸಿ-ದರು
ಸೇರ-ಬಲ್ಲೆ
ಸೇರ-ಬೇಕು
ಸೇರ-ಬೇಕೆಂದು
ಸೇರ-ಬೇಕೆಂಬ
ಸೇರಲು
ಸೇರಲೇ-ಬೇಕು
ಸೇರಲೊಂದೇ
ಸೇರಿ
ಸೇರಿದ
ಸೇರಿ-ದಂತೆ
ಸೇರಿ-ದ-ಮೇಲೆ
ಸೇರಿ-ದರು
ಸೇರಿ-ದರೆ
ಸೇರಿ-ದ-ವನು
ಸೇರಿ-ದ-ವ-ರಲ್ಲ
ಸೇರಿ-ದ-ವ-ರಲ್ಲಿ
ಸೇರಿ-ದ-ವ-ರಲ್ಲಿಯೇ
ಸೇರಿ-ದ-ವ-ರಾಗಿ-ರುತ್ತಾರೆ
ಸೇರಿ-ದ-ವರು
ಸೇರಿ-ದ-ವ-ರೆಂದು
ಸೇರಿ-ದಾಗ
ಸೇರಿ-ದುದು
ಸೇರಿ-ದು-ದೆಂದು
ಸೇರಿದೆ
ಸೇರಿದ್ದ
ಸೇರಿದ್ದರು
ಸೇರಿದ್ದರೂ
ಸೇರಿದ್ದರೆ
ಸೇರಿದ್ದರೋ
ಸೇರಿದ್ದಾಗಿದೆ
ಸೇರಿದ್ದಾ-ರಲ್ಲಾ
ಸೇರಿದ್ದಾರೆ
ಸೇರಿದ್ದು
ಸೇರಿಯೇ
ಸೇರಿ-ರಲಿ
ಸೇರಿ-ರ-ಲಿಲ್ಲ
ಸೇರಿ-ರಲಿಲ್ಲ-ವೆಂದೂ
ಸೇರಿ-ರುವ
ಸೇರಿ-ರುವ-ವ-ರಲ್ಲಿ
ಸೇರಿಲ್ಲ-ವಲ್ಲ
ಸೇರಿ-ಸದೆ
ಸೇರಿ-ಸ-ಬ-ಹುದು
ಸೇರಿ-ಸ-ಬೇಕು
ಸೇರಿ-ಸ-ಬೇಕೆಂದಿದ್ದೇನೆ
ಸೇರಿ-ಸ-ಲಾಗಿತ್ತು
ಸೇರಿಸಿ
ಸೇರಿ-ಸಿ-ಕೊಂಡರು
ಸೇರಿ-ಸಿ-ಕೊಂಡು
ಸೇರಿ-ಸಿ-ಕೊಂಡು-ಬಿಟ್ಟು
ಸೇರಿ-ಸಿ-ಕೊಳ್ಳು-ವು-ದಿಲ್ಲ
ಸೇರಿ-ಸಿ-ದಂತಾಗುತ್ತ-ದಷ್ಟೇ
ಸೇರಿ-ಸಿ-ದಾಗ
ಸೇರಿ-ಸಿದ್ದು
ಸೇರಿ-ಸಿರಿ
ಸೇರಿ-ಸಿ-ರು-ವುದೇ
ಸೇರಿಸು
ಸೇರಿ-ಸು-ವುದಕ್ಕೆ
ಸೇರಿ-ಸು-ವುದ-ರಲ್ಲಿ
ಸೇರಿ-ಸು-ವುದು
ಸೇರಿ-ಹೋಗುತ್ತದೆ
ಸೇರಿ-ಹೋಗು-ವರು
ಸೇರಿ-ಹೋಗು-ವುವು
ಸೇರಿ-ಹೋ-ಯಿತು
ಸೇರುತ್ತಿತ್ತು
ಸೇರುತ್ತಿದ್ದರು
ಸೇರುವ
ಸೇರು-ವಂತೆ
ಸೇರುವನು
ಸೇರುವರು
ಸೇರುವಿ-ರೆಂಬುದು
ಸೇರು-ವುದಕ್ಕೆ
ಸೇರು-ವು-ದರ
ಸೇರು-ವು-ದಿಲ್ಲ
ಸೇರು-ವುದು
ಸೇರು-ವುದೋ
ಸೇರು-ವುವೋ
ಸೇರು-ವೆನು
ಸೇವಕ
ಸೇವಕ-ನಿಗೆ
ಸೇವ-ಕನು
ಸೇವ-ಕರ
ಸೇವಾ-ಧರ್ಮ
ಸೇವಾ-ಧರ್ಮ-ವನ್ನು
ಸೇವಾಧಿ-ಕಾರ
ಸೇವಾಧಿ-ಕಾರ-ವನ್ನು
ಸೇವಾ-ಪರ-ನಾಗಿ
ಸೇವಾ-ಭಾವ
ಸೇವಾ-ಭಾವ-ದಿಂದ
ಸೇವಾಶ್ರಮ
ಸೇವಾಶ್ರಮ-ವನ್ನು
ಸೇವಾಶ್ರಮವು
ಸೇವಾ-ಸತ್ರ-ದಲ್ಲಿ
ಸೇವಾ-ಸಾರೈ-ರಭಿನುತಂ
ಸೇವಿಛೆ
ಸೇವಿಸ-ಬೇಕು
ಸೇವಿ-ಸಲು
ಸೇವಿಸಿ-ದರೂ
ಸೇವಿ-ಸಿದರೆ
ಸೇವಿ-ಸುತ್ತ
ಸೇವಿ-ಸುತ್ತಾ
ಸೇವಿ-ಸುತ್ತಿದ್ದಾರೆ
ಸೇವಿ-ಸುತ್ತಿರು-ವಿ-ರೆಂದು
ಸೇವಿ-ಸುವ
ಸೇವಿ-ಸು-ವಂತೆ
ಸೇವಿ-ಸುವರು
ಸೇವಿಸು-ವಿರಿ
ಸೇವಿ-ಸು-ವು-ದ-ರಿಂದ
ಸೇವಿಸ್ತಾರ
ಸೇವೆ
ಸೇವೆ-ಗಳ
ಸೇವೆ-ಗಾಗಿ
ಸೇವೆ-ಗಿಂತ
ಸೇವೆಗೂ
ಸೇವೆಗೆ
ಸೇವೆ-ಗೋಸ್ಕರ
ಸೇವೆ-ಮಾ-ಡಲು
ಸೇವೆ-ಮಾಡುತ್ತ
ಸೇವೆ-ಮಾಡುತ್ತಾ
ಸೇವೆ-ಮಾಡುತ್ತಿದ್ದಾಗ
ಸೇವೆ-ಮಾಡು-ವು-ದರ
ಸೇವೆಯ
ಸೇವೆ-ಯನ್ನು
ಸೇವೆ-ಯನ್ನೂ
ಸೇವೆ-ಯಲ್ಲಿ
ಸೇವೆಯು
ಸೇವೆಯೆ
ಸೇವೆಯೇ
ಸೇವೆ-ಸಲ್ಲಿ-ಸ-ಬೇಕು
ಸೇವೆ-ಸಲ್ಲಿ-ಸಿ-ದಂತೆ
ಸೇವೋ
ಸೇವೋ-ತಬು-ಲಾಗೆ
ಸೈತಾನ
ಸೈತಾನ-ನಿಂದ
ಸೈತಾ-ನನೇ
ಸೈನಿ-ಕರು
ಸೈನ್ಯ-ಗಳ
ಸೊಂಟ-ಕಟ್ಟಿ
ಸೊಂಟ-ದಲ್ಲಿ
ಸೊಂಪಾಗಿ
ಸೊಗ-ದಲಿ
ಸೊಗ-ದೊಲವನು
ಸೊಗವ
ಸೊಗಸಾಗಿ
ಸೊನ್ನೆಯಲೆ
ಸೊನ್ನೆಯಿಂ
ಸೊನ್ನೆಯೆ
ಸೊಪ್ಪು
ಸೊಬ-ಗನ್ನು
ಸೊಬಗು
ಸೊರಗಿ
ಸೊಳ್ಳೆಗೆ
ಸೊಸೆಯರು
ಸೊಹಾಗ್
ಸೋಂಕಿಸಿ
ಸೋಂಕು
ಸೋಂಕೇ
ಸೋಕಿ
ಸೋಕಿ-ಸಿ-ಕೊಳ್ಳುತ್ತಿದ್ದರು
ಸೋಕೆ
ಸೋಗಿ-ನಲ್ಲಿ
ಸೋಗು
ಸೋಗೆಂದು
ಸೋಗೆಯ
ಸೋಜಿಗ
ಸೋಜಿಗ-ವೇ-ನಲ್ಲ
ಸೋತಿದೆ
ಸೋತು
ಸೋದರ-ತೆ-ಯನು
ಸೋದರ-ಳಿಯ
ಸೋದರ-ಸಂನ್ಯಾಸಿ-ಗಳಿಗೆ
ಸೋದರಿ
ಸೋದರಿ-ಯ-ರಿಗೆ
ಸೋದರಿ-ಯರು
ಸೋಪಾನ
ಸೋಪಾನ-ಗಳ
ಸೋಪಾನ-ಗಳಿವೆ
ಸೋಪಾನ-ಗಳು
ಸೋಮಾರಿ-ಗಳಂತೆ
ಸೋಮಾರಿ-ಗಳಾಗಿ
ಸೋಮಾರಿ-ಗಳು
ಸೋಮಾರಿ-ತನ
ಸೋಮಾರಿ-ತನಕ್ಕಾಗಿ
ಸೋಮಾರಿ-ತ-ನಕ್ಕೆ
ಸೋಮಾರಿ-ತನ-ದಿಂದ
ಸೋಮಾರಿ-ಯಂತೆ
ಸೋಮಾರಿ-ಯಾಗಿ
ಸೋರೆಬುರುಡೆಯ
ಸೋಲನ್ನು
ಸೋಲಾ-ಯಿತೆಂದು
ಸೋಲಿ-ಸಲು
ಸೋಲಿ-ಸಲ್ಪಟ್ಟನು
ಸೋಲಿಸಿ-ಬಿಟ್ಟು
ಸೋಲಿ-ಸುವ
ಸೋಲುಂಡು
ಸೋಲು-ತಲಿದ್ದರು
ಸೋಲುವ
ಸೋಲೊಪ್ಪದ
ಸೋಹಂ
ಸೋಽಯಂ
ಸೋಽಹಂ
ಸೋಽಹಮಸ್ಮಿ
ಸೌಂದರ್ಯ
ಸೌಂದರ್ಯ-ಗಳನ್ನು
ಸೌಂದರ್ಯದ
ಸೌಂದರ್ಯ-ದಲ್ಲಿ
ಸೌಂದರ್ಯದಿ
ಸೌಂದರ್ಯಪ್ರತಿ-ಬಿಂಬ-ದಲ್ಲಿ
ಸೌಂದರ್ಯ-ವನ್ನು
ಸೌಂದರ್ಯ-ವಿದೆ
ಸೌಜನ್ಯ-ವನ್ನು
ಸೌದೆ
ಸೌಮ್ಯ
ಸೌಮ್ಯ-ವಾಗಿ
ಸೌರಭೆ
ಸೌರವ್ಯೂಹ-ಗಳನ್ನೊಳ-ಗೊಂಡ
ಸೌಲಭ್ಯ
ಸೌಹಾರ್ದತೆ-ಯನ್ನು
ಸೌಹಾರ್ದ-ದಿಂದ
ಸ್ಕೂಲಿಗೆ
ಸ್ಕೂಲಿ-ನಲ್ಲಿ
ಸ್ಕೂಲು
ಸ್ಕೂಲು-ಗಳೂ
ಸ್ಖಲನಂ
ಸ್ಟರ್ಜೆಸ್
ಸ್ಟಾರ್
ಸ್ಟಿಮರ್
ಸ್ತಂಭ-ಗಳು
ಸ್ತಂಭೀ-ಭೂತ-ನಾಗಿ
ಸ್ತಂಭೀ-ಭೂತ-ನಾದ
ಸ್ತಂಭೀ-ಭೂತ-ರಾಗಿ
ಸ್ತಂಭೀ-ಭೂತ-ರಾಗಿದ್ದರು
ಸ್ತಂಭೀ-ಭೂತ-ರಾಗಿದ್ದು
ಸ್ತಂಭೀ-ಭೂತ-ರಾದರು
ಸ್ತಂಭೀ-ಭೂತ-ರಾ-ದೆವು
ಸ್ತಂಭೀ-ಭೂತ-ವಾಗಿ
ಸ್ತಬ್ಧ
ಸ್ತಬ್ಧ-ಗೊಳಿಸುತ
ಸ್ತಬ್ಧ-ರಾಗಿ
ಸ್ತಬ್ಧ-ವಾಗಿ
ಸ್ತಬ್ಧ-ವಾಗಿ-ಬಿಡುತ್ತಿತ್ತು
ಸ್ತಬ್ಧ-ವಾಗಿ-ರು-ವುದು
ಸ್ತಬ್ಧೀ-ಕೃತ್ಯ
ಸ್ತರದ
ಸ್ತರದಲ್ಲಿದ್ದರು
ಸ್ತರದಲ್ಲಿ-ರುವ
ಸ್ತರವಿನ್ಯಾಸ-ದಲ್ಲಿ
ಸ್ತಿಮಿತ-ಸಲಿ-ಲ-ರಾಶಿಪ್ರಖ್ಯಮಾಖ್ಯಾವಿ-ಹೀನಂ
ಸ್ತುತಿ
ಸ್ತುತಿ-ಯನ್ನು
ಸ್ತುತಿ-ಸಲಿ
ಸ್ತುತಿ-ಸಲ್ಪಡುತ್ತಿರುವ
ಸ್ತುತಿ-ಸು-ವೆನು
ಸ್ತುತೋ
ಸ್ತುವಂತು
ಸ್ತೂಪ-ಗಳು
ಸ್ತೂಪ-ವನ್ನು
ಸ್ತೋತ್ರ
ಸ್ತೋತ್ರಂ
ಸ್ತೋತ್ರ-ಗಳ
ಸ್ತೋತ್ರ-ಗಳನ್ನು
ಸ್ತೋತ್ರ-ಗಳು
ಸ್ತೋತ್ರ-ದಲ್ಲಿ
ಸ್ತೋತ್ರ-ವನ್ನು
ಸ್ತೋತ್ರ-ವನ್ನೂ
ಸ್ತೋತ್ರ-ವೆಲ್ಲಿ
ಸ್ತೋತ್ರಾಣಿ
ಸ್ತ್ರೀ
ಸ್ತ್ರೀಗೆ
ಸ್ತ್ರೀಪುರುಷರ
ಸ್ತ್ರೀಪುರುಷ-ರಲ್ಲಿ
ಸ್ತ್ರೀಪುರುಷರು
ಸ್ತ್ರೀಪುರುಷ-ರೆಂದು
ಸ್ತ್ರೀಪುರುಷ-ರೆಂಬ
ಸ್ತ್ರೀಪುರುಷ-ರೊ-ಳಗೆ
ಸ್ತ್ರೀಪೂಜೆ-ಯನ್ನು
ಸ್ತ್ರೀಯನ್ನು
ಸ್ತ್ರೀಯರ
ಸ್ತ್ರೀಯ-ರನ್ನು
ಸ್ತ್ರೀಯ-ರನ್ನೂ
ಸ್ತ್ರೀಯ-ರಲ್ಲಿ
ಸ್ತ್ರೀಯ-ರಾಗ-ಬಲ್ಲರು
ಸ್ತ್ರೀಯ-ರಿ-ಗಾಗಿ
ಸ್ತ್ರೀಯ-ರಿಗೆ
ಸ್ತ್ರೀಯರು
ಸ್ತ್ರೀಯರೂ
ಸ್ತ್ರೀಯ-ರೆಲ್ಲ
ಸ್ತ್ರೀಯರೇ
ಸ್ತ್ರೀಯ-ರೊಡನೆ
ಸ್ತ್ರೀಯು
ಸ್ತ್ರೀಲಂಪ-ಟ-ವಾಗಿದೆ
ಸ್ತ್ರೀವಿದ್ಯಾಭ್ಯಾಸ-ದಲ್ಲಿ
ಸ್ಥಂಭೀ-ಭೂತ-ನಾಗಿ
ಸ್ಥರ-ದಲ್ಲಿ-ರುವ
ಸ್ಥಳ
ಸ್ಥಳಕ್ಕೆ
ಸ್ಥಳ-ಗಳಲ್ಲಿ
ಸ್ಥಳ-ಗಳಲ್ಲಿಯೇ
ಸ್ಥಳ-ಗಳಿಗೆ
ಸ್ಥಳದ
ಸ್ಥಳ-ದಲ್ಲಿ
ಸ್ಥಳ-ದಲ್ಲಿ-ಡಲು
ಸ್ಥಳ-ದಲ್ಲಿ-ಡು-ವರು
ಸ್ಥಳ-ದಲ್ಲಿಯೂ
ಸ್ಥಳ-ದಿಂದ
ಸ್ಥಳ-ವನ್ನು
ಸ್ಥಳ-ವನ್ನೂ
ಸ್ಥಳ-ವಾಗು-ವುದು
ಸ್ಥಳ-ವಿದೆ
ಸ್ಥಳ-ವಿದೆ-ಯಲ್ಲಾ
ಸ್ಥಳವು
ಸ್ಥಳ-ವೆಲ್ಲಿದೆ
ಸ್ಥಳವೇ
ಸ್ಥಳೀಯ
ಸ್ಥಾನ
ಸ್ಥಾನಕ್ಕಾಗಿ
ಸ್ಥಾನಕ್ಕೆ
ಸ್ಥಾನ-ಗಳ
ಸ್ಥಾನ-ಗಳನ್ನು
ಸ್ಥಾನ-ಗಳಿಗೆ
ಸ್ಥಾನದ
ಸ್ಥಾನ-ದಲ್ಲಿ
ಸ್ಥಾನ-ದಲ್ಲಿದ್ದರೂ
ಸ್ಥಾನ-ವನ್ನಾಕ್ರಮಿ-ಸಲು
ಸ್ಥಾನ-ವನ್ನು
ಸ್ಥಾನ-ವಾಗು-ವುದು
ಸ್ಥಾನ-ವಾ-ವುದು
ಸ್ಥಾನ-ವಿದ್ದು
ಸ್ಥಾನವೂ
ಸ್ಥಾನ-ವೊಂದು
ಸ್ಥಾಪಕ
ಸ್ಥಾಪಕ-ನೆಂದು
ಸ್ಥಾಪಕ-ಳಾದ
ಸ್ಥಾಪ-ಕಾಯ
ಸ್ಥಾಪನೆ
ಸ್ಥಾಪನೆ-ಯಾಗ-ಬೇಕು
ಸ್ಥಾಪನೆ-ಯಾದಾಗ
ಸ್ಥಾಪಿತ-ವಾಗ-ದಿದ್ದುದು
ಸ್ಥಾಪಿ-ತ-ವಾಗಿ-ರು-ವು-ದ-ರಿಂದ
ಸ್ಥಾಪಿತ-ವಾದ
ಸ್ಥಾಪಿತ-ವಾ-ದಲ್ಲಿ
ಸ್ಥಾಪಿಸ-ಬೇಕಾಗಿದೆ
ಸ್ಥಾಪಿಸ-ಬೇಕು
ಸ್ಥಾಪಿಸ-ಬೇಕೆಂದು
ಸ್ಥಾಪಿ-ಸಲು
ಸ್ಥಾಪಿ-ಸಲ್ಪಟ್ಟ
ಸ್ಥಾಪಿ-ಸಲ್ಪಟ್ಟಿ-ರುವುದು
ಸ್ಥಾಪಿ-ಸಲ್ಪಟ್ಟಿವೆ
ಸ್ಥಾಪಿ-ಸಲ್ಪಡು-ವುದು
ಸ್ಥಾಪಿಸಿ
ಸ್ಥಾಪಿಸಿ-ದು-ದ-ರಿಂದ
ಸ್ಥಾಪಿ-ಸಿದ್ದ
ಸ್ಥಾಪಿಸಿದ್ದಾರೆ
ಸ್ಥಾಪಿ-ಸುತ್ತೇನೆ
ಸ್ಥಾಪಿ-ಸು-ವುದಕ್ಕಾಗಿ
ಸ್ಥಾಪಿ-ಸು-ವುದಕ್ಕೆ
ಸ್ಥಾಪಿ-ಸು-ವುದು
ಸ್ಥಾಪಿ-ಸು-ವುದೇ
ಸ್ಥಾಪಿ-ಸುವೆ
ಸ್ಥಾಪಿ-ಸು-ವೆನು
ಸ್ಥಾಯಿ-ಯಾಗಿ
ಸ್ಥಾವರ
ಸ್ಥಿತ-ವಾದ-ಮೇಲೆ
ಸ್ಥಿತಾ
ಸ್ಥಿತಿ
ಸ್ಥಿತಿ-ಗತಿ-ಗಳೂ
ಸ್ಥಿತಿ-ಗತಿ-ಗಳೆ-ನಿತೆಂದು
ಸ್ಥಿತಿ-ಗನು-ಗುಣ-ವಾಗಿ
ಸ್ಥಿತಿ-ಗಳ
ಸ್ಥಿತಿ-ಗಳನ್ನೆಲ್ಲಾ
ಸ್ಥಿತಿ-ಗಳಲ್ಲಿ
ಸ್ಥಿತಿ-ಗಳೂ
ಸ್ಥಿತಿ-ಗಿಳಿ-ದಿಲ್ಲ
ಸ್ಥಿತಿಗೆ
ಸ್ಥಿತಿಯ
ಸ್ಥಿತಿ-ಯನ್ನು
ಸ್ಥಿತಿ-ಯನ್ನೇ
ಸ್ಥಿತಿ-ಯಲ್ಲಿ
ಸ್ಥಿತಿ-ಯಲ್ಲಿತ್ತು
ಸ್ಥಿತಿ-ಯಲ್ಲಿದೆ
ಸ್ಥಿತಿ-ಯಲ್ಲಿದ್ದರು
ಸ್ಥಿತಿ-ಯಲ್ಲಿಯೂ
ಸ್ಥಿತಿ-ಯಲ್ಲಿಯೇ
ಸ್ಥಿತಿ-ಯಲ್ಲಿ-ರ-ಬೇಕು
ಸ್ಥಿತಿ-ಯಲ್ಲಿ-ರಲಿ
ಸ್ಥಿತಿ-ಯಲ್ಲಿ-ರ-ಲಿಲ್ಲ
ಸ್ಥಿತಿ-ಯಲ್ಲಿ-ರುತ್ತಾನೆ
ಸ್ಥಿತಿ-ಯಲ್ಲಿ-ರುವರು
ಸ್ಥಿತಿ-ಯಲ್ಲಿ-ರು-ವು-ದ-ರಿಂದ
ಸ್ಥಿತಿ-ಯಷ್ಟೇ
ಸ್ಥಿತಿ-ಯಿಂದ
ಸ್ಥಿತಿಯು
ಸ್ಥಿತಿ-ಯುಂಟಾ-ದರೆ
ಸ್ಥಿತಿ-ಯೆಲ್ಲಾ
ಸ್ಥಿತಿಯೇ
ಸ್ಥಿರ
ಸ್ಥಿರ-ಗೊಳಿ-ಸಲು
ಸ್ಥಿರ-ಚಿತ್ತ
ಸ್ಥಿರ-ತೆ-ಯಿತ್ತು
ಸ್ಥಿರದಿ
ಸ್ಥಿರ-ನಾಗಿ-ಬಿಟ್ಟಿದ್ದನ್ನು
ಸ್ಥಿರ-ಪಡಿ-ಸಲು
ಸ್ಥಿರ-ಪಡಿ-ಸಿಕೊಳ್ಳ-ಬೇಕೆಂಬ
ಸ್ಥಿರ-ಬುದ್ಧಿ-ಯಿಂದ
ಸ್ಥಿರ-ಮಾಡಿ-ಕೊಂಡೆ
ಸ್ಥಿರ-ರಾಗಿ
ಸ್ಥಿರ-ವಲ್ಲ
ಸ್ಥಿರ-ವಾಗಿ
ಸ್ಥಿರ-ವಾಗಿದ್ದರು
ಸ್ಥಿರ-ವಾಗಿ-ರ-ಲಿಲ್ಲ
ಸ್ಥಿರ-ವಾಗಿರು
ಸ್ಥಿರ-ವಾಗಿ-ರು-ವು-ದಿಲ್ಲ
ಸ್ಥಿರ-ವಾದ
ಸ್ಥಿರವೂ
ಸ್ಥೂಲ
ಸ್ಥೂಲ-ದೇಹ-ದಲ್ಲಿ
ಸ್ಥೂಲ-ಪದಾರ್ಥ-ಗಳ
ಸ್ಥೂಲ-ರೂಪ
ಸ್ಥೂಲ-ರೂಪ-ದಲ್ಲಿ
ಸ್ಥೂಲ-ವಾಗಿದೆ
ಸ್ಥೂಲ-ವಾಗು-ವುದು
ಸ್ಥೂಲ-ವಾದ
ಸ್ಥೂಲ-ವಾ-ದದ್ದು
ಸ್ಥೂಲ-ವಾ-ದು-ದೆಲ್ಲ-ವನ್ನೂ
ಸ್ಥೂಲ-ವಿಕಾಸ
ಸ್ಥೂಲಾಂಶ
ಸ್ಥೂಲಾ-ಕಾರ-ವನ್ನು
ಸ್ಥೈರ್ಯ-ದಿಂದ
ಸ್ನಾನ
ಸ್ನಾನಕ್ಕೆ
ಸ್ನಾನ-ಮಾಡಿ
ಸ್ನಾನ-ಮಾಡಿ-ಕೊಂಡು
ಸ್ನಾನ-ಮಾಡಿದ್ದಾರೆ
ಸ್ನಾನ-ವನ್ನೇ
ಸ್ನಾನ-ವಾದ
ಸ್ನಿಗ್ಧ
ಸ್ನಿಗ್ಧೋಜ್ವಲ
ಸ್ನೇಹ
ಸ್ನೇಹದ
ಸ್ನೇಹ-ಪೂರ್ವ-ಕ-ವಾಗಿ
ಸ್ನೇಹ-ಭಾವ-ನೆ-ಯಿಂದ
ಸ್ನೇಹಾಶೀರ್ವಾದ-ಗಳನ್ನು
ಸ್ನೇಹಾಶೀರ್ವಾದ-ಗಳಿಂದಲೇ
ಸ್ನೇಹಾಶೀರ್ವಾದ-ವನ್ನು
ಸ್ನೇಹಿತ
ಸ್ನೇಹಿತನ
ಸ್ನೇಹಿತ-ನಂತೆ
ಸ್ನೇಹಿತ-ನಾದ
ಸ್ನೇಹಿತ-ನಿಗೆ
ಸ್ನೇಹಿತನು
ಸ್ನೇಹಿತನೆ
ಸ್ನೇಹಿತ-ನೊಬ್ಬ-ನಿಗೆ
ಸ್ನೇಹಿತ-ನೊಬ್ಬನು
ಸ್ನೇಹಿತ-ರಲ್ಲೊಬ್ಬರು
ಸ್ನೇಹಿತರು
ಸ್ನೇಹಿತರೂ
ಸ್ನೇಹಿತ-ರೊಡನೆ
ಸ್ನೇಹಿತ-ರೊಬ್ಬ-ರನ್ನು
ಸ್ನೇಹಿತ-ರೊಬ್ಬ-ರೊಡನೆ
ಸ್ಪಂದನ
ಸ್ಪಂದಿ-ಸಲು
ಸ್ಪಂದಿಸಿ
ಸ್ಪಂದಿಸಿ-ರಲು
ಸ್ಪಂದಿ-ಸುತ್ತಿತ್ತು
ಸ್ಪಟಿಕ
ಸ್ಪಟಿಕ-ದಂತೆ
ಸ್ಪರ್ಧೆ
ಸ್ಪರ್ಧೆ-ಗಳಿಲ್ಲಿ
ಸ್ಪರ್ಧೆಗೆ
ಸ್ಪರ್ಧೆ-ಯನ್ನು
ಸ್ಪರ್ಶ
ಸ್ಪರ್ಶದಿ
ಸ್ಪರ್ಶ-ದಿಂದ
ಸ್ಪರ್ಶ-ಮಣಿ-ಯಂತೆ
ಸ್ಪರ್ಶ-ಮಾಡ-ಲಾ-ರವು
ಸ್ಪರ್ಶ-ಮಾಡೋಣ
ಸ್ಪರ್ಶವು
ಸ್ಪರ್ಶಾನು-ಭೂತಿ-ಯನ್ನು
ಸ್ಪರ್ಶಿ-ಸಲೋಸುಗ
ಸ್ಪರ್ಶೇಂದ್ರಿಯ
ಸ್ಪಳ
ಸ್ಪಷ್ಟ
ಸ್ಪಷ್ಟ-ಪಡಿ-ಸು-ವು-ದಾಗಿದೆ
ಸ್ಪಷ್ಟ-ವಾಗಿ
ಸ್ಪಷ್ಟ-ವಾಗು-ವುದು
ಸ್ಪಷ್ಟ-ವಾಣಿ-ಯೊಳೊ
ಸ್ಪಷ್ಟ-ವಾದ
ಸ್ಪಷ್ಟ-ವಾದರೆ
ಸ್ಪುಟ-ಗೊಳ್ಳುವುವು
ಸ್ಪುರಿ-ಸಿ-ದುವು
ಸ್ಪೂರ್ತಿ-ದಾ-ಯಕ-ವಾದ
ಸ್ಪೆನ್ಸರ್
ಸ್ಪೆನ್ಸರ್ನ
ಸ್ಫುರಿಸಿತ್ತೆಂದು
ಸ್ಫೂರ್ತಗಾ-ಯಕಿಯು
ಸ್ಫೂರ್ತಿ
ಸ್ಫೂರ್ತಿ-ಗೊಂಡು
ಸ್ಫೂರ್ತಿ-ಗೊಳಿಸಿ-ದವು
ಸ್ಫೂರ್ತಿ-ದಾ-ಯಕ
ಸ್ಫೂರ್ತಿ-ನೀಡಿ
ಸ್ಫೂರ್ತಿಯ
ಸ್ಫೂರ್ತಿ-ಯನ್ನು
ಸ್ಫೂರ್ತಿ-ಯಿಂದ
ಸ್ಫೂರ್ತಿ-ಯಿದೆ
ಸ್ಫೂರ್ತಿ-ಯುಂಟಾಗಿತ್ತು
ಸ್ಫೂರ್ತಿ-ಯುತ
ಸ್ಫೂರ್ತಿ-ಯುತ-ವಾದ
ಸ್ಫೂರ್ತಿ-ಯೆಲ್ಲ
ಸ್ಫೂರ್ತಿ-ವಾಣಿಯು
ಸ್ಫೋಟ-ಗೊಳ್ಳುವಂತೆ
ಸ್ಫೋಟಿ-ಸಲು
ಸ್ಮ
ಸ್ಮರಿ-ಸ-ಬಾ-ರದು
ಸ್ಮರಿಸಿ
ಸ್ಮರಿ-ಸಿ-ಕೊಂಡು
ಸ್ಮರಿಸು
ಸ್ಮಶಾನ
ಸ್ಮಶಾನದ
ಸ್ಮಾರ-ಕ-ಗಳು
ಸ್ಮಾರ-ಕ-ವನ್ನು
ಸ್ಮೃತ-ಪುಣ್ಯ-ರಾದ
ಸ್ಮೃತಿ
ಸ್ಮೃತಿ-ಕಾರ-ರಾದ
ಸ್ಮೃತಿ-ಗಳ
ಸ್ಮೃತಿ-ಗಳನ್ನೆಲ್ಲಾ
ಸ್ಮೃತಿ-ಗಳಲ್ಲಿ
ಸ್ಮೃತಿ-ಗಳವು
ಸ್ಮೃತಿ-ಪಠೆ
ಸ್ಮೃತಿ-ಯನ್ನು
ಸ್ಮೃತಿ-ಯಿಂದ
ಸ್ಮೃತಿ-ಯೊಂದು
ಸ್ಯಾತ್
ಸ್ಯಾಮಾಜಿ
ಸ್ಯಾಮಾಜಿ-ಯವರ
ಸ್ಯಾಮೀಜಿ
ಸ್ರೋತ-ದಲಿ
ಸ್ವಂತ
ಸ್ವಂತಕ್ಕಾಗಿ
ಸ್ವಂತ-ವಾಗಿ
ಸ್ವಃ
ಸ್ವಕಲಿ-ತೈರ್ಲಲಿತೈರ್ವಿಲಾಸೈಃ
ಸ್ವಕಲ್ಪಿತ
ಸ್ವಚ್ಛ
ಸ್ವಚ್ಛಂದ
ಸ್ವಚ್ಛ-ವಾ-ದೊಂದು
ಸ್ವತಂತ್ರ
ಸ್ವತಂತ್ರ-ನಾಗಿ
ಸ್ವತಂತ್ರ-ನಾಗಿಯೇ
ಸ್ವತಂತ್ರನು
ಸ್ವತಂತ್ರ-ರಾಗುತ್ತೇವೆಂದು
ಸ್ವತಂತ್ರರು
ಸ್ವತಂತ್ರ-ವಲ್ಲ
ಸ್ವತಂತ್ರ-ವಾಗಿ
ಸ್ವತಂತ್ರ-ವಾಗಿ-ರು-ವು-ದೊಂದು
ಸ್ವತಂತ್ರ-ವಾದ
ಸ್ವತಂತ್ರಾಲೋಚ-ನೆ-ಯಿಲ್ಲದೆ
ಸ್ವತಂತ್ರೇಚ್ಛೆ-ಅಂದರೆ
ಸ್ವತಂತ್ರೈಃ
ಸ್ವತಃ
ಸ್ವತ್ತನ್ನಾಗಿ
ಸ್ವತ್ತಾಗ-ಲಾರದು
ಸ್ವದೇಶ
ಸ್ವದೇಶ-ಗಳಿಗೆ
ಸ್ವದೇಶಿ
ಸ್ವದೇಶೀ-ಯರೇ
ಸ್ವಧರ್ಮ-ವನ್ನೇ
ಸ್ವಪನ
ಸ್ವಪ್ನ
ಸ್ವಪ್ನ-ದಂತಿದೆ
ಸ್ವಪ್ನ-ದಂತೆ
ಸ್ವಪ್ನ-ದಲ್ಲಿಯೂ
ಸ್ವಪ್ನ-ಲೋಕದ
ಸ್ವಪ್ನ-ವನ್ನು
ಸ್ವಪ್ನ-ವೆಲ್ಲಾ
ಸ್ವಪ್ನವೋ
ಸ್ವಪ್ನ-ಸಮ
ಸ್ವಪ್ರ-ಯತ್ನಕ್ಕೆ
ಸ್ವಪ್ರ-ಯತ್ನ-ವನ್ನೂ
ಸ್ವಭಾವ
ಸ್ವಭಾ-ವಕ್ಕೆ
ಸ್ವಭಾವ-ಗಳು
ಸ್ವಭಾವ-ಗಳೂ
ಸ್ವಭಾವತಃ
ಸ್ವಭಾವದ
ಸ್ವಭಾವ-ದಲ್ಲಿ
ಸ್ವಭಾವ-ದ-ವರು
ಸ್ವಭಾವ-ದ-ವರೇ
ಸ್ವಭಾವ-ದಿಂದ
ಸ್ವಭಾವ-ವನ್ನು
ಸ್ವಭಾವ-ವನ್ನೆಲ್ಲ
ಸ್ವಭಾವ-ವನ್ನೇ
ಸ್ವಭಾವ-ವಾದ
ಸ್ವಭಾವ-ವಾದರೋ
ಸ್ವಭಾವ-ವಿರು-ವಂತೆ
ಸ್ವಭಾ-ವವು
ಸ್ವಭಾವ-ವುಳ್ಳ-ವರೋ
ಸ್ವಭಾವ-ವೆನ್ನು-ವುದು
ಸ್ವಭಾವವೇ
ಸ್ವಮೀಜಿ
ಸ್ವಯಂ
ಸ್ವಯಂಪ್ರಕಾಶ-ಮಾನ
ಸ್ವಯಂಪ್ರಕಾಶ-ಮಾನ-ವಾಗಿ
ಸ್ವಯಂಭೂಃ
ಸ್ವರ
ಸ್ವರಕೆ
ಸ್ವರಕ್ಕೆ
ಸ್ವರ-ಗಳ
ಸ್ವರ-ಗಳಲ್ಲಿ
ಸ್ವರ-ದಲಿ
ಸ್ವರ-ದಲ್ಲಿ
ಸ್ವರ-ದಿಂದ
ಸ್ವರ-ಮಯ
ಸ್ವರ-ಮೈತ್ರಿ
ಸ್ವರ-ಮೈತ್ರಿಯು
ಸ್ವರ-ಲಹರಿ-ಯನ್ನು
ಸ್ವರ-ವೀ-ಯುತದು
ಸ್ವರಾಜ್ಯ-ದೊಳು
ಸ್ವರೂಪ
ಸ್ವರೂಪಕ್ಕೆ
ಸ್ವರೂಪ-ನಾಗ-ಬೇಕು
ಸ್ವರೂಪರು
ಸ್ವರೂಪಳೆ
ಸ್ವರೂಪ-ವನ್ನು
ಸ್ವರೂಪ-ವಾದ
ಸ್ವರೂಪವು
ಸ್ವರೂಪವೂ
ಸ್ವರೂಪ-ವೇನು
ಸ್ವರೂಪಿ-ಣಿಯ-ರಾದ
ಸ್ವರೂಪಿ-ಣಿಯಾದ
ಸ್ವರೂಪಿ-ಣಿಯೇ
ಸ್ವರೂಪಿನ
ಸ್ವರ್
ಸ್ವರ್ಗ
ಸ್ವರ್ಗಂ
ಸ್ವರ್ಗಕ್ಕೆ
ಸ್ವರ್ಗ-ಗಳು
ಸ್ವರ್ಗ-ದಲ್ಲಾ-ಗಲಿ
ಸ್ವರ್ಗ-ದಲ್ಲಿಯೂ
ಸ್ವರ್ಗದ್ವಾರ
ಸ್ವರ್ಗ-ಮಯ-ಕೇನಾ
ಸ್ವರ್ಗ-ವಾಸಿ
ಸ್ವರ್ಗಸ್ಥ-ರಾದರು
ಸ್ವರ್ಗೀಯ
ಸ್ವರ್ಣತುಲಿ-ಕರ
ಸ್ವರ್ಣ-ಪಾತ್ರೆ-ಯಲ್ಲಿ
ಸ್ವರ್ಣಸ್ವಪ್ನವ
ಸ್ವಲ್ಪ
ಸ್ವಲ್ಪ-ಕಾಲ
ಸ್ವಲ್ಪ-ಕಾಲಕ್ಕೆ
ಸ್ವಲ್ಪ-ಕಾಲದ
ಸ್ವಲ್ಪ-ದ-ರಲ್ಲಿಯೆ
ಸ್ವಲ್ಪ-ದ-ರಲ್ಲೇ
ಸ್ವಲ್ಪ-ಮಟ್ಟಿ-ಗಾ-ದರೂ
ಸ್ವಲ್ಪ-ಮಟ್ಟಿಗೆ
ಸ್ವಲ್ಪ-ವನ್ನು
ಸ್ವಲ್ಪ-ವಿದ್ದು-ವೇನು
ಸ್ವಲ್ಪವೂ
ಸ್ವಲ್ಪವೇ
ಸ್ವಲ್ಪ-ಹೇಳು-ವಂತೆ
ಸ್ವಲ್ಪ-ಹೊತ್ತಿಗೆ
ಸ್ವಲ್ಪ-ಹೊತ್ತು
ಸ್ವಶಿ
ಸ್ವಸಂಭ್ರಮ
ಸ್ವಸಂವೇದ್ಯವೊ
ಸ್ವಸು-ಖ-ವನ್ನೆಣಿ-ಸು-ವು-ದಿಲ್ಲ
ಸ್ವಸು-ಖೇಚ್ಛೆ-ಯನ್ನು
ಸ್ವಸ್ತಿ
ಸ್ವಸ್ಥ
ಸ್ವಸ್ಥ-ನಾಗಿ-ಬಿಟ್ಟನು
ಸ್ವಸ್ಥ-ರಾಗಿದ್ದ-ವ-ರಲ್ಲಿಯೂ
ಸ್ವಸ್ಥ-ವಾದ
ಸ್ವಸ್ಥ-ವಿಲ್ಲ
ಸ್ವಸ್ಥೇ
ಸ್ವಸ್ವ-ರೂಪದ
ಸ್ವಸ್ವ-ರೂಪ-ದಲ್ಲಿಯೇ
ಸ್ವಸ್ವ-ರೂಪ-ಲಾಭ-ವೆಂದು
ಸ್ವಸ್ವ-ರೂಪ-ವನ್ನು
ಸ್ವಾಗತ
ಸ್ವಾಗ-ತಕ್ಕೆ
ಸ್ವಾಗ-ತ-ವನ್ನು
ಸ್ವಾಗ-ತವು
ಸ್ವಾಗ-ತಿಸಿ
ಸ್ವಾಚ್ಛಂದ್ಯ-ಗಳನ್ನು
ಸ್ವಾತಂತ್ರ್ಯ
ಸ್ವಾತಂತ್ರ್ಯದ
ಸ್ವಾತಂತ್ರ್ಯ-ದಿಂದ
ಸ್ವಾತಂತ್ರ್ಯ-ವನ್ನು
ಸ್ವಾತಂತ್ರ್ಯ-ವನ್ನೂ
ಸ್ವಾತಂತ್ರ್ಯ-ವಿದೆ
ಸ್ವಾತಂತ್ರ್ಯ-ವಿದ್ದರೆ
ಸ್ವಾತಂತ್ರ್ಯ-ವಿಲ್ಲ
ಸ್ವಾತಂತ್ರ್ಯವು
ಸ್ವಾತಂತ್ರ್ಯ-ವೆಂಬ
ಸ್ವಾತಂತ್ರ್ಯವೇ
ಸ್ವಾತ್ಮನಿರ್ಭರ-ರಾಗಿ
ಸ್ವಾದ
ಸ್ವಾಧೀನ-ತಾ-ಭಾವ-ನೆ-ಯಿಂದ
ಸ್ವಾಧೀನತೆ
ಸ್ವಾಧೀನ-ತೆ-ಯುಳ್ಳ
ಸ್ವಾಧೀನ-ದಲ್ಲಿ-ರಿಸಿ-ಕೊಂಡಿದ್ದ
ಸ್ವಾಧೀನ-ಪಡಿ-ಸಿ-ಕೊಂಡು
ಸ್ವಾಭಾವಿಕ
ಸ್ವಾಭಾವಿಕ-ವಾಗಿ
ಸ್ವಾಭಾವಿಕ-ವಾಗಿಯೇ
ಸ್ವಾಭಾವಿಕ-ವಾದ
ಸ್ವಾಮಾಜಿ
ಸ್ವಾಮಾಜಿ-ಯನ್ನು
ಸ್ವಾಮಿ
ಸ್ವಾಮಿ-ಗಳ
ಸ್ವಾಮಿ-ಗಳನ್ನು
ಸ್ವಾಮಿ-ಗಳನ್ನೂ
ಸ್ವಾಮಿ-ಗಳಲ್ಲಿ
ಸ್ವಾಮಿ-ಗಳಲ್ಲಿಗೆ
ಸ್ವಾಮಿ-ಗಳಿಂದ
ಸ್ವಾಮಿ-ಗಳಿಗೂ
ಸ್ವಾಮಿ-ಗಳಿಗೆ
ಸ್ವಾಮಿ-ಗಳು
ಸ್ವಾಮಿ-ಗಳೂ
ಸ್ವಾಮಿ-ಗಳೆ
ಸ್ವಾಮಿ-ಗಳೇ
ಸ್ವಾಮಿ-ಗಳೊ-ಡನೆ
ಸ್ವಾಮಿಗೆ
ಸ್ವಾಮಿಜಿ
ಸ್ವಾಮಿ-ಜಿ-ಗಳು
ಸ್ವಾಮಿ-ಜಿಗೆ
ಸ್ವಾಮಿ-ಜಿಯ
ಸ್ವಾಮಿ-ಜಿ-ಯನ್ನು
ಸ್ವಾಮಿ-ಜಿ-ಯರು
ಸ್ವಾಮಿ-ಜಿ-ಯಲ್ಲಿ
ಸ್ವಾಮಿ-ಜಿ-ಯ-ವರ
ಸ್ವಾಮಿ-ಜಿ-ಯ-ವ-ರಿಗೆ
ಸ್ವಾಮಿ-ಜಿ-ಯ-ವರು
ಸ್ವಾಮಿ-ಜಿ-ಯ-ವರೇ
ಸ್ವಾಮಿ-ಜಿ-ಯಿಂದ
ಸ್ವಾಮಿ-ಜಿಯೂ
ಸ್ವಾಮಿ-ಜಿ-ಯೊ-ಡನೆ
ಸ್ವಾಮಿ-ಯಲ್ಲಿ
ಸ್ವಾಮಿಯು
ಸ್ವಾಮೀ
ಸ್ವಾಮೀಜಿ
ಸ್ವಾಮೀ-ಜಿ-ಗಳನ್ನು
ಸ್ವಾಮೀ-ಜಿ-ಗಳಿಗೆ
ಸ್ವಾಮೀ-ಜಿ-ಗಾಗಿ
ಸ್ವಾಮೀ-ಜಿಗೂ
ಸ್ವಾಮೀ-ಜಿಗೆ
ಸ್ವಾಮೀ-ಜಿಯ
ಸ್ವಾಮೀ-ಜಿ-ಯನ್ನು
ಸ್ವಾಮೀ-ಜಿ-ಯನ್ನೂ
ಸ್ವಾಮೀ-ಜಿ-ಯಲ್ಲಿ
ಸ್ವಾಮೀ-ಜಿ-ಯ-ವರ
ಸ್ವಾಮೀ-ಜಿ-ಯ-ವ-ರನ್ನು
ಸ್ವಾಮೀ-ಜಿ-ಯ-ವ-ರಿಗೆ
ಸ್ವಾಮೀ-ಜಿ-ಯ-ವರು
ಸ್ವಾಮೀ-ಜಿ-ಯ-ವ-ರೊಡನೆ
ಸ್ವಾಮೀ-ಜಿ-ಯ-ವರೊಬ್ಬರೇ
ಸ್ವಾಮೀ-ಜಿ-ಯೊ-ಡನೆ
ಸ್ವಾರಸ್ಯ-ಕರ
ಸ್ವಾರ್ಥ
ಸ್ವಾರ್ಥಕ್ಕಾಗಿ
ಸ್ವಾರ್ಥ-ಜೀವಿ-ಗಳು
ಸ್ವಾರ್ಥತೆ
ಸ್ವಾರ್ಥ-ತೆ-ಯನ್ನು
ಸ್ವಾರ್ಥ-ತೆ-ಯಿಂದ
ಸ್ವಾರ್ಥ-ತೆಯೇ
ಸ್ವಾರ್ಥತ್ಯಾಗವೇ
ಸ್ವಾರ್ಥದ
ಸ್ವಾರ್ಥ-ದಾರಾಧ-ನೆಯು
ಸ್ವಾರ್ಥ-ದಿಂದ
ಸ್ವಾರ್ಥ-ದೃಷ್ಟಿ-ಯಿಂದ
ಸ್ವಾರ್ಥ-ಪರ-ತೆ-ಯಲ್ಲಿ
ಸ್ವಾರ್ಥ-ಪರ-ತೆಯೇ
ಸ್ವಾರ್ಥ-ಪರ-ರಾದ
ಸ್ವಾರ್ಥ-ಪರಾಯಣ
ಸ್ವಾರ್ಥ-ಮ-ಲಿನತ
ಸ್ವಾರ್ಥ-ವನ್ನು-ಳಿದು
ಸ್ವಾರ್ಥ-ವಿಲ್ಲದೆ
ಸ್ವಾರ್ಥವೂ
ಸ್ವಾರ್ಥ-ವೆಲ್ಲವು
ಸ್ವಾರ್ಥವೇ
ಸ್ವಾರ್ಥ-ಶೂನ್ಯ-ರಾಗಿ
ಸ್ವಾರ್ಥಶ್ರುತಿ
ಸ್ವಾರ್ಥ-ಸಿದ್ಧಮ್
ಸ್ವಾರ್ಥ-ಸಿದ್ಧಿ
ಸ್ವಾರ್ಥ-ಹೀನ
ಸ್ವಾರ್ಥೋದ್ದೇಶ
ಸ್ವಾವಲಂಬನ
ಸ್ವೀಕರಿ-ಸದೆ
ಸ್ವೀಕರಿ-ಸಬೇ-ಕಾ-ಯಿತು
ಸ್ವೀಕರಿ-ಸ-ಬೇಕು
ಸ್ವೀಕರಿ-ಸ-ಬೇಕೆಂಬ
ಸ್ವೀಕರಿ-ಸ-ಲಿಲ್ಲ
ಸ್ವೀಕರಿ-ಸಲು
ಸ್ವೀಕರಿ-ಸಲೇ-ಬೇಕು
ಸ್ವೀಕ-ರಿಸಿ
ಸ್ವೀಕರಿ-ಸಿ-ದಾಗ
ಸ್ವೀಕರಿ-ಸಿದ್ದಾರೋ
ಸ್ವೀಕರಿಸು
ಸ್ವೀಕರಿ-ಸುತ್ತಲೇ
ಸ್ವೀಕರಿ-ಸುತ್ತಿದ್ದರು
ಸ್ವೀಕರಿ-ಸುತ್ತಿ-ರ-ಲಿಲ್ಲ
ಸ್ವೀಕರಿ-ಸುತ್ತೇನೆ
ಸ್ವೀಕರಿ-ಸುವ
ಸ್ವೀಕರಿ-ಸು-ವರೆ
ಸ್ವೀಕರಿ-ಸು-ವಲ್ಲಿ
ಸ್ವೀಕರಿ-ಸು-ವುದಕ್ಕೆ
ಸ್ವೀಕರಿ-ಸು-ವು-ದ-ರಿಂದ
ಸ್ವೀಕರಿ-ಸು-ವು-ದೆಂದು
ಸ್ವೀಕರಿ-ಸೋಣ
ಸ್ವೀಕಾರ
ಸ್ವೀಕೃತ-ವಾಗು-ವುದು
ಸ್ವೇಚ್ಛಾ-ಚಾರ-ಗಳೊಂದಿಗೆ
ಹಂಕರ-ದೇವ
ಹಂಕರ-ದೇವ-ನೆಂಬುದು
ಹಂಚ-ಬ-ಹುದು
ಹಂಚಿ-ಕೊಳ್ಳುವ
ಹಂಚಿಬಿಡ-ಬೇಕೆನ್ನಿ-ಸು-ವುದು
ಹಂಚಿ-ಹೋಗಿದ್ದುವು
ಹಂಚಿ-ಹೋಗಿ-ರುತ್ತದೆ
ಹಂಚಿ-ಹೋದರು
ಹಂತ-ಗಳನ್ನು
ಹಂತ-ಗಳಲ್ಲಿಯೂ
ಹಂತಹಂತ-ವಾಗಿ
ಹಂದಿ-ಗಳ
ಹಂದಿಯ
ಹಂಬಲಿ-ಸುತ್ತದೆ
ಹಂಬಲಿ-ಸುತ್ತಿದ್ದರು
ಹಂಸದ
ಹಂಸವು
ಹಂಸವೂ
ಹಕ್ಕಿ
ಹಕ್ಕಿ-ಗಳ
ಹಕ್ಕಿ-ಗಳನ್ನು
ಹಕ್ಕಿಗೆ
ಹಕ್ಕಿದೆ
ಹಕ್ಕಿಯು
ಹಕ್ಕಿಯೆ
ಹಕ್ಕಿ-ರ-ಬಾ-ರದು
ಹಕ್ಕಿಲ್ಲ
ಹಕ್ಕಿಲ್ಲದೆ
ಹಕ್ಕಿಲ್ಲ-ವೆಂಬು-ದನ್ನು
ಹಕ್ಕು
ಹಕ್ಕು-ಗಳನ್ನು
ಹಕ್ಕು-ಗಳಿ-ಗಾಗಿ
ಹಕ್ಕು-ಗಳಿ-ಗೋಸ್ಕರ
ಹಕ್ಕು-ಗ-ಳಿದ್ದಿತೋ
ಹಕ್ಕು-ಗಳು
ಹಕ್ಕು-ಬಾಧ್ಯ-ತೆ-ಗಳನ್ನು
ಹಕ್ಕು-ಬಾಧ್ಯತೆ-ಗಳಿ-ಗಾಗಿ
ಹಕ್ಕು-ಬಾಧ್ಯತೆ-ಯನ್ನು
ಹಗಲ
ಹಗ-ಲಿ-ನಲ್ಲಿ
ಹಗಲಿ-ನಷ್ಟೇ
ಹಗಲಿ-ರುಳು
ಹಗಲಿ-ರುಳು-ಗಳು
ಹಗಲು
ಹಗಲು-ಹೊತ್ತು
ಹಗಲೂ
ಹಗ-ಲೆಲ್ಲಾ
ಹಗುರ-ಮೋಡ-ಗಳಲ್ಲಿ
ಹಗೆ
ಹಗೆಗಳ
ಹಗೆಗಳ-ನೆಲ್ಲ
ಹಗೆತನ
ಹಗೆ-ಯಾಗಿ
ಹಗೆಯು
ಹಗೆ-ಯೆನ-ಗಿಲ್ಲವು
ಹಗ್ಗ
ಹಗ್ಗದ
ಹಗ್ಗ-ದಲ್ಲಿ
ಹಗ್ಗವ
ಹಗ್ಗ-ವನ್ನಾಗಿಯೆ
ಹಗ್ಗ-ವನ್ನು
ಹಗ್ಗ-ವನ್ನೇ
ಹಗ್ಗ-ವೆಂದು
ಹಗ್ಗ-ವೇನು
ಹಚಿ-ಕೊಂಡರೆ
ಹಚ್ಚದೇ
ಹಚ್ಚನೆ
ಹಚ್ಚ-ಬೇಕು
ಹಚ್ಚಿ
ಹಚ್ಚಿಸಿ-ಕೊಂಡು
ಹಚ್ಚಿಸು
ಹಚ್ಚುತ್ತ
ಹಟ
ಹಟ-ಮಾಡಿ-ಕೊಂಡು
ಹಟ-ಮಾಡುತ್ತ
ಹಠ
ಹಠ-ಯೋಗ
ಹಠಾತ್ತಾಗಿ
ಹಡಗನ್ನು
ಹಡಗಿನ
ಹಡಗಿ-ನಲ್ಲಿ
ಹಡಗು
ಹಡಗು-ಗಳಿಂದ
ಹಡ್ಸನ್
ಹಣ
ಹಣಕ್ಕೆ
ಹಣತೆ-ಯನ್ನು
ಹಣದ
ಹಣ-ದಲ್ಲಿ
ಹಣ-ದಾಸೆ-ಗಾಗಿ
ಹಣ-ದಿಂದ
ಹಣ-ವಂತ-ರಾಗಿ-ಬಿಟ್ಟಿದ್ದಾರೆ
ಹಣ-ವಂತ-ರಾದ
ಹಣ-ವಂತ-ರಿಗೆ
ಹಣ-ವನ್ನರ-ಸುತ್ತಿ-ರುವರೋ
ಹಣ-ವನ್ನು
ಹಣ-ವನ್ನೆಲ್ಲಾ
ಹಣ-ವಿದ್ದರೆ
ಹಣ-ವಿಲ್ಲದೆ
ಹಣವು
ಹಣವೂ
ಹಣ-ವೆಲ್ಲ-ವನ್ನೂ
ಹಣವೇ
ಹಣೆ-ಯಲ್ಲಿ
ಹಣೇ-ಬರಹವು
ಹಣ್ಣನ್ನು
ಹಣ್ಣಾ-ಯಿತು
ಹಣ್ಣಿ-ನಿಂದ
ಹಣ್ಣು
ಹಣ್ಣು-ಗಳನ್ನು
ಹಣ್ಣು-ಗಳೆಲ್ಲ
ಹತಾಃ
ಹತಾಶ-ರಾಗಿ
ಹತಾಶರೋ
ಹತಾಶೆಗೊಳ-ದಿರು
ಹತೆ
ಹತೇ
ಹತೋ
ಹತೋಟಿ-ಯಿಂದ
ಹತೋಟಿ-ಯಿಲ್ಲದೆ
ಹತ್ತ-ಬೇಕು
ಹತ್ತ-ರಷ್ಟು
ಹತ್ತ-ಲಿಲ್ಲ
ಹತ್ತಾಗಿ-ರಲಿ
ಹತ್ತಾರು
ಹತ್ತಿ
ಹತ್ತಿ-ಕೊಂಡಾಗ
ಹತ್ತಿಕ್ಕಲು
ಹತ್ತಿದ
ಹತ್ತಿ-ದಂತಾಯಿತು
ಹತ್ತಿ-ದರು
ಹತ್ತಿ-ದೆವು
ಹತ್ತಿ-ಬಿಟ್ಟಿದ್ದಾರೆ
ಹತ್ತಿರ
ಹತ್ತಿರಕ್ಕೂ
ಹತ್ತಿರಕ್ಕೆ
ಹತ್ತಿರದ
ಹತ್ತಿರ-ದಲ್ಲಿ
ಹತ್ತಿರ-ದಲ್ಲಿದ್ದ
ಹತ್ತಿರ-ದಲ್ಲಿದ್ದಾನೆ
ಹತ್ತಿರ-ದಲ್ಲಿ-ರುವ
ಹತ್ತಿರ-ದಲ್ಲಿರು-ವುದು
ಹತ್ತಿರ-ದವನು
ಹತ್ತಿರ-ವಿದೆಯೋ
ಹತ್ತಿರ-ವಿದ್ದ
ಹತ್ತಿರವೆ
ಹತ್ತಿರ-ವೆಂದರೆ
ಹತ್ತಿರವೇ
ಹತ್ತಿಸ-ಬೇಕಾಗಿದೆ
ಹತ್ತಿಸಿ-ದರು
ಹತ್ತಿಹುದೆನ್ನ
ಹತ್ತು
ಹತ್ತು-ಕಡೆ
ಹತ್ತುತ್ತವೆ
ಹತ್ತುದಿನ-ದಿಂದ
ಹತ್ತು-ವಾಗ
ಹತ್ತು-ವುದು
ಹತ್ತೊಂಬತ್ತನೇ
ಹದ-ಗೊಳಿಸು
ಹದ-ವಾದ
ಹದಿನಾರ-ರಲ್ಲಿ
ಹದಿನಾರಾಣೆಯ
ಹದಿನಾಲ್ಕು
ಹದಿನೆಂಟು
ಹದಿನೇಳನೆಯ
ಹದಿನೈದು
ಹದಿ-ಮೂರನೇ
ಹದಿ-ಮೂರು
ಹದಿಹರೆ-ಯದ
ಹನಿ-ಗಳು
ಹನಿ-ಯನ್ನು
ಹನಿ-ಯನ್ನೂ
ಹನುಮಂತ
ಹನುಮಂತನ
ಹನುಮಂತನು
ಹನ್ನೆರಡು
ಹನ್ನೆರಡು-ವರ್ಷದ
ಹನ್ನೆರಡೂವರೆ
ಹನ್ನೊಂದನೆಯದನ್ನಾಗಲೇ
ಹನ್ನೊಂದು
ಹಬೆ
ಹಬ್ಬ-ಗಳುಂಟಷ್ಟೆ
ಹಬ್ಬ-ವನ್ನಾಚರಿಸು-ವುದು
ಹಬ್ಬಿತ್ತು
ಹಬ್ಬಿದೆ
ಹಬ್ಬಿರು-ವನು
ಹಬ್ಬಿಹುದು
ಹಮ್ಮಿನ
ಹಯ
ಹಯಿ
ಹಯೆ
ಹರ
ಹರಕು
ಹರಕೆಯ-ದೆಲ್ಲವು
ಹರಟುತ್ತಿದ್ದೆ
ಹರಟೆ
ಹರಡಲು
ಹರಡಿ
ಹರಡಿ-ಕೊಂಡಿತು
ಹರಡಿ-ಕೊಂಡಿತ್ತೆಂದು
ಹರಡಿ-ಕೊಂಡಿದ್ದರು
ಹರಡಿತು
ಹರಡಿದರೆ
ಹರಡಿದೆ
ಹರಡಿರುವೆ
ಹರಡಿವೆ
ಹರಡು
ಹರಡು-ತಿರುವೆ
ಹರಡುತ್ತವೆ
ಹರಡು-ವಂತಹ
ಹರಡು-ವಂತೆಯೂ
ಹರಡು-ವರು
ಹರಡುವಿಕೆ-ಯಿಂದ
ಹರಡುವಿಕೆ-ಯಿಂದಾಗಿ
ಹರಡು-ವುದರ
ಹರಡು-ವುವು
ಹರ-ನನ್ನು
ಹರ-ಮೋಹನ
ಹರಸಲಿ
ಹರಸಿ
ಹರಸಿ-ದನು
ಹರಸಿ-ದರು
ಹರಹರ
ಹರಿ-ದರು
ಹರಿದಾಡುತ್ತಿರುವ
ಹರಿದಾಡುವ
ಹರಿ-ದಿದೆ
ಹರಿದು
ಹರಿದು-ಹೋಗು-ತ್ತಿರುತ್ತದೆ
ಹರಿದೆಸೆದೆ
ಹರಿ-ನಾಮ
ಹರಿ-ನಾಮ-ವನ್ನು
ಹರಿ-ನಾಮೋಚ್ಛಾರಣೆ
ಹರಿಮೀಡೇ
ಹರಿಯ-ದಂತೆ
ಹರಿಯ-ಬಿಟ್ಟರೆ
ಹರಿಯ-ಲಾರದೋ
ಹರಿಯಲಿ
ಹರಿ-ಯಿತದೊ
ಹರಿ-ಯಿತಲ್ಲಿ
ಹರಿ-ಯಿತೊ
ಹರಿಯುತಿಹ
ಹರಿ-ಯುತ್ತಲೇ
ಹರಿ-ಯುತ್ತಿತ್ತು
ಹರಿ-ಯುತ್ತಿದೆ
ಹರಿಯುತ್ತಿದ್ದರೂ
ಹರಿಯುತ್ತಿದ್ದು-ದ-ರಿಂದ
ಹರಿಯು-ತ್ತಿದ್ದುದು
ಹರಿಯು-ತ್ತಿರುವ
ಹರಿಯುವ
ಹರಿಯು-ವಂತೆ
ಹರಿಯು-ವುದು
ಹರಿವುದು
ಹರಿವುದೊ
ಹರಿಸುತ್ತ
ಹರಿಸು-ತ್ತಿರು
ಹರಿಸು-ವುದು
ಹರಿಹರಬ್ರಹ್ಮಾದಿ
ಹರುಷ-ಚಿತ್ತ-ದಿಂದ
ಹರೆ-ಯುವ
ಹರೇರ್ನಾಮ
ಹರೇರ್ನಾಮೈವ
ಹರ್ಬರ್ಟ್
ಹರ್ಷ
ಹರ್ಷಕ್ಕಿಂತಲೂ
ಹರ್ಷ-ದಿಂದ
ಹರ್ಷ-ಭಾವನೆ
ಹರ್ಷ-ವುಂಟಾಯಿತು
ಹಲವನ್ನು
ಹಲವರ
ಹಲವರಿಗೆ
ಹಲವರು
ಹಲವಲ್ಲ
ಹಲವಾರು
ಹಲವಿಲ್ಲ-ವಿಲ್ಲಿ
ಹಲವು
ಹಲಾಹಲ
ಹಳಿದರೆ
ಹಳಿದು-ಕೊಳ್ಳು-ತ್ತಿರು-ವವ-ನಿಗೆ
ಹಳಿಯ-ಲಿಲ್ಲ
ಹಳಿಯು-ತ್ತಿದ್ದರು
ಹಳಿಯು-ವುದಕ್ಕೋಸ್ಕರವೇ
ಹಳೆಯ
ಹಳ್ಳಿ
ಹಳ್ಳಿ-ಗಳಲ್ಲಿ
ಹಳ್ಳಿ-ಗಳಲ್ಲಿಯೂ
ಹಳ್ಳಿ-ಗಳಲ್ಲಿ-ರುವ
ಹಳ್ಳಿ-ಗಳಿಗೆ
ಹಳ್ಳಿಗೂ
ಹಳ್ಳಿಗೆ
ಹಳ್ಳಿಯ
ಹಳ್ಳಿಯಲ್ಲಿ
ಹಳ್ಳಿಯಲ್ಲಿಯೂ
ಹಳ್ಳಿಯ-ವರು
ಹಳ್ಳಿಯ-ವರೆಲ್ಲ
ಹವಾ
ಹವಿಸ್ಸನ್ನು
ಹವಿಸ್ಸಿನ
ಹವಿಸ್ಸೆಲ್ಲ
ಹವೆ
ಹವೋ
ಹವ್ಯಾಸ-ಗಳನ್ನು
ಹಸನಾಗ-ಬೇಕು
ಹಸನ್ಮುಖ-ದಿಂದ
ಹಸಿದ
ಹಸಿಯೊ
ಹಸಿರ
ಹಸಿರುಟ್ಟು
ಹಸಿ-ವನ್ನಡಗಿಸಲು
ಹಸಿ-ವನ್ನಿಂಗಿಸಲು
ಹಸಿವಿ-ನಿಂದ
ಹಸಿವು
ಹಸಿ-ವೆಂಬ
ಹಸಿವೇ
ಹಸಿಹಸಿ-ಯಾಗಿರು-ವಾಗಲೇ
ಹಸು
ಹಸು-ಗಳ
ಹಸು-ಗಳನ್ನು
ಹಸು-ರಾಗಿದ್ದುವು
ಹಸುರೆ
ಹಸು-ವನ್ನಾಗಲಿ
ಹಸುವು
ಹಸುವೋ
ಹಸ್ತ-ಗಳಿಂದ
ಹಸ್ತದ
ಹಸ್ತ-ಪಾತಃ
ಹಸ್ತಪ್ರಕ್ಷಾಲನ
ಹಸ್ತ-ವನ್ನಿರಿಸಿ
ಹಸ್ತ-ವನ್ನು
ಹಸ್ತಾಂತರಿತ-ವಾದ
ಹಸ್ಥಪ್ರತಿ-ಯನ್ನು
ಹಾ
ಹಾಕ-ಕೂಡ-ದೆಂದೂ
ಹಾಕ-ತೊಡಗಿ-ದರು
ಹಾಕದ
ಹಾಕದೆ
ಹಾಕ-ಬಹುದು
ಹಾಕ-ಬೇಕು
ಹಾಕಲಿ
ಹಾಕಲು
ಹಾಕಲೆತ್ನಿ-ಸುತ್ತಿ-ರುವ
ಹಾಕಲ್ಪಟ್ಟಿದೆ
ಹಾಕಲ್ಪಟ್ಟು
ಹಾಕಿ
ಹಾಕಿ-ಕೊಂಡಿದ್ದಾರೆ
ಹಾಕಿ-ಕೊಂಡಿರುವೆ
ಹಾಕಿ-ಕೊಂಡಿಲ್ಲ
ಹಾಕಿ-ಕೊಂಡು
ಹಾಕಿ-ಕೊಳ್ಳದೆ
ಹಾಕಿ-ಕೊಳ್ಳ-ಬಹುದು
ಹಾಕಿ-ಕೊಳ್ಳ-ಲಾಗ-ದಿದ್ದರೂ
ಹಾಕಿ-ಕೊಳ್ಳಲಿ
ಹಾಕಿ-ಕೊಳ್ಳುವ
ಹಾಕಿ-ಕೊಳ್ಳುವರು
ಹಾಕಿ-ಕೊಳ್ಳು-ವಷ್ಟು
ಹಾಕಿ-ಕೊಳ್ಳುವು-ದಕ್ಕೂ
ಹಾಕಿ-ಕೊಳ್ಳು-ವುದಕ್ಕೆ
ಹಾಕಿ-ಕೊಳ್ಳು-ವುದನ್ನು
ಹಾಕಿ-ಕೊಳ್ಳುವು-ದರಲ್ಲಿ
ಹಾಕಿ-ಕೊಳ್ಳುವುದು
ಹಾಕಿಟ್ಟಿದ್ದರು
ಹಾಕಿತು
ಹಾಕಿದ
ಹಾಕಿ-ದಂತಲ್ಲವೆ
ಹಾಕಿ-ದಂತೆ
ಹಾಕಿ-ದನು
ಹಾಕಿ-ದರು
ಹಾಕಿ-ದರೆ
ಹಾಕಿ-ದಳು
ಹಾಕಿದೆ
ಹಾಕಿದ್ದ
ಹಾಕಿದ್ದು-ಬಿಡು
ಹಾಕಿ-ಬಿಟ್ಟಿತ್ತು
ಹಾಕಿ-ಬಿಟ್ಟು
ಹಾಕಿ-ಬಿಡು
ಹಾಕಿ-ಬಿಡುತ್ತೇನೆ
ಹಾಕಿ-ರುವ
ಹಾಕಿ-ರುವರು
ಹಾಕಿಸ-ಬೇಕು
ಹಾಕಿಸಿ-ಕೊಂಡು
ಹಾಕಿಸಿ-ಕೊಡುತ್ತೇನೆ
ಹಾಕಿಸು-ತ್ತೇನೆ
ಹಾಕಿಸು-ವುದಕ್ಕೆ
ಹಾಕುತ್ತ
ಹಾಕುತ್ತಾ
ಹಾಕುತ್ತಾರೆ
ಹಾಕುತ್ತಿದ್ದನು
ಹಾಕುತ್ತಿದ್ದರು
ಹಾಕುತ್ತಿದ್ದಾಗ
ಹಾಕುತ್ತಿದ್ದೀ-ಯಲ್ಲವೆ
ಹಾಕುತ್ತಿರು
ಹಾಕುತ್ತೇನೆ
ಹಾಕುವ
ಹಾಕು-ವುದಕ್ಕೆ
ಹಾಕುವುದು
ಹಾಗಲ್ಲ
ಹಾಗಲ್ಲ-ದಿದ್ದರೂ
ಹಾಗಲ್ಲ-ದಿದ್ದರೆ
ಹಾಗಲ್ಲದೆ
ಹಾಗಲ್ಲದೇ
ಹಾಗಲ್ಲ-ವಯ್ಯ
ಹಾಗಾ-ಗದೇ
ಹಾಗಾಗಲು
ಹಾಗಾಗಿಯೇ
ಹಾಗಾ-ಗುತ್ತದೆ
ಹಾಗಾದ
ಹಾಗಾ-ದರೂ
ಹಾಗಾ-ದರೆ
ಹಾಗಾ-ದರೇ
ಹಾಗಾದ-ರೇನು
ಹಾಗಾದ-ರೇನೆ
ಹಾಗಾ-ದಾಗ
ಹಾಗಾ-ಯಿತು
ಹಾಗಿದೆ
ಹಾಗಿದ್ದ
ಹಾಗಿದ್ದರೂ
ಹಾಗಿದ್ದರೆ
ಹಾಗಿದ್ದಲ್ಲಿ
ಹಾಗಿದ್ದಿದ್ದರೆ
ಹಾಗಿದ್ದಿರ-ಬಹುದು
ಹಾಗಿರ-ಬಹುದು
ಹಾಗಿರಲಿ
ಹಾಗಿರ-ಲಿಲ್ಲ
ಹಾಗಿಲ್ಲ
ಹಾಗಿಲ್ಲ-ದಿದ್ದರೆ
ಹಾಗೂ
ಹಾಗೆ
ಹಾಗೆಂದರೆ
ಹಾಗೆಂದ-ವರಾರು
ಹಾಗೆಂದು
ಹಾಗೆಯೆ
ಹಾಗೆಯೇ
ಹಾಗೆಲ್ಲಾ
ಹಾಗೆಲ್ಲಿ
ಹಾಗೆ-ಹಾಗೇ
ಹಾಗೇ
ಹಾಗೇಕೆ
ಹಾಗೇ-ನಾದರೂ
ಹಾಜಾರ್
ಹಾಟೆಂಟಾಟ್
ಹಾಡ-ತೊಡಗಿ-ದರು
ಹಾಡ-ದಿಹ
ಹಾಡನು
ಹಾಡನುಲಿಯೈ
ಹಾಡನ್ನು
ಹಾಡ-ಲಾಗುತ್ತದೆ
ಹಾಡ-ಲಾರಂಭಿಸಿ-ದರು
ಹಾಡಲು
ಹಾಡಲು-ಪಕ್ರಮಿಸಿ-ದರು
ಹಾಡಿ
ಹಾಡಿ-ಕೊಳ್ಳುತ್ತಿದ್ದರು
ಹಾಡಿ-ಗೊಪ್ಪುವ
ಹಾಡಿ-ದರಲ್ಲ
ಹಾಡಿ-ದರು
ಹಾಡಿದ್ದಾರೆ
ಹಾಡಿನ
ಹಾಡಿ-ನಲಿ
ಹಾಡಿ-ನಿಂದೆಚ್ಚರಿಸು
ಹಾಡಿರು-ವರೋ
ಹಾಡಿಸ-ಬೇಕೆಂದು
ಹಾಡು
ಹಾಡು-ಗಳ
ಹಾಡು-ಗಳನ್ನು
ಹಾಡು-ಗಳನ್ನೆಲ್ಲಾ
ಹಾಡು-ಗಳೆಂದೂ
ಹಾಡು-ಗಾರಿಕೆ-ಯಲ್ಲಿ
ಹಾಡು-ಗಾರಿಕೆಯೂ
ಹಾಡುತ್ತ
ಹಾಡುತ್ತಾ
ಹಾಡುತ್ತಿದ್ದ
ಹಾಡುತ್ತೇನೆ
ಹಾಡುವ
ಹಾಡು-ವನು
ಹಾಡು-ವರು
ಹಾಡು-ವರೆಂದು
ಹಾಡು-ವವರು
ಹಾಡು-ವಾಗ
ಹಾಡು-ವುದಿಲ್ಲ
ಹಾಡು-ವುದು
ಹಾಡುವೆ
ಹಾಡು-ವೆನು
ಹಾಡೈ
ಹಾಡೊಂದ
ಹಾತೀ
ಹಾತೊರೆ-ಯುತ್ತಾನೆ
ಹಾತೊರೆ-ಯುತ್ತಿರು-ವರೊ
ಹಾದಿ
ಹಾದಿ-ಗಳೂ
ಹಾದಿ-ಗಳೆ
ಹಾದಿಗೆ
ಹಾದಿ-ಗೇಕೆ
ಹಾದಿಯ
ಹಾದಿ-ಯನು
ಹಾದಿ-ಯನ್ನು
ಹಾದಿ-ಯಲ್ಲಿ
ಹಾದಿ-ಯಲ್ಲೂ
ಹಾದಿ-ಯಲ್ಲೇ
ಹಾದಿ-ಯಿಂದ
ಹಾದಿ-ಯುದ್ದಕು
ಹಾದಿಯೂ
ಹಾದಿಯೇ
ಹಾದು
ಹಾದು-ಹೋದಂತೆ
ಹಾನಿ
ಹಾನಿ-ಗಳೆಲ್ಲಾ
ಹಾನಿಗೂ
ಹಾನಿ-ಗೆಲ್ಲಾ
ಹಾನಿ-ಯನ್ನುಂಟು-ಮಾಡುತ್ತಿದೆ
ಹಾನಿ-ಯುಂಟಾದರೆ
ಹಾನಿಯೂ
ಹಾಯಾಗಿ
ಹಾಯಿ-ಪಟ-ವನ್ನು
ಹಾಯಿ-ಸುತ್ತಾನೆ
ಹಾಯು-ತ್ತಿರು-ವಾಗ
ಹಾಯು-ತ್ತಿರು-ವುದನ್ನು
ಹಾರ-ಗಳಿಂದ
ಹಾರಾಡು-ತ್ತಿರು-ವುದು
ಹಾರಾಡುವ
ಹಾರಿ
ಹಾರಿ-ದರೆ
ಹಾರಿ-ಹೋಗಿ
ಹಾರುತ್ತಿರುವ
ಹಾರೈಕೆ
ಹಾರೈಕೆ-ಯೆಲ್ಲಾ
ಹಾರೈಸಿವೆ
ಹಾಲನ್ನು
ಹಾಲಾಹಲವೆ
ಹಾಲಿ-ನಲ್ಲಿ
ಹಾಲಿ-ನಿಂದ
ಹಾಲು
ಹಾಲು-ಗಲ್ಲಿನ
ಹಾಳಾಗಿ
ಹಾಳಾಗಿ-ಹೋಗಿದೆ
ಹಾಳಾಗಿ-ಹೋಗಿವೆ
ಹಾಳಾಗು-ವನು
ಹಾಳಾಗು-ವುದು
ಹಾಳಾ-ಯಿತು
ಹಾಳು
ಹಾಳು-ಮಾಡ-ಕೂಡದು
ಹಾಳು-ಮಾಡಿತು
ಹಾಳು-ಮಾಡುತ್ತಾರೆ
ಹಾಳು-ಮಾಡು-ವನು
ಹಾಳು-ಮಾಡು-ವಿರಿ
ಹಾಳು-ಮಾಡುವು-ದರಲ್ಲಿ-ರುವರು
ಹಾವನ್ನು
ಹಾವನ್ನೂ
ಹಾವ-ಭಾವ-ಗಳನ್ನು
ಹಾವಳಿ
ಹಾವಳಿಯೂ
ಹಾವಾಗಿ-ಬಿಟ್ಟಿತೊ
ಹಾವಿನ
ಹಾವು
ಹಾವೆಂದು
ಹಾವೆಂದು-ಕೊಂಡು
ಹಾವೇ
ಹಾಸಗೆಯ
ಹಾಸಿ
ಹಾಸಿಗೆ
ಹಾಸಿಗೆಗೆ
ಹಾಸಿಗೆಯ
ಹಾಸಿಗೆ-ಯನ್ನು
ಹಾಸಿಗೆ-ಯಲ್ಲಿ
ಹಾಸಿಗೆ-ಯಿಂದ
ಹಾಸಿದ್ದ
ಹಾಸಿಮುಖ
ಹಾಸಿರುವ
ಹಾಸು-ಗಳ
ಹಾಸುಹೊಕ್ಕಾಗಿವೆ
ಹಾಸುಹೊಕ್ಕಿ-ನಲಿ
ಹಾಸ್ಯ
ಹಾಸ್ಯ-ಕ್ಕಾದರೂ
ಹಾಸ್ಯ-ಕ್ಕೋಸ್ಕರ
ಹಾಸ್ಯ-ದಿಂದ
ಹಾಸ್ಯ-ಮಾಡಿ-ದರಷ್ಟೆ
ಹಾಸ್ಯ-ಮಾಡುತ್ತ
ಹಾಸ್ಯ-ಮಾಡುತ್ತಿದ್ದರು
ಹಾಸ್ಯಾಸ್ಪದ-ವಾದುವು
ಹಾಹಾಕಾರ
ಹಾಹಾಕಾರ-ದಿಂದ
ಹಾಹಾಕಾರ-ಪಡುತ್ತಿರುವ
ಹಾಹಾಕಾರ-ವನ್ನೂ
ಹಾಹಾಕಾರ-ವೇಳು-ವುದು
ಹಿ
ಹಿಂಜರಿ-ದರೆ
ಹಿಂಜರಿ-ಯದಿರು
ಹಿಂಜರಿ-ಯುತ್ತಾ
ಹಿಂಜರಿ-ಯುವ
ಹಿಂಜರಿ-ಯುವು-ದಿಲ್ಲ
ಹಿಂಜರಿ-ಯುವೆನು
ಹಿಂಜಿ-ದರಳೆಯ
ಹಿಂಡಿ
ಹಿಂಡಿ-ದಂತಾಯಿತು
ಹಿಂಡಿ-ದರೆ
ಹಿಂಡು
ಹಿಂತಿರುಗ-ಬೇಕು
ಹಿಂತಿರುಗ-ಬೇಕೆಂದು
ಹಿಂತಿರುಗ-ಲಾರದು
ಹಿಂತಿರುಗ-ಲಿಲ್ಲ
ಹಿಂತಿರುಗಲು
ಹಿಂತಿ-ರುಗಿ
ಹಿಂತಿ-ರುಗಿತು
ಹಿಂತಿ-ರುಗಿದ
ಹಿಂತಿ-ರುಗಿ-ದನು
ಹಿಂತಿ-ರುಗಿ-ದರು
ಹಿಂತಿ-ರುಗಿ-ದಾಗ
ಹಿಂತಿ-ರುಗಿದೆ
ಹಿಂತಿ-ರುಗಿ-ದೆವು
ಹಿಂತಿ-ರುಗಿ-ಬಂದು
ಹಿಂತಿ-ರುಗಿ-ಬ-ರುವ
ಹಿಂತಿ-ರುಗಿ-ರುವ-ರೆಂದೂ
ಹಿಂತಿರುಗುವ
ಹಿಂತಿರುಗು-ವುದಕ್ಕೆ
ಹಿಂತಿರುಗು-ವುದ-ರಿಂದಾಗುವ
ಹಿಂತಿರುಗು-ವುದಿಲ್ಲ
ಹಿಂತೆಗೆಯದೆ
ಹಿಂತೆಗೆಯ-ಬೇಡ
ಹಿಂದಕೆ
ಹಿಂದಕ್ಕೂ
ಹಿಂದಕ್ಕೆ
ಹಿಂದಣ
ಹಿಂದಿ
ಹಿಂದಿನ
ಹಿಂದಿ-ನಂತೆ
ಹಿಂದಿ-ನಂತೆಯೇ
ಹಿಂದಿ-ನವ-ನಾಗಿದ್ದನು
ಹಿಂದಿ-ನವನು
ಹಿಂದಿ-ನಷ್ಟೇ
ಹಿಂದಿ-ನಿಂದ
ಹಿಂದಿ-ರುಗಿ
ಹಿಂದಿ-ರುಗಿ-ದನು
ಹಿಂದಿ-ರುಗು-ವಂತಿಲ್ಲ
ಹಿಂದಿ-ರುಗುವಿಕೆ
ಹಿಂದಿ-ರುಗುವಿಕೆ-ಯಿಲ್ಲ
ಹಿಂದಿ-ರುವ
ಹಿಂದಿ-ರುವಾಗ
ಹಿಂದೀ
ಹಿಂದು
ಹಿಂದು-ಗಡೆ
ಹಿಂದು-ಗಡೆ-ಯಿಂದ
ಹಿಂದು-ಮುಂದು
ಹಿಂದು-ಮುಂದೆಲ್ಲ-ವನು
ಹಿಂದು-ವೆಂದು
ಹಿಂದೂ
ಹಿಂದೂ-ಗಳ
ಹಿಂದೂ-ಗಳನ್ನಾಗಿ
ಹಿಂದೂ-ಗಳನ್ನು
ಹಿಂದೂ-ಗಳಲ್ಲಿ
ಹಿಂದೂ-ಗಳಾದರೋ
ಹಿಂದೂ-ಗಳಿಂದ
ಹಿಂದೂ-ಗಳಿಗೆ
ಹಿಂದೂ-ಗಳು
ಹಿಂದೂ-ಗಳೆಲ್ಲ
ಹಿಂದೂ-ಗಳೆಲ್ಲರೂ
ಹಿಂದೂ-ದೇಶದ
ಹಿಂದೂ-ದೇಶೀಯ-ರಾದ
ಹಿಂದೂ-ಧರ್ಮ
ಹಿಂದೂ-ಧರ್ಮಕ್ಕೂ
ಹಿಂದೂ-ಧರ್ಮದ
ಹಿಂದೂ-ಧರ್ಮ-ದಲ್ಲಿ
ಹಿಂದೂ-ಧರ್ಮ-ದಲ್ಲಿದ್ದ
ಹಿಂದೂ-ಧರ್ಮ-ದಲ್ಲಿನ
ಹಿಂದೂ-ಧರ್ಮ-ದಿಂದ
ಹಿಂದೂ-ಧರ್ಮ-ವನ್ನು
ಹಿಂದೂ-ಧರ್ಮವು
ಹಿಂದೂ-ಧರ್ಮವೇ
ಹಿಂದೂಮತ
ಹಿಂದೂ-ಮುಸಲ್ಮಾನಕ್ರಿಸ್ತ-ಧರ್ಮ-ಗಳಲ್ಲಿದೆ
ಹಿಂದೂ-ರಕ್ತ-ವಾದು-ದರಿಂದ
ಹಿಂದೂ-ರಕ್ತವೇ
ಹಿಂದೂ-ವಿಗೆ
ಹಿಂದೂ-ವಿನಂತೆ
ಹಿಂದೂವು
ಹಿಂದೂವೇ
ಹಿಂದೂಸ್ತಾನಕ್ಕೆ
ಹಿಂದೂಸ್ತಾನ-ದಲ್ಲಿ
ಹಿಂದೂಸ್ಥಾನದ
ಹಿಂದೂಸ್ಥಾನ-ದಲ್ಲಿ
ಹಿಂದೂಸ್ಥಾನ-ದವ-ನೆಂದು
ಹಿಂದೂಸ್ಥಾನಿ
ಹಿಂದೆ
ಹಿಂದೆ-ಮುಂದಕೆ
ಹಿಂದೆಯೂ
ಹಿಂದೆಯೇ
ಹಿಂದೆಲ್ಲ
ಹಿಂದೆ-ಹಿಂದೆಯೆ
ಹಿಂದೇಟು
ಹಿಂದೋಡುತ
ಹಿಂದೋಲ್
ಹಿಂಬಾಗಿಲಿ-ನಿಂದ
ಹಿಂಬಾಲಿಸಿ
ಹಿಂಬಾಲಿಸು-ವರು
ಹಿಂಬಾಲಿಸು-ವುದು
ಹಿಂಬಾಲಿಸು-ವುವು
ಹಿಂಸಾ-ಜನಕರೂ
ಹಿಂಸಾ-ದೋಷಾದಿ-ಗಳು
ಹಿಂಸಿಸ-ಬೇಡಿ
ಹಿಂಸಿಸು-ವುದು
ಹಿಂಸೆ
ಹಿಂಸೆ-ಯನ್ನೂ
ಹಿಂಸೆ-ಯಾಗದೆ
ಹಿಂಸೆ-ಯಿಂದ
ಹಿಗ್ಗ-ಬೇಕು
ಹಿಗ್ಗಿ
ಹಿಟ್ಟನ್ನು
ಹಿಟ್ಟಿಲ್ಲ
ಹಿಟ್ಟು-ಬಟ್ಟೆ
ಹಿಡಿ
ಹಿಡಿತ-ದಲ್ಲಿಟ್ಟು-ಕೊಂಡಿದ್ದಾನೆ
ಹಿಡಿತ-ದಲ್ಲಿಟ್ಟು-ಕೊಂಡು
ಹಿಡಿತ-ದಲ್ಲಿಟ್ಟು-ಕೊಳ್ಳ-ಬಹುದು
ಹಿಡಿತ-ದಲ್ಲಿಟ್ಟು-ಕೊಳ್ಳಲು
ಹಿಡಿತ-ದಲ್ಲೇ
ಹಿಡಿತ-ದಿಂದ
ಹಿಡಿದ
ಹಿಡಿ-ದಂತಾಗು-ವುದು
ಹಿಡಿದ-ನಲ್ಲ
ಹಿಡಿದರೆ
ಹಿಡಿದ-ವನು-ರುಳಿದ
ಹಿಡಿ-ದಿಟ್ಟಿದೆ
ಹಿಡಿ-ದಿಟ್ಟಿದ್ದಾರೆ
ಹಿಡಿದಿಡ-ಲಾಗಿದೆ
ಹಿಡಿದಿಡ-ಲಾರಳು
ಹಿಡಿ-ದಿದೆ
ಹಿಡಿ-ದಿದ್ದರು
ಹಿಡಿದಿರ-ಬೇಕು
ಹಿಡಿದಿರ-ಬೇಕೆಂದರೆ
ಹಿಡಿದಿರು-ವವ-ಳೆಂದು
ಹಿಡಿದಿರು-ವುದರ
ಹಿಡಿದು
ಹಿಡಿದು-ಕೊಂಡಿದೆ
ಹಿಡಿದು-ಕೊಂಡಿದ್ದ
ಹಿಡಿದು-ಕೊಂಡಿರ-ಬೇಕು
ಹಿಡಿದು-ಕೊಂಡಿರುತ್ತೀಯೋ
ಹಿಡಿದು-ಕೊಂಡು
ಹಿಡಿದು-ಕೊಂಡೆ
ಹಿಡಿದು-ಕೊಳ್ಳುತ್ತಿದ್ದ
ಹಿಡಿದೆ
ಹಿಡಿಯ-ಬೇಕಲ್ಲವೆ
ಹಿಡಿಯ-ಲಿಲ್ಲ
ಹಿಡಿ-ಯಲ್ಪಟ್ಟ
ಹಿಡಿ-ಯಿತಂತೆ
ಹಿಡಿ-ಯಿತು
ಹಿಡಿ-ಯಿರಿ
ಹಿಡಿ-ಯುತ್ತವೆ
ಹಿಡಿ-ಯುತ್ತಾನೆ
ಹಿಡಿ-ಯುತ್ತಾರೆ
ಹಿಡಿಯು-ವಂತೆ
ಹಿಡಿಯು-ವುದಕ್ಕೆ
ಹಿಡಿಯು-ವುದು
ಹಿಡಿಯು-ವುದೆಂದೆನ್ನಿ-ಸು-ವುದು
ಹಿಡಿಯು-ವುವು
ಹಿಡಿಸ-ಲಾರ-ದಷ್ಟು
ಹಿಡಿಸಿ-ದುವು
ಹಿಡಿಸು-ವಂಥದು
ಹಿಡಿಸು-ವಷ್ಟು
ಹಿಡಿಸು-ವುದು
ಹಿಡಿಸು-ವುದೋ
ಹಿತ
ಹಿತ-ಕರ-ವಲ್ಲ-ದು-ದನ್ನು
ಹಿತ-ಕರ-ವಲ್ಲ-ವಾದರೂ
ಹಿತ-ಕರ-ವಾಗಿರು-ವುದು
ಹಿತ-ಕರ-ವಾದ
ಹಿತ-ಕರವೂ
ಹಿತಕ್ಕಾಗಿ
ಹಿತಕ್ಕೆ
ಹಿತಕ್ಕೋಸ್ಕರ
ಹಿತ-ಚಿಂತಕರು
ಹಿತದ
ಹಿತ-ವನ್ನು
ಹಿತ-ವಲ್ಲ
ಹಿತ-ವಾಗು-ವುದು
ಹಿತ-ವಾದ
ಹಿತ-ಸಾಧನೆ-ಗಾಗಿ
ಹಿತಾಯ
ಹಿತ್ತಾಳೆ-ಯಂತೆ
ಹಿತ್ತಾಳೆಯೇ
ಹಿತ್ವಾ
ಹಿನ್ನೆಲೆ-ಯಲ್ಲಿ
ಹಿನ್ನೆಲೆ-ಯಲ್ಲಿರು-ವುದು
ಹಿನ್ನೆಲೆ-ಯಿದೆ
ಹಿನ್ನೆಲೆ-ಯೊಂದು
ಹಿಮ
ಹಿಮಕರ
ಹಿಮ-ಗಿರಿಯು
ಹಿಮದ
ಹಿಮ-ಮಣಿ-ಗಳಿಂದಿಡಿದ
ಹಿಮ-ಶಿಖರ
ಹಿಮ-ಶಿಖರ-ವದು
ಹಿಮಶಿಲ
ಹಿಮಾನೀ
ಹಿಮಾಲಯ
ಹಿಮಾಲಯಕ್ಕೆ
ಹಿಮಾಲಯದ
ಹಿಮಾಲಯ-ದಂತೆ
ಹಿಮಾಲಯ-ದಲ್ಲಿ
ಹಿಮ್ಮುಖ-ರಾಗ-ಕೂಡದು
ಹಿಮ್ಮೆಟ್ಟ-ಬೇಡ
ಹಿಮ್ಮೆಟ್ಟು-ವಂತೆ
ಹಿಯಾ
ಹಿರಿ
ಹಿರಿಮೆ
ಹಿರಿಮೆ-ಯನ್ನು
ಹಿರಿಯ
ಹಿರಿಯರು
ಹಿರಿಯರೆದು-ರಿಗೆ
ಹಿಲ್ಲೋಲ
ಹಿಸುಕು
ಹೀಗಾಗಿ
ಹೀಗಾಗು-ವುದು
ಹೀಗಾದ
ಹೀಗಾದ-ಮೇಲೆ
ಹೀಗಾದರೆ
ಹೀಗಿತ್ತೆಂದು
ಹೀಗಿದೆ
ಹೀಗಿದ್ದ
ಹೀಗಿದ್ದು
ಹೀಗಿರ-ಬಹುದು
ಹೀಗಿರ-ಲಾಗುವು-ದಿಲ್ಲವೋ
ಹೀಗಿರಲು
ಹೀಗಿರುತ್ತಿದ್ದರೆ
ಹೀಗಿರು-ವಾಗ
ಹೀಗಿರು-ವುದು
ಹೀಗೂ
ಹೀಗೆ
ಹೀಗೆಂದನು
ಹೀಗೆಂದರು
ಹೀಗೆಂದು
ಹೀಗೆಂದು-ಕೊಂಡು
ಹೀಗೆಯೆ
ಹೀಗೆಯೇ
ಹೀಗೆಲ್ಲ
ಹೀಗೇ
ಹೀನ
ಹೀನ-ಕೆಲಸ-ಗಳ
ಹೀನ-ಗತಿಗೆ
ಹೀನತೆ
ಹೀನ-ನಾಗಿ
ಹೀನ-ಬುದ್ಧಿ
ಹೀನ-ಬುದ್ಧೇಃ
ಹೀನರು
ಹೀನ-ವಾದ
ಹೀನ-ವಾದುದೇ
ಹೀನ-ವಾಯ್ತು
ಹೀನ-ವೀರ್ಯ-ರಾದ-ವರು
ಹೀನಸ್ಥಿತಿಗೆ
ಹೀನಸ್ಥಿತಿ-ಯಲ್ಲಿದ್ದಾಗ
ಹೀನಾಚರಣೆ-ಯನ್ನು
ಹೀನಾಯ-ವಾಗಿ
ಹೀನೇನ
ಹೀಯಾಳಿಸ-ಬೇಡಿ
ಹೀರಲು
ಹೀರಿ
ಹೀರಿ-ಕೊಂಡು-ಬಿಟ್ಟಿದೆ
ಹೀರಿ-ಕೊಳ್ಳುವ
ಹೀರಿ-ಕೊಳ್ಳುವುದು
ಹೀರಿ-ಕೊಳ್ಳುವುದೇ
ಹೀರಿ-ಬಿಡು-ವನು
ಹೀರು
ಹುಂ
ಹುಂಬರು
ಹುಕ್ಕ-ವನ್ನು
ಹುಚ್ಚ
ಹುಚ್ಚನ
ಹುಚ್ಚನು
ಹುಚ್ಚರ
ಹುಚ್ಚ-ರನ್ನಾಗಿ
ಹುಚ್ಚ-ರಾಗಿ-ಬಿಡು-ವರು
ಹುಚ್ಚರೆ
ಹುಚ್ಚ-ರೆನ್ನುತ್ತಾರೆ
ಹುಚ್ಚಿ-ನಿಂದ
ಹುಚ್ಚು
ಹುಚ್ಚು-ದಾರಿ-ಯಲಿ
ಹುಚ್ಚು-ವರಿದು
ಹುಚ್ಚು-ಹುಚ್ಚಾಗಿ
ಹುಚ್ಚೆಂದು
ಹುಟ್ಟಡಗಿಸಲೇ-ಬೇಕು
ಹುಟ್ಟ-ದಂತಾಗಿದೆ
ಹುಟ್ಟ-ದಿದ್ದರೆ
ಹುಟ್ಟದೆ
ಹುಟ್ಟ-ಬಹು-ದಲ್ಲವೆ
ಹುಟ್ಟ-ಬೇಕು
ಹುಟ್ಟಲು
ಹುಟ್ಟಿ
ಹುಟ್ಟಿ-ಕೊಂಡಿತು
ಹುಟ್ಟಿ-ಕೊಂಡಿವೆ
ಹುಟ್ಟಿ-ಕೊಳ್ಳುತ್ತಿದ್ದುವು
ಹುಟ್ಟಿತು
ಹುಟ್ಟಿದ
ಹುಟ್ಟಿ-ದರು
ಹುಟ್ಟಿ-ದರೆ
ಹುಟ್ಟಿ-ದಾಗಲೇ
ಹುಟ್ಟಿದ್ದ-ರಿಂದ
ಹುಟ್ಟಿದ್ದ-ರೆಂಬು-ದನ್ನು
ಹುಟ್ಟಿದ್ದಾರೆ
ಹುಟ್ಟಿದ್ದಾರೆಂದು
ಹುಟ್ಟಿದ್ದು
ಹುಟ್ಟಿ-ನಿಂದಲೇ
ಹುಟ್ಟಿಯೂ
ಹುಟ್ಟಿಯೇ
ಹುಟ್ಟಿರ-ಬೇಕು
ಹುಟ್ಟಿರು-ವುದು
ಹುಟ್ಟಿಲ್ಲ
ಹುಟ್ಟಿ-ಸದೆ
ಹುಟ್ಟಿಸ-ಬೇಕಾದು-ದಿಲ್ಲ
ಹುಟ್ಟಿಸಿದರೆ
ಹುಟ್ಟಿಸು-ವಂತಿದ್ದುವು
ಹುಟ್ಟಿ-ಹೋಗಿದೆ
ಹುಟ್ಟಿ-ಹೋಗುತ್ತದೆ
ಹುಟ್ಟು
ಹುಟ್ಟುತ
ಹುಟ್ಟುತ್ತದೆ
ಹುಟ್ಟು-ತ್ತದೆಯೋ
ಹುಟ್ಟುತ್ತಲೇ
ಹುಟ್ಟುತ್ತವೆ
ಹುಟ್ಟುತ್ತಾನೆ
ಹುಟ್ಟುತ್ತಾರೆ
ಹುಟ್ಟು-ತ್ತಾರೆಯೋ
ಹುಟ್ಟುತ್ತಿರ-ಬೇಕು
ಹುಟ್ಟು-ತ್ತಿಲ್ಲವೋ
ಹುಟ್ಟುವ
ಹುಟ್ಟು-ವಂತೆ
ಹುಟ್ಟು-ವರು
ಹುಟ್ಟು-ವವರು
ಹುಟ್ಟು-ವುದಕ್ಕಿಂತ
ಹುಟ್ಟು-ವುದಕ್ಕೆ
ಹುಟ್ಟು-ವುದಿಲ್ಲ
ಹುಟ್ಟು-ವುದಿಲ್ಲ-ವೆಂದು
ಹುಟ್ಟು-ವುದು
ಹುಟ್ಟು-ವುವು
ಹುಟ್ಟು-ಹಬ್ಬಕ್ಕಾಗಿ
ಹುಟ್ಟು-ಹಬ್ಬಕ್ಕೆಂದು
ಹುಟ್ಟು-ಹಾಕಿದ್ದಾರೆ
ಹುಟ್ಟು-ಹೊಡೆ-ಯುವ-ವರ
ಹುಟ್ಟೇ
ಹುಡಿಯು
ಹುಡಿಯೊಳ
ಹುಡುಕ-ಬೇಕೆಂದು
ಹುಡುಕಾಟ
ಹುಡುಕಿ-ಕೊಂಡು
ಹುಡುಕಿ-ಕೊಂಡು-ಹೋಗೋಣ
ಹುಡುಕಿ-ದರೂ
ಹುಡುಕಿ-ದೆನು
ಹುಡುಕಿದ್ದೇನೆ
ಹುಡುಕಿ-ನೋಡು
ಹುಡುಕುತಿತ್ತು
ಹುಡುಕುವೆ
ಹುಡುಕು-ವೆಯ-ವನ
ಹುಡುಕು-ವೆವು
ಹುಡುಗ
ಹುಡುಗ-ನಂತೆ
ಹುಡುಗ-ನಯ್ಯ
ಹುಡುಗ-ನಲ್ಲಿ
ಹುಡುಗ-ನಾಗಿದ್ದಾಗ
ಹುಡುಗ-ನಿಗೆ
ಹುಡುಗನೂ
ಹುಡುಗರ
ಹುಡುಗ-ರನ್ನು
ಹುಡುಗ-ರಾಗಿದ್ದಾಗ
ಹುಡುಗ-ರಾಗಿದ್ದು-ದರಿಂದ
ಹುಡುಗ-ರಿಂದಲೇ
ಹುಡುಗ-ರಿಗೆ
ಹುಡುಗ-ರಿಗೆಲ್ಲಾ
ಹುಡುಗರು
ಹುಡುಗ-ರೆಲ್ಲಾ
ಹುಡುಗ-ರೊಡನೆ
ಹುಡುಗಾಟಿಕೆ-ಯವರು
ಹುಡುಗಿ
ಹುಡುಗಿಗೆ
ಹುಡುಗಿಯ
ಹುಡುಗಿ-ಯನ್ನು
ಹುಡುಗಿ-ಯರ
ಹುಡುಗಿ-ಯ-ರನ್ನು
ಹುಡುಗಿ-ಯರಿಗೆ
ಹುಡುಗಿ-ಯರು
ಹುಡುಗಿ-ಯರೆಲ್ಲಾ
ಹುಡುಗಿಯು
ಹುಡುಗಿ-ಯೊಡನೆ
ಹುಣ್ಣ-ನರಸಿ
ಹುಣ್ಣಾಗು-ವಂತೆ
ಹುದು-ಗಲಹುದು
ಹುದುಗಿ-ಕೊಂಡಿರು-ವು-ದಂತೆ
ಹುದುಗಿತ್ತು
ಹುದುಗಿದೆ
ಹುದುಗಿದ್ದ
ಹುದುಗಿ-ದ್ದಾನೆ
ಹುದ್ದೆ-ಗಳು
ಹುಯಿಲಿಡುತ್ತಿದ್ದನು
ಹುರಿ-ಗಾಳನ್ನು
ಹುರಿ-ಗಾಳು
ಹುರಿದುಂಬಿಸು-ವೆನು
ಹುರುಳಿಲ್ಲದ
ಹುಲಿ
ಹುಲಿ-ಗಳೇ-ನಾದರೂ
ಹುಲಿ-ಗಾಗಿ
ಹುಲ್ಲಿ-ನೆಸಳಿನ-ವರೆಗೆ
ಹುಲ್ಲು-ಕಡ್ಡಿಯ-ವರೆಗೆ
ಹುಳುಕು
ಹುಳು-ಗಳಂತೆ
ಹುಳು-ಗಳು
ಹುಳು-ಗಳೆಂದು
ಹುಳು-ಗಳೆಂದೆಣಿಸ-ಬೇಕು
ಹುಳುವಿಗೂ
ಹುವ್ವಿನ
ಹುಷಾರ್
ಹುಸಿ-ಮಾತು
ಹುಸಿ-ರಾಶಿ
ಹುಹುಂಕುರ
ಹುಹೂಂಕಾರ
ಹೂ
ಹೂಂಕರಿಸಿಹುದು
ಹೂಂಕಾರ
ಹೂಗಳಿಂದ
ಹೂಗೊಂಚಲನ್ನು
ಹೂಗ್ಲಿ
ಹೂಜಿ
ಹೂಜಿ-ಗಳು
ಹೂಡಿ
ಹೂಡಿ-ದರು
ಹೂಡಿ-ದರೆ
ಹೂನಗೆ-ಮಳೆ-ಯನು
ಹೂಮ್ಮಿ
ಹೂರತು
ಹೂವಾಯಿತು
ಹೂವಿನ
ಹೂವು-ಗಳ
ಹೂವು-ಗಳು
ಹೃತ್ಕಂದರ
ಹೃತ್ಪಿಂಡವು
ಹೃತ್ಪೂರ್ವಕ
ಹೃತ್ಪೂರ್ವಕ-ವಾಗಿ
ಹೃತ್ಪ್ರದೇಶದ
ಹೃದಯ
ಹೃದಯಕೆ
ಹೃದಯಕ್ಕೆ
ಹೃದಯಕ್ಕೋಸ್ಕರ
ಹೃದಯಗ್ರಂಥ
ಹೃದಯಗ್ರಂಥಿಃ
ಹೃದಯದ
ಹೃದಯ-ದಂತೆ
ಹೃದಯ-ದಲ್ಲಿ
ಹೃದಯ-ದಲ್ಲಿ-ರುವ
ಹೃದಯ-ದಲ್ಲೇ
ಹೃದಯ-ದಾಳದ
ಹೃದಯ-ದಾಳದಿಂ
ಹೃದಯ-ದಿಂದ
ಹೃದಯನೂ
ಹೃದಯರು
ಹೃದಯರು-ಧಿರ
ಹೃದಯರು-ಧಿರರಸ
ಹೃದಯ-ವಂತಿಕೆಯು
ಹೃದಯ-ವನ್ನು
ಹೃದಯ-ವನ್ನೆಲ್ಲಾ
ಹೃದಯವು
ಹೃದಯ-ವುಳ್ಳವರಲ್ಲ-ವೇನು
ಹೃದಯ-ವುಳ್ಳವರಾದ್ದ-ರಿಂದಲೇ
ಹೃದಯವೂ
ಹೃದಯವೆ
ಹೃದಯ-ವೆಂಥದ್ದೆಂದರೆ
ಹೃದಯ-ವೆಲ್ಲ
ಹೃದಯ-ಸ್ಮಶಾನ
ಹೃದಯಾಂತರಾಳ-ದಲಿ
ಹೃದಯಾಂತರಾಳ-ದಲ್ಲಿ
ಹೃದಯಾಂತರಾಳ-ವನೆ
ಹೃದಯಾಕಾಂಕ್ಷೆ-ಯನ್ನು
ಹೃದಯೆ
ಹೃದಯೇ
ಹೃದಯೇರ
ಹೃದಿ
ಹೃದಿ-ಕಂದರ
ಹೃದಿ-ಗಮ್ಯಃ
ಹೃದಿ-ವಾನ್
ಹೆ
ಹೆಂಗಸನ್ನು
ಹೆಂಗಸರ
ಹೆಂಗಸ-ರನ್ನು
ಹೆಂಗಸರನ್ನೆಲ್ಲಿಂದ
ಹೆಂಗಸರಲ್ಲ-ದಿದ್ದರೆ
ಹೆಂಗಸ-ರಲ್ಲಿ
ಹೆಂಗಸರಲ್ಲಿಯೂ
ಹೆಂಗಸರಿ-ಗಾಗಿ
ಹೆಂಗಸರಿಗೆ
ಹೆಂಗಸರಿ-ಗೆಲ್ಲಾ
ಹೆಂಗಸರಿ-ಗೋಸ್ಕರ
ಹೆಂಗಸರಿ-ಗೋಸ್ಕರವೂ
ಹೆಂಗಸರಿಲ್ಲವೆ
ಹೆಂಗಸರು
ಹೆಂಗಸರೂ
ಹೆಂಗಸರೆಂದೇ
ಹೆಂಗಸ-ರೊಡನೆ
ಹೆಂಗಸಾಗು-ವುದು
ಹೆಂಗಸಿನ
ಹೆಂಗಸು
ಹೆಂಡತಿ
ಹೆಂಡತಿಗೂ
ಹೆಂಡತಿಗೆ
ಹೆಂಡತಿಯ
ಹೆಂಡತಿ-ಯರಿಗೆ
ಹೆಂಡತಿ-ಯಲ್ಲದೆ
ಹೆಂಡತಿ-ಯಾದ-ವಳು
ಹೆಂಡಿರು
ಹೆಗಲ
ಹೆಗಲನು
ಹೆಗಲನ್ನು
ಹೆಗಲ-ಮೇಲಿನ
ಹೆಗಲಿನ
ಹೆಗಲು
ಹೆಚ್ಚಲ್ಲ
ಹೆಚ್ಚಾಗಿ
ಹೆಚ್ಚಾಗಿತ್ತು
ಹೆಚ್ಚಾಗಿರ-ದಿದ್ದರೆ
ಹೆಚ್ಚಾಗುತ್ತಿತ್ತೆಂದು
ಹೆಚ್ಚಾಗೆ
ಹೆಚ್ಚಾದ
ಹೆಚ್ಚಿಗೆ
ಹೆಚ್ಚಿನ
ಹೆಚ್ಚಿರ-ಬೇಕು
ಹೆಚ್ಚಿಸಿ-ಕೊಂಡು-ಬಿಟ್ಟ
ಹೆಚ್ಚಿಸುತ್ತಿದೆ
ಹೆಚ್ಚಿಸು-ವರು
ಹೆಚ್ಚಿಸು-ವಿರಾ
ಹೆಚ್ಚಿಸು-ವುದು
ಹೆಚ್ಚಿ-ಹೋಗಿ-ರುವು-ದರಿಂದ
ಹೆಚ್ಚು
ಹೆಚ್ಚು-ಕಡಿಮೆ
ಹೆಚ್ಚು-ಕಾಲ
ಹೆಚ್ಚುತ್ತದೆ
ಹೆಚ್ಚುತ್ತಾ
ಹೆಚ್ಚುತ್ತಿತ್ತು
ಹೆಚ್ಚೆಂದರೆ
ಹೆಚ್ಚೆಂದು
ಹೆಚ್ಚೇ-ನನ್ನು
ಹೆಚ್ಚೇನು
ಹೆಚ್ಚೇನೂ
ಹೆಚ್ಚೋ
ಹೆಜ್ಜೆ
ಹೆಜ್ಜೆಗೂ
ಹೆಜ್ಜೆಗೆ
ಹೆಜ್ಜೆ-ಯನ್ನೂ
ಹೆಜ್ಜೆ-ಯನ್ನೇ
ಹೆಜ್ಜೆ-ಯಿಟ್ಟು
ಹೆಜ್ಜೆಯೆ
ಹೆಡೆಯ
ಹೆಡೆಯಾಡಿಸಿ
ಹೆಡೆಯೆತ್ತು-ವುದು
ಹೆಣ-ಗಳ
ಹೆಣಗಾಡುತ್ತಿದ್ದಾರೆ
ಹೆಣಗಾಡುತ್ತಿದ್ದೀರಿ
ಹೆಣಗಿ-ದವರು
ಹೆಣಗು-ತಿರು-ವುದು
ಹೆಣಗುತ್ತಿದ್ದಾರೆ
ಹೆಣಗು-ಬವಣೆ-ಗಳಿರದ
ಹೆಣವು
ಹೆಣೆದು-ಕೊಂಡಿದೆ
ಹೆಣೆಯಲ್ಪಡುವ
ಹೆಣೆಯಲ್ಪಡು-ವಂತೆ
ಹೆಣೆಯು-ವಾಗ
ಹೆಣ್ಣಿಗ-ಜಾತಿ-ಯಾಗಿದೆ
ಹೆಣ್ಣಿನ
ಹೆಣ್ಣು
ಹೆಣ್ಣು-ಮಕ್ಕಳನ್ನು
ಹೆತ್ತ
ಹೆತ್ತಾರು
ಹೆದರದೆ
ಹೆದರ-ಬೇಕಾದ
ಹೆದರಿಕೆ
ಹೆದರಿಕೆ-ಯನ್ನು
ಹೆದರಿಕೆ-ಯಾಗುತ್ತದೆ
ಹೆದರಿಕೆ-ಯಾಗುವುದು
ಹೆದರಿಕೆ-ಯಾಯಿತು
ಹೆದರಿಕೆ-ಯಿಂದ
ಹೆದರಿಕೆಯೂ
ಹೆದರಿ-ಕೊಂಡು
ಹೆದರಿ-ಕೊಳ್ಳುವು-ದಿಲ್ಲ
ಹೆದರಿಸಿ
ಹೆದರು-ತಿಹೆನು
ಹೆದರುತ್ತೇನೆ
ಹೆದರುತ್ತೇವೆಯೆ
ಹೆದ-ರುವರು
ಹೆದ-ರುವೆ
ಹೆನ್ರಿ
ಹೆಪ್ಪುಗಟ್ಟಿ
ಹೆಪ್ಪುಗಟ್ಟಿ-ದರು
ಹೆಬ್ಬಾಗಿಲಿ-ನಿಂದ
ಹೆಬ್ಬಾಗಿಲು
ಹೆಮ್ಮೆ
ಹೆಮ್ಮೆ-ಪಡು-ವವ-ರಿಗೂ
ಹೆಮ್ಮೆ-ಪಡು-ವಿರಿ
ಹೆಮ್ಮೆ-ಪಡುವೆ
ಹೆಮ್ಮೆ-ಯನ್ನು
ಹೆರುತ್ತ
ಹೆರುವ
ಹೆರು-ವುದು
ಹೆಳವನಾದ-ನೆಂದು
ಹೆಸರನ್ನು
ಹೆಸರನ್ನುಚ್ಚರಿ-ಸುತ್ತಾ
ಹೆಸರನ್ನೂ
ಹೆಸರಲ್ಲಿಯೂ
ಹೆಸರಿ-ಗಷ್ಟೇ
ಹೆಸರಿಗೆ
ಹೆಸರಿಟ್ಟಿದ್ದಾ-ಯಿತು
ಹೆಸರಿಟ್ಟು
ಹೆಸರಿನ
ಹೆಸರಿ-ನಡಿ
ಹೆಸರಿ-ನಲ್ಲಿ
ಹೆಸರಿ-ನಲ್ಲಿಯೆ
ಹೆಸರಿ-ನಲ್ಲಿ-ರುವ
ಹೆಸರಿ-ನಿಂದ
ಹೆಸರಿ-ರುವು-ದಾಗಿ
ಹೆಸರಿಲ್ಲದ
ಹೆಸರಿಲ್ಲ-ದಂತಾಯಿತು
ಹೆಸರು
ಹೆಸರು-ಗಳನ್ನು
ಹೆಸರು-ಗಳಿದ್ದುವು
ಹೆಸರು-ತಾನೆ
ಹೆಸರು-ವಾಸಿ-ಯಾದ
ಹೆಸರು-ವಾಸಿ-ಯಾದುವು
ಹೆಸರೆ
ಹೆಸರೇ
ಹೆಸರೊಳೆಸೆಯುತಲಿರು-ವನು
ಹೇ
ಹೇಗಾ-ಗುತ್ತದೆ
ಹೇಗಾದ-ರಾಗಲಿ
ಹೇಗಾದರಿ-ರಲಿ
ಹೇಗಾ-ದರೂ
ಹೇಗಾ-ದೀತು
ಹೇಗಿತ್ತು
ಹೇಗಿದೆ
ಹೇಗಿದೆ-ಯೆಂದರೆ
ಹೇಗಿದ್ದರೂ
ಹೇಗಿದ್ದವೋ
ಹೇಗಿದ್ದಾಳೆ
ಹೇಗಿದ್ದೀರಿ
ಹೇಗಿರ-ಬಹುದು
ಹೇಗಿ-ರುತ್ತದೆ
ಹೇಗೆ
ಹೇಗೆಂದರೆ
ಹೇಗೆಂಬ
ಹೇಗೆ-ತಾನೆ
ಹೇಗೆನ್ನಿ-ಸುತ್ತದೋ
ಹೇಗೊ
ಹೇಗೋ
ಹೇಡಿ
ಹೇಡಿ-ಗಲ್ಲ
ಹೇಡಿ-ಗಳ
ಹೇಡಿ-ಗಳನ್ನಾಗಿ
ಹೇಡಿ-ಗಳಾಗ-ಬೇಡಿ
ಹೇಡಿ-ಗಳು
ಹೇಡಿತನ
ಹೇಡಿತನಕ್ಕೆ
ಹೇಡಿತನ-ಗಳನ್ನು
ಹೇಡಿತನವೂ
ಹೇಡಿಯ
ಹೇಡಿ-ಯಂತೆ
ಹೇತುವು
ಹೇಥಾ
ಹೇಯ-ರೆಂದು
ಹೇಯ-ವಾದದ್ದು
ಹೇಯ-ವಾದು-ದಲ್ಲವೇ
ಹೇಯ-ವಾದುದು
ಹೇರಲ್ಪಡುತ್ತವೆ
ಹೇರಳ-ವಾಗಿ
ಹೇರಳ-ವಾಗಿವೆ
ಹೇರಿ
ಹೇರ್
ಹೇಲ್
ಹೇಲ್ಗೆ
ಹೇಳ
ಹೇಳ-ತಕ್ಕದ್ದೇನು
ಹೇಳ-ತೀರದು
ಹೇಳ-ತೊಡಗಿ-ದರು
ಹೇಳ-ತೊಡಗಿದ್ದೀರಿ
ಹೇಳ-ದಿದ್ದರೆ
ಹೇಳದೆ
ಹೇಳ-ಬಲ್ಲನು
ಹೇಳ-ಬಲ್ಲರು
ಹೇಳ-ಬಲ್ಲಿರಾ
ಹೇಳ-ಬಲ್ಲಿರಾದರೆ
ಹೇಳ-ಬಲ್ಲೆ
ಹೇಳ-ಬಲ್ಲೆಯಾ
ಹೇಳ-ಬಹುದು
ಹೇಳ-ಬಹು-ದೆಂದರೆ
ಹೇಳ-ಬಹು-ದೇನೆಂದರೆ
ಹೇಳ-ಬಾರದೇಕೆ
ಹೇಳ-ಬೇಕಾಗಿದ್ದಾಗ
ಹೇಳ-ಬೇಕಾಗಿರು-ವುದು
ಹೇಳ-ಬೇಕಾಗಿಲ್ಲ
ಹೇಳ-ಬೇಕಾಗು-ತ್ತದೆ
ಹೇಳ-ಬೇಕಾಗು-ವುದು
ಹೇಳ-ಬೇಕಾದರೆ
ಹೇಳ-ಬೇಕಾದ್ದೆ
ಹೇಳ-ಬೇಕಾದ್ದೇ
ಹೇಳ-ಬೇಕಾದ್ದೇನು
ಹೇಳ-ಬೇಕಾ-ಯಿತು
ಹೇಳ-ಬೇಕು
ಹೇಳ-ಬೇಕೆ
ಹೇಳ-ಬೇಕೆಂದರೆ
ಹೇಳ-ಬೇಕೆಂದು
ಹೇಳ-ಬೇಕೆಂಬಂತಿದ್ದ
ಹೇಳ-ಬೇಡ
ಹೇಳ-ಲಾಗದೆ
ಹೇಳ-ಲಾಗುವು-ದಿಲ್ಲ
ಹೇಳ-ಲಾಗುವು-ದಿಲ್ಲವೆ
ಹೇಳ-ಲಾರದೆ
ಹೇಳ-ಲಾರೆ
ಹೇಳ-ಲಾರೆವು
ಹೇಳಲಿ
ಹೇಳ-ಲಿಲ್ಲ
ಹೇಳ-ಲಿಲ್ಲವೆ
ಹೇಳಲು
ಹೇಳಲೂ
ಹೇಳಲೇ-ಬೇಕೆಂದು
ಹೇಳಲ್ಪಟ್ಟಿ-ದೆ-ಯಲ್ಲಾ
ಹೇಳಿ
ಹೇಳಿಕೆ
ಹೇಳಿಕೆಗೂ
ಹೇಳಿಕೆಯ
ಹೇಳಿಕೆ-ಯನ್ನಾಗಲೀ
ಹೇಳಿ-ಕೊಂಡು
ಹೇಳಿ-ಕೊಂಡೆನು
ಹೇಳಿ-ಕೊಟ್ಟರೆ
ಹೇಳಿ-ಕೊಟ್ಟರೇ
ಹೇಳಿ-ಕೊಡ-ಬೇಕು
ಹೇಳಿ-ಕೊಡಿ
ಹೇಳಿ-ಕೊಡುತ್ತಿದ್ದನು
ಹೇಳಿ-ಕೊಡುತ್ತಿದ್ದರು
ಹೇಳಿ-ಕೊಡುತ್ತೇನೆ
ಹೇಳಿ-ಕೊಡು-ವು-ದಕ್ಕೂ
ಹೇಳಿ-ಕೊಡು-ವುದು
ಹೇಳಿ-ಕೊಳ್ಳಲಿ
ಹೇಳಿ-ಕೊಳ್ಳುತ್ತಿರು-ವರು
ಹೇಳಿ-ಕೊಳ್ಳುತ್ತೀರಿ
ಹೇಳಿ-ಕೊಳ್ಳುವ-ವರು
ಹೇಳಿದ
ಹೇಳಿ-ದಂತಿತ್ತು
ಹೇಳಿ-ದಂತೆ
ಹೇಳಿ-ದಂತೆಯೇ
ಹೇಳಿ-ದನಲ್ಲವೆ
ಹೇಳಿ-ದನು
ಹೇಳಿ-ದನೋ
ಹೇಳಿ-ದರಲ್ಲವೆ
ಹೇಳಿ-ದರು
ಹೇಳಿ-ದರೂ
ಹೇಳಿದರೆ
ಹೇಳಿದ-ರೆಂದರೆ
ಹೇಳಿದ-ವರು
ಹೇಳಿದ-ಹಾಗಾಯಿ-ತೆಂದರೆ
ಹೇಳಿ-ದಾಗ
ಹೇಳಿದಿ-ರಲ್ಲ
ಹೇಳಿದಿ-ರಲ್ಲಾ
ಹೇಳಿದಿರಿ
ಹೇಳಿದು-ದನ್ನು
ಹೇಳಿದು-ದರ
ಹೇಳಿದು-ದರಿಂದ
ಹೇಳಿದು-ದೆಲ್ಲಾ
ಹೇಳಿದೆ
ಹೇಳಿದೆನು
ಹೇಳಿದೆ-ಯಂತೆ
ಹೇಳಿದೆ-ಯಲ್ಲವೆ
ಹೇಳಿದೆ-ಯಲ್ಲಾ
ಹೇಳಿದೆಯೇ
ಹೇಳಿದೆ-ಯೇನು
ಹೇಳಿದೆಯೋ
ಹೇಳಿದ್ದನ್ನು
ಹೇಳಿದ್ದನ್ನೆಲ್ಲಾ
ಹೇಳಿದ್ದ-ರಿಂದ
ಹೇಳಿದ್ದರು
ಹೇಳಿದ್ದಲ್ಲದೆ
ಹೇಳಿದ್ದಾರೆ
ಹೇಳಿದ್ದಾರೆಂದ
ಹೇಳಿದ್ದಾರೆ-ಯೇನು
ಹೇಳಿದ್ದಿರಿ
ಹೇಳಿದ್ದೀರಲ್ಲಾ
ಹೇಳಿದ್ದೀರಿ
ಹೇಳಿದ್ದು
ಹೇಳಿದ್ದೆ
ಹೇಳಿದ್ದೇಕೆ
ಹೇಳಿದ್ದೇನು
ಹೇಳಿದ್ದೇನೆ
ಹೇಳಿದ್ದೇ-ನೆಂದರೆ
ಹೇಳಿದ್ದೇವೆ
ಹೇಳಿ-ಬಿಟ್ಟರು
ಹೇಳಿ-ಬಿಟ್ಟೆ
ಹೇಳಿ-ಬಿಡ-ಬಾರದೇಕೆ
ಹೇಳಿ-ಬಿಡುತ್ತೇನೆ
ಹೇಳಿಯೂ
ಹೇಳಿರ-ಲಿಲ್ಲವೆ
ಹೇಳಿ-ರುವ
ಹೇಳಿರು-ವಂತೆ
ಹೇಳಿರು-ವವ-ರೆಲ್ಲ
ಹೇಳಿರು-ವುದಕ್ಕೆ
ಹೇಳಿರು-ವುದನ್ನು
ಹೇಳಿರು-ವುದರ
ಹೇಳಿರು-ವುದು
ಹೇಳಿರು-ವುದೇನು
ಹೇಳಿರು ವುದೇನೆಂದರೆ
ಹೇಳಿ-ರುವೆ
ಹೇಳಿಲ್ಲ
ಹೇಳಿಸಿ-ದರು
ಹೇಳು
ಹೇಳುತ್ತ
ಹೇಳುತ್ತದೆ
ಹೇಳುತ್ತವೆ
ಹೇಳುತ್ತ-ವೆಯೋ
ಹೇಳುತ್ತಾ
ಹೇಳುತ್ತಾನೆ
ಹೇಳು-ತ್ತಾನೆಂಬು-ದನ್ನು
ಹೇಳುತ್ತಾರಲ್ಲ
ಹೇಳುತ್ತಾರೆ
ಹೇಳುತ್ತಾರೆ-ಯಲ್ಲಾ
ಹೇಳುತ್ತಾರೆಯೋ
ಹೇಳುತ್ತಾರೋ
ಹೇಳುತ್ತಿ
ಹೇಳುತ್ತಿತ್ತು
ಹೇಳುತ್ತಿ-ದೆಯೋ
ಹೇಳುತ್ತಿದ್ದ
ಹೇಳು-ತ್ತಿದ್ದಂತೆಯೇ
ಹೇಳು-ತ್ತಿದ್ದದ್ದು
ಹೇಳು-ತ್ತಿದ್ದನು
ಹೇಳು-ತ್ತಿದ್ದ-ರಲ್ಲ
ಹೇಳು-ತ್ತಿದ್ದ-ರಲ್ಲವೇ
ಹೇಳು-ತ್ತಿದ್ದರು
ಹೇಳು-ತ್ತಿದ್ದರೆ
ಹೇಳು-ತ್ತಿದ್ದ-ರೆಂದರೆ
ಹೇಳು-ತ್ತಿದ್ದ-ರೆಂಬು-ದನ್ನು
ಹೇಳು-ತ್ತಿದ್ದ-ರೆಂಬು-ದರ
ಹೇಳು-ತ್ತಿದ್ದರೊ
ಹೇಳು-ತ್ತಿದ್ದರೋ
ಹೇಳು-ತ್ತಿದ್ದಾಗ
ಹೇಳು-ತ್ತಿದ್ದಾರೆ
ಹೇಳು-ತ್ತಿದ್ದೀರಿ
ಹೇಳು-ತ್ತಿದ್ದೀರೇನು
ಹೇಳು-ತ್ತಿದ್ದುದನ್ನು
ಹೇಳು-ತ್ತಿದ್ದು-ದೇನೆಂದರೆ
ಹೇಳು-ತ್ತಿದ್ದುದೊಂದೇ
ಹೇಳು-ತ್ತಿದ್ದೆ
ಹೇಳು-ತ್ತಿದ್ದೆನೋ
ಹೇಳು-ತ್ತಿದ್ದೆ-ಯೇನು
ಹೇಳು-ತ್ತಿದ್ದೇನೆ
ಹೇಳು-ತ್ತಿರಬೇಕು
ಹೇಳು-ತ್ತಿರಲಿಲ್ಲ
ಹೇಳು-ತ್ತಿರುವ
ಹೇಳು-ತ್ತಿರುವಂತಿದ್ದೇವೆ
ಹೇಳು-ತ್ತಿರು-ವಂತೆಯೇ
ಹೇಳು-ತ್ತಿರುವರು
ಹೇಳು-ತ್ತಿರುವಾಗ
ಹೇಳು-ತ್ತಿರು-ವುದು
ಹೇಳು-ತ್ತಿರು-ವುದೆಲ್ಲಾ
ಹೇಳು-ತ್ತಿರು-ವುದೇನು
ಹೇಳು-ತ್ತಿರುವೆ
ಹೇಳು-ತ್ತಿಲ್ಲ
ಹೇಳುತ್ತೀಯೆ
ಹೇಳುತ್ತೀಯೇನು
ಹೇಳುತ್ತೀರಿ
ಹೇಳು-ತ್ತೀರೊ
ಹೇಳು-ತ್ತೇನೆ
ಹೇಳು-ತ್ತೇನೆಯೋ
ಹೇಳು-ತ್ತೇವೆ
ಹೇಳುವ
ಹೇಳು-ವಂತಹ
ಹೇಳು-ವಂತೆ
ಹೇಳು-ವನು
ಹೇಳು-ವರಲ್ಲ
ಹೇಳು-ವರು
ಹೇಳು-ವರೋ
ಹೇಳು-ವಳು
ಹೇಳು-ವಷ್ಟು
ಹೇಳು-ವಾಗ
ಹೇಳು-ವಿರಲ್ಲ
ಹೇಳು-ವಿರಿ
ಹೇಳು-ವುದಕ್ಕಾಗಲೀ
ಹೇಳು-ವುದಕ್ಕಾ-ಗುವು-ದಿಲ್ಲ
ಹೇಳು-ವುದಕ್ಕಾ-ಗುವು-ದಿಲ್ಲವೇ
ಹೇಳು-ವು-ದಕ್ಕೂ
ಹೇಳು-ವುದಕ್ಕೆ
ಹೇಳು-ವುದಕ್ಕೇ-ನಿದೆ
ಹೇಳು-ವು-ದನ್ನು
ಹೇಳು-ವುದ-ರಲ್ಲಿ
ಹೇಳು-ವು-ದಷ್ಟೇ
ಹೇಳು-ವು-ದಾದರೆ
ಹೇಳು-ವು-ದಿಲ್ಲ
ಹೇಳು-ವು-ದಿಲ್ಲವೆ
ಹೇಳು-ವುದಿಷ್ಟೆ
ಹೇಳು-ವುದು
ಹೇಳು-ವುದೂ
ಹೇಳು-ವು-ದೆಂದರೆ
ಹೇಳು-ವು-ದೆಲ್ಲ
ಹೇಳು-ವುದೇ
ಹೇಳು-ವು-ದೇನು
ಹೇಳು-ವುದೇನೆಂದರೆ
ಹೇಳು-ವು-ದೇನೋ
ಹೇಳು-ವುವು
ಹೇಳುವೆ
ಹೇಳುವೆನು
ಹೇಳುವೆ-ಯೇನು
ಹೇಸಿಗೆ
ಹೇಸಿಗೆಯೆನಿ-ಸುತ್ತದೆ
ಹೈಡ್ರಾಲಿಕ್
ಹೈಮವತಿಯವಳು
ಹೊಂಚಿ
ಹೊಂದ-ಬಲ್ಲೆವು
ಹೊಂದ-ಬಹುದು
ಹೊಂದ-ಬೇಕು
ಹೊಂದಲಿ
ಹೊಂದಲು
ಹೊಂದಲೇ-ಬೇಕು
ಹೊಂದಾಣಿಕೆ
ಹೊಂದಾಣಿಕೆ-ಗನುಸಾರ-ವಾಗಿ
ಹೊಂದಾಣಿಕೆಯೇ
ಹೊಂದಿ
ಹೊಂದಿ-ಕೊಂಡಿದ್ದರೆ
ಹೊಂದಿ-ಕೊಂಡು
ಹೊಂದಿ-ಕೊಳ್ಳಲು
ಹೊಂದಿ-ಕೊಳ್ಳುವ
ಹೊಂದಿ-ಕೊಳ್ಳುವುದು
ಹೊಂದಿತ್ತು
ಹೊಂದಿದ
ಹೊಂದಿ-ದರೆ
ಹೊಂದಿ-ದವರು
ಹೊಂದಿದೆ
ಹೊಂದಿದ್ದಾರೆ
ಹೊಂದಿದ್ದೀಯೇ
ಹೊಂದಿದ್ದೀರಿ
ಹೊಂದಿದ್ದುವು
ಹೊಂದಿದ್ದು-ವೆಂದರೆ
ಹೊಂದಿದ್ದೇನೆ
ಹೊಂದಿದ್ದೇವೆ
ಹೊಂದಿರ-ಬಹುದು
ಹೊಂದಿರ-ಬಾರದು
ಹೊಂದಿರ-ಬೇಕು
ಹೊಂದಿರಲೇ-ಬೇಕು
ಹೊಂದಿರು-ತ್ತವೆ
ಹೊಂದಿರುವ
ಹೊಂದಿರು-ವವನು
ಹೊಂದಿರು-ವುದು
ಹೊಂದಿರು-ವುದೆಲ್ಲಾ
ಹೊಂದಿರು-ವುವು
ಹೊಂದಿರುವೆ
ಹೊಂದಿಲ್ಲವೊ
ಹೊಂದು
ಹೊಂದು-ತಲಿರಲು
ಹೊಂದುತ್ತದೆ
ಹೊಂದುತ್ತವೆ
ಹೊಂದುತ್ತಾರೆ
ಹೊಂದುತ್ತಿತ್ತು
ಹೊಂದುತ್ತಿದೆ
ಹೊಂದು-ವಂತೆ
ಹೊಂದು-ವಂತೆಯೇ
ಹೊಂದು-ವನೆಂದಲ್ಲ
ಹೊಂದು-ವರು
ಹೊಂದು-ವಿರಿ
ಹೊಂದು-ವುದಕ್ಕೆ
ಹೊಂದು-ವುದಿಲ್ಲ
ಹೊಂದು-ವುದು
ಹೊಂದು-ವುದೋ
ಹೊಂದುವೆ
ಹೊಂದು-ವೆವು
ಹೊಂದೇ
ಹೊಂಬಣ್ಣದ
ಹೊಕ್ಕಿತು
ಹೊಕ್ಕಿತ್ತು
ಹೊಕ್ಕಿದ್ದರೆ
ಹೊಕ್ಕು-ಗಳ
ಹೊಕ್ಕು-ದರ
ಹೊಗ-ಲಾರದು
ಹೊಗಳ-ದಿರ-ಲಾರರು
ಹೊಗಳ-ಬೇಡಿ
ಹೊಗಳಲಿ
ಹೊಗಳಿ
ಹೊಗಳಿಕೆ
ಹೊಗಳಿಕೆಯೊ
ಹೊಗಳಿಸಿ-ಕೊಂಬರು
ಹೊಗಳುತ್ತಿದ್ದರು
ಹೊಗಳುತ್ತಿದ್ದಿರ-ಬಹುದು
ಹೊಗಳುವ
ಹೊಗಳು-ವರಾರು
ಹೊಗಳು-ವರು
ಹೊಗಳು-ವಾಗ
ಹೊಗಳು-ವುದಕ್ಕೂ
ಹೊಗಳು-ವುದಕ್ಕೆ
ಹೊಗಿ-ನೋಡು
ಹೊಗು-ತಿಹುದು
ಹೊಗುತ್ತ
ಹೊಗೆ
ಹೊಗೆ-ಬಂಡಿ
ಹೊಗೆ-ಬಂಡಿ-ಯಲ್ಲಿ
ಹೊಗೆ-ಯಿಂದ
ಹೊಗೆ-ಸೊಪ್ಪನ್ನು
ಹೊಟ್ಟೆ
ಹೊಟ್ಟೆ-ಗಾಗಿಯೇ
ಹೊಟ್ಟೆ-ಗಿಲ್ಲ-ದಿರುವಿಕೆ
ಹೊಟ್ಟೆ-ಗಿಲ್ಲದೆ
ಹೊಟ್ಟೆಗೆ
ಹೊಟ್ಟೆ-ಪಾಡಿಗೆ
ಹೊಟ್ಟೆ-ಬಟ್ಟೆ-ಗಾಗುವಷ್ಟನ್ನು
ಹೊಟ್ಟೆಯ
ಹೊಟ್ಟೆ-ಯನ್ನು
ಹೊಟ್ಟೆ-ಯಲ್ಲಿ
ಹೊಡಿ
ಹೊಡೆತ
ಹೊಡೆತವೇ
ಹೊಡೆದಟ್ಟ-ಬೇಕು
ಹೊಡೆ-ದರೆ
ಹೊಡೆದಾಟ
ಹೊಡೆದಾಟ-ವಾಗುತ್ತದೆ
ಹೊಡೆ-ದಾಡುತ್ತಾರೆ
ಹೊಡೆ-ದಾಡುತ್ತಿದ್ದೆ
ಹೊಡೆ-ದಾಡುತ್ತಿರುವ-ರೆಂಬುದು
ಹೊಡೆ-ದಾಡುತ್ತೇವೆ
ಹೊಡೆದು
ಹೊಡೆದು-ಕೊಳ್ಳಲು
ಹೊಡೆದೆಬ್ಬಿಸು
ಹೊಡೆ-ದೋಡಿಸಲು
ಹೊಡೆ-ಯಿತು
ಹೊಡೆ-ಯುತ್ತಾ
ಹೊಡೆ-ಯುತ್ತಾರೆ
ಹೊಡೆ-ಯುವ
ಹೊಡೆ-ಯುವುದು
ಹೊಣೆ
ಹೊಣೆ-ಗಳವು
ಹೊಣೆ-ಯಲ್ಲವೆ
ಹೊಣೆ-ಯಾಗಿರ-ಲಿಲ್ಲ
ಹೊಣೆ-ಯಿದು
ಹೊಣೆ-ಯೆಂಬ
ಹೊತೆ
ಹೊತ್ತ
ಹೊತ್ತನ್ನು
ಹೊತ್ತಾಗಿ
ಹೊತ್ತಾಗುವು-ದಿಲ್ಲ
ಹೊತ್ತಾದ
ಹೊತ್ತಾದರೂ
ಹೊತ್ತಾಯಿತು
ಹೊತ್ತಿ-ಗಾಗಲೇ
ಹೊತ್ತಿಗೆ
ಹೊತ್ತಿಗೇ
ಹೊತ್ತಿನ
ಹೊತ್ತಿನ-ಮೇಲೆ
ಹೊತ್ತಿನಲ್ಲಿಯೇ
ಹೊತ್ತಿ-ನಿಂದಲೂ
ಹೊತ್ತಿಸಿ
ಹೊತ್ತಿಸಿ-ಕೊಂಡು
ಹೊತ್ತು
ಹೊತ್ತು-ಕೊಂಡನು
ಹೊತ್ತು-ಕೊಂಡು
ಹೊತ್ತು-ಗೊತ್ತಿಲ್ಲದೆ
ಹೊತ್ತೂ
ಹೊತ್ತೆಲ್ಲಾ
ಹೊದ
ಹೊದಿಕೆ
ಹೊದಿಕೆ-ಯಲಿ
ಹೊದಿಸಿ
ಹೊನಲಿನ
ಹೊನಲು
ಹೊನ್ನನ್ನು
ಹೊನ್ನಿನ
ಹೊಮ್ಮಿ
ಹೊಮ್ಮಿತು
ಹೊಮ್ಮಿದೆ
ಹೊಮ್ಮಿ-ಸುತ
ಹೊಮ್ಮು-ವಂತಿರ-ಬೇಕು
ಹೊಯ
ಹೊಯಿ
ಹೊಯೆ-ಜಾಯ್
ಹೊಯೇ
ಹೊರ
ಹೊರಕ್ಕೆ
ಹೊರಗಡೆ
ಹೊರಗಡೆ-ಯಿಂದ
ಹೊರಗಣ
ಹೊರಗಿ-ದ್ದಾನೆ
ಹೊರಗಿನ
ಹೊರಗಿ-ನದೆ
ಹೊರಗಿನ-ವರಿಗೆ
ಹೊರಗಿ-ನಿಂದ
ಹೊರಗಿನಿರವನು
ಹೊರಗಿ-ರಲಿ
ಹೊರಗು-ಗಳೆಂಬ
ಹೊರಗೂ
ಹೊರಗೆ
ಹೊರಗೆಡಹು
ಹೊರ-ಗೆಲ್ಲ
ಹೊರಗೇ
ಹೊರ-ಗೇಕೆ
ಹೊರ-ಚೆಲ್ಲಿ
ಹೊರಟ
ಹೊರಟನು
ಹೊರಟ-ರಂತೆ
ಹೊರಟರು
ಹೊರಟರೆ
ಹೊರಟಿತು
ಹೊರಟಿದ್ದ
ಹೊರಟಿದ್ದಾರೆ
ಹೊರಟಿದ್ದೇನೆ
ಹೊರಟಿದ್ದೇವೆ
ಹೊರಟಿರ-ಬೇಕು
ಹೊರಟಿ-ರುವ
ಹೊರಟಿರು-ವರೋ
ಹೊರಟಿರು-ವೆಯೋ
ಹೊರಟಿವೆ
ಹೊರಟು
ಹೊರಟು-ಬಂದೆ
ಹೊರಟು-ಹೋಗಿ
ಹೊರಟು-ಹೋಗಿ-ಬಿಡುತ್ತಿದ್ದರು
ಹೊರಟು-ಹೋಗು
ಹೊರಟು-ಹೋದ
ಹೊರಟು-ಹೋದ-ನಲ್ಲ
ಹೊರಟು-ಹೋದನು
ಹೊರಟು-ಹೋದರು
ಹೊರಟು-ಹೋದರೆ
ಹೊರಟು-ಹೋದೆ
ಹೊರಟು-ಹೋಯಿತು
ಹೊರಟೆ
ಹೊರಡದಂತಾಯಿತು
ಹೊರಡದೆ
ಹೊರಡ-ಬೇಕು
ಹೊರಡ-ಬೇಕೆಂದರ್ಥ-ವಲ್ಲ
ಹೊರಡ-ಲಿಲ್ಲ
ಹೊರಡಲು
ಹೊರಡಿ
ಹೊರಡಿಸ-ಬೇಕೆಂದು
ಹೊರಡಿಸಿದ
ಹೊರಡಿಸಿ-ದನು
ಹೊರಡಿಸಿ-ದಾಗ
ಹೊರಡಿಸಿ-ದ್ದರು
ಹೊರಡಿಸಿ-ಬಿಡು-ವರು
ಹೊರಡಿಸು-ವುದೇ
ಹೊರಡು
ಹೊರಡುತ್ತವೆ
ಹೊರಡುತ್ತಾರೆ
ಹೊರಡು-ತ್ತಾರೆಯೋ
ಹೊರಡುತ್ತಿದ್ದ
ಹೊರಡು-ವನು
ಹೊರಡು-ವರು
ಹೊರಡು-ವಾಗ
ಹೊರಡು-ವಿರೋ
ಹೊರಡು-ವುದಕ್ಕೆ
ಹೊರಡು-ವುದೇ
ಹೊರಡು-ವೆಯಾ
ಹೊರ-ತರುತ್ತದೆ
ಹೊರ-ತಾಗಿ
ಹೊರ-ತಿನ್ನಾರು
ಹೊರತು
ಹೊರ-ತೆಗೆದು
ಹೊರ-ದೂಡ-ಬೇಕಾಗಿದೆ
ಹೊರ-ದೇಶ-ಗಳಲ್ಲಿ
ಹೊರ-ಪಡಿಸಲು
ಹೊರ-ಬಾಗಿಲಿನ-ವರೆಗೆ
ಹೊರ-ಬಿದ್ದ
ಹೊರ-ಬೀಳು-ವುವು
ಹೊರ-ಬೇಕಲ್ಲದೆ
ಹೊರ-ಬೇಕಾಗು-ವುದು
ಹೊರ-ಬೇಕೆಂಬ
ಹೊರಳಿಸಿ-ರುವೆ
ಹೊರಳು-ತಿದ್ದೆ
ಹೊರಳು-ತಿಹ
ಹೊರ-ಸೂಸುತ್ತಾ
ಹೊರ-ಹೊಮ್ಮಲಿ
ಹೊರ-ಹೊಮ್ಮಿ
ಹೊರ-ಹೊಮ್ಮಿತು
ಹೊರ-ಹೊಮ್ಮುತ್ತವೆ
ಹೊರ-ಹೊಮ್ಮು-ವನು
ಹೊರಿ-ಸಿದ್ದರೂ
ಹೊರಿಸಿ-ಬಿಡುತ್ತಾರೆ
ಹೊರಿಸುವ
ಹೊರಿಸು-ವುದು
ಹೊರುವ
ಹೊರು-ವುದು
ಹೊರೆ
ಹೊರೆದು-ಕೊಳ್ಳು-ತ್ತದೆಯೋ
ಹೊಲ
ಹೊಲ-ದಲ್ಲಿ
ಹೊಲಸು
ಹೊಲಿಗೆ
ಹೊಲಿಯು-ವವರು
ಹೊಲಿಯು-ವುದು
ಹೊಲೆಯ-ನನ್ನು
ಹೊಲೆಯ-ರನ್ನು
ಹೊಲೆಯರು
ಹೊಳಪು
ಹೊಳೆ
ಹೊಳೆದು
ಹೊಳೆ-ಯದೆ
ಹೊಳೆ-ಯಲ್ಲಿ
ಹೊಳೆ-ಯುತ್ತದೆ
ಹೊಳೆ-ಯುತ್ತಿತ್ತು
ಹೊಳೆ-ಯುತ್ತಿದೆ
ಹೊಳೆ-ಯುತ್ತಿದ್ದ
ಹೊಳೆ-ಯುವನು
ಹೊಳೆ-ಯುವುದು
ಹೊಳೆವ
ಹೊಳೆ-ವುದರ
ಹೊಳೆ-ಹೊಳೆ-ಯುತ-ಲಿರೆ
ಹೊಳೆ-ಹೊಳೆವ
ಹೊಸ
ಹೊಸ-ತಾಗಿ
ಹೊಸ-ತಾದದ್ದೇ-ನನ್ನಾದರೂ
ಹೊಸ-ತಾದುದೇ-ನಾದರೂ
ಹೊಸತು
ಹೊಸ-ದಾಗಿ
ಹೊಸ-ದಾಗಿ-ರುವ
ಹೊಸ-ದೇನೂ
ಹೊಸ-ದೊಂದು
ಹೊಸ-ಪರಿಯ
ಹೊಸ-ಮಠ
ಹೊಸ-ಮಠದ
ಹೊಸ-ಮಠ-ವನ್ನು
ಹೊಸ-ರಾಗ-ವನ್ನು
ಹೊಸ-ರೀತಿ-ಯಲ್ಲಿ
ಹೊಸ-ವರ್ಗ-ದ-ವನು
ಹೋ
ಹೋಕ
ಹೋಕ್
ಹೋಗ-ಕೂ-ಡದು
ಹೋಗ-ಗೊಡಿ-ಸು-ವು-ದಿಲ್ಲ
ಹೋಗ-ದಿ-ರು-ವುದು
ಹೋಗದೆ
ಹೋಗ-ಬಲ್ಲರೆ
ಹೋಗ-ಬ-ಹುದು
ಹೋಗ-ಬಾ-ರದೆಂದು
ಹೋಗ-ಬೇಕಾಗಿದೆ
ಹೋಗ-ಬೇಕಾಗಿ-ದೆ-ಯೆಂಬು-ದನ್ನೂ
ಹೋಗ-ಬೇಕಾಗಿದ್ದು-ದ-ರಿಂದ
ಹೋಗ-ಬೇ-ಕಾದರೆ
ಹೋಗಬೇ-ಕಾ-ಯಿತು
ಹೋಗ-ಬೇಕು
ಹೋಗ-ಬೇಕೆ
ಹೋಗ-ಬೇಕೆಂದು
ಹೋಗ-ಬೇಕೆಂಬ
ಹೋಗ-ಬೇಕೆಂಬು-ದರ
ಹೋಗ-ಬೇಕೆನ್ನುತ್ತೀ-ಯಲ್ಲಾ
ಹೋಗ-ಬೇಡ
ಹೋಗ-ಬೇಡಿ
ಹೋಗ-ಲಾಗುವು-ದಿಲ್ಲ
ಹೋಗ-ಲಾಡಿಸ-ಲಾಗುವು-ದಿಲ್ಲ
ಹೋಗ-ಲಾಡಿಸ-ಲಾರ-ದಿದ್ದರೂ
ಹೋಗ-ಲಾಡಿಸಲು
ಹೋಗ-ಲಾಡಿಸಿದನೋ
ಹೋಗ-ಲಾಡಿಸುವ
ಹೋಗ-ಲಾಡಿಸು-ವನೊ
ಹೋಗ-ಲಾಡಿಸು-ವುದಕ್ಕೆ
ಹೋಗ-ಲಾರದು
ಹೋಗ-ಲಾರೆವು
ಹೋಗಲಿ
ಹೋಗಲಿಚ್ಛಿಸುವೆ
ಹೋಗಲಿಲ್ಲ
ಹೋಗಲು
ಹೋಗಲೇ
ಹೋಗಲೇ-ಬೇಕಾಗಿತ್ತು
ಹೋಗಲೇ-ಬೇಕು
ಹೋಗಲೊಪ್ಪಿ-ದನು
ಹೋಗಿ
ಹೋಗಿತ್ತು
ಹೋಗಿತ್ತೋ
ಹೋಗಿದೆ
ಹೋಗಿ-ದೆಯೊ
ಹೋಗಿದ್ದ
ಹೋಗಿದ್ದನು
ಹೋಗಿದ್ದರು
ಹೋಗಿದ್ದರೆ
ಹೋಗಿದ್ದಾನೆ
ಹೋಗಿದ್ದಾರೆ
ಹೋಗಿದ್ದೀ-ರೆಂದು
ಹೋಗಿದ್ದೆ
ಹೋಗಿ-ಬಂದ
ಹೋಗಿ-ಬಂದರೆ
ಹೋಗಿ-ಬಂದಿದ್ದ-ರೆಂದು
ಹೋಗಿ-ಬರು-ತ್ತಿದ್ದರಿಂದ
ಹೋಗಿ-ಬರು-ತ್ತಿದ್ದಾನೆ
ಹೋಗಿ-ಬರು-ವುದಕ್ಕೆ
ಹೋಗಿ-ಬರು-ವುದಕ್ಕೆಂದು
ಹೋಗಿ-ಬಿಟ್ಟಿತ್ತು
ಹೋಗಿ-ಬಿಟ್ಟಿದ್ದೆ-ನೆಂದು
ಹೋಗಿ-ಬಿಡಲಿ
ಹೋಗಿ-ಬಿಡು
ಹೋಗಿ-ಬಿಡುತ್ತದೆ
ಹೋಗಿ-ಬಿಡುತ್ತದೆಯೋ
ಹೋಗಿ-ಬಿಡುತ್ತಿದ್ದರು
ಹೋಗಿ-ಬಿಡುತ್ತಿದ್ದರೂ
ಹೋಗಿ-ಬಿಡು-ವನೊ
ಹೋಗಿ-ಬಿಡು-ವರು
ಹೋಗಿಯೇ
ಹೋಗಿ-ರುತ್ತಾ-ನೆಯೋ
ಹೋಗಿ-ರುವಂತಿವೆ
ಹೋಗಿ-ರುವರು
ಹೋಗಿ-ರುವರೋ
ಹೋಗಿ-ರುವು-ದನ್ನು
ಹೋಗಿ-ರುವು-ದರಿಂದ
ಹೋಗಿ-ರುವೆ
ಹೋಗಿ-ರೆಂದು
ಹೋಗಿವೆ
ಹೋಗು
ಹೋಗು-ತಿಹ
ಹೋಗುತ್ತ
ಹೋಗುತ್ತದೆ
ಹೋಗು-ತ್ತದೆಯೆ
ಹೋಗು-ತ್ತದೆಯೇನು
ಹೋಗು-ತ್ತದೆಯೋ
ಹೋಗುತ್ತವೆ
ಹೋಗುತ್ತಾ
ಹೋಗು-ತ್ತಾನೆ
ಹೋಗು-ತ್ತಾನೆಯೆ
ಹೋಗು-ತ್ತಾನೆಯೋ
ಹೋಗು-ತ್ತಾರೆ
ಹೋಗು-ತ್ತಾರೆಂದು
ಹೋಗು-ತ್ತಾರೆ-ಯೇನು
ಹೋಗು-ತ್ತಿತ್ತು
ಹೋಗುತ್ತಿದೆ
ಹೋಗುತ್ತಿದೆ-ಯೆಂಬುದು
ಹೋಗುತ್ತಿದೆಯೋ
ಹೋಗುತ್ತಿದ್ದ
ಹೋಗುತ್ತಿದ್ದರು
ಹೋಗುತ್ತಿದ್ದರೆ
ಹೋಗುತ್ತಿದ್ದ-ರೆಂದು
ಹೋಗುತ್ತಿದ್ದಾಗ
ಹೋಗುತ್ತಿದ್ದಾನೆ
ಹೋಗುತ್ತಿದ್ದಾರೆ
ಹೋಗುತ್ತಿದ್ದಾರೆಂಬು-ದನ್ನು
ಹೋಗುತ್ತಿದ್ದೀಯೆ
ಹೋಗುತ್ತಿದ್ದು-ದನ್ನು
ಹೋಗುತ್ತಿದ್ದೆನು
ಹೋಗುತ್ತಿದ್ದೆವು
ಹೋಗುತ್ತಿದ್ದೇನೆ
ಹೋಗುತ್ತಿದ್ದೇನೆಂದು
ಹೋಗುತ್ತಿದ್ದೇವೆ-ಯೇನೊ
ಹೋಗುತ್ತಿರ-ಬೇಕು
ಹೋಗುತ್ತಿರ-ಲಿಲ್ಲ
ಹೋಗುತ್ತಿರ-ಲಿಲ್ಲ-ವೆಂದು
ಹೋಗುತ್ತಿರುತ್ತವೆ
ಹೋಗುತ್ತಿರುತ್ತಾರೆ
ಹೋಗುತ್ತಿರುವ
ಹೋಗುತ್ತಿರುವು-ದನ್ನು
ಹೋಗುತ್ತಿರುವೆ
ಹೋಗುತ್ತಿಲ್ಲವೆ
ಹೋಗು-ತ್ತೀಯೆ
ಹೋಗು-ತ್ತೇನೆ
ಹೋಗು-ತ್ತೇವೆ
ಹೋಗುವ
ಹೋಗು-ವಂತೆ
ಹೋಗು-ವ-ತನಕ
ಹೋಗು-ವನು
ಹೋಗು-ವರಲ್ಲದೆ
ಹೋಗು-ವರು
ಹೋಗು-ವರೋ
ಹೋಗು-ವವ-ನಿಗೂ
ಹೋಗು-ವವ-ರಾಗಿದ್ದರೊ
ಹೋಗು-ವವರು
ಹೋಗು-ವಷ್ಟ-ರಲ್ಲಿಯೇ
ಹೋಗು-ವಾಗ
ಹೋಗು-ವಿರಿ
ಹೋಗು-ವುದಕ್ಕಿಂತ
ಹೋಗು-ವುದಕ್ಕೆ
ಹೋಗು-ವುದನ್ನು
ಹೋಗು-ವುದನ್ನೇ
ಹೋಗು-ವುದ-ರಲ್ಲಿ
ಹೋಗು-ವುದ-ರಿಂದ
ಹೋಗು-ವುದಷ್ಟೇ
ಹೋಗು-ವುದಿಲ್ಲ
ಹೋಗು-ವುದಿಲ್ಲವೇ
ಹೋಗು-ವುದು
ಹೋಗುವುದೂ
ಹೋಗುವು-ದೆಂದು
ಹೋಗುವುದೇನು
ಹೋಗು-ವುದೊಳ್ಳೆ-ಯದೇ
ಹೋಗು-ವುದೋ
ಹೋಗು-ವುವು
ಹೋಗು-ವುವೋ
ಹೋಗುವೆ
ಹೋಗು-ವೆನು
ಹೋಗು-ವೆ-ನೆಂದು
ಹೋಗು-ವೆ-ನೆಂಬ
ಹೋಗು-ವೆಯೋ
ಹೋಗು-ವೆವೋ
ಹೋಗೆಲೈ
ಹೋಗೋಣ
ಹೋಗೋಣ-ವೆಂದು
ಹೋದ
ಹೋದಂತೆ
ಹೋದಂತೆಲ್ಲಾ
ಹೋದ-ಕೂಡಲೆ
ಹೋದದ್ದ-ರಿಂದ
ಹೋದದ್ದಾಯಿತು
ಹೋದದ್ದೇನು
ಹೋದ-ನಲ್ಲವೆ
ಹೋದನು
ಹೋದ-ನೆಂದು
ಹೋದ-ಮೇಲೆ
ಹೋದರು
ಹೋದರೂ
ಹೋದರೆ
ಹೋದ-ರೆಯೆ
ಹೋದ-ರೆಷ್ಟು
ಹೋದರೊ
ಹೋದಲ್ಲಿ
ಹೋದಳು
ಹೋದ-ವನು
ಹೋದ-ವರಿಗೆ
ಹೋದ-ವರು
ಹೋದ-ವರೆಲ್ಲಾ
ಹೋದವು
ಹೋದಾಗ
ಹೋದಿರಲ್ಲ
ಹೋದಿರಿ
ಹೋದು-ದಕ್ಕೆ
ಹೋದುದು
ಹೋದುವು
ಹೋದೆ
ಹೋದೆನಲ್ಲಾ
ಹೋದೆನು
ಹೋದೆ-ನೆಂದು
ಹೋದೆ-ಯಲ್ಲ
ಹೋದೆ-ಯೇನು
ಹೋದೆವು
ಹೋಮ
ಹೋಮ-ಗಳನ್ನು
ಹೋಮ-ಮಾಡಿ
ಹೋಮಾಗ್ನಿ-ಯನ್ನು
ಹೋಮಾಗ್ನಿ-ಯಲ್ಲಿ
ಹೋಮಾದಿ-ಗಳನ್ನು
ಹೋಯಿತಹ
ಹೋಯಿತು
ಹೋಯಿ-ತೆಂದರೆ
ಹೋರಾಟ
ಹೋರಾಟಕ್ಕೆ
ಹೋರಾಟದ
ಹೋರಾಟ-ವನ್ನು
ಹೋರಾಟ-ವಿರುವು-ದಿಲ್ಲ
ಹೋರಾಟ-ವೆಲ್ಲ
ಹೋರಾಟವೇ
ಹೋರಾಡ-ಬಲ್ಲ
ಹೋರಾಡ-ಬೇಕಾಗಿರ-ಲಿಲ್ಲ
ಹೋರಾಡ-ಬೇಕಾಗಿಲ್ಲ
ಹೋರಾಡ-ಬೇಕಾಗು-ವುದೆಂಬ
ಹೋರಾಡ-ಬೇಕಾ-ಯಿತು
ಹೋರಾಡಲು
ಹೋರಾಡಿ
ಹೋರಾಡಿ-ದರು
ಹೋರಾಡಿದೆ
ಹೋರಾಡಿಯೂ
ಹೋರಾಡಿರು-ವೆವು
ಹೋರಾಡುತ್ತಾ
ಹೋರಾಡು-ತ್ತಿರು-ವನೋ
ಹೋರಾಡು-ತ್ತಿರುವ-ರೆಂದು
ಹೋರಾಡು-ತ್ತಿರು-ವುದು
ಹೋರಾಡು-ತ್ತೀರಿ
ಹೋರಾಡು-ತ್ತೇನೆ
ಹೋರಾಡು-ವರು
ಹೋರಾಡುವು-ದರಿಂದಲ್ಲ
ಹೋರಾಡುವು-ದಿಲ್ಲ
ಹೋರಾಡು-ವುದು
ಹೋರಿ-ಗಳಲ್ಲೊಂದು
ಹೋರ್
ಹೋಲಸು
ಹೋಲಿಕೆ-ಗಳೊಂದಿಗೆ
ಹೋಲಿಕೆ-ಯಿಂದ
ಹೋಲಿಕೆ-ಯಿರುವು-ದರಿಂದ
ಹೋಲಿಕೆಯು
ಹೋಲಿ-ಸಲ-ಸದಳ
ಹೋಲಿಸಿ
ಹೋಲಿಸಿ-ದರೆ
ಹೋಲಿಸು-ತ್ತಿದ್ದರು
ಹೋಲಿಸು-ವನು
ಹೋಲಿಸು-ವುದಕ್ಕೆ
ಹೋಲಿಸು-ವುದು
ಹೋಲುತ್ತವೆ
ಹೋಲು-ವನು
ಹೋಲು-ವುದಿಲ್ಲ
ಹೋಲು-ವುವು
ಹೌದು
ಹೌದೇನು
ಹೌದೋ
ಹೌರಾ
ಹೌರಾಕ್ಕೆ
ಹ್ಯಾಮಿಲ್ಟನ್
ಹ್ರದನೀಲ
ಹ್ರೀಂ
}
