
\chapter{ಅಧ್ಯಾಯ ೧೧: ಹಿಂದಿರುಗುವಾಗ ಶ‍್ರೀನಗರದಲ್ಲಿ}

ವ್ಯಕ್ತಿಗಳು: ಸ್ವಾಮಿ ವಿವೇಕಾನಂದರು, ಗುರುಭಾಯಿಗಳು, ಧೀರಮಾತಾ, ಜಯಾ ಎಂಬ ಹೆಸರಿನವಳು ಮತ್ತು ಸೋದರಿ ನಿವೇದಿತಾಳನ್ನೊಳಗೊಂಡ ಯೂರೋಪಿ ಯನ್ ಶಿಷ್ಯರುಗಳ ಮತ್ತು ಅತಿಥಿಗಳ ಗುಂಪು.

ಸ್ಥಳ: ಕಾಶ್ಮೀರ - ಶ‍್ರೀನಗರ.

ಕಾಲ: ೧೮೯೮ರ ಆಗಸ್ಟ್ ೯ರಿಂದ ಆಗಸ್ಟ್ ೧೩ರ ವರೆಗೆ.

ಆಗಸ್ಟ್ ೯.

ಈ ಸಮಯದಲ್ಲಿ ಗುರುದೇವರು ಯಾವಾಗಲೂ ನಮ್ಮನ್ನು ಬಿಟ್ಟು ಹೊರಟು ಹೋಗುವ ಮಾತನ್ನೇ ಆಡುತ್ತಿದ್ದರು. “ಹರಿಯುವ ನದಿ ಶುದ್ಧ, ಚಲಿಸುವ ಸಂನ್ಯಾಸಿ ಶುದ್ಧ” ಎಂಬ ಮಾತನ್ನು ಕೇಳಿದಾಗಲೆಲ್ಲ ಅದರ ನಿಜವಾದ ಅರ್ಥವೇನು ಎನ್ನುವುದು ನನಗೆ ತಿಳಿಯುತ್ತದೆ. “ತುತ್ತು ಅನ್ನಕ್ಕಾಗಿ ಭಿಕ್ಷೆ ಬೇಡುತ್ತ ಕಷ್ಟಪಡುವ ತಪೋಜೀವನ ಕ್ಕಾಗಿ ನನ್ನ ಜೀವ ಕಾತರಿಸುತ್ತಿದೆ” ಎಂಬ ಆರ್ತತೆಯ ಕೂಗು, ಸ್ವಾತಂತ್ರ್ಯಕ್ಕಾಗಿ, ಜನಸಾಮಾನ್ಯರ ಸಂಸ್ಪರ್ಶಕ್ಕಾಗಿ ಹಂಬಲ, ತಾವು ದೇಶಾದ್ಯಂತ ಕಾಲ್ನಡಿಗೆಯಲ್ಲಿ ಚಲಿಸುತ್ತ ನಡೆದು, ಬಂದಲ್ಲಿಗೆ ಹಿಂದಿರುಗುವುದಕ್ಕಾಗಿ ಮತ್ತೆ ನಮ್ಮನ್ನು ಬಾರಾಮುಲ್ಲಾದಲ್ಲಿ ಬಂದು ಕೂಡಿಕೊಳ್ಳುವ ಕಾಲ್ಪನಿಕ ಚಿತ್ರ.

ಎರಡು ಋತುಗಳಷ್ಟು ದೀಘಕಾಲ ಅವರಿಗೆ ಅತ್ಯಂತ ಆಪ್ತಮಿತ್ರರಾಗಿದ್ದ ದೋಣಿ ಕುಟುಂಬದವರು ಇಂದು ನಮ್ಮನ್ನು ಬೀಳ್ಕೊಂಡು ಹೊರಟುಹೋದರು. ಮುಂದೆಂದೋ ಸ್ವಾಮಿಗಳು ಅವರೊಂದಿಗಿನ ಆಪ್ತತೆಯ ನೆನಪುಗಳನ್ನು ಸ್ಮರಿಸಿಕೊಂಡು, ಔದಾರ್ಯ ತಾಳ್ಮೆಗಳೂ ಸಹ ಅದೆಷ್ಟು ದೂರ ಹೋಗಬಲ್ಲವು ಎಂಬುದಕ್ಕೆ ಇದೊಂದು ಸಾಕ್ಷಿ ಎಂಬಂತೆ ಉಲ್ಲೇಖಿಸುವರು.

\textbf{ಆಗಸ್ಟ್ ೧೦.}

ಸಂಜೆಹೊತ್ತು ನಾವೆಲ್ಲರೂ ಎಲ್ಲಿಗೋ ಭೇಟಿ ಕೊಡುವುದಕ್ಕೆಂದು ಹೋಗಿದ್ದೆವು. ಹಿಂದಿರುಗಿ ಬರುತ್ತಿರುವಾಗ, ಅವರು ತಮ್ಮ ಶಿಷ್ಯೆ ನಿವೇದಿತಾಳನ್ನು ತಮ್ಮ ಜೊತೆಗೆ ಹೊಲ ಗಳಲ್ಲಿ ನಡೆದು ಬರುವಂತೆ ಕರೆದರು. ಅವರ ಮಾತೆಲ್ಲವೂ ಮಾಡಬೇಕಾಗಿರುವ ಕಾರ್ಯ ಹಾಗೂ ಅದರಲ್ಲಿನ ಉದ್ದೇಶ ಇವುಗಳನ್ನು ಕುರಿತಾಗಿದ್ದಿತು. ಇಡಿಯ ರಾಷ್ಟ್ರವನ್ನು ಹಾಗೂ ಅದರಲ್ಲಿನ ಧರ್ಮಗಳನ್ನು ಒಳಗೊಳ್ಳುವಿಕೆಯ ತಮ್ಮ ಕಲ್ಪನೆಯನ್ನು ಕುರಿತು ಮಾತನಾಡಿದರು; ತಮ್ಮ ವೈಶಿಷ್ಟ್ಯವು ಪೂರ್ಣವಾಗಿ ಅಡಗಿರುವುದು ಹಿಂದೂ ಧರ್ಮವನ್ನು ಕ್ರಿಯಾಶೀಲವನ್ನಾಗಿಸುವ, ಆತ್ಮವಿಶ್ವಾಸದ ಕೆಚ್ಚಿನದನ್ನಾಗಿಸುವ, ಒಂದು ಶ್ರದ್ಧೆಯ ಪ್ರಚಾರವನ್ನಾಗಿಸುವ ತಮ್ಮ ಅಭೀಪ್ಸೆಯಲ್ಲಿಯೇ ಎಂದರು; ತಾವು ಒಪ್ಪದಿರು ವುದು ಹಾಗೂ ನಿರಾಕರಿಸುವುದು ಕೇವಲ ಅಸ್ಪೃಶ್ಯತೆಯನ್ನು ಮಾತ್ರ ಎಂದರು. ಅನಂತರ, ತಮ್ಮ ಎದೆಯಾಳದಿಂದ, ಭಾವುಕರಾಗಿ, ಅತ್ಯಂತ ಸಾಂಪ್ರದಾಯಿಕರಾಗಿರುವ ಅನೇಕ ರಲ್ಲಿ ಬೃಹತ್ತಾಗಿ ಹುದುಗಿರುವ ಆಧ್ಯಾತ್ಮಿಕತೆಯನ್ನು ಕುರಿತು ಮಾತನಾಡಿದರು. ಭಾರತಕ್ಕೆ ವಾಸ್ತವಿಕವಾದ ಕ್ರಿಯಾಶೀಲತೆಯೇನೋ ಬೇಕು, ಆದರೆ ಅದಕ್ಕಾಗಿ ಅದು ತನ್ನ ಪ್ರಾಚೀನವಾದ ಧ್ಯಾನಜೀವನವನ್ನು ಬಿಡತಕ್ಕದ್ದಲ್ಲ ಎಂದು ನುಡಿದರು. “ಸಾಗರ ದಷ್ಟು ಆಳ, ಆಕಾಶದಷ್ಟು ವಿಶಾಲ” ಆಗಿರುವುದೇ ಶ‍್ರೀರಾಮಕೃಷ್ಣರು ಬೋಧಿಸಿದ ಆದರ್ಶ. ಆದರೆ ಈ ಸಾಂಪ್ರದಾಯಿಕತೆಯಲ್ಲಿ ಅಡಗಿರುವ ಆತ್ಮದಲ್ಲಿನ ಅದ್ಭುತ ಅಂತ ರಂಗದ ಜೀವನವು ಒಂದು ಆಕಸ್ಮಿಕವಾದ, ಆದರೆ ಅತ್ಯಗತ್ಯವಲ್ಲದ ಒಡನಾಟದ ಫಲ. “ಇಲ್ಲಿ ನಮ್ಮನ್ನು ನಾವು ಸರಿಪಡಿಸಿಕೊಂಡದ್ದೇ ಆದರೆ, ಲೋಕವು ನಮ್ಮ ಪಾಲಿಗೆ ಸರಿಯಾಗುತ್ತದೆ-ಏಕೆಂದರೆ ನಾವೆಲ್ಲರೂ ಒಂದೇ ಅಲ್ಲವೆ? ಶ‍್ರೀರಾಮಕೃಷ್ಣ ಪರಮ ಹಂಸರು ತಮ್ಮ ಅಸ್ತಿತ್ವದ ಆಳದುದ್ದಕ್ಕೂ ಜೀವಂತವಾಗಿದ್ದರು; ಆದರೂ ಬಾಹ್ಯದಲ್ಲಿಯೂ ಸಹ ಪರಿಪೂರ್ಣ ಕ್ರಿಯಾಶೀಲರೂ ಸಮರ್ಥರೂ ಆಗಿದ್ದರು”.

ಇನ್ನು ತಮ್ಮ ಗುರುಗಳ ಆರಾಧನೆಯ ಆ ಆಯಕಟ್ಟಿನ ಪ್ರಶ್ನೆಯ ಬಗ್ಗೆ, “ನನ್ನ ಜೀವನವೇನೋ ಆ ಮಹಾವ್ಯಕ್ತಿತ್ವದ ಸ್ಫೂರ್ತಿಯಿಂದ ನಿರ್ದೇಶಿತವಾಗಿ ನಡೆಯುತ್ತಿ ರುವುದು; ಆದರೆ ಇತರರು ಇದು ತಮ್ಮ ಮಟ್ಟಿಗೆ ಅದೆಷ್ಟು ನಿಜವಾಗಿರುವುದೆಂಬು ದನ್ನು ತಾವೇ ನಿರ್ಧರಿಸಿಕೊಳ್ಳುವರು. ಸ್ಫೂರ್ತಿ ಎಂಬುದು ಒಬ್ಬ ಮನುಷ್ಯನ ಮೂಲಕ ವಾಗಿ ಮಾತ್ರವೇ ಲೋಕಕ್ಕೆ ಬರುವಂಥದೇನೂ ಅಲ್ಲ”.

\textbf{ಆಗಸ್ಟ್ ೧೧.}

ಈ ದಿವಸ ಸ್ವಾಮಿಗಳು ನಮ್ಮ ಗುಂಪಿನ ಸದಸ್ಯರೊಬ್ಬರಿಗೆ ಹಸ್ತಸಾಮುದ್ರಿಕೆಯನ್ನು ಅನುಷ್ಠಾನಮಾಡಿದುದಕ್ಕಾಗಿ ಬೈದ ಸಂದರ್ಭ ಒದಗಿ ಬಂತು. ಅದು ಪ್ರತಿಯೊಬ್ಬರಿಗೂ ಆಪ್ಯಾಯಮಾನವಾದದ್ದೇನೋ ಸರಿ. ಆದರೂ ಇಡೀ ಭಾರತವೇ ದ್ವೇಷಿಸುವಂಥದು, ಹೀಗಳೆಯುವಂಥದು ಎಂದರು. ಎದುರಾದ ಒಂದಿನಿತು ಸಮರ್ಥನೆಗೆ ಉತ್ತರವಾಗಿ, ಇನ್ನೊಬ್ಬರ ಮನಸ್ಸನ್ನು ತಿಳಿದುಕೊಳ್ಳಲೆತ್ನಿಸುವುದನ್ನೂ ಸಹ ತಾವು ಒಪ್ಪುವುದಿಲ್ಲ ಎಂದರು. “ನಿಜವಾಗಿ ಹೇಳಬೇಕೆಂದರೆ, ಅವರೂ ಅವರ ಶಿಷ್ಯರೂ ಯಾವ ಪವಾಡ ವನ್ನೂ ಮಾಡದೇ ಇದ್ದಿದ್ದರೆ, ನಿಮ್ಮಗಳ ಅವತರಣವನ್ನೂ ಸಹ ನಾನು ಹೆಚ್ಚು ಪ್ರಾಮಾಣಿಕವೆನ್ನುತ್ತಿದ್ದೆ. ಇದನ್ನು ಮಾಡಿದ್ದಕ್ಕಾಗಿ ಬುದ್ಧನು ಸಂನ್ಯಾಸಿಯೊಬ್ಬನನ್ನು ಒದ್ದೋಡಿಸಿದ.” ಆ ನಂತರ ತಮ್ಮ ಗಮನವನ್ನು ವಿಷಯಾಂತರ ಮಾಡಿ ಮೇಲೆ, ಅಂಥದನ್ನು ಅದೆಷ್ಟು ಅಲ್ಪಪ್ರಮಾಣದಲ್ಲಿ ಪ್ರದರ್ಶಿಸಿದರೂ ಅದು ಖಂಡಿತ ವಾಗಿಯೂ ಭಯಾನಕನಿಸುವಂತಹ ಪ್ರತಿಕ್ರಿಯೆಯನ್ನು ತರುತ್ತದೆ ಎಂದರು.

\textbf{ಆಗಸ್ಟ್ ೧೨ ಮತ್ತು ೧೩.
 }

ಸ್ವಾಮಿಗಳು ಈಗ ಬ್ರಾಹ್ಮಣ ಅಡಿಗೆಯವರೊಬ್ಬರನ್ನು ನೇಮಿಸಿಕೊಂಡಿದ್ದರು. ತಮಗಾಗಿ ಮುಸ್ಲಿಮನೊಬ್ಬನು ಅಡುಗೆ ಮಾಡಿದರೂ ಆದೀತು ಎಂಬ ಅವರ ನಿಲುವಿಗೆ ವಿರೋಧವಾಗಿ ಅಮರನಾಥದ ಸಾಧುಗಳು ಮನಮುಟ್ಟವ ಹಾಗೆ ವಾದ ಮಾಡಿದ್ದರು. “ಕನಿಷ್ಠಪಕ್ಷ ಸಿಖ್ಖರ ದೇಶದಲ್ಲಿಯಾದರೂ ಹಾಗೆ ಮಾಡಬೇಡಿ, ಸ್ವಾಮೀಜಿ” ಎಂದಿದ್ದ ಅವರ ಮಾತಿಗೆ ಸ್ವಾಮೀಜಿ ಕೊನೆಗೆ ಒಪ್ಪಿದ್ದರು. ಆದರೀಗ ಅವರು ತಮ್ಮ ಪುಟ್ಟ ಮುಸ್ಲಿಂ ದೋಣಿ-ಮಗುವನ್ನು ಉಮಾ ಎಂದು ಕರೆದು ಆರಾಧಿಸುತ್ತಿದ್ದರು. ಆ ಮಗುವಿನ ಪ್ರೀತಿ ಎಂದರೆ ಸೇವೆಮಾಡುವುದೇ ಆಗಿತ್ತು; ಸ್ವಾಮಿಗಳು ಕಾಶ್ಮೀರವನ್ನು ಬಿಟ್ಟ ದಿನ, ಆ ಪುಟ್ಟ ಮಗು, ಸಂತೋಷದಿಂದ ನಲಿಯುತ್ತ, ತನ್ನ ಪುಟ್ಟ ಕೈಗಳಿಂದ ಸೇಬಿನ ಬುಟ್ಟಿಯೊಂದನ್ನು ಟಾಂಗಾದವರೆಗೆ ತಾನೇ ತಂದು ಕೊಟ್ಟಿತ್ತು. ಆಗ ಸ್ವಲ್ಪ ಉದಾಸೀನರಾಗಿದ್ದಂತೆ ತೋರಿ ದರೂ, ಸ್ವಾಮಿಗಳು ಆ ಮಗುವನ್ನೆಂದಿಗೂ ಮರೆಯಲಿಲ್ಲ. ಕಾಶ್ಮೀರದಲ್ಲಿದ್ದಾಗಲೂ, ದೋಣಿಯನ್ನೆಳೆದೊಯ್ಯುವ ಹಾದಿಯಲ್ಲಿ ಆ ಮಗು ಒಂದು ನೀಲಿ ಹೂವನ್ನು ಕಂಡುದು, ಅದನ್ನೇ ನೋಡುತ್ತ ಆದರ ಬಳಿಯಲ್ಲಿಯೇ ಕುಳಿತು ಹೀಗೊಮ್ಮೆ ಹಾಗೊಮ್ಮೆ ಅದನ್ನು ಆಡಿಸುತ್ತ “ಒಬ್ಬಳೇ ಆ ಹೂವಿನ ಜೊತೆಗೆ ಇಪ್ಪತ್ತು ನಿಮಿಷಗಳ ಕಾಲ ಕುಳಿತಿದ್ದಳು” ಎಂದಿತ್ಯಾದಿಯಾಗಿ ಅದನ್ನು ಜ್ಞಾಪಿಸಿಕೊಳ್ಳುವುದಕ್ಕೆ ಬಹಳ ಆನಂದಪಡುತ್ತಿದ್ದರು.

ನದೀತೀರದಲ್ಲಿ ಮೂರು ಚೆನ್ನಾರ್ ವೃಕ್ಷಗಳು ಬೆಳೆದಿದ್ದ ಒಂದು ತುಂಡು ಭೂಮಿಯ ಕಡೆಗೆ ಈ ಸಲ ನಮ್ಮ ಯೋಚನೆಗಳು ವಿಚಿತ್ರವಾದ ಪ್ರೀತಿಯಿಂದ ಹರಿದುವು. ಏಕೆಂದರೆ ಕಾಶ್ಮೀರದ ಮಹಾರಾಜರು ಅದನ್ನು ಸ್ವಾಮೀಜಿಯವರಿಗೆ ಕೊಡಲು ಉತ್ಸುಕರಾಗಿ ದ್ದರು. ನಾವೆಲ್ಲ ಅದನ್ನು ನಮ್ಮ ಭವಿಷ್ಯದ ಕಾರ್ಯಕ್ಷೇತ್ರವನ್ನಾಗಿ ಊಹಿಸಿ ಕೊಂಡೆವು- “ಜನಗಳಿಂದ, ಜನಗಳಿಗಾಗಿ, ದುಡಿಯುವವನ ಹಾಗೂ ಸೇವಿತರ ಸಂತಸ ಕ್ಕಾಗಿ” ದುಡಿಯುವ ಮಹತ್ಕಲ್ಪನೆಯ ಸಾಕ್ಷಾತ್ಕಾರಕ್ಕಾಗಿ.

ಗೃಹಿಣೀ ಗೃಹಮುಚ್ಯತೇ ಎಂಬ ಭಾರತೀಯ ಭಾವಕ್ಕನುಗುಣವಾಗಿ, ನಾವುಗಳು ಆ ಕ್ಷೇತ್ರಕ್ಕೆ ಹೋಗಿ ಸ್ವಲ್ಪಕಾಲ ಅಲ್ಲಿ ಬೀಡುಬಿಡಬೇಕೆಂಬ ಸಲಹೆ ಬಂತು. ಅಲ್ಲದೆ ನಮ್ಮ ಲ್ಲೊಬ್ಬರಿಗೆ ಈಗ ಸ್ವಲ್ಪಕಾಲ ವಿಶೇಷವಾಗಿ ಮೌನವಾಚರಿಸಬೇಕೆಂಬ ಅಪೇಕ್ಷೆಯೂ ಮೊಳೆದಿತ್ತು. ಆದ್ದರಿಂದ, ಮಹಾರಾಜರು ಆ ಭೂಮಿಯನ್ನು ಕೈಗೆ ತೆಗೆದುಕೊಂಡು ಸ್ವಾಮಿಗಳಿಗೆ ಕೊಡುವ ಮುನ್ನವೇ ನಾವು ಅಲ್ಲಿಗೆ ಹೋಗಿ “ಮಹಿಳೆಯರ ಮಠ” ವೊಂದನ್ನು ಸ್ಥಾಪಿಸಿಬಿಡಬೇಕೆಂದೇ ನಿರ್ಧರಿಸಿಬಿಟ್ಟೆವು. ಅದು ಯೂರೋಪಿ ಯನ್ನರು ತಾತ್ಕಾಲಿಕವಾಗಿ ಬೀಡುಬಿಡಲು ಆಗಿಂದಾಗ್ಗೆ ಬಳಸುತ್ತಿದ್ದ ಜಾಗಗಳಲ್ಲಿ ಒಂದಾಗಿದ್ದುದರಿಂದ ಹಾಗೆ ಮಾಡುವುದು ಸಾಧ್ಯವಾಯಿತು.

