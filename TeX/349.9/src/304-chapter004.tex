
\chapter{ಬ್ರಹ್ಮಾಂಡ ಮತ್ತು ಪಿಂಡಾಂಡ}

(ಆಲ್ಮೋರದ ಹತ್ತಿರ ಕಾಂಕಡಿ ಘಾಟ್ನಲ್ಲಿ ಧ್ಯಾನದಲ್ಲಿ ಮಗ್ನರಾಗಿರುವಾಗ ಸ್ವಾಮಿ ವಿವೇಕಾ ನಂದರು ಬ್ರಹ್ಮಾಂಡ ಮತ್ತು ಪಿಂಡಾಂಡಗಳ ಅನುಭವವನ್ನು ಪಡೆದರು. ಅದನ್ನು ಬಂಗಾಳಿ ಯಲ್ಲಿ ಅವರು ಬರೆದುಕೊಂಡರು. ಆದಿಯಲ್ಲಿ ಶಬ್ದವಿದ್ದಿತು ಇತ್ಯಾದಿ.)

ಬ್ರಹ್ಮಾಂಡ ಮತ್ತು ಪಿಂಡಾಂಡಗಳು ಒಂದೇ ರೀತಿಯಲ್ಲಿ ನಿರ್ಮಿತವಾಗಿವೆ. ಜೀವಾತ್ಮವು ಜೀವಂತ ದೇಹದಲ್ಲಿ ಹುದುಗಿರುವಂತೆ ವಿಶ್ವಾತ್ಮವು ಜೀವಂತ ಪ್ರಕೃತಿಯಲ್ಲಿ ಬಾಹ್ಯ ಜಗತ್ತಿನಲ್ಲಿ ಅಂತರ್ಹಿತವಾಗಿರುವುದು. ಕಾಳಿಯು ಶಿವನನ್ನು ಆಲಿಂಗಿಸಿರುವಳು: ಇದು ಊಹೆಯಲ್ಲ. ಹೀಗೆ ಆತ್ಮವು ಪ್ರಕೃತಿಯಿಂದ ಆವೃತವಾಗಿರುವುದು ಶಬ್ದ ಮತ್ತು ಅದು ವ್ಯಕ್ತಪಡಿಸುವ ಭಾವನೆಗೆ ಸಮಾನವಾಗಿದೆ. ಅವೆರೆಡೂ ಒಂದೇ. ಮಾನಸಿಕ ಕಲ್ಪನೆಯ ಮೂಲಕ ಮಾತ್ರ ಅವುಗಳನ್ನು ಬೇರ್ಪಡಿಸಬಹುದು. ಶಬ್ದವಿಲ್ಲದೆ ಆಲೋಚನೆ ಅಸಾಧ್ಯ. ಆದ್ದರಿಂದ ಆದಿಯಲ್ಲಿ ಶಬ್ದವಿತ್ತು ಇತ್ಯಾದಿ.

ವಿಶ್ವಾತ್ಮದ ಈ ದ್ವೈತ ರೂಪಗಳು ಶಾಶ್ವತವಾದವುಗಳು. ಆದ್ದರಿಂದ ನಾವು ಏನನ್ನು ಅನುಭವಿಸುತ್ತೇವೊ ಅದು ನಿತ್ಯರೂಪ ಮತ್ತು ನಿತ್ಯರೂಪ ಇವುಗಳ ಸಂಯೋಗ.

