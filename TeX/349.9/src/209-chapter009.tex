
\chapter{ಬೇರೆ ಬೇರೆ ಉಪನ್ಯಾಸಗಳಿಂದ ಸಂಗ್ರಹಿಸಿದ ಹೇಳಿಕೆಗಳು}

(ಫ್ಯಾಂಕ್ ರೊಡ್ಹಾಮಲ್ರಿಂದ ಸಂಗ್ರಹವಾದುವುಗಳು. ಇವು ಬಹುಶಃ ಭಗವದ್ಗೀತೆಯ ಮೇಲಿನ ಎರಡನೆಯ ಉಪನ್ಯಾಸಕ್ಕೆ ಸಂಬಂಧಿಸಿದುವು.)

“ಭೌತಿಕತೆ ಹೊಗದೆ ಆಧ್ಯಾತ್ಮಿಕತೆಯನ್ನು ಪಡೆಯಲಾಗುವುದಿಲ್ಲ.”

ಗೀತೆಯ ಮೊದಲನೆಯ ಅಧ್ಯಾಯವನ್ನು ಒಂದು ರೂಪಕವಾಗಿ ತೆಗೆದುಕೊಳ್ಳ ಬಹುದು. “ವೇದಗಳು ಪ್ರಕೃತಿಯ ವಿಷಯವನ್ನೇ ಬೋಧಿಸುತ್ತವೆ.”

ನಾವು ಭಾವೋದ್ವೇಗವು ಕರ್ತವ್ಯದ ಸ್ಥಾನವನ್ನು ಆಕ್ರಮಿಸಲು ಅವಕಾಶಕೊಡು ತ್ತೇವೆ ಮತ್ತು ನಿಜವಾದ ಪ್ರೀತಿಯಿಂದ ವರ್ತಿಸುತ್ತಿರುವೆವೆಂದು ಹೆಮ್ಮೆಪಡುತ್ತೇವೆ.

ನಾವು ತ್ಯಾಗಮಾಡಬೇಕಾದರೆ ಭಾವೋದ್ವೇಗದಿಂದ ಮುಕ್ತರಾಗಬೇಕು. ಭಾವೋದ್ವೇಗವು ಪ್ರಾಣಿಸಹಜವಾದುದು. ಪ್ರಾಣಿಗಳು ಸಂಪೂರ್ಣವಾಗಿ ಭಾವೋದ್ವೇಗದ ಸೃಷ್ಟಿ.

ತನ್ನ ಮಕ್ಕಳಿಗಾಗಿ ತ್ಯಾಗಮಾಡುವುದು ಅತ್ಯುನ್ನತ ಆದರ್ಶವಲ್ಲ. ಪ್ರಾಣಿಗಳು ಅದನ್ನು ಮಾಡುತ್ತವೆ ಹಾಗೆಯೇ ಯಾವುದೇ ಸಾಮಾನ್ಯ ಮಹಿಳೆಯೂ ಮಾಡುತ್ತಾಳೆ. ಹಾಗೆ ಮಾಡುವುದು ಪ್ರೀತಿಯ ಚಿಹ್ನೆಯಲ್ಲ - ಅದು ಅಂಧ ಭಾವವಷ್ಟೆ.

ನಾವು ನಮ್ಮ ದೌರ್ಬಲ್ಯವು ಶಕ್ತಿಯಂತೆಯೂ ನಮ್ಮ ಭಾವೋದ್ವೇಗವು ಪ್ರೀತಿ ಯಂತೆಯೂ, ನಮ್ಮ ಹೇಡಿತನವನ್ನು ಧೈರ್ಯದಂತೆಯೂ ಕಾಣುವಂತೆ ಮಾಡಲು ಯಾವಾಗಲೂ ಪ್ರಯತ್ನಿಸುತ್ತೇವೆ.

ದುರಹಂಕಾರ, ದುರ್ಬಲತೆ ಇತ್ಯಾದಿಗಳ ವಿಷಯದಲ್ಲಿ ಇವು ನನಗೆ ಯೋಗ್ಯ ವಾದುವಲ್ಲ ಎಂದು ನಿಮಗೆ ನೀವೆ ಹೇಳಿಕೊಳ್ಳಿ.

