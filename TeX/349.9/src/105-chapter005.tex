
\chapter{ಆರ್ಯಜನಾಂಗದ ಇತಿಹಾಸ}

(೧೮೯೬, ಮೇ ೭ರಂದು ಲಂಡನ್ನಿನಲ್ಲಿ ನೀಡಿದ ಉಪನ್ಯಾಸ. ಶ‍್ರೀ ಜೆ.ಜೆ. ಗುಡ್ವಿನ್ ಅವರು ಬರೆದುಕೊಂಡುದು.)

ಯೋಗದ ನಾಲ್ಕು ವಿಭಾಗಗಳನ್ನು ನಾನು ನಿಮಗೆ ಹೇಳಿರುವೆನು. ಈ ನಾಲ್ಕು ಯೋಗಗಳ ಗುರಿಯೂ ಒಂದೇ. ತಾತ್ತ್ವಿಕ ಭಾಗವಾದ ಜ್ಞಾನ ಯೋಗವನ್ನು ಕುರಿತು ಮೊದಲು ಹೇಳುವುದು ಒಳ್ಳೆಯದು. ವೇದಾಂತ ತತ್ತ್ವದ ಬಗೆಗೆ ಹೇಳುವುದಕ್ಕೆ ಮುಂಚೆ ಈ ತತ್ತ್ವವು ಪ್ರಾರಂಭವಾಗಿ ಕಾಲಾನುಕ್ರಮವಾಗಿ ಬೆಳೆದುಬಂದ ರೀತಿಯನ್ನು ಕುರಿತು ಹೇಳುತ್ತೇನೆ. ನಿಮ್ಮಲ್ಲಿ ಅನೇಕರು ‘ಆರ್ಯ’ ಎಂಬ ಶಬ್ದವನ್ನು ಕೇಳಿರುವಿರಿ ಮತ್ತು ಈ ಶಬ್ದ ದ ಮೇಲೆ ಎಷ್ಟೊ ವಿಷಯಗಳು ಬರೆಯಲ್ಪಟ್ಟಿವೆ.

ಸುಮಾರು ನೂರು ವರ್ಷಗಳ ಹಿಂದೆ ಬಂಗಾಳದಲ್ಲಿ ಸರ್ ವಿಲಿಯಂ ಜೋನ್ಸ್ ಎಂಬ ಆಂಗ್ಲ ನ್ಯಾಯಾಧೀಶನಿದ್ದ. ನಿಮಗೆ ತಿಳಿದಿರುವಂತೆ ಭಾರತದಲ್ಲಿ ಮಹಮ್ಮದೀ ಯರು ಮತ್ತು ಹಿಂದೂಗಳು ಎಂಬ ಎರಡು ವರ್ಗದ ಜನರಿದ್ದಾರೆ. ಹಿಂದೂಗಳು ಮೂಲ ಜನರು, ಮಹಮ್ಮದೀಯರು ಅನಂತರ ಬಂದು ಭಾರತವನ್ನು ಆಕ್ರಮಿಸಿ ಏಳು ನೂರು ವರ್ಷಗಳು ಆಳಿದರು. ಭಾರತದ ಮೇಲೆ ಅನೇಕ ಆಕ್ರಮಣಗಳು ನಡೆದವು. ಹೊಸ ಆಕ್ರಮಣ ನಡೆದಾಗಲೆಲ್ಲ ದೇಶದ ದಂಡ ಶಾಸನ ಬದಲಾಗುತ್ತಿತ್ತು. ದಂಡ ಶಾಸನವು ಯಾವಾಗಲೂ ಆಕ್ರಮಣಕಾರಿಗಳ ಕೈಯಲ್ಲಿರುತ್ತದೆ. ಆದರೆ ಪೌರ ಶಾಸನ ಮಾತ್ರ ಹಾಗೇ ಉಳಿಯುತ್ತದೆ. ಆಂಗ್ಲರು ಭಾರತವನ್ನು ಆಕ್ರಮಿಸಿದ ಮೇಲೆ ಅವರು ದಂಡಶಾಸನವನ್ನು ಬದಲಾಯಿಸಿದರು, ಆದರೆ ಪೌರಶಾಸನವು ಹಾಗೆಯೇ ಉಳಿ ಯಿತು. ನ್ಯಾಯಾಧೀಶರು ಆಂಗ್ಲರಾಗಿದ್ದುದರಿಂದ ಪೌರಶಾಸನವು ಇರುವ ದೇಶ ಭಾಷೆಯು ಅವರಿಗೆ ತಿಳಿದಿರಲಿಲ್ಲ. ಆದ್ದರಿಂದ ಅವರು ಅದನ್ನು ವಿವರಿಸುವುದಕ್ಕೆ ಭಾರತೀಯ ವಕೀಲರ ಸಹಾಯವನ್ನು ಪಡೆಯಬೇಕಾಗಿತ್ತು. ಭಾರತೀಯ ಶಾಸನದ ಬಗ್ಗೆ ಯಾವುದೇ ಪ್ರಶ್ನೆ ಉಂಟಾದರೂ ಈ ಶಾಸನವೇತ್ತರನ್ನು ವಿಚಾರಿಸಬೇಕಾಗಿತ್ತು.

ಸರ್ ವಿಲಿಯಂ ಜೋನ್ಸ್ ತುಂಬ ಪಳಗಿದ ವಿದ್ವಾಂಸನಾಗಿದ್ದ. ಶಾಸನದ ವಿವರಣೆಗೆ ಇತರರನ್ನು ಅವಲಂಬಿಸುವುದರ ಬದಲು ತಾನೇ ಆ ಶಾಸನವಿರುವ ಮೂಲ ಭಾಷೆಯನ್ನು ಅಧ್ಯಯನ ಮಾಡಬೇಕೆಂದು ಬಯಸಿದ. ಏಕೆಂದರೆ ಈ ಶಾಸನವೇತ್ತರು ಲಂಚದ ಆಸೆಯಿಂದ ತಪ್ಪು ವಿವರಣೆ ನೀಡುವ ಸಾಧ್ಯತೆಯಿತ್ತು. ಆದ್ದರಿಂದ ಅವನು ತಾನೇ ಸ್ವತಃ ಹಿಂದೂ ಶಾಸನದ ಅಧ್ಯಯನ ಪ್ರಾರಂಭಿಸಿದಾಗ, ಅದನ್ನು ಆಂಗ್ಲಭಾಷೆಗೆ ಅನುವಾದಿ ಸುವುದು ತುಂಬ ಕಠಿನವೆಂದು ಅವನಿಗೆ ತಿಳಿದುಬಂತು. ಮೊದಲು ಅದನ್ನು ಲ್ಯಾಟಿನ್ನಿಗೆ ಅನುವಾದಿಸಿ ಅದರಿಂದ ಇಂಗ್ಲಿಷಿಗೆ ಅನುವಾದಿಸುವುದು ಸುಲಭವೆಂದು ಅವನು ಕಂಡು ಹಿಡಿದನು. ಆಗ ಅವನಿಗೆ ತಿಳಿದು ಬಂತು, ಬಹುತೇಕ ಸಂಸ್ಕೃತ ಶಬ್ದಗಳು ಲ್ಯಾಟಿನ್ ಪದಗಳಂತೆಯೇ ಇವೆ ಎಂದು. ಅವನೇ ಸಂಸ್ಕೃತ ಭಾಷಾಧ್ಯಯನವನ್ನು ಯೂರೋಪಿಯ ನರಿಗೆ ಪರಿಚಯಿಸಿದವನು. ಆಗ ಜರ್ಮನರು ಮತ್ತು ಫ್ರೆಂಚರು ಪಾಂಡಿತ್ಯದಲ್ಲಿ ಹೆಚ್ಚು ಮುಂದೆ ಬರುತ್ತಿದ್ದುದರಿಂದ, ಅವರು ಈ ಭಾಷೆಯನ್ನು ಮೊದಲು ಅಧ್ಯಯನ ಮಾಡಲು ಪ್ರಾರಂಭಿಸಿದರು.

ತಮ್ಮ ಅದ್ಭುತ ವಿಶ್ಲೇಷಣಾ ಶಕ್ತಿಯ ಮೂಲಕ ಜರ್ಮನರು, ಸಂಸ್ಕೃತ ಮತ್ತು ಎಲ್ಲ ಐರೋಪ್ಯ ಭಾಷೆಗಳ ನಡುವೆ ಸಮಾನತೆಯಿದೆಯೆಂಬುದನ್ನು ಕಂಡುಹಿಡಿದರು. ಪುರಾತನ ಭಾಷೆಗಳಲ್ಲಿ ಗ್ರೀಕ್ ಭಾಷೆಯು ಸಂಸ್ಕೃತಕ್ಕೆ ಹೆಚ್ಚು ಸಮೀಪದಲ್ಲಿದೆ. ಮುಂದೆ ಲಿಥುಯೇನಿಯನ್ ಎಂಬ ಭಾಷೆ ಇತ್ತು ಎಂಬುದು ತಿಳಿದುಬಂತು. ಇದು ಬಾಲ್ಟಿಕ್ ಸಮುದ್ರತೀರದಲ್ಲಿ ಪ್ರಚಲಿತವಿತ್ತು. ಲಿಥುಯೇನಿಯನ್ ಭಾಷೆಯು ಸಂಸ್ಕೃತಕ್ಕೆ ಅತ್ಯಂತ ಸಮೀಪವರ್ತಿಯಾಗಿದೆ. ಕೆಲವು ಲಿಥುಯೇನಿಯನ್ ವಾಕ್ಯಗಳು ಉತ್ತರಭಾರತದ ಭಾಷೆಗಳಿಗಿಂತಲೂ ಕಡಿಮೆ ಸಂಸ್ಕೃತದಿಂದ ಬೇರೆಯಾಗಿವೆ. ಐರೋಪ್ಯ ಭಾಷೆಗಳು ಮತ್ತು ಏಶಿಯಾದ ಎರಡು ಭಾಷೆಯಗಳಾದ ಸಂಸ್ಕೃತ ಹಾಗೂ ಪಾರ್ಸಿ ಭಾಷೆಗಳು ಇವುಗಳ ನಡುವೆ ನಿಕಟ ಸಂಪರ್ಕವಿದೆಯೆಂಬುದು ಬೆಳಕಿಗೆ ಬಂತು. ಹೇಗೆ ಈ ಸಂಪರ್ಕ ಉಂಟಾಯಿತು ಎಂಬುದನ್ನು ವಿವರಿಸಲು ಅನೇಕ ಸಿದ್ಧಾಂತಗಳು ಹುಟ್ಟಿ ಕೊಂಡವು. ಪ್ರತಿದಿನವೂ ಹೊಸ ಹೊಸ ಸಿದ್ಧಾಂತಗಳು ಹುಟ್ಟಿಕೊಳ್ಳುತ್ತಿವೆ, ಪ್ರತಿದಿನವೂ ನುಚ್ಚುನೂರಾಗಿ ಹೋಗುತ್ತಿವೆ. ಇದು ಎಲ್ಲಿ ನಿಲ್ಲುವುದೊ ತಿಳಿಯದು.

ತಮ್ಮನ್ನು ಆರ್ಯರು ಎಂದು ಕರೆದುಕೊಳ್ಳುತ್ತಿದ್ದ ಒಂದು ಜನಾಂಗವು ಪುರಾತನ ಕಾಲದಲ್ಲಿ ಇತ್ತು ಎಂಬ ಸಿದ್ಧಾಂತವು ಅನಂತರ ಹುಟ್ಟಿಕೊಂಡಿತು. ತಮ್ಮನ್ನು ಆರ್ಯರು ಎಂದು ಕರೆದುಕೊಂಡು ಸಂಸ್ಕೃತ ಭಾಷೆಯನ್ನು ಮಾತನಾಡುತ್ತಿದ್ದ ಜನರಿದ್ದರು ಎಂಬ ವಿಷಯವು ಅವರಿಗೆ ಸಂಸ್ಕೃತ ಸಾಹಿತ್ಯದಲ್ಲಿ ದೊರೆಯಿತು. ಪಾರ್ಸಿ ಭಾಷೆಯಲ್ಲಿಯೂ ಇದರ ಪ್ರಸ್ತಾಪವಿದೆ. ಇದರ ಆಧಾರದ ಮೇಲೆ ಸಂಸ್ಕೃತ ಭಾಷೆಯನ್ನು ಮಾತನಾಡು ತ್ತಿದ್ದ ಆರ್ಯರು ಎಂಬ ಜನಾಂಗದವರು ಮಧ್ಯ ಏಷ್ಯಾದಲ್ಲಿ ವಾಸಿಸುತ್ತಿದ್ದರು ಎಂಬ ಸಿದ್ಧಾಂತವನ್ನು ರೂಪಿಸಿದರು. ಈ ಜನಾಂಗದವರು ವಿವಿಧ ಶಾಖೆಗಳಾಗಿ ವಿಭಾಗ ಗೊಂಡು ಯೂರೋಪ್ ಮತ್ತು ಪರ್ಸಿಯಾಕ್ಕೆ ವಲಸೆ ಹೋದರು. ಅವರು ಎಲ್ಲಿ ಹೋದರೂ ತಮ್ಮ ಭಾಷೆಯನ್ನು ತಮ್ಮೊಡನೆ ತೆಗೆದುಕೊಂಡು ಹೋದರು. ಜರ್ಮನ್, ಗ್ರೀಕ್ ಮತ್ತು ಫ್ರೆಂಚ್ ಭಾಷೆಗಳು ಹಳೆಯ ಭಾಷೆಯ ಪಳೆಯುಳಿಕೆಗಳು ಹಾಗೂ ಸಂಸ್ಕೃತವು ಇವುಗಳಲ್ಲೆಲ್ಲ ಅತ್ಯಂತ ಪ್ರಬುದ್ಧ ಭಾಷೆ.

ಇವೆಲ್ಲ ಸಿದ್ಧಾಂತಗಳು, ಯಾವುದೂ ಪ್ರಮಾಣೀಕರಿಸಿಲ್ಲ - ಇವೆಲ್ಲ ಊಹೆಗಳು. ಅನೇಕ ತೊಡರುಗಳಿವೆ. ಉದಾಹರಣೆಗೆ ಭಾರತೀಯರು ಕಪ್ಪಾಗಿದ್ದು ಐರೋಪ್ಯರು ಬೆಳ್ಳಗಿರುವುದಕ್ಕೆ ಕಾರಣವೇನು? ಈ ಭಾಷೆಗಳನ್ನು ಮಾತನಾಡುವ ಒಂದೇ ದೇಶದಲ್ಲಿ ಅನೇಕರು ಹಳದಿ ಕೂದಲಿನವರೂ ಅನೇಕರು ಕಪ್ಪು ಕೂದಲಿನವರೂ ಇರುವರು. ಹೀಗಾಗಿ ಪರಿಹರಿಸಬೇಕಾದ ಅನೇಕ ಪ್ರಶ್ನೆಗಳು ಉಳಿದಿವೆ.

ಆದರೆ ಇಷ್ಟಂತೂ ನಿಜ: ಬಾಲ್ಕರು, ಹಂಗೇರಿಯನರು, ಟಾರ್ಟರರು ಮತ್ತು ಫಿನ್ನರನ್ನು ಬಿಟ್ಟರೆ, ಯೂರೋಪಿನ ಎಲ್ಲ ಜನಾಂಗದವರು, ಎಲ್ಲ ಉತ್ತರ ಭಾರತದವರು ಮತ್ತು ಪರ್ಸಿಯನರು, ಇವೆರಲ್ಲರೂ ಮಾತನಾಡುವ ಭಾಷೆಗಳು ಒಂದೇ ಮೂಲ ಭಾಷೆಯ ವಿವಿಧ ಶಾಖೆಗಳು. ಗ್ರೀಕ್, ಲ್ಯಾಟಿನ್, ಜರ್ಮನ್, ಇಂಗ್ಲಿಷ್, ಫ್ರೆಂಚ್, ಪುರಾತನ ಹಾಗೂ ಆಧುನಿಕ ಪಾರ್ಸಿ ಮತ್ತು ಸಂಸ್ಕೃತ - ಈ ಎಲ್ಲ ಆರ್ಯ ಭಾಷೆ ಗಳಲ್ಲಿಯೂ ಅಪಾರ ಸಾಹಿತ್ಯರಾಶಿ ಅಸ್ತಿತ್ವದಲ್ಲಿದೆ.

ಮೊದಲನೆಯದಾಗಿ ಸಂಸ್ಕೃತ ಸಾಹಿತ್ಯರಾಶಿಯೇ ಅತ್ಯಂತ ಬೃಹತ್ತಾದುದು. ಬಹುಶಃ ಅದರ ಮುಕ್ಕಾಲು ಭಾಗವು ಆಗಾಗ ನಡೆದ ವಿದೇಶೀ ಆಕ್ರಮಣದಿಂದ ನಾಶವಾಗಿದ್ದರೂ ಈಗಿರುವ ಒಟ್ಟು ಸಂಸ್ಕೃತ ಸಾಹಿತ್ಯವು ಯಾವುದಾದರೂ ಮೂರು ಅಥವಾ ನಾಲ್ಕು ಐರೋಪ್ಯ ಭಾಷಾ ಸಾಹಿತ್ಯವನ್ನು ಮೀರಿಸುತ್ತದೆ. ಎಷ್ಟು ಪುಸ್ತಕಗಳು ಎಲ್ಲಿ ಇವೆ ಎಂಬುದು ಯಾರಿಗೂ ಗೊತ್ತಿಲ್ಲ, ಏಕೆಂದರೆ ಇದು ಎಲ್ಲ ಆರ್ಯ ಭಾಷೆಗಳಿಗಿಂತ ಅತ್ಯಂತ ಪುರಾತನ ವಾದುದು. ಸಂಸ್ಕೃತ ಮಾತನಾಡುತ್ತಿದ್ದ ಆರ್ಯಜನಾಂಗದ ಶಾಖೆಯವರು ಪ್ರಪ್ರಥಮ ವಾಗಿ ನಾಗರಿಕರಾದವರು ಮತ್ತು ಗ್ರಂಥ ಹಾಗೂ ಸಾಹಿತ್ಯರಚನೆಯಲ್ಲಿ ಮೊಟ್ಟ ಮೊದಲಿಗರು. ಸಾವಿರಾರು ವರ್ಷಗಳು ಹೀಗೆಯೇ ಕಳೆದವು. ಎಷ್ಟು ಸಾವಿರ ವರ್ಷಗಳು ಬರೆದರೊ ಯಾರಿಗೂ ತಿಳಿಯದು. ಅನೇಕ ಊಹೆಗಳಿವೆ - ಕ್ರಿ.ಪೂ. ೩೦೦೦ ದಿಂದ ಕ್ರಿ.ಪೂ. ೮೦೦೦ ದಷ್ಟೂ ಹಿಂದಿನವರೆಗೂ ಅವರ ಕಾಲನಿರ್ಣಯ ಬದಲಾಗುತ್ತದೆ.

ಈ ಪುರಾತನ ಗ್ರಂಥಗಳ ಮತ್ತು ಕಾಲಗಳ ಬಗ್ಗೆ ಬರೆಯುವ ಪ್ರತಿಯೊಬ್ಬನೂ ತನ್ನ ಪೂರ್ವ ಶಿಕ್ಷಣದಿಂದಲೂ ತನ್ನ ಧರ್ಮದಿಂದಲೂ ಹಾಗೂ ರಾಷ್ಟ್ರೀಯತೆಯಿಂದಲೂ ಪ್ರಭಾವಿತನಾಗಿರುತ್ತಾನೆ. ಮಹಮ್ಮದನು ಹಿಂದೂಗಳ ಬಗ್ಗೆ ಬರೆಯುವಾಗ ತನ್ನ ಧರ್ಮವನ್ನು ವೈಭವೀಕರಿಸದ ಪ್ರತಿಯೊಂದನ್ನೂ ಅವನು ತಳ್ಳಿಹಾಕುತ್ತಾನೆ. ಹಾಗೆಯೇ ಕ್ರಿಶ್ಚಿಯನರೂ ಕೂಡ - ನಿಮ್ಮ ಗ್ರಂಥಕರ್ತರಲ್ಲಿಯೂ ನೀವಿದನ್ನು ನೋಡಬಹುದು. ಕಳೆದ ಹತ್ತು ವರ್ಷಗಳಲ್ಲಿ ನಿಮ್ಮ ಸಾಹಿತ್ಯ ಹೆಚ್ಚು ಗೌರವಾನ್ವಿತವಾಗಿರುವುದು. ಅಲ್ಲಿಯವರೆಗೂ ಕ್ರಿಶ್ಚಿಯನ್ನರದೇ ಆಟವಾಗಿತ್ತು - ಅವರು ಇಂಗ್ಲಿಷ್ನಲ್ಲಿ ಬರೆಯುತ್ತಿದ್ದುದರಿಂದ ಹಿಂದೂ ವಿಮರ್ಶಕರ ಕೈಗೆ ಸಿಕ್ಕುತ್ತಿರಲಿಲ್ಲ. ಆದರೆ ಇಪ್ಪತ್ತು ವರ್ಷಗಳಿಂದೀಚೆಗೆ ಹಿಂದೂಗಳೂ ಇಂಗ್ಲಿಷ್ನಲ್ಲಿ ಬರೆಯುವುದಕ್ಕೆ ಪ್ರಾರಂಭಿಸಿರುವರು, ಆದ್ದರಿಂದ ಅವರು ಸ್ವಲ್ಪ ಎಚ್ಚರಿಕೆಯಿಂದಿರುವರು. ಹತ್ತು ಅಥವಾ ಇಪ್ಪತ್ತು ವರ್ಷಗಳಿಂದೀಚೆಗೆ ಅವರ ಶ್ರುತಿ ಬದಲಾಗಿರುವುದನ್ನು ನೀವು ಗಮನಿಸಬಹುದು.

ಸಂಸ್ಕೃತದ ಇನ್ನೊಂದು ವೈಚಿತ್ರ್ಯವೇನೆಂದರೆ, ಬೇರೆ ಭಾಷೆಗಳಂತೆಯೇ ಅದೂ ಕೂಡ ತುಂಬ ಬದಲಾವಣೆಯನ್ನು ಹೊಂದಿರುವುದು. ಗ್ರೀಕ್, ಲ್ಯಾಟಿನ್ ಮುಂತಾದ ಆರ್ಯ ಭಾಷೆಗಳ ಎಲ್ಲ ಸಾಹಿತ್ಯವನ್ನು ಒಟ್ಟಿಗೆ ತೆಗೆದುಕೊಂಡರೆ, ಐರೋಪ್ಯ ಭಾಷೆಗಳ ಸಾಹಿತ್ಯವು ಇತ್ತೀಚಿನವುಗಳೆಂಬುದು ತಿಳಿದುಬರುತ್ತದೆ. ಗ್ರೀಕ್ ಸಾಹಿತ್ಯ ಬಹಳ ಕಾಲದ ನಂತರ ಬೆಳೆಯಿತು - ಈಜಿಪ್ಟ್ ಅಥವಾ ಬ್ಯಾಬಿಲೋನಿಯನ್ ಭಾಷೆಗೆ ಹೋಲಿಸಿದರೆ ಅದು ಮಗು ಇದ್ದಂತೆ.

ಈಜಿಪ್ಷಿಯನ್ ಮತ್ತು ಬ್ಯಾಬಿಲೋನಿಯನ್ ಭಾಷೆಗಳು ಆರ್ಯ ಭಾಷೆಗಳಲ್ಲ, ನಿಜ, ಅವರದೇ ಪ್ರತ್ಯೇಕ ಜನಾಂಗ, ಅವರ ನಾಗರಿಕತೆ ಎಲ್ಲ ಐರೋಪ್ಯ ನಾಗರಿಕತೆಗಳಿ ಗಿಂತ ಹಿಂದಿನದು. ಪುರಾತನ ಈಜಿಪ್ಷಿಯನರನ್ನು ಬಿಟ್ಟರೆ, ಅವರು ಆರ್ಯರ ಸಮ ಕಾಲೀನರೆಂದು ಹೇಳಬಹುದು. ಕೆಲವರ ಹೇಳಿಕೆಯ ಪ್ರಕಾರ ಅವರು ಆರ್ಯರಿಗಿಂತ ಹಿಂದಿನವರು. ಆದರೂ ಈಜಿಪ್ಷಿಯನ್ ಸಾಹಿತ್ಯದಲ್ಲಿ ಇನ್ನೂ ನಿರ್ಧರಿಸಬೇಕಾದ ಕೆಲವು ವಿಷಯಗಳಿವೆ - ಉದಾಹರಣೆಗೆ ಹಳೆಯ ದೇವಸ್ಥಾನಗಳಲ್ಲಿ ಭಾರತೀಯ (ಗಂಗಾ) ತಾವರೆಯ ಬಳಕೆ. ಈ ತಾವರೆಯು ಭಾರತದಲ್ಲಿ ಮಾತ್ರ ಬೆಳೆಯುವುದೆಂಬುದು ಸರ್ವವಿದಿತ. ಅಲ್ಲದೆ ಅದರಲ್ಲಿ ಪಂತ್ ಭೂಮಿಯ ಬಗ್ಗೆ ಪ್ರಸ್ತಾಪವಿದೆ. ಈ ಪಂತ್ ಭೂ ಭಾಗವು ಅರಬ್ ಪ್ರಾಂತ್ಯಕ್ಕೆ ಸೇರಿದ್ದೆಂದು ಪ್ರಮಾಣೀಕರಿಸಲು ಬಹಳ ಪ್ರಯತ್ನ ನಡೆ ದಿದ್ದರೂ, ಅದು ಇನ್ನೂ ಅನಿಶ್ಚಿತವಾಗಿಯೇ ಉಳಿದಿದೆ. ಮತ್ತೆ ದಕ್ಷಿಣ ಭಾರತದಲ್ಲಿ ಮಾತ್ರ ಕಾಣಸಿಗುವ ಮಂಗಗಳು ಮತ್ತು ಗಂಧದ ಮರಗಳ ಪ್ರಸ್ತಾಪವಿದೆ ಅದರಲ್ಲಿ.

ಯಹೂದಿಗಳು ಗ್ರೀಕ್ ಆರ್ಯರಿಗಿಂತ ತುಂಬ ಇತ್ತೀಚಿನವರು. ಬ್ಯಾಬಿಲೋನಿ ಯನ್ನಿಗೆ ಸೇರಿದ ಸೆಮಿಟಿಕ್ ಜನಾಂಗದ ಒಂದು ಶಾಖೆಯವರು ಮತ್ತು ಯಾವ ಮೂಲದವರೆಂದು ತಿಳಿಯಲಾಗದ ಈ ಈಜಿಪ್ಷಿಯನರು ಮಾತ್ರ, ಹಿಂದೂಗಳನ್ನು ಹೊರತಾಗಿ, ಆರ್ಯರಿಗಿಂತ ತುಂಬ ಹಿಂದಿನವರು.

ಸಾವಿರಾರು ವರ್ಷಗಳ ಸಂಭಾಷಣೆ ಮತ್ತು ಬರವಣಿಗೆಯ ಪರಿಣಾಮವಾಗಿ ಈ ಸಂಸ್ಕೃತ ಭಾಷೆಯು ಬಹಳ ಬದಲಾವಣೆಯನ್ನು ಹೊಂದಿದೆ. ಗ್ರೀಕ್ ಮತ್ತು ರೋಮನ್ ಮುಂತಾದ ಇತರ ಆರ್ಯ ಭಾಷೆಗಳ ಸಾಹಿತ್ಯವು ಸಂಸ್ಕೃತಕ್ಕಿಂತ ತುಂಬ ಇತ್ತೀಚಿನ ದೆಂದು ಇದರಿಂದ ತಿಳಿದುಬರುತ್ತದೆ. ಇನ್ನೊಂದು ವಿಶೇಷವೇನೆಂದರೆ, ಇಂದು ಜಗತ್ತಿ ನಲ್ಲಿರುವ ನಿರ್ದಿಷ್ಟ ಗ್ರಂಥಗಳಲ್ಲೆಲ್ಲ ಅತ್ಯಂತ ಪುರಾತನವಾದುದು ಸಂಸ್ಕೃತದಲ್ಲಿದೆ - ಆ ಗ್ರಂಥರಾಶಿಯನ್ನು ವೇದಗಳೆಂದು ಕರೆಯುತ್ತಾರೆ. ಬ್ಯಾಬಿಲೋನಿಯನ್ ಅಥವಾ ಈಜಿಪ್ಷಿಯನ್ ಭಾಷೆಗಳಲ್ಲಿ ಇದಕ್ಕಿಂತಲೂ ಪುರಾತನವಾದ ಬರಹಗಳಿವೆ; ಆದರೆ ಅವುಗಳನ್ನು ಸಾಹಿತ್ಯ ಅಥವಾ ಗ್ರಂಥಗಳೆಂದು ಕರೆಯಲಾಗುವುದಿಲ್ಲ, ಅವು ಕೇವಲ ಕೆಲವು ಶಬ್ದಗಳು ಅಥವಾ ಟಿಪ್ಪಣಿಗಳು ಅಷ್ಟೇ. ಪರಿಪೂರ್ಣ ಸಾಂಸ್ಕೃತಿಕ ಸಾಹಿತ್ಯವಾಗಿ ವೇದಗಳೇ ಅತ್ಯಂತ ಪುರಾತನ.

ಈ ವೇದಗಳು ಪ್ರಾಚೀನ ಸಂಸ್ಕೃತದಲ್ಲಿ ಬರೆಯಲ್ಪಟ್ಟಿವೆ. ಇಂದಿಗೂ ಕೂಡ ಪುರಾತನ ಶಾಸ್ತ್ರಜ್ಞರು, ಈ ವೇದಗಳು ಬರೆಯಲ್ಪಟ್ಟಿಲ್ಲ, ವಂಶಪಾರಂಪರ್ಯವಾಗಿ ಅಥವಾ ಗುರುಶಿಷ್ಯ ಪರಂಪರೆಯಿಂದ ಕಂಠಪಾಠದ ಮೂಲಕ ಅವು ಕಾಯ್ದಿರಿಸಲ್ಪಟ್ಟಿವೆ ಎಂದು ಹೇಳುತ್ತಾರೆ. ಈ ಕೆಲವು ವರ್ಷಗಳಿಂದ ಈ ಅಭಿಪ್ರಾಯ ಬದಲಾಗುತ್ತಿದೆ; ತುಂಬ ಪ್ರಾಚೀನ ಕಾಲದಲ್ಲಿ ಅವು ಬರೆಯಲ್ಪಟ್ಟಿದ್ದವು ಎಂದು ಹೇಳಲಾಗುತ್ತಿದೆ.

ಹೀಗೆಯೇ ಸಿದ್ಧಾಂತಗಳನ್ನು ಅವರು ರೂಪಿಸಬೇಕಾಗುತ್ತದೆ. ಸತ್ಯ ಹೊರಬರು ವವರೆಗೂ ಒಂದಾದ ಮೇಲೆ ಒಂದು ಸಿದ್ಧಾಂತಗಳು ಬರುತ್ತವೆ ಮತ್ತು ನಾಶವಾಗುತ್ತವೆ. ಇದು ಸ್ವಾಭಾವಿಕವೇ. ಆದರೆ ವಿಷಯವು ಭಾರತ ಅಥವಾ ಈಜಿಪ್ಟಿಗೆ ಸಂಬಂಧಿಸಿದು ದಾದರೆ, ಕ್ರೈಸ್ತ ತತ್ತ್ವಜ್ಞರು ಆತುರದಲ್ಲಿ ಯಾವುದಾದರೂ ಸಿದ್ಧಾಂತವನ್ನು ತಯಾರಿಸು ತ್ತಾರೆ; ಆದರೆ ತಮ್ಮ ಹತ್ತಿರದ ವಿಷಯಕ್ಕೆ ಸಂಬಂಧಿಸಿದುದಾದರೆ ಸಿದ್ಧಾಂತವನ್ನು ರೂಪಿಸು ವುದಕ್ಕೆ ಮುಂಚೆ ತುಂಬ ಯೋಚಿಸುತ್ತಾರೆ. ಆದ್ದರಿಂದಲೇ ಅವರು ತುಂಬ ಎಡುವುತ್ತಾರೆ ಮತ್ತು ಪ್ರತಿ ಐದು ವರ್ಷಕ್ಕೆ ಹೊಸ ಹೊಸ ಸಿದ್ಧಾಂತಗಳನ್ನು ರೂಪಿಸುತ್ತ ಹೋಗುತ್ತಾರೆ. ಆದರೆ ಇಷ್ಟಂತೂ ಸತ್ಯ, ಈ ವೇದ ಸಾಹಿತ್ಯವು, ಹಿಂದೆ ಬರೆಯಲ್ಪಟ್ಟಿರಲಿ ಅಥವಾ ಇಲ್ಲ ದಿರಲಿ, ನಮ್ಮ ಪಾಲಿಗೆ ಅವು ಬಂದಿರುವುದು ಬಾಯಿಯ ಮೂಲಕವೇ. ಆದ್ದರಿಂದಲೇ ಅವನ್ನು ಪವಿತ್ರವೆಂದು ಪರಿಗಣಿಸಲಾಗಿದೆ.

ಒಂದು ಹೊಸ ಭಾವನೆ, ಒಂದು ಹೊಸ ರೂಪ, ಒಂದು ಹೊಸ ಆವಿಷ್ಕಾರ ಅಥವಾ ಸಂಶೋಧನೆ ಬೆಳಕಿಗೆ ಬಂದರೆ ಹಳೆಯ ವಿಷಯಗಳು ಹಾಗೆಯೇ ಹೊರಟುಹೋಗು ವುದಿಲ್ಲ; ಪವಿತ್ರ ಧಾರ್ಮಿಕ ಚಿಹ್ನೆಗಳಾಗಿ ಉಳಿಯುತ್ತವೆ. ಪ್ರತಿಯೊಂದು ದೇಶದಲ್ಲಿಯೂ ಹೀಗಾಗುವುದನ್ನು ನೋಡುತ್ತೇವೆ. ಪುರಾತನ ಹಿಂದೂಗಳು ತಾಳೆಯಗರಿ ಮತ್ತು ಭೂರ್ಜದ ತೊಗಟೆಯ ಮೇಲೆ ಬರೆಯುತ್ತಿದ್ದರು. ಕಾಗದವು ಬಂದ ಮೇಲೆ ತಾಳೆಯಗರಿ ಮತ್ತು ಭೂರ್ಜದ ತೊಗಟೆಯನ್ನು ಅವರು ಬಿಸಾಕಲಿಲ್ಲ; ಅವುಗಳ ಮೇಲೆ ಬರೆಯುವುದು ಪವಿತ್ರವೆಂದು ಪರಿಗಣಿಸಿದರು. ಹೀಗೆಯೇ ಯಹೂದಿಗಳೂ ಕೂಡ. ಅವರು ಚರ್ಮದ ಹಾಳೆಗಳ ಮೇಲೆ ಬರೆಯುತ್ತಿದ್ದರು, ಈಗ ಅವನ್ನು ದೇವಸ್ಥಾನ ಗಳಲ್ಲಿ ಬರೆಯುವುದಕ್ಕೆ ಉಪಯೋಗಿಸುತ್ತಾರೆ. ಹೊಸ ಪದ್ಧತಿಗಳು ಜಾರಿಗೆ ಬಂದ ಮೇಲೆ ಹಳೆಯವು ಪವಿತ್ರವಾಗುತ್ತವೆ. ಹಾಗೆಯೇ ಗುರು - ಶಿಷ್ಯ ಪರಂಪರೆಯಿಂದ ಬಾಯಿಯ ಮೂಲಕ ಉಳಿಸಿಕೊಂಡು ಬರುವ ಪದ್ಧತಿಯು ಹಳತಾಗಿ ಹೋಗಿದ್ದರೂ ಈಗಲೂ ಅದೊಂದುಪವಿತ್ರ ಪದ್ಧತಿಯಾಗಿ ಉಳಿದಿದೆ. ವಿದ್ಯಾರ್ಥಿಯು ನೆನಪಿಟ್ಟುಕೊಳ್ಳು ವುದಕ್ಕಾಗಿ ಗ್ರಂಥಗಳ ಸಹಾಯವನ್ನು ಪಡೆಯಬಹುದಾದರೂ ಅವನದನ್ನು ಗುರುವಿ ನಿಂದ ಬಾಯಿ ಮೂಲಕವೇ ಕಲಿಯಬೇಕು. ಈ ಪವಿತ್ರತೆಯನ್ನು ವಿಚಾರಪೂರ್ಣವ ನ್ನಾಗಿ ಮಾಡಲು ಅನೇಕ ವಿಷಯಗಳು ಇದರ ಸುತ್ತ ಸೇರಿಕೊಳ್ಳುತ್ತವೆ. ಆದರೆ ನಿಜವಾದ ನಿಯಮ ಬೇರೆ.

ವೇದಗಳು ಒಂದು ಅಪಾರ ಸಾಹಿತ್ಯರಾಶಿ. ಪುರಾತನ ಕಾಲದಲ್ಲಿ ಪ್ರತಿಯೊಂದು ದೇಶದಲ್ಲಿಯೂ ಮನುಷ್ಯನ ಹೃದಯದಿಂದ ಹೊರಚಿಮ್ಮುವ ಪ್ರಥಮ ಆದರ್ಶವೇ ಧರ್ಮವಾಗಿತ್ತು - ಲೌಕಿಕ ಜ್ಞಾನಗಳೆಲ್ಲ ಅನಂತರ ಬಂದವುಗಳು.

ಎರಡನೆಯದಾಗಿ, ಧಾರ್ಮಿಕ ವಿಷಯವನ್ನು ನೋಡಿಕೊಳ್ಳುವವರನ್ನು ಮುಂದೆ ಪುರೋಹಿತರೆಂದು ಕರೆಯುತ್ತಿದ್ದರು ಮತ್ತು ಇವರೇ ಪ್ರಥಮ ಚಿಂತಕರು. ಇವರು ಧಾರ್ಮಿಕ ವಿಷಯಗಳನ್ನು ಮಾತ್ರವಲ್ಲದೆ ಲೌಕಿಕ ವಿಷಯಗಳನ್ನು ಕುರಿತೂ ಚಿಂತಿಸುತ್ತಿ ದ್ದರು. ಎಲ್ಲ ಜ್ಞಾನವೂ ಅವರಿಗೇ ಸೀಮಿತವಾಗಿತ್ತು. ಈ ಲೌಕಿಕ ಮತ್ತು ಧಾರ್ಮಿಕ ಜ್ಞಾನರಾಶಿಗಳೆಲ್ಲ ಒಟ್ಟು ಕಲೆತು ಒಂದು ಅಪಾರ ಸಾಹಿತ್ಯ ರಾಶಿಯಾಗಿ ಪರಿಣಮಿಸಿತು.

ಮುಂದೆಯೂ ಬಹಳ ಕಾಲ ಹೀಗೆಯೇ ಆಯಿತು. ಉದಾಹರಣೆಗೆ ಬೈಬಲನ್ನು ಓದಿದರೆ ಇದೇ ರೀತಿ ಆಗಿರುವುದನ್ನು ನೋಡುತ್ತೇವೆ. ತಾಲ್ಮುಡ್ ಮತ್ತು ಪೆಂಟಾಟೆಕ್ ಎಲ್ಲ ವಿಷಯಗಳಿಗೆ ಸಂಬಂಧಿಸಿದ ಮಾಹಿತಿಗಳ ರಾಶಿಯನ್ನೇ ಹೊಂದಿವೆ. ಹಾಗೆಯೇ ವೇದಗಳೂ ಕೂಡ ವಿವಿಧ ವಿಷಯಗಳಿಗೆ ಸಂಬಂಧಿಸಿದ ಮಾಹಿತಿಗಳನ್ನು ಒಳಗೊಂಡಿವೆ. ಈ ಮಾಹಿತಿಗಳೆಲ್ಲ ಒಟ್ಟುಗೂಡಿ ಒಂದು ಬೃಹತ್ ಗ್ರಂಥವಾಗಿದೆ. ಮುಂದೆ ಖಗೋಳ ಶಾಸ್ತ್ರ, ಜ್ಯೋತಿಶ್ಶಾಸ್ತ್ರ ಮುಂತಾದ ವಿಷಯಗಳು ಧರ್ಮದಿಂದ ಬೇರ್ಪಟ್ಟವು. ಆದರೆ ಅವು ವೇದಗಳಲ್ಲಿ ಉಕ್ತವಾಗಿರುವುದರಿಂದಲೂ ಪ್ರಾಚೀನವಾದುದರಿಂದಲೂ ಅವನ್ನು ಪವಿತ್ರವೆಂದು ಪರಿಗಣಿಸಲಾಗಿದೆ.

ವೇದಗಳ ಬಹುಭಾಗ ನಷ್ಟವಾಗಿಹೋಗಿವೆ. ವೇದಸಂಪತ್ತಿನ ಅಧಿಕಾರಿಗಳಾಗಿದ್ದ ಬ್ರಾಹ್ಮಣರು ಅನೇಕ ವಂಶಗಳಾಗಿ ವಿಭಾಗಗೊಂಡಿದ್ದರು. ಅದರಂತೆಯೇ ವೇದಗಳು ಅನೇಕ ಭಾಗಗಳಾಗಿ ವಿಭಾಗಗೊಂಡಿದ್ದವು. ಒಂದು ಭಾಗವೂ ಒಂದೊಂದು ವಂಶಕ್ಕೆ ಸೀಮಿತವಾಗಿತ್ತು. ಆಯಾಯ ವಂಶದ ಆಚರಣೆಗಳು, ಕ್ರಿಯಾವಿಧಿಗಳು, ಪೂಜಾಪದ್ಧತಿ ಗಳು ಆಯಾಯ ವಂಶಕ್ಕೆ ಸೇರಿದ ವೇದಭಾಗದಿಂದ ಪಡೆಯಬೇಕಾಗಿತ್ತು. ಆಯಾಯ ವಂಶದವರು ಅವುಗಳನ್ನು ಆಚರಿಸಿ ಉಳಿಸಿಕೊಂಡು ಬರುತ್ತಿದ್ದರು. ಕಾಲಕ್ರಮೇಣ ಇಂಥ ಕೆಲವು ವಂಶಗಳು ನಿರ್ನಾಮವಾಗಿ ಹೋದವು, ಅವುಗಳ ಜೊತೆ ಆ ವಂಶಗಳಿಗೆ ಸೇರಿದ ವೇದಭಾಗಗಳೂ ನಶಿಸಿಹೋದವು.

ವೇದಗಳು ಋಗ್, ಯಜುಸ್, ಸಾಮ ಮತ್ತು ಅಥರ್ವಣ ಎಂದು ನಾಲ್ಕು ಭಾಗ ಗಳಾಗಿ ವಿಭಾಗಗೊಂಡಿರುವುದು ನಿಮ್ಮಲ್ಲಿ ಕೆಲವರಿಗೆ ಗೊತ್ತು. ಇವುಗಳಲ್ಲಿ ಪ್ರತಿ ಯೊಂದೂ ಮತ್ತೆ ಹಲವು ಶಾಖೆಗಳಾಗಿ ವಿಭಾಗಗೊಂಡಿವೆ. ಉದಾಹರಣೆಗೆ ಸಾಮ ವೇದವು ಒಂದು ಸಾವಿರ ಶಾಖೆಗಳನ್ನು ಒಳಗೊಂಡಿದ್ದು, ಅವುಗಳಲ್ಲಿ ಕೇವಲ ಐದು ಅಥವಾ ಆರು ಶಾಖೆಗಳು ಮಾತ್ರ ಉಳಿದಿವೆ; ಉಳಿದವೆಲ್ಲ ನಷ್ಟವಾಗಿವೆ. ಇದೇ ರೀತಿ ಇತರ ಭಾಗಗಳೂ ಕೂಡ. ಋಗ್ವೇದಕ್ಕೆ ೧೦೮ ಶಾಖೆಗಳಿದ್ದವು, ಈಗ ಒಂದು ಮಾತ್ರ ಉಳಿದಿದೆ. ಉಳಿದವೆಲ್ಲ ನಷ್ಟವಾದವು.

ಇಷ್ಟೇ ಅಲ್ಲದೆ ಎಷ್ಟೋ ಹೊರಗಿನ ಆಕ್ರಮಣಗಳಿದ್ದವು. ಯಾವುದೇ ರಾಷ್ಟ್ರವು ಶಕ್ತಿಯುತವಾದಕೂಡಲೆ ಭಾರತವನ್ನು ಆಕ್ರಮಿಸಿ ಗೆಲ್ಲಬೇಕೆಂದು ಬಯಸುತ್ತಿತ್ತು, ಏಕೆಂದರೆ ಅದು ಶ‍್ರೀಮಂತ ರಾಷ್ಟ್ರವಾಗಿತ್ತು. ಅತ್ಯಂತ ಪುರಾತನ ಕಾಲದಿಂದಿಂದಲೂ ಭಾರತ ಐಶ್ವರ್ಯಕ್ಕೆ ಅತಿ ಪ್ರಸಿದ್ಧವಾಗಿತ್ತು. ಅನೇಕ ವಿದೇಶೀ ಆಕ್ರಮಣಕಾರರು ಶ‍್ರೀಮಂತ ರಾಗಲು ಭಾರತಕ್ಕೆ ನುಗ್ಗಿ ಬಂದರು. ಇಂಥ ಪ್ರತಿಯೊಂದು ಆಕ್ರಮಣವೂ ಈ ಕೆಲವು ವಂಶಗಳನ್ನೂ ನಾಶಮಾಡಿತು, ಪುಸ್ತಕಭಂಡಾರಗಳನ್ನೂ ಮನೆಗಳನ್ನೂ ಸುಟ್ಟುಹಾಕಿತು. ಹೀಗೆ ಹೆಚ್ಚಿನ ಸಾಹಿತ್ಯ ನಾಶವಾಯಿತು. ಇತ್ತೀಚಿನ ವರ್ಷಗಳಲ್ಲಿ ಮಾತ್ರ ಈ ವಿವಿಧ ಧರ್ಮಗಳನ್ನೂ ಗ್ರಂಥಗಳನ್ನೂ ಉಳಿಸಿಕೊಳ್ಳಬೇಕೆಂಬ ಭಾವನೆ ಉದಿಸಲು ಪ್ರಾರಂಭ ವಾಯಿತು. ಅದಕ್ಕೆ ಮುಂಚೆ ಕೊಳ್ಳೆ ಹೊಡೆಯುವುದು, ಧ್ವಂಸಮಾಡುವುದನ್ನು ಮಾನವ ಕೋಟಿಯು ಅನುಭವಿಸಬೇಕಾಗಿತ್ತು. ಅತ್ಯದ್ಭುತವಾದ ಕಲಾ ಕೃತಿಗಳು ಹೇಳ ಹೆಸರಿಲ್ಲದೆ ನಾಶವಾದವು. ಅತ್ಯಂತ ಸುಂದರ ಕಟ್ಟಡಗಳು ನಾಶವಾದವು; ಅವುಗಳ ಉಳಿದ ಸ್ವಲ್ಪ ಭಾಗಗಳು ಅವು ಎಷ್ಟು ಅದ್ಭುತವಾಗಿದ್ದವು ಎಂಬುದಕ್ಕೆ ಸಾಕ್ಷಿಯಾಗಿವೆ.

ತಮ್ಮ ಮತಕ್ಕೆ ಸೇರದವರಾರಿಗೂ ಬದುಕಲು ಹಕ್ಕಿಲ್ಲ ಎಂಬ ಮತಾಂಧ ನಂಬಿಕೆ ಯನ್ನು ಈ ಆಕ್ರಮಣಕಾರರಲ್ಲಿ ಅನೇಕರು ಹೊಂದಿದ್ದರು. ಸತ್ತ ಮೇಲೆ ಅವರು ನರಕಾಗಿ ಯಲ್ಲಿ ಬೇಯಬೇಕೆಂದೂ ಇಲ್ಲಿರುವಾಗ ಅವರು ಗುಲಾಮಗಿರಿ ಮತ್ತು ಸಾವಿಗೆ ಮಾತ್ರ ಅರ್ಹರೆಂದೂ ಈ ಆಕ್ರಮಣಕಾರರು ನಂಬಿದ್ದರು. ಅನ್ಯ ಮತೀಯರು ತಮಗೆ ಗುಲಾಮ ರಾಗಿ ಮಾತ್ರ ಇರಬೇಕೆಂದೂ ಸ್ವತಂತ್ರವಾಗಿ ಜೀವಿಸಲು ಅವರಿಗೆ ಹಕ್ಕಿಲ್ಲವೆಂದೂ ಅವರು ಭಾವಿಸಿದರು. ಇಂಥ ಮತಾಂಧ ಆಕ್ರಮಣಕಾರರ ಪ್ರವಾಹ ಭಾರತದ ಮೇಲೆ ಬಂದೆರಗಿದಾಗ ಎಲ್ಲವನ್ನೂ ಅದು ಮುಳುಗಿಸಿತು. ಗ್ರಂಥಗಳು, ಸಾಹಿತ್ಯ ಮತ್ತು ನಾಗರಿಕತೆ ಎಲ್ಲ ನುಚ್ಚುನೂರಾಯಿತು.

ಆದರೆ ಆ ಜನಾಂಗದಲ್ಲಿ ಒಂದು ಸತ್ತ್ವವಿದೆ. ಅದೇ ಅದರ ವೈಶಿಷ್ಟ್ಯ ಬಹುಶಃ ಈ ಸತ್ತ್ವವು ಅಪ್ರತೀಕಾರ ಮನೋಭಾವದಿಂದ ಬಂದಿರುವುದು. ಅಪ್ರತೀಕಾರವೇ ಅತ್ಯಂತ ಶ್ರೇಷ್ಠ ಶಕ್ತಿ. ದೈನ್ಯತೆಯಲ್ಲಿ, ಮೃದುತ್ವದಲ್ಲಿ ಶ್ರೇಷ್ಠ ಶಕ್ತಿಯಿದೆ. ಕರ್ಮದಲ್ಲಿರುವು ದಕ್ಕಿಂತ ದುಃಖಾನುಭವದಲ್ಲಿ ಹೆಚ್ಚು ಶಕ್ತಿಯಿದೆ. ಇತರರನ್ನು ನೋಯಿಸುವುದಕ್ಕಿಂತ ಕೋಪವನ್ನು ತಡೆಹಿಡಿಯುವುದರಲ್ಲಿ ಅತ್ಯಧಿಕ ಶಕ್ತಿಯಿದೆ. ಎಲ್ಲ ಕಷ್ಟಸಂಕಟಗಳನ್ನು ಅನುಭವಿಸಿದ, ಸಂಪತ್ತನ್ನು ಅನುಭವಿಸಿದ ಆ ಜನಾಂಗದ ಮುಖ್ಯ ಪಲ್ಲವಿಯೇ ಇದಾಗಿದೆ. ನೆರೆ ದೇಶದವರನ್ನು ನಾಶಮಾಡಲು ತನ್ನ ಗಡಿಯಿಂದಾಚೆಗೆ ಹೋಗದ ದೇಶ ಅದೊಂದೇ. ಇಲ್ಲಿಯೇ ಅದರ ವೈಭವವಿರುವುದು. ನಾನೋರ್ವ ಹಿಂದೂ ಎಂದು ಭಾವಿಸಲು, ಯಾರನ್ನೂ ನೋಯಿಸಲು ಹೊರಡದ ಏಕೈಕ ಜನಾಂಗಕ್ಕೆ ಸೇರಿದವನು ಎಂದು ಚಿಂತಿಸಲು ನನಗೆ ಹೆಮ್ಮೆಯಾಗುತ್ತದೆ. ಆ ರಾಷ್ಟ್ರದ ಏಕೈಕ ಕಾರ್ಯವೇ ಮಾನವ ಕೋಟಿಗೆ ಏನಾದರೂ ಕೊಡುವುದು, ಜ್ಞಾನದಾನ ಮಾಡುವುದು ಶುದ್ಧೀಕರಿಸುವುದು ಆದರೆ ಎಂದಿಗೂ ದೋಚುವುದಲ್ಲ.

ಜಗತ್ತಿನ ಮುಕ್ಕಾಲು ಪಾಲು ಐಶ್ವರ್ಯವು ಭಾರತದಿಂದ ಬಂದಿದೆ ಮತ್ತು ಈಗಲೂ ಕೂಡ ಹಾಗೆಯೇ. ಭಾರತದ ವಾಣಿಜ್ಯ ಜಗತ್ತಿನ ಇತಿಹಾಸದಲ್ಲಿ ಒಂದು ನವೋದಯಕ್ಕೆ ಕಾರಣವಾಯಿತು. ಅದನ್ನು ಪಡೆದ ದೇಶಗಳೆಲ್ಲ ಶಕ್ತಿಯುತವೂ ನಾಗರಿಕವೂ ಆದವು. ಗ್ರೀಕರು ಅದನ್ನು ಪಡೆದು ಅತ್ಯಂತ ಬಲಶಾಲಿಗಳಾದರು, ರೋಮನರು ಅದನ್ನು ಪಡೆದು ಹಾಗೆಯೇ ಬಲಶಾಲಿಗಳಾದರು. ಫೋನಿಸಿಯನ್ನರ ಕಾಲದಲ್ಲಿಯೂ ಕೂಡ ಹೀಗೆಯೇ. ರೋಮನರ ಪತನವಾದನಂತರ ಜೆನೋಸರು ಮತ್ತು ವೆನೀಶಿಯನರು ಅದನ್ನು ಪಡೆದರು. ಅನಂತರ ಅರಬರು ಪ್ರವರ್ಧಮಾನಕ್ಕೆ ಬಂದು ವೆನಿಸ್ ಮತ್ತು ಭಾರತಗಳ ನಡುವೆ ಒಂದು ಗೋಡೆಯನ್ನು ನಿರ್ಮಿಸಿದರು. ಭಾರತಕ್ಕೆ ಹೊಸ ಮಾರ್ಗವನ್ನು ಕಂಡುಹಿಡಿ ಯುವ ಹೋರಾಟದಲ್ಲಿ ಅಮೆರಿಕ ಖಂಡದ ಆವಿಷ್ಕಾರವಾಯಿತು. ಅದಕ್ಕಾಗಿಯೇ ಅಮೆರಿಕದ ಮೂಲಪುರುಷರು ಇಂಡಿಯನರು ಅಥವಾ “ಇಂಜನರು” ಎಂದು ಕರೆಯಲ್ಪಟ್ಟರು. ಡಚ್ಚರೂ ಅದನ್ನು ಪಡೆದರು. ಆಂಗ್ಲರು ಅದನ್ನು ಪಡೆದು ಇಂಗ್ಲೆಂಡ್ ಜಗತ್ತಿನಲ್ಲೇ ಅತ್ಯಂತ ಶಕ್ತಿಶಾಲಿ ರಾಷ್ಟ್ರವಾಯಿತು. ಅದನ್ನು ಪಡೆಯುವ ಮುಂದಿನ ಯಾವುದೇ ರಾಷ್ಟ್ರವಾದರೂ ಕೂಡಲೇ ಅತ್ಯಂತ ಶಕ್ತಿಶಾಲಿಯಾಗುವುದು.

ನಮ್ಮ ದೇಶದಲ್ಲಿ ಕಂಡುಬರುವ ಈ ಅಪಾರ ಶಕ್ತಿಯನ್ನು ಯೋಚಿಸಿ ನೋಡಿ! ಅದು ಎಲ್ಲಿಂದ ಬರುತ್ತಿದೆ? ಭಾರತೀಯರು ಉತ್ಪಾದಕರು ಮತ್ತು ನೀವು ಭೋಗಿಸುವವರು ಎಂಬುದರಲ್ಲಿ ಸಂಶಯವಿಲ್ಲ. ಸಹಿಷ್ಣುಗಳಾದ, ಶ್ರಮಜೀವಿಗಳಾದ ಮಿಲಿಯಗಟ್ಟಳೆ ಹಿಂದೂಗಳು ಪ್ರತಿಯೊಬ್ಬರ ದಬ್ಬಾಳಿಕೆಗೆ ತುತ್ತಾಗಿ ಸಂಪತ್ತನ್ನು ಉತ್ಪಾದಿಸಿದರು. ಮಿಲಿಯ ಗಟ್ಟಳೆ ಭಾರತೀಯರನ್ನು ಶಪಿಸುವ ಮಿಷನರಿಗಳೂ ಕೂಡ ಅವರ ಶ್ರಮದಿಂದಲೇ ಸ್ಥೂಲಕಾಯರಾಗಿರುವರು. ಇದು ಆ ಮಿಷನರಿಗಳಿಗೆ ತಿಳಿಯದು ಅಷ್ಟೆ. ಅವರ ಬಲಿದಾನ ದಿಂದಲೇ ಜಗತ್ತಿನ ಇತಿಹಾಸ ಇದುವರೆಗೂ ಬದಲಾಗುತ್ತಿರುವುದು; ಮುಂದೆಯೂ ಸಾವಿರಾರು ವರ್ಷಗಳ ವರೆಗೆ ಹೀಗೆಯೇ ಬದಲಾಗುತ್ತ ಹೋಗುತ್ತದೆ. ಇದರ ಪ್ರಯೋಜನ ವೇನು? ಇದು ಆ ದೇಶಕ್ಕೆ ಶಕ್ತಿಯನ್ನು ನೀಡುತ್ತದೆ. ಅವರು ಒಂದು ದೃಷ್ಟಾಂತವಿದ್ದಂತೆ. ಅವರು ದುಃಖವನ್ನು ಅನುಭವಿಸಬೇಕು; ಅದರ ನಡುವೆಯೇ ತಲೆಯೆತ್ತಿ ನಿಂತು, ಧರ್ಮದ ಸತ್ಯಗಳಿಗಾಗಿ ಹೋರಾಡುತ್ತ ಅಪ್ರತೀಕಾರವೇ ಹೆಚ್ಚು ಶ್ರೇಷ್ಠವಾದುದು, ದುಃಖಾನು ಭವವೇ ಹೆಚ್ಚು ಉನ್ನತವಾದುದು, ತಮ್ಮ ಆಕ್ರಮಣಕಾರರೂ ಒಪ್ಪಿಕೊಳ್ಳುವಂತೆ, ಜೀವನವೇ ಗುರಿಯಾದರೆ ನಮ್ಮ ಜನಾಂಗವೊಂದೇ ಅಮೃತವಾದುದು ಎಂದು ಇಡೀ ಜಗತ್ತಿಗೆ ಸಾರಬೇಕು.

ಯಾರ ಸೈನಿಕರ ದಂಡು ಪ್ರಪಂಚವನ್ನೆಲ್ಲ ಸುತ್ತಾಡಿತೊ ಆ ಗ್ರೀಕರು ಈಗ ಎಲ್ಲಿ ದ್ದಾರೆ? ಸಾವಿರಾರು ವರ್ಷಗಳ ಹಿಂದೆಯೇ ಅವರು ಕಣ್ಮರೆಯಾದರು. ಉತ್ತರದ ಬರ್ಬರರು ಅವರ ಮೇಲೆ ಆಕ್ರಮಣ ಮಾಡಿದಾಗ ಅವರು ನಿರ್ನಾಮವಾದರು. ಯಾರ ಸೈನ್ಯದಳ ಭೂಮಿಯನ್ನು ಮೆಟ್ಟಿನಿಂತಿತ್ತೊ ಆ ಬಲಿಷ್ಠ ರೋಮನರು ಈಗ ಎಲ್ಲಿ? ಬೆಳಗಿನ ಹಿಮಮಣಿಯಂತೆ ಕರಗಿಹೋದರು.

ಆದರೆ ಮೂವತ್ತು ಕೋಟಿ ಹಿಂದೂಗಳು ಶಕ್ತಿ ಶಾಲಿಗಳಾಗಿ ಇನ್ನೂ ಬದುಕಿರು ವರು. ಆ ಜನಾಂಗದ ಜೀವಸತ್ತ್ವವನ್ನು ಆಲೋಚಿಸಿ ನೋಡಿ! ಇಡೀ ಪ್ರಪಂಚವೇ ಅವರನ್ನು ಸಾಯಿಸುವುದಕ್ಕಿಂತ ಹೆಚ್ಚಾಗಿ ಅವರು ವೃದ್ಧಿಯಾಗಬಲ್ಲರು. ಇದು ಆ ಜನಾಂಗದ ಪ್ರಾಣಶಕ್ತಿ. ನಮ್ಮ ಮುಖ್ಯ ವಿಷಯಕ್ಕೆ ಅಷ್ಟಾಗಿ ಸಂಬಂಧಿಸದೆ ಇದ್ದರೂ ಈ ವಿಷಯಗಳನ್ನು ನಿಮ್ಮ ಗಮನಕ್ಕೆ ತರಬೇಕೆಂದು ಬಯಸಿದೆನು.

ಸಾಮಾನ್ಯವಾಗಿ ಅಶಿಕ್ಷಿತರು, ನಗರಗಳ ಅಪ್ರಬುದ್ಧ ಜನಜಂಗುಳಿಯಂತಿರುವ ಪ್ರತಿ ರಾಷ್ಟ್ರದ ಸ್ಥೂಲ ಮನಸ್ಸಿನವರು ಸೂಕ್ಷ್ಮ ಚಲನೆಯನ್ನು ಗ್ರಹಿಸಲಾರರು, ಅರ್ಥ ಮಾಡಿ ಕೊಳ್ಳಲಾರರು. ಈ ನಮ್ಮ ಪ್ರಪಂಚದಲ್ಲಿ ಕಾರಣಭೂತವಾಗಿರುವ ನಿಜವಾದ ಚಲನೆ ಗಳೆಲ್ಲ ತುಂಬ ಸೂಕ್ಷ್ಮವಾದುವುಗಳು. ಕಾರ್ಯವು ಮಾತ್ರ ಸ್ಥೂಲವಾಗಿರುತ್ತದೆ. ಮನಸ್ಸು ಈ ದೇಹದ ನಿಜವಾದ ಕಾರಣ - ಇದರ ಹಿಂದೆ ಇರುವ ಸೂಕ್ಷ್ಮ ಚಲನೆ. ದೇಹ ಸ್ಥೂಲವೂ ಬಾಹ್ಯವೂ ಆದುದು. ಆದರೆ ಎಲ್ಲರೂ ದೇಹವನ್ನು ನೋಡುತ್ತಾರೆಯೇ ಹೊರತು ಮನಸ್ಸನ್ನು ನೋಡುವವರು ಅಪರೂಪ. ಎಲ್ಲ ವಿಷಯದಲ್ಲಿಯೂ ಹೀಗೆಯೇ. ಪ್ರತಿ ಯೊಂದು ದೇಶದ ಪಾಶವಿಕ ಸ್ವಭಾವದ, ಅಜ್ಞಾನಿಗಳಾದ ಜನ ಸಾಧಾರಣರು ವಿಧ್ವಂಸಕ ಸೈನಿಕ ಕಾರ್ಯಾಚರಣೆಯನ್ನೂ ಅಶ್ವದಳದ ಪದಾಘಾತವನ್ನೂ ಅಸ್ತ್ರಶಸ್ತ್ರಗಳ ರುದ್ರ ನಿನಾದವನ್ನೂ ಗಮನಿಸುತ್ತಾರೆ ಅಷ್ಟೇ - ಇವು ಮಾತ್ರ ಅವರಿಗೆ ಅರ್ಥವಾಗುತ್ತವೆ. ಆದರೆ ಇವುಗಳ ಹಿಂದೆ ನಡೆಯುವ ಸೂಕ್ಷ್ಮವಾದ ಮೃದವಾದ ಕ್ರಿಯೆಯನ್ನು ತುಂಬ ಸುಸಂಸ್ಕೃತ ರಾದ ವ್ಯಕ್ತಿಗಳು, ತತ್ತ್ವ ಜ್ಞಾನಿಗಳು ಮಾತ್ರ ಅರ್ಥಮಾಡಿಕೊಳ್ಳಬಲ್ಲರು.

ಈಗ ವೇದಾಂತದ ವಿಷಯಕ್ಕೆ ಬರೋಣ. ವೈದಿಕ ಸಂಸ್ಕೃತವು ಮುಂದೆ ಸುಮಾರು ಒಂದು ಸಾವಿರ ವರ್ಷಗಳ ನಂತರ ಬರೆದ ಸಂಸ್ಕೃತ ಗ್ರಂಥಗಳ ಭಾಷೆಗಿಂತ ಬೇರೆಯಾದುದು ಎಂದು ನಾನಾಗಲೆ ನಿಮಗೆ ಹೇಳಿರುವೆ. ವೇದಗಳ ಸಂಸ್ಕೃತವು ಸರಳವೂ ಪ್ರಾಚೀನವೂ ಆದುದು; ಬಹುಶಃ ಆಗ ಅದು ಆಡುಭಾಷೆಯಾಗಿದ್ದಿರಬಹುದು. ಆದರೆ ಈಗ ನಮ್ಮೊಂದಿಗಿರುವ ಸಂಸ್ಕೃತವು, ಮೂರುಸಾವಿರ ವರ್ಷಗಳ ಹಿಂದಿನದಾದರೂ, ಆಡುಭಾಷೆಯಲ್ಲ. ವಿಚಿತ್ರವೇನೇದರೆ ಮೂರುಸಾವಿರ ವರ್ಷಗಳಿಂದಿತ್ತೀಚೆಗೆ ಬಂದ ಅಪಾರ ಸಾಹಿತ್ಯ ರಾಶಿಯೆಲ್ಲ ಮೃತ ಭಾಷೆಯಲ್ಲಿ ಬರೆಯಲ್ಪಟ್ಟಿತು. ನಾಟಕಗಳು ಮತ್ತು ಕಾದಂಬರಿ ಗಳು ಈ ಮೃತ ಭಾಷೆಯಲ್ಲಿ ಬರೆಯಲ್ಪಟ್ಟವು. ಈ ಕಾಲಾವಧಿಯಲ್ಲಿ ಅದು ಆಡು ಭಾಷೆಯಾಗಿರಲಿಲ್ಲ - ಅದು ವಿದ್ವಾಂಸ ಭಾಷೆಯಾಗಿತ್ತಷ್ಟೆ.

ಬುದ್ಧನ ಕಾಲವಾದ ಕ್ರಿ.ಪೂ. ೫೬೦ರ ವೇಳೆಗಾಗಲೇ ಸಂಸ್ಕೃತವು ಆಡುಭಾಷೆ ಯಾಗಿರಲಿಲ್ಲ. ಅವನ ಕೆಲವು ಶಿಷ್ಯರು ಸಂಸ್ಕೃತದಲ್ಲಿ ಬೋಧಿಸಬೇಕೆಂದು ಬಯಸಿದ್ದರು, ಆದರೆ ಬುದ್ಧನು ಅದನ್ನು ನಿರಾಕರಿಸಿದನು. ಜನರ ಭಾಷೆಯಲ್ಲಿ ಬೋಧಿಸಬೇಕೆಂದು ಅವನು ಬಯಸಿದನು, ಏಕೆಂದರೆ ತಾನು ಜನಾಸಾಮಾನ್ಯರ ಪ್ರವಾದಿ ಎಂದು ಅವನು ಹೇಳಿದನು. ಆದ್ದರಿಂದಲೇ ಬೌದ್ಧ ಸಾಹಿತ್ಯವು ಆಗಿನ ಕಾಲದಲ್ಲಿ ಆಡುಭಾಷೆಯಾಗಿದ್ದ ಪಾಲಿಯಲ್ಲಿರುವುದು.

ಈ ಬೃಹತ್ ವೇದಸಾಹಿತ್ಯವು ಮೂರು ಭಾಗಗಳಾಗಿ ವಿಂಗಡಗೊಂಡಿವೆ. ಮೊದಲನೆ ಯದಾದ ಸಂಹಿತಾಭಾಗವು ಪ್ರಾರ್ಥನಾ ಮಂತ್ರಗಳಿಂದ ಕೂಡಿದೆ. ಎರಡೆಯದು ಬ್ರಾಹ್ಮಣ - ಇದು ವಿವಿಧ ಬಗೆಯ ಯಜ್ಞಯಾಗಾದಿಗಳ ವಿವರಣೆಯನ್ನು ಒಳಗೊಂಡಿದೆ. ‘ಬ್ರಾಹ್ಮಣ’ ಶಬ್ದದ ಅರ್ಥ ಯಜ್ಞದ ಮೂಲಕ ಪಡೆದುಕೊಂಡದ್ದು ಎಂದು. ಮೂರನೆ ಯದು ತಾತ್ತ್ವಿಕ ಭಾಗವಾದ ಉಪನಿಷತ್ತುಗಳು. ಸಂಹಿತೆ ಮತ್ತು ಬ್ರಾಹ್ಮಣಗಳೆರಡನ್ನು ಒಟ್ಟಿಗೆ ಕರ್ಮಕಾಂಡವೆಂದು ಕರೆಯುತ್ತಾರೆ ಮತ್ತು ಉಪನಿಷತ್ತುಗಳನ್ನು ಜ್ಞಾನಕಾಂಡ ವೆಂದು ಕರೆಯುತ್ತಾರೆ. ‘ಜ್ಞಾನ’ ವೆಂಬ ಶಬ್ದವು ಆಂಗ್ಲಭಾಷೆಯ \enginline{Knowledge} ಮತ್ತು ಗ್ರೀಕ್ ಭಾಷೆಯ \enginline{gnos} ಶಬ್ದಗಳೊಡನೆ ಸಮಾನಾರ್ಥಕವಾಗಿದೆ.

ಮೊದಲನೆಯ ಸಂಹಿತಾಭಾಗವು ಅಗ್ನಿ, ಮಿತ್ರ, ಸೂರ್ಯ ಮುಂತಾದ ದೇವತೆ ಗಳನ್ನು ಕುರಿತ ಪ್ರಾರ್ಥನೆಗಳ ಸಂಗ್ರಹವಾಗಿದೆ. ಅವರನ್ನು ಸ್ತುತಿಸಿ ನೈವೇದ್ಯವನ್ನು ಅರ್ಪಿಸಲಾಗುತ್ತಿತ್ತು. ನಾನು ಇಲ್ಲಿ \enginline{‘gods’} ಎಂಬ ಶಬ್ದವನ್ನು ಉಪಯೋಗಸಿದ್ದರೂ ಅದು ಅಷ್ಟು ಸಮರ್ಪಕವಾದ ಶಬ್ದವಲ್ಲ. ದೇವತೆಗಳು ಎಂದರೆ ‘ಬೆಳಗುತ್ತಿರುವವರು’ ಎಂದು. ಭಾರತದಲ್ಲಿ ಈ ದೇವತೆಗಳ ಹೆಸರುಗಳು ಸ್ಥಾನಗಳನ್ನು ಸೂಚಿಸುತ್ತವೆಯೇ ಹೊರತು ವ್ಯಕ್ತಿಗಳನ್ನಲ್ಲ. ಉದಾಹರಣೆಗೆ ಇಂದ್ರ ಮತ್ತು ಅಗ್ನಿ ವಿಶಿಷ್ಟ ವ್ಯಕ್ತಿಗಳ ಹೆಸರಲ್ಲ, ವಿಶ್ವದಲ್ಲಿ ಕೆಲವು ಸ್ಥಾನಗಳ ಹೆಸರುಗಳು. ಭೂತಗಳ ಅಧಿಷ್ಠಾನ ದೇವತೆ, ಬ್ರಹ್ಮಾಂಡದ ಅಧಿಷ್ಠಾನ ದೇವತೆ - ಹೀಗೆ ಬೇರೆ ಬೇರೆ ದೇವತೆಗಳಿರುತ್ತಾರೆ. ನಮ್ಮ ಮತ್ತು ನಿಮ್ಮಲ್ಲಿ ಕೆಲವರು ಬಹುಶಃ ಕೆಲವು ಕಾಲ ಈ ಸ್ಥಾನಗಳಲ್ಲಿದ್ದಿರಬಹುದು. ತಾತ್ಕಾಲಿಕವಾಗಿ ಒಬ್ಬ ಜೀವನು ಈ ಸ್ಥಾನವನ್ನು ಅಲಂಕರಿಸುತ್ತಾನೆ. ಅವನ ಸರದಿ ಮುಗಿದಮೇಲೆ ಆ ಸ್ಥಾನವನ್ನು ಅವನು ತೆರವು ಮಾಡಬೇಕು. ಇನ್ನೊಬ್ಬನು ತನ್ನ ಪುಣ್ಯದ ಪ್ರಭಾವದಿಂದ ಆ ಸ್ಥಾನಕ್ಕೆ ಏರುತ್ತಾನೆ - ಉದಾ ಹರಣೆಗೆ ಅವನು ಅಗ್ನಿಯಾಗುತ್ತಾನೆ. ಸಂಸ್ಕೃತ ದೇವತಾಶಾಸ್ತ್ರವನ್ನು ಓದುವವರಿಗೆ ಈ ದೇವತೆಗಳ ಬದಲಾವಣೆ ಗೊಂದಲವನ್ನು ಉಂಟುಮಾಡುತ್ತದೆ. ಆದರೆ ಇದು ಸಿದ್ಧಾಂತ - ಈ ದೇವತೆಗಳೆಲ್ಲ ಸ್ಥಾನಗಳ ಹೆಸರುಗಳು ಮತ್ತು ಎಲ್ಲ ಜೀವಗಳೂ ಮತ್ತೆ ಮತ್ತೆ ಆ ಸ್ಥಾನ ಗಳನ್ನು ಆಕ್ರಮಿಸಬೇಕು. ಈ ಸ್ಥಾನಗಳನ್ನು ಪಡೆದ ಜೀವಗಳು ಅಂದರೆ ದೇವತೆಗಳು ಮಾನವ ಕೋಟಿಗೆ ಸಹಾಯಮಾಡಬಹುದು. ಆದ್ದರಿಂದಲೇ ಕಾಣಿಕೆಗಳನ್ನು ಮತ್ತು ಪ್ರಾರ್ಥನೆಗಳನ್ನು ಅವರಿಗೆ ಅರ್ಪಿಸಲಾಗುತ್ತದೆ. ಈ ಭಾವನೆಯನ್ನು ಆರ್ಯನರು ಹೇಗೆ ಪಡೆದರೊ ನಮಗೆ ತಿಳಿಯದು. ಆದರೆ ಋಗ್ವೇದದ ಅತ್ಯಂತ ಪುರಾತನ ಭಾಗದಲ್ಲಿಯೂ ಈ ಭಾವನೆ ಪರಿಪೂರ್ಣವಾಗಿ ವ್ಯಕ್ತವಾಗಿರುವುದನ್ನು ಕಾಣುತ್ತೇವೆ.

ಈ ದೇವತೆಗಳು, ಮಾನವರು, ಪ್ರಾಣಿಗಳು ಮತ್ತು ಬ್ರಹ್ಮಾಂಡಗಳ ಹಿಂದೆ, ಇವುಗಳ ಆಚೆ ವಿಶ್ವನಿಯಾಮಕನಾದ ಈಶ್ವರನಿರುವನು. ಇವನು ಹೊಸ ಒಡಂಬಡಿಕೆಯಲ್ಲಿ ಹೇಳಿರುವ ಸೃಷ್ಟಿ ಕರ್ತನಾದ ದೇವರಿಗೆ \enginline{(God)} ಸಮಾನನೆಂದು ಹೇಳಬಹುದು. ಈ ದೇವತೆಗಳನ್ನು ಈಶ್ವರನೆಂದು ಖಂಡಿತ ಭಾವಿಸಬಾರದು. ಆದರೆ ಆಂಗ್ಲಭಾಷೆಯಲ್ಲಿ ಇವರಿಬ್ಬರಿಗೂ ಒಂದೇ ಶಬ್ದವನ್ನು ಬಳಸಲಾಗಿದೆ. ನೀವು \enginline{God}ಶಬ್ದವನ್ನು ಏಕವಚನ ಮತ್ತು ಬಹುವಚನ ಎರಡರಲ್ಲೂ ಉಪಯೋಗಿಸುತ್ತೀರಿ. ಆದರೆ \enginline{gods}(ಬಹುವಚನ) ಎಂದರೆ ದೇವತೆಗಳು, \enginline{(God)} ಎಂದರೆ ಈಶ್ವರ ವೇದಗಳ ಅತ್ಯಂತ ಪ್ರಾಚೀನ ಭಾಗದಲ್ಲಿಯೂ ಈ ಭಾವನೆಯು ಕಂಡುಬರುತ್ತದೆ.

ಇನ್ನೊಂದು ವೈಶಿಷ್ಟ್ಯವೇನೆಂದರೆ ಈ ಈಶ್ವರನು, ಈ ದೇವರು \enginline{(God)} ವಿವಿಧ ದೇವತೆಗಳ ರೂಪದಲ್ಲಿ ಅಭಿವ್ಯಕ್ತಗೊಳ್ಳುತ್ತಿರುವನು. ಒಬ್ಬನೇ ದೇವರು ವಿವಿಧ ರೂಪಗಳಲ್ಲಿ ಅಭಿವ್ಯಕ್ತವಾಗುತ್ತಿರುವನು ಎಂಬ ಭಾವನೆ ವೇದಗಳಲ್ಲಿ ಅತ್ಯಂತ ಸಾಮಾನ್ಯವಾದುದು. ಒಂದು ಸಂದರ್ಭದಲ್ಲಿ ಏಕೇಶ್ವರವಾದವು ವೇದಗಳನ್ನು ಪ್ರವೇಶಿಸಿದ್ದುಂಟು. ಆದರೆ ಅದನ್ನು ಬಹುಬೇಗ ಕೈಬಿಡಲಾಯಿತು. ಅದನ್ನು ಹಾಗೆ ಕೈ ಬಿಟ್ಟಿದ್ದು ಒಳ್ಳೆದೇ ಆಯಿತು ಎಂಬ ನನ್ನ ಅಭಿಪ್ರಾಯವನ್ನು ನೀವು ಒಪ್ಪು ತ್ತೀರಿ ಎಂದು ಭಾವಿಸುತ್ತೇನೆ.

ಸಂಹಿತೆಯ ಪ್ರಾಚೀನ ಭಾಗಗಳಲ್ಲಿ ವಿವಿಧ ದೇವತೆಗಳ ಪೈಕಿ ಯಾವನಾದರು ಒಬ್ಬನನ್ನು ತೆಗೆದುಕೊಂಡು ಎಲ್ಲ ಗುಣವಾಚಕಗಳನ್ನೂ ಅವನ ಮೇಲೆ ಆರೋಪಿಸಿ ಅವನನ್ನೇ ಜಗತ್ತಿನ ಈಶ್ವರ ಪಟ್ಟಕ್ಕೆ ಏರಿಸುವುದನ್ನು ನೋಡುತ್ತೇವೆ. ಬೇರೆ ಬೇರೆ ಸಂದರ್ಭಗಳಲ್ಲಿ ಬೇರೆ ಬೇರೆ ದೇವತೆಗಳನ್ನು ಹೀಗೆ ಈಶ್ವರ ಪಟ್ಟಕ್ಕೆ ಏರಿಸುತ್ತಾರೆ. ಅನಂತರ ಈ ರೀತಿಯ ಹೇಳಿಕೆಗಳು ಕಾಣಿಸಿಕೊಳ್ಳುತ್ತವೆ: “ಅಗ್ನಿಯಂತೆ ಪೂಜಿಸ ಲ್ಪಡುವ ದೇವರು ನಾನೆ.” “ಸತ್ಯ ಒಂದೇ; ವಿಪ್ರರು ಅದನ್ನು ಬೇರೆ ಬೇರೆ ಹೆಸರಿನಿಂದ ಕರೆಯುತ್ತಾರೆ -” ಈ ಭಾವನೆಯೇ ಭಾರತದ ಮುಖ್ಯ ಪಲ್ಲವಿ ಎಂಬುದನ್ನು ನೀವು ನೆನಪಿನಲ್ಲಿಡಬೇಕು. ಇದೇ ಹಿಂದೂತತ್ತ್ವದ - ಅದು ಸೇಶ್ವರವಾದವಾಗಲಿ, ನಿರೀಶ್ವರ ವಾದವಾಗಲಿ, ಏಕೇಶ್ವರವಾಗಲಿ, ದ್ವೈತವಾಗಲಿ ಅಥವಾ ಅದ್ವೈತವಾಗಲಿ - ಕೇಂದ್ರ ಭಾವನೆ. ಸಾವಿರಾರು ವರ್ಷಗಳ ಹಿಂದೂ ಸಂಸ್ಕೃತಿಯಲ್ಲಿ ಈ ಭಾವನೆ ಅದರ ಜೀವಾಳ ವಾಗಿದೆ, ಆ ಜನಾಂಗಕ್ಕೆ ಇದನ್ನು ಬಿಟ್ಟುಬಿಡುವುದು ಅಸಾಧ್ಯ.

ಈ ಬೀಜವು ಹೆಮ್ಮರವಾಯಿತು. ಆದ್ದರಿಂದಲೆ ಭಾರತದಲ್ಲಿ ಹಿಂದೂ ಗಳಿಂದಂತೂ ಧಾರ್ಮಿಕ ಹಿಂಸೆ ನಡೆಯಲೇ ಇಲ್ಲ. ಜಗತ್ತಿನ ಯಾವುದೇ ಭಾಗದಿಂದ ಯಾವುದೇ ಧರ್ಮ ಬಂದರೂ ಹಿಂದೂಗಳು ಔದಾರ್ಯದಿಂದ ಅದನ್ನು ಸ್ವಾಗತಿಸಿ ಅದಕ್ಕೆ ಅಲ್ಲಿ ತಳವೂರಲು ಅವಕಾಶ ಕೊಟ್ಟರು. ಕೆಲವು ಮುಸ್ಲಿಮರು ಹಿಂದೂ ಗಳನ್ನು ಕೊಲ್ಲುವುದರಲ್ಲಿ ಅತ್ಯಾತುರರಾಗಿದ್ದರೂ, ಭಾರತೀಯ ರಾಜರು ಮಸೀದಿ ಗಳಿಗೆ ಹೋಗುತ್ತಾರೆ ಮತ್ತು ಮುಸ್ಲಿಮರ ಉತ್ಸವಗಳನ್ನೂ ಆಚರಿಸುತ್ತಾರೆ.

“ಏಕಂ ಸದ್ವಿಪ್ರಾ ಬಹುಧಾ ವದಂತಿ.”

ಧರ್ಮಗಳು ಹೇಗೆ ಬೆಳೆದವು ಎಂಬ ವಿಷಯವನ್ನೂ ಕುರಿತಂತೆ ಎರಡು ಸಿದ್ಧಾಂತ ಗಳು ಇವೆ: ಒಂದು ಆದಿವಾಸಿ ಸಿದ್ಧಾಂತ ಮತ್ತೊಂದು ಪ್ರೇತ ಸಿದ್ಧಾಂತ. ಆದಿವಾಸಿ ಸಿದ್ಧಾಂತದ ಪ್ರಕಾರ ಮಾನವಕೋಟಿಯು ಪ್ರಾರಂಭದಲ್ಲಿ ಅನೇಕ ಬುಡಕಟ್ಟು ಗಳಾಗಿ ವಿಭಾಗಗೊಂಡಿತ್ತು. ಒಂದೊಂದು ಬುಡಕಟ್ಟಿನವರೂ ತಮ್ಮದೇ ಆದ ದೇವತೆ ಯನ್ನು ಹೊಂದಿದ್ದರು, ಅಥವಾ ಒಂದೇ ದೇವತೆಯ ಅನೇಕ ರೂಪಗಳಿದ್ದವು. ಒಂದು ನಗರದ ದೇವತೆ ಇನ್ನೊಂದು ನಗರಕ್ಕೆ ಹೋಗುವುದು ಇತ್ಯಾದಿಗಳಿದ್ದವು. ಉದಾಹರಣೆಗೆ ಆ ನಗರ ಅಥವಾ ಪರ್ವತದ ಜಿಹೋವನು ಈ ನಗರ ಅಥವಾ ಪರ್ವತಕ್ಕೆ ಬಂದನು. ಬುಡಕಟ್ಟಿನ ಜನಾಂಗಗಳು ಒಟ್ಟು ಕಲೆತಮೇಲೆ ಅವರಲ್ಲಿ ಒಬ್ಬ ದೇವತೆಯು ಬಲಯುತನಾದನು.

ಯಹೊದಿಗಳ ಉದಾಹರಣೆಯನ್ನು ತೆಗೆದುಕೊಳ್ಳಿ. ಅವರು ಅನೇಕ ಜನಾಂಗ ಗಳಾಗಿ ವಿಭಾಗಗೊಂಡಿದ್ದರು. ಪ್ರತಿಯೊಂದು ಜನಾಂಗಕ್ಕೂ ಬಾಲ್ ಅಥವಾ ಮೊಲಾಕ್ ಎಂದು ಕರೆಯಲ್ಪಡುವ ತನ್ನದೇ ಆದ ದೇವತೆ ಇತ್ತು. ಆ ದೇವತೆಯನ್ನು ನಿಮ್ಮ ಹಳೆಯ ಒಡಂಬಡಿಕೆಯಲ್ಲಿ “ಪ್ರಭ” ಎಂದು ಕರೆಯಲಾಗಿದೆ. ಈ ರಾಜ್ಯದ ಮೊಲಾಕ್, ಆ ರಾಜ್ಯದ ಮೊಲಾಕ್, ಈ ಪರ್ವತದ ಮೊಲಾಕ್, ಆ ಪರ್ವತದ ಮೊಲಾಕ್ - ಹೀಗೆ ದೇವತೆಗಳ ಕಾರ್ಯವ್ಯಾಪ್ತಿ ಸೀಮಿತಗೊಂಡಿತ್ತು. ಈ ಮೊಲಾಕ್ ದೇವತೆಯನ್ನು ಪೂಜಿಸುವ ಜನಾಂಗವು ಬಲಯುತವಾಗಿ ಬೇರೆ ಜನಾಂಗಗಳನ್ನು ತನ್ನ ವಶಪಡಿಸಿ ಕೊಂಡಾಗ ಈ ಮೊಲಾಕ್ ಸರ್ವಶ್ರೇಷ್ಠ ದೇವತೆಯಾಯಿತು. ಅವನೇ ಎಲ್ಲ ಬಾಲ್ ದೇವತೆಗಳ ಮತ್ತು ಬೇರೆ ಮೊಲಾಕ್ ದೇವತೆಗಳ ಒಡೆಯನಾದನು.

ಪ್ರೇತಸಿದ್ಧಾಂತದ ಪ್ರಕಾರ ಪಿತೃಪೂಜೆಯಿಂದ ಧರ್ಮವು ಪ್ರಾರಂಭವಾಯಿತು. ಈ ಪಿತೃಪೂಜೆಯು ಈಜಿಪ್ಷಿಯನರು, ಬ್ಯಾಬಿಲೋನಿಯರು ಮತ್ತು ಇನ್ನೂ ಅನೇಕ ಜನಾಂಗಗಳಲ್ಲಿ ಪ್ರಚಲಿತವಿತ್ತು. ಯಾವುದಾದರೊಂದು ರೂಪದಲ್ಲಿ ಪಿತೃಪೂಜೆಯು ಕಾಣಿಸಿಕೊಳ್ಳದ ಯಾವ ಧರ್ಮವೂ ಇಲ್ಲ.

ಈ ದೇಹದ ಒಳಗೆ ಇನ್ನೊಂದು ಜೋಡಿ ಇದೆ ಮತ್ತು ಈ ದೇಹವು ಮೃತಪಟ್ಟ ಮೇಲೆ ಈ ಜೋಡಿಯು ಹೊರಗೆ ಬಂದು ಸ್ಥೂಲ ದೇಹವು ಅಸ್ತಿತ್ವದಲ್ಲಿರುವವರೆಗೂ ಉಳಿಯುತ್ತದೆ ಎಂದು ಜನರು ಹಿಂದೆ ನಂಬಿದ್ದರು. ಈ ಜೋಡಿಗೆ ಹಸಿವು ನೀರಡಿಕೆ ಗಳಿರುತ್ತವೆ ಮತ್ತು ಈ ಪ್ರಪಂಚದ ಒಳ್ಳೆಯ ಪದಾರ್ಥಗಳನ್ನು ಭೋಗಿಸುವ ಆಸೆ ಇರುತ್ತದೆ. ಅದು ಆಹಾರವನ್ನು ಪಡೆಯಲು ಬರುತ್ತದೆ, ಅದು ಸಿಕ್ಕದೆ ಇದ್ದರೆ ತನ್ನ ಮಕ್ಕಳನ್ನೇ ಪೀಡಿಸುತ್ತದೆ. ಎಲ್ಲಿಯವರೆಗೆ ಮೃತ ದೇಹವನ್ನು ಸುರಕ್ಷಿತವಾಗಿ ಇರಿಸಿಕೊಳ್ಳ ಲಾಗುತ್ತದೊ ಅಲ್ಲಿಯವರೆಗೆ ಜೋಡಿಯೂ ಉಳಿಯುತ್ತದೆ. ಸ್ವಾಭಾವಿಕವಾಗಿಯೇ ಮೃತದೇಹವನ್ನು ಶಾಶ್ವತವಾಗಿ ಉಳಿಸಿಕೊಳ್ಳುವ ಪ್ರಯತ್ನ ನಡೆಯಿತು.

ಬ್ಯಾಬಿಲೋನಿಯನರಲ್ಲಿ ಈ ತೆರನಾದ ಪಿತೃಪೂಜೆಯು ಪ್ರಚಲಿತವಿತ್ತು. ಮುಂದೆ ಜನಾಂಗವು ನಾಗರಿಕವಾದಂತೆಲ್ಲ ಕ್ರೂರ ರೂಪಗಳು ಅಳಿದು ಉತ್ತಮ ರೂಪಗಳು ಉಳಿದುಕೊಂಡವು. ಜೋಡಿ ಸ್ವರ್ಗದಲ್ಲಿರುತ್ತದೆಂದೂ ಇಲ್ಲಿ ಅರ್ಪಿಸಿದ ಆಹಾರವು ಅಲ್ಲಿರುವ ಜೋಡಿಗೆ ತಲಪುತ್ತದೆಂದೂ ಭಾವಿಸತೊಡಗಿದರು. ಈಗಲೂ ಕೂಡ ನಿಷ್ಠಾವಂತ ಹಿಂದೂಗಳು ವರ್ಷಕ್ಕೊಮ್ಮೆ ಪಿತೃಗಳಿಗಾಗಿ ಆಹಾರವನ್ನು ಅರ್ಪಿಸುತ್ತಾರೆ. ಹೀಗೆ ಪಿತೃ ಪೂಜೆಯು ಎಲ್ಲ ಧರ್ಮಗಳ ಮೂಲಕಾರಣವೆಂದು ಕೆಲವು ಆಧುನಿಕ ತತ್ತ್ವಜ್ಞಾನಿಗಳು ಹೇಳುತ್ತಾರೆ. ಮತ್ತೆ ಕೆಲವರು ಜನಾಂಗೀಯ ದೇವತೆಗಳೆಲ್ಲ ಒಂದು ದೇವರಲ್ಲಿ ಏಕೀಭವಿಸುವುದೇ ಎಲ್ಲ ಧರ್ಮಗಳ ಮೂಲಕರಣವೆಂದು ಹೇಳುತ್ತಾರೆ.

ಹಳೆಯ ಒಡಂಬಡಿಕೆಯ ಯಹೂದಿಗಳಲ್ಲಿ ಜೀವಾತ್ಮದ ಕಲ್ಪನೆಯೇ ಇರಲಿಲ್ಲ. ತಾಲ್ಮುಡ್ನಲ್ಲಿ ಮಾತ್ರ ಅದರ ಪ್ರಸ್ತಾಪವಿದೆ. ಅಲೆಕ್ಸಾಂಡ್ರಿಯನರಿಂದ ಈ ಭಾವನೆ ಅವರಿಗೆ ಬಂದಿತು, ಅಲೆಕ್ಸಾಂಡ್ರಿಯನರು ಹಿಂದೂಗಳಿಂದ ಅದನ್ನು ಪಡೆದರು. ಮುಂದೆ ತಾಲ್ಮುಡ್ ಪುನರ್ಜನ್ಮ ಭಾವನೆಯನ್ನು ಪ್ರಚಾರಕ್ಕೆ ತಂದಿತು. ಆದರೆ ಪುರಾತನ ಯಹೂದಿಗಳಲ್ಲಿ ಅದ್ಭುತವಾದ ಭಗವದ್ಭಾವನೆಗಳಿದ್ದವು. ಯಹೂದಿಗಳ ದೇವರು ಪರಮೇಶ್ವರನಾಗಿದ್ದ - ಅವನು ಸರ್ವಶಕ್ತ, ಸರ್ವವ್ಯಾಪಿ, ದಯಾಮಯ. ಈ ಭಗವದ್ಭಾ ವನೆ ಅವರಿಗೆ ಬಂದಿರುವುದು ಹಿಂದೂಗಳಿಂದ, ಜೀವಾತ್ಮನ ಭಾವನೆಯ ಮೂಲಕ ವಲ್ಲ. ಆದ್ದರಿಂದ ಪ್ರೇತಸಿದ್ಧಾಂತವು ಅದರಲ್ಲಿ ಯಾವ ಪಾತ್ರವನ್ನೂ ವಹಿಸಿರಲಿಕ್ಕಿಲ್ಲ, ಏಕೆಂದರೆ ಮೃತ್ಯುವಿನನಂತರವೂ ಜೀವಾತ್ಮನ ಅಸ್ತಿತ್ವವಿರುತ್ತದೆ ಎಂದು ನಂಬದವರಿಗೂ ಪ್ರೇತ ಸಿದ್ಧಾಂತಕ್ಕೂ ಯಾವ ಸಂಬಂಧವೂ ಇಲ್ಲ.

ವೇದಗಳ ಬಹು ಪ್ರಾಚೀನ ಭಾಗದಲ್ಲಿಯೂ ಪ್ರೇತಸಿದ್ಧಾಂತದ ಕಲ್ಪನೆ ಇಲ್ಲವೆಂದೇ ಹೇಳಬಹುದು. ವೇದಗಳಲ್ಲಿ ಬರುವ ಈ ದೇವತೆಗಳಿಗೂ ಪ್ರೇತ ಸಿದ್ಧಾಂತಕ್ಕೂ ಯಾವ ಸಂಬಂಧವೂ ಇಲ್ಲ. ಈ ದೇವತೆಗಳ ಹಿಂದೆ ಯಾವನೊ ಇದ್ದಾನೆ, ಇವರು ಅವನ ಅಭಿವ್ಯಕ್ತಿಗಳು ಎಂಬ ಭಾವನೆಯು ಅತ್ಯಂತ ಪ್ರಾಚೀನ ವೇದಭಾಗಗಳಲ್ಲಿಯೂ ಇದೆ.

ಇನ್ನೊಂದು ಭಾವನೆ ಯಾವುದೆಂದರೆ, ದೇಹವು ಮೃತಪಟ್ಟಮೇಲೆ ಅಮೃತ ವಾದ ಜೀವಾತ್ಮವು ಧನ್ಯತೆಯನ್ನು ಪಡೆಯುವುದು. ಅತ್ಯಂತ ಪುರಾತನ ಆರ್ಯ ಸಾಹಿತ್ಯದಲ್ಲಿಯೂ - ಅದು ಜರ್ಮನ್ ಅಥವಾ ಗ್ರೀಕ್ ಆಗಿರಲಿ - ಈ ಜೀವಾತ್ಮನ ಕಲ್ಪನೆಯಿದೆ. ಜೀವಾತ್ಮನ ಕಲ್ಪನೆ ಹಿಂದೂಗಳಿಂದ ಬಂದಿರುವುದು.

ಎರಡು ಜನಾಂಗಗಳು ಪ್ರಪಂಚದಲ್ಲಿರುವ ಎಲ್ಲ ಧರ್ಮಗಳನ್ನೂ ನೀಡಿರು ವರು - ಅವರೇ ಹಿಂದೂಗಳು ಮತ್ತು ಯಹೂದಿಗಳು. ಆದರೆ ಜೀವಾತ್ಮನ ಕಲ್ಪನೆ ಮೊದಲು ಪ್ರಾರಂಭವಾದುದು ಹಿಂದೂಗಳಲ್ಲಿ ಮತ್ತು ಬೇರೆ ಆರ್ಯ ಜನಾಂಗ ಗಳೆಲ್ಲವೂ ಅದನ್ನು ಪಡೆದವು.

ಸೆಮಿಟಿಕ್ ಜನಾಂಗಗಳು ಮತ್ತು ಈಜಿಪ್ಟಿಯನರು ಮೃತದೇಹವನ್ನು ಸುರಕ್ಷಿತ ವಾಗಿಡಲು ಪ್ರಯತ್ನಿಸುತ್ತಾರೆ ಮತ್ತು ಆರ್ಯನರು ಅದನ್ನು ನಾಶಮಾಡಲು ಪ್ರಯತ್ನಿ ಸುತ್ತಾರೆ. ನಿಮ್ಮ ಪೂರ್ವಜರಾದ ಗ್ರೀಕರು, ಜರ್ಮನರು, ರೋಮನರು ಮುಂತಾ ದವರು ಅವರು ಕ್ರೈಸ್ತರಾಗುವುದಕ್ಕೆ ಮುಂಚೆ ಮೃತದೇಹಗಳನ್ನು ಸುಡುತ್ತಿದ್ದರು. ಚಾರ್ಲ್ ಮೇನನು ಕತ್ತಿಯ ಸಹಾಯದಿಂದ ನಿಮ್ಮನ್ನು ಕ್ರೈಸ್ತರನ್ನಾಗಿ ಮಾಡಿದನು. ಇದಕ್ಕೆ ಒಪ್ಪದ ನೂರಾರು ಜನರ ತಲೆ ಕತ್ತರಿಸಿದನು, ಉಳಿದವರನೇಕರು ನೀರಿಗೆ ಧುಮುಕಿದರು. ಹೆಣ ಹೂಳುವ ಪದ್ಧತಿ ಹೀಗೆ ಜಾರಿಗೆ ಬಂದಿತು. ಹೆಣ ಹೂಳುವ ಪದ್ಧತಿ (ಹೂಳು ವುದರ ಮೂಲಕ ಮೃತದೇಹವನ್ನು ರಕ್ಷಿಸುವುದು) ಜೀವಾತ್ಮದ ಭಾವನೆ ಇಲ್ಲದೆ ದೇಹವೇ ಸರ್ವಸ್ವವಾದಾಗ ಮಾತ್ರ ಉಳಿಯಬಲ್ಲದು. ಹೆಚ್ಚೆಂದರೆ, ಈ ಮೃತ ದೇಹವೇ ಅನೇಕ ವರ್ಷಗಳ ನಂತರ ಮತ್ತೊಂದು ಜೀವನವನ್ನು ಪ್ರಾರಂಭಿಸುವುದು - ಅಂದರೆ ಮಮ್ಮಿಗಳು ಹೊರಬಂದು ಬೀದಿಯಲ್ಲಿ ಓಡಾಡತೊಡಗುವುವು - ಎಂಬ ಭಾವನೆ ಮುಂದೆ ಪ್ರಚಲಿತವಾಯಿತು.

ಆದರೆ ಆರ್ಯರು ಪ್ರಾರಂಭದಿಂದಲೂ ಜೀವಾತ್ಮವು ದೇಹವಲ್ಲ, ಅದು ದೇಹ ಮೃತವಾದಮೇಲೂ ಉಳಿಯುತ್ತದೆ ಎಂದು ಭಾವಿಸುತ್ತಿದ್ದರು. ದೇಹ ದಹನವಾದ ಮೇಲೆ ಹೇಳಲ್ಪಡುವ ಈ ಮಂತ್ರ ಪ್ರಾಚೀನ ಋಗ್ವೇದದಲ್ಲಿ ಬರುತ್ತದೆ: “ಅವನನ್ನು ಮೃದುವಾಗಿ ತೆಗೆದುಕೊಂಡುಹೋಗು, ಅವನನ್ನು ಶುದ್ಧಗೊಳಿಸು, ಅವನಿಗೆ ದಿವ್ಯ ಶರೀರವನ್ನು ನೀಡು, ದುಃಖವಿಲ್ಲದ, ಶಾಶ್ವತ ಆನಂದದಿಂದ ಕೂಡಿದ ಪಿತೃಲೋಕಕ್ಕೆ ಅವನನ್ನು ಕರೆದುಕೊಂಡು ಹೋಗು.”\footnote{1. ಋಗ್ವೇದ, ೧೦.೧೬.೪}

ಆಧುನಿಕ ಕಾಲದಲ್ಲಿ ಅನೇಕ ವಿಕೃತ ಮತ್ತು ಕ್ರೂರ ಧಾರ್ಮಿಕ ಮತಗಳು ಭಾರತವನ್ನು ಪ್ರವೇಶಿಸಿದ್ದರೂ, ಒಂದು ಅಂಶವಂತೂ ಆರ್ಯನರನ್ನು ಪ್ರಪಂಚದ ಬೇರೆ ಎಲ್ಲ ಜನಾಂಗಗಳಿಂದ ಬೇರ್ಪಡಿಸುತ್ತದೆ: ಅದಾವುದೆಂದರೆ ಅವರ ಹಿಂದೂ ಧರ್ಮವು ಇಂದ್ರನನ್ನು ಅತ್ಯಂತ ಸತ್ಯ ಎಂದು ಒಪ್ಪಿಕೊಳ್ಳುವುದು. ವೇದಗಳ ಮುಕ್ಕಾಲು ಪಾಲು ಪುರಾಣವು ಗ್ರೀಕರದಂತೆಯೇ ಇದೆ. ಆದರೆ ಹಳೆಯ ದೇವತೆಗಳು ಹೊಸ ಧರ್ಮದಲ್ಲಿ ಸಂತರಾಗುತ್ತಾರೆ. ಆದರೆ ಅವರು ಮೂಲತಃ ಸಂಹಿತೆಯ ದೇವತೆಗಳು.

ಇನ್ನೊಂದು ವೈಶಿಷ್ಟ್ಯವೇನೆಂದರೆ ಅವರ ಧರ್ಮವು ಉತ್ಸಾಹಪೂರ್ಣವೂ ಆನಂದ ಪೂರ್ಣವೂ ಆದುದು, ಕೆಲವೊಮ್ಮೆ ಹರ್ಷಾತಿರೇಕದಿಂದ ಕೂಡಿರುತ್ತದೆ. ಅದರಲ್ಲಿ ನೈರಾಶ್ಯವೆಂಬುದು ಇಲ್ಲವೇ ಇಲ್ಲ. ಭೂಮಿ ಸುಂದರವಾಗಿದೆ, ಸ್ವರ್ಗಗಳು ಸುಂದರವಾಗಿವೆ, ಜೀವನ ಶಾಶ್ವತವಾದುದು. ಸತ್ತನಂತರವೂ ಈ ದೇಹದ ಅಪರಿಪೂರ್ಣತೆಗಳಿಲ್ಲದ ಹೆಚ್ಚು ಸುಂದರವಾದ ದೇಹವನ್ನು ಅವರು ಪಡೆಯುತ್ತಾರೆ ಮತ್ತು ಸ್ವರ್ಗದಲ್ಲಿ ದೇವತೆಗಳೊಡನೆ ಶಾಶ್ವತ ಆನಂದದಿಂದ ಇರುತ್ತಾರೆ.

ಆದರೆ ಸೆಮಿಟಿಕ್ ಜನಾಂಗಗಳಲ್ಲಿ ಧರ್ಮದ ಪ್ರಥಮ ಅಂಕುರವೇ ಬೀಭತ್ಸ ವಾದುದು. ಮನುಷ್ಯನು ತನ್ನ ಬಡಕುಟೀರದಲ್ಲಿ ಭಯದಿಂದ ಮುದುಡಿಕೊಂಡಿ ರುತ್ತಾನೆ. ಅವನ ಮನೆಯು ಜೋಡಿಗಳಿಂದ ಸುತ್ತುವರಿದಿರುತ್ತದೆ. ಯಹೂದಿಗಳ ಪೂರ್ವಿಕ ಪಿತೃಗಳು ತಮಗೆ ರಕ್ತಯಜ್ಞ ಮಾಡದಿದ್ದರೆ ಯಾರ ಮೇಲಾದರೂ ಬಿದ್ದು ಅವನನ್ನು ಸೀಳಿಹಾಕುವುದಕ್ಕೆ ಹವಣಿಸುತ್ತಿರುತ್ತಾರೆ. ಈ ಜೋಡಿಗಳ ಐಕ್ಯತೆಯ ಭಾವನೆಯು ಬಂದಮೇಲೂ ಕೂಡ ಈ ಯಜ್ಞದ ಭಾವನೆ ಉಳಿದುಕೊಂಡಿತು.

ಭಾರತದಲ್ಲಿ ಈ ಯಜ್ಞದ ಭಾವನೆ ಪ್ರಾರಂಭದಲ್ಲಿ ಇರಲಿಲ್ಲ. ಆದರೆ ಮುಂದೆ ಬ್ರಾಹ್ಮಣ ಭಾಗದಲ್ಲಿ ಈ ಭಾವನೆ ಪ್ರಚಲಿತವಾಯಿತು. ಮೂಲತಃ ಯಜ್ಞವೆಂದರೆ ದೇವತೆಗಳಿಗೆ ಆಹಾರವನ್ನು ನೀಡುವುದು ಎಂದು. ಆದರೆ ಇದು ಕ್ರಮೇಣ ಮೇಲ್ಮಟ್ಟಕ್ಕೆ ಏರಿಸಲ್ಪಟ್ಟು ಭಗವಂತನಿಗಾಗಿ ಮಾಡುವ ಯಜ್ಞವಾಗಿ ಮಾರ್ಪಟ್ಟಿತು. ತತ್ತ್ವಶಾಸ್ತ್ರವು ಮುಂದೆ ಬಂದು ಅದನ್ನು ಇನ್ನೂ ರಹಸ್ಯಗೊಳಿಸಿತು, ಅದರ ಸುತ್ತ ತಾರ್ಕಿಕ ಬಲೆಯನ್ನು ಹಣೆಯಿತು. ಕ್ರಮೇಣ ರಕ್ತಯಜ್ಞವು ಜಾರಿಗೆ ಬಂತು. ಮುನ್ನೂರು ಎತ್ತುಗಳನ್ನು ಬಲಿಕೊಡಲಾಯಿತೆಂದೂ ದೇವತೆಗಳು ಯಜ್ಞವನ್ನು ಆಘ್ರಾಣಿಸುತ್ತ ಆನಂದಿಸುತ್ತಿರು ವರೆಂದೂ ನಾವು ಕೆಲವು ಕಡೆ ಓದುತ್ತೇವೆ. ಅನಂತರ ಅನೇಕ ರಹಸ್ಯ ಭಾವನೆಗಳು ಇದ ರೊಡನೆ ಸೇರಿಕೊಂಡವು. ಯಜ್ಞವನ್ನು ಹೇಗೆ ಮಾಡಬೇಕು. ಯಜ್ಞವೇದಿಕೆಯನ್ನು ಹೇಗೆ ನಿರ್ಮಿಸಬೇಕು, ಅದು ತ್ರಿಕೋನಾಕಾರವಾಗಿರಬೇಕೆ ಅಥವಾ ಚೌಕಾಕಾರ ನಾಗಿರಬೇಕೆ - ಹೀಗೆ ಅನೇಕಾನೇಕ ವಿಚಾರಗಳು ಸೇರಿಕೊಂಡು ಅತ್ಯಂತ ಸಂಕೀರ್ಣ ಕ್ರಿಯೆಯಾಗಿ ಮಾರ್ಪಟ್ಟಿತು. ಇದರ ಒಂದು ಬಹುಮುಖ್ಯ ಪ್ರಯೋಜನವೇ ಜ್ಯಾಮಿತಿಯು ಅಸ್ತಿತ್ವಕ್ಕೆ ಬಂದುದು. ವಿವಿಧಾಕಾರದ ಯಜ್ಞವೇದಿಕೆಗಳನ್ನು ರಚಿಸುವ ವಿಸ್ತಾರವಾದ ವಿವರಣೆಗಳು ದೊರೆಯುತ್ತವೆ. ಎಷ್ಟು ಇಟ್ಟಿಗೆಗಳನ್ನು ಉಪಯೋಗಿಸ ಬೇಕು, ಅವುಗಳನ್ನು ಹೇಗೆ ಜೋಡಿಸಬೇಕು. ಅವು ಎಷ್ಟು ಗಾತ್ರದವಾಗಿರಬೇಕು - ಈ ಎಲ್ಲ ಕಟ್ಟುನಿಟ್ಟಾದ ನಿಯಮಗಳನ್ನು ರೂಪಿಸಿದರು. ಸ್ವಾಭಾವಿಕವಾಗಿಯೇ ಇವೆಲ್ಲ ದರ ಪರಿಣಾಮವಾಗಿ ಜ್ಯಾಮಿತಿಯು ಬಳಕೆಗೆ ಬಂತು. ಈಜಿಪ್ಟಿಯನರು ನೀರಾ ವರಿಯ ಮೂಲಕ - ನೈಲ್ನದಿಯ ನೀರನ್ನು ಕಾಲುವೆಗಳ ಸಹಾಯದಿಂದ ತಮ್ಮ ಹೊಲಗಳಿಗೆ ಹಾಯಿಸುವ ಮೂಲಕವೂ ಹಿಂದೂಗಳು ಯಜ್ಞವೇದಿಗಳ ಮೂಲಕವೂ ಈ ಜ್ಯಾಮಿತಿಯನ್ನು ಬಳಕೆಗೆ ತಂದರು.

ಭಾರತದ ಮತ್ತು ಯಹೂದಿಗಳ ಯಜ್ಞದ ಕಲ್ಪನೆಯಲ್ಲಿ ಇನ್ನೊಂದು ವ್ಯತ್ಯಾಸವಿದೆ. ಯಜ್ಞದ ನಿಜವಾದ ಅರ್ಥ ಪೂಜೆ - ನೈವೇದ್ಯದ ಮೂಲಕ ಪೂಜಿಸುವುದು ಎಂದು ಪ್ರಾರಂಭದಲ್ಲಿ ಅದು ದೇವತೆಗಳಿಗೆ ಆಹಾರವನ್ನು ಅರ್ಪಿಸುವ ಒಂದು ವಿಧಾನವಾಗಿತ್ತು. ನಮ್ಮಂತೆಯೇ ಅವರೂ ಸ್ಥೂಲ ಆಹಾರವನ್ನು ಸ್ವೀಕರಿಸುತ್ತಿದ್ದರು. ಅನಂತರ ತತ್ತ್ವ ಶಾಸ್ತ್ರವು ತಲೆಹಾಕಿ ಉನ್ನತ ಸ್ತರದಲ್ಲಿರುವ ದೇವತೆಗಳು ನಮ್ಮಂತೆ ಆಹಾರವನ್ನು ಸ್ವೀಕರಿಸ ಲಾರರು ಎಂಬ ಭಾವನೆ ಬೆಳೆಯಿತು. ಅವರ ದೇಹಗಳು ಸೂಕ್ಷ್ಮ ಕಣಗಳಿಂದ ಆಗಿರು ತ್ತವೆ. ನಮ್ಮ ದೇಹಗಳು ಗೋಡೆಯ ಮೂಲಕ ತೂರಿಹೋಗಲಾರವು. ಆದರೆ ಅವರ ದೇಹಗಳಿಗೆ ಸ್ಥೂಲವಸ್ತುಗಳು ಅಡೆತಡೆಯಾಗಲಾರದು. ಆದ್ದರಿಂದ ಅವರೂ ಕೂಡ ನಮ್ಮಂತೆಯೇ ಸ್ಥೂಲರೂಪದಲ್ಲಿ ಆಹಾರವನ್ನು ಸ್ವೀಕರಿಸುತ್ತಾರೆ ಎಂದು ನಿರೀಕ್ಷಿಸಲಾಗದು.

(ಈ ಉಪನ್ಯಾಸದ ಉಳಿದ ಭಾಗದ ಮೂಲಪ್ರತಿಯು ತುಂಬ ಕೆಟ್ಟುಹೋಗಿದ್ದುದ ರಿಂದ ಓದಲು ಸಾಧ್ಯವಾದುದನ್ನು ಮಾತ್ರ ಕೆಳಗೆ ಕೊಡಲಾಗಿದೆ.)

“ಹೇ ಇಂದ್ರ, ನಾನು ನಿನಗೆ ಈ ಆಹುತಿಯನ್ನು ಅರ್ಪಿಸುತ್ತೇನೆ. ಹೇ ಅಗ್ನಿ, ನಾನು ನಿನಗೆ ಈ ಆಹುತಿಯನ್ನು ಅರ್ಪಿಸುತ್ತೇನೆ.” ಸಂಸ್ಕೃತದಲ್ಲಿ ಈ ಶಬ್ದಗಳಿಗೆ ರಹಸ್ಯ ಶಕ್ತಿಗಳಿವೆ. ಮನುಷ್ಯನು ಒಂದು ವಿಶಿಷ್ಟ ಮಾನಸಿಕ ಸ್ಥಿತಿಯಲ್ಲಿ ಈ ಶಬ್ದಗಳನ್ನು ಉಚ್ಚರಿಸಿ ದಾಗ ಒಂದು ಶಕ್ತಿಯು ಉತ್ಪನ್ನವಾಗಿ ಕೆಲವು ಪರಿಣಾಮಗಳನ್ನು ಉಂಟುಮಾಡುತ್ತದೆ. ಇದೇ ಆಲೋಚನೆಯ ವಿಕಾಸ.

ಉದಾಹರಣೆಗೆ, ಮಕ್ಕಳಿಲ್ಲದ ಒಬ್ಬ ವ್ಯಕ್ತಿಯು ಪ್ರಜಾಕಾಮನಾಗಿ ಇಂದ್ರನನ್ನು ಪೂಜಿಸುತ್ತಾನೆ. ಅವನಿಗೆ ಮಗುವಾದರೆ ಅದು ಇಂದ್ರನಿಂದ ಪ್ರಾಪ್ತವಾದುದೆಂದು ಹೇಳುತ್ತಾನೆ. ಮುಂದೆ ಇಂದ್ರ ಎಂಬುವವನಿಲ್ಲ ಎಂದು ಹೇಳಿದರು. ಹಾಗಾದರೆ ಆ ವ್ಯಕ್ತಿಗೆ ಮಗುವನ್ನು ನೀಡಿದವರು ಯಾರು? ಇದೆಲ್ಲವೂ ಕಾರ್ಯ - ಕಾರಣಕ್ಕೆ ಸಂಬಂಧ ಪಟ್ಟ ವಿಷಯ....

.... ದೇವತೆಗಳಿಗೆ ಆಹಾರವನ್ನು ನೀಡುವುದಲ್ಲ, ಆದರೆ ನನ್ನ ಪಾಪವನ್ನು ಇನ್ನೊಬ್ಬರ ತಲೆಯ ಮೇಲೆ ಹೊರಿಸುವುದು ಎಂದು ಅವರು ಹೇಳಿದರು. “ನನ್ನ ಅಪರಾಧಗಳು ಆಡನ್ನು ಸೇರುತ್ತದೆ. ಆಡನ್ನು ಬಲಿಕೊಟ್ಟಾಗ ನನ್ನ ಅಪರಾಧಗಳು ಕ್ಷಮಿಸಲ್ಪಡುತ್ತವೆ.” ಯಹೂದಿಗಳ ಈ ಯಜ್ಞದ ಭಾವನೆ ಭಾರತವನ್ನು ಪ್ರವೇಶಿಸಲೇ ಇಲ್ಲ. ಇದರಿಂದಾಗಿ ನಾವು ಬಹುಶಃ ಅನೇಕ ಕಷ್ಟ ಸಂಕಟಗಳಿಂದ ಪಾರಾದೆವು.

ಮಾನವ ಸ್ವಭಾವವೇ ಸ್ವಾರ್ಥಪರವಾದುದು ಮತ್ತು ಬಹುಮಂದಿ ಸ್ತ್ರೀ ಪುರುಷರು ದುರ್ಬಲರು. ಪಾಪವರ್ಗಾವಣೆಯ ಭಾವನೆ ಅವರನ್ನು ಇನ್ನೂ ದುರ್ಬಲರನ್ನಾಗಿ ಮಾಡುತ್ತದೆ. ಒಂದು ಮಗುವು ಸಂಪೂರ್ಣ ಕೈಲಾಗದ ವ್ಯಕ್ತಿಯಾಗುವವರೆಗೂ ‘ನೀನು ಕೆಲಸಕ್ಕೆ ಬಾರದವನೆಂದು’ ಅದಕ್ಕೆ ಬೋಧಿಸಲಾಗುತ್ತದೆ. ಅವನು ತನ್ನನ್ನು ಅವಲಂಬಿಸುವು ದನ್ನು ಬಿಟ್ಟು ಯಾರಿಗಾದರೂ ಹೋಗಿ ಅಂಟಿಕೊಳ್ಳಲು ಪ್ರಯತ್ನಿಸುತ್ತಾನೆ.

