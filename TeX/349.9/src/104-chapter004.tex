
\chapter{ಮುಂಡಕ ಉಪನಿಷತ್}

(೧೮೯೬, ಜನವರಿ ೨೯ರಂದು ನ್ಯೂಯಾರ್ಕ್ನಲ್ಲಿ ಜ್ಞಾನಯೋಗದ ಮೇಲೆ ಮಾಡಿದ ಉಪನ್ಯಾಸ. ಜೆ.ಜೆ. ಗುಡ್ವಿನ್ ಅವರು ಬರೆದುಕೊಂಡುದು.)

ಜ್ಞಾನಯೋಗದ ಮೇಲಿನ ಹಿಂದಿನ ಉಪನ್ಯಾಸದಲ್ಲಿ ಒಂದು ಉಪನಿಷತ್ತಿನ ಬಗ್ಗೆ ತಿಳಿದು ಕೊಂಡೆವು, ಈಗ ಇನ್ನೊಂದು (ಮುಂಡಕ) ಉಪನಿಷತ್ತಿನ ಬಗ್ಗೆ ತಿಳಿದು ಕೊಳ್ಳೋಣ. ಬ್ರಹ್ಮನು ದೇವತೆಗಳಲ್ಲೆಲ್ಲ ಮೊದಲನೆಯವನು, ಈ ಕಲ್ಪದ ಒಡೆಯನು ಮತ್ತು ಅದರ ರಕ್ಷಕನು. ಎಲ್ಲ ಜ್ಞಾನಗಳ ಸಾರವಾದ ಬ್ರಹ್ಮಜ್ಞಾನವನ್ನು ತನ್ನ ಮಗನಾದ ಅಥರ್ವಣನಿಗೆ ನೀಡಿದನು. ಅವನು ತನ್ನ ಮಗನಾದ ಅಂಗಿರಸನಿಗೂ, ಅಂಗಿರಸನು ಭರದ್ವಾಜನಿಗೂ ನೀಡಿದರು. ಹೀಗೆ ಇದರ ಪರಂಪರೆ ಮುಂದುವರಿಯಿತು.

ಶೌನಕ ಎಂಬ ಶ‍್ರೀಮಂತನು ಶಿಷ್ಯನಾಗಿ ಅಂಗಿರಸನೆಡೆಗೆ ಹೋಗಿ, “ಯಾವುದನ್ನು ತಿಳಿದುಕೊಂಡರೆ ಎಲ್ಲವನ್ನೂ ತಿಳಿದುಕೊಂಡಂತಾಗುವುದೊ ಅದನ್ನು ನನಗೆ ಬೋಧಿಸಿ” ಎಂದು ಕೇಳಿಕೊಂಡನು.

ಒಂದು ಪರಾ ವಿದ್ಯೆ, ಇನ್ನೊಂದು ಅಪರಾ ವಿದ್ಯೆ. ಋಗ್ವೇದವು ವೇದಗಳಲ್ಲಿ ಒಂದು ಭಾಗ. ಶಿಕ್ಷಾ ಎನ್ನುವುದು ಇನ್ನೊಂದು ಭಾಗ. ವಿವಿಧ ಶಾಸ್ತ್ರಗಳೆಲ್ಲವೂ ಅಪರಾ ವಿದ್ಯೆ. ಯಾವುದು ಪರಾ ವಿದ್ಯೆ? ಯಾವುದರಿಂದ ಅಕ್ಷರವಾದುದನ್ನು ಅರಿಯು ತ್ತೇವೆಯೊ ಅದು ಪರಾ ವಿದ್ಯೆ. ಆದರೆ ಆ ಅಕ್ಷರವು ಅದೃಶ್ಯವಾದುದು, ಅಗ್ರಾಹ್ಯವಾದುದು, ನಿರ್ದೇಶಿಸಲಾಗದುದು. ಅದು ವರ್ಣರಹಿತವಾದುದು, ಕಣ್ಣು, ಕಿವಿ, ಮೂಗು, ಪಾಣಿ, ಪಾದಗಳಿಲ್ಲದುದು. ಅದು ನಿತ್ಯವೂ ವಿಭುವೂ ಸರ್ವಗತವೂ ಆದುದು. ಅದರಿಂದಲೇ ಎಲ್ಲವೂ ಹುಟ್ಟಿಬಂದಿರುವುದು. ಋಷಿಗಳು ಅದನ್ನು ಅನುಭವಿಸುತ್ತಾರೆ, ಅದೇ ಪರಾ ವಿದ್ಯೆ.

ಜೇಡರ ಹುಳವು ಬಲೆಯನ್ನು ತನ್ನ ದೇಹದಿಂದಲೇ ಹೊರತೆಗೆದು ಮತ್ತೆ ಒಳಕ್ಕೆ ಎಳೆದುಕೊಳ್ಳುವಂತೆ, ಸಸ್ಯಗಳು ತಮ್ಮ ಸ್ವಭಾವದಿಂದಲೇ ಬೆಳೆಯುವಂತೆ ಈ ಜಗತ್ತು ಆ ಅಕ್ಷರದಿಂದ ಬಂದಿದೆ. ಹೃದಯವು ಮನುಷ್ಯ ದೇಹದ ಬೇರೆ ಭಾಗದಿಂದ ಬೇರೆ ಯಾಗಿ ಕಂಡರೂ, ಸಸ್ಯಗಳು ಭೂಮಿಯಿಂದ ಬೇರೆಯಾಗಿ ಕಂಡರೂ, ಬಲೆಯು ಜೇಡರಹುಳದಿಂದ ಬೇರೆಯಾಗಿ ಕಂಡರೂ ಅವಾವುದೂ ನಿಜವಾಗಿಯೂ ಬೇರೆಯಾ ಗಿರದೆ, ಕಾರಣದಲ್ಲಿಯೇ ಕಾರ್ಯವು ನೆಲೆಸಿರುವಂತೆ, ಈ ಜಗತ್ತೂ ಆ ಅಕ್ಷರದಿಂದ ಬೇರೆಯಲ್ಲ.

ಆ ಬ್ರಹ್ಮನಲ್ಲಿ ಮೊದಲು ಸೃಷ್ಟಿಸಂಕಲ್ಪ ಉಂಟಾಗುತ್ತದೆ, ಮತ್ತು ಆ ಸಂಕಲ್ಪದಿಂದ ಸೃಷ್ಟಿಕರ್ತನು (ಹಿರಣ್ಯಗರ್ಭ) ಆವಿರ್ಭವಿಸುತ್ತಾನೆ. ಅವನಿಂದ ಮಹತ್ ಉಂಟಾಗುತ್ತದೆ ಮತ್ತು ಅದರಿಂದ ಭೂತಗಳು ಮತ್ತು ವಿವಿಧ ವಿಶ್ವಗಳು ಉಂಟಾಗುತ್ತವೆ.

ಇದು ಸತ್ಯ - ಮುಕ್ತಿಯನ್ನು ಬಯಸುವವರಿಗಾಗಿ ಮತ್ತು ಬೇರೆ ಸುಖವನ್ನು ಬಯಸುವವರಿಗಾಗಿ ವೇದಗಳು ವಿವಿಧ ಮಾರ್ಗಗಳನ್ನು ತಿಳಿಸುತ್ತವೆ. ಮುಂಡಕ ಉಪ ನಿಷತ್ತು ಈ ಮಾರ್ಗಗಳನ್ನು ವಿವರಿಸುತ್ತದೆ. ಸ್ವರ್ಗವನ್ನು ಬಯಸುವವರು ಮರಣಾ ನಂತರ ಅರ್ಚಿರಾದಿ ಮಾರ್ಗದ ಮೂಲಕ ಸ್ವರ್ಗಕ್ಕೆ ಹೋಗಿ ಅಲ್ಲಿ ಬಹುಕಾಲ ಸುಖವನ್ನು ಅನುಭವಿಸುತ್ತಾರೆ. ಆದರೆ ಅವರು ಪುನಃ ಭೂಮಿಗೆ ಹಿಂತಿರುಗಬೇಕಾಗುತ್ತದೆ.

ಇಷ್ಟಾ ಮತ್ತು ಪೂರ್ತ ಎಂಬ ಎರಡು ಶಬ್ದಗಳಿವೆ. ಯಾಗ ಯಜ್ಞಾದಿಗಳು ಇಷ್ಟಾ, ರಸ್ತೆ ಮಾಡಿಸುವುದು, ಆಸ್ಪತ್ರೆ ಕಟ್ಟಿಸುವುದು ಮುಂತಾದ ಸಮಾಜ ಹಿತಕಾರ್ಯಗಳು ಪೂರ್ತ. ಮೂರ್ಖರು ಇವಿಷ್ಟೇ ಶ್ರೇಷ್ಠವೆಂದೂ ಇವಕ್ಕಿಂತ ಶ್ರೇಷ್ಠವಾದುದು ಇನ್ನೇನೂ ಇಲ್ಲವೆಂದೂ ಭಾವಿಸುತ್ತಾರೆ. ಅವರು ಬಯಸಿದುದನ್ನು ಅವರು ಪಡೆಯುತ್ತಾರೆ. ಅವರು ಸ್ವರ್ಗಕ್ಕೆ ಹೋಗುತ್ತಾರೆ. ಆದರೆ ಪ್ರತಿಯೊಂದು ಸುಖಕ್ಕೂ ಮತ್ತು ದುಃಖಕ್ಕೂ ಒಂದು ಅಂತ್ಯವಿದ್ದೇ ಇದೆ. ಸ್ವರ್ಗಸುಖ ಮುಗಿದ ಮೇಲೆ ಕೆಳಗಿಳಿದು ಮನುಷ್ಯರೊ ಪ್ರಾಣಿಗಳೊ ಆಗಿ ಹುಟ್ಟುತ್ತಾರೆ. ಸಂಸಾರವನ್ನು ತ್ಯಜಿಸಿದವರು ಮತ್ತು ಇಂದ್ರಿಯ ನಿಗ್ರಹ ವುಳ್ಳವರು ಕಾಡಿನಲ್ಲಿ ವಾಸಿಸುತ್ತಾರೆ. ಅವರು ಮರಣದನಂತರ ಬ್ರಹ್ಮ ಲೋಕವನ್ನೂ ಸೇರುತ್ತಾರೆ.

ಶುಭಾಶುಭ ಕರ್ಮಗಳನ್ನೆಲ್ಲ ಪರಿಕ್ಷಿಸಿ ಋಷಿಯು ಎಲ್ಲ ಕರ್ತವ್ಯಗಳನ್ನು ಬಿಟ್ಟು, ಯಾವುದನ್ನು ಪಡೆದರೆ ಹಿಂತಿರುಗುವುದಿಲ್ಲವೊ, ಬದಲಾವಣೆ ಎಂಬುದಿಲ್ಲವೊ ಆ ಸತ್ಯವನ್ನು ಅರಿಯ ಬಯಸುತ್ತಾನೆ. ಅದನ್ನು ಅರಿಯಲು ಸಮಿತ್ಪಾಣಿಯಾಗಿ ಗುರುವಿನ ಸಮೀಪಕ್ಕೆ ಹೋಗುತ್ತಾನೆ. ಗುರುವು ಬೇರಾವ ಕಾಣಿಕೆಗಳನ್ನು ಸ್ವೀಕರಿಸುವುದಿಲ್ಲ. ಹೋಮಕ್ಕೆ ಬೇಕಾದ ಸಮಿತ್ತನ್ನು ಮಾತ್ರ ಅವನು ಬಯಸುತ್ತಾನೆ.

ಗುರು ಯಾರು? ಅವನು ಶ್ರೋತ್ರಿಯ, ಎಂದರೆ ಶಾಸ್ತ್ರಸಾರವನ್ನು ಅರಿತಿರಬೇಕು; ಅವನು ಬ್ರಹ್ಮನಿಷ್ಠ ಎಂದರೆ ಅವನಾತ್ಮವು ಪರಬ್ರಹ್ಮನಲ್ಲಿ ನೆಲೆಸಿರಬೇಕು; ಅವನು ಅಕಾಮಹತ ಎಂದರೆ ಎಲ್ಲ ಆಸೆಗಳಿಂದಲೂ ಮುಕ್ತನಾಗಿರಬೇಕು.

ಮನೊನಿಗ್ರಹವುಳ್ಳ, ಶಾಂತನೂ ದಾಂತನೂ ಆದ, ಪುತ್ರೇಷಣ ವಿತ್ತೇಷಣ ಲೋಕೇಷಣಗಳಿಂದ ಮುಕ್ತನಾದ ಶಿಷ್ಯನಿಗೆ ಗುರುವು ಬ್ರಹ್ಮಜ್ಞಾನವನ್ನು ಬೋಧಿಸುತ್ತಾನೆ.

ಅಂಗಿರಸನು ಶಿಷ್ಯನಿಗೆ ಹೀಗೆ ಉಪದೇಶಿಸುತ್ತಾನೆ:

“ಇದು ಸತ್ಯ. ಪ್ರಜ್ವಲಿಸುತ್ತಿರುವ ಅಗ್ನಿಯಿಂದ ಅಗ್ನಿರೂಪವಾದ ಸಾವಿರಾರು ಕಿಡಿಗಳು ಹೇಗೆ ಹುಟ್ಟು ತ್ತವೆಯೊ ಹಾಗೆಯೇ, ಎಲೈ ಸೌಮ್ಯನೆ, ಅಕ್ಷರದಿಂದ ಈ ಎಲ್ಲ ರೂಪಗಳು, ಈ ಎಲ್ಲ ಭಾವನೆಗಳು, ಎಲ್ಲ ಸೃಷ್ಟಿಯೂ ಹೊರಬಂದಿವೆ, ಮತ್ತು ಅದ ರಲ್ಲಿಯೇ ಲೀನವಾಗುತ್ತವೆ.

“ಆ ಪುರುಷನು ಮಾತ್ರ ನಿತ್ಯನು, ರೂಪರಹಿತನು, ಅನಾದಿಯು, ಎಲ್ಲದರ ಒಳಗೂ ಹೊರಗೂ ಇರುವವನು. ಅವನು ಪ್ರಾಣ ಮನಸ್ಸುಗಳಾಚೆ ಇರುವನು. ಶುದ್ಧನಾದ ಅವನು ಅಕ್ಷರಕ್ಕಿಂತಲೂ ಶ್ರೇಷ್ಠನು.

“ಪ್ರಾಣ, ಮನಸ್ಸು, ಎಲ್ಲ ಇಂದ್ರಿಯಗಳು, ಆಕಾಶ, ವಾಯು, ಅಗ್ನಿ, ನೀರು, ಎಲ್ಲ ವನ್ನೂ ಧರಿಸಿರುವ ಪೃಥ್ವಿ - ಇವೆಲ್ಲವೂ ಈ ಪುರುಷನಿಂದಲೇ ಹುಟ್ಟುತ್ತವೆ. ಸ್ವರ್ಗಗಳು ಅವನ ತಲೆ, ಸೂರ್ಯಚಂದ್ರರು ಅವನ ಕಣ್ಣುಗಳು, ದಿಕ್ಕುಗಳು ಅವನ ಕಿವಿ, ವೇದಗಳು ಅವನ ವಾಕ್ಕು. ವಾಯುವೇ ಅವನ ಪ್ರಾಣ, ಈ ವಿಶ್ವವೇ ಅವನ ಹೃದಯ, ಈ ಜಗತ್ತೇ ಅವನ ಪಾದ, ಸರ್ವಭೂತಗಳ ಅಂತರಾತ್ಮನೂ ಅವನೇ.

“ವೇದಗಳು ಅವನಿಂದಲೇ ಬಂದಿರುವುವು. ಸಾಧ್ಯರು ಅವನಿಂದಲೇ ಹುಟ್ಟಿರು ವರು. ಈ ಸಾಧ್ಯರು ಸಾಮಾನ್ಯ ಮಾನವರಿಗಿಂತ ಅತ್ಯಂತ ಶ್ರೇಷ್ಠರು ಮತ್ತು ದೇವತೆಗಳಿಗೆ ಹೆಚ್ಚು ಹೋಲುತ್ತಾರೆ. ಅವನಿಂದಲೇ ಎಲ್ಲ ಮನುಷ್ಯರೂ ಪ್ರಾಣಿಗಳೂ ಹುಟ್ಟಿರುತ್ತಾರೆ. ಮಾನಸಿಕ ಶಕ್ತಿಗಳು, ತಪಸ್ಸು, ಸತ್ಯ, ಬ್ರಹ್ಮಚರ್ಯ ಎಲ್ಲವೂ ಅವನಿಂದಲೇ ಹುಟ್ಟಿರುವುವು.

“ಏಳು ಇಂದ್ರಿಯಗಳು ಅವನಿಂದಲೇ ಹುಟ್ಟು ತ್ತವೆ. ಏಳು ಇಂದ್ರಿಯ ವಿಷಯ ಗಳೂ ಅವನಿಂದಲೇ. ಏಳು ಇಂದ್ರಿಯ ಕ್ರಿಯೆಗಳು ಕೂಡ ಅವನಿಂದಲೇ. ಪ್ರಾಣಗಳು ಸಂಚರಿಸುತ್ತಿರುವ ಈ ಏಳು ಲೋಕಗಳು ಅವನಿಂದಲೇ ಉಂಟಾಗಿವೆ. ಈ ಪರ್ವತ ಸಮುದ್ರಗಳೆಲ್ಲ ಅವನಿಂದಲೇ ಹುಟ್ಟಿವೆ. ಸಮುದ್ರಕ್ಕೆ ಸೇರುವ ಎಲ್ಲ ನದಿಗಳೂ ಅವ ನಿಂದಲೇ ಬಂದಿವೆ. ಅವನಿಂದಲೇ ಎಲ್ಲ ಸಸ್ಯಗಳೂ ರಸಗಳೂ ಬಂದಿವೆ.

“ಎಲ್ಲದರ ಒಳಗೂ ಅವನೇ. ಎಲ್ಲ ಭೂತಗಳ ಅಂತರಾತ್ಮನು ಅವನು. ಈ ಪರಮ ಪುರುಷನೇ ಈ ವಿಶ್ವವಾಗಿರುವನು. ಕರ್ಮವೂ ಅವನೆ, ಯಜ್ಞವೂ ಅವನೆ. ಅವನೇ ಪರಬ್ರಹ್ಮ, ಅವನೇ ತ್ರಿಮೂರ್ತಿ. ಅವನನ್ನು ಅರಿತುಕೊಳ್ಳುವವನು ಅವಿದ್ಯಾಗ್ರಂಥಿ ಗಳನ್ನು ಕಳಚಿಕೊಂಡು ಮುಕ್ತನಾಗುತ್ತಾನೆ.

“ಅವನು ಸ್ವಪ್ರಕಾಶನು, ಪ್ರತಿಯೊಬ್ಬ ಮಾನವನ ಆತ್ಮದೊಳಗೂ ಅವನಿರುವನು. ಅವನಿಂದಲೇ ಎಲ್ಲ ನಾಮ ರೂಪಗಳು ಬಂದಿವೆ. ಅವನಿಂದಲೇ ಮನುಷ್ಯರೂ ಪ್ರಾಣಿಗಳೂ ಬಂದಿರುವುವು. ಅವನೇ ಪರಮ ಶ್ರೇಷ್ಠನು. ಅವನನ್ನು ಅರಿತವನು ಮುಕ್ತನಾಗುತ್ತಾನೆ.

“ಅವನನ್ನು ಅರಿಯುವುದು ಹೇಗೆ? ಉಪನಿಷತ್ ಎಂಬ ಧನುಸ್ಸನ್ನು ತೆಗೆದು ಕೊಂಡು, ಧ್ಯಾನದಿಂದ ಹರಿತವಾದ ಬಾಣವನ್ನು ಅದರಲ್ಲಿ ಇರಿಸಿ, ಅವನ ಚಿಂತನೆಯ ಮೂಲಕವೇ ಧನುಸ್ಸನ್ನು ಏಳೆದು ಅವನನ್ನೇ ಗುರಿಯಾಗಿಟ್ಟುಕೊಂಡು ಹೊಡೆ.

“ಓಂಕಾರವು ಬಿಲ್ಲು, ಜೀವಾತ್ಮನೆ ಬಾಣ, ಬ್ರಹ್ಮವೇ ಲಕ್ಷ್ಯ. ಮನಸ್ಸಿನ ಏಕಾಗ್ರತೆಯ ಮೂಲಕ ಲಕ್ಷ್ಯವನ್ನು ಹೊಡೆಯಬೇಕು. ಬಾಣವು ಲಕ್ಷ್ಯವನ್ನು ತಾಕಿದಾಗ ಅದರೊಡನೆ ಒಂದಾಗುತ್ತದೆ. ಹೀಗೆಯೇ ಜೀವಾತ್ಮವಾದ ಬಾಣವನ್ನು ಆ ವಸ್ತುವಿಗೆ ಗುರಿಯಿಟ್ಟು ಹೊಡೆಯಬೇಕು - ಅದು ವಸ್ತುವಿನೊಡನೆ ಒಂದಾಗಬೇಕು. ಆ ಪರಬ್ರಹ್ಮ ವಸ್ತುವಿ ನಲ್ಲಿಯೇ ಭೂಮಿ ಆಕಾಶಗಳು, ಮನಸ್ಸು ಪ್ರಾಣಗಳು ನೆಲಸಿವೆ.”

ಉಪನಿಷತ್ತಿನಲ್ಲಿ ಬರುವ ಕೆಲವು ವಾಕ್ಯಗಳನ್ನು ಮಹಾವಾಕ್ಯಗಳೆಂದು ಕರೆಯು ತ್ತಾರೆ. ಅವನ್ನು ಆಗಾಗ ಉದ್ಧರಿಸುತ್ತಾರೆ.

“ಅವನಲ್ಲಿಯೇ, ಆ ಆತ್ಮನಲ್ಲಿಯೇ ಎಲ್ಲ ಜಗತ್ತುಗಳೂ ನೆಲಸಿವೆ. ಬೇರೆ ಮಾತು ಗಳಿಂದ ಏನು ಪ್ರಯೋಜನ? ಅವನನ್ನೆ ಅರಿಯಿರಿ. ವಿಶ್ವಾತ್ಮಭಾವವನ್ನು ಪಡೆಯು ವುದಕ್ಕೆ ಇದೇ ಸೇತುವೆ.”

ಅಂಗಿರಸನು ಅನುಷ್ಠಾನ ಮಾರ್ಗವನ್ನು ತೋರಿಸುತ್ತಾನೆ. ರೂಪಕಗಳ ಮೂಲಕ ಇದನ್ನು ವಿವರಿಸುತ್ತಾನೆ.

“ರಥ ಚಕ್ರದ ಗುಂಬದಲ್ಲಿ ಅರೆಕಾಲುಗಳು ಸೇರಿಕೊಂಡಿರುವಂತೆ ದೇಹದ ಒಂದು ಕೇಂದ್ರದಲ್ಲಿ ನಾಡಿಗಳು ಬಂದುಸೇರುವುವು, ಮತ್ತು ಅಲ್ಲಿಂದ ಹೊರ ಹೋಗುವುವು. ಆ ಹೃತ್ಕೇಂದ್ರದಲ್ಲಿ ಓಂ ಎಂದು ಆತ್ಮನನ್ನು ಧ್ಯಾನಿಸು. ನಿನಗೆ ಯಶಸ್ವಿಯಾಗಲಿ.

“ಸೋಮ್ಯನು ಯಶಸ್ವಿಯಾಗಿ ಗುರಿಯನ್ನು ಸೇರಲಿ. ನೀನು ಎಲ್ಲ ಅಂಧಕಾರವನ್ನು ದಾಟಿಹೋಗಿ ಸರ್ವಜ್ಞನೂ ಸರ್ವವಿದನೂ ಆದವನನ್ನು ಸೇರುವಂತಾಗಲಿ. ಅವನ ಮಹಿಮೆ ಸ್ವರ್ಗದಲ್ಲಿ, ಭೂಮಂಡಲದಲ್ಲಿ ಎಲ್ಲೆಲ್ಲಿಯೂ ಬೆಳಗುತ್ತಿದೆ.

“ಅವನು ಮನಸ್ಸಾಗಿರುವನು, ಪ್ರಾಣವಾಗಿರುವನು, ಈ ಶರೀರವನ್ನು ಆಳು ವವನು. ಜೀವಶಕ್ತಿಯಾದ ಆಹಾರದಲ್ಲಿ ಅವನು ನೆಲಸಿರುವನು. ಅಮೃತಸ್ವರೂಪನೂ ಅನಂತ ಸ್ವರೂಪನೂ ಆದ ಅವನನ್ನು ಋಷಿಗಳು ವಿಜ್ಞಾನದ ಮೂಲಕ ನೋಡುತ್ತಾರೆ.

ಜ್ಞಾನ ಮತ್ತು ವಿಜ್ಞಾನ ಎಂಬ ಎರಡು ಶಬ್ದಗಳಿವೆ. ಜ್ಞಾನವೆಂದರೆ ಬೌದ್ಧಿಕ ಜ್ಞಾನ ಮತ್ತು ವಿಜ್ಞಾನವೆಂದರೆ ಸಾಕ್ಷಾತ್ಕಾರ. ಬೌದ್ಧಿಕ ಜ್ಞಾನದ ಮೂಲಕ ಭಗವಂತನನ್ನು ತಿಳಿಯ ಲಾಗುವುದಿಲ್ಲ. ವಿಜ್ಞಾನದ ಮೂಲಕ ಭಗವತ್ಸಾಕ್ಷಾತ್ಕಾರವನ್ನು ಪಡೆದವನಿಗೆ ಏನಾಗುತ್ತದೆ?

“ಅವನ ಹೃದಯಗ್ರಂಥಿಗಳೆಲ್ಲ ಕತ್ತರಿಸಲ್ಪಡುವುವು. ನೀನು ಸತ್ಯವನ್ನು ಅರಿತ ಮೇಲೆ ಎಲ್ಲ ಅಂಧಕಾರವೂ ಶಾಶ್ವತವಾಗಿ ಪರಿಹರಿಸಲ್ಪಡುವುವು.”

ಆಗ ನಿಮಗೆ ಸಂಶಯವೆಲ್ಲಿರುತ್ತದೆ? ಬೇರೆ ಬೇರೆ ವಿಜ್ಞಾನಗಳು ಮತ್ತು ತತ್ತ್ವ ಶಾಸ್ತ್ರಗಳು ಇವುಗಳ ನಡುವೆ ಇರುವ ಸಂಘರ್ಷಣೆ ಕಲಹಗಳೆಲ್ಲ ಆಗ ಮೂರ್ಖತನ ಮತ್ತು ಬಾಲಿಶ ಎಂದೆನಿಸುತ್ತದೆ. ಅವರು ಕಾದಾಡುವುದನ್ನು ನೋಡಿ ನೀವು ನಗುತ್ತೀರಿ. ಎಲ್ಲ ಸಂಶಯಗಳೂ ಮಾಯವಾಗುವುವು, ಕರ್ಮಗಳೆಲ್ಲ ದೂರವಾಗುವುವು.

“ಹಿರಣ್ಮಯವೂ ಶ್ರೇಷ್ಠವೂ ಆದ ಕೋಶದಲ್ಲಿ ಅಶುದ್ಧಿಯಿಲ್ಲದ, ವಿಭಾಗವಿಲ್ಲದ ಬ್ರಹ್ಮವಿರುವುದು. ಅದು ಶುಭ್ರವು, ಜ್ಯೋತಿಗಳಿಗೆಲ್ಲ ಜ್ಯೋತಿಯು. ಆ ಬ್ರಹ್ಮವನ್ನು ಆತ್ಮ ವಿದರು ತಿಳಿದಿರುತ್ತಾರೆ.

“ಅದನ್ನು ಸೂರ್ಯನಾಗಲಿ, ಚಂದ್ರನಾಗಲಿ, ನಕ್ಷತ್ರಗಳಾಗಲಿ ಪ್ರಕಾಶಿಸಲಾರವು. ಈ ಮಿಂಚುಗಳೂ ಪ್ರಕಾಶಿಸಲಾರವು. ಇನ್ನು ಅಗ್ನಿಯ ಮಾತೇಕೆ? ಪ್ರಕಾಶಿಸುತ್ತಿರುವ ಅವನನ್ನೇ ಅನುಸರಿಸಿ ಎಲ್ಲವೂ ಪ್ರಕಾಶಿಸುತ್ತಿರುವುವು. ಈ ಜಗತ್ತು ಅವನ ಬೆಳಕಿನಿಂದ ಬೆಳಗುತ್ತಿದೆ.”

ನೀವು ಉಪನಿಷತ್ತುಗಳನ್ನು ಅಧ್ಯಯನ ಮಾಡುವಾಗ ಇಂತಹ ಹೇಳಿಕೆಗಳನ್ನು ಕಂಠ ಪಾಠ ಮಾಡಬೇಕು.

ಹಿಂದೂ ಮನಸ್ಸಿಗೂ ಐರೋಪ್ಯ ಮನಸ್ಸಿಗೂ ಇರುವ ವ್ಯತ್ಯಾಸವೇನೆಂದರೆ, ಪಶ್ಚಿಮದಲ್ಲಿ ಸತ್ಯಾನ್ವೇಷಣೆ ವಿಶೇಷದ ವಿಶ್ಲೇಷಣೆಯ ಮೂಲಕ ಪ್ರಾರಂಭವಾಗುತ್ತದೆ. ಹಿಂದೂ ವಿನ ಅನ್ವೇಷಣೆ ಇದಕ್ಕೆ ವಿರುದ್ಧವಾದುದು. ಉಪನಿಷತ್ತುಗಳಲ್ಲಿರುವಂತಹ ತಾತ್ತ್ವಿಕ ಭವ್ಯತೆ ಬೇರೆಲ್ಲಿಯೂ ಇಲ್ಲ.

ಮುಂಡಕ ಉಪನಿಷತ್ತು ನಿಮ್ಮನ್ನು ಇಂದ್ರಿಯಗಳಾಚೆ, ರವಿತಾರೆಗಳಾಚೆ ಅತ್ಯಂತ ಭವ್ಯವಾಗಿರುವ ನೆಲೆಗೆ ಕೊಂಡೊಯ್ಯುವುದು. ಮೊದಲು ಅಂಗಿರಸನು ಭೂಮಿ ಭಗವಂತನ ಪಾದ, ಆಕಾಶ ಅವನ ಮಸ್ತಕ ಎಂಬಿತ್ಯಾದಿಯಾಗಿ ಇಂದ್ರಿಯಗ್ರಾಹ್ಯವಾಗಿ ಅವನ ಭವ್ಯತೆ ಯನ್ನು ವರ್ಣಿಸುತ್ತಾನೆ. ಆದರೆ ಅದು ಅವನಿಗೆ ತೃಪ್ತಿಯನ್ನು ನೀಡುವುದಿಲ್ಲ. ಮೊದಲು ಈ ರೀತಿಯ ವಿವರಣೆಯನ್ನಿತ್ತು ಶಿಷ್ಯನನ್ನು ನಿಧಾನವಾಗಿ ಮೇಲೆ ಮೇಲೆಕ್ಕೆ ತೆಗೆದು ಕೊಂಡುಹೋಗಿ ಅತ್ಯುನ್ನತವಾದ ನೇತ್ಯಾತ್ಮಕ ಭಾವನೆಯನ್ನು ನೀಡುತ್ತಾನೆ.

“ಅವನು ಅಮೃತನು. ಅವನು ನಮ್ಮ ಮುಂದೆಯೂ ಹಿಂದೆಯೂ ಇರುವನು, ಎಡಬಲಗಳಲ್ಲಿರುವನು, ಮೇಲೆ ಕೆಳಗೆ ಎಲ್ಲೆಲ್ಲಿಯೂ ಇರುವನು.

“ಒಂದೇ ವೃಕ್ಷದ ಮೇಲೆ ಸುಂದರವಾದ ರೆಕ್ಕೆಗಳುಳ್ಳ ಎರಡು ಹಕ್ಕಿಗಳು ಕುಳಿತು ಕೊಂಡಿವೆ. ಅವು ಯಾವಾಗಲೂ ಒಟ್ಟಿಗೆ ಇರುತ್ತವೆ. ಅವುಗಳಲ್ಲಿ ಒಂದು ಮರದ ಹಣ್ಣು ಗಳನ್ನು ತಿನ್ನು ತ್ತಿದೆ. ಇನ್ನೊಂದು ಏನನ್ನೂ ತಿನ್ನದೆ ಸುಮ್ಮನೆ ನೋಡುತ್ತಿದೆ.

“ಈ ದೇಹದಲ್ಲಿ ಯಾವಾಗಲೂ ಒಟ್ಟಿಗೆ ಇರುವ ಎರಡು ಹಕ್ಕಿಗಳಿವೆ. ಎರಡೂ ಒಂದೇ ರೂಪವುಳ್ಳವುಗಳು, ಸುಂದರ ರೆಕ್ಕೆಯುಳ್ಳವುಗಳು. ಒಂದು ಹಣ್ಣುಗಳನ್ನು ತಿನ್ನು ತ್ತಿರುವ ಜೀವಾತ್ಮ; ಇನ್ನೊಂದು ಅದೇ ಸ್ವರೂಪವುಳ್ಳ ಭಗವಂತನೇ. ಅವನು ನಮ್ಮ ಆತ್ಮದ ಆತ್ಮನಾಗಿ ಈ ದೇಹದಲ್ಲಿರುವನು. ಅವನು ಒಳ್ಳೆಯ ಮತ್ತು ಕೆಟ್ಟ ಹಣ್ಣುಗಳನ್ನು ತಿನ್ನು ತ್ತಿಲ್ಲ - ಸುಮ್ಮನೆ ನೋಡುತ್ತಿರುತ್ತಾನೆ.”

ಕೆಳಗಿನ ಹಕ್ಕಿಗೆ ಗೊತ್ತು ತಾನು ದುರ್ಬಲ, ಅಲ್ಪ, ದೀನ ಎಂದು. ಅದು ಎಲ್ಲ ಬಗೆಯ ಸುಳ್ಳನ್ನೂ ಹೇಳುತ್ತದೆ. ತಾನು ಸ್ತ್ರೀ, ತಾನು ಪುರುಷ, ತಾನು ಹುಡುಗ ಎಂದೆಲ್ಲ ಹೇಳು ತ್ತದೆ. ತಾನು ಒಳ್ಳೆಯದನ್ನು ಕೆಟ್ಟದ್ದನ್ನು ಮಾಡುತ್ತೇನೆ ಎನ್ನು ತ್ತದೆ. ಅದು ಸ್ವರ್ಗಕ್ಕೆ ಹೋಗುತ್ತದೆ, ಏನೇನೋ ಮಾಡುತ್ತದೆ. ಒಂದು ರೀತಿಯ ಅಮಲು ಅದನ್ನು ಈ ರೀತಿ ಕುಣಿಸುತ್ತದೆ. ತಾನು ದುರ್ಬಲ ಎನ್ನುವುದೇ ಈ ಅಮಲಿಗೆ ಮೂಲ ಕಾರಣ.

ತಾನೇನೂ ಅಲ್ಲ ಎಂದು ಭಾವಿಸುವುದರಿಂದ ಅವನು ಎಲ್ಲ ದುಃಖವನ್ನೂ ಪಡೆಯು ತ್ತಾನೆ. ತಾನು ಸೃಷ್ಟಿಸಲ್ಪಟ್ಟ ಅಲ್ಪಜೀವಿ, ತಾನು ಗುಲಾಮ, ಯಾವುದೊ ದೇವ ದೇವತೆಯಿಂದ ನಿಯಂತ್ರಿಸಲ್ಪಟ್ಟವನು ಎಂದು ಭಾವಿಸುತ್ತಾನೆ. ಆದ್ದರಿಂದಲೇ ಅವನು ಅಸುಖಿ.

ಆದರೆ ಅವನು ದೇವರೊಡನೆ ಒಂದಾದಾಗ, ಅವನು ಯೋಗಿಯಾದಾಗ, ಬೇರೆ ಪಕ್ಷಿಯು ತನ್ನದೇ ವೈಭವವೆಂದು ತಿಳಿಯುತ್ತಾನೆ. “ನನ್ನದೇ ವೈಭವವನ್ನು ನಾನು ದೇವ ರೆಂದು ಕರೆಯುತ್ತಾ ಇದ್ದೆ. ಈ ಅಲ್ಪ ‘ಅಹಂ’, ಈ ದುಃಖ, ಎಲ್ಲವೂ ಭ್ರಾಂತಿ - ಅದಿರಲೇ ಇಲ್ಲ. ನಾನು ಸ್ತ್ರೀಯೂ ಆಗಿರಲಿಲ್ಲ, ಪುರುಷನೂ ಆಗಿರಲಿಲ್ಲ, ಬೇರಿನ್ನಾವುದೂ ಆಗಿರ ಲಿಲ್ಲ.” ಆಗ ಅವನ ದುಃಖ ಅಂತ್ಯಗೊಳ್ಳುವುದು.

“ಯಾವಾಗ ಸಾಧಕನು ಹಿರಣ್ಯವರ್ಣದವನೂ, ಕರ್ತೃವೂ, ಈಶನೂ, ಬ್ರಹ್ಮ ಯೋನಿಯೂ ಆದ ಪುರುಷನನ್ನು ನೋಡುತ್ತಾನೆಯೋ ಆಗ ಅವನು ಪುಣ್ಯಪಾಪ ಗಳಿಂದ ಮುಕ್ತನಾಗಿ, ನಿರಂಜನ ಬ್ರಹ್ಮದೊಡನೆ ಪರಮಸಾಮ್ಯವನ್ನು ಪಡೆಯುತ್ತಾನೆ.

“ಆತ್ಮಗಳ ಆತ್ಮನಾದ ಆ ಪರಮಾತ್ಮನೇ ಎಲ್ಲದರ ಮೂಲಕ ಪ್ರಕಾಶಿಸುತ್ತಿರುವನು ಎಂಬುದನ್ನು ಅವನು ತಿಳಿಯುತ್ತಾನೆ.”

ಅವನೇ ಪುರುಷ, ಸ್ತ್ರೀ, ಹಸು, ನಾಯಿ ಎಲ್ಲವೂ. ಅವನೇ ಎಲ್ಲ ಪ್ರಾಣಿಗಳಲ್ಲಿಯೂ ಪಾಪದಲ್ಲಿಯೂ ಪಾಪಿಯಲ್ಲಿಯೂ ಇರುವನು. ಅವನೇ ಸಂನ್ಯಾಸಿ, ಅವನೇ ಅಳುವ ವನಲ್ಲಿರುವವನು, ಅವನೇ ಎಲ್ಲೆಲ್ಲಿಯೂ ಇರುವನು.

“ಋಷಿಯು ಮಾತನಾಡುವುದಿಲ್ಲ. (ಅವನು ಯಾರನ್ನೂ ನಿಂದಿಸುವುದಿಲ್ಲ, ಯಾರನ್ನೂ ಬಯ್ಯುವುದಿಲ್ಲ, ಯಾರ ಕೆಟ್ಟದ್ದನ್ನೂ ಬಯಸುವುದಿಲ್ಲ.) ಅವನ ಆಸೆಗಳೆ ಲ್ಲವೂ ಆತ್ಮನಲ್ಲಿಯೇ ಲೀನವಾಗಿರುತ್ತವೆ. ಬ್ರಹ್ಮಜ್ಞಾನಿಗಳ ಮುಖ್ಯ ಲಕ್ಷಣ ಇದು: ಅವರು ಅವನನ್ನಲ್ಲದೆ ಮತ್ತೇನನ್ನೂ ನೋಡುವುದಿಲ್ಲ.”

ಅವನು ಈ ವಸ್ತುಗಳ ಮೂಲಕ ಆಟವಾಡುತ್ತಿರುವನು. ಉನ್ನತ ದೇವತೆಗಳಿಂದ ಹಿಡಿದು ಕನಿಷ್ಠ ಹುಳದವರೆಗೆ ಎಲ್ಲ ರೂಪಗಳೂ ಅವನೇ. ಭಾವನೆಗಳನ್ನು ದೃಷ್ಟಾಂತ ಗಳ ಮೂಲಕ ವಿವರಿಸಬೇಕಾಗುತ್ತದೆ.

ಮೊದಲನೆಯದಾಗಿ ಸ್ವರ್ಗವೇ ಮುಂತಾದ ಸ್ಥಳಗಳಿಗೆ ಹೋಗಬೇಕೆಂದಿದ್ದರೆ ಅದೂ ಸಾಧ್ಯ ಎಂಬುದನ್ನು ಉಪನಿಷತ್ಕಾರನು ತೋರಿಸುತ್ತಾನೆ. ಅಂದರೆ ವೇದಗಳ ಪ್ರಕಾರ ಒಬ್ಬನು ಏನನ್ನು ಆಸೆಪಟ್ಟರೂ ಅದನ್ನು ಪಡೆಯುತ್ತಾನೆ.

ಹಿಂದೆಯೇ ಹೇಳಿದಂತೆ ಆತ್ಮನು ಬರುವುದೂ ಇಲ್ಲ ಹೋಗುವುದೂ ಇಲ್ಲ. ಅವನಿಗೆ ಹುಟ್ಟುಸಾವುಗಳಿಲ್ಲ. ನೀವೆಲ್ಲರೂ ಸರ್ವವ್ಯಾಪಿ, ಆತ್ಮಸ್ವರೂಪರು. ನೀವು ಈ ಕ್ಷಣ ಸ್ವರ್ಗದಲ್ಲೂ ಇದ್ದೀರಿ, ಅತ್ಯಂತ ಕತ್ತಲೆಯ ಸ್ಥಳದಲ್ಲಿಯೂ ಇದ್ದೀರಿ. ನೀವು ಎಲ್ಲೆ ಲ್ಲಿಯೂ ಇರುವಿರಿ. ನೀವು ಎಲ್ಲಿಲ್ಲ? ಆದ್ದರಿಂದ ನೀವು ಎಲ್ಲಿಗಾದರೂ ಹೋಗುವುದು ಹೇಗೆ? ಈ ಬರುವುದು ಹೋಗುವುದು ಎಲ್ಲ ಭ್ರಮೆ. ಆತ್ಮನು ಬರುವುದೂ ಇಲ್ಲ ಹೋಗುವುದೂ ಇಲ್ಲ.

ಈ ದರ್ಶನಗಳೆಲ್ಲ ಬದಲಾಗುತ್ತವೆ. ಮನಸ್ಸು ಒಂದು ಸ್ಥಿತಿಯಲ್ಲಿರುವಾಗ ಏನನ್ನೊ ನೋಡುತ್ತದೆ, ಏನೊ ಕನಸು ಕಾಣುತ್ತದೆ. ನಮ್ಮ ಈಗಿನ ಸ್ಥಿತಿಯಲ್ಲಿ ಈ ಜಗತ್ತನ್ನು, ಮನುಷ್ಯ ಪ್ರಾಣಿ, ವಿವಿಧ ವಸ್ತುಗಳನ್ನು, ನೋಡುತ್ತಿದ್ದೇವೆ. ಆದರೆ ಈ ಸ್ಥಳದಲ್ಲೇ ಈ ಸ್ಥಿತಿ ಬದಲಾಗಬಹುದು. ನಾವು ನೋಡುತ್ತಿರುವ ಈ ಭೂಮಿಯೇ ಸ್ವರ್ಗವಾಗಿಯೊ, ನರಕವಾಗಿಯೊ ಅಥವಾ ಇನ್ನಾವ ಸ್ಥಳವಾಗಿಯೊ ಕಾಣಬಹುದು.

ಇದೆಲ್ಲವೂ ನಮ್ಮ ಆಸೆಗಳನ್ನು ಅವಲಂಬಿಸಿರುತ್ತವೆ. ಈ ಕನಸು ಶಾಶ್ವತವಲ್ಲ, ನಮಗೆ ತಿಳಿದಿರುವಂತೆ, ರಾತ್ರಿ ಕಾಣುವ ಕನಸು ನಿಂತುಹೋಗಲೇಬೇಕು. ಈ ಕನಸು ಗಳಾವುವೂ ಶಾಶ್ವತವಲ್ಲ. ನಾವು ಏನು ಮಾಡಬೇಕೆಂದು ಬಯಸುತ್ತೇವೆಯೊ ಅದರ ಕನಸು ಕಾಣುತ್ತೇವೆ. ಆದ್ದರಿಂದ ಯಾರು ಈ ಜೀವನದಲ್ಲಿ ತಾವು ಸ್ವರ್ಗಕ್ಕೆ ಹೋಗಬೇಕು ಮತ್ತು ಅಲ್ಲಿ ತಮ್ಮ ಬಂಧುಗಳನ್ನು ಸಂಧಿಸಬೇಕು ಎಂದು ಭಾವಿಸುವವರು ಈ ಜೀವನದ ಕನಸು ಮುಗಿದನಂತರ ಅಲ್ಲಿಗೆ ಹೋಗುತ್ತಾರೆ. ಈ ಜೀವನದ ಆಸೆಗಳ ಒತ್ತಡದಿಂದಾಗಿ ಅವರು ಈ ಕನಸುಗಳನ್ನು ನೋಡಲೇ ಬೇಕಾಗುತ್ತದೆ. ಮೂಢನಂಬಿಕೆಯುಳ್ಳ ಮತ್ತು ನರಕದ ಭಯದಿಂದ ತತ್ತರಿಸುತ್ತಿರುವ ಜನರು ತಾವು ನರಕದಲ್ಲಿಯರುವುದಾಗಿ ಕನಸು ಕಾಣುತ್ತಾರೆ. ಪ್ರಾಣಿಸಹಜ ಭಾವನೆಗಳುಳ್ಳವರು ಸತ್ತನಂತರ ಪ್ರಾಣಿಗಳೇ ಆಗುತ್ತಾರೆ. ಪ್ರತಿಯೊಬ್ಬನೂ ತನ್ನ ಆಸೆಗೆ ಅನುಗುಣವಾದುದನ್ನು ಪಡೆಯುತ್ತಾನೆ.

ರಸ್ತೆ ನಿರ್ಮಿಸುವುದು, ಆಸ್ಪತ್ರೆ ಕಟ್ಟುವುದು ಮುಂತಾದ ಸತ್ಕಾರ್ಯಗಳನ್ನಲ್ಲದೆ ಮತ್ತೇನನ್ನೂ ಅರಿಯದವರು ಸತ್ತಮೇಲೆ ಒಳ್ಳೆಯ ಕನಸನ್ನು ಕಾಣುತ್ತಾರೆ, ಎಂದು ಈ ಉಪನಿಷತ್ತು ಹೇಳುತ್ತದೆ. ದೇವತೆಗಳ ಶರೀರವನ್ನು ಪಡೆದು, ಏನನ್ನಾದರು ತಿನ್ನು ತ್ತ ಎಲ್ಲಿಗೆ ಬೇಕಾದರೂ ಹೋಗುತ್ತ, ಗೋಡೆಯ ಮೂಲಕ ತೂರಿಹೋಗುತ್ತ, ಕೆಲವೊಮ್ಮೆ ಕೆಳಗೆ ಬಂದು ಇತರರನ್ನು ಹೆದರಿಸುತ್ತ ಇರುವುದಾಗಿ ಅವರು ಕನಸುಕಾಣುತ್ತಾರೆ.

ನಮ್ಮ ಪುರಾಣಗಳಲ್ಲಿ ಸ್ವರ್ಗದಲ್ಲಿ ವಾಸಿಸುವ ದೇವತೆಗಳ ವಿಷಯವೂ, ಹೆಚ್ಚು ಕಡಿಮೆ ದೇವತೆಗಳಂತಿರುವ ಆದರೆ ಅವರಿಗಿಂತ ಕ್ರೂರಿಗಳಾದ ದೇವಕರ ವಿಷಯವೂ ಬರುತ್ತದೆ. ದೇವತೆಗಳು ನಿಮ್ಮ ಏಂಜೆಲ್ಸ್ \enginline{(angles)} ಇದ್ದಹಾಗೆ. ಅವರಲ್ಲಿ ಕೆಲವರು ಆಗಾಗ ದುಷ್ಟರಾಗಿ ಮನುಷ್ಯರ ಹೆಣ್ಣುಮಕ್ಕಳ ಮೇಲೆ ಕಣ್ಣು ಹಾಯಿಸುತ್ತಾರೆ. ಇಂತಹ ಕೃತ್ಯಗಳಿಗಾಗಿ ನಮ್ಮ ದೇವತೆಗಳನ್ನು ವೈಭವೀಕರಿಸುವುದುಂಟು. ಅವರಿಂದ ನೀವು ಏನನ್ನು ನಿರೀಕ್ಷಿಸಬಲ್ಲಿರಿ? ಅವರು ಇಲ್ಲಿ ಆಸ್ಪತ್ರೆ ಕಟ್ಟುವವರಾಗಿದ್ದರು, ಬೇರೆ ಮನುಷ್ಯರ ಬಗ್ಗೆ ಯಲ್ಲದೆ ಅವರಿಗೆ ಮತ್ತೇನೂ ತಿಳಿಯದು. ಅವರು ಸ್ವಲ್ಪ ಒಳ್ಳೆಯ ಕೆಲಸ ಮಾಡಿದ್ದರಿಂದ ದೇವತೆಗಳಾದರು. ಅವರು ಸತ್ಕರ್ಮ ಮಾಡುವುದು ಹೆಸರು ಕೀರ್ತಿಗಾಗಿ, ಯಾವುದೋ ಪ್ರತಿಫಲಕ್ಕಾಗಿ. ಅನಂತರ ತಾವು ಸ್ವರ್ಗದಲ್ಲಿ ತಾವು ಬಯಸಿದುದನ್ನೆಲ್ಲ ಪಡೆದಿರುವುದಾಗಿ ಕನಸು ಕಾಣುತ್ತಾರೆ.

ಹಾಗೆಯೇ ಈ ಜೀವನದಲ್ಲಿ ಕೆಟ್ಟದ್ದನ್ನು ಮಾಡಿದ ರಾಕ್ಷಸರಿದ್ದಾರೆ. ಆದರೆ ಈ ಕನಸು ಗಳೆಲ್ಲ ಶಾಶ್ವತವಲ್ಲ. ಅವರು ಮತ್ತೆ ಹಿಂದಿರುಗಿ ಬಂದು ಹಳೆಯ ಮಾನವ ಜೀವನದ ಕನಸನ್ನು ಕಾಣಬೇಕು. ಈ ಸ್ವರ್ಗ ನರಕಗಳ ಮೂರ್ಖ ಭಾವನೆಗಳನ್ನು ಸಂಪೂರ್ಣ ಬಿಟ್ಟುಬಿಡಬೇಕೆಂದು ಉಪನಿಷತ್ತುಗಳು ಹೇಳುತ್ತವೆ.

ಈ ಪ್ರಪಂಚದಲ್ಲಿ ಎರಡು ವಿಷಯಗಳಿವೆ - ಕನಸು ಮತ್ತು ಸತ್ಯ. ನಾವು ಯಾವು ದನ್ನು ಜೀವನ ಎಂದು ಕರೆಯುತ್ತೇವೆಯೊ ಅದು ಒಂದು ಕನಸುಗಳ ಸರಣಿ - ಕನಸಿ ನೊಳಗೆ ಕನಸು. ಒಂದು ಕನಸನ್ನು ಸ್ವರ್ಗವೆಂದೂ, ಮತ್ತೊಂದನ್ನು ಭೂಮಿಯೆಂದೂ, ಇನ್ನೊಂದನ್ನು ನರಕವೆಂದೂ ಕರೆಯುತ್ತೇವೆ. ಒಂದು ಕನಸನ್ನು ಮಾನವದೇಹವೆಂದೂ, ಇನ್ನೊಂದನ್ನು ಪ್ರಾಣಿದೇಹವೆಂದೂ ಮುಂತಾಗಿ ಕರೆಯುತ್ತೇವೆ. ಎಲ್ಲ ಕನಸುಗಳೇ. ಸತ್ಯ ಯಾವುದೆಂದರೆ ಸಚ್ಚಿದಾನಂದ ಬ್ರಹ್ಮ.

ಈ ಎಲ್ಲ ಕನಸುಗಳನ್ನು ತೊಡೆದುಹಾಕಿ, ಈ ನಿದ್ರಾಲೋಕವನ್ನು ಮೀರಿ ತನ್ನ ನಿಜ ಸ್ವರೂಪವನ್ನು ಅರಿತವನೇ ಗುರು.

ನಾವು ಏನನ್ನಾದರೂ ಆಸೆಪಟ್ಟರೆ ನಮ್ಮನ್ನು ನಾವೇ ಮೋಸಪಡಿಸಿಕೊಳ್ಳುತ್ತೇವೆ. “ನಾನು ಸ್ವರ್ಗಕ್ಕೆ ಹೋಗಬೇಕು ಎಂದು ಆಸೆಪಟ್ಟರೆ ನಾವು ಸತ್ತಮೇಲೆ ಸ್ವರ್ಗದಲ್ಲಿರು ವಂತೆಯೂ ದೇವತೆಗಳಿಂದ, ಸುಖಭೋಗಗಳಿಂದ ಸುತ್ತುವರಿದಿರುವಂತೆಯೂ ಭಾವಿಸ ತೊಡಗುತ್ತೇನೆ. ಮೃತ್ಯುಮುಖದಿಂದ ಹೊರಬಂದ ಸುಮಾರು ಐವತ್ತು ಜನರನ್ನು ನಾನು ಸಂಧಿಸಿರುವೆನು. ಅವರೆಲ್ಲರೂ ತಾವು ಸ್ವರ್ಗದಲ್ಲಿದ್ದ ಕಥೆಯನ್ನು ಹೇಳಿದರು. ಇವೆಲ್ಲ ಪುರಾಣಕಥೆಗಳು, ಇವೆಲ್ಲ ಸಮ್ಮೋಹನವೆಂಬುದು ಗೊತ್ತಾಗುತ್ತದೆ.

ಪಾಶ್ಚಾತ್ಯರು ಇಲ್ಲಿಯೇ ದೊಡ್ಡ ತಪ್ಪು ಮಾಡುವುದು. ನಿಮ್ಮಲ್ಲೂ ಈ ಸ್ವರ್ಗ ನರಕಗಳ ಭಾವನೆಗಳಿವೆ - ಇದನ್ನು ನಾವೂ ಒಪ್ಪುತ್ತೇವೆ. ಆದರೆ ಈ ಭೂಮಿ ಸತ್ಯ ಎಂದು ನೀವು ಹೇಳುತ್ತೀರಿ. ಅದು ಸಾಧ್ಯವಿಲ್ಲ. ಇದು ಸತ್ಯವಾದರೆ ಸ್ವರ್ಗನರಕಗಳೂ ಸತ್ಯ, ಏಕೆಂದರೆ ಇವುಗಳ ಪ್ರತಿಯೊಂದು ಪ್ರಮಾಣವೂ ಒಂದೇ. ಒಂದು ಕನಸಾದರೆ ಇವೆಲ್ಲವೂ ಕನಸೇ.

ಸ್ವರ್ಗಗಳು ಮಾತ್ರ ಕನಸಾಗಿರುವುದಲ್ಲದೆ, ಈ ಜೀವನ ಮತ್ತು ಇಲ್ಲಿರುವುದೆಲ್ಲವೂ ಕನಸಿನ ರಾಜ್ಯಕ್ಕೇ ಸೇರಿದ್ದು. ಕೆಲವರು ಒಂದು ಕನಸಿನ ಸ್ಥಿತಿಯಿಂದ ಇನ್ನೊಂದು ಕನಸಿನ ಸ್ಥಿತಿಗೆ ಹೋಗಲು ಬಯಸುತ್ತಾರೆ. ಇವರೇ ಸಂಸಾರಿಗಳು - ಒಂದು ಕನಸಿನಿಂದ ಮತ್ತೊಂದಕ್ಕೆ ಸಂಚರಿಸುವವರು.

ಆತ್ಮವನ್ನು ಸ್ತ್ರೀ ಅಥವಾ ಪುರುಷ ಎಂದು ಕರೆಯುವುದು ಎಂಥ ಮೂರ್ಖತನ! ಆತ್ಮನಲ್ಲಿ ಲಿಂಗಭೇದ ಹೇಗೆ ಸಾಧ್ಯ? ಇದು ಆತ್ಮವಿಸ್ಮರಣೆ. ನಿಮ್ಮನ್ನು ನೀವೇ ಮರೆತು ನಾನು ಸ್ತ್ರೀ ಅಥವಾ ಪುರುಷ ಎಂದು ಭಾವಿಸುತ್ತೀರಿ. ಇನ್ನೂ ಮೋಹಕ್ಕೆ ಒಳಗಾಗಿ ಸ್ವರ್ಗಕ್ಕೆ ಹೋಗಬೇಕೆಂದು ಬಯಸುತ್ತೀರಿ, ದೇವ ದೇವತೆಗಳನ್ನು ಕಲ್ಪಿಸುತ್ತೀರಿ, ಅವರಿಗೆ ಮಂಡಿಯೂರಿ ಪ್ರಾರ್ಥಿಸುತ್ತೀರಿ. ಆನಂತರ ದೇವ - ಶರೀರವನ್ನು ಪಡೆದು ಲಕ್ಷಾಂತರ ಜನರಿಂದ ಪೂಜಿಸಲ್ಪಡುತ್ತೀರಿ. ಇದೆಲ್ಲ ಆದಮೇಲೆ ಮತ್ತೆ ಇನ್ನೊಂದು ಸ್ವಪ್ನಲೋಕಕ್ಕೇ ಹೋಗುತ್ತೇವೆ.

ನಾವೆಲ್ಲರೂ ಇಲ್ಲಿ ಒಂದೇ ದೋಣಿಯಲ್ಲಿದ್ದೇವೆ. ಒಂದೇ ದೋಣಿಯಲ್ಲಿರುವ ವರೆಲ್ಲರೂ ಪರಸ್ಪರರನ್ನು ನೋಡಬಹುದು. ಇದರಿಂದ ಹೊರಗೆ ಬನ್ನಿ, ಈ ಸ್ವಪ್ನಲೋಕವನ್ನು ಮೀರಿಹೋಗಿ. ಅಜ್ಞಾನಿಗಳು ತಮಗೆ ದೇಹಗಳಿವೆ, ಪತ್ನಿಯರಿದ್ದಾರೆ ಎಂದೆಲ್ಲ ಭಾವಿಸುತ್ತಾರೆ. ನಾನೂ ಒಬ್ಬ ಮೂರ್ಖ, ನನಗೆ ಇಂದ್ರಿಯಗಳಿವೆ ಎಂದೆಲ್ಲ ಭಾವಿಸು ತ್ತೇನೆ. ಆದ್ದರಿಂದ ನಾವೆಲ್ಲರೂ ಒಂದೇ ದೋಣಿಯಲ್ಲಿದ್ದೇವೆ, ನಾವು ಪರಸ್ಪರ ನೋಡುತ್ತೇವೆ. ನಾವು ನೋಡಲಾಗದ, ಸ್ಪರ್ಶಿಸಲಾಗದ ಲಕ್ಷಾಂತರ ಜನರು ಇಲ್ಲಿ ಇರಬಹುದು. ಸಮ್ಮೋಹನ ಸ್ಥಿತಿಯಲ್ಲಿ ನಿಮ್ಮ ಮೂರು ಪುಸ್ತಕಗಳಿದ್ದರೆ ನೀವು ಎರಡೇ ಪುಸ್ತಕಗಳನ್ನು ಕಾಣುವಂತೆ ಮಾಡಬಹುದು. ನೀವು ಅದೇ ಸ್ಥಿತಿಯಲ್ಲಿ ಅನೇಕ ವರ್ಷ ಗಳಿದ್ದರೂ ನಿಮಗೆ ಗೊತ್ತಾಗುವುದೇ ಇಲ್ಲ. ಮೂವತ್ತು ಜನರು ಒಂದೇ ಸಮ್ಮೋಹನ ಸ್ಥಿತಿಯಲ್ಲಿದ್ದು ಈ ಪುಸ್ತಕ ಅಲ್ಲಿಲ್ಲ ಎಂದು ಅವರಿಗೆ ಹೇಳಿದರೆ ಅವರಾರಿಗೂ ಈ ಪುಸ್ತಕ ಕಾಣುವುದೇ ಇಲ್ಲ. ಸ್ತ್ರೀ ಪುರುಷ ಪ್ರಾಣಿಗಳೆಲ್ಲರೂ ಇಂತಹ ಸಮ್ಮೋಹನ ಸ್ಥಿತಿಯಲ್ಲಿ ರುವರು, ಎಲ್ಲರೂ ಒಂದೇ ಕನಸನ್ನು ಕಾಣುತ್ತಿರುವರು, ಏಕೆಂದರೆ ಅವರೆಲ್ಲರೂ ಒಂದೇ ದೋಣಿಯಲ್ಲಿರುವರು.

ಮಾನಸಿಕ, ಭೌತಿಕ, ನೈತಿಕ ಜಗತ್ತೆಲ್ಲವೂ, ಇಡೀ ವಿಶ್ವವೇ ಒಂದು ನಿದ್ರೆ ಎಂದು ವೇದಾಂತ ಹೇಳುತ್ತದೆ. ಈ ನಿದ್ರೆಗೆ ಕಾರಣ ಯಾರು? ನೀವೇ. ಈ ಅಳು ನಗು, ಈ ಹೋರಾಟ, ಈ ಪರದಾಟ ಇವುಗಳಿಂದ ನಿಮಗಾವ ಪ್ರಯೋಜನವೂ ಇಲ್ಲ.

ಆದರೆ ಇವುಗಳೇ ವೈರಾಗ್ಯ ಮನೋಭಾವವನ್ನು ನೀಡುತ್ತವೆ. ಈ ನಿದ್ರೆಗೆ ಹೆಚ್ಚು ಹೆಚ್ಚು ಅಂಟಿಕೊಳ್ಳುವುದೇ ವ್ಯಾಮೋಹ. ಆದ್ದರಿಂದಲೇ ಎಲ್ಲ ಧರ್ಮಗಳೂ ಸಂಸಾರ ತ್ಯಾಗಕ್ಕೆ ಹೆಚ್ಚು ಒತ್ತನ್ನು ನೀಡುತ್ತವೆ. ಆದರೆ ಈ ಸಂಸಾರ ತ್ಯಾಗದ ನಿಜವಾದ ಅರ್ಥವನ್ನು ಅನೇಕರು ಅರಿಯರು. ಅವರು ಕಾಡಿನಲ್ಲಿ ಉಪವಾಸ ಮಾಡುತ್ತಿದ್ದರು ಮತ್ತು ದೆವ್ವಗಳು ತಮ್ಮೆಡೆಗೆ ಬರುವುದನ್ನು ನೋಡುತ್ತಿದ್ದರು.

ನೀವು ಭಾರತದ ಐಂದ್ರಜಾಲಿಕರು ಹಗ್ಗವನ್ನು ನೆಟ್ಟಗೆ ನಿಲ್ಲಿಸುವ ಚಮತ್ಕಾರದ ಕಥೆಯನ್ನು ಕೇಳಿರಬಹುದು. ನಾನದನ್ನು ನೋಡಿಲ್ಲ. ಮೊಗಲ್ ಚಕ್ರವರ್ತಿ ಜಹಂಗೀರ್ ಅದನ್ನು ಪ್ರಸ್ತಾಪಿಸುತ್ತಾನೆ. ಅವನು ಹೇಳುತ್ತಾನೆ: “ಅಲ್ಲಾಹ್, ಏನಿದು ಈ ದೆವ್ವಗಳು ಮಾಡುತ್ತಿರುವುದು!? ಅವರು ಹಗ್ಗವನ್ನೊ ಸರಪಳಿಯನ್ನೊ ಮೇಲಕ್ಕೆ ಎಸೆಯುತ್ತಾರೆ, ಅದು ಯಾವುದಕ್ಕೊ ನಾಟಿದಂತೆ ಸ್ಥಿರವಾಗಿ ನಿಂತುಕೊಂಡು ಬಿಡುತ್ತದೆ. ಅನಂತರ ಒಂದು ಬೆಕ್ಕು ಆ ಹಗ್ಗದ ಮೇಲೇರಿ ಹೋಗುತ್ತದೆ, ಅನಂತರ ನಾಯಿ, ತೋಳ, ಹುಲಿ, ಸಿಂಹ ಇವುಗಳು ಹೋಗುತ್ತವೆ. ಅವೆಲ್ಲವೂ ಹಗ್ಗದ ಮೇಲೆ ಹತ್ತಿಹೋಗಿ ಕಣ್ಮರೆ ಯಾಗುತ್ತವೆ. ಕೆಲವೊಮ್ಮೆ ಅವರು ಮನುಷ್ಯರನ್ನೂ ಹಗ್ಗದ ಮೇಲೆ ಕಳಿಸುತ್ತಾರೆ. ಇಬ್ಬರು ವ್ಯಕ್ತಿಗಳು ಮೇಲೆ ಹೋಗಿ ಅಲ್ಲಿ ಹೋರಾಡುತ್ತಾರೆ, ಅನಂತರ ಕಣ್ಮರೆಯಾಗುತ್ತಾರೆ. ಸ್ವಲ್ಪ ಹೊತ್ತಿನ ನಂತರ ಜಗಳವಾಡುವ ಶಬ್ದಕೇಳಿಸುತ್ತದೆ, ಮುಂದೆ ತಲೆ ಕೈ ಕಾಲುಗಳು ಕೆಳಗೆ ಬೀಳುತ್ತವೆ. ಮೂರು ನಾಲ್ಕು ಸಾವಿರ ಜನರು ಸುತ್ತಲೂ ನೆರೆದಿರುತ್ತಾರೆ. ಇದನ್ನು ತೋರಿಸುವವನು ಕೇವಲ ಕೌಪೀನಧಾರಿಯಾಗಿರುತ್ತಾನೆ.” ಇದನ್ನೇ ಸಮ್ಮೋಹಿನಿ ಎನ್ನುತ್ತಾರೆ - ಇದು ಇಡೀ ಪ್ರೇಕ್ಷಕರ ಕಣ್ಮರೆಸುವುದು.

ಅದನ್ನೇ ಅವರು ತಮ್ಮ ವಿಜ್ಞಾನವೆನ್ನುತ್ತಾರೆ. ಅದು ಕೆಲವು ಮಿತಿಗೆ ಒಳಪಟ್ಟಿ ರುತ್ತದೆ. ಅದರೆ ನೀವು ಆ ಮಿತಿಯನ್ನು ಮೀರಿಹೋದರೆ, ಅಥವಾ ಅದರೊಳಗೆ ಬಂದರೆ ನಿಮಗದು ಕಾಣುವುದಿಲ್ಲ. ಈ ಆಟ ಆಡುವವನು ಇದಾವುದನ್ನೂ ನೋಡು ವುದಿಲ್ಲ, ಆದ್ದರಿಂದ ನೀವು ಅವನ ಹತ್ತಿರ ನಿಂತರೆ ನಿಮಗೂ ಏನೂ ಕಾಣಿಸುವುದಿಲ್ಲ. ಸಮ್ಮೋಹಿನಿ ಎಂದರೆ ಇದು.

ಆದ್ದರಿಂದ ನಾವು ಈ ವೃತ್ತದಿಂದ ದೂರಹೋಗಬೇಕು (ಜ್ಞಾನ) ಅಥವಾ ಈ ವೃತ್ತದೊಳಗೆ ಐಂದ್ರಜಾಲಿಕನಾದ ದೇವರ ಹತ್ತಿರ ಇರಬೇಕು (ಭಕ್ತಿ). ಇಡೀ ಜಗತ್ತು ಅವನ ಇಂದ್ರಜಾಲ.

ಅಧ್ಯಾಯಗಳಾದ ಮೇಲೆ ಅಧ್ಯಾಯಗಳು ಬಂದು ಹೋಗುತ್ತವೆ. ಇದೇ ಮಾಯೆ, ಈ ಎಲ್ಲ ವೈಚಿತ್ರ್ಯಗಳನ್ನು ಸೃಷ್ಟಿಸುವ ಶಕ್ತಿ. ಈ ಮಾಯೆಯನ್ನು ಆಳುವವನೇ ದೇವರು, ಮತ್ತು ಈ ಮಾಯೆಯಿಂದ ಆಳಲ್ಪಡುವವನೇ ಜೀವಾತ್ಮ. ಹಗ್ಗದ ಕಣ್ಕಟ್ಟಿನಲ್ಲಿ ಕೇಂದ್ರದಲ್ಲಿ ನಿಂತಿರುವ ವ್ಯಕ್ತಿಯು ಆಟವಾಡಿಸುತ್ತಾನೆಯೇ ಹೊರತು ತಾನು ಮೋಹಕ್ಕೆ ಒಳ ಗಾಗುವುದಿಲ್ಲ, ಆದರೆ ಪ್ರೇಕ್ಷಕರೆಲ್ಲರೂ ಮಾಯೆಗೆ ಒಳಗಾಗಿರುತ್ತಾರೆ. ಮಾಯೆಯನ್ನು ಆಳುವ ಆತ್ಮವೇ ದೇವರು ಮತ್ತು ಮಾಯೆಗೆ ಒಳಗಾಗಿರುವ ಅಲ್ಪಾತ್ಮಗಳು ನಾನು, ನೀವು ಎಲ್ಲರೂ.

ಐಂದ್ರಜಾಲಿಕನ ಹತ್ತಿರ ತೆವಳುತ್ತ ಹೋಗು ಮತ್ತು ನೀನು ಕೇಂದ್ರವನ್ನು ತಲುಪಿ ದಾಗ ನಿನಗೇನೂ ಕಾಣಿಸುವುದಿಲ್ಲ - ದೃಷ್ಟಿ ಸ್ಪಷ್ಟವಾಗುತ್ತದೆ, ಎಂದು ಭಕ್ತನು ಹೇಳುತ್ತಾನೆ.

ಜ್ಞಾನಿಗೆ ಈ ತೊಂದರೆಯನ್ನು ತೆಗೆದುಕೊಳ್ಳುವುದು ಇಷ್ಟವಿಲ್ಲ - ಇದು ಅಪಾಯದ ಮಾರ್ಗ. ಹುಚ್ಚನಲ್ಲದೆ ಇರುವವನು ತನಗೆ ಮಣ್ಣು ಮೆತ್ತಿದಾಗ ಅದನ್ನು ತೊಳೆಯಲು ಮಣ್ಣನ್ನು ಉಪಯೋಗಿಸುತ್ತಾನೆಯೆ? ಈ ಸಮ್ಮೋಹಿನಿಯನ್ನೇ ಬೆಳಸುವುದೇಕೆ? ವೃತ್ತದಿಂದ ಹೊರಗೆ ಬನ್ನಿ, ಅದನ್ನು ಕತ್ತರಿಸಿ ಮುಕ್ತರಾಗಿ. ನೀವು ಮುಕ್ತ ರಾದಾಗ ಬಂಧನಕ್ಕೊಳಗಾಗದೆ ಆಟವಾಡಬಲ್ಲಿರಿ. ಈಗ ನೀವು ಹಿಡಿಯಲ್ಪಟ್ಟಿದ್ದೀರಿ, ಆಗ ನೀವೇ ಸಿಡಿಯುತ್ತೀರಿ - ಅದೇ ವ್ಯತ್ಯಾಸ.

ಆದ್ದರಿಂದ ಈ ಉಪನಿಷತ್ತಿನ ಮೊದಲ ಭಾಗದಲ್ಲಿ; ಈ ಸ್ವರ್ಗದ ಭಾವನೆ ಯನ್ನೂ ಹುಟ್ಟು ಸಾವುಗಳ ಭಾವನೆಯನ್ನೂ ಬಿಟ್ಟು ಬಿಡಬೇಕೆಂದು ಹೇಳಿದೆ. ಇದೆಲ್ಲ ಮೂರ್ಖತನ, ಯಾವನೂ ಹುಟ್ಟುವುದೂ ಇಲ್ಲ ಸಾಯುವುದೂ ಇಲ್ಲ. ಇದೆಲ್ಲ ಕನಸು. ಶಾಶ್ವತ ಜೀವನ, ಸ್ವರ್ಗ, ಭೂಮಿ ಎಲ್ಲ ಕನಸು.

ಸ್ವರ್ಗವು ಒಂದು ಮೂಢನಂಬಿಕೆ, ದೇವರು ಒಂದು ಮೂಢನಂಬಿಕೆ, ಆದರೆ ತಾನು ಮಾತ್ರ ಮೂಢನಂಬಿಕೆಯಲ್ಲ - ಎಂದು ಜಡವಾದಿಯ ಮತ ಸರಿಯಲ್ಲ. ಒಂದು ಮೂಢನಂಬಿಕೆಯಾದರೆ - ಒಂದು ಕೊಂಡಿ ಅಸತ್ಯವಾದರೆ - ಇಡೀ ಸರಪಳಿಯೇ ಅಸತ್ಯ. ಇಡೀ ಸರಪಳಿಯ ಅಸ್ತಿತ್ವ ಒಂದು ಕೊಂಡಿಯನ್ನು ಅವಲಂಬಿಸಿರುತ್ತದೆ - ಮತ್ತು ಒಂದು ಕೊಂಡಿಯು ಇಡಿ ಸರಪಳಿಯನ್ನು ಅವಲಂಬಿಸಿರುತ್ತದೆ.

ಸ್ವರ್ಗವಿಲ್ಲದಿದ್ದರೆ ಭೂಮಿಯೂ ಇಲ್ಲ; ದೇವರಿಲ್ಲದಿದ್ದರೆ ಮನುಷ್ಯನೂ ಇಲ್ಲ. ನೀವು ಈಗ ಸಮ್ಮೋಹಿನಿಗೆ ಒಳಗಾಗಿದ್ದೀರಿ. ಈ ಸ್ಥಿತಿಯಲ್ಲಿರುವವರೆಗೂ ನೀವು ದೇವರನ್ನೂ ಪ್ರಕೃತಿಯನ್ನೂ ಜೀವಿಗಳನ್ನೂ ನೋಡಲೇ ಬೇಕು. ನೀವು ಇದನ್ನು ಮೀರಿ ಹೋದಾಗ (ಸಗುಣ) ದೇವರೂ ಇಲ್ಲ, ಪ್ರಕೃತಿಯೂ ಇಲ್ಲ, ಜೀವಿಗಳೂ ಇಲ್ಲ.

ಆದ್ದರಿಂದ ನಾವು ಮೊದಲನೆಯದಾಗಿ ದೇವರು, ಸ್ವರ್ಗ, ಸುಖಾನುಭವ ಮುಂತಾದ ಭಾವನೆಗಳನ್ನೆಲ್ಲ ಬಿಟ್ಟುಬಿಡಬೇಕು. ಸ್ವರ್ಗಕ್ಕೆ ಹೋಗುವುದು ಮುಂತಾದ ವುಗಳೆಲ್ಲ ಬರಿ ಕನಸು.

ಇದನ್ನೆಲ್ಲ ಹೇಳಿದ ಮೇಲೆ ಈ ಕನಸಿನಿಂದ ಪಾರಾಗುವುದು ಹೇಗೆಂಬುದನ್ನು ಈ ಉಪನಿಷತ್ತು ತಿಳಿಸುತ್ತದೆ. ವಿಶ್ವಾತ್ಮದೊಡನೆ ಒಂದಾಗುವುದೇ ಆತ್ಯಂತ ಮುಖ್ಯವಾದ ಭಾವನೆ. ಇವೆಲ್ಲವೂ ಯಾರ ಅಭಿವ್ಯಕ್ತಿಯೊ ಆ ವಿಶ್ವಾತ್ಮನು ಇವಾವುದಕ್ಕೂ ಸಂಬಂಧಿ ಸಿಲ್ಲ - ಇವೆಲ್ಲವೂ ಮಾಯೆ. ಯಾರ ಆಧಾರದ ಮೇಲೆ ಈ ಎಲ್ಲ ಆಟ ನಡೆಯುತ್ತಿ ದೆಯೊ, ಯಾವ ಹಿನ್ನೆಲೆಯಲ್ಲಿ ಈ ಚಿತ್ರಗಳು ತೋರಿಕೊಳ್ಳುತ್ತಿವೆಯೋ ಆ ವಿಶ್ವಾತ್ಮ ದೊಡನೆ ನಾವು ಒಂದಾಗಿದ್ದೇವೆ. ನೀವು ಅವನೊಡನೆ ಒಂದಾಗಿರುವುದು ನಿಮಗೆ ಗೊತ್ತು, ಆದರೆ ಅದನ್ನು ಸಾಕ್ಷಾತ್ಕಾರ ಮಾಡಿಕೊಳ್ಳಬೇಕಷ್ಟೇ.

ಬೌದ್ಧಿಕ ಜ್ಞಾನ ಮತ್ತು ಸಾಕ್ಷಾತ್ಕಾರ ಎಂಬ ಎರಡು ಶಬ್ದಗಳಿವೆ. ಬೌದ್ಧಿಕ ಜ್ಞಾನ ಸಾಕ್ಷಾತ್ಕಾರದ ಪರಿಧಿಯೊಳಗೆ ಬರುತ್ತದೆ, ಆದರೆ ಸಾಕ್ಷಾತ್ಕಾರ ಅದನ್ನು ಮೀರಿರುವುದು. ಆದ್ದರಿಂದ ಬೌದ್ಧಿಕ ಜ್ಞಾನ ಸಾಕಾಗುವುದಿಲ್ಲ.

ಪ್ರತಿಯೊಬ್ಬ ವ್ಯಕ್ತಿಯೂ ಈ ಸಿದ್ಧಾಂತ ಸರಿಯೆಂದು ಹೇಳಬಹುದು, ಆದರೆ ಅದು ಸಾಕ್ಷಾತ್ಕಾರವಲ್ಲ. ನಾವೆಲ್ಲ ಹೇಳಬಹುದು ಇದೆಲ್ಲ ಕನಸು ಎಂಬುದು ನಮಗೆ ಗೊತ್ತು ಎಂದು, ಆದರೆ ಅದು ಸಾಕ್ಷಾತ್ಕಾರವಲ್ಲ. ಈ ಕನಸು ಮಾಯವಾಗಬೇಕು - ಒಂದು ಕ್ಷಣವಾದರೂ - ಅದು ಸಾಕ್ಷಾತ್ಕಾರ. ಅದು ತಾನೇ ತಾನಾಗಿ ಸುಙರಿಸುತ್ತದೆ. ನೀವು ಪ್ರಯತ್ನ ಪಟ್ಟರೆ ಅದು ಬಂದೇ ಬರುತ್ತದೆ.

ಈ ಸಾಕ್ಷಾತ್ಕಾರವಾದಾಗ ದೇಹಭಾವನೆ ಲಿಂಗಭಾವನೆಗಳೆಲ್ಲ ದೀಪ ಆರಿ ಹೋದಂತೆ ನಾಶವಾಗುತ್ತವೆ. ಆಗ ನಿಮಗೇನಾಗುತ್ತದೆ? ಕರ್ಮ ಇನ್ನೂ ಸ್ವಲ್ಪ ಉಳಿದಿದ್ದರೆ ಈ ಪ್ರಪಂಚ ಮತ್ತೆ ಕಾಣಿಸಿಕೊಳ್ಳುತ್ತದೆ - ಆದರೆ ಹಿಂದಿನ ರೀತಿಯಲ್ಲಲ್ಲ. ಅದು ಏನು ಎಂಬುದು ನಿಮಗೆ ಗೊತ್ತು, ನಿಮಗೆ ಬಂಧನವೆಂಬುದೇ ಇರುವುದಿಲ್ಲ. ಕಣ್ಣುಗಳಿರುವ ವರೆಗೂ ನೀವು ನೋಡಲೇ ಬೇಕು, ಕಿವಿಗಳಿರುವವರೆಗೂ ಕೇಳಲೇ ಬೇಕು - ಆದರೆ ಮೊದಲಿನಂತಲ್ಲ.

ನಾನು ಮರೀಚಿಕೆಯ ಬಗ್ಗೆ ಬಹಳಷ್ಟು ಓದಿದ್ದೆ. ಆದರೆ ನಾನವನ್ನು ನೋಡಿರಲಿಲ್ಲ. ನಾಲ್ಕು ವರ್ಷಗಳ ಹಿಂದೆ ನಾನು ಪಶ್ಚಿಮ ಭಾರತದಲ್ಲಿ ಪ್ರಯಾಣ ಮಾಡುತ್ತಿದ್ದೆ ಹಾಗೂ ನಿಧಾನವಾಗಿ ಹೋಗುತ್ತಿದ್ದೆ. ಆದ್ದರಿಂದ ಆ ಪ್ರದೇಶವನ್ನು ದಾಟಲು ನನಗೆ ಸುಮಾರು ಒಂದು ತಿಂಗಳು ಬೇಕಾಯಿತು. ನಾನೂ ಪ್ರತಿನಿತ್ಯವೂ ಸುಂದರ ಕೊಳ ಗಳನ್ನು ನೋಡುತ್ತಿದ್ದೆ. ಅದರ ದಡದ ಮೇಲಿರುವ ಮರಗಳ ನೆರಳು ಕಾಣಿಸುತ್ತಿತ್ತು. ಗಾಳಿಯಿಂದ ಎಲ್ಲ ಓಲಾಡುತ್ತಿದ್ದವು. ಪಶುಪಕ್ಷಿಗಳು ಓಡಾಡುತ್ತಿದ್ದವು. ಪ್ರತಿದಿನ ನಾನು ಈ ದೃಶ್ಯವನ್ನು ನೋಡುತ್ತಿದ್ದೆ ಮತ್ತು ಎಂತಹ ಸುಂದರವಾದ ಪ್ರದೇಶ ಇದು ಎಂದು ಭಾವಿಸಿದೆ. ಆದರೆ ನಾನು ಯಾವುದೊ ಹಳ್ಳಿಯನ್ನು ತಲುಪಿದಾಗ ಅದೆಲ್ಲ ಮರುಳು ಎಂದು ತಿಳಿದುಬಂತು. ಇದು ಹೇಗೆ ಸಾಧ್ಯ ಎಂದು ಭಾವಿಸಿದೆ.

ಒಂದುದಿನ ನನಗೆ ತುಂಬ ಬಾಯಾರಿಕೆಯಾಗಿತ್ತು, ಆ ಕೊಳದಲ್ಲಿ ನೀರು ಕುಡಿಯೋಣವೆಂದು ಹೋದೆ. ಆದರೆ ನಾನು ಹತ್ತಿರಹೋದಾಗ ಅದು ಪತ್ತೆಯೇ ಇಲ್ಲ. ಆಗ ಕೂಡಲೆ ನನಗೆ ಹೊಳೆಯಿತು: “ನಾನು ಹಿಂದೆಲ್ಲ ಓದಿದ ಮರೀಚಿಕೆ ಎಂದರೆ ಇದೇ.” ಆದರೆ ಆಶ್ಚರ್ಯ ವಿಷಯವೇನೆಂದರೆ, ನಾನು ಒಂದು ತಿಂಗಳು ಪ್ರಯಾಣ ಮಾಡುತ್ತಿದ್ದೆ, ಆದರೂ ಅದು ಮರೀಚಿಕೆ ಎಂದು ಗೊತ್ತಾಗಲಿಲ್ಲ - ಈಗ ಒಂದೇ ಕ್ಷಣದಲ್ಲಿ ಅದು ಮಾಯವಾಯಿತು. ನಾನು ಪುಸ್ತಕಗಳಲ್ಲಿ ಯಾವುದರ ಬಗ್ಗೆ ಓದಿ ದ್ದೆನೊ ಆ ಮರೀಚಿಕೆ ಇದೇ ಎಂಬುದನ್ನು ತಿಳಿದು ನನಗೆ ತುಂಬ ಸಂತೋಷವಾಯಿತು.

ಮರುದಿನ ಬೆಳಿಗ್ಗೆ ಮತ್ತೆ ಕೊಳ ಕಾಣಿಸಿತು, ತಕ್ಷಣವೇ ಇದು ಮರೀಚಿಕೆ ಎಂದು ಮನಸ್ಸಿಗೆ ಹೊಳೆಯಿತು. ಇಡಿ ತಿಂಗಳೆಲ್ಲ ಮರೀಚಿಕೆಯನ್ನು ನೋಡುತ್ತಿದ್ದರೂ ಸತ್ಯ ಮತ್ತು ಮರಿಚಿಕೆಯ ಭೇದ ನನಗೆ ತಿಳಿಯಲೇ ಇಲ್ಲ. ಆದರೆ ಆ ಒಂದು ಕ್ಷಣದಲ್ಲಿ ಅದು ಹೊಳೆಯಿತು.

ಅಂದಿನಿಂದ ಮರೀಚಿಕೆ ಕಂಡಾಕ್ಷಣ ‘ಇದು ಮರೀಚಿಕೆ’ ಎಂದು ಹೇಳುತ್ತೇನೆ, ಅದರ ಪ್ರಭಾವಕ್ಕೆ ಒಳಗಾಗುವುದಿಲ್ಲ. ಇದರಂತೆಯೇ ಜಗತ್ತು - ಒಮ್ಮೆ ಎಲ್ಲ ಮಾಯ ವಾಗುತ್ತದೆ, ಅನಂತರ ಪೂರ್ವಕರ್ಮಕ್ಕನುಸಾರವಾಗಿ ಬದುಕಬೇಕಾದರೂ ಅದ ರಿಂದ ಮೋಹಿತರಾಗುವುದಿಲ್ಲ.

ಎರಡು ಚಕ್ರದ ಗಾಡಿಯನ್ನು ತೆಗೆದುಕೊಳ್ಳಿ. ಒಂದು ಚಕ್ರವನ್ನು ಅಚ್ಚಿನಿಂದ ಕತ್ತರಿಸಿ ಹಾಕಿದ ಮೇಲೂ ಇನ್ನೊಂದು ಚಕ್ರವು ಸ್ವಲ್ಪ ಕಾಲ ಸುತ್ತುತ್ತದೆ. ಇದು ಹಿಂದಿನ ವೇಗದ ಪ್ರಭಾವದಿಂದ. ದೇಹವು ಒಂದು ಚಕ್ರ, ಜೀವ ಇನ್ನೊಂದು ಚಕ್ರ. ಅವು ಅಜ್ಞಾನ ವೆಂಬ ಅಚ್ಚಿನಲ್ಲಿ ಕೂಡಿಕೊಂಡಿವೆ. ಜ್ಞಾನವು ಅಚ್ಚನ್ನು ಕತ್ತರಿಸುವ ಕೊಡಲಿ. ಜೀವದ ಗತಿ ತಕ್ಷಣ ನಿಲ್ಲುವುದು - ಅದು ಎಲ್ಲ ವ್ಯರ್ಥ ಕನಸುಗಳನ್ನು ಪರಿತ್ಯಜಿಸುವುದು.

ಆದರೆ ಹಿಂದಿನ ವೇಗದ ಪರಿಣಾಮ ದೇಹದ ಮೇಲಿರುತ್ತದೆ. ಅದು ಏನೇನೊ ಕರ್ಮ ಮಾಡುತ್ತ ಸ್ವಲ್ಪ ಕಾಲ ಓಡಿ ನಂತರ ಬೀಳುತ್ತದೆ. ಆದರೆ ಒಳ್ಳೆಯ ಪರಿಣಾಮವೇ ಉಳಿದಿರುತ್ತದೆ, ಆದ್ದರಿಂದ ದೇಹ ಒಳ್ಳೆಯದನ್ನು ಮಾತ್ರ ಮಾಡುತ್ತದೆ. ಇದರಿಂದ ಮುಕ್ತ ಪುರುಷನನ್ನು ಸಾಮಾನ್ಯ ವ್ಯಕ್ತಿಯಿಂದ ಬೇರ್ಪಡಿಸಬಹುದು. ಮುಕ್ತ ಪುರುಷನು ಕೆಟ್ಟದ್ದನ್ನು ಮಾಡುವುದು ಸಾಧ್ಯವೇ ಇಲ್ಲ.

ನೀವು ಮುಕ್ತರಾದಾಗ ಇಡೀ ಸ್ವಪ್ನ ಮಾಯವಾಗುತ್ತದೆ, ಸತ್ಯ ಮತ್ತು ಮರೀಚಿಕೆ ಗಳ ಭೇದ ನಿಮಗೆ ತಿಳಿದುಬರುತ್ತದೆ. ಮರೀಚಿಕೆ ನಿಮಗಿನ್ನು ಬಂಧನವಾಗುವುದಿಲ್ಲ. ಎಂಥ ಭಯಂಕರ ವಿಷಯವೂ ನಿಮ್ಮನ್ನು ಸ್ತಂಭಿತಗೊಳಿಸುವುದಿಲ್ಲ. ಪರ್ವತವೇ ನಿಮ್ಮ ಮೇಲೆ ಬೀಳಬಹುದು, ನೀವದನ್ನು ಲೆಕ್ಕಿಸುವುದಿಲ್ಲ. ಅದು ಮರೀಚಿಕೆಯೆಂಬುದು ನಿಮಗೆ ಗೊತ್ತಿರುತ್ತದೆ.

