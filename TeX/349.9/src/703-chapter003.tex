
\chapter{: ಭಾರತದ ಪತ್ರಿಕಾವರದಿಗಳು}

\begin{center}
\textbf{ಒಬ್ಬ ಬಂಗಾಳಿ ಸಾಧು\supskpt{\footnote{1. ಶ‍್ರೀ ಶಂಕರೀಪ್ರಸಾದ ಬಸು, “ಮದರಾಸಿನಲ್ಲಿ ಸ್ವಾಮಿ ವಿವೇಕಾನಂದರು: ೧೮೯೨-೧೮೯೩- ಕೆಲವು ಹೊಸ ಸಂಶೋಧನೆ”, ಪ್ರಬುದ್ಧ ಭಾರತ, ೧೯೭೪, ಪುಟಗಳು ೨೯೬-೯೮.}}}
\end{center}

\begin{center}
(ಮಧುರಾ ಮೇಲ್, ೨೮ ಜನವರಿ ೧೮೯೩)
\end{center}

\begin{center}
\textbf{ಹಿಂದೂಧರ್ಮ ಮತ್ತು ಸಮಾಜವಿಜ್ಞಾನಗಳ ಬಗ್ಗೆ ಒಬ್ಬ ಬಂಗಾಳಿ ಸಾಧು}
\end{center}

ಕೊಲ್ಕತ್ತ ವಿಶ್ವವಿದ್ಯಾನಿಲಯದ ಕಲಾಸ್ನಾತಕೋತ್ತರ ಪದವೀಧರರಾದ ಸುಮಾರು ಮೂವತ್ತೆರಡು ವರ್ಷ ವಯಸ್ಸಿನ ಬಂಗಾಳಿ ತರುಣ ಸಂನ್ಯಾಸಿಯೊಬ್ಬರನ್ನು ಕಳೆದ ವಾರ ಟ್ರಿಪ್ಲಿಕೇನ್ ಲಿಟರರಿ ಸೊಸೈಟಿಯಲ್ಲಿ ಸುಮಾರು ಒಂದು ನೂರು ವಿದ್ಯಾವಂತ ಭಾರತೀಯರು - ದಿವಾನ್ ಬಹದ್ದೂರ್ ರಂಗನಾಥರಾವ್ ಅವರನ್ನೂ ಒಳಗೊಂಡಂತೆ -ಸಂದರ್ಶಿಸಿದರು. ಸಾಧುವು ಹೇಳಿದ್ದರ ಸಾರಾಂಶವನ್ನು ಇಂಡಿಯನ್ ಸೋಷಿಯಲ್ ರಿಫಾರ್ಮರ್ ಪತ್ರಿಕೆಯಲ್ಲಿ ಪ್ರಕಟಿಸಿದ್ದೇವೆ. ಅದರ ಕೆಲವು ಭಾಗಗಳನ್ನು ನಾವು ಇಲ್ಲಿ ಕೊಡುತ್ತಿದ್ದೇವೆವೆವು.\footnote{2. ಸ್ವಾಮಿ ವಿವೇಕಾನಂದರ ಜೀವನದ ಈ ಅವಧಿಯಲ್ಲಿ ಅವರ ಕಲ್ಪನೆಗಳ ಬಗ್ಗೆ ಲಭ್ಯವಿರುವ ಏಕಮಾತ್ರ ಪದಶಃ ವರದಿ. ಇದರಲ್ಲಿ ಸ್ವಾಮಿ ವಿವೇಕಾನಂದರ ನಾಮನಿರ್ದೇಶನವನ್ನು ಮಾಡದಿದ್ದರೂ, ಅವರೆಂದೂ ಗ್ರಹಿಸದೆ ಇದ್ದ ಎಂ.ಎ. ಪದವಿಯನ್ನು ಹಾಗೂ ಆಗಿದ್ದುದಕ್ಕಿಂತ ಎರಡು ವರ್ಷ ಹೆಚ್ಚು ವಯಸ್ಸನ್ನು ಸೂಚಿಸಿದ್ದರೂ - ಈ ಬಂಗಾಳಿ ಸಾಧು ಸ್ವಾಮಿ ವಿವೇಕಾನಂದರೇ ಎನ್ನುವ ಸಂದೇಹಕ್ಕವಕಾಶವಿಲ್ಲದಂತಹ ಸಾಕ್ಷ್ಯಾಧಾರಗಳಿವೆ. ಅದೂ ಅಲ್ಲದೆ, ದಿನಾಂಕವು ಸ್ವಾಮಿಗಳು ಮದರಾಸಿನಲ್ಲಿದ್ದ ಕಾಲಕ್ಕೆ ಸರಿಯಾಗಿ ಹೊಂದುತ್ತದೆ; ೧೯೦೨ ಜುಲೈ ೧೩ರ ಇಂಡಿಯನ್ ಸೋಷಿಯಲ್ ರಿಫಾರ್ಮರ್ನಲ್ಲಿ ಹಿಂದಿನ ಉಲ್ಲೇಖನವು ಇದನ್ನು ಇನ್ನೂ ಸ್ಥಿರಪಡಿಸುತ್ತದೆ. ಆದರೆ ೧೮೯೨-೯೩ರ ಈ ಪತ್ರಿಕೆಯ ಒಂದು ಪ್ರತಿಯೂ ಇಂದು ಸಿಕ್ಕುವುದಿಲ್ಲ.}

\begin{center}
\textbf{ವೈದಿಕ ಧರ್ಮ}
\end{center}

ಪರಿಪೂರ್ಣ ಧರ್ಮವೆಂದರೆ ವೈದಿಕ ಧರ್ಮ. ವೇದಗಳಲ್ಲಿ ಕಡ್ಡಾಯ ಮತ್ತು ಐಚ್ಚಿಕ ಎಂಬ ಎರಡು ಭಾಗಗಳಿವೆ. ಕಡ್ಡಾಯವಾದ ಕಟ್ಟಳೆಗಳಿಂದ ನಾವು ಚಿರಂತನವಾಗಿ ನಿಯುಕ್ತರಾಗಿರುತ್ತೇವೆ. ಹಿಂದೂಧರ್ಮವೆಂದರೆ, ಅವುಗಳೇ ಐಚ್ಛಿಕ ಭಾಗಗಳು ಹಾಗಲ್ಲ. ಅವು ಬದಲಾಗುತ್ತಿರುತ್ತವೆ; ಕಾಲಧರ್ಮವನ್ನನುಸರಿಸಿ ಋಷಿಗಳು ಅವುಗಳನ್ನು ಬದಲಿ ಸುತ್ತ ಬಂದಿರುವರು. ಒಂದು ಕಾಲದಲ್ಲಿ ಬ್ರಾಹ್ಮಣರು ಗೋಮಾಂಸ ತಿನ್ನುತ್ತಿದ್ದರು, ಶೂದ್ರರನ್ನು ಮದುವೆಯಾಗುತ್ತಿದ್ದರು. ಅತಿಥಿಯೊಬ್ಬರನ್ನು ಸಂತೋಷಪಡಿಸುವುದಕ್ಕಾಗಿ ಕರುವೊಂದನ್ನು ಕೊಲ್ಲಲಾಗುತ್ತಿತ್ತು. ಶೂದ್ರರು ಬ್ರಾಹ್ಮಣರಿಗಾಗಿ ಅಡುಗೆಮಾಡುತ್ತಿದ್ದರು. ಬ್ರಾಹ್ಮಣ ಪುರುಷನು ಮಾಡಿದ ಅಡುಗೆಯನ್ನು ಮೈಲಿಗೆಯೆಂದು ಭಾವಿಸಲಾಗುತ್ತಿತ್ತು. ಆದರೆ ನಾವು ಇಂದಿನ ಕಾಲಕ್ಕೆ ಹೊಂದುವ ಹಾಗೆ ನಮ್ಮ ಅಭ್ಯಾಸಗಳನ್ನು ಬದಲಿಸಿಕೊಂಡಿ ದ್ದೇವೆ. ಮನುವಿನ ಕಾಲದಿಂದ ಇಲ್ಲಿಯವರೆಗೆ ನಮ್ಮ ಜಾತಿ ನಿಯಮಗಳು ಬದಲಾಗುತ್ತ ಬಂದಿದ್ದರೂ, ಮನು ಏನಾದರೂ ಈ ಹೊತ್ತು ನಮ್ಮ ನಡುವೆ ಬಂದರೆ, ನಮ್ಮನ್ನು ಈಗಲೂ ಹಿಂದೂಗಳೆಂದೇ ಕರೆಯುವನು. ಜಾತಿ ಎನ್ನುವುದು ಒಂದು ಸಾಮಾಜಿಕ ಸಂಸ್ಥೆಯೇ ಹೊರತು ಧಾರ್ಮಿಕವಲ್ಲ. ಅದು ನಮ್ಮ ಸಮಾಜದ ಸಹಜ ವಿಕಾಸದ ಫಲಿತ. ಒಂದು ಕಾಲದಲ್ಲಿ ಅದು ಸುಲಭವೆಂದೂ ಆವಶ್ಯಕವೆಂದೂ ಕಂಡಿತ್ತು. ಅದು ತನ್ನ ಉದ್ದೇಶವನ್ನು ಸಫಲಗೊಳಿಸಿಕೊಂಡಾಯಿತು. ಅದು ಇಲ್ಲದೆ ಹೋಗಿದ್ದರೆ, ನಾವು ಎಂದೋ ಮಹಮ್ಮದೀ ಯರಾಗಿಬಿಟ್ಟಿರುತ್ತಿದ್ದೆವು. ಈಗ ಅದು ನಿರುಪಯುಕ್ತ. ಅದನ್ನು ತೆಗೆದುಹಾಕಬಹುದು. ಹಿಂದೂಧರ್ಮಕ್ಕೆ ಇನ್ನು ಜಾತಿವ್ಯವಸ್ಥೆಯ ಆಸರೆ ಬೇಕಿಲ್ಲ. ಬ್ರಾಹ್ಮಣನೊಬ್ಬನು ಯಾರೊಂದಿಗೆ ಬೇಕಾದರೂ - ಪರಯನೊಂದಿಗೆ ಸಹ - ಸಹಪಂಕ್ತಿಭೋಜನ ಮಾಡಬಹುದು. ಆ ಕಾರಣವಾಗಿ ಅವನ ಆಧ್ಯಾತ್ಮಿಕತೆ ನಾಶವಾಗುವುದಿಲ್ಲ. ಪರಯನ ಸ್ಪರ್ಶ ದಿಂದ ಹಾಳಾಗುವ ಆಧ್ಯಾತ್ಮಿಕತೆ ಆಧ್ಯಾತ್ಮಿಕತೆಯೇ ಅಲ್ಲ; ಅದು ಶೂನ್ಯವೇ ಸರಿ. ಬ್ರಾಹ್ಮಣನ ಆಧ್ಯಾತ್ಮಿಕತೆಯ ಪ್ರವಾಹ ತುಂಬಿ ಹರಿಯುತ್ತಿರಬೇಕು. ಒಬ್ಬ ಪರಯನನ್ನಲ್ಲ, ಅವನನ್ನು ಮುಟ್ಟುವ ಸಾವಿರಾರು ಪರಯರ ಆಂತರ್ಯದಲ್ಲಿ ಆಧ್ಯಾತ್ಮಿಕತೆಯನ್ನು ಉದ್ದೀಪಿಸುವ ಹಾಗೆ ಅವನ ಆಧ್ಯಾತ್ಮಿಕತೆ ಜ್ವಲಂತವಾಗಿರಬೇಕು, ಪ್ರದೀಪ್ತವಾಗಿರಬೇಕು. ಪ್ರಾಚೀನ ಋಷಿಗಳು ಆಹಾರದ ವಿಚಾರದಲ್ಲಿ ಯಾವ ವೈಶಿಷ್ಟ್ಯವನ್ನಾಗಲಿ ವ್ಯತ್ಯಾಸವನ್ನಾಗಲಿ ಇಟ್ಟು ಕೊಂಡಿರಲಿಲ್ಲ. ಕೀಳುಜಾತಿಯವನೊಬ್ಬನನ್ನು ನೋಡಿದ ಮಾತ್ರಕ್ಕೆ ತನ್ನ ಅಲ್ಪ ಆಧ್ಯಾತ್ಮಿಕತೆ ನಷ್ಟವಾಗಿಬಿಡುತ್ತದೆ ಎಂದು ಯಾರಿಗೆ ಅನ್ನಿಸುವುದೋ ಅವರು ಪರಯನ ಬಳಿಗೆ ಬಾರ ದಿರುವ ಮೂಲಕ ತಮ್ಮ ಮೌಲಿಕವಾದ ಆ ಅಲ್ಪವನ್ನು ತಮ್ಮೊಳಗೇ ಇಟ್ಟುಕೊಂಡಿರಲಿ.

\begin{center}
\textbf{ಹಿಂದೂ ಜೀವನಾದರ್ಶ}
\end{center}

ನಿವೃತ್ತಿ\footnote{1. ನಿವೃತ್ತಿ, ಪ್ರವೃತ್ತಿ ಎಂಬ ಎರಡು ಪದಗಳು ಹಿಂದೂ ತತ್ತ್ವಜ್ಞಾನದಲ್ಲಿಯ ಮುಖ್ಯವಾದ ಕಲ್ಪನೆಗಳು; ಸ್ವಾಮಿ ವಿವೇಕಾನಂದರು ಬಹಳಷ್ಟು ಸಲ ಅವುಗಳ ಪರಂಪರಾಗತ ನಿಲುವುಗಳನ್ನು ವಿವರವಾಗಿ ಪ್ರತಿಪಾದನೆ ಮಾಡಿದ್ದಾರೆ (ಉದಾಹರಣೆಗೆ, ಕರ್ಮಯೋಗದ ಆರನೆಯ ಅಧ್ಯಾಯವನ್ನು ನೋಡಿ). ಆದರೆ ಇಲ್ಲಿ ಇಂಡಿಯನ್ ಸೋಷಿಯಲ್ ರಿಫಾರ್ಮರ್ ಪತ್ರಿಕಾ ವರದಿಗಾರನು ಅವರದೆಂದು ಹೇಳಿರುವ ಅವುಗಳ ಪ್ರತಿಪಾದನೆ ಸ್ವಾಮಿಗಳು ಇನ್ನಿತರ ಕಡೆಗಳಲ್ಲಿ ಹೇಳಿರುವುದಕ್ಕೆ ಅನ್ವಯವಾಗಿಲ್ಲ.} ಯೇ ಹಿಂದೂವಿನ ಜೀವನಾದರ್ಶ. ನಿವೃತ್ತಿ ಎಂದರೆ ತಮೋರೂಪದ ರಾಗ, ದ್ವೇಷ, ಲೋಭ ಮುಂತಾದ ಕೆಟ್ಟ ಮನೋವಿಕಾರಗಳನ್ನು ನಿಗ್ರಹಿಸಿ ಜಯಿಸುವುದು ಎಂದರ್ಥ. ಎಲ್ಲಾ ಮನೋಭಿಲಾಷೆಗಳನ್ನೂ ಜಯಿಸುವುದು ಎಂದಲ್ಲ. ಐಂದ್ರಿಯಕವಾದ ದುರಾಸೆಗಳನ್ನು ನಾಶಪಡಿಸುವುದು ಎಂದಷ್ಟೇ. ಪ್ರತಿಯೊಬ್ಬನೂ ತನ್ನ ಸಹಜೀವಿಗಳನ್ನು ಪ್ರೀತಿಸಬೇಕು, ಅವುಗಳೊಂದಿಗೆ ಸಹಾನುಭೂತಿಯಿಂದಿರಬೇಕು. ಯಾವನು ತನ್ನ ಎಲ್ಲ ಸ್ವಾರ್ಥಪರ ಭಾವಗಳನ್ನು ನಿಗ್ರಹಿಸಿ ತನ್ನ ಜೀವನವನ್ನು ಇತರರ ಒಳಿತಿಗಾಗಿ ಮುಡಿ ಪಾಗಿಟ್ಟಿ ರುವನೋ ಅವನು ಸಂನ್ಯಾಸಿ. ಅವನು ಎಲ್ಲರನ್ನೂ ಪ್ರೀತಿಸುವನು. ಪ್ರವೃತ್ತಿ ಎಂದರೆ ಭಗವಂತನನ್ನೂ ಆತನ ಸೃಷ್ಟಿಯೆಲ್ಲವನ್ನೂ ಪ್ರೇಮಿಸುವುದು. ಸಂನ್ಯಾಸಿಗಳಿಗೂ ಹೊಟ್ಟೆ ತುಂಬಬೇಕಲ್ಲ! ತಮ್ಮ ಕೆಲಸಕ್ಕಾಗಿ ಸಹಸ್ರಾರು ಪೌಂಡ್ ವಾರ್ಷಿಕ ಆದಾಯವಿರುವ - ಆ ಎಲ್ಲ ಗಳಿಕೆಯನ್ನೂ ತಮ್ಮ ಸುಖಸೌಲಭ್ಯಗಳ ಮೇಲೆ, ಹೆಂಡತಿಮಕ್ಕಳ ಮೇಲೆ ವ್ಯಯಿ ಸುವ - ಕ್ರೈಸ್ತ ಬಿಷಪ್ ಆರ್ಚ್ಬಿಷಪ್ಗಳಂತಲ್ಲ ಅವರು. ಸಂನ್ಯಾಸಿಗೆ ಬೇಕಾದ್ದು ಒಂದು ತುತ್ತು ಅನ್ನ; ಆಗ ಅವನು ತನ್ನೆಲ್ಲ ಜ್ಞಾನವನ್ನೂ ಸೇವೆಯನ್ನೂ ಸಾರ್ವಜನಿಕರಿಗಾಗಿ ತೆರೆದಿಡುವನು. ಅವನೊಬ್ಬ ಅಲೆಮಾರಿ ಪ್ರಚಾರಕ. ವ್ಯಕ್ತಿಗಳಾಗಲಿ, ಸಮಾಜವಾಗಲಿ ‘ಮೃಗೀಯತೆಯಿಂದ ಮಾನವತೆಗೆ, ಅನಂತರ ದಿವ್ಯತೆಗೆ’ ಕಷ್ಟಪಟ್ಟು ಮೇಲೇರಬೇಕು. ಹಿಂದೂಗಳಲ್ಲಿನ ಅತ್ಯಂತ ಕೀಳು ಜಾತಿಯ ಪರಯನೂ ಸಹ ಅಂತಹುದೇ ಸಾಮಾಜಿಕ ಸ್ತರದಲ್ಲಿರುವ ಬ್ರಿಟನ್ನಿನವನಷ್ಟು ಮೃಗೀಯತೆಯನ್ನು ಹೊಂದಿರುವುದಿಲ್ಲ. ಇದು ಪ್ರಾಚೀನವೂ ಉತ್ಕೃಷ್ಟವೂ ಆದ ಧಾರ್ಮಿಕ ನಾಗರಿಕತೆಯ ಫಲ. ಉನ್ನತ ಆಧ್ಯಾತ್ಮಿಕ ಸ್ತರಕ್ಕೆ ಆಗುವ ಈ ವಿಕಾಸ ಶಿಸ್ತಿನ ಹಾಗೂ ಶಿಕ್ಷಣದ ಮೂಲಕ ಮಾತ್ರವೇ ಸಾಧ್ಯ.

\begin{center}
\textbf{ಶ್ರಾದ್ಧವೆಂಬ ಕ್ರಿಯಾವಿಧಿ\supskpt{\footnote{1. ಗತಿಸಿದ ಪಿತೃಗಳಿಗೆ ಮತ್ತು ಬಂಧುಗಳಿಗೆ ಆಹಾರ ಪಾನೀಯಗಳನ್ನು ನಿವೇದಿಸುವ ಒಂದು ಧಾರ್ಮಿಕ ಕ್ರಿಯೆ.}}}
\end{center}

“ಶಿಕ್ಷಣದ ಹಾದಿಯಲ್ಲಿ ಅಡ್ಡಿಯಾಗುವ ಜಾತಿ, ಬಾಲ್ಯವಿವಾಹ ಮುಂತಾದ ಪ್ರತಿ ಯೊಂದು ಆಚಾರವನ್ನೂ ತಡಮಾಡದೆ ಬಡಿದು ಹಾಕಬೇಕು. ‘ಶ್ರಾದ್ಧ’ ದ ಆಚರಣೆಯು ಸಹ ವ್ಯರ್ಥ ಕಾಲಹರಣವೆಂದು ತೋರಿದರೆ, ಆ ಕಾಲವನ್ನು ಇನ್ನೂ ಚೆನ್ನಾಗಿ ಆತ್ಮ ಶಿಕ್ಷಣಕ್ಕೆ ಬಳಸಿಕೊಳ್ಳುವಂತಿದ್ದರೆ, ಅದನ್ನೂ ಬಿಟ್ಟುಬಿಡಬಹುದು. ಇಲ್ಲವಾದರೆ ‘ಶ್ರಾದ್ಧ’ವನ್ನು ಬಿಡಕೂಡದು. ಮಂತ್ರಗಳ ಅರ್ಥವು ಆತ್ಮೋನ್ನತಿಯನ್ನುಂಟುಮಾಡುವಂಥದು. ನಮ್ಮ ತಂದೆತಾಯಿಗಳು ನಮಗಾಗಿ ಪಟ್ಟ ಕಷ್ಟ, ಪರಿಶ್ರಮಗಳನ್ನು ಅವು ಒಳಗೊಂಡಿವೆ. ಅದರ ಆಚರಣೆಯು ಯಾರ ಪುಣ್ಯವನ್ನು ನಾವು ಪಡೆದುಕೊಂಡಿರುವೆವೋ ಆ ಪಿತೃಗಳೆಲ್ಲರ ನೆನಪಿಗಾಗಿ ಸಲ್ಲಿಸುವ ಮರ್ಯಾದೆ. ಶ್ರಾದ್ಧಕ್ಕೂ ನಮ್ಮ ಮುಕ್ತಿಗೂ ಸಂಬಂಧವಿಲ್ಲ. ಅದರೂ ಸಹ ತನ್ನ ದೇಶವನ್ನು, ಧರ್ಮವನ್ನು, ಪೂರ್ವಿಕರಾದ ಪುಣ್ಯ ಪುರುಷರನ್ನು ಪ್ರೀತಿಸುವ ಯಾವ ಹಿಂದುವೂ ಶ್ರಾದ್ಧವನ್ನು ಬಿಡಕೂಡದು. ಬಾಹ್ಯ ಆಚರಣೆಗಳು, ಬ್ರಾಹ್ಮಣರಿಗೆ ಭೋಜನ ಮಾಡಿಸುವುದು ಮುಖ್ಯವಲ್ಲ. ಶ್ರಾದ್ಧದ ಭೋಜನಕ್ಕೆ ಅರ್ಹರಾಗಿರುವ ಬ್ರಾಹ್ಮಣರು ಇತ್ತೀಚಿನ ದಿನಗಳಲ್ಲಿ ಸಿಕ್ಕುವುದಿಲ್ಲ. ಚೆನ್ನಾಗಿ ಊಟಮಾಡಬಲ್ಲ ಬ್ರಾಹ್ಮಣರಲ್ಲ, ಶಿಷ್ಯರಿಗೆ ಅನ್ನದಾನವಿತ್ತು ವೇದಾಧ್ಯಯನ ಮಾಡಿಸುತ್ತಿರುವ ಬ್ರಾಹ್ಮಣರು ಮಾತ್ರ ಶ್ರಾದ್ಧದ ಭೋಜನಕ್ಕೆ ಅರ್ಹರು. ಇಂದಿನ ದಿನಗಳಲ್ಲಿ ಶ್ರಾದ್ಧವನ್ನು ಮಾನಸಿಕವಾಗಿ ಮಾಡಬಹುದು.

\begin{center}
\textbf{ಸ್ತ್ರೀ ಶಿಕ್ಷಣ}
\end{center}

ಸ್ತ್ರೀಯರ ರಕ್ಷಣೆಯಲ್ಲಿ ನಮಗಿರುವ ಪಕ್ಷಪಾತದೃಷ್ಟಿಯು, ನಾವು ಹಿಂದೂಗಳು ನಮ್ಮ ರಾಷ್ಟ್ರೀಯ ಮೌಲ್ಯಗಳಲ್ಲಿ ಅವನತಿ ಹೊಂದುತ್ತಿರುವೆವು, ‘ಮೃಗೀಯ ಸ್ಥಿತಿ’ ಯನ್ನು ತಲುಪುತ್ತಿರುವೆವು ಎಂಬುದನ್ನು ತೋರಿಸುತ್ತದೆ. ಸಮಸ್ತ ಸ್ತ್ರೀಯರನ್ನೂ ತನ್ನ ತಾಯಿ ಅಥವಾ ಸೋದರಿಯಂತೆ ಭಾವಿಸುವ ಹಾಗೆ ಪ್ರತಿಯೊಬ್ಬ ಮನುಷ್ಯನೂ ತನ್ನ ಮನಸ್ಸನ್ನು ಶಿಸ್ತಿಗೆಗೊಳಪಡಿಸಬೇಕು. ಸ್ತ್ರೀಯರಿಗೆ ಓದುವುದಕ್ಕೆ, ಪುರುಷರಷ್ಟೇ ಒಳ್ಳೆಯ ವಿದ್ಯಾಭ್ಯಾಸವನ್ನು ಪಡೆಯುವುದಕ್ಕೆ ಸ್ವಾತಂತ್ರ್ಯವಿರಬೇಕು. ಅಜ್ಞಾನ ಹಾಗೂ ದಾಸ್ಯಗಳಿರುವಾಗ ವ್ಯಕ್ತಿಗಳ ಪುರೋಭಿವೃದ್ಧಿ ಅಸಾಧ್ಯ.

\begin{center}
\textbf{ಹಿಂದೂಗಳ ವಿಮೋಚನೆ}
\end{center}

ಸಹಸ್ರಾರು ವರ್ಷಗಳ ದಾಸ್ಯದಿಂದಾಗಿ, ಹಿಂದೂಗಳು ಇಂದು ಅವನತಿಯ ಸ್ಥಿತಿಯಲ್ಲಿ ದ್ದಾರೆ. ಅವರು ತಮ್ಮ ಆತ್ಮಗೌರವವನ್ನೇ ಮರೆತಿದ್ದಾರೆ. ಪ್ರತಿಯೊಬ್ಬ ಇಂಗ್ಲಿಷ್ ಬಾಲಕ ನಿಗೂ ತನ್ನ ಪ್ರಾಮುಖ್ಯತೆಯನ್ನು ಅರಿತುಕೊಳ್ಳಲು ಕಲಿಸಲಾಗುತ್ತದೆ; ತಾನು ಈ ಭೂಮಿ ಯನ್ನೇ ಜಯಿಸಿದ ಶ್ರೇಷ್ಠ ಜನಾಂಗವೊಂದರ ಸದಸ್ಯನೆಂದು ಅವನು ಯೋಚಿಸುತ್ತಾನೆ. ಹಿಂದೂ ತನ್ನ ಬಾಲ್ಯಾದಾರಭ್ಯ ಇದಕ್ಕೆ ತದ್ವಿರುದ್ಧವಾಗಿ ತಾನೊಬ್ಬ ಗುಲಾಮತನಕ್ಕೇ ಹುಟ್ಟಿದವನೆಂದು ಭಾವಿಸುತ್ತಾನೆ. ನಮ್ಮ ದೇಶವನ್ನು, ನಮ್ಮ ಜನರನ್ನು, ನಮ್ಮ ಸಮಾಜವನ್ನು, ಕೊನೆಗೆ ನಮ್ಮನ್ನು ನಾವು ಗೌರವಿಸಲು ಕಲಿಯದೆ, ನಮ್ಮ ಧರ್ಮವನ್ನು ನಾವು ಪ್ರೀತಿಸದೆ ನಾವೊಂದು ಶ್ರೇಷ್ಠ ರಾಷ್ಟ್ರವೆನಿಸಿಕೊಳ್ಳಲಾರದು. ನವಯುಗದ ಹಿಂದೂ ಗಳು ಸಾಮಾನ್ಯವಾಗಿ ಆಷಾಢಭೂತಿಗಳು. ಅವರು ಎಚ್ಚೆತ್ತುಕೊಂಡು ನಿಜವಾದ ವೈದಿಕ ಧರ್ಮದಲ್ಲಿನ ಶ್ರದ್ಧೆಯ ಜೊತೆಗೆ ಯೂರೋಪಿಯನ್ನರ ರಾಜಕೀಯ ಹಾಗೂ ವೈಜ್ಞಾನಿಕ ಸತ್ಯಗಳ ಜ್ಞಾನವನ್ನು ಸಂಯೋಜಿಸಿಕೊಳ್ಳಬೇಕು. ಬಂಗಾಳಕ್ಕಿಂತ ದಕ್ಷಿಣದಲ್ಲಿಯೇ ಜಾತೀಯತೆಯ ಕೇಡು ಹೆಚ್ಚಾಗಿದೆ ಎನ್ನಿಸುತ್ತದೆ. ಬಂಗಾಳದಲ್ಲಿ ಒಬ್ಬ ಬ್ರಾಹ್ಮಣನು ಶೂದ್ರರು ಮುಟ್ಟಿದ ನೀರನ್ನು ಬಳಸುತ್ತಾನೆ; ಆದರೆ ಇಲ್ಲಿ ಬ್ರಾಹ್ಮಣನು ಶೂದ್ರನನ್ನು ದೂರ ಇಟ್ಟಿರುತ್ತಾನೆ. ಕಲಿಯುಗದಲ್ಲಿ ಬ್ರಾಹ್ಮಣರೆಂಬವರಿಲ್ಲ. ನಮ್ಮ ಸಹಚರರಾದ ಪರಯರನ್ನು ಉನ್ನತ ಜಾತಿಯವರು ಶಿಕ್ಷಣ ಕೊಟ್ಟು ಮೇಲೆತ್ತಬೇಕು; ಹಿಂದೂಧರ್ಮದ ಸತ್ಯಗಳನ್ನು (ತಿಳಿಸಿ ಕೊಟ್ಟು?) ಬ್ರಾಹ್ಮಣರನ್ನಾಗಿ (ಮಾಡಬೇಕು?). ಬ್ರಾಹ್ಮಣನ ಮೊದಲ ಕರ್ತವ್ಯವೇ ಎಲ್ಲರನ್ನೂ ಪ್ರೀತಿಸುವುದು. ಮೊದಲು ಬ್ರಾಹ್ಮಣರೆಲ್ಲರೂ ಒಂದಾಗಬೇಕು, ಆ ನಂತರ ದ್ವಿಜ\footnote{1. “ಎರಡು ಬಾರಿ ಹುಟ್ಟಿದವರು” - ಆಧ್ಯಾತ್ಮಿಕ ಪುನರ್ಜನ್ಮದ ಪ್ರತೀಕವಾದ ಜನಿವಾರವನ್ನು ಹಾಕಿಕೊಳ್ಳುವ ಕಾರಣದಿಂದ ಹಿಂದೂ ಸಮಾಜದ ಮೊದಲ ಮೂರು ವರ್ಣಗಳವರಿಗೆ ಸಲ್ಲುವ ಹೆಸರು.} ರೆಲ್ಲರೂ ಒಂದಾಗಬೇಕು, ತದನಂತರ ದ್ವಿಜರೂ ಶೂದ್ರರೂ ಒಂದಾಗಬೇಕು.”

\begin{center}
\textbf{ಸರ್ವಧರ್ಮ ಸಮ್ಮೇಳನ\supskpt{\footnote{\enginline{Vivekananda in Indian Newspaper, pp.4}}}}
\end{center}

\begin{center}
\textbf{ಹೆಚ್.ಆರ್. ಹ್ಯಾವೀಸ್ ಅವರಿಂದ}
\end{center}

\begin{center}
(ದಿ ಇಂಡಿಯನ್ ಮಿರರ್ (ದಿ ಡೈಲಿ ಕ್ರಾನಿಕಲ್ನಿಂದ), ೨೮ ನವೆಂಬರ್ ೧೮೯೩)
\end{center}

.... ಆಶ್ಚರ್ಯಕರವಾಗಿ ಬುದ್ಧನ ಸಾಂಪ್ರದಾಯಿಕ ಮುಖಲಕ್ಷಣವನ್ನೇ ಹೋಲುವ ಪ್ರಖ್ಯಾತ ಹಿಂದೂ ಸಂನ್ಯಾಸಿ ವಿವೇಕಾನಂದರು ನಮ್ಮ ವ್ಯಾವಹಾರಿಕ ಅಭ್ಯುದಯ, ನಮ್ಮ ರಕ್ತಪಾತದ ಯುದ್ಧಗಳು ಮತ್ತು ನಮ್ಮ ಧಾರ್ಮಿಕ ಅಸಹಿಷ್ಣುತೆ ಎಲ್ಲವನ್ನೂ ಖಂಡಿಸುತ್ತ, ಸೌಮ್ಯಸ್ವಭಾವದ ಹಿಂದೂ ಇಂತಹ ಬೆಲೆತೆತ್ತು ನೀವು ಬಡಾಯಿ ಕೊಚ್ಚುತ್ತಿರುವ ನಾಗರಿಕತೆಯನ್ನು ಬರಮಾಡಿಕೊಳ್ಳುವುದಿಲ್ಲ ಎಂದರು.... ಹಾಗೆಯೇ ಮುಂದುವರೆದು ಇಂತೆಂದರು:

“ನೀವು ಒಂದು ಕೈಯಲ್ಲಿ ಬೈಬಲ್ನ್ನೂ ಇನ್ನೊಂದು ಕೈಯಲ್ಲಿ ಯೋಧನ ಖಡ್ಗ ವನ್ನೂ ಹಿಡಿದುಕೊಂಡು ಬರುವಿರಿ - ನೆನ್ನೆ ಮೊನ್ನೆಯ ಧರ್ಮದ ನೀವು, ನಿಮ್ಮ ಕ್ರಿಸ್ತನ ಜೀವನದಷ್ಟೇ ಪವಿತ್ರವಾದ ಜೀವನದ ಔನ್ನತ್ಯವನ್ನು ಸಾವಿರಾರು ವರ್ಷಗಳ ಹಿಂದೆಯೇ ನಮ್ಮ ಋಷಿಗಳು ಕಲಿಸಿಕೊಟ್ಟಿರುವ ನಮ್ಮ ಬಳಿಗೆ ಬರುವಿರಿ. ಪದಾಘಾತದಿಂದ ನಮ್ಮನ್ನು ತುಳಿಯುತ್ತ, ನಮ್ಮನ್ನು ನಿಮ್ಮ ಕಾಲ ಕೆಳಗಿರುವ ಧೂಳಿನ ಹಾಗೆ ಕಾಣುವಿರಿ. ಪ್ರಾಣಿಗಳ ಬಹುಮೌಲ್ಯ ಬದುಕನ್ನು ನಾಶ ಮಾಡುವ ನೀವು, ಮಾಂಸಾಹಾರಿಗಳು. ಹೆಂಡ ಕುಡಿಸಿ ನಮ್ಮ ಜನರನ್ನು ಹಾಳುಮಾಡುತ್ತಿರುವಿರಿ. ನಮ್ಮ ಸ್ತ್ರೀಯರನ್ನು ಅಪಮಾನಗೊಳಿಸುತ್ತಿ ರುವಿರಿ. ಅನೇಕ ವಿಚಾರಗಳಲ್ಲಿ ನಿಮ್ಮದನ್ನು ಹೋಲುವ, ಆದರೆ ಇನ್ನೂ ಹೆಚ್ಚು ಮಾನವೀಯವಾಗಿರುವ ಕಾರಣ ಅದಕ್ಕಿಂತ ಮೇಲಾದ ನಮ್ಮ ಧರ್ಮವನ್ನು ಹೀಗಳೆ ಯುತ್ತಿರುವಿರಿ. ಅನಂತರ, ಕ್ರೈಸ್ತಧರ್ಮದ ಮುನ್ನಡೆ ಭಾರತದಲ್ಲಿ ಅದೇಕೆ ಇನ್ನೂ ನಿಧಾನವಾಗುತ್ತಿದೆ ಎಂದು ಅಚ್ಚರಿಪಡುವಿರಿ. ಇದಕ್ಕೆ ಕಾರಣ ನಾವು ಪ್ರೀತಿಸಿ ಗೌರವಿಸ ಬಹುದಾದ ನಿಮ್ಮ ಕ್ರಿಸ್ತನ ಹಾಗೆ ನೀವು ಇಲ್ಲದಿರುವುದೇ ಎಂದು ನಾನು ಹೇಳುತ್ತೇನೆ. ಅವನ ಹಾಗೆ ನೀವು ವಿನೀತರಾಗಿ, ಸಾಧುವಾಗಿ, ಇತರರಿಗಾಗಿ ನರಳುತ್ತ, ಇತರರಿಗಾಗಿ ದುಡಿಯುತ್ತ, ಇತರರಿಗಾಗಯೇ ಬದುಕುತ್ತ, ಪ್ರೇಮಸಂದೇಶವನ್ನು ಹೊತ್ತು ನಮ್ಮ ಬಳಿಗೆ ಬಂದರೆ ನಾವು ಕೇಳಿಸಿಕೊಳ್ಳದೆ ಇರುತ್ತೇವೆಯೇ? ಖಂಡಿತ ಇಲ್ಲ! ಹಿಂದೆ ನಮ್ಮ ಋಷಿ ಗಳು (ಪ್ರಬೋಧಕರು) ಹೇಳಿದ್ದನ್ನು ಕೇಳಿದ ಹಾಗೆಯೇ ಅವನನ್ನೂ ಸ್ವಾಗತಿಸಿ ಅವನು ಹೇಳುವುದನ್ನು ಕೇಳುತ್ತೇವೆ”...

\begin{center}
\textbf{ಚಿಕಾಗೋದಲ್ಲಿ ಸರ್ವಧರ್ಮ ಸಮ್ಮೇಳನ\supskpt{\footnote{\enginline{Vivekananda in Indian Newspaper, p.5-6}}}}
\end{center}

\begin{center}
(ದಿ ಇಂಡಿಯನ್ ಮಿರರ್, ೭ ಡಿಸೆಂಬರ್ ೧೮೯೩)
\end{center}

\begin{center}
\textbf{ಕ್ರೈಸ್ತಧರ್ಮದ ಮೇಲೆ ಹಿಂದೂವಿನ ಟೀಕೆ\\ವೇದಧರ್ಮವು ಪ್ರೇಮಧರ್ಮವೆನ್ನುತ್ತಾರೆ ಮಿ. ವಿವೇಕಾನಂದರು ಕ್ರೈಸ್ತಧರ್ಮವನ್ನು ಅಸಹಿಷ್ಣುವೆನ್ನುತ್ತಾರೆ ವಿವೇಕಾನಂದರು}
\end{center}

ಮಧ್ಯಾಹ್ನದ ಅಧಿವೇಶನಕ್ಕೆ ಡಾ. ನೋಬಲ್ ಅಧ್ಯಕ್ಷರಾಗಿದ್ದರು. ಕೊಲಂಬಸ್ ಸಭಾಂಗಣ ಜನರಿಂದ ಕಿಕ್ಕಿರಿದಿತ್ತು.... ಆಗ ಡಾ. ನೋಬಲ್ ಹಿಂದೂ ಸಂನ್ಯಾಸಿ ಸ್ವಾಮಿ ವಿವೇಕಾನಂದ ರನ್ನು ಪರಿಚಯಿಸಿದರು; ವಿವೇಕಾನಂದರು ವೇದಿಕೆಯ ನಡುವೆ ಬರುತ್ತಲೂ ಸಭಿಕರು ಗಟ್ಟಿಯಾಗಿ ಕರತಾಡನ ಮಾಡಿದರು. ಕಿತ್ತಳೆವರ್ಣದ ಉಡುಪನ್ನು ಧರಿಸಿದ್ದ ಅವರು ಕಡುಗೆಂಪು ಸೊಂಟಪಟ್ಟಿಯೊಂದನ್ನು ಕಟ್ಟಿದ್ದರು; ತಿಳಿಹಳದಿ ರುಮಾಲನ್ನು ಧರಿಸಿದ್ದರು. ಎಂದಿನ ದಿವ್ಯಮಂದಹಾಸವು ಅವರ ಸುಂದರ ವದನವನ್ನಲಂಕರಿಸಿತ್ತು; ಅವರ ಕಣ್ಣು ಗಳು ಜೀವಂತಿಕೆಯಿಂದ ತುಂಬಿದ್ದವು. ಅವರೆಂದರು:

“ಪೂರ್ವದೇಶದಿಂದ ಬಂದಿರುವ ನಾವು ದಿನದಿನವೂ ಈ ವೇದಿಕೆಯ ಮೇಲೆ ಬಂದು ಕುಳಿತುಕೊಂಡು, ಕ್ರೈಸ್ತ ದೇಶಗಳು ಅತ್ಯಂತ ಪುರೋಭಿವೃದ್ಧಿಯಲ್ಲಿವೆಯಾದ್ದರಿಂದ ನೀವು ಕ್ರೈಸ್ತಧರ್ಮವನ್ನು ಸ್ವೀಕರಿಸತಕ್ಕುದು ಎಂದು ಅನುಗ್ರಹ ಮಾಡುವ ರೀತಿಯಲ್ಲಿ ಹೇಳುವು ದನ್ನು ಕೇಳುತ್ತಲೇ ಇದ್ದೇವೆ. ಸುತ್ತಲೂ ತಿರುಗಿ ನೋಡಿದಾಗ ನಮಗೆ ಕಾಣಿಸುವುದು ಪುರೋಭಿವೃದ್ಧಿಯಲ್ಲಿ ಪ್ರಪಂಚದ ಮುಂಚೂಣಿಯಲ್ಲಿರುವ ಕ್ರೈಸ್ತ ದೇಶ ಇಂಗ್ಲೆಂಡ್, ಇಪ್ಪತ್ತು ಕೋಟಿ ಏಷ್ಯನ್ನರ ಕತ್ತನ್ನು ಮೆಟ್ಟಿ ನಿಂತಿರುವ ಇಂಗ್ಲೆಂಡ್. ಪೂರ್ವೇತಿಹಾಸವನ್ನು ನೋಡಿದಾಗ, ಕ್ರೈಸ್ತ ಯೂರೋಪಿನ ಮುನ್ನಡೆ ಪ್ರಾರಂಭವಾದುದು ಸ್ಪೇನ್ನಿಂದ. ಸ್ಪೇನ್ನ ಪುರೋಭಿವೃದ್ಧಿ ಪ್ರಾರಂಭವಾದುದು ಮೆಕ್ಸಿಕೋ ಮೇಲೆ ಆಕ್ರಮಣವಾದಂದಿನಿಂದ. ಕ್ರೈಸ್ತಧರ್ಮ ತನ್ನ ಅಭಿವೃದ್ಧಿಯನ್ನು ಸಾಧಿಸುವುದೇ ಸಹಮಾನವರ ಕುತ್ತಿಗೆ ಕೊಯ್ಯು ವುದರ ಮೂಲಕ. ಇಂತಹ ಬೆಲೆ ತೆತ್ತು ಹಿಂದೂವು ಪುರೋಭಿವೃದ್ಧಿಯನ್ನು ಪಡೆಯ ಲಿಚ್ಛಿಸುವುದಿಲ್ಲ.\footnote{2. ನೋಡಿ, “ಜಗಳಗಂಟಿಯ ಟೀಕೆಗಳು” ಕೃತಿಶ್ರೇಣಿ ೩ ಪುಟ ೪೦೪.}

“ನಾನು ಈ ಹೊತ್ತು ಇಲ್ಲಿ ಕುಳಿತು ಅಸಹಿಷ್ಣುತೆಯ ಪರಾಕಾಷ್ಠೆಯನ್ನೇ ಕೇಳುತ್ತಿ ದ್ದೇನೆ. ಮುಸ್ಲಿಂ ಪಂಥದವರಿಗೆ ಚಪ್ಪಾಳೆ ಹೊಡೆದುದನ್ನು ಕೇಳಿರುವೆ; ಆದರೆ ಇಂದು ಮುಸ್ಲಿಂ ಖಡ್ಗವು ಭಾರತದಲ್ಲಿ ವಿನಾಶದ ತಾಂಡವವನ್ನಾಡುತ್ತಿದೆ. ಹಿಂದೂವಿನ ಧರ್ಮವು ಪ್ರೇಮ ಶಾಸನವನ್ನಾಧರಿಸಿದೆ; ಅವನಿಗೆ ರಕ್ತವಾಗಲಿ ಖಡ್ಗವಾಗಲಿ ಸಲ್ಲದು.”\footnote{3. ಈ ಕೊನೆಯ ಪ್ಯಾರಾ ಈ ಮೊದಲು ಎಲ್ಲೂ ಪ್ರಕಟವಾಗಿಲ್ಲ}

ಕರತಾಡನದ ಶಬ್ದ ನಿಂತ ಮೇಲೆ, ಮಿ. ವಿವೇಕಾನಂದರು ತಮ್ಮ ಪ್ರಬಂಧವನ್ನು ಓದತೊಡ ಗಿದರು; ಅದರ ಸಾರಾಂಶವನ್ನು ಮುಂದೆ ಕೊಟ್ಟಿದೆ: (ನೋಡಿ, ಕೃತಿಶ್ರೇಣಿ, ೧, ಪುಟ ೮-೨೧)....

\begin{center}
\textbf{ಕ್ರೈಸ್ತ ಮತಾಂತರವನ್ನು ಕುರಿತು *\supskpt{\footnote{\enginline{Vivekananda in Indian Newspaper, p.25}}}}
\end{center}

\begin{center}
(ದಿ ಇಂಡಿಯನ್ ಮಿರರ್, ೧೪ ಜೂನ್ ೧೮೯೪)
\end{center}

ಭಾರತದಲ್ಲಿ ಕ್ರೈಸ್ತಧರ್ಮದವರ ಕೆಲಸ ಹಾಗೂ ಅದರ ಭವಿಷ್ಯವನ್ನು ಕುರಿತು ಒಬ್ಬರು ನಿವೃತ್ತ ಕ್ರೈಸ್ತಪ್ರಚಾರಕರಿಗೂ ಸ್ವಾಮಿ ವಿವೇಕಾನಂದರಿಗೂ ಸ್ವಲ್ಪ ಬಿಸಿಬಿಸಿ ಪತ್ರ ವ್ಯವಹಾರ ನಡೆದಿತ್ತು. ಇನ್ನಿತರ ಸಂಗತಿಗಳ ಜೊತೆಗೆ, ಸ್ವಾಮಿಗಳು “ಮತಾಂತರದ ಕ್ರಮ ಸಂಪೂರ್ಣವಾಗಿ ಅಸಂಬಂದ್ಧವಾಗಿದೆ” ಎಂದು ಹೇಳಿರುವರೆಂದು ವರದಿಯಾಗಿತ್ತು;

“ಪ್ರಚಾರಕ ವೈದ್ಯರುಗಳು ಮಾಡುತ್ತಿರುವ ಕೆಲಸ ಉಪಯುಕ್ತವಾಗಿಲ್ಲ, ಏಕೆಂದರೆ ಅವರು ಜನರೊಂದಿಗೆ ಬೆರೆತಿಲ್ಲ.... ಮತಾಂತರದ ಮೂಲಕ ಅವರು ಸಾಧಿಸುವು ದೇನೂ ಇಲ್ಲ, ತಮ್ಮ ತಮ್ಮಲ್ಲಿ ಒಳ್ಳೆಯ ಸಾಮಾಜಿಕ ಸಹವರ್ತನೆಯ ಸಮಯಗಳ ಹೊರತಾಗಿ, ಇತ್ಯಾದಿ.”

ಆ ಸಭ್ಯ ರೆವರೆಂಡ್ನಿಗೆ ಈ ವಾಕ್ಯಗಳಿಂದ ಕೋಪ ಬಂದಿತು; ದೇಶೀಯರಾರೂ ಭಾರತೀಯರನ್ನು ಕ್ರೈಸ್ತಪ್ರಚಾರಕರುಗಳಿಗಿಂತ ಉತ್ತಮವಾಗಿ ಸಹಾನುಭೂತಿಯಿಂದ ಕಾಣುವುದಿಲ್ಲ ಎಂದು ತನ್ನನ್ನು ತಾನು ಸಮರ್ಥಿಸಿಕೊಂಡನು. ಕ್ರೈಸ್ತ ಪ್ರಚಾರಕರುಗಳು ನಿಸ್ಸಂದೇಹವಾಗಿ ಒಳ್ಳೆಯವರು, ಸದುದ್ದೇಶವುಳ್ಳವರು; ಆದರೂ, ನಾವು ಅವರೊಂದಿಗೆ ಸಂಪರ್ಕದಲ್ಲಿಲ್ಲವೆಂಬ ಸ್ವಾಮಿಗಳ ಟೀಕೆಯೂ ಆಧಾರವಿಲ್ಲದೆ ಇಲ್ಲ; ಹಿಂದೂಧರ್ಮದ ಪುನರುತ್ಥಾನ ದೇಶದಲ್ಲಿ ಎಲ್ಲೆಲ್ಲೂ ಕಾಣುತ್ತಿದೆಯಾಗಿ, ಕ್ರೈಸ್ತಧರ್ಮವು ಹಿಂದೂಗಳನ್ನು ಸೆಳೆಯುವುದು ಅನುಮಾನಾಸ್ಪದವಾಗಿಯೇ ಕಾಣುತ್ತಿದೆ. ಸದ್ಯದಲ್ಲಿ ಕ್ರೈಸ್ತಪ್ರಚಾರಕರು ಗಳಿಗೆ ಭಾರತದಲ್ಲಿ ಸಂದಿಗ್ಧ ಪರಿಸ್ಥಿತಿ. ಸ್ವಾಮಿಗಳು ತಮ್ಮನ್ನು ಅವರ ದೇಶದ ಸಹನಾಗರಿಕ ಎಂದು ಕರೆದುದಕ್ಕೆ ಆ ಪ್ರಚಾರಕರಿಗೆ ಧನ್ಯವಾದಗಳನ್ನರ್ಪಿಸಿದರು; ಹೀಗೆ ಬರೆದರು:

“ಯೂರೋಪಿಯನ್ ಪರದೇಶೀಯನೊಬ್ಬ, ಅವನು ಭಾರತಸಂಜಾತನೇ ಆಗಿರಲಿ, ದೇಶೀಯನೆಂದರೆ ಹೇಸಿಗೆಪಟ್ಟುಕೊಳ್ಳುವ ಬದಲು ಹೀಗೆ ಕರೆಯುವ ಧೈರ್ಯ ಮಾಡಿದ್ದು ಇದೇ ಮೊದಲು - ಅವನು ಪ್ರಚಾರಕನಾಗಿರಲಿ, ಇಲ್ಲದಿರಲಿ. ಭಾರತದಲ್ಲಿಯೂ ಸಹ ನನ್ನನ್ನು ಹೀಗೆ ಕರೆಯುವ ಧೈರ್ಯ ನಿಮಗಿದೆಯ?”

\begin{center}
\textbf{ವೇದಗಳ ಕೇಂದ್ರಭಾವನೆ\supskpt{\footnote{\enginline{Vivekananda in Indian Newspaper, p.25}}}}
\end{center}

\begin{center}
(ದಿ ಇಂಡಿಯನ್ ಮಿರರ್, ೨೦ ಜೂನ್ ೧೮೯೪)
\end{center}

ಸ್ವಾಮಿ ವಿವೇಕಾನಂದರು ಅಮೆರಿಕಾದಲ್ಲಿ ವೇದಗಳಲ್ಲಿ ಕೇಂದ್ರಭಾವನೆ ಯನ್ನು ಹೀಗೆಂದು ವಿವರಿಸಿದರು:

“ಯಾರು ಏಕದೇವತಾವಾದದಲ್ಲಿ, ಇಷ್ಟದೇವತೆಯನ್ನು ಒಪ್ಪಿಕೊಳ್ಳುವುದರಲ್ಲಿ, ಪ್ರಕೃತಿಯನ್ನು ಮಾತ್ರವಲ್ಲದೆ ಬೌದ್ಧಿಕ ಉನ್ನತಿಯ ಪರಾಕಾಷ್ಠೆಯನ್ನು ಸಹ ಕಾಣು ವರೋ ಅವರೊಂದಿಗೆ ಭಿನ್ನಾಭಿಪ್ರಾಯ ನನಗಿರುವುದೆಂದು ವಿನೀತನಾಗಿ ತಿಳಿಸಬಯಸು ತ್ತೇನೆ. ಅಜ್ಞಾತವನ್ನು ತಿಳಿಯುವ ಮೊದಲ ಪ್ರಯತ್ನದಲ್ಲಿ ಮಾನವನ ಮನಸ್ಸು ಅಂತಹ ಮನುಷ್ಯತ್ವಾರೋಷಣೆಯಲ್ಲಿ ನಿಲ್ಲುವುದೆಂದು ನನ್ನ ನಂಬಿಕೆ. ಮನುಷ್ಯನ ವೈಚಾರಿಕತೆಗೆ ಕೊಟ್ಟಕೊನೆಯಲ್ಲಿ ಪರಮತೃಪ್ತಿ ಉಂಟಾಗುವುದು ಸಮಷ್ಟಿಸ್ವರೂಪವಾಗಿರುವ ವಿಶ್ವ ಸಾರದ ಸಾಕ್ಷಾತ್ಕಾರದಲ್ಲಿ. ಅಸಂಖ್ಯ ದೇವರುಗಳು ಮತ್ತು ಅವುಗಳ ಆವಾಹನೆ ಮುಂತಾ ದುವುಗಳ ಹೊರತಾಗಿಯೂ ವೇದಗಳಲ್ಲಿ ಈ ಕಲ್ಪನೆ ಇರುವುದೆಂಬುದಕ್ಕೆ ನನ್ನ ಬಳಿ ನಿರ್ವಿವಾದವಾದ ಪುರಾವೆ ಇದೆ. ಈ ಅರೂಪ ಸಮಷ್ಟಿ ಅಥವಾ ಸತ್, ಎಂದರೆ ಅಸ್ತಿತ್ವದ ಸಾರವನ್ನೇ ಉಪನಿಷತ್ತುಗಳಲ್ಲಿ ಆತ್ಮ ಹಾಗೂ ಬ್ರಹ್ಮ ಎಂದು ಕರೆದಿರುವುದು; ಆ ನಂತರ ದರ್ಶನಗಳಲ್ಲಿ ವಿವರಿಸಿರುವುದು. ಇದೇ ವೇದಗಳಲ್ಲಿನ ಕೇಂದ್ರಭಾವನೆ; ಅಲ್ಲಲ್ಲ, ಇದೇ ಹಿಂದೂ ಧರ್ಮದಲ್ಲಿನ ಸಾಮಾನ್ಯವಾದ ಮೂಲಭಾವನೆ.”\footnote{2. ಖಚಿತವಾಗಿ ತಿಳಿದಿಲ್ಲದ ಮೂಲದಿಂದ}

\begin{center}
\textbf{ಸಮುದ್ರಲಂಘನ ಚಳುವಳಿಯನ್ನು ಕುರಿತು ಸ್ವಾಮಿ ವಿವೇಕಾನಂದರು\supskpt{\footnote{\enginline{Vivekananda in Indian Newspaper, p. 260-62}}}}
\end{center}

\begin{center}
(ದಿ ಬಂಗಾಳಿ, ೧೮ ಮೇ ೧೮೯೫)
\end{center}

ಸ್ವಾಮಿ ವಿವೇಕಾನಂದರ ಬಗ್ಗೆ ಹೆಮ್ಮೆಪಡದ, ಅವರನ್ನೂ ಅವರ ಬೋಧನೆ ಗಳನ್ನೂ ಗೌರವಿಸದ ಹಿಂದೂ ಇಲ್ಲವೆಂದೇ ಹೇಳಬಹುದು. ಅವರು ತಮಗೆ, ತಮ್ಮ ಜನಾಂಗಕ್ಕೆ ಮತ್ತು ತಮ್ಮ ಧರ್ಮಕ್ಕೆ ಅತಿಶಯವಾದ ಗೌರವವನ್ನು ಸಂಪಾದಿಸಿಕೊಂಡಿ ರುವರು. ನಮ್ಮ ಈ ದೃಷ್ಟಿ ಸರಿಯಾದುದಾದಲ್ಲಿ, ವಿವೇಕಾನಂದರ ಅಭಿಪ್ರಾಯಗಳು ಅತ್ಯುನ್ನತ ಮಟ್ಟದಲ್ಲಿ ಪರಿಗಣನೆಗೆ ಅರ್ಹವಾಗಿರುವುವು. ಸಮುದ್ರಲಂಘನ ಚಳು ವಳಿಯ ಬಗ್ಗೆ ಅವರು ಹೀಗೆನ್ನುತ್ತಾರೆ:-

“ವಿಕಾಸವೇ ಜೀವನ; ಸಂಕೋಚವೇ ಮರಣ. ಪ್ರೇಮವೇ ಬದುಕು, ದ್ವೇಷವೇ ಸಾವು. ನಾವು ಇತರ ಬುಡಕಟ್ಟುಗಳವರನ್ನು ದ್ವೇಷಿಸುವ ಮೂಲಕ ಸಂಕುಚಿತತೆಯನ್ನು ಪ್ರಾರಂಭಿಸಿದ ದಿನವೇ ಸಾಯುವುದಕ್ಕಾರಂಭಿಸಿರುವೆವು. ನಾವು ಪುನಃ ಜೀವಂತಿಕೆಗೆ, ವಿಕಾಸಕ್ಕೆ ಮರಳುವವರೆಗೆ ಯಾವುದೂ ನಮ್ಮ ಸಾವನ್ನು ತಪ್ಪಿಸಲಾರದು. ಆದ್ದರಿಂದ, ನಾವು ಈ ಭೂಮಂಡಲದ ಎಲ್ಲ ಜನಾಂಗದವರೊಡನೆಯೂ ಬೆರೆಯಬೇಕು; ಗೊಂತಿಗೆ ಕಟ್ಟಿದ ನಾಯಿಯಂತೆಯೇ ಉಳಿಯುವ ಏಕಮಾತ್ರ ಧ್ಯೇಯವುಳ್ಳ ನೂರಾರು ಸ್ವಾರ್ಥದ, ಮೂಢನಂಬಿಕೆಗಳ ಕಂತೆಗಳ ಜನರಿಗಿಂತ ಪರದೇಶಗಳಿಗೆ ಪಯಣಿಸುವ ಪ್ರತಿಯೊಬ್ಬ ಹಿಂದುವೂ ತನ್ನ ದೇಶಕ್ಕೆ ಹೆಚ್ಚು ಒಳಿತನ್ನು ಮಾಡುವನು. ಪಶ್ಚಿಮ ದೇಶಗಳು ಕಟ್ಟಿರುವ ರಾಷ್ಟ್ರೀಯ ಜೀವನದ ಅದ್ಭುತ ನಿರ್ಮಾಣಗಳು ಆಧರಿಸಿರುವುದು ಶೀಲದ ಸ್ತಂಭಗಳ ಮೇಲೆ - ಅಂತಹ ನೂರಾರನ್ನು ನಾವೂ ರೂಪಿಸುವವರೆಗೆ ಈ ಶಕ್ತಿಯ ವಿರುದ್ಧ ಆ ಶಕ್ತಿಯ ವಿರುದ್ಧ ಎಂದು ದನಿಯೆತ್ತುವುದು, ಸಿಟ್ಟು ಸೆಡವು ಮಾಡುವುದು ವ್ಯರ್ಥವೇ ಸರಿ. ಇತರರಿಗೆ ಔದಾರ್ಯ ತೋರಲು ಸಿದ್ಧರಾಗಿಲ್ಲದವರೆಗೆ ಸ್ವಾತಂತ್ರ್ಯವನ್ನು ಪಡೆಯುವ ಯೋಗ್ಯತೆ ಇದೆಯೆ? ಅನವಶ್ಯಕ ಸಿಟ್ಟುಸೆಡವುಗಳಲ್ಲಿ ನಮ್ಮ ಶಕ್ತಿಯನ್ನು ವ್ಯರ್ಥಗೊಳಿಸಿ ಕೊಳ್ಳುವ ಬದಲು ಶಾಂತರಾಗಿ, ಪೌರುಷಯುಕ್ತರಾಗಿ ಕೆಲಸದಲ್ಲಿ ತೊಡಗೋಣ. ಯಾರೊಬ್ಬರಿಗೂ ನಿಜವಾದ ಯೋಗ್ಯತೆಯಿರುವುದನ್ನು ಪಡೆಯುವುದಕ್ಕೆ ಅಡ್ಡಿಪಡಿಸಲು ಈ ವಿಶ್ವದ ಯಾವ ಶಕ್ತಿಗೂ ಸಾಧ್ಯವಿಲ್ಲ ಎನ್ನುವುದು ನನ್ನ ಖಚಿತವಾದ ನಂಬಿಕೆ. ನಮ್ಮ ಹಿಂದಿನ ಕಾಲ ನಿಸ್ಸಂದೇಹವಾಗಿ ಮಹತ್ವದ್ದಾಗಿತ್ತು ನಿಜ; ಆದರೆ ನಮಗಾಗಿ ಕಾದಿರುವ ಭವಿಷ್ಯವು ಇನ್ನೂ ಮಹಿಮಾನ್ವಿತವಾಗಿರುವುದೆಂದು ನಾನು ಪ್ರಾಮಾಣಿಕವಾಗಿ ನಂಬುವೆ.\footnote{1. ನೋಡಿ, \enginline{Vide Complete Works IV, 366.}}

ನಾವು ಇತರ ದೇಶಗಳ ಜೊತೆಗೆ ಬೆರೆಯಬೇಕು ಮತ್ತು ಅವರು ನಮಗೆ ಕೊಡ ಬಹುದಾದ ಒಳಿತನ್ನು ತೆಗೆದುಕೊಳ್ಳಬೇಕು. ಪ್ರತ್ಯೇಕವಾಗಿರುವಿಕೆ ಹಾಗೂ ಪರದೇಶ ಗಳಿಂದ ಕಲಿಯಲು ಇಚ್ಛೆ ಇಲ್ಲದಿರುವಿಕೆ ನಮ್ಮ ಇಂದಿನ ದುರ್ಗತಿಗೆ ಕಾರಣ. ನಮ್ಮನ್ನು ನಾವು ದೇವಲೋಕದಿಂದಲೇ ಆರಿಸಲ್ಪಟ್ಟವರು, ಈ ಭೂಮಿಯ ಮೇಲಣ ದೇಶಗ ಳೆಲ್ಲವಕ್ಕಿಂತಲೂ ನಾವೇ ಸರ್ವಪ್ರಕಾರದಿಂದಲೂ ಮೇಲು ಎಂದುಕೊಂಡೆವು. ಅವರನ್ನು ನಾವು ರೂಕ್ಷರೆಂದು ಭಾವಿಸಿದೆವು; ಅವರ ಸ್ಪರ್ಶ ಮೈಲಿಗೆ, ಅವರ ತಿಳಿವು ಅಜ್ಞಾನ ಕ್ಕಿಂತಲೂ ಕಡೆ ಎಂದು ಪರಿಗಣಿಸಿದೆವು. ನಾವೇ ಸೃಜಿಸಿಕೊಂಡ ಪ್ರಪಂಚದಲ್ಲಿ ಬದುಕ ತೊಡಗಿದೆವು. ನಾವು ಅನ್ಯದೇಶೀಯರಿಗೆ ಏನನ್ನೂ ಬೋಧಿಸಲಿಲ್ಲ, ಅವರಿಂದ ಏನನ್ನೂ ಕಲಿತುಕೊಳ್ಳಲಿಲ್ಲ. ಕೊಟ್ಟಕೊನೆಗೆ ಭ್ರಮನಿರಸನ ಎದುರಾಗಿದೆ. ಪರದೇಶದವರು ನಮ್ಮ ಯಜಮಾನರಾಗಿದ್ದಾರೆ, ನಮ್ಮ ಭವಿಷ್ಯವನ್ನು ರೂಪಿಸುವವರಾಗಿದ್ದಾರೆ. ನಾವು ಕಾತರ ದಿಂದ ಅವರ ವಿದ್ಯೆಯನ್ನು ಕಲಿಯುತ್ತಿದ್ದೇವೆ. ಅದರಲ್ಲಿ ನವೀನವಾದುದು, ಉಪಯುಕ್ತ ವಾದುದು ಬೇಕಾದಷ್ಟಿದೆ ಎಂದು ನಮಗೆ ಗೊತ್ತಾಗಿದೆ. ಪ್ರಾಪಂಚಿಕ ಸುಖಸೌಲಭ್ಯಗಳ ವಿಚಾರದಲ್ಲಿ ಅವರು ನಮ್ಮನ್ನು ಬಹಳ ಹಿಂದೆ ಹಾಕಿರುವರೆಂಬುದು ನಮಗೆ ಅರಿವಾಗಿದೆ. ಪ್ರಕೃತಿಯ ಶಕ್ತಿಗಳ ಮೇಲೆ ಅವರ ನಿಯಂತ್ರಣ ನಾವು ಕನಸು ಕಂಡಿದ್ದುದಕ್ಕಿಂತಲೂ ತುಂಬ ಹೆಚ್ಚೆಂದು ಗೊತ್ತಾಗಿದೆ. ಅವರು ದೇಶಕಾಲಗಳನ್ನು ಶೂನ್ಯಗೊಳಿಸಿದ್ದಾರೆ, ಪ್ರಕೃತಿಯ ಶಕ್ತಿ ಗಳನ್ನು ಮಾನವನ ಅನುಕೂಲಕ್ಕೆ ಒದಗುವಂತೆ ಸ್ವಾಧೀನಪಡಿಸಿಕೊಂಡಿದ್ದಾರೆ. ನಮಗೆ ಕಲಿಸಲು ಅವರ ಬಳಿ ಅನೇಕ ಅಚ್ಚರಿಯ ಸಂಗತಿಗಳಿದ್ದವು, ನಾವು ಅವನ್ನು ಕಾತರದಿಂದ ಕಲಿತೆವು. ಆದರೆ ಇನ್ನೂ ನಾವು ಅವರ ದೇಶಕ್ಕೆ ಭೇಟಿ ಕೊಡುತ್ತಿಲ್ಲ. ಹಾಗೆ ಮಾಡಿದರೆ ಜಾತಿ ಹೋಗುವುದು! ನಾವೀಗ ಪರದೇಶದವರ ಆಳ್ವಿಕೆಯಲ್ಲಿದ್ದೇವೆ. ನಾವು ಕಾತರ ದಿಂದ ಅವರದೊಂದು ಭಾಷೆಯನ್ನು ಕಲಿತು, ಅವರ ಸಾಹಿತ್ಯವನ್ನು ಓದಿ, ಅವರಲ್ಲಿ ರುವ ಒಳ್ಳೆಯದನ್ನು, ಸುಂದರವಾದುದನ್ನು ಪ್ರಶಂಸಿಸುತ್ತೇವೆ. ಅವರ ವಸ್ತುಗಳನ್ನು ಉಡುಪಿಗಾಗಿ, ಆಹಾರಕ್ಕಾಗಿ ಬಳಸುತ್ತೇವೆ. ಆದರೂ ಸಹ, ಬಹಿಷ್ಕಾರಕ್ಕೆ ಅಂಜಿ ನಮ್ಮ ನ್ನಾಳುವ ಅವರ ದೇಶಕ್ಕೆ ಭೇಟಿ ಕೊಡುವ ಧೈರ್ಯ ಮಾಡುತ್ತಿಲ್ಲ. ಉದ್ದೇಶಪೂರ್ವಕ ವಲ್ಲದ ಈ ಪಕ್ಷಪಾತದ ವಿರುದ್ಧ ಹಿಂದೂಗಳಲ್ಲಿ ಹಿಂದೂ ಆಗಿರುವ ಮಹೋದಯ ರಾದ ಸ್ವಾಮಿಗಳು ಸಾತ್ವಿಕ ಕ್ರೋಧದಿಂದ ದನಿಯೆತ್ತುತ್ತಾರೆ. ಅವರು ನಿರ್ಭಿಡೆಯಿಂದ ಹೇಳಿದ ಹಾಗೆ, ಇದಕ್ಕೆ ಪ್ರತಿಹೇಳುವವರು ಗೊಂತಿನಲ್ಲಿ ಕಟ್ಟಿಹಾಕಲ್ಪಟ್ಟ ನಾಯಿಯಂತೆ. ಅವರು ತಾವೂ ಪರದೇಶಗಳಿಗೆ ಪ್ರವಾಸ ಹೋಗುವುದಿಲ್ಲ, ಇತರರನ್ನೂ ಹೋಗಲು ಬಿಡುವುದಿಲ್ಲ. ಆದರೂ, ಸ್ವಾಮಿಗಳೆಂದಂತೆ,

“ನೂರಾರು ಅಂತಹ ಗೊಂತಿಗೆ ಕಟ್ಟಿದ ನಾಯಿಗಳಂತಿರುವ ಸ್ವಾರ್ಥದ, ಮೂಢ ನಂಬಿಕೆಯ ಕಂತೆಗಳಿಗಿಂತ ಪರದೇಶಗಳಿಗೆ ಹೋಗಿ ಬಂದಿರುವ ಹಿಂದೂಗಳಿಂದ ದೇಶಕ್ಕೆ ಎಷ್ಟೋ ಪ್ರಯೋಜನವಾಗುತ್ತದೆ”....\footnote{1. ನೋಡಿ, \enginline{Complete Works IV, 366.}}

ಹಿಂದಿನ ಕಾಲದಲ್ಲಿ ಇದ್ದಂತಹ ಋಷಿಗಳು ಈಗ ನಮ್ಮ ನಡುವೆ ಇದ್ದಿದ್ದರೆ, ಸಮುದ್ರ ಲಂಘನದ ವಿರುದ್ಧವಾಗಿರುವ - ಒಂದುವೇಳೆ ಇದ್ದಿರಬಹುದಾದ - ಶಾಸ್ತ್ರಭಾಗವನ್ನು ವಾಪಸು ತೆಗೆದುಕೊಳ್ಳುತ್ತಿದ್ದರು. ಸಮಾಜವೆನ್ನುವುದು ಅಭ್ಯುದಯದ ಅಚಲ ನಿಯಮ ಗಳನ್ನು ಪಾಲಿಸುವ ಒಂದು ಸಂಸ್ಥೆ. ನಮ್ಮ ಯೋಗಕ್ಷೇಮಕ್ಕೆ, ಹಾಗೂ ನಿಜಕ್ಕೂ ಸಾಮಾಜಿಕ ವ್ಯವಸ್ಥೆಯನ್ನು ಕಾಪಾಡಿಕೊಳ್ಳುವುದಕ್ಕೆ ಎಚ್ಚರದ, ವಿವೇಕಯುತವಾಗಿರುವ ಬದಲಾವಣೆ ಅಗತ್ಯ. ಅದು ಹೇಗೇ ಇರಲಿ, ಅಷ್ಟು ಉನ್ನತ ಪ್ರೌಢಿಮೆಯ, ಅಷ್ಟು ಒಳ್ಳೆಯ ಹಿಂದೂ ಆಗಿರುವ ಸ್ವಾಮಿ ವಿವೇಕಾನಂದರು ಪರದೇಶಗಳಿಗೆ ಪಯಣಿಸುವುದನ್ನು ಬೆಂಬಲಿ ಸುತ್ತಾರೆ ಎನ್ನುವುದೇ ದೊಡ್ಡದು....

\begin{center}
\textbf{“ಬೌದ್ಧಧರ್ಮ, ಹಿಂದೂಧರ್ಮದ ಸಫಲತೆ” ಎನ್ನುವುದರ ಸಾರಾಂಶ *\supskpt{\footnote{\enginline{Vivekananda in Indian Newspaper, p. 73}}}}
\end{center}

\begin{center}
(ದಿ ಇಂಡಿಯನ್ ಮಿರರ್, ೨೯ಜೂನ್ ೧೮೯೫)
\end{center}

ಚಿಕಾಗೋದಲ್ಲಿ ೧೮೯೩ರ ಸೆಪ್ಟೆಬರ್ ೨೬ರಂದು ಬೌದ್ಧಧರ್ಮೀಯರ ವಿಚಾರ ಮಂಡನೆಯ ಸಮಯದಲ್ಲಿ ಸ್ವಾಮಿ ವಿವೇಕಾನಂದರು ಮಾಡಿದ ಭಾಷಣವನ್ನು “ಸರ್ವಧರ್ಮ ಸಮ್ಮೇಳನದ ಇತಿಹಾಸ” ದ ಮ್ಯಾಕ್ನೀಲೆ ಆವೃತ್ತಿಯಲ್ಲಿ ಪ್ರಕಟಿಸ ಲಾಗಿದೆ. ಅವರ ಉಪಸಂಹಾರದ ಮಾತುಗಳು ಹೀಗಿವೆ: -

“ನೀವಿಲ್ಲದೆ ನಾವು ಬದುಕಿರಲಾರೆವು, ನೀವು ನಾವಿಲ್ಲದೆ ಬದುಕಿರಲಾರಿರಿ, ಹಾಗಿದ್ದ ಮೇಲೆ ನೀವು ಬ್ರಾಹ್ಮಣರ ಬೌದ್ಧಿಕತೆ ತತ್ತ್ವಜ್ಞಾನಗಳಿಲ್ಲದೆ ನಿಲ್ಲಲಾರಿರಿ ಮತ್ತು ನಾವು ನಿಮ್ಮ ಹಾರ್ದಿಕತೆ ಇಲ್ಲದೆ ಇರಲಾರೆವು ಎನ್ನುವುದನ್ನು ನಂಬಿರಿ. ಬೌದ್ಧರೂ ಬ್ರಾಹ್ಮಣರೂ ಹೀಗೆ ಪ್ರತ್ಯೇಕವಾದುದೇ ಭರತಖಂಡದ ಅವನತಿಗೆ ಕಾರಣ. ಆದ್ದರಿಂದಲೇ ಭಾರತವು ಕಳೆದ ಒಂದು ಸಾವಿರ ವರ್ಷಗಳಿಂದಲೂ ಆಕ್ರಮಣಕಾರರ ದಾಸ್ಯದಲ್ಲಿರುವಂತಾಗಿದೆ. ಆದಕಾರಣ ನಾವು ಬ್ರಾಹ್ಮಣರ ಅಚ್ಚರಿಯ ನಿಶಿತಮತಿಯನ್ನೂ ಮಹಾಗುರು ಬುದ್ಧನ ಅಚ್ಚರಿಯ ಮಾನವೀಯಗೊಳಿಸುವ ಹೃದಯದಶಕ್ತಿ ದಿವ್ಯಾತ್ಮತೆಗಳನ್ನೂ ಒಂದುಗೂಡಿ ಸೋಣ.”\footnote{2. ಸ್ವಲ್ಪ ಭಿನ್ನವಾದ ಸಾರಾಂಶದ ಭಾಗಕ್ಕಾಗಿ ನೋಡಿ ಕೃತಿಶ್ರೇಣಿ - \enginline{21-23} ವಿಷಯ: ‘ \enginline{"Buddhism the fulfilment of Hindusim"}}

\begin{center}
\textbf{ಭಾರತೀಯ ತತ್ತ್ವಜ್ಞಾನ ಮತ್ತು ಪಾಶ್ಚಾತ್ಯ ಸಮಾಜ\supskpt{\footnote{\enginline{Vivekananda in Indian Newspaper, p. 73}}}}
\end{center}

\begin{center}
(ದಿ ಇಂಡಿಯನ್ ಮಿರರ್, ೧ ಡಿಸೆಂಬರ್ ೧೮೯೫)
\end{center}

ಬಲೂನ್ ಸೊಸೈಟಿಯವರ ಸಾಪ್ತಾಹಿಕ ಸಭೆಯಲ್ಲಿ ಸ್ವಾಮಿ ವಿವೇಕಾನಂದರಿಂದ “ವೇದಾಂತದ ಬೆಳಕಿನಲ್ಲಿ ಮಾನವ ಹಾಗೂ ಸಮಾಜ”\footnote{4. ಸ್ವಾಮಿ ವಿವೇಕಾನಂದರ ಪ್ರಕಾರ ವಿಷಯವು ‘ಭಾರತೀಯ ತತ್ತ್ವಜ್ಞಾನ ಮತ್ತು ಪಾಶ್ಚಾತ್ಯ ಸಮಾಜ’ ಇದರ ಪದಶಃ ವರದಿ ಲಭ್ಯವಿಲ್ಲ. ನೋಡಿ \enginline{New Discoveries, Vol.3, p.262.}} ಎಂಬ ಉಪನ್ಯಾಸ ಕೊಡ ಲ್ಪಟ್ಟಿತ್ತು. ತಮ್ಮ ಪಂಥದ ಕಡುಕೆಂಪು ಉಡುಪಿನಲ್ಲಿದ್ದ ಸ್ವಾಮಿಗಳು ಅತ್ಯುತ್ತಮ ಇಂಗ್ಲಿಷಿನಲ್ಲಿ ಅತ್ಯಂತ ನಿರರ್ಗಳವಾಗಿ ಒಂದೇ ಒಂದು ಟಿಪ್ಪಣಿಯ ಸಹಾಯ ವಿಲ್ಲದೆ ಒಂದು ಗಂಟೆ ಕಾಲಕ್ಕಿಂತಲೂ ಹೆಚ್ಚಾಗಿ ಮಾತನಾಡಿದರು. ಸಮಾಜವೆಂಬ ಸಂಘಟನೆಯಲ್ಲಿ ಧರ್ಮ ಎನ್ನುವುದೊಂದು ಅತ್ಯಂತ ಅಚ್ಚರಿಯ ಅಂಶ ಎಂದು ಅವರು ಹೇಳಿದರು. ವಿಜ್ಞಾನ ನೀಡಬಹುದಾದ ಅತಿ ದೊಡ್ಡ ಲಾಭ ಜ್ಞಾನ ಎನ್ನುವುದಾದರೆ, ಧರ್ಮದ ಅಧ್ಯಯನದಿಂದ ಮನುಷ್ಯನಿಗೆ ಲಭಿಸುವ ತನ್ನದೇ ಸ್ವರೂಪದ, ತನ್ನ ಆತ್ಮದ, ದೇವರನ್ನು ಕುರಿತಾದ ಜ್ಞಾನಕ್ಕಿಂತ ಹಿರಿದಾದುದು ಇನ್ನಾವುದಿದೆ? ಇಡಿಯ ಪ್ರಪಂಚಕ್ಕೆಲ್ಲ ಒಂದೇ ಒಂದು ಧರ್ಮ ಇರಬೇಕೆನ್ನುವುದು ಅಸಾಧ್ಯ ಮಾತ್ರವಲ್ಲ, ಅಪಾಯಕಾರಿಯೂ ಹೌದು. ಸಮಸ್ತ ಧಾರ್ಮಿಕ ಚಿಂತನೆಯೂ ಒಂದೇ ಮಟ್ಟದಲ್ಲಿ ಇರುವುದಾದರೆ, ಅದು ಧಾರ್ಮಿಕ ಚಿಂತನೆಯ ಸಾವು; ವೈವಿಧ್ಯತೆಯೇ ಅದರ ಜೀವಂತಿಕೆ. ನಾಲ್ಕು ರೀತಿಯ ಧರ್ಮಗಳಿವೆ - (೧) ಕ್ರಿಯಾಶಾಲಿ, (೨) ಭಾವನಾತ್ಮಕ (೩) ಅನುಭಾವಿ ಮತ್ತು (೪) ತಾತ್ತ್ವಿಕ. ದುರದೃಷ್ಟವಶಾತ್ ಪ್ರತಿಯೊಬ್ಬನೂ ತನ್ನದೇ ರೀತಿಗೆ ಕಟ್ಟುಬಿದ್ದಿರುವುದರಿಂದ ಅವನಿಗೆ ಪ್ರಪಂಚದಲ್ಲಿ ಏನಿದೆ ಎಂದು ನೋಡುವುದಕ್ಕೆ ಕಣ್ಣೇ ಇಲ್ಲದಂತಾಗಿದೆ. ಅವನು ಇತರರನ್ನೂ ತನ್ನದೇ ರೀತಿಗೆ ತರುವುದಕ್ಕೆ ಹೋರಾಡುತ್ತಾನೆ. ಎಲ್ಲ ರೀತಿಯ ಜನರಿಗೂ ಅವಕಾಶವೀಯುವ ಧರ್ಮವೇ ಪರಿಪೂರ್ಣ ಧರ್ಮವಾಗುತ್ತದೆ. ವೇದಾಂತ ಧರ್ಮವು ಎಲ್ಲರನ್ನೂ ಸ್ವೀಕರಿಸುತ್ತದೆ; ಒಬ್ಬೊಬ್ಬರೂ ತಮ್ಮ ಸ್ವಭಾವಕ್ಕೆ ಬೇಕಾದುದನ್ನು ಆರಿಸಿ ಕೊಳ್ಳುವುದಕ್ಕೆ ಅವಕಾಶವಿದೆ. ಉಪನ್ಯಾಸವಾದ ಮೇಲೆ ಚರ್ಚೆ ನಡೆಯಿತು.

\begin{center}
\textbf{ಅಮೆರಿಕಾದಲ್ಲಿ ಸ್ವಾಮಿ ವಿವೇಕಾನಂದರು\supskpt{\footnote{\enginline{Vivekananda in Indian Newspaper, pp. 89-90}}}}
\end{center}

\begin{center}
(ದಿ ಇಂಡಿಯನ್ ಮಿರರ್(ನ್ಯೂಯಾರ್ಕ್ ಹೆರಾಲ್ಡ್ನಿಂದ), ೨೫ ಮಾರ್ಚ್ ೧೮೯೬)
\end{center}

ಅನೇಕ ಸುಪ್ರಸಿದ್ಧರು ಸ್ವಾಮಿ ವಿವೇಕಾನಂದರ ತಾತ್ತ್ವಿಕ ಬೋಧನೆಯನ್ನು ಅನುಸರಿಸಲು ಪ್ರಯತ್ನಿಸುತ್ತಿರುವರು.

\begin{center}
\textbf{ಸ್ವಾಮಿಗಳಿಂದ ಒಂದು ಭಾಷಣ}
\end{center}

ಕಾವಿಯುಡುಗೆಯುಟ್ಟು ಸ್ವಾಮಿ ವಿವೇಕಾನಂದರು ಮಧ್ಯದಲ್ಲಿ ಕುಳಿತಿದ್ದರು. ಹಿಂದೂವಿನ ಸಭೆಯು ಅವರ ಇಕ್ಕೆಲದಲ್ಲಿ ಎರಡು ಭಾಗವಾಗಿ ನೆರೆದಿತ್ತು; ಸುಮಾರು ನೂರೈವತ್ತು ಜನರು ಆಸೀನರಾಗಿದ್ದರು. ಅದು ಕರ್ಮ ಮತ್ತು ಕರ್ತವ್ಯಗಳ ಮೂಲಕ ತನ್ನ ಆತ್ಮವನ್ನೇ ಭಗವಂತನೆಂಬುದಾಗಿ ಸಾಕ್ಷಾತ್ಕರಿಸಿಕೊಳ್ಳುವ ಮಾರ್ಗವೆಂದು ವಿವರಿಸಲಾದ ಕರ್ಮಯೋಗದ ಮೇಲಿನ ತರಗತಿಯಾಗಿತ್ತು.

ವಿಷಯವು “ಒಳ್ಳೆಯದಾಗಲಿ ಕೆಟ್ಟದ್ದಾಗಲಿ, ನೀನು ಬಿತ್ತಿದಂತೆ ನಿನ್ನ ಬೆಳೆ” ಎಂದಾಗಿತ್ತು. ಉಪನ್ಯಾಸ ಅಥವಾ ಸೂಚನೆಯ ಅವಧಿ ಮುಗಿದ ನಂತರ ಸ್ವಾಮಿಗಳು ಎಲ್ಲರನ್ನೂ ಪರಿಚಯ ಮಾಡಿಕೊಳ್ಳತೊಡಗಿದರು; ಅವರು ಹೇಳುವುದನ್ನು ಈವರೆಗೆ ಕೇಳುತ್ತಿದ್ದ ಜನರು ಈಗ ಕಾತರದಿಂದ ಅವರ ಕೈಕುಲಕುವುದಕ್ಕೆ ಮುನ್ನುಗ್ಗಿದ, ಪರಿಚಯದ ಲಾಭಕ್ಕಾಗಿ ಪರದಾಡಿದ ರೀತಿಯಿಂದ ಸ್ವಾಮಿಗಳ ಆಕರ್ಷಣಶಕ್ತಿಯನ್ನು ಕಾಣಬಹು ದಾಗಿತ್ತು. ಆದರೆ ತಮ್ಮ ಬಗ್ಗೆ ಸ್ವಾಮಿಗಳು ಅತ್ಯಂತ ಆವಶ್ಯಕವಾದುದಕ್ಕಿಂತ ಹೆಚ್ಚಾಗಿ ಏನನ್ನೂ ಹೇಳುತ್ತಿರಲಿಲ್ಲ. ಅವರ ಶಿಷ್ಯರಲ್ಲಿ ಕೆಲವರು ಹೇಳುವುದಕ್ಕೆ ವಿರೋಧವಾಗಿ, ತಾವು ಈ ದೇಶಕ್ಕೆ ಒಬ್ಬರೇ ಬಂದಿರುವುದಾಗಿಯೂ, ವಿಧಿವತ್ತಾಗಿ ತಾವು ಯಾವ ಹಿಂದೂ ಸಂನ್ಯಾಸಿ ಸಂಘವನ್ನೂ ಪ್ರತಿನಿಧಿಸುತ್ತಿಲ್ಲವೆಂದೂ ಅವರು ಉದ್ಘೋಷಿಸಿದರು. ನಾನೊಬ್ಬ ಸಂನ್ಯಾಸಿ; ಆದ್ದರಿಂದ ಜಾತಿ ಕಳೆದುಕೊಳ್ಳದೆ ಪ್ರವಾಸ ಮಾಡಲು ಸ್ವತಂತ್ರನಾಗಿರುವೆ ಎಂದು ಅವರು ಹೇಳಿದರು. ಹಿಂದೂಧರ್ಮ ಒಂದು ಮತಾಂತರಗೊಳಿಸುವ ಧರ್ಮವಲ್ಲ ಎಂಬುದನ್ನು ಅವರ ಗಮನಕ್ಕೆ ತಂದಾಗ, ಬುದ್ಧನಲ್ಲಿ ಪ್ರಾಚ್ಯಕ್ಕೊಂದು ಸಂದೇಶ ಇದ್ದ ಹಾಗೆ, ಪಶ್ಚಿಮಕ್ಕೆ ಕೊಡುವುದಕ್ಕಾಗಿ ಒಂದು ಸಂದೇಶ\footnote{1. ವಿವರಗಳಿಗೆ ನೋಡಿ, \enginline{Complete works, V. 314}} ತಮ್ಮಲ್ಲಿದೆ ಎಂದರು. ಹಿಂದೂಧರ್ಮದ ಬಗ್ಗೆ ಹಾಗೂ ಅದರ ಕ್ರಿಯಾವಿಧಿಗಳ ಆಚರಣೆಗಳನ್ನು ಈ ದೇಶದಲ್ಲಿ ಪರಿಚಯಿಸುವ ಉದ್ದೇಶವಿದೆಯೇ ಎಂದು ಕೇಳಿದಾಗ, ತಾವು ಕೇವಲ ತತ್ತ್ವಜ್ಞಾನವನ್ನು ಮಾತ್ರ ಬೋಧಿಸುವುದಾಗಿ ಘೋಷಿಸಿದರು.

\begin{center}
\textbf{ವಿದ್ಯಾಭ್ಯಾಸವನ್ನು ಕುರಿತು *\supskpt{\footnote{\enginline{Vivekananda in Indian Newspaper, p.101}}}}
\end{center}

\begin{center}
(ದಿ ಇಂಡಿಯನ್ ಮಿರರ್, ೧೯ ಜೂನ್ ೧೮೯೬)
\end{center}

ಬ್ರಹ್ಮವಾದಿನ್ ಪತ್ರಿಕೆಯ ಸಂಪಾದಕರಿಗೆ ಲಂಡನ್ನಿನಿಂದ ಬರೆದ ಪತ್ರವೊಂದರಲ್ಲಿ ಸ್ವಾಮಿ ಶಾರದಾನಂದರು ಹೀಗೆಂದು ಹೇಳುತ್ತಾರೆ:-

ಇಲ್ಲಿ ಸ್ವಾಮಿ ವಿವೇಕಾನಂದರು ಅತ್ಯುತ್ತಮವಾಗಿ ಪ್ರಾರಂಭ ಮಾಡಿದ್ದಾರೆ. ಅವರ ತರಗತಿಗಳಿಗೆ ಜನರು ಹೆಚ್ಚು ಸಂಖ್ಯೆಯಲ್ಲಿ ತಪ್ಪದೆ ಹಾಜರಾಗುತ್ತಾರೆ; ಮತ್ತು ಉಪನ್ಯಾಸ ಗಳು ಅತ್ಯಂತ ಆಸಕ್ತಿಕರವಾಗಿರುತ್ತವೆ. ಆಂಗ್ಲಿಕನ್ ಚರ್ಚ್ನ ಮುಂದಾಳುಗಳಲ್ಲೊಬ್ಬರಾದ ಕ್ಯಾನನ್ ಹ್ಯಾವೀಸ್ ಅವರು ಒಂದು ದಿನ ಆಗಮಿಸಿದರು; ಅವರಿಗೂ ತುಂಬ ಆಸಕ್ತಿ ಹುಟ್ಟಿತು. ಅವರು ಇಲ್ಲಿನ ಚಿಕಾಗೋ ಜಾತ್ರೆಗಳಲ್ಲಿ ಸ್ವಾಮಿಗಳನ್ನು ಅದಾಗಲೇ ನೋಡಿ ದ್ದರು; ಆವಾಗಿನಿಂದಲೂ ಅವರನ್ನು ಇಷ್ಟಪಟ್ಟಿದ್ದರು. ಕಳೆದ ಮಂಗಳವಾರ ಸ್ವಾಮಿಗಳು ಸಿಸೇಮ್​ ಕ್ಲಬ್ನಲ್ಲಿ “ವಿದ್ಯಾಭ್ಯಾಸ”ವನ್ನು ಕುರಿತು ಉಪನ್ಯಾಸ ಮಾಡಿದರು. ಮಹಿಳೆಯರೇ ಸ್ತ್ರೀ ವಿದ್ಯಾಭ್ಯಾಸ ಪ್ರಚಾರಕ್ಕಾಗಿ ಹುಟ್ಟುಹಾಕಿದ ಗೌರವಾನ್ವಿತ ಕ್ಲಬ್ ಅದು. ಇದ ರಲ್ಲಿ ಅವರು ಭಾರತದ ಪುರಾತನ ವಿದ್ಯಾಭ್ಯಾಸಕ್ರಮವನ್ನು ವಿವರಿಸುತ್ತ, ಮನುಷ್ಯನ ನಿರ್ಮಾಣ ಮಡುವುದೇ ಇದರ ಏಕೈಕ ಧ್ಯೇಯ ಎಂಬುದನ್ನು ಸ್ಪಷ್ಟವಾಗಿ, ಮನಮುಟ್ಟುವಂತೆ ತಿಳಿಸಿ ಕೊಟ್ಟರು; ಪ್ರಚಲಿತ ವ್ಯವಸ್ಥೆಯಲ್ಲಿರುವ ಹಾಗೆ ಮಾಹಿತಿಯನ್ನು ಸಂಗ್ರಹಿಸುವುದಲ್ಲ ಎಂದರು. ಮಾನವನ ಮನಸ್ಸು ಜ್ಞಾನದ ಒಂದು ಅಪರಿಮಿತವಾದ ಭಂಡಾರವಿದ್ದಂತೆ; ಭೂತ, ವರ್ತಮಾನ, ಭವಿಷ್ಯತ್ತುಗಳ ಸಮಸ್ತ ಜ್ಞಾನವೂ - ಅಭಿವ್ಯಕ್ತವಾಗಿರಲಿ ಇಲ್ಲದಿರಲಿ - ಮಾನವನಲ್ಲಿ ಅಡಗಿದೆ; ಪ್ರತಿಯೊಂದು ವಿದ್ಯಾಭ್ಯಾಸಕ್ರಮದ ಗುರಿಯೂ ಮನಸ್ಸಿಗೆ ಅದನ್ನು ಪ್ರಕಟಪಡಿಸಲು ಸಹಾಯ ಮಾಡುವುದಾಗಿರಬೇಕು. ಉದಾಹರಣೆಗೆ, ಗುರುತ್ವಾಕರ್ಷಣ ನಿಯಮ ಮಾನವನ ಮನಸ್ಸಿನಲ್ಲಿತ್ತು; ಸೇಬು ಬಿದ್ದುದರಿಂದ ನ್ಯೂಟನ್ನನಿಗೆ ಯೋಚಿಸಲು, ಯೋಚಿಸಿ ತನ್ನ ಮನಸ್ಸಿಗೊಳಗಿಂದ ಅದನ್ನು ಹೊರತರಲು ಸಹಾಯವಾಯಿತು. ಸ್ವಾಮಿಗಳ ತರಗತಿಗಳನ್ನು ಹೀಗೆ ಏರ್ಪಾಟು ಮಾಡಲಾಗಿದೆ: -

ಮಂಗಳವಾರ ಮತ್ತು ಗುರುವಾರಗಳಲ್ಲಿ ಬೆಳಗ್ಗೆ ಮತ್ತು ಸಾಯಂಕಾಲ; ಶುಕ್ರವಾರ ಗಳಲ್ಲಿ ಸಂಜೆ ಪ್ರಶ್ನೋತ್ತರ ತರಗತಿಗಳು. ಹೀಗೆ ಸ್ವಾಮಿಗಳು ವಾರದಲ್ಲಿ ನಾಲ್ಕು ಉಪ ನ್ಯಾಸಗಳನ್ನೂ ಒಂದು ಪ್ರಶ್ನೋತ್ತರ ತರಗತಿಯನ್ನೂ ನಡೆಸುವರು. ತರಗತಿ ಉಪನ್ಯಾಸ ಗಳನ್ನು ಜ್ಞಾನಯೋಗದಿಂದ ಪ್ರಾರಂಭಿಸಿರುವರು. ಈ ಉಪನ್ಯಾಸಗಳನ್ನು ಸ್ವಾಮಿಗಳ ಆರಾಧಕರಾದ ಮಿ. ಗುಡ್ವಿನ್ ಅವರು ಶೀಘ್ರಲಿಪಿಯಲ್ಲಿ ಬರೆದುಕೊಳ್ಳುತ್ತಿದ್ದರು. ಅವುಗಳನ್ನು ಮುಂದೆ ಪ್ರಕಟಿಸಲಾಗುವುದು.

\begin{center}
\textbf{ಇಂಗ್ಲೆಂಡಿನಲ್ಲಿ ಸ್ವಾಮಿ ವಿವೇಕಾನಂದರು\supskpt{\footnote{\enginline{Vivekananda in Indian Newspaper, pp. 493-95}}}}
\end{center}

\begin{center}
(ದಿ ಬ್ರಹ್ಮವಾದಿನ್, ೧೮ ಜುಲೈ ೧೮೯೬)
\end{center}

ಸರ್,

ಕೆಲವು ವಾರಗಳ ಹಿಂದೆ ಸ್ವಾಮಿ ಶಾರದಾನಂದರು ನಿಮಗೆ ಕಳುಹಿಸಿದ ಪತ್ರಕ್ಕೆ ಪೂರಕವಾಗಿ ಇಂಗ್ಲೆಂಡ್ನಲ್ಲಿ ಸ್ವಾಮಿಜಿಯವರ ಕೆಲಸ ಮುಂದುವರೆಯುತ್ತಿರುವುದರ ಬಗ್ಗೆ ತಿಳಿದುಕೊಳ್ಳಲು ನಿಮಗೆ ಸಂತೋಷವಾಗುತ್ತದೆ ಎಂದು ಖಂಡಿತವಾಗಿ ಭಾವಿಸು ತ್ತೇನೆ. ಆಗ ಸ್ವಾಮಿಗಳಿಂದ ಭಾನುವಾರಗಳ ಉಪನ್ಯಾಸ ಸರಮಾಲೆಯೊಂದನ್ನು ಯೋಜಿಸಲಾಗಿತ್ತು; ಅದರಲ್ಲಿ ಮೂರು ಉಪನ್ಯಾಸಗಳನ್ನು ಈಗ ಕೊಟ್ಟಿರುವರು. ಅವು ಗಳನ್ನು ೧೯೧, ಪಿಕಡಿಲ್ಲಿಯಲ್ಲಿರುವ ಜಲವರ್ಣ ಚಿತ್ರಕಾರರ ರಾಯಲ್ ಇನ್ಸ್ಪಿಟ್ಯೂಟ್ನ ಸಭಾಂಗಣವೊಂದರಲ್ಲಿ ಏರ್ಪಡಿಸಲಾಗಿತ್ತು; ಒಂದಲ್ಲ ಒಂದು ಕಾರಣಕ್ಕಾಗಿ ತರಗತಿ ಉಪನ್ಯಾಸಗಳಿಗೆ ಬರಲಾಗದವರನ್ನೂ ಒಳಗೊಂಡು ತಮ್ಮ ಧ್ಯೇಯವನ್ನು ಸಾಧಿಸುವುದರಲ್ಲಿ ಈ ಉಪನ್ಯಾಸಗಳು ತುಂಬ ಯಶಸ್ವಿಯಾದವು. ಈ ಸರಮಾಲೆಯಲ್ಲಿ ಮೊದಲನೆಯ ಉಪನ್ಯಾಸ “ಧರ್ಮದ ಅವಶ್ಯಕತೆ”\footnote{2. ನೋಡಿ, ಮಿ. ಜೆ.ಜೆ. ಗುಡ್ವಿನ್ ಅವರು ಪ್ರಕಟಿಸಿದ “ಧರ್ಮದ ಆವಶ್ಯಕತೆ”, ಕೃತಿಶ್ರೇಣಿ, ೨, ಪುಟ ೩.}. ಮಾನವ ಜನಾಂಗದ ಭವಿಷ್ಯವನ್ನು ನಿರ್ಮಾಣ ಮಾಡುವುದರಲ್ಲಿ ಧರ್ಮವು ಮಹತ್ತರವಾದ ಶಕ್ತಿಯಾಗಿದೆ ಎಂದು ಸ್ವಾಮಿಗಳು ಉದ್ಘೋಷಿಸಿದರು. ಅದರ ಮೂಲದ ವಿಚಾರವಾಗಿ ಇರುವ ಎರಡು ವಾದಗಳಲ್ಲಿ - (೧) ಚೇತನಮೂಲವಾದದ್ದು, (೨) ಅನಂತವನ್ನು ಹುಡುಕಿಕೊಂಡು ಹೋಗುವುದು - ಯಾವುದಾದರೂ ಸರಿಯೇ ಎಂದರು ಸ್ವಾಮಿಗಳು. ಅವರಿಗೆ ಅನ್ನಿಸಿದಂತೆ, ಇವೆರಡರಲ್ಲಿ ಒಂದು ಇನ್ನೊಂದನ್ನು ವಿರೋಧಿಸಲಾರದು; ಏಕೆಂದರೆ, ಬ್ಯಾಬಿಲೋನಿಯನ್ನರ ಮತ್ತು ಈಜಿಪ್ಟಿಯನ್ನರ ಹಾಗೆ ಸತ್ತವರ ಚೇತನ ಏನಾಯಿತು ಎಂದು ಹುಡುಕಿಕೊಂಡು ಹೋಗು ವುದು ಹಾಗೂ ಆರ್ಯರಂತೆ ಸಂಧ್ಯಾಕಾಲ ಉಷಃಕಾಲ ಸಿಡಿಲು ಬಿರುಗಾಳಿ ಇತ್ಯಾದಿ ಪ್ರಕೃತಿಯ ವಿದ್ಯಮಾನಗಳ ಹಿಂದೇನಿದೆ ಎಂದು ಹುಡುಕಿ ನೋಡುವುದು - ಇವೆರಡನ್ನೂ ಇಂದ್ರಿಯಾತೀತದ ಕಡೆಗೆ ಸಾಗುವ, ಆದಕಾರಣವೇ ಅನಂತ ಅಥವಾ ಅಪರಿಮಿತವನ್ನು ಕುರಿತಾದ ಅನ್ವೇಷಣೆಗಳೆಂದು ಭಾವಿಸಬಹುದು. ಈ ಅಪರಿಮಿತವು ಕಾಲಾನುಕ್ರಮದಲ್ಲಿ, ಮೊದಲು ಒಂದು ವ್ಯಕ್ತಿಯಾಗಿ, ಆ ನಂತರ ಒಂದು ಸನ್ನಿಧಿಯಾಗಿ, ನಿರಪೇಕ್ಷ ವಾಯಿತು; ಕೊನೆಗೆ ಸಮಸ್ತ ಅಸ್ತಿತ್ವದ ಸಾರವೆನಿಸಿಕೊಂಡಿತು. ಸ್ವಾಮಿಗಳಿಗೆ ಅನ್ನಿಸಿ ದಂತೆ, ಸ್ವಪ್ನಸ್ಥಿತಿಯೇ ಧಾರ್ಮಿಕ ಪರಿಪ್ರಶ್ನೆಯ ಮೊದಲ ಭಾವನೆಯೆನ್ನಬಹುದು; ಜಾಗೃತ್ ಅವಸ್ಥೆ ಇಲ್ಲಿಯವರೆಗೆ ಮತ್ತು ಇನ್ನೂ ಮುಂದೆ ಯಾವಾಗಲೂ ಸ್ವಪ್ನಾವಸ್ಥೆಯ ಸಹವರ್ತಿಯಾಗಿಯೇ ಇರುವುದರಿಂದ, ಜಾಗೃದವಸ್ಥೆ ಗಿಂತ ಸೂಕ್ಷ್ಮವಾದ ಅಸ್ತಿತ್ವ ವೊಂದು ಇರಬಹುದು, ಅದು ಜಾಗೃದವಸ್ಥೆಯಲ್ಲಿ ಇಲ್ಲವಾಗುತ್ತಿರಬಹುದು; ಹೀಗಿರು ವಾಗ ಮನುಷ್ಯನ ಮನಸ್ಸು ಚೇತನವೊಂದರ ಇರುವಿಕೆ ಹಾಗೂ ಭವಿಷ್ಯಜೀವನದ ಬಗ್ಗೆ ಯಾವಾಗಲೂ ಒಲಿದಿರುತ್ತದೆ. ಒಂದು ಅರ್ಥದಲ್ಲಿ ನಾವು ನಮ್ಮ ಸ್ವಪ್ನಾವಸ್ಥೆಯಲ್ಲಿಯೇ ನಮ್ಮ ಅಮೃತತ್ವವನ್ನು ಕಂಡುಕೊಳ್ಳುವುದು. ತದನಂತರ, ಸ್ವಪ್ನಗಳು ಜಾಗೃದವಸ್ಥೆಯ ಒಂದು ರೀತಿಯ ಸೌಮ್ಯವಾದ ಆವಿರ್ಭಾವವಾದ್ದರಿಂದ, ಮನಸ್ಸಿನ ಇನ್ನೂ ಆಳವಾದ ಭೂಮಿಕೆಗಳ - ಎಂದರೆ ಪ್ರಜ್ಞಾತೀತ ಸ್ಥಿತಿಗಳ ಅನ್ವೇಷಣೆಯ ಪ್ರಯತ್ನ ಆರಂಭವಾಗುತ್ತದೆ. ಎಲ್ಲಾ ಧರ್ಮಗಳೂ ಇಂತಹ ಒಂದು ಸ್ಥಿತಿಯಲ್ಲಿ ಕಂಡು ಹಿಡಿಯಲಾದ ವಾಸ್ತವಗಳನ್ನೇ ಆಧರಿಸಿರುವುದನ್ನು ಸಾರುತ್ತವೆ. ಇಲ್ಲಿ ಮುಖ್ಯವಾಗುವ ಎರಡು ಸಂಗತಿಗ ಳೆಂದರೆ, ಹೀಗೆ ಕಂಡುಹಿಡಿದ ಎಲ್ಲ ‘ವಾಸ್ತವ’ ಗಳೂ ಸಹ, ಅತ್ಯುನ್ನತ ಅರ್ಥದಲ್ಲಿ, ಅಮೂರ್ತ ಭಾವನೆಗಳು; ಎರಡನೆಯದಾಗಿ, ಈ ಆದರ್ಶಕ್ಕೆ ಬಂದು ತಲುಪುವು ದಕ್ಕಾಗಿ ಹೋರಾಟದ ಸ್ಪರ್ಧೆ ಏರ್ಪಡುವುದರಿಂದ, ಇದರ ಕಡೆಗೆ ಸಾಗುವ ಪಥದಲ್ಲಿ ಅಡಚಣೆ ಯಾಗಬಹುದಾದ ಎಲ್ಲವನ್ನೂ ನಾವು ಮಿತಿಗಳೇ ಎಂದುಕೊಳ್ಳುತ್ತೇವೆ. ಅಪರಿಮಿತ ಸುಖ, ಶಕ್ತಿ, ಜ್ಞಾನ, ಅಥವಾ ಇನ್ನಾವ ಬಗೆಯ ಅಪರಿಮಿತತೆಯೇ ಆಗಲಿ, ಅದನ್ನು ಇಂದ್ರಿಯಗಳ ಮೂಲಕ ಕಂಡುಕೊಳ್ಳಲಾಗುವುದಿಲ್ಲ ಎಂಬ ಸಂಶೋಧನೆಯೊಂದಿಗೆ ಈ ಹೋರಾಟ ಬೇಗನೆ ಕೊನೆಗೊಳ್ಳುತ್ತದೆ; ಅನಂತರ ವಿಕಾಸದ ಇನ್ನಿತರ ಮಾರ್ಗಗಳಿಗಾಗಿ ಹೋರಾಟ ಪ್ರಾರಂಭವಾಗುತ್ತದೆ; ಆಗ ಧರ್ಮದ ಆವಶ್ಯಕತೆ ನಮಗೆ ತಿಳಿಯುವುದು. ಎರಡನೆಯ ಉಪನ್ಯಾಸವು “ಒಂದು ವಿಶ್ವಧರ್ಮ”\footnote{1. ಇದರ ಪದಶಃ ವರದಿ ಲಭ್ಯವಿಲ್ಲ.} ಎನ್ನುವು ದಾಗಿತ್ತು. ಸ್ವಾಮಿಜಿ ಈ ಉಪನ್ಯಾಸವನ್ನು ಕೊಟ್ಟಾಗ, ಸಾರಾಂಶವಾಗಿ ಅದು ನಿಮ್ಮ ಓದುಗರೆಲ್ಲ ಈಗಾಗಲೇ ನೋಡಿರಬಹುದಾದ ನ್ಯೂಯಾರ್ಕ್ ಉಪನ್ಯಾಸದಂತೆಯೇ ಇತ್ತು. ಈ ಉಪನ್ಯಾಸವನ್ನು ಸ್ವಾಮಿಗಳ ‘ಸಮರನೀತಿ’ ಎಂದೇ ಕರೆಯಬಹುದಿತ್ತು. ಆದ್ದರಿಂದ ನಾವು ಅದನ್ನು ತುಂಬ ಕುತೂಹಲದಿಂದಲೂ ಅತ್ಯಂತ ಆಸಕ್ತಿಯಿಂದಲೂ ನಿರೀಕ್ಷಿಸು ತ್ತಿದ್ದೆವು; ಇಲ್ಲಿ ಲಂಡನ್ನಿನಲ್ಲಿ ಅದು ಮೂಡಿಸಿದ ಛಾಪು, ಅದು ನ್ಯೂಯಾರ್ಕ್ನ ಹಾರ್ಡ್ ಮನ್ ಹಾಲ್ನಲ್ಲಿ ಉಂಟುಮಾಡಿದ ಪ್ರಭಾವಕ್ಕೆ ಸರಿಸಮನಾಗಿಯೇ ಇತ್ತೆಂಬುದು ತುಂಬ ಹೆಮ್ಮೆಯ ಸಂಗತಿ. ಈ ಸರಣಿಯ ಮೂರನೆಯ ಉಪನ್ಯಾಸವನ್ನು ಕಳೆದ ಭಾನು ವಾರ, ಜೂನ್ ೨೧ರಂದು “ನೈಜ ಮತ್ತು ತೋರಿಕೆಯ ಮಾನವ”\footnote{1. ನೋಡಿ, ಮಿ. ಜೆ.ಜೆ. ಗುಡ್ವಿನ್ ಅವರ ಪ್ರಕಟಿಸಿದ “ಮಾನವನ ನೈಜ ಸ್ವರೂಪ”, ಕೃತಿಶ್ರೇಣಿ, ೨, ಪುಟ ೧೮೩.} ಎಂಬ ಶೀರ್ಷಿಕೆಯಡಿಯಲ್ಲಿ ಕೊಡಲಾಯಿತು. ಇದರಲ್ಲಿ ಸ್ವಾಮೀಜಿಯವರು ಮಾನವನಿಗೆ ದೇವರು ಹಾಗೂ ಇನ್ನಿತರ ವಿಶ್ವದಿಂದ ಭಿನ್ನವಾದ ಪ್ರತ್ಯೇಕ ಆಸ್ತಿತ್ವವಿದೆ ಎಂಬ ಪರಿಗಣನೆಯಿಂದ ಕ್ರಮವಾಗಿ ಮುಂದುವರಿಯುತ್ತ, ಕೊನೆಗೆ ಅನಂತವು ಅನೇಕವಿರುವುದು ಅಸಾಧ್ಯ ಎಂದು ಒಪ್ಪಿಕೊಳ್ಳುವವರೆಗೆ ಚಿಂತನೆಯ ಧಾರೆಯನ್ನು ಎಳೆ ಎಳೆಯಾಗಿ ಕೊಂಡೊಯ್ದರು. ಇದರ ಅವಶ್ಯ ಪರಿಣಾಮವು ನಿಜವಾದ ಏಕತೆಯಲ್ಲ; ನಿಜವಾದದ್ದು ಅಭೇದ್ಯ ವಾದುದು, ಬದಲಿಸಲಾಗದ್ದು, ಆಗಿರಬೇಕು; ವೈಚಾರಿಕತೆಯು ನಮ್ಮನ್ನು ಈ ವಿದ್ಯಮಾನ ಪ್ರಪಂಚವೆಲ್ಲ ಭ್ರಮಾತ್ಮಕವಾದುದು ಎಂಬ ತೀರ್ಮಾನಕ್ಕೆ ತಂದಾಗ, ಇದರ ಮೂಲಕವೇ ಭ್ರಮೆಯಿಂದ ಪ್ರತ್ಯೇಕ ಅಸ್ತಿತ್ವಗಳೆಂದುಕೊಂಡಿರುವ ನಾವೆಲ್ಲರೂ ಹಾಯ್ದು ನಮ್ಮ ನೈಜ ಸ್ವರೂಪವನ್ನು ಕಂಡುಕೊಳ್ಳಬೇಕು ಎನ್ನುವುದು ಗೊತ್ತಾಗುವುದು. “ಏಕಂಸತ್ ವಿಪ್ರಾಃ; ಬಹುಧಾ ವದಂತಿ”. ಆದರೆ ಸ್ವಾಮೀಜಿ ಸೈದ್ಧಾಂತಿಕ ನೆಲೆಯಲ್ಲಿ ನಿಲ್ಲಿಸಿಬಿಡಲಿಲ್ಲ; ಅಂತಹ ಸಿದ್ಧಾಂತದ ಪ್ರಾಯೋಗಿಕ ಪರಿಣಾಮ ಏನಾಗುವುದೆಂಬುದನ್ನು ತೋರಿಸಿಕೊಟ್ಟರು - ಸಮಾಜದಲ್ಲಿ ಕ್ರಮೇಣ ವರ್ಗಭೇದ ಇಲ್ಲದಂತಾಗುವುದು; ಹಣ, ಅಧಿಕಾರಗಳ ವಿಷಯದಲ್ಲಿ ಹೆಚ್ಚು ಹೆಚ್ಚು ನಿಸ್ವಾರ್ಥತೆಯಿಂದಾಗಿ ಮನುಷ್ಯ - ಮನುಷ್ಯರ ನಡುವೆ ವ್ಯತ್ಯಾಸವಿಲ್ಲದಂತಾಗುವುದು. ಅಂತಹ ಧರ್ಮವೆಂದರೆ ವ್ಯಕ್ತಿತ್ವದ ನಾಶ ಎಂಬ ಆಕ್ಷೇಪಕ್ಕೆ ಉತ್ತರಿಸುತ್ತ, ಬದಲಾವಣೆಗೆ ಒಳಗಾಗುವುದು ನಿಜವಾದ ವ್ಯಕ್ತಿತ್ವವೆನ್ನಿಸಿಕೊಳ್ಳಲಾರದು; ಕ್ರಮೇಣ ನಮ್ಮ ಹಿಂದಿರುವ ನೈಜತೆಯನ್ನು ಕಂಡುಕೊಳ್ಳುವುದೆಂದರೆ ವ್ಯಕ್ತಿತ್ವವನ್ನು ಪಡೆದುಕೊಳ್ಳುವುದೇ ಹೊರತು ಅದರ ನಾಶವಲ್ಲ ಎಂದು ವಾದಿಸಿದರು.

ಹೀಗೆ, ಸ್ವಾಮೀಜಿ ಕೊಟ್ಟ ಮೂರೂ ಉಪನ್ಯಾಸಗಳು ಅಷ್ಟೊಂದು ಚೆನ್ನಾಗಿ ಗ್ರಹಿಸ ಲ್ಪಟ್ಟವು; ಅವುಗಳನ್ನು ಮುಂದುವರಿಸುವುದಕ್ಕೆ ಬೇಡಿಕೆಗಳೂ ಬಂದವು, ಇನ್ನೂ ಮೂರು ಉಪನ್ಯಾಸಗಳನ್ನು ಕೊಡುವಂತಾಗಲಿ ಎಂಬ ಆಗಲೆಂಬ ಹಾರೈಕೆಗಳೂ ಬೇಕಾದಷ್ಟು ಬಂದವು....

\end{tabular}{}
\textbf{ಸ್ವಿಟ್ಸರ್ಲೆಂಡ್ನ ಆಲ್ಪ್ಸ್ ಪರ್ವತಗಳ ಮೇಲೆ *\supskpt{\footnote{\enginline{Vivekananda in Indian Newspaper, p.117}}}}

\begin{center}
(ದಿ ಇಂಡಿಯನ್ ಮಿರರ್, ೨೨ ಸೆಪ್ಟೆಂಬರ್ ೧೮೯೬)
\end{center}

ಕಳೆದ ಆಗಸ್ಟ್ ೨೩ರ ದಿನಾಂಕದಲ್ಲಿ ಸ್ವಾಮಿ ವಿವೇಕಾನಂದರು ಲೂಸರ್ನಿ ಸರೋವರದಿಂದ ಪತ್ರ ಬರೆದಿರುವರು\footnote{2. ಲೂಸೆರ್ನಿನಿಂದ ಸ್ವಾಮೀಜಿ ಬರೆದ ಮೂರು ಅಪ್ರಕಟಿತ ಪತ್ರಗಳಲ್ಲಿ ಒಂದರ ಸಾರಾಂಶ ವಿದು ನಿಶ್ಚಿತ. ನೋಡಿ \enginline{Epistces} ಕೃತಿಶ್ರೇಣಿ \enginline{V} ನ ಮತ್ತು \enginline{} IV}. ಸಿಸ್ - ಆಲ್ಪೈನ್ ಪ್ರದೇಶದ ಅನೇಕ ಭಾಗಗಳಲ್ಲಿ ನಡೆದಾಡುತ್ತಿರುವ ಅವರು ಅಲ್ಲಿಯ ಪ್ರಕೃತಿಯ ಸುಂದರ ದೃಶ್ಯಗಳನ್ನು ನೋಡಿ ಆನಂದಿಸುತ್ತಿರುವರು. ದೃಶ್ಯಗಳು ಹಿಮಾಲಯಕ್ಕಿಂತ ಭವ್ಯತೆಯಲ್ಲಿ ಕಡಿಮೆಯೇನಿಲ್ಲವೆಂದು ಹೇಳಿರುವರು. ಆದರೂ, ಈ ಎರಡು ಪರ್ವತಪ್ರದೇಶಗಳಲ್ಲಿ ಎರಡು ವ್ಯತ್ಯಾಸದ ಅಂಶಗಳನ್ನು ಅವರು ಗುರುತಿಸುತ್ತಾರೆ. ಜನನಿವಾಸಗಳ ದಟ್ಟಣೆ ಅಲ್ಲಿಯ ಸೌಂದರ್ಯವನ್ನೆಲ್ಲ ಹಾಳುಗೆಡವುತ್ತಿದೆ, ಆದರೆ ಇಲ್ಲಿ ಹಿಮಾಲಯದಲ್ಲಿ ಅಂತಹ ಗಮನಾರ್ಹ ಪ್ರವೃತ್ತಿ ಇನ್ನೂ ಕಂಡುಬರುತ್ತಿಲ್ಲ. ಇಲ್ಲಿಯ ಜನರು ಅದನ್ನು ಬೇಸಿಗೆಯ ತಾಣವಾಗಿ, ಆರೋಗ್ಯ ಧಾಮವಾಗಿ ಬಳಸುವರು; ಅಲ್ಲಿ ಜನರು ಬರುವುದು ಭಕ್ತರಾಗಿ, ತೀರ್ಥಯಾತ್ರೆಗೆಂದು. ಸ್ವಾಮಿಗಳು ಸದ್ಯದಲ್ಲೇ ಜರ್ಮನಿಗೆ ಭೇಟಿ ಕೊಡಲಿರುವರು; ಅಲ್ಲಿ ಪ್ರೊಫೆಸರ್ ಡಾಯ್ಸನ್ ಅವರೊಂದಿಗೆ ಸಂದರ್ಶನ ನಡೆಯಲಿದೆ; ಅನಂತರ, ಸೆಪ್ಟೆಂಬರ್ ೨೪ರ ವೇಳೆಗೆ ಇಂಗ್ಲೆಂಡಿಗೆ ಹಿಂದಿರುಗಲಿರುವರು. ಅವರೇ ಹೇಳುವಂತೆ, ಬಹುಶಃ ಮುಂದಿನ ಚಳಿ ಗಾಲದ ಹೊತ್ತಿಗೆ ಭಾರತಕ್ಕೆ ಹಿಂದಿರುಗುವರು. ಇಲ್ಲಿ ಹಿಮಾಲಯದಲ್ಲಿ ಸ್ವಲ್ಪ ಕಾಲ ವಾಸಿಸಬೇಕೆಂಬುದು ಅವರ ಇಷ್ಟ.

\begin{center}
\textbf{“ವಿಶ್ವಧರ್ಮದ ಆದರ್ಶ”\supskpt{\footnote{\enginline{Vivekananda in Indian Newspaper, pp.331-33}}}}
\end{center}

\begin{center}
(ದಿ ಜರ್ನಲ್ ಆಫ್ ದಿ ಮಹಾಬೋಧಿ ಸೊಸೈಟಿ, ನವೆಂಬರ್ ೧೮೯೬)
\end{center}

ಮದರಾಸಿನ ಬ್ರಹ್ಮವಾದಿನ್ ಪಬ್ಲಿಷಿಂಗ್ ಕಂಪನಿಯವರು ತಾವು ಪ್ರಕಟಿಸಿರುವ “ವಿಶ್ವಧರ್ಮದ ಆದರ್ಶ” ಎಂಬ ಕಿರುಹೊತ್ತಿಗೆಯನ್ನು ನಮಗೆ ಕೊಡುಗೆಯಾಗಿ ಕಳುಹಿಸಿರುವರು. ಇದು ಸ್ವಾಮಿ ವಿವೇಕಾನಂದರು ಅಮೆರಿಕಾದಲ್ಲಿ ಕೊಟ್ಟಿರುವ ಒಂದು ಉಪನ್ಯಾಸ. ಉಪನ್ಯಾಸವು ತುಂಬ ಆಸಕ್ತಿಕರವೂ ಪ್ರಬೋಧಕವೂ ಆಗಿದೆ. ಇದು ಧರ್ಮ ಗಳ ವೈವಿಧ್ಯದ ನಡುವೆ ಒಂದು ಸಾಮರಸ್ಯದ ಪ್ರಯತ್ನವಾಗಿದೆ. ಈ ಕಿರುಹೊತ್ತಿಗೆ ಯನ್ನು ನಾವು ಕಾಲಧರ್ಮದ ಲಕ್ಷಣವಾಗಿ ಎತ್ತಿ ಹಿಡಿಯುತ್ತೇವೆ; ಏಕೆಂದರೆ, ಮಾನವ ರೆಲ್ಲರೂ ಈ ಧಾರ್ಮಿಕ ಸಮನ್ವಯ ಸಮಸ್ಯೆಯ ಪ್ರಾಮುಖ್ಯತೆಗೆ ಎಚ್ಚೆತ್ತು ಕೊಳ್ಳಲು ಪ್ರಾರಂಭಿಸಿದ್ದಾರೆ ಎಂಬುದು ವೇದ್ಯವೇ ಇದೆ. ಇತ್ತೀಚೆಗೆ, ಈ ದೇಶಗಳಲ್ಲಿ ವಿಭಿನ್ನ ಧರ್ಮಪಂಥಗಳ ಮುಂದಾಳುಗಳು ಈ ಧಾರ್ಮಿಕ ವೈವಿಧ್ಯತೆಯಲ್ಲಿ ಸಾಮರಸ್ಯವನ್ನು ತರಲು ತಮ್ಮದೇ ಆದ ರೀತಿಯಲ್ಲಿ ಪ್ರಯತ್ನಿಸಿ ಸೋತಿದ್ದಾರೆ; ತಮ್ಮದೇ ಸಿದ್ಧಾಂತಗಳನ್ನು ಮತಪಂಥಗಳ ವಿಕೃತ ದೃಷ್ಟಿಯ ಆಧಾರದ ಮೇಲೆ ಎತ್ತಿಹಿಡಿಯುವ ಉತ್ಸುಕತೆಯನ್ನೇ ಪ್ರದರ್ಶಿಸಿದ್ದಾರೆ. ಧಾರ್ಮಿಕ ಸಮನ್ವಯದ ಈ ಸಮಸ್ಯೆಗೆ ಸ್ವಾಮಿ ವಿವೇಕಾನಂದರು ಒಂದು ತಾತ್ತ್ವಿಕವಾದ, ಅದೇ ಸಮಯಕ್ಕೆ ಅತ್ಯಂತ ಪ್ರಾಯೋಗಿಕವಾದ ಪರಿಹಾರವನ್ನು ಮಂಡಿಸಿದ್ದಾರೆ. ಅವರ ಪ್ರಕಾರ, ಎಂದೆಂದಿಗೂ ಮುಗಿಯದ ಹೋರಾಟದಲ್ಲಿ ತೊಡಗಿರುವ ಧಾರ್ಮಿಕ ಭಿನ್ನತೆಗಳ ನಡುವೆ ವೇದಾಂತವು ಒಗ್ಗೂಡಿಸುವ ಒಂದು ಬಂಧವಾಗಲಿದೆ. ಹೊರಪ್ರಪಂಚದಲ್ಲಿಯಂತೆಯೇ ಅಂತರಂಗ ಪ್ರಪಂಚದಲ್ಲೂ ಸಹ, ಕೇಂದ್ರತ್ಯಾಗಿ ಹಾಗೂ ಕೇಂದ್ರಾಭಿಗಮನ ಕ್ರಿಯೆಗಳಿವೆ. ನಾವು ಯಾವುದನ್ನೋ ವಿಕರ್ಷಿಸುತ್ತೇವೆ, ಯಾವುದಕ್ಕೋ ಆಕರ್ಷಿತರಾಗುತ್ತೇವೆ. ಇಂದು ಯಾವುದಕ್ಕೋ ಆಕರ್ಷಿತರಾಗಿರುವ ನಾವು, ನಾಳೆ ಇನ್ನಾ ವುದರಿಂದಲೋ ವಿಕರ್ಷಿತರಾಗುತ್ತೇವೆ. ಎಲ್ಲ ಸನ್ನಿವೇಶಗಳಲ್ಲೂ ಎಲ್ಲ ಕಾಲಕ್ಕೂ ಒಂದೇ ನಿಯಮವನ್ನು ಅನ್ವಯಿಸಲಾಗದು. “ಧರ್ಮವೆಂಬುದು ಮಾನವ ಚಿಂತನೆಯ ಮತ್ತು ಜೀವನದ ಅತ್ಯುನ್ನತ ಸ್ಥಳ; ಇಲ್ಲಿ ಈ ಎರಡು ಬಲಗಳ ಕಾರ್ಯ ಅತ್ಯಂತ ಗಮನಾರ್ಹವಾಗಿರುತ್ತದೆ.” ಮೊದಲಿಗೆ, ಘೋರ ಕಾಳಗದ ಈ ಸ್ತರದಲ್ಲಿ ಆವಿಚ್ಛಿನ್ನ ಸಮನ್ವಯದ ಅಧಿಪತ್ಯ ನಡೆಯಲಾರದು ಎಂದು ತೋರಬಹುದು. ಪ್ರತಿಯೊಂದು ಧರ್ಮದಲ್ಲಿಯೂ ಮೂರು ಭಾಗಗಳಿವೆ - ಅವೆಂದರೆ ತತ್ತ್ವ, ಶಾಸ್ತ್ರಪುರಾಣಾದಿಗಳು ಮತ್ತು ಕ್ರಿಯಾವಿಧಿಗಳು. ಧರ್ಮವೆಂದು ಮನ್ನಣೆ ಪಡೆದ ಪ್ರತಿಯೊಂದರಲ್ಲಿಯೂ ಈ ಮೂರು ಅಂಶಗಳಿವೆ. ಆದರೆ ಇಡೀ ಪ್ರಪಂಚಕ್ಕೆ ಒಂದು ಸಾರ್ವತ್ರಿಕ ತತ್ತ್ವ, ಪುರಾಣಪ್ರಣಾಳಿಕೆ, ಕ್ರಿಯೆಗಳು ಇರಲು ಸಾಧ್ಯವಿಲ್ಲ. ಎಂದಮೇಲೆ ವಿಶ್ವಮಾನ್ಯತೆ ಎಲ್ಲಿದೆ? ಧರ್ಮಕ್ಕೆ ಸಾರ್ವತ್ರಿಕ ರೂಪವನ್ನು ಕೊಡುವುದು ಹೇಗೆ ಸಾಧ್ಯ? ಎನ್ನು ತ್ತಾರೆ ಸ್ವಾಮಿ ವಿವೇಕಾನಂದರು.\footnote{1. ೧೮೯೬ ಜನವರಿ ೧೨ರಂದು ಅಮೇರಿಕಾದಲ್ಲಿ ನೀಡಿದ ಉಪನ್ಯಾಸ ನೋಡಿ, ಕೃತಿಶ್ರೇಣಿ ೨, ಪುಟ ೨೨೧}

“ವಿಶ್ವಭ್ರಾತೃತ್ವದ ಬಗ್ಗೆ ನಾವೆಲ್ಲರೂ ಕೇಳುತ್ತಲೇ ಇರುತ್ತೇವೆ; ಹೇಗೆ ಸಾಮಾಜಿಕ ಸಂಘಟನೆಗಳು ಇದನ್ನು ಬೋಧಿಸಲು ಮೇಲೆದ್ದು ನಿಲ್ಲು ತ್ತವೆ, ಹೇಗೆ ನಾವೆಲ್ಲರೂ ಸಮಾ ನರು, ಆದ್ದರಿಂದ ನಮ್ಮದೊಂದು ಪಂಥವನ್ನು ಸ್ಥಾಪಿಸೋಣ ಎಂದು ಮದ್ಯದ ಅಮಲಿನರುವವರಂತೆ ನಾವು ಘೋಷಣೆ ಹಾಕುತ್ತೇವೆ! ನೀವು ಪಂಥವೊಂದನ್ನು ಮಾಡಿದ ಕೂಡಲೆ, ಆ ಮೂಲಕವೆ ಸಮಾನತೆಯ ವಿರುದ್ಧವಾಗಿರುವಿರಿ ನೀವು! ಇನ್ನೆಲ್ಲಿದೆ ಸಮಾನತೆ?”

ಮಹಮ್ಮದೀಯರು ವಿಶ್ವಭ್ರಾತೃತ್ವದ ಮಾತನ್ನಾಡುತ್ತಾರೆ, ಆದರೆ ನಿಜವಾಗಿ ಅವರಿಂದ ಹೊರಬರುವುದೇನು? ಯಾರು ಮಹಮ್ಮದೀಯರಲ್ಲವೋ ಅವರನ್ನು ಭ್ರಾತೃತ್ವದ ಒಳ ಗಡೆ ಸೇರಿಸಿಕೊಳ್ಳುವುದಿಲ್ಲ; ಅವರ ಕತ್ತನ್ನು ಕತ್ತರಿಸಿ ಹಾಕುತ್ತಾರೆ. ಅವರ ಮಾತುಗಳನ್ನೇ ಉದ್ಧರಿಸಿ ಹೇಳುವುದಕ್ಕಿಂತ ಹೆಚ್ಚಿನದನ್ನು ನಾವೇನೂ ಮಾಡಲಾರೆವು ಎಂದೇ ಅನ್ನಿಸುತ್ತದೆ; ಅವರಾದರೋ ತಮ್ಮ ಅಚ್ಚರಿಯ ನಿಚ್ಚಳತೆಯಿಂದ, ಆಳ ದೃಷ್ಟಿಯಿಂದ, ಗಮನಾರ್ಹವಾದ ಮನೋ ವೈಶಾಲ್ಯದಿಂದ ವಿಶ್ವಧರ್ಮದ ತತ್ತ್ವವನ್ನು ಶಕ್ತಿಯುತವಾಗಿ ಪ್ರತಿಪಾದಿಸುತ್ತಾರೆ (ಕೃತಿಶ್ರೇಣಿ, ೨, ೨೧೮-೨೩೬)....

ಸಮಾಜದಲ್ಲಿ ನಾನಾ ಸ್ವಭಾವಗಳ ಜನರಿರುತ್ತಾರೆ. ಕೆಲವರು ಕ್ರಿಯಾಶೀಲರಾದ ಕೆಲಸಗಾರರು; ಇನ್ನು ಭಾವಜೀವಿ ಇರುವನು, ಅನುಭಾವಿ ಇರುವನು, ಕೊನೆಯದಾಗಿ ತಾತ್ತ್ವಿಕನಿರುವನು. ಈ ಎಲ್ಲ ನಾಲ್ಕು ದಿಕ್ಕುಗಳಲ್ಲೂ ಸುಂದರ ಸಮತೋಲನ ಸಾಧಿಸುವುದರಲ್ಲಿ ಮಾನವ ಜನತೆಗೆ ಸಹಾಯಮಾಡುವ ಪ್ರಯತ್ನದಲ್ಲಿ ತಮ್ಮ ಧರ್ಮದ ಆದರ್ಶ ಅಡಗಿದೆ, ಈ ಧರ್ಮವನ್ನು ಭಾರತದಲ್ಲಿ ಯೋಗ ಎಂದು ಕರೆಯುತ್ತಾರೆ, ಎಂದು ಸಾರು ವಾಗ ವಿವೇಕಾನಂದರು ತಮ್ಮ ತತ್ತ್ವವೆಲ್ಲದರ ಆಧಾರಶ್ರುತಿಗೆ ಬರುತ್ತಾರೆ. ಕೆಲಸಗಾರನನ್ನು ಕರ್ಮಯೋಗಿಯೆನ್ನುವರು; ಪ್ರೇಮದ ಮೂಲಕ ಮುಕ್ತಿಯನ್ನು ಪಡೆಯಲೆತ್ನಿಸು ವವನನ್ನು ಭಕ್ತಿಯೋಗಿಯೆನ್ನುವರು; ಅನುಭಾವದ ಮೂಲಕ ಅದನ್ನು ಸಾಧಿಸುವವನನ್ನು ರಾಜಯೋಗಿಯೆನ್ನುವರು; ತತ್ತ್ವಾನ್ವೇಷಣೆಯಿಂದ ಅದನ್ನೇ ಪಡೆಯಬಯಸುವವನನ್ನು ಜ್ಞಾನಯೋಗಿ ಎನ್ನುವರು. ಈ ಎಲ್ಲ ಸ್ವಭಾವಗಳ ಜನರಿಗೂ ಸ್ಥಳವಿರುವ ಧರ್ಮ, ವಿಭಿನ್ನ ಒಲವುಗಳುಳ್ಳ ಜನರ ದಾಹವನ್ನು ತೃಪ್ತಿಪಡಿಸಬಲ್ಲ ಧರ್ಮ, ಅದನ್ನು ವಿಶ್ವ ಧರ್ಮವೆನ್ನಬಹುದು; ಆ ಧರ್ಮವೇ ವೇದಾಂತ. ಈ ಪ್ರಶಂಸನೀಯ ಪುಟ್ಟ ಪುಸ್ತಕವನ್ನು ನಾವು ನಮ್ಮ ಓದುಗರಿಗೆ ಅತ್ಯಂತ ಹಾರ್ದಿಕವಾಗಿ ಶಿಫಾರಸು ಮಾಡುತ್ತೇವೆ. ಏಕೆಂದರೆ, ಇಂದು ದೇವತಾವಾದಿಗಳ ಗಮನವನ್ನು ಗಂಭೀರವಾಗಿ ಸೆಳೆಯುತ್ತಿರುವ ಅತ್ಯಂತ ಸೂಕ್ಷ್ಮವಾದ ಸಮಸ್ಯೆಯ ಮೇಲಣ ಸ್ಪಷ್ಟವಾದ ಹಾಗೂ ಖಚಿತವಾದ ದೃಷ್ಟಿಕೋನಗಳನ್ನು ಇದು ಒಳಗೊಂಡಿರುತ್ತದೆ. ಪುಸ್ತಕದ ಬೆಲೆ ಮೂರಾಣೆ; ಇದನ್ನು ಮದರಾಸಿನ ತಿರುವಲ್ಲಿ ಕ್ಕೇಣಿಯಲ್ಲಿರುವ ಬ್ರಹ್ಮವಾದಿನ್ ಕಛೇರಿಯಿಂದ ಪಡೆದುಕೊಳ್ಳಬಹುದು.

\begin{center}
\textbf{ರಣಜಿತ್ ಸಿಂಗರಿಗೆ ಕೊಟ್ಟ ಔತಣಕೂಟ*\supskpt{\footnote{\enginline{New Discoveries Vol. 4, pp. 479-80.}}}}
\end{center}

\begin{center}
(ದಿ ಇಂಡಿಯನ್ ಮಿರರ್, ೧೬ ಡಿಸೆಂಬರ್ ೧೮೯೬)
\end{center}

ಈ ತಿಂಗಳ (ನವೆಂಬರ್) ೨೧ರಂದು ಕೇಂಬ್ರಿಡ್ಜ್ “ಇಂಡಿಯನ್ ಮಜ್ಲಿಸ್” ಸಂಘದವರು ವಿಶ್ವವಿದ್ಯಾನಿಲಯದ ಆರ್ಮ್ಸ್ ಹೋಟೆಲ್ (ಕೇಂಬ್ರಿಡ್ಜ್ನಲ್ಲಿ)ನಲ್ಲಿ ರಾಜಕುಮಾರ ರಣಜಿತ್ ಸಿಂಗ್ಜಿ ಮತ್ತು ಮಿ. ಅತುಲ್ಚಂದ್ರ ಚಟರ್ಜಿ ಅವರುಗಳಿಗೆ ಒಂದು ಶುಭಹಾರೈಕೆಯ ಔತಣಕೂಟವನ್ನು ಏರ್ಪಡಿಸಿದ್ದರು. ಸೇಂಟ್ ಜಾನ್ಸ್ ಕಾಲೇಜಿನ ಮಿ. ಹಾಫಿಜ್ ಜಿ. ಸರವೀರ್ ಪೀಠವನ್ನು ಆಲಂಕರಿಸಿದ್ದರು. ಸುಮಾರು ಐವತ್ತು ಜನ ಭಾರತೀಯರು ಹಾಗೂ ಕೆಲವು ಇಂಗ್ಲಿಷ್ ಜನರು ಕೂಟದಲ್ಲಿದ್ದರು....

ನಂತರ ಕಿವಿಗಡಚಿಕ್ಕುವ ಕರತಾಡನ ಜಯಘೋಷಗಳ ನಡುವೆ ಸ್ವಾಮಿ ವಿವೇಕಾ ನಂದರು ಪ್ರತ್ಯುತ್ತರ\footnote{1. ಇದರ ಪದಶಃ ವರದಿ ಲಭ್ಯವಿಲ್ಲ. ವಿವರಗಳಿಗೆ \enginline{The Majkis in cambridge'} ೧೮೯೭ ಜನವರಿ ೮ರ \enginline{Indian Newspapers,} ವರದಿ ನೋಡಿ.} (ಭಾರತದ ಸ್ವಸ್ತಿಪಾನಕ್ಕಾಗಿ) ಕೊಡಲು ಎದ್ದುನಿಂತರು. ಸ್ವಸ್ತಿಪಾನದ ಪ್ರತ್ಯುತ್ತರಕ್ಕಾಗಿ ತಮ್ಮನ್ನು ಆಯ್ಕೆ ಮಾಡಿದ್ದೇಕೆ, ಗಾತ್ರದಲ್ಲಿ ಭಾರತದ ರಾಷ್ಟ್ರೀಯ ಪ್ರಾಣಿಯಂತಿರುವೆನೆಂದೇ (ನಗು), ಎಂಬುದು ನನಗೆ ಗೊತ್ತಾಗುತ್ತಿಲ್ಲ, ಎಂದು ಪ್ರಾರಂಭಿಸಿದ ಅವರು, ಆ ಸಂಜೆಯ ಅತಿಥಿಯನ್ನು ಅಭಿವಂದಿಸಿದರು; ಸಭಾಧ್ಯಕ್ಷರು ಮಿ. ಚಟರ್ಜಿ ಯವರು ಭಾರತದ ಈ ಹಿಂದಿನ ಚರಿತ್ರಕಾರರು ಮಾಡಿರುವ ತಪ್ಪನ್ನು ತಿದ್ದಲಿರುವರು ಎಂದು ಹೇಳಿದ್ದನ್ನು ಅಕ್ಷರಶಃ ನಿಜವೆಂದು ಅನುಮೋದಿಸಿದರು. ಏಕೆಂದರೆ, ಭೂತ ದಿಂದಲೇ ಭವಿಷ್ಯತ್ತು ಮೂಡಬೇಕು; ಭವಿಷ್ಯತ್ತಿಗೆ ಈ ಹಿಂದೆ ನಡೆದದ್ದರ ನೈಜ ತಿಳಿವಳಿಕೆಗಿಂತ ಉತ್ತಮವಾದ ಹಾಗೂ ಶಾಶ್ವತವಾದ ತಳಹದಿ ತಮಗೆ ತಿಳಿಯದೆಂದು ನುಡಿದರು. ಕಳೆದುಹೋದುದನ್ನು ಪ್ರತಿನಿಧಿಸುವ ಅಸಂಖ್ಯ ಕಾರಣಗಳ ಪರಿಣಾಮ ವಾಗಿ ರೂಪುಗೊಂಡಿರುವುದೇ ವರ್ತಮಾನ ಎಂದರು. ನಾವು ಯೂರೋಪಿಯನ್ನರಿಂದ ಕಲಿಯುವ ಅನೇಕ ಸಂಗತಿಗಳಿರುವುವಾದರೂ, ಕಳೆದು ಹೋದ ಭಾರತದ ವೈಭವವು ನಮಗೆ ಪಾಠ ಕಲಿಸಬೇಕು, ಸ್ಫೂರ್ತಿ ತುಂಬಬೇಕು. ಏಳುಬೀಳುಗಳು ಸಹಜ; ಲೋಕದಲ್ಲಿ ಅವು ಎಲ್ಲೆಲ್ಲೂ ಕಾಣಸಿಗುವುವು... (ಸ್ವಾಮಿಗಳ ಉಕ್ತಿಯ ಉದ್ಧರಣೆಗಾಗಿ ಮುಂದಿನ ವರದಿಯನ್ನು ನೋಡಿ).

\begin{center}
\textbf{ಕೇಂಬ್ರಿಡ್ಜ್ನಲ್ಲಿನ ಮಜ್ಲಿಸ್\supskpt{\footnote{\enginline{Vivekananda in Indian Newspaper, pp.310-11}}}}
\end{center}

\begin{center}
(ದಿ ಅಮೃತ ಬಜಾರ್ ಪತ್ರಿಕಾ, ೮ ಜನವರಿ ೧೮೯೭)
\end{center}

.... ಕೂಟವು ವಿಶಿಷ್ಟವಾಗಿತ್ತು. ಏಕೆಂದರೆ ಭಾರತೀಯರು ರಣಜಿತ್ ಸಿಂಗ್ ಮತ್ತು ಅತುಲ್ಚಂದ್ರ ಚಟರ್ಜಿಯ ಯಶಸ್ಸನ್ನು ಕುರಿತು ಮಾತನಾಡುವುದಕ್ಕೆಂದು ಸೇರಿದ್ದರು (ಮಜ್ಲಿಸ್ನಲ್ಲಿ ಅವರೆಲ್ಲರೂ ಮಾತನಾಡುವರು). ಈ ಇಬ್ಬರ ಜೊತೆಗೆ ಪ್ರೊಫೆಸರ್ ಬೋಸ್ ಅವರ ಹೆಸರನ್ನು ಸೇರಿಸದೆ ಇದ್ದುದು ವಿಷಾದನೀಯ; ಮತ್ತು ಸಂದರ್ಭದಲ್ಲಿ ಹಾಜರಿದ್ದ ಸ್ವಾಮಿ ವಿವೇಕಾನಂದರೂ ಸಹ ಮನ್ನಣೆಗೆ ಯೋಗ್ಯರಾಗಿದ್ದರೆಂದು ನಮಗೆ ಅನ್ನಿಸುತ್ತದೆ. ಪಶ್ಚಿಮದಲ್ಲಿ ಭಾರತೀಯರು ಏನನ್ನು ಸಾಧಿಸಬಲ್ಲರೆಂಬುದನ್ನು ತೋರಿಸಲು ಕೊನೆಯ ಈ ಇಬ್ಬರನ್ನು ಗಮನಿಸದೆ ಇರುವ ತಪ್ಪನ್ನು ನಾವಂತೂ ಮಾಡಲಾರೆವು.

ಅಮೆರಿಕನ್ನರ ಮನಸ್ಸಿನಲ್ಲಿದ್ದ ಭಾರತೀಯರೆಲ್ಲ ಅನಾಗರಿಕರು, ಮೂಢನಂಬಿಕೆಗಳಿಗೆ ಒಳಗಾದವರು, ಮತ್ತು ರಾಕ್ಷಸೀ ಕ್ರೂರಕೃತ್ಯಗಳಿಗೆ ಒಗ್ಗಿಹೋದವರು ಎಂಬ ಚಿತ್ರವನ್ನು ತೆಗೆದುಹಾಕಿದ್ದೇ ಸ್ವಾಮೀಜಿ ಮಾಡಿದ ಕೆಲಸ. ಸ್ವಾಮೀಜಿಯವರ ಪಶ್ಚಿಮ ದೇಶಗಳ ಭೇಟಿ ಇದನ್ನು ಸಾಧಿಸಿತು; ಸರ್ ಚಾರ್ಲ್ಸ್ ಎಲಿಯಟ್ ಹೇಳಿದ ಹಾಗೆ ಭಾರತೀಯರು ಕೀಳು ಜನಾಂಗವಲ್ಲ, ತತ್ತ್ವ ಧರ್ಮಗಳಂಥ ವಿಷಯಗಳಲ್ಲಿ ಪಾಶ್ಚಾತ್ಯರಿಗೆ ಸಹಗೊತ್ತಿಲ್ಲದ ಅನೇಕ ಸಂಗತಿಗಳನ್ನು ತಿಳಿದಿರುವರು ಎಂಬ ಅಭಿಪ್ರಾಯ ಅನೇಕ ಕಡೆಗಳಲ್ಲಿ ಮೂಡು ವಂತೆ ಮಾಡಿತು. ಸ್ವಾಮಿಜಿ ಪಾಶ್ಚಾತ್ಯ ದೇಶಗಳಿಗೆ ಹೋದುದು ಪಶ್ಚಿಮದಲ್ಲಿ ನಿಸ್ಸಂದೇಹ ವಾಗಿ ಭಾರತೀಯರ ಶೀಲಸಂವರ್ಧನೆಗೆ ಕಾರಣವಾಯಿತು....

ಸ್ವಾಮಿ ವಿವೇಕಾನಂದರು ಹೀಗೆಂದು ಹೇಳಿದರು:-

“ಇಂದು ಭಾರತವು ಅವನತ ಪರಿಸ್ಥಿತಿಯಲ್ಲಿದ್ದರೂ, ಮತ್ತೊಮ್ಮೆ ಅದು ಮೇಲಕ್ಕೇರು ವುದು ಖಂಡಿತ. ಭಾರತವು ಮಹಾ ತತ್ತ್ವಜ್ಞಾನಿಗಳನ್ನು, ಇನ್ನೂ ಮೇಲ್ಮಟ್ಟದ ಪ್ರಬೋಧ ಕರುಗಳನ್ನು, ಪ್ರವಾದಿಗಳನ್ನು ಹೊಂದಿದ್ದ ಕಾಲವೊಂದಿತ್ತು. ಆ ಕಾಲದ ನೆನಪು ಭಾರತೀಯ ರಿಗೆ ಭರವಸೆಯನ್ನು ಉಂಟುಮಾಡಬೇಕು, ಆತ್ಮವಿಶ್ವಾಸವನ್ನು ತುಂಬಬೇಕು. ಭಾರತದ ಇತಿಹಾಸದಲ್ಲಿ ದೇಶವು ಹೀನಸ್ಥಿತಿಗೆ ಹೋಗಿರುವುದು ಇದೇನು ಮೊದಲನೆಯ ಸಲವಲ್ಲ. ಅವನತಿಯ, ಕೀಳ್ಮಟ್ಟದ ಅವಧಿಗಳು ಹಿಂದೆಯೂ ಪ್ರಾಪ್ತವಾಗಿದ್ದವು; ಆದರೆ ಭಾರತವು ಕಾಲಾನುಕ್ರಮದಲ್ಲಿ ಅವುಗಳನ್ನೆಲ್ಲ ಜಯಿಸಿಕೊಂಡೇ ಬಂದಿರುವುದು; ಮುಂದೆಯೂ ಭಾರತವು ಜಯಿಸುವುದರಿಲ್ಲ ಸಂಶಯವಿಲ್ಲ.\footnote{1. ಇದರ ಪದಶಃ ವರದಿ ಲಭ್ಯವಿಲ್ಲ. ಹಿಂದಿನ ೧೮೯೬ ಡಿಸೆಂಬರ್ ೧೬ರ ಪತ್ರಿಕಾವರದಿ ‘ರಣಜಿತ್ ಸಿಂಗ್ರಿಗೆ ಕೊಟ್ಟ ಕ್ಷಾತಣ ಕೂಟ’ ನೋಡಿ.}

\begin{center}
\textbf{ಪಶ್ಚಿಮದೇಶಗಳಲ್ಲಿ ವಿವೇಕಾನಂದರು\supskpt{\footnote{\enginline{Vivekananda in Indian Newspaper, p. 312.}}}}
\end{center}

\begin{center}
(ದಿ ಅಮೃತ ಬಜಾರ್ ಪತ್ರಿಕಾ, ೨೦ ಜನವರಿ ೧೮೯೭)
\end{center}

ಸ್ವದೇಶಕ್ಕೆ ಹಿಂದಿರುಗಿದ ಸ್ವಾಮಿ ವಿವೇಕಾನಂದರಿಗೆ ವೀರೋಚಿತವಾದ ಉತ್ಸಾಹ ಪೂರ್ಣ ಸ್ವಾಗತವನ್ನು ನೀಡಲಾಯಿತು. ಇಂಗ್ಲೆಂಡಿನಿಂದ ಅವರ ಬಗ್ಗೆ ನಾವು ಕೇಳಿದ ಕೊನೆಯ ಸಂದರ್ಭ ಅವರ ಆಂಗ್ಲ ಶಿಷ್ಯರುಗಳು ಅವರಿಗೆ ವಿದಾಯವಾಚನವನ್ನು ಸಲ್ಲಿಸುತ್ತ, ಭಾರತದ ಬಗ್ಗೆ ತಮಗಿರುವ ತೀರಲಾರದ ಪ್ರೇಮವನ್ನು ಅಭಿವ್ಯಕ್ತಿಗೊಳಿಸಿದಾಗಿನದು....

ಸ್ವಾಮಿ ವಿವೇಕಾನಂದರು ಪಶ್ಚಿಮದೇಶಗಳಲ್ಲಿ ಏನು ಮಾಡುತ್ತಿದ್ದರು ಎಂಬ ಬಗ್ಗೆ ಯಾರೊಬ್ಬರಿಗೂ ಖಚಿತವಾಗಿ ತಿಳಿದಿಲ್ಲ. ಅವರು ಅಮೆರಿಕಾದಲ್ಲಿ ಮತ್ತು ಇಂಗ್ಲೆಂಡಿನಲ್ಲಿ ಸಹಾ ಸ್ವಲ್ಪ ಪ್ರಭಾವ ಬೀರಿರುವುದನ್ನು ಕೇಳುತ್ತಿರುವೆವು....

ಏನೇ ಇರಲಿ, ಸ್ವಾಮಿಗಳಿಗೆ ತಮ್ಮೆದುರಿಗೆ ಇರುವ ಪ್ರಚಾರಕಾರ್ಯದ ಪರಿಜ್ಞಾನ ಚೆನ್ನಾಗಿದೆ. ವೇದಾಂತವು ಸತ್ಯವನ್ನೇ ಬೋಧಿಸುತ್ತದೆ ಎನ್ನುತ್ತಾರೆ ಅವರು. ಮನುಷ್ಯನು ದಿವ್ಯತೆಯ ಆವಿರ್ಭಾವವಾಗಿದ್ದಾನೆ; ಉಚ್ಚ, ನೀಚ ಇಬ್ಬರೂ ಒಬ್ಬನೇ ಭಗವಂತನ ಆವಿರ್ಭಾವ ಎಂಬುದೇ ಆ ಸತ್ಯ. ಮನುಷ್ಯನ ಮುಕ್ತಿಗೆ ಜ್ಞಾನ ಮಾತ್ರವೇ ಸಾಕಾಗುತ್ತದೆ ಎಂದು ಅವರು ಒಪ್ಪಿಕೊಳ್ಳುವುದಿಲ್ಲ. ಅವರೆನ್ನುತ್ತಾರೆ:-

ಮನುಷ್ಯನ ಜ್ಞಾನವು ಕೇವಲ ಸಿದ್ಧಾಂತ ಮಾತ್ರವಾಗಿರದೆ, ನೈಜ ಜೀವನ ವಾಗಿರಬೇಕು. ಧರ್ಮವೆಂದರೆ ಸಾಕ್ಷಾತ್ಕಾರ. ಮಾತುಗಾರಿಕೆಷ್ಟೇ, ಸಿದ್ಧಾಂತಗಳು, ವಾದವಿವಾದ ಗಳು - ಇವು ಎಷ್ಟೋ ಆಹ್ಲಾದಕರವಾಗಿದ್ದರೂ ಇವು ಧರ್ಮವಲ್ಲ. ಧರ್ಮ ಇರುವುದು, ಆಗುವುದು, ಕೇಳುವುದು ಮತ್ತು ಅದಕ್ಕೆ ಪ್ರತಿಕ್ರಿಯೆ ನೀಡುವುದಷ್ಟೇ ಅಲ್ಲ. ಅದು ಬೌದ್ಧಿಕ ಒಪ್ಪಿಗೆಯಲ್ಲ; ಆದರೆ ಮನುಷ್ಯನ ಇಡಿಯ ಸ್ವರೂಪವೇ ಅದಕ್ಕನುಗುಣವಾಗಿ ಬದ ಲಾಗಬೇಕಾಗುವುದು. ಧರ್ಮವೆಂದರೆ ಅದು. ಬೌದ್ಧಿಕ ಒಪ್ಪಿಗೆಯಿಂದ ನಾವು ನೂರೆಂಟು ಬಗೆಯ ಮೂರ್ಖ ತೀರ್ಮಾನಗಳಿಗೆ ಬರಬಹುದು, ಮಾರನೆಯ ದಿನವೇ ಬದಲಿಸ ಬಹುದು; ಆದರೆ ಈ ಇರುವುದು ಮತ್ತು ಆಗುವುದು -ಇದೇ ಧರ್ಮವೆಂದರೆ.”

ಮೇಲ್ಕಂಡ ಉತ್ಕೃಷ್ಟ ಭಾವನೆಗಳಲ್ಲಿ, ಸ್ವಾಮಿಗಳು ತಾವು ಪರಿಸ್ಥಿತಿಯನ್ನು ತುಂಬ ಚೆನ್ನಾಗಿ ಅರ್ಥಮಾಡಿಕೊಂಡಿರುವುದನ್ನು ತೋರಿಸಿಕೊಡುತ್ತಾರೆ. ಮನುಷ್ಯನಿಗೆ ಯಾವುದು ಪುನರ್ಜನ್ಮವನ್ನು ತಂದುಕೊಡುವುದೋ ಅದೇ ಧರ್ಮ. ಧರ್ಮದ ಪ್ರಭಾವದಿಂದ ಮನುಷ್ಯನು ಈ ಹಿಂದೆ ಏನಾಗಿದ್ದನೋ ಅದಕ್ಕಿಂತ ಬೇರೆಯೇ ಆಗಿಬಿಡುತ್ತಾನೆ. ಧರ್ಮವು ಮನುಷ್ಯನಲ್ಲಿ ಈ ಪರಿಣಾಮವನ್ನು ಉಂಟುಮಾಡದೆ ಇದ್ದರೆ ಅವನ ಧರ್ಮವೆಂಬುದು ಒಂದು ಭ್ರಾಂತಿ.

\begin{center}
\textbf{ಭಕ್ತಿ *\supskpt{\footnote{1. ಸ್ವಲ್ಪ ಭಿನ್ನವಾದ ವರದಿಗೆ ನೋಡಿ “ಭಕ್ತಿ” ‘ದಿ ಟ್ರಿಬ್ರೂನ್’ ನಿಂದ ವರದಿಯಾದುದ್ದು ಕೃತಿಶ್ರೇಣಿ ೩ ಪುಟ ೨೫೧.}}}
\end{center}

\begin{center}
(ದಿ ಇಂಡಿಯನ್ ಮಿರರ್, ೨೪ ಫೆಬ್ರವರಿ ೧೮೯೮)
\end{center}

ಲಾಹೋರ್ ಹಾಗೂ ಸಿಯಾಲ್ಕೋಟ್ನ ಜನರಿಗೆ ಸ್ವಾಮಿ ವಿವೇಕಾನಂದರು ವಾಸ್ತವವಾಗಿ ಕೆಲಸದಲ್ಲಿ ತೊಡಗುವ ಆವಶ್ಯಕತೆಯ ಬಗ್ಗೆ ಒತ್ತಿ ಹೇಳುತ್ತಿದ್ದಾರೆ\footnote{\enginline{Vivekananda in Indian Newspaper, p. 215.}}. ಹಸಿದಿರುವ ಕೋಟ್ಯಂತರ ಜನರು ಯಾವುದೋ ನಿಗೂಢ ತಾತ್ತ್ವಿಕ ಪರಿಕಲ್ಪನೆಯನ್ನಾಧರಿಸಿ ಬದುಕಿರಲಾರರು; ಅವರಿಗೆ ತಿನ್ನಲು ರೊಟ್ಟಿ ಬೇಕು. ಭಕ್ತಿಯ ಮೇಲೆ ಲಾಹೋರಿನಲ್ಲಿ ಕೊಟ್ಟ ಒಂದು ಉಪನ್ಯಾಸದಲ್ಲಿ ಅವರು, ಈ ಹೊತ್ತಿನ ಅತ್ಯುತ್ತಮ ಧರ್ಮವೆಂದರೆ ಪ್ರತಿಯೊಬ್ಬನೂ ತನಗೆ ಸಾಧ್ಯವಿರುವಷ್ಟು ಮಟ್ಟಿಗೆ ಬೀದಿಗಿಳಿದು ಹಸಿದ ನಾರಾಯಣರನ್ನು ಹುಡುಕಿ ಅವರನ್ನು ತನ್ನ ಮನೆಗೆ ಕರೆದೊಯ್ದು ಅನ್ನವಿಡಬೇಕು, ಕೈಲಾದರೆ ಬಟ್ಟೆಯನ್ನೂ ಕೊಡಬೇಕು ಎಂದು ಸಲಹೆ ಮಾಡಿದರು. ಮನುಷ್ಯನೇ ಭಗವಂತನ ಅತ್ಯು ತ್ತಮ ದೇವಾಲಯವೆನ್ನುವುದನ್ನು ನೆನಪಿನಲ್ಲಿಟ್ಟುಕೊಂಡು, ದಾನಿಯು ಅವನಿಗೆ ಕೊಡಬೇಕು. ತಾನು ಅನೇಕ ದೇಶಗಳಲ್ಲಿ ಔದಾರ್ಯವನ್ನು ನೋಡಿರುವೆ, ಆದರೆ ಅದು ಸೋತಿರುವುದಕ್ಕೆ ಅದರ ಹಿಂದಿರುವ ಭಾವವೇ ಕಾರಣ ಎಂದರವರು. “ನೋಡಿಲ್ಲಿ, ಇದನ್ನು ತೆಗೆದುಕೊ, ಹೊರಟುಹೋಗು.” ಲೋಕದ ಚಪ್ಪಾಳೆ ಗಿಟ್ಟಿಸುವುದಕ್ಕಾಗಿಯೋ ಕೀರ್ತಿಗಳಿಸುವುದಕ್ಕಾಗಿಯೋ ದಾನ ಕೊಟ್ಟರೆ ಅದು ದಾನವೆಂಬ ಹೆಸರನ್ನೇ ಸುಳ್ಳಾಗಿಸುವುದು.

\begin{center}
\textbf{ಅಮೆರಿಕಾದಲ್ಲಿ ನಮ್ಮ ಪ್ರಚಾರಕಾರ್ಯ\supskpt{\footnote{\enginline{Vivekananda in Indian Newspaper, p. 208.}}}}
\end{center}

\begin{center}
(ದಿ ಇಂಡಿಯನ್ ಮಿರರ್, ೨೪ ಏಪ್ರಿಲ್ ೧೮೯೮)
\end{center}

ಸ್ವಾಮಿ ವಿವೇಕಾನಂದರು, ಉಪನ್ಯಾಸಕರಾದ ಸ್ವಾಮಿ ಶಾರದಾನಂದರನ್ನು ಪರಿಚಯ ಮಾಡಿಕೊಡುತ್ತ ಹೀಗೆಂದು ಹೇಳಿದರು:-

ಮಹಿಳೆಯರೆ ಮತ್ತು ಮಹನೀಯರೆ, - ಈ ದಿನದ ಭಾಷಣಕಾರರು ಈಗತಾನೆ ಅಮೆರಿಕಾದಿಂದ ಬಂದಿರುವರು. ನಮ್ಮ ಕೆಲವು ದೇಶೀಯರು, ಅದರಲ್ಲೂ ಸ್ವಾಮಿ ದಯಾನಂದ ಸರಸ್ವತಿಯವರು, ಅಮೆರಿಕಾವನ್ನು ಲ್ಯಾಪ್ಲ್ಯಾಂಡಿನವರು ರಾಕ್ಷಸರು ಅಸುರರು ಮುಂತಾದವರು ವಾಸಿಸುತ್ತಿರುವ ಪಾತಾಳ ಎಂದು ಕರೆದಿದ್ದರೂ, ನಿಮಗೆಲ್ಲ ಗೊತ್ತಿರುವಂತೆ ಅಮೆರಿಕಾವು ನಿಮ್ಮ ದೇಶದ ಪರವಾಗಿದೆ (ನಗು ಮತ್ತು ಜಯಘೋಷ ಗಳು). ಒಳ್ಳೆಯದು ಮಹನೀಯರೆ, ಅದು ಪಾತಾಳ ಹೌದೋ ಅಲ್ಲವೋ ಎನ್ನುವುದನ್ನು ನೀವು ಇಲ್ಲಿರುವ ಈ ಕೆಲವು ಮಹಿಳೆಯರನ್ನು ನೋಡಿ ತೀರ್ಮಾನಿಸಬೇಕು - ಪಾತಾಳ ವೆಂದು ಕರೆಸಿಕೊಳ್ಳುವ ಆ ದೇಶದಿಂದಲೇ ಬಂದಿರುವ ಇವರು ನಾಗಕನ್ಯೆಯರು ಹೌದೋ ಅಲ್ಲವೋ ಎಂದು (ಜಯಘೋಷ). ಈಗ, ಅಮೆರಿಕಾವು ಪರಿಪೂರ್ಣವಾದ ಒಂದು ಹೊಸ ರಾಷ್ಟ್ರ. ಕೊಲಂಬಸ್ ಎಂಬ ಇಟೆಲಿ ದೇಶದವನು ಅದನ್ನು ಕಂಡುಹಿಡಿದನು. ಅದಕ್ಕಿಂತ ಮುಂಚೆ ನಾರ್ವೇಜಿಯನ್ನರು ಅದರ ಉತ್ತರಭಾಗವನ್ನು ಕಂಡುಹಿಡಿದಿ ದ್ದರು ಎಂದು ಹೇಳುವರು; ಅದಕ್ಕಿಂತಲೂ ಮುಂಚೆ ಒಂದು ಕಾಲದಲ್ಲಿ ಚೀನೀಯರು ಬೌದ್ಧಧರ್ಮದ ಉನ್ನತ ಪ್ರಣಾಳಿಕೆಗಳನ್ನು ಲೋಕದ ಎಲ್ಲೆಡೆ ಬೋಧಿಸುತ್ತಿದ್ದಾಗ ಅಲ್ಲಿಗೂ ಹೋಗಿದ್ದರು ಎನ್ನುವರು; ಭಾರತದಿಂದಲೂ ಬೌದ್ಧ ಪ್ರಚಾರಕರನ್ನು ವಾಷಿಂಗ್ಟನ್ಗೆ ಕಳುಹಿಸಲಾಗಿತ್ತು ಎಂದೂ ಹೇಳುವರು. ಇಂಥವುಗಳನ್ನು ಸೂಚಿಸುವ ಕೆಲವು ಬಗೆಯ ದಾಖಲೆಗಳನ್ನು ಅಲ್ಲಿಗೆ ಹೋಗುವ ಯಾತ್ರಿಕರು ಇಂದಿಗೂ ಹುಡುಕಿ ನೋಡಬಹುದು. ಈಗ, ಕಳೆದ ಒಂದು ಶತಮಾನಕ್ಕಿಂತಲೂ ಹೆಚ್ಚು ಕಾಲದಿಂದ, ಅಮೆರಿಕಾವನ್ನು ಇತರರು ಸಂಶೋಧಿಸುವುದಕ್ಕಿಂತ ಅಮೆರಿಕಾವೇ ಅಲ್ಲಿಗೆ ಹೋಗುವ ವರನ್ನು ಸಂಶೋಧಿಸುತ್ತಿದೆ (ಗಟ್ಟಿಯಾಗಿ ಕರತಾಡನ). ಪ್ರಪಂಚದ ಎಲ್ಲ ಭಾಗಗ ಳಿಂದಲೂ ಬರುತ್ತಿರುವ ಅದೆಷ್ಟೋ ಜನರು ಸಂಯುಕ್ತ ಸಂಸ್ಥಾನಗಳಲ್ಲಿ ಸಂಶೋಧಿಸಲ್ಪಡುತ್ತಿರುವ ವಿದ್ಯಮಾನವನ್ನು ಅಲ್ಲಿ ಅನುದಿನವೂ ನೋಡುತ್ತಿರುತ್ತೇವೆ. ನಿಮಗೆಲ್ಲ ಗೊತ್ತಿರುವ ಹಾಗೆ, ನಮ್ಮ ದೇಶೀಯರೂ ಅನೇಕರು ಹಾಗೆ ಸಂಶೋಧಿಸಲ್ಪಟ್ಟವರಾಗಿ ದ್ದಾರೆ (ಜಯಘೋಷ). ಇಂದು ನಾನಿಲ್ಲಿ ನಿಮ್ಮ ಮುಂದೆ ಹಾಗೆ ಅಮೆರಿಕಾದವ ರಿಂದ ಸಂಶೋಧಿಸಲ್ಪಟ್ಟ, ನಿಮ್ಮವನೇ ಆದ ಕಲ್ಕತ್ತದ ಒಬ್ಬ ಹುಡುಗನನ್ನು ಪರಿಚಯಿಸುತ್ತಿದ್ದೇನೆ (ಜಯಘೋಷ).

\begin{center}
\textbf{(ಶಿಕ್ಷಣವನ್ನು ಕುರಿತು) ಬೇಲೂರಿನಲ್ಲಿ ಸ್ವಾಮಿ ವಿವೇಕಾನಂದರು\supskpt{\footnote{\enginline{Vivekananda in Indian Newspaper, p. 215.}}}}
\end{center}

\begin{center}
(ದಿ ಇಂಡಿಯನ್ ಮಿರರ್, ೧೫ ಫೆಬ್ರವರಿ ೧೯೦೧)
\end{center}

ವರದಿಗಾರರೊಬ್ಬರು ಬರೆಯುತ್ತಾರೆ: - “ಬೇಲೂರಿನ ಎಂ.ಇ. ಶಾಲೆಯಲ್ಲಿ ಬಹುಮಾನ ವಿತರಣೆಯ ದಿನ (೨೨ರಂದು ಭಾನುವಾರ) ಸ್ವಾಮಿ ವಿವೇಕಾನಂದರು ಸಮಾರಂಭದ ಅಧ್ಯಕ್ಷರಾಗಿ ಆಹ್ವಾನಿಸಲ್ಪಟ್ಟಾಗ ಆ ಶಾಲೆಯ ವಿದ್ಯಾರ್ಥಿಗಳನ್ನೂ ನೆರೆದಿದ್ದ ಬೇಲೂರಿನ ಹಿರಿಯ ನಾಗರಿಕರನ್ನೂ ಉದ್ದೇಶಿಸಿ ಮಾಡಿದ ಭಾಷಣದ ಸಾರಾಂಶ ಇದು.”

“ಆಧುನಿಕ ವಿದ್ಯಾರ್ಥಿ ವಾಸ್ತವಿಕ ದೃಷ್ಟಿಯನ್ನು ಹೊಂದಿಲ್ಲ.ಅವನು ನಿಸ್ಸಹಾಯಕ. ನಮ್ಮ ವಿದ್ಯಾರ್ಥಿಗಳಿಗೆ ಅಗತ್ಯವಾಗಿರುವುದು ಅಷ್ಟೊಂದು ದೇಹದ ಸೌಷ್ಠವವೂ ಅಲ್ಲ, ಕಾಠಿಣ್ಯವೂ ಅಲ್ಲ. ಅವರಿಗೆ ಬೇಕಾದುದು ಸ್ವಾವಲಂಬನೆಯ ಮನಸ್ಸು. ಅವರಿಗೆ ತಮ್ಮ ಕಣ್ಣು ಕೈಗಳನ್ನು ಬಳಸಿ ರೂಢಿಯೇ ಇಲ್ಲ. ಕೈಕೆಲಸ ಯಾವುದನ್ನೂ ಅವರಿಗೆ ಹೇಳಿ ಕೊಟ್ಟಿಲ್ಲ. ಈಗಿನ ಇಂಗ್ಲಿಷ್ ವಿದ್ಯಾಭ್ಯಾಸ ಸಂಪೂರ್ಣವಾಗಿ ಸಾಹಿತ್ಯಿಕವಾದದ್ದು. ವಿದ್ಯಾರ್ಥಿಯನ್ನು ಸ್ವತಂತ್ರವಾಗಿ ಚಿಂತಿಸುವ ಹಾಗೆ, ಸ್ವತಂತ್ರವಾಗಿ ಕೆಲಸ ಮಾಡಿಕೊಳ್ಳುವ ಹಾಗೆ ತಯಾರು ಮಾಡಬೇಕು. ಒಂದು ಕಡೆ ಬೆಂಕಿ ಹತ್ತಿಕೊಂಡಿದೆ ಎಂದಿಟ್ಟು ಕೊಳ್ಳೋಣ. ಯಾರು ಮುನ್ನುಗ್ಗಿ ಅದನ್ನು ನಂದಿಸಲು ಬರುತ್ತಾರೋ ಅವರಿಗೆ ಕಣ್ಣು ಕೈಕಾಲುಗಳನ್ನು ಬಳಸುವ ರೂಢಿಯಿದೆ ಎಂದರ್ಥ. ಬಂಗಾಳಿಗಳ ಸೋಮಾರಿತನದ ಬಗ್ಗೆ ಯೂರೋಪಿಯನ್ನರ ಲೇವಡಿಯಲ್ಲಿ ತುಂಬ ಸತ್ಯವಿದೆ - ಕೆಲಸ ಮಾಡುವುದರಲ್ಲಿ ಅವರಿಗಿರುವ ತಾತ್ಸಾರ ಹಾಗಿದೆ. ವಿದ್ಯಾರ್ಥಿಗಳಿಗೆ ಏನಾದರೂ ಕೈಕೆಲಸ ಹೇಳಿ ಕೊಡುವ ಮೂಲಕ ಇದನ್ನು ಸರಿಪಡಿಸಬಹುದು; ಉಪಯೋಗದ ದೃಷ್ಟಿಗಿಂತ ಹೆಚ್ಚಾಗಿ, ಅದೊಂದು ಮಹತ್ವದ ಶಿಕ್ಷಣವಾಗುತ್ತದೆ.

ಎರಡನೆಯದಾಗಿ, ಅತ್ಯಂತ ಕೆಟ್ಟ ಆಹಾರ ತಿಂದುಕೊಂಡು ಬದುಕುವ, ಭಯಾನಕ ಪರಿಸರದಲ್ಲಿ ವಾಸಿಸುತ್ತಿರುವ ಎಷ್ಟೊಂದು ಸಹಸ್ರ ವಿದ್ಯಾರ್ಥಿಗಳನ್ನು ನಾನು ನೋಡಿ ರುವೆ! ಅವರಲ್ಲಿ ಅಷ್ಟೊಂದು ಜನ ಹೆಡ್ಡರು, ಮಂಕರು, ಹೇಡಿಗಳು ಇರುವು ದರಲ್ಲಿ ಅಚ್ಚರಿಯೇನಿದೆ? ನೊಣಗಳ ಹಾಗೆ ಅವರು ಸಾಯುವರು. ಅವರಿಗೆ ಕೊಡ ಲಾಗುತ್ತಿರುವ ವಿದ್ಯಾಭ್ಯಾಸ ಏಕಪಕ್ಷೀಯವಾದದ್ದು, ದುರ್ಬಲಗೊಳಿಸುವಂಥದ್ದು; ಅದು ಅವರನ್ನು ಅಂಗುಲ ಅಂಗುಲವಾಗಿ ಕೊಲ್ಲುತ್ತಿದೆ. ಮಕ್ಕಳಿಗೆ ನಿರುಪಯುಕ್ತ ಮಾಹಿತಿಯನ್ನು ತುಂಬಲಾಗುತ್ತಿದೆ; ಐವತ್ತರಿಂದ ಎಪ್ಪತ್ತು ಮಕ್ಕಳನ್ನು ಐದಾರು ಗಂಟೆ ಕಾಲ ಇಕ್ಕಟ್ಟಾದ ತರಗತಿಗಳಲ್ಲಿ ಕೂಡಿಹಾಕಿರುತ್ತಾರೆ. ಅವರಿಗೆ ಕೆಟ್ಟ ಆಹಾರ ಕೊಡಲಾಗುತ್ತಿದೆ. ಮನುಷ್ಯನ ಭವಿಷ್ಯದ ಆರೋಗ್ಯ ಶೈಶವದಲ್ಲೇ ಇರುವುದು ಎನ್ನುವುದನ್ನು ಮರೆಯುತ್ತಾರೆ. ಪ್ರಕೃತಿಗೆ ಮೋಸಮಾಡಲು ಆಗದೆಂಬುದನ್ನು ಮರೆತು ತುಂಬ ಬೇಗನೆ ಎಲ್ಲವನ್ನೂ ಮಾಡಿಸಲು ಯತ್ನಿಸುತ್ತಾರೆ. ಮಗುವಿಗೆ ವಿದ್ಯಾಭ್ಯಾಸ ಕೊಡುವಾಗ ಬೆಳವಣಿಗೆಯ ನಿಯಮವನ್ನು ಪಾಲಿಸಬೇಕು. ನಾವು ಸ್ವಲ್ಪ ಕಾಯುವುದನ್ನು ಕಲಿಯ ಬೇಕು. ಮಗುವಿಗೆ ಸಶಕ್ತವೂ ಆರೋಗ್ಯಪೂರ್ಣವೂ ಆದ ಶರೀರಕ್ಕಿಂತ ಮುಖ್ಯವಾದುದು ಬೇರೇನೂ ಇಲ್ಲ. ಯೋಗ್ಯತೆಯನ್ನು ಸಂಪಾದಿಸುವುದರಲ್ಲಿ ಮೊದಲು ಅಗತ್ಯವಾಗಿರುವುದೇ ಶರೀರ. ನಮ್ಮದು ಅತ್ಯಂತ ಬಡದೇಶ, ನಾವು ಹೆಚ್ಚೇನೂ ಮಾಡಲಾರೆವು ಎನ್ನುವುದು ನನಗೆ ಗೊತ್ತಿದೆ. ನಾವು ಕೇವಲ ಅಲ್ಪ ಪ್ರತಿರೋಧದ ಮಾರ್ಗವಾಗಿ ಕೆಲಸ ಮಾಡಬಲ್ಲೆವು. ಕನಿಷ್ಠಪಕ್ಷ ನಾವು ನಮ್ಮ ಮಕ್ಕಳಿಗಾದರೂ ಒಳ್ಳೆಯ ಆಹಾರ ಕೊಡಬೇಕು. ಮಗುವಿನ ಶರೀರಯಂತ್ರವನ್ನು ತೀರ ಬಳಲಿಸಬಾರದು. ಯೂರೋಪ್ ಅಮೆರಿಕಾಗಳಲ್ಲಿ ಕೋಟ್ಯಾಧಿಪತಿಯಾದವನೂ ಸಹ ತಮ್ಮ ಮಗ ದುರ್ಬಲನಾಗಿದ್ದರೆ ಅವನನ್ನು ರೈತರೊಡನೆ ಭೂಮಿಯನ್ನು ಉಳಲು ಕಳುಹಿಸಿಕೊಡುವನು. ಮೂರು ವರ್ಷಗಳ ನಂತರ ಅವನು ಆರೋಗ್ಯವಾಗಿ, ಬಲಿಷ್ಠನಾಗಿ, ಗುಲಾಬಿಯಂತೆ ಕೆಂಪಗಾಗಿ ತಂದೆಯ ಬಳಿಗೆ ಹಿಂದಿರುಗುವನು. ಆಗ ಅವನು ಶಾಲೆಗೆ ಹೋಗಲು ಯೋಗ್ಯನಾಗುವನು. ಈ ಕಾರಣಗಳಿಗಾಗಿ ನಾವು ಪ್ರಸಕ್ತ ಶಿಕ್ಷಣಕ್ರಮವನ್ನು ಇನ್ನೂ ಒತ್ತಿ ಹೇರಬಾರದು.

ಮೂರನೆಯದಾಗಿ, ನಮ್ಮಲ್ಲಿ ಶೀಲವೆಂಬುದೇ ಅದೃಶ್ಯವಾಗಿದೆ. ನಮ್ಮ ಇಂಗ್ಲಿಷ್ ವಿದ್ಯಾಭ್ಯಾಸವು ಇದ್ದುದೆಲ್ಲವನ್ನೂ ನಾಶಪಡಿಸಿ, ಅದರ ಜಾಗದಲ್ಲಿ ಏನನ್ನೂ ಬಿಟ್ಟಿಲ್ಲ. ನಮ್ಮ ಮಕ್ಕಳು ಶಿಷ್ಟ ನಡವಳಿಕೆಯನ್ನೇ ಕಳೆದುಕೊಂಡಿದ್ದಾರೆ. ಸಭ್ಯವಾಗಿ ಮಾತನಾಡುವು ದೆಂದರೆ ಏನೋ ಹೀನಾಯ. ಹಿರಿಯರಿಗೆ ಗೌರವ ತೋರಿಸುವುದೆಂದರೆ ಏನೋ ಹೀನಾಯ. ಅಗೌರವದಿಂದಿರುವುದೇ ಸ್ವಾತಂತ್ರ್ಯದ ಚಿಹ್ನೆ. ನಮ್ಮ ಮೊದಲಿನ ಶಿಷ್ಟತೆಗೆ ನಾವು ಹಿಂದಿರುಗಲು ಈಗಾಗಲೇ ತಡವಾಗಿದೆ. ಸುಧಾರಕರಿಗೆ ನಮ್ಮಿಂದ ಕಸಿದು ಕೊಂಡಿರುವುದರ ಸ್ಥಳದಲ್ಲಿ ಕೊಡುವುದಕ್ಕೆ ಏನೂ ಇಲ್ಲವಾಗಿದೆ. ಆದರೂ, ವ್ಯತಿರಿಕ್ತ ಪರಿಸರದ ನಡುವೆಯೂ ಸಾಕಷ್ಟು ದುಡಿಯಲು ನಮಗೆ ಸಾಧ್ಯವಾಗಿದೆ; ಆದರೆ ನಾವಿನ್ನೂ ಹೆಚ್ಚು ದುಡಿಯಬೇಕಾಗಿದೆ. ನನ್ನ ಜನಾಂಗದ ಬಗ್ಗೆ ನನಗೆ ಹೆಮ್ಮೆಯಿದೆ; ನಾನು ನಿರಾಶನಾಗುವುದಿಲ್ಲ; ಅನುದಿನವೂ ನಾನು ಅಚ್ಚರಿಯ, ವೈಭವದ, ಭವಿಷ್ಯದ ಚಿತ್ರವನ್ನೇ ನನ್ನ ಬಗೆಗಣ್ಣಿನೆದುರು ಕಾಣುತ್ತೇನೆ. ನಮ್ಮ ಭವಿಷ್ಯ ಈ ಎಳೆಯರ ಮೇಲೆ ನಿಂತಿದೆ; ಇವರನ್ನು ಅತ್ಯಂತ ಎಚ್ಚರದಿಂದ ನೋಡಿಕೊಳ್ಳಿರಿ.”

\begin{center}
\textbf{ಹಿಂದೂ ವಿಧವೆಯರು *\supskpt{\footnote{\enginline{Vivekananda in Indian Newspaper, p. 458.}}}}
\end{center}

\begin{center}
(ದಿ ಇಂಡಿಯನ್ ಸೋಷಿಯಲ್ ರಿಫಾರ್ಮರ್, ೧೬ ಜೂನ್ ೧೯೦೧)
\end{center}

ಸಾಮಾಜಿಕ ಪ್ರಶ್ನೆಗಳ ಕಡೆಗೆ ಸ್ವಾಮಿ ವಿವೇಕಾನಂದರ ಧೋರಣೆಯನ್ನು ಕುರಿತು ಅಮೆರಿಕಾದಲ್ಲಿ ಪ್ರಶ್ನೆ ಉದ್ಭವವಾಗಿದ್ದರಿಂದ, ಮಹಿಳೆಯೊಬ್ಬರು ಅಮೆರಿಕಾದ ಪತ್ರಿಕೆ ಯೊಂದಕ್ಕೆ ಹೀಗೆಂದು ಬರೆಯುತ್ತಾರೆ: “ಪೌಚ್ ಅರಮನೆಯಲ್ಲಿ ಕೊಟ್ಟ ಒಂದು ಉಪ ನ್ಯಾಸ\footnote{2. ಬಹುಶಃ ೧೮೯೫ರ ಫೆಬ್ರವರಿ ೨೫ರಂದು ಕೊಟ್ಟ ‘ಲೋಕಕ್ಕೆ ಭಾರತದ ಕೊಡುಗೆ’ ಎಂಬ ಉಪನ್ಯಾಸವಿರಬೇಕು, ಇದರ ಪದಶಃ ವರದಿ ಲಭ್ಯವಿಲ್ಲ. ಸ್ವಲ್ಪ ಭಿನ್ನವಾದ ಎರಡು ಅಮೆರಿಕನ್ ಪತ್ರಿಕಾವರದಿಗಳಿಗೆ ನೋಡಿ ಕೃತಿಶ್ರೇಣಿ \enginline{2} ಪುಟ \enginline{510-514}}ದಲ್ಲಿ ಅವರು ಹಿಂದೂ ವಿಧವೆಯರ ಬಗ್ಗೆ ಮಾತನಾಡಿದರು; ಭಾರತೀಯ ಗೃಹಗಳಲ್ಲಿ ಅವರು ತುಳಿತಕ್ಕೆ, ಕ್ರೌರ್ಯಕ್ಕೆ ಒಳಗಾಗಿರುವರೆಂದು ಹೇಳುವುದು ನ್ಯಾಯವಲ್ಲ ಎಂದು ಸಾರಿದರು. ವಿಧವೆಯರ ಪುನರ್ವಿವಾಹದ ಬಗ್ಗೆ ಪೂರ್ವಗ್ರಹವಿರುವುದನ್ನು ಅವರು ಒಪ್ಪಿಕೊಂಡರು; ಪರಂಪರಾಗತ ಪದ್ಧತಿಯು ವಿಧವೆಯನ್ನು ತವರಿಗೆ ಹಿಂದಿರುಗದೆ ಪತಿಗೃಹದ ಸದಸ್ಯಳಾಗಿರುವಂತೆ ಮಾಡುವುದು ಭಾರತದಲ್ಲಿ ವಿಧವೆಯರಿಗೆ ಕೆಲವು ಕಷ್ಟಗಳನ್ನು ತಂದೊಡ್ಡುವುದು ಎನ್ನುವುದನ್ನೂ ಒಪ್ಪಿಕೊಂಡರು. ಅವರು ಇಂತಹ ಅವಸ್ಥೆಯಿಂದ ಪಾರಾಗುವುದಕ್ಕೆ ತಮ್ಮ ಕಾಲ ಮೇಲೆ ತಾವು ನಿಲ್ಲುವಂತೆ ಅವರಿಗೆ ವಿದ್ಯಾಭ್ಯಾಸವನ್ನು ಕೊಡಿಸುವ ಪ್ರಯತ್ನಗಳನ್ನು ಬೆಂಬಲಿಸಿದರು. ವಿಧವೆಯನ್ನೂ ಒಳಗೊಂಡಂತೆ ತಮ್ಮ ದೇಶದ ಸ್ತ್ರೀಯರ ವಿದ್ಯಾಭ್ಯಾಸ ಮತ್ತು ಉದ್ಧಾರವನ್ನು ತಾವು ಆಶಿಸುವುದಾಗಿ ಒತ್ತಿ ಹೇಳಿದರು. ತಮ್ಮ ಈ ಒಂದು ಉಪನ್ಯಾಸದಲ್ಲಿ ಬಂದ ಎಲ್ಲ ಹಣವನ್ನೂ ಕೊಲ್ಕತ್ತದ ಬಳಿಯ ಬಾರಾನಗರದ ಬಾಬು ಶಶಿಪಾದ ಬ್ಯಾನರ್ಜಿಯವರ ಶಾಲೆಗೆ ಸಹಾಯಾರ್ಥವಾಗಿ ಕೊಡುವುದಾಗಿಯೂ ಹೇಳಿದರು; ಹಾಗೆಯೇ ಕಳುಹಿಸಿಯೂ ಬಿಟ್ಟರು. ಈ ಸಂಸ್ಥೆಯ ಸ್ಫೂರ್ತಿಯಿಂದಲೇ ಪಂಡಿತ ರಮಾಬಾಯಿ ಪುಣೆಯಲ್ಲಿ ಶಾಲೆ ತೆರೆದದ್ದು.”

