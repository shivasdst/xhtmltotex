
\chapter{ಭಾರತ ಮಹಿಳೆಯರು}

(೧೮೯೪, ಡಿಸೆಂಬರ್ ೧೭ರಂದು ಕೇಂಬ್ರಿಜ್ನಲ್ಲಿ ನೀಡಿದ ಉಪನ್ಯಾಸ. ಮಿಸ್ ಫ್ರಾಂಸಿಸ್ ವಿಲ್ಲಾರ್ಡ್ ಅವರು ಬರೆದುಕೊಂಡದ್ದು.)

ಭಾರತದ ಮಹಿಳೆಯರನ್ನು ಕುರಿತು ಮಾತನಾಡುವಾಗ ಇನ್ನೊಂದು ಜನಾಂಗದ ತಾಯಂದಿರು ಮತ್ತು ಸಹೋದರಿಯರಿಗೆ ಭಾರತದ ನನ್ನ ತಾಯಂದಿರು ಮತ್ತು ಸಹೋದರಿಯರ ಬಗ್ಗೆ ಹೇಳುತ್ತಿರುವಂತೆ ನನಗನಿಸುತ್ತದೆ. ಆದರೆ ಇತ್ತೀಚಿನ ದಿನಗಳಲ್ಲಿ ನಮ್ಮ ದೇಶದ ಮಹಿಳೆಯರನ್ನು ಎಲ್ಲರೂ ಶಪಿಸುವವರೇ ಆಗಿದ್ದರೂ ಅವರ ಬಗ್ಗೆ ಒಳ್ಳೆಯ ಮಾತನಾಡುವವರೂ ಕೆಲವರು ಇದ್ದಾರೆ. ಮಿಸೆಸ್ ಓಲ್ ಬುಲ್, ಮಿಸ್ ಸಾರಾ ಫಾರ್ಮರ್, ಮಿಸ್ ಫ್ರಾಂಸಿಸ್ ವಿಲ್ಲಾರ್ಡ್ ಇವರಂತಹ ಉದಾತ್ತ ವ್ಯಕ್ತಿಗಳನ್ನು ನಾನು ಈ ದೇಶದಲ್ಲಿ ಸಂಧಿಸಿರುವೆನು. ಸಜ್ಜನಿಕೆಯ ಅತ್ಯುತ್ತಮ ಪ್ರತಿನಿಧಿಯಾದ ಲೇಡಿ ಹೆನ್ರಿ ಸೋಮರ್ಸೆಟ್ ಅವರನ್ನು ಸಂಧಿಸಿರುವೆನು. ಅವರ ಜೀವನವು ಕ್ರಿಸ್ತನ ಜನನಕ್ಕೆ ಆರನೂರು ವರ್ಷಗಳ ಹಿಂದೆ ಜನಸಾಮಾನ್ಯರೊಡನೆ ಬಾಳುವುದಕ್ಕಾಗಿ ಸಿಂಹಾಸನವನ್ನು ತ್ಯಾಗ ಮಾಡಿದ ಆ ಭಾರತೀಯ ಮಹಾಪುರುಷನನ್ನು ನೆನಪಿಸಿಕೊಡುತ್ತದೆ. ನನಗೂ ನನ್ನ ದೇಶದ ಸ್ತ್ರೀ ಪುರುಷರಿಗೂ ಯಾವಾಗಲೂ ಒಳಿತನ್ನೇ ಬಯಸುವ, ಶಾಪದ ನುಡಿಯನ್ನು ಉಚ್ಚರಿಸದ, ಮತ್ತು ಮಾನವಕೋಟಿಗೆ ಸೇವೆಯನ್ನು ಸಲ್ಲಿಸಲು ತನು ಮನ ಗಳನ್ನು ಅರ್ಪಿಸಿರುವ ಇಂತಹ ಉದಾತ್ತ ವ್ಯಕ್ತಿಗಳನ್ನು ನೋಡಿದಾಗ ನನಗೆ ಧೈರ್ಯ ಬರುತ್ತದೆ.

ಭಾರತದ ಇತಿಹಾಸದ ಪುರಾತನ ಕಾಲವನ್ನು ಒಮ್ಮೆ ವೀಕ್ಷಿಸೋಣ. ಅಲ್ಲಿ ಒಂದು ಅಪೂರ್ವ ವೈಶಿಷ್ಟ್ಯವನ್ನು ಕಾಣುತ್ತೇವೆ. ಅಮೆರಿಕನರಾದ ನೀವು, ಹಿಂದೂಗಳಾದ ನಾವು ಮತ್ತು ಐಸ್ಲ್ಯಾಂಡಿನ ಈ ಮಹಿಳೆ (ಮಿಸೆಸ್. ಸಿಗ್ರಿಡ್ ಮ್ಯಾಗ್ನುಸನ್) ಎಲ್ಲರೂ ಆರ್ಯವಂಶಜರೆಂಬುದು ನಿಮಗೆಲ್ಲರಿಗೂ ತಿಳಿದಿರಬಹುದು. ಆರ್ಯರು ಹೋದೆಡೆ ಯಲ್ಲೆಲ್ಲಾ ಮೂರು ಮುಖ್ಯಾಂಶಗಳು ನಮ್ಮ ಗಮನಕ್ಕೆ ಬರುತ್ತವೆ: ಗ್ರಾಮ ಜೀವನ, ಸ್ತ್ರೀಯರ ಹಕ್ಕುಗಳು ಮತ್ತು ಆನಂದಭರಿತ ಧರ್ಮ.\footnote{1. ಸ್ವಾಮಿ ವಿವೇಕಾನಂದರ ಅನಂತರ ಆರ್ಯ ಸಂಸ್ಕೃತಿಯ ವ್ಯಾಪ್ತಿಯ ಬಗ್ಗೆ ಅನೇಕ ಸಂಶೋಧನೆಗಳು ನಡೆದಿವೆ.}

ಮೊದಲನೆಯ ಅಂಶ ಗ್ರಾಮ ಜೀವನ. ಪ್ರತಿಯೊಬ್ಬನೂ ತನಗೆ ತಾನೇ ಒಡೆಯ ನಾಗಿದ್ದ, ತನ್ನದೇ ಭೂಮಿಯನ್ನು ಹೊಂದಿದ್ದ. ಈಗ ನಾವು ಪ್ರಪಂಚದಲ್ಲಿ ಕಾಣುತ್ತಿರುವ ರಾಜಕೀಯ ಸಂಸ್ಥೆಗಳೂ ಈ ಗ್ರಾಮ ಪದ್ಧತಿಯ ಬೆಳವಣಿಗೆಗಳು. ಆರ್ಯರು ಬೇರೆ ಬೇರೆ ದೇಶಗಳಿಗೆ ಹೋಗಿ ನೆಲೆಯೂರಿದ ಮೇಲೆ ಅಲ್ಲಿಯ ಪರಿಸ್ಥಿತಿಗನುಗುಣವಾಗಿ ವಿವಿಧ ರಾಜಕೀಯ ಪದ್ಧತಿಗಳು ಜಾರಿಗೆ ಬಂದವು.

ಆರ್ಯರ ಇನ್ನೊಂದು ಮುಖ್ಯ ಭಾವನೆ ಸ್ತ್ರೀ ಸ್ವಾತಂತ್ರ್ಯ. ಪುರಾತನ ಕಾಲದಲ್ಲಿ ಸ್ತ್ರೀಯರು ಪುರುಷರಷ್ಟೇ ಹಕ್ಕನ್ನು ಪಡೆದಿದ್ದರೆಂಬುದನ್ನು ಆರ್ಯರ ಸಾಹಿತ್ಯದಲ್ಲಿ ಓದುತ್ತೇವೆ. ಪ್ರಪಂಚದ ಇನ್ನಾವ ಸಾಹಿತ್ಯದಲ್ಲಿಯೂ ನಮಗಿದು ಕಂಡುಬರುವುದಿಲ್ಲ.

ಪ್ರಪಂಚದ ಅತ್ಯಂತ ಪುರಾತನ ಸಾಹಿತ್ಯ ವೇದಗಳು. ಇವುಗಳನ್ನು ರಚಿಸಿದವರು ನಮ್ಮ ಮತ್ತು ನಿಮ್ಮ ಪೂರ್ವಿಕರು (ಅವು ಭಾರತದಲ್ಲಿ ರಚಿತವಾದುದಲ್ಲ - ಬಹುಶಃ ಬಾಲ್ಟಿಕ್ ತೀರದಲ್ಲೊ ಅಥವಾ ಮಧ್ಯ ಏಷ್ಯಾದಲ್ಲೋ ಆಗಿರಬೇಕು - ನಮಗೆ ಸರಿ ಯಾಗಿ ತಿಳಿಯದು). ಅವುಗಳ ಅತ್ಯಂತ ಪುರಾತನ ಭಾಗವೇ ಸಂಹಿತಾ ಭಾಗ. ಇದು ಆರ್ಯರು ಪೂಜಿಸುತ್ತಿದ್ದ ದೇವತೆಗಳನ್ನು ಕುರಿತ ಸ್ತೋತ್ರಗಳಿಂದ ಕೂಡಿದೆ. ‘ದೇವತೆ’ ಎಂಬುದರ ಶಬ್ದಾರ್ಥ ಬೆಳಗುವುದು ಎಂದು. ಈ ಸ್ತೋತ್ರಗಳು ಅಗ್ನಿ, ಸೂರ್ಯ, ವರುಣ ಮುಂತಾದ ದೇವತೆಗಳನ್ನು ಕುರಿತಾದದ್ದು. “ಇಂಥ ಋಷಿಯು ಈ ಸ್ತೋತ್ರವನ್ನು ರಚಿಸಿ ಇಂಥ ದೇವತೆಗೆ ಅರ್ಪಿಸಿದನು” ಎಂದು ತಲೆಬರೆಹವನ್ನು ಕಾಣುತ್ತೇವೆ.

ಹತ್ತನೆಯ ಅಧ್ಯಾಯದಲ್ಲಿ ಒಂದು ವಿಶಿಷ್ಟ ಸ್ತೋತ್ರವಿದೆ.\footnote{1. “ದೇವೀ ಸೂಕ್ತ”, ಋಗ್ವೇದ, ೧೦.೧೨೫} ಈ ಸ್ತೋತ್ರದ ಋಷಿ ಓರ್ವ ಮಹಿಳೆ. ಇದು ಎಲ್ಲ ದೇವತೆಗಳ ಹಿನ್ನೆಲೆಯಾದ ಒಬ್ಬ ದೇವರನ್ನು ಕುರಿತಾದದ್ದು. ಬೇರೆ ಎಲ್ಲ ಸ್ತೋತ್ರಗಳು ಬೇರೊಬ್ಬ ವ್ಯಕ್ತಿಯು ದೇವತೆಗಳನ್ನು ಪ್ರಾರ್ಥಿಸುತ್ತಿರುವಂತೆ ಪ್ರಥಮ ಪುರುಷದಲ್ಲಿವೆ. ಆದರೆ ಈ ಸ್ತೋತ್ರದಲ್ಲಿ ದೇವರೇ (ದೇವಿ) ತನಗೆ - ತಾನೇ ಹೇಳಿಕೊಳ್ಳುತ್ತಿರುವಳು. ‘ನಾನು’ ಎಂಬ ಸರ್ವನಾಮವನ್ನು ಬಳಸಲಾಗಿದೆ. “ನಾನೇ ವಿಶ್ವದ ರಾಜ್ಞಿ, ಎಲ್ಲ ಪ್ರಾರ್ಥನೆಗಳನ್ನು ಪೂರ್ತಿಗೊಳಿಸುವವಳು”.

ವೇದಗಳಲ್ಲಿ ಸ್ತ್ರೀಯರ ಕೃತಿಯ ಬಗ್ಗೆ ನಮಗೆ ದೊರೆಯುವ ಪ್ರಥಮ ನೋಟ. ನಾವು ಮುಂದುವರಿದಂತೆ ಪೌರೋಹಿತ್ಯವನ್ನು ಮಾಡುವಂತಹ ಇನ್ನೂ ಹೆಚ್ಚಿನ ಚಟುವಟಿಕೆಗಳಲ್ಲಿ ಸ್ತ್ರೀಯರು ಭಾಗವಹಿಸುತ್ತಿದ್ದುದನ್ನು ನೋಡುತ್ತೇವೆ. ಸ್ತ್ರೀಯರು ಪೌರೋಹಿತ್ಯಕ್ಕೆ ಯೋಗ್ಯರಲ್ಲ ಎಂಬುದನ್ನು ಪರೋಕ್ಷವಾಗಿಯಾದರೂ ಸೂಚಿಸುವ ಒಂದೇ ಒಂದು ವಾಕ್ಯವೂ ಇಡೀ ವೇದಸಾಹಿತ್ಯದಲ್ಲಿ ಇಲ್ಲ. ಅವರು ಪೌರೋಹಿತ್ಯವನ್ನು ಮಾಡುವ ಅನೇಕ ಉದಾಹರಣೆಗಳಿವೆ.

ಈಗ ವೇದಗಳ ಅಂತ್ಯಭಾಗವನ್ನು ನೋಡೋಣ. ಇದೇ ನಿಜವಾಗಿಯೂ ಭಾರತದ ಧರ್ಮವಾಗಿದೆ. ಇದರಲ್ಲಿ ಬರುವ ಜ್ಞಾನದ ಉತ್ಕರ್ಷವನ್ನು ಈಗಿನ ಶತಮಾನ ದಲ್ಲೂ ಮೀರಲಾಗಿಲ್ಲ. ಇದರಲ್ಲೂ ಮಹಿಳೆಯರ ಮುಖ್ಯಪಾತ್ರವನ್ನು ನೋಡುತ್ತೇವೆ. ಈ ಕೃತಿಗಳ ಬಹುಭಾಗವು ಸ್ತ್ರೀಮುಖದಿಂದ ಬಂದಿರುವುದಾಗಿದೆ. ಅವರ ಹೆಸರು ಮತ್ತು ಬೋಧನೆಗಳೊಡನೆ ಅಲ್ಲಿ ಅದು ಲಿಖಿತವಾಗಿದೆ.

ಶ್ರೇಷ್ಠ ಋಷಿ ಯಾಜ್ಞವಲ್ಕ್ಯನ ಸುಂದರ ಕಥೆ ಅದರಲ್ಲಿ ಬರುತ್ತದೆ.\footnote{1. ಬೃಹದಾರಣ್ಯಕ ಉಪನಿಷತ್, ೩.೮.೧-೧೨} ಅವನು ಪ್ರಸಿದ್ಧ ರಾಜನಾದ ಜನಕನ ಆಸ್ಥಾನಕ್ಕೆ ಹೋಗುತ್ತಾನೆ. ಅಲ್ಲಿ ನೆರೆದಿದ್ದ ಪಂಡಿತರು ಅವನಿಗೆ ಹಲವಾರು ಪ್ರಶ್ನೆಗಳನ್ನು ಕೇಳುತ್ತಾರೆ. ಒಬ್ಬನು “ಈ ಯಜ್ಞವನ್ನು ಮಾಡುವ ಬಗೆ ಹೇಗೆ?” ಮತ್ತೊಬ್ಬನು “ಆ ಯಜ್ಞವನ್ನು ಮಾಡುವ ಬಗೆ ಹೇಗೆ?” ಎಂದೆಲ್ಲ ಕೇಳು ತ್ತಾರೆ. ಇವರ ಪ್ರಶ್ನೆಗಳಿಗೆಲ್ಲ ಉತ್ತರಿಸಿಯಾದ ಮೇಲೆ, ಓರ್ವ ಮಹಿಳೆ ಮುಂದೆ ಬಂದು, “ಇವೆಲ್ಲ ಬಾಲಿಶ ಪ್ರಶ್ನೆಗಳು. ಈಗ ನೋಡುತ್ತಿರು, ಬಾಣಗಳಂತೆ ಹರಿತವಾದ ಎರಡು ಪ್ರಶ್ನೆಗಳನ್ನು ಕೇಳುತ್ತೇನೆ. ನಿಮಗೆ ಸಾಧ್ಯವಾದರೆ ಅವುಗಳಿಗೆ ಉತ್ತರಿಸಿ. ಆಗ ನಿಮ್ಮನ್ನು ಋಷಿಯೆಂದು ಪರಿಗಣಿಸುತ್ತೇವೆ. ಮೊದಲನೆಯ ಪ್ರಶ್ನೆ: ಜೀವಾತ್ಮವೆಂದರೆ ಯಾವುದು? ಎರಡನೆಯ ಪ್ರಶ್ನೆ: ದೇವರು ಎಂದರೆ ಏನು?”

ಹೀಗೆ ಭಾರತದಲ್ಲಿ ಜೀವಾತ್ಮನ ಬಗೆಗಿನ, ದೇವರ ಬಗೆಗಿನ ಪ್ರಶ್ನೆಗಳು ಉದಿಸಿದವು. ಈ ಪ್ರಶ್ನೆಗಳು ಓರ್ವ ಮಹಿಳೆಯ ಬಾಯಿಯಿಂದ ಬಂದಿರುವುದೊಂದು ವಿಶೇಷ. ಆ ಋಷಿಯು ಆಕೆಯ ಮುಂದೆ ಪರೀಕ್ಷೆಯಲ್ಲಿ ಉತ್ತೀರ್ಣನಾಗಬೇಕಿತ್ತು, ಅವನು ಚೆನ್ನಾಗಿಯೇ ಉತ್ತೀರ್ಣನಾದನು.

ಮುಂದಿನ ಸಾಹಿತ್ಯವಾದ ಪುರಾಣ, ಮಹಾಕಾವ್ಯಗಳ ಸಮಯಕ್ಕೆ ಬಂದರೆ ಶಿಕ್ಷಣವು ಅವನತಿಯನ್ನು ಹೊಂದಿಲ್ಲದೆ ಇರುವುದನ್ನು ನೋಡುತ್ತೇವೆ. ವಿಶೇಷವಾಗಿ ಕ್ಷತ್ರಿಯ ವರ್ಗದಲ್ಲಿ ಈ ಆದರ್ಶವು ಬಹುಮಾನ್ಯತೆಯನ್ನು ಪಡೆದಿತ್ತು.

ವೇದ ಕಾಲದಲ್ಲಿ ಬಾಲಕ ಬಾಲಕಿಯರಿಗೆ ತಮಗಿಷ್ಟರಾದ ವಧು ವರರನ್ನು ಆರಿಸಿ ಕೊಳ್ಳಲು ಸ್ವಾತಂತ್ರ್ಯವಿತ್ತು. ಮುಂದಿನ ಹಂತದಲ್ಲಿ, ಒಂದು ವರ್ಣದವರನ್ನು ಬಿಟ್ಟರೆ, ತಂದೆ ತಾಯಿಗಳೇ ವಧುವರರನ್ನು ನಿಶ್ಚಯಿಸುತ್ತಿದ್ದರು.

ಇಲ್ಲಿಯೂ ಕೂಡ ಇನ್ನೊಂದು ಅಂಶವನ್ನು ನೀವು ಗಮನಿಸಬೇಕೆಂದು ನಾನು ಕೇಳಿ ಕೊಳ್ಳುತ್ತೇನೆ. ಹಿಂದೂಗಳ ವಿಷಯದಲ್ಲಿ ಏನೇ ಹೇಳಿಸಬಹುದು, ಅವರು ಪ್ರಪಂಚ ದಲ್ಲೇ ಅತ್ಯಂತ ವಿದ್ಯಾವಂತ ಜನಾಂಗ. ಹಿಂದುವು ತತ್ತ್ವಜ್ಞಾನಿ, ಅವನು ಪ್ರತಿಯೊಂದು ವಿಷಯದಲ್ಲಿಯೂ ಬುದ್ಧಿಯನ್ನು ಉಪಯೋಗಿಸುತ್ತಾನೆ. ಪ್ರತಿಯೊಂದೂ ಜ್ಯೋತಿ ಶ್ಶಾಸ್ತ್ರದ ಗಣನೆಯ ಪ್ರಕಾರ ನಡೆಯಬೇಕು.

ಪ್ರತಿಯೊಬ್ಬ ಸ್ತ್ರೀ ಪುರುಷರ ಅದೃಷ್ಟವೂ ನಕ್ಷತ್ರಗಳಿಂದ ನಿರ್ಧಾರಿತವಾಗುತ್ತದೆ. ಇಂದಿಗೂ ಕೂಡ ಒಂದು ಮಗು ಹುಟ್ಟಿದ ಕೂಡಲೆ ಅದರ ಜಾತಕವನ್ನು ತಯಾರಿಸು ತ್ತಾರೆ. ಅದು ಮಗುವಿನ ಚಾರಿತ್ರ್ಯವನ್ನು ನಿರ್ಧರಿಸುತ್ತದೆ. ಒಂದು ಮಗುವು ದಿವ್ಯ ಸ್ವಭಾವದಿಂದ ಹುಟ್ಟಿರಬಹುದು, ಇನ್ನೊಂದು ಮಾನವ ಸ್ವಭಾವದಿಂದಲೊ ಅಥವಾ ಇನ್ನೂ ಕೀಳು ಸ್ವಭಾವದಿಂದಲೋ ಹುಟ್ಟಿರಬಹುದು. ಆಸುರೀ ಸ್ವಭಾವದ ಮಗು ದೈವೀ ಸ್ವಭಾವದ ಮಗುವಿನೊಡನೆ ಒಂದಾದರೆ ಇಬ್ಬರೂ ಅವನತಿಯನ್ನು ಹೊಂದಬಹು ದಲ್ಲವೆ? ಇನ್ನೊಂದು ತೊಂದರೆ ಏನೆಂದರೆ, ಒಂದೇ ಕುಲದವರಲ್ಲಿ ಮದುವೆಯನ್ನು ನಮ್ಮ ಶಾಸನವು ಒಪ್ಪುವುದಿಲ್ಲ. ಒಬ್ಬನು ತನ್ನ ಕುಟುಂಬಕ್ಕೆ ಸೇರಿದವರನ್ನಾಗಲೀ, ದೊಡ್ಡಪ್ಪ ಅಥವಾ ಚಿಕ್ಕಪ್ಪನ ಮಕ್ಕಳನ್ನಾಗಲಿ ಮದುವೆಯಾಗುವುದು ನಿಷಿದ್ಧ ಮಾತ್ರವಲ್ಲದೆ ತಂದೆ ಅಥವಾ ತಾಯಿಯ ಕುಲಕ್ಕೆ ಸೇರಿದವರನ್ನೂ ಮದುವೆಯಾಗುವಂತಿಲ್ಲ. ಮೂರನೆಯ ತೊಂದರೆ ಯಾವುದೆಂದರೆ, ವಧು ಅಥವಾ ವರನ ಹಿಂದಿನ ಆರು ತಲೆಮಾರಿನವರಲ್ಲಿ ಯಾರಿಗಾದರೂ ಕುಷ್ಟ ಅಥವಾ ಕ್ಷಯರೋಗವಿದ್ದರೆ ಅವರು ಮದುವೆಯಾಗುವಂತಿಲ್ಲ.

ಈ ಮೂರು ಅಂಶಗಳನ್ನು ಗಣನೆಗೆ ತೆಗೆದುಕೊಂಡು ಬ್ರಾಹ್ಮಣನು ಹೇಳುತ್ತಾನೆ: “ಮದುವೆಯ ವಿಷಯದಲ್ಲಿ ಹುಡುಗ ಅಥವಾ ಹುಡುಗಿಯರಿಗೆ ಸ್ವಾತಂತ್ರ್ಯವನ್ನು ಕೊಟ್ಟರೆ ಅವರು ಮುಖ ಸೌಂದರ್ಯಕ್ಕೆ ಮಾರುಹೋಗುತ್ತಾರೆ. ಇದರಿಂದ ಮೇಲಿನ ದೋಷಗಳ ಪರಿಣಾಮವಾಗಿ ಇಡೀ ಕುಟುಂಬವೇ ಅವನತಿಯನ್ನು ಹೊಂದಬಹುದು”. ನಮ್ಮ ದೇಶದಲ್ಲಿ ವಿವಾಹ ಶಾಸನಗಳ ಹಿಂದಿರುವ ಮುಖ್ಯ ಭಾವನೆಯೇ ಇದಾಗಿದೆ. ಸರಿಯೋ ತಪ್ಪೊ, ಈ ತತ್ತ್ವ ಹಿನ್ನೆಲೆಯಾಗಿದೆ. ವ್ಯಾಧಿಯ ಚಿಕಿತ್ಸೆಗಿಂತ ನಿವಾರಣೆಯೇ ಲೇಸು.

ಈ ಪ್ರಪಂಚದಲ್ಲಿ ದುಃಖವಿರುವುದಕ್ಕೆ ಕಾರಣ ನಾವೇ, ದುಃಖಕ್ಕೆ ಜನ್ಮವೀಯು ತ್ತೇವೆ. ದುಃಖಿತ ಮಕ್ಕಳು ಹುಟ್ಟದಂತೆ ಹೇಗೆ ತಡೆಯುವುದೆಂಬುದೇ ಮುಖ್ಯ ಪ್ರಶ್ನೆ. ಸಮಾಜದ ಹಕ್ಕುಗಳು ಎಷ್ಟರ ಮಟ್ಟಿಗೆ ವ್ಯಕ್ತಿಯ ವರ್ತನೆಗಳನ್ನು ನಿಯಂತ್ರಿಸಬಹುದು ಎಂಬುದು ಒಂದು ದೊಡ್ಡ ಪ್ರಶ್ನೆ. ಆದರೆ ಹಿಂದೂಗಳು ವಿವಾಹದ ವಿಷಯದಲ್ಲಿ ಹುಡುಗ ಅಥವಾ ಹುಡುಗಿಯರಿಗೆ ಸ್ವಾತಂತ್ರ್ಯ ವಿರಬಾರದೆಂದು ಹೇಳುತ್ತಾರೆ.

ಈ ಪದ್ಧತಿಯೇ ಅತ್ಯುತ್ತಮವೆಂದು ನಾನು ಹೇಳುವುದಿಲ್ಲ. ವಿವಾಹದ ವಿಷಯವನ್ನು ಅವರಿಗೇ ಬಿಟ್ಟುಬಿಡುವುದೂ ಸರಿಯಾದ ಪರಿಹಾರವೆಂದು ನಾನು ಹೇಳುವು ದಿಲ್ಲ. ನನಗೂ ಇದುವರೆಗೆ ಯಾವ ಪರಿಹಾರವೂ ಹೊಳೆದಿಲ್ಲ; ಯಾವ ದೇಶವೂ ಇದನ್ನು ಕಂಡುಹಿಡಿದಿಲ್ಲ.

ನಾವೀಗ ಇನ್ನೊಂದು ದೃಶ್ಯವನ್ನು ನೋಡುತ್ತೇವೆ. ರಾಜಮನೆತನದವರಲ್ಲಿ ಸ್ವಯಂವರ ಎಂಬ ಇನ್ನೊಂದು ವಿಚಿತ್ರ ರೀತಿಯ ವಿವಾಹ ಪದ್ಧತಿಯಿತ್ತು. ಹುಡುಗಿಯ ತಂದೆಯು ಬೇರೆ ಬೇರೆ ರಾಜಕುಮಾರರನ್ನೂ, ಕ್ಷತ್ರಿಯವೀರರನ್ನೂ ಆಹ್ವಾನಿಸಿ ಸಭೆ ಸೇರಿಸುತ್ತಿದ್ದನು. ರಾಜಕುಮಾರಿಯನ್ನು ಪಲ್ಲ ಕ್ಕಿಯ ಮೇಲೆ ಕೂರಿಸಿ ಪ್ರತಿಯೊಬ್ಬ ರಾಜಕುಮಾರನ ಮುಂದೆಯೂ ಕರೆದುಕೊಂಡು ಹೋಗುತ್ತಿದ್ದರು. ರಾಜಭಟ ನೊಬ್ಬನು ‘ಇವನು ಇಂಥ ರಾಜಕುಮಾರ, ಇವನ ಗುಣಗಳು ಇಂಥವು’ ಎಂದು ಸಾರು ತ್ತಿದ್ದನು. ರಾಜಕುಮಾರಿಯೂ ‘ಸ್ವಲ್ಪ ಇರಿ’ ಅಥವಾ ‘ಮುಂದೆಹೋಗಿ’ ಎಂದು ಹೇಳು ತ್ತಿದ್ದಳು. ಮುಂದಿನ ರಾಜಕುಮಾರನ ಹತ್ತಿರ ಹೋಗುವುದಕ್ಕೆ ಮುಂಚೆಯೇ ಭಟನು ಅವನ ವಿಷಯವನ್ನು ಹೇಳಿಬಿಡುತ್ತಿದ್ದನು. ರಾಜಕುಮಾರಿಯು ‘ಮುಂದೆ ಹೋಗಿ’ ಎನ್ನು ತ್ತಿದ್ದಳು. (ಎಲ್ಲವೂ ಪೂರ್ವನಿಶ್ಚಿತವಾಗಿರುತ್ತಿತ್ತು, ಆಕೆ ಮೊದಲೇ ಒಬ್ಬನನ್ನು ವರಿಸಿರುತ್ತಿದ್ದಳು.) ಅನಂತರ ಕೊನೆಯಲ್ಲಿ ಒಬ್ಬನ ಕುತ್ತಿಗೆಗೆ ಹಾರವನ್ನು ಹಾಕಲು ಅವಳು ತನ್ನ ಪರಿವಾರದಲ್ಲಿ ಒಬ್ಬನಿಗೆ ಹೇಳುತ್ತಿದ್ದಳು - ಹಾರ ಹಾಕಿಸಿಕೊಂಡವನು ತನಗೆ ಒಪ್ಪಿಗೆ ಎಂದರ್ಥ. (ಈ ರೀತಿಯ ವಿವಾಹ ಪದ್ಧತಿಗಳಲ್ಲಿ ಕೊನೆಯದೇ ಮಹ ಮ್ಮದೀಯರ ಭಾರತದ ಆಕ್ರಮಣಕ್ಕೆ ಕಾರಣವಾಯಿತು.)\footnote{1. ಡೆಲ್ಲಿಯ ರಾಣಿಯಾದ ರಾಜಪೂತ ಪುತ್ರಿ ಸಂಯೋಗಿತೆಯ ಕಥೆಗೆ ಪುಟ ೨೨೭-೨೮ ನೋಡಿ.} ಈ ವಿವಾಹ ಪದ್ಧತಿಯು ವಿಶೇಷವಾಗಿ ಕ್ಷತ್ರಿಯರಿಗೆ ಮಾತ್ರ ಸೀಮಿತವಾಗಿತ್ತು.

ಅತ್ಯಂತ ಪುರಾತನ ಸಂಸ್ಕೃತ ಕಾವ್ಯವಾದ ರಾಮಾಯಣದಲ್ಲಿ ಸೀತೆ ಎಂಬ ಆದರ್ಶ ಮಹಿಳೆಯ ವಿಷಯ ಬರುತ್ತದೆ. ಅವಳು ಹಿಂದೂ ಆದರ್ಶ ಮಹಿಳೆಯ ಪರಮ ಶ್ರೇಷ್ಠ ಪ್ರತಿನಿಧಿ. ಅಪಾರ ತಾಳ್ಮೆ ಮತ್ತು ಸದ್ಗುಣಗಳಿಂದ ಕೂಡಿದ ಆಕೆಯ ಜೀವನದ ಬಗ್ಗೆ ಹೇಳಲು ಈಗ ಸಮಯವಿಲ್ಲ. ಅವಳನ್ನು ಭಗವದವತಾರವೆಂದು ಪೂಜಿಸುತ್ತೇವೆ ಮತ್ತು ಅವಳ ಪತಿ ರಾಮನ ಹೆಸರಿನ ಹಿಂದೆ ಆಕೆಯ ಹೆಸರನ್ನು ಸೇರಿಸು ತ್ತೇವೆ (ಸೀತಾರಾಮ). ನಾವು ‘ಶ‍್ರೀಮಾನ್’ ಮತ್ತು ‘ಶ‍್ರೀಮತಿ’ ಎನ್ನುವುದರ ಬದಲು ‘ಶ‍್ರೀಮತಿ’ ಮತ್ತು ‘ಶ‍್ರೀಮಾನ್’ ಎನ್ನು ತ್ತೇವೆ. ದೇವತೆಗಳ ಹೆಸರನ್ನು ಹೇಳುವಾಗಲೆಲ್ಲ ದೇವಿಯ ಹೆಸರನ್ನು ಮೊದಲು ಸೇರಿಸುತ್ತೇವೆ.

ಹಿಂದೂಗಳಲ್ಲಿ ಇನ್ನೊಂದು ವಿಚಿತ್ರ ಭಾವನೆಯಿದೆ. ನನ್ನೊಡನೆ ಅಧ್ಯಯನ ಮಾಡುತ್ತಿರುವವರಿಗೆ, ವಿಶ್ವದ ಹಿನ್ನೆಲೆಯಾದ ನಿರಪೇಕ್ಷ ಸತ್ಯವೇ ಹಿಂದೂತತ್ತ್ವದ ಕೇಂದ್ರ ಭಾವನೆ ಎಂಬುದು ಗೊತ್ತಿದೆ. ಅನಿರ್ವಚನೀಯವಾದ ಈ ನಿರಪೇಕ್ಷ ಸತ್ಯವು ತನ್ನದೇ ಆದ ಶಕ್ತಿಯನ್ನು ಪಡೆದಿದೆ. ಈ ಶಕ್ತಿಯನ್ನು ಸ್ತ್ರೀಲಿಂಗ ಶಬ್ದದಿಂದ ಸಂಬೋಧಿಸುತ್ತೇವೆ. ಭಾರತದಲ್ಲಿ ಆಕೆಯೇ ಸಗುಣ ದೇವರು.

ರಾಮನು ನಿರಪೇಕ್ಷ ಸತ್ಯವನ್ನು ಪ್ರತಿನಿಧಿಸಿದರೆ, ಸೀತೆಯು ಅವನ ಶಕ್ತಿ. ಆಕೆಯ ಪೂರ್ಣ ಜೀವನದ ಬಗ್ಗೆ ಹೇಳಲು ಈಗ ಸಮಯವಿಲ್ಲ. ಆದರೆ ಆಕೆಯ ಜೀವನಕ್ಕೆ ಸಂಬಂಧಪಟ್ಟ ಒಂದು ಪಠ್ಯಭಾಗವನ್ನು ಇಲ್ಲಿ ಉದಾಹರಿಸುತ್ತೇನೆ, ಅದು ಈ ದೇಶದ (ಅಮೆರಿಕ) ಸ್ತ್ರೀಯರಿಗೆ ಅನ್ವಯಿಸುತ್ತದೆ.

ಸೀತೆಯು ಕಾಡಿನಲ್ಲಿ ತನ್ನ ಪತಿಯೊಡನೆ ಇರುವ ದೃಶ್ಯ. ಅಲ್ಲಿ ಓರ್ವ ಸ್ತ್ರೀ ಋಷಿ ಇದ್ದಳು. ಅವಳನ್ನು ನೋಡಲು ಇವರು ಹೋಗುತ್ತಾರೆ. ಅವಳ ಉಪವಾಸ, ಸಾಧನೆಗ ಳಿಂದಾಗಿ ಅವಳ ದೇಹ ತುಂಬ ಕೃಶವಾಗಿಹೋಗಿತ್ತು. ಸೀತೆಯು ಅವಳಿಗೆ ನಮಸ್ಕರಿಸಿ ದಳು. ಆ ಸಾಧ್ವಿಯು ತನ್ನ ಹಸ್ತವನ್ನು ಸೀತೆಯ ತಲೆಯ ಮೇಲಿರಿಸಿ ಹೇಳಿದಳು: “ಸುಂದರ ದೇಹವನ್ನು ಪಡೆಯುವುದೊಂದು ದೊಡ್ಡ ಭಾಗ್ಯ. ನೀನದನ್ನು ಪಡೆದಿರುವೆ. ಸತ್ಪುರುಷನಾದ ಪತಿಯನ್ನು ಪಡೆಯುವುದೂ ಒಂದು ಮಹಾ ಭಾಗ್ಯ. ನಿನಗದೂ ಲಭಿ ಸಿದೆ. ಅಂತಹ ಪತಿಗೆ ಸಂಪೂರ್ಣ ವಿಧೇಯಳಾಗಿರುವುದು ಇನ್ನೂ ಹೆಚ್ಚಿನ ಭಾಗ್ಯ. ನಿನಗದೂ ಲಭ್ಯವಾಗಿದೆ. ನೀನು ಖಂಡಿತ ಸುಖಿಯಾಗಿರಬೇಕು”.

ಸೀತೆಯು ಉತ್ತರಿಸುತ್ತಾಳೆ: “ತಾಯಿ, ದೇವರು ನನಗೆ ಸುಂದರ ದೇಹವನ್ನು ಕೊಟ್ಟಿ ರುವನು, ಮತ್ತು ಉತ್ತಮ ಪತಿಯನ್ನು ಕೊಟ್ಟಿರುವನು. ಇದು ತುಂಬ ಸಂತಸದ ವಿಷಯ. ಆದರೆ ಮೂರನೆಯ ಭಾಗ್ಯದ ವಿಷಯವಾಗಿ ಹೇಳ ಬೇಕಾದರೆ ನಾನು ಅವನಿಗೆ ವಿಧೇಯ ಳಾಗಿರುವೆನೊ, ಅವನು ನನಗೆ ವಿಧೇಯನಾಗಿರುವನೊ ಎಂಬುದು ನನಗೆ ತಿಳಿಯ ದಾಗಿದೆ. ಒಂದು ವಿಷಯ ನನಗೆ ನೆನಪಿದೆ, ಹೋಮಾಗ್ನಿಯ ಮುಂದೆ ಅವನು ನನ್ನ ಕೈ ಹಿಡಿದಾಗ ನಾನು ಅವನವಳು ಮತ್ತು ಅವನು ನನ್ನವನು ಎಂಬುದು ನನಗೆ ಅರಿವಾ ಯಿತು. ಇದು ಅಗ್ನಿಯ ಪ್ರಭಾವವೊ ಅಥವಾ ದೇವರೇ ನನಗೆ ಆ ಅರಿವುಂಟುಮಾಡಿ ದನೊ ತಿಳಿಯದು. ಅಂದಿನಿಂದ ನಾನು ಅವನ ಜೀವನಕ್ಕೆ ಪೂರಕ, ಅವನು ನನ್ನ ಜೀವನಕ್ಕೆ ಪೂರಕ ಎಂದು ಭಾವಿಸುತ್ತಿದ್ದೇನೆ”.

ಈ ಪದ್ಯದ ಕೆಲವು ಭಾಗಗಳನ್ನು ಇಂಗ್ಲಿಷ್ ಭಾಷೆಗೆ ಅನುವಾದಿಸಲಾಗಿದೆ. ಭಾರತದ ಆದರ್ಶ ಮಹಿಳೆ ಸೀತೆ, ಅವಳನ್ನು ಅವತಾರವೆಂದು ಪೂಜಿಸುತ್ತಾರೆ.

ಪ್ರಸಿದ್ಧ ಶಾಸನಕಾರನಾದ ಮನುವಿನ ವಿಷಯಕ್ಕೆ ಬರೋಣ. ಮಗುವಿಗೆ ಹೇಗೆ ವಿದ್ಯಾಭ್ಯಾಸವನ್ನು ನೀಡಬೇಕೆಂಬುದರ ಬಗ್ಗೆ ವಿಸ್ತಾರವಾದ ವಿವರಣೆ ಮನುಸ್ಮೃತಿಯಲ್ಲಿ ಇದೆ. ಯಾವುದೇ ವರ್ಣದವರಾಗಿರಲಿ, ಮಕ್ಕಳು ವಿದ್ಯಾಭ್ಯಾಸವನ್ನು ಪಡೆಯಲೇಬೇಕೆಂಬ ಕಡ್ಡಾಯ ನಿಬಂಧನೆ ಆರ್ಯರಲ್ಲಿತ್ತು. ಮಗುವಿಗೆ ಹೇಗೆ ಶಿಕ್ಷಣವನ್ನು ಕೊಡಬೇಕೆಂಬು ದನ್ನು ವಿವರಿಸಿ ಮನುವು ಹೇಳುತ್ತಾನೆ: “ಹುಡುಗರಂತೆಯೇ ಹೆಣ್ಣುಮಕ್ಕಳಿಗೂ ವಿದ್ಯಾಭ್ಯಾಸವನ್ನು ನೀಡಬೇಕು”\footnote{1. ಈ ವಾಕ್ಯವು ಈಗ ಲಭ್ಯವಿರುವ ಮನ ಸಂಹಿತೆಯಲ್ಲಿ ನಮಗೆ ದೊರೆಯುವುದಿಲ್ಲ. ನೋಡಿ: ಮಹಾನಿರ್ವಾಣ ತಂತ್ರ, ೮.೪೭}.

ಗ್ರಂಥದ ಬೇರೆ ಭಾಗಗಳಲ್ಲಿ ಮಹಿಳೆಯರನ್ನು ನಿಂದಿಸಲಾಗಿದೆ ಎಂದು ನಾನು ಕೇಳಿದ್ದೇನೆ. ನಮ್ಮ ಶಾಸ್ತ್ರದ ಎಷ್ಟೋ ಕಡೆಗಳಲ್ಲಿ ಮಹಿಳೆಯರು ಪ್ರಲೋಭನೆಯನ್ನು ಉಂಟುಮಾಡುತ್ತಾರೆಂದು ಅವರನ್ನು ಖಂಡಿಸಿರುವುದು ನಿಜ. ಆದರೆ ಭಗವಂತನ ಶಕ್ತಿ ಯೆಂದು ಅವರನ್ನು ವೈಭವೀಕರಿಸುವ ಹೇಳಿಕೆಗಳು ಬಹಳ ಇವೆ. ಯಾವ ಮನೆಯಲ್ಲಿ ಮಹಿಳೆಯ ಕಣ್ಣೀರು ಕೆಳಕ್ಕೆ ಬೀಳುವುದೊ ಆ ಮನೆಯ ವಿಷಯದಲ್ಲಿ ದೇವತೆಗಳು ಅಸಂತುಷ್ಟರಾಗಿ ಅದು ನಾಶಹೊಂದುತ್ತದೆ - ಎಂಬ ಹೇಳಿಕೆಗಳೂ ಇವೆ. ಮದ್ಯಸೇವನೆ, ಸ್ತ್ರೀಹತ್ಯೆ ಮತ್ತು ಬ್ರಹ್ಮಹತ್ಯೆ - ಇವು ಹಿಂದೂ ಧರ್ಮದಲ್ಲಿ ಅತಿಹೀನವಾದ ಪಾಪ ಕೃತ್ಯಗಳು. ನಿಂದಾತ್ಮಕ ವಾಕ್ಯಗಳಿವೆ ಎಂಬುದನ್ನು ನಾನು ಒಪ್ಪುತ್ತೇನೆ. ಆದರೂ ಈ ಗ್ರಂಥಗಳ ಶ್ರೇಷ್ಠತೆಯನ್ನು ನಾನು ಒತ್ತಿಹೇಳಬೇಕಾಗಿದೆ, ಏಕೆಂದರೆ ಬೇರೆ ಜನಾಂಗದ ಗ್ರಂಥಗಳಲ್ಲಿ ಮಹಿಳೆಯರ ಬಗ್ಗೆ ನಿಂದೆಯೊಂದೇ ಇರುವುದು, ಒಂದೇ ಒಂದು ಒಳ್ಳೆಯ ಮಾತೂ ಇಲ್ಲ.

ಈಗ ನಮ್ಮ ಹಳೆಯ ನಾಟಕಗಳ ವಿಷಯಕ್ಕೆ ಬರೋಣ. ಗ್ರಂಥಗಳು ಏನೇ ಹೇಳ ಬಹುದು, ನಾಟಕಗಳು ಮಾತ್ರ ಆಗಿನ ಕಾಲದ ಸಮಾಜವನ್ನು ಯಥಾವತ್ತಾಗಿ ಪ್ರತಿ ನಿಧಿಸುತ್ತವೆ. ಕ್ರಿಸ್ತಪೂರ್ವ ನಾನೂರು ವರ್ಷಗಳ ಅನಂತರ ಬರೆಯಲ್ಪಟ್ಟ ಈ ನಾಟಕ ಗಳು, ಯುವಕ ಯುವತಿಯರಿಂದ ತುಂಬಿರುವ ವಿಶ್ವವಿದ್ಯಾಲಯಗಳಿದ್ದವೆಂಬು ದನ್ನು ಸೂಚಿಸುತ್ತವೆ. ಈಗ ಹಿಂದೂ ಮಹಿಳೆಯರು ಉನ್ನತ ಶಿಕ್ಷಣದಿಂದ ಹೊರತಾಗಿರುವರು.\footnote{1. ಸ್ವಾಮಿ ವಿವೇಕಾನಂದರ ನಂತರ ಉನ್ನತ ಶಿಕ್ಷಣವು ಮಹಿಳೆಯರಲ್ಲಿ ಬಹುಬೇಗ ಹರಡುತ್ತಿದೆ.} ಆದರೆ ಆಗಿನಕಾಲದಲ್ಲಿ ಈಗ ಈ ದೇಶದಲ್ಲಿಯಂತೆಯೇ ಮಹಿಳೆಯರು ಎಲ್ಲೆಲ್ಲಿಯೂ ಓಡಾಡಿ ಕೊಂಡಿರುತ್ತಿದ್ದರು - ತೋಟ ಉದ್ಯಾನಗಳಲ್ಲಿ ವಿಹರಿಸುತ್ತಿದ್ದರು.

ಇನ್ನೊಂದು ವಿಷಯವನ್ನು ನಿಮ್ಮ ಗಮನಕ್ಕೆ ತರುತ್ತೇನೆ. ಅದು ಮಹಿಳೆಯ ಹಕ್ಕು ಗಳಿಗೆ ಸಂಬಂಧಪಟ್ಟಿದ್ದು. ಈ ವಿಷಯದಲ್ಲಿ ಹಿಂದೂ ಮಹಿಳೆಯು ಪ್ರಪಂಚದ ಎಲ್ಲ ಮಹಿಳೆಯರಿಗಿಂತ ಶ್ರೇಷ್ಠಳು. ಭಾರತದಲ್ಲಿ ಪುರುಷರಿಗಿರುವಷ್ಟೇ ಅವಳಿಗೂ ಆಸ್ತಿಯ ಮೇಲೆ ಹಕ್ಕಿದೆ. ಸಾವಿರಾರು ವರ್ಷಗಳಿಂದ ಈ ಪದ್ಧತಿ ಅಲ್ಲಿ ಜಾರಿಯಲ್ಲಿತ್ತು. ನಿಮಗೆ ಯಾರಾದರೂ ವಕೀಲ ಸ್ನೇಹಿತನಿದ್ದರೆ ಅವನ ಸಹಾಯದಿಂದ ಸ್ತ್ರೀಯರ ಹಕ್ಕುಗಳ ವಿಷಯವು ಹಿಂದೂಶಾಸನದ ಮೇಲಿರುವ ಭಾಷ್ಯಗಳಲ್ಲಿ ಇರುವುದನ್ನು ನೀವೇ ನೋಡ ಬಹುದು. ವಧುವು ಗಂಡನ ಮನೆಗೆ ಬರುವಾಗ ಲಕ್ಷಗಟ್ಟಳೆ ಹಣವನ್ನು ತರಬಹುದು, ಆದರೆ ಅದೆಲ್ಲವೂ ಪೂರ್ಣ ಆಕೆಯ ಸ್ವತ್ತು. ಅದರಲ್ಲಿ ಒಂದು ಪೈಸೆಯನ್ನೂ ಮುಟ್ಟಲು ಯಾರಿಗೂ ಅಧಿಕಾರವಿಲ್ಲ. ಗಂಡನು ಮಕ್ಕಳಿಲ್ಲದೆ ಕಾಲವಾದರೆ, ಅವನ ತಂದೆ ಅಥವಾ ತಾಯಿ ಜೀವಿಸಿದ್ದರೂ ಕೂಡ, ಅವನ ಈ ಇಡೀ ಆಸ್ತಿ ಪತ್ನಿಗೆ ಹೋಗುತ್ತದೆ. ಪುರಾತನ ಕಾಲದಿಂದಲೂ ಇದೇ ನಿಯಮವಾಗಿದೆ. ಈ ವಿಷಯದಲ್ಲಿ ಹಿಂದೂ ಮಹಿಳೆ ಬೇರೆ ಎಲ್ಲ ದೇಶಗಳ ಮಹಿಳೆಯರನ್ನು ಮೀರಿರುತ್ತಾಳೆ.

ಪುರಾತನ ಗ್ರಂಥಗಳು - ಅವಕ್ಕಿಂತ ಸ್ವಲ್ಪ ಇತ್ತೀಚಿನ ಗ್ರಂಥಗಳೂ ಕೂಡ - ವಿಧವಾ ವಿವಾಹವನ್ನು ನಿಷೇಧಿಸುವುದಿಲ್ಲ. ಹಾಗೆ ನಿಷೇಧಿಸುತ್ತವೆ ಎನ್ನುವುದು ಒಂದು ತಪ್ಪು ಕಲ್ಪನೆ. ಈ ವಿಷಯದಲ್ಲಿ ಅವು ಅವರಿಗೆ ಸ್ವಾತಂತ್ರ್ಯವನ್ನು ನೀಡಿವೆ. ಪುರುಷರಿಗೂ ಇದೇ ಸ್ವಾತಂತ್ರ್ಯವಿದೆ. ವಿವಾಹವು ದುರ್ಬಲರಿಗಾಗಿ ಎನ್ನುವುದು ನಮ್ಮ ಧರ್ಮದ ಭಾವನೆಯಾಗಿದೆ, ಇಂದೂ ಈ ಭಾವನೆಯನ್ನು ಕೈ ಬಿಡಬೇಕಾದ ಆವಶ್ಯಕತೆ ನನಗೆ ಕಂಡು ಬರುವುದಿಲ್ಲ. ತಮಗೆ ತಾವೇ ಪೂರ್ಣರಾಗಿರುವವರಿಗೆ ಮದುವೆಯಿಂದೇನು ಪ್ರಯೋಜನ? ಯಾರು ವಿವಾಹವಾಗುತ್ತಾರೊ ಅವರಿಗೆ ಒಂದೇ ಅವಕಾಶ ನೀಡ ಲಾಗಿತ್ತು. ಆ ಅವಕಾಶವು ಪೂರ್ತಿಯಾದ ಮೇಲೆ ಪುರುಷ ಅಥವಾ ಸ್ತ್ರೀಯಾಗಲಿ ಪುನಃ ವಿವಾಹವಾದರೆ ಸಮಾಜವು ಅವರನ್ನು ಕೀಳಾಗಿ ಕಾಣುತ್ತಿತ್ತು. ಆದರೆ ಪುನರ್ವಿ ವಾಹ ಅವರಿಗೆ ನಿಷೇಧವಾಗಿರಲಿಲ್ಲ. ವಿಧವೆಯು ಮದುವೆಯಾಗಬಾರದೆಂದು ಎಲ್ಲಿಯೂ ಹೇಳಿಲ್ಲ. ಮದುವೆಯಾಗದ ವಿಧುರ ಮತ್ತು ವಿಧವೆಯರನ್ನು ಆಧ್ಯಾತ್ಮಿಕ ವ್ಯಕ್ತಿಗಳೆಂದು ಪರಿಗಣಿಸಲಾಗುತ್ತಿತ್ತು.

ಆದರೆ ಪುರುಷರು ಈ ನಿಯಮವನ್ನು ಉಲ್ಲಂಘಿಸಿ ಪುನರ್ವಿವಾಹವಾಗುತ್ತಿದ್ದರು. ಮಹಿಳೆಯರು ಹೆಚ್ಚು ಆಧ್ಯಾತ್ಮಿಕ ಸ್ವಭಾವದವರಾದುದರಿಂದ ನಿಯಮಕ್ಕೆ ಬದ್ಧರಾಗಿ ರುತ್ತಿದ್ದರು. ಉದಾಹರಣೆಗೆ, ಮಾಂಸಾಹಾರವು ಕೆಟ್ಟದ್ದು, ಪಾಪ ಎಂದು ನಮ್ಮ ಗ್ರಂಥ ಗಳು ಹೇಳುತ್ತವೆ. ಆದರೆ ಕೆಲವು ಬಗೆಯ ಮಾಂಸವನ್ನು, ಉದಾಹರಣೆಗೆ ಕುರಿಮಾಂಸವನ್ನು, ತಿನ್ನಬಹುದು. ಸಾವಿರಾರು ಪುರುಷರು ಕುರಿಮಾಂಸ ತಿನ್ನುವುದನ್ನು ನಾನು ನೋಡಿದ್ದೇನೆ. ಆದರೆ ಉನ್ನತ ವರ್ಗಕ್ಕೆ ಸೇರಿದ ಯಾವ ಮಹಿಳೆಯೂ ಯಾವುದೇ ಬಗೆಯ ಮಾಂಸ ತಿನ್ನುವುದನ್ನೂ ನಾನು ನೋಡಿಲ್ಲ. ಇದು ನಿಯಮಕ್ಕೆ ಬದ್ಧರಾಗಿ ನಡೆ ಯುವ, ಧರ್ಮದ ಕಡೆಗೆ ಒಲವುಳ್ಳ ಅವರ ಸ್ವಭಾವವನ್ನು ತೋರಿಸುತ್ತದೆ. ಆದರೂ ಹಿಂದೂ ಪುರುಷರ ಬಗ್ಗೆ ಕಟುವಾದ ನಿರ್ಣಯಕ್ಕೆ ಬರಬೇಡಿ. ನಾನು ಒಬ್ಬ ಹಿಂದೂ ಪುರುಷನಾಗಿರುವುದರಿಂದ, ಹಿಂದೂ ಶಾಸನವನ್ನು ನನ್ನ ದೃಷ್ಟಿಕೋಣದಿಂದ ನೋಡಲು ನೀವು ಪ್ರಯತ್ನಿಸ ಬೇಕು.

ವಿಧವೆಯರು ಅವಿವಾಹಿತರಾಗಿ ಉಳಿಯುವುದು ಕ್ರಮೇಣ ಒಂದು ಪದ್ಧತಿ ಯಾಯಿತು. ಭಾರತದಲ್ಲಿ ಒಂದು ಪದ್ಧತಿಯು ಕಟ್ಟುನಿಟ್ಟಾಗಿಬಿಟ್ಟರೆ ಅದನ್ನು ಮೀರಿ ಹೋಗುವುದು ಸಾಧ್ಯವೇ ಇಲ್ಲ. ಇದು ಈ ದೇಶದಲ್ಲಿ ಮೂರುದಿನದ ಒಂದು ಫ್ಯಾಶ ನ್ನನ್ನು ಮೀರಿ ಹೋಗುವುದು ಅಸಾಧ್ಯವಾಗಿರುವಂತೆ. ಎರಡು ವರ್ಣಗಳನ್ನು ಬಿಟ್ಟು ಕೆಳಗಿನ ವರ್ಣಗಳಲ್ಲಿ ವಿಧವಾ ಪುನರ್ವಿವಾಹವಿದೆ.

ಮುಂದಿನ ಶಾಸನ ಗ್ರಂಥಗಳಲ್ಲಿ ಮಹಿಳೆಯರು ವೇದಗಳನ್ನು ಓದಬಾರದೆಂಬ ಹೇಳಿಕೆಗಳಿವೆ. ಆದರೆ ದುರ್ಬಲ ಬ್ರಾಹ್ಮಣನ ಮೇಲೂ ಈ ನಿಷೇಧವನ್ನು ಹೇರಲಾಗಿದೆ. ಒಬ್ಬ ಬ್ರಾಹ್ಮಣನ ಮಗನು ದುರ್ಬಲ ಮನಸ್ಸಿನವನಾದರೆ ಈ ನಿಯಮವು ಅವನಿಗೆ ಅನ್ವಯಿಸುತ್ತದೆ. ಆದರೆ ಅವರಿಗೆ ವಿದ್ಯಾಭ್ಯಾಸವನ್ನು ನಿಷೇಧಿಸಿಲ್ಲ, ಏಕೆಂದರೆ ಹಿಂದೂ ಗಳ ಸಾಹಿತ್ಯವೆಂದರೆ ಕೇವಲ ವೇದಗಳು ಮಾತ್ರವಲ್ಲ. ಬೇರೆ ಯಾವ ಗ್ರಂಥವನ್ನಾದರೂ ಮಹಿಳೆ ಓದಬಹುದು. ವಿಜ್ಞಾನ, ನಾಟಕ, ಕಾವ್ಯವೇ ಮುಂತಾದ ಅಪಾರ ಸಾಹಿತ್ಯ ರಾಶಿ, ಆ ಸಂಸ್ಕೃತ ಸಾಹಿತ್ಯ ಸಮುದ್ರವೇ ಅವರ ಮುಂದೆ ಇದೆ. ವೈದಿಕ ಸಾಹಿತ್ಯವೊಂದನ್ನು ಬಿಟ್ಟರೆ ಇವೆಲ್ಲವನ್ನೂ ಅವರು ಓದಬಹುದು.\footnote{1. ಈಗ ಭಾರತದಲ್ಲಿ ವೇದವನ್ನೊಳಗೊಂಡು ಯಾವ ಸಾಹಿತ್ಯವನ್ನು ಬೇಕಾದರೂ ಓದಬಹುದು.}

ಆಗಿನ ಕಾಲದಲ್ಲಿ ಮಹಿಳೆಯರು ಪೌರೋಹಿತ್ಯ ಮಾಡಬೇಕಾಗಿರಲಿಲ್ಲ. ಆದ್ದ ರಿಂದ ಅವರಿಗೆ ವೇದಾಧ್ಯಯನದ ಆವಶ್ಯಕತೆ ಏನಿದೆ? ಈ ವಿಷಯದಲ್ಲಿ ಹಿಂದೂ ಗಳು ಬೇರೆ ದೇಶಗಳಿಗಿಂತ ಬಹಳ ಹಿಂದೆಯೇನೂ ಇಲ್ಲ. ಮಹಿಳೆಯರು ಸಂಸಾರವನ್ನು ತ್ಯಜಿಸಿ ಸಂನ್ಯಾಸಿನಿಗಳಾದರೆ ಅವರು ಮಹಿಳೆಯರೂ ಅಲ್ಲ, ಪುರುಷರೂ ಅಲ್ಲ. ಅವರಿಗೆ ಯಾವ ಲಿಂಗವೂ ಇಲ್ಲ. ಮೇಲು ಕೀಳು ಅಥವಾ ಸ್ತ್ರೀ ಪುರುಷ ಮುಂತಾದ ಭೇದಗಳೆಲ್ಲ ಸಂಪೂರ್ಣ ಲುಪ್ತವಾಗುವುವು.

ನಾನು ನನ್ನ ಧಾರ್ಮಿಕ ಜ್ಞಾನವನ್ನು ನನ್ನ ಗುರುವಿನಿಂದ ಪಡೆದೆ, ಅವರು ಅದನ್ನು ಓರ್ವ ಮಹಿಳೆಯಿಂದ ಪಡೆದರು.

ರಾಜಪೂತ ಮಹಿಳೆಯ ವಿಷಯಕ್ಕೆ ಬರೋಣ. ನಮ್ಮ ಹಳೆಯ ಗ್ರಂಥಗಳಲ್ಲಿ ಬರುವ ಒಂದು ಕಥೆಯನ್ನು ಹೇಳುತ್ತೇನೆ. ಮುಸ್ಲಿಂ ಆಕ್ರಮಣದ ಸಮಯದಲ್ಲಿ ಓರ್ವ ಮಹಿಳೆಯು ಭಾರತದ ಪರಾಭವಕ್ಕೆ ಹೇಗೆ ಮೂಲ ಕಾರಣಳಾದಳು ಎಂಬು ದನ್ನು ಈ ಕಥೆ ತೋರಿಸುತ್ತದೆ.

ಕನ್ಯಾಕುಬ್ಜದ ರಾಜನಿಗೆ ಸಂಯೋಗಿತಾ ಎಂಬ ಮಗಳಿದ್ದಳು. ಅಜ್ಮೀರ್ ಮತ್ತು ಡೆಲ್ಲಿಯ ರಾಜನಾದ ಪೃಥ್ವೀರಾಜನ ಪರಾಕ್ರಮ ಮತ್ತು ಕೀರ್ತಿಯ ಬಗ್ಗೆ ಆಕೆ ಕೇಳಿದ್ದಳು, ಮತ್ತು ಅವನಲ್ಲಿ ಅನುರಕ್ತಳಾಗಿದ್ದಳು. ಅವಳ ತಂದೆಯು ರಾಜಸೂಯ ಯಾಗವನ್ನು ಮಾಡುವ ಉದ್ದೇಶದಿಂದ ದೇಶದ ಎಲ್ಲ ರಾಜರುಗಳನ್ನೂ ಆಹ್ವಾನಿಸಿ ದನು. ಈ ಯಾಗದಲ್ಲಿ ಅವರೆಲ್ಲರೂ ಕಾಯಕರ್ಮ ಮಾಡಬೇಕಾಗಿತ್ತು, ಏಕೆಂದರೆ ಇವರೆಲ್ಲರಿಗಿಂತಲೂ ಅವನು ಶ್ರೇಷ್ಠನಾಗಿದ್ದನು. ಈ ಯಾಗದ ಸಮಯದಲ್ಲಿ ತನ್ನ ಮಗಳ ಸ್ವಯಂವರವಿರುತ್ತದೆ ಎಂದು ರಾಜನು ಸಾರಿದನು.

ಆದರೆ ಮಗಳು ಆಗಲೇ ಪೃಥ್ವೀರಾಜನನ್ನು ವರಿಸಿದ್ದಳು. ಅವನು ಬಹಳ ಬಲ ಶಾಲಿಯಾದುದರಿಂದ ಈ ರಾಜನಿಗೆ ವಿಧೇಯನಾಗಿರಲು ಒಪ್ಪದೆ ಅವನು ಆಹ್ವಾನವನ್ನು ತಿರಸ್ಕರಿಸಿದನು. ರಾಜನು ಪೃಥ್ವೀರಾಜನ ಚಿನ್ನದ ವಿಗ್ರಹವನ್ನು ನಿರ್ಮಿಸಿ ಬಾಗಿ ಲಿನ ಸಮೀಪ ನಿಲ್ಲಿಸಿ, ಅವನಿಗೆ ದ್ವಾರಪಾಲಕನ ಕೆಲಸ ಕೊಡಲಾಗಿದೆ ಎಂದು ಹೇಳಿದನು.

ಇದರ ಒಟ್ಟು ಪರಿಣಾಮವೇನಾಯಿತೆಂದರೆ ಪೃಥ್ವೀರಾಜನು ವೀರಾವೇಶ ದಿಂದ ಬಂದು ರಾಜಕುಮಾರಿಯನ್ನು ಕುದುರೆಯ ಮೇಲೆ ಕೂರಿಸಿಕೊಂಡು ಓಡಿ ದನು. ರಾಜನಿಗೆ ವಿಷಯ ತಿಳಿದ ಕೂಡಲೆ ತನ್ನ ಸೈನ್ಯದೊಡನೆ ಅವನನ್ನು ಅಟ್ಟಿಸಿ ಕೊಂಡು ಹೋದನು. ಅನಂತರ ದೊಡ್ಡ ಯುದ್ಧವಾಗಿ ಎರಡೂ ಕಡೆಯವರಲ್ಲಿ ಹೆಚ್ಚಿನ ಸಂಖ್ಯೆಯ ಸೈನಿಕರು ಸತ್ತು ಹೋದರು. ಇದರಿಂದಾಗಿ ರಜಪೂತರು ಅತ್ಯಂತ ದುರ್ಬಲ ರಾಗಿ ಮುಸ್ಲಿಂ ಸಾಮ್ರಾಜ್ಯವು ಭಾರತದಲ್ಲಿ ಪ್ರಾರಂಭವಾಗುವುದಕ್ಕೆ ಕಾರಣರಾದರು.

ಮುಸ್ಲಿಂ ಸಾಮ್ರಾಜ್ಯವು ಭಾರತದಲ್ಲಿ ವಿಸ್ತರಿಸುತ್ತಿರುವಾಗ ಚಿತ್ತೂರಿನ ರಾಣಿ ಪದ್ಮಿನಿಯು ಸೌಂದರ್ಯಕ್ಕೆ ಪ್ರಸಿದ್ಧಳಾಗಿದ್ದಳು. ಆಕೆಯ ಸೌಂದರ್ಯದ ವಿಷಯ ಸುಲ್ತಾನನಿಗೆ ತಿಳಿದು, ಅವಳನ್ನು ತನ್ನ ಅಂತಃಪುರಕ್ಕೆ ಕಳಿಸಬೇಕೆಂದು ಅವನು ಪತ್ರ ಬರೆಯುತ್ತಾನೆ. ಇದರ ಪರಿಣಾಮವಾಗಿ ಚಿತ್ತೂರಿನ ರಾಜನಿಗೂ ಸುಲ್ತಾನನಿಗೂ ದೊಡ್ಡ ಯುದ್ಧ ನಡೆಯುತ್ತದೆ. ಮಹಮ್ಮದೀಯರು ಚಿತ್ತೂರಿನ ಮೇಲೆ ಆಕ್ರಮಣ ಮಾಡು ತ್ತಾರೆ. ಸಾಮ್ರಾಜ್ಯದ ರಕ್ಷಣೆ ಅಸಾಧ್ಯವೆಂದು ರಜಪೂತರಿಗೆ ತಿಳಿದಾಗ ಗಂಡಸರೆಲ್ಲ ಕತ್ತಿಯನ್ನು ಹಿಡಿದುಕೊಂಡು ಹೋರಾಡಿ ಮರಣ ಹೊಂದಿದರು, ಮತ್ತು ಮಹಿಳೆ ಯರು ಬೆಂಕಿಯಲ್ಲಿ ಬಿದ್ದು ಪ್ರಾಣ ತೆತ್ತರು.

ಪುರುಷರೆಲ್ಲ ಮಡಿದ ಮೇಲೆ ಜಯಶಾಲಿಯಾದ ಸುಲ್ತಾನನು ನಗರವನ್ನು ಪ್ರವೇಶಿಸಿದನು. ಅಲ್ಲಿ ಬೀದಿಯಲ್ಲಿ ಭಯಂಕರ ಬೆಂಕಿ ಉರಿಯುತ್ತಿತ್ತು. ರಾಣಿಯನ್ನು ಹಿಂಬಾಲಿಸಿ ಮಹಿಳೆಯರು ಬೆಂಕಿಯ ಸುತ್ತ ತಿರುಗುವುದನ್ನು ಅವನು ಕಂಡನು. ಅವನು ಹತ್ತಿರ ಹೋಗಿ ಬೆಂಕಿಗೆ ಬೀಳುವುದು ಬೇಡವೆಂದು ಹೇಳಿದಾಗ ಅವಳು, “ರಜಪೂತ ಮಹಿಳೆಯು ನಿನ್ನನ್ನು ಆದರಿಸುವ ರೀತಿ ಇದು” ಎಂದು ಹೇಳಿ ಬೆಂಕಿಗೆ ಬೀಳುತ್ತಾಳೆ.

ಮಹಮ್ಮದೀಯರಿಂದ ಮಾನಹಾನಿಗೆ ಒಳಗಾಗುವುದನ್ನು ತಪ್ಪಿಸಿಕೊಳ್ಳಲು ಆ ದಿವಸ ೭೪, ೫೦೦ ಮಹಿಳೆಯರು ಬೆಂಕಿಗೆ ಆಹುತಿಯಾದರಂತೆ. ಇಂದಿಗೂ ಕೂಡ ಪತ್ರವನ್ನು ಬರೆದು ಅದನ್ನು ಅಂಟಿಸಿಯಾದ ಮೇಲೆ “೭೪-೧/೨” ಎಂದು ಬರೆಯುವ ರೂಢಿಯಿದೆ. ಇದರ ಅರ್ಥ, ಯಾರಾದರೂ ಆ ಪತ್ರವನ್ನು ತೆರೆದು ನೋಡಿ ದರೆ ಅವರು ೭೪, ೫೦೦ ಸ್ತ್ರೀಹತ್ಯಾ ಪಾಪಕ್ಕೆ ಗುರಿಯಾಗುತ್ತಾರೆ ಎಂದು.

ನಾನು ಇನ್ನೋರ್ವ ರಜಪೂತ ರಾಜಕುಮಾರಿ\footnote{1. ರಾಜಾಸ್ತಾನದಲ್ಲಿರುವ ಕೃಷ್ಣಘರ್ ಪ್ರಾಂತ್ಯದ ರಾಜನಾದ ವಿಕ್ರಮ ಸಿಂಘನ ಮಗಳು ಚಾರುಮತಿ ಅಥವಾ ರೂಪಮತಿ. ಬಂಕಿಮ ಚಂದ್ರರ ‘ರಾಜಸಿಂಹ’ ಎಂಬ ಚಾರಿತ್ರಿಕ ಕಾದಂಬರಿಯ ಕಥಾನಾಯಕಿ ಇವಳು.} ಯ ಕಥೆಯನ್ನು ಹೇಳುತ್ತೇನೆ. ನಮ್ಮ ದೇಶದಲ್ಲಿ ‘ರಕ್ಷಾಬಂಧ’ ಎಂಬ ಒಂದು ವಿಚಿತ್ರ ಪದ್ಧತಿಯಿದೆ. ಮಹಿಳೆಯರು ರೇಶ್ಮೆದಾರದ ಕಂಕಣವನ್ನು ಪುರುಷರಿಗೆ ಕಳುಹಿಸುತ್ತಾರೆ. ಹುಡುಗಿಯೊಬ್ಬಳು ಪುರುಷ ನಿಗೆ ಈ ರೀತಿ ಕಂಕಣವನ್ನು ಕಳಿಸಿದಾಗ ಅವನು ಆಕೆಯ ಸಹೋದರನಾಗುತ್ತಾನೆ.

ಭಾರತದಲ್ಲಿ ವೈಭವ ಪೂರ್ಣವಾದ ಮೊಗಲ್ ಸಾಮ್ರಾಜ್ಯದ ನಾಶಕ್ಕೆ ಕಾರಣನಾದ ಅದರ ಕೊನೆಯ ಕ್ರೂರ ರಾಜನ ಆಳ್ವಿಕೆಯ ಸಮಯದಲ್ಲಿ, ಅವನೂ ಕೂಡ ರಜಪೂತ ರಾಜನೊಬ್ಬನ ಮಗಳ ಅಪಾರ ಸೌಂದರ್ಯದ ವಿಷಯವನ್ನು ಕೇಳಿದನು. ಮೊಗಲ್ ರಾಜನ ಅಂತಃಪುರಕ್ಕೆ ಅವಳನ್ನು ಕಳಿಸಬೇಕೆಂದು ಆಜ್ಞೆಯನ್ನು ಹೊರಡಿಸಲಾಯಿತು. ದೂತನೊಬ್ಬನು ರಾಜನ ಚಿತ್ರದೊಡನೆ ಬಂದು ಅದನ್ನು ಆಕೆಗೆ ತೋರಿಸುತ್ತಾನೆ. ಅವಳು ಕುಪಿತಳಾಗಿ ಆ ಚಿತ್ರವನ್ನು ತುಳಿದು, “ನಿಮ್ಮ ಮೊಗಲ ರಾಜನನ್ನು ರಾಜಪೂತ ಯುವತಿಯು ಆದರಿಸುವ ಬಗೆ ಇದು” ಎಂದು ನುಡಿಯುತ್ತಾಳೆ. ಇದರ ಪರಿಣಾಮವಾಗಿ ಮೊಗಲರ ಸೇನೆಯು ರಾಜಪುತಾನದ ಮೇಲೆ ದಂಡೆತ್ತಿ ಬರುತ್ತದೆ.

ಈಗ ರಾಜಕುಮಾರಿಯು ಬೇರೆ ದಾರಿಗಾಣದೆ, ಅನೇಕ ಕಂಕಣಗಳನ್ನು ಅಣಿ ಗೊಳಿಸಿ ಬೇರೆ ಬೇರೆ ರಜಪೂತ ರಾಜರುಗಳಿಗೆ ಅವರ ಸಹಾಯವನ್ನು ಕೋರಿ ಕಳಿಸು ತ್ತಾಳೆ. ಅವರೆಲ್ಲರೂ ಸಹಾಯಕ್ಕೆ ಬಂದದ್ದರಿಂದ ಮೊಗಲ ಸೈನ್ಯವು ಹಿಂತಿರುಗಬೇಕಾಯಿತು.

ರಾಜಪುತಾನದಲ್ಲಿ ಪ್ರಚಲಿತವಿರುವ ಒಂದು ವಿಚಿತ್ರ ಗಾದೆಯನ್ನು ನಿಮಗೆ ಹೇಳು ತ್ತೇನೆ. ಭಾರತದಲ್ಲಿ ವೈಶ್ಯರೆಂದು ಕರೆಯಲ್ಪಡುವ ಒಂದು ವರ್ಣದವರಿದ್ದಾರೆ. ಅವರು ಬಹಳ ಬುದ್ಧಿವಂತರು, ಆದರೆ ಹಿಂದೂಗಳು ಅವರು ತುಂಬ ಸ್ವಹಿತಾಸಕ್ತರೆಂದು ಭಾವಿಸು ತ್ತಾರೆ. ಆ ವರ್ಣಕ್ಕೆ ಸೇರಿದ ಮಹಿಳೆಯರು ಪುರುಷರಷ್ಟು ಬುದ್ಧಿವಂತರಲ್ಲ. ಆದರೆ ರಜಪೂತ ಪುರುಷನಿಗೆ ರಜಪೂತ ಮಹಿಳೆಗಿರುವ ಅರ್ಧದಷ್ಟು ಬುದ್ಧಿಶಕ್ತಿಯೂ ಇಲ್ಲ. ರಾಜಪುತಾನದಲ್ಲಿ ಪ್ರಚಲಿತವಿರುವ ಸಾಮಾನ್ಯ ಗಾದೆ ಇದು: “ಬುದ್ಧಿವಂತ ಮಹಿಳೆಗೆ ದಡ್ಡ ಪುತ್ರನು ಹುಟ್ಟುತ್ತಾನೆ, ದಡ್ಡ ಮಹಿಳೆಗೆ ಬುದ್ಧಿವಂತ ಪುತ್ರನು ಹುಟ್ಟುತ್ತಾನೆ”. ನಿಜ ವಾದ ಸಂಗತಿಯೇನೆಂದರೆ, ರಜಪುತಾನದಲ್ಲಿ ಯಾವುದೇ ಸಾಮ್ರಾಜ್ಯವು ಮಹಿಳೆಯಿಂದ ಅಳಲ್ಪಟ್ಟಾಗಲೆಲ್ಲ ಅವಳು ಅತ್ಯಂತ ದಕ್ಷತೆಯಿಂದ ಆಳುತ್ತಿದ್ದಳು.

ಇನ್ನೊಂದು ವರ್ಗದ ಮಹಿಳೆಯರ ವಿಷಯಕ್ಕೆ ಬರೋಣ. ಈ ಮೃದು ಸ್ವಭಾವದ ಹಿಂದೂ ಜನಾಂಗವು ಆಗಾಗ ಮಹಿಳಾ ಯೋಧರನ್ನು ಸೃಷ್ಟಿಸುತ್ತದೆ. ನೀವು ಝಾನ್ಸಿ ರಾಣಿ ಲಕ್ಷ್ಮೀಬಾಯಿಯ ವಿಷಯವನ್ನು ಕೇಳಿರಬಹುದು. ಅವಳು ಬ್ರಿಟಿಷರನ್ನೇ ಎದುರಿಸಿ ಎರಡು ವರ್ಷಗಳವರೆಗೆ ಅವರಿಗೆ ಮಣಿಯದೆ, ಆಧುನಿಕ ಸೈನ್ಯದ ಮುಂಚೂಣಿ ಯಾಗಿ ಹೋರಾಡಿದಳು. ಈ ರಾಣಿಯು ಬ್ರಾಹ್ಮಣ ವರ್ಣಕ್ಕೆ ಸೇರಿದ ಮಹಿಳೆ.

ನನಗೆ ತಿಳಿದಿರುವ ಒಬ್ಬ ವ್ಯಕ್ತಿಯು ಆ ಯುದ್ಧದಲ್ಲಿ ತನ್ನ ಮೂವರು ಮಕ್ಕಳನ್ನು ಕಳೆದುಕೊಂಡನು. ಅವರ ಬಗ್ಗೆ ಹೇಳುವಾಗ ಅವನು ಶಾಂತವಾಗಿರುತ್ತಾನೆ. ಆದರೆ ಅವನು ಈ ಮಹಿಳೆಯ ಬಗ್ಗೆ ಮಾತನಾಡುವಾಗಲಂತೂ ತುಂಬ ಭಾವೋದ್ವೇಗಕ್ಕೆ ಒಳಗಾಗುತ್ತಾನೆ. ಅವಳು ಮಾನವಳಲ್ಲ, ದೇವಿಯೇ ಎಂದು ಅವನು ಹೇಳುತ್ತಿದ್ದನು. ಅವಳಿಗಿಂತ ಉತ್ತಮ ಸೇನಾಧಿಪತಿಯನ್ನು ತಾನು ನೋಡಿಲ್ಲವೆಂದು ಈ ವೃದ್ಧ ಯೋಧನು ಹೇಳುತ್ತಾನೆ.

ಚಾಂದ್ ಬೀಬಿ ಅಥವಾ ಚಾಂದ್ ಸುಲ್ತಾನಳ (೧೫೪೬-೧೫೯೯)\footnote{1. ಸೈನಿಕರು ಚಾಂದ್ ಬೀಬಿಯ ದೈರ್ಯ ಸಾಹಸಗಳಿಗೆ ಮೆಚ್ಚಿ ಅವಳನ್ನು ಚಾಂದ್ ಸುಲ್ತಾನಾ, ಅಂದರೆ ಚಾಂದ್ ಚಕ್ರವರ್ತಿನಿ ಎಂದು ಕರೆಯತೊಡಗಿದರು.} ಕಥೆ ಭಾರತದಲ್ಲಿ ಪ್ರಸಿದ್ಧವಾದುದು. ಅವಳು ಚಿನ್ನದ ಗಣಿ ಇದ್ದ ಗೋಲ್ಕೊಂಡದ ರಾಣಿಯಾಗಿದ್ದಳು. ಹಲವು ತಿಂಗಳು ಹೋರಾಡಿ ತನ್ನ ರಾಜ್ಯವನ್ನು ರಕ್ಷಿಸಿದಳು. ಕೊನೆಯಲ್ಲಿ ಕೋಟೆಯ ಗೋಡೆಯನ್ನು ಒಡೆಯಲಾಯಿತು. ಶತ್ರು ಸೈನ್ಯವು ಒಳನುಗ್ಗಲು ಪ್ರಯತ್ನಿಸಿದಾಗ ಅವಳು ಶಸ್ತ್ರ ಸನ್ನದ್ಧಳಾಗಿ ಹೋರಾಡಿ ಸೈನ್ಯವನ್ನು ಹಿಮ್ಮೆಟ್ಟಿಸಿದಳು.

ಇನ್ನೂ ಇತ್ತೀಚಿನ ಸಮಯಕ್ಕೆ ಬಂದರೆ, ಪ್ರಸಿದ್ಧ ಇಂಗ್ಲಿಷ್ ಸೇನಾನಿಯೊಬ್ಬನು ಹದಿನಾರು ವರ್ಷದ ಹಿಂದೂ ಹುಡುಗಿಯನ್ನು ಎದುರಿಸಬೇಕಾದ ಪ್ರಸಂಗವನ್ನು ಕೇಳಿ ನಿಮಗೆ ಆಶ್ಚರ್ಯವಾಗಬಹುದು.

ರಾಜಕೀಯ ಮತ್ಸದ್ದಿತನದಲ್ಲಿ, ಗಡಿರಕ್ಷಣೆಯಲ್ಲಿ, ರಾಜ್ಯಾಡಳಿತದಲ್ಲಿ ಹಾಗೂ ಯುದ್ಧ ಮಾಡುವುದರಲ್ಲಿಯೂ ಕೂಡ, ಪುರುಷರಿಗಿಂತ ಮೇಲಲ್ಲದಿದ್ದರೂ ತಾವು ಅವರಿಗೆ ಸಮಾನರೆಂಬುದನ್ನು ಮಹಿಳೆಯರು ತೋರಿಸಿರುವರು. ಭಾರತದಲ್ಲಂತೂ ಈ ವಿಷಯದಲ್ಲಿ ಸಂದೇಹವೇ ಇಲ್ಲ. ತಮಗೆ ಅವಕಾಶ ದೊರಕಿದಾಗಲೆಲ್ಲ ಪುರುಷ ರಷ್ಟೇ ತಾವು ಸಮರ್ಥರು ಎಂಬುದನ್ನು ಅವರು ಪ್ರಮಾಣೀಕರಿಸಿರುವರು. ಅಷ್ಟೇ ಅಲ್ಲದೆ ಅವರು ಪುರುಷರಂತೆ ಅವನತಿ ಹೊಂದುವುದಿಲ್ಲ. ಅವರು ಯಾವಾಗಲೂ ನೀತಿ ಬದ್ಧರಾಗಿರುತ್ತಾರೆ. ಅದು ಅವರ ಸ್ವಭಾವ. ಈ ವಿಷಯದಲ್ಲಿ - ಕೊನೆಯ ಪಕ್ಷ ಭಾರತದಲ್ಲಿ ಅವರು ರಾಜ್ಯಭಾರ ಮಾಡುವಾಗ ಪುರುಷರಿಗಿಂತ ತಾವು ಮೇಲೆಂಬುದನ್ನು ತೋರಿಸಿರುವರು. ಜಾನ್ ಸ್ಟುವರ್ಟ್ ಮಿಲ್ ಈ ವಿಷಯವನ್ನು ಪ್ರಸ್ತಾಪಿಸುತ್ತಾನೆ.

ಈಗಿನ ಕಾಲದಲ್ಲಿಯೂ ಕೂಡ ಭಾರತದಲ್ಲಿ ಮಹಿಳೆಯರು ದೊಡ್ಡ ದೊಡ್ಡ ಎಸ್ಟೇಟುಗಳನ್ನು ಸಮರ್ಥವಾಗಿ ನೋಡಿಕೊಳ್ಳುತ್ತಿರುವುದನ್ನು ನೋಡುತ್ತೇವೆ. ನಾನು ಹುಟ್ಟಿದ ಪ್ರದೇಶದಲ್ಲಿ ಇಬ್ಬರು ಮಹಿಳೆಯರಿದ್ದರು. ಅವರು ದೊಡ್ಡ ಎಸ್ಟೇಟ್ಗಳ ಒಡತಿಯರಾಗಿದ್ದರು ಮತ್ತು ಸಾಹಿತ್ಯ ಮತ್ತು ಕಲೆಗೆ ಹೆಚ್ಚು ಪ್ರೋತ್ಸಾಹ ಕೊಡುತ್ತಿ ದ್ದರು. ಅವರು ತಮ್ಮ ಸ್ವಂತ ಬುದ್ಧಿಶಕ್ತಿಯಿಂದಲೆ ವ್ಯವಹಾರದ ಎಲ್ಲ ವಿವರಗಳನ್ನು ನೋಡಿಕೊಳ್ಳುತ್ತಿದ್ದರು.

ಪ್ರತಿಯೊಂದು ದೇಶವೂ ಸಾಮಾನ್ಯ ಮಾನವ ಸ್ವಭಾವವನ್ನು ಮೀರಿದ, ತನ್ನದೇ ಆದ ವಿಶಿಷ್ಟ ಲಕ್ಷಣಗಳನ್ನು ಹೊಂದಿರುತ್ತದೆ. ಧರ್ಮ, ರಾಜಕೀಯ, ಭೌತಿಕ ದೇಹ, ಮಾನಸಿಕ ಪ್ರವೃತ್ತಿ, ಸ್ತ್ರೀಪುರುಷರು, ನಡತೆ - ಇತ್ಯಾದಿಗಳು ಬೇರೆ ಬೇರೆ ದೇಶಗಳಲ್ಲಿ ತಮ್ಮದೇ ವೈಶಿಷ್ಟ್ಯವನ್ನು ಪಡೆದಿರುತ್ತವೆ. ಒಂದು ದೇಶವು ಒಂದು ಬಗೆಯ ಶೀಲವನ್ನು ರೂಪಿಸುತ್ತದೆ, ಇನ್ನೊಂದು ದೇಶವು ಇನ್ನೊಂದು ಬಗೆಯದನ್ನು. ಕಳೆದ ಕೆಲವು ವರ್ಷಗಳಿಂದೀಚೆಗೆ ಪ್ರಪಂಚವು ಇದನ್ನು ಗಮನಿಸಲು ಪ್ರಾರಂಭಿಸಿತು.

ಭಾರತದಲ್ಲಿ ಮಹಿಳೆಯ ವೈಶಿಷ್ಟ್ಯ ಯಾವುದೆಂದರೆ ಅವಳ ತಾಯ್ತನ. ನೀವು ಹಿಂದೂ ಗೃಹವನ್ನು ಪ್ರವೇಶಿಸಿದರೆ, ಇಲ್ಲಿಯಂತೆ ಪತ್ನಿಯು ಪತಿಯ ಸಹಭಾಗಿಯಾಗಿರು ವುದನ್ನು ಕಾಣುವುದಿಲ್ಲ. ಆದರೆ ತಾಯಿಯು ಮನೆಯ ಮೂಲಸ್ತಂಭವಾಗಿರು ವುದನ್ನು ನೋಡುತ್ತೀರಿ. ಪತ್ನಿಯು ತಾಯಿಯಾಗುವವರೆಗೆ ಕಾಯಬೇಕು, ಅನಂತರ ಅವಳೇ ಸರ್ವಸ್ವವಾಗುತ್ತಾಳೆ.

ಒಬ್ಬನು ಸಂನ್ಯಾಸಿಯಾದರೆ ಅವನ ತಂದೆಯು ಅವನಿಗೆ ನಮಸ್ಕರಿಸಬೇಕು. ಏಕೆಂದರೆ ಅವನು ಸಂನ್ಯಾಸಿಯಾಗಿರುವುದರಿಂದ ತನ್ನ ತಂದೆಗಿಂತ ಮೇಲು. ಆದರೆ ತಾಯಿಗೆ, ಸಂನ್ಯಾಸಿಯಿರಲಿ ಇಲ್ಲದಿರಲಿ, ಮಂಡಿಯೂರಿ ಪ್ರಣಾಮ ಸಲ್ಲಿಸಬೇಕು. ಅವನು ತಾಯಿಯ ಮುಂದೆ ಒಂದು ಸಣ್ಣ ಲೋಟದಲ್ಲಿ ನೀರನ್ನು ಇಡುತ್ತಾನೆ. ಅವಳು ತನ್ನ ಕಾಲ್ಬೆರಳನ್ನು ಅದರಲ್ಲಿ ಅದ್ದುತ್ತಾಳೆ. ಅವನು ಆ ನೀರನ್ನು ತೀರ್ಥದಂತೆ ಸ್ವೀಕರಿಸು ತ್ತಾನೆ. ಹಿಂದೂ ಪುತ್ರನು ಇದನ್ನು ಸಾವಿರಾರು ಬಾರಿ ಸಂತೋಷದಿಂದ ಮಾಡುತ್ತಾನೆ.\footnote{1. ಇದು ತಾಯಿಯನ್ನು ಮೊದಲ ಗುರುವೆಂದು ಮಾತ್ರವಲ್ಲದೆ, ಪ್ರೇಮ - ಸ್ವರೂಪಿ ಭಗವಂತನ ಪ್ರತಿರೂಪನೆಂದು ಗೌರವಿಸುವ ಒಂದು ಪದ್ಧತಿ.}

ವೇದಗಳು ನೈತಿಕತೆಯನ್ನು ಬೋಧಿಸುವಾಗ ಮೊದಲನೆಯ ವಾಕ್ಯವೇ “ತಾಯಿ ಯು ನಿನ್ನ ದೇವರಾಗಲಿ”\footnote{2. ತೈ ತ್ತಿರೀಯ ಉಪನಿಷತ್ ೧.೧೧} ಎಂಬುದು. ಭಾರತದಲ್ಲಿ ನಾವು ಮಹಿಳೆಯ ವಿಷಯ ಮಾತ ನಾಡುವಾಗ, ಮಹಿಳೆ ಎಂದರೆ ತಾಯಿ ಎಂಬುದೇ ನಮ್ಮ ಭಾವನೆ. ಮಾನವ ಜನಾಂಗದ ತಾಯಂದಿರಾಗಿರುವುದರಲ್ಲಿಯೇ ಮಹಿಳೆಯರ ವೈಶಿಷ್ಟ್ಯವಿರುವುದು. ಇದು ಹಿಂದೂ ಭಾವನೆ.

ನಾನು ನನ್ನ ಗುರುವು ಚಿಕ್ಕ ಹುಡುಗಿಯರನ್ನು ಪೀಠದ ಮೇಲೆ ಕೂರಿಸಿ ಧೂಪ ದೀಪಾದಿಗಳೊಡನೆ ಪೂಜಿಸುವುದನ್ನು ನೋಡಿದ್ದೇನೆ. ಅವರನ್ನು ಜಗನ್ಮಾತೆಯ ಪ್ರತಿ ನಿಧಿಗಳೆಂದು ಭಾವಿಸಿ ಸಾಷ್ಟಾಂಗ ಪ್ರಣಾಮ ಮಾಡುತ್ತಿದ್ದರು.

ನಮ್ಮ ಕುಟುಂಬದಲ್ಲಿ ತಾಯಿಯೇ ದೇವತೆ. ಈ ಪ್ರಪಂಚದಲ್ಲಿ ನಿಜವಾದ ಪ್ರೀತಿ ಯನ್ನು ಅತ್ಯಂತ ನಿಸ್ವಾರ್ಥ ಪ್ರೀತಿಯನ್ನು ಕಾಣುವುದು ತಾಯಿಯಲ್ಲಿ ಮಾತ್ರವೇ. ಎಲ್ಲ ದುಃಖವನ್ನು ಅನುಭವಿಸಿ ಯಾವಾಗಲೂ ಪ್ರೀತಿಸುತ್ತಿರುವುದು ಅವಳ ಸ್ವಭಾವ. ತಾಯಿಯ ಪ್ರೇಮವಲ್ಲದೆ ಮತ್ತಾವುದು ಭಗವತ್ಪ್ರೇಮವನ್ನು ಪ್ರತಿನಿಧಿಸಬಲ್ಲದು? ಆದ್ದರಿಂದ ಹಿಂದೂಗಳಿಗೆ ತಾಯಿಯು ಭಗವಂತನ ಅವತಾರ.

“ಯಾವ ಹುಡುಗನು ಮೊದಲು ತಾಯಿಯಿಂದ ಬೋಧಿಸಲ್ಪಟ್ಟಿರುತ್ತಾನೊ ಅವನು ಮಾತ್ರ ದೇವರನ್ನು ಅರ್ಥಮಾಡಿಕೊಳ್ಳಬಲ್ಲ”. ನಮ್ಮ ಮಹಿಳೆಯರ ಅನ ಕ್ಷರತೆಯ ಬಗ್ಗೆ ನಾನು ವಿಕೃತ ಕಥೆಗಳನ್ನು ಕೇಳಿರುವೆನು. ನಾನು ಹತ್ತು ವರ್ಷದ ಬಾಲಕನಾಗುವವರೆಗೂ ನನ್ನ ತಾಯಿಯೇ ನನಗೆ ಬೋಧಿಸಿದವಳು. ನಾನು ನನ್ನ ಅಜ್ಜಿ ಮುತ್ತಜ್ಜಿಯರನ್ನು ನೋಡಿದ್ದೇನೆ. ನನ್ನ ವಂಶದಲ್ಲಿ ಓದು ಬರಹ ಬರದ, ಹೆಬ್ಬೆರಳೊತ್ತುವ ಪೂರ್ವದ ಮಹಿಳೆಯರಾರೂ ಇರಲಿಲ್ಲವೆಂದು ನಾನು ಖಂಡಿತವಾಗಿ ಹೇಳುತ್ತೇನೆ. ಓದು ಬರಹ ಬರದ ಮಹಿಳೆಯು ಇದ್ದಿದ್ದರೆ ನನ್ನ ಜನನವೇ ಸಾಧ್ಯವಾಗುತ್ತಿರಲಿಲ್ಲ - ಜಾತಿ ನಿಯಮವು ಇದನ್ನು ಕಡ್ಡಾಯವಾಗಿಸುತ್ತದೆ.

ಮಧ್ಯಯುಗದಲ್ಲಿ ಹಿಂದೂ ಮಹಿಳೆಯರಿಗೆ ವಿದ್ಯಾಭ್ಯಾಸ ನಿಷಿದ್ಧವಾಗಿತ್ತು ಎಂದು ಮುಂತಾದ ಸತ್ಯದೂರವಾದ ಕಥೆಗಳನ್ನು ನಾನು ಕೇಳಿರುವೆನು. ಸರ್ ವಿಲಿಯಂ ಹಂಟರನ \enginline{History of English People} ಗ್ರಂಥವನ್ನು ಓದಿ. ಅದರಲ್ಲಿ ಸೂರ್ಯ ಗ್ರಹಣ ಕಾಲವನ್ನು ಗುಣಿಸಿ ನಿಶ್ಚಯಿಸಬಲ್ಲ ಮಹಿಳೆಯರ ಪ್ರಸ್ತಾಪವಿದೆ.

ತಾಯಿಯನ್ನು ಅತಿಯಾಗಿ ಪೂಜಿಸುವುದು ಆಕೆಯನ್ನು ಸ್ವಾರ್ಥಿಯನ್ನಾಗಿಸುತ್ತದೆ ಅಥವಾ ತಾಯಿಯ ಅತಿಯಾದ ಪ್ರೀತಿಯಿಂದ ಮಕ್ಕಳು ಸ್ವಾರ್ಥಿಗಳಾಗುತ್ತಾರೆ ಎಂದು ನಾನು ಕೇಳಿದ್ದೇನೆ. ಆದರೆ ನನಗಿದರಲ್ಲಿ ನಂಬಿಕೆಯಿಲ್ಲ. ನನ್ನ ತಾಯಿಯು ನನಗೆ ತೋರಿದ ಪ್ರೀತಿಯಿಂದಾಗಿಯೇ ನಾನೀಗಿರುವಂತೆ ಆಗಿರುವುದು, ಆಕೆಯ ಋಣವನ್ನು ನಾನೆಂದೂ ತೀರಿಸಲಾರೆ.

ಹಿಂದುವು ತಾಯಿಯನ್ನು ಪೂಜಿಸುವುದಕ್ಕೆ ಕಾರಣವೇನು? ನಮ್ಮ ತತ್ತ್ವ ಜ್ಞಾನಿಗಳು ಇದಕ್ಕೆ ಒಂದು ಕಾರಣವನ್ನು ಕಂಡುಹಿಡಿಯಲು ಪ್ರಯತ್ನಿಸಿ ಅವರು ಈ ನಿರ್ಣಯಕ್ಕೆ ಬಂದರು: ನಾವು ಆರ್ಯರು. ಆರ್ಯ ಎಂದರೇನು? ಯಾರು ಧರ್ಮದ ಮೂಲಕ ಹುಟ್ಟುತ್ತಾನೊ ಅವನು ಆರ್ಯ. ಈ ದೇಶದಲ್ಲಿ ಇದೊಂದು ವಿಚಿತ್ರ ವಿಷಯ ವಾಗಿ ಕಾಣಬಹುದು. ಆದರೆ ವ್ಯಕ್ತಿಯು ಧರ್ಮದ ಮೂಲಕ, ಪ್ರಾರ್ಥನೆಯ ಮೂಲಕವೇ ಹುಟ್ಟಬೇಕು. ನಮ್ಮ ಸ್ಮೃತಿಗ್ರಂಥಗಳನ್ನು ಓದಿದರೆ, ಗರ್ಭಸ್ಥ ಮಗುವಿನ ಮೇಲೆ ಆಗುವ ತಾಯಿಯ ಪ್ರಭಾವದ ಬಗ್ಗೆ ವಿಸ್ತಾರವಾದ ವಿವರಣೆ ಕಂಡುಬರುತ್ತದೆ.

ನಾನು ಹುಟ್ಟುವುದಕ್ಕೆ ಮುಂಚೆ ನನ್ನ ತಾಯಿಯು ಉಪವಾಸ, ಪ್ರಾರ್ಥನೆ ಮುಂತಾದ ನೂರಾರು ವ್ರತಗಳನ್ನು ಆಚರಿಸುತ್ತಿದ್ದುದು ನನಗೆ ಗೊತ್ತು. ನಾನು ಐದು ನಿಮಿಷಗಳು ಕೂಡ ಅವುಗಳನ್ನು ಮಾಡಲಾರೆ. ಅವಳು ಎರಡು ವರ್ಷಗಳು ಅವನ್ನು ಆಚರಿಸಿದಳು. ನನ್ನಲ್ಲಿರುವ ಧಾರ್ಮಿಕ ಸಂಸ್ಕೃತಿಯೆಲ್ಲವೂ ಬಂದಿರುವುದು ಅವಳಿಂದ. ಪ್ರಜ್ಞಾಪೂರ್ವಕವಾಗಿ ನನ್ನನ್ನು ಈ ರೀತಿ ರೂಪಿಸಿದವಳೇ ನನ್ನ ತಾಯಿ. ನನ್ನಲ್ಲಿರುವ ಸತ್ಪ್ರವೃತ್ತಿಯೆಲ್ಲವನ್ನೂ ಯಾಂತ್ರಿಕವಾಗಿಯಲ್ಲ -ಪ್ರಜ್ಞಾಪೂರ್ವಕವಾಗಿ ತಾಯಿಯೇ ನನಗೆ ನೀಡಿರುವುದು.

“ಭೌತಿಕವಾಗಿ ಹುಟ್ಟಿದ ಮಗುವು ಆರ್ಯನಲ್ಲ; ಆಧ್ಯಾತ್ಮಿಕವಾಗಿ ಹುಟ್ಟಿದ ಮಗುವೇ ಆರ್ಯ. ಪವಿತ್ರ ಮಕ್ಕಳನ್ನು ಪಡೆಯುವುದಕ್ಕೆ ತಾಯಿಯು ಅತ್ಯಂತ ಪರಿಶ್ರಮದ ಮೂಲಕ ಪವಿತ್ರಳಾಗಬೇಕು. ಆದ್ದರಿಂದಲೇ ಅವಳು ಹಿಂದೂ ಮಗುವಿನ ಮೇಲೆ ವಿಶೇಷ ಹಕ್ಕನ್ನು ಹೊಂದಿರುತ್ತಾಳೆ. ಆಕೆಯ ಇತರ ಲಕ್ಷಣಗಳೆಲ್ಲ ಬೇರೆ ದೇಶಗಳ ಮಹಿಳೆಯರಂತೆಯೇ ಇವೆ. ಆದರೆ ನಮ್ಮ ಕುಟುಂಬಗಳಲ್ಲಿ ತಾಯಿಯು ಹೆಚ್ಚು ದುಃಖಪಡಬೇಕು.

ತಾಯಿಯು ಕೊನೆಯಲ್ಲಿ ಊಟಮಾಡಬೇಕು. ಭಾರತದಲ್ಲಿ ಪತಿಯು ಪತ್ನಿಯ ಜೊತೆ ಕುಳಿತು ಏಕೆ ಊಟಮಾಡುವುದಿಲ್ಲ ಎಂದು ಈ ದೇಶದಲ್ಲಿ ನನ್ನನ್ನು ಅನೇಕರು ಕೇಳಿರುವರು. ಅವಳು ತುಂಬ ಕೆಳಮಟ್ಟದವಳು ಎಂದು ಅವನು ಹಾಗೆ ಮಾಡುತ್ತಾನೆ ಎಂದು ನೀವು ಭಾವಿಸಿರಬಹುದು. ಆದರೆ ಅದು ಸರಿಯಲ್ಲ. ನಿಮಗೆ ತಿಳಿದಿರುವಂತೆ, ಹಂದಿಯ ಕೂದಲು ತುಂಬ ಅಶುದ್ಧವೆಂದು ಪರಿಗಣಿಸಲಾಗಿದೆ. ಹಿಂದುವು ಅದರಿಂದ ಮಾಡಿದ ಬ್ರಶ್ಶಿನ ಮೂಲಕ ಹಲ್ಲನ್ನು ಉಜ್ಜುವುದಿಲ್ಲ, ಅವನು ಗಿಡದ ನಾರನ್ನು ಉಪ ಯೋಗಿಸುತ್ತಾನೆ. ಒಬ್ಬ ವಿದೇಶೀ ಯಾತ್ರಿಕನು ಹಿಂದೂವೊಬ್ಬನು ಅದರಿಂದ ಹಲ್ಲು ಜ್ಜು ತ್ತಿರುವುದನ್ನು ನೋಡಿ, ಹೀಗೆ ಬರೆಯುತ್ತಾನೆ: “ಹಿಂದೂವು ಬೆಳಗ್ಗೆ ಮುಂಚೆ ಎದ್ದು ಒಂದು ಗಿಡವನ್ನು ಅಗಿದು ತಿನ್ನುತ್ತಾನೆ.” ಹೀಗೆಯೇ ಯಾರೋ ಪತಿ ಪತ್ನಿಯರು ಒಟ್ಟಿಗೆ ಊಟಮಾಡದಿರುವುದನ್ನು ನೋಡಿ ತಮ್ಮದೇ ಆದ ವಿವರಣೆಯನ್ನು ನೀಡಿ ರುವರು. ಈ ಪ್ರಪಂಚದಲ್ಲಿ ವಿವರಣೆಯನ್ನು ನೀಡುವವರು ಬಹಳ ಮಂದಿ ಇದ್ದಾರೆ, ಅವರ ವಿವರಣೆಗಾಗಿಯೇ ಪ್ರಪಂಚ ಕಾಯುತ್ತಿರುವುದೊ ಎಂಬಂತೆ! ಆದರೆ ನಿಜವಾದ ವಸ್ತುಸ್ಥಿತಿಯನ್ನು ಅರಿಯುವವರು ಬಹಳ ಕಡಿಮೆ. ಆದ್ದರಿಂದಲೇ ಮುದ್ರಣದ ಆವಿಷ್ಕ ರಣವು ನಿರ್ದುಷ್ಟ ಭಾಗ್ಯವೆಂದು ನಾನು ಪರಿಗಣಿಸುವುದಿಲ್ಲ. ವಾಸ್ತವಾಂಶವಿದು: ನಿಮ್ಮ ದೇಶದಲ್ಲಿ ಮಹಿಳೆಯರು ಪುರುಷರ ಮುಂದೆ ಕೆಲವು ಕ್ರಿಯೆಗಳನ್ನು ಮಾಡದಿರುವಂತೆ, ನಮ್ಮ ದೇಶದಲ್ಲಿಯೂ ಕೂಡ ಪುರುಷರ ಮುಂದೆ ಅಗಿಯುತ್ತ ಇರುವುದು ಅಸಭ್ಯ ವೆಂದು ಪರಿಗಣಿಸಲಾಗಿದೆ. ಮಹಿಳೆಯು ತನ್ನ ಸಹೋದರರ ಮುಂದೆ ಊಟಮಾಡ ಬಹುದು. ಆದರೆ ಪತಿಯು ಒಳಗೆ ಬಂದರೆ ಅವಳು ಕೂಡಲೆ ಉಣ್ಣುವುದನ್ನು ನಿಲ್ಲಿಸು ತ್ತಾಳೆ, ಪತಿಯೂ ಕೂಡಲೆ ಅಲ್ಲಿಂದ ಹೊರಟುಹೋಗುತ್ತಾನೆ. ನಾವು ಊಟದ ಮೇಜನ್ನು ಉಪಯೋಗಿಸುವುದಿಲ್ಲ, ಆದ್ದರಿಂದ ಪುರುಷನು ತನಗೆ ಹಸಿವಾದಾಗ ಒಳಗೆ ಬಂದು ಊಟ ಮಾಡಿ ಹೊರಟುಹೋಗುತ್ತಾನೆ. ಹಿಂದೂ ಪತಿಯು ತನ್ನ ಪತ್ನಿಯನ್ನು ಭಾರತ ಮಹಿಳೆಯರು ೨೪೧ ತನ್ನೊಡನೆ ಊಟದ ಮೇಜಿನ ಮುಂದೆ ಕುಳಿತುಕೊಳ್ಳಲು ಬಿಡುವುದಿಲ್ಲ ಎಂದು ಭಾವಿಸ ಬೇಡಿ. ಅವನಿಗೆ ಊಟದ ಮೇಜೇ ಇಲ್ಲ.

ಅಡಿಗೆಯು ಸಿದ್ಧವಾದ ಮೇಲೆ ಆಹಾರದ ಮೊದಲ ಭಾಗವು ಅತಿಥಿ ಮತ್ತು ಬಡವರಿಗೆ ಸೇರಿದ್ದು, ಎರಡನೆಯ ಭಾಗವು ಪ್ರಾಣಿಗಳಿಗೆ, ಮೂರನೆಯದು ಮಕ್ಕಳಿಗೆ, ನಾಲ್ಕನೆಯದು ಪತಿಗೆ, ತಾಯಿಗೆ ಕೊನೆಯ ಪಾಲು. ನಾನು ಎಷ್ಟೊ ಬಾರಿ ನನ್ನ ತಾಯಿಯು ದಿನದ ಮೊದಲ ಊಟವನ್ನು ಮಧ್ಯಾಹ್ನ ಎರಡು ಗಂಟೆಗೆ ಮಾಡುವುದನ್ನು ನೋಡಿದ್ದೇನೆ. ನಾವು ಹತ್ತು ಗಂಟೆಗೆ ಊಟಮಾಡಿದರೆ ಅವಳು ಎರಡು ಗಂಟೆಗೆ ಮಾಡು ತ್ತಿದ್ದಳು. ಏಕೆಂದರೆ ಅವಳು ಮಾಡಬೇಕಾದುದು ಎಷ್ಟೋ ಇರುತ್ತಿತ್ತು. ಉದಾಹರಣೆಗೆ ಒಬ್ಬ ಅತಿಥಿಯು ಬರಬಹುದು, ತನಗಾಗಿ ಇರಿಸಿದ ಊಟವಲ್ಲದೆ ಮತ್ತಾವ ಆಹಾರವೂ ಇಲ್ಲದಿರಬಹುದು. ತನ್ನದನ್ನೇ ಅವನಿಗೆ ಬಡಿಸಿ ಮತ್ತೆ ಅವಳು ಅಡಿಗೆ ಮಾಡಬೇಕಾಗು ತ್ತಿತ್ತು. ಅದು ಅವಳ ಜೀವನದ ರೀತಿ, ಅವಳಿಗದೇ ಇಷ್ಟ. ಇದರಿಂದಾಗಿಯೇ ನಾವು ತಾಯಂದಿರನ್ನು ದೇವತೆಗಳೆಂದು ಪೂಜಿಸುತ್ತೇವೆ.

ಕೇವಲ ಮುದ್ದಿಸಿಕೊಳ್ಳುವುದು ಮತ್ತು ಪ್ರೋತ್ಸಾಹಗಳಿಗಿಂತ ಹೆಚ್ಚಾಗಿ ಪೂಜ್ಯತೆ ಯನ್ನು ನೀವು ಬಯಸುತ್ತೀರೆಂದು ನಾನು ಭಾವಿಸುತ್ತೇನೆ. ನೀವು ಮಾನವ ವರ್ಗಕ್ಕೆ ಸೇರಿ ದವರು! ಕೇವಲ ಮುದ್ದಿಸಿಕೊಳ್ಳುವ ನಿಮ್ಮ ಪ್ರವೃತ್ತಿಯನ್ನು ಹಿಂದೂ ಅರ್ಥ ಮಾಡಿ ಕೊಳ್ಳುವುದಿಲ್ಲ. ಆದರೆ ನೀವು “ನಾವು ತಾಯಂದಿರು, ಗೌರವಕ್ಕೆ ಪಾತ್ರರು” ಎಂದು ಸಾರಿದಾಗ ಹಿಂದೂ ನಿಮಗೆ ತಲೆಬಾಗುತ್ತಾನೆ. ಇದು ಹಿಂದೂಗಳು ಬೆಳೆಸಿಕೊಂಡು ಬಂದ ಸ್ವಭಾವ.

ನಮ್ಮ ಸಿದ್ಧಾಂತಕ್ಕೆ ಹಿಂದಿರುಗೋಣ. ಪಾಶ್ಚಿಮಾತ್ಯರು ಕೇವಲ ನೂರು ವರ್ಷಗಳ ಹಿಂದಷ್ಟೇ ಪರಧರ್ಮ ಸಹಿಷ್ಣುತೆಯ ಆವಶ್ಯಕತೆಯನ್ನು ಮನಗಂಡರು. ಆದರೆ ಕೇವಲ ಸಹಿಷ್ಣುತೆ ಸಾಲದು ಬೇರೆ ಧರ್ಮವನ್ನು ಸ್ವೀಕರಿಸಬೇಕು ಎಂಬುದು ಈಗ ನಮಗೆ ಅರ್ಥ ವಾಗಿದೆ. ಇದು ಕಳೆಯುವುದಲ್ಲ, ಕೂಡುವ ಪ್ರಶ್ನೆ. ಎಲ್ಲ ಬೇರೆ ಬೇರೆ ಅಂಶಗಳು ಒಂದು ಗೂಡಿದ ಒಟ್ಟು ಪರಿಣಾಮವೇ ಪೂರ್ಣ ಸತ್ಯ. ಪ್ರತಿಯೊಂದು ಧರ್ಮವೂ ಈ ಸತ್ಯದ ಒಂದೊಂದು ಅಂಶವನ್ನು ಪ್ರತಿನಿಧಿಸುತ್ತವೆ. ಪೂರ್ಣವು ಇವೆಲ್ಲದರ ಒಟ್ಟು ಮೊತ್ತ. ಪ್ರತಿಯೊಂದು ವಿಜ್ಞಾನದಲ್ಲಿಯೂ ಹೀಗೆಯೇ - ಕೂಡುವುದೇ ನಿಯಮ.

ಹಿಂದೂವು ಈ ಅಂಶವನ್ನು ಬೆಳಸಿರುವನು. ಆದರೆ ಈ ಒಂದು ಅಂಶವೇ ಸಾಕೇನು? ತಾಯಿಯಾದ ಹಿಂದೂ ಮಹಿಳೆಯು ಆದರ್ಶ ಪತ್ನಿಯೂ ಆಗಿರಲಿ, ಆದರೆ ಮಾತೃತ್ವವನ್ನು ಕಳೆದುಕೊಳ್ಳಬಾರದು. ಇದೇ ನೀವು ಮಾಡಬೇಕಾದ ಉತ್ತಮ ವಿಧಾನ. ಹೀಗೆ ನೀವು ಜಗತ್ತಿನ ಬಗ್ಗೆ ಉತ್ತಮ ದೃಷ್ಟಿಕೋಣವನ್ನು ಪಡೆಯುತ್ತೀರಿ. ನೀವು ಜಗತ್ತನ್ನೆಲ್ಲ ಸುತ್ತಾಡಿ, ಬೇರೆ ಬೇರೆ ದೇಶಗಳಲ್ಲಿ ಹಾರಾಡಿ, ಅವರನ್ನೆಲ್ಲ ನಿಂದಿಸುತ್ತ “ಎಲ್ಲ ಭಯಂಕರ ನೀಚರು, ಅವರನ್ನೆಲ್ಲ ಕತ್ತರಿಸಿ ಬೇಯಿಸಬೇಕು” ಎಂದು ವಿಷಕಾರುವುದನ್ನು ನಿಲ್ಲಿಸುತ್ತೀರಿ.

ಭಗವಂತನ ಇಚ್ಛೆಯಿಂದ ಪ್ರತಿಯೊಂದು ದೇಶವೂ ಮಾನವ ಸ್ವಭಾವದ ಒಂದು ಅಂಶವನ್ನು ಬೆಳೆಸಿಕೊಂಡು ಬರುತ್ತಿದೆ ಎಂದು ನಾವು ಭಾವಿಸುವುದಾದರೆ ಯಾವ ದೇಶವೂ ವಿಫಲವಾಗಿಲ್ಲ ಎಂಬುದು ಕಂಡುಬರುತ್ತದೆ. ಇಷ್ಟರವರೆಗೂ ಅವರು ಚೆನ್ನಾಗಿಯೇ ಮಾಡಿರುವರು, ಈಗ ಇನ್ನೂ ಉತ್ತಮವಾಗಿ ಮಾಡಲಿ (ಚಪ್ಪಾಳೆ).

ಹಿಂದೂಗಳನ್ನು ‘ಧರ್ಮಭ್ರಷ್ಟರು’, ‘ನೀಚರು’, ‘ಗುಲಾಮರು’ ಎಂದು ಕರೆಯುವುದನ್ನು ಬಿಟ್ಟು, ಭಾರತಕ್ಕೆ ಹೋಗಿ, “ಇಲ್ಲಿಯವರೆಗೂ ನಿಮ್ಮ ಕಾರ್ಯ ಅದ್ಭುತವಾದುದು. ಆದರೆ ಅಷ್ಟೆ ಸಾಲದು. ನೀವು ಇನ್ನೂ ಮಾಡಬೇಕಾದುದು ಬಹಳ ಇದೆ. ಮಹಿಳೆಯ ಮಾತೃತ್ವದ ಶೀಲವನ್ನು ನೀವು ಬೆಳೆಸಿರುವುದಕ್ಕೆ ದೇವರು ನಿಮ್ಮನ್ನು ಆಶೀರ್ವದಿಸಲಿ. ಈಗ ಪತ್ನಿಯ ಆದರ್ಶವನ್ನು ಬೆಳಗಿಸಿ”, ಎಂದು ಅವರಿಗೆ ಹೇಳಿ.

ಹಾಗೆಯೇ, ನಿಮ್ಮ ರಾಷ್ಟ್ರೀಯ ಶೀಲಕ್ಕೆ ಹಿಂದೂ ಸ್ವಭಾವದ ಮಾತೃಭಾವವನ್ನು ಅಳವಡಿಸಿದರೆ ಒಳ್ಳೆಯದಾಗುವುದೆಂದು ನಾನು ಭಾವಿಸುತ್ತೇನೆ. ನಿಮ್ಮ ಹಿತದೃಷ್ಟಿಯಿಂದ ನಾನಿದನ್ನು ಹೇಳುತ್ತಿದ್ದೇನೆ. ನಾನು ಮೊದಲನೆಯ ದಿನ ಶಾಲೆಗೆ ಹೋದಾಗ ನನಗೆ ಕಲಿಸಿದ ಮೊದಲನೆಯ ಶ್ಲೋಕವಿದು: “ಯಾರು ಎಲ್ಲ ಮಹಿಳೆಯ ರನ್ನೂ ತನ್ನ ತಾಯಿ ಎಂದು ಭಾವಿಸುತ್ತಾನೊ, ಇತರರ ಆಸ್ತಿಯನ್ನು ಮಣ್ಣಿನ ಸಮಾನ ವೆಂದು ಪರಿಗಣಿಸುತ್ತಾನೊ, ಪ್ರತಿಯೊಂದು ಜೀವಿಯನ್ನೂ ತನ್ನಾತ್ಮವೆಂದು ತಿಳಿಯು ತ್ತಾನೊ ಅವನೇ ನಿಜವಾದ ಪಂಡಿತನು.”

ಪುರುಷನೊಡನೆ ಕೆಲಸಮಾಡುವ ಮಹಿಳೆಯ ಇನ್ನೊಂದು ಶೀಲವಿದೆ. ಹಿಂದೂ ಗಳಲ್ಲಿ ಈ ಆದರ್ಶವಿರಲಿಲ್ಲವೆಂದಲ್ಲ, ಆದರೆ ಅವರಿಗೆ ಅದನ್ನು ಬೆಳಸಲಾಗಲಿಲ್ಲ.

ಸಂಸ್ಕೃತ ಭಾಷೆಯೊಂದರಲ್ಲೇ ಪತಿ ಮತ್ತು ಪತ್ನಿ ಎಂಬ ಎರಡೂ ಅರ್ಥಗಳನ್ನು ಕೊಡುವ ನಾಲ್ಕು ಶಬ್ದಗಳಿರುವುವು. ನಮ್ಮಲ್ಲಿ ಮಾತ್ರ ವಿವಾಹದ ಸಂದರ್ಭದಲ್ಲಿ, “ನನ್ನ ಹೃದಯವು ಈಗ ನಿನ್ನದಾಗಲಿ” ಎಂದು ವಧುವರರಿಬ್ಬರೂ ಪ್ರತಿಜ್ಞೆ ಮಾಡುತ್ತಾರೆ. ಪತಿಯು ಧ್ರುವ ನಕ್ಷತ್ರವನ್ನು ನೋಡುತ್ತ, ತನ್ನ ಪತ್ನಿಯ ಕೈಹಿಡಿದುಕೊಂಡು, “ಧ್ರುವ ನಕ್ಷತ್ರವು ಆಕಾಶದಲ್ಲಿ ಸ್ಥಿರವಾಗಿರುವಂತೆ ನನ್ನ ಪ್ರೇಮವೂ ನಿನ್ನಲ್ಲಿ ಸ್ಥಿರವಾಗಿರಲಿ” ಎಂದು ಹೇಳುತ್ತಾನೆ. ಹಾಗೆಯೇ ಪತ್ನಿಯೂ ಹೇಳುತ್ತಾಳೆ.

ಭ್ರಷ್ಟ ಮಹಿಳೆಯೂ ಕೂಡ ತನ್ನ ಜೀವನಾಂಶವನ್ನು ಪಡೆಯಲು ಗಂಡನ ವಿರುದ್ಧ ಕೋರ್ಟಿಗೆ ಹೋಗಬಹುದು. ಈ ಭಾವನೆಗಳು ದೇಶಾದ್ಯಂತ ನಮ್ಮ ಗ್ರಂಥಗಳಲ್ಲಿ ರುವುದನ್ನು ಕಾಣಬಹುದು. ಆದರೆ ಶೀಲದ ಈ ಅಂಶವನ್ನು ಬೆಳಸುವಲ್ಲಿ ನಾವು ಸಮರ್ಥರಾಗಲಿಲ್ಲ.

ನಾವು ಇತರರ ಬಗ್ಗೆ ನಮ್ಮ ಅಭಿಪ್ರಾಯವನ್ನು ವ್ಯಕ್ತಪಡಿಸುವಾಗ ಭಾವುಕತೆಯಿಂದ ತುಂಬ ದೂರವಿರಬೇಕು. ಭಾವಾವೇಶವೊಂದೇ ಜಗತ್ತನ್ನು ಆಳುತ್ತಿರುವುದಲ್ಲ, ಅದನ್ನೂ ಮೀರಿದುದು ಇದೆ. ಆರ್ಥಿಕ ಕಾರಣಗಳು, ಸುತ್ತಲಿನ ಪರಿಸರ ಮತ್ತು ಇತರ ಅಂಶಗಳು ರಾಷ್ಟ್ರಗಳ ಬೆಳವಣಿಗೆಯಲ್ಲಿ ಬಹುಮುಖ್ಯ ಪಾತ್ರ ವಹಿಸುತ್ತವೆ. (ಮಹಿಳೆ ಯನ್ನು ಪತ್ನಿಯಾಗಿ ಚಿತ್ರಿಸುವುದು ಈಗ ನನ್ನ ಉದ್ದೇಶವಲ್ಲ).

ಪ್ರತಿಯೊಂದು ರಾಷ್ಟ್ರವೂ ಒಂದು ವಿಶಿಷ್ಟ ಪರಿಸರವನ್ನು ಪಡೆದಿದ್ದು, ಅದಕ್ಕನು ಗುಣವಾದ ತನ್ನದೇ ಲಕ್ಷಣವನ್ನು ಬೆಳಸಿಕೊಂಡಿದೆ. ಆದರೆ ಈ ಲಕ್ಷಣಗಳೂ ಮಿಶ್ರಣ ವಾಗುವ ಕಾಲವೊಂದು ಬರುತ್ತದೆ - ‘ಪ್ರತಿಯೊಬ್ಬರನ್ನೂ ದೋಚಿ ನನಗೆ ಕೊಡಿ’ ಎಂಬ ಕ್ರೂರ ದೇಶಭಕ್ತಿಯು ಕಣ್ಮರೆಯಾಗುತ್ತದೆ. ಆಗ ಇಡೀ ಪ್ರಪಂಚದಲ್ಲಿ ಏಕಮುಖ ಬೆಳವಣಿಗೆ ಇರುವುದಿಲ್ಲ, ಪ್ರತಿಯೊಂದು ರಾಷ್ಟ್ರವೂ ತನ್ನ ಕಾರ್ಯದ ಸಾರ್ಥಕತೆಯನ್ನು ಮನಗಾಣುತ್ತದೆ.

ಎಲ್ಲ ರಾಷ್ಟ್ರಗಳನ್ನು ಒಂದುಗೂಡಿಸುವ ಕಾರ್ಯ ಮಾಡೋಣ ಮತ್ತು ಒಂದು ಹೊಸ ರಾಷ್ಟ್ರವು ಮೈತಳೆಯಲಿ.

ನನ್ನ ದೃಢ ನಂಬಿಕೆ ಏನೆಂಬುದನ್ನು ಹೇಳುತ್ತೇನೆ. ಇಂದಿನ ಜಗತ್ತಿನಲ್ಲಿ ಕಾಣುವ ನಾಗರಿಕತೆಯ ಬಹುಭಾಗ ಆ ಒಂದು ವಿಶಿಷ್ಟ ಜನಾಂಗದಿಂದ ಬಂದಿದೆ - ಅದೇ ಆರ್ಯಜನಾಂಗ.

ಆರ್ಯ ನಾಗರಿಕತೆಯಲ್ಲಿ ಮೂರು ಬಗೆಗಳಿವೆ: ರೋಮನ್, ಗ್ರೀಕ್ ಮತ್ತು ಹಿಂದು. ರೋಮನ್ ನಾಗರಿಕತೆಯ ಲಕ್ಷಣ ಸಂಸ್ಥೆಯನ್ನು ಕಟ್ಟುವುದು, ಪರರಾಷ್ಟ್ರವನ್ನು ಗೆಲ್ಲುವುದು, ಮತ್ತು ಸ್ಥಿರತೆ. ಆದರೆ ಅದರಲ್ಲಿ ಭಾವುಕತೆ, ಸೌಂದರ್ಯಪ್ರಜ್ಞೆ ಮತ್ತು ಉನ್ನತ ಭಾವ ಇವುಗಳ ಕೊರತೆಯಿದೆ. ಅದರ ಒಂದು ದೋಷ ಕ್ರೌರ್ಯ. ಗ್ರೀಕ್ ಸಂಸ್ಕೃತಿಯ ಮುಖ್ಯ ಲಕ್ಷಣ ಸೌಂದರ್ಯೋಪಾಸನೆ. ಆದರೆ ಅದರಲ್ಲಿ ಗಾಂಭೀರ್ಯವಿಲ್ಲ, ಬೇಗ ಅನೈತಿಕವಾಗುವ ಪ್ರವೃತ್ತಿ ಬಹಳ. ಹಿಂದೂ ಸಂಸ್ಕೃತಿಯು ಮುಖ್ಯವಾಗಿ ಭೌತಾತೀತವಾದುದು, ಧಾರ್ಮಿಕವಾದುದು, ಆದರೆ ಸಂಘಟನೆ ಮತ್ತು ಚಟುವಟಿಕೆಗಳ ಕೊರತೆಯಿದೆ.

ಈಗ ಆಂಗ್ಲೋ - ಸ್ಯಾಕ್ಸನರು ರೋಮನ್ ಸಂಸ್ಕೃತಿಯನ್ನು ಪ್ರತಿನಿಧಿಸುತ್ತಾರೆ; ಗ್ರೀಕ್ ಸಂಸ್ಕೃತಿಯನ್ನು ಎಲ್ಲ ದೇಶಗಳಿಗಿಂತ ಹೆಚ್ಚಾಗಿ ಫ್ರೆಂಚರು ಪ್ರತಿನಿಧಿಸುತ್ತಾರೆ; ಹಳೆಯ ಹಿಂದೂಗಳಾದರೊ ಸಾಯುವುದೇ ಇಲ್ಲ! ಈ ಭರವಸೆಯ ಹೊಸರಾಜ್ಯದಲ್ಲಿ ಪ್ರತಿಯೊಂದು ಸಂಸ್ಕೃತಿಗೂ ಅವಕಾಶವಿದೆ. ಅಲ್ಲಿ ರೋಮನರ ಸಂಘಟನೆ ಇರುತ್ತದೆ, ಗ್ರೀಕರ ಸೌಂದರ್ಯಪ್ರಜ್ಞೆ ಇರುತ್ತದೆ, ಹಿಂದೂಗಳ ಬೆನ್ನೆಲುಬಾದ ಧರ್ಮ ಮತ್ತು ಭಗವದ್ಭ ಕ್ತಿಗಳಿರುತ್ತವೆ. ಈ ಮೂರನ್ನೂ ಬೆರಸಿ ಹೊಸ ನಾಗರೀಕತೆಯನ್ನು ಸೃಷ್ಟಿಸಿ.

ಇದು ಮಹಿಳೆಯರಿಂದ ಆಗಬೇಕಾಗಿದೆ. ಮುಂದಿನ ಅವತಾರವು ಸ್ತ್ರೀರೂಪದಲ್ಲಿರುತ್ತದೆ ಎಂದು ನಮ್ಮ ಕೆಲವು ಗ್ರಂಥಗಳು ಹೇಳುತ್ತವೆ.

ಜಗತ್ತಿನ ಸಂಪನ್ಮೂಲಗಳು ಇನ್ನೂ ಸಂಪೂರ್ಣ ಹೊರಬಂದಿಲ್ಲ, ಏಕೆಂದರೆ ಜಗತ್ತಿ ನಲ್ಲಿರುವ ಎಲ್ಲ ಶಕ್ತಿಗಳು ಇನ್ನೂ ಬಳಕೆಗೆ ಬಂದಿಲ್ಲ. ಇದುವರೆಗೂ ಕೈ ಮಾತ್ರ ಕೆಲಸ ಮಾಡುತ್ತಿದ್ದು ದೇಹದ ಉಳಿದ ಭಾಗಗಳು ಸ್ತಬ್ಧವಾಗಿ ಉಳಿದಿವೆ. ದೇಹದ ಉಳಿದ ಭಾಗಗಳನ್ನೂ ಜಾಗ್ರತಗೊಳಿಸಬೇಕು. ಸಮನ್ವಯಾತ್ಮಕ ಕ್ರಿಯೆಯ ಮೂಲಕ ಬಹುಶಃ ಎಲ್ಲ ದುಃಖವನ್ನೂ ಶಮನಗೊಳಿಸಬಹುದು. ಈ ಹೊಸ ಸಾಮ್ರಾಜ್ಯದಲ್ಲಿ, ನಿಮ್ಮ ಧಮನಿಯಲ್ಲಿ ಹರಿಯುತ್ತಿರುವ ಹೊಸ ರಕ್ತದ ಮೂಲಕ, ಅಮೆರಿಕದ ಮಹಿಳೆಯ ರಾದ ನೀವು ಹೊಸನಾಗರಿಕತೆಯನ್ನು ಅಸ್ತಿತ್ವಕ್ಕೆ ತರಬಹುದು.

ನನಗೆ ದೇಹವನ್ನು ನೀಡಿರುವ ಆ ಪುಣ್ಯ ಭೂಮಿಯನ್ನು ಅತ್ಯಂತ ಭಕ್ತಿಭಾವ ದಿಂದ ಕಾಣುತ್ತೇನೆ. ಭೂಮಿಯ ಮೇಲಿನ ಆ ಪರಮ ಪವಿತ್ರ ಸ್ಥಳದಲ್ಲಿ ಜನ್ಮವೆತ್ತು ವುದಕ್ಕೆ ಅವಕಾಶವನ್ನು ನೀಡಿದ ಆ ದಯಾಮಯನಿಗೆ ನಾನು ಋಣಿಯಾಗಿದ್ದೇನೆ. ಇಡೀ ಪ್ರಪಂಚವು, ಯೋಧರು ಅಥವಾ ಶ‍್ರೀಮಂತರು ತಮ್ಮ ಪೂಜ್ಯರೆಂದು ಗುರುತಿಸಲು ಪ್ರಯತ್ನಿಸುತ್ತಿದ್ದರೆ, ಹಿಂದೂಗಳು ಮಾತ್ರ ತಾವು ಋಷಿಸಂತಾನರೆಂದು ಹೆಮ್ಮೆಪಡುತ್ತಾರೆ.

ಯುಗಯುಗಾಂತರಗಳಿಂದ ಸ್ತ್ರೀಪುರುಷರನ್ನು ಭವಸಾಗರದಾಚೆಗೆ ಕರೆದು ಕೊಂಡು ಹೋಗುತ್ತಿರುವ ಆ ಅದ್ಭುತ ಹಡಗಿನಲ್ಲಿ ಈಗ ಅಲ್ಲಲ್ಲಿ ರಂಧ್ರಗಳು ಕಾಣಿಸಿ ಕೊಳ್ಳುತ್ತಿರಬಹುದು. ಇದಕ್ಕೆ ಹಿಂದೂಗಳೇ ಎಷ್ಟರ ಮಟ್ಟಿಗೆ ಕಾರಣ, ಅವರನ್ನು ತುಚ್ಛ ರೀತಿಯಿಂದ ಕಾಣುವವರು ಎಷ್ಟರಮಟ್ಟಿಗೆ ಕಾರಣ ಎಂಬುದು ದೇವರೊಬ್ಬನಿಗೆ ಗೊತ್ತು.

ಅಂಥ ರಂಧ್ರಗಳಿದ್ದರೆ, ಅತ್ಯಂತ ಸಾಮಾನ್ಯರಲ್ಲಿ ಒಬ್ಬನಾದ ನನ್ನ ಕರ್ತವ್ಯ ನನ್ನ ಜೀವನವನ್ನು ತೆತ್ತಾದರೂ ಆ ರಂಧ್ರಗಳನ್ನು ಮುಚ್ಚುವುದು. ನನ್ನ ಎಲ್ಲ ಹೋರಾಟವು ವ್ಯರ್ಥವಾದರೂ ಅವರಿಗೆ ನನ್ನ ಹೃತ್ಪೂರ್ವಕ ಆಶೀರ್ವಚನಗಳಿವು: “ನನ್ನ ಸಹೋದ ರರೇ, ನೀವು ಚೆನ್ನಾಗಿಯೇ ಮಾಡಿದ್ದೀರಿ - ಅಷ್ಟೇ ಅಲ್ಲ, ಇದೇ ಪರಿಸ್ಥಿತಿಯಲ್ಲಿ ಬೇರೆ ಯಾವ ಜನಾಂಗಕ್ಕಿಂತಲೂ ಚೆನ್ನಾಗಿಯೇ ಮಾಡಿದ್ದೀರಿ. ನಾನು ಪಡೆದಿರುವುದನ್ನೆಲ್ಲ ನೀಡಿರುವವರು ನೀವು. ಕೊನೆಯವರೆಗೂ ನಿಮ್ಮೊಂದಿಗಿರುವ ಭಾಗ್ಯವನ್ನು ನನಗೆ ಕೊಡಿ. ನಾವೆಲ್ಲರೂ ಒಟ್ಟಿಗೆ ಮುಳುಗೋಣ.”

