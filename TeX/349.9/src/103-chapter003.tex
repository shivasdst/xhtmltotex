
\chapter{ಭಕ್ತಿಯೋಗ}

(೧೮೯೬, ಜನವರಿ ೨೦ರಂದು ನ್ಯೂಯಾರ್ಕ್ನಲ್ಲಿ ನೀಡಿದ ತರಗತಿಯ ಉಪನ್ಯಾಸ - ಜೆ.ಜೆ. ಗುಡ್ವಿನ್ ಅವರು ಬರೆದುಕೊಂಡದ್ದು).

(ಕಳೆದ ತರಗತಿಯಲ್ಲಿ ಪ್ರತೀಕದ ವಿಷಯವಾಗಿ ಹೇಳಿದ್ದಾಯಿತು. ವೈಧೀ ಭಕ್ತಿಗೆ ಸಂಬಂಧಿಸಿದ ಇನ್ನೊಂದು ಭಾವನೆಯನ್ನು ಹೇಳಿ ಪರಾಭಕ್ತಿಯ ವಿಷಯವನ್ನು ತೆಗೆದುಕೊಳ್ಳುತ್ತೇನೆ. ಆ ಭಾವನೆ ಯಾವುದೆಂದರೆ ಏಕನಿಷ್ಠೆ).

ಎಲ್ಲ ಪೂಜಾ ಭಾವನೆಗಳೂ ಸರಿಯಾದವು ಮತ್ತು ಒಳ್ಳೆಯವು ಎಂಬುದು ನಮಗೆ ಗೊತ್ತು. ಭಗವಂತನ ಪೂಜೆಯೇ ಭಕ್ತಿ ಎಂಬುದನ್ನೂ ನಾವು ನೋಡಿದೆವು. ಬೇರೆ ಯಾವ ದೇವತೆಯ ಪೂಜೆಯೂ ಭಕ್ತಿಯಲ್ಲ. ಆದರೆ ಭಗವಂತನನ್ನು ಅನೇಕ ರೂಪಗಳಲ್ಲಿ, ಅನೇಕ ಭಾವನೆಗಳ ಮೂಲಕ ಪೂಜಿಸಬಹುದು. ಈ ಎಲ್ಲ ಭಾವನೆ ಗಳೂ ಸರಿಯಾದವು ಮತ್ತು ಒಳ್ಳೆಯವು. ಆದರೆ ಒಂದು ತೊಂದರೆ ಇದೆ: ಇದನ್ನು ಒಪ್ಪಿಕೊಂಡು ಅನುಸರಿಸಿದ್ದೇ ಆದರೆ ನಮ್ಮ ಶಕ್ತಿಯೆಲ್ಲ ಬೇರೆ ಬೇರೆ ಕಡೆ ಹರಿದು ಹೋಗಿ ಏನನ್ನೂ ಸಾಧಿಸಿದಂತಾಗುವುದಿಲ್ಲ.

ಉದಾರ ಮನಸ್ಸಿನವರ ಒಂದು ಪ್ರವೃತ್ತಿ ಏನೆಂದರೆ ಎಲ್ಲವನ್ನೂ ಅಭ್ಯಾಸ ಮಾಡಲು ಹೋಗಿ ಯಾವುದರಲ್ಲಿಯೂ ಪರಿಣತಿಯನ್ನು ಪಡೆಯದೆ ಇರುವುದು. ಈ ದೇಶದಲ್ಲಿ ಅನೇಕ ವೇಳೆ ಇದೊಂದು ಜಾಡ್ಯವಾಗುತ್ತದೆ - ಎಲ್ಲವನ್ನೂ ಕೇಳುವುದು, ಯಾವುದನ್ನೂ ಮಾಡುವುದಿಲ್ಲ.

ನಮ್ಮ ಪುರಾತನ ಭಕ್ತನ ಬೋಧನೆ ಹೀಗಿದೆ: “ಎಲ್ಲ ಹೂವುಗಳಿಂದಲೂ ಮಕ ರಂದವನ್ನು ಹೀರಿ, ಎಲ್ಲರೊಡನೆಯೂ ಗೌರವದಿಂದ ಬೆರೆಯಿರಿ, ಎಲ್ಲರಿಗೂ ‘ಹೌದು ಹೌದು’ ಎನ್ನಿ. ಆದರೆ ನಿಮ್ಮ ನೆಲೆಯಿಂದ ಮಾತ್ರ ಕದಲ ಬೇಡಿ.” ಈ ರೀತಿ ನಿಮ್ಮ ನೆಲೆಯನ್ನು ಬಿಡದಿರುವುದೇ ನಿಷ್ಠೆ. ಇತರರ ಆದರ್ಶಗಳನ್ನು ದ್ವೇಷಿಸಬೇಕಾಗಿಲ್ಲ, ನಿಂದಿಸಬೇಕಾಗಿಲ್ಲ. ಎಲ್ಲವೂ ಯಥಾರ್ಥವಾದವುಗಳೇ. ಆದರೆ ಅದೇ ಸಂದರ್ಭದಲ್ಲಿ, ನಮ್ಮ ಆದರ್ಶವನ್ನು ಗಟ್ಟಿಯಾಗಿ ಹಿಡಿದುಕೊಂಡಿರಬೇಕು.

ರಾಮಭಕ್ತ ಹನುಮಂತನ ಒಂದು ಕಥೆಯಿದೆ. (ಕ್ರೈಸ್ತರು ಕ್ರಿಸ್ತನನ್ನು ಭಗವದವತಾರ ವೆಂದು ಪೂಜಿಸುವಂತೆ, ಹಿಂದೂಗಳು ಭಗವಂತನ ಅನೇಕ ಅವತಾರಗಳನ್ನು ಪೂಜಿಸು ತ್ತಾರೆ. ಅವರ ಪ್ರಕಾರ, ಭಗವಂತನು ಆಗಲೆ ಒಂಬತ್ತು ಬಾರಿ ಅವತರಿಸಿರುವನು ಮತ್ತು ಇನ್ನೊಮ್ಮೆ ಬರುವವನಿದ್ದಾನೆ. ಅವನು ರಾಮನಾಗಿ ಅವತರಿಸಿದಾಗ ಹನುಮಂತ ಅವನ ಪರಮ ಭಕ್ತನಾಗಿದ್ದನು. ಅವನು ಬಹಳ ಕಾಲ ಬಾಳಿದನು ಮತ್ತು ದೊಡ್ಡ ಯೋಗಿಯಾಗಿದ್ದನು).

ಅವನ ಜೀವನಾವಧಿಯಲ್ಲಿಯೇ ರಾಮನು ಮತ್ತೆ ಕೃಷ್ಣನಾಗಿ ಅವತರಿಸಿದನು. ಯೋಗಿಯಾದ ಹನುಮಂತನಿಗೆ ದೇವರು ಪುನಃ ಶ‍್ರೀಕೃಷ್ಣನಾಗಿ ಅವತರಿಸಿರುವುದು ತಿಳಿದಿತ್ತು. ಅವನು ಬಂದು ಶ‍್ರೀಕೃಷ್ಣನ ಸೇವೆ ಮಾಡಿದನು. ಆದರೆ ಅವನಿಗೆ ಹೇಳಿ ದನು: “ನಾನು ನಿನ್ನ ರಾಮರೂಪವನ್ನು ನೋಡಲು ಇಚ್ಛಿಸುತ್ತೇನೆ.” ಕೃಷ್ಣನು ಹೇಳಿದನು: “ಈ ರೂಪವೇ ಸಾಲದೆ? ನಾನೇ ಈ ಕೃಷ್ಣ, ನಾನೇ ಈ ರಾಮ. ಈ ಎಲ್ಲ ರೂಪಗಳೂ ನನ್ನವೇ”. ಹನುಮಂತನು ಹೇಳಿದ: “ಅದು ನನಗೆ ಗೊತ್ತು. ಆದರೆ ನನಗೆ ರಾಮನ ರೂಪವೇ ಇಷ್ಟ. ಶ‍್ರೀನಾಥ ಮತ್ತು ಜಾನಕೀನಾಥ ಇಬ್ಬರೂ ಒಂದೇ. ಇಬ್ಬರೂ ಪರಮಾತ್ಮನ ಅವತಾರಗಳೇ. ಆದರೂ ಕಮಲಲೋಚನನಾದ ರಾಮನೇ ನನ್ನ ಸರ್ವಸ್ವ”. ಇದು ಏಕನಿಷ್ಠೆ - ವಿವಿಧ ರೂಪದ ಪೂಜೆಗಳು ಸರಿಯೆಂದು ಗೊತ್ತಿದ್ದರೂ, ಒಂದಕ್ಕೇ ನಿಷ್ಠರಾಗಿದ್ದು, ಉಳಿದುದನ್ನು ನಿರಾಕರಿಸುವುದು. ನಾವು ಇತರರನ್ನು ಪೂಜಿಸ ಕೂಡದು. ನಾವು ಅವರನ್ನು ದ್ವೇಷಿಸುವುದಾಗಲಿ ನಿಂದಿಸುವುದಾಗಲಿ ಮಾಡ ಕೂಡದು, ಆದರೆ ಗೌರವಿಸಬೇಕು.

(ಆನೆಗೆ ಎರಡು ಹಲ್ಲುಗಳು ಬಾಯಿಯಿಂದ ಹೊರಚಾಚಿರುತ್ತವೆ. ಇವು ಕೇವಲ ಶೋಭೆಗೆ ಮಾತ್ರ, ಅವುಗಳಿಂದ ತಿನ್ನುವುದಕ್ಕೆ ಆಗುವುದಿಲ್ಲ. ಆದರೆ ಒಳಗಿರುವ ಹಲ್ಲು ಗಳಿಂದ ಅದು ಆಹಾರವನ್ನು ಅಗಿಯುತ್ತದೆ. ಹಾಗೆಯೇ ಎಲ್ಲರೊಡನೆ ಬೆರೆಯಿರಿ, ಎಲ್ಲರ ಭಾವವನ್ನೂ ಸರಿ ಎನ್ನಿ. ಆದರೆ ಅವರ ಮಾರ್ಗವನ್ನು ಅನುಸರಿಸಬೇಡಿ, ನಿಮ್ಮ ಆದರ್ಶವನ್ನು ಬಲವಾಗಿ ಹಿಡಿದುಕೊಳ್ಳಿ, ನಿಮ್ಮ ಇಷ್ಟ ದೇವತೆಯೊಂದನ್ನೇ ಪೂಜಿಸಿ. ನೀವು ಹಾಗೆ ಮಾಡದಿದ್ದರೆ ನೀವು ಏನನ್ನೂ ಪಡೆಯಲಾರಿರಿ, ಪ್ರಗತಿಹೊಂದ ಲಾರಿರಿ).

ಗಿಡದ ಸುತ್ತ, ಪ್ರಾಣಿಗಳು ಅದನ್ನು ತಿನ್ನದಿರಲೆಂದು, ಬೇಲಿ ಕಟ್ಟಬೇಕು. ಆದರೆ ಅದು ಬೆಳೆದು ಹೆಮ್ಮರವಾದ ಮೇಲೆ ಯಾವ ಬೇಲಿಯ ಆವಶ್ಯಕತೆಯೂ ಇಲ್ಲ. ಹಾಗೆಯೇ ಆಧ್ಯಾತ್ಮಿಕತೆಯ ಬೀಜ ಬೆಳೆಯುತ್ತಿರುವಾಗ ಅದನ್ನು ಎಚ್ಚರಿಕೆಯಿಂದ ರಕ್ಷಿಸಬೇಕು. ಎಲ್ಲ ಧಾರ್ಮಿಕ ಭಾವನೆಗಳ ಕಡೆಗೂ ಮನಸ್ಸನ್ನು ಹರಿಸಿ ಶಕ್ತಿಯನ್ನು ವ್ಯಯಗೊಳಿಸಬಾರದು. ಕ್ರೈಸ್ತ, ಬೌದ್ಧ ಮುಂತಾದ ವಿವಿಧ ಆಧ್ಯಾತ್ಮಿಕ ಮಾರ್ಗಗಳನ್ನು ಅನುಸರಿಸಲು ಹೋಗಿ ಕೊನೆಗೆ ಯಾವುದೂ ಇಲ್ಲದಂತಾಗುತ್ತದೆ.

(ಎಲ್ಲವನ್ನೂ ಸ್ವೀಕರಿಸುವುದು ಒಳ್ಳೆಯದು. ಆದರೆ ಅದು ಕೊನೆಯಲ್ಲಿ ಬರ ಬೇಕು. ಕುದುರೆಯ ಮುಂದೆ ಗಾಡಿಯನ್ನು ಇಟ್ಟಂತಾಗಬಾರದು).

ಪ್ರಾರಂಭದ ಹಂತದಲ್ಲಿ ನಾವು ಮತೀಯರಾಗುವುದು ಅನಿವಾರ್ಯ. ಆದರೆ ಆ ಮತೀಯತೆ ಯಾರನ್ನೂ ನಿರಾಕರಿಸುವುದಾಗಿರಬಾರದು. ನಮ್ಮಲ್ಲಿ ಪ್ರತಿ ಯೊಬ್ಬರಿಗೂ ಒಂದೊಂದು ಮತವಿರಬೇಕು. ಆ ಮತವೇ ನಮ್ಮ ಇಷ್ಟ - ನಮ್ಮ ಆಯ್ಕೆಯ ಮಾರ್ಗ. ಆದರೆ ಆ ಆದರ್ಶವನ್ನು ಹಿಡಿದುಕೊಳ್ಳುವುದಕ್ಕಾಗಿ ನಾವು ಇತರರನ್ನು ನಾಶಮಾಡಬಾರದು. ನನ್ನ ಇಷ್ಟ ಪವಿತ್ರವಾದುದು - ಅದನ್ನು ನನ್ನ ಸಹೋದರನಿಗೂ ತಿಳಿಸುವಂತಿಲ್ಲ. ಪ್ರತಿಯೊಬ್ಬರಿಗೂ ಅವರವರ ಇಷ್ಟ ಪವಿತ್ರ. ಆದ್ದರಿಂದ ಆ ಇಷ್ಟವನ್ನು ನಿಮ್ಮದಾಗಿಯೇ ಇಟ್ಟುಕೊಳ್ಳಿ. ಪ್ರತಿಯೊಬ್ಬ ಸಾಧಕನು ಈ ಭಾವವನ್ನು ಅನುಸರಿಸಬೇಕು. ನಿಮ್ಮ ಇಷ್ಟ ದೇವತೆಯನ್ನು ಪ್ರಾರ್ಥಿಸುವಾಗ ಆ ದೇವತೆ ಯೊಂದೇ ನಿಮ್ಮ ಸರ್ವಸ್ವವಾಗಬೇಕು. ಭಗವಂತನ ಅನೇಕ ರೂಪಗಳಿರಬಹುದು, ಆದರೆ ಆ ಸಂದರ್ಭದಲ್ಲಿ ನಿಮ್ಮ ಇಷ್ಟ ದೇವತೆಯೊಂದೇ ನಿಮ್ಮ ಪಾಲಿಗಿರುವುದು.

ಈ ಇಷ್ಟಸಾಧನೆಯಲ್ಲಿ ತುಂಬ ಮುಂದುವರಿದ ಮೇಲೆ - ಆಧ್ಯಾತ್ಮಿಕ ಸಸಿಯು ಸಾಕಷ್ಟು ಬೆಳೆದ ಮೇಲೆ, ನಿಮ್ಮ ಅಂತರಂಗವು ಬಲಿಷ್ಠವಾದ ಮೇಲೆ, ಮತ್ತು ನಿಮ್ಮ ಇಷ್ಟ ದೇವತೆಯೇ ಎಲ್ಲೆಲ್ಲಿಯೂ ಇರುವುದೆಂದು ಅರಿತ ಮೇಲೆ - ಸ್ವಾಭಾವಿಕವಾಗಿಯೇ ಈ ಎಲ್ಲ ಬಂಧನಗಳೂ ಕಳಚಿ ಬೀಳುವುವು. ಹಣ್ಣು ಗಳಿತ ಮೇಲೆ ತನ್ನ ಭಾರದಿಂದಲೇ ಕೆಳಕ್ಕೆ ಬೀಳುತ್ತದೆ. ನೀವು ಅಪಕ್ವವಾದ ಹಣ್ಣನ್ನು ಕಿತ್ತರೆ ಅದು ಹುಳಿಯಾಗಿರುತ್ತದೆ. ಆದ್ದರಿಂದ ನಾವು ಈ ಆಲೋಚನೆಯಲ್ಲಿ ಬೆಳೆಯ ಬೇಕು.

(ಕೇವಲ ಉಪನ್ಯಾಸಗಳನ್ನು ಕೇಳುವುದು ಮುಂತಾದವುಗಳಿಂದ ಯಾವ ಪ್ರಯೋಜನವೂ ಇಲ್ಲ. ಇದರಿಂದ ಮಿದುಳು ಒಂದು ಯುದ್ಧರಂಗವಾಗುತ್ತದೆ ಅಷ್ಟೆ. ಅರಗಿಸಿಕೊಳ್ಳದ ಭಾವನೆಗಳಿಂದ ಯಾವ ಪ್ರಯೋಜನವೂ ಇಲ್ಲ. ಒಂದೇ ಭಾವನೆ ಯಲ್ಲಿ ನಿಷ್ಠೆಯುಳ್ಳವರು ಆಧ್ಯಾತ್ಮಿಕರಾಗುತ್ತಾರೆ, ಬೆಳಕನ್ನು ಕಾಣುತ್ತಾರೆ. ಪ್ರತಿಯೊಬ್ಬರೂ ಹೀಗೆ ದೂರುತ್ತಾರೆ: “ನಾನು ಈ ಸಾಧನೆ ಮಾಡಿದೆ, ಆ ಸಾಧನೆ ಮಾಡಿದೆ. ಯಾವ ಪ್ರಯೋಜನವೂ ಆಗಲಿಲ್ಲ” ಎಂದು. ಅವರನ್ನು ಸರಿಯಾಗಿ ಪ್ರಶ್ನಿಸಿದಾಗ ತಿಳಿದು ಬರುತ್ತದೆ, ಅವರು ಅಲ್ಲಿ ಇಲ್ಲಿ ಕೆಲವು ಉಪನ್ಯಾಸಗಳನ್ನು ಕೇಳಿರುತ್ತಾರೆ, ಅವರಿವರೊಡನೆ ಸ್ವಲ್ಪ ಮಾತನಾಡಿರುತ್ತಾರೆ ಅಷ್ಟೆ ಎಂದು. ಮೂರು ಗಂಟೆ ಅಥವಾ ಕೆಲವು ದಿವಸ ಸ್ವಲ್ಪ ಸಾಧನೆ ಮಾಡಿ ಅಷ್ಟೇ ಸಾಕೆಂದು ಭಾವಿಸುತ್ತಾರೆ. ಇದು ಮೂರ್ಖತನ, ಇದರಿಂದ ಪರಿಪೂರ್ಣತೆಯನ್ನು ಪಡೆಯಲಾಗುವುದಿಲ್ಲ - ಇದು ಆಧ್ಯಾತ್ಮಿಕತೆಯನ್ನು ಪಡೆ ಯುವ ಮಾರ್ಗವಲ್ಲ).

ಒಂದು ಭಾವನೆಯನ್ನು ತೆಗೆದುಕೊಳ್ಳಿ, ಒಂದು ಇಷ್ಟವನ್ನು ಸ್ವೀಕರಿಸಿ. ನಿಮ್ಮ ಇಡೀ ವ್ಯಕ್ತಿತ್ವವನ್ನು ಅದಕ್ಕೆ ಅರ್ಪಿಸಿ. ಪ್ರತಿಫಲ ದೊರೆಯುವವರೆಗೂ, ನಿಮ್ಮಾತ್ಮ ಬೆಳೆಯುವವರೆಗೂ ದಿನದಿನವೂ ಅದನ್ನೇ ಅಭ್ಯಾಸ ಮಾಡುತ್ತ ಹೋಗಿ. ಸಾಧನೆ ಪ್ರಾಮಾಣಿಕವಾಗಿದ್ದರೆ, ಸರಿಯಾಗಿದ್ದರೆ, ಆ ಇಷ್ಟವೇ ವಿಶಾಲವಾಗುತ್ತ ಹೋಗಿ ಇಡೀ ವಿಶ್ವವ್ಯಾಪಿಯಾಗುತ್ತದೆ. ಅದು ತಾನಾಗಿಯೇ ಬೆಳೆಯಲಿ, ಅದು ಒಳಗಿನಿಂದ ಹೊರಗೆ ಬೆಳೆಯುತ್ತದೆ. ಆಗ ನಿಮ್ಮ ಇಷ್ಟವೇ ಎಲ್ಲೆಲ್ಲಿಯೂ ಇದೆ, ಅದೇ ಎಲ್ಲವೂ ಆಗಿದೆ ಎಂದು ಹೇಳುತ್ತೀರಿ.

(ಆದರೆ ಇದೇ ಸಂದರ್ಭದಲ್ಲಿ ಇತರರ ಇಷ್ಟವನ್ನು - ಭಗವಂತನ ಇತರ ಭಾವನೆ ಗಳನ್ನು - ನಾವು ಗೌರವಿಸಬೇಕು. ಇಲ್ಲದೆ ಇದ್ದರೆ ನಮ್ಮ ಪೂಜೆಯು ಮತಾಂಧತೆಯಾಗಿ ಪರಿಣಮಿಸುತ್ತದೆ. ಶಿವಭಕ್ತನ ಪುರಾತನ ಕಥೆಯೊಂದಿದೆ. ಭಾರತದಲ್ಲಿ ಶಿವನನ್ನು ಆರಾಧಿಸುವ ಮತ್ತು ವಿಷ್ಣುವನ್ನು ಆರಾಧಿಸುವ ಮತಗಳು ಇವೆ. ಈ ವ್ಯಕ್ತಿಯು ದೊಡ್ಡ ಶಿವಭಕ್ತನಾಗಿದ್ದ. ಅಷ್ಟೇ ಅಲ್ಲದೆ ವಿಷ್ಣುವಿನ ಭಕ್ತರನ್ನು ತುಂಬ ದ್ವೇಷಿಸುತ್ತಿದ್ದ ಮತ್ತು ವಿಷ್ಣುವಿನ ಹೆಸರನ್ನು ಸಹ ಕೇಳುತ್ತಿರಲಿಲ್ಲ. ಭಾರತದಲ್ಲಿ ವಿಷ್ಣುವಿನ ಆರಾಧಕರು ಬಹಳ ಸಂಖ್ಯೆಯಲ್ಲಿ ಇದ್ದಾರೆ, ಆದ್ದರಿಂದ ಆ ಹೆಸರನ್ನು ಕೇಳದೆ ಇರುವುದು ಸಾಧ್ಯವೇ ಇರಲಿಲ್ಲ. ಅವನು ತನ್ನ ಕಿವಿಗೆ ಎರಡು ಘಂಟೆಗಳನ್ನು ಕಟ್ಟಿಕೊಂಡನು. ಯಾರಾದರೂ ವಿಷ್ಣುವಿನ ಹೆಸರನ್ನು ಪ್ರಸ್ತಾಪಿಸಿದರೆ ಸಾಕು ತನ್ನ ತಲೆಯನ್ನು ಅಲ್ಲಾಡಿಸಿ, ಘಂಟೆ ಬಾರಿಸಿ ಆ ಹೆಸರು ತನ್ನ ಕಿವಿಗೆ ಬೀಳದಂತೆ ಮಾಡುತ್ತಿದ್ದ).

(ಆದರೆ ಶಿವನು ಅವನ ಕನಸಿನಲ್ಲಿ ಕಾಣಿಸಿಕೊಂಡು ಹೇಳಿದ, “ಎಂಥ ಮೂರ್ಖ ನೀನು! ನಾನೇ ವಿಷ್ಣು ನಾನೇ ಶಿವ, ಕೇವಲ ನಾಮ ಭೇದ ಅಷ್ಟೇ. ಇಬ್ಬರು ದೇವರಿಲ್ಲ”, ಎಂದು. ಈ ವ್ಯಕ್ತಿ ಹೇಳಿದ, “ನನಗದೆಲ್ಲ ಬೇಕಿಲ್ಲ, ಈ ವಿಷ್ಣುವಿನ ಸಹವಾಸವೇ ನನಗೆ ಬೇಡ”, ಎಂದು).

(ಅವನ ಹತ್ತಿರ ಶಿವನ ಸುಂದರವಾದ ಚಿಕ್ಕ ವಿಗ್ರಹವಿತ್ತು. ಅದನ್ನು ಪೀಠದ ಮೇಲಿರಿಸಿ ಪೂಜಿಸುತ್ತಿದ್ದ. ಒಂದು ದಿನ ತುಂಬ ಸುವಾಸನೆಯುಳ್ಳ ಧೂಪವನ್ನು ಕೊಂಡು ಕೊಂಡು ಮನೆಗೆ ಹೋಗಿ ಅದನ್ನು ತನ್ನ ದೇವರ ಮುಂದೆ ಉರಿಸಿದನು. ಅದರ ಹೊಗೆ ಮೇಲೇರುತ್ತಿರುವಾಗ, ಅವನ ವಿಗ್ರಹ ಇಬ್ಭಾಗವಾಗಿ ಒಂದು ಭಾಗ ಶಿವನೂ ಇನ್ನೊಂದು ಭಾಗ ವಿಷ್ಣುವೂ ಆಗಿರುವುದನ್ನು ಕಂಡನು. ಅವನು ತಕ್ಷಣ ಮೇಲೆದ್ದು ಆ ಧೂಪದ ವಾಸನೆ ವಿಷ್ಣುವಿನ ಮೂಗಿನೊಳಗೆ ಹೋಗದಂತೆ ಅದನ್ನು ಒತ್ತಿ ಹಿಡಿದನು).

(ಶಿವನು ಕೋಪಿಸಿಕೊಂಡು ‘ನೀನು ರಾಕ್ಷಸನಾಗು’ ಎಂದು ಅವನಿಗೆ ಶಾಪ ಕೊಟ್ಟನು. ಘಂಟಾಕರ್ಣ ಎಂಬ ಈ ರಾಕ್ಷಸನೇ ಎಲ್ಲ ಮತಾಂಧತೆಯ ಪಿತ. ಭಾರತದಲ್ಲಿ ಹುಡುಗರು ಅವನನ್ನು ಆದರಿಸಿ ಪೂಜಿಸುತ್ತಾರೆ. ಅದೊಂದು ವಿಚಿತ್ರ ಬಗೆಯ ಪೂಜೆ. ಒಂದು ಮಣ್ಣಿನ ವಿಗ್ರಹ ತಯಾರಿಸಿ, ಅದರ ಮೂಲಕ ಎಲ್ಲ ಬಗೆಯ ಕೆಟ್ಟ ವಾಸನೆಯ ಹೂವುಗಳೊಂದಿಗೆ ಅವನನ್ನು ಪೂಜಿಸುತ್ತಾರೆ. ಭಾರತದ ಕಾಡುಗಳಲ್ಲಿ ಅತ್ಯಂತ ಹಾನಿಕರ ವಾದ ವಾಸನೆಯುಳ್ಳ ಹೂವುಗಳು ದೊರೆಯುತ್ತವೆ. ಈ ಹೂವುಗಳ ಮೂಲಕ ಅವನನ್ನು ಪೂಜಿಸಿ, ಅನಂತರ ದೊಡ್ಡ ದೊಣ್ಣೆಗಳಿಂದ ವಿಗ್ರಹವನ್ನು ಒಡೆಯುತ್ತಾರೆ. ತಮ್ಮ ದೇವರನ್ನು ಬಿಟ್ಟು ಮತ್ತೆಲ್ಲ ದೇವರನ್ನು ದ್ವೇಷಿಸುವ ಎಲ್ಲ ಮತಾಂಧರ ತಂದೆಯೇ ಈ ಘಂಟಾಕರ್ಣ ರಾಕ್ಷಸ).

ಈ ನಿಷ್ಠಾಭಕ್ತಿಯ ಒಂದು ಅಪಾಯವೇ ಅದು ರಾಕ್ಷಸೀಯ ಮತಾಂಧತೆಯಾಗಿ ಪರಿಣಮಿಸುವುದು. ಪ್ರಪಂಚ ಇಂಥವರಿಂದ ತುಂಬಿ ಹೋಗುವುದು. ದ್ವೇಷಿಸುವುದು ಬಹಳ ಸುಲಭ. ಜನ ಸಾಮಾನ್ಯರು ಎಷ್ಟು ದುರ್ಬಲರಾಗುತ್ತಾರೆಂದರೆ, ಒಬ್ಬನನ್ನು ಪ್ರೀತಿಸಬೇಕಾದರೆ ಮತ್ತೊಬ್ಬನನ್ನು ದ್ವೇಷಿಸುತ್ತಾರೆ. ಒಂದರಲ್ಲಿ ಅವರು ಶಕ್ತಿಯನ್ನು ಕೇಂದ್ರೀಕರಿಸಬೇಕಾದರೆ ಮತ್ತೊಂದರಿಂದ ಅದನ್ನು ತೆಗೆಯಬೇಕು. ಒಬ್ಬ ಪುರುಷನು ಒಬ್ಬಳು ಮಹಿಳೆಯನ್ನು ಪ್ರೀತಿಸುತ್ತಾನೆ, ಅನಂತರ ಮತ್ತೋರ್ವಳನ್ನು ಪ್ರೀತಿಸುತ್ತಾನೆ. ಹೀಗೆ ಮತ್ತೋರ್ವಳನ್ನೂ ಪ್ರೀತಿಸಬೇಕಾದರೆ ಮೊದಲಿನವಳನ್ನು ಅವನು ದ್ವೇಷಿಸಲೇ ಬೇಕು. ಸ್ತ್ರೀಯರ ವಿಷಯವೂ ಹೀಗೆಯೇ. ಈ ಲಕ್ಷಣ ನಮ್ಮ ಸ್ವಭಾವದಲ್ಲಿಯೇ ಇದೆ, ಆದ್ದರಿಂದ ಅದು ಧರ್ಮದಲ್ಲಿಯೂ ಇದೆ. ಅಪಕ್ವ ಮನಸ್ಸಿನ ಸಾಮಾನ್ಯ ವ್ಯಕ್ತಿಯು ಇನ್ನೊಬ್ಬನನ್ನು ದ್ವೇಷಿಸದೆ ಒಬ್ಬನನ್ನು ಪ್ರೀತಿಸಲಾರನು. ಈ ಲಕ್ಷಣವೇ ಧರ್ಮದಲ್ಲಿ ಮತಾಂಧತೆಯಾಗಿ ಕಾಣಿಸಿಕೊಳ್ಳುವುದು. ತಮ್ಮ ಆದರ್ಶವನ್ನು ಪ್ರೀತಿಸುವುದೆಂದರೆ ಇತರ ಎಲ್ಲ ಆದರ್ಶಗಳನ್ನೂ ದ್ವೇಷಿಸುವುದು.

ಇದನ್ನು ನಿವಾರಿಸಬೇಕು. ಇದೇ ಸಂದರ್ಭದಲ್ಲಿ ಬೇರೆ ಅಪಾಯವನ್ನೂ ನಿವಾರಿಸ ಬೇಕು. ನಮ್ಮ ಶಕ್ತಿಯನ್ನು ವೃಥಾ ವ್ಯಯಮಾಡಬಾರದು. ಹಾಗೆ ಮಾಡಿದರೆ ಧರ್ಮ ವೆಂದರೆ ನಮ್ಮ ಪಾಲಿಗೆ ಕೇವಲ ಉಪನ್ಯಾಸ ಕೇಳುವುದಾಗುತ್ತದೆ - ಅದಕ್ಕಿಂತ ಹೆಚ್ಚಲ್ಲ. ಇವು ಎರಡು ಅಪಾಯಗಳು. ಉದಾರ ಮನಸ್ಸಿನವರಲ್ಲಿ ವೈಶಾಲ್ಯತೆಯಿರುತ್ತದೆ, ತೀವ್ರತೆಯಿರುವುದಿಲ್ಲ. ಇತ್ತೀಚಿನ ದಿನಗಳಲ್ಲಿ ಧರ್ಮವು ತುಂಬ ವಿಶಾಲವಾಗುತ್ತಿದೆ. ಆದರೆ ಭಾವನೆಗಳು ವಿಶಾಲವಾದಷ್ಟೂ ಗಭೀರತೆಯನ್ನು ಕಳೆದುಕೊಳ್ಳುತ್ತವೆ. ಅನೇಕರಿಗೆ ಧರ್ಮವೆಂದರೆ ಸ್ವಲ್ಪ ದಾನ ಮಾಡುವುದು ಮತ್ತು ಶ್ರಮಪರಿಹಾರಕ್ಕಾಗಿ ಸ್ವಲ್ಪ ಮನ ರಂಜನೆಯನ್ನು ಪಡೆಯುವುದು - ಮನರಂಜನೆಗಾಗಿ ಅವರಿಗೆ ಐದು ನಿಮಿಷದ ಧರ್ಮ ಸಾಕು. ಉದಾರ ಭಾವನೆಯ ಅಪಾಯ ಇದು. ಇನ್ನೊಂದು ಕಡೆ ಮತೀಯ ಮನೋ ಭಾವದವರಲ್ಲಿ ಆಳವಿರುತ್ತದೆ, ತೀವ್ರತೆಯಿರುತ್ತದೆ, ಆದರೆ ಆ ತೀವ್ರತೆ ತುಂಬ ಸಂಕುಚಿತವಾಗಿರುತ್ತದೆ. ಅವರಲ್ಲಿ ಗಭೀರತೆಯಿದೆ, ಆದರೆ ವೈಶಾಲ್ಯತೆಯಿಲ್ಲ. ಅಷ್ಟೇ ಅಲ್ಲದೆ ಇದರ ಪರಿಣಾಮವಾಗಿ ಅವರು ಎಲ್ಲರನ್ನೂ ದ್ವೇಷಿಸುತ್ತಾರೆ.

ಈ ಎರಡೂ ಅಪಾಯಗಳನ್ನು ನಿವಾರಿಸಿ, ಅತ್ಯಂತ ಉದಾರಿಗಳಲ್ಲಿರುವ ವೈಶಾಲ್ಯತೆ ಯನ್ನೂ ಕ್ರೂರ ಮತಾಂಧರಲ್ಲಿರುವ ಗಭೀರತೆಯನ್ನೂ ಪಡೆದಾಗ ಸಮಸ್ಯೆಯ ಪರಿ ಹಾರವಾಗುತ್ತದೆ. ಇದನ್ನು ಸಾಧಿಸುವುದು ಹೇಗೆ ಎಂಬುದು ಪ್ರಶ್ನೆ. ಇಷ್ಟ - ನಿಷ್ಠೆಯ ಸಿದ್ಧಾಂತವೇ ಅದಕ್ಕೆ ಮಾರ್ಗ. ಎಲ್ಲ ಆದರ್ಶಗಳೂ ಸತ್ಯ ಮತ್ತು ಯಥಾರ್ಥ ಎಂದು ತಿಳಿಯಬೇಕು; ಇವೆಲ್ಲವೂ ಒಂದೇ ದೇವರ ವಿವಿಧ ಮುಖಗಳು; ಆದರೆ ಈ ಎಲ್ಲ ರೂಪಗಳ ಮೂಲಕ ದೇವರನ್ನು ಪೂಜಿಸಲು ನಾವಿನ್ನೂ ಅಸಮರ್ಥರು; ಆದ್ದರಿಂದ ಒಂದು ಆದರ್ಶವನ್ನು ಹಿಡಿದುಕೊಂಡು, ಅದನ್ನೇ ನಮ್ಮ ಜೀವನ ಸರ್ವಸ್ವವಾಗಿ ಮಾಡಿಕೊಳ್ಳಬೇಕು. ಇದರಲ್ಲಿ ನೀವು ಯಶಸ್ವಿಯಾದರೆ ಉಳಿದದ್ದೆಲ್ಲ ತಾನಾಗಿಯೇ ಬರುತ್ತದೆ. (ಇಲ್ಲಿಗೆ ಭಕ್ತಿಯ ಮೊದಲ ಭಾಗ, ಅಂದರೆ ಗೌಣೀ ಅಥವಾ ವೈಧೀ ಭಕ್ತಿ ಸಮಾಪ್ತಿಗೊಳ್ಳುತ್ತದೆ).

ಮುಂದಿನ ಪಾಠ ಪ್ರತೀಕಗಳಿಗೆ ಸಂಬಂಧಿಸಿದುದು. ದೇವರನ್ನು ಪೂಜಿಸುವುದೇ ಭಕ್ತಿ, ಇನ್ನಾರನ್ನಾದರೂ ಪೂಜಿಸುವುದು ಭಕ್ತಿಯಲ್ಲ. ಆದರೆ ಯಾವುದನ್ನಾದರೂ ದೇವರು ಎಂಬ ಭಾವದಿಂದ ಪೂಜಿಸಬಹುದು. ಅದನ್ನು ದೇವರು ಎಂದು ತಿಳಿಯದಿದ್ದರೆ ಆ ಪೂಜೆ ಭಕ್ತಿ ಎನಿಸುವುದಿಲ್ಲ. ಅದನ್ನು ದೇವರು ಎಂದು ಭಾವಿಸಿದರೆ ಅದು ಸರಿ.

ನದಿತೀರದ ಅರಣ್ಯ ಪ್ರದೇಶದಲ್ಲಿ ಯೋಗಿಯೊಬ್ಬನು ಧ್ಯಾನಾಭ್ಯಾಸ ಮಾಡುತ್ತಿದ್ದ. ಬಡ ಗೋಪಾಲಕನೊಬ್ಬನು ಅದೇ ಸ್ಥಳದಲ್ಲಿ ಹಸುಗಳನ್ನು ಮೇಯಿಸುತ್ತಿದ್ದ. ಪ್ರತಿ ದಿನವೂ ಆತ ಈ ಯೋಗಿಯು ಗಂಟೆಗಟ್ಟಳೆ ಧ್ಯಾನ ಮಾಡುವುದನ್ನೂ ತಪಸ್ಸನ್ನು ಆಚರಿಸುವು ದನ್ನೂ ಅಧ್ಯಯನ ಮಾಡುವುದನ್ನೂ ನೋಡುತ್ತಿದ್ದ. ಅವನು ಏನು ಮಾಡುತ್ತಾನೆ ಎಂಬುದನ್ನು ತಿಳಿಯುವ ಕುತೂಹಲವುಂಟಾಯಿತು ಗೋಪಾಲಕನಿಗೆ. ಅವನು ಯೋಗಿಯ ಹತ್ತಿರ ಬಂದು ಕೇಳಿದ: “ಸ್ವಾಮಿ, ದೇವರನ್ನು ಪಡೆಯುವ ಮಾರ್ಗವನ್ನು ನನಗೆ ಬೋಧಿಸುತ್ತೀರಾ?” ಆ ಯೋಗಿಯು ದೊಡ್ಡ ಪಂಡಿತನೂ ಶ್ರೇಷ್ಠ ವ್ಯಕ್ತಿಯೂ ಆಗಿದ್ದ. ಅವನು ಹೇಳಿದ: “ನೀನು ಸಾಮಾನ್ಯ ಗೋವಳ - ದೇವರನ್ನು ನೀನು ಹೇಗೆ ಅರ್ಥಮಾಡಿಕೊಳ್ಳ ಬಲ್ಲೆ? ದಡ್ಡ, ಸುಮ್ಮನೆ ಮನೆಗೆ ಹೋಗಿ ಹಸುವನ್ನು ಮೇಯಿಸಿ ಕೊಂಡಿರು. ಇಂಥ ವಿಷಯಗಳ ಬಗ್ಗೆ ತಲೆ ಕೆಡಿಸಿಕೊಳ್ಳ ಬೇಡ”.

ಆ ಮನುಷ್ಯ ಹೊರಟುಹೋದ. ಆದರೆ ನಿಜವಾದ ಬಯಕೆ ಅವನಲ್ಲಿ ಜಾಗ್ರತ ವಾಯಿತು. ಅವನು ಸುಮ್ಮನಿರಲಾಗದೆ ಮತ್ತೆ ಯೋಗಿಯ ಹತ್ತಿರ ಬಂದು ಹೇಳುತ್ತಾನೆ: “ಸ್ವಾಮಿ, ದೇವರ ಬಗ್ಗೆ ನನಗೆ ಏನಾದರು ಹೇಳುತ್ತೀರಾ?” ಮತ್ತೆ ಅದೇ ನಿರಾಕರಣೆ: “ಆಹ್, ನೀನು ದಡ್ಡ, ದೇವರ ಬಗ್ಗೆ ನೀನು ಏನು ತಿಳಿಯಬಲ್ಲೆ? ಮನೆಗೆ ಹೋಗು”. ಆದರೆ ಗೋಪಾಲಕನಿಗೆ ನಿದ್ರೆ ಮಾಡಲಾಗಲಿಲ್ಲ, ಊಟವೂ ರುಚಿಸಲಿಲ್ಲ. ತಾನು ದೇವರ ಬಗ್ಗೆ ತಿಳಿಯಲೇ ಬೇಕು ಎಂದೆನಿಸಿತು ಅವನಿಗೆ.

ಅವನು ಮತ್ತೆ ಬಂದನು. ಅವನನ್ನು ಸುಮ್ಮನಾಗಿಸಲು ಯೋಗಿಯು ಹೇಳಿದ: “ಆಗಲಿ, ನಿನಗೆ ದೇವರನ್ನು ಬೋಧಿಸುತ್ತೇನೆ”.

ಅವನು ಕೇಳಿದ: “ಸ್ವಾಮಿ, ದೇವರು ಎಂಥವನು? ಅವನ ರೂಪ ಯಾವುದು? ಅವನು ಹೇಗೆ ಕಾಣುತ್ತಾನೆ?”

ಯೋಗಿಯು ಹೇಳಿದ: “ದೇವರು ನಿನ್ನ ಹಸುವಿನ ಮಂದೆಯಲ್ಲಿರುವ ದೊಡ್ಡ ಬಸವನಂತಿದ್ದಾನೆ”. ಅದೇ ದೇವರು. ದೇವರು ಆ ದೊಡ್ಡ ಬಸವನಾಗಿದ್ದಾನೆ”.

ಗೋಪಾಲಕನು ಅವನ ಮಾತನ್ನು ನಂಬಿ ಹಿಂತಿರುಗಿದ. ಆ ಬಸವನನ್ನೇ ದೇವರೆಂದು ತಿಳಿದು ಹಗಲು ರಾತ್ರಿ ಅದನ್ನು ಪೂಜಿಸತೊಡಗಿದ. ಒಳ್ಳೆಯ ಹಸಿರು ಹುಲ್ಲನ್ನು ಅದಕ್ಕೆ ತಿನಿಸಿದ, ಅದರ ಹತ್ತಿರವೇ ವಿಶ್ರಮಿಸುತ್ತಿದ್ದ, ಅದಕ್ಕೆ ಧೂಪ ದೀಪಗಳನ್ನು ಅರ್ಪಿಸಿದ ಮತ್ತು ಅದನ್ನೇ ಅನುಸರಿಸಿದ. ಹೀಗೆ ದಿನ ತಿಂಗಳು ವರ್ಷಗಳು ಉರುಳಿದವು. ಅವನ ಪೂರ್ಣ ಆತ್ಮವೆ ಆ ಬಸವನಲ್ಲಿ ನೆಲೆಗೊಂಡಿತ್ತು.

ಒಂದು ದಿನ ಅವನಿಗೆ ಒಂದು ಧ್ವನಿ ಕೇಳಿಸಿತು - ಅದು ಬಸವನೊಳಗಿನಿಂದ ಬರುತ್ತಿರುವುದನ್ನು ಗಮನಿಸಿದ. “ಬಸವ ಮಾತನಾಡುತ್ತಿರುವುದು!” ಎಂದು ಅವನು ಚಕಿತನಾದ.

“ವತ್ಸ, ವತ್ಸ”.

“ಏನು, ಬಸವ ಮಾತನಾಡುತ್ತಿರುವುದೆ! ಇಲ್ಲ, ಬಸವ ಮಾತನಾಡಲಾರದು”.

ಅತ್ತ ಸರಿದು ತುಂಬ ಖಿನ್ನನಾಗಿ ಧ್ಯಾನ ಮಾಡುತ್ತ ಕುಳಿತನು. ಅವನಿಗೆ ಏನೂ ತಿಳಿಯಲಿಲ್ಲ. ಮತ್ತೆ ಬಸವನೊಳಗಿನಿಂದ ಧ್ವನಿ ಕೇಳಿಸಿತು: “ವತ್ಸ, ವತ್ಸ”. ಅವನು ಹತ್ತಿರ ಹೋದ. “ಇಲ್ಲ, ಬಸವ ಮಾತನಾಡಲಾರದು,” ಎಂದು ಭಾವಿಸಿ, ಹಿಂತಿರುಗಿ ವಿಷಾದ ದಿಂದ ಕುಳಿತುಕೊಂಡ.

ಮತ್ತೆ ಧ್ವನಿ ಕೇಳಿಸಿತು. ಈಗ ಅವನು ಕಂಡುಹಿಡಿದ, ತನ್ನ ಹೃದಯದಿಂದಲೇ ಧ್ವನಿ ಬರುತ್ತಿತ್ತು. ದೇವರು ತನ್ನೊಳಗೇ ಇರುವನೆಂದು ಅವನಿಗೆ ತಿಳಿಯಿತು. “ನಾನು ಯಾವಾಗಲೂ ನಿನ್ನೊಡನೆ ಇದ್ದೇನೆ” ಎಂಬ ಭಗವದ್ವಾಣಿಯ ಅದ್ಭುತ ಸತ್ಯ ಅವನಿಗೆ ಅರ್ಥವಾಯಿತು. ಬಡ ಗೋವಳನಿಗೆ ಇಡೀ ರಹಸ್ಯ ಅರ್ಥವಾಯಿತು.

ಅನಂತರ ಅವನು ಯೋಗಿಯನ್ನು ನೋಡಲು ಹೋಗುತ್ತಾನೆ. ಯೋಗಿಯು ಅವನನ್ನು ದೂರದಿಂದ ಬರುತ್ತಿರುವಾಗಲೇ ನೋಡಿದ. ಯೋಗಿಯು ದೊಡ್ಡ ಪಂಡಿತ, ಮಹಾ ತಪಸ್ವಿ. ಈ ಗೋಪಾಲಕನಾದರೊ ದಡ್ಡ, ಅನಕ್ಷರಕುಕ್ಷಿ. ಆದರೆ ಅವನ ಇಡೀ ದೇಹವೇ ಪರಿವರ್ತನೆಗೊಂಡಂತೆ ಕಾಣುತ್ತಿದೆ, ಸ್ವರ್ಗೀಯ ಜ್ಯೋತಿ ಅವನ ಮುಖದಲ್ಲಿ ಬೆಳಗುತ್ತಿದೆ. ಯೋಗಿಯು ಎದ್ದು ನಿಂತನು.

“ಇದು ಎಂಥ ಬದಲಾವಣೆ? ಇದು ಹೇಗಾಯಿತು?”

“ಸ್ವಾಮಿ, ಇದು ನಿಮ್ಮ ಅನುಗ್ರಹ”.

“ಅದು ಹೇಗೆ? ನಾನು ನಿನಗೆ ಅದನ್ನು ತಮಾಷೆಗೆ ಹೇಳಿದೆ.”

“ಆದರೆ ನಾನು ಅದನ್ನು ಗಂಭೀರವಾಗಿ ಸ್ವೀಕರಿಸಿದೆ. ನಾನು ಆ ಬಸವನಿಂದಲೇ ನಾನು ಬಯಸಿದುದೆಲ್ಲವನ್ನೂ ಪಡೆದೆ. ಏಕೆ, ದೇವರು ಎಲ್ಲೆಲ್ಲಿಯೂ ಇಲ್ಲವೇನು!”

ಆದ್ದರಿಂದ ಇಲ್ಲಿ ಆ ಬಸವನೇ ಪ್ರತೀಕ. ಆ ವ್ಯಕ್ತಿಯು ಆ ಪ್ರತೀಕವನ್ನೇ ದೇವರೆಂದು ಪೂಜಿಸಿ ಎಲ್ಲವನ್ನೂ ಪಡೆದ. ಆ ತೀವ್ರ ಪ್ರೀತಿ, ಆ ತೀವ್ರ ಬಯಕೆ ಎಲ್ಲವನ್ನೂ ಸಾಧಿಸಿ ಕೊಡುತ್ತದೆ. ಎಲ್ಲವೂ ನಮ್ಮೊಳಗೇ ಇದೆ. ಬಾಹ್ಯ ಜಗತ್ತು, ಬಾಹ್ಯ ಪೂಜೆ ಅದನ್ನು ಹೊರ ಪಡಿಸುವುದಕ್ಕೆ ಒಂದು ಸೂಚನೆಯಷ್ಟೆ, ಅದು ಬಲಯುತವಾದಾಗ ಒಳಗಿರುವ ದೇವರು ಜಾಗ್ರತವಾಗುತ್ತಾನೆ.

(ಬಾಹ್ಯ ಗುರು ಒಂದು ಸೂಚನೆಯಷ್ಟೆ. ಬಾಹ್ಯಗುರುವಿನಲ್ಲಿ ಶ್ರದ್ಧೆ ಬಲಯುತ ವಾಗಿದ್ದರೆ, ಒಳಗಿರುವ ಎಲ್ಲ ಗುರುಗಳ ಗುರುವಾದ ಭಗವಂತನು ಮಾತನಾಡುತ್ತಾನೆ, ಶಾಶ್ವತ ಜ್ಞಾನ ಹೃದಯದಲ್ಲಿ ಸುಙರಿಸುತ್ತದೆ. ಅವನಿಗೆ ಯಾವ ಗ್ರಂಥದ ಆವಶ್ಯಕತೆಯೂ ಇಲ್ಲ, ಯಾರೆಡಗೂ ಹೋಗಬೇಕಾಗಿಲ್ಲ, ಯಾವ ದೇವತೆಯನ್ನೂ ಆಶ್ರಯಿಸಬೇಕಾಗಿಲ್ಲ, ಯಾವ ಯಕ್ಷ ಗಂಧರ್ವರ ಉಪದೇಶವೂ ಅವನಿಗೆ ಬೇಕಾಗಿಲ್ಲ. ಭಗವಂತನೇ ಅವನ ಗುರುವಾಗುತ್ತಾನೆ. ತನ್ನಿಂದಲೇ ಅವನಿಗೆ ಬೇಕಾದುದಲ್ಲ ದೊರೆಯುತ್ತದೆ. ಅವನು ಯಾವ ದೇವಸ್ಥಾನ ಚರ್ಚುಗಳಿಗೂ ಹೋಗಬೇಕಾಗಿಲ್ಲ. ಅವನ ದೇಹವೇ ಶ್ರೇಷ್ಠ ದೇವಾಲಯವಾಗುತ್ತದೆ, ಅವರಲ್ಲಿ ಸೃಷ್ಟಿಕರ್ತ ಭಗವಂತನು ವಾಸಿಸುತ್ತಾನೆ. ಪ್ರತಿ ಯೊಂದು ದೇಶದಲ್ಲಿಯೂ ಮಹಾನ್ ಸಂತರು ಜನಿಸಿದ್ದಾರೆ; ಅವರ ಜೀವನ ಅದ್ಭುತ ವಾದುದು - ಇದೆಲ್ಲ ಪ್ರೇಮ ಶಕ್ತಿಯ ಪ್ರಭಾವ).

ನಾಮಜಪ, ಪ್ರತೀಕೋಪಾಸನೆ, ಇಷ್ಟನಿಷ್ಠೆ - ಇವೆಲ್ಲ ಭಕ್ತಿಯ ಬಾಹ್ಯ ರೂಪಗಳು. ಇವು ಆ ಶಾಶ್ವತ ಶಕ್ತಿಯು ಜಾಗ್ರತವಾಗುವುದಕ್ಕೆ ಸಿದ್ಧತೆಗಳು. ಈ ಎಲ್ಲ ವಿಧಿ ನಿಯಮ ಗಳನ್ನು ಮೀರಿ ಹೋದ ಮೇಲೆ ನಿಜವಾದ ಆಧ್ಯಾತ್ಮಿಕತೆ ಬರುತ್ತದೆ. ಆಗ ಎಲ್ಲ ನಿಯಮ ಗಳು ಕಳಚಿ ಬೀಳುವುವು, ಎಲ್ಲ ರೂಪಗಳು ಮಾಯವಾಗುವುವು, ದೇವಸ್ಥಾನ ಚರ್ಚು ಗಳೆಲ್ಲ ಉರುಳಿ ಮಣ್ಣುಪಾಲಾಗುವುವು. (ಒಂದು ಚರ್ಚಿನಲ್ಲಿ ಹುಟ್ಟುವುದು ಒಳ್ಳೆ ಯದು, ಆದರೆ ಅದರಲ್ಲಿಯೇ ಪ್ರಾಣಬಿಡುವುದು ಅತ್ಯಂತ ಶೋಚನೀಯವಾದುದು. ಒಂದು ಮತಪಂಥದಲ್ಲಿ ಹುಟ್ಟುವುದು ಒಳ್ಳೆಯದು, ಆದರೆ ಅದೇ ಮತದಲ್ಲಿ ಮತೀಯ ಭಾವನೆಗಳೊಂದಿಗೆ ಸಾಯುವುದು ತುಂಬ ಶೋಚನೀಯ).

ಭಗವಂತನ ಸಂತಾನವನ್ನು ಯಾವ ಮತ ತಾನೆ ಕಟ್ಟಿಹಾಕಬಲ್ಲದು? ಅವನಿಗೆ ಯಾವ ನಿಯಮವಿದೆ? ಅವನು ಅನುಸರಿಸಬೇಕಾದ ಕ್ರಿಯಾವಿಧಿಗಳಾವುವಿವೆ? ಅವನು ಪೂಜಿಸುವುದಾದರೂ ಯಾರನ್ನು? ಅವನು ಪರಮಾತ್ಮನನ್ನೇ ಪೂಜಿಸುತ್ತಾನೆ. ಪರಮಾತ್ಮನೇ ಅವನಿಗೆ ಬೋಧಿಸುತ್ತಾನೆ. ದೇವಾಲಯಗಳಲ್ಲೆಲ್ಲ ಅತ್ಯಂತ ಶ್ರೇಷ್ಠ ದೇವಾಲಯವಾದ ಮಾನವನ ಆತ್ಮನಲ್ಲಿ ಅವನು ವಾಸಿಸುತ್ತಾನೆ.

ಇದೇ ನಾವು ಪಡೆಯಬೇಕಾದ ಗುರಿ, ಇದೇ ಪರಾಭಕ್ತಿ. ಉಳಿದುವೆಲ್ಲ ಗೌಣ. ಆದರೆ ಅವು ಈ ಸ್ಥಿತಿಯನ್ನು ಪಡೆಯಲು ಆವಶ್ಯಕ. ಅನಂತಾತ್ಮವು ಗ್ರಂಥಗಳು, ಮತ ಗಳು ಮತ್ತು ಕ್ರಿಯಾವಿಧಿಗಳ ಬಂಧನದಿಂದ ಹೊರಬರಲು ಇವು ಸಹಾಯಕವಾಗುತ್ತವೆ. ಇವು ಕೊನೆಯಲ್ಲಿ ಅದೃಶ್ಯವಾಗಿ ಆತ್ಮನನ್ನು ಹಾಗೆಯೇ ಉಳಿಸುತ್ತವೆ. ಇವು ಅನಂತ ಕಾಲ ದಿಂದ ಬಂದ ಮೂಢನಂಬಿಕೆಗಳು. ಇದು ನನ್ನ ತಂದೆಯ ಧರ್ಮ, ಇದು ನನ್ನ ದೇಶದ ಧರ್ಮ, ಇದು ನನ್ನ ಗ್ರಂಥ - ಮುಂತಾದುವುಗಳೆಲ್ಲ ತಲೆತಲಾಂತರದಿಂದ ಬಂದ ಮೂಢನಂಬಿಕೆಗಳು. ಇವೆಲ್ಲ ಮಾಯವಾಗುತ್ತವೆ. ಇದು ಒಂದು ಮುಳ್ಳಿನಿಂದ ಮತ್ತೊಂದು ಮುಳ್ಳನ್ನು ತೆಗೆಯುವಂತೆ.

ಅನೇಕ ದೇಶಗಳಲ್ಲಿ, ಎಳೆಯ ಮಗುವಿನಲ್ಲಿಯೂ ಕೂಡ, ಅತ್ಯಂತ ಭಯಂಕರ ವಾದ, ಬರ್ಬರವಾದ ಮತೀಯ ಭಾವನೆಗಳನ್ನು ತುಂಬುತ್ತಾರೆ. ತಂದೆತಾಯಿಯರು ತಾವು ಮಗುವಿಗೆ ಒಳ್ಳೆಯದನ್ನು ಮಾಡುತ್ತಿರುವೆವು ಎಂದು ಭಾವಿಸುತ್ತಾರೆ. ಸಂಪ್ರ ದಾಯ ಪಾಲನೆಯ ಹೆಸರಿನಲ್ಲಿ ಮಗುವಿಗೆ ಹಾನಿಯುಂಟುಮಾಡುತ್ತಿರುವೆವು ಎಂಬುದು ಅವರಿಗೆ ತಿಳಿಯದು. ಇದೆಲ್ಲ ಸ್ವಾರ್ಥತೆ. ತಮ್ಮ ತೃಪ್ತಿಗಾಗಿ, ಸಮಾಜವನ್ನು ತೃಪ್ತಿಪಡಿಸುವುದಕ್ಕಾಗಿ ಜನರು ಏನು ಮಾಡುವುದಕ್ಕೂ ಸಿದ್ಧರಿರುತ್ತಾರೆ. ತಮ್ಮ ಮಕ್ಕಳನ್ನೇ ಬೇಕಾದರೆ ಕೊಲ್ಲುತ್ತಾರೆ, ತಾಯಂದಿರು ಕುಟುಂಬವನ್ನು ಉಪವಾಸ ಕೆಡುವುತ್ತಾರೆ, ಸಹೋದರರು ಸಹೋದರರನ್ನು ದ್ವೇಷಿಸುತ್ತಾರೆ. ಇದೆಲ್ಲ ಯಾರನ್ನೊ ತೃಪ್ತಿಪಡಿಸುವುದಕ್ಕಾಗಿ.

ಬಹುಪಾಲು ಜನರು ಯಾವುದೊ ಒಂದು ಸಾಂಪ್ರದಾಯಿಕ ಧರ್ಮದಲ್ಲಿ ಜನಿಸಿ ಅದರಿಂದ ಹೊರಬರುವುದೇ ಇಲ್ಲ. ಏಕೆ? ಈ ಸಂಪ್ರದಾಯಗಳು ಆಧ್ಯಾತ್ಮಿಕ ಬೆಳವಣಿಗೆಗೆ ಸಹಾಯಕವಾಗಿರುವವೆ? ಇವುಗಳ ಮೂಲಕ ಪ್ರೀತಿಯ ಅತ್ಯುನ್ನತ ನೆಲೆಯನ್ನು ತಲುಪ ಬಹುದಾದರೆ, ಬಹುಪಾಲು ಜನರು ಇನ್ನೂ ಯಾವುದಾದರೂ ಸಂಪ್ರದಾಯಕ್ಕೆ ಬದ್ಧ ರಾಗಿ ನರಳುತ್ತಿರುವರಲ್ಲ ಏಕೆ? ಮತೀಯ ಭಾವನೆಗಳನ್ನು ಮೀರಿಹೋಗಲಿಲ್ಲವಲ್ಲ ಏಕೆ? ಅವರೆಲ್ಲ ನಾಸ್ತಿಕರು, ಅವರಿಗೆ ಯಾವ ಧರ್ಮವೂ ಬೇಕಿಲ್ಲ.

ನೀವು ಏನನ್ನು ಬಯಸುತ್ತೀರೊ ನಿಮಗದು ದೊರೆಯುತ್ತದೆ. ಭಗವಂತನು ಎಲ್ಲ ಆಸೆಗಳನ್ನೂ ಪೂರ್ತಿಗೊಳಿಸುತ್ತಾನೆ. ಸಮಾಜದಲ್ಲಿ ಒಂದು ಸ್ಥಾನವನ್ನು ಉಳಿಸಿಕೊಳ್ಳ ಬೇಕೆಂದು ನೀವು ಬಯಸಿದರೆ, ನಿಮಗದು ದೊರೆಯುತ್ತದೆ. ನಿಮಗೆ ಚರ್ಚು ಬೇಕಾದರೆ ಅದು ಸಿಕ್ಕುತ್ತದೆ, ಆದರೆ ದೇವರು ಸಿಕ್ಕುವುದಿಲ್ಲ. ಈ ಎಲ್ಲ ಚರ್ಚುಗಳು, ಸಂಸ್ಥೆಗಳೆಂಬ ಮಕ್ಕಳಾಟದಲ್ಲಿ ತೊಡಗಿರಬೇಕೆಂದು ನೀವು ಬಯಸಿದರೆ, ಅದರಲ್ಲಿಯೇ ನೀವು ಕೊನೆಯವರೆಗೂ ಇರಬೇಕಾಗುತ್ತದೆ. “ಪಿತೃಗಳನ್ನು ಪೂಜಿಸುವವರು ಪಿತೃಗಳನ್ನು ಪಡೆಯುತ್ತಾರೆ; ಭೂತಾರಾಧಕರು ಭೂತಗಳನ್ನು ಪಡೆಯುತ್ತಾರೆ. ಆದರೆ ದೇವರನ್ನು ಪ್ರೀತಿಸುವವರು ದೇವರನ್ನೇ ಪಡೆಯುತ್ತಾರೆ”. ದೇವರನ್ನು ಪ್ರೀತಿಸುವವರು ಅವನೆಡೆಗೆ ಹೋಗುತ್ತಾರೆ, ಇತರರನ್ನು ಪ್ರೀತಿಸುವವರು ಅವರ ಪ್ರೀತಿಯ ವಸ್ತುವನ್ನು ಸೇರುತ್ತಾರೆ.

ದೇವಸ್ಥಾನ ಮತ್ತು ಚರ್ಚುಗಳಲ್ಲಿ ಮಂಡಿಯೂರುವುದು, ಕೈಮುಗಿಯುವುದು, ಕಣ್ಣು ಮುಚ್ಚುವುದು ಸಾಷ್ಟಾಂಗ ಪ್ರಣಾಮ ಮಾಡುವುದು - ಎಲ್ಲ ಕವಾಯಿತಿನಂತೆ ಯಾಂತ್ರಿಕವಾಗಿ ನಡೆಯುತ್ತದೆ. ಮನಸ್ಸು ಮಾತ್ರ ಬೇರೆಲ್ಲೋ ಇರುತ್ತದೆ. ಇದಾವುದೂ ನಿಜವಾದ ಧರ್ಮವಲ್ಲ.

ಸುಮಾರು ನಾನೂರು ವರ್ಷಗಳ ಹಿಂದೆ ಭಾರತದಲ್ಲಿ ಗುರು ನಾನಕ್ ಎಂಬ ಶ್ರೇಷ್ಠ ಸಂತನಿದ್ದ. ಯೋದ್ಧೃಗಳಾದ ಸಿಖ್ಖರ ಬಗ್ಗೆ ನೀವು ಕೇಳಿರಬಹುದು. ಗುರುನಾನಕನು ಸಿಖ್ ಧರ್ಮದ ಸಂಸ್ಥಾಪಕ.

ಒಂದು ದಿನ ಮಹಮ್ಮದೀಯರ ಮಸೀದಿಗೆ ಹೋಗುತ್ತಾನೆ. ಅವರ ದೇಶದಲ್ಲಿಯೇ ಮಹಮದೀಯರೆಂದರೆ ಎಲ್ಲರಿಗೂ ಭಯ. ಕ್ರೈಸ್ತ ದೇಶದಲ್ಲಿಯಂತೆಯೇ ಯಾರೂ ತಮ್ಮ ಧರ್ಮದ ವಿರುದ್ಧ ಒಂದು ಮಾತೂ ಎತ್ತುವಂತಿಲ್ಲ. ಗುರು ನಾನಕನು ಮಸೀದಿಯ ಒಳಗೆ ಹೋದನು. ಮಹಮದೀಯರು ಪ್ರಾರ್ಥನೆಗಾಗಿ ನಿಂತಿದ್ದರು. ಅವರು ಸಾಲಾಗಿ ನಿಲ್ಲುತ್ತಾರೆ: ಕೆಳಗೆ ಬಗ್ಗುತ್ತಾರೆ, ಮೇಲೇಳುತ್ತಾರೆ, ಅದೇ ಸಂದರ್ಭದಲ್ಲಿ ಬಾಯಲ್ಲಿ ಏನೊ ಉಚ್ಚರಿಸುತ್ತಾರೆ, ಇದಕ್ಕೆ ಒಬ್ಬ ಮುಂದಾಳಿರುತ್ತಾನೆ. ಗುರು ನಾನಕನು ಅಲ್ಲಿಗೆ ಹೋದನು. ಮುಲ್ಲಾಗಳು, “ಗುರುಗಳಿಗೆ ಗುರುವಾದ, ಅತ್ಯಂತ ದಯಾಮಯನಾದ ಭಗವಂತನ ಹೆಸರಿನಲ್ಲಿ” ಎಂದು ಹೇಳಿದಾಗ ಗುರುನಾನಕನು ಮುಗುಳ್ನಕ್ಕು ಹೇಳಿದನು, “ಇದೆಲ್ಲ ಧಾರ್ಮಿಕತೆಯ ಸೋಗು”. ಮುಲ್ಲಾನು ಕೋಪೋದ್ರಿಕ್ತನಾಗಿ, “ಯಾಕೆ ನಗುತ್ತಿರುವೆ?” ಎಂದು ರೇಗಿದನು.

“ಏಕೆಂದರೆ ನೀವು ಪ್ರಾರ್ಥಿಸುತ್ತಿಲ್ಲ. ಅದಕ್ಕೇ ನಕ್ಕೆ.”

“ಏನು, ಪ್ರಾರ್ಥಿಸುತ್ತಿಲ್ಲವೆ?”

“ಖಂಡಿತ ಇಲ್ಲ. ನಿಮ್ಮಲ್ಲಿ ಪ್ರಾರ್ಥನೆಯೆಂಬುದೇ ಇಲ್ಲ.”

ಮುಲ್ಲಾನು ಕೋಪದಿಂದ ಮ್ಯಾಜಿಸ್ಟ್ರೇಟ್ ಹತ್ತಿರ ಹೋಗಿ ದೂರಿತ್ತನು: “ಈ ಪಾಷಂಡನು ನಮ್ಮ ಮಸೀದಿಗೆ ಬಂದು, ನಾವು ಪ್ರಾರ್ಥಿಸುವುದನ್ನು ನೋಡಿ ಅಣಕಿಸು ತ್ತಾನೆ. ಇವನಿಗೆ ಒಂದೇ ಶಿಕ್ಷೆಯೆಂದರೆ ಮರಣದಂಡನೆ. ಕೂಡಲೇ ಇವನನ್ನು ಕೊಲ್ಲಿ”.

ಗುರು ನಾನಕನನ್ನು ಮ್ಯಾಜಿಸ್ಟ್ರೇಟರಿನ ಮುಂದೆ ಕರೆತರಲಾಯಿತು. ಮುಗುಳ್ನಗೆಗೆ ಕಾರಣವನ್ನು ಕೇಳಿದಾಗ ಅವನೆಂದನು:

“ಅವರು ಪ್ರಾರ್ಥಿಸುತ್ತಿರಲಿಲ್ಲ, ಅದಕ್ಕೆ”.

“ಅವರು ಏನು ಮಾಡುತ್ತಿದ್ದರು?” ಮ್ಯಾಜಿಸ್ಟ್ರೇಟ್ ಕೇಳಿದ.

“ಅವರನ್ನು ಇಲ್ಲಿಗೆ ಕರೆಸಿದರೆ ಅವರು ಏನು ಮಾಡುತ್ತಿದ್ದರು ಎಂಬುದನ್ನು ಹೇಳುತ್ತೇನೆ.”

ಮುಲ್ಲಾನನ್ನು ಕರೆಸಲಾಯಿತು. ಮ್ಯಾಜಿಸ್ಟ್ರೇಟ್ ಹೇಳಿದ: “ನೋಡು ಮುಲ್ಲ ಬಂದಿದ್ದಾರೆ. ಈಗ ನೀನು ನಕ್ಕಿದ್ದಕ್ಕೆ ಕಾರಣವನ್ನು ವಿವರಿಸು.”

ಗುರು ನಾನಕ್ ಹೇಳಿದ: “ಮುಲ್ಲಾರಿಗೆ ಕೊರಾನ್ ಗ್ರಂಥವನ್ನು ಕೊಡಿ (ಪ್ರಮಾಣ ಮಾಡಲು). ಅವರು ಮಸೀದಿಯಲ್ಲಿ ‘ಅಲ್ಲ, ಅಲ್ಲ’ ಎಂದು ಕೂಗುತ್ತಿರುವಾಗ ಅವರ ಮನಸ್ಸು ಮನೆಯಲ್ಲಿದ್ದ ಕೋಳಿ ಮರಿಯನ್ನು ಚಿಂತಿಸುತ್ತಿತ್ತು.”

ಪಾಪ, ಮುಲ್ಲ ನಿರುತ್ತರನಾದ. ಅವನು ಇತರ ಎಲ್ಲರಿಗಿಂತ ಹೆಚ್ಚು ಪ್ರಾಮಾಣಿಕ ನಾಗಿದ್ದ. ತಾನು ಕೋಳಿಮರಿಯನ್ನು ಚಿಂತಿಸುತ್ತಿದ್ದುದು ನಿಜವೆಂದು ಒಪ್ಪಿಕೊಂಡನು. ನಾನಕನಿಗೆ ಬಿಡುಗಡೆಯಾಯಿತು. ಮ್ಯಾಜಿಸ್ಟ್ರೇಟ್ ಮುಲ್ಲಾನಿಗೆ ಹೇಳಿದ: “ಮತ್ತೆ ಮಸೀದಿಗೆ ಹೋಗಬೇಡಿ. ಅಲ್ಲಿ ಈ ರೀತಿ ಅಪ್ರಾಮಾಣಿಕವಾಗಿ ವರ್ತಿಸುವುದಕ್ಕಿಂತ ಅಲ್ಲಿಗೆ ಹೋಗದಿರುವುದೇ ಒಳ್ಳೆಯದು. ನಿಮಗೆ ಪ್ರಾರ್ಥನೆ ಮಾಡಬೇಕೆನಿಸದೇ ಇದ್ದಾಗ ಅಲ್ಲಿಗೆ ಹೋಗಬೇಡಿ. ಆಷಾಢಭೂತಿತನ ಬೇಡ. ಮನಸ್ಸಿನಲ್ಲಿ ಕೋಳಿ ಮರಿಯನ್ನು ಚಿಂತಿಸುತ್ತ ದಯಾಮಯ ಮತ್ತು ಆನಂದಮಯ ಭಗವಂತನ ಹೆಸರನ್ನು ಉಚ್ಚರಿಸಬೇಡಿ.”

ಮಹಮದೀಯನೊಬ್ಬನು ತೋಟವೊಂದರಲ್ಲಿ ಪ್ರಾರ್ಥಿಸುತ್ತಿದ್ದ. ಅವರು ಪ್ರಾರ್ಥನೆ ಯನ್ನು ತಪ್ಪದೆ ಮಾಡುತ್ತಾರೆ. ಆ ಸಮಯ ಬಂದ ಕೂಡಲೆ, ಅವರು ಎಲ್ಲೇ ಇರಲಿ, ಪ್ರಾರಂಭಿಸುತ್ತಾರೆ. ನೆಲದ ಮೇಲೆ ಬೀಳುವುದು, ಮೇಲೇಳುವುದು, ಮತ್ತೆ ಬೀಳುವುದು, ಏಳುವುದು - ಹೀಗೆ ಮುಂದುವರಿಯುತ್ತದೆ. ಅವನು ತೋಟದಲ್ಲಿದ್ದಾಗ ಪ್ರಾರ್ಥನೆಯ ಸಮಯ ಬಂತು. ಅವನು ಪ್ರಾರ್ಥಿಸುವುದಕ್ಕಾಗಿ ನೆಲದ ಮೇಲೆ ಉದ್ದಂಡ ಮಲಗಿದ. ಅದೇ ತೋಟದಲ್ಲಿ ಯುವತಿಯೋರ್ವಳು ತನ್ನ ಪ್ರಿಯತಮನಿಗಾಗಿ ಕಾಯುತ್ತಿದ್ದಳು. ಅವನ ಹತ್ತಿರಕ್ಕೆ ಹೋಗುವ ಆತುರದಲ್ಲಿ ಅವಳು ಇವನು ಮಲಗಿರುವುದನ್ನು ನೋಡದೆ ಅವನ ಮೈಮೇಲೆಯೇ ನಡೆದು ಕೊಂಡು ಹೋದಳು. ಆ ಮಹಮದನು ತುಂಬ ಮತಾಂಧನಾಗಿದ್ದ. ಅವನ ದೇಹವನ್ನು ಹೀಗೆ ತುಳಿದಾಗ ಅವನಿಗೆ ಎಂಥ ಕೋಪ ಬಂದಿ ರಬೇಕೆಂಬುದನ್ನು ನೀವು ಊಹಿಸಬಹುದು. ಅವನಿಗೆ ಆ ಯುವತಿಯನ್ನು ತಕ್ಷಣ ಕೊಲ್ಲಬೇಕೆನಿಸಿತು. ಆ ಯುವತಿ ತುಂಬ ಚುರುಕು, ಅವಳೆಂದಳು: “ನಿಲ್ಲಿಸು. ನೀನೊಬ್ಬ ಮೂರ್ಖ, ಕಪಟಿ.”

“ಏನೆಂದೆ! ನಾನು ಕಪಟಿಯೆ?”

“ನಾನು ಮಾನವ ಪ್ರಿಯತಮನನ್ನು ಸಂಧಿಸಲು ಹೋಗುತ್ತಿದ್ದೇನೆ, ನನಗೆ ನೀನಿರು ವುದೇ ಕಾಣಿಸಲಿಲ್ಲ. ಆದರೆ ನೀನು ಸ್ವರ್ಗೀಯ ಪ್ರಿಯತಮನನ್ನು ಸಂಧಿಸಲು ಹೋಗು ತ್ತಿರುವೆ. ಯುವತಿ ನಿನ್ನ ದೇಹದ ಮೇಲೆ ನಡೆದುಕೊಂಡು ಹೋದುದು ನಿನಗೆ ತಿಳಿಯಬಾರದಿತ್ತು.”

