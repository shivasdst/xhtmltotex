
\chapter{ಅಧ್ಯಾಯ ೪. ಕಥಗೋಡಮ್​ನ ಹಾದಿಯಲ್ಲಿ}

\textbf{ಜೂನ್ ೧೧.}

ಶನಿವಾರ ಬೆಳಗ್ಗೆ ನಾವು ಆಲ್ಮೋರವನ್ನು ಬಿಟ್ಟು ಹೊರಟೆವು. ಕಥಗೋಡಮ್​ ತಲುಪು ವುದಕ್ಕೆ ನಮಗೆ ಎರಡೂವರೆ ದಿನಗಳು ಹಿಡಿದವು...

ಹಾದಿಯಲ್ಲಿ ಒಂದು ಹಳೆಯ ನೀರ್ಗಾಲಿ ಯಂತ್ರ ಹಾಗೂ ಯಾರೂ ಇಲ್ಲದೆ ಹಾಳುಬಿದ್ದ ಕಮ್ಮಾರಸಾಲೆ. ಅಲ್ಲಿ ಸ್ವಾಮಿಗಳು ಧೀರಮಾತಾಳಿಗೆ ಈ ಪರ್ವತಪ್ರದೇಶದಲ್ಲಿ ನರಾಶ್ವರೂಪದ ಭೂತಗಳಿರುವುವು ಎಂಬ ದಂತಕಥೆಯ ಬಗ್ಗೆ ಹೇಳಿದರು. ಆಕಾರಗಳನ್ನು ನೋಡಿದ ಅಲ್ಲಿನ ತಮ್ಮ ಅನುಭವ ಮತ್ತು ಅನಂತರ ತಾವು ಕೇಳಿದ್ದ, ಅಲ್ಲಿ ಪ್ರಚಲಿತವಿದ್ದ ಜನಪದ ಕಥೆಯನ್ನು ಸಹ ಹೇಳಿದರು.

ಗುಲಾಬಿ ಹೂಗಳ ಕಾಲ ಕಳೆದುಹೋಗಿತ್ತಾದರೂ, ಅಲ್ಲೊಂದು ಗುಲಾಬಿ ಹೂ ಅರಳಿ ನಿಂತಿತ್ತು; ಆದರೆ ಮುಟ್ಟಿದೊಡನೆ ಅದರ ಪಕಳೆಗಳೆಲ್ಲ ಉದುರಿಹೋದವು. ಭಾರತೀಯ ಕಾವ್ಯಕ್ಕೂ ಗುಲಾಬಿಗೂ ಇರುವ ಗಾಢವಾದ ನಂಟಿನಿಂದಾಗಿ ಅದನ್ನು ಅವರು ತೋರಿಸಿದರು.

\textbf{ಜೂನ್ ೧೨.}

ಭಾನುವಾರ ಸಂಜೆ ಬಯಲುಪ್ರದೇಶದ ಬಳಿ ನಾವು ಸರೋವರವೊಂದರ ದಡದಲ್ಲಿದ್ದ, ಹತ್ತಿರದ ಜಲಪಾತವಿದ್ದ, ನಾವು ವಿಚಿತ್ರವೆಂದುಕೊಂಡ ಹೋಟಲೊಂದರಲ್ಲಿ ವಿಶ್ರಾಂತಿ ಪಡೆದೆವು. ಅಲ್ಲಿ ಅವರು ನಮಗಾಗಿ ರುದ್ರಪ್ರಾರ್ಥನೆಯೊಂದನ್ನು ಅನುವಾದಿಸಿ ಹೇಳಿದರು:

\begin{myquote}
ಸತ್ಯವಲ್ಲದರಿಂದ ಸತ್ಯದೆಡೆ ನಡೆಸೆಮ್ಮ\\ಗಾಢಾಂಧಕಾರದಿಂ ಬೆಳಕಿನಡೆಗೆ;\\ಸಾವಿನಿಂದೆಮ್ಮನಮೃತದ ಕಡೆಗೆ.\\ಆತ್ಮನೊಳ ಗೋತಪ್ರೋತನಾಗುತ ನೀನು\\ನಮ್ಮೊಳಗೆ ಬಂದು ನೀ ನಮ್ಮನೆಳಸು.\\ಅಜ್ಞಾನದಿಂದೆಮ್ಮ ರಕ್ಷಿಸೆಲೆ ಭಯಂಕರನೆ!\\ದಯೆಸೂಸುತಿಹ ನಿನ್ನ ಸೊಗಮೊಗವ ತೋರು.
\end{myquote}

ನಾಲ್ಕನೆಯ ಸಾಲಿನ ಬಳಿ ಬಹಳ ಹೊತ್ತು ಹಿಂದೆಗೆಯುತ್ತ ಯೋಚಿಸಿದರು; “ಆಲಿಂಗಿ ಸೆಮ್ಮ ನೀ ಹೃದಯ ಗಹ್ವರದೊಳಗೆ!” ಎಂದು ಹೇಳುವುದೆಂದುಕೊಂಡರು; ಕೊನೆಗೆ ನಾಚುತ್ತ, ತಮ್ಮ ಮನಸ್ಸಿನ ಗಲಿಬಿಲಿಯನ್ನು ಸೂಚಿಸುವಂತೆ “ನಿಜವಾದ ಅರ್ಥವೆಂದರೆ, ಆತ್ಮನೊಳಗೆ ಓತಪ್ರೋತನಾಗುತ್ತ ನೀನು ಬಂದು ನಮ್ಮನ್ನು ಸೇರು ಎನ್ನುವುದೇ” ಎಂದರು. ಅಸಾಧಾರಣ ನಿಗೂಢತೆಯನ್ನೊಳಗೊಂಡ ಈ ವಾಕ್ಯದ ಒಳ್ಳೆಯ ಅರ್ಥವನ್ನು ಆಂಗ್ಲ ಭಾಷೆ ಸಂವಹಿಸಬಲ್ಲುದೇ ಎನ್ನುವುದು ಅವರ ಹೆದರಿಕೆಯಾಗಿದ್ದಿರಬೇಕು... ನನಗೆ ಅರ್ಥವಾದಂತೆ, ಅದರ ವಾಚ್ಯಾರ್ಥವನ್ನು ಹೀಗೆ ಹೇಳಬಹುದೇನೋ: “ಕೇವಲ ನಿನ್ನಲ್ಲಿ ನೀನೇ ಆವಿರ್ಭವಿಸುವವನೇ, ನಮ್ಮ ಮೂಲಕ ಕೂಡ ನೀನು ಆವಿರ್ಭವಿಸು!” ಅವರ ಆ ಕ್ಷಿಪ್ರ ಅನುವಾದವು, ಈಗ ನನಗನ್ನಿಸುವಂತೆ, ಸಮಾಧಿಯ ಪ್ರತ್ಯಕ್ಷ ಅನುಭವವನ್ನೇ ನೇರ ವಾಗಿ ಅಭಿವ್ಯಕ್ತಗೊಳಿಸಿದುದಾಗಿತ್ತು. ಅದು ಜೀವಂತ ಹೃದಯವನ್ನೇ ಸಂಸ್ಕೃತದಿಂದ ಹೊರತೆಗೆದು, ಆಂಗ್ಲರೂಪಕ್ಕೆ ತಂದಂತಿತ್ತು.

ಅದು ನಿಜಕ್ಕೂ ಭಾಷಾಂತರದ್ದೇ ಮಧ್ಯಾಹ್ನವಾಗಿತ್ತು; ಮಡುಗಟ್ಟಿದ್ದ ದುಃಖ ಕಳೆದ ಮೇಲೆ, ಅವರು ಹಿಂದೂ ಧಾರ್ಮಿಕ ಪರಿಭಾವನೆಗಳಲ್ಲಿ ಅತ್ಯಂತ ಸುಂದರವಾದ ಸ್ವಸ್ತಿ ವಾಚನವೊಂದರ ಕೆಲವು ಭಾಗಗಳನ್ನು ನಮಗೆ ಹೇಳಿದರು:

\begin{myquote}
ಸುಖಸಮೀರಣನೆಮಗೆ ಸುಯ್ಯಲಿನಿದಾಗಿ\\ಜಲಧಿಯುದಕಗಳೆಲ್ಲ ಉಕ್ಕಲಿನಿದಾಗಿ\\ಹೊಲದ ಧಾನ್ಯಗಳೆಮಗೆ ಆನಂದವಾಗಿರಲಿ\\ಸಸ್ಯರಸಮಾಧುರ್ಯ ಪಥ್ಯವಾಗಿರಲಿ\\ತರಲಿ ಗೋವುಗಳೆಮಗೆ ಆನಂದವನ್ನು\\ಅಂತರಿಕ್ಷದ ಪಿತನೆ ಆನಂದವೀಯೆಮಗೆ!\\ಆನಂದ ತುಂಬಿಹುದು ಭೂಧೂಳಿಯಲಿ ಕೂಡ.
\end{myquote}

ಅನಂತರ, ಸ್ವರವು ಧ್ಯಾನದಲ್ಲಿ ಕ್ರಮೇಣ ಲೀನವಾಗುವಂತೆ:

\begin{myquote}
ಎಲ್ಲವೂ ಆನಂದ - ಆನಂದ - ಆನಂದ.
\end{myquote}

ಮತ್ತೊಮ್ಮೆ ಖೇತ್ರಿಯ ಆ ನರ್ತಕಿಯಿಂದ ಸ್ವಾಮಿಗಳು ಕೇಳಿದ ಸೂರದಾಸನ ಆ ಗಾನದ ಪ್ರಸ್ತಾವನೆ ಬಂದಿತು:

\begin{myquote}
ನೋಡದಿರೆನ್ನಯ ಅವಗುಣ, ಪ್ರಭುವೆ!\\ಸಮದರ್ಶಿಯು ನೀನಲ್ಲವೆ, ಹರಿಯೆ?\\ಬ್ರಹ್ಮನೊಳೊಂದಾಗಿಸು ಇಬ್ಬರನು!...
\end{myquote}

ಅವರೊಮ್ಮೆ ಬನಾರಸ್ನಲ್ಲಿ ಕಪಿಗಳ ಹಿಂಡೊಂದರಿಂದ ಪೀಡಿಸಲ್ಪಟ್ಟಾಗ, ನೋಡು ತ್ತಿದ್ದ ವೃದ್ಧ ಸಂನ್ಯಾಸಿಯೊಬ್ಬರು, ಹಿಂದಿರುಗಿ ಓಡಿಯಾರು ಇವರು ಎಂಬ ಭಯದಿಂದ “ಯಾವಾಗಲೂ ಮೃಗವನ್ನು ಎದುರಿಸು!” ಎಂದು ಕೂಗಿಕೊಂಡ ಪ್ರಸಂಗವನ್ನು ನಮಗೆ ಹೇಳಿದ್ದು ಅಂದೇಯೋ ಅಥವಾ ಇನ್ನೊಂದು ದಿನವೋ?

ಆ ಪರ್ಯಟನಗಳು ಅತ್ಯಂತ ಸಂತೋಷದಾಯಕವಾಗಿದ್ದವು. ಗಮ್ಯಸ್ಥಾನವನ್ನು ಸೇರಿ ಬಿಟ್ಟೆವಲ್ಲ ಎಂದು ಯಾವಾಗಲೂ ಖೇದವೇ ಆಗುತ್ತಿತ್ತು. ಬಹುಶಃ ಈ ಸಮಯದಲ್ಲೇ ನಾವು ಒಂದು ಇಡೀ ಮಧ್ಯಾಹ್ನ ರೈಲಿನಲ್ಲಿ ಕುಳಿತು ಮಲೇರಿಯಾ ಪೀಡಿತ ಟೆರಾಯ್​ ಪ್ರಾಂತ್ಯವನ್ನು ದಾಟಿದ್ದು, ಇದೇ ಪ್ರಾಂತ್ಯದಲ್ಲಿಯೇ ಬುದ್ಧ ಜನಿಸಿದ ಎಂದು ಸ್ವಾಮಿ ಗಳು ನಮಗೆ ಜ್ಞಾಪಿಸಿದ್ದು.

ಪರ್ವತಪ್ರದೇಶದ ರಸ್ತೆಗಳಲ್ಲಿ ನಾವು ಕೆಳಗಿಳಿದು ಬರುತ್ತಿದ್ದಂತೆ, ಮಳೆ ಬಿದ್ದೊಡನೆ ಮಲೇರಿಯಾ ತಮ್ಮ ಮೇಲೆರಗುವುದು ಎಂಬ ಭಯದಿಂದ ಹಳ್ಳಿಗಾಡಿನ ಜನರ ಗುಂಪು ಗಳು ಗುಡ್ಡದ ಮೇಲ್ಭಾಗಕ್ಕೆ ವಲಸೆ ಹೋಗುತ್ತಿದ್ದುದನ್ನು ನೋಡಿದೆವು. ಈಗ ರೈಲಿನಲ್ಲಿ ಕುಳಿತು ನೋಡುವಾಗ ಸಸ್ಯವರ್ಗದಲ್ಲಿ ಕ್ರಮೇಣ ಬದಲಾವಣೆಯಾಗುತ್ತಿದ್ದುದು ನಮ್ಮ ಗಮನಕ್ಕೆ ಬಂದಿತು. ಗುರುದೇವರು ಕಾಡಿನಲ್ಲಿಯ ನವಿಲುಗಳನ್ನು, ಅಲ್ಲೊಂದು ಇಲ್ಲೊಂದು ಆನೆಗಳನ್ನು, ಅಥವಾ ಸಾಗುತ್ತಿದ್ದ ಒಂಟೆಯ ಸಾಲುಗಳನ್ನು, ಅವುಗಳ ಯಜಮಾನ ನಿಗಿಂತಲೂ ಹೆಚ್ಚಿನ ಹೆಮ್ಮೆಯಿಂದ, ಸಂತೋಷದಿಂದ ನಮಗೆ ತೋರಿಸತೊಡಗಿದರು...

