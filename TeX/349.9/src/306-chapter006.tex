
\chapter{ಪ್ಲೇಗುರೋಗವನ್ನು ಕುರಿತ ಘೋಷಣೆ\\ಓಂ ನಮೋ ಭಗವತೇ ರಾಮಕೃಷ್ಣಾಯ}

೧. ನೀವು ಸುಖವಾಗಿದ್ದರೆ ನಾವು ಸುಖವಾಗಿರುತ್ತೇವೆ, ನೀವು ದುಃಖಪಟ್ಟರೆ ನಾವೂ ದುಃಖಪಡುತ್ತೇವೆ. ಆದ್ದರಿಂದ, ಈ ತೀವ್ರ ಪ್ರತಿಕೂಲ ಪರಿಸ್ಥಿತಿಯಲ್ಲಿ, ನಿಮ್ಮ ಹಿತ ಕ್ಕಾಗಿ, ಈ ಸಾಂಕ್ರಾಮಿಕ ವ್ಯಾಧಿಯಿಂದ ಬೇಗ ಮುಕ್ತರಾಗುವುದಕ್ಕಾಗಿ ನಾವು ಪ್ರಯತ್ನಿ ಸುತ್ತಿದ್ದೇವೆ ಮತ್ತು ನಿರಂತರವಾಗಿ ಭಗವಂತನನ್ನು ಪ್ರಾರ್ಥಿಸುತ್ತಿದ್ದೇವೆ.

೨. ಈ ದಾರುಣ ವ್ಯಾಧಿಯಿಂದ ಮೇಲು ಕೀಳು ವರ್ಗದವರು, ಬಡವರು ಶ‍್ರೀಮಂತರು ಎಲ್ಲರೂ ಭಯಗ್ರಸ್ತರಾಗಿದ್ದಾರೆ. ಅಂಥವರ ಸೇವೆ ಮಾಡುವುದೇ ನಮ್ಮ ಉದ್ದೇಶವಾಗಿದೆ. ಈ ಸೇವೆಯಲ್ಲಿ ನಾವು ಪ್ರಾಣ ತೆತ್ತರೂ ಅದು ನಮ್ಮ ಭಾಗ್ಯವೆಂದು ತಿಳಿಯುತ್ತೇವೆ. ಏಕೆಂದರೆ ನೀವೆಲ್ಲರೂ ಭಗವಂತನ ವಿವಿಧ ರೂಪಗಳು. ದುರಹಂಕಾರ ದಿಂದ ಯಾರು ಹೀಗೆ ಭಾವಿಸುವುದಿಲ್ಲವೊ ಅವನು ದೇವರಿಗೆ ಅಪರಾಧವೆಸಗುತ್ತಾನೆ ಮತ್ತು ಅದರಿಂದ ಮಹಾಪಾಪಕ್ಕೆ ಒಳಗಾಗುತ್ತಾನೆ. ಇದರಲ್ಲಿ ಸ್ವಲ್ಪವೂ ಸಂಶಯವಿಲ್ಲ.

೩. ನಿರಾಧಾರಿತ ಭಯದಿಂದ ಅನವಶ್ಯಕವಾಗಿ ಕಾತರರಾಗಬೇಡಿ ಎಂದು ಕೇಳಿ ಕೊಳ್ಳುತ್ತೇವೆ. ದೇವರನ್ನು ಅವಲಂಬಿಸಿ ಶಾಂತವಾಗಿ ಈ ಸಮಸ್ಯೆಯ ಪರಿಹಾರೋ ಪಾಯವನ್ನು ಕಂಡುಹಿಡಿಯಿರಿ. ಇಲ್ಲದಿದ್ದರೆ ಆಗಲೇ ಯಾರು ಈ ಕಾರ್ಯದಲ್ಲಿ ನಿರತರಾಗಿರುವರೊ ಅವರಿಗೆ ನೆರವಾಗಿ.

೪. ಭಯವೇಕೆ? ಪ್ಲೇಗಿನ ದೆಸೆಯಿಂದ ಜನರ ಹೃದಯವನ್ನು ಹೊಕ್ಕಿರುವ ಭೀತಿಗೆ ಯಾವ ತಳಹದಿಯೂ ಇಲ್ಲ. ದೇವರ ದಯೆಯಿಂದ, ಬೇರೆ ಕಡೆಗಳಲ್ಲಿಯಂತೆ ಪ್ಲೇಗು ರೋಗವು ಕಲ್ಕತ್ತೆಯಲ್ಲಿ ಎಂದೂ ಅಂಥ ಭೀಕರ ರೂಪವನ್ನು ತಾಳಲಿಲ್ಲ. ಸರ್ಕಾರದವರೂ ಕೂಡ ನಮಗೆ ವಿಶೇಷ ನೆರವನ್ನು ನೀಡುತ್ತಿರುವರು. ಆದ್ದರಿಂದ ಭಯಕ್ಕೆ ಎಡೆ ಎಲ್ಲಿ?

೫. ಈ ನಿರಾಧಾರಿತ ಭಯವನ್ನು ಬಿಡೋಣ. ಭಗವಂತನ ಅನಂತ ಕೃಪೆಯಲ್ಲಿ ಶ್ರದ್ಧೆ ಇಡೋಣ. ಸೊಂಟಕಟ್ಟಿ ಕಾರ್ಯಕ್ಷೇತ್ರಕ್ಕೆ ಧುಮುಕೋಣ. ನಾವು ಸ್ವಚ್ಛ ಜೀವನವನ್ನು ನಡೆಸೋಣ. ರೋಗ, ರೋಗಭಯ ಇವು ಭಗವಂತನ ಕೃಪೆಯಿಂದ ಗಾಳಿಯಲ್ಲಿ ಕರಗಿ ಹೋಗುತ್ತವೆ.

೬. (ಅ) ಮನೆ, ಮನೆಯ ಸುತ್ತಮುತ್ತ, ಕೊಠಡಿ, ಬಟ್ಟೆ, ಹಾಸಿಗೆ, ಚರಂಡಿ ಇತ್ಯಾದಿ ಗಳನ್ನು ಯಾವಾಗಲೂ ಸ್ವಚ್ಛವಾಗಿಟ್ಟಿರಿ.

(ಆ) ಹಳಸಿದ ಮತ್ತು ಕೆಟ್ಟುಹೋದ ಆಹಾರವನ್ನು ಸ್ವೀಕರಿಸಬೇಡಿ. ಶುದ್ಧವಾದ ಪೌಷ್ಟಿಕವಾದ ಆಹಾರವನ್ನೇ ಸ್ವೀಕರಿಸಿ. ದುರ್ಬಲ ದೇಹವು ಬಹು ಬೇಗನೆ ರೋಗಕ್ಕೆ ಒಳಗಾಗುತ್ತದೆ.

(ಇ) ಯಾವಾಗಲೂ ಮನಸ್ಸು ಉಲ್ಲಾಸದಿಂದಿರಲಿ. ಪ್ರತಿಯೊಬ್ಬನೂ ಒಮ್ಮೆ ಸಾಯಲೇಬೇಕು. ಹೇಡಿಯು ಮತ್ತೆ ಮತ್ತೆ ಸಾವಿನ ನೋವನ್ನು ಅನುಭವಿಸುತ್ತಾನೆ. ಏಕೆಂದರೆ ಅವನ ಮನಸ್ಸಿನಲ್ಲಿಯೇ ಭಯವು ತುಂಬಿರುತ್ತದೆ.

(ಈ) ಯಾರು ಅನೈತಿಕ ಜೀವನವನ್ನು ನಡೆಸುತ್ತಾರೊ, ಇತರರಿಗೆ ಹಿಂಸೆಯನ್ನು ಉಂಟುಮಾಡುತ್ತಾರೊ ಅವರನ್ನು ಭಯ ಬಿಟ್ಟಿರುವುದೇ ಇಲ್ಲ. ಆದ್ದರಿಂದ ಈ ಆಪತ್ಕಾಲದಲ್ಲಿ ಆ ರೀತಿಯ ವರ್ತನೆಗಳೆಲ್ಲದರಿಂದ ದೂರವಿರಿ.

(ಉ) ಸಾಂಕ್ರಾಮಿಕರೋಗದ ಸಮಯದಲ್ಲಿ ನೀವು ಗೃಹಸ್ಥರಾಗಿದ್ದರೂ ಕೋಪ ಮತ್ತು ಕಾಮಗಳಿಂದ ದೂರವಿರಿ.

(ಊ) ಗಾಳಿಸುದ್ದಿಗೆ ಎಂದೂ ಕಿವಿಗೊಡಬೇಡಿ.

(ಎ) ಬ್ರಿಟಿಷ್ ಸರ್ಕಾರವು ಯಾರಿಗೂ ಬಲಾತ್ಕಾರವಾಗಿ ಚುಚ್ಚುಮದ್ದನ್ನು ನೀಡು ವುದಿಲ್ಲ. ಯಾರು ಇಷ್ಟಪಡುತ್ತಾರೊ ಅವರಿಗೆ ಮಾತ್ರ ಅದನ್ನು ನೀಡುತ್ತಾರೆ.

(ಏ) ರೋಗಗ್ರಸ್ತರನ್ನು ನಮ್ಮ ಆಸ್ಪತ್ರೆಗಳಲ್ಲಿ ನೋಡಿಕೊಳ್ಳಲು ಎಲ್ಲ ಪ್ರಯತ್ನ ವನ್ನೂ ಮಾಡಲಾಗುತ್ತದೆ. ಅವರವರ ಧರ್ಮ, ಜಾತಿ, ಸ್ತ್ರೀಯರ ಮಾನ್ಯತೆ ಇವುಗಳಿ ಗಾವುದಕ್ಕೂ ಭಂಗವಾಗದಂತೆ ನೋಡಿಕೊಳ್ಳಲಾಗುತ್ತದೆ. ಶ‍್ರೀಮಂತರು ಬೇಕಾದರೆ ಓಡಿಹೋಗಲಿ. ಆದರೆ ನಾವು ಬಡವರು, ನಾವು ಬಡವರ ಹೃದಯವೇದನೆ ಯನ್ನು ಅರ್ಥಮಾಡಿಕೊಳ್ಳುತ್ತೇವೆ. ಜಗನ್ಮಾತೆಯೇ ಅಸಹಾಯಕರಿಗೆ ಆಶ್ರಯ ನೀಡು ತ್ತಾಳೆ. ಅವಳೇ ‘ಭಯಪಡಬೇಡಿ, ಭಯಪಡಬೇಡಿ’ ಎಂದು ಭರವಸೆ ನೀಡುತ್ತಿರುವಳು.

೭. ಸಹೋದರನೆ, ನಿನಗೆ ಯಾರೂ ಸಹಾಯ ಮಾಡುವವರಿಲ್ಲದಿದ್ದರೆ ಬೇಲೂರು ಮಠದಲ್ಲಿರುವ ಶ‍್ರೀರಾಮಕೃಷ್ಣರ ಸೇವಕರಿಗೆ ಸುದ್ದಿಯನ್ನು ಮುಟ್ಟಿಸು. ಭೌತಿಕವಾಗಿ ಸಾಧ್ಯವಾಗುವ ಯಾವುದೇ ಸಹಾಯಕ್ಕೂ ಕೊರತೆಯಿಲ್ಲ. ಜಗನ್ಮಾತೆಯ ದಯೆಯಿಂದ ಆರ್ಥಿಕ ಸಹಾಯವೂ ಸಾಧ್ಯ.

ವಿ.ಸೂ.: ಸಾಂಕ್ರಾಮಿಕದ ಭಯವನ್ನು ಹೋಗಲಾಡಿಸಲು ಪ್ರತಿಸಂಜೆ, ಪ್ರತಿ ಯೊಂದು ಸ್ಥಳದಲ್ಲಿಯೂ ಭಗವನ್ನಾಮ ಸಂಕೀರ್ತನೆಯನ್ನು ನಡೆಸಬೇಕು.

