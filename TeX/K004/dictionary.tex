\sethyphenation{kannada}{
ಅ
ಅಂಕ-ಗಳನ್ನು
ಅಂಕು-ರಾ-ರ್ಪಣ
ಅಂಗ
ಅಂಗ-ಗಳು
ಅಂಗಡಿ
ಅಂಗ-ಡಿ-ಗಳ
ಅಂಗ-ಡಿಗೆ
ಅಂಗ-ಡಿಯ
ಅಂಗ-ಡಿ-ಯ-ವನ
ಅಂಗ-ಡಿ-ಯ-ವ-ನನ್ನು
ಅಂಗ-ಡಿ-ಯ-ವ-ನಿಗೆ
ಅಂಗ-ಡಿ-ಯ-ವನು
ಅಂಗ-ಡಿ-ಯ-ವ-ನೊಂ-ದಿಗೆ
ಅಂಗ-ಡಿ-ಯಿಂದ
ಅಂಗ-ಲಾಚಿ
ಅಂಗಳ
ಅಂಗ-ವಲ್ಲ
ಅಂಗ-ವಾ-ಗಲಿ
ಅಂಗ-ವಾಗಿ
ಅಂಗ-ವಾ-ಗಿ-ರುವ
ಅಂಗ-ವಾದ
ಅಂಗ-ವೆಂದು
ಅಂಗವೇ
ಅಂಗಾಂ-ಗ-ಗ-ಳಿಗೆ
ಅಂಗಾಂ-ಗ-ಗಳೂ
ಅಂಗೀ-ಕ-ರಿ-ಸಲು
ಅಂಗೀ-ಕ-ರಿ-ಸಿದ
ಅಂಗೀ-ಕ-ರಿ-ಸಿ-ದರು
ಅಂಗೀ-ಕ-ರಿ-ಸು-ತ್ತವೆ
ಅಂಗೀ-ಕೃತ
ಅಂಗೀ-ಕೃ-ತ-ವಾ-ಯಿತು
ಅಂಗುಲ
ಅಂಗು-ಲ-ವನ್ನೂ
ಅಂಗು-ಲ-ವಾಗಿ
ಅಂಗೈ
ಅಂಚನ್ನು
ಅಂಚಿ-ನಲ್ಲೇ
ಅಂಜ
ಅಂಜದೆ
ಅಂಜ-ಲಿ-ಬ-ದ್ಧ-ರಾಗಿ
ಅಂಜಿಕೆ
ಅಂಜಿ-ಕೆ-ಯಿಂದ
ಅಂಜಿ-ದೊ-ಡ-ನೆಯೇ
ಅಂಟಿ-ಕೊಂ-ಡ-ವರು
ಅಂಟಿ-ಕೊಂ-ಡಿದೆ
ಅಂಟಿ-ಕೊಂ-ಡಿದ್ದ
ಅಂಟಿ-ಕೊಂ-ಡಿ-ದ್ದರೂ
ಅಂಟಿ-ಕೊಂ-ಡಿ-ದ್ದರೆ
ಅಂಟಿ-ಕೊಂ-ಡಿ-ದ್ದಳು
ಅಂಟಿ-ಕೊ-ಳ್ಳದೆ
ಅಂಟು
ಅಂಟು-ಜಾ-ಡ್ಯ-ಗಳು
ಅಂಡ್
ಅಂತ
ಅಂತಃ-ಕ-ರಣ
ಅಂತಃ-ಕ-ರ-ಣದ
ಅಂತಃ-ಕ-ರ-ಣ-ವನ್ನು
ಅಂತಃ-ಕ-ರ-ಣವೂ
ಅಂತಃ-ಪ್ರ-ಜ್ಞೆಯು
ಅಂತಃ-ಸತ್ವ
ಅಂತಃ-ಸ್ವ-ಭಾ-ವ-ದೊಂ-ದಿಗೇ
ಅಂತರ
ಅಂತ-ರಂಗ
ಅಂತ-ರಂ-ಗಕ್ಕೂ
ಅಂತ-ರಂ-ಗದ
ಅಂತ-ರಂ-ಗ-ದಲ್ಲಿ
ಅಂತ-ರಂ-ಗ-ದ-ಲ್ಲಿದ್ದ
ಅಂತ-ರಂ-ಗ-ದಲ್ಲೂ
ಅಂತ-ರಂ-ಗ-ದಾ-ಳ-ದಲ್ಲಿ
ಅಂತ-ರಂ-ಗ-ವ-ನ್ನ-ರಿ-ಯಲು
ಅಂತ-ರಂ-ಗ-ವನ್ನು
ಅಂತ-ರಂ-ಗವು
ಅಂತ-ರ-ರಾ-ಷ್ಟ್ರೀಯ
ಅಂತ-ರ-ವಿದೆ
ಅಂತ-ರಾ-ತ್ಮದ
ಅಂತ-ರಾ-ತ್ಮನ
ಅಂತ-ರಾ-ಳದ
ಅಂತ-ರಾ-ಳ-ದಿಂದ
ಅಂತ-ರಾ-ಳ-ದೊ-ಳ-ಗಿನ
ಅಂತ-ರಿ-ಕ್ಷದ
ಅಂತರ್
ಅಂತ-ರ್ದೃಷ್ಟಿ
ಅಂತ-ರ್ದೃ-ಷ್ಟಿಗೆ
ಅಂತ-ರ್ದೃ-ಷ್ಟಿಯ
ಅಂತ-ರ್ದೃ-ಷ್ಟಿ-ಯಿಂದ
ಅಂತ-ರ್ಮು-ಖ-ವಾ-ಗ-ತೊ-ಡ-ಗಿತ್ತು
ಅಂತ-ರ್ಮು-ಖ-ವಾ-ಗಿದೆ
ಅಂತ-ರ್ಮು-ಖ-ವಾ-ಗುತ್ತ
ಅಂತ-ರ್ಮುಖಿ
ಅಂತ-ರ್ಮು-ಖಿ-ಗ-ಳಾ-ಗ-ತೊ-ಡ-ಗಿ-ದ್ದರು
ಅಂತ-ರ್ಮು-ಖಿ-ಗ-ಳಾ-ಗಿ-ದ್ದರು
ಅಂತ-ರ್ಮು-ಖಿ-ಯಾಗಿ
ಅಂತ-ರ್ಯ-ದಲ್ಲಿ
ಅಂತ-ರ್ಯಾ-ಮಿ-ಯಾ-ಗಿ-ರುವ
ಅಂತ-ರ್ಯು-ದ್ಧಕ್ಕೆ
ಅಂತಲೇ
ಅಂತ-ಶ್ಶಕ್ತಿ
ಅಂತ-ಶ್ಶಕ್ತಿಯ
ಅಂತ-ಶ್ಶಕ್ತಿ-ಯಲ್ಲಿ
ಅಂತ-ಸ್ತಿಗೆ
ಅಂತ-ಸ್ತಿನ
ಅಂತ-ಸ್ತಿ-ನಲ್ಲಿ
ಅಂತಸ್ತು
ಅಂತ-ಸ್ಸತ್ತ್ವ-ದಿಂದ
ಅಂತ-ಸ್ಸತ್ವ-ವನ್ನು
ಅಂತಹ
ಅಂತಿಮ
ಅಂತಿ-ಮ-ವಲ್ಲ
ಅಂತೂ
ಅಂತೆ-ಕಂ-ತೆಯ
ಅಂತೆಯೆ
ಅಂತೆಯೇ
ಅಂತೆಲ್ಲ
ಅಂತ್ಯ
ಅಂತ್ಯಕ್ಕೆ
ಅಂತ್ಯ-ಕ್ರಿಯೆ
ಅಂತ್ಯ-ಜ-ರಿ-ಗಾಗಿ
ಅಂತ್ಯದ
ಅಂತ್ಯ-ದಲ್ಲಿ
ಅಂತ್ಯ-ಭಾ-ಗ-ದಲ್ಲಿ
ಅಂತ್ಯ-ವನ್ನು
ಅಂಥ
ಅಂಥ-ದನ್ನು
ಅಂಥದು
ಅಂಥದೇ
ಅಂಥ-ದೊಂದು
ಅಂಥವ-ನಿಂದ
ಅಂಥವನು
ಅಂಥವರ
ಅಂಥವ-ರನ್ನು
ಅಂಥವ-ರ-ಲ್ಲೊಬ್ಬ
ಅಂಥವ-ರಾ-ಗಿ-ದ್ದರು
ಅಂಥವ-ರಿ-ಗಾಗಿ
ಅಂಥವ-ರಿ-ಗಾ-ಗಿಯೇ
ಅಂಥವ-ರಿಗೆ
ಅಂಥವ-ರಿ-ಗೆಲ್ಲ
ಅಂಥವರು
ಅಂಥವರೆ-ಲ್ಲರ
ಅಂಥವು-ಗಳ
ಅಂಥವು-ಗಳನ್ನು
ಅಂಥವು-ಗಳನ್ನೆಲ್ಲ
ಅಂಥಾ
ಅಂಥಾದ್ದು
ಅಂದ
ಅಂದ-ಮೇಲೆ
ಅಂದರೆ
ಅಂದ-ವಾಗಿ
ಅಂದ-ಹಾಗೆ
ಅಂದಾಜು
ಅಂದಿ-ಗಿಂತ
ಅಂದಿಗೂ
ಅಂದಿಗೆ
ಅಂದಿನ
ಅಂದಿ-ನ-ವ-ರೆಗೂ
ಅಂದಿ-ನಿಂದ
ಅಂದಿ-ನಿಂ-ದಲೂ
ಅಂದಿ-ನಿಂ-ದಲೇ
ಅಂದು
ಅಂದು-ಕೊಂಡೆ
ಅಂದೇ
ಅಂಧ
ಅಂಧ-ಕಾ-ರ-ದಲ್ಲಿ
ಅಂಧ-ಕಾ-ರ-ಮ-ಯ-ವಾಗಿ
ಅಂಧಾ-ನು-ಕ-ರಣೆ
ಅಂಧಾ-ನು-ಕ-ರ-ಣೆ-ಯಿಂದ
ಅಂಬಾ-ಲಾಗೆ
ಅಂಬಿಗ
ಅಂಬಿ-ಗನ
ಅಂಬೋಣ
ಅಂಶ
ಅಂಶಕ್ಕೆ
ಅಂಶ-ಗಳ
ಅಂಶ-ಗಳನ್ನು
ಅಂಶ-ಗಳನ್ನೆಲ್ಲ
ಅಂಶ-ಗಳಲ್ಲಿ
ಅಂಶ-ಗ-ಳಾ-ವುವು
ಅಂಶ-ಗ-ಳಿವೆ
ಅಂಶ-ಗಳು
ಅಂಶ-ಗಳೂ
ಅಂಶ-ಗ-ಳೆಲ್ಲ
ಅಂಶ-ಗಳೇ
ಅಂಶ-ಗ-ಳೊಂ-ದಿಗೆ
ಅಂಶದ
ಅಂಶ-ದಿಂ-ದಾಗಿ
ಅಂಶ-ವಂತೂ
ಅಂಶ-ವನ್ನು
ಅಂಶ-ವನ್ನೂ
ಅಂಶ-ವನ್ನೇ
ಅಂಶ-ವಾ-ಗಿತ್ತು
ಅಂಶ-ವಾದ
ಅಂಶ-ವಾ-ದರೆ
ಅಂಶ-ವಿ-ತ್ತು-ಸ್ವಾ-ಮೀ-ಜಿ-ಯ-ವರು
ಅಂಶ-ವಿದೆ
ಅಂಶವೂ
ಅಂಶ-ವೆಂದರೆ
ಅಂಶ-ವೇನೆಂದರೆ
ಅಕ-ಳಂ-ಕನೋ
ಅಕ-ಸ್ಮಾ-ತ್ತಾಗಿ
ಅಕಾ-ಡೆಮಿ
ಅಕಾ-ರಣ
ಅಕಾರಣ-ವಾಗಿ
ಅಕಾಲ
ಅಕಾ-ಲಿಕ
ಅಕ್ಕ
ಅಕ್ಕ-ತಂ-ಗಿ-ಯರು
ಅಕ್ಕ-ನಂ-ತೆಯೇ
ಅಕ್ಕ-ಪಕ್ಕ
ಅಕ್ಕ-ಪಕ್ಕ-ದ-ಲ್ಲಿ-ರು-ವ-ವರೆಲ್ಲ
ಅಕ್ಕಿ
ಅಕ್ಟೋ-ಬರ್
ಅಕ್ಬ-ರ-ನಿಗೆ
ಅಕ್ಬರ್
ಅಕ್ಷಮ್ಯ
ಅಕ್ಷ-ಮ್ಯ-ವಾದ
ಅಕ್ಷ-ರ-ದಾನ
ಅಕ್ಷ-ರ-ವನ್ನೂ
ಅಕ್ಷ-ರ-ವಿದ್ಯಾ
ಅಕ್ಷ-ರ-ವ್ಯ-ತ್ಯಾ-ಸ-ವನ್ನೇ
ಅಕ್ಷ-ರಶಃ
ಅಕ್ಷ-ರಾ-ಭ್ಯಾ-ಸ-ಕ್ಕೆಂದು
ಅಖಂಡ
ಅಖಂ-ಡಾ-ನಂದ
ಅಖಂಡಾನಂದ-ರಿಗೆ
ಅಖಂಡಾನಂದರು
ಅಖಿಲ
ಅಗತ್ಯ
ಅಗತ್ಯ-ವನ್ನು
ಅಗತ್ಯ-ವಾಗಿ
ಅಗತ್ಯ-ವಾ-ದರೂ
ಅಗತ್ಯ-ವಾ-ದಲ್ಲಿ
ಅಗತ್ಯ-ವಿತ್ತು
ಅಗತ್ಯ-ವಿಲ್ಲ
ಅಗತ್ಯವೇ
ಅಗತ್ಯ-ವೇನೂ
ಅಗಲ
ಅಗ-ಲ-ಎ-ತ್ತ-ರದ
ಅಗ-ಲ-ಬೇ-ಕಾ-ದೀ-ತೆಂದು
ಅಗ-ಲ-ವಾದ
ಅಗ-ಲ-ವಿದೆ
ಅಗ-ಲಿ-ದರೆ
ಅಗ-ಲು-ವಿ-ಕೆ-ಯನ್ನು
ಅಗ-ಲು-ವಿ-ಕೆ-ಯಿಂದ
ಅಗಾಧ
ಅಗಾ-ಧ-ತೆ-ಯನ್ನು
ಅಗಾ-ಧ-ತೆ-ಯಿದು
ಅಗಾ-ಧ-ತೆಯು
ಅಗಾ-ಧ-ವಾ-ಗಿದೆ
ಅಗಾ-ಧ-ವಾ-ಗಿ-ದ್ದಿ-ರ-ಬೇ-ಕೆಂದು
ಅಗಾ-ಧ-ವಾ-ದದ್ದು
ಅಗಿ-ಯು-ವು-ದಕ್ಕೂ
ಅಗೆತ
ಅಗೆದು
ಅಗೋ
ಅಗೋ-ಚರ
ಅಗೌ-ರವ
ಅಗೌ-ರ-ವ-ವನ್ನು
ಅಗ್ಗದ
ಅಗ್ಗಿ
ಅಗ್ಗಿ-ಷ್ಟಿ-ಕೆಗೆ
ಅಗ್ಗಿ-ಷ್ಟಿ-ಕೆಯ
ಅಗ್ಗಿ-ಷ್ಟಿ-ಕೆ-ಯನ್ನು
ಅಗ್ನಿ
ಅಗ್ನಿ-ಕುಂ-ಡದ
ಅಗ್ನಿ-ಕುಂ-ಡ-ದಲ್ಲಿ
ಅಗ್ನಿಗೆ
ಅಗ್ನಿ-ಚಿ-ಹ್ನೆಯ
ಅಗ್ನಿ-ದೇವ
ಅಗ್ನಿ-ಪರೀಕ್ಷೆ-ಯ
ಅಗ್ನಿ-ಪ-ರ್ವ-ತದ
ಅಗ್ನಿ-ಪ-ರ್ವ-ತ-ದಿಂದ
ಅಗ್ನಿ-ಪ-ರ್ವ-ತ-ವನ್ನು
ಅಗ್ನಿ-ಪ-ರ್ವ-ತ-ವಿ-ರು-ವುದು
ಅಗ್ನಿ-ಮಂ-ತ್ರೋ-ಪ-ದೇ-ಶ-ವನ್ನು
ಅಗ್ನಿ-ಮುಖ
ಅಗ್ನಿ-ಮು-ಖಿ-ಯಾಗಿ
ಅಗ್ನಿ-ಯಂತೆ
ಅಗ್ನಿ-ಯನ್ನು
ಅಗ್ನಿಯು
ಅಗ್ನಿ-ಸ-ದೃಶ
ಅಗ್ನಿ-ಸ-ದೃ-ಶ-ವಾದ
ಅಗ್ನಿ-ಸ್ಪರ್ಶ
ಅಗ್ರ-ಗಣ್ಯ
ಅಗ್ರ-ಗ-ಣ್ಯನೋ
ಅಗ್ರ-ಸ್ಥಾನ
ಅಚಲ
ಅಚ-ಲ-ರಾಗಿ
ಅಚ-ಲ-ವಾಗಿ
ಅಚ-ಲ-ವಾದ
ಅಚ-ಲಾ-ನಂದ
ಅಚಾ-ಬಾಲ್
ಅಚಾ-ಬಾ-ಲ್ನಲ್ಲಿ
ಅಚಾ-ಬಾ-ಲ್ನ-ಲ್ಲಿನ
ಅಚ್ಚ
ಅಚ್ಚರಿ
ಅಚ್ಚರಿಗೆ
ಅಚ್ಚರಿ-ಗೊಂ-ಡ-ರಾ-ದರೂ
ಅಚ್ಚರಿಯ
ಅಚ್ಚರಿ-ಯ-ನ್ನುಂ-ಟು-ಮಾ-ಡಿತು
ಅಚ್ಚರಿ-ಯಿಂದ
ಅಚ್ಚರಿ-ಯೇ-ನಿದೆ
ಅಚ್ಚರಿ-ಯೇ-ನಿಲ್ಲ
ಅಚ್ಚರಿ-ಯೇನೂ
ಅಚ್ಚ-ಳಿ-ಯದ
ಅಚ್ಚಿನ
ಅಚ್ಚು
ಅಚ್ಚೊತ್ತಿ
ಅಚ್ಯು-ತಾ-ನಂ-ದ-ರಿಗೆ
ಅಚ್ಯು-ತಾ-ನಂ-ದರು
ಅಚ್ಯು-ತಾ-ನಂ-ದ-ರೊಂ-ದಿಗೆ
ಅಜಯ
ಅಜ-ಯನ
ಅಜ-ಯ-ನಿಗೆ
ಅಜ-ಯ-ಹರಿ
ಅಜಾ-ಗ-ರೂ-ಕ-ತೆಯೂ
ಅಜಿತ್
ಅಜಿ-ತ್ಸಿಂಗ
ಅಜಿ-ತ್ಸಿಂ-ಗನ
ಅಜಿ-ತ್ಸಿಂ-ಗ-ನಿ-ಗಂತೂ
ಅಜಿ-ತ್ಸಿಂ-ಗನು
ಅಜಿ-ತ್ಸಿಂ-ಗನೂ
ಅಜಿ-ತ್ಸಿಂ-ಗನೆ
ಅಜಿ-ತ್ಸಿಂ-ಗ-ನೊಂ-ದಿಗೆ
ಅಜಿ-ತ್ಸಿಂಗ್
ಅಜೀ-ರ್ಣ-ವಾ-ಗು-ವ-ಷ್ಟಿದೆ
ಅಜ್ಜಿ
ಅಜ್ಜಿಯ
ಅಜ್ಜಿ-ಯನ್ನು
ಅಜ್ಞಾನದ
ಅಜ್ಞಾನ-ದಲ್ಲಿ
ಅಜ್ಞಾನ-ದಿಂ-ದಲೂ
ಅಜ್ಞಾನ-ವನ್ನು
ಅಜ್ಞಾನವು
ಅಜ್ಞಾನಿ-ಗ-ಳಿ-ಗೆ-ದೀ-ನ-ದ-ಲಿತ
ಅಜ್ಞಾನಿ-ಗಳು
ಅಜ್ಮೀರ್
ಅಟ್ಟಲೂ
ಅಟ್ಟ-ಹಾ-ಸದ
ಅಟ್ಟ-ಹಾ-ಸ-ದಿಂದ
ಅಟ್ಟ-ಹಾ-ಸ-ವನ್ನೂ
ಅಟ್ಟ-ಹಾ-ಸ-ವಿ-ರ-ಲಿಲ್ಲ
ಅಟ್ಟು-ತ್ತೇನೆ
ಅಟ್ಲಾಂ-ಟಿಕ್
ಅಡ-ಕ-ವಾ-ಗಿ-ತ್ತೆಂದು
ಅಡ-ಕ-ವಾ-ಗಿದೆ
ಅಡಗಿ
ಅಡಗಿ-ಕೊಂ-ಡಿದೆ
ಅಡಗಿ-ಕೊಂ-ಡಿವೆ
ಅಡಗಿತು
ಅಡಗಿತ್ತು
ಅಡಗಿದೆ
ಅಡಗಿ-ದೆ-ಯೆಂದು
ಅಡಗಿದ್ದ
ಅಡಗಿ-ರ-ಬ-ಹು-ದೇನೋ
ಅಡಗಿರು
ಅಡಗಿ-ರು-ತ್ತವೆ
ಅಡಗಿ-ರುವ
ಅಡಗಿ-ಸಿ-ಕೊಂ-ಡಿದ್ದ
ಅಡಗಿ-ಸಿ-ಕೊಂಡು
ಅಡಗಿ-ಸಿ-ಟ್ಟು-ಕೊ-ಳ್ಳು-ತ್ತದೆ
ಅಡಗಿ-ಹೋ-ಗಿತ್ತು
ಅಡಗಿ-ಹೋ-ದುವು
ಅಡಗಿ-ಹೋ-ಯಿತು
ಅಡ-ಚಣೆ
ಅಡ-ಚ-ಣೆ-ಗ-ಳಿ-ಲ್ಲ-ದಿ-ರು-ವುದು
ಅಡ-ಚ-ಣೆ-ಗಳೂ
ಅಡ-ಚ-ಣೆಯೇ
ಅಡಿ
ಅಡಿ-ಗಲ್ಲು
ಅಡಿಗೆ
ಅಡಿ-ಗೆ-ಗಾಗಿ
ಅಡಿ-ಗೆ-ಪಾ-ತ್ರೆ-ಗಳೇ
ಅಡಿ-ಗೆ-ಮನೆ
ಅಡಿ-ಗೆ-ಮ-ನೆಯ
ಅಡಿ-ಗೆ-ಮ-ನೆ-ಯನ್ನು
ಅಡಿ-ಗೆ-ಮ-ನೆ-ಯಲ್ಲಿ
ಅಡಿ-ಗೆ-ಮ-ನೆ-ಯೊ-ಳಗೆ
ಅಡಿ-ಗೆಯ
ಅಡಿ-ಗೆ-ಯನ್ನು
ಅಡಿ-ಗೆ-ಯನ್ನೇ
ಅಡಿ-ಗೆ-ಯ-ವನು
ಅಡಿ-ಗೆ-ಯ-ವಳು
ಅಡಿ-ಗೆ-ಯಾ-ಗು-ತ್ತಿ-ತ್ತೇನೋ
ಅಡಿ-ದಾ-ವರೆ-ಗಳಲ್ಲಿ
ಅಡಿ-ಮೇ-ಲಾ-ದುವು
ಅಡಿ-ಮೇಲು
ಅಡಿಯ
ಅಡಿ-ಯಾ-ಳ-ಲ್ಲವೆ
ಅಡಿ-ಯಿ-ಟ್ಟ-ಲ್ಲೆಲ್ಲ
ಅಡಿ-ಯಿಟ್ಟೆ
ಅಡೆ-ತ-ಡೆ-ಗಳನ್ನು
ಅಡೆ-ತ-ಡೆ-ಗ-ಳಿಂ-ದಾಗಿ
ಅಡೆ-ತ-ಡೆ-ಗಳು
ಅಡೆ-ತ-ಡೆ-ಗ-ಳೆ-ಲ್ಲವೂ
ಅಡೆ-ತ-ಡೆ-ಯಿ-ಲ್ಲದೆ
ಅಡ್ಡ-ಗೋ-ಡೆ-ಯೊಂ-ದರ
ಅಡ್ಡ-ಪ-ಟ್ಟಿ-ಗಳನ್ನೂ
ಅಡ್ಡ-ವಾಗಿ
ಅಡ್ಡಾ-ಡಿ-ಕೊಂಡು
ಅಡ್ಡಾಡು
ಅಡ್ಡಾ-ಡುತ್ತ
ಅಡ್ಡಾ-ಡು-ತ್ತಲೋ
ಅಡ್ಡಾ-ಡು-ತ್ತಿ-ದ್ದರು
ಅಡ್ಡಾ-ಡು-ತ್ತಿ-ದ್ದಾಗ
ಅಡ್ಡಾ-ಡು-ವಾಗ
ಅಡ್ಡಾ-ಡೋಣ
ಅಡ್ಡಿ
ಅಡ್ಡಿ-ಗಳು
ಅಡ್ಡಿ-ಯಾಗಿ
ಅಡ್ಡಿ-ಯುಂ-ಟು-ಮಾ-ಡ-ಬ-ಲ್ಲುದು
ಅಡ್ವೊ-ಕೇಟ್
ಅಣಿ-ಯಾ-ಗಿ-ರ-ಬೇಕು
ಅಣಿ-ಯಾ-ದರು
ಅಣು-ಅಣು
ಅಣು-ಕ-ಣದ
ಅಣು-ಗಾ-ತ್ರ-ವೆಂ-ಬಂತೆ
ಅಣು-ಭಾ-ಷ್ಯ-ವನ್ನು
ಅಣು-ವಿ-ನಷ್ಟು
ಅಣ್ಣ
ಅಣ್ಣ-ತ-ಮ್ಮಂ-ದಿರು
ಅಣ್ಣನ
ಅಣ್ಣಾ
ಅಣ್ವ-ಸ್ತ್ರ-ಗಳನ್ನು
ಅತಂ-ಸ್ಸ-ತ್ವವೇ
ಅತಿ
ಅತಿ-ಕ-ಷ್ಟದ
ಅತಿ-ಕ್ರಮ
ಅತಿ-ಕ್ರ-ಮಿ-ಸ-ಬಲ್ಲ
ಅತಿ-ಕ್ರ-ಮಿಸಿ
ಅತಿ-ಗಳನ್ನೂ
ಅತಿಥಿ
ಅತಿ-ಥಿ-ಗಳನ್ನು
ಅತಿ-ಥಿ-ಗ-ಳಾಗಿ
ಅತಿ-ಥಿ-ಗಳಿಂದ
ಅತಿ-ಥಿ-ಗಳು
ಅತಿ-ಥಿ-ಗೃ-ಹಕ್ಕೆ
ಅತಿ-ಥಿ-ಗೃ-ಹ-ದಲ್ಲಿ
ಅತಿ-ಥಿ-ಯಾಗಿ
ಅತಿ-ಥಿ-ಯಾ-ಗಿ-ದ್ದಾಗ
ಅತಿ-ದೊಡ್ಡ
ಅತಿ-ಧಾ-ರಾ-ಳ-ವಂ-ತಿ-ಕೆಯೇ
ಅತಿ-ಮಾ-ನುಷ
ಅತಿ-ಯಾಗಿ
ಅತಿ-ಯಾದ
ಅತಿ-ಯಾ-ದರೆ
ಅತಿ-ಯಾ-ದೀತು
ಅತಿ-ಯಾ-ಯಿತು
ಅತಿ-ವಿ-ನ-ಯ-ದಿಂದ
ಅತಿ-ಶಯ
ಅತಿ-ಶ-ಯ-ತೆ-ಯನ್ನು
ಅತಿ-ಶ-ಯೋ-ಕ್ತಿ-ಯಲ್ಲ
ಅತೀತ
ಅತೀ-ತ-ದಲ್ಲಿ
ಅತೀ-ತ-ರಾ-ಗ-ಬೇ-ಕಾ-ಗು-ತ್ತದೆ
ಅತೀ-ತ-ರಾ-ಗಿ-ದ್ದ-ರೆಂ-ಬು-ದೇನೋ
ಅತೀ-ತ-ವಾಗಿ
ಅತೀ-ತ-ವಾದ
ಅತೀ-ತ-ವಾ-ದದ್ದು
ಅತೀವ
ಅತು-ಲಾ-ನಂ-ದರು
ಅತು-ಲಾ-ನಂ-ದ-ರೆಂದು
ಅತ್ತ
ಅತ್ತ-ದ್ದಾ-ಯಿತು
ಅತ್ತಿಂ-ದಿತ್ತ
ಅತ್ತು-ಕೊಂಡೇ
ಅತ್ತೇ
ಅತ್ಯ
ಅತ್ಯಂತ
ಅತ್ಯ-ಗತ್ಯ
ಅತ್ಯ-ಗ-ತ್ಯ-ವಾಗಿ
ಅತ್ಯ-ಗ-ತ್ಯ-ವಾ-ಗಿದ್ದ
ಅತ್ಯ-ಗ-ತ್ಯ-ವಾದ
ಅತ್ಯ-ಗ-ತ್ಯ-ವಾ-ದಾಗ
ಅತ್ಯ-ಗ-ತ್ಯ-ವಾ-ದು-ದೇ-ನೆಂ-ದರೆ
ಅತ್ಯ-ದ್ಭುತ
ಅತ್ಯ-ಧಿಕ
ಅತ್ಯ-ನರ್ಘ್ಯ
ಅತ್ಯ-ಪೂ-ರ್ವ-ವಾ-ದು-ದೆ-ನ್ನ-ಬ-ಹುದು
ಅತ್ಯ-ಮೂಲ್ಯ
ಅತ್ಯಲ್ಪ
ಅತ್ಯಾ-ಚಾ-ರ-ವೆ-ಸಗಿ
ಅತ್ಯಾ-ದ-ರ-ದಿಂದ
ಅತ್ಯಾ-ಧು-ನಿಕ
ಅತ್ಯಾ-ಧು-ನಿ-ಕ-ವಾದ
ಅತ್ಯಾ-ನಂ-ದದ
ಅತ್ಯಾ-ನಂ-ದ-ವನ್ನು
ಅತ್ಯಾ-ನಂ-ದ-ವ-ನ್ನುಂಟು
ಅತ್ಯಾ-ನಂ-ದ-ವಾ-ಯಿತು
ಅತ್ಯಾ-ನಂ-ದಿ-ತ-ರಾದ
ಅತ್ಯಾ-ವ-ಶ್ಯಕ
ಅತ್ಯಾ-ವ-ಶ್ಯ-ಕ-ವಾ-ಗಿದೆ
ಅತ್ಯಾ-ಶ್ಚರ್ಯ
ಅತ್ಯಾ-ಶ್ಚ-ರ್ಯ-ಕ-ರ-ವಾಗಿ
ಅತ್ಯಾ-ಶ್ಚ-ರ್ಯ-ಪ-ಟ್ಟರು
ಅತ್ಯಾ-ಶ್ಚ-ರ್ಯ-ಭ-ರಿ-ತ-ರಾ-ದರು
ಅತ್ಯಾ-ಶ್ಚ-ರ್ಯ-ವಾ-ಗು-ತ್ತದೆ
ಅತ್ಯಾ-ಸ-ಕ್ತಿ-ಯಿಂದ
ಅತ್ಯು
ಅತ್ಯುಚ್ಚ
ಅತ್ಯು-ತ್ಕೃಷ್ಟ
ಅತ್ಯು-ತ್ಕೃ-ಷ್ಟ-ವಾದ
ಅತ್ಯು-ತ್ಕೃ-ಷ್ಟ-ವಾ-ದುದು
ಅತ್ಯು-ತ್ತಮ
ಅತ್ಯು-ತ್ತ-ಮ-ವಾದ
ಅತ್ಯು-ತ್ತ-ಮ-ವಾ-ದ-ದ್ದನ್ನೇ
ಅತ್ಯು-ತ್ಸಾ-ಹ-ದಿಂದ
ಅತ್ಯು-ನ್ನತ
ಅತ್ಯು-ನ್ನ-ತ-ವಾದ
ಅತ್ಯು-ನ್ನ-ತ-ವಾ-ದು-ದನ್ನು
ಅತ್ಯು-ನ್ನ-ತವೂ
ಅಥವಾ
ಅಥ-ವಾ-ವಿ-ರ-ಜಾ-ಹೋ-ಮ-ವೊಂ-ದನ್ನು
ಅಥೆ-ನ್ಸಿ-ನತ್ತ
ಅಥೆ-ನ್ಸ್
ಅದ
ಅದಂತೂ
ಅದ-ಕ್ಕ-ನು-ಗು-ಣ-ವಾಗಿ
ಅದ-ಕ್ಕ-ನು-ಸಾ-ರ-ವಾಗಿ
ಅದ-ಕ್ಕಾಗಿ
ಅದ-ಕ್ಕಾ-ಗಿಯೇ
ಅದ-ಕ್ಕಿಂತ
ಅದ-ಕ್ಕಿಂ-ತಲೂ
ಅದ-ಕ್ಕೀಗ
ಅದ-ಕ್ಕು-ತ್ತ-ರ-ವಾಗಿ
ಅದಕ್ಕೂ
ಅದಕ್ಕೆ
ಅದ-ಕ್ಕೆಲ್ಲ
ಅದ-ಕ್ಕೇನು
ಅದ-ಕ್ಕೇನೂ
ಅದ-ಕ್ಕೊಂದು
ಅದ-ಕ್ಕೊಪ್ಪಿ
ಅದ-ಕ್ಕೊ-ಪ್ಪಿ-ಕೊ-ಳ್ಳ-ಲಿಲ್ಲ
ಅದ-ಕ್ಕೊ-ಪ್ಪುವ
ಅದ-ನ್ನಲ್ಲ
ಅದ-ನ್ನ-ವ-ನಿಗೆ
ಅದ-ನ್ನ-ವರು
ಅದ-ನ್ನ-ವಳು
ಅದ-ನ್ನೀಗ
ಅದನ್ನು
ಅದನ್ನೆ
ಅದ-ನ್ನೆಂದೂ
ಅದ-ನ್ನೆಲ್ಲ
ಅದನ್ನೇ
ಅದ-ನ್ನೇನೂ
ಅದ-ನ್ನೇರಿ
ಅದ-ನ್ನೋ-ದು-ವಲ್ಲಿ
ಅದಮ್ಯ
ಅದರ
ಅದ-ರಂತೆ
ಅದ-ರಂ-ತೆಯೇ
ಅದ-ರ-ದರ
ಅದ-ರದೇ
ಅದ-ರಲ್ಲಿ
ಅದ-ರ-ಲ್ಲಿದ್ದ
ಅದ-ರ-ಲ್ಲಿನ
ಅದ-ರ-ಲ್ಲಿ-ರ-ಲಾ-ರದು
ಅದ-ರ-ಲ್ಲಿ-ರುವ
ಅದ-ರಲ್ಲೂ
ಅದ-ರಲ್ಲೇ
ಅದ-ರ-ಲ್ಲೇ-ನಾ-ದರೂ
ಅದ-ರಿಂದ
ಅದ-ರಿಂ-ದಲೂ
ಅದ-ರಿಂ-ದಾ-ಚೆಗೆ
ಅದ-ರಿಂದೇ
ಅದ-ರಿಂ-ದೇ-ನಂತೆ
ಅದ-ರಿಂ-ದೇನು
ಅದ-ರಿಂ-ದೇನೂ
ಅದರೂ
ಅದರೆ
ಅದ-ರೊಂ-ದಿಗೆ
ಅದ-ರೊಂ-ದಿಗೇ
ಅದ-ರೊ-ಳ-ಗಿನ
ಅದ-ರೊ-ಳ-ಗಿ-ನಿಂದ
ಅದ-ಲ್ಲವೇ
ಅದಾ-ಗಲೇ
ಅದಾ-ಗಿಯೇ
ಅದಾ-ಗು-ವುದು
ಅದಾ-ದ-ನಂ-ತರ
ಅದಾ-ವುದೂ
ಅದಿನ್ನೂ
ಅದು
ಅದು-ಇದು
ಅದು-ಆ-ತ್ಮ-ಸಾ-ಕ್ಷಾ-ತ್ಕಾರ
ಅದುಮಿ
ಅದುರಿ
ಅದು-ವರೆ-ಗಿನ
ಅದು-ವರೆ-ವಿಗೂ
ಅದು-ಸ-ಹ-ಮಾ-ನ-ವ-ರಿಗೆ
ಅದೂ
ಅದೃಷ್ಟ
ಅದೃ-ಷ್ಟಕ್ಕೆ
ಅದೃ-ಷ್ಟ-ವಂ-ತರ
ಅದೃ-ಷ್ಟ-ವಂ-ತ-ರೆಂದು
ಅದೃ-ಷ್ಟ-ವ-ಶಾತ್
ಅದೃ-ಷ್ಟ-ವಾ-ದರೆ
ಅದೃ-ಷ್ಟ-ವೇನೂ
ಅದೃ-ಷ್ಟ-ಶಾಲೀ
ಅದೆಂ-ತಹ
ಅದೆಂಥ
ಅದೆಂ-ದಾ-ದರೂ
ಅದೆಲ್ಲ
ಅದೆ-ಲ್ಲವೂ
ಅದೆಷ್ಟು
ಅದೆಷ್ಟೋ
ಅದೇ
ಅದೇ-ಅ-ವರು
ಅದೇಕೆ
ಅದೇ-ಕೆಂ-ದರೆ
ಅದೇಕೋ
ಅದೇ-ನಾ-ದರೂ
ಅದೇ-ನಾ-ನಂದ
ಅದೇನು
ಅದೇನೂ
ಅದೇ-ನೆಂ-ದರೆ
ಅದೇನೇ
ಅದೇನೋ
ಅದೊ
ಅದೊಂದು
ಅದೊಂದೇ
ಅದೋ
ಅದ್ಧೂರಿ
ಅದ್ಧೂ-ರಿಯ
ಅದ್ಭುತ
ಅದ್ಭು-ತ-ರಮ್ಯ
ಅದ್ಭು-ತ-ಕಾ-ವ್ಯ-ವಾದ
ಅದ್ಭು-ತ-ಗಳನ್ನು
ಅದ್ಭು-ತ-ಗಳು
ಅದ್ಭು-ತ-ಗ-ಳೆಲ್ಲ
ಅದ್ಭು-ತ-ಯಂ-ತ್ರ-ವನ್ನೋ
ಅದ್ಭು-ತ-ವನ್ನು
ಅದ್ಭು-ತ-ವ-ಲ್ಲವೆ
ಅದ್ಭು-ತ-ವಾಗಿ
ಅದ್ಭು-ತ-ವಾ-ಗಿತ್ತು
ಅದ್ಭು-ತ-ವಾ-ಗಿದೆ
ಅದ್ಭು-ತ-ವಾ-ಗಿ-ರ-ಬ-ಹುದು
ಅದ್ಭು-ತ-ವಾದ
ಅದ್ಭು-ತ-ವಾ-ದ-ದ್ದನ್ನು
ಅದ್ಭು-ತ-ವಾ-ದದ್ದು
ಅದ್ಭು-ತ-ವಾ-ದುದು
ಅದ್ಭು-ತ-ವೆಂದರೆ
ಅದ್ಭು-ತವೇ
ಅದ್ಭುತಾ
ಅದ್ಭು-ತಾ-ನಂ-ದರ
ಅದ್ಭು-ತಾ-ನಂ-ದ-ರಂ-ತೆಯೇ
ಅದ್ಭು-ತಾ-ನಂ-ದ-ರನ್ನೂ
ಅದ್ಭು-ತಾ-ನಂ-ದರು
ಅದ್ಯೈವ
ಅದ್ವಿ-ತೀಯ
ಅದ್ವಿ-ತೀ-ಯ-ವಾ-ದುದು
ಅದ್ವೈತ
ಅದ್ವೈ-ತಕ್ಕೆ
ಅದ್ವೈ-ತ-ಕ್ಕೆ-ಕೇ-ವಲ
ಅದ್ವೈ-ತ-ಕ್ಕೆ-ಮೀ-ಸ-ಲಾಗಿದೆ
ಅದ್ವೈ-ತದ
ಅದ್ವೈ-ತ-ಭಾ-ವ-ವನ್ನು
ಅದ್ವೈ-ತ-ವನ್ನು
ಅದ್ವೈ-ತ-ವಾ-ದ-ವನ್ನು
ಅದ್ವೈ-ತ-ವಾ-ದ-ವನ್ನೂ
ಅದ್ವೈ-ತ-ವಾ-ದಿ-ಗ-ಳೆಂ-ಬುದೇ
ಅದ್ವೈ-ತ-ವೆಂಬ
ಅದ್ವೈ-ತ-ವೆಂ-ಬುದು
ಅದ್ವೈ-ತವೇ
ಅದ್ವೈ-ತ-ವೇ-ದಾಂ-ತದ
ಅದ್ವೈ-ತ-ವೇ-ದಾಂ-ತ-ವನ್ನು
ಅದ್ವೈ-ತ-ವೇ-ದಾಂ-ತವು
ಅದ್ವೈ-ತ-ವೊಂದೇ
ಅದ್ವೈತಾ
ಅದ್ವೈ-ತಾ-ನಂದ
ಅದ್ವೈ-ತಾ-ನಂ-ದ-ರಿಗೆ
ಅದ್ವೈ-ತಾ-ನಂ-ದರು
ಅದ್ವೈ-ತಾ-ನಂ-ದ-ರೊಂ-ದಿಗೆ
ಅದ್ವೈ-ತಾ-ನು-ಭ-ವದ
ಅದ್ವೈ-ತಾ-ವ-ಲಂ-ಬಿ-ಗಳಲ್ಲಿ
ಅದ್ವೈ-ತಾ-ಶ್ರಮ
ಅದ್ವೈ-ತಾ-ಶ್ರ-ಮಕ್ಕೆ
ಅದ್ವೈ-ತಾ-ಶ್ರ-ಮದ
ಅದ್ವೈ-ತಾ-ಶ್ರ-ಮ-ದಲ್ಲಿ
ಅದ್ವೈ-ತಾ-ಶ್ರ-ಮ-ದ-ಲ್ಲಿದ್ದ
ಅದ್ವೈ-ತಾ-ಶ್ರ-ಮ-ವನ್ನು
ಅದ್ವೈತಿ
ಅದ್ವೈ-ತಿ-ಗಳೇ
ಅದ್ವೈ-ತಿ-ಯಾಗಿ
ಅದ್ವೈ-ತಿ-ಯಾ-ಗಿ-ದ್ದರೂ
ಅದ್ವೈ-ತಿ-ಯಾದ
ಅಧಃ-ಪ-ತ-ನಕ್ಕೆ
ಅಧಃ-ಪ-ತ-ನದ
ಅಧಃ-ಪಾ-ತಾ-ಳ-ಕ್ಕಿ-ಳಿ-ಯ-ದಂತೆ
ಅಧ-ಮಾ-ಧ-ಮ-ರಂತೆ
ಅಧರ್ಮ
ಅಧಿ
ಅಧಿಕ
ಅಧಿ-ಕ-ವಾಗಿ
ಅಧಿ-ಕ-ವಾ-ಗಿ-ತ್ತೆಂ-ದರೆ
ಅಧಿ-ಕ-ವಾ-ಗಿ-ದೆ-ಯೆಂ-ದರೆ
ಅಧಿ-ಕ-ವಾ-ಗಿ-ಬಿ-ಟ್ಟಿತು
ಅಧಿ-ಕ-ವಾ-ಗು-ವಂತೆ
ಅಧಿ-ಕಾರ
ಅಧಿ-ಕಾ-ರ-ಸ-ವ-ಲ-ತ್ತು-ಗ-ಳಿ-ಗಾಗಿ
ಅಧಿ-ಕಾ-ರ-ಗಳ
ಅಧಿ-ಕಾ-ರ-ಗ-ಳೆಲ್ಲ
ಅಧಿ-ಕಾ-ರದ
ಅಧಿ-ಕಾ-ರ-ಯುತ
ಅಧಿ-ಕಾ-ರ-ವಾ-ದದ
ಅಧಿ-ಕಾ-ರ-ವಾ-ದ-ವೆಂದರೆ
ಅಧಿ-ಕಾ-ರ-ವಿದೆ
ಅಧಿ-ಕಾ-ರ-ವಿಲ್ಲ
ಅಧಿ-ಕಾರಿ
ಅಧಿ-ಕಾ-ರಿ-ಗಳ
ಅಧಿ-ಕಾ-ರಿ-ಗ-ಳಾಗಿ
ಅಧಿ-ಕಾ-ರಿ-ಗ-ಳಾ-ಗಿಯೋ
ಅಧಿ-ಕಾ-ರಿ-ಗ-ಳಾ-ದ-ವರು
ಅಧಿ-ಕಾ-ರಿ-ಗಳಿಂದ
ಅಧಿ-ಕಾ-ರಿ-ಗ-ಳಿಗೆ
ಅಧಿ-ಕಾ-ರಿ-ಗಳು
ಅಧಿ-ಕಾ-ರಿಯ
ಅಧಿ-ಕಾ-ರಿ-ಯಂತೆ
ಅಧಿ-ಕಾ-ರಿ-ಯ-ನ್ನಾಗಿ
ಅಧಿ-ಕಾ-ರಿ-ಯಾ-ಗಿದ್ದ
ಅಧಿ-ಕಾ-ರಿ-ಯಾ-ಗಿದ್ದು
ಅಧಿ-ಕಾ-ರಿ-ಯಾ-ಗಿ-ರಲಿ
ಅಧಿ-ಕಾ-ರಿ-ಯಾದ
ಅಧಿ-ಕಾ-ರಿ-ಯೊ-ಬ್ಬರು
ಅಧಿ-ಕಾ-ರಿ-ವ-ರ್ಗ-ದ-ವರ
ಅಧಿ-ವೇ-ಶನ
ಅಧಿ-ವೇ-ಶ-ನಕ್ಕೆ
ಅಧಿ-ವೇ-ಶ-ನ-ಗ-ಳಾ-ಗಿ-ದ್ದುವು
ಅಧಿ-ವೇ-ಶ-ನದ
ಅಧಿ-ವೇ-ಶ-ನ-ದಲ್ಲಿ
ಅಧೀನ
ಅಧೀ-ನ-ದ-ಲ್ಲಿದೆ
ಅಧೀ-ನ-ವಾ-ಗಿ-ರ-ಬೇಕು
ಅಧೀ-ನ-ವಾ-ಗಿ-ರು-ವ-ವ-ರೆಗೆ
ಅಧೀ-ನ-ವಾ-ಗಿ-ರು-ವುದು
ಅಧೀ-ನ-ವಾ-ದಂ-ಥವು
ಅಧೀ-ನ-ವೆಂ-ಬು-ದನ್ನು
ಅಧೈರ್ಯ
ಅಧೋ-ಗ-ತಿ-ಗಿಳಿ-ಯು-ತ್ತಿ-ರುವ
ಅಧೋ-ಗ-ತಿ-ಗಿಳಿ-ಯು-ವುದನ್ನು
ಅಧೋ-ಗ-ತಿ-ಗಿಳಿ-ಸು-ವಂ-ತಹ
ಅಧ್ಯ
ಅಧ್ಯಕ್ಷ
ಅಧ್ಯ-ಕ್ಷತೆ
ಅಧ್ಯ-ಕ್ಷ-ತೆ-ಯಲ್ಲಿ
ಅಧ್ಯ-ಕ್ಷ-ನಾ-ಗಿದ್ದ
ಅಧ್ಯ-ಕ್ಷ-ರ-ನ್ನಾಗಿ
ಅಧ್ಯ-ಕ್ಷ-ರಾಗಿ
ಅಧ್ಯ-ಕ್ಷ-ರಾ-ಗಿ-ದ್ದರು
ಅಧ್ಯ-ಕ್ಷ-ರಾದ
ಅಧ್ಯ-ಕ್ಷ-ರಾ-ದರು
ಅಧ್ಯ-ಕ್ಷ-ರಾ-ದ-ವರು
ಅಧ್ಯ-ಕ್ಷರು
ಅಧ್ಯ-ಕ್ಷ-ಸ್ಥಾನ
ಅಧ್ಯ-ಕ್ಷ-ಸ್ಥಾ-ನಕ್ಕೆ
ಅಧ್ಯ-ಕ್ಷ-ಸ್ಥಾ-ನ-ದಿಂದ
ಅಧ್ಯ-ಕ್ಷ-ಸ್ಥಾ-ನವೆ
ಅಧ್ಯ-ಯನ
ಅಧ್ಯ-ಯ-ನಕ್ಕೆ
ಅಧ್ಯ-ಯ-ನದ
ಅಧ್ಯ-ಯ-ನ-ದಲ್ಲಿ
ಅಧ್ಯ-ಯ-ನ-ದಲ್ಲೂ
ಅಧ್ಯ-ಯ-ನ-ದಿಂದ
ಅಧ್ಯ-ಯ-ನ-ವನ್ನು
ಅಧ್ಯ-ಯ-ನವು
ಅಧ್ಯ-ಯ-ನವೇ
ಅಧ್ಯ-ಯ-ನಾ-ದಿ-ಗ-ಳಿಗೆ
ಅಧ್ಯ-ಯ-ನಾ-ದಿ-ಗ-ಳೊಂ-ದಿಗೇ
ಅಧ್ಯ-ಯಿ-ಸಲು
ಅಧ್ಯ-ಯಿಸಿ
ಅಧ್ಯಾತ್ಮ
ಅಧ್ಯಾ-ತ್ಮದ
ಅಧ್ಯಾ-ತ್ಮ-ದಲ್ಲಿ
ಅಧ್ಯಾ-ತ್ಮ-ವೆಂದು
ಅಧ್ಯಾ-ತ್ಮ-ವೆಂಬ
ಅಧ್ಯಾ-ತ್ಮ-ಶೀಲ
ಅಧ್ಯಾ-ತ್ಮ-ಶೀ-ಲ-ರಾ-ಗು-ವ-ವ-ರೆಗೆ
ಅಧ್ಯಾ-ಪ-ಕ-ರನ್ನು
ಅಧ್ಯಾಯ
ಅಧ್ಯಾ-ಯದ
ಅಧ್ಯಾ-ಯ-ವನ್ನು
ಅಧ್ಯಾ-ಯ-ವೊಂದು
ಅನಂತ
ಅನಂ-ತ-ಅ-ಖಂ-ಡ-ಅ-ವ್ಯ-ಕ್ತ-ಸ-ಚ್ಚಿ-ದಾ-ನಂದ
ಅನಂ-ತ-ತೆಗೆ
ಅನಂ-ತ-ನಾ-ಗಕ್ಕೆ
ಅನಂ-ತ-ನಾ-ಗದ
ಅನಂ-ತ-ನಾ-ಗ-ದಲ್ಲಿ
ಅನಂ-ತ-ನಾ-ಗ-ವನ್ನು
ಅನಂ-ತ-ಮೌ-ನ-ವನ್ನು
ಅನಂ-ತರ
ಅನಂ-ತ-ರದ
ಅನಂ-ತ-ರವೂ
ಅನಂ-ತ-ರವೇ
ಅನಂ-ತ-ಶಕ್ತಿ
ಅನಂ-ತ-ಶ-ಕ್ತಿ-ಯನ್ನು
ಅನಂ-ತ-ಸ್ವ-ರೂ-ಪರು
ಅನ-ಕ್ಷ-ರಸ್ಥ
ಅನತಿ
ಅನನು
ಅನ-ನು-ಕೂಲ
ಅನ-ನು-ಕೂ-ಲತೆ
ಅನ-ನು-ಕೂ-ಲ-ತೆ-ಗ-ಳಿಗೆ
ಅನ-ನು-ಕೂ-ಲ-ತೆ-ಗಳೂ
ಅನ-ನು-ಕೂ-ಲವೇ
ಅನರ್ಘ್ಯ
ಅನ-ರ್ಹರು
ಅನ-ವ-ರತ
ಅನಾ
ಅನಾ-ಗ-ರಿಕ
ಅನಾ-ಗ-ರಿ-ಕರ
ಅನಾಥ
ಅನಾ-ಥ-ರಾ-ದಂತೆ
ಅನಾ-ಥ-ರಿ-ಗೆ-ಪ-ತಿ-ತ-ರಿಗೆ
ಅನಾಥಾ
ಅನಾ-ಥಾ-ಲಯ
ಅನಾ-ಥಾ-ಶ್ರಮ
ಅನಾ-ದ-ರಕ್ಕೆ
ಅನಾ-ದಿ-ಕಾ-ಲ-ದಿಂದ
ಅನಾ-ದಿ-ಕಾ-ಲ-ದಿಂ-ದಲೂ
ಅನಾ-ದಿ-ಯಿಂದ
ಅನಾ-ದಿ-ಯಿಂ-ದಲೂ
ಅನಾ-ಮ-ಧೇಯ
ಅನಾ-ಮ-ಧೇ-ಯ-ರಾಗಿ
ಅನಾ-ರೋಗ್ಯ
ಅನಾ-ರೋ-ಗ್ಯಕ್ಕೆ
ಅನಾ-ರೋ-ಗ್ಯ-ಗ್ರ-ಸ್ತ-ರಾ-ಗಿದ್ದ
ಅನಾ-ರೋ-ಗ್ಯದ
ಅನಾ-ರೋ-ಗ್ಯ-ದಿಂ-ದಾಗಿ
ಅನಾ-ರೋ-ಗ್ಯ-ದೊಂ-ದಿಗೆ
ಅನಾ-ರೋ-ಗ್ಯ-ಪೀ-ಡಿ-ತ-ರಾ-ಗಿ-ದ್ದು-ದನ್ನು
ಅನಾ-ರೋ-ಗ್ಯ-ವಾ-ಗಿದೆ
ಅನಾ-ರೋ-ಗ್ಯ-ವಾ-ಗಿ-ದೆ-ಯೆಂದೂ
ಅನಾ-ರೋ-ಗ್ಯ-ವಿ-ದ್ದರೂ
ಅನಾ-ರೋ-ಗ್ಯ-ಸ್ಥಿತಿ
ಅನಾ-ವ-ಶ್ಯ-ಕ-ವಾದ
ಅನಾ-ಸಕ್ತಿ
ಅನಾ-ಸ-ಕ್ತಿ-ಭಾ-ವ-ದಿಂದ
ಅನಾ-ಹುತ
ಅನಿತ್ಯ
ಅನಿ-ತ್ಯ-ವಾದ
ಅನಿದ್ರೆ
ಅನಿ-ರೀ-ಕ್ಷಿತ
ಅನಿ-ರೀ-ಕ್ಷಿ-ತ-ವಾಗಿ
ಅನಿ-ರೀ-ಕ್ಷಿ-ತ-ವಾದ
ಅನಿ-ರೀ-ಕ್ಷಿ-ತ-ವಾ-ದದ್ದು
ಅನಿ-ರೀ-ಕ್ಷಿ-ತ-ವಾ-ದ-ದ್ದೇನೂ
ಅನಿ-ರ್ಬಂ-ಧಿತ
ಅನಿ-ರ್ವ-ಚ-ನೀಯ
ಅನಿ-ವಾರ್ಯ
ಅನಿ-ವಾ-ರ್ಯ-ವಾ-ಗು-ತ್ತದೆ
ಅನಿ-ವಾ-ರ್ಯ-ವಾ-ದುದು
ಅನಿ-ವಾ-ರ್ಯ-ವೆಂ-ಬುದು
ಅನಿ-ಸಿಕೆ
ಅನಿ-ಸಿ-ಕೆ-ಗ-ಳಿಗೇ
ಅನಿ-ಸು-ತ್ತದೆ
ಅನು
ಅನು-ಕಂಪೆ
ಅನು-ಕಂ-ಪೆಯ
ಅನು-ಕಂ-ಪೆ-ಯಿಂದ
ಅನು-ಕಂ-ಪೆ-ಯಿಂ-ದೊ-ಡ-ಗೂಡಿ
ಅನು-ಕಂ-ಪೆ-ಯಿ-ದೆಯೆ
ಅನು-ಕಂ-ಪೆ-ಯು-ಳ್ಳ-ವ-ರಾ-ಗಿ-ದ್ದರು
ಅನು-ಕ-ರ-ಣೆಯು
ಅನು-ಕ-ರ-ಣೆಯೂ
ಅನು-ಕ-ರಿ-ಸ-ಬೇ-ಕಾದ್ದು
ಅನು-ಕ-ರಿ-ಸಲು
ಅನು-ಕ-ರಿಸಿ
ಅನು-ಕ-ರಿ-ಸುವ
ಅನು-ಕ-ರಿ-ಸು-ವುದು
ಅನು-ಕೂಲ
ಅನು-ಕೂ-ಲ-ಕ-ರವಾ
ಅನು-ಕೂ-ಲ-ಕ-ರ-ವಾ-ಗಿತ್ತು
ಅನು-ಕೂ-ಲ-ಕ-ರ-ವಾದ
ಅನು-ಕೂ-ಲತೆ
ಅನು-ಕೂ-ಲ-ತೆ-ಗ-ಳಾ-ವುವೂ
ಅನು-ಕೂ-ಲ-ತೆ-ಗ-ಳುಳ್ಳ
ಅನು-ಕೂ-ಲ-ತೆ-ಗ-ಳೊಂ-ದಿಗೆ
ಅನು-ಕೂ-ಲ-ತೆ-ಯನ್ನು
ಅನು-ಕೂ-ಲ-ತೆಯೂ
ಅನು-ಕೂ-ಲ-ರಾಗಿ
ಅನು-ಕೂ-ಲ-ವಾಗ
ಅನು-ಕೂ-ಲ-ವಾ-ಗ-ಲೆಂಬ
ಅನು-ಕೂ-ಲ-ವಾ-ಗು-ತ್ತದೆ
ಅನು-ಕೂ-ಲ-ವಾ-ಗು-ವಂತೆ
ಅನು-ಕೂ-ಲ-ವಾ-ದಾಗ
ಅನು-ಕೂ-ಲ-ವಾ-ಯಿ-ತಾ-ದರೂ
ಅನು-ಕೂ-ಲ-ವಾ-ಯಿತು
ಅನು-ಕೂ-ಲ-ವಿಲ್ಲ
ಅನು-ಕೂ-ಲವೇ
ಅನು-ಕೂ-ಲಸ್ಥ
ಅನು-ಕೂ-ಲಿ-ಸ-ಲಾಗಿದೆ
ಅನು-ಗು-ಣ-ವಾಗಿ
ಅನು-ಗು-ಣ-ವಾ-ಗಿದೆ
ಅನು-ಗು-ಣ-ವಾ-ಗಿಯೇ
ಅನು-ಗು-ಣ-ವಾ-ಗಿವೆ
ಅನು-ಗ್ರಹ
ಅನು-ಗ್ರ-ಹ-ಕಾ-ರಕ
ಅನು-ಗ್ರ-ಹ-ಕಾ-ರ-ಕ-ನಾದ
ಅನು-ಗ್ರ-ಹವೇ
ಅನು-ಗ್ರ-ಹಿಸಿ
ಅನು-ಗ್ರ-ಹಿ-ಸಿ-ದ್ದುಂಟು
ಅನು-ಚ-ರ-ರೊಂ-ದಿಗೆ
ಅನು-ಜ್ಞೆ-ಯನ್ನು
ಅನು-ಪ-ಸ್ಥಿ-ತಿ-ಯಲ್ಲಿ
ಅನು-ಭವ
ಅನು-ಭ-ವ-ಇ-ವು-ಗಳೇ
ಅನು-ಭ-ವಕ್ಕೆ
ಅನು-ಭ-ವಕ್ಕೇ
ಅನು-ಭ-ವ-ಗಳ
ಅನು-ಭ-ವ-ಗಳನ್ನು
ಅನು-ಭ-ವ-ಗಳಲ್ಲಿ
ಅನು-ಭ-ವ-ಗ-ಳಾ-ಗಿ-ದ್ದುವು
ಅನು-ಭ-ವ-ಗಳಿಂದ
ಅನು-ಭ-ವ-ಗ-ಳಿಂ-ದಾಗಿ
ಅನು-ಭ-ವ-ಗಳೂ
ಅನು-ಭ-ವ-ಗಳೇ
ಅನು-ಭ-ವ-ಗ-ಳೊಂ-ದಿಗೆ
ಅನು-ಭ-ವದ
ಅನು-ಭ-ವ-ದಲ್ಲಿ
ಅನು-ಭ-ವ-ದಿಂದ
ಅನು-ಭ-ವ-ದಿಂ-ದಲೇ
ಅನು-ಭ-ವ-ದಿಂ-ದೊ-ಡ-ಗೂ-ಡಿದ
ಅನು-ಭ-ವ-ಪೂ-ರ್ಣ-ವಾದ
ಅನು-ಭ-ವ-ವನ್ನು
ಅನು-ಭ-ವ-ವನ್ನೇ
ಅನು-ಭ-ವ-ವಾ-ಗ-ಬೇಕು
ಅನು-ಭ-ವ-ವಾಗಿ
ಅನು-ಭ-ವ-ವಾ-ಗಿತ್ತು
ಅನು-ಭ-ವ-ವಾ-ಗಿ-ರ-ಲಿಲ್ಲ
ಅನು-ಭ-ವ-ವಾ-ಗು-ತ್ತಿತ್ತು
ಅನು-ಭ-ವ-ವಾದ
ಅನು-ಭ-ವ-ವಾ-ದರೆ
ಅನು-ಭ-ವ-ವಾ-ಯಿತು
ಅನು-ಭ-ವವೂ
ಅನು-ಭವಿ
ಅನು-ಭ-ವಿ-ಸ-ಬ-ಹು-ದಾ-ಗಿತ್ತು
ಅನು-ಭ-ವಿ-ಸ-ಬೇಕು
ಅನು-ಭ-ವಿಸಿ
ಅನು-ಭ-ವಿ-ಸಿದ
ಅನು-ಭ-ವಿ-ಸಿದೆ
ಅನು-ಭ-ವಿ-ಸಿ-ದ್ದೇನೆ
ಅನು-ಭ-ವಿ-ಸಿ-ರ-ದಿದ್ದ
ಅನು-ಭ-ವಿ-ಸಿ-ರ-ಲಿ-ಕ್ಕಿಲ್ಲ
ಅನು-ಭ-ವಿ-ಸಿ-ಲ್ಲವೋ
ಅನು-ಭ-ವಿ-ಸು-ತ್ತಿದ್ದ
ಅನು-ಭ-ವಿ-ಸು-ತ್ತಿ-ದ್ದಂತೆ
ಅನು-ಭ-ವಿ-ಸು-ತ್ತಿ-ದ್ದೇನೆ
ಅನು-ಭ-ವಿ-ಸು-ತ್ತಿ-ದ್ದೇವೆ
ಅನು-ಭ-ವಿ-ಸು-ತ್ತಿ-ರುವ
ಅನು-ಭ-ವಿ-ಸುವ
ಅನು-ಭ-ವಿ-ಸು-ವಂ-ತಾ-ಗಲಿ
ಅನು-ಭ-ವಿ-ಸು-ವಂತೆ
ಅನು-ಭ-ವಿ-ಸು-ವ-ವ-ರಿಗೂ
ಅನು-ಭ-ವಿ-ಸು-ವಾಗ
ಅನು-ಭವೀ
ಅನು-ಭೂತಿ
ಅನು-ಮತಿ
ಅನು-ಮ-ತಿಯ
ಅನು-ಮ-ತಿ-ಯನ್ನು
ಅನು-ಮ-ತಿ-ಯಿ-ರ-ಲಿಲ್ಲ
ಅನು-ಮಾನ
ಅನು-ಮಾ-ನ-ದಿಂ-ದಲೇ
ಅನು-ಮಾ-ನ-ವಿ-ದ್ದದ್ದು
ಅನು-ಮಾನವೂ
ಅನು-ಮಾ-ನ-ವೇಕೆ
ಅನು-ಮಾ-ನಿ-ಸದೆ
ಅನು-ಮಾ-ನಿ-ಸು-ವೆಯಾ
ಅನು-ಮೋ-ದನೆ
ಅನು-ಮೋ-ದಿ-ಸಿ-ದರು
ಅನು-ಮೋ-ದಿ-ಸಿ-ದ್ದರು
ಅನು-ಮೋ-ದಿ-ಸಿ-ರ-ಲಿಲ್ಲ
ಅನು-ಮೋ-ದಿಸು
ಅನು-ಮೋ-ದಿ-ಸು-ವು-ದಿಲ್ಲ
ಅನುಯಾ
ಅನು-ಯಾಯಿ
ಅನು-ಯಾ-ಯಿ-ಸ್ನೇ-ಹಿತ
ಅನು-ಯಾ-ಯಿ-ಗ-ಳಂತೆ
ಅನು-ಯಾ-ಯಿ-ಗಳನ್ನು
ಅನು-ಯಾ-ಯಿ-ಗಳಲ್ಲಿ
ಅನು-ಯಾ-ಯಿ-ಗ-ಳ-ಲ್ಲೊ-ಬ್ಬ-ಳಾ-ದಳು
ಅನು-ಯಾ-ಯಿ-ಗ-ಳಾದ
ಅನು-ಯಾ-ಯಿ-ಗ-ಳಾ-ದರು
ಅನು-ಯಾ-ಯಿ-ಗ-ಳಿಗೆ
ಅನು-ಯಾ-ಯಿ-ಗಳು
ಅನು-ಯಾ-ಯಿ-ಗಳೂ
ಅನು-ಯಾ-ಯಿ-ಗ-ಳೊಂ-ದಿಗೆ
ಅನು-ಯಾ-ಯಿ-ಯಾಗಿ
ಅನು-ಯಾ-ಯಿಯೂ
ಅನು-ರ-ಣಿ-ತ-ವಾ-ಗು-ತ್ತಿತ್ತು
ಅನು-ರ-ಣಿ-ತ-ವಾ-ಗು-ತ್ತಿ-ದೆಈ
ಅನು-ರಾಗ
ಅನು-ರಾ-ಧ-ಪು-ರದ
ಅನು-ರೂ-ಪ-ವಾದ
ಅನು-ವ-ರ್ತಿ-ಗ-ಳೊಂ-ದಿಗೆ
ಅನು-ವಾಗು
ಅನು-ವಾ-ಗು-ತ್ತಾನೆ
ಅನು-ವಾ-ಗು-ವಂತೆ
ಅನು-ವಾದ
ಅನು-ವಾ-ದರು
ಅನು-ವಾ-ದಿಸ
ಅನು-ವಾ-ದಿಸಿ
ಅನು-ವಾ-ದಿ-ಸಿದ
ಅನು-ವಾ-ದಿ-ಸು-ತ್ತಿ-ದ್ದರು
ಅನುವು
ಅನು-ವು-ಮಾ-ಡಿ-ಕೊ-ಟ್ಟಿ-ದ್ದಾರೆ
ಅನು-ಷ್ಠಾನ
ಅನು-ಷ್ಠಾ-ನಕ್ಕೆ
ಅನು-ಷ್ಠಾ-ನ-ಗಳನ್ನು
ಅನು-ಷ್ಠಾ-ನ-ಗೊ-ಳಿ-ಸು-ವುದು
ಅನು-ಷ್ಠಾ-ನದ
ಅನು-ಷ್ಠಾ-ನ-ದಲ್ಲಿ
ಅನು-ಷ್ಠಾ-ನ-ದ-ಲ್ಲಿ-ಸಾ-ಕ್ಷಾ-ತ್ಕಾ-ರ-ದಲ್ಲಿ
ಅನು-ಷ್ಠಾ-ನ-ಧರ್ಮ
ಅನು-ಷ್ಠಾ-ನ-ಧ-ರ್ಮದ
ಅನು-ಷ್ಠಾ-ನ-ಧ-ರ್ಮ-ವೆಂದರೆ
ಅನು-ಷ್ಠಾ-ನ-ಯೋಗ್ಯ
ಅನು-ಷ್ಠಾ-ನ-ವನ್ನು
ಅನು-ಷ್ಠಾ-ನವೂ
ಅನು-ಷ್ಠಾ-ನಾ-ತ್ಮ-ಕ-ವಾದ
ಅನು-ಷ್ಠಾ-ನಾ-ತ್ಮ-ಕ-ವೆಂ-ಬು-ದನ್ನೂ
ಅನು-ಸಂ-ಧಾ-ನಕ್ಕೂ
ಅನು-ಸ-ರಣೆ
ಅನು-ಸ-ರಿ-ಸ-ಬೇಕು
ಅನು-ಸ-ರಿ-ಸಲು
ಅನು-ಸ-ರಿ-ಸ-ಲ್ಪ-ಡ-ಬೇ-ಕಾದ
ಅನು-ಸ-ರಿಸಿ
ಅನು-ಸ-ರಿ-ಸಿ-ರ-ಲಿಲ್ಲ
ಅನು-ಸ-ರಿಸು
ಅನು-ಸ-ರಿ-ಸುತ್ತ
ಅನು-ಸ-ರಿ-ಸು-ತ್ತಿ-ದ್ದರು
ಅನು-ಸ-ರಿ-ಸು-ವಾಗ
ಅನು-ಸ-ರಿ-ಸು-ವು-ದಿ-ಲ್ಲವೊ
ಅನು-ಸ-ರಿ-ಸು-ವುದೇ
ಅನು-ಸಾರ
ಅನು-ಸಾ-ರ-ವಾಗಿ
ಅನು-ಸಾ-ರ-ವಾದ
ಅನೂ-ಚಾ-ನ-ವಾಗಿ
ಅನೂ-ರಾ-ಧ-ಪುರ
ಅನೂ-ರಾ-ಧ-ಪು-ರಕ್ಕೆ
ಅನೂ-ರಾ-ಧ-ಪು-ರ-ದಿಂದ
ಅನೇಕ
ಅನೇ-ಕ-ರ-ಲ್ಲಿತ್ತು
ಅನೇ-ಕ-ರಿಗೆ
ಅನೇ-ಕರು
ಅನೇ-ಕಾ-ನೇಕ
ಅನೈ-ಕ್ಯ-ಗಳ
ಅನೈ-ತಿಕ
ಅನೌ-ಪ-ಚಾ-ರಿಕ
ಅನ್ನ
ಅನ್ನ-ದಾ-ತರು
ಅನ್ನ-ದಾನ
ಅನ್ನ-ದಾ-ನ-ವ-ಸ್ತ್ರ-ದಾ-ನ-ಗಳನ್ನು
ಅನ್ನ-ದಾ-ನದ
ಅನ್ನ-ದಾ-ನಾ-ದಿ-ಗಳು
ಅನ್ನ-ವನ್ನು
ಅನ್ನ-ವನ್ನೂ
ಅನ್ನ-ವನ್ನೇ
ಅನ್ನ-ವಷ್ಟೂ
ಅನ್ನ-ವಿ-ಲ್ಲದೆ
ಅನ್ನ-ಸತ್ರ
ಅನ್ನ-ಸ-ತ್ರ-ಗಳು
ಅನ್ನ-ಸ-ತ್ರದ
ಅನ್ನ-ಸ-ತ್ರ-ದಲ್ಲಿ
ಅನ್ನ-ಸ-ತ್ರ-ವನ್ನು
ಅನ್ನಾ-ಹಾ-ರ-ಗ-ಳಿಗೇ
ಅನ್ನಿ-ಸ-ತೊ-ಡ-ಗಿತ್ತು
ಅನ್ನಿ-ಸ-ದಿ-ರ-ಬ-ಹುದು
ಅನ್ನಿ-ಸಲೇ
ಅನ್ನಿ-ಸಿತು
ಅನ್ನಿ-ಸಿತ್ತು
ಅನ್ನಿ-ಸಿ-ದರೂ
ಅನ್ನಿ-ಸಿ-ದ್ದನ್ನೂ
ಅನ್ನಿಸು
ಅನ್ನಿ-ಸು-ತ್ತಿ-ತ್ತು-ತಾನು
ಅನ್ನಿ-ಸು-ತ್ತಿದೆ
ಅನ್ನಿ-ಸು-ತ್ತಿ-ದೆ-ಎ-ಲ್ಲವೂ
ಅನ್ನಿ-ಸು-ತ್ತಿ-ರು-ವ-ವ-ರೆಗೂ
ಅನ್ನಿ-ಸು-ವುದೇ
ಅನ್ನು
ಅನ್ನು-ವುದು
ಅನ್ಯಥಾ
ಅನ್ಯ-ಮ-ನ-ಸ್ಕತೆ
ಅನ್ಯ-ಮ-ನ-ಸ್ಕ-ರಾಗಿ
ಅನ್ಯಾಯ
ಅನ್ಯಾ-ಯ-ಗಾ-ರರೂ
ಅನ್ಯಾ-ಯ-ವ-ಲ್ಲವೆ
ಅನ್ಯಾ-ಯ-ವಾಗಿ
ಅನ್ಯೋ-ನ್ಯತೆ
ಅನ್ವ-ಯ-ವಾ-ಗ-ಬ-ಲ್ಲುದು
ಅನ್ವ-ಯ-ವಾ-ಗ-ಬ-ಹು-ದಾದ
ಅನ್ವ-ಯ-ವಾ-ಗು-ತ್ತದೆ
ಅನ್ವ-ಯ-ವಾ-ಗು-ತ್ತ-ದೆಂದು
ಅನ್ವ-ಯ-ವಾ-ಗುವ
ಅನ್ವ-ಯ-ವಾ-ಗು-ವು-ದಿಲ್ಲ
ಅನ್ವ-ಯಿ-ಸು-ತ್ತದೆ
ಅನ್ವ-ಯಿ-ಸುವ
ಅನ್ವ-ಯಿ-ಸು-ವಂ-ಥದು
ಅನ್ವೇ-ಷಕ
ಅನ್ವೇ-ಷ-ಕ
ಅಪ-ಘಾತ
ಅಪ-ಘಾ-ತ-ಇವೇ
ಅಪ-ಘಾ-ತ-ದಲ್ಲಿ
ಅಪ-ಚಾ-ರ-ವ-ನ್ನೆ-ಸ-ಗಿದ್ದೂ
ಅಪ-ನಂ-ಬಿ-ಕೆ-ಗಳಿಂದ
ಅಪ-ನಂ-ಬಿ-ಕೆ-ಗ-ಳೆಲ್ಲ
ಅಪ-ನಿಂದೆ
ಅಪ-ಪ್ರ-ಚಾರ
ಅಪ-ಪ್ರ-ಚಾ-ರ-ಕ್ಕಿಂ-ತಲೂ
ಅಪ-ಪ್ರ-ಚಾ-ರ-ಗಳು
ಅಪ-ಪ್ರ-ಚಾ-ರದ
ಅಪ-ಪ್ರ-ಚಾ-ರ-ವನ್ನೂ
ಅಪ-ಮಾನ
ಅಪ-ಮಾ-ನ-ಕ-ರ-ವಾದ
ಅಪ-ಮಾ-ನ-ಗೊ-ಳಿ-ಸ-ಬಾ-ರದು
ಅಪ-ಮಾ-ನ-ದಿಂ-ದಲೇ
ಅಪ-ಮಾ-ನ-ವಾಯಿ
ಅಪ-ಯ-ಶ-ಸ್ಸಿ-ನಂತೆ
ಅಪ-ರಾ-ಧ-ಗಳನ್ನು
ಅಪ-ರಾ-ಧವೋ
ಅಪ-ರಾ-ಧಿ-ಗಳನ್ನು
ಅಪ-ರಾ-ವಿದ್ಯೆ
ಅಪ-ರಾಹ್ನ
ಅಪ-ರಾ-ಹ್ನದ
ಅಪರಿ
ಅಪ-ರಿ-ಚಿತ
ಅಪ-ರಿ-ಪಕ್ವ
ಅಪ-ರಿ-ಮಿತ
ಅಪ-ರಿ-ವ-ರ್ತ-ನೀಯ
ಅಪ-ರೂಪ
ಅಪ-ರೂ-ಪಕ್ಕೆ
ಅಪ-ರೂ-ಪಕ್ಕೊ-ಮ್ಮೊಮ್ಮೆ
ಅಪ-ರೂ-ಪದ
ಅಪ-ರೂ-ಪ-ವಾಗಿ
ಅಪ-ರೂ-ಪವೇ
ಅಪ-ವಾ-ದ-ಗಳ
ಅಪ-ವಾ-ದದ
ಅಪ-ವಾ-ದ-ವಾ-ಗಿ-ರ-ಲಿಲ್ಲ
ಅಪ-ವಾ-ದವೂ
ಅಪ-ವಿತ್ರ
ಅಪ-ವಿ-ತ್ರ-ಗೊಂ-ಡಿದ್ದ
ಅಪ-ವಿ-ತ್ರ-ಗೊಂ-ಡಿ-ವೆ-ಯಲ್ಲ
ಅಪ-ವಿ-ತ್ರ-ತೆಯೂ
ಅಪ-ಹಾಸ್ಯ
ಅಪಾ-ತ್ರ-ದಾನ
ಅಪಾ-ತ್ರ-ದಾ-ನದ
ಅಪಾ-ತ್ರ-ದಾ-ನ-ವನ್ನು
ಅಪಾ-ತ್ರ-ದಾ-ನ-ವೆಂ-ಬುದು
ಅಪಾಯ
ಅಪಾ-ಯ-ಕರ
ಅಪಾ-ಯ-ಕ-ರ-ವಾ-ಗಿತ್ತು
ಅಪಾ-ಯ-ಕಾರಿ
ಅಪಾ-ಯ-ಕಾ-ರಿ-ಯಾದ
ಅಪಾ-ಯಕ್ಕೆ
ಅಪಾ-ಯದ
ಅಪಾ-ಯ-ದ-ಲ್ಲಿ-ರು-ತ್ತೀರಿ
ಅಪಾ-ಯ-ವನ್ನು
ಅಪಾ-ಯ-ವಿ-ದ್ದಾಗ
ಅಪಾರ
ಅಪಾ-ರ-ವಾ-ಗಿದೆ
ಅಪಾ-ರ-ವಾದ
ಅಪಾ-ರ-ವಾ-ದುದು
ಅಪಾ-ರ್ಥಕ್ಕೆ
ಅಪೂರ್ವ
ಅಪೂ-ರ್ವ-ಅ-ದ್ಭುತ
ಅಪೂ-ರ್ವ-ಭವ್ಯ
ಅಪೂ-ರ್ವ-ವಾದ
ಅಪೂ-ರ್ವ-ವಾ-ದದ್ದು
ಅಪೂ-ರ್ವ-ವಾ-ದು-ದೆ-ನ್ನ-ಬೇಕು
ಅಪೂ-ರ್ವ-ಶಕ್ತಿ
ಅಪೇ-ಕ್ಷ-ಣೀ-ಯ-ವಲ್ಲ
ಅಪೇ-ಕ್ಷಿ-ಸಿ-ದರು
ಅಪೇ-ಕ್ಷಿ-ಸಿ-ದ್ದಳು
ಅಪೇಕ್ಷೆ
ಅಪೇ-ಕ್ಷೆಯ
ಅಪೇ-ಕ್ಷೆ-ಯಂತೆ
ಅಪೇರಾ
ಅಪ್ಪಣೆ
ಅಪ್ಪ-ಳಿಸಿ
ಅಪ್ಪ-ಳಿ-ಸು-ತ್ತಿ-ದ್ದರು
ಅಪ್ಪಿ-ಕೊಂ-ಡಿ-ದ್ದೇನೆ
ಅಪ್ಪಿ-ಕೊ-ಳ್ಳು-ವಂತೆ
ಅಪ್ರ-ತಿ-ಭ-ನಾ-ದರೂ
ಅಪ್ರ-ತಿ-ಮರು
ಅಪ್ರ-ತಿ-ಹತ
ಅಪ್ರ-ತಿ-ಹ-ತ-ವಾಗಿ
ಅಪ್ರ-ತಿ-ಹ-ತ-ವಾ-ಗಿತ್ತು
ಅಪ್ರ-ಯ-ತ್ನ-ವಾಗಿ
ಅಪ್ರ-ಯೋ-ಜ-ಕರು
ಅಪ್ರಿಯ
ಅಪ್ರಿ-ಯರೂ
ಅಪ್ರಿ-ಯ-ವಾದ
ಅಪ್ಸ-ರೆ-ಯರೂ
ಅಬ್ಬ
ಅಬ್ಬ-ರ-ಆ-ಡಂ-ಬ-ರ-ಗಳ
ಅಬ್ಬ-ರ-ವನ್ನು
ಅಬ್ಬ-ರ-ವೆಲ್ಲ
ಅಭ-ಯಾ-ನಂದಾ
ಅಭಾವ
ಅಭಾ-ವ-ದಿಂ-ದಾಗಿ
ಅಭಾ-ವ-ದಿಂ-ದಾ-ಗಿ-ಯಲ್ಲ
ಅಭಿ
ಅಭಿ-ನಂ-ದನಾ
ಅಭಿ-ನಂ-ದ-ನಾ-ಪ-ತ್ರ-ವನ್ನು
ಅಭಿ-ನಂ-ದನೆ
ಅಭಿ-ನಂ-ದ-ನೆ-ಗಳ
ಅಭಿ-ನಂ-ದ-ನೆ-ಯ-ನ್ನ-ರ್ಪಿ-ಸಲು
ಅಭಿ-ನಂ-ದ-ನೆ-ಯನ್ನು
ಅಭಿ-ನಂ-ದಿಸಿ
ಅಭಿ-ನಂ-ದಿ-ಸಿತು
ಅಭಿ-ನಂ-ದಿ-ಸಿ-ದರು
ಅಭಿ-ನಂ-ದಿ-ಸುವ
ಅಭಿ-ನ-ಯಿ-ಸಿದ್ದ
ಅಭಿ-ಪ್ರಾಯ
ಅಭಿ-ಪ್ರಾ-ಯಕ್ಕೆ
ಅಭಿ-ಪ್ರಾ-ಯ-ಗಳನ್ನು
ಅಭಿ-ಪ್ರಾ-ಯ-ಗಳನ್ನೂ
ಅಭಿ-ಪ್ರಾ-ಯ-ಗಳಿಂದ
ಅಭಿ-ಪ್ರಾ-ಯ-ಗ-ಳಿಗೆ
ಅಭಿ-ಪ್ರಾ-ಯದ
ಅಭಿ-ಪ್ರಾ-ಯ-ದಂತೆ
ಅಭಿ-ಪ್ರಾ-ಯ-ದಲ್ಲಿ
ಅಭಿ-ಪ್ರಾ-ಯ-ಪ-ಟ್ಟರು
ಅಭಿ-ಪ್ರಾ-ಯ-ಪ-ಟ್ಟಿ-ದ್ದಾನೆ
ಅಭಿ-ಪ್ರಾ-ಯ-ವನ್ನು
ಅಭಿ-ಪ್ರಾ-ಯ-ವನ್ನೇ
ಅಭಿ-ಪ್ರಾ-ಯ-ವಾ-ಗಿ-ರ-ಬೇಕು
ಅಭಿ-ಪ್ರಾ-ಯ-ವಿತ್ತು
ಅಭಿ-ಪ್ರಾ-ಯ-ವಿ-ದ್ದು-ದ-ರಿಂದ
ಅಭಿ-ಪ್ರಾ-ಯವು
ಅಭಿ-ಪ್ರಾ-ಯ-ವೇನು
ಅಭಿ-ಪ್ರಾ-ಯ-ವೇನೆಂದರೆ
ಅಭಿ-ಪ್ರಾ-ಯ-ವೇ-ನೆಂದು
ಅಭಿ-ಪ್ರಾ-ಯ-ವೇ-ನೆಂ-ಬು-ದನ್ನು
ಅಭಿ-ಪ್ರಾ-ಯ-ವೇ-ನೆಂ-ಬು-ದರ
ಅಭಿ-ಮತ
ಅಭಿ-ಮ-ತ-ವಾ-ಗಿತ್ತು
ಅಭಿ-ಮಾನ
ಅಭಿ-ಮಾ-ನ-ಗೌ-ರ-ವ-ಭಕ್ತಿ
ಅಭಿ-ಮಾ-ನ-ದಿಂದ
ಅಭಿ-ಮಾ-ನ-ವ-ನ್ನಿ-ಟ್ಟು-ಕೊಂ-ಡಿ-ದ್ದೇನೆ
ಅಭಿ-ಮಾ-ನ-ವನ್ನು
ಅಭಿ-ಮಾನವು
ಅಭಿ-ಮಾ-ನ-ಶೂ-ನ್ಯ-ರಾ-ಗಿದ್ದ
ಅಭಿ-ಮಾ-ನಿ-ಗ-ಳಾ-ಗಲಿ
ಅಭಿ-ಮಾ-ನಿ-ಗ-ಳಾದ
ಅಭಿ-ಮಾ-ನಿ-ಗಳು
ಅಭಿ-ಮಾ-ನಿ-ಗಳೂ
ಅಭಿ-ಮಾನೀ
ಅಭಿ-ಮುಖ
ಅಭಿ-ಮು-ಖ-ಳಾಗಿ
ಅಭಿ-ಮು-ಖ-ವಾಗಿ
ಅಭಿ-ರುಚಿ
ಅಭಿ-ರು-ಚಿ-ಯ-ನ್ನುಂ-ಟು-ಮಾ-ಡು-ವು-ದರ
ಅಭಿ-ಲಾಷೆ
ಅಭಿ-ಲಾ-ಷೆ-ಯಾ-ಗಿತ್ತು
ಅಭಿ-ಲಾ-ಷೆ-ಯೆಂ-ದರೆ
ಅಭಿ-ವಾ-ದಿ-ಸಿ-ದರು
ಅಭಿ-ವೃದ್ಧಿ
ಅಭಿ-ವೃ-ದ್ಧಿಯ
ಅಭಿ-ವೃ-ದ್ಧಿ-ಯಿಂದ
ಅಭಿ-ವ್ಯಕ್ತಿ
ಅಭಿ-ವ್ಯ-ಕ್ತಿ-ಗೊ-ಳಿ-ಸು-ತ್ತಿವೆ
ಅಭಿ-ವ್ಯ-ಕ್ತಿಯ
ಅಭಿ-ಷೇಕ
ಅಭಿ-ಷೇ-ಕ-ವಾ-ಗು-ತ್ತಿ-ರುವ
ಅಭೀಃ
ಅಭೂ-ತ-ಪೂರ್ವ
ಅಭೂ-ತ-ಪೂ-ರ್ವ-ವಾ-ದುದು
ಅಭೇದಾ
ಅಭೇ-ದಾ-ನಂದ
ಅಭೇ-ದಾ-ನಂ-ದರ
ಅಭೇ-ದಾ-ನಂ-ದ-ರನ್ನು
ಅಭೇ-ದಾ-ನಂ-ದ-ರಿಗೆ
ಅಭೇ-ದಾ-ನಂ-ದರು
ಅಭೇ-ದಾ-ನಂ-ದ-ರೆ-ನ್ನು-ತ್ತಾರೆ
ಅಭ್ಯಂ-ತ-ರವೂ
ಅಭ್ಯ-ಸಿ-ಸ-ಲಾ-ಗು-ವುದು
ಅಭ್ಯಾ-ಗ-ತರ
ಅಭ್ಯಾಸ
ಅಭ್ಯಾ-ಸ-ಸಂ-ಪ್ರ-ದಾ-ಯ-ಗಳೇ
ಅಭ್ಯಾ-ಸ-ಗಳ
ಅಭ್ಯಾ-ಸ-ಗಳನ್ನು
ಅಭ್ಯಾ-ಸ-ಗಳು
ಅಭ್ಯಾ-ಸ-ಗ-ಳು-ಳ್ಳ-ವ-ನಾ-ಗಿದ್ದು
ಅಭ್ಯಾ-ಸ-ದಂತೆ
ಅಭ್ಯಾ-ಸ-ದಲ್ಲಿ
ಅಭ್ಯಾ-ಸ-ವನ್ನು
ಅಭ್ಯಾ-ಸ-ವಿ-ಟ್ಟು-ಕೊಂ-ಡಿ-ದ್ದರು
ಅಭ್ಯಾ-ಸ-ವಿ-ದ್ದ-ವರು
ಅಭ್ಯಾ-ಸ-ವಿ-ರ-ಬೇಕು
ಅಭ್ಯಾ-ಸವೂ
ಅಭ್ಯು-ದಯ
ಅಭ್ಯು-ದ-ಯ-ಇವು
ಅಭ್ಯು-ದ-ಯಕ್ಕೆ
ಅಭ್ಯು-ದ-ಯದ
ಅಮಂ-ಗ-ಳ-ಕರ
ಅಮಂ-ಗ-ಳವೂ
ಅಮರ
ಅಮ-ರ-ತ್ವದ
ಅಮ-ರ-ನಾಥ
ಅಮ-ರ-ನಾ-ಥಕ್ಕೆ
ಅಮ-ರ-ನಾ-ಥದ
ಅಮ-ರ-ನಾ-ಥ-ದಲ್ಲಿ
ಅಮ-ರ-ನಾ-ಥ-ದಿಂದ
ಅಮ-ರ-ನಾ-ಥನ
ಅಮ-ರ-ನಾ-ಥ-ನಿಂದ
ಅಮ-ರ-ನಾ-ಥ-ವನ್ನು
ಅಮ-ರ-ಪ್ರೇ-ಮದ
ಅಮಾ-ನುಷ
ಅಮಾ-ನು-ಷ-ತೆ-ಯಿಂದ
ಅಮಾ-ರ್ಕಾಲೊ
ಅಮೂಲ್ಯ
ಅಮೃತ
ಅಮೃ-ತತ್ವ
ಅಮೃ-ತ-ತ್ವದ
ಅಮೃ-ತ-ಪು-ತ್ರರು
ಅಮೃ-ತ-ಪ್ರ-ವಾಹ
ಅಮೃ-ತ-ಮ-ಯ-ವಾದ
ಅಮೃ-ತ-ವ-ನ್ನ-ರಸಿ
ಅಮೃ-ತ-ವಾ-ಣಿ-ಯನ್ನು
ಅಮೃ-ತ-ಶಿ-ಲೆಯ
ಅಮೃ-ತ-ಶಿ-ಲೆ-ಯಿಂದ
ಅಮೃ-ತ-ಸ-ರಕ್ಕೆ
ಅಮೃ-ತ-ಸ್ವ-ರೂಪಿ
ಅಮೃ-ತಾತ್ಮ
ಅಮೃ-ತಾ-ನಂದ
ಅಮೃ-ತಾ-ನಂ-ದ-ರಿ-ಗೊಮ್ಮೆ
ಅಮೆ-ರಿಕ
ಅಮೆ-ರಿ-ಕ
ಅಮೆ-ರಿ-ಕ-ಇಂ-ಗ್ಲೆಂ-ಡು-ಗಳಿಂದ
ಅಮೆ-ರಿ-ಕ-ನಾರ್ವೆ
ಅಮೆ-ರಿ-ಕಕ್ಕೂ
ಅಮೆ-ರಿ-ಕಕ್ಕೆ
ಅಮೆ-ರಿ-ಕ-ಗಳಲ್ಲಿ
ಅಮೆ-ರಿ-ಕ-ಗ-ಳಿಗೆ
ಅಮೆ-ರಿ-ಕದ
ಅಮೆ-ರಿ-ಕ-ದಂ-ತಹ
ಅಮೆ-ರಿ-ಕ-ದ-ಲ್ಲಂತೂ
ಅಮೆ-ರಿ-ಕ-ದಲ್ಲಿ
ಅಮೆ-ರಿ-ಕ-ದ-ಲ್ಲಿ-ದ್ದಾಗ
ಅಮೆ-ರಿ-ಕ-ದ-ಲ್ಲಿ-ದ್ದಾ-ಗಲೇ
ಅಮೆ-ರಿ-ಕ-ದ-ಲ್ಲಿ-ದ್ದಾ-ಗಿ-ನಿಂ-ದಲೂ
ಅಮೆ-ರಿ-ಕ-ದ-ಲ್ಲಿನ
ಅಮೆ-ರಿ-ಕ-ದ-ಲ್ಲಿ-ಸಾ-ಧಿ-ಸಿದ
ಅಮೆ-ರಿ-ಕ-ದಲ್ಲೂ
ಅಮೆ-ರಿ-ಕ-ದ-ವನೊ
ಅಮೆ-ರಿ-ಕ-ದಾ-ದ್ಯಂತ
ಅಮೆ-ರಿ-ಕ-ದಿಂದ
ಅಮೆ-ರಿ-ಕ-ದೆ-ಡೆಗೆ
ಅಮೆ-ರಿ-ಕನ್
ಅಮೆ-ರಿ-ಕ-ನ್ನರ
ಅಮೆ-ರಿ-ಕ-ನ್ನ-ರಿಗೆ
ಅಮೆ-ರಿ-ಕ-ನ್ನರು
ಅಮೆ-ರಿ-ಕ-ನ್ನರೂ
ಅಮೆ-ರಿ-ಕ-ನ್ನರೆ
ಅಮೆ-ರಿ-ಕೆಗೆ
ಅಮೆ-ರಿ-ಕೆಯ
ಅಮೇ-ಧ್ಯ-ದಂತೆ
ಅಮೇ-ರಿ-ಕಕ್ಕೆ
ಅಮೇ-ರಿ-ಕ-ದಲ್ಲಿ
ಅಮೇ-ರಿ-ಕೆ-ಯಲ್ಲಿ
ಅಮ್ಮ
ಅಮ್ಮ-ನ-ವರ
ಅಮ್ಮಾ
ಅಯ-ಶ-ಸ್ವಿ-ಯಾ-ಗಿ-ಬಿ-ಡು-ತ್ತದೋ
ಅಯ-ಸ್ಕಾಂ-ತತ್ವ
ಅಯ-ಸ್ಕಾಂ-ತೀಯ
ಅಯಾ-ಚಿ-ತ-ವಾಗಿ
ಅಯೋಗ್ಯ
ಅಯ್ಯ
ಅಯ್ಯಂ-ಗಾ-ರರ
ಅಯ್ಯಂ-ಗಾ-ರ-ರಿಗೆ
ಅಯ್ಯಂ-ಗಾ-ರರು
ಅಯ್ಯಂ-ಗಾರ್
ಅಯ್ಯಯ್ಯೊ
ಅಯ್ಯಯ್ಯೋ
ಅಯ್ಯ-ರರ
ಅಯ್ಯ-ರ-ರಾ-ಗಲಿ
ಅಯ್ಯ-ರ-ರಿಗೆ
ಅಯ್ಯ-ರರು
ಅಯ್ಯ-ರ-ರೊಂ-ದಿಗೆ
ಅಯ್ಯರ್
ಅಯ್ಯಾ
ಅಯ್ಯೊ
ಅಯ್ಯೋ
ಅರ-ಗಳು
ಅರ-ಗಿ-ಸಿ-ಕೊಂ-ಡಳೋ
ಅರ-ಗಿ-ಸಿ-ಕೊಂ-ಡ-ವರು
ಅರ-ಗಿ-ಸಿ-ಕೊಂಡು
ಅರ-ಗಿ-ಸಿ-ಕೊ-ಳ್ಳ-ಬ-ಲ್ಲುದು
ಅರ-ಗಿ-ಸಿ-ಕೊ-ಳ್ಳ-ಬೇ-ಕೆಂ-ಬುದು
ಅರ-ಗಿ-ಸಿ-ಕೊಳ್ಳಿ
ಅರ-ಗು-ತ್ತಿದೆ
ಅರ-ಚಿತು
ಅರ-ಚುತ್ತ
ಅರಣ್ಯ
ಅರ-ಣ್ಯ-ಗಳ
ಅರ-ಣ್ಯ-ಗಳನ್ನು
ಅರ-ಣ್ಯ-ಗಳಲ್ಲಿ
ಅರ-ಣ್ಯದ
ಅರ-ಣ್ಯ-ದಲ್ಲಿ
ಅರ-ಮನೆ
ಅರ-ಮ-ನೆ-ಗ-ಳಲ್ಲೂ
ಅರ-ಮ-ನೆ-ಗಳು
ಅರ-ಮ-ನೆಗೆ
ಅರ-ಮ-ನೆಯ
ಅರ-ಮ-ನೆ-ಯಂ-ತಹ
ಅರ-ಮ-ನೆ-ಯನ್ನೂ
ಅರ-ಮ-ನೆ-ಯಲ್ಲಿ
ಅರಳಿ
ಅರ-ಳಿದ
ಅರ-ಳಿ-ಸುವ
ಅರಳು
ಅರ-ವತ್ತು
ಅರಸ
ಅರ-ಸರ
ಅರ-ಸಾ-ದರೂ
ಅರ-ಸು-ತ್ತಾ-ನೆಂಬ
ಅರ-ಸು-ತ್ತಿ-ದ್ದೇ-ನೆಯೋ
ಅರ-ಸು-ವು-ದ-ರ-ಲ್ಲಿಯೇ
ಅರಿ-ಕೆ-ಯನ್ನು
ಅರಿ-ತಂ-ತೆಲ್ಲ
ಅರಿ-ತಳು
ಅರಿ-ತ-ವನು
ಅರಿ-ತ-ವರ
ಅರಿ-ತ-ವರು
ಅರಿತಿ
ಅರಿ-ತಿದ್ದ
ಅರಿ-ತಿ-ದ್ದ-ರಾ-ದರೂ
ಅರಿ-ತಿ-ದ್ದರು
ಅರಿ-ತಿ-ದ್ದೇನೆ
ಅರಿ-ತಿ-ದ್ದೇ-ನೆ-ಪ್ರ-ತಿ-ಯೊ-ಬ್ಬ-ನಲ್ಲೂ
ಅರಿ-ತಿ-ರ-ಬೇ-ಕಾ-ಗು-ತ್ತದೆ
ಅರಿ-ತಿ-ರುವ
ಅರಿತು
ಅರಿ-ತು-ಕೊಂಡ
ಅರಿ-ತು-ಕೊಂ-ಡರೆ
ಅರಿ-ತು-ಕೊಂ-ಡ-ವ-ರಿಗೆ
ಅರಿ-ತು-ಕೊಂ-ಡ-ವ-ರೊ-ಬ್ಬ-ರಿಂದ
ಅರಿ-ತು-ಕೊಂ-ಡಾಗ
ಅರಿ-ತು-ಕೊಂ-ಡಾ-ಗಿ-ನಿಂ-ದಲೂ
ಅರಿ-ತು-ಕೊಂಡು
ಅರಿ-ತು-ಕೊ-ಳ್ಳ-ತೊ-ಡ-ಗಿ-ದ್ದರು
ಅರಿ-ತು-ಕೊ-ಳ್ಳ-ದಿ-ರುವ
ಅರಿ-ತು-ಕೊ-ಳ್ಳಲು
ಅರಿ-ತು-ಕೊ-ಳ್ಳು-ತ್ತಾನೆ
ಅರಿ-ತು-ಕೊ-ಳ್ಳು-ವಂತೆ
ಅರಿ-ತು-ಕೊ-ಳ್ಳು-ವಲ್ಲಿ
ಅರಿ-ತು-ಕೊ-ಳ್ಳು-ವ-ವ-ರೆಗೆ
ಅರಿತೆ
ಅರಿತೋ
ಅರಿ-ಯದ
ಅರಿ-ಯ-ದವ
ಅರಿ-ಯ-ದ-ವರ
ಅರಿ-ಯದೆ
ಅರಿ-ಯ-ದೆಯೋ
ಅರಿ-ಯ-ಬ-ಲ್ಲರು
ಅರಿ-ಯ-ಬ-ಲ್ಲೆ-ನ-ಲ್ಲದೆ
ಅರಿ-ಯ-ಬ-ಹು-ದಾ-ಗಿತ್ತು
ಅರಿ-ಯ-ಬೇ-ಕಾ-ಗಿದೆ
ಅರಿ-ಯ-ಬೇ-ಕಾ-ದರೆ
ಅರಿ-ಯ-ಬೇಕು
ಅರಿ-ಯರು
ಅರಿ-ಯಲು
ಅರಿ-ಯು-ತ್ತೇನೆ
ಅರಿ-ಯುವ
ಅರಿ-ಯು-ವಲ್ಲಿ
ಅರಿ-ಯು-ವುದ
ಅರಿ-ಯು-ವು-ದ-ರಿಂದ
ಅರಿ-ವನ್ನು
ಅರಿ-ವ-ನ್ನುಂ-ಟು-ಮಾ-ಡಿ-ಸುವ
ಅರಿ-ವ-ನ್ನುಂ-ಟು-ಮಾ-ಡಿ-ಸು-ವಲ್ಲಿ
ಅರಿವಾ
ಅರಿ-ವಾ-ಗ-ತೊ-ಡಿ-ಗಿತ್ತು
ಅರಿ-ವಾ-ಗ-ದಿ-ರದು
ಅರಿ-ವಾ-ಗದು
ಅರಿ-ವಾ-ಗಿತ್ತು
ಅರಿ-ವಾ-ಗಿ-ದೆಯೋ
ಅರಿ-ವಾ-ಗಿ-ರ-ಬೇ-ಕು-ತಾವು
ಅರಿ-ವಾ-ಗಿ-ರ-ಲಿಲ್ಲ
ಅರಿ-ವಾ-ಗು-ತ್ತದೆ
ಅರಿ-ವಾ-ಗು-ವ-ವ-ರೆಗೆ
ಅರಿ-ವಾ-ದದ್ದು
ಅರಿ-ವಾ-ದಾಗ
ಅರಿ-ವಾ-ಯಿತು
ಅರಿ-ವಾ-ಯಿ-ತು-ತಾವು
ಅರಿ-ವಿಗೂ
ಅರಿ-ವಿಗೆ
ಅರಿ-ವಿಗೇ
ಅರಿ-ವಿತ್ತು
ಅರಿ-ವಿ-ದೆಯೋ
ಅರಿ-ವಿ-ರುವ
ಅರಿ-ವಿ-ವಿಲ್ಲ
ಅರಿವು
ಅರಿವೇ
ಅರುಣೋ
ಅರು-ಣೋ-ದ-ಯದ
ಅರು-ಣೋ-ದ-ಯ-ದಲ್ಲಿ
ಅರು-ವತ್ತ
ಅರು-ಹಿ-ದರು
ಅರೆ
ಅರೆ-ಕ್ಷಣ
ಅರೆ-ನಾ-ಗ-ರಿಕ
ಅರೆ-ಬರೆ
ಅರೆ-ಬೆ-ತ್ತಲೆ
ಅರೆ-ಮು-ಚ್ಚಿದ
ಅರ್ಚಕ
ಅರ್ಚ-ಕನ
ಅರ್ಚ-ಕ-ರಾ-ಗಿ-ದ್ದರೋ
ಅರ್ಚ-ಕರು
ಅರ್ಚಿಸಿ
ಅರ್ಚಿ-ಸಿ-ದರು
ಅರ್ಜಿ
ಅರ್ಜು-ನ-ನಿಗೆ
ಅರ್ಥ
ಅರ್ಥ-ದಲ್ಲಿ
ಅರ್ಥ-ದ-ಲ್ಲಿಯೂ
ಅರ್ಥ-ಪೂರ್ಣ
ಅರ್ಥ-ಪೂ-ರ್ಣ-ವಾಗಿ
ಅರ್ಥ-ಪೂ-ರ್ಣ-ವಾ-ಗಿದೆ
ಅರ್ಥ-ಪೂ-ರ್ಣ-ವಾ-ಗಿ-ದ್ದುವು
ಅರ್ಥ-ಪೂ-ರ್ಣ-ವಾದ
ಅರ್ಥ-ಮಾಡಿ
ಅರ್ಥ-ಮಾ-ಡಿ-ಕೊಂಡ
ಅರ್ಥ-ಮಾ-ಡಿ-ಕೊಂ-ಡ-ವ-ರೆಂ-ದರೆ
ಅರ್ಥ-ಮಾ-ಡಿ-ಕೊಂ-ಡ-ವರೊ
ಅರ್ಥ-ಮಾ-ಡಿ-ಕೊಂ-ಡಿ-ದ್ದಾರೋ
ಅರ್ಥ-ಮಾ-ಡಿ-ಕೊಂ-ಡಿದ್ದೆ
ಅರ್ಥ-ಮಾ-ಡಿ-ಕೊಂ-ಡಿ-ದ್ದೇನೆ
ಅರ್ಥ-ಮಾ-ಡಿ-ಕೊಂ-ಡಿ-ರ-ಲಿಲ್ಲ
ಅರ್ಥ-ಮಾ-ಡಿ-ಕೊಂ-ಡಿ-ರುವ
ಅರ್ಥ-ಮಾ-ಡಿ-ಕೊಂಡು
ಅರ್ಥ-ಮಾ-ಡಿ-ಕೊಳ್ಳ
ಅರ್ಥ-ಮಾ-ಡಿ-ಕೊ-ಳ್ಳ-ಬ-ಲ್ಲರು
ಅರ್ಥ-ಮಾ-ಡಿ-ಕೊ-ಳ್ಳ-ಬ-ಲ್ಲೆವು
ಅರ್ಥ-ಮಾ-ಡಿ-ಕೊ-ಳ್ಳ-ಬೇ-ಕಾ-ಗಿತ್ತು
ಅರ್ಥ-ಮಾ-ಡಿ-ಕೊ-ಳ್ಳ-ಬೇಕು
ಅರ್ಥ-ಮಾ-ಡಿ-ಕೊ-ಳ್ಳ-ಲಾ-ಗದು
ಅರ್ಥ-ಮಾ-ಡಿ-ಕೊ-ಳ್ಳ-ಲಾ-ರದೆ
ಅರ್ಥ-ಮಾ-ಡಿ-ಕೊ-ಳ್ಳ-ಲಾ-ರರು
ಅರ್ಥ-ಮಾ-ಡಿ-ಕೊ-ಳ್ಳಲು
ಅರ್ಥ-ಮಾ-ಡಿ-ಕೊ-ಳ್ಳು-ವಲ್ಲಿ
ಅರ್ಥ-ಮಾ-ಡಿ-ಕೊ-ಳ್ಳು-ವುದು
ಅರ್ಥ-ರ-ಹಿತ
ಅರ್ಥ-ವ-ಡ-ಗಿತ್ತು
ಅರ್ಥ-ವನ್ನು
ಅರ್ಥ-ವಾ-ಗ-ತ್ತದೆ
ಅರ್ಥ-ವಾ-ಗ-ಲಿಲ್ಲ
ಅರ್ಥ-ವಾ-ಗ-ಲೊ-ಲ್ಲವು
ಅರ್ಥ-ವಾ-ಗು-ತ್ತದೆ
ಅರ್ಥ-ವಾ-ಗು-ತ್ತ-ದೆಯೆ
ಅರ್ಥ-ವಾ-ಗು-ತ್ತ-ದೆಯೇ
ಅರ್ಥ-ವಾ-ಗು-ತ್ತಿತ್ತು
ಅರ್ಥ-ವಾ-ಗು-ತ್ತಿದೆ
ಅರ್ಥ-ವಾ-ಗು-ತ್ತಿಲ್ಲ
ಅರ್ಥ-ವಾ-ಗು-ವಂ-ತಹ
ಅರ್ಥ-ವಾ-ಗು-ವ-ವ-ರೆಗೂ
ಅರ್ಥ-ವಾ-ದಂತೆ
ಅರ್ಥ-ವಾ-ದರೂ
ಅರ್ಥ-ವಾ-ಯಿತು
ಅರ್ಥ-ವಾ-ಯಿ-ತು-ಶ್ರೀ-ರಾ-ಮ-ಕೃ-ಷ್ಣರು
ಅರ್ಥ-ವಾ-ಯಿತೇ
ಅರ್ಥ-ವಿದೆ
ಅರ್ಥ-ವಿಲ್ಲ
ಅರ್ಥ-ವಿ-ಲ್ಲದ
ಅರ್ಥ-ವಿ-ಷ್ಟೆ-ರ-ಸ್ತೆ-ಯನ್ನು
ಅರ್ಥವು
ಅರ್ಥ-ವೆಂದು
ಅರ್ಥವೇ
ಅರ್ಥ-ವೇನು
ಅರ್ಥ-ವೊಂದು
ಅರ್ಥ-ವ್ಯಾಪ್ತಿ
ಅರ್ಥ-ಸ-ಹಿ-ತ-ವಾಗಿ
ಅರ್ಥ-ಹೀನ
ಅರ್ಥ-ಹೀ-ನ-ತೆ-ಪೊ-ಳ್ಳು-ತ-ನ-ಗಳನ್ನು
ಅರ್ಥ-ಹೀ-ನ-ತೆ-ಯೆಲ್ಲ
ಅರ್ಥ-ಹೀ-ನ-ವ-ಲ್ಲವೆ
ಅರ್ಥಾತ್
ಅರ್ಥೈ-ಸ-ಬ-ಹುದು
ಅರ್ಥೈ-ಸಲು
ಅರ್ಥೈ-ಸಿ-ದ್ದ-ರಿಂ-ದಲೇ
ಅರ್ಧ
ಅರ್ಧಂ-ಬರ್ಧ
ಅರ್ಧ-ಕ್ಕರ್ಧ
ಅರ್ಧಕ್ಕೆ
ಅರ್ಧಕ್ಕೇ
ಅರ್ಧ-ಗಂ-ಟೆಗೂ
ಅರ್ಧ-ಗಂ-ಟೆ-ಗೆಲ್ಲ
ಅರ್ಧ-ದಾರಿಗೆ
ಅರ್ಧ-ನಿ-ಮೀ-ಲಿತ
ಅರ್ಧ-ಭಾಗ
ಅರ್ಧೋ-ನ್ಮೀ-ಲಿತ
ಅರ್ಪಣೆ
ಅರ್ಪಿಸ
ಅರ್ಪಿ-ಸ-ಬೇಕು
ಅರ್ಪಿ-ಸ-ಲಾದ
ಅರ್ಪಿ-ಸ-ಲಾ-ಯಿತು
ಅರ್ಪಿ-ಸಲು
ಅರ್ಪಿಸಿ
ಅರ್ಪಿ-ಸಿ-ಕೊಂ-ಡರು
ಅರ್ಪಿ-ಸಿದ
ಅರ್ಪಿ-ಸಿ-ದರು
ಅರ್ಪಿ-ಸಿ-ಬಿ-ಡು-ತ್ತೇನೆ
ಅರ್ಪಿ-ಸು-ತ್ತೇನೆ
ಅರ್ಪಿ-ಸುವೆ
ಅರ್ಪಿ-ಸೋಣ
ಅರ್ಹ-ತೆ-ಗಳ
ಅರ್ಹ-ತೆ-ಯನ್ನು
ಅರ್ಹ-ತೆ-ಯಿ-ರು-ವುದನ್ನು
ಅರ್ಹ-ರ-ನ್ನಾ-ಗಿ-ಸು-ತ್ತದೆ
ಅರ್ಹ-ರಾ-ಗ-ಬ-ಲ್ಲರು
ಅರ್ಹ-ರಾ-ಗಿ-ದ್ದರು
ಅರ್ಹ-ರಾ-ಗಿ-ರು-ತ್ತಾರೆ
ಅರ್ಹ-ರಾ-ಗು-ವು-ದಿಲ್ಲ
ಅರ್ಹ-ಳಾಗಿ
ಅರ್ಹ-ಳಾ-ಗಿ-ಲ್ಲ-ದಿ-ರು-ವುದು
ಅರ್ಹ-ವಾ-ಗಿದೆ
ಅರ್ಹ-ವಾ-ಗಿ-ದ್ದರೆ
ಅಲಂ
ಅಲಂ-ಕ-ರಿಸ
ಅಲಂ-ಕ-ರಿ-ಸ-ಲಾ-ಗಿತ್ತು
ಅಲಂ-ಕ-ರಿ-ಸ-ಲಾ-ಗಿದ್ದ
ಅಲಂ-ಕ-ರಿ-ಸ-ಲ್ಪಟ್ಟ
ಅಲಂ-ಕ-ರಿಸಿ
ಅಲಂ-ಕ-ರಿ-ಸಿದ್ದ
ಅಲಂ-ಕ-ರಿ-ಸು-ತ್ತಾರೆ
ಅಲಂ-ಕಾರ
ಅಲಂ-ಕಾ-ರ-ಗಳನ್ನೆಲ್ಲ
ಅಲಂ-ಕಾ-ರ-ದಲ್ಲಿ
ಅಲಂ-ಕೃ-ತ-ವಾಗಿ
ಅಲಂ-ಕೃ-ತ-ವಾ-ಗಿದೆ
ಅಲಂ-ಕೃ-ತ-ವಾ-ಗಿದ್ದ
ಅಲಂ-ಕೃ-ತ-ವಾದ
ಅಲ-ಕ್ಷಿ-ಸಿ-ದರು
ಅಲ-ಕ್ಷ್ಯ-ವನ್ನು
ಅಲನ್
ಅಲನ್ಗೆ
ಅಲ-ನ್ನಿಗೆ
ಅಲ-ಮೇಡ
ಅಲ-ಮೇ-ಡ-ಗ-ಳ-ಲ್ಲೂ-ಅ-ನೌ-ಪ-ಚಾ-ರಿಕ
ಅಲ-ಮೇ-ಡ-ದಲ್ಲಿ
ಅಲು-ಗಾ-ಡದೆ
ಅಲು-ಗಾ-ಡಲೂ
ಅಲು-ಗಾಡಿ
ಅಲು-ಗಾ-ಡಿ-ಸ-ಬ-ಲ್ಲರು
ಅಲು-ಗಾ-ಡಿ-ಸ-ಬ-ಲ್ಲಿರಿ
ಅಲು-ಗಾ-ಡಿ-ಸ-ಬಲ್ಲೆ
ಅಲು-ಗಾ-ಡಿ-ಸಿದ
ಅಲು-ಗಾ-ಡಿ-ಸಿ-ಬಿ-ಡು-ತ್ತೇನೆ
ಅಲು-ಗಾ-ಡು-ತ್ತಿದೆ
ಅಲು-ಗಿತು
ಅಲು-ಗಿ-ಸ-ಬಲ್ಲ
ಅಲೆ
ಅಲೆ-ಕ್ಸಾಂ-ಡ-ರರೂ
ಅಲೆ-ಗಳನ್ನು
ಅಲೆ-ಗಳಲ್ಲಿ
ಅಲೆ-ಗ-ಳಿಂ-ದಾಗಿ
ಅಲೆ-ಗಳು
ಅಲೆ-ಗ-ಳು-ಇಂ-ಥದೇ
ಅಲೆ-ಗ-ಳೆದ್ದು
ಅಲೆ-ಗ್ಸಾಂ-ಡ-ರ್ನಂ-ಥ-ವ-ನನ್ನೇ
ಅಲೆ-ದಾ-ಡಿ-ಬಿ-ಟ್ಟಿ-ದ್ದಾರೆ
ಅಲೆ-ದಾ-ಡುವ
ಅಲೆಯ
ಅಲೆ-ಯಾ-ಗಿ-ರ-ಬ-ಹುದು
ಅಲೆಯು
ಅಲೆ-ಯುತ್ತ
ಅಲೆ-ಯು-ವುದು
ಅಲೆ-ಯೆ-ದ್ದಿತ್ತು
ಅಲೆಯೇ
ಅಲೆ-ಯೋ-ಪಾ-ದಿ-ಯಲ್ಲಿ
ಅಲೊ-ನನ್ನ
ಅಲೌ-ಕಿಕ
ಅಲೌ-ಕಿ-ಕ-ತೆ-ಯನ್ನು
ಅಲ್ಪ
ಅಲ್ಪ-ತ-ನ-ವನ್ನು
ಅಲ್ಪ-ನೆ-ನಿ-ಸು-ತ್ತಾನೆ
ಅಲ್ಪ-ಸ್ವಲ್ಪ
ಅಲ್ಪಾ-ವ-ಧಿ-ಯಲ್ಲೇ
ಅಲ್ಲ
ಅಲ್ಲ-ಇಡೀ
ಅಲ್ಲ-ಇನ್ನು
ಅಲ್ಲ-ಗ-ಳೆ-ಯ-ಲಾರೆ
ಅಲ್ಲ-ಗ-ಳೆ-ಯಲು
ಅಲ್ಲ-ದಿ-ದ್ದರೆ
ಅಲ್ಲದೆ
ಅಲ್ಲ-ದೆ-ಮ್ಲೇ-ಚ್ಛ-ರೊಂ-ದಿಗೆ
ಅಲ್ಲದೇ
ಅಲ್ಲಲ್ಲಿ
ಅಲ್ಲ-ಲ್ಲಿನ
ಅಲ್ಲಲ್ಲೇ
ಅಲ್ಲವೆ
ಅಲ್ಲವೋ
ಅಲ್ಲ-ಹ-ಣದ
ಅಲ್ಲಾ
ಅಲ್ಲಾ-ಡಿಸಿ
ಅಲ್ಲಾ-ಡಿ-ಸುತ್ತ
ಅಲ್ಲಾನ
ಅಲ್ಲಾನೊ
ಅಲ್ಲಿ
ಅಲ್ಲಿಂದ
ಅಲ್ಲಿಂ-ದಲೇ
ಅಲ್ಲಿಂ-ದೀ-ಚೆಗೆ
ಅಲ್ಲಿಗೂ
ಅಲ್ಲಿಗೆ
ಅಲ್ಲಿಗೇ
ಅಲ್ಲಿದ್ದ
ಅಲ್ಲಿ-ದ್ದರು
ಅಲ್ಲಿ-ದ್ದವ
ಅಲ್ಲಿ-ದ್ದ-ವ-ನೊಬ್ಬ
ಅಲ್ಲಿ-ದ್ದ-ವರ
ಅಲ್ಲಿ-ದ್ದ-ವ-ರನ್ನು
ಅಲ್ಲಿ-ದ್ದ-ವ-ರ-ನ್ನು-ದ್ದೇ-ಶಿಸಿ
ಅಲ್ಲಿ-ದ್ದ-ವ-ರಲ್ಲಿ
ಅಲ್ಲಿ-ದ್ದ-ವ-ರ-ಲ್ಲೊ-ಬ್ಬರು
ಅಲ್ಲಿ-ದ್ದ-ವರು
ಅಲ್ಲಿ-ದ್ದ-ವರೆಲ್ಲ
ಅಲ್ಲಿ-ದ್ದ-ವರೆ-ಲ್ಲರ
ಅಲ್ಲಿ-ದ್ದ-ವ-ರೊ-ಬ್ಬರು
ಅಲ್ಲಿ-ದ್ದು-ಕೊಂಡು
ಅಲ್ಲಿನ
ಅಲ್ಲಿ-ನ-ವರ
ಅಲ್ಲಿ-ನ-ವರು
ಅಲ್ಲಿ-ಯದು
ಅಲ್ಲಿ-ಯ-ವ-ರನ್ನು
ಅಲ್ಲಿ-ಯ-ವ-ರೆಗೂ
ಅಲ್ಲಿ-ಯ-ವ-ರೆಗೆ
ಅಲ್ಲಿಯೂ
ಅಲ್ಲಿಯೇ
ಅಲ್ಲಿ-ರುವ
ಅಲ್ಲಿ-ರು-ವಾಗ
ಅಲ್ಲಿ-ರು-ವುದು
ಅಲ್ಲಿ-ರು-ವು-ದೆಲ್ಲ
ಅಲ್ಲೆಲ್ಲ
ಅಲ್ಲೇ
ಅಲ್ಲೇ-ನಿ-ರ-ಬ-ಹುದು
ಅಲ್ಲೊಂದು
ಅಲ್ಲೋಲ
ಅಲ್ಲೋ-ಲ-ಕ-ಲ್ಲೋ-ಲ-ಗಳ
ಅಲ್ವ-ರಿಗೆ
ಅಲ್ವ-ರಿ-ನಲ್ಲಿ
ಅಲ್ವರ್
ಅಳ-ತೆಗೆ
ಅಳ-ತೊ-ಡ-ಗಿ-ದರು
ಅಳ-ಲೆ-ಕಾಯಿ
ಅಳ-ವ-ಡಿಸಿ
ಅಳ-ವ-ಡಿ-ಸಿ-ಕೊಂ-ಡಿ-ದ್ದೇನೆ
ಅಳ-ವ-ಡಿ-ಸಿ-ಕೊಂಡು
ಅಳ-ವ-ಡಿ-ಸಿ-ಕೊ-ಳ್ಳದೆ
ಅಳ-ವ-ಡಿ-ಸಿ-ಕೊ-ಳ್ಳಲು
ಅಳ-ವ-ಡಿ-ಸಿ-ಕೊ-ಳ್ಳು-ವು-ದರ
ಅಳ-ಸಿಂಗ
ಅಳ-ಸಿಂ-ಗ-ನಂ-ತ-ಹ-ವರು
ಅಳ-ಸಿಂ-ಗ-ನಿಗೂ
ಅಳ-ಸಿಂ-ಗ-ರಿಗೆ
ಅಳ-ಸಿಂ-ಗರೊ
ಅಳಿದು
ಅಳಿ-ದು-ಳಿದ
ಅಳಿ-ದು-ಹೋ-ದರೂ
ಅಳಿ-ವಿನ
ಅಳಿ-ವಿ-ಲ್ಲದ
ಅಳಿ-ಸಿ-ಕೊಂ-ಡು-ಬಿ-ಡ-ಬೇಕು
ಅಳು
ಅಳು-ಕದೆ
ಅಳುಕು
ಅಳು-ಕುಂ-ಟಾ-ಗು-ತ್ತಿತ್ತು
ಅಳು-ತ್ತೇವೆ
ಅಳು-ವು-ದಿಲ್ಲ
ಅಳು-ವುದು
ಅಳು-ವು-ದೇ-ಕೆಂ-ದರೆ
ಅಳೆದು
ಅಳೆ-ದು-ಬಿ-ಟ್ಟರು
ಅಳೆ-ದು-ಬಿ-ಡುವ
ಅಳೆ-ಯ-ಬ-ಲ್ಲ-ವ-ರಾರು
ಅಳೆ-ಯು-ತ್ತಿ-ದ್ದುದು
ಅವ
ಅವಂ-ತಿ-ಪು-ರದ
ಅವ-ಕಾಶ
ಅವ-ಕಾ-ಶ-ಗ-ಳಿ-ರ-ಬೇಕು
ಅವ-ಕಾ-ಶ-ಗಳು
ಅವ-ಕಾ-ಶ-ವನ್ನು
ಅವ-ಕಾ-ಶ-ವನ್ನೇ
ಅವ-ಕಾ-ಶ-ವಾ-ಗ-ದ್ದ-ರಿಂದ
ಅವ-ಕಾ-ಶ-ವಾ-ಗಿತ್ತು
ಅವ-ಕಾ-ಶ-ವಾ-ಗಿ-ರ-ಲಿಲ್ಲ
ಅವ-ಕಾ-ಶ-ವಿತ್ತು
ಅವ-ಕಾ-ಶ-ವಿ-ದ್ದದ್ದು
ಅವ-ಕಾ-ಶ-ವಿ-ರ-ಬಾ-ರದು
ಅವ-ಕಾ-ಶ-ವಿ-ರ-ಲಿಲ್ಲ
ಅವ-ಕಾ-ಶ-ವಿಲ್ಲ
ಅವ-ಕಾ-ಶ-ವೆಂದು
ಅವ-ಕಾ-ಶವೇ
ಅವ-ಗಢ
ಅವ-ತ-ರಿಸಿ
ಅವ-ತ-ರಿ-ಸಿದ
ಅವ-ತಾರ
ಅವ-ತಾ-ರ-ಇ-ವೆಲ್ಲ
ಅವ-ತಾ-ರ-ಗಳ
ಅವ-ತಾ-ರ-ಗಳನ್ನು
ಅವ-ತಾ-ರ-ಗ-ಳಿ-ದ್ದಾರೆ
ಅವ-ತಾ-ರದ
ಅವ-ತಾ-ರ-ಪು-ರು-ಷ-ನಿಗೂ
ಅವ-ತಾ-ರ-ಪು-ರು-ಷರ
ಅವ-ತಾ-ರ-ಪು-ರು-ಷ-ರಿಗೂ
ಅವ-ತಾ-ರ-ಪು-ರು-ಷ-ರಿಗೆ
ಅವ-ತಾ-ರ-ಪು-ರು-ಷರು
ಅವ-ತಾ-ರ-ಪು-ರು-ಷರೆ
ಅವ-ತಾ-ರ-ಪು-ರು-ಷ-ರೆಂದು
ಅವ-ತಾ-ರ-ವ-ರಿ-ಷ್ಠಾಯ
ಅವ-ತಾ-ರ-ವಾ-ದದ
ಅವ-ತಾ-ರ-ವೆಂದು
ಅವ-ತಾ-ರ-ವೆಂದೂ
ಅವ-ತಾ-ರ-ವೆಂದೆ
ಅವ-ತಾ-ರ-ವೆ-ತ್ತುವ
ಅವ-ತಾ-ರ-ವೆ-ತ್ತು-ವು-ದಿಲ್ಲ
ಅವ-ತಾ-ರವೇ
ಅವ-ತಾ-ರ-ಸ-ಮಾಪ್ತಿ
ಅವ-ತಾ-ರ-ಸ-ಮಾ-ಪ್ತಿ-ಕಾಲ
ಅವ-ತಾ-ರೋ-ದ್ದೇ-ಶದ
ಅವ-ತಾ-ರೋ-ದ್ದೇ-ಶ-ವನ್ನು
ಅವಧಿ
ಅವ-ಧಿ-ಗಳ
ಅವ-ಧಿ-ಗ-ಳ-ಲ್ಲ-ಲ್ಲದೆ
ಅವ-ಧಿಗೆ
ಅವ-ಧಿಯ
ಅವ-ಧಿ-ಯನ್ನು
ಅವ-ಧಿ-ಯ-ಲ್ಲ-ಲ್ಲವೆ
ಅವ-ಧಿ-ಯಲ್ಲಿ
ಅವ-ಧಿ-ಯಲ್ಲೇ
ಅವ-ಧಿಯು
ಅವನ
ಅವ-ನಂತೆ
ಅವ-ನಂ-ತೆಯೇ
ಅವ-ನತಿ
ಅವ-ನ-ತಿಯ
ಅವ-ನತ್ತ
ಅವ-ನದೇ
ಅವ-ನದ್ದು
ಅವ-ನನ್ನು
ಅವ-ನ-ನ್ನು-ದ್ದೇ-ಶಿಸಿ
ಅವ-ನನ್ನೂ
ಅವ-ನನ್ನೇ
ಅವ-ನಲ್ಲಿ
ಅವ-ನ-ಲ್ಲಿಯೇ
ಅವ-ನ-ಲ್ಲಿ-ರುವ
ಅವ-ನ-ಲ್ಲೇನೂ
ಅವ-ನ-ವನ
ಅವ-ನಷ್ಟು
ಅವ-ನಿಂದ
ಅವ-ನಿ-ಗ-ನ್ನಿ-ಸಿತು
ಅವ-ನಿ-ಗಿತ್ತು
ಅವ-ನಿ-ಗಿ-ನ್ನೇನು
ಅವ-ನಿ-ಗಿ-ರದು
ಅವ-ನಿಗೂ
ಅವ-ನಿಗೆ
ಅವ-ನಿ-ಗೆಲ್ಲ
ಅವ-ನಿಗೇ
ಅವ-ನಿನ್ನೂ
ಅವನು
ಅವನೂ
ಅವ-ನೆಂದ
ಅವ-ನೆಂ-ದಂ-ತೆಯೇ
ಅವನೇ
ಅವ-ನೇಕೆ
ಅವ-ನೇಕೋ
ಅವ-ನೊಂ-ದಿಗೆ
ಅವ-ನೊಂದು
ಅವ-ನೊಬ್ಬ
ಅವ-ನೊ-ಳ-ಗಿನ
ಅವನ್ನು
ಅವ-ನ್ನೆಲ್ಲ
ಅವ-ಮಾ-ನ-ಕರ
ಅವ-ಮಾ-ನ-ವನ್ನು
ಅವರ
ಅವ-ರಂ-ತಹ
ಅವ-ರಂತೂ
ಅವ-ರಂ-ತೆಯೇ
ಅವ-ರಂ-ಥ-ವ-ರು-ಎಂ-ದರೆ
ಅವ-ರ-ದನ್ನು
ಅವ-ರ-ದಾ-ಗಿತ್ತು
ಅವ-ರದು
ಅವ-ರದೂ
ಅವ-ರದೇ
ಅವ-ರ-ದ್ದಾ-ಗಿತ್ತು
ಅವ-ರದ್ದು
ಅವ-ರ-ದ್ದೊಂದು
ಅವ-ರ-ನ್ನ-ಲ್ಲದೆ
ಅವ-ರನ್ನು
ಅವ-ರ-ನ್ನು-ಅ-ವರ
ಅವ-ರ-ನ್ನು-ದ್ದೇ-ಶಿಸಿ
ಅವ-ರನ್ನೂ
ಅವ-ರ-ನ್ನೆಲ್ಲ
ಅವ-ರನ್ನೇ
ಅವ-ರ-ಲ್ಲಾ-ಗಿದ್ದ
ಅವ-ರಲ್ಲಿ
ಅವ-ರ-ಲ್ಲಿಗೆ
ಅವ-ರ-ಲ್ಲಿತ್ತು
ಅವ-ರ-ಲ್ಲಿನ
ಅವ-ರ-ಲ್ಲಿ-ರ-ಲಿಲ್ಲ
ಅವ-ರ-ಲ್ಲಿ-ರುವ
ಅವ-ರ-ಲ್ಲೀಗ
ಅವ-ರ-ಲ್ಲುಂ-ಟಾ-ಗು-ತ್ತಿತ್ತು
ಅವ-ರ-ಲ್ಲುಂ-ಟಾ-ಯಿತು
ಅವ-ರ-ಲ್ಲು-ದಿ-ಸಿತು
ಅವ-ರಲ್ಲೂ
ಅವ-ರ-ಲ್ಲೆಲ್ಲ
ಅವ-ರಲ್ಲೇ
ಅವ-ರ-ಲ್ಲೊಂದು
ಅವ-ರ-ಲ್ಲೊ-ಬ್ಬನ
ಅವ-ರ-ಲ್ಲೊ-ಬ್ಬರು
ಅವ-ರ-ವರ
ಅವ-ರ-ವ-ರದೇ
ಅವ-ರ-ವರು
ಅವ-ರ-ವರೇ
ಅವ-ರ-ಷ್ಟಕ್ಕೆ
ಅವರಾ
ಅವ-ರಾ-ಗ-ಬೇಡಿ
ಅವ-ರಾ-ಡಿದ
ಅವ-ರಾ-ಡು-ತ್ತಿದ್ದ
ಅವ-ರಾ-ಡುವ
ಅವ-ರಾ-ದರೋ
ಅವ-ರಾರೂ
ಅವರಿ
ಅವ-ರಿಂದ
ಅವ-ರಿಂ-ದಲೇ
ಅವ-ರಿ-ಗ-ನ್ನಿ-ಸಿತು
ಅವ-ರಿ-ಗ-ನ್ನಿ-ಸಿ-ತು-ತಮ್ಮ
ಅವ-ರಿ-ಗ-ನ್ನಿ-ಸಿತ್ತು
ಅವ-ರಿ-ಗ-ರ್ಪಿ-ಸಿದ
ಅವ-ರಿ-ಗಾಗಿ
ಅವ-ರಿ-ಗಾದ
ಅವ-ರಿ-ಗಿಂತ
ಅವ-ರಿ-ಗಿಂ-ತಲೂ
ಅವ-ರಿ-ಗಿತ್ತು
ಅವ-ರಿ-ಗಿದ್ದ
ಅವ-ರಿ-ಗಿನ್ನೂ
ಅವ-ರಿ-ಗಿ-ರ-ಲಿಲ್ಲ
ಅವ-ರಿ-ಗಿ-ರುವ
ಅವ-ರಿ-ಗಿಲ್ಲ
ಅವ-ರಿ-ಗೀಗ
ಅವ-ರಿಗೂ
ಅವ-ರಿಗೆ
ಅವ-ರಿ-ಗೆಂ-ದರು
ಅವ-ರಿ-ಗೆಲ್ಲ
ಅವ-ರಿ-ಗೆಷ್ಟು
ಅವ-ರಿಗೇ
ಅವ-ರಿ-ಗೇನೂ
ಅವ-ರಿ-ಗೊಂದು
ಅವ-ರಿ-ಗೊಬ್ಬ
ಅವ-ರಿ-ಗೊ-ಬ್ಬರು
ಅವ-ರಿದ್ದ
ಅವ-ರಿ-ದ್ದ-ಲ್ಲಿಗೇ
ಅವ-ರಿನ್ನು
ಅವ-ರಿನ್ನೂ
ಅವ-ರಿ-ಬ್ಬ-ರಿಗೂ
ಅವ-ರಿ-ಬ್ಬರೂ
ಅವ-ರಿ-ಬ್ಬರೇ
ಅವ-ರಿ-ಬ್ಬ-ರೊ-ಡನೆ
ಅವ-ರಿ-ಳಿ-ದು-ಕೊಂ-ಡಿ-ದ್ದ-ಲ್ಲಿಗೆ
ಅವ-ರೀಗ
ಅವ-ರೀ-ರ್ವರೂ
ಅವರು
ಅವ-ರು-ಅ-ವರ
ಅವ-ರು-ಗಳ
ಅವ-ರು-ಗಳು
ಅವ-ರು-ಗ-ಳೊಂ-ದಿಗೆ
ಅವರೂ
ಅವ-ರೆಂ-ತಹ
ಅವ-ರೆಂ-ದರು
ಅವ-ರೆಂ-ದ-ರು-ಎಂ-ತಹ
ಅವ-ರೆಂ-ದ-ರು-ಹೋಗಿ
ಅವ-ರೆಂ-ದಿಗೂ
ಅವ-ರೆಂದೂ
ಅವರೆ-ಡೆಗೆ
ಅವರೆ-ದು-ರಿಗೆ
ಅವರೆ-ದೆ-ಯನ್ನು
ಅವರೆ-ನ್ನು-ತ್ತಾರೆ
ಅವರೆ-ನ್ನು-ತ್ತಿ-ದ್ದರು
ಅವರೆಲ್ಲ
ಅವರೆ-ಲ್ಲರ
ಅವರೆ-ಲ್ಲ-ರನ್ನೂ
ಅವರೆ-ಲ್ಲ-ರಿಗೂ
ಅವರೆ-ಲ್ಲರೂ
ಅವರೆ-ಲ್ಲ-ರೊಂ-ದಿಗೆ
ಅವರೆ-ಲ್ಲಿ-ದ್ದಾರೆ
ಅವ-ರೆಷ್ಟು
ಅವರೇ
ಅವ-ರೇಆ
ಅವ-ರೇನು
ಅವ-ರೇನೂ
ಅವ-ರೊಂ
ಅವ-ರೊಂದಿ
ಅವ-ರೊಂ-ದಿ-ಗಿ-ದ್ದ-ವ-ರೆಂ-ದರೆ
ಅವ-ರೊಂ-ದಿ-ಗಿದ್ದು
ಅವ-ರೊಂ-ದಿ-ಗಿನ
ಅವ-ರೊಂ-ದಿಗೆ
ಅವ-ರೊಂ-ದಿಗೇ
ಅವ-ರೊಂದು
ಅವ-ರೊ-ಡನೆ
ಅವ-ರೊಬ್ಬ
ಅವ-ರೊ-ಬ್ಬರು
ಅವ-ರೊ-ಬ್ಬರೇ
ಅವ-ರೊಮ್ಮೆ
ಅವ-ರೊ-ಳ-ಗಿನ
ಅವ-ರೋ-ಹಣ
ಅವ-ಲಂಬಿ
ಅವ-ಲಂ-ಬಿ-ತ-ರಾ-ಗು-ತ್ತಿ-ದ್ದ-ರೆಂ-ದರೆ
ಅವ-ಲಂ-ಬಿಸಿ
ಅವ-ಲಂ-ಬಿ-ಸಿ-ಕೊಂ-ಡಿ-ರು-ವುದು
ಅವ-ಲಂ-ಬಿ-ಸಿದೆ
ಅವ-ಲಕ್ಕಿ
ಅವ-ಲ-ಕ್ಷಣ
ಅವಳ
ಅವ-ಳದೇ
ಅವ-ಳದ್ದೇ
ಅವ-ಳನ್ನು
ಅವ-ಳ-ನ್ನೇನೂ
ಅವ-ಳಲ್ಲಿ
ಅವ-ಳ-ಲ್ಲುಂ-ಟಾ-ಯಿತು
ಅವ-ಳಲ್ಲೂ
ಅವ-ಳ-ಷ್ಟಕ್ಕೆ
ಅವಳಿ
ಅವ-ಳಿಂದ
ಅವ-ಳಿ-ಗ-ದ-ರಲ್ಲಿ
ಅವ-ಳಿ-ಗದು
ಅವ-ಳಿ-ಗನ್ನಿ
ಅವ-ಳಿ-ಗ-ನ್ನಿ-ಸ-ಲಿಲ್ಲ
ಅವ-ಳಿ-ಗಿನ್ನೂ
ಅವ-ಳಿ-ಗಿ-ರ-ಲಿಲ್ಲ
ಅವ-ಳಿಗೂ
ಅವ-ಳಿಗೆ
ಅವ-ಳಿಗೇ
ಅವ-ಳಿ-ಗೊಂದು
ಅವ-ಳಿನ್ನೂ
ಅವ-ಳಿ-ನ್ನೇನು
ಅವ-ಳಿ-ಲ್ಲ-ದಿ-ರು-ವುದು
ಅವ-ಳೀಗ
ಅವಳು
ಅವಳೂ
ಅವ-ಳೆಂದೂ
ಅವಳೇ
ಅವ-ಳೊಂ-ದಿಗೆ
ಅವ-ಳೊ-ಬ್ಬ-ಳಿಗೆ
ಅವ-ಶೇ-ಷ-ಗಳ
ಅವ-ಶೇ-ಷ-ಗಳನ್ನು
ಅವ-ಶೇ-ಷ-ಗ-ಳ-ನ್ನೊ-ಳ-ಗೊಂಡ
ಅವ-ಶೇ-ಷ-ಗ-ಳಾ-ಗಿ-ರುವ
ಅವ-ಶೇ-ಷ-ಗ-ಳಿ-ಗಿಂತ
ಅವ-ಶೇ-ಷ-ಗಳು
ಅವ-ಶೇ-ಷದ
ಅವ-ಶ್ಯ-ವಾಗಿ
ಅವ-ಸರ
ಅವ-ಸ-ರದ
ಅವ-ಸ-ರ-ವ-ಸ-ರ-ವಾಗಿ
ಅವ-ಸ್ಥೆಗೆ
ಅವ-ಸ್ಥೆ-ಯನ್ನು
ಅವ-ಹೇ-ಳನ
ಅವ-ಹೇ-ಳ-ನಕ್ಕೆ
ಅವ-ಹೇ-ಳ-ನದ
ಅವಿ-ಚ-ಲ-ರಾಗಿ
ಅವಿ-ಚ್ಛಿನ್ನ
ಅವಿ-ದ್ಯಾ-ವಂ-ತ-ರಿಗೆ
ಅವಿ-ಧೇ-ಯ-ತೆಯೇ
ಅವಿ-ನಾ-ಶಿ-ಯಾದ
ಅವಿ-ರತ
ಅವಿ-ವಾ-ಹಿತ
ಅವಿ-ವೇ-ಕದ
ಅವಿ-ವೇ-ಕ-ದಿಂದ
ಅವಿ-ಶ್ರಾಂತ
ಅವಿ-ಶ್ರಾಂ-ತ-ವಾಗಿ
ಅವಿ-ಸ್ಮ-ರ-ಣೀಯ
ಅವಿ-ಸ್ಮ-ರ-ಣೀ-ಯ-ವಾದ
ಅವು
ಅವು-ಗಳ
ಅವು-ಗ-ಳ-ನ್ನೀಗ
ಅವು-ಗಳನ್ನು
ಅವು-ಗಳನ್ನೂ
ಅವು-ಗಳನ್ನೆಲ್ಲ
ಅವು-ಗ-ಳನ್ನೇ
ಅವು-ಗಳಲ್ಲಿ
ಅವು-ಗ-ಳ-ಲ್ಲೆಲ್ಲ
ಅವು-ಗಳಿಂದ
ಅವು-ಗ-ಳಿಂ-ದಲೇ
ಅವು-ಗ-ಳಿಂ-ದೆಲ್ಲ
ಅವು-ಗ-ಳಿ-ಗ-ನು-ಸಾ-ರ-ವಾಗಿ
ಅವು-ಗ-ಳಿಗೆ
ಅವು-ಗ-ಳಿ-ಗೆಲ್ಲ
ಅವು-ಗಳು
ಅವು-ಗ-ಳೆಲ್ಲ
ಅವು-ಗ-ಳೆ-ಲ್ಲ-ದಕ್ಕೂ
ಅವು-ಗ-ಳೆ-ಲ್ಲ-ದರ
ಅವು-ಗ-ಳೆ-ಲ್ಲ-ದ-ರಿಂ-ದಲೂ
ಅವು-ಗ-ಳೆ-ಲ್ಲ-ವನ್ನೂ
ಅವು-ಗ-ಳೊಂ-ದಿಗೆ
ಅವೆ-ರ-ಡನ್ನೂ
ಅವೆ-ರ-ಡರ
ಅವೆ-ರಡೂ
ಅವೆಲ್ಲ
ಅವೆ-ಲ್ಲಕ್ಕೂ
ಅವೆ-ಲ್ಲ-ವನ್ನೂ
ಅವೆ-ಲ್ಲವೂ
ಅವೈ-ಜ್ಞಾ-ನಿಕ
ಅವ್ಯಕ್ತ
ಅವ್ಯ-ಕ್ತವು
ಅವ್ಯ-ವ-ಸ್ಥಿತ
ಅಶ-ರೀ-ರ-ವಾಣಿ
ಅಶ-ರೀ-ರ-ವಾ-ಣಿಯ
ಅಶ-ರೀ-ರ-ವಾ-ಣಿ-ಯಂತೆ
ಅಶಾಂ-ತ-ರ-ನ್ನಾ-ಗಿ-ಸಿ-ದೆಯೆ
ಅಶಾಂ-ತಿಯ
ಅಶಿ-ಸ್ತಿ-ನಿಂ-ದಿ-ರು-ವುದೇ
ಅಶುದ್ಧ
ಅಶು-ಭವೂ
ಅಶೋ-ಕನ
ಅಶೋಕಾ
ಅಶೋ-ಕಾ-ಷ್ಟ-ಮಿಯೇ
ಅಶ್ರದ್ಧೆ
ಅಶ್ರು
ಅಶ್ರು-ತ-ರ್ಪ-ಣ-ವ-ನ್ನ-ರ್ಪಿ-ಸಿ-ದರು
ಅಶ್ರು-ತ-ರ್ಪ-ಣ-ವನ್ನು
ಅಶ್ರು-ಧಾರೆ
ಅಶ್ರು-ಪ್ರ-ವಾಹ
ಅಶ್ಲೀಲ
ಅಶ್ವಿನೀ
ಅಶ್ವಿ-ನೀ-ಕು-ಮಾರ
ಅಶ್ವಿ-ನೀ-ಕು-ಮಾ-ರರು
ಅಶ್ವಿ-ನೀ-ಬಾಬು
ಅಶ್ವಿ-ನೀ-ಬಾ-ಬು-ಗಳ
ಅಶ್ವಿ-ನೀ-ಬಾ-ಬು-ಗಳನ್ನು
ಅಶ್ವಿ-ನೀ-ಬಾ-ಬು-ಗಳು
ಅಷ್ಟ-ಕ್ಕಷ್ಟೆ
ಅಷ್ಟಕ್ಕೇ
ಅಷ್ಟನ್ನು
ಅಷ್ಟಮಿ
ಅಷ್ಟ-ರ-ಮ-ಟ್ಟಿಗೆ
ಅಷ್ಟ-ರ-ಲ್ಲಾ-ಗಲೇ
ಅಷ್ಟ-ರಲ್ಲಿ
ಅಷ್ಟ-ರಲ್ಲೇ
ಅಷ್ಟ-ರೊ-ಳಗೆ
ಅಷ್ಟ-ಲ್ಲದೆ
ಅಷ್ಟಷ್ಟು
ಅಷ್ಟಾಗಿ
ಅಷ್ಟಾ-ಧ್ಯಾಯಿ
ಅಷ್ಟಾ-ವಿಂ-ಶತಿ
ಅಷ್ಟಿ-ಷ್ಟಲ್ಲ
ಅಷ್ಟು
ಅಷ್ಟೆ
ಅಷ್ಟೆ-ನ-ನ-ಗಿ-ನ್ನೇನೂ
ಅಷ್ಟೆಲ್ಲ
ಅಷ್ಟೇ
ಅಷ್ಟೇಕೆ
ಅಷ್ಟೇ-ಕೆ-ತಮ್ಮ
ಅಷ್ಟೇನೂ
ಅಷ್ಟೊಂದು
ಅಸಂ
ಅಸಂಖ್ಯ
ಅಸಂ-ಖ್ಯಾತ
ಅಸಂ-ಗ-ತ-ವಾಗಿ
ಅಸಂ-ಬದ್ಧ
ಅಸಂ-ಬ-ದ್ಧ-ತೆ-ಗ-ಳಾ-ಗಲಿ
ಅಸಂ-ಬ-ದ್ಧ-ರಾ-ಗಿ-ರಲು
ಅಸಡ್ಡೆ
ಅಸ-ಡ್ಡೆ-ಯುಂ-ಟಾ-ಗಿ-ರು-ವುದು
ಅಸ-ತ್ಯ-ದಿಂದ
ಅಸ-ತ್ಯ-ವಾದ
ಅಸ-ತ್ಯವು
ಅಸ-ತ್ಯ-ವೆಂದು
ಅಸ-ದೃಶ
ಅಸ-ದೃ-ಶ-ಅ-ದ್ಭುತ
ಅಸ-ದೃ-ಶ-ವಾ-ದದ್ದು
ಅಸಭ್ಯ
ಅಸ-ಭ್ಯರ
ಅಸ-ಮಂ-ಜ-ಸ-ತೆ-ಗಳನ್ನು
ಅಸ-ಮಂ-ಜ-ಸ-ತೆ-ಯನ್ನು
ಅಸ-ಮಂ-ಜ-ಸ-ವಾ-ಗ-ಬ-ಹುದು
ಅಸ-ಮಂ-ಜ-ಸ-ವಾ-ದದ್ದು
ಅಸ-ಮರ್ಥ
ಅಸ-ಮ-ರ್ಥ-ರಾಗಿ
ಅಸ-ಮ-ರ್ಥ-ರಾ-ದರು
ಅಸ-ಮ-ರ್ಥ-ರಾ-ದುದು
ಅಸ-ಮ-ರ್ಥರು
ಅಸ-ಮ-ರ್ಥ-ಳಾಗಿ
ಅಸ-ಮ-ರ್ಥ-ವಾ-ಗಿ-ರು-ವು-ದ-ರಿಂದ
ಅಸ-ಮಾ-ಜದ
ಅಸ-ಮಾ-ಧಾನ
ಅಸ-ಮಾ-ಧಾ-ನ-ಗೊ-ಳ್ಳಲು
ಅಸ-ಮಾ-ಧಾ-ನ-ವಾ-ಯಿತು
ಅಸ-ಮಾ-ಧಾ-ನ-ವಿತ್ತು
ಅಸಹ
ಅಸ-ಹ-ಜ-ವಾ-ದ-ದ್ದಾ-ಗಲಿ
ಅಸ-ಹ-ಜ-ವಾ-ದದ್ದು
ಅಸ-ಹ-ನೀಯ
ಅಸ-ಹ-ನೀ-ಯ-ವಾಗಿ
ಅಸ-ಹ-ನೀ-ಯ-ವಾ-ಗಿತ್ತು
ಅಸ-ಹ-ನೀ-ಯ-ವಾ-ಗು-ತ್ತದೆ
ಅಸ-ಹನೆ
ಅಸ-ಹ-ನೆಯ
ಅಸ-ಹ-ನೆ-ಯಿಂದ
ಅಸ-ಹಾ-ಯಕ
ಅಸ-ಹಾ-ಯ-ಕ-ತೆ-ಯನ್ನು
ಅಸ-ಹಾ-ಯ-ಕ-ನಾಗಿ
ಅಸ-ಹಾ-ಯ-ಕ-ರನ್ನು
ಅಸ-ಹಾ-ಯ-ಕ-ರಾಗಿ
ಅಸ-ಹಾ-ಯ-ಕ-ರಿಗೆ
ಅಸ-ಹ್ಯ-ಕರ
ಅಸ-ಹ್ಯ-ಮ-ಡಿ-ವಂ-ತಿ-ಕೆ-ಯನ್ನು
ಅಸಾಂ-ಪ್ರ-ದಾ-ಯಿ-ಕ-ವಾದ
ಅಸಾ-ಧಾ-ರಣ
ಅಸಾಧ್ಯ
ಅಸಾ-ಧ್ಯ-ವಾ-ಗಿತ್ತು
ಅಸಾ-ಧ್ಯ-ವಾ-ದುದು
ಅಸಾ-ಮರ್ಥ್ಯ
ಅಸಾ-ಮಾನ್ಯ
ಅಸೀಮ
ಅಸೂ-ಯಾ-ಪರ
ಅಸೂಯೆ
ಅಸೂ-ಯೆ-ಗಳ
ಅಸೋ-ಸಿ-ಯೇ-ಶನ್
ಅಸೌ-ಖ್ಯ-ವಿ-ತ್ತಾ-ದರೂ
ಅಸೌ-ಖ್ಯವೋ
ಅಸ್ತಂ-ಗ-ತ-ನಾ-ಗುವ
ಅಸ್ತ-ಮಿ-ಸು-ತ್ತಿದ್ದ
ಅಸ್ತ-ಮಿ-ಸು-ತ್ತಿ-ದ್ದು-ದು-ಇ-ವೆಲ್ಲ
ಅಸ್ತ-ವ್ಯಸ್ತ
ಅಸ್ತಿ
ಅಸ್ತಿತ್ವ
ಅಸ್ತಿ-ತ್ವಕ್ಕೆ
ಅಸ್ತಿ-ತ್ವದ
ಅಸ್ತಿ-ತ್ವ-ದಲ್ಲಿ
ಅಸ್ತಿ-ತ್ವ-ದ-ಲ್ಲಿ-ರುವ
ಅಸ್ತಿ-ತ್ವ-ದಷ್ಟೇ
ಅಸ್ತಿ-ತ್ವ-ವನ್ನು
ಅಸ್ತಿ-ಭಾ-ರದ
ಅಸ್ತಿ-ಭಾ-ರ-ವನ್ನು
ಅಸ್ತಿ-ಭಾ-ರವೇ
ಅಸ್ತ್ರ-ದೊಂ-ದಿಗೆ
ಅಸ್ಥಿ
ಅಸ್ಥಿ-ಮಾಂ-ಸ-ಗಳ
ಅಸ್ಥಿಯ
ಅಸ್ಥಿ-ಯ-ನ್ನಿ-ಟ್ಟಿದ್ದ
ಅಸ್ಥಿ-ಯನ್ನು
ಅಸ್ಥಿರ
ಅಸ್ಥಿ-ರ-ವಾದ
ಅಸ್ಥಿ-ಶ-ಕ್ತಿ-ಯನ್ನೇ
ಅಸ್ಥಿ-ಸಂ-ಪು-ಟ-ವನ್ನು
ಅಸ್ಪಷ್ಟ
ಅಸ್ಪ-ಷ್ಟ-ವಾಗಿ
ಅಸ್ಪೃ-ಶ್ಯತಾ
ಅಸ್ಪೃ-ಶ್ಯ-ತೆ-ಯೆಂಬ
ಅಸ್ಪೃ-ಶ್ಯರ
ಅಸ್ಪೃ-ಶ್ಯ-ರ-ನ್ನಾಗಿ
ಅಸ್ಪೃ-ಶ್ಯ-ರೆಂ-ಬ-ವರು
ಅಸ್ಸಾ-ಮಿನ
ಅಹಂ
ಅಹಂ-ಕಾರ
ಅಹಂ-ಕಾ-ರ-ಸ್ವಾ-ರ್ಥ-ತೆ-ಗಳ
ಅಹಂ-ಕಾ-ರ-ಗಳನ್ನು
ಅಹಂ-ಕಾ-ರ-ದಿಂ-ದಾಗಿ
ಅಹಂ-ಕಾ-ರ-ದೊಂ-ದಿಗೆ
ಅಹಂ-ಕಾ-ರ-ವ-ನ್ನೆಲ್ಲ
ಅಹ-ಮಿ-ಕೆ-ಯನ್ನು
ಅಹ-ಮಿ-ಕೆಯು
ಅಹ-ಮಿ-ಕೆ-ಯೆಂ-ಬುದು
ಅಹಿಂ-ಸಾ-ತ-ತ್ತ್ವ-ವನ್ನೇ
ಅಹಿಂ-ಸೆಯ
ಅಹಿಂ-ಸೆ-ಯನ್ನು
ಅಹಿ-ತ-ಕರ
ಅಹಿ-ತ-ಕ-ರ-ವಾ-ಗು-ವಷ್ಟು
ಅಹೈ-ತುಕ
ಅಹ್
ಆ
ಆಂಗ್ಲ
ಆಂಗ್ಲ-ಭಾ-ಷೆಯ
ಆಂಗ್ಲರ
ಆಂಗ್ಲ-ರನ್ನು
ಆಂಗ್ಲರು
ಆಂಗ್ಲ-ರೆ-ಲ್ಲರೂ
ಆಂಗ್ಲೇಯ
ಆಂಗ್ಲೇ-ಯ-ನಲ್ಲಿ
ಆಂಜ-ನೇ-ಯನ
ಆಂತ-ರಂ-ಗಿಕ
ಆಂತ-ರಿಕ
ಆಂತ-ರ್ಯದ
ಆಂತ-ರ್ಯ-ದಲ್ಲಿ
ಆಂತ-ರ್ಯ-ದಲ್ಲೇ
ಆಂತ-ರ್ಯ-ವನ್ನು
ಆಂತ-ರ್ಯ-ವನ್ನೇ
ಆಂದೋ-ಲ-ನ-ಗಳ
ಆಂದೋ-ಲ-ನ-ವಾ-ಗ-ಬೇಕು
ಆಂಧ್ರ-ಪ್ರ-ದೇ-ಶದ
ಆಕ-ರ್ಷ-ಕ-ವಾ-ಗಿತ್ತು
ಆಕ-ರ್ಷಣ
ಆಕ-ರ್ಷ-ಣೀ-ಯ-ವಾ-ಗಿಯೇ
ಆಕ-ರ್ಷಣೆ
ಆಕ-ರ್ಷ-ಣೆಯ
ಆಕ-ರ್ಷ-ಣೆ-ಯನ್ನು
ಆಕ-ರ್ಷ-ಣೆಯು
ಆಕ-ರ್ಷಿತ
ಆಕ-ರ್ಷಿ-ತ-ನಾಗಿ
ಆಕ-ರ್ಷಿ-ತ-ನಾ-ಗಿದ್ದ
ಆಕ-ರ್ಷಿ-ತ-ರಾ-ಗಿದ್ದ
ಆಕ-ರ್ಷಿ-ತ-ರಾ-ಗಿ-ದ್ದರು
ಆಕ-ರ್ಷಿ-ತ-ರಾ-ದರು
ಆಕ-ರ್ಷಿ-ತ-ಳಾ-ದಳು
ಆಕ-ರ್ಷಿ-ಸ-ಬ-ಹು-ದೆಂದು
ಆಕ-ರ್ಷಿ-ಸಲು
ಆಕ-ರ್ಷಿಸಿ
ಆಕ-ರ್ಷಿ-ಸಿ-ದುವು
ಆಕ-ರ್ಷಿ-ಸಿ-ದ್ದಾರೆ
ಆಕ-ಸ್ಮಿಕ
ಆಕ-ಸ್ಮಿ-ಕ-ವಾಗಿ
ಆಕಾಂಕ್ಷೆ
ಆಕಾಂ-ಕ್ಷೆ-ನಿ-ರೀ-ಕ್ಷೆ-ಗಳೂ
ಆಕಾಂ-ಕ್ಷೆಯ
ಆಕಾಂ-ಕ್ಷೆ-ಯಿಲ್ಲ
ಆಕಾರ
ಆಕಾ-ರ-ಗಾ-ತ್ರ-ವ-ರ್ಣ-ಗಳಿಂದ
ಆಕಾ-ರಕ್ಕೆ
ಆಕಾ-ರ-ಗಳಿಂದ
ಆಕಾ-ರ-ವನ್ನು
ಆಕಾಶ
ಆಕಾ-ಶ-ದಂತೆ
ಆಕಾ-ಶ-ದಲ್ಲಿ
ಆಕಾ-ಶ-ದಲ್ಲೆಲ್ಲ
ಆಕಾ-ಶ-ದಿಂದ
ಆಕಾ-ಶ-ಬಾ-ಣ-ಗಳನ್ನು
ಆಕಾ-ಶ-ಯಾ-ನದ
ಆಕೃ-ತಿ-ಗ-ಳಿವೆ
ಆಕೃ-ತಿ-ಯನ್ನು
ಆಕೆ
ಆಕೆ-ಗದು
ಆಕೆ-ಗಾಗಿ
ಆಕೆಗೆ
ಆಕೆ-ಗೊಂದು
ಆಕೆಯ
ಆಕೆ-ಯತ್ತ
ಆಕೆ-ಯನ್ನು
ಆಕೆ-ಯಲ್ಲಿ
ಆಕೆಯು
ಆಕೆ-ಯೆಂ-ದಳು
ಆಕೆ-ಯೆಂ-ದಾಗ
ಆಕೆಯೇ
ಆಕ್ರಂ-ದ-ನಕ್ಕೆ
ಆಕ್ರ-ಮಣ
ಆಕ್ರ-ಮ-ಣ-ಗಳ
ಆಕ್ರ-ಮ-ಣ-ದಿಂದ
ಆಕ್ರ-ಮಿ-ಸಿ-ಬಿ-ಟ್ಟಿತ್ತು
ಆಕ್ರೋ-ಶಕ್ಕೆ
ಆಕ್ರೋ-ಶ-ದಿಂದ
ಆಕ್ಷೇ-ಪ-ಣೆ-ಯೆತ್ತಿ
ಆಕ್ಷೇ-ಪಿ-ಸಿದ
ಆಗ
ಆಗ-ತಾನೆ
ಆಗ-ದಿ-ರಲಿ
ಆಗ-ದಿ-ರು-ವು-ದ-ಕ್ಕಿಂತ
ಆಗ-ದಿ-ರು-ವು-ದಿಲ್ಲ
ಆಗ-ಬ-ಹು-ದಾದ
ಆಗ-ಬ-ಹುದು
ಆಗ-ಬಾ-ರದು
ಆಗ-ಬಾ-ರದ್ದು
ಆಗ-ಬೇ-ಕಾ-ಗಿ-ದೆಯೇ
ಆಗ-ಬೇ-ಕಾ-ಗಿದ್ದ
ಆಗ-ಬೇ-ಕಾ-ಗಿ-ರುವ
ಆಗ-ಬೇ-ಕಾ-ಗು-ತ್ತಿತ್ತು
ಆಗ-ಬೇ-ಕಾದ
ಆಗ-ಬೇಕು
ಆಗ-ಮನ
ಆಗ-ಮ-ನ-ಕ್ಕಾಗಿ
ಆಗ-ಮ-ನಕ್ಕೆ
ಆಗ-ಮ-ನದ
ಆಗ-ಮ-ನ-ದಿಂದ
ಆಗ-ಮ-ನ-ದಿಂ-ದಾಗಿ
ಆಗ-ಮ-ನ-ವನ್ನು
ಆಗ-ಮ-ನ-ವನ್ನೇ
ಆಗ-ಮಿಸಿ
ಆಗ-ಮಿ-ಸಿದ
ಆಗ-ಮಿ-ಸಿ-ದರು
ಆಗ-ಮಿ-ಸಿ-ದಾಗ
ಆಗ-ಮಿ-ಸಿ-ದ್ದ-ರಿಂದ
ಆಗ-ಮಿ-ಸಿ-ದ್ದರು
ಆಗ-ಮಿ-ಸಿ-ದ್ದಾಗ
ಆಗ-ಮಿ-ಸಿ-ದ್ದಾರೆ
ಆಗ-ಮಿ-ಸಿ-ರುವ
ಆಗ-ಮಿ-ಸೀತು
ಆಗ-ಮಿಸು
ಆಗ-ಮಿ-ಸು-ತ್ತಿ-ದ್ದಾ-ರೆಂದು
ಆಗ-ಮಿ-ಸುವ
ಆಗ-ಲಾರ
ಆಗ-ಲಾ-ರದು
ಆಗಲಿ
ಆಗ-ಲಿ-ತೆ-ರೆದ
ಆಗ-ಲಿ-ಯಾ-ರಿಗೂ
ಆಗ-ಲಿ-ಯಾ-ವಾ-ಗಲೂ
ಆಗ-ಲಿಲ್ಲ
ಆಗಲು
ಆಗಲೂ
ಆಗಲೆ
ಆಗಲೇ
ಆಗ-ಸ-ಸಾ-ಗ-ರ-ಗಳನ್ನು
ಆಗ-ಸಕ್ಕೆ
ಆಗ-ಸದ
ಆಗ-ಸ-ದಂತೆ
ಆಗ-ಸ-ದಲ್ಲಿ
ಆಗ-ಸ-ವನು
ಆಗ-ಸ್ಟಿ-ನಲ್ಲಿ
ಆಗ-ಸ್ಟ್
ಆಗ-ಸ್ಟ-್-ಸೆ-ಪ್ಟೆಂ-ಬರ್
ಆಗಾಗ
ಆಗಾ-ಗಲೇ
ಆಗಿ
ಆಗಿ-ತ್ತಾ-ದರೂ
ಆಗಿತ್ತು
ಆಗಿ-ತ್ತೆಂದು
ಆಗಿ-ತ್ತೆಂಬು
ಆಗಿ-ತ್ತೆ-ನ್ನ-ಬೇಕು
ಆಗಿದೆ
ಆಗಿ-ದೆ-ಯೆಂದು
ಆಗಿ-ದೆಯೇ
ಆಗಿ-ದೆಯೋ
ಆಗಿದ್ದ
ಆಗಿ-ದ್ದರು
ಆಗಿ-ದ್ದರೂ
ಆಗಿ-ದ್ದ-ರೆಂ-ಬುದು
ಆಗಿ-ದ್ದಳು
ಆಗಿ-ದ್ದಾನೆ
ಆಗಿ-ದ್ದಾರೆ
ಆಗಿ-ದ್ದಾ-ರೆ-ಸ್ಟ-ರ್ಡಿಯೂ
ಆಗಿ-ದ್ದಿ-ರ-ಬೇಕು
ಆಗಿ-ದ್ದೀಯೆ
ಆಗಿ-ದ್ದೀರಿ
ಆಗಿದ್ದು
ಆಗಿ-ದ್ದುವು
ಆಗಿದ್ದೆ
ಆಗಿ-ದ್ದೇವೆ
ಆಗಿನ
ಆಗಿ-ನಿಂದ
ಆಗಿ-ನಿಂ-ದಲೂ
ಆಗಿನ್ನೂ
ಆಗಿ-ಬಿ-ಟ್ಟಳು
ಆಗಿ-ಬಿ-ಟ್ಟಿತು
ಆಗಿ-ಬಿ-ಟ್ಟಿತ್ತು
ಆಗಿ-ಬಿ-ಟ್ಟಿ-ದ್ದರು
ಆಗಿ-ಬಿ-ಟ್ಟಿ-ದ್ದ-ವಂತೆ
ಆಗಿ-ಬಿ-ಟ್ಟಿ-ದ್ದಾರೆ
ಆಗಿ-ಬಿ-ಟ್ಟಿ-ದ್ದುವು
ಆಗಿ-ಬಿ-ಟ್ಟಿದ್ದೆ
ಆಗಿ-ಬಿ-ಟ್ಟಿ-ದ್ದೇವೆ
ಆಗಿ-ಬಿ-ಡು-ತ್ತಾರೆ
ಆಗಿ-ಬಿ-ಡು-ತ್ತಿ-ದ್ದಿರಿ
ಆಗಿ-ರ-ಬ-ಹು-ದ-ಲ್ಲವೆ
ಆಗಿ-ರ-ಬೇಕು
ಆಗಿ-ರಲಿ
ಆಗಿ-ರ-ಲಿ-ಅ-ದ-ರಲ್ಲಿ
ಆಗಿ-ರ-ಲಿ-ಎ-ತ್ತಿ-ಹಿ-ಡಿ-ಯುವ
ಆಗಿ-ರ-ಲಿಲ್ಲ
ಆಗಿ-ರು-ತ್ತಿದ್ದ
ಆಗಿ-ರು-ತ್ತಿ-ದ್ದರು
ಆಗಿ-ರುವ
ಆಗಿ-ರು-ವಂ-ತಹ
ಆಗಿ-ರು-ವನೋ
ಆಗಿ-ರು-ವು-ದ-ಕ್ಕಿಂತ
ಆಗಿ-ರು-ವು-ದೆಲ್ಲ
ಆಗಿಲ್ಲ
ಆಗಿವೆ
ಆಗಿ-ವೆ-ಯೆಂ-ಬು-ದನ್ನು
ಆಗಿ-ಹೋ-ಗಿತ್ತು
ಆಗಿ-ಹೋ-ಗಿ-ರ-ಬ-ಹುದು
ಆಗಿ-ಹೋ-ಗಿವೆ
ಆಗಿ-ಹೋ-ದ-ದ್ದನ್ನೇ
ಆಗಿ-ಹೋ-ಯಿತು
ಆಗು-ತ್ತದೆ
ಆಗು-ತ್ತ-ದೆ-ಯೆಂದು
ಆಗು-ತ್ತಾನೆ
ಆಗು-ತ್ತಿತ್ತು
ಆಗು-ತ್ತಿ-ರ-ಲಿಲ್ಲ
ಆಗು-ತ್ತೇನೆ
ಆಗುವ
ಆಗು-ವಿರಿ
ಆಗು-ವು-ದ-ಕ್ಕಿ-ಲ್ಲವೆ
ಆಗು-ವು-ದ-ರಲ್ಲಿ
ಆಗು-ವು-ದಿಲ್ಲ
ಆಗು-ವುದು
ಆಗು-ಹೋ-ಗು-ಗಳ
ಆಗು-ಹೋ-ಗು-ಗಳನ್ನು
ಆಗೆಲ್ಲ
ಆಗೇನೂ
ಆಗೊಂದು
ಆಗೊಮ್ಮೆ
ಆಗ್ರಹ
ಆಗ್ರ-ಹ-ಪ-ಡಿ-ಸಿ-ದ್ದ-ರಿಂದ
ಆಗ್ರ-ಹ-ಪೂ-ರ್ವಕ
ಆಗ್ರ-ಹ-ಪೂ-ರ್ವ-ಕ-ವಾಗಿ
ಆಗ್ರಾ
ಆಗ್ರಾದ
ಆಘಾತ
ಆಘಾ-ತ-ಕರ
ಆಘಾ-ತ-ಗ-ಳಾ-ದರೂ
ಆಘಾ-ತ-ಗ-ಳಿಂ-ದಾಗಿ
ಆಘಾ-ತ-ದಂತೆ
ಆಘಾ-ತ-ವ-ನ್ನಿ-ತ್ತ-ರು-ನೀವು
ಆಘಾ-ತ-ವನ್ನು
ಆಘಾ-ತ-ವ-ನ್ನುಂ-ಟು-ಮಾ-ಡಿದ
ಆಘಾ-ತ-ವಾ-ಗ-ಬ-ಹುದು
ಆಘಾ-ತ-ವಾಗಿ
ಆಘಾ-ತ-ವಾ-ದರೆ
ಆಘಾ-ತವೇ
ಆಚ
ಆಚಂ-ಡಾ-ಲಾ-ಪ್ರ-ತಿ-ಹ-ತ-ರಯೋ
ಆಚ-ರ-ಣಾ-ವಿ-ಧಾ-ನ-ವನ್ನು
ಆಚ-ರಣೆ
ಆಚ-ರ-ಣೆ-ಗಳನ್ನು
ಆಚ-ರ-ಣೆ-ಗಳಿಂದ
ಆಚ-ರ-ಣೆಗೆ
ಆಚ-ರ-ಣೆ-ಯಲ್ಲೇ
ಆಚರಿ
ಆಚ-ರಿ-ಸ-ಬೇ-ಕೆಂಬ
ಆಚ-ರಿ-ಸ-ಲಾ-ಗುವ
ಆಚ-ರಿ-ಸ-ಲಾ-ಗು-ವು-ದಿಲ್ಲ
ಆಚ-ರಿ-ಸ-ಲಾ-ಯಿತು
ಆಚ-ರಿ-ಸಿದ
ಆಚ-ರಿ-ಸಿ-ದರು
ಆಚ-ರಿ-ಸಿದ್ದು
ಆಚ-ರಿ-ಸಿಯೇ
ಆಚ-ರಿ-ಸು-ತ್ತಾರೆ
ಆಚ-ರಿ-ಸು-ತ್ತಿ-ದ್ದಳು
ಆಚ-ರಿ-ಸುವ
ಆಚ-ರಿ-ಸು-ವಂ-ತಿಲ್ಲ
ಆಚ-ರಿ-ಸು-ವಂತೆ
ಆಚ-ರಿ-ಸು-ವ-ವನು
ಆಚ-ರಿ-ಸು-ವ-ವರು
ಆಚ-ರಿ-ಸು-ವು-ದರ
ಆಚಾರ
ಆಚಾ-ರ-ಪ-ದ್ಧ-ತಿ-ಗಳನ್ನೂ
ಆಚಾ-ರ-ಸಂಪ್ರ
ಆಚಾ-ರ-ಗ-ಳಾ-ಗಲಿ
ಆಚಾ-ರ-ವಂತ
ಆಚಾ-ರ-ವಂ-ತ-ರಾದ
ಆಚಾ-ರ-ವಂ-ತ-ರಿ-ಗಿಂ-ತಲೂ
ಆಚಾ-ರ-ವಂ-ತರು
ಆಚಾರ್ಯ
ಆಚಾ-ರ್ಯ-ನ-ನ್ನಾಗಿ
ಆಚಾ-ರ್ಯ-ರಾಗಿ
ಆಚೆ
ಆಚೆಗೆ
ಆಚೆಯ
ಆಜ-ನ್ಮ-ಬ್ರ-ಹ್ಮ-ಚಾ-ರಿಣಿ
ಆಜೀವ
ಆಜ್ಞಾ-ಪನೆ
ಆಜ್ಞಾ-ಪಿ-ಸಿ-ದರು
ಆಜ್ಞೆ
ಆಜ್ಞೆ-ಗಳನ್ನು
ಆಜ್ಞೆಗೆ
ಆಜ್ಞೆಯ
ಆಜ್ಞೆ-ಯನ್ನು
ಆಟ
ಆಟದ
ಆಟ-ದಲ್ಲಿ
ಆಟ-ವಾಡಿ
ಆಟ-ವಾ-ಡಿ-ಕೊಂಡು
ಆಟ-ವಾ-ಡುತ್ತ
ಆಟ-ವಾ-ಡು-ತ್ತಿ-ದ್ದರು
ಆಟ-ವಾ-ಡು-ತ್ತಿ-ದ್ದಾ-ಗ್ಗಿಂತ
ಆಟ-ವಾ-ಡು-ತ್ತಿದ್ದೆ
ಆಟ-ವಾ-ಡು-ತ್ತಿ-ರು-ವುದನ್ನು
ಆಟಿ-ಕೆ-ಗ-ಳೆಂದು
ಆಟಿ-ಕೆ-ಯನ್ನೋ
ಆಡಂ-ಬ-ರ-ಅ-ಲಂ-ಕಾ-ರ-ಗಳನ್ನು
ಆಡದೆ
ಆಡ-ಬ-ಲ್ಲಿರಾ
ಆಡ-ಬೇ-ಕಾ-ಗಿಲ್ಲ
ಆಡ-ಲಿಲ್ಲ
ಆಡಲು
ಆಡ-ಳಿತ
ಆಡ-ಳಿ-ತಕ್ಕೆ
ಆಡ-ಳಿ-ತ-ಕ್ಕೊ-ಳ-ಪ-ಟ್ಟಿತ್ತು
ಆಡ-ಳಿ-ತ-ಕ್ಕೊ-ಳ-ಪ-ಟ್ಟಿ-ರುವ
ಆಡ-ಳಿ-ತದ
ಆಡ-ಳಿ-ತ-ದಲ್ಲಿ
ಆಡ-ಳಿ-ತ-ವನ್ನು
ಆಡ-ಳಿ-ತವು
ಆಡಿ
ಆಡಿ-ಕೊಂಡು
ಆಡಿದ
ಆಡಿ-ದಂ-ತಹ
ಆಡಿ-ದರು
ಆಡಿ-ದ-ವು-ಗ-ಳಲ್ಲ
ಆಡಿ-ದಷ್ಟು
ಆಡಿನ
ಆಡಿ-ಬಿ-ಡು-ತ್ತೇನೆ
ಆಡು-ಗಳು
ಆಡು-ಭಾ-ಷೆಯ
ಆಡು-ವು-ದರ
ಆಣತಿ
ಆಣ-ತಿ-ಗ-ಳಿಗೆ
ಆಣ-ತಿಯ
ಆತ
ಆತಂಕ
ಆತಂ-ಕ-ಕಾ-ತರ
ಆತಂ-ಕ-ಅದೇ
ಆತಂ-ಕಕ್ಕೆ
ಆತಂ-ಕ-ಗಳನ್ನು
ಆತಂ-ಕ-ಗೊಂ-ಡರು
ಆತಂ-ಕ-ವನ್ನು
ಆತಂ-ಕ-ವ-ಲ್ಲವೆ
ಆತನ
ಆತ-ನನ್ನು
ಆತ-ನಿಗೆ
ಆತನೇ
ಆತ-ನೊಬ್ಬ
ಆತಿಥೇ
ಆತಿ-ಥೇ-ಯ-ನಾ-ಗಿ-ದ್ದ-ವನು
ಆತಿ-ಥೇ-ಯ-ನಾದ
ಆತಿ-ಥೇ-ಯ-ರಾ-ಗ-ಲಿದ್ದ
ಆತಿ-ಥೇ-ಯ-ರಾ-ಗ-ಲಿ-ದ್ದ-ವರು
ಆತಿ-ಥೇ-ಯ-ರಾ-ಗಿದ್ದ
ಆತಿ-ಥೇ-ಯ-ರಾದ
ಆತಿ-ಥೇ-ಯರೂ
ಆತಿ-ಥೇ-ಯ-ಳಾ-ಗ-ಲಿ-ದ್ದಳು
ಆತಿ-ಥೇ-ಯ-ಳಾ-ಗಿದ್ದ
ಆತಿಥ್ಯ
ಆತಿ-ಥ್ಯ-ದಿಂ-ದಾಗಿ
ಆತಿ-ಥ್ಯ-ವನ್ನು
ಆತುರ
ಆತು-ರ-ಗೊಂಡು
ಆತು-ರ-ದ-ಲ್ಲಿದ್ದ
ಆತ್ಮ
ಆತ್ಮ-ಜೀ-ವ-ಗಳ
ಆತ್ಮಕ್ಕೂ
ಆತ್ಮಕ್ಕೆ
ಆತ್ಮ-ಗೌ-ರವ
ಆತ್ಮ-ಗೌ-ರ-ವಕ್ಕೆ
ಆತ್ಮ-ಗೌ-ರ-ವದ
ಆತ್ಮ-ಗೌ-ರ-ವ-ದಿಂದ
ಆತ್ಮ-ಗೌ-ರ-ವ-ವನ್ನು
ಆತ್ಮ-ಘಾ-ತ-ಕ-ವಾ-ಗ-ಬ-ಹು-ದಾ-ದರೂ
ಆತ್ಮ-ಘಾ-ತ-ಕ-ವಾದ
ಆತ್ಮ-ಚ-ರಿತ್ರೆ
ಆತ್ಮ-ಚ-ರಿ-ತ್ರೆ-ಯಲ್ಲಿ
ಆತ್ಮ-ಜಾ-ಗೃ-ತಿ-ಯ-ನ್ನುಂ-ಟು-ಮಾ-ಡಲು
ಆತ್ಮ-ಜ್ಞಾನ
ಆತ್ಮ-ಜ್ಞಾ-ನ-ಕ್ಕಾಗಿ
ಆತ್ಮ-ತ-ತ್ತ್ವ-ವನ್ನು
ಆತ್ಮದ
ಆತ್ಮ-ದಂ-ತೆಯೇ
ಆತ್ಮ-ದಲ್ಲಿ
ಆತ್ಮ-ದ-ಲ್ಲಿಯೂ
ಆತ್ಮ-ದೊಂ-ದಿಗೆ
ಆತ್ಮ-ದೊ-ಳ-ಗೆಯೇ
ಆತ್ಮ-ನನ್ನು
ಆತ್ಮ-ನಾದ
ಆತ್ಮ-ನಿ-ಗ್ರಹ
ಆತ್ಮ-ನೆಂ-ದ-ರಿ-ತರೆ
ಆತ್ಮ-ನೆಂ-ದ-ರಿ-ಯು-ವುದು
ಆತ್ಮ-ನೆಂದು
ಆತ್ಮನೇ
ಆತ್ಮನೋ
ಆತ್ಮ-ಬ-ಲ-ಗಳಿಂದ
ಆತ್ಮ-ಭಾ-ವ-ನೆ-ಯಿಂದ
ಆತ್ಮ-ರ-ಕ್ಷಣೆ
ಆತ್ಮ-ರಾಜ್ಯ
ಆತ್ಮ-ರಾ-ಜ್ಯ-ವನ್ನು
ಆತ್ಮ-ವ-ಡ-ಗಿ-ದೆಯೋ
ಆತ್ಮ-ವನ್ನು
ಆತ್ಮ-ವನ್ನೇ
ಆತ್ಮ-ವಾಗಿ
ಆತ್ಮ-ವಿ-ದ್ಯೆ-ಯನ್ನು
ಆತ್ಮ-ವಿ-ಶ್ವಾಸ
ಆತ್ಮ-ವಿ-ಶ್ವಾ-ಸ-ಗಳ
ಆತ್ಮ-ವಿ-ಶ್ವಾ-ಸ-ದಿಂ-ದಾಗಿ
ಆತ್ಮ-ವಿ-ಶ್ವಾ-ಸ-ಭ-ರಿತ
ಆತ್ಮ-ವಿ-ಶ್ವಾ-ಸ-ವನ್ನು
ಆತ್ಮ-ವಿ-ಶ್ವಾ-ಸ-ವಿಡಿ
ಆತ್ಮ-ವಿ-ಶ್ವಾ-ಸ-ವಿತ್ತು
ಆತ್ಮ-ವಿ-ಶ್ವಾ-ಸ-ವುಂ-ಟಾ-ಗಲಿ
ಆತ್ಮ-ವಿ-ಶ್ವಾ-ಸ-ವುಂ-ಟು-ಮಾ-ಡಲು
ಆತ್ಮ-ವಿ-ಶ್ವಾ-ಸವೇ
ಆತ್ಮವು
ಆತ್ಮವೂ
ಆತ್ಮ-ವೆಂಬ
ಆತ್ಮವೇ
ಆತ್ಮ-ವೇನೂ
ಆತ್ಮ-ಶಕ್ತಿ
ಆತ್ಮ-ಶ-ಕ್ತಿಯ
ಆತ್ಮ-ಶ-ಕ್ತಿ-ಯಾ-ಗಲಿ
ಆತ್ಮ-ಶ-ಕ್ತಿ-ಸಂ-ಪ-ನ್ನ-ನಾದ
ಆತ್ಮ-ಶ್ರ-ದ್ಧೆ-ಯನ್ನು
ಆತ್ಮ-ಶ್ರ-ದ್ಧೆ-ಯು-ಳ್ಳವ
ಆತ್ಮ-ಸಂ-ಯ-ಮ-ವು-ಳ್ಳವ
ಆತ್ಮ-ಸಂ-ರ-ಕ್ಷ-ಣೆಗೆ
ಆತ್ಮ-ಸ-ಮ-ರ್ಪಣ
ಆತ್ಮ-ಸಾ-ಕ್ಷಾ-ತ್ಕಾರ
ಆತ್ಮ-ಸಾ-ಕ್ಷಾ-ತ್ಕಾ-ರ-ಕ್ಕಾಗಿ
ಆತ್ಮ-ಸಾ-ಕ್ಷಾ-ತ್ಕಾ-ರಕ್ಕೆ
ಆತ್ಮ-ಸಾ-ಕ್ಷಾ-ತ್ಕಾ-ರ-ದೆ-ಡೆಗೆ
ಆತ್ಮ-ಸಾ-ಕ್ಷಾ-ತ್ಕಾ-ರ-ವನ್ನು
ಆತ್ಮ-ಸಾ-ಕ್ಷಾ-ತ್ಕಾ-ರ-ವೆಂ-ದ-ರೇನು
ಆತ್ಮ-ಸಾ-ಕ್ಷಾ-ತ್ಕಾ-ರ-ವೆಂಬ
ಆತ್ಮ-ಸ್ವ-ರೂ-ಪರು
ಆತ್ಮ-ಸ್ವ-ರೂ-ಪಿ-ಯೆಂದು
ಆತ್ಮ-ಹತ್ಯೆ
ಆತ್ಮಾ
ಆತ್ಮಾ-ನಂದ
ಆತ್ಮಾನು
ಆತ್ಮಾ-ಭಿ-ಮಾನ
ಆತ್ಮಾ-ಭಿ-ಮುಖಿ
ಆತ್ಮಾ-ರ್ಪಣೆ
ಆತ್ಮಾ-ರ್ಪ-ಣೆ-ಯನ್ನೇ
ಆತ್ಮೀಯ
ಆತ್ಮೀ-ಯತೆ
ಆತ್ಮೀ-ಯ-ತೆಯ
ಆತ್ಮೀ-ಯ-ತೆ-ಯಿಂದ
ಆತ್ಮೀ-ಯ-ತೆ-ಯಿಂ-ದಿದ್ದು
ಆತ್ಮೀ-ಯ-ರಿಗೆ
ಆತ್ಮೀ-ಯರು
ಆತ್ಮೀ-ಯ-ವಾಗಿ
ಆತ್ಯಾ-ಧು-ನಿಕ
ಆದ
ಆದಂ-ತಹ
ಆದಂತೆ
ಆದಂ-ತೆಯೇ
ಆದ-ಕಾ-ರಣ
ಆದ-ದ್ದಾ-ಗಿ-ರ-ಲಿಲ್ಲ
ಆದದ್ದು
ಆದರ
ಆದ-ರ-ಅ-ಭಿ-ಮಾ-ನ
ಆದ-ರ-ಕ್ಕಾಗಿ
ಆದ-ರಕ್ಕೆ
ಆದ-ರ-ಗ-ಳಿಗೆ
ಆದ-ರದ
ಆದ-ರ-ದಿಂದ
ಆದ-ರದು
ಆದ-ರ-ವಿಲ್ಲ
ಆದ-ರಾ-ಭಿ-ಮಾನ
ಆದ-ರಿಂದ
ಆದ-ರಿದು
ಆದ-ರಿಸಿ
ಆದ-ರಿ-ಸಿ-ದರು
ಆದ-ರಿ-ಸಿ-ದಳು
ಆದರು
ಆದ-ರು-ತ-ತ್ಕಾ-ಲಕ್ಕೆ
ಆದರೂ
ಆದರೆ
ಆದ-ರೇ-ನಂತೆ
ಆದ-ರೇ-ನೀಗ
ಆದ-ರೇನು
ಆದ-ರೇನೂ
ಆದರ್ಶ
ಆದ-ರ್ಶ-ಇ-ವು-ಗ-ಳೆಲ್ಲ
ಆದ-ರ್ಶ-ಕ್ಕಾಗಿ
ಆದ-ರ್ಶ-ಕ್ಕಿಂತ
ಆದ-ರ್ಶಕ್ಕೂ
ಆದ-ರ್ಶಕ್ಕೆ
ಆದ-ರ್ಶ-ಗಳ
ಆದ-ರ್ಶ-ಗ-ಳ-ನ್ನಲ್ಲ
ಆದ-ರ್ಶ-ಗಳನ್ನು
ಆದ-ರ್ಶ-ಗ-ಳ-ನ್ನು
ಆದ-ರ್ಶ-ಗಳನ್ನೂ
ಆದ-ರ್ಶ-ಗಳಿಂದ
ಆದ-ರ್ಶ-ಗ-ಳಿಗೂ
ಆದ-ರ್ಶ-ಗ-ಳಿಗೆ
ಆದ-ರ್ಶ-ಗಳು
ಆದ-ರ್ಶ-ಗಳೂ
ಆದ-ರ್ಶದ
ಆದ-ರ್ಶ-ದಿಂದ
ಆದ-ರ್ಶ-ದೊಂ-ದಿಗೆ
ಆದ-ರ್ಶ-ನಾರಿ
ಆದ-ರ್ಶ-ಪ್ರಾ-ಯ-ವಾಗಿ
ಆದ-ರ್ಶ-ವ-ನ್ನಾಗಿ
ಆದ-ರ್ಶ-ವ-ನ್ನಿ-ಟ್ಟು-ಕೊ-ಳ್ಳು-ವುದು
ಆದ-ರ್ಶ-ವನ್ನು
ಆದ-ರ್ಶ-ವಾಗಿ
ಆದ-ರ್ಶ-ವಾ-ದಿ-ಗಳು
ಆದ-ರ್ಶ-ವಿಲ್ಲ
ಆದ-ರ್ಶವು
ಆದ-ರ್ಶವೂ
ಆದ-ರ್ಶವೆ
ಆದ-ರ್ಶ-ವೆಂದರೆ
ಆದ-ರ್ಶ-ವೊಂ-ದಿದೆ
ಆದಲ್ಲಿ
ಆದ-ವ-ರಿ-ದ್ದಾರೋ
ಆದಾ-ಯ-ದಲ್ಲಿ
ಆದಾ-ಯ-ವನ್ನು
ಆದಿ-ಕಾ-ಲ-ದಿಂ-ದಲೂ
ಆದಿ-ಗಂ-ಗೆ-ಯಲ್ಲಿ
ಆದಿ-ಯಿಂ-ದಲೇ
ಆದು-ದ-ರಿಂದ
ಆದು-ದ-ರಿಂ-ದಲೇ
ಆದೇಶ
ಆದೇ-ಶ-ಗಳನ್ನು
ಆದೇ-ಶ-ಗ-ಳಿಗೆ
ಆದೇ-ಶ-ದಂತೆ
ಆದೇ-ಶ-ವನ್ನು
ಆದೇ-ಶ-ವಾ-ಗಿತ್ತು
ಆದೇ-ಶಿ-ಸಿದ
ಆದೇ-ಶಿ-ಸಿ-ದರು
ಆದ್ದ
ಆದ್ದ-ರಿಂದ
ಆದ್ದ-ರಿಂ-ದಲೆ
ಆದ್ದ-ರಿಂ-ದಲೇ
ಆದ್ಯ
ಆದ್ಯಂ-ತ-ರ-ಹಿತ
ಆದ್ಯಂ-ತ-ರ-ಹಿ-ತ-ವಾದ
ಆದ್ಯಂ-ತ-ರ-ಹಿ-ತ-ವಾ-ದುದು
ಆದ್ಯತೆ
ಆಧ-ರಿಸಿ
ಆಧಾ-ತ್ಮಿಕ
ಆಧಾರ
ಆಧಾ-ರ-ಗಳು
ಆಧಾ-ರ-ಗ್ರಂಥ
ಆಧಾ-ರದ
ಆಧಾ-ರ-ಪು-ರು-ಷ-ರೆಂ-ದರೆ
ಆಧಾ-ರ-ಭೂ-ತ-ವಾದ
ಆಧಾ-ರ-ಭೂ-ತೆ-ಯಾಗಿ
ಆಧಾ-ರ-ರ-ಹಿ-ತ-ವೆಂದು
ಆಧಾ-ರ-ವಾದ
ಆಧಾ-ರವೂ
ಆಧಾ-ರ-ವೇನು
ಆಧಾ-ರ-ವೊ-ದಗಿ
ಆಧಾ-ರ-ಸ್ತಂ-ಭ-ದಂ-ತಿ-ರ-ಬೇಕು
ಆಧಾ-ರಿ-ತ-ವಾದ
ಆಧಿ-ಕ್ಯ-ದಿಂದ
ಆಧಿ-ಕ್ಯ-ದಿಂ-ದಾಗಿ
ಆಧಿ-ಪತ್ಯ
ಆಧು-ನಿಕ
ಆಧು-ನಿ-ಕರ
ಆಧ್ಯ-ಕ್ಷ-ತೆ-ಯನ್ನು
ಆಧ್ಯಾ
ಆಧ್ಯಾ-ತ್ಮ-ಶೀ-ಲ-ಳಾ-ಗಿ-ದ್ದು-ದ-ರಿಂದ
ಆಧ್ಯಾ-ತ್ಮಿಕ
ಆಧ್ಯಾ-ತ್ಮಿ-ಕ-ಜ್ಞಾ-ನ-ದಾನ
ಆಧ್ಯಾ-ತ್ಮಿ-ಕತೆ
ಆಧ್ಯಾ-ತ್ಮಿ-ಕ-ತೆ-ಗಳಿಂದ
ಆಧ್ಯಾ-ತ್ಮಿ-ಕ-ತೆಗೂ
ಆಧ್ಯಾ-ತ್ಮಿ-ಕ-ತೆಯ
ಆಧ್ಯಾ-ತ್ಮಿ-ಕ-ತೆ-ಯತ್ತ
ಆಧ್ಯಾ-ತ್ಮಿ-ಕ-ತೆ-ಯನ್ನು
ಆಧ್ಯಾ-ತ್ಮಿ-ಕ-ತೆ-ಯನ್ನೂ
ಆಧ್ಯಾ-ತ್ಮಿ-ಕ-ತೆ-ಯನ್ನೇ
ಆಧ್ಯಾ-ತ್ಮಿ-ಕ-ತೆ-ಯಿಂದ
ಆಧ್ಯಾ-ತ್ಮಿ-ಕ-ತೆ-ಯಿ-ರು-ವಂ-ತೆಯೇ
ಆಧ್ಯಾ-ತ್ಮಿ-ಕ-ತೆಯು
ಆಧ್ಯಾ-ತ್ಮಿ-ಕ-ತೆ-ಯೆಂ-ಬುದು
ಆಧ್ಯಾ-ತ್ಮಿ-ಕ-ತೆ-ಯೊಂ-ದಿಗೆ
ಆಧ್ಯಾ-ತ್ಮಿ-ಕ-ವಾಗಿ
ಆಧ್ಯಾ-ತ್ಮಿ-ಕ-ವಾದ
ಆಧ್ಯಾ-ತ್ಮಿ-ಕವೂ
ಆಧ್ಯಾ-ತ್ಮಿ-ಕಾ-ನಂ-ದ-ದಿಂದ
ಆಧ್ಯಾ-ತ್ಮಿ-ಕಾ-ನು-ಭ-ವ-ದಿಂದ
ಆಧ್ಯಾ-ಯದ
ಆನಂದ
ಆನಂ-ದ-ಕ-ರ-ವಾ-ಗಿತ್ತು
ಆನಂ-ದ-ಕ-ರ-ವಾದ
ಆನಂ-ದಕ್ಕೆ
ಆನಂ-ದ-ಗಳನ್ನು
ಆನಂ-ದ-ಗ-ಳಿಂ-ದಲೂ
ಆನಂ-ದ-ಗೊಂಡು
ಆನಂ-ದ-ತುಂ-ದಿಲ
ಆನಂ-ದ-ತುಂ-ದಿ-ಲ-ರಾ-ಗಿ-ಬಿ-ಟ್ಟಿ-ದ್ದರು
ಆನಂ-ದದ
ಆನಂ-ದ-ದಲ್ಲಿ
ಆನಂ-ದ-ದಿಂ
ಆನಂ-ದ-ದಿಂದ
ಆನಂ-ದ-ದಿಂ-ದಿದ್ದ
ಆನಂ-ದ-ದಿಂ-ದಿ-ದ್ದರು
ಆನಂ-ದ-ದಿಂ-ದಿದ್ದು
ಆನಂ-ದ-ದಿಂ-ದಿ-ದ್ದೇನೆ
ಆನಂ-ದ-ದಿಂ-ದಿರ
ಆನಂ-ದ-ದಿಂ-ದಿ-ರುವ
ಆನಂ-ದ-ಭ-ರಿ-ತ-ನಾ-ಗಿ-ಬಿ-ಟ್ಟಿ-ದ್ದೇನೆ
ಆನಂ-ದ-ಭ-ರಿ-ತ-ರಾ-ಗಿ-ದ್ದ-ರೆಂ-ದರೆ
ಆನಂ-ದ-ಭ-ರಿ-ತ-ರಾ-ಗಿಯೇ
ಆನಂ-ದ-ಮ-ಯ-ರಾದ
ಆನಂ-ದ-ರಿಂದ
ಆನಂ-ದ-ವನು
ಆನಂ-ದ-ವನ್ನು
ಆನಂ-ದ-ವ-ನ್ನುಂಟು
ಆನಂ-ದ-ವ-ನ್ನುಂ-ಟು-ಮಾ-ಡಿತು
ಆನಂ-ದ-ವನ್ನೂ
ಆನಂ-ದ-ವಾ-ಗಿ-ಬಿ-ಟ್ಟಿದೆ
ಆನಂ-ದ-ವಾ-ಗಿ-ರ-ಬ-ಹು-ದೆಂ-ಬು-ದನ್ನು
ಆನಂ-ದ-ವಾ-ಗು-ತ್ತದೆ
ಆನಂ-ದ-ವಾ-ಯಿತು
ಆನಂ-ದ-ವಿದೆ
ಆನಂ-ದ-ವಿ-ರ-ಬ-ಲ್ಲದು
ಆನಂ-ದ-ವೊಂದು
ಆನಂದಾ
ಆನಂ-ದಾ-ಶ್ಚ-ರ್ಯ-ಗಳಿಂದ
ಆನಂ-ದಾ-ಶ್ಚ-ರ್ಯ-ಗ-ಳಿಗೆ
ಆನಂ-ದಾ-ಶ್ಚ-ರ್ಯ-ಗೊಂ-ಡರು
ಆನಂ-ದಾ-ಶ್ಚ-ರ್ಯ-ವಿ-ಮ-ತ್ತ-ರಾಗಿ
ಆನಂ-ದಿ-ತ-ರಾದ
ಆನಂ-ದಿ-ತ-ರಾ-ದರು
ಆನಂ-ದಿ-ಸ-ಬ-ಲ್ಲ-ವ-ರಿಗೆ
ಆನಂ-ದಿ-ಸ-ಲಾ-ಗು-ವಂತೆ
ಆನಂ-ದಿಸಿ
ಆನಂ-ದಿ-ಸಿ-ದರು
ಆನಂ-ದಿ-ಸಿದೆ
ಆನಂ-ದಿ-ಸಿ-ದ್ದಳೋ
ಆನಂ-ದಿಸು
ಆನಂ-ದಿ-ಸುತ್ತ
ಆನಂ-ದಿ-ಸು-ತ್ತಿ-ದ್ದರು
ಆನಂ-ದಿ-ಸು-ತ್ತೇನೆ
ಆನಂ-ದೋ-ತ್ಸಾಹ
ಆನಂ-ದೋ-ತ್ಸಾ-ಹ-ದಿಂದ
ಆನಂ-ದೋ-ತ್ಸಾ-ಹ-ಭ-ರಿ-ತ-ರಾ-ಗಿ-ದ್ದರು
ಆನ-ರ-ಬಲ್
ಆನೆ
ಆನೆಯ
ಆಪಾ-ದ-ನೆ-ಯನ್ನು
ಆಪ್ತ
ಆಪ್ತ-ಭ-ಕ್ತರ
ಆಪ್ತ-ಭ-ಕ್ತರು
ಆಪ್ತ-ರಲ್ಲಿ
ಆಪ್ತರು
ಆಪ್ತ-ರೊಂ-ದಿಗೆ
ಆಪ್ತ-ರೊ-ಬ್ಬರ
ಆಪ್ತ-ವಾ-ದುದು
ಆಪ್ತ-ಶಿ-ಷ್ಯೆ-ಯ-ರಾದ
ಆಫೀ-ಸನ್ನು
ಆಫ್
ಆಫ್ರಿ-ಕದ
ಆಫ್ರಿ-ಕ-ದಿಂದ
ಆಬಾಲ
ಆಭಾ-ರಿ-ಗ-ಳಾ-ಗಿ-ದ್ದೇವೆ
ಆಭಾ-ರಿ-ಯಾ-ಗಿರ
ಆಮಂ
ಆಮಂ-ತ್ರಣ
ಆಮಂ-ತ್ರ-ಣ-ಗಳ
ಆಮಂ-ತ್ರ-ಣ-ಗಳು
ಆಮಂ-ತ್ರ-ಣ-ವನ್ನು
ಆಮಂ-ತ್ರಿ-ಸು-ತ್ತಿ-ದ್ದಾರೆ
ಆಮ-ಶಂಕೆ
ಆಮಿ-ಷ-ಗಳನ್ನೂ
ಆಮೂ-ಲಾ-ಗ್ರ-ವಾಗಿ
ಆಮೇಲೂ
ಆಮೇಲೆ
ಆಮೋದ
ಆಮ್ಲ
ಆಯ
ಆಯ-ಸ್ಕಾಂ-ತೀಯ
ಆಯಾ
ಆಯಾಮ
ಆಯಾಯ
ಆಯಾಸ
ಆಯಾ-ಸ-ಕ-ರವೂ
ಆಯಾ-ಸ-ಗೊಂ-ಡಿದ್ದ
ಆಯಾ-ಸ-ಗೊಂ-ಡಿ-ದ್ದರೂ
ಆಯಾ-ಸದ
ಆಯಾ-ಸ-ದಿಂದ
ಆಯಾ-ಸ-ವ-ನ್ನುಂ-ಟು-ಮಾ-ಡು-ವು-ದಿಲ್ಲ
ಆಯಾ-ಸ-ವಾ-ಗಿತ್ತು
ಆಯಾ-ಸ-ವಾ-ಗಿ-ದ್ದರೆ
ಆಯಾ-ಸ-ವಾ-ಗಿ-ರು-ತ್ತದೆ
ಆಯಾ-ಸವೂ
ಆಯಾ-ಸ-ವೆಂ-ದ-ರೇ-ನೆಂದು
ಆಯಾ-ಸ್ಥ-ಳ-ಗ-ಳಿಗೆ
ಆಯಿ-ತಲ್ಲ
ಆಯಿತು
ಆಯು-ಷ್ಯ-ವ-ನ್ನೆಲ್ಲ
ಆಯು-ಸ್ಸಾ-ದರೂ
ಆಯು-ಸ್ಸಿ-ನಿಂದ
ಆಯುಸ್ಸು
ಆಯುಸ್ಸೂ
ಆಯ್ಕೆ-ಯನ್ನು
ಆಯ್ದ
ಆಯ್ದು
ಆರಂ-ಭ-ದಲ್ಲಿ
ಆರಂ-ಭ-ವಷ್ಟೆ
ಆರಂ-ಭ-ವಾ-ದದ್ದು
ಆರಂ-ಭ-ವಾ-ಯಿತು
ಆರಂ-ಭಿ-ಸ-ಬ-ಹುದು
ಆರಂ-ಭಿ-ಸಿದ
ಆರಂ-ಭಿ-ಸಿ-ದರು
ಆರತಿ
ಆರ-ತಿಯ
ಆರ-ತಿ-ಯೆತ್ತಿ
ಆರದ
ಆರ-ದಿ-ರಲಿ
ಆರನೇ
ಆರ-ರಂದು
ಆರಾ-ಧ-ನೆಯ
ಆರಾ-ಧ-ನೆಯು
ಆರಾ-ಧಿ-ಸ-ಬೇಕು
ಆರಾ-ಧಿ-ಸಲು
ಆರಾ-ಧಿಸಿ
ಆರಾ-ಧಿ-ಸಿ-ದರೆ
ಆರಾ-ಧಿ-ಸಿ-ದ್ದನ್ನು
ಆರಾ-ಧಿ-ಸು-ತ್ತೀರಿ
ಆರಾ-ಧಿ-ಸುವ
ಆರಾ-ಧಿ-ಸೋಣ
ಆರಾಮ
ಆರಾ-ಮ-ಕು-ರ್ಚಿಯ
ಆರಾ-ಮ-ವಾಗಿ
ಆರಾ-ಮ-ವಾ-ಗಿ-ರಲು
ಆರಿರಿ
ಆರಿಸಿ
ಆರಿ-ಸಿ-ಕೊಂಡ
ಆರಿ-ಸಿ-ಕೊಂ-ಡರು
ಆರಿ-ಸಿ-ಕೊಂ-ಡಿ-ದ್ದೇನೆ
ಆರಿ-ಸಿ-ಕೊ-ಳ್ಳು-ತ್ತೀ-ರಿ-ವಿ-ಗ್ರಹ
ಆರಿ-ಸಿ-ಕೊ-ಳ್ಳು-ವಂತೆ
ಆರಿ-ಸಿ-ದರು
ಆರು
ಆರು-ನೂರು
ಆರೆ
ಆರೈಕೆ
ಆರೈ-ಕೆ-ಯಿಂದ
ಆರೋಗ್ಯ
ಆರೋ-ಗ್ಯ-ಕ-ರ-ವಾದ
ಆರೋ-ಗ್ಯ-ಕ್ಕಿಂತ
ಆರೋ-ಗ್ಯಕ್ಕೆ
ಆರೋ-ಗ್ಯದ
ಆರೋ-ಗ್ಯ-ದಿಂ-ದಲೇ
ಆರೋ-ಗ್ಯ-ದಿಂ-ದಿ-ರ-ಲಿಲ್ಲ
ಆರೋ-ಗ್ಯ-ದೊಂ-ದಿಗೆ
ಆರೋ-ಗ್ಯ-ವಂ-ತ-ನಾ-ಗಿ-ದ್ದೇನೆ
ಆರೋ-ಗ್ಯ-ವಂ-ತ-ರಂತೆ
ಆರೋ-ಗ್ಯ-ವಂ-ತ-ರ-ನ್ನಾಗಿ
ಆರೋ-ಗ್ಯ-ವಂ-ತ-ರಾ-ಗಿ-ದ್ದೇವೆ
ಆರೋ-ಗ್ಯ-ವಂತೂ
ಆರೋ-ಗ್ಯ-ವನ್ನು
ಆರೋ-ಗ್ಯ-ವನ್ನೂ
ಆರೋ-ಗ್ಯ-ವಾ-ಗಿ-ದ್ದೇನೆ
ಆರೋ-ಗ್ಯವು
ಆರೋ-ಗ್ಯವೂ
ಆರೋ-ಗ್ಯ-ಶಾ-ಸ್ತ್ರ-ಇ-ವು-ಗಳ
ಆರೋ-ಗ್ಯ-ಸು-ಧಾ-ರ-ಣೆ-ಗಾಗಿ
ಆರೋ-ಗ್ಯ-ಸ್ಥಿತಿ
ಆರೋ-ಗ್ಯ-ಸ್ಥಿ-ತಿ-ಯಿಂ-ದಾಗಿ
ಆರೋ-ಗ್ಯಾ-ಧಿ-ಕಾ-ರಿಗೆ
ಆರೋಪ
ಆರೋ-ಪ-ಗಳನ್ನು
ಆರೋ-ಪ-ಗಳು
ಆರೋ-ಹಣ
ಆರ್
ಆರ್ಚ್
ಆರ್ಜನೆ
ಆರ್ತ-ನಾದ
ಆರ್ತರ
ಆರ್ಥಿಕ
ಆರ್ಥಿ-ಕ-ಔ-ದ್ಯೋ-ಗಿ-ಕ
ಆರ್ಥಿ-ಕ-ವಾಗಿ
ಆರ್ದ್ರತೆ
ಆರ್ಭ-ಟ-ವನ್ನು
ಆರ್ಭ-ಟಿ-ಸು-ತ್ತಿ-ದ್ದರು
ಆರ್ಯ
ಆರ್ಯ-ಜ-ನಾಂಗ
ಆರ್ಯ-ಧ-ರ್ಮದ
ಆರ್ಯರ
ಆರ್ಯರು
ಆರ್ಯ-ರು
ಆರ್ಯ-ಸಂ-ಸ್ಕೃ-ತಿ-ಯನ್ನೂ
ಆರ್ಯ-ಸ-ಮಾಜ
ಆರ್ಯ-ಸ-ಮಾ-ಜದ
ಆರ್ಯ-ಸ-ಮಾ-ಜ-ದ-ವರ
ಆರ್ಯ-ಸ-ಮಾ-ಜ-ದ-ವ-ರನ್ನು
ಆರ್ಯ-ಸ-ಮಾ-ಜ-ದ-ವ-ರಾ-ಗಿ-ರ-ಬ-ಹುದು
ಆರ್ಯ-ಸ-ಮಾ-ಜ-ದ-ವರು
ಆರ್ಯ-ಸ-ಮಾ-ಜ-ದ-ವರೂ
ಆರ್ಯ-ಸ-ಮಾ-ಜೀ-ಯರು
ಆರ್ಯಾ-ವ-ರ್ತಕ್ಕೆ
ಆರ್ಯಾ-ವ-ರ್ತದ
ಆರ್ಯಾ-ವ-ರ್ತ-ದಲ್ಲಿ
ಆರ್ಯಾ-ವ-ರ್ತ-ದ-ಲ್ಲಿಯೂ
ಆಲಂ-ಗಿ-ಸಿ-ಕೊಂಡು
ಆಲಂ-ಬ-ಜಾ-ರಿನ
ಆಲಂ-ಬ-ಜಾ-ರಿ-ನಿಂದ
ಆಲಂ-ಬ-ಜಾರ್
ಆಲದ
ಆಲ-ಸ್ಯಕ್ಕೆ
ಆಲ-ಸ್ಯದ
ಆಲ-ಸ್ಯ-ದಲ್ಲಿ
ಆಲಿಂ-ಗ-ನ-ಗಳು
ಆಲಿಂ-ಗಿ-ಸಲು
ಆಲಿಂ-ಗಿ-ಸಿ-ಕೊಂ-ಡರು
ಆಲಿ-ಸ-ಬೇ-ಕಾ-ದರೆ
ಆಲಿ-ಸಲು
ಆಲಿ-ಸಿದ
ಆಲಿ-ಸಿ-ದರು
ಆಲಿ-ಸುತ್ತ
ಆಲಿ-ಸು-ತ್ತಿತ್ತು
ಆಲಿ-ಸು-ತ್ತಿ-ದ್ದರು
ಆಲಿ-ಸೋಣ
ಆಲೂ-ಗಡ್ಡೆ
ಆಲೋ
ಆಲೋ-ಚನಾ
ಆಲೋ-ಚ-ನಾ-ಮ-ಗ್ನ-ರಾಗಿ
ಆಲೋ-ಚನೆ
ಆಲೋ-ಚ-ನೆ-ಭಾ-ವ-ನೆ
ಆಲೋ-ಚ-ನೆ-ಭಾ-ವ-ನೆ-ಗಳು
ಆಲೋ-ಚ-ನೆ-ಮಾ-ತು-ಕ-ತೆ-ಗ-ಳಲ್ಲೇ
ಆಲೋ-ಚ-ನೆ-ಗಳನ್ನು
ಆಲೋ-ಚ-ನೆ-ಗಳಲ್ಲಿ
ಆಲೋ-ಚ-ನೆ-ಗಳಿಂದ
ಆಲೋ-ಚ-ನೆ-ಗಳು
ಆಲೋ-ಚ-ನೆ-ಗ-ಳೊಂ-ದಿಗೆ
ಆಲೋ-ಚ-ನೆಯ
ಆಲೋ-ಚ-ನೆ-ಯನ್ನು
ಆಲೋ-ಚ-ನೆ-ಯನ್ನೂ
ಆಲೋ-ಚ-ನೆ-ಯನ್ನೇ
ಆಲೋ-ಚ-ನೆ-ಯಲ್ಲಿ
ಆಲೋ-ಚ-ನೆ-ಯಾ-ಗಿತ್ತು
ಆಲೋ-ಚ-ನೆ-ಯಿಂದ
ಆಲೋ-ಚ-ನೆಯೂ
ಆಲೋ-ಚ-ನೆ-ಯೆಂ-ದರೆ
ಆಲೋ-ಚ-ನೆಯೇ
ಆಲೋ-ಚಿಸ
ಆಲೋ-ಚಿ-ಸ-ದಂ-ತೆಲ್ಲ
ಆಲೋ-ಚಿ-ಸದೆ
ಆಲೋ-ಚಿ-ಸ-ಬ-ಲ್ಲ-ವ-ರಾ-ಗಿ-ದ್ದರು
ಆಲೋ-ಚಿ-ಸಲು
ಆಲೋ-ಚಿ-ಸಲೂ
ಆಲೋ-ಚಿಸಿ
ಆಲೋ-ಚಿ-ಸಿತು
ಆಲೋ-ಚಿ-ಸಿದ
ಆಲೋ-ಚಿ-ಸಿ-ದಓ
ಆಲೋ-ಚಿ-ಸಿ-ದರು
ಆಲೋ-ಚಿ-ಸಿ-ದಾಗ
ಆಲೋ-ಚಿ-ಸಿ-ದ್ದರು
ಆಲೋ-ಚಿ-ಸಿದ್ದೆ
ಆಲೋ-ಚಿ-ಸಿ-ರುವ
ಆಲೋ-ಚಿಸು
ಆಲೋ-ಚಿ-ಸುತ್ತ
ಆಲೋ-ಚಿ-ಸು-ತ್ತ-ದೆ-ಯೆಂದು
ಆಲೋ-ಚಿ-ಸು-ತ್ತಲೇ
ಆಲೋ-ಚಿ-ಸು-ತ್ತಿದ್ದ
ಆಲೋ-ಚಿ-ಸು-ತ್ತಿ-ದ್ದರು
ಆಲೋ-ಚಿ-ಸು-ತ್ತಿ-ದ್ದೇನೆ
ಆಲೋ-ಚಿ-ಸು-ತ್ತಿ-ರು-ತ್ತೇ-ನೆಈ
ಆಲೋ-ಚಿ-ಸು-ತ್ತೇವೆ
ಆಲೋ-ಚಿ-ಸುವ
ಆಲೋ-ಚಿ-ಸು-ವಂ-ತಾ-ದಾಗ
ಆಲೋ-ಚಿ-ಸು-ವಂತೆ
ಆಲೋ-ಚಿ-ಸು-ವು-ದಿಲ್ಲ
ಆಲೋ-ಚಿ-ಸು-ವುದು
ಆಲೋ-ಚಿ-ಸು-ವುದೂ
ಆಲ್ಬರ್ಟಾ
ಆಲ್ಬ-ರ್ಟ್ಸ್
ಆಲ್ಬು-ಮಿನ್ನ
ಆಲ್ಮೋ-ರಕ್ಕೆ
ಆಲ್ಮೋ-ರ-ಗಳಲ್ಲಿ
ಆಲ್ಮೋ-ರದ
ಆಲ್ಮೋ-ರ-ದತ್ತ
ಆಲ್ಮೋ-ರ-ದಲ್ಲಿ
ಆಲ್ಮೋ-ರ-ದ-ಲ್ಲಿದ್ದ
ಆಲ್ಮೋ-ರ-ದ-ಲ್ಲಿ-ದ್ದಾಗ
ಆಲ್ಮೋ-ರ-ದ-ಲ್ಲಿದ್ದು
ಆಲ್ಮೋ-ರ-ದ-ವ-ರೆಗೂ
ಆಲ್ಮೋ-ರ-ದಿಂದ
ಆಲ್ಮೋ-ರ-ದೊಂ-ದಿಗೆ
ಆಲ್ಮೋ-ರ-ವನ್ನು
ಆಳ
ಆಳ-ಆ-ಳಕ್ಕೆ
ಆಳ-ಇದು
ಆಳ-ಕ್ಕಿ-ಳಿದು
ಆಳಕ್ಕೆ
ಆಳದ
ಆಳ-ದ-ಲ್ಲಿಯೇ
ಆಳ-ದಿಂ-ದಲೂ
ಆಳ-ಬಲ್ಲ
ಆಳ-ರ-ಸರ
ಆಳ-ವನ್ನು
ಆಳ-ವಾಗಿ
ಆಳ-ವಾ-ಗಿ-ತ್ತೆಂ-ದರೆ
ಆಳ-ವಾ-ಗಿಯೇ
ಆಳ-ವಾದ
ಆಳ-ವಾ-ದದ್ದು
ಆಳವೂ
ಆಳಿದ
ಆಳು-ಗ-ಳು-ಎ-ಲ್ಲರೂ
ಆಳು-ವ-ವರು
ಆಳ್ವಿ-ಕೆಗೆ
ಆಳ್ವಿ-ಕೆ-ಯಲ್ಲೇ
ಆವ-ರಣ
ಆವ-ರ-ಣದ
ಆವ-ರ-ಣ-ದ-ಲ್ಲಂತೂ
ಆವ-ರ-ಣ-ದಲ್ಲಿ
ಆವ-ರ-ಣ-ದಿಂದ
ಆವ-ರಿ-ಸ-ತೊ-ಡ-ಗಿತು
ಆವ-ರಿ-ಸ-ಬ-ಹುದು
ಆವ-ರಿ-ಸ-ಲಿ-ರು-ವು-ದುಈ
ಆವ-ರಿ-ಸ-ಲಿವೆ
ಆವ-ರಿಸಿ
ಆವ-ರಿ-ಸಿ-ಕೊಂ-ಡರೆ
ಆವ-ರಿ-ಸಿ-ಕೊಂ-ಡಿತು
ಆವ-ರಿ-ಸಿ-ಕೊಂ-ಡಿದ್ದ
ಆವ-ರಿ-ಸಿ-ಕೊಂ-ಡಿ-ರುವ
ಆವ-ರಿ-ಸಿ-ಕೊಂಡು
ಆವ-ರಿ-ಸಿ-ಕೊಂಡೇ
ಆವ-ರಿ-ಸಿ-ಕೊ-ಳ್ಳು-ತ್ತಿತ್ತು
ಆವ-ರಿ-ಸಿ-ಕೊ-ಳ್ಳು-ವಂತೆ
ಆವ-ರಿ-ಸಿ-ದೆಯೆ
ಆವ-ರಿ-ಸಿ-ಬಿ-ಟ್ಟಿತು
ಆವ-ರಿ-ಸಿ-ಬಿ-ಟ್ಟಿತ್ತು
ಆವ-ರಿ-ಸಿ-ವೆ-ಉ-ಪ-ನಿ-ಷ-ತ್ತು-ಗಳ
ಆವ-ರಿ-ಸು-ತ್ತಿದೆ
ಆವ-ರಿ-ಸು-ತ್ತಿ-ರು-ವುದನ್ನು
ಆವರು
ಆವಶ್ಯ
ಆವ-ಶ್ಯಕ
ಆವ-ಶ್ಯ-ಕತೆ
ಆವ-ಶ್ಯ-ಕ-ತೆ-ಗಳನ್ನು
ಆವ-ಶ್ಯ-ಕ-ತೆ-ಗಳನ್ನೂ
ಆವ-ಶ್ಯ-ಕ-ತೆ-ಗ-ಳಿಗೆ
ಆವ-ಶ್ಯ-ಕ-ತೆಗೆ
ಆವ-ಶ್ಯ-ಕ-ತೆಯ
ಆವ-ಶ್ಯ-ಕ-ತೆ-ಯನ್ನು
ಆವ-ಶ್ಯ-ಕ-ತೆ-ಯನ್ನೇ
ಆವ-ಶ್ಯ-ಕ-ತೆ-ಯಿತ್ತು
ಆವ-ಶ್ಯ-ಕ-ತೆಯು
ಆವ-ಶ್ಯ-ಕ-ತೆಯೂ
ಆವ-ಶ್ಯ-ಕ-ತೆ-ಯೆ-ಷ್ಟೆಂಬು
ಆವ-ಶ್ಯ-ಕ-ವಲ್ಲ
ಆವ-ಶ್ಯ-ಕ-ವ-ಲ್ಲವೆ
ಆವ-ಶ್ಯ-ಕ-ವಾ-ಗಿದ್ದ
ಆವ-ಶ್ಯ-ಕ-ವಾದ
ಆವಾ-ಹನೆ
ಆವಿ-ಯಾ-ಗಿ-ಸು-ತ್ತವೆ
ಆವಿರ್
ಆವಿ-ರ್ಭ-ವಿ-ಸಿ-ರು-ವಂತೆ
ಆವಿ-ರ್ಭ-ವಿ-ಸಿ-ರು-ವುದನ್ನು
ಆವಿ-ರ್ಭ-ವಿ-ಸು-ವು-ದಿ-ಲ್ಲವೋ
ಆವಿ-ರ್ಭಾ-ವ-ವಿ-ಕಾಸ
ಆವಿ-ರ್ಭಾ-ವ-ಗೊಂ-ಡಂ-ತಿತ್ತು
ಆವಿ-ರ್ಭಾ-ವ-ವಾ-ದಂ-ತಿತ್ತು
ಆವಿ-ಷ್ಕ-ರಿ-ಸು-ವುದು
ಆವಿ-ಷ್ಕಾರ
ಆವೃತ
ಆವೃ-ತ-ರಾ-ಗಿದ್ದ
ಆವೃ-ತ-ರಾ-ಗಿ-ಬಿ-ಟ್ಟಿ-ದ್ದರು
ಆವೃ-ತ-ರಾದ
ಆವೃ-ತ-ವಾಗಿ
ಆವೃ-ತ-ವಾ-ಗು-ವ-ವ-ರೆಗೆ
ಆವೃ-ತ್ತಿ-ಯನ್ನು
ಆವೃ-ತ್ತಿ-ಯಾಗಿ
ಆವೆ-ಮ-ಣ್ಣಿನ
ಆವೇ-ಗ-ಕ್ಕೊ-ಳ-ಗಾ-ದರು
ಆವೇ-ಗ-ವನ್ನು
ಆವೇ-ಶ-ದಿಂದ
ಆವೇ-ಶ-ಭ-ರಿತ
ಆವೇ-ಶ-ಭ-ರಿ-ತ-ರಾಗಿ
ಆವೇ-ಶ-ವನ್ನು
ಆಶಯ
ಆಶ-ಯ-ದಲ್ಲಿ
ಆಶ-ಯ-ವನ್ನೇ
ಆಶ-ಯ-ವಾ-ಗಿತ್ತು
ಆಶಾ-ಕಿ-ರ-ಣ-ವೆಂದರೆ
ಆಶಾ-ದಾ-ಯ-ಕವಾ
ಆಶಾ-ದಾ-ಯ-ಕ-ವಾ-ಗಿ-ದೆಯೇ
ಆಶಾ-ವಾ-ದಿ-ಗಳು
ಆಶಿಷ್ಠ
ಆಶಿ-ಷ್ಠ-ದೃ-ಢಿ-ಷ್ಠ-ಬ-ಲಿ-ಷ್ಠ-ರಾದ
ಆಶಿ-ಸ-ಬ-ಹುದು
ಆಶಿಸಿ
ಆಶಿ-ಸಿ-ದ್ದರು
ಆಶಿ-ಸಿದ್ದೆ
ಆಶಿ-ಸಿ-ದ್ದೇನೆ
ಆಶಿ-ಸು-ತ್ತಿ-ದ್ದರು
ಆಶಿ-ಸು-ತ್ತೇನೆ
ಆಶಿ-ಸು-ತ್ತೇವೆ
ಆಶೀ-ರ್ವ-ಚನ
ಆಶೀ-ರ್ವ-ದಿ-ಸದೆ
ಆಶೀ-ರ್ವ-ದಿ-ಸ-ಬೇಕು
ಆಶೀ-ರ್ವ-ದಿಸಿ
ಆಶೀ-ರ್ವ-ದಿ-ಸಿ-ದರು
ಆಶೀ-ರ್ವ-ದಿ-ಸಿ-ದರೆ
ಆಶೀ-ರ್ವ-ದಿ-ಸುತ್ತ
ಆಶೀ-ರ್ವ-ದಿ-ಸು-ತ್ತೀ-ರ-ಲ್ಲವೆ
ಆಶೀ-ರ್ವಾದ
ಆಶೀ-ರ್ವಾ-ದ-ಗಳನ್ನು
ಆಶೀ-ರ್ವಾ-ದ-ಗಳು
ಆಶೀ-ರ್ವಾ-ದ-ಗಳೇ
ಆಶೀ-ರ್ವಾ-ದದ
ಆಶೀ-ರ್ವಾ-ದ-ದಿಂದ
ಆಶೀ-ರ್ವಾ-ದ-ವನ್ನು
ಆಶೀ-ರ್ವಾ-ದ-ವಿದೆ
ಆಶೀ-ವಾ-ರ್ದ-ಗಳು
ಆಶು-ಭಾ-ಷ-ಣ-ಗಳನ್ನು
ಆಶೋ-ತ್ತ-ರ-ಗಳನ್ನು
ಆಶ್ಚರ್ಯ
ಆಶ್ಚ-ರ್ಯ-ಕರ
ಆಶ್ಚ-ರ್ಯ-ಕ-ರ-ವಾಗಿ
ಆಶ್ಚ-ರ್ಯ-ಕ-ರ-ವಾದ
ಆಶ್ಚ-ರ್ಯಕ್ಕೆ
ಆಶ್ಚ-ರ್ಯ-ಚ-ಕಿ-ತ-ನಾದೆ
ಆಶ್ಚ-ರ್ಯ-ಚ-ಕಿ-ತ-ರನ್ನಾ
ಆಶ್ಚ-ರ್ಯ-ಚ-ಕಿ-ತ-ರ-ನ್ನಾ-ಗಿ-ಸು-ತ್ತಿ-ದ್ದರು
ಆಶ್ಚ-ರ್ಯ-ಚ-ಕಿ-ತ-ರಾ-ಗಿ-ದ್ದರು
ಆಶ್ಚ-ರ್ಯ-ಚ-ಕಿ-ತ-ರಾ-ದರು
ಆಶ್ಚ-ರ್ಯದ
ಆಶ್ಚ-ರ್ಯ-ದಿಂದ
ಆಶ್ಚ-ರ್ಯ-ಪ-ಡು-ತ್ತಿ-ದ್ದರು
ಆಶ್ಚ-ರ್ಯ-ವ-ನ್ನಲ್ಲ
ಆಶ್ಚ-ರ್ಯ-ವ-ನ್ನುಂ-ಟು-ಮಾ-ಡುವ
ಆಶ್ಚ-ರ್ಯ-ವ-ಲ್ಲವೆ
ಆಶ್ಚ-ರ್ಯ-ವಾ-ಗಿ-ರ-ಬೇ-ಕಾ-ದರೆ
ಆಶ್ಚ-ರ್ಯ-ವಾ-ಗು-ತ್ತದೆ
ಆಶ್ಚ-ರ್ಯ-ವಾ-ಗು-ತ್ತಿ-ರ-ಲಿಲ್ಲ
ಆಶ್ಚ-ರ್ಯ-ವಾ-ದದ್ದು
ಆಶ್ಚ-ರ್ಯ-ವಿಲ್ಲ
ಆಶ್ಚ-ರ್ಯ-ವೇ-ನಿದೆ
ಆಶ್ಚ-ರ್ಯ-ವೇನು
ಆಶ್ಚ-ರ್ಯವೋ
ಆಶ್ಚ-ರ್ಯಾ-ನಂದ
ಆಶ್ರಮ
ಆಶ್ರ-ಮ-ಗಳ
ಆಶ್ರ-ಮ-ಗಳನ್ನು
ಆಶ್ರ-ಮ-ಗಳು
ಆಶ್ರ-ಮದ
ಆಶ್ರ-ಮ-ದಂತೆ
ಆಶ್ರ-ಮ-ದಲ್ಲೇ
ಆಶ್ರ-ಮ-ದಿಂದ
ಆಶ್ರ-ಮ-ದೊ-ಳಗೆ
ಆಶ್ರ-ಮ-ವನ್ನು
ಆಶ್ರ-ಮ-ವಾಸಿ
ಆಶ್ರ-ಮ-ವಾ-ಸಿಗ
ಆಶ್ರ-ಮ-ವಾ-ಸಿ-ಗಳ
ಆಶ್ರ-ಮ-ವಾ-ಸಿ-ಗಳನ್ನೆಲ್ಲ
ಆಶ್ರ-ಮ-ವಾ-ಸಿ-ಗಳಲ್ಲಿ
ಆಶ್ರ-ಮ-ವಾ-ಸಿ-ಗ-ಳಿ-ಗಿ-ದ್ದಂ-ತೆಯೇ
ಆಶ್ರ-ಮ-ವಾ-ಸಿ-ಗ-ಳಿಗೆ
ಆಶ್ರ-ಮ-ವಾ-ಸಿ-ಗ-ಳಿ-ಗೆಲ್ಲ
ಆಶ್ರ-ಮ-ವಾ-ಸಿ-ಗಳು
ಆಶ್ರ-ಮ-ವಾ-ಸಿ-ಗ-ಳೆಲ್ಲ
ಆಶ್ರ-ಮ-ವಾ-ಸಿ-ಗ-ಳೆ-ಲ್ಲರ
ಆಶ್ರ-ಮ-ವಾ-ಸಿ-ಗ-ಳೆ-ಲ್ಲ-ರಿಗೂ
ಆಶ್ರ-ಮ-ವಾ-ಸಿ-ಗ-ಳೆ-ಲ್ಲ-ರೊಂ-ದಿಗೆ
ಆಶ್ರ-ಮ-ವಾ-ಸಿ-ಗ-ಳೊಂ-ದಿಗೆ
ಆಶ್ರ-ಮವು
ಆಶ್ರ-ಮ-ವೊಂ-ದನ್ನು
ಆಶ್ರಯ
ಆಶ್ರ-ಯ-ದಲ್ಲಿ
ಆಶ್ರ-ಯ-ವನ್ನು
ಆಶ್ರ-ಯ-ವೊ-ದ-ಗಿ-ಸು-ವುದು
ಆಶ್ವಾ-ಸನೆ
ಆಶ್ವಾ-ಸ-ನೆಯ
ಆಶ್ವಾ-ಸ-ನೆ-ಯನ್ನು
ಆಶ್ವಾ-ಸ-ನೆ-ಯನ್ನೂ
ಆಶ್ವಾ-ಸ-ನೆ-ಯಿ-ತ್ತರು
ಆಷಾ-ಢ-ಭೂ-ತಿ-ತ-ನದ
ಆಷಾ-ಢ-ಭೂ-ತಿ-ಯ-ನ್ನಾ-ಗಿ-ಸು-ತ್ತದೆ
ಆಸಕ್ತ
ಆಸ-ಕ್ತ-ರಾಗಿ
ಆಸ-ಕ್ತ-ರಾ-ಗಿದ್ದ
ಆಸ-ಕ್ತ-ರಾ-ಗಿ-ದ್ದಾರೆ
ಆಸಕ್ತಿ
ಆಸ-ಕ್ತಿ-ಕರ
ಆಸ-ಕ್ತಿ-ಕ-ರ-ವ-ಲ್ಲ-ದಿ-ರಲು
ಆಸ-ಕ್ತಿ-ಯನ್ನು
ಆಸ-ಕ್ತಿ-ಯಿಂದ
ಆಸ-ಕ್ತಿ-ಯಿ-ದ್ದ-ವ-ರಿಗೆ
ಆಸ-ಕ್ತಿ-ಯಿ-ರ-ಲಿಲ್ಲ
ಆಸ-ಕ್ತಿ-ಯುಂ-ಟಾ-ಯಿತು
ಆಸ-ಕ್ತಿ-ಯುಳ್ಳ
ಆಸ-ನದ
ಆಸ-ನ-ದಲ್ಲಿ
ಆಸ-ನ-ದಿಂದ
ಆಸ-ನ-ವ-ನ್ನ-ಲಂ-ಕ-ರಿ-ಸಿ-ದರು
ಆಸ-ನ-ವನ್ನು
ಆಸರೆ
ಆಸ-ರೆ-ಯಾಗಿ
ಆಸೀ-ನ-ರಾ-ಗಿದ್ದ
ಆಸೀ-ನ-ರಾ-ಗು-ತ್ತಿ-ದ್ದಂ-ತೆಯೇ
ಆಸು-ಪಾ-ಸಿನ
ಆಸು-ಪಾ-ಸಿ-ನಲ್ಲಿ
ಆಸೆ
ಆಸೆ-ಗಳಿಂದ
ಆಸೆಗೆ
ಆಸೆ-ಪ-ಟ್ಟರು
ಆಸೆ-ಯನ್ನು
ಆಸೆ-ಯಾಗಿ
ಆಸೆ-ಯಾ-ಗಿದೆ
ಆಸೆ-ಯಿಂದ
ಆಸೆ-ಯಿ-ತ್ತು-ಅದು
ಆಸೆ-ಯಿದೆ
ಆಸೆಯೇ
ಆಸೆ-ಯೇನೋ
ಆಸ್ಟ್ರಿಯಾ
ಆಸ್ತಮಾ
ಆಸ್ತ-ಮಾ-ಗಳೂ
ಆಸ್ತ-ಮಾದ
ಆಸ್ತ-ಮಾ-ದಿಂ-ದಾಗಿ
ಆಸ್ತಿ
ಆಸ್ತಿ-ಪಾಸ್ತಿ
ಆಸ್ತಿಗೆ
ಆಸ್ತಿ-ಪಾಸ್ತಿ
ಆಸ್ತಿ-ಪಾ-ಸ್ತಿ-ಯ-ನ್ನೆಲ್ಲ
ಆಸ್ತಿ-ಯನ್ನು
ಆಸ್ತಿ-ಯ-ನ್ನೆಲ್ಲ
ಆಸ್ತಿ-ಯಾಗಿ
ಆಸ್ತಿ-ಯೆ-ಲ್ಲವೂ
ಆಸ್ಥಾ-ನಿ-ಕರು
ಆಸ್ಥೆ
ಆಸ್ಪತ್ರೆ
ಆಸ್ಪ-ತ್ರೆ-ಗಳನ್ನು
ಆಸ್ಫೋ-ಟ-ದಂತೆ
ಆಸ್ಫೋ-ಟಿ-ಸು-ವುದು
ಆಸ್ವಾ-ದಿ-ಸ-ಬ-ಲ್ಲ-ವರು
ಆಸ್ವಾ-ದಿ-ಸ-ಬೇಕು
ಆಸ್ವಾ-ದಿ-ಸಿ-ದರು
ಆಸ್ವಾ-ದಿ-ಸಿ-ದಳು
ಆಸ್ವಾ-ದಿ-ಸಿ-ದ-ವರು
ಆಸ್ವಾ-ದಿ-ಸುತ್ತ
ಆಸ್ವಾ-ದಿ-ಸು-ತ್ತಿ-ದ್ದ-ಳೆಂ-ಬುದು
ಆಸ್ವಾ-ದಿ-ಸುವ
ಆಹಾ
ಆಹಾರ
ಆಹಾ-ರಕ್ಕೆ
ಆಹಾ-ರದ
ಆಹಾ-ರ-ವನ್ನು
ಆಹಾ-ರ-ವನ್ನೇ
ಆಹಾ-ರ-ವಿತ್ತು
ಆಹಾ-ರವೂ
ಆಹಾ-ರಾದಿ
ಆಹಾ-ರಾ-ದಿ-ಗಳ
ಆಹಾ-ರಾ-ದಿ-ಗಳನ್ನು
ಆಹಾ-ರಾ-ಭ್ಯಾ-ಸ-ಗಳು
ಆಹುತಿ
ಆಹು-ತಿ-ಗಳನ್ನು
ಆಹು-ತಿ-ಯಾ-ಗಿಸಿ
ಆಹು-ತಿ-ಯಾ-ಗಿ-ಸಿ-ಕೊ-ಳ್ಳು-ವು-ದರ
ಆಹು-ತಿ-ಯಾ-ಗು-ತ್ತಿವೆ
ಆಹ್
ಆಹ್ಲಾ-ದ-ಕ-ರ-ವಾಗಿ
ಆಹ್ಲಾ-ದ-ಕ-ರ-ವಾ-ಗಿಯೇ
ಆಹ್ಲಾ-ದ-ಕ-ರ-ವೇ-ನಾ-ಗಿ-ರ-ಲಿಲ್ಲ
ಆಹ್ವಾನ
ಆಹ್ವಾ-ನ-ಗ-ಳಿಗೂ
ಆಹ್ವಾ-ನ-ಗ-ಳಿಗೆ
ಆಹ್ವಾ-ನ-ಗಳು
ಆಹ್ವಾ-ನದ
ಆಹ್ವಾ-ನ-ವನ್ನು
ಆಹ್ವಾ-ನಿ-ತ-ರಾಗಿ
ಆಹ್ವಾ-ನಿ-ತ-ರಿಗೆ
ಆಹ್ವಾ-ನಿ-ತ-ರೊಂ-ದಿಗೆ
ಆಹ್ವಾ-ನಿ-ಸ-ಲಾ-ಗಿದ್ದ
ಆಹ್ವಾ-ನಿ-ಸಲು
ಆಹ್ವಾ-ನಿಸಿ
ಆಹ್ವಾ-ನಿ-ಸಿದ
ಆಹ್ವಾ-ನಿ-ಸಿ-ದರು
ಆಹ್ವಾ-ನಿ-ಸಿ-ದಳು
ಆಹ್ವಾ-ನಿ-ಸಿದ್ದ
ಆಹ್ವಾ-ನಿ-ಸಿ-ದ್ದರು
ಆಹ್ವಾ-ನಿ-ಸಿ-ದ್ದಳು
ಆಹ್ವಾ-ನಿ-ಸಿ-ರ-ಲಿ-ಲ್ಲವೆ
ಆಹ್ವಾ-ನಿ-ಸು-ತ್ತಲೇ
ಆಹ್ವಾ-ನಿ-ಸು-ತ್ತಿ-ದ್ದರು
ಆಹ್ವಾ-ನಿ-ಸುವ
ಆಹ್ವಾ-ನಿ-ಸು-ವುದನ್ನು
ಆಹ್ವಾ-ನಿ-ಸು-ವುದು
ಆ್ಯಂ-ಟೋನ್
ಆ್ಯಂ-ಟೋ-ನ್ನಲ್ಲಿ
ಆ್ಯ-ಡ-ಮ್ಸ್
ಆ್ಯನಿ
ಆ್ಯ-ಲಿಸ್
ಆ್ಯ-ಷ್ಟನ್
ಆ್ಯ-ಸ್ಪಿ-ನಾ-ಲ-ಳಿಗೆ
ಆ್ಯ-ಸ್ಪಿ-ನಾಲ್
ಇ
ಇಂಕ್ವಾಯ
ಇಂಗಿತ
ಇಂಗಿ-ತ-ವ-ನ್ನ-ರಿ-ಯದೆ
ಇಂಗಿ-ತ-ವನ್ನು
ಇಂಗ್ಗಿ-ಷರ
ಇಂಗ್ಲಿ-ಷರ
ಇಂಗ್ಲಿ-ಷ-ರನ್ನು
ಇಂಗ್ಲಿ-ಷ-ರಾ-ಗಿ-ರ-ಬ-ಹುದು
ಇಂಗ್ಲಿ-ಷ-ರಿಗೂ
ಇಂಗ್ಲಿ-ಷ-ರಿ-ಗೆಲ್ಲ
ಇಂಗ್ಲಿ-ಷಿಗೂ
ಇಂಗ್ಲಿ-ಷಿನ
ಇಂಗ್ಲಿ-ಷಿ-ನಲ್ಲಿ
ಇಂಗ್ಲಿ-ಷಿ-ನ-ವರ
ಇಂಗ್ಲಿಷ್
ಇಂಗ್ಲಿ-ಷ್ಕ್ಲ-ಬ್ನಲ್ಲಿ
ಇಂಗ್ಲಿ-ಷ್ಮನ್
ಇಂಗ್ಲೆಂ-ಡನ್ನು
ಇಂಗ್ಲೆಂ-ಡನ್ನೂ
ಇಂಗ್ಲೆಂಡಿ
ಇಂಗ್ಲೆಂ-ಡಿಗೂ
ಇಂಗ್ಲೆಂ-ಡಿಗೆ
ಇಂಗ್ಲೆಂ-ಡಿನ
ಇಂಗ್ಲೆಂ-ಡಿ-ನಲ್ಲಿ
ಇಂಗ್ಲೆಂ-ಡಿ-ನ-ಲ್ಲಿದ್ದ
ಇಂಗ್ಲೆಂ-ಡಿ-ನ-ಲ್ಲಿನ
ಇಂಗ್ಲೆಂ-ಡಿ-ನ-ಲ್ಲಿಯೂ
ಇಂಗ್ಲೆಂ-ಡಿ-ನ-ಲ್ಲಿ-ರ-ಬೇಕು
ಇಂಗ್ಲೆಂ-ಡಿ-ನ-ಲ್ಲಿ-ರು-ವುದು
ಇಂಗ್ಲೆಂ-ಡಿ-ನಿಂದ
ಇಂಗ್ಲೆಂ-ಡಿ-ನೊಂ-ದಿಗೆ
ಇಂಗ್ಲೆಂ-ಡು-ಗ-ಳ-ಲ್ಲಿನ
ಇಂಗ್ಲೆಂ-ಡು-ಗ-ಳಿಗೆ
ಇಂಗ್ಲೆಂಡ್
ಇಂಟು
ಇಂಡಿ-ಯನ್
ಇಂಡಿ-ಯಾ-ದ-ವರು
ಇಂತಹ
ಇಂತ-ಹದೇ
ಇಂತಿಂ-ತಹ
ಇಂತಿತ್ತು
ಇಂತಿದೆ
ಇಂತಿ-ದ್ದುವು
ಇಂತಿಷ್ಟು
ಇಂಥ
ಇಂಥ-ದ-ಕ್ಕಾ-ಗಿಯೆ
ಇಂಥ-ದ-ನ್ನೆಲ್ಲ
ಇಂಥದು
ಇಂಥ-ದೆಲ್ಲ
ಇಂಥದೇ
ಇಂಥ-ದೊಂ-ದನ್ನು
ಇಂಥ-ದೊಂದು
ಇಂಥ-ವ-ರಿಂದ
ಇಂಥ-ವರು
ಇಂಥ-ವು-ಗ-ಳೆಲ್ಲ
ಇಂಥಾ
ಇಂಥಾ-ದ್ದ-ನ್ನೆಲ್ಲ
ಇಂದಲ್ಲ
ಇಂದಿಗೂ
ಇಂದಿನ
ಇಂದಿ-ನ-ವರೂ
ಇಂದಿ-ನ-ವ-ರೆಗೂ
ಇಂದಿ-ನಿಂದ
ಇಂದಿ-ನಿಂ-ದಲೇ
ಇಂದು
ಇಂದೇ
ಇಂದೋರ್
ಇಂದ್ರಿಯ
ಇಂದ್ರಿ-ಯ-ಗಳಲ್ಲಿ
ಇಂದ್ರಿ-ಯ-ಗ-ಳಿಗೂ
ಇಂದ್ರಿ-ಯ-ಗ-ಳಿಗೆ
ಇಂದ್ರಿ-ಯ-ಗ-ಳೆಲ್ಲ
ಇಂದ್ರಿ-ಯ-ಜೀ-ವನ
ಇಂದ್ರಿ-ಯ-ಭೋ-ಗ-ಗ-ಳೆ-ಲ್ಲ-ವನ್ನೂ
ಇಂದ್ರಿ-ಯ-ಸುಖ
ಇಂದ್ರಿ-ಯ-ಸು-ಖ-ವನ್ನು
ಇಂದ್ರಿ-ಯ-ಸು-ಖವೇ
ಇಂಪು-ದನಿ
ಇಕೊ
ಇಕ್ಕ-ಟ್ಟಿನ
ಇಕ್ಕ-ಟ್ಟಿ-ನದು
ಇಕ್ಕೆ-ಲ-ಗಳಲ್ಲಿ
ಇಕ್ಕೆ-ಲ-ಗ-ಳ-ಲ್ಲಿ-ರುವ
ಇಕ್ಕೆ-ಲ-ಗ-ಳಲ್ಲೂ
ಇಗ-ರ್ಜಿಗೂ
ಇಗ-ರ್ಜಿ-ಯಾದ
ಇಗೊ
ಇಚ್ಛಾ
ಇಚ್ಛಾ-ನು-ಸಾರ
ಇಚ್ಛಾ-ಮ-ರ-ಣದ
ಇಚ್ಛಾ-ಮಾ-ತ್ರ-ದಿಂ-ದಲೇ
ಇಚ್ಛಾ-ಶಕ್ತಿ
ಇಚ್ಛಾ-ಶ-ಕ್ತಿ-ಯೋ-ಗ-ಶ-ಕ್ತಿ-ಗಳು
ಇಚ್ಛಾ-ಶ-ಕ್ತಿ-ಯನ್ನು
ಇಚ್ಛಾ-ಶ-ಕ್ತಿಯು
ಇಚ್ಛಿ-ಸಲಿ
ಇಚ್ಛಿಸಿ
ಇಚ್ಛಿ-ಸಿದ
ಇಚ್ಛಿ-ಸಿ-ದರು
ಇಚ್ಛಿ-ಸಿ-ದರೂ
ಇಚ್ಛಿ-ಸಿ-ದರೆ
ಇಚ್ಛಿ-ಸಿ-ದ್ದನ್ನು
ಇಚ್ಛಿ-ಸಿ-ದ್ದ-ರಿಂದ
ಇಚ್ಛಿ-ಸಿ-ದ್ದರು
ಇಚ್ಛಿ-ಸಿ-ದ್ದಾ-ಳೆಂಬ
ಇಚ್ಛಿ-ಸು-ತ್ತಾರೋ
ಇಚ್ಛಿ-ಸುವ
ಇಚ್ಛೆ
ಇಚ್ಛೆಗೆ
ಇಚ್ಛೆಯ
ಇಚ್ಛೆ-ಯಂತೆ
ಇಚ್ಛೆ-ಯಂ-ತೆಯೇ
ಇಚ್ಛೆ-ಯನ್ನು
ಇಚ್ಛೆ-ಯಾ-ಗಿತ್ತು
ಇಚ್ಛೆ-ಯಿಂದ
ಇಚ್ಛೆ-ಯಿತ್ತು
ಇಚ್ಛೆ-ಯಿ-ದ್ದರೆ
ಇಚ್ಛೆ-ಯಿ-ರ-ಲಿಲ್ಲ
ಇಚ್ಛೆ-ಯುಂ-ಟಾ-ಗು-ತ್ತಿತ್ತು
ಇಚ್ಛೆ-ಯುಂ-ಟಾ-ದಾ-ಗ-ಲೆಲ್ಲ
ಇಚ್ಛೆ-ಯು-ಳ್ಳ-ವ-ರಾ-ಗಿ-ರು-ತ್ತಾರೆ
ಇಚ್ಛೆಯೂ
ಇಚ್ಛೆಯೇ
ಇಚ್ಛೆ-ಯೇನೂ
ಇಚ್ಛೆ-ಯೇ-ನೆಂದು
ಇಟಲಿ
ಇಟ-ಲಿಗೆ
ಇಟ-ಲಿಯ
ಇಟ-ಲಿ-ಯನ್ನು
ಇಟ್ಟದ್ದೇ
ಇಟ್ಟಿಗೆ
ಇಟ್ಟಿ-ದ್ದರು
ಇಟ್ಟಿರು
ಇಟ್ಟು
ಇಟ್ಟುಕೊ
ಇಟ್ಟು-ಕೊಂ-ಡಳು
ಇಟ್ಟು-ಕೊಂ-ಡಿದ್ದ
ಇಟ್ಟು-ಕೊಂ-ಡಿ-ದ್ದೇವೆ
ಇಟ್ಟು-ಕೊಂ-ಡಿ-ರು-ತ್ತೇನೆ
ಇಟ್ಟು-ಕೊಂ-ಡಿಲ್ಲ
ಇಟ್ಟು-ಕೊಂಡೇ
ಇಟ್ಟು-ಕೊ-ಅ-ದ-ರಿಂದ
ಇಟ್ಟು-ಕೊ-ಳ್ಳದೆ
ಇಟ್ಟು-ಕೊ-ಳ್ಳ-ಬೇ-ಕಾ-ಗು-ತ್ತದೆ
ಇಟ್ಟು-ಕೊ-ಳ್ಳ-ಬೇ-ಕೆಂದು
ಇಟ್ಟು-ಕೊಳ್ಳಿ
ಇಟ್ಟು-ಕೊ-ಳ್ಳು-ವಂ-ತೆಯೇ
ಇಡ-ಲಾರೆ
ಇಡಿ-ಕಿ-ರಿ-ದಿಹ
ಇಡಿಯ
ಇಡೀ
ಇಣಿಕಿ
ಇಣುಕು
ಇತರ
ಇತ-ರರ
ಇತ-ರ-ರಂ-ತೆಯೇ
ಇತ-ರ-ರ-ನ್ನಾ-ಗಲಿ
ಇತ-ರ-ರನ್ನು
ಇತ-ರ-ರನ್ನೂ
ಇತ-ರ-ರನ್ನೇ
ಇತ-ರ-ರಲ್ಲ
ಇತ-ರ-ರಲ್ಲಿ
ಇತ-ರ-ರಲ್ಲೂ
ಇತ-ರ-ರಾ-ಗ-ಬೇಡಿ
ಇತ-ರರಿ
ಇತ-ರ-ರಿಂದ
ಇತ-ರ-ರಿಂ-ದೆಲ್ಲ
ಇತ-ರ-ರಿ-ಗಾಗಿ
ಇತ-ರ-ರಿ-ಗಾ-ಗಿಯೇ
ಇತ-ರ-ರಿ-ಗಿಂತ
ಇತ-ರ-ರಿಗೂ
ಇತ-ರ-ರಿಗೆ
ಇತ-ರ-ರಿ-ಗೆಲ್ಲ
ಇತ-ರರು
ಇತ-ರರೂ
ಇತ-ರ-ರೆಲ್ಲ
ಇತ-ರ-ರೊಂ-ದಿಗೆ
ಇತ-ರೆ-ಡೆ-ಗಳಲ್ಲಿ
ಇತ-ರೆಲ್ಲ
ಇತಾವೂ
ಇತಿ
ಇತಿ-ಮಿ-ತಿ-ಯೊ-ಳಗೆ
ಇತಿ-ಹಾಸ
ಇತಿ-ಹಾ-ಸ-ಪು-ರಾ-ಣ-ಗಳ
ಇತಿ-ಹಾ-ಸಕ್ಕೆ
ಇತಿ-ಹಾ-ಸಜ್ಞ
ಇತಿ-ಹಾ-ಸದ
ಇತಿ-ಹಾ-ಸ-ದಲ್ಲಿ
ಇತಿ-ಹಾ-ಸ-ದಲ್ಲೆಲ್ಲ
ಇತಿ-ಹಾ-ಸ-ದಲ್ಲೇ
ಇತಿ-ಹಾ-ಸ-ಪ್ರ-ಸಿದ್ಧ
ಇತಿ-ಹಾ-ಸ-ವನ್ನು
ಇತಿ-ಹಾ-ಸ-ವನ್ನೇ
ಇತಿ-ಹಾ-ಸವು
ಇತಿ-ಹಾ-ಸ-ವೆಂದರೆ
ಇತಿ-ಹಾ-ಸವೇ
ಇತ್ತ
ಇತ್ತೀ-ಚಿನ
ಇತ್ತೀ-ಚಿ-ನ-ವರೂ
ಇತ್ತೀ-ಚೆ-ಗಷ್ಟೇ
ಇತ್ತೀ-ಚೆಗೆ
ಇತ್ತು
ಇತ್ತೆಂದೂ
ಇತ್ಯ-ರ್ಥಕ್ಕೆ
ಇತ್ಯ-ರ್ಥ-ವಾ-ಯಿತು
ಇತ್ಯಾದಿ
ಇತ್ಯಾ-ದಿ-ಗಳ
ಇತ್ಯಾ-ದಿ-ಗಳನ್ನೆಲ್ಲ
ಇತ್ಯಾ-ದಿ-ಗ-ಳೆಲ್ಲ
ಇತ್ಯಾ-ದಿ-ಯಾಗಿ
ಇತ್ಯಾ-ದಿ-ಯೆಲ್ಲ
ಇದ
ಇದಂ
ಇದಂತೂ
ಇದ-ಕ್ಕಾಗಿ
ಇದ-ಕ್ಕಾ-ಗಿಯೇ
ಇದ-ಕ್ಕಿಂತ
ಇದ-ಕ್ಕಿಂ-ತಲೂ
ಇದ-ಕ್ಕಿ-ದ್ದಂತೆ
ಇದಕ್ಕೆ
ಇದ-ಕ್ಕೆಲ್ಲ
ಇದ-ಕ್ಕೊಂದು
ಇದನ್ನು
ಇದ-ನ್ನು-ಏ-ನೆಂ-ದರೆ
ಇದನ್ನೂ
ಇದ-ನ್ನೆಲ್ಲ
ಇದನ್ನೇ
ಇದ-ನ್ನೇಕೆ
ಇದ-ನ್ನೊಂದು
ಇದರ
ಇದ-ರಲ್ಲಿ
ಇದ-ರಿಂದ
ಇದ-ರಿಂ-ದಾಗಿ
ಇದ-ರಿಂ-ದಾ-ಗಿಯೇ
ಇದ-ರಿಂ-ದಾದ
ಇದ-ರಿಂ-ದೆಲ್ಲ
ಇದ-ರೊಂ-ದಿಗೆ
ಇದ-ಲ್ಲದೆ
ಇದಾದ
ಇದಾ-ದ-ಮೇಲೆ
ಇದಾ-ದರೂ
ಇದಾ-ವು-ದಕ್ಕೂ
ಇದಾ-ವು-ದನ್ನೂ
ಇದಿನ್ನು
ಇದಿನ್ನೂ
ಇದಿರು
ಇದೀಗ
ಇದು
ಇದು-ಏ-ನೆಂ-ದರೆ
ಇದು-ತ-ನ್ನ-ತ-ನ-ವನ್ನು
ಇದು-ತಾವು
ಇದು-ನಿ-ಮ್ಮಿಂದ
ಇದು-ಮ-ದ್ರಾ-ಸಿಗೆ
ಇದು-ಲೋಕ್
ಇದು-ವ-ರೆಗೂ
ಇದು-ವ-ರೆಗೆ
ಇದು-ವರೆ-ವಿಗೂ
ಇದೂ
ಇದೆ
ಇದೆಂಥ
ಇದೆ-ಅದೇ
ಇದೆ-ಯಲ್ಲ
ಇದೆ-ಯಾ-ದರೂ
ಇದೆಯೇ
ಇದೆಯೋ
ಇದೆಲ್ಲ
ಇದೆ-ಲ್ಲವೂ
ಇದೆ-ಸಭೆ
ಇದೇ
ಇದೇಕೆ
ಇದೇ-ನಯ್ಯ
ಇದೇ-ನಿದು
ಇದೇನು
ಇದೊಂ-ದನ್ನೇ
ಇದೊಂದು
ಇದೊಂದೇ
ಇದೊಳ್ಳೆ
ಇದೋ
ಇದ್ದ
ಇದ್ದಂ-ತಿತ್ತು
ಇದ್ದಂತೆ
ಇದ್ದಂ-ತೆಯೇ
ಇದ್ದ-ಕಿ-ದ್ದಂತೆ
ಇದ್ದಕ್ಕಿ
ಇದ್ದ-ಕ್ಕಿ-ದಂತೆ
ಇದ್ದ-ಕ್ಕಿ-ದ್ದಂತೆ
ಇದ್ದದ್ದು
ಇದ್ದದ್ದೂ
ಇದ್ದ-ರಲ್ಲ
ಇದ್ದ-ರಾ-ದರೂ
ಇದ್ದ-ರಿತ್ತು
ಇದ್ದರು
ಇದ್ದರೂ
ಇದ್ದರೆ
ಇದ್ದ-ರೆಂ-ಬುದು
ಇದ್ದಲ್ಲಿ
ಇದ್ದಳು
ಇದ್ದ-ವ-ರೆಂ-ದರೆ
ಇದ್ದ-ಹಾಗೆ
ಇದ್ದಾ-ಗ-ಲೆಲ್ಲ
ಇದ್ದಾನೆ
ಇದ್ದಾ-ನೆಯೋ
ಇದ್ದಾರೆ
ಇದ್ದಾ-ರೆಯೇ
ಇದ್ದಾಳೆ
ಇದ್ದಿ-ರ-ಬ-ಹು-ದು-ಅದೇ
ಇದ್ದಿ-ರ-ಲಾ-ರದು
ಇದ್ದಿ-ರ-ಲೇ-ಬೇಕು
ಇದ್ದೀಯೆ
ಇದ್ದೀರಿ
ಇದ್ದು
ಇದ್ದು-ಕೊಂಡು
ಇದ್ದು-ಕೊ-ಳ್ಳಲಿ
ಇದ್ದು-ದ-ರಿಂದ
ಇದ್ದು-ಬಿ-ಟ್ಟ-ದ್ದನ್ನು
ಇದ್ದು-ಬಿ-ಟ್ಟರು
ಇದ್ದು-ಬಿ-ಟ್ಟರೆ
ಇದ್ದು-ಬಿ-ಡು-ತ್ತಿದ್ದೆ
ಇದ್ದು-ಬಿ-ಡು-ತ್ತೀರಿ
ಇದ್ದುವು
ಇದ್ದು-ವೆಂದು
ಇದ್ದು-ವೆಂ-ಬುದು
ಇದ್ದೇ
ಇದ್ದೇನೆ
ಇದ್ದೇ-ವೆಂದು
ಇದ್ಯಾ-ರಿದು
ಇನ್ನರ್ಧ
ಇನ್ನಷ್ಟು
ಇನ್ನಾಕೆ
ಇನ್ನಾ-ದರೂ
ಇನ್ನಾರ
ಇನ್ನಾ-ರದ್ದೂ
ಇನ್ನಾ-ರನ್ನು
ಇನ್ನಾರು
ಇನ್ನಾರೇ
ಇನ್ನಾವ
ಇನ್ನಾ-ವು-ದನ್ನೋ
ಇನ್ನಾ-ವು-ದಾ-ದರೂ
ಇನ್ನಾ-ವುದೇ
ಇನ್ನಿತ
ಇನ್ನಿ-ತರ
ಇನ್ನಿ-ತ-ರರ
ಇನ್ನಿ-ತ-ರರು
ಇನ್ನಿ-ತ-ರ-ರೊಂ-ದಿಗೆ
ಇನ್ನಿ-ರ-ಲಾ-ರರು
ಇನ್ನಿ-ಲ್ಲ-ವಾ-ದರು
ಇನ್ನೀಗ
ಇನ್ನು
ಇನ್ನು-ಮೇಲೆ
ಇನ್ನು-ಳಿದ
ಇನ್ನು-ಳಿ-ದ-ವರು
ಇನ್ನೂ
ಇನ್ನೂರು
ಇನ್ನೆಂದೂ
ಇನ್ನೆ-ರಡು
ಇನ್ನೆಲ್ಲ
ಇನ್ನೆಲ್ಲಿ
ಇನ್ನೆಲ್ಲೂ
ಇನ್ನೆಷ್ಟು
ಇನ್ನೆಷ್ಟೋ
ಇನ್ನೇನು
ಇನ್ನೇನೂ
ಇನ್ನೈದು
ಇನ್ನೊಂದು
ಇನ್ನೊಂ-ದೆಡೆ
ಇನ್ನೊಬ್ಬ
ಇನ್ನೊ-ಬ್ಬ-ನನ್ನು
ಇನ್ನೊ-ಬ್ಬ-ನಿ-ದ್ದಿ-ರ-ಲಾರ
ಇನ್ನೊ-ಬ್ಬರ
ಇನ್ನೊ-ಬ್ಬರು
ಇನ್ನೊ-ಬ್ಬ-ಳೆಂ-ದರೆ
ಇನ್ನೊಮ್ಮೆ
ಇನ್ಫರ್ನೋ
ಇನ್ಯಾರ
ಇನ್ಯಾ-ರಿ-ಗೋ-ಸ್ಕ-ರ-ವಾ-ದರೂ
ಇನ್ಸು-ಲಿ-ನ್ನನ್ನು
ಇಪ್ಪ
ಇಪ್ಪ-ತ್ತ-ನಾಲ್ಕು
ಇಪ್ಪ-ತ್ತ-ನೆಯ
ಇಪ್ಪ-ತ್ತಾರು
ಇಪ್ಪತ್ತು
ಇಪ್ಪ-ತ್ತು-ಇ-ಪ್ಪ-ತ್ತೈದು
ಇಪ್ಪ-ತ್ತೆಂಟು
ಇಪ್ಪ-ತ್ತೇಳು
ಇಪ್ಪ-ತ್ತೈದು
ಇಪ್ಪ-ತ್ತೊಂದು
ಇಬ್ಬ-ಗೆಯ
ಇಬ್ಬರ
ಇಬ್ಬ-ರದೂ
ಇಬ್ಬ-ರನ್ನೂ
ಇಬ್ಬ-ರಿಗೂ
ಇಬ್ಬರು
ಇಬ್ಬರೂ
ಇಬ್ಬರೇ
ಇಬ್ಬಿ-ಬ್ಬ-ರಾಗಿ
ಇಮ್ಮ-ಡಿ-ಯಾ-ಯಿತು
ಇರ
ಇರ-ತೊಡ
ಇರ-ತೊ-ಡ-ಗಿ-ದರು
ಇರ-ದಂ-ತಹ
ಇರ-ದಿದ್ದ
ಇರ-ದಿ-ದ್ದು-ದನ್ನು
ಇರ-ಬ-ಯ-ಸದೆ
ಇರ-ಬ-ಲ್ಲರೋ
ಇರ-ಬ-ಲ್ಲಿ-ರೇನು
ಇರ-ಬ-ಹು-ದ-ಲ್ಲವೆ
ಇರ-ಬ-ಹುದು
ಇರ-ಬ-ಹುದೇ
ಇರ-ಬ-ಹು-ದೇನೋ
ಇರ-ಬಾ-ರದು
ಇರ-ಬೇ-ಕ-ಲ್ಲವೆ
ಇರ-ಬೇ-ಕಾಗಿ
ಇರ-ಬೇ-ಕಾ-ಗು-ತ್ತದೆ
ಇರ-ಬೇ-ಕಾದ
ಇರ-ಬೇ-ಕಾ-ದು-ದರ
ಇರ-ಬೇ-ಕಾದ್ದು
ಇರ-ಬೇ-ಕಾ-ಯಿತು
ಇರ-ಬೇಕು
ಇರ-ಬೇ-ಕೆಂದು
ಇರ-ಲಾ-ರಂ-ಭಿ-ಸಿ-ದರು
ಇರ-ಲಾ-ರದು
ಇರಲಿ
ಇರ-ಲಿ-ಚ್ಛಿ-ಸದ
ಇರ-ಲಿಲ್ಲ
ಇರ-ಲಿ-ಲ್ಲ-ವಲ್ಲ
ಇರ-ಲಿ-ಲ್ಲ-ವಾ-ದ್ದ-ರಿಂದ
ಇರ-ಲಿ-ಲ್ಲ-ವೆಂದು
ಇರಲು
ಇರಲೇ
ಇರ-ಲೇ-ಬೇ-ಕ-ಲ್ಲವೆ
ಇರಿದು
ಇರಿ-ದುವು
ಇರಿಸ
ಇರಿ-ಸ-ಲಾ-ಯಿತು
ಇರಿಸಿ
ಇರಿ-ಸಿ-ಕೊಂ-ಡ-ರ-ಲ್ಲದೆ
ಇರಿ-ಸಿ-ಕೊಂ-ಡ-ವರೇ
ಇರಿ-ಸಿ-ಕೊಂಡು
ಇರಿ-ಸಿದ
ಇರಿ-ಸಿದ್ದು
ಇರು
ಇರು-ತ್ತದೆ
ಇರು-ತ್ತ-ದೆಯೋ
ಇರು-ತ್ತಲೇ
ಇರು-ತ್ತವೆ
ಇರು-ತ್ತಾನೆ
ಇರು-ತ್ತಾರೆ
ಇರು-ತ್ತಾ-ರೆಂ-ಬುದೂ
ಇರುತ್ತಿ
ಇರು-ತ್ತಿತ್ತು
ಇರು-ತ್ತಿ-ದ್ದರು
ಇರು-ತ್ತಿ-ದ್ದುವು
ಇರು-ತ್ತಿದ್ದೆ
ಇರು-ತ್ತಿ-ರ-ಲಿಲ್ಲ
ಇರು-ತ್ತೇನೆ
ಇರು-ತ್ತೇವೆ
ಇರು-ಳಿನ
ಇರುವ
ಇರು-ವಂ-ತಾ-ಗಿದೆ
ಇರು-ವಂ-ತಾ-ಗಿ-ಬಿ-ಟ್ಟರೆ
ಇರು-ವಂ-ತಿ-ರ-ಲಿಲ್ಲ
ಇರು-ವಂತೆ
ಇರು-ವನೋ
ಇರು-ವ-ಲ್ಲಿಗೇ
ಇರು-ವ-ವರು
ಇರು-ವಾಗ
ಇರು-ವಾ-ಗಲೇ
ಇರು-ವಿ-ಕೆ-ನಂ-ಬಿಕೆ
ಇರು-ವಿ-ಕೆ-ಯಿಂದ
ಇರುವು
ಇರು-ವುದನ್ನು
ಇರು-ವು-ದರ
ಇರು-ವು-ದಾ-ದರೆ
ಇರು-ವು-ದಿಲ್ಲ
ಇರು-ವು-ದಿ-ಲ್ಲ-ವೆಂದೂ
ಇರು-ವುದು
ಇರು-ವು-ದೆಂ-ದರೆ
ಇರು-ವು-ದೆಲ್ಲ
ಇರು-ವು-ದೆಲ್ಲಿ
ಇರು-ವುದೇ
ಇರು-ವು-ದೊಂದೇ
ಇರು-ವೆ-ಗಿಂ-ತಲೂ
ಇಲ್ಲ
ಇಲ್ಲ-ಇ-ವೆಲ್ಲ
ಇಲ್ಲದ
ಇಲ್ಲ-ದಂ-ತಹ
ಇಲ್ಲ-ದಂ-ತಾ-ಗಿತ್ತು
ಇಲ್ಲ-ದ-ವ-ರಾ-ಗು-ತ್ತಿ-ದ್ದೀರಿ
ಇಲ್ಲ-ದ-ವ-ರಿಗೆ
ಇಲ್ಲ-ದ-ವರು
ಇಲ್ಲ-ದಷ್ಟು
ಇಲ್ಲ-ದಿದ್ದ
ಇಲ್ಲ-ದಿ-ದ್ದ-ರಿಲ್ಲ
ಇಲ್ಲ-ದಿ-ದ್ದರೂ
ಇಲ್ಲ-ದಿ-ದ್ದರೆ
ಇಲ್ಲ-ದಿ-ದ್ದು-ದ-ರಿಂದ
ಇಲ್ಲ-ದಿ-ದ್ದುದು
ಇಲ್ಲ-ದಿ-ರಲಿ
ಇಲ್ಲ-ದಿ-ರು-ವಾಗ
ಇಲ್ಲ-ದಿಲ್ಲ
ಇಲ್ಲದೆ
ಇಲ್ಲ-ದೆ-ಹೋ-ಗಿ-ದ್ದರೆ
ಇಲ್ಲದೇ
ಇಲ್ಲ-ದ್ದ-ರಿಂದ
ಇಲ್ಲ-ವಲ್ಲ
ಇಲ್ಲ-ವಾ-ಗಿದೆ
ಇಲ್ಲ-ವಾ-ದರೆ
ಇಲ್ಲವೆ
ಇಲ್ಲ-ವೆಂ-ದಲ್ಲ
ಇಲ್ಲ-ವೆಂ-ಬುದು
ಇಲ್ಲ-ವೆ-ನ್ನ-ಬ-ಹುದು
ಇಲ್ಲವೇ
ಇಲ್ಲ-ವೇನೋ
ಇಲ್ಲ-ವೈ-ರಾ-ಗ್ಯ-ಮೇ-ವಾ-ಭ-ಯಮ್
ಇಲ್ಲವೊ
ಇಲ್ಲವೋ
ಇಲ್ಲ-ಸ-ಲ್ಲದ
ಇಲ್ಲಿ
ಇಲ್ಲಿಂದ
ಇಲ್ಲಿಂ-ದಲೇ
ಇಲ್ಲಿ-ಇ-ಳಿ-ದು-ಕೊಂಡ
ಇಲ್ಲಿಗೂ
ಇಲ್ಲಿಗೆ
ಇಲ್ಲಿಗೇ
ಇಲ್ಲಿದೆ
ಇಲ್ಲಿ-ದೆ-ಬಾ-ಳಿಗೆ
ಇಲ್ಲಿದ್ದ
ಇಲ್ಲಿ-ದ್ದಾಗ
ಇಲ್ಲಿ-ದ್ದಾರೆ
ಇಲ್ಲಿ-ದ್ದಾಳೆ
ಇಲ್ಲಿ-ದ್ದಿ-ದ್ದರೆ
ಇಲ್ಲಿನ
ಇಲ್ಲಿ-ನಂತೆ
ಇಲ್ಲಿ-ನಷ್ಟು
ಇಲ್ಲಿನ್ನೂ
ಇಲ್ಲಿಯ
ಇಲ್ಲಿ-ಯ-ವ-ರಿಗೆ
ಇಲ್ಲಿ-ಯ-ವ-ರಿನ್ನೂ
ಇಲ್ಲಿ-ಯ-ವರೆ-ಗಿತ್ತು
ಇಲ್ಲಿ-ಯ-ವ-ರೆಗೂ
ಇಲ್ಲಿ-ಯ-ವ-ರೆಗೆ
ಇಲ್ಲಿಯೂ
ಇಲ್ಲಿಯೆ
ಇಲ್ಲಿಯೇ
ಇಲ್ಲಿ-ರಲು
ಇಲ್ಲಿ-ರುವ
ಇಲ್ಲಿಲ್ಲ
ಇಲ್ಲೀಗ
ಇಲ್ಲೂ
ಇಲ್ಲೇ
ಇಲ್ಲೊಂದು
ಇಲ್ಲೊ-ಬ್ಬರು
ಇಳಿ-ಗಾ-ಲ-ದಲ್ಲಿ
ಇಳಿ-ಜಾ-ರಿ-ನಿಂದ
ಇಳಿ-ಜಾರು
ಇಳಿ-ತಾ-ಯ-ವಾ-ಗಿತ್ತು
ಇಳಿದ
ಇಳಿ-ದರು
ಇಳಿ-ದರೋ
ಇಳಿದು
ಇಳಿ-ದು-ಕೊಂಡ
ಇಳಿ-ದು-ಕೊಂ-ಡರು
ಇಳಿ-ದು-ಕೊಂ-ಡಳು
ಇಳಿ-ದು-ಕೊಂ-ಡಿದ್ದ
ಇಳಿ-ದು-ಕೊಂ-ಡಿ-ದ್ದಾಗ
ಇಳಿ-ದು-ಕೊಂಡು
ಇಳಿ-ದು-ಕೊ-ಳ್ಳಲು
ಇಳಿ-ದು-ಬಂ-ದಾರು
ಇಳಿ-ದು-ಹೋ-ಗಿ-ದ್ದರು
ಇಳಿ-ದು-ಹೋ-ಗಿ-ದ್ದು-ದನ್ನು
ಇಳಿ-ದು-ಹೋ-ಗಿ-ದ್ದೇನೆ
ಇಳಿ-ದು-ಹೋ-ಗು-ತ್ತಿ-ದ್ದ-ರೆಂ-ಬು-ದನ್ನು
ಇಳಿ-ದು-ಹೋ-ಯಿತು
ಇಳಿ-ಬಿ-ಟ್ಟಂ-ತಿದೆ
ಇಳಿ-ಮು-ಖ-ವಾ-ಗು-ತ್ತಿತ್ತು
ಇಳಿ-ಮು-ಖ-ವಾ-ಯಿತು
ಇಳಿ-ಯಲು
ಇಳಿ-ಯಿತು
ಇಳಿ-ಯುತ್ತ
ಇಳಿ-ಯು-ತ್ತದೆ
ಇಳಿ-ಯು-ತ್ತಿ-ದ್ದಂ-ತೆಯೇ
ಇಳಿ-ಯುವ
ಇಳಿ-ಯು-ವಂ-ತಿ-ರ-ಲಿಲ್ಲ
ಇಳಿ-ಯು-ವುದೇ
ಇಳಿ-ಸಿ-ಕೊ-ಳ್ಳುವ
ಇಳೆಗೆ
ಇವತ್ತು
ಇವ-ನಿಗೆ
ಇವನು
ಇವ-ನೆ-ಲ್ಲಿಂದ
ಇವ-ನೊಬ್ಬ
ಇವನ್ನು
ಇವ-ನ್ನೆಲ್ಲ
ಇವರ
ಇವ-ರದು
ಇವ-ರನ್ನು
ಇವ-ರನ್ನೂ
ಇವ-ರ-ನ್ನೆಲ್ಲ
ಇವ-ರ-ಲ್ಲದೆ
ಇವ-ರಲ್ಲಿ
ಇವ-ರ-ಲ್ಲಿ-ಬ್ಬರು
ಇವರಿ
ಇವ-ರಿಗೆ
ಇವ-ರಿ-ಗೆಲ್ಲ
ಇವ-ರಿ-ಬ್ಬರ
ಇವ-ರಿ-ಬ್ಬ-ರ-ನ್ನ-ಲ್ಲದೆ
ಇವ-ರಿ-ಬ್ಬ-ರನ್ನೂ
ಇವ-ರಿ-ಬ್ಬ-ರಿಗೂ
ಇವ-ರಿ-ಬ್ಬ-ರಿಗೆ
ಇವ-ರಿ-ಬ್ಬರೂ
ಇವರು
ಇವ-ರು-ಗಳ
ಇವ-ರು-ಗ-ಳ-ಲ್ಲದೆ
ಇವ-ರು-ಗ-ಳಿ-ಗೆಲ್ಲ
ಇವ-ರು-ಗ-ಳೊಂ-ದಿಗೆ
ಇವ-ರೆಡು
ಇವರೆಲ್ಲ
ಇವರೆ-ಲ್ಲರ
ಇವರೆ-ಲ್ಲ-ರನ್ನೂ
ಇವರೆ-ಲ್ಲ-ರಿಗೂ
ಇವರೆ-ಲ್ಲರೂ
ಇವರೇ
ಇವ-ರೊಂ-ದಿಗೆ
ಇವ-ರೊ-ಬ್ಬರು
ಇವ-ರೊಮ್ಮೆ
ಇವ-ರ್ಯಾ-ರನ್ನೂ
ಇವ-ರ್ಯಾ-ರಿಗೂ
ಇವ-ರ್ಯಾರೋ
ಇವ-ಲ್ಲದೆ
ಇವಳ
ಇವ-ಳ-ಲ್ಲದೆ
ಇವಳು
ಇವಳೂ
ಇವಳೇ
ಇವ-ಳೊಬ್ಬ
ಇವಿಷ್ಟು
ಇವು
ಇವು-ಗಳ
ಇವು-ಗಳನ್ನು
ಇವು-ಗಳನ್ನೆಲ್ಲ
ಇವು-ಗ-ಳ-ಲ್ಲದೆ
ಇವು-ಗಳಲ್ಲಿ
ಇವು-ಗ-ಳ-ಲ್ಲೊಂ-ದೆಂ-ದರೆ
ಇವು-ಗಳಿಂದ
ಇವು-ಗ-ಳಿಂ-ದೆಲ್ಲ
ಇವು-ಗ-ಳಿ-ಗಾಗಿ
ಇವು-ಗ-ಳಿ-ಗಿ-ರುವ
ಇವು-ಗ-ಳಿಗೆ
ಇವು-ಗ-ಳಿ-ಗೆಲ್ಲ
ಇವು-ಗಳು
ಇವು-ಗಳೂ
ಇವು-ಗ-ಳೆಲ್ಲ
ಇವು-ಗ-ಳೆ-ಲ್ಲ-ದರ
ಇವು-ಗ-ಳೊಂ-ದಿಗೆ
ಇವೆ
ಇವೆ-ರ-ಡನ್ನೂ
ಇವೆ-ರ-ಡರ
ಇವೆ-ರಡೂ
ಇವೆಲ್ಲ
ಇವೆ-ಲ್ಲ-ಕ್ಕಿಂತ
ಇವೆ-ಲ್ಲಕ್ಕೂ
ಇವೆ-ಲ್ಲ-ದ-ರೊಂ-ದಿಗೆ
ಇವೆ-ಲ್ಲ-ವನ್ನೂ
ಇವೆ-ಲ್ಲ-ವು-ಗಳ
ಇವೆ-ಲ್ಲ-ವು-ಗ-ಳಿ-ಗಿಂತ
ಇವೆ-ಲ್ಲವೂ
ಇವೇ
ಇಷ್ಟ
ಇಷ್ಟಕ್ಕೇ
ಇಷ್ಟ-ದೇ-ವತೆ
ಇಷ್ಟ-ದೇ-ವ-ತೆಯ
ಇಷ್ಟ-ನ-ನ-ಗ-ವಳು
ಇಷ್ಟನ್ನು
ಇಷ್ಟ-ಪಟ್ಟ
ಇಷ್ಟ-ಪ-ಟ್ಟರು
ಇಷ್ಟ-ಪ-ಟ್ಟರೆ
ಇಷ್ಟ-ಪ-ಡದೆ
ಇಷ್ಟ-ಪ-ಡ-ಲಿಲ್ಲ
ಇಷ್ಟ-ಪ-ಡು-ತ್ತಾರೆ
ಇಷ್ಟ-ಪ-ಡು-ತ್ತಿ-ರ-ಲಿಲ್ಲ
ಇಷ್ಟ-ಪ-ಡು-ವಂತೆ
ಇಷ್ಟ-ಮಿತ್ರ
ಇಷ್ಟರ
ಇಷ್ಟ-ರಲ್ಲೇ
ಇಷ್ಟ-ರಿಂ-ದಲೇ
ಇಷ್ಟ-ವಾಗ
ಇಷ್ಟ-ವಾದ
ಇಷ್ಟ-ವಾ-ದರೆ
ಇಷ್ಟ-ವಾ-ಯಿ-ತೆಂದು
ಇಷ್ಟ-ವಿರ
ಇಷ್ಟ-ವಿ-ರು-ವ-ವರ
ಇಷ್ಟ-ವಿ-ಲ್ಲದ
ಇಷ್ಟ-ವಿ-ಲ್ಲ-ದಿ-ದ್ದರೂ
ಇಷ್ಟ-ವಿ-ಲ್ಲ-ದಿ-ರು-ವ-ವರ
ಇಷ್ಟ-ವೆಂಬ
ಇಷ್ಟಾ-ದರೂ
ಇಷ್ಟಾ-ನಿ-ಷ್ಟ-ಗಳನ್ನು
ಇಷ್ಟು
ಇಷ್ಟು-ಹೊ-ತ್ತಿಗೆ
ಇಷ್ಟೆ
ಇಷ್ಟೆಲ್ಲ
ಇಷ್ಟೊಂ-ದ-ನ್ನೆಲ್ಲ
ಇಷ್ಟೊಂದು
ಇಸಾ-ಬೆಲ್
ಇಸ್ತಾಂ-ಬುಲ್
ಇಸ್ಲಾಂ
ಇಹ
ಇಹ-ಜೀ-ವ-ನದ
ಇಹ-ಲೋಕ
ಇಹ-ಲೋ-ಕ-ದಲ್ಲಿ
ಇಹ-ಲೋ-ಕ-ವನ್ನು
ಇಹ-ಲೋ-ಕ-ವನ್ನೇ
ಈ
ಈಕೆ
ಈಕೆಗೆ
ಈಕ್ಷಿ-ಸುತ್ತ
ಈಗ
ಈಗಂತೂ
ಈಗ-ತಾನೆ
ಈಗ-ಲಂತೂ
ಈಗಲಾ
ಈಗ-ಲಾ-ದರೂ
ಈಗಲೂ
ಈಗಲೇ
ಈಗಾ
ಈಗಾ-ಗಲೇ
ಈಗಿದ್ದ
ಈಗಿನ
ಈಗಿ-ನಂ-ತಹ
ಈಗಿ-ರುವ
ಈಗಿ-ರು-ವಂ-ತೆಯೇ
ಈಗಿ-ರು-ವ-ವರೆಲ್ಲ
ಈಗಿ-ರು-ವ-ಷ್ಟ-ಲ್ಲದೆ
ಈಗಿ-ರು-ವು-ದೆಲ್ಲ
ಈಗಿಲ್ಲ
ಈಗಿ-ಲ್ಲ-ವೆಂದು
ಈಗೀಗ
ಈಗೆ-ಲ್ಲಿ-ರು-ತ್ತಿದ್ದೆ
ಈಗೆಲ್ಲೋ
ಈಗೇಕೆ
ಈಗೊಂದು
ಈಗೊಮ್ಮೆ
ಈಚಿನ
ಈಚೆಗೆ
ಈಚೆಯ
ಈಜಾ-ಡಿ-ದರು
ಈಜಾ-ಡು-ತ್ತಿ-ದ್ದರು
ಈಜಿ-ನಲ್ಲಿ
ಈಜಿ-ಪ್ಟಿಗೆ
ಈಜಿ-ಪ್ಟಿನ
ಈಜಿ-ಪ್ಟಿ-ನಲ್ಲಿ
ಈಜಿ-ಪ್ಟಿ-ನೆ-ಡೆಗೆ
ಈಜಿ-ಪ್ಟ್ಗಳನ್ನು
ಈಜಿ-ಪ್ಷಿ-ಯನ್
ಈಡಿಗ
ಈಡಿ-ಗ-ರದ್ದು
ಈಡೇರಿ
ಈಡೇ-ರಿ-ರ-ಲಿಲ್ಲ
ಈಡೇ-ರಿಸಿ
ಈಡೇ-ರಿ-ಸಿ-ಕೊ-ಳ್ಳ-ಬೇ-ಕಾದ
ಈಡೇ-ರಿ-ಸಿ-ದರೋ
ಈಡೇ-ರಿ-ಸುವ
ಈಡೇ-ರು-ತ್ತಿ-ದೆ-ಯೆಂಬ
ಈತ
ಈತನ
ಈತ-ನನ್ನು
ಈತ-ನಲ್ಲಿ
ಈತ-ನಿಗೆ
ಈತನೇ
ಈತ-ನೊಂ-ದಿಗೆ
ಈತ-ನೊಬ್ಬ
ಈವರೆ-ಗಿನ
ಈವ-ರೆಗೂ
ಈವ-ರೆಗೆ
ಈಶ್ವ-ರ-ಕೋ-ಟಿ-ಗ-ಳೆಂದು
ಈಶ್ವ-ರ-ಕೋ-ಟಿ-ಗ-ಳೆಂ-ಬ-ವರ
ಈಶ್ವ-ರ-ಚಂದ್ರ
ಈಶ್ವ-ರತ್ವ
ಈಶ್ವ-ರನ
ಈಶ್ವ-ರ-ನನ್ನು
ಈಸ್ಟರ್
ಉಂಟಾಗ
ಉಂಟಾ-ಗ-ದಂ-ತಿತ್ತು
ಉಂಟಾ-ಗ-ಬೇ-ಕಾ-ದರೆ
ಉಂಟಾ-ಗ-ಲಾ-ರದು
ಉಂಟಾ-ಗ-ಲಿಲ್ಲ
ಉಂಟಾಗಿ
ಉಂಟಾ-ಗಿದೆ
ಉಂಟಾ-ಗಿದ್ದ
ಉಂಟಾ-ಗಿ-ರ-ಬ-ಹು-ದೆಂ-ಬುದು
ಉಂಟಾ-ಗು-ತ್ತದೆ
ಉಂಟಾ-ಗು-ತ್ತ-ದೆಯೋ
ಉಂಟಾ-ಗು-ತ್ತಿತ್ತು
ಉಂಟಾ-ಗು-ತ್ತಿವೆ
ಉಂಟಾ-ಗುವ
ಉಂಟಾ-ಗು-ವು-ದಿ-ಲ್ಲವೋ
ಉಂಟಾದ
ಉಂಟಾ-ದದ್ದು
ಉಂಟಾ-ದಾಗ
ಉಂಟಾದು
ಉಂಟಾ-ಯಿತು
ಉಂಟು
ಉಂಟು-ಮಾ-ಡಿತು
ಉಂಟು-ಮಾ-ಡಿ-ತೆ-ನ್ನ-ಬೇಕು
ಉಂಟು-ಮಾ-ಡಿದ
ಉಂಟು-ಮಾ-ಡಿ-ದುವು
ಉಂಟು-ಮಾ-ಡು-ತ್ತಿ-ದ್ದರು
ಉಂಟು-ಮಾ-ಡುವ
ಉಂಡ
ಉಂಡ-ದ್ದ-ಲ್ಲದೆ
ಉಂಡರೆ
ಉಂಡಿಲ್ಲ
ಉಂಡು
ಉಕ್ಕಿ
ಉಕ್ಕಿನ
ಉಕ್ಕಿ-ನಂ-ತಹ
ಉಕ್ಕಿ-ಬರು
ಉಕ್ಕಿ-ಸುವ
ಉಕ್ಕಿ-ಹ-ರಿದ
ಉಕ್ಕಿ-ಹ-ರಿ-ಯು-ತ್ತಿ-ರುವ
ಉಗ
ಉಗಮ
ಉಗ-ಮದ
ಉಗ-ಮ-ವಾ-ಗ-ಬೇಕು
ಉಗ-ಮ-ಸ್ಥಾನ
ಉಗ-ಮ-ಸ್ಥಾ-ನಕ್ಕೆ
ಉಗ-ಮ-ಸ್ಥಾ-ನ-ವಾದ
ಉಗಿ-ದು-ಬಿಟ್ಟೆ
ಉಗಿ-ದೋ-ಣಿಯ
ಉಗಿ-ಹ-ಡಗು
ಉಗ್ರ
ಉಗ್ರ-ವಾಗಿ
ಉಗ್ರಾ-ಣ-ಎ-ಲ್ಲವೂ
ಉಗ್ರಾ-ಣ-ದಂ-ತಿ-ರು-ತ್ತಾನೆ
ಉಚಿ-ತ-ವಾದ
ಉಚಿ-ತ-ವೆಂದೂ
ಉಚ್ಚ
ಉಚ್ಚ-ಕಂ-ಠ-ದಿಂದ
ಉಚ್ಚ-ತಮ
ಉಚ್ಚ-ರಿಸ
ಉಚ್ಚ-ರಿ-ಸಿದ
ಉಚ್ಚ-ರಿ-ಸುವ
ಉಚ್ಚ-ವ-ರ್ಣ-ದ-ವರ
ಉಚ್ಚ-ಸ್ವ-ರ-ದಲ್ಲಿ
ಉಚ್ಚಾ
ಉಚ್ಚಾ-ರ-ಣೆ-ಯನ್ನು
ಉಚ್ಛಾ-ಟಿ-ಸ-ಬೇಕು
ಉಜ್ಜಲ
ಉಜ್ವಲ
ಉಜ್ವ-ಲತೆ
ಉಜ್ವ-ಲ-ವಾಗಿ
ಉಠೋ
ಉಡಿಗೆ
ಉಡಿ-ಗೆ-ತೊ-ಡಿಗೆ
ಉಡಿ-ಗೆ-ತೊ-ಡಿ-ಗೆ-ಇವು
ಉಡು
ಉಡು-ಗಿ-ಹೋ-ಗಿದೆ
ಉಡುಗೆ
ಉಡು-ಗೆ-ಯತ್ತ
ಉಡು-ಗೆ-ಯನ್ನು
ಉಡು-ಗೆ-ಯನ್ನೇ
ಉಡು-ಗೊರೆ
ಉಡು-ಗೊ-ರೆ-ಗ-ಳೊಂ-ದಿಗೆ
ಉಡು-ಗೊ-ರೆ-ಯನ್ನು
ಉಡು-ಗೊ-ರೆ-ಯಾಗಿ
ಉಡುಪ
ಉಣ-ಬ-ಡಿ-ಸ-ಬೇಕು
ಉಣ-ಬ-ಡಿ-ಸಿ-ದರು
ಉಣ-ಬ-ಡಿ-ಸಿ-ದ-ರೆಂ-ದರೆ
ಉಣ-ಬ-ಡಿ-ಸುವ
ಉಣ-ಬ-ಡಿ-ಸು-ವುದನ್ನು
ಉಣ-ಬ-ಡಿ-ಸು-ವು-ದರ
ಉಣ್ಣುತ್ತ
ಉಣ್ಣು-ವು-ದಾ-ದರೂ
ಉತ್ಕಂ
ಉತ್ಕಂ-ಠಿ-ತ-ನಾ-ಗಿದ್ದ
ಉತ್ಕಂ-ಠಿ-ತ-ರಾ-ಗಿ-ದ್ದರು
ಉತ್ಕಟ
ಉತ್ಕ-ಟಾ-ವಸ್ಥೆ
ಉತ್ಕ-ಟೇಚ್ಛೆ
ಉತ್ಕ-ಟೇ-ಚ್ಛೆ-ಯನ್ನು
ಉತ್ಕ-ರ್ಷ-ಗೊ-ಳಿ-ಸಲು
ಉತ್ಕಾಂ-ಕ್ಷೆ-ಗಳನ್ನು
ಉತ್ಕಾಂ-ಕ್ಷೆಯ
ಉತ್ಕೃಷ್ಟ
ಉತ್ಕೃ-ಷ್ಟ-ವಾ-ದದ್ದು
ಉತ್ಕ್ರಾಂತಿ
ಉತ್ಕ್ರಾಂ-ತಿಯೇ
ಉತ್ತಮ
ಉತ್ತ-ಮ-ವಾ-ಗಿ-ರ-ಬ-ಹುದು
ಉತ್ತ-ಮ-ವಾ-ಗಿ-ರು-ವು-ದಾ-ದರೆ
ಉತ್ತ-ಮ-ವಾ-ಗು-ವು-ದರ
ಉತ್ತ-ಮ-ವಾ-ದಂತೆ
ಉತ್ತ-ಮ-ವಾ-ದ-ದ್ದ-ಕ್ಕೇನೂ
ಉತ್ತ-ಮವೂ
ಉತ್ತ-ಮ-ವೆಂದು
ಉತ್ತರ
ಉತ್ತ-ರ-ಕೊ-ಡಲು
ಉತ್ತ-ರ-ಗಳನ್ನು
ಉತ್ತ-ರ-ಗಳಲ್ಲಿ
ಉತ್ತ-ರದ
ಉತ್ತ-ರ-ದಲ್ಲಿ
ಉತ್ತ-ರ-ದಿಂದ
ಉತ್ತ-ರ-ಪ್ರ-ದೇ-ಶದ
ಉತ್ತ-ರ-ಭಾ-ರ-ತ-ದಲ್ಲಿ
ಉತ್ತ-ರ-ರೂ-ಪ-ವಾಗಿ
ಉತ್ತ-ರ-ವನ್ನು
ಉತ್ತ-ರ-ವಲ್ಲ
ಉತ್ತ-ರ-ವಾ-ಗ-ಲಾ-ರ-ದಷ್ಟೆ
ಉತ್ತ-ರ-ವಾಗಿ
ಉತ್ತ-ರ-ವಿದೆ
ಉತ್ತ-ರವು
ಉತ್ತ-ರ-ವೆಂ-ಬು-ದೇನೋ
ಉತ್ತ-ರಾ-ಭಿ-ಮು-ಖ-ವಾಗಿ
ಉತ್ತ-ರಿ-ಸದೆ
ಉತ್ತ-ರಿ-ಸ-ಬ-ಹುದು
ಉತ್ತ-ರಿ-ಸ-ಬೇ-ಕಾ-ಯಿತೋ
ಉತ್ತ-ರಿ-ಸಲು
ಉತ್ತ-ರಿಸಿ
ಉತ್ತ-ರಿ-ಸಿದ
ಉತ್ತ-ರಿ-ಸಿ-ದ-ರ-ಲ್ಲದೆ
ಉತ್ತ-ರಿ-ಸಿ-ದರು
ಉತ್ತ-ರಿ-ಸಿದೆ
ಉತ್ತ-ರಿ-ಸಿ-ದ್ದರು
ಉತ್ತ-ರಿ-ಸಿ-ದ್ದಳು
ಉತ್ತ-ರಿ-ಸಿದ್ದು
ಉತ್ತ-ರಿ-ಸಿ-ರ-ಲಿಲ್ಲ
ಉತ್ತ-ರಿ-ಸುತ್ತ
ಉತ್ತ-ರಿ-ಸು-ತ್ತಾರೆ
ಉತ್ತ-ರಿ-ಸು-ತ್ತಿದ್ದ
ಉತ್ತ-ರಿ-ಸು-ತ್ತಿ-ದ್ದರು
ಉತ್ತ-ರಿ-ಸು-ತ್ತಿ-ದ್ದ-ರೆಂ-ದರೆ
ಉತ್ತ-ರಿ-ಸು-ತ್ತಿ-ದ್ದು-ದೇನೋ
ಉತ್ತ-ರಿ-ಸು-ತ್ತೇನೆ
ಉತ್ತಿ-ಷ್ಠತ
ಉತ್ತೀ-ರ್ಣ-ರಾ-ದಂ-ತಾ-ಯಿತು
ಉತ್ತೇ-ಜನ
ಉತ್ತೇ-ಜಿ-ಸು-ವುದು
ಉತ್ಥಾ-ನದ
ಉತ್ಪ-ತ್ತಿ-ದಾ-ಯ-ಕ-ವ-ಲ್ಲದ
ಉತ್ಪ-ನ್ನ-ವಾ-ಗ-ಬೇ-ಕೆಂದು
ಉತ್ಪ-ನ್ನ-ವಾ-ಯಿತು
ಉತ್ಸವ
ಉತ್ಸ-ವಕ್ಕೆ
ಉತ್ಸ-ವ-ಗಳ
ಉತ್ಸ-ವ-ಗಳನ್ನು
ಉತ್ಸ-ವ-ಗಳು
ಉತ್ಸ-ವ-ಗ-ಳೆಲ್ಲ
ಉತ್ಸ-ವದ
ಉತ್ಸ-ವ-ದಂ-ತಿ-ದ್ದುವು
ಉತ್ಸ-ವ-ದಲ್ಲಿ
ಉತ್ಸ-ವ-ಮೂ-ರ್ತಿ-ಯನ್ನು
ಉತ್ಸ-ವ-ವನ್ನು
ಉತ್ಸ-ವ-ವಾ-ಗಿ-ರದೆ
ಉತ್ಸ-ವಾ-ದಿ-ಗಳನ್ನು
ಉತ್ಸಾಹ
ಉತ್ಸಾ-ಹ-ಕು-ತೂ-ಹ-ಲ-ಕಾ-ತ-ರ-ಸಂ-ಭ್ರ-ಮದ
ಉತ್ಸಾ-ಹ-ಕು-ತೂ-ಹ-ಲ-ಗಳಿಂದ
ಉತ್ಸಾ-ಹ-ಸಂ-ಭ್ರ-ಮದ
ಉತ್ಸಾ-ಹಕ್ಕೂ
ಉತ್ಸಾ-ಹ-ಗಳನ್ನು
ಉತ್ಸಾ-ಹ-ಗಳಿಂದ
ಉತ್ಸಾ-ಹ-ಗಳು
ಉತ್ಸಾ-ಹ-ಗೊಂಡು
ಉತ್ಸಾ-ಹದ
ಉತ್ಸಾ-ಹ-ದಲ್ಲಿ
ಉತ್ಸಾ-ಹ-ದಿಂದ
ಉತ್ಸಾ-ಹ-ದಿಂ-ದಲೇ
ಉತ್ಸಾ-ಹ-ಪೂರ್ಣ
ಉತ್ಸಾ-ಹ-ಪೂ-ರ್ಣ-ವಾದ
ಉತ್ಸಾ-ಹ-ಪೂ-ರ್ಣವೂ
ಉತ್ಸಾ-ಹ-ಭ-ರಿ-ತ-ರಾ-ಗಿ-ದ್ದರು
ಉತ್ಸಾ-ಹ-ಭ-ರಿ-ತ-ರಾದ
ಉತ್ಸಾ-ಹ-ಭ-ರಿ-ತ-ವಾದ
ಉತ್ಸಾ-ಹ-ಯುತ
ಉತ್ಸಾ-ಹ-ವಂತೂ
ಉತ್ಸಾ-ಹ-ವನ್ನು
ಉತ್ಸಾ-ಹವೇ
ಉತ್ಸಾಹಿ
ಉತ್ಸಾ-ಹಿ-ಗಳು
ಉತ್ಸಾ-ಹಿತ
ಉತ್ಸಾ-ಹಿ-ತ-ರಾಗಿ
ಉತ್ಸಾ-ಹಿ-ತ-ರಾ-ಗಿ-ದ್ದರು
ಉತ್ಸಾ-ಹಿ-ತ-ರಾದ
ಉತ್ಸಾಹೀ
ಉತ್ಸು-ಕ-ತೆ-ಯಿಂದ
ಉತ್ಸು-ಕ-ನಾಗಿ
ಉತ್ಸು-ಕ-ರಾಗಿ
ಉತ್ಸು-ಕ-ರಾ-ಗಿದ್ದ
ಉತ್ಸು-ಕ-ರಾ-ಗಿ-ದ್ದ-ರು-ಅ-ಥವಾ
ಉತ್ಸು-ಕ-ರಾ-ದರು
ಉದಯ
ಉದ-ಯ-ವಾ-ಗು-ವುದು
ಉದ-ಯಿ-ಸಲಿ
ಉದ-ಯಿ-ಸ-ಲಿ-ರುವ
ಉದ-ಯಿ-ಸುವ
ಉದಾ
ಉದಾತ್ತ
ಉದಾ-ತ್ತ-ಭವ್ಯ
ಉದಾರ
ಉದಾ-ರ-ದೃ-ಷ್ಟಿಯ
ಉದಾ-ರ-ಬುದ್ಧಿ
ಉದಾ-ರ-ಭಾ-ವ-ವನ್ನು
ಉದಾ-ರ-ವಾಗಿ
ಉದಾ-ರಿ-ಗಳನ್ನು
ಉದಾ-ರಿ-ಯಾ-ಗಿ-ದ್ದು-ದ-ರಿಂದ
ಉದಾ-ಸೀ-ನ-ರಾಗಿ
ಉದಾ-ಹ-ರಣೆ
ಉದಾ-ಹ-ರ-ಣೆ-ಗಳ
ಉದಾ-ಹ-ರ-ಣೆ-ಗ-ಳಷ್ಟೆ
ಉದಾ-ಹ-ರ-ಣೆಗೆ
ಉದಾ-ಹ-ರ-ಣೆ-ಯಾ-ದರು
ಉದಾ-ಹ-ರಿ-ಸ-ಬ-ಹುದು
ಉದಾ-ಹ-ರಿಸಿ
ಉದಾ-ಹ-ರಿ-ಸಿಲ್ಲ
ಉದಿ-ಸ-ಬ-ಹುದು
ಉದಿ-ಸ-ಬೇ-ಕಾ-ಗಿದೆ
ಉದಿ-ಸ-ಲಿ-ದ್ದಾರೆ
ಉದಿ-ಸ-ಲಿ-ದ್ದಾ-ರೆಯೆ
ಉದಿ-ಸ-ಲಿ-ರು-ವರೋ
ಉದಿಸಿ
ಉದಿ-ಸಿತು
ಉದಿ-ಸಿದ
ಉದಿ-ಸಿ-ದು-ದು-ಮತ್ತು
ಉದಿಸು
ಉದಿ-ಸು-ತ್ತಾರೆ
ಉದಿ-ಸು-ತ್ತಿದೆ
ಉದಿ-ಸು-ತ್ತಿದ್ದ
ಉದು-ರ-ಲಾ-ರಂ-ಭಿ-ಸಿತ್ತು
ಉದುರಿ
ಉದು-ರು-ತ್ತದೆ
ಉದ್ಗ-ರಿಸಿ
ಉದ್ಗ-ರಿ-ಸಿದ
ಉದ್ಗ-ರಿ-ಸಿ-ದರು
ಉದ್ಗ-ರಿ-ಸಿ-ದಳು
ಉದ್ಗ-ರಿ-ಸಿ-ದ್ದರು
ಉದ್ಗ-ರಿ-ಸಿ-ರ-ಲಿ-ಲ್ಲ-ವೆ-ಓಹ್
ಉದ್ಗ-ರಿಸು
ಉದ್ಗ-ರಿ-ಸುತ್ತ
ಉದ್ಗ-ರಿ-ಸು-ತ್ತಾನೆ
ಉದ್ಗ-ರಿ-ಸು-ತ್ತಾರೆ
ಉದ್ಗ-ರಿ-ಸು-ತ್ತಾಳೆ
ಉದ್ಗ-ರಿ-ಸು-ತ್ತಿ-ದ್ದರು
ಉದ್ಗ-ರಿ-ಸು-ತ್ತಿ-ದ್ದ-ರುಈ
ಉದ್ಗಾರ
ಉದ್ಗಾ-ರ-ಗಳಿಂದ
ಉದ್ಗಾ-ರವೇ
ಉದ್ಗಾ-ರ-ವೊಂದು
ಉದ್ಘಾ-ಟ-ನೆ-ಯೊಂ-ದಿಗೆ
ಉದ್ಘಾ-ಟಿ-ಸಿ-ದರು
ಉದ್ಘೋಷ
ಉದ್ಘೋ-ಷ-ದೊಂ-ದಿಗೆ
ಉದ್ಘೋ-ಷಿ-ಸಿ-ದರು
ಉದ್ಘೋ-ಷಿ-ಸುತ್ತ
ಉದ್ದ
ಉದ್ದಕ್ಕೂ
ಉದ್ದ-ಗ-ಲಕ್ಕೂ
ಉದ್ದದ
ಉದ್ದ-ವಿ-ರು-ತ್ತದೆ
ಉದ್ದಿ-ಶ್ಯ-ಪೂ-ರ್ಣ-ವಾ-ಗಿ-ದ್ದುವು
ಉದ್ದಿ-ಶ್ಯ-ವೇನು
ಉದ್ದೀ-ಪ-ನ-ಗೊ-ಳಿ-ಸುವ
ಉದ್ದೇಶ
ಉದ್ದೇ-ಶ-ಕ್ಕಾಗಿ
ಉದ್ದೇ-ಶ-ಕ್ಕಾ-ಗಿಯೇ
ಉದ್ದೇ-ಶಕ್ಕೆ
ಉದ್ದೇ-ಶ-ಗಳನ್ನು
ಉದ್ದೇ-ಶ-ಗಳಲ್ಲಿ
ಉದ್ದೇ-ಶ-ಗ-ಳಿ-ರ-ಬ-ಹು-ದೆಂದು
ಉದ್ದೇ-ಶ-ಗ-ಳು
ಉದ್ದೇ-ಶ-ದತ್ತ
ಉದ್ದೇ-ಶ-ದಿಂದ
ಉದ್ದೇ-ಶ-ದಿಂ-ದಲೇ
ಉದ್ದೇ-ಶ-ಪೂ-ರ್ವ-ಕ-ವಾ-ಗಿಯೂ
ಉದ್ದೇ-ಶ-ಪೂ-ರ್ವ-ಕ-ವಾ-ಗಿಯೇ
ಉದ್ದೇ-ಶ-ವನ್ನು
ಉದ್ದೇ-ಶ-ವಾ-ಗಿತ್ತು
ಉದ್ದೇ-ಶ-ವಾ-ಗಿದೆ
ಉದ್ದೇ-ಶ-ವಾ-ಗಿ-ರ-ಲಿಲ್ಲ
ಉದ್ದೇ-ಶ-ವಾ-ದರೂ
ಉದ್ದೇ-ಶ-ವಾ-ದರೆ
ಉದ್ದೇ-ಶ-ವಿ-ಟ್ಟು-ಕೊಂ-ಡಿ-ದ್ದರು
ಉದ್ದೇ-ಶ-ವಿತ್ತು
ಉದ್ದೇ-ಶ-ವಿ-ರಿ-ಸಿ-ಕೊಂಡ
ಉದ್ದೇ-ಶವೂ
ಉದ್ದೇ-ಶ-ವೆಂದರೆ
ಉದ್ದೇ-ಶ-ವೆಂದು
ಉದ್ದೇ-ಶವೇ
ಉದ್ದೇ-ಶ-ವೇ-ನಿ-ರ-ಬ-ಹು-ದೆಂದು
ಉದ್ದೇ-ಶ-ವೇನು
ಉದ್ದೇ-ಶ-ವೇನೆಂದರೆ
ಉದ್ದೇ-ಶ-ವೇ-ನೆಂದು
ಉದ್ದೇ-ಶ-ವೇ-ನೆಂ-ಬುದು
ಉದ್ದೇ-ಶ-ವೊಂದು
ಉದ್ದೇ-ಶ-ವ್ಯಾ-ಪ್ತಿಯ
ಉದ್ದೇ-ಶಿತ
ಉದ್ದೇ-ಶಿಸ
ಉದ್ದೇ-ಶಿಸಿ
ಉದ್ದೇ-ಶಿ-ಸಿದ್ದ
ಉದ್ದೇ-ಶಿ-ಸಿಯೇ
ಉದ್ದೇ-ಶಿ-ಸಿ-ರುವ
ಉದ್ದೇ-ಶಿ-ಸಿ-ರು-ವಿ-ರೇನು
ಉದ್ಧರಿ
ಉದ್ಧ-ರಿಸಿ
ಉದ್ಧ-ರಿ-ಸಿ-ಕೊ-ಳ್ಳ-ಬೇಕು
ಉದ್ಧ-ರಿ-ಸಿ-ದರು
ಉದ್ಧ-ರಿ-ಸುತ್ತ
ಉದ್ಧರೇ
ಉದ್ಧಾರ
ಉದ್ಧಾ-ರ-ಕ-ನಂ-ತಿ-ದ್ದರು
ಉದ್ಧಾ-ರ-ಕ್ಕಾಗಿ
ಉದ್ಧಾ-ರಕ್ಕೆ
ಉದ್ಧಾ-ರ-ವಾ-ಗ-ಲಾ-ರೆವು
ಉದ್ಧಾ-ರ-ವಾ-ಗ-ಲೆಂದು
ಉದ್ಧಾ-ರ-ವಾ-ಗು-ವುದು
ಉದ್ಧಾ-ರ-ವಿಲ್ಲ
ಉದ್ಬೋ-ಧಕ
ಉದ್ಬೋ-ಧನ
ಉದ್ಭವ
ಉದ್ಭ-ವಿ-ಸ-ಬ-ಹು-ದಾ-ದಂ-ತಹ
ಉದ್ಭ-ವಿ-ಸಿದ
ಉದ್ಯಮ
ಉದ್ಯಾನ
ಉದ್ಯಾ-ನಕ್ಕೆ
ಉದ್ಯಾ-ನ-ಗಳನ್ನು
ಉದ್ಯಾ-ನ-ಗಳನ್ನೂ
ಉದ್ಯಾ-ನ-ಗಳಲ್ಲಿ
ಉದ್ಯಾ-ನ-ಗೃ-ಹಕ್ಕೆ
ಉದ್ಯಾ-ನ-ಗೃ-ಹ-ದಲ್ಲಿ
ಉದ್ಯಾ-ನ-ಗೃ-ಹ-ದ-ಲ್ಲಿ-ದ್ದಾಗ
ಉದ್ಯಾ-ನ-ಗೃ-ಹ-ದ-ಲ್ಲಿನ
ಉದ್ಯಾ-ನ-ದಲ್ಲಿ
ಉದ್ಯಾ-ನ-ವೊಂ-ದ-ರಲ್ಲಿ
ಉದ್ಯುಕ್ತ
ಉದ್ಯು-ಕ್ತ-ರಾ-ದರು
ಉದ್ಯೋಗ
ಉದ್ಯೋ-ಗಕ್ಕೆ
ಉದ್ಯೋ-ಗದ
ಉದ್ಯೋ-ಗ-ದ-ಲ್ಲಿ-ದ್ದರು
ಉದ್ಯೋ-ಗ-ವನ್ನು
ಉದ್ರೇ-ಕ-ಗೊ-ಳಿ-ಸುವ
ಉದ್ರೇ-ಕ-ಗೊ-ಳ್ಳದೆ
ಉದ್ವಿ-ಗ್ನ-ರಾಗಿ
ಉದ್ವೇ-ಗ-ದಿಂದ
ಉದ್ವೇ-ಗ-ಭ-ರ-ದಲ್ಲಿ
ಉದ್ವೇ-ಗ-ಭ-ರಿ-ತ-ರಾಗಿ
ಉನ್ನತ
ಉನ್ನ-ತ-ಮ-ಟ್ಟದ
ಉನ್ನ-ತ-ವಾ-ಗಿದೆ
ಉನ್ನ-ತ-ವಾದ
ಉನ್ನ-ತ-ವಾ-ದ-ದ್ದೇನೋ
ಉನ್ನ-ತಾ-ಧಿ-ಕಾ-ರಿ-ಗಳೂ
ಉನ್ನ-ತಿ-ಗಾಗಿ
ಉನ್ನ-ತಿಗೆ
ಉನ್ನ-ತಿ-ಗೇ-ರಲು
ಉನ್ಮ-ತ್ತ-ರಾ-ಗಿ-ದ್ದಾರೆ
ಉನ್ಮ-ತ್ತ-ವಾ-ಗಿತ್ತು
ಉಪ
ಉಪ-ಕ-ರಣ
ಉಪ-ಕ-ರ-ಣ-ಗ-ಳ-ನ್ನಾ-ಗಿ-ಸಿ-ಕೊಳ್ಳಿ
ಉಪ-ಕ-ರ-ಣ-ವ-ನ್ನಾಗಿ
ಉಪ-ಕಾರ
ಉಪ-ಕಾ-ರ-ಇ-ವೆಲ್ಲ
ಉಪ-ಕಾ-ರಕ
ಉಪ-ಕಾ-ರ-ಕ-ವಾ-ಗಿ-ರು-ವು-ದ-ರಿಂದ
ಉಪ-ಕಾ-ರದ
ಉಪ-ಕಾ-ರ-ವನ್ನು
ಉಪ-ಕಾ-ರ-ವಾ-ಗ-ಬೇ-ಕಾ-ಗಿದೆ
ಉಪ-ಕಾ-ರ್ಯ-ದ-ರ್ಶಿ-ಗ-ಳಾ-ದರು
ಉಪ-ಕೃ-ತ-ರಾ-ಗಿ-ದ್ದೇ-ವೆಂದು
ಉಪ-ಚ-ರಿ-ಸ-ಬೇ-ಕೆಂಬ
ಉಪ-ಚ-ರಿ-ಸಿ-ದರು
ಉಪ-ಚ-ರಿ-ಸಿ-ದಳು
ಉಪ-ಚ-ರಿ-ಸಿ-ದು-ದುನ್ನು
ಉಪ-ಚ-ರಿ-ಸಿ-ಯೇ-ನೆಂಬ
ಉಪ-ಚಾ-ರ-ಕ್ಕಾಗಿ
ಉಪ-ಚಾ-ರ-ದಿಂದ
ಉಪ-ಚಾ-ರ-ವನ್ನು
ಉಪ-ದೇಶ
ಉಪ-ದೇ-ಶಕ್ಕೆ
ಉಪ-ದೇ-ಶ-ಗಳ
ಉಪ-ದೇ-ಶ-ಗ-ಳಿ-ಗಿಂತ
ಉಪ-ದೇ-ಶಾ-ಮೃ-ತವು
ಉಪ-ದೇ-ಶಿ-ಸಿದ
ಉಪ-ದೇ-ಶಿ-ಸು-ವು-ದ-ಕ್ಕಾಗಿ
ಉಪ-ದ್ರ-ವದ
ಉಪ-ನ-ಯ-ನದ
ಉಪನಿ
ಉಪ-ನಿ-ಷತ್
ಉಪ-ನಿ-ಷ-ತ್ತನ್ನು
ಉಪ-ನಿ-ಷ-ತ್ತಿಗೆ
ಉಪ-ನಿ-ಷ-ತ್ತಿನ
ಉಪ-ನಿ-ಷತ್ತು
ಉಪ-ನಿ-ಷ-ತ್ತು-ಗಳ
ಉಪ-ನಿ-ಷ-ತ್ತು-ಗಳನ್ನು
ಉಪ-ನಿ-ಷ-ತ್ತು-ಗಳಲ್ಲಿ
ಉಪ-ನಿ-ಷ-ತ್ತು-ಗ-ಳಲ್ಲೂ
ಉಪ-ನಿ-ಷ-ತ್ತು-ಗಳಿಂದ
ಉಪ-ನಿ-ಷ-ತ್ತು-ಗ-ಳಿಗೆ
ಉಪ-ನಿ-ಷ-ತ್ತು-ಗಳು
ಉಪ-ನಿ-ಷ-ತ್ತು-ಗ-ಳೆ-ಡೆಗೆ
ಉಪ-ನಿ-ಷತ್ತೇ
ಉಪ-ನ್ಯಾಸ
ಉಪ-ನ್ಯಾ-ಸ-ಚ-ರ್ಚೆ-ಗಳು
ಉಪ-ನ್ಯಾ-ಸ-ತ-ರ-ಗ-ತಿ-ಗಳನ್ನು
ಉಪ-ನ್ಯಾ-ಸ-ಕನ
ಉಪ-ನ್ಯಾ-ಸ-ಕ-ರ-ನ್ನಾಗಿ
ಉಪ-ನ್ಯಾ-ಸಕ್ಕೆ
ಉಪ-ನ್ಯಾ-ಸ-ಗಳ
ಉಪ-ನ್ಯಾ-ಸ-ಗಳನ್ನು
ಉಪ-ನ್ಯಾ-ಸ-ಗ-ಳ-ಲ್ಲದೆ
ಉಪ-ನ್ಯಾ-ಸ-ಗಳಲ್ಲಿ
ಉಪ-ನ್ಯಾ-ಸ-ಗ-ಳಲ್ಲೂ
ಉಪ-ನ್ಯಾ-ಸ-ಗ-ಳ-ಲ್ಲೆಲ್ಲ
ಉಪ-ನ್ಯಾ-ಸ-ಗಳಿಂದ
ಉಪ-ನ್ಯಾ-ಸ-ಗ-ಳಿ-ಗಿಂ-ತಲೂ
ಉಪ-ನ್ಯಾ-ಸ-ಗ-ಳಿಗೂ
ಉಪ-ನ್ಯಾ-ಸ-ಗ-ಳಿಗೆ
ಉಪ-ನ್ಯಾ-ಸ-ಗಳು
ಉಪ-ನ್ಯಾ-ಸ-ಗಳೂ
ಉಪ-ನ್ಯಾ-ಸ-ಗ-ಳೆ-ಲ್ಲಕ್ಕೂ
ಉಪ-ನ್ಯಾ-ಸದ
ಉಪ-ನ್ಯಾ-ಸ-ದಂ-ತೆಯೇ
ಉಪ-ನ್ಯಾ-ಸ-ದಲ್ಲಿ
ಉಪ-ನ್ಯಾ-ಸ-ದೊಂ-ದಿಗೆ
ಉಪ-ನ್ಯಾ-ಸ-ಮಾ-ಲಿ-ಕೆಯು
ಉಪ-ನ್ಯಾ-ಸ-ಮಾ-ಲಿ-ಕೆ-ಯೊಂ-ದರ
ಉಪ-ನ್ಯಾ-ಸ-ಮಾ-ಲೆ-ಯಿಂದ
ಉಪ-ನ್ಯಾ-ಸ-ವನ್ನು
ಉಪ-ನ್ಯಾ-ಸ-ವಾದ
ಉಪ-ನ್ಯಾ-ಸವು
ಉಪ-ನ್ಯಾ-ಸವೂ
ಉಪ-ನ್ಯಾ-ಸ-ವೆಂದು
ಉಪ-ನ್ಯಾ-ಸವೇ
ಉಪ-ನ್ಯಾ-ಸ-ವೊಂ-ದನ್ನು
ಉಪ-ನ್ಯಾ-ಸ-ವೊಂ-ದ-ರಲ್ಲಿ
ಉಪ-ನ್ಯಾ-ಸ-ವೊಂದು
ಉಪ-ನ್ಯಾ-ಸಾ-ದಿ-ಗಳನ್ನು
ಉಪ-ಫ-ಲ-ಗ-ಳಲ್ಲ
ಉಪ-ಯುಕ್ತ
ಉಪ-ಯು-ಕ್ತ-ವ-ಲ್ಲದ
ಉಪ-ಯು-ಕ್ತ-ವಾಗಿ
ಉಪ-ಯು-ಕ್ತ-ವಾ-ಗಿ-ದ್ದುವು
ಉಪ-ಯು-ಕ್ತ-ವಾ-ಗುವ
ಉಪ-ಯು-ಕ್ತ-ವಾದ
ಉಪ-ಯು-ಕ್ತ-ವಾ-ದದ್ದು
ಉಪ-ಯು-ಕ್ತವೂ
ಉಪ-ಯೋ-ಗ-ಕ್ಕಾಗಿ
ಉಪ-ಯೋ-ಗದ
ಉಪ-ಯೋಗಿ
ಉಪ-ಯೋ-ಗಿ-ಸ-ಬೇಕು
ಉಪ-ಯೋ-ಗಿಸಿ
ಉಪ-ಯೋ-ಗಿ-ಸಿ-ಕೊಂಡು
ಉಪ-ಯೋ-ಗಿ-ಸಿ-ಕೊ-ಳ್ಳ-ಬ-ಹು-ದ-ಲ್ಲವೆ
ಉಪ-ಯೋ-ಗಿ-ಸಿ-ಕೊ-ಳ್ಳ-ಬಾ-ರದು
ಉಪ-ಯೋ-ಗಿ-ಸಿ-ಕೊ-ಳ್ಳಲು
ಉಪ-ಯೋ-ಗಿ-ಸಿ-ಕೊ-ಳ್ಳು-ತ್ತಿದ್ದ
ಉಪ-ಯೋ-ಗಿ-ಸಿ-ದ-ವ-ರಲ್ಲ
ಉಪ-ಯೋ-ಗಿ-ಸಿ-ಬಿಡಿ
ಉಪ-ವಾಸ
ಉಪ-ವಾ-ಸ-ಜಾ-ಗ-ರ-ಣೆಯೇ
ಉಪ-ವಾ-ಸ-ದಿಂದ
ಉಪ-ವಾ-ಸ-ಬಿದ್ದು
ಉಪ-ವಾ-ಸ-ವಿ-ರು-ವುದೇ
ಉಪ-ಸ್ಥಿತ
ಉಪ-ಸ್ಥಿ-ತ-ರಿ-ದ್ದರು
ಉಪಾ
ಉಪಾ-ಧ್ಯ-ಕ್ಷ-ರಾ-ದರು
ಉಪಾ-ಧ್ಯ-ಕ್ಷ-ರು-ಗ-ಳಾ-ದರು
ಉಪಾ-ಧ್ಯಾ-ಯಿ-ನಿ-ಯರೇ
ಉಪಾಯ
ಉಪಾ-ಯ-ಗಳ
ಉಪಾ-ಯ-ಗ-ಳ-ನ್ನ-ನು-ಸ-ರಿಸಿ
ಉಪಾ-ಯ-ಗಳನ್ನು
ಉಪಾ-ಯ-ಗ-ಳಿವೆ
ಉಪಾ-ಯ-ವಿಲ್ಲ
ಉಪಾ-ಯ-ವೇ-ನಾ-ದರೂ
ಉಪಾ-ಯ-ವೇನು
ಉಪಾ-ಸನೆ
ಉಪಾ-ಸ-ನೆ-ಗಳು
ಉಪಾ-ಹಾರ
ಉಪಾ-ಹಾ-ರ-ಗಳನ್ನು
ಉಪಾ-ಹಾ-ರದ
ಉಪಾ-ಹಾ-ರ-ವನ್ನು
ಉಪಾ-ಹಾ-ರ-ವನ್ನೂ
ಉಪಾ-ಹಾ-ರ-ವಿತ್ತು
ಉಪ್ಪನ್ನು
ಉಪ್ಪನ್ನೂ
ಉಪ್ಪಿ-ಲ್ಲದೆ
ಉಪ್ಪು
ಉಬ್ಬ-ದಿ-ರದು
ಉಬ್ಬಸ
ಉಬ್ಬ-ಸದ
ಉಬ್ಬ-ಸ-ದಿಂದ
ಉಬ್ಬ-ಸ-ರೋ-ಗ-ದಿಂದ
ಉಬ್ಬಾ-ದರೆ
ಉಮಾ
ಉಮಾ-ಕು-ಮಾ-ರಿಯ
ಉರಿ
ಉರಿದು
ಉರಿ-ದು-ಹೋ-ಯಿತು
ಉರಿ-ಬಿ-ಸಿ-ಲಿ-ನಲ್ಲಿ
ಉರಿ-ಬಿ-ಸಿ-ಲಿ-ನಿಂದ
ಉರಿ-ಬಿ-ಸಿಲು
ಉರಿ-ಯನ್ನು
ಉರಿ-ಯ-ಲಿಲ್ಲ
ಉರಿ-ಯಲ್ಲಿ
ಉರಿ-ಯು-ತ್ತಿದ್ದು
ಉರಿ-ಸು-ತ್ತಿದ್ದ
ಉರು-ಳಿ-ಕೊ-ಳ್ಳ-ಬ-ಹು-ದಾ-ಗಿತ್ತು
ಉರು-ಳಿ-ಕೊ-ಳ್ಳು-ತ್ತೇನೆ
ಉರು-ಳಿ-ದರು
ಉರು-ಳಿ-ದರೂ
ಉರು-ಳಿ-ದುವು
ಉರು-ಳಿ-ಬಿ-ದ್ದರೂ
ಉರು-ಳುವ
ಉರು-ಳು-ಸೇವೆ
ಉಲ್ಬ-ಣಿಸಿ
ಉಲ್ಬ-ಣಿ-ಸಿತು
ಉಲ್ಬ-ಣಿ-ಸಿ-ತೆಂ-ದರೆ
ಉಲ್ಬ-ಣಿ-ಸು-ತ್ತಿ-ದ್ದುವು
ಉಲ್ಲಂ-ಘಿ-ಸ-ಲೇ-ಬೇಕು
ಉಲ್ಲ-ಸಿ-ತ-ರಾ-ಗಿ-ದ್ದರು
ಉಲ್ಲ-ಸಿ-ತ-ರಾ-ದರು
ಉಲ್ಲಾಸ
ಉಲ್ಲಾ-ಸ-ಕ-ರವೂ
ಉಲ್ಲಾ-ಸ-ಭ-ರಿತ
ಉಲ್ಲಾ-ಸ-ಭ-ರಿ-ತ-ರಾಗಿ
ಉಲ್ಲಾ-ಸ-ಯು-ತ-ರಾ-ಗಿ-ದ್ದ-ರಿಂದ
ಉಲ್ಲೇ-ಖಿ-ಸಿ-ದ್ದಾರೆ
ಉಳಿದ
ಉಳಿ-ದಳು
ಉಳಿ-ದ-ವ-ರಂ-ತಲ್ಲ
ಉಳಿ-ದ-ವ-ರ-ನ್ನೆಲ್ಲ
ಉಳಿ-ದ-ವ-ರಿಗೆ
ಉಳಿ-ದ-ವ-ರಿ-ಗೆಲ್ಲ
ಉಳಿ-ದ-ವರು
ಉಳಿ-ದ-ವರೆಲ್ಲ
ಉಳಿ-ದ-ವು-ಗಳ
ಉಳಿ-ದಿತ್ತು
ಉಳಿ-ದಿ-ದೆಯೆ
ಉಳಿ-ದಿದ್ದ
ಉಳಿ-ದಿ-ದ್ದಾರೆ
ಉಳಿ-ದಿ-ಬ್ಬ-ರಾದ
ಉಳಿ-ದಿ-ರುವ
ಉಳಿ-ದಿಲ್ಲ
ಉಳಿ-ದಿವೆ
ಉಳಿ-ದೀತು
ಉಳಿದು
ಉಳಿ-ದುಕೊ
ಉಳಿ-ದು-ಕೊಂಡ
ಉಳಿ-ದು-ಕೊಂ-ಡ-ರ-ಲ್ಲದೆ
ಉಳಿ-ದು-ಕೊಂ-ಡರು
ಉಳಿ-ದು-ಕೊಂ-ಡರೂ
ಉಳಿ-ದು-ಕೊಂ-ಡ-ಳ-ಲ್ಲದೆ
ಉಳಿ-ದು-ಕೊಂ-ಡಳು
ಉಳಿ-ದು-ಕೊಂ-ಡ-ವ-ನಲ್ಲ
ಉಳಿ-ದು-ಕೊಂ-ಡ-ವರು
ಉಳಿ-ದು-ಕೊಂ-ಡಾ-ರೆಂಬ
ಉಳಿ-ದು-ಕೊಂ-ಡಿತು
ಉಳಿ-ದು-ಕೊಂ-ಡಿದೆ
ಉಳಿ-ದು-ಕೊಂ-ಡಿದ್ದ
ಉಳಿ-ದು-ಕೊಂ-ಡಿ-ದ್ದರು
ಉಳಿ-ದು-ಕೊಂ-ಡಿ-ದ್ದರೆ
ಉಳಿ-ದು-ಕೊಂ-ಡಿ-ರ-ಬ-ಹುದು
ಉಳಿ-ದು-ಕೊಂ-ಡಿವೆ
ಉಳಿ-ದು-ಕೊಂಡು
ಉಳಿ-ದು-ಕೊಂ-ಡು-ಬಿ-ಟ್ಟರು
ಉಳಿ-ದು-ಕೊಂ-ಡುವು
ಉಳಿ-ದು-ಕೊ-ಳ್ಳ-ಬ-ಹುದು
ಉಳಿ-ದು-ಕೊ-ಳ್ಳ-ಬೇ-ಕಾ-ದರೆ
ಉಳಿ-ದು-ಕೊ-ಳ್ಳ-ಬೇಕು
ಉಳಿ-ದು-ಕೊ-ಳ್ಳ-ಬೇ-ಕೆಂದು
ಉಳಿ-ದು-ಕೊ-ಳ್ಳ-ಬೇ-ಕೆಂ-ಬುದು
ಉಳಿ-ದು-ಕೊ-ಳ್ಳ-ಲಿ-ಲ್ಲ-ವಾ-ದರೂ
ಉಳಿ-ದು-ಕೊ-ಳ್ಳಲು
ಉಳಿ-ದು-ಕೊ-ಳ್ಳು-ತ್ತಾ-ರಂತೆ
ಉಳಿ-ದು-ಕೊ-ಳ್ಳುವ
ಉಳಿ-ದು-ಕೊ-ಳ್ಳು-ವಂ-ತಾ-ಗಲು
ಉಳಿ-ದು-ಕೊ-ಳ್ಳು-ವಂತೆ
ಉಳಿ-ದು-ಕೊ-ಳ್ಳು-ವುದು
ಉಳಿ-ದು-ಕೊ-ಳ್ಳು-ವು-ದೆಂದು
ಉಳಿ-ದು-ಕೊ-ಳ್ಳು-ವು-ದೆಂದೂ
ಉಳಿ-ದು-ದನ್ನು
ಉಳಿ-ದು-ದೆಲ್ಲ
ಉಳಿ-ದು-ದೆ-ಲ್ಲವೂ
ಉಳಿ-ದು-ಹೋ-ಗಿದ್ದು
ಉಳಿ-ಯ-ದಂ-ತಾ-ದಾಗ
ಉಳಿ-ಯದು
ಉಳಿ-ಯ-ಬಾ-ರದು
ಉಳಿ-ಯ-ಬೇ-ಕಾದ
ಉಳಿ-ಯ-ಬೇ-ಕಾ-ದರೆ
ಉಳಿ-ಯ-ಬೇ-ಕಾ-ಯಿತು
ಉಳಿ-ಯ-ಲಾರ
ಉಳಿ-ಯ-ಲಾ-ರದು
ಉಳಿ-ಯ-ಲಿಲ್ಲ
ಉಳಿ-ಯು-ತ್ತದೆ
ಉಳಿ-ಯು-ವಂ-ತೆಯೇ
ಉಳಿ-ವಿ-ಗಾಗಿ
ಉಳಿ-ವಿನ
ಉಳಿವು
ಉಳಿ-ಸ-ಬೇಕು
ಉಳಿ-ಸ-ಲಾ-ಗದು
ಉಳಿಸಿ
ಉಳಿ-ಸಿ-ಕೊಂ-ಡ-ರು-ಸ-ಭಿ-ಕರ
ಉಳಿ-ಸಿ-ಕೊಂ-ಡಿದೆ
ಉಳಿ-ಸಿ-ಕೊಂ-ಡಿ-ದ್ದರು
ಉಳಿ-ಸಿ-ಕೊಂಡು
ಉಳಿ-ಸಿ-ಕೊ-ಳ್ಳ-ಬೇ-ಕೆಂದು
ಉಳಿ-ಸಿ-ಕೊ-ಳ್ಳ-ಬೇ-ಕೆಂ-ಬುದೇ
ಉಳಿ-ಸಿ-ಕೊ-ಳ್ಳಲು
ಉಳಿ-ಸಿ-ಕೊ-ಳ್ಳು-ವುದು
ಉಳಿ-ಸಿ-ಡ-ಲಾಗಿದೆ
ಉಳು-ವು-ದಕ್ಕೂ
ಉಳ್ಳ-ದ್ದಾ-ದರೂ
ಉಷಃ-ಕಾ-ಲ-ದಲ್ಲಿ
ಉಷಃ-ಕಾ-ಲ-ದಲ್ಲೇ
ಉಷ್ಟ್ರ-ಪ-ಕ್ಷಿಯು
ಉಷ್ಣ-ತೆ-ಯನ್ನು
ಉಷ್ಣವೂ
ಉಸಿ-ರನ್ನೂ
ಉಸಿ-ರಾಟ
ಉಸಿ-ರಾ-ಟದ
ಉಸಿ-ರಾ-ಟ-ವಿ-ಲ್ಲ-ದಿ-ರು-ವುದನ್ನು
ಉಸಿ-ರಾ-ಡು-ತ್ತೇವೆ
ಉಸಿ-ರಾ-ಡು-ವು-ದಕ್ಕೆ
ಉಸಿ-ರಾ-ಡು-ವುದನ್ನು
ಉಸಿ-ರಾ-ಡು-ವು-ದು-ಇ-ವು-ಗಳ
ಉಸಿರಿ
ಉಸಿ-ರಿ-ಗಾಗಿ
ಉಸಿರು
ಉಸಿ-ರು-ಗ-ಟ್ಟಿ-ಸುವ
ಉಸಿರೂ
ಉಸಿ-ರೆ-ಳೆ-ದರು
ಉಸಿ-ರೆ-ಳೆದು
ಉಸಿ-ರೆ-ಳೆ-ಯು-ತ್ತಿ-ದ್ದಂತೆ
ಉಸು-ರಿ-ದರು
ಉಸ್ತು-ವಾ-ರಿ-ಯನ್ನು
ಊಟ
ಊಟ
ಊಟ-ತಿಂ-ಡಿ-ಗಳ
ಊಟ-ಕ್ಕಾಗಿ
ಊಟಕ್ಕೆ
ಊಟದ
ಊಟ-ಮಾ-ಡಿ-ದರು
ಊಟ-ಮಾ-ಡು-ತ್ತಿ-ದ್ದುದು
ಊಟ-ವನ್ನು
ಊಟ-ವಾದ
ಊಟ-ಹಾ-ಕ-ಬೇಕು
ಊಟಿಗೆ
ಊದಿ
ಊದಿ-ಕೊಂ-ಡಿ-ದ್ದು-ದನ್ನು
ಊದಿ-ಕೊ-ಳ್ಳು-ವ-ವ-ರೆಗೆ
ಊದುತ್ತ
ಊರ
ಊರನ್ನು
ಊರ-ಲ್ಲೆಲ್ಲ
ಊರಾದ
ಊರಿಗೂ
ಊರಿಗೆ
ಊರಿನ
ಊರಿ-ನಲ್ಲಿ
ಊರು
ಊರು-ಹಳ್ಳಿ
ಊರು-ಗಳ
ಊರು-ಗಳಲ್ಲಿ
ಊರು-ಗಳಿಂದ
ಊರು-ಗ-ಳಿಂ-ದ-ಸಂ-ಘ-ಸಂ-ಸ್ಥೆ-ಗಳಿಂದ
ಊರು-ಗ-ಳಿಂ-ದ-ಲ್ಲದೆ
ಊರು-ಗ-ಳಿಗೂ
ಊರು-ಗ-ಳಿಗೆ
ಊರು-ಗೋಲು
ಊರೂ
ಊಹಿ
ಊಹಿ-ಸದ
ಊಹಿ-ಸ-ಬ-ಲ್ಲಿರಾ
ಊಹಿ-ಸ-ಬಲ್ಲೆ
ಊಹಿ-ಸ-ಬ-ಹುದು
ಊಹಿ-ಸ-ಲ-ಸಾಧ್ಯ
ಊಹಿ-ಸ-ಲಾ-ಗ-ಲಿಲ್ಲ
ಊಹಿ-ಸ-ಲಾ-ರರು
ಊಹಿ-ಸ-ಲಾರೆ
ಊಹಿ-ಸಲು
ಊಹಿ-ಸಲೂ
ಊಹಿಸಿ
ಊಹಿ-ಸಿ-ಕೊ-ಳ್ಳು-ವುದು
ಊಹಿ-ಸಿ-ದರು
ಊಹಿ-ಸಿದ್ದ
ಊಹಿ-ಸಿ-ದ್ದಿ-ರ-ಲಾ-ರರು
ಊಹಿ-ಸಿ-ರ-ದಿದ್ದ
ಊಹಿ-ಸಿ-ರ-ಲಿಲ್ಲ
ಊಹಿ-ಸು-ತ್ತಾಳೆ
ಊಹೆಗೂ
ಋಷಿ-ಯಾ-ಗಲಿ
ಎ
ಎಂ
ಎಂಜಿ-ನಿ-ಯ-ರಾಗಿ
ಎಂಜಿ-ನಿ-ಯ-ರಾ-ದ್ದ-ರಿಂದ
ಎಂಟಡಿ
ಎಂಟ-ರಂದು
ಎಂಟ-ರಿಂದ
ಎಂಟಾಣೆ
ಎಂಟು
ಎಂಟು-ಒಂ-ಬತ್ತು
ಎಂಟು-ದಿನ
ಎಂಟು-ನೂರು
ಎಂಟೂ-ವರೆ
ಎಂತಹ
ಎಂತ-ಹ-ದೆಂಬ
ಎಂತಿವೆ
ಎಂತೇ
ಎಂಥ
ಎಂಥದು
ಎಂಥವ
ಎಂಥ-ವನೂ
ಎಂಥ-ವನೇ
ಎಂಥ-ವ-ರನ್ನೂ
ಎಂಥ-ವ-ರಿ-ಗಾ-ದರೂ
ಎಂಥಾ
ಎಂದ
ಎಂದಂ-ತಾ-ಯಿತು
ಎಂದಂತೂ
ಎಂದಂತೆ
ಎಂದ-ಮೇಲೆ
ಎಂದ-ರಾ-ಯಿತು
ಎಂದ-ರಿತು
ಎಂದ-ರಿ-ತು-ಕೊಳ್ಳ
ಎಂದರು
ಎಂದರೆ
ಎಂದರ್ಥ
ಎಂದಲ್ಲ
ಎಂದ-ಲ್ಲವೆ
ಎಂದಳು
ಎಂದಷ್ಟೇ
ಎಂದಾಕೆ
ಎಂದಾಗ
ಎಂದಾ-ಗು-ತ್ತೀರಿ
ಎಂದಾ-ಗು-ವುದು
ಎಂದಾ-ದರೂ
ಎಂದಾ-ದ-ರೊಮ್ಮೆ
ಎಂದಿಗೂ
ಎಂದಿಗೆ
ಎಂದಿ-ಟ್ಟು-ಕೊಂ-ಡರೂ
ಎಂದಿ-ಟ್ಟು-ಕೊಂ-ಡರೆ
ಎಂದಿನ
ಎಂದಿ-ನಂತೆ
ಎಂದಿ-ನಷ್ಟೇ
ಎಂದಿ-ನಿಂ-ದಲೂ
ಎಂದು
ಎಂದು-ಆ-ಕೆಗೆ
ಎಂದು-ಕೊಂ-ಡಿ-ದ್ದೇನೆ
ಎಂದು-ಕೊಂ-ಡಿರು
ಎಂದು-ಕೊಂಡು
ಎಂದು-ಕೊಳ್ಳ
ಎಂದು-ಕೊಳ್ಳು
ಎಂದು-ಕೊ-ಳ್ಳು-ತ್ತಿ-ದ್ದ-ಳಂತೆ
ಎಂದು-ಕೊ-ಳ್ಳು-ವುದೇ
ಎಂದು-ತ್ತ-ರಿಸಿ
ಎಂದು-ತ್ತ-ರಿ-ಸಿ-ದರು
ಎಂದು-ತ್ತ-ರಿ-ಸಿ-ದಳು
ಎಂದು-ತ್ತ-ರಿಸು
ಎಂದು-ದ್ಗ-ರಿ-ಸಿದ
ಎಂದು-ದ್ಗ-ರಿ-ಸಿ-ದರು
ಎಂದು-ದ್ಗ-ರಿ-ಸಿ-ದಳು
ಎಂದು-ಬಿ-ಟ್ಟರು
ಎಂದೂ
ಎಂದೆ
ಎಂದೆಂ-ದಿಗೂ
ಎಂದೆಂದು
ಎಂದೆಂದೂ
ಎಂದೆ-ನ್ನಿಸಿ
ಎಂದೆ-ನ್ನಿ-ಸು-ತ್ತದೆ
ಎಂದೆಲ್ಲ
ಎಂದೇ
ಎಂದೇಕೆ
ಎಂದೊಮ್ಮೆ
ಎಂಬ
ಎಂಬಂ
ಎಂಬಂ-ತಹ
ಎಂಬಂ-ತಿತ್ತು
ಎಂಬಂ-ತಿದೆ
ಎಂಬಂ-ತಿ-ದ್ದರು
ಎಂಬಂ-ತಿ-ದ್ದು-ಬಿ-ಟ್ಟರು
ಎಂಬಂತೆ
ಎಂಬಂ-ತೆಯೇ
ಎಂಬ-ತ್ತ-ನಾಲ್ಕು
ಎಂಬ-ತ್ತಾರು
ಎಂಬತ್ತು
ಎಂಬಲ್ಲಿ
ಎಂಬ-ಲ್ಲಿಗೆ
ಎಂಬ-ಲ್ಲಿನ
ಎಂಬ-ಲ್ಲಿ-ನ-ವ-ರೆಗೂ
ಎಂಬ-ಲ್ಲಿ-ಯ-ವ-ರೆಗೆ
ಎಂಬ-ಲ್ಲಿಯೇ
ಎಂಬವ
ಎಂಬ-ವನ
ಎಂಬ-ವನು
ಎಂಬ-ವರ
ಎಂಬ-ವ-ರಿಂದ
ಎಂಬ-ವ-ರಿಗೆ
ಎಂಬ-ವರು
ಎಂಬ-ವಳ
ಎಂಬ-ವಳು
ಎಂಬವು
ಎಂಬಾಕೆ
ಎಂಬಾತ
ಎಂಬಿ-ತ್ಯಾದಿ
ಎಂಬಿ-ತ್ಯಾ-ದಿ-ಯಾಗಿ
ಎಂಬಿ-ಬ್ಬ-ರಿಗೆ
ಎಂಬೀ
ಎಂಬು
ಎಂಬುದ
ಎಂಬು-ದಕ್ಕೆ
ಎಂಬು-ದ-ನ್ನ-ರಿ-ಯ-ಬೇಕು
ಎಂಬು-ದ-ನ್ನಾ-ಗಲಿ
ಎಂಬು-ದ-ನ್ನಿಲ್ಲಿ
ಎಂಬು-ದನ್ನು
ಎಂಬು-ದ-ನ್ನು-ನೋ-ಡಿ-ಕೊಂಡು
ಎಂಬು-ದನ್ನೂ
ಎಂಬು-ದ-ನ್ನೆಲ್ಲ
ಎಂಬು-ದರ
ಎಂಬು-ದ-ರಲ್ಲಿ
ಎಂಬು-ದಷ್ಟೇ
ಎಂಬು-ದಾ-ಗಲಿ
ಎಂಬುದು
ಎಂಬುದೂ
ಎಂಬುದೆ
ಎಂಬುದೇ
ಎಂಬು-ದೇನೋ
ಎಂಬು-ದೊಂದು
ಎಂಬು-ದೊಂದೇ
ಎಂಬು-ವ-ನನ್ನು
ಎಂಬು-ವನು
ಎಂಬು-ವರ
ಎಂಬು-ವರು
ಎಂಬು-ವ-ರೊ-ಬ್ಬರು
ಎಂಬು-ವಳು
ಎಂಬು-ವು-ದಲ್ಲ
ಎಂಬೆಲ್ಲ
ಎಂಬೊಬ್ಬ
ಎಂಬೊ-ಬ್ಬರು
ಎಂಬೊ-ಬ್ಬಳು
ಎಕರೆ
ಎಕ-ರೆಯ
ಎಕ-ರೆ-ಯಷ್ಟು
ಎಕ್ಸ್ಪ-್ರೆಸ್
ಎಗ್ಬ-ರ್ಟ್
ಎಗ್ಮೋರ್
ಎಗ್ಸಿ-ಕ್ಯೂ-ಟಿವ್
ಎಚ್ಚತ್ತ
ಎಚ್ಚ-ತ್ತಿ-ದ್ದ-ರೆಂ-ದರೆ
ಎಚ್ಚತ್ತು
ಎಚ್ಚರ
ಎಚ್ಚ-ರ-ಗೊಂಡು
ಎಚ್ಚ-ರ-ಗೊ-ಳಿ-ಸ-ಬೇಕು
ಎಚ್ಚ-ರ-ಗೊ-ಳಿಸಿ
ಎಚ್ಚ-ರ-ಗೊ-ಳ್ಳ-ಲಾ-ರದು
ಎಚ್ಚ-ರ-ಗೊಳ್ಳಿ
ಎಚ್ಚ-ರ-ಗೊ-ಳ್ಳು-ತ್ತಿದೆ
ಎಚ್ಚ-ರ-ದಿಂದ
ಎಚ್ಚ-ರ-ವ-ಹಿ-ಸಿ-ದರೂ
ಎಚ್ಚ-ರಿಕೆ
ಎಚ್ಚ-ರಿ-ಕೆ-ಗಾಗಿ
ಎಚ್ಚ-ರಿ-ಕೆ-ಯ-ನ್ನಿ-ತ್ತಿ-ದ್ದರು
ಎಚ್ಚ-ರಿ-ಕೆ-ಯನ್ನೂ
ಎಚ್ಚ-ರಿ-ಕೆ-ಯಿಂದ
ಎಚ್ಚ-ರಿ-ಸಲು
ಎಚ್ಚ-ರಿಸಿ
ಎಚ್ಚ-ರಿ-ಸಿ-ದರು
ಎಚ್ಚ-ರಿ-ಸಿ-ದರೂ
ಎಚ್ಚ-ರಿ-ಸುವ
ಎಚ್ಚ-ರಿ-ಸು-ವ-ವರು
ಎಟ್ನಾ
ಎಡ-ಗ-ಣ್ಣಿ-ನಲ್ಲಿ
ಎಡ-ಗಣ್ಣು
ಎಡ-ಗೈಗೆ
ಎಡ-ಗೈ-ಯಿಂದ
ಎಡಿತ್
ಎಡೆ-ಎ-ಡೆ-ಯೊಳು
ಎಡೆ-ಬಿ-ಡದ
ಎಡೆ-ಬಿ-ಡದೆ
ಎಡೆ-ಯಿ-ರ-ಲಿಲ್ಲ
ಎಡ್ಗರ್
ಎಡ್ವ-ರ್ಡ್ಸ್
ಎಡ್ವ-ರ್ಡ್ಸ-್ಳೊಂ-ದಿ-ಗಿ-ದ್ದಳು
ಎಣಿ-ಕೆಗೆ
ಎಣಿಸಿ
ಎಣಿ-ಸಿ-ದವ
ಎಣಿ-ಸುತ್ತ
ಎಣೆಯೇ
ಎತ್ತ
ಎತ್ತ-ದಿ-ದ್ದರೆ
ಎತ್ತರ
ಎತ್ತ-ರ-ಬಿ-ತ್ತ-ರದ
ಎತ್ತ-ರಕ್ಕೆ
ಎತ್ತ-ರದ
ಎತ್ತ-ರ-ದ-ಲ್ಲಿ-ರ-ಬೇಕು
ಎತ್ತ-ರ-ದ-ಲ್ಲಿ-ರುವ
ಎತ್ತ-ರ-ದಿಂದ
ಎತ್ತ-ರ-ವಿದೆ
ಎತ್ತ-ರ-ವಿ-ರ-ಬ-ಹುದೆ
ಎತ್ತಲು
ಎತ್ತಿ
ಎತ್ತಿ-ಕೊ-ಡು-ವ-ವರು
ಎತ್ತಿ-ತೋ-ರಿ-ಸಿ-ದರು
ಎತ್ತಿ-ತೋ-ರಿಸು
ಎತ್ತಿ-ತೋ-ರಿ-ಸು-ತ್ತಾ-ರೆ-ಅದು
ಎತ್ತಿದ
ಎತ್ತಿ-ದ್ದೀಯೆ
ಎತ್ತಿನ
ಎತ್ತಿ-ಹಾ-ಕಲು
ಎತ್ತಿ-ಹಾ-ಕಿ-ದ್ದಾಳೆ
ಎತ್ತಿ-ಹಿಡಿ
ಎತ್ತಿ-ಹಿ-ಡಿ-ದರು
ಎತ್ತಿ-ಹಿ-ಡಿ-ದಾಗ
ಎತ್ತಿ-ಹಿ-ಡಿದು
ಎತ್ತಿ-ಹಿ-ಡಿ-ಯಿರಿ
ಎತ್ತಿ-ಹಿ-ಡಿ-ಯುತ್ತ
ಎತ್ತಿ-ಹಿ-ಡಿ-ಯುವ
ಎತ್ತಿ-ಹಿ-ಡಿ-ಯು-ವು-ದ-ರಿಂ-ದಾ-ಗಲಿ
ಎತ್ತು-ವುದು
ಎಥಿ-ಕಲ್
ಎದು-ರಾಗಿ
ಎದು-ರಾ-ಗುವ
ಎದು-ರಾ-ಯಿತು
ಎದು-ರಿ-ನಲ್ಲೆ
ಎದು-ರಿಸ
ಎದು-ರಿ-ಸ-ಬಲ್ಲ
ಎದು-ರಿ-ಸ-ಬ-ಲ್ಲ-ವ-ರಾರು
ಎದು-ರಿ-ಸ-ಬೇ-ಕಾ-ಗ-ಬ-ಹು-ದಾದ
ಎದು-ರಿ-ಸ-ಬೇ-ಕಾಗಿ
ಎದು-ರಿ-ಸ-ಬೇ-ಕಾ-ಗು-ತ್ತದೆ
ಎದು-ರಿ-ಸ-ಬೇ-ಕಾ-ಗುವ
ಎದು-ರಿ-ಸ-ಬೇ-ಕಾದ
ಎದು-ರಿ-ಸ-ಬೇ-ಕಾ-ಯಿತು
ಎದು-ರಿ-ಸ-ಬೇಕು
ಎದು-ರಿ-ಸಲು
ಎದು-ರಿ-ಸ-ಲೇ-ಬೇಕು
ಎದು-ರಿಸಿ
ಎದು-ರಿ-ಸಿದ
ಎದು-ರಿ-ಸಿದ್ದ
ಎದು-ರಿಸು
ಎದು-ರಿ-ಸು-ತ್ತಿದ್ದ
ಎದುರು
ಎದು-ರು-ಗೊಂ-ಡರು
ಎದು-ರು-ಗೊ-ಳ್ಳಲು
ಎದು-ರು-ನೋ-ಡು-ತ್ತಿದೆ
ಎದು-ರು-ಹಾ-ಕಿ-ಕೊ-ಳ್ಳು-ತ್ತೀ-ರೇನೊ
ಎದುರ್
ಎದು-ರ್ಗೊಂ-ಡದ್ದೂ
ಎದು-ರ್ಗೊಂಡು
ಎದು-ರ್ಗೊ-ಳ್ಳ-ಲಾ-ಯಿತು
ಎದು-ರ್ಗೊ-ಳ್ಳಲು
ಎದು-ರ್ಗೊ-ಳ್ಳ-ಲೆಂದು
ಎದೆ
ಎದೆ-ಕೆಚ್ಚು
ಎದೆ-ಗಾ-ರಿ-ಕೆ-ಇವು
ಎದೆ-ಗಾ-ರಿ-ಕೆಯ
ಎದೆಗೂ
ಎದೆಯ
ಎದೆ-ಯನ್ನು
ಎದೆ-ಯಲ್ಲಿ
ಎದೆ-ಯೊಡ್ಡಿ
ಎದ್ದರು
ಎದ್ದಿತು
ಎದ್ದಿತ್ತು
ಎದ್ದಿ-ರ-ಬಹು
ಎದ್ದು
ಎದ್ದು-ಕಂ-ಡಿತು
ಎದ್ದು-ಕಾಣು
ಎದ್ದು-ಕಾ-ಣು-ತ್ತದೆ
ಎದ್ದು-ಕಾ-ಣು-ತ್ತಿತ್ತು
ಎದ್ದು-ಕಾ-ಣು-ವಂ-ತಿದ್ದ
ಎದ್ದು-ನಿಂ-ತರು
ಎದ್ದು-ನಿಂ-ತ-ರೆಂ-ದರೆ
ಎದ್ದು-ನಿಂ-ತಾಗ
ಎದ್ದು-ನಿಂತು
ಎದ್ದು-ನಿಂ-ತೊ-ಡನೆ
ಎದ್ದು-ನಿ-ಲ್ಲು-ತ್ತಿ-ದ್ದ-ವರೂ
ಎದ್ದು-ನಿ-ಲ್ಲು-ವುದು
ಎದ್ದು-ಬಿ-ಟ್ಟಿತು
ಎದ್ದೇಳಿ
ಎದ್ದೇಳು
ಎನಿತೊ
ಎನಿ-ಸಿ-ಕೊಂಡ
ಎನ್
ಎನ್ನ
ಎನ್ನ-ಬ-ಹು-ದಾ-ದಂ-ತಹ
ಎನ್ನ-ಬ-ಹುದು
ಎನ್ನ-ಬ-ಹು-ದು-ಒಂದು
ಎನ್ನ-ಬೇ-ಕಾ-ಗು-ತ್ತದೆ
ಎನ್ನ-ಬೇಕು
ಎನ್ನಲು
ಎನ್ನಿ
ಎನ್ನಿ-ಸ-ಬ-ಹುದು
ಎನ್ನಿ-ಸಿ-ಕೊಂ-ಡರು
ಎನ್ನಿ-ಸಿ-ಕೊಂ-ಡವ
ಎನ್ನಿ-ಸಿ-ಕೊಂ-ಡ-ವರ
ಎನ್ನಿ-ಸಿ-ಕೊ-ಳ್ಳ-ಬೇ-ಕಾ-ದರೆ
ಎನ್ನಿ-ಸಿ-ಕೊ-ಳ್ಳ-ಲಿ-ಲ್ಲವೆ
ಎನ್ನಿ-ಸಿತು
ಎನ್ನಿ-ಸಿ-ದರೆ
ಎನ್ನಿ-ಸಿ-ಬಿ-ಟ್ಟಿತ್ತು
ಎನ್ನಿ-ಸು-ತ್ತಿದೆ
ಎನ್ನು
ಎನ್ನುತ್ತ
ಎನ್ನು-ತ್ತದೆ
ಎನ್ನು-ತ್ತಲೋ
ಎನ್ನು-ತ್ತವೆ
ಎನ್ನುತ್ತಾ
ಎನ್ನು-ತ್ತಾನೆ
ಎನ್ನು-ತ್ತಾರೆ
ಎನ್ನು-ತ್ತಾರೋ
ಎನ್ನು-ತ್ತಾಳೆ
ಎನ್ನು-ತ್ತಿದ್ದ
ಎನ್ನು-ತ್ತಿ-ದ್ದರು
ಎನ್ನು-ತ್ತಿ-ದ್ದಳು
ಎನ್ನು-ತ್ತೀಯಾ
ಎನ್ನು-ತ್ತೀರಿ
ಎನ್ನು-ತ್ತೇನೆ
ಎನ್ನುವ
ಎನ್ನು-ವಂತೆ
ಎನ್ನು-ವಂ-ಥ-ದ್ದಾ-ಗಿ-ರ-ಲಿಲ್ಲ
ಎನ್ನು-ವ-ವ-ನೇ-ನಾ-ದರೂ
ಎನ್ನು-ವ-ವ-ರೆಗೂ
ಎನ್ನು-ವ-ವಳು
ಎನ್ನು-ವ-ಷ್ಟ-ರಲ್ಲೇ
ಎನ್ನು-ವಾಗ
ಎನ್ನು-ವಿಯೊ
ಎನ್ನು-ವಿ-ರಲ್ಲ
ಎನ್ನು-ವು-ದಕ್ಕೆ
ಎನ್ನು-ವುದನ್ನು
ಎನ್ನು-ವು-ದರ
ಎನ್ನು-ವು-ದಾ-ಗಲಿ
ಎನ್ನು-ವು-ದಾ-ದರೆ
ಎನ್ನು-ವುದು
ಎನ್ನು-ವುದೂ
ಎನ್ನು-ವು-ದೆಲ್ಲ
ಎನ್ನು-ವುದೇ
ಎನ್ನು-ವೆ-ಯಲ್ಲ
ಎನ್ಸೈ-ಕ್ಲೋ-ಪೀ-ಡಿಯ
ಎಪ್ಪತ್ತು
ಎಬ್ಬಿ-ಸ-ಲಿ-ದ್ದೇನೆ
ಎಬ್ಬಿಸಿ
ಎಬ್ಬಿ-ಸಿತು
ಎಬ್ಬಿ-ಸಿ-ಬಿ-ಟ್ಟರು
ಎಮಿಲಿ
ಎಮ್
ಎಮ್ಮಾ
ಎರ-ಗಿತು
ಎರ-ಗಿ-ದಳು
ಎರ-ಡಕ್ಕೂ
ಎರ-ಡ-ನೆಯ
ಎರ-ಡ-ನೆ-ಯ-ದಾಗಿ
ಎರ-ಡ-ನೆ-ಯದು
ಎರ-ಡ-ನೆ-ಯ-ವ-ಳಾದ
ಎರ-ಡನೇ
ಎರ-ಡನ್ನೂ
ಎರ-ಡರ
ಎರ-ಡ-ರಂದು
ಎರಡು
ಎರ-ಡು-ಮೂರು
ಎರ-ಡು-ಮೂರು
ಎರ-ಡು-ಸಂ-ಶಯ
ಎರಡೂ
ಎರ-ಡೂ-ವರೆ
ಎರ-ಡೆ-ರಡು
ಎರಡೇ
ಎರ-ವಲು
ಎರಿಕ್
ಎರೆ-ದ-ರೆಂ-ಬುದು
ಎಲಿ-ಫೆಂ-ಟ್ಪಾಸ್
ಎಲಿ-ಫೆಂ-ಟ್ಪಾ-ಸ್ನ-ವ-ರೆಗೂ
ಎಲೆ
ಎಲೆ-ಕ್ಟ್ರಿಕ್
ಎಲೆ-ಗಳ
ಎಲೆನ್
ಎಲೆ-ಯನ್ನು
ಎಲೆಲಾ
ಎಲ್
ಎಲ್ಲ
ಎಲ್ಲ-ಕ್ಕಿಂತ
ಎಲ್ಲ-ದರ
ಎಲ್ಲ-ದ-ರಲ್ಲೂ
ಎಲ್ಲ-ದ-ರಿಂ-ದಲೂ
ಎಲ್ಲರ
ಎಲ್ಲ-ರಂ-ತೆಯೇ
ಎಲ್ಲ-ರನ್ನೂ
ಎಲ್ಲ-ರಲ್ಲೂ
ಎಲ್ಲರಿ
ಎಲ್ಲ-ರಿಂ-ದಲೂ
ಎಲ್ಲ-ರಿ-ಗಿಂತ
ಎಲ್ಲ-ರಿಗೂ
ಎಲ್ಲರೂ
ಎಲ್ಲ-ರೆ-ಡೆಗೂ
ಎಲ್ಲ-ರೆ-ದು-ರಿಗೆ
ಎಲ್ಲ-ರೆ-ದು-ರಿ-ನಲ್ಲಿ
ಎಲ್ಲ-ರೊಂ-ದಿಗೂ
ಎಲ್ಲ-ರೊಂ-ದಿಗೆ
ಎಲ್ಲ-ರೊ-ಳ-ಗೊಂ-ದಾ-ಗಿ-ಬಿ-ಡು-ತ್ತಿ-ದ್ದರು
ಎಲ್ಲ-ವನು
ಎಲ್ಲ-ವನ್ನು
ಎಲ್ಲ-ವನ್ನೂ
ಎಲ್ಲವೂ
ಎಲ್ಲಾ
ಎಲ್ಲಾ-ದರೂ
ಎಲ್ಲಿ
ಎಲ್ಲಿಂದ
ಎಲ್ಲಿಂ-ದಲೋ
ಎಲ್ಲಿಗೆ
ಎಲ್ಲಿ-ಡು-ವುದು
ಎಲ್ಲಿದೆ
ಎಲ್ಲಿ-ದ್ದಾರೆ
ಎಲ್ಲಿಯ
ಎಲ್ಲಿ-ಯ-ವ-ರೆಗೆ
ಎಲ್ಲಿ-ಯ-ವರೆ-ಗೆಂ-ದರೆ
ಎಲ್ಲಿ-ಯಾ-ದರೂ
ಎಲ್ಲಿಯೂ
ಎಲ್ಲಿಯೋ
ಎಲ್ಲಿ-ಲ್ಲದ
ಎಲ್ಲೆ
ಎಲ್ಲೆಂ-ದ-ರಲ್ಲಿ
ಎಲ್ಲೆಂ-ದರೆ
ಎಲ್ಲೆ-ಡೆ-ಗಳಲ್ಲಿ
ಎಲ್ಲೆ-ಡೆ-ಗ-ಳಲ್ಲೂ
ಎಲ್ಲೆ-ಡೆ-ಗಳಿಂದ
ಎಲ್ಲೆ-ಡೆ-ಗ-ಳಿಂ-ದಲೂ
ಎಲ್ಲೆ-ಡೆ-ಗ-ಳಿಗೂ
ಎಲ್ಲೆ-ಡೆ-ಯಲ್ಲೂ
ಎಲ್ಲೆ-ಡೆ-ಯಿಂದ
ಎಲ್ಲೆ-ಡೆಯೂ
ಎಲ್ಲೆಲ್ಲಿ
ಎಲ್ಲೆ-ಲ್ಲಿಯೂ
ಎಲ್ಲೆಲ್ಲೂ
ಎಲ್ಲೋ
ಎಳ-ನೀ-ರನ್ನು
ಎಳ-ನೀ-ರನ್ನೇ
ಎಳ-ನೀರು
ಎಳ-ನೀ-ರು-ಗಳನ್ನು
ಎಳ-ನೀ-ರು-ಗಳು
ಎಳೆ-ತಂ-ದರು
ಎಳೆದ
ಎಳೆ-ದತ್ತ
ಎಳೆ-ದರು
ಎಳೆ-ದರೂ
ಎಳೆದು
ಎಳೆ-ದುವು
ಎಳೆ-ದೊ-ಯ್ದಿವೆ
ಎಳೆ-ಯ-ರನ್ನು
ಎಳೆ-ಯ-ರಿ-ಗಾಗಿ
ಎಳೆ-ಯು-ತ್ತಿರು
ಎಳೆ-ಯು-ವು-ದಿ-ಲ್ಲವೆ
ಎಳ್ಳಷ್ಟೂ
ಎವ-ರೆಟ್
ಎಷ್ಟರ
ಎಷ್ಟ-ರ-ಮ-ಟ್ಟಿಗೆ
ಎಷ್ಟ-ರ-ಮ-ಟ್ಟಿ-ಗೆಂ-ದರೆ
ಎಷ್ಟಾ-ದರೂ
ಎಷ್ಟಾ-ಯಿತು
ಎಷ್ಟಿದೆ
ಎಷ್ಟು
ಎಷ್ಟೆಂ-ದರೆ
ಎಷ್ಟೆಷ್ಟು
ಎಷ್ಟೇ
ಎಷ್ಟೊಂ-ದನ್ನು
ಎಷ್ಟೊಂದು
ಎಷ್ಟೋ
ಎಷ್ಟೋ-ವೇಳೆ
ಎಸೆತ
ಎಸೆದ
ಎಸೆ-ದಂತೆ
ಎಸೆ-ದರು
ಎಸೆ-ದರೂ
ಎಸೆ-ದಿ-ದ್ದರು
ಎಸೆದು
ಎಸೆ-ದು-ಬಿಟ್ಟ
ಎಸೆ-ದು-ಬಿ-ಟ್ಟಿ-ದ್ದರು
ಎಸೆ-ದು-ಬಿ-ಡ-ಬೇಕು
ಎಸೆ-ದು-ಬಿ-ಡು-ತ್ತೇನೆ
ಎಸೆದೇ
ಎಸೆ-ಯಲೂ
ಎಸೆ-ಯಿರಿ
ಎಸೆ-ಯು-ತ್ತಾ-ರಂ-ತಲ್ಲ
ಎಸೆ-ಯು-ತ್ತೀ-ರಂತೆ
ಎಸೆ-ಯುವ
ಎಸೆ-ಯು-ವುದು
ಎಸ್
ಎಸ್-ಎಸ್
ಎಸ್ಪ್ಲ-ನೇಡ್
ಏ
ಏಂಜ-ಲಿಸ್
ಏಂಜ-ಲಿ-ಸ್ನಲ್ಲಿ
ಏಂಜೆ-ಲಿಸ್
ಏಂಜೆ-ಲಿಸ್ಗೆ
ಏಂಜೆ-ಲಿ-ಸ್ನಲ್ಲಿ
ಏಕಂ
ಏಕ-ಕಂ-ಠ-ದಿಂದ
ಏಕ-ಕಾ-ಲ-ದಲ್ಲಿ
ಏಕತೆ
ಏಕ-ತೆಯ
ಏಕ-ತೆ-ಯನ್ನು
ಏಕ-ತೆ-ಯಿದೆ
ಏಕ-ಪ್ರ-ಕಾ-ರ-ವಾಗಿ
ಏಕ-ಭಾ-ವ-ದಿಂದ
ಏಕ-ಮಾತ್ರ
ಏಕ-ಮೇ-ವಾ-ದ್ವಿ-ತೀ-ಯಳು
ಏಕ-ಸೂ-ತ್ರತೆ
ಏಕ-ಸೂ-ತ್ರ-ತೆ-ಯನ್ನು
ಏಕ-ಸೂ-ತ್ರ-ತೆ-ಯ-ನ್ನು-ಅವು
ಏಕಾಂಗಿ
ಏಕಾಂ-ಗಿ-ಗ-ಳಾಗಿ
ಏಕಾಂ-ಗಿ-ಯಾಗಿ
ಏಕಾಂ-ಗಿ-ಯಾ-ಗಿಯೇ
ಏಕಾಂ-ಗಿ-ಯಾ-ಗಿ-ರದೆ
ಏಕಾಂ-ಗಿ-ಯಾ-ಗಿ-ರಲು
ಏಕಾಂತ
ಏಕಾಂ-ತ-ಪ್ರ-ಶಾಂತ
ಏಕಾಂ-ತಕ್ಕೆ
ಏಕಾಂ-ತ-ಜೀ-ವನ
ಏಕಾಂ-ತ-ಜೀ-ವ-ನದ
ಏಕಾಂ-ತದ
ಏಕಾಂ-ತ-ದಲ್ಲಿ
ಏಕಾಂ-ತ-ದ-ಲ್ಲಿದ್ದು
ಏಕಾಂ-ತ-ದ-ಲ್ಲಿ-ದ್ದು-ಕೊಂಡು
ಏಕಾಂ-ತ-ದ-ಲ್ಲಿ-ರಲು
ಏಕಾಂ-ತ-ದ-ಲ್ಲಿ-ರುತ್ತ
ಏಕಾಂ-ತ-ದ-ಲ್ಲಿ-ರು-ವುದು
ಏಕಾಂ-ತ-ದಿಂ-ದಿ-ರುವ
ಏಕಾಂ-ತ-ದೆ-ಡೆಗೆ
ಏಕಾಂ-ತ-ವಾಸ
ಏಕಾಂ-ತ-ವಾ-ಸದ
ಏಕಾ-ಗ್ರ-ಗೊ-ಳಿ-ಸಲು
ಏಕಾ-ಗ್ರ-ಗೊ-ಳಿ-ಸಿದೆ
ಏಕಾ-ಗ್ರತೆ
ಏಕಾ-ಗ್ರ-ತೆ-ಗಳನ್ನು
ಏಕಾ-ಗ್ರ-ತೆ-ಯಿಂದ
ಏಕಾ-ಗ್ರ-ತೆ-ಯಿಂ-ದಲೋ
ಏಕಾ-ಗ್ರ-ತೆಯೇ
ಏಕಾ-ಗ್ರ-ತೆ-ಯೊಂ-ದಿ-ದ್ದು-ಬಿ-ಟ್ಟರೆ
ಏಕಾ-ತ್ಮ-ವಾ-ದ-ವನ್ನು
ಏಕಾ-ದಶಿ
ಏಕಿ-ರ-ಬಹು
ಏಕಿ-ರ-ಬ-ಹುದು
ಏಕೀ-ಕ-ರ-ಣವು
ಏಕೆ
ಏಕೆಂ
ಏಕೆಂ-ದರೆ
ಏಕೈಕ
ಏಟು-ಗಳನ್ನು
ಏಡನ್
ಏಡನ್ಗೆ
ಏಡ-ನ್ನನ್ನು
ಏಡ-ನ್ನಿ-ನ-ವರೆ-ಗಿನ
ಏನಂ
ಏನದು
ಏನನ್ನಾ
ಏನ-ನ್ನಾ-ದರೂ
ಏನ-ನ್ನಿ-ಸಿತೋ
ಏನನ್ನು
ಏನನ್ನೂ
ಏನನ್ನೇ
ಏನನ್ನೋ
ಏನಪ್ಪ
ಏನಮ್ಮ
ಏನಾ-ಗ-ಬ-ಹು-ದೆಂ-ಬು-ದನ್ನು
ಏನಾ-ಗ-ಬೇಕು
ಏನಾ-ಗಿ-ದೆ-ಯೆಂ-ದರೆ
ಏನಾ-ಗಿ-ದ್ದೇನೆ
ಏನಾ-ಗಿ-ದ್ದೇವೋ
ಏನಾ-ಗಿ-ರು-ತ್ತಿ-ದ್ದಿರೋ
ಏನಾ-ಗು-ತ್ತದೆ
ಏನಾ-ದರೂ
ಏನಾ-ದ-ರೇ-ನಂತೆ
ಏನಾ-ದ-ರೊಂ-ದನ್ನು
ಏನಾ-ದ-ರೊಂದು
ಏನಾ-ಶ್ಚರ್ಯ
ಏನಿತ್ತೋ
ಏನಿ-ದ-ಕ್ಕೆಲ್ಲ
ಏನಿದು
ಏನಿದೆ
ಏನಿ-ದೆ-ಯೆಂ-ಬುದು
ಏನಿ-ದೆಯೊ
ಏನಿ-ದೆಲ್ಲ
ಏನಿ-ದ್ದರೂ
ಏನಿರ
ಏನಿ-ರ-ಬ-ಹುದು
ಏನಿ-ರ-ಲಿಲ್ಲ
ಏನು
ಏನೂ
ಏನೆಂ-ದರೆ
ಏನೆಂ-ದ-ರೆಈ
ಏನೆಂ-ದ-ರೆ-ಪ್ರತಿ
ಏನೆಂದು
ಏನೆಂದೆ
ಏನೆಂ-ಬು-ದನ್ನು
ಏನೆಂ-ಬು-ದನ್ನೂ
ಏನೆ-ನ್ನ-ಬ-ಹುದು
ಏನೆನ್ನಿ
ಏನೇ
ಏನೇ-ನನ್ನು
ಏನೇ-ನಾ-ಯಿ-ತೆಂ-ಬುದು
ಏನೇನು
ಏನೇನೂ
ಏನೇನೋ
ಏನೋ
ಏಪ್ರಿ-ಲಿನ
ಏಪ್ರಿಲ್
ಏರ-ಬ-ಲ್ಲು-ದಾ-ದ್ದ-ರಿಂದ
ಏರ-ಬೇ-ಕಾ-ಗಿ-ದೆಯೋ
ಏರಿ
ಏರಿತು
ಏರಿದ
ಏರಿ-ದರು
ಏರಿ-ದ್ದೇನು
ಏರಿ-ಸ-ಬ-ಲ್ಲ-ವ-ರಾಗಿ
ಏರಿ-ಸ-ಬೇಕು
ಏರಿ-ಸಲು
ಏರಿಸಿ
ಏರಿ-ಸಿ-ದರೆ
ಏರಿ-ಸುವ
ಏರಿ-ಸು-ವುದೇ
ಏರು-ತ್ತಾನೆ
ಏರು-ತ್ತಿದೆ
ಏರು-ಪೇ-ರಾ-ಗಿ-ಬಿ-ಡು-ತ್ತದೆ
ಏರು-ಪೇ-ರಾ-ಯಿತು
ಏರ್ಪಟ್ಟಿ
ಏರ್ಪ-ಡಿಸ
ಏರ್ಪ-ಡಿ-ಸ-ಬೇ-ಕೆಂದು
ಏರ್ಪ-ಡಿ-ಸಲಾ
ಏರ್ಪ-ಡಿ-ಸ-ಲಾ-ಗಿತ್ತು
ಏರ್ಪ-ಡಿ-ಸ-ಲಾ-ಯಿತು
ಏರ್ಪ-ಡಿ-ಸಲು
ಏರ್ಪ-ಡಿಸಿ
ಏರ್ಪ-ಡಿ-ಸಿದ್ದ
ಏರ್ಪ-ಡಿ-ಸಿ-ದ್ದರು
ಏರ್ಪ-ಡಿ-ಸಿ-ದ್ದಳು
ಏರ್ಪ-ಡಿ-ಸು-ತ್ತಿ-ದ್ದರು
ಏರ್ಪ-ಡಿ-ಸುವ
ಏರ್ಪಾ-ಡಾ-ಗಿತ್ತು
ಏರ್ಪಾ-ಡಾ-ಗಿದೆ
ಏರ್ಪಾ-ಡಾ-ದುವು
ಏರ್ಪಾಡು
ಏರ್ಪಾ-ಡು-ಗಳನ್ನೂ
ಏರ್ಪಾ-ಡು-ಗಳನ್ನೆಲ್ಲ
ಏರ್ಪಾ-ಡು-ಗ-ಳಾ-ಗಿ-ದ್ದುವು
ಏಲ-ಕ್ಕಿ-ಮೆ-ಣ-ಸು-ತೆಂ-ಗು-ಬಾಳೆ
ಏಳನೇ
ಏಳ-ರಂದು
ಏಳ-ಲೇ-ಬೇ-ಕೆಂಬ
ಏಳಿ
ಏಳಿ-ಗೆಗೆ
ಏಳು
ಏಳು-ತ್ತಿದ್ದ
ಏಳು-ತ್ತಿ-ದ್ದರು
ಏಳು-ತ್ತಿ-ದ್ದು-ದ-ರಿಂದ
ಏಳು-ತ್ತಿ-ರು-ವುದನ್ನು
ಏಳು-ತ್ತೇನೆ
ಏಳು-ನೂ-ರಕ್ಕೂ
ಏಳುವ
ಏಳು-ವು-ದುಂಟು
ಏಳು-ಸಾ-ವಿರ
ಏಳೂ-ವರೆ
ಏಳ್ಗೆ-ಗಾಗಿ
ಏಳ್ನೂರು
ಏಷಿಯಾ
ಏಷಿ-ಯಾ-ದ-ವನೊ
ಏಸು-ಕ್ರಿಸ್ತ
ಏಸು-ಕ್ರಿ-ಸ್ತನ
ಏಸು-ಕ್ರಿ-ಸ್ತ-ನನ್ನು
ಏಸು-ಕ್ರಿ-ಸ್ತನು
ಏಸು-ವಾ-ಗಿದ್ದೆ
ಐ
ಐಕ್ಯ-ಇ-ವು-ಗಳನ್ನು
ಐತಿ-ಹಾ-ಸಿಕ
ಐತಿ-ಹಾ-ಸಿ-ಕ-ತೆಯ
ಐತಿ-ಹ್ಯ-ಗಳು
ಐದ-ರಂದು
ಐದಾ-ಯಿತು
ಐದಾರು
ಐದು
ಐದು-ನೂರು
ಐನೂರು
ಐರಿಶ್
ಐರೋಪ್ಯ
ಐರೋ-ಪ್ಯರ
ಐರೋ-ಪ್ಯ-ರಂ-ತೆಯೇ
ಐರೋ-ಪ್ಯರು
ಐರೋ-ಪ್ಯರೂ
ಐರೋ-ಪ್ಯ-ರೊಂ-ದಿಗೆ
ಐರ್ಲೆಂ-ಡಿನ
ಐವ-ತ್ತಕ್ಕೂ
ಐವತ್ತು
ಐವ-ತ್ತು-ಮೈಲಿ
ಐವರು
ಐಶ್ವ-ರ್ಯದ
ಐಷಾ-ರಾ-ಮಿ-ಗಳು
ಐಹಿಕ
ಒ
ಒಂಟಿ-ಯ-ನ್ನಿ-ಟ್ಟರು
ಒಂಟೆ
ಒಂಟೆ-ಗಳು
ಒಂದಂಶ
ಒಂದ-ಕ್ಕಿಂತ
ಒಂದಕ್ಕೆ
ಒಂದ-ಕ್ಕೊಂದು
ಒಂದನ್ನು
ಒಂದ-ನ್ನೇ-ನೀವು
ಒಂದ-ನ್ನೊಂದು
ಒಂದ-ರಲ್ಲಿ
ಒಂದ-ರಿಂದ
ಒಂದಲ್ಲ
ಒಂದಷ್ಟು
ಒಂದಾ
ಒಂದಾಗಿ
ಒಂದಾ-ಗಿ-ಬಿ-ಟ್ಟಿತ್ತು
ಒಂದಾ-ಗು-ವು-ದರ
ಒಂದಾದ
ಒಂದಾ-ದ-ಮೇ-ಲೊಂ-ದ-ರಂತೆ
ಒಂದಾ-ದ-ಮೇ-ಲೊಂದು
ಒಂದಾ-ದರೂ
ಒಂದಾ-ನೊಂದು
ಒಂದಿ-ನಿ-ತಾ-ದರೂ
ಒಂದಿ-ನಿತೂ
ಒಂದಿ-ಬ್ಬ-ರನ್ನು
ಒಂದಿ-ಬ್ಬರು
ಒಂದಿ-ಬ್ಬ-ರು-ಮೂ-ವ-ರಿಗೆ
ಒಂದಿ-ರು-ತ್ತದೆ
ಒಂದಿ-ಷ್ಟ-ನ್ನಾ-ದರೂ
ಒಂದಿಷ್ಟು
ಒಂದಿಷ್ಟೂ
ಒಂದು
ಒಂದು-ಒ-ಳಿ-ತಿನ
ಒಂದು-ಗೂಡಿ
ಒಂದು-ಗೂ-ಡಿದ
ಒಂದು-ಗೂ-ಡಿ-ಸಿ-ಬಿಡ
ಒಂದು-ಗೂ-ಡಿ-ಸುವ
ಒಂದೂ-ವರೆ
ಒಂದೆಂ-ದರೆ
ಒಂದೆಡೆ
ಒಂದೆ-ಡೆ-ಯಾ-ದರೆ
ಒಂದೆ-ರಡು
ಒಂದೇ
ಒಂದೈದು
ಒಂದೊಂ-ದಾಗಿ
ಒಂದೊಂದು
ಒಂದೊಂದೂ
ಒಂದೋ
ಒಂಬ-ತ್ತ-ರಂದು
ಒಂಬ-ತ್ತ-ರ-ವ-ರೆಗೆ
ಒಂಬ-ತ್ತಾ-ಯಿತು
ಒಂಬತ್ತು
ಒಕಾ-ಕುರ
ಒಕಾ-ಕು-ರ-ನಿಗೂ
ಒಕಾ-ಕು-ರ-ನಿಗೆ
ಒಕಾ-ಕು-ರನು
ಒಕ್ಕ-ಣಿ-ಸ-ಲಾ-ಗಿತ್ತು
ಒಕ್ಕ-ಣಿ-ಸಿದ
ಒಕ್ಕ-ಣ್ಣ-ನೆಂ-ಬುದು
ಒಕ್ಕೂಟ
ಒಕ್ಕೂ-ಟ-ಗಳ
ಒಕ್ಕೂ-ಟ-ಗಳಲ್ಲಿ
ಒಕ್ಕೂ-ಟ-ಗಳಿಂದ
ಒಗ್ಗಿ-ಸುವ
ಒಗ್ಗೂ-ಡಿ-ಸಲು
ಒಗ್ಗೂ-ಡಿ-ಸಿ-ಕೊಂಡು
ಒಗ್ಗೂ-ಡಿ-ಸುವ
ಒಗ್ಗೂ-ಡಿ-ಸು-ವು-ದ-ರಲ್ಲಿ
ಒಟ್ಟಾಗಿ
ಒಟ್ಟಾ-ಗಿ-ದ್ದಾಗ
ಒಟ್ಟಾರೆ
ಒಟ್ಟಾ-ರೆ-ಯಾಗಿ
ಒಟ್ಟಿಗೆ
ಒಟ್ಟಿ-ನಲ್ಲಿ
ಒಟ್ಟು
ಒಟ್ಟೊ-ಟ್ಟಾಗಿ
ಒಟ್ಟೊ-ಟ್ಟಾ-ಗಿಯೇ
ಒಡಂ-ಬ-ಡಿ-ಕೆಯು
ಒಡಕು
ಒಡ-ಕು-ದ-ನಿ-ಯೆ-ತ್ತಿದ
ಒಡತಿ
ಒಡ-ನಾ-ಡಿ-ಗಳು
ಒಡ-ನಾ-ಡು-ವುದು
ಒಡ-ನೆಯೇ
ಒಡ-ಮೂ-ಡಿ-ದಂ-ಥವು
ಒಡ-ಮೂ-ಡಿ-ದ್ದಾ-ದರೆ
ಒಡ-ಲಲ್ಲಿ
ಒಡಾ
ಒಡೆ-ದಿತ್ತು
ಒಡೆದು
ಒಡೆ-ಯ-ಅ-ದರ
ಒಡ್ಡ-ಲಾ-ಗು-ತ್ತದೆ
ಒಡ್ಡಿ-ದುದು
ಒಡ್ಡು-ತ್ತ-ವೆಯೋ
ಒಡ್ಡೊ-ಡ್ಡಾಗಿ
ಒಣ
ಒಣಗಿ
ಒಣ-ಗಿ-ಸ-ಲಾ-ರದು
ಒಣ-ಗಿ-ಹೋ-ಗಿ-ರು-ವಂತೆ
ಒಣ-ಗಿ-ಹೋ-ಯಿತು
ಒಣ-ವೇ-ದಾಂ-ತಿ-ಗ-ಳಲ್ಲ
ಒಣ-ಹ-ವೆ-ಯಿಂದ
ಒತ್ತ-ಟ್ಟಿ-ಗಿಟ್ಟು
ಒತ್ತಡ
ಒತ್ತ-ಡ-ಕ್ಕ-ನು-ಗು-ಣ-ವಾಗಿ
ಒತ್ತ-ಡಕ್ಕೆ
ಒತ್ತ-ಡ-ಗಳ
ಒತ್ತ-ಡದ
ಒತ್ತ-ಡ-ದಿಂದ
ಒತ್ತ-ಡ-ವನ್ನು
ಒತ್ತ-ಡ-ವಿತ್ತು
ಒತ್ತ-ರಿಸಿ
ಒತ್ತಾಯ
ಒತ್ತಾ-ಯಕ್ಕೆ
ಒತ್ತಾ-ಯದ
ಒತ್ತಾ-ಯ-ದಿಂದ
ಒತ್ತಾ-ಯ-ಪ-ಡಿ-ಸಿ-ದರು
ಒತ್ತಾ-ಯ-ಪ-ಡಿ-ಸು-ತ್ತಿ-ದ್ದ-ರಿಂದ
ಒತ್ತಾ-ಯ-ಪೂ-ರ್ವಕ
ಒತ್ತಾ-ಯ-ಪೂ-ರ್ವ-ಕ-ವಾಗಿ
ಒತ್ತಾ-ಯವೂ
ಒತ್ತಾ-ಯಿ-ಸಲು
ಒತ್ತಾ-ಯಿಸಿ
ಒತ್ತಾ-ಯಿ-ಸಿ-ದರು
ಒತ್ತಾ-ಯಿ-ಸಿ-ದಾಗ
ಒತ್ತಾ-ಯಿ-ಸಿ-ದ್ದ-ರಿಂದ
ಒತ್ತಾ-ಯಿ-ಸು-ತ್ತಿದ್ದ
ಒತ್ತಾ-ಯಿ-ಸು-ತ್ತಿ-ದ್ದರು
ಒತ್ತಾ-ಯಿ-ಸು-ತ್ತಿ-ದ್ದ-ವ-ರನ್ನು
ಒತ್ತಾ-ಯಿ-ಸು-ತ್ತಿ-ರು-ವು-ದ-ರಿಂದ
ಒತ್ತಾ-ಯಿ-ಸುವ
ಒತ್ತಿ
ಒತ್ತಿ-ಒತ್ತಿ
ಒತ್ತಿ-ಕೊಂ-ಡರು
ಒತ್ತಿ-ಕೊಂ-ಡಳು
ಒತ್ತಿ-ಹೇ-ಳಿ-ದರು
ಒತ್ತು
ಒದಗಿ
ಒದ-ಗಿತು
ಒದ-ಗಿ-ಬಂತು
ಒದ-ಗಿ-ಬಂ-ದ-ದ್ದ-ರಿಂದ
ಒದ-ಗಿ-ಬಂ-ದದ್ದು
ಒದ-ಗಿ-ಬಂ-ದಾಗ
ಒದ-ಗಿ-ಬಂ-ದಿತು
ಒದ-ಗಿ-ಬಂ-ದಿತ್ತು
ಒದ-ಗಿ-ಬಂ-ದಿದೆ
ಒದ-ಗಿ-ಬ-ರ-ಲಿ-ಲ್ಲ-ವಲ್ಲ
ಒದ-ಗಿ-ಬ-ರಲೇ
ಒದ-ಗಿ-ಬ-ರು-ತ್ತಿ-ದ್ದರು
ಒದ-ಗಿ-ಸ-ಬ-ಹು-ದಾದ
ಒದ-ಗಿ-ಸ-ಲಾ-ಗು-ವು-ದಿಲ್ಲ
ಒದ-ಗಿ-ಸಿ-ಕೊಡು
ಒದ-ಗಿ-ಸಿವೆ
ಒದ-ಗಿ-ಸುವ
ಒದ-ಗಿ-ಸು-ವಲ್ಲಿ
ಒದ-ರಿದ
ಒದ್ದಾ-ಡಿ-ದರು
ಒದ್ದಾ-ಡಿದೆ
ಒದ್ದೆ
ಒಪ್ಪಂ-ದಕ್ಕೆ
ಒಪ್ಪಂ-ದ-ಪ-ತ್ರದ
ಒಪ್ಪಂ-ದ-ಪ-ತ್ರ-ವನ್ನು
ಒಪ್ಪಂ-ದವು
ಒಪ್ಪ-ದ-ವನು
ಒಪ್ಪ-ದಿ-ದ್ದಾಗ
ಒಪ್ಪ-ದಿ-ದ್ದು-ದ-ರಲ್ಲಿ
ಒಪ್ಪ-ದಿ-ದ್ದುದು
ಒಪ್ಪ-ದಿ-ರ-ಬ-ಹುದು
ಒಪ್ಪದೆ
ಒಪ್ಪ-ಬ-ಹು-ದಾದ
ಒಪ್ಪಲಿ
ಒಪ್ಪ-ಲಿಲ್ಲ
ಒಪ್ಪಲು
ಒಪ್ಪಲೇ
ಒಪ್ಪ-ಲೇ-ಬೇಕು
ಒಪ್ಪಿ
ಒಪ್ಪಿ-ಕೊಂಡ
ಒಪ್ಪಿ-ಕೊಂ-ಡ-ದ್ದಕ್ಕೆ
ಒಪ್ಪಿ-ಕೊಂ-ಡ-ರಾ-ದರೂ
ಒಪ್ಪಿ-ಕೊಂ-ಡರು
ಒಪ್ಪಿ-ಕೊಂ-ಡರೂ
ಒಪ್ಪಿ-ಕೊಂ-ಡಾ-ಗಲೇ
ಒಪ್ಪಿ-ಕೊಂ-ಡಿ-ದ್ದರು
ಒಪ್ಪಿ-ಕೊಂ-ಡಿ-ದ್ದೇ-ನೆಯೋ
ಒಪ್ಪಿ-ಕೊಂ-ಡಿ-ರ-ಬೇ-ಕ-ಲ್ಲವೆ
ಒಪ್ಪಿ-ಕೊಂಡು
ಒಪ್ಪಿ-ಕೊಂ-ಡು-ಬಿ-ಟ್ಟರು
ಒಪ್ಪಿ-ಕೊ-ಳ್ಳ-ಬೇ-ಕಾ-ಗು-ತ್ತದೆ
ಒಪ್ಪಿ-ಕೊ-ಳ್ಳ-ಬೇಕು
ಒಪ್ಪಿ-ಕೊ-ಳ್ಳ-ಲಾ-ಗು-ತ್ತಿದ್ದ
ಒಪ್ಪಿ-ಕೊ-ಳ್ಳ-ಲಾ-ರಂ-ಭಿ-ಸಿ-ದರು
ಒಪ್ಪಿ-ಕೊ-ಳ್ಳ-ಲಿಲ್ಲ
ಒಪ್ಪಿ-ಕೊಳ್ಳಿ
ಒಪ್ಪಿ-ಕೊಳ್ಳು
ಒಪ್ಪಿ-ಕೊ-ಳ್ಳುವ
ಒಪ್ಪಿ-ಕೊ-ಳ್ಳು-ವಂತೆ
ಒಪ್ಪಿಗೆ
ಒಪ್ಪಿ-ಗೆ-ಯನ್ನು
ಒಪ್ಪಿ-ಗೆ-ಯಾ-ಗದೆ
ಒಪ್ಪಿ-ಗೆ-ಯಾ-ಗ-ಲಾ-ರವು
ಒಪ್ಪಿ-ಗೆ-ಯಾ-ಗ-ಲಿಲ್ಲ
ಒಪ್ಪಿ-ಗೆ-ಯಾ-ಗಿ-ರ-ಲಿಲ್ಲ
ಒಪ್ಪಿ-ಗೆ-ಯಾ-ಗು-ವಂ-ತಹ
ಒಪ್ಪಿ-ಗೆ-ಯಾ-ಗು-ವಂ-ತಿವೆ
ಒಪ್ಪಿ-ಗೆ-ಯಿದೆ
ಒಪ್ಪಿ-ಗೆ-ಯಿ-ಲ್ಲದೆ
ಒಪ್ಪಿ-ದರು
ಒಪ್ಪಿ-ದ-ರು-ವಿ-ರ-ಜಾ-ನಂ-ದ-ರೊ-ಬ್ಬ-ರನ್ನು
ಒಪ್ಪಿ-ದಳು
ಒಪ್ಪಿದ್ದ
ಒಪ್ಪಿ-ರುವ
ಒಪ್ಪಿ-ಸಲು
ಒಪ್ಪಿಸಿ
ಒಪ್ಪಿ-ಸಿ-ಕೊಂ-ಡ-ಮೇಲೆ
ಒಪ್ಪಿ-ಸಿದ
ಒಪ್ಪಿ-ಸಿ-ದರು
ಒಪ್ಪಿ-ಸಿ-ಬಿ-ಟ್ಟರು
ಒಪ್ಪಿ-ಸಿ-ಯೇ-ಬಿ-ಟ್ಟಳು
ಒಪ್ಪಿ-ಸಿ-ರ-ಲಿ-ಲ್ಲ-ವಾ-ದ್ದ-ರಿಂದ
ಒಪ್ಪಿ-ಸುತ್ತ
ಒಪ್ಪಿ-ಸು-ವರೋ
ಒಪ್ಪಿ-ಸು-ವುದು
ಒಪ್ಪು-ತ್ತಿ-ರ-ಲಿಲ್ಲ
ಒಪ್ಪು-ತ್ತೇನೆ
ಒಪ್ಪು-ತ್ತೇವೆ
ಒಪ್ಪುವ
ಒಪ್ಪು-ವರೇ
ಒಪ್ಪು-ವ-ವ-ರಲ್ಲ
ಒಪ್ಪು-ವಿರಾ
ಒಪ್ಪು-ವು-ದಿಲ್ಲ
ಒಪ್ಪು-ವು-ದಿ-ಲ್ಲ-ವಲ್ಲ
ಒಪ್ಪು-ವುದು
ಒಪ್ಪೊತ್ತು
ಒಬ್ಬ
ಒಬ್ಬನ
ಒಬ್ಬ-ನಂತೂ
ಒಬ್ಬ-ನನ್ನು
ಒಬ್ಬ-ನನ್ನೂ
ಒಬ್ಬ-ನಿ-ಗಾ-ದರೂ
ಒಬ್ಬ-ನಿಗೂ
ಒಬ್ಬ-ನಿಗೆ
ಒಬ್ಬನು
ಒಬ್ಬನೇ
ಒಬ್ಬರ
ಒಬ್ಬ-ರನ್ನು
ಒಬ್ಬ-ರ-ನ್ನೊ-ಬ್ಬರು
ಒಬ್ಬ-ರಾದ
ಒಬ್ಬ-ರಿಂ-ದೊ-ಬ್ಬ-ರಿಗೆ
ಒಬ್ಬ-ರಿಗೂ
ಒಬ್ಬ-ರಿಗೆ
ಒಬ್ಬರು
ಒಬ್ಬರೂ
ಒಬ್ಬ-ರೆಂ-ದರೆ
ಒಬ್ಬರೇ
ಒಬ್ಬ-ಳಿಗೆ
ಒಬ್ಬಳು
ಒಬ್ಬಾಕೆ
ಒಬ್ಬಾತ
ಒಬ್ಬೊ-ಬ್ಬ-ರನ್ನೇ
ಒಬ್ಬೊ-ಬ್ಬ-ರಾಗಿ
ಒಬ್ಬೊ-ಬ್ಬ-ರಿಗೂ
ಒಬ್ಬೊ-ಬ್ಬರು
ಒಬ್ಬೊ-ಬ್ಬರೂ
ಒಬ್ಬೊ-ಬ್ಬ-ಳಿಗೆ
ಒಮ್ಮ-ತ-ವ-ನ್ನುಂ-ಟು-ಮಾ-ಡು-ವಂಥ
ಒಮ್ಮೆ
ಒಮ್ಮೆಗೇ
ಒಮ್ಮೆ-ಯಂತೂ
ಒಮ್ಮೆ-ಯಾ-ದರೂ
ಒಮ್ಮೊಮ್ಮೆ
ಒಯ್ದು
ಒಯ್ಯ-ಬೇಕು
ಒಯ್ಯು-ತ್ತದೆ
ಒಯ್ಯುವ
ಒರ-ಗಿ-ಕೊಂ-ಡಂ-ತಿತ್ತು
ಒರ-ಗಿ-ಕೊಂ-ಡರು
ಒರ-ಗಿ-ಕೊಂಡಿ
ಒರ-ಗಿ-ಕೊ-ಳ್ಳು-ತ್ತಿ-ದ್ದರು
ಒರ-ಗಿ-ಬಿ-ಟ್ಟರು
ಒರ-ಟಾಗಿ
ಒರಟು
ಒರ-ಟು-ತ-ನ-ವೆಲ್ಲ
ಒರ-ತೆ-ಗಳು
ಒರ-ಸು-ವು-ದ-ಕ್ಕಾಗಿ
ಒರೆ
ಒರೆ-ಯೊ-ಳಗೆ
ಒರೆ-ಸಿ-ದರು
ಒರೆ-ಹಚ್ಚಿ
ಒಲ-ವನ್ನು
ಒಲವು
ಒಲ-ವು-ಇ-ವೆಲ್ಲ
ಒಲು-ಮೆ-ಯಾ-ಳವ
ಒಲೆ
ಒಲೆ-ಗಳು
ಒಲ್ಲ
ಒಲ್ಲದ
ಒಲ್ಲೆ-ನೆಂ-ದರೂ
ಒಳ
ಒಳ-ಹೊರ
ಒಳ-ಉ-ದ್ದೇಶ
ಒಳ-ಕ-ರೆದು
ಒಳಕ್ಕೆ
ಒಳ-ಗ-ಡಿ-ಯಿ-ರಿ-ಸಿದ
ಒಳ-ಗಡೆ
ಒಳ-ಗ-ಡೆ-ಯಿದ್ದ
ಒಳ-ಗ-ಡೆಯೇ
ಒಳ-ಗಾ-ಗದೆ
ಒಳ-ಗಾ-ಗ-ಬೇ-ಕಾ-ಗಿತ್ತು
ಒಳ-ಗಾ-ಗು-ತ್ತಿ-ರ-ಲಿಲ್ಲ
ಒಳ-ಗಿನ
ಒಳ-ಗಿ-ನಿಂದ
ಒಳ-ಗಿ-ನಿಂ-ದಲೇ
ಒಳ-ಗಿ-ರು-ವು-ದ-ನ್ನೆಲ್ಲ
ಒಳ-ಗುಟ್ಟು
ಒಳಗೂ
ಒಳಗೆ
ಒಳ-ಗೆಂ-ದಿಗೂ
ಒಳ-ಗೆಯೇ
ಒಳ-ಗೆಲ್ಲ
ಒಳಗೇ
ಒಳ-ಗೊ-ಳಗೇ
ಒಳ-ಗೊ-ಳ್ಳುವಂ
ಒಳ-ನ-ಡೆ-ದರು
ಒಳ-ಪಂ-ಗ-ಡ-ಗಳ
ಒಳ-ಪ-ಟ್ಟಿತ್ತು
ಒಳ-ಪ-ಡಿ-ಸಿ-ಕೊ-ಳ್ಳ-ಬೇಕು
ಒಳ-ಪ್ರ-ದೇಶ
ಒಳ-ಭಾಗ
ಒಳ-ಭಾ-ಗ-ವನ್ನು
ಒಳ-ಹೊ-ಗಲು
ಒಳಾ-ಡ-ಳಿ-ತದ
ಒಳಿ
ಒಳಿ-ತನ್ನು
ಒಳಿ-ತ-ನ್ನುಂ-ಟು-ಮಾ-ಡು-ತ್ತದೆ
ಒಳಿ-ತಾ-ಗಲು
ಒಳಿ-ತಾ-ಗು-ತ್ತದೆ
ಒಳಿ-ತಾ-ದು-ದೆ-ಲ್ಲ-ವನ್ನೂ
ಒಳಿ-ತಿ-ಗಾಗಿ
ಒಳಿ-ತಿ-ಗಾ-ಗಿ-ಕ-ನಿಷ್ಠ
ಒಳಿ-ತಿ-ಗಾ-ಗಿಯೂ
ಒಳಿ-ತಿಗೆ
ಒಳಿತು
ಒಳಿತೂ
ಒಳಿತೇ
ಒಳ್ಳೆ
ಒಳ್ಳೆಯ
ಒಳ್ಳೆ-ಯ-ತ-ನದ
ಒಳ್ಳೆ-ಯ-ದ-ಕ್ಕಾ-ಗಿಯೇ
ಒಳ್ಳೆ-ಯ-ದಕ್ಕೇ
ಒಳ್ಳೆ-ಯ-ದ-ಕ್ಕೋ-ಸ್ಕರ
ಒಳ್ಳೆ-ಯ-ದನ್ನು
ಒಳ್ಳೆ-ಯ-ದಲ್ಲ
ಒಳ್ಳೆ-ಯ-ದ-ಲ್ಲವೆ
ಒಳ್ಳೆ-ಯ-ದ-ಲ್ಲ-ವೆಂದು
ಒಳ್ಳೆ-ಯ-ದಾಗಿ
ಒಳ್ಳೆ-ಯ-ದಾ-ಗಿಯೂ
ಒಳ್ಳೆ-ಯ-ದಾ-ಗು-ವುದು
ಒಳ್ಳೆ-ಯ-ದಿ-ತ್ತ-ಲ್ಲವೆ
ಒಳ್ಳೆ-ಯದು
ಒಳ್ಳೆ-ಯ-ದು-ಕೆ-ಟ್ಟದ್ದು
ಒಳ್ಳೆ-ಯದೂ
ಒಳ್ಳೆ-ಯ-ದೆಂದು
ಒಳ್ಳೆ-ಯದೇ
ಒಳ್ಳೆ-ಯದೋ
ಒಳ್ಳೆ-ಯ-ವ-ನಾ-ಗ-ಬೇಕು
ಒಳ್ಳೆ-ಯ-ವ-ನಾ-ಗಿ-ದ್ದು-ಕೊಂಡು
ಒಳ್ಳೆ-ಯ-ವ-ನಾ-ಗಿ-ರು-ವುದು
ಒಳ್ಳೆ-ಯ-ವ-ನಾಗು
ಒಳ್ಳೆ-ಯ-ವ-ನೆ-ನ್ನಿ-ಸ-ಬೇ-ಕಾ-ದರೆ
ಒಳ್ಳೆ-ಯ-ವ-ರಾ-ಗಲು
ಒಳ್ಳೆ-ಯ-ವ-ರಾ-ಗಿ-ರಲು
ಒಳ್ಳೆ-ಯ-ವ-ರಾ-ಗಿ-ರು-ವು-ದರ
ಒಳ್ಳೆ-ಯ-ವ-ರಾ-ಗಿ-ರು-ವು-ದೇ-ಕೆಂ-ದರೆ
ಒಳ್ಳೇ
ಒಳ್ಳೇದು
ಓ
ಓಂ
ಓಂಕಾ-ರದ
ಓಕ್
ಓಕ್ಲ್ಯಾಂ-ಡಿಗೆ
ಓಕ್ಲ್ಯಾಂ-ಡಿನ
ಓಕ್ಲ್ಯಾಂ-ಡಿ-ನಲ್ಲಿ
ಓಕ್ಲ್ಯಾಂಡ್
ಓಗೊಟ್ಟು
ಓಗೊ-ಡ-ಬೇ-ಕಾ-ಗಿದೆ
ಓಗೊ-ಡು-ವಿರಾ
ಓಟ
ಓಡ-ತೊ-ಡ-ಗಿ-ದುವು
ಓಡಾ-ಟ-ಗಳ
ಓಡಾ-ಡ-ಬೇ-ಕಾ-ಗಿತ್ತು
ಓಡಾ-ಡಲು
ಓಡಾ-ಡಿ-ಕೊಂ-ಡಿತ್ತು
ಓಡಾ-ಡಿ-ಕೊಂ-ಡಿ-ದ್ದುವು
ಓಡಾ-ಡಿ-ದರು
ಓಡಾ-ಡುತ್ತ
ಓಡಾ-ಡು-ತ್ತಾರೆ
ಓಡಾ-ಡು-ತ್ತಿ-ದ್ದರು
ಓಡಾ-ಡು-ತ್ತಿ-ದ್ದರೆ
ಓಡಾ-ಡು-ತ್ತಿ-ದ್ದುದು
ಓಡಾ-ಡು-ತ್ತಿ-ದ್ದೇ-ನೆ-ಆ-ಗ-ಲಾ-ದರೂ
ಓಡಾ-ಡು-ತ್ತಿ-ರು-ವಾಗ
ಓಡಾ-ಡುವ
ಓಡಾ-ಡು-ವುದು
ಓಡಿ
ಓಡಿ-ದರು
ಓಡಿ-ದ್ದರು
ಓಡಿ-ಬಂ-ದರು
ಓಡಿ-ಬಂದು
ಓಡಿ-ಬಂದೆ
ಓಡಿ-ಬ-ರುವ
ಓಡಿ-ಸ-ಬಾ-ರದು
ಓಡಿ-ಸಲು
ಓಡಿಸಿ
ಓಡಿ-ಸಿ-ದರು
ಓಡಿ-ಹೋಗಿ
ಓಡಿ-ಹೋ-ದರು
ಓಡಿ-ಹೋ-ದುವು
ಓಡು-ತ್ತಿದ್ದ
ಓಡು-ತ್ತಿ-ರುವ
ಓಡೋಡಿ
ಓಡೋ-ಡುತ್ತ
ಓದ
ಓದದೇ
ಓದ-ಬಲ್ಲೆ
ಓದ-ಲಾ-ಯಿತು
ಓದಲು
ಓದಿ
ಓದಿ-ಕೊಂಡ
ಓದಿ-ಕೊಂ-ಡ-ವನು
ಓದಿ-ಕೊ-ಳ್ಳ-ಲಿಲ್ಲ
ಓದಿ-ಕೊ-ಳ್ಳು-ವು-ದ-ರಿಂ-ದೇನು
ಓದಿದ
ಓದಿ-ದಂ-ತೆಲ್ಲ
ಓದಿ-ದರು
ಓದಿ-ದರೂ
ಓದಿ-ದ್ದನ್ನು
ಓದಿ-ದ್ದೇನೆ
ಓದಿ-ನೋ-ಡು-ಪ್ರ-ತಿ-ಯೊಂದು
ಓದಿ-ರು-ವಿರಾ
ಓದು
ಓದು-ಗರ
ಓದು-ಗ-ರನ್ನು
ಓದು-ಗ-ರಿಗೆ
ಓದು-ಗ-ರಿ-ಗೊಂದು
ಓದು-ಗರು
ಓದುತ್ತ
ಓದು-ತ್ತಲೋ
ಓದು-ತ್ತಿದ್ದ
ಓದು-ತ್ತಿ-ದ್ದಂತೆ
ಓದು-ತ್ತಿ-ದ್ದರು
ಓದು-ತ್ತಿ-ರುವ
ಓದು-ಬ-ರಹ
ಓದುವ
ಓದು-ವಂ-ತಿ-ರ-ಲಿಲ್ಲ
ಓದು-ವಂ-ತಿಲ್ಲ
ಓದು-ವಂತೆ
ಓದು-ವ-ವ-ರಿಗೆ
ಓದು-ವ-ವರೇ
ಓದುವು
ಓದು-ವು-ದಕ್ಕೆ
ಓದು-ವುದು
ಓದು-ವು-ದು-ಬ-ರೆ-ಯು-ವುದು
ಓರಿ-ಯೆಂ-ಟಲ್
ಓರೆ
ಓರ್ವ
ಓಲಿಯಾ
ಓಲೆ
ಓಲ್
ಓಲ್ಕಾಟ್
ಓವೊ
ಓಹೋ
ಓಹ್
ಔತಣ
ಔತ-ಣ-ವನ್ನೇ
ಔತ-ಣ-ವ-ನ್ನೇ-ರ್ಪ-ಡಿ-ಸಿ-ದರು
ಔದಾ-ರ್ಯದ
ಔದಾ-ರ್ಯ-ವನ್ನು
ಔದಾ-ರ್ಯ-ವೊಂ-ದಿ-ದ್ದರೆ
ಔನ್ನ-ತ್ಯ-ವ-ನ್ನಾ-ಗಲಿ
ಔನ್ನ-ತ್ಯ-ವನ್ನು
ಔಪ-ಚಾ-ರಿ-ಕತೆ
ಔಷಧ
ಔಷ-ಧ-ವಲ್ಲ
ಔಷಧಿ
ಔಷ-ಧಿ-ಗಳೂ
ಔಷ-ಧಿ-ಯನ್ನೇ
ಔಷಧೋ
ಔಷ-ಧೋ-ಪ-ಚಾ-ರ-ವನ್ನು
ಕ
ಕಂಕ-ಣ-ಬ-ದ್ಧ-ರಾ-ಗು-ವಂತೆ
ಕಂಖ-ಲಿ-ನಲ್ಲಿ
ಕಂಗಳನ್ನೂ
ಕಂಗಳಿಂದ
ಕಂಗಳು
ಕಂಗಾ-ಲಾಗಿ
ಕಂಗಾ-ಲಾ-ಗಿ-ದ್ದಾ-ಗಲೇ
ಕಂಗಾ-ಲಾ-ಗಿ-ರುವ
ಕಂಗಾ-ಲಾ-ದು-ದನ್ನು
ಕಂಗೆ-ಡಿ-ಸಿತು
ಕಂಗೆ-ಡಿ-ಸು-ವಂ-ಥದು
ಕಂಗೊ-ಳಿ-ಸಿ-ದರು
ಕಂಗೊ-ಳಿ-ಸು-ತ್ತಿದೆ
ಕಂಗೊ-ಳಿ-ಸು-ತ್ತಿದ್ದ
ಕಂಗೊ-ಳಿ-ಸು-ತ್ತಿ-ದ್ದರು
ಕಂಗೊ-ಳಿ-ಸು-ತ್ತಿ-ದ್ದುವು
ಕಂಗೊ-ಳಿ-ಸು-ತ್ತಿ-ರು-ತ್ತದೆ
ಕಂಠ
ಕಂಠ-ಗಳಿಂದ
ಕಂಠ-ದಿಂದ
ಕಂಠ-ದಿಂ-ದಲೇ
ಕಂಠ-ದಿಂ-ದಾಗಿ
ಕಂಠ-ಧ್ವ-ನಿಯೂ
ಕಂಠ-ಮಾ-ಧು-ರ್ಯದ
ಕಂಠ-ಶ್ರೀ-ಯನ್ನು
ಕಂಠ-ಸ್ಥ-ವಾ-ಗಿವೆ
ಕಂಡ
ಕಂಡಂತೆ
ಕಂಡಂ-ದಿ-ನಿಂದ
ಕಂಡ-ಕೂ-ಡಲೇ
ಕಂಡ-ದ್ದಾ-ಗಿದೆ
ಕಂಡದ್ದು
ಕಂಡ-ರಿ-ಯದು
ಕಂಡರು
ಕಂಡರೂ
ಕಂಡರೆ
ಕಂಡ-ವನು
ಕಂಡ-ವನೇ
ಕಂಡ-ವರ
ಕಂಡ-ವ-ರ-ಲ್ಲವೆ
ಕಂಡ-ವ-ರಿಗೆ
ಕಂಡ-ವರು
ಕಂಡಾಗ
ಕಂಡಾ-ಗ-ಲಂತೂ
ಕಂಡಾ-ಗಲೂ
ಕಂಡಾ-ಗಲೇ
ಕಂಡಿತು
ಕಂಡಿತ್ತು
ಕಂಡಿದೆ
ಕಂಡಿದ್ದ
ಕಂಡಿ-ದ್ದರು
ಕಂಡಿ-ದ್ದಳು
ಕಂಡಿದ್ದಾ
ಕಂಡಿ-ದ್ದಾರೆ
ಕಂಡಿ-ದ್ದೇನೆ
ಕಂಡಿ-ದ್ದೇವೆ
ಕಂಡಿ-ರದ
ಕಂಡಿ-ರ-ದಿದ್ದ
ಕಂಡಿ-ರ-ಬ-ಹುದು
ಕಂಡಿ-ರ-ಲಿಲ್ಲ
ಕಂಡಿ-ರ-ಲಿ-ಲ್ಲವೋ
ಕಂಡಿರಾ
ಕಂಡಿರು
ಕಂಡಿ-ರುವ
ಕಂಡಿ-ರೆಂ-ಬು-ದನ್ನು
ಕಂಡಿಲ್ಲ
ಕಂಡಿ-ಲ್ಲವೆ
ಕಂಡು
ಕಂಡು-ಕೊಂಡ
ಕಂಡು-ಕೊಂ-ಡಂ-ತಾ-ಗು-ವುದು
ಕಂಡು-ಕೊಂ-ಡರು
ಕಂಡು-ಕೊಂ-ಡರೋ
ಕಂಡು-ಕೊಂ-ಡ-ವ-ರೆಂ-ದರೆ
ಕಂಡು-ಕೊಂ-ಡಾ-ಗಲೇ
ಕಂಡು-ಕೊಂಡಿ
ಕಂಡು-ಕೊಂ-ಡಿತ್ತು
ಕಂಡು-ಕೊಂ-ಡಿದ್ದ
ಕಂಡು-ಕೊಂ-ಡಿ-ದ್ದರು
ಕಂಡು-ಕೊಂ-ಡಿ-ದ್ದಾ-ನಲ್ಲ
ಕಂಡು-ಕೊಂ-ಡಿ-ದ್ದಾರೆ
ಕಂಡು-ಕೊಂ-ಡಿ-ದ್ದೇ-ನೆ-ಅ-ದಕ್ಕೆ
ಕಂಡು-ಕೊಂಡು
ಕಂಡು-ಕೊಂಡೆ
ಕಂಡು-ಕೊಳ್ಳ
ಕಂಡು-ಕೊ-ಳ್ಳ-ಬ-ಯ-ಸು-ವುದು
ಕಂಡು-ಕೊ-ಳ್ಳ-ಬ-ಲ್ಲರು
ಕಂಡು-ಕೊ-ಳ್ಳಲಿ
ಕಂಡು-ಕೊ-ಳ್ಳಲು
ಕಂಡು-ಕೊ-ಳ್ಳು-ತ್ತೀರಿ
ಕಂಡು-ಕೊ-ಳ್ಳು-ವುದು
ಕಂಡು-ಕೊ-ಳ್ಳುವೆ
ಕಂಡು-ಬಂತು
ಕಂಡು-ಬಂ-ದದ್ದು
ಕಂಡು-ಬಂ-ದರು
ಕಂಡು-ಬಂ-ದರೂ
ಕಂಡು-ಬಂ-ದರೆ
ಕಂಡು-ಬಂ-ದಾಗ
ಕಂಡು-ಬಂ-ದಿತು
ಕಂಡು-ಬಂ-ದುವು
ಕಂಡು-ಬರ
ಕಂಡು-ಬ-ರ-ತೊ-ಡ-ಗಿತು
ಕಂಡು-ಬ-ರ-ಬ-ಹು-ದಾದ
ಕಂಡು-ಬ-ರ-ಬ-ಹುದೋ
ಕಂಡು-ಬ-ರ-ಲಿಲ್ಲ
ಕಂಡು-ಬ-ರ-ಲಿ-ಲ್ಲ-ವಾ-ದರೂ
ಕಂಡು-ಬರು
ಕಂಡು-ಬ-ರು-ತ್ತದೆ
ಕಂಡು-ಬ-ರು-ತ್ತ-ದೆಂ-ಬು-ದ-ರಲ್ಲಿ
ಕಂಡು-ಬ-ರು-ತ್ತಿತ್ತು
ಕಂಡು-ಬ-ರು-ತ್ತಿದೆ
ಕಂಡು-ಬ-ರು-ತ್ತಿ-ದ್ದಾ-ರಲ್ಲ
ಕಂಡು-ಬ-ರು-ತ್ತಿ-ದ್ದಾರೆ
ಕಂಡು-ಬ-ರು-ತ್ತಿಲ್ಲ
ಕಂಡು-ಬ-ರುವ
ಕಂಡು-ಬ-ರು-ವಂ-ತಹ
ಕಂಡು-ಬ-ರು-ವು-ದಿಲ್ಲ
ಕಂಡು-ಹಿ-ಡಿ-ಯ-ಬೇಕು
ಕಂಡು-ಹಿ-ಡಿ-ಯು-ತ್ತಲೇ
ಕಂಡು-ಹಿ-ಡಿ-ಯು-ತ್ತಿ-ರು-ವುದು
ಕಂಡು-ಹಿ-ಡಿ-ಯು-ವಂ-ತಾ-ಗ-ಬೇಕು
ಕಂಡೂ
ಕಂಡೆ
ಕಂಡೇ
ಕಂಡೊ-ಡನೆ
ಕಂಡೊ-ಡ-ನೆಯೇ
ಕಂತೆ
ಕಂತೆ-ಗಳು
ಕಂತೆಯ
ಕಂತೆಯೇ
ಕಂದ-ಕ-ವಿ-ರು-ವಂತೆ
ಕಂದಾ-ಚಾ-ರ-ಗಳನ್ನು
ಕಂದಾ-ಚಾ-ರ-ಗ-ಳಲ್ಲೇ
ಕಂದಾ-ಚಾ-ರ-ಗ-ಳಿಗೆ
ಕಂದಾ-ಚಾ-ರ-ಗಳು
ಕಂಪ
ಕಂಪ-ನ-ಗಳ
ಕಂಪ-ನ-ಗಳು
ಕಂಪ-ನನ
ಕಂಪ-ನಿ-ಯ-ವರು
ಕಂಪ-ನಿ-ಯೊಂ-ದರ
ಕಂಪಿ-ಸ-ತೊ-ಡ-ಗಿತು
ಕಂಪಿ-ಸ-ಲಾ-ರಂ-ಭಿ-ಸಿತು
ಕಂಪಿ-ಸು-ತ್ತಿತ್ತು
ಕಂಪೆಯ
ಕಂಪ್ಯೂ-ಟ-ರನ್ನೋ
ಕಂಬಕ್ಕೆ
ಕಂಬ-ಗಳ
ಕಂಬ-ಗಳಿಂದ
ಕಂಬ-ಗಳು
ಕಕುಸೋ
ಕಗ್ಗ-ತ್ತಲ
ಕಚ್ಚಾ
ಕಚ್ಚಿ
ಕಛೇ-ರಿ-ಗಳಲ್ಲಿ
ಕಟ-ಕ-ಟೆಯ
ಕಟಾ-ಕ್ಷವು
ಕಟಿ-ಬ-ದ್ಧ-ರಾಗಿ
ಕಟು
ಕಟು-ಕರ
ಕಟು-ಕರೇ
ಕಟು-ಟೀಕೆ
ಕಟು-ವಾಗಿ
ಕಟು-ವಾ-ಗಿ-ದ್ದಿ-ರಲಿ
ಕಟು-ವಾದ
ಕಟ್ಟ-ಕ-ಡೆಯ
ಕಟ್ಟ-ಕ-ಡೆ-ಯ-ದಾಗಿ
ಕಟ್ಟಡ
ಕಟ್ಟ-ಡಕ್ಕೆ
ಕಟ್ಟ-ಡ-ಗಳನ್ನೆಲ್ಲ
ಕಟ್ಟ-ಡ-ಗಳಿಂದ
ಕಟ್ಟ-ಡ-ಗ-ಳಿ-ದ್ದುವು
ಕಟ್ಟ-ಡ-ಗಳು
ಕಟ್ಟ-ಡ-ಗ-ಳೆಲ್ಲ
ಕಟ್ಟ-ಡದ
ಕಟ್ಟ-ಡ-ದಲ್ಲೇ
ಕಟ್ಟ-ಡ-ದಷ್ಟು
ಕಟ್ಟ-ಡ-ದಿಂದ
ಕಟ್ಟ-ಡ-ದೊ-ಳಗೇ
ಕಟ್ಟ-ಡ-ವನ್ನು
ಕಟ್ಟ-ಡ-ವೊಂ-ದನ್ನು
ಕಟ್ಟ-ಡ-ವೊಂ-ದರ
ಕಟ್ಟನ್ನು
ಕಟ್ಟ-ಬ-ಹುದು
ಕಟ್ಟ-ಬೇಕು
ಕಟ್ಟ-ಬೇ-ಕೆಂ-ದಿ-ರುವ
ಕಟ್ಟ-ಲಾ-ರಂ-ಭಿ-ಸಿ-ದ್ದರು
ಕಟ್ಟಲು
ಕಟ್ಟ-ಲೆಯೂ
ಕಟ್ಟ-ಲ್ಪಟ್ಟ
ಕಟ್ಟ-ಲ್ಪ-ಡು-ತ್ತಿದ್ದ
ಕಟ್ಟಾ
ಕಟ್ಟಿ
ಕಟ್ಟಿ-ಕೊಂಡ
ಕಟ್ಟಿ-ಕೊಂಡು
ಕಟ್ಟಿ-ಕೊಂಡೇ
ಕಟ್ಟಿ-ಕೊಳ್ಳ
ಕಟ್ಟಿ-ಕೊ-ಳ್ಳ-ಬೇಕು
ಕಟ್ಟಿ-ಕೊಳ್ಳು
ಕಟ್ಟಿ-ಟ್ಟದ್ದು
ಕಟ್ಟಿತು
ಕಟ್ಟಿದ
ಕಟ್ಟಿ-ದರು
ಕಟ್ಟಿ-ದ್ದರು
ಕಟ್ಟಿ-ದ್ದೆ-ಲ್ಲವೂ
ಕಟ್ಟಿ-ನಲ್ಲಿ
ಕಟ್ಟಿ-ನಿಂ-ದಾಚೆ
ಕಟ್ಟಿ-ಸ-ಬೇಕು
ಕಟ್ಟಿ-ಸಿ-ಕೊಂ-ಡಿದ್ದ
ಕಟ್ಟಿ-ಸಿ-ಕೊಟ್ಟು
ಕಟ್ಟಿ-ಸಿ-ಕೊ-ಳ್ಳ-ಬಲ್ಲೆ
ಕಟ್ಟಿ-ಸಿ-ದರೆ
ಕಟ್ಟಿ-ಸಿ-ದ-ರೇನು
ಕಟ್ಟಿ-ಸಿ-ದ-ವನು
ಕಟ್ಟಿ-ಸಿದ್ದ
ಕಟ್ಟಿ-ಸು-ವು-ದು-ಇವು
ಕಟ್ಟಿ-ಸು-ವು-ದು-ಇವೇ
ಕಟ್ಟಿ-ಸು-ವುದೇ
ಕಟ್ಟಿ-ಹಾ-ಕಲು
ಕಟ್ಟಿ-ಹಾ-ಕ-ಲ್ಪ-ಟ್ಟಿ-ದ್ದೇವೆ
ಕಟ್ಟಿ-ಹಾ-ಕ-ಲ್ಪ-ಟ್ಟಿ-ರ-ಲಿಲ್ಲ
ಕಟ್ಟಿ-ಹಾಕಿ
ಕಟ್ಟಿ-ಹಾ-ಕಿದ
ಕಟ್ಟು
ಕಟ್ಟು-ಕ-ಟ್ಟ-ಲೆ-ಗ-ಳಾ-ಗಲಿ
ಕಟ್ಟು-ತ್ತಲೇ
ಕಟ್ಟು-ನಿ-ಟ್ಟಾಗಿ
ಕಟ್ಟು-ನಿ-ಟ್ಟಾ-ಗಿತ್ತು
ಕಟ್ಟು-ನಿ-ಟ್ಟಾ-ಗಿಲ್ಲ
ಕಟ್ಟು-ನಿ-ಟ್ಟಾದ
ಕಟ್ಟು-ನಿ-ಟ್ಟಿನ
ಕಟ್ಟು-ನಿ-ಟ್ಟಿ-ನಿಂ-ದಿ-ದ್ದು-ಬಿ-ಟ್ಟರು
ಕಟ್ಟು-ನಿಟ್ಟು
ಕಟ್ಟು-ಪಾ-ಡು-ಗಳ
ಕಟ್ಟು-ಬಿ-ದ್ದಿರ
ಕಟ್ಟು-ಬಿ-ದ್ದಿ-ರು-ವು-ದಾ-ಗಲಿ
ಕಟ್ಟು-ಮ-ಸ್ತಾದ
ಕಟ್ಟುವ
ಕಟ್ಟು-ವಂತೆ
ಕಟ್ಟು-ವ-ವ-ರಲ್ಲ
ಕಟ್ಟು-ವು-ದ-ರಿಂ
ಕಟ್ಟೆ
ಕಟ್ಟೆ-ಗಳನ್ನು
ಕಟ್ಟೆ-ಯಲ್ಲಿ
ಕಟ್ಟೆ-ಯೊ-ಡೆದು
ಕಠಿಣ
ಕಠಿ-ಣ-ಒಂದು
ಕಠಿ-ಣ-ತಮ
ಕಠಿ-ಣ-ತ-ರ-ವಾದ
ಕಠಿ-ಣ-ವಾ-ಗಿಯೇ
ಕಠಿ-ಣ-ವಾ-ಗಿ-ರ-ಬೇ-ಕಾಗಿ
ಕಠಿ-ಣ-ವಾದ
ಕಠಿ-ಣವೇ
ಕಠೋರ
ಕಡ-ಗ-ಗಳನ್ನು
ಕಡಮೆ
ಕಡ-ಮೆ-ಯಾ-ಗ-ಲಿಲ್ಲ
ಕಡಲ
ಕಡ-ಲಿ-ನ-ಲೆ-ಗಳ
ಕಡಿ-ದಾದ
ಕಡಿದು
ಕಡಿ-ದು-ಕೊಂಡು
ಕಡಿ-ದು-ಕೊಂ-ಡು-ಬಿ-ಟ್ಟಳು
ಕಡಿಮೆ
ಕಡಿ-ಮೆ-ಯಾ-ಗ-ಲಿತ್ತು
ಕಡಿ-ಮೆ-ಯಾ-ಗ-ಲಿಲ್ಲ
ಕಡಿ-ಮೆ-ಯಾ-ಗಿ-ರ-ಲಿಲ್ಲ
ಕಡಿ-ಮೆ-ಯಾಗು
ಕಡಿ-ಮೆ-ಯಾ-ಗುತ್ತ
ಕಡಿ-ಮೆ-ಯಾ-ಗು-ತ್ತದೆ
ಕಡಿ-ಮೆ-ಯಾ-ಗು-ತ್ತಿ-ದ್ದವು
ಕಡಿ-ಮೆ-ಯಿ-ದ್ದಿ-ರ-ಲಾ-ರದು
ಕಡಿ-ಮೆ-ಯಿ-ರ-ಲಿಲ್ಲ
ಕಡಿ-ಮೆ-ಯಿ-ರುವ
ಕಡಿ-ಮೆ-ಯಿ-ರು-ವು-ದ-ರಿಂದ
ಕಡಿ-ಮೆಯೆ
ಕಡಿ-ಮೆಯೇ
ಕಡಿ-ಯೆ-ಯಾ-ಗುತ್ತ
ಕಡು
ಕಡು-ಶ-ತ್ರುವೂ
ಕಡೆ
ಕಡೆ-ಗ-ಣಿಸಿ
ಕಡೆ-ಗ-ಣಿ-ಸಿ-ದ-ವರು
ಕಡೆ-ಗ-ಣಿ-ಸಿ-ದ್ದ-ರಿಂದ
ಕಡೆ-ಗ-ಣಿ-ಸಿ-ದ್ದಾರೆ
ಕಡೆ-ಗ-ಣಿಸು
ಕಡೆ-ಗ-ಣಿ-ಸುವ
ಕಡೆ-ಗ-ಣಿ-ಸು-ವಂ-ತಿಲ್ಲ
ಕಡೆ-ಗಳಲ್ಲಿ
ಕಡೆ-ಗಳಿಂದ
ಕಡೆ-ಗ-ಳಿಂ-ದಲೂ
ಕಡೆ-ಗ-ಳಿ-ಗಿಂ-ತಲೂ
ಕಡೆ-ಗಿಂದು
ಕಡೆ-ಗಿನ
ಕಡೆ-ಗೀಗ
ಕಡೆಗೂ
ಕಡೆಗೆ
ಕಡೆ-ಗೆಲ್ಲ
ಕಡೆಗೇ
ಕಡೆ-ಗೊಂದು
ಕಡೆ-ಗೊಮ್ಮೆ
ಕಡೆಗೋ
ಕಡೆ-ದಂಥ
ಕಡೆಯ
ಕಡೆ-ಯ-ದಾಗಿ
ಕಡೆ-ಯ-ದಾದ
ಕಡೆ-ಯ-ಪಕ್ಷ
ಕಡೆ-ಯ-ಲ್ಪಟ್ಟ
ಕಡೆ-ಯಲ್ಲಿ
ಕಡೆ-ಯ-ವ-ರಿಗೂ
ಕಡೆ-ಯ-ವ-ರೆಗೂ
ಕಡೆ-ಯಿಂದ
ಕಡೆ-ಯಿ-ರು-ವ-ಮ-ತ್ತೊಂದು
ಕಡೇ
ಕಡ್ಡಿ-ಗಳನ್ನು
ಕಡ್ಡಿ-ಮು-ರಿ-ದಂತೆ
ಕಣಿ-ವೆಯ
ಕಣಿ-ವೆ-ಯಲ್ಲಿ
ಕಣಿ-ವೆ-ಯೊ-ಳಗೆ
ಕಣೊ
ಕಣ್ಕಣ್ಣು
ಕಣ್ಣಂ-ಚಿ-ನಲ್ಲಿ
ಕಣ್ಣ-ಮುಂ-ದಿ-ರಿ-ಸಿ-ಕೊಂ-ಡಿ-ದ್ದರು
ಕಣ್ಣ-ಮುಂದೆ
ಕಣ್ಣಲ್ಲಿ
ಕಣ್ಣಾ-ಮು-ಚ್ಚಾಲೆ
ಕಣ್ಣಾರೆ
ಕಣ್ಣಿಂದ
ಕಣ್ಣಿಗೂ
ಕಣ್ಣಿಗೆ
ಕಣ್ಣಿಗೇ
ಕಣ್ಣಿ-ಗೊ-ತ್ತಿ-ಕೊಂಡು
ಕಣ್ಣಿ-ಟ್ಟಿ-ದ್ದರು
ಕಣ್ಣಿ-ಟ್ಟಿರು
ಕಣ್ಣಿ-ಟ್ಟಿ-ರುವ
ಕಣ್ಣಿ-ದ್ದ-ವ-ರಿಗೆ
ಕಣ್ಣಿ-ದ್ದ-ವರು
ಕಣ್ಣಿನ
ಕಣ್ಣಿ-ನಿಂದ
ಕಣ್ಣೀರ
ಕಣ್ಣೀ-ರನ್ನು
ಕಣ್ಣೀ-ರಿ-ನಲ್ಲಿ
ಕಣ್ಣೀರು
ಕಣ್ಣೀ-ರ್ಗ-ರೆ-ದರು
ಕಣ್ಣು
ಕಣ್ಣು-ಗಳ
ಕಣ್ಣು-ಗ-ಳನ್ನೇ
ಕಣ್ಣು-ಗಳಲ್ಲಿ
ಕಣ್ಣು-ಗ-ಳಷ್ಟೇ
ಕಣ್ಣು-ಗಳಿಂದ
ಕಣ್ಣು-ಗಳು
ಕಣ್ಣು-ಗ-ಳು-ಆಹ್
ಕಣ್ಣು-ಮುಚ್ಚಿ
ಕಣ್ಣೆ-ದು-ರಿಗೆ
ಕಣ್ಣೆ-ದು-ರಿಗೇ
ಕಣ್ಣೆ-ದು-ರಿ-ನಲ್ಲೇ
ಕಣ್ಣೆ-ದುರು
ಕಣ್ಣೊ-ರೆ-ಸುವ
ಕಣ್ತುಂಬ
ಕಣ್ತೆ-ರೆದು
ಕಣ್ದೆ-ರೆ-ದರು
ಕಣ್ದೆ-ರೆದು
ಕಣ್ಮ-ರೆ-ಯಾ-ಗು-ತ್ತದೆ
ಕಣ್ಮ-ರೆ-ಯಾದ
ಕಣ್ಮ-ರೆ-ಯಾ-ದರೂ
ಕಣ್ಮ-ರೆ-ಯಾ-ಯಿತು
ಕಣ್ಮುಂದೆ
ಕಣ್ಮುಚ್ಚಿ
ಕಣ್ಮು-ಚ್ಚಿ-ದರೆ
ಕಣ್ರೆ-ಪ್ಪೆ-ಗಳು
ಕಣ್ಸೆ-ಳೆ-ಯು-ವಂ-ತಿದ್ದ
ಕತೆ
ಕತೆ-ಗಳ
ಕತೆ-ಗಳನ್ನು
ಕತೆ-ಗಳು
ಕತೆಯ
ಕತೆ-ಯಿಲ್ಲ
ಕತೆಯೇ
ಕತ್ತನ್ನೇ
ಕತ್ತ-ರಿ-ಕೆ-ಗಳು
ಕತ್ತ-ರಿ-ಸ-ಲಾ-ರದು
ಕತ್ತ-ರಿಸಿ
ಕತ್ತ-ರಿಸು
ಕತ್ತ-ರಿ-ಸು-ವುದು
ಕತ್ತಲ
ಕತ್ತ-ಲಲಿ
ಕತ್ತ-ಲಲ್ಲಿ
ಕತ್ತ-ಲಲ್ಲೇ
ಕತ್ತ-ಲಾ-ಗಿತ್ತು
ಕತ್ತ-ಲಾ-ವ-ರಿ-ಸಿತು
ಕತ್ತಲು
ಕತ್ತ-ಲೆಯ
ಕತ್ತಿ
ಕತ್ತೆ-ತ್ತದೆ
ಕಥ-ಗೋಡಂ
ಕಥ-ಗೋ-ಡಂಗೆ
ಕಥ-ಗೋ-ಡಂ-ನಲ್ಲೇ
ಕಥ-ಗೋ-ಡಂ-ವ-ರೆಗೆ
ಕಥ-ನ-ವನ್ನು
ಕಥೆ
ಕಥೆ-ಗಳ
ಕಥೆ-ಗಳನ್ನು
ಕಥೆ-ಗ-ಳ-ನ್ನೊ-ಳ-ಗೊಂ-ಡಂತೆ
ಕಥೆ-ಗ-ಳಿವೆ
ಕಥೆ-ಗಳು
ಕಥೆ-ಗ-ಳೊಂ-ದಿಗೆ
ಕಥೆ-ಗಳೋ
ಕಥೆ-ಯನ್ನು
ಕಥೆಯು
ಕಥೆ-ಯೊಂ-ದಿಗೆ
ಕದ-ಡಲು
ಕದ-ಡಿತು
ಕದಡು
ಕದ-ಡು-ವಂ-ತಾ-ಯಿತು
ಕದನ
ಕದ-ನಕೆ
ಕದ-ಲಿ-ಸ-ಬಾ-ರದು
ಕದ-ಲಿ-ಸ-ಲಾ-ರ-ದಂ-ತಹ
ಕದ-ಲಿ-ಸಿ-ಬಿ-ಡ-ಬ-ಹು-ದೆಂಬ
ಕದ-ಲು-ವು-ದಿಲ್ಲ
ಕದಾ-ಚನ
ಕದ್ದರೆ
ಕದ್ದು
ಕನ-ವ-ರಿ-ಸು-ವಂತೆ
ಕನ-ಸಾ-ಗಿತ್ತು
ಕನ-ಸಾ-ಯಿತು
ಕನ-ಸಿಗೆ
ಕನ-ಸಿನ
ಕನ-ಸಿ-ನಲ್ಲಿ
ಕನ-ಸಿ-ನಲ್ಲೂ
ಕನಸು
ಕನ-ಸು-ಗಳನ್ನು
ಕನ-ಸು-ಗಳಲ್ಲಿ
ಕನ-ಸು-ಗ-ಳು-ಇ-ವೆಲ್ಲ
ಕನ-ಸು-ಗ-ಳೆಲ್ಲ
ಕನ-ಸು-ಮ-ನ-ಸ್ಸಿ-ನಲ್ಲೂ
ಕನಿ-ಕರ
ಕನಿಷ್ಠ
ಕನಿ-ಷ್ಠ-ಕ್ಕಿ-ಳಿಸಿ
ಕನಿ-ಷ್ಠ-ಪಕ್ಷ
ಕನಿ-ಷ್ಠ-ರಲ್ಲಿ
ಕನಿ-ಷ್ಠ-ರಾ-ದ-ವ-ರ-ಲ್ಲಿಯೂ
ಕನ್
ಕನ್ನಡ
ಕನ್ನ-ಡಕ್ಕೂ
ಕನ್ನ-ಡ-ದಲ್ಲಿ
ಕನ್ನಡಿ
ಕನ್ನ-ಡಿ-ಗ-ರು-ತ-ಮಿ-ಳರು
ಕನ್ನ-ಡಿ-ಯಲ್ಲಿ
ಕನ್ಯಾ-ಕು-ಮಾ-ರಿ-ಯ-ವ-ರೆಗೆ
ಕನ್ಯೆಗೂ
ಕನ್ಯೆಗೆ
ಕನ್ಹಾಯ್
ಕಪಿ-ಮು-ಷ್ಟಿ-ಯಿಂದ
ಕಪ್ಪ-ಗಿ-ದ್ದರೂ
ಕಪ್ಪ-ಗಿ-ಹಳೆ
ಕಪ್ಪು
ಕಪ್ಪೆ-ಗಳು
ಕಬೀ-ರರು
ಕಬ್ಬಿ-ಣದ
ಕಮಲ
ಕಮ-ಲ-ಗಳು
ಕಮ-ಲದ
ಕಮ-ಲ-ನ-ಯನ
ಕಮ-ಲ-ನ-ಯ-ನ-ಗಳಿಂದ
ಕಮ-ಲ-ವೊಂದು
ಕಮಾ-ನಿನ
ಕಮಾ-ನು-ಗಳನ್ನು
ಕಮಾ-ನು-ಗಳು
ಕಮಾ-ನೊಂ-ದನ್ನು
ಕಮೀ-ಷ-ನ-ರನೂ
ಕರಂ-ಡ-ದ-ಲ್ಲಿಟ್ಟು
ಕರ-ಕ-ರೆ-ಯಾಗು
ಕರ-ಗತ
ಕರ-ಗದ
ಕರ-ಗ-ದಿ-ರಲು
ಕರ-ಗ-ಲಿಲ್ಲ
ಕರಗಿ
ಕರ-ಗಿತು
ಕರ-ಗಿ-ದ್ದು-ದ-ರಿಂದ
ಕರ-ಗಿಯೇ
ಕರ-ಗಿಸಿ
ಕರ-ಗಿ-ಸಿ-ಕೊಂ-ಡ-ವರೇ
ಕರ-ಗಿ-ಸಿ-ಕೊ-ಳ್ಳು-ವು-ದರ
ಕರ-ಗಿ-ಸಿ-ಬಿ-ಡು-ವು-ದಿಲ್ಲ
ಕರ-ಗಿ-ಹೋ-ಗು-ತ್ತಿತ್ತು
ಕರ-ಗಿ-ಹೋ-ದರು
ಕರ-ಗಿ-ಹೋ-ಯಿತು
ಕರ-ಗು-ತಿವೆ
ಕರ-ಗು-ತ್ತಿ-ದ್ದರೆ
ಕರಡಿ
ಕರ-ಡಿಗೆ
ಕರ-ಣ-ಗ-ಳಷ್ಟೆ
ಕರ-ತಾ-ಡನ
ಕರ-ತಾ-ಡ-ನ-ಹ-ರ್ಷೋ-ದ್ಗಾ-ರ-ಗಳ
ಕರ-ತಾ-ಡ-ನದ
ಕರ-ತಾ-ಡ-ನ-ದಿಂದ
ಕರ-ತಾ-ಡ-ನವು
ಕರ-ಪ-ತ್ರ-ಗಳ
ಕರ-ಪ-ತ್ರ-ಗಳನ್ನು
ಕರ-ಪ-ತ್ರ-ವೊಂ-ದ-ರಿಂದ
ಕರ-ರಾದ
ಕರ-ವಲ್ಲ
ಕರ-ವ-ಸ್ತ್ರ-ಗಳನ್ನು
ಕರಾಚಿ
ಕರಾ-ರು-ವಾ-ಕ್ಕಾಗಿ
ಕರಾ-ಳ-ಸ-ತ್ಯ-ವನ್ನು
ಕರಾ-ವಳಿ
ಕರಾ-ವ-ಳಿಯ
ಕರಾ-ವ-ಳಿ-ಯಲ್ಲಿ
ಕರಿ-ಕ-ರಾಳ
ಕರಿ-ಯ-ರಿಗೇ
ಕರೀ
ಕರು
ಕರು-ಣಾ-ಜ-ನ-ಕ-ಭಾ-ವ-ದಿಂದ
ಕರು-ಣಾ-ನಿ-ಧಿಯೆ
ಕರು-ಣಾ-ಪೂ-ರ್ಣ-ವಾದ
ಕರು-ಣಾ-ಪೂ-ರ್ಣ-ವಾ-ದದ್ದು
ಕರು-ಣಾ-ಮಯ
ಕರು-ಣಾ-ಸಾ-ಗರ
ಕರು-ಣಿಸ
ಕರು-ಣಿ-ಸಿ-ದರೂ
ಕರು-ಣಿ-ಸಿದ್ದೇ
ಕರುಣೆ
ಕರು-ಣೆಯ
ಕರು-ಣೆ-ಯನ್ನು
ಕರು-ಣೆ-ಯಿಂದ
ಕರು-ಣೆ-ಯಿಡು
ಕರು-ಳಿಗೆ
ಕರೆ
ಕರೆ-ಕ-ರೆ-ಯಾ-ಗು-ವಂ-ತಹ
ಕರೆ-ಗಳು
ಕರೆಗೆ
ಕರೆ-ತಂದ
ಕರೆ-ತಂ-ದ-ದ್ದ-ರಿಂದ
ಕರೆ-ತಂ-ದರು
ಕರೆ-ತಂ-ದ-ವನು
ಕರೆ-ತಂ-ದಿತು
ಕರೆ-ತಂ-ದಿ-ರ-ಲಿಲ್ಲ
ಕರೆ-ತಂದು
ಕರೆ-ತ-ರ-ಲಾ-ಯಿತು
ಕರೆ-ತ-ರಲು
ಕರೆ-ತ-ರು-ತ್ತಾನೆ
ಕರೆ-ತ-ರುವ
ಕರೆದ
ಕರೆ-ದರು
ಕರೆ-ದರೋ
ಕರೆ-ದಾಗ
ಕರೆ-ದಿ-ದ್ದಂತೆ
ಕರೆ-ದಿ-ದ್ದರೋ
ಕರೆದು
ಕರೆ-ದು-ಕೊಂಡು
ಕರೆ-ದು-ಕೊ-ಳ್ಳಲು
ಕರೆ-ದು-ಕೊ-ಳ್ಳು-ತ್ತಿ-ದ್ದವ
ಕರೆ-ದು-ಕೊ-ಳ್ಳುವ
ಕರೆ-ದು-ಕೊ-ಳ್ಳು-ವಂ-ತಾ-ದೀತು
ಕರೆ-ದು-ಕೊ-ಳ್ಳು-ವುದು
ಕರೆ-ದು-ದಕ್ಕೆ
ಕರೆ-ದುದು
ಕರೆ-ದೊಯ್ದ
ಕರೆ-ದೊ-ಯ್ದದ್ದು
ಕರೆ-ದೊ-ಯ್ದರು
ಕರೆ-ದೊ-ಯ್ದಾ-ಗಲೂ
ಕರೆ-ದೊ-ಯ್ದಿ-ದ್ದರು
ಕರೆ-ದೊಯ್ದು
ಕರೆ-ದೊಯ್ಯ
ಕರೆ-ದೊ-ಯ್ಯ-ಬೇಕು
ಕರೆ-ದೊ-ಯ್ಯ-ಬೇ-ಕೆಂ-ಬುದು
ಕರೆ-ದೊ-ಯ್ಯ-ಲಾ-ಗಿತ್ತು
ಕರೆ-ದೊ-ಯ್ಯ-ಲಾ-ಯಿತು
ಕರೆ-ದೊ-ಯ್ಯಲು
ಕರೆ-ದೊಯ್ಯು
ಕರೆ-ದೊ-ಯ್ಯು-ತ್ತದೆ
ಕರೆ-ದೊ-ಯ್ಯು-ತ್ತಾರೆ
ಕರೆ-ದೊ-ಯ್ಯುವ
ಕರೆ-ದೊ-ಯ್ಯು-ವುದು
ಕರೆ-ದೊ-ಯ್ಯು-ವು-ದೆಂದು
ಕರೆ-ನೀಡಿ
ಕರೆಯ
ಕರೆ-ಯನ್ನು
ಕರೆ-ಯ-ಬಲ್ಲೆ
ಕರೆ-ಯ-ಬ-ಹು-ದಾ-ದರೆ
ಕರೆ-ಯ-ಬ-ಹುದು
ಕರೆ-ಯ-ಬ-ಹುದೆ
ಕರೆ-ಯ-ಬೇ-ಕಾ-ಗಿ-ಬ-ರ-ಲಿಲ್ಲ
ಕರೆ-ಯ-ಬೇಕು
ಕರೆ-ಯ-ಲಾ-ಗಿತ್ತು
ಕರೆ-ಯ-ಲ್ಪ-ಡುವ
ಕರೆ-ಯ-ಲ್ಪ-ಡು-ವುದು
ಕರೆ-ಯು-ತ್ತಾರೆ
ಕರೆ-ಯು-ತ್ತಿದ್ದ
ಕರೆ-ಯು-ತ್ತಿ-ದ್ದರು
ಕರೆ-ಯು-ತ್ತಿ-ದ್ದ-ರು-ಇ-ವರು
ಕರೆ-ಯು-ತ್ತಿ-ದ್ದರೋ
ಕರೆ-ಯು-ತ್ತಿ-ದ್ದುದು
ಕರೆ-ಯು-ತ್ತೇನೆ
ಕರೆ-ಯು-ತ್ತೇ-ವೆಯೋ
ಕರೆ-ಯುವ
ಕರೆ-ಯು-ವ-ವ-ರಲ್ಲಿ
ಕರೆ-ಯು-ವಷ್ಟು
ಕರೆ-ಯು-ವಿರೋ
ಕರೆ-ಯು-ವು-ದರ
ಕರೆ-ಯು-ವು-ದಾ-ದ-ರೆ-ನೆ-ನ-ಪಿ-ರಲಿ
ಕರೆ-ಯು-ವುದು
ಕರೆ-ಯು-ವೆಯಾ
ಕರೆಯೂ
ಕರೆ-ಸ-ಬಲ್ಲೆ
ಕರೆ-ಸ-ಬೇ-ಕಾ-ದರೇ
ಕರೆ-ಸ-ಲಾ-ಯಿತು
ಕರೆಸಿ
ಕರೆ-ಸಿ-ಕೊಂ-ಡದ್ದು
ಕರೆ-ಸಿ-ಕೊಂ-ಡರು
ಕರೆ-ಸಿ-ಕೊಂ-ಡಿದ್ದ
ಕರೆ-ಸಿ-ಕೊ-ಳ್ಳಲು
ಕರೆ-ಸಿ-ದರು
ಕರ್ಜ-ನ-ನಿಂದ
ಕರ್ತವ್ಯ
ಕರ್ತ-ವ್ಯ-ಇ-ವು-ಗಳನ್ನು
ಕರ್ತ-ವ್ಯ-ಗಳನ್ನು
ಕರ್ತ-ವ್ಯದ
ಕರ್ತ-ವ್ಯ-ದಲ್ಲಿ
ಕರ್ತ-ವ್ಯ-ನಿ-ಷ್ಠೆಯ
ಕರ್ತ-ವ್ಯ-ಬು-ದ್ಧಿ-ಯಿಂದ
ಕರ್ತ-ವ್ಯ-ವನ್ನು
ಕರ್ತ-ವ್ಯ-ವ-ಲ್ಲವೆ
ಕರ್ತ-ವ್ಯ-ವಾ-ಗಿ-ರ-ಬೇಕು
ಕರ್ತ-ವ್ಯ-ವಾ-ಯಿತು
ಕರ್ತ-ವ್ಯ-ವೇನೆಂದರೆ
ಕರ್ತೃ
ಕರ್ತೃ-ವನ್ನು
ಕರ್ನನ್
ಕರ್ನ-ನ್ನಿಗೆ
ಕರ್ನ-ನ್ನಿ-ನಲ್ಲಿ
ಕರ್ನ-ನ್ನಿ-ನಿಂದ
ಕರ್ನಲ
ಕರ್ನಲ್
ಕರ್ನಾ-ಟ-ಕ-ದಲ್ಲಿ
ಕರ್ಮ
ಕರ್ಮ-ಕಾಂಡ
ಕರ್ಮ-ಕಾಂ-ಡದ
ಕರ್ಮ-ಕಾಂ-ಡವು
ಕರ್ಮ-ಕಾಂ-ಡವೇ
ಕರ್ಮ-ಕು-ಶ-ಲಿ-ಯನ್ನು
ಕರ್ಮ-ಕ್ಕ-ನು-ಸಾ-ರ-ವಾಗಿ
ಕರ್ಮಕ್ಕೆ
ಕರ್ಮ-ಗಳ
ಕರ್ಮ-ಗಳನ್ನು
ಕರ್ಮ-ಗಳನ್ನೂ
ಕರ್ಮ-ಣ್ಯೇ-ವಾ-ಧಿ-ಕಾ-ರಸ್ತೇ
ಕರ್ಮದ
ಕರ್ಮ-ದಂತೆ
ಕರ್ಮ-ದಿಂದ
ಕರ್ಮ-ದೋ-ಷ-ದಿಂದ
ಕರ್ಮ-ದೋ-ಷ-ವಾ-ಗು-ವುದು
ಕರ್ಮ-ಭೂ-ಮಿ-ಕೆಗೆ
ಕರ್ಮ-ಭೂ-ಮಿ-ಯೊಂ-ದಿ-ದ್ದರೆ
ಕರ್ಮ-ಭ್ರ-ಷ್ಟ-ರಾ-ಗಿ-ದ್ದಾರೆ
ಕರ್ಮ-ಮಾ-ಡ-ಬಲ್ಲ
ಕರ್ಮ-ಮಾಡಿ
ಕರ್ಮ-ಮಾ-ಡುವ
ಕರ್ಮ-ಮಾ-ಡು-ವ-ವರ
ಕರ್ಮ-ಯೋಗ
ಕರ್ಮ-ಯೋ-ಗ-ಗಳನ್ನು
ಕರ್ಮ-ಯೋಗಿ
ಕರ್ಮ-ಯೋ-ಗಿ-ಗ-ಳಾದ
ಕರ್ಮ-ಯೋ-ಗಿ-ಗಳು
ಕರ್ಮ-ಯೋ-ಗಿಯ
ಕರ್ಮ-ರಾ-ಗಿ-ದ್ದಾ-ರೆಯೇ
ಕರ್ಮ-ವಂತೆ
ಕರ್ಮ-ವನ್ನು
ಕರ್ಮ-ವ-ಲ್ಲವೆ
ಕರ್ಮ-ವಾ-ದರೆ
ಕರ್ಮವು
ಕರ್ಮವೂ
ಕರ್ಮ-ವೆ-ನ್ನುವ
ಕರ್ಮವೊ
ಕರ್ಮ-ಸಿ-ದ್ಧಾಂತ
ಕರ್ಮ-ಸಿ-ದ್ಧಾಂ-ತ-ವನ್ನು
ಕರ್ಮ-ಸಿ-ದ್ಧಾಂ-ತ-ವೆಂ-ದೊ-ಡನೆ
ಕಲ-ಶ-ವ-ನ್ನಿ-ರಿ-ಸಿ-ದರು
ಕಲಾ
ಕಲಾ-ಕಾರ
ಕಲಾ-ಕೃತಿ
ಕಲಾ-ಕೃ-ತಿ-ಗಳನ್ನು
ಕಲಾ-ಕೃ-ತಿ-ಯಾದ
ಕಲಾ-ವಿದೆ
ಕಲಾ-ವಿ-ನ್ಯಾ-ಸ-ಗಳನ್ನು
ಕಲಾ-ಶಾ-ಲೆ-ಯೊಂ-ದರ
ಕಲಿ-ಕೆಗೆ
ಕಲಿ-ಗಾ-ಲ-ದಲ್ಲಿ
ಕಲಿತ
ಕಲಿ-ತದ್ದು
ಕಲಿ-ತಾಗ
ಕಲಿ-ತಿ-ದ್ದೇನೆ
ಕಲಿ-ತಿ-ದ್ದೇವೆ
ಕಲಿ-ತಿ-ರ-ಲಿಲ್ಲ
ಕಲಿ-ತಿ-ರು-ವು-ದೆಲ್ಲ
ಕಲಿತು
ಕಲಿ-ತುಕೊ
ಕಲಿ-ತು-ಕೊಂಡು
ಕಲಿ-ತು-ಕೊ-ಳ್ಳ-ಬೇ-ಕಾ-ದದ್ದು
ಕಲಿ-ತು-ಕೊ-ಳ್ಳ-ಲಾ-ರನೋ
ಕಲಿ-ತು-ಕೊಳ್ಳಿ
ಕಲಿಯ
ಕಲಿ-ಯ-ದಿ-ದ್ದರೆ
ಕಲಿ-ಯ-ಬ-ಹುದು
ಕಲಿ-ಯ-ಬೇ-ಕಾ-ಗಿ-ರುವ
ಕಲಿ-ಯ-ಬೇ-ಕಾ-ದದ್ದೂ
ಕಲಿ-ಯ-ಬೇ-ಕಾ-ದುದು
ಕಲಿ-ಯ-ಬೇ-ಕಾದ್ದು
ಕಲಿ-ಯ-ಬೇ-ಕಾ-ಯಿತು
ಕಲಿ-ಯ-ಬೇಕು
ಕಲಿ-ಯಲಿ
ಕಲಿ-ಯಲು
ಕಲಿ-ಯ-ಲೇ-ಬೇಕು
ಕಲಿ-ಯಿರಿ
ಕಲಿ-ಯಿ-ರಿಯ
ಕಲಿ-ಯಿ-ರಿ-ಯಲ್ಲಿ
ಕಲಿ-ಯು-ಗ-ದಲ್ಲಿ
ಕಲಿ-ಯುತ್ತ
ಕಲಿ-ಯು-ತ್ತಿ-ರುವ
ಕಲಿ-ಯುವ
ಕಲಿ-ಯು-ವಂ-ತಾ-ಗಲಿ
ಕಲಿ-ಯು-ವುದು
ಕಲಿಸ
ಕಲಿ-ಸ-ಬೇಕು
ಕಲಿ-ಸ-ಲಾ-ಗು-ತ್ತದೆ
ಕಲಿ-ಸಲು
ಕಲಿಸಿ
ಕಲಿ-ಸಿ-ದರೆ
ಕಲಿ-ಸು-ತ್ತಿತ್ತು
ಕಲಿ-ಸು-ವು-ದ-ಕ್ಕಾಗಿ
ಕಲಿ-ಸು-ವು-ದ-ಕ್ಕಾ-ಗಿಯೇ
ಕಲೆ
ಕಲೆ-ಕ್ಟ-ರರೂ
ಕಲೆ-ತಿ-ದ್ದರು
ಕಲೆತು
ಕಲೆ-ಯನ್ನು
ಕಲೆ-ಯಲ್ಲಿ
ಕಲೆಯೇ
ಕಲೆ-ಹಾ-ಕು-ವುದು
ಕಲ್ಕತ್ತ
ಕಲ್ಕ-ತ್ತಕ್ಕೆ
ಕಲ್ಕ-ತ್ತ-ಕ್ಕೊಂದು
ಕಲ್ಕ-ತ್ತ-ಗಳಲ್ಲಿ
ಕಲ್ಕ-ತ್ತ-ಗ-ಳಲ್ಲೂ
ಕಲ್ಕ-ತ್ತದ
ಕಲ್ಕ-ತ್ತ-ದತ್ತ
ಕಲ್ಕ-ತ್ತ-ದ-ಲ್ಲಾ-ದರೂ
ಕಲ್ಕ-ತ್ತ-ದಲ್ಲಿ
ಕಲ್ಕ-ತ್ತ-ದ-ಲ್ಲಿದ್ದ
ಕಲ್ಕ-ತ್ತ-ದ-ಲ್ಲಿ-ದ್ದಾಗ
ಕಲ್ಕ-ತ್ತ-ದ-ಲ್ಲಿ-ದ್ದಾ-ಗ-ಲೆಲ್ಲ
ಕಲ್ಕ-ತ್ತ-ದ-ಲ್ಲಿನ
ಕಲ್ಕ-ತ್ತ-ದಲ್ಲೆಲ್ಲ
ಕಲ್ಕ-ತ್ತ-ದಲ್ಲೇ
ಕಲ್ಕ-ತ್ತ-ದ-ಲ್ಲೊಂದು
ಕಲ್ಕ-ತ್ತ-ದ-ವ-ರೆಗೆ
ಕಲ್ಕ-ತ್ತ-ದಿಂದ
ಕಲ್ಕ-ತ್ತ-ವಂತೂ
ಕಲ್ಕ-ತ್ತ-ವನ್ನು
ಕಲ್ಕ-ತ್ತವು
ಕಲ್ಕ-ತ್ತವೇ
ಕಲ್ಗುಂ-ಡು-ಗ-ಳೊ-ಡನೆ
ಕಲ್ಚರ್
ಕಲ್ಪನೆ
ಕಲ್ಪ-ನೆ-ಗಳನ್ನು
ಕಲ್ಪ-ನೆ-ಗ-ಳಾ-ವು-ದನ್ನೂ
ಕಲ್ಪ-ನೆ-ಗ-ಳಿ-ದ್ದುವು
ಕಲ್ಪ-ನೆ-ಗ-ಳಿರು
ಕಲ್ಪ-ನೆ-ಗಳು
ಕಲ್ಪ-ನೆ-ಗಳೂ
ಕಲ್ಪ-ನೆಗೂ
ಕಲ್ಪ-ನೆಯ
ಕಲ್ಪ-ನೆ-ಯನ್ನು
ಕಲ್ಪ-ನೆ-ಯಲ್ಲ
ಕಲ್ಪ-ನೆ-ಯಾ-ಗದು
ಕಲ್ಪ-ನೆ-ಯಾ-ಗು-ತ್ತಿತ್ತು
ಕಲ್ಪ-ನೆ-ಯಾ-ದರೂ
ಕಲ್ಪ-ನೆ-ಯಾ-ದೀತು
ಕಲ್ಪ-ನೆಯು
ಕಲ್ಪ-ನೆಯೂ
ಕಲ್ಪ-ನೆಯೇ
ಕಲ್ಪ-ನೆ-ಯೊಂ-ದನ್ನು
ಕಲ್ಪಿಸ
ಕಲ್ಪಿ-ಸಲಾ
ಕಲ್ಪಿ-ಸ-ಲಾ-ಗಿತ್ತು
ಕಲ್ಪಿ-ಸ-ಲಾಗಿದೆ
ಕಲ್ಪಿ-ಸ-ಲಾ-ಯಿತು
ಕಲ್ಪಿಸಿ
ಕಲ್ಪಿ-ಸಿ-ಕೊ-ಟ್ಟಿತು
ಕಲ್ಪಿ-ಸಿ-ಕೊ-ಟ್ಟಿ-ದ್ದಳು
ಕಲ್ಪಿ-ಸಿ-ಕೊಟ್ಟು
ಕಲ್ಪಿ-ಸಿ-ಕೊ-ಡು-ತ್ತಿ-ದ್ದರು
ಕಲ್ಪಿ-ಸಿ-ಕೊ-ಳ್ಳ-ಬ-ಹುದು
ಕಲ್ಪಿ-ಸಿ-ಕೊ-ಳ್ಳ-ಬೇ-ಕಾ-ಗು-ತ್ತದೆ
ಕಲ್ಪಿ-ಸಿ-ಕೊ-ಳ್ಳಲು
ಕಲ್ಪಿ-ಸಿ-ಯಾನೆ
ಕಲ್ಯಾಣ
ಕಲ್ಯಾ-ಣ-ಕಾರ್ಯ
ಕಲ್ಯಾ-ಣ-ಕ್ಕಾಗಿ
ಕಲ್ಯಾ-ಣಕ್ಕೆ
ಕಲ್ಯಾ-ಣದ
ಕಲ್ಯಾಣಾ
ಕಲ್ಯಾ-ಣಾ-ನಂ-ದರು
ಕಲ್ಲಿನ
ಕಲ್ಲಿ-ನಂತೆ
ಕಲ್ಲಿ-ನಷ್ಟು
ಕಲ್ಲು
ಕಲ್ಲೂ
ಕಲ್ಲೋ-ಲ-ವ-ನ್ನುಂ-ಟು-ಮಾ-ಡಿ-ದುವು
ಕಳಂ-ಕ-ದಿಂದ
ಕಳಂ-ಕ-ವನ್ನು
ಕಳ-ಕಳಿ
ಕಳ-ಕ-ಳಿಯ
ಕಳ-ಕ-ಳಿ-ಯಿಂದ
ಕಳ-ಚ-ಬ-ಲ್ಲ-ವ-ರಾಗು
ಕಳ-ಚಲು
ಕಳಚಿ
ಕಳ-ಚಿ-ಕೊ-ಳ್ಳಲು
ಕಳಪೆ
ಕಳ-ಪೆಯ
ಕಳ-ವಳ
ಕಳ-ವ-ಳ-ಗೊಂ-ಡರು
ಕಳ-ವ-ಳ-ಗೊಂ-ಡಿ-ದ್ದಾರೆ
ಕಳ-ವ-ಳ-ಗೊ-ಳ್ಳ-ಬೇ-ಕಾ-ಗಿಲ್ಲ
ಕಳ-ವ-ಳ-ಗೊ-ಳ್ಳು-ತ್ತಿ-ದ್ದರು
ಕಳ-ವ-ಳ-ಗೊ-ಳ್ಳುವ
ಕಳಿ-ಸ-ಬೇಕು
ಕಳಿ-ಸ-ಲಾ-ಯಿತು
ಕಳಿ-ಸ-ಲ್ಪ-ಟ್ಟ-ವ-ನು-ಭ-ಗ-ವಂ-ತನ
ಕಳಿಸಿ
ಕಳಿ-ಸಿ-ಕೊಟ್ಟ
ಕಳಿ-ಸಿ-ಕೊ-ಟ್ಟರು
ಕಳಿ-ಸಿ-ಕೊಟ್ಟಿ
ಕಳಿ-ಸಿ-ಕೊ-ಟ್ಟಿದ್ದ
ಕಳಿ-ಸಿ-ಕೊ-ಡ-ಬೇಕು
ಕಳಿ-ಸಿ-ಕೊ-ಡ-ಬೇ-ಕೆಂದು
ಕಳಿ-ಸಿ-ಕೊ-ಡಲು
ಕಳಿ-ಸಿ-ಕೊ-ಡುತ್ತ
ಕಳಿ-ಸಿ-ಕೊ-ಡು-ತ್ತಾಳೆ
ಕಳಿ-ಸಿ-ಕೊ-ಡು-ತ್ತಿದ್ದ
ಕಳಿ-ಸಿ-ಕೊ-ಡು-ತ್ತೇನೆ
ಕಳಿ-ಸಿ-ಕೊ-ಡು-ವಂತೆ
ಕಳಿ-ಸಿ-ಕೊ-ಡು-ವು-ದಾಗಿ
ಕಳಿ-ಸಿ-ಕೊ-ಡು-ವೆಯಾ
ಕಳಿ-ಸಿತು
ಕಳಿ-ಸಿದ
ಕಳಿ-ಸಿ-ದ-ಮ-ಠ-ದಲ್ಲಿ
ಕಳಿ-ಸಿ-ದರು
ಕಳಿ-ಸಿದ್ದ
ಕಳಿ-ಸಿ-ದ್ದರು
ಕಳಿ-ಸಿ-ಬಿ-ಡು-ತ್ತಿ-ದ್ದರು
ಕಳಿ-ಸುತ್ತ
ಕಳಿ-ಸು-ತ್ತಲೇ
ಕಳಿ-ಸು-ತ್ತಿ-ದ್ದೇನೆ
ಕಳಿ-ಸು-ತ್ತಿ-ರು-ವುದು
ಕಳಿ-ಸುತ್ತೀ
ಕಳಿ-ಸು-ವಂತೆ
ಕಳಿ-ಸು-ವುದು
ಕಳು-ಹಿಸಿ
ಕಳೆ
ಕಳೆ-ಕ-ಟ್ಟು-ವಂತೆ
ಕಳೆ-ಕೊ-ಡು-ವಂ-ತಿ-ರುವ
ಕಳೆ-ಗ-ಳಂತೆ
ಕಳೆದ
ಕಳೆ-ದಂ-ತಹ
ಕಳೆ-ದಂತೆ
ಕಳೆ-ದಂ-ತೆಯೂ
ಕಳೆ-ದಂ-ತೆಲ್ಲ
ಕಳೆ-ದ-ನಂ-ತರ
ಕಳೆ-ದರು
ಕಳೆದು
ಕಳೆ-ದು-ಕೊಂಡ
ಕಳೆ-ದು-ಕೊಂ-ಡಂತೆ
ಕಳೆ-ದು-ಕೊಂ-ಡರು
ಕಳೆ-ದು-ಕೊಂ-ಡರೆ
ಕಳೆ-ದು-ಕೊಂ-ಡಿ-ದ್ದೇನೆ
ಕಳೆ-ದು-ಕೊಂ-ಡಿ-ರು-ವುದನ್ನು
ಕಳೆ-ದು-ಕೊಂಡು
ಕಳೆ-ದು-ಕೊಳ್ಳ
ಕಳೆ-ದು-ಕೊ-ಳ್ಳದೆ
ಕಳೆ-ದು-ಕೊ-ಳ್ಳಲಿ
ಕಳೆ-ದು-ಕೊ-ಳ್ಳು-ತ್ತಿದೆ
ಕಳೆ-ದು-ಕೊ-ಳ್ಳು-ತ್ತಿದ್ದ
ಕಳೆ-ದು-ಕೊ-ಳ್ಳು-ತ್ತಿ-ದ್ದಾರೆ
ಕಳೆ-ದು-ಕೊ-ಳ್ಳುವ
ಕಳೆ-ದು-ಕೊ-ಳ್ಳು-ವು-ದಿಲ್ಲ
ಕಳೆ-ದುವು
ಕಳೆ-ದು-ಹೋ-ಗಿವೆ
ಕಳೆ-ದು-ಹೋ-ಗು-ತ್ತಿತ್ತು
ಕಳೆ-ದು-ಹೋದ
ಕಳೆ-ದು-ಹೋ-ದುವು
ಕಳೆ-ದು-ಹೋ-ಯಿತು
ಕಳೆ-ದೆ-ರಡು
ಕಳೆ-ಯ-ಬ-ಲ್ಲುದೆ
ಕಳೆ-ಯ-ಬೇ-ಕಲ್ಲ
ಕಳೆ-ಯ-ಬೇಕಾ
ಕಳೆ-ಯ-ಬೇ-ಕಾ-ಯಿತು
ಕಳೆ-ಯ-ಬೇಕು
ಕಳೆ-ಯ-ಬೇ-ಕೆಂದು
ಕಳೆ-ಯ-ಲಾ-ರಂ-ಭಿ-ಸಿ-ದ್ದರು
ಕಳೆ-ಯ-ಲಿ-ದ್ದ-ರು-ಯಾರೂ
ಕಳೆ-ಯಲು
ಕಳೆ-ಯ-ಲೆಂದು
ಕಳೆ-ಯಿತು
ಕಳೆ-ಯಿತೊ
ಕಳೆ-ಯುತ್ತ
ಕಳೆ-ಯು-ತ್ತಿದೆ
ಕಳೆ-ಯು-ತ್ತಿ-ದ್ದರು
ಕಳೆ-ಯುವ
ಕಳೆ-ಯು-ವಂ-ತಾ-ಗಲಿ
ಕಳೆ-ಯು-ವು-ದಾ-ಗಿಯೂ
ಕಳ್ಳ-ನಾ-ಗಲಿ
ಕಳ್ಳ-ನಾದ
ಕಳ್ಳ-ರಿ-ಗಾಗಿ
ಕಳ್ಳರೂ
ಕಳ್ಳ-ಸಂ-ನ್ಯಾ-ಸಿ-ಯಿ-ರ-ಬ-ಹುದು
ಕವನ
ಕವ-ನ-ಗಳನ್ನು
ಕವ-ನದ
ಕವ-ನ-ವನ್ನು
ಕವ-ನವು
ಕವ-ನ-ವೊಂ-ದನ್ನು
ಕವಿ-ಗಳು
ಕವಿ-ತೆಯ
ಕವಿ-ತೆ-ಯೊಂ-ದನ್ನು
ಕವಿತ್ವ
ಕವಿದ
ಕವಿ-ದಂತೆ
ಕವಿ-ದಿತ್ತು
ಕವಿಯ
ಕವಿ-ಯುತ್ತಿ
ಕವಿ-ಯು-ತ್ತಿದ್ದು
ಕವಿ-ರಾ-ಜ-ರಿಂದ
ಕವಿ-ರಾ-ಜ-ರೊ-ಬ್ಬ-ರಿಂದ
ಕವಿ-ವಾ-ಕ್ಯದ
ಕವಿ-ಸ-ಹಜ
ಕವಿ-ಸುವ
ಕವಿ-ಸು-ವಂ-ತಹ
ಕಷ್ಟ
ಕಷ್ಟ-ಕಂ-ಟ-ಕ-ಗಳನ್ನು
ಕಷ್ಟ-ಕರ
ಕಷ್ಟ-ಕ-ರ-ವಾ-ಗಿತ್ತು
ಕಷ್ಟ-ಕ-ರ-ವಾ-ಗಿದ್ದು
ಕಷ್ಟ-ಕ-ರ-ವಾ-ಗುತ್ತ
ಕಷ್ಟ-ಕ-ರ-ವಾದ
ಕಷ್ಟ-ಕ-ರ-ವಾ-ದದ್ದು
ಕಷ್ಟಕ್ಕೆ
ಕಷ್ಟ-ಕ್ಕೇ-ನಾ-ದರೂ
ಕಷ್ಟ-ಗಳನ್ನು
ಕಷ್ಟ-ಗಳನ್ನೆಲ್ಲ
ಕಷ್ಟ-ಗ-ಳಿ-ಗೆಲ್ಲ
ಕಷ್ಟ-ಗಳು
ಕಷ್ಟ-ಗ-ಳು-ಎ-ಲ್ಲ-ವನ್ನೂ
ಕಷ್ಟ-ಗಳೂ
ಕಷ್ಟ-ಗ-ಳೇ-ನಾ-ದರೂ
ಕಷ್ಟದ
ಕಷ್ಟ-ದಲ್ಲಿ
ಕಷ್ಟ-ದ-ಲ್ಲಿ-ದ್ದಂತೆ
ಕಷ್ಟ-ದ-ಲ್ಲಿ-ರು-ವ-ವರ
ಕಷ್ಟ-ದಿಂದ
ಕಷ್ಟ-ದ್ದಾ-ಗಿತ್ತು
ಕಷ್ಟ-ಪಟ್ಟು
ಕಷ್ಟ-ಪಟ್ಟೆ
ಕಷ್ಟ-ಪ-ಡ-ಬೇ-ಕಾ-ಗಿತ್ತು
ಕಷ್ಟ-ಪ-ಡು-ತ್ತಾರೆ
ಕಷ್ಟ-ಪ-ಡು-ತ್ತೇವೆ
ಕಷ್ಟ-ವನ್ನು
ಕಷ್ಟ-ವಾ-ಗದು
ಕಷ್ಟ-ವಾ-ಗ-ಬ-ಹು-ದೆಂದು
ಕಷ್ಟ-ವಾ-ಗ-ಲಿಲ್ಲ
ಕಷ್ಟ-ವಾ-ಗಿ-ದೆಯೇ
ಕಷ್ಟ-ವಾ-ಗಿ-ಬಿ-ಟ್ಟಿದೆ
ಕಷ್ಟ-ವಾ-ಗು-ತ್ತದೆ
ಕಷ್ಟ-ವಾ-ಗು-ತ್ತಿದೆ
ಕಷ್ಟ-ವಾ-ಗು-ತ್ತಿ-ದ್ದು-ದುಂಟು
ಕಷ್ಟ-ವಾ-ಗು-ವು-ದಿಲ್ಲ
ಕಷ್ಟ-ವಾ-ದರೂ
ಕಷ್ಟ-ವಾ-ಯಿ-ತಲ್ಲಾ
ಕಷ್ಟ-ವಾ-ಯಿತು
ಕಷ್ಟ-ವಿದೆ
ಕಷ್ಟ-ವಿಲ್ಲ
ಕಷ್ಟವೂ
ಕಷ್ಟ-ವೆಂದರೆ
ಕಷ್ಟವೇ
ಕಷ್ಟ-ಸಂ-ಕ-ಟ-ಗಳ
ಕಷ್ಟ-ಸು-ಖ-ಗಳನ್ನು
ಕಸ-ಪೊ-ರ-ಕೆ-ಯನ್ನು
ಕಸ-ವನ್ನು
ಕಸ-ವಾಗಿ
ಕಸಿ-ದು-ಕೊಂ-ಡು-ಬಿ-ಡು-ತ್ತಾರೋ
ಕಸಿ-ದು-ಕೊ-ಳ್ಳಲು
ಕಸಿ-ವಿಸಿ
ಕಸಿ-ವಿ-ಸಿ-ಗೊಂ-ಡರು
ಕಸಿ-ವಿ-ಸಿ-ಗೊಂ-ಡ-ವರು
ಕಸಿ-ವಿ-ಸಿ-ಯಾ-ಯಿ-ತಂತೆ
ಕಸೂತಿ
ಕಹಳೆ
ಕಹ-ಳೆ-ಯೂದಿ
ಕಹಿ
ಕಹಿ-ವಿ-ವಾ-ದಾ-ಸ್ಪದ
ಕಹಿ-ರ-ಸ-ವನ್ನು
ಕಾಂಗ್ರೆ-ಸಿ-ಗರು
ಕಾಂಗ್ರೆಸ್
ಕಾಂಗ್ರೆ-ಸ್ಸಿ-ಗ-ರಲ್ಲಿ
ಕಾಂಗ್ರೆ-ಸ್ಸಿನ
ಕಾಂಗ್ರೆಸ್ಸು
ಕಾಂಚ-ನ-ಗಳನ್ನು
ಕಾಂಚ-ನದ
ಕಾಂಚ-ನ-ವನ್ನು
ಕಾಂತಿ-ಯನ್ನು
ಕಾಂತಿಯು
ಕಾಂತಿ-ಯೊಂದು
ಕಾಂಪೌಂಡು
ಕಾಗ-ದ-ಕ್ಕಿ-ಳಿದು
ಕಾಗ-ದ-ದಲ್ಲಿ
ಕಾಗ-ದ-ವನ್ನು
ಕಾಟನ್
ಕಾಟ-ವನ್ನು
ಕಾಟೇಜ್
ಕಾಡ-ಲಾ-ರಂ-ಭಿ-ಸಿದ
ಕಾಡಾ-ನೆ-ಯಂ-ತಹ
ಕಾಡಿ
ಕಾಡಿನ
ಕಾಡಿ-ನಲ್ಲಿ
ಕಾಡು-ಕೋ-ಳಿಯು
ಕಾಡು-ತ್ತಿದ್ದ
ಕಾಡು-ತ್ತಿದ್ದಂ
ಕಾಡು-ತ್ತಿ-ರುವ
ಕಾಣ
ಕಾಣದ
ಕಾಣ-ದಂತೆ
ಕಾಣ-ದಿ-ದ್ದಾಗ
ಕಾಣ-ದಿ-ರದು
ಕಾಣ-ದಿ-ರು-ತ್ತಿ-ರ-ಲಿಲ್ಲ
ಕಾಣ-ದಿ-ರು-ವಂ-ತೆ-ಭಾ-ಸ-ವಾ-ಗು-ತ್ತಿತ್ತು
ಕಾಣದು
ಕಾಣ-ಬ-ರದ
ಕಾಣ-ಬ-ರುವ
ಕಾಣ-ಬ-ಲ್ಲರು
ಕಾಣ-ಬ-ಲ್ಲಿರಿ
ಕಾಣ-ಬ-ಲ್ಲೆಯಾ
ಕಾಣ-ಬಹು
ಕಾಣ-ಬ-ಹುದಾ
ಕಾಣ-ಬ-ಹು-ದಾ-ಗಿತ್ತು
ಕಾಣ-ಬ-ಹು-ದಾ-ಗಿದೆ
ಕಾಣ-ಬ-ಹುದು
ಕಾಣ-ಬ-ಹು-ದೇನೋ
ಕಾಣ-ಬೇಕು
ಕಾಣ-ಬೇ-ಕೆಂ-ದಲ್ಲ
ಕಾಣ-ಬೇ-ಕೆಂಬ
ಕಾಣ-ಲಾ-ರಿ-ರಾ-ದರೆ
ಕಾಣ-ಲಾ-ರೆ-ಯಾ-ದರೂ
ಕಾಣ-ಲಿಲ್ಲ
ಕಾಣಲು
ಕಾಣ-ಲೆಂದೇ
ಕಾಣ-ಲೇ-ಬೇಕು
ಕಾಣ-ಸಿ-ಕ್ಕಾವು
ಕಾಣ-ಸಿ-ಗು-ವು-ದಿಲ್ಲ
ಕಾಣ-ಸಿ-ಗು-ವುದು
ಕಾಣಿಕೆ
ಕಾಣಿ-ಕೆ-ಗಳನ್ನು
ಕಾಣಿ-ಕೆ-ಗ-ಳೊಂ-ದಿಗೆ
ಕಾಣಿ-ಕೆ-ಯನ್ನು
ಕಾಣಿ-ಕೆ-ಯಾಗಿ
ಕಾಣಿ-ಕೆ-ಯೆಂ-ದರೆ
ಕಾಣಿ-ಸ-ಬ-ಹುದು
ಕಾಣಿಸಿ
ಕಾಣಿ-ಸಿ-ಕೊಂಡ
ಕಾಣಿ-ಸಿ-ಕೊಂ-ಡದ್ದು
ಕಾಣಿ-ಸಿ-ಕೊಂ-ಡಿ-ತಾ-ದರೂ
ಕಾಣಿ-ಸಿ-ಕೊಂ-ಡಿತು
ಕಾಣಿ-ಸಿ-ಕೊಂ-ಡಿ-ರ-ಬೇಕು
ಕಾಣಿ-ಸಿ-ಕೊ-ಳ್ಳ-ಬ-ಹುದು
ಕಾಣಿ-ಸಿ-ಕೊ-ಳ್ಳಲಿ
ಕಾಣಿ-ಸಿ-ಕೊ-ಳ್ಳು-ತ್ತಿ-ದ್ದಂ-ತೆಯೇ
ಕಾಣಿ-ಸಿ-ಕೊ-ಳ್ಳು-ತ್ತಿ-ದ್ದರು
ಕಾಣಿ-ಸಿ-ಕೊ-ಳ್ಳು-ತ್ತಿವೆ
ಕಾಣಿ-ಸಿ-ದೆಯೆ
ಕಾಣಿ-ಸು-ತ್ತಿದೆ
ಕಾಣಿ-ಸು-ತ್ತಿದ್ದ
ಕಾಣಿ-ಸು-ತ್ತಿ-ಲ್ಲವೆ
ಕಾಣು
ಕಾಣುತ್ತ
ಕಾಣು-ತ್ತದೆ
ಕಾಣು-ತ್ತಿತ್ತು
ಕಾಣು-ತ್ತಿದೆ
ಕಾಣು-ತ್ತಿ-ದೆ-ಯಲ್ಲ
ಕಾಣು-ತ್ತಿದ್ದ
ಕಾಣು-ತ್ತಿ-ದ್ದರು
ಕಾಣು-ತ್ತಿ-ದ್ದ-ರೆಂ-ಬುದು
ಕಾಣು-ತ್ತಿ-ದ್ದಳು
ಕಾಣು-ತ್ತಿ-ದ್ದಾರೆ
ಕಾಣು-ತ್ತಿ-ದ್ದಾ-ರೆ-ಇಲ್ಲಿ
ಕಾಣು-ತ್ತಿ-ದ್ದಾಳೆ
ಕಾಣು-ತ್ತಿ-ದ್ದೇನೆ
ಕಾಣು-ತ್ತಿ-ದ್ದೇವೆ
ಕಾಣು-ತ್ತಿ-ದ್ದೇ-ವೆಯೋ
ಕಾಣು-ತ್ತಿ-ರ-ಲಿಲ್ಲ
ಕಾಣು-ತ್ತಿ-ರುವ
ಕಾಣು-ತ್ತಿ-ರು-ವ-ವ-ರೆಗೂ
ಕಾಣು-ತ್ತಿಲ್ಲ
ಕಾಣು-ತ್ತಿವೆ
ಕಾಣು-ತ್ತೇವೆ
ಕಾಣುವ
ಕಾಣು-ವಂ-ತಾ-ಗ-ಬೇಕು
ಕಾಣು-ವಂ-ತಾ-ಗಲಿ
ಕಾಣು-ವಂತೆ
ಕಾಣು-ವ-ವ-ರಿಗೆ
ಕಾಣು-ವಷ್ಟು
ಕಾಣು-ವಿರಿ
ಕಾಣು-ವುದನ್ನು
ಕಾಣು-ವು-ದಿಲ್ಲ
ಕಾಣು-ವುದು
ಕಾಣು-ವುದೇ
ಕಾಣುವೆ
ಕಾಣೆ
ಕಾಣೆ-ಯಾ-ಗು-ತ್ತಿದೆ
ಕಾಣ್ಕೆ-ಯ-ದೆಂದು
ಕಾತರ
ಕಾತ-ರ-ಗೊಂ-ಡಿತ್ತು
ಕಾತ-ರತೆ
ಕಾತ-ರದ
ಕಾತ-ರ-ದಿಂದ
ಕಾತ-ರ-ನಾ-ಗಿದ್ದ
ಕಾತ-ರ-ರಾಗಿ
ಕಾತ-ರ-ರಾ-ಗಿ-ದ್ದರು
ಕಾತ-ರ-ರಾ-ಗಿ-ದ್ದಾರೆ
ಕಾತ-ರ-ರಾ-ದರು
ಕಾತ-ರ-ಳಾಗಿ
ಕಾತ-ರ-ಳಾ-ಗಿ-ದ್ದಳು
ಕಾಥೇ-ವಾಡ
ಕಾಥೇ-ವಾ-ಡಕ್ಕೆ
ಕಾಥೇ-ವಾ-ಡದ
ಕಾಥೇ-ವಾ-ಡ-ದ-ಲ್ಲಿದ್ದ
ಕಾದ
ಕಾದಂ-ಬರಿ
ಕಾದ-ಮೇಲೆ
ಕಾದರು
ಕಾದಾ-ಡು-ತ್ತಿದ್ದ
ಕಾದಿ
ಕಾದಿತ್ತು
ಕಾದಿದ್ದ
ಕಾದಿ-ದ್ದರು
ಕಾದಿ-ದ್ದುವು
ಕಾದಿರು
ಕಾದಿ-ರು-ತ್ತಿ-ತ್ತು-ಹಾ-ಗಾ-ದರೆ
ಕಾದು
ಕಾದು-ಕು-ಳಿ-ತರು
ಕಾದು-ಕು-ಳಿ-ತಿ-ದ್ದಾರೆ
ಕಾದು-ನಿಂ-ತಿ-ದ್ದರು
ಕಾನ-ನವ
ಕಾನಿ-ಷ್ಕನ
ಕಾನೂ-ನಿನ
ಕಾನೂನು
ಕಾನ್ಫ-ರೆ-ನ್ಸಸ್
ಕಾನ್ಫ-ರೆ-ನ್ಸಸ್ನ
ಕಾನ್ಫ-ರೆ-ನ್ಸ-ಸ್ಸಿನ
ಕಾನ್ಸ್ಟಾಂ-ಟಿ-ನೋ-ಪಲ್
ಕಾನ್ಸ್ಟಾಂ-ಟಿ-ನೋ-ಪ-ಲ್ನಲ್ಲಿ
ಕಾನ್ಸ್ಟಾಂ-ಟಿ-ನೋ-ಪ-ಲ್ಲಿಗೆ
ಕಾಪಾ-ಡ-ಬೇಕು
ಕಾಪಾ-ಡಲು
ಕಾಪಾ-ಡಿ-ಕೊಂ-ಡಿ-ರು-ವು-ದಕ್ಕೆ
ಕಾಪಾ-ಡಿ-ಕೊ-ಳ್ಳ-ಬೇ-ಕೆಂಬ
ಕಾಪಾ-ಡಿ-ಕೊ-ಳ್ಳು-ತ್ತಿ-ದ್ದರೋ
ಕಾಪಾ-ಡು-ತ್ತಾನೆ
ಕಾಪಾ-ಡು-ವ-ವರು
ಕಾಪಾ-ಡು-ವುದು
ಕಾಪಿ-ಹೊ-ಡೆ-ಯುವ
ಕಾಮ
ಕಾಮ-ಕಾಂ-ಚ-ನ-ವನ್ನು
ಕಾಮ-ಕಾಂ-ಚನ
ಕಾಮ-ಕಾಂ-ಚ-ನ-ಗಳ
ಕಾಮ-ದೇ-ವ-ನೆಂದೂ
ಕಾಮ-ನಿಲ್ಲ
ಕಾಮ-ರೂ-ಪ-ಕಾ-ಮಾಖ್ಯ
ಕಾಮ-ವಿ-ರು-ವುದೊ
ಕಾಮ-ವಿ-ರು-ವೆಡೆ
ಕಾಮ-ವಿಲ್ಲ
ಕಾಮ-ವಿ-ಲ್ಲ-ದಿ-ದ್ದರೆ
ಕಾಮಾಖ್ಯ
ಕಾಮಾ-ಖ್ಯ-ದೇವಿ
ಕಾಮಿ-ನೀ-ಕಾಂ-ಚ-ನ-ವನ್ನು
ಕಾಮ್
ಕಾಯ-ಕದ
ಕಾಯ-ಕಲ್ಪ
ಕಾಯಲಿ
ಕಾಯಲು
ಕಾಯಾ
ಕಾಯಿ-ದೆ-ಗಳ
ಕಾಯಿ-ದೆ-ಗಳು
ಕಾಯಿಲೆ
ಕಾಯಿ-ಲೆ-ನ-ರ-ಳಾ-ಟ-ಗಳು
ಕಾಯಿ-ಲೆ-ಗಳಿಂದ
ಕಾಯಿ-ಲೆ-ಗ-ಳಿಂ-ದಾಗಿ
ಕಾಯಿ-ಲೆ-ಗ-ಳಿಗೆ
ಕಾಯಿ-ಲೆ-ಗಳು
ಕಾಯಿ-ಲೆಗೆ
ಕಾಯಿ-ಲೆ-ಬಿ-ದ್ದರು
ಕಾಯಿ-ಲೆ-ಯನ್ನು
ಕಾಯಿ-ಲೆ-ಯಾ-ಗಿ-ದೆ-ಯೆಂದು
ಕಾಯಿ-ಲೆ-ಯಾದ
ಕಾಯಿ-ಲೆ-ಯಾ-ದ-ವ-ರನ್ನು
ಕಾಯಿ-ಲೆ-ಯಿಂದ
ಕಾಯಿ-ಲೆ-ಯಿಂ-ದಾಗಿ
ಕಾಯಿ-ಲೆ-ಯಿಂ-ದಿ-ದ್ದರೂ
ಕಾಯಿ-ಲೆ-ಯಿಂ-ದಿ-ರು-ವುದನ್ನು
ಕಾಯಿ-ಲೆಯು
ಕಾಯಿಸಿ
ಕಾಯು-ತಿತ್ತು
ಕಾಯುತ್ತ
ಕಾಯು-ತ್ತಿತ್ತು
ಕಾಯು-ತ್ತಿದ್ದ
ಕಾಯು-ತ್ತಿ-ದ್ದಳು
ಕಾಯು-ತ್ತಿ-ದ್ದಾಳೆ
ಕಾಯು-ತ್ತಿರಿ
ಕಾಯು-ವುದನ್ನು
ಕಾಯ್ದು-ಕೊ-ಳ್ಳಲೇ
ಕಾರ
ಕಾರ-ಕೂ-ನ-ನಾ-ಗಿ-ರಲಿ
ಕಾರಣ
ಕಾರ-ಣ-ಕ್ಕಲ್ಲ
ಕಾರ-ಣ-ಕ್ಕಾಗಿ
ಕಾರ-ಣ-ಕ್ಕಾ-ಗಿಯೂ
ಕಾರ-ಣ-ಕ್ಕಾ-ಗಿಯೇ
ಕಾರ-ಣಕ್ಕೋ
ಕಾರ-ಣ-ಗ-ಳಲ್ಲ
ಕಾರ-ಣ-ಗಳಿಂದ
ಕಾರ-ಣ-ಗ-ಳಿಂ-ದಲೋ
ಕಾರ-ಣ-ಗ-ಳಿಂ-ದಾಗಿ
ಕಾರ-ಣ-ಗ-ಳಿಗೆ
ಕಾರ-ಣ-ಗ-ಳಿ-ದ್ದುವು
ಕಾರ-ಣ-ಗ-ಳಿ-ದ್ದು-ವೆಂ-ಬುದು
ಕಾರ-ಣ-ಗ-ಳಿವೆ
ಕಾರ-ಣ-ಗಳೂ
ಕಾರ-ಣ-ಗಳೇ
ಕಾರ-ಣ-ಜ-ನ-ಗ-ಳಿಗೆ
ಕಾರ-ಣ-ದಿಂದ
ಕಾರ-ಣ-ದಿಂ-ದಲೇ
ಕಾರ-ಣ-ದಿಂ-ದಾಗಿ
ಕಾರ-ಣ-ನಾ-ಗು-ತ್ತಾನೆ
ಕಾರ-ಣ-ಭೂ-ತ-ರಾ-ದ-ವ-ರೆಂ-ದರೆ
ಕಾರ-ಣ-ರಾ-ಗಿ-ದ್ದಾರೆ
ಕಾರ-ಣ-ರಾ-ದರು
ಕಾರ-ಣರೂ
ಕಾರ-ಣ-ವನ್ನು
ಕಾರ-ಣ-ವಾ-ಗ-ಬ-ಹು-ದೆಂದೋ
ಕಾರ-ಣ-ವಾ-ಗ-ಬೇ-ಕಿಲ್ಲ
ಕಾರ-ಣ-ವಾ-ಗಿತ್ತು
ಕಾರ-ಣ-ವಾ-ಗಿದ್ದ
ಕಾರ-ಣ-ವಾ-ಗಿ-ರ-ಬ-ಹು-ದಾ-ದರೂ
ಕಾರ-ಣ-ವಾ-ಗಿ-ರ-ಬ-ಹುದು
ಕಾರ-ಣ-ವಾ-ಗು-ತ್ತಿತ್ತು
ಕಾರ-ಣ-ವಾ-ಗುವ
ಕಾರ-ಣ-ವಾದ
ಕಾರ-ಣ-ವಾ-ದುದು
ಕಾರ-ಣ-ವಾ-ಯಿತು
ಕಾರ-ಣ-ವಿದೆ
ಕಾರ-ಣ-ವಿ-ರ-ಲಿಲ್ಲ
ಕಾರ-ಣ-ವಿಲ್ಲ
ಕಾರ-ಣವೂ
ಕಾರ-ಣ-ವೆಂದರೆ
ಕಾರ-ಣ-ವೆಂದು
ಕಾರ-ಣವೇ
ಕಾರ-ಣ-ವೇ-ನಿ-ದ್ದಿ-ರ-ಬ-ಹುದು
ಕಾರ-ಣ-ವೇನು
ಕಾರ-ಣ-ವೇನೆಂದರೆ
ಕಾರ-ಣ-ವೇ-ನೆಂದು
ಕಾರ-ಣಾಂ-ತ-ರ-ಗಳಿಂದ
ಕಾರದ
ಕಾರನ್ನು
ಕಾರ-ರಾ-ಹಿ-ತ್ಯವು
ಕಾರ-ವಾನ್
ಕಾರಾ-ಗೃ-ಹ-ಗ-ಳಿವೆ
ಕಾರಿ-ಗ-ಳೊಂ-ದಿಗೆ
ಕಾರಿದ
ಕಾರಿ-ದುವು
ಕಾರಿ-ದ್ದಕ್ಕೆ
ಕಾರಿ-ನಲ್ಲಿ
ಕಾರಿ-ಯಾ-ಗಿದೆ
ಕಾರಿ-ರು-ಳಿನ
ಕಾರು-ಣಿ-ಕ-ನಾದ
ಕಾರು-ವುದನ್ನು
ಕಾರ್ತಿ-ಕೇ-ಯನ
ಕಾರ್ಬಾ-ಲಿಕ್
ಕಾರ್ಮಿ-ಕರ
ಕಾರ್ಮೋ-ಡ-ದಂತೆ
ಕಾರ್ಯ
ಕಾರ್ಯ-ಕ-ರ್ತ-ರಾ-ಗಿ-ದ್ದರು
ಕಾರ್ಯ-ಕ-ರ್ತ-ರಿಗೂ
ಕಾರ್ಯ-ಕ-ರ್ತರು
ಕಾರ್ಯ-ಕ-ಲಾಪ
ಕಾರ್ಯ-ಕ-ಲಾ-ಪ-ಗಳ
ಕಾರ್ಯ-ಕ-ಲಾ-ಪ-ಗಳನ್ನು
ಕಾರ್ಯ-ಕ-ಲಾ-ಪ-ಗಳಲ್ಲಿ
ಕಾರ್ಯ-ಕ-ಲಾ-ಪ-ಗ-ಳ-ಷ್ಟನ್ನೂ
ಕಾರ್ಯ-ಕ-ಲಾ-ಪ-ಗಳಾ
ಕಾರ್ಯ-ಕ-ಲಾ-ಪ-ಗಳಿಂದ
ಕಾರ್ಯ-ಕ-ಲಾ-ಪ-ಗಳು
ಕಾರ್ಯ-ಕ-ಲಾ-ಪ-ಗ-ಳೆಲ್ಲ
ಕಾರ್ಯ-ಕೌ-ಶ-ಲ-ವನ್ನು
ಕಾರ್ಯ-ಕ್ಕಾಗಿ
ಕಾರ್ಯಕ್ಕೂ
ಕಾರ್ಯಕ್ಕೆ
ಕಾರ್ಯ-ಕ್ಕೆಂದು
ಕಾರ್ಯ-ಕ್ರಮ
ಕಾರ್ಯ-ಕ್ರ-ಮಕ್ಕೆ
ಕಾರ್ಯ-ಕ್ರ-ಮ-ಗಳ
ಕಾರ್ಯ-ಕ್ರ-ಮ-ಗಳನ್ನು
ಕಾರ್ಯ-ಕ್ರ-ಮ-ಗಳನ್ನೂ
ಕಾರ್ಯ-ಕ್ರ-ಮ-ಗಳಲ್ಲಿ
ಕಾರ್ಯ-ಕ್ರ-ಮ-ಗ-ಳಾದ
ಕಾರ್ಯ-ಕ್ರ-ಮ-ಗ-ಳಾ-ದರೂ
ಕಾರ್ಯ-ಕ್ರ-ಮ-ಗ-ಳಿಗೆ
ಕಾರ್ಯ-ಕ್ರ-ಮ-ಗ-ಳೆಲ್ಲ
ಕಾರ್ಯ-ಕ್ರ-ಮದ
ಕಾರ್ಯ-ಕ್ರ-ಮ-ದಲ್ಲಿ
ಕಾರ್ಯ-ಕ್ರ-ಮ-ದಲ್ಲೂ
ಕಾರ್ಯ-ಕ್ರ-ಮ-ದಿಂ-ದಾಗಿ
ಕಾರ್ಯ-ಕ್ರ-ಮ-ವನ್ನು
ಕಾರ್ಯ-ಕ್ರ-ಮ-ವ-ಷ್ಟನ್ನೂ
ಕಾರ್ಯ-ಕ್ರ-ಮ-ವಾ-ಗಿತ್ತು
ಕಾರ್ಯ-ಕ್ರ-ಮ-ವಾದ
ಕಾರ್ಯ-ಕ್ರ-ಮ-ವಿತ್ತು
ಕಾರ್ಯ-ಕ್ರ-ಮ-ವಿ-ರ-ಲಿಲ್ಲ
ಕಾರ್ಯ-ಕ್ರ-ಮವೂ
ಕಾರ್ಯ-ಕ್ರ-ಮ-ವೆಂದರೆ
ಕಾರ್ಯ-ಕ್ರ-ಮ-ವೊಂ-ದನ್ನು
ಕಾರ್ಯ-ಕ್ಷೇ-ತ್ರಕ್ಕೆ
ಕಾರ್ಯ-ಕ್ಷೇ-ತ್ರದ
ಕಾರ್ಯ-ಕ್ಷೇ-ತ್ರ-ದತ್ತ
ಕಾರ್ಯ-ಕ್ಷೇ-ತ್ರ-ವನ್ನು
ಕಾರ್ಯ-ಗ-ತ-ಗೊಳಿ
ಕಾರ್ಯ-ಗ-ತ-ಗೊ-ಳಿ-ಸ-ಬಲ್ಲ
ಕಾರ್ಯ-ಗ-ತ-ಗೊ-ಳಿ-ಸ-ಬೇಕಾ
ಕಾರ್ಯ-ಗ-ತ-ಗೊ-ಳಿ-ಸ-ಬೇ-ಕಾ-ಗಿದೆ
ಕಾರ್ಯ-ಗ-ತ-ಗೊ-ಳಿ-ಸಲು
ಕಾರ್ಯ-ಗ-ತ-ಗೊ-ಳಿ-ಸು-ತ್ತಾರೊ
ಕಾರ್ಯ-ಗ-ತ-ಗೊ-ಳಿ-ಸುವ
ಕಾರ್ಯ-ಗ-ತ-ಗೊ-ಳಿ-ಸು-ವುದು
ಕಾರ್ಯ-ಗ-ತ-ವಾ-ಗ-ದಿ-ದ್ದರೂ
ಕಾರ್ಯ-ಗ-ತ-ವಾ-ಗಿಲ್ಲ
ಕಾರ್ಯ-ಗಳ
ಕಾರ್ಯ-ಗ-ಳಂತೂ
ಕಾರ್ಯ-ಗ-ಳ-ನ್ನ-ಲ್ಲದೆ
ಕಾರ್ಯ-ಗಳನ್ನು
ಕಾರ್ಯ-ಗಳನ್ನೆಲ್ಲ
ಕಾರ್ಯ-ಗಳಲ್ಲಿ
ಕಾರ್ಯ-ಗ-ಳಲ್ಲೂ
ಕಾರ್ಯ-ಗ-ಳಾ-ಗಿ-ದ್ದುವು
ಕಾರ್ಯ-ಗಳಿಂದ
ಕಾರ್ಯ-ಗ-ಳಿಂ-ದಾಗಿ
ಕಾರ್ಯ-ಗ-ಳಿ-ಗಾಗಿ
ಕಾರ್ಯ-ಗ-ಳಿಗೆ
ಕಾರ್ಯ-ಗಳು
ಕಾರ್ಯ-ಗಳೂ
ಕಾರ್ಯ-ಗ-ಳೆಲ್ಲ
ಕಾರ್ಯ-ಗಳೇ
ಕಾರ್ಯ-ಚ-ಟು-ವ-ಟಿ-ಕೆ-ಗಳಲ್ಲಿ
ಕಾರ್ಯ-ಚ-ಟು-ವ-ಟಿ-ಕೆ-ಗ-ಳಿಂ-ದಾಗಿ
ಕಾರ್ಯ-ಚ-ಟು-ವ-ಟಿ-ಕೆ-ಗಳು
ಕಾರ್ಯತಃ
ಕಾರ್ಯದ
ಕಾರ್ಯ-ದ-ಕ್ಷತೆ
ಕಾರ್ಯ-ದತ್ತ
ಕಾರ್ಯ-ದರ್ಶಿ
ಕಾರ್ಯ-ದ-ರ್ಶಿ-ಗ-ಳ-ನ್ನಾಗಿ
ಕಾರ್ಯ-ದ-ರ್ಶಿ-ಗ-ಳಾ-ದರು
ಕಾರ್ಯ-ದ-ರ್ಶಿ-ಯಾದ
ಕಾರ್ಯ-ದಲ್ಲಿ
ಕಾರ್ಯ-ದ-ಲ್ಲಿಯೇ
ಕಾರ್ಯ-ದಲ್ಲೇ
ಕಾರ್ಯ-ದಲ್ಲೋ
ಕಾರ್ಯ-ದಿಂದ
ಕಾರ್ಯ-ಧೋ-ರ-ಣೆ-ಗಳ
ಕಾರ್ಯ-ನಿ-ಮಿ-ತ್ತ-ವಾಗಿ
ಕಾರ್ಯ-ನಿ-ರ-ತ-ರಾಗಿ
ಕಾರ್ಯ-ನಿ-ರ-ತ-ರಾ-ಗಿದ್ದ
ಕಾರ್ಯ-ನಿ-ರ್ವ-ಹ-ಣೆಯ
ಕಾರ್ಯ-ಪ್ರ-ಣಾ-ಳಿ-ಕೆ-ಯನ್ನು
ಕಾರ್ಯ-ಪ್ರ-ಣಾ-ಳಿ-ಕೆಯು
ಕಾರ್ಯ-ಪ್ರ-ಣಾ-ಳಿ-ಯನ್ನೇ
ಕಾರ್ಯ-ಪ್ರ-ವೃ-ತ್ತ-ರಾ-ಗು-ವಂತೆ
ಕಾರ್ಯ-ಪ್ರ-ವೃ-ತ್ತ-ವಾ-ಗ-ಬೇಕು
ಕಾರ್ಯ-ಬಾ-ಹು-ಳ್ಯ-ಗಳ
ಕಾರ್ಯ-ಭಾ-ರ-ಗಳ
ಕಾರ್ಯ-ಭಾ-ರದ
ಕಾರ್ಯ-ಭಾ-ರ-ದಲ್ಲಿ
ಕಾರ್ಯ-ಭಾ-ರ-ವನ್ನು
ಕಾರ್ಯ-ಮ-ಗ್ನ-ನಾಗು
ಕಾರ್ಯ-ಮ-ಗ್ನ-ರಾ-ಗಿ-ದ್ದೀ-ರೆಂದು
ಕಾರ್ಯ-ಮ-ಗ್ನ-ರಾ-ದರು
ಕಾರ್ಯ-ಮಾ-ಡುತ್ತಾ
ಕಾರ್ಯ-ಯೋ-ಜನೆ
ಕಾರ್ಯ-ಯೋ-ಜ-ನೆ-ಗಳ
ಕಾರ್ಯ-ಯೋ-ಜ-ನೆ-ಗಳನ್ನು
ಕಾರ್ಯ-ಯೋ-ಜ-ನೆ-ಗಳು
ಕಾರ್ಯ-ಯೋ-ಜ-ನೆ-ಗ-ಳೆಲ್ಲ
ಕಾರ್ಯ-ಯೋ-ಜ-ನೆಯ
ಕಾರ್ಯ-ರಂ-ಗ-ಕ್ಕಿ-ಳಿ-ಯಿರಿ
ಕಾರ್ಯ-ರಂ-ಗಕ್ಕೆ
ಕಾರ್ಯ-ರಂ-ಗ-ಗಳು
ಕಾರ್ಯ-ರೂ-ಪಕ್ಕೆ
ಕಾರ್ಯ-ವ-ನ್ನಾ-ದರೂ
ಕಾರ್ಯ-ವನ್ನು
ಕಾರ್ಯ-ವಾ-ಗಿತ್ತು
ಕಾರ್ಯ-ವಾ-ಗಿ-ರುವು
ಕಾರ್ಯ-ವಿ-ಧಾನ
ಕಾರ್ಯ-ವಿ-ಧಾ-ನ-ಗಳನ್ನೂ
ಕಾರ್ಯ-ವಿ-ಧಾ-ನ-ಗಳು
ಕಾರ್ಯ-ವಿ-ಧಾ-ನ-ಗ-ಳೆಲ್ಲ
ಕಾರ್ಯ-ವಿ-ಧಾ-ನದ
ಕಾರ್ಯ-ವಿ-ಧಾ-ನ-ದಲ್ಲಿ
ಕಾರ್ಯ-ವಿ-ಧಾ-ನ-ವನ್ನು
ಕಾರ್ಯ-ವಿ-ಧಾ-ನ-ವೆಂ-ಥದು
ಕಾರ್ಯ-ವಿನ್ನೂ
ಕಾರ್ಯ-ವಿ-ರು-ವು-ದಾಗಿ
ಕಾರ್ಯವು
ಕಾರ್ಯವೂ
ಕಾರ್ಯ-ವೆಂದರೆ
ಕಾರ್ಯ-ವೆಲ್ಲ
ಕಾರ್ಯ-ವೆ-ಸ-ಗಿದ
ಕಾರ್ಯವೇ
ಕಾರ್ಯ-ವೇನು
ಕಾರ್ಯ-ವೇನೆಂದರೆ
ಕಾರ್ಯ-ವೊಂದು
ಕಾರ್ಯ-ಶೀ-ಲ-ನಾ-ಗಿ-ರ-ಬೇಕು
ಕಾರ್ಯ-ಶೀ-ಲ-ರಾ-ಗಿಲ್ಲ
ಕಾರ್ಯ-ಶೀ-ಲ-ವಾ-ಗಿ-ರ-ಲಿಲ್ಲ
ಕಾರ್ಯ-ಸಾ-ಧ-ನೆ-ಯಲ್ಲಿ
ಕಾರ್ಯಾ-ರಂಭ
ಕಾರ್ಯಾ-ರ್ಥ-ವಾಗಿ
ಕಾರ್ಯಾ-ವ-ಕಾ-ಶ-ಗಳು
ಕಾರ್ಯೋ
ಕಾರ್ಯೋ-ದ್ದೇಶ
ಕಾರ್ಯೋ-ದ್ದೇ-ಶ-ಕ್ಕಾಗಿ
ಕಾರ್ಯೋ-ದ್ದೇ-ಶಕ್ಕೆ
ಕಾರ್ಯೋ-ದ್ದೇ-ಶ-ಗಳ
ಕಾರ್ಯೋ-ದ್ದೇ-ಶ-ಗಳನ್ನು
ಕಾರ್ಯೋ-ದ್ದೇ-ಶ-ಗಳಲ್ಲಿ
ಕಾರ್ಯೋ-ದ್ದೇ-ಶ-ಗ-ಳಿಗೆ
ಕಾರ್ಯೋ-ದ್ದೇ-ಶದ
ಕಾರ್ಯೋ-ದ್ದೇ-ಶ-ದಲ್ಲಿ
ಕಾರ್ಯೋ-ದ್ದೇ-ಶವು
ಕಾಲ
ಕಾಲ-ದೇ-ಶ-ಗ-ಳೆಲ್ಲ
ಕಾಲ-ಕ-ಳೆ-ಯು-ವು-ದ-ಕ್ಕಿಂತ
ಕಾಲ-ಕಾ-ಲಕ್ಕೆ
ಕಾಲ-ಕ್ಕಿಂತ
ಕಾಲಕ್ಕೆ
ಕಾಲ-ಕ್ರ-ಮ-ದಲ್ಲಿ
ಕಾಲ-ಗ-ರ್ಭ-ದಾ-ಳ-ದಿಂದ
ಕಾಲ-ಗ-ರ್ಭ-ದೊ-ಳ-ಕ್ಕೆ-ಚಾ-ರಿ-ತ್ರಿಕ
ಕಾಲ-ಗಳ
ಕಾಲ-ಗಳಲ್ಲಿ
ಕಾಲ-ಗ-ಳ-ಲ್ಲುಂ-ಟಾದ
ಕಾಲ-ಗ-ಳಲ್ಲೂ
ಕಾಲ-ಜ್ಞಾ-ನದ
ಕಾಲ-ಜ್ಞಾ-ನಿ-ಯೊ-ಬ್ಬನ
ಕಾಲ-ಡಿಗೆ
ಕಾಲದ
ಕಾಲ-ದಲ್ಲಿ
ಕಾಲ-ದಲ್ಲೂ
ಕಾಲ-ದಲ್ಲೇ
ಕಾಲ-ದಿಂದ
ಕಾಲ-ದಿಂ-ದಲೂ
ಕಾಲ-ದೇ-ಶ-ಗಳ
ಕಾಲ-ದೇ-ಶವ
ಕಾಲ-ದೊ-ಳಕ್ಕೆ
ಕಾಲದ್ದು
ಕಾಲ-ಧ-ರ್ಮ-ಕ್ಕನು
ಕಾಲ-ಧ-ರ್ಮ-ಗಳ
ಕಾಲ-ಯಾ-ಪನೆ
ಕಾಲರಾ
ಕಾಲ-ರೂಪಿ
ಕಾಲ-ರೂ-ಪಿಣಿ
ಕಾಲ-ವನ್ನು
ಕಾಲ-ವ-ರ್ಣದ
ಕಾಲ-ವ-ಲ್ಲವೆ
ಕಾಲ-ವಾ-ಯಿತು
ಕಾಲ-ವಿನ್ನೂ
ಕಾಲವೂ
ಕಾಲ-ವೆಂದು
ಕಾಲ-ವೊಂದು
ಕಾಲ-ಹ-ರಣ
ಕಾಲಾಂ-ತ-ರ-ದಲ್ಲಿ
ಕಾಲಾ-ನಂ-ತರ
ಕಾಲಾ-ವ-ಕಾ-ಶ-ವಿಲ್ಲ
ಕಾಲಾ-ವಧಿ
ಕಾಲಿ-ಕವೂ
ಕಾಲಿಗೆ
ಕಾಲಿ-ಟ್ಟದ್ದು
ಕಾಲಿ-ಟ್ಟಾಗ
ಕಾಲಿ-ಟ್ಟಾ-ಗಿ-ನಿಂದ
ಕಾಲಿ-ಟ್ಟಿದೆ
ಕಾಲಿ-ಡಲು
ಕಾಲಿ-ಡಲೂ
ಕಾಲಿನ
ಕಾಲಿ-ರಿ-ಸಿ-ದ-ರೆಂಬ
ಕಾಲು
ಕಾಲು-ಕಿತ್ತ
ಕಾಲು-ಗಳ
ಕಾಲು-ಗಳನ್ನು
ಕಾಲು-ಗಳು
ಕಾಲು-ದಾರಿ-ಯಲ್ಲಿ
ಕಾಲು-ವೆಯ
ಕಾಲು-ವೆ-ಯನ್ನು
ಕಾಲು-ವೆ-ಯಲ್ಲೇ
ಕಾಲು-ವೆ-ಯಿಂದ
ಕಾಲು-ವೆಯೇ
ಕಾಲೇ-ಜನ್ನು
ಕಾಲೇಜಿ
ಕಾಲೇ-ಜಿಗೆ
ಕಾಲೇ-ಜಿನ
ಕಾಲೇ-ಜಿ-ನಲ್ಲಿ
ಕಾಲೇಜು
ಕಾಲೇ-ಜು-ಗಳ
ಕಾಲೇ-ಜೊಂ-ದನ್ನು
ಕಾಲೇಜ್
ಕಾಲೊ
ಕಾಲೊ-ನಿಗೆ
ಕಾಲ್ತೊ-ಡ-ರಿ-ಸುವ
ಕಾಲ್ನ-ಡಿ-ಗೆಯ
ಕಾಲ್ನ-ಡಿ-ಗೆ-ಯಲ್ಲಿ
ಕಾಲ್ವೆ
ಕಾಲ್ವೆಯ
ಕಾಲ್ವೆ-ಯನ್ನು
ಕಾಳ-ಗ-ತ್ತ-ಲೆ-ಯಂ-ತಿ-ರುವ
ಕಾಳ-ಜಿಯ
ಕಾಳ-ಮೇ-ಘ-ಗ-ಳೆಲ್ಲ
ಕಾಳ-ರಾ-ತ್ರಿಯ
ಕಾಳಿ
ಕಾಳಿ-ಕಾ-ದೇ-ವಿ-ಯನ್ನು
ಕಾಳಿಗೂ
ಕಾಳಿಗೆ
ಕಾಳಿಯ
ಕಾಳಿ-ಯನ್ನೇ
ಕಾಳೀ
ಕಾಳೀ-ಕೃಷ್ಣ
ಕಾಳೀ-ಕೃ-ಷ್ಣನ
ಕಾಳೀ-ಘಾ-ಟಿನ
ಕಾಳೀ-ದ-ರ್ಶ-ನದ
ಕಾಳೀ-ದೇ-ವಾ-ಲ-ಯಕ್ಕೆ
ಕಾಳೀ-ಪೂ-ಜೆ-ಯಂದು
ಕಾಳೀ-ಪೂ-ಜೆ-ಯನ್ನೂ
ಕಾಳೀ-ಪ್ರ-ಸಾ-ದ-ನಿಗೆ
ಕಾಳೀ-ಪ್ರ-ಸಾ-ದ-ರೊಂ-ದಿಗೆ
ಕಾಳೀ-ಮಾ-ತೆಯ
ಕಾಳ್ಗಿ-ಚ್ಚಿ-ನಂತೆ
ಕಾಳ್ಗಿಚ್ಚು
ಕಾವಲು
ಕಾವಿ
ಕಾವಿಗೇ
ಕಾವಿ-ಛ-ತ್ರಿ-ಗಳ
ಕಾವಿ-ನಿಂದ
ಕಾವು
ಕಾವ್ಯ
ಕಾವ್ಯ-ಮಯ
ಕಾವ್ಯಾ-ತ್ಮ-ಕ-ವಾಗಿ
ಕಾಶ
ಕಾಶಿಗೆ
ಕಾಶಿಯ
ಕಾಶಿ-ಯಲ್ಲಿ
ಕಾಶಿ-ಯ-ಲ್ಲಿ-ದ್ದ-ರಾ-ದರೂ
ಕಾಶೀ
ಕಾಶೀ-ಕ್ಷೇ-ತ್ರದ
ಕಾಶೀ-ಪು-ರದ
ಕಾಶ್ಮೀರ
ಕಾಶ್ಮೀ-ರಕ್ಕೆ
ಕಾಶ್ಮೀ-ರ-ಗಳ
ಕಾಶ್ಮೀ-ರದ
ಕಾಶ್ಮೀ-ರ-ದತ್ತ
ಕಾಶ್ಮೀ-ರ-ದಲ್ಲಿ
ಕಾಶ್ಮೀ-ರ-ದ-ಲ್ಲಿದ್ದ
ಕಾಶ್ಮೀ-ರ-ದ-ಲ್ಲಿ-ದ್ದಾಗ
ಕಾಶ್ಮೀ-ರ-ದಿಂದ
ಕಾಶ್ಮೀ-ರ-ಪು-ರ-ವಾಸ
ಕಾಶ್ಮೀ-ರ-ವನ್ನು
ಕಾಶ್ಮೀ-ರ-ವಲ್ಲ
ಕಾಶ್ಮೀ-ರ-ವಾ-ಸ-ದಿಂದ
ಕಾಷಾಯ
ಕಾಷಾ-ಯ-ಧಾರಿ
ಕಾಷಾ-ಯ-ವನ್ನು
ಕಾಷಾ-ಯ-ವಸ್ತ್ರ
ಕಾಷಾ-ಯ-ವ-ಸ್ತ್ರದ
ಕಾಷಾ-ಯ-ವ-ಸ್ತ್ರ-ಧಾ-ರಿ-ಯಾದ
ಕಾಷಾ-ಯ-ವ-ಸ್ತ್ರ-ವನ್ನು
ಕಾಷಾ-ಯ-ವ-ಸ್ತ್ರ-ವೊಂದು
ಕಾಸನ್ನೂ
ಕಾಸಿ-ಲ್ಲದೆ
ಕಿ
ಕಿಂಕ-ರರು
ಕಿಂಚಿತ್
ಕಿಂಚಿ-ತ್ತಾ-ದರೂ
ಕಿಂಚಿತ್ತೂ
ಕಿಕ್ಕಿ-ರಿದ
ಕಿಕ್ಕಿ-ರಿ-ದಿತ್ತು
ಕಿಕ್ಕಿ-ರಿ-ದಿದ್ದ
ಕಿಕ್ಕಿ-ರಿ-ದಿ-ದ್ದರು
ಕಿಕ್ಕಿ-ರಿ-ದಿ-ದ್ದುದು
ಕಿಕ್ಕಿ-ರಿದು
ಕಿಚುಡಿ
ಕಿಚು-ಡಿಯ
ಕಿಟಕಿ
ಕಿಟ-ಕಿ-ಗ-ವಾ-ಕ್ಷ-ಗಳ
ಕಿಟ-ಕಿ-ಗಳನ್ನೆಲ್ಲ
ಕಿಟ-ಕಿ-ಗಳು
ಕಿಟ-ಕಿಯ
ಕಿಟ-ಕಿ-ಯಿಂದ
ಕಿಡಿ
ಕಿಡಿ-ಗಳು
ಕಿಡಿ-ಯೊಂ-ದನ್ನು
ಕಿತ್ತ-ಡಿ-ಯಿ-ಡುತ
ಕಿತ್ತು
ಕಿತ್ತು-ಕೊಂಡು
ಕಿತ್ತು-ಕೊಂ-ಡು-ಬಿ-ಟ್ಟಿವೆ
ಕಿತ್ತು-ಕೊ-ಳ್ಳಲು
ಕಿತ್ತೊ
ಕಿತ್ತೊ-ಗೆ-ಯ-ಬೇ-ಕಾ-ದರೆ
ಕಿತ್ತೊ-ಗೆ-ಯ-ಬೇಕು
ಕಿತ್ತೊ-ಗೆ-ಯಲು
ಕಿತ್ತೊ-ಗೆ-ಯಿರಿ
ಕಿತ್ತೊ-ಗೆ-ಯುವ
ಕಿತ್ತೊ-ಗೆ-ಯು-ವಂ-ತಿ-ದ್ದುವು
ಕಿನಿಂದ
ಕಿನ್ನರ
ಕಿರ-ಣ-ಗಳನ್ನು
ಕಿರ-ಣ-ಗ-ಳಿ-ದ್ದಂತೆ
ಕಿರ-ಣ-ಗಳು
ಕಿರಿ-ಕಿ-ರಿಯೇ
ಕಿರಿ-ದಾದ
ಕಿರಿಯ
ಕಿರಿ-ಯರ
ಕಿರಿ-ಯ-ರಾದ
ಕಿರಿ-ಯರು
ಕಿರೀಟ
ಕಿರು-ಕುಳ
ಕಿರು-ಕು-ಳದ
ಕಿರು-ನ-ಗೆ-ಯೊಂದು
ಕಿರು-ಪ-ರಿ-ಚಯ
ಕಿಲುಬು
ಕಿಲೋ-ಮೀ-ಟ-ರಿಗೂ
ಕಿಲೋ-ಮೀ-ಟ-ರಿ-ನಷ್ಟು
ಕಿವಿ
ಕಿವಿ-ಗ-ಡ-ಚಿ-ಕ್ಕುವ
ಕಿವಿ-ಗ-ಡ-ಚಿ-ಕ್ಕು-ವಂತೆ
ಕಿವಿ-ಗಳಿಂದ
ಕಿವಿ-ಗ-ಳಿಗೆ
ಕಿವಿಗೂ
ಕಿವಿಗೆ
ಕಿವಿ-ಗೊಟ್ಟು
ಕಿವಿ-ಗೊ-ಡ-ದಿ-ದ್ದರೂ
ಕಿವಿ-ಗೊ-ಡ-ಲಿಲ್ಲ
ಕಿವಿ-ಗೊ-ಡು-ವು-ದಾ-ದರೆ
ಕಿವಿ-ತೆ-ರೆದು
ಕಿವಿ-ದೆ-ರೆದು
ಕಿವಿಯ
ಕಿವಿ-ಯ-ರ-ಳಿ-ಸಿ-ಕೊಂಡು
ಕಿವಿ-ಯಲ್ಲಿ
ಕಿವಿ-ಯ-ಲ್ಲೀಗ
ಕಿವಿ-ಯಾಗಿ
ಕಿವು-ಡಾ-ಗು-ವಂ-ತಿತ್ತು
ಕಿವು-ಡಾ-ಗು-ವಷ್ಟು
ಕಿವುಡು
ಕಿಶ-ನ್ಘರ್
ಕೀ
ಕೀಟವೇ
ಕೀಯ
ಕೀಯ-ವಾಗಿ
ಕೀರ್ತ-ನೆ-ಗಳನ್ನು
ಕೀರ್ತ-ನೆ-ಯೊಂ-ದನ್ನು
ಕೀರ್ತ-ನೋ-ತ್ಸಾಹ
ಕೀರ್ತಿ
ಕೀರ್ತಿ-ಗಳನ್ನು
ಕೀರ್ತಿಗೆ
ಕೀರ್ತಿ-ಪ-ತಾ-ಕೆ-ಯನ್ನು
ಕೀರ್ತಿಯೂ
ಕೀರ್ತಿ-ಯೆಲ್ಲ
ಕೀರ್ತಿ-ಶಾ-ಲಿ-ಯಾದ
ಕೀರ್ತಿ-ಶಿ-ಖ-ರ-ವ-ನ್ನೇ-ರಿದ
ಕೀಲಿಕೈ
ಕೀಳುವ
ಕುಂಚ-ದಿಂದ
ಕುಂಟಿ-ಕೊಂಡು
ಕುಂಟೆ-ಗ-ಳೆಲ್ಲ
ಕುಂಠಿ-ತ-ಗೊಂಡ
ಕುಂಡ-ಗ-ಳಿದ್ದು
ಕುಂಡ-ಲಿ-ನಿ-ಯನ್ನು
ಕುಂಡ-ವನ್ನೇ
ಕುಂದ-ಲಿಲ್ಲ
ಕುಂದಿ-ರ-ಲಿಲ್ಲ
ಕುಂದಿ-ರ-ಲಿ-ಲ್ಲ-ವೆಂ-ಬು-ದಕ್ಕೆ
ಕುಂದು-ಕೊ-ರತೆ
ಕುಂದು-ತ್ತಿ-ರುವ
ಕುಂಭ-ಕೋಣಂ
ಕುಂಭ-ಕೋ-ಣಂಗೆ
ಕುಂಭ-ಕೋ-ಣಂ-ನಲ್ಲಿ
ಕುಂಭ-ಕೋ-ಣಂ-ನ-ಲ್ಲಿ-ದ್ದಾಗ
ಕುಂಭ-ಕೋ-ಣಂ-ನಿಂದ
ಕುಂಭಾ-ಭಿ-ಷೇಕ
ಕುಂಯ್ಗು-ಟ್ಟಿತು
ಕುಕ್ಕ-ರಿ-ಸು-ತ್ತಿ-ದ್ದರು
ಕುಗ್ಗಿ
ಕುಗ್ಗಿ-ಹೋ-ದಳು
ಕುಗ್ಗು-ತ್ತಿ-ದ್ದುದು
ಕುಗ್ಗು-ತ್ತಿ-ರಲೂ
ಕುಗ್ರಾ-ಮ-ದಲ್ಲಿ
ಕುಚೋದ್ಯ
ಕುಚೋ-ದ್ಯ-ವೆಂದು
ಕುಟಿ-ಲೋ-ಪಾ-ಯ-ಗಳನ್ನು
ಕುಟಿ-ಲೋ-ಪಾ-ಯ-ಗ-ಳಷ್ಟೇ
ಕುಟೀರ
ಕುಟೀ-ರಕ್ಕೆ
ಕುಟೀ-ರದ
ಕುಟೀ-ರ-ದಲ್ಲಿ
ಕುಟೀ-ರ-ವನ್ನು
ಕುಟುಂ
ಕುಟುಂಬ
ಕುಟುಂ-ಬಕ್ಕೆ
ಕುಟುಂ-ಬದ
ಕುಟುಂ-ಬ-ದ-ವರ
ಕುಟುಂ-ಬ-ದ-ವರು
ಕುಟುಂ-ಬ-ದ-ವರೆಲ್ಲ
ಕುಟುಂ-ಬ-ದ-ವರೆ-ಲ್ಲರ
ಕುಟುಂ-ಬ-ವ-ರ್ಗಕ್ಕೆ
ಕುಟುಂ-ಬ-ವ-ರ್ಗ-ದ-ಲ್ಲಿದ್ದ
ಕುಟುಂ-ಬ-ವ-ರ್ಗ-ದ-ವರು
ಕುಟ್ಟಿ
ಕುಡಿದ
ಕುಡಿ-ದರು
ಕುಡಿ-ದಿ-ರ-ಬಹು
ಕುಡಿ-ದಿ-ರ-ಲಿಲ್ಲ
ಕುಡಿ-ದಿರಾ
ಕುಡಿದು
ಕುಡಿ-ದು-ದ-ಲ್ಲದೆ
ಕುಡಿ-ದು-ಬಿಟ್ಟೆ
ಕುಡಿದೆ
ಕುಡಿದೇ
ಕುಡಿ-ಯದೆ
ಕುಡಿ-ಯ-ಬ-ಹುದು
ಕುಡಿ-ಯ-ಬ-ಹುದೆ
ಕುಡಿ-ಯ-ಬಾ-ರ-ದೆಂದು
ಕುಡಿ-ಯ-ಬೇಕೆ
ಕುಡಿ-ಯ-ಬೇ-ಕೆಂದು
ಕುಡಿ-ಯ-ಲಿಲ್ಲ
ಕುಡಿ-ಯಲು
ಕುಡಿ-ಯು-ತ್ತಿ-ದ್ದ-ವರು
ಕುಡಿ-ಯುವ
ಕುಡಿ-ಯು-ವಂ-ತಿಲ್ಲ
ಕುಡಿ-ಯು-ವಂತೆ
ಕುಡಿ-ಯು-ವು-ದಿಲ್ಲ
ಕುಣಿ-ಕು-ಣಿ-ಯುತ
ಕುಣಿ-ದಾ-ಡಿ-ದರು
ಕುಣಿ-ದಾ-ಡು-ತ್ತಿದ್ದ
ಕುತಂತ್ರ
ಕುತರ್ಕ
ಕುತ-ರ್ಕ-ವನ್ನು
ಕುತೂ
ಕುತೂ-ಹಲ
ಕುತೂ-ಹ-ಲ-ಕರ
ಕುತೂ-ಹ-ಲ-ಕ-ರ-ವಾ-ಗಿತ್ತು
ಕುತೂ-ಹ-ಲ-ಕ-ರ-ವಾ-ಗಿದೆ
ಕುತೂ-ಹ-ಲ-ಕ-ರ-ವಾದ
ಕುತೂ-ಹ-ಲ-ಕಾ-ರಿ-ಯಾಗಿ
ಕುತೂ-ಹ-ಲ-ಗೊಂಡು
ಕುತೂ-ಹ-ಲ-ತಮ್ಮ
ಕುತೂ-ಹ-ಲದ
ಕುತೂ-ಹ-ಲ-ದಿಂದ
ಕುತೂ-ಹ-ಲಿ-ಗಳ
ಕುತೂ-ಹ-ಲಿ-ಗ-ಳಾ-ಗಿ-ದ್ದರು
ಕುತೂ-ಹ-ಲಿ-ಗ-ಳಾ-ಗಿ-ದ್ದೇವೆ
ಕುತೂ-ಹ-ಲಿ-ಗಳು
ಕುತ್ತು
ಕುದಿದ
ಕುದುರೆ
ಕುದು-ರೆ-ಗಳ
ಕುದು-ರೆ-ಗಳನ್ನು
ಕುದು-ರೆ-ಗ-ಳ-ನ್ನೇರಿ
ಕುದು-ರೆ-ಗಳಿಂದ
ಕುದು-ರೆ-ಗಳೂ
ಕುದು-ರೆ-ಗಳೋ
ಕುದು-ರೆ-ಗಾ-ಡಿ-ಯ-ಲ್ಲಿ-ಅದೂ
ಕುದು-ರೆಯ
ಕುದು-ರೆ-ಯಂತೂ
ಕುದು-ರೆ-ಯನ್ನು
ಕುಪ್ರ-ಸಿದ್ಧ
ಕುಬ್ಜ-ನಂತೆ
ಕುಮಾರ
ಕುಮಾ-ರರು
ಕುಮಾ-ರ-ಸ್ವಾಮಿ
ಕುಮಾ-ರ-ಸ್ವಾ-ಮಿ-ಯ-ವರು
ಕುಮಾರಿ
ಕುಮಾ-ರಿ-ಲ-ಭ-ಟ್ಟ-ನಂತೆ
ಕುಮಾ-ರೀ-ಪೂಜೆ
ಕುಮಾವೂ
ಕುಯು-ಕ್ತಿ-ಯಿಂ-ದಲೋ
ಕುರಿ-ಗ-ಳಂತಾ
ಕುರಿ-ಗಳನ್ನು
ಕುರಿತ
ಕುರಿ-ತ-ದ್ದಾ-ಗಿತ್ತು
ಕುರಿ-ತದ್ದು
ಕುರಿತಾ
ಕುರಿ-ತಾಗಿ
ಕುರಿ-ತಾ-ಗಿಯೇ
ಕುರಿ-ತಾ-ಗಿಯೋ
ಕುರಿ-ತಾದ
ಕುರಿತು
ಕುರಿತೂ
ಕುರಿತೇ
ಕುರು-ಕ್ಷೇ-ತ್ರದ
ಕುರು-ಕ್ಷೇ-ತ್ರ-ದಲ್ಲಿ
ಕುರು-ಡ-ನಾ-ಗಿ-ಬಿ-ಟ್ಟಿದ್ದ
ಕುರು-ಡ-ನಿಗೆ
ಕುರು-ಡನು
ಕುರು-ಡ-ರಿಗೆ
ಕುರು-ಡಾಗಿ
ಕುರು-ಡಾ-ಗಿ-ದ್ದೀರಿ
ಕುರುಡಿ
ಕುರು-ಡು-ತ-ನ-ವನ್ನು
ಕುರು-ಬರ
ಕುರು-ಹಾಗಿ
ಕುರುಹು
ಕುರ್ಚಿಯ
ಕುರ್ಚಿ-ಯನ್ನು
ಕುಲಕ್ಕೆ
ಕುಲದ
ಕುಲ-ದ-ವರು
ಕುಲ-ಪ-ತಿ-ಗಳ
ಕುಲಾ-ಭಿ-ಮಾನವೂ
ಕುಳಿ
ಕುಳಿತ
ಕುಳಿ-ತ-ದ್ದನ್ನು
ಕುಳಿ-ತದ್ದು
ಕುಳಿ-ತರು
ಕುಳಿ-ತರೆ
ಕುಳಿ-ತಲ್ಲಿ
ಕುಳಿ-ತ-ಲ್ಲಿಂ-ದಲೇ
ಕುಳಿ-ತ-ಲ್ಲಿಂ-ದೆದ್ದು
ಕುಳಿ-ತಲ್ಲೇ
ಕುಳಿ-ತಳು
ಕುಳಿ-ತ-ವ-ರಿಗೆ
ಕುಳಿ-ತ-ವ-ರಿ-ಗೆಲ್ಲ
ಕುಳಿ-ತ-ವರು
ಕುಳಿ-ತಾಗ
ಕುಳಿತಿ
ಕುಳಿ-ತಿತ್ತು
ಕುಳಿ-ತಿದೆ
ಕುಳಿ-ತಿದ್ದ
ಕುಳಿ-ತಿ-ದ್ದರು
ಕುಳಿ-ತಿ-ದ್ದ-ವರ
ಕುಳಿ-ತಿ-ದ್ದ-ವರೆಲ್ಲ
ಕುಳಿ-ತಿ-ದ್ದ-ವರೆ-ಲ್ಲರ
ಕುಳಿ-ತಿ-ದ್ದಾಗ
ಕುಳಿ-ತಿ-ದ್ದಾರೆ
ಕುಳಿ-ತಿ-ದ್ದೀರಿ
ಕುಳಿ-ತಿದ್ದೆ
ಕುಳಿ-ತಿರ
ಕುಳಿ-ತಿ-ರ-ದ-ವರು
ಕುಳಿ-ತಿ-ರ-ಬ-ಹು-ದು-ಏನೇ
ಕುಳಿ-ತಿ-ರ-ಬೇಕು
ಕುಳಿ-ತಿ-ರ-ಬೇಡಿ
ಕುಳಿ-ತಿ-ರ-ಲಾ-ರದೆ
ಕುಳಿ-ತಿ-ರಲು
ಕುಳಿ-ತಿರು
ಕುಳಿ-ತಿ-ರು-ತ್ತಿ-ದ್ದರು
ಕುಳಿ-ತಿ-ರು-ತ್ತಿದ್ದೆ
ಕುಳಿ-ತಿ-ರು-ತ್ತೇವೆ
ಕುಳಿ-ತಿ-ರುವ
ಕುಳಿ-ತಿ-ರು-ವಂ-ತಿದೆ
ಕುಳಿ-ತಿ-ರು-ವಂತೆ
ಕುಳಿ-ತಿ-ರು-ವು-ದ-ಕ್ಕಿಂತ
ಕುಳಿ-ತಿ-ರು-ವುದು
ಕುಳಿತು
ಕುಳಿ-ತುಕೊ
ಕುಳಿ-ತು-ಕೊಂಡ
ಕುಳಿ-ತು-ಕೊಂ-ಡರು
ಕುಳಿ-ತು-ಕೊಂ-ಡಿತು
ಕುಳಿ-ತು-ಕೊಂ-ಡಿ-ರು-ವುದು
ಕುಳಿ-ತು-ಕೊಂಡು
ಕುಳಿ-ತು-ಕೊ-ಳ್ಳದೆ
ಕುಳಿ-ತು-ಕೊ-ಳ್ಳ-ಬ-ಯ-ಸು-ವ-ವ-ರಿಗೆ
ಕುಳಿ-ತು-ಕೊ-ಳ್ಳ-ಬೇ-ಕಾ-ಯಿ-ತು-ಬ-ಹಳ
ಕುಳಿ-ತು-ಕೊ-ಳ್ಳಲು
ಕುಳಿ-ತು-ಕೊ-ಳ್ಳಲೂ
ಕುಳಿ-ತು-ಕೊ-ಳ್ಳುತ್ತ
ಕುಳಿ-ತು-ಕೊ-ಳ್ಳು-ತ್ತಾರೆ
ಕುಳಿ-ತು-ಕೊ-ಳ್ಳು-ವಂತೆ
ಕುಳಿ-ತು-ಕೊ-ಳ್ಳು-ವ-ವ-ರಲ್ಲ
ಕುಳಿ-ತು-ಕೊ-ಳ್ಳು-ವ-ವ-ರಿಗೆ
ಕುಳಿ-ತು-ಕೊ-ಳ್ಳು-ವುದ
ಕುಳಿ-ತು-ಕೊ-ಳ್ಳು-ವು-ದರ
ಕುಳಿ-ತು-ಕೊ-ಳ್ಳು-ವು-ದೆಂ-ದರೆ
ಕುಳಿ-ತು-ಬಿಟ್ಟ
ಕುಳಿ-ತು-ಬಿ-ಟ್ಟರು
ಕುಳಿ-ತು-ಬಿ-ಟ್ಟರೆ
ಕುಳಿ-ತು-ಬಿ-ಟ್ಟಿತು
ಕುಳಿ-ತು-ಬಿ-ಟ್ಟಿ-ದ್ದರು
ಕುಳಿ-ತು-ಬಿ-ಟ್ಟಿ-ದ್ದ-ವರು
ಕುಳಿ-ತು-ಬಿಟ್ಟು
ಕುಳಿ-ತು-ಬಿಡು
ಕುಳಿ-ತು-ಬಿ-ಡು-ತ್ತಿ-ದ್ದರು
ಕುಳಿ-ತು-ಬಿ-ಡು-ತ್ತೇವೆ
ಕುಳಿ-ತೆವು
ಕುಳ್ಳಿ-ರಿಸ
ಕುಳ್ಳಿ-ರಿಸಿ
ಕುಳ್ಳಿ-ರಿ-ಸಿ-ಕೊಂ-ಡರು
ಕುಳ್ಳಿ-ರಿ-ಸಿ-ಕೊಂಡು
ಕುಳ್ಳಿ-ರಿ-ಸಿ-ದರು
ಕುಳ್ಳಿ-ರಿ-ಸು-ತ್ತಾನೆ
ಕುಶ
ಕುಶ-ಲ-ಕ-ಲೆ-ಗಳನ್ನು
ಕುಶ-ಲೋ-ಪ-ರಿಯ
ಕುಷಾ-ಣರು
ಕುಷ್ಠ-ರೋ-ಗಿ-ಯಾಗಿ
ಕುಸಿದು
ಕುಸಿ-ದು-ಬಿ-ದ್ದರೂ
ಕುಸಿ-ಯುತ್ತ
ಕುಸಿ-ಯು-ತ್ತಿದ್ದ
ಕುಸು-ಮ-ಗಳನ್ನು
ಕುಸು-ಮ-ರಾ-ಶಿ-ಗ-ಳಂತೆ
ಕುಹಕ
ಕುಹ-ಕದ
ಕುಹ-ಕ-ವಾ-ಡಿ-ದ-ಶ್ರೀ-ಮತಿ
ಕುಹ-ಕಿ-ಗಳ
ಕುಹ-ಕಿ-ಗ-ಳಿ-ಗೆ-ವಿ-ಕೃ-ತ-ಬು-ದ್ಧಿ-ಯ-ವ-ರಿಗೆ
ಕೂಗ
ಕೂಗನ್ನು
ಕೂಗಾ-ಟ-ವನ್ನು
ಕೂಗಾ-ಡು-ತ್ತಿ-ದ್ದರು
ಕೂಗಾ-ಡು-ವುದು
ಕೂಗಿ
ಕೂಗಿ-ಕೊಂ-ಡ-ನಂ-ತೆ-ದೇ-ವರು
ಕೂಗಿ-ಕೊಂ-ಡರು
ಕೂಗಿ-ಕೊ-ಳ್ಳು-ತ್ತಿ-ದ್ದನು
ಕೂಗಿಗೆ
ಕೂಗು
ಕೂಗುತ್ತ
ಕೂಚ್ಬಿ-ಹಾ-ರಿನ
ಕೂಟ-ದಲ್ಲಿ
ಕೂಡ
ಕೂಡದು
ಕೂಡಲೆ
ಕೂಡಲೇ
ಕೂಡಾ
ಕೂಡಿ
ಕೂಡಿ-ಕೊಂ-ಡರು
ಕೂಡಿ-ಕೊಂ-ಡಿದ್ದ
ಕೂಡಿ-ಕೊಂಡು
ಕೂಡಿ-ಕೊ-ಳ್ಳ-ಬೇಕು
ಕೂಡಿ-ಕೊ-ಳ್ಳ-ಲಿದ್ದ
ಕೂಡಿ-ಕೊ-ಳ್ಳಲು
ಕೂಡಿ-ಕೊ-ಳ್ಳುವ
ಕೂಡಿ-ಕೊ-ಳ್ಳು-ವಂತೆ
ಕೂಡಿ-ಟ್ಟು-ಕೊಂ-ಡಿದ್ದ
ಕೂಡಿತ್ತು
ಕೂಡಿದ
ಕೂಡಿ-ದುದು
ಕೂಡಿದೆ
ಕೂಡಿ-ದ್ದರು
ಕೂಡಿದ್ದು
ಕೂಡಿ-ದ್ದೆ-ನ್ನ-ಬ-ಹುದು
ಕೂಡಿ-ಬಂ-ದಿತ್ತು
ಕೂಡಿ-ಬ-ರಲೇ
ಕೂಡಿ-ರ-ಬೇಕೆ
ಕೂಡಿ-ರು-ತ್ತಿ-ದ್ದರು
ಕೂಡಿ-ರು-ವುದು
ಕೂದಲು
ಕೂಪ-ದಲ್ಲಿ
ಕೂಪ-ಮಂ-ಡೂ-ಕ-ಗ-ಳಾ-ಗಿ-ರುವ
ಕೂಪ-ಮಂ-ಡೂ-ಕ-ಗಳೂ
ಕೂಲ-ಕ-ರ-ವಾ-ಗಿಯೇ
ಕೂಲ-ತೆ-ಗಳು
ಕೂಲಿ
ಕೂಲಿ-ಗಳನ್ನು
ಕೂಲಿ-ಗಳನ್ನೂ
ಕೂಲಿ-ಗ-ಳಿ-ಗೆಲ್ಲ
ಕೂಲಿ-ಗಳು
ಕೂಲಿ-ಗಳೂ
ಕೂಲಿ-ಗ-ಳೊಂ-ದಿಗೆ
ಕೂಲಿಗೆ
ಕೂಳಿ-ಗಾಗಿ
ಕೃತಕ
ಕೃತ-ಕೃತ್ಯ
ಕೃತ-ಘ್ನ-ನಾ-ಗು-ತ್ತೇನೆ
ಕೃತಜ್ಞ
ಕೃತ-ಜ್ಞ-ತಾ-ಪೂ-ರ್ವ-ಕ-ವಾಗಿ
ಕೃತ-ಜ್ಞ-ತಾ-ಭಾ-ವ-ದಿಂದ
ಕೃತ-ಜ್ಞತೆ
ಕೃತ-ಜ್ಞ-ತೆ-ಅಭಿ
ಕೃತ-ಜ್ಞ-ತೆ-ಗ-ಳ-ನ್ನರ್ಪಿ
ಕೃತ-ಜ್ಞ-ತೆ-ಗ-ಳ-ನ್ನ-ರ್ಪಿ-ಸುತ್ತ
ಕೃತ-ಜ್ಞ-ತೆ-ಗಳನ್ನು
ಕೃತ-ಜ್ಞ-ತೆ-ಗಳು
ಕೃತ-ಜ್ಞ-ತೆ-ಯ-ನ್ನರ್ಪಿ
ಕೃತ-ಜ್ಞ-ತೆ-ಯ-ನ್ನ-ರ್ಪಿ-ಸಿ-ದರು
ಕೃತ-ಜ್ಞ-ತೆ-ಯ-ನ್ನ-ರ್ಪಿ-ಸುವ
ಕೃತ-ಜ್ಞ-ತೆ-ಯನ್ನು
ಕೃತ-ಜ್ಞ-ತೆ-ಯಿಂದ
ಕೃತ-ಜ್ಞ-ರಾ-ಗಿ-ದ್ದರೂ
ಕೃತ-ಜ್ಞ-ವಾ-ಗಿ-ರ-ಬೇಕು
ಕೃತಾ-ರ್ಥ-ರಾ-ದೆ-ವೆಂಬ
ಕೃತಿ
ಕೃತಿ-ಗಳ
ಕೃತಿ-ಗಳನ್ನು
ಕೃತಿ-ಗಳಲ್ಲಿ
ಕೃತಿ-ಗ-ಳ-ಲ್ಲೊಂದು
ಕೃತಿ-ಗ-ಳೆ-ಲ್ಲ-ವನ್ನೂ
ಕೃತಿಯ
ಕೃತಿ-ಯನ್ನು
ಕೃತಿ-ಯಾ-ದರೆ
ಕೃತಿಯು
ಕೃತಿ-ಶ್ರೇ-ಣಿ-ಯಲ್ಲಿ
ಕೃತ್ಯ
ಕೃತ್ಯವೂ
ಕೃಪಾ
ಕೃಪಾ-ಮ-ಯ-ರಾ-ಗಿರ
ಕೃಪಾ-ಶೀ-ರ್ವಾ-ದದ
ಕೃಪಾ-ಶೀ-ರ್ವಾ-ದ-ವನ್ನು
ಕೃಪೆ
ಕೃಪೆಗೆ
ಕೃಪೆ-ಗೈ-ಯಲಿ
ಕೃಪೆ-ದೋರಿ
ಕೃಪೆ-ಮಾ-ಡ-ಲೆಂದು
ಕೃಪೆ-ಮಾಡಿ
ಕೃಪೆಯ
ಕೃಪೆ-ಯನ್ನು
ಕೃಪೆ-ಯಿಂದ
ಕೃಪೆ-ಯಿಂ-ದಾಗಿ
ಕೃಪೆ-ಯಿಟ್ಟು
ಕೃಪೆ-ಯೆಂ-ದರೆ
ಕೃಪೆ-ಯೊಂ-ದ-ರಿಂ-ದಲೇ
ಕೃಷ್ಣ
ಕೃಷ್ಣ-ವಿಷ್ಣು
ಕೃಷ್ಣ-ಮೂರ್ತಿ
ಕೃಷ್ಣರ
ಕೃಷ್ಣ-ರನ್ನು
ಕೃಷ್ಣ-ರಿಗೆ
ಕೃಷ್ಣರು
ಕೃಷ್ಣ-ಲಾ-ಲರು
ಕೃಷ್ಣ-ಲಾ-ಲ-ರೊ-ಬ್ಬ-ರನ್ನು
ಕೃಷ್ಣ-ಲಾಲ್
ಕೃಷ್ಣ-ಸ್ವಾಮಿ
ಕೃಷ್ಣಾ-ಜಿ-ನ-ವನ್ನು
ಕೆ
ಕೆಂಟ್
ಕೆಂಡ
ಕೆಂಪ-ಗಿತ್ತು
ಕೆಂಪಾ-ಗಿದೆ
ಕೆಂಪಾ-ಯಿತು
ಕೆಂಪು
ಕೆಚ್ಚೆ-ದೆ-ಯಿಂದ
ಕೆಟ್ಟ
ಕೆಟ್ಟ-ದಾ-ಗಿಯೂ
ಕೆಟ್ಟ-ದ್ದನ್ನು
ಕೆಟ್ಟ-ದ್ದನ್ನೂ
ಕೆಟ್ಟ-ದ್ದ-ರಿಂದ
ಕೆಟ್ಟ-ದ್ದಾಗಿ
ಕೆಟ್ಟ-ದ್ದು-ಎ-ರ-ಡನ್ನೂ
ಕೆಟ್ಟದ್ದೂ
ಕೆಟ್ಟ-ದ್ದೇ-ನಲ್ಲ
ಕೆಟ್ಟ-ವನಾ
ಕೆಟ್ಟ-ವರು
ಕೆಟ್ಟ-ವೇ-ನಲ್ಲ
ಕೆಟ್ಟಿತು
ಕೆಟ್ಟಿ-ದ್ದರೂ
ಕೆಟ್ಟಿಲ್ಲ
ಕೆಟ್ಟಿ-ಲ್ಲ-ತಾನೆ
ಕೆಟ್ಟು
ಕೆಟ್ಟು-ಹೋ-ಗ-ಬ-ಲ್ಲುದು
ಕೆಟ್ಟು-ಹೋಗಿ
ಕೆಟ್ಟೆನೋ
ಕೆಡ-ಕನ್ನೂ
ಕೆಡ-ಕು-ಗಳನ್ನು
ಕೆಡ-ಬಾ-ರ-ದೆಂದು
ಕೆಡ-ಲಾ-ರಂ-ಭಿ-ಸಿತು
ಕೆಡಿಸಿ
ಕೆಡಿ-ಸಿ-ಕೊ-ಳ್ಳ-ದಿ-ರು-ವಂತೆ
ಕೆಡಿ-ಸಿ-ಕೊ-ಳ್ಳ-ಬೇಡ
ಕೆಡಿ-ಸಿ-ಕೊ-ಳ್ಳುವ
ಕೆಡುಕಿ
ಕೆಡುಕೇ
ಕೆಡುವ
ಕೆಣಕಿ
ಕೆಣ-ಕಿ-ದಾಗ
ಕೆತ್ತನೆ
ಕೆತ್ತ-ನೆ-ಗ-ಳುಳ್ಳ
ಕೆತ್ತ-ನೆಯ
ಕೆತ್ತಿ-ರುವ
ಕೆದಕಿ
ಕೆನೋ-ವರ
ಕೆಮ್ಮು-ವು-ದುಂಟು
ಕೆರ-ಳಿತು
ಕೆರ-ಳಿ-ಸಿತು
ಕೆರ-ಳಿ-ಸಿವೆ
ಕೆರ-ಳಿ-ಸುವ
ಕೆರ-ಳು-ವಂತೆ
ಕೆರೆ
ಕೆರೆ-ಗಳು
ಕೆರೆ-ಯಲ್ಲಿ
ಕೆರೆ-ಯು-ತ್ತದೆ
ಕೆಲ
ಕೆಲ-ಕಾಲ
ಕೆಲ-ಕಾ-ಲದ
ಕೆಲ-ಕಾ-ಲ-ದಲ್ಲೇ
ಕೆಲ-ಕಾ-ಲ-ದಿಂದ
ಕೆಲ-ಕಾ-ಲ-ದಿಂ-ದಲೂ
ಕೆಲ-ಕಾ-ಲ-ವಾ-ದರೂ
ಕೆಲ-ಕೆ-ಲ-ವರು
ಕೆಲ-ಕೆ-ಲವು
ಕೆಲ-ತಿಂ-ಗಳು
ಕೆಲ-ದಿನ
ಕೆಲ-ದಿ-ನ-ಗಳ
ಕೆಲ-ದಿ-ನ-ಗ-ಳಂತೂ
ಕೆಲ-ದಿ-ನ-ಗಳನ್ನು
ಕೆಲ-ದಿ-ನ-ಗಳಲ್ಲಿ
ಕೆಲ-ದಿ-ನ-ಗ-ಳಲ್ಲೇ
ಕೆಲ-ದಿ-ನ-ಗ-ಳ-ವ-ರೆಗೆ
ಕೆಲ-ದಿ-ನ-ಗಳು
ಕೆಲ-ದಿ-ನ-ವಿದ್ದು
ಕೆಲ-ಭಾ-ಗ-ವನ್ನು
ಕೆಲ-ಮ-ಟ್ಟಿಗೆ
ಕೆಲರು
ಕೆಲ-ವಂತೂ
ಕೆಲ-ವಂ-ಶ-ಗಳು
ಕೆಲ-ವ-ನ್ನಾ-ದರೂ
ಕೆಲ-ವನ್ನು
ಕೆಲ-ವರ
ಕೆಲ-ವ-ರದು
ಕೆಲ-ವ-ರನ್ನು
ಕೆಲ-ವ-ರಲ್ಲಿ
ಕೆಲ-ವರಿ
ಕೆಲ-ವ-ರಿ-ಗಾ-ದರೂ
ಕೆಲ-ವ-ರಿ-ಗಾ-ದರೆ
ಕೆಲ-ವ-ರಿಗೆ
ಕೆಲ-ವ-ರಿ-ದ್ದರು
ಕೆಲ-ವರು
ಕೆಲ-ವ-ರೆಂ-ದರೆ
ಕೆಲ-ವರೊ
ಕೆಲವು
ಕೆಲವೇ
ಕೆಲ-ವೊಂದು
ಕೆಲ-ವೊಮ್ಮೆ
ಕೆಲಸ
ಕೆಲ-ಸ-ಕಾ-ರ್ಯ-ಗಳ
ಕೆಲ-ಸ-ಕಾ-ರ್ಯ-ಗಳನ್ನು
ಕೆಲ-ಸ-ಕಾ-ರ್ಯ-ಗ-ಳ-ನ್ನೇನೂ
ಕೆಲ-ಸ-ಕಾ-ರ್ಯ-ಗಳು
ಕೆಲ-ಸ-ಕ್ಕಾಗಿ
ಕೆಲ-ಸಕ್ಕೆ
ಕೆಲ-ಸ-ಗಳ
ಕೆಲ-ಸ-ಗಳನ್ನು
ಕೆಲ-ಸ-ಗಳನ್ನೂ
ಕೆಲ-ಸ-ಗಳಲ್ಲಿ
ಕೆಲ-ಸ-ಗ-ಳ-ಲ್ಲೊಂ-ದೆಂ-ದರೆ
ಕೆಲ-ಸ-ಗಳಿಂದ
ಕೆಲ-ಸ-ಗ-ಳಿ-ಗಿಂ-ತಲೂ
ಕೆಲ-ಸ-ಗ-ಳಿಗೆ
ಕೆಲ-ಸ-ಗ-ಳಿ-ಗೆಲ್ಲ
ಕೆಲ-ಸ-ಗ-ಳಿವೆ
ಕೆಲ-ಸ-ಗಳು
ಕೆಲ-ಸ-ಗಳೂ
ಕೆಲ-ಸ-ಗ-ಳೆಲ್ಲ
ಕೆಲ-ಸ-ಗಾ-ರ-ನಾ-ಗ-ಬಲ್ಲ
ಕೆಲ-ಸ-ಗಾ-ರ-ನೊ-ಬ್ಬ-ನನ್ನು
ಕೆಲ-ಸ-ಗಾ-ರ-ರಿಗೆ
ಕೆಲ-ಸ-ಗಾ-ರರು
ಕೆಲ-ಸದ
ಕೆಲ-ಸ-ದಲ್ಲಿ
ಕೆಲ-ಸ-ದ-ಲ್ಲಿ-ದ್ದರೆ
ಕೆಲ-ಸ-ದ-ಲ್ಲಿಯೂ
ಕೆಲ-ಸ-ದಲ್ಲೂ
ಕೆಲ-ಸ-ದಲ್ಲೇ
ಕೆಲ-ಸ-ದಿಂದ
ಕೆಲ-ಸ-ಬೇ-ರಾ-ವು-ದನ್ನೂ
ಕೆಲ-ಸ-ಮಯ
ಕೆಲ-ಸ-ಮಾ-ಡುವ
ಕೆಲ-ಸ-ವನ್ನು
ಕೆಲ-ಸ-ವನ್ನೂ
ಕೆಲ-ಸ-ವನ್ನೇ
ಕೆಲ-ಸ-ವಿ-ರು-ತ್ತಿ-ತ್ತಾ-ದರೂ
ಕೆಲ-ಸವೂ
ಕೆಲ-ಸವೆ
ಕೆಲ-ಸ-ವೆಂದರೆ
ಕೆಲ-ಸ-ವೆಂ-ಬುದು
ಕೆಲ-ಸವೇ
ಕೆಲ-ಸ-ವೇ-ನಾ-ಗಿರ
ಕೆಲ-ಸ-ವೇ-ನಾ-ದರೂ
ಕೆಲ-ಸ-ವೇ-ನಿ-ದ್ದರೂ
ಕೆಲ-ಸ-ವೊಂ-ದನ್ನು
ಕೆಲ-ಹೊ-ತ್ತಿ-ನ-ವ-ರೆಗೆ
ಕೆಲೇಸ್
ಕೆಳ
ಕೆಳಕ್ಕೆ
ಕೆಳ-ಗಿನ
ಕೆಳ-ಗಿ-ನ-ವ-ರನ್ನು
ಕೆಳ-ಗಿ-ನಿಂದ
ಕೆಳ-ಗಿ-ಳಿದು
ಕೆಳ-ಗಿಳಿ-ಯು-ವುದು
ಕೆಳ-ಗಿಳಿ-ಸ-ಬೇಡಿ
ಕೆಳ-ಗಿ-ಳಿಸಿ
ಕೆಳಗೂ
ಕೆಳಗೆ
ಕೆಳ-ಗೆ-ಳೆ-ಯ-ಬಾ-ರದು
ಕೆಳ-ಗೆ-ಳೆ-ಯು-ವು-ದಲ್ಲ
ಕೆಳಗೇ
ಕೆಳ-ಗೊ-ತ್ತ-ಲಾ-ರದು
ಕೆಳ-ತುಟಿ
ಕೆಳ-ದ-ರ್ಜೆಯ
ಕೆಳ-ಮು-ಖ-ವಾ-ಗು-ತ್ತಿ-ದ್ದುವು
ಕೆಳ-ವ-ರ್ಗ-ಗಳ
ಕೆಳ-ವ-ರ್ಗ-ದಲ್ಲಿ
ಕೆಳ-ವ-ರ್ಗ-ದ-ವರ
ಕೆಳ-ವ-ರ್ಗ-ದ-ವ-ರ-ವ-ರೆಗೆ
ಕೆಳ-ಹಂ-ತದ
ಕೆಳಿದ
ಕೆಷ್ಟೊ
ಕೆಷ್ಟೊ-ನನ್ನು
ಕೆಷ್ಟೊನೂ
ಕೆಸ-ರನ್ನು
ಕೆಸ-ರಾ-ಗಿ-ದ್ದರೂ
ಕೆಸ-ರಿ-ನಲ್ಲಿ
ಕೆಸ-ರಿ-ನಿಂದ
ಕೆಸ-ರಿ-ನಿಂ-ದಾಗಿ
ಕೆಸರು
ಕೇಂದ್ರ
ಕೇಂದ್ರ-ಗಳ
ಕೇಂದ್ರ-ಗಳನ್ನು
ಕೇಂದ್ರ-ಗ-ಳ-ಲ್ಲಿನ
ಕೇಂದ್ರ-ಗ-ಳಾಗಿ
ಕೇಂದ್ರ-ಗಳು
ಕೇಂದ್ರದ
ಕೇಂದ್ರ-ದ-ಲ್ಲಿಯೇ
ಕೇಂದ್ರ-ದಿಂದ
ಕೇಂದ್ರ-ಬಿಂದು
ಕೇಂದ್ರ-ಬಿಂ-ದು-ವಾಗಿ
ಕೇಂದ್ರ-ಬಿಂ-ದುವಿ
ಕೇಂದ್ರ-ಬಿಂ-ದುವೇ
ಕೇಂದ್ರ-ವ-ನ್ನಾ-ಗಿ-ಪು-ಣ್ಯ-ಕ್ಷೇ-ತ್ರ-ವ-ನ್ನಾ-ಗಿ-ಮಾ-ಡ-ಲೆಂದು
ಕೇಂದ್ರ-ವನ್ನು
ಕೇಂದ್ರ-ವಾಗಿ
ಕೇಂದ್ರ-ವಾ-ಗಿ-ರ-ಬೇಕು
ಕೇಂದ್ರ-ವಾಗು
ಕೇಂದ್ರ-ವಾ-ಗು-ತ್ತದೆ
ಕೇಂದ್ರ-ವಾದ
ಕೇಂದ್ರ-ವಾ-ದರೂ
ಕೇಂದ್ರ-ವೆಂದರೆ
ಕೇಂದ್ರ-ವೊಂ-ದರ
ಕೇಂದ್ರ-ವೊಂದು
ಕೇಂದ್ರ-ಸ್ಥಾ-ನ-ವಾ-ಗಿ-ಟ್ಟು-ಕೊಂಡು
ಕೇಂದ್ರ-ಸ್ಥಾ-ನ-ವಾ-ಗಿ-ರು-ತ್ತದೆ
ಕೇಂದ್ರಿತ
ಕೇಂದ್ರಿ-ತ-ವಾ-ದಂತೆ
ಕೇಂದ್ರೀ-ಕ-ರಿ-ಸಿ-ದ್ದರು
ಕೇಂದ್ರೀ-ಕೃ-ತ-ವಾ-ಗಿದೆ
ಕೇಂಬ್ರಿ-ಡ್ಜಿಗೂ
ಕೇಂಬ್ರಿ-ಡ್ಜಿ-ನಲ್ಲಿ
ಕೇಂಬ್ರಿ-ಡ್ಜ್
ಕೇಕಿ-ರವ
ಕೇದಾ-ರ-ನಾಥ
ಕೇದಾ-ರ-ನಾ-ಥನ
ಕೇರಳ
ಕೇರ-ಳದ
ಕೇಲ್ಕ-ರರು
ಕೇಳ-ಕೇ-ಳು-ತ್ತಿ-ದ್ದಂ-ತೆಯೇ
ಕೇಳ-ತೊ-ಡ-ಗಿ-ದರು
ಕೇಳದೆ
ಕೇಳ-ಬ-ಹು-ದಾ-ಗಿತ್ತು
ಕೇಳ-ಬ-ಹು-ದಾದ
ಕೇಳ-ಬ-ಹುದು
ಕೇಳ-ಬ-ಹುದೆ
ಕೇಳ-ಬ-ಹು-ದೆಂಬ
ಕೇಳ-ಬೇ-ಕಾಗಿ
ಕೇಳ-ಬೇ-ಕಿತ್ತು
ಕೇಳ-ಬೇಕೆ
ಕೇಳ-ಬೇ-ಕೆಂದು
ಕೇಳ-ಬೇ-ಕೆಂಬ
ಕೇಳ-ಲಾರೆ
ಕೇಳಲು
ಕೇಳಲೇ
ಕೇಳ-ಲೇ-ಬೇಕು
ಕೇಳಿ
ಕೇಳಿ-ಕೊಂಡ
ಕೇಳಿ-ಕೊಂ-ಡರು
ಕೇಳಿ-ಕೊಂ-ಡಾಗ
ಕೇಳಿ-ಕೊಂ-ಡಿತು
ಕೇಳಿ-ಕೊ-ಳ್ಳ-ಲಾ-ಯಿತು
ಕೇಳಿ-ಕೊ-ಳ್ಳಲು
ಕೇಳಿ-ಕೊ-ಳ್ಳು-ತ್ತಿ-ದ್ದರು
ಕೇಳಿ-ಕೊ-ಳ್ಳು-ವಂ-ತಾ-ಗು-ತ್ತದೆ
ಕೇಳಿ-ಕೊ-ಳ್ಳು-ವ-ವರೇ
ಕೇಳಿತು
ಕೇಳಿದ
ಕೇಳಿ-ದರು
ಕೇಳಿ-ದರೂ
ಕೇಳಿ-ದರೆ
ಕೇಳಿ-ದಳು
ಕೇಳಿ-ದ-ವನು
ಕೇಳಿ-ದ-ವ-ನೊಬ್ಬ
ಕೇಳಿ-ದ-ವ-ರಲ್ಲಿ
ಕೇಳಿ-ದ-ವ-ರ-ಲ್ಲೊಂದು
ಕೇಳಿ-ದ-ವರು
ಕೇಳಿ-ದ-ವರೆಲ್ಲ
ಕೇಳಿ-ದ-ಸ್ವಾ-ಮೀಜಿ
ಕೇಳಿ-ದಾಗ
ಕೇಳಿ-ದಾ-ಗಿ-ನಿಂದ
ಕೇಳಿ-ದುದು
ಕೇಳಿದೆ
ಕೇಳಿ-ದೆ-ನಲ್ಲ
ಕೇಳಿ-ದೊ-ಡನೆ
ಕೇಳಿ-ದೊ-ಡ-ನೆಯೇ
ಕೇಳಿದ್ದ
ಕೇಳಿ-ದ್ದನ್ನು
ಕೇಳಿ-ದ್ದರು
ಕೇಳಿ-ದ್ದಳು
ಕೇಳಿ-ದ್ದಳೋ
ಕೇಳಿದ್ದು
ಕೇಳಿದ್ದೆ
ಕೇಳಿದ್ದೇ
ಕೇಳಿ-ದ್ದೇ-ನೆ-ಕೆ-ಳ-ವ-ರ್ಗ-ದ-ವ-ರಿ-ಗೆಲ್ಲ
ಕೇಳಿ-ದ್ದೇವೆ
ಕೇಳಿ-ನೀನು
ಕೇಳಿ-ಬಂತು
ಕೇಳಿ-ಬಂದ
ಕೇಳಿ-ಬಂ-ದಂ-ತಾ-ಗು-ತ್ತಿತ್ತು
ಕೇಳಿ-ಬಂ-ದರೆ
ಕೇಳಿ-ಬಂ-ದಿತು
ಕೇಳಿ-ಬ-ರದೆ
ಕೇಳಿ-ಬರು
ಕೇಳಿ-ಬ-ರು-ತ್ತಲೇ
ಕೇಳಿ-ಬ-ರು-ತ್ತಿತ್ತು
ಕೇಳಿ-ಬ-ರು-ತ್ತಿದ್ದ
ಕೇಳಿ-ಬ-ರು-ತ್ತಿ-ದ್ದುವು
ಕೇಳಿ-ಬ-ರುವ
ಕೇಳಿಯೂ
ಕೇಳಿಯೇ
ಕೇಳಿ-ಯೇ-ಬಿಟ್ಟ
ಕೇಳಿ-ರ-ದಂ-ಥದು
ಕೇಳಿ-ರ-ಲಿ-ಅ-ವರು
ಕೇಳಿ-ರ-ಲಿಲ್ಲ
ಕೇಳಿ-ಲ್ಲ-ವಲ್ಲ
ಕೇಳಿ-ಲ್ಲ-ವೆ-ಕೆ-ಲವು
ಕೇಳಿ-ಲ್ಲ-ವೆ-ನನ್ನ
ಕೇಳಿ-ಸ-ದಂ-ತಾ-ಯಿತು
ಕೇಳಿ-ಸಲೇ
ಕೇಳಿಸಿ
ಕೇಳಿ-ಸಿ-ಕೊಂ-ಡರೋ
ಕೇಳಿ-ಸಿ-ಕೊಂಡು
ಕೇಳಿ-ಸಿತು
ಕೇಳಿ-ಸಿ-ತು-ಧ್ಯಾ-ನಕ್ಕೆ
ಕೇಳಿ-ಸಿ-ತು-ನೀನು
ಕೇಳಿ-ಸಿದ್ದ
ಕೇಳಿ-ಸು-ತ್ತಲೇ
ಕೇಳಿ-ಸು-ವಂ-ತಹ
ಕೇಳಿ-ಸು-ವಂ-ತಿತ್ತು
ಕೇಳಿ-ಸು-ವಂ-ತಿ-ರ-ಲಿಲ್ಲ
ಕೇಳಿ-ಸು-ವಂ-ತೆಯೇ
ಕೇಳು
ಕೇಳು-ಗರ
ಕೇಳು-ಗ-ರನ್ನು
ಕೇಳು-ಗ-ರಿಗೆ
ಕೇಳುತ್ತ
ಕೇಳು-ತ್ತಾರೆ
ಕೇಳು-ತ್ತಿದ್ದ
ಕೇಳು-ತ್ತಿ-ದ್ದಂತೆ
ಕೇಳು-ತ್ತಿ-ದ್ದಂ-ತೆಯೇ
ಕೇಳು-ತ್ತಿ-ದ್ದರು
ಕೇಳು-ತ್ತಿ-ದ್ದ-ರು-ಏನ್
ಕೇಳು-ತ್ತಿ-ದ್ದರೆ
ಕೇಳು-ತ್ತಿ-ದ್ದ-ವ-ರಲ್ಲಿ
ಕೇಳು-ತ್ತಿ-ದ್ದ-ವ-ರಿಗೆ
ಕೇಳು-ತ್ತಿ-ದ್ದ-ವ-ರಿ-ಗೆಲ್ಲ
ಕೇಳು-ತ್ತಿ-ದ್ದಾಗ
ಕೇಳು-ತ್ತಿ-ದ್ದಾರೆ
ಕೇಳು-ತ್ತಿ-ದ್ದುದು
ಕೇಳು-ತ್ತಿದ್ದೆ
ಕೇಳು-ತ್ತಿ-ದ್ದೇ-ನೆ-ನಾವು
ಕೇಳು-ತ್ತಿ-ದ್ದೇವೆ
ಕೇಳು-ತ್ತಿ-ರು-ವಷ್ಟೇ
ಕೇಳು-ತ್ತೀರಿ
ಕೇಳು-ತ್ತೇನೆ
ಕೇಳು-ತ್ತೇ-ವೆ-ಒ-ಬ್ಬನು
ಕೇಳುವ
ಕೇಳು-ವಂ-ತೆಯೇ
ಕೇಳು-ವ-ವರ
ಕೇಳು-ವ-ವರು
ಕೇಳು-ವಷ್ಟು
ಕೇಳು-ವಷ್ಟೇ
ಕೇಳು-ವಾಗ
ಕೇಳು-ವು-ದ-ಕ್ಕಿಂತ
ಕೇಳು-ವು-ದಕ್ಕೂ
ಕೇಳು-ವು-ದಾ-ದರೆ
ಕೇಳು-ವು-ದಿಲ್ಲ
ಕೇಳು-ವು-ದುಂಟು
ಕೇವಲ
ಕೇಸರಿ
ಕೇಸ-ರಿ-ಯಾದ
ಕೇಸೀ
ಕೈ
ಕೈಕಾ-ಲು-ಗಳನ್ನು
ಕೈಕೆ-ಲ-ಸ-ವನ್ನು
ಕೈಕೆ-ಳ-ಗಿ-ಟ್ಟು-ಕೊ-ಳ್ಳುವ
ಕೈಕೆ-ಳ-ಗಿನ
ಕೈಕೈ
ಕೈಗಂಟೆ
ಕೈಗ-ಡಿ-ಯಾ-ರ-ವನ್ನು
ಕೈಗ-ಳ-ನ್ನಿ-ರಿಸಿ
ಕೈಗಳನ್ನು
ಕೈಗಳನ್ನೂ
ಕೈಗಳಿಂದ
ಕೈಗಳು
ಕೈಗಾ-ರಿಕಾ
ಕೈಗಾ-ರಿ-ಕೆ-ಗ-ಳಿಗೆ
ಕೈಗಾ-ರಿ-ಕೋ-ದ್ಯ-ಮಿ-ಗ-ಳಾದ
ಕೈಗಿತ್ತು
ಕೈಗೂ
ಕೈಗೂಡ
ಕೈಗೂ-ಡ-ದಿ-ದ್ದು-ದ-ರಿಂದ
ಕೈಗೂ-ಡ-ಲಿಲ್ಲ
ಕೈಗೂ-ಡಲೇ
ಕೈಗೂ-ಡು-ವುದು
ಕೈಗೆ
ಕೈಗೆ-ಟು-ಕದ
ಕೈಗೆ-ತ್ತಿ-ಕೊಂಡ
ಕೈಗೆ-ತ್ತಿ-ಕೊಂ-ಡಳು
ಕೈಗೆ-ತ್ತಿ-ಕೊಂಡು
ಕೈಗೆ-ತ್ತಿ-ಕೊ-ಳ್ಳ-ದಿ-ದ್ದರೆ
ಕೈಗೆ-ತ್ತಿ-ಕೊ-ಳ್ಳ-ಬ-ಹುದು
ಕೈಗೆ-ತ್ತಿ-ಕೊ-ಳ್ಳ-ಬೇ-ಕಾದ
ಕೈಗೆ-ತ್ತಿ-ಕೊ-ಳ್ಳಲು
ಕೈಗೆ-ತ್ತಿ-ಕೊ-ಳ್ಳ-ಲ್ಪ-ಟ್ಟರೆ
ಕೈಗೆ-ತ್ತಿ-ಕೊಳ್ಳಿ
ಕೈಗೆ-ತ್ತಿ-ಕೊಳ್ಳು
ಕೈಗೆ-ತ್ತಿ-ಕೊ-ಳ್ಳು-ತ್ತಿ-ದ್ದರು
ಕೈಗೇ-ನಾ-ದರೂ
ಕೈಗೊಂಡ
ಕೈಗೊಂ-ಡದ್ದು
ಕೈಗೊಂ-ಡರು
ಕೈಗೊಂ-ಡರೆ
ಕೈಗೊಂ-ಡಳು
ಕೈಗೊಂ-ಡಾಗ
ಕೈಗೊಂ-ಡಾ-ಗಲೇ
ಕೈಗೊಂಡಿ
ಕೈಗೊಂ-ಡಿತು
ಕೈಗೊಂ-ಡಿದ್ದ
ಕೈಗೊಂ-ಡಿ-ದ್ದರು
ಕೈಗೊಂ-ಡಿ-ರುವ
ಕೈಗೊಂಡು
ಕೈಗೊಳ್ಳ
ಕೈಗೊ-ಳ್ಳ-ತ-ಕ್ಕದ್ದು
ಕೈಗೊ-ಳ್ಳ-ಬೇಕು
ಕೈಗೊ-ಳ್ಳ-ಬೇ-ಕೆಂದು
ಕೈಗೊ-ಳ್ಳ-ಲಿ-ರುವ
ಕೈಗೊ-ಳ್ಳ-ಲಿ-ರು-ವು-ದಾಗಿ
ಕೈಗೊ-ಳ್ಳಲು
ಕೈಗೊಳ್ಳಿ
ಕೈಗೊ-ಳ್ಳುವ
ಕೈಗೊ-ಳ್ಳು-ವಂ-ತಹ
ಕೈಗೊ-ಳ್ಳು-ವಂತೆ
ಕೈಗೊ-ಳ್ಳು-ವಾಗ
ಕೈಗೊ-ಳ್ಳೋಣ
ಕೈಚಾ-ಚುವ
ಕೈಚೆಲ್ಲಿ
ಕೈಜಾ-ರಲು
ಕೈಜೋ-ಡಿಸಿ
ಕೈಜೋ-ಡಿ-ಸಿ-ಕೊಂಡು
ಕೈತುಂಬ
ಕೈತೊ-ಳೆ-ಯಲು
ಕೈತೋ-ರಿ-ಸುತ್ತ
ಕೈದಿ-ಯಾ-ಗಿದ್ದ
ಕೈದೀ-ವಿ-ಗೆ-ಯಿಂದು
ಕೈದೋರಿ
ಕೈಬಿ-ಡ-ಬೇ-ಕಾ-ಯಿತು
ಕೈಬಿ-ಡ-ಲಾ-ಯಿತು
ಕೈಬಿ-ಡು-ತ್ತಿ-ದ್ದಾರೆ
ಕೈಬಿ-ಡು-ವು-ದಿಲ್ಲ
ಕೈಬೀಸಿ
ಕೈಬೆ-ರ-ಳೆ-ಣಿಕೆ
ಕೈಮೀ-ರಿ-ಹೋ-ಯಿತು
ಕೈಮು-ಗಿದು
ಕೈಯನ್ನು
ಕೈಯಲ್ಲಿ
ಕೈಯ-ಲ್ಲಿದ್ದ
ಕೈಯ-ಲ್ಲಿನ
ಕೈಯ-ಲ್ಲಿ-ರ-ಬೇಕು
ಕೈಯ-ಲ್ಲಿ-ರುವ
ಕೈಯ-ಲ್ಲೆತ್ತಿ
ಕೈಯ-ಲ್ಲೆ-ರಡು
ಕೈಯ-ಲ್ಲೊಂದು
ಕೈಯಾರೆ
ಕೈಯಿಂದ
ಕೈಯಿಂ-ದಲೇ
ಕೈಯಿಟ್ಟು
ಕೈಯೆತ್ತಿ
ಕೈರೋ
ಕೈರೋ-ಎ-ಲ್ಲವೂ
ಕೈರೋದ
ಕೈರೋ-ದಲ್ಲಿ
ಕೈರೋ-ದಿಂದ
ಕೈಲಾದ
ಕೈಲಾ-ದ-ಷ್ಟನ್ನು
ಕೈಲಾ-ದು-ದ-ನ್ನೆಲ್ಲ
ಕೈಲಾಸ
ಕೈಲಾ-ಸದ
ಕೈಲಾ-ಸ-ಯಾತ್ರೆ
ಕೈವಾ-ಡ-ವನ್ನು
ಕೈಸೇ-ರಿತು
ಕೈಸೇ-ರು-ವ-ಷ್ಟ-ರಲ್ಲೇ
ಕೈಹಾಕಿ
ಕೈಹಾ-ಕು-ತ್ತಾರೋ
ಕೈಹಿ-ಡಿ-ತ-ದ-ಲ್ಲಿದ್ದ
ಕೈಹಿ-ಡಿದು
ಕೊಂಡ
ಕೊಂಡದ್ದು
ಕೊಂಡ-ರಾ-ದರೂ
ಕೊಂಡರು
ಕೊಂಡರೆ
ಕೊಂಡಳು
ಕೊಂಡ-ವ-ರಿಗೆ
ಕೊಂಡ-ವರು
ಕೊಂಡಷ್ಟು
ಕೊಂಡಾಗ
ಕೊಂಡಾ-ಡ-ಬ-ಹುದು
ಕೊಂಡಾ-ಡಿದ
ಕೊಂಡಾ-ಡಿ-ದರು
ಕೊಂಡಾ-ಡಿ-ದ್ದ-ರಾ-ದರೂ
ಕೊಂಡಾ-ಡಿ-ದ್ದಾನೆ
ಕೊಂಡಾ-ಡಿ-ದ್ದಾರೆ
ಕೊಂಡಾ-ಡು-ತ್ತಿ-ದ್ದರು
ಕೊಂಡಾರು
ಕೊಂಡಿತು
ಕೊಂಡಿ-ತೆಂ-ದರೆ
ಕೊಂಡಿದೆ
ಕೊಂಡಿದ್ದ
ಕೊಂಡಿ-ದ್ದರು
ಕೊಂಡಿ-ದ್ದ-ವರ
ಕೊಂಡಿ-ದ್ದ-ವರು
ಕೊಂಡಿ-ರುವ
ಕೊಂಡಿ-ರು-ವ-ವ-ರೆಗೂ
ಕೊಂಡು
ಕೊಂಡು-ಕೊಂಡ
ಕೊಂಡು-ಕೊಂ-ಡರು
ಕೊಂಡು-ಕೊಂ-ಡಾಗ
ಕೊಂಡು-ಕೊಳ್ಳ
ಕೊಂಡು-ಕೊ-ಳ್ಳಲು
ಕೊಂಡು-ಕೊ-ಳ್ಳು-ವು-ದ-ರಿಂದ
ಕೊಂಡೆ
ಕೊಂಡೊ-ಯ್ಯಲು
ಕೊಂಡೊ-ಯ್ಯು-ತ್ತ-ದೆಂ-ಬು-ದನ್ನು
ಕೊಂಡೊ-ಯ್ಯು-ತ್ತಿ-ರು-ವುದು
ಕೊಂಡೊ-ಯ್ಯು-ವುದೂ
ಕೊಂದದ್ದು
ಕೊಂದರೆ
ಕೊಂದ-ಲ್ಲದೆ
ಕೊಂದ-ವ-ನನ್ನೂ
ಕೊಂದು
ಕೊಂಬೆ
ಕೊಂಬೆ-ಗಳ
ಕೊಂಬೆ-ಯನ್ನು
ಕೊಕ್ಕ-ರೆ-ಗಂತೂ
ಕೊಚ್ಚಿ-ಕೊಂಡು
ಕೊಚ್ಚಿ-ಹೋ-ಗು-ತ್ತಿ-ದ್ದರು
ಕೊಚ್ಚಿ-ಹೋ-ಗು-ತ್ತಿ-ದ್ದವು
ಕೊಟ್ಟ
ಕೊಟ್ಟ-ಕೊ-ನೆ-ಯ-ದಾಗಿ
ಕೊಟ್ಟ-ದ್ದ-ಲ್ಲದೆ
ಕೊಟ್ಟದ್ದು
ಕೊಟ್ಟ-ಮೇಲೆ
ಕೊಟ್ಟರು
ಕೊಟ್ಟ-ರು-ಜ-ನ-ಸಾ-ಮಾ-ನ್ಯ-ರೊಂ-ದಿಗೆ
ಕೊಟ್ಟರೂ
ಕೊಟ್ಟರೆ
ಕೊಟ್ಟ-ರೆಂ-ದರೆ
ಕೊಟ್ಟ-ಲ್ಲೆಲ್ಲ
ಕೊಟ್ಟಳು
ಕೊಟ್ಟ-ವನು
ಕೊಟ್ಟಾಗ
ಕೊಟ್ಟಿತು
ಕೊಟ್ಟಿ-ತೆಂ-ದರೆ
ಕೊಟ್ಟಿದೆ
ಕೊಟ್ಟಿದ್ದ
ಕೊಟ್ಟಿ-ದ್ದ-ರಿಂದ
ಕೊಟ್ಟಿ-ದ್ದರು
ಕೊಟ್ಟಿ-ದ್ದ-ರು-ಆ-ಹಾ-ರದ
ಕೊಟ್ಟಿದ್ದೆ
ಕೊಟ್ಟಿ-ದ್ದೇನೆ
ಕೊಟ್ಟಿಲ್ಲ
ಕೊಟ್ಟು
ಕೊಟ್ಟು-ಬಿಟ್ಟ
ಕೊಟ್ಟು-ಬಿ-ಟ್ಟರು
ಕೊಟ್ಟು-ಬಿ-ಟ್ಟಳು
ಕೊಟ್ಟು-ಬಿ-ಡು-ತ್ತದೆ
ಕೊಟ್ಟು-ಬಿ-ಡು-ತ್ತಾ-ರಂತೆ
ಕೊಡ-ಕೂ-ಡದು
ಕೊಡಗು
ಕೊಡ-ದಿ-ರಲು
ಕೊಡದೆ
ಕೊಡ-ಬಲ್ಲ
ಕೊಡ-ಬ-ಲ್ಲಿರಾ
ಕೊಡ-ಬಲ್ಲೆ
ಕೊಡ-ಬಹು
ಕೊಡ-ಬ-ಹುದು
ಕೊಡ-ಬಾ-ರದು
ಕೊಡ-ಬೇ-ಕಾ-ಗಿದೆ
ಕೊಡ-ಬೇ-ಕಾದ
ಕೊಡ-ಬೇ-ಕಾ-ದದ್ದು
ಕೊಡ-ಬೇ-ಕಾ-ದರೆ
ಕೊಡ-ಬೇಕು
ಕೊಡ-ಬೇ-ಕೆಂದು
ಕೊಡ-ಬೇ-ಕೆಂ-ಬುದು
ಕೊಡ-ಲಾಗ
ಕೊಡ-ಲಾದ
ಕೊಡ-ಲಾ-ಯಿತು
ಕೊಡ-ಲಿಲ್ಲ
ಕೊಡಲು
ಕೊಡಲೂ
ಕೊಡ-ಲೇ-ಬಾ-ರ-ದಾ-ಗಿತ್ತು
ಕೊಡ-ಲೇ-ಬಾ-ರದು
ಕೊಡ-ಲೇ-ಬೇಡಿ
ಕೊಡ-ವಿ-ಹಾಕಿ
ಕೊಡಿ
ಕೊಡಿ-ಸಲು
ಕೊಡಿ-ಸಿದ್ದೆ
ಕೊಡಿಸು
ಕೊಡಿ-ಸು-ವುದು
ಕೊಡು
ಕೊಡುಗೆ
ಕೊಡುತ್ತ
ಕೊಡು-ತ್ತದೆ
ಕೊಡು-ತ್ತಾ-ರೆಂದು
ಕೊಡು-ತ್ತಾ-ರೆಯೇ
ಕೊಡು-ತ್ತಿತ್ತು
ಕೊಡು-ತ್ತಿದ್ದ
ಕೊಡು-ತ್ತಿ-ದ್ದರು
ಕೊಡು-ತ್ತಿ-ದ್ದ-ರೆಂದು
ಕೊಡು-ತ್ತಿ-ದ್ದು-ದನ್ನು
ಕೊಡು-ತ್ತಿ-ದ್ದುದು
ಕೊಡು-ತ್ತಿರ
ಕೊಡು-ತ್ತಿ-ರ-ಬ-ಹು-ದೆಂದು
ಕೊಡು-ತ್ತಿರು
ಕೊಡು-ತ್ತೀಯಾ
ಕೊಡು-ತ್ತೇನೆ
ಕೊಡು-ತ್ತೇವೆ
ಕೊಡುವ
ಕೊಡು-ವಂತೆ
ಕೊಡು-ವ-ವ-ನಿ-ಗಿಂತ
ಕೊಡು-ವ-ವ-ರಲ್ಲ
ಕೊಡು-ವ-ವರೆ-ಗಿನ
ಕೊಡುವು
ಕೊಡು-ವುದನ್ನು
ಕೊಡು-ವು-ದ-ರ-ವ-ರೆಗೆ
ಕೊಡು-ವು-ದಾಗಿ
ಕೊಡು-ವು-ದಿಲ್ಲ
ಕೊಡು-ವುದು
ಕೊಡು-ವು-ದೆಂದು
ಕೊಡು-ವು-ದೇಕೆ
ಕೊಡು-ವುದೋ
ಕೊನೆ-ಗಾ-ಣು-ತ್ತಿದೆ
ಕೊನೆ-ಗಾ-ಲ-ದಲ್ಲಿ
ಕೊನೆಗೂ
ಕೊನೆಗೆ
ಕೊನೆ-ಗೊಂ-ಡಿತು
ಕೊನೆ-ಗೊಂ-ಡಿ-ದ್ದ-ರಿಂದ
ಕೊನೆ-ಗೊಂದು
ಕೊನೆ-ಗೊ-ಳಿ-ಸ-ಲಿ-ರು-ವಳೋ
ಕೊನೆ-ಗೊ-ಳ್ಳ-ಬೇಕು
ಕೊನೆ-ಗೊ-ಳ್ಳುವ
ಕೊನೆಯ
ಕೊನೆ-ಯ-ದಾಗಿ
ಕೊನೆ-ಯಲ್ಲಿ
ಕೊನೆ-ಯ-ವ-ರೆಗೂ
ಕೊನೆ-ಯಿ-ಲ್ಲದ
ಕೊನೆ-ಯು-ಸಿ-ರಿ-ನ-ವ-ರೆಗೂ
ಕೊನೆ-ಯು-ಸಿ-ರಿ-ರು-ವ-ವ-ರೆಗೂ
ಕೊನೆ-ಯು-ಸಿ-ರೆ-ಳೆದ
ಕೊನೆ-ಯು-ಸಿ-ರೆ-ಳೆ-ದರು
ಕೊನೆ-ಯು-ಸಿ-ರೆ-ಳೆ-ಯು-ತ್ತೇವೆ
ಕೊನೆ-ಯು-ಸಿ-ರೆ-ಳೆ-ಯು-ವ-ವ-ನನ್ನು
ಕೊನೆ-ಯು-ಸಿ-ರೆ-ಳೆ-ಯು-ವುದು
ಕೊಬ್ಬಿ
ಕೊಬ್ಬಿಸಿ
ಕೊರ-ಗ-ದಿ-ರಲಿ
ಕೊರ-ಗುತ್ತ
ಕೊರ-ಡಿ-ನಂತೆ
ಕೊರ-ಡು-ಗಳು
ಕೊರತೆ
ಕೊರ-ತೆ-ಯಿಂ-ದಾಗಿ
ಕೊರ-ತೆ-ಯಿ-ರ-ಲಿಲ್ಲ
ಕೊರ-ತೆಯೆ
ಕೊರಳ
ಕೊರ-ಳಲ್ಲಿ
ಕೊರ-ಳಿಗೂ
ಕೊರ-ಳಿಗೆ
ಕೊರ-ಳಿ-ನಲ್ಲಿ
ಕೊರೆಯು
ಕೊರೆ-ಯು-ತ್ತಿದೆ
ಕೊರೆ-ಯು-ತ್ತಿದ್ದ
ಕೊರೆ-ಯುವ
ಕೊರ್ಸಿಕಾ
ಕೊಲಂ-ಬೊ-ದಿಂದ
ಕೊಲಂಬೋ
ಕೊಲಂ-ಬೋಗೆ
ಕೊಲಂ-ಬೋದ
ಕೊಲಂ-ಬೋ-ದತ್ತ
ಕೊಲಂ-ಬೋ-ದಲ್ಲಿ
ಕೊಲಂ-ಬೋ-ದಲ್ಲೇ
ಕೊಲಂ-ಬೋ-ದಿಂದ
ಕೊಲಂ-ಬೋ-ವ-ರೆಗೆ
ಕೊಲೆ-ಗಾ-ರರೂ
ಕೊಲ್ಲ-ಲ್ಪ-ಟ್ಟದ್ದು
ಕೊಲ್ಲಿ-ಯನ್ನು
ಕೊಲ್ಲಿ-ಯನ್ನೂ
ಕೊಲ್ವ
ಕೊಳ-ಕಾ-ಗಿ-ದ್ದರೆ
ಕೊಳ-ಕಾ-ಗಿ-ರು-ತ್ತದೆ
ಕೊಳಕು
ಕೊಳ-ವೆ-ಯನ್ನು
ಕೊಳ-ವೆಯು
ಕೊಳೆ-ಗೇ-ರಿ-ಗಳನ್ನು
ಕೊಳೆತ
ಕೊಳೆತು
ಕೊಳೆ-ಯನ್ನು
ಕೊಳೆ-ಸು-ವಂ-ತಹ
ಕೊಳ್ಳ-ತೊ-ಡ-ಗಿದ
ಕೊಳ್ಳ-ಬ-ಲ್ಲಿರಾ
ಕೊಳ್ಳ-ಬ-ಲ್ಲೆ-ವಾ-ದರೆ
ಕೊಳ್ಳ-ಬ-ಹುದು
ಕೊಳ್ಳ-ಬೇಕು
ಕೊಳ್ಳ-ಬೇ-ಕೆಂದು
ಕೊಳ್ಳ-ಬೇಡಿ
ಕೊಳ್ಳ-ಲಾ-ಯಿತು
ಕೊಳ್ಳ-ಲಾ-ರೆ-ನೆಂದು
ಕೊಳ್ಳಲು
ಕೊಳ್ಳ-ಲು-ಐದೇ
ಕೊಳ್ಳುತ್ತ
ಕೊಳ್ಳು-ತ್ತಾನೋ
ಕೊಳ್ಳು-ತ್ತಾ-ರೆ-ಎಂಬ
ಕೊಳ್ಳು-ತ್ತಿ-ದ್ದರು
ಕೊಳ್ಳು-ತ್ತಿ-ದ್ದರೂ
ಕೊಳ್ಳು-ತ್ತಿ-ದ್ದೇನೆ
ಕೊಳ್ಳುವ
ಕೊಳ್ಳು-ವತ್ತ
ಕೊಳ್ಳು-ವುದನ್ನು
ಕೊಳ್ಳು-ವುದು
ಕೋಕಿ-ಲ-ಕಂ-ಠ-ದಿಂದ
ಕೋಟನ್ನು
ಕೋಟಿ
ಕೋಟಿ-ಗ-ಟ್ಟಲೆ
ಕೋಟಿನ
ಕೋಟೆ
ಕೋಟೆ-ಯನ್ನು
ಕೋಟ್ಯಂ-ತರ
ಕೋಟ್ಯ-ಧಿ-ಪ-ತಿ-ಯನ್ನು
ಕೋಟ್ಯ-ನು-ಕೋಟಿ
ಕೋಡಿ
ಕೋಣೆ
ಕೋಣೆ-ಗ-ಳ-ಲ್ಲದೆ
ಕೋಣೆ-ಗಳಲ್ಲಿ
ಕೋಣೆ-ಗ-ಳಿಗೆ
ಕೋಣೆ-ಗ-ಳಿ-ರ-ಬೇಕು
ಕೋಣೆ-ಗಳು
ಕೋಣೆಗೆ
ಕೋಣೆಯ
ಕೋಣೆ-ಯನ್ನು
ಕೋಣೆ-ಯಲ್ಲಿ
ಕೋಣೆ-ಯ-ಲ್ಲೂ-ಇ-ಬ್ಬರು
ಕೋಣೆ-ಯಲ್ಲೆ
ಕೋಣೆ-ಯಲ್ಲೇ
ಕೋಣೆ-ಯಿಂದ
ಕೋಣೆ-ಯೊ-ಳಗೆ
ಕೋಣೆ-ಯೊ-ಳಗೇ
ಕೋತಿ-ಗಳಲ್ಲಿ
ಕೋತಿ-ಗಳು
ಕೋಪ
ಕೋಪದ
ಕೋಪ-ದಿಂದ
ಕೋಪ-ವಿಲ್ಲ
ಕೋಪ-ವೆಲ್ಲ
ಕೋಪಿ-ಸಿ-ಕೊ-ಳ್ಳದೆ
ಕೋಮಿನ
ಕೋಮು
ಕೋಮು-ವಾರು
ಕೋರಿ
ಕೋರಿಕೆ
ಕೋರಿ-ಕೆ-ಭಾ-ರ-ತ-ವನ್ನು
ಕೋರಿ-ಕೆಯ
ಕೋರಿ-ಕೆ-ಯನ್ನು
ಕೋರಿ-ಕೆ-ಯಿಂದ
ಕೋರಿತು
ಕೋರಿ-ದ-ರ-ಲ್ಲದೆ
ಕೋರಿ-ದರು
ಕೋರು-ತ್ತಿ-ದ್ದಾರೆ
ಕೋರು-ವು-ದ-ಕ್ಕಾಗಿ
ಕೋರ್ಟು
ಕೋರ್ಟು-ಗಳಲ್ಲಿ
ಕೋಲಾ-ಹಲ
ಕೋಲಾ-ಹ-ಲ-ಕ್ಕೇನೂ
ಕೋಲಾ-ಹ-ಲ-ವನ್ನೇ
ಕೋಲಾ-ಹ-ಲ-ವೇ-ರ್ಪ-ಟ್ಟಿತ್ತು
ಕೋಳಿ
ಕೋಸಿಗೂ
ಕೌಂಟೆಸ್
ಕೌಟುಂ-ಬಿಕ
ಕೌತು-ಕ-ಮಯ
ಕೌಪೀನ
ಕೌಪೀ-ನಕ್ಕೂ
ಕೌಪೀ-ನ-ಧಾರಿ
ಕೌಪೀ-ನ-ವನ್ನು
ಕೌಶಲ
ಕೌಶ-ಲ-ದಿಂದ
ಕ್ಕಾಗಿ
ಕ್ಕಾಗಿಯೇ
ಕ್ಕಿಂತ
ಕ್ಕಿದ್ದಂತೆ
ಕ್ಕಿರುವ
ಕ್ಕೇರ-ಬೇ-ಕಾ-ಗಿದೆ
ಕ್ಕೇರ-ಲಾ-ರರು
ಕ್ಕೊಂದು
ಕ್ಕೊಪ್ಪದೆ
ಕ್ಕೊಮ್ಮೆ
ಕ್ತಿಕ
ಕ್ಯಾ
ಕ್ಯಾಂಡಿ
ಕ್ಯಾಂಡಿಯ
ಕ್ಯಾಂಪೊ-ಸ್ಯಾಂಟೊ
ಕ್ಯಾಂಪ್
ಕ್ಯಾಟ್ಸ್ಕಿಲ್
ಕ್ಯಾಥೋ-ಲಿ-ಕರ
ಕ್ಯಾಥೋ-ಲಿಕ್
ಕ್ಯಾಪಿ-ಟ-ಲೀನ್
ಕ್ಯಾಪಿ-ಟಲ್
ಕ್ಯಾಪ್ಟನ್
ಕ್ಯಾಪ್ಟ-ನ್ನನೂ
ಕ್ಯಾಮೆರಾ
ಕ್ಯಾರೀ
ಕ್ಯಾಲಿ
ಕ್ಯಾಲಿ-ಪೋ-ರ್ನಿಯಾ
ಕ್ಯಾಲಿ-ಫೋ-ರ್ನಿಯ
ಕ್ಯಾಲಿ-ಫೋ-ರ್ನಿ-ಯಕ್ಕೆ
ಕ್ಯಾಲಿ-ಫೋ-ರ್ನಿ-ಯದ
ಕ್ಯಾಲಿ-ಫೋ-ರ್ನಿ-ಯ-ದತ್ತ
ಕ್ಯಾಲಿ-ಫೋ-ರ್ನಿ-ಯ-ದಲ್ಲಿ
ಕ್ಯಾಲಿ-ಫೋ-ರ್ನಿ-ಯ-ದ-ಲ್ಲಿದ್ದ
ಕ್ಯಾಲಿ-ಫೋ-ರ್ನಿ-ಯ-ದಲ್ಲೇ
ಕ್ಯಾಲಿ-ಫೋ-ರ್ನಿ-ಯ-ದಿಂದ
ಕ್ಯಾಲಿ-ಫೋ-ರ್ನಿ-ಯ-ದೆ-ಡೆಗೆ
ಕ್ಯಾಲಿ-ಫೋ-ರ್ನಿ-ಯ-ವ-ವರೆ-ಗಿನ
ಕ್ಯಾಲಿ-ಫೋ-ರ್ನಿಯಾ
ಕ್ಯಾಲಿ-ಫೋ-ರ್ನಿ-ಯಾಕ್ಕೆ
ಕ್ಯಾಲ್ಕಿ-ನ್ಸ್
ಕ್ಯಾಸಲ್
ಕ್ರಮ
ಕ್ರಮಕ್ಕೂ
ಕ್ರಮಕ್ಕೆ
ಕ್ರಮ-ಗಳ
ಕ್ರಮ-ಗಳನ್ನು
ಕ್ರಮ-ಗಳು
ಕ್ರಮ-ದಲ್ಲಿ
ಕ್ರಮ-ವಾಗಿ
ಕ್ರಮ-ವಿದೆ
ಕ್ರಮ-ವೊಂದು
ಕ್ರಮಾ-ಗ-ತ-ವಾಗಿ
ಕ್ರಮಿ-ಸಲು
ಕ್ರಮಿಸಿ
ಕ್ರಮೇಣ
ಕ್ರಾಂತಿ-ಕಾ-ರಕ
ಕ್ರಾಂತಿ-ಕಾರಿ
ಕ್ರಾಂತಿ-ಕಾ-ರಿ-ಗಳ
ಕ್ರಾಂತಿ-ಕಾ-ರಿ-ಗ-ಳೆಲ್ಲ
ಕ್ರಾಂತಿ-ಕಾರೀ
ಕ್ರಾಂತಿಯ
ಕ್ರಾಂತಿ-ಯನ್ನೇ
ಕ್ರಾಂತಿ-ಯೆಂ-ಬುದು
ಕ್ರಿಮಿ-ಕೀ-ಟ-ಗ-ಳಂತೆ
ಕ್ರಿಮಿ-ಗಳು
ಕ್ರಿಮಿ-ನಾ-ಶ-ಕ-ಗಳನ್ನು
ಕ್ರಿಯಾ-ಕ-ಲಾ-ಪ-ಗಳಲ್ಲಿ
ಕ್ರಿಯಾ-ಕ-ಲಾ-ಪ-ಗಳು
ಕ್ರಿಯಾ-ತ್ಮ-ಕ-ವಾ-ಗಿ-ಸ-ಕಾ-ರಾ-ತ್ಮ-ಕ-ವಾಗಿ
ಕ್ರಿಯಾ-ಶೀಲ
ಕ್ರಿಯಾ-ಶೀ-ಲ-ರಾದ
ಕ್ರಿಯೆಯ
ಕ್ರಿಶ್ಚಿ-ಯನ್
ಕ್ರಿಸ್ಟೀನ
ಕ್ರಿಸ್ಟೀ-ನಳ
ಕ್ರಿಸ್ಟೀ-ನ-ಳಂ-ತೆಯೇ
ಕ್ರಿಸ್ಟೀ-ನ-ಳಿಗೆ
ಕ್ರಿಸ್ಟೀ-ನ-ಳಿ-ಗೊಂದು
ಕ್ರಿಸ್ಟೀ-ನಳು
ಕ್ರಿಸ್ಟೀ-ನಳೂ
ಕ್ರಿಸ್ಟೀ-ನ-ಳೊಂ-ದಿಗೆ
ಕ್ರಿಸ್ಟೀನಾ
ಕ್ರಿಸ್ಟೋ-ಫರ್
ಕ್ರಿಸ್ತ
ಕ್ರಿಸ್ತನ
ಕ್ರಿಸ್ತ-ನಂ-ತಹ
ಕ್ರಿಸ್ತ-ನಂತೆ
ಕ್ರಿಸ್ತ-ನಂ-ತೆಯೇ
ಕ್ರಿಸ್ತ-ನಿ-ಗಿಂತ
ಕ್ರಿಸ್ತ-ನಿಗೂ
ಕ್ರಿಸ್ತನು
ಕ್ರಿಸ್ತನೊ
ಕ್ರಿಸ್ತ-ಭ-ಗ-ವಂ-ತನೇ
ಕ್ರಿಸ್ತ-ಭಾ-ವ-ದಲ್ಲಿ
ಕ್ರಿಸ್ತರೇ
ಕ್ರಿಸ್ಮಸ್
ಕ್ರಿಸ್ಮಸ್ನ
ಕ್ರೀಟ್
ಕ್ರೀಡಾಂ-ಗ-ಣ-ಗಳು
ಕ್ರೂರ
ಕ್ರೈಸ್ತ
ಕ್ರೈಸ್ತ-ಧ-ರ್ಮದ
ಕ್ರೈಸ್ತ-ಧ-ರ್ಮ-ದಲ್ಲಿ
ಕ್ರೈಸ್ತ-ಧ-ರ್ಮ-ವನ್ನು
ಕ್ರೈಸ್ತ-ಧ-ರ್ಮವೇ
ಕ್ರೈಸ್ತ-ಧ-ರ್ಮಾ-ಧಿ-ಕಾ-ರಿ-ಗಳ
ಕ್ರೈಸ್ತ-ಧ-ರ್ಮೀ-ಯರ
ಕ್ರೈಸ್ತ-ಪಾ-ದ್ರಿ-ಗಳು
ಕ್ರೈಸ್ತ-ಮ-ತೀ-ಯರ
ಕ್ರೈಸ್ತ-ಮ-ತೀ-ಯ-ಳಾ-ದರೂ
ಕ್ರೈಸ್ತರ
ಕ್ರೈಸ್ತ-ರಾ-ಗಲಿ
ಕ್ರೈಸ್ತ-ರಾಗಿ
ಕ್ರೈಸ್ತ-ರಾ-ಗಿ-ರ-ಬ-ಹುದು
ಕ್ರೈಸ್ತ-ರಾ-ದ್ದ-ರಿಂದ
ಕ್ರೈಸ್ತ-ರಿ-ದ್ದರು
ಕ್ರೈಸ್ತರು
ಕ್ರೈಸ್ತೀ-ಕ-ರಣ
ಕ್ರೈಸ್ತೀ-ಕ-ರ-ಣ-ಗೊ-ಳಿ-ಸಿ-ಬಿ-ಟ್ಟರೆ
ಕ್ರೋಡೀ-ಕ-ರಿಸಿ
ಕ್ರೋಧದ
ಕ್ರೌರ್ಯ-ಅ-ಪ-ರಾಧ
ಕ್ಲಬ್
ಕ್ಲಬ್ನಲ್ಲಿ
ಕ್ಲಿಷ್ಟ-ವಾ-ದ-ವು-ಗಳು
ಕ್ಲೇರ್
ಕ್ಷಂತವ್ಯಂ
ಕ್ಷಣ
ಕ್ಷಣ-ಕಾಲ
ಕ್ಷಣಕ್ಕೆ
ಕ್ಷಣ-ಕ್ಷ-ಣಕ್ಕೂ
ಕ್ಷಣ-ಗ-ಳಿಂ-ದಲೂ
ಕ್ಷಣ-ದ-ಲ್ಲಾ-ದರೂ
ಕ್ಷಣ-ದಲ್ಲಿ
ಕ್ಷಣ-ದ-ಲ್ಲಿಯೇ
ಕ್ಷಣ-ದಲ್ಲೂ
ಕ್ಷಣ-ದಲ್ಲೇ
ಕ್ಷಣ-ದಿಂದ
ಕ್ಷಣ-ದಿಂ-ದಲೇ
ಕ್ಷಣ-ಭಂ-ಗು-ರ-ತೆಯು
ಕ್ಷಣ-ಮಾ-ತ್ರ-ದಲ್ಲಿ
ಕ್ಷಣ-ಮಾ-ತ್ರ-ವಾ-ದರೂ
ಕ್ಷಣವೂ
ಕ್ಷಣವೇ
ಕ್ಷಣಾ-ರ್ಧ-ದಲ್ಲಿ
ಕ್ಷಣಿಕ
ಕ್ಷತ್ರಿಯ
ಕ್ಷತ್ರಿ-ಯ-ವೈಶ್ಯ
ಕ್ಷತ್ರಿ-ಯನ
ಕ್ಷತ್ರಿ-ಯ-ನಿ-ಗಿಂತ
ಕ್ಷತ್ರಿ-ಯನು
ಕ್ಷತ್ರಿ-ಯ-ರಾ-ಜ-ನಂತೆ
ಕ್ಷತ್ರಿ-ಯ-ರಿಗೂ
ಕ್ಷತ್ರಿ-ಯರು
ಕ್ಷಮಾ-ಶೀ-ಲರು
ಕ್ಷಮಿ-ಸ-ಬೇ-ಕೆಂದೂ
ಕ್ಷಮಿಸಿ
ಕ್ಷಮಿ-ಸು-ವಂತೆ
ಕ್ಷಮಿ-ಸು-ವ-ವ-ರೆಗೂ
ಕ್ಷಮಿ-ಸು-ವಿ-ರಂತೆ
ಕ್ಷಮೆ
ಕ್ಷಮೆಗೆ
ಕ್ಷಮೆ-ಯನ್ನು
ಕ್ಷಮ್ಯ-ವಾದ
ಕ್ಷಾತ್ರ-ವನ್ನು
ಕ್ಷಾತ್ರ-ವೀ-ರ್ಯ-ವನ್ನು
ಕ್ಷಾತ್ರ-ಶ-ಕ್ತಿ-ಯನ್ನು
ಕ್ಷಾಮ
ಕ್ಷಾಮ
ಕ್ಷಾಮ-ಗಳು
ಕ್ಷಾಮ-ಗೀ-ಮ-ದಲ್ಲಿ
ಕ್ಷಾಮ-ಡಾ-ಮ-ರ-ಗಳ
ಕ್ಷಾಮ-ಡಾ-ಮ-ರ-ಗ-ಳನ್ನೇ
ಕ್ಷಾಮ-ಡಾ-ಮ-ರ-ಗಳು
ಕ್ಷಾಮ-ಪ-ರಿ-ಸ್ಥಿತಿ
ಕ್ಷಾಮ-ಪ-ರಿ-ಹಾರ
ಕ್ಷಾಮ-ಸಂ-ತ್ರ-ಸ್ತ-ರಿ-ಗಾಗಿ
ಕ್ಷೀಣ-ಗೊಂಡ
ಕ್ಷೀಣ-ವಾ-ಗಿ-ದ್ದುದು
ಕ್ಷೀಣ-ವಾ-ಗುತ್ತ
ಕ್ಷೀಣಿ-ಸುತ್ತ
ಕ್ಷೀಣಿ-ಸು-ತ್ತಿದ್ದ
ಕ್ಷೀರ-ಭ-ವಾನಿ
ಕ್ಷೀರ-ಭ-ವಾ-ನಿಗೂ
ಕ್ಷೀರ-ಭ-ವಾ-ನಿಗೆ
ಕ್ಷೀರ-ಭ-ವಾ-ನಿಯ
ಕ್ಷೀರ-ಭ-ವಾ-ನಿ-ಯಲ್ಲಿ
ಕ್ಷುದ್ರ
ಕ್ಷುದ್ರ-ಜಂ-ತು-ಗಳೂ
ಕ್ಷುದ್ರ-ವಾದ
ಕ್ಷುದ್ರ-ವಾ-ದದ್ದು
ಕ್ಷುಲ್ಲ-ಕ-ವಾದ
ಕ್ಷೇತ್ರ
ಕ್ಷೇತ್ರಕ್ಕೆ
ಕ್ಷೇತ್ರ-ಗಳನ್ನು
ಕ್ಷೇತ್ರ-ಗಳಲ್ಲಿ
ಕ್ಷೇತ್ರ-ಗ-ಳಲ್ಲೂ
ಕ್ಷೇತ್ರದ
ಕ್ಷೇತ್ರ-ದಲ್ಲಿ
ಕ್ಷೇತ್ರ-ದ-ಲ್ಲಿನ
ಕ್ಷೇತ್ರ-ಪಂ-ಜಾ-ಬ್ದ-ಲ್ಲಿಯೇ
ಕ್ಷೇತ್ರ-ವನ್ನು
ಕ್ಷೇತ್ರ-ವನ್ನೇ
ಕ್ಷೇತ್ರ-ವಾದ
ಕ್ಷೇತ್ರ-ವಿ-ರು-ವುದು
ಕ್ಷೇತ್ರವೂ
ಕ್ಷೇಮ
ಕ್ಷೇಮ-ವಾಗಿ
ಕ್ಷೋಭೆ
ಖಂಡ-ದಲ್ಲಿ
ಖಂಡ-ದಿಂದ
ಖಂಡ-ದಿಂ-ದಲೇ
ಖಂಡನ
ಖಂಡ-ನೆಯ
ಖಂಡಾಂ-ತರ
ಖಂಡಿ
ಖಂಡಿತ
ಖಂಡಿ-ತ-ವಾಗಿ
ಖಂಡಿ-ತ-ವಾ-ಗಿಯೂ
ಖಂಡಿ-ಸ-ಬೇ-ಕೆಂದು
ಖಂಡಿಸಿ
ಖಂಡಿ-ಸಿದ
ಖಂಡಿ-ಸಿ-ದ-ರಾ-ದರೂ
ಖಂಡಿ-ಸಿ-ದರು
ಖಂಡಿ-ಸಿ-ದ-ವ-ರಲ್ಲ
ಖಂಡಿ-ಸಿ-ದಾಗ
ಖಂಡಿ-ಸಿ-ರುವ
ಖಂಡಿ-ಸುತ್ತ
ಖಂಡಿ-ಸು-ತ್ತಾರೆ
ಖಂಡಿ-ಸು-ತ್ತಿ-ದ್ದ-ರಾ-ದರೂ
ಖಂಡಿ-ಸು-ತ್ತಿ-ದ್ದರು
ಖಂಡಿ-ಸು-ವುದು
ಖಗೇಂದ್ರ
ಖಗೇನ್
ಖಚಿತ
ಖಚಿ-ತ-ವಾಗಿ
ಖಚಿ-ತ-ವಾ-ಗಿತ್ತು
ಖಚಿ-ತ-ವಾ-ಗು-ತ್ತಿತ್ತು
ಖಚಿ-ತ-ವಾದ
ಖಜಾಂ-ಚಿ-ಗಳು
ಖಜಾಂ-ಚಿ-ಯಾದ
ಖಡ-ಕ್ಕಾಗಿ
ಖಡಾ-ಖಂ-ಡಿ-ತ-ವಾಗಿ
ಖಡ್ಗ
ಖನಿ
ಖನಿ-ಗ-ಳ-ಲ್ಲೊಂದು
ಖನಿ-ಯಾದ
ಖರೀ-ದಿಸಿ
ಖರೀ-ದಿಸು
ಖರ್ಚನ್ನು
ಖರ್ಚ-ನ್ನೆಲ್ಲ
ಖರ್ಚಾ-ಗಿ-ಹೋ-ಗುವ
ಖರ್ಚಿ-ಗಾಗಿ
ಖರ್ಚಿಗೆ
ಖರ್ಚಿ-ನಲ್ಲೇ
ಖರ್ಚು
ಖರ್ಚು-ಗಳನ್ನೂ
ಖರ್ಚು-ವೆ-ಚ್ಚ-ವ-ನ್ನೆಲ್ಲ
ಖಾಂಡ್ವಾ
ಖಾಂಡ್ವಾ-ದಲ್ಲಿ
ಖಾಂಡ್ವಾ-ದಿಂದ
ಖಾನೆ
ಖಾಯಂ
ಖಾರ
ಖಾರ-ವಾಗಿ
ಖಾರ-ವಾ-ಗಿಯೇ
ಖಾಲಿ
ಖಾಲಿ-ಮಾ-ಡ-ಲಾ-ಯಿತು
ಖಾಸಗಿ
ಖಾಸ-ಗಿ-ಯಾಗಿ
ಖಾಸಗೀ
ಖೀರನ್ನು
ಖುದ್ದಾಗಿ
ಖುಲಾ-ಯಿ-ಸ-ಲಿಲ್ಲ
ಖುಶಿ
ಖುಷಿ-ಗೊ-ಳಿ-ಸಲು
ಖುಷಿ-ಯಾಗಿ
ಖುಷಿ-ಯಾ-ಗಿದೆ
ಖುಷಿ-ಯಾ-ಗಿ-ದ್ದಾ-ರಲ್ಲ
ಖೇತ್ರಿ
ಖೇತ್ರಿಗೆ
ಖೇತ್ರಿಯ
ಖೇತ್ರಿ-ಯತ್ತ
ಖೇತ್ರಿ-ಯಲ್ಲಿ
ಖೇತ್ರಿ-ಯಿಂದ
ಖೇದ-ಗೊಂ-ಡ-ರಾ-ದರೂ
ಖೇದದ
ಖೇದ-ವಾ-ಯಿತು
ಖೋಲನ್ನು
ಖೋಲ್
ಖ್ಯಾತ
ಖ್ಯಾತಿ-ಗೊಂ-ಡಿರು
ಗಂಗಾ
ಗಂಗಾ-ಜ-ಲ-ವನ್ನು
ಗಂಗಾ-ತೀ-ರ-ದಲ್ಲಿ
ಗಂಗಾ-ತೀ-ರ-ದ-ಲ್ಲಿ-ದ್ದು-ದ-ರಿಂದ
ಗಂಗಾ-ತೀ-ರ-ದ-ಲ್ಲಿನ
ಗಂಗಾ-ಧರ
ಗಂಗಾ-ನದಿ
ಗಂಗಾ-ನ-ದಿಯ
ಗಂಗಾ-ನ-ದಿ-ಯಂ-ತಿ-ರು-ತ್ತಿತ್ತು
ಗಂಗಾ-ನ-ದಿ-ಯ-ಗ-ಲಕ್ಕೂ
ಗಂಗಾ-ನ-ದಿ-ಯಲ್ಲಿ
ಗಂಗೆಗೆ
ಗಂಗೆಯ
ಗಂಗೆ-ಯತ್ತ
ಗಂಗೆ-ಯನ್ನು
ಗಂಗೆ-ಯಲ್ಲಿ
ಗಂಗೆಯು
ಗಂಗೋ-ದ-ಕ-ದಿಂದ
ಗಂಜಿ
ಗಂಟ-ಲಿನ
ಗಂಟಲು
ಗಂಟ-ಲೊ-ಳಗೆ
ಗಂಟು
ಗಂಟು-ಗಳನ್ನು
ಗಂಟು-ಬಿ-ದ್ದರು
ಗಂಟು-ಬಿ-ದ್ದಿ-ದ್ದರೆ
ಗಂಟು-ಹಾ-ಕಿ-ದ್ದೇವೆ
ಗಂಟೆ
ಗಂಟೆ-ಗ-ಟ್ಟಲೆ
ಗಂಟೆ-ಗಳ
ಗಂಟೆ-ಗಳನ್ನು
ಗಂಟೆ-ಗ-ಳಲ್ಲೇ
ಗಂಟೆ-ಗ-ಳಷ್ಟು
ಗಂಟೆ-ಗಳು
ಗಂಟೆ-ಗಳೇ
ಗಂಟೆಗೂ
ಗಂಟೆಗೆ
ಗಂಟೆ-ಗೊಮ್ಮೆ
ಗಂಟೆಯ
ಗಂಟೆ-ಯ-ವ-ರೆಗೂ
ಗಂಟೆ-ಯ-ವ-ರೆಗೆ
ಗಂಟೆ-ಯಿಂ-ದಲೇ
ಗಂಟೆಯೂ
ಗಂಟೆ-ಯೆಂ-ಬಂತೆ
ಗಂಡ
ಗಂಡ-ನನ್ನು
ಗಂಡ-ನಿಗೆ
ಗಂಡ-ನಿ-ಲ್ಲ-ದಂತಾ
ಗಂಡ-ಸ-ರಿಗೇ
ಗಂಡ-ಸರು
ಗಂಡ-ಸಿನ
ಗಂಡಾಂ-ತ-ರ-ವ-ನ್ನೆ-ದು-ರಿ-ಸಿ-ದಾ-ಗ-ಲೆಲ್ಲ
ಗಂಡು
ಗಂಡು-ಗ-ಲಿ-ಗ-ಳ-ಲ್ಲೊಬ್ಬ
ಗಂಧ-ಗಳ
ಗಂಧ-ಗಳನ್ನು
ಗಂಧ-ಚಂ-ದ-ನ-ಗಳನ್ನು
ಗಂಧದ
ಗಂಧ-ರ್ವ-ರೂ-ಪಿ-ಯಾದ
ಗಂಧ-ರ್ವ-ಸ್ತಸ್ಯ
ಗಂಭೀರ
ಗಂಭೀ-ರದ
ಗಂಭೀ-ರ-ಭಾವ
ಗಂಭೀ-ರ-ಭಾ-ವ-ದಲ್ಲಿ
ಗಂಭೀ-ರ-ವ-ದ-ನ-ರಾಗಿ
ಗಂಭೀ-ರ-ವಾಗಿ
ಗಂಭೀ-ರ-ವಾ-ಗಿತ್ತು
ಗಂಭೀ-ರ-ವಾ-ಗಿದೆ
ಗಂಭೀ-ರ-ವಾ-ಗಿ-ಬಿ-ಡು-ತ್ತಿ-ದ್ದರು
ಗಂಭೀ-ರ-ವಾ-ಗಿ-ರು-ವಾಗ
ಗಂಭೀ-ರ-ವಾ-ಗಿ-ರು-ವುದನ್ನು
ಗಂಭೀ-ರ-ವಾದ
ಗಂಭೀ-ರ-ವಾ-ಯಿತು
ಗಂಭೀ-ರ-ವಾ-ಹಿ-ನಿ-ಗ-ಳಾದ
ಗಕ್ಕೂ
ಗಗನ
ಗಗ-ನ-ದಿಂದ
ಗಗ-ನ-ದೆ-ತ್ತ-ರಕ್ಕೆ
ಗಚ್ಛತು
ಗಟ್ಟ-ಲೆ-ಯಿಂದ
ಗಟ್ಟಿ
ಗಟ್ಟಿ-ಮ-ನ-ಸ್ಸಿ-ನ-ವ-ನಾ-ಗಿ-ದ್ದೇನೆ
ಗಟ್ಟಿ-ಯಾಗಿ
ಗಟ್ಟಿ-ಯಾದ
ಗಟ್ಟಿ-ಯಾ-ಯಿತು
ಗಡ-ಗಡ
ಗಡ-ಗ-ಡನೆ
ಗಡಿ-ಯಲ್ಲಿ
ಗಡಿ-ಯಾರ
ಗಡಿ-ಯಾ-ರದ
ಗಡಿ-ಯಾ-ರ-ವನ್ನು
ಗಡೀ-ಪಾ-ರಿನ
ಗಡೀ-ಪಾರು
ಗಡ್ಡ-ಧಾ-ರಿ-ಯಾದ
ಗಣ
ಗಣ-ನೀ-ಯ-ವಾ-ಗಿ-ರುವ
ಗಣ-ನೆಗೆ
ಗಣ-ನೆಗೇ
ಗಣಿ
ಗಣಿತ
ಗಣಿ-ತ-ಶಾ-ಸ್ತ್ರದ
ಗಣಿ-ಯಾದ
ಗಣಿ-ಸದೆ
ಗಣೇ-ಶನ
ಗಣೇಶ್
ಗಣ್ಯ
ಗಣ್ಯರು
ಗಣ್ಯರೂ
ಗಣ್ಯ-ವ್ಯ-ಕ್ತಿ-ಗಳ
ಗಣ್ಯ-ವ್ಯ-ಕ್ತಿ-ಗಳನ್ನು
ಗಣ್ಯ-ವ್ಯ-ಕ್ತಿ-ಗಳಲ್ಲಿ
ಗಣ್ಯ-ವ್ಯ-ಕ್ತಿ-ಗ-ಳಾದ
ಗಣ್ಯ-ವ್ಯ-ಕ್ತಿ-ಗ-ಳಿಗೆ
ಗಣ್ಯ-ವ್ಯ-ಕ್ತಿ-ಗಳು
ಗಣ್ಯ-ವ್ಯ-ಕ್ತಿ-ಗ-ಳೆಲ್ಲ
ಗಣ್ಯ-ವ್ಯ-ಕ್ತಿ-ಗ-ಳೊಂ-ದಿಗೆ
ಗಣ್ಯ-ವ್ಯ-ಕ್ತಿ-ಯೆಂ-ದರೆ
ಗತ-ಕಾ-ಲದ
ಗತ-ಗೊ-ಳಿ-ಸ-ಲಿ-ದ್ದುದು
ಗತ-ಗೊ-ಳಿ-ಸುವ
ಗತ-ವಾ-ಗಲಿ
ಗತ-ವೈ-ಭ-ವ-ವನ್ನು
ಗತಿ
ಗತಿ-ಗೋ-ತ್ರ-ವಿ-ಲ್ಲದ
ಗತಿಯ
ಗತಿ-ಯನ್ನೇ
ಗತಿ-ಯಲ್ಲಿ
ಗತಿ-ಯಾ-ದ್ದ-ರಿಂದ
ಗತಿ-ಯಾ-ಯಿತು
ಗತಿ-ಯಿ-ಲ್ಲ-ದಂ-ತಾ-ಗಿದೆ
ಗತಿ-ಯಿ-ಲ್ಲ-ದಿ-ರು-ವಾಗ
ಗತಿ-ಯೇನು
ಗತ್ತನ್ನು
ಗತ್ಯಂ-ತ-ರ-ವಿಲ್ಲ
ಗದರಿ
ಗದ-ರಿ-ಸ-ಬೇ-ಕಾ-ಯಿತು
ಗದ-ರಿ-ಸಿ-ದಾಗ
ಗದೆ
ಗದ್ಗದ
ಗದ್ಗ-ದ-ವಾ-ಯಿತು
ಗದ್ದ-ಲ-ನೂ-ಕು-ನು-ಗ್ಗ-ಲಿ-ನಲ್ಲಿ
ಗದ್ದ-ಲ-ದಿಂದ
ಗದ್ದ-ಲ-ವನ್ನು
ಗದ್ದ-ಲ-ವಿ-ದ್ದು-ದ-ರಿಂದ
ಗದ್ದೆ-ಗಳು
ಗನು-ಸಾ-ರ-ವಾಗಿ
ಗನ್ನಿನ
ಗಮನ
ಗಮ-ನಕ್ಕೂ
ಗಮ-ನಕ್ಕೆ
ಗಮ-ನ-ದ-ಲ್ಲಿ-ಟ್ಟು-ಕೊಂಡು
ಗಮ-ನ-ವನ್ನು
ಗಮ-ನ-ವಾ-ಗಲಿ
ಗಮ-ನ-ವಿಟ್ಟು
ಗಮ-ನ-ಸೆ-ಳೆ-ದರು
ಗಮ-ನಾರ್ಹ
ಗಮ-ನಾ-ರ್ಹ-ವಾಗಿ
ಗಮ-ನಾ-ರ್ಹ-ವಾದ
ಗಮ-ನಾ-ರ್ಹ-ವಾ-ದಂ-ಥದು
ಗಮ-ನಾ-ರ್ಹ-ವಾ-ದದ್ದು
ಗಮ-ನಿ-ಸದೆ
ಗಮ-ನಿ-ಸ-ಬಲ್ಲ
ಗಮ-ನಿ-ಸ-ಬೇಕು
ಗಮ-ನಿ-ಸ-ಲಿ-ಲ್ಲ-ವೆಂ-ದಲ್ಲ
ಗಮ-ನಿ-ಸಲು
ಗಮ-ನಿಸಿ
ಗಮ-ನಿ-ಸಿತು
ಗಮ-ನಿ-ಸಿದ
ಗಮ-ನಿ-ಸಿ-ದರು
ಗಮ-ನಿ-ಸಿ-ದರೆ
ಗಮ-ನಿ-ಸಿ-ದರೋ
ಗಮ-ನಿ-ಸಿ-ದಳೋ
ಗಮ-ನಿ-ಸಿ-ದಾಗ
ಗಮ-ನಿ-ಸಿದೆ
ಗಮ-ನಿ-ಸಿ-ದ್ದರು
ಗಮ-ನಿ-ಸಿ-ದ್ದರೂ
ಗಮ-ನಿ-ಸಿದ್ದೆ
ಗಮ-ನಿ-ಸಿಯೂ
ಗಮ-ನಿಸು
ಗಮ-ನಿ-ಸುತ್ತ
ಗಮ-ನಿ-ಸು-ತ್ತಲೇ
ಗಮ-ನಿ-ಸು-ತ್ತಿ-ದ್ದರು
ಗಮ-ನಿ-ಸು-ತ್ತಿ-ದ್ದುವು
ಗಮ-ನಿ-ಸು-ತ್ತಿ-ರ-ಲಿಲ್ಲ
ಗಮ-ನಿ-ಸುವ
ಗಮ-ನಿ-ಸು-ವ-ವ-ರಲ್ಲ
ಗಮ-ನಿ-ಸು-ವ-ಷ್ಟ-ರಲ್ಲೇ
ಗಮ-ನೀಯ
ಗಮ-ನೀ-ಯ-ವಾದ
ಗಮಿ-ನಿ-ಸಿದ
ಗಮ್ಯ-ವಾದ
ಗಯೆಗೆ
ಗಯೆಯ
ಗಯೆ-ಯನ್ನು
ಗಯೆ-ಯಲ್ಲಿ
ಗರ-ಮಾ-ಗರಂ
ಗರಿ-ಗಳನ್ನೂ
ಗರಿ-ಗಳಿಂದ
ಗರಿ-ಗೆ-ದರಿ
ಗರಿ-ಮೆ-ಗಳನ್ನು
ಗರಿ-ಮೆ-ಯನ್ನು
ಗರು
ಗರೆ-ದರು
ಗರ್ಗ-ರಿ-ಸುತ
ಗರ್ಜನೆ
ಗರ್ಜ-ನೆಯೂ
ಗರ್ಜ-ರಿ-ಸುತ
ಗರ್ಜಿಸಿ
ಗರ್ಜಿ-ಸಿ-ದರು
ಗರ್ಜಿ-ಸಿ-ದ-ವರು
ಗರ್ಜಿ-ಸು-ತ್ತಾರೆ
ಗರ್ನ್ಸೆ-ಯ-ವರ
ಗರ್ನ್ಸೇ
ಗರ್ನ್ಸೇ-ಯ-ವರ
ಗರ್ಭ-ಗು-ಡಿಗೆ
ಗರ್ಭ-ಗು-ಡಿ-ಯಲ್ಲಿ
ಗರ್ಭ-ಗು-ಡಿ-ಯ-ಲ್ಲಿ-ಅ-ರ್ಚಕ
ಗರ್ಭ-ದಲ್ಲಿ
ಗರ್ಭ-ಪಾತ
ಗರ್ಭಿ-ಣಿ-ಯಾಗಿ
ಗರ್ಲ್ಸ್
ಗರ್ವಿ-ಷ್ಠಳೂ
ಗಲಭೆ
ಗಲ-ಭೆ-ಗ-ದ್ದಲ
ಗಲ-ಭೆಗೆ
ಗಲ-ಭೆಯ
ಗಲ-ಭೆ-ಯ-ನ್ನೆ-ಬ್ಬಿ-ಸಲು
ಗಲ-ಭೆ-ಯಾ-ಗಿತ್ತು
ಗಲಾಟೆ
ಗಲು
ಗಲೂ
ಗಲೇ
ಗಲ್ಲಿ-ಗ-ಲ್ಲಿ-ಗಳನ್ನೂ
ಗಳ
ಗಳಂ-ತಹ
ಗಳಂತೆ
ಗಳ-ನ್ನ-ರ್ಪಿಸಿ
ಗಳ-ನ್ನ-ರ್ಪಿ-ಸಿ-ದರು
ಗಳ-ನ್ನ-ವ-ಲಂ-ಬಿ-ಸಿದ
ಗಳ-ನ್ನಾ-ಡಿ-ದ-ರಂತೂ
ಗಳ-ನ್ನಿಟ್ಟು
ಗಳ-ನ್ನಿ-ರಿ-ಸ-ಲಾ-ಗಿತ್ತು
ಗಳ-ನ್ನಿ-ರಿ-ಸಿ-ದರು
ಗಳನ್ನು
ಗಳನ್ನೂ
ಗಳ-ನ್ನೆ-ಣಿ-ಸದೆ
ಗಳ-ನ್ನೆಲ್ಲ
ಗಳನ್ನೇ
ಗಳನ್ನೋ
ಗಳಲ್ಲಿ
ಗಳ-ಲ್ಲಿನ
ಗಳಲ್ಲೂ
ಗಳ-ಲ್ಲೂ-ಅ-ನೌ-ಪ-ಚಾ-ರಿಕ
ಗಳ-ಲ್ಲೆಲ್ಲ
ಗಳಲ್ಲೇ
ಗಳ-ಲ್ಲೊಂ-ದಾ-ಗಿತ್ತು
ಗಳ-ಲ್ಲೊಂದು
ಗಳ-ವ-ರೆಗೆ
ಗಳ-ಷ್ಟಾ-ದರೂ
ಗಳಾ-ಗ-ದಂತೆ
ಗಳಾ-ಗ-ಬೇ-ಕೆಂದು
ಗಳಾ-ಗಲಿ
ಗಳಾಗಿ
ಗಳಾ-ಗಿ-ಬಿ-ಡು-ತ್ತಿ-ದ್ದರು
ಗಳಾಚೆ
ಗಳಾದ
ಗಳಾರೂ
ಗಳಿಂದ
ಗಳಿಂ-ದಲೂ
ಗಳಿ-ಗಾಗಿ
ಗಳಿ-ಗಿಂತ
ಗಳಿ-ಗಿಂ-ತಲೂ
ಗಳಿಗೂ
ಗಳಿಗೆ
ಗಳಿ-ಗೆಲ್ಲ
ಗಳಿ-ಗೆ-ಲ್ಲಕ್ಕೂ
ಗಳಿ-ಗೆ-ಸಂ-ಸ್ಕೃತ
ಗಳಿಗೇ
ಗಳಿ-ರ-ಲೇ-ಬೇಕು
ಗಳಿ-ರುವ
ಗಳಿ-ಲ್ಲದೆ
ಗಳಿವೆ
ಗಳಿ-ಸ-ಬ-ಹು-ದಾದ
ಗಳಿ-ಸ-ಬೇ-ಕೆಂಬ
ಗಳಿ-ಸಿ-ಕೊಂಡೆ
ಗಳಿ-ಸಿ-ಕೊ-ಟ್ಟಿದೆ
ಗಳಿ-ಸಿ-ಕೊ-ಳ್ಳಲೂ
ಗಳಿ-ಸಿ-ಕೊ-ಳ್ಳು-ತ್ತಾರೆ
ಗಳಿ-ಸಿ-ಕೊ-ಳ್ಳು-ತ್ತೀರಿ
ಗಳಿ-ಸಿ-ಕೊ-ಳ್ಳು-ತ್ತೇನೆ
ಗಳಿ-ಸಿ-ಕೊ-ಳ್ಳುವ
ಗಳಿ-ಸಿತು
ಗಳಿ-ಸಿದ
ಗಳಿ-ಸಿ-ದರು
ಗಳಿ-ಸಿ-ದುವು
ಗಳಿ-ಸಿ-ದ್ದರು
ಗಳಿ-ಸಿ-ದ್ದ-ವಳು
ಗಳಿ-ಸಿದ್ದೇ
ಗಳಿ-ಸು-ವು-ದ-ಕ್ಕಿಂ-ತಲೂ
ಗಳಿ-ಸು-ವುದೇ
ಗಳು
ಗಳು-ಇ-ವು-ಗ-ಳೆ-ಲ್ಲ-ದರ
ಗಳು-ಸಂ-ಭಾ-ಷ-ಣೆ-ಗಳು
ಗಳೂ
ಗಳೆಂ-ದರೆ
ಗಳೆಂದು
ಗಳೆಲ್ಲ
ಗಳೆ-ಲ್ಲರೂ
ಗಳೆ-ಲ್ಲವೂ
ಗಳೊಂ-ದಿಗೆ
ಗಹನ
ಗಹ-ನ-ವಾದ
ಗಾಂಧಿ-ಕ-ಲ್ಕ-ತ್ತ-ದಲ್ಲಿ
ಗಾಂಧಿ-ಯ-ವ-ರು-ಆಗ
ಗಾಂಧಿ-ಯ-ವರೂ
ಗಾಂಧೀ-ಜಿ-ಯ-ವ-ರಿ-ಗಾದ
ಗಾಂಧೀ-ಜಿ-ಯ-ವರು
ಗಾಂಧೀಜೀ
ಗಾಂಭೀರ್ಯ
ಗಾಂಭೀ-ರ್ಯದ
ಗಾಂಭೀ-ರ್ಯ-ದಿಂದ
ಗಾಂಭೀ-ರ್ಯ-ವಿ-ರು-ತ್ತಿತ್ತು
ಗಾಂವನ್ನು
ಗಾಂವ್ನತ್ತ
ಗಾಗ-ಬ-ಹು-ದಾದ
ಗಾಗಿ
ಗಾಗಿದ್ದ
ಗಾಜಿ-ನಂತೆ
ಗಾಡಿ
ಗಾಡಿ-ಗ-ಟ್ಟಲೆ
ಗಾಡಿ-ಗಳಲ್ಲಿ
ಗಾಡಿಯ
ಗಾಡಿ-ಯನ್ನು
ಗಾಡಿ-ಯಲ್ಲಿ
ಗಾಡಿ-ಯಲ್ಲೇ
ಗಾಡಿ-ಯ-ವನು
ಗಾಡಿ-ಯಾ-ದ್ದ-ರಿಂದ
ಗಾಡಿ-ಯಿಂ-ದಿ-ಳಿದು
ಗಾಡಿ-ಯಿಂ-ದಿ-ಳಿ-ಯು-ತ್ತಿ-ದ್ದಂತೆ
ಗಾಡಿ-ಯೊಂ-ದರ
ಗಾಢ
ಗಾಢ-ನೀ-ರವ
ಗಾಢ-ಮೌನ
ಗಾಢ-ವಾಗಿ
ಗಾಢ-ವಾ-ದದ್ದು
ಗಾಢ-ವಾ-ಯಿತು
ಗಾತ್ರಕ್ಕೆ
ಗಾತ್ರದ
ಗಾದ
ಗಾದರೂ
ಗಾದ-ವ-ರಂತೆ
ಗಾದೆ
ಗಾದೆಯೇ
ಗಾನ
ಗಾಬರಿ
ಗಾಬ-ರಿ-ಕೊಂ-ಡರು
ಗಾಬ-ರಿ-ಗೊಂಡ
ಗಾಬ-ರಿ-ಗೊಂ-ಡಿ-ದ್ದಾನೆ
ಗಾಬ-ರಿಯ
ಗಾಯ
ಗಾಯಕಿ
ಗಾಯ-ಕಿ
ಗಾಯತ್ರೀ
ಗಾಯ-ವನ್ನು
ಗಾಯ-ವಾ-ಗಿದ್ದ
ಗಾರಂಭ
ಗಾರರು
ಗಾರಿಕೆ
ಗಾರಿ-ಕೆ-ಯಿಂದ
ಗಾರ್ಗಿ
ಗಾರ್ಟ್
ಗಾರ್ಡಿನ
ಗಾಳಿ
ಗಾಳಿ-ಭೂ-ಮಿ-ಹು-ಲ್ಲು-ಗಿ-ಡ-ಮ-ರ-ಪ-ರ್ವ-ತ-ಹಿ-ಮ-ಮಾ-ನ-ವಾ-ಕೃ-ತಿ-ಎ-ಲ್ಲವೂ
ಗಾಳಿ-ಗ-ಳಿಂ-ದಾಗಿ
ಗಾಳಿ-ಯಲ್ಲಿ
ಗಿಂತ
ಗಿಂತಲೂ
ಗಿಂದು
ಗಿಜಿ-ಗು-ಟ್ಟ-ಲಾ-ರಂ-ಭಿ-ಸಿತ್ತು
ಗಿಟಾ-ರಿನ
ಗಿಟ್ಟಿ-ರಲಿ
ಗಿಟ್ಟು-ಕೊಂ-ಡಿ-ದ್ದಾ-ರೆಯೇ
ಗಿಡ-ಗಳು
ಗಿಡಲು
ಗಿಡ್ಡಿ-ಸರ
ಗಿಡ್ಡಿ-ಸ-ರಿಗೆ
ಗಿಡ್ಡಿ-ಸ-ರೊಂ-ದಿಗೆ
ಗಿಡ್ಡಿಸ್
ಗಿತ್ತು
ಗಿತ್ತೆ
ಗಿದರು
ಗಿದೆ
ಗಿದ್ದ
ಗಿದ್ದು-ದ-ರಿಂದ
ಗಿದ್ದು-ಬಿ-ಟ್ಟರು
ಗಿದ್ದುವು
ಗಿದ್ದೆ
ಗಿನ್ನೂ
ಗಿರ-ಲಿಲ್ಲ
ಗಿರಲೂ
ಗಿರಿ
ಗಿರಿ-ತ-ರಂ-ಗ-ವೆ-ಬ್ಬಿ-ಸುತ್ತ
ಗಿರಿ-ಧಾಮ
ಗಿರಿ-ಧಾ-ಮ-ಗಳ
ಗಿರಿ-ಧಾ-ಮ-ದಲ್ಲಿ
ಗಿರಿ-ಧಾ-ಮ-ವೊಂ-ದಕ್ಕೆ
ಗಿರಿ-ಶೃಂ-ಗ-ವಾದ
ಗಿರೀಶ
ಗಿರೀ-ಶ-ಚಂ-ದ್ರ-ನಲ್ಲಿ
ಗಿರೀ-ಶನ
ಗಿರೀ-ಶ-ನಿಗೆ
ಗಿರೀ-ಶನು
ಗಿರೀ-ಶನೂ
ಗಿರೀಶ್
ಗಿರೀ-ಶ್ಚಂದ್ರ
ಗಿರೀ-ಶ್ಚಂ-ದ್ರ-ನನ್ನು
ಗಿರೀ-ಶ್ಚಂ-ದ್ರ-ನಿಗೆ
ಗಿರುವ
ಗಿರು-ವುದನ್ನು
ಗಿರು-ವು-ದರ
ಗಿರು-ವು-ದೇ-ಕೆಂ-ದರೆ
ಗಿರು-ವೆವೋ
ಗಿವೆ
ಗಿವೆಯೇ
ಗಿಸ-ದಿ-ರದು
ಗಿಸ-ಬೇಕು
ಗಿಸಿ-ದೆಯೆ
ಗೀಚಿ
ಗೀಚು-ತ್ತಿ-ದ್ದೇನೆ
ಗೀತ-ಗೋ-ವಿಂ-ದದ
ಗೀತಾ-ಧ್ಯ-ಯ-ನ-ದಲ್ಲಿ
ಗೀತಾ-ಪಾ-ರಾ-ಯಣ
ಗೀತಾ-ಶೈ-ಲಿ-ಯಲ್ಲಿ
ಗೀತೆ
ಗೀತೆ-ಗಳ
ಗೀತೆ-ಗಳನ್ನು
ಗೀತೆ-ಯ-ನ್ನೇನೋ
ಗೀತೆ-ಯಲ್ಲಿ
ಗೀಳಿ-ನ-ವರು
ಗುಂಡಿ-ನೇ-ಟಿಗೆ
ಗುಂಡು
ಗುಂಡು-ಗಳ
ಗುಂಡು-ಗಳನ್ನು
ಗುಂಡು-ಗ-ಳಿಗೆ
ಗುಂಡು-ಗ-ಳೆಲ್ಲ
ಗುಂಪನ್ನು
ಗುಂಪಾ-ಗಿದೆ
ಗುಂಪಿಗೆ
ಗುಂಪಿ-ನಲ್ಲಿ
ಗುಂಪಿ-ನ-ಲ್ಲಿದ್ದ
ಗುಂಪಿ-ನಲ್ಲೂ
ಗುಂಪಿ-ನ-ವರು
ಗುಂಪಿ-ನಿಂದ
ಗುಂಪಿ-ನೊಂ-ದಿಗೆ
ಗುಂಪು
ಗುಂಪು-ಗಳಲ್ಲಿ
ಗುಂಪು-ಗ-ಳಾಗಿ
ಗುಂಪು-ಗಾ-ರಿ-ಕೆ-ಮ-ತಾಂ-ಧ-ತೆ-ಗಳು
ಗುಂಪು-ಗುಂ-ಪಾಗಿ
ಗುಂಪು-ಗೂ-ಡಿಸಿ
ಗುಂಪೊಂದು
ಗುಂಯಿ-ಗು-ಡು-ತ್ತಿ-ರು-ತ್ತದೆ
ಗುಜ
ಗುಜ-ರಾತ್
ಗುಜು-ಗುಜು
ಗುಟುಕು
ಗುಟ್ಟನ್ನು
ಗುಟ್ಟಾಗಿ
ಗುಟ್ಟು
ಗುಡಿ
ಗುಡಿ-ಗಾ-ರ-ರಿಗೆ
ಗುಡಿ-ಗಾ-ರ-ರೆಲ್ಲ
ಗುಡಿ-ಗೋ-ಪು-ರ-ಗಳು
ಗುಡಿ-ಸಲ
ಗುಡಿ-ಸ-ಲಿ-ನ-ಲ್ಲಿದೆ
ಗುಡಿ-ಸ-ಲು-ಗಳಲ್ಲಿ
ಗುಡಿ-ಸ-ಲು-ಗ-ಳಲ್ಲೂ
ಗುಡಿ-ಸಿ-ಲಿ-ನಲ್ಲಿ
ಗುಡಿಸು
ಗುಡುಗಿ
ಗುಡು-ಗಿ-ದರು
ಗುಡು-ಗುಡಿ
ಗುಡು-ಗು-ತ್ತಾರೆ
ಗುಡು-ಗು-ತ್ತಿ-ದ್ದರು
ಗುಡ್
ಗುಡ್ಡ-ಗಾ-ಡಿನ
ಗುಡ್ಡ-ಗಾಡು
ಗುಡ್ಡದ
ಗುಡ್ಡ-ವನ್ನು
ಗುಡ್ಡ-ವ-ನ್ನೇ-ರಿ-ದರೆ
ಗುಡ್ವಿನ್
ಗುಡ್ವಿನ್ನ
ಗುಡ್ವಿ-ನ್ನನ
ಗುಡ್ವಿ-ನ್ನ-ನನ್ನು
ಗುಡ್ವಿ-ನ್ನ-ನಿಗೂ
ಗುಡ್ವಿ-ನ್ನನೂ
ಗುಣ
ಗುಣ
ಗುಣ-ಗಳ
ಗುಣ-ಗಳನ್ನೂ
ಗುಣ-ಗ-ಳನ್ನೇ
ಗುಣ-ಗಳಲ್ಲಿ
ಗುಣ-ಗ-ಳಿ-ದ್ದರೆ
ಗುಣ-ಗಳು
ಗುಣ-ಗ-ಳು-ಹೃ-ದ-ಯ-ದಲ್ಲಿ
ಗುಣ-ಗಾನ
ಗುಣ-ಗಾ-ನವೇ
ಗುಣ-ಮ-ಟ್ಟದ
ಗುಣ-ಮು-ಖ-ನಾ-ಗು-ವ-ವ-ರೆಗೂ
ಗುಣ-ಮು-ಖ-ನಾದ
ಗುಣ-ವಾಗಿ
ಗುಣ-ವಾ-ಗು-ತ್ತೀರಿ
ಗುಣ-ವಾ-ಗುವ
ಗುಣ-ವಾ-ದರೆ
ಗುಣವೇ
ಗುಣ-ಹೊಂದಿ
ಗುತ್ತದೆ
ಗುತ್ತಲೇ
ಗುತ್ತಿತ್ತು
ಗುನು-ಗಿ-ಕೊಳ್ಳು
ಗುನು-ಗಿ-ಕೊ-ಳ್ಳು-ತ್ತಿದ್ದ
ಗುನು-ಗು-ಟ್ಟಿ-ದರು
ಗುಪ್ತ
ಗುಪ್ತರ
ಗುಪ್ತ-ರನ್ನು
ಗುಪ್ತ-ರಿಗೆ
ಗುಪ್ತ-ವಾಗಿ
ಗುಪ್ತ-ವಾ-ಗಿ-ದ್ದುವು
ಗುಪ್ತ-ಶ್ರೀ-ರಾ-ಮ-ಕೃಷ್ಣ
ಗುಮಾಸ್ತೆ
ಗುರಿ
ಗುರಿ-ಮು-ಟ್ಟಲು
ಗುರಿ-ಮು-ಟ್ಟಿ-ದುವು
ಗುರಿ-ಮು-ಟ್ಟು-ವ-ವ-ರೆಗೂ
ಗುರಿ-ಯತ್ತ
ಗುರಿ-ಯನ್ನು
ಗುರಿಯಾ
ಗುರಿ-ಯಾ-ಗ-ಲಿತ್ತು
ಗುರಿ-ಯಾ-ಗ-ಲಿಲ್ಲ
ಗುರಿ-ಯಾಗಿ
ಗುರಿ-ಯಾ-ಗಿದ್ದ
ಗುರಿ-ಯಾ-ಗಿ-ದ್ದರು
ಗುರಿ-ಯಾ-ಗಿ-ದ್ದೀರಿ
ಗುರಿ-ಯಾ-ಗಿರು
ಗುರಿ-ಯಾ-ಗು-ತ್ತಾರೆ
ಗುರಿ-ಯಾ-ಗು-ತ್ತಿ-ದ್ದಳು
ಗುರಿ-ಯಾ-ಗು-ತ್ತಿರು
ಗುರಿ-ಯಾದ
ಗುರಿ-ಯಾ-ದರೆ
ಗುರಿ-ಯಾ-ದಾಗ
ಗುರಿ-ಯಾ-ಯಿತು
ಗುರಿ-ಯಿಟ್ಟು
ಗುರಿಯೂ
ಗುರಿಯೆ
ಗುರಿ-ಯೆಂ-ಬು-ದನ್ನು
ಗುರಿ-ಯೆ-ನ್ನು-ವು-ದಾ-ದರೆ
ಗುರಿ-ಯೇನು
ಗುರು
ಗುರು-ಶಿಷ್ಯ
ಗುರು-ಅ-ರ್ಥಾತ್
ಗುರು-ಕುಲ
ಗುರು-ಗಳಲ್ಲಿ
ಗುರು-ಗ-ಳಾದ
ಗುರು-ಗ-ಳಾ-ದ-ವರು
ಗುರು-ಗ-ಳಿಗೆ
ಗುರು-ಗಳು
ಗುರು-ಗ-ಳೆ-ನ್ನಿ-ಸಿ-ಕೊಂ-ಡ-ವರು
ಗುರು-ಗಿ-ರಿಯ
ಗುರು-ಗು-ಟ್ಟಿತು
ಗುರು-ತಿನ
ಗುರು-ತಿಸ
ಗುರು-ತಿ-ಸ-ದಿ-ರು-ವಂ-ತಿ-ರ-ಲಿಲ್ಲ
ಗುರು-ತಿ-ಸ-ಬ-ಹು-ದಾ-ಗಿತ್ತು
ಗುರು-ತಿ-ಸ-ಬ-ಹುದು
ಗುರು-ತಿ-ಸ-ಲಿಲ್ಲ
ಗುರು-ತಿ-ಸಲು
ಗುರು-ತಿಸಿ
ಗುರು-ತಿ-ಸಿ-ದರು
ಗುರು-ತಿ-ಸಿ-ದ-ವರೇ
ಗುರು-ತಿ-ಸಿ-ದ್ದ-ರಿಂದ
ಗುರು-ತಿ-ಸಿ-ದ್ದರು
ಗುರು-ತಿಸು
ಗುರು-ತಿ-ಸು-ತ್ತಿ-ದ್ದರು
ಗುರು-ತಿ-ಸುವ
ಗುರು-ತಿ-ಸು-ವಂ-ತಾ-ದದ್ದು
ಗುರು-ತಿ-ಸು-ವಲ್ಲಿ
ಗುರು-ತಿ-ಸು-ವು-ದಾ-ದರೆ
ಗುರುತು
ಗುರು-ತ್ವ-ಕೇಂದ್ರ
ಗುರು-ದಾಸ್
ಗುರು-ದೇವ
ಗುರು-ದೇ-ವನ
ಗುರು-ದೇ-ವ-ನನ್ನು
ಗುರು-ದೇ-ವ-ನಾದ
ಗುರು-ದೇ-ವ-ನಿಗೆ
ಗುರು-ದೇ-ವರ
ಗುರು-ಪ-ತ್ನಿ-ಯಲ್ಲ
ಗುರು-ಪು-ತ್ರೇಷು
ಗುರು-ಭಕ್ತಿ
ಗುರು-ಭ-ಕ್ತಿಯ
ಗುರು-ಭ-ಕ್ತಿ-ಯನ್ನು
ಗುರು-ಭಾಯಿ
ಗುರು-ಭಾ-ಯಿ-ಗಳ
ಗುರು-ಭಾ-ಯಿ-ಗ-ಳಾದ
ಗುರು-ಭಾ-ಯಿ-ಗ-ಳಿಗೂ
ಗುರು-ಭಾ-ಯಿ-ಗ-ಳಿಗೆ
ಗುರು-ಭಾ-ಯಿ-ಗ-ಳಿ-ಗೆಲ್ಲ
ಗುರು-ಭಾ-ಯಿ-ಗಳು
ಗುರು-ಭಾ-ಯಿ-ಗಳೂ
ಗುರು-ಭಾ-ಯಿ-ಗ-ಳೊಂ-ದಿ-ಗಿನ
ಗುರು-ಭಾ-ಯಿ-ಗ-ಳೊಂ-ದಿಗೆ
ಗುರು-ಭಾ-ಯಿ-ಗ-ಳೊ-ಬ್ಬ-ರಿಂದ
ಗುರು-ಭಾ-ಯಿಯ
ಗುರು-ಭಾ-ಯಿಯೂ
ಗುರು-ಮಹಾ
ಗುರು-ಮ-ಹಾ-ರಾ-ಜರ
ಗುರು-ಮ-ಹಾ-ರಾ-ಜರು
ಗುರು-ಮ-ಹಾ-ರಾ-ಜರೂ
ಗುರು-ಮ-ಹಾ-ರಾಜ್
ಗುರು-ವತ್
ಗುರು-ವನ್ನು
ಗುರು-ವಾಗಿ
ಗುರು-ವಾ-ಗಿ-ದ್ದಾ-ರೆಯೋ
ಗುರು-ವಾದ
ಗುರು-ವಾರ
ಗುರು-ವಿನ
ಗುರು-ವಿ-ನಂತೆ
ಗುರು-ವಿ-ನಂ-ತೆಯೇ
ಗುರು-ವಿ-ನ-ಡಿ-ಯಲ್ಲಿ
ಗುರು-ವಿ-ನ-ಲ್ಲಿ-ರ-ಬೇ-ಕಾದ
ಗುರು-ವಿ-ನೊಂ-ದಿಗೆ
ಗುರುವು
ಗುರು-ವೆಂದು
ಗುರು-ಶಿ-ಷ್ಯ-ರಿ-ಬ್ಬರೂ
ಗುರು-ಸ್ಥಾ-ನಕ್ಕೆ
ಗುಲಾ-ಬಿ-ಯೊಂ-ದನ್ನು
ಗುಲಾ-ಮ-ಗಿರಿ
ಗುಲಾ-ಮ-ಗಿ-ರಿಯ
ಗುಲಾ-ಮ-ಗಿ-ರಿ-ಯನ್ನು
ಗುಲಾ-ಮ-ಗಿ-ರಿ-ಯಲ್ಲಿ
ಗುಲಾ-ಮ-ಗಿ-ರಿ-ಯ-ಲ್ಲಿದ್ದ
ಗುಲಾ-ಮ-ನಲ್ಲ
ಗುಲಾ-ಮ-ಬು-ದ್ಧಿಯ
ಗುಲಾ-ಮ-ರಾ-ಗಿ-ರು-ವುದು
ಗುಲ್ಮಾ-ರ್ಗಿಗೆ
ಗುಳ್ಳೆ-ಯಾ-ಗಿ-ರ-ಬ-ಹುದು
ಗುವಂ-ತಿ-ರ-ಲಿಲ್ಲ
ಗುವ-ವ-ರೆಗೂ
ಗುವಾ-ಗಲೂ
ಗುಹ
ಗುಹಾಂ-ತ-ರ-ಗ-ಳ-ಲ್ಲಿ-ರುವ
ಗುಹೆ
ಗುಹೆ-ಗಳಲ್ಲಿ
ಗುಹೆಯ
ಗುಹೆ-ಯಂ-ತಹ
ಗುಹೆ-ಯನ್ನು
ಗುಹೆ-ಯನ್ನೂ
ಗುಹೆ-ಯಲ್ಲಿ
ಗುಹೆ-ಯಿಂದ
ಗುಹೆ-ಯಿ-ರು-ವುದು
ಗುಹೆ-ಯೊ-ಳಗೆ
ಗೂಂಡಾ
ಗೂಡನ್ನು
ಗೂಡಿತು
ಗೂಡಿ-ಸಿ-ಕೊ-ಳ್ಳು-ವುದು
ಗೂಡೇ
ಗೂಢ-ಶಾ-ಸ್ತ್ರ-ದಲ್ಲಿ
ಗೃಹ-ಕೃತ್ಯ
ಗೃಹ-ಕೃ-ತ್ಯ-ದಲ್ಲಿ
ಗೃಹಕ್ಕೆ
ಗೃಹ-ಪ್ರ-ವೇ-ಶ-ವನ್ನೂ
ಗೃಹ-ಮು-ಚ್ಯತೇ
ಗೃಹಸ್ಥ
ಗೃಹ-ಸ್ಥ-ಜೀ-ವ-ನ-ವನ್ನು
ಗೃಹ-ಸ್ಥ-ಧ-ರ್ಮದ
ಗೃಹ-ಸ್ಥ-ಧ-ರ್ಮ-ವನ್ನು
ಗೃಹ-ಸ್ಥನ
ಗೃಹ-ಸ್ಥ-ನಾ-ಗಿ-ದ್ದರೂ
ಗೃಹ-ಸ್ಥ-ಭ-ಕ್ತರು
ಗೃಹ-ಸ್ಥ-ಭ-ಕ್ತೆ-ಯೊ-ಬ್ಬ-ಳಿಗೆ
ಗೃಹ-ಸ್ಥರ
ಗೃಹ-ಸ್ಥ-ರನ್ನು
ಗೃಹ-ಸ್ಥ-ರಾದ
ಗೃಹ-ಸ್ಥ-ರಾ-ದರೆ
ಗೃಹ-ಸ್ಥ-ರಿಗೂ
ಗೃಹ-ಸ್ಥ-ರಿಗೆ
ಗೃಹ-ಸ್ಥ-ರಿಗೇ
ಗೃಹ-ಸ್ಥರು
ಗೃಹ-ಸ್ಥರೂ
ಗೃಹ-ಸ್ಥ-ಶಿ-ಷ್ಯ-ರಲ್ಲಿ
ಗೃಹಿ-ಣಿ-ಯರ
ಗೃಹಿಣೀ
ಗೃಹೀ
ಗೃಹೀ-ಭ-ಕ್ತ-ನಾದ
ಗೃಹೀ-ಭ-ಕ್ತರ
ಗೃಹೀ-ಭ-ಕ್ತ-ರಲ್ಲಿ
ಗೃಹೀ-ಭ-ಕ್ತ-ರ-ಲ್ಲೊ-ಬ್ಬ-ನಾದ
ಗೃಹೀ-ಭ-ಕ್ತ-ರಾದ
ಗೃಹೀ-ಭ-ಕ್ತರು
ಗೃಹೀ-ಭ-ಕ್ತರೂ
ಗೆ
ಗೆಂದ-ರಾ-ವಾಗ
ಗೆದು
ಗೆದ್ದು
ಗೆದ್ದು-ಕೊಂ-ಡಿ-ದ್ದರು
ಗೆದ್ದು-ಕೊ-ಳ್ಳಲು
ಗೆರಾ-ಲ್ಡ್
ಗೆರೆ-ಗಳು
ಗೆರೆ-ಯ-ನ್ನ-ವರು
ಗೆಲು-ವೆ-ನ್ನದೆ
ಗೆಲ್ಲ-ಬ-ಹುದು
ಗೆಲ್ಲ-ಬೇ-ಕಾ-ದರೆ
ಗೆಲ್ಲ-ಬೇಕು
ಗೆಲ್ಲು-ವನು
ಗೆಲ್ಲು-ವುದು
ಗೆಲ್ಲು-ವು-ದೆಂ-ದರೆ
ಗೆಲ್ಲೋ
ಗೆಳೆ-ದರು
ಗೆಳೆ-ಯ-ರಾಗಿ
ಗೇಟ್
ಗೇನೂ
ಗೇರಿದ್ದೇ
ಗೇರಿ-ರು-ವುದನ್ನು
ಗೇಲಿ
ಗೈಯು-ವ-ವ-ನಿ-ದ್ದೇನೆ
ಗೈರಿಕ
ಗೈರು-ಹಾ-ಜ-ರಾ-ಗಿ-ರ-ಬೇಕೆ
ಗೈರು-ಹಾ-ಜ-ರಿ-ಯನ್ನು
ಗೊಂಡ
ಗೊಂಡರು
ಗೊಂಡ-ವರ
ಗೊಂಡಿತು
ಗೊಂಡಿತ್ತು
ಗೊಂಡಿ-ರ-ಬೇ-ಕು-ಇ-ವು-ಗ-ಳಿ-ಲ್ಲದೆ
ಗೊಂಡಿ-ರುವ
ಗೊಂದ-ಲಕ್ಕೆ
ಗೊಂದು
ಗೊಜ್ಜನ್ನು
ಗೊಜ್ಜು
ಗೊಡ-ವೆಯೇ
ಗೊಡ್ಡು
ಗೊಡ್ಡು-ಕಂತೆ
ಗೊಣ-ಗು-ಟ್ಟದೆ
ಗೊಣ-ಗುತ್ತ
ಗೊತ್ತಾ
ಗೊತ್ತಾ-ಗ-ದಂತೆ
ಗೊತ್ತಾ-ಗ-ದಿ-ರ-ಲಿಲ್ಲ
ಗೊತ್ತಾ-ಗ-ಲಿಲ್ಲ
ಗೊತ್ತಾ-ಗಲೇ
ಗೊತ್ತಾ-ಗಿದೆ
ಗೊತ್ತಾ-ಗಿ-ದೆಯೆ
ಗೊತ್ತಾಗು
ಗೊತ್ತಾ-ಗು-ತ್ತದೆ
ಗೊತ್ತಾ-ದೊ-ಡ-ನೆಯೇ
ಗೊತ್ತಾ-ಯಿತು
ಗೊತ್ತಾ-ಯಿತೆ
ಗೊತ್ತಾ-ಯಿ-ತೆಂ-ಬುದೇ
ಗೊತ್ತಾಯ್ತೆ
ಗೊತ್ತಿ
ಗೊತ್ತಿತ್ತು
ಗೊತ್ತಿ-ತ್ತು-ಸ್ವಾ-ಮೀಜಿ
ಗೊತ್ತಿದೆ
ಗೊತ್ತಿ-ದ್ದರೆ
ಗೊತ್ತಿ-ದ್ದುದೇ
ಗೊತ್ತಿ-ರ-ಬೇ-ಕ-ಲ್ಲವೇ
ಗೊತ್ತಿ-ರ-ಲಿಲ್ಲ
ಗೊತ್ತಿ-ರುವ
ಗೊತ್ತಿ-ರು-ವಂ-ತೆ-ಹಾ-ಗೇ-ನಾ-ದರೂ
ಗೊತ್ತಿ-ರು-ವ-ಬೇ-ರೊಂದು
ಗೊತ್ತಿಲ್ಲ
ಗೊತ್ತಿ-ಲ್ಲದ
ಗೊತ್ತಿ-ಲ್ಲಪ್ಪ
ಗೊತ್ತಿ-ಲ್ಲವೆ
ಗೊತ್ತು
ಗೊತ್ತು-ಪ-ಡಿ-ಸ-ಬೇ-ಕಾ-ಯಿತು
ಗೊತ್ತು-ಪ-ಡಿ-ಸ-ಲಾ-ಯಿತು
ಗೊತ್ತು-ಪ-ಡಿ-ಸು-ವುದು
ಗೊತ್ತು-ಮಾ-ಡಿ-ಕೊಂಡು
ಗೊತ್ತು-ವಳಿ
ಗೊತ್ತೆ
ಗೊತ್ತೇನು
ಗೊತ್ತೋ
ಗೊರ-ಗೊರ
ಗೊರೆ-ಯನ್ನು
ಗೊಳಿ-ಸ-ಬೇಕು
ಗೊಳಿ-ಸ-ಲಾ-ಗಿತ್ತು
ಗೊಳಿ-ಸಲು
ಗೊಳಿಸಿ
ಗೊಳಿ-ಸಿ-ಕೊ-ಳ್ಳುವ
ಗೊಳಿ-ಸಿದ
ಗೊಳಿ-ಸಿ-ದ-ರ-ಲ್ಲದೆ
ಗೊಳಿ-ಸಿ-ದರೆ
ಗೊಳಿ-ಸಿ-ದರೋ
ಗೊಳಿಸು
ಗೊಳಿ-ಸು-ತ್ತಿ-ದ್ದುವು
ಗೊಳಿ-ಸುವ
ಗೊಳ್ಳಲು
ಗೊಳ್ಳಿ
ಗೊಳ್ಳು-ತ್ತಾನೆ
ಗೋ
ಗೋಗ-ರೆದ
ಗೋಚ-ರ-ವಾ-ಗಿ-ದ್ದಿ-ರ-ಬೇಕು
ಗೋಚ-ರ-ವಾ-ಗು-ತ್ತದೆ
ಗೋಚ-ರ-ವಾ-ಗು-ತ್ತಿತ್ತು
ಗೋಚ-ರ-ವಾ-ಗು-ವುದು
ಗೋಚ-ರ-ವಾ-ದುವು
ಗೋಚ-ರ-ವಾ-ಯಿತು
ಗೋಚ-ರಿ-ಸಿ-ದರು
ಗೋಚ-ರಿ-ಸಿ-ದ್ದುವು
ಗೋಚ-ರಿ-ಸು-ತ್ತಿತ್ತು
ಗೋಜಿಗೆ
ಗೋಡೆ
ಗೋಡೆ-ಗಳನ್ನೂ
ಗೋಡೆಗೆ
ಗೋಡೆಯ
ಗೋಡೆ-ಯಂತೆ
ಗೋಡೆ-ಯಾ-ಗಿ-ರು-ವು-ದ-ಕ್ಕಿಂತ
ಗೋಡೆಯೇ
ಗೋಪಾ-ಲ-ಲಾಲ್
ಗೋಪಾ-ಲೇರ್
ಗೋಪಾಲ್
ಗೋಪುರ
ಗೋಪು-ರ-ಗ-ಳ-ನ್ನೊ-ಳ-ಗೊಂ-ಡಿದೆ
ಗೋಪು-ರ-ಗಳು
ಗೋಪು-ರದ
ಗೋಪು-ರ-ವನ್ನು
ಗೋಪ್ಯ-ವಾಗಿ
ಗೋಮಾತೆ
ಗೋಮಾ-ತೆ-ಯರ
ಗೋಮಾ-ತೆ-ಯ-ರನ್ನು
ಗೋಮಾ-ಳ-ಗಳನ್ನು
ಗೋಯ-ಲುಂಡೊ
ಗೋಯ-ಲ್ಪಾ-ರಾ-ದಲ್ಲಿ
ಗೋರ-ಕ್ಷಣಾ
ಗೋರಿ-ಗ-ಳಿವೆ
ಗೋರಿ-ಗಳೂ
ಗೋರಿ-ಯನ್ನು
ಗೋಲ್ಕೊಂಡ
ಗೋಲ್ಡನ್
ಗೋಳ-ಬಾ-ಳಲಿ
ಗೋಳಾ-ಡಿದ್ದಾ
ಗೋಳಿ-ಡು-ತ್ತಿ-ರುವ
ಗೋವಿಂದ
ಗೋವಿಂ-ದ-ಚೆಟ್ಟಿ
ಗೋವಿಂ-ದ-ನನ್ನು
ಗೋವಿಂ-ದ-ಸಾ-ಹರೂ
ಗೋವಿಂ-ದಾ-ನಂ-ದಜಿ
ಗೋವಿಂ-ದಾ-ನಂ-ದರು
ಗೋವು
ಗೋವು-ಗಳನ್ನೆಲ್ಲ
ಗೋಸ್ವಾ-ಮಿ-ಯ-ವರ
ಗೋಸ್ವಾ-ಮಿ-ಯ-ವರು
ಗೌಜು
ಗೌಜು-ಗ-ದ್ದಲ
ಗೌಣ-ವಾ-ದು-ದೆಂದೂ
ಗೌರವ
ಗೌರ-ವ-ಪೂಜ್ಯ
ಗೌರ-ವ-ಪ್ರೀ-ತಿ-ಗಳ
ಗೌರ-ವ-ಮ-ನ್ನಣೆ
ಗೌರ-ವ-ಸ-ನ್ಮಾ-ನದ
ಗೌರ-ವಕ್ಕೆ
ಗೌರ-ವ-ಗಳನ್ನು
ಗೌರ-ವ-ಗಳನ್ನೂ
ಗೌರ-ವ-ಗ-ಳೊಂ-ದಿಗೆ
ಗೌರ-ವದ
ಗೌರ-ವ-ಪೂ-ರ್ಣ-ವಾ-ಗಿಯೇ
ಗೌರ-ವ-ಭಾ-ವ-ವನ್ನು
ಗೌರ-ವ-ರ-ಕ್ಷೆ-ಯೊ-ಡನೆ
ಗೌರ-ವ-ವನ್ನು
ಗೌರ-ವ-ವನ್ನೂ
ಗೌರ-ವ-ವಾ-ಗಲಿ
ಗೌರ-ವ-ವಿತ್ತು
ಗೌರ-ವ-ವಿಲ್ಲ
ಗೌರ-ವ-ಸ್ಥಾ-ನ-ದ-ಲ್ಲಿ-ಟ್ಟಿದ್ದ
ಗೌರ-ವಾ-ದರ
ಗೌರ-ವಾ-ದ-ರ-ಗಳನ್ನು
ಗೌರ-ವಾ-ದ-ರ-ಗಳಿಂದ
ಗೌರ-ವಾ-ದ-ರ-ದಿಂದ
ಗೌರ-ವಾ-ನ್ವಿತ
ಗೌರ-ವಾ-ರ್ಥ-ವಾಗಿ
ಗೌರ-ವಿಸ
ಗೌರ-ವಿ-ಸ-ಬ-ಲ್ಲದು
ಗೌರ-ವಿ-ಸ-ಬೇಕು
ಗೌರ-ವಿ-ಸಲೇ
ಗೌರ-ವಿಸಿ
ಗೌರ-ವಿ-ಸಿತು
ಗೌರ-ವಿ-ಸಿ-ದರು
ಗೌರ-ವಿಸು
ಗೌರ-ವಿ-ಸು-ತ್ತಾರೆ
ಗೌರ-ವಿ-ಸು-ತ್ತಿ-ದ್ದಾರೆ
ಗೌರ-ವಿ-ಸು-ತ್ತೇನೆ
ಗೌರ-ವಿ-ಸು-ವ-ವರು
ಗೌರ-ವಿ-ಸು-ವುದು
ಗೌರೀ
ಗೌಹಾತಿ
ಗೌಹಾ-ತಿಯ
ಗೌಹಾ-ತಿ-ಯಲ್ಲಿ
ಗೌಹಾ-ತಿಯು
ಗ್ಯಾರಂಟಿ
ಗ್ರಂಥ
ಗ್ರಂಥ-ಗ್ರಂ-ಥ-ಕ-ರ್ತರ
ಗ್ರಂಥ-ಗಳ
ಗ್ರಂಥ-ಗ-ಳಂತೂ
ಗ್ರಂಥ-ಗಳನ್ನು
ಗ್ರಂಥ-ಗಳಲ್ಲಿ
ಗ್ರಂಥ-ಗಳಿಂದ
ಗ್ರಂಥ-ಗ-ಳಿಲ್ಲ
ಗ್ರಂಥ-ಗಳು
ಗ್ರಂಥ-ಗ-ಳೆಂ-ದರೆ
ಗ್ರಂಥ-ಗ-ಳೊ-ಳ-ಗಿ-ನಿಂದ
ಗ್ರಂಥದ
ಗ್ರಂಥ-ದಲ್ಲಿ
ಗ್ರಂಥ-ದಿಂದ
ಗ್ರಂಥ-ಮಾ-ಲೆ-ಯನ್ನು
ಗ್ರಂಥ-ರ-ಚ-ನೆ-ಯಲ್ಲಿ
ಗ್ರಂಥ-ವನ್ನು
ಗ್ರಂಥ-ವನ್ನೂ
ಗ್ರಂಥಾ-ಲ-ಯ-ದಲ್ಲಿ
ಗ್ರಹ-ಗಳನ್ನೂ
ಗ್ರಹಣ
ಗ್ರಹ-ಣ-ಕಾ-ಲ-ದಲ್ಲಿ
ಗ್ರಹ-ಣ-ವೆಂದರೆ
ಗ್ರಹ-ಣ-ಶ-ಕ್ತಿ-ಯನ್ನು
ಗ್ರಹಿ
ಗ್ರಹಿ-ಸದೆ
ಗ್ರಹಿ-ಸಲು
ಗ್ರಹಿ-ಸಿದ
ಗ್ರಹಿ-ಸುತ್ತ
ಗ್ರಹಿ-ಸು-ವುದು
ಗ್ರಾಫರ್
ಗ್ರಾಮಕ್ಕೆ
ಗ್ರಾಮ-ದಿಂದ
ಗ್ರಾಮಾಂ-ತ-ರ-ಗಳಲ್ಲಿ
ಗ್ರಾಹ್ಯ-ವಾ-ಗು-ತ್ತ-ವೆಂದು
ಗ್ರೀಕರ
ಗ್ರೀಕರು
ಗ್ರೀಕ್
ಗ್ರೀನ್
ಗ್ರೀನ್ಸ್ಟೈ-ಡಲ್
ಗ್ರೀಸಿನ
ಗ್ರೀಸಿ-ನಲ್ಲಿ
ಗ್ರೀಸ್
ಗ್ರೇಸರ
ಗ್ರ್ಯಾಂಡ್
ಗ್ಲಾಸ್ಗೋಗೆ
ಗ್ಲೇಷಿ-ಯರ್ಗೆ
ಗ್ವಾಲಿ-ಯ-ರಿನ
ಗ್ವಾಲಿ-ಯರ್
ಘಂಟಾ-ಘೋ-ಷ-ವಾಗಿ
ಘಂಟಾ-ನಾದ
ಘಂಟೆಗೆ
ಘಟ
ಘಟ-ನಾ-ವ-ಳಿ-ಗಳನ್ನು
ಘಟನೆ
ಘಟ-ನೆ-ಗಳ
ಘಟ-ನೆ-ಗಳಿಂದ
ಘಟ-ನೆ-ಗ-ಳಿಗೆ
ಘಟ-ನೆ-ಗಳು
ಘಟ-ನೆ-ಗಳೂ
ಘಟ-ನೆಗೆ
ಘಟ-ನೆಯ
ಘಟ-ನೆ-ಯನ್ನು
ಘಟ-ನೆ-ಯಲ್ಲೂ
ಘಟ-ನೆ-ಯಾಗಿ
ಘಟ-ನೆ-ಯಿಂದ
ಘಟ-ನೆ-ಯಿಂ-ದಾಗಿ
ಘಟ-ನೆಯು
ಘಟ-ನೆಯೂ
ಘಟ-ನೆ-ಯೆಂ-ದರೆ
ಘಟ-ನೆಯೇ
ಘಟ-ನೆ-ಯೊಂ-ದನ್ನು
ಘಟ-ನೆ-ಯೊಂದು
ಘಟಿ-ಸ-ಬ-ಹುದೇ
ಘಟಿಸಿ
ಘಟಿ-ಸಿ-ರ-ದಿ-ದ್ದರೆ
ಘಟ್ಟದ
ಘನ
ಘನ-ಗಂ-ಭೀ-ರ-ವಾಗಿ
ಘನ-ತರ
ಘನ-ತ-ರ-ವಾದ
ಘನತೆ
ಘನ-ತೆಗೆ
ಘನ-ತೆ-ಯನ್ನು
ಘನ-ವಾದ
ಘನ-ವಾ-ದ-ದ್ದೇ-ನನ್ನೋ
ಘನ-ಸ-ತ್ಯ-ಗಳನ್ನು
ಘಮ-ಘ-ಮಿ-ಸು-ತ್ತಿ-ರು-ವಾ-ಗಲೇ
ಘಮಲು
ಘರ್ಷ-ಣೆ-ತೊ-ಳ-ಲಾ-ಟ-ಗಳ
ಘರ್ಷ-ಣೆಗೆ
ಘಳಿ-ಗೆಗೆ
ಘಳಿ-ಗೆ-ಯಲ್ಲಿ
ಘಳಿ-ಗೆ-ಯ-ವ-ರೆಗೂ
ಘಳಿ-ಗೆ-ಯಿಂದ
ಘಾಟಿ-ನಲ್ಲಿ
ಘಾಟ್ನತ್ತ
ಘಾತಕ
ಘಾಸಿ-ಗೊಂ-ಡಿತು
ಘಾಸಿ-ಯಾ-ಗಿ-ದ್ದಂ-ತಿತ್ತು
ಘಾಸಿ-ಯುಂ-ಟು-ಮಾಡು
ಘೋರ
ಘೋರಾಂ-ಧ-ಕಾ-ರ-ದಲ್ಲಿ
ಘೋಷ
ಘೋಷಣೆ
ಘೋಷ-ಣೆ-ಯಾ-ಗಿತ್ತು
ಘೋಷ-ಣೆ-ಯೆಂ-ದರೆ
ಘೋಷದ
ಘೋಷನ
ಘೋಷ-ವನ್ನು
ಘೋಷಾ-ಲ-ರಿ-ಬ್ಬರೂ
ಘೋಷಾ-ಲಳು
ಘೋಷಿ-ಸ-ಲಾ-ಯಿತು
ಘೋಷಿ-ಸ-ಲಿಲ್ಲ
ಘೋಷಿಸಿ
ಘೋಷಿ-ಸಿ-ಕೊಳ್ಳ
ಘೋಷಿ-ಸಿತು
ಘೋಷಿ-ಸಿದ
ಘೋಷಿ-ಸಿ-ದರು
ಘೋಷಿ-ಸಿ-ದಳು
ಘೋಷಿ-ಸು-ತ್ತಾರೆ
ಘೋಷಿ-ಸು-ತ್ತಿ-ದ್ದ-ರಂತೆ
ಘೋಷಿ-ಸು-ತ್ತಿ-ದ್ದರು
ಘೋಷಿ-ಸುವ
ಘೋಷಿ-ಸು-ವುದು
ಘೋಷ್
ಚ
ಚಂಚ-ಲ-ಗೊ-ಳಿ-ಸು-ವುದು
ಚಂಡ-ಮಾ-ರುತ
ಚಂಡ-ಮಾ-ರು-ತದ
ಚಂಡ-ಮಾ-ರು-ತ-ದಂತೆ
ಚಂಡ-ಮಾ-ರು-ತ-ದಿಂ-ದಾಗಿ
ಚಂಡ-ಮಾ-ರು-ತವೇ
ಚಂಡಾ-ಲ-ನಿಗೆ
ಚಂಡಾ-ಲ-ರನ್ನು
ಚಂಡಾ-ಲ-ರನ್ನೂ
ಚಂಡಾ-ಲ-ರ-ವರೆ-ಗಿನ
ಚಂಡಿಯೇ
ಚಂಡೀ
ಚಂದ-ನ-ವನ್ನು
ಚಂದ-ನ್ವಾರ್
ಚಂದಾ-ದಾ-ರರ
ಚಂದಾ-ದಾ-ರ-ರನ್ನು
ಚಂದಾ-ದಾ-ರ-ರಿ-ದ್ದರು
ಚಂದ್ರ
ಚಂದ್ರ-ಕಾಂತ
ಚಂದ್ರದ
ಚಂದ್ರನ
ಚಂದ್ರ-ನಾಥ
ಚಂದ್ರ-ನಾ-ಥ-ಕಾ-ಮಾ-ಖ್ಯ-ಗಳ
ಚಂದ್ರ-ನಾ-ಥಕ್ಕೆ
ಚಂದ್ರ-ನಾ-ಥ-ದಿಂದ
ಚಂದ್ರನು
ಚಂದ್ರಮ
ಚಂದ್ರ-ಮಾ-ಧ-ವ-ಘೋಷ್
ಚಂದ್ರ-ಶೇ-ಖರ
ಚಂಪಾ-ವತ್
ಚಂಪಾ-ವ-ತ್ನಿಂದ
ಚಕಿ-ತ-ರಾ-ದರೂ
ಚಕಿ-ತ-ಳಾದ
ಚಕ್ರ-ಗಳಲ್ಲಿ
ಚಕ್ರದ
ಚಕ್ರ-ದಲ್ಲಿ
ಚಕ್ರ-ವರ್ತಿ
ಚಕ್ರ-ವ-ರ್ತಿಗೆ
ಚಕ್ರ-ವ-ರ್ತಿಯ
ಚಕ್ರ-ವ-ರ್ತಿ-ಯಂ-ತಹ
ಚಕ್ರ-ವ-ರ್ತಿ-ಯನ್ನು
ಚಜ-ಗದ
ಚಟರ್ಜಿ
ಚಟ-ರ್ಜಿಯ
ಚಟ-ರ್ಜಿಯು
ಚಟಾಕಿ
ಚಟಾ-ಕಿ-ಗಳನ್ನು
ಚಟು-ವ-ಟಿಕೆ
ಚಟು-ವ-ಟಿ-ಕೆ-ಗಳ
ಚಟು-ವ-ಟಿ-ಕೆ-ಗಳನ್ನು
ಚಟು-ವ-ಟಿ-ಕೆ-ಗಳನ್ನೆಲ್ಲ
ಚಟು-ವ-ಟಿ-ಕೆ-ಗಳಿಂದ
ಚಟು-ವ-ಟಿ-ಕೆಯ
ಚಟು-ವ-ಟಿ-ಕೆ-ಯಲ್ಲಿ
ಚಟು-ವ-ಟಿ-ಕೆ-ಯ-ಲ್ಲಿಯೂ
ಚಟು-ವ-ಟಿ-ಕೆ-ಯಿಂದ
ಚಟು-ವ-ಟಿ-ಕೆ-ಯಿಂ-ದಿ-ರು-ವಂತೆ
ಚಟು-ವ-ಟಿ-ಕೆಯೆ
ಚಡ-ಪ-ಡಿ-ಕೆ-ಯನ್ನು
ಚಡ-ಪ-ಡಿ-ಸ-ಲಾ-ರಂ-ಭಿ-ಸಿ-ದರು
ಚಡ-ಪ-ಡಿ-ಸಿ-ದರು
ಚಡ-ಪ-ಡಿ-ಸುತ್ತ
ಚಡ-ಪ-ಡಿ-ಸು-ತ್ತಿತ್ತು
ಚಡ-ಪ-ಡಿ-ಸು-ತ್ತಿ-ರು-ವಂತೆ
ಚತು-ರರೂ
ಚತು-ರೋ-ಪಾ-ಯ-ಗಳೂ
ಚತು-ಷ್ಪಾ-ದಿ-ಗಳೇ
ಚದು-ರ-ಬೇ-ಕೆಂಬ
ಚದು-ರಿರ
ಚದು-ರಿ-ಹೋ-ಯಿತು
ಚನಾ
ಚನ್ನ-ಪುರಿ
ಚನ್ನ-ಪು-ರಿಯ
ಚಪಾತಿ
ಚಪಾ-ತಿ-ಎಂದು
ಚಪಾ-ತಿ-ಗಳನ್ನು
ಚಪಾ-ತಿ-ಗಳು
ಚಪಾ-ತಿಗೆ
ಚಪ್ಪರ
ಚಪ್ಪ-ರಕ್ಕೆ
ಚಪ್ಪ-ರದ
ಚಪ್ಪ-ರ-ದಲ್ಲಿ
ಚಪ್ಪ-ರ-ದೊ-ಳಗೆ
ಚಪ್ಪ-ರ-ವನ್ನು
ಚಪ್ಪ-ರ-ವೊಂ-ದನ್ನು
ಚಪ್ಪಲಿ
ಚಪ್ಪ-ಲಿ-ಗಳನ್ನು
ಚಪ್ಪ-ಲಿ-ಯನ್ನು
ಚಪ್ಪಾಳೆ
ಚಪ್ಪಾ-ಳೆ-ಗಳ
ಚಬು-ಕಿ-ನಿಂದ
ಚಮ್ಮಾ
ಚಯಾ-ದಿ-ಗಳನ್ನೂ
ಚರಂ-ಡಿ-ಗಳನ್ನು
ಚರಾ-ಚ-ರ-ಗಳ
ಚರಿತ
ಚರಿತ್ರೆ
ಚರಿ-ತ್ರೆ-ಸಂ-ಪ್ರ-ದಾ-ಯ-ಗ-ಳು-ಸ್ವ-ಭಾ-ವ-ಗಳು
ಚರಿ-ತ್ರೆ-ಗಳು
ಚರಿ-ತ್ರೆ-ಯನ್ನು
ಚರಿ-ತ್ರೆ-ಯನ್ನೂ
ಚರಿ-ತ್ರೆ-ಯಲ್ಲಿ
ಚರಿ-ತ್ರೆ-ಯಲ್ಲೇ
ಚರ್ಚಾ-ಗೋ-ಷ್ಠಿಗೋ
ಚರ್ಚಿನ
ಚರ್ಚಿ-ನಲ್ಲಿ
ಚರ್ಚಿ-ಸಲು
ಚರ್ಚಿಸಿ
ಚರ್ಚಿ-ಸಿದ
ಚರ್ಚಿ-ಸಿ-ದರು
ಚರ್ಚಿ-ಸಿ-ದಳು
ಚರ್ಚಿ-ಸುತ್ತ
ಚರ್ಚಿ-ಸು-ತ್ತಿ-ದ್ದರು
ಚರ್ಚಿ-ಸು-ತ್ತಿ-ದ್ದು-ದುಂಟು
ಚರ್ಚಿ-ಸು-ವಂತೆ
ಚರ್ಚೀ
ಚರ್ಚು
ಚರ್ಚು-ಗಳ
ಚರ್ಚು-ಗಳನ್ನು
ಚರ್ಚು-ಗಳಿಂದ
ಚರ್ಚು-ಗಳೂ
ಚರ್ಚು-ಜೀ-ವ-ನ-ವೊಂದು
ಚರ್ಚೂ
ಚರ್ಚೆ
ಚರ್ಚೆ-ಗಳನ್ನು
ಚರ್ಚೆ-ಗಳಲ್ಲಿ
ಚರ್ಚೆ-ಗಳು
ಚರ್ಚೆ-ಮಾ-ಡುತ್ತ
ಚರ್ಚೆಯ
ಚರ್ಚೆ-ಯನ್ನು
ಚರ್ಚೆ-ಯಲ್ಲಿ
ಚರ್ಚ್
ಚರ್ಯ
ಚರ್ಯವು
ಚಲ-ಚ್ಚಿ-ತ್ರ-ಗಳು
ಚಲ-ನ-ವ-ಲನ
ಚಲ-ನ-ವ-ಲ-ನ-ಗಳನ್ನು
ಚಲ-ನ-ವ-ಲ-ನವೂ
ಚಲ-ನ-ಶೀ-ಲ-ವಾದ
ಚಲನೆ
ಚಲಾ-ವ-ಣೆ-ಯಾ-ಗು-ತ್ತಿತ್ತು
ಚಲಿ-ಸು-ತ್ತಿದೆ
ಚಲಿ-ಸು-ತ್ತಿ-ರು-ವಂತೆ
ಚಲಿ-ಸು-ವ-ವ-ರೆಗೂ
ಚಲ್ಲಪ್ಪ
ಚಲ್ಲಯ್ಯ
ಚಳ-ವಳಿ
ಚಳ-ವ-ಳಿ-ಗಳ
ಚಳ-ವ-ಳಿ-ಗಳು
ಚಳ-ವ-ಳಿಗೆ
ಚಳ-ವ-ಳಿಯ
ಚಳ-ವ-ಳಿ-ಯಲ್ಲಿ
ಚಳಿ
ಚಳಿ-ಗಾಲ
ಚಳಿ-ಗಾ-ಲದ
ಚಳಿ-ಗಾ-ಲ-ದಲ್ಲಿ
ಚಳಿ-ಗಾ-ಲ-ವಾ-ದ್ದ-ರಿಂದ
ಚಳಿ-ಗಾ-ಲ-ವೆಂದರೆ
ಚಳಿ-ಬಿ-ಸಿ-ಲು-ಗಳನ್ನೆಲ್ಲ
ಚಳಿ-ಯನ್ನು
ಚಳಿಯೇ
ಚಳು-ವ-ಳಿಯು
ಚಹ-ರೆ-ಚ-ಹ-ರೆಯೂ
ಚಹಾ
ಚಹಾ-ಕೂ-ಟ-ಗಳ
ಚಹಾ-ಕೂ-ಟ-ಗಳನ್ನು
ಚಹಾಕ್ಕೆ
ಚಹಾ-ಕ್ಕೆಂದು
ಚಹಾದ
ಚಾಕು-ವನ್ನು
ಚಾಚಿ
ಚಾಚೂ
ಚಾಟಿ-ಯೇ-ಟು-ಗಳನ್ನು
ಚಾಟೂ-ಕ್ತಿ-ಗ-ಳಿಂ-ದಲೂ
ಚಾತುರ್
ಚಾತುರ್ಯ
ಚಾತು-ರ್ಯದ
ಚಾಪ-ಲ್ಯ-ದಿಂದ
ಚಾಪೆ
ಚಾಮ-ರ-ಗಳೂ
ಚಾರ
ಚಾರ-ಕಂ-ದಾ-ಚಾ-ರ-ಗ-ಳಲ್ಲ
ಚಾರ-ವಿ-ಷ-ಯ-ಲಂ-ಪ-ಟ-ತ-ನ-ಗ-ಳಿ-ಗಾಗಿ
ಚಾರ-ಕರೂ
ಚಾರಿ
ಚಾರಿ-ಗ-ಳಿಗೆ
ಚಾರಿ-ಗ-ಳೊ-ಬ್ಬರು
ಚಾರಿ-ತ್ರಿಕ
ಚಾರಿತ್ರ್ಯ
ಚಾರಿ-ತ್ರ್ಯದ
ಚಾರಿ-ತ್ರ್ಯ-ಬಲ
ಚಾರಿ-ತ್ರ್ಯ-ವಂತ
ಚಾರು-ಚಂದ್ರ
ಚಾರು-ಚಂ-ದ್ರನ
ಚಾರ್ಲ್ಸ್
ಚಾರ್ವಾ-ಕ-ರಾ-ಗುವ
ಚಾಲ-ನೆ-ಯಲ್ಲಿ
ಚಾಲಾ-ಕಿ-ತ-ನಕ್ಕೆ
ಚಾಲ್ತಿಯ
ಚಾವಟಿ
ಚಿಂತಕ
ಚಿಂತ-ಕರು
ಚಿಂತ-ಕ-ರೊ-ಡನೆ
ಚಿಂತನ
ಚಿಂತ-ನ-ಮಂ-ಥನ
ಚಿಂತ-ನ-ಧಾ-ರೆ-ಯೊಂ-ದಿಗೆ
ಚಿಂತ-ನ-ಲ-ಹ-ರಿ-ಗ-ಳು-ಇವು
ಚಿಂತನೆ
ಚಿಂತ-ನೆ-ಗಳ
ಚಿಂತ-ನೆಯ
ಚಿಂತ-ನೆ-ಯಲ್ಲಿ
ಚಿಂತ-ನೆ-ಯಲ್ಲೆ
ಚಿಂತಿ-ನ-ಶೀಲ
ಚಿಂತಿ-ಸ-ಬೇ-ಕಾ-ಗಿಲ್ಲ
ಚಿಂತಿ-ಸ-ಬೇ-ಕಾದ
ಚಿಂತಿ-ಸ-ಬೇಡ
ಚಿಂತಿ-ಸ-ಬೇಡಿ
ಚಿಂತಿಸಿ
ಚಿಂತಿ-ಸುತ್ತ
ಚಿಂತಿ-ಸು-ತ್ತಾರೆ
ಚಿಂತಿ-ಸು-ತ್ತಿ-ದ್ದಂ-ತೆಯೂ
ಚಿಂತಿ-ಸು-ತ್ತಿ-ದ್ದರು
ಚಿಂತಿ-ಸು-ತ್ತಿ-ದ್ದೆ-ನಮ್ಮ
ಚಿಂತಿ-ಸುವ
ಚಿಂತಿ-ಸು-ವ-ಷ್ಟ-ರಲ್ಲೇ
ಚಿಂತಿ-ಸು-ವು-ದಿ-ಲ್ಲ-ವಲ್ಲ
ಚಿಂತೆ
ಚಿಂತೆ-ಗೊ-ಳ-ಗಾ-ಗಿದ್ದ
ಚಿಂತೆ-ಯ-ನ್ನುಂಟು
ಚಿಂತೆ-ಯಲ್ಲಿ
ಚಿಂತೆ-ಯಿತ್ತು
ಚಿಂತೆ-ಯಿಲ್ಲ
ಚಿಂತೆಯು
ಚಿಂದಿ
ಚಿಂದಿ-ಚಿಂ-ದಿ-ಯಾ-ಗಿದೆ
ಚಿಕಿತ್ಸೆ
ಚಿಕಿ-ತ್ಸೆಗೂ
ಚಿಕಿ-ತ್ಸೆಗೆ
ಚಿಕಿ-ತ್ಸೆಯ
ಚಿಕಿ-ತ್ಸೆ-ಯನ್ನು
ಚಿಕಿ-ತ್ಸೆ-ಯಲ್ಲಿ
ಚಿಕಿ-ತ್ಸೆ-ಯಿಂ-ದಲೂ
ಚಿಕಿ-ತ್ಸೆ-ಯೇನೋ
ಚಿಕಿ-ತ್ಸೆಯೋ
ಚಿಕ್ಕ
ಚಿಕ್ಕಂ-ದಿ-ನ-ಲ್ಲಿಯೇ
ಚಿಕ್ಕಂ-ದಿ-ನಿಂ-ದಲೇ
ಚಿಕ್ಕ-ಚಿಕ್ಕ
ಚಿಕ್ಕಮ್ಮ
ಚಿಕ್ಕ-ವ-ರಿಗೆ
ಚಿಗರೆ
ಚಿಗುರಿ
ಚಿತಾ-ಗ್ನಿ-ಯನ್ನು
ಚಿತಾ-ಗ್ನಿ-ಯಿಂದ
ಚಿತಾ-ವೇ-ದಿ-ಕೆ-ಯನ್ನು
ಚಿತೆಗೆ
ಚಿತೆಯ
ಚಿತೆ-ಯ-ನ್ನಾ-ವ-ರಿ-ಸಿದ
ಚಿತ್-ಆ-ನಂ-ದ-ವನ್ನು
ಚಿತ್ತ-ರಾ-ಗಿ-ದ್ದಾರೆ
ಚಿತ್ತ-ವೃ-ತ್ತಿ-ಯನ್ನು
ಚಿತ್ತ-ಶುದ್ಧಿ
ಚಿತ್ತ-ಶು-ದ್ಧಿ-ಯಿಂದ
ಚಿತ್ತ-ಶೋ-ಧ-ನೆ-ಯಿಂದ
ಚಿತ್ತಾ-ಕ-ರ್ಷ-ಕ-ವಾದ
ಚಿತ್ರ
ಚಿತ್ರ-ಕಲಾ
ಚಿತ್ರ-ಕಲೆ
ಚಿತ್ರ-ಕಾ-ರರು
ಚಿತ್ರ-ಗಳನ್ನು
ಚಿತ್ರ-ಗು-ಪ್ತರ
ಚಿತ್ರ-ಣ-ವಿದೆ
ಚಿತ್ರ-ಣವು
ಚಿತ್ರ-ಣ-ವೊಂದು
ಚಿತ್ರದ
ಚಿತ್ರ-ಪು-ಸ್ತ-ಕ-ಗಳನ್ನು
ಚಿತ್ರ-ವನ್ನು
ಚಿತ್ರ-ವಾಗಿ
ಚಿತ್ರ-ವಿ-ಚಿತ್ರ
ಚಿತ್ರವೇ
ಚಿತ್ರ-ಸ-ಮೇ-ತ-ವಾಗಿ
ಚಿತ್ರಿಸಿ
ಚಿತ್ರಿ-ಸಿ-ದ್ದಾರೆ
ಚಿದಾ-ಕಾ-ಶ-ದೆ-ಡೆಗೆ
ಚಿನ್ನದ
ಚಿನ್ನ-ವನ್ನು
ಚಿಪ್ಪು-ಗಳನ್ನು
ಚಿಮು-ಕಿಸಿ
ಚಿಮ್ಮ-ಬೇ-ಕಾ-ಗಿದೆ
ಚಿಮ್ಮ-ಲ್ಪಟ್ಟ
ಚಿಮ್ಮಿ
ಚಿಮ್ಮಿತು
ಚಿಮ್ಮಿ-ಸಿತು
ಚಿಮ್ಮು-ತ್ತಿತ್ತು
ಚಿಮ್ಮು-ವುದು
ಚಿರ
ಚಿರ-ಋ-ಣಿ-ಯಾ-ಗಿ-ದ್ದರು
ಚಿರ-ಪ-ರಿ-ಚಿತ
ಚಿರ-ಪ-ರಿ-ಚಿ-ತರು
ಚಿರ-ಸ್ಥಾ-ಯಿ-ಯಾಗಿ
ಚಿರ-ಸ್ಮ-ರ-ಣೀಯ
ಚಿಲುಮೆ
ಚಿಲು-ಮೆ-ಗ-ಳಾ-ಗಿ-ದ್ದರು
ಚಿಲು-ಮೆಯ
ಚಿಲು-ಮೆ-ಯನ್ನು
ಚಿಸು-ವಂತೆ
ಚಿಹ್ನೆ
ಚೀಟಿ
ಚೀಟಿ-ಯನ್ನು
ಚೀನಾ-ಗ-ಳಿಗೆ
ಚೀಫ್
ಚೀಲ-ದಿಂದ
ಚುಕ್ಕಿ
ಚುಕ್ಕೆ-ಗಳಿಂದ
ಚುಚ್ಚಿ
ಚುಚ್ಚಿ-ಕೊಂಡ
ಚುಚ್ಚಿ-ದಂ-ತಿ-ರು-ತ್ತದೆ
ಚುಚ್ಚಿ-ದರು
ಚುಚ್ಚು-ಮ-ದ್ದನ್ನು
ಚುಚ್ಚುವ
ಚುಟು-ಕಾಗಿ
ಚುಟ್ಟಾ
ಚುನಾ-ಯಿ-ತ-ರಾ-ದರು
ಚುನಾ-ಯಿಸು
ಚುನಾ-ಯಿ-ಸು-ವಂತೆ
ಚುನಾ-ವ-ಣೆ-ಗ-ಳು-ಅದ
ಚುರುಕಾ
ಚುರು-ಕಾ-ಯಿತು
ಚುರುಕು
ಚುರು-ಕು-ಗೊ-ಳಿ-ಸು-ವು-ದ-ಕ್ಕಾಗಿ
ಚೂಡಾ-ಮಣಿ
ಚೂರಾ-ಗು-ತ್ತದೆ
ಚೂರು
ಚೆಂಗ-ಲ್ಪೇ-ಟೆ-ಯನ್ನು
ಚೆಂಗೀ-ಸ್ಖಾ-ನನ
ಚೆಂಡುಹೂ
ಚೆಕ್
ಚೆಟ್ಟಿ
ಚೆನ್ನ
ಚೆನ್ನಾ
ಚೆನ್ನಾಗಿ
ಚೆನ್ನಾ-ಗಿದೆ
ಚೆನ್ನಾ-ಗಿ-ದೆ-ಯೆಂದು
ಚೆನ್ನಾ-ಗಿ-ದ್ದೀಯಾ
ಚೆನ್ನಾ-ಗಿ-ದ್ದೇನೆ
ಚೆನ್ನಾ-ಗಿಯೇ
ಚೆನ್ನಾ-ಗಿ-ರಲಿ
ಚೆನ್ನಾ-ಗಿರು
ಚೆನ್ನಾ-ಗಿ-ರು-ತ್ತದೆ
ಚೆನ್ನಾ-ಗಿ-ರು-ತ್ತ-ದೆಂ-ಬುದು
ಚೆನ್ನಾ-ಗಿ-ರು-ತ್ತಿತ್ತು
ಚೆನ್ನಾ-ಗಿ-ಲ್ಲ-ದಿ-ದ್ದರೂ
ಚೆನ್ನಾ-ಗಿ-ಲ್ಲ-ವಲ್ಲ
ಚೆನ್ನಾ-ಗಿ-ಲ್ಲ-ವೆಂದೇ
ಚೆನ್ನಾ-ಗಿ-ಲ್ಲ-ವೆಂ-ಬುದು
ಚೆನ್ನಾ-ಯಿ-ತಲ್ಲ
ಚೆರ್ರಿ
ಚೆಲ್ಲಾ-ಟ-ಗ-ಳೆಲ್ಲ
ಚೆಲ್ಲಾ-ಪಿ-ಲ್ಲಿ-ಯಾ-ಗಿ-ದ್ದರೆ
ಚೆಲ್ಲಿದ
ಚೆಲ್ಲಿ-ದುದು
ಚೆಲ್ಲು-ತ್ತಿದ್ದಾ
ಚೇತನ
ಚೇತ-ನದ
ಚೇತ-ನ-ವನ್ನು
ಚೇತ-ನವು
ಚೇತ-ರಿ-ಸಿ-ಕೊಂ-ಡರು
ಚೇತ-ರಿ-ಸಿ-ಕೊಂ-ಡಿರ
ಚೇತ-ರಿ-ಸಿ-ಕೊಂಡು
ಚೇತೋ-ದಾ-ಯ-ಕ-ವಾ-ಗು-ವಂ-ತಾ-ಗಲು
ಚೇಷ್ಟೆ
ಚೈತನ್ಯ
ಚೈತ-ನ್ಯ-ದಾ-ಯಕ
ಚೈತ-ನ್ಯ-ಭ-ರಿ-ತ-ರಾ-ದಂ-ತಹ
ಚೈತ-ನ್ಯರು
ಚೈತ-ನ್ಯ-ವನ್ನು
ಚೈತ-ನ್ಯ-ವ-ನ್ನೆ-ರೆ-ಯುವ
ಚೈತ್ರ-ಮಾ-ಸ-ದಲ್ಲಿ
ಚೈನಾ
ಚೊಕ್ಕ
ಚೊಕ್ಕ-ಟ-ವಾಗಿ
ಚೊಕ್ಕ-ಟ-ವಾ-ಗಿ-ರ-ಬೇ-ಕೆಂ-ಬುದು
ಚೌಕ-ಟ್ಟನ್ನು
ಚೌಕ-ಟ್ಟಿನ
ಚೌಕಟ್ಟು
ಚೌರಂಘೀ
ಚ್ಚಂದ್ರನ
ಚ್ಚರಿಕೆ
ಛಂಗನೆ
ಛತ-ರ್ಪು-ರದ
ಛತ್ರ-ದಲ್ಲಿ
ಛತ್ರಿ
ಛಲ
ಛಾಯಾ-ಚಿ-ತ್ರ-ಗಳನ್ನೂ
ಛಾಯಾ-ಚಿ-ತ್ರ-ಗಳು
ಛಾಯಾ-ಚಿ-ತ್ರ-ವನ್ನು
ಛಾಯೆ-ಯನ್ನು
ಛಾವ-ಣಿಯ
ಛಾವ-ಣಿಯು
ಛಿದ್ರ-ಛಿ-ದ್ರ-ಗೊ-ಳಿ-ಸಿ-ದರು
ಛೀಮಾರಿ
ಛೆ
ಛೇಡಿಸು
ಜಂಕ್ಷ-ನ್ನಿ-ನ-ವ-ರೆಗೂ
ಜಂಜ-ಡ-ಗಳಿಂದ
ಜಂಬ
ಜಖಂ
ಜಗ
ಜಗಕೆ
ಜಗ-ತ್ಕ-ಲ್ಯಾ-ಣದ
ಜಗ-ತ್ತನ್ನು
ಜಗ-ತ್ತನ್ನೆ
ಜಗ-ತ್ತನ್ನೇ
ಜಗ-ತ್ತಾ-ಗಿ-ರು-ವಾಗ
ಜಗ-ತ್ತಿ-ಗಾಗಿ
ಜಗ-ತ್ತಿ-ಗಿಂತ
ಜಗ-ತ್ತಿಗೆ
ಜಗ-ತ್ತಿಗೇ
ಜಗ-ತ್ತಿನ
ಜಗ-ತ್ತಿ-ನ-ಲ್ಲಾ-ಗಲಿ
ಜಗ-ತ್ತಿ-ನಲ್ಲಿ
ಜಗ-ತ್ತಿ-ನ-ಲ್ಲೆಲ್ಲೂ
ಜಗ-ತ್ತಿ-ನಲ್ಲೇ
ಜಗ-ತ್ತಿನಾ
ಜಗ-ತ್ತಿ-ನಾ-ದ್ಯಂತ
ಜಗ-ತ್ತಿನ್ನೂ
ಜಗತ್ತು
ಜಗತ್ತೂ
ಜಗ-ತ್ತೆಲ್ಲ
ಜಗತ್ತೇ
ಜಗ-ತ್ಪ್ರ-ಸಿದ್ಧ
ಜಗ-ತ್ಪ್ರ-ಸಿ-ದ್ಧ-ವ-ನ್ನಾಗಿ
ಜಗ-ತ್ಪ್ರ-ಸಿ-ದ್ಧ-ವಾದ
ಜಗದ
ಜಗ-ದಂ-ಬೆಯ
ಜಗ-ದಾ-ತ್ಮಾ-ನಂದ
ಜಗ-ದಾ-ತ್ಮಾ-ನಂ-ದರ
ಜಗ-ದಾ-ತ್ಮಾ-ನಂ-ದರು
ಜಗ-ದಿಂ-ದ್ರ-ನಾಥ
ಜಗ-ದೀ-ಶ್ಚಂ-ದ್ರ-ಬೋಸ
ಜಗ-ದ್ಗು-ರು-ವನ್ನು
ಜಗ-ದ್ಧಾತ್ರೀ
ಜಗ-ದ್ಧಿ-ತ-ಕ್ಕಾಗಿ
ಜಗ-ದ್ಧಿ-ತಾಯ
ಜಗ-ದ್ವಿ-ಖ್ಯಾ-ತ-ಗೊ-ಳಿ-ಸಿ-ಬಿಡ
ಜಗ-ದ್ವಿ-ಖ್ಯಾ-ತ-ನಾ-ದಾಗ
ಜಗ-ನ್ನಾಥ
ಜಗ-ನ್ಮಾ-ತೃ-ತ್ವದ
ಜಗ-ನ್ಮಾತೆ
ಜಗ-ನ್ಮಾ-ತೆಗೆ
ಜಗ-ನ್ಮಾ-ತೆಯ
ಜಗ-ನ್ಮಾ-ತೆ-ಯ-ತ್ತಲೇ
ಜಗ-ನ್ಮಾ-ತೆ-ಯನ್ನೂ
ಜಗ-ನ್ಮಾ-ತೆಯೆ
ಜಗ-ನ್ಮಾ-ತೆಯೇ
ಜಗ-ಮೋ-ಹನ
ಜಗ-ಮೋ-ಹ-ನ-ಲಾ-ಲ-ನನ್ನು
ಜಗ-ಮೋ-ಹ-ನ-ಲಾ-ಲ-ನೊಂ-ದಿಗೆ
ಜಗ-ಮೋ-ಹ-ನ-ಲಾಲ್
ಜಗಳ
ಜಗ-ಳ-ಗಳನ್ನೆಲ್ಲ
ಜಗ-ಳದ
ಜಗ-ಳ-ವನ್ನು
ಜಗ-ಳ-ವಾ-ಡುವ
ಜಗ-ಳ-ವಿ-ಲ್ಲ-ದಿ-ರಲಿ
ಜಗವ
ಜಗ-ವಂ-ದನ
ಜಗ-ವ-ನೆ-ಚ್ಚ-ರಿ-ಸು-ತಿದೆ
ಜಗವು
ಜಗ-ವೆ-ನಿತು
ಜಗ್ಗ-ದಿ-ರು-ವುದನ್ನು
ಜಗ್ಗಾ-ಡುತ್ತ
ಜಗ್ಗಿ
ಜಟಿಲ
ಜಟಿ-ಲ-ವಾದ
ಜಟೆ-ಯನ್ನು
ಜಡ
ಜಡತೆ
ಜಡ-ನಾ-ಗ-ರಿ-ಕ-ತೆಯ
ಜಡ-ನಿ-ದ್ರೆ-ಯನ್ನು
ಜಡ-ಭ-ರ-ತನ
ಜಡ-ವ-ಸ್ತು-ವಿ-ಗಿ-ರುವ
ಜಡ-ವ-ಸ್ತು-ವೆಂದೂ
ಜಡ-ವಾಗಿ
ಜಡ-ವಾದ
ಜಡ-ವಾ-ದದ
ಜಡ-ಸ-ಮಾಧಿ
ಜಡಿ-ಮಳೆ
ಜತಿ-ನ್ಬಾಬು
ಜತೀಂದ್ರ
ಜನ
ಜನ-ಲ-ಕ್ಷ-ಗ-ಟ್ಟಲೆ
ಜನಕ
ಜನ-ಕ-ನಂತೆ
ಜನ-ಕೋಟಿ
ಜನ-ಕೋ-ಟಿಗೆ
ಜನ-ಕೋ-ಟಿಯ
ಜನ-ಕೋ-ಟಿಯೇ
ಜನಕ್ಕೆ
ಜನ-ಗಳ
ಜನ-ಗ-ಳ-ನ್ನೇಕೆ
ಜನ-ಗಳಲ್ಲಿ
ಜನ-ಗ-ಳಿ-ಗಿಂತ
ಜನ-ಗ-ಳಿಗೂ
ಜನ-ಗ-ಳಿಗೆ
ಜನ-ಗಳು
ಜನ-ಗ-ಳೆಂ-ದರೆ
ಜನ-ಗ-ಳೆಲ್ಲ
ಜನ-ಜಾ-ಗೃ-ತಿಯ
ಜನ-ಜಾತ್ರೆ
ಜನ-ಜಾ-ತ್ರೆಯೇ
ಜನ-ಜೀ-ವನ
ಜನ-ಜೀ-ವ-ನಕ್ಕೂ
ಜನ-ಜೀ-ವ-ನದ
ಜನ-ಜೀ-ವ-ನ-ದಲ್ಲಿ
ಜನ-ಜೀ-ವ-ನ-ವನ್ನು
ಜನತಾ
ಜನ-ತೆಗೆ
ಜನ-ತೆಯ
ಜನ-ತೆಯೂ
ಜನನ
ಜನ-ನ-ಮ-ರ-ಣ-ಗಳ
ಜನ-ನ-ಮ-ರ-ಣದ
ಜನ-ನದ
ಜನ-ನ-ದಂ-ತೆಯೇ
ಜನ-ನ-ವಾದ
ಜನ-ನ-ವಿ-ಲ್ಲ-ದ-ವರು
ಜನ-ನಾಶ
ಜನ-ನಾ-ಶಕ್ಕೆ
ಜನ-ಪ್ರ-ವಾಹ
ಜನ-ಪ್ರಿಯ
ಜನ-ಪ್ರಿ-ಯ-ತೆ-ಯನ್ನು
ಜನ-ಪ್ರಿ-ಯರಾ
ಜನ-ಪ್ರಿ-ಯ-ವಾ-ಗಿ-ಬಿ-ಟ್ಟಿತು
ಜನ-ಪ್ರಿ-ಯ-ವಾ-ದುವು
ಜನ-ಪ್ರಿ-ಯ-ವಾ-ಯಿ-ತೆಂ-ದರೆ
ಜನ-ಮ-ನದ
ಜನ-ಮ-ನ-ದಲ್ಲಿ
ಜನ-ಮ-ನ-ನಾ-ಯ-ಕ-ರಾ-ಗಿ-ದ್ದರು
ಜನ-ಮ-ನ-ವನ್ನು
ಜನ-ಮಾ-ನ-ಸ-ದಲ್ಲಿ
ಜನರ
ಜನ-ರಂತೆ
ಜನ-ರನ್ನು
ಜನ-ರ-ನ್ನು-ದ್ದೇ-ಶಿಸಿ
ಜನ-ರನ್ನೂ
ಜನ-ರಲ್ಲಿ
ಜನ-ರ-ಲ್ಲಿ-ಅ-ದ-ರಲ್ಲೂ
ಜನ-ರಾ-ದರೂ
ಜನ-ರಿಂದ
ಜನ-ರಿ-ಗಾಗಿ
ಜನ-ರಿ-ಗಾದ
ಜನ-ರಿಗೂ
ಜನ-ರಿಗೆ
ಜನ-ರಿ-ಗೆ-ಮ-ನ-ಸೋ-ತಿ-ದ್ದರು
ಜನ-ರಿ-ದ್ದರು
ಜನರು
ಜನ-ರು-ಸಾ-ಧಾ-ರ-ಣ-ವಾಗಿ
ಜನರೂ
ಜನ-ರೆಲ್ಲ
ಜನರೇ
ಜನರೊ
ಜನ-ರೊಂ-ದಿಗೆ
ಜನ-ವರಿ
ಜನ-ವ-ರಿ-ಫೆ-ಬ್ರು-ವರಿ
ಜನ-ವ-ರಿಯ
ಜನ-ವ-ರಿ-ಯಲ್ಲಿ
ಜನ-ವ-ರ್ಗ-ದಲ್ಲಿ
ಜನ-ವ-ಸ-ತಿಯ
ಜನ-ವ-ಸ-ತಿ-ಯಿಂದ
ಜನ-ವೆಲ್ಲ
ಜನ-ಸಂಖ್ಯೆ
ಜನ-ಸಂ-ಖ್ಯೆ-ಯಲ್ಲಿ
ಜನ-ಸಂ-ದಣಿ
ಜನ-ಸಂ-ದ-ಣಿಯ
ಜನ-ಸಂ-ದ-ಣಿ-ಯನ್ನು
ಜನ-ಸಂ-ದ-ಣಿ-ಯಿ-ರು-ತ್ತದೆ
ಜನ-ಸ-ಮು-ದಾ-ಯದ
ಜನ-ಸ-ಮು-ದಾ-ಯವು
ಜನ-ಸ-ಮು-ದಾ-ಯವೇ
ಜನ-ಸ-ಮೂಹ
ಜನ-ಸ-ಮೂ-ಹಕ್ಕೆ
ಜನ-ಸ-ಮೂ-ಹದ
ಜನ-ಸ-ಮೂ-ಹ-ದಲ್ಲಿ
ಜನ-ಸ-ಮೂ-ಹ-ವನ್ನು
ಜನ-ಸ-ಮೂ-ಹವೇ
ಜನ-ಸಾ-ಗ-ರವು
ಜನ-ಸಾ-ಗ-ರವೇ
ಜನ-ಸಾ-ಮಾನ್ಯ
ಜನ-ಸಾ-ಮಾ-ನ್ಯರ
ಜನ-ಸಾ-ಮಾ-ನ್ಯ-ರನ್ನು
ಜನ-ಸಾ-ಮಾ-ನ್ಯ-ರಿಗೂ
ಜನ-ಸಾ-ಮಾ-ನ್ಯ-ರಿ-ಗೆಲ್ಲ
ಜನ-ಸಾ-ಮಾ-ನ್ಯ-ರಿ-ರಲಿ
ಜನ-ಸಾ-ಮಾ-ನ್ಯ-ರೆಲ್ಲ
ಜನ-ಸೇವಾ
ಜನ-ಸೇವೆ
ಜನ-ಸೇ-ವೆಯ
ಜನ-ಸ್ತೋ-ಮದ
ಜನ-ಸ್ತೋ-ಮ-ದಿಂದ
ಜನ-ಸ್ತೋ-ಮ-ವನ್ನು
ಜನ-ಸ್ತೋ-ಮವು
ಜನ-ಸ್ತೋ-ಮವೇ
ಜನ-ಹಿತ
ಜನ-ಹಿ-ತ-ಕ್ಕಾ-ಗಿಯೇ
ಜನಾಂ
ಜನಾಂಗ
ಜನಾಂ-ಗ-ಕ್ಕಾಗಿ
ಜನಾಂ-ಗಕ್ಕೂ
ಜನಾಂ-ಗಕ್ಕೆ
ಜನಾಂ-ಗ-ಗಳ
ಜನಾಂ-ಗ-ಗ-ಳಲ್ಲೂ
ಜನಾಂ-ಗ-ಗ-ಳಿಗೆ
ಜನಾಂ-ಗ-ಜೀ-ವ-ನಕ್ಕೂ
ಜನಾಂ-ಗದ
ಜನಾಂ-ಗ-ದಲ್ಲಿ
ಜನಾಂ-ಗ-ದ್ವೇಷ
ಜನಾಂ-ಗವೇ
ಜನಾನಾ
ಜನಿಕ
ಜನಿ-ಕ-ವಾಗಿ
ಜನಿ-ವಾರ
ಜನಿಸಿ
ಜನಿ-ಸಿದ
ಜನಿ-ಸಿ-ದುದು
ಜನಿ-ಸಿದ್ದು
ಜನಿ-ಸಿ-ಬ-ರು-ವಂ-ತಾ-ಗಲಿ
ಜನೋ-ಪ-ಕಾರಿ
ಜನ್ಮ
ಜನ್ಮ-ಕ್ಕಿ-ರಲಿ
ಜನ್ಮ-ಗಳೂ
ಜನ್ಮತಃ
ಜನ್ಮ-ತ-ಳೆದ
ಜನ್ಮ-ತ-ಳೆದು
ಜನ್ಮ-ತಾ-ಳಿ-ತೆಂದು
ಜನ್ಮ-ತಾ-ಳಿದ
ಜನ್ಮ-ತಾ-ಳಿ-ದ್ದಾರೆ
ಜನ್ಮದ
ಜನ್ಮ-ದಲ್ಲಿ
ಜನ್ಮ-ದಲ್ಲೇ
ಜನ್ಮ-ದಾ-ತೆಯ
ಜನ್ಮ-ದಿನ
ಜನ್ಮ-ದಿ-ನದ
ಜನ್ಮ-ದಿ-ನೋ-ತ್ಸವ
ಜನ್ಮ-ದಿ-ನೋ-ತ್ಸ-ವಕ್ಕೆ
ಜನ್ಮ-ದಿ-ನೋ-ತ್ಸ-ವದ
ಜನ್ಮ-ವನ್ನು
ಜನ್ಮ-ವಾ-ಗದು
ಜನ್ಮ-ವಿತ್ತ
ಜನ್ಮ-ವಿ-ತ್ತಾರು
ಜನ್ಮ-ವೆ-ತ್ತಲೂ
ಜನ್ಮ-ವೆತ್ತಿ
ಜನ್ಮ-ವೆ-ತ್ತಿದ
ಜನ್ಮ-ವೆ-ತ್ತಿ-ದ್ದೀರಿ
ಜನ್ಮ-ವೆ-ತ್ತಿದ್ದು
ಜನ್ಮ-ವೆ-ತ್ತಿ-ರ-ಬೇ-ಕಾ-ಗು-ತ್ತದೆ
ಜನ್ಮ-ವೆ-ತ್ತಿ-ರು-ವುದು
ಜನ್ಮ-ವೆ-ತ್ತು-ತ್ತಾನೆ
ಜನ್ಮ-ವೆ-ತ್ತುವ
ಜನ್ಮವೇ
ಜನ್ಮ-ಸಾ-ಲದು
ಜನ್ಮ-ಹೀನ
ಜಪ
ಜಪಕ್ಕೆ
ಜಪ-ಗ-ಳಿ-ಗಾಗಿ
ಜಪ-ತ-ಪ-ಗಳನ್ನು
ಜಪ-ಧ್ಯಾನ
ಜಪ-ಧ್ಯಾ-ನ-ಗಳಲ್ಲಿ
ಜಪ-ಧ್ಯಾ-ನ-ಗಳೂ
ಜಪ-ಮಾಲೆ
ಜಪ-ಮಾ-ಲೆ-ಯನ್ನು
ಜಪಾ-ನಿಗೆ
ಜಪಾ-ನಿನ
ಜಪಾ-ನಿ-ನಲ್ಲಿ
ಜಪಾ-ನಿ-ನ-ಲ್ಲಿ-ದ್ದಾಳೆ
ಜಪಾ-ನಿ-ನಿಂದ
ಜಪಾನೀ
ಜಪಾನ್
ಜಪಿ-ಸುತ್ತ
ಜಬ-ರ-ದ-ಸ್ತಿಗೆ
ಜಬ್ಬ-ಲ್ಪು-ರದ
ಜಮ-ಖಾ-ನದ
ಜಮ-ಖಾ-ನೆಯ
ಜಮ-ಖಾ-ನೆ-ಯನ್ನು
ಜಮೀ-ನನ್ನು
ಜಮೀ-ನಿ-ನಲ್ಲಿ
ಜಮೀ-ನ್ದಾ-ರ-ನಾ-ಗಿ-ದ್ದ-ವನು
ಜಮೀ-ನ್ದಾ-ರರ
ಜಮ್ಮುವಿ
ಜಮ್ಮು-ವಿಗೆ
ಜಮ್ಮು-ವಿಗೇ
ಜಮ್ಮು-ವಿ-ನಲ್ಲಿ
ಜಮ್ಮು-ವಿ-ನ-ಲ್ಲಿದ್ದ
ಜಯ
ಜಯಂ-ತಿಯ
ಜಯಂ-ತಿ-ಯನ್ನು
ಜಯಂ-ತ್ಯು-ತ್ಸವ
ಜಯಂ-ತ್ಯು-ತ್ಸ-ವಕ್ಕೆ
ಜಯಂ-ತ್ಯು-ತ್ಸ-ವ-ದಲ್ಲಿ
ಜಯಂ-ತ್ಯು-ತ್ಸ-ವನ್ನು
ಜಯ-ಕಾರ
ಜಯ-ಘೋಷ
ಜಯ-ಘೋ-ಷ-ಗಳ
ಜಯ-ಘೋ-ಷ-ಗಳು
ಜಯ-ಘೋ-ಷ-ಗ-ಳೆಲ್ಲ
ಜಯ-ಘೋ-ಷ-ಗ-ಳೊಂ-ದಿಗೆ
ಜಯ-ಘೋ-ಷದ
ಜಯ-ಘೋ-ಷ-ದೊಂ-ದಿಗೆ
ಜಯ-ಜ-ಯ-ವೆಂ-ದಿವೆ
ಜಯ-ದೇವ
ಜಯ-ನ-ಗರ
ಜಯ-ಭೇರಿ
ಜಯ-ರಾಂ-ಬಾ-ಟಿ-ಯಿಂದ
ಜಯ-ವಾ-ಗಲಿ
ಜಯ-ವೆನ್ನಿ
ಜಯಾ-ಪ-ಜ-ಯ-ಗ-ಳಿಗೆ
ಜಯಿಸಿ
ಜಯಿ-ಸಿದ
ಜಯಿ-ಸಿ-ದ-ವ-ರನ್ನು
ಜಯಿ-ಸುತ್ತ
ಜಯಿ-ಸು-ತ್ತದೆ
ಜಯಿ-ಸು-ವು-ದ-ಕ್ಕಾಗಿ
ಜರೆ-ದ-ವ-ರಲ್ಲ
ಜರೆ-ದಾಗ
ಜರ್ಜ-ರ-ಗೊಂ-ಡಿತ್ತು
ಜರ್ಜ-ರಿ-ತ-ರಾ-ಗಿ-ಬಿ-ಟ್ಟರು
ಜರ್ಜ-ರಿ-ತ-ವಾ-ಗಿದ್ದ
ಜರ್ಜ-ರಿ-ತ-ವಾ-ಗಿ-ಬಿ-ಟ್ಟಿತ್ತು
ಜರ್ಮನಿ
ಜರ್ರೋ
ಜಲದ
ಜಲ-ಪಾತ
ಜಲ-ವನ್ನು
ಜಲ-ಸಂ-ಚಾ-ರದ
ಜಲ-ಸಂ-ಧಿಯ
ಜಲ-ಸ-ಮಾ-ಧಿ-ಯಾ-ಗೋಣ
ಜಲಾ-ಶ-ಯ-ವನ್ನು
ಜವದಿ
ಜವ-ನಿಕೆ
ಜವಾ-ಬ್ದಾರಿ
ಜವಾ-ಬ್ದಾ-ರಿ-ಗಳನ್ನು
ಜವಾ-ಬ್ದಾ-ರಿ-ಯನ್ನು
ಜವಾ-ಬ್ದಾ-ರಿ-ಯನ್ನೂ
ಜವಾ-ಬ್ದಾ-ರಿ-ಯ-ನ್ನೆಲ್ಲ
ಜವಾ-ಬ್ದಾ-ರಿಯೂ
ಜಸ್
ಜಸ್ಟಿಸ್
ಜಹಜು
ಜಹಾಂ
ಜಾಗ
ಜಾಗಕ್ಕೆ
ಜಾಗ-ಗ-ಳ-ಲ್ಲಂತೂ
ಜಾಗಟೆ
ಜಾಗ-ಟೆ-ಗಳ
ಜಾಗ-ಟೆ-ಗಳನ್ನು
ಜಾಗ-ತಿಕ
ಜಾಗ-ದಲ್ಲಿ
ಜಾಗ-ರಣ
ಜಾಗ-ರ-ಣ-ಇವು
ಜಾಗ-ರೂ-ಕಳೂ
ಜಾಗ-ವನ್ನು
ಜಾಗ-ವಿ-ರ-ಲೇ-ಬೇಕು
ಜಾಗೃತ
ಜಾಗೃ-ತ-ಗೊ-ಳಿ-ಸಿಕೊ
ಜಾಗೃ-ತ-ಗೊ-ಳಿ-ಸಿ-ಕೊಂಡು
ಜಾಗೃ-ತ-ಗೊ-ಳಿ-ಸಿ-ಕೊಂ-ಡುದೇ
ಜಾಗೃ-ತ-ಗೊ-ಳಿ-ಸಿ-ದರೋ
ಜಾಗೃ-ತ-ಗೊ-ಳಿಸು
ಜಾಗೃ-ತ-ಗೊ-ಳಿ-ಸು-ತ್ತಾನೆ
ಜಾಗೃ-ತ-ಗೊ-ಳಿ-ಸು-ವಂತೆ
ಜಾಗೃ-ತ-ಗೊ-ಳಿ-ಸು-ವು-ದ-ಕ್ಕಾಗಿ
ಜಾಗೃ-ತ-ಗೊ-ಳಿ-ಸು-ವುದು
ಜಾಗೃ-ತ-ರಾಗಿ
ಜಾಗೃ-ತ-ರಾ-ದರು
ಜಾಗೃ-ತ-ವಾಗಿ
ಜಾಗೃ-ತ-ವಾ-ಗು-ವಲ್ಲಿ
ಜಾಗೃ-ತ-ವಾ-ಯಿತು
ಜಾಗೃ-ತಿ-ಯ-ನ್ನುಂ-ಟು-ಮಾ-ಡು-ವಂತೆ
ಜಾಗ್ರತ
ಜಾಗ್ರ-ತೆ-ಯಿಂ-ದಿ-ರ-ಬೇ-ಕಾ-ಗು-ತ್ತದೆ
ಜಾಜ್ವ-ಲ್ಯ-ಮಾ-ನ-ವಾದ
ಜಾಡ-ಮಾಲಿ
ಜಾಡ-ಮಾ-ಲಿ-ಗಳ
ಜಾಣ
ಜಾಣರು
ಜಾಣೆ
ಜಾತಕ
ಜಾತಿ
ಜಾತಿ-ಮ-ತ-ಗಳ
ಜಾತಿ-ಗ-ಳಂತೂ
ಜಾತಿ-ಗಳಲ್ಲಿ
ಜಾತಿ-ಗ-ಳ-ವ-ರಿಗೆ
ಜಾತಿ-ಪ-ದ್ಧತಿ
ಜಾತಿ-ಪ-ದ್ಧ-ತಿಯ
ಜಾತಿ-ಭೇ-ದ-ಗಳನ್ನು
ಜಾತಿ-ಭ್ರ-ಷ್ಟ-ನಂತೆ
ಜಾತಿ-ಭ್ರ-ಷ್ಟ-ನಾ-ಗು-ವು-ದಿಲ್ಲ
ಜಾತಿ-ಮತ
ಜಾತಿ-ಮ-ತ-ಗಳ
ಜಾತಿಯ
ಜಾತಿ-ಯ-ವನ
ಜಾತಿ-ಯ-ವರು
ಜಾತಿ-ಯಾ-ದ-ರೇನು
ಜಾತಿ-ಯಿಂದ
ಜಾತಿಯು
ಜಾತಿಯೂ
ಜಾತಿಯೆ
ಜಾತ್ಯ-ಭಿ-ಮಾನವೂ
ಜಾತ್ರೆಯೂ
ಜಾನ್
ಜಾನ್ಸನ್
ಜಾನ್ಸ-ನ್ರಂ-ಥ-ವರು
ಜಾನ್ಸ್ಟನ್
ಜಾಫ್ನಾ
ಜಾಫ್ನಾದ
ಜಾಫ್ನಾ-ದತ್ತ
ಜಾಫ್ನಾ-ದಿಂದ
ಜಾಯ-ಸ್ವ-ಮ್ರಿ-ಯಸ್ವ
ಜಾಯ್
ಜಾರ-ಬೇಡಿ
ಜಾರಿಗೆ
ಜಾರಿ-ತೆಂ-ದರೆ
ಜಾರಿ-ದರು
ಜಾರಿ-ಬಿ-ದ್ದಿತು
ಜಾರುತ್ತ
ಜಾರ್ಜ್
ಜಾಲ-ದಲ್ಲಿ
ಜಾಲ-ದಿಂದ
ಜಾವ
ಜಾವ-ದಲ್ಲಿ
ಜಾಸ್ತಿ
ಜಾಸ್ತಿ-ಯಾಗಿ
ಜಾಸ್ತಿ-ಯಾ-ಯಿತು
ಜಾಹೀ-ರಾತು
ಜಿ
ಜಿಂಕೆ
ಜಿಂಕೆ-ಯೇ-ನಾ-ದರೂ
ಜಿಂಕೆಯೋ
ಜಿಕ
ಜಿಜ್ಞಾ-ಸು-ಗಳ
ಜಿಜ್ಞಾ-ಸು-ಗಳು
ಜಿಜ್ಞಾ-ಸು-ಗಳೇ
ಜಿಜ್ಞಾಸೆ
ಜಿಜ್ಞಾ-ಸೆಯ
ಜಿಡು-ಕಿ-ನಿಂ-ದಾಗಿ
ಜಿತೇಂ-ದ್ರಿ-ಯ-ತ್ವ-ವನ್ನು
ಜಿನನೊ
ಜಿಲ್ಲಾ
ಜಿಲ್ಲೆ
ಜಿಲ್ಲೆಯ
ಜಿಲ್ಲೆ-ಯಲ್ಲಿ
ಜಿಸಿ
ಜಿಸಿಗೂ
ಜಿಸಿ-ಯಲ್ಲಿ
ಜೀ
ಜೀತ-ದ-ವರೇ
ಜೀನಾ
ಜೀರ್ಣ-ವ-ಸ್ತ್ರ-ವೆಂ-ಬಂತೆ
ಜೀರ್ಣ-ವಾಗಿ
ಜೀರ್ಣ-ವಾ-ಗಿದ್ದ
ಜೀರ್ಣ-ವಾದ
ಜೀರ್ಣಿ-ಸಿ-ಕೊ-ಳ್ಳ-ಬೇಕು
ಜೀರ್ಣಿ-ಸಿ-ಕೊ-ಳ್ಳಲು
ಜೀರ್ಣೋ-ದ್ದಾರ
ಜೀವ
ಜೀವ-ಜೀ-ವ-ನ-ವನ್ನೇ
ಜೀವ-ಶಿವ
ಜೀವಂತ
ಜೀವಂ-ತ-ವಾಗಿ
ಜೀವಂ-ತ-ವಾ-ಗಿ-ಜ್ವ-ಲಂ-ತ-ವಾಗಿ
ಜೀವಂ-ತ-ವಾ-ಗಿದೆ
ಜೀವಂ-ತ-ವಾ-ಗಿ-ರುವು
ಜೀವ-ಕ-ಳೆ-ಯ-ನ್ನೀ-ಯು-ವಂತೆ
ಜೀವ-ಗಳನ್ನು
ಜೀವ-ಗಳು
ಜೀವ-ಚ್ಛ-ವ-ದಂ-ತಿ-ರುವ
ಜೀವದ
ಜೀವ-ದಾನ
ಜೀವನ
ಜೀವ-ನ
ಜೀವ-ನ-ಬೋ-ಧ-ನೆ-ಗಳು
ಜೀವ-ನ-ಸಂ-ದೇಶ
ಜೀವ-ನ-ಸಂ-ದೇ-ಶ-ಗಳ
ಜೀವ-ನ-ಸಂ-ದೇ-ಶ-ಗಳನ್ನು
ಜೀವ-ನಕ್ಕೆ
ಜೀವ-ನ-ಕ್ಕೇ-ರುವ
ಜೀವ-ನ-ಕ್ಕೊಂದು
ಜೀವ-ನ-ಕ್ರ-ಮಕ್ಕೂ
ಜೀವ-ನ-ಕ್ರ-ಮಕ್ಕೆ
ಜೀವ-ನ-ಕ್ರ-ಮ-ವನ್ನೂ
ಜೀವ-ನ-ಗಳನ್ನು
ಜೀವ-ನ-ಚ-ಕ್ರದ
ಜೀವ-ನದ
ಜೀವ-ನ-ದತ್ತ
ಜೀವ-ನ-ದಲ್ಲಿ
ಜೀವ-ನ-ದಲ್ಲೂ
ಜೀವ-ನ-ದಲ್ಲೇ
ಜೀವ-ನ-ದ-ಲ್ಲೊಂದು
ಜೀವ-ನ-ದಿಯು
ಜೀವ-ನ-ದು-ದ್ದಕ್ಕೂ
ಜೀವ-ನ-ದೊಂ-ದಿಗೆ
ಜೀವ-ನ-ವನ್ನು
ಜೀವ-ನ-ವನ್ನೇ
ಜೀವ-ನ-ವಾ-ಗದೇ
ಜೀವ-ನ-ವಾ-ಹಿ-ನಿಯು
ಜೀವ-ನವು
ಜೀವ-ನವೂ
ಜೀವ-ನ-ವೆಂದರೆ
ಜೀವ-ನ-ವೆಂ-ಬುದು
ಜೀವ-ನವೇ
ಜೀವ-ನವೊ
ಜೀವ-ನ-ವೊಂದು
ಜೀವ-ನ-ಸಾ-ಗ-ರ-ದೊ-ಳಗೆ
ಜೀವ-ನಾಡಿ
ಜೀವ-ನಾ-ದರ್ಶ
ಜೀವ-ನಾ-ದ-ರ್ಶ-ಗಳು
ಜೀವ-ನಾ-ದ-ರ್ಶ-ವ-ನ್ನಾಗಿ
ಜೀವ-ನಾ-ದ-ರ್ಶ-ವಿದೆ
ಜೀವ-ನಾ-ದ-ರ್ಶವು
ಜೀವನೇ
ಜೀವನೋ
ಜೀವ-ನೋ-ದ್ದೇ-ಶ-ವನ್ನೂ
ಜೀವ-ನೋ-ದ್ದೇ-ಶ-ವಿ-ರು-ವಂತೆ
ಜೀವ-ನೋ-ದ್ದೇ-ಶವು
ಜೀವ-ನೋ-ಪ-ಯೋಗೀ
ಜೀವ-ನೋ-ಪಾ-ಯ-ಕ್ಕಾಗಿ
ಜೀವ-ನೋ-ಪಾ-ಯ-ವನ್ನು
ಜೀವ-ನ್ಮುಕ್ತ
ಜೀವ-ನ್ಮು-ಕ್ತರು
ಜೀವ-ನ್ಮುಕ್ತಿ
ಜೀವ-ನ್ಮು-ಕ್ತಿ-ಯನ್ನು
ಜೀವ-ಮಾನ
ಜೀವ-ಮಾ-ನ-ದಲ್ಲೇ
ಜೀವ-ಮಾ-ನ-ದಾ-ದ್ಯಂತ
ಜೀವ-ಮಾ-ನ-ವೆಲ್ಲ
ಜೀವರ
ಜೀವ-ರನ್ನೂ
ಜೀವ-ರಲ್ಲಿ
ಜೀವ-ರ-ಲ್ಲಿ-ರುವ
ಜೀವ-ರಲ್ಲೂ
ಜೀವ-ರೂ-ಪದ
ಜೀವ-ರೊ-ಳ-ಗಿನ
ಜೀವ-ವನ್ನು
ಜೀವ-ವನ್ನೂ
ಜೀವ-ವನ್ನೇ
ಜೀವ-ವಿ-ರು-ವುದೇ
ಜೀವವು
ಜೀವ-ಸ್ವರ
ಜೀವಾತ್ಮ
ಜೀವಾ-ತ್ಮ-ಪ-ರ-ಮಾ-ತ್ಮರ
ಜೀವಾ-ತ್ಮನ
ಜೀವಾ-ತ್ಮನು
ಜೀವಾ-ತ್ಮ-ರನ್ನು
ಜೀವಾಳ
ಜೀವಾ-ಳ-ವ-ನ್ನಾಗಿ
ಜೀವಾ-ಳ-ವನ್ನೇ
ಜೀವಾ-ಳ-ವಿ-ರು-ವುದು
ಜೀವಿ
ಜೀವಿ-ಗಳ
ಜೀವಿ-ಗಳಲ್ಲಿ
ಜೀವಿ-ಗ-ಳಲ್ಲೂ
ಜೀವಿ-ಗ-ಳಿಗೆ
ಜೀವಿ-ಗಳು
ಜೀವಿ-ಗಳೋ
ಜೀವಿಗೆ
ಜೀವಿ-ತದ
ಜೀವಿ-ತ-ದಲ್ಲಿ
ಜೀವಿ-ತಾ-ವ-ಧಿಯ
ಜೀವಿ-ತಾ-ವ-ಧಿ-ಯ-ಲ್ಲಂತೂ
ಜೀವಿ-ತಾ-ವ-ಧಿ-ಯಲ್ಲಿ
ಜೀವಿ-ತಾ-ವ-ಧಿ-ಯಲ್ಲೂ
ಜೀವಿ-ತಾ-ವ-ಧಿ-ಯಲ್ಲೇ
ಜೀವಿತೋ
ಜೀವಿ-ತೋ-ದ್ದೇ-ಶ-ಗಳನ್ನು
ಜೀವಿಯ
ಜೀವಿ-ಯನ್ನೂ
ಜೀವಿ-ಯಲ್ಲೂ
ಜೀವಿ-ಯ-ಲ್ಲೂ-ವಿ-ಶ್ವದ
ಜೀವಿಯು
ಜೀವಿಯೂ
ಜೀವಿ-ಸ-ಬೇ-ಕಾ-ಗು-ತ್ತದೆ
ಜೀವಿ-ಸ-ಬೇಕು
ಜೀವಿ-ಸಿದ
ಜೀವಿ-ಸಿ-ದ-ವ-ರ-ಲ್ಲವೆ
ಜೀವಿ-ಸಿ-ದ್ದರು
ಜೀವಿ-ಸಿ-ದ್ದರೂ
ಜೀವಿ-ಸಿ-ದ್ದಾಗ
ಜೀವಿ-ಸಿ-ದ್ದಾ-ಗಲೇ
ಜೀವಿ-ಸಿದ್ದೇ
ಜೀವಿ-ಸಿ-ರು-ವ-ವರು
ಜೀವಿ-ಸುತ್ತ
ಜೀವಿ-ಸು-ತ್ತಿ-ದ್ದಾರೆ
ಜೀವಿ-ಸು-ತ್ತಿ-ರುವ
ಜೀವಿ-ಸು-ತ್ತೇವೆ
ಜೀವಿ-ಸು-ವು-ದರ
ಜೀವಿ-ಸು-ವು-ದಿ-ಲ್ಲ-ವೆಂದು
ಜುಗು-ಪ್ಸಾ-ಕಾ-ರ-ಕ-ವಾಗಿ
ಜುಗುಪ್ಸೆ
ಜುಗು-ಪ್ಸೆ-ಗೊಂ-ಡರು
ಜುಗು-ಪ್ಸೆಯ
ಜುಟ್ಟು
ಜುದಾ-ಸನೂ
ಜುಮ್ಮೆ-ನ್ನು-ವಂ-ತಿದೆ
ಜುಲೈ
ಜುಲೈ-ನಲ್ಲಿ
ಜುಳು-ಜುಳು
ಜೂನಿ-ನಲ್ಲಿ
ಜೂನಿ-ನಲ್ಲೇ
ಜೂನ್
ಜೂನ್ನಲ್ಲಿ
ಜೂಲ್ಸ್
ಜೆ
ಜೆಮ್
ಜೆರೂ-ಸ-ಲೆ-ಮ್ಮಿಗೆ
ಜೆಲೋನ
ಜೆಹೋ-ವನೊ
ಜೇನ್
ಜೇನ್ಸ್
ಜೇಬಿ-ನಲ್ಲಿ
ಜೇಬಿ-ನ-ಲ್ಲಿಟ್ಟು
ಜೇಮ್ಸ್
ಜೈ
ಜೈಕಾರ
ಜೈಕಾ-ರ-ಹ-ರ್ಷೋ-ದ್ಗಾ-ರ-ಗಳ
ಜೈಕಾ-ರ-ದಿಂದ
ಜೈತ್ರ-ಯಾ-ತ್ರೆಗೆ
ಜೈನ
ಜೈನರ
ಜೈಪು-ರಕ್ಕೆ
ಜೈಪು-ರ-ದಲ್ಲಿ
ಜೈಪು-ರ-ದ-ವ-ರೆಗೂ
ಜೈಪು-ರ-ದಿಂದ
ಜೈಲು
ಜೈಲು-ಗಳನ್ನು
ಜೊತೆ-ಗಾರ
ಜೊತೆ-ಗಾ-ರ-ರಿಗೆ
ಜೊತೆ-ಗಾ-ರರು
ಜೊತೆ-ಗಾ-ರ-ರೆಲ್ಲ
ಜೊತೆ-ಗಿದ್ದು
ಜೊತೆ-ಗೂ-ಡ-ಬೇ-ಕೆಂದು
ಜೊತೆ-ಗೂ-ಡಿ-ಕೊಂ-ಡರು
ಜೊತೆಗೆ
ಜೊತೆಗೇ
ಜೊತೆ-ಗೊ-ಡ-ಲೆಂದು
ಜೊತೆ-ಗೊ-ಡು-ವು-ದಾಗಿ
ಜೊತೆ-ಜೊ-ತೆ-ಯಾ-ಗಿಯೇ
ಜೊತೆ-ಯಲ್ಲಿ
ಜೊತೆ-ಯ-ಲ್ಲಿದ್ದ
ಜೊತೆ-ಯ-ಲ್ಲಿ-ದ್ದ-ವ-ರಾ-ರಿಗೂ
ಜೊತೆ-ಯ-ಲ್ಲಿ-ದ್ದ-ವ-ರಿ-ಗಂತೂ
ಜೊತೆ-ಯ-ಲ್ಲಿ-ದ್ದ-ವರು
ಜೊತೆ-ಯ-ಲ್ಲಿ-ದ್ದ-ವ-ರೆಂ-ದರೆ
ಜೊತೆ-ಯ-ಲ್ಲಿ-ರು-ವುದು
ಜೊತೆ-ಯಲ್ಲೇ
ಜೊತೆ-ಯ-ವ-ರೊಂ-ದಿಗೆ
ಜೊತೆ-ಯಾಗಿ
ಜೊಸೆ-ಫಿನ್
ಜೊಸೈಯ
ಜೋ
ಜೋಗೇಂದ್ರ
ಜೋಗೇಂ-ದ್ರ-ನಾಥ
ಜೋಗೇ-ಶ್ಚಂದ್ರ
ಜೋಡಿ-ಸ-ಲಾ-ಯಿತು
ಜೋಡಿಸಿ
ಜೋಡು
ಜೋಡು-ಕು-ದು-ರೆ-ಗಾ-ಡಿ-ಯಲ್ಲಿ
ಜೋತಾ-ಡು-ತ್ತಿತ್ತು
ಜೋತಾ-ಡುವ
ಜೋಧ್ಪುರ್
ಜೋನಾ-ಥನ್
ಜೋಪಡಿ
ಜೋಪ-ಡಿಗೆ
ಜೋಪ-ಡಿಯ
ಜೋಪ-ಡಿ-ಯನ್ನು
ಜೋಪ-ಡಿ-ಯಲ್ಲಿ
ಜೋಪ-ಡಿ-ಯಲ್ಲೇ
ಜೋಪ-ಡಿ-ಯಿ-ದ್ದುದು
ಜೋಪ-ಡಿಯೇ
ಜೋಪ-ಡಿ-ಯೊಂದು
ಜೋಪಾನ
ಜೋಪಾ-ನ-ವಾಗಿ
ಜೋರಾಗಿ
ಜೋರಾ-ಗಿದೆ
ಜೋರಾ-ಗು-ತ್ತಿದೆ
ಜೋಲು
ಜೋಳಿ-ಗೋ-ಸ್ಕರ
ಜೋಶಿ-ಯ-ವರು
ಜೋಸೆ-ಫಿನ್
ಜೋಸೆ-ಫಿ-ನ್ನ-ರಿಗೆ
ಜೋಸೆ-ಫಿ-ನ್ನರು
ಜೋಸೆ-ಫಿ-ನ್ನ-ರೊಂ-ದಿಗೆ
ಜೋಸೆ-ಫಿ-ನ್ನಳ
ಜೋಸೆ-ಫಿ-ನ್ನ-ಳಿಗೆ
ಜೋಸೆ-ಫಿ-ನ್ನಳು
ಜೋಸೆ-ಫಿ-ನ್ನಳೂ
ಜೋಸೆ-ಫಿ-ನ್ನ-ಳೊಂ-ದಿಗೆ
ಜ್ಞಾನ
ಜ್ಞಾನ-ಅ-ದಮ್ಯ
ಜ್ಞಾನ-ಪ್ರೇ-ಮ-ಕ-ರ್ಮ-ಗ-ಳೆಂಬ
ಜ್ಞಾನ-ಬು-ದ್ಧಿ-ಹೃ-ದ-ಯ-ಗಳು
ಜ್ಞಾನ-ಕಾಂಡ
ಜ್ಞಾನ-ಕಾಂ-ಡಕ್ಕೆ
ಜ್ಞಾನ-ಕಾಂ-ಡದ
ಜ್ಞಾನ-ಕಾಂ-ಡವೇ
ಜ್ಞಾನಕ್ಕೆ
ಜ್ಞಾನದ
ಜ್ಞಾನ-ದಲ್ಲಿ
ಜ್ಞಾನ-ದಾನ
ಜ್ಞಾನ-ದಾ-ನವು
ಜ್ಞಾನ-ದೀಪ್ತಿ
ಜ್ಞಾನ-ನಿಧಿ
ಜ್ಞಾನ-ಪಿ-ಪಾ-ಸು-ಗ-ಳಾಗಿ
ಜ್ಞಾನ-ಪಿ-ಪಾಸೆ
ಜ್ಞಾನ-ಪ್ರ-ಚೋ-ದಕ
ಜ್ಞಾನ-ಪ್ರ-ದ-ವಾ-ಗಿತ್ತು
ಜ್ಞಾನ-ಪ್ರ-ದ-ವಾ-ಗಿ-ರ-ಲೇ-ಬೇಕು
ಜ್ಞಾನ-ಪ್ರ-ದವೂ
ಜ್ಞಾನ-ಪ್ರಾಪ್ತಿ
ಜ್ಞಾನ-ಭಂ-ಡಾ-ರದ
ಜ್ಞಾನ-ಭಂ-ಡಾ-ರ-ದಿಂದ
ಜ್ಞಾನ-ಯೋಗ
ಜ್ಞಾನ-ಯೋ-ಗದ
ಜ್ಞಾನ-ರಾ-ಶಿಯ
ಜ್ಞಾನ-ರಾ-ಶಿ-ಯನ್ನು
ಜ್ಞಾನ-ರಾ-ಶಿಯು
ಜ್ಞಾನ-ವನ್ನು
ಜ್ಞಾನ-ವಾ-ಹಿನಿ
ಜ್ಞಾನ-ವುಂ-ಟಾ-ಗು-ತ್ತದೆ
ಜ್ಞಾನವೂ
ಜ್ಞಾನ-ವೃ-ದ್ಧರು
ಜ್ಞಾನ-ವೆಂಬ
ಜ್ಞಾನ-ಶಕ್ತಿ
ಜ್ಞಾನಾ-ನಂ-ದರು
ಜ್ಞಾನಾ-ರ್ಜ-ನೆ-ಗಳ
ಜ್ಞಾನಿ-ಗಳು
ಜ್ಞಾನಿ-ಯಾ-ಗಿದ್ದ
ಜ್ಞಾನಿಯೋ
ಜ್ಞಾಪಿ-ಸ-ಬೇಕು
ಜ್ಞಾಪಿ-ಸಿ-ದರೆ
ಜ್ಯೋತಿ
ಜ್ಯೋತಿ-ಕಿ-ರ-ಣ-ಗಳು
ಜ್ಯೋತಿಯ
ಜ್ಯೋತಿ-ಯನ್ನು
ಜ್ಯೋತಿ-ಯಲ್ಲೇ
ಜ್ಯೋತಿ-ರೂ-ಪ-ರಾಗಿ
ಜ್ವರ
ಜ್ವರ-ನೆ-ಗ-ಡಿ-ಕೆಮ್ಮು
ಜ್ವರ-ದಿಂದ
ಜ್ವರ-ಪೀ-ಡಿ-ತ-ರಾದ
ಜ್ವರ-ವಂ-ತೆ-ಹಾ-ಗೆಂ-ದ-ರೇನೋ
ಜ್ವಲಂತ
ಜ್ವಲಿ-ಸು-ತ್ತಿದ್ದ
ಜ್ವಲಿ-ಸುವ
ಜ್ವಾಲಾ-ದತ್ತ
ಜ್ವಾಲಾ-ಮು-ಖಿಯ
ಜ್ವಾಲಾ-ಮು-ಖಿ-ಯನ್ನು
ಜ್ವಾಲೆ
ಜ್ವಾಲೆ-ಗಳಿಂದ
ಜ್ವಾಲೆಗೆ
ಜ್ವಾಲೆ-ಯಲ್ಲಿ
ಜ್ಸಾರ್ನ್ನು
ಝಗ-ಝ-ಗಿ-ಸುವ
ಝರಿ-ಗ-ಳಿದ್ದು
ಝರಿ-ಯಂತೆ
ಝರಿ-ಯಲ್ಲಿ
ಝೇಲಂ
ಝೇಲ-ಮ್ಮಿನ
ಟನ್
ಟರು
ಟರ್ಕಿ
ಟರ್ಕ್
ಟರ್ಕ್ಸ-್ಟ್ರೀ-ಟಿನ
ಟರ್ನ್ಬುಲ್
ಟವೆ-ಲ್ಲಿ-ನಿಂದ
ಟಾಂಗಾ
ಟಾಂಗಾ-ಗಳಲ್ಲಿ
ಟಾಂಗಾ-ದಲ್ಲಿ
ಟಾಂಗಾ-ದ-ವ-ರೆಗೆ
ಟಾಗ
ಟಾಗೋ-ರರ
ಟಾಗೋ-ರ-ರನ್ನು
ಟಾಗೋರ್
ಟಾಟಾ
ಟಾಟಾರ
ಟಾಟಾ-ರ-ವ-ರಿಂದ
ಟಾಟಾ-ರ-ವರು
ಟಾಟಾ-ರೊಂ-ದಿಗೂ
ಟಾಯಿತು
ಟಾಯ್ನಬೀ
ಟಿ
ಟಿಕೆ-ಟಿಗೆ
ಟಿಕೆ-ಟಿನ
ಟಿಕೆಟ್
ಟಿಕೆ-ಟ್ಟನ್ನು
ಟಿಕೆ-ಟ್ಟಿನ
ಟಿಕೆಟ್ಟು
ಟಿಪ್ಪಣಿ
ಟಿಪ್ಪ-ಣಿ-ಗಳು
ಟೀ
ಟೀಕಿ-ಸ-ಲಾ-ರಂ-ಭಿ-ಸಿ-ದ್ದರು
ಟೀಕಿ-ಸಲಿ
ಟೀಕಿ-ಸಲು
ಟೀಕಿಸಿ
ಟೀಕಿ-ಸಿದ
ಟೀಕಿ-ಸಿ-ದರು
ಟೀಕಿ-ಸಿ-ದ-ರು-ನಿಮ್ಮ
ಟೀಕಿ-ಸಿ-ದರೂ
ಟೀಕಿ-ಸಿ-ದರೆ
ಟೀಕಿ-ಸಿ-ದ-ವ-ರಲ್ಲ
ಟೀಕಿ-ಸಿ-ದ್ದ-ಕ್ಕಾಗಿ
ಟೀಕಿ-ಸಿ-ರ-ಬ-ಹುದು
ಟೀಕಿ-ಸಿ-ರಲು
ಟೀಕಿ-ಸು-ತ್ತಿ-ದ್ದರು
ಟೀಕಿ-ಸು-ತ್ತಿ-ದ್ದ-ವ-ರಲ್ಲಿ
ಟೀಕಿ-ಸು-ತ್ತಿ-ದ್ದ-ವ-ರಿಗೆ
ಟೀಕಿ-ಸು-ವ-ವರು
ಟೀಕಿ-ಸು-ವು-ದರ
ಟೀಕಿ-ಸು-ವು-ದ-ರಿಂ-ದಾ-ಗಲಿ
ಟೀಕಿ-ಸು-ವುದು
ಟೀಕಿ-ಸು-ವು-ದು-ಪ್ರ-ತಿ-ಭ-ಟಿ-ಸು-ವುದು
ಟೀಕೆ
ಟೀಕೆ-ಟಿ-ಪ್ಪ-ಣಿ-ಗಳನ್ನು
ಟೀಕೆ-ಗಳ
ಟೀಕೆ-ಗಳನ್ನೂ
ಟೀಕೆ-ಗ-ಳಿಗೆ
ಟೀಕೆ-ಗಳು
ಟೀಕೆ-ಗಾ-ರರ
ಟೀಕೆಗೆ
ಟೀಕೆ-ಯನ್ನು
ಟೀಕೆ-ಯನ್ನೂ
ಟೀಕೆ-ಯ-ನ್ನೆಲ್ಲ
ಟೀಕೆ-ಯನ್ನೇ
ಟೀಕೆ-ಯಿಂದ
ಟೀಕೆ-ಯೇ-ನೆಂ-ದರೆ
ಟೆಬೆ-ಟ್ಟಿನ
ಟೆಸ್ಟ-ಮೆಂಟ್
ಟೈಫಾ-ಯ್ಡ್
ಟೈಲರ್
ಟೈಲ-ರ್-ಇ-ವ-ರು-ಗಳು
ಟೊಂಕ-ಕಟ್ಟಿ
ಟೊಂಗೆ-ಗಳನ್ನು
ಟೊಕು-ನೋಗೆ
ಟೊಯ್ನಬೀ
ಟೌನಿಗೆ
ಟೌನಿಗೋ
ಟೌನಿನ
ಟೌನಿ-ನ-ಲ್ಲೆಲ್ಲ
ಟ್ರಂಕ್
ಟ್ರಸ್ಟನ್ನು
ಟ್ರಸ್ಟಿ-ಗಳ
ಟ್ರಸ್ಟಿ-ಗ-ಳ-ಲ್ಲೊ-ಬ್ಬ-ರ-ನ್ನಾಗಿ
ಟ್ರಸ್ಟಿ-ಗಳು
ಟ್ರಸ್ಟಿ-ಗ-ಳೆಂದು
ಟ್ರಸ್ಟಿನ
ಟ್ರಸ್ಟಿ-ನಿಂದ
ಟ್ರಸ್ಟ್
ಟ್ರಾಮು-ಗಳಲ್ಲಿ
ಟ್ರಾಲಿ-ಯಲ್ಲಿ
ಟ್ರಿಪ್ಲಿ-ಕೇ-ನಿನ
ಟ್ರಿಪ್ಲಿ-ಕೇನ್
ಟ್ರಿಬ್ಯೂ-ನ್-ಎಂಬ
ಟ್ರೂತ್
ಟ್ರೂತ್ನ
ಟ್ರೂತ್ನ-ಲ್ಲಿ-ಸ್ವಾ-ಮೀಜಿ
ಟ್ರೈನನ್ನು
ಟ್ರೈನ-ನ್ನೇರಿ
ಟ್ರೈನಿನ
ಟ್ರೈನಿ-ನಲ್ಲಿ
ಟ್ರೈನು
ಠರಾ-ವನ್ನು
ಠರಾ-ವು-ಗಳನ್ನು
ಠಾಕೂ-ರ-ಣಿ-ಯ-ವರು
ಠಾಕೂ-ರರು
ಠಾಕೂರ್
ಠಿತ-ರಾ-ಗಿ-ದ್ದ-ರೆಂ-ದರೆ
ಠೇವ-ಣಾ-ತಿ-ಯ-ಲ್ಲಿದೆ
ಡಂಗುರ
ಡಂಬೆ-ಲ್ಲು-ಗಳಿಂದ
ಡಂಭಾ-ಚಾ-ರದ
ಡನೆಯೂ
ಡಯಾ
ಡಯಾ-ಬಿ-ಟಿ-ಸಿನ
ಡಯಾ-ಬಿ-ಟೀ-ಸಿಗೆ
ಡಯಾ-ಬಿ-ಟೀಸ್
ಡಯಾ-ಬಿ-ಟೀ-ಸ್-ಇದು
ಡಲು
ಡವ್ಸನ್
ಡಾ
ಡಾಂಭಿಕ
ಡಾಕ್ಟ-ರರ
ಡಾಕ್ಟ-ರ-ರನ್ನು
ಡಾಕ್ಟ-ರರು
ಡಾಕ್ಟ-ರರೂ
ಡಾಕ್ಟ-ರಿಗೆ
ಡಾಕ್ಟರು
ಡಾಕ್ಟ-ರು-ಗಳ
ಡಾಕ್ಟ-ರು-ಗಳು
ಡಾಕ್ಟ-ರು-ಗ-ಳೆಲ್ಲ
ಡಾಕ್ಟ-ರು-ಗಳೇ
ಡಾಕ್ಟ-ರು-ಗ-ಳೊಂ-ದಿಗೆ
ಡಾಕ್ಟರ್
ಡಾಮರ
ಡಾಮ-ರಕ್ಕೆ
ಡಾರ್ಜಿ-ಲಿಂಗಿ
ಡಾರ್ಜಿ-ಲಿಂ-ಗಿಗೆ
ಡಾರ್ಜಿ-ಲಿಂ-ಗಿನ
ಡಾರ್ಜಿ-ಲಿಂ-ಗಿ-ನತ್ತ
ಡಾರ್ಜಿ-ಲಿಂ-ಗಿ-ನಲ್ಲಿ
ಡಾರ್ಜಿ-ಲಿಂ-ಗಿ-ನಿಂದ
ಡಾರ್ವಿ-ನ್ನನ
ಡಾಲ-ರಿ-ನಷ್ಟು
ಡಾಲರು
ಡಾಲ-ರು-ಗಳನ್ನು
ಡಾಲರ್
ಡಾಲ-ರ್ಗ-ಳಷ್ಟು
ಡಿಗ್ರಿ
ಡಿಗ್ರಿ-ಯ-ವ-ರೆಗೂ
ಡಿಸೆಂ-ಬ-ರಾದ್ದ
ಡಿಸೆಂ-ಬ-ರಿ-ನಲ್ಲಿ
ಡಿಸೆಂ-ಬರ್
ಡುತ್ತ
ಡುತ್ತೇನೆ
ಡೂನಿಗೆ
ಡೂವರೆ
ಡೆಕ್ಕಿನ
ಡೆಕ್ಕಿ-ನಲ್ಲಿ
ಡೆಟ್ರಾ-ಯ್ಟಿಗೆ
ಡೆಟ್ರಾ-ಯ್ಟಿ-ನಲ್ಲಿ
ಡೆಟ್ರಾ-ಯ್ಟಿ-ನಿಂದ
ಡೆಟ್ರಾ-ಯ್ಟ್
ಡೆಪ್ಯುಟಿ
ಡೆಮಿ-ಡಾಫ್
ಡೆಹರಾ
ಡೆಹ-ರಾ-ಡೂ-ನಿಗೆ
ಡೆಹ-ರಾ-ಡೂ-ನಿಗೇ
ಡೆಹ-ರಾ-ಡೂ-ನಿ-ನಲ್ಲಿ
ಡೆಹ-ರಾ-ಡೂನ್
ಡೇರೆ
ಡೇರೆ-ಗಳ
ಡೇರೆ-ಗಳನ್ನು
ಡೇರೆ-ಗ-ಳಿ-ಗಿಂತ
ಡೇರೆ-ಗ-ಳಿಗೂ
ಡೇರೆಗೆ
ಡೇರೆಯ
ಡೇರೆ-ಯನ್ನು
ಡೇರೆಯು
ಡೈರೆ-ಕ್ಟರ್
ಡೊರೋತಿ
ಡೊರೋ-ತಿ-ಯನ್ನು
ಡೋಲಾ-ಯ-ಮಾ-ನ-ವಾ-ದರೂ
ಡೋಲು
ಡೋವ-ರ್ವ-ರೆಗೂ
ಢಾಕಾ
ಢಾಕಾಕ್ಕೆ
ಢಾಕಾಗೆ
ಢಾಕಾದ
ಢಾಕಾ-ದಲ್ಲಿ
ಢಾಕಾ-ದ-ಲ್ಲಿದ್ದ
ಢಾಕಾ-ದ-ಲ್ಲಿ-ದ್ದಾಗ
ಢಾಕಾ-ದ-ಲ್ಲಿನ
ಢಾಕಾ-ದಲ್ಲೂ
ಢಾಕಾ-ದಿಂದ
ಣಸಿ-ಯಲ್ಲಿ
ಣಾಮ-ವಾಗಿ
ಣಿಕ-ರಲ್ಲಿ
ಣಿಕರೇ
ಣೀಯ
ತಂಗಾ-ಳಿ-ಯಂತೆ
ತಂಗಾ-ಳಿ-ಯಲ್ಲಿ
ತಂಗಿ
ತಂಗು-ದಾಣ
ತಂಗು-ದಾ-ಣ-ಗಳಲ್ಲಿ
ತಂಗು-ದಾ-ಣದ
ತಂಗು-ದಾ-ಣ-ದ-ಲ್ಲಿಯೂ
ತಂಗು-ದಾ-ಣ-ವಾ-ಯಿತು
ತಂಜಾ-ವೂ-ರು-ಗಳ
ತಂಟೆ
ತಂಡ
ತಂಡ-ಗಳು
ತಂಡದ
ತಂಡ-ದಲ್ಲಿ
ತಂಡ-ದ-ಲ್ಲಿದ್ದ
ತಂಡ-ದ-ವ-ರಿ-ಗೆಲ್ಲ
ತಂಡ-ದ-ವರು
ತಂಡ-ದ-ವರೆಲ್ಲ
ತಂಡ-ದೊ-ಡನೆ
ತಂಡ-ವನ್ನು
ತಂಡ-ವ-ರಿ-ಗೆಲ್ಲ
ತಂಡ-ವಾಗಿ
ತಂಡವೇ
ತಂಡ-ವೊಂದು
ತಂಡೋಪ
ತಂಡೋ-ಪ-ತಂ-ಡ-ವಾಗಿ
ತಂತಿ
ತಂತಿ-ಗಳ
ತಂತಿ-ಗಳು
ತಂತಿಯ
ತಂತಿ-ಯನ್ನೂ
ತಂತಿ-ಯಿಂ-ದಲೇ
ತಂತ್ರ
ತಂತ್ರ-ಗಳಲ್ಲಿ
ತಂತ್ರ-ಗಳು
ತಂತ್ರ-ಜ್ಞಾನ
ತಂತ್ರ-ಧಾ-ರಕ
ತಂತ್ರ-ಧಾ-ರ-ಕ-ರೆಂ-ದರೆ
ತಂತ್ರ-ಶಾ-ಸ್ತ್ರ-ಗಳೂ
ತಂತ್ರ-ಸಾ-ಧನೆ
ತಂತ್ರ-ಸಾ-ಧ-ನೆಯ
ತಂದ
ತಂದಂ-ಥವು
ತಂದದ್ದು
ತಂದ-ಮೇ-ಲೆಯೇ
ತಂದರು
ತಂದಿ-ಡು-ವಂತೆ
ತಂದಿತು
ತಂದಿತ್ತು
ತಂದಿದ್ದ
ತಂದಿ-ದ್ದ-ನ-ಲ್ಲದೆ
ತಂದಿ-ದ್ದರು
ತಂದಿ-ದ್ದಳು
ತಂದಿ-ದ್ದಾರೆ
ತಂದಿ-ರಪ್ಪ
ತಂದಿ-ರಿ-ಸ-ಲಾ-ಯಿತು
ತಂದು
ತಂದುಕೊ
ತಂದು-ಕೊಂಡು
ತಂದು-ಕೊ-ಟ್ಟಿತ್ತು
ತಂದು-ಕೊ-ಟ್ಟಿ-ದ್ದುವು
ತಂದು-ಕೊ-ಡ-ಬ-ಲ್ಲುದು
ತಂದು-ಕೊ-ಡಲು
ತಂದು-ಕೊಡು
ತಂದು-ಕೊ-ಡು-ತ್ತದೆ
ತಂದು-ಕೊ-ಡು-ತ್ತ-ದೆಯೋ
ತಂದು-ಕೊಳ್ಳು
ತಂದು-ಕೊ-ಳ್ಳು-ತ್ತೇನೆ
ತಂದು-ಕೊ-ಳ್ಳು-ವು-ದ-ರಲ್ಲೂ
ತಂದು-ಕೊ-ಳ್ಳು-ವು-ದ-ಲ್ಲದೆ
ತಂದೆ
ತಂದೆ-ತಾ-ಯಿ-ಸ-ಖ-ನಾದ
ತಂದೆ-ತಾಯಂ
ತಂದೆ-ತಾ-ಯಂ-ದಿರು
ತಂದೆ-ತಾಯಿ
ತಂದೊ-ಡನೆ
ತಂದೊ-ಡ್ಡಲು
ತಂಪನ್ನು
ತಂಪ-ನ್ನುಂ-ಟು-ಮಾ-ಡ-ಬೇ-ಕೆಂಬ
ತಂಪಾ-ಗಿತ್ತು
ತಂಪಾ-ಗಿ-ರು-ವು-ದ-ಲ್ಲದೆ
ತಂಪಾದ
ತಂಪೆ-ರೆ-ಯು-ವಂ-ತಿ-ದ್ದು-ದ-ರಿಂದ
ತಂಬಾಕು
ತಂಬಿ-ಗೆ-ಯನ್ನು
ತಂಬುರ
ತಕ್ಕ
ತಕ್ಕಂ-ತಹ
ತಕ್ಕಂತೆ
ತಕ್ಕಂ-ಥವು
ತಕ್ಕ-ದ್ದಲ್ಲ
ತಕ್ಕ-ಮ-ಟ್ಟಿಗೆ
ತಕ್ಕ-ಮ-ಟ್ಟಿನ
ತಕ್ಕು-ದಲ್ಲ
ತಕ್ಕು-ದಾ-ದಂ-ತಹ
ತಕ್ಕುದೆ
ತಕ್ಷಣ
ತಕ್ಷ-ಣ-ದಿಂದ
ತಕ್ಷ-ಣವೇ
ತಗು-ಲಿ-ಕೊಂ-ಡಿದೆ
ತಗು-ಲಿತು
ತಗು-ಲಿತೋ
ತಗು-ಲಿದ
ತಗು-ಲಿ-ದಾಗ
ತಗೋ-ತಾರೆ
ತಗ್ಗ-ಲಿಲ್ಲ
ತಗ್ಗಾ-ಗಿ-ರು-ತ್ತದೆ
ತಗ್ಗಿ
ತಗ್ಗಿರ
ತಗ್ಗಿ-ಸಿ-ಕೊಂಡು
ತಜ್ಞ-ರಾದ
ತಟ್ಟಿ
ತಟ್ಟಿತು
ತಟ್ಟಿದ
ತಟ್ಟಿ-ದರು
ತಟ್ಟಿಯೂ
ತಟ್ಟಿಯೋ
ತಟ್ಟು-ತ್ತ-ದೆಯೆ
ತಟ್ಟು-ತ್ತಿತ್ತು
ತಟ್ಟು-ವ-ವ-ರಿಂ-ದಲ್ಲ
ತಟ್ಟೆ-ಗಳ
ತಟ್ಟೆ-ಯನ್ನು
ತಟ್ಟೆ-ಯಲ್ಲಿ
ತಡ
ತಡ-ಕಾ-ಡು-ತ್ತಿ-ದ್ದೇವೆ
ತಡ-ಮಾ-ಡಿ-ದರೆ
ತಡ-ವ-ರಿ-ಸುತ್ತ
ತಡ-ವಾ-ಗ-ಬ-ಹುದು
ತಡ-ವಾ-ಗಿ-ಅ-ದರೆ
ತಡ-ವಾ-ಗಿ-ಯಾ-ದರೂ
ತಡ-ವಾ-ಯಿತು
ತಡೆ-ಗ-ಟ್ಟದೆ
ತಡೆ-ಗ-ಟ್ಟಲು
ತಡೆ-ಗ-ಟ್ಟು-ವಂ-ತಹ
ತಡೆ-ಗ-ಟ್ಟು-ವುದನ್ನು
ತಡೆ-ಗಳು
ತಡೆ-ದ-ದ್ದು-ಆಹ್
ತಡೆ-ದರು
ತಡೆದು
ತಡೆ-ದು-ಕೊಂ-ಡೀತೇ
ತಡೆ-ದು-ಕೊಳ್ಳ
ತಡೆ-ದು-ಕೊ-ಳ್ಳ-ಲಾ-ರದ
ತಡೆ-ದು-ಕೊ-ಳ್ಳ-ಲಾ-ರದೆ
ತಡೆ-ದು-ಕೊ-ಳ್ಳ-ಲಾ-ರರು
ತಡೆ-ದು-ಕೊ-ಳ್ಳಲು
ತಡೆ-ದು-ಕೊ-ಳ್ಳುತ್ತ
ತಡೆ-ಯ-ದಲೆ
ತಡೆ-ಯ-ಬಲ್ಲ
ತಡೆ-ಯ-ಬ-ಲ್ಲ-ವ-ರಿಲ್ಲ
ತಡೆ-ಯ-ಬ-ಲ್ಲ-ವರು
ತಡೆ-ಯ-ಬಾ-ರ-ದಿತ್ತೆ
ತಡೆ-ಯ-ಲಾ-ರದೆ
ತಡೆ-ಯಲು
ತಡೆ-ಯ-ಹೊ-ರ-ಟಂತೆ
ತಡೆ-ಯು-ತ್ತಿ-ದ್ದರು
ತಡೆ-ಯು-ತ್ತಿ-ದ್ದಾ-ರೆಂದು
ತಡೆ-ಯುವ
ತಡೆ-ಯು-ವು-ದಿಲ್ಲ
ತಡೆ-ಯು-ವುದು
ತಡೆ-ಯು-ವುದೇ
ತಡೆ-ಹಿ-ಡಿ-ದಿ-ಡಲು
ತಡೆ-ಹಿ-ಡಿದು
ತಣಿ-ಸಿ-ಕೊ-ಳ್ಳಲು
ತಣ್ಣ-ಗಾ-ಗಿ-ಬಿಡು
ತಣ್ಣ-ಗಾ-ಗು-ವಷ್ಟು
ತಣ್ಣ-ಗಾದ
ತಣ್ಣಗೆ
ತಣ್ಣ-ನೆಯ
ತಣ್ಣಿ-ಯಂ-ತಹ
ತಣ್ಣಿ-ಯನ್ನು
ತಣ್ಣೀ-ರೆ-ರ-ಚಿದಂ
ತತ್
ತತ್ಕಾ-ಲಕ್ಕೆ
ತತ್ಕ್ಷಣ
ತತ್ಕ್ಷ-ಣ-ದಲ್ಲಿ
ತತ್ತ-ರಿ-ಸು-ತ್ತಿ-ರು-ವುದನ್ನು
ತತ್ತ-ರಿ-ಸು-ತ್ತಿವೆ
ತತ್ತ್ವ
ತತ್ತ್ವ-ಆ-ದ-ರ್ಶ-ಗಳನ್ನು
ತತ್ತ್ವ-ಶಾ-ಸ್ತ್ರ-ಸಿ-ದ್ಧಾಂ-ತ-ಗಳ
ತತ್ತ್ವಕ್ಕೆ
ತತ್ತ್ವ-ಗಳ
ತತ್ತ್ವ-ಗಳನ್ನು
ತತ್ತ್ವ-ಗಳನ್ನೂ
ತತ್ತ್ವ-ಗಳಲ್ಲಿ
ತತ್ತ್ವ-ಗ-ಳಾದ
ತತ್ತ್ವ-ಗ-ಳಾ-ವುವು
ತತ್ತ್ವ-ಗಳಿಂದ
ತತ್ತ್ವ-ಗ-ಳಿಗೆ
ತತ್ತ್ವ-ಗ-ಳಿವೆ
ತತ್ತ್ವ-ಗಳು
ತತ್ತ್ವ-ಗಳೂ
ತತ್ತ್ವ-ಗ-ಳೆ-ರ-ಡನ್ನೂ
ತತ್ತ್ವ-ಗಳೇ
ತತ್ತ್ವ-ಜ್ಞಾ-ನ-ಗಳ
ತತ್ತ್ವ-ಜ್ಞಾ-ನ-ಗಳು
ತತ್ತ್ವ-ಜ್ಞಾ-ನ-ವ-ನ್ನೊ-ಳ-ಗೊಂಡಿ
ತತ್ತ್ವದ
ತತ್ತ್ವ-ನಿ-ಷ್ಠ-ರ-ನ್ನಾ-ಗಿ-ಸು-ವುದು
ತತ್ತ್ವ-ಬೋ-ಧ-ಕನೆ
ತತ್ತ್ವ-ಬೋ-ಧ-ಕರು
ತತ್ತ್ವ-ಮಸಿ
ತತ್ತ್ವ-ವಡ
ತತ್ತ್ವ-ವ-ನ್ನ-ಲ್ಲದೆ
ತತ್ತ್ವ-ವನ್ನು
ತತ್ತ್ವ-ವನ್ನೇ
ತತ್ತ್ವ-ವಾ-ದ-ವ-ಲ್ಲ-ವೆಂ-ಬು-ದನ್ನು
ತತ್ತ್ವ-ವೆಂದರೆ
ತತ್ತ್ವ-ವೆಲ್ಲ
ತತ್ತ್ವವೇ
ತತ್ತ್ವ-ಶಾಸ್ತ್ರ
ತತ್ತ್ವ-ಶಾ-ಸ್ತ್ರ-ಸಂ-ಪ್ರ-ದಾ-ಯ-ಆ-ಚ-ರ-ಣೆ-ಗಳ
ತತ್ತ್ವ-ಶಾ-ಸ್ತ್ರ-ಗಳ
ತತ್ತ್ವ-ಶಾ-ಸ್ತ್ರ-ಗಳಲ್ಲಿ
ತತ್ತ್ವ-ಶಾ-ಸ್ತ್ರ-ಜ್ಞ-ರಾದ
ತತ್ತ್ವ-ಶಾ-ಸ್ತ್ರದ
ತತ್ತ್ವ-ಶಾ-ಸ್ತ್ರ-ದಿಂದ
ತತ್ತ್ವ-ಶಾ-ಸ್ತ್ರ-ದೊಂ-ದಿಗೆ
ತತ್ತ್ವ-ಶಾ-ಸ್ತ್ರ-ವನ್ನು
ತತ್ಪರಿ
ತತ್ಪ-ರಿ-ಣಾ-ಮ-ವಾಗಿ
ತತ್ವ
ತತ್ವ-ಗಳನ್ನು
ತತ್ವ-ಜ್ಞಾ-ನಿ-ಗಳೂ
ತತ್ವ-ವನ್ನು
ತತ್ವ-ವಾ-ದಿ-ಗಳು
ತತ್ವ-ಶಾಸ್ತ್ರ
ತಥ್ಯ
ತಥ್ಯ-ವನ್ನು
ತದ
ತದೇ-ಕ-ಚಿ-ತ್ತ-ದಿಂದ
ತದ್ವಿ-ರು-ದ್ಧ-ವಾ-ಗಲಿ
ತದ್ವಿ-ರು-ದ್ಧ-ವಾಗಿ
ತದ್ವಿ-ರು-ದ್ಧ-ವಾದ
ತದ್ವಿ-ರು-ದ್ಧ-ವಾದು
ತದ್ವಿ-ರು-ದ್ಧ-ವಾ-ದು-ದನ್ನು
ತದ್ವಿ-ರು-ದ್ಧ-ವಾ-ಯಿತು
ತನ-ಕ್ಪು-ರಕ್ಕೆ
ತನ-ಕ್ಪು-ರ-ದ-ಲ್ಲಿನ
ತನ-ಗದು
ತನ-ಗ-ರಿ-ವಿ-ಲ್ಲ-ದಂ-ತೆಯೇ
ತನ-ಗಾಗಿ
ತನ-ಗಿಷ್ಟ
ತನಗೆ
ತನಗೇ
ತನಿ-ಯಾದ
ತನು-ಮನ
ತನು-ಮ-ನ-ಗ-ಳನ್ನೇ
ತನು-ಮ-ನ-ಗಳು
ತನು-ಮ-ನ-ಗ-ಳೆಲ್ಲ
ತನ್ನ
ತನ್ನಂ-ತಹ
ತನ್ನ-ತ-ನ-ವನ್ನೇ
ತನ್ನದೇ
ತನ್ನನ್ನು
ತನ್ನನ್ನೇ
ತನ್ನಲ್ಲಿ
ತನ್ನ-ಲ್ಲಿಗೆ
ತನ್ನ-ಲ್ಲಿ-ರುವ
ತನ್ನ-ಲ್ಲೊಂದು
ತನ್ನ-ವ-ರನ್ನೂ
ತನ್ನ-ವ-ಳೆಂದೇ
ತನ್ನ-ಷ್ಟಕ್ಕೆ
ತನ್ನಿ
ತನ್ನಿಂದ
ತನ್ನಿಂ-ದಾದ
ತನ್ನಿಚ್ಛೆ
ತನ್ನಿ-ಚ್ಛೆ-ಯಂತೆ
ತನ್ನು
ತನ್ನೆ-ದೆ-ಯನ್ನು
ತನ್ನೆಲ್ಲ
ತನ್ನೆ-ಲ್ಲ-ವನ್ನೂ
ತನ್ನೊಂ-ದಿಗೆ
ತನ್ನೊ-ಡನೆ
ತನ್ನೊ-ಡ-ಲೊ-ಳಗೆ
ತನ್ನೊ-ಬ್ಬಳ
ತನ್ನೊ-ಳ-ಗಿನ
ತನ್ನೊ-ಳಗೆ
ತನ್ಮಯ
ತನ್ಮ-ಯ-ತೆ-ಯಿಂದ
ತನ್ಮ-ಯ-ರಾಗಿ
ತನ್ಮ-ಯ-ರಾ-ಗಿ-ದ್ದ-ರೆಂ-ದರೆ
ತನ್ಮ-ಯ-ರಾ-ಗಿ-ದ್ದು-ದನ್ನು
ತನ್ಮ-ಯ-ರಾ-ಗಿ-ಬಿ-ಟ್ಟರು
ತನ್ಮೂ
ತನ್ಮೂ-ಲಕ
ತಪ
ತಪ-ಶ್ಚ-ರ್ಯೆಯ
ತಪ-ಶ್ಚ-ರ್ಯೆ-ಯನ್ನು
ತಪ-ಶ್ಚ-ರ್ಯೆ-ಯಿಂದ
ತಪ-ಶ್ಶಕ್ತಿ-ಯಿಂದ
ತಪ-ಸ್ವಿ-ಗ-ಳಿ-ಗಾ-ದರೆ
ತಪ-ಸ್ಸ-ನ್ನಾ-ಚ-ರಿ-ಸು-ವಂತೆ
ತಪ-ಸ್ಸಲ್ಲ
ತಪ-ಸ್ಸಾಗಿ
ತಪ-ಸ್ಸಿನ
ತಪ-ಸ್ಸಿ-ನಷ್ಟೇ
ತಪ-ಸ್ಸಿ-ನಿಂದ
ತಪಸ್ಸು
ತಪಸ್ಸೇ
ತಪಾ-ದಿ-ಗಳ
ತಪಾ-ದಿ-ಗಳನ್ನು
ತಪಾ-ಸಣೆ
ತಪೋ
ತಪೋ-ಮ-ಗ್ನ-ರಾ-ಗಿ-ದ್ದಾರೆ
ತಪೋ-ಮಯ
ತಪೋ-ಮೂರ್ತಿ
ತಪ್ಪಂತೂ
ತಪ್ಪ-ದ-ವನ
ತಪ್ಪದೆ
ತಪ್ಪನ್ನು
ತಪ್ಪ-ಲಿನ
ತಪ್ಪ-ಲಿಲ್ಲ
ತಪ್ಪಲು
ತಪ್ಪ-ಲ್ಲವೆ
ತಪ್ಪಾ-ಗ-ಲಾ-ರದು
ತಪ್ಪಾಗಿ
ತಪ್ಪಾ-ಗಿ-ರಲೇ
ತಪ್ಪಾ-ಯಿತು
ತಪ್ಪಿ
ತಪ್ಪಿತು
ತಪ್ಪಿದ
ತಪ್ಪಿ-ದರೂ
ತಪ್ಪಿ-ದರೆ
ತಪ್ಪಿ-ದ್ದ-ಲ್ಲ-ವೆಂದ
ತಪ್ಪಿನ
ತಪ್ಪಿ-ನಿಂ-ದಲೇ
ತಪ್ಪಿ-ಸ-ಬ-ಹುದು
ತಪ್ಪಿ-ಸ-ಬೇಕು
ತಪ್ಪಿ-ಸಲು
ತಪ್ಪಿಸಿ
ತಪ್ಪಿ-ಸಿ-ಕೊಂಡು
ತಪ್ಪಿ-ಸಿ-ಕೊಂ-ಡು-ಬಿಟ್ಟ
ತಪ್ಪಿ-ಸಿ-ಕೊ-ಳ್ಳ-ಬೇ-ಕೆಂದೋ
ತಪ್ಪಿ-ಸಿ-ಕೊ-ಳ್ಳ-ಲಾ-ರದೆ
ತಪ್ಪಿ-ಸಿ-ಕೊ-ಳ್ಳ-ಲಾ-ರಿರಿ
ತಪ್ಪಿ-ಸಿ-ಕೊ-ಳ್ಳಲು
ತಪ್ಪಿ-ಸಿ-ದ್ದೇಕೆ
ತಪ್ಪಿ-ಸು-ವಂ-ತಹ
ತಪ್ಪಿ-ಹೋ-ಗಿತ್ತು
ತಪ್ಪಿ-ಹೋ-ದ-ದ್ದಂತೂ
ತಪ್ಪಿ-ಹೋ-ಯಿತು
ತಪ್ಪು
ತಪ್ಪು-ಗಳನ್ನು
ತಪ್ಪು-ಗಳಿಂದ
ತಪ್ಪು-ಗ-ಳಿಗೂ
ತಪ್ಪು-ಗಳು
ತಪ್ಪು-ಗಳೂ
ತಪ್ಪು-ದಾರಿ-ಯಲ್ಲಿ
ತಪ್ಪೆ
ತಪ್ಪೆಂದೇ
ತಪ್ಪೇ-ನಾ-ದೀತು
ತಬ್ಬ-ಲೆಂದು
ತಬ್ಬಿ-ಕೊಂ-ಡರು
ತಬ್ಬಿ-ಬ್ಬಾದ
ತಬ್ಬಿಬ್ಬು
ತಬ್ಬಿ-ಬ್ಬು-ಗೊ-ಳಿ-ಸು-ತ್ತಿ-ದ್ದುವು
ತಬ್ಬಿ-ಹಿ-ಡಿದ
ತಬ್ಬುವ
ತಬ್ಬು-ವ-ರೆಗೂ
ತಮ
ತಮ-ಗಾಗಿ
ತಮ-ಗಾ-ಗಿದ್ದ
ತಮ-ಗಾ-ಗಿಯೇ
ತಮ-ಗಾದ
ತಮ-ಗಿತ್ತ
ತಮ-ಗುಂ-ಟಾದ
ತಮಗೂ
ತಮಗೆ
ತಮ-ಗೆ-ತಾವೇ
ತಮ-ಗೆ-ದು-ರಾಗಿ
ತಮ-ಗೆಲ್ಲ
ತಮ-ಗೆಷ್ಟೇ
ತಮ-ಗೊಂದು
ತಮ-ಗೊ-ಪ್ಪಿ-ಸಿದ್ದ
ತಮ-ಗೊ-ಪ್ಪಿ-ಸುವ
ತಮ-ಸ್ಸಿನ
ತಮ-ಸ್ಸಿ-ನಲ್ಲಿ
ತಮಾಷೆ
ತಮಾ-ಷೆಗೆ
ತಮಾ-ಷೆಯ
ತಮಾ-ಷೆ-ಯಂತೆ
ತಮಾ-ಷೆ-ಯಾಗಿ
ತಮಾ-ಷೆ-ಯಾ-ಗಿಯೇ
ತಮಾ-ಷೆ-ಯೆಂದು
ತಮಾ-ಷೆ-ಯೆ-ನಿ-ಸಿತು
ತಮಾ-ಷೆಯೋ
ತಮಿ-ಳರು
ತಮಿ-ಳಿಗೆ
ತಮಿ-ಳಿ-ನಲ್ಲಿ
ತಮಿಳು
ತಮಿ-ಳು-ನಾ-ಡಿನ
ತಮೋ-ಗು-ಣಿ-ಗಳೂ
ತಮ್ಮ
ತಮ್ಮ-ತ-ನ-ದಲ್ಲಿ
ತಮ್ಮ-ತಮ್ಮ
ತಮ್ಮ-ತ-ಮ್ಮ-ವ-ರನ್ನು
ತಮ್ಮ-ದನ್ನು
ತಮ್ಮ-ದಾ-ಗಿ-ಸಿ-ಕೊ-ಳ್ಳು-ವ-ವ-ರೆಗೆ
ತಮ್ಮದೇ
ತಮ್ಮ-ನನ್ನು
ತಮ್ಮನ್ನು
ತಮ್ಮನ್ನೂ
ತಮ್ಮ-ನ್ನೆಲ್ಲ
ತಮ್ಮನ್ನೇ
ತಮ್ಮ-ಲ್ಲ-ಡ-ಗಿದ್ದ
ತಮ್ಮಲ್ಲಿ
ತಮ್ಮ-ಲ್ಲಿಗೆ
ತಮ್ಮ-ಲ್ಲಿದ್ದ
ತಮ್ಮ-ಲ್ಲಿಯೇ
ತಮ್ಮಲ್ಲೇ
ತಮ್ಮ-ವ-ರಲ್ಲೇ
ತಮ್ಮ-ವ-ರಾದ
ತಮ್ಮ-ವರು
ತಮ್ಮ-ವ-ರೊ-ಡನೆ
ತಮ್ಮ-ವರೋ
ತಮ್ಮ-ಷ್ಟಕ್ಕೆ
ತಮ್ಮ-ಷ್ಟಕ್ಕೇ
ತಮ್ಮಿಂದ
ತಮ್ಮಿಂ-ದಾದ
ತಮ್ಮಿ-ಬ್ಬರ
ತಮ್ಮಿ-ಬ್ಬರು
ತಮ್ಮಿ-ಷ್ಟ-ಮಿ-ತ್ರ-ರೆ-ಲ್ಲರ
ತಮ್ಮೂ-ರಿನ
ತಮ್ಮೆ
ತಮ್ಮೆ-ದು-ರಿ-ನ-ಲ್ಲಿಯೇ
ತಮ್ಮೆ-ದುರು
ತಮ್ಮೆಲ್ಲ
ತಮ್ಮೆ-ಲ್ಲರ
ತಮ್ಮೊಂ
ತಮ್ಮೊಂ-ದಿ-ಗಿದ್ದ
ತಮ್ಮೊಂ-ದಿ-ಗಿ-ದ್ದರೆ
ತಮ್ಮೊಂ-ದಿ-ಗಿನ
ತಮ್ಮೊಂ-ದಿಗೆ
ತಮ್ಮೊ-ಡ-ನಿದ್ದ
ತಮ್ಮೊ-ಡನೆ
ತಮ್ಮೊಬ್ಬ
ತಮ್ಮೊ-ಳಗೇ
ತಯಾ
ತಯಾ-ರಾ-ಗ-ಲಾರ
ತಯಾ-ರಾ-ಗಿ-ರ-ಬೇಕು
ತಯಾ-ರಾ-ದರು
ತಯಾರಿ
ತಯಾ-ರಿ-ಸ-ಲಾದ
ತಯಾ-ರಿ-ಸಲು
ತಯಾ-ರಿಸಿ
ತಯಾ-ರಿ-ಸಿ-ಕೊಂ-ಡಿ-ದ್ದರು
ತಯಾ-ರಿ-ಸಿದ
ತಯಾ-ರಿ-ಸಿ-ದ್ದಾರೆ
ತಯಾ-ರಿ-ಸು-ತ್ತಿರು
ತಯಾ-ರಿ-ಸುವ
ತಯಾ-ರಿ-ಸು-ವಂತೆ
ತಯಾ-ರಿ-ಸು-ವು-ದ-ಲ್ಲದೆ
ತಯಾರು
ತಯಾ-ರು-ಗೊ-ಳಿ-ಸುವ
ತರ
ತರಂ-ಗ-ಗಳ
ತರಂ-ಗ-ಗ-ಳಾಗಿ
ತರಂ-ಗ-ಗಳು
ತರಂ-ಗ-ಗಳೇ
ತರ-ಕಾರಿ
ತರ-ಕಾ-ರಿ-ಗಳ
ತರ-ಗತಿ
ತರ-ಗ-ತಿ-ಗಳ
ತರ-ಗ-ತಿ-ಗಳನ್ನು
ತರ-ಗ-ತಿ-ಗಳನ್ನೂ
ತರ-ಗ-ತಿ-ಗ-ಳ-ಲ್ಲದೆ
ತರ-ಗ-ತಿ-ಗಳಲ್ಲಿ
ತರ-ಗ-ತಿ-ಗ-ಳಲ್ಲೂ
ತರ-ಗ-ತಿ-ಗಳು
ತರ-ಗ-ತಿ-ಗಳೂ
ತರ-ಗ-ತಿಗೆ
ತರ-ಗ-ತಿಗೋ
ತರ-ಗ-ತಿ-ಯನ್ನು
ತರ-ಗ-ತಿ-ಯಲ್ಲಿ
ತರ-ಗ-ತಿಯು
ತರ-ಗ-ತಿ-ಯೊಂ-ದ-ರಲ್ಲಿ
ತರ-ಗ-ಳಿಂ-ದಲೂ
ತರ-ಚಿ-ದ್ದ-ರಿಂದ
ತರ-ತೊ-ಡ-ಗಿತು
ತರ-ಬಲ್ಲ
ತರ-ಬ-ಹು-ದಾದ
ತರ-ಬಾ-ರದು
ತರ-ಬೇಕು
ತರ-ಬೇತಿ
ತರ-ಬೇ-ತಿ-ಗಾಗಿ
ತರ-ಬೇ-ತಿ-ಗೊ-ಳ-ಗಾದ
ತರ-ಬೇ-ತಿ-ಗೊ-ಳಿ-ಸ-ಬೇಕು
ತರ-ಬೇ-ತಿ-ಗೊ-ಳಿ-ಸಲು
ತರ-ಬೇ-ತಿ-ಗೊ-ಳಿ-ಸು-ವಂ-ತಹ
ತರ-ಬೇ-ತಿ-ಗೊ-ಳಿ-ಸು-ವುದು
ತರ-ಬೇ-ತಿ-ಗೊ-ಳಿ-ಸು-ವು-ದೊಂದು
ತರ-ಬೇ-ತಿಯ
ತರ-ಬೇ-ತಿ-ಯನ್ನು
ತರ-ಬೇ-ತಿ-ಯಲ್ಲಿ
ತರ-ಬೇ-ತು-ಗೊಂಡ
ತರ-ಬೇ-ತು-ಗೊ-ಳಿ-ಸುತ್ತ
ತರ-ಬೇ-ತು-ಗೊ-ಳಿ-ಸುವ
ತರ-ಲಾ-ಗಿತ್ತು
ತರ-ಲಾ-ಯಿತು
ತರಲು
ತರಲೆ
ತರ-ಲೇ-ಬೇಡಿ
ತರ-ಲೋ-ಸುಗ
ತರ-ವಲ್ಲ
ತರ-ವ-ಲ್ಲ-ವೆಂದು
ತರವೆ
ತರಾ-ಟೆಗೆ
ತರಾ-ತುರಿ
ತರಿ-ಸ-ಲಾ-ಯಿತು
ತರಿಸಿ
ತರುಣ
ತರು-ಣನ
ತರು-ಣರ
ತರು-ಣ-ರಿಂದ
ತರು-ಣ-ರಿರಾ
ತರು-ಣರು
ತರು-ಣಿ-ಯರ
ತರು-ತ-ಲ-ವಾ-ಸಿ-ಗ-ಳಾದ
ತರು-ತ್ತಿ-ದ್ದಾಗ
ತರುವ
ತರು-ವಂ-ತಿ-ರ-ಲಿಲ್ಲ
ತರು-ವಂತೆ
ತರು-ವಂಥ
ತರು-ವಾಗ
ತರು-ವಾಯ
ತರುವು
ತರು-ವುದಾ
ತರು-ವು-ದಿಲ್ಲ
ತರು-ವುದು
ತರೆ-ದು-ಕೊಳ್ಳು
ತರೆ-ಯು-ವು-ದ-ರೊ-ಳ-ಗಾಗಿ
ತರ್ಕ
ತರ್ಕ-ದಲ್ಲಿ
ತರ್ಕ-ದಿಂ-ದಾ-ದರೂ
ತರ್ಕ-ಬ-ದ್ಧ-ತೆ-ಯನ್ನು
ತರ್ಕ-ಬ-ದ್ಧ-ವಾಗಿ
ತರ್ಕ-ಬ-ದ್ಧ-ವಾ-ಗಿದೆ
ತರ್ಕ-ಬ-ದ್ಧ-ವಾದ
ತರ್ಕ-ವನ್ನು
ತರ್ಕ-ವಿ-ತ-ರ್ಕದ
ತರ್ಕ-ಶ-ಕ್ತಿ-ಯಲ್ಲ
ತರ್ಕ-ಸ-ರ-ಣಿ-ಯಿಂದ
ತರ್ಕ-ಸಾ-ಮ-ರ್ಥ್ಯ-ದಿಂದ
ತಲ-ಪಿತು
ತಲ-ಪು-ತ್ತಿ-ದ್ದಂ-ತೆಯೇ
ತಲ-ಪುವ
ತಲುಪ
ತಲು-ಪ-ಬ-ಹು-ದಾ-ಗಿತ್ತು
ತಲು-ಪ-ಬೇ-ಕಾ-ಗಿತ್ತು
ತಲು-ಪ-ಬೇ-ಕಾ-ದರೆ
ತಲು-ಪ-ಬೇ-ಕೆಂಬ
ತಲು-ಪ-ಲಾ-ರವು
ತಲುಪಿ
ತಲು-ಪಿತು
ತಲು-ಪಿತ್ತು
ತಲು-ಪಿದ
ತಲು-ಪಿ-ದ-ರಲ್ಲ
ತಲು-ಪಿ-ದರು
ತಲು-ಪಿ-ದಳು
ತಲು-ಪಿ-ದ-ವರೇ
ತಲು-ಪಿ-ದಾಗ
ತಲು-ಪಿ-ದುವು
ತಲು-ಪಿ-ದ್ದರು
ತಲು-ಪಿ-ದ್ದಾರೆ
ತಲು-ಪಿ-ದ್ದುದು
ತಲು-ಪಿ-ಬಿ-ಟ್ಟೆವು
ತಲು-ಪಿ-ರು-ತ್ತಾರೆ
ತಲು-ಪಿವೆ
ತಲು-ಪಿಸ
ತಲು-ಪಿ-ಸಲು
ತಲು-ಪಿಸಿ
ತಲು-ಪಿ-ಸಿದೆ
ತಲು-ಪಿ-ಸು-ತ್ತೇವೆ
ತಲು-ಪಿ-ಸುವ
ತಲು-ಪು-ತ್ತದೆ
ತಲು-ಪು-ತ್ತಲೇ
ತಲು-ಪು-ತ್ತಿ-ದ್ದಂತೆ
ತಲು-ಪು-ತ್ತಿ-ದ್ದಂ-ತೆಯೇ
ತಲು-ಪುವ
ತಲು-ಪು-ವಂ-ತಾ-ಯಿತು
ತಲು-ಪು-ವಂತೆ
ತಲು-ಪು-ವವ
ತಲು-ಪು-ವ-ವ-ರೆಗೂ
ತಲು-ಪು-ವು-ದಾಗಿ
ತಲೆ
ತಲೆ-ಕೆ-ಡಿ-ಸಿ-ಕೊಂ-ಡ-ದ್ದುಂಟು
ತಲೆ-ಕೆ-ಳಗು
ತಲೆ-ಗಿಲೆ
ತಲೆ-ಗೂ-ದಲು
ತಲೆಗೆ
ತಲೆ-ತ-ಪ್ಪಿ-ಸಿ-ಕೊಂಡ
ತಲೆ-ದೋ-ರಿ-ದಾಗ
ತಲೆ-ನೋ-ವಿಗೆ
ತಲೆ-ನೋವು
ತಲೆ-ಬಾ-ಗದೆ
ತಲೆ-ಬಾಗಿ
ತಲೆ-ಬಾ-ಗಿ-ದರು
ತಲೆ-ಬಾ-ಗು-ವುದು
ತಲೆ-ಬಾ-ಗು-ವುದೇ
ತಲೆ-ಬು-ರ-ಡೆ-ಅ-ಡ್ಡ-ಮೂ-ಳೆ-ಗಳ
ತಲೆ-ಮ-ರೆ-ಸಿ-ಕೊಂ-ಡಿದ್ದ
ತಲೆ-ಮಾ-ರಿನ
ತಲೆ-ಮಾ-ರಿ-ನ-ವರು
ತಲೆಯ
ತಲೆ-ಯನ್ನು
ತಲೆ-ಯ-ನ್ನೊಮ್ಮೆ
ತಲೆ-ಯಲ್ಲಿ
ತಲೆ-ಯ-ಲ್ಲಿವೆ
ತಲೆ-ಯಾ-ಡಿ-ಸಿ-ದ್ದಲ್ಲ
ತಲೆ-ಯಿ-ರಿಸಿ
ತಲೆ-ಯೆ-ತ್ತದೆ
ತಲೆ-ಯೆತ್ತಿ
ತಲೆ-ಯೊ-ಳಕ್ಕೆ
ತಲೆ-ಯೊ-ಳಗೆ
ತಲೆ-ಹ-ರಟೆ
ತಲೆ-ಹಾ-ಕು-ವಂ-ತಿಲ್ಲ
ತಲ್ಲ-ಣ-ದಿಂ
ತಳ-ಪಾಯ
ತಳ-ಪಾ-ಯ-ವನ್ನು
ತಳ-ಪಾ-ಯ-ವೆ-ಲ್ಲಿದೆ
ತಳ-ಮ-ಳಿ-ಸಿತು
ತಳ-ಮ-ಳಿ-ಸಿ-ದರು
ತಳ-ಹ-ದಿಯ
ತಳ-ಹ-ದಿ-ಯ-ನ್ನಾಗಿ
ತಳ-ಹ-ದಿ-ಯಾ-ಯಿತು
ತಳಿ-ರು-ತೋ-ರಣ
ತಳಿ-ರು-ತೋ-ರ-ಣ-ಗಳಿಂದ
ತಳಿ-ರು-ತೋ-ರ-ಣ-ಗಳು
ತಳೆದ
ತಳೆದು
ತಳ್ಳ
ತಳ್ಳ-ಲಾ-ಗ-ದಂ-ತಹ
ತಳ್ಳಿ
ತಳ್ಳಿ-ಕೊಂಡು
ತಳ್ಳಿ-ಹಾ-ಕಲು
ತಳ್ಳಿ-ಹಾಕಿ
ತಳ್ಳಿ-ಹಾ-ಕಿತು
ತಳ್ಳಿ-ಹಾ-ಕಿದ
ತಳ್ಳಿ-ಹಾ-ಕಿ-ದರು
ತಳ್ಳಿ-ಹಾ-ಕು-ತ್ತಾರೆ
ತವಕ
ತವ-ಕ-ವನ್ನು
ತವ-ರೂರು
ತವಾ-ಗಿವೆ
ತಸ್ಮೈ
ತಹಸೀ
ತಹ-ಸೀ-ಲ್ದಾ-ರರ
ತಹಾಂ
ತಾ
ತಾಂಡ-ವ-ವನ್ನು
ತಾಂಡ-ವ-ವಾ-ಡಿ-ದಾಗ
ತಾಂಡ-ವ-ವಾಡು
ತಾಂತ್ರಿ-ಕ-ವಾಗಿ
ತಾಂತ್ರಿ-ಕ-ವಿದ್ಯೆ
ತಾಕೀತು
ತಾಗ-ಬೇಕು
ತಾಗಿತು
ತಾಗು-ವುದು
ತಾಣ
ತಾಣ-ವನ್ನು
ತಾತನೇ
ತಾತ್ಕಾ-ಲಿಕ
ತಾತ್ಕಾ-ಲಿ-ಕ-ವಾಗಿ
ತಾತ್ತ್ವಿಕ
ತಾತ್ತ್ವಿ-ಕ-ಆಧ್ಯಾ
ತಾತ್ವಿಕ
ತಾದಾತ್ಮ್ಯ
ತಾದಾ-ತ್ಮ್ಯ-ಗೊ-ಳಿ-ಸಿ-ಕೊಂ-ಡ-ದ್ದರ
ತಾದಾ-ತ್ಮ್ಯ-ಗೊ-ಳಿ-ಸಿ-ಕೊಂಡೇ
ತಾದಾ-ತ್ಮ್ಯ-ವನ್ನು
ತಾದುವು
ತಾನ
ತಾನ-ಗಳು
ತಾನ-ದಕ್ಕೆ
ತಾನ-ದಲ್ಲಿ
ತಾನಾ-ಗಿ-ದ್ದಾ-ನೆಯೋ
ತಾನಾ-ಗಿಯೇ
ತಾನಾ-ಯಿತು
ತಾನಿನ್ನು
ತಾನೀಗ
ತಾನು
ತಾನೂ
ತಾನೆ
ತಾನೇ
ತಾನೇ-ತಾ-ನಾಗಿ
ತಾನೇ-ತಾ-ನಾ-ಗಿತ್ತು
ತಾನೊಂದು
ತಾನೊಂದೇ
ತಾನೊಬ್ಬ
ತಾಪ-ಗಳು
ತಾಭ್ಯಃ
ತಾಯ
ತಾಯಂ-ದಿರ
ತಾಯಿ
ತಾಯಿ-ಕೋಳಿ
ತಾಯಿ-ಕೋ-ಳಿಗೆ
ತಾಯಿಗೂ
ತಾಯಿಗೆ
ತಾಯಿತು
ತಾಯಿಯ
ತಾಯಿ-ಯನ್ನು
ತಾಯಿ-ಯ-ಲ್ಲ-ವೆಂ-ಬುದು
ತಾಯಿ-ಯಾದ
ತಾಯಿ-ಯೊಂ-ದಿಗೆ
ತಾಯೀ
ತಾಯೆ
ತಾಯ್ನಾ-ಡಾದ
ತಾಯ್ನಾ-ಡಿಗೆ
ತಾರಕ
ತಾರ-ತ-ಮ್ಯ-ದಿಂ-ದಾಗಿ
ತಾರ-ತ-ಮ್ಯ-ವಿ-ಲ್ಲದೆ
ತಾರ-ತ-ಮ್ಯವು
ತಾರದೆ
ತಾರ-ಸಿಯ
ತಾರ-ಸ್ವ-ರ-ದಲ್ಲಿ
ತಾರೀ-ಕಿನ
ತಾರೀ-ಕಿ-ನಂದು
ತಾರೀ-ಕಿ-ನಿಂದ
ತಾರೀಕು
ತಾರೀ-ಕೇನೋ
ತಾರು-ಣ್ಯದ
ತಾರು-ಣ್ಯ-ದಲ್ಲಿ
ತಾರು-ಣ್ಯ-ದಿಂ-ದಲೂ
ತಾರು-ಣ್ಯ-ಭ-ರಿ-ತ-ವಾದ
ತಾರೆ-ಗ-ಳನು
ತಾಲ್ಗೆ
ತಾಳ-ಮೃ-ದಂ-ಗ-ಗ-ಳೊಂ-ದಿಗೆ
ತಾಳ-ತ-ಪ್ಪಿ-ಹೋ-ಗಿತ್ತು
ತಾಳ-ಬ-ಹುದು
ತಾಳ-ಬ-ಹುದೆ
ತಾಳ-ಬೇ-ಕಾಗಿ
ತಾಳ-ಲಾರ
ತಾಳ-ಲಾ-ರದೆ
ತಾಳಲು
ತಾಳಿ
ತಾಳಿ-ಕೊಂ-ಡರೋ
ತಾಳಿ-ಕೊ-ಳ್ಳುವ
ತಾಳಿದ
ತಾಳಿ-ದರು
ತಾಳಿ-ದ-ವರು
ತಾಳಿ-ದ್ದರು
ತಾಳಿ-ದ್ದರೂ
ತಾಳಿ-ದ್ದೀರಿ
ತಾಳಿ-ದ್ದೇ-ನೆಂದು
ತಾಳು-ತ್ತಲೂ
ತಾಳು-ತ್ತೇನೆ
ತಾಳು-ವು-ದರ
ತಾಳು-ವು-ದ-ರಿಂದ
ತಾಳು-ವುದು
ತಾಳೆಯ
ತಾಳ್ಮೆ-ಗೆ-ಡದೆ
ತಾಳ್ಮೆ-ಯಿಂದ
ತಾವ-ದನ್ನು
ತಾವಲ್ಲಿ
ತಾವಾಗಿ
ತಾವಾ-ಗಿದ್ದು
ತಾವಾ-ಗಿಯೇ
ತಾವಾ-ಡಿದ
ತಾವಿಟ್ಟ
ತಾವಿನ್ನು
ತಾವಿ-ರು-ವ-ಲ್ಲಿಗೆ
ತಾವಿ-ಳಿದು
ತಾವೀಗ
ತಾವೀ-ಗಾ-ಗಲೇ
ತಾವು
ತಾವೂ
ತಾವೆಲ್ಲ
ತಾವೇ
ತಾವೇ-ತಾ-ವಾಗಿ
ತಾವೇನೂ
ತಾವೊಬ್ಬ
ತಾವೊ-ಬ್ಬರೇ
ತಾಸು
ತಿಂಗಳ
ತಿಂಗಳನ್ನು
ತಿಂಗಳಲ್ಲಿ
ತಿಂಗ-ಳಲ್ಲೇ
ತಿಂಗ-ಳಷ್ಟು
ತಿಂಗ-ಳಷ್ಟೇ
ತಿಂಗ-ಳಾ-ದರೂ
ತಿಂಗ-ಳಿ-ಗಿಂತ
ತಿಂಗ-ಳಿ-ನಲ್ಲಿ
ತಿಂಗ-ಳಿ-ನಿಂದ
ತಿಂಗ-ಳಿ-ನಿಂ-ದಲೂ
ತಿಂಗ-ಳಿ-ನಿಂ-ದಲೇ
ತಿಂಗ-ಳಿ-ರು-ವಾಗ
ತಿಂಗಳು
ತಿಂಗ-ಳು-ಗಳಲ್ಲಿ
ತಿಂಗಳೂ
ತಿಂಡಿ
ತಿಂಡಿಗೆ
ತಿಂಡಿ-ತೀ-ರ್ಥ-ಗಳ
ತಿಂಡಿಯ
ತಿಂದರೆ
ತಿಂದ-ವನ
ತಿಂದು
ತಿಂದುಂಡು
ತಿಂದು-ಕೊಂಡು
ತಿಂದೇ
ತಿಕ್ಕಾ-ಟ-ಗ-ಳುಂ-ಟಾ-ದುವು
ತಿಕ್ಕು
ತಿಕ್ಕು-ತ್ತೀಯಾ
ತಿತ್ತು
ತಿಥಿ-ಗಳು
ತಿದ್ದ
ತಿದ್ದ-ಲಾಗಿದೆ
ತಿದ್ದಲು
ತಿದ್ದಿ
ತಿದ್ದಿ-ಕೊ-ಳ್ಳಲು
ತಿದ್ದು-ವುದು
ತಿನ್ನಿ-ಸು-ವು-ದ-ಕ್ಕಾಗಿ
ತಿನ್ನು
ತಿನ್ನುತ್ತ
ತಿನ್ನು-ತ್ತಾನೆ
ತಿನ್ನು-ತ್ತೇನೆ
ತಿನ್ನು-ವಂತೆ
ತಿನ್ನು-ವ-ವರೋ
ತಿನ್ನು-ವುದನ್ನು
ತಿನ್ನು-ವು-ದ-ರಿಂದ
ತಿನ್ನು-ವುದು
ತಿಪ್ಪೇ-ಸ್ವಾಮಿ
ತಿರ-ಸ್ಕ-ರಿ-ಸ-ಬೇ-ಕಾ-ಗಿಲ್ಲ
ತಿರ-ಸ್ಕ-ರಿಸಿ
ತಿರ-ಸ್ಕ-ರಿ-ಸಿ-ದ್ದಾರೆ
ತಿರ-ಸ್ಕ-ರಿ-ಸಿ-ಬಿ-ಟ್ಟರೆ
ತಿರ-ಸ್ಕಾರ
ತಿರ-ಸ್ಕಾ-ರಕ್ಕೆ
ತಿರ-ಸ್ಕಾ-ರ-ಗಳನ್ನು
ತಿರ-ಸ್ಕಾ-ರ-ದಿಂದ
ತಿರ-ಸ್ಕಾ-ರಾರ್ಥ
ತಿರಿ-ಚುತ್ತ
ತಿರಿ-ಚುವ
ತಿರುಕ
ತಿರು-ಕ-ನಾ-ಗಿಯೇ
ತಿರು-ಗಾ-ಡಿ-ಕೊಂಡು
ತಿರು-ಗಾ-ಡು-ವು-ದ-ರಿಂ-ದೇನು
ತಿರುಗಿ
ತಿರು-ಗಿ-ಕೊಂಡ
ತಿರು-ಗಿ-ಕೊಂ-ಡಾಗ
ತಿರು-ಗಿ-ಕೊಂ-ಡಿತು
ತಿರು-ಗಿ-ಕೊಂಡು
ತಿರು-ಗಿತು
ತಿರು-ಗಿತ್ತು
ತಿರು-ಗಿ-ದರೂ
ತಿರು-ಗಿ-ಬೀ-ಳು-ವು-ದಕ್ಕೆ
ತಿರು-ಗಿ-ಸುವ
ತಿರುಗು
ತಿರು-ಗು-ತ್ತ-ದೆಯೋ
ತಿರು-ಗು-ತ್ತಿದೆ
ತಿರು-ಗು-ತ್ತಿ-ರು-ವುದನ್ನು
ತಿರು-ಗು-ವಂತೆ
ತಿರು-ಚನಾ
ತಿರು-ಚ-ನಾ-ಪಳ್ಳಿ
ತಿರು-ಚ-ನಾ-ಪ-ಳ್ಳಿ-ಯ-ವರ
ತಿರುಚಿ
ತಿರು-ನಾಮ
ತಿರು-ಪ್ಪ-ತ್ತೂ-ರಿನ
ತಿರು-ಳನ್ನು
ತಿರು-ಳಿನ
ತಿರುಳು
ತಿರು-ವ-ನಂತ
ತಿರು-ವ-ನಂ-ತ-ಪು-ರ-ದಲ್ಲಿ
ತಿರು-ವನ್ನು
ತಿರು-ವಾಂ-ಕೂ-ರಿನ
ತಿರುವು
ತಿಲ-ಕರ
ತಿಲ-ಕ-ರನ್ನು
ತಿಲ-ಕರು
ತಿಲ-ಕರೂ
ತಿಲ್ಲ
ತಿಳಿ
ತಿಳಿದ
ತಿಳಿ-ದಂ-ತಿಲ್ಲ
ತಿಳಿ-ದಂತೆ
ತಿಳಿ-ದದ್ದು
ತಿಳಿ-ದ-ವರು
ತಿಳಿ-ದಾಗ
ತಿಳಿ-ದಿತ್ತು
ತಿಳಿ-ದಿ-ತ್ತು-ಜ-ನರು
ತಿಳಿ-ದಿದೆ
ತಿಳಿ-ದಿ-ದೆ-ನಾನು
ತಿಳಿ-ದಿ-ದೆಯೆ
ತಿಳಿ-ದಿ-ದೆಯೋ
ತಿಳಿ-ದಿದ್ದ
ತಿಳಿ-ದಿ-ದ್ದರು
ತಿಳಿ-ದಿ-ದ್ದರೂ
ತಿಳಿ-ದಿ-ದ್ದಳು
ತಿಳಿ-ದಿ-ದ್ದಾ-ರಾ-ದರೂ
ತಿಳಿ-ದಿ-ದ್ದೇನೆ
ತಿಳಿ-ದಿರ
ತಿಳಿ-ದಿ-ರ-ಬ-ಹು-ದಾದ
ತಿಳಿ-ದಿ-ರ-ಬೇ-ಕು-ಅವು
ತಿಳಿ-ದಿ-ರಲಿ
ತಿಳಿ-ದಿ-ರ-ಲಿಲ್ಲ
ತಿಳಿ-ದಿ-ರ-ಲಿ-ಲ್ಲ-ವೆಂದು
ತಿಳಿ-ದಿ-ರಲೇ
ತಿಳಿ-ದಿರಾ
ತಿಳಿ-ದಿರಿ
ತಿಳಿ-ದಿ-ರುವ
ತಿಳಿ-ದಿ-ರು-ವಂತೆ
ತಿಳಿ-ದಿ-ರು-ವಂ-ಥದೇ
ತಿಳಿ-ದಿ-ರು-ವಿರೋ
ತಿಳಿ-ದಿ-ರು-ವು-ದಿಲ್ಲ
ತಿಳಿ-ದಿ-ರು-ವುದು
ತಿಳಿ-ದಿ-ರು-ವೆ-ನಿನ್ನ
ತಿಳಿ-ದಿಲ್ಲ
ತಿಳಿದು
ತಿಳಿ-ದುಕೊ
ತಿಳಿ-ದು-ಕೊಂಡಂ
ತಿಳಿ-ದು-ಕೊಂ-ಡ-ದ್ದನ್ನು
ತಿಳಿ-ದು-ಕೊಂ-ಡದ್ದು
ತಿಳಿ-ದು-ಕೊಂ-ಡರು
ತಿಳಿ-ದು-ಕೊಂ-ಡರೆ
ತಿಳಿ-ದು-ಕೊಂ-ಡಾಗ
ತಿಳಿ-ದು-ಕೊಂ-ಡಿ-ದ್ದಂತೆ
ತಿಳಿ-ದು-ಕೊಂ-ಡಿ-ದ್ದ-ವನು
ತಿಳಿ-ದು-ಕೊಂ-ಡಿ-ದ್ದೀರಿ
ತಿಳಿ-ದು-ಕೊಂ-ಡಿದ್ದೆ
ತಿಳಿ-ದು-ಕೊಂ-ಡಿರಿ
ತಿಳಿ-ದು-ಕೊಂಡು
ತಿಳಿ-ದು-ಕೊಂಡೂ
ತಿಳಿ-ದು-ಕೊಂಡೆ
ತಿಳಿ-ದು-ಕೊ-ಆ-ತ-ನ-ಲ್ಲಿ-ರು-ವುದು
ತಿಳಿ-ದು-ಕೊ-ಜ-ನ್ಮತಃ
ತಿಳಿ-ದು-ಕೊಳ್ಳ
ತಿಳಿ-ದು-ಕೊ-ಳ್ಳದೆ
ತಿಳಿ-ದು-ಕೊ-ಳ್ಳದೇ
ತಿಳಿ-ದು-ಕೊ-ಳ್ಳ-ಬಲ್ಲೆ
ತಿಳಿ-ದು-ಕೊ-ಳ್ಳ-ಬೇಕು
ತಿಳಿ-ದು-ಕೊ-ಳ್ಳಲು
ತಿಳಿ-ದು-ಕೊ-ಳ್ಳ-ಲೆಂದು
ತಿಳಿ-ದು-ಕೊ-ಳ್ಳ-ಲೆಂದೇ
ತಿಳಿ-ದು-ಕೊಳ್ಳಿ
ತಿಳಿ-ದು-ಕೊ-ಳ್ಳಿ-ಅದು
ತಿಳಿ-ದು-ಕೊ-ಳ್ಳಿ-ನಿ-ಮ್ಮಲ್ಲಿ
ತಿಳಿ-ದು-ಕೊ-ಳ್ಳು-ತ್ತಿ-ದ್ದರು
ತಿಳಿ-ದು-ಕೊ-ಳ್ಳುವ
ತಿಳಿ-ದು-ಕೊ-ಳ್ಳು-ವ-ವರೂ
ತಿಳಿ-ದು-ಕೊ-ಳ್ಳು-ವುದು
ತಿಳಿ-ದುದೇ
ತಿಳಿ-ದು-ಬಂತು
ತಿಳಿ-ದು-ಬಂದ
ತಿಳಿ-ದು-ಬಂ-ದಾಗ
ತಿಳಿ-ದು-ಬಂ-ದಿತು
ತಿಳಿ-ದು-ಬಂ-ದಿದೆ
ತಿಳಿ-ದು-ಬಂ-ದಿ-ದ್ದರೂ
ತಿಳಿ-ದು-ಬಂ-ದಿ-ರ-ಬೇಕು
ತಿಳಿ-ದು-ಬಂ-ದಿ-ರ-ಲಿಲ್ಲ
ತಿಳಿ-ದು-ಬಂ-ದಿರು
ತಿಳಿ-ದು-ಬಂ-ದುವು
ತಿಳಿ-ದು-ಬ-ರು-ತ್ತದೆ
ತಿಳಿ-ದು-ಬ-ರು-ತ್ತ-ದೆ-ಬೆ-ಳಿಗ್ಗೆ
ತಿಳಿ-ದು-ಬ-ರು-ತ್ತಿ-ದ್ದುವು
ತಿಳಿ-ದು-ಬ-ರು-ತ್ತಿವೆ
ತಿಳಿ-ದು-ಬಿ-ಡು-ತ್ತಿತ್ತು
ತಿಳಿದೇ
ತಿಳಿದೋ
ತಿಳಿ-ನೀ-ರಿನ
ತಿಳಿ-ಯ-ಗೊ-ಡದೆ
ತಿಳಿ-ಯದ
ತಿಳಿ-ಯ-ದಿ-ರಲಿ
ತಿಳಿ-ಯದು
ತಿಳಿ-ಯ-ದು-ಅದು
ತಿಳಿ-ಯ-ದೆಯೋ
ತಿಳಿ-ಯದ್ದೆ
ತಿಳಿ-ಯ-ಪ-ಡಿ-ಸಲು
ತಿಳಿ-ಯ-ಪ-ಡಿ-ಸಿ-ದರು
ತಿಳಿ-ಯ-ಪ-ಡಿ-ಸು-ತ್ತಿ-ದ್ದರು
ತಿಳಿ-ಯ-ಬ-ಲ್ಲರು
ತಿಳಿ-ಯ-ಬಲ್ಲೆ
ತಿಳಿ-ಯ-ಬ-ಹು-ದಾ-ಗಿದೆ
ತಿಳಿ-ಯ-ಬ-ಹುದು
ತಿಳಿ-ಯ-ಬಾ-ರ-ದಷ್ಟು
ತಿಳಿ-ಯ-ಬೇಕು
ತಿಳಿ-ಯ-ಬೇಡ
ತಿಳಿ-ಯ-ಬೇಡಿ
ತಿಳಿ-ಯಲಿ
ತಿಳಿ-ಯ-ಲಿಲ್ಲ
ತಿಳಿ-ಯಲು
ತಿಳಿ-ಯಾ-ಗಿದೆ
ತಿಳಿ-ಯಾ-ಗಿ-ರು-ವಂತೆ
ತಿಳಿ-ಯಾ-ದ-ನಿ-ರ್ಮ-ಲ-ವಾದ
ತಿಳಿ-ಯಿತು
ತಿಳಿ-ಯಿ-ತು-ಅ-ವ-ರಿ-ಬ್ಬರೂ
ತಿಳಿ-ಯಿ-ತು-ಕ-ಲ್ಕ-ತ್ತ-ದಲ್ಲಿ
ತಿಳಿ-ಯಿರಿ
ತಿಳಿಯು
ತಿಳಿ-ಯು-ತ್ತದೆ
ತಿಳಿ-ಯು-ತ್ತ-ದೆ-ಉ-ಪ-ನಿ-ಷ-ತ್ತು-ಗ-ಳ-ಲ್ಲದೆ
ತಿಳಿ-ಯು-ತ್ತಲೇ
ತಿಳಿ-ಯು-ತ್ತಾರೆ
ತಿಳಿ-ಯು-ತ್ತಿ-ತ್ತು-ಅ-ವರು
ತಿಳಿ-ಯು-ತ್ತಿ-ತ್ತುಈ
ತಿಳಿ-ಯು-ತ್ತಿ-ದ್ದಂ-ತೆಯೇ
ತಿಳಿ-ಯು-ತ್ತಿ-ದ್ದರು
ತಿಳಿ-ಯು-ತ್ತಿಲ್ಲ
ತಿಳಿ-ಯು-ತ್ತೇನೆ
ತಿಳಿ-ಯುವ
ತಿಳಿ-ಯು-ವಂ-ತಾ-ಗಿದೆ
ತಿಳಿ-ಯು-ವಂ-ತಿಲ್ಲ
ತಿಳಿ-ಯು-ವನೊ
ತಿಳಿ-ಯು-ವು-ದಿಲ್ಲ
ತಿಳಿ-ಯು-ವುದು
ತಿಳಿ-ವ-ಳಿಕೆ
ತಿಳಿ-ವ-ಳಿ-ಕೆ-ಯನ್ನು
ತಿಳಿ-ವ-ಳಿ-ಕೆ-ಯಿ-ಲ್ಲದ
ತಿಳಿ-ವ-ಳಿ-ಕೆ-ಯಿ-ಲ್ಲ-ದ-ವ-ರಿಗೆ
ತಿಳಿ-ವ-ಳಿ-ಕೆ-ಯಿ-ಲ್ಲ-ವೆಂ-ದರ್ಥ
ತಿಳಿ-ಸ-ಬ-ಹು-ದೆಂದು
ತಿಳಿ-ಸ-ಬೇ-ಕೆಂದು
ತಿಳಿ-ಸ-ಲಾ-ಯಿತು
ತಿಳಿ-ಸ-ಲಿಲ್ಲ
ತಿಳಿ-ಸ-ಲಿ-ಲ್ಲವೆ
ತಿಳಿ-ಸಲು
ತಿಳಿಸಿ
ತಿಳಿ-ಸಿ-ಕೊ-ಟ್ಟರು
ತಿಳಿ-ಸಿ-ಕೊ-ಟ್ಟ-ವರು
ತಿಳಿ-ಸಿ-ಕೊ-ಟ್ಟಿ-ದ್ದಾರೆ
ತಿಳಿ-ಸಿ-ಕೊಟ್ಟು
ತಿಳಿ-ಸಿ-ಕೊಡ
ತಿಳಿ-ಸಿ-ಕೊ-ಡ-ಬೇಕು
ತಿಳಿ-ಸಿ-ಕೊ-ಡಲು
ತಿಳಿ-ಸಿ-ಕೊಡಿ
ತಿಳಿ-ಸಿ-ಕೊಡು
ತಿಳಿ-ಸಿ-ಕೊ-ಡು-ತ್ತಿ-ದ್ದಾರೆ
ತಿಳಿ-ಸಿ-ಕೊ-ಡು-ತ್ತೇನೆ
ತಿಳಿ-ಸಿ-ಕೊ-ಡು-ವು-ದು-ಇ-ವು-ಗ-ಳಿ-ಗಾಗಿ
ತಿಳಿ-ಸಿದ
ತಿಳಿ-ಸಿ-ದಂತೆ
ತಿಳಿ-ಸಿ-ದರು
ತಿಳಿ-ಸಿ-ದಳು
ತಿಳಿ-ಸಿ-ದಾಗ
ತಿಳಿ-ಸಿ-ದುದೂ
ತಿಳಿ-ಸಿದೆ
ತಿಳಿ-ಸಿದ್ದ
ತಿಳಿ-ಸಿ-ದ್ದ-ರ-ಲ್ಲದೆ
ತಿಳಿ-ಸಿ-ದ್ದ-ರ-ಲ್ಲವೆ
ತಿಳಿ-ಸಿ-ದ್ದರು
ತಿಳಿ-ಸಿ-ದ್ದಳು
ತಿಳಿ-ಸಿ-ಬಿ-ಡ-ಬೇ-ಕೆಂದು
ತಿಳಿ-ಸಿ-ಬಿಡಿ
ತಿಳಿ-ಸಿ-ರ-ಲಿಲ್ಲ
ತಿಳಿ-ಸಿ-ರುವೆ
ತಿಳಿಸು
ತಿಳಿ-ಸುತ್ತ
ತಿಳಿ-ಸು-ತ್ತದೆ
ತಿಳಿ-ಸು-ತ್ತಾ-ನೆ-ನಾವು
ತಿಳಿ-ಸು-ತ್ತಾರೆ
ತಿಳಿ-ಸು-ತ್ತಾಳೆ
ತಿಳಿ-ಸು-ತ್ತಿ-ದ್ದರು
ತಿಳಿ-ಸು-ತ್ತಿ-ದ್ದಳು
ತಿಳಿ-ಸು-ತ್ತೇನೆ
ತಿಳಿ-ಸುವ
ತಿಳಿ-ಸು-ವಂ-ತಹ
ತಿಳಿ-ಸು-ವಂತೆ
ತಿಳಿ-ಸು-ವಾಗ
ತಿಳಿ-ಸುವು
ತಿಳಿ-ಸು-ವುದನ್ನು
ತಿಳಿ-ಸು-ವುದು
ತಿಳಿ-ಸು-ವು-ವಾ-ದರೂ
ತಿಳಿ-ಸೋಣ
ತೀಕ್ಷ
ತೀಕ್ಷ್ಣ
ತೀಕ್ಷ್ಣತೆ
ತೀಕ್ಷ್ಣ-ತೆ-ಯನ್ನು
ತೀಕ್ಷ್ಣ-ವಾಗಿ
ತೀಕ್ಷ್ಣ-ವಾದ
ತೀಯ
ತೀರ
ತೀರಕ್ಕೆ
ತೀರಕ್ಕೋ
ತೀರದ
ತೀರ-ದಲ್ಲಿ
ತೀರ-ದ-ಲ್ಲಿದ್ದ
ತೀರ-ದಿಂದ
ತೀರ-ಪ್ರ-ದೇ-ಶ-ದಲ್ಲಿ
ತೀರ-ವನ್ನು
ತೀರಾ
ತೀರಿ-ಕೊಂಡ
ತೀರಿ-ಕೊಂ-ಡರು
ತೀರಿ-ಕೊಂ-ಡ-ರೆಂದು
ತೀರಿ-ಕೊಂ-ಡ-ರೆಂಬ
ತೀರಿ-ಕೊಂ-ಡಿ-ದ್ದ-ರಿಂದ
ತೀರಿ-ಕೊಂ-ಡಿ-ದ್ದಳು
ತೀರಿ-ಕೊಂಡು
ತೀರಿ-ಕೊ-ಳ್ಳುವ
ತೀರಿ-ಸ-ಲಾ-ಗದೆ
ತೀರಿ-ಸ-ಲಾರೆ
ತೀರಿ-ಸಲು
ತೀರಿ-ಸಿ-ದಂ-ತಾ-ದೀ-ತೇನು
ತೀರಿ-ಹೋ-ಗಿ-ದ್ದರೂ
ತೀರಿ-ಹೋ-ಗು-ವು-ದಕ್ಕೆ
ತೀರಿ-ಹೋದ
ತೀರಿ-ಹೋ-ದಂತೆ
ತೀರಿ-ಹೋ-ದರೂ
ತೀರಿ-ಹೋ-ದರೆ
ತೀರು-ತ್ತದೆ
ತೀರು-ತ್ತಾ-ರೆಂ-ಬುದು
ತೀರು-ವಂತೆ
ತೀರು-ವುದು
ತೀರ್ಥ
ತೀರ್ಥ-ಕ್ಷೇತ್ರ
ತೀರ್ಥ-ಕ್ಷೇ-ತ್ರ-ಗಳ
ತೀರ್ಥ-ಕ್ಷೇ-ತ್ರ-ಗಳನ್ನು
ತೀರ್ಥ-ಕ್ಷೇ-ತ್ರ-ಗ-ಳ-ಲ್ಲಿ-ಸ್ವಾ-ಮೀ-ಜಿ-ಯ-ವ-ರಿಗೆ
ತೀರ್ಥ-ಕ್ಷೇ-ತ್ರ-ಗ-ಳ-ಲ್ಲೊಂದು
ತೀರ್ಥ-ಕ್ಷೇ-ತ್ರ-ಗ-ಳಿಗೆ
ತೀರ್ಥ-ಕ್ಷೇ-ತ್ರ-ಗ-ಳಿವೆ
ತೀರ್ಥ-ಕ್ಷೇ-ತ್ರದ
ತೀರ್ಥ-ಕ್ಷೇ-ತ್ರ-ವನ್ನು
ತೀರ್ಥ-ಕ್ಷೇ-ತ್ರ-ವಾದ
ತೀರ್ಥ-ಕ್ಷೇ-ತ್ರವೇ
ತೀರ್ಥ-ಗ-ಳಲ್ಲೂ
ತೀರ್ಥ-ಯಾ-ತ್ರಿ-ಕನ
ತೀರ್ಥ-ಯಾ-ತ್ರಿ-ಕರ
ತೀರ್ಥ-ಯಾತ್ರೆ
ತೀರ್ಥ-ಯಾ-ತ್ರೆಗೆ
ತೀರ್ಥ-ಯಾ-ತ್ರೆ-ಯಲ್ಲಿ
ತೀರ್ಥ-ಯಾ-ತ್ರೆ-ಯಿಂದ
ತೀರ್ಥ-ಯಾ-ತ್ರೆಯೂ
ತೀರ್ಥ-ರಾಮ
ತೀರ್ಥ-ರಾ-ಮರ
ತೀರ್ಥ-ರಾ-ಮ-ರಿದ್ದ
ತೀರ್ಥ-ರಾ-ಮರು
ತೀರ್ಥ-ರಾ-ಮರೂ
ತೀರ್ಥ-ವನ್ನು
ತೀರ್ಪು
ತೀರ್ಮಾನ
ತೀರ್ಮಾ-ನಕ್ಕೆ
ತೀರ್ಮಾ-ನ-ಗಳ
ತೀರ್ಮಾ-ನ-ಗಳನ್ನು
ತೀರ್ಮಾ-ನ-ವಾ-ಗಿ-ತ್ತಾ-ದರೂ
ತೀರ್ಮಾ-ನ-ವಾ-ದಾ-ಗಲೇ
ತೀರ್ಮಾ-ನ-ವಾ-ಯಿತು
ತೀರ್ಮಾ-ನಿಸಿ
ತೀರ್ಮಾ-ನಿ-ಸಿ-ದಳು
ತೀರ್ಮಾ-ನಿ-ಸಿ-ದ-ವರು
ತೀರ್ಮಾ-ನಿ-ಸಿ-ದ್ದರು
ತೀರ್ಮಾ-ನಿ-ಸಿ-ಬಿ-ಟ್ಟಿ-ದ್ದರು
ತೀವ್ರ
ತೀವ್ರ-ಗೊಂಡು
ತೀವ್ರ-ಗೊ-ಳಿಸಿ
ತೀವ್ರ-ಜ್ವ-ರ-ದಿಂದ
ತೀವ್ರ-ತರ
ತೀವ್ರ-ತೆ-ಯನ್ನು
ತೀವ್ರ-ವಾಗಿ
ತೀವ್ರ-ವಾ-ಗಿತ್ತು
ತೀವ್ರ-ವಾ-ಗಿ-ದ್ದರ
ತೀವ್ರ-ವಾ-ಗಿ-ಬಿ-ಟ್ಟಿತು
ತೀವ್ರ-ವಾ-ಗಿ-ರ-ಲಿಲ್ಲ
ತೀವ್ರ-ವಾದ
ತೀವ್ರ-ವಾ-ಯಿತು
ತುಂಟ
ತುಂಟ-ತ-ನದ
ತುಂಡನ್ನು
ತುಂಡ-ರಿ-ಸಿ-ದರೂ
ತುಂಡು
ತುಂಬ
ತುಂಬ-ಬೇಕು
ತುಂಬ-ಲಾ-ರದ
ತುಂಬಲು
ತುಂಬಿ
ತುಂಬಿ-ಕೊಂಡ
ತುಂಬಿ-ಕೊಂ-ಡಂತೆ
ತುಂಬಿ-ಕೊಂ-ಡರೋ
ತುಂಬಿ-ಕೊಂ-ಡಿತು
ತುಂಬಿ-ಕೊಂ-ಡಿದೆ
ತುಂಬಿ-ಕೊಂ-ಡಿ-ದ್ದುವು
ತುಂಬಿ-ಕೊಂ-ಡಿ-ರದ
ತುಂಬಿ-ಕೊಂಡು
ತುಂಬಿ-ಕೊ-ಳ್ಳಲೂ
ತುಂಬಿತು
ತುಂಬಿತ್ತು
ತುಂಬಿದ
ತುಂಬಿ-ದರು
ತುಂಬಿದೆ
ತುಂಬಿದ್ದ
ತುಂಬಿ-ದ್ದಂತೆ
ತುಂಬಿ-ದ್ದ-ರಿಂ-ದಲೇ
ತುಂಬಿ-ದ್ದರು
ತುಂಬಿ-ದ್ದುದು
ತುಂಬಿ-ದ್ದುವು
ತುಂಬಿ-ಬಂತು
ತುಂಬಿ-ಬಂ-ದಿತು
ತುಂಬಿ-ಬಂದು
ತುಂಬಿ-ರಲಿ
ತುಂಬಿ-ರು-ತ್ತಿತ್ತು
ತುಂಬಿ-ರುವ
ತುಂಬಿವೆ
ತುಂಬಿ-ಸಲು
ತುಂಬಿ-ಸಿ-ದರು
ತುಂಬಿಸು
ತುಂಬಿ-ಸುವ
ತುಂಬಿಹ
ತುಂಬಿ-ಹೋ-ಗಿತ್ತು
ತುಂಬಿ-ಹೋ-ಗಿ-ದ್ದರು
ತುಂಬಿ-ಹೋ-ಗಿ-ರುವ
ತುಂಬಿ-ಹೋ-ಯಿತು
ತುಂಬು-ತ್ತಿ-ದ್ದರು
ತುಂಬು-ತ್ತಿ-ದ್ದುವು
ತುಂಬು-ತ್ತೇನೆ
ತುಂಬುವ
ತುಂಬು-ವಂ-ತಹ
ತುಂಬು-ವುದು
ತುಕ್ಕು
ತುಚ್ಛ-ವಾಗಿ
ತುಚ್ಛ-ವಾದ
ತುಚ್ಛ-ವಾ-ದ-ದ್ದಲ್ಲ
ತುಚ್ಛ-ವಾ-ದ-ದ್ದೇ-ನಲ್ಲ
ತುಚ್ಛೀ-ಕ-ರಿ-ಸಿತ್ತು
ತುಚ್ಛೀ-ಕ-ರಿಸು
ತುಟಿ
ತುಟಿ-ಗಳಿಂದ
ತುಟಿ-ಪಿ-ಟ-ಕ್ಕೆ-ನ್ನದೆ
ತುಡಿ-ಯು-ತ್ತಿತ್ತು
ತುಣು-ಕನ್ನು
ತುತ್ತ
ತುತ್ತ-ತುದಿ
ತುತ್ತ-ತು-ದಿ-ಯ-ಲ್ಲಿದ್ದ
ತುತ್ತ-ನ್ನಿ-ಡುವು
ತುತ್ತಾ-ಗ-ಬೇ-ಕಾ-ಗಿತ್ತು
ತುತ್ತಾ-ಗಿತ್ತು
ತುತ್ತಾದ
ತುತ್ತಾ-ಯಿತು
ತುತ್ತಿಗೆ
ತುತ್ತಿಗೇ
ತುತ್ತು
ತುದಿ-ಗಾ-ಲಿ-ನಲ್ಲಿ
ತುದಿಗೆ
ತುದಿ-ಭಾ-ಗದ
ತುದಿಯ
ತುದಿ-ಯನ್ನು
ತುದಿ-ಯ-ವ-ರೆಗೂ
ತುದಿ-ಯಿಂದ
ತುಪ್ಪ
ತುಮುಲ
ತುಮು-ಲದ
ತುಮು-ಲ-ವಿ-ಲ್ಲ-ದಂ-ತಾ-ಗ-ಬೇ-ಕಾ-ದರೆ
ತುರೀಯಾ
ತುರೀ-ಯಾ-ನಂದ
ತುರೀ-ಯಾ-ನಂ-ದ-ಇ-ವರು
ತುರೀ-ಯಾ-ನಂ-ದರ
ತುರೀ-ಯಾ-ನಂ-ದ-ರನ್ನು
ತುರೀ-ಯಾ-ನಂ-ದ-ರನ್ನೂ
ತುರೀ-ಯಾ-ನಂ-ದ-ರಿಗೂ
ತುರೀ-ಯಾ-ನಂ-ದ-ರಿಗೆ
ತುರೀ-ಯಾ-ನಂ-ದರು
ತುರೀ-ಯಾ-ನಂ-ದರೂ
ತುರೀ-ಯಾ-ನಂ-ದ-ರೊ-ಡನೆ
ತುರೀ-ಯಾ-ನು-ಭವ
ತುರೀ-ಯಾ-ವಸ್ಥೆ
ತುರೀ-ಯಾ-ವ-ಸ್ಥೆ-ಯೊಂದೇ
ತುರು-ಕ-ಲಾಗಿದೆ
ತುರು-ಕ-ಲ್ಪಟ್ಟ
ತುರುಕಿ
ತುರ್ತಿನ
ತುರ್ತು
ತುಲ-ನಾ-ತ್ಮ-ಕ-ವಾಗಿ
ತುಲ-ಸೀ-ದಾ-ಸರ
ತುಳಿ-ತ-ಕ್ಕೊ-ಳ-ಗಾದ
ತುಳಿದು
ತುಳಿ-ಯಲು
ತುಳಿ-ಯು-ತ್ತಿ-ರು-ವುದು
ತುಳು-ಕಾ-ಡುವ
ತುಳು-ಕು-ತ್ತಿತ್ತು
ತುಷ-ಹೊಟ್ಟು
ತುಷಾಗ್ನಿ
ತುಷಾ-ಗ್ನಿ-ಯನ್ನು
ತುಷಾರ
ತುಸು
ತೂಕ-ಡಿ-ಸಲು
ತೂಗಾ-ಟ-ದಲ್ಲಿ
ತೂಗಿದ
ತೂಗು
ತೂಗು-ತ್ತಿವೆ
ತೂಗುವ
ತೂಗು-ಹಾಕಿ
ತೂಗು-ಹಾ-ಕಿ-ದ್ದಳು
ತೂತು
ತೂರಾಟ
ತೂರಾ-ಡ-ಲಾ-ರಂ-ಭಿ-ಸಿತು
ತೂರಿ
ತೂರಿ-ಕೊಂಡು
ತೂರಿ-ಹೋ-ಗಲಾ
ತೃಣ-ಮಾತ್ರ
ತೃಣ-ಮಾ-ತ್ರ-ವಾ-ದರೂ
ತೃಪ್ತ-ನಾ-ಗು-ತ್ತೇನೆ
ತೃಪ್ತ-ರಾ-ಗದೆ
ತೃಪ್ತ-ರಾ-ಗ-ಬೇಕು
ತೃಪ್ತಿ
ತೃಪ್ತಿ-ಸಂ-ತೋ-ಷ-ಗ-ಳ-ನ್ನುಂ-ಟು-ಮಾ-ಡಿತು
ತೃಪ್ತಿ-ಕರ
ತೃಪ್ತಿ-ಕ-ರ-ವಾಗಿ
ತೃಪ್ತಿ-ಪ-ಡಿ-ಸ-ಬೇಕು
ತೃಪ್ತಿ-ಯಾ-ಗದೆ
ತೃಪ್ತಿ-ಯಾ-ಗ-ಲಿಲ್ಲ
ತೃಪ್ತಿ-ಯಾಗಿ
ತೃಪ್ತಿ-ಯಾ-ಗಿದೆ
ತೃಪ್ತಿ-ಯಾ-ಗುವ
ತೃಪ್ತಿ-ಯಾ-ಗು-ವ-ವ-ರೆಗೂ
ತೃಪ್ತಿ-ಯಾ-ಗು-ವಷ್ಟು
ತೃಪ್ತಿ-ಯಾ-ದರೂ
ತೃಪ್ತಿ-ಯಿ-ರ-ಲಿಲ್ಲ
ತೃಪ್ತಿ-ಯಿಲ್ಲ
ತೃಪ್ತಿ-ಯುಂ-ಟು-ಮಾ-ಡಿತು
ತೆಂಗ-ಲೆಯ
ತೆಂಗಿನ
ತೆಂದರೆ
ತೆಂಬಂತೆ
ತೆಕ್ಕೆ-ಯಲ್ಲಿ
ತೆಗ-ಳಲಿ
ತೆಗೆದ
ತೆಗೆ-ದರೆ
ತೆಗೆ-ದಾ-ಯಿ-ತಲ್ಲ
ತೆಗೆ-ದಿ-ರಿ-ಸಿ-ಕೊ-ಳ್ಳ-ಬೇಕು
ತೆಗೆದು
ತೆಗೆ-ದುಕೊ
ತೆಗೆ-ದು-ಕೊಂಡ
ತೆಗೆ-ದು-ಕೊಂ-ಡರು
ತೆಗೆ-ದು-ಕೊಂ-ಡಾ-ಗಿತ್ತು
ತೆಗೆ-ದು-ಕೊಂಡು
ತೆಗೆ-ದು-ಕೊ-ಟ್ಟಂತೆ
ತೆಗೆ-ದು-ಕೊ-ಡಲು
ತೆಗೆ-ದು-ಕೊಳ್ಳ
ತೆಗೆ-ದು-ಕೊ-ಳ್ಳ-ದಿ-ದ್ದರೆ
ತೆಗೆ-ದು-ಕೊ-ಳ್ಳದೆ
ತೆಗೆ-ದು-ಕೊ-ಳ್ಳ-ಬೇ-ಕಾ-ಗಿಯೇ
ತೆಗೆ-ದು-ಕೊ-ಳ್ಳ-ಬೇ-ಕೆಂದು
ತೆಗೆ-ದು-ಕೊ-ಳ್ಳ-ಲಾ-ಯಿತು
ತೆಗೆ-ದು-ಕೊ-ಳ್ಳ-ಲಿಲ್ಲ
ತೆಗೆ-ದು-ಕೊ-ಳ್ಳಲು
ತೆಗೆ-ದು-ಕೊಳ್ಳಿ
ತೆಗೆ-ದು-ಕೊಳ್ಳು
ತೆಗೆ-ದು-ಕೊ-ಳ್ಳು-ತ್ತಾರೆ
ತೆಗೆ-ದು-ಕೊ-ಳ್ಳು-ತ್ತಿತ್ತು
ತೆಗೆ-ದು-ಕೊ-ಳ್ಳು-ತ್ತಿ-ದ್ದರು
ತೆಗೆ-ದು-ಕೊ-ಳ್ಳು-ತ್ತೇನೆ
ತೆಗೆ-ದು-ಕೊ-ಳ್ಳುವ
ತೆಗೆ-ದು-ಕೊ-ಳ್ಳು-ವಂತೆ
ತೆಗೆ-ದು-ಕೊ-ಳ್ಳು-ವು-ದ-ಕ್ಕಾ-ಗಿಯೇ
ತೆಗೆ-ದು-ಕೊ-ಳ್ಳು-ವು-ದ-ಕ್ಕಿಂತ
ತೆಗೆ-ದು-ಕೊ-ಳ್ಳೋಣ
ತೆಗೆ-ದು-ಹಾ-ಕಲು
ತೆಗೆ-ಯ-ಲಾ-ಯಿತು
ತೆಗೆ-ಯಲು
ತೆಗೆಸಿ
ತೆತ್ತರು
ತೆತ್ತು
ತೆಪ್ಪ-ಗಿ-ರಲು
ತೆರದಿ
ತೆರ-ನಾದ
ತೆರ-ಳ-ಬೇ-ಕಾಗಿ
ತೆರ-ಳ-ಬೇ-ಕಾ-ಯಿತು
ತೆರ-ಳ-ಬೇ-ಕೆಂಬ
ತೆರಳಿ
ತೆರ-ಳಿದ
ತೆರ-ಳಿ-ದರು
ತೆರ-ಳಿದ್ದ
ತೆರ-ಳಿ-ದ್ದಳು
ತೆರ-ಳುವ
ತೆರ-ಳು-ವ-ವ-ರೆಗೂ
ತೆರ-ಳು-ವು-ದಕ್ಕೆ
ತೆರ-ಳು-ವು-ದೆಂದು
ತೆರಿಗೆ
ತೆರಿ-ಗೆ-ಯಿಂದ
ತೆರೆದ
ತೆರೆ-ದರು
ತೆರೆ-ದಿಟ್ಟ
ತೆರೆ-ದಿ-ಟ್ಟರು
ತೆರೆ-ದಿ-ಡು-ತ್ತಿ-ದ್ದರು
ತೆರೆ-ದಿ-ರ-ಲಿ-ಲ್ಲವೊ
ತೆರೆ-ದಿ-ರುವ
ತೆರೆದು
ತೆರೆ-ದು-ಕೊಂ-ಡಿತು
ತೆರೆ-ದು-ಕೊಂಡು
ತೆರೆ-ದು-ಕೊಳ್ಳ
ತೆರೆ-ದು-ಕೊ-ಳ್ಳು-ತ್ತದೆ
ತೆರೆ-ದು-ತೋ-ರಲು
ತೆರೆ-ಮ-ನ-ಸ್ಸಿ-ನಿಂದ
ತೆರೆ-ಯನ್ನು
ತೆರೆ-ಯ-ಬೇ-ಕಾ-ಯಿತು
ತೆರೆ-ಯ-ಬೇಕು
ತೆರೆ-ಯ-ಲಿ-ದ್ದಾ-ರೆಂದೂ
ತೆರೆ-ಯ-ಲಿ-ರು-ವುದನ್ನು
ತೆರೆ-ಯಲು
ತೆರೆ-ಯ-ಲ್ಪ-ಟ್ಟುವು
ತೆರೆ-ಯಿತು
ತೆರೆ-ಯಿರಿ
ತೆರೆ-ಯಿ-ಸು-ತ್ತದೆ
ತೆರೆ-ಯು-ತ್ತಿವೆ
ತೆರೆ-ಯು-ವಂತೆ
ತೆರೆ-ಯು-ವು-ದ-ಕ್ಕಾಗಿ
ತೆರೆ-ಯು-ವುದು
ತೆರೆ-ಸು-ವಂ-ತಿತ್ತು
ತೆರೈ
ತೇ
ತೇಜಃ-ಪುಂಜ
ತೇಜ-ಸ್ಸನ್ನು
ತೇಜ-ಸ್ಸಿ-ನಿಂದ
ತೇಜಸ್ಸು
ತೇಜ-ಸ್ಸು
ತೇಜಸ್ಸೇ
ತೇಜೋ-ಮಯ
ತೇಜೋ-ಮ-ಯ-ವಾದ
ತೇದ-ರ-ವರು
ತೇದು
ತೇದು-ಕೊಂ-ಡರು
ತೇಪೆ
ತೇಯು-ತ್ತಿ-ರು-ವುದು
ತೇಯ್ದರು
ತೇರ-ಪುತ್ತ
ತೇರ-ಪು-ತ್ತರ
ತೇರ-ಪು-ತ್ತ-ರಲ್ಲಿ
ತೇರ-ಪು-ತ್ತರು
ತೇಲಾಡು
ತೇಲಾ-ಡು-ತ್ತಿ-ರುವ
ತೇಲಾ-ಡು-ವ-ವರು
ತೇಲಿ-ಕೊಂಡೇ
ತೇಲಿ-ಬಿಟ್ಟು
ತೇಲಿ-ಹೋ-ಗು-ತ್ತಿ-ದ್ದರು
ತೇಲು
ತೇಲು-ತ್ತಿ-ದ್ದೇನೆ
ತೇಲು-ತ್ತಿ-ರುವ
ತೇವ-ರ-ಹಿ-ತ-ವಾ-ದ್ದ-ರಿಂದ
ತೇವಾ-ರ-ಗ-ಳೆಂದು
ತೊಂದರೆ
ತೊಂದ-ರೆ-ಇ-ವು-ಗಳನ್ನೆಲ್ಲ
ತೊಂದ-ರೆ-ಗಳನ್ನು
ತೊಂದ-ರೆ-ಗ-ಳಿ-ಗೇನೂ
ತೊಂದ-ರೆ-ಗ-ಳಿ-ದ್ದುವು
ತೊಂದ-ರೆ-ಗಳೂ
ತೊಂದ-ರೆ-ಗ-ಳೆಲ್ಲ
ತೊಂದ-ರೆಗೆ
ತೊಂದ-ರೆ-ಗೊ-ಳ-ಗಾ-ಗಿ-ದ್ದರು
ತೊಂದ-ರೆ-ಯಾ-ಗ-ಲಿಲ್ಲ
ತೊಂದ-ರೆ-ಯಾ-ಗುವು
ತೊಂದ-ರೆ-ಯಾ-ಯಿತು
ತೊಂದ-ರೆ-ಯಿಂದ
ತೊಂದ-ರೆಯೂ
ತೊಂದ-ರೆಯೇ
ತೊಂಬತ್ತು
ತೊಟ್ಟ-ವರು
ತೊಟ್ಟಿದ್ದ
ತೊಟ್ಟಿ-ಯನ್ನು
ತೊಟ್ಟಿ-ಲಿಗೆ
ತೊಟ್ಟಿ-ಲು-ಗ-ಳೆ-ಷ್ಟಿ-ವೆಯೋ
ತೊಟ್ಟು
ತೊಟ್ಟೂ
ತೊಡ-ಗ-ದಿ-ದ್ದಾ-ಗಲೂ
ತೊಡ-ಗ-ಬ-ಲ್ಲಂ-ತಹ
ತೊಡ-ಗ-ಬಾ-ರ-ದೆಂದು
ತೊಡ-ಗ-ಬೇ-ಕಾ-ದರೆ
ತೊಡ-ಗಲು
ತೊಡಗಿ
ತೊಡ-ಗಿತು
ತೊಡ-ಗಿತ್ತು
ತೊಡ-ಗಿ-ದಂ-ದಿ-ನಿಂದ
ತೊಡ-ಗಿ-ದರು
ತೊಡ-ಗಿ-ದ-ವರು
ತೊಡ-ಗಿ-ದಾಗ
ತೊಡ-ಗಿ-ದಾ-ಗಲೂ
ತೊಡ-ಗಿ-ದ್ದರು
ತೊಡ-ಗಿ-ದ್ದು-ದನ್ನು
ತೊಡ-ಗಿ-ರ-ಲಿಲ್ಲ
ತೊಡ-ಗಿ-ರು-ತ್ತಿ-ದ್ದರು
ತೊಡ-ಗಿ-ರು-ತ್ತಿ-ದ್ದೆವು
ತೊಡ-ಗಿ-ಸ-ಲಾರ
ತೊಡ-ಗಿ-ಸಲು
ತೊಡ-ಗಿ-ಸಿ-ಕೊ-ಳ್ಳ-ಬೇಕು
ತೊಡ-ಗಿ-ಸಿ-ಕೊ-ಳ್ಳಲು
ತೊಡ-ಗಿ-ಸಿದ್ದ
ತೊಡ-ಗಿ-ಸಿ-ದ್ದೇ-ನೆಂ-ದ-ರಿತು
ತೊಡಗು
ತೊಡ-ಗುತ್ತ
ತೊಡ-ಗುವ
ತೊಡ-ಗು-ವಂತೆ
ತೊಡ-ರಿಸಿ
ತೊಡಿಗೆ
ತೊಡಿ-ಗೆ-ಗಳನ್ನೆಲ್ಲ
ತೊಡಿ-ಗೆ-ಗಳು
ತೊಡಿ-ಸ-ಲಾ-ಗಿತ್ತು
ತೊಡಿ-ಸಿ-ದರು
ತೊಡಿ-ಸುತ್ತ
ತೊಡೆ-ದು-ಹಾಕಿ
ತೊಡೆ-ದು-ಹಾಕು
ತೊಡೆ-ಯ-ಬ-ಲ್ಲುದೆ
ತೊನೆ-ದಾ-ಡಿ-ಸಿ-ದರೋ
ತೊರೆ
ತೊರೆ-ದರು
ತೊರೆದು
ತೊರೆ-ದು-ಬಿಡು
ತೊಲ-ಗ-ಬೇಕು
ತೊಲ-ಗಿ-ಹೋ-ಗು-ತ್ತದೆ
ತೊಳ-ಲಾಟ
ತೊಳ-ಲಾಡು
ತೊಳ-ಲಾ-ಡುವ
ತೊಳ-ಲು-ವ-ವರು
ತೊಳೆದ
ತೊಳೆದು
ತೊಳೆ-ದು-ಕೊಂ-ಡರು
ತೊಳೆ-ದು-ಕೊಂಡು
ತೊಳೆ-ದು-ಕೊ-ಳ್ಳಲು
ತೊಳೆ-ಯ-ಬ-ಹುದೆ
ತೊಳೆ-ಯಲು
ತೊಳೆ-ಯು-ವು-ದೆಂ-ದರೆ
ತೊಳೆ-ಯು-ವು-ದೆಂ-ದ-ರೇನು
ತೊವ್ವೆ
ತೋಚ
ತೋಚ-ಲಿಲ್ಲ
ತೋಟ-ಗಳು
ತೋಟ-ಗಾ-ರಿಕೆ
ತೋಟದ
ತೋಟ-ದಲ್ಲಿ
ತೋಟ-ದ-ಲ್ಲಿದ್ದ
ತೋಟ-ದೊ-ಳಕ್ಕೆ
ತೋಟ-ವಿ-ದ್ದದ್ದು
ತೋಟ-ವೊಂ-ದ-ರಲ್ಲಿ
ತೋಡಿ-ಕೊಂ-ಡ-ನೆಂದೇ
ತೋಡಿ-ಕೊಂಡಿ
ತೋಡಿ-ಕೊಂಡು
ತೋಡಿದ
ತೋಡಿ-ಸಿ-ದರು
ತೋಡು-ತ್ತಿ-ರುವ
ತೋಡು-ವ-ವರು
ತೋಮಾಯ್
ತೋಯಿ-ಸಿ-ದುದು
ತೋರ-ಣ-ಬಾ-ಳೆ-ಕಂ-ಬ-ಗಳನ್ನು
ತೋರ-ಣ-ಗಳನ್ನು
ತೋರದೆ
ತೋರ-ಬ-ಹುದು
ತೋರ-ಬಾ-ರ-ದೆಂ-ಬುದು
ತೋರ-ಲಿಲ್ಲ
ತೋರಾ-ಣಿ-ಕೆಗೆ
ತೋರಿ
ತೋರಿತು
ತೋರಿತ್ತು
ತೋರಿದ
ತೋರಿ-ದರು
ತೋರಿ-ದರೂ
ತೋರಿ-ದರೆ
ತೋರಿ-ದ-ರೆಂದು
ತೋರಿ-ದುದು
ತೋರಿ-ದ್ದನ್ನು
ತೋರಿದ್ದು
ತೋರಿ-ರ-ಬೇಕು
ತೋರಿ-ಸ-ಕೊ-ಡ-ಬಲ್ಲೆ
ತೋರಿ-ಸ-ಬ-ಹುದು
ತೋರಿ-ಸ-ಬೇ-ಕಾ-ಗಿದೆ
ತೋರಿ-ಸ-ಬೇಡ
ತೋರಿ-ಸ-ಲಾ-ಯಿತು
ತೋರಿ-ಸ-ಲಿಲ್ಲ
ತೋರಿ-ಸಲು
ತೋರಿಸಿ
ತೋರಿ-ಸಿ-ಕೊ-ಟ್ಟರು
ತೋರಿ-ಸಿ-ಕೊ-ಟ್ಟರೆ
ತೋರಿ-ಸಿ-ಕೊ-ಟ್ಟರೋ
ತೋರಿ-ಸಿ-ಕೊ-ಟ್ಟಿದೆ
ತೋರಿ-ಸಿ-ಕೊ-ಟ್ಟಿ-ದ್ದರು
ತೋರಿ-ಸಿ-ಕೊ-ಟ್ಟಿ-ರುವ
ತೋರಿ-ಸಿ-ಕೊಟ್ಟು
ತೋರಿ-ಸಿ-ಕೊ-ಟ್ಟು-ದ-ಕ್ಕಾಗಿ
ತೋರಿ-ಸಿ-ಕೊ-ಡ-ಬೇ-ಕಾ-ಗಿದೆ
ತೋರಿ-ಸಿ-ಕೊ-ಡ-ಬೇ-ಕಾದ
ತೋರಿ-ಸಿ-ಕೊ-ಡ-ಬೇಕು
ತೋರಿ-ಸಿ-ಕೊ-ಡಲು
ತೋರಿ-ಸಿ-ಕೊ-ಡು-ತ್ತಿ-ದ್ದರು
ತೋರಿ-ಸಿ-ಕೊ-ಡುವ
ತೋರಿ-ಸಿದ
ತೋರಿ-ಸಿ-ದಂತೆ
ತೋರಿ-ಸಿ-ದ-ರ-ಲ್ಲದೆ
ತೋರಿ-ಸಿ-ದರು
ತೋರಿ-ಸಿ-ದ್ದಾರೆ
ತೋರಿ-ಸುತ್ತ
ತೋರಿ-ಸು-ತ್ತದೆ
ತೋರಿ-ಸು-ತ್ತಾರೆ
ತೋರಿ-ಸು-ತ್ತಿ-ದ್ದರು
ತೋರಿ-ಸು-ತ್ತಿ-ದ್ದ-ರೆಂ-ದರೆ
ತೋರಿ-ಸು-ತ್ತಿ-ದ್ದಾರೆ
ತೋರಿ-ಸು-ತ್ತಿ-ರುವ
ತೋರಿ-ಸುವ
ತೋರಿ-ಸು-ವಂ-ತಹ
ತೋರಿ-ಸು-ವಂತೆ
ತೋರಿ-ಸು-ವಂ-ಥವು
ತೋರಿ-ಸು-ವ-ವರೂ
ತೋರು
ತೋರು-ತಿ-ಹುದು
ತೋರು-ತ್ತ-ದಲ್ಲ
ತೋರು-ತ್ತದೆ
ತೋರು-ತ್ತ-ದೆಯೋ
ತೋರು-ತ್ತಿತ್ತು
ತೋರು-ತ್ತಿದೆ
ತೋರು-ತ್ತಿ-ರುವ
ತೋರು-ತ್ತೀ-ರಲ್ಲ
ತೋರುವ
ತೋರು-ವಂತೆ
ತೋರು-ವ-ವರು
ತೋರು-ವುದನ್ನು
ತೋರು-ವು-ದಿ-ಲ್ಲವೋ
ತೋರು-ವುದೆ
ತೋರು-ವುದೇ
ತೋಳ-ಗಳ
ತೋಳಿಗೆ
ತೋಳು-ಗಳನ್ನು
ತೋಳು-ಗ-ಳಿಗೆ
ತ್ಕರಿ-ಸಿ-ಕೊಂ-ಡ-ರೆಂ-ಬು-ದನ್ನು
ತ್ಕೃಷ್ಟ
ತ್ತದೆ
ತ್ತದೆಂದು
ತ್ತದೆಯೋ
ತ್ತರ
ತ್ತಲೇ
ತ್ತವೆ
ತ್ತಾನೆ
ತ್ತಾನೆಯೋ
ತ್ತಾರೆ
ತ್ತಾರೆಂಬ
ತ್ತಾರೆ-ಮಿಸ್
ತ್ತಾರೆಯೇ
ತ್ತಾರೆಯೋ
ತ್ತಾಳೆ
ತ್ತಿತ್ತು
ತ್ತಿದೆ
ತ್ತಿದೆ-ಯೆಂದು
ತ್ತಿದೆಯೋ
ತ್ತಿದ್ದ
ತ್ತಿದ್ದಂತೆ
ತ್ತಿದ್ದಂ-ತೆಯೇ
ತ್ತಿದ್ದ-ನ-ಲ್ಲದೆ
ತ್ತಿದ್ದ-ರಂ-ತೆ-ಅ-ವರು
ತ್ತಿದ್ದರು
ತ್ತಿದ್ದ-ರು-ನ-ನ್ನಂ-ತೆಯೇ
ತ್ತಿದ್ದರೂ
ತ್ತಿದ್ದ-ರೆಂ-ದಲ್ಲ
ತ್ತಿದ್ದಳು
ತ್ತಿದ್ದಾಗ
ತ್ತಿದ್ದಾರೆ
ತ್ತಿದ್ದೀಯೆ
ತ್ತಿದ್ದೀ-ಯೆ-ಯಾ-ವುದೋ
ತ್ತಿದ್ದೀರಿ
ತ್ತಿದ್ದು
ತ್ತಿದ್ದುದು
ತ್ತಿದ್ದುವು
ತ್ತಿದ್ದೇನೆ
ತ್ತಿನ
ತ್ತಿರ-ಬೇಕು
ತ್ತಿರ-ಲಿಲ್ಲ
ತ್ತಿರ-ಲಿ-ಲ್ಲ-ವೆಂ-ಬಂ-ತೆ-ಎ-ಷ್ಟರ
ತ್ತಿರು-ತ್ತೀಯೋ
ತ್ತಿರುವ
ತ್ತಿರು-ವಾಗ
ತ್ತಿರು-ವುದನ್ನು
ತ್ತಿರು-ವುದು
ತ್ತಿರು-ವು-ದೆಲ್ಲ
ತ್ತಿಲ್ಲ
ತ್ತೀರಾ
ತ್ತೀರಿ
ತ್ತೀರೋ
ತ್ತುವ
ತ್ತೆಂದರೆ
ತ್ತೇನೆ
ತ್ತೇನೆ-ನೀವು
ತ್ತೇವೆಯೋ
ತ್ತೊಂದನೇ
ತ್ಥಾನ
ತ್ಮಿಕ
ತ್ಯಜಿ-ಸ-ಬೇಡಿ
ತ್ಯಜಿ-ಸಲು
ತ್ಯಜಿಸಿ
ತ್ಯಜಿ-ಸಿದ
ತ್ಯಜಿ-ಸಿ-ದ-ವರು
ತ್ಯಜಿ-ಸಿ-ಬಿ-ಟ್ಟಿ-ದ್ದ-ನಂತೆ
ತ್ಯಜಿ-ಸಿ-ಬಿ-ಟ್ಟಿ-ದ್ದಾನೆ
ತ್ಯಜಿ-ಸಿ-ಬಿ-ಡಲು
ತ್ಯಜಿ-ಸಿ-ಬಿ-ಡು-ತ್ತಾನೆ
ತ್ಯಜಿ-ಸಿರಿ
ತ್ಯಜಿ-ಸುವ
ತ್ಯಜಿ-ಸು-ವಂತೆ
ತ್ಯಜಿ-ಸು-ವುದು
ತ್ಯಜಿ-ಸು-ವುದೇ
ತ್ಯಾಗ
ತ್ಯಾಗ-ಪೂ-ಜೆ-ಗ-ಳೆಲ್ಲ
ತ್ಯಾಗ-ವೈ-ರಾಗ್ಯ
ತ್ಯಾಗ-ವೈ-ರಾ-ಗ್ಯ-ಗಳನ್ನು
ತ್ಯಾಗ-ಗಳ
ತ್ಯಾಗ-ಜೀ-ವ-ನದ
ತ್ಯಾಗ-ಜೀ-ವ-ನವು
ತ್ಯಾಗದ
ತ್ಯಾಗ-ದಲ್ಲಿ
ತ್ಯಾಗ-ಮ-ನೋ-ಭಾವ
ತ್ಯಾಗ-ಮಾಡಿ
ತ್ಯಾಗ-ಮಾ-ಡಿ-ದಷ್ಟೂ
ತ್ಯಾಗ-ವನ್ನು
ತ್ಯಾಗ-ವಿ-ದೆಯೋ
ತ್ಯಾಗವು
ತ್ಯಾಗ-ವೆಂಬ
ತ್ಯಾಗ-ವೆಂ-ಬಂ-ತ-ಲ್ಲದೆ
ತ್ಯಾಗವೇ
ತ್ಯಾಗ-ಸೂ-ಚ-ಕ-ವಾದ
ತ್ಯಾಗಿ-ಗ-ಳಾದ
ತ್ಯಾಗಿ-ಗಳು
ತ್ಯಾಗಿ-ಗ-ಳೊಂ-ದಿಗೆ
ತ್ರಣ-ಗಳು
ತ್ರಯ-ಮೇ-ವೈ-ತತ್
ತ್ರಾಣ-ವಿ-ರದ
ತ್ರಿಗು-ಣಾ-ತೀ-ತಾ-ನಂದ
ತ್ರಿಗು-ಣಾ-ತೀ-ತಾ-ನಂ-ದರ
ತ್ರಿಗು-ಣಾ-ತೀ-ತಾ-ನಂ-ದ-ರತ್ತ
ತ್ರಿಗು-ಣಾ-ತೀ-ತಾ-ನಂ-ದರು
ತ್ರಿಗು-ಣಾ-ತೀ-ತಾ-ನಂ-ದರೂ
ತ್ರಿವಿ-ಕ್ರಮ
ತ್ರಿಶೂ-ಲ-ವನ್ನು
ತ್ರೈಲಿಂ-ಗ-ಸ್ವಾಮಿ
ತ್ರೈಲೋಕ್ಯ
ತ್ರೈಲೋ-ಕ್ಯ-ನಾಥ
ತ್ರೈಲೋ-ಕ್ಯ-ನಾ-ಥನ
ತ್ರೈಲೋ-ಕ್ಯ-ನಿಂ-ದಲೇ
ತ್ರೈಲೋ-ಕ್ಯ-ಬಾಬು
ತ್ರೈಲೋ-ಕ್ಯ-ಬಾ-ಬು-ವಿಗೆ
ತ್ವರಿ-ತದ
ತ್ವರಿ-ತದ್ದೂ
ತ್ವರೆ-ಯಿಂದ
ಥಂಡಿ-ಯಿಂದ
ಥರ್ಸ್ಬಿ
ಥವೂ
ಥಾಮಸ್
ಥಿಯಾ-ಸೊ-ಫಿ-ಕಲ್
ಥಿಯಾ-ಸೊ-ಫಿಕ್
ಥಿಯಾ-ಸೊ-ಫಿಯ
ಥಿಯಾ-ಸೊ-ಫಿ-ಸ್ಟ-ನಾದ
ಥಿಯಾ-ಸೊ-ಫಿ-ಸ್ಟರ
ಥಿಯಾ-ಸೊ-ಫಿ-ಸ್ಟರು
ಥಿಯಾ-ಸೊ-ಫಿ-ಸ್ಟರೇ
ಥಿಯಾ-ಸೊ-ಫಿ-ಸ್ಟ-ರೊ-ಬ್ಬರು
ಥಿಯಾ-ಸೊ-ಫಿ-ಸ್ಟ್
ಥಿಯಾ-ಸೋ-ಫಿ-ಸ್ಟರು
ಥಿಯೇ-ಟ-ರಿ-ನಲ್ಲಿ
ಥಿಯೊ-ಡರ್
ಥಿಯೊ-ಲಾ-ಜಿ-ಕಲ್
ಥಿಯೊ-ಸಾ-ಫಿ-ಕಲ್
ಥೂ
ಥೂತ್ಕ-ರಿ-ಸಿ-ದರು
ದ
ದಂ
ದಂಗಾಗಿ
ದಂಗಾದ
ದಂಗಾ-ದರು
ದಂಗು-ಬ-ಡಿ-ದು-ಹೋ-ದರು
ದಂಗೆಯ
ದಂಡ-ಕ-ಮಂ-ಡ-ಲು-ಧಾ-ರಿ-ಯಾದ
ದಂಡದ
ದಂಡ-ನಾ-ಯ-ಕ-ರನ್ನು
ದಂಡನ್ನೇ
ದಂಡಿ
ದಂಡಿ-ಸ-ಲ್ಪ-ಟ್ಟರು
ದಂಡಿ-ಸಿ-ದ್ದರ
ದಂಡೆತ್ತಿ
ದಂಡೆಯ
ದಂಡೇ
ದಂತ-ಗಳು
ದಂತಹ
ದಂತ-ಹ-ಮ-ಹಾ-ಕಾಳೀ
ದಂತಾ-ಯಿತು
ದಂತಿತ್ತು
ದಂತಿ-ರ-ಬೇಕು
ದಂತೆ
ದಂಥವು
ದಂಪ-ತಿ-ಗಳ
ದಂಪ-ತಿ-ಗಳನ್ನು
ದಂಪ-ತಿ-ಗಳನ್ನೂ
ದಂಪ-ತಿ-ಗಳಿಂದ
ದಂಪ-ತಿ-ಗ-ಳಿಗೂ
ದಂಪ-ತಿ-ಗ-ಳಿಗೆ
ದಂಪ-ತಿ-ಗ-ಳಿಗೇ
ದಂಪ-ತಿ-ಗಳು
ದಂಪ-ತಿ-ಗಳೂ
ದಂಪ-ತಿ-ಗ-ಳೊಂ-ದಿಗೆ
ದಂಬೂಲ್
ದಂಬೂ-ಲ್ನಲ್ಲಿ
ದಕರೂ
ದಕ್ಕಾಗಿ
ದಕ್ಕಿಂತ
ದಕ್ಕು-ವಂ-ಥ-ದಲ್ಲ
ದಕ್ಷತೆ
ದಕ್ಷಿಣ
ದಕ್ಷಿ-ಣದ
ದಕ್ಷಿ-ಣ-ದಲ್ಲಿ
ದಕ್ಷಿ-ಣ-ದ-ವ-ರಲ್ಲ
ದಕ್ಷಿ-ಣ-ದಿ-ಕ್ಕಿ-ನಲ್ಲಿ
ದಕ್ಷಿ-ಣ-ಭಾ-ರ-ತಕ್ಕೂ
ದಕ್ಷಿ-ಣ-ಭಾ-ರ-ತದ
ದಕ್ಷಿ-ಣ-ಭಾ-ರ-ತ-ದಲ್ಲಿ
ದಕ್ಷಿ-ಣೀ-ರಂ-ಜನ್
ದಕ್ಷಿಣೇ
ದಕ್ಷಿ-ಣೇ-ಶ್ವರ
ದಕ್ಷಿ-ಣೇ-ಶ್ವ-ರಕ್ಕೆ
ದಕ್ಷಿ-ಣೇ-ಶ್ವ-ರ-ಗಳ
ದಕ್ಷಿ-ಣೇ-ಶ್ವ-ರದ
ದಕ್ಷಿ-ಣೇ-ಶ್ವ-ರ-ದತ್ತ
ದಕ್ಷಿ-ಣೇ-ಶ್ವ-ರ-ದಲ್ಲಿ
ದಕ್ಷಿ-ಣೇ-ಶ್ವ-ರ-ದ-ಲ್ಲಿದ್ದ
ದಕ್ಷಿ-ಣೇ-ಶ್ವ-ರ-ದಲ್ಲೇ
ದಗ್ಧ-ಹೃ-ದ-ಯಕ್ಕೆ
ದಟ್ಟ
ದಟ್ಟ-ವಾಗಿ
ದಟ್ಟ-ವಾದ
ದಟ್ಟೈಸಿ
ದಡ
ದಡಕ್ಕೆ
ದಡದ
ದಡ-ದತ್ತ
ದಡ-ದಲ್ಲಿ
ದಡ-ದ-ಲ್ಲಿದ್ದ
ದಡ-ದ-ಲ್ಲಿ-ರುವ
ದಡ-ದಲ್ಲೋ
ದಡ-ದಿಂದ
ದಡ-ವನ್ನು
ದಢೂತಿ
ದಣಿದ
ದಣಿ-ದಿ-ದ್ದಿರಿ
ದಣಿ-ಯ-ಲಿಲ್ಲ
ದಣಿ-ವರಿ
ದಣಿ-ವಾ-ದರೂ
ದಣಿ-ವಿ-ಲ್ಲದ
ದಣಿವು
ದಣಿ-ವುಂ-ಟು-ಮಾ-ಡು-ವಂ-ತಹ
ದಣಿವೇ
ದತ್ತ
ದತ್ತ-ಇ-ವ-ರನ್ನು
ದತ್ತನ
ದತ್ತನೂ
ದತ್ತರ
ದತ್ತರು
ದತ್ತಿಯ
ದದ್ದು
ದಧೀ-ಚಿಯ
ದನಿ
ದನಿ-ಗಾಗಿ
ದನಿ-ಗೂ-ಡಿಸಿ
ದನಿ-ಗೂ-ಡಿ-ಸುತ್ತ
ದನಿ-ಯನ್ನು
ದನಿ-ಯಲ್ಲಿ
ದನಿ-ಯಷ್ಟೇ
ದನಿ-ಯಿಲ್ಲ
ದನಿಯೂ
ದನಿ-ಯೆ-ತ್ತ-ಲೇ-ಬೇ-ಕಾ-ಗಿದೆ
ದನ್ನು
ದನ್ನೆಲ್ಲ
ದನ್ನೇ
ದಪ್ಪಕ್ಕೆ
ದಪ್ಪ-ಕ್ಷ-ರ-ಗಳಲ್ಲಿ
ದಪ್ಪ-ಗಿದ್ದ
ದಪ್ಪ-ಗಿ-ದ್ದೆ-ನೆಂ-ದರೆ
ದಪ್ಪ-ನೆಯ
ದಬ್ಬಾ-ಳಿಕೆ
ದಮ
ದಮನ
ದಮ-ನ-ಕ್ಕೊ-ಳ-ಗಾ-ಗಿ-ರುವ
ದಮ-ನ-ಗೊ-ಳಿ-ಸುವ
ದಯ-ದಿಂದ
ದಯ-ನೀಯ
ದಯ-ಪಾ-ಲಿ-ಸಿ-ದರು
ದಯ-ಪಾ-ಲಿ-ಸಿ-ದ್ದಾ-ರಲ್ಲ
ದಯ-ಮಾ-ಡಿ-ಸ-ಬೇಕು
ದಯ-ವಾ-ಗು-ತ್ತಿ-ದ್ದಂ-ತೆಯೇ
ದಯ-ವಿಟ್ಟು
ದಯಾ
ದಯಾ-ಪ-ರರು
ದಯಾ-ಪೂ-ರಿತ
ದಯಾ-ಮಯ
ದಯಾ-ಮಯಿ
ದಯಾರ್ದ್ರ
ದಯಾ-ಲರು
ದಯಾ-ಳು-ವಾದ
ದಯೆ
ದಯೆಯ
ದಯೆ-ಯಿಂದ
ದರ
ದರ-ಕ್ಕಿಂ-ತಲೂ
ದರಲ್ಲೋ
ದರಿ
ದರಿಂದ
ದರಿ-ದ್ರ-ನಾ-ರಾ-ಯಣ
ದರಿ-ದ್ರ-ರಲ್ಲಿ
ದರಿ-ದ್ರ-ರಾದ
ದರಿ-ದ್ರ-ರಿ-ಗಾಗಿ
ದರಿ-ದ್ರ-ರಿಗೆ
ದರಿ-ದ್ರರು
ದರು
ದರು-ನಾ-ಗ-ಮ-ಹಾ-ಶ-ಯ-ರಂ-ಥ-ವರು
ದರೂ
ದರೆ
ದರ್ಜಿಗೆ
ದರ್ಜಿಯ
ದರ್ಜೆಯ
ದರ್ಬಾ-ರನ್ನು
ದರ್ಬಾ-ರು-ಇ-ವು-ಗಳನ್ನೆಲ್ಲ
ದರ್ಭಾಂ-ಗದ
ದರ್ಭೆ-ಯಿಂದ
ದರ್ಶ-ಕ-ರಾ-ಗಿ-ದ್ದರು
ದರ್ಶ-ಕ-ರಾ-ಗಿ-ರ-ಬೇಕು
ದರ್ಶನ
ದರ್ಶ-ನ-ಸಂ-ದ-ರ್ಶನ
ದರ್ಶ-ನ-ಸಂ-ದ-ರ್ಶ-ನ-ಗ-ಳ-ನ್ನ-ರಸಿ
ದರ್ಶ-ನ-ಕ್ಕಾಗಿ
ದರ್ಶ-ನ-ಕ್ಕಾ-ಗಿಯೇ
ದರ್ಶ-ನಕ್ಕೆ
ದರ್ಶ-ನ-ಗಳ
ದರ್ಶ-ನ-ಗಳು
ದರ್ಶ-ನ-ಗ-ಳು-ಇ-ವು-ಗಳ
ದರ್ಶ-ನ-ಗ-ಳೆಲ್ಲ
ದರ್ಶ-ನದ
ದರ್ಶ-ನ-ದಿಂದ
ದರ್ಶ-ನ-ಭಾಗ್ಯ
ದರ್ಶ-ನ-ವನ್ನು
ದರ್ಶ-ನ-ವಾಗ
ದರ್ಶ-ನ-ವಾ-ಗ-ಲೇ-ಬೇಕು
ದರ್ಶ-ನ-ವಾ-ಗಿದೆ
ದರ್ಶ-ನ-ವಾ-ದರೆ
ದರ್ಶ-ನ-ವಾ-ಯಿತು
ದರ್ಶ-ನವು
ದರ್ಶ-ನವೂ
ದರ್ಶ-ನಾ-ನು-ಗ್ರ-ಹ-ಕ್ಕಾಗಿ
ದರ್ಶ-ನಾ-ನು-ಭವ
ದರ್ಶ-ನಾರ್ಥಿ
ದರ್ಶ-ನಾ-ರ್ಥಿ-ಗಳನ್ನು
ದರ್ಶ-ನಾ-ರ್ಥಿ-ಗ-ಳಾಗಿ
ದರ್ಶ-ನಾ-ರ್ಥಿ-ಗಳು
ದರ್ಶ-ನಾ-ರ್ಥಿ-ಗ-ಳೊಂ-ದಿಗೆ
ದರ್ಶ-ನಾ-ವ-ಕಾಶ
ದರ್ಶಿ-ತ್ವವೂ
ದರ್ಶಿ-ಯ-ನ್ನಾಗಿ
ದರ್ಶಿ-ಸ-ಬ-ಹುದು
ದರ್ಶಿ-ಸಲು
ದರ್ಶಿ-ಸಿದ
ದರ್ಶಿ-ಸಿ-ದರು
ದಲಿತ
ದಲಿ-ತ-ರ-ಲ್ಲಿ-ರುವ
ದಲ್ಲ
ದಲ್ಲವೆ
ದಲ್ಲಿ
ದಲ್ಲಿದ್ದ
ದಲ್ಲಿ-ದ್ದೇನೆ
ದಲ್ಲಿನ
ದಲ್ಲಿಯೂ
ದಲ್ಲಿಯೇ
ದಲ್ಲಿ-ರುವ
ದಲ್ಲಿ-ರು-ವಾ-ಗಲೇ
ದಲ್ಲೂ
ದಲ್ಲೇ
ದಲ್ಲೊಂದು
ದಳದ
ದಳ್ಳು-ರಿ-ಯನ್ನು
ದವನು
ದವನೋ
ದವ-ರ-ದ್ದಾ-ಗ-ಲಿದೆ
ದವ-ರಿಗೂ
ದವರು
ದವರೂ
ದವಳು
ದವು-ಗ-ಳಿಗೆ
ದಶ-ಕ-ಗ-ಳಲ್ಲೇ
ದಶ-ದಿ-ಶೆಗೂ
ದಶ-ಭು-ಜ-ಧಾ-ರಿ-ಣಿ-ಯಾದ
ದಷ್ಟು
ದಸ್ತಾ-ವೇ-ಜಿ-ನಲ್ಲಿ
ದಹನ
ದಹ-ನ-ಕಾ-ರ್ಯ-ವನ್ನು
ದಹಿ-ಸುತ್ತ
ದಹಿ-ಸು-ತ್ತಿ-ರುವ
ದಾ
ದಾಂತೆಯ
ದಾಂಪತ್ಯ
ದಾಕ್ಷಿ-ಣ್ಯಕ್ಕೆ
ದಾಖ-ಲಾ-ಗಿವೆ
ದಾಖ-ಲೆ-ಗಳ
ದಾಖ-ಲೆ-ಗಳನ್ನು
ದಾಖ-ಲೆ-ಗಳಲ್ಲಿ
ದಾಖ-ಲೆ-ಗಳು
ದಾಖ-ಲೆಯ
ದಾಗ
ದಾಗ-ಲಿಲ್ಲ
ದಾಗಲೂ
ದಾಗಿ
ದಾಗಿತ್ತು
ದಾಗಿಯೂ
ದಾಟ-ಬ-ಹು-ದಾ-ಗಿತ್ತು
ದಾಟ-ಬೇ-ಕಾ-ಯಿತು
ದಾಟಲು
ದಾಟಿ
ದಾಟಿದ
ದಾಟಿ-ದರು
ದಾಟಿ-ದುದೂ
ದಾಟಿದ್ದೇ
ದಾಟಿ-ಯೇನು
ದಾಟಿ-ಸಲು
ದಾಟುವ
ದಾಟು-ವು-ದಿಲ್ಲ
ದಾಡುತ್ತ
ದಾತ್ಮ-ನಾ-ತ್ಮಾ-ನಂ-ಅ-ವ-ರ-ವ-ರನ್ನು
ದಾದರೂ
ದಾದರೆ
ದಾದಾ
ದಾದೂ
ದಾನ
ದಾನ-ಗಳಲ್ಲಿ
ದಾನ-ಗ-ಳ-ಲ್ಲೆಲ್ಲ
ದಾನದ
ದಾನ-ಧ-ರ್ಮದ
ದಾನ-ವನ್ನು
ದಾನ-ವಾಗಿ
ದಾನವೇ
ದಾನಿ
ದಾಯ-ಕ-ವಾ-ದ-ದ್ದನ್ನು
ದಾಯಕ್ಕೂ
ದಾಯ-ಗಳ
ದಾಯ-ಪರ
ದಾಯಸ್ಥ
ದಾಯಿತು
ದಾರಿ
ದಾರಿ-ಖ-ರ್ಚನ್ನು
ದಾರಿ-ಗಾ-ಣದೆ
ದಾರಿ-ತ-ಪ್ಪಿ-ಸಿ-ಬಿಡು
ದಾರಿ-ತೋ-ರಲಿ
ದಾರಿ-ದೀಪ
ದಾರಿದ್ರ್ಯ
ದಾರಿ-ದ್ರ್ಯ-ಅ-ನ-ಕ್ಷ-ರ-ತೆ-ಗಳು
ದಾರಿ-ದ್ರ್ಯ-ಗಳು
ದಾರಿ-ದ್ರ್ಯದ
ದಾರಿ-ದ್ರ್ಯ-ವನ್ನು
ದಾರಿಯ
ದಾರಿ-ಯನ್ನು
ದಾರಿ-ಯಲ್ಲಿ
ದಾರಿ-ಯ-ಲ್ಲಿ-ದ್ದೇವೆ
ದಾರಿ-ಯ-ಲ್ಲಿನ
ದಾರಿ-ಯ-ಲ್ಲಿ-ರುವ
ದಾರಿ-ಯ-ಲ್ಲೊಂದು
ದಾರಿ-ಯಾಗಿ
ದಾರಿ-ಯಾಗೇ
ದಾರಿ-ಯಿ-ರ-ಲಿಲ್ಲ
ದಾರಿ-ಯಿ-ಲ್ಲ-ದಿ-ರು-ವು-ದ-ರಿಂದ
ದಾರಿ-ಯು-ದ್ದಕ್ಕೂ
ದಾರಿ-ಯೆಲ್ಲ
ದಾರಿಯೇ
ದಾರಿ-ಯೇ-ನೆಂದು
ದಾರಿ-ಯೊಂ-ದರ
ದಾರಿ-ಹೋಕ
ದಾರುಣ
ದಾಳ-ವನ್ನು
ದಾಳಿ-ಯಿ-ಡ-ಬೇಕು
ದಾವ್
ದಾಸ
ದಾಸ-ನಾ-ಗ-ದಂತೆ
ದಾಸರ
ದಾಸ-ರು-ಇ-ವರೇ
ದಾಸರೂ
ದಾಸೀ-ಪು-ತ್ರ-ರಾದ
ದಾಸೋ
ದಾಸ್
ದಾಸ್-ಇ-ವರು
ದಾಸ್ಯ-ಇ-ವು-ಗಳ
ದಾಸ್ಯಈ
ದಾಸ್ಯಕ್ಕೆ
ದಾಸ್ಯ-ದಿಂದ
ದಾಸ್ರ-ವರು
ದಿಂದ
ದಿಂದಲೆ
ದಿಂದಲೇ
ದಿಂದಲೋ
ದಿಂದಾಗಿ
ದಿಂದಿ-ದ್ದಳು
ದಿಂದೆದ್ದು
ದಿಂದೊ-ಡ-ಗೂಡಿ
ದಿಂಬಿನ
ದಿಕ್ಕಿಗೆ
ದಿಕ್ಕಿ-ನಲ್ಲಿ
ದಿಕ್ಕಿ-ನ-ಲ್ಲಿ-ರುವ
ದಿಕ್ಕಿ-ನಲ್ಲೇ
ದಿಕ್ಕು
ದಿಕ್ಕು-ಗಳಲ್ಲಿ
ದಿಕ್ಕು-ಗಾ-ಣ-ದಂ-ತಾಗಿ
ದಿಗಂ-ತ-ದಲ್ಲಿ
ದಿಗಂ-ತ-ದ-ವ-ರೆಗೂ
ದಿಗಂ-ಬರ
ದಿಗಳ
ದಿಗ-ಳಿ-ದ್ದು-ದ-ರಿಂದ
ದಿಗಿ-ಲು-ಗೊಂಡೆ
ದಿಗಿ-ಲು-ಗೊ-ಳ್ಳ-ಬೇಡ
ದಿಗೆ
ದಿಗ್ಬ್ರ-ಮೆ-ಗೊಂ-ಡ-ವ-ರಾಗಿ
ದಿಗ್ಬ್ರಾಂ-ತ-ನಾದ
ದಿಗ್ಭಾಂ-ತ-ರಾ-ದರು
ದಿಗ್ಭ್ರಾಂ-ತ-ರಾ-ದರು
ದಿಗ್ಭ್ರಾಂ-ತ-ಳಾಗಿ
ದಿಗ್ವಿ-ಜ-ಯದ
ದಿಙ್ಮೂ-ಢ-ರಾ-ದರು
ದಿಟ್ಟ-ತನ
ದಿಟ್ಟಿಸಿ
ದಿಟ್ಟಿ-ಸಿದ
ದಿಟ್ಟಿ-ಸಿ-ದರು
ದಿಟ್ಟಿ-ಸುತ್ತ
ದಿಟ್ಟಿ-ಸು-ತ್ತಿ-ದ್ದರು
ದಿಢೀರ್
ದಿತೆಯ
ದಿದ್ದರೂ
ದಿದ್ದರೆ
ದಿದ್ದಲ್ಲಿ
ದಿದ್ದ-ವ-ರ-ನ್ನು-ದ್ದೇ-ಶಿಸಿ
ದಿದ್ದಾಗ
ದಿದ್ದುದು
ದಿದ್ದು-ಬಿ-ಟ್ಟಿ-ದ್ದಾರೆ
ದಿನ
ದಿನಂ-ಪ್ರ-ತಿಯ
ದಿನ-ಅ-ನೇಕ
ದಿನ-ಕರ
ದಿನ-ಕ-ಳೆ-ದಂ-ತೆಲ್ಲ
ದಿನ-ಕ-ಳೆ-ದರು
ದಿನಕ್ಕೆ
ದಿನ-ಕ್ಕೆ-ಎಂಬ
ದಿನ-ಕ್ಕೆ-ರಡ
ದಿನ-ಗ-ಟ್ಟಲೆ
ದಿನ-ಗಳ
ದಿನ-ಗ-ಳಂತೂ
ದಿನ-ಗ-ಳಂದು
ದಿನ-ಗ-ಳ-ನ್ನಾ-ದರೂ
ದಿನ-ಗಳನ್ನು
ದಿನ-ಗಳಲ್ಲಿ
ದಿನ-ಗ-ಳ-ಲ್ಲೆಲ್ಲ
ದಿನ-ಗ-ಳಲ್ಲೇ
ದಿನ-ಗ-ಳ-ಲ್ಲೊಮ್ಮೆ
ದಿನ-ಗ-ಳ-ವ-ರೆಗೆ
ದಿನ-ಗ-ಳವು
ದಿನ-ಗ-ಳಾ-ಗಿ-ದ್ದು-ವಷ್ಟೇ
ದಿನ-ಗ-ಳಾ-ದ-ಮೇಲೆ
ದಿನ-ಗ-ಳಾ-ದರೂ
ದಿನ-ಗ-ಳಿಂ-ದಂತೂ
ದಿನ-ಗ-ಳಿಂ-ದಲೂ
ದಿನ-ಗ-ಳಿಂ-ದಲೇ
ದಿನ-ಗ-ಳಿ-ಗಿಂತ
ದಿನ-ಗ-ಳಿಗೇ
ದಿನ-ಗ-ಳಿವೆ
ದಿನ-ಗಳು
ದಿನ-ಗಳೇ
ದಿನ-ಗ-ಳೊ-ಳ-ಗಾಗಿ
ದಿನ-ಚರಿ
ದಿನ-ಚ-ರಿಗೆ
ದಿನ-ಚ-ರಿಯ
ದಿನ-ಚ-ರಿ-ಯನ್ನು
ದಿನ-ಜುಲೈ
ದಿನ-ಜ್ಪು-ರದ
ದಿನ-ಜ್ಪು-ರ-ದಲ್ಲಿ
ದಿನ-ಡಿ-ಸೆಂ-ಬರ್
ದಿನದ
ದಿನ-ದಂದು
ದಿನ-ದಂದೇ
ದಿನ-ದಲ್ಲಿ
ದಿನ-ದಿಂದ
ದಿನ-ದಿಂ-ದಲೇ
ದಿನ-ದಿ-ನಕ್ಕೂ
ದಿನ-ದಿ-ನಕ್ಕೆ
ದಿನ-ದಿ-ನವೂ
ದಿನ-ನಿ-ತ್ಯದ
ದಿನ-ಪ-ತ್ರಿ-ಕೆ-ಗ-ಳಿ-ಗೆಲ್ಲ
ದಿನ-ಪ-ತ್ರಿ-ಕೆ-ಗ-ಳೆಲ್ಲ
ದಿನ-ಪ-ತ್ರಿ-ಕೆಯ
ದಿನ-ಪ-ತ್ರಿ-ಕೆ-ಯನ್ನು
ದಿನ-ಬೆ-ಳ-ಗಾ-ದರೆ
ದಿನ-ವಂತೂ
ದಿನ-ವನ್ನು
ದಿನ-ವಾ-ದರೂ
ದಿನ-ವಿಡೀ
ದಿನ-ವಿದ್ದ
ದಿನವೂ
ದಿನ-ವೆಂ-ಬುದು
ದಿನ-ವೆಲ್ಲ
ದಿನವೇ
ದಿನ-ವೊಂ-ದನ್ನು
ದಿನಾಂಕ
ದಿನಾಂ-ಕ-ವ-ನ್ನುಂ
ದಿನಾಂ-ಕ-ವನ್ನೋ
ದಿನೇ-ದಿನೇ
ದಿರದು
ದಿರ-ಬ-ಹುದು
ದಿರು
ದಿರು-ವುದು
ದಿಲ್ಲ
ದಿಲ್ಲವೋ
ದಿವಂ-ಗತ
ದಿವ-ಸ-ದ-ವ-ರೆಗೂ
ದಿವಾ-ನ-ನಾದ
ದಿವ್ಯ
ದಿವ್ಯ-ಕ-ನ್ಯೆ-ಯಾದ
ದಿವ್ಯ-ಗು-ರು-ವಿನ
ದಿವ್ಯ-ಜ್ಞಾ-ನದ
ದಿವ್ಯ-ಜ್ಞಾ-ನ-ವನ್ನು
ದಿವ್ಯ-ಜ್ಯೋತಿ
ದಿವ್ಯ-ಜ್ಯೋ-ತಿಯ
ದಿವ್ಯ-ಜ್ಯೋ-ತಿ-ಯನ್ನೇ
ದಿವ್ಯತೆ
ದಿವ್ಯ-ತೆಗೂ
ದಿವ್ಯ-ತೆ-ಯನ್ನು
ದಿವ್ಯ-ತೆ-ಯನ್ನೂ
ದಿವ್ಯ-ತೆ-ಯೊಂದು
ದಿವ್ಯ-ದ-ರ್ಶ-ನ-ವಾ-ಯಿತು
ದಿವ್ಯ-ಧ್ಯಾ-ನ-ದಲ್ಲಿ
ದಿವ್ಯ-ಭಾವ
ದಿವ್ಯ-ಮಾ-ತೆಯ
ದಿವ್ಯ-ವಾಗಿ
ದಿವ್ಯ-ವಾ-ಗಿ-ತ್ತೆಂ-ಬು-ದನ್ನು
ದಿವ್ಯ-ವಾ-ಣಿ-ಯನ್ನು
ದಿವ್ಯ-ವಾದ
ದಿವ್ಯ-ಸಂ-ದೇ-ಶದ
ದಿವ್ಯಾ-ತ್ಮರು
ದಿವ್ಯೋ-ತ್ಸಾಹ
ದಿಸೆ
ದಿಸೆಯ
ದಿಸೆ-ಯತ್ತ
ದಿಸೆ-ಯಲ್ಲಿ
ದಿಸೆ-ಯಲ್ಲೇ
ದೀಕ್ಷಾ-ಕಾರ್ಯ
ದೀಕ್ಷಾ-ಕಾ-ರ್ಯಕ್ಕೆ
ದೀಕ್ಷಾ-ಕಾ-ರ್ಯವು
ದೀಕ್ಷಾ-ರ್ಥಿ-ಗಳ
ದೀಕ್ಷೆ
ದೀಕ್ಷೆಯ
ದೀಕ್ಷೆ-ಯನ್ನು
ದೀತು
ದೀನ
ದೀನ-ದ-ಲಿ-ತ-ದುಃ-ಖಿ-ಗಳಲ್ಲಿ
ದೀನ-ತೆ-ಯನ್ನು
ದೀನ-ದ-ಯಾ-ಪ-ರ-ತೆಯ
ದೀನ-ದ-ಯಾಳು
ದೀನ-ದ-ಲಿ-ತರ
ದೀನ-ದ-ಲಿ-ತ-ರಿ-ಗಾಗಿ
ದೀನ-ನಾದ
ದೀನರ
ದೀನ-ರಲ್ಲಿ
ದೀನ-ರೆಂದು
ದೀನಾ-ರ್ತರ
ದೀಪ
ದೀಪ-ಕೈ-ದೀ-ವಿ-ಗೆ-ಗಳು
ದೀಪ-ಗಳ
ದೀಪ-ಗಳು
ದೀಪೋ-ತ್ಸವ
ದೀರ್ಘ
ದೀರ್ಘ-ಕಾಲ
ದೀರ್ಘ-ಕಾ-ಲದ
ದೀರ್ಘ-ಕಾ-ಲ-ದಿಂದ
ದೀರ್ಘ-ಕಾ-ಲ-ವಾದ
ದೀರ್ಘ-ದಂಡ
ದೀರ್ಘ-ದಂ-ಡ-ಪ್ರ-ಣಾಮ
ದೀರ್ಘ-ಪ-ಟ್ಟಿ-ಯನ್ನು
ದೀರ್ಘ-ವಾಗಿ
ದೀರ್ಘ-ವಾ-ಗಿ-ತ್ತೆಂ-ದರೆ
ದೀರ್ಘ-ವಾದ
ದೀರ್ಘ-ವಾ-ದಷ್ಟೂ
ದೀರ್ಘಾಯು
ದುಂಟು
ದುಂಟೆ
ದುಃಖ
ದುಃಖ-ದಾರಿ-ದ್ರ್ಯ-ದೌ-ರ್ಜ-ನ್ಯ-ವಿ-ಧ್ವಂ-ಸ-ಕ-ತೆ-ಗ-ಳಲ್ಲೂ
ದುಃಖ-ದು-ಮ್ಮಾ-ನ-ಗಳ
ದುಃಖಂ-ಸ್ವಾ-ವ-ಲಂ-ಬ-ನೆಯೇ
ದುಃಖ-ಕರ
ದುಃಖ-ಕ್ಕಿಂ-ತೇನೂ
ದುಃಖ-ಕ್ಕೀ-ಡಾ-ಗಿ-ದ್ದರು
ದುಃಖ-ಗಳನ್ನು
ದುಃಖ-ತ-ಪ್ತ-ರಾ-ಗಿದ್ದ
ದುಃಖ-ತ-ಪ್ತ-ವಾ-ಗು-ವುದು
ದುಃಖದ
ದುಃಖ-ದಾರಿ-ದ್ರ್ಯದ
ದುಃಖ-ದಿಂದ
ದುಃಖ-ದಿಂ-ದಲೋ
ದುಃಖ-ದು-ರಂ-ತ-ಗಳನ್ನು
ದುಃಖ-ಪ-ಡ-ಬ-ಹುದೆ
ದುಃಖ-ವನ್ನು
ದುಃಖ-ವಾ-ಗು-ತ್ತಿತ್ತು
ದುಃಖ-ವಾ-ಗು-ತ್ತಿದೆ
ದುಃಖ-ವಾ-ಯಿ-ತಾ-ದರೂ
ದುಃಖ-ವಾ-ಯಿತು
ದುಃಖವೂ
ದುಃಖ-ವೇನೂ
ದುಃಖ-ಶೋ-ಕವೂ
ದುಃಖ-ಸಂ-ಕ-ಟ-ಗಳ
ದುಃಖ-ಸಂ-ಕ-ಟ-ಗಳನ್ನು
ದುಃಖಾರ್ತ
ದುಃಖಾ-ರ್ತರೇ
ದುಃಖಿ-ಗಳ
ದುಃಖಿ-ಗಳಲ್ಲಿ
ದುಃಖಿ-ಗ-ಳಿ-ಗಾಗಿ
ದುಃಖಿ-ಗಳು
ದುಃಖಿ-ತ-ರಾ-ಗಿ-ರು-ವಂತೆ
ದುಃಖಿ-ತ-ರಾ-ದರು
ದುಃಖಿ-ತ-ಳಾ-ದಳು
ದುಃಖಿನೀ
ದುಃಖಿ-ಸಿ-ದಳು
ದುಃಖಿ-ಸಿ-ದಾ-ಗಲೂ
ದುಃಖೋ-ದ್ವೇ-ಗ-ದಿಂದ
ದುಃಸ್ಥಿ-ತಿಗೆ
ದುಃಸ್ಥಿ-ತಿ-ಯಿಂದ
ದುಗುಡ
ದುಗು-ಡ-ಗ-ಳೆಲ್ಲ
ದುಗು-ಡದ
ದುಡಿ
ದುಡಿದ
ದುಡಿ-ದರು
ದುಡಿ-ದರೂ
ದುಡಿ-ದಿ-ದ್ದರು
ದುಡಿ-ದಿ-ದ್ದೇನೆ
ದುಡಿದು
ದುಡಿ-ಮೆಯ
ದುಡಿ-ಮೆ-ಯಲ್ಲಿ
ದುಡಿ-ಯಲು
ದುಡಿ-ಯಿರಿ
ದುಡಿಯು
ದುಡಿ-ಯು-ತ್ತಲೇ
ದುಡಿ-ಯು-ತ್ತಿ-ದ್ದರು
ದುಡಿ-ಯು-ತ್ತಿ-ರುವ
ದುಡಿ-ಯು-ತ್ತೇನೆ
ದುಡಿ-ಯುವ
ದುಡಿ-ಯು-ವ-ವರು
ದುಡಿ-ಯೋಣ
ದುದು
ದುಬಾರಿ
ದುಭಾ-ಷಿ-ಗಳು
ದುರಂತ
ದುರಂ-ತಕ್ಕೆ
ದುರಂ-ತ-ದಂ-ಚಿ-ನಲ್ಲಿ
ದುರ-ದೃ-ಷ್ಟಕ್ಕೆ
ದುರ-ದೃ-ಷ್ಟ-ದಿಂದ
ದುರ-ದೃ-ಷ್ಟ-ವ-ಶಾತ್
ದುರ-ದೃ-ಷ್ಟ-ವಾತ್
ದುರ-ದೃ-ಷ್ಟ-ವಾ-ಶಾತ್
ದುರ-ಭಿ-ಮಾನ
ದುರ-ಭಿ-ಮಾ-ನಕ್ಕೆ
ದುರ-ಭಿ-ಮಾ-ನ-ವ-ನ್ನಿನ್ನು
ದುರ-ವ-ಸ್ಥೆ-ಯನ್ನು
ದುರಸ್ತಿ
ದುರ-ಹಂ-ಕಾ-ರ-ಕ್ಕಾಗಿ
ದುರ-ಹಂ-ಕಾ-ರದ
ದುರಾ
ದುರಾ-ಕ್ರ-ಮಣ
ದುರಾ-ಕ್ರ-ಮ-ಣ-ಕಾರಿ
ದುರಾ-ಕ್ರ-ಮ-ಣಕ್ಕೆ
ದುರಾ-ಕ್ರ-ಮ-ಣ-ಗ-ಳ-ನ್ನೆ-ದು-ರಿ-ಸಿಯೂ
ದುರಾ-ಚಾ-ರ-ಗಳ
ದುರಾ-ಚಾ-ರ-ಗ-ಳಿಗೆ
ದುರಾಸೆ
ದುರಾ-ಸೆ-ಗ-ಳೆಲ್ಲ
ದುರು-ದ್ದೇ-ಶ-ಪೂ-ರಿತ
ದುರು-ಪ-ಯೋಗ
ದುರು-ಪ-ಯೋ-ಗ-ಇವು
ದುರು-ಪ-ಯೋ-ಗ-ಪ-ಡಿ-ಸಿ-ಕೊ-ಳ್ಳು-ವುದೇ
ದುರು-ಪ-ಯೋ-ಗ-ಪ-ಡಿ-ಸು-ವ-ವರೂ
ದುರು-ಪ-ಯೋ-ಗ-ವನ್ನು
ದುರ್ಗಾ
ದುರ್ಗಾ-ಪೂಜೆ
ದುರ್ಗಾ-ಪೂ-ಜೆ-ಗಾಗಿ
ದುರ್ಗಾ-ಪೂ-ಜೆಗೆ
ದುರ್ಗಾ-ಪೂ-ಜೆಯ
ದುರ್ಗಾ-ಪೂ-ಜೆ-ಯನ್ನು
ದುರ್ಗಾ-ಮಾತೆ
ದುರ್ಗಾ-ಷ್ಟಮಿ
ದುರ್ಗೆ
ದುರ್ಗೆಯ
ದುರ್ಗೆ-ಯ-ದುವೆ
ದುರ್ಘ-ಟನೆ
ದುರ್ಜ-ನರೂ
ದುರ್ಜ-ನರೋ
ದುರ್ದೈ-ವವೆ
ದುರ್ಬಲ
ದುರ್ಬ-ಲ-ರೋ-ಗಿಷ್ಠ
ದುರ್ಬ-ಲ-ಗೊ-ಳಿ-ಸಲಾ
ದುರ್ಬ-ಲ-ಗೊ-ಳಿ-ಸುವ
ದುರ್ಬ-ಲ-ಗೊ-ಳಿ-ಸು-ವಂ-ಥವು
ದುರ್ಬ-ಲ-ಗೊ-ಳಿ-ಸು-ವು-ವೆಂ-ಬುದೇ
ದುರ್ಬ-ಲತೆ
ದುರ್ಬ-ಲನ
ದುರ್ಬ-ಲ-ನಾ-ಗಿ-ದ್ದೇನೆ
ದುರ್ಬ-ಲ-ರ-ನ್ನಾ-ಗಿ-ಸಿ-ರು-ವುದೋ
ದುರ್ಬ-ಲ-ರನ್ನು
ದುರ್ಬ-ಲ-ರಾ-ಗಿದ್ದು
ದುರ್ಬ-ಲ-ರಾ-ಗಿ-ರು-ವು-ದ-ರಿಂ-ದಲೇ
ದುರ್ಬ-ಲ-ರಾ-ಗು-ವುದನ್ನು
ದುರ್ಬ-ಲ-ರಾ-ದ-ವರು
ದುರ್ಬ-ಲರು
ದುರ್ಬ-ಲ-ರು-ದ-ಲಿ-ತರು
ದುರ್ಬ-ಲರೂ
ದುರ್ಬ-ಲರೇ
ದುರ್ಬ-ಲ-ವಾ-ಗಿದ್ದ
ದುರ್ಬು-ದ್ಧಿಯ
ದುರ್ಭರ
ದುರ್ಭ-ರ-ವಾಗಿ
ದುರ್ಭೇ-ದ್ಯ-ವಾದ
ದುರ್ಮ-ರಣ
ದುರ್ಲಭ
ದುರ್ಲಭಂ
ದುರ್ಲ-ಭ-ವಾ-ದುವು
ದುರ್ವ-ರ್ತ-ನೆ-ಗಾಗಿ
ದುರ್ವ-ರ್ತ-ನೆಗೆ
ದುರ್ವ-ರ್ತ-ನೆ-ಯನ್ನು
ದುವು
ದುಷ್ಕೃತ್ಯ
ದುಷ್ಟ
ದುಷ್ಟ-ತನ
ದುಷ್ಟ-ತ-ನ-ವೆಂ-ದರೂ
ದುಷ್ಟ-ನಾ-ಗ-ಲೇ-ಬೇ-ಕಿ-ದ್ದರೆ
ದುಷ್ಟ-ನಾಗು
ದುಷ್ಟರ
ದುಷ್ಟ-ರಿ-ಗೊಂದು
ದುಷ್ಟರು
ದುಷ್ಪ-ರಿ-ಣಾಮ
ದುಸ್ವ-ಪ್ನವೇ
ದುಸ್ಸಾಧ್ಯ
ದುಸ್ಸಾ-ಧ್ಯ-ವಾಗಿ
ದುಸ್ಸಾ-ಧ್ಯ-ವೆ-ನಿ-ಸಿ-ದರೆ
ದೂಡುವ
ದೂತ
ದೂತ-ನಾ-ಗು-ತ್ತೇನೆ
ದೂತ-ರನ್ನು
ದೂರ
ದೂರಕ್ಕೆ
ದೂರಕ್ಕೇ
ದೂರ-ಗೊ-ಳಿಸಿ
ದೂರ-ಗೊ-ಳಿ-ಸುತ್ತ
ದೂರ-ಗೊ-ಳಿ-ಸು-ವು-ದ-ಕ್ಕಾಗಿ
ದೂರದ
ದೂರ-ದ-ರ್ಶ-ಕದ
ದೂರ-ದಲ್ಲಿ
ದೂರ-ದ-ಲ್ಲಿತ್ತು
ದೂರ-ದ-ಲ್ಲಿದೆ
ದೂರ-ದ-ಲ್ಲಿದ್ದ
ದೂರ-ದ-ಲ್ಲಿ-ದ್ದು-ದ-ರಿಂದ
ದೂರ-ದ-ಲ್ಲಿ-ರುವ
ದೂರ-ದಲ್ಲೇ
ದೂರ-ದ-ಲ್ಲೊಂದು
ದೂರ-ದಿಂದ
ದೂರ-ದಿಂ-ದಲೇ
ದೂರ-ದೂ-ರದ
ದೂರ-ದೂ-ರ-ದಿಂದ
ದೂರ-ಮಾ-ಡಲು
ದೂರ-ವನ್ನು
ದೂರವಾ
ದೂರ-ವಾ-ಗ-ಬ-ಯ-ಸು-ತ್ತೀರಿ
ದೂರ-ವಾ-ಗ-ಬಲ್ಲ
ದೂರ-ವಾ-ಗ-ಬೇಕು
ದೂರ-ವಾ-ಗಿದ್ದ
ದೂರ-ವಾ-ಗಿ-ದ್ದರೂ
ದೂರ-ವಾ-ಗಿ-ದ್ದುದು
ದೂರ-ವಾ-ಗಿ-ದ್ದೇನೆ
ದೂರ-ವಾ-ಗು-ತ್ತಿ-ದ್ದರು
ದೂರ-ವಾ-ಗುವ
ದೂರ-ವಾ-ಗು-ವಂ-ತಿಲ್ಲ
ದೂರ-ವಾ-ಗು-ವುದು
ದೂರ-ವಾ-ಗು-ವು-ವೆಂದು
ದೂರ-ವಾದ
ದೂರ-ವಾ-ದರು
ದೂರ-ವಾ-ದ-ರೆ-ನ್ನು-ವುದೇ
ದೂರ-ವಾ-ದ-ವರೂ
ದೂರ-ವಾ-ದ-ವ-ರೆಂ-ದರೆ
ದೂರ-ವಾ-ದಷ್ಟೂ
ದೂರ-ವಾ-ದುದು
ದೂರ-ವಾ-ದುವೆ
ದೂರ-ವಾದೆ
ದೂರ-ವಾ-ಯಿತು
ದೂರ-ವಿಟ್ಟು
ದೂರ-ವಿ-ಡಲು
ದೂರ-ವಿ-ಡು-ತ್ತಾರೆ
ದೂರ-ವಿದ್ದ
ದೂರ-ವಿ-ರ-ಲಿಲ್ಲ
ದೂರ-ವಿ-ರುವ
ದೂರವೂ
ದೂರ-ಸ್ಥ-ಳ-ಗ-ಳ-ಲ್ಲಿನ
ದೂರೀ-ಕ-ರಿ-ಸಿ-ಬಿ-ಟ್ಟಳು
ದೂರು
ದೂರುತ್ತ
ದೂರುವ
ದೂಷ-ಣೆಯ
ದೂಷಿ-ಸು-ವಂ-ತಿಲ್ಲ
ದೂಷಿ-ಸು-ವು-ದಿಲ್ಲ
ದೃಗ್ಗೋ-ಚ-ರ-ವಾದ
ದೃಢ
ದೃಢ-ಕಾ-ಯರೂ
ದೃಢ-ಚಿ-ತ್ತರು
ದೃಢತೆ
ದೃಢ-ತೆ-ಯನ್ನು
ದೃಢ-ನಿ-ರ್ಧಾ-ರ-ದಿಂದ
ದೃಢ-ನಿ-ಶ್ಚ-ಯ-ಮಾಡಿ
ದೃಢ-ಪ-ಟ್ಟಿತು
ದೃಢ-ಪ-ಡಿ-ಸ-ಬೇಕು
ದೃಢ-ಪ-ಡಿ-ಸಿ-ಕೊಂ-ಡರೆ
ದೃಢ-ಪ-ಡಿ-ಸಿ-ಕೊ-ಳ್ಳ-ಬ-ಹುದು
ದೃಢ-ಪ-ಡಿ-ಸಿ-ಕೊಳ್ಳಿ
ದೃಢ-ವಾಗಿ
ದೃಢ-ವಾ-ಗಿತ್ತು
ದೃಢ-ವಾ-ಗಿದೆ
ದೃಢ-ವಾ-ಗಿ-ಸು-ತ್ತಿ-ದ್ದುದು
ದೃಢ-ವಾ-ಗುತ್ತ
ದೃಢ-ವಾ-ಗು-ತ್ತಿದೆ
ದೃಢ-ವಾದ
ದೃಢ-ವಾ-ಯಿತು
ದೃಢ-ವಿ-ಶ್ವಾ-ಸ-ವಿತ್ತು
ದೃಢಿಷ್ಠ
ದೃಢಿ-ಷ್ಠ-ನಾ-ಗುತ್ತೀ
ದೃಢೀ-ಕ-ರಿ-ಸು-ತ್ತವೆ
ದೃಶ್ಯ
ದೃಶ್ಯ-ಗಳ
ದೃಶ್ಯ-ಗ-ಳತ್ತ
ದೃಶ್ಯ-ಗಳನ್ನು
ದೃಶ್ಯ-ಗಳನ್ನೆಲ್ಲ
ದೃಶ್ಯ-ಗಳಿಂದ
ದೃಶ್ಯ-ಗ-ಳಿವೆ
ದೃಶ್ಯ-ಗಳು
ದೃಶ್ಯದ
ದೃಶ್ಯ-ವನ್ನು
ದೃಶ್ಯ-ವನ್ನೇ
ದೃಶ್ಯ-ವಾ-ಗಿತ್ತು
ದೃಶ್ಯವೇ
ದೃಶ್ಯವೊ
ದೃಶ್ಯ-ವೊಂದು
ದೃಶ್ಯಾ-ವ-ಳಿ-ಗಳು
ದೃಷ್ಟಿ
ದೃಷ್ಟಿ-ಕೋನ
ದೃಷ್ಟಿ-ಕೋ-ನ-ಗಳಿಂದ
ದೃಷ್ಟಿ-ಕೋ-ನದ
ದೃಷ್ಟಿ-ಕೋ-ನ-ದಲ್ಲಿ
ದೃಷ್ಟಿ-ಕೋ-ನ-ದಿಂದ
ದೃಷ್ಟಿ-ಕೋ-ನ-ವನ್ನು
ದೃಷ್ಟಿ-ಕೋ-ನ-ವಿತ್ತು
ದೃಷ್ಟಿ-ಗಳಿಂದ
ದೃಷ್ಟಿಗೆ
ದೃಷ್ಟಿ-ಮಾ-ತ್ರ-ದಿಂದ
ದೃಷ್ಟಿಯ
ದೃಷ್ಟಿ-ಯನ್ನು
ದೃಷ್ಟಿ-ಯನ್ನೂ
ದೃಷ್ಟಿ-ಯಲ್ಲಿ
ದೃಷ್ಟಿ-ಯ-ಲ್ಲಿ-ಟ್ಟು-ಕೊಂ-ಡಲ್ಲ
ದೃಷ್ಟಿ-ಯಿಂದ
ದೃಷ್ಟಿ-ಯಿಂ-ದಲೂ
ದೃಷ್ಟಿ-ಯಿಂ-ದಲೇ
ದೃಷ್ಟಿ-ಯಿಂ-ದ-ಲ್ಲದೆ
ದೃಷ್ಟಿಯೇ
ದೃಷ್ಟಿ-ಶಕ್ತಿ
ದೃಷ್ಟಿ-ಶ-ಕ್ತಿಯೂ
ದೆಂದರೆ
ದೆಂದು
ದೆಂಬು-ದರ
ದೆನೋ
ದೆಯಾ
ದೆಲ್ಲವೂ
ದೆವ್ವದ
ದೆಸೆ
ದೆಹಲಿ
ದೆಹ-ಲಿಗೆ
ದೆಹ-ಲಿಯ
ದೇಕೆ
ದೇಣಿಗೆ
ದೇನು
ದೇನೋ
ದೇವ
ದೇವ-ಗ-ರ್ಪಿ-ತ-ವಾದ
ದೇವ-ಘರ್
ದೇವತಾ
ದೇವ-ತಾ-ರಾ-ಧ-ನೆಯ
ದೇವ-ತಾ-ಸ್ವ-ರೂ-ಪಿ-ಗಳೂ
ದೇವತೆ
ದೇವ-ತೆ-ಗಳ
ದೇವ-ತೆ-ಗ-ಳಿಗೆ
ದೇವ-ತೆ-ಗಳು
ದೇವ-ತೆ-ಗ-ಳೆಲ್ಲ
ದೇವ-ತೆ-ಯಂತೆ
ದೇವ-ತೆ-ಯಾದ
ದೇವ-ದಾ-ರು-ಪೀ-ತ-ದಾರು
ದೇವ-ದು-ರ್ಲಭ
ದೇವ-ದೂತ
ದೇವ-ದೇ-ವಿ-ಯ-ರಲ್ಲಿ
ದೇವ-ದೇ-ವಿ-ಯ-ರಿಗೆ
ದೇವನ
ದೇವ-ನನ್ನೂ
ದೇವ-ನಾದ
ದೇವನೇ
ದೇವ-ಭೋ-ಗಕ್ಕೆ
ದೇವ-ಭೋ-ಗ-ಕ್ಕೊಮ್ಮೆ
ದೇವ-ಭೋ-ಗದ
ದೇವ-ಭೋ-ಗ-ದಿಂದ
ದೇವ-ಮಂ-ದಿ-ರ-ದಿಂದ
ದೇವ-ಮಾನವ
ದೇವ-ಮಾ-ನ-ವ-ನನ್ನು
ದೇವ-ಮಾ-ನ-ವ-ನೊ-ಬ್ಬನ
ದೇವರ
ದೇವ-ರನ್ನು
ದೇವ-ರನ್ನೇ
ದೇವ-ರ-ಲ್ಲವೆ
ದೇವ-ರಾ-ಗಿ-ದ್ದಾರೆ
ದೇವ-ರಾದ
ದೇವ-ರಾ-ದರೆ
ದೇವ-ರಿಂದ
ದೇವ-ರಿ-ಗಿಂ-ತಲೂ
ದೇವ-ರಿಗೂ
ದೇವ-ರಿಗೆ
ದೇವ-ರಿ-ದ್ದಾನೆ
ದೇವ-ರಿ-ದ್ದಾ-ನೆಂ-ಬು-ದಕ್ಕೆ
ದೇವ-ರಿ-ರು-ವನೇ
ದೇವ-ರಿಲ್ಲ
ದೇವರು
ದೇವ-ರು-ಅ-ವರೇ
ದೇವ-ರು-ಇ-ವ-ರಿ-ಬ್ಬರ
ದೇವ-ರು-ಗಳು
ದೇವ-ರು-ಗಳೇ
ದೇವರೂ
ದೇವ-ರೆಂದು
ದೇವ-ರೆಂದೇ
ದೇವರೇ
ದೇವ-ಲ್ಧಾರ್
ದೇವ-ಶಿಲ್ಪ
ದೇವ-ಸಂ-ತಾ-ನರು
ದೇವ-ಸ-ದೃಶ
ದೇವ-ಸೇ-ವ-ಕ-ರಾದ
ದೇವ-ಸ್ಕರ್
ದೇವ-ಸ್ಥಾನ
ದೇವ-ಸ್ಥಾ-ನಕ್ಕೆ
ದೇವ-ಸ್ಥಾ-ನ-ಗಳ
ದೇವ-ಸ್ಥಾ-ನ-ಗಳನ್ನು
ದೇವ-ಸ್ಥಾ-ನ-ಗ-ಳಿಗೆ
ದೇವ-ಸ್ಥಾ-ನ-ಗಳು
ದೇವ-ಸ್ಥಾ-ನದ
ದೇವ-ಸ್ಥಾ-ನ-ದಲ್ಲಿ
ದೇವ-ಸ್ಥಾ-ನ-ದಲ್ಲೇ
ದೇವ-ಸ್ಥಾ-ನ-ವನ್ನು
ದೇವ-ಸ್ಥಾ-ನ-ವ-ನ್ನೊಮ್ಮೆ
ದೇವ-ಸ್ಥಾ-ನವೇ
ದೇವಾ
ದೇವಾ-ಲಯ
ದೇವಾ-ಲ-ಯಕ್ಕೆ
ದೇವಾ-ಲ-ಯ-ಗಳ
ದೇವಾ-ಲ-ಯ-ಗಳನ್ನು
ದೇವಾ-ಲ-ಯ-ಗ-ಳಿಗೆ
ದೇವಾ-ಲ-ಯ-ಗಳು
ದೇವಾ-ಲ-ಯ-ಗ-ಳು-ಯಾ-ವುವೂ
ದೇವಾ-ಲ-ಯದ
ದೇವಾ-ಲ-ಯ-ದಲ್ಲಿ
ದೇವಾ-ಲ-ಯ-ದಲ್ಲೇ
ದೇವಾ-ಲ-ಯ-ದಿಂದ
ದೇವಾ-ಲ-ಯ-ದೊ-ಳಗೆ
ದೇವಾ-ಲ-ಯ-ವನ್ನು
ದೇವಾ-ಲ-ಯ-ವ-ನ್ನೆಲ್ಲ
ದೇವಾ-ಲ-ಯ-ವಾ-ಗಿತ್ತು
ದೇವಾ-ಲ-ಯ-ವಿ-ರುವ
ದೇವಾ-ಲ-ಯವು
ದೇವಿ
ದೇವಿಯ
ದೇವಿ-ಯ-ರನ್ನು
ದೇವೀ
ದೇವೀ-ಪ-ರ-ವಾದ
ದೇವ್
ದೇಶ
ದೇಶ-ವಿ-ದೇ-ಶ-ಗ-ಳ-ಲ್ಲಿನ
ದೇಶ-ಅ-ಮೆ-ರಿ-ಕ-ದಲ್ಲಿ
ದೇಶ-ಕಾಲ
ದೇಶ-ಕ್ಕಾಗಿ
ದೇಶಕ್ಕೆ
ದೇಶ-ಕ್ಕೆಂಥ
ದೇಶ-ಗಳ
ದೇಶ-ಗಳಲ್ಲಿ
ದೇಶ-ಗ-ಳ-ಲ್ಲಿ-ದ್ದಾ-ಗಲೂ
ದೇಶ-ಗ-ಳ-ಲ್ಲಿನ
ದೇಶ-ಗ-ಳ-ಲ್ಲಿಯೂ
ದೇಶ-ಗ-ಳ-ಲ್ಲೆಲ್ಲ
ದೇಶ-ಗಳಿಂದ
ದೇಶ-ಗ-ಳಿ-ಗಿಂತ
ದೇಶ-ಗ-ಳಿಗೆ
ದೇಶ-ಗಳು
ದೇಶದ
ದೇಶ-ದಲ್ಲಿ
ದೇಶ-ದ-ಲ್ಲಿನ
ದೇಶ-ದ-ಲ್ಲಿಯೂ
ದೇಶ-ದಲ್ಲೂ
ದೇಶ-ದಲ್ಲೇ
ದೇಶ-ದ-ವಳೇ
ದೇಶ-ದಾ-ದ್ಯಂತ
ದೇಶ-ದೇ-ಶಾಂ-ತ-ರ-ಗಳನ್ನು
ದೇಶ-ದ್ರೋ-ಹಿ-ಗಳು
ದೇಶ-ಪ್ರೇಮ
ದೇಶ-ಪ್ರೇ-ಮದ
ದೇಶ-ಪ್ರೇ-ಮವು
ದೇಶ-ಬಾಂ-ಧ-ವ-ರನ್ನು
ದೇಶ-ಬಾಂ-ಧ-ವ-ರಿ-ಗಾಗಿ
ದೇಶ-ಬಾಂ-ಧ-ವ-ರಿಗೂ
ದೇಶ-ಬಾಂ-ಧ-ವ-ರಿಗೆ
ದೇಶ-ಬಾಂ-ಧ-ವರೇ
ದೇಶ-ಭಕ್ತ
ದೇಶ-ಭ-ಕ್ತರೇ
ದೇಶ-ಭ-ಕ್ತಿ-ಯನ್ನೂ
ದೇಶ-ವನ್ನು
ದೇಶ-ವಿ-ದೇ-ಶ-ಗಳ
ದೇಶವು
ದೇಶ-ವೊಂ-ದನ್ನು
ದೇಶಾ
ದೇಹ
ದೇಹ-ಬು-ದ್ಧಿ-ಐ-ಶ್ವ-ರ್ಯ-ಗಳನ್ನು
ದೇಹ-ಎ-ಲ್ಲ-ವನ್ನೂ
ದೇಹಕ್ಕೆ
ದೇಹ-ಗಳು
ದೇಹ-ತ್ಯಾಗ
ದೇಹ-ತ್ಯಾ-ಗದ
ದೇಹದ
ದೇಹ-ದಂ-ಡನೆ
ದೇಹ-ದಂ-ಡ-ನೆಯ
ದೇಹ-ದ-ಲ್ಲಿ-ರು-ವ-ವ-ರೆಗೆ
ದೇಹ-ದೊಂ-ದಿಗೇ
ದೇಹ-ದೊ-ಳಕ್ಕೆ
ದೇಹ-ದೊ-ಳಗೆ
ದೇಹ-ಮಂ-ದಿ-ರ-ದ-ಲ್ಲಿ-ರುವ
ದೇಹ-ವನ್ನು
ದೇಹ-ವನ್ನೂ
ದೇಹ-ವಲ್ಲ
ದೇಹ-ವೇ-ನಿ-ದ್ದರೂ
ದೇಹ-ಶ್ರಮ
ದೇಹ-ಸ್ಥಿತಿ
ದೇಹ-ಸ್ಥಿ-ತಿಯ
ದೇಹ-ಸ್ಥಿ-ತಿ-ಯನ್ನು
ದೇಹ-ಸ್ಥಿ-ತಿ-ಯಲ್ಲಿ
ದೇಹ-ಸ್ಥಿ-ತಿಯು
ದೇಹ-ಸ್ಥಿ-ತಿ-ಯೊಂ-ದಿಗೆ
ದೇಹಾ-ತೀತ
ದೇಹಾ-ರೋಗ್ಯ
ದೇಹಾ-ರೋ-ಗ್ಯದ
ದೇಹಾ-ರೋ-ಗ್ಯ-ವನ್ನು
ದೇಹಾ-ರೋ-ಗ್ಯವು
ದೇಹಾ-ರೋ-ಗ್ಯವೂ
ದೇಹಾ-ರೋ-ಗ್ಯ-ವೊಂದೇ
ದೈತ್ಯಾ-ಕಾ-ರದ
ದೈನಂ-ದಿನ
ದೈನ್ಯ-ತೆಯ
ದೈನ್ಯ-ದಿಂದ
ದೈವ
ದೈವತ್ವ
ದೈವ-ತ್ವದ
ದೈವ-ತ್ವ-ವನ್ನು
ದೈವದ
ದೈವ-ಭಕ್ತಿ
ದೈವ-ಭೀ-ತಿ-ಯು-ಳ್ಳ-ವ-ರಾಗಿ
ದೈವ-ವನ್ನು
ದೈವಾ-ನು-ಗ್ರ-ಹ-ಹೇ-ತು-ಕಮ್
ದೈವಿ-ಕ-ತೆ-ಯನ್ನು
ದೈವೀ
ದೈವೀ-ಗು-ಣದ
ದೈವೀ-ಭಾ-ವ-ದೀ-ಪ್ತಿ-ಯಿಂದ
ದೈವೀ-ಭಾ-ವೋ-ನ್ಮ-ತ್ತ-ರಾ-ಗಿ-ಬಿ-ಡು-ತ್ತಿ-ದ್ದರು
ದೈವೀ-ಶ-ಕ್ತಿ-ಯನ್ನು
ದೈಹಿಕ
ದೈಹಿ-ಕ-ಮಾ-ನ-ಸಿಕ
ದೈಹಿ-ಕ-ವಾಗಿ
ದೈಹಿ-ಕ-ವಾ-ಗಿಯೂ
ದೊಂದಿಗೆ
ದೊಂದು
ದೊಡ್ಡ
ದೊಡ್ಡ-ದಲ್ಲ
ದೊಡ್ಡ-ದಾದ
ದೊಡ್ಡದು
ದೊಡ್ಡ-ದೊಂದು
ದೊಡ್ಡ-ದೊಡ್ಡ
ದೊಡ್ಡ-ಮ-ನು-ಷ್ಯ-ನಾಗಿ
ದೊಡ್ಡ-ಮ-ನು-ಷ್ಯ-ನಾ-ಗಿ-ಬಿಟ್ಟೆ
ದೊಡ್ಡ-ಮ-ನು-ಷ್ಯರು
ದೊಡ್ಡ-ವ-ನಾದ
ದೊಡ್ಡ-ವ-ನಾ-ದಂ-ತೆಲ್ಲ
ದೊಡ್ಡ-ವನು
ದೊಡ್ಡ-ವರು
ದೊಡ್ಡ-ವಾ-ಗ-ಲಾ-ರವು
ದೊಯ್ದರು
ದೊಯ್ಯ-ಲಾ-ಯಿತು
ದೊರ-ಕಲಿ
ದೊರ-ಕ-ಲಿಲ್ಲ
ದೊರ-ಕ-ಲಿ-ಲ್ಲ-ವಲ್ಲ
ದೊರ-ಕ-ಲಿ-ಲ್ಲ-ವೆಂದ
ದೊರ-ಕ-ಲಿ-ಲ್ಲ-ವೆಂಬ
ದೊರ-ಕಿತು
ದೊರ-ಕಿತ್ತು
ದೊರ-ಕಿದ
ದೊರ-ಕಿ-ದಂ-ತಾ-ಯಿತು
ದೊರ-ಕಿ-ದ್ದ-ರಿಂದ
ದೊರ-ಕಿ-ದ್ದುವು
ದೊರ-ಕಿ-ಬಿ-ಟ್ಟರೆ
ದೊರ-ಕಿ-ರ-ಲಿಲ್ಲ
ದೊರ-ಕಿಸಿ
ದೊರ-ಕಿ-ಸಿ-ಕೊಟ್ಟ
ದೊರ-ಕಿ-ಸಿ-ಕೊ-ಡ-ಬಲ್ಲ
ದೊರ-ಕಿ-ಸಿ-ಕೊಡಿ
ದೊರ-ಕಿ-ಸಿ-ಕೊ-ಳ್ಳುವೆ
ದೊರ-ಕೀ-ತೆಂ-ಬು-ದನ್ನು
ದೊರಕು
ದೊರ-ಕು-ತ್ತದೆ
ದೊರ-ಕು-ತ್ತಲೇ
ದೊರ-ಕು-ತ್ತಿ-ರುವ
ದೊರ-ಕು-ವಂ-ತಾ-ಗಿ-ಬಿ-ಟ್ಟರೆ
ದೊರ-ಕು-ವಂ-ತಾ-ಯಿತು
ದೊರ-ಕು-ವುದು
ದೊರೆ
ದೊರೆತ
ದೊರೆ-ತಂ-ತಾ-ಗು-ವುದು
ದೊರೆ-ತ-ದ್ದ-ಕ್ಕಾಗಿ
ದೊರೆ-ತದ್ದು
ದೊರೆ-ತರೂ
ದೊರೆ-ತಿಲ್ಲ
ದೊರೆ-ತೀತು
ದೊರೆ-ಯದ
ದೊರೆ-ಯ-ಬಲ್ಲ
ದೊರೆ-ಯ-ಲಾ-ರ-ದೆಂದು
ದೊರೆ-ಯ-ಲೇ-ಬೇಕು
ದೊರೆ-ಯಿತು
ದೊರೆ-ಯು-ತ್ತದೆ
ದೊರೆ-ಯು-ತ್ತಿದ್ದ
ದೊರೆ-ಯು-ತ್ತಿ-ದ್ದುವು
ದೊರೆ-ಯು-ತ್ತಿ-ದ್ದು-ವೆಂದು
ದೊರೆ-ಯು-ತ್ತಿಲ್ಲ
ದೊರೆ-ಯು-ವಂ-ತಾ-ದದ್ದು
ದೊಳಕ್ಕೇ
ದೊಳಗೆ
ದೋಣಿ
ದೋಣಿ-ಗಳ
ದೋಣಿ-ಗಳಲ್ಲಿ
ದೋಣಿ-ಗಳಿಂದ
ದೋಣಿ-ಗ-ಳಿಗೆ
ದೋಣಿ-ಗಳು
ದೋಣಿಗೆ
ದೋಣಿ-ಮನೆ
ದೋಣಿ-ಮ-ನೆ-ಗಳಲ್ಲಿ
ದೋಣಿ-ಮ-ನೆ-ಗ-ಳಲ್ಲೇ
ದೋಣಿ-ಮ-ನೆ-ಗ-ಳಿಗೆ
ದೋಣಿ-ಮ-ನೆಗೇ
ದೋಣಿ-ಮ-ನೆಯ
ದೋಣಿ-ಮ-ನೆ-ಯನ್ನು
ದೋಣಿಯ
ದೋಣಿ-ಯಲ್ಲಿ
ದೋಣಿ-ಯಲ್ಲೇ
ದೋಣಿ-ಯ-ವನು
ದೋಣಿ-ಯ-ವನೂ
ದೋಣಿ-ಯ-ವರು
ದೋಣಿಯು
ದೋಣಿ-ಯೊಂ-ದ-ರಲ್ಲಿ
ದೋಣಿ-ಯೊಂ-ದ-ರ-ಲ್ಲಿಈ
ದೋಷ
ದೋಷ-ಗಳ
ದೋಷ-ಗಳನ್ನು
ದೋಷ-ಗಳನ್ನೂ
ದೋಷ-ಗ-ಳಾ-ವು-ದನ್ನೂ
ದೋಷ-ಗ-ಳಿಂ-ದೊ-ಡ-ಗೂಡಿ
ದೋಷ-ಗ-ಳಿ-ಗೆಲ್ಲ
ದೋಷ-ಗಳೂ
ದೋಷ-ವನ್ನು
ದೋಷ-ವಾ-ಗು-ತ್ತದೆ
ದೋಷ-ವಿತ್ತು
ದೋಷವೂ
ದೋಷ-ವೆಂದರೆ
ದೌರ್ಬಲ್ಯ
ದೌರ್ಬ-ಲ್ಯ-ಕಾ-ರ-ಕ-ವಾ-ದದ್ದು
ದೌರ್ಬ-ಲ್ಯ-ಕ್ಕಾ-ಗಲಿ
ದೌರ್ಬ-ಲ್ಯ-ಗಳ
ದೌರ್ಬ-ಲ್ಯ-ಗಳನ್ನು
ದೌರ್ಬ-ಲ್ಯ-ಗಳನ್ನೂ
ದೌರ್ಬ-ಲ್ಯ-ಗಳೂ
ದೌರ್ಬ-ಲ್ಯ-ದಿಂದ
ದೌರ್ಬ-ಲ್ಯ-ವನ್ನು
ದೌರ್ಬ-ಲ್ಯ-ವಿದೆ
ದೌರ್ಬ-ಲ್ಯ-ವಿ-ಲ್ಲವೆ
ದೌರ್ಬ-ಲ್ಯವೂ
ದೌರ್ಬ-ಲ್ಯ-ವೆ-ನಿ-ಸಿ-ಕೊ-ಳ್ಳು-ತ್ತದೆ
ದೌರ್ಬ-ಲ್ಯವೇ
ದ್ಗಾರ-ಗಳ
ದ್ಗಾರ-ಗ-ಳೆಲ್ಲ
ದ್ಗೀತೆಯು
ದ್ದಂತೆ
ದ್ದಂತೆಯೇ
ದ್ದಕ್ಕಾಗಿ
ದ್ದನು
ದ್ದನ್ನು
ದ್ದರಿಂದ
ದ್ದರು
ದ್ದರು-ಇದೇ
ದ್ದರೂ
ದ್ದರೆ
ದ್ದರೋ
ದ್ದಳು
ದ್ದವ-ರಲ್ಲಿ
ದ್ದವರು
ದ್ದವರೆಲ್ಲ
ದ್ದವಳು
ದ್ದಾಗ
ದ್ದಾನೆ
ದ್ದಾನೆಂದು
ದ್ದಾನೆಯೋ
ದ್ದಾರಂ-ತಲ್ಲ
ದ್ದಾರೆ
ದ್ದಾರೆಯೆ
ದ್ದಿರಾ-ಅಂತ
ದ್ದೀಯಾ
ದ್ದೀರಾ
ದ್ದೀರಿ
ದ್ದೀರೋ
ದ್ದುದು
ದ್ದುವು
ದ್ದೆಂದರೆ
ದ್ದೇನಿದೆ
ದ್ದೇನೂ
ದ್ದೇನೆ
ದ್ದೇವೆ
ದ್ದೇವೆಂಬ
ದ್ದೇಶ
ದ್ದೇಶ-ಗಳ
ದ್ದೇಶ-ದತ್ತ
ದ್ದೇಶವು
ದ್ದೇಶವೇ
ದ್ದೇಶಿಸಿ
ದ್ದೊಂದು
ದ್ಯಂತ
ದ್ಯಮಿ-ಗಳು
ದ್ಯೋತಕ
ದ್ರವ್ಯ-ಗಳನ್ನು
ದ್ರವ್ಯ-ಗ-ಳು-ಇ-ವು-ಗಳ
ದ್ರಷ್ಟಾ-ರರು
ದ್ರಾವಿಡ
ದ್ರಾವಿ-ಡರು
ದ್ರೋಹ-ಗಳನ್ನು
ದ್ರೋಹ-ವೆ-ಸ-ಗಿದ್ದಾ
ದ್ವಂದ್ವ
ದ್ವಂದ್ವಾ-ರ್ಥ-ಕ್ಕೆ-ಡೆ-ಯಿ-ಲ್ಲ-ದಂತೆ
ದ್ವಾರ-ಗ-ಳಿವೆ
ದ್ವಾರದ
ದ್ವಿಗು-ಣ-ವಾ-ಗು-ವು-ದ-ಲ್ಲವೆ
ದ್ವೀಪಕ್ಕೆ
ದ್ವೀಪ-ಗ-ಳಿಗೆ
ದ್ವೀಪದ
ದ್ವೀಪ-ದಿಂದ
ದ್ವೀಪವ
ದ್ವೀಪ-ವನ್ನೂ
ದ್ವೇಷದ
ದ್ವೇಷ-ದಿಂದ
ದ್ವೇಷ-ದಿಂ-ದಲೊ
ದ್ವೇಷ-ವನ್ನೇ
ದ್ವೇಷ-ಶು-ಷ್ಕ-ಹೃ-ದ-ಯತೆ
ದ್ವೇಷಾ-ಸೂಯೆ
ದ್ವೇಷಾ-ಸೂ-ಯೆ-ಗಳನ್ನು
ದ್ವೇಷಿ-ಸು-ವ-ವರು
ದ್ವೈತ
ದ್ವೈತ-ಅ-ದ್ವೈ-ತ-ಗ-ಳೆ-ರ-ಡಕ್ಕೂ
ದ್ವೈತ-ವಿ-ಶಿ-ಷ್ಟಾ-ದ್ವೈ-ತ-ಅ-ದ್ವೈ-ತ-ಗ-ಳೆ-ಲ್ಲಕ್ಕೂ
ದ್ವೈತದ
ದ್ವೈತ-ಭಾ-ವದ
ದ್ವೈತ-ಭಾ-ವ-ನೆ-ಗ-ಳೆಂಬ
ದ್ವೈತ-ವಾ-ಗಿ-ರಲಿ
ದ್ವೈತ-ವಾ-ದದ
ದ್ವೈತ-ಸಾ-ಧ-ನೆ-ಯಾ-ಗಲಿ
ದ್ವೈತಿ
ದ್ವೈತಿ-ಗಳು
ದ್ವೈತಿ-ಯಾ-ಗಿದ್ದ
ಧಕ್ಕೆ
ಧಕ್ಕೆ-ಯ-ನ್ನುಂಟು
ಧಕ್ಕೆ-ಯಾ-ಗ-ಲಾ-ರ-ದೆಂದು
ಧಕ್ಕೆ-ಯಾ-ಗು-ವಷ್ಟು
ಧಕ್ಕೆ-ಯುಂ-ಟಾ-ಗು-ವಂತೆ
ಧಗ-ಧ-ಗಿ-ಸು-ತ್ತಲೇ
ಧಗ-ಧ-ಗಿ-ಸು-ತ್ತಿತ್ತು
ಧಗ-ಧ-ಗಿ-ಸುವ
ಧಗೆ
ಧಡ-ಕಾ-ಬ-ಢಕ್
ಧನ
ಧನ-ಮಾನ
ಧನ-ದಾ-ಹಿ-ಗ-ಳಾದ
ಧನ-ಸಂ-ಗ್ರಹ
ಧನ-ಸಂ-ಗ್ರ-ಹಣೆ
ಧನ-ಸಂ-ಗ್ರ-ಹ-ಣೆಯ
ಧನ-ಸಂ-ಗ್ರ-ಹ-ಣೆ-ಯಂತೂ
ಧನ-ಸ-ಹಾಯ
ಧನ-ಸ-ಹಾ-ಯ-ವನ್ನು
ಧನಾ
ಧನಾ-ರ್ಜ-ನೆ-ಗಾ-ಗಲಿ
ಧನಿಕ
ಧನ್ಯ
ಧನ್ಯ-ತಾ-ಭಾವ
ಧನ್ಯ-ತೆ-ಗಳ
ಧನ್ಯ-ತೆ-ಯನ್ನು
ಧನ್ಯ-ನಾ-ಗಿ-ರ-ದಿ-ದ್ದಲ್ಲಿ
ಧನ್ಯ-ನಾ-ಗು-ತ್ತಿದ್ದೆ
ಧನ್ಯನೇ
ಧನ್ಯ-ರಾ-ದೆ-ವೆಂದು
ಧನ್ಯರು
ಧನ್ಯರೇ
ಧನ್ಯ-ವಾದ
ಧನ್ಯ-ವಾ-ದ-ಗಳ
ಧನ್ಯ-ವಾ-ದ-ಗ-ಳ-ನ್ನ-ರ್ಪಿ-ಸುವ
ಧನ್ಯ-ವಾ-ದ-ಗಳನ್ನು
ಧನ್ಯ-ವಾ-ದ-ಗಳು
ಧನ್ಯಾ-ತ್ಮ-ರನ್ನು
ಧಮ-ನಿ-ಗಳಲ್ಲಿ
ಧರ-ಮ್ಘ-ರ-ವನ್ನ
ಧರರೂ
ಧರಿ-ಸ-ಬೇ-ಕೆಂದು
ಧರಿ-ಸ-ಲಾ-ರಂ-ಭಿ-ಸಿ-ದಳು
ಧರಿಸಿ
ಧರಿ-ಸಿದ
ಧರಿ-ಸಿದ್ದ
ಧರಿ-ಸಿ-ದ್ದು-ದನ್ನು
ಧರಿ-ಸಿ-ದ್ದೇಕೆ
ಧರಿ-ಸಿ-ಯಾ-ದರೂ
ಧರಿ-ಸು-ತ್ತದೆ
ಧರಿ-ಸು-ತ್ತಾನೆ
ಧರಿ-ಸು-ತ್ತಿ-ದ್ದು-ದನ್ನು
ಧರಿ-ಸು-ವುದು
ಧರ್ಮ
ಧರ್ಮ
ಧರ್ಮ-ಅ-ಧ್ಯಾತ್ಮ
ಧರ್ಮ-ಸಂ-ಸ್ಕೃ-ತಿ-ಗಳ
ಧರ್ಮ-ಸಂ-ಸ್ಕೃ-ತಿ-ಗಳು
ಧರ್ಮ-ಸಂ-ಸ್ಕೃ-ತಿಯ
ಧರ್ಮ-ಇವು
ಧರ್ಮ-ಕಾ-ರ್ಯಕ್ಕೆ
ಧರ್ಮ-ಕ್ಕಾಗಿ
ಧರ್ಮ-ಕ್ಕಾ-ಗಿ-ಮಾ-ನ-ವ-ತೆ-ಗಾಗಿ
ಧರ್ಮಕ್ಕೆ
ಧರ್ಮ-ಕ್ಷೇ-ತ್ರವು
ಧರ್ಮ-ಗಳ
ಧರ್ಮ-ಗ-ಳ-ಲ್ಲಿನ
ಧರ್ಮ-ಗಳು
ಧರ್ಮ-ಗಳೂ
ಧರ್ಮ-ಗು-ರು-ಗ-ಳಾದ
ಧರ್ಮ-ಗು-ರು-ವ-ನ್ನಾಗಿ
ಧರ್ಮ-ಗ್ರಂ-ಥ-ಗ-ಳ-ಲ್ಲ-ಡ-ಗಿ-ರುವ
ಧರ್ಮ-ಗ್ರಂ-ಥ-ಗ-ಳ-ಲ್ಲಿ-ರುವ
ಧರ್ಮ-ಗ್ರಂ-ಥ-ವೆಂದು
ಧರ್ಮ-ಚ-ರಿತ
ಧರ್ಮ-ಛ-ತ್ರ-ವ-ನ್ನಾಗಿ
ಧರ್ಮ-ತತ್ವ
ಧರ್ಮದ
ಧರ್ಮ-ದಲ್ಲಿ
ಧರ್ಮ-ದಿ-ಗಂ-ತ-ದಲ್ಲಿ
ಧರ್ಮ-ದೂ-ತ-ರ-ನ್ನಾಗಿ
ಧರ್ಮ-ನೌ-ಕೆಯು
ಧರ್ಮ-ಪಾ-ಲರ
ಧರ್ಮ-ಪಾ-ಲ-ರನ್ನು
ಧರ್ಮ-ಪಾ-ಲರು
ಧರ್ಮ-ಪಾ-ಲ-ರೊ-ಬ್ಬರೇ
ಧರ್ಮ-ಪ್ರ-ಚಾರ
ಧರ್ಮ-ಪ್ರ-ಚಾ-ರ-ಕ-ನಾಗಿ
ಧರ್ಮ-ಪ್ರ-ಚಾ-ರ-ಕ-ರಂ-ತೆಯೇ
ಧರ್ಮ-ಪ್ರ-ಚಾ-ರ-ಕ್ಕಾ-ಗಲಿ
ಧರ್ಮ-ಪ್ರ-ಧಾ-ನ-ವಾ-ದುದು
ಧರ್ಮ-ಪ್ರ-ಸಾರ
ಧರ್ಮ-ಪ್ರ-ಸಾ-ರ-ಕ-ನಂ-ತಲ್ಲ
ಧರ್ಮ-ಪ್ರ-ಸಾ-ರ-ಕ್ಕಾಗಿ
ಧರ್ಮ-ಪ್ರ-ಸಾ-ರಕ್ಕೆ
ಧರ್ಮ-ಬು-ದ್ಧಿ-ಯಾ-ಗಲಿ
ಧರ್ಮ-ಬೋ-ಧ-ಕ-ರಾಗಿ
ಧರ್ಮ-ಬೋ-ಧನೆ
ಧರ್ಮ-ಬೋ-ಧ-ನೆಯು
ಧರ್ಮ-ರ-ಕ್ಷ-ಕರ
ಧರ್ಮ-ರ-ಕ್ಷ-ಕ-ರೆ-ನ್ನಿ-ಸಿ-ಕೊಂ-ಡ-ವರು
ಧರ್ಮ-ವಂತೂ
ಧರ್ಮ-ವನ್ನು
ಧರ್ಮ-ವನ್ನೂ
ಧರ್ಮ-ವನ್ನೇ
ಧರ್ಮ-ವಲ್ಲ
ಧರ್ಮ-ವಾಗಿ
ಧರ್ಮ-ವಿ-ರು-ವುದು
ಧರ್ಮವು
ಧರ್ಮವೂ
ಧರ್ಮ-ವೆಂದರೆ
ಧರ್ಮ-ವೆಂ-ದ-ರೇನು
ಧರ್ಮ-ವೆಂದು
ಧರ್ಮ-ವೆಂಬ
ಧರ್ಮ-ವೆಂ-ಬುದು
ಧರ್ಮವೇ
ಧರ್ಮ-ವೇ-ದಾಂ-ತದ
ಧರ್ಮ-ಶಾಲಾ
ಧರ್ಮ-ಶಾ-ಸ್ತ್ರ-ಗಳು
ಧರ್ಮ-ಸಂ-ಬಂ-ಧ-ವಾಗಿ
ಧರ್ಮ-ಸಂ-ಸ್ಥಾ-ಪ-ಕನೂ
ಧರ್ಮ-ಸಂ-ಸ್ಥಾ-ಪ-ನಾ-ಚಾ-ರ್ಯರು
ಧರ್ಮ-ಸ-ಮ್ಮೇ-ಳ-ನಕ್ಕೆ
ಧರ್ಮ-ಸ-ಮ್ಮೇ-ಳ-ನ-ದಲ್ಲಿ
ಧರ್ಮ-ಸಾ-ಧನೆ
ಧರ್ಮ-ಸೂ-ರ್ಯ-ನೀಗ
ಧರ್ಮಸ್ಯ
ಧರ್ಮಾ-ನು-ಷ್ಠಾ-ನದ
ಧರ್ಮಿ-ಷ್ಠ-ರನ್ನು
ಧವಳ
ಧಾಟಿ-ಯಲ್ಲಿ
ಧಾಟಿ-ಯಲ್ಲೇ
ಧಾಟಿಯೂ
ಧಾಮ
ಧಾರ-ಗಳ
ಧಾರಣೆ
ಧಾರಾ-ಕಾ-ರ-ವಾಗಿ
ಧಾರಾ-ಳ-ವಾಗಿ
ಧಾರಾ-ವಾ-ಹಿ-ಯಾಗಿ
ಧಾರಿ
ಧಾರಿ-ಯಾಗಿ
ಧಾರೆ
ಧಾರೆ-ಯನ್ನು
ಧಾರೆ-ಯಾಗಿ
ಧಾರೆ-ಯೆ-ರೆದ
ಧಾರೆ-ಯೆ-ರೆ-ದಿ-ದ್ದರು
ಧಾರೆ-ಯೆ-ರೆ-ದಿ-ದ್ದಾ-ರೆಂಬು
ಧಾರೆ-ಯೆ-ರೆದು
ಧಾರೆ-ಹ-ರಿ-ಸುತ್ತ
ಧಾರೆ-ಹ-ರಿ-ಸು-ತ್ತಿ-ದ್ದರು
ಧಾರ್ಮಿಕ
ಧಾರ್ಮಿ-ಕ-ಆ-ಧ್ಯಾ-ತ್ಮಿಕ
ಧಾರ್ಮಿ-ಕ-ನಾ-ಗಿದ್ದೂ
ಧಾರ್ಮಿ-ಕ-ರಾದ
ಧಾವಿ-ಸ-ತೊ-ಡ-ಗಿ-ದರು
ಧಾವಿ-ಸ-ಬೇ-ಕು-ಅ-ದ-ಕ್ಕಾಗಿ
ಧಾವಿಸಿ
ಧಾವಿ-ಸಿ-ದರು
ಧಾವಿ-ಸಿ-ದಳು
ಧಾವಿ-ಸು-ತ್ತಿ-ದ್ದರು
ಧಾವಿ-ಸು-ತ್ತಿ-ದ್ದರೆ
ಧಾವಿ-ಸುವ
ಧಿಕ್ಕಾ-ರ-ವಿ-ರಲಿ
ಧಿಮಾಕು
ಧಿಯ
ಧೀಮಂತ
ಧೀಮಂ-ತರು
ಧೀಮಂ-ತರೂ
ಧೀಮಂ-ತಿ-ಕೆ-ಯನ್ನು
ಧೀರ
ಧೀರತೆ
ಧೀರ-ತೆ-ಯನ್ನು
ಧೀರ-ನೆ-ದೆಯ
ಧೀರ-ಪು-ತ್ರ-ನಿಗೆ
ಧೀರರ
ಧೀರ-ರಾ-ಗ-ಬೇಕು
ಧೀರ-ರಾದ
ಧೀರ-ರಾ-ದರೂ
ಧೀರರೇ
ಧೀರ-ವಾಣಿ
ಧೀರಾಃ
ಧೀರಾ-ಗ್ರ-ಣಿ-ಗಳ
ಧೀರಾ-ತ್ಮಳು
ಧುನಿಯ
ಧುಮುಕಿ
ಧುಮು-ಕಿ-ದೆನೊ
ಧುಮುಕು
ಧುಮು-ಕುವ
ಧುರೀ-ಣ-ರಲ್ಲಿ
ಧುರೀ-ಣರು
ಧುರೀ-ಣ-ರೆ-ಲ್ಲರೂ
ಧುರೀ-ಣ-ರೊಂ-ದಿಗೆ
ಧೂಪ-ದೀ-ಪ-ಭ-ಜ-ನೆ-ಗಳ
ಧೂಪ-ಫ-ಲ-ಪು-ಷ್ಪ
ಧೂಮ-ಪಾನ
ಧೂಳೀ-ದೂ-ಸ-ರ-ಗೈವ
ಧೂಳು
ಧೃತಿ-ಗೆಟ್ಟು
ಧೈರ್ಯ
ಧೈರ್ಯ-ಗುಂ-ದ-ದಿರಿ
ಧೈರ್ಯ-ಗೊಂಡ
ಧೈರ್ಯ-ದಿಂದ
ಧೈರ್ಯ-ವಂ-ತ-ರ-ನ್ನಾ-ಗಿ-ಸುವ
ಧೈರ್ಯ-ವಂ-ತ-ಳಾಗು
ಧೈರ್ಯ-ವನ್ನು
ಧೈರ್ಯ-ವಾ-ಗ-ಲಿಲ್ಲ
ಧೈರ್ಯ-ವಾಗಿ
ಧೈರ್ಯವೋ
ಧೈರ್ಯ-ಶಾ-ಲಿ-ಗ-ಳಾ-ಗ-ಬೇಕು
ಧೈರ್ಯ-ಶಾ-ಲಿ-ಗ-ಳಾಗಿ
ಧೋರ-ಣೆ-ಗಳ
ಧೋರ-ಣೆ-ಗಾಗಿ
ಧೋರ-ಣೆಯೂ
ಧ್ಯಯನ
ಧ್ಯವೇ
ಧ್ಯಾನ
ಧ್ಯಾನ-ಸಾ-ಧ-ನೆ-ಗಳ
ಧ್ಯಾನ-ಇ-ವೆ-ಲ್ಲವೂ
ಧ್ಯಾನ-ಕಾರ್ಯ
ಧ್ಯಾನ-ಕ್ಕಾಗಿ
ಧ್ಯಾನಕ್ಕೆ
ಧ್ಯಾನಕ್ಕೋ
ಧ್ಯಾನ-ಜೀ-ವ-ನಕ್ಕೆ
ಧ್ಯಾನದ
ಧ್ಯಾನ-ದಲ್ಲಿ
ಧ್ಯಾನ-ದಲ್ಲೂ
ಧ್ಯಾನ-ದಿಂ-ದುಂ-ಟಾದ
ಧ್ಯಾನ-ನಿ-ರ-ತ-ರಾ-ಗಿದ್ದು
ಧ್ಯಾನ-ಬ-ಲ-ದಿಂದ
ಧ್ಯಾನ-ಭಾ-ವ-ದಲ್ಲಿ
ಧ್ಯಾನ-ಭಾ-ವ-ದಿಂದ
ಧ್ಯಾನ-ಮಗ್ನ
ಧ್ಯಾನ-ಮ-ಗ್ನ-ರಾ-ಗಲು
ಧ್ಯಾನ-ಮ-ಗ್ನ-ರಾಗಿ
ಧ್ಯಾನ-ಮ-ಗ್ನ-ರಾ-ಗಿ-ದ್ದು-ದನ್ನು
ಧ್ಯಾನ-ಮ-ಗ್ನ-ರಾ-ಗು-ತ್ತಿ-ದ್ದರು
ಧ್ಯಾನ-ಮ-ಗ್ನ-ರಾ-ಗು-ತ್ತಿ-ದ್ದು-ದನ್ನು
ಧ್ಯಾನ-ಮ-ಗ್ನ-ರಾ-ದರು
ಧ್ಯಾನ-ಮಯ
ಧ್ಯಾನ-ಮಾ-ಡು-ವಂತೆ
ಧ್ಯಾನ-ಮು-ದ್ರೆ-ಯಲ್ಲಿ
ಧ್ಯಾನ-ಲೀ-ನ-ವಾ-ಗಿ-ರಲು
ಧ್ಯಾನವೇ
ಧ್ಯಾನ-ಶೀಲ
ಧ್ಯಾನ-ಸಿ-ದ್ಧ-ರಾದ
ಧ್ಯಾನ-ಸಿ-ದ್ಧರೂ
ಧ್ಯಾನ-ಸ್ಥ-ರಾಗಿ
ಧ್ಯಾನ-ಸ್ಥ-ರಾ-ದರು
ಧ್ಯಾನಾ
ಧ್ಯಾನಾ-ದಿ-ಗಳನ್ನು
ಧ್ಯಾನಾ-ದಿ-ಗಳು
ಧ್ಯಾನಾ-ನಂ-ದ-ದಲ್ಲಿ
ಧ್ಯಾನಾ-ಭ್ಯಾಸ
ಧ್ಯಾನಾ-ಭ್ಯಾ-ಸಕ್ಕೆ
ಧ್ಯಾನಿಸಿ
ಧ್ಯಾನಿಸು
ಧ್ಯಾನ್
ಧ್ಯೇಯ
ಧ್ಯೇಯಕ್ಕೆ
ಧ್ಯೇಯ-ಗಳಾ
ಧ್ಯೇಯ-ಧೋ-ರ-ಣೆ-ಗ-ಳನ್ನೇ
ಧ್ಯೇಯ-ವನ್ನು
ಧ್ಯೇಯ-ವಿ-ರ-ಬೇಕು
ಧ್ಯೇಯ-ವೆಂ-ಬುದು
ಧ್ಯೇಯ-ವೊಂ-ದನ್ನು
ಧ್ಯೇಯೋ-ದ್ದೇಶ
ಧ್ಯೇಯೋ-ದ್ದೇ-ಶ-ಗಳ
ಧ್ಯೇಯೋ-ದ್ದೇ-ಶ-ಗಳನ್ನು
ಧ್ಯೇಯೋ-ದ್ದೇ-ಶ-ಗಳಲ್ಲಿ
ಧ್ಯೇಯೋ-ದ್ದೇ-ಶ-ಗಳು
ಧ್ರುವ-ಗ-ಳಷ್ಟು
ಧ್ವಜ-ಪ-ತಾ-ಕೆ-ಗಳೂ
ಧ್ವಜ-ದ-ಲ್ಲಿ-ರುವ
ಧ್ವಜ-ವನ್ನು
ಧ್ವನಿ
ಧ್ವನಿ-ಮು-ದ್ರಿ-ಕೆ-ಗಳು
ಧ್ವನಿಯ
ಧ್ವನಿ-ಯನ್ನು
ಧ್ವನಿಯು
ಧ್ವನಿ-ವ-ರ್ಧಕ
ಧ್ವನಿ-ಸು-ತ್ತಿತ್ತು
ನ
ನಂಗುವೆ
ನಂಜುಂ-ಡ-ರಾವ್
ನಂಟು
ನಂತರ
ನಂತ-ರದ
ನಂತ-ರವೇ
ನಂತೆ
ನಂತೆಯೇ
ನಂದ
ನಂದಜೀ
ನಂದನಾ
ನಂದ-ನೆ-ಗಳು
ನಂದರ
ನಂದ-ರನ್ನು
ನಂದ-ರಿಗೆ
ನಂದ-ರಿ-ದ್ದಾ-ರೆಯೇ
ನಂದರು
ನಂದರೂ
ನಂದ-ರೆಂ-ದರು
ನಂದರೇ
ನಂದ-ಲೀ-ನ-ರಾ-ಗು-ತ್ತಿ-ದ್ದರು
ನಂದಾ-ದೀ-ಪ-ಗಳನ್ನು
ನಂದಿ-ಸ-ಲ್ಪ-ಡ-ದಿ-ದ್ದರೆ
ನಂಬ-ಬ-ಹು-ದಾದ
ನಂಬ-ಲಾ-ಗ-ಲಿಲ್ಲ
ನಂಬ-ಲಾ-ಗಿತ್ತು
ನಂಬಲು
ನಂಬ-ಲುಂಟೆ
ನಂಬಿ
ನಂಬಿಕೆ
ನಂಬಿ-ಕೆ-ಗ-ನು-ಸಾ-ರ-ವಾ-ಗಿಯೇ
ನಂಬಿ-ಕೆ-ಗಳ
ನಂಬಿ-ಕೆ-ಗಳನ್ನು
ನಂಬಿ-ಕೆ-ಗಳಿಂದ
ನಂಬಿ-ಕೆ-ಗ-ಳಿಗೆ
ನಂಬಿ-ಕೆ-ಗ-ಳಿ-ದ್ದುವು
ನಂಬಿ-ಕೆ-ಗಳು
ನಂಬಿ-ಕೆ-ಗ-ಳು-ಪೂ-ರ್ವ-ಗ್ರ-ಹ-ಗಳು
ನಂಬಿ-ಕೆಗೆ
ನಂಬಿ-ಕೆಯ
ನಂಬಿ-ಕೆ-ಯ-ನ್ನಾ-ಗಲಿ
ನಂಬಿ-ಕೆ-ಯನ್ನು
ನಂಬಿ-ಕೆ-ಯಿಂ-ದಲೇ
ನಂಬಿ-ಕೆ-ಯಿ-ದ್ದರೆ
ನಂಬಿ-ಕೆ-ಯಿ-ರಲಿ
ನಂಬಿ-ಕೆ-ಯಿ-ರ-ಲಿಲ್ಲ
ನಂಬಿ-ಕೆ-ಯಿ-ರುವ
ನಂಬಿ-ಕೆ-ಯುಂ-ಟಾ-ಗ-ಲಿಲ್ಲ
ನಂಬಿ-ಕೆ-ಯುಂ-ಟು-ಮಾ-ಡಿ-ಸುವ
ನಂಬಿ-ಕೆಯೂ
ನಂಬಿ-ಕೊಂ-ಡಿ-ದ್ದೀ-ರಿ-ಅದು
ನಂಬಿ-ಕೊಂಡು
ನಂಬಿ-ದಳು
ನಂಬಿದ್ದ
ನಂಬಿ-ದ್ದರು
ನಂಬಿ-ದ್ದಳು
ನಂಬಿ-ದ್ದೀರೋ
ನಂಬಿದ್ದೇ
ನಂಬಿ-ದ್ದೇನೆ
ನಂಬಿ-ದ್ದೇವೆ
ನಂಬು
ನಂಬು-ತ್ತಾನೆ
ನಂಬು-ತ್ತಾರೆ
ನಂಬು-ತ್ತೇನೆ
ನಂಬುವ
ನಂಬು-ವ-ವ-ರಲ್ಲ
ನಂಬು-ವ-ವರು
ನಂಬು-ವು-ದಕ್ಕೇ
ನಂಬು-ವು-ದೇ-ನೆಂ-ದರೆ
ನಕಾ-ರಾ-ತ್ಮ-ಕ-ವಾದ
ನಕ್ಕರು
ನಕ್ಕ-ರೆಂ-ದರೆ
ನಕ್ಕು
ನಕ್ಕು-ನ-ಗಿ-ಸು-ವು-ದುಂಟು
ನಕ್ಕು-ಬಿ-ಟ್ಟರು
ನಕ್ಷತ್ರ
ನಕ್ಷ-ತ್ರ-ಗಳ
ನಕ್ಷ-ತ್ರ-ಗ-ಳಂತೆ
ನಕ್ಷ-ತ್ರ-ಗಳನ್ನು
ನಕ್ಷ-ತ್ರ-ಗಳನ್ನೂ
ನಕ್ಷ-ತ್ರ-ಗ-ಳಿಗೆ
ನಕ್ಷ-ತ್ರ-ಗಳೇ
ನಕ್ಷ-ತ್ರ-ಮ-ಕು-ಟ-ಗಳ
ನಕ್ಷ-ತ್ರಾ-ಣ್ಯ-ಪ್ಸ-ರಸೋ
ನಕ್ಷೆ
ನಖ-ಶಿ-ಖಾಂತ
ನಗರ
ನಗ-ರ-ಪ-ಟ್ಟ-ಣ-ಗಳ
ನಗ-ರಕ್ಕೆ
ನಗ-ರ-ಗಳ
ನಗ-ರ-ಗಳನ್ನು
ನಗ-ರ-ಗಳಲ್ಲಿ
ನಗ-ರ-ಗ-ಳ-ಲ್ಲೊಂದು
ನಗ-ರ-ಗ-ಳಾದ
ನಗ-ರ-ಗಳಿಂದ
ನಗ-ರದ
ನಗ-ರ-ದ-ರ್ಶನ
ನಗ-ರ-ದಲ್ಲಿ
ನಗ-ರ-ದ-ಲ್ಲಿ-ದ್ದಳು
ನಗ-ರ-ದ-ಲ್ಲಿಯೇ
ನಗ-ರ-ದ-ಲ್ಲಿ-ರ-ಲಿಲ್ಲ
ನಗ-ರ-ದಲ್ಲೂ
ನಗ-ರ-ದಲ್ಲೆಲ್ಲ
ನಗ-ರ-ದಲ್ಲೇ
ನಗ-ರ-ದ-ವ-ರೆಗೂ
ನಗ-ರ-ದಿಂದ
ನಗ-ರ-ದೆ-ಡೆಗೆ
ನಗ-ರ-ಪ್ರ-ವಾ-ಸ-ದಲ್ಲಿ
ನಗ-ರ-ವನ್ನು
ನಗ-ರ-ವನ್ನೂ
ನಗ-ರ-ವಾದ
ನಗ-ರ-ವಿದ್ದ
ನಗ-ರವು
ನಗ-ರವೂ
ನಗ-ರವೇ
ನಗಿ-ಸು-ವುದನ್ನು
ನಗು
ನಗುತ್ತ
ನಗು-ತ್ತಿ-ದ್ದಂತೆ
ನಗು-ತ್ತಿ-ದ್ದರು
ನಗು-ನ-ಗುತ್ತ
ನಗು-ಬಂ-ದಿತು
ನಗು-ಮು-ಖ-ದಿಂದ
ನಗು-ಮೊ-ಗ-ದಿಂದ
ನಗು-ವ-ಅ-ಳುವ
ನಗು-ವನ್ನು
ನಗು-ವಿನ
ನಗು-ವಿ-ನಂತೆ
ನಗು-ವುದು
ನಗೆ-ಗ-ಡ-ಲಲ್ಲಿ
ನಗೆ-ಗ-ಡ-ಲಿ-ನಲ್ಲಿ
ನಗ್ನ
ನಗ್ನ-ರೂ-ಪ-ವನ್ನು
ನಚಿ-ಕೇ-ತ-ನಂತೆ
ನಜರ್
ನಟ-ಕೃಷ್ಣ
ನಟ-ನ-ಟಿ-ಯರು
ನಟಿ
ನಟಿ-ಸು-ವ-ವ-ರಿ-ಗಲ್ಲ
ನಟೇಶ್
ನಡ-ವ-ಳಿಕೆ
ನಡ-ವ-ಳಿ-ಕೆ-ಗ-ಳಿಗೆ
ನಡ-ವ-ಳಿ-ಕೆಯ
ನಡಿ-ದರು
ನಡು
ನಡುಕ
ನಡು-ಗಿದ
ನಡು-ಗುತ್ತ
ನಡು-ದಾರಿ-ಯಲ್ಲಿ
ನಡು-ನ-ಡುವೆ
ನಡು-ನೆ-ತ್ತಿಯ
ನಡು-ಭಾಗ
ನಡು-ರಾ-ತ್ರಿ-ಯಲ್ಲಿ
ನಡು-ವಣ
ನಡು-ವಿನ
ನಡುವೆ
ನಡು-ವೆಯೂ
ನಡು-ವೆಯೇ
ನಡೆ
ನಡೆದ
ನಡೆ-ದದ್ದು
ನಡೆ-ದ-ರ-ವರು
ನಡೆ-ದರು
ನಡೆ-ದಳು
ನಡೆ-ದವು
ನಡೆ-ದಾಗ
ನಡೆ-ದಾಟ
ನಡೆ-ದಾ-ಟ-ವನ್ನು
ನಡೆ-ದಾ-ಡ-ತೊ-ಡ-ಗಿ-ದರು
ನಡೆ-ದಾಡಿ
ನಡೆ-ದಾ-ಡುತ್ತ
ನಡೆ-ದಾ-ಡು-ತ್ತಲೇ
ನಡೆ-ದಾ-ಡು-ತ್ತಿ-ದ್ದರು
ನಡೆ-ದಾ-ಡು-ತ್ತಿ-ದ್ದ-ರೆಂ-ದರೆ
ನಡೆ-ದಿತ್ತು
ನಡೆ-ದಿದೆ
ನಡೆ-ದಿ-ದೆ-ನ-ಡೆ-ಯು-ತ್ತಿದೆ
ನಡೆ-ದಿದ್ದ
ನಡೆ-ದಿ-ದ್ದಂತೆ
ನಡೆ-ದಿ-ದ್ದುದು
ನಡೆ-ದಿ-ದ್ದುವು
ನಡೆ-ದೀತು
ನಡೆದು
ನಡೆ-ದು-ಕೊಂ-ಡರೂ
ನಡೆ-ದು-ಕೊಂಡು
ನಡೆ-ದು-ಕೊಂ-ಡು-ಬಂ-ದಿತು
ನಡೆ-ದು-ಕೊಂಡೇ
ನಡೆ-ದು-ಕೊ-ಳ್ಳ-ಲಾ-ರಂ-ಭಿ-ಸಿದ
ನಡೆ-ದು-ಕೊ-ಳ್ಳಲು
ನಡೆ-ದು-ಕೊ-ಳ್ಳು-ತ್ತಿ-ದ್ದ-ರೆಂ-ದಲ್ಲ
ನಡೆ-ದು-ಕೊ-ಳ್ಳು-ವಂತೆ
ನಡೆ-ದು-ದನ್ನು
ನಡೆ-ದು-ಬಂ-ದರು
ನಡೆ-ದು-ಬಂ-ದಿತ್ತು
ನಡೆ-ದು-ಬ-ರಲು
ನಡೆ-ದು-ಬ-ರು-ತ್ತಿ-ದ್ದಂತೆ
ನಡೆ-ದು-ಬ-ರು-ತ್ತಿ-ದ್ದಳು
ನಡೆ-ದು-ಬಿ-ಟ್ಟರು
ನಡೆ-ದುವು
ನಡೆ-ದು-ಹೋ-ಗು-ತ್ತವೆ
ನಡೆ-ದು-ಹೋದ
ನಡೆ-ದು-ಹೋ-ದರು
ನಡೆ-ದು-ಹೋ-ದುವು
ನಡೆ-ದು-ಹೋ-ಯಿತು
ನಡೆ-ದು-ಹೋ-ಯಿತೋ
ನಡೆದೇ
ನಡೆ-ದೇನು
ನಡೆ-ದೇ-ಬಿ-ಟ್ಟರು
ನಡೆ-ನು-ಡಿ-ಗಳನ್ನು
ನಡೆ-ನು-ಡಿ-ಗಳನ್ನೂ
ನಡೆ-ನು-ಡಿ-ಗ-ಳೆಲ್ಲ
ನಡೆ-ನು-ಡಿ-ಯಲ್ಲಿ
ನಡೆಯ
ನಡೆ-ಯದು
ನಡೆ-ಯ-ಬ-ಲ್ಲುದು
ನಡೆ-ಯ-ಬಾ-ರ-ದೆಂ-ಬುದು
ನಡೆ-ಯ-ಬೇ-ಕಾ-ಗಿ-ದೆ-ಯೆಂದು
ನಡೆ-ಯ-ಬೇ-ಕಾದ
ನಡೆ-ಯ-ಬೇ-ಕಾ-ದಂ-ಥವು
ನಡೆ-ಯ-ಬೇಕು
ನಡೆ-ಯ-ಬೇ-ಕೆಂಬು
ನಡೆ-ಯ-ಲಾ-ರಂಭಿ
ನಡೆ-ಯಲಿ
ನಡೆ-ಯ-ಲಿತ್ತು
ನಡೆ-ಯ-ಲಿ-ದೆಯೊ
ನಡೆ-ಯ-ಲಿದ್ದ
ನಡೆ-ಯ-ಲಿಲ್ಲ
ನಡೆ-ಯ-ಲಿವೆ
ನಡೆ-ಯಲು
ನಡೆ-ಯಲೆ
ನಡೆ-ಯ-ಲೇ-ಬೇಕು
ನಡೆ-ಯಿತು
ನಡೆ-ಯಿ-ತೆಂ-ದರೆ
ನಡೆ-ಯಿತೋ
ನಡೆ-ಯಿರಿ
ನಡೆ-ಯಿ-ಸಲು
ನಡೆ-ಯಿ-ಸಿದ
ನಡೆಯು
ನಡೆ-ಯುತ್ತ
ನಡೆ-ಯು-ತ್ತದೆ
ನಡೆ-ಯು-ತ್ತ-ದೆ-ಯಂತೆ
ನಡೆ-ಯು-ತ್ತಲೇ
ನಡೆ-ಯು-ತ್ತಿತ್ತು
ನಡೆ-ಯು-ತ್ತಿದೆ
ನಡೆ-ಯು-ತ್ತಿ-ದೆಯೊ
ನಡೆ-ಯು-ತ್ತಿದ್ದ
ನಡೆ-ಯು-ತ್ತಿ-ದ್ದಂ-ತೆಯೇ
ನಡೆ-ಯು-ತ್ತಿ-ದ್ದರು
ನಡೆ-ಯು-ತ್ತಿ-ದ್ದಾಗ
ನಡೆ-ಯು-ತ್ತಿ-ದ್ದುದು
ನಡೆ-ಯು-ತ್ತಿ-ದ್ದುವು
ನಡೆ-ಯು-ತ್ತಿದ್ದೆ
ನಡೆ-ಯು-ತ್ತಿ-ರ-ಲಿಲ್ಲ
ನಡೆ-ಯು-ತ್ತಿ-ರುವ
ನಡೆ-ಯು-ತ್ತಿ-ರು-ವಾಗ
ನಡೆ-ಯು-ತ್ತಿ-ರು-ವುದನ್ನು
ನಡೆ-ಯು-ತ್ತಿಲ್ಲ
ನಡೆ-ಯು-ತ್ತಿ-ವೆ-ಯೆಂ-ಬು-ದರ
ನಡೆ-ಯುವ
ನಡೆ-ಯು-ವಂ-ತಾ-ಗ-ಬೇ-ಕೆಂ-ಬುದು
ನಡೆ-ಯು-ವಂ-ತಾ-ದುದು
ನಡೆ-ಯು-ವಂತೆ
ನಡೆ-ಯು-ವ-ವರು
ನಡೆ-ಯು-ವ-ಷ್ಟ-ರಲ್ಲಿ
ನಡೆ-ಯು-ವು-ದಕ್ಕೆ
ನಡೆ-ಯು-ವುದನ್ನು
ನಡೆ-ಯು-ವುದು
ನಡೆ-ಯು-ವು-ದೆಂದು
ನಡೆ-ಯು-ವುದೇ
ನಡೆ-ಯು-ವುವು
ನಡೆ-ಸದೆ
ನಡೆ-ಸ-ಬೇ-ಕಾ-ಗು-ತ್ತದೆ
ನಡೆ-ಸ-ಬೇಕು
ನಡೆ-ಸ-ಬೇ-ಕೆಂದು
ನಡೆ-ಸ-ಬೇ-ಕೆಂಬ
ನಡೆ-ಸ-ಬೇ-ಕೆಂ-ಬುದು
ನಡೆ-ಸ-ಬೇ-ಕೆ-ನ್ನು-ವು-ದರ
ನಡೆ-ಸ-ಲಾ-ಗು-ವು-ದೆಂದು
ನಡೆ-ಸ-ಲಾ-ಗು-ವು-ದೆಂದೂ
ನಡೆ-ಸ-ಲಾ-ಯಿತು
ನಡೆ-ಸ-ಲಾ-ರಂ-ಭಿ-ಸಿ-ದರು
ನಡೆ-ಸ-ಲಿದ್ದ
ನಡೆ-ಸಲು
ನಡೆ-ಸ-ಲು-ದ್ದೇ-ಶಿ-ಸ-ಲಾ-ಗಿದ್ದ
ನಡೆಸಿ
ನಡೆ-ಸಿ-ಕೊಂ-ಡರೂ
ನಡೆ-ಸಿ-ಕೊಂಡು
ನಡೆ-ಸಿ-ಕೊ-ಟ್ಟಿದ್ದ
ನಡೆ-ಸಿ-ಕೊಟ್ಟು
ನಡೆ-ಸಿ-ಕೊ-ಡು-ವಂತೆ
ನಡೆ-ಸಿದ
ನಡೆ-ಸಿ-ದಂತೆ
ನಡೆ-ಸಿ-ದರು
ನಡೆ-ಸಿ-ದರೆ
ನಡೆ-ಸಿ-ದ-ವರು
ನಡೆ-ಸಿ-ದುದು
ನಡೆ-ಸಿ-ದುವು
ನಡೆ-ಸಿ-ದ್ದರು
ನಡೆಸು
ನಡೆ-ಸುತ್ತ
ನಡೆ-ಸು-ತ್ತಲೂ
ನಡೆ-ಸು-ತ್ತಿದ್ದ
ನಡೆ-ಸು-ತ್ತಿ-ದ್ದಂ-ತಹ
ನಡೆ-ಸು-ತ್ತಿ-ದ್ದಂತೆ
ನಡೆ-ಸು-ತ್ತಿ-ದ್ದರು
ನಡೆ-ಸು-ತ್ತಿ-ದ್ದ-ರು-ತ-ನ್ಮೂ-ಲಕ
ನಡೆ-ಸು-ತ್ತಿ-ದ್ದಳು
ನಡೆ-ಸು-ತ್ತಿ-ದ್ದ-ವ-ಳ-ಲ್ಲವೆ
ನಡೆ-ಸು-ತ್ತಿ-ದ್ದಾ-ರೆಂದು
ನಡೆ-ಸು-ತ್ತಿ-ದ್ದಾ-ಳೆಂದು
ನಡೆ-ಸು-ತ್ತಿದ್ದು
ನಡೆ-ಸು-ತ್ತಿ-ದ್ದುದು
ನಡೆ-ಸು-ತ್ತಿ-ರು-ವಂತೆ
ನಡೆ-ಸು-ತ್ತಿ-ರು-ವುದು
ನಡೆ-ಸುವ
ನಡೆ-ಸು-ವಂತೆ
ನಡೆ-ಸು-ವು-ದರ
ನಡೆ-ಸು-ವು-ದ-ರೊಂ-ದಿಗೆ
ನಡೆ-ಸು-ವುದು
ನತ-ದೃಷ್ಟ
ನತ್ತ
ನದಿ
ನದಿ-ಸ-ರೋ-ವ-ರ-ಗಳನ್ನು
ನದಿ-ಗಳ
ನದಿ-ಗಳನ್ನು
ನದಿ-ಗಳು
ನದಿಗೆ
ನದಿಯ
ನದಿ-ಯಲ್ಲಿ
ನದಿ-ಯಾ-ಚೆಗೆ
ನದಿ-ಯಿಂದ
ನದಿಯು
ನದಿಯೇ
ನದೀ-ತಲ
ನದೀ-ಸ್ನಾನ
ನನ-ಗಂತೂ
ನನ-ಗ-ದನ್ನು
ನನ-ಗ-ದ-ರಲ್ಲಿ
ನನ-ಗದು
ನನ-ಗ-ದೆಲ್ಲ
ನನ-ಗ-ನಿ-ಸಿತು
ನನ-ಗ-ನಿ-ಸು-ತ್ತದೆ
ನನ-ಗ-ನಿ-ಸು-ತ್ತ-ದೆ-ನನ್ನ
ನನ-ಗನ್ನಿ
ನನ-ಗ-ನ್ನಿ-ಸಿತು
ನನ-ಗ-ನ್ನಿ-ಸಿ-ತು-ಸ್ವಾ-ಮೀ-ಜಿ-ಯ-ವ-ರಲ್ಲಿ
ನನ-ಗ-ನ್ನಿ-ಸಿದ್ದು
ನನ-ಗ-ನ್ನಿ-ಸು-ತ್ತದೆ
ನನ-ಗ-ನ್ನಿ-ಸು-ತ್ತ-ದೆ-ಆಗ
ನನ-ಗ-ನ್ನಿ-ಸು-ತ್ತ-ದೆ-ಪಾ-ಶ್ಚಾ-ತ್ಯ-ರ-ದ್ದೊಂದು
ನನ-ಗ-ನ್ನಿ-ಸು-ತ್ತ-ದೆ-ಮೊ-ದ-ಲ-ನೆಯ
ನನ-ಗ-ನ್ನಿ-ಸು-ತ್ತಿದೆ
ನನ-ಗ-ನ್ನಿ-ಸು-ವು-ದಿಲ್ಲ
ನನ-ಗ-ವರು
ನನ-ಗಾಗಿ
ನನ-ಗಾ-ದ-ದ್ದೇನು
ನನ-ಗಿಂತ
ನನ-ಗಿಂ-ತಲೂ
ನನ-ಗಿ-ದಕ್ಕೆ
ನನ-ಗಿದೆ
ನನ-ಗಿ-ದ್ದಂ-ತಹ
ನನ-ಗಿ-ದ್ದಿ-ದ್ದರೂ
ನನ-ಗಿಷ್ಟ
ನನ-ಗೀಗ
ನನಗೂ
ನನಗೆ
ನನ-ಗೆಂ-ಥದೂ
ನನ-ಗೆಂ-ದಿಗೂ
ನನ-ಗೆಂದೂ
ನನ-ಗೆ-ಲ್ಲಿಂದ
ನನ-ಗೆಷ್ಟು
ನನಗೇ
ನನ-ಗೇನು
ನನ-ಗೇನೂ
ನನ-ಗೊಂ-ದಿಷ್ಟು
ನನ-ಗೊಂದು
ನನ-ಗೊಬ್ಬ
ನನ-ಗೊ-ಲಿ-ದಿ-ದ್ದಾಳೆ
ನನ-ಸಾ-ಗ-ಬಹು
ನನ-ಸಾ-ಗ-ಲಿತ್ತು
ನನ-ಸಾ-ಗಿ-ಸಲು
ನನ-ಸಾ-ಗು-ವುದೊ
ನನಸು
ನನ್ನ
ನನ್ನಂ-ತ-ರಂ-ಗದ
ನನ್ನಂ-ತಹ
ನನ್ನಂ-ತೆಯೇ
ನನ್ನಂ-ಥ-ವನ
ನನ್ನಂ-ಥ-ವ-ನೊ-ಬ್ಬನ
ನನ್ನ-ತ-ನ-ವನ್ನು
ನನ್ನ-ದಲ್ಲ
ನನ್ನ-ದಾ-ಗಿತ್ತು
ನನ್ನ-ದಾ-ಗು-ತ್ತದೆ
ನನ್ನದು
ನನ್ನದೇ
ನನ್ನನ್ನು
ನನ್ನ-ನ್ನುಈ
ನನ್ನನ್ನೂ
ನನ್ನನ್ನೇ
ನನ್ನ-ನ್ನೇನೂ
ನನ್ನ-ನ್ನೊಂದು
ನನ್ನಲ್ಲಿ
ನನ್ನ-ಲ್ಲಿಗೆ
ನನ್ನ-ಲ್ಲಿ-ರುವ
ನನ್ನಲ್ಲೂ
ನನ್ನವೇ
ನನ್ನ-ಷ್ಟಕ್ಕೇ
ನನ್ನಾಸೆ
ನನ್ನಿಂದ
ನನ್ನಿಂ-ದೆ-ಲ್ಲಾ-ದೀ-ತಪ್ಪ
ನನ್ನಿಂ-ದೇ-ನಾ-ದರೂ
ನನ್ನು
ನನ್ನು-ನಾನು
ನನ್ನೆದೆ
ನನ್ನೆ-ದೆಯ
ನನ್ನೆ-ದೆ-ಯನ್ನು
ನನ್ನೆಲ್ಲ
ನನ್ನೊಂ
ನನ್ನೊಂ-ದಿ-ಗಿ-ದ್ದಾನೆ
ನನ್ನೊಂ-ದಿಗೆ
ನನ್ನೊ-ಡನೆ
ನನ್ನೊ-ಬ್ಬ-ಳಿಗೇ
ನನ್ನೊ-ಲ-ವಿನ
ನನ್ನೊ-ಳಗೆ
ನಮ
ನಮಃ
ನಮ-ಗದು
ನಮ-ಗ-ನ್ನಿ-ಸು-ತ್ತದೆ
ನಮ-ಗಾಗಿ
ನಮ-ಗಾದ
ನಮ-ಗಿಂದು
ನಮ-ಗಿತ್ತ
ನಮ-ಗಿ-ರುವ
ನಮ-ಗಿಲ್ಲ
ನಮಗೂ
ನಮಗೆ
ನಮ-ಗೆಲ್ಲ
ನಮ-ಗೆ-ಲ್ಲ-ರಿಗೂ
ನಮ-ಗೆಲ್ಲಿ
ನಮ-ಗೆ-ಲ್ಲಿದೆ
ನಮಗೇ
ನಮ-ಗೇನೂ
ನಮ-ಗೊಂದು
ನಮ-ನ-ಗಳು
ನಮ-ಸ್ಕ-ರಿಸಿ
ನಮ-ಸ್ಕ-ರಿ-ಸಿ-ಕೊ-ಳ್ಳೋಣ
ನಮ-ಸ್ಕ-ರಿ-ಸಿದ
ನಮ-ಸ್ಕ-ರಿ-ಸಿ-ದರು
ನಮ-ಸ್ಕ-ರಿ-ಸಿ-ದಳು
ನಮ-ಸ್ಕ-ರಿ-ಸುತ್ತ
ನಮ-ಸ್ಕ-ರಿ-ಸು-ತ್ತೀರೋ
ನಮ-ಸ್ಕ-ರಿ-ಸುವ
ನಮ-ಸ್ಕ-ರಿ-ಸುವು
ನಮ-ಸ್ಕಾರ
ನಮ-ಸ್ಕಾ-ರ-ವನ್ನು
ನಮ-ಸ್ಕಾರ್
ನಮಿಸಿ
ನಮಿ-ಸು-ತ್ತಿ-ದ್ದಾರೆ
ನಮಿ-ಸೋಣ
ನಮೋ
ನಮ್ಮ
ನಮ್ಮಂ-ತಹ
ನಮ್ಮಂ-ತೆಯೆ
ನಮ್ಮಂ-ಥ-ವ-ರಿ-ಗಲ್ಲ
ನಮ್ಮತ್ತ
ನಮ್ಮ-ದೆ-ಲ್ಲ-ವನ್ನೂ
ನಮ್ಮದೇ
ನಮ್ಮನ್ನು
ನಮ್ಮ-ನ್ನೆಲ್ಲ
ನಮ್ಮನ್ನೇ
ನಮ್ಮಲ್ಲಿ
ನಮ್ಮ-ಲ್ಲಿಗೆ
ನಮ್ಮ-ಲ್ಲಿ-ದ್ದುದು
ನಮ್ಮ-ಲ್ಲಿನ
ನಮ್ಮ-ಲ್ಲಿಯೂ
ನಮ್ಮ-ಲ್ಲೊ-ಬ್ಬರು
ನಮ್ಮ-ವ-ರಿಗೆ
ನಮ್ಮ-ವರು
ನಮ್ಮಿಂದ
ನಮ್ಮಿಂ-ದೇನು
ನಮ್ಮೀ
ನಮ್ಮೆಲ್ಲ
ನಮ್ಮೆ-ಲ್ಲರ
ನಮ್ಮೆ-ಲ್ಲ-ರಿಗೂ
ನಮ್ಮೊಂ-ದಿಗೆ
ನಮ್ಮೊ-ಡನೆ
ನಮ್ಮೊ-ಳ-ಗಿನ
ನಮ್ಮೊ-ಳಗೆ
ನಮ್ರ
ನಮ್ರ-ತೆ-ಯನ್ನು
ನಮ್ರ-ಭಾ-ವ-ದಿಂದ
ನಯ-ಗೊ-ಳಿ-ಸಿ-ಕೊ-ಳ್ಳು-ವಂತೆ
ನಯನ
ನಯ-ನ-ಗಳು
ನಯ-ನ-ದ್ವ-ಯ-ವನ್ನು
ನಯ-ವಲ್ಲ
ನಯ-ವಾ-ಗು-ತ್ತದೆ
ನಯ-ವಾದ
ನರ
ನರಕ
ನರ-ಕಕ್ಕೆ
ನರ-ಕ-ಗ-ಳ-ಲ್ಲಾ-ದರೆ
ನರ-ಕದ
ನರ-ಕ-ದಲ್ಲಿ
ನರ-ಕ-ವಲ್ಲ
ನರ-ಕ-ವೆಂದು
ನರ-ಕ-ಸ-ದೃ-ಶ-ವಾಗಿ
ನರ-ಕಾ-ಸು-ರ-ನೆಂದೂ
ನರ-ಗಳು
ನರ-ತಂ-ತು-ವೊಂದು
ನರ-ನಾ-ಡಿ-ಗಳಲ್ಲಿ
ನರ-ನಾ-ಡಿ-ಗ-ಳ-ಲ್ಲೆಲ್ಲ
ನರ-ನಾ-ರಿ-ಯರು
ನರ-ಮಂ-ಡಲ
ನರ-ಮಂ-ಡ-ಲಕ್ಕೆ
ನರ-ಮಂ-ಡ-ಲದ
ನರ-ಮಂ-ಡ-ಲ-ವನ್ನು
ನರ-ರಲ್ಲಿ
ನರ-ರೂಪೀ
ನರ-ಳಾ-ಟ-ಗಳಲ್ಲಿ
ನರ-ಳಾ-ಟ-ಗಳು
ನರ-ಳಿ-ದ-ರಾ-ದರೂ
ನರಳು
ನರ-ಳು-ತ್ತಲೇ
ನರ-ಳು-ತ್ತಿ-ದ್ದರು
ನರ-ಳು-ತ್ತಿ-ದ್ದರೂ
ನರ-ಳು-ತ್ತಿ-ದ್ದಳು
ನರ-ಳು-ತ್ತಿ-ದ್ದೇನೆ
ನರ-ಳು-ತ್ತಿ-ರುವ
ನರ-ಳು-ತ್ತಿ-ರು-ವಾಗ
ನರ-ಳು-ತ್ತಿ-ರು-ವುದನ್ನು
ನರ-ಳು-ಳು-ತ್ತಿ-ದ್ದೇನೆ
ನರ-ಳು-ವ-ವ-ರಿ-ಗೆಲ್ಲ
ನರ-ಳು-ವಾಗ
ನರ-ಸಿಂಹ
ನರ-ಸಿಂ-ಹಾ-ಚಾರ್ಯ
ನರಸು
ನರ-ಹರಿ
ನರಿ-ಯಾ-ಗು-ತ್ತಾನೆ
ನರೆ-ಗೂ-ದಲೆ
ನರೇಂದ್ರ
ನರೇಂ-ದ್ರನ
ನರೇಂ-ದ್ರ-ನನ್ನು
ನರೇಂ-ದ್ರ-ನಲ್ಲಿ
ನರೇಂ-ದ್ರ-ನಾ-ಗಿ-ದ್ದಾಗ
ನರೇಂ-ದ್ರ-ನಾ-ಗಿ-ದ್ದಾ-ಗ-ಲೇ-ಶ್ರೀ-ರಾ-ಮ-ಕೃ-ಷ್ಣರು
ನರೇಂ-ದ್ರ-ನಾ-ಗಿಯೇ
ನರೇಂ-ದ್ರ-ನಾಥ
ನರೇಂ-ದ್ರ-ನಿಂದು
ನರೇಂ-ದ್ರ-ನಿ-ಗಿಂತ
ನರೇಂ-ದ್ರ-ನಿಗೆ
ನರೇಂ-ದ್ರನೂ
ನರೇಂ-ದ್ರನೇ
ನರೇಂ-ದ್ರ-ನೊಂ-ದಿಗೆ
ನರೇ-ನ-ನಾಗಿ
ನರೇ-ನನೇ
ನರೇನ್
ನರೇ-ನ್ದತ್ತ
ನರ್ತ-ನ-ಗೈ-ಯುವ
ನರ್ತ-ನ-ವ-ನ್ನೇಕೆ
ನರ್ತಿ-ಸು-ತ್ತಿದೆ
ನಲ-ವ-ತ್ತನೇ
ನಲ-ವ-ತ್ತನ್ನು
ನಲ-ವತ್ತು
ನಲ-ವ-ತ್ತು-ಎಂಬ
ನಲ-ವ-ತ್ತೆ-ರಡು
ನಲಿ-ದಾ-ಡು-ತ್ತಿ-ದ್ದುವು
ನಲಿ-ನ-ಲಿ-ಯುತ
ನಲಿ-ಯು-ತ್ತಿ-ದ್ದ-ರಾ-ದರೂ
ನಲಿ-ವನ್ನೇ
ನಲಿವು
ನಲು-ಗಿ-ದಂತೆ
ನಲ್ಮೆಯ
ನಲ್ಲಿ
ನಲ್ಲಿಗೆ
ನಲ್ಲಿನ
ನಲ್ಲೂ
ನಲ್ಲೆಲ್ಲ
ನಲ್ಲೇ
ನಲ್ವ-ತ್ತ-ನೆಯ
ನವ
ನವ-ಕು-ಸು-ಮ-ಗಳನ್ನು
ನವ-ಗೋ-ಪಾಲ
ನವ-ಗೋ-ಪಾ-ಲನ
ನವ-ಚಿಂ-ತ-ನೆಯ
ನವ-ಚೇ-ತನ
ನವ-ಚೇ-ತ-ನ-ದಿಂದ
ನವ-ಚೇ-ತ-ನ-ವೊಂದು
ನವ-ಜಾತ
ನವ-ಜೀ-ವ-ನ-ಚೇ-ತ-ನ-ವನ್ನು
ನವ-ತಾ-ರು-ಣ್ಯ-ದಿಂದ
ನವ-ನೂ-ತನ
ನವ-ಭಾ-ರ-ತದ
ನವಮಿ
ನವ-ಮಿ-ಯಂದು
ನವ-ಯು-ಗದ
ನವ-ರಾತ್ರಿ
ನವ-ರಾ-ತ್ರಿಗೆ
ನವ-ರಾ-ತ್ರಿ-ಯನ್ನು
ನವ-ಶಕ್ತಿ
ನವ-ಸಂ-ದೇ-ಶದ
ನವ-ಸ್ಫೂ-ರ್ತಿ-ಯನ್ನು
ನವ-ಸ್ವಾ-ಗ-ತವು
ನವಾ-ಬ-ನನ್ನು
ನವಾ-ಬರ
ನವಿ-ರೇ-ಳುವ
ನವಿ-ಲು-ಗಳನ್ನು
ನವೀನ
ನವೆಂ-ಬ-ರಿನ
ನವೆಂ-ಬರ್
ನವೋ-ತ್ಸಾ-ಹ-ದಿಂದ
ನವೋ-ತ್ಸಾ-ಹ-ವನ್ನು
ನವೋ-ತ್ಸಾ-ಹ-ವ-ನ್ನುಂ-ಟು-ಮಾ-ಡು-ತ್ತಿತ್ತು
ನಶಿ-ಸಿ-ಹೋ-ಗಿದೆ
ನಶಿ-ಸಿ-ಹೋ-ಗಿ-ದೆಯೋ
ನಶ್ವರ
ನಷ್ಟ
ನಷ್ಟ-ವಾ-ಗದೆ
ನಷ್ಟ-ವಾ-ಗಿ-ಲ್ಲ-ವಷ್ಟೇ
ನಷ್ಟವೇ
ನಷ್ಟ-ವೇ-ನಿ-ದ್ದರೂ
ನಷ್ಟ-ವೇನೂ
ನಷ್ಟೆ
ನಸು-ಕಿ-ನಲ್ಲಿ
ನಸು-ಗ-ತ್ತ-ಲೆ-ಯಲ್ಲಿ
ನಸು-ನಕ್ಕು
ನಸು-ನ-ಗುತ್ತ
ನಸು-ಬೆ-ಳ-ಕಿ-ನಲ್ಲಿ
ನಹ-ಬತ್
ನಹಿ
ನಾ
ನಾಂದಿ-ಯಾ-ಗ-ಲಿತ್ತು
ನಾಗ
ನಾಗ-ಮಹಾ
ನಾಗ-ಮ-ಹಾ-ಶ-ಯರ
ನಾಗ-ಮ-ಹಾ-ಶ-ಯ-ರತ್ತ
ನಾಗ-ಮ-ಹಾ-ಶ-ಯ-ರನ್ನು
ನಾಗ-ಮ-ಹಾ-ಶ-ಯ-ರಿದ್ದ
ನಾಗ-ಮ-ಹಾ-ಶ-ಯರು
ನಾಗ-ಮ-ಹಾ-ಶ-ಯ-ರೊಂ-ದಿಗೆ
ನಾಗರಿ
ನಾಗ-ರಿಕ
ನಾಗ-ರಿ-ಕತೆ
ನಾಗ-ರಿ-ಕ-ತೆ-ಚ-ರಿ-ತ್ರೆ-ಗಳ
ನಾಗ-ರಿ-ಕ-ತೆ-ಗಳನ್ನೂ
ನಾಗ-ರಿ-ಕ-ತೆ-ಗ-ಳ-ಲ್ಲೊಂ-ದಾದ
ನಾಗ-ರಿ-ಕ-ತೆ-ಗ-ಳಿ-ಗಿಂ-ತಲೂ
ನಾಗ-ರಿ-ಕ-ತೆ-ಗ-ಳೆಲ್ಲ
ನಾಗ-ರಿ-ಕ-ತೆಯ
ನಾಗ-ರಿ-ಕ-ತೆ-ಯಲ್ಲಿ
ನಾಗ-ರಿ-ಕ-ತೆ-ಯಿಂದ
ನಾಗ-ರಿ-ಕ-ತೆಯು
ನಾಗ-ರಿ-ಕರ
ನಾಗ-ರಿ-ಕ-ರ-ನ್ನು-ದ್ದೇ-ಶಿಸಿ
ನಾಗ-ರಿ-ಕ-ರಾದ
ನಾಗ-ರಿ-ಕ-ರಿಂದ
ನಾಗ-ರಿ-ಕ-ರಿಗೆ
ನಾಗ-ರಿ-ಕರು
ನಾಗ-ರಿ-ಕರೂ
ನಾಗರೀ
ನಾಗ-ರೀ-ಕ-ತೆಯ
ನಾಗಾ
ನಾಗಿದ್ದ
ನಾಗಿ-ದ್ದಾ-ಗಿ-ನಿಂ-ದಲೂ
ನಾಗಿ-ದ್ದುದು
ನಾಗಿ-ದ್ದೇನೆ
ನಾಗಿ-ಬಿ-ಟ್ಟಿ-ದ್ದೇನೆ
ನಾಗಿ-ರ-ಬೇಕು
ನಾಗು-ತ್ತಾನೆ
ನಾಚಿ-ಕೆ-ಗಳಿಂದ
ನಾಚಿ-ಕೆ-ಯಾ-ಗು-ತ್ತದೆ
ನಾಚಿ-ಕೆ-ಯಾ-ಗು-ತ್ತಿದೆ
ನಾಚಿ-ಕೆ-ಯಾ-ಗು-ವು-ದಿ-ಲ್ಲವೆ
ನಾಚಿ-ಕೆ-ಯಾ-ಯಿತು
ನಾಜೂ-ಕಾಗಿ
ನಾಟಕ
ನಾಟ-ಕ-ಕ-ರ್ತನೂ
ನಾಟ-ಕ-ಕಾ-ರರು
ನಾಟ-ಕ-ಗಳಲ್ಲಿ
ನಾಟ-ಕ-ವನ್ನು
ನಾಟ-ಕ-ವಾ-ಡಿ-ದ್ದರು
ನಾಟ-ಕೀ-ಯ-ವಾಗಿ
ನಾಟ್ಯ
ನಾಡ-ದೆ-ಲ್ಲವು
ನಾಡ-ಲಿಲ್ಲ
ನಾಡಾಗಿ
ನಾಡಾದ
ನಾಡಿ
ನಾಡಿ-ಗಳ
ನಾಡಿಗೆ
ನಾಡಿ-ದರು
ನಾಡಿ-ದ-ರೆಂ-ದರೆ
ನಾಡಿನ
ನಾಡಿ-ನಲ್ಲಿ
ನಾಡಿ-ನ-ಲ್ಲಿಯೇ
ನಾಡಿ-ನಲ್ಲೇ
ನಾಡಿ-ನಿಂ-ದಲೇ
ನಾಡಿಯ
ನಾಡು
ನಾಡು-ಗಳಲ್ಲಿ
ನಾಡುತ್ತ
ನಾಡು-ತ್ತಿ-ರ-ಲಿಲ್ಲ
ನಾಡುವ
ನಾಡು-ವಾ-ಗ-ಲೆಲ್ಲ
ನಾಣ್ಯ-ಗ-ಳ-ನ್ನಿ-ಟ್ಟಿ-ರುವ
ನಾಣ್ಯದ
ನಾಣ್ಯ-ವನ್ನು
ನಾಥಕ್ಕೆ
ನಾಥವು
ನಾದ
ನಾದ-ನೆಂ-ದರೆ
ನಾದರೂ
ನಾದ-ವನ
ನಾದ-ವನು
ನಾದವು
ನಾದಾ-ಮೃ-ತ-ದಿಂದ
ನಾನಂತೂ
ನಾನಂದು
ನಾನ-ಕ-ರನ್ನು
ನಾನ-ಕರು
ನಾನ-ದನ್ನು
ನಾನ-ದರ
ನಾನ-ದ-ರಲ್ಲಿ
ನಾನ-ರಿಯೆ
ನಾನಲ್ಲ
ನಾನ-ವನ
ನಾನ-ವ-ನಿಗೆ
ನಾನ-ವ-ರನ್ನು
ನಾನ-ವ-ರಿಗೆ
ನಾನ-ವ-ಳಿಗೆ
ನಾನ-ವು-ಗಳನ್ನು
ನಾನಾ
ನಾನಾಗ
ನಾನಾ-ದರೂ
ನಾನಾ-ರೀ-ತಿ-ಯಲ್ಲಿ
ನಾನಿಂದು
ನಾನಿ-ಚ್ಛಿ-ಸಿ-ದರೆ
ನಾನಿ-ದನ್ನು
ನಾನಿ-ದ್ದೇನೆ
ನಾನಿನ್ನು
ನಾನಿನ್ನೂ
ನಾನಿಲ್ಲಿ
ನಾನಿ-ಲ್ಲಿಗೆ
ನಾನೀ
ನಾನೀಗ
ನಾನು
ನಾನು-ನನ್ನ
ನಾನು-ಗಳ
ನಾನೂ
ನಾನೂರು
ನಾನೆಂ-ದಿಗೂ
ನಾನೆಂದೂ
ನಾನೆಂದೆ
ನಾನೆ-ಣಿ-ಸಿದ್ದೆ
ನಾನೆ-ಲ್ಲಿ-ರು-ತ್ತಿದ್ದೆ
ನಾನೆಷ್ಟು
ನಾನೇ
ನಾನೇಕೆ
ನಾನೇ-ನಾ-ಗಿ-ರು-ವೆನೋ
ನಾನೇ-ನಾ-ದರೂ
ನಾನೇನು
ನಾನೇನೂ
ನಾನೇ-ನೆಂದು
ನಾನೇನೋ
ನಾನೊಂದು
ನಾನೊ-ಪ್ಪು-ವುದು
ನಾನೊಬ್ಬ
ನಾನೊ-ಬ್ಬನೇ
ನಾನ್ನೂರು
ನಾಪ-ಭಾ-ಷಿತಂ
ನಾಮ
ನಾಮ-ಧೇ-ಯ-ದಿಂದ
ನಾಮ-ಧೇ-ಯ-ವನ್ನು
ನಾಮ-ಧೇ-ಯ-ಸೋ-ದರಿ
ನಾಮವು
ನಾಮ-ಸ್ಮ-ರ-ಣೆ-ಯಿಂ-ದಲೇ
ನಾಮೋ-ಚ್ಚಾ-ರಣೆ
ನಾಮೋ-ಪ-ದೇ-ಶ-ವನ್ನು
ನಾಯಕ
ನಾಯ-ಕ-ತ್ವದ
ನಾಯ-ಕ-ತ್ವ-ದಲ್ಲಿ
ನಾಯ-ಕ-ತ್ವ-ವನ್ನು
ನಾಯ-ಕನ
ನಾಯ-ಕ-ನಂತೆ
ನಾಯ-ಕ-ನನ್ನೂ
ನಾಯ-ಕ-ನಾ-ಗು-ವು-ದ-ರ-ಲ್ಲಿ-ರುವ
ನಾಯ-ಕ-ನಿಂದ
ನಾಯ-ಕನೂ
ನಾಯ-ಕ-ನೆಂದು
ನಾಯ-ಕ-ನೊಬ್ಬ
ನಾಯ-ಕ-ರಾ-ಗ-ಬೇ-ಕೆಂದು
ನಾಯ-ಕ-ರಾದ
ನಾಯ-ಕರು
ನಾಯಿ
ನಾಯಿ-ಗ-ಳಂತೆ
ನಾಯಿ-ಗ-ಳಿ-ದ್ದು-ವು-ಒಂದು
ನಾಯಿ-ಗಳೂ
ನಾಯಿಗೆ
ನಾಯಿ-ಮರಿ
ನಾಯಿಯೂ
ನಾಯಿ-ಯೊಂದು
ನಾಯ್ಡು
ನಾರ-ದ-ರು-ಇ-ವರೇ
ನಾರಾ-ಯಣ
ನಾರಾ-ಯ-ಣ-ಗಂಜ್ಗೆ
ನಾರಾ-ಯ-ಣನ
ನಾರಾ-ಯ-ಣ-ನನ್ನು
ನಾರಾ-ಯ-ಣರು
ನಾರಾ-ಯ-ಣ-ಸ್ವ-ರೂ-ಪಿ-ಯೆಂದು
ನಾರಾ-ಯ-ಣಾಯ
ನಾಲಿಗೆ
ನಾಲಿ-ಗೆಗೆ
ನಾಲಿ-ಗೆಯ
ನಾಲಿ-ಗೆ-ಯಲ್ಲಿ
ನಾಲ್ಕಂ-ತ-ಸ್ತು-ಗಳ
ನಾಲ್ಕಂ-ಶದ
ನಾಲ್ಕ-ನೆಯ
ನಾಲ್ಕನೇ
ನಾಲ್ಕ-ರಂದು
ನಾಲ್ಕ-ರಂದೇ
ನಾಲ್ಕಾರು
ನಾಲ್ಕು
ನಾಲ್ಕೂ
ನಾಲ್ಕೇ
ನಾಲ್ಕೈದು
ನಾಲ್ಕೋ
ನಾಲ್ವ-ರಿಗೆ
ನಾಲ್ವರು
ನಾಳಿನ
ನಾಳೆ
ನಾಳೆಯೇ
ನಾವ-ದನ್ನು
ನಾವ-ವ-ರಿಗೆ
ನಾವಿಂದು
ನಾವಿನ್ನು
ನಾವಿ-ನ್ನು-ಮೇಲೆ
ನಾವಿನ್ನೂ
ನಾವಿ-ಬ್ಬರೂ
ನಾವಿಲ್ಲಿ
ನಾವೀಗ
ನಾವೀ-ಗಾ-ಗಲೇ
ನಾವು
ನಾವೂ
ನಾವೆಂದೂ
ನಾವೆಂ-ಬಂತೆ
ನಾವೆಲ್ಲ
ನಾವೆ-ಲ್ಲರೂ
ನಾವೆಷ್ಟು
ನಾವೇ
ನಾವೇ-ನಾ-ದರೂ
ನಾವೇನು
ನಾವೇನೂ
ನಾವೊಂದು
ನಾವ್ಯಾರೂ
ನಾಶ
ನಾಶಕ್ಕೆ
ನಾಶ-ಗೈ-ಯುವ
ನಾಶ-ಗೊ-ಳಿ-ಸಿ-ಬಿ-ಡು-ತ್ತದೆ
ನಾಶ-ಗೊ-ಳಿ-ಸು-ತ್ತದೆ
ನಾಶ-ಗೊ-ಳಿ-ಸು-ತ್ತಿದೆ
ನಾಶ-ಮಾ-ಡದೆ
ನಾಶ-ಮಾ-ಡಿದ
ನಾಶ-ವಾ-ಗ-ದಿ-ರು-ವುದು
ನಾಶ-ವಾ-ಗದೆ
ನಾಶ-ವಾಗು
ನಾಶ-ವಾ-ಗು-ತ್ತದೆ
ನಾಶ-ವಾ-ಗು-ತ್ತ-ದೆಂದು
ನಾಶ-ವಾ-ಗು-ವು-ದಿಲ್ಲ
ನಾಶ-ವಾ-ದಾಗ
ನಾಶ-ವಿಲ್ಲ
ನಾಸ್ತಿ-ಕರು
ನಾಸ್ತಿ-ಕ-ವಾ-ದ-ಕ್ಕಿಂ-ತಲೂ
ನಿಂತ
ನಿಂತಂತೆ
ನಿಂತ-ದ್ದನ್ನು
ನಿಂತದ್ದು
ನಿಂತ-ಮೇಲೆ
ನಿಂತರು
ನಿಂತರೂ
ನಿಂತರೆ
ನಿಂತ-ಲ್ಲದೆ
ನಿಂತಲ್ಲಿ
ನಿಂತ-ಲ್ಲಿಂದ
ನಿಂತ-ಲ್ಲೆಲ್ಲ
ನಿಂತವು
ನಿಂತಾಗ
ನಿಂತಾ-ಗ-ಲೆಲ್ಲ
ನಿಂತಿ
ನಿಂತಿತು
ನಿಂತಿತ್ತು
ನಿಂತಿದೆ
ನಿಂತಿದ್ದ
ನಿಂತಿ-ದ್ದ-ರಿಂದ
ನಿಂತಿ-ದ್ದರು
ನಿಂತಿ-ದ್ದರೆ
ನಿಂತಿ-ದ್ದಾಗ
ನಿಂತಿ-ದ್ದಾರೆ
ನಿಂತಿ-ದ್ದು-ದ-ರಿಂದ
ನಿಂತಿ-ದ್ದೇನೆ
ನಿಂತಿ-ರ-ಬ-ಲ್ಲೆಯಾ
ನಿಂತಿ-ರ-ಬೇಕು
ನಿಂತಿ-ರು-ತ್ತಾನೆ
ನಿಂತಿ-ರುವ
ನಿಂತಿ-ರು-ವಂತೆ
ನಿಂತಿ-ರು-ವಷ್ಟೇ
ನಿಂತಿ-ರು-ವು-ದಾಗಿ
ನಿಂತಿ-ರು-ವುದು
ನಿಂತಿಲ್ಲ
ನಿಂತಿವೆ
ನಿಂತಿ-ಹನು
ನಿಂತು
ನಿಂತು-ಕೊಂ-ಡರು
ನಿಂತು-ಕೊಂಡು
ನಿಂತು-ಕೊ-ಳ್ಳುವ
ನಿಂತು-ಬಿ-ಟ್ಟರು
ನಿಂತು-ಬಿ-ಟ್ಟಿದೆ
ನಿಂತು-ಹೋ-ಗಿದೆ
ನಿಂತು-ಹೋ-ಗಿದ್ದ
ನಿಂತು-ಹೋ-ಯಿತು
ನಿಂತೇ
ನಿಂತೇ-ಬಿ-ಟ್ಟರು
ನಿಂತೇ-ಹೋ-ದಂತೆ
ನಿಂದ
ನಿಂದಂತು
ನಿಂದ-ನೆಯ
ನಿಂದ-ನೆ-ಯಲ್ಲಿ
ನಿಂದಲೂ
ನಿಂದಿ-ಸ-ತೊ-ಡ-ಗಿ-ದರು
ನಿಂದಿ-ಸ-ತೊ-ಡ-ಗಿ-ದಾಗ
ನಿಂದಿ-ಸ-ದಿ-ರೋಣ
ನಿಂದಿ-ಸ-ಬ-ಹುದು
ನಿಂದಿಸಿ
ನಿಂದಿ-ಸಿ-ಕೊ-ಳ್ಳ-ಬೇಕು
ನಿಂದಿ-ಸಿ-ದ-ವ-ರಲ್ಲ
ನಿಂದಿ-ಸಿ-ಬಿ-ಟ್ಟೆನೋ
ನಿಂದಿ-ಸುವ
ನಿಂದಿ-ಸು-ವಿರಾ
ನಿಂದೆ
ನಿಂದೆ-ಟೀ-ಕೆ-ಗಳ
ನಿಂದೆ-ಟೀ-ಕೆ-ಗ-ಳಿಗೆ
ನಿಂದೆ-ಗ-ಳಿ-ಗಿಂ-ತಲೂ
ನಿಂದೆಯ
ನಿಃಸ್ವಾರ್ಥ
ನಿಃಸ್ವಾ-ರ್ಥ-ತೆಯ
ನಿಃಸ್ವಾ-ರ್ಥ-ದಿಂದ
ನಿಃಸ್ವಾ-ರ್ಥ-ವಾ-ಗಿಯೇ
ನಿಃಸ್ವಾರ್ಥಿ
ನಿಃಸ್ವಾ-ರ್ಥಿ-ಯಾ-ದ-ವರು
ನಿಕಟ
ನಿಕ-ಟ-ವ-ರ್ತಿ-ಗ-ಳಾ-ಗಿದ್ದ
ನಿಕ-ಟ-ವ-ರ್ತಿ-ಗ-ಳಿ-ಗೆಲ್ಲ
ನಿಕ-ಟ-ವ-ರ್ತಿ-ಗ-ಳೆಲ್ಲ
ನಿಕ-ಟ-ವಾ-ಗಿ-ತ್ತೆಂ-ದರೆ
ನಿಕ-ಟ-ವಾ-ದಷ್ಟೂ
ನಿಕೃಷ್ಟ
ನಿಕೇ-ತ-ನ-ಗಳನ್ನು
ನಿಖ-ರ-ತೆ-ಯಿಂದ
ನಿಖ-ರ-ವಾ-ಗಿ-ರಲೂ
ನಿಖ-ರ-ವಾದ
ನಿಗ-ದಿ-ಗೊ-ಳಿಸ
ನಿಗ-ದಿತ
ನಿಗ-ದಿ-ಪ-ಡಿ-ಸ-ಲಾ-ಗಿ-ತ್ತು-ಬೆಲೆ
ನಿಗ-ದಿ-ಪ-ಡಿ-ಸಿದ
ನಿಗ-ದಿ-ಪ-ಡಿ-ಸಿ-ದರು
ನಿಗ-ದಿ-ಯಾ-ಗಿತ್ತು
ನಿಗಾ
ನಿಗಾಗಿ
ನಿಗಾದ
ನಿಗೂ-ಢ-ತೆಯು
ನಿಗೂ-ಢ-ವಾಗಿ
ನಿಗೆ
ನಿಗೇ
ನಿಚ್ಚ-ಳ-ವಾಗಿ
ನಿಜ
ನಿಜಕ್ಕೂ
ನಿಜ-ಜೀ-ವ-ನ-ವನ್ನು
ನಿಜ-ವಲ್ಲ
ನಿಜ-ವ-ಲ್ಲದ
ನಿಜ-ವಾ-ಗ-ಬ-ಹು-ದೆಂದು
ನಿಜ-ವಾಗಿ
ನಿಜ-ವಾ-ಗಿದೆ
ನಿಜ-ವಾ-ಗಿಯೂ
ನಿಜ-ವಾ-ಗಿ-ರು-ವುದನ್ನು
ನಿಜ-ವಾ-ಗು-ತ್ತವೆ
ನಿಜ-ವಾದ
ನಿಜ-ವಾ-ದಂ-ತಾ-ಯಿತು
ನಿಜ-ವಾ-ದರೂ
ನಿಜವೆ
ನಿಜ-ವೆಂದು
ನಿಜ-ವೆ-ನ್ನು-ವು-ದಾ-ದರೆ
ನಿಜವೇ
ನಿಜವೋ
ನಿಜ-ಸಂ-ಗ-ತಿ-ಯನ್ನು
ನಿಜ-ಸ್ವ-ರೂಪ
ನಿಜ-ಸ್ವ-ರೂ-ಪವು
ನಿಜಾರ್ಥ
ನಿಜಾ-ರ್ಥ-ದಲ್ಲಿ
ನಿಟ್ಟಾದ
ನಿಟ್ಟಿ-ನಲ್ಲಿ
ನಿಟ್ಟಿ-ಸುತ್ತ
ನಿಡು-ಸು-ಯ್ಯುವ
ನಿತ್ಯ
ನಿತ್ಯ-ಶಾ-ಶ್ವ-ತ-ಸ-ನಾ-ತನ
ನಿತ್ಯ-ಶು-ದ್ಧ-ಬು-ದ್ಧ-ಮು-ಕ್ತ-ನಾದ
ನಿತ್ಯ-ಕ-ಲಹ
ನಿತ್ಯ-ಜೀ-ವ-ನ-ದಲ್ಲಿ
ನಿತ್ಯತ್ವ
ನಿತ್ಯ-ನೂ-ತನ
ನಿತ್ಯ-ಮುಕ್ತ
ನಿತ್ಯ-ವಾದ
ನಿತ್ಯವೂ
ನಿತ್ಯ-ಶ್ರ-ದ್ಧೆ-ಯಿಂದ
ನಿತ್ಯಾ-ನಂದ
ನಿತ್ಯಾ-ನಂ-ದ-ರನ್ನು
ನಿತ್ಯಾ-ನಂ-ದರು
ನಿದ-ರ್ಶನ
ನಿದ-ರ್ಶ-ನ-ಗಳನ್ನು
ನಿದ-ರ್ಶ-ನ-ದಂ-ತಿದ್ದ
ನಿದ್ದಾ-ನೆಂಬ
ನಿದ್ರಾ
ನಿದ್ರಾ-ಶ-ಕ್ತಿ-ಯಿ-ರ-ಲಿಲ್ಲ
ನಿದ್ರಿ-ಸ-ಬೇ-ಕೆಂದು
ನಿದ್ರಿ-ಸಿದ
ನಿದ್ರಿ-ಸಿ-ದರು
ನಿದ್ರಿ-ಸಿ-ದರೆ
ನಿದ್ರಿ-ಸು-ತ್ತಿ-ದ್ದಂ-ತೆಯೇ
ನಿದ್ರಿ-ಸು-ತ್ತಿ-ದ್ದರೋ
ನಿದ್ರಿ-ಸು-ತ್ತಿ-ದ್ದಾಳೆ
ನಿದ್ರಿ-ಸು-ತ್ತಿ-ದ್ದಾ-ಳೆಂ-ದರೆ
ನಿದ್ರಿ-ಸು-ತ್ತಿ-ದ್ದೇನೆ
ನಿದ್ರಿ-ಸುವ
ನಿದ್ರಿ-ಸು-ವು-ದಿಲ್ಲ
ನಿದ್ರೆ
ನಿದ್ರೆ-ವಿ-ಶ್ರಾಂ-ತಿ-ಯೆಂ-ಬುದು
ನಿದ್ರೆ-ಹ-ಸಿ-ವು-ಗಳು
ನಿದ್ರೆ-ಗಿದು
ನಿದ್ರೆ-ಗೆ-ಡ-ಬೇ-ಕಾ-ಯಿತು
ನಿದ್ರೆ-ಗೆ-ಡು-ವಂತೆ
ನಿದ್ರೆ-ಮಾ-ಡಿದ್ದೇ
ನಿದ್ರೆ-ಮಾ-ಡು-ವು-ದ-ಕ್ಕಾಗಿ
ನಿದ್ರೆಯ
ನಿದ್ರೆ-ಯನ್ನು
ನಿದ್ರೆ-ಯಲ್ಲಿ
ನಿದ್ರೆ-ಯಿಂದ
ನಿದ್ರೆ-ಯಿ-ಲ್ಲ-ದಂ-ತಾಗಿ
ನಿದ್ರೆಯೂ
ನಿದ್ರೆ-ಯೆಂ-ಬುದು
ನಿಧನ
ನಿಧ-ನದ
ನಿಧ-ನ-ದಿಂ-ದಾಗಿ
ನಿಧ-ನ-ದೊಂ-ದಿಗೆ
ನಿಧ-ನ-ರಾದ
ನಿಧಾನ
ನಿಧಾ-ನ-ವಾಗಿ
ನಿಧಾ-ನ-ವಾ-ದು-ದಕ್ಕೆ
ನಿಧಿ
ನಿಧಿ-ಗಾಗಿ
ನಿಧಿಗೆ
ನಿಧಿ-ಯನ್ನು
ನಿಧಿ-ಯೊಂದು
ನಿನ-ಗ-ದನ್ನು
ನಿನ-ಗದು
ನಿನ-ಗ-ನಿ-ಸು-ತ್ತ-ದೆಯೆ
ನಿನ-ಗ-ನ್ನಿ-ಸು-ತ್ತದೆ
ನಿನ-ಗ-ರ್ಪಿತ
ನಿನ-ಗ-ವೆಲ್ಲ
ನಿನ-ಗಾ-ಗಲಿ
ನಿನ-ಗಾಗಿ
ನಿನ-ಗಾದ
ನಿನ-ಗಿದು
ನಿನ-ಗಿ-ರುವ
ನಿನ-ಗಿ-ರು-ವಂ-ತಹ
ನಿನ-ಗಿಲ್ಲ
ನಿನ-ಗಿಷ್ಟ
ನಿನ-ಗೀಗ
ನಿನ-ಗುಂ-ಟಾದ
ನಿನಗೂ
ನಿನಗೆ
ನಿನ-ಗೆಂಥ
ನಿನಗೇ
ನಿನ-ಗೇನು
ನಿನ-ಗೇ-ನೆ-ನ್ನಿ-ಸು-ತ್ತದೆ
ನಿನ-ಗೊಂದು
ನಿನ-ಗೊ-ದ-ಗಿದ
ನಿನಾದ
ನಿನಾ-ದದ
ನಿನಾ-ದ-ವೊಂದು
ನಿನ್ನ
ನಿನ್ನಂ-ತಹ
ನಿನ್ನ-ಇ-ಬ್ಬರ
ನಿನ್ನ-ಡಿ-ಗಳ
ನಿನ್ನ-ತ-ನ-ವನ್ನೇ
ನಿನ್ನ-ದಾ-ಗಲಿ
ನಿನ್ನದೇ
ನಿನ್ನ-ನ-ರ-ಸುತ
ನಿನ್ನನ್ನು
ನಿನ್ನನ್ನೂ
ನಿನ್ನನ್ನೆ
ನಿನ್ನನ್ನೇ
ನಿನ್ನಲ್ಲಿ
ನಿನ್ನ-ಲ್ಲಿ-ರುವ
ನಿನ್ನವ
ನಿನ್ನ-ವನೇ
ನಿನ್ನ-ವಾ-ಗಲಿ
ನಿನ್ನಿಂದ
ನಿನ್ನು-ಸಿರು
ನಿನ್ನೆ
ನಿನ್ನೆಯ
ನಿನ್ನೆಯೇ
ನಿನ್ನೊಂ-ದಿಗೆ
ನಿನ್ನೊ-ಲವ
ನಿನ್ನೊ-ಲ-ವಿ-ನೆದೆ
ನಿನ್ನೊ-ಳ-ಗಿ-ನಿಂ-ದಲೇ
ನಿನ್ನೊ-ಳಗೆ
ನಿಬೋ-ಧತ
ನಿಬ್ಬೆ-ರ-ಗಾ-ದರು
ನಿಮ
ನಿಮ-ಗ-ದನ್ನು
ನಿಮ-ಗಾಗಿ
ನಿಮ-ಗಿನ್ನೂ
ನಿಮ-ಗಿಲ್ಲ
ನಿಮ-ಗಿಷ್ಟ
ನಿಮಗೂ
ನಿಮಗೆ
ನಿಮ-ಗೆ-ದು-ರಾಗಿ
ನಿಮ-ಗೆಲ್ಲ
ನಿಮಗೇ
ನಿಮ-ಗೇ-ನಾ-ದರೂ
ನಿಮ-ಗೊಂದು
ನಿಮ-ಗ್ನ-ರಾ-ಗಿ-ರು-ವುದನ್ನು
ನಿಮಿತ್ತ
ನಿಮಿಷ
ನಿಮಿ-ಷ-ಗಳ
ನಿಮಿ-ಷ-ಗಳಲ್ಲಿ
ನಿಮಿ-ಷ-ಗ-ಳಲ್ಲೇ
ನಿಮಿ-ಷದ
ನಿಮೀ-ಲಿತ
ನಿಮ್ಮ
ನಿಮ್ಮಂ-ತಹ
ನಿಮ್ಮಂಥ
ನಿಮ್ಮಂ-ಥ-ವರು
ನಿಮ್ಮ-ಕೈಲಿ
ನಿಮ್ಮ-ತ-ನ-ವನ್ನು
ನಿಮ್ಮ-ದಿನ್ನೂ
ನಿಮ್ಮದು
ನಿಮ್ಮ-ದೆಲ್ಲ
ನಿಮ್ಮದೇ
ನಿಮ್ಮ-ನಿ-ಮ್ಮೊ-ಳಗೇ
ನಿಮ್ಮ-ನ್ನಾ-ಗಲಿ
ನಿಮ್ಮ-ನ್ನೀಗ
ನಿಮ್ಮನ್ನು
ನಿಮ್ಮ-ನ್ನು-ದ್ದೇ-ಶಿಸಿ
ನಿಮ್ಮನ್ನೂ
ನಿಮ್ಮ-ನ್ನೆಲ್ಲ
ನಿಮ್ಮ-ನ್ನೆ-ಲ್ಲಿಗೂ
ನಿಮ್ಮನ್ನೇ
ನಿಮ್ಮ-ನ್ನೇನೂ
ನಿಮ್ಮ-ಲಿ-ದೆ-ಯೇನು
ನಿಮ್ಮಲ್ಲಿ
ನಿಮ್ಮ-ಲ್ಲಿಗೆ
ನಿಮ್ಮ-ಲ್ಲಿದೆ
ನಿಮ್ಮ-ಲ್ಲಿ-ದೆ-ಯೇನು
ನಿಮ್ಮ-ಲ್ಲಿ-ದ್ದರೆ
ನಿಮ್ಮ-ಲ್ಲಿನ
ನಿಮ್ಮ-ಲ್ಲಿಯೇ
ನಿಮ್ಮ-ಲ್ಲಿ-ರಲಿ
ನಿಮ್ಮ-ಲ್ಲಿ-ರು-ವಂ-ತೆಯೇ
ನಿಮ್ಮಲ್ಲೇ
ನಿಮ್ಮ-ಲ್ಲೊಂದು
ನಿಮ್ಮ-ವ-ರಲ್ಲ
ನಿಮ್ಮ-ಷ್ಟಕ್ಕೆ
ನಿಮ್ಮಿಂದ
ನಿಮ್ಮೆ-ದು-ರಿ-ನಲ್ಲಿ
ನಿಮ್ಮೆಲ್ಲ
ನಿಮ್ಮೆ-ಲ್ಲರ
ನಿಮ್ಮೆ-ಲ್ಲ-ರಲ್ಲಿ
ನಿಮ್ಮೆ-ಲ್ಲ-ರಿಗೂ
ನಿಮ್ಮೊಂ-ದಿಗೆ
ನಿಮ್ಮೊ-ಡ-ನಿದ್ದು
ನಿಮ್ಮೊ-ಡ-ನಿ-ರ-ಲೆಂದೇ
ನಿಮ್ಮೊ-ಳ-ಗಣ
ನಿಮ್ಮೊ-ಳ-ಗಿನ
ನಿಮ್ಮೊ-ಳಗೂ
ನಿಮ್ಮೊ-ಳಗೆ
ನಿಮ್ಮೊ-ಳಗೇ
ನಿಯಂ-ತ್ರ-ಣಕ್ಕೆ
ನಿಯಂ-ತ್ರಿ-ಸ-ದಿ-ದ್ದರೆ
ನಿಯಂ-ತ್ರಿ-ಸಲು
ನಿಯಂ-ತ್ರಿ-ಸು-ತ್ತಾರೆ
ನಿಯಂ-ತ್ರಿ-ಸುವ
ನಿಯ-ತ-ಕಾ-ಲಿಕ
ನಿಯ-ತ-ವಾಗಿ
ನಿಯಮ
ನಿಯ-ಮಕ್ಕೆ
ನಿಯ-ಮ-ಗಳ
ನಿಯ-ಮ-ಗಳನ್ನು
ನಿಯ-ಮ-ಗಳನ್ನೂ
ನಿಯ-ಮ-ಗಳನ್ನೆಲ್ಲ
ನಿಯ-ಮ-ಗಳಲ್ಲಿ
ನಿಯ-ಮ-ಗಳಿಂದ
ನಿಯ-ಮ-ಗ-ಳಿ-ಗ-ನು-ಸಾ-ರ-ವಾಗಿ
ನಿಯ-ಮ-ಗಳು
ನಿಯ-ಮ-ಗ-ಳೆಲ್ಲ
ನಿಯ-ಮ-ಗ-ಳೆ-ಲ್ಲಕ್ಕೂ
ನಿಯ-ಮ-ದಂತೆ
ನಿಯ-ಮ-ಬ-ದ್ಧ-ವಾಗಿ
ನಿಯ-ಮ-ವ-ನ್ನ-ನು-ಸ-ರಿಸಿ
ನಿಯ-ಮ-ವನ್ನು
ನಿಯ-ಮವು
ನಿಯ-ಮವೂ
ನಿಯ-ಮಾ-ವ-ಳಿ-ಯನ್ನು
ನಿಯ-ಮಿಸು
ನಿಯೋ-ಗದ
ನಿಯೋ-ಗ-ವೊಂದು
ನಿಯೋ-ಜಿ-ಸ-ಬೇಕು
ನಿಯೋ-ಜಿ-ಸ-ಲ್ಪ-ಟ್ಟ-ವ-ರ-ಲ್ಲವೆ
ನಿಯೋ-ಜಿ-ಸ-ಲ್ಪ-ಟ್ಟಿದ್ದ
ನಿರಂ-ಜನಾ
ನಿರಂ-ಜ-ನಾ-ನಂ-ದರ
ನಿರಂ-ಜ-ನಾ-ನಂ-ದ-ರತ್ತ
ನಿರಂ-ಜ-ನಾ-ನಂ-ದರು
ನಿರಂ-ಜ-ನಾ-ನಂ-ದರೂ
ನಿರಂ-ಜ-ನಾ-ನಂ-ದ-ರೊಂ-ದಿಗೆ
ನಿರಂ-ತರ
ನಿರಂ-ತ-ರ-ವಾಗಿ
ನಿರತ
ನಿರ-ತ-ನಾ-ಗಿದ್ದ
ನಿರ-ತ-ನಾ-ಗಿ-ದ್ದು-ಬಿ-ಟ್ಟರೆ
ನಿರ-ತ-ನಾ-ಗಿ-ರು-ವುದು
ನಿರ-ತ-ನಾ-ಗು-ತ್ತಾನೆ
ನಿರ-ತ-ನಾದ
ನಿರ-ತ-ರಾಗಿ
ನಿರ-ತ-ರಾ-ಗಿದ್ದ
ನಿರ-ತ-ರಾ-ಗಿ-ದ್ದರು
ನಿರ-ತ-ರಾ-ಗಿ-ದ್ದಾರೆ
ನಿರ-ತ-ರಾ-ಗಿ-ರ-ಬೇ-ಕೆಂಬ
ನಿರ-ತ-ರಾ-ಗಿ-ರು-ತ್ತಿ-ದ್ದರು
ನಿರ-ತ-ರಾ-ದರು
ನಿರ-ತ-ಳಾ-ಗ-ಬೇಕು
ನಿರ-ತ-ಳಾ-ಗಿ-ದ್ದರೆ
ನಿರ-ತ-ಳಾ-ಗು-ವ-ವ-ಳಾ-ದ್ದ-ರಿಂದ
ನಿರ-ತ-ವಾ-ಗಿತ್ತು
ನಿರ-ತ-ವಾದ
ನಿರ-ರ್ಗಳ
ನಿರ-ರ್ಗ-ಳ-ತೆ-ಯನ್ನು
ನಿರ-ರ್ಗ-ಳ-ವಾಗಿ
ನಿರ-ರ್ಥ-ಕ-ವಾ-ದದ್ದು
ನಿರ-ಹಂ-ಭಾವ
ನಿರಾ-ಕ-ರಿ-ಸಲು
ನಿರಾ-ಕ-ರಿಸಿ
ನಿರಾ-ಕ-ರಿ-ಸಿ-ದಾಗ
ನಿರಾ-ಕ-ರಿ-ಸಿ-ಬಿಟ್ಟ
ನಿರಾ-ಕ-ರಿ-ಸಿ-ಬಿ-ಟ್ಟುವು
ನಿರಾ-ಕ-ರಿಸು
ನಿರಾ-ಕ-ರಿ-ಸು-ತ್ತಾರೆ
ನಿರಾ-ಕಾರ
ನಿರಾ-ಕಾ-ರ-ಬ್ರ-ಹ್ಮದ
ನಿರಾ-ಕಾ-ರ-ಬ್ರ-ಹ್ಮ-ದಲ್ಲಿ
ನಿರಾ-ತಂಕ
ನಿರಾ-ತಂ-ಕ-ವಾಗಿ
ನಿರಾ-ಳವಾ
ನಿರಾ-ಳ-ವಾ-ಗೇನೂ
ನಿರಾ-ಳ-ವಾದ
ನಿರಾ-ಶ-ರಾ-ಗ-ಲಿಲ್ಲ
ನಿರಾ-ಶ-ರಾಗಿ
ನಿರಾ-ಶಾ-ದಾ-ಯ-ಕ-ವಾಗಿ
ನಿರಾ-ಶಾ-ದಾ-ಯ-ಕ-ವಾದ
ನಿರಾ-ಶಾ-ಭಾವ
ನಿರಾ-ಶಾ-ಭಾ-ವ-ವನ್ನು
ನಿರಾ-ಶಾ-ಭಾ-ವ-ವ-ನ್ನುಂಟು
ನಿರಾಶೆ
ನಿರಾ-ಶೆ-ಗಾ-ಗಲಿ
ನಿರಾ-ಶೆ-ಗೊ-ಳಿ-ಸು-ವುದು
ನಿರಾ-ಶೆ-ಗೊ-ಳ್ಳು-ವಂ-ತಾ-ಯಿತು
ನಿರಾ-ಶೆಯ
ನಿರಾ-ಶೆ-ಯ-ನ್ನುಂ-ಟು-ಮಾ-ಡಿತು
ನಿರಾ-ಶೆ-ಯ-ನ್ನುಂ-ಟು-ಮಾ-ಡಿತ್ತು
ನಿರಾ-ಶೆ-ಯನ್ನೇ
ನಿರಾ-ಶೆ-ಯಾ-ಯಿತು
ನಿರಾ-ಶೆ-ಯಿಂದ
ನಿರಾ-ಶೆಯು
ನಿರಾ-ಶೆಯೇ
ನಿರಾ-ಶೆ-ಯೇನೋ
ನಿರಾ-ಶ್ರಿತ
ನಿರಾಸೆ
ನಿರಾ-ಸೆ-ಯಾ-ಗ-ದಂತೆ
ನಿರಾ-ಸೆ-ಯಾ-ದರೂ
ನಿರೀ-ಕ್ಷಿ-ಸ-ಬ-ಹುದು
ನಿರೀ-ಕ್ಷಿ-ಸ-ಬ-ಹುದೇ
ನಿರೀ-ಕ್ಷಿ-ಸ-ಲಾ-ಗಿತ್ತು
ನಿರೀ-ಕ್ಷಿಸಿ
ನಿರೀ-ಕ್ಷಿ-ಸಿದ್ದ
ನಿರೀ-ಕ್ಷಿ-ಸಿ-ದ್ದಂ-ತಹ
ನಿರೀ-ಕ್ಷಿ-ಸಿ-ದ್ದಂತೆ
ನಿರೀ-ಕ್ಷಿ-ಸಿ-ದ್ದರು
ನಿರೀ-ಕ್ಷಿ-ಸಿಯೇ
ನಿರೀ-ಕ್ಷಿ-ಸಿ-ರ-ಬೇಕು
ನಿರೀ-ಕ್ಷಿ-ಸಿ-ರ-ಲಿಲ್ಲ
ನಿರೀ-ಕ್ಷಿ-ಸುತ್ತ
ನಿರೀ-ಕ್ಷಿ-ಸು-ತ್ತಿ-ದ್ದರು
ನಿರೀ-ಕ್ಷಿ-ಸು-ತ್ತಿ-ದ್ದಾರೆ
ನಿರೀ-ಕ್ಷಿ-ಸು-ತ್ತಿ-ರು-ತ್ತೇವೆ
ನಿರೀ-ಕ್ಷಿ-ಸು-ತ್ತೀರಿ
ನಿರೀ-ಕ್ಷಿ-ಸುವ
ನಿರೀ-ಕ್ಷಿ-ಸು-ವುದು
ನಿರೀ-ಕ್ಷೆ-ಗಳನ್ನು
ನಿರೀ-ಕ್ಷೆ-ಗಳೂ
ನಿರೀ-ಕ್ಷೆ-ಯನ್ನೂ
ನಿರೀ-ಕ್ಷೆ-ಯಿಂದ
ನಿರೂ-ಪಿ-ಸಿ-ದರು
ನಿರೂ-ಪಿ-ಸಿ-ದ್ದಾ-ರೆ-ಪ-ರ್ವ-ತದ
ನಿರೂ-ಪಿ-ಸಿ-ದ್ದಾಳೆ
ನಿರೂ-ಪಿ-ಸು-ತ್ತಿದ್ದ
ನಿರೂ-ಪಿ-ಸು-ತ್ತಿ-ರು-ವಾಗ
ನಿರೋ-ಧವೂ
ನಿರೋ-ಧ-ಶಕ್ತಿ
ನಿರೋ-ಧಿ-ಸಲು
ನಿರೋ-ಧಿಸಿ
ನಿರ್ಗ-ತಿಕ
ನಿರ್ಗ-ತಿ-ಕನ
ನಿರ್ಗ-ತಿ-ಕ-ನಾ-ಗಿ-ದ್ದನೋ
ನಿರ್ಗ-ತಿ-ಕರೊ
ನಿರ್ಗ-ಮ-ನಕ್ಕೆ
ನಿರ್ಗ-ಮ-ನದ
ನಿರ್ಗ-ಮ-ನ-ದಿಂ-ದಾಗಿ
ನಿರ್ಗ-ಮಿ-ಸಿದ
ನಿರ್ಗ-ಮಿ-ಸಿ-ದರು
ನಿರ್ಜನ
ನಿರ್ಜ-ನ-ಪ್ರ-ಶಾಂತ
ನಿರ್ಜೀ-ವ-ವಾ-ಗಿಲ್ಲ
ನಿರ್ಜೀ-ವ-ವಾ-ದಂ-ತಿದ್ದ
ನಿರ್ಣಯ
ನಿರ್ಣ-ಯ-ಗಳನ್ನು
ನಿರ್ಣ-ಯ-ಗಳು
ನಿರ್ಣ-ಯ-ವನ್ನು
ನಿರ್ಣ-ಯಿ-ಸು-ವುದು
ನಿರ್ದ-ಯ-ವಾಗಿ
ನಿರ್ದ-ಯ-ವಾದ
ನಿರ್ದಾ-ಕ್ಷಿ-ಣ್ಯ-ವಾಗಿ
ನಿರ್ದಿಷ್ಟ
ನಿರ್ದಿ-ಷ್ಟ-ವಾದ
ನಿರ್ದೇ-ಶ-ಕರ
ನಿರ್ದೇ-ಶ-ಕ-ರಿರ
ನಿರ್ದೇ-ಶನ
ನಿರ್ದೇ-ಶಿ-ಸಿ-ದರು
ನಿರ್ದೇ-ಶಿ-ಸಿದ್ದ
ನಿರ್ದೇ-ಶಿ-ಸುವ
ನಿರ್ದೇ-ಶಿ-ಸು-ವಂ-ತೆಯೂ
ನಿರ್ದೋ-ಷಿ-ಗ-ಳೆಂದು
ನಿರ್ಧ-ರಿ-ಸ-ಬೇಕು
ನಿರ್ಧ-ರಿ-ಸ-ಲಾ-ಗ-ದಂ-ತಹ
ನಿರ್ಧ-ರಿ-ಸ-ಲಾ-ಯಿತು
ನಿರ್ಧ-ರಿಸಿ
ನಿರ್ಧ-ರಿ-ಸಿ-ದರು
ನಿರ್ಧ-ರಿ-ಸಿ-ದಳು
ನಿರ್ಧ-ರಿ-ಸಿ-ದ್ದಂ-ತಿತ್ತು
ನಿರ್ಧ-ರಿ-ಸಿ-ದ್ದರು
ನಿರ್ಧ-ರಿ-ಸಿ-ದ್ದರೇ
ನಿರ್ಧ-ರಿ-ಸಿ-ದ್ದು-ದ-ರಿಂದ
ನಿರ್ಧ-ರಿ-ಸಿ-ದ್ದೇನೆ
ನಿರ್ಧ-ರಿ-ಸಿ-ಬಿಟ್ಟ
ನಿರ್ಧ-ರಿ-ಸಿ-ಬಿ-ಟ್ಟಿದ್ದೆ
ನಿರ್ಧ-ರಿ-ಸಿ-ಬಿ-ಟ್ಟಿ-ದ್ದೇನೆ
ನಿರ್ಧ-ರಿ-ಸಿ-ರು-ವೆ-ಯೆಂಬ
ನಿರ್ಧ-ರಿ-ಸು-ತ್ತೀರೋ
ನಿರ್ಧ-ರಿ-ಸುವ
ನಿರ್ಧಾರ
ನಿರ್ಧಾ-ರ-ಕ್ಕಾಗಿ
ನಿರ್ಧಾ-ರಕ್ಕೂ
ನಿರ್ಧಾ-ರಕ್ಕೆ
ನಿರ್ಧಾ-ರ-ಗಳನ್ನು
ನಿರ್ಧಾ-ರದ
ನಿರ್ಧಾ-ರ-ವನ್ನು
ನಿರ್ಧಾ-ರ-ವಾ-ಗಿ-ರ-ಲಿಲ್ಲ
ನಿರ್ಧಾ-ರ-ವಾ-ಯಿತು
ನಿರ್ಧಾ-ರ-ವೇ-ನೆಂದು
ನಿರ್ಧಾ-ರಿ-ತ-ವಾ-ಗಿದ್ದ
ನಿರ್ನಾ-ಮ-ಗೈ-ಯಲು
ನಿರ್ನಾ-ಮ-ವಾ-ಗು-ತ್ತದೆ
ನಿರ್ನಾ-ಮ-ವಾ-ಗು-ವಿರಿ
ನಿರ್ನಾ-ಮ-ವಾ-ಯಿತು
ನಿರ್ನಾ-ಮವು
ನಿರ್ಬಂ-ಧ-ಗಳನ್ನು
ನಿರ್ಭಯ
ನಿರ್ಭ-ಯತೆ
ನಿರ್ಭ-ಯ-ತೆ-ಇದು
ನಿರ್ಭ-ಯ-ವಾಗಿ
ನಿರ್ಭಯಾ
ನಿರ್ಭ-ಯಾ-ನಂದ
ನಿರ್ಭ-ಯಾ-ನಂ-ದ-ರಿಗೆ
ನಿರ್ಭ-ಯಾ-ನಂ-ದರು
ನಿರ್ಭ-ಯಾ-ನಂ-ದರೂ
ನಿರ್ಭ-ಯಾ-ನಂರೂ
ನಿರ್ಭಾ-ಗ್ಯರ
ನಿರ್ಭಾ-ಗ್ಯರು
ನಿರ್ಭಿ-ಡೆಯ
ನಿರ್ಭಿ-ಡೆ-ಯಿಂದ
ನಿರ್ಭೀತ
ನಿರ್ಭೀ-ತ-ರಾಗಿ
ನಿರ್ಭೀ-ತರು
ನಿರ್ಮ-ಲೀ-ಕ-ರ-ಣದ
ನಿರ್ಮಾಣ
ನಿರ್ಮಾ-ಣ-ಕಾ-ರಿ-ಯಾದ
ನಿರ್ಮಾ-ಣ-ಕಾ-ರ್ಯ-ದಲ್ಲಿ
ನಿರ್ಮಾ-ಣ-ಕಾ-ರ್ಯ-ವನ್ನು
ನಿರ್ಮಾ-ಣ-ಕ್ಕಾಗಿ
ನಿರ್ಮಾ-ಣಕ್ಕೆ
ನಿರ್ಮಾ-ಣ-ಗೊಂ-ಡದ್ದು
ನಿರ್ಮಾ-ಣ-ಗೊಂ-ಡಿತು
ನಿರ್ಮಾ-ಣ-ಗೊಂ-ಡಿತ್ತು
ನಿರ್ಮಾ-ಣ-ಗೊಂ-ಡಿದೆ
ನಿರ್ಮಾ-ಣ-ಗೊಂ-ಡಿದ್ದ
ನಿರ್ಮಾ-ಣ-ಗೊಂ-ಡಿಲ್ಲ
ನಿರ್ಮಾ-ಣ-ಗೊ-ಳ್ಳ-ಬೇ-ಕಾ-ಗಿದೆ
ನಿರ್ಮಾ-ಣ-ಗೊ-ಳ್ಳ-ಬೇಕು
ನಿರ್ಮಾ-ಣ-ಗೊ-ಳ್ಳು-ವಂ-ತಿ-ದ್ದರೆ
ನಿರ್ಮಾ-ಣ-ದ-ವರೆ-ಗಿನ
ನಿರ್ಮಾ-ಣ-ದಿಂದ
ನಿರ್ಮಾ-ಣ-ಮಾ-ಡಲು
ನಿರ್ಮಾ-ಣ-ವಾ-ಗ-ಬೇ-ಕೆಂದು
ನಿರ್ಮಾ-ಣ-ವಾ-ಗಿ-ತ್ತೆಂ-ದರೆ
ನಿರ್ಮಾ-ಣ-ವಾ-ಗು-ತ್ತಿದೆ
ನಿರ್ಮಾ-ಣ-ವಾ-ಯಿತು
ನಿರ್ಮಾ-ಣವೇ
ನಿರ್ಮಾತೃ
ನಿರ್ಮಾ-ಪಕ
ನಿರ್ಮಿ
ನಿರ್ಮಿತ
ನಿರ್ಮಿ-ತ-ವಾದ
ನಿರ್ಮಿ-ತಿ-ಗಳು
ನಿರ್ಮಿ-ಸ-ಬಲ್ಲ
ನಿರ್ಮಿ-ಸ-ಬ-ಲ್ಲಳು
ನಿರ್ಮಿ-ಸ-ಬ-ಹುದು
ನಿರ್ಮಿ-ಸ-ಬೇ-ಕಾ-ಗಿತ್ತು
ನಿರ್ಮಿ-ಸ-ಲಾ-ಗಿತ್ತು
ನಿರ್ಮಿ-ಸ-ಲಾಗಿದೆ
ನಿರ್ಮಿ-ಸ-ಲಾ-ಗಿದ್ದ
ನಿರ್ಮಿ-ಸಿ-ಕೊಂ-ಡಿ-ದ್ದರು
ನಿರ್ಮಿ-ಸಿ-ಕೊ-ಳ್ಳ-ಬಲ್ಲೆ
ನಿರ್ಮಿ-ಸಿ-ದ್ದುವು
ನಿರ್ಮಿ-ಸು-ತ್ತಿ-ದ್ದುವು
ನಿರ್ಮಿ-ಸು-ವುದು
ನಿರ್ಮೂ-ಲ-ನ-ಕ್ಕಾಗಿ
ನಿರ್ಮೂ-ಲ-ನ-ಗೊ-ಳಿ-ಸಲು
ನಿರ್ಮೂ-ಲ-ಮಾ-ಡು-ವ-ಕ-ಲ-ಬೆ-ರ-ಕೆಯ
ನಿರ್ಯಾಣ
ನಿರ್ಯಾ-ಣದ
ನಿರ್ಯಾ-ಣ-ವನ್ನು
ನಿರ್ಯಾ-ಣ-ಹೊಂ-ದಿದ
ನಿರ್ಯಾ-ಣಾ-ನಂ-ತ-ರವೂ
ನಿರ್ಲ-ಕ್ಷಿ-ಸ-ಬ-ಲ್ಲರು
ನಿರ್ಲ-ಕ್ಷಿ-ಸಿ-ದರು
ನಿರ್ಲಕ್ಷ್ಯ
ನಿರ್ಲಿ-ಪ್ತತೆ
ನಿರ್ಲಿ-ಪ್ತ-ತೆಯ
ನಿರ್ಲಿ-ಪ್ತ-ತೆ-ಯಿಂದ
ನಿರ್ಲಿ-ಪ್ತ-ಭಾವ
ನಿರ್ಲಿ-ಪ್ತ-ರಾ-ಗಿ-ದ್ದರು
ನಿರ್ಲಿ-ಪ್ತ-ರಾ-ಗಿ-ದ್ದ-ರೆಂ-ದರೆ
ನಿರ್ಲಿ-ಪ್ತ-ರಾ-ಗಿ-ದ್ದಾರೋ
ನಿರ್ಲಿ-ಪ್ತ-ವಾಗಿ
ನಿರ್ವ-ಹಣೆ
ನಿರ್ವ-ಹ-ಣೆ-ಗಾಗಿ
ನಿರ್ವ-ಹ-ಣೆ-ಯನ್ನು
ನಿರ್ವ-ಹ-ಣೆ-ಯಿಂದ
ನಿರ್ವಹಿ
ನಿರ್ವ-ಹಿ-ಸ-ಬೇಕು
ನಿರ್ವ-ಹಿ-ಸ-ಲಾ-ಗು-ವುದು
ನಿರ್ವ-ಹಿ-ಸ-ಲಾ-ರರು
ನಿರ್ವ-ಹಿ-ಸಲು
ನಿರ್ವ-ಹಿಸಿ
ನಿರ್ವ-ಹಿ-ಸಿ-ದರು
ನಿರ್ವ-ಹಿ-ಸಿ-ದು-ದ-ಕ್ಕಾಗಿ
ನಿರ್ವ-ಹಿ-ಸು-ತ್ತಿ-ದ್ದರು
ನಿರ್ವ-ಹಿ-ಸು-ತ್ತಿ-ರು-ವಿರೋ
ನಿರ್ವಾಣ
ನಿರ್ವಾ-ತ-ವನ್ನು
ನಿರ್ವಿ-ಕಲ್ಪ
ನಿರ್ವೀ-ರ್ಯ-ವಾ-ಗಿಲ್ಲ
ನಿಲು-ಕ-ದಷ್ಟು
ನಿಲು-ಕದ್ದು
ನಿಲು-ಕು-ತ್ತವೆ
ನಿಲು-ಗ-ಡೆ-ಯಿ-ರ-ಲಿಲ್ಲ
ನಿಲು-ವಂಗಿ
ನಿಲು-ವಂ-ಗಿ-ಯನ್ನು
ನಿಲು-ವನ್ನು
ನಿಲುವಿ
ನಿಲು-ವಿನ
ನಿಲು-ವಿ-ನಿಂ-ದಾಗಿ
ನಿಲುವು
ನಿಲ್ದಾಣ
ನಿಲ್ದಾ-ಣಕ್ಕೆ
ನಿಲ್ದಾ-ಣದ
ನಿಲ್ದಾ-ಣ-ದಲ್ಲಿ
ನಿಲ್ದಾ-ಣ-ದ-ಲ್ಲಿಯೂ
ನಿಲ್ದಾ-ಣ-ದಿಂದ
ನಿಲ್ದಾ-ಣ-ವನ್ನು
ನಿಲ್ದಾ-ಣ-ವನ್ನೂ
ನಿಲ್ದಾ-ಣ-ವೆಂದರೆ
ನಿಲ್ಲ-ದಿರಿ
ನಿಲ್ಲದೆ
ನಿಲ್ಲ-ಬಲ್ಲ
ನಿಲ್ಲ-ಬ-ಲ್ಲ-ವ-ರಾರು
ನಿಲ್ಲ-ಬ-ಲ್ಲಿರಿ
ನಿಲ್ಲ-ಬ-ಲ್ಲು-ದೆಂ-ಬು-ದನ್ನು
ನಿಲ್ಲ-ಬೇ-ಕಾ-ಗು-ತ್ತದೆ
ನಿಲ್ಲ-ಬೇ-ಕಾ-ಯಿತು
ನಿಲ್ಲ-ಬೇಕು
ನಿಲ್ಲ-ಬೇಕೇ
ನಿಲ್ಲ-ಲಾ-ರದು
ನಿಲ್ಲ-ಲಾ-ರರು
ನಿಲ್ಲಲಿ
ನಿಲ್ಲ-ಲಿತ್ತೋ
ನಿಲ್ಲ-ಲಿಲ್ಲ
ನಿಲ್ಲಲು
ನಿಲ್ಲಲೂ
ನಿಲ್ಲ-ಲೇ-ಬೇಕು
ನಿಲ್ಲಸ
ನಿಲ್ಲಿ
ನಿಲ್ಲಿಸ
ನಿಲ್ಲಿ-ಸ-ಬೇ-ಕಾ-ಯಿ-ತಾ-ದರೂ
ನಿಲ್ಲಿ-ಸ-ಬೇ-ಕಾ-ಯಿತು
ನಿಲ್ಲಿ-ಸ-ಬೇಕು
ನಿಲ್ಲಿ-ಸ-ಬೇ-ಕೆಂದು
ನಿಲ್ಲಿ-ಸ-ಬೇ-ಕೆಂದೇ
ನಿಲ್ಲಿ-ಸಲು
ನಿಲ್ಲಿಸಿ
ನಿಲ್ಲಿ-ಸಿದ
ನಿಲ್ಲಿ-ಸಿ-ದರು
ನಿಲ್ಲಿ-ಸಿದೆ
ನಿಲ್ಲಿ-ಸಿ-ಬಿ-ಟ್ಟರು
ನಿಲ್ಲಿ-ಸಿ-ಬಿ-ಟ್ಟಿದ್ದ
ನಿಲ್ಲಿ-ಸಿ-ಬಿ-ಟ್ಟಿ-ದ್ದ-ರಂತೆ
ನಿಲ್ಲಿ-ಸಿ-ಬಿ-ಟ್ಟಿ-ದ್ದೇನೆ
ನಿಲ್ಲಿ-ಸಿ-ಬಿ-ಡು-ವಂತೆ
ನಿಲ್ಲಿ-ಸಿವೆ
ನಿಲ್ಲಿ-ಸುವ
ನಿಲ್ಲಿ-ಸು-ವಂತೆ
ನಿಲ್ಲಿ-ಸು-ವುದು
ನಿಲ್ಲು
ನಿಲ್ಲು-ತ್ತದೆ
ನಿಲ್ಲು-ತ್ತದೊ
ನಿಲ್ಲು-ತ್ತಲೇ
ನಿಲ್ಲು-ತ್ತಿತ್ತು
ನಿಲ್ಲು-ತ್ತಿ-ದ್ದಂ-ತೆಯೇ
ನಿಲ್ಲು-ತ್ತಿ-ದ್ದರು
ನಿಲ್ಲು-ತ್ತಿ-ದ್ದಾಳೆ
ನಿಲ್ಲು-ತ್ತಿವೆ
ನಿಲ್ಲು-ತ್ತೇನೆ
ನಿಲ್ಲು-ವಂ-ತಾ-ಗ-ಬೇ-ಕಾ-ದರೆ
ನಿಲ್ಲು-ವಂ-ತಿಲ್ಲ
ನಿಲ್ಲು-ವಂ-ತಿ-ಲ್ಲ-ವಲ್ಲ
ನಿಲ್ಲು-ವಂತೆ
ನಿಲ್ಲು-ವುದನ್ನು
ನಿಲ್ಲು-ವು-ದಿಲ್ಲ
ನಿವಾ-ರಣೆ
ನಿವಾ-ರ-ಣೆ-ಗಾಗಿ
ನಿವಾ-ರ-ಣೆ-ಯಾ-ಗ-ಲಿಲ್ಲ
ನಿವಾ-ರಿ-ಸಲು
ನಿವಾ-ರಿ-ಸ-ಲೋ-ಸ್ಕರ
ನಿವಾ-ಸಕ್ಕೆ
ನಿವಾ-ಸ-ದಲ್ಲಿ
ನಿವಾ-ಸ-ದ-ವ-ರೆಗೆ
ನಿವಾ-ಸ-ವನ್ನು
ನಿವಾ-ಸ-ವಾದ
ನಿವಾ-ಸಿ-ಗ-ಳ-ನ್ನು-ದ್ದೇ-ಶಿಸಿ
ನಿವಾ-ಸಿ-ಗ-ಳಾದ
ನಿವಾ-ಸಿ-ಗ-ಳಿಗೆ
ನಿವಾ-ಸಿ-ಗಳು
ನಿವೃತ್ತ
ನಿವೃ-ತ್ತ-ರಾದ
ನಿವೃತ್ತಿ
ನಿವೆಯೋ
ನಿವೇ
ನಿವೇ-ದಿ-ತಳ
ನಿವೇ-ದಿ-ತ-ಳ-ನ್ನಾ-ಗಿ-ಸಿದ
ನಿವೇ-ದಿತಾ
ನಿವೇ-ದಿ-ತಾ-ಇ-ವರ
ನಿವೇ-ದಿ-ತಾ-ಕೇ-ವಲ
ನಿವೇ-ದಿ-ತಾ-ತಮ್ಮ
ನಿವೇ-ದಿ-ತಾ-ಭಾ-ರ-ತದ
ನಿವೇ-ದಿ-ತಾಳ
ನಿವೇ-ದಿತೆ
ನಿವೇ-ದಿ-ತೆ-ಗಾಗಿ
ನಿವೇ-ದಿ-ತೆ-ಗಿನ್ನೂ
ನಿವೇ-ದಿ-ತೆಗೂ
ನಿವೇ-ದಿ-ತೆಗೆ
ನಿವೇ-ದಿ-ತೆ-ಗೆ-ತೀವ್ರ
ನಿವೇ-ದಿ-ತೆ-ಗೆ-ಶಿ-ಕ್ಷಣ
ನಿವೇ-ದಿ-ತೆ-ಗೊಂದು
ನಿವೇ-ದಿ-ತೆಯ
ನಿವೇ-ದಿ-ತೆ-ಯತ್ತ
ನಿವೇ-ದಿ-ತೆ-ಯ-ದೆಂ-ಬುದು
ನಿವೇ-ದಿ-ತೆ-ಯ-ನ್ನಾಗಿ
ನಿವೇ-ದಿ-ತೆ-ಯನ್ನು
ನಿವೇ-ದಿ-ತೆ-ಯ-ನ್ನು-ದ್ದೇ-ಶಿಸಿ
ನಿವೇ-ದಿ-ತೆ-ಯಲ್ಲೂ
ನಿವೇ-ದಿ-ತೆ-ಯಾ-ಗ-ಲಿ-ರುವ
ನಿವೇ-ದಿ-ತೆ-ಯಾ-ದರೂ
ನಿವೇ-ದಿ-ತೆ-ಯಾ-ದರೋ
ನಿವೇ-ದಿ-ತೆಯು
ನಿವೇ-ದಿ-ತೆಯೂ
ನಿವೇ-ದಿ-ತೆ-ಯೊಂ-ದಿ-ಗಿನ
ನಿವೇ-ದಿ-ತೆ-ಯೊಂ-ದಿಗೆ
ನಿವೇ-ದಿ-ಸ-ಲ್ಪ-ಟ್ಟ-ವ-ಳ-ನ್ನಾಗಿ
ನಿವೇ-ದಿ-ಸ-ಲ್ಪ-ಟ್ಟ-ವಳು
ನಿವೇ-ಶ-ನಕ್ಕೆ
ನಿವೇ-ಶ-ನದ
ನಿವೇ-ಶ-ನ-ದಲ್ಲಿ
ನಿವೇ-ಶ-ನ-ವನ್ನು
ನಿವೇ-ಶ-ನ-ವ-ನ್ನೆಲ್ಲ
ನಿವೇ-ಶ-ನವು
ನಿಶಿ-ತ-ವಾದ
ನಿಶ್ಚಯ
ನಿಶ್ಚ-ಯ-ವಾಗಿ
ನಿಶ್ಚ-ಯ-ವಾ-ಗಿತ್ತು
ನಿಶ್ಚ-ಯ-ವಾ-ಗಿದೆ
ನಿಶ್ಚ-ಯ-ವಾ-ಗಿಯೂ
ನಿಶ್ಚ-ಯ-ವಾ-ಯಿತು
ನಿಶ್ಚಯಿ
ನಿಶ್ಚ-ಯಿ-ಸ-ಲಾ-ಯಿತು
ನಿಶ್ಚ-ಯಿಸಿ
ನಿಶ್ಚ-ಯಿ-ಸಿದ
ನಿಶ್ಚ-ಯಿ-ಸಿ-ದಂ-ತಿತ್ತು
ನಿಶ್ಚ-ಯಿ-ಸಿ-ದರು
ನಿಶ್ಚ-ಯಿ-ಸಿ-ದರೆ
ನಿಶ್ಚ-ಯಿ-ಸಿ-ದ್ದರು
ನಿಶ್ಚ-ಯಿ-ಸಿ-ಬಿ-ಟ್ಟರು
ನಿಶ್ಚ-ಯಿ-ಸುವ
ನಿಶ್ಚಲ
ನಿಶ್ಚ-ಲ-ವಾಗಿ
ನಿಶ್ಚಿಂ-ತೆ-ಯಿಂದ
ನಿಶ್ಚಿತ
ನಿಶ್ಚಿ-ತ-ತೆ-ಯಿತ್ತು
ನಿಶ್ಚಿ-ತ-ವಾಗಿ
ನಿಶ್ಚಿ-ತ-ವಾ-ಗಿದೆ
ನಿಶ್ಚಿ-ತ-ವಾದ
ನಿಶ್ಚೇ-ಷ್ಟಿ-ತ-ರಾಗಿ
ನಿಶ್ಶಬ್ದ
ನಿಶ್ಶೇ-ಷ-ವಾಗಿ
ನಿಷ-ತ್ತು-ಗಳನ್ನು
ನಿಷೇಧ
ನಿಷ್ಕ-ಪಟ
ನಿಷ್ಕ-ಪ-ಟ-ತೆ-ಯನ್ನು
ನಿಷ್ಕ-ಪ-ಟಿ-ಗಳು
ನಿಷ್ಕಾಮ
ನಿಷ್ಕಾ-ಮ-ಕರ್ಮ
ನಿಷ್ಕಾ-ಮ-ಕ-ರ್ಮದ
ನಿಷ್ಕಾ-ಮ-ಕ-ರ್ಮ-ದಲ್ಲಿ
ನಿಷ್ಕಾ-ಮ-ಕ-ರ್ಮ-ದಿಂದ
ನಿಷ್ಕಾ-ಮ-ಕ-ರ್ಮ-ದಿಂ-ದಲೂ
ನಿಷ್ಕಾ-ಮ-ಕ-ರ್ಮವೂ
ನಿಷ್ಕಾ-ಮ-ಕ-ರ್ಮವೇ
ನಿಷ್ಕಾ-ಮ-ಸೇವೆ
ನಿಷ್ಠಾ-ಯುತ
ನಿಷ್ಠಾ-ವಂತ
ನಿಷ್ಠಾ-ವಂ-ತ-ನಾ-ಗಿ-ದ್ದೇನೆ
ನಿಷ್ಠು-ರದ
ನಿಷ್ಠು-ರ-ರಾ-ಗಿ-ದ್ದರು
ನಿಷ್ಠು-ರರೂ
ನಿಷ್ಠು-ರ-ವಾಗಿ
ನಿಷ್ಠೆ
ನಿಷ್ಠೆ-ಉ-ತ್ಸಾ-ಹ-ಗ-ಳನ್ನೇ
ನಿಷ್ಠೆ-ಯನ್ನು
ನಿಷ್ಪ-ಕ್ಷ-ಪಾತ
ನಿಷ್ಪ್ರ-ಯೋ-ಜ-ಕ-ರಾ-ಗುತ್ತ
ನಿಷ್ಪ್ರ-ಯೋ-ಜ-ಕರು
ನಿಷ್ಫ-ಲ-ವ-ಲ್ಲವೇ
ನಿಷ್ಫ-ಲ-ವಾ-ಗು-ತ್ತವೆ
ನಿಷ್ಫ-ಲ-ವಾ-ದರೆ
ನಿಸರ್ಗ
ನಿಸ್ತೇ-ಜ-ತೆ-ಯನ್ನೂ
ನಿಸ್ವಾರ್ಥ
ನಿಸ್ವಾ-ರ್ಥ-ಕ-ರ್ಮ-ವನ್ನು
ನಿಸ್ವಾರ್ಥಿ
ನಿಸ್ವಾ-ರ್ಥಿ-ಗ-ಳಾ-ಗಿ-ದ್ದರೂ
ನಿಸ್ಸಂ-ಕೋ-ಚ-ವಾಗಿ
ನಿಸ್ಸಂ-ದೇ-ಹ-ವಾಗಿ
ನಿಸ್ಸಂ-ಶಯ
ನಿಸ್ಸಂ-ಶ-ಯ-ವಾಗಿ
ನಿಸ್ಸ-ಹಾ-ಯ-ಕ-ರಾಗಿ
ನಿಸ್ಸ-ಹಾ-ಯ-ಕ-ರಾ-ದ-ದ್ದು-ಇ-ವ-ನ್ನೆಲ್ಲ
ನಿಸ್ಸ-ಹಾ-ಯ-ಕರು
ನಿಸ್ಸೀಮ
ನೀ
ನೀಚ
ನೀಚ-ತನ
ನೀಚ-ಬು-ದ್ಧಿ-ಪ್ರೇ-ರಿ-ತ-ವಾದ
ನೀಚ-ಮ-ಟ್ಟದ
ನೀಚರೂ
ನೀಚಾ-ವಾಚಾ
ನೀಡ-ತೊ-ಡ-ಗಿ-ದರು
ನೀಡ-ಬಲ್ಲ
ನೀಡ-ಬ-ಲ್ಲ-ವ-ರಾರು
ನೀಡ-ಬ-ಲ್ಲಿರಾ
ನೀಡ-ಬ-ಹುದು
ನೀಡ-ಬೇ-ಕಾ-ಗು-ತ್ತದೆ
ನೀಡ-ಬೇ-ಕಾದ
ನೀಡ-ಬೇ-ಕಾ-ದ-ದ್ದಿದೆ
ನೀಡ-ಬೇಕು
ನೀಡ-ಬೇ-ಕೆಂದು
ನೀಡ-ಬೇ-ಕೆಂಬ
ನೀಡ-ಬೇ-ಕೆಂ-ಬುದು
ನೀಡ-ಲಾ-ಗಿತ್ತು
ನೀಡ-ಲಾ-ಗು-ತ್ತದೆ
ನೀಡ-ಲಾ-ಗು-ತ್ತಿದ್ದ
ನೀಡ-ಲಾ-ಗು-ವುದು
ನೀಡ-ಲಾದ
ನೀಡ-ಲಾ-ಯಿತು
ನೀಡಲಿ
ನೀಡ-ಲಿ-ರುವ
ನೀಡ-ಲಿಲ್ಲ
ನೀಡಲು
ನೀಡ-ಲೇ-ಬೇ-ಕೆಂದು
ನೀಡ-ಲೊ-ಪ್ಪಿ-ದರು
ನೀಡ-ವು-ದ-ಕ್ಕಾಗಿ
ನೀಡಿ
ನೀಡಿತು
ನೀಡಿದ
ನೀಡಿ-ದರು
ನೀಡಿ-ದ-ರು-ನಮ್ಮ
ನೀಡಿ-ದರೆ
ನೀಡಿ-ದ-ರೆಂದು
ನೀಡಿ-ದರೋ
ನೀಡಿ-ದ-ವ-ನಲ್ಲ
ನೀಡಿ-ದ-ವರು
ನೀಡಿ-ದಾಗ
ನೀಡಿ-ದುದು
ನೀಡಿ-ದುವು
ನೀಡಿ-ದು-ವೆಂದು
ನೀಡಿ-ದೆಯೇ
ನೀಡಿದ್ದ
ನೀಡಿ-ದ್ದ-ಕ್ಕಾಗಿ
ನೀಡಿ-ದ್ದರು
ನೀಡಿ-ದ್ದರೂ
ನೀಡಿ-ದ್ದರೆ
ನೀಡಿ-ದ್ದ-ಲ್ಲದೆ
ನೀಡಿ-ದ್ದಳು
ನೀಡಿ-ದ್ದಾರೆ
ನೀಡಿದ್ದು
ನೀಡಿ-ದ್ದೇನೆ
ನೀಡಿ-ಯಾ-ಯಿತು
ನೀಡಿಯೇ
ನೀಡಿ-ರುವ
ನೀಡಿ-ರುವೆ
ನೀಡಿಲ್ಲ
ನೀಡು
ನೀಡುತ್ತ
ನೀಡು-ತ್ತ-ದೆಂದು
ನೀಡುತ್ತಾ
ನೀಡು-ತ್ತಾ-ನೆಯೋ
ನೀಡು-ತ್ತಾರೆ
ನೀಡು-ತ್ತಿತ್ತು
ನೀಡು-ತ್ತಿದ್ದ
ನೀಡು-ತ್ತಿ-ದ್ದ-ರ-ಲ್ಲದೆ
ನೀಡು-ತ್ತಿ-ದ್ದರು
ನೀಡು-ತ್ತಿ-ದ್ದು-ದ-ರಿಂದ
ನೀಡು-ತ್ತಿ-ದ್ದುದು
ನೀಡು-ತ್ತಿ-ರ-ಲಿ-ಲ್ಲ-ವೆಂ-ದೇನೂ
ನೀಡು-ತ್ತಿ-ರುವ
ನೀಡು-ತ್ತಿ-ರು-ವ-ವರು
ನೀಡು-ತ್ತೇನೆ
ನೀಡು-ತ್ತೇ-ನೆ-ನಾನು
ನೀಡುವ
ನೀಡು-ವಂ-ತಾ-ಗ-ಬೇಕು
ನೀಡು-ವಂ-ತಿ-ರ-ಬೇಕು
ನೀಡು-ವಂತೆ
ನೀಡು-ವಲ್ಲಿ
ನೀಡು-ವ-ವನು
ನೀಡು-ವ-ವರು
ನೀಡು-ವಾಗ
ನೀಡು-ವಿ-ರೆಂದು
ನೀಡು-ವುದ
ನೀಡು-ವು-ದ-ಕ್ಕಾಗಿ
ನೀಡು-ವು-ದ-ರಲ್ಲಿ
ನೀಡು-ವು-ದ-ರೊಂ-ದಿಗೆ
ನೀಡು-ವು-ದಾಗಿ
ನೀಡು-ವುದು
ನೀಡು-ವುದೇ
ನೀತಿ
ನೀತಿ-ಗಳನ್ನೂ
ನೀತಿ-ಗಳು
ನೀತಿಗೆ
ನೀತಿ-ನಿ-ಪು-ಣರು
ನೀತಿ-ನಿ-ಪು-ಣಾಃ
ನೀತಿ-ನಿ-ಯ-ಮ-ಗ-ಳ-ನ್ನೊ-ಳ-ಗೊಂಡ
ನೀತಿ-ನಿ-ಯ-ಮ-ಗ-ಳಿ-ದ್ದು-ವ-ಲ್ಲವೆ
ನೀತಿ-ಪ-ರ-ವಾದ
ನೀತಿಯ
ನೀನಂ-ದದ್ದೇ
ನೀನ-ದನ್ನು
ನೀನಾಗು
ನೀನಿಂದು
ನೀನಿ-ದನ್ನು
ನೀನಿನ್ನೂ
ನೀನಿಲ್ಲಿ
ನೀನಿ-ಲ್ಲಿಗೆ
ನೀನೀಗ
ನೀನು
ನೀನೂ
ನೀನೆ
ನೀನೆಂ-ದಂತೆ
ನೀನೆಂ-ದಾ-ದರೂ
ನೀನೆ-ನ್ನು-ವುದೇ
ನೀನೇ
ನೀನೇ-ನಾ-ದರೂ
ನೀನೇನು
ನೀನೇನೂ
ನೀನೊಬ್ಬ
ನೀನೊ-ಬ್ಬನೇ
ನೀಯ-ವಾ-ಗು-ತ್ತದೆ
ನೀಯವೇ
ನೀರನ್ನು
ನೀರನ್ನೂ
ನೀರ-ನ್ನೆಲ್ಲ
ನೀರನ್ನೇ
ನೀರನ್ನೋ
ನೀರವ
ನೀರ-ವತೆ
ನೀರಸ
ನೀರ-ಸ-ವಾ-ದದ್ದು
ನೀರ-ಸ-ವಾ-ದ-ನಿ-ರ-ರ್ಥ-ಕ-ವಾದ
ನೀರ-ಸವೂ
ನೀರಾ-ದರು
ನೀರಿ-ಗಾಗಿ
ನೀರಿ-ಗಿಳಿ-ಸು-ತ್ತಾರೆ
ನೀರಿ-ಗಿಳಿ-ಸುವ
ನೀರಿಗೆ
ನೀರಿನ
ನೀರಿ-ನಲ್ಲಿ
ನೀರಿ-ಳಿ-ಯ-ದಂತೆ
ನೀರು
ನೀರು-ಕ್ಕು-ವಂತೆ
ನೀರು-ಶ-ತ-ಮಾ-ನ-ಗಳಿಂದ
ನೀರೆ-ರೆ-ಯು-ವುದೂ
ನೀರೇ
ನೀರೊ-ಳಗೆ
ನೀಲ
ನೀಲ-ವರ್ಣ
ನೀಲಾಂ-ಬರ
ನೀಲಾ-ಕಾಶ
ನೀಲಾ-ಕಾ-ಶ-ದಿಂದ
ನೀಲಾ-ಕಾ-ಶ-ದೆ-ಡೆಗೆ
ನೀಲಾ-ಕಾ-ಶ-ವನ್ನು
ನೀವಂದು
ನೀವಲ್ಲ
ನೀವಾ-ದರೆ
ನೀವಿಂದು
ನೀವಿ-ದ್ದೀ-ರಲ್ಲ
ನೀವಿನ್ನೂ
ನೀವಿ-ಬ್ಬರೂ
ನೀವಿ-ರುವ
ನೀವೀಗ
ನೀವು
ನೀವು-ಎಂ-ದರೆ
ನೀವು-ಗಳು
ನೀವೂ
ನೀವೆ-ನ್ನು-ವು-ದೇನೋ
ನೀವೆಲ್ಲ
ನೀವೆ-ಲ್ಲರೂ
ನೀವೇ
ನೀವೇಕೆ
ನೀವೇ-ನಾ-ದರೂ
ನೀವೇ-ನಿ-ದ್ದರೂ
ನೀವೇನು
ನೀವೇನೂ
ನೀವೇನೋ
ನೀವೊಂದು
ನೀವೊಬ್ಬ
ನೀವೊ-ಬ್ಬರು
ನುಂಗ-ಲಾ-ರದ
ನುಂಗಲು
ನುಂಗಿ
ನುಂಗಿ-ಕೊಂಡು
ನುಂಗಿ-ದಂತೆ
ನುಂಗಿ-ಹಾ-ಕು-ವಷ್ಟು
ನುಂಗು-ತಿ-ಹುದು
ನುಗ್ಗ-ದಂತೆ
ನುಗ್ಗ-ಬೇಕೆ
ನುಗ್ಗಲು
ನುಗ್ಗಿ
ನುಗ್ಗಿದ
ನುಗ್ಗಿ-ದರೆ
ನುಗ್ಗಿ-ಬಿ-ಡ-ಬೇಕೆ
ನುಗ್ಗು-ತಿ-ಹುದು
ನುಗ್ಗು-ತ್ತದೆ
ನುಗ್ಗು-ತ್ತಿ-ದ್ದರು
ನುಗ್ಗು-ವುದನ್ನು
ನುಚ್ಚು
ನುಡಿ
ನುಡಿ-ಕಿ-ಡಿ-ಗಳು
ನುಡಿ-ಗಳನ್ನು
ನುಡಿ-ಗಳನ್ನೆಲ್ಲ
ನುಡಿ-ಗ-ಳನ್ನೇ
ನುಡಿ-ಗಳಿಂದ
ನುಡಿ-ಗ-ಳಿವು
ನುಡಿ-ಗಳು
ನುಡಿ-ಗಳೋ
ನುಡಿದ
ನುಡಿ-ದರು
ನುಡಿ-ದ-ರು-ಗು-ರು-ಮ-ಹಾ-ರಾ-ಜರು
ನುಡಿ-ದ-ರು-ತಾವು
ನುಡಿ-ದ-ರು-ಮಾ-ನ-ವ-ರಲ್ಲಿ
ನುಡಿ-ದ-ರು-ಶಿ-ವ-ನನ್ನು
ನುಡಿ-ದ-ರು-ಶ್ರೀ-ರಾ-ಮ-ಕೃ-ಷ್ಣ-ರನ್ನು
ನುಡಿ-ದ-ರು-ಹೀಗೆ
ನುಡಿ-ದರೂ
ನುಡಿ-ದಳು
ನುಡಿ-ದಿದ್ದ
ನುಡಿ-ದಿ-ದ್ದರು
ನುಡಿ-ದಿ-ದ್ದರೆ
ನುಡಿ-ಮು-ತ್ತು-ಗಳ
ನುಡಿ-ಮು-ತ್ತು-ಗಳನ್ನೆಲ್ಲ
ನುಡಿ-ಯನ್ನು
ನುಡಿ-ಯಲು
ನುಡಿ-ಯಿತು
ನುಡಿ-ಯು-ತ್ತಾರೆ
ನುಡಿ-ಯು-ತ್ತಿ-ದ್ದಾ-ರೆಂ-ಬುದು
ನುಡಿ-ಯು-ತ್ತಿ-ರುವ
ನುಡಿ-ಯೊಂದು
ನುಡಿ-ಸಿ-ದರು
ನುಡಿ-ಸುತ್ತ
ನುಣು-ಪಾದ
ನುಣ್ಣಗೆ
ನುರಿ-ತ-ವ-ರ-ಲ್ಲದ
ನುರಿ-ತ-ವರು
ನುಸು-ಳ-ಬ-ಹು-ದಾ-ದ್ದನ್ನೂ
ನುಸುಳಿ
ನೂಕು
ನೂಕು-ನು-ಗ್ಗಲು
ನೂಕು-ನು-ಗ್ಗ-ಲು-ಗ-ಲಭೆ
ನೂತನ
ನೂರನೇ
ನೂರರ
ನೂರ-ರಷ್ಟು
ನೂರಾ-ಗಿದೆ
ನೂರಾರು
ನೂರಾ-ರು-ಸಾ-ವಿ-ರಾರು
ನೂರು
ನೂರು-ಗ-ಟ್ಟ-ಲೆ
ನೂರು-ಸಾ-ಸಿರ
ನೂರೇಳು
ನೂರೈದು
ನೆಂದರೆ
ನೆಂದು
ನೆಗಡಿ
ನೆಗ-ಡಿ-ಯಾ-ಗು-ತ್ತದೆ
ನೆಗೆದು
ನೆಗೆ-ಯುತ್ತ
ನೆಚ್ಚಿ-ಕೊಂ-ಡಿ-ರದೆ
ನೆಚ್ಚಿನ
ನೆಟ್ಟ-ಗಿ-ರ-ಲಿ-ಕ್ಕಿಲ್ಲ
ನೆಟ್ಟಗೆ
ನೆಟ್ಟಿದ್ದ
ನೆಟ್ಟಿವೆ
ನೆಟ್ಟು
ನೆತ್ತ-ರನ್ನು
ನೆತ್ತರು
ನೆತ್ತಿಯ
ನೆನ-ಕೆ-ಗಳನ್ನು
ನೆನ-ಪನ್ನು
ನೆನ-ಪನ್ನೇ
ನೆನ-ಪಾ-ಗು-ತ್ತಿ-ದ್ದವು
ನೆನ-ಪಾ-ಗು-ವು-ದಿ-ಲ್ಲವೆ
ನೆನ-ಪಾ-ಯಿತು
ನೆನ-ಪಿ-ಗಾಗಿ
ನೆನ-ಪಿಗೆ
ನೆನ-ಪಿ-ಟ್ಟು-ಕೊಂ-ಡಿ-ರ-ಬ-ಹು-ದೆಂದು
ನೆನ-ಪಿ-ಟ್ಟು-ಕೊಂಡು
ನೆನ-ಪಿ-ಟ್ಟು-ಕೊ-ಳ್ಳ-ಬೇಕು
ನೆನ-ಪಿ-ಡ-ಬೇಕು
ನೆನ-ಪಿ-ಡ-ಬೇ-ಕು-ಅ-ನು-ಕ-ರಣೆ
ನೆನ-ಪಿಡಿ
ನೆನ-ಪಿ-ಡು-ಕ-ರ್ತ-ವ್ಯ-ವೆಂ-ಬುದು
ನೆನ-ಪಿದೆ
ನೆನ-ಪಿ-ದೆಯೆ
ನೆನ-ಪಿ-ದೆಯೇ
ನೆನ-ಪಿನ
ನೆನ-ಪಿ-ನ-ಲ್ಲಿ-ಡ-ಬೇ-ಕಾದ
ನೆನ-ಪಿ-ನ-ಲ್ಲಿಡು
ನೆನ-ಪಿ-ರ-ದಿದ್ದ
ನೆನ-ಪಿ-ರ-ಬ-ಹು-ದೆಂದು
ನೆನ-ಪಿ-ರು-ತ್ತದೆ
ನೆನ-ಪಿಸಿ
ನೆನ-ಪಿ-ಸಿ-ಕೊಂಡು
ನೆನ-ಪಿ-ಸಿ-ಕೊಂ-ಡೊ-ಡ-ನೆಯೇ
ನೆನ-ಪಿ-ಸಿ-ಕೊ-ಡು-ತ್ತದೆ
ನೆನ-ಪಿ-ಸಿ-ಕೊ-ಡು-ತ್ತಿ-ದ್ದರು
ನೆನ-ಪಿ-ಸಿ-ಕೊ-ಡು-ತ್ತಿ-ದ್ದೇನೆ
ನೆನ-ಪಿ-ಸಿ-ಕೊಳ್ಳ
ನೆನ-ಪಿ-ಸಿ-ಕೊ-ಳ್ಳ-ಬ-ಹುದು
ನೆನ-ಪಿ-ಸಿ-ಕೊ-ಳ್ಳು-ತ್ತಿ-ದ್ದೇನೆ
ನೆನ-ಪಿ-ಸಿ-ದರು
ನೆನ-ಪಿಸು
ನೆನ-ಪಿ-ಸು-ತ್ತಿ-ದ್ದರು
ನೆನಪು
ನೆನ-ಪು-ಗಳಲ್ಲಿ
ನೆನ-ಪು-ಗ-ಳೆಲ್ಲ
ನೆನೆದು
ನೆನೆ-ಯುತ್ತ
ನೆನೆ-ಸಿ-ಕೊ-ಳ್ಳ-ಲಾರ
ನೆನೆ-ಸಿ-ರದ
ನೆಪ-ವೊಡ್ಡಿ
ನೆಪೋ
ನೆಪೋ-ಲಿ-ಯ-ನ್ನ-ನಿಗೆ
ನೆಮ್ಮ-ದಿ-ಗ-ಳಿಗೆ
ನೆಮ್ಮ-ದಿ-ಯೆ-ನಿ-ಸಿತು
ನೆಯ
ನೆಯೂ
ನೆಯೋ
ನೆರ-ಳಲ್ಲಿ
ನೆರ-ಳಲ್ಲೋ
ನೆರ-ಳಿ-ನಲ್ಲಿ
ನೆರಳು
ನೆರ-ಳು-ಗಳ
ನೆರ-ಳೆಂಬ
ನೆರಳೇ
ನೆರವ
ನೆರ-ವನ್ನು
ನೆರವಾ
ನೆರ-ವಾ-ಗ-ಬಲ್ಲ
ನೆರ-ವಾ-ಗ-ಬ-ಲ್ಲಳು
ನೆರ-ವಾ-ಗ-ಬ-ಲ್ಲ-ವರು
ನೆರ-ವಾ-ಗ-ಬಲ್ಲೆ
ನೆರ-ವಾ-ಗ-ಬೇಕಾ
ನೆರ-ವಾ-ಗ-ಬೇ-ಕಾ-ಗಿದೆ
ನೆರ-ವಾ-ಗ-ಬೇಕು
ನೆರ-ವಾ-ಗ-ಬೇ-ಕೆಂ-ದಿ-ರು-ವುದು
ನೆರ-ವಾ-ಗ-ಬೇ-ಕೆಂದು
ನೆರ-ವಾ-ಗ-ಬೇ-ಕೆಂಬ
ನೆರ-ವಾ-ಗಲಿ
ನೆರ-ವಾ-ಗಲು
ನೆರ-ವಾ-ಗ-ಲೆಂದು
ನೆರ-ವಾ-ಗ-ಲೇ-ಬೇಕು
ನೆರ-ವಾಗಿ
ನೆರ-ವಾ-ಗಿದ್ದ
ನೆರ-ವಾ-ಗಿ-ದ್ದಾರೆ
ನೆರ-ವಾಗು
ನೆರ-ವಾ-ಗುತ್ತ
ನೆರ-ವಾ-ಗು-ತ್ತಲೇ
ನೆರ-ವಾ-ಗು-ತ್ತ-ವೆಯೋ
ನೆರ-ವಾ-ಗು-ತ್ತಿದ್ದ
ನೆರ-ವಾ-ಗು-ತ್ತಿ-ದ್ದರು
ನೆರ-ವಾ-ಗುವ
ನೆರ-ವಾ-ಗು-ವಂ-ತಿ-ದ್ದರೆ
ನೆರ-ವಾ-ಗು-ವನೋ
ನೆರ-ವಾ-ಗು-ವ-ವರು
ನೆರ-ವಾ-ಗು-ವು-ದಾ-ಗಿತ್ತು
ನೆರ-ವಾ-ಗು-ವು-ದು-ಇದು
ನೆರ-ವಾದ
ನೆರ-ವಾ-ದಂತೆ
ನೆರ-ವಾ-ದರು
ನೆರ-ವಾ-ದರೆ
ನೆರ-ವಾ-ದ-ವರು
ನೆರ-ವಾ-ದೇನು
ನೆರ-ವಾ-ಯಿತು
ನೆರ-ವಿ-ಗಾಗಿ
ನೆರ-ವಿ-ಗಾ-ಗಿ-ರುವ
ನೆರ-ವಿಗೆ
ನೆರ-ವಿನ
ನೆರ-ವಿ-ನಿಂದ
ನೆರ-ವಿ-ಲ್ಲದೆ
ನೆರವು
ನೆರ-ವೇ-ರಿತು
ನೆರ-ವೇ-ರಿದ್ದು
ನೆರ-ವೇ-ರಿ-ರು-ತ್ತ-ದೆಂಬ
ನೆರ-ವೇ-ರಿ-ರು-ವು-ದ-ರಿಂದ
ನೆರ-ವೇ-ರಿ-ಸ-ಬೇ-ಕಾದ
ನೆರ-ವೇ-ರಿ-ಸಲು
ನೆರ-ವೇ-ರಿಸಿ
ನೆರ-ವೇ-ರಿ-ಸಿ-ದರು
ನೆರ-ವೇ-ರಿ-ಸಿ-ದ-ರೆಂ-ದರೆ
ನೆರ-ವೇ-ರಿ-ಸು-ವಲ್ಲಿ
ನೆರೆ
ನೆರೆ-ಗೂ-ದ-ಲಿನ
ನೆರೆ-ದರು
ನೆರೆದಿ
ನೆರೆ-ದಿತ್ತು
ನೆರೆ-ದಿದ್ದ
ನೆರೆ-ದಿ-ದ್ದಂ-ತಹ
ನೆರೆ-ದಿ-ದ್ದರು
ನೆರೆ-ದಿ-ದ್ದರೂ
ನೆರೆ-ದಿ-ದ್ದವ
ನೆರೆ-ದಿ-ದ್ದ-ವರ
ನೆರೆ-ದಿ-ದ್ದ-ವ-ರನ್ನು
ನೆರೆ-ದಿ-ದ್ದ-ವ-ರಲ್ಲಿ
ನೆರೆ-ದಿ-ದ್ದ-ವ-ರಿಗೆ
ನೆರೆ-ದಿ-ದ್ದ-ವರೆ-ಡೆಗೆ
ನೆರೆ-ದಿ-ದ್ದ-ವರೆಲ್ಲ
ನೆರೆ-ದಿ-ದ್ದ-ವರೆ-ಲ್ಲರೂ
ನೆರೆ-ದು-ಬಿ-ಟ್ಟಿ-ದ್ದರು
ನೆರೆ-ಯ-ಬೇ-ಕಾ-ಗಿತ್ತು
ನೆರೆ-ಯ-ಲಾ-ರಂ-ಭಿ-ಸಿ-ದ್ದರು
ನೆರೆ-ಯಿತು
ನೆರೆ-ರಾ-ಜ್ಯದ
ನೆಲ
ನೆಲ-ಅಂ-ತ-ಸ್ತನ್ನು
ನೆಲ-ಕ್ಕು-ರು-ಳಿ-ದರು
ನೆಲಕ್ಕೆ
ನೆಲದ
ನೆಲ-ದ-ಡಿ-ಯಲ್ಲಿ
ನೆಲ-ದಲ್ಲಿ
ನೆಲ-ದು-ದ್ದಕ್ಕೂ
ನೆಲ-ಮ-ಟ್ಟ-ದಿಂದ
ನೆಲ-ವನ್ನು
ನೆಲ-ವನ್ನೇ
ನೆಲ-ವೆಲ್ಲ
ನೆಲಸಿ
ನೆಲ-ಸಿತು
ನೆಲ-ಸಿತ್ತು
ನೆಲ-ಸಿ-ದರು
ನೆಲ-ಸಿ-ದ್ದರು
ನೆಲ-ಸಿ-ದ್ದ-ವ-ರೆಂ-ದರೆ
ನೆಲ-ಸಿ-ದ್ದುದು
ನೆಲ-ಸಿ-ರ-ದಿ-ದ್ದರೆ
ನೆಲ-ಸಿ-ರು-ವು-ದಾ-ದರೆ
ನೆಲ-ಸು-ತ್ತಾರೆ
ನೆಲ-ಸು-ತ್ತೇನೆ
ನೆಲ-ಸು-ವಂತೆ
ನೆಲ-ಸು-ವು-ದಾಗಿ
ನೆಲೆ
ನೆಲೆ-ಗ-ಟ್ಟನ್ನು
ನೆಲೆ-ಗ-ಟ್ಟನ್ನೇ
ನೆಲೆ-ಗ-ಟ್ಟಿ-ನ-ಮೇಲೆ
ನೆಲೆ-ಗಟ್ಟು
ನೆಲೆ-ಗೊಂಡಿ
ನೆಲೆ-ಗೊಂಡು
ನೆಲೆ-ಗೊಂ-ಡು-ಬಿ-ಡು-ತ್ತಿತ್ತು
ನೆಲೆ-ಗೊ-ಳಿ-ಸ-ಬೇ-ಕ-ಲ್ಲವೆ
ನೆಲೆ-ಗೊ-ಳಿ-ಸ-ಬೇ-ಕಾ-ಗಿದೆ
ನೆಲೆ-ಗೊ-ಳಿ-ಸ-ಬೇಕು
ನೆಲೆ-ಗೊ-ಳಿ-ಸು-ವು-ದ-ಕ್ಕಾ-ಗಿಯೇ
ನೆಲೆ-ಗೊ-ಳಿ-ಸು-ವು-ದಾ-ಗಿತ್ತು
ನೆಲೆ-ಗೊ-ಳ್ಳು-ವು-ದ-ಕ್ಕಾಗಿ
ನೆಲೆ-ಗೊ-ಳ್ಳು-ವೆವು
ನೆಲೆ-ನಿಂ-ತಿದೆ
ನೆಲೆ-ನಿ-ಲ್ಲಲೂ
ನೆಲೆ-ವೀ-ಡಾದ
ನೆಲೆ-ವೀಡು
ನೆಲ್ಲಿ-ಕಾಯಿ
ನೆವ
ನೇ
ನೇತೃತ್ವ
ನೇತೃ-ತ್ವ-ಮಾ-ರ್ಗ-ದ-ರ್ಶ-ನ-ಗಳಲ್ಲಿ
ನೇತೃ-ತ್ವ-ದಲ್ಲಿ
ನೇತ್ರ-ಗಳು
ನೇತ್ರ-ರಾಗಿ
ನೇಪಲ್ಸಿ
ನೇಪ-ಲ್ಸಿ-ನಿಂದ
ನೇಪ-ಲ್ಸ್
ನೇಮಕ
ನೇಮ-ಕ-ಗೊಂ-ಡರು
ನೇಮ-ಕ-ಗೊಂ-ಡಿ-ದ್ದ-ವನು
ನೇಮ-ಕ-ವಾ-ಗ-ಬೇ-ಕಾ-ಗಿದೆ
ನೇಮಿ-ಸ-ಲಾ-ಯಿತು
ನೇಮಿ-ಸಿ-ಕೊ-ಳ್ಳ-ಲಾ-ಗಿತ್ತು
ನೇಮಿ-ಸಿ-ದರು
ನೇಮಿ-ಸಿ-ದ್ದರು
ನೇರ
ನೇರ-ವಾಗಿ
ನೇರ-ವಾ-ಗಿದ್ದು
ನೇರ-ವೇ-ರಿ-ಸ-ಲಾ-ಯಿತು
ನೇಶನ್
ನೇಹಿ-ಗ-ರ-ನು-ಳಿ-ದ-ವರು
ನೈಜ-ವಾಗಿ
ನೈತಿಕ
ನೈತಿ-ಕ-ತೆ-ಗಳ
ನೈತಿ-ಕ-ವಾಗಿ
ನೈದು
ನೈನಿ
ನೈನಿ-ತಾ-ಲಿಗೆ
ನೈನಿ-ತಾ-ಲಿ-ನಲ್ಲಿ
ನೈನಿ-ತಾ-ಲಿ-ನ-ಲ್ಲಿ-ದ್ದಾಗ
ನೈನಿ-ತಾ-ಲಿ-ನಲ್ಲೇ
ನೈನಿ-ತಾ-ಲಿ-ನಿಂದ
ನೈನಿ-ತಾಲ್
ನೈವೇದ್ಯ
ನೈವೇ-ದ್ಯ-ಮಾ-ಡಿ-ದರು
ನೈಷ್ಠಿಕ
ನೈಸ-ರ್ಗಿಕ
ನೊಂದ-ವ-ರಿಗೆ
ನೊಂದಿಗೆ
ನೊಗ-ದಿಂದ
ನೊಡನೆ
ನೊಣೆ-ಯು-ವಂತೆ
ನೊಬ್ಬ
ನೊಳಗೆ
ನೋಟ
ನೋಟಕ್ಕೆ
ನೋಟ-ಗಳನ್ನು
ನೋಟ-ದ-ಲ್ಲಿಯೇ
ನೋಟ-ದಿಂದ
ನೋಟ-ದಿಂ-ದಲೆ
ನೋಟನ್ನು
ನೋಟ-ಮಾ-ತ್ರ-ದಿಂ-ದಲೇ
ನೋಟ-ವಿದೆ
ನೋಟವು
ನೋಟ-ವೇ-ರ್ಪ-ಟ್ಟಿತ್ತು
ನೋಡ
ನೋಡ-ತೊ-ಡ-ಗಿ-ದರು
ನೋಡದೆ
ನೋಡ-ನೋ-ಡುತ್ತಿ
ನೋಡ-ನೋ-ಡು-ತ್ತಿ-ದ್ದಂ-ತೆಯೇ
ನೋಡ-ನೋ-ಡು-ತ್ತಿ-ದ್ದಂತೇ
ನೋಡ-ಬ-ಯ-ಸ-ಬ-ಹುದು
ನೋಡ-ಬ-ಯ-ಸು-ತ್ತೇನೆ
ನೋಡ-ಬ-ರು-ತ್ತಿದ್ದ
ನೋಡ-ಬ-ಲ್ಲ-ವ-ರಿಲ್ಲ
ನೋಡ-ಬ-ಹುದು
ನೋಡ-ಬೇ-ಕಂತೆ
ನೋಡ-ಬೇ-ಕ-ಲ್ಲವೆ
ನೋಡ-ಬೇ-ಕಾ-ದರೆ
ನೋಡ-ಬೇಕು
ನೋಡ-ಬೇ-ಕೆಂದು
ನೋಡ-ಬೇ-ಕೆಂ-ಬುದು
ನೋಡ-ಬೇಡಿ
ನೋಡಲಿ
ನೋಡ-ಲಿ-ದ್ದೀಯೆ
ನೋಡ-ಲಿ-ದ್ದೇವೆ
ನೋಡಲು
ನೋಡಲೂ
ನೋಡ-ಲೇ-ಬೇಕು
ನೋಡಾ
ನೋಡಿ
ನೋಡಿಕೊ
ನೋಡಿ-ಕೊಂಡ
ನೋಡಿ-ಕೊಂ-ಡರು
ನೋಡಿ-ಕೊಂಡು
ನೋಡಿ-ಕೊಳ್ಳ
ನೋಡಿ-ಕೊ-ಳ್ಳದೆ
ನೋಡಿ-ಕೊ-ಳ್ಳ-ಬಲ್ಲ
ನೋಡಿ-ಕೊ-ಳ್ಳ-ಬ-ಹುದು
ನೋಡಿ-ಕೊ-ಳ್ಳ-ಬೇಕು
ನೋಡಿ-ಕೊ-ಳ್ಳ-ಬೇ-ಕು-ಇದು
ನೋಡಿ-ಕೊ-ಳ್ಳ-ಬೇ-ಕು-ಮನೆ
ನೋಡಿ-ಕೊ-ಳ್ಳಲು
ನೋಡಿ-ಕೊಳ್ಳಿ
ನೋಡಿ-ಕೊಳ್ಳು
ನೋಡಿ-ಕೊ-ಳ್ಳು-ತ್ತಾಳೆ
ನೋಡಿ-ಕೊ-ಳ್ಳು-ತ್ತಿ-ದ್ದರು
ನೋಡಿ-ಕೊ-ಳ್ಳು-ತ್ತಿ-ರುವ
ನೋಡಿ-ಕೊ-ಳ್ಳುವ
ನೋಡಿ-ಕೊ-ಳ್ಳು-ವಿ-ಯಾ-ದರೆ
ನೋಡಿ-ಕೊ-ಳ್ಳು-ವು-ದ-ಕ್ಕಾಗಿ
ನೋಡಿ-ಕೊ-ಳ್ಳು-ವುದು
ನೋಡಿದ
ನೋಡಿ-ದಂತೆ
ನೋಡಿ-ದಂ-ತೆಲ್ಲ
ನೋಡಿ-ದರು
ನೋಡಿ-ದರೂ
ನೋಡಿ-ದರೆ
ನೋಡಿ-ದಳು
ನೋಡಿ-ದ-ವನು
ನೋಡಿ-ದ-ವ-ರಿಗೆ
ನೋಡಿ-ದ-ವ-ರಿ-ಗೆಲ್ಲ
ನೋಡಿ-ದ-ವರು
ನೋಡಿ-ದಾಗ
ನೋಡಿ-ದಾ-ಗಲೂ
ನೋಡಿ-ದಿರಾ
ನೋಡಿ-ದೆಯಾ
ನೋಡಿ-ದೆವು
ನೋಡಿ-ದೆ-ವೆಂಬ
ನೋಡಿ-ದ್ದರೆ
ನೋಡಿ-ದ್ದೀಯಾ
ನೋಡಿ-ದ್ದೀರಾ
ನೋಡಿ-ದ್ದೆವು
ನೋಡಿ-ದ್ದೇವೆ
ನೋಡಿ-ನಿ-ದ್ರಾ-ಪ-ರ-ವ-ಶ-ವಾ-ಗಿದ್ದ
ನೋಡಿ-ಬಿ-ಡ-ಬೇಕು
ನೋಡಿ-ರ-ದಿ-ದ್ದರೆ
ನೋಡಿ-ರ-ಲಿ-ಲ್ಲ-ವಾ-ದ್ದ-ರಿಂದ
ನೋಡಿ-ರ-ಲಿ-ಲ್ಲವೆ
ನೋಡಿ-ರು-ವಂತೆ
ನೋಡಿ-ಶ್ರೀ-ರಾ-ಮ-ಕೃ-ಷ್ಣರ
ನೋಡು
ನೋಡು-ಜೀ-ವ-ನದ
ನೋಡುತ್ತ
ನೋಡು-ತ್ತಲೇ
ನೋಡು-ತ್ತಾನೆ
ನೋಡು-ತ್ತಾರೆ
ನೋಡು-ತ್ತಾ-ರೆ-ಅ-ವ-ರನ್ನು
ನೋಡು-ತ್ತಾ-ರೆ-ಕೂ-ಲಿ-ಗ-ಳೆಲ್ಲ
ನೋಡು-ತ್ತಾ-ರೆ-ಧೂಪ
ನೋಡು-ತ್ತಾ-ರೆ-ಮ-ದ್ರಾ-ಸಿನ
ನೋಡು-ತ್ತಾ-ರೆಯೋ
ನೋಡು-ತ್ತಾ-ರೆ-ರಾ-ಜನ
ನೋಡು-ತ್ತಾ-ರೆ-ಸ್ವಾ-ಮೀ-ಜಿ-ಯ-ವರು
ನೋಡು-ತ್ತಾ-ಳೆ-ಅ-ವರು
ನೋಡು-ತ್ತಿದ್ದ
ನೋಡು-ತ್ತಿ-ದ್ದಂತೆ
ನೋಡು-ತ್ತಿ-ದ್ದರು
ನೋಡು-ತ್ತಿ-ದ್ದಾರೆ
ನೋಡು-ತ್ತಿ-ದ್ದು-ದರ
ನೋಡು-ತ್ತಿ-ದ್ದೆ-ನಾ-ದ್ದ-ರಿಂದ
ನೋಡು-ತ್ತಿ-ದ್ದೇನೆ
ನೋಡು-ತ್ತಿ-ರ-ಲಿಲ್ಲ
ನೋಡು-ತ್ತಿರು
ನೋಡು-ತ್ತಿ-ರು-ವಂತೆ
ನೋಡು-ತ್ತಿ-ಲ್ಲವೆ
ನೋಡು-ತ್ತೇನೆ
ನೋಡು-ತ್ತೇವೆ
ನೋಡು-ತ್ತೇ-ವೆ-ಭಾರೀ
ನೋಡು-ತ್ತೇ-ವೆಯೋ
ನೋಡುವ
ನೋಡು-ವಂ-ತಿತ್ತು
ನೋಡು-ವಂತೆ
ನೋಡು-ವಂ-ಥದೂ
ನೋಡು-ವ-ವ-ರಿಗೂ
ನೋಡು-ವ-ವರೇ
ನೋಡು-ವ-ಷ್ಟ-ರಿಂ-ದಲೇ
ನೋಡು-ವಾಗ
ನೋಡು-ವಿರಿ
ನೋಡು-ವು-ದ-ಕ್ಕಾಗಿ
ನೋಡು-ವು-ದ-ಕ್ಕಾ-ಗಿಯೇ
ನೋಡು-ವು-ದ-ರಿಂದ
ನೋಡು-ವು-ದಾ-ದರೆ
ನೋಡು-ವುದು
ನೋಡು-ವುದೇ
ನೋಡುವೆ
ನೋಡೋಣ
ನೋದ್ದೇ-ಶ-ದಿಂದ
ನೋಬೆಲ್
ನೋಬೆ-ಲ್ಲ-ರಂ-ತಹ
ನೋಬೆ-ಲ್ಲಳ
ನೋಬೆ-ಲ್ಲ-ಳಿಗೂ
ನೋಬೆ-ಲ್ಲ-ಳಿಗೆ
ನೋಬೆ-ಲ್ಲಳು
ನೋಬೆ-ಲ್ಲಳೂ
ನೋಯಿ-ಸದೇ
ನೋಯಿ-ಸ-ಬೇ-ಕೆಂ-ಬುದೂ
ನೋಯಿ-ಸು-ವುದು
ನೋವನ್ನು
ನೋವ-ನ್ನುಂ-ಟು-ಮಾ-ಡಿತು
ನೋವನ್ನೇ
ನೋವಾ-ಗ-ಬ-ಹುದು
ನೋವಾ-ಗು-ವು-ದಿ-ಲ್ಲವೆ
ನೋವಾ-ದಾಗ
ನೋವಾ-ಯಿತು
ನೋವಿನ
ನೋವಿ-ನಿಂದ
ನೋವು
ನೋವೂ
ನೌಕ-ರರು
ನೌಕಾ-ಧಿ-ಕಾ-ರಿ-ಯನ್ನು
ನ್ನಂತೂ
ನ್ನರ್ಪಿಸಿ
ನ್ನರ್ಪಿ-ಸುವ
ನ್ನಲ್ಲ
ನ್ನಾಗಿ
ನ್ನಾಗಿ-ಸುವ
ನ್ನಾಡಿದ
ನ್ನಾದರೂ
ನ್ನಾಧ-ರಿಸಿ
ನ್ನಾಲಿ-ಸುತ್ತ
ನ್ನಿಟ್ಟು-ಕೊಂ-ಡಿ-ದ್ದಾರೆ
ನ್ನಿಟ್ಟು-ಕೊಂ-ಡಿಲ್ಲ
ನ್ನುಂಟು-ಮಾ-ಡು-ವು-ದರ
ನ್ನುದ್ದೇ-ಶಿಸಿ
ನ್ನೆಲ್ಲ
ನ್ನೇರಿ
ನ್ನೇರಿ-ದರು
ನ್ನೊಮ್ಮೆ
ನ್ನೊಳ-ಗೊಂಡ
ನ್ಮಾತೆಯ
ನ್ಯಾಯ
ನ್ಯಾಯದ
ನ್ಯಾಯ-ಬದ್ಧ
ನ್ಯಾಯ-ಬ-ದ್ಧ-ವಾಗಿ
ನ್ಯಾಯ-ಬ-ದ್ಧ-ವಾದ
ನ್ಯಾಯ-ವಂ-ತ-ರಾಗಿ
ನ್ಯಾಯ-ವನ್ನು
ನ್ಯಾಯ-ಶಾಸ್ತ್ರ
ನ್ಯಾಯ-ಶಾ-ಸ್ತ್ರ-ತ-ರ್ಕ-ಶಾ-ಸ್ತ್ರ-ಗಳಲ್ಲಿ
ನ್ಯಾಯ-ಸಿಂ-ಹಾ-ಸ-ನದ
ನ್ಯಾಯಾಂಗ
ನ್ಯಾಯಾತ್
ನ್ಯಾಯಾ-ಧೀ-ಶ-ರಾದ
ನ್ಯಾಯಾ-ಧೀ-ಶರು
ನ್ಯಾಯಾ-ಲ-ಯ-ದಲ್ಲಿ
ನ್ಯಾಯಾ-ಲ-ಯವು
ನ್ಯಾಸದ
ನ್ಯುಮಿ-ಡಿ-ಯನ್
ನ್ಯೂ
ನ್ಯೂಕ್ಯಾ-ಸಲ್ನ
ನ್ಯೂಜೆ-ರ್ಸಿ-ಯ-ಲ್ಲಿ-ರುವ
ನ್ಯೂಯಾರ್
ನ್ಯೂಯಾ-ರ್ಕನ್ನು
ನ್ಯೂಯಾರ್ಕಿ
ನ್ಯೂಯಾ-ರ್ಕಿಗೆ
ನ್ಯೂಯಾ-ರ್ಕಿನ
ನ್ಯೂಯಾ-ರ್ಕಿ-ನಲ್ಲಿ
ನ್ಯೂಯಾ-ರ್ಕಿ-ನಿಂದ
ನ್ಯೂಯಾ-ರ್ಕ್
ನ್ವಯ
ಪಂಕ್ತಿ-ಗಳನ್ನು
ಪಂಕ್ತಿ-ಪಂಕ್ತಿ
ಪಂಗ-ಡಕ್ಕೆ
ಪಂಗ-ಡ-ಗಳನ್ನೂ
ಪಂಗ-ಡ-ಗಳಲ್ಲಿ
ಪಂಗ-ಡ-ಗ-ಳ-ವ-ರನ್ನೂ
ಪಂಗ-ಡ-ಗಳೂ
ಪಂಗ-ಡ-ದ-ವರೂ
ಪಂಗ-ಡ-ವಿದೆ
ಪಂಚ-ತ-ರ-ಣಿ-ಯನ್ನು
ಪಂಚ-ತಾರಾ
ಪಂಚ-ನ-ದಿ-ಗಳ
ಪಂಚ-ಭೂ-ತ-ಗಳಲ್ಲಿ
ಪಂಚ-ವಟಿ
ಪಂಚಾಂಗ
ಪಂಚಾಂ-ಗ-ವನ್ನು
ಪಂಚೆ
ಪಂಜ-ರದ
ಪಂಜ-ರವು
ಪಂಜಾ-ಬಿ-ಗ-ರೆ-ಲ್ಲರೂ
ಪಂಜಾ-ಬಿ-ಗಳ
ಪಂಜಾ-ಬಿಗೆ
ಪಂಜಾ-ಬಿನ
ಪಂಜಾ-ಬಿ-ನಲ್ಲಿ
ಪಂಜಾ-ಬಿ-ನ-ಲ್ಲಿ-ದ್ದಾಗ
ಪಂಜಾ-ಬಿ-ನ-ವ-ರೆಂ-ಬು-ದನ್ನು
ಪಂಜಾಬೀ
ಪಂಜಾಬು
ಪಂಜಾಬ್
ಪಂಜಿನ
ಪಂಜು
ಪಂಜು-ಗಳನ್ನು
ಪಂಡಿತ
ಪಂಡಿ-ತನ
ಪಂಡಿ-ತ-ನಂ-ತಾ-ಗಲಿ
ಪಂಡಿ-ತ-ನಿಗೆ
ಪಂಡಿ-ತನು
ಪಂಡಿ-ತರ
ಪಂಡಿ-ತ-ರಂತೆ
ಪಂಡಿ-ತ-ರಾ-ದರೂ
ಪಂಡಿ-ತ-ರಿಂ-ದ-ತ-ತ್ವ-ಜ್ಞಾ-ನಿ-ಗಳಿಂದ
ಪಂಡಿ-ತ-ರಿಗೂ
ಪಂಡಿ-ತ-ರಿಗೆ
ಪಂಡಿ-ತ-ರಿಗೇ
ಪಂಡಿ-ತರು
ಪಂಡಿ-ತ-ರು-ಬುದ್ಧಿ
ಪಂಡಿ-ತ-ರು-ಸಂ-ನ್ಯಾ-ಸಿ-ಗಳು
ಪಂಡಿ-ತರೂ
ಪಂಡಿ-ತ-ಸ-ಭೆ-ಯ-ನ್ನು-ದ್ದೇ-ಶಿಸಿ
ಪಂಡಿ-ತಾ-ನಾಂ
ಪಂಡಿತ್ಜೀ
ಪಂತುಲು
ಪಂಥ
ಪಂಥಕ್ಕೆ
ಪಂಥ-ಗಳ
ಪಂಥ-ಗ-ಳ-ನ್ನ-ಲ್ಲದೆ
ಪಂಥ-ಗಳನ್ನು
ಪಂಥ-ಗಳನ್ನೂ
ಪಂಥ-ಗಳು
ಪಂಥ-ಗಳೂ
ಪಂಥ-ವ-ನ್ನಾ-ಗಿಯೇ
ಪಂಥವು
ಪಂಥ-ವೊ-ಡ್ಡಿದ
ಪಂದ್ರೇ-ಥಾನ
ಪಕ್ಕಕ್ಕೆ
ಪಕ್ಕದ
ಪಕ್ಕ-ದಲ್ಲಿ
ಪಕ್ಕ-ದ-ಲ್ಲಿದ್ದ
ಪಕ್ಕ-ದ-ಲ್ಲಿಯೇ
ಪಕ್ಕ-ದಲ್ಲೇ
ಪಕ್ಕಾ
ಪಕ್ವ-ವಾ-ಗಿ-ರ-ಲಿಲ್ಲ
ಪಕ್ಷ
ಪಕ್ಷ-ಗ-ಳ-ವ-ರಂತೆ
ಪಕ್ಷ-ದಲ್ಲಿ
ಪಕ್ಷಿ
ಪಕ್ಷಿ-ಗಳ
ಪಕ್ಷಿ-ಗಳನ್ನು
ಪಕ್ಷಿ-ಗಳಲ್ಲಿ
ಪಕ್ಷಿ-ಗ-ಳೆಲ್ಲ
ಪಕ್ಷಿ-ನೋಟ
ಪಚಾರ
ಪಚ್ಚ-ಯ್ಯಪ್ಪ
ಪಚ್ಚೆ
ಪಚ್ಚೆ-ಗಳು
ಪಚ್ಚೆ-ಪ-ಯಿರು
ಪಟ-ವನ್ನು
ಪಟ್ಟ
ಪಟ್ಟ-ಣ-ಗಳಲ್ಲಿ
ಪಟ್ಟ-ಣ-ಗಳಿಂದ
ಪಟ್ಟ-ಣ-ಗಳು
ಪಟ್ಟ-ಣದ
ಪಟ್ಟ-ಣ-ದಲ್ಲಿ
ಪಟ್ಟರೂ
ಪಟ್ಟ-ವನೂ
ಪಟ್ಟಿ
ಪಟ್ಟಿ-ದ್ದೇನೆ
ಪಟ್ಟಿ-ಯನ್ನು
ಪಟ್ಟಿ-ಯನ್ನೂ
ಪಟ್ಟಿ-ಯನ್ನೇ
ಪಟ್ಟಿ-ಯ-ಲ್ಲಿ-ರು-ತ್ತದೆ
ಪಟ್ಟು
ಪಟ್ಟು-ಕೊಂಡು
ಪಟ್ಟು-ಕೊ-ಳ್ಳ-ಬೇಡ
ಪಟ್ಟು-ಕೊ-ಳ್ಳು-ತ್ತಿ-ದ್ದರೂ
ಪಟ್ಟು-ಕೊ-ಳ್ಳು-ವುದೆ
ಪಠಣ
ಪಠಿ-ಸಿದ
ಪಠಿ-ಸಿ-ದರು
ಪಠಿ-ಸು-ತ್ತಿ-ದ್ದರು
ಪಠಿ-ಸು-ವು-ದ-ರೊಂ-ದಿಗೆ
ಪಡ-ಬೇ-ಕಾಗಿ
ಪಡ-ಬೇ-ಕಾ-ಗಿಲ್ಲ
ಪಡ-ಬೇ-ಕಾ-ಯಿತು
ಪಡ-ಬೇಡಿ
ಪಡಲಿ
ಪಡಿಸಿ
ಪಡಿ-ಸಿ-ಕೊ-ಳ್ಳಲು
ಪಡಿ-ಸಿ-ದರು
ಪಡಿ-ಸುತ್ತ
ಪಡಿ-ಸು-ತ್ತಿ-ದ್ದರು
ಪಡಿ-ಸು-ವು-ದ-ಕ್ಕಾಗಿ
ಪಡುತ್ತ
ಪಡು-ತ್ತಿದ್ದ
ಪಡು-ತ್ತಿ-ದ್ದರು
ಪಡು-ತ್ತಿ-ದ್ದಳು
ಪಡು-ತ್ತಿ-ದ್ದಾ-ರೆಂದು
ಪಡು-ತ್ತಿ-ದ್ದೇನೆ
ಪಡು-ವಂತೆ
ಪಡು-ವಂ-ಥ-ದೇನು
ಪಡು-ವು-ದ-ಲ್ಲದೆ
ಪಡು-ವುದು
ಪಡೆ
ಪಡೆದ
ಪಡೆ-ದರು
ಪಡೆ-ದ-ವ-ರಾಗಿ
ಪಡೆ-ದ-ವ-ರಿಂ-ದಲೇ
ಪಡೆ-ದ-ವರು
ಪಡೆ-ದಿ-ದ್ದರು
ಪಡೆ-ದಿ-ದ್ದ-ವ-ಳಿಗೆ
ಪಡೆ-ದಿ-ದ್ದಾ-ರೆಯೇ
ಪಡೆ-ದಿ-ದ್ದೇನೆ
ಪಡೆ-ದಿ-ದ್ದೇವೆ
ಪಡೆ-ದಿ-ರ-ಬೇ-ಕಾ-ಗಿತ್ತು
ಪಡೆ-ದಿ-ರು-ವುದನ್ನು
ಪಡೆದು
ಪಡೆ-ದುಕೊ
ಪಡೆ-ದು-ಕೊಂಡ
ಪಡೆ-ದು-ಕೊಂ-ಡರು
ಪಡೆ-ದು-ಕೊಂ-ಡರೂ
ಪಡೆ-ದು-ಕೊಂ-ಡರೆ
ಪಡೆ-ದು-ಕೊಂ-ಡ-ವರು
ಪಡೆ-ದು-ಕೊಂ-ಡಿ-ದೆ-ಯೆಂ-ಬು-ದನ್ನು
ಪಡೆ-ದು-ಕೊಂ-ಡಿ-ದ್ದಾ-ರೆಯೋ
ಪಡೆ-ದು-ಕೊಂ-ಡಿದ್ದು
ಪಡೆ-ದು-ಕೊಂ-ಡಿ-ದ್ದೇನೆ
ಪಡೆ-ದು-ಕೊಂ-ಡಿ-ರ-ಬೇ-ಕೆಂದು
ಪಡೆ-ದು-ಕೊಂಡು
ಪಡೆ-ದು-ಕೊ-ಳ್ಳ-ಬೇ-ಕೆಂಬ
ಪಡೆ-ದು-ಕೊ-ಳ್ಳಲು
ಪಡೆ-ದು-ಕೊಳ್ಳಿ
ಪಡೆ-ದು-ಕೊಳ್ಳು
ಪಡೆ-ದು-ಕೊ-ಳ್ಳು-ತ್ತದೆ
ಪಡೆ-ದು-ಕೊ-ಳ್ಳುವ
ಪಡೆ-ದು-ಕೊ-ಳ್ಳು-ವು-ದ-ಕ್ಕೋ-ಸ್ಕರ
ಪಡೆ-ದು-ಕೊ-ಳ್ಳು-ವುದು
ಪಡೆ-ದು-ಕೊ-ಳ್ಳು-ವುದೂ
ಪಡೆದೂ
ಪಡೆಯ
ಪಡೆ-ಯ-ದಿ-ರು-ವ-ವರು
ಪಡೆ-ಯದು
ಪಡೆ-ಯದೆ
ಪಡೆ-ಯ-ಬ-ಹುದು
ಪಡೆ-ಯ-ಬ-ಹು-ದೆಂದು
ಪಡೆ-ಯ-ಬೇ-ಕಾ-ಗಿದೆ
ಪಡೆ-ಯ-ಬೇ-ಕಾದ
ಪಡೆ-ಯ-ಬೇ-ಕಾ-ದರೆ
ಪಡೆ-ಯ-ಬೇಕು
ಪಡೆ-ಯ-ಬೇ-ಕೆಂಬ
ಪಡೆ-ಯ-ಲಾ-ಗದ
ಪಡೆ-ಯ-ಲಿ-ದ್ದಂ-ತಹ
ಪಡೆ-ಯ-ಲಿ-ರುವ
ಪಡೆ-ಯಲು
ಪಡೆ-ಯಲೂ
ಪಡೆ-ಯ-ಲೆಂದು
ಪಡೆಯು
ಪಡೆ-ಯುತ್ತ
ಪಡೆ-ಯು-ತ್ತಾನೆ
ಪಡೆ-ಯು-ತ್ತಿ-ದ್ದರು
ಪಡೆ-ಯು-ತ್ತಿ-ದ್ದಾರೆ
ಪಡೆ-ಯು-ತ್ತಿ-ದ್ದುದು
ಪಡೆ-ಯು-ತ್ತಿ-ರುವು
ಪಡೆ-ಯು-ತ್ತೀರಿ
ಪಡೆ-ಯು-ತ್ತೇನೆ
ಪಡೆ-ಯುವ
ಪಡೆ-ಯು-ವಂ-ತಹ
ಪಡೆ-ಯು-ವಂ-ತಾ-ಗ-ಬೇಕು
ಪಡೆ-ಯು-ವಂತೆ
ಪಡೆ-ಯು-ವ-ವನು
ಪಡೆ-ಯು-ವಿರಿ
ಪಡೆ-ಯು-ವು-ದ-ಕ್ಕಿಂ-ತಲೂ
ಪಡೆ-ಯು-ವುದು
ಪಡೆ-ಯು-ವುದೇ
ಪಡೆ-ವ-ವ-ರೆಗೂ
ಪಣ-ತೊ-ಟ್ಟರು
ಪಣ-ತೊಟ್ಟು
ಪಣ-ವಾ-ಗಿ-ಡು-ವುದೇ
ಪಣ-ವಾ-ಗೊಡ್ಡಿ
ಪತಂ-ಜ-ಲಿ-ಗಳ
ಪತ-ನ-ದಿಂ-ದಲೇ
ಪತಾ-ಕೆ-ಯೇ-ರಿಸು
ಪತಿ
ಪತಿ-ತ-ರನ್ನು
ಪತಿ-ತ-ರಿ-ಗಾಗಿ
ಪತಿ-ಯಂ-ತೆಯೇ
ಪತ್ತೇ-ದಾರಿ-ಗಳನ್ನು
ಪತ್ನಿ
ಪತ್ನಿ-ಯನ್ನು
ಪತ್ರ
ಪತ್ರ-ಕ-ರ್ತ-ರಾದ
ಪತ್ರಕ್ಕೆ
ಪತ್ರ-ಗಳ
ಪತ್ರ-ಗಳನ್ನು
ಪತ್ರ-ಗಳನ್ನೂ
ಪತ್ರ-ಗ-ಳನ್ನೋ
ಪತ್ರ-ಗಳಲ್ಲಿ
ಪತ್ರ-ಗಳಿಂದ
ಪತ್ರ-ಗ-ಳಿಗೆ
ಪತ್ರ-ಗಳು
ಪತ್ರದ
ಪತ್ರ-ದಲ್ಲಿ
ಪತ್ರ-ದಲ್ಲೇ
ಪತ್ರ-ದಿಂದ
ಪತ್ರ-ವನ್ನು
ಪತ್ರ-ವಾ-ಗಿ-ಬಿ-ಡು-ವುದೋ
ಪತ್ರವು
ಪತ್ರವೇ
ಪತ್ರ-ವೊಂ-ದನ್ನು
ಪತ್ರ-ವೊಂ-ದರ
ಪತ್ರ-ವೊಂ-ದ-ರಲ್ಲಿ
ಪತ್ರ-ವೊಂದು
ಪತ್ರ-ವ್ಯ-ವ-ಹಾರ
ಪತ್ರ-ವ್ಯ-ವ-ಹಾ-ರ-ದಲ್ಲೋ
ಪತ್ರ-ವ್ಯ-ವ-ಹಾ-ರ-ವನ್ನು
ಪತ್ರಿಕಾ
ಪತ್ರಿಕೆ
ಪತ್ರಿ-ಕೆ-ಗಳ
ಪತ್ರಿ-ಕೆ-ಗಳನ್ನು
ಪತ್ರಿ-ಕೆ-ಗಳಲ್ಲಿ
ಪತ್ರಿ-ಕೆ-ಗ-ಳಲ್ಲೂ
ಪತ್ರಿ-ಕೆ-ಗ-ಳ-ಲ್ಲೊಂದು
ಪತ್ರಿ-ಕೆ-ಗಳು
ಪತ್ರಿ-ಕೆ-ಗ-ಳೊಂ-ದಿಗೆ
ಪತ್ರಿ-ಕೆ-ಗಾಗಿ
ಪತ್ರಿ-ಕೆಗೆ
ಪತ್ರಿ-ಕೆಯ
ಪತ್ರಿ-ಕೆ-ಯನ್ನು
ಪತ್ರಿ-ಕೆ-ಯಲ್ಲಿ
ಪತ್ರಿ-ಕೆ-ಯಾ-ಗಿತ್ತು
ಪತ್ರಿ-ಕೆ-ಯಾದ
ಪತ್ರಿ-ಕೆಯು
ಪತ್ರಿ-ಕೆ-ಯೊಂ-ದನ್ನು
ಪತ್ರಿ-ಕೆ-ಯೊಂ-ದ-ರಲ್ಲಿ
ಪತ್ರಿಕೋ
ಪತ್ರಿ-ಕೋ-ದ್ಯಮಿ
ಪಥ
ಪಥಂ
ಪಥ-ಗಳ
ಪಥದ
ಪಥ-ದಲ್ಲಿ
ಪಥದಿ
ಪಥ-ದಿಂದ
ಪಥ-ಭ್ರ-ಷ್ಟ-ನಾ-ಗು-ವುದೆ
ಪಥ-ವನ್ನು
ಪಥ-ವನ್ನೇ
ಪಥ್ಯ
ಪಥ್ಯ-ಪಾನ
ಪಥ್ಯ-ಪಾ-ನದ
ಪಥ್ಯ-ಪಾ-ನ-ದ-ಲ್ಲಿ-ರ-ಬೇ-ಕಾ-ಯಿತು
ಪಥ್ಯ-ವನ್ನು
ಪಥ್ಯ-ವಾ-ಗ-ದಿ-ರ-ಬ-ಹುದು
ಪದಂ
ಪದ-ಗಳನ್ನು
ಪದ-ತ-ಲ-ದಲ್ಲಿ
ಪದ-ತ-ಲದಿ
ಪದದ
ಪದ-ದಲ್ಲಿ
ಪದ-ದಲ್ಲೂ
ಪದ-ರ-ಗಳನ್ನು
ಪದ-ಲಾ-ಲಿತ್ಯ
ಪದ-ವನ್ನು
ಪದ-ವನ್ನೂ
ಪದ-ವಿಲ್ಲ
ಪದವೀ
ಪದ-ವೀ-ಧ-ರರು
ಪದವು
ಪದವೂ
ಪದಾ-ರ್ಥ-ಗಳನ್ನು
ಪದಾ-ರ್ಥ-ವನ್ನು
ಪದಾ-ರ್ಪಣ
ಪದೇ-ಪದೇ
ಪದೇಶ
ಪದ್ಧತಿ
ಪದ್ಧ-ತಿ-ಗಳನ್ನು
ಪದ್ಧ-ತಿ-ಗ-ಳಾ-ಗ-ಬ-ಹುದು
ಪದ್ಧ-ತಿ-ಗ-ಳು-ಇ-ವು-ಗಳನ್ನೆಲ್ಲ
ಪದ್ಧ-ತಿ-ಗಳೊ
ಪದ್ಧ-ತಿಯ
ಪದ್ಧ-ತಿ-ಯನ್ನು
ಪದ್ಧ-ತಿ-ಯಲ್ಲಿ
ಪದ್ಧ-ತಿ-ಯಿಂದ
ಪದ್ಧ-ತಿಯು
ಪದ್ಧ-ತಿ-ಯೊಂ-ದನ್ನು
ಪದ್ಮಾ-ಸ-ನ-ದಲ್ಲಿ
ಪದ್ಯದ
ಪನ್ನರೂ
ಪನ್ನೀರು
ಪಯ-ಣ-ಇದೇ
ಪಯ-ಣಿ-ಸಲು
ಪಯ-ಣಿಸಿ
ಪಯ-ಣಿ-ಸಿದ
ಪಯ-ಣಿ-ಸಿ-ದರು
ಪಯ-ಣಿ-ಸುವ
ಪರ
ಪರಂ-ಪ-ರಾ-ಗತ
ಪರಂ-ಪ-ರಾ-ಗ-ತ-ವಾಗಿ
ಪರಂ-ಪ-ರೆಯ
ಪರಂ-ಪ-ರೆ-ಯತ್ತ
ಪರಂ-ಪ-ರೆ-ಯನ್ನು
ಪರಂ-ಪ-ರೆ-ಯನ್ನೂ
ಪರ-ಕೀ-ಯರ
ಪರ-ಕೀ-ಯ-ರ-ನ್ನಂ-ತಿ-ರಲಿ
ಪರ-ಕೀ-ಯ-ರನ್ನು
ಪರ-ಕೀ-ಯ-ರಾದ
ಪರ-ಕೀ-ಯರು
ಪರ-ಗ-ಣದ
ಪರ-ದಾ-ಡ-ಬೇಡಿ
ಪರದೆ
ಪರ-ದೆ-ಯೊಂದು
ಪರ-ಧರ್ಮ
ಪರ-ಧ-ರ್ಮ-ಗ-ಳ
ಪರ-ಧ-ರ್ಮೀ-ಯರ
ಪರ-ಬ್ರಹ್ಮ
ಪರ-ಬ್ರ-ಹ್ಮದ
ಪರ-ಬ್ರ-ಹ್ಮ-ದಲ್ಲಿ
ಪರ-ಬ್ರ-ಹ್ಮನೊ
ಪರ-ಬ್ರ-ಹ್ಮ-ವನ್ನೇ
ಪರ-ಬ್ರ-ಹ್ಮ-ವ-ಸ್ತು-ವಿನ
ಪರ-ಬ್ರ-ಹ್ಮ-ವ-ಸ್ತು-ವಿ-ನೊಂ-ದಿಗೆ
ಪರ-ಬ್ರ-ಹ್ಮ-ವ-ಸ್ತುವು
ಪರ-ಬ್ರ-ಹ್ಮ-ವ-ಸ್ತುವೇ
ಪರ-ಬ್ರ-ಹ್ಮ-ವೆಂಬ
ಪರಮ
ಪರ-ಮ-ಚ-ರಮ
ಪರ-ಮ-ಕುಡಿ
ಪರ-ಮ-ಕು-ಡಿಯ
ಪರ-ಮ-ತ-ತ್ತ್ವ-ವನ್ನು
ಪರ-ಮ-ಪ-ರಿ-ಶುದ್ಧ
ಪರ-ಮ-ಪ-ವಿತ್ರ
ಪರ-ಮ-ಪಾ-ವ-ನ-ರಾದ
ಪರ-ಮ-ಪ್ರಿಯ
ಪರ-ಮ-ಭ-ಕ್ತಿಗೆ
ಪರ-ಮ-ಭಾ-ಗ್ಯ-ವೆಂದು
ಪರ-ಮ-ಶಾ-ಸ್ತ್ರ-ಕ್ಕಿಂತ
ಪರ-ಮ-ಶಿ-ಷ್ಯ-ನಾದ
ಪರ-ಮ-ಸಂ-ಪ್ರ-ದಾ-ಯಸ್ಥ
ಪರ-ಮ-ಸ-ತ್ಯ-ವನ್ನು
ಪರ-ಮ-ಹಂಸ
ಪರ-ಮ-ಹಂ-ಸ-ಎಂ-ಬಿ-ತ್ಯಾ-ದಿ-ಯಾಗಿ
ಪರ-ಮ-ಹಂ-ಸರ
ಪರ-ಮ-ಹಂ-ಸರು
ಪರ-ಮ-ಹಂ-ಸ-ರು-ಯಾವ
ಪರ-ಮ-ಹಂ-ಸಾ-ವ-ಸ್ಥೆ-ಗೇ-ರಿದ
ಪರ-ಮಾತ್ಮ
ಪರ-ಮಾ-ತ್ಮನ
ಪರ-ಮಾ-ತ್ಮ-ನಂ-ತೆಯೇ
ಪರ-ಮಾ-ತ್ಮ-ನನ್ನು
ಪರ-ಮಾ-ತ್ಮ-ನ-ಲ್ಲಿ-ರುವ
ಪರ-ಮಾ-ತ್ಮ-ನಲ್ಲೇ
ಪರ-ಮಾ-ತ್ಮನೇ
ಪರ-ಮಾ-ತ್ಮ-ನೊಂ-ದಿಗೆ
ಪರ-ಮಾ-ತ್ಮ-ನೊ-ಡನೆ
ಪರ-ಮಾ-ದ್ಭುತ
ಪರ-ಮಾ-ದ್ಭು-ತ-ವ-ಲ್ಲದೆ
ಪರ-ಮಾ-ದ್ಭು-ತ-ವಾ-ದ-ದ್ದನ್ನು
ಪರ-ಮಾ-ನಂದ
ಪರ-ಮಾ-ನಂ-ದ-ಗೊಂಡ
ಪರ-ಮಾರ್ಥ
ಪರ-ಮಾ-ರ್ಥ-ವನ್ನು
ಪರ-ಮಾ-ವ-ಧಿ-ಯನ್ನು
ಪರ-ಮಾ-ಶ್ಚರ್ಯ
ಪರ-ಮಾ-ಶ್ಚ-ರ್ಯದ
ಪರ-ಮೇ-ಶ್ವ-ರನ
ಪರ-ಮೇ-ಶ್ವ-ರ-ನನ್ನು
ಪರರ
ಪರ-ರಾ-ಷ್ಟ್ರ-ಗಳ
ಪರ-ರಾ-ಷ್ಟ್ರ-ಗಳನ್ನು
ಪರ-ರಾ-ಷ್ಟ್ರೀ-ಯರು
ಪರ-ರಿ-ಗಾಗಿ
ಪರ-ರಿಗೆ
ಪರ-ವ-ಶ-ರಾದ
ಪರ-ವಾಗಿ
ಪರ-ವಾ-ಗಿಯೇ
ಪರ-ವಾ-ಗಿಲ್ಲ
ಪರ-ವಾದ
ಪರ-ಶ್ರೇಷ್ಠ
ಪರ-ಸ್ಪರ
ಪರ-ಸ್ಪ-ರರ
ಪರ-ಸ್ಪ-ರ-ರನ್ನು
ಪರ-ಸ್ಪ-ರ-ರಿಗೆ
ಪರ-ಹಿತ
ಪರ-ಹಿ-ತ-ಕ್ಕಾಗಿ
ಪರಾ-ಕಾಷ್ಠೆ
ಪರಾ-ಕಾ-ಷ್ಠೆ-ಯನ್ನು
ಪರಾ-ಕ್ರ-ಮ-ಶಾ-ಲಿ-ಗಳು
ಪರಾ-ಕ್ರ-ಮ-ಶಾ-ಲಿ-ಯಾಗಿ
ಪರಾ-ಕ್ರ-ಮಿ-ಗಳು
ಪರಾ-ಧೀ-ನ-ತೆ-ಯನ್ನು
ಪರಾ-ಧೀ-ನ-ತೆ-ಯನ್ನೂ
ಪರಾ-ನು-ಕ-ರ-ಣ-ಶೀ-ಲ-ತೆಯೂ
ಪರಾ-ಭ-ಕ್ತಿ-ಯನ್ನು
ಪರಾರಿ
ಪರಾ-ರಿ-ಯಾ-ದುವು
ಪರಾ-ವ-ಲಂ-ಬನಂ
ಪರಾ-ವ-ಲಂ-ಬ-ನೆಯೇ
ಪರಾ-ವ-ಲಂ-ಬಿ-ಗಳೂ
ಪರಾ-ವಿ-ದ್ಯೆ-ಗಳನ್ನು
ಪರಿ
ಪರಿ-ಇವು
ಪರಿ-ಗ-ಣಿ-ತ-ರಾ-ಗುವ
ಪರಿ-ಗ-ಣಿಸ
ಪರಿ-ಗ-ಣಿ-ಸದೆ
ಪರಿ-ಗ-ಣಿ-ಸ-ಬೇ-ಕಾ-ಗು-ತ್ತದೆ
ಪರಿ-ಗ-ಣಿ-ಸ-ಲಾಗಿದೆ
ಪರಿ-ಗ-ಣಿ-ಸಲೇ
ಪರಿ-ಗ-ಣಿ-ಸ-ಲ್ಪ-ಟ್ಟಿತ್ತು
ಪರಿ-ಗ-ಣಿ-ಸ-ಲ್ಪ-ಟ್ಟಿದ್ದ
ಪರಿ-ಗ-ಣಿಸಿ
ಪರಿ-ಗ-ಣಿ-ಸಿ-ದರು
ಪರಿ-ಗ-ಣಿ-ಸಿ-ದರೂ
ಪರಿ-ಗ-ಣಿ-ಸಿ-ದ್ದರು
ಪರಿ-ಗ-ಣಿ-ಸಿ-ದ್ದ-ರೆಂ-ಬುದು
ಪರಿ-ಗ-ಣಿ-ಸು-ತ್ತಾರೆ
ಪರಿ-ಗ-ಣಿ-ಸು-ತ್ತಿ-ದ್ದರು
ಪರಿ-ಗ್ರ-ಹವೂ
ಪರಿ-ಗ್ರ-ಹಿ-ಸಿದ
ಪರಿ-ಗ್ರಾ-ಹ್ಯವೂ
ಪರಿ-ಚಯ
ಪರಿ-ಚ-ಯ-ಪತ್ರ
ಪರಿ-ಚ-ಯ-ವಾ-ಗಿತ್ತು
ಪರಿ-ಚ-ಯ-ವಾ-ಯಿತು
ಪರಿ-ಚ-ಯ-ವಿ-ದೆ-ಹೌದು
ಪರಿ-ಚ-ಯ-ವಿದ್ದ
ಪರಿ-ಚ-ಯ-ವಿ-ದ್ದ-ವಳು
ಪರಿ-ಚ-ಯ-ವಿ-ರ-ಲಿಲ್ಲ
ಪರಿ-ಚ-ಯ-ವಿ-ರು-ವುದು
ಪರಿ-ಚ-ಯ-ವಿ-ಲ್ಲ-ದ-ವ-ರಿಗೆ
ಪರಿ-ಚ-ಯ-ವುಂ-ಟಾ-ಯಿತು
ಪರಿ-ಚ-ಯವೂ
ಪರಿ-ಚ-ಯ-ಸ್ಥನ
ಪರಿ-ಚ-ಯ-ಸ್ಥ-ನಾದ
ಪರಿ-ಚ-ಯ-ಸ್ಥ-ರ-ನ್ನೆಲ್ಲ
ಪರಿ-ಚ-ಯ-ಸ್ಥರು
ಪರಿ-ಚ-ಯ-ಸ್ಥ-ಳಾದ
ಪರಿ-ಚ-ಯಿ-ಸಿ-ಕೊಟ್ಟ
ಪರಿ-ಚ-ಯಿ-ಸಿ-ಕೊ-ಟ್ಟರು
ಪರಿ-ಚ-ಯಿ-ಸಿ-ಕೊ-ಡ-ಲಾ-ಯಿತು
ಪರಿ-ಚ-ಯಿ-ಸಿ-ಕೊಡು
ಪರಿ-ಚ-ಯಿ-ಸಿ-ಕೊ-ಡುತ್ತ
ಪರಿ-ಚ-ಯಿ-ಸಿ-ಕೊ-ಡು-ತ್ತಿ-ದ್ದಂ-ತಹ
ಪರಿ-ಚ-ಯಿ-ಸಿ-ಕೊ-ಡು-ತ್ತಿ-ರುವ
ಪರಿ-ಚ-ಯಿ-ಸು-ವಾಗ
ಪರಿ-ಚರ್ಯ
ಪರಿ-ಚರ್ಯೆ
ಪರಿ-ಚ-ರ್ಯೆಗೆ
ಪರಿ-ಚಾ-ರ-ಕ-ರನ್ನು
ಪರಿ-ಚಾ-ರ-ಕರು
ಪರಿ-ಚಿತ
ಪರಿ-ಚಿ-ತ-ನಾ-ದ-ವನು
ಪರಿ-ಚಿ-ತರ
ಪರಿ-ಚಿ-ತ-ರನ್ನೂ
ಪರಿ-ಚಿ-ತ-ರಲ್ಲಿ
ಪರಿ-ಚಿ-ತ-ರಾದ
ಪರಿ-ಚಿ-ತ-ರಿಗೆ
ಪರಿ-ಚಿ-ತರು
ಪರಿ-ಚಿ-ತರೂ
ಪರಿ-ಚಿ-ತ-ರೆ-ಲ್ಲ-ರನ್ನೂ
ಪರಿ-ಚಿ-ತ-ಳಾ-ಗಿದ್ದ
ಪರಿ-ಚಿ-ತ-ಳಾದ
ಪರಿ-ಚಿ-ತ-ವಾ-ಗಿ-ರು-ವಂತೆ
ಪರಿ-ಜ್ಞಾ-ನ-ವನ್ನು
ಪರಿ-ಣತ
ಪರಿ-ಣ-ತ-ರ-ನ್ನಾಗಿ
ಪರಿ-ಣ-ತ-ರಾದ
ಪರಿ-ಣ-ತರು
ಪರಿ-ಣ-ತರೂ
ಪರಿ-ಣ-ತಿ-ಯನ್ನು
ಪರಿ-ಣ-ತಿ-ಯಿ-ದ್ದ-ವರು
ಪರಿ-ಣ-ಮಿ-ಸ-ಬೇ-ಕೆಂದು
ಪರಿ-ಣ-ಮಿ-ಸ-ಲಿಲ್ಲ
ಪರಿ-ಣ-ಮಿಸಿ
ಪರಿ-ಣ-ಮಿ-ಸಿತು
ಪರಿ-ಣ-ಮಿ-ಸಿತ್ತು
ಪರಿ-ಣ-ಮಿ-ಸಿ-ದುವು
ಪರಿ-ಣ-ಮಿ-ಸಿಲ್ಲ
ಪರಿ-ಣ-ಮಿ-ಸು-ತ್ತಿವೆ
ಪರಿ-ಣ-ಮಿ-ಸು-ವ-ವ-ರೆಗೆ
ಪರಿ-ಣಾಮ
ಪರಿ-ಣಾ-ಮ-ಕಾ-ರಿ-ಯಾಗಿ
ಪರಿ-ಣಾ-ಮ-ಕಾ-ರಿ-ಯಾ-ಗಿತ್ತು
ಪರಿ-ಣಾ-ಮ-ವನ್ನು
ಪರಿ-ಣಾ-ಮ-ವ-ನ್ನುಂಟು
ಪರಿ-ಣಾ-ಮ-ವ-ನ್ನುಂ-ಟು-ಮಾ-ಡು-ತ್ತಿದೆ
ಪರಿ-ಣಾ-ಮ-ವನ್ನೂ
ಪರಿ-ಣಾ-ಮ-ವಾಗಿ
ಪರಿ-ಣಾ-ಮ-ವಾ-ಗಿ-ಲ್ಲ-ವೆ-ನ್ನ-ಬ-ಹುದು
ಪರಿ-ಣಾ-ಮವು
ಪರಿ-ಣಾ-ಮ-ವುಂ-ಟಾ-ಗ-ಬ-ಹು-ದೆಂ-ಬು-ದನ್ನೂ
ಪರಿ-ಣಾ-ಮ-ವೆಂದರೆ
ಪರಿ-ಣಾ-ಮವೋ
ಪರಿ-ತ-ಪಿ-ಸು-ತ್ತ-ದೆಯೆ
ಪರಿ-ತ್ಯಾಗ
ಪರಿ-ನಿ-ರ್ಯಾಣ
ಪರಿ-ಪಕ್ವ
ಪರಿ-ಪ-ಕ್ವ-ಗೊಂಡ
ಪರಿ-ಪ-ರಿ-ಯಾಗಿ
ಪರಿ-ಪಾ-ಠ-ವಿ-ಟ್ಟು-ಕೊಂ-ಡಿ-ದ್ದ-ವರು
ಪರಿ-ಪಾ-ಠ-ವಿದೆ
ಪರಿ-ಪಾ-ಲಿಸ
ಪರಿ-ಪಾ-ಲಿ-ಸಲು
ಪರಿ-ಪಾ-ಲಿ-ಸಿ-ಕೊಂಡು
ಪರಿ-ಪೂರ್ಣ
ಪರಿ-ಪೂ-ರ್ಣ-ಗೊ-ಳಿ-ಸು-ವು-ದ-ಕ್ಕಾ-ಗಿಯೇ
ಪರಿ-ಪೂ-ರ್ಣ-ಗೊ-ಳ್ಳು-ವುದನ್ನು
ಪರಿ-ಪೂ-ರ್ಣ-ಜ್ಞಾ-ನಿ-ಯಾ-ಗಲು
ಪರಿ-ಪೂ-ರ್ಣತೆ
ಪರಿ-ಪೂ-ರ್ಣ-ತೆ-ಗಳು
ಪರಿ-ಪೂ-ರ್ಣ-ತೆಯ
ಪರಿ-ಪೂ-ರ್ಣ-ತೆ-ಯನ್ನು
ಪರಿ-ಪೂ-ರ್ಣ-ತೆಯು
ಪರಿ-ಪೂ-ರ್ಣ-ವಾ-ಗ-ಲಾ-ರದು
ಪರಿ-ಮ-ಳ-ಭ-ರಿತ
ಪರಿ-ಮಾ-ಣದ
ಪರಿ-ಮಿತಿ
ಪರಿ-ಮಿ-ತಿ-ಗಳನ್ನು
ಪರಿ-ಮಿ-ತಿ-ಯನ್ನು
ಪರಿ-ಮಿ-ತಿ-ಯೊ-ಳಗೆ
ಪರಿ-ಯನ್ನು
ಪರಿ-ಯಲ್ಲಿ
ಪರಿ-ವ-ರ್ತ-ನ-ಶೀಲ
ಪರಿ-ವ-ರ್ತನೆ
ಪರಿ-ವ-ರ್ತ-ನೆ-ಗೊಂಡ
ಪರಿ-ವ-ರ್ತ-ನೆ-ಗೊ-ಳಿ-ಸಲು
ಪರಿ-ವ-ರ್ತ-ನೆ-ಯಾ-ಗದೇ
ಪರಿ-ವ-ರ್ತ-ನೆ-ಯಾ-ಗುವು
ಪರಿ-ವ-ರ್ತ-ನೆ-ಯಾ-ಯಿತು
ಪರಿ-ವ-ರ್ತಿ-ಸ-ಬೇ-ಕಾ-ಗು-ತ್ತದೆ
ಪರಿ-ವಾ-ಜ್ರ-ಕ-ರಾ-ಗಿ-ದ್ದಾ-ಗಿನ
ಪರಿ-ವಾರ
ಪರಿ-ವಾ-ರದ
ಪರಿ-ವಾ-ರ-ದಲ್ಲಿ
ಪರಿ-ವಾ-ರ-ದ-ವರ
ಪರಿ-ವಾ-ರ-ದ-ವ-ರ-ಲ್ಲದೆ
ಪರಿ-ವಾ-ರ-ದ-ವರಿ
ಪರಿ-ವಾ-ರ-ದ-ವ-ರಿ-ಗಾಗಿ
ಪರಿ-ವಾ-ರ-ದ-ವರು
ಪರಿ-ವಾ-ರ-ದ-ವರೂ
ಪರಿ-ವಾ-ರ-ದೊಂ-ದಿಗೆ
ಪರಿ-ವಾ-ರ-ವನ್ನು
ಪರಿ-ವಾ-ರ-ಸ-ಮೇ-ತ-ನಾಗಿ
ಪರಿ-ವಾ-ರ-ಸ-ಮೇ-ತ-ರಾಗಿ
ಪರಿ-ವಾ-ರ-ಸ-ಹಿ-ತ-ರಾಗಿ
ಪರಿ-ವೆಯೇ
ಪರಿ-ವ್ರ-ಜ-ನಕ್ಕೂ
ಪರಿ-ವ್ರ-ಜ-ನ-ವನ್ನು
ಪರಿ-ವ್ರಾ-ಜಕ
ಪರಿ-ವ್ರಾ-ಜ-ಕ-ನಾಗಿ
ಪರಿ-ವ್ರಾ-ಜ-ಕ-ರಾಗಿ
ಪರಿ-ವ್ರಾ-ಜ-ಕ-ರಾ-ಗಿ-ದ್ದಾ-ಗಲೂ
ಪರಿ-ವ್ರಾ-ಜ-ಕ-ರಾ-ಗಿ-ದ್ದಾ-ಗಲೇ
ಪರಿ-ಶೀ-ಲಿಸಿ
ಪರಿ-ಶೀ-ಲಿ-ಸಿ-ದರು
ಪರಿ-ಶೀ-ಲಿ-ಸು-ತ್ತಿ-ದ್ದರು
ಪರಿ-ಶುದ್ಧ
ಪರಿ-ಶು-ದ್ಧತೆ
ಪರಿ-ಶು-ದ್ಧ-ತೆ-ದಿ-ವ್ಯತೆ
ಪರಿ-ಶು-ದ್ಧ-ತೆ-ಸಂ-ಸ್ಕೃ-ತಿ-ತ್ಯಾಗ
ಪರಿ-ಶು-ದ್ಧ-ತೆ-ಗಳ
ಪರಿ-ಶು-ದ್ಧ-ವಾ-ಗು-ವು-ದ-ರಿಂದ
ಪರಿ-ಶು-ದ್ಧ-ವಾದ
ಪರಿ-ಶು-ದ್ಧ-ವಾ-ದುದು
ಪರಿ-ಶು-ದ್ಧಾ-ತ್ಮರು
ಪರಿ-ಶ್ರಮ
ಪರಿ-ಶ್ರ-ಮದ
ಪರಿ-ಶ್ರ-ಮ-ದಿಂದ
ಪರಿ-ಶ್ರ-ಮ-ದಿಂ-ದಲೂ
ಪರಿ-ಶ್ರ-ಮ-ವನ್ನೂ
ಪರಿ-ಷ-ತ್ತಿನ
ಪರಿ-ಸ-ಮಾ-ಪ್ತ-ವಾ-ಗುವ
ಪರಿ-ಸ-ಮಾ-ಪ್ತಿ-ಗೊ-ಳಿ-ಸಿ-ದರು
ಪರಿ-ಸರ
ಪರಿ-ಸ-ರ-ಸ-ನ್ನಿ-ವೇ-ಶ-ಗಳನ್ನು
ಪರಿ-ಸ-ರದ
ಪರಿ-ಸ-ರ-ದಲ್ಲಿ
ಪರಿ-ಸ-ರ-ವನ್ನು
ಪರಿ-ಸ-ರ-ವನ್ನೇ
ಪರಿ-ಸ-ರ-ಹೀಗೆ
ಪರಿ-ಸ್ಥಿತಿ
ಪರಿ-ಸ್ಥಿ-ತಿ-ಗಳ
ಪರಿ-ಸ್ಥಿ-ತಿಯ
ಪರಿ-ಸ್ಥಿ-ತಿ-ಯ-ನ್ನ-ರಿತ
ಪರಿ-ಸ್ಥಿ-ತಿ-ಯನ್ನು
ಪರಿ-ಸ್ಥಿ-ತಿ-ಯಲ್ಲಿ
ಪರಿ-ಸ್ಥಿ-ತಿ-ಯಲ್ಲೂ
ಪರಿ-ಸ್ಥಿ-ತಿ-ಯಿಂದ
ಪರಿ-ಸ್ಥಿ-ತಿಯೂ
ಪರಿ-ಸ್ಥಿ-ತಿಯೇ
ಪರಿ-ಹ-ರಿ-ಸಲು
ಪರಿ-ಹ-ರಿ-ಸಿ-ಕೊ-ಳ್ಳು-ವು-ದ-ರಲ್ಲೇ
ಪರಿ-ಹಾರ
ಪರಿ-ಹಾ-ರ-ಕಾ-ರ್ಯದ
ಪರಿ-ಹಾ-ರ-ಕಾ-ರ್ಯ-ವನ್ನು
ಪರಿ-ಹಾ-ರ-ಕ್ಕಾಗಿ
ಪರಿ-ಹಾ-ರಕ್ಕೆ
ಪರಿ-ಹಾ-ರ-ವನ್ನು
ಪರಿ-ಹಾ-ರ-ವಾ-ಗಲೂ
ಪರಿ-ಹಾ-ರ-ವಾ-ಗಿ-ಬಿ-ಡು-ತ್ತವೆ
ಪರಿ-ಹಾ-ರ-ವಾದ
ಪರಿ-ಹಾ-ರ-ವಾ-ದುವು
ಪರಿ-ಹಾ-ರ-ವೇ-ನಾ-ದರೂ
ಪರಿ-ಹಾ-ರೋ-ಪಾಯ
ಪರೀಕ್ಷಾ
ಪರೀ-ಕ್ಷಿ-ಸ-ಬೇ-ಕೆಂದು
ಪರೀ-ಕ್ಷಿ-ಸ-ಬೇ-ಕೆಂಬ
ಪರೀ-ಕ್ಷಿ-ಸ-ಲಾ-ರಂ-ಭ-ಗ-ರ್ಭ-ಗು-ಡಿ-ಯಲ್ಲಿ
ಪರೀ-ಕ್ಷಿ-ಸಲು
ಪರೀ-ಕ್ಷಿಸಿ
ಪರೀ-ಕ್ಷಿ-ಸಿದ
ಪರೀ-ಕ್ಷಿ-ಸಿ-ದರು
ಪರೀ-ಕ್ಷಿ-ಸಿ-ದರೆ
ಪರೀ-ಕ್ಷಿ-ಸು-ವಂತೆ
ಪರೀ-ಕ್ಷಿ-ಸು-ವು-ದ-ಕ್ಕಾಗಿ
ಪರೀಕ್ಷೆ
ಪರೀ-ಕ್ಷೆ-ನಿ-ರೀ-ಕ್ಷೆ-ಗ-ಳೆಲ್ಲ
ಪರೀ-ಕ್ಷೆ-ಗಳನ್ನು
ಪರೀ-ಕ್ಷೆ-ಗಳಿಂದ
ಪರೀ-ಕ್ಷೆಗೆ
ಪರೀ-ಕ್ಷೆಯ
ಪರೀ-ಕ್ಷೆ-ಯನ್ನು
ಪರೀ-ಕ್ಷೆ-ಯಲ್ಲೂ
ಪರೀ-ಕ್ಷೆ-ಯಿ-ಟ್ಟರು
ಪರೀ-ಕ್ಷೆ-ಯಿ-ಡಲು
ಪರೆ-ಯನ್ನು
ಪರೋಕ್ಷ
ಪರೋ-ಕ್ಷ-ವಾಗಿ
ಪರೋ-ಪ-ಕಾರ
ಪರ್ಣ-ಕು-ಟಿ-ಗ-ಳಲ್ಲೂ
ಪರ್ಯಂತ
ಪರ್ಯ-ವ-ಸಾ-ನ-ಗೊ-ಳ್ಳು-ತ್ತದೆ
ಪರ್ಯಾ
ಪರ್ಯಾ-ಲೋ-ಚಿಸಿ
ಪರ್ವತ
ಪರ್ವ-ತ-ಗಳ
ಪರ್ವ-ತ-ಗ-ಳದು
ಪರ್ವ-ತ-ಗಳನ್ನು
ಪರ್ವ-ತ-ಗಳಲ್ಲಿ
ಪರ್ವ-ತ-ಗ-ಹ್ವ-ರ-ಗಳಲ್ಲಿ
ಪರ್ವ-ತದ
ಪರ್ವ-ತ-ದಂತೆ
ಪರ್ವ-ತ-ದಷ್ಟು
ಪರ್ವ-ತವ
ಪರ್ವ-ತವು
ಪರ್ವ-ತ-ವೆಂ-ಬಂತೆ
ಪರ್ವ-ತ-ಶಿ-ಖ-ರ-ಗ-ಳೆಲ್ಲ
ಪರ್ವ-ತ-ಶ್ರೇ-ಣಿ-ಗಳಿಂದ
ಪರ್ವ-ತ-ಶ್ರೇ-ಣಿಯ
ಪರ್ವ-ತಾ-ಗ್ರಕ್ಕೆ
ಪರ್ವ-ತಾ-ರೋ-ಹಣ
ಪರ್ವ-ತೋ-ಪ-ಮವಾ
ಪರ್ವ-ತೋ-ಪ-ಮ-ವಾದ
ಪರ್ಶಿ-ಯದ
ಪರ್ಷಿ-ಯ-ನ್ನರು
ಪಲ್ಯ
ಪಲ್ಲ-ಕ್ಕಿ-ಯನ್ನು
ಪಲ್ಲ-ಕ್ಕಿ-ಯ-ನ್ನೇರಿ
ಪಲ್ಲ-ಕ್ಕಿ-ಯೇ-ರು-ವಂತೆ
ಪಲ್ಲವಿ
ಪಲ್ಲ-ವಿ-ಯಾ-ಗಿತ್ತು
ಪಳೆ-ಯು-ಳಿಕೆ
ಪಳೆ-ಯು-ಳಿ-ಕೆ-ಗಳನ್ನು
ಪಳ್ಳಿ-ಯ-ವರ
ಪವಾಡ
ಪವಾ-ಡ-ಗಳನ್ನು
ಪವಾ-ಡ-ವನ್ನೇ
ಪವಾ-ಡ-ಶಕ್ತಿ
ಪವಾ-ಡ-ಶ-ಕ್ತಿ-ಯಿದೆ
ಪವಾ-ಡ-ಶ-ಕ್ತಿ-ಯೆಲ್ಲ
ಪವಾ-ಹಾರಿ
ಪವಿತ್ರ
ಪವಿ-ತ್ರ-ಗೊ-ಳಿ-ಸು-ತ್ತಿ-ರು-ತ್ತಾರೆ
ಪವಿ-ತ್ರ-ಜೀ-ವ-ನ-ವನ್ನು
ಪವಿ-ತ್ರತೆ
ಪವಿ-ತ್ರ-ತೆ-ಶಾಂ-ತಿ-ಇ-ವು-ಗಳಲ್ಲಿ
ಪವಿ-ತ್ರ-ತೆಯ
ಪವಿ-ತ್ರ-ತೆ-ಯೆಂ-ಬುದು
ಪವಿ-ತ್ರ-ರಾಗು
ಪವಿ-ತ್ರ-ವಾದ
ಪವಿ-ತ್ರ-ವಾ-ದದ್ದು
ಪವಿ-ತ್ರ-ವಾ-ದುದೊ
ಪವಿ-ತ್ರಾ-ತ್ಮರು
ಪವೇಶ
ಪಶು-ಪ-ತಿ-ನಾಥ್
ಪಶು-ಸ-ಮಾ-ನ-ರಾ-ಗಿ-ರು-ವುದನ್ನು
ಪಶ್ಚಾ-ತ್ತಾಪ
ಪಶ್ಚಿಮ
ಪಶ್ಚಿ-ಮದ
ಪಶ್ಚಿ-ಮ-ದತ್ತ
ಪಶ್ಚಿ-ಮ-ದಲ್ಲಿ
ಪಶ್ಚಿ-ಮ-ದ-ವ-ರೆಗೆ
ಪಶ್ಚಿ-ಮ-ದೇಶ
ಪಶ್ಚಿ-ಮ-ದೇ-ಶ-ಗ-ಳಿಗೆ
ಪಶ್ಚಿ-ಮ-ದೇ-ಶ-ದಲ್ಲಿ
ಪಶ್ಚಿ-ಮ-ರಾ-ಷ್ಟ್ರ-ಗ-ಳಿಗೆ
ಪಸ-ರಿ-ಸಲು
ಪಸ-ರಿ-ಸಿತ್ತು
ಪಸ-ರಿ-ಸಿದ್ದು
ಪಸ-ರಿ-ಸಿ-ರುವೆ
ಪಸ-ರಿ-ಸಿವೆ
ಪಸಾ-ಡೆನ
ಪಸಾ-ಡೆ-ನಕ್ಕೆ
ಪಸಾ-ಡೆ-ನ-ಗಳ
ಪಸಾ-ಡೆ-ನ-ಗಳಲ್ಲಿ
ಪಸಾ-ಡೆ-ನದ
ಪಸಾ-ಡೆ-ನ-ದ-ಲ್ಲ-ಲ್ಲದೆ
ಪಸಾ-ಡೆ-ನ-ದಲ್ಲಿ
ಪಸಾ-ಡೆ-ನ-ದ-ಲ್ಲಿಯೂ
ಪಸಾ-ಡೆ-ನ-ದ-ವ-ಳಾದ
ಪಸಾ-ಡೆ-ನ-ದಿಂದ
ಪಹಲ್
ಪಹ-ಲ್ಗಾಂವ್
ಪಹ-ಲ್ಗಾಂ-ವ್ನ-ವ-ರೆಗೆ
ಪಹ-ಲ್ಗಾಂ-ವ್ನಿಂದ
ಪಹಾಡೀ
ಪಾಂಚ್ಕೋರಿ
ಪಾಂಚ್ಕೋ-ರಿಯ
ಪಾಂಚ್ಕೋ-ರಿ-ಯಿಂದ
ಪಾಂಚ್ಕೋ-ರಿಯು
ಪಾಂಡಿತ್ಯ
ಪಾಂಡಿ-ತ್ಯದ
ಪಾಂಡಿ-ತ್ಯ-ದಿಂದ
ಪಾಂಡಿ-ತ್ಯ-ಪೂ-ರ್ಣವೂ
ಪಾಂಡಿ-ತ್ಯ-ವನ್ನು
ಪಾಂಡಿ-ತ್ಯ-ವಿ-ದ್ದ-ವನು
ಪಾಂಡಿ-ತ್ಯವೂ
ಪಾಂಡು-ರಂಗ
ಪಾಂಪೆ
ಪಾಂಬ-ನ್ನಿಗೆ
ಪಾಂಬ-ನ್ನಿ-ನಲ್ಲಿ
ಪಾಂಬ-ನ್ನಿ-ನಿಂದ
ಪಾಕ್ಷಿಕ
ಪಾಟರ್
ಪಾಟ್ನಾ
ಪಾಠ
ಪಾಠ-ಪ್ರ-ವ-ಚ-ನ-ಸಂ-ಭಾ-ಷ-ಣೆ-ಗಳಲ್ಲಿ
ಪಾಠ-ವನ್ನು
ಪಾಠ-ವನ್ನೋ
ಪಾಠವೇ
ಪಾಠ-ಶಾಲೆ
ಪಾಠ-ಶಾ-ಲೆ-ಗೆಗೆ
ಪಾಠ-ಶಾ-ಲೆ-ಯಂ-ತಹ
ಪಾಡಿಗೆ
ಪಾಣಿನಿ
ಪಾಣಿ-ನಿಯ
ಪಾತ-ಕ-ಗ-ಳಿಗೆ
ಪಾತ-ಕ-ಗ-ಳಿ-ವೆ-ಒಂದು
ಪಾತಾ-ಳಕ್ಕೆ
ಪಾತಾ-ಳ-ಗಳನ್ನು
ಪಾತು
ಪಾತ್ರ
ಪಾತ್ರ-ಧ-ರಿಸಿ
ಪಾತ್ರ-ನಲ್ಲ
ಪಾತ್ರ-ನ-ಲ್ಲ-ವೆಂ-ಬುದು
ಪಾತ್ರ-ನಾ-ದ-ವನು
ಪಾತ್ರ-ರಾ-ಗಿ-ದ್ದರು
ಪಾತ್ರ-ರಾ-ಗು-ತ್ತಿ-ದ್ದರು
ಪಾತ್ರ-ರಾ-ದರು
ಪಾತ್ರ-ಳಾ-ದಳು
ಪಾತ್ರ-ವನ್ನು
ಪಾತ್ರ-ವ-ಹಿ-ಸಿ-ತೆಂ-ಬು-ದನ್ನು
ಪಾತ್ರ-ವ-ಹಿ-ಸಿ-ದ್ದರೂ
ಪಾತ್ರ-ವಾ-ಗ-ಲಿದೆ
ಪಾತ್ರ-ವಿದೆ
ಪಾತ್ರ-ವೇನು
ಪಾತ್ರೆ
ಪಾತ್ರೆ-ಯನ್ನು
ಪಾದ
ಪಾದ-ಗಳನ್ನು
ಪಾದ-ಗಳಲ್ಲಿ
ಪಾದ-ಗ-ಳಿಗೂ
ಪಾದ-ಗ-ಳಿಗೆ
ಪಾದದ
ಪಾದ-ಧೂ-ಳಿ-ಯನ್ನು
ಪಾದ-ಧೂ-ಳಿ-ಯಿಂದ
ಪಾದ-ಪ-ದ್ಮ-ಗಳಲ್ಲಿ
ಪಾದ-ಪ-ದ್ಮ-ಗ-ಳಿಗೆ
ಪಾದ-ಯಾತ್ರೆ
ಪಾದ-ರ-ಕ್ಷೆ-ಗಳನ್ನು
ಪಾದ-ರ-ಕ್ಷೆ-ಯಿ-ದ್ದ-ವ-ರೆಂ-ದರೆ
ಪಾದ-ರಿ-ಯಾದ
ಪಾದ-ವನ್ನು
ಪಾದ-ವನ್ನೇ
ಪಾದ-ಸೇ-ವೆ-ಯಲ್ಲಿ
ಪಾದ-ಸ್ಪರ್ಶ
ಪಾದ-ಸ್ಪ-ರ್ಶವು
ಪಾದ್ರಿ
ಪಾದ್ರಿ-ಗ-ಳಿಗೆ
ಪಾದ್ರಿ-ಗಳು
ಪಾದ್ರಿ-ಗಳೂ
ಪಾದ್ರಿ-ಯಂ-ತಾ-ಗಲಿ
ಪಾನ
ಪಾನಂ-ದ-ರನ್ನು
ಪಾನ-ಮಾ-ಡುವ
ಪಾನ್
ಪಾಪ
ಪಾಪ-ಕ-ರ್ಮವೂ
ಪಾಪ-ಕ-ಲ್ಪ-ನೆ-ಯೆಲ್ಲ
ಪಾಪಕ್ಕೆ
ಪಾಪ-ಗಳ
ಪಾಪ-ಗಳನ್ನು
ಪಾಪ-ಗ-ಳಲ್ಲ
ಪಾಪ-ಗ-ಳಿಗೆ
ಪಾಪ-ಗಳೂ
ಪಾಪದ
ಪಾಪ-ದಂತೆ
ಪಾಪ-ದ-ವ-ನನ್ನು
ಪಾಪ-ದಿಂ-ದಲೇ
ಪಾಪ-ಪ-ರಿ-ಹಾ-ರಕ್ಕೆ
ಪಾಪ-ರಾ-ಶಿಯೂ
ಪಾಪ-ವನ್ನು
ಪಾಪ-ವಲ್ಲ
ಪಾಪವು
ಪಾಪವೂ
ಪಾಪ-ವೆಂದು
ಪಾಪವೇ
ಪಾಪಿ
ಪಾಪಿಯ
ಪಾಪಿ-ಯಾಗಿ
ಪಾಪಿಷ್ಠ
ಪಾಮರ್
ಪಾಯ-ಸಾ-ನ್ನ-ವನ್ನು
ಪಾರಂ-ಗ-ತನೋ
ಪಾರಂ-ಗ-ತ-ರಾದ
ಪಾರಂ-ಗ-ತರು
ಪಾರ-ತಂ-ತ್ರ್ಯ-ದಿಂದ
ಪಾರ-ಮಾ-ರ್ಥಿಕ
ಪಾರ-ಮಾ-ರ್ಥಿ-ಕ-ವಾ-ಗಿ-ರ-ಬ-ಹು-ದು-ವಿ-ಧ್ಯು-ಕ್ತ-ವಾಗಿ
ಪಾರ-ಲೌ-ಕಿಕ
ಪಾರವೇ
ಪಾರಾ
ಪಾರಾ-ಗ-ಬೇ-ಕಾ-ಗಿದೆ
ಪಾರಾ-ಗ-ಬೇ-ಕಾ-ಯಿತು
ಪಾರಾ-ಗಲು
ಪಾರಾಗಿ
ಪಾರಾ-ಗಿ-ದ್ದಾರೆ
ಪಾರಾಗು
ಪಾರಾ-ಗು-ತ್ತಾರೆ
ಪಾರಾ-ಗು-ವಂತೆ
ಪಾರಾ-ಗು-ವು-ದ-ಕ್ಕಾ-ಗಿಯೇ
ಪಾರಾ-ಯ-ಣ-ಪ-ಠ-ಣ-ಗ-ಳಿಗೂ
ಪಾರಾ-ಯ-ಣ-ದ-ಲ್ಲಲ್ಲ
ಪಾರಾ-ಯಿತು
ಪಾರು
ಪಾರು-ಮಾ-ಡಿ-ಕೊ-ಳ್ಳ-ಬೇಕು
ಪಾರು-ಮಾ-ಡಿದೆ
ಪಾರು-ಮಾ-ಡು-ತ್ತದೆ
ಪಾರ್ಕಿಗೋ
ಪಾರ್ಟಿ
ಪಾರ್ಥ-ಸಾ-ರಥಿ
ಪಾರ್ಲಿ-ಮೆಂಟು
ಪಾರ್ವ-ತಿ-ಪ-ತಯೇ
ಪಾರ್ವ-ತೀ-ಪ-ತಯೇ
ಪಾರ್ಶ್ವ-ವಾಯು
ಪಾರ್ಶ್ವ-ವಾ-ಯು-ವಿ-ಗೀ-ಡಾ-ದರೆ
ಪಾಲನೂ
ಪಾಲ-ನೆ-ಯಿಂದ
ಪಾಲ-ನ್ನಾ-ದರೂ
ಪಾಲನ್ನು
ಪಾಲಿ-ಗಿದು
ಪಾಲಿ-ಗಿಲ್ಲ
ಪಾಲಿಗೆ
ಪಾಲಿ-ಗೇನೋ
ಪಾಲಿ-ಗೊಂದು
ಪಾಲಿಗೋ
ಪಾಲಿನ
ಪಾಲಿ-ನ-ದಾ-ಗಿತ್ತು
ಪಾಲಿ-ಸ-ಬಲ್ಲ
ಪಾಲಿ-ಸ-ಬೇಕು
ಪಾಲಿ-ಸ-ಬೇ-ಕೆಂದು
ಪಾಲಿ-ಸಲು
ಪಾಲಿ-ಸ-ಲೇ-ಬೇಕು
ಪಾಲಿ-ಸ-ಲೇ-ಬೇ-ಕೆಂ-ಬುದು
ಪಾಲಿ-ಸಿ-ಕೊಂಡು
ಪಾಲಿ-ಸಿ-ದರು
ಪಾಲಿ-ಸಿ-ದಲ್ಲಿ
ಪಾಲಿ-ಸಿ-ದ್ದೇ-ವೆಂಬ
ಪಾಲಿಸು
ಪಾಲಿ-ಸು-ತ್ತಿದ್ದ
ಪಾಲಿ-ಸು-ತ್ತಿ-ದ್ದರು
ಪಾಲಿ-ಸು-ತ್ತಿ-ದ್ದೇನೆ
ಪಾಲಿ-ಸು-ತ್ತಿಲ್ಲ
ಪಾಲಿ-ಸು-ವಲ್ಲಿ
ಪಾಲು
ಪಾಲೂ
ಪಾಲ್ಗೊಂ-ಡರು
ಪಾಲ್ಗೊಂ-ಡ-ವ-ನೊಬ್ಬ
ಪಾಲ್ಗೊಂ-ಡಿದ್ದು
ಪಾಲ್ಗೊಂಡು
ಪಾಲ್ಗೊ-ಳ್ಳಲಿ
ಪಾಲ್ಗೊ-ಳ್ಳಲು
ಪಾಲ್ಗೊ-ಳ್ಳ-ಲೆಂದೇ
ಪಾಲ್ಗೊ-ಳ್ಳು-ತ್ತಾರೋ
ಪಾಲ್ಗೊ-ಳ್ಳು-ವಂ-ಥದೂ
ಪಾಲ್ಗೊ-ಳ್ಳು-ವು-ದರ
ಪಾಳು-ಬಿದ್ದ
ಪಾವ-ನ-ವಾ-ಗು-ವುದು
ಪಾವಿತ್ರ್ಯ
ಪಾವಿ-ತ್ರ್ಯಕ್ಕೆ
ಪಾವಿ-ತ್ರ್ಯದ
ಪಾವಿ-ತ್ರ್ಯ-ವನ್ನು
ಪಾಶ್ಚಾತ್ಯ
ಪಾಶ್ಚಾ-ತ್ಯ
ಪಾಶ್ಚಾ-ತ್ಯ-ಭಾ-ರ-ತೀಯ
ಪಾಶ್ಚಾ-ತ್ಯ-ದೇ-ಶ-ಗಳಲ್ಲಿ
ಪಾಶ್ಚಾ-ತ್ಯ-ದೇ-ಶ-ಗ-ಳಿಗೆ
ಪಾಶ್ಚಾ-ತ್ಯ-ನಿಗೆ
ಪಾಶ್ಚಾ-ತ್ಯರ
ಪಾಶ್ಚಾ-ತ್ಯ-ರನ್ನು
ಪಾಶ್ಚಾ-ತ್ಯ-ರಲ್ಲಿ
ಪಾಶ್ಚಾ-ತ್ಯ-ರಿಗೆ
ಪಾಶ್ಚಾ-ತ್ಯ-ರಿಗೇ
ಪಾಶ್ಚಾ-ತ್ಯ-ರಿ-ಗೊಂದು
ಪಾಶ್ಚಾ-ತ್ಯ-ರಿ-ದ್ದೀ-ರ-ಲ್ಲ-ತುಂಬ
ಪಾಶ್ಚಾ-ತ್ಯರು
ಪಾಶ್ಚಾ-ತ್ಯರೂ
ಪಾಶ್ಚಾ-ತ್ಯ-ವಾದ
ಪಾಸು-ಮಾಡಿ
ಪಾಸ್
ಪಾಸ್ಗಳನ್ನು
ಪಿ
ಪಿಂಗಾ-ಣಿಯ
ಪಿಂಡ-ಪ್ರ-ದಾನ
ಪಿಂಡಾರಿ
ಪಿಂಡಿಯ
ಪಿಟೀಲು
ಪಿತ
ಪಿತರು
ಪಿತೃ-ಗಳನ್ನು
ಪಿತೃ-ಪೂಜೆ
ಪಿತೃ-ಪೂ-ಜೆಯೇ
ಪಿತ್ರಾ-ರ್ಜಿತ
ಪಿಪಾ-ಸು-ಗ-ಳಾಗಿ
ಪಿಯರ್
ಪಿಯಾನೋ
ಪಿರ-ಮಿ-ಡ್ಡಿ-ನಂ-ತಿದೆ
ಪಿರ-ಮಿಡ್ಡು
ಪಿರ-ಮಿ-ಡ್ಡು-ಗಳ
ಪಿಲಿ-ಭಿತ್
ಪಿಲಿ-ಭಿ-ತ್ತಿ-ನಲ್ಲಿ
ಪಿಳ್ಳೆ-ಯ-ವರು
ಪಿಶಾಚಿ
ಪಿಶಾ-ಚಿ-ಗ-ಳೂ-ಇ-ವರೆ-ಲ್ಲರೂ
ಪಿಶು
ಪಿಸು-ಗು-ಟ್ಟಿ-ದರು
ಪಿಸು-ದ-ನಿ-ಯಲ್ಲಿ
ಪಿಸು-ದ-ನಿ-ಯಾಗಿ
ಪೀಟ-ರನೂ
ಪೀಟ-ರ್ಸ್
ಪೀಠದ
ಪೀಠ-ದಲ್ಲಿ
ಪೀಠ-ವ-ನ್ನೇ-ರಿ-ದರು
ಪೀಠವೂ
ಪೀಠಿ-ಕೆ-ಯೊಂ-ದಿಗೆ
ಪೀಡಿತ
ಪೀಡಿ-ತ-ನಾಗ
ಪೀಡಿ-ತರ
ಪೀಡಿ-ತ-ರಾಗಿ
ಪೀಡಿ-ತ-ವಾದ
ಪೀಡಿ-ತ-ವಾ-ದ್ದ-ರಿಂದ
ಪೀಳಿಗೆ
ಪೀಳಿ-ಗೆಗೆ
ಪೀಳಿ-ಗೆಯು
ಪೀಸಾ
ಪೀಸಾ-ದಿಂದ
ಪುಕ್ಕ
ಪುಕ್ಕಟೆ
ಪುಕ್ಕಲು
ಪುಗ್ವೇ-ದದ
ಪುಜು-ತ್ವ-ವನ್ನೇ
ಪುಜು-ಮಾ-ರ್ಗ-ಗಳಿಂದ
ಪುಟ
ಪುಟ-ಗ-ಟ್ಟಲೆ
ಪುಟ-ಗ-ಳನ್ನೇ
ಪುಟ-ಪು-ಟ-ದಲ್ಲೂ
ಪುಟಿ-ದೆ-ದ್ದುದು
ಪುಟಿ-ಯುವ
ಪುಟ್ಟ
ಪುಟ್ಟ-ರಾ-ಶಿ-ಯನ್ನು
ಪುಡಿ-ಗೈದು
ಪುಡಿ-ಗೈ-ಯ-ಲೆ-ತ್ನಿ-ಸುವ
ಪುಡಿ-ಪು-ಡಿ-ಯಾ-ಗು-ತ್ತದೆ
ಪುಣ-ದಿಂದ
ಪುಣವ
ಪುಣ-ವನ್ನು
ಪುಣಿ-ಯಾ-ಗಿ-ದ್ದೇನೆ
ಪುಣ್ಯ
ಪುಣ್ಯಕ್ಕೆ
ಪುಣ್ಯ-ಕ್ಷೇತ್ರ
ಪುಣ್ಯ-ಕ್ಷೇ-ತ್ರ-ಗಳ
ಪುಣ್ಯ-ಗ-ಳಷ್ಟೇ
ಪುಣ್ಯ-ಭೂಮಿ
ಪುಣ್ಯ-ಭೂ-ಮಿ-ಕ-ರ್ಮ-ಭೂಮಿ
ಪುಣ್ಯ-ಮು-ಹೂ-ರ್ತ-ದಲ್ಲಿ
ಪುಣ್ಯವೋ
ಪುಣ್ಯಾ-ತ್ಮ-ಪಾ-ಪಾ-ತ್ಮ-ರೆ-ನ್ನದೆ
ಪುತು-ವಿನ
ಪುತ್ಥಳಿ
ಪುತ್ರ
ಪುತ್ರನ
ಪುತ್ರ-ನನ್ನು
ಪುತ್ರ-ನಾ-ಗು-ತ್ತೇನೆ
ಪುತ್ರರ
ಪುತ್ರ-ರ-ತ್ನ-ರಿಗೆ
ಪುತ್ರ-ರಿಗೂ
ಪುತ್ರ-ರಿರಾ
ಪುತ್ರರು
ಪುತ್ರರೆ
ಪುತ್ರರೇ
ಪುತ್ರ-ಶೋಕ
ಪುತ್ರ-ಶೋ-ಕಕ್ಕೆ
ಪುತ್ರಿಗೆ
ಪುತ್ರಿಯ
ಪುತ್ರಿ-ಯನ್ನು
ಪುತ್ರಿ-ಯೆಂದು
ಪುತ್ರಿ-ಯೆಂಬ
ಪುನಃ
ಪುನಃ-ಸ್ಥಾ-ಪ-ನೆಯ
ಪುನ-ರಾ-ರಂ-ಭ-ಗೊ-ಳಿ-ಸಲು
ಪುನ-ರಾ-ರಂ-ಭ-ವಾ-ಗು-ತ್ತಿತ್ತು
ಪುನ-ರಾ-ರಂ-ಭ-ವಾ-ಯಿತು
ಪುನ-ರಾ-ರಂ-ಭಿ-ಸಿ-ದರು
ಪುನ-ರಾ-ವ-ರ್ತನೆ
ಪುನರು
ಪುನ-ರು-ಚ್ಚ-ರಿ-ಸಿ-ದರು
ಪುನ-ರು-ಚ್ಚ-ರಿ-ಸು-ತ್ತಿಲ್ಲ
ಪುನ-ರು-ಜ್ಜೀ-ವ-ನ-ಗೊಳಿ
ಪುನ-ರು-ಜ್ಜೀ-ವ-ನ-ಗೊ-ಳಿಸಿ
ಪುನ-ರು-ತ್ಥಾನ
ಪುನ-ರು-ತ್ಥಾ-ನ-ಕ್ಕಾಗಿ
ಪುನ-ರು-ತ್ಥಾ-ನದ
ಪುನ-ರು-ತ್ಥಾ-ನ-ವನ್ನು
ಪುನ-ರು-ತ್ಥಾ-ನ-ವಿಲ್ಲ
ಪುನ-ರು-ದ್ಧಾರ
ಪುನ-ರೂ-ರ್ಜಿ-ತ-ಗೊ-ಳಿ-ಸುವ
ಪುನ-ರ್ಜಾ-ಗೃ-ತಿಯ
ಪುನ-ರ್ನಿ-ರ್ಮಾಣ
ಪುನ-ರ್ನಿ-ರ್ಮಾ-ಣಕ್ಕೆ
ಪುನ-ರ್ರ-ಚಿಸ
ಪುನ-ರ್ರೂ-ಪಿ-ಸಿ-ಕೊಂ-ಡರು
ಪುನ-ಶ್ಚೇ-ತ-ನ-ಗೊ-ಳಿ-ಸ-ಬೇ-ಕಾ-ಗಿದೆ
ಪುನ-ಸ್ಸಂ-ಸ್ಥಾ-ಪನೆ
ಪುನ-ಸ್ಸಂ-ಸ್ಥಾ-ಪಿಸು
ಪುನೀ-ತ-ಗೊ-ಳಿ-ಸಿದ
ಪುನೀ-ತ-ನಾ-ಗ-ದಿ-ದ್ದಲ್ಲಿ
ಪುನೀ-ತ-ನಾ-ಗಲು
ಪುನೀ-ತ-ವಾದ
ಪುರ-ದಲ್ಲಿ
ಪುರ-ಭ-ವ-ನ-ದಲ್ಲಿ
ಪುರ-ಸಭೆ
ಪುರ-ಸ-ಭೆಯ
ಪುರ-ಸ-ಭೆಯು
ಪುರ-ಸ್ಕ-ರಿ-ಸಿ-ದಾಗ
ಪುರ-ಸ್ಕಾರ
ಪುರಾಣ
ಪುರಾ-ಣ-ಗಳ
ಪುರಾ-ಣ-ಗಳನ್ನೂ
ಪುರಾ-ಣ-ಗಳಲ್ಲಿ
ಪುರಾ-ಣ-ಗಳು
ಪುರಾ-ಣ-ವಾಗಿ
ಪುರಾ-ಣಾ-ದಿ-ಗಳನ್ನು
ಪುರಾ-ಣಾ-ಧಿ-ಷ್ಠಾ-ನ-ಅ-ರ್ಥಾತ್
ಪುರಾ-ತತ್ವ
ಪುರಾ-ತನ
ಪುರಾ-ತ-ನ-ವಾದ
ಪುರಾ-ತ-ನ-ವಾ-ದುದು
ಪುರಾವೆ
ಪುರಾ-ವೆ-ಗಳ
ಪುರಾ-ವೆ-ಯನ್ನು
ಪುರಾ-ವೆ-ಯೆಂ-ದರೆ
ಪುರುಷ
ಪುರು-ಷತ್ವ
ಪುರು-ಷ-ತ್ವ-ವನ್ನೇ
ಪುರು-ಷನ
ಪುರು-ಷ-ನಲ್ಲ
ಪುರು-ಷ-ನಾ-ಗುವೆ
ಪುರು-ಷರ
ಪುರು-ಷರು
ಪುರು-ಷರೂ
ಪುರು-ಷ-ರೊಂ-ದಿಗೆ
ಪುರು-ಷ-ರೊ-ಳ-ಗಿನ
ಪುರು-ಷರೋ
ಪುರು-ಷ-ಸಿಂಹ
ಪುರು-ಷ-ಸಿಂ-ಹ-ರ-ನ್ನಾಗಿ
ಪುರು-ಷ-ಸಿಂ-ಹ-ರನ್ನು
ಪುರು-ಷ-ಸಿಂ-ಹ-ರಾಗಿ
ಪುರು-ಷ-ಸಿಂ-ಹರು
ಪುರು-ಷಾ-ರ್ಥ-ವಾ-ದ್ದ-ರಿಂದ
ಪುರೋ-ಗಾಮಿ
ಪುರೋ-ಭಿ-ವೃ-ದ್ಧಿ-ಗಾಗಿ
ಪುರೋ-ಹಿ-ತರ
ಪುಲಿ
ಪುಳಕ
ಪುಷಿ-ಗಳ
ಪುಷಿ-ಗ-ಳಿಗೂ
ಪುಷಿ-ಗಳು
ಪುಷಿ-ಗ-ಳೆಲ್ಲ
ಪುಷಿ-ಗಳೇ
ಪುಷಿ-ಗಳೋ
ಪುಷಿ-ಪ-ದವಿ
ಪುಷಿ-ಮು-ನಿ-ಗ-ಳಿಗೆ
ಪುಷಿ-ಮು-ನಿ-ಗಳು
ಪುಷಿ-ಮು-ನಿ-ಯೋಗಿ
ಪುಷಿಯೂ
ಪುಷಿಯೇ
ಪುಷಿ-ಶ್ರೇ-ಷ್ಠರ
ಪುಷಿ-ಸಂ-ತಾ-ನರು
ಪುಷಿ-ಸ-ದೃಶ
ಪುಷ್ಕ-ರಿ-ಣಿ-ಯೊಂ-ದ-ರಲ್ಲಿ
ಪುಷ್ಕ-ಳ-ವಾಗಿ
ಪುಷ್ಟಿ
ಪುಷ್ಟಿ-ಕೊ-ಡು-ವಂ-ತಿ-ದ್ದುವು
ಪುಷ್ಪ-ಬಿ-ಲ್ವ-ಪ-ತ್ರ-ಗಳಿಂದ
ಪುಷ್ಪ-ಗಳ
ಪುಷ್ಪ-ಗಳಿಂದ
ಪುಷ್ಪ-ಮಾ-ಲೆ-ಗಂ-ಧ-ಚಂ-ದ-ನ-ಸು-ಗಂ-ಧ-ದ್ರ-ವ್ಯ-ಗಳನ್ನು
ಪುಷ್ಪ-ವನ್ನು
ಪುಷ್ಪ-ವೃ-ಷ್ಟಿ-ಯಾ-ಗು-ತ್ತಲೇ
ಪುಷ್ಪ-ವೃ-ಷ್ಟಿ-ಯಾ-ಯಿತು
ಪುಷ್ಪ-ಹಾ-ರ-ಗಳನ್ನು
ಪುಷ್ಪ-ಹಾ-ರ-ಗಳಿಂದ
ಪುಷ್ಪ-ಹಾ-ರ-ವನ್ನು
ಪುಷ್ಪಾಂ-ಜ-ಲಿ-ಯ-ನ್ನ-ರ್ಪಿಸಿ
ಪುಷ್ಪಾ-ದಿ-ಗಳನ್ನು
ಪುಷ್ಪಾ-ದಿ-ಗಳಿಂದ
ಪುಸ್ತಕ
ಪುಸ್ತ-ಕ-ಕ್ಕಿ-ಳಿ-ಸಿದ
ಪುಸ್ತ-ಕ-ಗಳ
ಪುಸ್ತ-ಕ-ಗಳನ್ನು
ಪುಸ್ತ-ಕ-ಗಳಲ್ಲಿ
ಪುಸ್ತ-ಕ-ಗಳಿಂದ
ಪುಸ್ತ-ಕ-ಗಳು
ಪುಸ್ತ-ಕ-ಗಳೂ
ಪುಸ್ತ-ಕ-ಜ್ಞಾನ
ಪುಸ್ತ-ಕ-ಜ್ಞಾ-ನವೂ
ಪುಸ್ತ-ಕದ
ಪುಸ್ತ-ಕ-ದಲ್ಲಿ
ಪುಸ್ತ-ಕ-ವನ್ನು
ಪುಸ್ತ-ಕ-ವನ್ನೂ
ಪುಸ್ತ-ಕ-ವಾ-ಗಿತ್ತು
ಪುಸ್ತ-ಕ-ವಿ-ದ್ಯೆ-ಅ-ಕ್ಷ-ರ-ವಿ-ದ್ಯೆ-ಗ-ಳೇ-ನೇನೂ
ಪುಸ್ತ-ಕ-ವೊಂ-ದನ್ನು
ಪುಸ್ತ-ಕ-ವೊಂ-ದ-ರಲ್ಲಿ
ಪೂಜಕ
ಪೂಜ-ಕ-ನಾ-ಗಿ-ರು-ತ್ತಾನೆ
ಪೂಜ-ನೀಯ
ಪೂಜಾ
ಪೂಜಾ-ಕಾ-ರ್ಯದ
ಪೂಜಾ-ಗೃಹ
ಪೂಜಾ-ಗೃ-ಹಕ್ಕೆ
ಪೂಜಾ-ಗೃ-ಹ-ಗಳೂ
ಪೂಜಾ-ಗೃ-ಹದ
ಪೂಜಾ-ಗೃ-ಹ-ದಲ್ಲಿ
ಪೂಜಾ-ಗೃ-ಹದಿ
ಪೂಜಾ-ಗೃ-ಹ-ವನ್ನು
ಪೂಜಾ-ಗೃ-ಹ-ವನ್ನೂ
ಪೂಜಾ-ಗೃ-ಹ-ವನ್ನೇ
ಪೂಜಾದಿ
ಪೂಜಾ-ದಿ-ಗಳ
ಪೂಜಾ-ದಿ-ಗಳನ್ನು
ಪೂಜಾ-ದಿ-ಗಳನ್ನೆಲ್ಲ
ಪೂಜಾ-ದಿ-ಗಳಲ್ಲಿ
ಪೂಜಾ-ದಿ-ಗಳು
ಪೂಜಾ-ದಿ-ಗ-ಳೆಲ್ಲ
ಪೂಜಾ-ಫ-ಲ-ವನ್ನು
ಪೂಜಾ-ಭಾ-ವ-ದಿಂದ
ಪೂಜಾ-ಮಂ-ದಿ-ರ-ವನ್ನು
ಪೂಜಾ-ರೂ-ಪದ
ಪೂಜಾ-ರ್ಹನೇ
ಪೂಜಾ-ವಿ-ಧಿ-ಗ-ಳಲ್ಲೂ
ಪೂಜಾ-ವಿ-ಧಿ-ಗ-ಳಿ-ಗಾಗಿ
ಪೂಜಾ-ಸ-ನದ
ಪೂಜಾ-ಸಾ-ಮ-ಗ್ರಿ-ಗಳು
ಪೂಜಾ-ಸ್ಥಾ-ನಕ್ಕೆ
ಪೂಜಿ-ಸ-ಬೇಕು
ಪೂಜಿ-ಸ-ಬೇ-ಕೆಂ-ದಿ-ದ್ದೇನೆ
ಪೂಜಿ-ಸ-ಲಾ-ಗು-ತ್ತಿತ್ತು
ಪೂಜಿ-ಸ-ಲಾ-ಗು-ತ್ತಿ-ದೆಯೋ
ಪೂಜಿ-ಸ-ಲಾ-ರಂ-ಭಿ-ಸಿ-ದ್ದರು
ಪೂಜಿ-ಸಲು
ಪೂಜಿ-ಸ-ಲ್ಪ-ಟ್ಟಿ-ದ್ದಾರೆ
ಪೂಜಿ-ಸ-ಲ್ಪ-ಡ-ಲಿ-ದ್ದಾರೆ
ಪೂಜಿ-ಸ-ಲ್ಪ-ಡು-ತ್ತಾರೆ
ಪೂಜಿ-ಸ-ಲ್ಪ-ಡು-ತ್ತಿದ್ದ
ಪೂಜಿ-ಸ-ಲ್ಪ-ಡುವೆ
ಪೂಜಿಸಿ
ಪೂಜಿ-ಸಿ-ದರು
ಪೂಜಿ-ಸಿ-ದರೆ
ಪೂಜಿ-ಸಿ-ದ-ವ-ರ-ಲ್ಲವೆ
ಪೂಜಿ-ಸಿ-ದ್ದಾರೆ
ಪೂಜಿಸು
ಪೂಜಿ-ಸು-ತ್ತಿ-ದ್ದರು
ಪೂಜಿ-ಸು-ತ್ತಿ-ದ್ದಾರೆ
ಪೂಜಿ-ಸು-ತ್ತಿ-ರು-ವಾಗ
ಪೂಜಿ-ಸುವ
ಪೂಜಿ-ಸು-ವಂ-ತಾ-ಗಲು
ಪೂಜಿ-ಸು-ವಷ್ಟು
ಪೂಜಿ-ಸು-ವು-ದ-ರಿಂದ
ಪೂಜಿ-ಸು-ವು-ದೊಂದು
ಪೂಜೆ
ಪೂಜೆ-ಜ-ಪ-ಗಳನ್ನು
ಪೂಜೆ-ಪಿಂ-ಡ-ಪ್ರ-ದಾನ
ಪೂಜೆ-ಗಾಗಿ
ಪೂಜೆ-ಗೀಗ
ಪೂಜೆಗೂ
ಪೂಜೆಗೆ
ಪೂಜೆ-ಗೆಂದು
ಪೂಜೆ-ಗೈ-ದರು
ಪೂಜೆಯ
ಪೂಜೆ-ಯನ್ನು
ಪೂಜೆ-ಯ-ನ್ನು-ಅದೂ
ಪೂಜೆ-ಯಲ್ಲೇ
ಪೂಜೆ-ಯಾ-ದಂ-ತಾಗು
ಪೂಜೆಯು
ಪೂಜೆಯೂ
ಪೂಜೆ-ಯೆಂದು
ಪೂಜೆಯೇ
ಪೂಜೆ-ಯೊಂ-ದಿಗೆ
ಪೂಜ್ಯ
ಪೂಜ್ಯತೆ
ಪೂಜ್ಯ-ತೆ-ಗೌ-ರ-ವ
ಪೂಜ್ಯ-ತೆ-ಯಿಂದ
ಪೂಜ್ಯ-ಬು-ದ್ಧಿ-ಯಾ-ಗಲಿ
ಪೂಜ್ಯ-ಭಾವ
ಪೂಜ್ಯ-ಭಾ-ವ-ದಿಂದ
ಪೂಜ್ಯ-ಭಾ-ವ-ವ-ನ್ನಿ-ಟ್ಟು-ಕೊಂ-ಡಿ-ದ್ದರು
ಪೂಜ್ಯ-ಭಾ-ವ-ವನ್ನು
ಪೂಜ್ಯ-ರಾದ
ಪೂಣ-ಚಂದ್ರ
ಪೂನಾ
ಪೂರ-ಕ-ವಾ-ಗ-ಬಲ್ಲ
ಪೂರ-ಕ-ವಾಗಿ
ಪೂರಿ
ಪೂರಿತ
ಪೂರೈಕೆ
ಪೂರೈ-ಸಲು
ಪೂರೈಸಿ
ಪೂರೈ-ಸಿ-ಕೊ-ಟ್ಟದ್ದು
ಪೂರೈ-ಸುವ
ಪೂರೈ-ಸು-ವು-ದ-ಕ್ಕಾಗಿ
ಪೂರೈ-ಸು-ವುದೇ
ಪೂರ್ಣ
ಪೂರ್ಣ-ಕಾಂ-ತಿ-ಯಿಂದ
ಪೂರ್ಣ-ಕಾ-ಲ-ವನ್ನು
ಪೂರ್ಣ-ಕುಂಭ
ಪೂರ್ಣಕ್ಕೆ
ಪೂರ್ಣ-ಗೊ-ಳಿ-ಸ-ಲೆಂದು
ಪೂರ್ಣ-ಗೊ-ಳಿ-ಸಿ-ದರು
ಪೂರ್ಣ-ಗೊ-ಳಿ-ಸುವ
ಪೂರ್ಣ-ಗೊ-ಳಿ-ಸು-ವು-ದೆಂಬ
ಪೂರ್ಣ-ತೆಯೇ
ಪೂರ್ಣ-ವಲ್ಲ
ಪೂರ್ಣ-ವಾಗಿ
ಪೂರ್ಣ-ವಾ-ಗಿತ್ತು
ಪೂರ್ಣ-ವಾ-ಗು-ತ್ತದೆ
ಪೂರ್ಣ-ವಾ-ಗು-ವಲ್ಲಿ
ಪೂರ್ಣ-ವಾದ
ಪೂರ್ಣ-ವಾಯ್ತು
ಪೂರ್ಣ-ವಿ-ರಾಮ
ಪೂರ್ಣ-ವೇ-ಗ-ದಿಂದ
ಪೂರ್ಣಿಮೆ
ಪೂರ್ತಿ
ಪೂರ್ತಿ-ಯಾಗಿ
ಪೂರ್ವ
ಪೂರ್ವ-ಪ-ಶ್ಚಿ-ಮ-ಗ-ಳೆ-ರಡೂ
ಪೂರ್ವ-ಕ-ವಾದ
ಪೂರ್ವ-ಕಾ-ಲದ
ಪೂರ್ವ-ಕ್ಕಿ-ರುವ
ಪೂರ್ವ-ಗ್ರಹ
ಪೂರ್ವ-ಗ್ರ-ಹ-ಗಳನ್ನು
ಪೂರ್ವ-ಗ್ರ-ಹ-ಗ-ಳಿಗೆ
ಪೂರ್ವ-ಗ್ರ-ಹ-ಪೀ-ಡಿ-ತರು
ಪೂರ್ವ-ಗ್ರ-ಹ-ಪೀ-ಡಿ-ತ-ವಾ-ಗಿ-ರ-ಲಿ-ಲ್ಲವೋ
ಪೂರ್ವ-ಜ-ನ್ಮ-ಗಳ
ಪೂರ್ವ-ಜ-ನ್ಮದ
ಪೂರ್ವ-ಜ-ನ್ಮ-ದಲ್ಲಿ
ಪೂರ್ವ-ಜ-ನ್ಮ-ವನ್ನೂ
ಪೂರ್ವ-ಜರ
ಪೂರ್ವ-ಜ-ರಿಂದ
ಪೂರ್ವ-ತ-ಯಾ-ರಿಯೂ
ಪೂರ್ವ-ದಲ್ಲಿ
ಪೂರ್ವ-ದಿಂದ
ಪೂರ್ವ-ಪ-ಶ್ಚಿ-ಮ-ಗಳು
ಪೂರ್ವ-ಬಂ-ಗಾ-ಳಕ್ಕೆ
ಪೂರ್ವ-ಬಂ-ಗಾ-ಳದ
ಪೂರ್ವ-ಬಂ-ಗಾ-ಳ-ದಿಂದ
ಪೂರ್ವ-ಬಂ-ಗಾ-ಳ-ದಿಂ-ದಲೂ
ಪೂರ್ವ-ಭಾ-ವಿ-ಯಾಗಿ
ಪೂರ್ವ-ಮೀ-ಮಾಂಸೆ
ಪೂರ್ವಾ-ಪ-ರ-ಗ-ಳ-ನ್ನ-ರಿ-ಯದೆ
ಪೂರ್ವಿ-ಕರ
ಪೂರ್ವಿ-ಕರು
ಪೃಥ್ವಿ-ಯ-ನ್ನೆಲ್ಲ
ಪೆಟ್ಟಿಗೆ
ಪೆಟ್ಟಿ-ಗೆಯ
ಪೆಟ್ಟು
ಪೆಟ್ಟು-ಗಳು
ಪೆನ್ನಿ-ನಿಂದ
ಪೆರು-ಮಾ-ಳರ
ಪೆರು-ಮಾ-ಳ-ರಿಗೆ
ಪೆರು-ಮಾ-ಳರು
ಪೆರು-ಮಾ-ಳರೇ
ಪೆರು-ಮಾಳ್
ಪೆಸಿ-ಫಿಕ್
ಪೇಚಿ-ಗೀ-ಡಾ-ಗ-ಬೇ-ಕಾ-ಗು-ತ್ತಿತ್ತು
ಪೇಟ-ಗಳನ್ನು
ಪೇಟ-ದಂತೆ
ಪೇಟ-ವನ್ನು
ಪೇದೆ-ಗ-ಳಿ-ಬ್ಬರೂ
ಪೇನ್ಸ್
ಪೇರಿ-ಸು-ತ್ತಲೇ
ಪೈಕಿ
ಪೈಗಂ-ಬರ್
ಪೈಥಾ-ಗೊ-ರಸ್
ಪೈರಿ-ನೊಂ-ದಿಗೇ
ಪೈರು
ಪೊದೆ
ಪೊನ್ನಂ-ಪೇಟೆ
ಪೊರೆ-ಯನ್ನು
ಪೊರೆ-ಯಲಿ
ಪೊಳ್ಳು
ಪೊಳ್ಳು-ತ-ನದ
ಪೊಳ್ಳು-ತ-ನ-ವನ್ನು
ಪೋಕ್
ಪೋಗೋಸ್
ಪೋಪರು
ಪೋರ್ಟ್
ಪೋರ್ಟ್ಸೆಡ್
ಪೋಲಿ-ಸರ
ಪೋಲಿ-ಸ-ರನ್ನು
ಪೋಲಿ-ಸರು
ಪೋಲೀ-ಸರು
ಪೋಲೀಸು
ಪೋಷ-ಕ-ವಾದ
ಪೋಷಾ-ಕಿ-ನಲ್ಲಿ
ಪೌಂಡು
ಪೌಂಡು-ಗಳನ್ನು
ಪೌಜ್ದಾರೀ
ಪೌರ-ವ-ನಿಂದ
ಪೌರುಷ
ಪೌರು-ಷ-ದಿಂದ
ಪೌರು-ಷ-ವಂ-ತ-ರಾದ
ಪೌರ್ವಾತ್ಯ
ಪೌರ್ವಾ-ತ್ಯ-ಧ್ವ-ಜ-ವನ್ನು
ಪೌರ್ವಾ-ತ್ಯರ
ಪೌರ್ವಾ-ತ್ಯ-ರಾ-ಗಿ-ದ್ದಾ-ರೆಯೇ
ಪೌರ್ವಾ-ತ್ಯರು
ಪೌರ್ವಾ-ತ್ಯರೂ
ಪೌರ್ಹಾ-ಪಾನಿ
ಪೌರ್ಹಾ-ಪಾ-ನಿಗೆ
ಪೌರ್ಹಾ-ಪಾ-ನಿ-ಯನ್ನು
ಪೌಷ್ಟಿಕ
ಪೌಷ್ಟಿ-ಕಾ-ಹಾ-ರ-ವನ್ನು
ಪ್ತೋತ್ಸಾಹಿ
ಪ್ಯಾಟ-ರ್ಸನ್
ಪ್ಯಾಟ-ರ್ಸ-ನ್ನರು
ಪ್ಯಾಟ್ರಿಕ್
ಪ್ಯಾರಿ-ಸಿಗೆ
ಪ್ಯಾರಿ-ಸಿನ
ಪ್ಯಾರಿ-ಸಿ-ನಲ್ಲಿ
ಪ್ಯಾರಿ-ಸಿ-ನಿಂದ
ಪ್ಯಾರಿಸ್
ಪ್ಯಾರಿ-ಸ್ಸನ್ನು
ಪ್ಯಾಲೆ-ಸ್ಟೀ-ನಿ-ನಲ್ಲಿ
ಪ್ಯಾಲ್ಟೀನ್
ಪ್ಯಾಸ್ಟನ್
ಪ್ರಕಟ
ಪ್ರಕ-ಟ-ಗೊಂ-ಡಿತು
ಪ್ರಕ-ಟ-ಗೊಂಡು
ಪ್ರಕ-ಟ-ಗೊ-ಳಿಸಿ
ಪ್ರಕ-ಟ-ಗೊ-ಳಿ-ಸಿದ್ದೇ
ಪ್ರಕ-ಟ-ಣೆಗೆ
ಪ್ರಕ-ಟ-ಣೆಯ
ಪ್ರಕ-ಟ-ಣೆ-ಯನ್ನು
ಪ್ರಕ-ಟ-ಣೆ-ಯಲ್ಲಿ
ಪ್ರಕ-ಟನ
ಪ್ರಕ-ಟ-ಪ-ಡಿ-ಸಿ-ದಳು
ಪ್ರಕ-ಟ-ಪ-ಡಿ-ಸುತ್ತ
ಪ್ರಕ-ಟ-ವಾ-ಗ-ತೊ-ಡ-ಗಿ-ದ್ದುವು
ಪ್ರಕ-ಟ-ವಾ-ಗಿತ್ತು
ಪ್ರಕ-ಟ-ವಾ-ಗಿದ್ದ
ಪ್ರಕ-ಟ-ವಾ-ಗಿ-ರು-ವುದು
ಪ್ರಕ-ಟ-ವಾ-ಗಿವೆ
ಪ್ರಕ-ಟ-ವಾ-ಗು-ತ್ತಿತ್ತು
ಪ್ರಕ-ಟ-ವಾ-ಗು-ತ್ತಿದೆ
ಪ್ರಕ-ಟ-ವಾ-ಗು-ತ್ತಿಲ್ಲ
ಪ್ರಕ-ಟ-ವಾ-ಗುವ
ಪ್ರಕ-ಟ-ವಾದ
ಪ್ರಕ-ಟ-ವಾ-ದದ್ದು
ಪ್ರಕ-ಟ-ವಾ-ದಾಗ
ಪ್ರಕ-ಟ-ವಾ-ದುವು
ಪ್ರಕ-ಟ-ವಾ-ಯಿತು
ಪ್ರಕ-ಟ-ವಾ-ಯಿ-ತು-ಶೂದ್ರ
ಪ್ರಕ-ಟಿ-ಸ-ಲಾ-ಗಿದ್ದು
ಪ್ರಕ-ಟಿ-ಸ-ಲಾ-ಯಿತು
ಪ್ರಕ-ಟಿ-ಸಲು
ಪ್ರಕ-ಟಿಸಿ
ಪ್ರಕ-ಟಿ-ಸಿತು
ಪ್ರಕ-ಟಿ-ಸಿ-ದರು
ಪ್ರಕ-ಟಿ-ಸಿ-ದುವು
ಪ್ರಕ-ಟಿ-ಸಿ-ದ್ದರು
ಪ್ರಕ-ಟಿ-ಸಿ-ರು-ವಂತೆ
ಪ್ರಕ-ಟಿಸು
ಪ್ರಕ-ಟಿ-ಸುತ್ತ
ಪ್ರಕ-ಟಿ-ಸು-ತ್ತದೆ
ಪ್ರಕಾಂಡ
ಪ್ರಕಾರ
ಪ್ರಕಾ-ರ-ನಿ-ನಗೆ
ಪ್ರಕಾ-ರವೂ
ಪ್ರಕಾ-ರವೇ
ಪ್ರಕಾಶ
ಪ್ರಕಾ-ಶ-ಕರ
ಪ್ರಕಾ-ಶ-ಕ-ರಿಗೆ
ಪ್ರಕಾ-ಶ-ಕರು
ಪ್ರಕಾ-ಶಕ್ಕೆ
ಪ್ರಕಾ-ಶ-ದಿಂದ
ಪ್ರಕಾ-ಶನ
ಪ್ರಕಾ-ಶ-ನಕ್ಕೆ
ಪ್ರಕಾ-ಶ-ನದ
ಪ್ರಕಾ-ಶ-ಮಾ-ನ-ನಾ-ಗಿ-ದ್ದಾನೆ
ಪ್ರಕಾ-ಶ-ಮಾ-ನ-ವಾ-ಗಿತ್ತು
ಪ್ರಕಾ-ಶ-ಮಾ-ನ-ವಾ-ದದ್ದು
ಪ್ರಕಾ-ಶ-ವನ್ನು
ಪ್ರಕಾ-ಶಾ-ನಂದ
ಪ್ರಕಾ-ಶಾ-ನಂ-ದರ
ಪ್ರಕಾ-ಶಾ-ನಂ-ದ-ರನ್ನು
ಪ್ರಕಾ-ಶಾ-ನಂ-ದರು
ಪ್ರಕಾ-ಶಾ-ನಂ-ದರೂ
ಪ್ರಕಾ-ಶಾ-ನಂ-ದರೇ
ಪ್ರಕಾ-ಶಿ-ಸ-ತೊ-ಡ-ಗಿತು
ಪ್ರಕಾ-ಶಿ-ಸು-ತ್ತಿತ್ತು
ಪ್ರಕಾ-ಶಿ-ಸು-ತ್ತಿದೆ
ಪ್ರಕಾ-ಶಿ-ಸು-ವುದನ್ನು
ಪ್ರಕೃತಿ
ಪ್ರಕೃ-ತಿಗೆ
ಪ್ರಕೃ-ತಿ-ದೇವಿ
ಪ್ರಕೃ-ತಿ-ನಿ-ಯ-ಮ-ಗ-ಳೊಂ-ದಿಗೆ
ಪ್ರಕೃ-ತಿಯ
ಪ್ರಕೃ-ತಿ-ಯನ್ನು
ಪ್ರಕೃ-ತಿ-ಯಿಂದ
ಪ್ರಕೃ-ತಿ-ಯೊಂ-ದಿಗೆ
ಪ್ರಕೃ-ತಿ-ವಿ-ಹಿತ
ಪ್ರಕೃ-ತಿ-ಶ-ಕ್ತಿಯೇ
ಪ್ರಕೃ-ತಿ-ಸೌಂ-ದ-ರ್ಯ-ವನ್ನು
ಪ್ರಕ್ರಿ-ಯೆ-ಗಳ
ಪ್ರಕ್ರಿ-ಯೆ-ಗಳನ್ನು
ಪ್ರಕ್ಷುಬ್ಧ
ಪ್ರಕ್ಷು-ಬ್ಧ-ಗೊಂಡ
ಪ್ರಖರ
ಪ್ರಖ-ರ-ವಾದ
ಪ್ರಗತಿ
ಪ್ರಗ-ತಿ-ಕಾ-ರ್ಯದ
ಪ್ರಗ-ತಿ-ಗಾಗಿ
ಪ್ರಗ-ತಿಗೆ
ಪ್ರಗ-ತಿ-ಪರ
ಪ್ರಗ-ತಿಯ
ಪ್ರಗ-ತಿ-ಯನ್ನು
ಪ್ರಗ-ತಿ-ಯಾ-ಗ-ಬೇ-ಕಾ-ದರೆ
ಪ್ರಗ-ತಿ-ಯಾ-ಗಿ-ದ್ದು-ದನ್ನು
ಪ್ರಗ-ತಿಯು
ಪ್ರಗ-ತಿ-ಶೀಲ
ಪ್ರಚಂಡ
ಪ್ರಚಂ-ಡ-ವಾದ
ಪ್ರಚಂ-ಡ-ವಾ-ದುದು
ಪ್ರಚ-ಲಿತ
ಪ್ರಚ-ಲಿ-ತ-ಗೊ-ಳಿ-ಸಲು
ಪ್ರಚ-ಲಿ-ತ-ವಾ-ಗಿದ್ದ
ಪ್ರಚ-ಲಿ-ತ-ವಾ-ಗಿವೆ
ಪ್ರಚ-ಲಿ-ತ-ವಾ-ಗು-ತ್ತಿದ್ದ
ಪ್ರಚ-ಲಿ-ತ-ವಿದ್ದ
ಪ್ರಚಾರ
ಪ್ರಚಾ-ರಕ
ಪ್ರಚಾ-ರ-ಕರು
ಪ್ರಚಾ-ರ-ಕಾರ್ಯ
ಪ್ರಚಾ-ರ-ಕಾ-ರ್ಯ-ಗಳಲ್ಲಿ
ಪ್ರಚಾ-ರ-ಕ್ಕಾಗಿ
ಪ್ರಚಾ-ರಕ್ಕೆ
ಪ್ರಚಾ-ರ-ಗಳ
ಪ್ರಚಾ-ರ-ವಾ-ಗಿತ್ತು
ಪ್ರಚಾ-ರ-ವಾ-ಗಿ-ರ-ಲಿಲ್ಲ
ಪ್ರಚಾ-ರವೂ
ಪ್ರಚಾ-ರ-ವೆಲ್ಲ
ಪ್ರಚೋ-ದ-ಕ-ವಾದ
ಪ್ರಚೋ-ದನೆ
ಪ್ರಚೋ-ದಿ-ಸಿ-ದರು
ಪ್ರಚೋ-ದಿ-ಸಿ-ದ-ರು-ನಿ-ಮಗೆ
ಪ್ರಚೋ-ದಿ-ಸಿ-ದರೆ
ಪ್ರಚೋ-ದಿ-ಸಿ-ದ್ದಾನೆ
ಪ್ರಚೋ-ದಿ-ಸುತ್ತ
ಪ್ರಚೋ-ದಿ-ಸು-ತ್ತಿ-ದ್ದಾರೆ
ಪ್ರಚೋ-ದಿ-ಸು-ತ್ತಿ-ದ್ದುವು
ಪ್ರಚೋ-ದಿ-ಸು-ವಂ-ತಿ-ರ-ಬೇಕು
ಪ್ರಜಾ-ಪ್ರ-ಭುತ್ವ
ಪ್ರಜಾ-ಪ್ರ-ಭು-ತ್ವ-ಗಳನ್ನು
ಪ್ರಜಾ-ಪ್ರ-ಭು-ತ್ವದ
ಪ್ರಜೆ-ಗ-ಳ-ವ-ರೆಗೆ
ಪ್ರಜೆ-ಗ-ಳಾದ
ಪ್ರಜೆ-ಗಳಿ
ಪ್ರಜೆ-ಗಳೂ
ಪ್ರಜೆ-ಗ-ಳೆ-ಲ್ಲ-ರಿಗೂ
ಪ್ರಜೆಗೂ
ಪ್ರಜ್ಞಾನ
ಪ್ರಜ್ಞೆ
ಪ್ರಜ್ಞೆ-ಯನ್ನು
ಪ್ರಜ್ಞೆ-ಯ-ಲ್ಲಿ-ದ್ದಾ-ಗಲೂ
ಪ್ರಜ್ಞೆ-ಯು-ಳ್ಳ-ವರು
ಪ್ರಜ್ಞೆಯೂ
ಪ್ರಜ್ಞೆಯೇ
ಪ್ರಜ್ವ-ಲ-ಗೊ-ಳಿ-ಸ-ಬೇಕು
ಪ್ರಜ್ವ-ಲ-ನ-ಗೊ-ಳಿ-ಸು-ತ್ತಿ-ದ್ದರು
ಪ್ರಜ್ವ-ಲಿಸ
ಪ್ರಜ್ವ-ಲಿ-ಸ-ತೊ-ಡ-ಗಿ-ದುವು
ಪ್ರಜ್ವ-ಲಿಸಿ
ಪ್ರಜ್ವ-ಲಿಸು
ಪ್ರಜ್ವ-ಲಿ-ಸು-ತ್ತಿತ್ತು
ಪ್ರಜ್ವ-ಲಿ-ಸು-ತ್ತಿದ್ದ
ಪ್ರಜ್ವ-ಲಿ-ಸುವ
ಪ್ರಜ್ವ-ಲಿ-ಸು-ವಂತೆ
ಪ್ರಣಾಮ
ಪ್ರಣಾ-ಮ-ಗಳು
ಪ್ರಣಾ-ಮ-ಗೈ-ಯು-ತ್ತಿ-ದ್ದರು
ಪ್ರಣಾ-ಮ-ವ-ನ್ನ-ರ್ಪಿ-ಸಿ-ದರು
ಪ್ರಣೀ-ತ-ವಾದ
ಪ್ರತಾಪ
ಪ್ರತಾ-ಪ-ಸಿಂಗ್
ಪ್ರತಿ
ಪ್ರತಿ-ಕ್ರಿ-ಯಿ-ಸ-ಬ-ಹು-ದೆಂದು
ಪ್ರತಿ-ಕ್ರಿಯೆ
ಪ್ರತಿ-ಕ್ರಿ-ಯೆ-ಅ-ವರು
ಪ್ರತಿ-ಕ್ರಿ-ಯೆ-ಗಳ
ಪ್ರತಿ-ಕ್ರಿ-ಯೆ-ಯೇನು
ಪ್ರತಿ-ಕ್ಷ-ಣ-ದಲ್ಲೂ
ಪ್ರತಿ-ಗಳನ್ನು
ಪ್ರತಿ-ಗಳು
ಪ್ರತಿಜ್ಞೆ
ಪ್ರತಿ-ದಿನ
ಪ್ರತಿ-ದಿ-ನದ
ಪ್ರತಿ-ದಿ-ನವೂ
ಪ್ರತಿ-ದಿ-ನ-ವೆಂ-ಬಂತೆ
ಪ್ರತಿ-ಧ್ವನಿ
ಪ್ರತಿ-ಧ್ವ-ನಿ-ಯಷ್ಟೇ
ಪ್ರತಿ-ಧ್ವ-ನಿ-ಸಿತು
ಪ್ರತಿ-ಧ್ವ-ನಿ-ಸಿ-ದುವು
ಪ್ರತಿ-ಧ್ವ-ನಿ-ಸು-ತ್ತಿತ್ತು
ಪ್ರತಿ-ಧ್ವ-ನಿ-ಸು-ತ್ತಿದೆ
ಪ್ರತಿ-ಧ್ವ-ನಿ-ಸು-ವಂ-ತಾ-ಗ-ಬೇಕು
ಪ್ರತಿ-ನ-ಮ-ಸ್ಕಾರ
ಪ್ರತಿ-ನಿಧಿ
ಪ್ರತಿ-ನಿ-ಧಿ-ಗಳ
ಪ್ರತಿ-ನಿ-ಧಿ-ಗಳು
ಪ್ರತಿ-ನಿ-ಧಿ-ಗಳೂ
ಪ್ರತಿ-ನಿ-ಧಿ-ಯ-ನ್ನಾಗಿ
ಪ್ರತಿ-ನಿ-ಧಿ-ಯಾಗಿ
ಪ್ರತಿ-ನಿ-ಧಿ-ಯಾ-ಗಿ-ರು-ತ್ತಾನೆ
ಪ್ರತಿ-ನಿ-ಧಿ-ಯಾದ
ಪ್ರತಿ-ನಿ-ಧಿ-ಯಾ-ದರೆ
ಪ್ರತಿ-ನಿ-ಧಿ-ಯೆಂದೇ
ಪ್ರತಿ-ನಿ-ಧಿಯೇ
ಪ್ರತಿ-ನಿ-ಧಿ-ಸು-ತ್ತಿ-ದ್ದರೆ
ಪ್ರತಿ-ನಿ-ಧಿ-ಸು-ತ್ತಿ-ದ್ದಾರೆ
ಪ್ರತಿ-ನಿ-ಧಿ-ಸುವ
ಪ್ರತಿ-ನಿ-ಧಿ-ಸು-ವಂತೆ
ಪ್ರತಿ-ನಿ-ಧಿ-ಸು-ವು-ದಿಲ್ಲ
ಪ್ರತಿ-ಪಾ-ದ-ಕ-ರಾ-ಗಿ-ದ್ದರೂ
ಪ್ರತಿ-ಪಾ-ದಿ-ಸ-ಬೇ-ಕಾ-ಗಿಲ್ಲ
ಪ್ರತಿ-ಪಾ-ದಿ-ಸ-ಬೇಕು
ಪ್ರತಿ-ಪಾ-ದಿ-ಸಿ-ದರು
ಪ್ರತಿ-ಪಾ-ದಿ-ಸುವ
ಪ್ರತಿ-ಫಲ
ಪ್ರತಿ-ಫ-ಲ-ನ-ವಷ್ಟೆ
ಪ್ರತಿ-ಫ-ಲಿ-ಸು-ತ್ತಿತ್ತು
ಪ್ರತಿ-ಫ-ಲಿ-ಸು-ವ-ರೆಗೂ
ಪ್ರತಿ-ಫ-ಲಿ-ಸು-ವು-ದಿಲ್ಲ
ಪ್ರತಿ-ಬಂ-ಧ-ಕವೇ
ಪ್ರತಿ-ಬಿಂಬ
ಪ್ರತಿ-ಬಿಂ-ಬಿ-ಸು-ತ್ತದೆ
ಪ್ರತಿ-ಬಿಂ-ಬಿ-ಸು-ತ್ತಿತ್ತು
ಪ್ರತಿ-ಬಿಂ-ಬಿ-ಸುವ
ಪ್ರತಿ-ಬಿಂ-ಬಿ-ಸು-ವಂ-ತಿತ್ತು
ಪ್ರತಿ-ಭ-ಟ-ನೆಯ
ಪ್ರತಿ-ಭ-ಟಿಸಿ
ಪ್ರತಿ-ಭ-ಟಿ-ಸಿ-ದರು
ಪ್ರತಿ-ಭ-ಟಿ-ಸು-ತ್ತಿ-ದ್ದರು
ಪ್ರತಿ-ಭ-ಟಿ-ಸು-ವ-ವರ
ಪ್ರತಿ-ಭ-ಟಿ-ಸು-ವುದು
ಪ್ರತಿ-ಭ-ಟಿ-ಸು-ವುದೇ
ಪ್ರತಿಭಾ
ಪ್ರತಿ-ಭಾ-ನ್ವಿತ
ಪ್ರತಿ-ಭಾ-ನ್ವಿ-ತ-ರಿಗೆ
ಪ್ರತಿ-ಭಾ-ನ್ವಿ-ತರು
ಪ್ರತಿ-ಭಾ-ನ್ವಿತೆ
ಪ್ರತಿ-ಭಾ-ಪೂರ್ಣ
ಪ್ರತಿ-ಭಾ-ವಂತ
ಪ್ರತಿ-ಭೆ-ಯನ್ನು
ಪ್ರತಿ-ಭೆಯು
ಪ್ರತಿ-ಭೆ-ಯುಳ್ಳ
ಪ್ರತಿ-ಮಾ-ಸ-ಹಿ-ತ-ವಾಗಿ
ಪ್ರತಿಮೆ
ಪ್ರತಿ-ಮೆ-ಚಿ-ತ್ರ-ಸಂ-ಕೇ-ತ-ಗಳ
ಪ್ರತಿ-ಮೆಯ
ಪ್ರತಿ-ಮೆ-ಯ-ನ್ನಿಟ್ಟು
ಪ್ರತಿ-ಮೆ-ಯನ್ನು
ಪ್ರತಿ-ಮೆ-ಯಲ್ಲಿ
ಪ್ರತಿಯ
ಪ್ರತಿ-ಯನ್ನು
ಪ್ರತಿ-ಯಾಗಿ
ಪ್ರತಿಯೊ
ಪ್ರತಿ-ಯೊಂ-ದನ್ನೂ
ಪ್ರತಿ-ಯೊಂದು
ಪ್ರತಿ-ಯೊಂದೂ
ಪ್ರತಿ-ಯೊಬ್ಬ
ಪ್ರತಿ-ಯೊ-ಬ್ಬನ
ಪ್ರತಿ-ಯೊ-ಬ್ಬ-ನನ್ನೂ
ಪ್ರತಿ-ಯೊ-ಬ್ಬ-ನಲ್ಲೂ
ಪ್ರತಿ-ಯೊ-ಬ್ಬ-ನಿಗೂ
ಪ್ರತಿ-ಯೊ-ಬ್ಬನೂ
ಪ್ರತಿ-ಯೊ-ಬ್ಬ-ನೂ-ಪ್ರ-ತಿ-ಯೊಂದೂ
ಪ್ರತಿ-ಯೊ-ಬ್ಬರ
ಪ್ರತಿ-ಯೊ-ಬ್ಬ-ರನ್ನೂ
ಪ್ರತಿ-ಯೊ-ಬ್ಬ-ರಿಗೂ
ಪ್ರತಿ-ಯೊ-ಬ್ಬರೂ
ಪ್ರತಿ-ಯೋರ್ವ
ಪ್ರತಿ-ರೋ-ಧ-ವನ್ನು
ಪ್ರತಿ-ರೋ-ಧ-ವನ್ನೇ
ಪ್ರತಿ-ವರ್ಷ
ಪ್ರತಿ-ವಾದಿ
ಪ್ರತಿ-ವಾ-ದಿಯ
ಪ್ರತಿ-ಷ್ಠಾ-ಪನಾ
ಪ್ರತಿ-ಷ್ಠಾ-ಪನೆ
ಪ್ರತಿ-ಷ್ಠಾ-ಪ-ನೆಗೆ
ಪ್ರತಿ-ಷ್ಠಾ-ಪ-ನೆಯ
ಪ್ರತಿ-ಷ್ಠಾ-ಪಿ-ಸ-ಲಾ-ಯಿತು
ಪ್ರತಿ-ಷ್ಠಾ-ಪಿ-ಸಲು
ಪ್ರತಿ-ಷ್ಠಾ-ಪಿಸಿ
ಪ್ರತಿ-ಷ್ಠಾ-ಪಿ-ಸಿದ
ಪ್ರತಿ-ಷ್ಠಾ-ಪಿ-ಸಿ-ದರೆ
ಪ್ರತಿ-ಷ್ಠಾ-ಪಿ-ಸು-ಮ-ರದ
ಪ್ರತಿ-ಷ್ಠಾ-ಪಿ-ಸುವ
ಪ್ರತಿ-ಷ್ಠಿ-ತರು
ಪ್ರತಿ-ಷ್ಠಿ-ತ-ವಾ-ಗಿ-ರು-ವುದು
ಪ್ರತಿ-ಸ-ಲವೂ
ಪ್ರತಿ-ಸ್ಪಂ-ದಿ-ಸು-ವಂತೆ
ಪ್ರತೀಕ
ಪ್ರತೀ-ಕವೇ
ಪ್ರತೀ-ಕ್ಷಿಸು
ಪ್ರತೀ-ತಿ-ಯಿದೆ
ಪ್ರತ್ಯಕ್ಷ
ಪ್ರತ್ಯ-ಕ್ಷ-ದ-ರ್ಶಿ-ಗಳು
ಪ್ರತ್ಯ-ಕ್ಷ-ವಾಗಿ
ಪ್ರತ್ಯ-ಕ್ಷ-ವಾ-ಗಿಯೋ
ಪ್ರತ್ಯೇಕ
ಪ್ರತ್ಯೇ-ಕ-ವಾಗಿ
ಪ್ರತ್ಯೇ-ಕ-ವಾದ
ಪ್ರತ್ಯೇ-ಕ-ವೆಂದು
ಪ್ರತ್ಯೇ-ಕಿಸಿ
ಪ್ರತ್ಯೇ-ಕಿ-ಸಿದ
ಪ್ರತ್ಯೇ-ಕಿ-ಸುವ
ಪ್ರಥಮ
ಪ್ರಥ-ಮತಃ
ಪ್ರದ-ಕ್ಷಿಣೆ
ಪ್ರದ-ಕ್ಷಿ-ಣೆ-ಗಳನ್ನು
ಪ್ರದರ್
ಪ್ರದ-ರ್ಶನ
ಪ್ರದ-ರ್ಶ-ನ-ಇ-ವೆಲ್ಲ
ಪ್ರದ-ರ್ಶ-ನಕ್ಕೆ
ಪ್ರದ-ರ್ಶ-ನ-ಗಳು
ಪ್ರದ-ರ್ಶ-ನದ
ಪ್ರದ-ರ್ಶ-ನ-ವನ್ನು
ಪ್ರದ-ರ್ಶ-ನ-ವೊಂ-ದಕ್ಕೆ
ಪ್ರದ-ರ್ಶ-ನ-ವೊಂದು
ಪ್ರದ-ರ್ಶ-ನಾ-ಲ-ಯ-ಗಳನ್ನು
ಪ್ರದ-ರ್ಶಿ-ಸಲು
ಪ್ರದ-ರ್ಶಿ-ಸುವ
ಪ್ರದೇಶ
ಪ್ರದೇ-ಶಕ್ಕೆ
ಪ್ರದೇ-ಶ-ಗಳನ್ನು
ಪ್ರದೇ-ಶ-ಗಳಲ್ಲಿ
ಪ್ರದೇ-ಶ-ಗಳು
ಪ್ರದೇ-ಶ-ಗಳೂ
ಪ್ರದೇ-ಶದ
ಪ್ರದೇ-ಶ-ದತ್ತ
ಪ್ರದೇ-ಶ-ದಲ್ಲಿ
ಪ್ರದೇ-ಶ-ದ-ಲ್ಲಿ-ದ್ದೇವೆ
ಪ್ರದೇ-ಶ-ದಲ್ಲೆಲ್ಲ
ಪ್ರದೇ-ಶ-ದ-ವನು
ಪ್ರದೇ-ಶ-ದಿಂದ
ಪ್ರದೇ-ಶ-ದಿಂ-ದಲೇ
ಪ್ರದೇ-ಶ-ವನ್ನು
ಪ್ರದೇ-ಶ-ವಾದ
ಪ್ರದೇ-ಶವೇ
ಪ್ರಧಾನ
ಪ್ರಧಾ-ನ-ವಾಗಿ
ಪ್ರಧಾ-ನ-ವಾದ
ಪ್ರಧಾ-ನ-ವಾ-ದದ್ದು
ಪ್ರಪಂಚ
ಪ್ರಪಂ-ಚಕ್ಕೆ
ಪ್ರಪಂ-ಚ-ಕ್ಕೇನು
ಪ್ರಪಂ-ಚದ
ಪ್ರಪಂ-ಚ-ದತ್ತ
ಪ್ರಪಂ-ಚ-ದಲ್ಲಿ
ಪ್ರಪಂ-ಚ-ದಲ್ಲೆಲ್ಲ
ಪ್ರಪಂ-ಚ-ದಲ್ಲೇ
ಪ್ರಪಂ-ಚ-ದಾ-ದ್ಯಂತ
ಪ್ರಪಂ-ಚ-ದಿಂದ
ಪ್ರಪಂ-ಚ-ದೆ-ಲ್ಲೆ-ಡೆ-ಯಲ್ಲಿ
ಪ್ರಪಂ-ಚ-ದೊ-ಳಗೆ
ಪ್ರಪಂ-ಚ-ವನ್ನು
ಪ್ರಪಂ-ಚ-ವನ್ನೇ
ಪ್ರಪಂ-ಚವೇ
ಪ್ರಪಂ-ಚವೋ
ಪ್ರಪಾ-ತ-ದೆ-ಡೆಗೆ
ಪ್ರಪ್ರ-ಥಮ
ಪ್ರಬಲ
ಪ್ರಬ-ಲ-ಪ-ರಿ-ಶುದ್ಧ
ಪ್ರಬ-ಲರೋ
ಪ್ರಬ-ಲ-ವಾಗಿ
ಪ್ರಬ-ಲ-ವಾ-ಗಿತ್ತು
ಪ್ರಬ-ಲ-ವಾ-ಗಿಯೇ
ಪ್ರಬ-ಲ-ವಾ-ದದ್ದು
ಪ್ರಬ-ಲ-ವಾ-ದುದು
ಪ್ರಬುದ್ಧ
ಪ್ರಬುದ್ಧಾ
ಪ್ರಭ-ವಾ-ನಂ-ದ-ರಿಂದ
ಪ್ರಭಾ
ಪ್ರಭಾವ
ಪ್ರಭಾ-ವ-ಕ್ಕ-ನು-ಗು-ಣ-ವಾಗಿ
ಪ್ರಭಾ-ವ-ಕ್ಕಾಗಿ
ಪ್ರಭಾ-ವಕ್ಕೆ
ಪ್ರಭಾ-ವ-ಕ್ಕೊಳ
ಪ್ರಭಾ-ವದ
ಪ್ರಭಾ-ವ-ಪೂ-ರ್ಣವೂ
ಪ್ರಭಾ-ವ-ಯು-ತ-ವಾ-ಗಿತ್ತು
ಪ್ರಭಾ-ವ-ವನ್ನು
ಪ್ರಭಾ-ವ-ವ-ಳಿ-ಯದ
ಪ್ರಭಾ-ವವು
ಪ್ರಭಾ-ವ-ವೆಂ-ತ-ಹ-ದೆಂಬ
ಪ್ರಭಾ-ವ-ಶಾಲಿ
ಪ್ರಭಾ-ವ-ಶಾ-ಲಿ-ಯಾಗಿ
ಪ್ರಭಾ-ವ-ಶಾ-ಲಿ-ಯಾ-ಗಿತ್ತು
ಪ್ರಭಾ-ವ-ಶಾ-ಲಿ-ಯಾ-ಗಿವೆ
ಪ್ರಭಾ-ವ-ಶಾಲೀ
ಪ್ರಭಾ-ವಿತ
ಪ್ರಭಾ-ವಿ-ತ-ಗೊ-ಳಿ-ಸಿ-ದ-ರೆಂ-ಬು-ದನ್ನು
ಪ್ರಭಾ-ವಿ-ತ-ನಾ-ಗಿದ್ದ
ಪ್ರಭಾ-ವಿ-ತ-ನಾದ
ಪ್ರಭಾ-ವಿ-ತ-ನಾ-ದ-ನೆಂ-ದರೆ
ಪ್ರಭಾ-ವಿ-ತ-ರಾ-ಗ-ದಿ-ರಲು
ಪ್ರಭಾ-ವಿ-ತ-ರಾ-ಗಲೇ
ಪ್ರಭಾ-ವಿ-ತ-ರಾಗಿ
ಪ್ರಭಾ-ವಿ-ತ-ರಾ-ಗಿದ್ದ
ಪ್ರಭಾ-ವಿ-ತ-ರಾ-ಗಿ-ದ್ದರು
ಪ್ರಭಾ-ವಿ-ತ-ರಾದ
ಪ್ರಭಾ-ವಿ-ತ-ರಾ-ದ-ರ-ಲ್ಲದೆ
ಪ್ರಭಾ-ವಿ-ತ-ರಾ-ದ-ರೆಂ-ಬುದು
ಪ್ರಭಾ-ವಿ-ತ-ಳಾ-ಗಿ-ದ್ದಳು
ಪ್ರಭಾ-ವಿ-ತ-ಳಾ-ಗಿ-ದ್ದ-ಳೆಂ-ದರೆ
ಪ್ರಭಾ-ವಿ-ತ-ಳಾದ
ಪ್ರಭಾ-ವಿ-ತ-ವಾ-ಗಿದೆ
ಪ್ರಭು
ಪ್ರಭುತ್ವ
ಪ್ರಭು-ತ್ವ-ವಿದೆ
ಪ್ರಭು-ಭೋ-ಜನ
ಪ್ರಭು-ರಾ-ಷ್ಟ್ರ-ದವ
ಪ್ರಭು-ವಾದ
ಪ್ರಭು-ವಿನ
ಪ್ರಭುವೆ
ಪ್ರಭೆ
ಪ್ರಭೆ-ಯನು
ಪ್ರಭೆ-ಯನ್ನೇ
ಪ್ರಭೆ-ಯಲ್ಲಿ
ಪ್ರಭೆ-ಯಿಂದ
ಪ್ರಭೇ-ದ-ಗ-ಳಿ-ರು-ವು-ದ-ರತ್ತ
ಪ್ರಭೇ-ದದ
ಪ್ರಮ-ದ-ದಾಸ
ಪ್ರಮಾಣ
ಪ್ರಮಾ-ಣಕ್ಕೆ
ಪ್ರಮಾ-ಣ-ಗ್ರಂ-ಥ-ವಾದ
ಪ್ರಮಾ-ಣ-ದಲ್ಲಿ
ಪ್ರಮಾ-ಣ-ದ್ದಾ-ಗಿ-ರಲಿ
ಪ್ರಮಾ-ಣ-ವಾ-ಕ್ಯ-ಗ-ಳಲ್ಲ
ಪ್ರಮಾ-ಣ-ವಾಗಿ
ಪ್ರಮಾ-ಣವು
ಪ್ರಮಾ-ಣ-ವೆಂದು
ಪ್ರಮಾ-ಣ-ಸಿದ್ಧ
ಪ್ರಮುಖ
ಪ್ರಮು-ಖ-ರ-ನ್ನೆಲ್ಲ
ಪ್ರಮು-ಖ-ರಲ್ಲಿ
ಪ್ರಮು-ಖ-ರಿಂದ
ಪ್ರಮು-ಖ-ರೊಂ-ದಿಗೆ
ಪ್ರಮು-ಖ-ವಾದ
ಪ್ರಮು-ಖ-ವಾ-ದ-ದ್ದೆಂಬ
ಪ್ರಮು-ಖ-ವಾ-ದವು
ಪ್ರಮೋ-ದ-ಗಳ
ಪ್ರಯತ್ನ
ಪ್ರಯ-ತ್ನ-ಗಳೂ
ಪ್ರಯ-ತ್ನ-ದಲ್ಲಿ
ಪ್ರಯ-ತ್ನ-ಪ-ಟ್ಟರು
ಪ್ರಯ-ತ್ನ-ಪ-ಟ್ಟರೂ
ಪ್ರಯ-ತ್ನ-ಪಟ್ಟು
ಪ್ರಯ-ತ್ನ-ಪ-ಡ-ಲಿ-ಲ್ಲವೆ
ಪ್ರಯ-ತ್ನ-ಪಡಿ
ಪ್ರಯ-ತ್ನ-ಪೂ-ರ್ವ-ಕ-ವಾಗಿ
ಪ್ರಯ-ತ್ನ-ಮಾ-ಡದ
ಪ್ರಯ-ತ್ನ-ವನ್ನು
ಪ್ರಯ-ತ್ನ-ವ-ನ್ನೆಲ್ಲ
ಪ್ರಯ-ತ್ನ-ವನ್ನೇ
ಪ್ರಯ-ತ್ನವೂ
ಪ್ರಯತ್ನಿ
ಪ್ರಯ-ತ್ನಿಸ
ಪ್ರಯ-ತ್ನಿ-ಸ-ದಿ-ರು-ವುದೇ
ಪ್ರಯ-ತ್ನಿ-ಸ-ಬ-ಲ್ಲಿ-ರೇನು
ಪ್ರಯ-ತ್ನಿ-ಸ-ಬಾ-ರದೆ
ಪ್ರಯ-ತ್ನಿ-ಸ-ಬೇ-ಕಾ-ದದ್ದು
ಪ್ರಯ-ತ್ನಿ-ಸ-ಬೇಕು
ಪ್ರಯ-ತ್ನಿ-ಸ-ಬೇಡಿ
ಪ್ರಯ-ತ್ನಿ-ಸ-ಲಿಲ್ಲ
ಪ್ರಯ-ತ್ನಿಸಿ
ಪ್ರಯ-ತ್ನಿ-ಸಿತ್ತು
ಪ್ರಯ-ತ್ನಿ-ಸಿದ
ಪ್ರಯ-ತ್ನಿ-ಸಿ-ದ-ರಾ-ದರೂ
ಪ್ರಯ-ತ್ನಿ-ಸಿ-ದರು
ಪ್ರಯ-ತ್ನಿ-ಸಿ-ದರೂ
ಪ್ರಯ-ತ್ನಿ-ಸಿ-ದರೆ
ಪ್ರಯ-ತ್ನಿ-ಸಿ-ದ-ವ-ರಲ್ಲ
ಪ್ರಯ-ತ್ನಿ-ಸಿ-ದೆ-ನಷ್ಟೇ
ಪ್ರಯ-ತ್ನಿ-ಸಿ-ದ್ದೇನೆ
ಪ್ರಯ-ತ್ನಿಸು
ಪ್ರಯ-ತ್ನಿ-ಸು-ತ್ತಾರೆ
ಪ್ರಯ-ತ್ನಿ-ಸು-ತ್ತಾರೋ
ಪ್ರಯ-ತ್ನಿ-ಸು-ತ್ತಿದ್ದ
ಪ್ರಯ-ತ್ನಿ-ಸು-ತ್ತಿ-ದ್ದೇನೆ
ಪ್ರಯ-ತ್ನಿ-ಸು-ತ್ತೇನೆ
ಪ್ರಯ-ತ್ನಿ-ಸು-ತ್ತೇವೆ
ಪ್ರಯ-ತ್ನಿ-ಸು-ವ-ವ-ರನ್ನು
ಪ್ರಯಾ
ಪ್ರಯಾಣ
ಪ್ರಯಾ-ಣ-ಭಾ-ಷ-ಣ-ಸಂ-ಭಾ-ಷ-ಣೆ-ಗಳಿಂದ
ಪ್ರಯಾ-ಣ-ಕಾ-ಲ-ಕ್ಕೆಂದು
ಪ್ರಯಾ-ಣ-ಕಾ-ಲ-ದಲ್ಲಿ
ಪ್ರಯಾ-ಣ-ಕಾ-ಲವು
ಪ್ರಯಾ-ಣಕ್ಕೆ
ಪ್ರಯಾ-ಣ-ಗಳ
ಪ್ರಯಾ-ಣದ
ಪ್ರಯಾ-ಣ-ದಲ್ಲಿ
ಪ್ರಯಾ-ಣ-ದಿಂದ
ಪ್ರಯಾ-ಣ-ದಿಂ-ದಾಗಿ
ಪ್ರಯಾ-ಣ-ದೊಂ-ದಿಗೆ
ಪ್ರಯಾ-ಣ-ವನ್ನು
ಪ್ರಯಾ-ಣವು
ಪ್ರಯಾ-ಣ-ಸು-ಮಾರು
ಪ್ರಯಾ-ಣಿಕ
ಪ್ರಯಾ-ಣಿ-ಕರ
ಪ್ರಯಾ-ಣಿ-ಕ-ರನ್ನು
ಪ್ರಯಾ-ಣಿ-ಕ-ರಿಗೋ
ಪ್ರಯಾ-ಣಿ-ಕರು
ಪ್ರಯಾ-ಣಿ-ಕ-ರೆಲ್ಲ
ಪ್ರಯಾ-ಣಿ-ಕ-ರೆ-ಲ್ಲರೂ
ಪ್ರಯಾ-ಣಿಸಿ
ಪ್ರಯಾ-ಣಿ-ಸಿ-ದರು
ಪ್ರಯಾ-ಣಿ-ಸು-ತ್ತಿ-ದ್ದರು
ಪ್ರಯಾಸ
ಪ್ರಯಾ-ಸ-ಕರ
ಪ್ರಯಾ-ಸ-ಕ-ರ-ವಾ-ಗಿತ್ತು
ಪ್ರಯಾ-ಸ-ಕಾರಿ
ಪ್ರಯಾ-ಸ-ದಿಂದ
ಪ್ರಯಾ-ಸ-ವಾಗಿ
ಪ್ರಯಾ-ಸ-ವಾ-ಗಿತ್ತು
ಪ್ರಯು-ಕ್ತ-ವಾ-ಗಿಯೇ
ಪ್ರಯೋ
ಪ್ರಯೋಗ
ಪ್ರಯೋ-ಗಿ-ಸ-ಬೇ-ಕಾ-ಯಿ-ತೆಂ-ಬು-ದನ್ನು
ಪ್ರಯೋ-ಗಿ-ಸು-ತ್ತಿ-ದ್ದರು
ಪ್ರಯೋ-ಗಿ-ಸುವ
ಪ್ರಯೋ-ಗಿ-ಸು-ವು-ದಾಗಿ
ಪ್ರಯೋ-ಜನ
ಪ್ರಯೋ-ಜ-ನಕ್ಕೆ
ಪ್ರಯೋ-ಜ-ನ-ವನ್ನೂ
ಪ್ರಯೋ-ಜ-ನ-ವಾಗ
ಪ್ರಯೋ-ಜ-ನ-ವಾ-ಗ-ಲಾ-ರದು
ಪ್ರಯೋ-ಜ-ನ-ವಾ-ಗ-ಲಿಲ್ಲ
ಪ್ರಯೋ-ಜ-ನ-ವಾ-ಗಿತ್ತು
ಪ್ರಯೋ-ಜ-ನ-ವಾ-ಗಿದೆ
ಪ್ರಯೋ-ಜ-ನ-ವಾ-ಗಿ-ದೆ-ಯೆಂದು
ಪ್ರಯೋ-ಜ-ನ-ವಾ-ಗಿ-ರ-ಲಿಲ್ಲ
ಪ್ರಯೋ-ಜ-ನ-ವಾ-ದಂ-ತಾ-ಯಿತು
ಪ್ರಯೋ-ಜ-ನ-ವಾ-ದಂತೆ
ಪ್ರಯೋ-ಜ-ನ-ವಾ-ದೀತು
ಪ್ರಯೋ-ಜ-ನ-ವಾ-ದೀ-ತೆಂದೂ
ಪ್ರಯೋ-ಜ-ನ-ವಾ-ಯಿತು
ಪ್ರಯೋ-ಜ-ನ-ವಿ-ದೆ-ಯೆಂದು
ಪ್ರಯೋ-ಜ-ನ-ವಿ-ರ-ಲಿಲ್ಲ
ಪ್ರಯೋ-ಜ-ನ-ವಿಲ್ಲ
ಪ್ರಯೋ-ಜ-ನವೂ
ಪ್ರಯೋ-ಜ-ನ-ವೇ-ನಾ-ಯಿತು
ಪ್ರಯೋ-ಜ-ನ-ವೇನು
ಪ್ರಲಾಪ
ಪ್ರಲಾ-ಪಿ-ಸು-ತ್ತಿ-ರುವ
ಪ್ರಲೋ-ಭ-ನೆಯ
ಪ್ರವ-ಚನ
ಪ್ರವ-ಚ-ನ-ಗಳನ್ನು
ಪ್ರವ-ಚ-ನ-ಗಳು
ಪ್ರವ-ಚ-ನ-ವನ್ನು
ಪ್ರವ-ಹಿ-ಸಿವೆ
ಪ್ರವ-ಹಿ-ಸು-ವಂತೆ
ಪ್ರವಾದಿ
ಪ್ರವಾ-ದಿ-ಗ-ಳಾಗಿ
ಪ್ರವಾ-ದಿ-ಗಳು
ಪ್ರವಾ-ದಿಯ
ಪ್ರವಾ-ದಿ-ಯಾ-ಗು-ತ್ತೇನೆ
ಪ್ರವಾ-ದಿಯೂ
ಪ್ರವಾಸ
ಪ್ರವಾ-ಸ-ಕ-ಥ-ನ-ವನ್ನು
ಪ್ರವಾ-ಸ-ಕಾ-ಲ-ದಲ್ಲಿ
ಪ್ರವಾ-ಸ-ಕ್ಕಾಗಿ
ಪ್ರವಾ-ಸಕ್ಕೆ
ಪ್ರವಾ-ಸ-ಗ-ಳಿಗೂ
ಪ್ರವಾ-ಸ-ಗ-ಳಿಗೆ
ಪ್ರವಾ-ಸ-ಗ-ಳೆಲ್ಲ
ಪ್ರವಾ-ಸದ
ಪ್ರವಾ-ಸ-ದಲ್ಲಿ
ಪ್ರವಾ-ಸ-ದ-ಲ್ಲಿ-ದ್ದರು
ಪ್ರವಾ-ಸ-ದ-ಲ್ಲಿದ್ದೆ
ಪ್ರವಾ-ಸ-ದಿಂದ
ಪ್ರವಾ-ಸ-ದಿಂ-ದಾಗಿ
ಪ್ರವಾ-ಸ-ಪ್ರಿ-ಯರು
ಪ್ರವಾ-ಸ-ವನ್ನು
ಪ್ರವಾ-ಸವು
ಪ್ರವಾಸಿ
ಪ್ರವಾ-ಸಿ-ಗರ
ಪ್ರವಾಸೀ
ಪ್ರವಾಹ
ಪ್ರವಾ-ಹಕ್ಕೆ
ಪ್ರವಾ-ಹ-ಗಳು
ಪ್ರವಾ-ಹದ
ಪ್ರವಾ-ಹ-ದಂತೆ
ಪ್ರವಾ-ಹ-ದಲ್ಲಿ
ಪ್ರವಾ-ಹ-ದಲ್ಲೂ
ಪ್ರವಾ-ಹ-ದಿಂದ
ಪ್ರವಾ-ಹ-ದೋ-ಪಾ-ದಿ-ಯಲ್ಲಿ
ಪ್ರವಾ-ಹ-ವನ್ನು
ಪ್ರವಾ-ಹವು
ಪ್ರವಾ-ಹವೂ
ಪ್ರವಾ-ಹಿನಿ
ಪ್ರವಿ-ಚ-ಲಂತಿ
ಪ್ರವೀ-ಣ-ನಾದ
ಪ್ರವೃ-ತ್ತ-ರಾ-ಗು-ವಂತೆ
ಪ್ರವೃತ್ತಿ
ಪ್ರವೃ-ತ್ತಿ-ಗಳನ್ನು
ಪ್ರವೃ-ತ್ತಿ-ಯನ್ನು
ಪ್ರವೃ-ತ್ತಿ-ಯ-ವಳು
ಪ್ರವೃ-ತ್ತಿ-ಯಾ-ಗದೇ
ಪ್ರವೇಶ
ಪ್ರವೇ-ಶ-ಗ-ಳಾ-ದಾಗ
ಪ್ರವೇ-ಶ-ದಿಂ-ದಾಗಿ
ಪ್ರವೇ-ಶ-ವಿ-ಲ್ಲದ
ಪ್ರವೇ-ಶ-ಶು-ಲ್ಕ-ವಿ-ದ್ದಿ-ತಾ-ದರೂ
ಪ್ರವೇ-ಶಾ-ವ-ಕಾಶ
ಪ್ರವೇ-ಶಿ-ಸ-ದಂತೆ
ಪ್ರವೇ-ಶಿಸಿ
ಪ್ರವೇ-ಶಿ-ಸಿತು
ಪ್ರವೇ-ಶಿ-ಸಿದ
ಪ್ರವೇ-ಶಿ-ಸಿ-ದರು
ಪ್ರವೇ-ಶಿ-ಸಿ-ದರೆ
ಪ್ರವೇ-ಶಿ-ಸಿ-ದಾಗ
ಪ್ರವೇ-ಶಿ-ಸಿದೆ
ಪ್ರವೇ-ಶಿ-ಸಿ-ದೊ-ಡ-ನೆಯೇ
ಪ್ರವೇ-ಶಿ-ಸಿ-ದ್ದರೋ
ಪ್ರವೇ-ಶಿ-ಸಿದ್ದು
ಪ್ರವೇ-ಶಿ-ಸಿದ್ದೇ
ಪ್ರವೇ-ಶಿ-ಸಿ-ಬಿ-ಟ್ಟಳು
ಪ್ರವೇ-ಶಿ-ಸಿ-ಬಿ-ಟ್ಟಿತ್ತು
ಪ್ರವೇ-ಶಿ-ಸಿ-ರು-ವು-ದಿ-ಲ್ಲವೋ
ಪ್ರವೇ-ಶಿ-ಸು-ತ್ತವೆ
ಪ್ರವೇ-ಶಿ-ಸು-ತ್ತಿದ್ದ
ಪ್ರವೇ-ಶಿ-ಸು-ತ್ತಿ-ದ್ದಂತೆ
ಪ್ರವೇ-ಶಿ-ಸು-ತ್ತಿ-ದ್ದಂ-ತೆಯೇ
ಪ್ರವೇ-ಶಿ-ಸುವ
ಪ್ರವೇ-ಶಿ-ಸು-ವಂ-ತಾ-ಯಿತು
ಪ್ರವೇ-ಶಿ-ಸು-ವಂತೆ
ಪ್ರವೇ-ಶಿ-ಸು-ವಾಗ
ಪ್ರಶಂ-ಸಿಸಿ
ಪ್ರಶಂ-ಸಿ-ಸಿ-ದರು
ಪ್ರಶಂ-ಸೆಗೆ
ಪ್ರಶಂ-ಸೆಯ
ಪ್ರಶಸ್ತ
ಪ್ರಶಾಂತ
ಪ್ರಶಾಂ-ತ-ಪ್ರ-ಚಂಡ
ಪ್ರಶಾಂ-ತ-ವಾ-ಗಿತ್ತು
ಪ್ರಶಾಂ-ತ-ವಾದ
ಪ್ರಶ್ನ-ಪ-ತ್ರಿ-ಕೆ-ಗಳನ್ನು
ಪ್ರಶ್ನಾ-ತೀತ
ಪ್ರಶ್ನಾರ್ಥಿ
ಪ್ರಶ್ನಿ-ಸಲು
ಪ್ರಶ್ನಿ-ಸಿದ
ಪ್ರಶ್ನಿ-ಸಿ-ದರು
ಪ್ರಶ್ನಿ-ಸಿ-ದಾಗ
ಪ್ರಶ್ನಿ-ಸುತ್ತಿ
ಪ್ರಶ್ನೆ
ಪ್ರಶ್ನೆ-ಗಳ
ಪ್ರಶ್ನೆ-ಗಳನ್ನು
ಪ್ರಶ್ನೆ-ಗಳನ್ನೆಲ್ಲ
ಪ್ರಶ್ನೆ-ಗ-ಳಿಗೂ
ಪ್ರಶ್ನೆ-ಗ-ಳಿಗೆ
ಪ್ರಶ್ನೆ-ಗ-ಳಿ-ಗೆಲ್ಲ
ಪ್ರಶ್ನೆ-ಗಳು
ಪ್ರಶ್ನೆ-ಗ-ಳೆಲ್ಲ
ಪ್ರಶ್ನೆಗೆ
ಪ್ರಶ್ನೆಯ
ಪ್ರಶ್ನೆ-ಯ-ನ್ನಾ-ದರೂ
ಪ್ರಶ್ನೆ-ಯನ್ನು
ಪ್ರಶ್ನೆ-ಯನ್ನೇ
ಪ್ರಶ್ನೆ-ಯಲ್ಲಿ
ಪ್ರಶ್ನೆಯೇ
ಪ್ರಶ್ನೆ-ಯೇ-ನೆಂ-ದರೆ
ಪ್ರಶ್ನೆ-ಯೇನೋ
ಪ್ರಶ್ನೆ-ಯೊಂ-ದನ್ನು
ಪ್ರಶ್ನೆ-ಯೊಂದು
ಪ್ರಶ್ನೆಯೋ
ಪ್ರಶ್ನೋ-ತ್ತರ
ಪ್ರಶ್ನೋ-ತ್ತ-ರ-ಗಳ
ಪ್ರಶ್ನೋ-ತ್ತ-ರ-ಗಳು
ಪ್ರಸಂಗ
ಪ್ರಸಂ-ಗ-ಗಳನ್ನು
ಪ್ರಸಂ-ಗ-ಗಳೂ
ಪ್ರಸಂ-ಗ-ವನ್ನು
ಪ್ರಸನ್ನ
ಪ್ರಸ-ನ್ನ-ಗಂ-ಭೀರ
ಪ್ರಸ-ನ್ನ-ವಾ-ಯಿತು
ಪ್ರಸ-ವಿ-ಸ-ಲಾ-ರದು
ಪ್ರಸಾದ
ಪ್ರಸಾರ
ಪ್ರಸಾ-ರ-ಪ್ರ-ವಾ-ಸ-ಕಾರ್ಯ
ಪ್ರಸಾ-ರ-ಕಾ-ರ್ಯ-ಗಳನ್ನು
ಪ್ರಸಾ-ರ-ಕ್ಕಾಗಿ
ಪ್ರಸಾ-ರ-ಕ್ಕಾ-ಗಿ-ರುವ
ಪ್ರಸಾ-ರಕ್ಕೆ
ಪ್ರಸಾ-ರ-ಗೊ-ಳ್ಳು-ತ್ತವೆ
ಪ್ರಸಾ-ರ-ದಲ್ಲಿ
ಪ್ರಸಾ-ರ-ವಾ-ಗಿಯೇ
ಪ್ರಸಾ-ರ-ವಾ-ದು-ವ-ಲ್ಲದೆ
ಪ್ರಸಿದ್ದ
ಪ್ರಸಿದ್ಧ
ಪ್ರಸಿ-ದ್ಧ-ರಾಗಿ
ಪ್ರಸಿ-ದ್ಧ-ರಾ-ಗಿದ್ದ
ಪ್ರಸಿ-ದ್ಧ-ರಾದ
ಪ್ರಸಿ-ದ್ಧ-ರಾ-ದರು
ಪ್ರಸಿ-ದ್ಧ-ಳಾದ
ಪ್ರಸಿ-ದ್ಧ-ವಾ-ಗಿತ್ತು
ಪ್ರಸಿ-ದ್ಧ-ವಾ-ಗಿದೆ
ಪ್ರಸಿ-ದ್ಧ-ವಾದ
ಪ್ರಸಿ-ದ್ಧ-ವಾ-ಯಿತು
ಪ್ರಸಿ-ದ್ಧಿಗೆ
ಪ್ರಸ್ತಾಪ
ಪ್ರಸ್ತಾ-ಪಕ್ಕೆ
ಪ್ರಸ್ತಾ-ಪಿ-ಸದೆ
ಪ್ರಸ್ತಾ-ಪಿ-ಸ-ಬೇ-ಕಾ-ಯಿತು
ಪ್ರಸ್ತಾ-ಪಿ-ಸ-ಲಿಲ್ಲ
ಪ್ರಸ್ತಾ-ಪಿ-ಸಲೇ
ಪ್ರಸ್ತಾ-ಪಿ-ಸ-ಲೇ-ಬೇಡಿ
ಪ್ರಸ್ತಾ-ಪಿ-ಸ-ಲ್ಪ-ಟ್ಟದ್ದು
ಪ್ರಸ್ತಾ-ಪಿ-ಸ-ಲ್ಪ-ಡದ
ಪ್ರಸ್ತಾ-ಪಿಸಿ
ಪ್ರಸ್ತಾ-ಪಿ-ಸಿದ
ಪ್ರಸ್ತಾ-ಪಿ-ಸಿ-ದರು
ಪ್ರಸ್ತಾ-ಪಿ-ಸಿ-ದ-ರು-ಅ-ವರು
ಪ್ರಸ್ತಾ-ಪಿ-ಸಿ-ದ್ದರು
ಪ್ರಸ್ತಾ-ಪಿ-ಸಿ-ದ್ದಲ್ಲ
ಪ್ರಸ್ತಾ-ಪಿಸು
ಪ್ರಸ್ತಾ-ಪಿ-ಸುತ್ತ
ಪ್ರಸ್ತಾ-ವಿಸಿ
ಪ್ರಸ್ತುತ
ಪ್ರಸ್ತು-ತ-ಪ-ಡಿ-ಸಿದ
ಪ್ರಸ್ತು-ತ-ಪ-ಡಿ-ಸಿ-ದ್ದೇನೆ
ಪ್ರಹಾ-ರ-ವನ್ನು
ಪ್ರಹ್ಲಾ-ದನ
ಪ್ರಾಂಗ-ಣ-ದಲ್ಲಿ
ಪ್ರಾಂತ
ಪ್ರಾಂತಕ್ಕೆ
ಪ್ರಾಂತದ
ಪ್ರಾಂತ-ದಲ್ಲಿ
ಪ್ರಾಂತ-ದ-ಲ್ಲಿಯೂ
ಪ್ರಾಂತ್ಯ-ಗಳಲ್ಲಿ
ಪ್ರಾಂತ್ಯದ
ಪ್ರಾಂತ್ಯ-ದಲ್ಲಿ
ಪ್ರಾಂತ್ಯ-ದಲ್ಲೇ
ಪ್ರಾಂತ್ಯ-ದಿಂದ
ಪ್ರಾಕಾರ
ಪ್ರಾಚೀನ
ಪ್ರಾಚ್ಯ
ಪ್ರಾಚ್ಯ
ಪ್ರಾಚ್ಯ-ಪಾ-ಶ್ಚಾತ್ಯ
ಪ್ರಾಚ್ಯ-ವಸ್ತು
ಪ್ರಾಜ್ಞ-ನಲ್ಲ
ಪ್ರಾಟೆ-ಸ್ಟೆಂ-ಟರ
ಪ್ರಾಣ
ಪ್ರಾಣ-ತ್ಯಾಗ
ಪ್ರಾಣದ
ಪ್ರಾಣ-ಪ್ರಿಯ
ಪ್ರಾಣ-ಬಿಟ್ಟ
ಪ್ರಾಣ-ಬಿ-ಟ್ಟಳು
ಪ್ರಾಣ-ವಂತೂ
ಪ್ರಾಣ-ವನ್ನು
ಪ್ರಾಣ-ವನ್ನೇ
ಪ್ರಾಣ-ವಾ-ಗಿತ್ತು
ಪ್ರಾಣ-ವಾಯು
ಪ್ರಾಣ-ವಾ-ಯು-ವನ್ನು
ಪ್ರಾಣ-ಸ-ಮಾ-ನ-ರಾದ
ಪ್ರಾಣಾ
ಪ್ರಾಣಾ-ಪಾ-ಯ-ವುಂ-ಟಾ-ಗಿತ್ತು
ಪ್ರಾಣಾ-ಯಾಮ
ಪ್ರಾಣಾ-ಯಾ-ಮದ
ಪ್ರಾಣಿ
ಪ್ರಾಣಿ-ಗಳ
ಪ್ರಾಣಿ-ಗಳನ್ನು
ಪ್ರಾಣಿ-ಗ-ಳಿ-ಗಾಗಿ
ಪ್ರಾಣಿ-ಗಳು
ಪ್ರಾಣಿ-ಗ-ಳೊಂ-ದಿಗೆ
ಪ್ರಾಣಿ-ಜೀ-ವ-ನಕ್ಕೆ
ಪ್ರಾಣಿ-ಜೀ-ವ-ನ-ದಲ್ಲಿ
ಪ್ರಾಣಿ-ಧ-ರ್ಮ-ಕ್ಕ-ನು-ಸಾ-ರ-ವಾಗಿ
ಪ್ರಾಣಿ-ಪಕ್ಷಿ
ಪ್ರಾಣಿ-ಯೆಂ-ಬಂತೆ
ಪ್ರಾಣಿ-ವ-ನಕ್ಕೆ
ಪ್ರಾಣಿ-ವ-ನದ
ಪ್ರಾಣಿ-ಶಾಸ್ತ್ರ
ಪ್ರಾಣಿ-ಸ-ಹ-ಜ-ವಾದ
ಪ್ರಾಣಿ-ಸಾ-ಮ್ರಾಜ್ಯ
ಪ್ರಾತಃ-ಕಾಲ
ಪ್ರಾಧಾನ್ಯ
ಪ್ರಾಧಾ-ನ್ಯ-ವಿತ್ತು
ಪ್ರಾಧ್ಯಾ-ಪ-ಕ-ರಾಗಿ
ಪ್ರಾಧ್ಯಾ-ಪ-ಕರು
ಪ್ರಾಪಂ-ಚಿಕ
ಪ್ರಾಪಂ-ಚಿ-ಕತೆ
ಪ್ರಾಪಂ-ಚಿ-ಕ-ತೆಯ
ಪ್ರಾಪಂ-ಚಿ-ಕ-ತೆಯು
ಪ್ರಾಪಂ-ಚಿ-ಕರ
ಪ್ರಾಪಂ-ಚಿ-ಕ-ರಿ-ಗಿಂ-ತಲೂ
ಪ್ರಾಪಂ-ಚಿ-ಕ-ವಾ-ಗಿ-ರ-ಬ-ಹುದು
ಪ್ರಾಪ್ತ
ಪ್ರಾಪ್ತ-ವಾ-ಗಲಿ
ಪ್ರಾಪ್ತ-ವಾ-ಗು-ತ್ತದೆ
ಪ್ರಾಪ್ತ-ವಾ-ಗುವ
ಪ್ರಾಪ್ತ-ವಾದ
ಪ್ರಾಪ್ತಿಗೆ
ಪ್ರಾಪ್ತಿ-ಮಾ-ಡಿ-ಕೊಂ-ಡಂ-ತಹ
ಪ್ರಾಪ್ಯ
ಪ್ರಾಪ್ಯ-ವ-ರಾ-ನ್ನಿ-ಬೋ-ಧತ
ಪ್ರಾಬಲ್ಯ
ಪ್ರಾಮಾ-ಣಿಕ
ಪ್ರಾಮಾ-ಣಿ-ಕ-ಧೀರ
ಪ್ರಾಮಾ-ಣಿ-ಕತೆ
ಪ್ರಾಮಾ-ಣಿ-ಕ-ತೆಯ
ಪ್ರಾಮಾ-ಣಿ-ಕ-ನಾಗಿ
ಪ್ರಾಮಾ-ಣಿ-ಕ-ನಾ-ಗಿ-ದ್ದೇನೆ
ಪ್ರಾಮಾ-ಣಿ-ಕ-ಬುದ್ಧಿ
ಪ್ರಾಮಾ-ಣಿ-ಕ-ರಾಗಿ
ಪ್ರಾಮಾ-ಣಿ-ಕ-ರಾದ
ಪ್ರಾಮಾ-ಣಿ-ಕರು
ಪ್ರಾಮಾ-ಣಿ-ಕರೂ
ಪ್ರಾಮಾ-ಣಿ-ಕ-ಳಾ-ಗಿದ್ದ
ಪ್ರಾಮಾ-ಣಿ-ಕ-ವಾಗಿ
ಪ್ರಾಮಾ-ಣಿ-ಕ-ವಾದ
ಪ್ರಾಮುಖ್ಯ
ಪ್ರಾಮು-ಖ್ಯತೆ
ಪ್ರಾಮು-ಖ್ಯ-ವಾ-ದ-ದ್ದಾ-ಗಿತ್ತು
ಪ್ರಾಯ
ಪ್ರಾಯವು
ಪ್ರಾಯಶಃ
ಪ್ರಾಯ-ಶ್ಚಿತ್ತ
ಪ್ರಾಯ-ಶ್ಚಿ-ತ್ತಕ್ಕೆ
ಪ್ರಾಯ-ಶ್ಚಿ-ತ್ತ-ವಾ-ಗಿಯೋ
ಪ್ರಾಯ-ಶ್ಚಿ-ತ್ತ-ವೆಂ-ಬಂತೆ
ಪ್ರಾಯ-ಶ್ಚಿ-ತ್ತ-ವೇನು
ಪ್ರಾಯೋ
ಪ್ರಾರಂಭ
ಪ್ರಾರಂ-ಭ-ಬೆ-ಳ-ವ-ಣಿ-ಗೆ-ಗಳ
ಪ್ರಾರಂ-ಭದ
ಪ್ರಾರಂ-ಭ-ದಲ್ಲಿ
ಪ್ರಾರಂ-ಭ-ದಿಂ-ದಲೂ
ಪ್ರಾರಂ-ಭ-ವಾ-ಗ-ಬ-ಹು-ದೆಂದೂ
ಪ್ರಾರಂ-ಭ-ವಾ-ಗ-ಬೇ-ಕಿತ್ತು
ಪ್ರಾರಂ-ಭ-ವಾ-ಗ-ಲಿತ್ತು
ಪ್ರಾರಂ-ಭ-ವಾ-ಗಲೇ
ಪ್ರಾರಂ-ಭ-ವಾ-ಗಿತ್ತು
ಪ್ರಾರಂ-ಭ-ವಾ-ಗಿದೆ
ಪ್ರಾರಂ-ಭ-ವಾ-ಗಿದ್ದ
ಪ್ರಾರಂ-ಭ-ವಾ-ಗಿ-ದ್ದುವು
ಪ್ರಾರಂ-ಭ-ವಾ-ಗು-ತ್ತ-ವೆಂದು
ಪ್ರಾರಂ-ಭ-ವಾ-ಗು-ತ್ತಿತ್ತು
ಪ್ರಾರಂ-ಭ-ವಾ-ಗು-ತ್ತಿದೆ
ಪ್ರಾರಂ-ಭ-ವಾ-ಗುವ
ಪ್ರಾರಂ-ಭ-ವಾದ
ಪ್ರಾರಂ-ಭ-ವಾ-ದಂ-ತಿತ್ತು
ಪ್ರಾರಂ-ಭ-ವಾ-ದದ್ದು
ಪ್ರಾರಂ-ಭ-ವಾ-ದದ್ದೇ
ಪ್ರಾರಂ-ಭ-ವಾ-ದರೆ
ಪ್ರಾರಂ-ಭ-ವಾ-ದುವು
ಪ್ರಾರಂ-ಭ-ವಾ-ಯಿತು
ಪ್ರಾರಂ-ಭ-ವಾ-ಯಿ-ತು-ಆ-ಕೆಗೆ
ಪ್ರಾರಂ-ಭ-ವಾ-ಯಿ-ತೆಂ-ಬು-ದನ್ನು
ಪ್ರಾರಂ-ಭ-ವಾ-ಯಿತೋ
ಪ್ರಾರಂ-ಭ-ವೇನೋ
ಪ್ರಾರಂಭಿ
ಪ್ರಾರಂ-ಭಿಕ
ಪ್ರಾರಂ-ಭಿ-ಸ-ಬ-ಹುದು
ಪ್ರಾರಂ-ಭಿ-ಸ-ಬೇ-ಕಾ-ದು-ದರ
ಪ್ರಾರಂ-ಭಿ-ಸ-ಬೇಕು
ಪ್ರಾರಂ-ಭಿ-ಸ-ಬೇ-ಕೆಂಬ
ಪ್ರಾರಂ-ಭಿ-ಸ-ಲಾ-ಗಿ-ತ್ತಾ-ದರೂ
ಪ್ರಾರಂ-ಭಿ-ಸ-ಲಾ-ಗಿತ್ತು
ಪ್ರಾರಂ-ಭಿ-ಸ-ಲಾ-ಯಿತು
ಪ್ರಾರಂ-ಭಿ-ಸಲು
ಪ್ರಾರಂ-ಭಿಸಿ
ಪ್ರಾರಂ-ಭಿ-ಸಿದ
ಪ್ರಾರಂ-ಭಿ-ಸಿ-ದರು
ಪ್ರಾರಂ-ಭಿ-ಸಿ-ದ-ರು-ಇ-ದ್ದ-ಕ್ಕಿ-ದ್ದಂತೆ
ಪ್ರಾರಂ-ಭಿ-ಸಿ-ದುದು
ಪ್ರಾರಂ-ಭಿ-ಸಿ-ದೆಯೋ
ಪ್ರಾರಂ-ಭಿ-ಸಿ-ದ್ದನ್ನು
ಪ್ರಾರಂ-ಭಿ-ಸಿ-ದ್ದರು
ಪ್ರಾರಂ-ಭಿ-ಸಿ-ದ್ದ-ವರೂ
ಪ್ರಾರಂ-ಭಿ-ಸಿ-ಬಿ-ಟ್ಟರು
ಪ್ರಾರಂ-ಭಿ-ಸಿ-ರು-ವು-ದ-ರಿಂದ
ಪ್ರಾರಂ-ಭಿಸು
ಪ್ರಾರಂ-ಭಿ-ಸುವ
ಪ್ರಾರಂ-ಭಿ-ಸು-ವು-ದೇನೋ
ಪ್ರಾರಂ-ಭಿ-ಸೋಣ
ಪ್ರಾರಂ-ಭೋ-ತ್ಸವ
ಪ್ರಾರ-ಬ್ಧ-ಕ-ರ್ಮ-ವನ್ನು
ಪ್ರಾರ್ಥನಾ
ಪ್ರಾರ್ಥ-ನಾ-ಮಂ-ದಿ-ರದ
ಪ್ರಾರ್ಥನೆ
ಪ್ರಾರ್ಥ-ನೆ-ಗಳನ್ನೂ
ಪ್ರಾರ್ಥ-ನೆಯ
ಪ್ರಾರ್ಥ-ನೆ-ಯನ್ನು
ಪ್ರಾರ್ಥ-ನೆ-ಯಿಂದ
ಪ್ರಾರ್ಥ-ನೆ-ಯಿಂ-ದೇನೂ
ಪ್ರಾರ್ಥಿ-ಸಿ-ಕೊಂಡ
ಪ್ರಾರ್ಥಿ-ಸಿ-ಕೊಂ-ಡ-ದ್ದ-ರಿಂದ
ಪ್ರಾರ್ಥಿ-ಸಿ-ಕೊಂ-ಡಳು
ಪ್ರಾರ್ಥಿ-ಸಿ-ಕೊಂ-ಡಾಗ
ಪ್ರಾರ್ಥಿ-ಸಿ-ಕೊಂ-ಡಿ-ದ್ದರು
ಪ್ರಾರ್ಥಿ-ಸಿ-ಕೊ-ಳ್ಳು-ತ್ತಿ-ದ್ದರು
ಪ್ರಾರ್ಥಿ-ಸಿ-ದರು
ಪ್ರಾರ್ಥಿ-ಸುತ್ತ
ಪ್ರಾರ್ಥಿ-ಸು-ತ್ತೇನೆ
ಪ್ರಾರ್ಥಿ-ಸು-ವುದು
ಪ್ರಾರ್ಥಿ-ಸೋಣ
ಪ್ರಾಶಸ್ತ್ಯ
ಪ್ರಾಶ-ಸ್ತ್ಯ-ವನ್ನು
ಪ್ರಾಶ್ನಿ-ಕರ
ಪ್ರಾಶ್ನಿ-ಕ-ರಿಗೆ
ಪ್ರಾಶ್ನಿ-ಕ-ರೆ-ಲ್ಲರ
ಪ್ರಿನ್ಸ್
ಪ್ರಿಯ
ಪ್ರಿಯ-ತಮ
ಪ್ರಿಯ-ನಾಥ
ಪ್ರಿಯ-ನಾ-ಥ-ನೊಂ-ದಿಗೆ
ಪ್ರಿಯ-ರಾದ
ಪ್ರಿಯ-ರಾ-ದ-ವ-ರೊ-ಬ್ಬರು
ಪ್ರಿಯ-ಳಾದ
ಪ್ರಿಯ-ವಾ-ಗಿತ್ತು
ಪ್ರಿಯ-ವಾ-ಗಿ-ಸು-ತ್ತಿ-ದ್ದುದು
ಪ್ರಿಯ-ವಾದ
ಪ್ರಿಯ-ವಾ-ದ-ದ್ದಾ-ಗಿ-ರ-ಲಿಲ್ಲ
ಪ್ರಿಯ-ವಾ-ದದ್ದು
ಪ್ರಿಯ-ವಾ-ದು-ವಲ್ಲ
ಪ್ರಿಯ-ವಾ-ದುವು
ಪ್ರಿಯ-ಸಖಿ
ಪ್ರಿಯೆ
ಪ್ರೀತಿ
ಪ್ರೀತಿ-ಅ-ಭಿ-ಮಾನ
ಪ್ರೀತಿ-ಸ-ಹಾ-ನು-ಭೂ-ತಿ-ಕ-ಳ-ಕ-ಳಿ-ಗಳನ್ನು
ಪ್ರೀತಿ-ಸ-ಹಾ-ನು-ಭೂ-ತಿ-ಗಳಿಂದ
ಪ್ರೀತಿ-ಗಳನ್ನು
ಪ್ರೀತಿಗೆ
ಪ್ರೀತಿ-ಪಾ-ತ್ರ-ರಾದ
ಪ್ರೀತಿ-ಪೂ-ರ್ವಕ
ಪ್ರೀತಿ-ಪೂ-ರ್ವ-ಕ-ವಾಗಿ
ಪ್ರೀತಿಯ
ಪ್ರೀತಿ-ಯನ್ನು
ಪ್ರೀತಿ-ಯಿಂದ
ಪ್ರೀತಿ-ಯಿತ್ತು
ಪ್ರೀತಿಯು
ಪ್ರೀತಿ-ಯುತ
ಪ್ರೀತಿಯೂ
ಪ್ರೀತಿಯೇ
ಪ್ರೀತಿ-ಯೇನೂ
ಪ್ರೀತಿ-ಯೊಂದು
ಪ್ರೀತಿ-ವಿ-ಶ್ವಾ-ಸ-ಪಾ-ತ್ರ-ರಾದ
ಪ್ರೀತಿ-ಸ-ಬೇಕು
ಪ್ರೀತಿ-ಸ-ಲೆಂದೇ
ಪ್ರೀತಿ-ಸಿ-ದಂತೆ
ಪ್ರೀತಿ-ಸಿ-ದರೆ
ಪ್ರೀತಿಸು
ಪ್ರೀತಿ-ಸು-ತ್ತಾರೆ
ಪ್ರೀತಿ-ಸು-ತ್ತಿ-ದ್ದೀ-ಯೆಂದು
ಪ್ರೀತಿ-ಸು-ತ್ತೇನೆ
ಪ್ರೀತಿ-ಸು-ತ್ತೇ-ನೆ-ಭಾ-ರ-ತೀಯ
ಪ್ರೀತಿ-ಸು-ತ್ತೇ-ನೆಯೋ
ಪ್ರೀತಿ-ಸುವ
ಪ್ರೀತಿ-ಸುವು
ಪ್ರೀತಿ-ಸು-ವು-ದ-ರಲ್ಲಿ
ಪ್ರೀತಿ-ಸು-ವು-ದಿಲ್ಲ
ಪ್ರೀತ್ಯಾ-ದರ
ಪ್ರೀತ್ಯಾ-ದ-ರ-ಗಳನ್ನು
ಪ್ರೀತ್ಯಾ-ದ-ರ-ಗ-ಳಿಗೆ
ಪ್ರೀತ್ಯಾ-ದ-ರ-ದಿಂದ
ಪ್ರೆಸ್
ಪ್ರೇಕ್ಷ
ಪ್ರೇಕ್ಷ-ಕರ
ಪ್ರೇಕ್ಷ-ಕ-ರತ್ತ
ಪ್ರೇಕ್ಷ-ಣೀಯ
ಪ್ರೇತ
ಪ್ರೇತ-ವೊಂ-ದನ್ನು
ಪ್ರೇತಾ-ರಾ-ಧ-ಕ-ನಾದ
ಪ್ರೇಮ
ಪ್ರೇಮ-ಕ್ಕಿಂತ
ಪ್ರೇಮದ
ಪ್ರೇಮ-ಪೂರ್ಣ
ಪ್ರೇಮ-ಭಾ-ವ-ರಂ-ಜಿತ
ಪ್ರೇಮ-ಮಯ
ಪ್ರೇಮ-ವನ್ನು
ಪ್ರೇಮ-ವೆಂದರೆ
ಪ್ರೇಮ-ವೆಂಬ
ಪ್ರೇಮ-ವೆಂ-ಬುದು
ಪ್ರೇಮ-ವೆ-ಲ್ಲವೂ
ಪ್ರೇಮವೇ
ಪ್ರೇಮ-ಸುಧೆ
ಪ್ರೇಮಾ
ಪ್ರೇಮಾ-ನಂದ
ಪ್ರೇಮಾ-ನಂ-ದರ
ಪ್ರೇಮಾ-ನಂ-ದ-ರನ್ನು
ಪ್ರೇಮಾ-ನಂ-ದ-ರ-ನ್ನೊ-ಡ-ಗೂಡಿ
ಪ್ರೇಮಾ-ನಂ-ದ-ರಿಗೂ
ಪ್ರೇಮಾ-ನಂ-ದ-ರಿ-ದ್ದರು
ಪ್ರೇಮಾ-ನಂ-ದರು
ಪ್ರೇಮಾ-ನಂ-ದರೇ
ಪ್ರೇಮಾ-ನಂ-ದ-ರೊ-ಡನೆ
ಪ್ರೇರ-ಕ-ವಾ-ಗು-ವಂಥ
ಪ್ರೇರಣೆ
ಪ್ರೇರ-ಣೆ-ಗಳಿಂದ
ಪ್ರೇರ-ಣೆ-ಯನ್ನು
ಪ್ರೇರಿ-ತ-ರಾಗಿ
ಪ್ರೇರಿ-ತ-ರಾದ
ಪ್ರೇರಿ-ತ-ವಾ-ಗಿ-ರು-ತ್ತದೆ
ಪ್ರೇರಿ-ತ-ವಾದ
ಪ್ರೇರಿ-ಸುವ
ಪ್ರೇರೇ-ಪಿ-ಸಿ-ದರು
ಪ್ರೇರೇ-ಪಿ-ಸಿವೆ
ಪ್ರೇರೇ-ಪಿ-ಸುವ
ಪ್ರೊ
ಪ್ರೊಫೆ-ಸ-ರ-ರಾ-ಗಿ-ದ್ದ-ರ-ಲ್ಲದೆ
ಪ್ರೊಫೆ-ಸ-ರು-ಗಳ
ಪ್ರೊಫೆ-ಸರ್
ಪ್ರೋಕ್ಷಿ-ಸ-ಲಾ-ಯಿತು
ಪ್ರೋಕ್ಷಿಸಿ
ಪ್ರೋತ್ಸಾಹ
ಪ್ರೋತ್ಸಾ-ಹ-ನೆ-ರವು
ಪ್ರೋತ್ಸಾ-ಹಕಿ
ಪ್ರೋತ್ಸಾ-ಹ-ಗಳೇ
ಪ್ರೋತ್ಸಾ-ಹ-ದಿಂದ
ಪ್ರೋತ್ಸಾ-ಹ-ವನ್ನು
ಪ್ರೋತ್ಸಾ-ಹ-ವನ್ನೂ
ಪ್ರೋತ್ಸಾ-ಹ-ವಾ-ಗಲಿ
ಪ್ರೋತ್ಸಾ-ಹ-ವಿತ್ತು
ಪ್ರೋತ್ಸಾ-ಹಿ-ಸ-ಬೇಕು
ಪ್ರೋತ್ಸಾ-ಹಿ-ಸಿ-ದರು
ಪ್ರೋತ್ಸಾ-ಹಿ-ಸಿ-ರ-ಲಿಲ್ಲ
ಪ್ರೋತ್ಸಾ-ಹಿ-ಸು-ತ್ತಿ-ದ್ದರು
ಪ್ರೋತ್ಸಾ-ಹಿ-ಸು-ತ್ತಿ-ದ್ದ-ವರು
ಪ್ಲಾಟ್ಫಾರಂ
ಪ್ಲೇಗು
ಪ್ಲೇಗ್
ಪ್ಲೇಟೋ
ಫಂಕೆ
ಫಂಕೆ-ಯನ್ನು
ಫಕೀರ
ಫಕೀ-ರ-ಗು-ರು-ವಿಗೆ
ಫಕೀ-ರನ
ಫಲ
ಫಲ-ಕ-ಗಳನ್ನು
ಫಲ-ಕ-ಗಳಲ್ಲಿ
ಫಲ-ಕ-ಗ-ಳಿ-ದ್ದುವು
ಫಲ-ಕ-ವಿದೆ
ಫಲ-ಕಾ-ರಿ-ಯಾ-ಗ-ಲಿಲ್ಲ
ಫಲ-ಕ್ಕ-ಲ್ಲ-ಇದು
ಫಲದ
ಫಲ-ದಿಂದ
ಫಲ-ಪುಷ್ಪ
ಫಲ-ಪು-ಷ್ಪ-ಗ-ಳ-ನ್ನಿ-ತ್ತರು
ಫಲ-ಪ್ರ-ದ-ವಾಗು
ಫಲ-ವ-ತ್ತಾದ
ಫಲ-ವ-ತ್ತಾ-ದದ್ದು
ಫಲ-ವನ್ನು
ಫಲ-ವಾಗಿ
ಫಲವೇ
ಫಲಿ-ಸಿತು
ಫಲಿ-ಸು-ವು-ದಿಲ್ಲ
ಫಲೇಷು
ಫಾಕ್ಸ್
ಫಾದರ್
ಫಾರ್
ಫಿರಂಗಿ
ಫಿರಂ-ಗಿಯ
ಫೀಸಿಗೇ
ಫೆ
ಫೆಬ್ರ-ವರಿ
ಫೆಬ್ರು-ವರಿ
ಫೆಬ್ರು-ವ-ರಿಯ
ಫೆಬ್ರು-ವ-ರಿ-ಯಲ್ಲಿ
ಫೋಟೋ
ಫೋಟೋ-ಗ್ರಾ-ಫರ್
ಫೋಟೋ-ವನ್ನು
ಫೋನ್
ಫೋರ್ನಿಯ
ಫ್ಯಾಶನ್
ಫ್ಯಾಷನ್ನೇ
ಫ್ರಾನ್ಸನ್ನು
ಫ್ರಾನ್ಸಿಗೆ
ಫ್ರಾನ್ಸಿನ
ಫ್ರಾನ್ಸಿ-ನಿಂದ
ಫ್ರಾನ್ಸಿಸ್
ಫ್ರಾನ್ಸಿಸ್ಕೋ
ಫ್ರಾನ್ಸಿ-ಸ್ಕೋಗೆ
ಫ್ರಾನ್ಸಿ-ಸ್ಕೋ-ದ-ಲ್ಲಿ-ದ್ದಾಗ
ಫ್ರಾನ್ಸಿ-ಸ್ಕೋ-ದಲ್ಲೂ
ಫ್ರಾನ್ಸ್
ಫ್ರೆಂಚಿನ
ಫ್ರೆಂಚಿ-ನಲ್ಲೇ
ಫ್ರೆಂಚ್
ಫ್ರೈಡೇ
ಫ್ಲಾರೆ-ನ್ಸಿ-ನಿಂದ
ಫ್ಲಾರೆ-ನ್ಸ್
ಫ್ಲೋರಲ್
ಬಂಕಿ-ಮ-ಚಂ-ದ್ರರ
ಬಂಗಲೆ
ಬಂಗ-ಲೆಗೆ
ಬಂಗ-ಲೆಯ
ಬಂಗ-ಲೆ-ಯನ್ನು
ಬಂಗ-ಲೆ-ಯಲ್ಲಿ
ಬಂಗ-ಲೆ-ಯಲ್ಲೇ
ಬಂಗ-ಲೆ-ಯ-ವ-ರೆಗೂ
ಬಂಗ-ಲೆ-ಯೊ-ಳಗೆ
ಬಂಗಾ-ರ-ವನ್ನು
ಬಂಗಾಳ
ಬಂಗಾ-ಳ-ಕೊ-ಲ್ಲಿ-ಯನ್ನು
ಬಂಗಾ-ಳಕ್ಕೆ
ಬಂಗಾ-ಳದ
ಬಂಗಾ-ಳ-ದಲ್ಲಿ
ಬಂಗಾ-ಳ-ದ-ಲ್ಲಿ-ದ್ದರು
ಬಂಗಾ-ಳ-ದ-ಲ್ಲಿಯೇ
ಬಂಗಾ-ಳ-ದಲ್ಲೆಲ್ಲ
ಬಂಗಾ-ಳ-ವನ್ನು
ಬಂಗಾ-ಳವು
ಬಂಗಾ-ಳವೇ
ಬಂಗಾಳಿ
ಬಂಗಾ-ಳಿ-ಗಳು
ಬಂಗಾ-ಳಿಯ
ಬಂಗಾ-ಳಿ-ಯಲ್ಲಿ
ಬಂಗಾಳೀ
ಬಂಗ್ಲಾ-ದೇಶ
ಬಂಡೆ-ಯಂ-ತಾ-ಗಿ-ಬಿ-ಡು-ತ್ತೇನೆ
ಬಂತು
ಬಂತೆ-ನ್ನು-ತ್ತಾರೆ
ಬಂದ
ಬಂದಂ-ತಿ-ಲ್ಲ-ವೆಂ-ಬುದು
ಬಂದಂತೆ
ಬಂದಂ-ತೆ-ನ್ನಿ-ಸಿತು
ಬಂದಂ-ತೆಯೇ
ಬಂದಂ-ತೆಲ್ಲ
ಬಂದಂ-ದಿ-ನಿಂದ
ಬಂದ-ಕೂ-ಡಲೇ
ಬಂದ-ದ್ದಕ್ಕೆ
ಬಂದ-ದ್ದರ
ಬಂದ-ದ್ದ-ರಿಂದ
ಬಂದ-ದ್ದಾ-ದರೂ
ಬಂದದ್ದು
ಬಂದದ್ದೆ
ಬಂದದ್ದೇ
ಬಂದನೆ
ಬಂದ-ಬಂ-ದ-ವ-ರಿ-ಗೆಲ್ಲ
ಬಂದ-ಮೇಲೆ
ಬಂದ-ರನ್ನು
ಬಂದ-ರಿನ
ಬಂದ-ರಿ-ನಲ್ಲಿ
ಬಂದ-ರಿ-ನ-ಲ್ಲಿದ್ದ
ಬಂದರು
ಬಂದರೂ
ಬಂದರೆ
ಬಂದ-ಳಾ-ದರೆ
ಬಂದಳು
ಬಂದಳೋ
ಬಂದವ
ಬಂದ-ವ-ನಲ್ಲ
ಬಂದ-ವನು
ಬಂದ-ವರ
ಬಂದ-ವ-ರನ್ನು
ಬಂದ-ವ-ರಲ್ಲಿ
ಬಂದ-ವ-ರಾ-ದರೂ
ಬಂದ-ವ-ರಿಗೆ
ಬಂದ-ವ-ರಿ-ಗೆಲ್ಲ
ಬಂದ-ವರು
ಬಂದ-ವ-ಳ-ಲ್ಲವೆ
ಬಂದಷ್ಟು
ಬಂದಾಗ
ಬಂದಾ-ಗ-ಲಂತೂ
ಬಂದಾ-ಗ-ಲೆಲ್ಲ
ಬಂದಾ-ಗಲೇ
ಬಂದಾ-ಗಿ-ನಿಂದ
ಬಂದಾ-ಗಿ-ನಿಂ-ದಲೂ
ಬಂದಾ-ಗಿ-ರು-ವು-ದ-ರಿಂದ
ಬಂದಾ-ರೆಂದು
ಬಂದಿ
ಬಂದಿತು
ಬಂದಿ-ತು-ಅದು
ಬಂದಿ-ತು-ಕ-ಲ್ಕ-ತ್ತ-ದಲ್ಲಿ
ಬಂದಿ-ತು-ತಾವೇ
ಬಂದಿ-ತು-ನ-ಮಗೆ
ಬಂದಿ-ತು-ಸೋಽಹಂ
ಬಂದಿ-ತು-ಸ್ವಾ-ಮೀಜಿ
ಬಂದಿತ್ತು
ಬಂದಿದೆ
ಬಂದಿ-ದೆ-ಬಂ-ದು-ಬಿಡು
ಬಂದಿದ್ದ
ಬಂದಿ-ದ್ದಂತೆ
ಬಂದಿ-ದ್ದರು
ಬಂದಿ-ದ್ದರೂ
ಬಂದಿ-ದ್ದರೆ
ಬಂದಿ-ದ್ದ-ರೆಂದು
ಬಂದಿ-ದ್ದಳು
ಬಂದಿ-ದ್ದ-ವನು
ಬಂದಿ-ದ್ದ-ವರು
ಬಂದಿ-ದ್ದ-ವಳು
ಬಂದಿ-ದ್ದಾಗ
ಬಂದಿ-ದ್ದಾ-ಗಿನ
ಬಂದಿ-ದ್ದಾನೆ
ಬಂದಿ-ದ್ದಾರೆ
ಬಂದಿ-ದ್ದಾ-ರೆಂಬ
ಬಂದಿ-ದ್ದಾರೋ
ಬಂದಿ-ದ್ದಾಳೆ
ಬಂದಿ-ದ್ದಾ-ಳೆಂಬ
ಬಂದಿ-ದ್ದೀರಿ
ಬಂದಿದ್ದು
ಬಂದಿ-ದ್ದುವು
ಬಂದಿದ್ದೆ
ಬಂದಿ-ದ್ದೇನೆ
ಬಂದಿ-ದ್ದೇ-ನೆ-ನಾನು
ಬಂದಿ-ರ-ದಿ-ದ್ದರೆ
ಬಂದಿ-ರಪ್ಪಾ
ಬಂದಿ-ರ-ಬ-ಹುದು
ಬಂದಿ-ರ-ಬೇ-ಕ-ಲ್ಲವೆ
ಬಂದಿ-ರ-ಬೇಕು
ಬಂದಿ-ರ-ಲಾ-ರದು
ಬಂದಿ-ರಲಿ
ಬಂದಿ-ರ-ಲಿಲ್ಲ
ಬಂದಿ-ರಲು
ಬಂದಿರಿ
ಬಂದಿ-ರು-ತ್ತಿ-ದ್ದಳು
ಬಂದಿ-ರುವ
ಬಂದಿ-ರು-ವಂ-ತಾ-ದರೆ
ಬಂದಿ-ರು-ವಾಗ
ಬಂದಿ-ರು-ವುದು
ಬಂದಿ-ರು-ವುದೇ
ಬಂದಿ-ಳಿ-ದರು
ಬಂದಿ-ಳಿ-ದಾ-ಗಿ-ನಿಂದ
ಬಂದಿ-ಳಿ-ದು-ಕೊಂ-ಡರು
ಬಂದಿ-ಳಿ-ಯು-ತ್ತಿ-ದ್ದಂ-ತೆಯೇ
ಬಂದಿವೆ
ಬಂದೀತು
ಬಂದು
ಬಂದು-ದಕ್ಕೆ
ಬಂದು-ದನ್ನು
ಬಂದು-ಬಿ-ಟ್ಟರು
ಬಂದು-ಬಿ-ಟ್ಟಳು
ಬಂದು-ಬಿ-ಟ್ಟಿತು
ಬಂದು-ಬಿ-ಟ್ಟಿದೆ
ಬಂದು-ಬಿ-ಟ್ಟಿ-ದ್ದಳು
ಬಂದು-ಬಿ-ಟ್ಟೆವು
ಬಂದು-ಬಿಡು
ಬಂದು-ಬಿ-ಡು-ತ್ತದೆ
ಬಂದು-ಬಿ-ಡು-ತ್ತಾ-ರೆಂದು
ಬಂದು-ಬಿ-ಡು-ತ್ತಿ-ದ್ದರು
ಬಂದು-ಬಿ-ಡು-ತ್ತಿದ್ದೆ
ಬಂದು-ಬಿ-ಡು-ತ್ತೇನೆ
ಬಂದುವು
ಬಂದು-ಹೋ-ಗು-ವಂತೆ
ಬಂದು-ಹೋದ
ಬಂದೂ-ಕ-ವನ್ನು
ಬಂದೂ-ಕಿ-ನಿಂದ
ಬಂದೂಕು
ಬಂದೆ
ಬಂದೆಯೋ
ಬಂದೆ-ರ-ಗಿತು
ಬಂದೆ-ರ-ಗಿದ
ಬಂದೆವು
ಬಂದೇ
ಬಂದೇ-ಬಿ-ಟ್ಟಿತು
ಬಂದೊ-ದ-ಗು-ವುದು
ಬಂಧ
ಬಂಧ-ಗಳ
ಬಂಧನ
ಬಂಧ-ನ-ಕ್ಕೊ-ಳ-ಪ-ಡಿ-ಸಿ-ಕೊ-ಳ್ಳು-ತ್ತದೆ
ಬಂಧ-ನ-ಗಳ
ಬಂಧ-ನ-ಗಳನ್ನು
ಬಂಧ-ನ-ಗಳನ್ನೂ
ಬಂಧ-ನ-ಗಳಿಂದ
ಬಂಧ-ನ-ಗ-ಳಿಂ-ದೆಲ್ಲ
ಬಂಧ-ನದ
ಬಂಧ-ನ-ದಲ್ಲಿ
ಬಂಧ-ನ-ದಲ್ಲೂ
ಬಂಧ-ನ-ದಿಂದ
ಬಂಧ-ನ-ವಿಲ್ಲ
ಬಂಧಿ-ತ-ನಾ-ದ-ವನು
ಬಂಧಿ-ತ-ರಾ-ಗಿ-ರುವ
ಬಂಧಿ-ತ-ರಾ-ಗಿ-ರು-ವು-ದ-ಕ್ಕಿಂತ
ಬಂಧಿ-ತ-ವಾ-ಗು-ತ್ತದೆ
ಬಂಧಿ-ತ-ವಾದ
ಬಂಧಿ-ಸಿಟ್ಟು
ಬಂಧಿ-ಸಿ-ಡುವ
ಬಂಧಿ-ಸಿ-ದ್ದುದು
ಬಂಧು
ಬಂಧು-ಗಳ
ಬಂಧು-ಗ-ಳಾದ
ಬಂಧು-ಗ-ಳಿಗೆ
ಬಂಧು-ಗ-ಳೆ-ಲ್ಲ-ರನ್ನೂ
ಬಂಧು-ಗ-ಳೊ-ಡನೆ
ಬಂಧು-ಬಾಂ-ಧ-ವರು
ಬಕ್ಕೂ
ಬಗೆ
ಬಗೆ-ಇ-ವು-ಗ-ಳಿಗೆ
ಬಗೆಈ
ಬಗೆ-ಗಳಲ್ಲಿ
ಬಗೆ-ಗಿದ್ದ
ಬಗೆ-ಗಿನ
ಬಗೆಗೂ
ಬಗೆಗೆ
ಬಗೆ-ದರೆ
ಬಗೆದು
ಬಗೆ-ಬ-ಗೆಯ
ಬಗೆ-ಬ-ಗೆ-ಯಿಂದ
ಬಗೆಯ
ಬಗೆ-ಯ-ದಾ-ಗಿತ್ತು
ಬಗೆ-ಯದು
ಬಗೆ-ಯನ್ನು
ಬಗೆ-ಯಲ್ಲಿ
ಬಗೆ-ಯ-ವರೂ
ಬಗೆ-ಯಾಗಿ
ಬಗೆ-ಯಿತು
ಬಗೆ-ಹ-ರಿ-ಸಿ-ಕೊಂ-ಡರು
ಬಗೆ-ಹ-ರಿ-ಸಿ-ಕೊ-ಳ್ಳಲು
ಬಗ್ಗು
ಬಗ್ಗು-ಬ-ಡಿ-ಯಲು
ಬಗ್ಗೆ
ಬಗ್ಗೆ-ಯಂತೂ
ಬಗ್ಗೆ-ಯಾ-ಗಲಿ
ಬಗ್ಗೆಯೂ
ಬಗ್ಗೆಯೇ
ಬಗ್ಗೆಯೋ
ಬಚಾ-ವಾ-ಗ-ಬ-ಹುದು
ಬಚಾ-ವಾ-ದೆವು
ಬಚ್ಚಿ-ಟ್ಟು-ಕೊ-ಳ್ಳು-ತ್ತಿದ್ದೆ
ಬಜಾ-ರಿ-ನ-ಲ್ಲಿದ್ದ
ಬಜಾ-ರಿ-ನಲ್ಲೇ
ಬಜಾ-ರಿ-ನ-ವ-ರೆಗೆ
ಬಜ್ಬಜ್
ಬಟಾಣಿ
ಬಟ್ಟಲು
ಬಟ್ಟೆ
ಬಟ್ಟೆ-ಇದು
ಬಟ್ಟೆ-ಗಳ
ಬಟ್ಟೆ-ಗಳನ್ನು
ಬಟ್ಟೆ-ಗ-ಳಿ-ರ-ಬೇಕು
ಬಟ್ಟೆಗೂ
ಬಟ್ಟೆಯ
ಬಟ್ಟೆ-ಯಂತೆ
ಬಟ್ಟೆ-ಯನ್ನು
ಬಟ್ಟೆ-ಯಲ್ಲೇ
ಬಟ್ಟೆ-ಯಿ-ಲ್ಲ-ದಿ-ದ್ದಾ-ಗಲೇ
ಬಡ
ಬಡ-ಜನ
ಬಡ-ಜ-ನರ
ಬಡ-ಜ-ನ-ರಿ-ಗಾಗಿ
ಬಡ-ಜ-ನ-ರಿಗೆ
ಬಡ-ಜ-ನರೇ
ಬಡ-ಜೀವ
ಬಡ-ಜೀ-ವನ
ಬಡ-ಜೀ-ವಿ-ಗಳು
ಬಡ-ಜೋ-ಪ-ಡಿ-ಯಲ್ಲಿ
ಬಡ-ತನ
ಬಡ-ತ-ನದ
ಬಡ-ತ-ನ-ದೊಂ-ದಿಗೆ
ಬಡ-ತ-ನ-ವಿ-ದ್ದರೂ
ಬಡ-ತ-ನವೂ
ಬಡ-ಪ-ಟ್ಟಿಗೆ
ಬಡ-ಬ-ಗ್ಗರ
ಬಡ-ಬ-ಡಿ-ಕೆಗೆ
ಬಡ-ಬ-ಡಿ-ಸ-ಲಾ-ರಂಭಿ
ಬಡ-ಬಾಗ್ನಿ
ಬಡ-ಮ-ಹಿ-ಳೆ-ಯೊ-ಬ್ಬಳು
ಬಡ-ಯಾ-ತ್ರಿ-ಕ-ರಲ್ಲಿ
ಬಡ-ರಾ-ಷ್ಟ್ರ-ವಾ-ಗಿ-ರ-ಬ-ಹುದು
ಬಡವ
ಬಡ-ವ-ಬ-ಲ್ಲಿದ
ಬಡ-ವರ
ಬಡ-ವ-ರ-ದೀ-ನರ
ಬಡ-ವ-ರನ್ನು
ಬಡ-ವ-ರ-ಲ್ಲಿ-ರುವ
ಬಡ-ವ-ರಾ-ದ-ವರು
ಬಡ-ವ-ರಿ-ಗಿಂತ
ಬಡ-ವ-ರಿಗೆ
ಬಡ-ವರು
ಬಡ-ವರೊ
ಬಡ-ವಾ-ಗಿದೆ
ಬಡ-ವಾ-ಗು-ತ್ತದೆ
ಬಡ-ವಾ-ಗು-ತ್ತಿದೆ
ಬಡವಿ
ಬಡ-ಶಿ-ಷ್ಯ-ರ-ಲ್ಲೊಬ್ಬ
ಬಡ-ಸಂ-ನ್ಯಾ-ಸಿ-ಯಾಗಿ
ಬಡಿತ
ಬಡಿ-ತ-ಗ-ಳಾ-ಗಿ-ದ್ದುವು
ಬಡಿ-ತ-ದಲ್ಲಿ
ಬಡಿ-ತವೇ
ಬಡಿ-ದಂ-ತಾ-ಯಿತು
ಬಡಿ-ದಂ-ತಿತ್ತು
ಬಡಿ-ದಂತೆ
ಬಡಿ-ದಾ-ಗ-ಲಂತೂ
ಬಡಿ-ದಿದೆ
ಬಡಿ-ದಿ-ರುವ
ಬಡಿ-ದೆ-ಬ್ಬಿಸಿ
ಬಡಿ-ದೆ-ಬ್ಬಿ-ಸು-ವಂ-ತಾ-ದರೆ
ಬಡಿ-ದೋ-ಡಿ-ಸುವ
ಬಡಿ-ಯ-ಬೇಕು
ಬಡಿ-ಯುವ
ಬಡಿ-ಸ-ಬೇಕು
ಬಡಿ-ಸ-ಬೇ-ಕೆಂಬ
ಬಡಿ-ಸ-ಲಾ-ಯಿತು
ಬಡಿ-ಸಿ-ಕೊಂಡು
ಬಡಿ-ಸಿ-ದ-ರಾ-ಯಿತು
ಬಡಿ-ಸಿ-ದರು
ಬಡಿ-ಸುತ್ತ
ಬಡಿ-ಸು-ತ್ತೀರಿ
ಬಡ್ಡಿಗೆ
ಬಡ್ಡಿಯ
ಬಣ-ಗಳ
ಬಣ್ಣ
ಬಣ್ಣ-ಆ-ಕಾ-ರ-ಗ-ಳಿಂ-ದಾಗಿ
ಬಣ್ಣಕ್ಕೆ
ಬಣ್ಣದ
ಬಣ್ಣ-ದ-ವ-ರೆಂಬ
ಬಣ್ಣ-ಬ-ಣ್ಣದ
ಬಣ್ಣ-ವನ್ನೂ
ಬಣ್ಣಿಸ
ಬಣ್ಣಿ-ಸ-ತೊ-ಡ-ಗಿ-ದರು
ಬಣ್ಣಿ-ಸ-ಬ-ಹು-ದಾದ
ಬಣ್ಣಿ-ಸಲು
ಬಣ್ಣಿಸಿ
ಬಣ್ಣಿ-ಸಿ-ದರು
ಬಣ್ಣಿ-ಸಿ-ದ-ರೆಂ-ದರೆ
ಬಣ್ಣಿ-ಸಿ-ದ್ದರು
ಬಣ್ಣಿ-ಸಿ-ದ್ದಾಳೆ
ಬಣ್ಣಿಸು
ಬಣ್ಣಿ-ಸುತ್ತ
ಬಣ್ಣಿ-ಸು-ತ್ತಾರೆ
ಬಣ್ಣಿ-ಸು-ತ್ತಾ-ರೆ-ಒಂದು
ಬಣ್ಣಿ-ಸು-ತ್ತಾಳೆ
ಬಣ್ಣಿ-ಸು-ತ್ತಿ-ದ್ದರು
ಬಣ್ಣಿ-ಸುವ
ಬಣ್ಣಿ-ಸು-ವಾಗ
ಬಣ್ಣಿ-ಸು-ವು-ದಕ್ಕೇ
ಬಣ್ಣಿ-ಸು-ವುದನ್ನು
ಬತ್ತದ
ಬತ್ತಿ
ಬತ್ತಿಯೇ
ಬತ್ತಿ-ಹೋದ
ಬತ್ತಿ-ಹೋ-ದಂ-ತಿ-ತ್ತೆಂ-ಬುದೇ
ಬತ್ತಿ-ಹೋ-ಯಿತು
ಬದ
ಬದರೀ
ಬದ-ರೀ-ಸಾ-ಹರ
ಬದಲಾ
ಬದ-ಲಾ-ಗದ
ಬದ-ಲಾ-ಗ-ಲಿಲ್ಲ
ಬದ-ಲಾಗಿ
ಬದ-ಲಾ-ಗಿತ್ತು
ಬದ-ಲಾ-ಗಿ-ದೆಯೇ
ಬದ-ಲಾ-ಗಿದ್ದ
ಬದ-ಲಾ-ಗಿ-ರ-ಬ-ಹು-ದಷ್ಟೆ
ಬದ-ಲಾ-ಗಿಲ್ಲ
ಬದ-ಲಾ-ಗು-ತ್ತಿತ್ತು
ಬದ-ಲಾ-ಗು-ತ್ತಿ-ದ್ದಿ-ರ-ಬೇಕು
ಬದ-ಲಾ-ಗು-ವಂತೆ
ಬದ-ಲಾ-ಗು-ವುದೂ
ಬದ-ಲಾ-ದ-ದ್ದರ
ಬದ-ಲಾ-ದೀತೆ
ಬದ-ಲಾ-ಯಿತು
ಬದ-ಲಾ-ಯಿ-ಸ-ಬೇ-ಕಾ-ದರೂ
ಬದ-ಲಾ-ಯಿ-ಸ-ಬೇಕು
ಬದ-ಲಾ-ಯಿ-ಸ-ಲಾ-ಯಿತು
ಬದ-ಲಾ-ಯಿ-ಸಲೇ
ಬದ-ಲಾ-ಯಿಸಿ
ಬದ-ಲಾ-ಯಿ-ಸಿ-ಕೊಂಡ
ಬದ-ಲಾ-ಯಿ-ಸಿ-ಕೊ-ಳ್ಳು-ವು-ದ-ರಿಂದ
ಬದ-ಲಾ-ಯಿ-ಸಿ-ಕೊ-ಳ್ಳು-ವು-ದೇನು
ಬದ-ಲಾ-ಯಿ-ಸಿ-ದರು
ಬದ-ಲಾ-ಯಿ-ಸು-ತ್ತೇನೆ
ಬದ-ಲಾ-ವಣೆ
ಬದ-ಲಾ-ವ-ಣೆ-ಗಳನ್ನು
ಬದ-ಲಾ-ವ-ಣೆ-ಗಳಿಂದ
ಬದ-ಲಾ-ವ-ಣೆ-ಗಾಗಿ
ಬದ-ಲಾ-ವ-ಣೆಗೆ
ಬದ-ಲಾ-ವ-ಣೆ-ಯನ್ನೂ
ಬದ-ಲಾ-ವ-ಣೆ-ಯಾ-ಗಿ-ಬಿ-ಟ್ಟಿತು
ಬದ-ಲಾ-ವ-ಣೆ-ಯಾ-ಗಿ-ರು-ವುದು
ಬದ-ಲಾ-ವ-ಣೆ-ಯುಂ-ಟಾ-ಯಿತು
ಬದ-ಲಿಗೆ
ಬದ-ಲಿ-ಸದೇ
ಬದ-ಲಿ-ಸ-ಬ-ಲ್ಲಿರಿ
ಬದ-ಲಿ-ಸ-ಬ-ಹುದು
ಬದ-ಲಿ-ಸ-ಬೇ-ಕಾ-ಯಿತು
ಬದ-ಲಿ-ಸಿ-ಕೊಳ್ಳ
ಬದ-ಲಿ-ಸಿ-ದಂ-ತಷ್ಟೇ
ಬದ-ಲಿ-ಸಿ-ದರು
ಬದಲು
ಬದಿ-ಗಿಟ್ಟು
ಬದಿಗೆ
ಬದಿ-ಗೊ-ತ್ತ-ಬೇ-ಕಾ-ದರೆ
ಬದಿ-ಗೊತ್ತಿ
ಬದಿಯ
ಬದಿ-ಯಲ್ಲಿ
ಬದಿ-ಯಲ್ಲೇ
ಬದು-ಕನ್ನು
ಬದು-ಕ-ಬ-ಹುದು
ಬದು-ಕಲಿ
ಬದು-ಕಲು
ಬದು-ಕಲೂ
ಬದುಕಿ
ಬದು-ಕಿಗೆ
ಬದು-ಕಿ-ಗೊಂದು
ಬದು-ಕಿ-ದರೆ
ಬದು-ಕಿದ್ದ
ಬದು-ಕಿ-ದ್ದರೂ
ಬದು-ಕಿ-ದ್ದರೆ
ಬದು-ಕಿ-ದ್ದ-ವರು
ಬದು-ಕಿ-ದ್ದಾರೆ
ಬದು-ಕಿನ
ಬದು-ಕಿ-ನಲ್ಲಿ
ಬದು-ಕಿ-ನಲ್ಲೇ
ಬದು-ಕಿ-ನೊ-ಳ-ಗುಟ್ಟ
ಬದು-ಕಿ-ರುವ
ಬದು-ಕಿ-ಸ-ಬೇ-ಕಾ-ದದ್ದು
ಬದು-ಕಿ-ಸಲು
ಬದುಕು
ಬದು-ಕು-ಳಿದೆ
ಬದು-ಕು-ಳಿ-ಯುವ
ಬದು-ಕುವ
ಬದು-ಕುವು
ಬದ್ಧ-ನಾ-ಗಿ-ರು-ವ-ವ-ರೆಗೆ
ಬದ್ಧ-ರಾ-ಗಿ-ದ್ದ-ರೆಂ-ಬು-ದಕ್ಕೆ
ಬನ್ನಿ
ಬಬೈ
ಬಯಕೆ
ಬಯ-ಕೆ-ಯಂತೆ
ಬಯ-ಕೆ-ಯಾ-ಗು-ತ್ತದೆ
ಬಯ-ಕೆ-ಯಿತ್ತು
ಬಯ-ಕೆ-ಯೆಂ-ದರೆ
ಬಯ-ಕೆ-ಯೆಲ್ಲ
ಬಯ-ಕೆಯೇ
ಬಯ-ಲಲ್ಲಿ
ಬಯಲಿ
ಬಯ-ಲಿ-ಗೆ-ಳೆ-ದರು
ಬಯ-ಲಿ-ಗೆ-ಳೆ-ದಾಗ
ಬಯ-ಲಿ-ಗೆ-ಳೆದು
ಬಯ-ಲಿ-ಗೆ-ಳೆ-ಯ-ಲಿ-ದ್ದೇನೆ
ಬಯ-ಲಿ-ಗೆ-ಳೆ-ಯುವ
ಬಯ-ಲಿ-ಗೆ-ಳೆ-ಯು-ವಲ್ಲಿ
ಬಯ-ಲಿ-ನಲ್ಲಿ
ಬಯಲು
ಬಯ-ಲು-ಗಳಲ್ಲಿ
ಬಯ-ಲು-ಪ್ರ-ದೇಶ
ಬಯ-ಲು-ಸೀ-ಮೆಗೆ
ಬಯ-ಲು-ಸೀ-ಮೆಯ
ಬಯಸ
ಬಯ-ಸಲಿ
ಬಯ-ಸ-ಲಿಲ್ಲ
ಬಯಸಿ
ಬಯ-ಸಿ-ದರು
ಬಯ-ಸಿ-ದರೂ
ಬಯ-ಸಿ-ದಳು
ಬಯ-ಸಿ-ದಷ್ಟು
ಬಯ-ಸಿ-ದ್ದ-ರಿಂದ
ಬಯ-ಸಿ-ದ್ದರು
ಬಯ-ಸಿ-ದ್ದರೆ
ಬಯ-ಸಿ-ದ್ದಾ-ರೆಂಬ
ಬಯ-ಸಿದ್ದು
ಬಯ-ಸಿದ್ದೆ
ಬಯ-ಸಿ-ಯಾನೆ
ಬಯ-ಸಿ-ರ-ಬ-ಹುದು
ಬಯ-ಸು-ತ್ತದೆ
ಬಯ-ಸು-ತ್ತಾನೆ
ಬಯ-ಸು-ತ್ತಾರೊ
ಬಯ-ಸು-ತ್ತಾಳೆ
ಬಯ-ಸು-ತ್ತಿ-ದ್ದ-ರೆಂ-ಬು-ದನ್ನು
ಬಯ-ಸು-ತ್ತೇನೆ
ಬಯ-ಸುವ
ಬಯ-ಸು-ವಂತೆ
ಬಯ-ಸು-ವ-ವನು
ಬಯ-ಸು-ವ-ವರು
ಬಯ-ಸು-ವೆ-ಯಾ-ದರೆ
ಬರ
ಬರ-ಗಾಲ
ಬರ-ಗಾ-ಲ-ದಿಂದ
ಬರ-ಗಾ-ಲ-ಪ-ರಿ-ಹಾ-ರದ
ಬರ-ಡಾ-ಗಿ-ರು-ತ್ತಾನೋ
ಬರ-ತೊ-ಡ-ಗಿತ್ತು
ಬರ-ದಂತೆ
ಬರ-ದಿ-ದ್ದ-ವ-ರನ್ನು
ಬರದೆ
ಬರ-ಪ-ರಿ-ಹಾ-ರ-ಕ್ಕಾಗಿ
ಬರ-ಪೀ-ಡಿತ
ಬರ-ಬ-ರುತ್ತ
ಬರ-ಬ-ಲ್ಲರು
ಬರ-ಬ-ಹುದು
ಬರ-ಬೇ-ಕಾ-ಗ-ಬ-ಹುದು
ಬರ-ಬೇ-ಕಾ-ಗಿತ್ತು
ಬರ-ಬೇ-ಕಾ-ಗಿದ್ದ
ಬರ-ಬೇ-ಕಾ-ಗು-ತ್ತದೆ
ಬರ-ಬೇ-ಕಾ-ಗು-ತ್ತ-ದೆಯೆ
ಬರ-ಬೇ-ಕಾದ
ಬರ-ಬೇ-ಕಾ-ದದ್ದೇ
ಬರ-ಬೇ-ಕಾ-ದರೆ
ಬರ-ಬೇ-ಕಾ-ಯಿತು
ಬರ-ಬೇಕು
ಬರ-ಬೇ-ಕೆಂದು
ಬರ-ಬೇ-ಕೆಂಬ
ಬರ-ಬೇ-ಕೆಂ-ಬು-ವ-ವರು
ಬರ-ಮಾಡಿ
ಬರ-ಮಾ-ಡಿ-ಕೊಂಡ
ಬರ-ಮಾ-ಡಿ-ಕೊಂ-ಡರು
ಬರ-ಮಾ-ಡಿ-ಕೊಂ-ಡಳು
ಬರ-ಮಾ-ಡಿ-ಕೊಂ-ಡಿ-ರು-ವುದು
ಬರ-ಮಾ-ಡಿ-ಕೊಂಡು
ಬರ-ಮಾ-ಡಿ-ಕೊಳ್ಳ
ಬರ-ಮಾ-ಡಿ-ಕೊ-ಳ್ಳ-ಲಾ-ಯಿತು
ಬರ-ಮಾ-ಡಿ-ಕೊ-ಳ್ಳಲು
ಬರ-ಮಾ-ಡಿ-ಕೊಳ್ಳು
ಬರ-ಮಾ-ಡಿ-ಕೊ-ಳ್ಳುವ
ಬರ-ಮಾ-ಡಿ-ಕೊ-ಳ್ಳು-ವಂತೆ
ಬರ-ಮಾ-ಡಿ-ಕೊ-ಳ್ಳು-ವೆ-ವೆಂಬ
ಬರ-ಲಾ-ಗದ
ಬರ-ಲಾ-ಗ-ದ್ದಕ್ಕೆ
ಬರ-ಲಾಗಿದೆ
ಬರ-ಲಾ-ಗಿ-ರ-ಲಿಲ್ಲ
ಬರ-ಲಾ-ಯಿತು
ಬರ-ಲಾ-ರಂ-ಭಿ-ಸಿದ
ಬರ-ಲಾ-ರಂ-ಭಿ-ಸಿ-ದರು
ಬರ-ಲಾ-ರಂ-ಭಿ-ಸಿ-ದುವು
ಬರ-ಲಾ-ರಂ-ಭಿ-ಸಿ-ದ್ದುವು
ಬರಲಿ
ಬರ-ಲಿದೆ
ಬರ-ಲಿದ್ದ
ಬರ-ಲಿ-ದ್ದಾ-ನೆಂಬ
ಬರ-ಲಿ-ರುವ
ಬರ-ಲಿಲ್ಲ
ಬರ-ಲಿ-ಲ್ಲ-ವಲ್ಲ
ಬರ-ಲಿ-ಲ್ಲ-ವೆಂದು
ಬರಲು
ಬರಲೂ
ಬರಲೇ
ಬರ-ಲೇ-ಬೇ-ಕಾ-ಗಿತ್ತು
ಬರ-ಲೇ-ಬೇ-ಕಾ-ಗು-ತ್ತದೆ
ಬರ-ಲೇ-ಬೇಡ
ಬರ-ವ-ಣಿ-ಗೆ-ಯನ್ನು
ಬರ-ವನ್ನೇ
ಬರ-ವಿ-ಗಾಗಿ
ಬರ-ವಿ-ಗಾ-ಗಿಯೇ
ಬರಹ
ಬರ-ಹ-ಗಾ-ರರು
ಬರ-ಹ-ದಲ್ಲಿ
ಬರ-ಹ-ವನ್ನು
ಬರ-ಹವೂ
ಬರಾ-ಮು-ಲ್ಲ-ವನ್ನು
ಬರಾ-ಮುಲ್ಲಾ
ಬರಾ-ಮು-ಲ್ಲಾದ
ಬರಾ-ಮು-ಲ್ಲಾ-ದಲ್ಲಿ
ಬರಾ-ಮು-ಲ್ಲಾ-ದಿಂದ
ಬರಿ-ಗಾ-ಲಿ-ನಲ್ಲಿ
ಬರಿ-ಗಾ-ಲಿ-ನಲ್ಲೇ
ಬರಿ-ಗಾಲು
ಬರಿಯ
ಬರಿ-ಸುವ
ಬರಿ-ಹೊ-ಟ್ಟೆ-ಗಲ್ಲ
ಬರು
ಬರು-ತ್ತದೆ
ಬರು-ತ್ತ-ದೆಂದು
ಬರು-ತ್ತ-ದೆಂ-ಬು-ದ-ರಲ್ಲಿ
ಬರು-ತ್ತ-ದೆ-ಆಗ
ಬರು-ತ್ತ-ದೆ-ಯ-ಲ್ಲವೆ
ಬರು-ತ್ತ-ದೆಯೋ
ಬರು-ತ್ತಲೆ
ಬರು-ತ್ತಲೇ
ಬರು-ತ್ತವೆ
ಬರು-ತ್ತ-ವೆಯೇ
ಬರು-ತ್ತಾನೆ
ಬರು-ತ್ತಾ-ರಂತೆ
ಬರು-ತ್ತಾರೆ
ಬರು-ತ್ತಾರೋ
ಬರು-ತ್ತಾಳೆ
ಬರುತ್ತಿ
ಬರು-ತ್ತಿತ್ತು
ಬರು-ತ್ತಿದೆ
ಬರು-ತ್ತಿ-ದೆ-ನಾನು
ಬರು-ತ್ತಿದ್ದ
ಬರು-ತ್ತಿ-ದ್ದಂ-ತಿತ್ತು
ಬರು-ತ್ತಿ-ದ್ದಂತೆ
ಬರು-ತ್ತಿ-ದ್ದಂ-ತೆಯೇ
ಬರು-ತ್ತಿ-ದ್ದರು
ಬರು-ತ್ತಿ-ದ್ದ-ವ-ರಲ್ಲಿ
ಬರು-ತ್ತಿ-ದ್ದ-ವ-ರಿ-ಗಂತೂ
ಬರು-ತ್ತಿ-ದ್ದ-ವರೆಲ್ಲ
ಬರು-ತ್ತಿ-ದ್ದಾ-ಗ-ಯಾವ
ಬರು-ತ್ತಿ-ದ್ದಾ-ಗಲೇ
ಬರು-ತ್ತಿ-ದ್ದಾ-ನೆಂಬ
ಬರು-ತ್ತಿ-ದ್ದಾರೆ
ಬರು-ತ್ತಿ-ದ್ದೀರಿ
ಬರು-ತ್ತಿ-ದ್ದು-ದೊಂದು
ಬರು-ತ್ತಿ-ದ್ದುವು
ಬರು-ತ್ತಿ-ದ್ದು-ವೆಂ-ಬುದು
ಬರು-ತ್ತಿದ್ದೆ
ಬರು-ತ್ತಿ-ದ್ದೇವೆ
ಬರು-ತ್ತಿ-ರ-ಲಿಲ್ಲ
ಬರು-ತ್ತಿರು
ಬರು-ತ್ತಿ-ರು-ತ್ತದೆ
ಬರು-ತ್ತಿ-ರು-ತ್ತೇನೆ
ಬರು-ತ್ತಿ-ರುವ
ಬರು-ತ್ತಿ-ರು-ವಾಗ
ಬರು-ತ್ತಿ-ರು-ವುದು
ಬರು-ತ್ತಿಲ್ಲ
ಬರು-ತ್ತೀಯಾ
ಬರು-ತ್ತೀರಾ
ಬರು-ತ್ತೇನೆ
ಬರುವ
ಬರುವಂ
ಬರು-ವಂ-ತಾ-ಗ-ಬೇಕು
ಬರು-ವಂ-ತಾ-ಗಲೂ
ಬರು-ವಂ-ತಾ-ಗು-ತ್ತದೆ
ಬರು-ವಂ-ತಾ-ದದ್ದು
ಬರು-ವಂ-ತಾ-ದರೆ
ಬರು-ವಂ-ತಿ-ರ-ಲಿಲ್ಲ
ಬರು-ವಂ-ತಿಲ್ಲ
ಬರು-ವಂತೆ
ಬರು-ವಂ-ಥದು
ಬರು-ವಂ-ಥವೂ
ಬರು-ವ-ರಾ-ದರೂ
ಬರು-ವ-ವ-ನಲ್ಲ
ಬರು-ವ-ವ-ರೆಗೂ
ಬರು-ವ-ಷ್ಟ-ರಲ್ಲಿ
ಬರು-ವಾಗ
ಬರು-ವಿ-ರೆಂದು
ಬರು-ವು-ದ-ಕ್ಕಿಂತ
ಬರು-ವು-ದಕ್ಕೂ
ಬರು-ವು-ದಕ್ಕೆ
ಬರು-ವುದನ್ನು
ಬರು-ವು-ದನ್ನೇ
ಬರು-ವು-ದ-ರೊಂ-ದಿಗೆ
ಬರು-ವು-ದಾಗಿ
ಬರು-ವು-ದಿಲ್ಲ
ಬರು-ವು-ದಿ-ಲ್ಲ-ವೆಂದು
ಬರು-ವು-ದಿ-ಲ್ಲ-ವೆಂಬ
ಬರು-ವು-ದಿ-ಲ್ಲ-ವೆಂ-ಬುದು
ಬರು-ವು-ದಿ-ಲ್ಲವೋ
ಬರು-ವುದು
ಬರು-ವು-ದುಂಟು
ಬರು-ವು-ದೆಂ-ದರೆ
ಬರು-ವು-ದೆಂ-ಬುದು
ಬರು-ವು-ದೆ-ಲ್ಲವೂ
ಬರೆ
ಬರೆ-ಗಳನ್ನೆಲ್ಲ
ಬರೆದ
ಬರೆ-ದ-ದ್ದಾ-ದರೂ
ಬರೆ-ದ-ದ್ದಿಲ್ಲ
ಬರೆ-ದರು
ಬರೆ-ದ-ರು-ಅಂತೂ
ಬರೆ-ದ-ರು-ಅಯ್ಯೋ
ಬರೆ-ದ-ರು-ಈ-ಚೆಗೆ
ಬರೆ-ದ-ರು-ನನ್ನ
ಬರೆ-ದರೆ
ಬರೆ-ದಳು
ಬರೆದಿ
ಬರೆ-ದಿ-ಟ್ಟಿ-ದ್ದಾರೆ
ಬರೆ-ದಿಟ್ಟು
ಬರೆ-ದಿ-ಟ್ಟುಕೊ
ಬರೆ-ದಿ-ಟ್ಟು-ಕೊಂಡ
ಬರೆ-ದಿ-ಟ್ಟು-ಕೊ-ಳ್ಳ-ಬೇ-ಕೆಂದು
ಬರೆ-ದಿ-ಟ್ಟು-ಕೊ-ಳ್ಳಲು
ಬರೆ-ದಿ-ಡ-ಬೇ-ಕಾದ
ಬರೆ-ದಿ-ಡ-ಬೇ-ಕಾ-ದಂ-ತಹ
ಬರೆ-ದಿದೆ
ಬರೆ-ದಿದ್ದ
ಬರೆ-ದಿ-ದ್ದರು
ಬರೆ-ದಿ-ದ್ದ-ರು-ಅಂತೂ
ಬರೆ-ದಿ-ದ್ದಾರೆ
ಬರೆ-ದಿ-ದ್ದಾಳೆ
ಬರೆ-ದಿ-ದ್ದಿ-ರ-ಬೇಕು
ಬರೆ-ದಿ-ದ್ದೇನೆ
ಬರೆ-ದಿ-ರ-ಬೇಕು
ಬರೆ-ದಿ-ರುವ
ಬರೆ-ದಿ-ರುವೆ
ಬರೆದು
ಬರೆ-ದು-ಕೊಟ್ಟ
ಬರೆ-ದು-ಕೊ-ಡು-ವಂತೆ
ಬರೆ-ದುವು
ಬರೆ-ಯದೇ
ಬರೆ-ಯ-ಬ-ಲ್ಲರೋ
ಬರೆ-ಯ-ಬೇ-ಕಾ-ಗಿಲ್ಲ
ಬರೆ-ಯ-ಬೇ-ಕಾ-ಯಿತು
ಬರೆ-ಯ-ಬೇಕು
ಬರೆ-ಯ-ಲಾ-ಗಿತ್ತು
ಬರೆ-ಯ-ಲಾ-ರಂ-ಭಿ-ಸಿ-ದುವು
ಬರೆ-ಯ-ಲಿಲ್ಲ
ಬರೆ-ಯಲು
ಬರೆ-ಯಲೂ
ಬರೆ-ಯ-ಲೆಂದು
ಬರೆ-ಯ-ಲ್ಪ-ಟ್ಟಿತೇ
ಬರೆ-ಯ-ಲ್ಪ-ಟ್ಟಿ-ರುವ
ಬರೆ-ಯಿತು
ಬರೆಯು
ಬರೆ-ಯುತ್ತ
ಬರೆ-ಯು-ತ್ತಲೇ
ಬರೆ-ಯು-ತ್ತಲೋ
ಬರೆ-ಯು-ತ್ತಾನೆ
ಬರೆ-ಯು-ತ್ತಾರೆ
ಬರೆ-ಯು-ತ್ತಾ-ರೆ-ಅ-ವರೆಲ್ಲ
ಬರೆ-ಯು-ತ್ತಾಳೆ
ಬರೆ-ಯು-ತ್ತಾ-ಳೆ-ಸ್ವಾ-ಮೀಜಿ
ಬರೆ-ಯು-ತ್ತಿದ್ದ
ಬರೆ-ಯು-ತ್ತಿ-ದ್ದರು
ಬರೆ-ಯು-ತ್ತಿ-ದ್ದೇನೆ
ಬರೆ-ಯು-ತ್ತಿ-ರು-ವಾ-ಗಲೇ
ಬರೆ-ಯು-ತ್ತೇನೆ
ಬರೆ-ಯುವ
ಬರೆ-ಯು-ವ-ವನು
ಬರೆ-ಯು-ವ-ಷ್ಟಕ್ಕೇ
ಬರೆ-ಯು-ವು-ದಕ್ಕೆ
ಬರೆ-ಯು-ವು-ದ-ರಲ್ಲಿ
ಬರೆ-ಯು-ವು-ದಾಗಿ
ಬರೆ-ಯು-ವುದು
ಬರೆ-ಸಿದ
ಬರೇ-ಲಿಗೆ
ಬರೇ-ಲಿ-ಯಿಂದ
ಬರೋಡ
ಬರೋ-ಸರ
ಬರೋ-ಸ-ರಿಗೆ
ಬರೋ-ಸರು
ಬರೋ-ಸರೂ
ಬರೋ-ಸ್ರ-ವರು
ಬರ್ಕ್
ಬರ್ದ್ವಾ-ನಿನ
ಬರ್ನ್
ಬರ್ನ್ಹಾ-ರ್ಟ್
ಬರ್ನ್ಹಾ-ರ್ಡ್
ಬರ್ಬರ
ಬಲ
ಬಲ-ಖಾನಾ
ಬಲ-ಗ-ಣ್ಣಿನ
ಬಲಗೈ
ಬಲ-ಗೈ-ಯನ್ನು
ಬಲ-ಗೈ-ಯನ್ನೇ
ಬಲ-ಗೈ-ಯಿಂದ
ಬಲ-ದಿಂದ
ಬಲ-ದಿಂ-ದಲೋ
ಬಲ-ಪ-ಡಿ-ಸ-ಬೇ-ಕಾ-ಗಿದೆ
ಬಲ-ಪ-ಡಿ-ಸ-ಬೇ-ಕಾ-ದರೆ
ಬಲ-ಪ-ಡಿ-ಸಿ-ಕೊ-ಳ್ಳು-ವುದು
ಬಲ-ಪ್ರ-ಯೋಗ
ಬಲ-ಭಾ-ಗ-ದ-ಲ್ಲಿ-ದ್ದ-ವ-ರ-ನ್ನೆಲ್ಲ
ಬಲ-ಭು-ಜದ
ಬಲ-ಯು-ತ-ರಾಗಿ
ಬಲ-ರಾಮ
ಬಲ-ರಾ-ಮ-ಬಾ-ಬು-ವಿನ
ಬಲ-ರಾಮ್
ಬಲ-ವಂ-ತ-ದಿಂದ
ಬಲ-ವಂ-ತ-ವಾಗಿ
ಬಲ-ವ-ತ್ತ-ರ-ವಾದ
ಬಲ-ವ-ನ್ನೊ-ಳ-ಗೊಂ-ಡಿವೆ
ಬಲ-ವಾಗಿ
ಬಲ-ವಾ-ಗಿ-ತ್ತೆಂ-ದರೆ
ಬಲ-ವಾ-ಗಿ-ದ್ದರೂ
ಬಲ-ವಾ-ಗಿ-ದ್ದು-ವೆಂ-ಬು-ದಕ್ಕೆ
ಬಲ-ವಾ-ಗಿಯೇ
ಬಲ-ವಾದ
ಬಲ-ವಿದೆ
ಬಲಾ-ತ್ಕ-ರಿ-ಸಿ-ದರು
ಬಲಾ-ತ್ಕ-ರಿ-ಸಿ-ದ್ದರು
ಬಲಾ-ತ್ಕ-ರಿ-ಸು-ತ್ತದೆ
ಬಲಾ-ತ್ಕಾರ
ಬಲಾ-ತ್ಕಾ-ರಕ್ಕೆ
ಬಲಾ-ತ್ಕಾ-ರ-ವಾಗಿ
ಬಲಿತ
ಬಲಿ-ದಾ-ನ-ವಾ-ಗಲು
ಬಲಿ-ದಾ-ನ-ವಾ-ಗಿ-ಸಿ-ದರು
ಬಲಿ-ಪ-ಶು-ಗಳೇ
ಬಲಿ-ಬಿದ್ದು
ಬಲಿ-ಯನ್ನು
ಬಲಿ-ಯಾ-ಗದೆ
ಬಲಿ-ಯಾ-ಗಲು
ಬಲಿ-ಯಾ-ಗಿದ್ದ
ಬಲಿ-ಯಾ-ಗಿಸಿ
ಬಲಿ-ಯಾ-ಗಿ-ಸೋಣ
ಬಲಿ-ಯಾ-ಗು-ತ್ತಿ-ದ್ದರು
ಬಲಿ-ಯಾ-ಗು-ತ್ತಿ-ರುವ
ಬಲಿ-ಯಾ-ದರೂ
ಬಲಿ-ಯಾ-ದ-ವನ
ಬಲಿಷ್ಠ
ಬಲಿ-ಷ್ಠ-ನಾ-ಗಿ-ದ್ದೇನೆ
ಬಲಿ-ಷ್ಠ-ರನ್ನು
ಬಲಿ-ಷ್ಠ-ರಾಗಿ
ಬಲಿ-ಷ್ಠ-ರಾದ
ಬಲಿ-ಷ್ಠರು
ಬಲಿ-ಷ್ಠರೂ
ಬಲಿ-ಷ್ಠ-ಳಾಗಿ
ಬಲಿ-ಷ್ಠ-ವಾದ
ಬಲು
ಬಲೆ-ಯಲ್ಲಿ
ಬಲ್ಗೇ-ರಿ-ಯ-ಗಳ
ಬಲ್ಲ
ಬಲ್ಲರು
ಬಲ್ಲರೇ
ಬಲ್ಲ-ವ-ನೆಂದು
ಬಲ್ಲ-ವ-ರಾ-ಗಿ-ರ-ಬೇ-ಕೆಂದು
ಬಲ್ಲ-ವ-ರಾರು
ಬಲ್ಲ-ವರು
ಬಲ್ಲವು
ಬಲ್ಲುದೊ
ಬಲ್ಲೆನೆ
ಬಲ್ಲೆಯಾ
ಬಲ್ಲೆವು
ಬಳ-ಕೆಗೆ
ಬಳ-ಕೆ-ಯಲ್ಲಿ
ಬಳಗ
ಬಳಲಿ
ಬಳ-ಲಿ-ಕೆ-ಯುಂ-ಟಾಗಿ
ಬಳ-ಲಿತ್ತು
ಬಳ-ಲಿ-ದ-ರಾ-ದರೂ
ಬಳ-ಲಿದ್ದ
ಬಳ-ಲಿ-ದ್ದ-ರಾ-ದರೂ
ಬಳ-ಲಿ-ದ್ದುವು
ಬಳ-ಲಿ-ದ್ದೇನೆ
ಬಳ-ಲು-ತ್ತಿ-ದ್ದರೂ
ಬಳ-ಲು-ತ್ತಿ-ದ್ದು-ದ-ರಿಂದ
ಬಳ-ಸ-ಬೇಕು
ಬಳ-ಸ-ಲಾ-ಗುವ
ಬಳ-ಸ-ಲಾ-ಗು-ವು-ದೆಂದೂ
ಬಳಸಿ
ಬಳ-ಸಿ-ಕೊಂ-ಡಿ-ದ್ದೇನೆ
ಬಳ-ಸಿ-ಕೊಂಡು
ಬಳ-ಸಿ-ಕೊ-ಳ್ಳ-ಬ-ಹುದು
ಬಳ-ಸಿ-ಕೊ-ಳ್ಳಲೂ
ಬಳ-ಸಿ-ಕೊ-ಳ್ಳುವ
ಬಳ-ಸಿನ
ಬಳಸು
ಬಳ-ಸು-ತ್ತಿ-ದ್ದರು
ಬಳ-ಸು-ತ್ತಿ-ರ-ಲಿಲ್ಲ
ಬಳ-ಸು-ತ್ತೇನೆ
ಬಳ-ಸುವ
ಬಳ-ಸು-ವುದು
ಬಳಿ
ಬಳಿಕ
ಬಳಿ-ಕವೂ
ಬಳಿ-ಕವೇ
ಬಳಿಗೂ
ಬಳಿಗೆ
ಬಳಿ-ದರು
ಬಳಿದು
ಬಳಿ-ದು-ಕೊಂಡ
ಬಳಿ-ದು-ಕೊಂ-ಡರು
ಬಳಿ-ದು-ಕೊಂಡು
ಬಳಿಯ
ಬಳಿ-ಯಲ್ಲಿ
ಬಳಿ-ಯ-ಲ್ಲಿದ್ದ
ಬಳಿ-ಯ-ಲ್ಲಿ-ದ್ದವ
ಬಳಿ-ಯ-ಲ್ಲಿ-ದ್ದ-ವ-ರಿಗೆ
ಬಳಿ-ಯ-ಲ್ಲಿನ
ಬಳಿ-ಯ-ಲ್ಲಿಯೇ
ಬಳಿ-ಯ-ಲ್ಲಿ-ರು-ವಾಗ
ಬಳಿ-ಯಲ್ಲೇ
ಬಳಿ-ಯಿದ್ದ
ಬಳಿ-ಯಿ-ದ್ದ-ವರ
ಬಳಿ-ಯಿ-ದ್ದ-ವ-ರನ್ನು
ಬಳಿ-ಯಿ-ದ್ದ-ವ-ರಾರೂ
ಬಳಿ-ಯಿ-ದ್ದ-ವ-ರಿ-ಗೆಲ್ಲ
ಬಳಿ-ಯಿ-ರ-ಬೇ-ಕೆಂದು
ಬಳಿ-ಯಿ-ರುವ
ಬಳಿಯೇ
ಬಳಿ-ಸಾ-ರಲು
ಬಳಿ-ಸಾರಿ
ಬಳು-ವ-ಳಿ-ಯನ್ನು
ಬಳ್ಳಿ
ಬವಣೆ
ಬವ-ಣೆ-ಯನ್ನು
ಬಸಿದ
ಬಸಿ-ದಾ-ದರೂ
ಬಸಿ-ದಿ-ದ್ದೇನೆ
ಬಸಿ-ದಿ-ರ-ದಿ-ದ್ದರೆ
ಬಸಿದು
ಬಸಿ-ಯಲು
ಬಸಿ-ಯು-ತ್ತಿದ್ದೆ
ಬಸು
ಬಹ
ಬಹಳ
ಬಹ-ಳ-ವಾಗಿ
ಬಹ-ಳ-ವಾ-ಗಿಯೇ
ಬಹ-ಳ-ವಾ-ದ್ದ-ರಿಂದ
ಬಹ-ಳ-ವಿತ್ತು
ಬಹ-ಳವೇ
ಬಹ-ಳಷ್ಟಿ
ಬಹ-ಳ-ಷ್ಟಿದೆ
ಬಹ-ಳ-ಷ್ಟಿ-ದೆ-ಯೆಂಬ
ಬಹ-ಳ-ಷ್ಟಿದ್ದು
ಬಹ-ಳ-ಷ್ಟಿ-ರು-ವಾಗ
ಬಹ-ಳಷ್ಟು
ಬಹಾ-ದ್ದೂ-ರ-ರನ್ನು
ಬಹಾ-ದ್ದೂ-ರರು
ಬಹಿ-ರಂಗ
ಬಹಿ-ರಂ-ಗ-ಪ-ಡಿಸಿ
ಬಹಿ-ರಂ-ಗ-ಪ-ಡಿ-ಸಿ-ರ-ಲಿಲ್ಲ
ಬಹಿ-ರಂ-ಗ-ವಾಗಿ
ಬಹಿ-ರಂ-ಗ-ವಾ-ಗಿ-ರ-ಲಿಲ್ಲ
ಬಹಿ-ರಂ-ಗ-ವಾ-ಯಿತು
ಬಹಿ-ರ್ಮು-ಖ-ಗೊ-ಳಿ-ಸಲು
ಬಹಿ-ಷ್ಕ-ರಿ-ಸ-ಲ್ಪಟ್ಟ
ಬಹಿ-ಷ್ಕಾರ
ಬಹಿ-ಷ್ಕಾ-ರದ
ಬಹಿ-ಷ್ಕೃ-ತ-ರಾ-ಗಿ-ಲ್ಲ-ವೆಂ-ಬು-ದನ್ನು
ಬಹು
ಬಹು-ಕಾಲ
ಬಹು-ಕಾ-ಲದ
ಬಹು-ಕಾ-ಲ-ದಿಂ-ದಲೂ
ಬಹು-ಜನ
ಬಹು-ಜ-ನರ
ಬಹು-ತೇಕ
ಬಹು-ದಾ-ಗಿತ್ತು
ಬಹು-ದಾ-ಗಿದೆ
ಬಹು-ದಾದ
ಬಹು-ದೀರ್ಘ
ಬಹುದು
ಬಹು-ದೂ-ರ-ದಲ್ಲಿ
ಬಹು-ದೆಂದು
ಬಹು-ದೊಡ್ಡ
ಬಹುಧಾ
ಬಹು-ಪಾ-ಲನ್ನು
ಬಹು-ಭಾ-ಗ-ವನ್ನು
ಬಹು-ಮಂದಿ
ಬಹು-ಮ-ಟ್ಟಿಗೆ
ಬಹು-ಮ-ತ-ದಿಂದ
ಬಹು-ಮಾನ
ಬಹು-ಮುಖ
ಬಹು-ಮು-ಖ್ಯ-ವಾದ
ಬಹು-ವ-ರ್ಷ-ಗಳ
ಬಹು-ವಾಗಿ
ಬಹುಶಃ
ಬಹು-ಸಂ-ಖ್ಯಾತ
ಬಹು-ಸಂ-ಖ್ಯಾ-ತ-ರೆಂ-ಬು-ವರ
ಬಹು-ಸಂ-ಖ್ಯೆ-ಯಲ್ಲಿ
ಬಾ
ಬಾಂಗ್ಲಾ-ದೇಶ
ಬಾಂಧ-ವ-ರನ್ನು
ಬಾಂಧ-ವ-ರಲ್ಲೂ
ಬಾಂಧ-ವರು
ಬಾಂಧವ್ಯ
ಬಾಂಧ-ವ್ಯ-ಗಳ
ಬಾಂಧ-ವ್ಯ-ಗಳು
ಬಾಂಧ-ವ್ಯ-ವನ್ನು
ಬಾಂಬಿ-ನಂತೆ
ಬಾಂಬು-ಗಳ
ಬಾಂಬು-ಗಳನ್ನು
ಬಾಂಬು-ಗ-ಳ-ಲ್ಲಿನ
ಬಾಂಬೆ
ಬಾಂಬ್
ಬಾಕಿ
ಬಾಕಿ-ಯಿತ್ತು
ಬಾಕಿ-ಯಿದೆ
ಬಾಗಿ
ಬಾಗಿ-ಕೊಂ-ಡಿತ್ತು
ಬಾಗಿ-ಕೊಂಡು
ಬಾಗಿ-ಲನ್ನು
ಬಾಗಿ-ಲಲ್ಲಿ
ಬಾಗಿ-ಲಿ-ನಲ್ಲೇ
ಬಾಗಿಲು
ಬಾಗಿ-ಲು-ಗಳನ್ನೂ
ಬಾಗಿ-ಲು-ಗಳನ್ನೆಲ್ಲ
ಬಾಗಿ-ಲು-ಗಳಿಂದ
ಬಾಗ್
ಬಾಗ್ಬ-ಜಾ-ರಿ-ನಲ್ಲಿ
ಬಾಘಾ
ಬಾಘಾಗೂ
ಬಾಘಾಗೆ
ಬಾಘ್ಬ-ಜಾ-ರಿಗೆ
ಬಾಜಾ-ಬ-ಜಂ-ತ್ರಿ-ಗಳೂ
ಬಾಜಿ-ಸ-ಲಾ-ಯಿತು
ಬಾಜಿ-ಸಿದ
ಬಾಜಿ-ಸಿ-ದರು
ಬಾಜಿ-ಸುತ್ತ
ಬಾಟ-ಲಿ-ಯನ್ನು
ಬಾಟ-ಲಿ-ಯಲ್ಲಿ
ಬಾಡಿಗೆ
ಬಾಡಿ-ಗೆಗೆ
ಬಾಣ-ಗಳನ್ನು
ಬಾತು-ಕೊಂ-ಡಿ-ದ್ದುವು
ಬಾತು-ಕೋಳಿ
ಬಾತು-ಕೋ-ಳಿ-ಗ-ಳಿಗೂ
ಬಾತು-ಕೋ-ಳಿಗೂ
ಬಾತು-ಕೋ-ಳಿಗೆ
ಬಾತು-ಗಳ
ಬಾತು-ಗಳನ್ನೂ
ಬಾತು-ಗ-ಳಿಗೆ
ಬಾದಾ-ಮಿ-ಗಳಿಂದ
ಬಾಧ-ಕ-ವಾ-ಗ-ಬಾ-ರ-ದ-ಲ್ಲವೆ
ಬಾಧ-ಕ-ವಿಲ್ಲ
ಬಾಧೆ
ಬಾಧ್ಯ-ತೆ-ಗಳನ್ನು
ಬಾನು-ಬು-ವಿ-ಗಳ
ಬಾಬ
ಬಾಬಾ
ಬಾಬಾಜಿ
ಬಾಬಾ-ಜಿ-ಯ-ವರು
ಬಾಬಾರ
ಬಾಬಾ-ರಿಗೆ
ಬಾಬು
ಬಾಬು-ಗಳನ್ನು
ಬಾಬು-ಗ-ಳಿಗೆ
ಬಾಬು-ಗಳು
ಬಾಬು-ಗಳೂ
ಬಾಬು-ವಿನ
ಬಾಮ್
ಬಾಯನ್ನು
ಬಾಯನ್ನೂ
ಬಾಯಲ್ಲಿ
ಬಾಯ-ಲ್ಲೇನೋ
ಬಾಯಾ-ರಿ-ಕೆ-ಯನ್ನು
ಬಾಯಿ
ಬಾಯಿಂದ
ಬಾಯಿಗೆ
ಬಾಯಿ-ತಪ್ಪಿ
ಬಾಯಿ-ಪಾಠ
ಬಾಯಿ-ಮು-ಕ್ಕ-ಳಿ-ಸಲು
ಬಾಯ್ತುಂಬ
ಬಾಯ್ತೆ-ರೆದ
ಬಾಯ್ಮುಚ್ಚಿ
ಬಾರ
ಬಾರ-ತದ
ಬಾರ-ತ-ದಲ್ಲಿ
ಬಾರದ
ಬಾರ-ದಂತೆ
ಬಾರ-ದ-ವ-ರಾ-ಗು-ತ್ತೀರಿ
ಬಾರ-ದ-ವರು
ಬಾರ-ದಿ-ದ್ದಂ-ತಹ
ಬಾರ-ದಿ-ದ್ದ-ರಿಂದ
ಬಾರ-ದಿ-ರಲಿ
ಬಾರ-ದಿ-ರು-ವುದು
ಬಾರದು
ಬಾರದೆ
ಬಾರ-ದೆಂದು
ಬಾರ-ದೆಂ-ಬುದು
ಬಾರದ್ದು
ಬಾರಾ-ನಾ-ಗೋರ್
ಬಾರಿ
ಬಾರಿಗೆ
ಬಾರಿ-ಬಾ-ರಿಗೂ
ಬಾರಿಸ
ಬಾರಿಸಿ
ಬಾರಿ-ಸಿದ
ಬಾರಿ-ಸಿ-ದಾಗ
ಬಾರಿ-ಸುತ್ತ
ಬಾರಿ-ಸುವ
ಬಾಲ
ಬಾಲಕ
ಬಾಲ-ಕ-ನಂತೆ
ಬಾಲ-ಕ-ನಾಗಿ
ಬಾಲ-ಕ-ಬಾ-ಲ-ಕಿ-ಯ-ರಿ-ಗಾಗಿ
ಬಾಲ-ಕ್ರಿ-ಸ್ತನ
ಬಾಲ-ಗಂ-ಗಾ-ಧರ
ಬಾಲ-ಗೋ-ಪಾ-ಲ-ನನ್ನು
ಬಾಲ-ಬು-ದ್ಧಿ-ಯ-ವರೂ
ಬಾಲರ
ಬಾಲ-ವಾ-ಡಿಯ
ಬಾಲ-ವಿ-ಧ-ವೆ-ಯರ
ಬಾಲಾ-ಜಿ-ರಾವ್
ಬಾಲಿ
ಬಾಲಿಯ
ಬಾಲ್ಕನ್
ಬಾಲ್ಕ-ನ್ದೇ-ಶ-ಗಳು
ಬಾಲ್ಟಿ-ಮೋರ್
ಬಾಲ್ಯ
ಬಾಲ್ಯದ
ಬಾಲ್ಯ-ದಿಂದ
ಬಾಲ್ಯ-ದಿಂ-ದಲೂ
ಬಾಲ್ಯ-ದಿಂ-ದಲೇ
ಬಾಲ್ಯ-ವನ್ನು
ಬಾಲ್ಯ-ಸ್ನೇ-ಹಿತ
ಬಾಲ್ಯ-ಸ್ನೇ-ಹಿ-ತ-ನಾದ
ಬಾಲ್ಯ-ಸ್ನೇ-ಹಿ-ತ-ನಿ-ದ್ದ-ಅ-ವನು
ಬಾಲ್ಯಾ-ವ-ಸ್ಥೆಯ
ಬಾಳ
ಬಾಳ-ಕ-ಡ-ಲಿ-ನಾ-ಚೆಗೆ
ಬಾಳನ್ನು
ಬಾಳ-ಬು-ತ್ತಿಯ
ಬಾಳಲು
ಬಾಳಿ
ಬಾಳಿ-ದು-ಡಿ-ದು-ಮ-ಡಿ-ದೆ-ವೆಂಬ
ಬಾಳಿಕೆ
ಬಾಳಿ-ದರು
ಬಾಳಿ-ದರೆ
ಬಾಳಿದ್ದು
ಬಾಳು
ಬಾಳು-ವಂ-ತಾ-ಗಲಿ
ಬಾಳು-ವುದು
ಬಾಳೆ-ಕಂ-ಬ-ತ-ಳಿ-ರು-ತೋ-ರ-ಣ-ಗಳಿಂದ
ಬಾಳೆಯ
ಬಾಳೆ-ಹಣ್ಣು
ಬಾವಿ
ಬಾವುಟ
ಬಾವು-ಟ-ಗಳಿಂದ
ಬಾವು-ಟ-ವನ್ನು
ಬಾವು-ಟವು
ಬಾಷ-ಣ-ಗಳನ್ನು
ಬಾಷೆ-ಯಲ್ಲಿ
ಬಾಸ್ಟ-ನ್-ನ್ಯೂ-ಯಾ-ರ್ಕು-ಗಳ
ಬಾಸ್ಟ-ನ್ನಿಗೆ
ಬಾಹು-ಗಳನ್ನು
ಬಾಹ್ಯ
ಬಾಹ್ಯ-ಜ-ಗ-ತ್ತನ್ನು
ಬಾಹ್ಯ-ಜ-ಗ-ತ್ತಿ-ನಲ್ಲಿ
ಬಾಹ್ಯ-ನೋ-ಟಕ್ಕೆ
ಬಾಹ್ಯ-ಪೂ-ಜೆ-ಗಳನ್ನು
ಬಾಹ್ಯ-ಪೂ-ಜೆ-ಯಾ-ಗಲಿ
ಬಾಹ್ಯ-ಪೂ-ಜೆ-ಯಿ-ಲ್ಲ-ದಂ-ತಹ
ಬಾಹ್ಯ-ಪ್ರ-ಕೃ-ತಿಯ
ಬಾಹ್ಯ-ಪ್ರ-ಜ್ಞೆ-ಯನ್ನೇ
ಬಾಹ್ಯ-ಪ್ರ-ಪಂ-ಚ-ದತ್ತ
ಬಾಹ್ಯ-ಪ್ರ-ಪಂ-ಚ-ದ-ಲ್ಲಲ್ಲ
ಬಾಹ್ಯ-ಪ್ರ-ಪಂ-ಚ-ದಿಂದ
ಬಾಹ್ಯ-ಶ-ಕ್ತಿಯ
ಬಾಹ್ಯ-ಶ-ಕ್ತಿಯೂ
ಬಾಹ್ಯಾ-ಚ-ರ-ಣೆ-ಗಳ
ಬಾಹ್ಯಾ-ಚ-ರ-ಣೆ-ಗಳನ್ನು
ಬಾಹ್ಯಾ-ಚ-ರ-ಣೆ-ಗ-ಳನ್ನೇ
ಬಾಹ್ಯಾ-ಚ-ರ-ಣೆ-ಗಳಲ್ಲಿ
ಬಾಹ್ಯಾ-ಚ-ರ-ಣೆ-ಗ-ಳಿ-ಗೆಲ್ಲ
ಬಾಹ್ಯಾ-ಡಂ-ಬ-ರಕ್ಕೂ
ಬಾಹ್ಯಾ-ಡಂ-ಬ-ರ-ಗಳ
ಬಿಕ್ಕ-ಟ್ಟನ್ನು
ಬಿಕ್ಕುತ್ತ
ಬಿಕ್ಕು-ತ್ತಲೇ
ಬಿಗ-ಡಾ-ಯಿ-ಸಿದ್ದೂ
ಬಿಗಿ-ದಿ-ಟ್ಟು-ಕೊ-ಳ್ಳಲು
ಬಿಗಿ-ದು-ಕೊಂ-ಡಿದ್ದ
ಬಿಗಿ-ದು-ಬಂತು
ಬಿಗಿ-ಯ-ಲ್ಪ-ಟ್ಟಿದೆ
ಬಿಗಿ-ಯಾಗಿ
ಬಿಗಿ-ಯಾದ
ಬಿಗಿಯು
ಬಿಗಿ-ಹಿ-ಡಿ-ದಿ-ಟ್ಟು-ಕೊಂ-ಡಿ-ದ್ದರು
ಬಿಗಿ-ಹಿ-ಡಿದು
ಬಿಚ್ಚ-ಲಿಲ್ಲ
ಬಿಚ್ಚಿ
ಬಿಚ್ಚು-ಮ-ನ-ಸ್ಸಿ-ನಿಂದ
ಬಿಜ್ಬೆ-ಹರ
ಬಿಟಿಸ್
ಬಿಟ್ಟ
ಬಿಟ್ಟ-ಮೇಲೆ
ಬಿಟ್ಟರು
ಬಿಟ್ಟ-ರು-ನೀ-ವೆಲ್ಲ
ಬಿಟ್ಟರೆ
ಬಿಟ್ಟ-ವ-ನಲ್ಲ
ಬಿಟ್ಟ-ವರು
ಬಿಟ್ಟಾ-ಗಿ-ನಿಂ-ದಲೂ
ಬಿಟ್ಟಾ-ರೆಯೇ
ಬಿಟ್ಟಿತು
ಬಿಟ್ಟಿತ್ತು
ಬಿಟ್ಟಿದೆ
ಬಿಟ್ಟಿದ್ದ
ಬಿಟ್ಟಿ-ದ್ದರು
ಬಿಟ್ಟಿ-ದ್ದಾನೆ
ಬಿಟ್ಟಿರಿ
ಬಿಟ್ಟಿಲ್ಲ
ಬಿಟ್ಟು
ಬಿಟ್ಟು-ಕೊಟ್ಟ
ಬಿಟ್ಟು-ಕೊ-ಟ್ಟರು
ಬಿಟ್ಟು-ಕೊ-ಟ್ಟಿ-ದ್ದರು
ಬಿಟ್ಟು-ಕೊಟ್ಟು
ಬಿಟ್ಟು-ಕೊಡ
ಬಿಟ್ಟು-ಕೊ-ಡ-ಬೇ-ಕೆಂ-ಬುದು
ಬಿಟ್ಟು-ಕೊ-ಡ-ಲಾ-ಯಿತು
ಬಿಟ್ಟು-ಕೊ-ಡಲು
ಬಿಟ್ಟು-ಕೊ-ಡು-ವು-ದಕ್ಕೆ
ಬಿಟ್ಟು-ಬಿ-ಟ್ಟರು
ಬಿಟ್ಟು-ಬಿ-ಟ್ಟಿ-ದ್ದಾರೆ
ಬಿಟ್ಟು-ಬಿ-ಡ-ಬೇಕು
ಬಿಟ್ಟು-ಬಿ-ಡ-ಲಾ-ರಂ-ಭಿ-ಸಿ-ದರು
ಬಿಟ್ಟು-ಬಿಡಿ
ಬಿಟ್ಟು-ಬಿಡು
ಬಿಟ್ಟು-ಹೋ-ಗಲಿ
ಬಿಟ್ಟು-ಹೋ-ಗಲು
ಬಿಟ್ಟು-ಹೋ-ಗಿದ್ದ
ಬಿಟ್ಟು-ಹೋ-ಗಿ-ರುವ
ಬಿಟ್ಟು-ಹೋದ
ಬಿಟ್ಟು-ಹೋ-ದರು
ಬಿಟ್ಟೆ
ಬಿಟ್ಟೊ-ಡ-ನೆಯೇ
ಬಿಡ
ಬಿಡ-ದಂ-ತಹ
ಬಿಡ-ದಂತೆ
ಬಿಡ-ದಿ-ದ್ದರೆ
ಬಿಡ-ದಿರಿ
ಬಿಡದೆ
ಬಿಡ-ಬೇಕು
ಬಿಡ-ಲಾ-ರ-ರೆಂದು
ಬಿಡಲಿ
ಬಿಡ-ಲಿಲ್ಲ
ಬಿಡಲು
ಬಿಡಲೇ
ಬಿಡಿ
ಬಿಡಿ-ಗಾ-ಸನ್ನೂ
ಬಿಡಿ-ಸ-ಲಾ-ಗದ
ಬಿಡಿ-ಸ-ಲಾ-ರ-ದಂತೆ
ಬಿಡಿಸಿ
ಬಿಡಿ-ಸಿ-ಕೊಂಡು
ಬಿಡಿ-ಸಿ-ಕೊ-ಳ್ಳ-ಬೇ-ಕಾ-ಗು-ತ್ತದೆ
ಬಿಡಿ-ಸಿ-ಕೊ-ಳ್ಳ-ಬೇ-ಕಾ-ದರೆ
ಬಿಡಿ-ಸಿ-ಕೊ-ಳ್ಳಲು
ಬಿಡಿ-ಸಿ-ಬಿ-ಡು-ತ್ತಿ-ದ್ದರು
ಬಿಡು
ಬಿಡು-ಗಡೆ
ಬಿಡು-ಗ-ಡೆ-ಯನ್ನು
ಬಿಡು-ತ್ತದೆ
ಬಿಡು-ತ್ತವೆ
ಬಿಡು-ತ್ತಾ-ರಂ-ತಲ್ಲ
ಬಿಡು-ತ್ತಿ-ದ್ದರು
ಬಿಡು-ತ್ತಿ-ದ್ದ-ರು-ಎಂ-ದರೆ
ಬಿಡು-ತ್ತಿ-ರ-ಲಿಲ್ಲ
ಬಿಡು-ತ್ತೀ-ರಲ್ಲ
ಬಿಡು-ತ್ತೀರಿ
ಬಿಡು-ತ್ತೇನೆ
ಬಿಡುವ
ಬಿಡು-ವಂ-ತೆಯೂ
ಬಿಡು-ವ-ವ-ನಲ್ಲ
ಬಿಡು-ವ-ವನು
ಬಿಡು-ವ-ವರು
ಬಿಡು-ವ-ವ-ಳಲ್ಲ
ಬಿಡು-ವಿನ
ಬಿಡು-ವಿ-ಲ್ಲ-ದಂತೆ
ಬಿಡು-ವಿ-ಲ್ಲ-ದಷ್ಟು
ಬಿಡು-ವಿ-ಲ್ಲದೆ
ಬಿಡುವು
ಬಿಡು-ವು-ದಕ್ಕೆ
ಬಿಡು-ವು-ದಿಲ್ಲ
ಬಿಡು-ವು-ದಿ-ಲ್ಲ-ವೆಂದು
ಬಿಡು-ವುದೇ
ಬಿತ್ತ-ಬೇ-ಕೆಂದು
ಬಿತ್ತ-ರಿ-ಸಿದ
ಬಿತ್ತ-ರಿ-ಸು-ವು-ದರ
ಬಿತ್ತಲು
ಬಿತ್ತಿ
ಬಿತ್ತಿದ
ಬಿತ್ತಿದ್ದ
ಬಿತ್ತಿ-ದ್ದೇನೆ
ಬಿತ್ತು
ಬಿದ್ದ
ಬಿದ್ದಂ-ತಾ-ಯಿತು
ಬಿದ್ದಂ-ತೆಲ್ಲ
ಬಿದ್ದರೂ
ಬಿದ್ದರೆ
ಬಿದ್ದಾಗ
ಬಿದ್ದಿತು
ಬಿದ್ದಿ-ತೆಂ-ದರೆ
ಬಿದ್ದಿತೊ
ಬಿದ್ದಿತ್ತು
ಬಿದ್ದಿದೆ
ಬಿದ್ದಿದ್ದ
ಬಿದ್ದಿ-ದ್ದರೆ
ಬಿದ್ದಿ-ದ್ದಾರೆ
ಬಿದ್ದಿ-ರ-ಬ-ಹುದು
ಬಿದ್ದಿ-ರು-ತ್ತದೆ
ಬಿದ್ದಿ-ರು-ತ್ತಾರೆ
ಬಿದ್ದಿ-ರು-ವರೋ
ಬಿದ್ದು
ಬಿದ್ದು-ಕೊಂ-ಡಿ-ದ್ದನು
ಬಿದ್ದು-ಕೊಂ-ಡಿ-ದ್ದಾ-ಗಲೇ
ಬಿದ್ದು-ಬಿ-ಟ್ಟರೆ
ಬಿದ್ದುವು
ಬಿದ್ದು-ಹೋಗು
ಬಿದ್ದು-ಹೋ-ಗು-ತ್ತವೆ
ಬಿದ್ದು-ಹೋ-ಯಿತು
ಬಿನ್ನ-ವ-ತ್ತಳೆ
ಬಿನ್ನ-ವ-ತ್ತ-ಳೆ-ಗಳ
ಬಿನ್ನ-ವ-ತ್ತ-ಳೆ-ಗಳನ್ನು
ಬಿನ್ನ-ವ-ತ್ತ-ಳೆಗೆ
ಬಿನ್ನ-ವ-ತ್ತ-ಳೆ-ಯನ್ನು
ಬಿನ್ನ-ವ-ತ್ತ-ಳೆ-ಯ-ನ್ನೋದಿ
ಬಿನ್ನ-ವ-ತ್ತ-ಳೆ-ಯ-ನ್ನೋ-ದಿ-ದರು
ಬಿನ್ನ-ವ-ತ್ತ-ಳೆ-ಯಲ್ಲಿ
ಬಿನ್ನ-ವ-ತ್ತ-ಳೆ-ಯೊಂ-ದನ್ನು
ಬಿನ್ನ-ವ-ತ್ತ-ಳೆ-ಯೋ-ದಿ-ದರು
ಬಿನ್ನ-ಹಕ್ಕೂ
ಬಿನ್ನಿ
ಬಿರಿ-ಯು-ವು-ದ-ರ-ಲ್ಲಿದೆ
ಬಿರು-ಕು-ಗಳ
ಬಿರು-ಗಾಳಿ
ಬಿರು-ಗಾ-ಳಿ-ಮ-ಳೆ-ಗಳ
ಬಿರು-ಗಾ-ಳಿ-ಗೊಡ್ಡಿ
ಬಿರು-ಗಾ-ಳಿಯೇ
ಬಿರು-ಸಾ-ಯಿತು
ಬಿಲ್ಡಿಂಗ್
ಬಿಲ್ವ-ವೃ-ಕ್ಷದ
ಬಿಳಿ-ಕೂ-ದ-ಲು-ಗಳು
ಬಿಳಿ-ಗಿರಿ
ಬಿಳಿ-ಚ-ರ್ಮ-ದ-ವರು
ಬಿಳಿ-ಚ-ರ್ಮ-ದ-ವ-ರೆಂ-ದರೆ
ಬಿಳಿಯ
ಬಿಳಿ-ಯ-ಗಡ್ಡ
ಬಿಳಿ-ಯನ
ಬಿಳಿ-ಯ-ರನ್ನು
ಬಿಳಿ-ಯ-ರಿಗೇ
ಬಿಳಿ-ಯರು
ಬಿಳುಪು
ಬಿಷ-ಪ್ಪರ
ಬಿಸಿ
ಬಿಸಿ-ಬಿಸಿ
ಬಿಸಿ-ಯಾ-ಗಿತ್ತು
ಬಿಸಿ-ಯಾ-ಗು-ತ್ತದೆ
ಬಿಸಿ-ಯಿಂದ
ಬಿಸಿ-ಯೇ-ರಿದ
ಬಿಸಿ-ರ-ಕ್ತ-ವನ್ನು
ಬಿಸಿ-ಲಿಗೆ
ಬಿಸಿ-ಲಿನ
ಬಿಸಿ-ಲಿ-ನಲ್ಲಿ
ಬಿಸಿ-ಲಿ-ನಲ್ಲೂ
ಬಿಸಿಲು
ಬಿಸಿ-ಲೇ-ರಿತ್ತು
ಬಿಸಿ-ಲೇ-ರಿ-ದಾಗ
ಬಿಸಿ-ಲೇ-ರು-ತ್ತಿತ್ತು
ಬಿಸು-ಡ-ಬೇಕು
ಬಿಹಾರ್
ಬೀಗು-ತ್ತಿತ್ತು
ಬೀಚ್ನಲ್ಲಿ
ಬೀಜ
ಬೀಜ-ಗಳನ್ನು
ಬೀಜ-ಗಳು
ಬೀಜ-ರೂ-ಪ-ದಲ್ಲಿ
ಬೀಜ-ವ-ನ್ನಂತೂ
ಬೀಜ-ವನ್ನು
ಬೀಜ-ವೊಂದು
ಬೀಡಾ
ಬೀಡಿ
ಬೀಡು
ಬೀಡು-ಬಿ-ಟ್ಟರು
ಬೀದಿ-ಗಳಲ್ಲಿ
ಬೀದಿ-ಯಲ್ಲಿ
ಬೀರ-ಬ-ಲ್ಲು-ದೆಂದು
ಬೀರ-ಬ-ಲ್ಲುವು
ಬೀರಲು
ಬೀರಿತು
ಬೀರಿದ
ಬೀರಿ-ದುವು
ಬೀರಿದೆ
ಬೀರಿ-ದೆ-ಯೆಂ-ಬು-ದನ್ನು
ಬೀರಿ-ದ್ದುವು
ಬೀರಿ-ರ-ಬೇಕು
ಬೀರುತ್ತ
ಬೀರು-ವಂ-ತಹ
ಬೀರು-ವು-ದ-ಕ್ಕಾಗಿ
ಬೀಳ-ದಂ-ದದಿ
ಬೀಳ-ದಿ-ರುವ
ಬೀಳದೆ
ಬೀಳ-ಬ-ಹುದೆ
ಬೀಳಲಿ
ಬೀಳಲು
ಬೀಳ-ಲೇ-ಬಾ-ರದು
ಬೀಳಿ-ಸಲು
ಬೀಳು-ತ್ತಂತೆ
ಬೀಳು-ತ್ತದೋ
ಬೀಳು-ತ್ತಲೇ
ಬೀಳು-ತ್ತವೆ
ಬೀಳು-ತ್ತಿ-ದ್ದಂ-ತೆಯೇ
ಬೀಳು-ತ್ತಿ-ದ್ದವು
ಬೀಳು-ತ್ತಿ-ದ್ದುದೇ
ಬೀಳು-ತ್ತಿ-ರ-ಲಿಲ್ಲ
ಬೀಳು-ತ್ತಿ-ರು-ವುದು
ಬೀಳು-ತ್ತಿವೆ
ಬೀಳು-ವಂ-ತಾ-ಗಲಿ
ಬೀಳು-ವಂ-ತಾ-ಗಿತ್ತು
ಬೀಳು-ವ-ವರೆ-ಗಿನ
ಬೀಳುವೆ
ಬೀಳ್ಕೊಂಡ
ಬೀಳ್ಕೊಂ-ಡ-ದ್ದನ್ನು
ಬೀಳ್ಕೊಂ-ಡರು
ಬೀಳ್ಕೊಂಡು
ಬೀಳ್ಕೊಟ್ಟ
ಬೀಳ್ಕೊ-ಟ್ಟರು
ಬೀಳ್ಕೊ-ಟ್ಟಳು
ಬೀಳ್ಕೊ-ಡ-ಬೇ-ಕಾ-ಯಿತು
ಬೀಳ್ಕೊ-ಡಲು
ಬೀಳ್ಕೊ-ಡುಗೆ
ಬೀಳ್ಕೊ-ಡು-ಗೆಯ
ಬೀಳ್ಕೊ-ಳ್ಳುತ್ತ
ಬೀಳ್ಗೊಂ-ಡರು
ಬೀಳ್ಗೊಂಡು
ಬೀಳ್ಗೊ-ಳ್ಳು-ತ್ತಿ-ದ್ದಾರೆ
ಬೀಳ್ಗೊ-ಳ್ಳುವ
ಬೀಸ
ಬೀಸ-ಬೇ-ಕಾ-ಯಿತು
ಬೀಸಿ
ಬೀಸಿತು
ಬೀಸಿ-ದರೂ
ಬೀಸಿ-ದಳು
ಬೀಸಿ-ದ್ದ-ರಿಂ-ದಲೋ
ಬೀಸುತ್ತ
ಬೀಸು-ತ್ತ-ದೆಯೊ
ಬೀಸು-ತ್ತಲೇ
ಬೀಸು-ತ್ತಿದೆ
ಬೀಸು-ತ್ತಿದ್ದ
ಬೀಸು-ತ್ತಿ-ರುವ
ಬೀಸುವ
ಬುಡ-ಕ-ಟ್ಟಿಗೆ
ಬುಡ-ದ-ಲ್ಲಾ-ದರೂ
ಬುಡ-ದಲ್ಲಿ
ಬುಡ-ಮೇಲು
ಬುಡ-ವನ್ನೇ
ಬುಡ-ಸ-ಮೇತ
ಬುದು
ಬುದ್ದಿ-ವಂ-ತಿ-ಕೆ-ಯ-ನ್ನೆಲ್ಲ
ಬುದ್ಧ
ಬುದ್ಧ-ಯ-ಶೋ-ಧ-ರೆ-ಯ-ರಿಗೆ
ಬುದ್ಧ-ಶಂ-ಕರ
ಬುದ್ಧ-ಗಯೆ
ಬುದ್ಧ-ಗ-ಯೆಗೆ
ಬುದ್ಧ-ಗ-ಯೆಗೇ
ಬುದ್ಧ-ಗ-ಯೆಯ
ಬುದ್ಧ-ಗ-ಯೆ-ಯಲ್ಲಿ
ಬುದ್ಧ-ಗ-ಯೆ-ಯಲ್ಲೂ
ಬುದ್ಧ-ತ್ವ-ವನ್ನು
ಬುದ್ಧನ
ಬುದ್ಧ-ನ-ನ್ನಾಗಿ
ಬುದ್ಧ-ನನ್ನು
ಬುದ್ಧ-ನಲ್ಲಿ
ಬುದ್ಧ-ನಾ-ಗಲಿ
ಬುದ್ಧ-ನಾ-ಗು-ತ್ತಾನೆ
ಬುದ್ಧ-ನಿ-ಗಿಂತ
ಬುದ್ಧನು
ಬುದ್ಧನೊ
ಬುದ್ಧ-ಭ-ಗ-ವಂ-ತನ
ಬುದ್ಧ-ಭ-ಗ-ವಂ-ತ-ನನ್ನು
ಬುದ್ಧ-ಭ-ಗ-ವಂ-ತನೂ
ಬುದ್ಧಿ
ಬುದ್ಧಿ-ಹೃ-ದ-ಯ-ಗಳ
ಬುದ್ಧಿ-ಗಂ-ಟಿದ
ಬುದ್ಧಿಗೂ
ಬುದ್ಧಿಗೆ
ಬುದ್ಧಿ-ಗೆ-ಟ್ಟಂ-ತಾಗಿ
ಬುದ್ಧಿ-ಜೀವಿ
ಬುದ್ಧಿ-ಜೀ-ವಿ-ಗ-ಳೆ-ನ್ನಿ-ಸಿ-ಕೊಂ-ಡ-ವರ
ಬುದ್ಧಿ-ಪ್ರ-ಧಾನ
ಬುದ್ಧಿ-ಮತ್ತೆ
ಬುದ್ಧಿ-ಮ-ತ್ತೆ-ಗ-ಳನ್ನೇ
ಬುದ್ಧಿ-ಮ-ತ್ತೆ-ಯಿತ್ತು
ಬುದ್ಧಿ-ಮ-ತ್ತೆಯೂ
ಬುದ್ಧಿ-ಮ-ತ್ತೆ-ಯೆಂ-ಬುದು
ಬುದ್ಧಿ-ಯನ್ನು
ಬುದ್ಧಿ-ಯನ್ನೂ
ಬುದ್ಧಿ-ಯಲ್ಲಿ
ಬುದ್ಧಿಯು
ಬುದ್ಧಿಯೋ
ಬುದ್ಧಿ-ವಂತ
ಬುದ್ಧಿ-ವಂ-ತ-ರಿಗೆ
ಬುದ್ಧಿ-ವಂ-ತ-ರೆಂದು
ಬುದ್ಧಿ-ವಂ-ತಳೂ
ಬುದ್ಧಿ-ವಂ-ತಿ-ಕೆ-ಯಲ್ಲ
ಬುದ್ಧಿ-ವಂ-ತಿ-ಕೆ-ಯಲ್ಲಿ
ಬುದ್ಧಿ-ವಾದ
ಬುದ್ಧಿ-ಶ-ಕ್ತಿ-ಯನ್ನು
ಬುದ್ಧಿ-ಶ-ಕ್ತಿ-ಯಾ-ಗಲಿ
ಬುಧ-ವಾರ
ಬುಲ್
ಬುಲ್ಗೆ
ಬುಲ್ಲ-ಳನ್ನು
ಬುಲ್ಲ-ಳಿಗೆ
ಬುಲ್ಳಿಗೆ
ಬುಲ್ಳೊಂ-ದಿಗೆ
ಬುಸು-ಗು-ಟ್ಟ-ಬೇಕು
ಬುಹಶಃ
ಬೂಕ-ಳೊಂ-ದಿಗೆ
ಬೂಕ್
ಬೂಟಾ-ಟಿಕೆ
ಬೂದಿ
ಬೂದಿ-ಯನ್ನು
ಬೂದಿ-ಯಾ-ಗ-ಬೇಕು
ಬೂದಿ-ಯಾ-ಗ-ಲಿದೆ
ಬೂದು
ಬೃಹತ್
ಬೃಹ-ತ್ತರ
ಬೃಹ-ತ್ತಾದ
ಬೃಹ-ದಾ-ಕಾ-ರ-ವಾಗಿ
ಬೃಹದ್
ಬೃಹ-ದ್ವಿ-ಚಾರ
ಬೆಂಕಿ
ಬೆಂಕಿಗೆ
ಬೆಂಕಿ-ಯಂತೆ
ಬೆಂಕಿ-ಯನ್ನು
ಬೆಂಕಿ-ಯಲ್ಲಿ
ಬೆಂಕಿಯು
ಬೆಂಕಿ-ಯುಂಡೆ
ಬೆಂಕಿಯೇ
ಬೆಂಗ-ಳೂ-ರಿ-ನಲ್ಲಿ
ಬೆಂಗ-ಳೂರು
ಬೆಂಚಿನ
ಬೆಂಚಿ-ನೆ-ಡೆಗೆ
ಬೆಂಜ-ಮಿನ್
ಬೆಂಡಾಗಿ
ಬೆಂದ
ಬೆಂದಿದ್ದ
ಬೆಂದು-ಹೋ-ಗು-ವಂತೆ
ಬೆಂಬಲ
ಬೆಂಬ-ಲ-ವನ್ನು
ಬೆಂಬ-ಲ-ವಾಗಿ
ಬೆಂಬ-ಲ-ವಾ-ಗಿ-ರುವ
ಬೆಂಬ-ಲ-ವಿದೆ
ಬೆಂಬ-ಲವು
ಬೆಂಬ-ಲಿಗ
ಬೆಂಬ-ಲಿ-ಗನೂ
ಬೆಂಬ-ಲಿ-ಗ-ರಾ-ಗಲಿ
ಬೆಂಬ-ಲಿ-ಗ-ರಾಗಿ
ಬೆಂಬ-ಲಿ-ಗರೂ
ಬೆಂಬ-ಲಿ-ಗ-ಳಾ-ಗಿದ್ದ
ಬೆಂಬ-ಲಿಸು
ಬೆಂಬ-ಲಿ-ಸು-ತ್ತಿ-ದ್ದರು
ಬೆಂಬ-ಲಿ-ಸು-ವ-ವನು
ಬೆಕ್ಕಸ
ಬೆಕ್ಕ-ಸ-ಬೆ-ರ-ಗಾ-ಗ-ದಿ-ರ-ಲಾ-ರೆವು
ಬೆಕ್ಕ-ಸ-ಬೆ-ರ-ಗಾ-ಗ-ದಿ-ರ-ಲಿಲ್ಲ
ಬೆಚ್ಚ-ಗಿ-ಡ-ಲೆಂದು
ಬೆಚ್ಚ-ಗಿದ್ದು
ಬೆಚ್ಚ-ಗಿ-ರು-ತ್ತದೆ
ಬೆಚ್ಚಿ-ಬಿ-ದ್ದಳು
ಬೆಟ್ಟ
ಬೆಟ್ಟಕ್ಕೆ
ಬೆಟ್ಟ-ಗಳ
ಬೆಟ್ಟ-ಗಳಿಂದ
ಬೆಟ್ಟ-ಗು-ಡ್ಡ-ಗಳು
ಬೆಟ್ಟದ
ಬೆಟ್ಟ-ದೆ-ತ್ತ-ರದ
ಬೆಡ-ಗಿ-ನಿಂದ
ಬೆಡಗು
ಬೆಣ-ಚು-ಕ-ಲ್ಲು-ಗಳನ್ನು
ಬೆಣ್ಣೆಯ
ಬೆಣ್ಣೆ-ಯಂತೆ
ಬೆದ-ರದೆ
ಬೆದರಿ
ಬೆದ-ರಿಕೆ
ಬೆದ-ರಿದ
ಬೆದ-ರಿ-ಸ-ಲಾ-ಗಿ-ದೆ-ಯೆಂ-ದರೆ
ಬೆನ್ನ
ಬೆನ್ನಟ್ಟಿ
ಬೆನ್ನು
ಬೆನ್ನು-ಹತ್ತ
ಬೆನ್ನೆ-ಲು-ಬನ್ನೇ
ಬೆನ್ನೆ-ಲು-ಬಿ-ನಂ-ತಿ-ರುವ
ಬೆನ್ನೆ-ಲು-ಬಿ-ನಿಂದ
ಬೆನ್ನೆ-ಲುಬು
ಬೆನ್ನೇರಿ
ಬೆಪ್ಪಾಗಿ
ಬೆರ-ಗ-ನ್ನುಂ-ಟು-ಮಾ-ಡಲು
ಬೆರ-ಗಾ-ಗ-ದಿ-ರು-ವಂ-ತೆಯೇ
ಬೆರ-ಗಾಗಿ
ಬೆರ-ಗಾ-ಗಿ-ಸು-ವಂ-ತಹ
ಬೆರ-ಗಾ-ಗಿ-ಹೋದೆ
ಬೆರ-ಗಾ-ಗು-ತ್ತಿ-ದ್ದರು
ಬೆರ-ಗಾ-ದರು
ಬೆರ-ಗು-ಗ-ಣ್ಣಿ-ನಿಂದ
ಬೆರ-ಗು-ಗೊ-ಳಿ-ಸಲೂ
ಬೆರ-ಗು-ಗೊ-ಳಿ-ಸುವ
ಬೆರಳು
ಬೆರ-ಳೆ-ಣಿ-ಕೆ-ಯಷ್ಟು
ಬೆರೆತ
ಬೆರೆ-ತ-ವ-ರಾ-ಗಿ-ದ್ದರೂ
ಬೆರೆ-ತಿತ್ತು
ಬೆರೆ-ತಿರು
ಬೆರೆತು
ಬೆರೆಯ
ಬೆರೆ-ಯು-ತ್ತಿ-ದ್ದ-ರೆಂದೂ
ಬೆರೆ-ಯುವ
ಬೆರೆ-ಯು-ವಂ-ತಿಲ್ಲ
ಬೆರೆ-ಯು-ವಿಕೆ
ಬೆರೆಸಿ
ಬೆಲೆ
ಬೆಲೆ-ಯಿದೆ
ಬೆಲೆ-ಯಿಲ್ಲ
ಬೆಲೆ-ಯೆ-ಷ್ಟಾ-ಗು-ತ್ತದೆ
ಬೆಲ್
ಬೆಳ
ಬೆಳಕ
ಬೆಳ-ಕದು
ಬೆಳ-ಕನ್ನು
ಬೆಳ-ಕ-ನ್ನೆಲ್ಲ
ಬೆಳ-ಕಿಗೆ
ಬೆಳ-ಕಿನ
ಬೆಳ-ಕಿ-ನಲ್ಲಿ
ಬೆಳಕು
ಬೆಳ-ಗ-ಬ-ಹುದು
ಬೆಳ-ಗಲಿ
ಬೆಳಗಾ
ಬೆಳ-ಗಾ-ಗಲಿ
ಬೆಳ-ಗಾ-ಗು-ತ್ತಿ-ದ್ದಂ-ತೆಯೇ
ಬೆಳಗಿ
ಬೆಳ-ಗಿತು
ಬೆಳ-ಗಿ-ದರು
ಬೆಳ-ಗಿನ
ಬೆಳ-ಗಿ-ನಿಂದ
ಬೆಳ-ಗಿ-ಸ-ಲಾ-ಗಿತ್ತು
ಬೆಳಗು
ಬೆಳ-ಗು-ತ್ತಾನೆ
ಬೆಳ-ಗು-ತ್ತಿತ್ತು
ಬೆಳ-ಗು-ತ್ತಿ-ದೆಯೋ
ಬೆಳ-ಗು-ತ್ತಿದ್ದ
ಬೆಳ-ಗು-ತ್ತಿ-ರು-ವ-ಳಲ್ಲ
ಬೆಳ-ಗು-ತ್ತಿ-ರು-ವುದನ್ನು
ಬೆಳ-ಗುವ
ಬೆಳ-ಗು-ವಂ-ತಾ-ಗಲಿ
ಬೆಳಗ್ಗೆ
ಬೆಳ-ದಿಂ-ಗ-ಳಂತೂ
ಬೆಳ-ವ-ಣಿಗೆ
ಬೆಳ-ವ-ಣಿ-ಗೆಗೆ
ಬೆಳ-ವ-ಣಿ-ಗೆಯ
ಬೆಳ-ವ-ಣಿ-ಗೆ-ಯತ್ತ
ಬೆಳ-ವ-ಣಿ-ಗೆ-ಯನ್ನು
ಬೆಳ-ವ-ಣಿ-ಗೆ-ಯಿಂದ
ಬೆಳಸಿ
ಬೆಳಿಗ್ಗೆ
ಬೆಳೆದ
ಬೆಳೆ-ದರು
ಬೆಳೆ-ದ-ವಳು
ಬೆಳೆ-ದಿದೆ
ಬೆಳೆ-ದಿದ್ದ
ಬೆಳೆ-ದಿ-ರು-ವಷ್ಟು
ಬೆಳೆ-ದಿಲ್ಲ
ಬೆಳೆ-ದಿ-ಲ್ಲ-ವೇಕೆ
ಬೆಳೆದು
ಬೆಳೆ-ದು-ನಿಂತ
ಬೆಳೆ-ದು-ಬಂ-ದ-ವನು
ಬೆಳೆ-ದು-ಬಂ-ದಿತು
ಬೆಳೆ-ದು-ಬಂ-ದಿದೆ
ಬೆಳೆ-ಯ-ತೊ-ಡ-ಗಿ-ದುವು
ಬೆಳೆ-ಯನ್ನು
ಬೆಳೆ-ಯ-ಬೇಕು
ಬೆಳೆ-ಯ-ಬೇ-ಕೆಂದು
ಬೆಳೆ-ಯಲು
ಬೆಳೆ-ಯಿತು
ಬೆಳೆ-ಯಿಸಿ
ಬೆಳೆ-ಯುತ್ತ
ಬೆಳೆ-ಯು-ತ್ತದೆ
ಬೆಳೆ-ಯು-ತ್ತಿ-ದ್ದಂತೆ
ಬೆಳೆ-ಯು-ತ್ತಿ-ರುವ
ಬೆಳೆ-ಯುವ
ಬೆಳೆ-ಸ-ಬ-ಹು-ದಾ-ಗಿತ್ತು
ಬೆಳೆ-ಸ-ಬೇಕು
ಬೆಳೆ-ಸ-ಬೇ-ಕೆಂ-ದಿ-ದ್ದೇನೆ
ಬೆಳೆ-ಸಿ-ಕೊಂ-ಡಿ-ದ್ದಾ-ರೆಯೇ
ಬೆಳೆ-ಸಿ-ಕೊಂಡು
ಬೆಳೆ-ಸಿ-ಕೊಂ-ಡು-ಬಿ-ಟ್ಟಿ-ದ್ದೇವೆ
ಬೆಳೆ-ಸಿ-ಕೊ-ಳ್ಳ-ಬೇ-ಕಾ-ಗಿ-ದೆಯೋ
ಬೆಳೆ-ಸಿ-ಕೊ-ಳ್ಳ-ಬೇಕು
ಬೆಳೆ-ಸಿ-ಕೊಳ್ಳಿ
ಬೆಳೆ-ಸಿ-ಕೊ-ಳ್ಳು-ವಂತೆ
ಬೆಳೆ-ಸಿ-ದಂ-ಥದು
ಬೆಳೆ-ಸಿ-ದರು
ಬೆಳೆ-ಸಿ-ದ್ದು-ದ-ರಿಂದ
ಬೆಳೆ-ಸು-ತ್ತಿದ್ದ
ಬೆಳೆ-ಸುವ
ಬೆಳೆ-ಸು-ವು-ದು-ಇವೇ
ಬೆಳ್ದಿಂ-ಗಳ
ಬೆಳ್ಳ-ಗಾ-ಗ-ತೊ-ಡ-ಗಿತ್ತು
ಬೆಳ್ಳ-ಗಾ-ಗಿ-ರ-ಲಿಲ್ಲ
ಬೆಳ್ಳ-ಗಾ-ಗಿ-ಸ-ಬೇಕು
ಬೆಳ್ಳಗೆ
ಬೆಳ್ಳಿಯ
ಬೆವ-ರಿನ
ಬೆಸಂ-ಟರು
ಬೆಸಿ-ಲಿ-ಕಾ-ಗ-ಳೆಂದು
ಬೆಸೆಂ-ಟರ
ಬೆಸೆಂ-ಟ-ರನ್ನು
ಬೆಸೆಂ-ಟರು
ಬೆಸೆಂಟ್
ಬೆಸೆ-ಯ-ಬಲ್ಲ
ಬೆಸೆ-ಯುವು
ಬೆಸ್ತ-ನಾಗು
ಬೆಸ್ತ-ರ-ವನು
ಬೆಸ್ತ-ರ-ವಳ
ಬೆಸ್ಸಿ
ಬೇ
ಬೇಕಂತೆ
ಬೇಕಾ
ಬೇಕಾ-ಗಿತ್ತು
ಬೇಕಾ-ಗಿದೆ
ಬೇಕಾ-ಗಿ-ದ್ದರು
ಬೇಕಾ-ಗಿ-ದ್ದಾರೆ
ಬೇಕಾ-ಗಿರು
ಬೇಕಾ-ಗಿ-ರುವ
ಬೇಕಾ-ಗಿ-ರು-ವುದು
ಬೇಕಾ-ಗಿಲ್ಲ
ಬೇಕಾ-ಗಿವೆ
ಬೇಕಾಗು
ಬೇಕಾ-ಗು-ತ್ತಿ-ದ್ದುದು
ಬೇಕಾ-ಗುವ
ಬೇಕಾ-ಗು-ವಂ-ತಹ
ಬೇಕಾ-ಗು-ವಷ್ಟು
ಬೇಕಾದ
ಬೇಕಾ-ದಂ-ತಹ
ಬೇಕಾ-ದಂತೆ
ಬೇಕಾ-ದದ್ದು
ಬೇಕಾ-ದ-ದ್ದೇ-ನಿದೆ
ಬೇಕಾ-ದರೂ
ಬೇಕಾ-ದರೆ
ಬೇಕಾ-ದ-ವರು
ಬೇಕಾ-ದ-ಷ್ಟಾ-ಯಿತು
ಬೇಕಾ-ದಷ್ಟು
ಬೇಕಾ-ದು-ದ-ನ್ನೆಲ್ಲ
ಬೇಕಾ-ದುದು
ಬೇಕಾ-ದು-ದು-ಪ್ರ-ಚಂಡ
ಬೇಕಾ-ಬಿಟ್ಟಿ
ಬೇಕಾ-ಯಿತು
ಬೇಕಿತ್ತು
ಬೇಕಿ-ದ್ದರೆ
ಬೇಕಿಲ್ಲ
ಬೇಕು
ಬೇಕು-ಇದೇ
ಬೇಕು-ಒ-ಟ್ಟಿ-ನಲ್ಲಿ
ಬೇಕು-ಬೇ-ಕೆಂದೇ
ಬೇಕೆ
ಬೇಕೆಂ-ದಿ-ದ್ದೇನೆ
ಬೇಕೆಂದು
ಬೇಕೆಂದೇ
ಬೇಕೆಂಬ
ಬೇಕೆಂ-ಬವ
ಬೇಕೆಂ-ಬಷ್ಟು
ಬೇಕೆಂ-ಬು-ದನ್ನು
ಬೇಕೆಂ-ಬುದು
ಬೇಕೆಂ-ಬುದೇ
ಬೇಕೆಂ-ಬು-ವರ
ಬೇಕೆಂ-ಬು-ವ-ರಿಗೆ
ಬೇಕೇ
ಬೇಕೇ-ಬೇ-ಕ-ಲ್ಲವೆ
ಬೇಕೇ-ಬೇಕು
ಬೇಗ
ಬೇಗನೆ
ಬೇಗನೇ
ಬೇಗ-ಬೇಗ
ಬೇಗೆ-ಯಲ್ಲಿ
ಬೇಟಿ
ಬೇಟೆ-ಯ-ಲ್ಲಿನ
ಬೇಡ
ಬೇಡಲು
ಬೇಡ-ಲೆಂದೇ
ಬೇಡವೇ
ಬೇಡಿ
ಬೇಡಿಕೆ
ಬೇಡಿ-ಕೆಗೆ
ಬೇಡಿ-ಕೆಯ
ಬೇಡಿ-ಕೆ-ಯಂತೂ
ಬೇಡಿ-ಕೆ-ಯನ್ನು
ಬೇಡಿ-ಕೆ-ಯ-ನ್ನು-ಅದು
ಬೇಡಿ-ಕೊಂಡ
ಬೇಡಿ-ಕೊಂ-ಡರು
ಬೇಡಿ-ಕೊಂ-ಡರೂ
ಬೇಡಿ-ಕೊಂ-ಡಾಗ
ಬೇಡಿ-ಕೊ-ಳ್ಳು-ತ್ತಿ-ದ್ದ-ಳು-ತಮ್ಮ
ಬೇಡಿ-ಕೊ-ಳ್ಳು-ವು-ದಲ್ಲ
ಬೇಡಿದ್ದೆ
ಬೇಡಿ-ಯಾ-ದರೂ
ಬೇಡು-ವು-ದಲ್ಲ
ಬೇಯಿ-ಸಿ-ಕೊಂಡು
ಬೇರಾರ
ಬೇರಾ-ರಲ್ಲೂ
ಬೇರಾವ
ಬೇರಾ-ವುದೋ
ಬೇರು
ಬೇರು-ಬಿ-ಟ್ಟಿ-ರುವ
ಬೇರು-ಸ-ಮೇತ
ಬೇರು-ಸ-ಹಿತ
ಬೇರೂರಿ
ಬೇರೂ-ರಿತ್ತು
ಬೇರೂ-ರಿದೆ
ಬೇರೂ-ರಿದ್ದ
ಬೇರೂ-ರಿ-ದ್ದುವು
ಬೇರೂ-ರು-ತ್ತಿವೆ
ಬೇರೂ-ರು-ವಂ-ತಾ-ಗು-ತ್ತದೆ
ಬೇರೆ
ಬೇರೆ-ಡೆಗೆ
ಬೇರೆ-ಡೆ-ಯಿಂದ
ಬೇರೆ-ಬೇರೆ
ಬೇರೆ-ಯಾ-ಗ-ಬ-ಹುದು
ಬೇರೆಯೇ
ಬೇರೆಲ್ಲ
ಬೇರೆ-ಲ್ಲರಿ
ಬೇರೆಲ್ಲೂ
ಬೇರೇ-ನ-ನ್ನಾ-ದರೂ
ಬೇರೇ-ನನ್ನು
ಬೇರೇ-ನನ್ನೂ
ಬೇರೇ-ನಿಲ್ಲ
ಬೇರೇನು
ಬೇರೇನೂ
ಬೇರೊಂದು
ಬೇರೊಂದೂ
ಬೇರೊಂ-ದೆ-ಡೆ-ಯಲ್ಲಿ
ಬೇರ್ಪಟ್ಟು
ಬೇರ್ಪ-ಡಿಸಿ
ಬೇರ್ಪ-ಡಿ-ಸು-ವುದು
ಬೇಲಿ
ಬೇಲಿಯೂ
ಬೇಲೂ-ರಿಗೆ
ಬೇಲೂ-ರಿನ
ಬೇಲೂ-ರಿ-ನಲ್ಲಿ
ಬೇಲೂ-ರಿ-ನಿಂದ
ಬೇಲೂರು
ಬೇಸರ
ಬೇಸ-ರದ
ಬೇಸ-ರ-ವಾ-ದರೆ
ಬೇಸ-ರ-ವಿ-ರ-ಲಿಲ್ಲ
ಬೇಸ-ರ-ವುಂ-ಟು-ಮಾ-ಡಿತು
ಬೇಸ-ರಿ-ಸದೆ
ಬೇಸ-ರಿ-ಸಿ-ಕೊಂ-ಡಂತೆ
ಬೇಸ-ರಿ-ಸಿ-ಕೊ-ಳ್ಳದೆ
ಬೇಸ-ರಿ-ಸಿ-ಕೊ-ಳ್ಳ-ಬಾ-ರದು
ಬೇಸ-ರಿ-ಸಿ-ಕೊ-ಳ್ಳ-ಲಿಲ್ಲ
ಬೇಸ-ರಿ-ಸಿ-ಕೊಳ್ಳು
ಬೇಸ-ರಿಸು
ಬೇಸಿಗೆ
ಬೇಸಿ-ಗೆಯ
ಬೇಸಿ-ಗೆ-ಯನ್ನು
ಬೇಸಿ-ಗೆ-ಯಲ್ಲಿ
ಬೈಗಳನ್ನು
ಬೈಗು-ಳಕ್ಕೆ
ಬೈಗು-ಳ-ಗಳನ್ನೂ
ಬೈಗು-ಳದ
ಬೈದ
ಬೈದದ್ದೇ
ಬೈದರು
ಬೈದರೂ
ಬೈದಿ-ರ-ಲೂ-ಬ-ಹುದು
ಬೈದು
ಬೈದೆ
ಬೈಬ-ಲಿನ
ಬೈಬ-ಲ್ಲಿ-ನಲ್ಲಿ
ಬೈಯಲು
ಬೈಯಿಸಿ
ಬೈಯು-ತ್ತಾರೋ
ಬೈಯು-ತ್ತಿ-ದ್ದರು
ಬೈಯು-ತ್ತಿ-ದ್ದರೂ
ಬೈಯು-ತ್ತೇನೆ
ಬೈಯು-ವು-ದಿಲ್ಲ
ಬೈಯು-ವುದು
ಬೈಯ್ಯುವ
ಬೊಂಬೆ
ಬೊಂಬೆ-ಗ-ಳಂತೆ
ಬೊಂಬೆ-ಯಂತೆ
ಬೊಕ್ಕೆ-ಗಳು
ಬೊಗ-ಳುತ್ತ
ಬೋಗಿ
ಬೋಗಿಗೆ
ಬೋಗಿ-ಯನ್ನು
ಬೋಗಿ-ಯಲ್ಲಿ
ಬೋಗಿ-ಯಿಂದ
ಬೋಗಿ-ಯೊ-ಳಕ್ಕೆ
ಬೋಗಿ-ಯೊ-ಳಗೆ
ಬೋಧಕ
ಬೋಧ-ಕ-ನಂತೆ
ಬೋಧ-ಕರ
ಬೋಧ-ಕರು
ಬೋಧ-ನಾ-ಕಾರ್ಯ
ಬೋಧ-ನಾ-ಕಾ-ರ್ಯ-ಗಳನ್ನು
ಬೋಧನೆ
ಬೋಧ-ನೆ-ಇ-ವು-ಗಳು
ಬೋಧ-ನೆ-ಗಳ
ಬೋಧ-ನೆ-ಗಳನ್ನು
ಬೋಧ-ನೆ-ಗಳನ್ನೂ
ಬೋಧ-ನೆ-ಗ-ಳನ್ನೇ
ಬೋಧ-ನೆ-ಗಳಲ್ಲಿ
ಬೋಧ-ನೆ-ಗ-ಳಷ್ಟೇ
ಬೋಧ-ನೆ-ಗಳಿಂದ
ಬೋಧ-ನೆ-ಗ-ಳಿಗೂ
ಬೋಧ-ನೆ-ಗ-ಳಿಗೆ
ಬೋಧ-ನೆ-ಗಳು
ಬೋಧ-ನೆ-ಗ-ಳೆಲ್ಲ
ಬೋಧ-ನೆಗೆ
ಬೋಧ-ನೆಯ
ಬೋಧ-ನೆ-ಯನ್ನು
ಬೋಧ-ನೆ-ಯ-ನ್ನೆಲ್ಲ
ಬೋಧ-ನೆಯು
ಬೋಧ-ನೆಯೂ
ಬೋಧ-ಪ್ರ-ದ-ವಾ-ಗಿತ್ತು
ಬೋಧ-ಪ್ರ-ದ-ವಾ-ದದ್ದು
ಬೋಧಾ-ನಂದ
ಬೋಧಾ-ನಂ-ದ-ರಿಗೆ
ಬೋಧಾ-ನಂ-ದರು
ಬೋಧಿ
ಬೋಧಿಯ
ಬೋಧಿ-ವೃ-ಕ್ಷದ
ಬೋಧಿ-ಸ-ಬೇ-ಕಾ-ಗಿಯೇ
ಬೋಧಿ-ಸ-ಬೇ-ಕಾ-ಗಿ-ರು-ವು-ದೆಂ-ದರೆ
ಬೋಧಿ-ಸ-ಬೇಕು
ಬೋಧಿ-ಸ-ಬೇ-ಕೆಂದು
ಬೋಧಿ-ಸ-ಲಾಗಿದೆ
ಬೋಧಿ-ಸ-ಲಾಗು
ಬೋಧಿ-ಸ-ಲಾ-ಗು-ವು-ದಿಲ್ಲ
ಬೋಧಿ-ಸ-ಲಾ-ಗು-ವುದು
ಬೋಧಿ-ಸ-ಲಾ-ರಂ-ಭಿ-ಸಿ-ದರು
ಬೋಧಿ-ಸಲಿ
ಬೋಧಿ-ಸ-ಲಿ-ಲ್ಲವೆ
ಬೋಧಿ-ಸಲು
ಬೋಧಿ-ಸ-ಲ್ಪ-ಟ್ಟಿ-ರುವ
ಬೋಧಿಸಿ
ಬೋಧಿ-ಸಿದ
ಬೋಧಿ-ಸಿ-ದಂ-ತಹ
ಬೋಧಿ-ಸಿ-ದರು
ಬೋಧಿ-ಸಿ-ದರೂ
ಬೋಧಿ-ಸಿ-ದರೋ
ಬೋಧಿ-ಸಿ-ದ-ವರು
ಬೋಧಿ-ಸಿ-ದಾಗ
ಬೋಧಿ-ಸಿ-ದುದು
ಬೋಧಿ-ಸಿದೆ
ಬೋಧಿ-ಸಿ-ದ್ದರು
ಬೋಧಿ-ಸಿ-ದ್ದ-ವೆಂದು
ಬೋಧಿ-ಸಿದ್ದು
ಬೋಧಿ-ಸಿ-ರು-ವರೋ
ಬೋಧಿಸು
ಬೋಧಿ-ಸು-ತ್ತದೆ
ಬೋಧಿ-ಸುತ್ತಾ
ಬೋಧಿ-ಸು-ತ್ತಿದ್ದ
ಬೋಧಿ-ಸು-ತ್ತಿ-ದ್ದರು
ಬೋಧಿ-ಸು-ತ್ತಿ-ದ್ದ-ರೆಂ-ಬು-ದನ್ನು
ಬೋಧಿ-ಸು-ತ್ತಿ-ದ್ದೇನೆ
ಬೋಧಿ-ಸುವ
ಬೋಧಿ-ಸು-ವಂ-ತಹ
ಬೋಧಿ-ಸು-ವಷ್ಟು
ಬೋಧಿ-ಸು-ವು-ದಕ್ಕೆ
ಬೋಧಿ-ಸು-ವುದನ್ನು
ಬೋಧಿ-ಸು-ವು-ದಿಲ್ಲ
ಬೋಧಿ-ಸು-ವು-ದಿ-ಲ್ಲವೆ
ಬೋಧಿ-ಸು-ವುದು
ಬೋಧಿ-ಸು-ವುದೇ
ಬೋನು-ಗಳನ್ನು
ಬೋಳಿ-ಸಿ-ಕೊಂಡು
ಬೋಸರ
ಬೋಸ್
ಬೋಸ್ಪಾರಾ
ಬೌದ್ಧ
ಬೌದ್ಧ-ಕಾ-ಲೀನ
ಬೌದ್ಧ-ತ-ತ್ತ್ವ-ಗ-ಳೊಂ-ದಿಗೆ
ಬೌದ್ಧ-ಧ-ರ್ಮ-ಗ್ರಂ-ಥ-ಗಳ
ಬೌದ್ಧ-ಧ-ರ್ಮದ
ಬೌದ್ಧ-ಧ-ರ್ಮವು
ಬೌದ್ಧನೂ
ಬೌದ್ಧ-ಮ-ತೀ-ಯ-ರಲ್ಲಿ
ಬೌದ್ಧರ
ಬೌದ್ಧ-ರಿ-ದ್ದರು
ಬೌದ್ಧರು
ಬೌದ್ಧ-ರೆಂ-ದರೆ
ಬೌದ್ಧ-ರೆಂದು
ಬೌದ್ಧ-ಸಂನ್ಯಾಸಿ
ಬೌದ್ಧಾನು
ಬೌದ್ಧಿಕ
ಬೌದ್ಧಿ-ಕ-ವಾಗಿ
ಬ್ಬರಿಗೂ
ಬ್ಬರು
ಬ್ಬರೂ
ಬ್ಯಾಂಡು-ತು-ತ್ತೂ-ರಿ-ಗ-ಳೊಂ-ದಿಗೆ
ಬ್ಯಾಗ್ಲೀ
ಬ್ಯಾನರ್ಜಿ
ಬ್ಯಾನ-ರ್ಜಿ-ಯ-ವರ
ಬ್ಯೂಸ-ನೊಂ-ದಿಗೆ
ಬ್ಯೂಸ್
ಬ್ಯೂಸ್ನ
ಬ್ಯೂಸ್ನೊಂ-ದಿಗೆ
ಬ್ರಹ್ಮ
ಬ್ರಹ್ಮ-ಕ್ಷತ್ರಂ
ಬ್ರಹ್ಮ-ಚರ್ಯ
ಬ್ರಹ್ಮ-ಚ-ರ್ಯ-ಪಾ-ವಿ-ತ್ರ್ಯ-ಗಳನ್ನು
ಬ್ರಹ್ಮ-ಚ-ರ್ಯ-ಜೀ-ವ-ನದ
ಬ್ರಹ್ಮ-ಚ-ರ್ಯ-ಜೀ-ವ-ನ-ದಲ್ಲಿ
ಬ್ರಹ್ಮ-ಚ-ರ್ಯದ
ಬ್ರಹ್ಮ-ಚ-ರ್ಯ-ದಿಂ-ದೊ-ಡ-ಗೂ-ಡಿದ
ಬ್ರಹ್ಮ-ಚ-ರ್ಯ-ದೀಕ್ಷೆ
ಬ್ರಹ್ಮ-ಚ-ರ್ಯ-ದೀ-ಕ್ಷೆ-ಯನ್ನೂ
ಬ್ರಹ್ಮ-ಚ-ರ್ಯ-ವನ್ನು
ಬ್ರಹ್ಮ-ಚ-ರ್ಯವು
ಬ್ರಹ್ಮ-ಚಾರಿ
ಬ್ರಹ್ಮ-ಚಾ-ರಿ-ಗಳ
ಬ್ರಹ್ಮ-ಚಾ-ರಿ-ಗಳನ್ನು
ಬ್ರಹ್ಮ-ಚಾ-ರಿ-ಗಳಲ್ಲಿ
ಬ್ರಹ್ಮ-ಚಾ-ರಿ-ಗ-ಳಾಗಿ
ಬ್ರಹ್ಮ-ಚಾ-ರಿ-ಗ-ಳಿಗೂ
ಬ್ರಹ್ಮ-ಚಾ-ರಿ-ಗ-ಳಿಗೆ
ಬ್ರಹ್ಮ-ಚಾ-ರಿ-ಗಳು
ಬ್ರಹ್ಮ-ಚಾ-ರಿ-ಗಳೂ
ಬ್ರಹ್ಮ-ಚಾ-ರಿ-ಗ-ಳೆಲ್ಲ
ಬ್ರಹ್ಮ-ಚಾ-ರಿ-ಗ-ಳೊಂ-ದಿಗೆ
ಬ್ರಹ್ಮ-ಚಾ-ರಿಗೆ
ಬ್ರಹ್ಮ-ಚಾ-ರಿ-ಣಿಯ
ಬ್ರಹ್ಮ-ಚಾ-ರಿ-ಣಿ-ಯರ
ಬ್ರಹ್ಮ-ಚಾ-ರಿ-ಣಿ-ಯ-ರನ್ನೂ
ಬ್ರಹ್ಮ-ಚಾ-ರಿ-ಣಿ-ಯ-ರಿಗೆ
ಬ್ರಹ್ಮ-ಚಾ-ರಿ-ಣಿ-ಯಾ-ಗಿ-ದ್ದಾಳೆ
ಬ್ರಹ್ಮ-ಚಾ-ರಿ-ಯನ್ನು
ಬ್ರಹ್ಮ-ಜ್ಞಾನ
ಬ್ರಹ್ಮ-ಜ್ಞಾ-ನಕ್ಕೂ
ಬ್ರಹ್ಮ-ಜ್ಞಾ-ನದ
ಬ್ರಹ್ಮ-ಜ್ಞಾ-ನಿ-ಗಳಲ್ಲಿ
ಬ್ರಹ್ಮ-ಜ್ಞಾ-ನಿ-ಗಳೂ
ಬ್ರಹ್ಮ-ಜ್ಞಾ-ನಿ-ಗ-ಳೆಂದೂ
ಬ್ರಹ್ಮ-ಪುತ್ರ
ಬ್ರಹ್ಮ-ಪು-ತ್ರ-ದ-ವ-ರೆಗೆ
ಬ್ರಹ್ಮ-ರಾ-ಕ್ಷ-ಸನ
ಬ್ರಹ್ಮ-ರ್ಷಿ-ಯಂತೆ
ಬ್ರಹ್ಮ-ವಾ-ದಿನ್
ಬ್ರಹ್ಮ-ವಾ-ದಿ-ನ್ಗಾಗಿ
ಬ್ರಹ್ಮ-ವಾ-ದಿನ್ನ
ಬ್ರಹ್ಮ-ವಿ-ದ್ಯೆಗೆ
ಬ್ರಹ್ಮ-ಸೂ-ತ್ರ-ಗಳು
ಬ್ರಹ್ಮಾ
ಬ್ರಹ್ಮಾಂ-ಡವೂ
ಬ್ರಹ್ಮಾ-ನಂದ
ಬ್ರಹ್ಮಾ-ನಂ-ದರ
ಬ್ರಹ್ಮಾ-ನಂ-ದ-ರದು
ಬ್ರಹ್ಮಾ-ನಂ-ದ-ರನ್ನು
ಬ್ರಹ್ಮಾ-ನಂ-ದ-ರಿಗೆ
ಬ್ರಹ್ಮಾ-ನಂ-ದರು
ಬ್ರಹ್ಮಾ-ನಂ-ದರೂ
ಬ್ರಹ್ಮಾ-ನಂ-ದ-ರೆಂ-ದರೆ
ಬ್ರಹ್ಮಾ-ನಂ-ದರೇ
ಬ್ರಹ್ಮಾ-ವ-ರ್ತ-ವೆಂದು
ಬ್ರಹ್ಮೋ-ಪ-ದೇಶ
ಬ್ರಹ್ಮೋ-ಪ-ದೇ-ಶಕ್ಕೆ
ಬ್ರಹ್ಮೋ-ಪ-ದೇ-ಶ-ವಾ-ಗ-ಬೇ-ಕೆಂದು
ಬ್ರಾನ್ಸ್ಬಿ
ಬ್ರಾಹ್ಮ
ಬ್ರಾಹ್ಮಣ
ಬ್ರಾಹ್ಮ-ಣ-ತ್ವಕ್ಕೆ
ಬ್ರಾಹ್ಮ-ಣ-ತ್ವದ
ಬ್ರಾಹ್ಮ-ಣ-ತ್ವ-ವೆಂ-ಬುದು
ಬ್ರಾಹ್ಮ-ಣನ
ಬ್ರಾಹ್ಮ-ಣ-ನಾಗಿ
ಬ್ರಾಹ್ಮ-ಣ-ನಿ-ಗಿಂತ
ಬ್ರಾಹ್ಮ-ಣನು
ಬ್ರಾಹ್ಮ-ಣರ
ಬ್ರಾಹ್ಮ-ಣ-ರ-ನ್ನು-ದ್ದೇ-ಶಿಸಿ
ಬ್ರಾಹ್ಮ-ಣ-ರಾ-ಗಿ-ಬಿ-ಟ್ಟರೊ
ಬ್ರಾಹ್ಮ-ಣ-ರಾ-ಗು-ವುದು
ಬ್ರಾಹ್ಮ-ಣ-ರಾ-ದ-ವರು
ಬ್ರಾಹ್ಮ-ಣ-ರಿಂದ
ಬ್ರಾಹ್ಮ-ಣ-ರಿಗೆ
ಬ್ರಾಹ್ಮ-ಣರು
ಬ್ರಾಹ್ಮ-ಣ-ರು-ತೀರಾ
ಬ್ರಾಹ್ಮ-ಣರೂ
ಬ್ರಾಹ್ಮ-ಣರೇ
ಬ್ರಾಹ್ಮ-ಣ-ಶಿ-ಷ್ಯ-ನಾದ
ಬ್ರಾಹ್ಮ-ಣಿ-ಕೆ-ಯನ್ನು
ಬ್ರಾಹ್ಮ-ಣಿ-ಕೋಲೆ
ಬ್ರಾಹ್ಮ-ಣೇ-ತ-ರರ
ಬ್ರಾಹ್ಮ-ಣೇ-ತ-ರ-ರನ್ನು
ಬ್ರಾಹ್ಮ-ಸ-ಮಾ-ಜ-ದ-ವ-ರಾ-ಗಿ-ರ-ಬ-ಹುದು
ಬ್ರಾಹ್ಮ-ಸ-ಮಾ-ಜೀ-ಯರೇ
ಬ್ರಾಹ್ಮೀ
ಬ್ರಿಟಾನಿ
ಬ್ರಿಟಾ-ನಿ-ಕದ
ಬ್ರಿಟಾ-ನಿಗೆ
ಬ್ರಿಟಾ-ನಿ-ಯಲ್ಲಿ
ಬ್ರಿಟಾ-ನಿ-ಯಿಂದ
ಬ್ರಿಟಿ-ಷರ
ಬ್ರಿಟಿ-ಷ-ರಿಗೆ
ಬ್ರಿಟಿಷ್
ಬ್ರೂಕ್ಲಿನ್
ಬ್ರೆಡ್ಡನ್ನು
ಬ್ರೆಡ್ಡು
ಬ್ಲಾಂಕ-ರ್ಡ್
ಬ್ಲಾಜೆಟ್
ಬ್ಲಾಜೆ-ಟ್ಟಳ
ಬ್ಲಾಜೆ-ಟ್ಟ-ಳಿಗೆ
ಭಂಗ
ಭಂಗ-ವಾ-ಯಿತು
ಭಂಗ-ವುಂ-ಟಾ-ಗಿಯೇ
ಭಂಗಿ-ಯಲ್ಲಿ
ಭಂಗಿಸಿ
ಭಂಡಾರ
ಭಂಡಾ-ರ-ವನ್ನು
ಭಕ್ತ
ಭಕ್ತ-ಜ-ನ-ರಿಗೆ
ಭಕ್ತನ
ಭಕ್ತ-ನಾದ
ಭಕ್ತನೂ
ಭಕ್ತರ
ಭಕ್ತ-ರನ್ನು
ಭಕ್ತ-ರಾದ
ಭಕ್ತ-ರಿಂದ
ಭಕ್ತ-ರಿಗೂ
ಭಕ್ತ-ರಿಗೆ
ಭಕ್ತ-ರಿ-ಗೆಲ್ಲ
ಭಕ್ತರು
ಭಕ್ತರೂ
ಭಕ್ತ-ರೆಲ್ಲ
ಭಕ್ತ-ರೆ-ಲ್ಲರೂ
ಭಕ್ತ-ರೊಂ-ದಿಗೆ
ಭಕ್ತ-ರೊ-ಬ್ಬರ
ಭಕ್ತ-ವ-ರೇ-ಣ್ಯನೂ
ಭಕ್ತ-ವ-ರೇ-ಣ್ಯ-ರಾದ
ಭಕ್ತ-ವೃಂ-ದದ
ಭಕ್ತಾ
ಭಕ್ತಾ-ದಿ-ಗಳ
ಭಕ್ತಾ-ದಿ-ಗ-ಳಿಗೆ
ಭಕ್ತಾ-ದಿ-ಗಳು
ಭಕ್ತಾ-ದಿ-ಗಳೂ
ಭಕ್ತಾ-ದಿ-ಗ-ಳೆಲ್ಲ
ಭಕ್ತಾ-ದಿ-ಗಳೇ
ಭಕ್ತಿ
ಭಕ್ತಿ
ಭಕ್ತಿ-ಆ-ನಂ-ದೋ-ತ್ಸಾ-ಹ-ಗಳ
ಭಕ್ತಿ-ಗೌ-ರವ
ಭಕ್ತಿ-ಗೌ-ರ-ವ-ಗಳನ್ನು
ಭಕ್ತಿ-ಪ್ರೇ-ಮ-ಕಾ-ತ-ರ-ತೆ-ಗಳನ್ನು
ಭಕ್ತಿ-ಭಾವ
ಭಕ್ತಿ-ವಿ-ಶ್ವಾಸ
ಭಕ್ತಿ-ಶ್ರ-ದ್ಧೆ-ಗಳ
ಭಕ್ತಿ-ಶ್ರ-ದ್ಧೆ-ಗಳನ್ನು
ಭಕ್ತಿ-ಶ್ರ-ದ್ಧೆ-ಗಳಿಂದ
ಭಕ್ತಿ-ಶ್ರ-ದ್ಧೆ-ಗ-ಳಿ-ದ್ದರೆ
ಭಕ್ತಿ-ಶ್ರ-ದ್ಧೆ-ಯಿಂದ
ಭಕ್ತಿ-ಎಂಬ
ಭಕ್ತಿ-ಪೂ-ರಿತ
ಭಕ್ತಿ-ಪೂರ್ಣ
ಭಕ್ತಿ-ಪೂ-ರ್ವಕ
ಭಕ್ತಿ-ಪ್ರೇ-ಮಾ-ನಂ-ದ-ವನ್ನು
ಭಕ್ತಿ-ಭಾವ
ಭಕ್ತಿ-ಭಾ-ವಕ್ಕೆ
ಭಕ್ತಿ-ಭಾ-ವದ
ಭಕ್ತಿ-ಭಾ-ವ-ದಿಂದ
ಭಕ್ತಿ-ಭಾ-ವ-ಲೀ-ನ-ರಾಗಿ
ಭಕ್ತಿ-ಭಾ-ವ-ವನ್ನು
ಭಕ್ತಿ-ಭಾ-ವವು
ಭಕ್ತಿಯ
ಭಕ್ತಿ-ಯನ್ನು
ಭಕ್ತಿ-ಯಲ್ಲಿ
ಭಕ್ತಿ-ಯಿಂದ
ಭಕ್ತಿ-ಯೆಂ-ದರೆ
ಭಕ್ತಿ-ಯೆಲ್ಲ
ಭಕ್ತಿ-ಯೊಂ-ದಿಗೆ
ಭಕ್ತಿ-ಯೋಗ
ಭಕ್ತಿ-ಯೋ-ಗ-ವನ್ನು
ಭಕ್ತಿ-ಯೋ-ಗವೇ
ಭಕ್ತಿ-ರ-ಸ-ವನ್ನು
ಭಕ್ತಿ-ವಿ-ಶ್ವಾಸ
ಭಕ್ತಿ-ವಿ-ಶ್ವಾ-ಸ-ಗಳು
ಭಕ್ತಿ-ಶ್ರ-ದ್ಧೆ-ಗಳಿಂದ
ಭಕ್ತಿ-ಸಿ-ದ್ಧಾಂ-ತ-ಗಳ
ಭಕ್ತಿ-ಸ್ವ-ಭಾ-ವ-ದ-ವ-ರಾ-ಗಿದ್ದ
ಭಕ್ತೆ
ಭಕ್ತೆ-ಯ-ರಿಂ-ದೊ-ಡ-ಗೂಡಿ
ಭಕ್ತೆಯೂ
ಭಕ್ತ್ಯಾ-ಧಿ-ಕ್ಯ-ದಿಂದ
ಭಕ್ತ್ಯಾ-ವೇ-ಶದ
ಭಕ್ತ್ಯಾ-ವೇ-ಶ-ಭ-ರಿ-ತ-ರಾಗಿ
ಭಕ್ತ್ಯು-ತ್ಸಾ-ಹ-ದಿಂದ
ಭಕ್ಷೀ-ಸನ್ನು
ಭಕ್ಷೀಸು
ಭಕ್ಷ್ಯ-ಗಳ
ಭಕ್ಷ್ಯ-ಗಳನ್ನು
ಭಕ್ಷ್ಯ-ಭೋ-ಜ್ಯ-ಗಳನ್ನು
ಭಕ್ಷ್ಯ-ಭೋ-ಜ್ಯ-ಗಳನ್ನೆಲ್ಲ
ಭಕ್ಷ್ಯ-ಸ-ಮೇ-ತ-ವಾದ
ಭಗ
ಭಗವ
ಭಗ-ವಂತ
ಭಗ-ವಂ-ತ-ದನ
ಭಗ-ವಂ-ತನ
ಭಗ-ವಂ-ತ-ನನ್ನು
ಭಗ-ವಂ-ತ-ನ-ನ್ನು-ಎಲ್ಲ
ಭಗ-ವಂ-ತ-ನನ್ನೂ
ಭಗ-ವಂ-ತ-ನನ್ನೇ
ಭಗ-ವಂ-ತ-ನಲ್ಲಿ
ಭಗ-ವಂ-ತ-ನ-ಲ್ಲಿಗೇ
ಭಗ-ವಂ-ತ-ನಿಂದ
ಭಗ-ವಂ-ತ-ನಿಂ-ದಲೇ
ಭಗ-ವಂ-ತ-ನಿ-ಗಾಗಿ
ಭಗ-ವಂ-ತ-ನಿಗೆ
ಭಗ-ವಂ-ತ-ನಿಗೇ
ಭಗ-ವಂ-ತನು
ಭಗ-ವಂ-ತನೂ
ಭಗ-ವಂ-ತ-ನೆಂದು
ಭಗ-ವಂ-ತ-ನೆಂಬ
ಭಗ-ವಂ-ತ-ನೆ-ಡೆಗೆ
ಭಗ-ವಂ-ತನೇ
ಭಗ-ವ-ಚ್ಚಿಂ-ತನೆ
ಭಗ-ವ-ಚ್ಚಿಂ-ತ-ನೆಗೆ
ಭಗ-ವತಿ
ಭಗ-ವ-ತ್ಕೃ-ಪೆ-ಯಿಂದ
ಭಗ-ವ-ತ್ಕೃ-ಪೆ-ಯಿಂ-ದಾಗಿ
ಭಗ-ವ-ತ್ಪ್ರೇ-ಮದ
ಭಗ-ವ-ತ್ಪ್ರೇ-ಮ-ಭ-ರಿತ
ಭಗ-ವ-ತ್ಸಂ-ಕ-ಲ್ಪ-ದಂತೆ
ಭಗ-ವ-ತ್ಸಾ-ಕ್ಷಾ-ತ್ಕಾರ
ಭಗ-ವ-ತ್ಸಾ-ಕ್ಷಾ-ತ್ಕಾ-ರಕ್ಕೆ
ಭಗ-ವ-ತ್ಸೌಂ-ದ-ರ್ಯದ
ಭಗ-ವ-ತ್ಸ್ವ-ರೂ-ಪಿ-ಯಾದ
ಭಗ-ವ-ದ-ನು-ಗ್ರ-ಹ-ದಿಂದ
ಭಗ-ವ-ದಾಜ್ಞೆ
ಭಗ-ವ-ದಾ-ನಂ-ದ-ಮಯ
ಭಗ-ವ-ದಾ-ನಂ-ದ-ವನ್ನು
ಭಗ-ವ-ದಿಚ್ಛೆ
ಭಗ-ವದ್
ಭಗ-ವ-ದ್ಗೀತೆ
ಭಗ-ವ-ದ್ಗೀ-ತೆ-ವೇ-ದಾಂ-ತ-ಗಳ
ಭಗ-ವ-ದ್ಗೀ-ತೆಯ
ಭಗ-ವ-ದ್ಗೀ-ತೆ-ಯನ್ನು
ಭಗ-ವ-ದ್ಗೀ-ತೆಯೇ
ಭಗ-ವ-ದ್ಜ್ಞಾ-ನದ
ಭಗ-ವ-ದ್ದ-ರ್ಶನ
ಭಗ-ವ-ದ್ಭಾ-ವ-ದಿಂದ
ಭಗ-ವ-ದ್ಭಾ-ವ-ನೆಯು
ಭಗ-ವ-ನ್ನಾಮ
ಭಗ-ವ-ನ್ನಾ-ಮ-ವನ್ನು
ಭಗ-ವ-ನ್ಮ-ಯ-ವಾಗಿ
ಭಗ-ವ-ನ್ಮ-ಯ-ವಾದ
ಭಗ-ವಾ-ನರ
ಭಗ-ವಾ-ನರು
ಭಗ-ವಾನ್
ಭಗೀ-ರಥ
ಭಗ್ನ-ವಾ-ಗಿದೆ
ಭಜನ
ಭಜನೆ
ಭಜ-ನೆ-ಗಳನ್ನು
ಭಟ್ಟಾ-ಚಾ-ರ್ಯ-ರನ್ನು
ಭಟ್ಟಾ-ಚಾ-ರ್ಯರು
ಭಟ್ಟಾ-ಚಾ-ರ್ಯ-ರೊಂ-ದಿಗೆ
ಭದ್ರ
ಭದ್ರ-ವಾಗಿ
ಭದ್ರ-ವಾ-ಗಿ-ದೆ-ಯಲ್ಲ
ಭದ್ರ-ವಾ-ಗಿ-ದ್ದು-ದ-ರಿಂದ
ಭಯ
ಭಯ-ಸಂ-ಶ-ಯ-ಗ-ಳಾ-ಗಲಿ
ಭಯಂ
ಭಯಂ-ಕರ
ಭಯಂ-ಕ-ರ-ವಾ-ಗಿತ್ತು
ಭಯಂ-ಕ-ರ-ವಾ-ಗಿಯೇ
ಭಯಂ-ಕ-ರ-ವಾದ
ಭಯಂ-ಕ-ರ-ವಾ-ದ-ದ್ದೇ-ನನ್ನೂ
ಭಯಂ-ಕ-ರ-ವಾ-ದು-ದನ್ನು
ಭಯ-ಗ್ರಸ್ತ
ಭಯ-ಗ್ರ-ಸ್ತ-ರಾ-ದರು
ಭಯದ
ಭಯ-ದಿಂದ
ಭಯ-ಪ-ಡು-ತ್ತಲೇ
ಭಯ-ಪ-ಡು-ತ್ತಿ-ರುವ
ಭಯ-ವನ್ನು
ಭಯ-ವಿ-ದ್ದರೂ
ಭಯ-ವಿ-ಲ್ಲವೋ
ಭಯ-ವಿ-ಹ್ವ-ಲ-ವಾ-ಗು-ವುದು
ಭಯವೂ
ಭಯ-ವೆಂ-ದಿಗೂ
ಭಯವೇ
ಭಯಾ-ತ್ಈ
ಭಯಾ-ನಕ
ಭಯಾ-ನ-ಕ-ವಾ-ಗಿತ್ತು
ಭಯಾ-ನ-ಕ-ವಾ-ಗಿಲ್ಲ
ಭಯಾ-ಶ್ಚ-ರ್ಯ-ಚ-ಕಿ-ತ-ರಾ-ಗಿ-ರ-ಬೇಕು
ಭಯ್ಯಾ
ಭರತ
ಭರ-ತ-ಖಂಡ
ಭರ-ತ-ಖಂ-ಡದ
ಭರ-ತ-ಖಂ-ಡ-ದಲ್ಲಿ
ಭರ-ತ-ಖಂ-ಡ-ದಲ್ಲೇ
ಭರ-ತ-ಖಂ-ಡ-ದಿಂ-ದಲೇ
ಭರ-ತ-ಖಂ-ಡ-ವನ್ನು
ಭರ-ತ-ಖಂ-ಡ-ವ-ನ್ನೆಲ್ಲ
ಭರ-ತ-ಖಂ-ಡವು
ಭರ-ತ-ಖಂ-ಡವೇ
ಭರ-ದಲ್ಲಿ
ಭರ-ದಿಂದ
ಭರ-ವಸೆ
ಭರ-ವ-ಸೆ-ವಿ-ಶ್ವಾಸ
ಭರ-ವ-ಸೆಯ
ಭರ-ವ-ಸೆ-ಯ-ನ್ನಿ-ಟ್ಟಿ-ದ್ದರು
ಭರ-ವ-ಸೆ-ಯನ್ನು
ಭರ-ವ-ಸೆ-ಯನ್ನೂ
ಭರ-ವ-ಸೆ-ಯಿ-ರು-ವುದು
ಭರ-ವ-ಸೆ-ಯುಂ-ಟಾ-ಗಿ-ರ-ಲಿಲ್ಲ
ಭರಾಟೆ
ಭರಾ-ಟೆ-ಯೆಲ್ಲ
ಭರಿತ
ಭರಿ-ತ-ರಾ-ಗಿ-ದ್ದರು
ಭರಿ-ತ-ವಾ-ಗಿ-ದೆಯೋ
ಭರಿ-ಸ-ಲಾ-ಗದ
ಭರ್ಜರಿ
ಭರ್ತಿ-ಯಾ-ಗಿವೆ
ಭರ್ತೃ-ಹರಿ
ಭಲೆ
ಭವ
ಭವ-ನ-ದಲ್ಲಿ
ಭವ-ಬಂ-ಧನ
ಭವ-ವನ್ನು
ಭವ-ವಾ-ಗು-ತ್ತಿತ್ತು
ಭವ-ಸಾ-ಗರ
ಭವ-ಸಾ-ಗ-ರ-ದಿಂದ
ಭವಿ-ತವ್ಯ
ಭವಿ-ತ-ವ್ಯದ
ಭವಿಷ್ಯ
ಭವಿ-ಷ್ಯ-ಚಿಂ-ತನೆ
ಭವಿ-ಷ್ಯತಿ
ಭವಿ-ಷ್ಯ-ತ್ತನ್ನು
ಭವಿ-ಷ್ಯದ
ಭವಿ-ಷ್ಯ-ದಲ್ಲಿ
ಭವಿ-ಷ್ಯ-ಪು-ತ್ರ-ನಿಗೆ
ಭವಿ-ಷ್ಯ-ವನ್ನು
ಭವಿ-ಷ್ಯ-ವಾಣಿ
ಭವಿ-ಷ್ಯ-ವಾ-ಣಿ-ಯಂ-ತಿ-ದೆ-ಯೆಂದು
ಭವಿ-ಷ್ಯ-ವಿದೆ
ಭವಿ-ಷ್ಯವು
ಭವಿ-ಷ್ಯ-ವೊಂ-ದರ
ಭವಿ-ಸಿತ್ತು
ಭವ್ಯ
ಭವ್ಯತೆ
ಭವ್ಯ-ವಾಗಿ
ಭವ್ಯ-ವಾ-ಗಿ-ರು-ತ್ತದೆ
ಭವ್ಯ-ವಾದ
ಭಾಗ
ಭಾಗಕ್ಕೆ
ಭಾಗ-ಗಳನ್ನು
ಭಾಗ-ಗಳನ್ನೂ
ಭಾಗ-ಗಳಲ್ಲಿ
ಭಾಗ-ಗಳಿಂದ
ಭಾಗ-ಗ-ಳಿಗೆ
ಭಾಗ-ಗಳು
ಭಾಗ-ದಲ್ಲಿ
ಭಾಗ-ದ-ಲ್ಲಿ-ರುವ
ಭಾಗ-ವ-ತವೇ
ಭಾಗ-ವ-ನ್ನಿನ್ನೂ
ಭಾಗ-ವನ್ನು
ಭಾಗ-ವ-ಹಿ-ಸ-ಬೇ-ಕೆಂದು
ಭಾಗ-ವ-ಹಿ-ಸ-ಬೇ-ಕೆಂಬ
ಭಾಗ-ವ-ಹಿ-ಸ-ಲಿದ್ದ
ಭಾಗ-ವ-ಹಿ-ಸ-ಲಿಲ್ಲ
ಭಾಗ-ವ-ಹಿ-ಸಲು
ಭಾಗ-ವ-ಹಿಸಿ
ಭಾಗ-ವ-ಹಿ-ಸಿದ
ಭಾಗ-ವ-ಹಿ-ಸಿ-ದರು
ಭಾಗ-ವ-ಹಿ-ಸಿದ್ದ
ಭಾಗ-ವ-ಹಿ-ಸಿ-ದ್ದರು
ಭಾಗ-ವ-ಹಿ-ಸಿ-ದ್ದಳು
ಭಾಗ-ವ-ಹಿಸು
ಭಾಗ-ವ-ಹಿ-ಸು-ತ್ತಾರೆ
ಭಾಗ-ವ-ಹಿ-ಸು-ತ್ತಿ-ದ್ದರು
ಭಾಗ-ವ-ಹಿ-ಸು-ತ್ತಿ-ರು-ವ-ವ-ರಿ-ಗೆಲ್ಲ
ಭಾಗ-ವ-ಹಿ-ಸುವ
ಭಾಗ-ವ-ಹಿ-ಸು-ವಂ-ತಿ-ರ-ಬೇಕು
ಭಾಗ-ವ-ಹಿ-ಸು-ವಂತೆ
ಭಾಗ-ವ-ಹಿ-ಸು-ವು-ದ-ಕ್ಕಾಗಿ
ಭಾಗ-ವ-ಹಿ-ಸು-ವು-ದ-ರಲ್ಲಿ
ಭಾಗ-ವಾದ
ಭಾಗವು
ಭಾಗವೇ
ಭಾಗಿ-ಯಾ-ಗು-ವಂ-ಥ-ವನು
ಭಾಗ್ಯ
ಭಾಗ್ಯಂ
ಭಾಗ್ಯ-ಮಹೋ
ಭಾಗ್ಯ-ಲ-ಕ್ಷ್ಮಿಯು
ಭಾಗ್ಯ-ವನ್ನು
ಭಾಗ್ಯ-ವಿ-ದ್ದರೆ
ಭಾಗ್ಯ-ವಿ-ರ-ಲಿಲ್ಲ
ಭಾಗ್ಯವೇ
ಭಾಗ್ಯ-ವೇನು
ಭಾನು-ವಾರ
ಭಾನು-ವಾ-ರ-ಗ-ಳಂದು
ಭಾನು-ವಾ-ರ-ದಂದು
ಭಾಯಿ-ಗ-ಳಾದ
ಭಾಯಿ-ಗ-ಳಿಗೆ
ಭಾಯಿ-ಗಳೂ
ಭಾರ
ಭಾರತ
ಭಾರ-ತ-ಕ್ಕಾಗಿ
ಭಾರ-ತಕ್ಕೂ
ಭಾರ-ತಕ್ಕೆ
ಭಾರ-ತ-ಕ್ಕೆ-ಸ-ಮೀ-ಪ-ವಾ-ಗು-ತ್ತಿ-ದ್ದೇ-ವೆಂಬ
ಭಾರ-ತ-ಗಳಲ್ಲಿ
ಭಾರ-ತದ
ಭಾರ-ತ-ದಂ-ತಹ
ಭಾರ-ತ-ದಂಥ
ಭಾರ-ತ-ದತ್ತ
ಭಾರ-ತ-ದ-ಲ್ಲಾ-ದ-ರೂ-ಯಾವ
ಭಾರ-ತ-ದಲ್ಲಿ
ಭಾರ-ತ-ದ-ಲ್ಲಿ-ದ್ದಾಗ
ಭಾರ-ತ-ದ-ಲ್ಲಿನ
ಭಾರ-ತ-ದ-ಲ್ಲಿನ್ನೂ
ಭಾರ-ತ-ದಲ್ಲೂ
ಭಾರ-ತ-ದಲ್ಲೆಲ್ಲ
ಭಾರ-ತ-ದಲ್ಲೇ
ಭಾರ-ತ-ದಾ-ದ್ಯಂತ
ಭಾರ-ತ-ದಿಂದ
ಭಾರ-ತ-ದಿಂ-ದಲೇ
ಭಾರ-ತ-ದೆ-ಡೆಗೆ
ಭಾರ-ತ-ದೇ-ಶಕ್ಕೆ
ಭಾರ-ತ-ದೊಂ-ದಿಗೆ
ಭಾರ-ತ-ನನ್ನ
ಭಾರ-ತ-ಪ್ರೇ-ಮ-ವನ್ನು
ಭಾರ-ತ-ಭೂಮಿ
ಭಾರ-ತ-ಭೂ-ಮಿ-ಯನ್ನು
ಭಾರ-ತ-ಮ-ಟ್ಟದ
ಭಾರ-ತ-ಮಾ-ತೆಯು
ಭಾರ-ತ-ಮಾ-ತ್ರವೇ
ಭಾರ-ತ-ವಂತೂ
ಭಾರ-ತ-ವನ್ನು
ಭಾರ-ತ-ವ-ನ್ನು-ಪೂ-ಜಿಸಿ
ಭಾರ-ತ-ವ-ನ್ನೆಲ್ಲ
ಭಾರ-ತ-ವಿಂದು
ಭಾರ-ತ-ವಿನ್ನೂ
ಭಾರ-ತವು
ಭಾರ-ತವೇ
ಭಾರ-ತಾಂಬೆ
ಭಾರತಿ
ಭಾರತೀ
ಭಾರ-ತೀಯ
ಭಾರ-ತೀ-ಯ-ತೆಯ
ಭಾರ-ತೀ-ಯ-ತೆ-ಯನ್ನು
ಭಾರ-ತೀ-ಯ-ನಿಗೂ
ಭಾರ-ತೀ-ಯನೂ
ಭಾರ-ತೀ-ಯರ
ಭಾರ-ತೀ-ಯ-ರನ್ನು
ಭಾರ-ತೀ-ಯ-ರ-ನ್ನು-ದ್ದೇ-ಶಿಸಿ
ಭಾರ-ತೀ-ಯ-ರಲ್ಲಿ
ಭಾರ-ತೀ-ಯ-ರಲ್ಲೇ
ಭಾರ-ತೀ-ಯ-ರಾದ
ಭಾರ-ತೀ-ಯ-ರಿಂದ
ಭಾರ-ತೀ-ಯ-ರಿ-ಗಾಗಿ
ಭಾರ-ತೀ-ಯ-ರಿಗೆ
ಭಾರ-ತೀ-ಯ-ರಿ-ಗೆಲ್ಲ
ಭಾರ-ತೀ-ಯರು
ಭಾರ-ತೀ-ಯ-ರು-ದ-ಕ್ಷಿಣ
ಭಾರ-ತೀ-ಯ-ರು-ಅ-ದ-ರಲ್ಲೂ
ಭಾರ-ತೀ-ಯರೇ
ಭಾರ-ತೀ-ಯ-ರೊ-ಬ್ಬ-ರಿಗೆ
ಭಾರ-ದಿಂದ
ಭಾರ-ವನ್ನು
ಭಾರ-ವಾ-ಗು-ತ್ತಿ-ದ್ದಾರೆ
ಭಾರ-ವಾದ
ಭಾರವೂ
ಭಾರಿ-ಯಾಗೇ
ಭಾರೀ
ಭಾವ
ಭಾವಕ್ಕೂ
ಭಾವಕ್ಕೆ
ಭಾವ-ಕ್ಕೇ-ರ-ಬ-ಲ್ಲ-ವ-ರಾ-ಗಿ-ದ್ದರು
ಭಾವ-ಗಳ
ಭಾವ-ಗಳನ್ನು
ಭಾವ-ಗ-ಳಿಗೆ
ಭಾವ-ಗಳೂ
ಭಾವ-ಗ-ಳೆಲ್ಲ
ಭಾವ-ಗ್ರಾ-ಹ-ಕತೆ
ಭಾವ-ಚಿತ್ರ
ಭಾವ-ಚಿ-ತ್ರಕ್ಕೆ
ಭಾವ-ಚಿ-ತ್ರ-ವ-ನ್ನಿಟ್ಟು
ಭಾವ-ಚಿ-ತ್ರ-ವನ್ನು
ಭಾವ-ಚಿ-ತ್ರ-ವನ್ನೂ
ಭಾವ-ಚಿ-ತ್ರವು
ಭಾವ-ಚಿ-ತ್ರ-ವೊಂ-ದನ್ನು
ಭಾವ-ತಂ-ತು-ಗಳನ್ನು
ಭಾವ-ತಂ-ತು-ವೊಂ-ದನ್ನು
ಭಾವ-ತ-ರಂಗ
ಭಾವದ
ಭಾವ-ದ-ಲೆ-ಗಳಿಂದ
ಭಾವ-ದಲ್ಲಿ
ಭಾವ-ದ-ಲ್ಲಿ-ದ್ದರು
ಭಾವ-ದ-ಲ್ಲಿ-ದ್ದಾಗ
ಭಾವ-ದ-ಲ್ಲಿ-ರ-ಬ-ಹುದು
ಭಾವ-ದ-ಲ್ಲಿ-ರು-ತ್ತಿ-ದ್ದರು
ಭಾವ-ದಲ್ಲೇ
ಭಾವ-ದಾಳ
ಭಾವ-ದಿಂದ
ಭಾವ-ದಿಂ-ದಲೇ
ಭಾವ-ದಿಂ-ದಿ-ದ್ದು-ದನ್ನು
ಭಾವನಾ
ಭಾವ-ನಾ-ಜೀ-ವಿ-ಯಲ್ಲ
ಭಾವ-ನಾ-ತ-ರಂ-ಗ-ಗಳನ್ನು
ಭಾವ-ನಾ-ತ್ಮಕ
ಭಾವ-ನಾ-ಪ್ರ-ಪಂಚ
ಭಾವ-ನಾ-ಶ-ಕ್ತಿ-ಗಳ
ಭಾವ-ನಾ-ಶ-ಕ್ತಿ-ಯನ್ನು
ಭಾವನೆ
ಭಾವ-ನೆ-ಕೃತಿ
ಭಾವ-ನೆ-ಬೋ-ಧ-ನೆ-ಗ-ಳಿಂ-ದಲ್ಲ
ಭಾವ-ನೆ-ಗಳ
ಭಾವ-ನೆ-ಗ-ಳ-ನ್ನಷ್ಟೇ
ಭಾವ-ನೆ-ಗಳನ್ನು
ಭಾವ-ನೆ-ಗ-ಳ-ನ್ನು-ಅವು
ಭಾವ-ನೆ-ಗಳನ್ನೂ
ಭಾವ-ನೆ-ಗಳನ್ನೆಲ್ಲ
ಭಾವ-ನೆ-ಗಳಲ್ಲಿ
ಭಾವ-ನೆ-ಗಳಿಂದ
ಭಾವ-ನೆ-ಗಳು
ಭಾವ-ನೆ-ಗ-ಳು-ಆ-ಚ-ರ-ಣೆ-ಗಳು
ಭಾವ-ನೆ-ಗ-ಳು-ವಿ-ಚಾ-ರ-ಗಳು
ಭಾವ-ನೆ-ಗ-ಳೆಂದೂ
ಭಾವ-ನೆ-ಗ-ಳೆಲ್ಲ
ಭಾವ-ನೆ-ಗಳೇ
ಭಾವ-ನೆಗೆ
ಭಾವ-ನೆಯ
ಭಾವ-ನೆ-ಯತ್ತ
ಭಾವ-ನೆ-ಯನ್ನು
ಭಾವ-ನೆ-ಯನ್ನೇ
ಭಾವ-ನೆ-ಯಾ-ಗಲಿ
ಭಾವ-ನೆ-ಯಾ-ಗಿತ್ತು
ಭಾವ-ನೆ-ಯಿಂದ
ಭಾವ-ನೆಯು
ಭಾವ-ನೆಯೂ
ಭಾವ-ನೆಯೇ
ಭಾವ-ನೆ-ಯೊಂದು
ಭಾವ-ಪ-ರ-ವ-ಶ-ರಾಗಿ
ಭಾವ-ಪ-ರ-ವ-ಶ-ರಾ-ದರು
ಭಾವ-ಪೂರ್ಣ
ಭಾವ-ಪೂ-ರ್ಣ-ವಾಗಿ
ಭಾವ-ಪ್ರ-ಕಾ-ಶವೇ
ಭಾವ-ಪ್ರ-ಚೋ-ದ-ಕ-ವಾ-ಗಿತ್ತು
ಭಾವ-ಪ್ರ-ಚೋ-ದ-ಕ-ವಾದ
ಭಾವ-ಪ್ರ-ಪಂ-ಚ-ದ-ಲ್ಲೊಂದು
ಭಾವ-ಭ-ರಿತ
ಭಾವ-ಭ-ರಿ-ತ-ರಾಗಿ
ಭಾವ-ಭ-ರಿ-ತ-ವಾಗಿ
ಭಾವ-ಮು-ಖ-ದಲ್ಲಿ
ಭಾವ-ವನ್ನು
ಭಾವ-ವಲ್ಲ
ಭಾವ-ವಿತ್ತು
ಭಾವ-ವಿ-ನಿ-ಮಯ
ಭಾವವು
ಭಾವವೂ
ಭಾವ-ವೇನು
ಭಾವ-ವೊಂದು
ಭಾವ-ಸ-ಮಾ-ಧಿ-ಗಳ
ಭಾವ-ಸ-ಮಾ-ಧಿಯ
ಭಾವ-ಸ-ಮಾ-ಧಿ-ಯಲ್ಲಿ
ಭಾವಾ-ಭಿ-ನ-ಯದ
ಭಾವಾ-ವ-ಸ್ಥೆ-ಗೇ-ರಿ-ದ್ದರು
ಭಾವಾ-ವ-ಸ್ಥೆ-ಯಲ್ಲಿ
ಭಾವಾ-ವೇಗ
ಭಾವಾ-ವೇ-ಗದ
ಭಾವಾ-ವೇ-ಗ-ದಿಂದ
ಭಾವಾ-ವೇ-ಗ-ಭ-ರಿ-ತ-ರಾ-ಗಿ-ದ್ದರು
ಭಾವಾ-ವೇ-ಶದ
ಭಾವಾ-ವೇ-ಶ-ದಿಂದ
ಭಾವಾ-ವೇ-ಶ-ಭ-ರಿತ
ಭಾವಾ-ವೇ-ಶ-ಭ-ರಿ-ತ-ನಾಗಿ
ಭಾವಾ-ವೇ-ಶ-ಭ-ರಿ-ತ-ರಾಗಿ
ಭಾವಾ-ವೇ-ಶ-ಭ-ರಿ-ತ-ವಾ-ಗಿದೆ
ಭಾವಿ
ಭಾವಿಸ
ಭಾವಿ-ಸ-ಬ-ಹುದು
ಭಾವಿ-ಸ-ಬಾ-ರದು
ಭಾವಿ-ಸ-ಬೇಕು
ಭಾವಿ-ಸ-ಲಿಲ್ಲ
ಭಾವಿಸಿ
ಭಾವಿ-ಸಿದ
ಭಾವಿ-ಸಿ-ದರು
ಭಾವಿ-ಸಿ-ದರೂ
ಭಾವಿ-ಸಿ-ದರೆ
ಭಾವಿ-ಸಿ-ದಾಗ
ಭಾವಿ-ಸಿ-ದಿರಾ
ಭಾವಿ-ಸಿದೆ
ಭಾವಿ-ಸಿ-ದೆಯಾ
ಭಾವಿ-ಸಿದ್ದ
ಭಾವಿ-ಸಿ-ದ್ದರು
ಭಾವಿ-ಸಿ-ದ್ದ-ರು-ಸ್ವಾ-ಮೀಜಿ
ಭಾವಿ-ಸಿ-ದ್ದರೋ
ಭಾವಿ-ಸಿ-ದ್ದಾರೆ
ಭಾವಿ-ಸಿ-ದ್ದಿ-ರ-ಬ-ಹುದೆ
ಭಾವಿ-ಸಿದ್ದೆ
ಭಾವಿ-ಸಿ-ರ-ಬೇಕು
ಭಾವಿ-ಸಿ-ರ-ಲಿಲ್ಲ
ಭಾವಿ-ಸಿರಿ
ಭಾವಿ-ಸಿ-ರು-ವಂತೆ
ಭಾವಿ-ಸಿ-ರು-ವಷ್ಟು
ಭಾವಿ-ಸು-ತ್ತದೆ
ಭಾವಿ-ಸು-ತ್ತಾರೆ
ಭಾವಿ-ಸು-ತ್ತಾ-ರೆ-ನೀವು
ಭಾವಿ-ಸು-ತ್ತಾರೋ
ಭಾವಿ-ಸು-ತ್ತಿದ್ದ
ಭಾವಿ-ಸು-ತ್ತಿ-ದ್ದಾ-ರೆ-ಕೇ-ವಲ
ಭಾವಿ-ಸು-ತ್ತೀಯೋ
ಭಾವಿ-ಸು-ತ್ತೇನೆ
ಭಾವಿ-ಸು-ತ್ತೇ-ನೆ-ನನ್ನ
ಭಾವಿ-ಸುವ
ಭಾವಿ-ಸು-ವಂತೆ
ಭಾವಿ-ಸು-ವಷ್ಟು
ಭಾವಿ-ಸು-ವು-ದಾ-ದರೆ
ಭಾವೀ
ಭಾವೀ-ಭಾ-ರ-ತ-ವನ್ನು
ಭಾವೋ-ತ್ಕರ್ಷ
ಭಾವೋ-ದ್ದೀ-ಪ್ತ-ರಾ-ದರು
ಭಾವೋ-ದ್ರೇ-ಕ-ವನ್ನು
ಭಾವೋ-ದ್ವೇ-ಗಕ್ಕೆ
ಭಾವೋ-ದ್ವೇ-ಗ-ಗೊಂ-ಡರು
ಭಾವೋ-ದ್ವೇ-ಗ-ಭ-ರಿತ
ಭಾವೋ-ನ್ಮ-ತ್ತ-ರಾಗಿ
ಭಾಷಣ
ಭಾಷ-ಣ-ಪ್ರ-ವ-ಚ-ನ-ಗ-ಳೆಲ್ಲ
ಭಾಷ-ಣ-ಸಂ-ಭಾ-ಷ-ಣೆ-ಗಳ
ಭಾಷ-ಣ-ಕರ್ತೆ
ಭಾಷ-ಣ-ಕ-ಲೆಯ
ಭಾಷ-ಣಕ್ಕೆ
ಭಾಷ-ಣ-ಗಳ
ಭಾಷ-ಣ-ಗ-ಳಂ-ತೆಯೇ
ಭಾಷ-ಣ-ಗ-ಳನ್ನಾ
ಭಾಷ-ಣ-ಗಳನ್ನು
ಭಾಷ-ಣ-ಗಳನ್ನೂ
ಭಾಷ-ಣ-ಗಳನ್ನೆಲ್ಲ
ಭಾಷ-ಣ-ಗ-ಳ-ನ್ನೇನೋ
ಭಾಷ-ಣ-ಗಳಲ್ಲಿ
ಭಾಷ-ಣ-ಗ-ಳಲ್ಲೂ
ಭಾಷ-ಣ-ಗ-ಳ-ಲ್ಲೆಲ್ಲ
ಭಾಷ-ಣ-ಗ-ಳ-ಲ್ಲೊಂದು
ಭಾಷ-ಣ-ಗ-ಳಷ್ಟೇ
ಭಾಷ-ಣ-ಗ-ಳಿ-ಗಿಂತ
ಭಾಷ-ಣ-ಗಳು
ಭಾಷ-ಣ-ಗಳೂ
ಭಾಷ-ಣದ
ಭಾಷ-ಣ-ದಲ್ಲಿ
ಭಾಷ-ಣ-ದಲ್ಲೂ
ಭಾಷ-ಣ-ದ-ವ-ರೆಗೂ
ಭಾಷ-ಣ-ದಿಂದ
ಭಾಷ-ಣ-ದೊಂ-ದಿಗೆ
ಭಾಷ-ಣ-ಮಾ-ಲಿಕೆ
ಭಾಷ-ಣ-ವ-ನ್ನಾ-ರಂ-ಭಿ-ಸಿ-ದರು
ಭಾಷ-ಣ-ವ-ನ್ನಿ-ರಿ-ಸಿದ
ಭಾಷ-ಣ-ವನ್ನು
ಭಾಷ-ಣ-ವನ್ನೇ
ಭಾಷ-ಣ-ವ-ನ್ನೇ-ರ್ಪ-ಡಿ-ಸಿ-ದರು
ಭಾಷ-ಣ-ವಾ-ಗಿತ್ತು
ಭಾಷ-ಣವು
ಭಾಷ-ಣವೂ
ಭಾಷ-ಣ-ವೊಂ-ದನ್ನು
ಭಾಷ-ಣೋ-ಪ-ಯೋಗಿ
ಭಾಷಾಂ-ತ-ರಿಸಿ
ಭಾಷಾ-ಜ್ಞಾನ
ಭಾಷೆ
ಭಾಷೆ-ಇ-ವು-ಗಳನ್ನು
ಭಾಷೆ-ಗ-ಳಲ್ಲೂ
ಭಾಷೆ-ಗ-ಳಿಗೆ
ಭಾಷೆಗೆ
ಭಾಷೆಯ
ಭಾಷೆ-ಯನ್ನು
ಭಾಷೆ-ಯನ್ನೂ
ಭಾಷೆ-ಯಲ್ಲಿ
ಭಾಷೆ-ಯ-ಲ್ಲಿಯೇ
ಭಾಷೆ-ಯಲ್ಲೂ
ಭಾಷೆ-ಯಲ್ಲೇ
ಭಾಷೆ-ಯಿ-ದೆ-ಯಲ್ಲ
ಭಾಷ್ಯಂ
ಭಾಷ್ಯ-ಕಾ-ರನ
ಭಾಷ್ಯ-ಕಾ-ರರ
ಭಾಷ್ಯ-ಕಾ-ರರು
ಭಾಷ್ಯ-ಗಳ
ಭಾಷ್ಯ-ಗಳನ್ನು
ಭಾಷ್ಯ-ಗಳನ್ನೂ
ಭಾಷ್ಯ-ದಂ-ತಿತ್ತು
ಭಾಷ್ಯ-ವನ್ನೂ
ಭಾಸ-ವಾ-ಗಿತ್ತು
ಭಾಸ-ವಾ-ಗು-ತ್ತಿತ್ತು
ಭಾಸ-ವಾ-ಗು-ತ್ತಿದೆ
ಭಾಸ-ವಾ-ಯಿತು
ಭಾಸ್ಕರ
ಭಿಂಗಾದ
ಭಿಕ್ಷಾ-ಟ-ನೆಯ
ಭಿಕ್ಷಾ-ನ್ನ-ವ-ನ್ನ-ವ-ಲಂ-ಬಿಸಿ
ಭಿಕ್ಷಾ-ನ್ನ-ವನ್ನು
ಭಿಕ್ಷು-ಕರ
ಭಿಕ್ಷು-ಕರು
ಭಿಕ್ಷುಕಿ
ಭಿಕ್ಷು-ಗಳು
ಭಿಕ್ಷೆ
ಭಿಕ್ಷೆಗೆ
ಭಿಕ್ಷೆಯ
ಭಿಕ್ಷೆ-ಯನ್ನು
ಭಿಕ್ಷೆ-ಯೊ-ದ-ಗಿ-ಸು-ತ್ತಿದ್ದ
ಭಿಕ್ಷೆ-ಹಾ-ಕ-ಲೆಂದು
ಭಿನ್ನ-ತೆ-ಗಳ
ಭಿನ್ನ-ತೆ-ಗಳು
ಭಿನ್ನ-ಭಿನ್ನ
ಭಿನ್ನ-ವಾಗಿ
ಭಿನ್ನ-ವಾ-ಗಿದೆ
ಭಿನ್ನ-ವಾ-ಗಿ-ರ-ಬ-ಹುದು
ಭಿನ್ನ-ವಾದ
ಭಿನ್ನಾ-ಭಿ-ಪ್ರಾಯ
ಭೀಕರ
ಭೀಕ-ರ-ತೆಯ
ಭೀಕ-ರ-ತೆ-ಯನ್ನು
ಭೀತಿ
ಭೀಮ-ಕಾ-ಯನೂ
ಭೀರು-ತೆ-ಯನ್ನು
ಭೀಷ-ಣ-ತೆ-ಗೋ-ಸ್ಕ-ರವೇ
ಭೀಷ-ಣ-ತೆ-ಯನ್ನು
ಭುಜದ
ಭುಜ-ವನ್ನು
ಭುವ-ನೇ-ಶ್ವರಿ
ಭುವ-ನೇ-ಶ್ವ-ರಿಗೆ
ಭುವ-ನೇ-ಶ್ವ-ರಿ-ದೇವಿ
ಭುವ-ನೇ-ಶ್ವ-ರಿ-ದೇ-ವಿ-ಗೊಂದು
ಭುವಿ-ಯಲ್ಲಿ
ಭೂಕಂಪ
ಭೂಕಂ-ಪ-ದಿಂ-ದಾಗಿ
ಭೂಕಂ-ಪ-ನ-ವನ್ನು
ಭೂಗ-ತ-ವಾ-ಗು-ತ್ತದೆ
ಭೂತ
ಭೂತ-ವ-ರ್ತ-ಮಾ-ನ-ಗಳನ್ನು
ಭೂತ-ಕಾ-ಲವು
ಭೂತ-ಗ-ಣ-ಗಳು
ಭೂತ-ವನ್ನು
ಭೂತಿ-ಬೆಂ-ಬ-ಲ-ಗ-ಳಿ-ಗಾ-ಗಿಯೇ
ಭೂತಿ-ಯಾ-ಗದೇ
ಭೂತೋ
ಭೂಪ-ಟದ
ಭೂಪೇಂ-ದ್ರ-ನಾಥ
ಭೂಪೇಂ-ದ್ರ-ನಾ-ಥ-ರಿ-ಬ್ಬರೂ
ಭೂಭಾ-ಗ-ವನ್ನೂ
ಭೂಮಿ
ಭೂಮಿ-ಆಸ್ತಿ
ಭೂಮಿ-ಕೆ-ಯಲ್ಲಿ
ಭೂಮಿ-ಕೆ-ಯಲ್ಲೇ
ಭೂಮಿಯ
ಭೂಮಿ-ಯನ್ನು
ಭೂಮಿ-ಯಲ್ಲಿ
ಭೂಮಿಯು
ಭೂಮಿ-ಯೇನೋ
ಭೂಷಣ
ಭೂಷ-ಣ-ರಾ-ಗಿ-ದ್ದಾ-ರೆಯೇ
ಭೂಷ-ಣ-ವಲ್ಲ
ಭೂಷ-ಣ-ವಾ-ದುದು
ಭೇಕು-ರಯೋ
ಭೇಟಿ
ಭೇಟಿ-ಕೊ-ಟ್ಟರು
ಭೇಟಿ-ಕೊಟ್ಟು
ಭೇಟಿ-ಗಾಗಿ
ಭೇಟಿ-ಮಾ-ಡಿ-ದರು
ಭೇಟಿಯ
ಭೇಟಿ-ಯನ್ನು
ಭೇಟಿ-ಯಲ್ಲಿ
ಭೇಟಿ-ಯಾ-ಗ-ಲಿ-ಚ್ಛಿ-ಸಿ-ದರು
ಭೇಟಿ-ಯಾ-ಗಲು
ಭೇಟಿ-ಯಾ-ಗ-ಲೇ-ಬಾ-ರ-ದೆಂದು
ಭೇಟಿ-ಯಾಗಿ
ಭೇಟಿ-ಯಾ-ಗಿದ್ದ
ಭೇಟಿ-ಯಾ-ಗಿ-ದ್ದರು
ಭೇಟಿ-ಯಾ-ಗಿ-ದ್ದಳು
ಭೇಟಿ-ಯಾ-ಗಿ-ದ್ದಾಗ
ಭೇಟಿ-ಯಾ-ಗಿ-ದ್ದಾರೆ
ಭೇಟಿ-ಯಾ-ಗು-ತ್ತಿದ್ದ
ಭೇಟಿ-ಯಾ-ಗು-ತ್ತಿ-ದ್ದಂ-ತೆಯೇ
ಭೇಟಿ-ಯಾ-ಗು-ತ್ತಿ-ದ್ದರು
ಭೇಟಿ-ಯಾ-ಗು-ತ್ತಿ-ರುವ
ಭೇಟಿ-ಯಾ-ಗುವ
ಭೇಟಿ-ಯಾ-ಗು-ವಂತೆ
ಭೇಟಿ-ಯಾ-ಗು-ವುದು
ಭೇಟಿ-ಯಾದ
ಭೇಟಿ-ಯಾ-ದದ್ದು
ಭೇಟಿ-ಯಾ-ದರು
ಭೇಟಿ-ಯಾ-ದರೂ
ಭೇಟಿ-ಯಾ-ದಳು
ಭೇಟಿ-ಯಾ-ದಾಗ
ಭೇಟಿ-ಯಾ-ದೆವು
ಭೇಟಿ-ಯಾ-ಯಿತು
ಭೇಟಿ-ಯಿ-ತ್ತರು
ಭೇಟಿ-ಯಿ-ತ್ತಾಗ
ಭೇಟಿ-ಯಿ-ತ್ತಿ-ದ್ದರು
ಭೇಟಿ-ಯಿ-ತ್ತಿ-ದ್ದಾರೆ
ಭೇಟಿ-ಯಿತ್ತು
ಭೇಟಿಯು
ಭೇದ
ಭೇದ-ಗಳ
ಭೇದ-ಗಳನ್ನು
ಭೇದ-ಗ-ಳ-ನ್ನೆ-ಣಿ-ಸದೆ
ಭೇದ-ದೃ-ಷ್ಟಿ-ಯೆಂಬ
ಭೇದ-ಭಾ-ವ-ಕ್ಕಿಂತ
ಭೇದ-ಭಾ-ವ-ಗಳನ್ನೂ
ಭೇದ-ಭಾ-ವ-ವನ್ನು
ಭೇದ-ಭಾ-ವ-ವಿ-ಲ್ಲದೆ
ಭೇದ-ವನ್ನು
ಭೇದ-ವನ್ನೂ
ಭೇದವೇ
ಭೇದಿಸಿ
ಭೇರಿಯ
ಭೈರ-ವ-ಗ-ಣ-ಗ-ಳಂತೆ
ಭೈರ-ವನ
ಭೈರ-ವ-ಸ್ವ-ರೂಪ
ಭೋಗ
ಭೋಗ-ಕ್ಕಾಗಿ
ಭೋಗಕ್ಕೆ
ಭೋಗ-ಗಳನ್ನು
ಭೋಗ-ಭ-ರಿತ
ಭೋಗ-ಭೂ-ಮಿ-ಯಿಂದ
ಭೋಗ-ವಾ-ದಕ್ಕೂ
ಭೋಗ-ವಾ-ದವು
ಭೋಗ-ವಾದೀ
ಭೋಗ-ವಿ-ಲಾ-ಸ-ದಲ್ಲಿ
ಭೋಜನ
ಭೋಜ-ನಕ್ಕೆ
ಭೋಜ-ನ-ದಲ್ಲೂ
ಭೋಜ-ನ-ವ-ನ್ನಿಕ್ಕಿ
ಭೋಜ-ನ-ವಾದ
ಭೋರ್ಗ-ರೆ-ಯುತ್ತ
ಭೋರ್ಗ-ರೆ-ಯು-ತ್ತಿತ್ತು
ಭೋರ್ಗ-ರೆ-ಯು-ತ್ತಿ-ದ್ದರೆ
ಭೋಲಾ-ನಾಥ
ಭೌತಿಕ
ಭೌತಿ-ಕವೇ
ಭ್ಯಾಸ
ಭ್ಯಾಸ-ಕ್ಕಾಗಿ
ಭ್ರಮಿ-ಸು-ತ್ತಿ-ರು-ತ್ತೇವೆ
ಭ್ರಮೆ
ಭ್ರಮೆ-ಯನ್ನು
ಭ್ರಮೆ-ಯಿತ್ತು
ಭ್ರಷ್ಟಾ-ಚಾರ
ಭ್ರಷ್ಟಾ-ಚಾ-ರ-ಗಳ
ಭ್ರಾತೃತ್ವ
ಭ್ರಾತೃ-ತ್ವದ
ಭ್ರಾತೃ-ತ್ವ-ವನ್ನು
ಮಂಕಾ-ಗಿತ್ತು
ಮಂಕು
ಮಂಕು-ತ-ನ-ಗ-ಳ-ನ್ನಲ್ಲ
ಮಂಗ-ಮಾ-ಯ-ವಾ-ಗುವ
ಮಂಗಳ
ಮಂಗ-ಳ-ಕ-ರ-ವಾ-ಗಿದೆ
ಮಂಗ-ಳ-ಕ-ರ-ವಾದ
ಮಂಗ-ಳ-ದ್ರ-ವ್ಯ-ಗಳ
ಮಂಗ-ಳ-ಧ್ವನಿ
ಮಂಗ-ಳ-ವಾರ
ಮಂಗ-ಳ-ವೆಷ್ಟೋ
ಮಂಗೋ-ಲರೇ
ಮಂಚದ
ಮಂಚ-ವನ್ನು
ಮಂಜಾ-ದವು
ಮಂಜಿನ
ಮಂಜಿ-ನಲ್ಲಿ
ಮಂಜಿ-ನಿಂದ
ಮಂಜಿ-ನಿಂ-ದಾ-ವೃ-ತ-ವಾದ
ಮಂಟ-ಪ-ದಲ್ಲಿ
ಮಂಡ-ಲಿಯ
ಮಂಡ-ಳಿ-ಗಳು
ಮಂಡಿ-ಯೂರಿ
ಮಂಡಿ-ವ-ರೆಗೂ
ಮಂಡಿಸಿ
ಮಂಡಿ-ಸಿದ
ಮಂಡಿ-ಸಿ-ದರು
ಮಂತ್ರ
ಮಂತ್ರ-ಗಳ
ಮಂತ್ರ-ಗಳನ್ನು
ಮಂತ್ರ-ಗ-ಳಲ್ಲೂ
ಮಂತ್ರ-ಗಳೂ
ಮಂತ್ರ-ಘೋ-ಷ-ಗಳ
ಮಂತ್ರದ
ಮಂತ್ರ-ದೀಕ್ಷೆ
ಮಂತ್ರ-ದೀ-ಕ್ಷೆ-ಯನ್ನು
ಮಂತ್ರ-ಮು-ಗ್ಧ-ನಂತೆ
ಮಂತ್ರ-ಮು-ಗ್ಧ-ರಂತೆ
ಮಂತ್ರ-ಮು-ಗ್ಧ-ರ-ನ್ನಾ-ಗಿ-ಸಿ-ದರು
ಮಂತ್ರ-ಮು-ಗ್ಧ-ವಾ-ಗಿಸಿ
ಮಂತ್ರ-ವನ್ನು
ಮಂತ್ರ-ವಾ-ಗ-ಬೇಕು
ಮಂತ್ರ-ವಾ-ದಿಯ
ಮಂತ್ರ-ಸ್ಪರ್ಶ
ಮಂತ್ರಾ-ಕ್ಷತೆ
ಮಂತ್ರೋ
ಮಂತ್ರೋ-ಪ-ದೇಶ
ಮಂತ್ರೋ-ಪ-ದೇ-ಶದ
ಮಂತ್ರೋ-ಪ-ದೇ-ಶ-ವನ್ನು
ಮಂಥ-ನ-ಗಳು
ಮಂದ-ಗ-ತಿ-ಯಲ್ಲಿ
ಮಂದ-ಪ್ರ-ಕಾ-ಶವೂ
ಮಂದ-ಮಾ-ರುತ
ಮಂದ-ವಾ-ದರೂ
ಮಂದ-ವಾ-ಯಿತು
ಮಂದ-ಹಾಸ
ಮಂದ-ಹಾ-ಸದ
ಮಂದ-ಹಾ-ಸ-ವನ್ನು
ಮಂದಿ
ಮಂದಿ-ಗಾ-ದರೂ
ಮಂದಿಗೆ
ಮಂದಿಯ
ಮಂದಿರ
ಮಂದಿ-ರ-ದಲ್ಲಿ
ಮಂದಿ-ರ-ದೊ-ಳಗೆ
ಮಂದಿ-ರ-ವನ್ನು
ಮಂದಿ-ರ-ವೊಂದು
ಮಕು-ಟ-ಮ-ಣಿ-ಯಾ-ದರೆ
ಮಕ್ಕಳ
ಮಕ್ಕ-ಳಂತೆ
ಮಕ್ಕ-ಳಂ-ತೆಯೇ
ಮಕ್ಕ-ಳನ್ನು
ಮಕ್ಕ-ಳಲ್ಲಿ
ಮಕ್ಕ-ಳಾದ
ಮಕ್ಕ-ಳಿ-ಗಾಗಿ
ಮಕ್ಕ-ಳಿ-ಗಿಂತ
ಮಕ್ಕ-ಳಿಗೂ
ಮಕ್ಕ-ಳಿಗೆ
ಮಕ್ಕ-ಳಿ-ಗೆಲ್ಲ
ಮಕ್ಕಳು
ಮಕ್ಕ-ಳು-ಎ-ಲ್ಲರೂ
ಮಕ್ಕಳೆ
ಮಕ್ಕ-ಳೆಂದು
ಮಕ್ಕ-ಳೆಂದೇ
ಮಕ್ಕ-ಳೆಲ್ಲ
ಮಕ್ಕ-ಳೆ-ಲ್ಲರೂ
ಮಕ್ಕಳೇ
ಮಕ್ಕ-ಳೊಂ-ದಿಗೆ
ಮಕ್ಕ-ಳೊ-ಡನೆ
ಮಗ
ಮಗನ
ಮಗ-ನಂತೆ
ಮಗ-ನನ್ನು
ಮಗ-ನ-ಲ್ಲವೆ
ಮಗ-ನಾದ
ಮಗ-ನಿಗೆ
ಮಗ-ನೆಂದೇ
ಮಗಳ
ಮಗ-ಳಂತೆ
ಮಗಳನ್ನು
ಮಗ-ಳಿ-ಬ್ಬರೂ
ಮಗಳು
ಮಗಳೇ
ಮಗು
ಮಗು-ವಿ-ಗಿಂತ
ಮಗು-ವಿಗೆ
ಮಗು-ವಿನ
ಮಗು-ವಿ-ನಂ-ತಾ-ಗಿ-ಬಿ-ಟ್ಟರು
ಮಗು-ವಿ-ನಂತೆ
ಮಗುವು
ಮಗು-ವೊಂ-ದನ್ನು
ಮಗೂ
ಮಗ್ಗು-ಲಲ್ಲಿ
ಮಗ್ನ-ನಾ-ಗಿ-ದ್ದು-ಬಿ-ಡು-ತ್ತೇನೆ
ಮಗ್ನ-ರಾಗಿ
ಮಗ್ನ-ರಾ-ಗಿ-ದ್ದರೂ
ಮಗ್ನ-ರಾ-ಗಿ-ದ್ದು-ದನ್ನು
ಮಗ್ನ-ರಾ-ಗಿ-ದ್ದು-ಬಿ-ಡುವ
ಮಗ್ನ-ರಾ-ಗಿ-ರ-ಬೇ-ಕಾ-ಯಿತು
ಮಗ್ನ-ರಾ-ಗಿ-ರ-ಬೇ-ಕೆಂಬ
ಮಗ್ನ-ರಾ-ದುದು
ಮಜು-ಮ್ದಾ-ರ-ರಿಗೂ
ಮಜು-ಮ್ದಾರ್
ಮಟ-ಮಟ
ಮಟ್ಟ
ಮಟ್ಟಕ್ಕೆ
ಮಟ್ಟದ
ಮಟ್ಟ-ದ-ಲ್ಲಿ-ರುವ
ಮಟ್ಟ-ದ್ದಾ-ಗಿತ್ತು
ಮಟ್ಟ-ದ್ದಾ-ಗಿ-ರ-ಬೇಕು
ಮಟ್ಟ-ಹಾ-ಕ-ದಿ-ದ್ದು-ದಕ್ಕೆ
ಮಟ್ಟಿ-ಗಂತೂ
ಮಟ್ಟಿ-ಗಷ್ಟೇ
ಮಟ್ಟಿ-ಗಾ-ದರೂ
ಮಟ್ಟಿ-ಗಿ-ತ್ತೆಂ-ದರೆ
ಮಟ್ಟಿಗೂ
ಮಟ್ಟಿಗೆ
ಮಟ್ಟಿ-ಗೆಂ-ದರೆ
ಮಟ್ಟಿ-ಗೆ-ಹೋ-ಗ-ಬೇಕು
ಮಠ
ಮಠ-ಆ-ಶ್ರ-ಮ-ಗಳನ್ನು
ಮಠ-ಕ್ಕಾಗಿ
ಮಠಕ್ಕೆ
ಮಠ-ಕ್ಕೊಂದು
ಮಠ-ಗ-ಳಂತೆ
ಮಠ-ಗ-ಳ-ನ್ನು-ಸೇ-ವಾ-ಶ್ರ-ಮ-ಗಳನ್ನು
ಮಠ-ಗಳಲ್ಲಿ
ಮಠ-ಗ-ಳಲ್ಲೂ
ಮಠ-ಗಳು
ಮಠ-ಗ-ಳೊ-ಳ-ಗಿ-ನಿಂದ
ಮಠದ
ಮಠ-ದಲ್ಲಿ
ಮಠ-ದ-ಲ್ಲಿದ್ದ
ಮಠ-ದ-ಲ್ಲಿ-ದ್ದದ್ದು
ಮಠ-ದ-ಲ್ಲಿ-ದ್ದಾಗ
ಮಠ-ದ-ಲ್ಲಿ-ದ್ದಾರೆ
ಮಠ-ದ-ಲ್ಲಿನ
ಮಠ-ದ-ಲ್ಲಿ-ರ-ಲಿಲ್ಲ
ಮಠ-ದಲ್ಲೆಲ್ಲ
ಮಠ-ದಲ್ಲೇ
ಮಠ-ದ-ಲ್ಲೊಂದು
ಮಠ-ದಲ್ಲೋ
ಮಠ-ದಿಂದ
ಮಠ-ವನ್ನು
ಮಠವು
ಮಠವೇ
ಮಠ-ವೊಂ-ದನ್ನು
ಮಠ-ಸ್ಥಾ-ಪ-ನೆ-ಗಾಗಿ
ಮಠ-ಸ್ಥಾ-ಪ-ನೆಯ
ಮಡ-ಕೆ-ಯಂ-ತಹ
ಮಡದಿ
ಮಡ-ದಿ-ಮ-ಕ್ಕಳು
ಮಡ-ಲಿಗೆ
ಮಡಿ
ಮಡಿ-ಮೈ-ಲಿಗೆ
ಮಡಿ-ತ-ನದ
ಮಡಿದ
ಮಡಿ-ದರು
ಮಡಿ-ಯಲ್ಲಿ
ಮಡಿ-ಲಲ್ಲಿ
ಮಡಿ-ಲ-ಲ್ಲಿ-ರಿಸಿ
ಮಡಿ-ವಂತ
ಮಡಿ-ವಂ-ತ-ರಾದ
ಮಡಿ-ವಂ-ತ-ರಿಂದ
ಮಡಿ-ವಂ-ತರು
ಮಡಿ-ವಂ-ತಿ-ಕೆ-ಯನ್ನು
ಮಡಿ-ವಂ-ತಿ-ಕೆಯೇ
ಮಣ
ಮಣಿ-ದರು
ಮಣಿದು
ಮಣಿಯ
ಮಣಿ-ಯ-ಲೇ-ಬೇ-ಕಾ-ಯಿತು
ಮಣ್ಣ-ಧೂ-ಳಿ-ನಲ್ಲಿ
ಮಣ್ಣ-ಹೆಂ-ಟೆ-ಯಿಂದ
ಮಣ್ಣಿ
ಮಣ್ಣಿನ
ಮಣ್ಣು
ಮಣ್ಣು-ಬೂ-ದಿಯ
ಮಣ್ಣು-ಗಳ
ಮಣ್ಣು-ಪಾಲಾ
ಮಣ್ಣು-ಪಾ-ಲಾ-ಯಿತು
ಮಣ್ಣು-ಹುಡಿ
ಮಣ್ಣೆ-ರ-ಚುತ್ತ
ಮಣ್ಣೊ-ಳಗೆ
ಮತ
ಮತ-ಧ-ರ್ಮ-ಗಳ
ಮತ-ಪಂ-ಗ-ಡ-ಗ-ಳಲ್ಲೂ
ಮತ-ಪಂ-ಥ-ಗ-ಳ-ವರೂ
ಮತ-ಗಳನ್ನು
ಮತ-ಗ-ಳ-ವರ
ಮತ-ಗಳಿಂದ
ಮತ-ಗಳು
ಮತ-ಗ-ಳು-ಸಂ-ಪ್ರ-ದಾ-ಯ-ಗಳು
ಮತ-ಗಳೂ
ಮತದ
ಮತ-ಧ-ರ್ಮ-ಗಳ
ಮತ-ಧ-ರ್ಮ-ಗಳನ್ನೂ
ಮತ-ಪಂ-ಗ-ಡ-ಗಳನ್ನು
ಮತ-ಪಂ-ಗ-ಡ-ಗ-ಳೆ-ಲ್ಲಕ್ಕೂ
ಮತ-ಪಂ-ಗ-ಡ-ಗ-ಳೊ-ಳಕ್ಕೂ
ಮತ-ಪಂ-ಥ-ಗಳ
ಮತ-ಪಂ-ಥ-ಗಳಿಂದ
ಮತ-ಪಂ-ಥ-ಗ-ಳಿಗೂ
ಮತ-ಪಂ-ಥ-ಗ-ಳೆ-ಲ್ಲ-ದರ
ಮತ-ಪಂ-ಥ-ಗ-ಳೇಕೆ
ಮತ-ಪ್ರ-ಚಾ-ರವೂ
ಮತ-ಪ್ರ-ಮು-ಖರು
ಮತ-ವ-ನ್ನಾ-ಗಲಿ
ಮತ-ವನ್ನು
ಮತ-ವನ್ನೇ
ಮತ-ವಾದ
ಮತ-ವಾ-ದ-ರೇನು
ಮತವೂ
ಮತ-ಸ್ಥರು
ಮತ-ಸ್ಥರೂ
ಮತ-ಸ್ಥಳು
ಮತಾಂ-ತರ
ಮತಾಂಧ
ಮತಾಂ-ಧತೆ
ಮತಾಂ-ಧ-ತೆ-ಯನ್ನೂ
ಮತಾಂ-ಧರ
ಮತಾಂ-ಧರೂ
ಮತೀಯ
ಮತ್ತ-ರಾಗಿ
ಮತ್ತ-ರಾ-ಗಿ-ರು-ವಂತೆ
ಮತ್ತ-ವರ
ಮತ್ತಷ್ಟು
ಮತ್ತಾರು
ಮತ್ತಾರೂ
ಮತ್ತಿ-ತರ
ಮತ್ತಿ-ತ-ರರ
ಮತ್ತಿ-ತ-ರರು
ಮತ್ತಿ-ನ್ನೆ-ಲ್ಲಿಗೆ
ಮತ್ತಿ-ನ್ನೇನು
ಮತ್ತು
ಮತ್ತು-ಅ-ತ್ಯಂತ
ಮತ್ತು-ಇ-ಸ್ಲಾಂ
ಮತ್ತೂ
ಮತ್ತೆ
ಮತ್ತೆಂದೂ
ಮತ್ತೆ-ಮತ್ತೆ
ಮತ್ತೆ-ರಡು
ಮತ್ತೆಷ್ಟು
ಮತ್ತೇನು
ಮತ್ತೇನೂ
ಮತ್ತೊಂ-ದ-ಕ್ಕಿಂತ
ಮತ್ತೊಂ-ದಕ್ಕೆ
ಮತ್ತೊಂ-ದರ
ಮತ್ತೊಂ-ದ-ರಲ್ಲಿ
ಮತ್ತೊಂ-ದಿ-ರ-ಲಾ-ರದು
ಮತ್ತೊಂದು
ಮತ್ತೊಂ-ದೆಡೆ
ಮತ್ತೊಂ-ದೆ-ಡೆ-ಯಲ್ಲಿ
ಮತ್ತೊಬ್ಬ
ಮತ್ತೊ-ಬ್ಬನು
ಮತ್ತೊ-ಬ್ಬ-ರ-ನ್ನು-ನನ
ಮತ್ತೊ-ಬ್ಬರು
ಮತ್ತೊ-ಬ್ಬಳು
ಮತ್ತೊ-ಬ್ಬ-ಳೆಂ-ದರೆ
ಮತ್ತೊಮ್ಮೆ
ಮತ್ಸ-ರ-ದಿಂದ
ಮತ್ಸ-ರ-ಪ-ಡು-ವಂಥ
ಮದ-ರಾಸಿ
ಮದ-ರಾ-ಸಿನ
ಮದುವೆ
ಮದು-ವೆಯ
ಮದು-ವೆ-ಯನ್ನು
ಮದು-ವೆ-ಯಾ-ಗ-ಲಾರೆ
ಮದು-ವೆ-ಯಾಗಿ
ಮದು-ವೆ-ಯಾ-ಗುವ
ಮದು-ವೆ-ಯಾ-ಗು-ವುದು
ಮದು-ವೆಯೂ
ಮದು-ವೆ-ಯೊಂದು
ಮದ್ಯ-ಪಾ-ನಕ್ಕೆ
ಮದ್ರಾ
ಮದ್ರಾಸಿ
ಮದ್ರಾ-ಸಿ-ಗರು
ಮದ್ರಾ-ಸಿಗೆ
ಮದ್ರಾ-ಸಿನ
ಮದ್ರಾ-ಸಿ-ನಂ-ತಹ
ಮದ್ರಾ-ಸಿ-ನತ್ತ
ಮದ್ರಾ-ಸಿ-ನಲ್ಲಿ
ಮದ್ರಾ-ಸಿ-ನ-ಲ್ಲಿದ್ದ
ಮದ್ರಾ-ಸಿ-ನ-ಲ್ಲಿ-ದ್ದಷ್ಟು
ಮದ್ರಾ-ಸಿ-ನಲ್ಲೇ
ಮದ್ರಾ-ಸಿ-ನ-ವ-ರೆಗೆ
ಮದ್ರಾ-ಸಿ-ನಿಂದ
ಮದ್ರಾಸೀ
ಮದ್ರಾಸು
ಮದ್ರಾ-ಸು-ಕ-ಲ್ಕ-ತ್ತ-ಗ-ಳ-ಲ್ಲಂತೂ
ಮದ್ರಾ-ಸು-ಮುಂ-ಬ-ಯಿ-ಬಂ-ಗಾಳ
ಮದ್ರಾ-ಸು-ಗಳ
ಮದ್ರಾಸ್
ಮಧು-ಕರಿ
ಮಧು-ಕ-ರಿಯ
ಮಧುರ
ಮಧು-ರ-ಸ-ಹಜ
ಮಧು-ರ-ವಾಗಿ
ಮಧು-ರ-ವಾದ
ಮಧು-ರ-ವಾ-ದದ್ದು
ಮಧುರೆ
ಮಧು-ರೆಗೆ
ಮಧು-ರೆಯ
ಮಧು-ರೆ-ಯಲ್ಲಿ
ಮಧು-ರೆ-ಯಲ್ಲೂ
ಮಧ್ಯ
ಮಧ್ಯ-ದ-ಲ್ಲಿ-ದ್ದಾರೆ
ಮಧ್ಯ-ದಲ್ಲೂ
ಮಧ್ಯ-ದಲ್ಲೇ
ಮಧ್ಯ-ಬಾಗ
ಮಧ್ಯ-ಭಾ-ಗ-ದಲ್ಲಿ
ಮಧ್ಯ-ಭಾ-ರ-ತ-ದಲ್ಲಿ
ಮಧ್ಯ-ರಾತ್ರಿ
ಮಧ್ಯ-ರಾ-ತ್ರಿಯ
ಮಧ್ಯ-ವರ್ತಿ
ಮಧ್ಯ-ವ-ರ್ತಿ-ಯಾದ
ಮಧ್ಯಾಹ್ನ
ಮಧ್ಯಾ-ಹ್ನದ
ಮಧ್ಯಾ-ಹ್ನ-ವೆಲ್ಲ
ಮಧ್ಯೆ
ಮಧ್ಯೆ-ಮಧ್ಯೆ
ಮನ
ಮನಃ-ಪ-ಟ-ಲದ
ಮನಃ-ಪೂ-ರ್ವ-ಕ-ವಾಗಿ
ಮನಃ-ಶಾಸ್ತ್ರ
ಮನಃ-ಸ್ಥಿ-ತಿ-ಯನ್ನು
ಮನಃ-ಸ್ಥಿ-ತಿ-ಯಲ್ಲಿ
ಮನ-ಕ-ರ-ಗಿ-ಸಿ-ದ್ದು-ದನ್ನು
ಮನ-ಗಂಡ
ಮನ-ಗಂ-ಡರು
ಮನ-ಗಂ-ಡರೂ
ಮನ-ಗಂ-ಡಿದ್ದ
ಮನ-ಗಂ-ಡಿ-ದ್ದರು
ಮನ-ಗಂ-ಡಿ-ದ್ದೇನೆ
ಮನ-ಗಂ-ಡಿ-ರುವ
ಮನ-ಗಂಡು
ಮನ-ಗಂಡೇ
ಮನ-ಗಾ-ಣಿ-ಸ-ಬೇ-ಕಾ-ಗು-ತ್ತದೆ
ಮನ-ಗಾ-ಣಿ-ಸ-ಲಾ-ರೆವು
ಮನ-ಗಾ-ಣಿ-ಸಿ-ದರು
ಮನ-ಗಾ-ಣಿ-ಸು-ತ್ತಿ-ದ್ದರು
ಮನ-ಗಾ-ಣಿ-ಸು-ವಲ್ಲಿ
ಮನ-ಗಾ-ಣು-ತ್ತಿದೆ
ಮನದ
ಮನ-ದ-ಟ್ಟಾ-ಗಿತ್ತು
ಮನ-ದ-ಟ್ಟಾ-ಯಿತು
ಮನ-ದಟ್ಟು
ಮನ-ದಲ್ಲಿ
ಮನ-ದ-ಲ್ಲಿತ್ತು
ಮನ-ದಲ್ಲೂ
ಮನ-ದಲ್ಲೇ
ಮನ-ದ-ಲ್ಲೊಂದು
ಮನ-ದಾ-ಳ-ವನ್ನು
ಮನ-ದಾಸೆ
ಮನ-ದಾ-ಸೆ-ಯಂತೆ
ಮನ-ದಿಂ-ಗಿ-ತ-ವನ್ನು
ಮನ-ಮು-ಟ್ಟಿ-ಸಿ-ದರು
ಮನ-ಮು-ಟ್ಟು-ವಂತೆ
ಮನ-ಮೆಚ್ಚು
ಮನ-ಮೋ-ಹಕ
ಮನ-ಮೋ-ಹ-ಕ-ರಾಗಿ
ಮನ-ರಂ-ಜನೆ
ಮನ-ರಂ-ಜ-ನೆ-ಗಾಗಿ
ಮನ-ರಂ-ಜ-ನೆಯ
ಮನ-ರಂ-ಜ-ನೆಯೂ
ಮನ-ವ-ರಿಕೆ
ಮನ-ವ-ರಿ-ಕೆ-ಯಾ-ಗಿದೆ
ಮನ-ವ-ರಿ-ಕೆ-ಯಾ-ಗು-ತ್ತದೆ
ಮನ-ವ-ರಿ-ಕೆ-ಯಾ-ಗು-ತ್ತಿದೆ
ಮನ-ವ-ರಿ-ಕೆ-ಯಾ-ಯಿ-ತು-ಏ-ನೆಂ-ದರೆ
ಮನವಿ
ಮನ-ವಿ-ಯಂತೆ
ಮನ-ವಿ-ಯನ್ನು
ಮನ-ವೊ-ಲಿ-ಸಲು
ಮನ-ವೊ-ಲಿ-ಸಿ-ಕೊಂ-ಡರು
ಮನ-ವೊ-ಲಿ-ಸಿ-ದರು
ಮನ-ವೊ-ಲಿಸು
ಮನ-ವೊ-ಲಿ-ಸುವ
ಮನ-ವೊ-ಲಿ-ಸು-ವುದು
ಮನ-ಶ್ಶಾಂತಿ
ಮನ-ಶ್ಶಾಂ-ತಿ-ಗಾಗಿ
ಮನ-ಶ್ಶಾಂ-ತಿಯ
ಮನ-ಶ್ಶಾಂ-ತಿ-ಯನ್ನು
ಮನ-ಶ್ಶಾಸ್ತ್ರ
ಮನ-ಶ್ಶಾ-ಸ್ತ್ರ-ವನ್ನು
ಮನ-ಶ್ಶಾ-ಸ್ತ್ರೀಯ
ಮನ-ಶ್ಶಾ-ಸ್ತ್ರೀ-ಯ-ವಾಗಿ
ಮನ-ಶ್ಶುದ್ಧಿ-ಹೃ-ದ-ಯ-ಶುದ್ಧಿ
ಮನಸಾ
ಮನ-ಸಾರೆ
ಮನ-ಸೂರೆ
ಮನ-ಸೂ-ರೆ-ಗೊಂ-ಡಿತು
ಮನ-ಸೆ-ಳೆ-ದಿದ್ದ
ಮನ-ಸೆ-ಳೆ-ಯುವ
ಮನ-ಸೆ-ಳೆ-ಯು-ವಂ-ತಿ-ದ್ದುವು
ಮನ-ಸೋ-ತಿದ್ದ
ಮನ-ಸ್ಕ-ರಾ-ದ್ದ-ರಿಂದ
ಮನ-ಸ್ಥಿತಿ
ಮನ-ಸ್ಥಿ-ತಿ-ಯಲ್ಲಿ
ಮನ-ಸ್ಥಿ-ತಿ-ಯ-ಲ್ಲಿ-ರು-ವಂತೆ
ಮನ-ಸ್ಸ-ನ್ನಾ-ವ-ರಿ-ಸಿತು
ಮನ-ಸ್ಸನ್ನು
ಮನ-ಸ್ಸಾ-ಯಿತು
ಮನಸ್ಸಿ
ಮನ-ಸ್ಸಿಗೆ
ಮನ-ಸ್ಸಿ-ಗೊಂದು
ಮನ-ಸ್ಸಿಟ್ಟು
ಮನ-ಸ್ಸಿನ
ಮನ-ಸ್ಸಿ-ನಲ್ಲಿ
ಮನ-ಸ್ಸಿ-ನ-ಲ್ಲಿ-ಟ್ಟು-ಕೊಂಡೇ
ಮನ-ಸ್ಸಿ-ನ-ಲ್ಲಿತ್ತು
ಮನ-ಸ್ಸಿ-ನ-ಲ್ಲಿದ್ದ
ಮನ-ಸ್ಸಿ-ನ-ಲ್ಲಿ-ದ್ದುದು
ಮನ-ಸ್ಸಿ-ನ-ಲ್ಲಿಯೇ
ಮನ-ಸ್ಸಿ-ನ-ಲ್ಲಿ-ರು-ವು-ದ-ನ್ನೆಲ್ಲ
ಮನ-ಸ್ಸಿ-ನ-ಲ್ಲು-ದ್ಭ-ವಿ-ಸಿದ
ಮನ-ಸ್ಸಿ-ನಲ್ಲೂ
ಮನ-ಸ್ಸಿ-ನಲ್ಲೇ
ಮನ-ಸ್ಸಿ-ನ-ವರು
ಮನ-ಸ್ಸಿ-ನಿಂದ
ಮನ-ಸ್ಸಿ-ನೊಂ-ದಿಗೆ
ಮನ-ಸ್ಸಿ-ರು-ವ-ವರು
ಮನ-ಸ್ಸಿಲ್ಲ
ಮನ-ಸ್ಸಿ-ಲ್ಲದ
ಮನ-ಸ್ಸೀಗ
ಮನಸ್ಸು
ಮನ-ಸ್ಸು-ಬು-ದ್ಧಿ-ಗಳನ್ನೆಲ್ಲ
ಮನ-ಸ್ಸು-ಗಳನ್ನು
ಮನ-ಸ್ಸು-ಗ-ಳಿಗೆ
ಮನ-ಸ್ಸು-ಗ-ಳೆ-ರಡೂ
ಮನಸ್ಸೂ
ಮನಸ್ಸೇ
ಮನ-ಸ್ಸೊ-ಪ್ಪು-ತ್ತಿಲ್ಲ
ಮನು
ಮನು-ಯಾ-ಜ್ಞ-ವ-ಲ್ಕ್ಯರೇ
ಮನು-ಕು-ಲದ
ಮನುಷ್ಯ
ಮನು-ಷ್ಯ-ಕುಲ
ಮನು-ಷ್ಯ-ಕೃತ
ಮನು-ಷ್ಯ-ಜನ್ಮ
ಮನು-ಷ್ಯತ್ವ
ಮನು-ಷ್ಯತ್ವಂ
ಮನು-ಷ್ಯ-ತ್ವ-ದಲ್ಲಿ
ಮನು-ಷ್ಯ-ತ್ವ-ವನ್ನೇ
ಮನು-ಷ್ಯನ
ಮನು-ಷ್ಯ-ನ-ನ್ನಾಗಿ
ಮನು-ಷ್ಯ-ನನ್ನು
ಮನು-ಷ್ಯ-ನಲ್ಲಿ
ಮನು-ಷ್ಯ-ನಾ-ಗಲು
ಮನು-ಷ್ಯ-ನಾ-ಗಿ-ರು-ವ-ವ-ರೆಗೂ
ಮನು-ಷ್ಯ-ನಾ-ಗು-ವುದು
ಮನು-ಷ್ಯ-ನಿಂದ
ಮನು-ಷ್ಯ-ನಿಗೂ
ಮನು-ಷ್ಯ-ನಿಗೆ
ಮನು-ಷ್ಯನು
ಮನು-ಷ್ಯ-ನೆಂಬ
ಮನು-ಷ್ಯ-ನೆ-ನ್ನಿ-ಸಿ-ಕೊ-ಳ್ಳ-ಬ-ಹುದು
ಮನು-ಷ್ಯ-ನೊ-ಳ-ಗಿನ
ಮನು-ಷ್ಯರ
ಮನು-ಷ್ಯ-ರಂ-ತೆಯೇ
ಮನು-ಷ್ಯ-ರಲ್ಲಿ
ಮನು-ಷ್ಯ-ರಲ್ಲೂ
ಮನು-ಷ್ಯ-ರಾಗಿ
ಮನು-ಷ್ಯ-ರಾ-ಗು-ತ್ತಿ-ದ್ದಾ-ರೆಯೆ
ಮನು-ಷ್ಯ-ರಾ-ದರೆ
ಮನು-ಷ್ಯ-ರಿ-ಗಾ-ಗಿ-ಯಾ-ದರೂ
ಮನು-ಷ್ಯ-ರಿಗೇ
ಮನು-ಷ್ಯರು
ಮನು-ಷ್ಯ-ರೆಂ-ದರೆ
ಮನು-ಷ್ಯರೇ
ಮನು-ಷ್ಯ-ರೇನು
ಮನು-ಷ್ಯ-ಶ-ಕ್ತಿಗೆ
ಮನೆ
ಮನೆ-ಗಳ
ಮನೆ-ಗಳನ್ನು
ಮನೆ-ಗಳಲ್ಲಿ
ಮನೆ-ಗ-ಳಿಗೆ
ಮನೆ-ಗ-ಳಿ-ದ್ದ-ವ-ರೇನೋ
ಮನೆ-ಗಳು
ಮನೆ-ಗಿಂತ
ಮನೆಗೂ
ಮನೆಗೆ
ಮನೆಗೇ
ಮನೆ-ತನ
ಮನೆ-ತ-ನಕ್ಕೆ
ಮನೆ-ತ-ನದ
ಮನೆ-ತ-ನಸ್ಥ
ಮನೆ-ಮಂ-ದಿ-ಯ-ನ್ನೆಲ್ಲ
ಮನೆ-ಮ-ನೆ-ಗಳನ್ನೂ
ಮನೆ-ಮ-ನೆಗೂ
ಮನೆ-ಮಾ-ಡಲು
ಮನೆ-ಮಾ-ಡಿ-ಕೊಂ-ಡಿದ್ದ
ಮನೆ-ಮಾ-ತಾ-ಗಿವೆ
ಮನೆ-ಮಾ-ತಾ-ಯಿತು
ಮನೆ-ಮಾ-ರು-ಗಳನ್ನು
ಮನೆಯ
ಮನೆ-ಯನ್ನು
ಮನೆ-ಯನ್ನೂ
ಮನೆ-ಯಲ್ಲಿ
ಮನೆ-ಯ-ಲ್ಲಿದ್ದ
ಮನೆ-ಯ-ಲ್ಲಿ-ದ್ದಾಗ
ಮನೆ-ಯ-ಲ್ಲಿ-ದ್ದಾ-ಗಲೇ
ಮನೆ-ಯಲ್ಲೇ
ಮನೆ-ಯಲ್ಲೋ
ಮನೆ-ಯ-ವರ
ಮನೆ-ಯ-ವ-ರಿಂದ
ಮನೆ-ಯ-ವ-ರಿಗೂ
ಮನೆ-ಯ-ವ-ರಿಗೆ
ಮನೆ-ಯ-ವರು
ಮನೆ-ಯ-ವರೆಲ್ಲ
ಮನೆ-ಯಿಂದ
ಮನೆ-ಯಿಂ-ದಾ-ಚೆಗೆ
ಮನೆ-ಯಿ-ದ್ದುದು
ಮನೆಯು
ಮನೆಯೂ
ಮನೆ-ಯೆಂ-ದರೆ
ಮನೆಯೇ
ಮನೆ-ಯೊಂ-ದನ್ನು
ಮನೆ-ಯೊ-ಳಗೆ
ಮನೆ-ವಾ-ರ್ತೆಯು
ಮನೋ
ಮನೋ-ಧ-ರ್ಮದ
ಮನೋ-ಭಾವ
ಮನೋ-ಭಾ-ವಕ್ಕೆ
ಮನೋ-ಭಾ-ವ-ಗಳ
ಮನೋ-ಭಾ-ವ-ಗಳನ್ನು
ಮನೋ-ಭಾ-ವ-ತ-ನ್ಮೂ-ಲಕ
ಮನೋ-ಭಾ-ವದ
ಮನೋ-ಭಾ-ವ-ದಲ್ಲಿ
ಮನೋ-ಭಾ-ವ-ದ-ವ-ರಾ-ಗಿ-ರ-ಬೇಕು
ಮನೋ-ಭಾ-ವ-ದ-ವ-ರಾ-ದ್ದ-ರಿಂದ
ಮನೋ-ಭಾ-ವ-ವ-ನ್ನಾ-ದರೂ
ಮನೋ-ಭಾ-ವ-ವನ್ನು
ಮನೋ-ಭಿ-ಲಾ-ಷೆ-ಯನ್ನು
ಮನೋ-ರಂಗ
ಮನೋ-ರಂ-ಜ-ನೆಯ
ಮನೋ-ರಂ-ಜಿನೀ
ಮನೋ-ವೃ-ತ್ತಿ-ಯನ್ನು
ಮನೋ-ವೃ-ತ್ತಿಯೇ
ಮನೋ-ವೈ-ಜ್ಞಾ-ನಿಕ
ಮನೋ-ಹರ
ಮನೋ-ಹ-ರ-ಗಂ-ಭೀರ
ಮನೋ-ಹ-ರ-ವಾ-ಗಿತ್ತು
ಮನೋ-ಹ-ರ-ವಾ-ಗಿವೆ
ಮನೋ-ಹ-ರ-ವಾದ
ಮನೋ-ಹ-ರ-ವಾ-ದದ್ದು
ಮನೋ-ಹ-ರ-ವಾ-ದುದು
ಮನ್ನಣೆ
ಮನ್ನ-ಣೆ-ಕೊಟ್ಟು
ಮನ್ನಿ-ಸಪ್ಪ
ಮನ್ನಿ-ಸ-ಬೇಕು
ಮನ್ನಿ-ಸ-ಲೇ-ಬೇ-ಕಾ-ಗಿತ್ತು
ಮನ್ನಿಸಿ
ಮನ್ನಿ-ಸಿ-ದರು
ಮನ್ನಿ-ಸಿ-ದಳು
ಮನ್ನಿ-ಸಿ-ದ-ವ-ರಲ್ಲಿ
ಮನ್ಮ-ಥ-ನಾಥ
ಮಬ್ಬನು
ಮಬ್ಬು
ಮಬ್ಬು-ಗ-ವಿ-ದಂ-ತಾ-ಗಿದೆ
ಮಮ
ಮಮತೆ
ಮಮ-ತೆ-ಯಿತ್ತು
ಮಯ
ಮರ
ಮರ-ಇ-ವು-ಗಳ
ಮರ-ಕೋ-ತಿ-ಯಾ-ಡು-ತ್ತಿ-ದ್ದಾಗ
ಮರ-ಗ-ಟ್ಟಿ-ಕೊಂಡು
ಮರ-ಗ-ಟ್ಟಿ-ಹೋ-ಗಿತ್ತು
ಮರ-ಗಳ
ಮರ-ಗಳಿಂದ
ಮರ-ಗಳು
ಮರಣ
ಮರ-ಣ-ಕಾಲ
ಮರ-ಣ-ಕಾ-ಲ-ದಲ್ಲಿ
ಮರ-ಣ-ಕ್ಕಿಂ-ತಲೂ
ಮರ-ಣ-ಗಳ
ಮರ-ಣದ
ಮರ-ಣ-ದಿಂ-ದಾಗಿ
ಮರ-ಣ-ಭ-ಯವೂ
ಮರ-ಣ-ಮಸ್ತು
ಮರ-ಣ-ವ-ನ್ನ-ಪ್ಪಿ-ದ-ನೆಂಬ
ಮರ-ಣ-ವ-ನ್ನ-ಪ್ಪಿ-ದರು
ಮರ-ಣ-ವ-ನ್ನ-ಪ್ಪು-ತ್ತೇನೆ
ಮರ-ಣ-ವ-ನ್ನ-ಪ್ಪು-ವಂತೆ
ಮರ-ಣ-ವಿಲ್ಲ
ಮರ-ಣ-ಶ-ಯ್ಯೆ-ಯ-ಲ್ಲಿ-ದ್ದಾರೆ
ಮರ-ಣ-ಹೊಂ-ದು-ವಂತೆ
ಮರ-ಣಾಂ-ತಿಕ
ಮರ-ಣಾ-ನಂ-ತರ
ಮರ-ಣೋ-ನ್ಮು-ಖ-ರಾ-ಗಿ-ದ್ದರು
ಮರ-ಣೋ-ನ್ಮು-ಖ-ರಾ-ಗಿ-ರು-ವ-ವ-ರಂತೆ
ಮರದ
ಮರ-ದ-ಡಿ-ಯಲ್ಲಿ
ಮರ-ಮ-ರ-ಗಳ
ಮರ-ಳ-ಬೇ-ಕೆಂ-ದಿ-ರುವ
ಮರ-ಳಲು
ಮರಳಿ
ಮರ-ಳಿದ
ಮರ-ಳಿ-ದರು
ಮರ-ಳಿ-ದಳು
ಮರ-ಳಿ-ದಾಗ
ಮರ-ಳಿ-ದಾ-ಗಿ-ನಿಂ-ದಲೂ
ಮರ-ಳಿ-ದೊ-ಡ-ನೆಯೇ
ಮರ-ಳಿದ್ದು
ಮರಳು
ಮರ-ಳು-ತ್ತಿ-ದ್ದಂ-ತೆಯೇ
ಮರ-ಳುವ
ಮರ-ಳು-ವವ
ಮರ-ವಾಗಿ
ಮರ-ವಾ-ಗಿಯೇ
ಮರ-ವಾ-ದರೆ
ಮರ-ವಿದೆ
ಮರವು
ಮರ-ವೊಂದು
ಮರಿ
ಮರಿ-ಇವು
ಮರಿ-ಗಳ
ಮರಿ-ಗ-ಳಿಗೆ
ಮರು
ಮರು-ಕಕ್ಕೆ
ಮರು-ಕ-ಗೊಂಡು
ಮರು-ಕದ
ಮರು-ಕ-ಪಡು
ಮರು-ಕ-ಳಿ-ಸಿತು
ಮರು-ಕ-ಳಿ-ಸಿ-ದಂತೆ
ಮರು-ಕ-ಳಿ-ಸಿ-ದುವು
ಮರು-ಕ-ಳಿ-ಸಿಲ್ಲ
ಮರು-ಕ-ಳಿ-ಸು-ತ್ತಿತ್ತು
ಮರು-ಕ-ಳಿ-ಸುವ
ಮರು-ಕ-ಳಿ-ಸು-ವಂತೆ
ಮರು-ಕ-ವನ್ನು
ಮರು-ಕವು
ಮರು-ಕ್ಷಣ
ಮರು-ಕ್ಷ-ಣಕ್ಕೆ
ಮರು-ಕ್ಷ-ಣ-ದಲ್ಲೇ
ಮರು-ಕ್ಷ-ಣವೇ
ಮರು-ಗ-ಲಿ-ಲ್ಲವೆ
ಮರು-ಗಿತು
ಮರು-ಗಿದ
ಮರು-ಗಿ-ದರು
ಮರು-ಗು-ತ್ತ-ದೆಯೆ
ಮರು-ಗು-ತ್ತಿತ್ತು
ಮರು-ಗುವ
ಮರು-ಗು-ವು-ದಿ-ಲ್ಲವೋ
ಮರು-ಜೀವ
ಮರು-ಣೋ-ನ್ಮು-ಖ-ಳಾ-ಗಿ-ದ್ದಳು
ಮರು-ದ-ನಿ-ಯಾಗಿ
ಮರು-ದಿನ
ಮರು-ದಿ-ನ-ದ-ವ-ರೆಗೂ
ಮರು-ದಿ-ನವೇ
ಮರು-ಪ್ರ-ಯಾ-ಣ-ವನ್ನು
ಮರು-ಭೂ-ಮಿಯ
ಮರು-ಮಾ-ತ-ನಾ-ಡದೆ
ಮರು-ಮಾತಿ
ಮರು-ಮಾ-ತಿ-ಲ್ಲದೆ
ಮರು-ಳ-ನಂ-ತೆಯೇ
ಮರು-ಳರ
ಮರು-ವ-ರ್ಷದ
ಮರು-ವ-ರ್ಷವೂ
ಮರು-ಶಿ-ಕ್ಷಣ
ಮರೆ
ಮರೆತ
ಮರೆ-ತರು
ಮರೆ-ತರೆ
ಮರೆ-ತಿರ
ಮರೆ-ತಿ-ರ-ಲಿಲ್ಲ
ಮರೆ-ತಿ-ರು-ವಿರಾ
ಮರೆ-ತಿ-ಲ್ಲ-ವೆಂ-ಬು-ದನ್ನು
ಮರೆತು
ಮರೆ-ತು-ಬಿ-ಟ್ಟರು
ಮರೆ-ತು-ಬಿ-ಟ್ಟರೆ
ಮರೆ-ತು-ಬಿ-ಟ್ಟಿ-ದ್ದಾರೆ
ಮರೆ-ತು-ಬಿ-ಟ್ಟಿ-ದ್ದಾಳೆ
ಮರೆ-ತು-ಬಿ-ಟ್ಟಿರಿ
ಮರೆ-ತು-ಬಿ-ಡ-ಬೇಕು
ಮರೆ-ತು-ಬಿಡು
ಮರೆ-ತು-ಬಿ-ಡು-ತ್ತಾರೆ
ಮರೆ-ತು-ಬಿ-ಡು-ತ್ತಿ-ದ್ದರು
ಮರೆ-ತು-ಬಿ-ಡು-ತ್ತಿದ್ದೆ
ಮರೆ-ತು-ಬಿ-ಡು-ತ್ತೇವೆ
ಮರೆ-ತು-ಹೋ-ಗಿತ್ತು
ಮರೆ-ತು-ಹೋ-ಗಿದೆ
ಮರೆ-ತೇ-ಬಿ-ಡು-ತ್ತೇವೆ
ಮರೆ-ಮಾ-ಚ-ಬಲ್ಲೆ
ಮರೆ-ಮಾ-ಚಲು
ಮರೆ-ಮಾ-ಚಿ-ಕೊಂ-ಡಳು
ಮರೆ-ಮಾ-ಚಿ-ದ್ದಾಳೆ
ಮರೆ-ಮಾ-ಚು-ತ್ತಲೂ
ಮರೆ-ಮಾ-ಡಿ-ಬಿ-ಟ್ಚಿತ್ತು
ಮರೆಯ
ಮರೆ-ಯ-ದಂತೆ
ಮರೆ-ಯ-ದಿರಿ
ಮರೆ-ಯ-ಬ-ಲ್ಲರು
ಮರೆ-ಯ-ಬ-ಹು-ದಾ-ಗಿತ್ತು
ಮರೆ-ಯ-ಬೇ-ಕೆಂದು
ಮರೆ-ಯ-ಬೇ-ಡಿ-ಗೃ-ಹ-ಸ್ಥ-ನಿಗೂ
ಮರೆ-ಯ-ಲಾ-ಗ-ದಂ-ತಹ
ಮರೆ-ಯ-ಲಾರ
ಮರೆ-ಯ-ಲಾರೆ
ಮರೆ-ಯ-ಲಾ-ರೆವು
ಮರೆ-ಯ-ಲಿಲ್ಲ
ಮರೆ-ಯಲು
ಮರೆ-ಯಾ-ಗಿ-ರ-ಲಿಲ್ಲ
ಮರೆ-ಯಾ-ಗಿ-ಸು-ತ್ತಿತ್ತು
ಮರೆ-ಯಾದ
ಮರೆ-ಯಿ-ಸಿದೆ
ಮರೆ-ಯುತ್ತ
ಮರೆ-ಯುವ
ಮರೆ-ಯು-ವಂ-ತಿ-ರ-ಲಿಲ್ಲ
ಮರೆ-ಯು-ವಂತೆ
ಮರೆ-ಯು-ವು-ದಿಲ್ಲ
ಮರೆ-ಸಿ-ಟ್ಟು-ಕೊಂಡು
ಮರ್ತ್ಯ
ಮರ್ತ್ಯ-ಲೋ-ಕ-ದಲ್ಲಿ
ಮರ್ಮ
ಮರ್ಮ-ಮ-ಹಿ-ಮೆ-ಗಳ
ಮರ್ಮ-ವನ್ನು
ಮರ್ಮ-ವನ್ನೂ
ಮರ್ಮಾ-ಘಾ-ತ-ಕ-ವಾಗಿ
ಮಲ-ಗಲು
ಮಲಗಿ
ಮಲ-ಗಿ-ಕೊಂ-ಡರು
ಮಲ-ಗಿ-ಕೊಂ-ಡಿ-ದ್ದರೂ
ಮಲ-ಗಿ-ಕೊಂ-ಡು-ಬಿ-ಟ್ಟರು
ಮಲ-ಗಿ-ಕೊಂಡೆ
ಮಲ-ಗಿ-ಕೊಂಡೇ
ಮಲ-ಗಿ-ಕೊ-ಳ್ಳಲು
ಮಲ-ಗಿತ್ತು
ಮಲ-ಗಿದ್ದ
ಮಲ-ಗಿ-ದ್ದರು
ಮಲ-ಗಿ-ದ್ದಾನೆ
ಮಲ-ಗಿ-ಬಿ-ಡು-ತ್ತೇನೆ
ಮಲ-ಗಿ-ರುವ
ಮಲ-ಗಿ-ಸ-ಲಾಗಿದೆ
ಮಲ-ಗು-ತ್ತಿ-ದ್ದುದು
ಮಲ-ಗು-ತ್ತಿ-ದ್ದೇನೆ
ಮಲ-ಗುವ
ಮಲಯ
ಮಲ್ಲಿ-ಗೆಯ
ಮಳೆ
ಮಳೆ-ಗ-ರೆ-ದರು
ಮಳೆ-ಗ-ರೆ-ಯ-ಬಲ್ಲ
ಮಳೆ-ಗ-ರೆ-ಯಲಿ
ಮಳೆ-ಗಾಲ
ಮಳೆ-ಗಾ-ಲವು
ಮಳೆ-ಗಾ-ಳಿ-ಗಳಿಂದ
ಮಳೆಯ
ಮಳೆ-ಯಲ್ಲಿ
ಮಳೆ-ಯಾಗಿ
ಮಸ-ಣದ
ಮಸಾ-ಚು-ಸೆ-ಟ್ಸ್
ಮಸಿ-ಕು-ಡಿಕೆ
ಮಸಿ-ಬ-ಳಿ-ಯ-ಲೆ-ತ್ನಿ-ಸುವ
ಮಸೀದಿ
ಮಸೀ-ದಿ-ಗಳು
ಮಸು-ಕಿ-ನ-ಲ್ಲಿಯೇ
ಮಸುಕು
ಮಸು-ಳ-ದಲ್ಲಿ
ಮಸ್ತಕ
ಮಸ್ಲಿನ್
ಮಹಂ-ತರು
ಮಹ-ಡಿ-ಗಳ
ಮಹ-ಡಿಯ
ಮಹತೋ
ಮಹ-ತ್ಕಾ-ರ್ಯ-ಗಳನ್ನು
ಮಹ-ತ್ಕಾ-ರ್ಯ-ವನ್ನು
ಮಹ-ತ್ತರ
ಮಹ-ತ್ತ-ರ-ವಾಗಿ
ಮಹ-ತ್ತ-ರ-ವಾ-ಗಿತ್ತು
ಮಹ-ತ್ತ-ರ-ವಾದ
ಮಹ-ತ್ತ-ರ-ವಾ-ದದ್ದೋ
ಮಹ-ತ್ತಾದ
ಮಹತ್ವ
ಮಹ-ತ್ವ-ಇವು
ಮಹ-ತ್ವ-ಗಳನ್ನು
ಮಹ-ತ್ವದ
ಮಹ-ತ್ವ-ದಿಂದ
ಮಹ-ತ್ವ-ದ್ದಾಗಿ
ಮಹ-ತ್ವದ್ದೂ
ಮಹ-ತ್ವ-ಪೂರ್ಣ
ಮಹ-ತ್ವ-ಪೂ-ರ್ಣ-ವಾ-ಗಿತ್ತು
ಮಹ-ತ್ವ-ಪೂ-ರ್ಣ-ವಾದ
ಮಹ-ತ್ವ-ಪೂ-ರ್ಣ-ವಾ-ದದ್ದು
ಮಹ-ತ್ವ-ಪೂ-ರ್ಣವೂ
ಮಹ-ತ್ವ-ವನ್ನು
ಮಹ-ತ್ವ-ವನ್ನೂ
ಮಹ-ತ್ವ-ವು-ಳ್ಳ-ದ್ದಾಗಿ
ಮಹ-ತ್ವ-ವೇನು
ಮಹ-ತ್ವಾ-ಕಾಂ-ಕ್ಷೆಗೂ
ಮಹ-ತ್ವಿ-ಕೆ-ಗೇ-ರಿ-ಸಲು
ಮಹ-ದಾ-ನಂದ
ಮಹ-ದಾ-ನಂ-ದ-ವಾ-ಗಿದೆ
ಮಹ-ದಾ-ಸೆ-ಯಾ-ಗಿತ್ತು
ಮಹ-ದಾ-ಸೆ-ಯಿಂದ
ಮಹ-ನೀ-ಯರ
ಮಹ-ನೀ-ಯರು
ಮಹ-ನೀ-ಯರೆ
ಮಹ-ನೀ-ಯರೇ
ಮಹ-ಮದ್
ಮಹ-ಮ್ಮ-ದ-ನಾ-ದರೆ
ಮಹ-ಮ್ಮ-ದೀ-ಯರ
ಮಹ-ಮ್ಮ-ದೀ-ಯರೂ
ಮಹ-ಮ್ಮ-ದೀ-ಯ-ರೊಂ-ದಿಗೆ
ಮಹ-ರ್ಷಿ-ಗ-ಳಂ-ತೆಯೇ
ಮಹ-ರ್ಷಿ-ಗ-ಳ-ಲ್ಲ-ನೇ-ಕರು
ಮಹ-ರ್ಷಿ-ಗಳು
ಮಹ-ರ್ಷಿ-ಗ-ಳು-ಮ-ಹಾ-ತ್ಮರು
ಮಹ-ರ್ಷಿಯು
ಮಹಾ
ಮಹಾ-ಅ-ವ-ತಾರ
ಮಹಾ-ಕಾರ್ಯ
ಮಹಾ-ಕಾ-ರ್ಯ-ಗ-ಳೆಲ್ಲ
ಮಹಾ-ಕಾ-ರ್ಯದ
ಮಹಾ-ಕಾ-ರ್ಯ-ವನ್ನು
ಮಹಾ-ಕಾ-ರ್ಯ-ವೇನಾ
ಮಹಾ-ಕಾ-ರ್ಯ-ವೊಂ-ದನ್ನು
ಮಹಾ-ಕಾಳಿ
ಮಹಾ-ಕಾಳೀ
ಮಹಾ-ಕೇಂದ್ರ
ಮಹಾ-ಗುನು
ಮಹಾ-ಗುರು
ಮಹಾ-ಗು-ರು-ಗಳು
ಮಹಾ-ಚಂಡಿ
ಮಹಾ-ಚಾ-ರ್ಯನ
ಮಹಾ-ಚೇ-ತ-ನ-ಗಳ
ಮಹಾ-ಜ-ನ-ತೆಗೂ
ಮಹಾ-ಜ-ನರೇ
ಮಹಾ-ಜೀ-ವ-ನ-ವೊಂ-ದರ
ಮಹಾ-ಜ್ಞಾ-ನ-ವನ್ನು
ಮಹಾ-ಜ್ಞಾ-ನಿ-ಗ-ಳಾದ
ಮಹಾ-ಜ್ಞಾ-ನಿ-ಗಳೇ
ಮಹಾ-ತತ್ತ್ವ
ಮಹಾ-ತ-ತ್ತ್ವ-ಗಳನ್ನು
ಮಹಾ-ತಾ-ಯಿಯ
ಮಹಾತ್ಮ
ಮಹಾ-ತ್ಮನ
ಮಹಾ-ತ್ಮ-ನಾಗಿ
ಮಹಾ-ತ್ಮ-ನಾ-ದ-ವನು
ಮಹಾ-ತ್ಮ-ನೆ-ನಿ-ಸು-ತ್ತಾನೆ
ಮಹಾ-ತ್ಮ-ನೆ-ನ್ನಿ-ಸಿ-ಕೊ-ಳ್ಳು-ತ್ತಾನೆ
ಮಹಾ-ತ್ಮ-ನೆ-ನ್ನಿ-ಸು-ತ್ತಾನೆ
ಮಹಾ-ತ್ಮ-ನೊಬ್ಬ
ಮಹಾ-ತ್ಮ-ನೊ-ಬ್ಬ-ನಿಗೆ
ಮಹಾ-ತ್ಮರ
ಮಹಾ-ತ್ಮ-ರಲ್ಲಿ
ಮಹಾ-ತ್ಮ-ರ-ಲ್ಲೆಲ್ಲ
ಮಹಾ-ತ್ಮ-ರಾದ
ಮಹಾ-ತ್ಮರು
ಮಹಾ-ತ್ಮ-ರೆಂದು
ಮಹಾ-ತ್ಮ-ರೆಂದೂ
ಮಹಾ-ತ್ಮರೇ
ಮಹಾತ್ಮಾ
ಮಹಾ-ತ್ಮೆ-ಯುಳ್ಳ
ಮಹಾ-ತ್ಯಾ-ಗಕ್ಕೆ
ಮಹಾ-ತ್ಯಾ-ಗದ
ಮಹಾ-ದ-ರ್ಶ-ದೆ-ಡೆಗೆ
ಮಹಾ-ದ-ರ್ಶ-ವನ್ನು
ಮಹಾ-ದೇವ
ಮಹಾ-ದೇ-ವನ
ಮಹಾ-ದ್ವಾ-ರದ
ಮಹಾ-ಧ-ರ್ಮ-ಗ-ಳಿಗೆ
ಮಹಾ-ಧ-ರ್ಮ-ಗಳೇ
ಮಹಾ-ಧ್ಯ-ಕ್ಷ-ರಾಗಿ
ಮಹಾ-ಧ್ಯಾ-ನ-ಭಾವ
ಮಹಾ-ನಂದ
ಮಹಾ-ನ-ಗ-ರಕ್ಕೆ
ಮಹಾ-ನಾ-ಗ-ರಿ-ಕ-ತೆ-ಗಳು
ಮಹಾ-ನಾ-ಗ-ರಿ-ಕ-ತೆಯ
ಮಹಾ-ನಾ-ಯ-ಕ-ರಾ-ಗ-ಬ-ಲ್ಲರು
ಮಹಾ-ನು-ಭಾವ
ಮಹಾ-ನು-ಭಾ-ವ-ನನ್ನು
ಮಹಾ-ನು-ಭಾ-ವ-ರನ್ನು
ಮಹಾನ್
ಮಹಾ-ಪಂ-ಡಿ-ತರು
ಮಹಾ-ಪ-ರಾ-ಧ-ಗಳನ್ನು
ಮಹಾ-ಪಾ-ಪವು
ಮಹಾ-ಪೀ-ಠ-ಗ-ಳ-ಲ್ಲೊಂದು
ಮಹಾ-ಪು-ರುಷ
ಮಹಾ-ಪು-ರು-ಷನ
ಮಹಾ-ಪು-ರು-ಷ-ನೆ-ನಿ-ಸಿ-ಕೊಂ-ಡ-ವನು
ಮಹಾ-ಪು-ರು-ಷನೇ
ಮಹಾ-ಪು-ರು-ಷರ
ಮಹಾ-ಪು-ರು-ಷರು
ಮಹಾ-ಪು-ರು-ಷ-ಸಂ-ಶ್ರಯಃ
ಮಹಾ-ಪು-ರುಷ್
ಮಹಾ-ಪೂರ
ಮಹಾ-ಪೂ-ರ-ದಲ್ಲಿ
ಮಹಾ-ಪ್ರ-ಸಾ-ದ-ವನ್ನು
ಮಹಾ-ಪ್ರ-ಸ್ಥಾ-ನಕ್ಕೆ
ಮಹಾ-ಪ್ರ-ಸ್ಥಾ-ನದ
ಮಹಾ-ಬೋಧಿ
ಮಹಾ-ಭಕ್ತ
ಮಹಾ-ಭ-ಕ್ತನ
ಮಹಾ-ಭಕ್ತೆ
ಮಹಾ-ಭಾ-ರತ
ಮಹಾ-ಭಾ-ರ-ತದ
ಮಹಾ-ಭಾ-ರ-ತ-ದೊ-ಳಕ್ಕೆ
ಮಹಾ-ಭಾ-ರ-ದಲ್ಲಿ
ಮಹಾ-ಭಾ-ವ-ನೆ-ಗಳ
ಮಹಾ-ಮಹಾ
ಮಹಾ-ಮ-ಹಿಮ
ಮಹಾ-ಮ-ಹಿ-ಮ-ರಷ್ಟೇ
ಮಹಾ-ಮ-ಹಿ-ಮ-ರಾದ
ಮಹಾ-ಮ-ಹಿ-ಮರು
ಮಹಾ-ಮಾ-ಯೆಯ
ಮಹಾ-ಮೇ-ಧಾ-ವಿಯು
ಮಹಾ-ಯುದ್ಧ
ಮಹಾ-ಯು-ದ್ಧವು
ಮಹಾ-ಯು-ದ್ಧ-ವೊಂದು
ಮಹಾ-ರ-ಭ-ಸ-ದಿಂದ
ಮಹಾ-ರಾಜ
ಮಹಾ-ರಾ-ಜನ
ಮಹಾ-ರಾ-ಜ-ನನ್ನು
ಮಹಾ-ರಾ-ಜ-ನಿಂದ
ಮಹಾ-ರಾ-ಜ-ನಿಗೆ
ಮಹಾ-ರಾ-ಜನು
ಮಹಾ-ರಾ-ಜನೂ
ಮಹಾ-ರಾ-ಜನೇ
ಮಹಾ-ರಾ-ಜ-ನೇನೋ
ಮಹಾ-ರಾ-ಜ-ನೊಂ-ದಿಗೆ
ಮಹಾ-ರಾ-ಜ-ರಿಗೂ
ಮಹಾ-ರಾ-ಜರು
ಮಹಾ-ರಾ-ಜರೂ
ಮಹಾ-ರಾ-ಜರೇ
ಮಹಾ-ರಾಜಾ
ಮಹಾ-ರಾಜ್
ಮಹಾ-ರಾಯ
ಮಹಾ-ರಾ-ಷ್ಟ್ರದ
ಮಹಾ-ವಾಣಿ
ಮಹಾ-ವೀರ
ಮಹಾ-ಶಕ್ತಿ
ಮಹಾ-ಶ-ಕ್ತಿ-ಗಳ
ಮಹಾ-ಶ-ಕ್ತಿ-ಯನ್ನು
ಮಹಾ-ಶ-ಕ್ತಿ-ಶಾ-ಲಿ-ಗಳು
ಮಹಾ-ಶ-ಯರ
ಮಹಾ-ಶ-ಯ-ರಂ-ತೆಯೇ
ಮಹಾ-ಶ-ಯ-ರನ್ನು
ಮಹಾ-ಶ-ಯ-ರಿಗೆ
ಮಹಾ-ಶ-ಯರು
ಮಹಾ-ಶಿ-ವ-ರಾ-ತ್ರಿಯ
ಮಹಾ-ಶಿ-ಶು-ವಿನ
ಮಹಾ-ಶಿ-ಷ್ಯ-ನಿಗೆ
ಮಹಾ-ಸಂ-ಘಕ್ಕೆ
ಮಹಾ-ಸಂ-ಘದ
ಮಹಾ-ಸಂ-ಘ-ದಲ್ಲಿ
ಮಹಾ-ಸಂ-ಘವು
ಮಹಾ-ಸಂ-ಚ-ಲ-ನದ
ಮಹಾ-ಸಂ-ಚ-ಲ-ನ-ವನ್ನು
ಮಹಾ-ಸಂನ್ಯಾಸಿ
ಮಹಾ-ಸ-ತ್ಯದ
ಮಹಾ-ಸ-ತ್ಯ-ವನ್ನು
ಮಹಾ-ಸ-ತ್ವವೇ
ಮಹಾ-ಸ-ಮಾ-ಧಿಯ
ಮಹಾ-ಸ-ಮಾ-ಧಿ-ಯ-ನ್ನೈ-ದಿ-ದರು
ಮಹಾ-ಸ-ಮಾ-ಧಿ-ಯಾ-ಗಿತ್ತು
ಮಹಾ-ಸಾ-ಗ-ರ-ದಿಂದ
ಮಹಾ-ಸಾ-ಮ್ರಾ-ಜ್ಯ-ವನ್ನು
ಮಹಾ-ಸ್ವಾ-ಮಿ-ಗಳ
ಮಹಾ-ಹೃ-ದ-ಯ-ಕ್ಕಾಗಿ
ಮಹಿ-ಮಾ-ವಂ-ತರು
ಮಹಿಮೆ
ಮಹಿ-ಮೆ-ಯ-ನ್ನ-ರಿ-ಯದೇ
ಮಹಿ-ಮೆ-ಯನ್ನು
ಮಹಿ-ಮೆ-ಯೇನು
ಮಹಿಳಾ
ಮಹಿಳೆ
ಮಹಿ-ಳೆಯ
ಮಹಿ-ಳೆ-ಯಂತೂ
ಮಹಿ-ಳೆ-ಯರ
ಮಹಿ-ಳೆ-ಯ-ರನ್ನು
ಮಹಿ-ಳೆ-ಯ-ರನ್ನೂ
ಮಹಿ-ಳೆ-ಯ-ರ-ನ್ನೆಲ್ಲ
ಮಹಿ-ಳೆ-ಯ-ರ-ಲ್ಲದೆ
ಮಹಿ-ಳೆ-ಯ-ರ-ಲ್ಲೊ-ಬ್ಬರೂ
ಮಹಿ-ಳೆ-ಯ-ರಾಗಿ
ಮಹಿ-ಳೆ-ಯ-ರಿಂ-ದಲೇ
ಮಹಿ-ಳೆ-ಯ-ರಿ-ಗಾಗಿ
ಮಹಿ-ಳೆ-ಯ-ರಿಗೆ
ಮಹಿ-ಳೆ-ಯರು
ಮಹಿ-ಳೆ-ಯರೇ
ಮಹಿ-ಳೆ-ಯ-ರೊಂ-ದಿಗೆ
ಮಹಿ-ಳೆ-ಯಾಗಿ
ಮಹಿ-ಳೆ-ಯಾದ
ಮಹಿ-ಳೆಯೇ
ಮಹಿ-ಳೆ-ಯೊ-ಬ್ಬರು
ಮಹಿ-ಳೆ-ಶ್ರೀ-ಮಾತೆ
ಮಹಿ-ಷಾ-ಸು-ರ-ನನ್ನು
ಮಹೀ-ಧ-ರನ
ಮಹೇಂ-ದ್ರ-ನಾಥ
ಮಹೇ-ಶ-ಚಂದ್ರ
ಮಹೋ-ದ್ದೇ-ಶದ
ಮಹೋ-ದ್ದೇ-ಶ-ದಿಂದ
ಮಹೋ-ನ್ನತ
ಮಾ
ಮಾಂಟ್
ಮಾಂತ್ರಿಕ
ಮಾಂತ್ರಿ-ಕನ
ಮಾಂಸ-ಖಂ-ಡ-ಗಳನ್ನು
ಮಾಂಸ-ಖಂ-ಡ-ಗಳು
ಮಾಂಸ-ಖಂ-ಡ-ಗ-ಳು-ಉ-ಕ್ಕಿನ
ಮಾಂಸ-ಗಳ
ಮಾಂಸ-ಗ-ಳಿಗೆ
ಮಾಜಿ
ಮಾಟ-ವಾಗಿ
ಮಾಡ
ಮಾಡ-ತೊ-ಡ-ಗಿ-ದಾಗ
ಮಾಡ-ದ-ಮಾಡ
ಮಾಡ-ದಂತೆ
ಮಾಡ-ದಿ-ದ್ದಾಗ
ಮಾಡ-ದಿ-ರಲಿ
ಮಾಡ-ದಿ-ರುವ
ಮಾಡ-ದಿ-ರು-ವ-ವರೂ
ಮಾಡ-ದಿ-ರು-ವು-ದಾ-ದರೂ
ಮಾಡ-ದಿ-ರು-ವುದು
ಮಾಡದೆ
ಮಾಡದೇ
ಮಾಡನ್ನು
ಮಾಡ-ಬಲ್ಲ
ಮಾಡ-ಬ-ಲ್ಲಂ-ತಹ
ಮಾಡ-ಬ-ಲ್ಲದು
ಮಾಡ-ಬ-ಲ್ಲನೆ
ಮಾಡ-ಬ-ಲ್ಲರು
ಮಾಡ-ಬ-ಲ್ಲ-ರೆಂದು
ಮಾಡ-ಬ-ಲ್ಲ-ವ-ನಾ-ಗಿದ್ದ
ಮಾಡ-ಬ-ಲ್ಲ-ವ-ರಾ-ದರು
ಮಾಡ-ಬ-ಲ್ಲ-ವ-ರಾ-ದರೆ
ಮಾಡ-ಬ-ಲ್ಲಿರಿ
ಮಾಡ-ಬ-ಲ್ಲಿ-ರೇನು
ಮಾಡ-ಬ-ಲ್ಲುದೋ
ಮಾಡ-ಬಲ್ಲೆ
ಮಾಡ-ಬ-ಲ್ಲೆಯಾ
ಮಾಡ-ಬ-ಲ್ಲೆವು
ಮಾಡ-ಬ-ಹು-ದಾ-ಗಿತ್ತು
ಮಾಡ-ಬ-ಹು-ದಾದ
ಮಾಡ-ಬ-ಹುದು
ಮಾಡ-ಬ-ಹುದೆ
ಮಾಡ-ಬ-ಹು-ದೆಂದು
ಮಾಡ-ಬ-ಹು-ದೆಂ-ಬು-ದನ್ನು
ಮಾಡ-ಬಾ-ರದು
ಮಾಡ-ಬಾ-ರ-ದೆಂದು
ಮಾಡ-ಬೇಕಾ
ಮಾಡ-ಬೇ-ಕಾಗಿ
ಮಾಡ-ಬೇ-ಕಾ-ಗಿತ್ತು
ಮಾಡ-ಬೇ-ಕಾ-ಗಿದೆ
ಮಾಡ-ಬೇ-ಕಾ-ಗಿ-ದ್ದ-ವರು
ಮಾಡ-ಬೇ-ಕಾ-ಗಿ-ದ್ದುದು
ಮಾಡ-ಬೇ-ಕಾ-ಗಿ-ರುವ
ಮಾಡ-ಬೇ-ಕಾ-ಗು-ತ್ತದೆ
ಮಾಡ-ಬೇ-ಕಾ-ಗು-ವು-ದಿಲ್ಲ
ಮಾಡ-ಬೇ-ಕಾದ
ಮಾಡ-ಬೇ-ಕಾ-ದರೆ
ಮಾಡ-ಬೇ-ಕಾದ್ದು
ಮಾಡ-ಬೇ-ಕಿತ್ತು
ಮಾಡ-ಬೇಕು
ಮಾಡ-ಬೇಕೆ
ಮಾಡ-ಬೇ-ಕೆಂ-ದರೆ
ಮಾಡ-ಬೇ-ಕೆಂದು
ಮಾಡ-ಬೇ-ಕೆಂ-ದು-ಕೊಂ-ಡಿ-ದ್ದಾರೋ
ಮಾಡ-ಬೇ-ಕೆಂಬ
ಮಾಡ-ಬೇ-ಕೆಂ-ಬು-ದನ್ನು
ಮಾಡ-ಬೇ-ಕೆಂ-ಬುದು
ಮಾಡ-ಬೇ-ಕೆಂ-ಬು-ದೆಲ್ಲ
ಮಾಡ-ಬೇಡ
ಮಾಡ-ಬೇ-ಡವೆ
ಮಾಡ-ಬೇಡಿ
ಮಾಡ-ಲಮ್ಮ
ಮಾಡಲಾ
ಮಾಡ-ಲಾ-ಗಿತ್ತು
ಮಾಡ-ಲಾಗಿದೆ
ಮಾಡ-ಲಾ-ಗು-ವು-ದಿಲ್ಲ
ಮಾಡ-ಲಾ-ಯಿತು
ಮಾಡ-ಲಾರ
ಮಾಡ-ಲಾ-ರಂಭಿ
ಮಾಡ-ಲಾ-ರಂ-ಭಿ-ಸಿ-ದರು
ಮಾಡ-ಲಾ-ರದು
ಮಾಡ-ಲಾ-ರಿರಿ
ಮಾಡ-ಲಾರೆ
ಮಾಡಲಿ
ಮಾಡ-ಲಿ-ಕ್ಕಿದೆ
ಮಾಡ-ಲಿದ್ದ
ಮಾಡ-ಲಿ-ದ್ದೇನೆ
ಮಾಡ-ಲಿಲ್ಲ
ಮಾಡ-ಲಿ-ಲ್ಲ-ವಲ್ಲ
ಮಾಡಲು
ಮಾಡಲೂ
ಮಾಡ-ಲೆಂದೇ
ಮಾಡ-ಲೆ-ಳ-ಸುವ
ಮಾಡಲೇ
ಮಾಡ-ಲೇ-ಬೇ-ಕಾ-ದಂಥ
ಮಾಡ-ಲೇ-ಬೇ-ಕೆಂದು
ಮಾಡ-ಲ್ಪಟ್ಟ
ಮಾಡ-ಲ್ಪ-ಡು-ತ್ತದೆ
ಮಾಡ-ಹೊ-ರ-ಟ-ದ್ದಾ-ದರೂ
ಮಾಡ-ಹೊ-ರ-ಟರೋ
ಮಾಡ-ಹೊ-ರ-ಟ-ವ-ರ-ಲ್ಲವೆ
ಮಾಡ-ಹೋಗಿ
ಮಾಡಿ
ಮಾಡಿಕೊ
ಮಾಡಿ-ಕೊಂ-ಡಂ-ತಾ-ಗ-ಲಿ-ಲ್ಲವೆ
ಮಾಡಿ-ಕೊಂ-ಡರು
ಮಾಡಿ-ಕೊಂ-ಡಲ್ಲಿ
ಮಾಡಿ-ಕೊಂ-ಡಳು
ಮಾಡಿ-ಕೊಂ-ಡ-ವರ
ಮಾಡಿ-ಕೊಂಡಿ
ಮಾಡಿ-ಕೊಂ-ಡಿ-ದ್ದಾರೆ
ಮಾಡಿ-ಕೊಂ-ಡಿ-ದ್ದೀರಿ
ಮಾಡಿ-ಕೊಂಡು
ಮಾಡಿ-ಕೊಂ-ಡು-ಹೋ-ಗು-ತ್ತಿ-ರುವ
ಮಾಡಿ-ಕೊಂಡೆ
ಮಾಡಿ-ಕೊ-ಟ್ಟರು
ಮಾಡಿ-ಕೊ-ಡ-ಬೇ-ಕಾ-ದರೆ
ಮಾಡಿ-ಕೊ-ಡ-ಬೇಕು
ಮಾಡಿ-ಕೊ-ಡ-ಬೇ-ಕೆಂದು
ಮಾಡಿ-ಕೊ-ಡಲು
ಮಾಡಿ-ಕೊ-ಡುತ್ತ
ಮಾಡಿ-ಕೊ-ಡು-ತ್ತಿ-ದ್ದರು
ಮಾಡಿ-ಕೊಳ್ಳ
ಮಾಡಿ-ಕೊ-ಳ್ಳದೆ
ಮಾಡಿ-ಕೊ-ಳ್ಳ-ಬಾ-ರ-ದೆಂಬ
ಮಾಡಿ-ಕೊ-ಳ್ಳ-ಬೇಕಾ
ಮಾಡಿ-ಕೊ-ಳ್ಳ-ಬೇ-ಕಾ-ಗಿದೆ
ಮಾಡಿ-ಕೊ-ಳ್ಳ-ಬೇ-ಕಾ-ಗು-ತ್ತದೆ
ಮಾಡಿ-ಕೊ-ಳ್ಳ-ಬೇ-ಕಾದ
ಮಾಡಿ-ಕೊ-ಳ್ಳ-ಬೇಕು
ಮಾಡಿ-ಕೊ-ಳ್ಳ-ಬೇ-ಕೆಂದು
ಮಾಡಿ-ಕೊ-ಳ್ಳ-ಲಾ-ರ-ದೆ-ಹೋ-ದಳು
ಮಾಡಿ-ಕೊ-ಳ್ಳ-ಲಾ-ರೆವು
ಮಾಡಿ-ಕೊ-ಳ್ಳಲಿ
ಮಾಡಿ-ಕೊ-ಳ್ಳಲು
ಮಾಡಿ-ಕೊ-ಳ್ಳಲೋ
ಮಾಡಿ-ಕೊಳ್ಳಿ
ಮಾಡಿ-ಕೊಳ್ಳು
ಮಾಡಿ-ಕೊ-ಳ್ಳು-ತ್ತಲೋ
ಮಾಡಿ-ಕೊ-ಳ್ಳು-ತ್ತಾರೆ
ಮಾಡಿ-ಕೊ-ಳ್ಳು-ತ್ತಿದ್ದ
ಮಾಡಿ-ಕೊ-ಳ್ಳು-ತ್ತಿ-ದ್ದರು
ಮಾಡಿ-ಕೊ-ಳ್ಳು-ತ್ತೇನೆ
ಮಾಡಿ-ಕೊ-ಳ್ಳುವ
ಮಾಡಿ-ಕೊ-ಳ್ಳು-ವಂ-ತಾ-ಗ-ಬೇಕು
ಮಾಡಿ-ಕೊ-ಳ್ಳು-ವಂತೆ
ಮಾಡಿ-ಕೊ-ಳ್ಳು-ವು-ದ-ಕ್ಕಾಗಿ
ಮಾಡಿ-ಕೊ-ಳ್ಳು-ವು-ದಕ್ಕೂ
ಮಾಡಿ-ಕೊ-ಳ್ಳು-ವು-ದ-ಕ್ಕೊಂದು
ಮಾಡಿ-ಕೊ-ಳ್ಳು-ವು-ದರ
ಮಾಡಿ-ಕೊ-ಳ್ಳು-ವುದೇ
ಮಾಡಿ-ಕೊ-ಳ್ಳು-ವು-ದೊಂ-ದನ್ನೇ
ಮಾಡಿಟ್ಟ
ಮಾಡಿ-ಟ್ಟ-ರೆಂದು
ಮಾಡಿ-ಟ್ಟಿದೆ
ಮಾಡಿ-ತಲ್ಲ
ಮಾಡಿತು
ಮಾಡಿತ್ತು
ಮಾಡಿದ
ಮಾಡಿ-ದಂ-ತಾ-ಗು-ತ್ತದೆ
ಮಾಡಿ-ದಂತೆ
ಮಾಡಿ-ದ-ರಲ್ಲ
ಮಾಡಿ-ದ-ರಾ-ದರೂ
ಮಾಡಿ-ದರು
ಮಾಡಿ-ದ-ರುಆ
ಮಾಡಿ-ದ-ರು-ಯಾ-ವು-ದ-ರಲ್ಲೂ
ಮಾಡಿ-ದ-ರು-ಸಂ-ನ್ಯಾ-ಸಿ-ಯಾ-ದ-ವನು
ಮಾಡಿ-ದರೂ
ಮಾಡಿ-ದರೆ
ಮಾಡಿ-ದ-ರೆಂ-ಬುದು
ಮಾಡಿ-ದ-ರೆಈ
ಮಾಡಿ-ದ-ರೆ-ನ್ನ-ಲಾದ
ಮಾಡಿ-ದಳು
ಮಾಡಿ-ದ-ವನೂ
ಮಾಡಿ-ದ-ವ-ರ-ಲ್ಲವೆ
ಮಾಡಿ-ದ-ವ-ರಲ್ಲಿ
ಮಾಡಿ-ದ-ವರು
ಮಾಡಿ-ದಾಗ
ಮಾಡಿ-ದಾ-ಗ-ಲೆಲ್ಲ
ಮಾಡಿ-ದಿರಿ
ಮಾಡಿ-ದು-ದನ್ನು
ಮಾಡಿ-ದು-ದಲ್ಲ
ಮಾಡಿ-ದುವು
ಮಾಡಿದೆ
ಮಾಡಿ-ದೆ-ನೆಂದೇ
ಮಾಡಿ-ದೆ-ಯಪ್ಪ
ಮಾಡಿ-ದೆ-ಯಲ್ಲ
ಮಾಡಿ-ದೆಯೆ
ಮಾಡಿ-ದೆ-ಯೇನು
ಮಾಡಿದ್ದ
ಮಾಡಿ-ದ್ದಂತೆ
ಮಾಡಿ-ದ್ದಕ್ಕೆ
ಮಾಡಿ-ದ್ದನ್ನು
ಮಾಡಿ-ದ್ದರ
ಮಾಡಿ-ದ್ದ-ರಲ್ಲಿ
ಮಾಡಿ-ದ್ದ-ರಿಂದ
ಮಾಡಿ-ದ್ದ-ರಿಂ-ದಲೇ
ಮಾಡಿ-ದ್ದರು
ಮಾಡಿ-ದ್ದರೆ
ಮಾಡಿ-ದ್ದರೇ
ಮಾಡಿ-ದ್ದರೋ
ಮಾಡಿ-ದ್ದ-ಲ್ಲದೆ
ಮಾಡಿ-ದ್ದಳು
ಮಾಡಿ-ದ್ದ-ವರು
ಮಾಡಿ-ದ್ದಾನೆ
ಮಾಡಿ-ದ್ದಾರೆ
ಮಾಡಿ-ದ್ದೀ-ಯಲ್ಲ
ಮಾಡಿ-ದ್ದೀರಿ
ಮಾಡಿದ್ದು
ಮಾಡಿ-ದ್ದುಣ್ಣೋ
ಮಾಡಿ-ದ್ದೇನೆ
ಮಾಡಿ-ದ್ದೇ-ವಲ್ಲ
ಮಾಡಿ-ದ್ದೇವೆ
ಮಾಡಿ-ದ್ದೊಂದು
ಮಾಡಿ-ಬಂದು
ಮಾಡಿ-ಬಿ-ಟ್ಟರು
ಮಾಡಿ-ಬಿ-ಟ್ಟಿತು
ಮಾಡಿ-ಬಿ-ಟ್ಟಿದೆ
ಮಾಡಿ-ಬಿ-ಟ್ಟಿ-ರು-ತ್ತದೆ
ಮಾಡಿ-ಬಿಟ್ಟೆ
ಮಾಡಿ-ಬಿ-ಡ-ಲಿಲ್ಲ
ಮಾಡಿ-ಬಿಡು
ಮಾಡಿ-ಬಿ-ಡು-ತ್ತದೆ
ಮಾಡಿ-ಬಿ-ಡು-ತ್ತ-ದೆ-ಯೆಂಬ
ಮಾಡಿ-ಬಿ-ಡು-ತ್ತಿ-ದ್ದರು
ಮಾಡಿ-ಬಿ-ಡು-ತ್ತಿ-ದ್ದಾನೆ
ಮಾಡಿ-ಮು-ಗಿ-ಸಲು
ಮಾಡಿಯಾ
ಮಾಡಿ-ಯಾ-ದರೂ
ಮಾಡಿ-ಯಾ-ಯಿತು
ಮಾಡಿ-ಯಾರೆ
ಮಾಡಿಯೂ
ಮಾಡಿಯೇ
ಮಾಡಿ-ರ-ಬೇಕು
ಮಾಡಿ-ರುವ
ಮಾಡಿ-ರು-ವಿರಾ
ಮಾಡಿಲ್ಲ
ಮಾಡಿ-ಸ-ಬೇ-ಕಾ-ಗಿತ್ತು
ಮಾಡಿ-ಸ-ಬೇ-ಕೆಂದು
ಮಾಡಿ-ಸ-ಬೇ-ಕೆಂಬ
ಮಾಡಿ-ಸ-ಲಾ-ಯಿತು
ಮಾಡಿ-ಸಲು
ಮಾಡಿಸಿ
ಮಾಡಿ-ಸಿ-ಕೊಂ-ಡರು
ಮಾಡಿ-ಸಿ-ಕೊಂಡು
ಮಾಡಿ-ಸಿ-ಕೊ-ಟ್ಟರು
ಮಾಡಿ-ಸಿ-ಕೊ-ಟ್ಟಿತು
ಮಾಡಿ-ಸಿ-ಕೊ-ಳ್ಳಲು
ಮಾಡಿ-ಸಿ-ಟ್ಟಂ-ತಹ
ಮಾಡಿ-ಸಿ-ದರು
ಮಾಡಿ-ಸಿ-ದ್ದ-ರಿಂದ
ಮಾಡಿ-ಸಿ-ದ್ದರು
ಮಾಡಿ-ಸಿ-ಬಿ-ಟ್ಟಿದ್ದೆ
ಮಾಡಿ-ಸಿ-ಬಿಡು
ಮಾಡಿ-ಸು-ತ್ತಿ-ದ್ದಾರೆ
ಮಾಡಿ-ಸು-ತ್ತಿ-ದ್ದಾಳೆ
ಮಾಡಿ-ಸು-ವು-ದ-ಕ್ಕಾಗಿ
ಮಾಡಿ-ಸು-ವು-ದಾ-ಗಿಯೂ
ಮಾಡಿ-ಸು-ವುದು
ಮಾಡಿ-ಸು-ವುದೇ
ಮಾಡಿ-ಸೋಣ
ಮಾಡಿ-ಹಾ-ಕ-ಬೇಕು
ಮಾಡು
ಮಾಡುತ್ತ
ಮಾಡು-ತ್ತದೆ
ಮಾಡು-ತ್ತಲೂ
ಮಾಡು-ತ್ತಲೇ
ಮಾಡು-ತ್ತಾನೆ
ಮಾಡು-ತ್ತಾ-ನೆಂದು
ಮಾಡು-ತ್ತಾರೆ
ಮಾಡು-ತ್ತಾ-ರೆಂದು
ಮಾಡು-ತ್ತಾ-ಳೆ-ಕಡೇ
ಮಾಡುತ್ತಿ
ಮಾಡು-ತ್ತಿತ್ತು
ಮಾಡು-ತ್ತಿ-ತ್ತು-ಅ-ವಳು
ಮಾಡು-ತ್ತಿದೆ
ಮಾಡು-ತ್ತಿ-ದೆ-ಯೆಂದು
ಮಾಡು-ತ್ತಿದ್ದ
ಮಾಡು-ತ್ತಿ-ದ್ದರು
ಮಾಡು-ತ್ತಿ-ದ್ದಳು
ಮಾಡು-ತ್ತಿ-ದ್ದ-ವರ
ಮಾಡು-ತ್ತಿ-ದ್ದಾಗ
ಮಾಡು-ತ್ತಿ-ದ್ದಾ-ನೆಯೆ
ಮಾಡು-ತ್ತಿ-ದ್ದಾರೆ
ಮಾಡು-ತ್ತಿ-ದ್ದಾ-ರೆಯೋ
ಮಾಡು-ತ್ತಿ-ದ್ದೀರಿ
ಮಾಡು-ತ್ತಿ-ದ್ದು-ದನ್ನು
ಮಾಡು-ತ್ತಿ-ದ್ದುದೇ
ಮಾಡು-ತ್ತಿ-ದ್ದುವು
ಮಾಡು-ತ್ತಿ-ದ್ದೇನೆ
ಮಾಡು-ತ್ತಿರ
ಮಾಡು-ತ್ತಿ-ರ-ಬ-ಹುದು
ಮಾಡು-ತ್ತಿ-ರ-ಲಿಲ್ಲ
ಮಾಡು-ತ್ತಿರು
ಮಾಡು-ತ್ತಿ-ರು-ಕೆ-ಟ್ಟ-ದ್ದ-ನ್ನಾ-ದರೂ
ಮಾಡು-ತ್ತಿ-ರು-ತ್ತಾರೆ
ಮಾಡು-ತ್ತಿ-ರುವ
ಮಾಡು-ತ್ತಿ-ರು-ವಂ-ತಿದೆ
ಮಾಡು-ತ್ತಿ-ರು-ವು-ದೊಂ-ದೇ-ಕೆ-ಲಸ
ಮಾಡು-ತ್ತಿ-ರು-ವೆ-ವೆಂ-ಬುದು
ಮಾಡು-ತ್ತಿಲ್ಲ
ಮಾಡು-ತ್ತಿವೆ
ಮಾಡು-ತ್ತೀರಿ
ಮಾಡು-ತ್ತೇನೆ
ಮಾಡು-ತ್ತೇ-ನೆಂದು
ಮಾಡು-ತ್ತೇವೆ
ಮಾಡುವ
ಮಾಡು-ವಂ-ತಹ
ಮಾಡು-ವಂ-ತಾ-ಗ-ಬೇಕು
ಮಾಡು-ವಂ-ತಿ-ರ-ಲಿಲ್ಲ
ಮಾಡು-ವಂತೆ
ಮಾಡು-ವನೋ
ಮಾಡು-ವಲ್ಲಿ
ಮಾಡು-ವ-ವ-ನಿ-ದ್ದೇನೆ
ಮಾಡು-ವ-ವನು
ಮಾಡು-ವ-ವ-ರಿಗೆ
ಮಾಡು-ವ-ವ-ರಿ-ದ್ದಾರೆ
ಮಾಡು-ವ-ವರು
ಮಾಡು-ವ-ವ-ರೆಗೂ
ಮಾಡು-ವ-ವರೇ
ಮಾಡು-ವ-ಷ್ಟ-ರಲ್ಲೇ
ಮಾಡು-ವಷ್ಟು
ಮಾಡು-ವಷ್ಟೇ
ಮಾಡು-ವಾಗ
ಮಾಡುವು
ಮಾಡು-ವುದ
ಮಾಡು-ವು-ದಂತೂ
ಮಾಡು-ವು-ದ-ಕ್ಕಾಗಿ
ಮಾಡು-ವು-ದ-ಕ್ಕಿಂತ
ಮಾಡು-ವು-ದಕ್ಕೂ
ಮಾಡು-ವು-ದಕ್ಕೇ
ಮಾಡು-ವುದನ್ನು
ಮಾಡು-ವು-ದನ್ನೂ
ಮಾಡು-ವು-ದಪ್ಪ
ಮಾಡು-ವು-ದರ
ಮಾಡು-ವು-ದ-ರಲ್ಲಿ
ಮಾಡು-ವು-ದ-ರಿಂದ
ಮಾಡು-ವು-ದ-ರಿಂ-ದಾಗಿ
ಮಾಡು-ವು-ದ-ರಿಂ-ದೇನು
ಮಾಡು-ವು-ದ-ರೊಂ-ದಿಗೆ
ಮಾಡು-ವು-ದಷ್ಟೇ
ಮಾಡು-ವು-ದಾಗಿ
ಮಾಡು-ವು-ದಾ-ದರೆ
ಮಾಡು-ವು-ದಿಲ್ಲ
ಮಾಡು-ವುದು
ಮಾಡು-ವು-ದು-ಇ-ವೆಲ್ಲ
ಮಾಡು-ವುದೂ
ಮಾಡು-ವು-ದೆಂ-ದರೆ
ಮಾಡು-ವು-ದೆಂ-ದ-ರೇನು
ಮಾಡು-ವುದೇ
ಮಾಡು-ವು-ದೇನು
ಮಾಡು-ವುವು
ಮಾಡು-ವೆ-ಯಾ-ದರೆ
ಮಾಡೋಣ
ಮಾಡೋ-ಣ-ವೆಂದರೆ
ಮಾಡ್
ಮಾಡ್ತಿದ್ದೀ
ಮಾತ
ಮಾತನಾ
ಮಾತ-ನಾಡ
ಮಾತ-ನಾ-ಡ-ತೊ-ಡ-ಗಿ-ದರು
ಮಾತ-ನಾ-ಡ-ತೊ-ಡ-ಗಿ-ದರೂ
ಮಾತ-ನಾ-ಡದ
ಮಾತ-ನಾ-ಡ-ದಿ-ರಲು
ಮಾತ-ನಾ-ಡದೆ
ಮಾತ-ನಾ-ಡ-ಬ-ಲ್ಲ-ರೆಂದು
ಮಾತ-ನಾ-ಡ-ಬ-ಹುದು
ಮಾತ-ನಾ-ಡ-ಬೇ-ಕಲ್ಲ
ಮಾತ-ನಾ-ಡ-ಬೇ-ಕಾ-ಗಿದೆ
ಮಾತ-ನಾ-ಡ-ಬೇ-ಕಾ-ಗು-ತ್ತಿತ್ತು
ಮಾತ-ನಾ-ಡ-ಬೇ-ಕಾದ
ಮಾತ-ನಾ-ಡ-ಬೇ-ಕೆಂದು
ಮಾತ-ನಾ-ಡ-ಬೇ-ಕೆಂಬ
ಮಾತ-ನಾ-ಡ-ಬೇಕೋ
ಮಾತ-ನಾ-ಡ-ಬೇಡಿ
ಮಾತ-ನಾ-ಡ-ಲಾ-ರಂ-ಭಿ-ಸಿ-ದರು
ಮಾತ-ನಾ-ಡ-ಲಾ-ರಂ-ಭಿ-ಸಿ-ದರೆ
ಮಾತ-ನಾ-ಡ-ಲಾ-ರಂ-ಭಿ-ಸಿ-ದ-ರೆಂ-ದರೆ
ಮಾತ-ನಾ-ಡ-ಲಿ-ದ್ದರು
ಮಾತ-ನಾ-ಡ-ಲಿ-ದ್ದೇನೆ
ಮಾತ-ನಾ-ಡ-ಲಿಲ್ಲ
ಮಾತ-ನಾ-ಡಲು
ಮಾತ-ನಾ-ಡಲೂ
ಮಾತ-ನಾ-ಡಲೇ
ಮಾತ-ನಾಡಿ
ಮಾತ-ನಾ-ಡಿ-ಕೊಂ-ಡರು
ಮಾತ-ನಾ-ಡಿ-ಕೊಂ-ಡಿ-ದ್ದರು
ಮಾತ-ನಾ-ಡಿ-ಕೊಂಡು
ಮಾತ-ನಾ-ಡಿ-ಕೊಳ್ಳು
ಮಾತ-ನಾ-ಡಿ-ಕೊ-ಳ್ಳುತ್ತ
ಮಾತ-ನಾ-ಡಿ-ಕೊ-ಳ್ಳು-ತ್ತಾ-ರಂತೆ
ಮಾತ-ನಾ-ಡಿ-ಕೊ-ಳ್ಳು-ತ್ತಿ-ದ್ದರು
ಮಾತ-ನಾ-ಡಿ-ಕೊ-ಳ್ಳು-ತ್ತಿ-ದ್ದಾ-ರೆ-ಒ-ಳ್ಳೆ-ಯ-ವರು
ಮಾತ-ನಾ-ಡಿದ
ಮಾತ-ನಾ-ಡಿ-ದ-ರ-ಲ್ಲದೆ
ಮಾತ-ನಾ-ಡಿ-ದ-ರಾ-ದರೂ
ಮಾತ-ನಾ-ಡಿ-ದರು
ಮಾತ-ನಾ-ಡಿ-ದ-ರು-ಬಂ-ಧು-ಗಳೇ
ಮಾತ-ನಾ-ಡಿ-ದರೂ
ಮಾತ-ನಾ-ಡಿ-ದರೆ
ಮಾತ-ನಾ-ಡಿ-ದ-ರೆಂದು
ಮಾತ-ನಾ-ಡಿ-ದ-ರೆಂಬ
ಮಾತ-ನಾ-ಡಿ-ದರೋ
ಮಾತ-ನಾ-ಡಿ-ದಳು
ಮಾತ-ನಾ-ಡಿ-ದ-ವರು
ಮಾತ-ನಾ-ಡಿ-ದಷ್ಟು
ಮಾತ-ನಾ-ಡಿ-ದುದು
ಮಾತ-ನಾ-ಡಿ-ದೊ-ಡ-ನೆಯೇ
ಮಾತ-ನಾ-ಡಿ-ದ್ದನ್ನು
ಮಾತ-ನಾ-ಡಿದ್ದು
ಮಾತ-ನಾ-ಡಿದ್ದೇ
ಮಾತ-ನಾ-ಡಿ-ರ-ಬ-ಹುದು
ಮಾತ-ನಾ-ಡಿ-ರ-ಬೇಕು
ಮಾತ-ನಾ-ಡಿ-ರ-ಲಿಲ್ಲ
ಮಾತ-ನಾ-ಡಿ-ಸ-ದಿ-ರಲು
ಮಾತ-ನಾ-ಡಿಸಿ
ಮಾತ-ನಾ-ಡಿ-ಸಿ-ದರು
ಮಾತ-ನಾ-ಡಿ-ಸು-ತ್ತಿ-ದ್ದರು
ಮಾತ-ನಾ-ಡಿ-ಸುವ
ಮಾತ-ನಾ-ಡಿ-ಸು-ವುದು
ಮಾತ-ನಾಡು
ಮಾತ-ನಾ-ಡುತ್ತ
ಮಾತ-ನಾ-ಡು-ತ್ತಲೇ
ಮಾತ-ನಾ-ಡು-ತ್ತಲೋ
ಮಾತ-ನಾ-ಡು-ತ್ತಾರೆ
ಮಾತ-ನಾ-ಡುತ್ತಿ
ಮಾತ-ನಾ-ಡು-ತ್ತಿದ್ದ
ಮಾತ-ನಾ-ಡು-ತ್ತಿ-ದ್ದರು
ಮಾತ-ನಾ-ಡು-ತ್ತಿ-ದ್ದರೆ
ಮಾತ-ನಾ-ಡು-ತ್ತಿ-ದ್ದಳು
ಮಾತ-ನಾ-ಡು-ತ್ತಿ-ದ್ದ-ವರು
ಮಾತ-ನಾ-ಡು-ತ್ತಿ-ದ್ದಾಗ
ಮಾತ-ನಾ-ಡು-ತ್ತಿ-ದ್ದಾರೆ
ಮಾತ-ನಾ-ಡು-ತ್ತಿ-ದ್ದೀರಿ
ಮಾತ-ನಾ-ಡು-ತ್ತಿದ್ದು
ಮಾತ-ನಾ-ಡು-ತ್ತಿದ್ದೆ
ಮಾತ-ನಾ-ಡು-ತ್ತಿ-ದ್ದೇನೆ
ಮಾತ-ನಾ-ಡು-ತ್ತಿ-ರ-ಲಿಲ್ಲ
ಮಾತ-ನಾ-ಡು-ತ್ತಿ-ರುವ
ಮಾತ-ನಾ-ಡು-ತ್ತಿ-ರು-ವಂತೆ
ಮಾತ-ನಾ-ಡು-ತ್ತಿ-ರು-ವಾಗ
ಮಾತ-ನಾ-ಡು-ತ್ತೀರಿ
ಮಾತ-ನಾ-ಡು-ತ್ತೇನೋ
ಮಾತ-ನಾ-ಡುವ
ಮಾತ-ನಾ-ಡು-ವಂ-ತಿಲ್ಲ
ಮಾತ-ನಾ-ಡು-ವಂತೆ
ಮಾತ-ನಾ-ಡು-ವ-ಷ್ಟ-ರಲ್ಲಿ
ಮಾತ-ನಾ-ಡು-ವಾಗ
ಮಾತ-ನಾ-ಡು-ವಾ-ಗ-ಲೆಲ್ಲ
ಮಾತ-ನಾ-ಡುವು
ಮಾತ-ನಾ-ಡು-ವು-ದಕ್ಕೆ
ಮಾತ-ನಾ-ಡು-ವು-ದಕ್ಕೇ
ಮಾತ-ನಾ-ಡು-ವುದನ್ನು
ಮಾತ-ನಾ-ಡು-ವು-ದ-ರಲ್ಲೇ
ಮಾತ-ನಾ-ಡು-ವುದು
ಮಾತ-ನಾ-ಡು-ವು-ದೆಂ-ದರೆ
ಮಾತ-ನಾ-ಡು-ವು-ದೆಲ್ಲ
ಮಾತ-ನಾ-ಡು-ವುದೇ
ಮಾತ-ನ್ನಾ-ಡ-ಬಾ-ರದು
ಮಾತ-ನ್ನಾಡಿ
ಮಾತ-ನ್ನಾ-ಡಿದ್ದು
ಮಾತನ್ನು
ಮಾತನ್ನೂ
ಮಾತ-ನ್ನೆಲ್ಲ
ಮಾತನ್ನೇ
ಮಾತ-ನ್ನೊಂ-ದಿಷ್ಟು
ಮಾತನ್ನೋ
ಮಾತಲ್ಲ
ಮಾತ-ಲ್ಲವೆ
ಮಾತಾ-ಜಿ-ಯ-ವರು
ಮಾತಾ-ಡಿ-ಸ-ಬೇಡಿ
ಮಾತಾ-ಡುತ್ತ
ಮಾತಿ
ಮಾತಿ-ಗಾಗಿ
ಮಾತಿ-ಗಾ-ರಂ-ಭಿ-ಸಿ-ದರು
ಮಾತಿ-ಗಾ-ರಂ-ಭಿ-ಸಿ-ದಳು
ಮಾತಿಗೆ
ಮಾತಿದೆ
ಮಾತಿನ
ಮಾತಿ-ನಂತೆ
ಮಾತಿ-ನಂ-ತೆಯೇ
ಮಾತಿ-ನಲ್ಲಿ
ಮಾತಿ-ನಲ್ಲೇ
ಮಾತಿ-ನಿಂದ
ಮಾತಿ-ನಿಂ-ದಲೇ
ಮಾತಿಲ್ಲ
ಮಾತಿ-ಲ್ಲ-ದ-ವ-ನಾ-ಗುವೆ
ಮಾತಿ-ಲ್ಲದೆ
ಮಾತು
ಮಾತು-ನಗು
ಮಾತು-ಕತೆ
ಮಾತು-ಕ-ತೆ-ಓ-ಡಾ-ಟ-ಗಳ
ಮಾತು-ಕ-ತೆ-ವ-ರ್ತ-ನೆ-ಗಳನ್ನೂ
ಮಾತು-ಕ-ತೆ-ಗಳನ್ನೆಲ್ಲ
ಮಾತು-ಕ-ತೆ-ಗಳಲ್ಲಿ
ಮಾತು-ಕ-ತೆಯ
ಮಾತು-ಕ-ತೆ-ಯನ್ನು
ಮಾತು-ಕ-ತೆ-ಯನ್ನೂ
ಮಾತು-ಕ-ತೆ-ಯಲ್ಲಿ
ಮಾತು-ಕ-ತೆ-ಯಾ-ಡಿ-ದರು
ಮಾತು-ಕ-ತೆ-ಯಾ-ಡುತ್ತ
ಮಾತು-ಕ-ತೆ-ಯಾ-ಡು-ತ್ತಿ-ದ್ದಾಗ
ಮಾತು-ಕ-ತೆ-ಯಾ-ಡು-ವಂತೆ
ಮಾತು-ಕ-ತೆ-ಯಿಂದ
ಮಾತು-ಕೊಟ್ಟಿ
ಮಾತು-ಕೊ-ಟ್ಟಿ-ದ್ದರು
ಮಾತು-ಕೊ-ಡುವ
ಮಾತು-ಗಳ
ಮಾತು-ಗ-ಳ-ನ್ನಲ್ಲ
ಮಾತು-ಗ-ಳ-ನ್ನಾ-ಡ-ಬಾ-ರದು
ಮಾತು-ಗ-ಳ-ನ್ನಾಡಿ
ಮಾತು-ಗ-ಳ-ನ್ನಾ-ಡಿ-ದರು
ಮಾತು-ಗ-ಳ-ನ್ನಾ-ಡಿ-ದ್ದರೆ
ಮಾತು-ಗ-ಳ-ನ್ನಾ-ಡಿ-ದ್ದಾ-ರೆಂ-ಬುದು
ಮಾತು-ಗ-ಳ-ನ್ನಾ-ಡುತ್ತ
ಮಾತು-ಗ-ಳ-ನ್ನಾ-ಡು-ವಂತೆ
ಮಾತು-ಗ-ಳ-ನ್ನಾ-ಡು-ವಾಗ
ಮಾತು-ಗ-ಳ-ನ್ನಾ-ದರೂ
ಮಾತು-ಗ-ಳ-ನ್ನಾ-ಲಿ-ಸಿ-ದರು
ಮಾತು-ಗಳನ್ನು
ಮಾತು-ಗಳನ್ನೂ
ಮಾತು-ಗಳನ್ನೆಲ್ಲ
ಮಾತು-ಗ-ಳಲ್ಲ
ಮಾತು-ಗಳಲ್ಲಿ
ಮಾತು-ಗ-ಳಲ್ಲೇ
ಮಾತು-ಗ-ಳಷ್ಟೇ
ಮಾತು-ಗ-ಳಾ-ಗಿ-ದ್ದುವು
ಮಾತು-ಗ-ಳಾ-ಗಿ-ರ-ಲಿಲ್ಲ
ಮಾತು-ಗಳಿಂದ
ಮಾತು-ಗ-ಳಿ-ಗಿಂತ
ಮಾತು-ಗ-ಳಿಗೆ
ಮಾತು-ಗ-ಳಿ-ದ್ದುವು
ಮಾತು-ಗಳು
ಮಾತು-ಗಳೂ
ಮಾತು-ಗ-ಳೆಂ-ದರೆ
ಮಾತು-ಗ-ಳೆಲ್ಲ
ಮಾತು-ಗ-ಳೆ-ಲ್ಲವೂ
ಮಾತು-ಗಳೇ
ಮಾತು-ಗ-ಳೊಂ-ದಿಗೆ
ಮಾತು-ಗ-ಳೊಂ-ದೊಂದೂ
ಮಾತೂ
ಮಾತೃ
ಮಾತೃ-ದೃ-ಷ್ಟಿ-ಯಿಂದ
ಮಾತೃ-ಭಾ-ವಕ್ಕೆ
ಮಾತೃ-ಭಾ-ವ-ದಿಂದ
ಮಾತೃ-ಭಾಷೆ
ಮಾತೃ-ಭೂಮಿ
ಮಾತೃ-ಭೂ-ಮಿ-ಗಾಗಿ
ಮಾತೃ-ಭೂ-ಮಿಗೆ
ಮಾತೃ-ಭೂ-ಮಿಗೇ
ಮಾತೃ-ಭೂ-ಮಿಯ
ಮಾತೃ-ಭೂ-ಮಿ-ಯೆ-ಡೆಗೆ
ಮಾತೆಂ-ದರೆ
ಮಾತೆ-ತ್ತ-ಬೇಡಿ
ಮಾತೆ-ತ್ತಿ-ದರೆ
ಮಾತೆಯ
ಮಾತೆ-ಯನ್ನು
ಮಾತೆ-ಯರ
ಮಾತೆ-ಯ-ವ-ರಿ-ಗೇ-ನೆ-ನ್ನಿ-ಸೀತು
ಮಾತೆ-ಯ-ವರು
ಮಾತೆ-ಯೆಂದು
ಮಾತೆಯೇ
ಮಾತೆಲ್ಲ
ಮಾತೇ
ಮಾತೇ-ನಲ್ಲ
ಮಾತೇನೂ
ಮಾತೊಂದು
ಮಾತ್ರ
ಮಾತ್ರಕ್ಕೆ
ಮಾತ್ರ-ದಿಂ-ದಲೇ
ಮಾತ್ರ-ವಲ್ಲ
ಮಾತ್ರ-ವ-ಲ್ಲದೆ
ಮಾತ್ರವೆ
ಮಾತ್ರವೇ
ಮಾತ್ರ-ವೇ-ಅದು
ಮಾದಕ
ಮಾದ-ರಿ-ಗಳು
ಮಾದ-ರಿಯ
ಮಾದ-ರಿ-ಯಾ-ಗಿ-ರ-ಬೇಕು
ಮಾಧುರ್ಯ
ಮಾಧು-ರ್ಯ-ಆ-ನಂ-ದ-ಗಳಲ್ಲಿ
ಮಾಧು-ರ್ಯ-ಔ-ದಾ-ರ್ಯ
ಮಾಧು-ರ್ಯವೇ
ಮಾಧ್ಯಂ-ದಿನ
ಮಾಧ್ವ
ಮಾನಕ್ಕೆ
ಮಾನ-ದಲ್ಲಿ
ಮಾನ-ದಿಂದ
ಮಾನ-ಮ-ಧು-ರೆಗೆ
ಮಾನ-ಮ-ಧು-ರೆ-ಯಿಂದ
ಮಾನ-ರಾದ
ಮಾನವ
ಮಾನ-ವ-ಕು-ಲಕ್ಕೆ
ಮಾನ-ವ-ಕು-ಲದ
ಮಾನ-ವ-ಕೋ-ಟಿಯ
ಮಾನ-ವ-ಜ-ನಾಂ-ಗದ
ಮಾನ-ವ-ಜ-ನ್ಮದ
ಮಾನ-ವ-ಜೀ-ವ-ನವು
ಮಾನ-ವತಾ
ಮಾನ-ವ-ತೆಯ
ಮಾನ-ವನ
ಮಾನ-ವ-ನನ್ನು
ಮಾನ-ವ-ನಲ್ಲಿ
ಮಾನ-ವ-ನಲ್ಲೂ
ಮಾನ-ವ-ನಿಗೆ
ಮಾನ-ವನು
ಮಾನ-ವ-ನೆಂ-ದರೆ
ಮಾನ-ವ-ಪ್ರೇ-ಮಕ್ಕೆ
ಮಾನ-ವ-ಪ್ರೇ-ಮ-ವನ್ನು
ಮಾನ-ವ-ಪ್ರೇಮಿ
ಮಾನ-ವರ
ಮಾನ-ವ-ರಲ್ಲಿ
ಮಾನ-ವ-ರಲ್ಲೂ
ಮಾನ-ವ-ರೂ-ಪ-ದಿಂದ
ಮಾನ-ವರೇ
ಮಾನ-ವ-ಸ-ಮೂ-ಹದ
ಮಾನ-ವ-ಸೇವೆ
ಮಾನ-ವ-ಸ್ವ-ಭಾ-ವ-ವನ್ನೇ
ಮಾನ-ವಿಕ
ಮಾನ-ವೀಯ
ಮಾನ-ವೀ-ಯ-ತೆಯ
ಮಾನ-ವು-ಳಿ-ಸಿ-ಕೊ-ಳ್ಳಲು
ಮಾನ-ಸಿಕ
ಮಾನ-ಸಿ-ಕ-ವಾಗಿ
ಮಾನ-ಸಿ-ಕ-ವಾ-ಗಿಯೂ
ಮಾನ-ಸಿ-ಕ-ಸ್ಥಿ-ತಿಯೂ
ಮಾನಿ-ಗಳು
ಮಾನಿಯ
ಮಾನ್ಯರೇ
ಮಾಯ-ಮಂ-ತ್ರಾದಿ
ಮಾಯ-ವಾಗಿ
ಮಾಯ-ವಾ-ಗಿ-ಬಿ-ಟ್ಟಿದೆ
ಮಾಯ-ವಾ-ಗಿ-ಬಿ-ಡು-ತ್ತವೆ
ಮಾಯ-ವಾ-ಗಿ-ಹೋ-ಗಿ-ದ್ದುವು
ಮಾಯ-ವಾ-ಗು-ತ್ತಿವೆ
ಮಾಯ-ವಾ-ಗುವ
ಮಾಯ-ವಾ-ಗು-ವುದೋ
ಮಾಯ-ವಾ-ದುವು
ಮಾಯ-ವಾ-ಯಿತು
ಮಾಯಾ
ಮಾಯಾ-ಮಂ-ತ್ರ-ಪ-ವಾ-ಡ-ಗ-ಳಂ-ತಹ
ಮಾಯಾ-ದೇವಿ
ಮಾಯಾ-ಪ್ರ-ಪಂ-ಚಕ್ಕೆ
ಮಾಯಾ-ಪ್ರ-ಪಂ-ಚದ
ಮಾಯಾ-ವತಿ
ಮಾಯಾ-ವ-ತಿಗೆ
ಮಾಯಾ-ವ-ತಿಯ
ಮಾಯಾ-ವ-ತಿ-ಯತ್ತ
ಮಾಯಾ-ವ-ತಿ-ಯನ್ನು
ಮಾಯಾ-ವ-ತಿ-ಯಲ್ಲಿ
ಮಾಯಾ-ವ-ತಿ-ಯ-ಲ್ಲಿ-ದ್ದಾಗ
ಮಾಯಾ-ವ-ತಿ-ಯ-ಲ್ಲಿನ
ಮಾಯಾ-ವ-ತಿ-ಯಿಂದ
ಮಾಯಾ-ವ-ತಿಯು
ಮಾಯಾ-ವ-ರಂ-ನಲ್ಲಿ
ಮಾಯೆ
ಮಾಯೆಯ
ಮಾಯೆಯೇ
ಮಾರಕ
ಮಾರ-ಕ-ಶ-ಕ್ತಿಯ
ಮಾರ-ನೆಯ
ಮಾರ-ಬೇ-ಕಾಗಿ
ಮಾರಾ-ಟ-ದಿಂದ
ಮಾರಾ-ಮಾ-ರಿಯೇ
ಮಾರಿ
ಮಾರಿ-ಕೊಂಡು
ಮಾರಿ-ಕೊ-ಳ್ಳುವ
ಮಾರಿ-ಕೊ-ಳ್ಳು-ವುದು
ಮಾರಿ-ಬಿ-ಡು-ವುದು
ಮಾರು-ಕ-ಟ್ಟೆಗೆ
ಮಾರು-ಕ-ಟ್ಟೆ-ಯಲ್ಲಿ
ಮಾರು-ತದ
ಮಾರುವ
ಮಾರು-ಹೋ-ದರು
ಮಾರ್ಕೆ-ಟ್ಟಿ-ನಲ್ಲಿ
ಮಾರ್ಗ
ಮಾರ್ಗಕ್ಕೆ
ಮಾರ್ಗ-ಗಳಲ್ಲಿ
ಮಾರ್ಗ-ಗಳಿಂದ
ಮಾರ್ಗ-ಗ-ಳಿಗೂ
ಮಾರ್ಗ-ದ-ರ್ಶಕ
ಮಾರ್ಗ-ದ-ರ್ಶಕಿ
ಮಾರ್ಗ-ದ-ರ್ಶ-ಕಿ-ಯಾ-ಗಿ-ರು-ತ್ತಾಳೆ
ಮಾರ್ಗ-ದ-ರ್ಶನ
ಮಾರ್ಗ-ದ-ರ್ಶ-ನ-ವ-ನ್ನ-ರಸಿ
ಮಾರ್ಗ-ದ-ರ್ಶ-ನ-ವನ್ನು
ಮಾರ್ಗ-ದಲ್ಲಿ
ಮಾರ್ಗ-ದಿಂ-ದಲೇ
ಮಾರ್ಗ-ನಮ್ಮ
ಮಾರ್ಗ-ರೆಟ್
ಮಾರ್ಗ-ರೆ-ಟ್ಟಳ
ಮಾರ್ಗ-ರೆ-ಟ್ಟ-ಳನ್ನು
ಮಾರ್ಗ-ರೆ-ಟ್ಟ-ಳಿಗೆ
ಮಾರ್ಗ-ರೆ-ಟ್ಟ-ಳಿ-ಗೊಂದು
ಮಾರ್ಗ-ರೆ-ಟ್ಟಳು
ಮಾರ್ಗ-ರೆ-ಟ್ಟಳೂ
ಮಾರ್ಗ-ರೆ-ಟ್ಟ-ಳೊಂ-ದಿಗೆ
ಮಾರ್ಗ-ವನ್ನು
ಮಾರ್ಗ-ವಾ-ಗ-ಬೇಕು
ಮಾರ್ಗ-ವಾಗಿ
ಮಾರ್ಗವು
ಮಾರ್ಗವೂ
ಮಾರ್ಗ-ವೆಂದರೆ
ಮಾರ್ಗ-ವೆಂದು
ಮಾರ್ಚಿ-ಯಲ್ಲಿ
ಮಾರ್ಚಿ-ಯಲ್ಲೇ
ಮಾರ್ಚ್
ಮಾರ್ತಾಂ-ಡಕ್ಕೆ
ಮಾರ್ತಾಂ-ಡದ
ಮಾರ್ತಾಂ-ಡ-ದಿಂದ
ಮಾರ್ದ-ನಿ-ಗೊಳ್ಳು
ಮಾರ್ದ-ನಿ-ಗೊ-ಳ್ಳು-ತ್ತಿತ್ತು
ಮಾರ್ಪಾ-ಡಾ-ಗ-ಲಿಲ್ಲ
ಮಾರ್ಮಿಕ
ಮಾರ್ವಾ-ಡಿ-ಗಳ
ಮಾರ್ವಾ-ಡಿ-ಗಳು
ಮಾರ್ವಾ-ಡಿ-ಗಳೂ
ಮಾರ್ವಾ-ಡಿ-ಗಳೇ
ಮಾರ್ಸೆ-ಲ್ಸ್
ಮಾಲಾ-ರ್ಪಣೆ
ಮಾಲಿ
ಮಾಲಿಕ
ಮಾಲಿ-ಕ-ರಾದ
ಮಾಲಿ-ಕೆ-ಯನ್ನು
ಮಾಲಿ-ಕೆ-ಯೊಂದು
ಮಾಲೀ-ಕ-ರಿಗೆ
ಮಾಲೆ
ಮಾಲೆ-ಯನ್ನು
ಮಾಲೆ-ಯಿನ್ನೂ
ಮಾಲೆಯು
ಮಾಳ
ಮಾಳಂ-ತಹ
ಮಾಳೂ
ಮಾವಿನ
ಮಾಸದ
ಮಾಸ-ಪ-ತ್ರಿ-ಕೆ-ಯಾ-ಯಿತು
ಮಾಸ-ಲಿಲ್ಲ
ಮಾಸ್ಟ-ರನ
ಮಾಸ್ಟರು
ಮಾಸ್ಟರೂ
ಮಾಸ್ಟರ್
ಮಾಹಾತ್ಮ್ಯ
ಮಾಹಾ-ತ್ಮ್ಯಕ್ಕೆ
ಮಾಹಾ-ತ್ಮ್ಯ-ವನ್ನು
ಮಾಹಿತಿ
ಮಾಹಿ-ತಿಯ
ಮಾಹಿ-ತಿ-ಯನ್ನು
ಮಾಹು-ಲಾ-ದಿಂದ
ಮಿಂಚಿತು
ಮಿಂಚಿ-ನಂತೆ
ಮಿಂಚು
ಮಿಂಚೊಂದು
ಮಿಂದ
ಮಿಂದ-ದ್ದೇನು
ಮಿಂದರೆ
ಮಿಂದು
ಮಿಕ್ಕರೆ
ಮಿಕ್ಕ-ವರ
ಮಿಕ್ಕ-ವರೆಲ್ಲ
ಮಿಕ್ಕಿ-ದ್ದೆ-ಲ್ಲವೂ
ಮಿಕ್ಕೆಲ್ಲ
ಮಿಗಿ-ಲಾಗಿ
ಮಿಗಿ-ಲಾ-ದದ್ದು
ಮಿಗಿ-ಲಾ-ದುದು
ಮಿಡಿ-ತ-ದಲ್ಲಿ
ಮಿಡಿ-ದರು
ಮಿಡಿದಿ
ಮಿಡಿ-ಯದ
ಮಿಡಿ-ಯ-ಬ-ಲ್ಲ-ವ-ರಾ-ಗಿ-ದ್ದರು
ಮಿಡಿ-ಯು-ತ್ತಿ-ದೆಯೆ
ಮಿಡಿ-ಯು-ತ್ತಿ-ರಲಿ
ಮಿಡಿ-ಯು-ವಂ-ತಹ
ಮಿಣುಕು
ಮಿಣು-ಕು-ಹು-ಳು-ವಿಗೂ
ಮಿತ-ವಾ-ಗಿತ್ತು
ಮಿತ-ವಾ-ಗಿ-ತ್ತೆಂದು
ಮಿತಿ-ಯನ್ನು
ಮಿತ್ಥಂ
ಮಿತ್ರ
ಮಿತ್ರ-ರನ್ನು
ಮಿತ್ರ-ರಾ-ಗಲಿ
ಮಿತ್ರರು
ಮಿಥ್ಯ-ತ್ವ-ಗಳನ್ನು
ಮಿಥ್ಯಾ
ಮಿಥ್ಯೆ
ಮಿದು-ಳಿಗೆ
ಮಿನುಗಿ
ಮಿನು-ಗು-ತ್ತಿ-ದ್ದುವು
ಮಿನು-ಗುವ
ಮಿನ್ನಿ
ಮಿರರ್
ಮಿಲ-ನ-ಗೊಂಡು
ಮಿಲಾನ್
ಮಿಲಿ-ಯ-ಗ-ಟ್ಟಲೆ
ಮಿಲಿ-ಯ-ದಲ್ಲಿ
ಮಿಲ್ವ-ರ್ಡ್
ಮಿಲ್ಸರು
ಮಿಲ್ಸ್
ಮಿಳಿ-ತ-ಗೊಂ-ಡಿದೆ
ಮಿಳುಹು
ಮಿಷ-ನ-ರಿ-ಗಳ
ಮಿಷ-ನ-ರಿ-ಗ-ಳಾದ
ಮಿಷ-ನ-ರಿ-ಗ-ಳಿಗೆ
ಮಿಷ-ನ-ರಿ-ಗಳು
ಮಿಷ-ನ-ರಿ-ಗ-ಳು-ಇ-ವರೆಲ್ಲ
ಮಿಷ-ನ-ರಿ-ಗಳೆ
ಮಿಷನ್
ಮಿಷ-ನ್ನನ್ನು
ಮಿಷ-ನ್ನಿನ
ಮಿಷ-ನ್ನಿ-ನಲ್ಲಿ
ಮಿಸಿ
ಮಿಸು-ಕಾ-ಡದೆ
ಮಿಸು-ಕಾ-ಡ-ಲಿಲ್ಲ
ಮಿಸ್
ಮೀಡ್
ಮೀನಾಕ್ಷಿ
ಮೀನಿನ
ಮೀಮಾಂ-ಸ-ಕರ
ಮೀಮಾಂ-ಸೆ-ಗಳ
ಮೀಮಾಂ-ಸೆ-ಯೆಂ-ದರೆ
ಮೀರಾ-ಇ-ವರೇ
ಮೀರಿ
ಮೀರಿದ
ಮೀರಿ-ದ-ಬು-ದ್ಧಿ-ಗ್ರಾಹ್ಯ
ಮೀರಿ-ದ-ವ-ರಿಗೆ
ಮೀರಿ-ದ್ದಂತೂ
ಮೀರಿದ್ದು
ಮೀರಿ-ರು-ತ್ತಾರೆ
ಮೀರಿಸು
ಮೀಸಲಾ
ಮೀಸ-ಲಾಗಿದೆ
ಮೀಸ-ಲಾದ
ಮೀಸಲು
ಮೀಸೆ
ಮುಂಗಂ-ಡಿದ್ದ
ಮುಂಗಂ-ಡಿ-ದ್ದರು
ಮುಂಗಂಡು
ಮುಂಗಡ
ಮುಂಗಾ-ಣು-ತ್ತಿ-ದ್ದೇನೆ
ಮುಂಚಿತ
ಮುಂಚಿ-ತ-ವಾಗಿ
ಮುಂಚಿ-ತ-ವಾ-ಗಿಯೇ
ಮುಂಚೆ
ಮುಂಚೆ-ಯಲ್ಲ
ಮುಂಚೆ-ಯಾ-ದರೂ
ಮುಂಚೆಯೇ
ಮುಂಜಾನೆ
ಮುಂಜಾ-ನೆಯ
ಮುಂಜಾ-ನೆ-ಯಿಂದ
ಮುಂಜಾ-ನೆ-ಯಿಂ-ದಲೇ
ಮುಂಜಾ-ವಿ-ನಲ್ಲಿ
ಮುಂತಾದ
ಮುಂತಾ-ದ-ವು-ಗಳ
ಮುಂತಾ-ದು-ವೆಲ್ಲ
ಮುಂದಕ್ಕೆ
ಮುಂದಾ
ಮುಂದಾ-ಗಿ-ದ್ದರು
ಮುಂದಾ-ಗಿಯೇ
ಮುಂದಾ-ಗಿ-ರ-ಬ-ಹುದು
ಮುಂದಾ-ಗು-ತ್ತಾ-ರೆಯೋ
ಮುಂದಾ-ಗು-ವಂತೆ
ಮುಂದಾದ
ಮುಂದಾ-ದ-ರಲ್ಲ
ಮುಂದಾ-ದರೂ
ಮುಂದಾ-ದಳು
ಮುಂದಾ-ಳ್ತನ
ಮುಂದಿ
ಮುಂದಿಟ್ಟ
ಮುಂದಿ-ಟ್ಟಂ-ತಹ
ಮುಂದಿ-ಟ್ಟದ್ದು
ಮುಂದಿ-ಟ್ಟರು
ಮುಂದಿ-ಟ್ಟಳು
ಮುಂದಿಟ್ಟು
ಮುಂದಿ-ಡ-ಬೇಕು
ಮುಂದಿ-ಡುತ್ತ
ಮುಂದಿ-ಡುವ
ಮುಂದಿದ್ದ
ಮುಂದಿನ
ಮುಂದಿ-ನ-ದೊಂದು
ಮುಂದಿ-ರುವ
ಮುಂದಿವೆ
ಮುಂದು
ಮುಂದು-ಮುಂ-ದಕ್ಕೆ
ಮುಂದು-ವರಿ
ಮುಂದು-ವ-ರಿ-ದಂ-ತೆಲ್ಲ
ಮುಂದು-ವ-ರಿ-ದರು
ಮುಂದು-ವ-ರಿ-ದರೆ
ಮುಂದು-ವ-ರಿದು
ಮುಂದು-ವ-ರಿ-ದುವು
ಮುಂದು-ವ-ರಿ-ಯ-ತೊ-ಡ-ಗಿ-ದುವು
ಮುಂದು-ವ-ರಿ-ಯ-ಬೇ-ಕಾ-ಗಿದೆ
ಮುಂದು-ವ-ರಿ-ಯ-ಬೇಕು
ಮುಂದು-ವ-ರಿ-ಯ-ಬೇ-ಕೆಂದು
ಮುಂದು-ವ-ರಿ-ಯಿತು
ಮುಂದು-ವ-ರಿ-ಯುತ್ತಿ
ಮುಂದು-ವ-ರಿ-ಯು-ತ್ತಿದೆ
ಮುಂದು-ವ-ರಿ-ಯು-ತ್ತಿದ್ದ
ಮುಂದು-ವ-ರಿ-ಯು-ತ್ತಿ-ರುವ
ಮುಂದು-ವ-ರಿ-ಯು-ತ್ತಿ-ರು-ವುದು
ಮುಂದು-ವ-ರಿ-ಯುವ
ಮುಂದು-ವ-ರಿ-ಯು-ವಂತೆ
ಮುಂದು-ವ-ರಿ-ಯು-ವು-ದೆಲ್ಲ
ಮುಂದು-ವ-ರಿ-ಯು-ವುದೇ
ಮುಂದು-ವ-ರಿ-ಯು-ವುದೋ
ಮುಂದು-ವ-ರಿ-ಸ-ದಂತೆ
ಮುಂದು-ವ-ರಿ-ಸ-ಬ-ಹುದು
ಮುಂದು-ವ-ರಿ-ಸ-ಬೇಕು
ಮುಂದು-ವ-ರಿ-ಸಲು
ಮುಂದು-ವ-ರಿಸಿ
ಮುಂದು-ವ-ರಿ-ಸಿ-ಕೊಂಡು
ಮುಂದು-ವ-ರಿ-ಸಿ-ದರು
ಮುಂದು-ವ-ರಿಸು
ಮುಂದು-ವ-ರಿ-ಸುತ್ತ
ಮುಂದು-ವ-ರಿ-ಸುತ್ತಾ
ಮುಂದು-ವ-ರಿ-ಸು-ತ್ತಾ-ರೆಂದು
ಮುಂದು-ವ-ರಿ-ಸು-ತ್ತಿ-ರು-ವುದನ್ನು
ಮುಂದು-ವ-ರಿ-ಸುವ
ಮುಂದು-ವ-ರಿ-ಸು-ವಲ್ಲಿ
ಮುಂದು-ವ-ರಿ-ಸು-ವು-ದಕ್ಕೆ
ಮುಂದು-ವ-ರಿ-ಸು-ವುದು
ಮುಂದು-ವ-ರಿ-ಸು-ವು-ದೆಂದು
ಮುಂದೂ
ಮುಂದೂಡ
ಮುಂದೆ
ಮುಂದೆಂದು
ಮುಂದೆ-ಮುಂದೆ
ಮುಂದೆಯೂ
ಮುಂದೆಯೆ
ಮುಂದೆಯೇ
ಮುಂದೆ-ಸುಂ-ದರ
ಮುಂದೇ-ನಾ-ಗು-ತ್ತದೆ
ಮುಂದೇ-ನಾ-ದರೂ
ಮುಂದೇ-ನಾ-ದೀ-ತೆಂದು
ಮುಂದೇ-ನಾ-ಯಿತು
ಮುಂದೊಮ್ಮೆ
ಮುಂಬಯಿ
ಮುಂಬ-ಯಿಗೆ
ಮುಂಬ-ಯಿಯ
ಮುಂಬ-ಯಿ-ಯತ್ತ
ಮುಂಬ-ಯಿ-ಯ-ವರೆ-ಗಿನ
ಮುಂಬರಿ
ಮುಂಬ-ರಿ-ಯ-ಬೇ-ಕೆಂದು
ಮುಂಬ-ರಿ-ಯು-ತ್ತಿದೆ
ಮುಂಬ-ರಿ-ಯು-ತ್ತಿ-ದ್ದಂತೆ
ಮುಂಬ-ರು-ತ್ತಿ-ರು-ವಾಗ
ಮುಂಬಾ-ಗಿ-ಲಿಗೆ
ಮುಂಭಾ-ಗ-ದಲ್ಲಿ
ಮುಕ್ತ
ಮುಕ್ತ-ಗೊ-ಳಿ-ಸು-ತ್ತಿತ್ತು
ಮುಕ್ತ-ನ-ನ್ನಾ-ಗಿ-ಸಲು
ಮುಕ್ತ-ನಾ-ಗಲು
ಮುಕ್ತ-ನಾ-ಗಿದ್ದೆ
ಮುಕ್ತ-ನಾ-ಗಿ-ದ್ದೇನೆ
ಮುಕ್ತ-ನಾ-ಗಿ-ರು-ತ್ತೇ-ನೆ-ಇದೇ
ಮುಕ್ತ-ನಾ-ಗು-ವ-ವ-ರೆಗೂ
ಮುಕ್ತ-ಪು-ರು-ಷ-ನಿಗೂ
ಮುಕ್ತ-ಮ-ನ-ಸ್ಸಿ-ನಿಂದ
ಮುಕ್ತ-ರ-ನ್ನಾ-ಗಿ-ಸುವ
ಮುಕ್ತ-ರಾಗಿ
ಮುಕ್ತ-ರಾ-ಗಿ-ದ್ದರೂ
ಮುಕ್ತ-ರಾ-ಗು-ತ್ತೀರಿ
ಮುಕ್ತ-ರಾ-ಗುವ
ಮುಕ್ತ-ರಾ-ಗು-ವುದೇ
ಮುಕ್ತ-ರಾದ
ಮುಕ್ತರು
ಮುಕ್ತ-ವಾ-ಗು-ವ-ವ-ರೆಗೆ
ಮುಕ್ತ-ವಾದ
ಮುಕ್ತ-ಸ್ವ-ಭಾ-ವ-ವನ್ನು
ಮುಕ್ತಾತ್ಮ
ಮುಕ್ತಾ-ತ್ಮ-ರಾದ
ಮುಕ್ತಾ-ತ್ಮ-ರಿಗೂ
ಮುಕ್ತಾ-ತ್ಮವು
ಮುಕ್ತಾ-ಯ-ಗೊಂಡ
ಮುಕ್ತಾ-ಯ-ಗೊಂ-ಡ-ಮೇಲೆ
ಮುಕ್ತಾ-ಯ-ಗೊಂ-ಡಿತು
ಮುಕ್ತಾ-ಯ-ಗೊಂಡು
ಮುಕ್ತಾ-ಯ-ಗೊ-ಳಿಸಿ
ಮುಕ್ತಾ-ಯ-ಗೊ-ಳಿ-ಸಿ-ದರು
ಮುಕ್ತಾ-ಯ-ಗೊ-ಳಿ-ಸುತ್ತ
ಮುಕ್ತಾ-ಯ-ಗೊ-ಳಿ-ಸು-ತ್ತಿ-ದ್ದಂ-ತೆಯೇ
ಮುಕ್ತಾ-ಯದ
ಮುಕ್ತಾ-ಯ-ವಾ-ಗಿತ್ತು
ಮುಕ್ತಾ-ಯ-ವಾ-ಯಿತು
ಮುಕ್ತಾ-ಯ-ಸ-ಮಾ-ರಂ-ಭದ
ಮುಕ್ತಿ
ಮುಕ್ತಿ-ಇ-ವು-ಗ-ಳಿಗೆ
ಮುಕ್ತಿ-ಗ-ಲ್ಲದೆ
ಮುಕ್ತಿ-ಗಳೇ
ಮುಕ್ತಿ-ಗಾಗಿ
ಮುಕ್ತಿಗೆ
ಮುಕ್ತಿಯ
ಮುಕ್ತಿ-ಯ-ಡ-ಗಿದೆ
ಮುಕ್ತಿ-ಯನು
ಮುಕ್ತಿ-ಯನ್ನು
ಮುಕ್ತಿ-ಯ-ನ್ನೆಲ್ಲ
ಮುಕ್ತಿ-ಯನ್ನೇ
ಮುಕ್ತಿಯೂ
ಮುಕ್ತಿ-ಯೆಂ-ದರೆ
ಮುಕ್ತಿ-ಯೆ-ಡೆಗೆ
ಮುಖ
ಮುಖಂ-ಡರ
ಮುಖಂ-ಡ-ರಾ-ಗಿಯೋ
ಮುಖಂ-ಡರು
ಮುಖಂ-ಡರೂ
ಮುಖಂ-ಡ-ರೆಂ-ದರೆ
ಮುಖಂ-ಡ-ರೊಂ-ದಿಗೆ
ಮುಖ-ಕ-ಮ-ಲ-ದಿಂದ
ಮುಖಕ್ಕೆ
ಮುಖ-ಗಳಲ್ಲಿ
ಮುಖ-ಗ-ಳಿ-ರು-ತ್ತವೆ
ಮುಖ-ಗಳು
ಮುಖ-ಗ-ಳೆಂದು
ಮುಖ-ಗೊ-ಳ್ಳಲು
ಮುಖ-ತೆ-ಯಲ್ಲಿ
ಮುಖದ
ಮುಖ-ದ-ರ್ಶನ
ಮುಖ-ದ-ರ್ಶ-ನ-ವಾ-ಗು-ತ್ತಿ-ದ್ದಂತೆ
ಮುಖ-ದಲ್ಲಿ
ಮುಖ-ದ-ಲ್ಲೆ-ದ್ದು-ಕಾಣು
ಮುಖ-ದಿಂದ
ಮುಖ-ಪು-ಟ-ದಲ್ಲಿ
ಮುಖ-ಭಂ-ಗ-ವಾ-ದಂ-ತಾ-ಯಿತು
ಮುಖ-ಭಾವ
ಮುಖ-ಭಾ-ವದ
ಮುಖ-ಭಾ-ವ-ದಿಂದ
ಮುಖ-ಭಾ-ವ-ವನ್ನು
ಮುಖ-ಮಂ-ಡಲ
ಮುಖ-ಮಂ-ಡ-ಲದ
ಮುಖ-ಮಂ-ಡ-ಲ-ದಿಂದ
ಮುಖ-ಮು-ದ್ರೆ-ಯನ್ನು
ಮುಖರ್ಜಿ
ಮುಖ-ರ್ಜಿ-ಗಳ
ಮುಖ-ರ್ಜಿಯ
ಮುಖ-ರ್ಜಿ-ಯನ್ನು
ಮುಖ-ರ್ಜಿ-ಯ-ವರ
ಮುಖ-ರ್ಜಿ-ಯ-ವರು
ಮುಖ-ವನ್ನು
ಮುಖ-ವನ್ನೂ
ಮುಖ-ವ-ನ್ನೊ-ಬ್ಬರು
ಮುಖ-ವಿದೆ
ಮುಖ-ವಿ-ರಿಸಿ
ಮುಖವೇ
ಮುಖಾಂ-ತರ
ಮುಖಾ-ಮು-ಖಿ-ಯಾಗಿ
ಮುಖ್ಯ
ಮುಖ್ಯ-ಭಾಗ
ಮುಖ್ಯ-ವಲ್ಲ
ಮುಖ್ಯ-ವಾಗಿ
ಮುಖ್ಯ-ವಾದ
ಮುಖ್ಯ-ವಾ-ದದ್ದು
ಮುಖ್ಯ-ವಾ-ದು-ದೆಂ-ದರೆ
ಮುಖ್ಯ-ವಾ-ದು-ವೆನ್ನ
ಮುಖ್ಯಸ್ಥ
ಮುಖ್ಯ-ಸ್ಥ-ರಾದ
ಮುಖ್ಯಾಂ-ಶ-ಗಳನ್ನು
ಮುಖ್ಯಾಂ-ಶ-ಗಳು
ಮುಖ್ಯಾಂ-ಶ-ವನ್ನೇ
ಮುಖ್ಯೋ-ದ್ದೇ-ಶ-ಗ-ಳಾದ
ಮುಗಿದ
ಮುಗಿ-ದಂ-ತಾ-ಯಿತು
ಮುಗಿ-ದಂ-ತೆಯೇ
ಮುಗಿ-ದಾಗ
ಮುಗಿ-ದಿತ್ತು
ಮುಗಿ-ದಿದೆ
ಮುಗಿ-ದಿ-ದೆ-ಯೆಂದು
ಮುಗಿ-ದಿಲ್ಲ
ಮುಗಿ-ದಿ-ಲ್ಲ-ದಿ-ದ್ದರೆ
ಮುಗಿದು
ಮುಗಿ-ದುವು
ಮುಗಿ-ದು-ಹೋ-ಗ-ಬೇ-ಕೆಂದು
ಮುಗಿ-ದು-ಹೋ-ದ-ದ್ದನ್ನು
ಮುಗಿ-ದು-ಹೋ-ದೀತೇ
ಮುಗಿ-ದು-ಹೋ-ಯಿತು
ಮುಗಿ-ಯದು
ಮುಗಿ-ಯ-ಲಿಲ್ಲ
ಮುಗಿ-ಯುತ್ತ
ಮುಗಿ-ಯು-ವಂ-ತೆಯೇ
ಮುಗಿ-ಯು-ವ-ವ-ರೆಗೆ
ಮುಗಿಲ
ಮುಗಿಲು
ಮುಗಿ-ಸ-ಬೇಕು
ಮುಗಿ-ಸಲು
ಮುಗಿ-ಸ-ಲೇ-ಬೇ-ಕೆಂದು
ಮುಗಿಸಿ
ಮುಗಿ-ಸಿ-ಕೊಂಡು
ಮುಗಿ-ಸಿದ
ಮುಗಿ-ಸಿ-ದರು
ಮುಗಿ-ಸಿ-ದಳು
ಮುಗಿ-ಸಿ-ರ-ಲಿಲ್ಲ
ಮುಗಿ-ಸುತ್ತಿ
ಮುಗಿ-ಸು-ತ್ತಿ-ದ್ದಂ-ತೆಯೇ
ಮುಗಿ-ಸು-ತ್ತೇನೆ
ಮುಗಿ-ಸು-ವ-ವ-ರಲ್ಲ
ಮುಗಿ-ಸು-ವ-ಷ್ಟ-ರಲ್ಲೇ
ಮುಗಿ-ಸು-ವುದು
ಮುಗು
ಮುಗು-ಳ್ನ-ಕ್ಕರು
ಮುಗು-ಳ್ನ-ಗುತ್ತ
ಮುಗು-ಳ್ನ-ಗೆ-ಯೊ-ಡನೆ
ಮುಗ್ಗ-ಟ್ಟಿಗೆ
ಮುಗ್ಗ-ರಿ-ಸುತ
ಮುಗ್ಧ
ಮುಗ್ಧ-ರ-ನ್ನಾ-ಗಿ-ಸು-ತ್ತಿತ್ತು
ಮುಗ್ಧರು
ಮುಚ್ಚ-ಲಾ-ಗದ
ಮುಚ್ಚಿ
ಮುಚ್ಚಿ-ಕೊಂ-ಡಳು
ಮುಚ್ಚಿ-ಕೊಂಡು
ಮುಚ್ಚಿ-ಕೊಂ-ಡು-ಬಿ-ಟ್ಟಿ-ದ್ದಾರೆ
ಮುಚ್ಚಿ-ಕೊಂಡೇ
ಮುಚ್ಚಿ-ಕೊ-ಳ್ಳ-ಬೇಕು
ಮುಚ್ಚಿ-ಕೊ-ಳ್ಳಲೋ
ಮುಚ್ಚಿ-ಟ್ಟು-ಕೊಂ-ಡಿ-ರ-ಲಿಲ್ಲ
ಮುಚ್ಚಿ-ಡ-ಬಾ-ರದು
ಮುಚ್ಚಿ-ಡಲು
ಮುಚ್ಚಿ-ಡುವ
ಮುಚ್ಚಿದ್ದ
ಮುಚ್ಚಿ-ಯಾ-ದರೂ
ಮುಚ್ಚಿ-ಸಿ-ದುವು
ಮುಚ್ಚಿ-ಹೋ-ಗಿತ್ತು
ಮುಚ್ಚು-ತ್ತೇನೆ
ಮುಚ್ಚು-ಮ-ರೆ-ಯಿ-ಲ್ಲದೆ
ಮುಚ್ಚು-ಮ-ರೆಯೂ
ಮುಚ್ಚು-ವಷ್ಟು
ಮುಚ್ಚೋಣ
ಮುಜು-ಗರ
ಮುಜು-ಗ-ರ-ದ್ದಾ-ಗಿತ್ತು
ಮುಜು-ಗ-ರ-ವುಂ-ಟು-ಮಾ-ಡು-ವಂ-ಥ-ದಾ-ಗಿ-ದ್ದರೂ
ಮುಟ್ಟ
ಮುಟ್ಟದ
ಮುಟ್ಟ-ದಿ-ರು-ವಂ-ತಿ-ರ-ಲಿಲ್ಲ
ಮುಟ್ಟ-ಬ-ಹುದೆ
ಮುಟ್ಟ-ಬ-ಹುದೋ
ಮುಟ್ಟ-ಬಾ-ರ-ದೆಂಬ
ಮುಟ್ಟ-ಬೇ-ಕೆಂಬ
ಮುಟ್ಟ-ಬೇಡ
ಮುಟ್ಟ-ಬೇ-ಡ-ಪ್ಪ-ಗಳು
ಮುಟ್ಟ-ಲಿಲ್ಲ
ಮುಟ್ಟಲು
ಮುಟ್ಟಿ
ಮುಟ್ಟಿತು
ಮುಟ್ಟಿತ್ತು
ಮುಟ್ಟಿದ
ಮುಟ್ಟಿ-ದಂ-ತಾ-ಗು-ವು-ದುಈ
ಮುಟ್ಟಿ-ದರು
ಮುಟ್ಟಿ-ದರೂ
ಮುಟ್ಟಿ-ದ-ವನ
ಮುಟ್ಟಿದೆ
ಮುಟ್ಟಿ-ದ್ದನ್ನು
ಮುಟ್ಟಿ-ದ್ದ-ಲ್ಲದೆ
ಮುಟ್ಟಿ-ದ್ದಳು
ಮುಟ್ಟಿದ್ದು
ಮುಟ್ಟಿ-ದ್ದೇನೆ
ಮುಟ್ಟಿ-ಬಿ-ಟ್ಟೆ-ಯಲ್ಲ
ಮುಟ್ಟಿ-ರ-ಲಿಲ್ಲ
ಮುಟ್ಟಿ-ರು-ವು-ದಾಗಿ
ಮುಟ್ಟಿಲ್ಲ
ಮುಟ್ಟಿ-ಸ-ಬಲ್ಲ
ಮುಟ್ಟಿ-ಸ-ಬೇ-ಕಾ-ಯಿತು
ಮುಟ್ಟಿ-ಸ-ಲಾ-ರದು
ಮುಟ್ಟಿ-ಸಿ-ದರು
ಮುಟ್ಟು
ಮುಟ್ಟು-ತ್ತಿತ್ತು
ಮುಟ್ಟು-ತ್ತಿ-ದ್ದಂತೆ
ಮುಟ್ಟುವ
ಮುಟ್ಟು-ವಂತೆ
ಮುಟ್ಟು-ವಂ-ಥ-ವಾ-ದ್ದ-ರಿಂದ
ಮುಟ್ಟು-ವ-ವ-ರೆಗೂ
ಮುಟ್ಟು-ವ-ವ-ರೆಗೆ
ಮುಟ್ಟು-ವು-ದಿಲ್ಲ
ಮುಟ್ಟು-ವುದು
ಮುಟ್ಟು-ವು-ದೆಲ್ಲ
ಮುಡಿ-ಪಾ-ಗಿ-ಟ್ಟ-ವ-ರಾ-ದ್ದ-ರಿಂದ
ಮುಡಿ-ಪಾ-ಗಿ-ಡಲು
ಮುಡಿ-ಪಾ-ಗಿ-ಡುವ
ಮುಡಿ-ಪಾ-ಗಿ-ಡು-ವಂ-ತಾ-ದರೆ
ಮುಡಿ-ಪಾ-ಗಿ-ಡೋಣ
ಮುಡಿ-ಪಾ-ಗಿ-ರ-ಬೇ-ಕು-ಆ-ತ್ಮನೋ
ಮುತು-ವ-ರ್ಜಿ-ಯಿಂದ
ಮುತ್ತಲ
ಮುತ್ತ-ಲಾ-ರಂ-ಭಿ-ಸಿ-ದ್ದುವು
ಮುತ್ತಲು
ಮುತ್ತಿ
ಮುತ್ತಿ-ಕೊಂ-ಡರು
ಮುತ್ತಿ-ಕೊಂ-ಡಿ-ದ್ದುವು
ಮುತ್ತಿ-ಕೊಂಡು
ಮುತ್ತಿ-ಕೊಂಡೇ
ಮುತ್ತಿ-ಕೊ-ಳ್ಳು-ತ್ತಿದ್ದ
ಮುತ್ತಿ-ಗಾಗಿ
ಮುತ್ತಿಟ್ಟು
ಮುತ್ತಿ-ದ್ದರು
ಮುತ್ತು-ತ್ತಿದ್ದ
ಮುದ-ಲಿ-ಯಾರ್
ಮುದಿ
ಮುದಿ-ಯಾಗಿ
ಮುದಿ-ಯಾದ
ಮುದುಕ
ಮುದು-ಕನ
ಮುದು-ಕ-ನಾ-ಗಿ-ದ್ದೇನೆ
ಮುದು-ಕ-ನಿಗೆ
ಮುದು-ಕರು
ಮುದುಕಿ
ಮುದು-ಕಿಗೆ
ಮುದು-ಕಿಯ
ಮುದು-ಡಿ-ಕೊ-ಳ್ಳು-ತ್ತವೆ
ಮುದು-ರಿ-ಕೊಂ-ಡರು
ಮುದ್ದಿನ
ಮುದ್ದು
ಮುದ್ದು-ಪ್ರಾ-ಣಿ-ಗ-ಳಿಗೆ
ಮುದ್ದೆ
ಮುದ್ದೆ-ಯಂತೆ
ಮುದ್ರಕ
ಮುದ್ರ-ಕರು
ಮುದ್ರಣ
ಮುದ್ರ-ಣ-ಗಳನ್ನು
ಮುದ್ರ-ಣ-ದಲ್ಲಿ
ಮುದ್ರ-ಣಾ-ಲ-ಯ-ವನ್ನೂ
ಮುದ್ರಿ-ಕೆ-ಗಳು
ಮುದ್ರಿ-ತ-ವಾಗಿ
ಮುದ್ರಿಸಿ
ಮುದ್ರಿ-ಸಿ-ಕೊ-ಳ್ಳಲೂ
ಮುದ್ರಿ-ಸಿ-ದರೆ
ಮುದ್ರಿ-ಸಿ-ದ್ದೇವೆ
ಮುದ್ರೆ
ಮುದ್ರೆ-ಯನ್ನು
ಮುದ್ರೆ-ಯ-ನ್ನೊ-ತ್ತಲು
ಮುದ್ರೆ-ಯ-ನ್ನೊ-ತ್ತಿತು
ಮುದ್ರೆ-ಯಲ್ಲಿ
ಮುದ್ರೆ-ಯೊ-ತ್ತಲು
ಮುದ್ರೆ-ಯೊ-ತ್ತಿ-ದರು
ಮುನಿ-ಯಾ-ಗಲಿ
ಮುನಿ-ಸಿ-ಕೊಂಡ
ಮುನ್ನ
ಮುನ್ನ-ಡಿ-ಯಿ-ಡ-ಬೇ-ಕಾ-ಗಿದೆ
ಮುನ್ನಡೆ
ಮುನ್ನ-ಡೆ-ದರು
ಮುನ್ನ-ಡೆದು
ಮುನ್ನ-ಡೆ-ಯ-ಬೇಕು
ಮುನ್ನ-ಡೆ-ಯಲು
ಮುನ್ನ-ಡೆ-ಯಿಂದ
ಮುನ್ನ-ಡೆ-ಯಿತು
ಮುನ್ನ-ಡೆ-ಯಿರಿ
ಮುನ್ನ-ಡೆ-ಯು-ತ್ತಲೇ
ಮುನ್ನ-ಡೆ-ಯು-ತ್ತಿದೆ
ಮುನ್ನ-ಡೆ-ಯು-ತ್ತಿ-ದ್ದೇವೆ
ಮುನ್ನ-ಡೆ-ಯು-ವಂತೆ
ಮುನ್ನ-ಡೆ-ಯು-ವುದು
ಮುನ್ನ-ಡೆಯೂ
ಮುನ್ನ-ಡೆ-ಸಲಿ
ಮುನ್ನ-ಡೆ-ಸಿ-ದಂ-ತೆ-ಯೇ-ಅಲ್ಲ
ಮುನ್ನ-ಡೆ-ಸಿದ್ದು
ಮುನ್ನ-ಡೆ-ಸುತ್ತ
ಮುನ್ನ-ಡೆ-ಸು-ತ್ತದೆ
ಮುನ್ನುಗ್ಗಿ
ಮುನ್ನು-ಗ್ಗಿದ
ಮುನ್ನು-ಗ್ಗಿ-ದರು
ಮುನ್ನುಡಿ
ಮುನ್ನು-ಡಿ-ದರು
ಮುನ್ನು-ಡಿ-ದಿದ್ದ
ಮುನ್ನು-ಡಿ-ದಿ-ದ್ದಂತೆ
ಮುನ್ನು-ಡಿ-ದಿ-ದ್ದರು
ಮುನ್ನು-ಡಿ-ದಿ-ದ್ದಳು
ಮುನ್ನು-ಡಿ-ಯಲ್ಲಿ
ಮುನ್ನೂರು
ಮುನ್ನೆ
ಮುನ್ನೆ-ಚ್ಚ-ರಿಕೆ
ಮುನ್ಷಿ
ಮುನ್ಸೀ-ಫ-ರಾದ
ಮುನ್ಸೂ-ಚನೆ
ಮುನ್ಸೂ-ಚ-ನೆ-ಗಳು
ಮುನ್ಸೂ-ಚ-ನೆ-ಯನ್ನೂ
ಮುಮು-ಕ್ಷು-ಗ-ಳಾದ
ಮುಮು-ಕ್ಷು-ಗಳು
ಮುಮು-ಕ್ಷುತ್ವಂ
ಮುರಿ
ಮುರಿ-ದದ್ದೂ
ಮುರಿ-ದಿ-ರ-ಬ-ಹು-ದೆಂಬು
ಮುರಿದು
ಮುರಿ-ದು-ಬಿ-ದ್ದಂ-ತೆಯೇ
ಮುರಿ-ದು-ಬೀ-ಳು-ವುದು
ಮುರಿ-ದು-ಹಾ-ಕಿ-ದರು
ಮುರಿ-ದು-ಹಾ-ಕಿ-ಬಿಡ
ಮುರಿ-ಮು-ರಿ-ಯುತ
ಮುರಿಯ
ಮುರಿ-ಯು-ತ್ತದೆ
ಮುರಿ-ಯು-ವಂತೆ
ಮುರಿ-ಯು-ವ-ವರು
ಮುರಿ-ಯು-ವು-ದಕ್ಕೂ
ಮುರುಕು
ಮುರ್ರಿಯ
ಮುರ್ರೀ
ಮುರ್ಶಿ-ದಾ-ಬಾ-ದಿನ
ಮುರ್ಶಿ-ದಾ-ಬಾ-ದಿ-ನಲ್ಲಿ
ಮುರ್ಶಿ-ದಾ-ಬಾದ್
ಮುರ್ಷಿ-ದಾ-ಬಾ-ದಿ-ನಲ್ಲಿ
ಮುಲ್ಲ-ರ-ರಾಗಿ
ಮುಲ್ಲ-ರರು
ಮುಲ್ಲ-ರಳ
ಮುಲ್ಲ-ರ-ಳಿಂದ
ಮುಲ್ಲ-ರ-ಳಿಗೆ
ಮುಲ್ಲ-ರಳು
ಮುಲ್ಲ-ರಳೂ
ಮುಲ್ಲರ್
ಮುಲ್ಲ-ರ್ರಂತೆ
ಮುಳುಗಿ
ಮುಳು-ಗಿತು
ಮುಳು-ಗಿದ
ಮುಳು-ಗಿ-ದರು
ಮುಳು-ಗಿ-ದೆವು
ಮುಳು-ಗಿದ್ದ
ಮುಳು-ಗಿ-ದ್ದ-ರಾ-ಯಿತು
ಮುಳು-ಗಿ-ದ್ದರು
ಮುಳು-ಗಿ-ದ್ದ-ವರು
ಮುಳು-ಗಿ-ದ್ದಾಗ
ಮುಳು-ಗಿದ್ದು
ಮುಳು-ಗಿ-ಬಿಟ್ಟಿ
ಮುಳು-ಗಿ-ಬಿ-ಡು-ತ್ತಾರೆ
ಮುಳು-ಗಿ-ರಲಿ
ಮುಳು-ಗಿ-ರು-ತ್ತಿ-ದ್ದರು
ಮುಳು-ಗಿ-ರುವ
ಮುಳು-ಗಿ-ರು-ವ-ವರು
ಮುಳು-ಗಿ-ರು-ವು-ದ-ಕ್ಕಿಂತ
ಮುಳು-ಗಿ-ರು-ವುದನ್ನು
ಮುಳು-ಗಿ-ಸ-ಬ-ಹುದು
ಮುಳು-ಗಿಸಿ
ಮುಳು-ಗಿ-ಸಿ-ಬಿ-ಡು-ತ್ತಿ-ದ್ದರು
ಮುಳು-ಗಿಸು
ಮುಳು-ಗಿ-ಹೋಗಿ
ಮುಳು-ಗಿ-ಹೋ-ಗಿದೆ
ಮುಳು-ಗಿ-ಹೋ-ಗಿ-ದ್ದರು
ಮುಳು-ಗಿ-ಹೋ-ಗಿಲ್ಲ
ಮುಳು-ಗಿ-ಹೋ-ಗು-ತ್ತಿ-ದ್ದರು
ಮುಳು-ಗಿ-ಹೋ-ಗು-ತ್ತಿ-ದ್ದಾಗ
ಮುಳು-ಗಿ-ಹೋ-ಗು-ತ್ತಿ-ದ್ದೇವೆ
ಮುಳು-ಗಿ-ಹೋ-ದರು
ಮುಳು-ಗಿ-ಹೋ-ದ-ವರು
ಮುಳು-ಗು-ತ್ತಿ-ರುವ
ಮುಳು-ಗು-ವು-ದಾ-ದರೆ
ಮುಳ್ಳನ್ನು
ಮುಳ್ಳಿನ
ಮುಳ್ಳು
ಮುಳ್ಳು-ಗಳು
ಮುಷ್ಕ-ರ-ಹೂಡಿ
ಮುಸಲ
ಮುಸ-ಲ್ಮಾನ
ಮುಸ-ಲ್ಮಾ-ನ-ನಾದ
ಮುಸ-ಲ್ಮಾ-ನರ
ಮುಸ-ಲ್ಮಾ-ನ-ರದ್ದು
ಮುಸ-ಲ್ಮಾ-ನ-ರಲ್ಲಿ
ಮುಸ-ಲ್ಮಾ-ನ-ರಾ-ಗಲಿ
ಮುಸ-ಲ್ಮಾ-ನ-ರಾ-ಗಿರ
ಮುಸ-ಲ್ಮಾ-ನರೂ
ಮುಸ-ಲ್ಮಾ-ನಳು
ಮುಸ್ಲಿಂ
ಮುಹೂರ್ತ
ಮುಹೂ-ರ್ತ-ಗ-ಳಿಂ-ದಲೂ
ಮುಹೂ-ರ್ತ-ದಲ್ಲೇ
ಮೂಕ-ರಾಗಿ
ಮೂಕ-ಳಾ-ಗಿ-ಬಿ-ಟ್ಟಳು
ಮೂಕ-ಳಾ-ದಳು
ಮೂಗ-ರಾ-ಗಿ-ದ್ದರು
ಮೂಗು
ಮೂಟೆ-ಯ-ನ್ನೆ-ತ್ತಿ-ಕೊಂಡು
ಮೂಡಿ
ಮೂಡಿತು
ಮೂಡಿದ
ಮೂಡಿ-ದರೆ
ಮೂಡಿ-ದುವು
ಮೂಡಿ-ದೊ-ಡ-ನೆಯೇ
ಮೂಡಿ-ದ್ದಿ-ರ-ಬೇಕು
ಮೂಡಿ-ನಿಂ-ತಿತು
ಮೂಡಿ-ನಿಂ-ತುವು
ಮೂಡಿ-ಬಂದ
ಮೂಡಿ-ಬಂ-ದಿತು
ಮೂಡಿ-ಬ-ರು-ತ್ತಿತ್ತು
ಮೂಡಿಲ್ಲ
ಮೂಡಿ-ಸಿತು
ಮೂಡಿ-ಸು-ತ್ತಿ-ದ್ದರು
ಮೂಡಿ-ಸು-ವಲ್ಲಿ
ಮೂಡು-ತ್ತದೆ
ಮೂಡು-ತ್ತಿತ್ತು
ಮೂಡು-ತ್ತಿದೆ
ಮೂಡು-ತ್ತಿ-ದ್ದವು
ಮೂಡು-ತ್ತಿ-ದ್ದು-ದುಂಟು
ಮೂಡ್
ಮೂಢ
ಮೂಢ-ನಂ-ಬಿಕೆ
ಮೂಢ-ನಂ-ಬಿ-ಕೆ-ಗಳ
ಮೂಢ-ನಂ-ಬಿ-ಕೆ-ಗಳನ್ನೂ
ಮೂಢ-ನಂ-ಬಿ-ಕೆ-ಗ-ಳನ್ನೇ
ಮೂಢ-ನಂ-ಬಿ-ಕೆ-ಗಳಿಂದ
ಮೂಢ-ನಂ-ಬಿ-ಕೆ-ಗ-ಳಿಗೆ
ಮೂಢ-ನಂ-ಭಿಕೆ
ಮೂತ್ರ-ಪಿಂ-ಡದ
ಮೂದ-ಲಿ-ಸಿ-ದರು
ಮೂನ್ನೂರು
ಮೂರ
ಮೂರಂ-ಶ-ಗಳನ್ನು
ಮೂರ-ನೆಯ
ಮೂರ-ನೆ-ಯ-ದಾಗಿ
ಮೂರ-ರಂದು
ಮೂರ-ರಿಂದ
ಮೂರು
ಮೂರು-ನಾಲ್ಕು
ಮೂರು-ತಾವು
ಮೂರುತಿ
ಮೂರು-ನಾಲ್ಕು
ಮೂರೂ
ಮೂರೂ-ವರೆ
ಮೂರೋ
ಮೂರ್ಖ-ತನ
ಮೂರ್ಖ-ತ-ನ-ವನ್ನು
ಮೂರ್ಖ-ತ-ನ-ವ-ಲ್ಲವೇ
ಮೂರ್ಖ-ನಾಗಿ
ಮೂರ್ಖ-ರನ್ನು
ಮೂರ್ಖರು
ಮೂರ್ತ
ಮೂರ್ತ-ರೂಪ
ಮೂರ್ತ-ರೂ-ಪ-ರಾ-ಗಿ-ದ್ದಾರೆ
ಮೂರ್ತ-ರೂ-ಪವೇ
ಮೂರ್ತ-ಸ್ವ-ರೂ-ಪವೇ
ಮೂರ್ತಿ
ಮೂರ್ತಿ-ಗಳಲ್ಲಿ
ಮೂರ್ತಿ-ಪೂ-ಜೆಯ
ಮೂರ್ತಿ-ಪೂ-ಜೆ-ಯನ್ನೂ
ಮೂರ್ತಿ-ಯಾಗಿ
ಮೂರ್ನಾಲ್ಕು
ಮೂಲ
ಮೂಲ
ಮೂಲಕ
ಮೂಲ-ಕ-ಅದೇ
ಮೂಲ-ಕವೇ
ಮೂಲತಃ
ಮೂಲ-ತತ್ತ್ವ
ಮೂಲ-ತ-ತ್ತ್ವ-ಗಳನ್ನು
ಮೂಲ-ತ-ತ್ತ್ವ-ಗಳು
ಮೂಲ-ತ-ತ್ತ್ವ-ವನ್ನೂ
ಮೂಲ-ದಲ್ಲಿ
ಮೂಲ-ಭೂತ
ಮೂಲ-ಭೂ-ತ-ವಾ-ದದ್ದು
ಮೂಲ-ವನ್ನು
ಮೂಲ-ವಾ-ಗಿತ್ತು
ಮೂಲ-ವೆಂದು
ಮೂಲ-ಸ್ಥಾನ
ಮೂಲೆ
ಮೂಲೆ-ಗಳಿಂದ
ಮೂಲೆ-ಗ-ಳಿಗೆ
ಮೂಲೆ-ಮೂ-ಲೆಗೂ
ಮೂಲೆ-ಯಲ್ಲಿ
ಮೂಲೋ
ಮೂಲೋ-ತ್ಪಾ-ಟ-ನೆಯ
ಮೂಲೋ-ದ್ದೇ-ಶ-ವಾ-ಗಿತ್ತು
ಮೂಲ್ಯ
ಮೂಳೆ
ಮೂಳೆ-ಮೂಳೆ
ಮೂವ-ತ್ತಾರು
ಮೂವತ್ತು
ಮೂವ-ತ್ತು-ಕೋಟಿ
ಮೂವ-ತ್ತೆಂ-ಟ-ನೆಯ
ಮೂವ-ತ್ತೆಂಟು
ಮೂವ-ತ್ತೊಂ-ಬತ್ತು
ಮೂವ-ರನ್ನು
ಮೂವರು
ಮೂವರೂ
ಮೂಸದ
ಮೃಗ
ಮೃಗ-ಗ-ಳಂತೆ
ಮೃಗ-ಗ-ಳಂ-ತೆಯೇ
ಮೃಗ-ಗ-ಳಿ-ಗಿಂತ
ಮೃಗ-ಗಳು
ಮೃಗ-ರಾ-ಜ-ನಂತೆ
ಮೃಣಾ-ಲಿ-ನಿ-ಬೋಸ್
ಮೃತ-ನಾದ
ಮೃತ-ಪ್ರಾ-ಯ-ರಾ-ಗಿಯೇ
ಮೃತ-ಪ್ರಾ-ಯ-ವಾ-ಗಿದೆ
ಮೃತ-ರಾ-ದ-ವ-ರಂತೆ
ಮೃತ-ರಾ-ದ-ವರು
ಮೃತ-ಳಾ-ದಂ-ತೆಯೇ
ಮೃತ್ಯು
ಮೃತ್ಯು-ಛಾ-ಯೆಯ
ಮೃತ್ಯು-ಮು-ಖ-ಗಳು
ಮೃತ್ಯು-ವ-ನ್ನ-ಪ್ಪು-ತೇನೆ
ಮೃತ್ಯು-ವನ್ನು
ಮೃತ್ಯು-ವಿಗೆ
ಮೃತ್ಯು-ವಿನ
ಮೃತ್ಯುವು
ಮೃತ್ಯು-ಸ-ರೋ-ವರ
ಮೃತ್ಯು-ಸ-ರೋ-ವ-ರ-ವೆಂಬ
ಮೃತ್ಯು-ಹೀನ
ಮೃದಂ-ಗ-ದಂ-ತಹ
ಮೃದು
ಮೃದು-ದ-ನಿ-ಯಲ್ಲಿ
ಮೃದು-ಮ-ಧುರ
ಮೃದು-ವಾಗಿ
ಮೃದು-ವಾ-ಗು-ತ್ತದೆ
ಮೃದು-ವಾದ
ಮೃಽತ್ಯುವೆ
ಮೆಕ್
ಮೆಕ್ಲಾ-ಡರು
ಮೆಕ್ಲಾ-ಡ-ರೊಂ-ದಿಗೆ
ಮೆಕ್ಲಾ-ಡಳ
ಮೆಕ್ಲಾ-ಡ-ಳ-ಲ್ಲದೆ
ಮೆಕ್ಲಾ-ಡ-ಳಿಂದ
ಮೆಕ್ಲಾ-ಡ-ಳಿಗೂ
ಮೆಕ್ಲಾ-ಡ-ಳಿಗೆ
ಮೆಕ್ಲಾ-ಡ-ಳಿ-ಗೊಂದು
ಮೆಕ್ಲಾ-ಡಳು
ಮೆಕ್ಲಾ-ಡ-ಳೊಂ-ದಿಗೆ
ಮೆಕ್ಲಾ-ಡ-ಳೊ-ಬ್ಬಳೇ
ಮೆಕ್ಲಾಡ್
ಮೆಕ್ಲಾಡ್ರ
ಮೆಕ್ಲಾ-ಡ್ರನ್ನು
ಮೆಕ್ಲಾ-ಡ್ಳಿಗೆ
ಮೆಚ್ಚದೆ
ಮೆಚ್ಚಲು
ಮೆಚ್ಚಿ-ಕೊಂಡ
ಮೆಚ್ಚಿ-ಕೊಂ-ಡ-ರಾ-ದರೂ
ಮೆಚ್ಚಿ-ಕೊಂ-ಡರು
ಮೆಚ್ಚಿ-ಕೊಂ-ಡರೂ
ಮೆಚ್ಚಿ-ಕೊಂ-ಡ-ವರು
ಮೆಚ್ಚಿ-ಕೊಂ-ಡಿದ್ದ
ಮೆಚ್ಚಿ-ಕೊಂ-ಡಿ-ದ್ದರು
ಮೆಚ್ಚಿ-ಕೊಂ-ಡಿ-ರು-ವುದು
ಮೆಚ್ಚಿ-ಕೊಂಡು
ಮೆಚ್ಚಿಗೆ
ಮೆಚ್ಚಿ-ಗೆ-ಯಾ-ಗಿದೆ
ಮೆಚ್ಚಿ-ದರು
ಮೆಚ್ಚಿ-ದ್ದರು
ಮೆಚ್ಚಿನ
ಮೆಚ್ಚುಗೆ
ಮೆಚ್ಚು-ಗೆಯ
ಮೆಚ್ಚು-ಗೆ-ಯನ್ನು
ಮೆಚ್ಚು-ಗೆ-ಯನ್ನೇ
ಮೆಚ್ಚು-ತ್ತೇನೆ
ಮೆಚ್ಚು-ವ-ವರು
ಮೆಚ್ಚು-ವು-ದಿಲ್ಲ
ಮೆಟ್ಟ-ಲಿ-ಳಿದು
ಮೆಟ್ಟಲು
ಮೆಟ್ಟ-ಲು-ಗ-ಳ-ನ್ನೇರಿ
ಮೆಟ್ಟಿ-ಲಿಗೇ
ಮೆಟ್ಟಿಲು
ಮೆಟ್ಟಿ-ಲು-ಗಳನ್ನು
ಮೆಟ್ಟಿ-ಹಾ-ಕು-ವುದು
ಮೆಣ-ಸಿನ
ಮೆತ್ತ-ನೆಯ
ಮೆದು-ಳನ್ನು
ಮೆದು-ಳಿಗೆ
ಮೆದು-ಳಿ-ನಲ್ಲಿ
ಮೆದುಳು
ಮೆದು-ಳು-ಗಳಿಂದ
ಮೆದು-ಳೊ-ಳಗೆ
ಮೆದು-ವ-ಸ್ತು-ವಾ-ದರೆ
ಮೆದು-ವಾ-ಗಿದ್ದ
ಮೆನ್ಸ್
ಮೆರ
ಮೆರ-ವ-ಣಿಗೆ
ಮೆರ-ವ-ಣಿ-ಗೆಗೆ
ಮೆರ-ವ-ಣಿ-ಗೆ-ಗೊಂದು
ಮೆರ-ವ-ಣಿ-ಗೆಯ
ಮೆರ-ವ-ಣಿ-ಗೆ-ಯಲ್ಲಿ
ಮೆರ-ವ-ಣಿ-ಗೆಯು
ಮೆರ-ವ-ಣಿ-ಗೆ-ಯೊಂದು
ಮೆರೆ-ದಾ-ಡಿ-ಸು-ತ್ತಿ-ರು-ತ್ತೇವೊ
ಮೆರೆ-ಸುತ್ತ
ಮೆರೆ-ಸು-ತ್ತೇನೆ
ಮೆಲುಕು
ಮೆಲು-ದ-ನಿ-ಯಲ್ಲಿ
ಮೆಲು-ದ-ನಿ-ಯಲ್ಲೇ
ಮೆಲೆ
ಮೆಲ್ಲಗೆ
ಮೆಲ್ಲನೆ
ಮೆಲ್ಲ-ನೆದ್ದು
ಮೆಲ್ಲ-ಮೆ-ಲ್ಲನೆ
ಮೆಷಿನ್
ಮೆಸ್ಸಿನಾ
ಮೇ
ಮೇಕೆ
ಮೇಕೆ-ಗ-ಳಿಗೆ
ಮೇಕೆ-ಮ-ರಿಯೋ
ಮೇಘ-ಗ-ರ್ಜ-ನೆ-ಗೈ-ಯು-ತ್ತಿ-ರುವ
ಮೇಜಿನ
ಮೇಡಂ
ಮೇಡಮ್
ಮೇಣ
ಮೇಧಾವಿ
ಮೇಧಾ-ವಿ-ಗ-ಳಿಗೆ
ಮೇಧಾ-ವಿ-ಗಳೂ
ಮೇಧಾ-ವಿ-ಯೆಂ-ಬು-ದನ್ನು
ಮೇಧಾ-ಶಾ-ಲಿ-ಗ-ಳಾ-ಗು-ತ್ತೀರಿ
ಮೇನೆ
ಮೇನೆ-ಯನ್ನು
ಮೇನೆ-ಯ-ವರು
ಮೇಯ್ದ-ಬಿ-ಡು-ತ್ತಿತ್ತು
ಮೇರಿ
ಮೇರಿಗೆ
ಮೇರಿ-ಗೊಂದು
ಮೇರಿಯ
ಮೇರಿ-ಯನ್ನೂ
ಮೇರು-ಪ-ರ್ವ-ತಕ್ಕೂ
ಮೇರೆಗೆ
ಮೇರೆಗೇ
ಮೇರೆ-ಯಿ-ಲ್ಲದ
ಮೇರೆಯೇ
ಮೇಲಕ್ಕೆ
ಮೇಲ-ಕ್ಕೆ-ತ್ತಿ-ದರು
ಮೇಲಣ
ಮೇಲ-ಧಿ-ಕಾರಿ
ಮೇಲ-ಧಿ-ಕಾ-ರಿ-ಗಳ
ಮೇಲ-ಧಿ-ಕಾ-ರಿ-ಗಳು
ಮೇಲ-ಧಿ-ಕಾ-ರಿ-ಯ-ನ್ನಾಗಿ
ಮೇಲಲ್ಲ
ಮೇಲಾಗಿ
ಮೇಲಾದ
ಮೇಲಾ-ದರೂ
ಮೇಲಿಂದ
ಮೇಲಿ-ಟ್ಟು-ಕೊಂಡು
ಮೇಲಿದೆ
ಮೇಲಿದ್ದ
ಮೇಲಿದ್ದು
ಮೇಲಿನ
ಮೇಲಿ-ನಿಂದ
ಮೇಲಿ-ರಿಸ
ಮೇಲಿ-ರಿ-ಸಿ-ಕೊ-ಳ್ಳಲು
ಮೇಲಿ-ರುವ
ಮೇಲಿ-ರು-ವಂ-ತಹ
ಮೇಲಿ-ರು-ವ-ವ-ರನ್ನು
ಮೇಲಿ-ರು-ವ-ವ-ರಿಗೆ
ಮೇಲು
ಮೇಲು-ಕೀಳು
ಮೇಲು-ಕೀ-ಳೆ-ನ್ನದೇ
ಮೇಲು-ಕೋಟೆ
ಮೇಲುಗೈ
ಮೇಲು-ಸಿ-ರಿನ
ಮೇಲೂ
ಮೇಲೆ
ಮೇಲೆಂದು
ಮೇಲೆ-ತ್ತ-ಬ-ಲ್ಲ-ವ-ರಾ-ಗು-ವಿರಿ
ಮೇಲೆ-ತ್ತ-ಬೇ-ಕಾ-ಗಿದೆ
ಮೇಲೆ-ತ್ತ-ಬೇಕು
ಮೇಲೆ-ತ್ತಲು
ಮೇಲೆ-ತ್ತ-ಲ್ಪ-ಟ್ಟಿದ್ದ
ಮೇಲೆತ್ತಿ
ಮೇಲೆ-ತ್ತಿ-ಕೊಂ-ಡಿವೆ
ಮೇಲೆ-ತ್ತಿ-ಕೊಂಡು
ಮೇಲೆ-ತ್ತಿದ
ಮೇಲೆ-ತ್ತಿ-ದರು
ಮೇಲೆ-ತ್ತು-ತ್ತಿ-ರು-ವಂತೆ
ಮೇಲೆ-ತ್ತುವ
ಮೇಲೆ-ತ್ತು-ವ-ವರು
ಮೇಲೆ-ದ್ದರು
ಮೇಲೆದ್ದು
ಮೇಲೆ-ನ್ನಿ-ಸು-ತ್ತದೆ
ಮೇಲೆ-ಬ್ಬಿ-ಸಲು
ಮೇಲೆ-ಬ್ಬಿಸಿ
ಮೇಲೆ-ಬ್ಬಿ-ಸಿ-ದರು
ಮೇಲೆಯೂ
ಮೇಲೆಯೇ
ಮೇಲೆಲ್ಲ
ಮೇಲೆ-ಹೀಗೆ
ಮೇಲೇರ
ಮೇಲೇರಿ
ಮೇಲೇ-ರಿತು
ಮೇಲೇ-ರಿ-ದಾಗ
ಮೇಲೇ-ರಿ-ದಿರಿ
ಮೇಲೇ-ರು-ತ್ತದೆ
ಮೇಲೇ-ಳ-ಬೇಕು
ಮೇಲೇ-ಳ-ಲಿಲ್ಲ
ಮೇಲೇಳು
ಮೇಲೇ-ಳು-ತ್ತಿ-ದ್ದಾಳೆ
ಮೇಲೇ-ಳು-ವಂ-ತಾ-ಗ-ಬೇಕು
ಮೇಲೇ-ಳು-ವ-ವ-ರೆಗೆ
ಮೇಲೊಂ-ದ-ರಂತೆ
ಮೇಲೊಂದು
ಮೇಲೊಂ-ದು-ಸುಂ-ದ-ರ-ವಾದ
ಮೇಲೊ-ಬ್ಬ-ರಂತೆ
ಮೇಲೊ-ಬ್ಬರು
ಮೇಲೊ-ರ-ಗಿಸಿ
ಮೇಲೋ
ಮೇಲ್ಖ-ರ್ಚನ್ನು
ಮೇಲ್ಜಾ-ತಿಯ
ಮೇಲ್ಜಾ-ತಿ-ಯ-ವರ
ಮೇಲ್ಜಾ-ತಿ-ಯ-ವ-ರಿಗೆ
ಮೇಲ್ನೋ-ಟಕ್ಕೆ
ಮೇಲ್ನೋ-ಟ-ಕ್ಕೇನೋ
ಮೇಲ್ಭಾ-ಗ-ದ-ಲ್ಲಿನ
ಮೇಲ್ಮೆ-ಗಾಗಿ
ಮೇಲ್ವ-ರ್ಗ-ಗಳಲ್ಲಿ
ಮೇಲ್ವ-ರ್ಗದ
ಮೇಲ್ವ-ರ್ಗ-ದ-ವರ
ಮೇಲ್ವಿ
ಮೇಲ್ವಿ-ಚಾ-ರ-ಕನೂ
ಮೇಲ್ವಿ-ಚಾ-ರ-ಕ-ರಾ-ಗಿ-ರ-ಬೇ-ಕಾ-ದುದು
ಮೇಲ್ವಿ-ಚಾ-ರ-ಕ-ರಿಗೆ
ಮೇಲ್ವಿ-ಚಾ-ರ-ಕರು
ಮೇಲ್ವಿ-ಚಾ-ರಕಿ
ಮೇಲ್ವಿ-ಚಾ-ರಣೆ
ಮೇಲ್ವಿ-ಚಾ-ರ-ಣೆಗೆ
ಮೇಲ್ವಿ-ಚಾ-ರ-ಣೆಯ
ಮೇಲ್ವಿ-ಚಾ-ರ-ಣೆ-ಯನ್ನು
ಮೇಲ್ವಿ-ಚಾ-ರ-ಣೆ-ಯಲ್ಲಿ
ಮೇಳ
ಮೇಳದ
ಮೇಳ-ದಲ್ಲಿ
ಮೇಳ-ವಿ-ಶ್ವದ
ಮೇಳವು
ಮೇಸ್ತ್ರಿ
ಮೈ
ಮೈಕ-ಟ್ಟಿ-ನ-ವ-ರಿ-ದ್ದಾರೆ
ಮೈಕ-ಟ್ಟಿ-ರ-ಬ-ಹುದೆ
ಮೈಕಟ್ಟು
ಮೈಕ-ಟ್ಟು-ನಿ-ಲುವು
ಮೈಕ-ಲೇಂ
ಮೈಕೊ-ಡಹಿ
ಮೈಗ-ಳಿಗೂ
ಮೈಗ-ಳ್ಳ-ರಂತೆ
ಮೈಗ-ಳ್ಳರೂ
ಮೈಗೂ-ಡಿ-ಸಿ-ಕೊಂ-ಡಿ-ದ್ದಾ-ಳೆಂ-ಬು-ದನ್ನು
ಮೈಗೂ-ಡಿ-ಸಿ-ಕೊಂಡು
ಮೈಗೂ-ಡಿ-ಸಿ-ಕೊ-ಳ್ಳ-ಬೇ-ಕಾ-ಗು-ತ್ತದೆ
ಮೈಗೂ-ಡಿ-ಸಿ-ಕೊ-ಳ್ಳ-ಬೇಕು
ಮೈಗೂ-ಡಿ-ಸಿ-ಕೊಳ್ಳು
ಮೈಗೂ-ಡಿ-ಸಿ-ಕೊ-ಳ್ಳುತ್ತ
ಮೈಗೆ
ಮೈಗೆಲ್ಲ
ಮೈಥು-ನಾ-ದಿ-ಗಳ
ಮೈಥು-ನಾ-ದಿ-ಗ-ಳ-ಷ್ಟ-ರಿಂ-ದಲೇ
ಮೈದ-ಡವಿ
ಮೈದ-ಳೆ-ದಂ-ತಿತ್ತು
ಮೈದಾ-ನ-ಗಳಲ್ಲಿ
ಮೈದಾ-ನ-ಗಳು
ಮೈದಾ-ನದ
ಮೈದಾ-ನ-ದಲ್ಲಿ
ಮೈದಾ-ಳಿ-ದು-ವು-ಅದೇ
ಮೈನರ್
ಮೈನ-ವಿ-ರೆ-ಬ್ಬಿ-ಸುವ
ಮೈಬಣ್ಣ
ಮೈಮ-ರೆ-ತರು
ಮೈಮ-ರೆ-ತಿತ್ತು
ಮೈಮ-ರೆ-ತಿ-ದ್ದಾರೆ
ಮೈಮ-ರೆತು
ಮೈಮ-ರೆಸು
ಮೈಮೇ-ಲಿದ್ದ
ಮೈಮೇಲೆ
ಮೈಯ-ಲ್ಲೆಲ್ಲ
ಮೈಯೆಲ್ಲ
ಮೈಲಿ
ಮೈಲಿ-ಗ-ಟ್ಟಲೆ
ಮೈಲಿ-ಗಲ್ಲು
ಮೈಲಿ-ಗಳ
ಮೈಲಿ-ಗಳನ್ನು
ಮೈಲಿಗೂ
ಮೈಲಿಗೆ
ಮೈಲಿ-ಗೆಯ
ಮೈಲಿ-ಗೆ-ಯಾ-ಗದ
ಮೈಲಿ-ಗೆ-ಯಾ-ಗಿದ್ದ
ಮೈಲಿ-ಯಷ್ಟು
ಮೈವ-ಡೆದ
ಮೈವೆ-ತ್ತಂ-ತಿ-ದ್ದರು
ಮೈಸು-ಖ-ದಲ್ಲೇ
ಮೈಸೂ-ರಿನ
ಮೈಸೂರು
ಮೊಕ-ದ್ದಮೆ
ಮೊಕ-ದ್ದ-ಮೆ-ಯನ್ನು
ಮೊಘ-ಲರ
ಮೊಟ-ಕು-ಗೊ-ಳಿ-ಸ-ಬೇ-ಕಾ-ಯಿತು
ಮೊಟ-ಕು-ಗೊ-ಳಿಸಿ
ಮೊಟ-ಕು-ಗೊ-ಳಿ-ಸಿದೆ
ಮೊಟ್ಟ
ಮೊಟ್ಟ-ಮೊ-ದಲ
ಮೊಟ್ಟ-ಮೊ-ದ-ಲ-ನೆ-ಯ-ದಾಗಿ
ಮೊಟ್ಟ-ಮೊ-ದ-ಲಿಗೆ
ಮೊಟ್ಟ-ಮೊ-ದಲು
ಮೊಟ್ಟೆಯ
ಮೊತ್ತ
ಮೊತ್ತದ
ಮೊತ್ತ-ಮೊ-ದಲ
ಮೊತ್ತ-ಮೊ-ದ-ಲ-ನೆ-ಯ-ದಾಗಿ
ಮೊತ್ತ-ಮೊ-ದ-ಲಿಗೆ
ಮೊದ
ಮೊದ-ಮೊ-ದ-ಲಿಗೆ
ಮೊದ-ಮೊ-ದಲು
ಮೊದಲ
ಮೊದ-ಲ-ದ-ರಂ-ತೆಯೇ
ಮೊದ-ಲನೆ
ಮೊದ-ಲ-ನೆಯ
ಮೊದ-ಲ-ನೆ-ಯ-ದ-ಕ್ಕಿಂತ
ಮೊದ-ಲ-ನೆ-ಯ-ದಾಗಿ
ಮೊದ-ಲ-ನೆ-ಯದು
ಮೊದ-ಲ-ನೆ-ಯ-ವ-ರೆಂ-ದರೆ
ಮೊದ-ಲರ್ಧ
ಮೊದಲಾ
ಮೊದ-ಲಾದ
ಮೊದ-ಲಾ-ದ-ವನ್ನು
ಮೊದ-ಲಾ-ದ-ವರ
ಮೊದ-ಲಾ-ದ-ವ-ರನ್ನು
ಮೊದ-ಲಾ-ದ-ವ-ರ-ಲ್ಲದೆ
ಮೊದ-ಲಾ-ದ-ವರು
ಮೊದ-ಲಾ-ದ-ವರೂ
ಮೊದ-ಲಾ-ದ-ವರೆಲ್ಲ
ಮೊದ-ಲಾ-ದ-ವು-ಗಳ
ಮೊದ-ಲಾ-ದ-ವು-ಗಳನ್ನು
ಮೊದ-ಲಾ-ದ-ವು-ಗಳಲ್ಲಿ
ಮೊದ-ಲಾ-ದ-ವು-ಗ-ಳಿಗೆ
ಮೊದ-ಲಾ-ದ-ವು-ಗ-ಳೊಂ-ದಿಗೆ
ಮೊದ-ಲಾ-ದು-ವ-ನ್ನೆಲ್ಲ
ಮೊದ-ಲಾ-ದು-ವು-ಗಳು
ಮೊದಲಿ
ಮೊದ-ಲಿಗ
ಮೊದ-ಲಿ-ಗರೇ
ಮೊದ-ಲಿ-ಗಿಂತ
ಮೊದ-ಲಿ-ಗಿಂ-ತಲೂ
ಮೊದ-ಲಿಗೆ
ಮೊದ-ಲಿನ
ಮೊದ-ಲಿ-ನಂ-ತಾ-ದರು
ಮೊದ-ಲಿ-ನಂ-ತೆಯೇ
ಮೊದ-ಲಿ-ನಿಂದ
ಮೊದ-ಲಿ-ನಿಂ-ದಲೂ
ಮೊದ-ಲಿ-ಯಾರ್
ಮೊದಲು
ಮೊದಲೇ
ಮೊನಚು
ಮೊನೆ-ಯಷ್ಟು
ಮೊನ್ನೆ
ಮೊಮ್ಮ-ಗನೂ
ಮೊಮ್ಮ-ಗಳೂ
ಮೊರೆ
ಮೊರೆ-ಯಿ-ಟ್ಟಾ-ಯಿತು
ಮೊರೆ-ಯಿ-ಡು-ತ್ತಾರೆ
ಮೊರೆ-ಯು-ವೆನು
ಮೊಳ-ಕೆ-ಯೊ-ಡೆದು
ಮೊಳ-ಗ-ಲಾ-ರಂ-ಭ-ವಾ-ಯಿತು
ಮೊಳ-ಗಿತು
ಮೊಳ-ಗಿ-ದುವು
ಮೊಳ-ಗಿ-ಸ-ಲಾ-ಯಿತು
ಮೊಳಗು
ಮೊಳ-ಗು-ತ್ತದೆ
ಮೊಳ-ಗು-ವಿಕೆ
ಮೊಳ-ಗು-ವಿ-ಕೆಯ
ಮೊಳೆತು
ಮೊಸ-ರನ್ನ
ಮೊಸ-ಳೆ-ಗ-ಳಿಗೆ
ಮೊಸ-ಳೆ-ಗಳು
ಮೊಸ-ಳೆಗೆ
ಮೊಸ-ಳೆಯ
ಮೊಸ-ಳೆ-ಯನ್ನು
ಮೊಹ-ಮ್ಮ-ದಾ-ನಂದ
ಮೊಹ-ಮ್ಮದ್
ಮೊಹ-ರನ್ನು
ಮೊಹ-ರು-ಗಳನ್ನು
ಮೋಕ್ಷ
ಮೋಕ್ಷಾರ್ಥಂ
ಮೋಜಿನ
ಮೋಜು
ಮೋಡ
ಮೋಡ-ಗಳು
ಮೋಡಿ
ಮೋಡಿ-ಗೊ-ಳ-ಗಾಗಿ
ಮೋತಿ
ಮೋತಿ-ಲಾ-ಲನ
ಮೋತಿ-ಲಾಲ್
ಮೋತೀ
ಮೋತೀ-ಲಾ-ಲರ
ಮೋತೀ-ಲಾ-ಲರು
ಮೋತೀ-ಲಾಲ್
ಮೋರೆ
ಮೋರೆ-ಯಿ-ರುವ
ಮೋಸ-ಹೋ-ಗುತ್ತಿ
ಮೋಹಕ
ಮೋಹ-ಕತೆ
ಮೋಹ-ದಿಂದ
ಮೋಹನ
ಮೋಹ-ನ-ಲಾ-ಲ-ನೊಂ-ದಿಗೆ
ಮೋಹ-ವನ್ನು
ಮೋಹಿ-ನೀ-ಮೋ-ಹನ
ಮೌಂಟ್
ಮೌಢ್ಯ-ವನ್ನು
ಮೌನ
ಮೌನಕ್ಕೆ
ಮೌನ-ತಾಳಿ
ಮೌನ-ದಲ್ಲಿ
ಮೌನ-ದಿಂ-ದೆದ್ದು
ಮೌನ-ವನ್ನು
ಮೌನ-ವಾಗಿ
ಮೌನ-ವಾ-ಗಿದ್ದು
ಮೌನ-ವಾ-ಗಿ-ದ್ದು-ದ-ರಿಂದ
ಮೌನ-ವಾ-ಗಿ-ದ್ದು-ಬಿ-ಡು-ವುದು
ಮೌನ-ವಾ-ಗಿ-ಬಿ-ಟ್ಟರು
ಮೌನ-ವಾ-ಗಿಯೇ
ಮೌನ-ವಾ-ಗಿ-ರು-ತ್ತಿ-ದ್ದರು
ಮೌನ-ವಾ-ದಾ-ಗಲೇ
ಮೌನವೇ
ಮೌನಿಯೂ
ಮೌರ್ನಾ-ಲ-ದಲ್ಲಿ
ಮೌರ್ನಾ-ಲ-ದ-ವ-ರೆಗೂ
ಮೌರ್ನಾ-ಲ-ವನ್ನು
ಮೌಲಿ-ಕ-ನೊಂ-ದಿಗೆ
ಮೌಲಿ-ಕವೂ
ಮೌಲಿಕ್
ಮೌಲ್ಯ
ಮೌಲ್ಯ-ಗಳ
ಮೌಲ್ಯ-ಗಳನ್ನು
ಮೌಲ್ಯ-ಗಳಲ್ಲಿ
ಮೌಲ್ಯ-ಗ-ಳಿಗೆ
ಮೌಲ್ಯ-ಗಳು
ಮೌಲ್ಯದ
ಮ್ಯಾಕ್ಸಿಂ
ಮ್ಯಾಕ್ಸಿಮ್
ಮ್ಯಾಕ್ಸ್
ಮ್ಯಾನ-ರನ್ನು
ಮ್ಯಾನ-ರಿಂದ
ಮ್ಯಾನರ್
ಮ್ಯಾನೇ-ಜರ್
ಮ್ಯಾರ್ಗಟ್
ಮ್ಯೂಸಿಯಂ
ಮ್ಯೂಸಿ-ಯ-ಮ್ಮನ್ನೂ
ಮ್ಲೇಚ್ಛ
ಮ್ಲೇಚ್ಛ-ನಾಗಿ
ಮ್ಲೇಚ್ಛನೇ
ಮ್ಲೇಚ್ಛರ
ಮ್ಲೇಚ್ಛ-ರಾದ
ಮ್ಲೇಚ್ಛ-ರೆಂದು
ಮ್ಲೇಚ್ಛ-ರೊಂ-ದಿಗೆ
ಮ್ಲೇಚ್ಛ-ಳೆಂದು
ಮ್ಲೇಚ್ಛಿತಂ
ಯಂತಹ
ಯಂತಿ
ಯಂತೂ
ಯಂತೆ
ಯಂತ್ರ
ಯಂತ್ರ-ಕೌ-ಶ-ಲ-ದಲ್ಲೂ
ಯಂತ್ರ-ಗಳನ್ನು
ಯಂತ್ರ-ಗ-ಳ-ಲ್ಲದೆ
ಯಂತ್ರ-ಗ-ಳಿ-ಲ್ಲ-ದಿದ್ದ
ಯಂತ್ರದ
ಯಂತ್ರ-ವನ್ನು
ಯಂತ್ರವು
ಯಂತ್ರ-ವೊಂ-ದನ್ನು
ಯಂತ್ರಾ-ದಿ-ಗಳನ್ನು
ಯಂಥದು
ಯಂದು
ಯಃಕ-ಶ್ಚಿ-ತ್ತು-ಗ-ಳಂತೆ
ಯಃಕಿ-ಶ್ಚಿತ್
ಯಕೃತ್ತು
ಯಜ
ಯಜ-ಮಾನ
ಯಜ-ಮಾ-ನನ
ಯಜ-ಮಾ-ನ-ನಿಗೆ
ಯಜ-ಮಾನಿ
ಯಜ-ಮಾ-ನಿ-ಕೆಯ
ಯಜ್ಞ
ಯಜ್ಞ-ಗಳನ್ನು
ಯಜ್ಞ-ಗ-ಳೆ-ಲ್ಲ-ವನ್ನೂ
ಯಜ್ಞ-ದ-ಕ್ಷಿ-ಣೆ-ಯನ್ನು
ಯಜ್ಞೇ-ಶ್ವ-ರ-ನಿಗೆ
ಯಜ್ಞೋ-ಪ-ವೀತ
ಯಜ್ಞೋ-ಪ-ವೀ-ತ-ಗಳನ್ನು
ಯಜ್ಞೋ-ಪ-ವೀ-ತ-ಗಳು
ಯಜ್ಞೋ-ಪ-ವೀ-ತ-ಧಾ-ರಣೆ
ಯಜ್ಞೋ-ಪ-ವೀ-ತ-ವನ್ನು
ಯಣದ
ಯತ್ಕಿಂ-ಚಿತ್
ಯತ್ನ
ಯತ್ನ-ದ-ಲ್ಲಿ-ದ್ದಾರೆ
ಯತ್ನಿ-ಸ-ಲಾಗಿದೆ
ಯತ್ನಿ-ಸಲು
ಯತ್ನಿ-ಸಿ-ದ್ದೇನೆ
ಯತ್ನಿಸು
ಯತ್ನಿ-ಸುತ್ತ
ಯತ್ನಿ-ಸು-ತ್ತಾನೆ
ಯತ್ನಿ-ಸು-ತ್ತಿ-ರುವ
ಯತ್ನಿ-ಸು-ತ್ತಿ-ರು-ವುದು
ಯತ್ನಿ-ಸು-ತ್ತಿವೆ
ಯಥಾ
ಯಥಾ-ಪ್ರ-ಕಾರ
ಯಥಾರ್ಥ
ಯಥಾ-ರ್ಥ-ವಾಗಿ
ಯಥಾ-ವ-ತ್ತಾಗಿ
ಯಥಾ-ಸ್ಥಿ-ತಿ-ಯಲ್ಲಿ
ಯಥೇಷ್ಟಂ
ಯದ
ಯದನ್ನು
ಯದಾಗಿ
ಯದಿ
ಯದು
ಯನ
ಯನ್ನಾಗಿ
ಯನ್ನು
ಯನ್ನುಳ್ಳ
ಯನ್ನೂ
ಯನ್ನೆಲ್ಲ
ಯನ್ನೇ
ಯಮನ
ಯಮ-ನಾಗಿ
ಯರ
ಯರನ್ನೂ
ಯರಿಗೆ
ಯರು
ಯರೂ
ಯರೇ
ಯರ್
ಯಲು
ಯಲ್ಪಟ್ಟ
ಯಲ್ಲ
ಯಲ್ಲವೆ
ಯಲ್ಲವೇ
ಯಲ್ಲಿ
ಯಲ್ಲಿ-ಅ-ತ್ಯಂತ
ಯಲ್ಲಿ-ದ್ದಂತೆ
ಯಲ್ಲಿ-ರ-ಲಿಲ್ಲ
ಯಲ್ಲಿವೆ
ಯಲ್ಲೂ
ಯಲ್ಲೇ
ಯವ-ಕ-ರಿಗೆ
ಯವರ
ಯವ-ರನ್ನು
ಯವ-ರಲ್ಲಿ
ಯವ-ರಾ-ಡಿದ
ಯವ-ರಿಂದ
ಯವ-ರಿ-ಗಾ-ಗು-ತ್ತಿದ್ದ
ಯವ-ರಿ-ಗಿತ್ತು
ಯವ-ರಿ-ಗಿದ್ದ
ಯವ-ರಿಗೆ
ಯವ-ರಿಗೇ
ಯವ-ರಿನ್ನೂ
ಯವರು
ಯವರೂ
ಯವರೇ
ಯವ-ರೇನೋ
ಯವ-ರೊಂ-ದಿಗೆ
ಯವ-ರೊಬ್ಬ
ಯಶಸ್ವಿ
ಯಶ-ಸ್ವಿ-ಗ-ಳಾ-ಗಲಿ
ಯಶ-ಸ್ವಿ-ಗ-ಳಾ-ದಂತೆ
ಯಶ-ಸ್ವಿ-ಗಳೂ
ಯಶ-ಸ್ವಿ-ಯಾ-ಗ-ದಿ-ದ್ದರೆ
ಯಶ-ಸ್ವಿ-ಯಾಗಿ
ಯಶ-ಸ್ವಿ-ಯಾ-ಗಿ-ದ್ದರೂ
ಯಶ-ಸ್ವಿ-ಯಾ-ಗಿ-ದ್ದುವು
ಯಶ-ಸ್ವಿ-ಯಾ-ಗಿ-ದ್ದೇನೆ
ಯಶ-ಸ್ವಿ-ಯಾ-ಗಿ-ರು-ವು-ದ-ರಿಂದ
ಯಶ-ಸ್ವಿ-ಯಾ-ಗಿ-ಸಲು
ಯಶ-ಸ್ವಿ-ಯಾಗು
ಯಶ-ಸ್ವಿ-ಯಾ-ಗು-ವಂತೆ
ಯಶ-ಸ್ವಿ-ಯಾ-ದರು
ಯಶ-ಸ್ವಿ-ಯಾ-ದೇನು
ಯಶ-ಸ್ವಿ-ಯಾ-ದ್ದ-ರಿಂದ
ಯಶ-ಸ್ಸನ್ನು
ಯಶ-ಸ್ಸಿ-ಗಾಗಿ
ಯಶ-ಸ್ಸಿಗೆ
ಯಶ-ಸ್ಸಿಗೇ
ಯಶ-ಸ್ಸಿನ
ಯಶ-ಸ್ಸಿ-ನಲ್ಲೇ
ಯಶಸ್ಸು
ಯಶ-ಸ್ಸು-ಅ-ದ-ರಲ್ಲೂ
ಯಶ-ಸ್ಸು-ಗಳನ್ನು
ಯಶ-ಸ್ಸು-ಗಳಿಂದ
ಯಶ-ಸ್ಸು-ಗ-ಳಿಗೆ
ಯಶ-ಸ್ಸು-ಗಳು
ಯಷ್ಟು
ಯಸ್ಥ
ಯಹೂ-ದ್ಯರ
ಯಾಂಕಿ-ಗಳ
ಯಾಂಕಿ-ಗಳು
ಯಾಂತ್ರಿ-ಕ-ವಾಗಿ
ಯಾಕದು
ಯಾಕೆ
ಯಾಕೋ
ಯಾಗದ
ಯಾಗ-ಬಾ-ರದು
ಯಾಗಲಿ
ಯಾಗಲು
ಯಾಗಿ
ಯಾಗಿತ್ತು
ಯಾಗಿದ್ದ
ಯಾಗಿ-ದ್ದರು
ಯಾಗಿ-ದ್ದುವು
ಯಾಗಿದ್ದೆ
ಯಾಗಿ-ರ-ಲಿ-ಮೊ-ದಲು
ಯಾಗಿ-ರು-ವಂತೆ
ಯಾಗು-ತ್ತಾನೆ
ಯಾಗು-ತ್ತಿತ್ತು
ಯಾಗು-ತ್ತಿ-ರು-ವುದು
ಯಾಗುವ
ಯಾಗು-ವಂತೆ
ಯಾಚ-ನೆಯ
ಯಾಚಿಸಿ
ಯಾಡ-ಬೇಡಿ
ಯಾತ-ನೆ-ಗಳ
ಯಾತ-ನೆ-ಗಳನ್ನು
ಯಾತ-ನೆ-ಗಳೂ
ಯಾತ-ನೆಗೆ
ಯಾತ-ನೆ-ಯನ್ನು
ಯಾತ-ನೆ-ಯಾ-ಗು-ತ್ತಿತ್ತು
ಯಾತ-ನೆ-ಯಾ-ಯಿತು
ಯಾತ-ನೆ-ಯಿಂದ
ಯಾತ್ರಿ-ಕ-ನಿ-ಷ್ಠೆ-ಯನ್ನು
ಯಾತ್ರಿ-ಕರ
ಯಾತ್ರಿ-ಕ-ರಂತೆ
ಯಾತ್ರಿ-ಕ-ರನ್ನು
ಯಾತ್ರಿ-ಕ-ರ-ನ್ನೆಲ್ಲ
ಯಾತ್ರಿ-ಕ-ರಿಂದ
ಯಾತ್ರಿ-ಕ-ರಿಗೆ
ಯಾತ್ರಿ-ಕರು
ಯಾತ್ರಿ-ಕ-ರೆಲ್ಲ
ಯಾತ್ರಿ-ಕರೇ
ಯಾತ್ರಿ-ಕ-ರೊಂ-ದಿ-ಗಿದ್ದು
ಯಾತ್ರಿ-ಕ-ರೊಂ-ದಿಗೆ
ಯಾತ್ರೆ
ಯಾತ್ರೆಗೆ
ಯಾತ್ರೆಯ
ಯಾತ್ರೆ-ಯನ್ನು
ಯಾತ್ರೆ-ಯಿಂದ
ಯಾದ
ಯಾದದ್ದು
ಯಾದ-ವ-ಗಿರಿ
ಯಾನ
ಯಾನ-ದಂತೆ
ಯಾನ-ದಿಂದ
ಯಾಪನ
ಯಾಮದ
ಯಾಮಿನೀ
ಯಾಮಿ-ನೀ-ರಂ-ಜನ
ಯಾಮಿ-ನೀ-ರಂ-ಜ-ನನ
ಯಾಯಿ-ಗಳು
ಯಾಯಿತು
ಯಾಯಿ-ತು-ಎಂ-ದರೆ
ಯಾಯಿ-ಯಾ-ಗಲು
ಯಾಯಿ-ಯೊ-ಬ್ಬರು
ಯಾರ
ಯಾರದೂ
ಯಾರದೇ
ಯಾರ-ನ್ನಾ-ದರೂ
ಯಾರನ್ನು
ಯಾರನ್ನೂ
ಯಾರ-ಲ್ಲಿಯೂ
ಯಾರಾ
ಯಾರಾ-ದರೂ
ಯಾರಾ-ದ-ರೊಬ್ಬ
ಯಾರಾ-ದ-ರೊ-ಬ್ಬರ
ಯಾರಾ-ದ-ರೊ-ಬ್ಬ-ರಲ್ಲಿ
ಯಾರಾ-ದ-ರೊ-ಬ್ಬರು
ಯಾರಿ
ಯಾರಿಂದ
ಯಾರಿಂ-ದಲೂ
ಯಾರಿಂ-ದಾ-ದೀತು
ಯಾರಿ-ಗಾ-ದರೂ
ಯಾರಿ-ಗಿ-ದ್ದೀತು
ಯಾರಿಗೂ
ಯಾರಿಗೆ
ಯಾರಿಗೋ
ಯಾರಿ-ದ್ದರು
ಯಾರಿ-ಬ್ಬರು
ಯಾರಿ-ರ-ಬ-ಹು-ದಪ್ಪ
ಯಾರಿ-ರ-ಬ-ಹುದು
ಯಾರಿ-ಲ್ಲಿ-ಲೆ-ಕ್ಕಕ್ಕೆ
ಯಾರು
ಯಾರು-ಏನು
ಯಾರು-ಯಾರು
ಯಾರೂ
ಯಾರೆಂ-ದ-ರ-ವರು
ಯಾರೆಂ-ಬು-ದನ್ನು
ಯಾರೆಂ-ಬು-ದನ್ನೇ
ಯಾರೆಂ-ಬು-ದರ
ಯಾರೆಂ-ಬುದು
ಯಾರೆಂ-ಬುದೇ
ಯಾರೇ
ಯಾರೊ
ಯಾರೊಡ
ಯಾರೊ-ಡ-ನೆಯೂ
ಯಾರೊ-ಬ್ಬ-ರನ್ನೂ
ಯಾರೊ-ಬ್ಬ-ರಲ್ಲೂ
ಯಾರೊ-ಳಗೆ
ಯಾರೋ
ಯಾರ್ಕಿನ
ಯಾರ್ಯಾರ
ಯಾರ್ಯಾರು
ಯಾವ
ಯಾವತ್ತು
ಯಾವ-ನಾ-ದರೂ
ಯಾವನಿ
ಯಾವನು
ಯಾವನೋ
ಯಾವ-ಯಾವ
ಯಾವಾ
ಯಾವಾಗ
ಯಾವಾ-ಗ-ಲಾ-ದರೂ
ಯಾವಾ-ಗ-ಲಾ-ದ-ರೊಮ್ಮೆ
ಯಾವಾ-ಗಲೂ
ಯಾವಾ-ಗಲೋ
ಯಾವಾ-ಗೆಂ-ದ-ರಾ-ವಾಗ
ಯಾವು
ಯಾವುದ
ಯಾವು-ದಕ್ಕೂ
ಯಾವು-ದದು
ಯಾವು-ದ-ನ್ನಾ-ದರೂ
ಯಾವುದನ್ನು
ಯಾವು-ದನ್ನೂ
ಯಾವು-ದನ್ನೇ
ಯಾವು-ದ-ರಿಂ-ದಲೂ
ಯಾವು-ದಾ-ಗಿ-ತ್ತೆಂ-ದರೆ
ಯಾವು-ದಾ-ದರೂ
ಯಾವು-ದಾ-ದ-ರೊಂದು
ಯಾವು-ದಿದೆ
ಯಾವು-ದಿ-ದ್ದಿ-ರ-ಬ-ಹುದು
ಯಾವು-ದಿ-ರ-ಬ-ಹುದು
ಯಾವು-ದಿ-ರ-ಬ-ಹು-ದೆಂದು
ಯಾವು-ದಿ-ರ-ಬೇಕು
ಯಾವುದು
ಯಾವುದೂ
ಯಾವು-ದೆಂ-ದರೆ
ಯಾವುದೇ
ಯಾವುದೋ
ಯಾವು-ವೆಂದರೆ
ಯಾವೊಂದು
ಯಾಸ-ವಾಗಿ
ಯಿಂದ
ಯಿಂದಲೇ
ಯಿಂದಾಗಿ
ಯಿಂದಾ-ಚೆಗೆ
ಯಿಂದಿರು
ಯಿಟ್ಟು
ಯಿತು
ಯಿತ್ತರು
ಯಿದ್ದ
ಯಿದ್ದಾಗ
ಯಿದ್ದು-ದ-ರಿಂದ
ಯಿಯೂ
ಯಿರ-ಲಿಲ್ಲ
ಯಿರುವ
ಯಿಲ್ಲ
ಯಿಲ್ಲದೆ
ಯಿಸ-ಬೇಕು
ಯಿಸು-ವಂ-ತಾ-ದರೆ
ಯುಂಟಾ-ಗಲಿ
ಯುಂಟಾ-ಗು-ತ್ತದೆ
ಯುಂಟಾ-ಗು-ತ್ತ-ದೆಂದೋ
ಯುಂಟಾ-ಯಿತು
ಯುಕ್ತ
ಯುಕ್ತ-ವಾ-ದದ್ದು
ಯುಕ್ತವೂ
ಯುಕ್ತಾ-ಯು-ಕ್ತ-ತೆ-ಯನ್ನೂ
ಯುಗ
ಯುಗಕ್ಕೆ
ಯುಗ-ಗ-ಳಿಂ-ದಲೂ
ಯುಗದ
ಯುಗ-ದಲ್ಲಿ
ಯುಗ-ಪ್ರ-ವ-ರ್ತ-ಕರು
ಯುಗ-ಪ್ರ-ವ-ರ್ತ-ನಾ-ಚಾ-ರ್ಯ-ನನ್ನು
ಯುಗ-ಯು-ಗ-ಗಳ
ಯುಗ-ಯು-ಗ-ಗಳಲ್ಲಿ
ಯುಗ-ಯು-ಗ-ಗಳಿಂದ
ಯುಗ-ಯು-ಗಾಂ
ಯುಗ-ಯು-ಗಾಂ-ತ-ರ-ಗ-ಳಿಂ-ದಲೂ
ಯುಗ-ಳ-ಬಂದಿ
ಯುಗ-ವನ್ನು
ಯುಗಾಂ-ತ-ರ-ಗಳ
ಯುಗಾಂ-ತರೇ
ಯುಗಾ-ವ-ತಾರ
ಯುತ್ತಿ-ದ್ದರು
ಯುದ್ದಕ್ಕೂ
ಯುದ್ಧ
ಯುದ್ಧ-ಗಳು
ಯುದ್ಧದ
ಯುದ್ಧ-ವೀ-ರ-ನಾದ
ಯುರೋ-ಪಿನ
ಯುವ
ಯುವಂತೆ
ಯುವಕ
ಯುವ-ಕ-ನಾ-ಗಿ-ದ್ದಾ-ಗಿ-ನಿಂ-ದಲೂ
ಯುವ-ಕ-ನಿಗೆ
ಯುವ-ಕನೂ
ಯುವ-ಕ-ನೊಬ್ಬ
ಯುವ-ಕರ
ಯುವ-ಕ-ರನ್ನು
ಯುವ-ಕ-ರ-ನ್ನು-ದ್ದೇ-ಶಿಸಿ
ಯುವ-ಕ-ರಲ್ಲಿ
ಯುವ-ಕ-ರಿಂ-ದಲೇ
ಯುವ-ಕ-ರಿಗೆ
ಯುವ-ಕ-ರಿ-ದ್ದರು
ಯುವ-ಕ-ರಿ-ದ್ದೀರಿ
ಯುವ-ಕರು
ಯುವ-ಕರೂ
ಯುವ-ಕ-ರೆಂ-ದರೆ
ಯುವ-ಕ-ರೆಲ್ಲ
ಯುವ-ಕರೇ
ಯುವ-ಕ-ರೊಂ-ದಿಗೆ
ಯುವ-ಕ-ಶಿ-ಷ್ಯರ
ಯುವ-ಜ-ನತೆ
ಯುವ-ಜ-ನ-ತೆಗೆ
ಯುವ-ಜ-ನ-ರನ್ನು
ಯುವ-ಜ-ನ-ರಿಂದ
ಯುವ-ಜ-ನ-ರಿಗೆ
ಯುವ-ಜ-ನರು
ಯುವ-ಜ-ನರೇ
ಯುವ-ಜ-ನಾಂ-ಗ-ದಲ್ಲಿ
ಯುವ-ಜ-ನಾಂ-ಗ-ದಿಂದ
ಯುವತಿ
ಯುವ-ತಿಯ
ಯುವ-ತಿ-ಯೊ-ಬ್ಬಳು
ಯುವ-ಬ್ರ-ಹ್ಮ-ಚಾ-ರಿ-ಗಳು
ಯುವ-ಭಾ-ರತ
ಯುವ-ಸಂ-ನ್ಯಾ-ಸಿ-ಗ-ಳಾದ
ಯುವ-ಸಾ-ಧ-ಕ-ರಲ್ಲಿ
ಯೂನಿಟಿ
ಯೂನಿ-ಟೇ-ರಿ-ಯನ್
ಯೂನಿ-ಯನ್
ಯೂರೋಪಿ
ಯೂರೋ-ಪಿಗೆ
ಯೂರೋ-ಪಿನ
ಯೂರೋ-ಪಿ-ನ-ವನೊ
ಯೂರೋ-ಪಿ-ನಿಂದ
ಯೂರೋಪು
ಯೂರೋ-ಪು-ಅ-ಮೆ-ರಿ-ಕ-ಗ-ಳಲ್ಲೇ
ಯೂರೋಪ್
ಯೂರೋ-ಪ್-ಅ-ಮೆ-ರಿ-ಕ-ಗಳಲ್ಲಿ
ಯೆಂದರೆ
ಯೆಂದು
ಯೆಂದೂ
ಯೆಂಬು-ದನ್ನು
ಯೆಂಬು-ದಾ-ಗಲಿ
ಯೆಂಬುದು
ಯೆಡೆಗೆ
ಯೆರೆ-ಯಲು
ಯೆಲ್ಲ
ಯೇನು
ಯೇನೆಂ-ದರೆ
ಯೇರಿತು
ಯೊಂದರ
ಯೊಂದ-ರಲ್ಲಿ
ಯೊಂದಿಗೆ
ಯೊಂದು
ಯೊಕೋ-ಹಾ-ಮಾ-ದಿಂದ
ಯೊಬ್ಬ-ನಲ್ಲೂ
ಯೊಬ್ಬ-ನಿಗೆ
ಯೊಬ್ಬನು
ಯೊಬ್ಬನೂ
ಯೊಬ್ಬ-ರಿಗೂ
ಯೊಳ-ಗಾಗಿ
ಯೊಳಗೆ
ಯೋಗ
ಯೋಗ-ಕ್ಷೇ-ಮದ
ಯೋಗ-ಕ್ಷೇ-ಮ-ವನ್ನು
ಯೋಗ-ಗಳ
ಯೋಗ-ಗಳನ್ನು
ಯೋಗ-ಗ-ಳ-ಲ್ಲಂತೂ
ಯೋಗದ
ಯೋಗ-ಭ್ರ-ಷ್ಟ-ನಾ-ದ-ವನು
ಯೋಗ-ಭ್ರ-ಷ್ಟ-ರಾ-ದ-ವರ
ಯೋಗ-ಮಾ-ರ್ಗ-ದಲ್ಲಿ
ಯೋಗ-ವನ್ನು
ಯೋಗ-ಶಕ್ತಿ
ಯೋಗ-ಸೂ-ತ್ರ-ಗಳಲ್ಲಿ
ಯೋಗಾ
ಯೋಗಾ-ನಂ-ದರು
ಯೋಗಾ-ನಂ-ದ-ರೊಂ-ದಿ-ಗಿನ
ಯೋಗಾ-ನಂ-ದ-ರೊಂ-ದಿಗೆ
ಯೋಗಾ-ಭ್ಯಾ-ಸ-ದಲ್ಲಿ
ಯೋಗಿ
ಯೋಗಿ-ಗಳ
ಯೋಗಿ-ಗ-ಳಿಗೆ
ಯೋಗ್ಯ
ಯೋಗ್ಯತೆ
ಯೋಗ್ಯ-ತೆಗೆ
ಯೋಗ್ಯ-ನಲ್ಲ
ಯೋಗ್ಯ-ರಾದ
ಯೋಗ್ಯ-ರೀ-ತಿ-ಯಲ್ಲಿ
ಯೋಗ್ಯ-ವಾದ
ಯೋಗ್ಯ-ವಾ-ದಂ-ಥ-ದೇ-ನನ್ನೂ
ಯೋಗ್ಯ-ವಾ-ದದ್ದು
ಯೋಚ-ನೆ-ಯನ್ನೇ
ಯೋಚ-ನೆಯೇ
ಯೋಚಿ-ಸ-ದಿ-ದ್ದರೆ
ಯೋಚಿ-ಸ-ಬ-ಲ್ಲನೇ
ಯೋಚಿ-ಸ-ಲಾರ
ಯೋಚಿಸಿ
ಯೋಚಿ-ಸಿದಂ
ಯೋಚಿ-ಸಿ-ದಂ-ತಿಲ್ಲ
ಯೋಚಿ-ಸಿ-ದ್ದಿದೆ
ಯೋಚಿ-ಸು-ವ-ವ-ರಾರು
ಯೋಜನೆ
ಯೋಜ-ನೆ-ಗಳ
ಯೋಜ-ನೆ-ಗಳನ್ನು
ಯೋಜ-ನೆ-ಗಳನ್ನೆಲ್ಲ
ಯೋಜ-ನೆ-ಗಳಾ
ಯೋಜ-ನೆ-ಗ-ಳಾ-ವುವೂ
ಯೋಜ-ನೆ-ಗ-ಳಿ-ಗಾಗಿ
ಯೋಜ-ನೆ-ಗ-ಳಿಗೆ
ಯೋಜ-ನೆ-ಗಳು
ಯೋಜ-ನೆ-ಗಳೂ
ಯೋಜ-ನೆ-ಗ-ಳೆಲ್ಲ
ಯೋಜ-ನೆಗೆ
ಯೋಜ-ನೆಯ
ಯೋಜ-ನೆ-ಯನ್ನು
ಯೋಜ-ನೆ-ಯಲ್ಲಿ
ಯೋಜ-ನೆ-ಯಾಗಿ
ಯೋಜ-ನೆ-ಯಾ-ಗಿತ್ತು
ಯೋಜ-ನೆ-ಯಿತ್ತು
ಯೋಜ-ನೆಯು
ಯೋಜ-ನೆ-ಯೇ-ನನ್ನೋ
ಯೋಜ-ನೆ-ಯೇನು
ಯೋಜಿ-ಸ-ಲಾ-ಗಿತ್ತು
ಯೋಜಿ-ಸಿ-ಕೊಂಡು
ಯೋಧ
ಯೌವನ
ಯೌವ-ನದ
ರ
ರಂಗ-ಗಳನ್ನೂ
ರಂಗ-ನಟಿ
ರಂಗಾ-ಚಾ-ರ್ಯ-ರನ್ನು
ರಂಗಾ-ಚಾ-ರ್ಯರು
ರಂಗಾ-ಚಾ-ರ್ಯರೂ
ರಂಗಾ-ಲಯ
ರಂಜ-ನೀ-ಯ-ವಾ-ಗಿಯೂ
ರಂಜನ್
ರಂಜಿ-ತ-ರಾಗಿ
ರಂಜಿ-ಸಿ-ದರು
ರಂತಹ
ರಂತೂ
ರಂತೆ
ರಂತೆಯೇ
ರಂಥ-ವರು
ರಂಥ-ವ-ರೊ-ಬ್ಬ-ರನ್ನು
ರಂದು
ರಂದು-ಎಂ-ದರೆ
ರಂದೇ
ರಂಧ್ರ-ಗಳನ್ನು
ರಂಧ್ರ-ವನ್ನು
ರಂಭ
ರಂಭ-ಕ್ಕಾಗಿ
ರಂಭದ
ರಕ್ತ
ರಕ್ತ-ಇ-ವು-ಗಳನ್ನೆಲ್ಲ
ರಕ್ತ-ಕಾ-ರು-ವಂ-ತಾ-ಗಲಿ
ರಕ್ತ-ಗ-ತ-ಮಾ-ಡಿ-ಕೊಳ್ಳಿ
ರಕ್ತ-ಗ-ತ-ವಾ-ಗು-ವ-ವ-ರೆಗೆ
ರಕ್ತ-ದಲ್ಲಿ
ರಕ್ತ-ದಿಂದ
ರಕ್ತ-ಮಾಂ-ಸ-ಗಳ
ರಕ್ತ-ವನ್ನು
ರಕ್ತ-ವನ್ನೇ
ರಕ್ತ-ವ-ರ್ಣ-ವನ್ನು
ರಕ್ತವೇ
ರಕ್ತ-ಸ್ರಾ-ವ-ವನ್ನು
ರಕ್ಷ-ಕರು
ರಕ್ಷಕಿ
ರಕ್ಷಣೆ
ರಕ್ಷ-ಣೆಗೆ
ರಕ್ಷ-ಣೆ-ಮಾ-ಡಲು
ರಕ್ಷಿ-ಸ-ಲಿಲ್ಲ
ರಕ್ಷಿ-ಸಲು
ರಕ್ಷಿ-ಸಿ-ಕೊಂಡು
ರಕ್ಷಿ-ಸಿ-ಕೊ-ಳ್ಳಲು
ರಕ್ಷಿ-ಸಿತು
ರಕ್ಷಿಸು
ರಕ್ಷಿ-ಸು-ತ್ತಿದ್ದೆ
ರಕ್ಷಿ-ಸು-ತ್ತೇವೆ
ರಕ್ಷಿ-ಸುವ
ರಕ್ಷಿ-ಸು-ವ-ವನೋ
ರಕ್ಷಿ-ಸು-ವ-ವಳೋ
ರಕ್ಷಿ-ಸು-ವು-ದ-ಕ್ಕಾಗಿ
ರಗಳೆ
ರಘು-ನಾಥ
ರಚ-ನಾ-ತ್ಮಕ
ರಚ-ನಾ-ತ್ಮ-ಕ-ವಾಗಿ
ರಚ-ನೆಯ
ರಚ-ನೆ-ಯನ್ನು
ರಚ-ನೆ-ಯಾ-ಗ-ಬೇಕು
ರಚಿ
ರಚಿ-ತ-ವಾದ
ರಚಿ-ತ-ವಾ-ದ್ದ-ರಿಂದ
ರಚಿ-ತ-ವಾ-ಯಿತು
ರಚಿ-ಸ-ಲಾ-ಗಿತ್ತು
ರಚಿ-ಸ-ಲಾಗಿದೆ
ರಚಿ-ಸ-ಲಾ-ಯಿತು
ರಚಿಸಿ
ರಚಿ-ಸಿ-ಕೊಂ-ಡರು
ರಚಿ-ಸಿ-ಕೊ-ಳ್ಳಲು
ರಚಿ-ಸಿದ
ರಚಿ-ಸಿ-ದರು
ರಚಿ-ಸಿದ್ದ
ರಚಿ-ಸು-ತ್ತಿ-ರುವ
ರಚಿ-ಸು-ವುದು
ರಚ್ಚೆ
ರಜತ
ರಜ-ಪೂ-ತರ
ರಜಾ
ರಜಾ-ದಿ-ನ-ಗಳ
ರಜೆಯ
ರಜೋ-ಗು-ಣದ
ರಜೋ-ಗು-ಣ-ವನ್ನು
ರಜೋ-ಗು-ಣ-ವನ್ನೂ
ರಟ್ಲಾಮ್
ರಡೂ
ರಣ-ಕ-ಹಳೆ
ರಣ-ಕ-ಹ-ಳೆಯ
ರಣ-ರಂಗ
ರಣ-ರಂ-ಗ-ದಲ್ಲೇ
ರಣಾ
ರಣೆಯೂ
ರತ್ನ
ರತ್ನ-ಗಂ-ಬಳಿ
ರತ್ನ-ಗಳು
ರತ್ನಾ-ಭ-ರ-ಣ-ಗಳನ್ನು
ರತ್ನಾ-ಭ-ರ-ಣ-ಗಳನ್ನೂ
ರಥ-ದಲ್ಲಿ
ರಥೋ-ತ್ಸ-ವ-ದಂ-ತಿತ್ತು
ರದು
ರದೆ
ರದೋ
ರದ್ದು-ಗೊ-ಳಿ-ಸ-ಬೇ-ಕಾ-ಯಿ-ತೆಂದೂ
ರದ್ದು-ಪ-ಡಿ-ಸ-ಬೇ-ಕಾ-ಯಿತು
ರದ್ದು-ಪ-ಡಿ-ಸಿ-ದರು
ರನ್ನು
ರನ್ನು-ಶಿ-ಷ್ಯ-ರನ್ನು
ರನ್ನೂ
ರಬ್ಬರೇ
ರಭಸ
ರಭ-ಸ-ದಿಂದ
ರಭ-ಸ-ಪೂ-ರ್ಣ-ವಾಗಿ
ರಭ-ಸ-ಪೂ-ರ್ಣ-ವಾ-ಗಿ-ದ್ದುವು
ರಭ-ಸ-ವನ್ನು
ರಭ-ಸವು
ರಮ-ಣೀಯ
ರಮ-ಣೀ-ಯ-ವಾದ
ರಮಾ-ರ್ಚ್
ರಮೇ-ಶ-ಚಂದ್ರ
ರಮ್ಯ
ರಮ್ಯ-ನೋ-ಟ-ವನ್ನು
ರಮ್ಯ-ವಾದ
ರರ
ರರು
ರರೂ
ರಲ್ಲ
ರಲ್ಲವೆ
ರಲ್ಲ-ವೆಂ-ಬುದು
ರಲ್ಲಿ
ರಲ್ಲೇ
ರಲ್ಲೊ-ಬ್ಬರು
ರವರ
ರವರು
ರವ-ರೆಗೂ
ರವ-ರೆಗೆ
ರವೀಂ-ದ್ರ-ನಾಥ
ರಶ್ಮಿಃ
ರಷ್ಯಾ
ರಷ್ಯಾದ
ರಸ
ರಸ-ಕ-ವ-ಳದ
ರಸ-ದೌ-ತ-ಣ-ವ-ನ್ನು-ಣಿ-ಸಿತು
ರಸ-ಪೂ-ರ್ಣವೂ
ರಸ-ಮಯ
ರಸ-ಮ-ಯ-ವಾ-ಗಿತ್ತು
ರಸ-ವ-ತ್ತಾಗಿ
ರಸಾ-ನು-ಭವ
ರಸಾ-ನು-ಭ-ವ-ಗಳನ್ನೆಲ್ಲ
ರಸಾ-ನು-ಭ-ವ-ವನ್ನು
ರಸ್ತೆ
ರಸ್ತೆ-ಗಳನ್ನು
ರಸ್ತೆ-ಗಳನ್ನೂ
ರಸ್ತೆ-ಗಳಲ್ಲಿ
ರಸ್ತೆಗೆ
ರಸ್ತೆ-ಬ-ದಿ-ಯಲ್ಲಿ
ರಸ್ತೆಯ
ರಸ್ತೆ-ಯನ್ನೂ
ರಸ್ತೆ-ಯಲ್ಲಿ
ರಸ್ತೆ-ಯಿಂದ
ರಸ್ತೆ-ಯು-ದ್ದಕ್ಕೂ
ರಸ್ತೆಯೋ
ರಹಸ್ಯ
ರಹ-ಸ್ಯ-ಗ-ಳಿಗೆ
ರಹ-ಸ್ಯ-ಗ-ಳಿ-ಗೆಲ್ಲ
ರಹ-ಸ್ಯ-ವ-ಡ-ಗಿ-ರು-ವುದು
ರಹ-ಸ್ಯ-ವಾಗಿ
ರಹ-ಸ್ಯ-ವಿ-ದ್ಯೆ-ಗಳನ್ನು
ರಹ-ಸ್ಯ-ವಿ-ದ್ಯೆ-ಗಳಲ್ಲಿ
ರಹ-ಸ್ಯ-ವಿ-ದ್ಯೆ-ಗ-ಳು-ಪ-ವಾ-ಡ-ಗ-ಳೆಲ್ಲ
ರಹಿತ
ರಾಕ್ಷ-ಸರ
ರಾಕ್ಷ-ಸ-ರಿಗೆ
ರಾಕ್ಷಸೀ
ರಾಖೇಲ್
ರಾಗ-ಅ-ಲ್ಲಿನ
ರಾಗ-ದಿ-ರಲು
ರಾಗದೇ
ರಾಗ-ದ್ವೇಷ
ರಾಗಲು
ರಾಗಿ
ರಾಗಿದ್ದ
ರಾಗಿ-ದ್ದರು
ರಾಗಿ-ರ-ಬ-ಹುದೆ
ರಾಗಿ-ರುತ್ತ
ರಾಗಿ-ರು-ವರೋ
ರಾಗು-ತ್ತಾರೆ
ರಾಗು-ತ್ತೇ-ವೆಂದು
ರಾಗು-ವುದನ್ನು
ರಾಗು-ವುದು
ರಾಜ
ರಾಜಂ
ರಾಜ-ಕಾ-ರಣ
ರಾಜ-ಕಾರಣ-ದ-ಲ್ಲಲ್ಲ
ರಾಜ-ಕಾರಣ-ದಲ್ಲಿ
ರಾಜ-ಕಾರಣ-ದಿಂದ
ರಾಜ-ಕಾರಣ-ದೊಂ-ದಿಗೆ
ರಾಜ-ಕೀಯ
ರಾಜ-ಕೀ-ಯದ
ರಾಜ-ಕೀ-ಯ-ವನ್ನೋ
ರಾಜ-ಕೀ-ಯ-ಸ್ಥರು
ರಾಜ-ಕು-ವರಿ
ರಾಜ-ಗಾಂ-ಭೀ-ರ್ಯ-ದಿಂದ
ರಾಜ-ಗೋ-ಪಾಲಾ
ರಾಜ-ಠೀ-ವಿ-ಯನ್ನು
ರಾಜ-ತ್ವ-ವೆಂ-ಬುದು
ರಾಜ-ಧಾನಿ
ರಾಜ-ಧಾ-ನಿ-ಯ-ಲ್ಲಿ-ರ-ಲಿಲ್ಲ
ರಾಜ-ಧಾ-ನಿ-ಯಷ್ಟೇ
ರಾಜ-ಧಾ-ನಿ-ಯಾದ
ರಾಜನ
ರಾಜ-ನನ್ನು
ರಾಜ-ನಿಗೆ
ರಾಜನು
ರಾಜನೂ
ರಾಜನೇ
ರಾಜ-ನೊಂ-ದಿಗೆ
ರಾಜ-ನೌ-ಕೆ-ಯಲ್ಲಿ
ರಾಜ-ಯೋಗ
ರಾಜ-ಯೋ-ಗ-ಗಳನ್ನು
ರಾಜ-ಯೋ-ಗವೇ
ರಾಜರ
ರಾಜ-ರತ್ನ
ರಾಜ-ರನ್ನು
ರಾಜ-ರಿಗೆ
ರಾಜರು
ರಾಜ-ರು-ಗಳನ್ನೂ
ರಾಜ-ರು-ಗ-ಳಿಗೆ
ರಾಜರೇ
ರಾಜರ್ಷಿ
ರಾಜ-ವಾ-ಹ-ನ-ದಲ್ಲಿ
ರಾಜ-ಸ-ಹ-ಜ-ವಾದ
ರಾಜ-ಸ್ಥಾ-ನದ
ರಾಜಾ
ರಾಜಾ-ಧಿ-ರಾ-ಜರು
ರಾಜಾ-ರೋ-ಷ-ವಾಗಿ
ರಾಜಿ
ರಾಜಿಯ
ರಾಜೀ-ನಾಮೆ
ರಾಜೀ-ನಾ-ಮೆ-ಯಿ-ತ್ತಿ-ದ್ದರು
ರಾಜ್ಯಕ್ಕೆ
ರಾಜ್ಯಕ್ಕೇ
ರಾಜ್ಯದ
ರಾಜ್ಯ-ದಲ್ಲಿ
ರಾಜ್ಯ-ದಲ್ಲೂ
ರಾಜ್ಯ-ಪಾ-ಲ-ನೆಯ
ರಾಜ್ಯ-ಭಾರ
ರಾಜ್ಯ-ವನ್ನೇ
ರಾಜ್ಯ-ವಾಳ
ರಾಡಿ-ಯಾ-ಗಿ-ಬಿ-ಟ್ಟಿತ್ತು
ರಾಣಿ
ರಾಣೀ
ರಾತೀ
ರಾತ್ರಿ
ರಾತ್ರಿ-ಗಳಲ್ಲಿ
ರಾತ್ರಿ-ತಾನೆ
ರಾತ್ರಿಯ
ರಾತ್ರಿ-ಯಂತೂ
ರಾತ್ರಿ-ಯನ್ನು
ರಾತ್ರಿ-ಯಲ್ಲಿ
ರಾತ್ರಿ-ಯ-ವ-ರೆಗೂ
ರಾತ್ರಿ-ಯ-ವ-ರೆಗೆ
ರಾತ್ರಿ-ಯಾ-ಗಿತ್ತು
ರಾತ್ರಿ-ಯಿಡೀ
ರಾತ್ರಿಯು
ರಾತ್ರಿ-ಯೆಲ್ಲ
ರಾತ್ರಿಯೇ
ರಾತ್ರಿ-ಯೊ-ಳ-ಗಾಗಿ
ರಾದ
ರಾದರು
ರಾದರೂ
ರಾದರೆ
ರಾದ-ರೆಂ-ದರೆ
ರಾದ-ವರು
ರಾದ-ವರೆ-ಲ್ಲರೂ
ರಾದು-ದ-ರಿಂದ
ರಾಧಾ-ಕೃ-ಷ್ಣರ
ರಾಧಾ-ಕಾಂತ
ರಾಮ
ರಾಮ-ಕೃಷ್ಣ
ರಾಮ-ಕೃ-ಷ್ಣ-ವಿ-ವೇ-ಕಾ-ನಂ-ದರು
ರಾಮ-ಕೃ-ಷ್ಣ-ದೇವ
ರಾಮ-ಕೃ-ಷ್ಣ-ನಾ-ಮ-ವನ್ನು
ರಾಮ-ಕೃ-ಷ್ಣ-ಪು-ರದ
ರಾಮ-ಕೃ-ಷ್ಣ-ಪು-ರ-ದಲ್ಲಿ
ರಾಮ-ಕೃ-ಷ್ಣರ
ರಾಮ-ಕೃ-ಷ್ಣ-ರನ್ನು
ರಾಮ-ಕೃ-ಷ್ಣ-ರಲ್ಲಿ
ರಾಮ-ಕೃ-ಷ್ಣ-ರ-ಲ್ಲಿಗೆ
ರಾಮ-ಕೃ-ಷ್ಣ-ರಿಗೆ
ರಾಮ-ಕೃ-ಷ್ಣರು
ರಾಮ-ಕೃ-ಷ್ಣರೇ
ರಾಮ-ಕೃಷ್ಣಾ
ರಾಮ-ಕೃ-ಷ್ಣಾ-ನಂದ
ರಾಮ-ಕೃ-ಷ್ಣಾ-ನಂ-ದರ
ರಾಮ-ಕೃ-ಷ್ಣಾ-ನಂ-ದ-ರನ್ನು
ರಾಮ-ಕೃ-ಷ್ಣಾ-ನಂ-ದ-ರನ್ನೂ
ರಾಮ-ಕೃ-ಷ್ಣಾ-ನಂ-ದ-ರನ್ನೇ
ರಾಮ-ಕೃ-ಷ್ಣಾ-ನಂ-ದ-ರಿಗೆ
ರಾಮ-ಕೃ-ಷ್ಣಾ-ನಂ-ದ-ರಿಗೇ
ರಾಮ-ಕೃ-ಷ್ಣಾ-ನಂ-ದರು
ರಾಮ-ಕೃ-ಷ್ಣಾ-ನಂ-ದರೂ
ರಾಮ-ಕೃ-ಷ್ಣಾ-ನಂ-ದರೋ
ರಾಮ-ಕೃ-ಷ್ಣಾಯ
ರಾಮ-ಚಂದ್ರ
ರಾಮ-ಚಂ-ದ್ರನ
ರಾಮ-ಚಂ-ದ್ರ-ನಿಗೆ
ರಾಮ-ಚಂ-ದ್ರ-ನೊ-ಡನೆ
ರಾಮ-ತೀ-ರ್ಥ-ರೆಂದು
ರಾಮ-ತೀ-ರ್ಥ-ರೆಂಬ
ರಾಮನ
ರಾಮ-ನ-ರೇಂ-ದ್ರ-ನಾಥ
ರಾಮ-ನಾಗಿ
ರಾಮ-ನಾ-ಡನ್ನು
ರಾಮ-ನಾ-ಡಿಗೆ
ರಾಮ-ನಾ-ಡಿನ
ರಾಮ-ನಾ-ಡಿ-ನಿಂದ
ರಾಮ-ನಾ-ಮ-ದಿಂದ
ರಾಮ-ನಾ-ಮ-ಸಂ-ಕೀ-ರ್ತ-ನೆಯ
ರಾಮ-ನಾ-ಮೋ-ಚ್ಚಾ-ರಣೆ
ರಾಮ-ನಿ-ರು-ವನೊ
ರಾಮ-ನಿ-ರು-ವೆಡೆ
ರಾಮ-ನಿಲ್ಲ
ರಾಮ-ಪ್ರ-ಸಾ-ದ-ನೆಂಬ
ರಾಮ-ರಾ-ಜ್ಯ-ವನ್ನೇ
ರಾಮರು
ರಾಮ-ಲಾಲ
ರಾಮ-ಲಾ-ಲ-ನಿಗೆ
ರಾಮ-ಲಾಲ್
ರಾಮ-ಸಿಂಗ್
ರಾಮಾ-ನುಜ
ರಾಮಾ-ನು-ಜರು
ರಾಮಾ-ನು-ಜಾ-ಚಾರಿ
ರಾಮಾ-ಯಣ
ರಾಮೇ-ಶ್ವರ
ರಾಮೇ-ಶ್ವ-ರಕ್ಕೆ
ರಾಮೇ-ಶ್ವ-ರ-ಗ-ಳಂ-ತಹ
ರಾಮೇ-ಶ್ವ-ರದ
ರಾಮೇ-ಶ್ವ-ರ-ದತ್ತ
ರಾಮೇ-ಶ್ವ-ರ-ದಿಂದ
ರಾಮೇ-ಶ್ವ-ರನ
ರಾಮ್
ರಾಮ್ದಾದಾ
ರಾಮ್ಸ್ನೇಹಿ
ರಾಮ್ಸ್ನೇ-ಹಿ-ಗಾದ
ರಾಯ-ಭಾ-ರಿಯ
ರಾಯ್
ರಾಯ್ಸ್
ರಾರ್ಥ-ವಾಗಿ
ರಾಲ್ಫ್
ರಾಲ್ಫ್ನಿಗೆ
ರಾವಲ್
ರಾವ-ಲ್ಪಿಂಡಿ
ರಾವ-ಲ್ಪಿಂ-ಡಿಗೆ
ರಾವ-ಲ್ಪಿಂ-ಡಿ-ಯಲ್ಲಿ
ರಾಶಿ
ರಾಶಿ-ಯ-ಡಿ-ಯಿಂದ
ರಾಶಿ-ಯನ್ನು
ರಾಶಿ-ಯನ್ನೇ
ರಾಶಿ-ರಾ-ಶಿ-ಯಾಗಿ
ರಾಷ್ಟ್ರ
ರಾಷ್ಟ್ರ-ಕುಂ-ಡ-ಲಿನಿ
ರಾಷ್ಟ್ರಕ್ಕೂ
ರಾಷ್ಟ್ರಕ್ಕೆ
ರಾಷ್ಟ್ರ-ಗಳ
ರಾಷ್ಟ್ರ-ಗ-ಳಂತೆ
ರಾಷ್ಟ್ರ-ಗಳನ್ನು
ರಾಷ್ಟ್ರ-ಗಳಲ್ಲಿ
ರಾಷ್ಟ್ರ-ಗ-ಳ-ಲ್ಲಿನ
ರಾಷ್ಟ್ರ-ಗ-ಳಲ್ಲೂ
ರಾಷ್ಟ್ರ-ಗ-ಳಾದ
ರಾಷ್ಟ್ರ-ಗಳಿಂದ
ರಾಷ್ಟ್ರ-ಗ-ಳಿಗೆ
ರಾಷ್ಟ್ರ-ಗಳು
ರಾಷ್ಟ್ರ-ಗಳೇ
ರಾಷ್ಟ್ರ-ಜೀ-ವ-ನದ
ರಾಷ್ಟ್ರದ
ರಾಷ್ಟ್ರ-ದಲ್ಲಿ
ರಾಷ್ಟ್ರ-ದ-ಲ್ಲಿಯೂ
ರಾಷ್ಟ್ರ-ದಲ್ಲೇ
ರಾಷ್ಟ್ರ-ದ-ವ-ರನ್ನು
ರಾಷ್ಟ್ರ-ದ-ವ-ರ-ನ್ನುಈ
ರಾಷ್ಟ್ರ-ದಿಂ-ದಲೇ
ರಾಷ್ಟ್ರ-ಧ-ಮ-ನಿ-ಗಳಲ್ಲಿ
ರಾಷ್ಟ್ರ-ಧ-ರ್ಮ-ವಾ-ಗಲು
ರಾಷ್ಟ್ರ-ಧ್ವ-ಜ-ವನ್ನು
ರಾಷ್ಟ್ರ-ನಿ-ರ್ಮಾ-ಣಕ್ಕೆ
ರಾಷ್ಟ್ರ-ನಿ-ರ್ಮಾ-ಣ-ಗಳ
ರಾಷ್ಟ್ರ-ಪ್ರಜ್ಞೆ
ರಾಷ್ಟ್ರ-ಪ್ರ-ಜ್ಞೆ-ಯ-ನ್ನುಂ-ಟು-ಮಾ-ಡಿ-ದರು
ರಾಷ್ಟ್ರ-ಪ್ರೇಮ
ರಾಷ್ಟ್ರ-ಪ್ರೇ-ಮದ
ರಾಷ್ಟ್ರ-ಪ್ರೇ-ಮ-ವನ್ನು
ರಾಷ್ಟ್ರ-ಪ್ರೇ-ಮ-ವೆಂದರೆ
ರಾಷ್ಟ್ರ-ಪ್ರೇ-ಮವೇ
ರಾಷ್ಟ್ರ-ಪ್ರೇ-ಮಿ-ಯಾಗಿ
ರಾಷ್ಟ್ರ-ಪ್ರೇ-ಮಿಯೂ
ರಾಷ್ಟ್ರ-ಭ-ಕ್ತರೂ
ರಾಷ್ಟ್ರ-ಭ-ಕ್ತಿ-ಯನ್ನು
ರಾಷ್ಟ್ರ-ರಾ-ಷ್ಟ್ರ-ಗ-ಳ-ಲ್ಲಿ-ಮ-ತ-ಪಂ-ಥ-ಗಳಲ್ಲಿ
ರಾಷ್ಟ್ರ-ವನ್ನು
ರಾಷ್ಟ್ರ-ವನ್ನೂ
ರಾಷ್ಟ್ರ-ವನ್ನೇ
ರಾಷ್ಟ್ರ-ವಾ-ದ್ದ-ರಿಂದ
ರಾಷ್ಟ್ರವು
ರಾಷ್ಟ್ರ-ವೆಂದರೆ
ರಾಷ್ಟ್ರವೇ
ರಾಷ್ಟ್ರ-ವೊಂ-ದಿ-ದ್ದರೆ
ರಾಷ್ಟ್ರ-ಹಿ-ತ-ಕ್ಕಾಗಿ
ರಾಷ್ಟ್ರಾ-ಭಿ-ಮಾ-ನಕ್ಕೆ
ರಾಷ್ಟ್ರಾ-ಭಿ-ಮಾ-ನ-ವನ್ನು
ರಾಷ್ಟ್ರೀಯ
ರಾಷ್ಟ್ರೀ-ಯತೆ
ರಾಸ-ಮ-ಣಿಯ
ರಿಂದ
ರಿಂದಲೇ
ರಿಂದಲೋ
ರಿಂದಾಗಿ
ರಿಂದಾ-ಗುವ
ರಿಂದೇನು
ರಿಕ್
ರಿಗೂ
ರಿಗೆ
ರಿಗೆಲ್ಲ
ರಿಚ್ಮಂ-ಡ್ನಿಗೆ
ರಿಡ್ಜ್
ರಿಡ್ಜ್ಲಿ
ರಿಡ್ಜ್ಲಿಯ
ರಿಡ್ಜ್ಲಿ-ಯಿಂದ
ರಿದ್ದು-ದ-ರಿಂದ
ರಿಪೇರಿ
ರಿಪ್ಪನ್
ರಿಲಿ-ಜನ್
ರಿಲಿ-ಜ-ನ್ಸ್
ರಿಲಿ-ಜನ್ಸ್ನ
ರಿಸ್
ರೀತಿ
ರೀತಿ-ಗಳು
ರೀತಿ-ನೀತಿ
ರೀತಿ-ನೀ-ತಿ-ಗಳ
ರೀತಿ-ನೀ-ತಿ-ಗಳನ್ನು
ರೀತಿ-ನೀ-ತಿ-ಗ-ಳಿಗೆ
ರೀತಿ-ನೀ-ತಿ-ಗಳು
ರೀತಿಯ
ರೀತಿ-ಯ-ದಾ-ಗಿತ್ತು
ರೀತಿ-ಯನ್ನು
ರೀತಿ-ಯಲ್ಲ
ರೀತಿ-ಯ-ಲ್ಲಾ-ದರೂ
ರೀತಿ-ಯಲ್ಲಿ
ರೀತಿ-ಯಲ್ಲೂ
ರೀತಿ-ಯ-ಲ್ಲೆಲ್ಲ
ರೀತಿ-ಯಲ್ಲೇ
ರೀತಿ-ಯ-ಲ್ಲೇ-ಜ-ನ-ಸಾ-ಮಾ-ನ್ಯರ
ರೀತಿ-ಯಾಗಿ
ರೀತಿ-ಯಾ-ಗಿತ್ತು
ರೀತಿ-ಯಿಂದ
ರೀತಿಯು
ರೀತ್ಯಾ
ರೀವ್ಸ್
ರುಂಡ-ಮಾ-ಲೆ-ಯಿಂದ
ರುಂಡ-ವಿ-ಲ್ಲದ
ರುಚಿ
ರುಚಿ-ಕ-ರ-ವಾದ
ರುಚಿ-ಯಾ-ಗಿ-ರು-ವು-ದಿಲ್ಲ
ರುಚಿ-ಸ-ಲಿಲ್ಲ
ರುಜಿನ
ರುತ್ತಿ-ದ್ದಂ-ತೆಯೇ
ರುತ್ತಿ-ದ್ದರು
ರುದ್ರ
ರುದ್ರ-ಗಂ-ಭೀರ
ರುದ್ರ-ಮುಖೀ
ರುದ್ರ-ರೂ-ಪಿ-ಣಿ-ಯಾ-ದ-ರು-ದ್ರಾ-ಣಿ-ಯಾದ
ರುದ್ರಾಕ್ಷಿ
ರುಬ್ಯಾ-ಟಿನೋ
ರುಮಾ-ಲಿ-ನಿಂದ
ರುವ
ರುವುದು
ರೂ
ರೂಢ-ಮೂ-ಲ-ವಾ-ಗಿ-ರುವ
ರೂಢಿ
ರೂಢಿ-ಯ-ಲ್ಲಿ-ರುವ
ರೂಪ
ರೂಪಕ್ಕೆ
ರೂಪ-ಗಳ
ರೂಪ-ಗಳನ್ನು
ರೂಪ-ಗಳಲ್ಲಿ
ರೂಪ-ಗಳಿಂದ
ರೂಪ-ಗ-ಳೆಂದು
ರೂಪ-ಗಳೇ
ರೂಪ-ತಾ-ಳಿತ್ತು
ರೂಪ-ತಾ-ಳಿದ
ರೂಪ-ತಾಳು
ರೂಪ-ದಲ್ಲಿ
ರೂಪ-ರೇ-ಷೆ-ಗಳು
ರೂಪ-ವನ್ನು
ರೂಪ-ವಾಗಿ
ರೂಪ-ವೊಂ-ದನ್ನು
ರೂಪಾಯಿ
ರೂಪಾ-ಯಿ-ಗಳ
ರೂಪಾ-ಯಿ-ಗಳನ್ನು
ರೂಪಾ-ಯಿ-ಗ-ಳಿಂ-ದಲೇ
ರೂಪಾ-ಯಿಗೂ
ರೂಪಾ-ಯಿಗೆ
ರೂಪಾ-ಯಿ-ದೊ-ರೆ-ತರೆ
ರೂಪಾ-ಯಿಯ
ರೂಪಾ-ಯಿ-ಯನ್ನು
ರೂಪಾ-ಯಿ-ಯಾ-ದೀತು
ರೂಪಿ-ಯಾದ
ರೂಪಿಸ
ರೂಪಿ-ಸ-ಬಲ್ಲ
ರೂಪಿ-ಸ-ಬೇಕು
ರೂಪಿ-ಸ-ಲ್ಪ-ಟ್ಟಿದೆ
ರೂಪಿ-ಸ-ಲ್ಪ-ಟ್ಟಿ-ವೆಯೆ
ರೂಪಿಸಿ
ರೂಪಿ-ಸಿ-ಕೊ-ಳ್ಳ-ಬೇಕು
ರೂಪಿ-ಸಿ-ಕೊ-ಳ್ಳಲು
ರೂಪಿ-ಸಿ-ಕೊ-ಳ್ಳು-ವಂ-ತಿ-ರ-ಬೇಕು
ರೂಪಿ-ಸಿದ
ರೂಪಿ-ಸಿ-ದರು
ರೂಪಿ-ಸಿದ್ದ
ರೂಪಿ-ಸಿ-ದ್ದರು
ರೂಪಿ-ಸಿದ್ದು
ರೂಪಿ-ಸಿ-ರು-ವು-ದಾಗಿ
ರೂಪಿ-ಸುವ
ರೂಪಿ-ಸು-ವಲ್ಲಿ
ರೂಪಿ-ಸು-ವುದು
ರೂಪೀ
ರೂಪು-ಗೊಂಡ
ರೂಪು-ಗೊ-ಳ್ಳು-ತ್ತಲೇ
ರೂಪು-ರೇ-ಷೆ-ಯನ್ನು
ರೂಪೆ
ರೆಂದರೆ
ರೆಂದು
ರೆಂಬು-ದನ್ನು
ರೆಕ್ಕೆಯ
ರೆಗೂ
ರೆಜೆಂಟ್
ರೆಡ್
ರೆಡ್ವುಡ್
ರೆನ್ನಿ-ಸಿ-ಕೊಂ-ಡ-ವರ
ರೆನ್ನಿ-ಸಿ-ಕೊ-ಳ್ಳು-ತ್ತಾರೆ
ರೆಲ-ವೆಂಟೆ
ರೆಲ್ಲ
ರೆಲ್ಲರ
ರೆಲ್ಲ-ರಿಗೂ
ರೆವ-ರೆಂಡ್
ರೇಖೆ-ಗ-ಳ-ನ್ನೆ-ಳೆ-ದ-ರು-ಅ-ಲೆ-ಗಳು
ರೇಗಾಡಿ
ರೇಗಿ
ರೇಗಿತು
ರೇಗಿ-ಹೋ-ಯಿತು
ರೇನೋ
ರೇಶ್ಮೆಯ
ರೈಟ್
ರೈಟ್ಜಿ
ರೈಲಿ-ನಲ್ಲಿ
ರೈಲಿ-ನಲ್ಲೂ
ರೈಲಿ-ನಿಂದ
ರೈಲಿ-ನಿಂ-ದಿ-ಳಿ-ಯು-ತ್ತಿ-ದ್ದಂತೆ
ರೈಲು
ರೈಲು-ನಿ-ಲ್ದಾ-ಣದ
ರೈಲು-ನಿ-ಲ್ದಾ-ಣ-ದಲ್ಲಿ
ರೈಲು-ನಿ-ಲ್ದಾ-ಣ-ದಿಂದ
ರೈಲು-ಬಂಡಿ
ರೈಲು-ಮಾ-ರ್ಗ-ದಲ್ಲಿ
ರೊಂದಿ-ಗಿನ
ರೊಂದಿಗೆ
ರೊಂದು
ರೊಚ್ಚಿ-ಗೆ-ದ್ದರೂ
ರೊಟ್ಟಿ-ಯನ್ನು
ರೊಡನೆ
ರೊಬ್ಬರ
ರೊಬ್ಬರೇ
ರೊಮ್ಮೆ
ರೋಗ
ರೋಗ-ಸಂ-ಕ-ಟ-ಗಳ
ರೋಗ-ಗಳ
ರೋಗ-ಗ್ರಸ್ತ
ರೋಗ-ಗ್ರ-ಸ್ತ-ರಾ-ಗಿ-ದ್ದರು
ರೋಗದ
ರೋಗ-ದಂತೆ
ರೋಗ-ಪ-ರಿ-ಹಾ-ರ-ಕ್ಕಾಗಿ
ರೋಗ-ಪೀ-ಡಿ-ತ-ರಾಗಿ
ರೋಗ-ರು-ಜಿ-ನ-ಗಳ
ರೋಗ-ರು-ಜಿ-ನ-ಗಳನ್ನು
ರೋಗ-ರು-ಜಿ-ನ-ಗಳು
ರೋಗ-ವನ್ನು
ರೋಗ-ವಾದ
ರೋಗ-ವಿ-ರು-ವ-ವ-ರೆಗೂ
ರೋಗವೂ
ರೋಗಿ
ರೋಗಿ-ಗಳ
ರೋಗಿ-ಗ-ಳಿಗೆ
ರೋಗಿ-ಗಳು
ರೋಗಿಯ
ರೋಗಿ-ಯನ್ನು
ರೋಗಿ-ಯೊ-ಬ್ಬ-ನನ್ನು
ರೋಗಿ-ಯೊ-ಬ್ಬ-ನಿಂದ
ರೋಚಕ
ರೋಡ್ಹ್ಯಾ-ಮೆಲ್
ರೋದನ
ರೋದ-ನ-ವನ್ನು
ರೋದಿ-ಸ-ಲಾ-ರಂ-ಭಿ-ಸಿ-ದರು
ರೋಧದ
ರೋಮ-ಗಳು
ರೋಮನ್
ರೋಮ-ನ್ನರ
ರೋಮಾಂ-ಚ-ಕರ
ರೋಮಾಂ-ಚ-ಕ-ರ-ವಾಗಿ
ರೋಮಾಂ-ಚ-ಕಾ-ರಕ
ರೋಮಾಂ-ಚ-ಕಾರಿ
ರೋಮಾಂ-ಚ-ಕಾ-ರಿ-ಯಾ-ಗಿತ್ತು
ರೋಮಾಂ-ಚ-ಕಾ-ರಿ-ಯಾದ
ರೋಮಾಂ-ಚ-ನ-ವ-ನ್ನುಂ-ಟು-ಮಾ-ಡಿತು
ರೋಮಾಂ-ಚಿತ
ರೋಮ್
ರೋಮ್ಗೆ
ರೋಮ್ನ
ರೋಮ್ನ-ಲ್ಲಿ-ಎಂ-ದರೆ
ರೋಮ್ನಿಂದ
ರೋಷಾ-ವೇಶ
ರೋಸಿ-ಹೋಗಿ
ರೋಸಿ-ಹೋ-ಗಿತ್ತು
ರೋಸಿ-ಹೋ-ಗಿ-ದ್ದಾರೆ
ರೌದ್ರ
ರೌದ್ರ-ರೂ-ಪದ
ರೌದ್ರ-ವನ್ನು
ರೌದ್ರವೂ
ರೌರವ
ರ್ಹೆಟ
ಲಂಗರು
ಲಂಗಲ್
ಲಂಚ-ಪ್ರ-ಲೋ-ಭ-ನೆಯ
ಲಂಡ-ನ್ನಿಗೆ
ಲಂಡ-ನ್ನಿನ
ಲಂಡ-ನ್ನಿ-ನಂ-ತೆಯೇ
ಲಂಡ-ನ್ನಿ-ನಲ್ಲಿ
ಲಂಡ-ನ್ನಿ-ನ-ಲ್ಲಿದ್ದ
ಲಂಡ-ನ್ನಿ-ನಿಂದ
ಲಕ
ಲಕೋ-ಟೆ-ಯನ್ನು
ಲಕ್ನೋ
ಲಕ್ನೋದ
ಲಕ್ಷ
ಲಕ್ಷಕ್ಕೂ
ಲಕ್ಷ-ಗ-ಟ್ಟಲೆ
ಲಕ್ಷಣ
ಲಕ್ಷ-ಣ-ಗಳನ್ನು
ಲಕ್ಷ-ಣ-ಗಳನ್ನೆಲ್ಲ
ಲಕ್ಷ-ಣ-ಗಳು
ಲಕ್ಷ-ಣ-ಗಳೇ
ಲಕ್ಷ-ಣದ
ಲಕ್ಷ-ಣ-ವಾದ
ಲಕ್ಷ-ಣ-ವೇನು
ಲಕ್ಷ-ದ-ಲ್ಲೊಂದು
ಲಕ್ಷ-ಮಂದಿ
ಲಕ್ಷ-ಲಕ್ಷ
ಲಕ್ಷಾಂ
ಲಕ್ಷಾಂ-ತರ
ಲಕ್ಷಿ-ಸದೆ
ಲಕ್ಷಿ-ಸ-ಬೇ-ಕಾ-ಗಿಲ್ಲ
ಲಕ್ಷಿ-ಸು-ತ್ತೇ-ನೆಯೆ
ಲಕ್ಷಿ-ಸು-ವ-ವ-ನಲ್ಲ
ಲಕ್ಷೋ-ಪ-ಲಕ್ಷ
ಲಕ್ಷ್ಮಿ
ಲಕ್ಷ್ಮೀಃ
ಲಕ್ಷ್ಮೀ-ನ-ರಸು
ಲಕ್ಷ್ಮೀ-ಪೂ-ಜೆ-ಯನ್ನೂ
ಲಕ್ಷ್ಯ-ಕೊ-ಡದೆ
ಲಕ್ಷ್ಯಕ್ಕೆ
ಲಕ್ಷ್ಯವೇ
ಲಘು
ಲಘು-ವಾಗಿ
ಲಜ್ಜಾ-ಪ-ಟಾ-ವೃ-ತೆ-ಯ-ಲ್ಲವೆ
ಲಭಿ-ಸಿತ್ತು
ಲಭಿ-ಸಿಯೇ
ಲಭಿ-ಸು-ತ್ತದೆ
ಲಭಿ-ಸು-ತ್ತವೆ
ಲಭಿ-ಸು-ವಂತಾ
ಲಭ್ಯ-ವಾ-ಗ-ಲಿದೆ
ಲಭ್ಯ-ವಾ-ಗಿತ್ತು
ಲಭ್ಯ-ವಾ-ಗಿದೆ
ಲಭ್ಯ-ವಾ-ಗಿವೆ
ಲಭ್ಯ-ವಾ-ಗು-ತ್ತದೆ
ಲಭ್ಯ-ವಾ-ದೊ-ಡ-ನೆಯೇ
ಲಭ್ಯ-ವಾ-ಯಿತು
ಲಭ್ಯ-ವಿದೆ
ಲಭ್ಯ-ವಿದ್ದ
ಲಭ್ಯ-ವಿಲ್ಲ
ಲಯ-ಗಳನ್ನು
ಲಯ-ಗಳನ್ನೂ
ಲಯ-ಗೊಂಡು
ಲಯದ
ಲಯ-ದಿಂದ
ಲಯನ್
ಲವ-ಲ-ವಿ-ಕೆಯ
ಲವ-ಲೇ-ಶ-ವಾ-ದರೂ
ಲವ-ಲೇ-ಶವೂ
ಲವ್
ಲಹ-ರಿ-ಯನ್ನು
ಲಹ-ರಿ-ಯಲ್ಲಿ
ಲಾ
ಲಾಂಛನ
ಲಾಂಛ-ನ-ವನ್ನು
ಲಾಂಛ-ನ-ವೊಂ-ದನ್ನು
ಲಾಂದ್ರ
ಲಾಗದ
ಲಾಗ-ದಿ-ದ್ದರೆ
ಲಾಗಾ-ಯ್ತಿ-ನಿಂದ
ಲಾಗಿ
ಲಾಗಿತ್ತು
ಲಾಗಿದೆ
ಲಾಗಿದ್ದ
ಲಾಗಿದ್ದು
ಲಾಗು-ತ್ತಿತ್ತು
ಲಾಟೀನು
ಲಾಡಿ-ಸಲು
ಲಾಡ್
ಲಾಡ್ಜ್
ಲಾಡ್ನ
ಲಾದ
ಲಾಭ
ಲಾಭ-ವಾ-ದಂತೆ
ಲಾಭ-ವಾ-ದರೂ
ಲಾಭ-ವಾ-ಯಿತು
ಲಾಭ-ವಿಲ್ಲ
ಲಾಭ-ವೇನು
ಲಾಯಕ್ಖು
ಲಾಯಿತು
ಲಾರಂ-ಭಿ-ಸಿತ್ತು
ಲಾರಂ-ಭಿ-ಸಿ-ದರು
ಲಾರಂ-ಭಿ-ಸಿದೆ
ಲಾರದ
ಲಾರದು
ಲಾರದೆ
ಲಾರವು
ಲಾರ-ವೇನು
ಲಾರೆ
ಲಾರೆ-ನ-ನ್ನನ್ನು
ಲಾರ್ಡ್
ಲಾಲ-ಸೆ-ಯನ್ನು
ಲಾಲಾ
ಲಾಲ್
ಲಾಸ್
ಲಾಸ್-ಎಂ-ಜ-ಲಿಸ್
ಲಾಸ್-ಏಂ-ಜ-ಲಿಸ್ನ
ಲಾಹೋ
ಲಾಹೋ-ರಿಗೆ
ಲಾಹೋ-ರಿನ
ಲಾಹೋ-ರಿ-ನಲ್ಲಿ
ಲಾಹೋ-ರಿ-ನ-ಲ್ಲಿ-ದ್ದದ್ದು
ಲಾಹೋ-ರಿ-ನ-ಲ್ಲಿ-ದ್ದಾಗ
ಲಾಹೋ-ರಿ-ನಿಂದ
ಲಾಹೋರ್
ಲಾಹೋ-ರ್ಗಳ
ಲಾಹೋ-ರ್ನ-ಗ-ರದ
ಲಿಂಬ್ಡಿಯ
ಲಿಟ್ಲ್
ಲಿಡ್ಡರ್
ಲಿದ್ದ-ವನು
ಲಿನ
ಲಿಪಿ-ಯನ್ನೂ
ಲಿಯ
ಲಿಯನ್
ಲಿಯಾನ್
ಲಿಲ್ಲ
ಲಿಲ್ಲವೆ
ಲಿಲ್ಲ-ವೆಂ-ಬು-ದನ್ನು
ಲಿಲ್ಲವೋ
ಲಿಲ್ಲಿ
ಲಿವರ್
ಲೀನ-ಗೊ-ಳಿ-ಸು-ವು-ದ-ಕ್ಕಾಗಿ
ಲೀನ-ನಾ-ಗಿ-ಬಿ-ಡು-ತ್ತಾನೆ
ಲೀನ-ರಾ-ಗಲು
ಲೀನ-ರಾ-ಗಿ-ದ್ದರು
ಲೀನ-ರಾ-ಗಿ-ರು-ವಂತೆ
ಲೀನ-ರಾ-ಗು-ವುದನ್ನು
ಲೀನ-ರಾ-ದರು
ಲೀನ-ವಾ-ಗು-ವ-ವ-ರೆಗೆ
ಲೀನ-ವಾ-ಗು-ವುವು
ಲೀಲಾ
ಲೀಲಾ-ಜಾಲ
ಲೀಲಾ-ಜಾ-ಲ-ವಾಗಿ
ಲೀಲಾ-ನಾ-ಟ-ಕ-ದಲ್ಲಿ
ಲೀಲಾ-ವತಿ
ಲೀಲಾ-ಸ-ಹ-ಚ-ರ-ನಾದ
ಲೀಲೆ-ಯನ್ನು
ಲೀಲೆ-ಯಲ್ಲಿ
ಲೀಲೆ-ಯಾ-ಗಿ-ದೆಯೋ
ಲೀಲೆ-ಯಾ-ಡುವ
ಲುಪ್ತ-ವಾ-ಗಿತ್ತು
ಲುಯ್ಟ್ಪೋ-ಲ್ಡ್
ಲೂಟಿ
ಲೂಧಿ-ಯಾನ
ಲೂಯಿ
ಲೂಯಿಸ್
ಲೆಂದು
ಲೆಕ್ಕ
ಲೆಕ್ಕಕ್ಕೆ
ಲೆಕ್ಕ-ವಿ-ಲ್ಲ-ದಷ್ಟು
ಲೆಕ್ಕವೇ
ಲೆಕ್ಕ-ಹಾಕಿ
ಲೆಕ್ಕಾ-ಚಾರ
ಲೆಕ್ಕಾ-ಚಾ-ರ-ಗ-ಳಿಂ-ದಾಗಿ
ಲೆಕ್ಕಾ-ಚಾ-ರದ
ಲೆಕ್ಕಿ-ಸದೆ
ಲೆಕ್ಕಿ-ಸ-ಲಿಲ್ಲ
ಲೆಕ್ಕಿ-ಸಲೇ
ಲೆಕ್ಕಿ-ಸು-ವಂ-ತಿಲ್ಲ
ಲೆಕ್ಕಿ-ಸು-ವ-ವ-ನಲ್ಲ
ಲೆಕ್ಕಿ-ಸು-ವ-ವ-ರಲ್ಲ
ಲೆಗಟ್
ಲೆಗೆಟ್
ಲೆಗೆ-ಟ್ಟರ
ಲೆಗೆ-ಟ್ಳಿಗೆ
ಲೇಖಕ
ಲೇಖ-ಕ-ನಿಗೆ
ಲೇಖ-ಕರು
ಲೇಖನ
ಲೇಖ-ನ-ಗಳ
ಲೇಖ-ನ-ಗಳನ್ನು
ಲೇಖ-ನ-ಗ-ಳನ್ನೋ
ಲೇಖ-ನ-ಗಳಿಂದ
ಲೇಖ-ನ-ಗಳು
ಲೇಖ-ನ-ರೂ-ಪ-ದಲ್ಲಿ
ಲೇಖ-ನ-ವನ್ನು
ಲೇಖನಿ
ಲೇಖ-ನಿಯ
ಲೇಖ-ನಿ-ಯಿಂದ
ಲೇನಾರ್ದೊ
ಲೇನಿ-ನಲ್ಲಿ
ಲೇಪ-ವನ್ನು
ಲೇಪಿಸಿ
ಲೇವಡಿ
ಲೇಶವೂ
ಲೇಸು
ಲೊಚ-ಗು-ಟ್ಟಿತು
ಲೊಡಿಯಾ
ಲೊಹರು
ಲೋಕ
ಲೋಕ-ಕ-ಲ್ಯಾ-ಣ-ಕಾ-ರ್ಯದ
ಲೋಕ-ಕ-ಲ್ಯಾ-ಣಾ-ರ್ಥ-ವಾಗಿ
ಲೋಕಕ್ಕೆ
ಲೋಕ-ಕ್ಕೆ-ತಮ್ಮ
ಲೋಕ-ಗು-ರು-ಗ-ಳಾದ್ದ
ಲೋಕದ
ಲೋಕ-ದಲ್ಲಿ
ಲೋಕ-ದಾ-ದ್ಯಂತ
ಲೋಕ-ದೊ-ಳಕ್ಕೆ
ಲೋಕ-ವಾಗಿ
ಲೋಕ-ಸೇ-ವೆಗೆ
ಲೋಕ-ಸೇ-ವೆ-ಯಲ್ಲಿ
ಲೋಕ-ಹಿತ
ಲೋಕ-ಹಿ-ತದ
ಲೋಕ-ಹಿ-ತ-ವನ್ನು
ಲೋಕ-ಹಿ-ತಾ-ರ್ಥ-ವಾದ
ಲೋಕಾ-ನು-ಭ-ವ-ದಿಂದ
ಲೋಕಾ-ಭಿ-ರಾ-ಮ-ವಾಗಿ
ಲೋಕೋ-ಪ-ಕಾ-ರವು
ಲೋಕೋ-ಪ-ಕಾರಿ
ಲೋಕ್
ಲೋಗನ್
ಲೋಗ-ನ್ನರ
ಲೋಚಿ-ಸಿದ
ಲೋಪ
ಲೋಪ-ದೋಷ
ಲೋಪ-ದೋ-ಷ-ಗಳನ್ನು
ಲೋಪ-ವನ್ನು
ಲೋಪ-ವಾ-ಗ-ದಂತೆ
ಲೋಫ್
ಲೋವ್ಗೆ
ಲೌಕಿಕ
ಲೌಕಿ-ಕ-ಇ-ವು-ಗಳ
ಲೌಕಿ-ಕ-ದಲ್ಲಿ
ಲೌಕಿ-ಕ-ವಾ-ದು-ದಲ್ಲ
ಲ್ದಾಕ್ಕೆ
ಲ್ದಾರ
ಲ್ಪಟ್ಟಿದೆ
ಲ್ಪಟ್ಟು
ಲ್ಪಡು-ತ್ತಾನೆ
ಲ್ಯಾಂಡಿ-ನಲ್ಲಿ
ಲ್ಯಾಂಡ್
ಲ್ಯಾಂಡ್ಸ್ಬ-ರ್ಗ್
ಲ್ಲದೆ
ಲ್ಲಲ್ಲ
ಲ್ಲವಳು
ಲ್ಲವೆ
ಲ್ಲಾದರೂ
ಲ್ಲಿಂದ
ಲ್ಲಿಗೆ
ಲ್ಲಿಟ್ಟು-ಕೊಂಡು
ಲ್ಲಿದ್ದ
ಲ್ಲಿದ್ದಾ-ಗಲೂ
ಲ್ಲಿನ
ಲ್ಲಿಯೂ
ಲ್ಲಿರ-ಲಿಲ್ಲ
ಲ್ಲಿರುವ
ಲ್ಲೆಲ್ಲ
ಲ್ಲೊಂದಾ-ಗಿತ್ತು
ಲ್ಲೊಂದು
ಲ್ಲೊಂದೆಂ-ದರೆ
ಲ್ಲೊಬ್ಬ-ನನ್ನು
ಲ್ಲೊಬ್ಬರು
ಲ್ಲೊಬ್ಬರೂ
ಳಲ್ಲ
ಳಾದ
ಳಿದ್ದ
ಳೆಂದರೆ
ಳೊಂದಿಗೆ
ಳ್ನಕ್ಕರು
ಳ್ನಗುತ್ತ
ವಂಗ-ನಾ-ಡಿನ
ವಂಗ-ಬಂ-ಧು-ಗಳ
ವಂಗ-ಮಾ-ತೆಯ
ವಂಗ-ವಾಸಿ
ವಂಗ-ವಾ-ಸಿಗೆ
ವಂಗ-ವಾ-ಸಿಯ
ವಂಗ-ವಾ-ಸಿ-ಯಂ-ತಹ
ವಂಗ-ವಾ-ಸಿ-ಯನ್ನು
ವಂಗ-ವಾ-ಸಿಯು
ವಂಗ-ವಾ-ಸಿಯೇ
ವಂಚಿತ
ವಂಚಿ-ಸಲು
ವಂತ
ವಂತ-ನಲ್ಲಿ
ವಂತ-ನಾ-ಗಿ-ರ-ಬಲ್ಲೆ
ವಂತನೇ
ವಂತರು
ವಂತ-ರು-ಹಿ-ರಿ-ಯರು
ವಂತಹ
ವಂತಾ-ದದ್ದು
ವಂತಿಗೆ
ವಂತಿತ್ತು
ವಂತಿದೆ
ವಂತಿದ್ದ
ವಂತಿ-ರ-ಬೇಕು
ವಂತಿಲ್ಲ
ವಂತೆ
ವಂತೆ-ತಪ್ಪು
ವಂಥ-ದುಈ
ವಂಥ-ದೊಂದು
ವಂಥ-ವನು
ವಂದ-ನೆ-ಗಳು
ವಂದಿ
ವಂದ್ಯೋ-ಪಾ-ಧ್ಯಾಯ
ವಂದ್ಯೋ-ಪಾ-ಧ್ಯಾ-ಯನ
ವಂದ್ಯೋ-ಪಾ-ಧ್ಯಾ-ಯ-ನಂ-ಥ-ವರು
ವಂಶ-ದ-ವಳು
ವಂಶ-ಪಾ-ರಂ-ಪ-ರ್ಯ-ವಾಗಿ
ವಂಶ-ಸ್ಥ-ನೊಬ್ಬ
ವಂಶ-ಸ್ಥರೇ
ವಕ-ವಾಗಿ
ವಕೀ-ಲರ
ವಕೀ-ಲ-ರಾದ
ವಕೀ-ಲರು
ವಕ್ರ-ಮಾರ್ಗ
ವಕ್ರ-ವ್ಯಾ-ಖ್ಯಾ-ನ-ಗಳನ್ನು
ವಗೈರೆ
ವಚನ
ವಚ-ನಾ-ಮೃ-ತ-ವ-ನ್ನಾ-ಲಿ-ಸಲು
ವಜ್ರ-ಕಾ-ಯದ
ವಜ್ರದ
ವಜ್ರಾ-ಯು-ಧ-ದಿಂದ
ವಟ-ಗು-ಟ್ಟು-ತ್ತಲೇ
ವಟ-ವೃಕ್ಷ
ವಟ-ವೃ-ಕ್ಷ-ವಾ-ಗ-ಬೇಕು
ವಠಾ-ರ-ದೊ-ಳಗೆ
ವಣ
ವಣಿ-ಗೆ-ಯಲ್ಲಿ
ವಣಿ-ಗೆಯೇ
ವಣೆ-ಯನ್ನೂ
ವತಿಗೆ
ವತಿ-ಯಿಂದ
ವತ್
ವದಂತಿ
ವದಂ-ತಿ-ಗಳನ್ನು
ವದಂ-ತಿ-ಗಳು
ವನ
ವನ-ಗಳು
ವನದ
ವನ-ರಾಶಿ
ವನ-ರಾ-ಶಿ-ಯನ್ನು
ವನಲ್ಲ
ವನ-ವಿ-ಹಾ-ರ-ಕ್ಕೆಂದು
ವನ-ಸಿ-ರಿ-ಯಿಂದ
ವನ-ಸ್ಪ-ತಿ-ಗಳ
ವನು
ವನೋ
ವನ್ನಪ್ಪಿ
ವನ್ನಾಗಿ
ವನ್ನಿನ್ನೂ
ವನ್ನು
ವನ್ನುಂ-ಟು-ಮಾ-ಡಿತು
ವನ್ನುಂ-ಟು-ಮಾ-ಡುವ
ವನ್ನೂ
ವನ್ನೆಲ್ಲ
ವನ್ನೇ
ವನ್ಯ
ವಯ-ಸ್ಸನ್ನು
ವಯ-ಸ್ಸಿಗೆ
ವಯ-ಸ್ಸಿನ
ವಯ-ಸ್ಸಿ-ನಲ್ಲೇ
ವಯಸ್ಸು
ವಯೋ-ವೃ-ದ್ಧ-ರಷ್ಟೇ
ವಯೋ-ವೃ-ದ್ಧ-ರಾದ
ವರ
ವರಟೆ
ವರ-ದಾನ
ವರ-ದಾ-ಯಕ
ವರದಿ
ವರ-ದಿ-ಗಳ
ವರ-ದಿ-ಗಳನ್ನು
ವರ-ದಿ-ಗಳು
ವರ-ದಿ-ಯನ್ನು
ವರ-ದಿ-ಯಲ್ಲಿ
ವರ-ದಿ-ಯೊಂ-ದನ್ನು
ವರನ್ನು
ವರ-ಮಾನ
ವರವೇ
ವರಾಂ-ಡ-ಇ-ವಿಷ್ಟು
ವರಾಂ-ಡ-ದಲ್ಲಿ
ವರಾಗಿ
ವರಾನ್
ವರಿಗೂ
ವರಿದ
ವರಿ-ಷ್ಠನೂ
ವರಿ-ಸಿ-ಕೊಂಡು
ವರಿ-ಸಿ-ದಾಗ
ವರು
ವರೆ
ವರೆಗೂ
ವರೆಗೆ
ವರೇ
ವರೋ
ವರ್ಗ
ವರ್ಗಕ್ಕೆ
ವರ್ಗ-ಗಳ
ವರ್ಗ-ಗಳಲ್ಲಿ
ವರ್ಗ-ಗ-ಳಲ್ಲೂ
ವರ್ಗದ
ವರ್ಗ-ದ-ವರ
ವರ್ಗ-ದ-ವ-ರನ್ನು
ವರ್ಗ-ದ-ವ-ರಿಗೆ
ವರ್ಗ-ದ-ವ-ರೊಂ-ದಿಗೆ
ವರ್ಗವು
ವರ್ಗಾ-ಯಿ-ಸ-ಲಾ-ಯಿತು
ವರ್ಗಾ-ಯಿ-ಸಿದ್ದು
ವರ್ಚಸ್ಸು
ವರ್ಣ
ವರ್ಣ-ಚಿ-ತ್ರ-ಗಳಿಂದ
ವರ್ಣ-ಚಿ-ತ್ರ-ವನ್ನು
ವರ್ಣ-ಚಿ-ತ್ರವು
ವರ್ಣದ
ವರ್ಣ-ಧ-ರ್ಮವು
ವರ್ಣನಾ
ವರ್ಣ-ನಾ-ತೀತ
ವರ್ಣ-ನಾ-ತೀ-ತ-ವಾ-ಗಿತ್ತು
ವರ್ಣ-ನಾ-ತೀ-ತ-ವಾ-ದದ್ದು
ವರ್ಣನೆ
ವರ್ಣ-ನೆ-ಯನ್ನು
ವರ್ಣ-ನೆ-ಯಿದೆ
ವರ್ಣ-ಭೇ-ದ-ಕ್ಕಿಂತ
ವರ್ಣ-ಭೇ-ದವೇ
ವರ್ಣ-ಮಯ
ವರ್ಣ-ಮ-ಯ-ಗೊ-ಳಿ-ಸು-ತ್ತಿ-ದ್ದರು
ವರ್ಣಾ
ವರ್ಣಿ-ತ-ವಾ-ಗಿತ್ತು
ವರ್ಣಿ-ಸ-ಲ-ಸಾಧ್ಯ
ವರ್ಣಿ-ಸಲು
ವರ್ಣಿ-ಸಿ-ದರು
ವರ್ಣಿ-ಸು-ತ್ತಾನೆ
ವರ್ಣಿ-ಸು-ತ್ತಾರೆ
ವರ್ಣಿ-ಸು-ತ್ತಾಳೆ
ವರ್ಣಿ-ಸು-ತ್ತಿದ್ದ
ವರ್ಣಿ-ಸು-ತ್ತಿ-ದ್ದರು
ವರ್ಣಿ-ಸು-ತ್ತಿ-ದ್ದರೆ
ವರ್ಣ್ಯದ
ವರ್ತ-ಕರ
ವರ್ತ-ಕ-ರಾದ
ವರ್ತ-ಕರು
ವರ್ತನೆ
ವರ್ತ-ನೆ-ಗಳನ್ನು
ವರ್ತ-ನೆಗೆ
ವರ್ತ-ನೆ-ಯನ್ನು
ವರ್ತ-ನೆ-ಯಾ-ಗಲಿ
ವರ್ತ-ನೆ-ಯಿಂದ
ವರ್ತ-ನೆ-ಯಿಂ-ದಲೇ
ವರ್ತ-ನೆಯು
ವರ್ತ-ನೆ-ಯೊಂ-ದೇ-ಎಂ-ದರೆ
ವರ್ತ-ಮಾನ
ವರ್ತ-ಮಾ-ನದ
ವರ್ತ-ಮಾ-ನ-ದಿಂದ
ವರ್ತಿ-ಸ-ಬೇ-ಕೆಂ-ಬು-ದರ
ವರ್ತಿಸಿ
ವರ್ತಿ-ಸಿದ
ವರ್ತಿ-ಸಿ-ದರೂ
ವರ್ತಿ-ಸು-ತ್ತಾನೆ
ವರ್ತಿ-ಸು-ತ್ತಿತ್ತು
ವರ್ತಿ-ಸು-ತ್ತಿ-ದ್ದರು
ವರ್ತಿ-ಸು-ತ್ತಿದ್ದೆ
ವರ್ಧಿಸಿ
ವರ್ಧಿ-ಸು-ತ್ತ-ಲಿವೆ
ವರ್ಷ
ವರ್ಷ-ಕಾಲ
ವರ್ಷಕ್ಕೆ
ವರ್ಷ-ಗ-ಟ್ಟ-ಲೆಯ
ವರ್ಷ-ಗಳ
ವರ್ಷ-ಗ-ಳನ್ನೇ
ವರ್ಷ-ಗಳಲ್ಲಿ
ವರ್ಷ-ಗ-ಳಲ್ಲೇ
ವರ್ಷ-ಗ-ಳ-ವ-ರೆಗೆ
ವರ್ಷ-ಗ-ಳಷ್ಟು
ವರ್ಷ-ಗ-ಳಷ್ಟೆ
ವರ್ಷ-ಗಳಿಂದ
ವರ್ಷ-ಗ-ಳಿಂ-ದಲೂ
ವರ್ಷ-ಗ-ಳಿಗೂ
ವರ್ಷ-ಗ-ಳಿಗೆ
ವರ್ಷ-ಗಳು
ವರ್ಷ-ಗ-ಳೊ-ಳ-ಗಾಗಿ
ವರ್ಷ-ಗ-ಳೊ-ಳಗೆ
ವರ್ಷದ
ವರ್ಷ-ದಂತೆ
ವರ್ಷ-ದ-ವ-ರೆಗೂ
ವರ್ಷ-ದಿಂದ
ವರ್ಷ-ದಿಂ-ದಲೂ
ವರ್ಷ-ದೊ-ಳ-ಗಾಗಿ
ವರ್ಷ-ವನ್ನು
ವರ್ಷ-ವಾ-ದರೂ
ವರ್ಷ-ವಿಡೀ
ವರ್ಷವೂ
ವರ್ಷಾಂ-ತ-ರ-ಗಳ
ವರ್ಷಾ-ಕಾ-ಲದ
ವಲ-ಯಕ್ಕೆ
ವಲ-ಯ-ಗಳಲ್ಲಿ
ವಲ-ಯ-ದಲ್ಲೆಲ್ಲ
ವಲ-ಯ-ದೊ-ಳಗೆ
ವಲ್ಲ
ವಲ್ಲದ
ವಲ್ಲದೆ
ವಲ್ಲ-ಭಾ-ಚಾ-ರ್ಯರ
ವಲ್ಲಿ
ವವ-ನಲ್ಲ
ವವ-ರನ್ನೂ
ವವ-ರಲ್ಲ
ವವ-ರಿಗೂ
ವವ-ರು-ಹೀಗೆ
ವವ-ರೆಗೂ
ವವಳೂ
ವವೂ-ನಿಯ
ವಶ-ರಾ-ದರು
ವಶ್ಯಕ
ವಷ್ಟೇ
ವಸತಿ
ವಸ-ತಿ-ಗಾಗಿ
ವಸ-ತಿ-ಗೃ-ಹ-ವೊಂ-ದನ್ನು
ವಸ-ತಿಗೆ
ವಸ-ತಿ-ಯನ್ನು
ವಸ-ನ-ಧಾ-ರಿ-ಯಾಗಿ
ವಸ-ರ-ವಾಗಿ
ವಸಿ-ಷ್ಠ-ವಿ-ಶ್ವಾ-ಮಿ-ತ್ರರ
ವಸಿ-ಷ್ಠರ
ವಸಿ-ಷ್ಠರು
ವಸ್ತು
ವಸ್ತು-ಗಳನ್ನು
ವಸ್ತು-ಗ-ಳಾ-ಗಿದ್ದ
ವಸ್ತು-ಗಳು
ವಸ್ತು-ಗ-ಳು-ಎ-ರಡು
ವಸ್ತು-ಗಳೂ
ವಸ್ತು-ವನ್ನು
ವಸ್ತು-ವಾ-ಗ-ಬೇಕು
ವಸ್ತು-ವಾ-ಗಿತ್ತು
ವಸ್ತು-ವಾ-ಗಿ-ಬಿ-ಟ್ಟಿತ್ತು
ವಸ್ತು-ವಾದ
ವಸ್ತು-ವಿ-ಗಿಂತ
ವಸ್ತು-ವಿಗೆ
ವಸ್ತು-ವಿನ
ವಸ್ತು-ವಿ-ನಂತೆ
ವಸ್ತು-ವಿ-ನಿಂದ
ವಸ್ತು-ವಿ-ನೊಂ-ದಿಗೆ
ವಸ್ತುವೂ
ವಸ್ತುವೇ
ವಸ್ತು-ಸಂ-ಗ್ರಹಾ
ವಸ್ತು-ಸಂ-ಗ್ರ-ಹಾ-ಲಯ
ವಸ್ತು-ಸಂ-ಗ್ರ-ಹಾ-ಲ-ಯ-ವನ್ನು
ವಸ್ತ್ರ
ವಸ್ತ್ರ-ದೊ-ಳಗೆ
ವಸ್ತ್ರ-ಧ-ರಿಸಿ
ವಸ್ತ್ರ-ಧಾ-ರಿ-ಗ-ಳಾದ
ವಸ್ತ್ರ-ಧಾ-ರಿ-ಗ-ಳೊ-ಬ್ಬರು
ವಸ್ತ್ರ-ವನ್ನು
ವಹಾಂ
ವಹಿ-ಸ-ಬೇ-ಕಾ-ಗಿತ್ತು
ವಹಿ-ಸ-ಬೇಕು
ವಹಿ-ಸ-ಲಿ-ದ್ದರು
ವಹಿ-ಸಲು
ವಹಿಸಿ
ವಹಿ-ಸಿ-ಕೊಂ-ಡರು
ವಹಿ-ಸಿ-ಕೊಂ-ಡರೆ
ವಹಿ-ಸಿ-ಕೊಂ-ಡಳು
ವಹಿ-ಸಿ-ಕೊಂ-ಡ-ವರು
ವಹಿ-ಸಿ-ಕೊಟ್ಟು
ವಹಿ-ಸಿ-ಕೊ-ಳ್ಳ-ತೊ-ಡ-ಗಿ-ದರು
ವಹಿ-ಸಿ-ಕೊ-ಳ್ಳಲು
ವಹಿ-ಸಿ-ಕೊ-ಳ್ಳು-ತ್ತಿ-ದ್ದರು
ವಹಿ-ಸಿ-ಕೊ-ಳ್ಳು-ವು-ದಾಗಿ
ವಹಿ-ಸಿ-ದರು
ವಹಿ-ಸಿ-ದಳು
ವಹಿ-ಸಿ-ದ್ದಕ್ಕೆ
ವಹಿ-ಸಿ-ದ್ದರು
ವಹಿ-ಸುತ್ತ
ವಹಿ-ಸು-ತ್ತಿ-ದ್ದರು
ವಹಿ-ಸುವ
ವಹಿ-ಸು-ವಂತೆ
ವಾ
ವಾಂತಿ
ವಾಂತಿ-ಯಾ-ದದ್ದ
ವಾಕ್
ವಾಕ್ಚಾ-ತು-ರ್ಯ-ದಿಂದ
ವಾಕ್ಯ-ಗಳು
ವಾಕ್ಯವೇ
ವಾಕ್ಲ-ಹರಿ
ವಾಕ್ಸಾ-ಮ-ರ್ಥ್ಯ-ದಿಂದ
ವಾಕ್ಸಾ-ಮ-ರ್ಥ್ಯ-ವಾ-ಗಲಿ
ವಾಗ
ವಾಗ-ಬ-ಲ್ಲು-ದೆಂ-ಬು-ದನ್ನು
ವಾಗಲಿ
ವಾಗ-ಲಿ-ದ್ದುದು
ವಾಗ-ಲಿಲ್ಲ
ವಾಗಲೂ
ವಾಗಲೇ
ವಾಗಿ
ವಾಗಿತ್ತು
ವಾಗಿತ್ತೋ
ವಾಗಿದೆ
ವಾಗಿ-ದೆ-ಯೆಂದು
ವಾಗಿ-ದೆ-ಯೆಂ-ಬುದು
ವಾಗಿ-ದ್ದರೂ
ವಾಗಿದ್ದು
ವಾಗಿಯೂ
ವಾಗಿಯೇ
ವಾಗಿ-ಯೋ-ಅ-ವ-ರಿಂದ
ವಾಗಿ-ರ-ಬ-ಹುದು
ವಾಗಿ-ರ-ಬೇಕು
ವಾಗಿ-ರು-ತ್ತದೆ
ವಾಗಿ-ರು-ವಂತೆ
ವಾಗಿ-ರು-ವ-ವ-ರೆಗೆ
ವಾಗಿ-ರು-ವುದು
ವಾಗಿಲ್ಲ
ವಾಗು-ತ್ತದೆ
ವಾಗು-ತ್ತಲೇ
ವಾಗು-ವು-ದಿಲ್ಲ
ವಾಗ್ಝರಿ
ವಾಗ್ಝ-ರಿ-ಯಿಂದ
ವಾಗ್ದಾನ
ವಾಗ್ಬಾ-ಣ-ಗಳು
ವಾಗ್ಮಿ
ವಾಗ್ಮಿ-ತೆಯೂ
ವಾಗ್ಮಿ-ಯಾ-ಗು-ತ್ತಾನೆ
ವಾಗ್ವಾದ
ವಾಗ್ವೈ-ಖ-ರಿ-ಯಿಂದ
ವಾಚ-ನಾ-ಲಯ
ವಾಚ-ನಾ-ಲ-ಯ-ದಿಂದ
ವಾಚಾ
ವಾಚಾ-ಮ-ಗೋ-ಚ-ರ-ವಾಗಿ
ವಾಜ-ಸ-ನೇಯೀ
ವಾಟ್
ವಾಡು-ತ್ತಿ-ದ್ದೇನೆ
ವಾಣಿ
ವಾಣಿ-ಅ-ವರ
ವಾಣಿಯ
ವಾಣಿಯೇ
ವಾಣಿ-ಯೊಂದು
ವಾತಾ-ವ-ರಣ
ವಾತಾ-ವ-ರ-ಣ-ದಲ್ಲಿ
ವಾತಾ-ವ-ರ-ಣ-ದಲ್ಲೆಲ್ಲ
ವಾತಾ-ವ-ರ-ಣ-ದಿಂದ
ವಾತಾ-ವ-ರ-ಣ-ವನ್ನು
ವಾತಾ-ವ-ರ-ಣ-ವನ್ನೇ
ವಾತಾ-ವ-ರ-ಣ-ವಿತ್ತು
ವಾತಾ-ವ-ರ-ಣವು
ವಾತಾ-ವ-ರ-ಣವೇ
ವಾತಾ-ವ-ರ-ಣ-ವೇ-ರ್ಪ-ಟ್ಟಿತು
ವಾತಾ-ವ-ರ-ಣ-ವೊಂ-ದನ್ನು
ವಾತಾ-ವ-ರ-ಣ-ವೊಂದು
ವಾತ್ಯ-ಲ್ಯ-ಧಾ-ರೆ-ಯನ್ನೇ
ವಾತ್ಸಲ್ಯ
ವಾತ್ಸ-ಲ್ಯ-ಪ್ರೇಮ
ವಾತ್ಸ-ಲ್ಯದ
ವಾತ್ಸ-ಲ್ಯ-ದಿಂದ
ವಾತ್ಸ-ಲ್ಯ-ಭ-ರಿತ
ವಾತ್ಸ-ಲ್ಯ-ವನ್ನು
ವಾತ್ಸ-ಲ್ಯ-ಶಕ್ತಿ
ವಾದ
ವಾದಂತೆ
ವಾದ-ಕ್ಕೆ-ಳೆದು
ವಾದ-ಗಳ
ವಾದ-ಗಳನ್ನು
ವಾದ-ಗಳು
ವಾದದ
ವಾದ-ದ-ಲ್ಲಂತೂ
ವಾದ-ದಲ್ಲಿ
ವಾದ-ದ್ದನ್ನು
ವಾದದ್ದು
ವಾದ-ದ್ದೇನೂ
ವಾದ-ನ-ವನ್ನು
ವಾದ-ಮಾ-ಡಲು
ವಾದರೂ
ವಾದರೆ
ವಾದ-ವನ್ನು
ವಾದ-ವನ್ನೂ
ವಾದ-ವನ್ನೇ
ವಾದ-ವಿ-ವಾ-ದ-ಗ-ಳಾ-ವುವೂ
ವಾದವು
ವಾದಷ್ಟು
ವಾದಿ-ಯೆಂದು
ವಾದಿ-ಸಿ-ದರು
ವಾದಿ-ಸು-ತ್ತಿದೆ
ವಾದುದು
ವಾದು-ದೆಂ-ದರೆ
ವಾದು-ದೆಂದು
ವಾದುದೊ
ವಾದುವು
ವಾದ್ದ-ರಿಂದ
ವಾದ್ಯ
ವಾದ್ಯ-ಕು-ಟೀ-ರ-ಗಳಿಂದ
ವಾದ್ಯ-ಗಳನ್ನು
ವಾದ್ಯ-ಗಳು
ವಾದ್ಯ-ಗಳೂ
ವಾದ್ಯ-ಗ-ಳೊಂ-ದಿಗೆ
ವಾದ್ಯ-ವೃಂದ
ವಾನ್
ವಾನ್ಹಾ-ಗೆ-ನ್ನನ
ವಾಪ-ಸಾ-ಗ-ಲಿ-ದ್ದರು
ವಾಪಸು
ವಾಪಸ್ಸು
ವಾಮಾ-ಚಾ-ರ-ಗಳ
ವಾಮಾ-ಚಾ-ರ-ವನ್ನು
ವಾಯಿತು
ವಾಯಿತೆ
ವಾಯಿತೊ
ವಾಯು
ವಾಯು-ವೇ-ಗ-ದಲ್ಲಿ
ವಾಯು-ಸಂ-ಚಾರ
ವಾಯು-ಸೇ-ವ-ನೆ-ಗಾಗಿ
ವಾರ
ವಾರಕ್ಕೆ
ವಾರ-ಕ್ಕೊಮ್ಮೆ
ವಾರ-ಗಳ
ವಾರ-ಗಳನ್ನು
ವಾರ-ಗಳಲ್ಲಿ
ವಾರ-ಗ-ಳ-ವ-ರೆಗೆ
ವಾರ-ಗಳಿಂದ
ವಾರ-ಗ-ಳಿಂ-ದಲೂ
ವಾರ-ಗ-ಳಿಗೆ
ವಾರದ
ವಾರ-ದಲ್ಲಿ
ವಾರ-ದಲ್ಲೇ
ವಾರ-ದ-ಲ್ಲೊಂದು
ವಾರ-ದಿಂದ
ವಾರ-ದಿಂ-ದಲೂ
ವಾರ-ಪ-ತ್ರಿಕೆ
ವಾರ-ವನ್ನು
ವಾರ-ಸು-ದಾ-ರತ್ವ
ವಾರಾ
ವಾರಾಂ-ಗನೆ
ವಾರಾ-ಣಸಿ
ವಾರಾ-ಣ-ಸಿ-ಗಳಿಂದ
ವಾರಾ-ಣ-ಸಿಗೆ
ವಾರಾ-ಣ-ಸಿಗೇ
ವಾರಾ-ಣ-ಸಿಯ
ವಾರಾ-ಣ-ಸಿ-ಯಲ್ಲಿ
ವಾರಾ-ಣ-ಸಿ-ಯ-ಲ್ಲಿದ್ದ
ವಾರಾ-ಣ-ಸಿ-ಯ-ಲ್ಲಿ-ದ್ದಾಗ
ವಾರಾ-ಣ-ಸಿ-ಯಲ್ಲೂ
ವಾರು
ವಾರ್ತೆ
ವಾರ್ತೆ-ಯನ್ನು
ವಾರ್ತೆ-ಯ-ನ್ನೆಂದೂ
ವಾರ್ಷಿ-ಕೋ-ತ್ಸ-ವ-ದಲ್ಲಿ
ವಾರ್ಷಿ-ಕೋ-ತ್ಸ-ವ-ವನ್ನು
ವಾಲ-ಗ-ಡೋ-ಲು-ಕ-ಹ-ಳೆ-ಗಳ
ವಾಲ್ಟ್
ವಾಲ್ಡೊ
ವಾವ್
ವಾಸ
ವಾಸಕ್ಕೆ
ವಾಸ-ಕ್ಕೊಂದು
ವಾಸ-ದಿಂ-ದಾಗಿ
ವಾಸ-ಯೋ-ಗ್ಯ-ವಾ-ಗಿ-ಸಲು
ವಾಸ-ವಾ-ಗಿದ್ದ
ವಾಸ-ವಾ-ಗಿದ್ದು
ವಾಸ-ವಾ-ಗಿ-ರ-ಬೇ-ಕಾಗಿ
ವಾಸ-ವಾ-ಗಿ-ರ-ಬೇ-ಕೆಂಬ
ವಾಸ-ವಾ-ಗಿ-ರು-ತ್ತಾರೆ
ವಾಸ-ವಾ-ಗಿ-ರು-ವು-ದರ
ವಾಸ-ವಾ-ಗಿ-ರೋಣ
ವಾಸ-ಸ್ಥ-ಳ-ವನ್ನು
ವಾಸಾದಿ
ವಾಸಿ
ವಾಸಿ-ಗ-ಳ-ನ್ನಾ-ಗಿ-ಸಿ-ತಂತೆ
ವಾಸಿ-ಗ-ಳೆ-ಲ್ಲರೂ
ವಾಸಿ-ಯಾ-ಗು-ವಂ-ತಾ-ಯಿತು
ವಾಸಿ-ಯಾ-ಗು-ವು-ದೆಂಬ
ವಾಸಿ-ಯಾ-ದೀತು
ವಾಸಿ-ಸಿ-ದ-ವ-ರನ್ನೂ
ವಾಸಿ-ಸಿ-ದ್ದೇನೆ
ವಾಸಿ-ಸು-ತ್ತಾರೆ
ವಾಸಿ-ಸು-ತ್ತಿ-ದ್ದರೂ
ವಾಸಿ-ಸುವ
ವಾಸಿ-ಸು-ವುದು
ವಾಸ್ತ-ವಾಂಶ
ವಾಸ್ತ-ವಾಂ-ಶ-ವನ್ನು
ವಾಸ್ತ-ವಿ-ಕತೆ
ವಾಸ್ತ-ವಿ-ಕ-ತೆ-ಗಳನ್ನೂ
ವಾಸ್ತ-ವಿ-ಕ-ತೆ-ಯನ್ನು
ವಾಸ್ತ-ವಿ-ಕ-ತೆ-ಯಲ್ಲ
ವಾಸ್ತವ್ಯ
ವಾಸ್ತ-ವ್ಯ-ಕ್ಕಾಗಿ
ವಾಸ್ತ-ವ್ಯಕ್ಕೆ
ವಾಸ್ತು-ಶಿಲ್ಪ
ವಾಸ್ತು-ಶಿ-ಲ್ಪ-ಗಳನ್ನು
ವಾಸ್ತು-ಶಿ-ಲ್ಪದ
ವಾಸ್ತು-ಶಿ-ಲ್ಪ-ವನ್ನೂ
ವಾಸ್ತು-ಶಿ-ಲ್ಪಿ-ಗಳು
ವಾಹನ
ವಾಹ-ನ-ದಿಂದ
ವಾಹ-ನ-ದೊಂ-ದಿಗೆ
ವಿ
ವಿಂಚಿಯ
ವಿಂಬ-ಲ್ಡ-ನ್ನಿಗೆ
ವಿಕಟ
ವಿಕ-ಟಾ-ಟ್ಟ-ಹಾ-ಸ-ಗೈ-ಯು-ತ್ತಿ-ದ್ದಾನೆ
ವಿಕ-ಸಿ-ತ-ಗೊಂ-ಡಿ-ವೆಯೋ
ವಿಕ-ಸಿ-ತ-ವಾ-ದುದು
ವಿಕಾರ
ವಿಕಾ-ರ-ದ-ನಿ-ಯಿಂದ
ವಿಕಾಸ
ವಿಕಾ-ಸಕ್ಕೆ
ವಿಕಾ-ಸ-ಗೊ-ಳ್ಳ-ಬೇ-ಕಾ-ದರೆ
ವಿಕಾ-ಸದ
ವಿಕಾ-ಸ-ವಾದ
ವಿಕಾ-ಸ-ವಾ-ದದ
ವಿಕಾ-ಸ-ವಾ-ದ-ವೆಂ-ಬುದು
ವಿಕೃತ
ವಿಕೃ-ತ-ವಾಗಿ
ವಿಕ್ಟೋ-ರಿಯಾ
ವಿಕ್ರಮ
ವಿಕ್ರ-ಮಾ-ದಿ-ತ್ಯನ
ವಿಖ್ಯಾತ
ವಿಖ್ಯಾ-ತ-ಳಾದ
ವಿಖ್ಯಾ-ತಿ-ಗೊ-ಳಿ-ಸಿ-ದರೋ
ವಿಗೆ
ವಿಗ್ರಹ
ವಿಗ್ರ-ಹ-ಗಳನ್ನು
ವಿಗ್ರ-ಹ-ಗ-ಳಿವೆ
ವಿಗ್ರ-ಹ-ಗಳೂ
ವಿಗ್ರ-ಹದ
ವಿಗ್ರ-ಹ-ದಂತೆ
ವಿಗ್ರ-ಹ-ದಲ್ಲಿ
ವಿಗ್ರ-ಹ-ವನ್ನು
ವಿಗ್ರ-ಹ-ವಿದೆ
ವಿಗ್ರ-ಹಾ-ರಾ-ಧಕ
ವಿಗ್ರ-ಹಾ-ರಾ-ಧನೆ
ವಿಗ್ರ-ಹಾ-ರಾ-ಧ-ನೆಯ
ವಿಗ್ರ-ಹಾ-ರಾ-ಧ-ನೆ-ಯನ್ನು
ವಿಗ್ರ-ಹಾ-ರಾ-ಧ-ನೆ-ಯಿಂದ
ವಿಚ-ಲಿ-ತ-ಗೊ-ಳಿ-ಸ-ಬ-ಹು-ದಾ-ಗಿತ್ತು
ವಿಚ-ಲಿ-ತ-ರಾ-ಗ-ದಿ-ದ್ದ-ವ-ರಲ್ಲಿ
ವಿಚ-ಲಿ-ತ-ರಾ-ಗದೆ
ವಿಚ-ಲಿ-ತ-ರಾ-ಗ-ಲಿಲ್ಲ
ವಿಚ-ಲಿ-ತ-ರಾ-ಗು-ವ-ವ-ರಲ್ಲ
ವಿಚ-ಲಿ-ತ-ರಾ-ಗು-ವು-ದಿಲ್ಲ
ವಿಚಾರ
ವಿಚಾ-ರ-ಅ-ವಿ-ಚಾರ
ವಿಚಾ-ರ-ವಿ-ಹಾರ
ವಿಚಾ-ರಕ್ಕೆ
ವಿಚಾ-ರ-ಗಳ
ವಿಚಾ-ರ-ಗ-ಳ-ನ್ನಾ-ಗಲಿ
ವಿಚಾ-ರ-ಗಳನ್ನು
ವಿಚಾ-ರ-ಗಳನ್ನೂ
ವಿಚಾ-ರ-ಗ-ಳ-ಲ್ಲವೆ
ವಿಚಾ-ರ-ಗಳಲ್ಲಿ
ವಿಚಾ-ರ-ಗಳಿಂದ
ವಿಚಾ-ರ-ಗ-ಳಿಗೆ
ವಿಚಾ-ರ-ಗ-ಳಿವೆ
ವಿಚಾ-ರ-ಗಳು
ವಿಚಾ-ರ-ಗ-ಳೆಂ-ದರೆ
ವಿಚಾ-ರ-ಗ-ಳೆಲ್ಲ
ವಿಚಾ-ರ-ಗೋಷ್ಠಿ
ವಿಚಾ-ರದ
ವಿಚಾ-ರ-ದಲ್ಲಿ
ವಿಚಾ-ರ-ದ-ಲ್ಲಿನ
ವಿಚಾ-ರ-ದಲ್ಲೂ
ವಿಚಾ-ರ-ಧಾರೆ
ವಿಚಾ-ರ-ಧಾ-ರೆಗೆ
ವಿಚಾ-ರ-ಧಾ-ರೆ-ಯನ್ನು
ವಿಚಾ-ರ-ಧಾ-ರೆ-ಯನ್ನೇ
ವಿಚಾ-ರ-ಧಾ-ರೆ-ಯಿಂದ
ವಿಚಾ-ರ-ಧಾ-ರೆಯು
ವಿಚಾ-ರ-ಲ-ಹ-ರಿ-ಯನ್ನೇ
ವಿಚಾ-ರ-ಲ-ಹ-ರಿ-ಯಲ್ಲಿ
ವಿಚಾ-ರ-ಲ-ಹ-ರಿಯು
ವಿಚಾ-ರ-ಲ-ಹ-ರಿಯೇ
ವಿಚಾ-ರ-ವಂತ
ವಿಚಾ-ರ-ವನ್ನು
ವಿಚಾ-ರ-ವನ್ನೂ
ವಿಚಾ-ರ-ವ-ನ್ನೆಲ್ಲ
ವಿಚಾ-ರ-ವನ್ನೇ
ವಿಚಾ-ರ-ವಲ್ಲ
ವಿಚಾ-ರ-ವಾಗಿ
ವಿಚಾ-ರ-ವಾದ
ವಿಚಾ-ರ-ವಿ-ಧಾನ
ವಿಚಾ-ರ-ವಿ-ನಿ-ಮಯ
ವಿಚಾ-ರ-ವಿ-ನಿ-ಮ-ಯದ
ವಿಚಾ-ರ-ವೆಂದರೆ
ವಿಚಾ-ರ-ಸ-ರ-ಣಿ-ಯಿಂದ
ವಿಚಾ-ರಿ-ಸ-ಲಾಗಿ
ವಿಚಾ-ರಿಸಿ
ವಿಚಾ-ರಿ-ಸಿ-ಕೊಂಡ
ವಿಚಾ-ರಿ-ಸಿ-ಕೊಂ-ಡರು
ವಿಚಾ-ರಿ-ಸಿ-ದರು
ವಿಚಾ-ರಿ-ಸಿ-ದಳು
ವಿಚಾ-ರಿ-ಸಿ-ದಾಗ
ವಿಚಾ-ರಿ-ಸಿದ್ದು
ವಿಚಾ-ರಿ-ಸಿ-ನೋಡಿ
ವಿಚಾ-ರಿ-ಸಿ-ಯಾ-ದರೂ
ವಿಚಿತ್ರ
ವಿಚಿ-ತ್ರ-ವಾಗಿ
ವಿಚಿ-ತ್ರ-ವಾದ
ವಿಚಿ-ತ್ರ-ವಾ-ದ-ದ್ದಾ-ಗಲಿ
ವಿಜಯ
ವಿಜ-ಯಕ್ಕೆ
ವಿಜ-ಯ-ದ-ಶ-ಮಿ-ಯಂದು
ವಿಜ-ಯ-ಪ-ತಾ-ಕೆ-ಗಳ
ವಿಜ-ಯ-ಯಾತ್ರೆ
ವಿಜ-ಯ-ಯಾ-ತ್ರೆ-ಯನ್ನು
ವಿಜ-ಯ-ಲಕ್ಷ್ಮಿ
ವಿಜ-ಯ-ವನ್ನು
ವಿಜ-ಯ-ವಾ-ಗಿತ್ತು
ವಿಜ-ಯಿ-ಗ-ಳಾಗಿ
ವಿಜ-ಯಿ-ಯಾಗಿ
ವಿಜ-ಯೋ-ತ್ಸ-ವದ
ವಿಜ-ಯೋ-ತ್ಸಾ-ಹ-ದಿಂದ
ವಿಜೃಂ-ಭ-ಣೆಯ
ವಿಜೃಂ-ಭ-ಣೆ-ಯಿಂದ
ವಿಜ್ಞಾನ
ವಿಜ್ಞಾ-ನಕ್ಕೂ
ವಿಜ್ಞಾ-ನ-ಗಳ
ವಿಜ್ಞಾ-ನದ
ವಿಜ್ಞಾ-ನ-ನಿ-ಷ್ಠ-ನಾ-ಗಿ-ದ್ದು-ಕೊಂಡು
ವಿಜ್ಞಾ-ನ-ವನ್ನು
ವಿಜ್ಞಾ-ನ-ಶಾ-ಸ್ತ್ರದ
ವಿಜ್ಞಾ-ನಾ-ನಂ-ದರ
ವಿಜ್ಞಾ-ನಾ-ನಂ-ದ-ರಾ-ದರು
ವಿಜ್ಞಾನಿ
ವಿಜ್ಞಾ-ನಿ-ಗಳು
ವಿಜ್ಞಾ-ನಿಯ
ವಿಜ್ಞಾ-ನಿ-ಯಾದ
ವಿಟ್ಮನ್
ವಿಟ್ಮ-ನ್ನ-ನನ್ನು
ವಿಟ್ಮಾ-ರ್ಶ್
ವಿತ-ರಣಾ
ವಿತ-ರಣೆ
ವಿತ-ರಾಗಿ
ವಿತ-ರಾ-ಗಿದ್ದ
ವಿದಾಯ
ವಿದಿ-ತ-ವಾ-ಯಿತು
ವಿದೀ-ರ್ಣ-ಗೊ-ಳಿ-ಸು-ತ್ತಿವೆ
ವಿದೆಈ
ವಿದೇ-ಶ-ಗಳ
ವಿದೇ-ಶ-ಗಳಲ್ಲಿ
ವಿದೇ-ಶ-ಗ-ಳ-ಲ್ಲಿನ
ವಿದೇ-ಶದ
ವಿದೇಶೀ
ವಿದೇ-ಶೀ-ಯನೂ
ವಿದೇ-ಶೀ-ಯಳು
ವಿದ್ಯಾ
ವಿದ್ಯಾ-ದಾನ
ವಿದ್ಯಾ-ದಾ-ನಕ್ಕೆ
ವಿದ್ಯಾ-ನಿಧಿ
ವಿದ್ಯಾ-ಭ್ಯಾಸ
ವಿದ್ಯಾ-ಭ್ಯಾ-ಸ-ಐ-ಶ್ವರ್ಯ
ವಿದ್ಯಾ-ಭ್ಯಾ-ಸ-ಕ್ಕಾಗಿ
ವಿದ್ಯಾ-ಭ್ಯಾ-ಸಕ್ಕೆ
ವಿದ್ಯಾ-ಭ್ಯಾ-ಸದ
ವಿದ್ಯಾ-ಭ್ಯಾ-ಸ-ದಲ್ಲಿ
ವಿದ್ಯಾ-ಭ್ಯಾ-ಸ-ವನ್ನು
ವಿದ್ಯಾ-ಭ್ಯಾ-ಸ-ವಲ್ಲ
ವಿದ್ಯಾ-ಭ್ಯಾ-ಸವು
ವಿದ್ಯಾ-ಭ್ಯಾ-ಸ-ವೆಲ್ಲ
ವಿದ್ಯಾರ್ಥಿ
ವಿದ್ಯಾ-ರ್ಥಿ-ಗಳ
ವಿದ್ಯಾ-ರ್ಥಿ-ಗ-ಳ-ನ್ನು-ದ್ದೇ-ಶಿಸಿ
ವಿದ್ಯಾ-ರ್ಥಿ-ಗ-ಳಿ-ಗಾಗಿ
ವಿದ್ಯಾ-ರ್ಥಿ-ಗ-ಳಿಗೆ
ವಿದ್ಯಾ-ರ್ಥಿ-ಗಳು
ವಿದ್ಯಾ-ರ್ಥಿ-ಗ-ಳೊಂ-ದಿಗೆ
ವಿದ್ಯಾ-ರ್ಥಿ-ನಿ-ಯ-ರಿಗೆ
ವಿದ್ಯಾ-ರ್ಥಿ-ನಿ-ಯೊ-ಬ್ಬಳು
ವಿದ್ಯಾ-ರ್ಥಿ-ಯಾ-ಗಿ-ದ್ದಾಗ
ವಿದ್ಯಾ-ರ್ಥಿಯು
ವಿದ್ಯಾ-ವಂತ
ವಿದ್ಯಾ-ವಂ-ತರ
ವಿದ್ಯಾ-ವಂ-ತ-ರನ್ನು
ವಿದ್ಯಾ-ವಂ-ತ-ರಾದ
ವಿದ್ಯಾ-ವಂ-ತ-ರಿಗೆ
ವಿದ್ಯಾ-ವಂ-ತರು
ವಿದ್ಯಾ-ವಂ-ತರೂ
ವಿದ್ಯಾ-ವಂ-ತ-ರೆ-ನ್ನಿ-ಸಿ-ಕೊಂ-ಡ-ವರು
ವಿದ್ಯಾ-ವ-ತಿ-ಯ-ರಾದ
ವಿದ್ಯಾ-ಸಂ-ಸ್ಥೆ-ಗ-ಳಿಗೆ
ವಿದ್ಯಾ-ಸಂ-ಸ್ಥೆ-ಗಳು
ವಿದ್ಯಾ-ಸಂ-ಸ್ಥೆಗೆ
ವಿದ್ಯು-ಚ್ಛಕ್ತಿ
ವಿದ್ಯುತ್
ವಿದ್ಯುತ್ತು
ವಿದ್ಯು-ತ್ಸಂ-ಚಾರ
ವಿದ್ಯು-ತ್ಸಂ-ಚಾ-ರ-ವನ್ನೇ
ವಿದ್ಯು-ದಾ-ಘಾ-ತ-ವನ್ನೇ
ವಿದ್ಯೆ
ವಿದ್ಯೆಯ
ವಿದ್ಯೆ-ಯನ್ನು
ವಿದ್ಯೆ-ಯಿಂದ
ವಿದ್ಯೆ-ಯಿ-ಲ್ಲದ
ವಿದ್ಯೆ-ಯೆಂ-ದರೆ
ವಿದ್ವ-ತ್ತನ್ನೂ
ವಿದ್ವನ್
ವಿದ್ವಾಂ-ಸರ
ವಿದ್ವಾಂ-ಸ-ರಿಗೆ
ವಿದ್ವಾಂ-ಸರು
ವಿದ್ವಾಂ-ಸ-ರೊಂ-ದಿಗೆ
ವಿದ್ವಾಂ-ಸ-ರೊ-ಬ್ಬರು
ವಿದ್ವೇ-ಷದ
ವಿಧ
ವಿಧದ
ವಿಧ-ದಲ್ಲಿ
ವಿಧ-ದ-ಲ್ಲಿಯೂ
ವಿಧ-ದಲ್ಲೂ
ವಿಧ-ದಿಂದ
ವಿಧ-ರ್ಮೀಯ
ವಿಧವಾ
ವಿಧ-ವಾದ
ವಿಧ-ವಾ-ವಿ-ವಾ-ಹದ
ವಿಧ-ವಿ-ಧದ
ವಿಧ-ವಿ-ಧ-ವಾದ
ವಿಧವೆ
ವಿಧ-ವೆ-ಯರ
ವಿಧ-ವೆ-ಯ-ರಾದ
ವಿಧ-ವೆ-ಯ-ರಿ-ಗಾಗಿ
ವಿಧ-ವೆ-ಯ-ರಿಗೂ
ವಿಧ-ವೆ-ಯ-ರಿಗೆ
ವಿಧ-ವೆ-ಯ-ರಿ-ದ್ದರೆ
ವಿಧ-ವೆ-ಯರು
ವಿಧ-ವೆ-ಯೊ-ಬ್ಬಳು
ವಿಧಾನ
ವಿಧಾ-ನ-ಕ್ಕಿಂತ
ವಿಧಾ-ನ-ಗಳ
ವಿಧಾ-ನ-ಗಳನ್ನು
ವಿಧಾ-ನ-ಗಳನ್ನೂ
ವಿಧಾ-ನ-ಗಳಲ್ಲಿ
ವಿಧಾ-ನ-ಗಳಿಂದ
ವಿಧಾ-ನ-ಗ-ಳಿಗೂ
ವಿಧಾ-ನ-ಗಳು
ವಿಧಾ-ನದ
ವಿಧಾ-ನ-ದಲ್ಲಿ
ವಿಧಾ-ನ-ದಿಂದ
ವಿಧಾ-ನ-ವಂತೂ
ವಿಧಾ-ನ-ವನ್ನು
ವಿಧಾ-ನ-ವಿ-ರ-ಲಾ-ರದು
ವಿಧಾ-ನವು
ವಿಧಾ-ನ-ವೊಂ-ದನ್ನು
ವಿಧಾ-ಯಕ
ವಿಧಿ
ವಿಧಿ-ನಿ-ಷೇ-ಧ-ಗಳನ್ನೆಲ್ಲ
ವಿಧಿ-ಗಳನ್ನು
ವಿಧಿ-ಗಳನ್ನೂ
ವಿಧಿ-ನಿ-ಯ-ಮ-ಗಳಲ್ಲಿ
ವಿಧಿ-ನಿ-ಯ-ಮ-ಗಳು
ವಿಧಿ-ಬದ್ಧ
ವಿಧಿ-ಬ-ದ್ಧ-ವಾಗಿ
ವಿಧಿ-ಯನ್ನೇ
ವಿಧಿ-ಯಿಲ್ಲ
ವಿಧಿ-ಯಿ-ಲ್ಲದೆ
ವಿಧಿ-ಯು-ಕ್ತ-ವಾಗಿ
ವಿಧಿ-ವ-ತ್ತಾಗಿ
ವಿಧಿ-ವಾದ
ವಿಧಿ-ವಿ-ಧಾನ
ವಿಧಿ-ವಿ-ಧಾ-ನ-ಗಳನ್ನು
ವಿಧಿ-ವಿ-ಧಾ-ನ-ಗಳು
ವಿಧಿ-ಸ-ಬೇ-ಕೆಂದು
ವಿಧಿ-ಸಿ-ದ-ರಾ-ದರೂ
ವಿಧಿ-ಸಿಯೇ
ವಿಧಿ-ಸು-ವಾಗ
ವಿಧಿ-ಸು-ವುದನ್ನು
ವಿಧೇಯ
ವಿಧೇ-ಯತೆ
ವಿಧೇ-ಯ-ನಾಗಿ
ವಿಧೇ-ಯ-ನಾ-ಗಿ-ರು-ವುದನ್ನು
ವಿಧೇ-ಯ-ರಾ-ಗಿ-ರ-ಬ-ಲ್ಲರೋ
ವಿಧೇ-ಯ-ರಾ-ಗಿ-ರ-ಬೇಕು
ವಿಧೇ-ಯ-ರಾ-ಗಿ-ರು-ವು-ದೆಂ-ದರೆ
ವಿಧೇ-ಯ-ವಾಗಿ
ವಿಧ್ಯುಕ್ತ
ವಿಧ್ವಂ-ಸಕ
ವಿನಂ-ತಿಸಿ
ವಿನಂ-ತಿ-ಸಿ-ಕೊಂಡ
ವಿನಂ-ತಿ-ಸಿ-ಕೊಂ-ಡ-ರಾ-ದರೂ
ವಿನಂ-ತಿ-ಸಿ-ಕೊಂ-ಡರು
ವಿನಂ-ತಿ-ಸಿ-ಕೊಂ-ಡಾಗ
ವಿನಂ-ತಿ-ಸಿ-ಕೊ-ಳ್ಳ-ಲಾ-ಯಿತು
ವಿನಮ್ರ
ವಿನ-ಮ್ರ-ತೆ-ಯಿಂದ
ವಿನ-ಮ್ರ-ಭಾ-ವ-ದಿಂದ
ವಿನಯ
ವಿನ-ಯ-ಕೃಷ್ಣ
ವಿನ-ಯ-ಕೃ-ಷ್ಣ-ದೇವ್
ವಿನ-ಯ-ದಿಂದ
ವಿನ-ಯ-ದಿಂ-ದಲೇ
ವಿನ-ಯ-ವನ್ನು
ವಿನ-ಲ್ಲೂ-ನಿನ್ನ
ವಿನಾ
ವಿನಾ-ಯತಿ
ವಿನಾ-ಯಿತಿ
ವಿನಾ-ಯಿ-ತಿ-ಯಿಲ್ಲ
ವಿನಾಶ
ವಿನಾ-ಶಕ್ಕೆ
ವಿನಾ-ಶ-ದತ್ತ
ವಿನಾ-ಶ-ದ-ತ್ತಲೇ
ವಿನಾ-ಶ-ವನ್ನು
ವಿನಿ-ಮಯ
ವಿನಿ-ಮ-ಯಿ-ಸಿ-ಕೊಳ್ಳು
ವಿನಿ-ಯೋಗ
ವಿನಿ-ಯೋ-ಗ-ವಾ-ಗು-ವಂತೆ
ವಿನಿ-ಯೋ-ಗಿ-ಸ-ಬೇಕು
ವಿನಿ-ಯೋ-ಗಿ-ಸಿ-ದ-ರಾ-ಯಿತು
ವಿನಿ-ಯೋ-ಗಿ-ಸಿ-ದರು
ವಿನಿ-ಯೋ-ಗಿಸು
ವಿನಿ-ಯೋ-ಗಿ-ಸು-ವಂ-ತಾ-ಗ-ಬೇಕು
ವಿನಿ-ಯೋ-ಗಿ-ಸು-ವಂತೆ
ವಿನಿ-ಯೋ-ಗಿ-ಸು-ವುದು
ವಿನೀತ
ವಿನೀ-ತ-ನಾ-ಗಿಯೇ
ವಿನೀ-ತ-ಭಾ-ವದ
ವಿನೂ-ತನ
ವಿನೋ-ದ-ಕ್ಕಾಗಿ
ವಿನೋ-ದದ
ವಿನೋ-ದ-ದಲ್ಲಿ
ವಿನ್
ವಿನ್ನನ
ವಿನ್ನನೂ
ವಿಪ-ತ್ತಿ-ನಿಂದ
ವಿಪ-ರೀತ
ವಿಪ-ರ್ಯಾ-ಸ-ವಾಗಿ
ವಿಪ-ರ್ಯಾ-ಸ-ವಾ-ಗು-ತ್ತ-ದೆ-ಯೆಂದು
ವಿಪ್ರಾ
ವಿಫ-ಲ-ಗೊ-ಳಿ-ಸಲು
ವಿಫ-ಲ-ನಾ-ದ-ವ-ನಿ-ರ-ಬ-ಹು-ದು-ಆ-ದರೆ
ವಿಫ-ಲ-ರಾಗಿ
ವಿಫ-ಲ-ವಾ-ಗ-ಲಿಲ್ಲ
ವಿಭಾಗ
ವಿಭಾ-ಗ-ಗಳನ್ನು
ವಿಭಾ-ಗ-ಗಳಲ್ಲಿ
ವಿಭಾ-ಗದ
ವಿಭಿನ್ನ
ವಿಭಿ-ನ್ನ-ತೆ-ಗ-ಳೆಲ್ಲ
ವಿಭಿ-ನ್ನ-ವಾ-ಗಿತ್ತು
ವಿಭಿ-ನ್ನ-ವಾ-ಗಿ-ದ್ದರೂ
ವಿಭಿ-ನ್ನ-ವಾ-ಗಿ-ದ್ದು-ದೇಕೆ
ವಿಭಿ-ನ್ನ-ವಾ-ಗಿ-ರ-ಲಿಲ್ಲ
ವಿಭಿ-ನ್ನ-ವಾದ
ವಿಭಿ-ನ್ನ-ವಾ-ದುದು
ವಿಭಿ-ನ್ನವೂ
ವಿಭೂತಿ
ವಿಭೂ-ತಿ-ಗ-ಳೆಲ್ಲ
ವಿಭೂ-ತಿ-ಪು-ರು-ಷ-ರನ್ನು
ವಿಭೂ-ತಿ-ಯನ್ನು
ವಿಭೂ-ತಿಯು
ವಿಭ್ರ-ಮ-ಗಳು
ವಿಭ್ರಾಂ-ತ-ನಾದ
ವಿಭ್ರಾಂ-ತ-ರಾ-ದರು
ವಿಮ-ರ್ಶಿಸಿ
ವಿಮ-ರ್ಶಿ-ಸು-ವುದು
ವಿಮರ್ಶೆ
ವಿಮ-ರ್ಶೆ-ಗ-ಳಾಗಿ
ವಿಮ-ರ್ಶೆ-ಗಳು
ವಿಮ-ಲಾ-ನಂದ
ವಿಮ-ಲಾ-ನಂ-ದರು
ವಿಮು-ಕ್ತ-ಗೊ-ಳಿ-ಸಲು
ವಿಮು-ಕ್ತ-ರಾ-ಗ-ತೊ-ಡ-ಗಿ-ದ್ದರು
ವಿಮು-ಕ್ತಿ-ಯನ್ನು
ವಿಮು-ಖತೆ
ವಿಮು-ಖ-ನಾದ
ವಿಮು-ಖ-ಳಾ-ಗ-ತೊ-ಡಗಿ
ವಿಮು-ಖ-ವಾ-ಗು-ತ್ತಿ-ತ್ತೆಂ-ದರೆ
ವಿಯೆ-ನ್ನಾ-ದಲ್ಲಿ
ವಿಯೆ-ನ್ನಾ-ದಿಂದ
ವಿಯೋಗ
ವಿಯೋ-ಗ-ಗಳು
ವಿರಕ್ತ
ವಿರಕ್ತಿ
ವಿರ-ಜಾ-ನಂದ
ವಿರ-ಜಾ-ನಂ-ದರ
ವಿರ-ಜಾ-ನಂ-ದ-ರ-ದ್ದಾ-ಗಿತ್ತು
ವಿರ-ಜಾ-ನಂ-ದ-ರಿಗೂ
ವಿರ-ಜಾ-ನಂ-ದ-ರಿಗೆ
ವಿರ-ಜಾ-ನಂ-ದರು
ವಿರ-ಜಾ-ನಂ-ದರೂ
ವಿರ-ಜಾ-ಹೋ-ಮ-ವನ್ನು
ವಿರ-ಮಿ-ಸಲು
ವಿರ-ಮಿಸಿ
ವಿರ-ಮಿ-ಸಿ-ದರು
ವಿರ-ಮಿ-ಸು-ತ್ತೇನೆ
ವಿರ-ಮಿ-ಸು-ವುದೂ
ವಿರ-ಲಿಲ್ಲ
ವಿರ-ಳ-ವಾ-ಗಿ-ರು-ವಲ್ಲಿ
ವಿರಾ
ವಿರಾ-ಗಿ-ಯಾದ
ವಿರಾಜ
ವಿರಾ-ಜಿ-ಸುತಿ
ವಿರಾ-ಜಿ-ಸು-ತ್ತಿತ್ತು
ವಿರಾ-ಜಿ-ಸು-ತ್ತಿದ್ದ
ವಿರಾ-ಜಿ-ಸು-ತ್ತಿ-ರು-ವ-ವರೂ
ವಿರಾ-ಜಿ-ಸು-ವುದನ್ನು
ವಿರಾಟ್
ವಿರಾಮ
ವಿರಾ-ಮ-ವಿ-ಲ್ಲದೆ
ವಿರಿ
ವಿರುದ್ಧ
ವಿರು-ದ್ಧದ
ವಿರು-ದ್ಧ-ವಾ-ಗಲು
ವಿರು-ದ್ಧ-ವಾಗಿ
ವಿರು-ದ್ಧ-ವಾ-ಗಿ-ದ್ದಳು
ವಿರು-ದ್ಧ-ವಾ-ಗಿಯೇ
ವಿರು-ದ್ಧ-ವಾ-ಗಿ-ರುವ
ವಿರು-ದ್ಧ-ವಾ-ಗಿ-ರುವು
ವಿರು-ದ್ಧ-ವಾ-ಡಿ-ದ-ವ-ರನ್ನು
ವಿರು-ದ್ಧ-ವಾದ
ವಿರು-ದ್ಧ-ವಾ-ದದ್ದು
ವಿರು-ದ್ಧ-ವಾ-ದ-ವು-ಗ-ಳಲ್ಲ
ವಿರು-ದ್ಧವೇ
ವಿರು-ದ್ಧಾ-ರ್ಥ-ದಲ್ಲಿ
ವಿರು-ವಂತೆ
ವಿರು-ವುದು
ವಿರೋಧ
ವಿರೋ-ಧದ
ವಿರೋ-ಧ-ದಿಂ-ದಾಗಿ
ವಿರೋ-ಧ-ವನ್ನು
ವಿರೋ-ಧ-ವನ್ನೇ
ವಿರೋ-ಧ-ವಾಗಿ
ವಿರೋ-ಧ-ವಾದ
ವಿರೋ-ಧವೂ
ವಿರೋ-ಧ-ವೆಲ್ಲ
ವಿರೋ-ಧಾ-ಭಿ-ಪ್ರಾ-ಯ-ಗ-ಳಿವೆ
ವಿರೋಧಿ
ವಿರೋ-ಧಿ-ಗಳ
ವಿರೋ-ಧಿ-ಗಳು
ವಿರೋ-ಧಿ-ಯಾ-ಗಿದ್ದ
ವಿರೋ-ಧಿ-ಯಾ-ಗಿ-ದ್ದರೂ
ವಿರೋ-ಧಿ-ಸ-ಲಾ-ರ-ದವ
ವಿರೋ-ಧಿ-ಸಲು
ವಿರೋ-ಧಿಸಿ
ವಿರೋ-ಧಿ-ಸಿ-ದರೂ
ವಿರೋ-ಧಿ-ಸು-ತ್ತಿ-ದ್ದರು
ವಿರೋ-ಧಿ-ಸು-ತ್ತೇವೆ
ವಿರೋ-ಧಿ-ಸು-ವ-ವರು
ವಿರೋ-ಧೀ-ಶ-ಕ್ತಿ-ಗಳ
ವಿಲಾ-ಸ
ವಿಲಾ-ಸ-ಪ್ರಿ-ಯರು
ವಿಲಾ-ಸವೂ
ವಿಲಿ-ಯಮ್
ವಿಲೀ-ನ-ಗೊಂ-ಡಿತು
ವಿಲ್ಲ
ವಿಲ್ಲದ
ವಿಲ್ಲ-ದಿ-ದ್ದರೆ
ವಿಲ್ಲಾ
ವಿಳಾ-ಸದ
ವಿವರ
ವಿವ-ರ-ಗಳ
ವಿವ-ರ-ಗಳನ್ನು
ವಿವ-ರ-ಗಳನ್ನೂ
ವಿವ-ರ-ಗಳನ್ನೆಲ್ಲ
ವಿವ-ರ-ಗಳಿಂದ
ವಿವ-ರ-ಗಳು
ವಿವ-ರಣೆ
ವಿವ-ರ-ಣೆ-ಗಳನ್ನೂ
ವಿವ-ರ-ಣೆ-ಗಳು
ವಿವ-ರ-ಣೆಯು
ವಿವ-ರ-ವನ್ನೂ
ವಿವ-ರ-ವಾಗಿ
ವಿವ-ರ-ವಾದ
ವಿವ-ರಿ-ಸ-ಲಾ-ಗಿತ್ತು
ವಿವ-ರಿ-ಸ-ಲಾ-ಗಿ-ದೆಯೋ
ವಿವ-ರಿ-ಸಲು
ವಿವ-ರಿಸಿ
ವಿವ-ರಿ-ಸಿದ
ವಿವ-ರಿ-ಸಿ-ದರು
ವಿವ-ರಿ-ಸಿ-ದರೆ
ವಿವ-ರಿ-ಸಿ-ದಳು
ವಿವ-ರಿ-ಸಿ-ದಾಗ
ವಿವ-ರಿ-ಸಿ-ದಾ-ಗ-ಲೆಲ್ಲ
ವಿವ-ರಿ-ಸುತ್ತ
ವಿವ-ರಿ-ಸು-ತ್ತಾರೆ
ವಿವ-ರಿ-ಸು-ತ್ತಿದ್ದ
ವಿವ-ರಿ-ಸು-ತ್ತಿ-ದ್ದಂತೆ
ವಿವ-ರಿ-ಸು-ತ್ತಿ-ದ್ದರು
ವಿವ-ರಿ-ಸುವ
ವಿವ-ರಿ-ಸು-ವಂತೆ
ವಿವ-ರಿ-ಸು-ವಿರಾ
ವಿವಾ-ದಾ-ತ್ಮ-ಕ-ವಾದ
ವಿವಾಹ
ವಿವಾ-ಹದ
ವಿವಾ-ಹ-ವಾ-ಗ-ಲಿದ್ದ
ವಿವಾ-ಹ-ವಾ-ಗಲೇ
ವಿವಾ-ಹ-ವೆಂ-ಬುದು
ವಿವಿಧ
ವಿವಿ-ಧ-ತೆಯ
ವಿವಿ-ಧ-ತೆ-ಯಲ್ಲಿ
ವಿವಿ-ಧೋ-ಪಾ-ಯ-ಗಳನ್ನು
ವಿವೇಕ
ವಿವೇ-ಕ-ಚೂಡಾ
ವಿವೇಕಾ
ವಿವೇಕಾನಂದ
ವಿವೇಕಾನಂದ-ಜಿ-ಯ-ವರು
ವಿವೇಕಾನಂದಜೀ
ವಿವೇಕಾನಂದನ
ವಿವೇಕಾನಂದ-ನನ್ನು
ವಿವೇಕಾನಂದ-ನಲ್ಲ
ವಿವೇಕಾನಂದ-ನಾ-ಗ-ಲಿ-ದ್ದೇನೆ
ವಿವೇಕಾನಂದ-ನಿ-ದ್ದಿ-ದ್ದರೆ
ವಿವೇಕಾನಂದರ
ವಿವೇಕಾನಂದ-ರಂ-ತಹ
ವಿವೇಕಾನಂದ-ರದೇ
ವಿವೇಕಾನಂದ-ರನ್ನು
ವಿವೇಕಾನಂದ-ರಲ್ಲಿ
ವಿವೇಕಾನಂದ-ರಾಗಿ
ವಿವೇಕಾನಂದ-ರಿಂದ
ವಿವೇಕಾನಂದ-ರಿಂ-ದಲೇ
ವಿವೇಕಾನಂದ-ರಿ-ಗಿಂತ
ವಿವೇಕಾನಂದ-ರಿಗೂ
ವಿವೇಕಾನಂದ-ರಿಗೆ
ವಿವೇಕಾನಂದ-ರಿತ್ತ
ವಿವೇಕಾನಂದರು
ವಿವೇಕಾನಂದರೇ
ವಿವೇಕಾನಂದ-ರೊ-ಬ್ಬರೇ
ವಿವೇಕಾನಂದರೋ
ವಿವೇ-ಕಿ-ಗ-ಳಾಗಿ
ವಿವೇ-ಕಿಯೇ
ವಿವೇ-ಚನಾ
ವಿವೇ-ಚ-ನೆ-ಯಿ-ಲ್ಲದೆ
ವಿಶ-ದ-ವಾಗಿ
ವಿಶ-ದೀ-ಕ-ರಿ-ಸ-ದಿ-ರ-ಲಿಲ್ಲ
ವಿಶಾಲ
ವಿಶಾ-ಲ-ತೇ-ಜೋ-ವಂತ
ವಿಶಾ-ಲ-ದೃ-ಷ್ಟಿಯ
ವಿಶಾ-ಲ-ನ-ಯ-ನ-ಗಳಲ್ಲಿ
ವಿಶಾ-ಲ-ವಾಗಿ
ವಿಶಾ-ಲ-ವಾ-ಗು-ತ್ತಿ-ದ್ದೇನೆ
ವಿಶಾ-ಲ-ವಾ-ಗು-ತ್ತಿ-ದ್ದೇ-ನೆಂ-ದರೆ
ವಿಶಾ-ಲ-ವಾದ
ವಿಶಾ-ಲ-ವಾ-ದದ್ದು
ವಿಶಾ-ಲ-ವಾ-ದ-ದ್ದೆಂ-ಬು-ದನ್ನು
ವಿಶಾ-ಲವೂ
ವಿಶಾ-ಲ-ಹೃ-ದ-ಯ-ರಾ-ಗಲು
ವಿಶಿಷ್ಟ
ವಿಶಿ-ಷ್ಟ-ವಾ-ಗಿ-ದ್ದುವು
ವಿಶಿ-ಷ್ಟ-ವಾದ
ವಿಶಿ-ಷ್ಟ-ವಾ-ದವು
ವಿಶಿ-ಷ್ಟವೂ
ವಿಶಿ-ಷ್ಟಾ-ದ್ವೈತ
ವಿಶಿ-ಷ್ಟಾ-ದ್ವೈ-ತ-ವಾ-ಗಿ-ರಲಿ
ವಿಶೇಷ
ವಿಶೇ-ಷತಃ
ವಿಶೇ-ಷತೆ
ವಿಶೇ-ಷ-ದಿಂದ
ವಿಶೇ-ಷ-ವಾಗಿ
ವಿಶೇ-ಷ-ವಾದ
ವಿಶೇ-ಷ-ವಾ-ದ-ವು-ಗಳೇ
ವಿಶೇ-ಷ-ವಾ-ದುದು
ವಿಶೇ-ಷ-ವೆಂದೇ
ವಿಶೇ-ಷವೇ
ವಿಶೇ-ಷ-ವೇನೂ
ವಿಶ್ರ-ಮಿ-ಸಲು
ವಿಶ್ರ-ಮಿಸಿ
ವಿಶ್ರ-ಮಿ-ಸಿಕೊ
ವಿಶ್ರ-ಮಿ-ಸಿ-ಕೊಂ-ಡರು
ವಿಶ್ರ-ಮಿ-ಸಿ-ಕೊ-ಳ್ಳ-ಬೇಕು
ವಿಶ್ರ-ಮಿ-ಸಿ-ಕೊ-ಳ್ಳಲು
ವಿಶ್ರ-ಮಿ-ಸಿ-ಕೊ-ಳ್ಳು-ವಂತೆ
ವಿಶ್ರ-ಮಿ-ಸಿ-ದರು
ವಿಶ್ರ-ಮಿ-ಸು-ವಂತೆ
ವಿಶ್ರಾಂತಿ
ವಿಶ್ರಾಂ-ತಿ-ನಿ-ದ್ರೆ-ಗ-ಳಿ-ಲ್ಲ-ದಂತೆ
ವಿಶ್ರಾಂ-ತಿ-ಇ-ವು-ಗಳಿಂದ
ವಿಶ್ರಾಂ-ತಿ-ಗಳು
ವಿಶ್ರಾಂ-ತಿ-ಗಳೂ
ವಿಶ್ರಾಂ-ತಿ-ಗಾಗಿ
ವಿಶ್ರಾಂ-ತಿ-ಗೆಂದು
ವಿಶ್ರಾಂ-ತಿ-ದಾ-ಯ-ಕವೂ
ವಿಶ್ರಾಂ-ತಿಯ
ವಿಶ್ರಾಂ-ತಿ-ಯನ್ನು
ವಿಶ್ರಾಂ-ತಿ-ಯನ್ನೂ
ವಿಶ್ರಾಂ-ತಿ-ಯ-ಲ್ಲಿ-ರು-ವು-ದ-ರಿಂದ
ವಿಶ್ರಾಂ-ತಿ-ಯಲ್ಲೇ
ವಿಶ್ರಾಂ-ತಿ-ಯಾ-ದರೂ
ವಿಶ್ರಾಂ-ತಿ-ಯಿಂದ
ವಿಶ್ರಾಂ-ತಿ-ಯಿ-ಲ್ಲ-ವೆಂ-ಬು-ದನ್ನು
ವಿಶ್ರಾಂ-ತಿಯೂ
ವಿಶ್ರಾಂ-ತಿಯೇ
ವಿಶ್ಲೇ-ಷಿಸಿ
ವಿಶ್ವ
ವಿಶ್ವ-ಕ-ಲ್ಯಾಣ
ವಿಶ್ವ-ಕವಿ
ವಿಶ್ವ-ಕೋಶ
ವಿಶ್ವಕ್ಕೇ
ವಿಶ್ವದ
ವಿಶ್ವ-ದಲ್ಲಿ
ವಿಶ್ವ-ದೆ-ಲ್ಲರ
ವಿಶ್ವ-ದೊ-ಳ-ಗೆ-ಲ್ಲೆಲ್ಲೂ
ವಿಶ್ವ-ಧರ್ಮ
ವಿಶ್ವ-ಧ-ರ್ಮ-ಚ-ರಿತ
ವಿಶ್ವ-ಧ-ರ್ಮ-ವೆ-ನಿ-ಸ-ಬಲ್ಲ
ವಿಶ್ವ-ಧ-ರ್ಮ-ವೊಂ-ದನ್ನು
ವಿಶ್ವ-ಧ-ರ್ಮ-ವೊಂ-ದರ
ವಿಶ್ವ-ನಾಥ
ವಿಶ್ವ-ನಿ-ಯಮ
ವಿಶ್ವ-ಮ-ಟ್ಟದ
ವಿಶ್ವ-ಮಾನವ
ವಿಶ್ವ-ಮಾ-ನ-ವ-ರಾದ
ವಿಶ್ವ-ರೂಪೀ
ವಿಶ್ವ-ವ-ಡ-ಗಿ-ದೆಯೋ
ವಿಶ್ವ-ವನ್ನೇ
ವಿಶ್ವ-ವಿ-ಖ್ಯಾತ
ವಿಶ್ವ-ವಿ-ಖ್ಯಾ-ತ-ರಾಗಿ
ವಿಶ್ವ-ವಿ-ಖ್ಯಾ-ತ-ರಾದ
ವಿಶ್ವ-ವಿ-ಖ್ಯಾ-ತಿ-ಯನ್ನು
ವಿಶ್ವ-ವಿ-ಜೇತ
ವಿಶ್ವ-ವಿ-ಜೇ-ತ-ನಾಗಿ
ವಿಶ್ವ-ವಿ-ಜೇ-ತ-ನಾದ
ವಿಶ್ವ-ವಿ-ಜೇ-ತ-ರಾ-ದದ್ದು
ವಿಶ್ವ-ವಿ-ದ್ಯಾ-ನಿ-ಲ-ಯದ
ವಿಶ್ವ-ವಿ-ದ್ಯಾ-ನಿ-ಲ-ಯ-ವನ್ನೆ
ವಿಶ್ವ-ವಿ-ದ್ಯಾ-ಲ-ಯ-ಗಳ
ವಿಶ್ವ-ವಿ-ದ್ಯಾ-ಲ-ಯ-ಗಳು
ವಿಶ್ವ-ವಿ-ದ್ಯಾ-ಲ-ಯದ
ವಿಶ್ವವೇ
ವಿಶ್ವ-ವ್ಯಾ-ಪ-ಕ-ತೆ-ಯನ್ನು
ವಿಶ್ವ-ಶಾಂ-ತಿ-ಯನ್ನು
ವಿಶ್ವ-ಸಂ-ಸ್ಥೆಯ
ವಿಶ್ವಾ-ತ್ಮಕ
ವಿಶ್ವಾ-ಮಿ-ತ್ರನ
ವಿಶ್ವಾ-ಮಿ-ತ್ರರು
ವಿಶ್ವಾಸ
ವಿಶ್ವಾ-ಸ-ಬೆಂ-ಬ-ಲ-ಗಳನ್ನು
ವಿಶ್ವಾ-ಸ-ಮ-ಮತೆ
ವಿಶ್ವಾ-ಸ-ಶ್ರ-ದ್ಧೆ-ಗಳಿಂದ
ವಿಶ್ವಾ-ಸ-ಸ-ಹಾ-ನು-ಭೂ-ತಿ-ಗಳನ್ನು
ವಿಶ್ವಾ-ಸ-ಸ್ನೇ-ಹ-ಗಳಲ್ಲಿ
ವಿಶ್ವಾ-ಸ-ಕ್ಕಾಗಿ
ವಿಶ್ವಾ-ಸಕ್ಕೆ
ವಿಶ್ವಾ-ಸ-ಗ-ಳ-ನ್ನಿ-ಟ್ಟು-ಕೊಂ-ಡಿದ್ದ
ವಿಶ್ವಾ-ಸ-ಗಳಿಂದ
ವಿಶ್ವಾ-ಸ-ಗ-ಳಿಂ-ದಲೇ
ವಿಶ್ವಾ-ಸ-ಗ-ಳಿನ್ನೂ
ವಿಶ್ವಾ-ಸ-ಗಳು
ವಿಶ್ವಾ-ಸದ
ವಿಶ್ವಾ-ಸ-ದಿಂದ
ವಿಶ್ವಾ-ಸ-ದಿಂ-ದಲೇ
ವಿಶ್ವಾ-ಸ-ದಿಂ-ದಾ-ಡಿದ
ವಿಶ್ವಾ-ಸನೇ
ವಿಶ್ವಾ-ಸ-ಪಾ-ತ್ರ-ರ-ಲ್ಲವೆ
ವಿಶ್ವಾ-ಸ-ಯುತ
ವಿಶ್ವಾ-ಸ-ವನ್ನು
ವಿಶ್ವಾ-ಸ-ವಾ-ಗಲಿ
ವಿಶ್ವಾ-ಸ-ವಿಟ್ಟು
ವಿಶ್ವಾ-ಸ-ವಿದೆ
ವಿಶ್ವಾ-ಸ-ವಿಲ್ಲ
ವಿಶ್ವಾ-ಸ-ವಿ-ಲ್ಲವೆ
ವಿಶ್ವಾ-ಸ-ವುಂ-ಟಾ-ಯಿತು
ವಿಶ್ವಾ-ಸ-ವೇನೂ
ವಿಶ್ವಾ-ಸಾ-ದ-ರ-ಗಳು
ವಿಶ್ವಾ-ಸಾ-ರ್ಹರು
ವಿಶ್ವಾಸಿ
ವಿಶ್ವಾ-ಸಿ-ಗರ
ವಿಶ್ವಾ-ಸಿ-ಗ-ರಲ್ಲಿ
ವಿಶ್ವಾ-ಸಿ-ಗ-ರ-ಲ್ಲೊ-ಬ್ಬ-ಳಾ-ಗಿ-ದ್ದಳು
ವಿಶ್ವಾ-ಸಿ-ಗ-ರಿಂದ
ವಿಶ್ವಾ-ಸಿ-ಗ-ರಿಗೆ
ವಿಶ್ವಾ-ಸಿ-ಗ-ರು-ಎ-ಲ್ಲರೂ
ವಿಶ್ವಾ-ಸಿ-ಗರೂ
ವಿಶ್ವಾ-ಸಿ-ಗ-ರೊಂ-ದಿಗೆ
ವಿಶ್ವಾ-ಸಿ-ಗ-ಳಾದ
ವಿಶ್ವಾ-ಸಿ-ಗಳಿಂದ
ವಿಶ್ವಾ-ಸಿ-ಗಳು
ವಿಶ್ವಾಸೀ
ವಿಷ
ವಿಷ-ದಂತೆ
ವಿಷ-ನ-ರಿ-ಗಳ
ವಿಷಮ
ವಿಷಯ
ವಿಷ-ಯ-ವಿ-ಚಾ-ರ-ಗಳ
ವಿಷ-ಯ-ಅ-ತ್ಯಂತ
ವಿಷ-ಯಕ್ಕೂ
ವಿಷ-ಯಕ್ಕೆ
ವಿಷ-ಯ-ಗಳ
ವಿಷ-ಯ-ಗ-ಳಂತೂ
ವಿಷ-ಯ-ಗ-ಳತ್ತ
ವಿಷ-ಯ-ಗಳನ್ನು
ವಿಷ-ಯ-ಗಳನ್ನೂ
ವಿಷ-ಯ-ಗಳನ್ನೆಲ್ಲ
ವಿಷ-ಯ-ಗ-ಳನ್ನೇ
ವಿಷ-ಯ-ಗ-ಳ-ನ್ನೊ-ಳ-ಗೊಂಡು
ವಿಷ-ಯ-ಗ-ಳ-ಲ್ಲದೆ
ವಿಷ-ಯ-ಗಳಲ್ಲಿ
ವಿಷ-ಯ-ಗ-ಳಲ್ಲೂ
ವಿಷ-ಯ-ಗ-ಳಲ್ಲೇ
ವಿಷ-ಯ-ಗ-ಳಾ-ಗಿ-ದ್ದುವು
ವಿಷ-ಯ-ಗ-ಳಾ-ವುವೂ
ವಿಷ-ಯ-ಗ-ಳಿಗೂ
ವಿಷ-ಯ-ಗ-ಳಿಗೆ
ವಿಷ-ಯ-ಗಳು
ವಿಷ-ಯ-ಗ-ಳು-ಜ-ನನ
ವಿಷ-ಯ-ಗ-ಳು-ಪೂಜ್ಯ
ವಿಷ-ಯ-ಗಳೂ
ವಿಷ-ಯ-ಗ-ಳೆಂ-ದರೆ
ವಿಷ-ಯ-ಗ-ಳೆಂದು
ವಿಷ-ಯ-ಗ-ಳೆಲ್ಲ
ವಿಷ-ಯದ
ವಿಷ-ಯ-ದಲ್ಲಿ
ವಿಷ-ಯ-ದ-ಲ್ಲಿಯೂ
ವಿಷ-ಯ-ದಲ್ಲೂ
ವಿಷ-ಯ-ದಲ್ಲೇ
ವಿಷ-ಯ-ದಾ-ಳ-ದಲ್ಲಿ
ವಿಷ-ಯ-ವನ್ನು
ವಿಷ-ಯ-ವನ್ನೂ
ವಿಷ-ಯ-ವ-ನ್ನೆಲ್ಲ
ವಿಷ-ಯ-ವನ್ನೇ
ವಿಷ-ಯ-ವಲ್ಲ
ವಿಷ-ಯ-ವ-ಲ್ಲವೆ
ವಿಷ-ಯ-ವಾಗಿ
ವಿಷ-ಯ-ವಾ-ಗಿತ್ತು
ವಿಷ-ಯ-ವಾ-ಗಿಯೇ
ವಿಷ-ಯ-ವಾ-ಗೇನೋ
ವಿಷ-ಯ-ವಾ-ದರೆ
ವಿಷ-ಯವು
ವಿಷ-ಯವೂ
ವಿಷ-ಯ-ವೆಂದರೆ
ವಿಷ-ಯವೇ
ವಿಷ-ಯ-ವೇನು
ವಿಷ-ಯ-ವೇನೆಂದರೆ
ವಿಷ-ಯ-ವೇ-ನೆಂದು
ವಿಷಾದ
ವಿಷಾ-ದದ
ವಿಷಾ-ದ-ದಿಂದ
ವಿಷಾ-ದ-ವನ್ನು
ವಿಷಾ-ದ-ವಾ-ಯಿತು
ವಿಷಾ-ದಿ-ಸ-ಬೇ-ಕಾಗಿ
ವಿಷಾ-ದಿ-ಸ-ಬೇಡಿ
ವಿಷ್ಣು-ಪಾ-ದ-ಪ-ದ್ಮ-ವನ್ನು
ವಿಷ್ಣು-ವ-ನ್ನಾಗಿ
ವಿಷ್ಣುವೊ
ವಿಸ-ರ್ಜ-ನೆ-ಗೊಂ-ಡಿತು
ವಿಸ-ರ್ಜ-ನೆ-ಯಾ-ದಾಗ
ವಿಸ-ರ್ಜಿ-ಸ-ಲಾ-ಯಿತು
ವಿಸ್ತ-ರಿ-ಸಲು
ವಿಸ್ತ-ರಿ-ಸುತ್ತ
ವಿಸ್ತ-ರಿ-ಸು-ತ್ತ-ಲಿದೆ
ವಿಸ್ತ-ರಿ-ಸುವ
ವಿಸ್ತಾ-ರದ
ವಿಸ್ತಾ-ರ-ವಾದ
ವಿಸ್ತಾ-ರವೂ
ವಿಸ್ಮಯ
ವಿಸ್ಮ-ಯ-ಗೊ-ಳ್ಳು-ವುದು
ವಿಸ್ಮ-ಯ-ಮೂ-ಕ-ನಾಗಿ
ವಿಸ್ಮ-ಯ-ಮೂ-ಕ-ರ-ನ್ನಾ-ಗಿ-ಸಿತು
ವಿಸ್ಮ-ಯ-ಮೂ-ಕ-ರಾ-ಗು-ತ್ತಿ-ದ್ದರು
ವಿಸ್ಮ-ಯ-ಮೂ-ಕ-ರಾ-ದರು
ವಿಸ್ಮ-ಯಾ-ದ್ಭುತ
ವಿಸ್ಮಿ-ತ-ಮೂ-ಕ-ರಾ-ಗಿ-ದ್ದರು
ವಿಸ್ಮಿ-ತ-ರ-ನ್ನಾ-ಗಿ-ಸು-ತ್ತಿತ್ತು
ವಿಸ್ಮಿ-ತ-ರಾ-ಗ-ದಿ-ರು-ವಂ-ತಿಲ್ಲ
ವಿಸ್ಮಿ-ತ-ಳಾ-ದಳು
ವಿಹಂ-ಗಮ
ವಿಹ-ರಿ-ಸಿದ
ವಿಹ-ರಿ-ಸಿ-ದ್ದರು
ವಿಹಾರ
ವಿಹಾ-ರ-ಕ್ಕಾಗಿ
ವಿಹಾ-ರ-ವನ್ನು
ವಿಹಾ-ರಾ-ರ್ಥ-ವಾಗಿ
ವಿಹಿ-ತ-ವೆಂದು
ವಿಹೀ-ನ-ವಾ-ಗಿದ್ದ
ವೀಕ್ಷ-ಕರ
ವೀಕ್ಷ-ಣಾ-ಲ-ಯಕ್ಕೆ
ವೀಕ್ಷಿ-ಸಲು
ವೀಕ್ಷಿಸಿ
ವೀಕ್ಷಿ-ಸಿದ
ವೀಕ್ಷಿ-ಸಿ-ದರು
ವೀಕ್ಷಿ-ಸಿ-ದಳು
ವೀಕ್ಷಿ-ಸಿ-ರ-ಲಿಲ್ಲ
ವೀಕ್ಷಿ-ಸುತ್ತ
ವೀಕ್ಷಿ-ಸು-ತ್ತಿದ್ದ
ವೀಕ್ಷಿ-ಸು-ತ್ತಿ-ದ್ದರು
ವೀಕ್ಷಿ-ಸು-ತ್ತಿ-ದ್ದ-ವ-ರಿಗೆ
ವೀಕ್ಷಿ-ಸುವ
ವೀಕ್ಷಿ-ಸು-ವಾಗ
ವೀಕ್ಷಿ-ಸು-ವು-ದ-ರಲ್ಲಿ
ವೀಕ್ಷಿ-ಸು-ವು-ದೆಂ-ದರೆ
ವೀಣೆ
ವೀರ
ವೀರ-ಪು-ತ್ರ-ನನ್ನು
ವೀರ-ಮ-ರಣ
ವೀರರ
ವೀರ-ರಂತೆ
ವೀರ-ರನ್ನು
ವೀರ್ಯ-ವಂತ
ವೀರ್ಯ-ವ-ತ್ತ-ರ-ವಾದ
ವೀಲರ್
ವುದ-ಕ್ಕಾಗಿ
ವುದ-ಕ್ಕಾ-ಗು-ವು-ದಿಲ್ಲ
ವುದ-ಕ್ಕಿಂತ
ವುದಕ್ಕೂ
ವುದನ್ನು
ವುದರ
ವುದ-ರಲ್ಲಿ
ವುದ-ರಿಂದ
ವುದಾಗಿ
ವುದಾ-ಗಿಯೂ
ವುದಾ-ದರೆ
ವುದಿತ್ತು
ವುದಿಲ್ಲ
ವುದು
ವುದು-ಇ-ವಿಷ್ಟು
ವುದೂ
ವುದೆಂ-ದರೆ
ವುದೆಂದು
ವುದೇ
ವುದೇನೋ
ವುವೂ
ವೃಕ್ಷ-ಗಳ
ವೃಕ್ಷ-ಗಳು
ವೃಕ್ಷದ
ವೃಕ್ಷ-ವನ್ನು
ವೃಕ್ಷ-ಸಂ-ಪ-ತ್ತಿ-ನಿಂದ
ವೃತ್ತದ
ವೃತ್ತ-ಪ-ತ್ರಿ-ಕೆ-ಗಳ
ವೃತ್ತ-ಪ-ತ್ರಿ-ಕೆ-ಗಳನ್ನು
ವೃತ್ತ-ಪ-ತ್ರಿ-ಕೆ-ಗ-ಳಿಗೆ
ವೃತ್ತ-ಪ-ತ್ರಿ-ಕೆ-ಗಳು
ವೃತ್ತ-ಪ-ತ್ರಿ-ಕೆ-ಯಲ್ಲಿ
ವೃತ್ತಾಂ-ತ-ವನ್ನು
ವೃತ್ತಾಂ-ತ-ವೊಂ-ದನ್ನು
ವೃತ್ತಿ-ಯನ್ನು
ವೃತ್ತಿ-ಯ-ವ-ರಿಗೂ
ವೃತ್ತಿ-ಯಿಂದ
ವೃತ್ರಾ-ಸುರ
ವೃಥಾ
ವೃದ್ದ-ನೊಬ್ಬ
ವೃದ್ಧ
ವೃದ್ಧ-ರಾದ
ವೃದ್ಧ-ರಿಗೂ
ವೃದ್ಧ-ರಿಗೆ
ವೃದ್ಧರು
ವೃದ್ಧೆಯ
ವೃದ್ಧೆ-ಯನ್ನು
ವೆಂಕ-ಟ-ರಾವ್
ವೆಂದರೆ
ವೆಂದಾ-ದರೂ
ವೆಂದು
ವೆಂಬ
ವೆಂಬಂತೆ
ವೆಂಬು-ದಂತೂ
ವೆಂಬುದು
ವೆಚ್ಚ
ವೆಚ್ಚಕ್ಕೆ
ವೆಚ್ಚ-ವನ್ನು
ವೆಸೂ-ವಿ-ಯಸ್
ವೇಗಕ್ಕೆ
ವೇಗ-ದಿಂದ
ವೇಗ-ವಾಗಿ
ವೇದ
ವೇದ-ಶಾ-ಸ್ತ್ರ-ಗಳನ್ನೆಲ್ಲ
ವೇದ-ಕಾ-ಲ-ದ-ವ-ರೆಗೆ
ವೇದ-ಕಾ-ಲ-ದಿಂ-ದಲೂ
ವೇದ-ಗಳ
ವೇದ-ಗಳನ್ನು
ವೇದ-ಗಳಲ್ಲಿ
ವೇದ-ಗ-ಳಿಗೆ
ವೇದ-ಗಳು
ವೇದ-ಗಳೇ
ವೇದ-ಗೀ-ದ-ಗಳನ್ನು
ವೇದ-ಘೋ-ಷ-ಗಳ
ವೇದದ
ವೇದ-ಧ-ರ್ಮ-ದಿಂ-ದಲೇ
ವೇದ-ನೆ-ಯನ್ನು
ವೇದ-ಪಾ-ರಂ-ಗ-ತ-ರೆಂಬ
ವೇದ-ಮಂ-ತ್ರ-ಗಳ
ವೇದ-ಮಂ-ತ್ರ-ಗಳನ್ನು
ವೇದ-ಮಂ-ತ್ರ-ವನ್ನು
ವೇದ-ಮ-ಹ-ರ್ಷಿ-ಗಳ
ವೇದ-ವನ್ನು
ವೇದ-ವ-ನ್ನೋ-ದು-ವು-ದ-ರಲ್ಲಿ
ವೇದ-ವಾ-ಕ್ಯ-ಗ-ಳಿಗೆ
ವೇದ-ವಾ-ಕ್ಯ-ವೆಂದು
ವೇದ-ವಾ-ಣಿ-ಯೊಂದು
ವೇದ-ವೇ-ದಾಂ-ಗ-ಗಳಲ್ಲಿ
ವೇದ-ವೇ-ದಾಂ-ತ-ಗಳ
ವೇದ-ವೇ-ದಾಂ-ತ-ಗಳನ್ನು
ವೇದ-ಶಾ-ಸ್ತ್ರ-ಗ-ಳಿ-ಗೆಲ್ಲ
ವೇದ-ಸ್ವ-ರೂ-ಪ-ರಾದ
ವೇದಾಂ
ವೇದಾಂತ
ವೇದಾಂ-ತ-ಕೃ-ತಿ-ಗಳನ್ನು
ವೇದಾಂ-ತ-ಕೇ-ಸರಿ
ವೇದಾಂ-ತ-ಗಳ
ವೇದಾಂ-ತ-ಜ್ಯೋ-ತಿ-ಯನ್ನು
ವೇದಾಂ-ತ-ತತ್ತ್ವ
ವೇದಾಂ-ತದ
ವೇದಾಂ-ತ-ದಲ್ಲಿ
ವೇದಾಂ-ತ-ದಲ್ಲೇ
ವೇದಾಂ-ತ-ದಿಂದ
ವೇದಾಂ-ತ-ಧರ್ಮ
ವೇದಾಂ-ತ-ಧ-ರ್ಮದ
ವೇದಾಂ-ತ-ಪ್ರ-ಸಾರ
ವೇದಾಂ-ತ-ಪ್ರ-ಸಾ-ರ-ಕಾ-ರ್ಯಕ್ಕೆ
ವೇದಾಂ-ತ-ಪ್ರ-ಸಾ-ರದ
ವೇದಾಂ-ತ-ಭಾ-ವ-ನೆ-ಗಳನ್ನು
ವೇದಾಂ-ತ-ವನ್ನು
ವೇದಾಂ-ತ-ವನ್ನೇ
ವೇದಾಂ-ತ-ವ-ನ್ನೊಳ
ವೇದಾಂ-ತವು
ವೇದಾಂ-ತವೆ
ವೇದಾಂ-ತ-ವೆಂದರೆ
ವೇದಾಂ-ತ-ವೆಂಬ
ವೇದಾಂ-ತವೇ
ವೇದಾಂ-ತ-ವೊಂದೇ
ವೇದಾಂ-ತ-ಸಿ-ದ್ಧಾಂ-ತ-ಗ-ಳೆಂ-ಬುದು
ವೇದಾಂ-ತ-ಸೂ-ತ್ರ-ಗಳು
ವೇದಾಂತಿ
ವೇದಾಂ-ತಿ-ಗಳ
ವೇದಾಂ-ತಿ-ಗಳು
ವೇದಾಂ-ತಿ-ಗಳೂ
ವೇದಾಂ-ತಿ-ಯಿ-ರ-ಬ-ಹುದು
ವೇದಾ-ಧ್ಯ-ಯನ
ವೇದಾ-ಧ್ಯ-ಯ-ನ-ವನ್ನು
ವೇದಿ-ಕೆಯ
ವೇದಿ-ಕೆ-ಯನ್ನೇ
ವೇದಿ-ಕೆ-ಯ-ನ್ನೇರಿ
ವೇದಿ-ಕೆ-ಯ-ನ್ನೇ-ರುವ
ವೇದಿ-ಕೆ-ಯಲ್ಲಿ
ವೇದೋ-ಕ್ತಿ-ಗಳನ್ನು
ವೇದೋಪ
ವೇದೋ-ಪ-ನಿ-ಷ-ತ್ತು-ಗಳನ್ನು
ವೇದೋ-ಪ-ನಿ-ಷ-ತ್ತು-ಗಳಲ್ಲಿ
ವೇದೋ-ಪ-ನಿ-ಷ-ತ್ತು-ಗಳು
ವೇದ್ಯ-ವಾ-ಗಿತ್ತು
ವೇನಿ-ದ್ದರೂ
ವೇನು
ವೇನೆಂ-ದರೆ
ವೇನೆಂದು
ವೇಲು
ವೇಲ್ಯೂ
ವೇಲ್ಸ್
ವೇಲ್ಸ್ನಿಂದ
ವೇಳೆ
ವೇಳೆ-ಗಳಲ್ಲಿ
ವೇಳೆ-ಗಾ-ಗಲೇ
ವೇಳೆಗೆ
ವೇಳೆ-ಯನ್ನು
ವೇಳೆ-ಯ-ನ್ನೆಲ್ಲ
ವೇಳೆ-ಯಲ್ಲಿ
ವೇಳೆ-ಯಾ-ಯಿತು
ವೇಶ್ಯಾ-ವಾ-ಟಿ-ಕೆ-ಯ-ಲ್ಲಿ-ರುವ
ವೇಶ್ಯಾ-ವೃ-ತ್ತಿ-ಗಿ-ಳಿದ
ವೇಶ್ಯೆ-ಯ-ರಿಗೆ
ವೇಶ್ಯೆ-ಯರು
ವೇಶ್ಯೆ-ಯೊ-ಬ್ಬಳು
ವೇಷ
ವೇಷ-ದಲ್ಲಿ
ವೇಷ-ದಿಂದ
ವೇಷ-ಭೂ-ಷಣ
ವೇಷ-ಭೂ-ಷ-ಣ-ಗಳು
ವೈ
ವೈ-ಯ-ಕ್ತಿ-ಕ-ತೆ-ಗ-ಳ-ನ್ನು
ವೈಕಾಫ್
ವೈಕಾಫ್ಳ
ವೈಚಾ-ರಿಕ
ವೈಚಾ-ರಿ-ಕ-ತೆಯ
ವೈಜ-ಯಂ-ತಿ-ಯನ್ನು
ವೈಜ್ಞಾ-ನಿಕ
ವೈದಿಕ
ವೈದಿ-ಕ-ರ-ಲ್ಲದೆ
ವೈದಿ-ಕರು
ವೈದಿ-ಕ-ವಾ-ದಂ-ತಹ
ವೈದ್ಯ
ವೈದ್ಯ-ಕೀಯ
ವೈದ್ಯನ
ವೈದ್ಯ-ನಾ-ಗಿ-ರಲಿ
ವೈದ್ಯ-ನಾ-ಥಕ್ಕೆ
ವೈದ್ಯ-ನಾ-ಥ-ದಲ್ಲಿ
ವೈದ್ಯ-ನಾ-ಥ-ದ-ಲ್ಲಿದ್ದ
ವೈದ್ಯ-ನಾ-ಥ-ದ-ಲ್ಲಿ-ದ್ದಾಗ
ವೈದ್ಯ-ನಾ-ಥ-ದಿಂದ
ವೈದ್ಯ-ನೊ-ಬ್ಬನ
ವೈದ್ಯರ
ವೈದ್ಯ-ರಾದ
ವೈದ್ಯ-ರಿಗೆ
ವೈದ್ಯ-ರಿ-ದ್ದೀ-ರಲ್ಲ
ವೈದ್ಯರು
ವೈದ್ಯರೂ
ವೈದ್ಯ-ಸ್ನೇ-ಹಿ-ತ-ರಾದ
ವೈಧೀ
ವೈಭವ
ವೈಭ-ವ-ಗಳ
ವೈಭ-ವ-ಗಳನ್ನು
ವೈಭ-ವ-ಗಳನ್ನೆಲ್ಲ
ವೈಭ-ವದ
ವೈಭ-ವ-ದಿಂದ
ವೈಭ-ವ-ಪೂರ್ಣ
ವೈಭ-ವ-ಪೂ-ರ್ಣ-ವಾಗಿ
ವೈಭ-ವ-ಪೂ-ರ್ಣ-ವಾದ
ವೈಭ-ವ-ಯುತ
ವೈಭ-ವ-ಯು-ತ-ಳಾಗಿ
ವೈಭ-ವ-ಯು-ತ-ವಾದ
ವೈಭ-ವ-ವ-ನ್ನಾ-ಗಲಿ
ವೈಭ-ವ-ವನ್ನು
ವೈಭ-ವ-ವನ್ನೂ
ವೈಭ-ವೋ-ಪೇ-ತ-ವಾದ
ವೈಯ
ವೈಯ-ಕ್ತಿಕ
ವೈಯ-ಕ್ತಿ-ಕತೆ
ವೈಯ-ಕ್ತಿ-ಕ-ತೆ-ಯನ್ನು
ವೈಯ-ಕ್ತಿ-ಕ-ತೆ-ಯನ್ನೆ
ವೈಯ-ಕ್ತಿ-ಕ-ತೆ-ಯೆಂ-ಬುದೇ
ವೈಯ-ಕ್ತಿ-ಕ-ವಾಗಿ
ವೈಯ-ಕ್ತಿ-ಕ-ವಾ-ದದ್ದು
ವೈರ
ವೈರಾಗ್ಯ
ವೈರಾ-ಗ್ಯದ
ವೈರಾ-ಗ್ಯ-ದಲ್ಲಿ
ವೈರಾ-ಗ್ಯ-ಶೀ-ಲ-ರಾದ
ವೈರಾ-ಗ್ಯ-ಸೂ-ಚಕ
ವೈರಿ-ಗ-ಳಾದ
ವೈವಿಧ್ಯ
ವೈವಿ-ಧ್ಯ-ಗಳ
ವೈವಿ-ಧ್ಯ-ಪೂರ್ಣ
ವೈವಿ-ಧ್ಯ-ಮಯ
ವೈವಿ-ಧ್ಯ-ಮ-ಯ-ವಾಗಿ
ವೈವಿ-ಧ್ಯ-ಮ-ಯ-ವಾದ
ವೈವಿ-ಧ್ಯ-ಮ-ಯ-ವಾ-ದವು
ವೈಶಾಲ್ಯ
ವೈಶಾ-ಲ್ಯ-ದಲ್ಲಿ
ವೈಶಾ-ಲ್ಯ-ದೊಂ-ದಿಗೆ
ವೈಶಾ-ಲ್ಯ-ವಿತ್ತು
ವೈಶಿಷ್ಟ್ಯ
ವೈಶಿ-ಷ್ಟ್ಯ-ಗಳನ್ನು
ವೈಶಿ-ಷ್ಟ್ಯದ
ವೈಶಿ-ಷ್ಟ್ಯ-ಪೂ-ರ್ಣ-ವಾ-ಗಿ-ತ್ತೆಂ-ಬುದು
ವೈಶಿ-ಷ್ಟ್ಯ-ವಿದೆ
ವೈಶಿ-ಷ್ಟ್ಯ-ವಿ-ರ-ಬ-ಹುದು
ವೈಶಿ-ಷ್ಟ್ಯ-ವೆಂದರೆ
ವೈಶಿ-ಷ್ಟ್ಯ-ವೇನೆಂದರೆ
ವೈಶ್ಯ-ರಿಗೂ
ವೈಷ್ಣವ
ವೈಷ್ಣ-ವರ
ವೈಸ್ರಾ-ಯ-ನಾದ
ವೊಂದಕ್ಕೆ
ವೊಂದನ್ನು
ವೊಂದರ
ವೊಂದಿ-ರು-ತ್ತದೆ
ವೊಂದು
ವೊಂದೇ
ವೊಮ್ಮೆ
ವ್ಯಂಗ್ಯ-ವಾಗಿ
ವ್ಯಕ್ತ
ವ್ಯಕ್ತ-ಗೊಂ-ಡಂ-ತಹ
ವ್ಯಕ್ತ-ಗೊ-ಳಿ-ಸಲು
ವ್ಯಕ್ತ-ಗೊ-ಳಿ-ಸಿ-ದರು
ವ್ಯಕ್ತ-ಗೊ-ಳಿ-ಸು-ತ್ತದೆ
ವ್ಯಕ್ತ-ಗೊ-ಳ್ಳು-ತ್ತಿತ್ತು
ವ್ಯಕ್ತ-ತೆಯ
ವ್ಯಕ್ತ-ಪ-ಡಿ-ಸ-ಬ-ಲ್ಲುದು
ವ್ಯಕ್ತ-ಪ-ಡಿ-ಸ-ಬ-ಹು-ದಂತೆ
ವ್ಯಕ್ತ-ಪ-ಡಿ-ಸ-ಲಿಲ್ಲ
ವ್ಯಕ್ತ-ಪ-ಡಿ-ಸಲು
ವ್ಯಕ್ತ-ಪ-ಡಿಸಿ
ವ್ಯಕ್ತ-ಪ-ಡಿ-ಸಿತು
ವ್ಯಕ್ತ-ಪ-ಡಿ-ಸಿದ
ವ್ಯಕ್ತ-ಪ-ಡಿ-ಸಿದಂ
ವ್ಯಕ್ತ-ಪ-ಡಿ-ಸಿ-ದರು
ವ್ಯಕ್ತ-ಪ-ಡಿ-ಸಿ-ದಳು
ವ್ಯಕ್ತ-ಪ-ಡಿ-ಸಿ-ದಾಗ
ವ್ಯಕ್ತ-ಪ-ಡಿ-ಸಿ-ದ್ದರು
ವ್ಯಕ್ತ-ಪ-ಡಿ-ಸಿ-ದ್ದಾನೆ
ವ್ಯಕ್ತ-ಪ-ಡಿ-ಸಿ-ದ್ದಾ-ರೆಂದೂ
ವ್ಯಕ್ತ-ಪ-ಡಿ-ಸಿ-ರ-ಲಿಲ್ಲ
ವ್ಯಕ್ತ-ಪ-ಡಿ-ಸಿ-ರುವ
ವ್ಯಕ್ತ-ಪ-ಡಿಸು
ವ್ಯಕ್ತ-ಪ-ಡಿ-ಸುತ್ತ
ವ್ಯಕ್ತ-ಪ-ಡಿ-ಸುವ
ವ್ಯಕ್ತ-ಪ್ರ-ಪಂ-ಚ-ದಲ್ಲಿ
ವ್ಯಕ್ತ-ವಾ-ಗ-ಬಲ್ಲ
ವ್ಯಕ್ತ-ವಾ-ಗಲು
ವ್ಯಕ್ತ-ವಾಗು
ವ್ಯಕ್ತ-ವಾ-ಗು-ತ್ತದೆ
ವ್ಯಕ್ತ-ವಾ-ಗು-ತ್ತ-ದೆ-ನ-ನ್ನಂತೆ
ವ್ಯಕ್ತ-ವಾ-ಗು-ವಂ-ತಹ
ವ್ಯಕ್ತ-ವಾ-ಗು-ವಂತೆ
ವ್ಯಕ್ತ-ವಾ-ದರೂ
ವ್ಯಕ್ತ-ವಾ-ಯಿತು
ವ್ಯಕ್ತಿ
ವ್ಯಕ್ತಿ-ಗಳ
ವ್ಯಕ್ತಿ-ಗಳನ್ನು
ವ್ಯಕ್ತಿ-ಗಳಲ್ಲಿ
ವ್ಯಕ್ತಿ-ಗ-ಳಾ-ಗ-ಬ-ಯ-ಸಿ-ದರೆ
ವ್ಯಕ್ತಿ-ಗ-ಳಾ-ಗು-ವ-ವ-ರೆಗೆ
ವ್ಯಕ್ತಿ-ಗಳಿಂದ
ವ್ಯಕ್ತಿ-ಗ-ಳಿ-ಗಾಗಿ
ವ್ಯಕ್ತಿ-ಗ-ಳಿಗೆ
ವ್ಯಕ್ತಿ-ಗ-ಳಿ-ದ್ದಾರೆ
ವ್ಯಕ್ತಿ-ಗ-ಳಿಲ್ಲ
ವ್ಯಕ್ತಿ-ಗಳು
ವ್ಯಕ್ತಿ-ಗ-ಳೆ-ನ್ನಿ-ಸಿ-ಕೊಂ-ಡ-ವರು
ವ್ಯಕ್ತಿ-ಗ-ಳೆಲ್ಲ
ವ್ಯಕ್ತಿ-ಗಳೇ
ವ್ಯಕ್ತಿ-ಗ-ಳೊಂ-ದಿಗೆ
ವ್ಯಕ್ತಿಗೂ
ವ್ಯಕ್ತಿಗೆ
ವ್ಯಕ್ತಿ-ತ್ತ್ವ-ದಿಂದ
ವ್ಯಕ್ತಿತ್ವ
ವ್ಯಕ್ತಿ-ತ್ವ-ಜೀ-ವ-ನ-ಗಳ
ವ್ಯಕ್ತಿ-ತ್ವ-ಸ್ವ-ಭಾ-ವ-ಗ-ಳಿಗೆ
ವ್ಯಕ್ತಿ-ತ್ವಕ್ಕೆ
ವ್ಯಕ್ತಿ-ತ್ವ-ಗಳ
ವ್ಯಕ್ತಿ-ತ್ವ-ಗಳಲ್ಲಿ
ವ್ಯಕ್ತಿ-ತ್ವ-ಗಳು
ವ್ಯಕ್ತಿ-ತ್ವದ
ವ್ಯಕ್ತಿ-ತ್ವ-ದಲ್ಲಿ
ವ್ಯಕ್ತಿ-ತ್ವ-ದಲ್ಲೇ
ವ್ಯಕ್ತಿ-ತ್ವ-ದಿಂದ
ವ್ಯಕ್ತಿ-ತ್ವ-ವ-ನ್ನಿ-ರಿ-ಸ-ಬೇಕು
ವ್ಯಕ್ತಿ-ತ್ವ-ವನ್ನು
ವ್ಯಕ್ತಿ-ತ್ವ-ವಿದೆ
ವ್ಯಕ್ತಿ-ತ್ವ-ವಿ-ರು-ವಂತೆ
ವ್ಯಕ್ತಿ-ತ್ವ-ವಿಲ್ಲ
ವ್ಯಕ್ತಿ-ತ್ವವು
ವ್ಯಕ್ತಿ-ತ್ವ-ವೆಂ-ಬುದು
ವ್ಯಕ್ತಿ-ತ್ವವೇ
ವ್ಯಕ್ತಿ-ತ್ವ-ವೊಂ-ದನ್ನು
ವ್ಯಕ್ತಿ-ನಿ-ರ್ಮಾಣ
ವ್ಯಕ್ತಿ-ನಿ-ರ್ಮಾ-ಣ-ಕಾ-ರಿ-ಯಾದ
ವ್ಯಕ್ತಿ-ನಿ-ರ್ಮಾ-ಣಕ್ಕೆ
ವ್ಯಕ್ತಿ-ನಿ-ರ್ಮಾ-ಣವೇ
ವ್ಯಕ್ತಿ-ನಿ-ಷ್ಠ-ರ-ನ್ನಾ-ಗಿ-ಸು-ವು-ದ-ಕ್ಕಿಂತ
ವ್ಯಕ್ತಿಯ
ವ್ಯಕ್ತಿ-ಯ-ನ್ನಾಗಿ
ವ್ಯಕ್ತಿ-ಯನ್ನು
ವ್ಯಕ್ತಿ-ಯನ್ನೂ
ವ್ಯಕ್ತಿ-ಯಲ್ಲ
ವ್ಯಕ್ತಿ-ಯಲ್ಲಿ
ವ್ಯಕ್ತಿ-ಯಾಗಿ
ವ್ಯಕ್ತಿ-ಯಾ-ಗಿರ
ವ್ಯಕ್ತಿ-ಯಾದ
ವ್ಯಕ್ತಿಯೂ
ವ್ಯಕ್ತಿ-ಯೆಂ-ದರೆ
ವ್ಯಕ್ತಿ-ಯೆಂದು
ವ್ಯಕ್ತಿಯೇ
ವ್ಯಕ್ತಿ-ಯೊಂ-ದಿಗೆ
ವ್ಯಕ್ತಿ-ಯೊಬ್ಬ
ವ್ಯಕ್ತಿ-ಯೊ-ಬ್ಬನ
ವ್ಯಕ್ತಿ-ಯೊ-ಬ್ಬ-ನನ್ನು
ವ್ಯಕ್ತಿ-ಯೊ-ಬ್ಬ-ರನ್ನು
ವ್ಯಕ್ತಿ-ಯೊ-ಬ್ಬ-ರಿಂದ
ವ್ಯಕ್ತಿ-ಯೊ-ಬ್ಬಳು
ವ್ಯಕ್ತಿ-ವ್ಯಕ್ತಿ
ವ್ಯಕ್ತಿ-ವ್ಯ-ಕ್ತಿ-ಗಳ
ವ್ಯಗ್ರ-ಗೊಂಡು
ವ್ಯಗ್ರ-ವಾಗಿ
ವ್ಯತಿ-ರಿಕ್ತ
ವ್ಯತಿ-ರಿ-ಕ್ತ-ತೆ-ಯಾ-ಗಲಿ
ವ್ಯತಿ-ರಿ-ಕ್ತ-ವಾಗಿ
ವ್ಯತಿ-ರಿ-ಕ್ತ-ವಾ-ದದ್ದು
ವ್ಯತಿ-ರಿ-ಕ್ತ-ವಾ-ದಾ-ಗ-ಲೆಲ್ಲ
ವ್ಯತ್ಯಾಸ
ವ್ಯತ್ಯಾ-ಸ-ವನ್ನು
ವ್ಯತ್ಯಾ-ಸ-ವನ್ನೂ
ವ್ಯತ್ಯಾ-ಸ-ವಿ-ರು-ವುದು
ವ್ಯತ್ಯಾ-ಸ-ವಿಷ್ಟೆ
ವ್ಯತ್ಯಾ-ಸವೇ
ವ್ಯತ್ಯಾ-ಸ-ವೇನು
ವ್ಯತ್ಯಾ-ಸ-ವೇನೂ
ವ್ಯತ್ಯಾ-ಸ-ವೇ-ನೆಂದು
ವ್ಯಥೆ-ಯಾ-ಯಿತು
ವ್ಯಭಿ-ಚಾ-ರವೇ
ವ್ಯಭಿ-ಚಾ-ರಿ-ಣಿಯೇ
ವ್ಯಯ-ಮಾ-ಡು-ತ್ತಿ-ದ್ದಾರೆ
ವ್ಯಯ-ಮಾ-ಡು-ತ್ತಿವೆ
ವ್ಯಯಿ-ಸಿ-ದ-ವ-ನಲ್ಲ
ವ್ಯಯಿ-ಸು-ತ್ತಾ-ರೆ-ಮ-ಕ್ಕಳು
ವ್ಯರ್ಥ
ವ್ಯರ್ಥ-ಗೊ-ಳಿ-ಸಿ-ದೆ-ವೆಂ-ದಾ-ಯಿ-ತಲ್ಲ
ವ್ಯರ್ಥ-ವಾ-ಗ-ಬೇ-ಕೆಂ-ದರೆ
ವ್ಯರ್ಥ-ವಾ-ಗ-ಲಿ-ಲ್ಲ-ವೆಂದು
ವ್ಯರ್ಥ-ವಾಗಿ
ವ್ಯರ್ಥ-ವಾ-ಗು-ವಂ-ತಿಲ್ಲ
ವ್ಯರ್ಥ-ವೆಂದ
ವ್ಯವ-ಧಾ-ನ-ವಿಲ್ಲ
ವ್ಯವ-ಧಾ-ನ-ವೆಲ್ಲಿ
ವ್ಯವ-ಸಾಯ
ವ್ಯವ-ಸ್ಥಿತ
ವ್ಯವ-ಸ್ಥಿ-ತ-ಗೊ-ಳಿ-ಸುವ
ವ್ಯವ-ಸ್ಥಿ-ತ-ಗೊ-ಳಿ-ಸು-ವುದು
ವ್ಯವಸ್ಥೆ
ವ್ಯವ-ಸ್ಥೆ-ಗಳನ್ನು
ವ್ಯವ-ಸ್ಥೆ-ಗಳನ್ನೂ
ವ್ಯವ-ಸ್ಥೆಗೆ
ವ್ಯವ-ಸ್ಥೆ-ಯನ್ನು
ವ್ಯವ-ಸ್ಥೆ-ಯನ್ನೋ
ವ್ಯವ-ಸ್ಥೆ-ಯಾ-ಗಿತ್ತು
ವ್ಯವ-ಸ್ಥೆ-ಯಾ-ಗು-ವ-ವ-ರೆಗೆ
ವ್ಯವ-ಸ್ಥೆ-ಯಿಲ್ಲ
ವ್ಯವ-ಸ್ಥೆ-ಯಿ-ಲ್ಲ-ದಿದ್ದ
ವ್ಯವ-ಹ-ರಿಸ
ವ್ಯವ-ಹ-ರಿ-ಸು-ತ್ತಿ-ದ್ದರು
ವ್ಯವ-ಹ-ರಿ-ಸು-ವುದು
ವ್ಯವ-ಹಾರ
ವ್ಯವ-ಹಾ-ರ-ಧೋ-ರ-ಣೆ-ಗಳನ್ನು
ವ್ಯವ-ಹಾ-ರ-ಗಳ
ವ್ಯವ-ಹಾ-ರ-ಗ-ಳಿಗೂ
ವ್ಯವ-ಹಾ-ರದ
ವ್ಯವ-ಹಾ-ರ-ದಲ್ಲಿ
ವ್ಯವ-ಹಾ-ರ-ವನ್ನು
ವ್ಯಾಂಕೋ-ವರ್ಗೆ
ವ್ಯಾಂಕೋ-ವ-ರ್ನಿಂದ
ವ್ಯಾಕ-ರಣ
ವ್ಯಾಕ-ರ-ಣದ
ವ್ಯಾಕ-ರ-ಣ-ದಂ-ತಹ
ವ್ಯಾಕುಲ
ವ್ಯಾಕು-ಲ-ಗೊಂಡು
ವ್ಯಾಕು-ಲತೆ
ವ್ಯಾಕು-ಲನೂ
ವ್ಯಾಕು-ಲಿ-ತ-ರಾದ
ವ್ಯಾಖ್ಯಾನ
ವ್ಯಾಖ್ಯಾ-ನ-ಕಾ-ರ-ರಾಗಿ
ವ್ಯಾಖ್ಯಾ-ನ-ಗಳ
ವ್ಯಾಖ್ಯಾ-ನ-ಗಳನ್ನೆಲ್ಲ
ವ್ಯಾಖ್ಯಾ-ನ-ಗ-ಳ-ನ್ನೊ-ಳ-ಗೊಂ-ಡಿವೆ
ವ್ಯಾಖ್ಯಾ-ನ-ಗಳಿಂದ
ವ್ಯಾಖ್ಯಾ-ನ-ಗ-ಳಿ-ಗಾಗಿ
ವ್ಯಾಖ್ಯಾ-ನ-ಗಳು
ವ್ಯಾಖ್ಯಾ-ನ-ವನ್ನು
ವ್ಯಾಖ್ಯಾ-ನ-ವಾ-ಗಿತ್ತು
ವ್ಯಾಖ್ಯಾ-ನ-ವಾ-ಚ-ಸ್ಪತಿ
ವ್ಯಾಖ್ಯಾ-ನ-ವೊಂ-ದನ್ನು
ವ್ಯಾಖ್ಯಾ-ನಿ-ಸಲು
ವ್ಯಾಖ್ಯಾ-ನಿಸಿ
ವ್ಯಾಖ್ಯಾ-ನಿ-ಸು-ತ್ತಿ-ದ್ದರು
ವ್ಯಾಖ್ಯಾ-ನಿ-ಸು-ತ್ತಿ-ರು-ವಾಗ
ವ್ಯಾಖ್ಯಾ-ನಿ-ಸುವ
ವ್ಯಾಖ್ಯೆ
ವ್ಯಾಘ್ರ
ವ್ಯಾಘ್ರದ
ವ್ಯಾಟಿ
ವ್ಯಾಟಿ-ಕನ್
ವ್ಯಾಟಿ-ಕ-ನ್ನಿಗೆ
ವ್ಯಾಧಿ-ಗ್ರಸ್ತ
ವ್ಯಾಪಾ-ರ-ವಾ-ಣಿಜ್ಯ
ವ್ಯಾಪಾ-ರಕ್ಕೆ
ವ್ಯಾಪಾ-ರಿ-ಗಳ
ವ್ಯಾಪಾ-ರೋ-ದ್ಯ-ಮಿ-ಗಳು
ವ್ಯಾಪಿ-ಸಿ-ಕೊಂ-ಡಿ-ರು-ವಂ-ತಹ
ವ್ಯಾಪಿ-ಸು-ತ್ತದೆ
ವ್ಯಾಮೋಹ
ವ್ಯಾಯಾಮ
ವ್ಯಾಯಾ-ಮ-ಗಳನ್ನು
ವ್ಯಾಯಾ-ಮದ
ವ್ಯಾಯಾ-ಮ-ವನ್ನು
ವ್ಯಾಯಾ-ಮ-ಶಾ-ಲೆ-ಗಳ
ವ್ಯಾಸ-ಮ-ಹ-ರ್ಷಿ-ಗಳ
ವ್ಯಾಸ-ಮ-ಹ-ರ್ಷಿ-ಗ-ಳಿಗೆ
ವ್ಯಾಸರು
ವ್ಯುತ್ಪ-ತ್ತಿ-ಯನ್ನು
ವ್ರತ
ವ್ರತ-ಗ-ಳ-ನ್ನಾ-ಚ-ರಿ-ಸುತ್ತ
ವ್ರತ-ಗಳು
ವ್ರತ-ದಿಂ-ದಾಗಿ
ವ್ರತ-ನಿ-ಯ-ಮ-ಗ-ಳಿ-ರ-ಬೇಕು
ವ್ರತ-ನಿ-ಷ್ಠೆ-ಗಳನ್ನು
ವ್ರತ-ಭ್ರ-ಷ್ಟ-ನಾದ
ವ್ರತ-ವನ್ನು
ವ್ರತ-ವನ್ನೂ
ವ್ರಾಜಕ
ಶಂಕರ
ಶಂಕ-ರರ
ಶಂಕ-ರರು
ಶಂಕ-ರಾ-ಚಾರ್ಯ
ಶಂಕ-ರಾ-ಚಾ-ರ್ಯರ
ಶಂಕಿ-ಸಿದ್ದ
ಶಂಕೆ
ಶಂಕೆ-ಅ-ಸ-ಮಾ-ಧಾನ
ಶಂಕೆ-ಯನ್ನು
ಶಂಕೆ-ಯಿತ್ತು
ಶಂಖ
ಶಂಖ-ಜಾ-ಗ-ಟೆ-ಗಂ-ಟೆ-ಗಳ
ಶಂಖ-ಜಾ-ಗ-ಟೆ-ಗಳ
ಶಂಖ-ಗಳನ್ನು
ಶಂಖ-ಜಾ-ಗ-ಟೆ-ಗಳ
ಶಂಖದ
ಶಕರು
ಶಕ್ತ-ರಾ-ಗಿ-ದ್ದರು
ಶಕ್ತ-ವಾಗಿ
ಶಕ್ತ-ವಾ-ಗಿ-ರ-ಬೇಕು
ಶಕ್ತಿ
ಶಕ್ತಿ-ಉ-ತ್ಸಾ-ಹದ
ಶಕ್ತಿ-ಪ-ವಿ-ತ್ರ-ತೆ
ಶಕ್ತಿ-ಸ್ಫೂ-ರ್ತಿ-ಗಳನ್ನು
ಶಕ್ತಿ-ಗಳ
ಶಕ್ತಿ-ಗ-ಳಾದ
ಶಕ್ತಿ-ಗಳಿಂದ
ಶಕ್ತಿ-ಗಳು
ಶಕ್ತಿ-ಗಳೂ
ಶಕ್ತಿ-ಗಳೇ
ಶಕ್ತಿ-ಗಿಂ-ತಲೂ
ಶಕ್ತಿ-ದಾ-ಯ-ಕ-ವಾದ
ಶಕ್ತಿ-ಪೂ-ರ್ಣ-ವಾ-ಗಿತ್ತು
ಶಕ್ತಿ-ಪೂ-ರ್ಣವೂ
ಶಕ್ತಿ-ಪ್ರ-ದ-ರ್ಶ-ನ-ವನ್ನು
ಶಕ್ತಿ-ಪ್ರ-ದ-ವಾ-ಗಿತ್ತು
ಶಕ್ತಿ-ಪ್ರ-ದ-ವಾ-ಗಿ-ರ-ಲೇ-ಬೇಕು
ಶಕ್ತಿ-ಪ್ರ-ದ-ವಾ-ದುದು
ಶಕ್ತಿ-ಭ-ರಿ-ತ-ವಾ-ಗಿ-ರು-ತ್ತಿ-ದ್ದುವು
ಶಕ್ತಿ-ಭ-ರಿ-ತವೂ
ಶಕ್ತಿ-ಮೀರಿ
ಶಕ್ತಿ-ಮೂಲ
ಶಕ್ತಿಯ
ಶಕ್ತಿ-ಯನ್ನು
ಶಕ್ತಿ-ಯನ್ನೂ
ಶಕ್ತಿ-ಯ-ನ್ನೆಲ್ಲ
ಶಕ್ತಿ-ಯನ್ನೇ
ಶಕ್ತಿ-ಯಲ್ಲಿ
ಶಕ್ತಿ-ಯ-ಲ್ಲಿಯೆ
ಶಕ್ತಿ-ಯಾ-ಗಿ-ದ್ದರು
ಶಕ್ತಿ-ಯಾದ
ಶಕ್ತಿ-ಯಿಂದ
ಶಕ್ತಿ-ಯಿತ್ತು
ಶಕ್ತಿ-ಯಿ-ರು-ತ್ತದೆ
ಶಕ್ತಿ-ಯಿ-ರು-ವು-ದಿಲ್ಲ
ಶಕ್ತಿ-ಯಿಲ್ಲ
ಶಕ್ತಿಯು
ಶಕ್ತಿ-ಯುಂ-ಟಾ-ಗು-ತ್ತದೆ
ಶಕ್ತಿ-ಯುತ
ಶಕ್ತಿ-ಯು-ತ-ವಾಗಿ
ಶಕ್ತಿ-ಯು-ತ-ವಾ-ಗಿ-ರು-ವಂತೆ
ಶಕ್ತಿ-ಯು-ತ-ವಾದ
ಶಕ್ತಿ-ಯು-ತವೂ
ಶಕ್ತಿ-ಯು-ಳ್ಳದ್ದು
ಶಕ್ತಿಯೂ
ಶಕ್ತಿ-ಯೆಂ-ಥದು
ಶಕ್ತಿ-ಯೆ-ಲ್ಲಿಂದ
ಶಕ್ತಿಯೇ
ಶಕ್ತಿ-ಯೇನು
ಶಕ್ತಿ-ಯೊಂ-ದರ
ಶಕ್ತಿ-ಯೊಂದು
ಶಕ್ತಿ-ರೂ-ಪಿಣಿ
ಶಕ್ತಿ-ಲೀಲೆ
ಶಕ್ತಿ-ವಂ-ತ-ನಾ-ಗಿ-ದ್ದರೂ
ಶಕ್ತಿ-ವಂ-ತ-ನಾ-ಗಿ-ರ-ಬೇಕು
ಶಕ್ತಿ-ವಂ-ತ-ನಾಗು
ಶಕ್ತಿ-ವಂ-ತ-ನಾದ
ಶಕ್ತಿ-ವಂ-ತ-ರಾಗಿ
ಶಕ್ತಿ-ವಂ-ತ-ರಾದ
ಶಕ್ತಿ-ವಂ-ತರು
ಶಕ್ತಿ-ವಂ-ತರೂ
ಶಕ್ತಿ-ವಂ-ತರೋ
ಶಕ್ತಿ-ಶಾ-ಲಿ-ಗ-ಳಾದ
ಶಕ್ತಿ-ಶಾ-ಲಿ-ಗಳು
ಶಕ್ತಿ-ಶಾ-ಲಿ-ಯಾ-ಗಿತ್ತು
ಶಕ್ತಿ-ಶಾ-ಲಿ-ಯಾಗು
ಶಕ್ತಿ-ಶಾ-ಲಿ-ಯಾದ
ಶಕ್ತಿ-ಸಂ-ಚಾರ
ಶಕ್ತಿ-ಸ್ಪ-ರ್ಶ-ದಿಂದ
ಶಕ್ತಿ-ಸ್ವ-ರೂ-ಪಿ-ಯಾ-ಗಿ-ದ್ದರು
ಶಕ್ತಿ-ಹ-ರಣ
ಶಕ್ತ್ಯು-ತ್ಸಾ-ಹ-ಗಳನ್ನು
ಶಕ್ತ್ಯು-ತ್ಸಾ-ಹ-ಗಳಿಂದ
ಶಕ್ಯ-ವಿ-ರ-ಲಿಲ್ಲ
ಶತ
ಶತ-ಮಾ-ನ-ಗಳ
ಶತ-ಮಾ-ನ-ಗಳಲ್ಲಿ
ಶತ-ಮಾ-ನ-ಗ-ಳಿಂ-ದಲೂ
ಶತ-ಮಾ-ನದ
ಶತ-ಮಾ-ನ-ದಲ್ಲಿ
ಶತ-ಮಾ-ನ-ದಲ್ಲೇ
ಶತ-ಮಾ-ನದ್ದು
ಶತ-ಶ-ತ-ಮಾನ
ಶತ-ಶ-ತ-ಮಾ-ನ-ಗಳ
ಶತ-ಶ-ತ-ಮಾ-ನ-ಗಳಿಂದ
ಶತ-ಶ-ತ-ಮಾ-ನ-ಗ-ಳಿಂ-ದಲೂ
ಶತ್ರು
ಶತ್ರು-ಗಳ
ಶತ್ರು-ಗಳೂ
ಶತ್ರು-ಗಳೇ
ಶತ್ರು-ವಾದ
ಶನದ
ಶನಿ-ವಾರ
ಶಪಥ
ಶಪಿ-ಸ-ಬ-ಹುದು
ಶಪಿ-ಸಿ-ಬಿ-ಟ್ಟಿದ್ದ
ಶಪಿ-ಸು-ವಿರಾ
ಶಬ್ದ
ಶಬ್ದ-ಕೋ-ಶ-ವನ್ನೇ
ಶಬ್ದಕ್ಕೆ
ಶಬ್ದ-ಗ-ಳಲ್ಲ
ಶಬ್ದ-ಗಳಿಂದ
ಶಬ್ದ-ಗಳು
ಶಬ್ದ-ಗ-ಳು-ತ್ಯಾಗ
ಶಬ್ದ-ಗಳೇ
ಶಬ್ದದ
ಶಬ್ದ-ದಲ್ಲಿ
ಶಬ್ದ-ವನ್ನು
ಶಬ್ದ-ವನ್ನೂ
ಶಮ-ನ-ಗೊ-ಳಿ-ಸ-ಬಲ್ಲ
ಶಮ-ನ-ಗೊ-ಳಿ-ಸಿ-ದರು
ಶಮ-ನ-ವಾ-ದಂತೆ
ಶಯ
ಶಯರ
ಶಯ-ರೊ-ಬ್ಬರು
ಶರ
ಶರ-ಚ್ಚಂದ್ರ
ಶರ-ಚ್ಚಂ-ದ್ರನ
ಶರ-ಚ್ಚಂ-ದ್ರ-ನತ್ತ
ಶರ-ಚ್ಚಂ-ದ್ರ-ನಿಗೆ
ಶರ-ಚ್ಚಂ-ದ್ರ-ನಿ-ಗೇಕೋ
ಶರ-ಚ್ಚಂ-ದ್ರನು
ಶರ-ಚ್ಚಂ-ದ್ರ-ನೊಂ-ದಿಗೆ
ಶರ-ಣಾ-ಗ-ತಿಯ
ಶರ-ಣಾಗಿ
ಶರ-ಣಾ-ಗಿ-ರ-ಬೇ-ಕಾ-ದು-ದರ
ಶರ-ಣಾ-ಗಿಸಿ
ಶರ-ಣಾದ
ಶರ-ಣಾ-ದ-ವರು
ಶರ-ಣೀತ
ಶರಣು
ಶರೀರ
ಶರೀ-ರ-ಐ-ಶ್ವ-ರ್ಯ-ಗಳ
ಶರೀ-ರ-ಮ-ನ-ಸ್ಸು-ಗಳಿಂದ
ಶರೀ-ರಕ್ಕೂ
ಶರೀ-ರಕ್ಕೆ
ಶರೀ-ರ-ಗ-ಳಿಗೆ
ಶರೀ-ರ-ಗಳು
ಶರೀ-ರದ
ಶರೀ-ರ-ದಲ್ಲಿ
ಶರೀ-ರ-ದಾ-ದ್ಯಂತ
ಶರೀ-ರ-ದಿಂದ
ಶರೀ-ರ-ದೊಂ-ದಿಗೆ
ಶರೀ-ರ-ದೊ-ಳಗೆ
ಶರೀ-ರ-ವನ್ನು
ಶರೀ-ರ-ವನ್ನೂ
ಶರೀ-ರ-ವನ್ನೇ
ಶರೀ-ರ-ವಲ್ಲ
ಶರೀ-ರ-ವ-ಲ್ಲವೆ
ಶರೀ-ರ-ವಿಂದು
ಶರೀ-ರವು
ಶರೀ-ರವೂ
ಶರೀ-ರವೇ
ಶರೀ-ರ-ವೊಂ-ದಿ-ದ್ದಿ-ದ್ದರೆ
ಶರೀ-ರ-ಸೌ-ಷ್ಠ-ವ-ವನ್ನು
ಶರೀ-ರ-ಸ್ಥಿ-ತಿ-ಯನ್ನು
ಶರೀ-ರಾ-ಸ-ಕ್ತಿ-ಯಿಂ-ದಲೇ
ಶವ-ದಂತೆ
ಶವ-ವನ್ನು
ಶಶಿ
ಶಶಿ-ಭೂ-ಷಣ
ಶಸ್ತ್ರ-ಗಳ
ಶಸ್ತ್ರಾ-ಸ್ತ್ರಕ್ಕೆ
ಶಸ್ತ್ರಾ-ಸ್ತ್ರ-ಗಳ
ಶಸ್ತ್ರಾ-ಸ್ತ್ರ-ಗಳನ್ನು
ಶಹ-ಬಾಸ್
ಶಹ-ರಾ-ನ್ಪು-ರದ
ಶಾಂಕ-ರ-ಭಾಷ್ಯ
ಶಾಂತ
ಶಾಂತ-ಗಂ-ಭೀರ
ಶಾಂತ-ತ-ಟಸ್ಥ
ಶಾಂತ-ಸ್ಥಿರ
ಶಾಂತ-ಸ್ವ-ತಂತ್ರ
ಶಾಂತ-ಗೊ-ಳಿ-ಸ-ಬೇಕು
ಶಾಂತ-ಗೊ-ಳಿ-ಸುವ
ಶಾಂತ-ಚಿ-ತ್ತತೆ
ಶಾಂತ-ಚಿ-ತ್ತ-ದಿಂದ
ಶಾಂತ-ನಾದ
ಶಾಂತ-ಮು-ದ್ರೆಯ
ಶಾಂತ-ರಾ-ಗಿರ
ಶಾಂತ-ರಾದ
ಶಾಂತ-ರೀ-ತಿ-ಯಲ್ಲಿ
ಶಾಂತ-ವಾಗಿ
ಶಾಂತ-ವಾ-ಗಿ-ಅ-ಗೋ-ಚ-ರ-ವಾಗಿ
ಶಾಂತ-ವಾ-ಗಿತ್ತು
ಶಾಂತ-ವಾ-ಗಿ-ದ್ದು-ಬಿ-ಡ-ಬೇ-ಕೆಂಬ
ಶಾಂತ-ವಾ-ಗಿ-ದ್ದೇ-ನೆಂ-ದರೆ
ಶಾಂತ-ವಾ-ಗಿ-ಬಿ-ಟ್ಟವು
ಶಾಂತ-ವಾ-ಗಿಯೇ
ಶಾಂತ-ವಾ-ಗಿ-ರುವ
ಶಾಂತ-ವಾ-ಗುತ್ತ
ಶಾಂತ-ವಾದ
ಶಾಂತ-ಸ್ವ-ಭಾ-ವ-ವನ್ನು
ಶಾಂತಿ
ಶಾಂತಿ-ನೆ-ಮ್ಮದಿ
ಶಾಂತಿ-ಶು-ಭ-ವೆಂ-ದೆಂದು
ಶಾಂತಿ-ಜ-ಲ-ಸೇ-ಚನ
ಶಾಂತಿ-ದೂ-ತ-ನಾಗಿ
ಶಾಂತಿ-ದೂ-ತ-ನಿಗೆ
ಶಾಂತಿಯ
ಶಾಂತಿ-ಯ-ನ್ನ-ನು-ಭ-ವಿ-ಸಿ-ದರು
ಶಾಂತಿ-ಯನ್ನು
ಶಾಂತಿ-ಯನ್ನೂ
ಶಾಂತಿ-ಯಿಂದ
ಶಾಂತಿ-ಯುತ
ಶಾಂತಿಯೂ
ಶಾಂತಿ-ಸ್ವ-ರೂಪ
ಶಾಂಪೇನ್
ಶಾಖ
ಶಾಖ-ವನ್ನು
ಶಾಖೆ-ಗ-ಳಿವೆ
ಶಾಖೆ-ಗಳು
ಶಾಖೆಯ
ಶಾಖೆ-ಯನ್ನು
ಶಾಖೆ-ಯಲ್ಲಿ
ಶಾಖೆಯೂ
ಶಾಪ
ಶಾಪ-ವಿತ್ತು
ಶಾಪವೂ
ಶಾಮಿ-ಯಾ-ನ-ದೊ-ಳಕ್ಕೆ
ಶಾಮಿ-ಯಾ-ನ-ವನ್ನು
ಶಾರ-ದ-ನಂ-ದರ
ಶಾರದಾ
ಶಾರ-ದಾ-ದೇವಿ
ಶಾರ-ದಾ-ದೇ-ವಿ-ಯರ
ಶಾರ-ದಾ-ದೇ-ವಿ-ಯರು
ಶಾರ-ದಾ-ದೇ-ವಿ-ಯ-ವರ
ಶಾರ-ದಾ-ದೇ-ವಿ-ಯ-ವ-ರನ್ನು
ಶಾರ-ದಾ-ದೇ-ವಿ-ಯ-ವರು
ಶಾರ-ದಾ-ನಂದ
ಶಾರ-ದಾ-ನಂ-ದರ
ಶಾರ-ದಾ-ನಂ-ದ-ರಾದಿ
ಶಾರ-ದಾ-ನಂ-ದ-ರಿಗೂ
ಶಾರ-ದಾ-ನಂ-ದ-ರಿಗೆ
ಶಾರ-ದಾ-ನಂ-ದರು
ಶಾರ-ದಾ-ನಂ-ದರೂ
ಶಾರ-ದಾ-ನಂ-ದ-ರೊ-ಡನೆ
ಶಾರ-ದಾ-ಶ್ರಮ
ಶಾರೀ-ರಕ
ಶಾರೀ-ರಿಕ
ಶಾರೀ-ರಿ-ಕ-ವಾಗಿ
ಶಾಲಾ
ಶಾಲಾ-ಕಾ-ಲೇ-ಜು-ಗಳ
ಶಾಲಿ
ಶಾಲೀ
ಶಾಲೆ
ಶಾಲೆ-ಗಳನ್ನು
ಶಾಲೆ-ಗಳು
ಶಾಲೆ-ಗಳೇ
ಶಾಲೆ-ಗಾಗಿ
ಶಾಲೆಗೆ
ಶಾಲೆಯ
ಶಾಲೆ-ಯನ್ನು
ಶಾಲೆ-ಯಲ್ಲಿ
ಶಾಲೆ-ಯಿಂದ
ಶಾಲೆ-ಯೊಂ-ದನ್ನು
ಶಾಶ್ವತ
ಶಾಶ್ವ-ತ-ವ-ಲ್ಲ-ವಲ್ಲ
ಶಾಶ್ವ-ತ-ವಾಗಿ
ಶಾಶ್ವ-ತ-ವಾದ
ಶಾಶ್ವ-ತವೂ
ಶಾಸ್ತ್ರ
ಶಾಸ್ತ್ರ-ಗಳ
ಶಾಸ್ತ್ರ-ಗಳನ್ನು
ಶಾಸ್ತ್ರ-ಗಳನ್ನೂ
ಶಾಸ್ತ್ರ-ಗಳಲ್ಲಿ
ಶಾಸ್ತ್ರ-ಗ-ಳಲ್ಲೂ
ಶಾಸ್ತ್ರ-ಗಳಿಂದ
ಶಾಸ್ತ್ರ-ಗ-ಳಿಗೆ
ಶಾಸ್ತ್ರ-ಗಳು
ಶಾಸ್ತ್ರ-ಗಳೇ
ಶಾಸ್ತ್ರ-ಗ-ಳೇನು
ಶಾಸ್ತ್ರ-ಗ್ರಂ-ಥ-ಗಳ
ಶಾಸ್ತ್ರ-ಗ್ರಂ-ಥ-ಗಳನ್ನು
ಶಾಸ್ತ್ರ-ಗ್ರಂ-ಥ-ಗಳಲ್ಲಿ
ಶಾಸ್ತ್ರ-ಗ್ರಂ-ಥ-ವಾ-ದರೂ
ಶಾಸ್ತ್ರ-ಜ್ಞ-ನೊ-ಬ್ಬ-ನಿಗೆ
ಶಾಸ್ತ್ರ-ಜ್ಞ-ರಾದ
ಶಾಸ್ತ್ರ-ಜ್ಞಾ-ನದ
ಶಾಸ್ತ್ರ-ಪಾ-ಠ-ಗಳನ್ನು
ಶಾಸ್ತ್ರ-ಪ್ರ-ಣೀ-ತ-ವಾದ
ಶಾಸ್ತ್ರ-ವಾಕ್ಯ
ಶಾಸ್ತ್ರ-ವಾ-ಕ್ಯ-ಗಳನ್ನು
ಶಾಸ್ತ್ರ-ವಿ-ಚಾರ
ಶಾಸ್ತ್ರ-ವಿ-ಚಾ-ರ-ಗಳ
ಶಾಸ್ತ್ರ-ವಿ-ಚಾ-ರ-ಗಳನ್ನು
ಶಾಸ್ತ್ರ-ವಿ-ಧಿ-ಯನ್ನು
ಶಾಸ್ತ್ರ-ವಿ-ಧಿ-ಯನ್ನೂ
ಶಾಸ್ತ್ರಾ-ಧ್ಯ-ಯನ
ಶಾಸ್ತ್ರಾ-ಧ್ಯ-ಯ-ನಕ್ಕೆ
ಶಾಸ್ತ್ರಾ-ಧ್ಯ-ಯ-ನ-ಗಳು
ಶಾಸ್ತ್ರಾ-ಧ್ಯ-ಯ-ನದ
ಶಾಸ್ತ್ರಾ-ಭ್ಯಾ-ಸದ
ಶಾಸ್ತ್ರಾ-ರ್ಥ-ವನ್ನು
ಶಿಕಾಗೊ
ಶಿಕಾಗೋ
ಶಿಕಾ-ಗೋದ
ಶಿಕಾ-ಗೋ-ದಲ್ಲಿ
ಶಿಕಾ-ಗೋ-ದ-ಲ್ಲಿ-ದ್ದಾಗ
ಶಿಕಾ-ಗೋ-ದ-ವರೆ-ಗಿನ
ಶಿಕಾ-ಗೋ-ದ-ವ-ರೆಗೆ
ಶಿಕಾ-ಗೋ-ದಿಂದ
ಶಿಕ್ಷ-ಕ-ನಾ-ಗಿ-ರಲಿ
ಶಿಕ್ಷ-ಕರು
ಶಿಕ್ಷಕಿ
ಶಿಕ್ಷ-ಕಿ-ಯರ
ಶಿಕ್ಷ-ಕಿ-ಯ-ರಿಗೆ
ಶಿಕ್ಷಣ
ಶಿಕ್ಷ-ಣ-ತ-ರ-ಬೇತಿ
ಶಿಕ್ಷ-ಣ-ತ-ರ-ಬೇ-ತಿಯೂ
ಶಿಕ್ಷ-ಣ-ಕ್ರ-ಮದ
ಶಿಕ್ಷ-ಣ-ಗಳು
ಶಿಕ್ಷ-ಣ-ಗ-ಳೆ-ರ-ಡನ್ನೂ
ಶಿಕ್ಷ-ಣ-ತಜ್ಞ
ಶಿಕ್ಷ-ಣ-ತ-ಜ್ಞೆ-ಯೆಂದು
ಶಿಕ್ಷ-ಣದ
ಶಿಕ್ಷ-ಣ-ದಿಂದ
ಶಿಕ್ಷ-ಣ-ವನ್ನು
ಶಿಕ್ಷ-ಣವು
ಶಿಕ್ಷಿ-ಸ-ಬೇ-ಕಾ-ಗ-ಬ-ಹು-ದಾ-ದರೂ
ಶಿಕ್ಷಿ-ಸ-ಬೇ-ಕೆಂದೋ
ಶಿಕ್ಷೆ
ಶಿಕ್ಷೆಗೆ
ಶಿಕ್ಷೆ-ಯ-ನ್ನೇನೋ
ಶಿಕ್ಷೆ-ಯಾ-ಗ-ಬೇ-ಕೆಂದು
ಶಿಖ-ರಕ್ಕೆ
ಶಿಖ-ರ-ಗಳ
ಶಿಖ-ರ-ಗಳನ್ನು
ಶಿಖ-ರ-ಗ-ಳ-ನ್ನೇ-ರಿ-ದೆ-ಇ-ವಿಷ್ಟು
ಶಿಖ-ರ-ಗಳಿಂದ
ಶಿಖ-ರ-ಗಳು
ಶಿಖ-ರದ
ಶಿಖ-ರ-ವನ್ನು
ಶಿಖ-ರ-ವಾದ
ಶಿಖೆ-ಯನ್ನು
ಶಿಥಿಲ
ಶಿಥಿ-ಲ-ವಾ-ಗಿ-ದ್ದರೂ
ಶಿಥಿ-ಲಾ-ವ-ಸ್ಥೆ-ಯನ್ನು
ಶಿಯಾ-ದೇವಿ
ಶಿರ-ಗಳ
ಶಿರ-ಬಾಗಿ
ಶಿರ-ಬಾ-ಗು-ವ-ವನು
ಶಿರ-ಮೆಟ್ಟಿ
ಶಿರ-ವನ್ನು
ಶಿರ-ಸಾ-ವ-ಹಿ-ಸಲು
ಶಿಲೆ
ಶಿಲೆಯ
ಶಿಲೆ-ಯಾ-ದರೆ
ಶಿಲೆ-ಯಿಂದ
ಶಿಲ್ಪ-ಕಲೆ
ಶಿಲ್ಪ-ಗ-ಳಿವೆ
ಶಿಲ್ಪ-ಶಾ-ಸ್ತ್ರದ
ಶಿಲ್ಲಾಂ-ಗಿಗೆ
ಶಿಲ್ಲಾಂ-ಗಿ-ನಲ್ಲಿ
ಶಿಲ್ಲಾಂ-ಗಿ-ನ-ಲ್ಲಿ-ದ್ದಷ್ಟು
ಶಿವ
ಶಿವ-ಕ್ಷೇತ್ರ
ಶಿವ-ಗಂ-ಗೆಯ
ಶಿವ-ದೇ-ವಾ-ಲಯ
ಶಿವ-ದೇ-ವಾ-ಲ-ಯಕ್ಕೆ
ಶಿವನ
ಶಿವ-ನಂತೆ
ಶಿವ-ನ-ಗ-ರಿ-ಯಾದ
ಶಿವ-ನ-ನ್ನಾಗಿ
ಶಿವ-ನನ್ನು
ಶಿವ-ನನ್ನೇ
ಶಿವ-ನಿಗೆ
ಶಿವನು
ಶಿವನೇ
ಶಿವನೊ
ಶಿವ-ನೊಂ-ದಿಗೆ
ಶಿವ-ಪೂ-ಜೆಯ
ಶಿವ-ಪೂ-ಜೆ-ಯನ್ನು
ಶಿವ-ಭ-ಕ್ತಿ-ಯಿಂದ
ಶಿವ-ಭಾ-ವವೇ
ಶಿವ-ಮ-ಯವೇ
ಶಿವ-ರಾತ್ರಿ
ಶಿವ-ರಾ-ತ್ರಿ-ಯಂದು
ಶಿವ-ರಾಮ
ಶಿವ-ಲಿಂಗ
ಶಿವ-ಲಿಂ-ಗ-ದಿಂದ
ಶಿವ-ಲಿಂ-ಗ-ವನ್ನು
ಶಿವ-ಲಿಂ-ಗವು
ಶಿವ-ಸೇ-ವೆಯ
ಶಿವಾ
ಶಿವಾಜಿ
ಶಿವಾ-ನಂದ
ಶಿವಾ-ನಂ-ದರ
ಶಿವಾ-ನಂ-ದ-ರಂ-ತಹ
ಶಿವಾ-ನಂ-ದ-ರಿಗೆ
ಶಿವಾ-ನಂ-ದರು
ಶಿವಾ-ನಂ-ದರೂ
ಶಿವಾ-ನಂ-ದರೇ
ಶಿವಾ-ನಂ-ದ-ರೊ-ಡನೆ
ಶಿವಾ-ನು-ಭವ
ಶಿವೋ
ಶಿಶು
ಶಿಶು-ಗ-ಳಂತೆ
ಶಿಶು-ಗಳನ್ನು
ಶಿಶು-ಗಳನ್ನೆಲ್ಲ
ಶಿಶು-ಗ-ಳನ್ನೇ
ಶಿಶು-ಪ್ರಾ-ಯ-ರಾ-ದ-ವರು
ಶಿಶು-ಭಾ-ವವು
ಶಿಶು-ವನ್ನು
ಶಿಶು-ವಾ-ಗ-ಹೊ-ರ-ಟಿ-ದ್ದಾರೋ
ಶಿಶು-ವಾದ
ಶಿಶು-ವಿಗೆ
ಶಿಶು-ವಿನ
ಶಿಶುವು
ಶಿಶು-ಸ-ಹಜ
ಶಿಷ್ಟಾ-ಚಾ-ರ-ಗಳನ್ನು
ಶಿಷ್ಯ
ಶಿಷ್ಯ-ತ್ವದ
ಶಿಷ್ಯನ
ಶಿಷ್ಯ-ನನ್ನು
ಶಿಷ್ಯ-ನ-ಲ್ಲವೆ
ಶಿಷ್ಯ-ನಲ್ಲಿ
ಶಿಷ್ಯ-ನಾದ
ಶಿಷ್ಯ-ನಾ-ದ-ನೆಂದೂ
ಶಿಷ್ಯ-ನಿ-ಗಿ-ರುವ
ಶಿಷ್ಯ-ನಿಗೆ
ಶಿಷ್ಯನು
ಶಿಷ್ಯನೂ
ಶಿಷ್ಯ-ನೆಂದು
ಶಿಷ್ಯನೇ
ಶಿಷ್ಯ-ನೊಂ-ದಿಗೆ
ಶಿಷ್ಯ-ನೊ-ಡನೆ
ಶಿಷ್ಯ-ನೊ-ಬ್ಬ-ನನ್ನು
ಶಿಷ್ಯರ
ಶಿಷ್ಯ-ರಂತೂ
ಶಿಷ್ಯ-ರತ್ತ
ಶಿಷ್ಯ-ರ-ನ್ನಾಗಿ
ಶಿಷ್ಯ-ರನ್ನು
ಶಿಷ್ಯ-ರನ್ನೂ
ಶಿಷ್ಯ-ರ-ನ್ನೆಲ್ಲ
ಶಿಷ್ಯ-ರಲ್ಲಿ
ಶಿಷ್ಯ-ರ-ಲ್ಲೆಲ್ಲ
ಶಿಷ್ಯ-ರ-ಲ್ಲೊ-ಬ್ಬ-ರಾದ
ಶಿಷ್ಯ-ರಾ-ಗ-ಬ-ಯ-ಸು-ವು-ದಾ-ದರೆ
ಶಿಷ್ಯ-ರಾ-ಗಲಿ
ಶಿಷ್ಯ-ರಾ-ಗಿ-ದ್ದರು
ಶಿಷ್ಯ-ರಾದ
ಶಿಷ್ಯ-ರಾ-ದರು
ಶಿಷ್ಯ-ರಾ-ದ್ದ-ರಿಂದ
ಶಿಷ್ಯ-ರಿಂದ
ಶಿಷ್ಯ-ರಿ-ಗಾಗಿ
ಶಿಷ್ಯ-ರಿ-ಗಾದ
ಶಿಷ್ಯ-ರಿಗೂ
ಶಿಷ್ಯ-ರಿಗೆ
ಶಿಷ್ಯ-ರಿ-ಗೆಲ್ಲ
ಶಿಷ್ಯರು
ಶಿಷ್ಯ-ರು-ಅಭಿ
ಶಿಷ್ಯ-ರು-ಸ್ನೇ-ಹಿ-ತ-ರಿಂದ
ಶಿಷ್ಯರೂ
ಶಿಷ್ಯ-ರೆಂದು
ಶಿಷ್ಯ-ರೆ-ನ್ನಿ-ಸಿ-ಕೊಂ-ಡರೂ
ಶಿಷ್ಯ-ರೆಲ್ಲ
ಶಿಷ್ಯ-ರೊಂ-ದಿ-ಗಿ-ದ್ದ-ರೆಂ-ಬು-ದೇನೋ
ಶಿಷ್ಯ-ರೊಂ-ದಿಗೆ
ಶಿಷ್ಯ-ರೊಂ-ದಿ-ಗೆಲ್ಲ
ಶಿಷ್ಯ-ರೊ-ಬ್ಬರ
ಶಿಷ್ಯ-ರೊ-ಬ್ಬ-ರಿಗೆ
ಶಿಷ್ಯ-ರೊ-ಬ್ಬರು
ಶಿಷ್ಯ-ವ-ರ್ಗ-ದವ
ಶಿಷ್ಯಾ-ಗ್ರಣಿ
ಶಿಷ್ಯೆ
ಶಿಷ್ಯೆಗೆ
ಶಿಷ್ಯೆಯ
ಶಿಷ್ಯೆ-ಯನ್ನು
ಶಿಷ್ಯೆ-ಯರ
ಶಿಷ್ಯೆ-ಯ-ರ-ನ್ನಾಗಿ
ಶಿಷ್ಯೆ-ಯ-ರನ್ನು
ಶಿಷ್ಯೆ-ಯ-ರನ್ನೂ
ಶಿಷ್ಯೆ-ಯ-ರ-ನ್ನೆಲ್ಲ
ಶಿಷ್ಯೆ-ಯ-ರಾ-ಗಿ-ದ್ದರು
ಶಿಷ್ಯೆ-ಯ-ರಾದ
ಶಿಷ್ಯೆ-ಯ-ರಿಂದ
ಶಿಷ್ಯೆ-ಯ-ರಿಗೂ
ಶಿಷ್ಯೆ-ಯ-ರಿಗೆ
ಶಿಷ್ಯೆ-ಯ-ರಿ-ಗೆ-ಮು-ಖ್ಯ-ವಾಗಿ
ಶಿಷ್ಯೆ-ಯ-ರಿ-ಗೆಲ್ಲ
ಶಿಷ್ಯೆ-ಯ-ರಿ-ಗೊಂದು
ಶಿಷ್ಯೆ-ಯ-ರಿದ್ದ
ಶಿಷ್ಯೆ-ಯರು
ಶಿಷ್ಯೆ-ಯ-ರು-ಅ-ವ-ರಾ-ಡುವ
ಶಿಷ್ಯೆ-ಯ-ರು-ಮು-ಖ್ಯ-ವಾಗಿ
ಶಿಷ್ಯೆ-ಯ-ರೆಲ್ಲ
ಶಿಷ್ಯೆ-ಯ-ರೊಂ-ದಿಗೆ
ಶಿಷ್ಯೆ-ಯಾಗಿ
ಶಿಷ್ಯೆ-ಯಾದ
ಶಿಷ್ಯೆಯು
ಶಿಷ್ಯೆಯೂ
ಶಿಷ್ಯೆ-ಯೊ-ಡನೆ
ಶಿಸಿ-ದರು
ಶಿಸ್ತಾಗಿ
ಶಿಸ್ತಿಗೂ
ಶಿಸ್ತಿಗೆ
ಶಿಸ್ತಿನ
ಶಿಸ್ತಿ-ನಿಂದ
ಶಿಸ್ತು
ಶಿಸ್ತು-ಶಾಂತಿ
ಶಿಸ್ತು-ಬ-ದ್ಧ-ವಾಗಿ
ಶೀಘ್ರ
ಶೀಘ್ರ-ದ-ಲ್ಲಿಯೇ
ಶೀಘ್ರ-ದಲ್ಲೇ
ಶೀಘ್ರ-ಲಿ-ಪಿ-ಕಾರ
ಶೀಘ್ರ-ಲಿ-ಪಿ-ಕಾ-ರ-ಶಿಷ್ಯ
ಶೀಘ್ರ-ಲಿ-ಪಿಯ
ಶೀಘ್ರ-ವಾಗಿ
ಶೀಘ್ರವೇ
ಶೀತ
ಶೀತ-ಗಾಳಿ
ಶೀತದ
ಶೀತ-ದಿಂದ
ಶೀತ-ವಾ-ಗಿತ್ತು
ಶೀತವೂ
ಶೀತ-ವೆಂದರೆ
ಶೀರ್ಷಿಕೆ
ಶೀರ್ಷಿ-ಕೆ-ಯ-ಡಿ-ಯಲ್ಲಿ
ಶೀರ್ಷಿ-ಕೆ-ಯಲ್ಲಿ
ಶೀಲ
ಶೀಲ-ಚಾ-ರಿತ್ರ್ಯ
ಶೀಲ-ನ-ಡ-ತೆ-ಗಳ
ಶೀಲ-ಸ್ವ-ಭಾ-ವ-ಗಳನ್ನು
ಶೀಲ-ಗೆ-ಟ್ಟಿ-ರು-ವುದೇ
ಶೀಲ-ತೆಯೇ
ಶೀಲ-ದಲ್ಲಿ
ಶೀಲ-ನಿ-ರ್ಮಾ-ಣ-ವಾ-ಗ-ಬೇಕು
ಶೀಲ-ಬಲ
ಶೀಲ-ವಂತ
ಶೀಲ-ವನ್ನು
ಶೀಲವು
ಶೀಲ-ಸಂ-ವ-ರ್ಧ-ನೆಯ
ಶುಕ-ದೇ-ವನ
ಶುಕಲ್
ಶುಕ್ರ-ವಾರ
ಶುಕ್ರಾ-ಚಾರ್ಯ
ಶುಕ್ರಾ-ಚಾ-ರ್ಯನು
ಶುಕ್ಲ-ಯ-ಜು-ರ್ವೇದ
ಶುಚಿ
ಶುಚಿ-ಗೊ-ಳಿ-ಸ-ಲಾ-ಯಿತು
ಶುಚಿ-ಗೊ-ಳಿಸಿ
ಶುಚಿ-ಗೊ-ಳಿಸು
ಶುಚಿ-ಗೊ-ಳಿ-ಸು-ವಂ-ತಹ
ಶುಚಿ-ಗೊ-ಳಿ-ಸು-ವಂತೆ
ಶುಚಿ-ಗೊ-ಳಿ-ಸು-ವು-ದ-ರಲ್ಲೇ
ಶುಚಿ-ಗೊ-ಳಿ-ಸು-ವುದು
ಶುಚಿ-ತ್ವ-ಗಳ
ಶುಚಿ-ತ್ವದ
ಶುಚಿಯಾ
ಶುಚಿ-ಯಾ-ಗಿ-ರು-ವ-ವರು
ಶುದ್ಧ
ಶುದ್ಧ-ಗೊ-ಳಿ-ಸಿಕೊ
ಶುದ್ಧ-ತೆಗೆ
ಶುದ್ಧ-ರೂ-ಪ-ದಲ್ಲಿ
ಶುದ್ಧ-ರೂ-ಪ-ದಿಂದ
ಶುದ್ಧ-ವಾಗಿ
ಶುದ್ಧ-ವಾ-ಗಿ-ದ್ದರೆ
ಶುದ್ಧ-ವಾ-ಗು-ತ್ತದೆ
ಶುದ್ಧ-ವಾದ
ಶುದ್ಧಾ-ನಂದ
ಶುದ್ಧಾ-ನಂ-ದರ
ಶುದ್ಧಾ-ನಂ-ದ-ರಿಗೆ
ಶುದ್ಧಾ-ನಂ-ದರು
ಶುದ್ಧಾ-ನಂ-ದ-ರೆಂಬ
ಶುದ್ಧಾ-ನಂ-ದ-ರೊಂ-ದಿಗೆ
ಶುದ್ಧಿ-ಮಾ-ಡಿ-ಕೊಂ-ಡಾಗ
ಶುಭ
ಶುಭ-ಕ-ರ-ವಾ-ಗಿದೆ
ಶುಭ-ಕ-ರ-ವಾ-ದ-ದ್ದೇ-ನನ್ನೂ
ಶುಭ-ದಿ-ನ-ಗಳನ್ನು
ಶುಭ-ದಿ-ನ-ವನ್ನು
ಶುಭ-ಮು-ಹೂ-ರ್ತ-ವನ್ನು
ಶುಭ-ವನ್ನು
ಶುಭ-ವನ್ನೇ
ಶುಭ-ವಾ-ಗಲಿ
ಶುಭ-ವೆಷ್ಟೋ
ಶುಭಾ-ನಂದ
ಶುಭಾ-ಶಯ
ಶುಭಾ-ಶ-ಯ-ಗಳನ್ನು
ಶುಭ್ರ
ಶುಭ್ರ-ವಾ-ಗದೇ
ಶುಭ್ರ-ವಾ-ಗಿತ್ತು
ಶುಭ್ರ-ವಾದ
ಶುರು-ಮಾಡಿ
ಶುರು-ಮಾ-ಡಿ-ದರು
ಶುರು-ವಾ-ಗ-ಬೇಕು
ಶುರು-ವಾ-ಯಿತು
ಶುಲ್ಕ
ಶುಲ್ಕ-ವಿದ್ದು
ಶುಲ್ಕ-ವಿ-ರ-ಲಿಲ್ಲ
ಶುಶ್ರೂಷೆ
ಶುಶ್ರೂ-ಷೆ-ಗಾಗಿ
ಶುಶ್ರೂ-ಷೆಗೆ
ಶುಷ್ಕ
ಶುಷ್ಕ-ವಾ-ಗಿದೆ
ಶೂದ್ರ
ಶೂದ್ರನ
ಶೂದ್ರ-ನಾ-ದರೆ
ಶೂದ್ರರ
ಶೂದ್ರ-ರಾ-ಗಿದ್ದೂ
ಶೂದ್ರ-ರಾ-ಗಿ-ರ-ಬ-ಹುದು
ಶೂದ್ರರು
ಶೂದ್ರರೂ
ಶೂನ್ಯ
ಶೂನ್ಯ-ದಿಂದ
ಶೂನ್ಯ-ರಾದ
ಶೃಂಖ-ಲೆ-ಗಳನ್ನು
ಶೃಂಗ-ಕ್ಕೇ-ರಿ-ಸಿ-ದ-ವ-ನನ್ನು
ಶೃಂಗ-ಗಳಲ್ಲಿ
ಶೇಕಡಾ
ಶೇಕ್ಸ್ಪಿ-ಯರ್
ಶೇಖ-ರಿ-ಸಿ-ಕೊ-ಳ್ಳಲು
ಶೇಟ್ಜಿ
ಶೇಷ-ನಾ-ಗದ
ಶೇಷಾ-ಚಾ-ರಿ-ಯರ್
ಶೈಕ್ಷ-ಣಿಕ
ಶೈತ್ಯ-ದಲ್ಲಿ
ಶೈಲಿ
ಶೈಲಿ-ಗಳ
ಶೈಲಿ-ಗಿಂತ
ಶೈಲಿ-ಯನ್ನು
ಶೈಲಿ-ಯಲ್ಲಿ
ಶೈಲಿ-ಯಿಂದ
ಶೈವ
ಶೈವರ
ಶೈವ-ರಿಗೆ
ಶೈವರು
ಶೋಕ
ಶೋಕದ
ಶೋಕ-ವನ್ನು
ಶೋಕ-ಸಾ-ಗ-ರ-ದಲ್ಲಿ
ಶೋಕಿ-ಸ-ದಿ-ದ್ದರೆ
ಶೋಚ-ನೀಯ
ಶೋಧಿಸಿ
ಶೋಭಾ-ಯ-ಮಾ-ನ-ವಾಗಿ
ಶೋಭಾ-ಯ-ಮಾ-ನ-ವಾದ
ಶೋಭಿ
ಶೋಭಿಸು
ಶೋಭಿ-ಸು-ತ್ತಿದ್ದ
ಶೋಭೆ-ಇವು
ಶೋಭೆ-ಗೊಂಡ
ಶೋಷ-ಣೆ-ಗೊ-ಳ-ಗಾ-ಗಿ-ದ್ದಾರೆ
ಶೌಚ-ಕ್ಕೆಂದು
ಶ್ಚರ್ಯ-ದಿಂದ
ಶ್ಚರ್ಯೆ-ಗಳಲ್ಲಿ
ಶ್ಯಾಮ-ಮೇ-ಘಾ-ಚ್ಛಾ-ದಿತ
ಶ್ರದ್ಧಾಂ-ಜ-ಲಿ-ಯ-ನ್ನ-ರ್ಪಿ-ಸಿ-ದರು
ಶ್ರದ್ಧಾಂ-ಜ-ಲಿ-ಯನ್ನು
ಶ್ರದ್ಧಾಂ-ಜ-ಲಿ-ಯನ್ನೂ
ಶ್ರದ್ಧಾ-ಕೇಂ-ದ್ರ-ವಿದೆ
ಶ್ರದ್ಧಾ-ಕೇಂ-ದ್ರವೇ
ಶ್ರದ್ಧಾ-ಭ-ಕ್ತಿ-ಯಿಂದ
ಶ್ರದ್ಧಾ-ವಂತ
ಶ್ರದ್ಧೆ
ಶ್ರದ್ಧೆ-ಉ-ತ್ಸಾ-ಹ-ಗಳಿಂದ
ಶ್ರದ್ಧೆ-ಶ-ಕ್ತಿ-ಗಳು
ಶ್ರದ್ಧೆ-ಗಳಿಂದ
ಶ್ರದ್ಧೆ-ಗಳು
ಶ್ರದ್ಧೆಯ
ಶ್ರದ್ಧೆ-ಯನ್ನು
ಶ್ರದ್ಧೆ-ಯನ್ನೂ
ಶ್ರದ್ಧೆ-ಯ-ನ್ನೆಲ್ಲ
ಶ್ರದ್ಧೆ-ಯಿಂದ
ಶ್ರದ್ಧೆ-ಯಿಡಿ
ಶ್ರದ್ಧೆ-ಯಿತ್ತು
ಶ್ರದ್ಧೆ-ಯಿ-ದ್ದುದೇ
ಶ್ರದ್ಧೆ-ಯಿಲ್ಲ
ಶ್ರದ್ಧೆ-ಯಿ-ಲ್ಲ-ದಿ-ರು-ವುದನ್ನು
ಶ್ರದ್ಧೆಯು
ಶ್ರಮ
ಶ್ರಮಕ್ಕೆ
ಶ್ರಮ-ಗಳಲ್ಲಿ
ಶ್ರಮ-ಗಳು
ಶ್ರಮ-ಜೀ-ವಿ-ಗಳು
ಶ್ರಮದ
ಶ್ರಮ-ದಿಂದ
ಶ್ರಮ-ಪಟ್ಟು
ಶ್ರಮ-ವನ್ನು
ಶ್ರಮ-ವ-ಹಿಸಿ
ಶ್ರಮ-ವಾ-ಗಿ-ರ-ಬ-ಹು-ದೆಂ-ಬುದು
ಶ್ರಮ-ವಾ-ಯಿತು
ಶ್ರಮವೇ
ಶ್ರಮಿಸ
ಶ್ರಮಿ-ಸ-ಬ-ಲ್ಲರು
ಶ್ರಮಿ-ಸ-ಬೇ-ಕಾ-ಯಿತು
ಶ್ರಮಿ-ಸ-ಬೇಕು
ಶ್ರಮಿಸಿ
ಶ್ರಮಿ-ಸಿದ
ಶ್ರಮಿ-ಸಿ-ದರು
ಶ್ರಮಿ-ಸಿ-ದಾಗ
ಶ್ರಮಿ-ಸಿ-ದ್ದರ
ಶ್ರಮಿ-ಸಿ-ದ್ದರು
ಶ್ರಮಿ-ಸಿರಿ
ಶ್ರಮಿಸು
ಶ್ರಮಿ-ಸುತ್ತ
ಶ್ರಮಿ-ಸು-ತ್ತಿ-ದ್ದರು
ಶ್ರಮಿ-ಸು-ತ್ತಿ-ರು-ವ-ವನು
ಶ್ರಮಿ-ಸು-ವಂ-ತಹ
ಶ್ರಮಿ-ಸು-ವಂತೆ
ಶ್ರಮಿ-ಸು-ವ-ವರ
ಶ್ರಮಿ-ಸು-ವ-ವರು
ಶ್ರಮಿ-ಸು-ವು-ದಿ-ರಲಿ
ಶ್ರಮಿ-ಸು-ವುದು
ಶ್ರಮಿ-ಸು-ವುದೇ
ಶ್ರವಣ
ಶ್ರಾದ್ಧ-ಕ-ರ್ಮ-ಗಳನ್ನೂ
ಶ್ರಾದ್ಧ-ಕ-ರ್ಮ-ಗ-ಳೆಲ್ಲ
ಶ್ರಾದ್ಧ-ಕ-ರ್ಮದ
ಶ್ರಾದ್ಧ-ಕ-ರ್ಮಾ-ದಿ-ಗಳನ್ನೆಲ್ಲ
ಶ್ರಾದ್ಧ-ಕಾರ್ಯ
ಶ್ರೀ
ಶ್ರೀಕೃಷ್ಣ
ಶ್ರೀಕೃ-ಷ್ಣನ
ಶ್ರೀಕೃ-ಷ್ಣನು
ಶ್ರೀಗಂ-ಧ-ದಂತೆ
ಶ್ರೀಗುರು
ಶ್ರೀಗು-ರು-ಮ-ಹಾ-ರಾ-ಜರ
ಶ್ರೀಗು-ರು-ಮ-ಹಾ-ರಾ-ಜ-ರನ್ನು
ಶ್ರೀಚೈ-ತ-ನ್ಯರ
ಶ್ರೀಚೈ-ತ-ನ್ಯ-ರಿಗೆ
ಶ್ರೀದು-ರ್ಗಾ-ಪೂ-ಜೆಯ
ಶ್ರೀದು-ರ್ಗಾ-ಪೂ-ಜೆ-ಯನ್ನು
ಶ್ರೀದು-ರ್ಗಾ-ಹೋ-ಮ-ವನ್ನು
ಶ್ರೀದು-ರ್ಗೆಯು
ಶ್ರೀನ-ಗರ
ಶ್ರೀನ-ಗ-ರಕ್ಕೆ
ಶ್ರೀನ-ಗ-ರದ
ಶ್ರೀನ-ಗ-ರ-ದಲ್ಲಿ
ಶ್ರೀನ-ಗ-ರ-ದಿಂದ
ಶ್ರೀನ-ಗ-ರ-ವನ್ನು
ಶ್ರೀನ-ಗ-ರವು
ಶ್ರೀನಾ-ಗೇಂ-ದ್ರ-ನಾಥ
ಶ್ರೀಭಾಷ್ಯ
ಶ್ರೀಭಾ-ಷ್ಯದ
ಶ್ರೀಮಂತ
ಶ್ರೀಮಂ-ತನ
ಶ್ರೀಮಂ-ತ-ನಾದ
ಶ್ರೀಮಂ-ತ-ನಿ-ಗಿ-ರುವ
ಶ್ರೀಮಂ-ತ-ಭ-ಕ್ತರು
ಶ್ರೀಮಂ-ತರ
ಶ್ರೀಮಂ-ತ-ರಾ-ಗ-ಬ-ಲ್ಲರು
ಶ್ರೀಮಂ-ತ-ರಿಗೆ
ಶ್ರೀಮಂ-ತರು
ಶ್ರೀಮಂ-ತರೇ
ಶ್ರೀಮಂ-ತ-ರೊ-ಬ್ಬರ
ಶ್ರೀಮಂ-ತ-ಳಲ್ಲ
ಶ್ರೀಮಂ-ತ-ವಾ-ದರೆ
ಶ್ರೀಮಂ-ತೆ-ಯಾ-ದರೂ
ಶ್ರೀಮತಿ
ಶ್ರೀಮಾತೆ
ಶ್ರೀಮಾ-ತೆ-ಯವ
ಶ್ರೀಮಾ-ತೆ-ಯ-ವರ
ಶ್ರೀಮಾ-ತೆ-ಯ-ವ-ರನ್ನು
ಶ್ರೀಮಾ-ತೆ-ಯ-ವ-ರಿಗೆ
ಶ್ರೀಮಾ-ತೆ-ಯ-ವ-ರಿ-ರಲಿ
ಶ್ರೀಮಾ-ತೆ-ಯ-ವರು
ಶ್ರೀಮಾ-ತೆ-ಯ-ವರೂ
ಶ್ರೀಮಾ-ತೆ-ಯ-ವ-ರೊಂ-ದಿ-ಗಿದ್ದ
ಶ್ರೀಯುತ
ಶ್ರೀರಾಮ
ಶ್ರೀರಾ-ಮ-ಕೃಷ್ಣ
ಶ್ರೀರಾ-ಮ-ಕೃ-ಷ್ಣ-ವಿ-ವೇ-ಕಾ-ನಂದ
ಶ್ರೀರಾ-ಮ-ಕೃ-ಷ್ಣ-ದೇ-ವರ
ಶ್ರೀರಾ-ಮ-ಕೃ-ಷ್ಣ-ದೇ-ವರು
ಶ್ರೀರಾ-ಮ-ಕೃ-ಷ್ಣ-ನಾ-ಮ-ವನ್ನು
ಶ್ರೀರಾ-ಮ-ಕೃ-ಷ್ಣರ
ಶ್ರೀರಾ-ಮ-ಕೃ-ಷ್ಣ-ರನ್ನು
ಶ್ರೀರಾ-ಮ-ಕೃ-ಷ್ಣ-ರನ್ನೂ
ಶ್ರೀರಾ-ಮ-ಕೃ-ಷ್ಣ-ರಲ್ಲಿ
ಶ್ರೀರಾ-ಮ-ಕೃ-ಷ್ಣ-ರಿಂದ
ಶ್ರೀರಾ-ಮ-ಕೃ-ಷ್ಣ-ರಿ-ಗಾಗಿ
ಶ್ರೀರಾ-ಮ-ಕೃ-ಷ್ಣ-ರಿಗೆ
ಶ್ರೀರಾ-ಮ-ಕೃ-ಷ್ಣ-ರಿಗೇ
ಶ್ರೀರಾ-ಮ-ಕೃ-ಷ್ಣರು
ಶ್ರೀರಾ-ಮ-ಕೃ-ಷ್ಣ-ರು-ವಿ-ವೇ-ಕಾ-ನಂ-ದರ
ಶ್ರೀರಾ-ಮ-ಕೃ-ಷ್ಣ-ರೂ-ಪ-ದಿಂದ
ಶ್ರೀರಾ-ಮ-ಕೃ-ಷ್ಣರೇ
ಶ್ರೀರಾ-ಮ-ಕೃ-ಷ್ಣ-ರೇ-ನಾ-ದರೂ
ಶ್ರೀರಾ-ಮ-ಕೃ-ಷ್ಣ-ಸಂ-ಘದ
ಶ್ರೀರಾ-ಮ-ಕೃಷ್ಣಾ
ಶ್ರೀರಾ-ಮ-ಚಂ-ದ್ರನು
ಶ್ರೀರಾ-ಮನ
ಶ್ರೀಲಂಕಾ
ಶ್ರೀಶಂ-ಕ-ರಾ-ಚಾರ್ಯ
ಶ್ರೀಶಂ-ಕ-ರಾ-ಚಾ-ರ್ಯ-ರಿಗೆ
ಶ್ರೀಶಾ-ರ-ದಾ-ದೇ-ವಿ-ಯ-ವರು
ಶ್ರುತಿ
ಶ್ರುತಿ-ಗ-ಳಿಗೆ
ಶ್ರುತಿ-ಗಳು
ಶ್ರುತಿ-ಗ-ಳೆಂ-ದರೆ
ಶ್ರುತಿಯ
ಶ್ರುತಿ-ಯೆಂದು
ಶ್ರುತಿ-ಸ್ಮೃ-ತಿ-ಗ-ಳೆಂಬ
ಶ್ರೇಣಿಯ
ಶ್ರೇಣಿ-ಯಂ-ತಿ-ರು-ವ-ವರೆಲ್ಲ
ಶ್ರೇಣಿ-ಯಲ್ಲೂ
ಶ್ರೇಯ
ಶ್ರೇಯ-ಸ್ಕರ
ಶ್ರೇಯ-ಸ್ಸಿ-ಗಾಗಿ
ಶ್ರೇಯ-ಸ್ಸಿ-ಗಾ-ಗಿಯೂ
ಶ್ರೇಯ-ಸ್ಸಿಗೂ
ಶ್ರೇಯ-ಸ್ಸಿಗೆ
ಶ್ರೇಯ-ಸ್ಸೆಂಬ
ಶ್ರೇಯೋ-ಭಿ-ವೃ-ದ್ಧಿ-ಗಾಗಿ
ಶ್ರೇಷ್ಠ
ಶ್ರೇಷ್ಠ-ಉ-ನ್ನತ
ಶ್ರೇಷ್ಠ-ತಮ
ಶ್ರೇಷ್ಠ-ತ-ಮ-ವಾದ
ಶ್ರೇಷ್ಠ-ನಾ-ದ-ವನು
ಶ್ರೇಷ್ಠ-ಮ-ಟ್ಟ-ದ್ದಾ-ಗಿತ್ತು
ಶ್ರೇಷ್ಠ-ಮ-ಟ್ಟದ್ದು
ಶ್ರೇಷ್ಠ-ರೆಂ-ದೇನೂ
ಶ್ರೇಷ್ಠ-ವಾದ
ಶ್ರೇಷ್ಠ-ವಾ-ದವು
ಶ್ರೇಷ್ಠ-ವಾ-ದುದು
ಶ್ರೇಷ್ಠ-ಸ್ಥಾ-ನ-ದಲ್ಲಿ
ಶ್ರೋತೃ
ಶ್ರೋತೃ-ಗಳ
ಶ್ರೋತೃ-ಗ-ಳತ್ತ
ಶ್ರೋತೃ-ಗಳನ್ನು
ಶ್ರೋತೃ-ಗಳಲ್ಲಿ
ಶ್ರೋತೃ-ಗ-ಳಲ್ಲೇ
ಶ್ರೋತೃ-ಗಳಿಂದ
ಶ್ರೋತೃ-ಗ-ಳಿಗೆ
ಶ್ರೋತೃ-ಗಳು
ಶ್ರೋತೃ-ಗಳೂ
ಶ್ಲಾಘಿಸಿ
ಶ್ಲಾಘಿ-ಸಿ-ದರು
ಶ್ಲಾಘಿ-ಸುವ
ಶ್ಲೋಕ
ಶ್ಲೋಕ-ಗಳ
ಶ್ಲೋಕ-ಗಳನ್ನು
ಶ್ಲೋಕ-ಗ-ಳೆಲ್ಲ
ಶ್ಲೋಕದ
ಶ್ಲೋಕ-ದಲ್ಲಿ
ಶ್ಲೋಕ-ವನ್ನು
ಶ್ಲೋಕ-ವನ್ನೂ
ಶ್ಲೋಕ-ವೊಂ-ದನ್ನು
ಶ್ವರದ
ಶ್ವೇತ
ಶ್ವೇತ-ಚ್ಛತ್ರ
ಶ್ವೇತ-ಜ್ಯೋ-ತಿ-ಯಿಂದ
ಶ್ವೇತ-ವ-ರ್ಣ-ಗಳು
ಶ್ವೇತ-ವ-ಸ್ತ್ರ-ವನ್ನು
ಶ್ವೇತಾ-ಶ್ವ-ಗಳು
ಶ್ಶಕ್ತಿ-ಯನ್ನು
ಷಂಡ-ತನ
ಷಂಡ-ರ-ನ್ನಾಗಿ
ಷಣೆ-ಗಳಲ್ಲಿ
ಷತ್ತು-ಗಳ
ಷಷ್ಠಿಗೆ
ಷೆಡ್ಡಿ-ನಲ್ಲಿ
ಷೆಲ್
ಷ್ಟಮಿಯ
ಷ್ಟಿಕೆಯ
ಷ್ಠಾನ
ಸಂಕಟ
ಸಂಕ-ಟಕ್ಕೆ
ಸಂಕ-ಟ-ಗ-ಳ-ನೀ-ಡಾ-ಡುತ
ಸಂಕ-ಟ-ಗ-ಳಿಗೆ
ಸಂಕ-ಟ-ಗಳು
ಸಂಕ-ಟ-ದ-ಲ್ಲಿ-ರುವ
ಸಂಕ-ಟ-ಪ-ಟ್ಟರು
ಸಂಕ-ಟ-ಮ-ಯ-ವಾ-ದುದು
ಸಂಕ-ಟ-ವನ್ನು
ಸಂಕ-ಟ-ವನ್ನೂ
ಸಂಕ-ಟ-ವಾ-ಗ-ದಿ-ರ-ಲಿಲ್ಲ
ಸಂಕ-ಟ-ವಾ-ಗು-ತ್ತಿತ್ತು
ಸಂಕ-ಟ-ವಾ-ಯಿತು
ಸಂಕ-ಲನ
ಸಂಕಲ್ಪ
ಸಂಕ-ಲ್ಪ-ವನ್ನು
ಸಂಕ-ಲ್ಪ-ವೆಂದರೆ
ಸಂಕ-ಲ್ಪ-ಶಕ್ತಿ
ಸಂಕ-ಲ್ಪಿ-ಸಿ-ದರೆ
ಸಂಕ-ಲ್ಪಿ-ಸಿ-ದ್ದೇನೆ
ಸಂಕ-ಷ್ಟ-ದಲ್ಲಿ
ಸಂಕೀ-ರ್ಣ-ಬ-ಹು-ಮುಖ
ಸಂಕೀ-ರ್ಣ-ವಾಗಿ
ಸಂಕೀ-ರ್ತನೆ
ಸಂಕೀ-ರ್ತ-ನೆ-ಗಳೂ
ಸಂಕೀ-ರ್ತ-ನೆಯ
ಸಂಕೀ-ರ್ತ-ನೆ-ಯ-ನ್ನೊಂ-ದಿಷ್ಟು
ಸಂಕೀ-ರ್ತ-ನೆ-ಯಲ್ಲಿ
ಸಂಕು-ಚಿತ
ಸಂಕು-ಚಿ-ತ-ಗೊ-ಳಿ-ಸಿ-ಕೊ-ಳ್ಳುತ್ತ
ಸಂಕು-ಚಿ-ತ-ತೆ-ಯನ್ನು
ಸಂಕೇತ
ಸಂಕೇ-ತ-ಗಳ
ಸಂಕೇ-ತ-ವಾಗಿ
ಸಂಕೇ-ತವೇ
ಸಂಕೋ-ಚ-ಹಿಂ-ಜ-ರಿ-ಕೆ-ಯೆಲ್ಲ
ಸಂಕೋ-ಚ-ಹಿಂ-ಜ-ರಿತ
ಸಂಕೋ-ಚ-ದಿಂದ
ಸಂಕೋ-ಚ-ಪ-ಟ್ಟು-ಕೊಂಡು
ಸಂಕೋ-ಚ-ವಾ-ಯಿತು
ಸಂಕೋ-ಚವೂ
ಸಂಕೋ-ಲೆ-ಗಳ
ಸಂಕ್ಷಿ-ಪ್ತ-ವಾ-ಗಿ-ಯಾ-ದರೂ
ಸಂಕ್ಷೇಪ
ಸಂಕ್ಷೇ-ಪ-ವಾಗಿ
ಸಂಖ್ಯೆ
ಸಂಖ್ಯೆ-ಗಿಂತ
ಸಂಖ್ಯೆಯ
ಸಂಖ್ಯೆ-ಯಲ್ಲಿ
ಸಂಖ್ಯೆ-ಯ-ಲ್ಲಿ-ದ್ದರು
ಸಂಖ್ಯೆಯು
ಸಂಗಡಿ
ಸಂಗ-ಡಿಗ
ಸಂಗ-ಡಿ-ಗರ
ಸಂಗ-ಡಿ-ಗ-ರನ್ನು
ಸಂಗ-ಡಿ-ಗ-ರನ್ನೂ
ಸಂಗ-ಡಿ-ಗ-ರಾದ
ಸಂಗ-ಡಿ-ಗ-ರಿಂದ
ಸಂಗ-ಡಿ-ಗ-ರಿ-ಗಾಗಿ
ಸಂಗ-ಡಿ-ಗ-ರಿಗೆ
ಸಂಗ-ಡಿ-ಗ-ರಿ-ಗೊಂದು
ಸಂಗ-ಡಿ-ಗರು
ಸಂಗ-ಡಿ-ಗರೂ
ಸಂಗ-ಡಿ-ಗ-ರೆಲ್ಲ
ಸಂಗ-ಡಿ-ಗ-ರೊಂ-ದಿಗೆ
ಸಂಗ-ಡಿ-ರೊಂ-ದಿಗೆ
ಸಂಗತಿ
ಸಂಗ-ತಿ-ಗಳನ್ನು
ಸಂಗ-ತಿ-ಗ-ಳಲ್ಲ
ಸಂಗ-ತಿ-ಗ-ಳಿವೆ
ಸಂಗ-ತಿ-ಗಳು
ಸಂಗ-ತಿ-ಯನ್ನು
ಸಂಗ-ತಿ-ಯಾ-ಗಿತ್ತು
ಸಂಗ-ತಿ-ಯೆಂ-ದರೆ
ಸಂಗ-ತಿಯೇ
ಸಂಗ-ತಿ-ಯೇ-ನೆಂ-ದರೆ
ಸಂಗ-ತಿ-ಯೊಂ-ದನ್ನು
ಸಂಗ-ದಲ್ಲಿ
ಸಂಗಮ
ಸಂಗ-ಮ-ದಂ-ತಿತ್ತು
ಸಂಗಾತಿ
ಸಂಗಾ-ತಿ-ಗಳು
ಸಂಗಾ-ತಿ-ಗಳೂ
ಸಂಗಾ-ತಿಗೆ
ಸಂಗಾ-ತಿ-ಯಾದ
ಸಂಗೀತ
ಸಂಗೀ-ತ-ಗಾ-ರರು
ಸಂಗೀ-ತದ
ಸಂಗೀ-ತ-ಮಯ
ಸಂಗೀ-ತ-ಮ-ಯ-ವಾ-ದದ್ದು
ಸಂಗೀ-ತ-ವನ್ನು
ಸಂಗ್ರಹ
ಸಂಗ್ರ-ಹ-ಕಾ-ರ್ಯ-ದಲ್ಲಿ
ಸಂಗ್ರ-ಹಕ್ಕೆ
ಸಂಗ್ರ-ಹಣೆ
ಸಂಗ್ರ-ಹ-ಣೆಯ
ಸಂಗ್ರ-ಹ-ಣೆಯೂ
ಸಂಗ್ರ-ಹದ
ಸಂಗ್ರ-ಹ-ಮಾ-ಡಿ-ಡ-ಬ-ಹುದು
ಸಂಗ್ರ-ಹ-ವಲ್ಲ
ಸಂಗ್ರ-ಹ-ವಾಗಿ
ಸಂಗ್ರ-ಹ-ವಾ-ಗಿದೆ
ಸಂಗ್ರ-ಹ-ವಾ-ಗು-ತ್ತಿತ್ತು
ಸಂಗ್ರ-ಹ-ವಾದ
ಸಂಗ್ರ-ಹಾ-ಲಯ
ಸಂಗ್ರ-ಹಾ-ಲ-ಯ-ವನ್ನೂ
ಸಂಗ್ರ-ಹಿಸಿ
ಸಂಗ್ರ-ಹಿ-ಸಿ-ಟ್ಟು-ಕೊಂ-ಡರು
ಸಂಗ್ರ-ಹಿ-ಸಿದ
ಸಂಗ್ರ-ಹಿ-ಸಿ-ದಳು
ಸಂಗ್ರ-ಹಿ-ಸುತ್ತ
ಸಂಗ್ರ-ಹಿ-ಸು-ತ್ತ-ಲಿದೆ
ಸಂಗ್ರ-ಹಿ-ಸುವ
ಸಂಘ
ಸಂಘ-ಸಂ-ಸ್ಥೆ-ಗ-ಳಿಗೆ
ಸಂಘ-ಕ್ಕಾಗಿ
ಸಂಘಕ್ಕೂ
ಸಂಘಕ್ಕೆ
ಸಂಘ-ಗಳ
ಸಂಘ-ಗಳನ್ನು
ಸಂಘ-ಗಳನ್ನೂ
ಸಂಘ-ಗಳು
ಸಂಘ-ಜೀ-ವ-ನ-ಸೇ-ವಾ-ವ್ರತ
ಸಂಘ-ಟ-ಕರು
ಸಂಘ-ಟನಾ
ಸಂಘ-ಟನೆ
ಸಂಘ-ಟ-ನೆ-ಯಲ್ಲಿ
ಸಂಘ-ಟ-ನೆ-ಯಾ-ಗಿತ್ತು
ಸಂಘ-ಟ-ನೆ-ಯಿ-ಲ್ಲದೆ
ಸಂಘ-ಟಿತ
ಸಂಘ-ಟಿ-ತ-ರಾಗಿ
ಸಂಘ-ಟಿ-ತ-ರಾದ
ಸಂಘ-ಟಿ-ಸುವ
ಸಂಘ-ಟಿ-ಸು-ವಂತೆ
ಸಂಘ-ಟಿ-ಸು-ವು-ದ-ರಲ್ಲೂ
ಸಂಘದ
ಸಂಘ-ದಲ್ಲಿ
ಸಂಘ-ದ-ವರು
ಸಂಘ-ರ್ಷಿ-ಸು-ತ್ತಿ-ರುವ
ಸಂಘ-ವನ್ನು
ಸಂಘವು
ಸಂಘವೆ
ಸಂಘ-ಸಂ-ಸ್ಥೆ-ಗಳು
ಸಂಚ-ಯ-ನ-ದಲ್ಲಿ
ಸಂಚರಿ
ಸಂಚ-ರಿ-ಸ-ಬೇಕು
ಸಂಚ-ರಿಸಿ
ಸಂಚ-ರಿ-ಸಿದ್ದು
ಸಂಚ-ರಿ-ಸಿ-ದ್ದೇನೆ
ಸಂಚ-ರಿಸು
ಸಂಚ-ರಿ-ಸು-ತ್ತಿ-ದ್ದಾಗ
ಸಂಚ-ಲ-ನದ
ಸಂಚಾರ
ಸಂಚಾ-ರದ
ಸಂಚಾ-ರ-ವನ್ನು
ಸಂಚಾ-ರ-ವಾ-ಗು-ವು-ದ-ರಲ್ಲಿ
ಸಂಜೆ
ಸಂಜೆಗೆ
ಸಂಜೆಯ
ಸಂಜೆ-ಯ-ವ-ರೆಗೆ
ಸಂಜೆಯಾ
ಸಂಜೆ-ಯಾ-ಗುತ್ತ
ಸಂಜೆ-ಯಾ-ದರೂ
ಸಂಜೆ-ಯೆಲ್ಲ
ಸಂಜೆಯೇ
ಸಂತ
ಸಂತ-ರಾ-ಷ್ಟ್ರ-ಭ-ಕ್ತರೂ
ಸಂತ-ತಿಗೆ
ಸಂತನ
ಸಂತ-ನನ್ನು
ಸಂತ-ನಲ್ಲಿ
ಸಂತನಾ
ಸಂತ-ನಾದ
ಸಂತನೂ
ಸಂತ-ನೆಂದು
ಸಂತರ
ಸಂತ-ರ-ಮಹಾ
ಸಂತರು
ಸಂತಸ
ಸಂತ-ಸ-ಗೊಂಡ
ಸಂತ-ಸ-ಗೊಂ-ಡರು
ಸಂತ-ಸ-ಗೊ-ಳ್ಳು-ತ್ತಿತ್ತು
ಸಂತ-ಸದಿ
ಸಂತ-ಸ-ವ-ನ್ನುಂ-ಟು-ಮಾ-ಡಿತು
ಸಂತ-ಸ-ವುಂ-ಟು-ಮಾ-ಡಿ-ದರು
ಸಂತ-ಸ-ವುಂ-ಟು-ಮಾ-ಡು-ತ್ತಿತ್ತು
ಸಂತ-ಸ-ವೆ-ಷ್ಟೆಂದು
ಸಂತಾನ
ಸಂತಾ-ನರು
ಸಂತಾಪ
ಸಂತಾ-ಪ-ಗೊಂಡು
ಸಂತಾಲ
ಸಂತಾ-ಲರು
ಸಂತಾಲ್
ಸಂತುಷ್ಟ
ಸಂತು-ಷ್ಟ-ರಾಗಿ
ಸಂತು-ಷ್ಟ-ರಾ-ಗಿ-ದ್ದರು
ಸಂತು-ಷ್ಟ-ರಾ-ಗಿ-ದ್ದ-ರೆಂ-ದರೆ
ಸಂತು-ಷ್ಟ-ರಾ-ದ-ರೆಂ-ದರೆ
ಸಂತು-ಷ್ಟಿ-ಯಿಂದ
ಸಂತೃಪ್ತ
ಸಂತೃ-ಪ್ತ-ನಾ-ಗಿ-ದ್ದೇನೆ
ಸಂತೃ-ಪ್ತ-ರಾಗಿ
ಸಂತೃಪ್ತಿ
ಸಂತೃ-ಪ್ತಿ-ಯಿಂದ
ಸಂತೆ-ಕ-ಟ್ಟೆಗೆ
ಸಂತೆಯೇ
ಸಂತೈ-ಸಲಿ
ಸಂತೈ-ಸಲು
ಸಂತೈ-ಸಿ-ದರು
ಸಂತೈ-ಸುತ್ತ
ಸಂತೋಷ
ಸಂತೋ-ಷ-ಕ್ಕಾಗಿ
ಸಂತೋ-ಷಕ್ಕೆ
ಸಂತೋ-ಷ-ಗ-ಳಿ-ಲ್ಲ-ದ-ಆ-ದರೆ
ಸಂತೋ-ಷ-ಗೊಂ-ಡರು
ಸಂತೋ-ಷ-ಗೊ-ಳಿ-ಸ-ಬ-ಹುದು
ಸಂತೋ-ಷದ
ಸಂತೋ-ಷ-ದಿಂದ
ಸಂತೋ-ಷ-ಪ-ಟ್ಟರು
ಸಂತೋ-ಷ-ಪ-ಡು-ತ್ತಿ-ದ್ದರು
ಸಂತೋ-ಷ-ಪ್ರ-ದ-ವಾ-ಗಿತ್ತು
ಸಂತೋ-ಷ-ವ-ನ್ನುಂ-ಟು-ಮಾ-ಡಿತು
ಸಂತೋ-ಷ-ವ-ನ್ನುಂ-ಟು-ಮಾ-ಡಿ-ದರು
ಸಂತೋ-ಷ-ವ-ನ್ನುಂ-ಟು-ಮಾ-ಡುವ
ಸಂತೋ-ಷ-ವನ್ನೇ
ಸಂತೋ-ಷ-ವಾಗಿ
ಸಂತೋ-ಷ-ವಾ-ಗಿದೆ
ಸಂತೋ-ಷ-ವಾ-ಗಿ-ಬಿ-ಟ್ಟಿತ್ತು
ಸಂತೋ-ಷ-ವಾ-ಗು-ತ್ತದೆ
ಸಂತೋ-ಷ-ವಾ-ಗು-ತ್ತಿದೆ
ಸಂತೋ-ಷ-ವಾ-ಯಿ-ತಾ-ದರೂ
ಸಂತೋ-ಷ-ವಾ-ಯಿತು
ಸಂತೋ-ಷ-ವಾ-ಯಿ-ತೆಂ-ದರೆ
ಸಂತೋ-ಷ-ವುಂ-ಟು-ಮಾ-ಡಿದ
ಸಂತೋ-ಷ-ವೆಲ್ಲ
ಸಂತೋ-ಷ-ವೆ-ಷ್ಟಿ-ದ್ದಿ-ರ-ಬ-ಹು-ದೆಂ-ಬು-ದರ
ಸಂತೋ-ಷವೇ
ಸಂತೋ-ಷಿ-ಸುತ್ತ
ಸಂತೋ-ಷಿ-ಸು-ತ್ತೇನೆ
ಸಂತ್ರ-ಸ್ತರ
ಸಂದ-ಣಿಯೇ
ಸಂದರ್
ಸಂದರ್ಭ
ಸಂದ-ರ್ಭ-ಕ್ಕ-ನು-ಸಾ-ರ-ವಾಗಿ
ಸಂದ-ರ್ಭ-ಕ್ಕಾಗಿ
ಸಂದ-ರ್ಭಕ್ಕೆ
ಸಂದ-ರ್ಭ-ಗಳಲ್ಲಿ
ಸಂದ-ರ್ಭ-ಗ-ಳಲ್ಲೂ
ಸಂದ-ರ್ಭ-ಗ-ಳ-ಲ್ಲೆಲ್ಲ
ಸಂದ-ರ್ಭ-ಗ-ಳಿಗೆ
ಸಂದ-ರ್ಭದ
ಸಂದ-ರ್ಭ-ದಲ್ಲಿ
ಸಂದ-ರ್ಭ-ದಲ್ಲೂ
ಸಂದ-ರ್ಭ-ದಲ್ಲೇ
ಸಂದ-ರ್ಭ-ದ-ಲ್ಲೊಮ್ಮೆ
ಸಂದ-ರ್ಭ-ವನ್ನು
ಸಂದ-ರ್ಭವೂ
ಸಂದ-ರ್ಭ-ವೊ-ದ-ಗ-ಬ-ಹುದು
ಸಂದ-ರ್ಭಾ-ನು-ಸಾರ
ಸಂದ-ರ್ಭಾ-ನು-ಸಾ-ರ-ವಾಗಿ
ಸಂದ-ರ್ಭೋ-ಚಿತ
ಸಂದ-ರ್ಭೋ-ಚಿ-ತ-ವಾಗಿ
ಸಂದ-ರ್ಶಕ
ಸಂದ-ರ್ಶ-ಕ-ನೊಬ್ಬ
ಸಂದ-ರ್ಶ-ಕರ
ಸಂದ-ರ್ಶ-ಕ-ರನ್ನು
ಸಂದ-ರ್ಶ-ಕ-ರಲ್ಲಿ
ಸಂದ-ರ್ಶ-ಕ-ರಿಗೆ
ಸಂದ-ರ್ಶ-ಕರು
ಸಂದ-ರ್ಶ-ಕರೂ
ಸಂದ-ರ್ಶ-ಕ-ರೊಂ-ದಿಗೆ
ಸಂದ-ರ್ಶ-ಕ-ರೊ-ಡನೆ
ಸಂದ-ರ್ಶನ
ಸಂದ-ರ್ಶ-ನ-ಕ್ಕಾಗಿ
ಸಂದ-ರ್ಶ-ನಕ್ಕೆ
ಸಂದ-ರ್ಶ-ನ-ಗ-ಳಿಂ-ದಾಗಿ
ಸಂದ-ರ್ಶ-ನ-ಗಳೇ
ಸಂದ-ರ್ಶ-ನದ
ಸಂದ-ರ್ಶ-ನ-ದಿಂದ
ಸಂದ-ರ್ಶ-ನ-ವನ್ನು
ಸಂದ-ರ್ಶ-ನ-ವಾ-ಗ-ಬ-ಹುದು
ಸಂದ-ರ್ಶ-ನವು
ಸಂದರ್ಶಿ
ಸಂದ-ರ್ಶಿಸ
ಸಂದ-ರ್ಶಿ-ಸ-ದಿ-ದ್ದರೆ
ಸಂದ-ರ್ಶಿಸಿ
ಸಂದ-ರ್ಶಿ-ಸಿದ
ಸಂದ-ರ್ಶಿ-ಸಿ-ದರು
ಸಂದ-ರ್ಶಿ-ಸಿ-ದಾ-ಗಲೂ
ಸಂದ-ರ್ಶಿ-ಸಿದ್ದ
ಸಂದ-ರ್ಶಿ-ಸಿ-ದ್ದರು
ಸಂದ-ರ್ಶಿ-ಸಿ-ದ್ದರೂ
ಸಂದ-ರ್ಶಿ-ಸಿ-ರ-ಲಿಲ್ಲ
ಸಂದ-ರ್ಶಿ-ಸುವ
ಸಂದ-ರ್ಶಿ-ಸು-ವುದು
ಸಂದ-ರ್ಶಿ-ಸು-ವುದೂ
ಸಂದಿ-ಗೊಂ-ದಿ-ಗಳಿಂದ
ಸಂದಿ-ಗ್ಧತೆ
ಸಂದುಕ್ಫೂ
ಸಂದೇಶ
ಸಂದೇ-ಶ-ಗಳ
ಸಂದೇ-ಶ-ಗಳನ್ನು
ಸಂದೇ-ಶ-ಗಳನ್ನೂ
ಸಂದೇ-ಶ-ಗಳು
ಸಂದೇ-ಶದ
ಸಂದೇ-ಶ-ವ-ನ್ನಿನ್ನೂ
ಸಂದೇ-ಶ-ವನ್ನು
ಸಂದೇ-ಶ-ವಾ-ಗಲಿ
ಸಂದೇ-ಶ-ವಾ-ಗಿತ್ತು
ಸಂದೇ-ಶ-ವಿದೆ
ಸಂದೇ-ಶವೂ
ಸಂದೇ-ಶ-ವೊಂ-ದಿ-ರು-ತ್ತದೆ
ಸಂದೇಹ
ಸಂದೇ-ಹ-ಗಳನ್ನು
ಸಂದೇ-ಹ-ಗ-ಳಿಗೆ
ಸಂದೇ-ಹ-ಗ-ಳೆಲ್ಲ
ಸಂದೇ-ಹ-ವನ್ನು
ಸಂದೇ-ಹ-ವಿಲ್ಲ
ಸಂದೇ-ಹ-ವೇ-ಳ-ಬ-ಹುದು
ಸಂಧಿ-ಕಾಲ
ಸಂಧಿ-ಕಾ-ಲ-ದಲ್ಲೇ
ಸಂಧಿ-ಪೂಜೆ
ಸಂಧಿ-ಪೂ-ಜೆಯ
ಸಂಧಿ-ಸ-ಲಿ-ದ್ದರು
ಸಂಧಿ-ಸಿದ
ಸಂಧಿ-ಸಿ-ದರು
ಸಂಧಿ-ಸಿ-ದ್ದರು
ಸಂಧಿ-ಸಿ-ದ್ದು-ದನ್ನು
ಸಂಧಿ-ಸುವ
ಸಂಧ್ಯಾ
ಸಂಧ್ಯಾ-ಕಾ-ಲ-ವಾ-ಗಿತ್ತು
ಸಂಧ್ಯಾ-ರತಿ
ಸಂನ್ಯಾಸ
ಸಂನ್ಯಾ-ಸ-ಸಾ-ಕ್ಷಾ-ತ್ಕಾ-ರ-ಗಳ
ಸಂನ್ಯಾ-ಸ-ಅ-ದರ
ಸಂನ್ಯಾ-ಸ-ಜೀ-ವ-ನದ
ಸಂನ್ಯಾ-ಸದ
ಸಂನ್ಯಾ-ಸ-ದೀಕ್ಷೆ
ಸಂನ್ಯಾ-ಸ-ದೀ-ಕ್ಷೆಯ
ಸಂನ್ಯಾ-ಸ-ದೀ-ಕ್ಷೆ-ಯನ್ನು
ಸಂನ್ಯಾ-ಸ-ಧ-ರ್ಮದ
ಸಂನ್ಯಾ-ಸ-ಧ-ರ್ಮ-ವನ್ನು
ಸಂನ್ಯಾ-ಸ-ಧ-ರ್ಮವೇ
ಸಂನ್ಯಾ-ಸ-ನಾ-ಮ-ಗಳು
ಸಂನ್ಯಾ-ಸ-ವನ್ನು
ಸಂನ್ಯಾ-ಸ-ವೆಂದರೆ
ಸಂನ್ಯಾ-ಸ-ವೆಂಬ
ಸಂನ್ಯಾ-ಸ-ಸಂ-ಪ್ರ-ದಾ-ಯ-ಕ್ಕಿಂ-ತಲೂ
ಸಂನ್ಯಾ-ಸಾ-ಧಿ-ಕಾರ
ಸಂನ್ಯಾಸಿ
ಸಂನ್ಯಾ-ಸಿ
ಸಂನ್ಯಾ-ಸಿ-ಬ್ರಹ್ಮ
ಸಂನ್ಯಾ-ಸಿ-ಬ್ರ-ಹ್ಮ-ಚಾರಿ
ಸಂನ್ಯಾ-ಸಿ-ಬ್ರ-ಹ್ಮ-ಚಾ-ರಿ-ಗ-ಳಿಗೆ
ಸಂನ್ಯಾ-ಸಿ-ಬ್ರ-ಹ್ಮ-ಚಾ-ರಿ-ಗಳು
ಸಂನ್ಯಾ-ಸಿ-ಬ್ರ-ಹ್ಮ-ಚಾ-ರಿ-ಗ-ಳೆಲ್ಲ
ಸಂನ್ಯಾ-ಸಿ-ಮ-ಹಾ-ರಾ-ಜರ
ಸಂನ್ಯಾ-ಸಿಆ
ಸಂನ್ಯಾ-ಸಿ-ಗಳ
ಸಂನ್ಯಾ-ಸಿ-ಗ-ಳ-ತ್ಯಾ-ಗಿ-ಗಳ
ಸಂನ್ಯಾ-ಸಿ-ಗ-ಳಂತೆ
ಸಂನ್ಯಾ-ಸಿ-ಗ-ಳತ್ತ
ಸಂನ್ಯಾ-ಸಿ-ಗ-ಳ-ನ್ನಾಗಿ
ಸಂನ್ಯಾ-ಸಿ-ಗಳನ್ನು
ಸಂನ್ಯಾ-ಸಿ-ಗಳನ್ನೆಲ್ಲ
ಸಂನ್ಯಾ-ಸಿ-ಗ-ಳ-ಲ್ಲದೆ
ಸಂನ್ಯಾ-ಸಿ-ಗ-ಳ-ಲ್ಲವೆ
ಸಂನ್ಯಾ-ಸಿ-ಗಳಲ್ಲಿ
ಸಂನ್ಯಾ-ಸಿ-ಗ-ಳಲ್ಲೂ
ಸಂನ್ಯಾ-ಸಿ-ಗಳಾ
ಸಂನ್ಯಾ-ಸಿ-ಗ-ಳಾದ
ಸಂನ್ಯಾ-ಸಿ-ಗ-ಳಾ-ದಿರಿ
ಸಂನ್ಯಾ-ಸಿ-ಗಳಿಂದ
ಸಂನ್ಯಾ-ಸಿ-ಗ-ಳಿಗೂ
ಸಂನ್ಯಾ-ಸಿ-ಗ-ಳಿಗೆ
ಸಂನ್ಯಾ-ಸಿ-ಗ-ಳಿ-ಗೆಲ್ಲ
ಸಂನ್ಯಾ-ಸಿ-ಗ-ಳಿ-ಗೇ-ನಾ-ದರೂ
ಸಂನ್ಯಾ-ಸಿ-ಗಳು
ಸಂನ್ಯಾ-ಸಿ-ಗಳೂ
ಸಂನ್ಯಾ-ಸಿ-ಗ-ಳೆಲ್ಲ
ಸಂನ್ಯಾ-ಸಿ-ಗಳೇ
ಸಂನ್ಯಾ-ಸಿ-ಗ-ಳೇ-ಆ-ಗಿ-ರ-ಬೇಕು
ಸಂನ್ಯಾ-ಸಿ-ಗ-ಳೊಂ-ದಿಗೆ
ಸಂನ್ಯಾ-ಸಿ-ಗ-ಳೊ-ಬ್ಬರು
ಸಂನ್ಯಾ-ಸಿಗೂ
ಸಂನ್ಯಾ-ಸಿಗೆ
ಸಂನ್ಯಾ-ಸಿ-ಗೇನೂ
ಸಂನ್ಯಾ-ಸಿ-ನಿ-ಯ-ನ್ನಾಗಿ
ಸಂನ್ಯಾ-ಸಿ-ನಿ-ಯ-ರಂತೆ
ಸಂನ್ಯಾ-ಸಿ-ನಿ-ಯ-ರಿ-ದ್ದರು
ಸಂನ್ಯಾ-ಸಿ-ನಿ-ಯ-ರೊಂ-ದಿಗೆ
ಸಂನ್ಯಾ-ಸಿ-ನಿ-ಯಾ-ಗ-ಬೇ-ಕೆಂದೂ
ಸಂನ್ಯಾ-ಸಿ-ನಿ-ಯಾ-ಗುವ
ಸಂನ್ಯಾ-ಸಿಯ
ಸಂನ್ಯಾ-ಸಿ-ಯಂ-ತಲ್ಲ
ಸಂನ್ಯಾ-ಸಿ-ಯಂತೆ
ಸಂನ್ಯಾ-ಸಿ-ಯಂ-ತೆಯೇ
ಸಂನ್ಯಾ-ಸಿ-ಯಲ್ಲ
ಸಂನ್ಯಾ-ಸಿ-ಯಾಗಿ
ಸಂನ್ಯಾ-ಸಿ-ಯಾ-ಗಿ-ದ್ದಾಗ
ಸಂನ್ಯಾ-ಸಿ-ಯಾ-ಗಿ-ದ್ದಾ-ಗಲೇ
ಸಂನ್ಯಾ-ಸಿ-ಯಾ-ಗಿ-ದ್ದೇನೆ
ಸಂನ್ಯಾ-ಸಿ-ಯಾದ
ಸಂನ್ಯಾ-ಸಿ-ಯಿ-ರ-ಬ-ಹುದು
ಸಂನ್ಯಾ-ಸಿಯು
ಸಂನ್ಯಾ-ಸಿಯೂ
ಸಂನ್ಯಾ-ಸಿ-ಯೆಂ-ದರೆ
ಸಂನ್ಯಾ-ಸಿ-ಯೆಂದು
ಸಂನ್ಯಾ-ಸಿ-ಯೆಂಬ
ಸಂನ್ಯಾ-ಸಿಯೇ
ಸಂನ್ಯಾ-ಸಿ-ಯೊಂ-ದಿಗೆ
ಸಂನ್ಯಾ-ಸಿ-ಯೊಬ್ಬ
ಸಂನ್ಯಾ-ಸಿ-ಯೊ-ಬ್ಬನು
ಸಂನ್ಯಾ-ಸಿ-ಯೊ-ಬ್ಬ-ರನ್ನು
ಸಂನ್ಯಾ-ಸಿ-ಯೋ-ರ್ವನ
ಸಂನ್ಯಾ-ಸಿ-ಯೋ-ರ್ವ-ನನ್ನು
ಸಂನ್ಯಾ-ಸಿ-ಯೋ-ರ್ವ-ನಿಗೆ
ಸಂನ್ಯಾಸೀ
ಸಂನ್ಯಾ-ಸೀ-ಬಂ-ಧು-ಗಳಲ್ಲಿ
ಸಂಪ-ತ್ತನ್ನು
ಸಂಪ-ತ್ತಿನ
ಸಂಪತ್ತು
ಸಂಪ-ದ್ಭ-ರಿತ
ಸಂಪ-ನ್ನರು
ಸಂಪರ್ಕ
ಸಂಪ-ರ್ಕಕ್ಕೂ
ಸಂಪ-ರ್ಕಕ್ಕೆ
ಸಂಪ-ರ್ಕ-ದ-ಲ್ಲಿ-ದ್ದರು
ಸಂಪ-ರ್ಕ-ದಿಂದ
ಸಂಪ-ರ್ಕ-ವಾ-ಯಿತು
ಸಂಪ-ರ್ಕವು
ಸಂಪ-ರ್ಕವೇ
ಸಂಪಾ
ಸಂಪಾ-ದಕ
ಸಂಪಾ-ದ-ಕ-ತ್ವ-ದಲ್ಲಿ
ಸಂಪಾ-ದ-ಕ-ನಾದ
ಸಂಪಾ-ದ-ಕ-ರ-ನ್ನಾಗಿ
ಸಂಪಾ-ದ-ಕ-ರಾಗಿ
ಸಂಪಾ-ದ-ಕ-ರಾದ
ಸಂಪಾ-ದ-ಕ-ರಾ-ದರು
ಸಂಪಾ-ದ-ಕರು
ಸಂಪಾ-ದ-ಕರೂ
ಸಂಪಾ-ದ-ಕಿಯೂ
ಸಂಪಾ-ದ-ಕೀ-ಯ-ಗಳನ್ನು
ಸಂಪಾ-ದ-ಕೀ-ಯದ
ಸಂಪಾ-ದ-ಕೀ-ಯ-ದಿಂದ
ಸಂಪಾ-ದನೆ
ಸಂಪಾ-ದಿ-ಸಲು
ಸಂಪಾ-ದಿಸಿ
ಸಂಪಾ-ದಿ-ಸಿ-ಕೊಂ-ಡರು
ಸಂಪಾ-ದಿ-ಸಿ-ಕೊಂಡು
ಸಂಪಾ-ದಿ-ಸಿದ
ಸಂಪಾ-ದಿ-ಸು-ತ್ತಿ-ದ್ದಾರೆ
ಸಂಪು-ಟ-ಕ-ಣ್ದೆ-ರೆ-ಸಿದ
ಸಂಪು-ಟ-ಗಳನ್ನು
ಸಂಪು-ಟ-ಗಳಲ್ಲಿ
ಸಂಪು-ಟ-ದಲ್ಲಿ
ಸಂಪೂರ್ಣ
ಸಂಪೂ-ರ್ಣ-ವಾಗಿ
ಸಂಪ್ರ
ಸಂಪ್ರದಾ
ಸಂಪ್ರ-ದಾಯ
ಸಂಪ್ರ-ದಾ-ಯ-ಮ-ಡಿ-ವಂ-ತಿ-ಕೆ-ಗಳ
ಸಂಪ್ರ-ದಾ-ಯಕ್ಕೆ
ಸಂಪ್ರ-ದಾ-ಯ-ಗಳ
ಸಂಪ್ರ-ದಾ-ಯ-ಗಳನ್ನು
ಸಂಪ್ರ-ದಾ-ಯ-ಗಳನ್ನೂ
ಸಂಪ್ರ-ದಾ-ಯ-ಗಳನ್ನೆಲ್ಲ
ಸಂಪ್ರ-ದಾ-ಯ-ಗ-ಳನ್ನೇ
ಸಂಪ್ರ-ದಾ-ಯ-ಗಳಲ್ಲಿ
ಸಂಪ್ರ-ದಾ-ಯ-ಗ-ಳಲ್ಲೂ
ಸಂಪ್ರ-ದಾ-ಯ-ಗ-ಳ-ವರ
ಸಂಪ್ರ-ದಾ-ಯ-ಗ-ಳಿ-ಗ-ನು-ಗು-ಣ-ವಾಗಿ
ಸಂಪ್ರ-ದಾ-ಯ-ಗ-ಳಿಗೆ
ಸಂಪ್ರ-ದಾ-ಯ-ಗಳು
ಸಂಪ್ರ-ದಾ-ಯ-ಗ-ಳೆಲ್ಲ
ಸಂಪ್ರ-ದಾ-ಯದ
ಸಂಪ್ರ-ದಾ-ಯ-ದಲ್ಲಿ
ಸಂಪ್ರ-ದಾ-ಯ-ದಲ್ಲೂ
ಸಂಪ್ರ-ದಾ-ಯ-ದ-ವರು
ಸಂಪ್ರ-ದಾ-ಯ-ನಿಷ್ಠ
ಸಂಪ್ರ-ದಾ-ಯ-ನಿ-ಷ್ಠ-ರಾ-ಗಿ-ರ-ಬೇಕು
ಸಂಪ್ರ-ದಾ-ಯ-ನಿ-ಷ್ಠ-ರಿಂದ
ಸಂಪ್ರ-ದಾ-ಯ-ನಿ-ಷ್ಠೆಯ
ಸಂಪ್ರ-ದಾ-ಯ-ಬದ್ಧ
ಸಂಪ್ರ-ದಾ-ಯ-ಬ-ದ್ಧ-ವಾಗಿ
ಸಂಪ್ರ-ದಾ-ಯ-ವನ್ನು
ಸಂಪ್ರ-ದಾ-ಯ-ವಾ-ದಿ-ಗಳ
ಸಂಪ್ರ-ದಾ-ಯ-ವಾ-ದಿ-ಗ-ಳಿಗೂ
ಸಂಪ್ರ-ದಾ-ಯ-ವಿ-ರು-ದ್ಧ-ವಾದ
ಸಂಪ್ರ-ದಾ-ಯ-ಶ-ರ-ಣ-ತೆ-ಯನ್ನು
ಸಂಪ್ರ-ದಾ-ಯ-ಸಿ-ದ್ಧ-ವಾದ
ಸಂಪ್ರ-ದಾ-ಯಸ್ಥ
ಸಂಪ್ರ-ದಾ-ಯ-ಸ್ಥ-ರಿಗೆ
ಸಂಪ್ರ-ದಾ-ಯ-ಸ್ಥರು
ಸಂಪ್ರ-ದಾ-ಯ-ಸ್ಥ-ರೆ-ನ್ನಿಸಿ
ಸಂಪ್ರೀ-ತ-ನಾಗಿ
ಸಂಪ್ರೀ-ತ-ನಾದ
ಸಂಪ್ರೀ-ತ-ರಾ-ಗಿ-ದ್ದರು
ಸಂಬಂಧ
ಸಂಬಂ-ಧ-ಗ-ಳಿಂ-ದಲೂ
ಸಂಬಂ-ಧ-ದಂ-ತಹ
ಸಂಬಂ-ಧ-ದಲ್ಲಿ
ಸಂಬಂ-ಧ-ದಿಂದ
ಸಂಬಂ-ಧ-ದಿಂ-ದಾಗಿ
ಸಂಬಂ-ಧ-ಪಟ್ಟ
ಸಂಬಂ-ಧ-ಪ-ಟ್ಟಂತೆ
ಸಂಬಂ-ಧ-ಪ-ಟ್ಟ-ದ್ದಾ-ಗಿದೆ
ಸಂಬಂ-ಧ-ಪ-ಟ್ಟ-ವು-ಗಳು
ಸಂಬಂ-ಧ-ಮೂ-ರ್ತಿಯ
ಸಂಬಂ-ಧ-ವ-ನ್ನಿ-ಟ್ಟು-ಕೊಳ್ಳ
ಸಂಬಂ-ಧ-ವನ್ನು
ಸಂಬಂ-ಧ-ವನ್ನೂ
ಸಂಬಂ-ಧ-ವಾಗಿ
ಸಂಬಂ-ಧ-ವಾದ
ಸಂಬಂ-ಧವು
ಸಂಬಂ-ಧ-ವುಂ-ಟಾ-ಗುವ
ಸಂಬಂ-ಧ-ವೆ-ನ್ನು-ವುದು
ಸಂಬಂ-ಧವೇ
ಸಂಬಂಧಿ
ಸಂಬಂ-ಧಿ-ಕ-ನೊ-ಡನೆ
ಸಂಬಂ-ಧಿಸಿ
ಸಂಬಂ-ಧಿ-ಸಿದ
ಸಂಬಂ-ಧಿ-ಸಿ-ದಂತೆ
ಸಂಬಂ-ಧಿ-ಸಿ-ದಂ-ಥವೂ
ಸಂಬಂ-ಧಿ-ಸಿ-ದುದು
ಸಂಬಂ-ಧಿ-ಸಿ-ದ್ದಲ್ಲ
ಸಂಬಳ
ಸಂಬೋ-ಧಿಸಿ
ಸಂಬೋ-ಧಿ-ಸಿ-ದರೆ
ಸಂಬೋ-ಧಿ-ಸಿ-ದಾಗ
ಸಂಭವ
ಸಂಭ-ವ-ವಿತ್ತು
ಸಂಭ-ವ-ವಿದೆ
ಸಂಭ-ವ-ವಿ-ದೆಯೇ
ಸಂಭ-ವ-ವುಂಟೆ
ಸಂಭ-ವಿ-ಸ-ದಂತೆ
ಸಂಭ-ವಿ-ಸ-ದಿ-ದ್ದರೆ
ಸಂಭ-ವಿ-ಸ-ಬ-ಹು-ದೆಂಬ
ಸಂಭ-ವಿ-ಸಿತ್ತು
ಸಂಭ-ವಿ-ಸಿದ
ಸಂಭ-ವಿ-ಸು-ತ್ತವೆ
ಸಂಭಾ
ಸಂಭಾ-ಷ-ಣಾ-ಸಾ-ಮ-ರ್ಥ್ಯ-ದಿಂದ
ಸಂಭಾ-ಷಣೆ
ಸಂಭಾ-ಷ-ಣೆ-ಪ್ರ-ವ-ಚ-ನ-ಗಳನ್ನು
ಸಂಭಾ-ಷ-ಣೆ-ಗಳ
ಸಂಭಾ-ಷ-ಣೆ-ಗಳನ್ನು
ಸಂಭಾ-ಷ-ಣೆ-ಗಳಲ್ಲಿ
ಸಂಭಾ-ಷ-ಣೆ-ಗ-ಳಿಗೆ
ಸಂಭಾ-ಷ-ಣೆ-ಗಳು
ಸಂಭಾ-ಷ-ಣೆ-ಗಳೂ
ಸಂಭಾ-ಷ-ಣೆಯ
ಸಂಭಾ-ಷ-ಣೆ-ಯನ್ನೂ
ಸಂಭಾ-ಷ-ಣೆ-ಯಲ್ಲಿ
ಸಂಭಾ-ಷ-ಣೆ-ಯ-ಲ್ಲೆಲ್ಲ
ಸಂಭಾ-ಷ-ಣೆಯು
ಸಂಭಾ-ಷ-ಣೆ-ಯೆಲ್ಲ
ಸಂಭಾಷಿ
ಸಂಭಾ-ಷಿಸ
ಸಂಭಾ-ಷಿ-ಸ-ತೊ-ಡ-ಗಿ-ದರು
ಸಂಭಾ-ಷಿ-ಸ-ಬ-ಲ್ಲ-ವರು
ಸಂಭಾ-ಷಿಸಿ
ಸಂಭಾ-ಷಿ-ಸಿ-ದರು
ಸಂಭಾ-ಷಿ-ಸುತ್ತ
ಸಂಭಾ-ಷಿ-ಸು-ತ್ತಿ-ದರು
ಸಂಭಾ-ಷಿ-ಸು-ತ್ತಿ-ದ್ದ-ರೆಂದು
ಸಂಭಾ-ಷಿ-ಸು-ತ್ತಿ-ದ್ದಾಗ
ಸಂಭ್ರಮ
ಸಂಭ್ರ-ಮ-ಕ್ಕಂತೂ
ಸಂಭ್ರ-ಮಕ್ಕೆ
ಸಂಭ್ರ-ಮ-ಗ-ಳೆಲ್ಲ
ಸಂಭ್ರ-ಮದ
ಸಂಭ್ರ-ಮ-ದಿಂದ
ಸಂಮಿ-ಶ್ರ-ಣ-ಗೊಂ-ಡಿ-ದ್ದಾ-ರೆಯೇ
ಸಂಯಮ
ಸಂಯ-ಮ-ಗೊ-ಳಿ-ಸ-ಬ-ಲ್ಲನೋ
ಸಂಯ-ಮವೇ
ಸಂಯುಕ್ತ
ಸಂಯೋ-ಜಿ-ಸಿ-ದ್ದರೆ
ಸಂರ-ಕ್ಷ-ಕರು
ಸಂರ-ಕ್ಷಣಾ
ಸಂರ-ಕ್ಷಿ-ತ-ವಾ-ಗಿವೆ
ಸಂರ-ಕ್ಷಿ-ಸಿ-ಕೊ-ಳ್ಳಲೂ
ಸಂರ-ಕ್ಷಿ-ಸಿ-ಕೊ-ಳ್ಳು-ವುದು
ಸಂರ-ಕ್ಷಿ-ಸಿ-ಟ್ಟಿ-ದ್ದಾರೆ
ಸಂರ-ಕ್ಷಿ-ಸು-ತ್ತೇವೆ
ಸಂವ-ರ್ಧ-ನೆಯ
ಸಂವಾದ
ಸಂವಿ-ಧಾ-ನ-ಗಳು
ಸಂಶಯ
ಸಂಶ-ಯ-ಗಳನ್ನು
ಸಂಶ-ಯ-ಗ-ಳಿಗೆ
ಸಂಶ-ಯ-ಗಳು
ಸಂಶ-ಯ-ಗ್ರ-ಸ್ತ-ರಾ-ಗ-ಬೇ-ಕಾ-ಗಿಲ್ಲ
ಸಂಶ-ಯದ
ಸಂಶ-ಯ-ದೃ-ಷ್ಟಿ-ಯಿಂದ
ಸಂಶ-ಯ-ಮ-ತಿಯೇ
ಸಂಶ-ಯ-ವನ್ನು
ಸಂಶ-ಯ-ವಿಲ್ಲ
ಸಂಶ-ಯವೂ
ಸಂಶ-ಯವೇ
ಸಂಶ-ಯ-ವೇ-ನಿದೆ
ಸಂಶೋ
ಸಂಶೋ-ಧ-ಕರ
ಸಂಶೋ-ಧನೆ
ಸಂಶೋ-ಧ-ನೆ-ಗಳ
ಸಂಶೋ-ಧ-ನೆ-ಗಳನ್ನು
ಸಂಶೋ-ಧ-ನೆ-ಗ-ಳೆಲ್ಲ
ಸಂಶೋ-ಧ-ನೆಯ
ಸಂಶೋ-ಧ-ನೆ-ಯನ್ನು
ಸಂಶೋ-ಧಿಸಿ
ಸಂಶೋ-ಧಿ-ಸುವ
ಸಂಸಾ-ರ-ದಲ್ಲಿ
ಸಂಸಾ-ರ-ದಿಂದ
ಸಂಸಾ-ರ-ವನ್ನೂ
ಸಂಸಾ-ರವು
ಸಂಸಾ-ರ-ಸ-ಮೇ-ತ-ನಾಗಿ
ಸಂಸ್ಕಾರ
ಸಂಸ್ಕಾ-ರ-ಗಳನ್ನು
ಸಂಸ್ಕಾ-ರ-ಗ-ಳುಳ್ಳ
ಸಂಸ್ಕಾ-ರ-ವುಂ-ಟಾ-ಗು-ತ್ತದೆ
ಸಂಸ್ಕೃತ
ಸಂಸ್ಕೃ-ತ-ಗಳಲ್ಲಿ
ಸಂಸ್ಕೃ-ತದ
ಸಂಸ್ಕೃ-ತ-ದಲ್ಲಿ
ಸಂಸ್ಕೃ-ತ-ದ-ಲ್ಲಿ-ಸ್ಥ-ವಿ-ರ-ಪುತ್ರ
ಸಂಸ್ಕೃ-ತ-ದ-ಲ್ಲೊಂದು
ಸಂಸ್ಕೃತಾ
ಸಂಸ್ಕೃ-ತಾ-ಧ್ಯ-ಯನ
ಸಂಸ್ಕೃತಿ
ಸಂಸ್ಕೃ-ತಿ-ಧ್ಯೇ-ಯ-ಗಳನ್ನು
ಸಂಸ್ಕೃ-ತಿ-ಪ-ರಂ-ಪ-ರೆ-ಗಳ
ಸಂಸ್ಕೃ-ತಿ-ಸಂ-ಪ್ರ-ದಾ-ಯ-ಗಳ
ಸಂಸ್ಕೃ-ತಿ-ಗ-ನು-ಗು-ಣ-ವಾಗಿ
ಸಂಸ್ಕೃ-ತಿ-ಗಳ
ಸಂಸ್ಕೃ-ತಿಯ
ಸಂಸ್ಕೃ-ತಿ-ಯನ್ನು
ಸಂಸ್ಕೃ-ತಿ-ಯನ್ನೂ
ಸಂಸ್ಕೃ-ತಿ-ಯಲ್ಲ
ಸಂಸ್ಕೃ-ತಿ-ಯಲ್ಲಿ
ಸಂಸ್ಥಾ-ನದ
ಸಂಸ್ಥಾ-ಪ-ಕ-ರಾದ
ಸಂಸ್ಥಾ-ಪ-ಕರು
ಸಂಸ್ಥಾ-ಪನೆ
ಸಂಸ್ಥಾ-ಪ-ನೆಗೆ
ಸಂಸ್ಥಾ-ಪಿ-ತ-ಳಾದ
ಸಂಸ್ಥಾ-ಪಿಸ
ಸಂಸ್ಥೆ
ಸಂಸ್ಥೆ-ಗಳ
ಸಂಸ್ಥೆ-ಗಳನ್ನು
ಸಂಸ್ಥೆ-ಗ-ಳ-ವರು
ಸಂಸ್ಥೆ-ಗಳೂ
ಸಂಸ್ಥೆಗೆ
ಸಂಸ್ಥೆಗೇ
ಸಂಸ್ಥೆಯ
ಸಂಸ್ಥೆ-ಯನ್ನು
ಸಂಸ್ಥೆ-ಯ-ವರು
ಸಂಸ್ಥೆ-ಯಾಗಿ
ಸಂಸ್ಥೆ-ಯಾದ
ಸಂಸ್ಥೆಯು
ಸಂಸ್ಥೆ-ಯೆಂ-ದರೆ
ಸಂಸ್ಥೆ-ಯೆಂದು
ಸಂಸ್ಥೆ-ಯೊಂ-ದನ್ನು
ಸಂಸ್ಥೆ-ಯೊಂದು
ಸಂಹ-ರಿ-ಸಲು
ಸಂಹ-ರಿ-ಸಿ-ದ್ದ-ರಿಂದ
ಸಂಹಿತೆ
ಸಕಲ
ಸಕ-ಲರ
ಸಕ-ಲ-ರನ್ನೂ
ಸಕ-ಲ-ರಿಗೂ
ಸಕ-ಲರೂ
ಸಕ-ಲ-ವನ್ನೂ
ಸಕ-ಲವೂ
ಸಕಾಲ
ಸಕಾ-ಲಕ್ಕೆ
ಸಕಾ-ಲ-ದಲ್ಲಿ
ಸಕಾ-ಲಿಕ
ಸಕ್ಕ-ರೆ-ಕಾ-ಯಿ-ಲೆ-ಯಲ್ಲಿ
ಸಕ್ಕ-ರೆಯ
ಸಕ್ರಿಯ
ಸಖಾ-ರಾಂ
ಸಖೇ-ದಾ-ಶ್ಚ-ರ್ಯ-ಕರ
ಸಚೇ-ತನ
ಸಚೇ-ತ-ನ-ವಾಗಿ
ಸಚೇ-ತ-ನ-ವಾ-ಗು-ತ್ತಿದೆ
ಸಚ್ಚ-ರಿ-ತೆಯ
ಸಚ್ಚಿ-ದಾ-ನಂದ
ಸಚ್ಚಿ-ದಾ-ನಂ-ದನ
ಸಚ್ಚಿ-ದಾ-ನಂ-ದ-ರಾ-ದರು
ಸಚ್ಚಿ-ದಾ-ನಂ-ದರು
ಸಜ್ಜ-ನ-ದು-ರ್ಜ-ನ-ರೆ-ನ್ನದೆ
ಸಜ್ಜ-ನರ
ಸಜ್ಜ-ನರು
ಸಜ್ಜು-ಗೊ-ಳಿ-ಸ-ಲಾ-ಗಿತ್ತು
ಸಜ್ಜು-ಗೊ-ಳಿ-ಸ-ಲಾ-ಯಿತು
ಸಡ-ಗರ
ಸಡ-ಗ-ರ-ಗಳ
ಸಡಿಲ
ಸಡಿ-ಲ-ಗೊಂಡು
ಸಡಿ-ಲ-ಗೊ-ಳಿಸಿ
ಸಡಿ-ಲ-ವಾಗಿ
ಸಡಿ-ಲಿಸು
ಸಣ್ಣ
ಸಣ್ಣ-ಕೊ-ಳಕ್ಕೂ
ಸಣ್ಣಗೆ
ಸಣ್ಣ-ಪುಟ್ಟ
ಸಣ್ಣ-ಸಣ್ಣ
ಸತ-ತ-ವಾಗಿ
ಸತಾ-ತನ
ಸತೀ-ಶ-ಚಂ-ದ್ರನೂ
ಸತ್
ಸತ್
ಸತ್ಕ-ರಿ-ಸ-ಬೇ-ಕೆಂಬ
ಸತ್ಕ-ರಿ-ಸ-ಲಾ-ಯಿತು
ಸತ್ಕ-ರಿಸಿ
ಸತ್ಕ-ರಿ-ಸಿ-ದಳು
ಸತ್ಕ-ರಿ-ಸಿ-ದ್ದಳು
ಸತ್ಕ-ರಿ-ಸಿ-ರ-ಲಿ-ಲ್ಲವೆ
ಸತ್ಕ-ರ್ಮ-ಫ-ಲವು
ಸತ್ಕ-ರ್ಮ-ವನ್ನು
ಸತ್ಕ-ರ್ಮ-ವೆ-ನಿ-ಸು-ತ್ತದೆ
ಸತ್ಕಾ
ಸತ್ಕಾರ
ಸತ್ಕಾ-ರ-ಕೂ-ಟ-ದ-ಲ್ಲಿ-ದ್ದ-ವ-ರಲ್ಲಿ
ಸತ್ಕಾ-ರ-ಬುದ್ಧಿ
ಸತ್ಕಾ-ರ-ಬು-ದ್ಧಿ-ಯನ್ನು
ಸತ್ಕಾ-ರ-ವನ್ನು
ಸತ್ಕಾ-ರ್ಯ-ಗಳಿಂದ
ಸತ್ಕಾ-ರ್ಯದ
ಸತ್ತ
ಸತ್ತಂತೆ
ಸತ್ತಂ-ತೆಯೇ
ಸತ್ತರೂ
ಸತ್ತರೆ
ಸತ್ತ-ವರ
ಸತ್ತ-ವ-ರನ್ನು
ಸತ್ತಿದೆ
ಸತ್ತಿ-ದ್ದಾರೆ
ಸತ್ತಿಲ್ಲ
ಸತ್ತು
ಸತ್ತು-ಹೋ-ಗಿದೆ
ಸತ್ತು-ಹೋ-ಗು-ವುದೇ
ಸತ್ತು-ಹೋ-ಯಿತು
ಸತ್ತೆನೋ
ಸತ್ತೇ
ಸತ್ತೇ-ಹೋ-ಗಿ-ಬಿ-ಡು-ತ್ತಿ-ದ್ದರು
ಸತ್ತ್ವ-ಶಾ-ಲಿ-ಗ-ಳ-ನ್ನಾ-ಗಿ-ಸು-ತ್ತದೆ
ಸತ್ಪ-ಥ-ದಲ್ಲಿ
ಸತ್ಪ-ರಿ-ಣಾಮ
ಸತ್ಪ-ರಿ-ಣಾ-ಮ-ವ-ನ್ನುಂಟು
ಸತ್ಪ-ರಿ-ಣಾ-ಮ-ವನ್ನೇ
ಸತ್ಪಾ-ತ್ರ-ದಾ-ನ-ಗಳ
ಸತ್ಪು-ತ್ರ-ನಿಗೆ
ಸತ್ಪ್ರ-ಜೆ-ಗ-ಳ-ನ್ನಾ-ಗಿ-ಸ-ಬ-ಹುದು
ಸತ್ಪ್ರ-ಭಾವ
ಸತ್ಪ್ರ-ವೃ-ತ್ತಿಗೆ
ಸತ್ಯ
ಸತ್ಯ-ದಿ-ವ್ಯ-ವಾದ
ಸತ್ಯ-ಗಳನ್ನು
ಸತ್ಯ-ಗಳನ್ನೂ
ಸತ್ಯ-ಗ-ಳ-ನ್ನೊ-ಳ-ಗೊಂಡ
ಸತ್ಯ-ಗಳು
ಸತ್ಯ-ಗ-ಳೆಲ್ಲ
ಸತ್ಯ-ಚಿ-ತ್ರ-ಣ-ವನ್ನು
ಸತ್ಯ-ತೆ-ಯನ್ನು
ಸತ್ಯತ್ವ
ಸತ್ಯದ
ಸತ್ಯ-ದಂತೆ
ಸತ್ಯ-ದ-ಆ-ತ್ಮ-ವಿ-ಚಾ-ರ-ದ-ಭ-ಯ-ನಿ-ವಾ-ರಕ
ಸತ್ಯ-ದ-ರ್ಶ-ನ-ಗಳು
ಸತ್ಯ-ದಲಿ
ಸತ್ಯ-ದಲ್ಲಿ
ಸತ್ಯ-ದಿಂದ
ಸತ್ಯ-ದೆ-ಡೆ-ಗಲ್ಲ
ಸತ್ಯ-ದೆ-ಡೆಗೆ
ಸತ್ಯ-ವ-ನ್ನ-ರಿ-ಯಲು
ಸತ್ಯ-ವನ್ನು
ಸತ್ಯ-ವನ್ನೂ
ಸತ್ಯ-ವನ್ನೇ
ಸತ್ಯ-ವಾ-ಕ್ಯ-ವನ್ನು
ಸತ್ಯ-ವಾಗಿ
ಸತ್ಯ-ವಾ-ಗಿತ್ತು
ಸತ್ಯ-ವಾ-ಗಿದೆ
ಸತ್ಯ-ವಾ-ಗಿ-ದೆಯೋ
ಸತ್ಯ-ವಾ-ಗಿರ
ಸತ್ಯ-ವಾ-ಗು-ವುದನ್ನು
ಸತ್ಯ-ವಾ-ದುದೋ
ಸತ್ಯ-ವಾ-ಯಿತು
ಸತ್ಯ-ವಿ-ರ-ಬ-ಹುದು
ಸತ್ಯ-ವಿ-ರು-ವು-ದಾ-ದರೆ
ಸತ್ಯವು
ಸತ್ಯವೆ
ಸತ್ಯ-ವೆಂದರೆ
ಸತ್ಯ-ವೆಂದು
ಸತ್ಯ-ವೆಂ-ಬಂತೆ
ಸತ್ಯವೇ
ಸತ್ಯವೋ
ಸತ್ಯ-ಶೀ-ಲರು
ಸತ್ಯ-ಸಂ-ಗ-ತಿ-ಯೇ-ನೆಂ-ದರೆ
ಸತ್ಯ-ಸ್ವ-ರೂ-ಪಿ-ಯಾದ
ಸತ್ಯಾಂ-ಶ-ವನ್ನು
ಸತ್ಯಾ-ನ್ವೇ-ಷಿ-ಗ-ಳಾ-ಗಲಿ
ಸತ್ಯಾ-ಸ-ತ್ಯ-ತೆ-ಯನ್ನು
ಸತ್ರಕ್ಕೆ
ಸತ್ವ-ಗಳು
ಸತ್ವ-ಗುಣ
ಸತ್ವ-ಗು-ಣ-ಕ್ಕೇ-ರ-ಬ-ಹುದು
ಸತ್ವ-ಪೂರ್ಣ
ಸತ್ವ-ಯುತ
ಸತ್ವ-ವಿ-ರ-ಲಾ-ರದು
ಸತ್ವ-ಶ-ಕ್ತಿ-ಗಳನ್ನು
ಸತ್ವ-ಸ್ಫೂ-ರ್ತಿ-ಗಳಿಂದ
ಸತ್ವ-ಹೀ-ನ-ರ-ನ್ನಾ-ಗಿ-ಸುವ
ಸತ್ಸಂ-ಕ-ಲ್ಪವು
ಸತ್ಸಂಗ
ಸದ-ರ್ನ್
ಸದ-ವ-ಕಾಶ
ಸದ-ವ-ಕಾ-ಶ-ವಾ-ಗಿತ್ತು
ಸದಸ್ಯ
ಸದ-ಸ್ಯ-ತ್ವಕ್ಕೆ
ಸದ-ಸ್ಯನೂ
ಸದ-ಸ್ಯರ
ಸದ-ಸ್ಯ-ರನ್ನು
ಸದ-ಸ್ಯ-ರ-ನ್ನೆಲ್ಲ
ಸದ-ಸ್ಯ-ರಲ್ಲಿ
ಸದ-ಸ್ಯ-ರ-ಲ್ಲೊ-ಬ್ಬ-ನಾ-ಗಿದ್ದ
ಸದ-ಸ್ಯ-ರಾದ
ಸದ-ಸ್ಯ-ರಿಗೆ
ಸದ-ಸ್ಯ-ರಿ-ಗೆಲ್ಲ
ಸದ-ಸ್ಯರು
ಸದ-ಸ್ಯರೂ
ಸದ-ಸ್ಯ-ರೆಂದು
ಸದ-ಸ್ಯ-ರೆಲ್ಲ
ಸದ-ಸ್ಯ-ರೆ-ಲ್ಲರೂ
ಸದ-ಸ್ಯೆ-ಯ-ನ್ನಾಗಿ
ಸದ-ಸ್ಯೆಯೂ
ಸದಾ
ಸದಾ-ಕಾ-ಲಕ್ಕೂ
ಸದಾ-ಕಾ-ಲ-ದ-ಲ್ಲಿಯೂ
ಸದಾ-ಚಾರ
ಸದಾ-ನಂ-ದರ
ಸದಾ-ನಂ-ದ-ರಿಗೂ
ಸದಾ-ನಂ-ದರು
ಸದಾ-ನಂ-ದರೂ
ಸದಾ-ನಂ-ದ-ರೊಂ-ದಿಗೆ
ಸದಾ-ನಂ-ದ-ರೊ-ಡನೆ
ಸದಾ-ನಂರು
ಸದಾ-ಶಿ-ವನೇ
ಸದಿ-ದ್ದುದು
ಸದು-ಪ-ಯೋಗ
ಸದು-ಪ-ಯೋ-ಗ-ಪ-ಡಿಸಿ
ಸದೆ-ಬ-ಡಿ-ಯಲು
ಸದ್ಗುಣ
ಸದ್ಗು-ಣ-ಗಳನ್ನೂ
ಸದ್ಗು-ಣ-ಗ-ಳೇನೇ
ಸದ್ಗು-ಣ-ವನ್ನು
ಸದ್ಗು-ಣ-ವೊಂ-ದನ್ನು
ಸದ್ಗೃ-ಹಸ್ಥ
ಸದ್ಗ್ರಂ-ಥ-ಗಳ
ಸದ್ದಿ-ಲ್ಲದೆ
ಸದ್ದು
ಸದ್ಭಾ-ವ-ನೆ-ಗ-ಳಿ-ಗಾಗಿ
ಸದ್ಭಾ-ವ-ನೆ-ಯಿಂದ
ಸದ್ಯ
ಸದ್ಯ-ಕ್ಕಂತೂ
ಸದ್ಯಕ್ಕೆ
ಸದ್ಯದ
ಸದ್ಯ-ದಲ್ಲೇ
ಸದ್ವ-ರ್ತ-ನೆ-ಗ-ಳನ್ನೇ
ಸದ್ವಿ-ಚಾ-ರ-ಗಳ
ಸದ್ವಿ-ಚಾ-ರ-ಗಳನ್ನು
ಸನ
ಸನದ
ಸನಾ-ತನ
ಸನಾ-ತ-ನ-ಧ-ರ್ಮ-ದಲ್ಲಿ
ಸನಾ-ತ-ನಿ-ಗ-ಳಾದ
ಸನಿ-ಹ-ದಲ್ಲೇ
ಸನೆ
ಸನ್ನ-ದ್ಧ-ರಾ-ದರು
ಸನ್ನ-ದ್ಧ-ವಾ-ಗಿತ್ತು
ಸನ್ನಾ-ಹ-ದ-ಲ್ಲಿ-ದ್ದಾನೆ
ಸನ್ನಿ
ಸನ್ನಿಗೆ
ಸನ್ನಿಧಿ
ಸನ್ನಿ-ಧಿಗೆ
ಸನ್ನಿ-ಧಿ-ಯಲ್ಲೇ
ಸನ್ನಿ-ಧಿ-ಯಿಂದ
ಸನ್ನಿ-ಧಿ-ಯಿಂ-ದಾಗಿ
ಸನ್ನಿ-ಧಿಯು
ಸನ್ನಿ-ಪಾ-ತಕ್ಕೆ
ಸನ್ನಿ-ವೇಶ
ಸನ್ನಿ-ವೇ-ಶ-ದಲ್ಲಿ
ಸನ್ನಿ-ವೇ-ಶ-ವನ್ನು
ಸನ್ನಿ-ಹಿತ
ಸನ್ನಿ-ಹಿ-ತ-ವಾ-ಗಿತ್ತು
ಸನ್ನಿ-ಹಿ-ತ-ವಾ-ಗಿದೆ
ಸನ್ನಿ-ಹಿ-ತ-ವಾ-ಗು-ತ್ತಿದೆ
ಸನ್ನಿ-ಹಿ-ತ-ವಾ-ಗು-ತ್ತಿ-ದೆ-ಯೆಂಬ
ಸನ್ನಿ-ಹಿ-ತ-ವಾ-ಗು-ವು-ದೆಂದು
ಸನ್ನಿ-ಹಿ-ತ-ವಾ-ದಂತೆ
ಸನ್ನಿ-ಹಿ-ತ-ವಾ-ಯಿತು
ಸನ್ನು
ಸನ್ಮಂ-ಗ-ಳ-ವ-ನ್ನುಂ-ಟು-ಮಾ-ಡಲಿ
ಸನ್ಮಾನ
ಸನ್ಮಾ-ನ-ಗಿ-ನ್ಮಾ-ನ-ಗ-ಳಿಂ-ದೆಲ್ಲ
ಸನ್ಮಾ-ನ-ಕ್ಕಾಗಿ
ಸನ್ಮಾ-ನಿ-ಸಲು
ಸನ್ಮಾ-ನಿಸಿ
ಸನ್ಮಾ-ರ್ಗ-ವನ್ನು
ಸಪ್ತಮಿ
ಸಪ್ಪೆ-ಯಾ-ಗಿ-ಸು-ವಂ-ತಹ
ಸಭಾಂ-ಗಣ
ಸಭಾಂ-ಗ-ಣಕ್ಕೆ
ಸಭಾಂ-ಗ-ಣ-ಗ-ಳನ್ನೇ
ಸಭಾಂ-ಗ-ಣದ
ಸಭಾಂ-ಗ-ಣ-ದಲ್ಲಿ
ಸಭಾಂ-ಗ-ಣ-ದಾಚೆ
ಸಭಾಂ-ಗ-ಣ-ದಾ-ಚೆ-ಯಿಂದ
ಸಭಾಂ-ಗ-ಣ-ದಿಂದ
ಸಭಾಂ-ಗ-ಣ-ವನ್ನೇ
ಸಭಾಂ-ಗ-ಣವು
ಸಭಾ-ಕಂಪ
ಸಭಾ-ಧ್ಯ-ಕ್ಷ-ರಾ-ಗಿದ್ದ
ಸಭಾ-ಧ್ಯ-ಕ್ಷ-ರಾದ
ಸಭಿ-ಕರ
ಸಭಿ-ಕ-ರನ್ನು
ಸಭಿ-ಕ-ರ-ನ್ನು-ದ್ದೇ-ಶಿಸಿ
ಸಭಿ-ಕ-ರಲ್ಲಿ
ಸಭಿ-ಕ-ರ-ಲ್ಲೊ-ಬ್ಬನು
ಸಭಿ-ಕ-ರಿಂದ
ಸಭಿ-ಕ-ರಿ-ಗಾಗಿ
ಸಭಿ-ಕ-ರಿಗೆ
ಸಭಿ-ಕರು
ಸಭಿ-ಕ-ರು-ಮತ್ತು
ಸಭಿ-ಕ-ರೆಲ್ಲ
ಸಭೆ
ಸಭೆ-ಗಳನ್ನು
ಸಭೆ-ಗಳಲ್ಲಿ
ಸಭೆ-ಗೂ-ಡಿ-ಸಿ-ದ್ದರು
ಸಭೆಗೆ
ಸಭೆಯ
ಸಭೆ-ಯನ್ನು
ಸಭೆ-ಯ-ನ್ನು-ದ್ದೇ-ಶಿಸಿ
ಸಭೆ-ಯ-ನ್ನೇನೂ
ಸಭೆ-ಯಲ್ಲಿ
ಸಭೆ-ಯ-ಲ್ಲೆಲ್ಲ
ಸಭೆ-ಯೊಂದ
ಸಭೆ-ಯೊಂ-ದನ್ನು
ಸಭ್ಯ-ನ-ಡತೆ
ಸಭ್ಯ-ವಾಗಿ
ಸಭ್ಯ-ವಾ-ಗಿ-ತ್ತೆ-ನ್ನಲೂ
ಸಮ
ಸಮಂ-ಜ-ಸ-ವಾ-ಗಿ-ದೆಯೋ
ಸಮಂ-ಜ-ಸ-ವಾ-ಗಿ-ದ್ದರೆ
ಸಮಂ-ಜ-ಸ-ವಾದ
ಸಮ-ಗೊ-ಳಿಸಿ
ಸಮಗ್ರ
ಸಮ-ಗ್ರ-ಭಾ-ರ-ತ-ವನ್ನು
ಸಮ-ಗ್ರ-ವಾದ
ಸಮ-ಜಾ-ಯಿ-ಷಿ-ಯನ್ನು
ಸಮ-ತ-ಟ್ಟು-ಗೊ-ಳಿ-ಸಲು
ಸಮ-ತ-ಟ್ಟು-ಗೊ-ಳಿ-ಸು-ವು-ದ-ಕ್ಕಾಗಿ
ಸಮ-ತೋ-ಲನ
ಸಮ-ತೋ-ಲ-ನದ
ಸಮ-ನಾಗಿ
ಸಮ-ನಾ-ಗಿ-ವೆಯೊ
ಸಮ-ನಾದ
ಸಮ-ನಾ-ದದ್ದು
ಸಮನೆ
ಸಮ-ನ್ವಯ
ಸಮ-ನ್ವ-ಯ-ಗಳ
ಸಮ-ನ್ವ-ಯ-ಗೊ-ಳಿ-ಸ-ಲೆಂದೆ
ಸಮ-ನ್ವ-ಯ-ತೆಯು
ಸಮ-ನ್ವ-ಯದ
ಸಮ-ನ್ವ-ಯ-ವನ್ನು
ಸಮಯ
ಸಮ-ಯಕ್ಕೆ
ಸಮ-ಯ-ಗಳಲ್ಲಿ
ಸಮ-ಯದ
ಸಮ-ಯ-ದಲ್ಲಿ
ಸಮ-ಯ-ದ-ಲ್ಲಿ-ದ್ದಷ್ಟು
ಸಮ-ಯ-ದಲ್ಲೂ
ಸಮ-ಯ-ದಲ್ಲೇ
ಸಮ-ಯ-ದಷ್ಟೇ
ಸಮ-ಯ-ಪ್ರ-ಜ್ಞೆ-ಯಿಂದ
ಸಮ-ಯ-ಪ್ರ-ಜ್ಞೆ-ಯಿ-ಲ್ಲ-ದಿ-ರು-ವುದನ್ನು
ಸಮ-ಯ-ವ-ನ್ನಾ-ದರೂ
ಸಮ-ಯ-ವನ್ನು
ಸಮ-ಯ-ವ-ನ್ನೆಲ್ಲ
ಸಮ-ಯ-ವಲ್ಲ
ಸಮ-ಯ-ವಾ-ದಾ-ಗ-ಲೆಲ್ಲ
ಸಮ-ಯ-ವಾ-ದ್ದ-ರಿಂದ
ಸಮ-ಯ-ವಿತ್ತು
ಸಮ-ಯ-ವಿದೆ
ಸಮ-ಯ-ವಿ-ರ-ಲಿಲ್ಲ
ಸಮ-ಯ-ವಿಲ್ಲ
ಸಮ-ಯ-ವೀಗ
ಸಮ-ಯವೂ
ಸಮ-ಯ-ವೆಲ್ಲ
ಸಮ-ಯ-ವೊ-ದ-ಗಿ-ದಾಗ
ಸಮ-ಯಾ-ಸ-ಮ-ಯ-ವೆ-ನ್ನದೆ
ಸಮರ
ಸಮ-ರ-ನೀತಿ
ಸಮ-ರಸ
ಸಮ-ರ-ಸ-ವಾಗಿ
ಸಮರ್ಥ
ಸಮ-ರ್ಥ-ಕ-ನೊ-ಬ್ಬನು
ಸಮ-ರ್ಥ-ಕರೂ
ಸಮ-ರ್ಥ-ನಾಗಿ
ಸಮ-ರ್ಥ-ನಾ-ಗಿ-ದ್ದರೆ
ಸಮ-ರ್ಥ-ನಾ-ಗಿ-ದ್ದೇನೆ
ಸಮ-ರ್ಥ-ನಾದ
ಸಮ-ರ್ಥ-ನಾದೆ
ಸಮ-ರ್ಥ-ರಾ-ಗಲಿ
ಸಮ-ರ್ಥ-ರಾ-ಗಿ-ದ್ದರು
ಸಮ-ರ್ಥ-ರಾ-ಗಿ-ದ್ದಾ-ರೆಯೇ
ಸಮ-ರ್ಥ-ರಾ-ಗಿ-ದ್ದೀರಿ
ಸಮ-ರ್ಥ-ರಾ-ಗಿದ್ದು
ಸಮ-ರ್ಥ-ರಾ-ಗಿ-ರ-ಲಿಲ್ಲ
ಸಮ-ರ್ಥ-ರಾ-ಗಿಲ್ಲ
ಸಮ-ರ್ಥ-ರಾ-ಗು-ತ್ತಾರೆ
ಸಮ-ರ್ಥ-ರಾ-ಗು-ವಂತೆ
ಸಮ-ರ್ಥ-ರಾದ
ಸಮ-ರ್ಥ-ರಾ-ದರು
ಸಮ-ರ್ಥ-ರಾ-ದರೆ
ಸಮ-ರ್ಥ-ರಾ-ದ-ರೆ-ನ್ನ-ಬ-ಹುದು
ಸಮ-ರ್ಥರು
ಸಮ-ರ್ಥ-ಳಾ-ಗಿ-ದ್ದೇನೆ
ಸಮ-ರ್ಥ-ಳಾ-ದಳು
ಸಮ-ರ್ಥ-ವಾಗಿ
ಸಮ-ರ್ಥ-ವಾ-ಗಿಲ್ಲ
ಸಮ-ರ್ಥ-ವಾದ
ಸಮರ್ಥಿ
ಸಮ-ರ್ಥಿ-ಸಲು
ಸಮ-ರ್ಥಿಸಿ
ಸಮ-ರ್ಥಿ-ಸಿ-ಕೊಂ-ಡರು
ಸಮ-ರ್ಥಿ-ಸಿ-ಕೊಂ-ಡಿ-ದ್ದಿ-ರ-ಬ-ಹುದು
ಸಮ-ರ್ಥಿ-ಸಿ-ಕೊ-ಳ್ಳಲು
ಸಮ-ರ್ಥಿ-ಸಿ-ಕೊ-ಳ್ಳುತ್ತ
ಸಮ-ರ್ಥಿ-ಸಿ-ಕೊ-ಳ್ಳುವ
ಸಮ-ರ್ಥಿ-ಸಿ-ದುದು
ಸಮ-ರ್ಥಿ-ಸು-ತ್ತಾ-ರೆಂದು
ಸಮ-ರ್ಥಿ-ಸು-ತ್ತಿ-ದ್ದಳು
ಸಮ-ರ್ಥಿ-ಸುವ
ಸಮ-ರ್ಥಿ-ಸು-ವಂ-ಥ-ವರೂ
ಸಮ-ರ್ಪಕ
ಸಮ-ರ್ಪ-ಕ-ವಾಗಿ
ಸಮ-ರ್ಪ-ಕ-ವಾ-ದ-ದ್ದಲ್ಲ
ಸಮ-ರ್ಪ-ಣಾ-ಪೂ-ರ್ವ-ಕ-ವಾಗಿ
ಸಮ-ರ್ಪಣೆ
ಸಮ-ರ್ಪಿ-ತ-ಳಾ-ದ-ವಳು
ಸಮ-ರ್ಪಿ-ಸಲಾ
ಸಮ-ರ್ಪಿ-ಸ-ಲಾ-ಯಿತು
ಸಮ-ರ್ಪಿ-ಸಲು
ಸಮ-ರ್ಪಿಸಿ
ಸಮ-ರ್ಪಿ-ಸಿ-ಕೊಂಡ
ಸಮ-ರ್ಪಿ-ಸಿ-ಕೊಂ-ಡ-ರೆಂ-ಬು-ದನ್ನು
ಸಮ-ರ್ಪಿ-ಸಿ-ಕೊ-ಳ್ಳಲು
ಸಮ-ರ್ಪಿ-ಸಿ-ಕೊಳ್ಳಿ
ಸಮ-ರ್ಪಿ-ಸಿ-ಕೊ-ಳ್ಳು-ವೆ-ನೆಂದು
ಸಮ-ರ್ಪಿ-ಸಿದ
ಸಮ-ರ್ಪಿ-ಸಿ-ದರು
ಸಮ-ರ್ಪಿ-ಸಿ-ದ್ದಾಳೆ
ಸಮ-ರ್ಪಿ-ಸಿ-ದ್ದೇ-ವೆಂದು
ಸಮ-ರ್ಪಿ-ಸಿ-ಬಿ-ಟ್ಟರು
ಸಮ-ರ್ಪಿ-ಸುತ್ತ
ಸಮ-ರ್ಪಿ-ಸು-ತ್ತಿ-ದ್ದರು
ಸಮ-ರ್ಪಿ-ಸುವ
ಸಮ-ವಸ್ತ್ರ
ಸಮ-ವಾ-ಗಿತ್ತು
ಸಮಷ್ಟಿ
ಸಮಸ್ತ
ಸಮಸ್ಯೆ
ಸಮ-ಸ್ಯೆ-ಗಳ
ಸಮ-ಸ್ಯೆ-ಗಳನ್ನು
ಸಮ-ಸ್ಯೆ-ಗಳನ್ನೆಲ್ಲ
ಸಮ-ಸ್ಯೆ-ಗಳಿಂದ
ಸಮ-ಸ್ಯೆ-ಗ-ಳಿ-ಗಿಂ-ತಲೂ
ಸಮ-ಸ್ಯೆ-ಗ-ಳಿಗೂ
ಸಮ-ಸ್ಯೆ-ಗ-ಳಿಗೆ
ಸಮ-ಸ್ಯೆ-ಗ-ಳಿ-ಗೆಲ್ಲ
ಸಮ-ಸ್ಯೆ-ಗಳು
ಸಮ-ಸ್ಯೆ-ಗಳೂ
ಸಮ-ಸ್ಯೆ-ಗ-ಳೆ-ಡೆಗೆ
ಸಮ-ಸ್ಯೆ-ಗ-ಳೆಲ್ಲ
ಸಮ-ಸ್ಯೆಗೆ
ಸಮ-ಸ್ಯೆಯ
ಸಮ-ಸ್ಯೆ-ಯ-ನ್ನಿ-ಟ್ಟರು
ಸಮ-ಸ್ಯೆ-ಯನ್ನು
ಸಮ-ಸ್ಯೆಯೇ
ಸಮಾ
ಸಮಾ-ಗ-ಮ-ವನ್ನು
ಸಮಾ-ಚಾ-ರ-ವನ್ನು
ಸಮಾಜ
ಸಮಾ-ಜಕ್ಕೂ
ಸಮಾ-ಜಕ್ಕೆ
ಸಮಾ-ಜ-ಜೀ-ವ-ನ-ವೆಂ-ಬುದು
ಸಮಾ-ಜದ
ಸಮಾ-ಜ-ದಲ್ಲಿ
ಸಮಾ-ಜ-ದ-ವರು
ಸಮಾ-ಜ-ದಿಂದ
ಸಮಾ-ಜ-ವನ್ನು
ಸಮಾ-ಜ-ವ-ನ್ನುಂ-ಟು-ಮಾ-ಡ-ಬ-ಲ್ಲುದು
ಸಮಾ-ಜ-ವಾ-ಗಲಿ
ಸಮಾ-ಜ-ವಾ-ದದ
ಸಮಾ-ಜವು
ಸಮಾ-ಜವೂ
ಸಮಾ-ಜ-ವೆಂದರೆ
ಸಮಾ-ಜವೇ
ಸಮಾ-ಜ-ಶಾ-ಸ್ತ್ರ-ಜ್ಞ-ನಂತೆ
ಸಮಾ-ಜ-ಶಾ-ಸ್ತ್ರ-ಜ್ಞರು
ಸಮಾ-ಜ-ಶಾ-ಸ್ತ್ರದ
ಸಮಾ-ಜ-ಸೇ-ವ-ಕರು
ಸಮಾ-ಜ-ಸೇ-ವೆ-ಯನ್ನು
ಸಮಾ-ಜ-ಹಿ-ತ-ಚಿಂ-ತ-ನೆಯ
ಸಮಾ-ಜೀ-ಯ-ರಾದ
ಸಮಾ-ಜೀ-ಯರು
ಸಮಾ-ಧಾನ
ಸಮಾ-ಧಾ-ನ-ಆ-ಶ್ರ-ಯ-ಗ-ಳಿ-ಗಾಗಿ
ಸಮಾ-ಧಾ-ನ-ಹರ್ಷ
ಸಮಾ-ಧಾ-ನ-ಕ-ರ-ವಾದ
ಸಮಾ-ಧಾ-ನ-ಕ್ಕಾಗಿ
ಸಮಾ-ಧಾ-ನ-ಗೊಂ-ಡರು
ಸಮಾ-ಧಾ-ನ-ಚಿ-ತ್ತ-ರಾ-ಗಿ-ದ್ದುದು
ಸಮಾ-ಧಾ-ನ-ಪ-ಟ್ಟು-ಕೊ-ಳ್ಳ-ಬ-ಹುದು
ಸಮಾ-ಧಾ-ನ-ವ-ನ್ನುಂಟು
ಸಮಾ-ಧಾ-ನ-ವ-ನ್ನುಂ-ಟು-ಮಾ-ಡ-ಲಿಲ್ಲ
ಸಮಾ-ಧಾ-ನ-ವ-ನ್ನುಂ-ಟು-ಮಾ-ಡಿತು
ಸಮಾ-ಧಾ-ನವಾ
ಸಮಾ-ಧಾ-ನ-ವಾ-ಗದೆ
ಸಮಾ-ಧಾ-ನ-ವಾ-ಗ-ಲಿಲ್ಲ
ಸಮಾ-ಧಾ-ನ-ವಾ-ಗಿದೆ
ಸಮಾ-ಧಾ-ನ-ವಾ-ದದ್ದು
ಸಮಾ-ಧಾ-ನ-ವಾ-ಯಿತು
ಸಮಾಧಿ
ಸಮಾ-ಧಿ-ಇ-ವು-ಗಳ
ಸಮಾ-ಧಿ-ಮಗ್ನ
ಸಮಾ-ಧಿ-ಮ-ಗ್ನ-ರಾಗಿ
ಸಮಾ-ಧಿಯ
ಸಮಾ-ಧಿ-ಯಲ್ಲಿ
ಸಮಾ-ಧಿ-ಯಿಂದ
ಸಮಾ-ಧಿಯು
ಸಮಾ-ಧಿ-ಸ್ಥಿ-ತಿ-ಗೇ-ರ-ದಂತೆ
ಸಮಾ-ಧಿ-ಸ್ಥಿ-ತಿ-ಗೇ-ರಿ-ರ-ಬೇ-ಕೆಂದು
ಸಮಾ-ಧಿ-ಸ್ಥಿ-ತಿ-ಯನ್ನು
ಸಮಾ-ಧಿ-ಸ್ಥಿ-ತಿಯು
ಸಮಾನ
ಸಮಾ-ನ-ವಾಗಿ
ಸಮಾ-ನ-ವಾದ
ಸಮಾ-ನಾಂ-ಶ-ವೆಂದರೆ
ಸಮಾ-ನಾ-ಭಿ-ಪ್ರಾಯ
ಸಮಾನೋ
ಸಮಾ-ನೋ-ದ್ದೇ-ಶ-ದಿಂದ
ಸಮಾ-ಪ್ತ-ವಾ-ಯಿತು
ಸಮಾ-ರಂಭ
ಸಮಾ-ರಂ-ಭ-ಕ್ಕಾಗಿ
ಸಮಾ-ರಂ-ಭಕ್ಕೆ
ಸಮಾ-ರಂ-ಭ-ಗಳಲ್ಲಿ
ಸಮಾ-ರಂ-ಭ-ಗ-ಳಲ್ಲೂ
ಸಮಾ-ರಂ-ಭ-ಗ-ಳ-ಲ್ಲೆಲ್ಲ
ಸಮಾ-ರಂ-ಭ-ಗಳಿಂದ
ಸಮಾ-ರಂ-ಭ-ಗ-ಳಿ-ಗಿಂ-ತಲೂ
ಸಮಾ-ರಂ-ಭ-ಗ-ಳಿಗೆ
ಸಮಾ-ರಂ-ಭ-ಗಳು
ಸಮಾ-ರಂ-ಭ-ಗ-ಳು-ಭಾ-ಷಣ
ಸಮಾ-ರಂ-ಭದ
ಸಮಾ-ರಂ-ಭ-ದಲ್ಲಿ
ಸಮಾ-ರಂ-ಭ-ದಲ್ಲೂ
ಸಮಾ-ರಂ-ಭ-ವನ್ನು
ಸಮಾ-ರಂ-ಭ-ವಾದ
ಸಮಾ-ರಂ-ಭವು
ಸಮಾ-ರಂ-ಭವೇ
ಸಮಾ-ರಂ-ಭ-ವೇನೂ
ಸಮಾ-ರಂ-ಭ-ವೊಂ-ದನ್ನು
ಸಮಾ-ರಾ-ಧನೆ
ಸಮಾ-ಲೋ-ಚನೆ
ಸಮಾ-ಲೋ-ಚಿಸಿ
ಸಮಾ-ಲೋ-ಚಿ-ಸು-ತ್ತಿ-ದ್ದರು
ಸಮಾ-ಲೋ-ಚಿ-ಸುವ
ಸಮಾ-ವಿ-ಶತು
ಸಮಿತಿ
ಸಮಿ-ತಿ-ಗಳನ್ನು
ಸಮಿ-ತಿಯ
ಸಮಿ-ತಿ-ಯ-ವ-ರಿಗೂ
ಸಮಿ-ತಿ-ಯ-ವರು
ಸಮಿ-ತಿ-ಯೊಂ-ದನ್ನು
ಸಮೀ-ಕ-ರ-ಣ-ಗೊಂ-ಡಿ-ದ್ದುವು
ಸಮೀಪ
ಸಮೀ-ಪಕ್ಕೆ
ಸಮೀ-ಪದ
ಸಮೀ-ಪ-ದಲ್ಲಿ
ಸಮೀ-ಪ-ದ-ಲ್ಲಿ-ದ್ದಾನೆ
ಸಮೀ-ಪ-ದ-ಲ್ಲಿ-ದ್ದೀಯೆ
ಸಮೀ-ಪ-ದ-ಲ್ಲಿಯೇ
ಸಮೀ-ಪ-ದಲ್ಲೇ
ಸಮೀ-ಪ-ದ-ವ-ರಾ-ಗು-ತ್ತಿ-ದ್ದರು
ಸಮೀ-ಪ-ದ-ವರೇ
ಸಮೀ-ಪ-ದಿಂದ
ಸಮೀ-ಪ-ವಾ-ದಂ-ತಹ
ಸಮೀ-ಪ-ವಾ-ದದ್ದು
ಸಮೀ-ಪಿ-ಸಿತು
ಸಮೀ-ಪಿ-ಸಿ-ದಂತೆ
ಸಮೀ-ಪಿ-ಸಿ-ದಂ-ತೆಲ್ಲ
ಸಮೀ-ಪಿಸು
ಸಮೀ-ಪಿ-ಸು-ತ್ತದೆ
ಸಮೀ-ಪಿ-ಸು-ತ್ತಲೇ
ಸಮೀ-ಪಿ-ಸು-ತ್ತಿದೆ
ಸಮೀ-ಪಿ-ಸು-ತ್ತಿ-ದ್ದಂತೆ
ಸಮೀ-ಪಿ-ಸು-ತ್ತಿ-ದ್ದು-ದ-ರಿಂದ
ಸಮೀ-ಪಿ-ಸು-ತ್ತಿ-ದ್ದೇನೆ
ಸಮೀ-ಪಿ-ಸು-ತ್ತಿ-ರು-ವು-ದರ
ಸಮು-ದಾ-ಯದ
ಸಮುದ್ರ
ಸಮು-ದ್ರ-ಇವು
ಸಮು-ದ್ರ-ಕ್ಕಿ-ರುವ
ಸಮು-ದ್ರಕ್ಕೆ
ಸಮು-ದ್ರ-ಕ್ಕೆ-ದು-ರಾದ
ಸಮು-ದ್ರ-ತೀ-ರದ
ಸಮು-ದ್ರ-ತೀ-ರ-ದ-ಲ್ಲಿದ್ದ
ಸಮು-ದ್ರದ
ಸಮು-ದ್ರ-ದಂ-ತಿತ್ತು
ಸಮು-ದ್ರ-ದಲ್ಲಿ
ಸಮು-ದ್ರ-ಪ್ರ-ಯಾಣ
ಸಮು-ದ್ರ-ಪ್ರ-ಯಾ-ಣದ
ಸಮು-ದ್ರ-ಮ-ಟ್ಟ-ದಿಂದ
ಸಮು-ದ್ರ-ಮಾರ್ಗ
ಸಮು-ದ್ರ-ಮಾ-ರ್ಗ-ವಾಗಿ
ಸಮು-ದ್ರ-ಯಾನ
ಸಮು-ದ್ರ-ಯಾ-ನದ
ಸಮು-ದ್ರ-ರಾಜ
ಸಮು-ದ್ರ-ರಾ-ಜನೂ
ಸಮು-ದ್ರ-ರೋ-ಗ-ವುಂ-ಟಾ-ಯಿತು
ಸಮು-ದ್ರ-ವನ್ನು
ಸಮು-ನ್ನ-ತಿ-ಯನ್ನು
ಸಮೂಹ
ಸಮೂ-ಹ-ಗಾ-ನ-ವನ್ನು
ಸಮೃದ್ಧ
ಸಮೇತ
ಸಮೇ-ತ-ನಾಗಿ
ಸಮೇ-ತ-ರಾಗಿ
ಸಮ್ಮ-ತ-ವಾಗಿ
ಸಮ್ಮತಿ
ಸಮ್ಮ-ತಿಯ
ಸಮ್ಮ-ತಿಸಿ
ಸಮ್ಮ-ತಿ-ಸಿ-ದರು
ಸಮ್ಮ-ತಿ-ಸಿ-ದಾಗ
ಸಮ್ಮ-ತಿ-ಸು-ತ್ತಾರೆ
ಸಮ್ಮಾನ
ಸಮ್ಮಾ-ನಿ-ತ-ರಾದ
ಸಮ್ಮಿ-ಲನ
ಸಮ್ಮಿ-ಲ-ನ-ಗೊಂಡು
ಸಮ್ಮಿ-ಲ-ನ-ದಂ-ತಿತ್ತು
ಸಮ್ಮಿ-ಳ-ನ-ಗೊ-ಳಿಸಿ
ಸಮ್ಮಿ-ಳಿ-ತ-ಗೊಂಡ
ಸಮ್ಮಿ-ಳಿ-ತ-ಗೊ-ಳ್ಳ-ಬೇ-ಕೆಂದು
ಸಮ್ಮಿ-ಶ್ರಣ
ಸಮ್ಮಿ-ಶ್ರ-ಣ-ವಾ-ಗಿ-ರು-ವಂತೆ
ಸಮ್ಮುಖ
ಸಮ್ಮು-ಖ-ದಲ್ಲಿ
ಸಮ್ಮೇ-ಳನ
ಸಮ್ಮೇ-ಳ-ನ-ಕ್ಕಾಗಿ
ಸಮ್ಮೇ-ಳ-ನ-ಕ್ಕಾ-ಗಿಯೇ
ಸಮ್ಮೇ-ಳ-ನಕ್ಕೆ
ಸಮ್ಮೇ-ಳ-ನದ
ಸಮ್ಮೇ-ಳ-ನ-ದಲ್ಲಿ
ಸಮ್ಮೇ-ಳ-ನ-ದಲ್ಲೇ
ಸಮ್ಮೇ-ಳ-ನ-ದಿಂದ
ಸಮ್ಮೇ-ಳ-ನ-ವನ್ನು
ಸಮ್ಮೇ-ಳ-ನ-ವಾಗಿ
ಸಮ್ಮೇ-ಳ-ನವು
ಸಮ್ಮೇ-ಳ-ನ-ವೆಲ್ಲ
ಸಮ್ಮೇ-ಳ-ನ-ವೊಂ-ದನ್ನು
ಸಮ್ಮೇ-ಳ-ನಾ-ಧ್ಯ-ಕ್ಷ-ರಾದ
ಸಮ್ಮೇ-ಳ-ನಾ-ಧ್ಯ-ಕ್ಷರು
ಸಮ್ಮೋ-ಹ-ಕ-ವಾದ
ಸಮ್ಮೋ-ಹಿನೀ
ಸರ-ಕಾ-ರದ
ಸರಕು
ಸರ-ಣಿ-ಗಳನ್ನು
ಸರ-ಣಿಯ
ಸರ-ಣಿ-ಯನ್ನು
ಸರ-ಣಿ-ಯನ್ನೂ
ಸರ-ಣಿ-ಯಲ್ಲಿ
ಸರ-ಣಿಯೂ
ಸರದಿ
ಸರ-ದಿಯ
ಸರ-ಪಣಿ
ಸರ-ಪ-ಣಿ-ಯಲ್ಲಿ
ಸರ-ಪ-ಣಿ-ಯಿಂದ
ಸರಲಾ
ಸರಳ
ಸರ-ಳ-ಅ-ಶಿ-ಕ್ಷಿತ
ಸರ-ಳ-ಮುಗ್ಧ
ಸರ-ಳಈ
ಸರ-ಳ-ಜೀ-ವಿ-ಗ-ಳೊಂ-ದಿಗೆ
ಸರ-ಳ-ತ-ನ-ವನ್ನು
ಸರ-ಳತೆ
ಸರ-ಳ-ತೆಯ
ಸರ-ಳ-ತೆ-ಯನ್ನು
ಸರ-ಳ-ತೆ-ಯಿಂದ
ಸರ-ಳನ್ನು
ಸರ-ಳ-ವಾಗಿ
ಸರ-ಳ-ವಾದ
ಸರ-ಳ-ವಾ-ದ-ವು-ಗಳು
ಸರ-ಳವೂ
ಸರಳಾ
ಸರಸ
ಸರ-ಸ-ವಾಗಿ
ಸರಸಿ
ಸರ-ಸ್ವ-ತಿಯ
ಸರಾ-ಗ-ವಾಗಿ
ಸರಿ
ಸರಿ-ಇ-ಲ್ಲದೆ
ಸರಿ-ದು-ಕೊ-ಳ್ಳು-ತ್ತಿ-ದ್ದರು
ಸರಿ-ದು-ಬಿ-ಟ್ಟರೆ
ಸರಿ-ದೂ-ಗ-ಬ-ಲ್ಲರು
ಸರಿ-ಪ-ಡಿ-ಸ-ಬೇ-ಕಾ-ದರೆ
ಸರಿ-ಪ-ಡಿ-ಸ-ಲಾ-ಗ-ದಂ-ತಹ
ಸರಿ-ಪ-ಡಿ-ಸಲು
ಸರಿ-ಪ-ಡಿ-ಸಿ-ಕೊಂ-ಡರು
ಸರಿ-ಪ-ಡಿ-ಸಿ-ಕೊ-ಳ್ಳ-ಬೇಕು
ಸರಿ-ಪ-ಡಿ-ಸಿ-ಕೊ-ಳ್ಳು-ತ್ತೇವೆ
ಸರಿ-ಪ-ಡಿ-ಸಿ-ದರು
ಸರಿ-ಪ-ಡಿ-ಸಿ-ದರೆ
ಸರಿ-ಪ-ಡಿ-ಸು-ವು-ದಷ್ಟೇ
ಸರಿ-ಪ-ಡಿ-ಸು-ವುದು
ಸರಿ-ಪ-ಡಿ-ಸೋಣ
ಸರಿ-ಬು-ದ್ಧಿ-ಮ-ತ್ತೆ-ಯಲ್ಲಿ
ಸರಿ-ಮಾ-ಡಿ-ದರೆ
ಸರಿ-ಯಲ್ಲ
ಸರಿ-ಯ-ಲ್ಲವೇ
ಸರಿ-ಯಾಗಿ
ಸರಿ-ಯಾ-ಗಿ-ನೈ-ಷ್ಠಿಕ
ಸರಿ-ಯಾ-ಗಿ-ರು-ತ್ತದೆ
ಸರಿ-ಯಾ-ಗು-ತ್ತದೆ
ಸರಿ-ಯಾದ
ಸರಿ-ಯಾ-ದದ್ದು
ಸರಿ-ಯಾ-ದೀತೆ
ಸರಿ-ಯಾ-ದು-ದಲ್ಲ
ಸರಿ-ಯಿ-ರ-ಬ-ಹುದು
ಸರಿ-ಯಿ-ಲ್ಲ-ದಿ-ರ-ಬ-ಹುದು
ಸರಿ-ಯುತ್ತ
ಸರಿ-ಯು-ದಾ-ಹ-ರಣೆ
ಸರಿಯೆ
ಸರಿ-ಯೆಂ-ದ-ವರು
ಸರಿ-ಯೆಂದು
ಸರಿ-ಯೆ-ಆ-ದರೆ
ಸರಿ-ಯೆ-ಕೆ-ಲ-ವೊಂದು
ಸರಿ-ಯೆ-ಮಾ-ಡಿ-ಕೊಂಡು
ಸರಿಯೇ
ಸರಿ-ಸ-ಬೇಕು
ಸರಿ-ಸ-ಮ-ನಾಗಿ
ಸರಿ-ಸ-ಮ-ವಾಗಿ
ಸರಿ-ಸಾಟಿ
ಸರಿ-ಸಾ-ಟಿ-ಯಾದ
ಸರಿಸಿ
ಸರಿ-ಹೋ-ಗ-ಬ-ಹು-ದೆಂದು
ಸರಿ-ಹೋ-ಗು-ತ್ತದೆ
ಸರಿ-ಹೋ-ಗು-ತ್ತವೆ
ಸರಿ-ಹೋ-ಗು-ವುದು
ಸರೀ-ಸೃ-ಪ-ಗಳ
ಸರೋ-ವರ
ಸರೋ-ವ-ರದ
ಸರೋ-ವ-ರ-ದಂ-ತಾ-ಗಿತ್ತು
ಸರೋ-ವ-ರ-ದಂತೆ
ಸರೋ-ವ-ರ-ದಲ್ಲಿ
ಸರೋ-ವ-ರ-ವನ್ನು
ಸರೋ-ವ-ರ-ವೊಂ-ದನ್ನು
ಸರ್
ಸರ್ಕಸ್
ಸರ್ಕಾರ
ಸರ್ಕಾ-ರಕ್ಕೆ
ಸರ್ಕಾ-ರದ
ಸರ್ಕಾ-ರವೂ
ಸರ್ಕಾರೀ
ಸರ್ಕಾ-ರ್-ಇ-ವರು
ಸರ್ದಾರ್
ಸರ್ಪ
ಸರ್ಪ-ಗಳಿಂದ
ಸರ್ಪದ
ಸರ್ಫ್ರಾಜ್
ಸರ್ಬಿಯ
ಸರ್ರನೆ
ಸರ್ವ
ಸರ್ವಂ
ಸರ್ವ-ಜನ
ಸರ್ವ-ಜೀ-ವ-ರಲ್ಲೂ
ಸರ್ವ-ಜ್ಞರೇ
ಸರ್ವ-ತೋ-ಮುಖ
ಸರ್ವತ್ರ
ಸರ್ವಥಾ
ಸರ್ವದಾ
ಸರ್ವ-ಧರ್ಮ
ಸರ್ವ-ಧ-ರ್ಮ-ಗಳನ್ನೂ
ಸರ್ವ-ಧ-ರ್ಮ-ಸ-ಮ-ನ್ವ-ಯದ
ಸರ್ವ-ಧ-ರ್ಮ-ಸ-ಹಿ-ಷ್ಣು-ವಾದ
ಸರ್ವ-ಧ-ರ್ಮ-ಸ್ವ-ರೂ-ಪಿಣೇ
ಸರ್ವ-ಧ-ರ್ಮ-ಸ್ವ-ರೂ-ಪಿಯೂ
ಸರ್ವ-ನಾ-ಶದ
ಸರ್ವ-ನಾ-ಶ-ವಾ-ಗು-ತ್ತೇ-ವೆಂದು
ಸರ್ವ-ನಾ-ಶ-ವಾ-ಗು-ವು-ದೆಂದೂ
ಸರ್ವ-ಪ್ರ-ಧಾ-ನ-ರಾ-ದ-ವ-ರೆಂ-ದರೆ
ಸರ್ವರ
ಸರ್ವರೂ
ಸರ್ವ-ವಿ-ದಿತ
ಸರ್ವ-ವ್ಯಾ-ಪ-ಕ-ವಾ-ದ-ಸ-ಕಲ
ಸರ್ವ-ವ್ಯಾ-ಪಿ-ಯಾ-ದುದೋ
ಸರ್ವ-ವ್ಯಾ-ಪಿಯೋ
ಸರ್ವ-ಶಕ್ತ
ಸರ್ವ-ಶ-ಕ್ತ-ರಾ-ಗು-ವಿರಿ
ಸರ್ವ-ಶ-ಕ್ತ-ವಾದ
ಸರ್ವ-ಶ-ಕ್ತ-ವಾ-ದುದು
ಸರ್ವ-ಶ-ಕ್ತಿ-ಸ್ವ-ರೂ-ಪಿಣಿ
ಸರ್ವ-ಶ್ರೇ-ಷ್ಠ-ರ-ಲ್ಲವೆ
ಸರ್ವ-ಶ್ರೇ-ಷ್ಠ-ರಾದ
ಸರ್ವ-ಶ್ರೇ-ಷ್ಠ-ವಾದ
ಸರ್ವ-ಶ್ರೇ-ಷ್ಠ-ವಾ-ದುದು
ಸರ್ವ-ಶ್ರೇ-ಷ್ಠ-ವೆಂದು
ಸರ್ವ-ಸಂಗ
ಸರ್ವ-ಸಂ-ಗ-ಪ-ರಿ-ತ್ಯಾಗ
ಸರ್ವ-ಸ-ಮ-ರ್ಪಕ
ಸರ್ವ-ಸ-ಮ-ರ್ಪಣ
ಸರ್ವ-ಸ-ಮೀ-ಕ್ಷ-ಕ-ನಾದ
ಸರ್ವಸ್ವ
ಸರ್ವ-ಸ್ವ-ವ-ನ್ನ-ರ್ಪಿ-ಸುವ
ಸರ್ವ-ಸ್ವ-ವನ್ನೂ
ಸರ್ವ-ಸ್ವ-ವೆ-ನ್ನುವ
ಸರ್ವಾಂ-ಗ-ಸುಂ-ದ-ರ-ವ-ನ್ನಾಗಿ
ಸರ್ವಾಂ-ಗೀಣ
ಸರ್ವಾಂ-ತ-ರ್ಯಾಮಿ
ಸರ್ವಾಂ-ತ-ರ್ಯಾ-ಮಿ-ಯಾದ
ಸರ್ವಾ-ಧಿ-ಕಾ-ರಿ-ಯಾದ
ಸರ್ವಾ-ನು-ಮ-ತ-ದಿಂದ
ಸರ್ವಾ-ಭ-ರ-ಣ-ಭೂ-ಷಿ-ತೆ-ಯಾ-ಗಿದ್ದ
ಸರ್ವಾ-ಲಂ-ಕೃ-ತ-ವಾದ
ಸರ್ವಿಸ್
ಸಲ
ಸಲ-ಕ-ರ-ಣೆ-ಗಳ
ಸಲ-ಕ-ರ-ಣೆ-ಗಳನ್ನು
ಸಲಕ್ಕೆ
ಸಲಕ್ಕೇ
ಸಲ-ದಂತೆ
ಸಲ-ದಂ-ತೆಯೇ
ಸಲ-ವಂತೂ
ಸಲ-ವಾ-ದರೂ
ಸಲವೂ
ಸಲಹೆ
ಸಲ-ಹೆ-ಗಳನ್ನು
ಸಲ-ಹೆ-ಗಳನ್ನೂ
ಸಲ-ಹೆ-ಗಳು
ಸಲ-ಹೆಗೆ
ಸಲ-ಹೆಯ
ಸಲ-ಹೆ-ಯನ್ನು
ಸಲ-ಹೆ-ಯನ್ನೂ
ಸಲ-ಹೆ-ಯೆಂ-ಬುದು
ಸಲ-ಹೆ-ಯೊಂ-ದಿಗೆ
ಸಲಹೊ
ಸಲಾಮು
ಸಲಿ-ಗೆ-ಯಿಂದ
ಸಲಿ-ಗೆ-ಯಿಂ-ದಲೇ
ಸಲಿ-ಲ-ದಿಂದ
ಸಲು
ಸಲು-ವಾಗಿ
ಸಲು-ವಾ-ಗಿಯೇ
ಸಲ್ಕಿಯಾ
ಸಲ್ಲದು
ಸಲ್ಲ-ಬೇಕು
ಸಲ್ಲಾಪ
ಸಲ್ಲಾ-ಪ-ಗಳ
ಸಲ್ಲಿ-ಸ-ಬೇ-ಕಾ-ದರೆ
ಸಲ್ಲಿ-ಸ-ಲಾ-ಯಿತು
ಸಲ್ಲಿ-ಸಲು
ಸಲ್ಲಿಸಿ
ಸಲ್ಲಿ-ಸಿದ
ಸಲ್ಲಿ-ಸಿ-ದರು
ಸಲ್ಲಿ-ಸಿ-ದ-ವ-ರಿಗೆ
ಸಲ್ಲಿ-ಸಿ-ದ್ದರು
ಸಲ್ಲಿ-ಸಿ-ದ್ದ-ವ-ರೊ-ಬ್ಬರು
ಸಲ್ಲಿ-ಸಿಯೇ
ಸಲ್ಲಿ-ಸುತ್ತ
ಸಲ್ಲಿ-ಸು-ತ್ತಿ-ದ್ದರು
ಸಲ್ಲಿ-ಸು-ತ್ತೇನೆ
ಸಲ್ಲಿ-ಸುವ
ಸಲ್ಲಿ-ಸು-ವು-ದ-ಕ್ಕಾಗಿ
ಸಲ್ಲಿ-ಸು-ವುದು
ಸಲ್ಲಿ-ಸು-ವುದೇ
ಸಲ್ಲು-ತ್ತದೆ
ಸಲ್ಲು-ತ್ತವೆ
ಸಲ್ಲು-ವಂತೆ
ಸಲ್ಲು-ವುದು
ಸವ-ರಿದ
ಸವ-ರಿ-ದರು
ಸವ-ರು-ತ್ತಲೂ
ಸವ-ಲ-ತ್ತು-ಗಳನ್ನು
ಸವಾ-ರನೂ
ಸವಾರಿ
ಸವಾ-ಲೆ-ಸೆ-ಯು-ವಂತೆ
ಸವಾ-ಲೊ-ಡ್ಡಿ-ದರು
ಸವಿದ
ಸವಿದೆ
ಸವಿ-ದೆ-ಯಾ-ದರೆ
ಸವಿ-ಯನ್ನು
ಸವಿ-ಯಾ-ಗಿ-ರಲು
ಸವೆದು
ಸವೆ-ದು-ಹೋ-ದ-ಮೇ-ಲಲ್ಲ
ಸವೆದೆ
ಸವೆ-ಸಲು
ಸವೆಸಿ
ಸವೆ-ಸು-ತ್ತಾನೆ
ಸಶ-ಕ್ತ-ವಾದ
ಸಸಿಯ
ಸಸಿ-ಯನ್ನು
ಸಸ್ಯ
ಸಸ್ಯ-ಗಳ
ಸಸ್ಯದ
ಸಸ್ಯ-ವಿ-ಜ್ಞಾನಿ
ಸಸ್ಯ-ಶಾ-ಸ್ತ್ರ-ಗಳಲ್ಲಿ
ಸಸ್ಯ-ಸಾ-ಮ್ರಾ-ಜ್ಯ-ಗ-ಳಿಗೆ
ಸಹ
ಸಹ-ಕ-ರಿ-ಸಲು
ಸಹ-ಕಾರ
ಸಹ-ಕಾ-ರ-ಗಳು
ಸಹ-ಕಾ-ರದ
ಸಹ-ಕಾ-ರ-ದಿಂದ
ಸಹ-ಕಾ-ರ-ವನ್ನೂ
ಸಹ-ಕಾ-ರ-ವಿ-ಲ್ಲ-ದಿ-ದ್ದರೆ
ಸಹ-ಕಾ-ರ-ವಿ-ಲ್ಲದೆ
ಸಹ-ಕಾ-ರ್ಯ-ಕ-ರ್ತರೂ
ಸಹ-ಚ-ರರು
ಸಹ-ಚ-ರ-ರೆಲ್ಲ
ಸಹಜ
ಸಹ-ಜ-ವ-ಲ್ಲದ
ಸಹ-ಜ-ವಾಗಿ
ಸಹ-ಜ-ವಾ-ಗಿಯೇ
ಸಹ-ಜ-ವಾದ
ಸಹ-ಜ-ವಾ-ದದ್ದೇ
ಸಹ-ಜವೇ
ಸಹ-ಜ-ಸ್ಥಿತಿ
ಸಹ-ಜ-ಸ್ಥಿ-ತಿಗೆ
ಸಹ-ಜ-ಸ್ಥಿ-ತಿ-ಯನ್ನು
ಸಹ-ಜೀ-ವಿ-ಗ-ಳಿ-ಗಾಗಿ
ಸಹ-ನಾ-ಪೂರ್ಣ
ಸಹ-ನಾ-ಮೂರ್ತಿ
ಸಹ-ನೀ-ಯ-ವಾ-ಗ-ಲಿ-ಲ್ಲ-ವೆಂದು
ಸಹ-ನೀ-ಯ-ವಾ-ಗು-ವಂತೆ
ಸಹ-ನೀ-ಯ-ವಾ-ಯಿತು
ಸಹನೆ
ಸಹ-ನೆಯ
ಸಹ-ನೆ-ಯಿಂದ
ಸಹ-ಪಾ-ಠಿ-ಯಾ-ಗಿದ್ದ
ಸಹ-ಪಾ-ಠಿಯೂ
ಸಹ-ಪ್ರ-ಯಾ-ಣಿಕ
ಸಹ-ಪ್ರ-ಯಾ-ಣಿ-ಕರ
ಸಹ-ಪ್ರ-ಯಾ-ಣಿ-ಕ-ರಾ-ಗಿದ್ದು
ಸಹ-ಪ್ರ-ಯಾ-ಣಿ-ಕ-ರಿ-ಗಿಂತ
ಸಹ-ಪ್ರ-ಯಾ-ಣಿ-ಕ-ರಿ-ಗೆಲ್ಲ
ಸಹ-ಭಾ-ಗಿ-ಯಾ-ಗಿ-ದ್ದೇನೆ
ಸಹ-ಮಾ-ನ-ವರ
ಸಹ-ಮಾ-ನ-ವ-ರಲ್ಲಿ
ಸಹ-ಮಾ-ನ-ವ-ರಿಗೆ
ಸಹ-ವ-ರ್ತಿ-ಗ-ಳಿಗೂ
ಸಹ-ವಾಸ
ಸಹ-ವಾ-ಸದ
ಸಹ-ವಾ-ಸವೇ
ಸಹ-ಸಂನ್ಯಾಸಿ
ಸಹ-ಸಂ-ನ್ಯಾ-ಸಿ-ಗಳ
ಸಹ-ಸಂ-ನ್ಯಾ-ಸಿ-ಗಳನ್ನು
ಸಹ-ಸಂ-ನ್ಯಾ-ಸಿ-ಗ-ಳಿಗೆ
ಸಹ-ಸಂ-ನ್ಯಾ-ಸಿ-ಗ-ಳೆ-ಲ್ಲರ
ಸಹಸ್ರ
ಸಹ-ಸ್ರಾರು
ಸಹಾ-ನು-ಭೂತಿ
ಸಹಾ-ನು-ಭೂ-ತಿ-ಗಳನ್ನು
ಸಹಾ-ನು-ಭೂ-ತಿ-ಯನ್ನು
ಸಹಾ-ನು-ಭೂ-ತಿ-ಯಿಂದ
ಸಹಾ-ನು-ಭೂ-ತಿ-ಯಿ-ಲ್ಲವೋ
ಸಹಾ-ನು-ಭೂ-ತಿಯೂ
ಸಹಾಯ
ಸಹಾ-ಯ-ಸ-ಹ-ಕಾ-ರ-ದಿಂದ
ಸಹಾ-ಯ-ಸ-ಹಾನು
ಸಹಾ-ಯಕ
ಸಹಾ-ಯ-ಕ-ಶಿಷ್ಯ
ಸಹಾ-ಯ-ಕ-ರಾಗಿ
ಸಹಾ-ಯ-ಕ-ರಾ-ಗಿದ್ದು
ಸಹಾ-ಯ-ಕ-ವಾ-ಗ-ಬಲ್ಲ
ಸಹಾ-ಯ-ಕ-ವಾ-ದುವು
ಸಹಾ-ಯ-ಕಾರಿ
ಸಹಾ-ಯ-ಕಾ-ರಿ-ಯಾಗಿ
ಸಹಾ-ಯಕ್ಕೆ
ಸಹಾ-ಯ-ಕ್ಕೆಂದು
ಸಹಾ-ಯದ
ಸಹಾ-ಯ-ದಿಂದ
ಸಹಾ-ಯ-ಮಾ-ಡುವ
ಸಹಾ-ಯ-ವಾ-ಗ-ಲೆಂದು
ಸಹಾ-ಯ-ವಾ-ಗಿದೆ
ಸಹಾ-ಯ-ವಾ-ಗು-ತ್ತದೆ
ಸಹಾ-ಯ-ವಾ-ಗು-ವಂ-ತಹ
ಸಹಾ-ಯ-ವಾ-ಗು-ವಂತೆ
ಸಹಾ-ಯ-ವಾ-ಗು-ವು-ದೆಂದೂ
ಸಹಾ-ಯ-ವಿ-ಲ್ಲದೆ
ಸಹಾ-ಯವೂ
ಸಹಾ-ಯವೇ
ಸಹಾ-ಯ-ಹಸ್ತ
ಸಹಾ-ಯ-ಹ-ಸ್ತ-ವನ್ನು
ಸಹಾ-ಯಾ-ರ್ಥ-ವಾಗಿ
ಸಹಿ
ಸಹಿ-ಯಿದ್ದ
ಸಹಿ-ಷ್ಣುತೆ
ಸಹಿ-ಷ್ಣು-ತೆ-ಪ-ರ-ಧರ್ಮ
ಸಹಿ-ಷ್ಣು-ತೆ-ಯೊಂದು
ಸಹಿಸ
ಸಹಿ-ಸ-ಲಾ-ರದ
ಸಹಿ-ಸ-ಲಾ-ರರು
ಸಹಿ-ಸ-ಲಾರೆ
ಸಹಿ-ಸಲು
ಸಹಿಸಿ
ಸಹಿ-ಸಿ-ಕೊಂಡ
ಸಹಿ-ಸಿ-ಕೊಂ-ಡರು
ಸಹಿ-ಸಿ-ಕೊ-ಳ್ಳಲು
ಸಹಿ-ಸಿ-ಕೊ-ಳ್ಳುವ
ಸಹಿ-ಸಿ-ಕೊ-ಳ್ಳು-ವುದನ್ನು
ಸಹಿ-ಸುವ
ಸಹಿ-ಸುವು
ಸಹೃ-ದಯ
ಸಹೃ-ದ-ಯ-ತೆ-ಯನ್ನು
ಸಹೃ-ದ-ಯ-ತೆ-ಯಿಂದ
ಸಹೃ-ದ-ಯರ
ಸಹೋ-ದ-ರ-ತೆ-ಇ-ವೆ-ರಡೂ
ಸಹೋ-ದ-ರ-ತೆಯ
ಸಹೋ-ದ-ರ-ತ್ವದ
ಸಹೋ-ದ-ರರು
ಸಹೋ-ದ-ರಿ-ಯ-ರಲ್ಲಿ
ಸಹ್ಯ-ವಾಗ
ಸಹ್ಯ-ವಾ-ಗ-ದ್ದ-ರಿಂದ
ಸಹ್ಯ-ವಾ-ಗ-ಲಿಲ್ಲ
ಸಾ-ಮ-ರ-ಸ್ಯ
ಸಾಂಕೇ-ತಿ-ಕ-ವಾಗಿ
ಸಾಂಕ್ರಾ-ಮಿಕ
ಸಾಂಖ್ಯ
ಸಾಂಖ್ಯ-ಶಾ-ಸ್ತ್ರದ
ಸಾಂತ್ವನ
ಸಾಂತ್ವ-ನದ
ಸಾಂದ-ರ್ಭಿ-ಕ-ವಾಗಿ
ಸಾಂದ್ರ-ವಾದ
ಸಾಂಪ್ರ-ದಾ-ಯಿಕ
ಸಾಂಸಾ-ರಿಕ
ಸಾಂಸ್ಕೃ-ತಿಕ
ಸಾಕ-ಷ್ಟಿದೆ
ಸಾಕಷ್ಟು
ಸಾಕಾಗಿ
ಸಾಕಾ-ಗು-ವಷ್ಟು
ಸಾಕಾ-ದೀತು
ಸಾಕಾರ
ಸಾಕಾ-ರ-ಗೊ-ಳಿ-ಸಿ-ಕೊ-ಳ್ಳು-ತ್ತದೆ
ಸಾಕಾ-ರ-ಮೂರ್ತಿ
ಸಾಕು
ಸಾಕುಈ
ಸಾಕು-ಸಾ-ಕಾ-ಗಿದೆ
ಸಾಕು-ಸಾ-ಕೆ-ನಿ-ಸು-ವಷ್ಟು
ಸಾಕೆಂದು
ಸಾಕೆ-ನಿ-ಸಿ-ದರೆ
ಸಾಕ್ರೆ-ಟಿಸ್
ಸಾಕ್ಷ-ರ-ರ-ನ್ನಾಗಿ
ಸಾಕ್ಷಾ
ಸಾಕ್ಷಾತ್
ಸಾಕ್ಷಾ-ತ್ಕ-ರಿಸಿ
ಸಾಕ್ಷಾ-ತ್ಕ-ರಿ-ಸಿ-ಕೊಂ-ಡಂ-ತಹ
ಸಾಕ್ಷಾ-ತ್ಕ-ರಿ-ಸಿ-ಕೊಂ-ಡರೋ
ಸಾಕ್ಷಾ-ತ್ಕ-ರಿ-ಸಿ-ಕೊಂ-ಡ-ವರ
ಸಾಕ್ಷಾ-ತ್ಕ-ರಿ-ಸಿ-ಕೊಂ-ಡ-ವಳು
ಸಾಕ್ಷಾ-ತ್ಕ-ರಿ-ಸಿ-ಕೊಂ-ಡಾಗ
ಸಾಕ್ಷಾ-ತ್ಕ-ರಿ-ಸಿ-ಕೊಂ-ಡಿ-ರುವ
ಸಾಕ್ಷಾ-ತ್ಕ-ರಿ-ಸಿ-ಕೊಂಡು
ಸಾಕ್ಷಾ-ತ್ಕ-ರಿ-ಸಿ-ಕೊ-ಳ್ಳ-ಬೇಕು
ಸಾಕ್ಷಾ-ತ್ಕ-ರಿ-ಸಿ-ಕೊ-ಳ್ಳಲು
ಸಾಕ್ಷಾ-ತ್ಕ-ರಿ-ಸಿ-ಕೊ-ಳ್ಳ-ಲೇ-ಬೇಕು
ಸಾಕ್ಷಾ-ತ್ಕ-ರಿ-ಸಿ-ಕೊ-ಳ್ಳುವ
ಸಾಕ್ಷಾ-ತ್ಕ-ರಿ-ಸಿ-ಕೊ-ಳ್ಳು-ವಲ್ಲಿ
ಸಾಕ್ಷಾ-ತ್ಕ-ರಿ-ಸಿ-ಕೊ-ಳ್ಳು-ವುದೇ
ಸಾಕ್ಷಾ-ತ್ಕಾಗಿ
ಸಾಕ್ಷಾ-ತ್ಕಾರ
ಸಾಕ್ಷಾ-ತ್ಕಾ-ರ-ಎ-ಲ್ಲವೂ
ಸಾಕ್ಷಾ-ತ್ಕಾ-ರ-ಕ್ಕಾಗಿ
ಸಾಕ್ಷಾ-ತ್ಕಾ-ರಕ್ಕೆ
ಸಾಕ್ಷಾ-ತ್ಕಾ-ರ-ಗಳ
ಸಾಕ್ಷಾ-ತ್ಕಾ-ರ-ಗಳನ್ನು
ಸಾಕ್ಷಾ-ತ್ಕಾ-ರ-ಗಳೇ
ಸಾಕ್ಷಾ-ತ್ಕಾ-ರದ
ಸಾಕ್ಷಾ-ತ್ಕಾ-ರ-ವನ್ನು
ಸಾಕ್ಷಿ
ಸಾಕ್ಷಿ-ಯಂ-ತಿದ್ದು
ಸಾಕ್ಷಿ-ಯಾಗಿ
ಸಾಕ್ಷ್ಯ
ಸಾಕ್ಷ್ಯ-ದಂ-ತಿ-ದ್ದ-ವರು
ಸಾಕ್ಷ್ಯಾ
ಸಾಗ-ಣೆಯ
ಸಾಗ-ತೊ-ಡ-ಗಿತ್ತು
ಸಾಗ-ಬ-ಹುದು
ಸಾಗ-ಬೇಕು
ಸಾಗರ
ಸಾಗ-ರಕ್ಕೂ
ಸಾಗ-ರ-ಗ-ಳಿವು
ಸಾಗ-ರದ
ಸಾಗ-ರ-ದಂತೆ
ಸಾಗ-ರ-ದೊ-ಳಕ್ಕೆ
ಸಾಗ-ರ-ಭೈ-ರ-ವನು
ಸಾಗ-ರ-ವನ್ನು
ಸಾಗ-ರೋ-ಪ-ಮ-ವಾದ
ಸಾಗಾ-ಣಿ-ಕೆಯ
ಸಾಗಿ
ಸಾಗಿತು
ಸಾಗಿತ್ತು
ಸಾಗಿದ
ಸಾಗಿ-ದಂತೆ
ಸಾಗಿ-ದರು
ಸಾಗಿದೆ
ಸಾಗಿ-ದ್ದರು
ಸಾಗಿ-ಸ-ಲಾ-ಯಿತು
ಸಾಗಿ-ಸು-ತ್ತಿ-ದ್ದರೆ
ಸಾಗಿ-ಸು-ತ್ತಿ-ರುವ
ಸಾಗಿ-ಸುವ
ಸಾಗು
ಸಾಗುತ್ತ
ಸಾಗು-ತ್ತದೆ
ಸಾಗು-ತ್ತಿತ್ತು
ಸಾಗು-ತ್ತಿದೆ
ಸಾಗು-ತ್ತಿ-ದೆ-ಯೆಂದು
ಸಾಗು-ತ್ತಿದ್ದ
ಸಾಗು-ತ್ತಿ-ದ್ದಂತೆ
ಸಾಗು-ತ್ತಿ-ದ್ದರು
ಸಾಗು-ತ್ತಿ-ದ್ದಾಗ
ಸಾಗು-ತ್ತಿ-ದ್ದಾ-ಗಲೇ
ಸಾಗು-ತ್ತಿ-ರುವ
ಸಾಗು-ತ್ತಿ-ರು-ವಾಗ
ಸಾಗು-ತ್ತಿ-ವೆ-ಯೆಂದು
ಸಾಗುವ
ಸಾಗು-ವಾಗ
ಸಾಗು-ವು-ದಿಲ್ಲ
ಸಾತ್ವಿಕ
ಸಾತ್ವಿ-ಕ-ಗೊ-ಳು-ತ್ತದೆ
ಸಾತ್ವಿ-ಕ-ತೆಯ
ಸಾತ್ವಿ-ಕ-ಪ್ರ-ವೃ-ತ್ತಿ-ಯನ್ನು
ಸಾತ್ವಿ-ಕ-ವಾ-ಗಿವೆ
ಸಾತ್ವಿ-ಕವೂ
ಸಾದಾ
ಸಾದ್ಯ
ಸಾಧಕ
ಸಾಧ-ಕರ
ಸಾಧ-ಕ-ರಿಂದ
ಸಾಧ-ಕ-ರಿ-ಗಾಗಿ
ಸಾಧ-ಕರು
ಸಾಧ-ಕ-ವಲ್ಲ
ಸಾಧನ
ಸಾಧ-ನ-ಗಳು
ಸಾಧ-ನ-ವಾ-ಗ-ಬಾ-ರದು
ಸಾಧನಾ
ಸಾಧನೆ
ಸಾಧ-ನೆ-ಧ್ಯಾ-ನಾ-ಭ್ಯಾ-ಸ-ಗಳನ್ನು
ಸಾಧ-ನೆ-ಗಳ
ಸಾಧ-ನೆ-ಗಳನ್ನು
ಸಾಧ-ನೆ-ಗಳನ್ನೆಲ್ಲ
ಸಾಧ-ನೆ-ಗಳಲ್ಲಿ
ಸಾಧ-ನೆ-ಗ-ಳಿಂ-ದಲ್ಲ
ಸಾಧ-ನೆ-ಗ-ಳಿ-ಗಾಗಿ
ಸಾಧ-ನೆ-ಗಳೂ
ಸಾಧ-ನೆಗೆ
ಸಾಧ-ನೆಯ
ಸಾಧ-ನೆ-ಯನ್ನು
ಸಾಧ-ನೆ-ಯಲ್ಲಿ
ಸಾಧ-ನೆ-ಯಾಗಿ
ಸಾಧ-ನೆ-ಯಿಂ-ದಾಗಿ
ಸಾಧ-ನೆಯು
ಸಾಧ-ನೆಯೇ
ಸಾಧ-ನೆ-ಯೊಂದೇ
ಸಾಧಾ-ರಣ
ಸಾಧಾ-ರ-ಣ-ದ್ದಾ-ಗಿ-ರ-ಲಿಲ್ಲ
ಸಾಧಾ-ರ-ಣದ್ದು
ಸಾಧಾ-ರ-ಣ-ರಲ್ಲ
ಸಾಧಾ-ರ-ಣ-ವಾಗಿ
ಸಾಧಿ-ಸದೆ
ಸಾಧಿ-ಸ-ಬಲ್ಲ
ಸಾಧಿ-ಸ-ಬ-ಲ್ಲಿರಿ
ಸಾಧಿ-ಸ-ಬಲ್ಲೆ
ಸಾಧಿ-ಸ-ಬ-ಹುದು
ಸಾಧಿ-ಸ-ಬೇ-ಕಾ-ಗಿದೆ
ಸಾಧಿ-ಸ-ಬೇ-ಕಾ-ದರೆ
ಸಾಧಿ-ಸ-ಬೇ-ಕೆಂ-ದಿ-ದ್ದರೆ
ಸಾಧಿ-ಸ-ಬೇ-ಕೆಂಬ
ಸಾಧಿ-ಸ-ಲಾ-ರವು
ಸಾಧಿ-ಸ-ಲಾ-ರೆವು
ಸಾಧಿ-ಸ-ಲಿ-ದ್ದೇವೆ
ಸಾಧಿ-ಸ-ಲಿ-ರುವ
ಸಾಧಿ-ಸಲು
ಸಾಧಿ-ಸಲೂ
ಸಾಧಿ-ಸ-ಲ್ಪ-ಟ್ಟುವು
ಸಾಧಿ-ಸ-ಲ್ಪ-ಡ-ದಿ-ದ್ದರೂ
ಸಾಧಿ-ಸ-ಹೊ-ರಟ
ಸಾಧಿಸಿ
ಸಾಧಿ-ಸಿ-ಕೊಳ್ಳಿ
ಸಾಧಿ-ಸಿತ್ತು
ಸಾಧಿ-ಸಿದ
ಸಾಧಿ-ಸಿ-ದಂ-ತಾ-ಯಿತು
ಸಾಧಿ-ಸಿ-ದಂತೆ
ಸಾಧಿ-ಸಿ-ದ-ರಲ್ಲ
ಸಾಧಿ-ಸಿ-ದರು
ಸಾಧಿ-ಸಿ-ದುದು
ಸಾಧಿ-ಸಿ-ದ್ದೀರಿ
ಸಾಧಿ-ಸಿದ್ದು
ಸಾಧಿ-ಸು-ತ್ತಾ-ರೆಂ-ಬುದು
ಸಾಧಿ-ಸುವ
ಸಾಧಿ-ಸು-ವಂ-ತಿಲ್ಲ
ಸಾಧಿ-ಸು-ವತ್ತ
ಸಾಧಿ-ಸು-ವ-ವ-ನಲ್ಲ
ಸಾಧಿ-ಸು-ವಿರಿ
ಸಾಧಿ-ಸು-ವು-ದಕ್ಕೇ
ಸಾಧು
ಸಾಧು
ಸಾಧು-ಬ್ರ-ಹ್ಮ-ಚಾ-ರಿ-ಗಳು
ಸಾಧು-ಬ್ರ-ಹ್ಮ-ಚಾ-ರಿ-ಗ-ಳೆಲ್ಲ
ಸಾಧು-ಬ್ರ-ಹ್ಮ-ಚಾ-ರಿ-ಗ-ಳೆ-ಲ್ಲರ
ಸಾಧು-ಸಂ-ನ್ಯಾ-ಸಿ-ಗಳ
ಸಾಧು-ಗಳ
ಸಾಧು-ಗ-ಳಂ-ತೆಯೇ
ಸಾಧು-ಗಳನ್ನು
ಸಾಧು-ಗ-ಳಾಗಿ
ಸಾಧು-ಗಳಿ
ಸಾಧು-ಗಳಿಂದ
ಸಾಧು-ಗ-ಳಿ-ಗಾಗಿ
ಸಾಧು-ಗ-ಳಿಗೂ
ಸಾಧು-ಗ-ಳಿಗೆ
ಸಾಧು-ಗ-ಳಿ-ದ್ದಾರೆ
ಸಾಧು-ಗಳು
ಸಾಧು-ಗ-ಳೆಲ್ಲ
ಸಾಧು-ಗ-ಳೆ-ಲ್ಲರೂ
ಸಾಧು-ಗಳೇ
ಸಾಧು-ಗ-ಳೊಂ-ದಿಗೆ
ಸಾಧು-ಪು-ರು-ಷನೇ
ಸಾಧುವಾ
ಸಾಧು-ವಿಗೂ
ಸಾಧು-ಸಂ-ತರ
ಸಾಧು-ಸಂ-ತ-ರನ್ನು
ಸಾಧು-ಸಂ-ತ-ರಲ್ಲೂ
ಸಾಧು-ಸಂ-ನ್ಯಾ-ಸಿ-ಗಳ
ಸಾಧು-ಸ-ಜ್ಜ-ನರ
ಸಾಧ್ಯ
ಸಾಧ್ಯ-ತೆ-ಗಳನ್ನು
ಸಾಧ್ಯ-ತೆ-ಗಳಿಂದ
ಸಾಧ್ಯ-ತೆ-ಗಳು
ಸಾಧ್ಯ-ತೆಯ
ಸಾಧ್ಯ-ತೆ-ಯನ್ನು
ಸಾಧ್ಯ-ವಾಗ
ಸಾಧ್ಯ-ವಾ-ಗ-ದಿ-ದ್ದರೆ
ಸಾಧ್ಯ-ವಾ-ಗ-ದಿ-ದ್ದಾಗ
ಸಾಧ್ಯ-ವಾ-ಗ-ದಿ-ದ್ದು-ದಕ್ಕೆ
ಸಾಧ್ಯ-ವಾ-ಗ-ದಿ-ರು-ವುದನ್ನು
ಸಾಧ್ಯ-ವಾ-ಗದು
ಸಾಧ್ಯ-ವಾ-ಗದೆ
ಸಾಧ್ಯ-ವಾ-ಗ-ದ್ದನ್ನು
ಸಾಧ್ಯ-ವಾ-ಗ-ದ್ದ-ರಿಂದ
ಸಾಧ್ಯ-ವಾ-ಗ-ಬ-ಹು-ದಾದ
ಸಾಧ್ಯ-ವಾ-ಗ-ಬ-ಹುದು
ಸಾಧ್ಯ-ವಾ-ಗ-ಬ-ಹುದೆ
ಸಾಧ್ಯ-ವಾ-ಗ-ಲಾ-ರ-ದೆಂದು
ಸಾಧ್ಯ-ವಾ-ಗ-ಲಿಲ್ಲ
ಸಾಧ್ಯ-ವಾ-ಗಲೇ
ಸಾಧ್ಯ-ವಾ-ಗಿತ್ತು
ಸಾಧ್ಯ-ವಾ-ಗಿದೆ
ಸಾಧ್ಯ-ವಾ-ಗಿ-ದೆಯೋ
ಸಾಧ್ಯ-ವಾ-ಗಿ-ದ್ದರೆ
ಸಾಧ್ಯ-ವಾ-ಗಿ-ರ-ಲಿಲ್ಲ
ಸಾಧ್ಯ-ವಾ-ಗಿ-ರ-ಲಿ-ಲ್ಲವೊ
ಸಾಧ್ಯ-ವಾ-ಗಿ-ರು-ವುದು
ಸಾಧ್ಯ-ವಾ-ಗಿಲ್ಲ
ಸಾಧ್ಯ-ವಾಗು
ಸಾಧ್ಯ-ವಾ-ಗು-ತ್ತದೆ
ಸಾಧ್ಯ-ವಾ-ಗು-ತ್ತಿ-ರ-ಲಿಲ್ಲ
ಸಾಧ್ಯ-ವಾ-ಗು-ತ್ತಿಲ್ಲ
ಸಾಧ್ಯ-ವಾ-ಗು-ತ್ತಿ-ಲ್ಲ-ವಲ್ಲ
ಸಾಧ್ಯ-ವಾ-ಗುವ
ಸಾಧ್ಯ-ವಾ-ಗು-ವಂತೆ
ಸಾಧ್ಯ-ವಾ-ಗು-ವು-ದ-ಕ್ಕಿಂತ
ಸಾಧ್ಯ-ವಾ-ಗು-ವು-ದಾ-ದರೆ
ಸಾಧ್ಯ-ವಾ-ಗು-ವು-ದಿಲ್ಲ
ಸಾಧ್ಯ-ವಾ-ಗು-ವು-ದೆಂದೇ
ಸಾಧ್ಯ-ವಾ-ಗು-ವುದೋ
ಸಾಧ್ಯ-ವಾದ
ಸಾಧ್ಯ-ವಾ-ದ-ದ್ದ-ನ್ನೆಲ್ಲ
ಸಾಧ್ಯ-ವಾ-ದದ್ದು
ಸಾಧ್ಯ-ವಾ-ದರೆ
ಸಾಧ್ಯ-ವಾ-ದ-ಷ್ಟನ್ನು
ಸಾಧ್ಯ-ವಾ-ದಷ್ಟು
ಸಾಧ್ಯ-ವಾ-ದಷ್ಟೂ
ಸಾಧ್ಯ-ವಾ-ದಾಗ
ಸಾಧ್ಯ-ವಾ-ದಾ-ಗ-ಲೆಲ್ಲ
ಸಾಧ್ಯ-ವಾ-ದೀತೆ
ಸಾಧ್ಯ-ವಾ-ದೀ-ತೆಂ-ಬುದು
ಸಾಧ್ಯ-ವಾ-ದೀತೇ
ಸಾಧ್ಯ-ವಾ-ದುದು
ಸಾಧ್ಯ-ವಾ-ಯಿತು
ಸಾಧ್ಯ-ವಾ-ಯಿತೋ
ಸಾಧ್ಯ-ವಿತ್ತು
ಸಾಧ್ಯ-ವಿದೆ
ಸಾಧ್ಯ-ವಿ-ದ್ದಲ್ಲೆಲ್ಲ
ಸಾಧ್ಯ-ವಿ-ದ್ದಿ-ದ್ದರೆ
ಸಾಧ್ಯ-ವಿ-ರ-ಲಿಲ್ಲ
ಸಾಧ್ಯ-ವಿ-ರು-ವಂ-ತಹ
ಸಾಧ್ಯ-ವಿ-ರು-ವು-ದ-ನ್ನೆಲ್ಲ
ಸಾಧ್ಯ-ವಿಲ್ಲ
ಸಾಧ್ಯ-ವಿ-ಲ್ಲ-ಇದು
ಸಾಧ್ಯ-ವಿ-ಲ್ಲ-ದಂ-ತಹ
ಸಾಧ್ಯ-ವಿ-ಲ್ಲ-ದಿದ್ದ
ಸಾಧ್ಯ-ವಿ-ಲ್ಲ-ದಿ-ರು-ವಾಗ
ಸಾಧ್ಯ-ವಿ-ಲ್ಲ-ದು-ದ-ರಿಂದ
ಸಾಧ್ಯ-ವಿ-ಲ್ಲ-ದ್ದ-ಕ್ಕಾಗಿ
ಸಾಧ್ಯ-ವಿ-ಲ್ಲ-ದ್ದ-ರಿಂದ
ಸಾಧ್ಯ-ವಿ-ಲ್ಲ-ವಾ-ದರೂ
ಸಾಧ್ಯ-ವಿ-ಲ್ಲವೆ
ಸಾಧ್ಯ-ವಿ-ಲ್ಲ-ವೆಂದು
ಸಾಧ್ಯ-ವಿ-ಲ್ಲ-ವೆಂ-ಬುದು
ಸಾಧ್ಯವೆ
ಸಾಧ್ಯ-ವೆಂದು
ಸಾಧ್ಯವೇ
ಸಾಧ್ಯವೋ
ಸಾನ್ನಿ
ಸಾನ್ನಿ-ಧ್ಯದ
ಸಾನ್ನಿ-ಧ್ಯ-ದಲ್ಲಿ
ಸಾನ್ನಿ-ಧ್ಯ-ದಿಂದ
ಸಾನ್ನಿ-ಧ್ಯ-ವಿ-ರು-ವಂತೆ
ಸಾನ್ನಿ-ಧ್ಯವು
ಸಾನ್ನಿ-ಧ್ಯವೂ
ಸಾನ್ನಿ-ಧ್ಯ-ವೆಂದರೆ
ಸಾನ್ನಿ-ಧ್ಯವೇ
ಸಾಬೀ-ತು-ಗೊ-ಳಿ-ಸಿ-ದ್ದೇನೆ
ಸಾಬೀ-ತು-ಮಾಡಿ
ಸಾಮಗ್ರಿ
ಸಾಮ-ಗ್ರಿ-ಗಳನ್ನು
ಸಾಮ-ಗ್ರಿ-ಗಳಿಂದ
ಸಾಮ-ಗ್ರಿ-ಗ-ಳು-ಕಾ-ರ್ಯ-ರೂ-ಪಕ್ಕೆ
ಸಾಮ-ರ-ಸ್ಯ-ದಿಂದ
ಸಾಮ-ರ-ಸ್ಯ-ದಿಂ-ದಿ-ರ-ಲಿಲ್ಲ
ಸಾಮ-ರ-ಸ್ಯ-ದಿಂ-ದಿ-ರು-ತ್ತವೆ
ಸಾಮ-ರ-ಸ್ಯ-ವನ್ನೂ
ಸಾಮ-ರ-ಸ್ಯ-ವಿ-ರು-ತ್ತದೆ
ಸಾಮರ್ಥ್ಯ
ಸಾಮ-ರ್ಥ್ಯ-ಗಳ
ಸಾಮ-ರ್ಥ್ಯ-ಗ-ಳಿಗೆ
ಸಾಮ-ರ್ಥ್ಯ-ಗ-ಳಿ-ದ್ದಾ-ಗಲೇ
ಸಾಮ-ರ್ಥ್ಯ-ಗಳು
ಸಾಮ-ರ್ಥ್ಯದ
ಸಾಮ-ರ್ಥ್ಯ-ದಿಂದ
ಸಾಮ-ರ್ಥ್ಯ-ವನ್ನು
ಸಾಮ-ರ್ಥ್ಯ-ವಿದ್ದೂ
ಸಾಮ-ರ್ಥ್ಯ-ವಿ-ರು-ವುದನ್ನು
ಸಾಮ-ರ್ಥ್ಯ-ವಿಲ್ಲ
ಸಾಮಾ
ಸಾಮಾ-ಜಿಕ
ಸಾಮಾ-ನ-ವಾದ
ಸಾಮಾ-ನು-ಸ-ಲ-ಕ-ರ-ಣೆ-ಗಳನ್ನು
ಸಾಮಾ-ನು-ಗಳನ್ನೆಲ್ಲ
ಸಾಮಾ-ನು-ಗಳೇ
ಸಾಮಾನ್ಯ
ಸಾಮಾ-ನ್ಯ-ಜ-ನ-ಗ-ಳಿ-ಗಾಗಿ
ಸಾಮಾ-ನ್ಯ-ಮ-ಟ್ಟ-ವನ್ನು
ಸಾಮಾ-ನ್ಯರ
ಸಾಮಾ-ನ್ಯ-ರಲ್ಲೇ
ಸಾಮಾ-ನ್ಯ-ರಾದ
ಸಾಮಾ-ನ್ಯ-ರಿ-ಗಾಗಿ
ಸಾಮಾ-ನ್ಯ-ರಿಗೆ
ಸಾಮಾ-ನ್ಯರು
ಸಾಮಾ-ನ್ಯ-ರು-ಎ-ಲ್ಲರೂ
ಸಾಮಾ-ನ್ಯಳೆ
ಸಾಮಾ-ನ್ಯ-ವಾಗಿ
ಸಾಮಾ-ನ್ಯ-ವಾದ
ಸಾಮಾ-ನ್ಯ-ವಾ-ದವು
ಸಾಮಾ-ನ್ಯ-ವೆ-ನ್ನಿಸ
ಸಾಮೀಪ್ಯ
ಸಾಮೀ-ಪ್ಯದ
ಸಾಮೂ-ಹಿಕ
ಸಾಮೂ-ಹಿ-ಕ-ವಾಗಿ
ಸಾಮ್ಯ-ಗಳನ್ನು
ಸಾಮ್ಯ-ವನ್ನು
ಸಾಮ್ರಾ-ಜ್ಯ-ಕ್ಕಲ್ಲ
ಸಾಮ್ರಾ-ಜ್ಯ-ದಲ್ಲಿ
ಸಾಯಂ
ಸಾಯ-ಣಾ-ಚಾ-ರ್ಯರು
ಸಾಯ-ಣಾ-ಚಾ-ರ್ಯರೇ
ಸಾಯ-ಬೇಕು
ಸಾಯ-ಬೇ-ಕೆಂದು
ಸಾಯಲಿ
ಸಾಯ-ಲೇ-ಬೇ-ಕಾಗಿ
ಸಾಯ-ಲೇ-ಬೇಕು
ಸಾಯ-ಲೇಳು
ಸಾಯಿರಿ
ಸಾಯು
ಸಾಯುತ್ತ
ಸಾಯು-ತ್ತಿ-ರು-ತ್ತಾರೆ
ಸಾಯು-ತ್ತಿ-ರು-ವಾಗ
ಸಾಯುವ
ಸಾಯು-ವಂ-ತಾ-ಗಲಿ
ಸಾಯು-ವಿರಿ
ಸಾಯು-ವು-ದ-ಕ್ಕಿಂತ
ಸಾಯು-ವು-ದಿಲ್ಲ
ಸಾಯು-ವುದು
ಸಾಯು-ವುದೇ
ಸಾರ
ಸಾರ-ಭೂತ
ಸಾರ-ವನ್ನು
ಸಾರ-ವನ್ನೇ
ಸಾರ-ವಾಗಿ
ಸಾರ-ವಾದ
ಸಾರ-ವೆಂದರೆ
ಸಾರ-ವೆಲ್ಲ
ಸಾರವೇ
ಸಾರ-ಸ-ರ್ವಸ್ವ
ಸಾರ-ಸ-ರ್ವ-ಸ್ವ-ವಲ್ಲ
ಸಾರ-ಸ-ರ್ವ-ಸ್ವವೇ
ಸಾರಾ
ಸಾರಾಂಶ
ಸಾರಾ-ಬುಲ್
ಸಾರಾಳ
ಸಾರಾ-ಳನ್ನು
ಸಾರಾ-ಳಿಂದ
ಸಾರಾ-ಳಿಗೆ
ಸಾರಿ
ಸಾರಿದ
ಸಾರಿ-ದ-ರಾ-ದರೂ
ಸಾರಿ-ದರು
ಸಾರಿ-ದಾ-ಗ-ಲೆಲ್ಲ
ಸಾರಿದೆ
ಸಾರಿ-ದ್ದರು
ಸಾರಿ-ದ್ದವ
ಸಾರಿ-ಹೇ-ಳು-ತ್ತಿತ್ತು
ಸಾರು
ಸಾರುತ್ತ
ಸಾರು-ತ್ತದೆ
ಸಾರು-ತ್ತವೆ
ಸಾರು-ತ್ತಾರೆ
ಸಾರು-ತ್ತಿ-ದ್ದರು
ಸಾರು-ತ್ತಿವೆ
ಸಾರು-ತ್ತೇನೆ
ಸಾರುವ
ಸಾರು-ವು-ದ-ರಿಂ-ದಲೇ
ಸಾರು-ವು-ದಿಲ್ಲ
ಸಾರು-ವುದು
ಸಾರು-ವು-ದೇ-ನೆಂ-ದರೆ
ಸಾರೋ-ಟನ್ನು
ಸಾರೋ-ಟ-ನ್ನೇರಿ
ಸಾರೋಟಿ
ಸಾರೋ-ಟಿಗೆ
ಸಾರೋ-ಟಿನ
ಸಾರೋ-ಟಿ-ನಲ್ಲಿ
ಸಾರೋ-ಟಿ-ನಿಂದ
ಸಾರೋ-ಟಿ-ನಿಂ-ದಿ-ಳಿ-ಯು-ತ್ತಿ-ದ್ದಂತೆ
ಸಾರೋಟು
ಸಾರೋ-ಟು-ಗಳ
ಸಾರೋ-ಟು-ಗಳು
ಸಾರ್ಥ-ಕ-ದಿ-ನ-ಗಳು
ಸಾರ್ಥ-ಕ-ವಾ-ಗಿದೆ
ಸಾರ್ಥ-ಕ-ವಾ-ಗು-ತ್ತದೆ
ಸಾರ್ಥ-ಕ-ವಾ-ಯಿತು
ಸಾರ್ಥ-ಕ-ವಾ-ಯಿ-ತೆಂದು
ಸಾರ್ಥಕ್ಯ
ಸಾರ್ವ
ಸಾರ್ವ-ಕಾ-ಲಿಕ
ಸಾರ್ವ-ಜ-ನಿಕ
ಸಾರ್ವ-ಜ-ನಿ-ಕರ
ಸಾರ್ವ-ಜ-ನಿ-ಕ-ರಿಗೆ
ಸಾರ್ವ-ಜ-ನಿ-ಕರು
ಸಾರ್ವ-ಜ-ನಿ-ಕ-ವಾಗಿ
ಸಾರ್ವ-ತ್ರಿಕ
ಸಾರ್ವ-ತ್ರಿ-ಕ-ವಾಗಿ
ಸಾರ್ವ-ತ್ರಿ-ಕವೂ
ಸಾರ್ವ-ಭೌ-ಮ-ನಿಗೆ
ಸಾಲ
ಸಾಲಂ-ಕೃತ
ಸಾಲ-ಗಾರ
ಸಾಲ-ದಾ-ಗಿತ್ತು
ಸಾಲದು
ಸಾಲದೆ
ಸಾಲ-ದೆಂ-ಬಂತೆ
ಸಾಲ-ದ್ದಕ್ಕೆ
ಸಾಲ-ಲಾ-ರವು
ಸಾಲ-ಲಿಲ್ಲ
ಸಾಲಿಗೆ
ಸಾಲಿ-ಗ್ರಾ-ಮದ
ಸಾಲು
ಸಾಲು-ಗಳನ್ನು
ಸಾವ-ಕಾ-ಶ-ವಾಗಿ
ಸಾವ-ಧಾ-ನ-ದಿಂದ
ಸಾವ-ಧಾ-ನ-ವಾಗಿ
ಸಾವನ್ನು
ಸಾವ-ರಿ-ಸಿ-ಕೊಂಡು
ಸಾವಿ-ಗೀ-ಡಾದ
ಸಾವಿಗೆ
ಸಾವಿತ್ರಿ
ಸಾವಿನ
ಸಾವಿರ
ಸಾವಿ-ರಕ್ಕೂ
ಸಾವಿ-ರ-ಗ-ಟ್ಟಲೆ
ಸಾವಿ-ರಾರು
ಸಾವಿಲ್ಲ
ಸಾವು
ಸಾವು-ಬ-ದು-ಕಿನ
ಸಾವೆಂದರೆ
ಸಾವೇ
ಸಾಷ್ಟಾಂಗ
ಸಾಸಿ-ರ-ಮಡಿ
ಸಾಸಿ-ರ-ಮ-ಡಿ-ಯಾ-ಗಿತ್ತು
ಸಾಸಿವೆ
ಸಾಹ
ಸಾಹನೂ
ಸಾಹ-ರನ್ನೇ
ಸಾಹ-ರಿಗೂ
ಸಾಹರು
ಸಾಹಸ
ಸಾಹ-ಸ-ಕಾರ್ಯ
ಸಾಹ-ಸ-ಕಾ-ರ್ಯ-ದಲ್ಲಿ
ಸಾಹ-ಸ-ಪ್ರ-ವೃ-ತ್ತಿ-ಯನ್ನು
ಸಾಹ-ಸ-ಮಯ
ಸಾಹ-ಸ-ವನ್ನು
ಸಾಹ-ಸಿ-ಗಳ
ಸಾಹಿ-ತಿ-ಯಾ-ಗ-ಬ-ಹು-ದಾ-ಗಿತ್ತು
ಸಾಹಿತ್ಯ
ಸಾಹಿ-ತ್ಯಕ್ಕೆ
ಸಾಹಿ-ತ್ಯದ
ಸಾಹಿ-ತ್ಯ-ದಲ್ಲೂ
ಸಾಹಿ-ತ್ಯ-ದಲ್ಲೇ
ಸಾಹಿ-ತ್ಯವು
ಸಾಹೇಬ
ಸಾಹೇ-ಬ-ನಿಂದ
ಸಾಹೇ-ಬರು
ಸಿ
ಸಿಂಗನ
ಸಿಂಗ-ನನ್ನು
ಸಿಂಗ-ರಿ-ಸ-ಲಾ-ಗಿತ್ತು
ಸಿಂಗ-ರಿಸಿ
ಸಿಂಗ-ರಿ-ಸಿದ್ದ
ಸಿಂಗರು
ಸಿಂಗಾರ
ಸಿಂಗಾ-ರ-ವೇಲು
ಸಿಂಗ್ನನ್ನು
ಸಿಂಡ-ರಿ-ಸಿ-ಕೊಂಡು
ಸಿಂಧೂ
ಸಿಂಧೂ-ವಿ-ನಿಂದ
ಸಿಂಧ್
ಸಿಂಪ-ಡಿ-ಸ-ಲಾ-ಯಿತು
ಸಿಂಹ
ಸಿಂಹದ
ಸಿಂಹ-ದಂತೆ
ಸಿಂಹ-ದಷ್ಟು
ಸಿಂಹ-ಪಾ-ಲನ್ನು
ಸಿಂಹ-ಪು-ರು-ಷ-ಸಿಂಹ
ಸಿಂಹ-ಳದ
ಸಿಂಹ-ಳ-ದಲ್ಲಿ
ಸಿಂಹಳೀ
ಸಿಂಹ-ವನ್ನು
ಸಿಂಹ-ವಾ-ಗಲು
ಸಿಂಹ-ಸ-ದೃಶ
ಸಿಂಹಾ
ಸಿಂಹಾ-ಸ-ನದ
ಸಿಂಹಾ-ಸ-ನವು
ಸಿಕ್ಕಂ-ತಾ-ಗಿತ್ತು
ಸಿಕ್ಕಂ-ತಾ-ಗು-ತ್ತಿತ್ತು
ಸಿಕ್ಕರೆ
ಸಿಕ್ಕಾಗ
ಸಿಕ್ಕಾ-ಗ-ಲೆಲ್ಲ
ಸಿಕ್ಕಾರು
ಸಿಕ್ಕಿ
ಸಿಕ್ಕಿ-ಕೊಂ-ಡರು
ಸಿಕ್ಕಿ-ಕೊಂಡಿ
ಸಿಕ್ಕಿ-ಕೊ-ಳ್ಳದೆ
ಸಿಕ್ಕಿತು
ಸಿಕ್ಕಿ-ತೆನ್ನು
ಸಿಕ್ಕಿತೋ
ಸಿಕ್ಕಿತ್ತು
ಸಿಕ್ಕಿದ
ಸಿಕ್ಕಿ-ದಂ-ತಾ-ದ್ದ-ರಿಂದ
ಸಿಕ್ಕಿ-ದರೆ
ಸಿಕ್ಕಿದೆ
ಸಿಕ್ಕಿ-ದೊ-ಡ-ನೆಯೇ
ಸಿಕ್ಕಿ-ದ್ದ-ರಿಂದ
ಸಿಕ್ಕಿ-ದ್ದರೆ
ಸಿಕ್ಕಿದ್ದು
ಸಿಕ್ಕಿ-ಬಿ-ಟ್ಟರೆ
ಸಿಕ್ಕಿಯೇ
ಸಿಕ್ಕಿ-ರ-ಲಿಲ್ಲ
ಸಿಕ್ಕಿಲ್ಲ
ಸಿಕ್ಕಿಸಿ
ಸಿಕ್ಕಿ-ಹಾ-ಕಿ-ಕೊಂಡ
ಸಿಕ್ಕಿ-ಹಾ-ಕಿ-ಕೊಂ-ಡಿ-ದ್ದರು
ಸಿಕ್ಕಿ-ಹಾ-ಕಿ-ಕೊಂಡು
ಸಿಕ್ಕಿ-ಹಾ-ಕಿ-ಕೊ-ಳ್ಳು-ವುದೇ
ಸಿಕ್ಕೀತು
ಸಿಕ್ಕೇ
ಸಿಕ್ಖರ
ಸಿಕ್ಖ-ರಾ-ಗಲಿ
ಸಿಕ್ಖ್
ಸಿಗ-ದಿ-ದ್ದರೆ
ಸಿಗದೆ
ಸಿಗ-ಬ-ಹು-ದಾದ
ಸಿಗ-ಬ-ಹು-ದೆಂದು
ಸಿಗ-ಬೇಕು
ಸಿಗ-ರೇಟು
ಸಿಗ-ಲಾ-ರ-ದಂ-ಥವು
ಸಿಗ-ಲಾ-ರದು
ಸಿಗ-ಲಾ-ರ-ದೆಂ-ಬು-ದನ್ನು
ಸಿಗ-ಲಾ-ರವು
ಸಿಗಲಿ
ಸಿಗ-ಲಿಲ್ಲ
ಸಿಗಲೇ
ಸಿಗು-ತ್ತದೆ
ಸಿಗು-ತ್ತ-ದೆಂಬ
ಸಿಗು-ತ್ತದೊ
ಸಿಗು-ತ್ತಿತ್ತು
ಸಿಗು-ತ್ತಿ-ದ್ದರೂ
ಸಿಗು-ತ್ತಿ-ದ್ದುದು
ಸಿಗು-ತ್ತಿರ
ಸಿಗು-ತ್ತಿಲ್ಲ
ಸಿಗುವ
ಸಿಗು-ವಂತಾ
ಸಿಗು-ವಂ-ತಾ-ಗು-ವು-ದಲ್ಲ
ಸಿಗು-ವಂ-ತಾ-ದುವು
ಸಿಗು-ವಂ-ತಾ-ಯಿತು
ಸಿಗು-ವುದೇ
ಸಿಟ್ಟನ್ನೂ
ಸಿಟ್ಟಾ-ಗಿ-ದ್ದರು
ಸಿಟ್ಟಿ-ಗೆ-ದ್ದದ್ದು
ಸಿಟ್ಟಿ-ಗೆ-ದ್ದರು
ಸಿಟ್ಟಿ-ಗೆದ್ದು
ಸಿಟ್ಟಿ-ಗೆ-ಬ್ಬಿ-ಸಿ-ಯಾ-ದರೂ
ಸಿಟ್ಟಿ-ಗೇ-ಳದೆ
ಸಿಟ್ಟಿ-ನಿಂದ
ಸಿಟ್ಟಿನ್ನೂ
ಸಿಟ್ಟು
ಸಿಡ-ಲಾ-ಗಿತ್ತು
ಸಿಡಿ-ದರೆ
ಸಿಡಿದು
ಸಿಡಿ-ದೆ-ರ-ಗು-ವ-ವ-ರೆಗೂ
ಸಿಡಿ-ಮ-ದ್ದನ್ನು
ಸಿಡಿ-ಮ-ದ್ದಾ-ದರೆ
ಸಿಡಿ-ಮಿ-ಡಿ-ಗೊಂ-ಡರು
ಸಿಡಿ-ಯ-ಬ-ಹು-ದಾ-ಗಿತ್ತು
ಸಿಡಿ-ಯು-ತ್ತೇನೆ
ಸಿಡಿಲ
ಸಿಡಿ-ಲ-ನ್ನಾ-ಗಿ-ಸು-ತ್ತಿತ್ತು
ಸಿಡಿ-ಲಿ-ನಂತೆ
ಸಿಡಿ-ಲಿ-ನಷ್ಟು
ಸಿಡಿಲು
ಸಿತು
ಸಿತೋ
ಸಿದ
ಸಿದರು
ಸಿದರೂ
ಸಿದ-ರೇನು
ಸಿದುವು
ಸಿದೆ
ಸಿದ್ದ-ಮಾಡಿ
ಸಿದ್ಧ
ಸಿದ್ಧ-ಗೊ-ಳಿಸ
ಸಿದ್ಧ-ಗೊ-ಳಿ-ಸ-ಲಾ-ಯಿತು
ಸಿದ್ಧ-ಗೊ-ಳಿಸು
ಸಿದ್ಧ-ಗೊ-ಳಿ-ಸು-ವು-ದ-ಕ್ಕಾ-ಗಿಯೇ
ಸಿದ್ಧತೆ
ಸಿದ್ಧ-ತೆ-ಗಳನ್ನು
ಸಿದ್ಧ-ತೆ-ಗಳನ್ನೂ
ಸಿದ್ಧ-ತೆ-ಗಳನ್ನೆಲ್ಲ
ಸಿದ್ಧ-ತೆ-ಗಳು
ಸಿದ್ಧ-ತೆ-ಗ-ಳೆಲ್ಲ
ಸಿದ್ಧ-ತೆ-ಯಲ್ಲಿ
ಸಿದ್ಧ-ತೆ-ಯಾ-ಗಿತ್ತು
ಸಿದ್ಧ-ನಾಗಿ
ಸಿದ್ಧ-ನಾ-ಗಿ-ದ್ದೇನೆ
ಸಿದ್ಧ-ನಾಗು
ಸಿದ್ಧ-ನಾ-ಗು-ತ್ತಿ-ದ್ದೇನೆ
ಸಿದ್ಧ-ನಾದ
ಸಿದ್ಧ-ನಾ-ದಾಗ
ಸಿದ್ಧ-ನಿದ್ದ
ಸಿದ್ಧ-ನಿ-ದ್ದಾ-ನೆಂದು
ಸಿದ್ಧ-ನಿ-ದ್ದೇನೆ
ಸಿದ್ಧ-ನಿ-ರು-ವೆಯಾ
ಸಿದ್ಧ-ನಿ-ಲ್ಲವೋ
ಸಿದ್ಧ-ನೆಂ-ಬು-ದನ್ನು
ಸಿದ್ಧ-ಪ-ಡಿ-ಸ-ಬೇ-ಕಾ-ಗಿದೆ
ಸಿದ್ಧ-ಪ-ಡಿ-ಸ-ಲಾ-ಗಿದ್ದ
ಸಿದ್ಧ-ಪ-ಡಿಸಿ
ಸಿದ್ಧ-ಪ-ಡಿ-ಸಿ-ದರು
ಸಿದ್ಧ-ಪ-ಡಿ-ಸಿದ್ದ
ಸಿದ್ಧ-ಪ-ಡಿ-ಸು-ವು-ದ-ರಲ್ಲೂ
ಸಿದ್ಧ-ರಾಗಿ
ಸಿದ್ಧ-ರಾ-ಗಿ-ದ್ದರು
ಸಿದ್ಧ-ರಾ-ಗಿ-ದ್ದ-ವರು
ಸಿದ್ಧ-ರಾ-ಗಿ-ದ್ದಾರೆ
ಸಿದ್ಧ-ರಾ-ಗಿ-ದ್ದೇವೆ
ಸಿದ್ಧ-ರಾ-ಗಿಯೇ
ಸಿದ್ಧ-ರಾ-ಗಿ-ರ-ಬೇಕು
ಸಿದ್ಧ-ರಾ-ಗಿರಿ
ಸಿದ್ಧ-ರಾ-ಗಿ-ರು-ವ-ವರು
ಸಿದ್ಧ-ರಾ-ಗಿ-ರು-ವಿ-ರೇನು
ಸಿದ್ಧ-ರಾ-ಗು-ತ್ತಿದ್ದ
ಸಿದ್ಧ-ರಾ-ಗು-ತ್ತಿ-ದ್ದರು
ಸಿದ್ಧ-ರಾ-ದರು
ಸಿದ್ಧ-ರಿ-ದ್ದಾರೆ
ಸಿದ್ಧ-ರಿ-ದ್ದೇವೆ
ಸಿದ್ಧ-ರಿ-ರ-ಬೇಕು
ಸಿದ್ಧ-ರಿ-ರ-ಲಿಲ್ಲ
ಸಿದ್ಧ-ರಿ-ರುವ
ಸಿದ್ಧ-ರಿ-ರು-ವಂ-ತಹ
ಸಿದ್ಧ-ರಿ-ರು-ವು-ದಿಲ್ಲ
ಸಿದ್ಧ-ಳಾ-ಗ-ಬೇಕು
ಸಿದ್ಧ-ಳಾ-ಗಿ-ದ್ದೇ-ನೆಂದು
ಸಿದ್ಧ-ಳಾಗು
ಸಿದ್ಧ-ಳಾ-ಗು-ತ್ತಿ-ದ್ದಳು
ಸಿದ್ಧ-ಳಾ-ದಳು
ಸಿದ್ಧ-ಳಿ-ರ-ಲಿಲ್ಲ
ಸಿದ್ಧ-ವಾ-ಗ-ಲಿದೆ
ಸಿದ್ಧ-ವಾಗಿ
ಸಿದ್ಧ-ವಾ-ಗಿತ್ತು
ಸಿದ್ಧ-ವಾ-ಗಿದ್ದ
ಸಿದ್ಧ-ವಾ-ಗಿ-ದ್ದುವು
ಸಿದ್ಧ-ವಾ-ಗಿ-ರಿ-ಸಿದ್ದ
ಸಿದ್ಧ-ವಾ-ಗಿ-ರು-ವು-ದಿಲ್ಲ
ಸಿದ್ಧ-ವಾ-ದುವು
ಸಿದ್ಧ-ವಾ-ಯಿತು
ಸಿದ್ಧಾಂತ
ಸಿದ್ಧಾಂ-ತ-ಗಳ
ಸಿದ್ಧಾಂ-ತ-ಗಳನ್ನು
ಸಿದ್ಧಾಂ-ತ-ಗಳಿಂದ
ಸಿದ್ಧಾಂ-ತ-ಗ-ಳಿಗೆ
ಸಿದ್ಧಾಂ-ತ-ಗಳು
ಸಿದ್ಧಾಂ-ತದ
ಸಿದ್ಧಾಂ-ತ-ದ-ಲ್ಲಾ-ಗಲಿ
ಸಿದ್ಧಾಂ-ತ-ದಲ್ಲಿ
ಸಿದ್ಧಾಂ-ತ-ದಿಂದ
ಸಿದ್ಧಾಂ-ತ-ವನ್ನು
ಸಿದ್ಧಾಂ-ತ-ವನ್ನೂ
ಸಿದ್ಧಾಂ-ತ-ವಾ-ದಿ-ಗಳ
ಸಿದ್ಧಿಯ
ಸಿದ್ಧಿ-ಯನ್ನು
ಸಿದ್ಧಿ-ಸಿ-ಕೊಂಡ
ಸಿದ್ಧಿ-ಸಿ-ಕೊಂ-ಡರೂ
ಸಿದ್ಧಿ-ಸಿ-ಕೊಂಡು
ಸಿದ್ಧಿ-ಸಿ-ಕೊ-ಳ್ಳಲು
ಸಿದ್ಧಿ-ಸು-ತ್ತದೆ
ಸಿನ್ಹ
ಸಿಪಾಯಿ
ಸಿಮ್ಲಾ
ಸಿಯಾ
ಸಿಯಾಲ್
ಸಿಯಾ-ಲ್ಕೋ-ಟಿಗೆ
ಸಿಯಾ-ಲ್ಕೋ-ಟಿನ
ಸಿಯಾ-ಲ್ಕೋ-ಟಿ-ನಲ್ಲಿ
ಸಿಯಾಲ್ದಾ
ಸಿಯೇ
ಸಿಲು-ಕದು
ಸಿಲುಕಿ
ಸಿಲು-ಕಿ-ಕೊಂ-ಡ-ವರ
ಸಿಲು-ಕಿ-ಕೊಂ-ಡಿ-ದ್ದೇನೆ
ಸಿಲು-ಕಿ-ಕೊ-ಳ್ಳ-ಲಿ-ರು-ವುದನ್ನು
ಸಿಲು-ಕಿ-ಕೊ-ಳ್ಳುವ
ಸಿಲು-ಕಿ-ದ-ವ-ರನ್ನು
ಸಿಲು-ಕಿ-ದ್ದರು
ಸಿಲು-ಕಿ-ದ್ದಾರೋ
ಸಿಲು-ಕಿ-ಸಿವೆ
ಸಿಲೋ-ನನ್ನೂ
ಸಿಲೋ-ನಿನ
ಸಿಲೋ-ನಿ-ನಲ್ಲಿ
ಸಿಲೋ-ನಿ-ನ-ವರು
ಸಿಲೋನ್
ಸಿವಿಲ್
ಸಿಸಿಲಿ
ಸಿಸಿ-ಲಿಯು
ಸಿಹಿ
ಸಿಹಿ-ಖಾ-ರದ
ಸಿಹಿ-ತಿಂ-ಡಿ-ಗಳು
ಸಿಹಿ-ತಿ-ನಿ-ಸು-ಇ-ವು-ಗಳಲ್ಲಿ
ಸಿಹಿ-ಮೂತ್ರ
ಸಿಹಿ-ಮೊ-ಸರು
ಸೀತಾ
ಸೀತಾ-ದೇವಿ
ಸೀತಾ-ದೇ-ವಿಯ
ಸೀತೆ
ಸೀದಾ
ಸೀನು-ತ್ತವೆ
ಸೀಮಿ-ತ-ಗೊ-ಳಿ-ಸಿ-ದರು
ಸೀಮೆ-ಗಳ
ಸೀಮೆಯ
ಸೀಮೆ-ಯಲ್ಲಿ
ಸೀಲರ
ಸೀಲ್
ಸೀಳಿ-ಕೊಂಡು
ಸೀಸರ್ನ
ಸುಂದರ
ಸುಂದ-ರ-ಪ್ರ-ಶಾಂತ
ಸುಂದ-ರ-ರಾಮ
ಸುಂದ-ರ-ವಾಗಿ
ಸುಂದ-ರ-ವಾ-ಗಿತ್ತು
ಸುಂದ-ರ-ವಾ-ಗಿದೆ
ಸುಂದ-ರ-ವಾ-ಗಿ-ದ್ದರೆ
ಸುಂದ-ರ-ವಾ-ಗಿರ
ಸುಂದ-ರ-ವಾದ
ಸುಂದ-ರ-ವಾ-ದ-ದ್ದನ್ನು
ಸುಂಯಿ-ಗು-ಡುವ
ಸುಂಯ್ಯನೆ
ಸುಕೃ-ತ-ಫ-ಲವೇ
ಸುಕೋತ್ರ
ಸುಕೋ-ಮಲ
ಸುಕೋ-ಮ-ಲ-ವಾ-ಗಿ-ಬಿ-ಡು-ತ್ತದೆ
ಸುಖ
ಸುಖಂ
ಸುಖ-ಕರ
ಸುಖ-ಕ-ರ-ವಾ-ಗಿ-ರ-ಲಿಲ್ಲ
ಸುಖ-ಕ್ಕಾಗಿ
ಸುಖದ
ಸುಖ-ದಲ್ಲೇ
ಸುಖ-ದುಃ-ಖ-ಗಳ
ಸುಖ-ದುಃ-ಖ-ಗಳನ್ನೆಲ್ಲ
ಸುಖ-ದುಃ-ಖ-ಗಳಲ್ಲಿ
ಸುಖ-ನಿದ್ರೆ
ಸುಖ-ನಿ-ದ್ರೆಯ
ಸುಖ-ಪ್ರ-ಯಾ-ಣ-ವನ್ನು
ಸುಖ-ಭೋಗ
ಸುಖ-ಭೋ-ಗಾ-ಸ-ಕ್ತಿ-ಯನ್ನು
ಸುಖ-ಭೋ-ಜ-ನದ
ಸುಖ-ಮ-ಯ-ವಾಗಿ
ಸುಖ-ಮ-ಯ-ವಾ-ಗಿತ್ತು
ಸುಖ-ಮ-ಯ-ವಾ-ದೀತು
ಸುಖ-ವನ್ನು
ಸುಖ-ವಾಗಿ
ಸುಖ-ವಾ-ಗಿ-ದ್ದೆವು
ಸುಖವೇ
ಸುಖ-ಶಾಂತಿ
ಸುಖ-ಸೌ-ಕರ್ಯ
ಸುಖ-ಸೌ-ಕ-ರ್ಯ-ಗಳ
ಸುಖ-ಸೌ-ಕ-ರ್ಯ-ಗಳಲ್ಲಿ
ಸುಖಾ-ಸ-ನ-ದಲ್ಲಿ
ಸುಖಿ-ಯಾಗು
ಸುಖೀ
ಸುಖೋ-ಮ್ಲಿಂಸ್ಕಿ
ಸುಗಂಧ
ಸುಗಂ-ಧ-ದ್ರವ್ಯ
ಸುಗ-ಮ-ಗೊ-ಳಿ-ಸಲು
ಸುಗ-ಮ-ವಾಗಿ
ಸುಜ-ನ್ಸಿಂಗ್
ಸುಟ್ಟ
ಸುಟ್ಟು
ಸುಟ್ಟು-ಕೊ-ಳ್ಳದೆ
ಸುಟ್ಟು-ಹಾಕು
ಸುಡ
ಸುಡ-ಲೇ-ಬೇಕು
ಸುಡುತ್ತ
ಸುಡು-ತ್ತದೆ
ಸುಡು-ಬಿ-ಸಿಲು
ಸುಡು-ಮ-ದ್ದಿನ
ಸುಡು-ಮ-ರಳು
ಸುಡುವ
ಸುಣ್ಣ
ಸುತ-ನ-ಲ್ಲವೆ
ಸುತ-ರಾಂ
ಸುತರೆ
ಸುತ್ತ
ಸುತ್ತಣ
ಸುತ್ತದೆ
ಸುತ್ತ-ಮುತ್ತ
ಸುತ್ತ-ಮು-ತ್ತಲ
ಸುತ್ತ-ಮು-ತ್ತ-ಲಿನ
ಸುತ್ತ-ಮು-ತ್ತ-ಲಿ-ರುವ
ಸುತ್ತಲ
ಸುತ್ತಲಿ
ಸುತ್ತ-ಲಿ-ದ್ದ-ವರ
ಸುತ್ತ-ಲಿ-ದ್ದ-ವರು
ಸುತ್ತ-ಲಿನ
ಸುತ್ತ-ಲಿ-ರು-ವ-ವರೆಲ್ಲ
ಸುತ್ತಲೂ
ಸುತ್ತಾ-ಡ-ಬೇ-ಕಿಲ್ಲ
ಸುತ್ತಾ-ಡಲು
ಸುತ್ತಾ-ಡು-ತ್ತಿತ್ತು
ಸುತ್ತಾ-ಡು-ತ್ತಿ-ದ್ದಾಗ
ಸುತ್ತಾ-ರಾ-ದರೂ
ಸುತ್ತಿ
ಸುತ್ತಿ-ಕೊಂಡು
ಸುತ್ತಿ-ಕೊ-ಳ್ಳುತ್ತ
ಸುತ್ತಿತ್ತು
ಸುತ್ತಿದ
ಸುತ್ತಿ-ರುವ
ಸುತ್ತು
ಸುತ್ತು-ತ್ತಿ-ದ್ದುವು
ಸುತ್ತು-ವ-ರಿ-ದಿದೆ
ಸುತ್ತು-ವ-ರಿಯ
ಸುದೀರ್ಘ
ಸುದೀ-ರ್ಘ-ಭಾ-ವ-ಪೂ-ರ್ಣ-ರೋ-ಮಾಂ-ಚಕ
ಸುದೀ-ರ್ಘ-ಕಾಲ
ಸುದೀ-ರ್ಘ-ವಾಗಿ
ಸುದೀ-ರ್ಘ-ವಾ-ಗಿಯೇ
ಸುದೂ-ರ-ಭ-ವಿ-ಷ್ಯ-ವನ್ನು
ಸುದೃಢ
ಸುದೈವ
ಸುದ್ದಿ
ಸುದ್ದಿ-ಗಳನ್ನು
ಸುದ್ದಿ-ಗಳು
ಸುದ್ದಿಗೆ
ಸುದ್ದಿಯ
ಸುದ್ದಿ-ಯನ್ನು
ಸುದ್ದಿಯೂ
ಸುದ್ದಿ-ಯೆಂ-ದರೆ
ಸುಧಾ-ರಕ
ಸುಧಾ-ರ-ಕ-ನಲ್ಲ
ಸುಧಾ-ರ-ಕರ
ಸುಧಾ-ರ-ಕ-ರಿಗೆ
ಸುಧಾ-ರ-ಕರು
ಸುಧಾ-ರ-ಕ-ರೆ-ನ್ನಿಸಿ
ಸುಧಾ-ರ-ಕ-ರೆ-ನ್ನಿ-ಸಿ-ಕೊಂ-ಡ-ವರು
ಸುಧಾ-ರ-ಕರೇ
ಸುಧಾ-ರಣಾ
ಸುಧಾ-ರಣೆ
ಸುಧಾ-ರ-ಣೆ-ಎಂ-ಬು-ದನ್ನು
ಸುಧಾ-ರ-ಣೆ-ಗಳನ್ನು
ಸುಧಾ-ರ-ಣೆ-ಗಾಗಿ
ಸುಧಾ-ರ-ಣೆಯ
ಸುಧಾ-ರ-ಣೆ-ಯನ್ನು
ಸುಧಾ-ರ-ಣೆ-ಯಾ-ಗ-ಬೇ-ಕೆಂ-ಬ-ವರು
ಸುಧಾ-ರ-ಣೆ-ಯಾ-ಗ-ಬೇ-ಕೆ-ನ್ನು-ತ್ತೇನೆ
ಸುಧಾ-ರ-ಣೆ-ಯಾ-ಗಿ-ದೆಯೆ
ಸುಧಾ-ರ-ಣೆಯೇ
ಸುಧಾ-ರ-ಣೆ-ಯೇನೂ
ಸುಧಾ-ರ-ಣೆ-ಯೊಂ-ದಿಗೆ
ಸುಧಾ-ರ-ಣೋ-ಪಾ-ಯ-ವೆಂದರೆ
ಸುಧಾ-ರಿಸ
ಸುಧಾ-ರಿ-ಸ-ತೊ-ಡ-ಗಿತು
ಸುಧಾ-ರಿ-ಸ-ದಿ-ದ್ದರೂ
ಸುಧಾ-ರಿ-ಸ-ಬ-ಹು-ದೆಂದು
ಸುಧಾ-ರಿ-ಸ-ಬ-ಹು-ದೆಂಬ
ಸುಧಾ-ರಿ-ಸ-ಲಿಲ್ಲ
ಸುಧಾ-ರಿ-ಸಲು
ಸುಧಾ-ರಿಸಿ
ಸುಧಾ-ರಿ-ಸಿ-ಕೊಂ-ಡರು
ಸುಧಾ-ರಿ-ಸಿ-ಕೊಂಡು
ಸುಧಾ-ರಿ-ಸಿ-ಕೊ-ಳ್ಳಲು
ಸುಧಾ-ರಿ-ಸಿ-ಕೊ-ಳ್ಳು-ವುದು
ಸುಧಾ-ರಿ-ಸಿತು
ಸುಧಾ-ರಿ-ಸಿತ್ತು
ಸುಧಾ-ರಿ-ಸಿ-ತ್ತೆಂ-ದರೆ
ಸುಧಾ-ರಿ-ಸಿದ
ಸುಧಾ-ರಿ-ಸಿದೆ
ಸುಧಾ-ರಿ-ಸಿದ್ದ
ಸುಧಾ-ರಿ-ಸಿ-ದ್ದ-ರಿಂದ
ಸುಧಾ-ರಿ-ಸಿ-ರ-ಲಿಲ್ಲ
ಸುಧಾ-ರಿ-ಸುತ್ತ
ಸುಧಾ-ರಿ-ಸು-ತ್ತಿದೆ
ಸುಧಾ-ರಿ-ಸು-ತ್ತಿ-ದ್ದಂ-ತೆಯೇ
ಸುಧಾ-ರಿ-ಸು-ವಂ-ತಾ-ಗಲು
ಸುಧಾ-ರಿ-ಸು-ವ-ವ-ರೆಗೂ
ಸುಧಾ-ರಿ-ಸು-ವು-ದೆಂಬ
ಸುಧೀರ್
ಸುಧೀ-ರ್ಚ-ಕ್ರ-ವರ್ತಿ
ಸುಧೃಢ
ಸುಪ-ರಿ-ಚಿತ
ಸುಪ-ರಿ-ಚಿ-ತ-ನಾದ
ಸುಪ-ರಿ-ಚಿ-ತ-ರಾ-ಗಿ-ದ್ದರು
ಸುಪ-ರಿ-ಚಿ-ತ-ವಾದ
ಸುಪ್ತ
ಸುಪ್ತ-ಚೈ-ತ-ನ್ಯ-ವನ್ನು
ಸುಪ್ತ-ವಾಗಿ
ಸುಪ್ತ-ವಾ-ಗಿ-ರುವ
ಸುಪ್ತಾ-ವ-ಸ್ಥೆ-ಯ-ಲ್ಲಿದೆ
ಸುಪ್ತಾ-ವ-ಸ್ಥೆ-ಯ-ಲ್ಲಿನ
ಸುಪ್ರ-ಸಿದ್ಧ
ಸುಪ್ರ-ಸಿ-ದ್ಧ-ವಾದ
ಸುಬೋ-ಧಾ-ನಂದ
ಸುಬ್ಬ-ರಾವ್
ಸುಬ್ರ-ಮಣ್ಯ
ಸುಬ್ರ-ಹ್ಮಣ್ಯ
ಸುಭ-ದ್ರ-ವಾ-ಗಿ-ರು-ತ್ತದೆ
ಸುಮ-ಧುರ
ಸುಮ-ನೋ-ಹರ
ಸುಮಾರು
ಸುಮ್ಮ-ನಾಗಿ
ಸುಮ್ಮ-ನಾ-ಗಿ-ಬಿ-ಟ್ಟರು
ಸುಮ್ಮ-ನಾ-ಗಿ-ಸು-ತ್ತಿ-ದ್ದರು
ಸುಮ್ಮ-ನಾದ
ಸುಮ್ಮ-ನಾ-ದರು
ಸುಮ್ಮ-ನಾ-ದಳು
ಸುಮ್ಮ-ನಾದೆ
ಸುಮ್ಮ-ನಿದೆ
ಸುಮ್ಮ-ನಿ-ದ್ದರು
ಸುಮ್ಮ-ನಿ-ದ್ದಾ-ರಲ್ಲ
ಸುಮ್ಮ-ನಿ-ದ್ದಿ-ರ-ಬ-ಹುದು
ಸುಮ್ಮ-ನಿದ್ದು
ಸುಮ್ಮ-ನಿ-ರ-ಲಾ-ಗ-ಲಿಲ್ಲ
ಸುಮ್ಮ-ನಿ-ರಲು
ಸುಮ್ಮ-ನಿ-ರು-ವು-ದ-ರಿಂದ
ಸುಮ್ಮ-ನಿ-ರು-ವುದು
ಸುಮ್ಮನೆ
ಸುರಂಗ
ಸುರಂ-ಗ-ಗಳನ್ನು
ಸುರ-ಕ್ಷಿತ
ಸುರ-ಕ್ಷಿ-ತ-ವಾಗಿ
ಸುರ-ಕ್ಷಿ-ತ-ವಾ-ಗಿ-ದ್ದೇನೆ
ಸುರ-ಕ್ಷಿ-ತ-ವಾ-ಗಿ-ದ್ದೇ-ನೆಂದು
ಸುರ-ಕ್ಷಿ-ತ-ವಾ-ಗಿಯೇ
ಸುರ-ಕ್ಷಿ-ತ-ವಾ-ಗಿ-ರ-ಬ-ಲ್ಲು-ದೆಂದೇ
ಸುರಕ್ಷೆ
ಸುರಿದ
ಸುರಿದು
ಸುರಿ-ದು-ಕೊ-ಳ್ಳಲು
ಸುರಿದೆ
ಸುರಿ-ಮಳೆ
ಸುರಿ-ಮ-ಳೆ-ಯಾ-ಗು-ತ್ತಿತ್ತು
ಸುರಿ-ಮ-ಳೆಯೇ
ಸುರಿ-ಯ-ತೊ-ಡ-ಗಿತು
ಸುರಿ-ಯಿತು
ಸುರಿ-ಯು-ತ್ತಿದೆ
ಸುರಿ-ಸ-ತೊ-ಡ-ಗಿ-ದರು
ಸುರಿ-ಸಿ-ದರು
ಸುರಿ-ಸಿ-ದ್ದನ್ನು
ಸುರಿ-ಸುತ್ತ
ಸುರಿ-ಸುವ
ಸುರೇಂ-ದ್ರ-ನಾಥ
ಸುರೇ-ಶ್ವ-ರಾ-ನಂದ
ಸುರೇ-ಶ್ವ-ರಾ-ನಂ-ದ-ರನ್ನು
ಸುಲಭ
ಸುಲ-ಭದ
ಸುಲ-ಭ-ದಲ್ಲಿ
ಸುಲ-ಭ-ದ್ದೇನೂ
ಸುಲ-ಭ-ವಲ್ಲ
ಸುಲ-ಭ-ವಾ-ಗ-ಲಿಲ್ಲ
ಸುಲ-ಭ-ವಾಗಿ
ಸುಲ-ಭ-ವಾ-ಗಿತ್ತು
ಸುಲ-ಭ-ವಾ-ಗು-ವಂತೆ
ಸುಲ-ಭ-ವಾ-ದುದು
ಸುಲ-ಭವೆ
ಸುಲ-ಭ-ವೆಂದು
ಸುಲ-ಭವೇ
ಸುಲ-ಭ-ಸಾ-ಧ್ಯ-ವಾ-ಗು-ತ್ತದೆ
ಸುಲ-ಲಿತ
ಸುಲ-ಲಿ-ತ-ವಾಗಿ
ಸುಳಿ-ದಾ-ಡಿ-ದ್ದ-ರಿಂದ
ಸುಳಿ-ದಾ-ಡು-ತ್ತಿ-ದ್ದುವು
ಸುಳಿ-ಯಂತೆ
ಸುಳಿ-ವನ್ನು
ಸುಳಿವು
ಸುಳಿವೂ
ಸುಳಿವೇ
ಸುಳ್ಳಾ-ಗು-ವು-ದಿಲ್ಲ
ಸುಳ್ಳು
ಸುಳ್ಳು-ಧರ್ಮ
ಸುಳ್ಳೆಂದು
ಸುವ
ಸುವಂ-ತಾ-ಗಲು
ಸುವರ್ಣ
ಸುವ-ರ್ಣ-ವರ್ಣ
ಸುವ-ರ್ಣ-ವ-ರ್ಣ-ದಿಂದ
ಸುವ-ರ್ಣ-ವೆಂ-ಬುದೂ
ಸುವ-ರ್ಣಾ-ಕ್ಷರ
ಸುವ-ರ್ಣಾ-ಕ್ಷ-ರ-ಗಳಿಂದ
ಸುವಷ್ಟು
ಸುವಾಗ
ಸುವಾ-ಸ-ನೆ-ಇ-ವು-ಗಳ
ಸುವಿ-ಖ್ಯಾತ
ಸುವಿ-ಖ್ಯಾ-ತರು
ಸುವಿ-ಸ್ತಾ-ರ-ವಾಗಿ
ಸುವಿ-ಸ್ತಾ-ರ-ವಾದ
ಸುವು-ದ-ರ-ಲ್ಲಿಯೇ
ಸುವು-ದಾ-ಗಲಿ
ಸುವುದು
ಸುವು-ದೆಂದು
ಸುವ್ಯ-ಕ್ತ-ವಾ-ಗಿದೆ
ಸುವ್ಯ-ಕ್ತ-ವಾ-ಗು-ತ್ತಿತ್ತು
ಸುವ್ಯ-ವ-ಸ್ಥಿ-ತ-ವಾಗಿ
ಸುಶಿ-ಕ್ಷಿತ
ಸುಶಿ-ಕ್ಷಿ-ತ-ರಾದ
ಸುಶಿ-ಕ್ಷಿ-ತರೂ
ಸುಶೀಲ್
ಸುಶ್ರಾವ್ಯ
ಸುಶ್ರಾ-ವ್ಯ-ವಾಗಿ
ಸುಶ್ರಾ-ವ್ಯ-ವಾದ
ಸುಷಮ್ಣಃ
ಸುಷುಮ್ಣಃ
ಸುಷು-ಮ್ನದ
ಸುಷು-ಮ್ನ-ವಾ-ಗಿ-ದ್ದಾ-ನೆಯೋ
ಸುಷುಮ್ನಾ
ಸುಸಂ-ದರ್ಭ
ಸುಸಂ-ಸ್ಕೃತ
ಸುಸಂ-ಸ್ಕೃ-ತರು
ಸುಸಂ-ಸ್ಕೃ-ತರೂ
ಸುಸ-ಜ್ಜಿತ
ಸುಸ-ಜ್ಜಿ-ತ-ವಾದ
ಸುಸೂ-ತ್ರ-ವಾಗಿ
ಸುಸ್ಥಾ-ಪಿ-ತ-ವಾದ
ಸುಸ್ಥಿತಿ
ಸುಸ್ಥಿ-ತಿಗೆ
ಸುಸ್ಪಷ್ಟ
ಸುಸ್ಪ-ಷ್ಟ-ವಾ-ಗಿತ್ತು
ಸುಸ್ಪ-ಷ್ಟ-ವಾ-ದದ್ದು
ಸುಸ್ವಾ-ಗತ
ಸುಸ್ವಾ-ಗ-ತವ
ಸೂ
ಸೂಕ್ತ
ಸೂಕ್ತ-ವಾಗಿ
ಸೂಕ್ತ-ವಾ-ಗಿತ್ತು
ಸೂಕ್ತ-ವಾ-ಗಿದೆ
ಸೂಕ್ತ-ವಾದ
ಸೂಕ್ತ-ವಾ-ದು-ವೆಂದು
ಸೂಕ್ತವೂ
ಸೂಕ್ತ-ಸ್ಥಾ-ನ-ವಿದೆ
ಸೂಕ್ಷ್ಮ
ಸೂಕ್ಷ್ಮ-ಗ್ರಾ-ಹಿ-ಯಾ-ಗಿದ್ದ
ಸೂಕ್ಷ್ಮ-ತೆ-ಯನ್ನು
ಸೂಕ್ಷ್ಮ-ದ-ರ್ಶಕ
ಸೂಕ್ಷ್ಮ-ದ-ರ್ಶ-ಕ-ವನ್ನು
ಸೂಕ್ಷ್ಮ-ದೃಷ್ಟಿ
ಸೂಕ್ಷ್ಮ-ಪ-ರಿ-ಚ-ಯ-ವಾ-ಗು-ತ್ತಿತ್ತು
ಸೂಕ್ಷ್ಮ-ಮ-ತಿಯೂ
ಸೂಕ್ಷ್ಮ-ವಾಗಿ
ಸೂಕ್ಷ್ಮ-ವಾ-ಗಿ-ತ್ತೆಂ-ದರೆ
ಸೂಕ್ಷ್ಮ-ವಾ-ಗಿ-ರ-ಬೇ-ಕಾಗು
ಸೂಕ್ಷ್ಮ-ವಾದ
ಸೂಕ್ಷ್ಮ-ವಾ-ದದ್ದ
ಸೂಕ್ಷ್ಮ-ವಾ-ದದ್ದು
ಸೂಕ್ಷ್ಮ-ವಾ-ದುದು
ಸೂಕ್ಷ್ಮವೂ
ಸೂಚಕ
ಸೂಚ-ಕ-ವಾಗಿ
ಸೂಚನೆ
ಸೂಚ-ನೆ-ಗಳನ್ನು
ಸೂಚ-ನೆ-ಗ-ಳಿವೆ
ಸೂಚ-ನೆ-ಗಳು
ಸೂಚ-ನೆ-ಯಂತೆ
ಸೂಚ-ನೆ-ಯನ್ನು
ಸೂಚ-ನೆ-ಯಿತ್ತು
ಸೂಚ-ನೆಯೂ
ಸೂಚಿ-ತ-ವಾದ
ಸೂಚಿ-ಸ-ಲಾ-ಯಿತು
ಸೂಚಿ-ಸಲು
ಸೂಚಿಸಿ
ಸೂಚಿ-ಸಿ-ದರು
ಸೂಚಿ-ಸಿ-ದ-ರು-ಬೇ-ರೆಲ್ಲ
ಸೂಚಿ-ಸಿ-ದರೆ
ಸೂಚಿ-ಸಿ-ದ-ವರೂ
ಸೂಚಿ-ಸಿ-ದುವು
ಸೂಚಿ-ಸು-ತ್ತದೆ
ಸೂಚಿ-ಸು-ತ್ತಿ-ದ್ದಾರೆ
ಸೂಚಿ-ಸುವ
ಸೂಚಿ-ಸು-ವುದು
ಸೂಚ್ಯ-ವಾ-ಗಿತ್ತು
ಸೂಜಿ
ಸೂಟ್ಕೇ-ಸಿನ
ಸೂತ್ರ
ಸೂತ್ರ-ಗಳ
ಸೂತ್ರದ
ಸೂತ್ರ-ವಿ-ಲ್ಲದೆ
ಸೂತ್ರವೂ
ಸೂಯೆಜ್
ಸೂರೆ
ಸೂರೆ-ಗೊಂ-ಡಿತು
ಸೂರೆ-ಯಾಗಿ
ಸೂರ್ಯ
ಸೂರ್ಯ-ಕಿ-ರ-ಣ-ಗ-ಳಿಗೆ
ಸೂರ್ಯ-ಕಿ-ರ-ಣ-ಗಳು
ಸೂರ್ಯ-ಗ್ರ-ಹಣ
ಸೂರ್ಯನ
ಸೂರ್ಯ-ನಿ-ಗಂತೂ
ಸೂರ್ಯ-ನಿಗೂ
ಸೂರ್ಯನು
ಸೂರ್ಯ-ಬಿಂ-ಬ-ವನ್ನು
ಸೂರ್ಯ-ರ-ಶ್ಮಿ-ಶ್ಚಂ-ದ್ರಮಾ
ಸೂಸು-ತ್ತಿತ್ತು
ಸೂಸುವ
ಸೂಸು-ವಂ-ತಹ
ಸೃಜಿ-ಸ-ಬ-ಹುದು
ಸೃಷ್ಟಿ
ಸೃಷ್ಟಿಯು
ಸೃಷ್ಟಿ-ಸ-ಬ-ಲ್ಲುದು
ಸೃಷ್ಟಿ-ಸ-ಲ್ಪ-ಟ್ಟಿತು
ಸೃಷ್ಟಿ-ಸ-ಹೊ-ರ-ಟ-ವರು
ಸೃಷ್ಟಿಸಿ
ಸೃಷ್ಟಿ-ಸು-ವು-ದ-ಕ್ಕಾಗಿ
ಸೃಷ್ಟಿ-ಸು-ವುದು
ಸೃಷ್ಟಿ-ಸೌಂ-ದರ್ಯ
ಸೆಂಟ್
ಸೆಕೆ-ಕಾಲ
ಸೆಕೆ-ಗಾ-ಲದ
ಸೆಕೆ-ಗಾ-ಲ-ವೆಂದರೆ
ಸೆಖೆ
ಸೆಖೆ-ಗಾ-ಲಕ್ಕೆ
ಸೆಟೆದು
ಸೆಡ್ನಿಂದ
ಸೆಣ-ಸಾಡ
ಸೆಣ-ಸಿ-ದ್ದೇನೆ
ಸೆಣ-ಸುತ್ತ
ಸೆಣ-ಸು-ತ್ತಿ-ದ್ದಳು
ಸೆಪ್ಟೆಂ-ಬ-ರಿನ
ಸೆಪ್ಟೆಂ-ಬ-ರಿ-ನಲ್ಲಿ
ಸೆಪ್ಟೆಂ-ಬರ್
ಸೆರ-ಗಿ-ನಿಂದ
ಸೆರೆ-ಮ-ನೆಗೆ
ಸೆರೆ-ಮ-ನೆ-ಯ-ಲ್ಲಿ-ರುವ
ಸೆರೆ-ಮ-ನೆ-ಯಿಂ
ಸೆರ್ಮಿ-ನಾರ
ಸೆಳೆ
ಸೆಳೆತ
ಸೆಳೆ-ತಕ್ಕೆ
ಸೆಳೆ-ದರು
ಸೆಳೆ-ದಿದ್ದ
ಸೆಳೆ-ದು-ಬಿ-ಡು-ತ್ತಿ-ದ್ದುವು
ಸೆಳೆ-ದುವು
ಸೆಳೆ-ಯ-ಬೇ-ಕಾ-ಯಿತು
ಸೆಳೆ-ಯ-ಬೇಕು
ಸೆಳೆ-ಯಲು
ಸೆಳೆ-ಯ-ಲ್ಪ-ಟ್ಟರು
ಸೆಳೆ-ಯ-ಲ್ಪಟ್ಟೆ
ಸೆಳೆ-ಯ-ಲ್ಪ-ಡುತ್ತಿ
ಸೆಳೆಯು
ಸೆಳೆ-ಯು-ತ್ತಿ-ದೆ-ಯೆಂ-ಬು-ದನ್ನು
ಸೆಳೆ-ಯು-ತ್ತಿ-ದ್ದರು
ಸೆಳೆ-ಯು-ವಂ-ತಾ-ಗಲಿ
ಸೆಳೆ-ಯು-ವುದು
ಸೇಕ್ರ-ಮೆಂಟ್
ಸೇತು-ಪತಿ
ಸೇತು-ಪ-ತಿಯು
ಸೇತುವೆ
ಸೇತು-ವೆ-ಯಿದೆ
ಸೇತು-ವೆ-ಯೊಂ-ದನ್ನು
ಸೇದುತ್ತ
ಸೇದುವ
ಸೇದು-ವು-ದುಂಟು
ಸೇನರು
ಸೇನ್
ಸೇಬನ್ನೋ
ಸೇಬಿನ
ಸೇರ-ತೊ-ಡ-ಗಿ-ದ್ದರು
ಸೇರ-ದ-ವ-ರನ್ನು
ಸೇರ-ಬೇಕಾ
ಸೇರ-ಬೇ-ಕಾದ್ದು
ಸೇರ-ಬೇ-ಕಾ-ದ್ದೆಂದೂ
ಸೇರ-ಬೇಕು
ಸೇರ-ಬೇ-ಕೆಂದು
ಸೇರಲಿ
ಸೇರ-ಲಿ-ರು-ವುದು
ಸೇರಲು
ಸೇರಿ
ಸೇರಿ-ಕೊಂ-ಡರು
ಸೇರಿ-ಕೊಂ-ಡಳು
ಸೇರಿ-ಕೊಂಡಿ
ಸೇರಿ-ಕೊಂ-ಡಿತ್ತು
ಸೇರಿ-ಕೊಂ-ಡಿದ್ದ
ಸೇರಿ-ಕೊಂ-ಡಿ-ದ್ದರು
ಸೇರಿ-ಕೊಂ-ಡಿ-ದ್ದ-ವರೆಲ್ಲ
ಸೇರಿ-ಕೊಂ-ಡಿ-ರುವ
ಸೇರಿ-ಕೊಂ-ಡಿಲ್ಲ
ಸೇರಿ-ಕೊಂ-ಡಿವೆ
ಸೇರಿ-ಕೊಂಡು
ಸೇರಿ-ಕೊಂ-ಡುವು
ಸೇರಿ-ಕೊ-ಳ್ಳ-ಬೇ-ಕಾ-ಯಿತು
ಸೇರಿ-ಕೊ-ಳ್ಳ-ಬೇಕು
ಸೇರಿ-ಕೊ-ಳ್ಳು-ತ್ತೇನೆ
ಸೇರಿ-ಕೊ-ಳ್ಳುವ
ಸೇರಿ-ಕೊ-ಳ್ಳು-ವ-ವರು
ಸೇರಿತು
ಸೇರಿತ್ತು
ಸೇರಿದ
ಸೇರಿ-ದಂತೆ
ಸೇರಿ-ದರು
ಸೇರಿ-ದವ
ಸೇರಿ-ದ-ವ-ನಲ್ಲ
ಸೇರಿ-ದ-ವನು
ಸೇರಿ-ದ-ವ-ರು-ವಿ-ಶ್ವ-ಮಾ-ನ-ವರು
ಸೇರಿ-ದ-ವ-ರೆಂ-ಬು-ದನ್ನು
ಸೇರಿ-ದ-ವರೇ
ಸೇರಿ-ದ-ವಳು
ಸೇರಿ-ದವು
ಸೇರಿ-ದಾಗ
ಸೇರಿ-ದುವು
ಸೇರಿದ್ದ
ಸೇರಿ-ದ್ದರು
ಸೇರಿ-ದ್ದ-ವರು
ಸೇರಿದ್ದು
ಸೇರಿ-ದ್ದುವು
ಸೇರಿ-ಬಿ-ಟ್ಟರು
ಸೇರಿ-ಬಿ-ಟ್ಟಿತ್ತು
ಸೇರಿ-ಬಿ-ಟ್ಟಿದೆ
ಸೇರಿ-ಬಿ-ಟ್ಟಿ-ದ್ದರು
ಸೇರಿ-ರುವ
ಸೇರಿ-ರು-ವಾಗ
ಸೇರಿ-ರು-ವುದು
ಸೇರಿವೆ
ಸೇರಿ-ಸದೆ
ಸೇರಿ-ಸ-ಬ-ಹುದು
ಸೇರಿ-ಸ-ಲಾಗಿದೆ
ಸೇರಿಸಿ
ಸೇರಿ-ಸಿ-ಕೊ-ಳ್ಳ-ಬೇ-ಕಾ-ದದ್ದು
ಸೇರಿ-ಸಿ-ಕೊ-ಳ್ಳಲು
ಸೇರಿ-ಸಿ-ದರು
ಸೇರಿ-ಸಿ-ದ-ರು-ಅ-ವಳು
ಸೇರಿ-ಸಿ-ದುವು
ಸೇರಿ-ಸಿ-ಬಿ-ಡು-ತ್ತಿ-ದ್ದರು
ಸೇರಿ-ಸು-ತ್ತಾ-ರೆ-ನಾನೂ
ಸೇರಿ-ಸುವ
ಸೇರಿ-ಸು-ವಲ್ಲಿ
ಸೇರು-ತ್ತದೆ
ಸೇರು-ತ್ತಾರೆ
ಸೇರು-ತ್ತಿ-ದ್ದರು
ಸೇರುವ
ಸೇರು-ವಂ-ತಹ
ಸೇರು-ವ-ಷ್ಟ-ರಲ್ಲಿ
ಸೇರು-ವಾ-ಗಲೂ
ಸೇವಕ
ಸೇವ-ಕ-ನಾಗಿ
ಸೇವ-ಕ-ನಿಗೆ
ಸೇವ-ಕನು
ಸೇವ-ಕನೂ
ಸೇವ-ಕರ
ಸೇವ-ಕ-ರನ್ನು
ಸೇವ-ಕರು
ಸೇವಕಿ
ಸೇವ-ನೆಗೆ
ಸೇವಾ
ಸೇವಾ-ಕ-ರ್ತ-ರಲ್ಲಿ
ಸೇವಾ-ಕಾ-ರ್ಯ-ಗಳನ್ನು
ಸೇವಾ-ಕಾ-ರ್ಯದ
ಸೇವಾ-ಕಾ-ರ್ಯ-ದಲ್ಲಿ
ಸೇವಾ-ಕಾ-ರ್ಯ-ದೊಂ-ದಿಗೆ
ಸೇವಾ-ಕಾ-ರ್ಯ-ವನ್ನು
ಸೇವಾ-ಕಾ-ರ್ಯವು
ಸೇವಾ-ಕೇಂ-ದ್ರ-ವನ್ನು
ಸೇವಾ-ಕೇಂ-ದ್ರ-ವೊಂ-ದನ್ನು
ಸೇವಾ-ನಿ-ರ-ತ-ರಾ-ಗಿ-ರು-ವಾಗ
ಸೇವಾ-ನಿ-ರ-ತ-ರಾ-ಗು-ವು-ದ-ರಿಂದ
ಸೇವಾ-ನಿ-ಷ್ಠೆ-ಯನ್ನು
ಸೇವಾ-ಮ-ನೋ-ಭಾ-ವ-ಇ-ವು-ಗಳನ್ನು
ಸೇವಾ-ವ್ರ-ತಿ-ಯ-ರನ್ನೂ
ಸೇವಾ-ಶ್ರಮ
ಸೇವಾ-ಶ್ರ-ಮ-ಗಳ
ಸೇವಾ-ಶ್ರ-ಮ-ದಲ್ಲಿ
ಸೇವಾ-ಶ್ರ-ಮ-ವೊಂ-ದನ್ನು
ಸೇವಾ-ಸಂ-ಸ್ಥೆಯ
ಸೇವಾ-ಸಂ-ಸ್ಥೆಯು
ಸೇವಿ
ಸೇವಿ-ಯರ
ಸೇವಿ-ಯ-ರ-ನ್ನೊಮ್ಮೆ
ಸೇವಿ-ಯ-ರರ
ಸೇವಿ-ಯ-ರ-ರನ್ನು
ಸೇವಿ-ಯ-ರರು
ಸೇವಿ-ಯ-ರರೇ
ಸೇವಿ-ಯ-ರ-ರೊಂ-ದಿಗೆ
ಸೇವಿ-ಯರ್
ಸೇವಿ-ಯರ್ಗೆ
ಸೇವಿ-ಯ-ರ್ರನ್ನು
ಸೇವಿ-ಯ-ರ್ರನ್ನೂ
ಸೇವಿ-ಯ-ರ್ರಿಂದ
ಸೇವಿ-ಯ-ರ್ರಿಗೆ
ಸೇವಿಸಿ
ಸೇವೆ
ಸೇವೆ-ಇವು
ಸೇವೆ-ಇ-ವೆ-ರಡೂ
ಸೇವೆ-ಇ-ವೆಲ್ಲ
ಸೇವೆ-ಗಳನ್ನು
ಸೇವೆ-ಗಾಗಿ
ಸೇವೆಗೆ
ಸೇವೆ-ಗೈದು
ಸೇವೆ-ಗೈ-ಯು-ವು-ದ-ರಿಂದ
ಸೇವೆಯ
ಸೇವೆ-ಯನ್ನು
ಸೇವೆ-ಯನ್ನೇ
ಸೇವೆ-ಯಲ್ಲಿ
ಸೇವೆ-ಯಿಂದ
ಸೇವೆಯು
ಸೇವೆ-ಯೆನ್ನು
ಸೇವೆಯೇ
ಸೈ
ಸೈಂಟ್
ಸೈತಾನ
ಸೈತಾ-ನ-ಇ-ವ-ರಿ-ಬ್ಬರೂ
ಸೈತಾ-ನರ
ಸೈದ್ಧಾಂ-ತಿ-ಕ-ವಾ-ಗಿತ್ತೋ
ಸೈನಿ
ಸೈನಿ-ಕ-ನ-ಲ್ಲವೆ
ಸೈನ್ಯ
ಸೈನ್ಯ-ದಲ್ಲಿ
ಸೈನ್ಯ-ಶ-ಕ್ತಿ-ಯ-ಲ್ಲಲ್ಲ
ಸೈನ್ಯಾ-ಧಿ-ಕಾರಿ
ಸೈನ್ಸಸ್
ಸೈನ್ಸಸ್ನ
ಸೊಂಪಾಗಿ
ಸೊಗಸು
ಸೊತ್ತಾ-ಗು-ವಂತೆ
ಸೊಬ-ಗನ್ನು
ಸೊರ-ಗಿ-ಹೋ-ಗಿ-ದ್ದ-ರೆಂ-ದರೆ
ಸೊರ-ಗಿ-ಹೋದ
ಸೊರೋ-ಕಿನ್
ಸೊಳ್ಳೆಗೆ
ಸೊಸೆ
ಸೊಸೈಟಿ
ಸೊಸೈ-ಟಿ-ಗಳ
ಸೊಸೈ-ಟಿ-ಗಳನ್ನು
ಸೊಸೈ-ಟಿಗೆ
ಸೊಸೈ-ಟಿಯ
ಸೊಸೈ-ಟಿ-ಯನ್ನು
ಸೊಸೈ-ಟಿ-ಯಲ್ಲಿ
ಸೊಸೈ-ಟಿ-ಯಿಂದ
ಸೊಸೈ-ಟಿಯು
ಸೊಸೈ-ಟಿ-ಯೊಂ-ದರ
ಸೋಂಕು-ಇದ್ದೇ
ಸೋಂಕೇ
ಸೋತ-ರೇ-ನಂತೆ
ಸೋದರ
ಸೋದ-ರ-ನಂತೆ
ಸೋದ-ರ-ನಷ್ಟು
ಸೋದ-ರ-ನಾದ
ಸೋದ-ರನೂ
ಸೋದ-ರರ
ಸೋದ-ರ-ರಲ್ಲಿ
ಸೋದ-ರ-ರಾದ
ಸೋದ-ರ-ರಿಗೆ
ಸೋದ-ರರು
ಸೋದ-ರರೂ
ಸೋದ-ರರೆ
ಸೋದ-ರರೇ
ಸೋದ-ರ-ರೊಂ-ದಿ-ಗಿ-ರ-ಬೇಕು
ಸೋದ-ರ-ರೊಂ-ದಿಗೆ
ಸೋದ-ರ-ಳಿಯ
ಸೋದ-ರ-ಳಿ-ಯ-ನಾದ
ಸೋದ-ರ-ಸಂನ್ಯಾಸಿ
ಸೋದ-ರ-ಸಂ-ನ್ಯಾ-ಸಿ-ಗಳ
ಸೋದ-ರ-ಸಂ-ನ್ಯಾ-ಸಿ-ಗಳನ್ನೂ
ಸೋದ-ರ-ಸಂ-ನ್ಯಾ-ಸಿ-ಗಳೂ
ಸೋದ-ರ-ಸಂ-ನ್ಯಾ-ಸಿ-ಗ-ಳೆಲ್ಲ
ಸೋದರಿ
ಸೋದ-ರಿಗೂ
ಸೋದ-ರಿಗೆ
ಸೋದ-ರಿಯ
ಸೋದ-ರಿ-ಯರ
ಸೋದ-ರಿ-ಯ-ರಿಗೆ
ಸೋದ-ರಿ-ಯ-ರಿ-ದ್ದರು
ಸೋದ-ರಿ-ಯರು
ಸೋದ-ರಿ-ಯ-ರು-ಶ್ರೀ-ಮತಿ
ಸೋದ-ರಿ-ಯರೂ
ಸೋದ-ರಿಯೂ
ಸೋಮ-ವಾರ
ಸೋಮ-ವಾ-ರ-ದಂದು
ಸೋಮಾ-ನಂದ
ಸೋಮಾರಿ
ಸೋಮಾ-ರಿ-ತನ
ಸೋರು-ತ್ತಿದೆ
ಸೋಲ-ಬಾರ
ಸೋಲಿ-ಸ-ಬ-ಲ್ಲ-ವರು
ಸೋಲು
ಸೋಽಹಂ
ಸೌಂದರ್ಯ
ಸೌಂದ-ರ್ಯದ
ಸೌಂದ-ರ್ಯ-ದಿಂದ
ಸೌಂದ-ರ್ಯ-ವ-ನ್ನಂತೂ
ಸೌಂದ-ರ್ಯ-ವನ್ನು
ಸೌಂದ-ರ್ಯ-ವಿ-ರು-ವುದು
ಸೌಂದ-ರ್ಯ-ವೆ-ಷ್ಟಿ-ದೆಯೋ
ಸೌಖ್ಯವೋ
ಸೌಜನ್ಯ
ಸೌಜ-ನ್ಯ-ಕ್ಕಾಗಿ
ಸೌಜ-ನ್ಯ-ಪೂರ್ಣ
ಸೌತೆಯ
ಸೌಥಾಂ-ಪ್ಟ-ನ್ನಿ-ನಿಂದ
ಸೌದೆಯೇ
ಸೌಭಾಗ್ಯ
ಸೌಭಾ-ಗ್ಯಕ್ಕೆ
ಸೌಭಾ-ಗ್ಯದ
ಸೌಭಾ-ಗ್ಯ-ವನ್ನು
ಸೌಭಾ-ಗ್ಯವೂ
ಸೌಲ-ಭ್ಯ-ಸೌ-ಭಾಗ್ಯ
ಸೌಲ-ಭ್ಯ-ಗಳ
ಸೌಲ-ಭ್ಯ-ಗಳನ್ನೂ
ಸೌಲ-ಭ್ಯ-ಗ-ಳಿಲ್ಲ
ಸೌಲ-ಭ್ಯ-ಗಳೂ
ಸೌಲ-ಭ್ಯ-ವಿ-ರ-ಲಿಲ್ಲ
ಸೌಹಾ-ರ್ದ-ಭಾ-ವ-ವನ್ನು
ಸೌಹಾ-ರ್ದ-ಯುತ
ಸೌಹಾ-ರ್ದ-ವನ್ನು
ಸ್ಕ್ವೇರ್
ಸ್ಖಲನಂ
ಸ್ಟಮ್
ಸ್ಟರ್ಜಿಸ್
ಸ್ಟರ್ಡಿ
ಸ್ಟರ್ಡಿಗೆ
ಸ್ಟರ್ಡಿಯ
ಸ್ಟರ್ಡಿ-ಯ-ಲ್ಲದೆ
ಸ್ಟರ್ಡಿ-ಯಿಂದ
ಸ್ಟಾರ್
ಸ್ಟೀಮ-ರಿನ
ಸ್ಟೀಮ-ರಿ-ನಲ್ಲಿ
ಸ್ಟೀಮ-ರು-ಗಳಲ್ಲಿ
ಸ್ಟೀಮರ್
ಸ್ಟೇಷನ್
ಸ್ಟೈಡಲ್
ಸ್ಟ್ರೀಟಿನ
ಸ್ಟ್ರೀಟ್
ಸ್ತಂಭವು
ಸ್ತಂಭ-ವೊಂದೇ
ಸ್ತಂಭೀ-ಭೂತ
ಸ್ತಂಭೀ-ಭೂ-ತ-ರಾ-ದರು
ಸ್ತಬ್ಧ-ನೀ-ರ-ವ-ಮೌನ
ಸ್ತಬ್ಧ-ರಾಗಿ
ಸ್ತಬ್ಧ-ರಾ-ದರು
ಸ್ತಬ್ಧ-ವಾಗಿ
ಸ್ತಬ್ಧ-ವಾ-ಗು-ತ್ತವೆ
ಸ್ತಬ್ಧ-ವಾ-ಯಿತು
ಸ್ತರ-ಕ್ಕೇ-ರಿತ್ತು
ಸ್ತರ-ಗಳನ್ನು
ಸ್ತರ-ಗಳಲ್ಲಿ
ಸ್ತರ-ದಲ್ಲಿ
ಸ್ತುತಿ
ಸ್ತುತಿ-ಸ-ಲಾ-ರಂ-ಭಿ-ಸಿ-ದರು
ಸ್ತುವಂತು
ಸ್ತೋತ್ರ
ಸ್ತೋತ್ರ-ಗಳ
ಸ್ತೋತ್ರ-ವನ್ನು
ಸ್ತ್ರೀ
ಸ್ತ್ರೀತ್ವದ
ಸ್ತ್ರೀಪು-ರು-ಷರ
ಸ್ತ್ರೀಪು-ರು-ಷ-ರನ್ನು
ಸ್ತ್ರೀಪು-ರು-ಷ-ರನ್ನೂ
ಸ್ತ್ರೀಪು-ರು-ಷ-ರಿಗೂ
ಸ್ತ್ರೀಪು-ರು-ಷ-ರಿಗೆ
ಸ್ತ್ರೀಪು-ರು-ಷರು
ಸ್ತ್ರೀಪು-ರು-ಷ-ರೆ-ಲ್ಲ-ರೂಈ
ಸ್ತ್ರೀಪು-ರು-ಷರೇ
ಸ್ತ್ರೀಮ-ಠ-ಗಳು
ಸ್ತ್ರೀಯ
ಸ್ತ್ರೀಯಂತೆ
ಸ್ತ್ರೀಯಂ-ತೆಯೇ
ಸ್ತ್ರೀಯನ್ನೂ
ಸ್ತ್ರೀಯರ
ಸ್ತ್ರೀಯ-ರನ್ನು
ಸ್ತ್ರೀಯ-ರನ್ನೂ
ಸ್ತ್ರೀಯ-ರಲ್ಲಿ
ಸ್ತ್ರೀಯ-ರಿ-ಗಾಗಿ
ಸ್ತ್ರೀಯ-ರಿ-ಗಿಂತ
ಸ್ತ್ರೀಯ-ರಿಗೆ
ಸ್ತ್ರೀಯರು
ಸ್ತ್ರೀಯರೂ
ಸ್ತ್ರೀಯ-ರೊ-ಡನೆ
ಸ್ತ್ರೀಯಾ-ಗಿರ
ಸ್ತ್ರೀಯೂ
ಸ್ತ್ರೀವಿ-ದ್ಯಾ-ಭ್ಯಾಸ
ಸ್ತ್ರೀಸ-ಮು-ದಾ-ಯದ
ಸ್ತ್ರೀಸಿಂ-ಹಿ-ಣಿ-ಯಂ-ತಹ
ಸ್ಥಗಿ-ತ-ಗೊಂ-ಡಿದೆ
ಸ್ಥಗಿ-ತ-ಗೊ-ಳ್ಳ-ಲಿಲ್ಲ
ಸ್ಥಗಿ-ತ-ವಾ-ಗಿ-ಬಿ-ಟ್ಟಿತ್ತು
ಸ್ಥಳ
ಸ್ಥಳ-ಕ್ಕಿಂತ
ಸ್ಥಳಕ್ಕೂ
ಸ್ಥಳಕ್ಕೆ
ಸ್ಥಳ-ಗಳ
ಸ್ಥಳ-ಗಳನ್ನು
ಸ್ಥಳ-ಗಳನ್ನೂ
ಸ್ಥಳ-ಗಳನ್ನೆಲ್ಲ
ಸ್ಥಳ-ಗಳಲ್ಲಿ
ಸ್ಥಳ-ಗ-ಳ-ಲ್ಲಿ-ಅ-ದ-ರಲ್ಲೂ
ಸ್ಥಳ-ಗ-ಳಲ್ಲೂ
ಸ್ಥಳ-ಗ-ಳ-ಲ್ಲೆಲ್ಲ
ಸ್ಥಳ-ಗ-ಳಾ-ಗಿವೆ
ಸ್ಥಳ-ಗಳಿಂದ
ಸ್ಥಳ-ಗ-ಳಿಂ-ದಲೂ
ಸ್ಥಳ-ಗ-ಳಿಗೂ
ಸ್ಥಳ-ಗ-ಳಿಗೆ
ಸ್ಥಳದ
ಸ್ಥಳ-ದತ್ತ
ಸ್ಥಳ-ದಲ್ಲಿ
ಸ್ಥಳ-ದ-ಲ್ಲಿ-ರುವ
ಸ್ಥಳ-ದ-ಲ್ಲೀಗ
ಸ್ಥಳ-ದಲ್ಲೂ
ಸ್ಥಳ-ದಲ್ಲೇ
ಸ್ಥಳ-ದಿಂದ
ಸ್ಥಳ-ಪು-ರಾಣ
ಸ್ಥಳ-ಪು-ರಾ-ಣ-ವನ್ನು
ಸ್ಥಳ-ವನ್ನು
ಸ್ಥಳ-ವ-ನ್ನೆಲ್ಲ
ಸ್ಥಳ-ವನ್ನೇ
ಸ್ಥಳ-ವ-ಲ್ಲದೆ
ಸ್ಥಳ-ವಾ-ದ್ದ-ರಿಂ-ದಆ
ಸ್ಥಳ-ವಿತ್ತು
ಸ್ಥಳ-ವಿ-ದ್ದುದು
ಸ್ಥಳ-ವಿ-ರು-ವುದು
ಸ್ಥಳ-ವಿ-ಲ್ಲ-ದಂ-ತಾ-ಯಿತು
ಸ್ಥಳ-ವಿ-ಲ್ಲ-ದಂತೆ
ಸ್ಥಳವು
ಸ್ಥಳವೂ
ಸ್ಥಳ-ವೆಂದು
ಸ್ಥಳವೇ
ಸ್ಥಳ-ವೊಂ-ದ-ಕ್ಕಾಗಿ
ಸ್ಥಳ-ವೊಂ-ದ-ನ್ನ-ರಸಿ
ಸ್ಥಳ-ವೊಂ-ದನ್ನು
ಸ್ಥಳ-ವೊಂ-ದ-ರಲ್ಲಿ
ಸ್ಥಳ-ವೊಂ-ದಿ-ದ್ದರೆ
ಸ್ಥಳ-ವೊಂದು
ಸ್ಥಳಾಂ-ತ-ರಿ-ಸ-ಲಾ-ಯಿತು
ಸ್ಥಳಾ-ವ-ಕಾಶ
ಸ್ಥಳಾ-ವ-ಕಾ-ಶ-ವಿತ್ತು
ಸ್ಥಳೀಯ
ಸ್ಥಳೀ-ಯ-ನನ್ನು
ಸ್ಥಳೀ-ಯನೇ
ಸ್ಥಳೀ-ಯ-ರನ್ನು
ಸ್ಥಳೀ-ಯರು
ಸ್ಥವಿರ
ಸ್ಥಾನ
ಸ್ಥಾನಕ್ಕೂ
ಸ್ಥಾನಕ್ಕೆ
ಸ್ಥಾನ-ಕ್ಕೇ-ರಿಸಿ
ಸ್ಥಾನದ
ಸ್ಥಾನ-ದಲ್ಲಿ
ಸ್ಥಾನ-ದ-ಲ್ಲಿ-ದ್ದರೆ
ಸ್ಥಾನ-ದ-ಲ್ಲಿ-ದ್ದ-ವರೂ
ಸ್ಥಾನ-ದ-ಲ್ಲಿ-ದ್ದು-ಕೊಂಡು
ಸ್ಥಾನ-ದ-ಲ್ಲಿ-ರು-ವ-ವರು
ಸ್ಥಾನ-ದಿಂದ
ಸ್ಥಾನ-ಮಾನ
ಸ್ಥಾನ-ಮಾ-ನ-ವನ್ನು
ಸ್ಥಾನ-ಮಾನವೂ
ಸ್ಥಾನ-ವ-ನ್ನ-ಲಂ-ಕ-ರಿ-ಸಿದ್ದ
ಸ್ಥಾನ-ವನ್ನು
ಸ್ಥಾನ-ವನ್ನೂ
ಸ್ಥಾನ-ವಿದೆ
ಸ್ಥಾನ-ವಿ-ರು-ವು-ದಾ-ದರೆ
ಸ್ಥಾನ-ವಿ-ಲ್ಲ-ದೆ-ಯಿ-ರು-ತ್ತ-ದೆಯೆ
ಸ್ಥಾನಾ
ಸ್ಥಾಪ-ಕಾಯ
ಸ್ಥಾಪ-ಕಿ-ಯಾದ
ಸ್ಥಾಪ-ನಾ-ಕಾ-ರ್ಯ-ದಲ್ಲಿ
ಸ್ಥಾಪ-ನೆಗೆ
ಸ್ಥಾಪ-ನೆಯ
ಸ್ಥಾಪ-ನೆ-ಯಾ-ಗ-ಬೇಕು
ಸ್ಥಾಪ-ನೆ-ಯಾ-ಗಿ-ರು-ವುದು
ಸ್ಥಾಪ-ನೆ-ಯಾದ
ಸ್ಥಾಪ-ನೆ-ಯಾ-ದಾಗ
ಸ್ಥಾಪ-ನೆಯೂ
ಸ್ಥಾಪಿ-ತ-ವಾ-ಗಿದ್ದ
ಸ್ಥಾಪಿ-ತ-ವಾದ
ಸ್ಥಾಪಿ-ತ-ವಾ-ದದ್ದು
ಸ್ಥಾಪಿಸ
ಸ್ಥಾಪಿ-ಸ-ಬೇಕು
ಸ್ಥಾಪಿ-ಸ-ಬೇ-ಕೆಂದು
ಸ್ಥಾಪಿ-ಸ-ಲಾ-ಯಿತು
ಸ್ಥಾಪಿ-ಸ-ಲಿ-ರುವ
ಸ್ಥಾಪಿ-ಸಲು
ಸ್ಥಾಪಿ-ಸ-ಲ್ಪ-ಟ್ಟಿತು
ಸ್ಥಾಪಿಸಿ
ಸ್ಥಾಪಿ-ಸಿ-ಕೊಂ-ಡಿ-ದ್ದರು
ಸ್ಥಾಪಿ-ಸಿ-ಕೊ-ಳ್ಳು-ತ್ತಿ-ದ್ದಾರೆ
ಸ್ಥಾಪಿ-ಸಿದ
ಸ್ಥಾಪಿ-ಸಿ-ದರು
ಸ್ಥಾಪಿ-ಸಿ-ದಾಗ
ಸ್ಥಾಪಿ-ಸಿದ್ದ
ಸ್ಥಾಪಿ-ಸಿ-ದ್ದೇವೆ
ಸ್ಥಾಪಿ-ಸಿ-ಲಾ-ಗಿತ್ತು
ಸ್ಥಾಪಿಸು
ಸ್ಥಾಪಿ-ಸು-ತ್ತಿ-ದ್ದೇವೆ
ಸ್ಥಾಪಿ-ಸುವ
ಸ್ಥಾಪಿ-ಸು-ವಂ-ತಹ
ಸ್ಥಾಪಿ-ಸು-ವಂತೆ
ಸ್ಥಾಪಿ-ಸು-ವು-ದ-ಕ್ಕಾಗಿ
ಸ್ಥಾಪಿ-ಸು-ವು-ದಕ್ಕೂ
ಸ್ಥಾಪಿ-ಸು-ವು-ದರ
ಸ್ಥಾಪಿ-ಸು-ವುದು
ಸ್ಥಾಪಿ-ಸು-ವು-ದು-ಇವು
ಸ್ಥಾಪಿ-ಸು-ವುದೇ
ಸ್ಥಾಯಿತ್ವ
ಸ್ಥಾವ-ರ-ವಾದ
ಸ್ಥಿತಿ
ಸ್ಥಿತಿ-ಗ-ತಿ-ಸ್ಥಾ-ನ-ಮಾ-ನ-ಗಳು
ಸ್ಥಿತಿ-ಗ-ತಿ-ಗಳ
ಸ್ಥಿತಿ-ಗ-ತಿಯ
ಸ್ಥಿತಿಗೂ
ಸ್ಥಿತಿಗೆ
ಸ್ಥಿತಿ-ಗೇ-ರು-ತ್ತದೆ
ಸ್ಥಿತಿಯ
ಸ್ಥಿತಿ-ಯನ್ನು
ಸ್ಥಿತಿ-ಯಲ್ಲಿ
ಸ್ಥಿತಿ-ಯ-ಲ್ಲಿ-ದ್ದರು
ಸ್ಥಿತಿ-ಯ-ಲ್ಲಿ-ದ್ದು-ದ-ರಿಂದ
ಸ್ಥಿತಿ-ಯ-ಲ್ಲಿ-ರ-ಲಿಲ್ಲ
ಸ್ಥಿತಿ-ಯಲ್ಲೂ
ಸ್ಥಿತಿ-ಯಿಂದ
ಸ್ಥಿತಿಯು
ಸ್ಥಿತಿಯೇ
ಸ್ಥಿತಿ-ಯೇನೂ
ಸ್ಥಿರ
ಸ್ಥಿರ-ದೃಢ
ಸ್ಥಿರದಿ
ಸ್ಥಿರ-ವಾಗಿ
ಸ್ಥಿರ-ವಾ-ಗಿ-ಟ್ಟಿರಿ
ಸ್ಥಿರ-ವಾ-ಗು-ವುದನ್ನು
ಸ್ಥೂಲ
ಸ್ಥೂಲ-ಕಾ-ಯ-ದ-ವ-ನನ್ನು
ಸ್ಥೂಲ-ಕಾ-ಯರೇ
ಸ್ಥೂಲ-ವಾಗಿ
ಸ್ನಾನ
ಸ್ನಾನ-ಗೃ-ಹ-ಗಳು
ಸ್ನಾನದ
ಸ್ನೇಹ
ಸ್ನೇಹ-ಗಳು
ಸ್ನೇಹ-ಜೀ-ವಿಯೇ
ಸ್ನೇಹ-ದಿಂದ
ಸ್ನೇಹ-ಪ-ರ-ತೆ-ಯನ್ನು
ಸ್ನೇಹ-ಮಯ
ಸ್ನೇಹ-ವನ್ನು
ಸ್ನೇಹಿ
ಸ್ನೇಹಿತ
ಸ್ನೇಹಿ-ತನ
ಸ್ನೇಹಿ-ತ-ನನ್ನು
ಸ್ನೇಹಿ-ತ-ನಾದ
ಸ್ನೇಹಿ-ತನೂ
ಸ್ನೇಹಿ-ತ-ನೊಂ-ದಿಗೆ
ಸ್ನೇಹಿ-ತ-ನೊ-ಬ್ಬ-ನನ್ನು
ಸ್ನೇಹಿ-ತರ
ಸ್ನೇಹಿ-ತ-ರನ್ನು
ಸ್ನೇಹಿ-ತ-ರನ್ನೂ
ಸ್ನೇಹಿ-ತ-ರ-ನ್ನೆಲ್ಲ
ಸ್ನೇಹಿ-ತ-ರಾ-ಗಿ-ದ್ದರು
ಸ್ನೇಹಿ-ತ-ರಾದ
ಸ್ನೇಹಿ-ತ-ರಾ-ದರು
ಸ್ನೇಹಿ-ತ-ರಾರೂ
ಸ್ನೇಹಿ-ತ-ರಿಗೆ
ಸ್ನೇಹಿ-ತರು
ಸ್ನೇಹಿ-ತರೂ
ಸ್ನೇಹಿ-ತ-ರೆಲ್ಲ
ಸ್ನೇಹಿ-ತರೇ
ಸ್ನೇಹಿ-ತ-ರೊಂ-ದಿಗೆ
ಸ್ನೇಹಿ-ತ-ರೊ-ಬ್ಬರ
ಸ್ನೇಹಿತೆ
ಸ್ನೇಹಿ-ತೆ-ಯ-ರಿ-ಬ್ಬರೂ
ಸ್ನೇಹಿ-ತೆ-ಯರೂ
ಸ್ಪಂದಿ
ಸ್ಪಂದಿ-ಸು-ತ್ತಿತ್ತು
ಸ್ಪಂದಿ-ಸು-ತ್ತಿರ
ಸ್ಪಂದಿ-ಸು-ತ್ತಿ-ರು-ತ್ತದೆ
ಸ್ಪಂಧಿ-ಸು-ತ್ತಿ-ರು-ತ್ತವೆ
ಸ್ಪರ
ಸ್ಪರ್ಧಿಸಿ
ಸ್ಪರ್ಧೆ-ಹೋ-ರಾ-ಟ-ವ್ಯಾ-ಪಾ-ರ-ಬು-ದ್ಧಿ-ಗಳ
ಸ್ಪರ್ಧೆ-ಗಳು
ಸ್ಪರ್ಧೆ-ಯಿಂದ
ಸ್ಪರ್ಶ
ಸ್ಪರ್ಶ-ದಿಂದ
ಸ್ಪರ್ಶಿಸಿ
ಸ್ಪರ್ಶಿ-ಸಿ-ದರು
ಸ್ಪರ್ಶಿ-ಸಿ-ದಾಗ
ಸ್ಪಷ್ಟ
ಸ್ಪಷ್ಟ-ಗೊ-ಳಿ-ಸು-ತ್ತಿ-ದ್ದರು
ಸ್ಪಷ್ಟ-ಪ-ಡಿಸಿ
ಸ್ಪಷ್ಟ-ಪ-ಡಿ-ಸಿದ
ಸ್ಪಷ್ಟ-ಪ-ಡಿ-ಸಿ-ದರು
ಸ್ಪಷ್ಟ-ಪ-ಡಿ-ಸಿ-ದ್ದರು
ಸ್ಪಷ್ಟ-ಪ-ಡಿ-ಸು-ವಂ-ತಿತ್ತು
ಸ್ಪಷ್ಟ-ಪ-ರಿ-ಚಯ
ಸ್ಪಷ್ಟ-ಲ-ಕ್ಷಣ
ಸ್ಪಷ್ಟ-ವಾ-ಗ-ತೊ-ಡ-ಗಿತ್ತು
ಸ್ಪಷ್ಟ-ವಾಗಿ
ಸ್ಪಷ್ಟ-ವಾ-ಗಿತ್ತು
ಸ್ಪಷ್ಟ-ವಾ-ಗಿಯೇ
ಸ್ಪಷ್ಟ-ವಾಗು
ಸ್ಪಷ್ಟ-ವಾ-ಗು-ತ್ತದೆ
ಸ್ಪಷ್ಟ-ವಾ-ಗು-ತ್ತ-ದೆ-ಅ-ವನ
ಸ್ಪಷ್ಟ-ವಾದ
ಸ್ಪಷ್ಟ-ವಾ-ಯಿತು
ಸ್ಪೂರ್ತಿ-ಯುತ
ಸ್ಪೂರ್ತಿ-ಯು-ತ-ರಾಗಿ
ಸ್ಪೆನ್ಸ-ರನ
ಸ್ಪೆನ್ಸ-ರಳ
ಸ್ಪೆನ್ಸರ್
ಸ್ಪೈನ್-ಜ-ರ್ಮ-ನಿಯೇ
ಸ್ಪ್ಯಾನಿಷ್
ಸ್ಫಟಿ-ಕ-ದಷ್ಟು
ಸ್ಫಟಿ-ಕ-ವ-ರ್ಣದ
ಸ್ಫಿಂಕ್ಸ್
ಸ್ಫುಟ-ಗೊಂ-ಡಿತು
ಸ್ಫುಟ-ವಾಗಿ
ಸ್ಫುಟ-ವಾ-ಗಿತ್ತು
ಸ್ಫುರ-ಣೆ-ಗೊ-ಳಿ-ಸು-ವುದೋ
ಸ್ಫುರಿ-ಸ-ದಿ-ದ್ದರೆ
ಸ್ಫುರಿ-ಸ-ಬಲ್ಲ
ಸ್ಫುರಿಸಿ
ಸ್ಫುರಿ-ಸಿತು
ಸ್ಫುರಿ-ಸು-ತ್ತಿತ್ತು
ಸ್ಫೂರ್ತಿ
ಸ್ಫೂರ್ತಿ-ಉ-ತ್ಸಾ-ಹ-ದಾ-ಯ-ಕ-ವಾ-ಗಿ-ರ-ಲೇ-ಬೇಕು
ಸ್ಫೂರ್ತಿ-ಚೈ-ತನ್ಯ
ಸ್ಫೂರ್ತಿ-ಹು-ಮ್ಮಸ್ಸು
ಸ್ಫೂರ್ತಿ-ಎ-ಲ್ಲವೂ
ಸ್ಫೂರ್ತಿ-ಗಳ
ಸ್ಫೂರ್ತಿ-ಗೊಂಡ
ಸ್ಫೂರ್ತಿ-ಗೊಂ-ಡಿ-ದ್ದರು
ಸ್ಫೂರ್ತಿ-ಗೊ-ಳಿ-ಸಲು
ಸ್ಫೂರ್ತಿ-ಗೊ-ಳಿ-ಸಿ-ದರು
ಸ್ಫೂರ್ತಿ-ಗೊ-ಳಿ-ಸಿ-ದ್ದೀರಿ
ಸ್ಫೂರ್ತಿ-ಗೊ-ಳಿ-ಸುತ್ತ
ಸ್ಫೂರ್ತಿ-ಗೊ-ಳಿ-ಸು-ತ್ತಿ-ದ್ದರು
ಸ್ಫೂರ್ತಿ-ದಾ-ಯಕ
ಸ್ಫೂರ್ತಿ-ದಾ-ಯ-ಕ-ವಾದ
ಸ್ಫೂರ್ತಿ-ದಾ-ಯ-ಕವೂ
ಸ್ಫೂರ್ತಿ-ಧಾರೆ
ಸ್ಫೂರ್ತಿ-ಪ-ಡೆದ
ಸ್ಫೂರ್ತಿ-ಭ-ರಿ-ತ-ರಾಗಿ
ಸ್ಫೂರ್ತಿಯ
ಸ್ಫೂರ್ತಿ-ಯನ್ನು
ಸ್ಫೂರ್ತಿ-ಯನ್ನೂ
ಸ್ಫೂರ್ತಿ-ಯಿಂದ
ಸ್ಫೂರ್ತಿ-ಯಿಂ-ದಾಗಿ
ಸ್ಫೂರ್ತಿಯು
ಸ್ಫೂರ್ತಿ-ಯುತ
ಸ್ಫೂರ್ತಿ-ಯು-ತ-ವಾಗಿ
ಸ್ಫೂರ್ತಿ-ಯು-ತ-ವಾದ
ಸ್ಫೂರ್ತಿ-ಯೊಂ-ದನ್ನು
ಸ್ಫೋಟ-ಗೊಂಡು
ಸ್ಫೋಟಿಸಿ
ಸ್ಮರ-ಣ-ಶ-ಕ್ತಿ-ಗಳನ್ನು
ಸ್ಮರ-ಣ-ಶ-ಕ್ತಿ-ಯಾ-ಗಲಿ
ಸ್ಮರ-ಣಾ-ರ್ಥ-ವಾಗಿ
ಸ್ಮರ-ಣೀ-ಯ-ವಾ-ಗು-ವಂ-ತಾ-ಯಿತು
ಸ್ಮರ-ಣೀ-ಯ-ವಾ-ದದ್ದು
ಸ್ಮರ-ಣೆ-ಗಳನ್ನು
ಸ್ಮರ-ಣೆಗೆ
ಸ್ಮರ-ಣೆ-ಯಲ್ಲಿ
ಸ್ಮರಿ-ಸ-ಬ-ಹುದು
ಸ್ಮರಿಸಿ
ಸ್ಮರಿ-ಸಿ-ಕೊಳ್ಳ
ಸ್ಮರಿ-ಸುತ್ತ
ಸ್ಮರಿ-ಸು-ತ್ತಾರೆ
ಸ್ಮರಿ-ಸು-ವರು
ಸ್ಮಶಾ-ನದ
ಸ್ಮಶಾ-ನ-ವಾಗಿ
ಸ್ಮಶಾ-ನ-ವೆಂ-ಬುದು
ಸ್ಮಾರ-ಕ-ವಾಗಿ
ಸ್ಮಾೈಲ್ಸ್
ಸ್ಮೃತಿ
ಸ್ಮೃತಿ-ಪು-ರಾ-ಣ-ಗಳಲ್ಲಿ
ಸ್ಮೃತಿ-ಗಳನ್ನು
ಸ್ಮೃತಿ-ಗ-ಳವು
ಸ್ಮೃತಿ-ಗಳು
ಸ್ಮೃತಿ-ಗ-ಳೆಂ-ದರೆ
ಸ್ಮೃತಿಯ
ಸ್ಮೃತಿಯು
ಸ್ಯತ್ರಾ-ಯತೇ
ಸ್ಯಾನ್
ಸ್ಯಾನ್ಫ್ರಾ-ನ್ಸಿಸ್ಕೋ
ಸ್ಯಾನ್ಫ್ರಾ-ನ್ಸಿ-ಸ್ಕೋ-ಗಳಲ್ಲಿ
ಸ್ಯಾನ್ಫ್ರಾ-ನ್ಸಿ-ಸ್ಕೋಗೆ
ಸ್ಯಾನ್ಫ್ರಾ-ನ್ಸಿ-ಸ್ಕೋದ
ಸ್ಯಾನ್ಫ್ರಾ-ನ್ಸಿ-ಸ್ಕೋ-ದಲ್ಲಿ
ಸ್ಯಾನ್ಫ್ರಾ-ನ್ಸಿ-ಸ್ಕೋ-ದಿಂದ
ಸ್ಯಾಮ್ಯು-ಯೆಲ್
ಸ್ಯೂಯೆಜ್
ಸ್ವಂತ
ಸ್ವಂತ-ದವ
ಸ್ವಂತಿ-ಕೆ-ಯಿತ್ತು
ಸ್ವಂತಿ-ಕೆಯೂ
ಸ್ವಇಚ್ಛೆ
ಸ್ವಇ-ಚ್ಛೆ-ಯಿಂ-ದಲೇ
ಸ್ವಗ-ತವೋ
ಸ್ವಗ್ರಾ-ಮ-ವಾದ
ಸ್ವಚ್ಛ-ವಾಗಿ
ಸ್ವಚ್ಛ-ವಾದ
ಸ್ವತಂತ್ರ
ಸ್ವತಂ-ತ್ರ-ನಾ-ಗಿ-ದ್ದೇನೆ
ಸ್ವತಂ-ತ್ರ-ರಾ-ಗು-ವ-ವ-ರೆಗೂ
ಸ್ವತಂ-ತ್ರಳು
ಸ್ವತಂ-ತ್ರ-ವಾಗಿ
ಸ್ವತಂ-ತ್ರ-ವಾ-ಗಿಯೂ
ಸ್ವತಂ-ತ್ರ-ವಾ-ಗಿ-ರಲು
ಸ್ವತಂ-ತ್ರ-ವಾದ
ಸ್ವತಃ
ಸ್ವತಃ-ಸಿ-ದ್ಧ-ವಾ-ಗಿ-ರು-ತ್ತದೆ
ಸ್ವತ-ಸ್ಸಿದ್ಧ
ಸ್ವತ್ವ
ಸ್ವದೇಶ
ಸ್ವದೇ-ಶ-ಪ್ರೇ-ಮ-ದಿಂದ
ಸ್ವದೇಶೀ
ಸ್ವಧ-ರ್ಮಾಭಿ
ಸ್ವಪ್ನ-ದಲ್ಲಿ
ಸ್ವಭಾವ
ಸ್ವಭಾ-ವ-ಇ-ವು-ಗ-ಳಿಂ-ದಾಗಿ
ಸ್ವಭಾ-ವಕ್ಕೆ
ಸ್ವಭಾ-ವ-ಗಳ
ಸ್ವಭಾ-ವ-ಗಳನ್ನು
ಸ್ವಭಾ-ವತಃ
ಸ್ವಭಾ-ವದ
ಸ್ವಭಾ-ವ-ದ-ವ-ರಾದ
ಸ್ವಭಾ-ವ-ವ-ನ್ನ-ರಿ-ತಿದ್ದ
ಸ್ವಭಾ-ವ-ವನ್ನು
ಸ್ವಭಾ-ವ-ವಾ-ಗಿತ್ತು
ಸ್ವಭಾ-ವ-ವಿದೆ
ಸ್ವಭಾ-ವ-ವೆಂದು
ಸ್ವಭಾ-ವವೇ
ಸ್ವಭಾ-ವ-ಸ-ಹಜ
ಸ್ವಭಾ-ವ-ಸ-ಹ-ಜ-ವಾದ
ಸ್ವಯಂ
ಸ್ವಯಂ-ಪ್ರೇ-ರಿತ
ಸ್ವಯಂ-ಸೇ-ವ-ಕರ
ಸ್ವಯಂ-ಸೇ-ವ-ಕರು
ಸ್ವಯಂ-ಸ್ಫೂ-ರ್ತಿ-ಯಿಂದ
ಸ್ವರ
ಸ್ವರ-ಗ-ಳೆಲ್ಲ
ಸ್ವರ-ದಲ್ಲಿ
ಸ್ವರ-ದಿಂದ
ಸ್ವರ-ಬ-ದ್ಧ-ವಾಗಿ
ಸ್ವರ-ಸಂ-ಗ-ಮ-ಅ-ದೊಂದು
ಸ್ವರೂ
ಸ್ವರೂಪ
ಸ್ವರೂ-ಪ-ಇ-ವೆ-ಲ್ಲ-ವನ್ನೂ
ಸ್ವರೂ-ಪದ
ಸ್ವರೂ-ಪ-ನಂ-ತಹ
ಸ್ವರೂ-ಪ-ರೆಂದು
ಸ್ವರೂ-ಪ-ವನ್ನು
ಸ್ವರೂ-ಪ-ವಲ್ಲ
ಸ್ವರೂ-ಪಾ-ನಂದ
ಸ್ವರೂ-ಪಾ-ನಂ-ದ-ರನ್ನು
ಸ್ವರೂ-ಪಾ-ನಂ-ದ-ರಿಗೆ
ಸ್ವರೂ-ಪಾ-ನಂ-ದರು
ಸ್ವರೂ-ಪಾ-ನಂ-ದ-ರೊ-ಡನೆ
ಸ್ವರೂ-ಪಿ-ಗ-ಳಾಗಿ
ಸ್ವರೂ-ಪಿ-ಗಳು
ಸ್ವರೂ-ಪಿ-ಣಿ-ಯಾ-ಗಿ-ದ್ದಾಳೆ
ಸ್ವರ್ಗ
ಸ್ವರ್ಗಕ್ಕೆ
ಸ್ವರ್ಗ-ರಾ-ಜ್ಯದ
ಸ್ವರ್ಗ-ರಾ-ಜ್ಯ-ವೆಂದರೆ
ಸ್ವರ್ಗ-ವನ್ನು
ಸ್ವರ್ಗ-ವನ್ನೇ
ಸ್ವರ್ಗ-ವಾ-ಗ-ಲಾ-ರ-ದೆಂ
ಸ್ವರ್ಗ-ವೆಂದರೆ
ಸ್ವರ್ಗವೇ
ಸ್ವರ್ಣ-ದೇ-ವಾ-ಲ-ಯ-ವನ್ನೂ
ಸ್ವಲ್ಪ
ಸ್ವಲ್ಪ-ಕಾಲ
ಸ್ವಲ್ಪ-ಮಾತ್ರ
ಸ್ವಲ್ಪ-ವಾ-ದರೂ
ಸ್ವಲ್ಪವೂ
ಸ್ವಲ್ಪವೇ
ಸ್ವಲ್ಪ-ಸ್ವಲ್ಪ
ಸ್ವಲ್ಪ-ಸ್ವ-ಲ್ಪ-ವನ್ನೇ
ಸ್ವಲ್ಪ-ಹೊ-ತ್ತಿನ
ಸ್ವಲ್ಪ-ಹೊ-ತ್ತಿ-ನಲ್ಲೇ
ಸ್ವಲ್ಪ-ಹೊತ್ತು
ಸ್ವಸಂ-ತೋ-ಷ-ಕ್ಕಾಗಿ
ಸ್ವಸಂ-ರ-ಕ್ಷ-ಣಾ-ರ್ಥ-ವಾಗಿ
ಸ್ವಸ-ಹಾಯ
ಸ್ವಸು-ಖ-ವನ್ನು
ಸ್ವಸ್ತಿ
ಸ್ವಸ್ಥ-ವಾಗಿ
ಸ್ವಹಿತ
ಸ್ವಾಗತ
ಸ್ವಾಗ-ತ-ಆ-ದ-ರ-ಪೂ-ಜ್ಯತೆ
ಸ್ವಾಗ-ತ-ಪರಿ
ಸ್ವಾಗ-ತ-ಕ್ಕಾಗಿ
ಸ್ವಾಗ-ತಕ್ಕೂ
ಸ್ವಾಗ-ತಕ್ಕೆ
ಸ್ವಾಗ-ತದ
ಸ್ವಾಗ-ತ-ದೊಂ-ದಿಗೆ
ಸ್ವಾಗ-ತ-ವನ್ನು
ಸ್ವಾಗ-ತ-ವನ್ನೇ
ಸ್ವಾಗ-ತ-ವಾದ
ಸ್ವಾಗ-ತ-ವಿದೆ
ಸ್ವಾಗ-ತವು
ಸ್ವಾಗತಿ
ಸ್ವಾಗ-ತಿ-ಸ-ಲಾ-ಯಿತು
ಸ್ವಾಗ-ತಿ-ಸಲು
ಸ್ವಾಗ-ತಿ-ಸಲೋ
ಸ್ವಾಗ-ತಿಸಿ
ಸ್ವಾಗ-ತಿ-ಸಿತು
ಸ್ವಾಗ-ತಿ-ಸಿದ
ಸ್ವಾಗ-ತಿ-ಸಿ-ದಂ-ತಾ-ಗ-ಲಿಲ್ಲ
ಸ್ವಾಗ-ತಿ-ಸಿ-ದ-ರ-ಲ್ಲದೆ
ಸ್ವಾಗ-ತಿ-ಸಿ-ದರು
ಸ್ವಾಗ-ತಿ-ಸಿ-ದುವು
ಸ್ವಾಗ-ತಿ-ಸುತ್ತ
ಸ್ವಾಗ-ತಿ-ಸು-ತ್ತದೆ
ಸ್ವಾಗ-ತಿ-ಸು-ತ್ತೇವೆ
ಸ್ವಾಗ-ತಿ-ಸುವ
ಸ್ವಾಗ-ತಿ-ಸು-ವು-ದೆಂ-ಬು-ದನ್ನು
ಸ್ವಾತಂತ್ರ್ಯ
ಸ್ವಾತಂ-ತ್ರ್ಯದ
ಸ್ವಾತಂ-ತ್ರ್ಯ-ದಲ್ಲಿ
ಸ್ವಾತಂ-ತ್ರ್ಯ-ದಿಂದ
ಸ್ವಾತಂ-ತ್ರ್ಯ-ದಿ-ನದ
ಸ್ವಾತಂ-ತ್ರ್ಯ-ದೆ-ಡೆಗೆ
ಸ್ವಾತಂ-ತ್ರ್ಯ-ಪೂ-ರ್ವ-ದಲ್ಲಿ
ಸ್ವಾತಂ-ತ್ರ್ಯ-ವನ್ನು
ಸ್ವಾತಂ-ತ್ರ್ಯ-ವನ್ನೂ
ಸ್ವಾತಂ-ತ್ರ್ಯ-ವಿದೆ
ಸ್ವಾತಂ-ತ್ರ್ಯ-ವಿ-ದ್ದರೆ
ಸ್ವಾತಂ-ತ್ರ್ಯ-ವಿ-ರು-ತ್ತದೆ
ಸ್ವಾತಂ-ತ್ರ್ಯ-ವೆಂದರೆ
ಸ್ವಾತಂ-ತ್ರ್ಯವೇ
ಸ್ವಾತಂ-ತ್ರ್ಯ-ವೊಂ-ದಿ-ದ್ದರೆ
ಸ್ವಾಧೀ-ನ-ತೆ-ಯನ್ನು
ಸ್ವಾಧ್ಯಾ-ಯ-ಪ್ರ-ವ-ಚ-ನ-ಗಳಲ್ಲಿ
ಸ್ವಾಧ್ಯಾ-ಯ-ಪ್ರ-ವ-ಚ-ನ-ಗ-ಳಿಗೆ
ಸ್ವಾಭಾ-ವಿಕ
ಸ್ವಾಭಾ-ವಿ-ಕ-ವಾ-ಗಿಯೇ
ಸ್ವಾಭಾ-ವಿ-ಕ-ವಾದ
ಸ್ವಾಭಿ-ಮಾನ
ಸ್ವಾಭಿ-ಮಾ-ನ-ವನ್ನು
ಸ್ವಾಮಿ
ಸ್ವಾಮಿ-ಶಿಷ್ಯ
ಸ್ವಾಮಿ-ಗಳ
ಸ್ವಾಮಿ-ಗಳನ್ನು
ಸ್ವಾಮಿ-ಗಳನ್ನೆಲ್ಲ
ಸ್ವಾಮಿ-ಗ-ಳಾದ
ಸ್ವಾಮಿ-ಗಳಿ
ಸ್ವಾಮಿ-ಗ-ಳಿ-ಗಾದ
ಸ್ವಾಮಿ-ಗ-ಳಿಗೆ
ಸ್ವಾಮಿ-ಗ-ಳಿ-ದ್ದರು
ಸ್ವಾಮಿ-ಗ-ಳಿ-ಬ್ಬ-ರಿಗೂ
ಸ್ವಾಮಿ-ಗಳು
ಸ್ವಾಮಿ-ಗಳೂ
ಸ್ವಾಮಿ-ಗ-ಳೊಂ-ದಿಗೆ
ಸ್ವಾಮಿ-ಗ-ಳೊ-ಬ್ಬರ
ಸ್ವಾಮಿ-ಗ-ಳೊ-ಬ್ಬ-ರನ್ನು
ಸ್ವಾಮಿ-ಗ-ಳೊ-ಬ್ಬ-ರಿ-ಗಾಗಿ
ಸ್ವಾಮಿ-ಗ-ಳೊ-ಬ್ಬರು
ಸ್ವಾಮಿ-ಜಿ-ಯ-ವರ
ಸ್ವಾಮಿ-ಜಿ-ಯ-ವ-ರಿಗೆ
ಸ್ವಾಮಿ-ಯಾದೆ
ಸ್ವಾಮೀ
ಸ್ವಾಮೀಜಿ
ಸ್ವಾಮೀ-ಜಿ-ಗ-ಳ-ನ್ನೊ-ಳ-ಗೊಂಡ
ಸ್ವಾಮೀ-ಜಿಗೆ
ಸ್ವಾಮೀ-ಜಿಯ
ಸ್ವಾಮೀ-ಜಿ-ಯನ್ನು
ಸ್ವಾಮೀ-ಜಿ-ಯರ
ಸ್ವಾಮೀ-ಜಿ-ಯವ
ಸ್ವಾಮೀ-ಜಿ-ಯ-ವರ
ಸ್ವಾಮೀ-ಜಿ-ಯ-ವ-ರಂತೂ
ಸ್ವಾಮೀ-ಜಿ-ಯ-ವ-ರ-ದಲ್ಲ
ಸ್ವಾಮೀ-ಜಿ-ಯ-ವ-ರದು
ಸ್ವಾಮೀ-ಜಿ-ಯ-ವ-ರದೇ
ಸ್ವಾಮೀ-ಜಿ-ಯ-ವ-ರ-ದೊಂದು
ಸ್ವಾಮೀ-ಜಿ-ಯ-ವ-ರನ್ನು
ಸ್ವಾಮೀ-ಜಿ-ಯ-ವ-ರನ್ನೂ
ಸ್ವಾಮೀ-ಜಿ-ಯ-ವ-ರನ್ನೇ
ಸ್ವಾಮೀ-ಜಿ-ಯ-ವ-ರ-ನ್ನೊಮ್ಮೆ
ಸ್ವಾಮೀ-ಜಿ-ಯ-ವ-ರಲ್ಲಿ
ಸ್ವಾಮೀ-ಜಿ-ಯ-ವ-ರಾ-ಗಲಿ
ಸ್ವಾಮೀ-ಜಿ-ಯ-ವರಿ
ಸ್ವಾಮೀ-ಜಿ-ಯ-ವ-ರಿಂದ
ಸ್ವಾಮೀ-ಜಿ-ಯ-ವ-ರಿ-ಗಾಗಿ
ಸ್ವಾಮೀ-ಜಿ-ಯ-ವ-ರಿ-ಗಾದ
ಸ್ವಾಮೀ-ಜಿ-ಯ-ವ-ರಿ-ಗಾ-ದದ್ದು
ಸ್ವಾಮೀ-ಜಿ-ಯ-ವ-ರಿ-ಗಿಂ-ತಲೂ
ಸ್ವಾಮೀ-ಜಿ-ಯ-ವ-ರಿ-ಗಿತ್ತು
ಸ್ವಾಮೀ-ಜಿ-ಯ-ವ-ರಿ-ಗಿದ್ದ
ಸ್ವಾಮೀ-ಜಿ-ಯ-ವ-ರಿ-ಗಿ-ರುವ
ಸ್ವಾಮೀ-ಜಿ-ಯ-ವ-ರಿಗೂ
ಸ್ವಾಮೀ-ಜಿ-ಯ-ವ-ರಿಗೆ
ಸ್ವಾಮೀ-ಜಿ-ಯ-ವ-ರಿ-ಗೆಂತು
ಸ್ವಾಮೀ-ಜಿ-ಯ-ವ-ರಿಗೇ
ಸ್ವಾಮೀ-ಜಿ-ಯ-ವ-ರಿ-ಗೇನೂ
ಸ್ವಾಮೀ-ಜಿ-ಯ-ವ-ರಿ-ಗೊಂದು
ಸ್ವಾಮೀ-ಜಿ-ಯ-ವ-ರಿದ್ದ
ಸ್ವಾಮೀ-ಜಿ-ಯ-ವ-ರಿ-ಲ್ಲ-ದಿದ್ದ
ಸ್ವಾಮೀ-ಜಿ-ಯ-ವ-ರೀಗ
ಸ್ವಾಮೀ-ಜಿ-ಯ-ವರು
ಸ್ವಾಮೀ-ಜಿ-ಯ-ವ-ರು-ಅ-ವರೇ
ಸ್ವಾಮೀ-ಜಿ-ಯ-ವರೂ
ಸ್ವಾಮೀ-ಜಿ-ಯ-ವ-ರೆಂ-ದಿಗೂ
ಸ್ವಾಮೀ-ಜಿ-ಯ-ವರೇ
ಸ್ವಾಮೀ-ಜಿ-ಯ-ವ-ರೊಂ-ದಿ-ಗಿದ್ದ
ಸ್ವಾಮೀ-ಜಿ-ಯ-ವ-ರೊಂ-ದಿ-ಗಿನ
ಸ್ವಾಮೀ-ಜಿ-ಯ-ವ-ರೊಂ-ದಿ-ಗಿನ್ನೂ
ಸ್ವಾಮೀ-ಜಿ-ಯ-ವ-ರೊಂ-ದಿಗೆ
ಸ್ವಾಮೀ-ಜಿ-ಯ-ವ-ರೊ-ಡನೆ
ಸ್ವಾಮೀ-ಜಿಯು
ಸ್ವಾಮೀ-ಜಿ-ಯೆಂ-ದರು
ಸ್ವಾಮೀ-ಜಿ-ಹಾಗೆ
ಸ್ವಾಮ್ಯ-ವಿದೆ
ಸ್ವಾರ-ಸ್ಯ-ಕರ
ಸ್ವಾರ-ಸ್ಯ-ಕ-ರ-ವಾ-ಗಿದೆ
ಸ್ವಾರ-ಸ್ಯ-ಕ-ರ-ವಾ-ಗಿ-ದೆ-ಯ-ಲ್ಲವೆ
ಸ್ವಾರ-ಸ್ಯ-ಕ-ರ-ವಾದ
ಸ್ವಾರ-ಸ್ಯದ
ಸ್ವಾರ-ಸ್ಯ-ವಾಗಿ
ಸ್ವಾರ-ಸ್ಯ-ವಾ-ಗಿ-ರು-ತ್ತದೆ
ಸ್ವಾರ-ಸ್ಯ-ವಿದೆ
ಸ್ವಾರ-ಸ್ಯ-ವೇನು
ಸ್ವಾರ್ಥ
ಸ್ವಾರ್ಥ-ಕ್ಕಾಗಿ
ಸ್ವಾರ್ಥತೆ
ಸ್ವಾರ್ಥ-ತೆ-ಯಿಂದ
ಸ್ವಾರ್ಥದ
ಸ್ವಾರ್ಥ-ಪ್ರೇ-ರಿ-ತ-ವಾ-ದದ್ದು
ಸ್ವಾರ್ಥ-ಬು-ದ್ಧಿ-ಯನ್ನು
ಸ್ವಾರ್ಥ-ಮು-ಖದ
ಸ್ವಾರ್ಥ-ವನ್ನು
ಸ್ವಾರ್ಥಿ-ಗಳೇ
ಸ್ವಾರ್ಥೋ-ದ್ದೇಶ
ಸ್ವಾವ-ಲಂ-ಬನಂ
ಸ್ವಾವ-ಲಂಬಿ
ಸ್ವಾವ-ಲಂ-ಬಿ-ಗ-ಳ-ನ್ನಾ-ಗಿ-ಸು-ವು-ದ-ಕ್ಕಾಗಿ
ಸ್ವಾಹಾ
ಸ್ವಾಹಾ-ಕಾರ
ಸ್ವಿಟ್ಸರ್
ಸ್ವಿಟ್ಸ-ರ್ಲ್ಯಾಂ-ಡ್ಗಳನ್ನು
ಸ್ವೀಕ-ರಿಸ
ಸ್ವೀಕ-ರಿ-ಸ-ತ-ಕ್ಕದ್ದು
ಸ್ವೀಕ-ರಿ-ಸ-ದಿ-ದ್ದು-ದ-ರಿಂ-ದಲೇ
ಸ್ವೀಕ-ರಿ-ಸದೆ
ಸ್ವೀಕ-ರಿ-ಸ-ಬಲ್ಲ
ಸ್ವೀಕ-ರಿ-ಸ-ಬ-ಲ್ಲ-ಧ-ರ್ಮ-ವೆಂದು
ಸ್ವೀಕ-ರಿ-ಸ-ಬ-ಲ್ಲುದು
ಸ್ವೀಕ-ರಿ-ಸ-ಬ-ಹುದು
ಸ್ವೀಕ-ರಿ-ಸ-ಬೇ-ಕೆಂದು
ಸ್ವೀಕ-ರಿ-ಸ-ಬೇ-ಕೆಂಬ
ಸ್ವೀಕ-ರಿ-ಸಲು
ಸ್ವೀಕ-ರಿಸಿ
ಸ್ವೀಕ-ರಿ-ಸಿತು
ಸ್ವೀಕ-ರಿ-ಸಿದ
ಸ್ವೀಕ-ರಿ-ಸಿ-ದರು
ಸ್ವೀಕ-ರಿ-ಸಿ-ದ-ರೆಂದು
ಸ್ವೀಕ-ರಿ-ಸಿ-ದರೋ
ಸ್ವೀಕ-ರಿ-ಸಿ-ದ-ವರು
ಸ್ವೀಕ-ರಿ-ಸಿ-ದೆವು
ಸ್ವೀಕ-ರಿ-ಸಿ-ದ್ದರ
ಸ್ವೀಕ-ರಿ-ಸಿ-ದ್ದರು
ಸ್ವೀಕ-ರಿ-ಸಿ-ದ್ದಾಳೋ
ಸ್ವೀಕ-ರಿ-ಸಿದ್ದು
ಸ್ವೀಕ-ರಿ-ಸಿ-ದ್ದೇನೆ
ಸ್ವೀಕ-ರಿ-ಸಿ-ಯಾ-ರೆಂಬ
ಸ್ವೀಕ-ರಿ-ಸಿ-ಯಾ-ರೆಂ-ಬು-ದರ
ಸ್ವೀಕ-ರಿಸು
ಸ್ವೀಕ-ರಿ-ಸುತ್ತ
ಸ್ವೀಕ-ರಿ-ಸು-ತ್ತಿದ್ದ
ಸ್ವೀಕ-ರಿ-ಸು-ತ್ತಿ-ದ್ದಾರೆ
ಸ್ವೀಕ-ರಿ-ಸು-ತ್ತೇ-ನೆಂ-ಬು-ದನ್ನು
ಸ್ವೀಕ-ರಿ-ಸುವ
ಸ್ವೀಕ-ರಿ-ಸು-ವಂ-ತಾ-ಗ-ಬೇಕಾ
ಸ್ವೀಕ-ರಿ-ಸು-ವಂ-ತಾ-ಗ-ಬೇಕು
ಸ್ವೀಕ-ರಿ-ಸು-ವಂತೆ
ಸ್ವೀಕ-ರಿ-ಸು-ವ-ರೆಂದು
ಸ್ವೀಕ-ರಿ-ಸು-ವರೋ
ಸ್ವೀಕ-ರಿ-ಸು-ವು-ದಿಲ್ಲ
ಸ್ವೀಕ-ರಿ-ಸು-ವುದು
ಸ್ವೀಕಾರ
ಸ್ವೀಕಾ-ರಕ್ಕೆ
ಸ್ವೀಕಾ-ರದ
ಸ್ವೀಕಾ-ರಾ-ರ್ಹ-ವೆ-ನಿ-ಸು-ತ್ತದೆ
ಸ್ವೇಚ್ಛಾ
ಸ್ವೇಚ್ಛಾ-ಚಾ-ರಿ-ಗಳು
ಸ್ವೇಚ್ಛೆ-ಯಾಗಿ
ಸ್ವೇಚ್ಛೆ-ಯಿಂದ
ಸ್ಸಿಲ್ಲ
ಹಂ
ಹಂಗಾ-ದರೂ
ಹಂಗಿ-ಸುವ
ಹಂಗು
ಹಂಚಿ
ಹಂಚಿಕೆ
ಹಂಚಿ-ಕೊಂಡು
ಹಂಚಿ-ಕೊ-ಡ-ದಿ-ದ್ದರೆ
ಹಂಚಿ-ದರು
ಹಂಚಿ-ದ್ದರು
ಹಂಚಿ-ಬಿ-ಡ-ಬಾ-ರದು
ಹಂಚು-ತ್ತಿ-ದ್ದರು
ಹಂತ
ಹಂತ-ಗಳಲ್ಲಿ
ಹಂತ-ಗ-ಳಿಗೂ
ಹಂತ-ದಲ್ಲೂ
ಹಂತ-ದ-ವ-ರೆಗೆ
ಹಂತ-ದಿಂ-ದಲೂ
ಹಂತ-ವನ್ನು
ಹಂತ-ವಾಗಿ
ಹಂದಿ
ಹಂದಿಯು
ಹಂಪಲು
ಹಂಬಲ
ಹಂಬ-ಲ-ದಿಂ-ದಲೇ
ಹಂಬ-ಲ-ವ-ನ್ನಾ-ದರೂ
ಹಂಬ-ಲ-ವ-ನ್ನುಂ-ಟು-ಮಾ-ಡುವ
ಹಂಬ-ಲ-ವುಂ-ಟಾ-ಯಿತು
ಹಂಬ-ಲಿಸಿ
ಹಂಬ-ಲಿ-ಸಿ-ಯಾ-ರೆಂದು
ಹಂಸ
ಹಂಸ-ತೂ-ಲಿ-ಕಾ-ತಲ್ಪ
ಹಕ್ಕನ್ನು
ಹಕ್ಕನ್ನೂ
ಹಕ್ಕಿ
ಹಕ್ಕಿ-ಗಳ
ಹಕ್ಕಿ-ಗಳಿಂದ
ಹಕ್ಕಿದೆ
ಹಕ್ಕಿ-ಯಂ-ತೆಯೇ
ಹಕ್ಕಿ-ರು-ವು-ದಾ-ದರೆ
ಹಕ್ಕಿಲ್ಲ
ಹಕ್ಕು
ಹಕ್ಕು-ಗಳನ್ನು
ಹಕ್ಕು-ಗಳನ್ನೂ
ಹಕ್ಕು-ಗಳೂ
ಹಕ್ಕು-ಬಾ-ಧ್ಯ-ತೆ-ಗ-ಳಿ-ರು-ವಂ-ತಹ
ಹಗ-ಲಿ-ನಷ್ಟು
ಹಗ-ಲಿ-ರುಳು
ಹಗ-ಲಿ-ರುಳೂ
ಹಗ-ಲಿ-ರು-ಳೆ-ನ್ನದೆ
ಹಗಲು
ಹಗ-ಲು-ಇ-ರು-ಳು-ಗಳ
ಹಗಲೂ
ಹಗುರ
ಹಗು-ರ-ವಾ-ಗು-ತ್ತದೆ
ಹಗು-ರ-ವಾದ
ಹಗು-ರ-ವೆ-ನಿ-ಸಿತು
ಹಚ್ಚ
ಹಚ್ಚ-ಹ-ಸಿರು
ಹಚ್ಚಿ-ಕೊಂಡು
ಹಚ್ಚಿ-ಕೊ-ಳ್ಳು-ವ-ವ-ಳಲ್ಲ
ಹಚ್ಚಿ-ಡ-ಲಾ-ಯಿತು
ಹಚ್ಚಿ-ದರು
ಹಚ್ಚಿ-ಸಿ-ಕೊಂಡು
ಹಚ್ಚುತ್ತ
ಹಜಾರ
ಹಜಾ-ರ-ದಲ್ಲಿ
ಹಟ
ಹಟ-ಮಾ-ರಿ-ಗ-ಳಂತೆ
ಹಟ-ಮಾ-ರಿಯೂ
ಹಠಾತ್
ಹಠಾ-ತ್ತನೆ
ಹಠಾ-ತ್ತಾಗಿ
ಹಡ-ಗನ್ನು
ಹಡ-ಗನ್ನೇ
ಹಡ-ಗ-ನ್ನೇ-ರ-ಬೇ-ಕಾ-ಗಿ-ದ್ದುದು
ಹಡ-ಗ-ನ್ನೇರಿ
ಹಡ-ಗ-ನ್ನೇ-ರಿ-ದರು
ಹಡ-ಗ-ನ್ನೇ-ರು-ವುದು
ಹಡ-ಗಾಗಿ
ಹಡ-ಗಾದ
ಹಡ-ಗಾ-ದರೆ
ಹಡ-ಗಿಗೆ
ಹಡ-ಗಿನ
ಹಡ-ಗಿ-ನಲ್ಲಿ
ಹಡ-ಗಿ-ನ-ಲ್ಲಿದ್ದ
ಹಡ-ಗಿ-ನಲ್ಲೂ
ಹಡ-ಗಿ-ನಲ್ಲೇ
ಹಡ-ಗಿ-ನಿಂದ
ಹಡ-ಗಿ-ನಿಂ-ದಿ-ಳಿ-ದೊ-ಡನೆ
ಹಡ-ಗಿ-ನಿಂ-ದಿ-ಳಿ-ಯಲು
ಹಡಗು
ಹಡ-ಗು-ಕ-ಟ್ಟೆಗೆ
ಹಡ-ಗು-ಕ-ಟ್ಟೆಯ
ಹಡ-ಗು-ಕ-ಟ್ಟೆ-ಯನ್ನು
ಹಡ-ಗು-ಕ-ಟ್ಟೆ-ಯ-ಲ್ಲೊಂದು
ಹಡುಗು
ಹಡ್ಸನ್
ಹಣ
ಹಣ-ಕೀರ್ತಿ
ಹಣ-ಇವು
ಹಣ-ತೆ-ಗಳಿಂದ
ಹಣದ
ಹಣ-ದಲ್ಲೂ
ಹಣ-ದಿಂದ
ಹಣ-ದಿಂ-ದಲೇ
ಹಣ-ವಂ-ತರ
ಹಣ-ವನ್ನು
ಹಣ-ವನ್ನೂ
ಹಣ-ವ-ನ್ನೆಲ್ಲ
ಹಣ-ವ-ನ್ನೇಕೆ
ಹಣ-ವಿ-ರ-ಲಿಲ್ಲ
ಹಣ-ವಿ-ರಲೇ
ಹಣ-ವಿಲ್ಲ
ಹಣ-ವಿ-ಲ್ಲ-ದಿ-ದ್ದರೆ
ಹಣವೂ
ಹಣವೇ
ಹಣೆಗೆ
ಹಣೆಯ
ಹಣೆ-ಯ-ನ್ನೊಮ್ಮೆ
ಹಣ್ಣನ್ನೂ
ಹಣ್ಣಾಗಿ
ಹಣ್ಣಾ-ಗಿ-ಬಿ-ಟ್ಟಿ-ದ್ದ-ರಿಂದ
ಹಣ್ಣಾ-ಯಿತು
ಹಣ್ಣಿನ
ಹಣ್ಣು
ಹಣ್ಣು-ಕಾ-ಯಿಯ
ಹಣ್ಣು-ಕಾಯಿ
ಹಣ್ಣು-ಕಾ-ಯಿ-ಗಳನ್ನು
ಹಣ್ಣು-ಗ-ಳಿದ್ದ
ಹತಾ-ಶ-ನಾ-ಗಿಲ್ಲ
ಹತಾ-ಶ-ರಾಗಿ
ಹತೋಟಿ
ಹತೋ-ಟಿಗೆ
ಹತೋ-ಟಿ-ಯನ್ನು
ಹತೋ-ಟಿ-ಯ-ಲ್ಲಿ-ಟ್ಟು-ಕೊ-ಳ್ಳು-ವುದು
ಹತೋ-ಟಿ-ಯ-ಲ್ಲಿ-ಡಲು
ಹತ್ತನೇ
ಹತ್ತ-ಬೇ-ಕಾ-ಗಿತ್ತು
ಹತ್ತ-ರಂದು
ಹತ್ತ-ರಲ್ಲಿ
ಹತ್ತಾರು
ಹತ್ತಾ-ರು-ನೂ-ರಾರು
ಹತ್ತಿ
ಹತ್ತಿ-ಕ್ಕ-ಲಾ-ರದೆ
ಹತ್ತಿ-ಕ್ಕಲು
ಹತ್ತಿಕ್ಕಿ
ಹತ್ತಿ-ದರು
ಹತ್ತಿದ್ದ
ಹತ್ತಿರ
ಹತ್ತಿ-ರದ
ಹತ್ತಿ-ರ-ದಲ್ಲೆ
ಹತ್ತಿ-ರ-ದಿಂದ
ಹತ್ತಿ-ರವೂ
ಹತ್ತು
ಹತ್ತು-ಗಂ-ಟೆ-ಗಳ
ಹತ್ತುತ್ತ
ಹತ್ತು-ತ್ತಿ-ದ್ದಾಗ
ಹತ್ತು-ದಿ-ನ-ಗ-ಳಿಗೆ
ಹತ್ತು-ವ-ರ್ಷ-ಗಳ
ಹತ್ತು-ಹ-ದಿ-ನೈದು
ಹತ್ತು-ಹ-ನ್ನೆ-ರಡು
ಹತ್ತೂ
ಹತ್ತೂ-ವ-ರೆಗೆ
ಹತ್ತೂ-ವರೆ-ಯಾ-ಗಿತ್ತು
ಹತ್ತೊಂ-ಬ-ತ್ತನೇ
ಹತ್ಯೆ-ಗಳನ್ನೂ
ಹದ-ಗೆ-ಟ್ಟ-ದ್ದ-ರಿಂದ
ಹದ-ಗೆ-ಟ್ಟಿತು
ಹದ-ಗೆ-ಟ್ಟಿತ್ತು
ಹದ-ಗೆ-ಟ್ಟಿ-ತ್ತೆಂ-ದರೆ
ಹದ-ಗೆ-ಟ್ಟಿದೆ
ಹದ-ಗೆಟ್ಟು
ಹದ-ಗೆಡ
ಹದ-ಗೆ-ಡಿ-ಸಿತು
ಹದ-ಗೆ-ಡು-ತ್ತಲೇ
ಹದ-ಗೆ-ಡು-ತ್ತಿದ್ದ
ಹದ-ವಾಗಿ
ಹದಿ
ಹದಿ-ನಾ-ಲ್ಕ-ನೆಯ
ಹದಿ-ನಾಲ್ಕು
ಹದಿ-ನೆಂಟು
ಹದಿ-ನೆಂಟೇ
ಹದಿ-ನೇ-ಳನೇ
ಹದಿ-ನೇಳು
ಹದಿ-ನೈದು
ಹದಿ-ವ-ಯಸ್ಕ
ಹನಿ
ಹನಿ-ಗೂಡಿ
ಹನಿ-ಗೂ-ಡಿ-ದುವು
ಹನ್ನೆರ
ಹನ್ನೆ-ರ-ಡಕ್ಕೆ
ಹನ್ನೆ-ರಡು
ಹನ್ನೆ-ರಡೇ
ಹನ್ನೊಂ-ದನೆ
ಹನ್ನೊಂ-ದ-ರಂದು
ಹನ್ನೊಂದು
ಹಬ್ಬ-ಹ-ರಿ-ದಿ-ನ-ಗಳಲ್ಲಿ
ಹಬ್ಬ-ಗ-ಳ-ಲ್ಲೊಂ-ದಾಗಿ
ಹಬ್ಬದ
ಹಬ್ಬ-ದಂತೆ
ಹಬ್ಬ-ದೂಟ
ಹಬ್ಬ-ದೂ-ಟ-ವನ್ನು
ಹಬ್ಬಲು
ಹಬ್ಬ-ವುಂಟು
ಹಬ್ಬ-ವುಂ-ಟು-ಮಾ-ಡು-ತ್ತಿ-ದ್ದುವು
ಹಬ್ಬವೇ
ಹಬ್ಬಿ
ಹಬ್ಬಿ-ಸುವ
ಹಬ್ಬಿ-ಸು-ವಲ್ಲಿ
ಹಮ್ಮಿ-ಕೊ-ಳ್ಳ-ಲಿಲ್ಲ
ಹಯ-ಸಿಂತ್
ಹಯ-ಸಿಂ-ಥ-ರನ್ನು
ಹಯ-ಸಿಂ-ಥ-ರಿಗೇ
ಹಯ-ಸಿಂ-ಥರು
ಹಯ-ಸಿಂ-ಥ-ರೊಂ-ದಿಗೆ
ಹಯ-ಸಿಂಥ್
ಹರ
ಹರಕೆ
ಹರ-ಟಿ-ದರು
ಹರ-ಟುತ್ತ
ಹರಟೆ
ಹರ-ಟೆ-ಯಲ್ಲೇ
ಹರಡ
ಹರ-ಡ-ದಿ-ದ್ದರೆ
ಹರ-ಡ-ಬಲ್ಲ
ಹರ-ಡ-ಬ-ಹುದು
ಹರ-ಡ-ಬೇಕು
ಹರಡಿ
ಹರ-ಡಿ-ಕೊಂ-ಡಿತು
ಹರ-ಡಿ-ಕೊಂ-ಡಿತ್ತು
ಹರ-ಡಿ-ಕೊಂ-ಡಿದ್ದ
ಹರ-ಡಿ-ಕೊಂ-ಡಿ-ರುವ
ಹರ-ಡಿತು
ಹರ-ಡಿತೋ
ಹರ-ಡಿದ
ಹರ-ಡಿದೆ
ಹರ-ಡಿ-ಸಿದೆ
ಹರ-ಡು-ತ್ತವೆ
ಹರ-ಡು-ತ್ತಿ-ದ್ದುದು
ಹರ-ಡುವ
ಹರ-ಡು-ವಲ್ಲಿ
ಹರ-ಡು-ವು-ದ-ಕ್ಕಾಗಿ
ಹರ-ಣೆಗೆ
ಹರ-ನನ್ನು
ಹರ-ನಮಃ
ಹರ-ಸಲಿ
ಹರಸಿ
ಹರ-ಸಿ-ದರು
ಹರ-ಹರ
ಹರಿ
ಹರಿ-ತವಾ
ಹರಿ-ತ-ವಾದ
ಹರಿ-ತ-ವಾ-ಯಿತು
ಹರಿ-ತವೂ
ಹರಿದ
ಹರಿ-ದಂ-ತಿತ್ತು
ಹರಿ-ದರು
ಹರಿ-ದರೆ
ಹರಿ-ದಾಸ
ಹರಿ-ದಾ-ಸ-ನಿಗೆ
ಹರಿ-ದಾ-ಸ-ನಿ-ಗೇಕೆ
ಹರಿದು
ಹರಿ-ದುದು
ಹರಿ-ದು-ಬಂ-ದದ್ದು
ಹರಿ-ದು-ಬಂ-ದಿವೆ
ಹರಿ-ದು-ಬಂದು
ಹರಿ-ದು-ಬ-ರ-ತೊ-ಡ-ಗಿತು
ಹರಿ-ದು-ಬ-ರಲಿ
ಹರಿ-ದೆ-ಸೆದೆ
ಹರಿ-ದೊ-ಗೆದು
ಹರಿ-ದೊ-ಗೆ-ಯ-ಬೇ-ಕೆಂಬ
ಹರಿ-ದ್ವರ್ಣ
ಹರಿ-ದ್ವಾ-ರ-ಹೃ-ಷೀ-ಕೇಶ
ಹರಿ-ದ್ವಾ-ರಕ್ಕೆ
ಹರಿ-ಪದ
ಹರಿ-ಪ್ರ-ಸನ್ನ
ಹರಿ-ಭಾಯ್
ಹರಿ-ಯ-ತೊ-ಡ-ಗಿತ್ತು
ಹರಿ-ಯದೆ
ಹರಿ-ಯ-ಬೇಕು
ಹರಿ-ಯಿತು
ಹರಿ-ಯಿ-ಸಿ-ದರು
ಹರಿ-ಯಿ-ಸಿದೆ
ಹರಿ-ಯಿ-ಸಿ-ವೆ-ಯೆಂ-ಬು-ದನ್ನು
ಹರಿ-ಯಿ-ಸು-ತ್ತಿ-ದ್ದರೆ
ಹರಿ-ಯಿ-ಸು-ತ್ತೇನೆ
ಹರಿ-ಯಿ-ಸು-ವಂ-ತಾ-ಗು-ವಿರಿ
ಹರಿ-ಯಿ-ಸು-ವು-ದ-ರಿಂದ
ಹರಿ-ಯು-ತ್ತಿತ್ತು
ಹರಿ-ಯು-ತ್ತಿದೆ
ಹರಿ-ಯು-ತ್ತಿ-ರುವ
ಹರಿ-ಯು-ತ್ತಿ-ರು-ವುದನ್ನು
ಹರಿ-ಯು-ತ್ತಿ-ರು-ವುದು
ಹರಿ-ಯುವ
ಹರಿ-ಯು-ವಂ-ತಾ-ದಾಗ
ಹರಿ-ಯು-ವಂತೆ
ಹರಿ-ವಾ-ಣ-ಗಳ
ಹರಿ-ಸ-ಬ-ಲ್ಲ-ವ-ರಾ-ದಾಗ
ಹರಿ-ಸ-ಬ-ಹುದು
ಹರಿ-ಸ-ಬೇ-ಕಾ-ಗಿತ್ತು
ಹರಿ-ಸ-ಬೇ-ಕಾದ
ಹರಿ-ಸ-ಬೇ-ಕೆಂದು
ಹರಿಸಿ
ಹರಿ-ಸಿ-ದ-ರು-ಉ-ಪ್ಪಿ-ಲ್ಲದ
ಹರಿ-ಸಿದ್ದ
ಹರಿ-ಸುತ್ತ
ಹರಿ-ಸು-ತ್ತಿ-ದ್ದರು
ಹರಿ-ಸು-ತ್ತೀರಿ
ಹರಿ-ಸು-ವುದೇ
ಹರಿ-ಹಾ-ಯ್ದರು
ಹರೇಂ-ದ್ರ-ನಾ-ಥರು
ಹರ್ಬ-ರ್ಟ್
ಹರ್ಷ
ಹರ್ಷ-ಗೊಂಡ
ಹರ್ಷ-ಗೊಂ-ಡರು
ಹರ್ಷ-ಚಿ-ತ್ತ-ರಾ-ಗಿ-ದ್ದರು
ಹರ್ಷ-ಚಿ-ತ್ತ-ರಾ-ಗಿಯೇ
ಹರ್ಷ-ದಾ-ಯ-ಕ-ವಾ-ಗಿತ್ತು
ಹರ್ಷ-ಪೂ-ರ್ಣ-ವಾ-ಗಿ-ಡು-ತ್ತಿ-ದ್ದರು
ಹರ್ಷ-ವಾ-ಗು-ತ್ತದೆ
ಹರ್ಷಾ-ನಂ-ದಜೀ
ಹರ್ಷೋ
ಹರ್ಷೋ-ದ್ಗಾರ
ಹರ್ಷೋ-ದ್ಗಾ-ರ-ಗಳ
ಹರ್ಷೋ-ದ್ಗಾ-ರ-ಗಳಿಂದ
ಹರ್ಷೋ-ದ್ಗಾ-ರ-ಗ-ಳೊಂ-ದಿಗೆ
ಹರ್ಷೋ-ದ್ಗಾ-ರ-ಗೈ-ದರು
ಹರ್ಷೋ-ದ್ಗಾ-ರ-ದೊ-ಡನೆ
ಹಲ
ಹಲ-ಗೆಯ
ಹಲ-ಧಾರಿ
ಹಲ-ವರು
ಹಲ-ವಾರು
ಹಲ-ವಾ-ರು-ಬಾರಿ
ಹಲವು
ಹಲ-ವೆಡೆ
ಹಲ-ವೆ-ಡೆ-ಗಳಲ್ಲಿ
ಹಲ-ವೆ-ಡೆ-ಗ-ಳ-ಲ್ಲಿದ್ದ
ಹಲ-ವೆ-ಡೆ-ಗಳಿಂದ
ಹಲಿ-ಗ-ಳಾ-ಗಿ-ದ್ದರೆ
ಹಲ್ಲಿ-ಯೊಂದು
ಹಳಿ-ದು-ಕೊಂ-ಡರು
ಹಳಿಯ
ಹಳಿ-ಯು-ತ್ತಿ-ದ್ದರು
ಹಳೆಯ
ಹಳೆ-ಯ-ದ-ನ್ನೆಲ್ಲ
ಹಳೆ-ಯ-ದಾದ
ಹಳೆ-ಯದೇ
ಹಳ್ಳ-ತಿ-ಟ್ಟು-ಗಳಿಂದ
ಹಳ್ಳಿ
ಹಳ್ಳಿ-ಗರು
ಹಳ್ಳಿ-ಗ-ರೆ-ಲ್ಲರ
ಹಳ್ಳಿ-ಗಳ
ಹಳ್ಳಿ-ಗಳನ್ನು
ಹಳ್ಳಿ-ಗಳಲ್ಲಿ
ಹಳ್ಳಿ-ಗಳಿಂದ
ಹಳ್ಳಿ-ಗಳು
ಹಳ್ಳಿ-ಗ-ಳು-ಇ-ವು-ಗಳನ್ನೆಲ್ಲ
ಹಳ್ಳಿ-ಗಾಡು
ಹಳ್ಳಿಗೆ
ಹಳ್ಳಿಯ
ಹಳ್ಳಿ-ಯಲ್ಲಿ
ಹಳ್ಳಿ-ಯಲ್ಲೇ
ಹಳ್ಳಿ-ಯಿಂದ
ಹವಾ
ಹವಾ-ಗು-ಣದ
ಹವಾ-ಮಾನ
ಹವೆ
ಹವೆಯೋ
ಹಸ-ನು-ಗೊ-ಳಿ-ಸಿ-ಕೊ-ಳ್ಳ-ಬೇಕು
ಹಸ-ನ್ಮು-ಖಿ-ಗ-ಳಾ-ಗಿ-ರು-ವಂತೆ
ಹಸಿದ
ಹಸಿ-ದ-ವ-ರಿಗೆ
ಹಸಿದು
ಹಸಿರು
ಹಸಿ-ರೆ-ಲೆ-ಗಳಿಂದ
ಹಸಿ-ವಾ-ಗಿ-ಬಿ-ಟ್ಟಿತ್ತು
ಹಸಿ-ವಿ-ನಿಂದ
ಹಸಿವು
ಹಸಿವೂ
ಹಸಿ-ವೆ-ಯಿಂದ
ಹಸು-ಗ-ಳಿವೆ
ಹಸು-ರಿನ
ಹಸುಳೆ
ಹಸು-ಳೆ-ಯಂತೆ
ಹಸ್ತ-ಪ್ರತಿ
ಹಸ್ತಾಂ-ತ-ರಿ-ಸು-ತ್ತಿ-ದ್ದರು
ಹಹ್ಹ
ಹಾಕದೆ
ಹಾಕ-ಬೇ-ಕಾ-ಗು-ತ್ತಿತ್ತು
ಹಾಕ-ಲಾ-ಗಿತ್ತು
ಹಾಕ-ಲಾ-ಯಿತು
ಹಾಕಲು
ಹಾಕಿ
ಹಾಕಿ-ಕೊಂ-ಡರು
ಹಾಕಿ-ಕೊಂ-ಡರೆ
ಹಾಕಿ-ಕೊಂಡಿ
ಹಾಕಿ-ಕೊಂ-ಡಿದ್ದ
ಹಾಕಿ-ಕೊಂ-ಡಿ-ದ್ದರು
ಹಾಕಿ-ಕೊಂ-ಡಿ-ದ್ದರೊ
ಹಾಕಿ-ಕೊಂ-ಡಿ-ದ್ದು-ದೇನೋ
ಹಾಕಿ-ಕೊಂಡು
ಹಾಕಿ-ಕೊಟ್ಟ
ಹಾಕಿ-ಕೊ-ಟ್ಟ-ದ್ದ-ರಿಂದ
ಹಾಕಿ-ಕೊ-ಳ್ಳ-ಬೇ-ಕೆಂದು
ಹಾಕಿ-ಕೊ-ಳ್ಳ-ಲಾ-ಯಿತು
ಹಾಕಿ-ಕೊ-ಳ್ಳು-ತ್ತದೆ
ಹಾಕಿ-ಕೊ-ಳ್ಳು-ವಂತೆ
ಹಾಕಿ-ಕೊ-ಳ್ಳು-ವಲ್ಲಿ
ಹಾಕಿತು
ಹಾಕಿತ್ತು
ಹಾಕಿದ
ಹಾಕಿ-ದರು
ಹಾಕಿ-ದರೂ
ಹಾಕಿ-ದಷ್ಟು
ಹಾಕಿ-ದಾಗ
ಹಾಕಿದ್ದ
ಹಾಕಿ-ದ್ದ-ರಿಂ-ದಲೋ
ಹಾಕಿ-ದ್ದರು
ಹಾಕಿ-ದ್ದಾ-ರೆಯೇ
ಹಾಕಿದ್ದು
ಹಾಕಿ-ಬಿ-ಡು-ತ್ತಾರೆ
ಹಾಕಿಯೇ
ಹಾಕಿ-ಯೇ-ಬಿಟ್ಟ
ಹಾಕಿ-ರು-ತ್ತಿತ್ತು
ಹಾಕಿ-ಸ-ಬೇಕು
ಹಾಕಿಸಿ
ಹಾಕಿ-ಸಿ-ಕೊ-ಳ್ಳ-ಬೇ-ಕೆಂಬ
ಹಾಕಿ-ಸಿ-ದರು
ಹಾಕುತ್ತ
ಹಾಕು-ತ್ತಾರೆ
ಹಾಕು-ತ್ತಾ-ರೆ-ಮೊ-ದಲೇ
ಹಾಕು-ತ್ತಿತ್ತು
ಹಾಕು-ತ್ತಿದ್ದ
ಹಾಕು-ತ್ತಿ-ದ್ದ-ಅದೂ
ಹಾಕು-ತ್ತಿ-ದ್ದರು
ಹಾಕು-ತ್ತಿ-ದ್ದಳು
ಹಾಕು-ತ್ತಿ-ದ್ದ-ವರೇ
ಹಾಕು-ತ್ತಿ-ದ್ದೀ-ರ-ಲ್ಲವೆ
ಹಾಕು-ತ್ತಿ-ದ್ದೇನೆ
ಹಾಕು-ತ್ತೇ-ನೆಯೋ
ಹಾಕುವ
ಹಾಕು-ವಂತೆ
ಹಾಕು-ವಾಗ
ಹಾಕು-ವು-ದರ
ಹಾಕು-ವುದು
ಹಾಕು-ವು-ದೆಂ-ದರೆ
ಹಾಗ-ನ್ನಿ-ಸಲೇ
ಹಾಗ-ನ್ನಿ-ಸು-ತ್ತ-ದೆಯೇ
ಹಾಗಲ್ಲ
ಹಾಗ-ಲ್ಲ-ದಿ-ದ್ದರೆ
ಹಾಗಾ
ಹಾಗಾ-ಗ-ದಂತೆ
ಹಾಗಾ-ಗ-ಲಿಲ್ಲ
ಹಾಗಾ-ಗು-ವು-ದಿಲ್ಲ
ಹಾಗಾ-ದರೂ
ಹಾಗಾ-ದರೆ
ಹಾಗಾ-ದಾ-ಗ-ಲೆಲ್ಲ
ಹಾಗಾ-ಯಿತು
ಹಾಗಿದ್ದ
ಹಾಗಿ-ದ್ದರೆ
ಹಾಗಿ-ದ್ದಲ್ಲಿ
ಹಾಗಿ-ರದೆ
ಹಾಗಿ-ರಲಿ
ಹಾಗಿ-ರ-ಲಿ-ಪ-ರಂ-ಪ-ರಾ-ಗ-ತ-ವಾದ
ಹಾಗಿರು
ಹಾಗಿ-ರು-ವಾಗ
ಹಾಗೂ
ಹಾಗೂ-ಅ-ವರೇ
ಹಾಗೆ
ಹಾಗೆಂದ
ಹಾಗೆಂ-ದಿಗೂ
ಹಾಗೆನ್
ಹಾಗೆ-ನ್ನ-ದಿರು
ಹಾಗೆ-ನ್ನಿ-ಸು-ತ್ತ-ದೆಯೆ
ಹಾಗೆಯೆ
ಹಾಗೆಯೇ
ಹಾಗೆಲ್ಲ
ಹಾಗೆ-ಹಾ-ಗೆಯೇ
ಹಾಗೇ
ಹಾಗೇಕೆ
ಹಾಗೇ-ನಾ-ದರೂ
ಹಾಗೇನು
ಹಾಗೇನೂ
ಹಾಗೊಂದು
ಹಾಜ-ರಿದ್ದ
ಹಾಜ-ರಿ-ದ್ದರು
ಹಾಜ-ರಿ-ದ್ದ-ವ-ರಲ್ಲಿ
ಹಾಜ-ರಿ-ದ್ದುದು
ಹಾಜ-ರಿದ್ದೆ
ಹಾಜ-ರಿ-ಯನ್ನು
ಹಾಡನ್ನು
ಹಾಡ-ಬೇ-ಕೇಕೆ
ಹಾಡ-ಲಾ-ಗು-ತ್ತಿದೆ
ಹಾಡ-ಲಾ-ಯಿತು
ಹಾಡ-ಲಾ-ರಂ-ಭಿ-ಸಿ-ದರು
ಹಾಡ-ಹ-ಗ-ಲಲ್ಲೇ
ಹಾಡಿ
ಹಾಡಿ-ಕು-ಣಿದು
ಹಾಡಿ-ಕೊಂಡು
ಹಾಡಿ-ಕೊ-ಳ್ಳು-ತ್ತಿ-ದ್ದರು
ಹಾಡಿದ
ಹಾಡಿ-ದರು
ಹಾಡಿ-ದಾಗ
ಹಾಡಿನ
ಹಾಡಿ-ನಲ್ಲಿ
ಹಾಡು-ಗಳನ್ನು
ಹಾಡು-ಗಳಿಂದ
ಹಾಡು-ಗಳು
ಹಾಡುತ್ತ
ಹಾಡು-ತ್ತಾ-ರಂತೆ
ಹಾಡು-ತ್ತಾರೆ
ಹಾಡು-ತ್ತಿ-ದ್ದರು
ಹಾಡು-ತ್ತಿದ್ದೆ
ಹಾಡುವ
ಹಾಡು-ವಂತೆ
ಹಾಡು-ವುದು
ಹಾಡೊಂ
ಹಾಡೊಂ-ದನ್ನು
ಹಾತೊ-ರೆ-ಯು-ತ್ತಿತ್ತು
ಹಾದಿ-ಯನು
ಹಾದಿ-ಯಲ್ಲಿ
ಹಾದು
ಹಾದು-ಹೋ-ಗಿದೆ
ಹಾದು-ಹೋ-ಗು-ತ್ತಿದ್ದ
ಹಾದು-ಹೋ-ಗು-ತ್ತಿ-ದ್ದಾ-ಗ-ಅದು
ಹಾದು-ಹೋ-ಗು-ವು-ದೊಂದು
ಹಾದು-ಹೋ-ದರು
ಹಾನಿ
ಹಾನಿ-ಕ-ರ-ವಾ-ದುದು
ಹಾನಿ-ಗೊ-ಳ-ಗಾದ
ಹಾನಿ-ಯಾ-ಗು-ತ್ತ-ದೆಂ-ಬು-ದನ್ನು
ಹಾನಿ-ಯುಂ-ಟು-ಮಾ-ಡಿತ್ತು
ಹಾನಿಯೇ
ಹಾನ್ಸ್ಬ-್ರೋಗೆ
ಹಾಯಾಗಿ
ಹಾಯಿ-ಸಿ-ದರು
ಹಾಯಿ-ಸಿ-ದರೆ
ಹಾಯಿ-ಸುತ್ತ
ಹಾರ
ಹಾರ-ತು-ರಾ-ಯಿ-ಗ-ಳ-ನ್ನ-ರ್ಪಿ-ಸಲು
ಹಾರ-ಗಳನ್ನು
ಹಾರ-ವನ್ನು
ಹಾರಾ-ಟ-ಗಳಿಂದ
ಹಾರಾ-ಡಿದ
ಹಾರಿ
ಹಾರಿ-ಕೊಂಡು
ಹಾರಿ-ಬಂದ
ಹಾರಿ-ಸ-ಲಾ-ಯಿತು
ಹಾರಿ-ಸಲೂ
ಹಾರಿಸಿ
ಹಾರಿ-ಸಿ-ದರು
ಹಾರಿಸು
ಹಾರಿ-ಸುತ್ತ
ಹಾರಿ-ಹೋ-ಗು-ತ್ತಿ-ದ್ದೇನೆ
ಹಾರಿ-ಹೋ-ಗು-ವಂತೆ
ಹಾರಿ-ಹೋ-ಯಿತು
ಹಾರುವ
ಹಾರು-ವಂ-ತಾ-ಗಲಿ
ಹಾರೈಕೆ
ಹಾರೈ-ಕೆ-ಗ-ಳೊಂ-ದಿಗೆ
ಹಾರೈಸಿ
ಹಾರೈ-ಸಿ-ದರು
ಹಾರೈ-ಸಿ-ದಳು
ಹಾರೈ-ಸಿವೆ
ಹಾರೈ-ಸು-ತ್ತೀರಿ
ಹಾರೈ-ಸು-ತ್ತೇವೆ
ಹಾರ್ಟ್
ಹಾರ್ದಿ-ಕ-ವಾಗಿ
ಹಾರ್ಮ್ಸ-್ಟನ್
ಹಾರ್ವ-ರ್ಡ್
ಹಾಲನ್ನೂ
ಹಾಲಾ-ಹಲ
ಹಾಲಿ-ನಲ್ಲಿ
ಹಾಲಿ-ಸ್ಟರ್
ಹಾಲು
ಹಾಲು-ಹ-ಣ್ಣು-ಗಳ
ಹಾಲೆಂಡ್
ಹಾಲೆಂ-ಡ್ಗಳನ್ನು
ಹಾಲ್
ಹಾಲ್ನಲ್ಲಿ
ಹಾಳಾ
ಹಾಳಾಗಿ
ಹಾಳಾ-ಗು-ತ್ತವೆ
ಹಾಳಾ-ಗು-ವು-ದ-ಕ್ಕಿಂತ
ಹಾಳಾ-ಗು-ವು-ದಿಲ್ಲ
ಹಾಳು
ಹಾಳು-ಗೆ-ಡ-ವ-ಲ್ಪಟ್ಟು
ಹಾಳು-ಗೆ-ಡ-ವಿ-ದರೆ
ಹಾಳು-ಗೆ-ಡ-ವು-ತ್ತಿದೆ
ಹಾವ-ಭಾ-ವ-ಗ-ಳೊಂ-ದಿಗೆ
ಹಾವಿನ
ಹಾವು-ಗಳು
ಹಾಸ
ಹಾಸಿ
ಹಾಸಿಗೆ
ಹಾಸಿ-ಗೆ-ಗಳನ್ನು
ಹಾಸಿ-ಗೆ-ಗ-ಳಿ-ರ-ಬೇಕು
ಹಾಸಿ-ಗೆಗೇ
ಹಾಸಿ-ಗೆಯ
ಹಾಸಿ-ಗೆ-ಯಂತೆ
ಹಾಸಿದ
ಹಾಸಿ-ದಂಥ
ಹಾಸಿನ
ಹಾಸಿ-ರುವ
ಹಾಸು-ಹೊ-ಕ್ಕಾ-ಗಿದ್ದ
ಹಾಸು-ಹೊ-ಕ್ಕು-ಗಳು
ಹಾಸ್ಯ
ಹಾಸ್ಯ-ತ-ಮಾಷೆ
ಹಾಸ್ಯದ
ಹಾಸ್ಯ-ದಲ್ಲೂ
ಹಾಸ್ಯ-ದಿಂ-ದಲೂ
ಹಾಸ್ಯ-ಪ್ರ-ಜ್ಞೆ-ಇವು
ಹಾಸ್ಯ-ಪ್ರ-ದ-ವಾ-ಗಿದೆ
ಹಾಸ್ಯ-ಮಯ
ಹಾಸ್ಯ-ಮಾ-ಡು-ತ್ತಿ-ದ್ದರು
ಹಾಸ್ಯ-ವನ್ನೂ
ಹಾಸ್ಯ-ವಾಗಿ
ಹಿ
ಹಿಂಜ-ರಿ-ದರು
ಹಿಂಜ-ರಿ-ದಳು
ಹಿಂಜ-ರಿ-ಯ-ಬೇಡಿ
ಹಿಂಜ-ರಿ-ಯುತ್ತ
ಹಿಂಜ-ರಿ-ಯು-ತ್ತಾರೆ
ಹಿಂಡಿ-ನಲ್ಲಿ
ಹಿಂತೆ-ಗೆ-ದು-ಕೊಂ-ಡು-ಬಿ-ಟ್ಟಿ-ದ್ದರು
ಹಿಂದಕ್ಕೆ
ಹಿಂದಣ
ಹಿಂದಿ
ಹಿಂದಿ-ಗಿಂತ
ಹಿಂದಿ-ಗಿಂ-ತಲೂ
ಹಿಂದಿ-ಡ-ಬೇಡಿ
ಹಿಂದಿದ್ದ
ಹಿಂದಿದ್ದು
ಹಿಂದಿನ
ಹಿಂದಿ-ನಂತೆ
ಹಿಂದಿ-ನಂ-ತೆಯೇ
ಹಿಂದಿ-ನ-ವರೂ
ಹಿಂದಿ-ನಷ್ಟೇ
ಹಿಂದಿ-ನಿಂ-ದಲೂ
ಹಿಂದಿ-ನಿಂ-ದಲೇ
ಹಿಂದಿಯ
ಹಿಂದಿ-ಯಲ್ಲಿ
ಹಿಂದಿರು
ಹಿಂದಿ-ರು-ಗ-ಬ-ಯಸು
ಹಿಂದಿ-ರು-ಗ-ಬ-ಹುದು
ಹಿಂದಿ-ರು-ಗ-ಬೇ-ಕಾ-ಯಿತು
ಹಿಂದಿ-ರು-ಗ-ಲಿ-ದ್ದರು
ಹಿಂದಿ-ರು-ಗ-ಲಿಲ್ಲ
ಹಿಂದಿ-ರು-ಗಲು
ಹಿಂದಿ-ರುಗಿ
ಹಿಂದಿ-ರು-ಗಿತು
ಹಿಂದಿ-ರು-ಗಿದ
ಹಿಂದಿ-ರು-ಗಿ-ದಂ-ದಿ-ನಿಂ-ದಲೂ
ಹಿಂದಿ-ರು-ಗಿ-ದ-ರಾ-ಯಿ-ತ-ಲ್ಲವೆ
ಹಿಂದಿ-ರು-ಗಿ-ದರು
ಹಿಂದಿ-ರು-ಗಿ-ದಳು
ಹಿಂದಿ-ರು-ಗಿ-ದಾಗ
ಹಿಂದಿ-ರು-ಗಿ-ದಾ-ಗಿ-ನಿಂದ
ಹಿಂದಿ-ರು-ಗಿದ್ದ
ಹಿಂದಿ-ರು-ಗಿ-ದ್ದರೂ
ಹಿಂದಿ-ರು-ಗಿ-ದ್ದಾರೆ
ಹಿಂದಿ-ರು-ಗಿ-ದ್ದುವು
ಹಿಂದಿ-ರು-ಗಿ-ಬ-ರ-ಬೇಕು
ಹಿಂದಿ-ರು-ಗಿ-ಬಿ-ಟ್ಟರು
ಹಿಂದಿ-ರು-ಗಿ-ರ-ಲಿಲ್ಲ
ಹಿಂದಿ-ರು-ಗಿ-ರುವ
ಹಿಂದಿ-ರು-ಗಿ-ರುವೆ
ಹಿಂದಿ-ರು-ಗಿ-ಸ-ಬಾ-ರದು
ಹಿಂದಿ-ರು-ಗಿ-ಸಿ-ದರು
ಹಿಂದಿ-ರು-ಗಿ-ಸಿ-ಬಿಡು
ಹಿಂದಿ-ರುಗು
ಹಿಂದಿ-ರು-ಗು-ತ್ತಿ-ದ್ದಂ-ತೆಯೇ
ಹಿಂದಿ-ರು-ಗು-ತ್ತಿ-ದ್ದರು
ಹಿಂದಿ-ರು-ಗು-ತ್ತಿ-ದ್ದೇನೆ
ಹಿಂದಿ-ರು-ಗು-ತ್ತಿ-ರುವ
ಹಿಂದಿ-ರು-ಗು-ತ್ತಿ-ರು-ವಾಗ
ಹಿಂದಿ-ರು-ಗು-ತ್ತೇನೆ
ಹಿಂದಿ-ರು-ಗುವ
ಹಿಂದಿ-ರು-ಗು-ವಂ-ತಾ-ಗು-ವ-ವ-ರೆಗೆ
ಹಿಂದಿ-ರು-ಗು-ವಂ-ತಿ-ರ-ಲಿಲ್ಲ
ಹಿಂದಿ-ರು-ಗು-ವಂತೆ
ಹಿಂದಿ-ರು-ಗು-ವಾಗ
ಹಿಂದಿ-ರು-ಗು-ವು-ದಕ್ಕೆ
ಹಿಂದಿ-ರು-ಗು-ವುದನ್ನು
ಹಿಂದಿ-ರು-ಗು-ವು-ದೆಂದು
ಹಿಂದಿ-ರುವ
ಹಿಂದಿ-ರು-ವಾಗ
ಹಿಂದಿ-ರು-ವುದೂ
ಹಿಂದಿ-ರು-ವು-ದೆಲ್ಲ
ಹಿಂದೀ
ಹಿಂದು-ಗಳು
ಹಿಂದು-ಗಳೇ
ಹಿಂದು-ಳಿದ
ಹಿಂದು-ಳಿ-ದ-ವರೇ
ಹಿಂದು-ಳಿ-ದಿ-ದ್ದರ
ಹಿಂದು-ಳಿದು
ಹಿಂದುವೂ
ಹಿಂದು-ವೆಂದು
ಹಿಂದುವೇ
ಹಿಂದು-ವೊ-ಬ್ಬನು
ಹಿಂದೂ
ಹಿಂದೂ-ಬೌ-ದ್ಧರ
ಹಿಂದೂ-ಮು-ಸ್ಲಿಮ್
ಹಿಂದೂ-ಗಳ
ಹಿಂದೂ-ಗಳನ್ನು
ಹಿಂದೂ-ಗ-ಳ-ಲ್ಲ-ದ-ವರು
ಹಿಂದೂ-ಗ-ಳ-ಲ್ಲದೆ
ಹಿಂದೂ-ಗ-ಳಾದ
ಹಿಂದೂ-ಗಳಿಂದ
ಹಿಂದೂ-ಗ-ಳಿಗೂ
ಹಿಂದೂ-ಗ-ಳಿಗೆ
ಹಿಂದೂ-ಗಳು
ಹಿಂದೂ-ಗಳೂ
ಹಿಂದೂ-ಗ-ಳೆಂ-ದರೆ
ಹಿಂದೂ-ಗ-ಳೆ-ನ್ನು-ವಂ-ತಿಲ್ಲ
ಹಿಂದೂ-ಗ-ಳೆ-ಲ್ಲರ
ಹಿಂದೂ-ಗ-ಳೆ-ಲ್ಲ-ರಿಗೂ
ಹಿಂದೂ-ಗಳೇ
ಹಿಂದೂ-ಗ-ಳೊಂ-ದಿಗೆ
ಹಿಂದೂ-ಜ-ನ-ಗಳ
ಹಿಂದೂ-ಧರ್ಮ
ಹಿಂದೂ-ಧ-ರ್ಮಕ್ಕೆ
ಹಿಂದೂ-ಧ-ರ್ಮ-ಗಳ
ಹಿಂದೂ-ಧ-ರ್ಮದ
ಹಿಂದೂ-ಧ-ರ್ಮ-ದಲ್ಲಿ
ಹಿಂದೂ-ಧ-ರ್ಮ-ದಿಂದ
ಹಿಂದೂ-ಧ-ರ್ಮ-ದೊ-ಳ-ಗಿನ
ಹಿಂದೂ-ಧ-ರ್ಮ-ವನ್ನು
ಹಿಂದೂ-ಧ-ರ್ಮವು
ಹಿಂದೂ-ಧ-ರ್ಮವೇ
ಹಿಂದೂ-ವಾಗಿ
ಹಿಂದೂ-ವಿಗೂ
ಹಿಂದೂ-ಸಂ-ಸ್ಕೃ-ತಿ-ಗಳನ್ನು
ಹಿಂದೂ-ಸ-ಮು-ದಾ-ಯದ
ಹಿಂದೂ-ಸ್ತಾ-ನಿ-ಗ-ಳು-ಬಂ-ಗಾ-ಳಿ-ಗಳು
ಹಿಂದೂ-ಸ್ತಾನೀ
ಹಿಂದೆ
ಹಿಂದೆಂದಿ
ಹಿಂದೆಂ-ದಿ-ಗಿಂತ
ಹಿಂದೆಂ-ದಿ-ಗಿಂ-ತಲೂ
ಹಿಂದೆಂದೂ
ಹಿಂದೆ-ಗೆದು
ಹಿಂದೆ-ಗೆ-ದು-ಕೊ-ಳ್ಳ-ತೊ-ಡ-ಗಿ-ದರು
ಹಿಂದೆ-ಗೆ-ದು-ಕೊ-ಳ್ಳ-ಲಿಲ್ಲ
ಹಿಂದೆ-ಬಿ-ದ್ದರು
ಹಿಂದೆ-ಯಷ್ಟೇ
ಹಿಂದೆಯೂ
ಹಿಂದೆಯೇ
ಹಿಂದೆ-ಯೇ-ಅ-ವ-ರಿನ್ನೂ
ಹಿಂದೆಲ್ಲ
ಹಿಂದೇಟು
ಹಿಂದೊಮ್ಮೆ
ಹಿಂಬ-ದಿ-ಯ-ಲ್ಲಿದ್ದ
ಹಿಂಬಾಲಿ
ಹಿಂಬಾ-ಲಿಸಿ
ಹಿಂಬಾ-ಲಿ-ಸಿದ
ಹಿಂಬಾ-ಲಿ-ಸಿ-ದರು
ಹಿಂಬಾ-ಲಿ-ಸಿದೆ
ಹಿಂಬಾ-ಲಿ-ಸಿದ್ದ
ಹಿಂಬಾ-ಲಿ-ಸು-ತ್ತಿ-ದ್ದರು
ಹಿಂಭಾ-ಗ-ವನ್ನು
ಹಿಂಸೆ
ಹಿಂಸೆ-ಯನ್ನು
ಹಿಂಸೆ-ಯನ್ನೂ
ಹಿಗೆ
ಹಿಗ್ಗಲು
ಹಿಗ್ಗಿ-ಹೋ-ಗಿದೆ
ಹಿಗ್ಗು
ಹಿಗ್ಗುಂ
ಹಿಗ್ಗುತ್ತ
ಹಿಗ್ಗು-ತ್ತಿ-ರ-ಲಿಲ್ಲ
ಹಿಡಿ
ಹಿಡಿತ
ಹಿಡಿ-ತ-ದಿಂದ
ಹಿಡಿ-ತ-ವನ್ನು
ಹಿಡಿ-ತ-ವಿ-ರು-ವಂತೆ
ಹಿಡಿ-ತವು
ಹಿಡಿದ
ಹಿಡಿ-ದಂ-ತಾಗಿ
ಹಿಡಿ-ದದ್ದ
ಹಿಡಿ-ದ-ವನೇ
ಹಿಡಿ-ದಿ-ಟ್ಟಿ-ದ್ದರು
ಹಿಡಿ-ದಿ-ಟ್ಟು-ಕೊಂ-ಡಿ-ದ್ದಾರೆ
ಹಿಡಿ-ದಿ-ಟ್ಟು-ಕೊ-ಳ್ಳ-ಬೇ-ಕಾದ
ಹಿಡಿ-ದಿ-ಟ್ಟು-ಕೊ-ಳ್ಳ-ಲಾ-ರದೆ
ಹಿಡಿ-ದಿ-ಡ-ಲಾ-ರದು
ಹಿಡಿ-ದಿದ್ದ
ಹಿಡಿ-ದಿ-ದ್ದರು
ಹಿಡಿ-ದಿ-ದ್ದರೆ
ಹಿಡಿ-ದಿದ್ದಾ
ಹಿಡಿ-ದಿ-ರ-ಲಿಲ್ಲ
ಹಿಡಿ-ದಿ-ರುವ
ಹಿಡಿದು
ಹಿಡಿ-ದು-ಕೊಂಡ
ಹಿಡಿ-ದು-ಕೊಂ-ಡರು
ಹಿಡಿ-ದು-ಕೊಂ-ಡ-ವ-ರಾ-ದರೆ
ಹಿಡಿ-ದು-ಕೊಂ-ಡಿತು
ಹಿಡಿ-ದು-ಕೊಂ-ಡಿತ್ತು
ಹಿಡಿ-ದು-ಕೊಂ-ಡಿ-ದ್ದರು
ಹಿಡಿ-ದು-ಕೊಂಡು
ಹಿಡಿ-ದು-ಕೊಂ-ಡು-ಬಿ-ಟ್ಟಿತ್ತು
ಹಿಡಿ-ದು-ಕೊಳ್ಳಿ
ಹಿಡಿದೇ
ಹಿಡಿ-ಯ-ದಿ-ದ್ದಲ್ಲಿ
ಹಿಡಿ-ಯ-ಬ-ಲ್ಲ-ವ-ರಾರು
ಹಿಡಿ-ಯ-ಬೇ-ಕಾ-ಯಿತು
ಹಿಡಿ-ಯ-ಬೇಕು
ಹಿಡಿ-ಯ-ಲಾ-ಗದು
ಹಿಡಿ-ಯಲು
ಹಿಡಿ-ಯಲೂ
ಹಿಡಿ-ಯಿತು
ಹಿಡಿ-ಯಿರಿ
ಹಿಡಿ-ಯುತ್ತ
ಹಿಡಿ-ಯು-ತ್ತಾರೆ
ಹಿಡಿ-ಯು-ತ್ತಾಳೆ
ಹಿಡಿ-ಯುವ
ಹಿಡಿ-ಯು-ವಂ-ತಾ-ಯಿ-ತಲ್ಲ
ಹಿಡಿ-ಯು-ವು-ದಿಲ್ಲ
ಹಿಡಿ-ಯು-ವುದು
ಹಿಡಿ-ಸ-ಬ-ಹು-ದಾ-ದಂ-ತಹ
ಹಿಡಿ-ಸ-ಲಾರ
ಹಿಡಿ-ಸ-ಲಾ-ರ-ದಷ್ಟು
ಹಿಡಿ-ಸ-ಲಾರೆ
ಹಿಡಿ-ಸೀತೆ
ಹಿಡಿ-ಸು-ವಂ-ತಹ
ಹಿತ-ಕ-ರ-ವಾದ
ಹಿತ-ಕ್ಕಾಗಿ
ಹಿತ-ಕ್ಕಾ-ಗಿ
ಹಿತ-ಕ್ಕಾ-ಗಿಯೇ
ಹಿತ-ಚಿಂ-ತ-ನೆಯ
ಹಿತ-ಚಿಂ-ತ-ನೆ-ಯನ್ನು
ಹಿತ-ಚಿಂ-ತ-ನೆ-ಯಲ್ಲಿ
ಹಿತದ
ಹಿತ-ದೃ-ಷ್ಟಿ-ಯಿಂದ
ಹಿತ-ವ-ನ್ನುಂ-ಟು-ಮಾ-ಡುವ
ಹಿತ-ವಾ-ಗ-ಬ-ಹುದು
ಹಿತ-ವಾ-ಗು-ವಂತೆ
ಹಿತ-ವಾದ
ಹಿತ-ಸಾ-ಧ-ನೆ-ಗಾಗಿ
ಹಿತ-ಸಾ-ಧ-ನೆಯ
ಹಿತೈ-ಷಿ-ಗಳೂ
ಹಿತೋ-ಪ-ದೇ-ಶವೇ
ಹಿತ್ತಲ
ಹಿನ್ನೆಲೆ
ಹಿನ್ನೆ-ಲೆಯ
ಹಿನ್ನೆ-ಲೆ-ಯಲ್ಲಿ
ಹಿನ್ನೆ-ಲೆ-ಯ-ಲ್ಲಿ-ರುವ
ಹಿನ್ನೆ-ಲೆ-ಯಲ್ಲೂ
ಹಿನ್ನೆ-ಲೆ-ಯಾಗಿ
ಹಿಮ
ಹಿಮ-ಮ-ಳೆ-ನೀರು
ಹಿಮ-ಗಡ್ಡೆ
ಹಿಮದ
ಹಿಮ-ದಲ್ಲೇ
ಹಿಮ-ನ-ದಿ-ಯಲ್ಲಿ
ಹಿಮ-ನ-ದಿ-ಯೊಂದು
ಹಿಮ-ಪಾತ
ಹಿಮ-ಪಾ-ತಕ್ಕೆ
ಹಿಮ-ಪಾ-ತದ
ಹಿಮ-ಪಾ-ತ-ದಿಂ-ದಾಗಿ
ಹಿಮ-ಪಾ-ತವೂ
ಹಿಮ-ಬಂ-ಡೆ-ಗಳ
ಹಿಮ-ಬಂ-ಡೆಯ
ಹಿಮ-ಬೀ-ಳುವ
ಹಿಮ-ಮಣಿ
ಹಿಮ-ಮ-ಣಿ-ಗ-ಳಿಂ-ದಿ-ಡಿದ
ಹಿಮ-ರಾ-ಶಿಯ
ಹಿಮ-ವಂ-ತನ
ಹಿಮ-ವ-ತ್ಪ-ರ್ವ-ತದ
ಹಿಮ-ವ-ತ್ಪ-ರ್ವ-ತ-ಶ್ರೇ-ಣಿಯ
ಹಿಮ-ವನ್ನೇ
ಹಿಮಾ
ಹಿಮಾ-ಚ-ಲದ
ಹಿಮಾ-ಲಯ
ಹಿಮಾ-ಲ-ಯಕ್ಕೆ
ಹಿಮಾ-ಲ-ಯದ
ಹಿಮಾ-ಲ-ಯ-ದಲ್ಲಿ
ಹಿಮಾ-ಲ-ಯ-ದ-ಲ್ಲಿನ
ಹಿಮಾ-ಲ-ಯ-ದಲ್ಲೇ
ಹಿಮಾ-ಲ-ಯ-ದಿಂದ
ಹಿಮಾ-ಲ-ಯ-ವನ್ನೇ
ಹಿಮಾ-ಲ-ಯವು
ಹಿಮಾ-ಲ-ಯವೇ
ಹಿಮಾ-ವೃತ
ಹಿಮಾ-ವೃ-ತ-ವಾ-ಗಿದ್ದು
ಹಿರಂ
ಹಿರಮ್
ಹಿರಿ-ಕಿ-ರಿ-ಯ-ರೆ-ಲ್ಲ-ರಿಗೂ
ಹಿರಿ-ದಾಗಿ
ಹಿರಿದು
ಹಿರಿಮೆ
ಹಿರಿ-ಮೆ-ಗ-ರಿ-ಮೆ-ಗಳ
ಹಿರಿ-ಮೆ-ಗಳನ್ನು
ಹಿರಿ-ಮೆ-ಗಾಗಿ
ಹಿರಿ-ಮೆ-ಯನ್ನು
ಹಿರಿ-ಮೆಯು
ಹಿರಿಯ
ಹಿರಿ-ಯರ
ಹಿರಿ-ಯ-ರಾದ
ಹಿರಿ-ಯ-ರಿಗೆ
ಹಿರಿ-ಯರು
ಹಿರಿ-ಯ-ರೆ-ಲ್ಲರೂ
ಹಿರೋ-ಷಿ-ಮ-ವನ್ನು
ಹೀಗ-ಲ್ಲದೆ
ಹೀಗ-ಳೆ-ಯು-ತ್ತಿಲ್ಲ
ಹೀಗಾಗಿ
ಹೀಗಾ-ಗು-ತ್ತಿದೆ
ಹೀಗಿತ್ತು
ಹೀಗಿದೆ
ಹೀಗಿ-ದ್ದರೂ
ಹೀಗಿ-ದ್ದುವು
ಹೀಗಿದ್ದೂ
ಹೀಗಿ-ರ-ಬೇಕು
ಹೀಗಿ-ರು-ವಾಗ
ಹೀಗಿವೆ
ಹೀಗೂ
ಹೀಗೆ
ಹೀಗೆಂದ
ಹೀಗೆಂ-ದರು
ಹೀಗೆಂ-ದ-ವನೇ
ಹೀಗೆಂದು
ಹೀಗೆ-ನ್ನುತ್ತ
ಹೀಗೆ-ನ್ನು-ತ್ತಿ-ದ್ದಂ-ತೆಯೇ
ಹೀಗೆಯೆ
ಹೀಗೆಯೇ
ಹೀಗೆಲ್ಲ
ಹೀಗೇ
ಹೀಗೊಂದು
ಹೀನ
ಹೀನತೆ
ಹೀನ-ತೆ-ಯನ್ನೂ
ಹೀನ-ಮ-ಟ್ಟದ
ಹೀನಾ-ಯ-ವಾಗಿ
ಹೀರಿ-ಕೊಂಡು
ಹೀರಿ-ಕೊ-ಳ್ಳು-ವಂ-ತಾ-ಗು-ತ್ತದೆ
ಹೀರು-ತಿ-ಹುದು
ಹೀರುತ್ತ
ಹುಕ್ಕ
ಹುಕ್ಕಾ
ಹುಕ್ಕಾದ
ಹುಕ್ಕಾ-ವನ್ನೋ
ಹುಚ್ಚಪ್ಪ
ಹುಚ್ಚ-ಪ್ಪ-ಗಳ
ಹುಚ್ಚ-ರಂತಾ
ಹುಚ್ಚ-ರಾ-ಗಿ-ದ್ದರೆ
ಹುಚ್ಚು
ಹುಚ್ಚು-ಪೂ-ಜಾ-ರಿಯ
ಹುಚ್ಚು-ವ-ರಿದು
ಹುಚ್ಚೆದ್ದು
ಹುಟ್ಚಿ-ದ್ದ-ರಿಂದ
ಹುಟ್ಟ-ಲಿ-ಲ್ಲ-ವಲ್ಲ
ಹುಟ್ಟಾ
ಹುಟ್ಟಿ
ಹುಟ್ಟಿ-ಕೊಂ-ಡಂ-ಥವು
ಹುಟ್ಟಿ-ಕೊಂ-ಡ-ದ್ದ-ಲ್ಲದೆ
ಹುಟ್ಟಿ-ಕೊಂ-ಡರೆ
ಹುಟ್ಟಿ-ಕೊಂ-ಡಿತು
ಹುಟ್ಟಿ-ಕೊಂ-ಡುವು
ಹುಟ್ಟಿ-ಕೊಂ-ಡು-ವು-ಸ್ವಾ-ಮೀ-ಜಿ-ಯ-ವ-ರನ್ನು
ಹುಟ್ಟಿ-ಕೊ-ಳ್ಳ-ತೊ-ಡಗಿ
ಹುಟ್ಟಿತು
ಹುಟ್ಟಿದ
ಹುಟ್ಟಿ-ದಂ-ದಿ-ನಿಂ-ದಲೇ
ಹುಟ್ಟಿ-ದೆನೋ
ಹುಟ್ಟಿದ್ದು
ಹುಟ್ಟಿ-ದ್ದೇನೆ
ಹುಟ್ಟಿ-ನಿಂದ
ಹುಟ್ಟಿ-ಬಂದ
ಹುಟ್ಟಿ-ಬ-ರ-ಬ-ಲ್ಲರು
ಹುಟ್ಟಿ-ಬ-ರ-ಬೇ-ಕಾ-ಗು-ತ್ತದೆ
ಹುಟ್ಟಿ-ಬ-ರ-ಲಿರು
ಹುಟ್ಟಿ-ಬೆ-ಳೆದ
ಹುಟ್ಟಿ-ಯಾರೇ
ಹುಟ್ಟಿಯೇ
ಹುಟ್ಟಿ-ರ-ದಿ-ದ್ದರೆ
ಹುಟ್ಟಿ-ರ-ಬ-ಹುದು
ಹುಟ್ಟಿ-ರು-ತ್ತಾರೆ
ಹುಟ್ಟಿ-ರು-ವುದು
ಹುಟ್ಟಿಸಿ
ಹುಟ್ಟಿ-ಸಿತ್ತು
ಹುಟ್ಟಿ-ಸಿ-ದ-ವನು
ಹುಟ್ಟಿ-ಸು-ವಂ-ತಹ
ಹುಟ್ಟಿ-ಸು-ವುದನ್ನು
ಹುಟ್ಟು
ಹುಟ್ಟು-ಕು-ರು-ಡರೆ
ಹುಟ್ಟು-ಗಳ
ಹುಟ್ಟು-ಗೋ-ಲಿ-ನಿಂದ
ಹುಟ್ಟುವು
ಹುಟ್ಟು-ವುದೇ
ಹುಟ್ಟು-ಹಬ್ಬ
ಹುಟ್ಟು-ಹಾ-ಕ-ಬ-ಹು-ದ-ಲ್ಲವೆ
ಹುಟ್ಟೂ-ರಾದ
ಹುಡಿ-ಗೈ-ಯುತ
ಹುಡು-ಕಲು
ಹುಡು-ಕಾ-ಟದ
ಹುಡು-ಕಾಡಿ
ಹುಡು-ಕಾ-ಡಿತು
ಹುಡುಕಿ
ಹುಡು-ಕಿ-ಕೊಂ-ಡರೂ
ಹುಡು-ಕಿ-ಕೊಂಡು
ಹುಡು-ಕಿ-ಕೊಟ್ಟು
ಹುಡು-ಕಿಟ್ಟಿ
ಹುಡು-ಕಿ-ದರು
ಹುಡು-ಕಿ-ದರೂ
ಹುಡು-ಕಿ-ದ್ದೇನೆ
ಹುಡು-ಕಿ-ರು-ವಿರಾ
ಹುಡುಕು
ಹುಡು-ಕು-ತಿತ್ತು
ಹುಡು-ಕು-ತ್ತಾ-ನೆಯೋ
ಹುಡು-ಕು-ತ್ತಿ-ದ್ದಂತೆ
ಹುಡು-ಕು-ತ್ತಿ-ದ್ದರೂ
ಹುಡು-ಕು-ತ್ತಿ-ದ್ದ-ರೆಂ-ಬುದು
ಹುಡು-ಕು-ತ್ತಿ-ದ್ದಾರೆ
ಹುಡು-ಕು-ವುದು
ಹುಡುಗ
ಹುಡು-ಗನ
ಹುಡು-ಗ-ನಂ-ತಾ-ಗಿ-ಬಿ-ಟ್ಟಿ-ದ್ದರು
ಹುಡು-ಗ-ನಂತೆ
ಹುಡು-ಗ-ನಾ-ಗಿ-ದ್ದಾಗ
ಹುಡು-ಗ-ನಾ-ಗಿ-ದ್ದಾ-ಗಿ-ನಿಂ-ದಲೂ
ಹುಡು-ಗ-ರಂತೆ
ಹುಡು-ಗ-ರಿಗೆ
ಹುಡು-ಗರು
ಹುಡು-ಗರೂ
ಹುಡು-ಗ-ರೆಲ್ಲ
ಹುಡು-ಗಾ-ಟ-ವಾ-ಡು-ವುದು
ಹುಡು-ಗಾ-ಟವೆ
ಹುಡು-ಗಾ-ಟ-ವೆಂದು
ಹುಡು-ಗಾ-ಟಿಕೆ
ಹುಡುಗಿ
ಹುಡು-ಗಿ-ಯ-ರಿ-ಗಾ-ಗಿಯೇ
ಹುತಾ-ತ್ಮನ
ಹುತಾ-ತ್ಮ-ರಾ-ಗು-ವು-ದೆಂದು
ಹುದು-ಗಿ-ರುವ
ಹುದು-ಗಿ-ಸಿ-ಕೊಂ-ಡಳು
ಹುಬ್ಬು-ಗಳ
ಹುಮ್ಮ-ಸ್ಸನ್ನು
ಹುಯಿ-ಲೆ-ಬ್ಬಿ-ಸು-ತ್ತಿದ್ದ
ಹುಯಿ-ಲೆ-ಬ್ಬಿ-ಸು-ತ್ತಿ-ದ್ದರು
ಹುರಿದ
ಹುರಿ-ದುಂಬಿ
ಹುರಿ-ದುಂ-ಬಿ-ಸು-ತ್ತಿ-ದ್ದರು
ಹುರಿ-ದುಂ-ಬಿ-ಸು-ತ್ತಿ-ದ್ದಾಗ
ಹುರು-ಪನ್ನು
ಹುರುಪು
ಹುಲಿ-ಸಿಂ-ಹ-ಗ-ಳಿಗೆ
ಹುಲಿ-ಗಳನ್ನು
ಹುಲಿ-ಯಂ-ತೆಯೇ
ಹುಲ್ಲಿನ
ಹುಲ್ಲು
ಹುಲ್ಲು-ಕ-ಡ್ಡಿ-ಯಷ್ಟು
ಹುಲ್ಲು-ಹಾ-ಸಿನ
ಹುಲ್ಲು-ಹಾ-ಸಿ-ನೊ-ಳಗೆ
ಹುಳ
ಹುಳು-ಕನ್ನು
ಹುಳು-ಗಳು
ಹುಸೇನ್
ಹೂಗಳನ್ನು
ಹೂಗಳನ್ನೂ
ಹೂಗಳಿಂದ
ಹೂಗ್ಲಿ
ಹೂಡಿ
ಹೂಡಿತು
ಹೂಡಿ-ದರು
ಹೂಡಿ-ದಾಗ
ಹೂಡುತ್ತ
ಹೂಡು-ತ್ತದೆ
ಹೂಡು-ವು-ದ-ನ್ನೆಲ್ಲ
ಹೂಣರು
ಹೂತು-ಹೋ-ಗಿ-ಬಿ-ಟ್ಟಿತು
ಹೂದೋಟ
ಹೂಮ-ಳೆ-ಗ-ರೆ-ದರು
ಹೂಮಾ-ಲೆ-ಗಳ
ಹೂಮಾ-ಲೆ-ಗ-ಳ-ನ್ನ-ರ್ಪಿ-ಸಿ-ದರು
ಹೂಮಾ-ಲೆ-ಗಳಿಂದ
ಹೂವನ್ನು
ಹೂವು
ಹೂವು-ಗಳಿಂದ
ಹೂವು-ಗಳು
ಹೂಹಾ-ರ-ಗಳಿಂದ
ಹೃತ್ಪೂರ್
ಹೃತ್ಪೂ-ರ್ವಕ
ಹೃತ್ಪೂ-ರ್ವ-ಕ-ವಾಗಿ
ಹೃದಯ
ಹೃದ-ಯ-ನ-ರ-ಮಂ-ಡಲ
ಹೃದ-ಯ-ಇ-ವೆ-ರಡೂ
ಹೃದ-ಯ-ಕ್ಕಾ-ಗಿದ್ದ
ಹೃದ-ಯ-ಕ್ಕಾದ
ಹೃದ-ಯಕ್ಕೂ
ಹೃದ-ಯಕ್ಕೆ
ಹೃದ-ಯ-ಗಳನ್ನು
ಹೃದ-ಯ-ಗಳಲ್ಲಿ
ಹೃದ-ಯ-ಗು-ಡಿಗೇ
ಹೃದ-ಯ-ತೆ-ಯನ್ನು
ಹೃದ-ಯದ
ಹೃದ-ಯ-ದಲಿ
ಹೃದ-ಯ-ದಲ್ಲಿ
ಹೃದ-ಯ-ದಲ್ಲೂ
ಹೃದ-ಯ-ದಿಂದ
ಹೃದ-ಯ-ದೊ-ಳಗೆ
ಹೃದ-ಯ-ನಾಡಿ
ಹೃದ-ಯ-ಪ-ದ್ಮ-ವನ್ನು
ಹೃದ-ಯ-ವಂತ
ಹೃದ-ಯ-ವಂ-ತ-ರಾಗಿ
ಹೃದ-ಯ-ವಂ-ತ-ರಾ-ಗಿ-ರ-ಬ-ಹುದು
ಹೃದ-ಯ-ವನ್ನು
ಹೃದ-ಯ-ವನ್ನೂ
ಹೃದ-ಯ-ವಿ-ದ್ರಾ-ವಕ
ಹೃದ-ಯ-ವಿ-ಲ್ಲವೆ
ಹೃದ-ಯ-ವಿ-ಲ್ಲವೋ
ಹೃದ-ಯವು
ಹೃದ-ಯ-ವು-ಳ್ಳ-ವರು
ಹೃದ-ಯವೂ
ಹೃದ-ಯವೇ
ಹೃದ-ಯವೋ
ಹೃದ-ಯ-ಸ್ಪಂ-ದ-ನ-ದೊಂ-ದಿಗೆ
ಹೃದ-ಯ-ಸ್ಪರ್ಶಿ
ಹೃದ-ಯ-ಸ್ಪ-ರ್ಶಿ-ಯಾ-ಗಿತ್ತು
ಹೃದ-ಯ-ಸ್ಪ-ರ್ಶಿ-ಯಾದ
ಹೃದ-ಯಾಂ-ತ-ರಾ-ಳ-ದಲ್ಲಿ
ಹೃದ-ಯಾಂ-ತ-ರಾ-ಳ-ದ-ವ-ರೆಗೂ
ಹೃದ-ಯಾಂ-ತ-ರಾ-ಳ-ದಿಂದ
ಹೃದ-ಯಾಂ-ತ-ರಾ-ಳ-ದಿಂ-ದಲೇ
ಹೃದ-ಯಿ-ಗ-ಳಾದ
ಹೃದಿ-ಪ-ದ್ಮ-ಕೊರೆ
ಹೃನ್ಮ-ನ-ಗಳನ್ನು
ಹೃನ್ಮ-ನ-ಗಳಲ್ಲಿ
ಹೆಂಗ-ಸನ್ನು
ಹೆಂಗ-ಸ-ರಂತೆ
ಹೆಂಗ-ಸ-ರನ್ನು
ಹೆಂಗ-ಸ-ರಿ-ಗಿಂತ
ಹೆಂಗ-ಸರು
ಹೆಂಗ-ಸಿನ
ಹೆಂಗಸು
ಹೆಂಗಸೇ
ಹೆಂಡತಿ
ಹೆಂಡಿ-ರು-ಮ-ಕ್ಕಳು
ಹೆಕ್ಕಿ
ಹೆಗಲ
ಹೆಗ್ಗ-ಳಿಕೆ
ಹೆಗ್ಗ-ಳಿ-ಕೆ-ಗಳನ್ನೆಲ್ಲ
ಹೆಚ್
ಹೆಚ್ಚ-ಲಾ-ರಂ-ಭಿ-ಸಿತ್ತು
ಹೆಚ್ಚಲೂ
ಹೆಚ್ಚಾಗಿ
ಹೆಚ್ಚಾ-ಗಿತ್ತು
ಹೆಚ್ಚಾ-ಗಿದೆ
ಹೆಚ್ಚಾ-ಗಿ-ರುವ
ಹೆಚ್ಚಾ-ಗಿ-ರು-ವುದು
ಹೆಚ್ಚಾ-ದಷ್ಟೂ
ಹೆಚ್ಚಾ-ದಾಗ
ಹೆಚ್ಚಾ-ಯಿತು
ಹೆಚ್ಚಿಗೆ
ಹೆಚ್ಚಿ-ಗೆ-ಯೇನೂ
ಹೆಚ್ಚಿತು
ಹೆಚ್ಚಿ-ದಂತೆ
ಹೆಚ್ಚಿನ
ಹೆಚ್ಚಿ-ನಂ-ಶ-ವನ್ನು
ಹೆಚ್ಚಿ-ನ-ದನ್ನು
ಹೆಚ್ಚಿ-ನ-ದಾದ
ಹೆಚ್ಚಿ-ನದು
ಹೆಚ್ಚಿ-ನ-ದೇ-ನನ್ನು
ಹೆಚ್ಚಿ-ನ-ದೇ-ನನ್ನೂ
ಹೆಚ್ಚಿ-ನ-ದೊಂದು
ಹೆಚ್ಚಿ-ನ-ವ-ನೆಂದು
ಹೆಚ್ಚಿ-ನ-ವರು
ಹೆಚ್ಚಿ-ಸಿ-ಕೊಂಡು
ಹೆಚ್ಚಿ-ಸಿ-ಕೊ-ಳ್ಳುವ
ಹೆಚ್ಚು
ಹೆಚ್ಚು-ಕ-ಡ-ಮೆ-ಯಾ-ಗ-ಬ-ಹು-ದೆಂದು
ಹೆಚ್ಚು-ಕ-ಡಿಮೆ
ಹೆಚ್ಚು-ಕ-ಡಿ-ಮೆ-ಯಾ-ದಂತೆ
ಹೆಚ್ಚು-ಕಾಲ
ಹೆಚ್ಚು-ಗಾ-ರಿ-ಕೆ-ಯಿ-ದ್ದರೆ
ಹೆಚ್ಚುತ್ತ
ಹೆಚ್ಚು-ತ್ತಲೇ
ಹೆಚ್ಚು-ದಿನ
ಹೆಚ್ಚು-ಹೆ-ಚ್ಚಾಗಿ
ಹೆಚ್ಚು-ಹೆ-ಚ್ಚಾ-ಗುತ್ತ
ಹೆಚ್ಚು-ಹೆಚ್ಚು
ಹೆಚ್ಚು-ಹೊತ್ತು
ಹೆಚ್ಚೆಂ-ದರೆ
ಹೆಚ್ಚೆಚ್ಚು
ಹೆಚ್ಚೇನೂ
ಹೆಜ್ಜೆ
ಹೆಜ್ಜೆ-ಗ-ಳ-ನ್ನಿ-ಡುತ್ತ
ಹೆಜ್ಜೆ-ಯ-ನ್ನಿ-ಡದ
ಹೆಜ್ಜೆ-ಯ-ನ್ನಿ-ಡುತ್ತ
ಹೆಜ್ಜೆ-ಯನ್ನು
ಹೆಜ್ಜೆ-ಯನ್ನೂ
ಹೆಜ್ಜೆ-ಯಲ್ಲಿ
ಹೆಜ್ಜೆ-ಯಾಗಿ
ಹೆಜ್ಜೆ-ಯಾ-ಗಿತ್ತು
ಹೆಜ್ಜೆ-ಯೆಂ-ದರೆ
ಹೆಜ್ಜೆ-ಹೆ-ಜ್ಜೆಗೂ
ಹೆಡೆ
ಹೆಣ-ಗಿ-ದ-ವರು
ಹೆಣೆ-ದು-ಕೊಂ-ಡಿದ್ದ
ಹೆಣೆ-ದು-ಕೊ-ಳ್ಳುತ್ತಾ
ಹೆಣ್ಣಾಗಿ
ಹೆಣ್ಣು
ಹೆಣ್ಣು-ಮ-ಕ್ಕಳ
ಹೆಣ್ಣು-ಮ-ಕ್ಕ-ಳ-ನ್ನೆಲ್ಲ
ಹೆಣ್ಣು-ಮ-ಕ್ಕ-ಳಿ-ಗಾಗಿ
ಹೆಣ್ಣು-ಮ-ಕ್ಕ-ಳಿಗೂ
ಹೆಣ್ಣು-ಮ-ಕ್ಕಳೂ
ಹೆಣ್ಣೊ-ಬ್ಬ-ಳನ್ನು
ಹೆತ್ತ
ಹೆದ-ರ-ಬೇ-ಕಾ-ಗಿಲ್ಲ
ಹೆದ-ರ-ಬೇ-ಕಾದ
ಹೆದ-ರ-ಬೇಡಿ
ಹೆದ-ರ-ಲಿಲ್ಲ
ಹೆದರಿ
ಹೆದ-ರಿಕೆ
ಹೆದ-ರಿ-ಕೆ-ಯಿಂ-ದಲೊ
ಹೆದ-ರಿದ
ಹೆದ-ರಿ-ದ್ದರು
ಹೆದ-ರಿ-ಸಲು
ಹೆದ-ರಿಸಿ
ಹೆದ-ರು-ವ-ವ-ರಲ್ಲ
ಹೆನ್ರಿ
ಹೆನ್ರಿಟಾ
ಹೆಪ್ಪು-ಗ-ಟ್ಟಿತ್ತು
ಹೆಪ್ಪು-ಗ-ಟ್ಟಿ-ಸುವ
ಹೆಬ್ಬಂ-ಡೆ-ಗಿಂ-ತಲೂ
ಹೆಬ್ಬ-ಯಕೆ
ಹೆಮ್ಮೆ
ಹೆಮ್ಮೆಗೆ
ಹೆಮ್ಮೆಯ
ಹೆಮ್ಮೆ-ಯಿಂದ
ಹೆಲೆನ್
ಹೆಲೆ-ನ್-ಅ-ವರ
ಹೆಲ್ಮರ್
ಹೆಲ್ಮ-ರ್ರನ್ನು
ಹೆಸರ
ಹೆಸ-ರನ್ನು
ಹೆಸ-ರನ್ನೂ
ಹೆಸ-ರನ್ನೇ
ಹೆಸ-ರಾಂತ
ಹೆಸ-ರಾ-ಗಲಿ
ಹೆಸ-ರಾ-ಗಿದೆ
ಹೆಸ-ರಾ-ಗಿದ್ದ
ಹೆಸ-ರಾದ
ಹೆಸ-ರಾ-ದ-ರೇ-ನಂತೆ
ಹೆಸ-ರಾ-ಯಿತು
ಹೆಸರಿ
ಹೆಸ-ರಿಗೆ
ಹೆಸ-ರಿ-ಟ್ಟರು
ಹೆಸ-ರಿಟ್ಟಿ
ಹೆಸ-ರಿನ
ಹೆಸ-ರಿ-ನಲ್ಲಿ
ಹೆಸ-ರಿ-ನ-ಲ್ಲಿತ್ತು
ಹೆಸ-ರಿ-ನಲ್ಲೇ
ಹೆಸ-ರಿ-ನ-ವರು
ಹೆಸ-ರಿ-ನಿಂದ
ಹೆಸ-ರಿ-ರಲಿ
ಹೆಸರು
ಹೆಸ-ರು-ಕೀರ್ತಿ
ಹೆಸ-ರು-ಗೌ-ರವ
ಹೆಸ-ರು-ಗಳನ್ನು
ಹೆಸ-ರು-ಗಳಲ್ಲಿ
ಹೆಸ-ರು-ಗಳು
ಹೇ
ಹೇಗಾ-ದ-ರಾ-ಗಲಿ
ಹೇಗಾ-ದರೂ
ಹೇಗಾ-ದೀತು
ಹೇಗಾ-ಯಿತು
ಹೇಗಿತ್ತು
ಹೇಗಿ-ತ್ತೆಂ-ದರೆ
ಹೇಗಿ-ದ್ದರೂ
ಹೇಗಿ-ದ್ದಾರೆ
ಹೇಗಿ-ದ್ದಾಳೆ
ಹೇಗಿ-ದ್ದೀ-ತೆಂದು
ಹೇಗಿ-ರ-ಬ-ಹುದು
ಹೇಗಿ-ರ-ಬೇಕು
ಹೇಗಿ-ರ-ಬೇ-ಕೆಂ-ಬುದೂ
ಹೇಗಿ-ರು-ತ್ತ-ದೆಂ-ಬು-ದನ್ನು
ಹೇಗೆ
ಹೇಗೆಂ-ದರೆ
ಹೇಗೆಂ-ಬು-ದನ್ನು
ಹೇಗೆಂ-ಬುದೇ
ಹೇಗೆ-ತಾನೆ
ಹೇಗೆ-ನ್ನಿ-ಸ-ಬೇಡ
ಹೇಗೆ-ಹಿಂ-ದೂ-ಗಳು
ಹೇಗೇ
ಹೇಗೊ
ಹೇಗೋ
ಹೇಡಿ
ಹೇಡಿ-ಗ-ಳಾ-ಗು-ವು-ದ-ರಿಂದ
ಹೇಡಿ-ತ-ನ-ದಿಂ
ಹೇಡಿಯೂ
ಹೇಮ-ಚಂದ್ರ
ಹೇಯ-ವಾ-ಗಿತ್ತು
ಹೇರ-ಬ-ಹುದು
ಹೇರಲು
ಹೇರುವ
ಹೇರು-ವುದು
ಹೇಲ್
ಹೇಲ್ರನ್ನು
ಹೇಲ್ಳನ್ನು
ಹೇಲ್ಳೊಂ-ದಿಗೆ
ಹೇಳ-ತೀ-ರದು
ಹೇಳ-ತೊ-ಡ-ಗಿ-ದರು
ಹೇಳ-ತೊ-ಡ-ಗಿ-ದರೆ
ಹೇಳ-ದಿ-ದ್ದರೆ
ಹೇಳದೇ
ಹೇಳ-ಬ-ಯ-ಸು-ತ್ತೇನೆ
ಹೇಳ-ಬ-ಯ-ಸುವ
ಹೇಳ-ಬ-ಲ್ಲರು
ಹೇಳ-ಬ-ಲ್ಲಳು
ಹೇಳ-ಬ-ಲ್ಲಿರಾ
ಹೇಳ-ಬ-ಲ್ಲುವು
ಹೇಳ-ಬಲ್ಲೆ
ಹೇಳ-ಬ-ಲ್ಲೆ-ಎ-ಲ್ಲಿ-ಯ-ವ-ರೆಗೆ
ಹೇಳ-ಬ-ಹು-ದಾದ
ಹೇಳ-ಬ-ಹುದು
ಹೇಳ-ಬ-ಹು-ದೆಂದು
ಹೇಳ-ಬಾ-ರದು
ಹೇಳ-ಬೇ-ಕಾ-ಗಿಯೇ
ಹೇಳ-ಬೇ-ಕಾ-ದ್ದನ್ನು
ಹೇಳ-ಬೇ-ಕಿಲ್ಲ
ಹೇಳ-ಬೇಕು
ಹೇಳ-ಬೇ-ಕು-ನೀವೇ
ಹೇಳ-ಬೇ-ಕೆಂ-ದರೆ
ಹೇಳ-ಬೇ-ಕೆಂ-ದಿ-ರುವು
ಹೇಳ-ಬೇ-ಕೆಂ-ದಿ-ರು-ವು-ದ-ನ್ನೆಲ್ಲ
ಹೇಳ-ಬೇ-ಕೆಂ-ದಿ-ರು-ವುದು
ಹೇಳ-ಲಾ-ಗ-ದಂ-ತಹ
ಹೇಳ-ಲಾ-ಗದು
ಹೇಳ-ಲಾ-ಗಿತ್ತು
ಹೇಳ-ಲಾ-ಗು-ತ್ತಿತ್ತು
ಹೇಳ-ಲಾ-ಗು-ವು-ದಿಲ್ಲ
ಹೇಳ-ಲಾದ
ಹೇಳ-ಲಾ-ಯಿತು
ಹೇಳ-ಲಾ-ರಂ-ಭಿ-ಸಿದ
ಹೇಳ-ಲಾ-ರಂ-ಭಿ-ಸಿ-ದರು
ಹೇಳ-ಲಾ-ರದ
ಹೇಳ-ಲಾ-ರ-ದೆ-ಹೋದ
ಹೇಳ-ಲಾ-ರ-ರೆಂಬ
ಹೇಳಲಿ
ಹೇಳ-ಲಿಲ್ಲ
ಹೇಳ-ಲಿ-ಲ್ಲವೇ
ಹೇಳ-ಲಿ-ಲ್ಲ-ವೇಕೆ
ಹೇಳಲು
ಹೇಳಲೇ
ಹೇಳ-ಲೇನು
ಹೇಳ-ಹೆ-ಸ-ರಿ-ಲ್ಲ-ದಂ-ತಾ-ದರು
ಹೇಳ-ಹೊ-ರ-ಟಾಗ
ಹೇಳಿ
ಹೇಳಿ-ಏಳಿ
ಹೇಳಿ-ಕ-ಳಿ-ಸಿದ
ಹೇಳಿ-ಕ-ಳಿ-ಸಿ-ದರು
ಹೇಳಿ-ಕ-ಳಿ-ಸಿ-ದ-ರು-ತಾವು
ಹೇಳಿ-ಕ-ಳಿ-ಸಿ-ದ್ದರೂ
ಹೇಳಿ-ಕ-ಳಿ-ಸಿ-ಬಿ-ಟ್ಟರು
ಹೇಳಿಕೆ
ಹೇಳಿ-ಕೆ-ಗಳನ್ನು
ಹೇಳಿ-ಕೆ-ಯನ್ನು
ಹೇಳಿ-ಕೆ-ಯಲ್ಲಿ
ಹೇಳಿ-ಕೊಂಡ
ಹೇಳಿ-ಕೊಂ-ಡರು
ಹೇಳಿ-ಕೊಂ-ಡಿದ್ದ
ಹೇಳಿ-ಕೊಟ್ಟ
ಹೇಳಿ-ಕೊ-ಟ್ಟರು
ಹೇಳಿ-ಕೊ-ಡು-ತ್ತಿ-ದ್ದರು
ಹೇಳಿ-ಕೊ-ಳ್ಳಲಿ
ಹೇಳಿ-ಕೊಳ್ಳಿ
ಹೇಳಿ-ಕೊಳ್ಳು
ಹೇಳಿ-ಕೊ-ಳ್ಳು-ತ್ತಿ-ದ್ದರು
ಹೇಳಿ-ಕೊ-ಳ್ಳು-ತ್ತಿ-ದ್ದ-ರುಮಾ
ಹೇಳಿ-ಕೊ-ಳ್ಳು-ತ್ತೀರಿ
ಹೇಳಿ-ಕೊ-ಳ್ಳು-ವು-ದಿಲ್ಲ
ಹೇಳಿ-ಕೊ-ಳ್ಳು-ವುದು
ಹೇಳಿ-ಟ್ಟ-ವ-ರಿ-ಗಾಗಿ
ಹೇಳಿದ
ಹೇಳಿದಂ
ಹೇಳಿ-ದಂ-ತಾ-ಯಿತು
ಹೇಳಿ-ದಂತೆ
ಹೇಳಿ-ದಂ-ತೆ-ತಮ್ಮ
ಹೇಳಿ-ದಂ-ತೆ-ಸಾ-ವನ್ನು
ಹೇಳಿ-ದಈ
ಹೇಳಿ-ದ-ಯಾ-ವುದೋ
ಹೇಳಿ-ದ-ರಂ-ತಲ್ಲ
ಹೇಳಿ-ದ-ರಲ್ಲ
ಹೇಳಿ-ದ-ರ-ಲ್ಲದೆ
ಹೇಳಿ-ದರು
ಹೇಳಿ-ದ-ರು-ಇಲ್ಲಿ
ಹೇಳಿ-ದ-ರು-ತ-ರುಣ
ಹೇಳಿ-ದ-ರು-ಧ್ಯಾ-ನ-ಕಾ-ಲ-ದಲ್ಲಿ
ಹೇಳಿ-ದ-ರು-ಭಾ-ರ-ತದ
ಹೇಳಿ-ದರೂ
ಹೇಳಿ-ದರೆ
ಹೇಳಿ-ದ-ರೆಂ-ದರೆ
ಹೇಳಿ-ದ-ರೆಂ-ಬುದು
ಹೇಳಿ-ದರೇ
ಹೇಳಿ-ದಳು
ಹೇಳಿ-ದ-ವರು
ಹೇಳಿ-ದಾಗ
ಹೇಳಿ-ದಿ-ರಂ-ತೆ-ನಾನು
ಹೇಳಿ-ದು-ದನ್ನು
ಹೇಳಿ-ದುದು
ಹೇಳಿದೆ
ಹೇಳಿ-ದೆ-ನೆಂ-ದರೆ
ಹೇಳಿ-ದೆ-ಯಾ-ದರೂ
ಹೇಳಿದ್ದ
ಹೇಳಿ-ದ್ದಂತೆ
ಹೇಳಿ-ದ್ದನ್ನು
ಹೇಳಿ-ದ್ದ-ನ್ನೆಲ್ಲ
ಹೇಳಿ-ದ್ದನ್ನೇ
ಹೇಳಿ-ದ್ದರ
ಹೇಳಿ-ದ್ದ-ರಂತೆ
ಹೇಳಿ-ದ್ದ-ರಲ್ಲಿ
ಹೇಳಿ-ದ್ದ-ರಿಂದ
ಹೇಳಿ-ದ್ದರು
ಹೇಳಿ-ದ್ದ-ರು-ನ-ರೇನ್
ಹೇಳಿ-ದ್ದರೋ
ಹೇಳಿ-ದ್ದ-ಲ್ಲದೆ
ಹೇಳಿ-ದ್ದಳು
ಹೇಳಿ-ದ್ದಾರೆ
ಹೇಳಿ-ದ್ದಿಷ್ಟೆ
ಹೇಳಿದ್ದು
ಹೇಳಿ-ದ್ದುಂಟು
ಹೇಳಿ-ದ್ದುದು
ಹೇಳಿದ್ದೆ
ಹೇಳಿ-ನಾನು
ಹೇಳಿ-ಬಿ-ಟ್ಟ-ಮೇಲೆ
ಹೇಳಿ-ಬಿ-ಟ್ಟರು
ಹೇಳಿ-ಬಿ-ಟ್ಟಿದ್ದ
ಹೇಳಿ-ಬಿ-ಡ-ಲಾರೆ
ಹೇಳಿ-ಬಿಡಿ
ಹೇಳಿ-ಬಿ-ಡುತ್ತಿ
ಹೇಳಿ-ಬಿ-ಡು-ತ್ತೇನೆ
ಹೇಳಿ-ಬಿ-ಡು-ವ-ವರೆ
ಹೇಳಿಯೇ
ಹೇಳಿ-ಯೇನು
ಹೇಳಿ-ರ-ಬ-ಹುದೆ
ಹೇಳಿ-ರ-ಬೇಕಾ
ಹೇಳಿ-ರ-ಲಿಲ್ಲ
ಹೇಳಿ-ರುವ
ಹೇಳಿ-ರು-ವಂತೆ
ಹೇಳಿ-ರು-ವುದು
ಹೇಳಿಲ್ಲ
ಹೇಳಿ-ಸ-ಬೇ-ಕೆಂದು
ಹೇಳಿ-ಸಿ-ದ್ದಲ್ಲ
ಹೇಳಿ-ಹೇಳಿ
ಹೇಳೀರಿ
ಹೇಳು
ಹೇಳುತ್ತ
ಹೇಳು-ತ್ತದೆ
ಹೇಳು-ತ್ತ-ದೆ-ಅಳು
ಹೇಳು-ತ್ತಲೇ
ಹೇಳು-ತ್ತವೆ
ಹೇಳು-ತ್ತಾನೆ
ಹೇಳು-ತ್ತಾ-ರಂ-ತೆ-ನಿ-ನ್ನನ್ನು
ಹೇಳು-ತ್ತಾರೆ
ಹೇಳು-ತ್ತಾ-ರೆ-ಅಲ್ಲಿ
ಹೇಳು-ತ್ತಾ-ರೆ-ಒ-ಳ್ಳೆ-ಯ-ವ-ನಾಗು
ಹೇಳು-ತ್ತಾ-ರೆ-ತಿ-ಳಿ-ವ-ಳಿಕೆ
ಹೇಳು-ತ್ತಾ-ರೆ-ನಾನು
ಹೇಳು-ತ್ತಾ-ರೆ-ನಿಜ
ಹೇಳು-ತ್ತಾ-ರೆ-ಮೊ-ದ-ಲ-ನೆಯ
ಹೇಳು-ತ್ತಾಳೆ
ಹೇಳುತ್ತಿ
ಹೇಳು-ತ್ತಿದೆ
ಹೇಳು-ತ್ತಿದ್ದ
ಹೇಳು-ತ್ತಿ-ದ್ದಂತೆ
ಹೇಳು-ತ್ತಿ-ದ್ದಂ-ತೆಯೇ
ಹೇಳು-ತ್ತಿ-ದ್ದ-ರ-ಲ್ಲದೆ
ಹೇಳು-ತ್ತಿ-ದ್ದರು
ಹೇಳು-ತ್ತಿ-ದ್ದ-ರು-ಒಬ್ಬ
ಹೇಳು-ತ್ತಿ-ದ್ದ-ರು-ಗು-ರು-ವಾ-ದ-ವನು
ಹೇಳು-ತ್ತಿ-ದ್ದ-ರು-ನಮ್ಮ
ಹೇಳು-ತ್ತಿ-ದ್ದಾಗ
ಹೇಳು-ತ್ತಿ-ದ್ದಾನೆ
ಹೇಳು-ತ್ತಿ-ದ್ದಾರೆ
ಹೇಳು-ತ್ತಿ-ದ್ದಾ-ರೆಯೊ
ಹೇಳು-ತ್ತಿ-ದ್ದೀರಿ
ಹೇಳು-ತ್ತಿ-ದ್ದು-ದ-ನ್ನೆಲ್ಲ
ಹೇಳು-ತ್ತಿ-ದ್ದು-ದುಂಟು
ಹೇಳು-ತ್ತಿ-ರ-ಲಿಲ್ಲ
ಹೇಳು-ತ್ತಿ-ರುವ
ಹೇಳು-ತ್ತಿ-ರು-ವ-ವರು
ಹೇಳು-ತ್ತಿ-ರು-ವು-ದಿ-ಷ್ಟೆ-ಬ-ಹು-ತೇಕ
ಹೇಳು-ತ್ತಿ-ರು-ವುದು
ಹೇಳು-ತ್ತಿಲ್ಲ
ಹೇಳುತ್ತೀ
ಹೇಳು-ತ್ತೀ-ಯಲ್ಲ
ಹೇಳು-ತ್ತೀಯೆ
ಹೇಳು-ತ್ತೀ-ರಲ್ಲ
ಹೇಳು-ತ್ತೀರಾ
ಹೇಳು-ತ್ತೀರಿ
ಹೇಳು-ತ್ತೇನೆ
ಹೇಳು-ತ್ತೇ-ನೆ-ನಿಮ್ಮ
ಹೇಳು-ತ್ತೇ-ನೆ-ಶಾ-ರೀ-ರಿ-ಕ-ವಾಗಿ
ಹೇಳುವ
ಹೇಳು-ವಂ-ತಿ-ಲ್ಲ-ನಾವು
ಹೇಳು-ವಂತೆ
ಹೇಳು-ವಂ-ತೆ-ಸ್ವ-ಲ್ಪ-ಮ-ಪ್ಯಸ್ಯ
ಹೇಳು-ವ-ವರ
ಹೇಳು-ವ-ವ-ರಲ್ಲ
ಹೇಳು-ವ-ವ-ರಿಲ್ಲ
ಹೇಳು-ವಾಗ
ಹೇಳು-ವಿರಿ
ಹೇಳು-ವುದನ್ನು
ಹೇಳು-ವು-ದ-ನ್ನೆಲ್ಲ
ಹೇಳು-ವು-ದನ್ನೇ
ಹೇಳು-ವು-ದ-ರಲ್ಲಿ
ಹೇಳು-ವು-ದ-ರಿಂದ
ಹೇಳು-ವು-ದಾ-ದರೆ
ಹೇಳು-ವು-ದಿ-ಲ್ಲವೆ
ಹೇಳು-ವು-ದಿ-ಷ್ಟೆ-ಧ-ರ್ಮವೇ
ಹೇಳು-ವುದು
ಹೇಳು-ವು-ದುಂ-ಟುಓ
ಹೇಳು-ವುದೇ
ಹೇಳು-ವೆಯಾ
ಹೇಳೋಣ
ಹೇಸಿ-ಗೆ-ಯಾ-ಗಿತ್ತು
ಹೈಕೋರ್ಟು
ಹೈದ-ರಾ-ಬಾದ್
ಹೈಸ್ಕೂ
ಹೈಸ್ಕೂ-ಲಿಗೆ
ಹೊಂಚಿಕೆ
ಹೊಂದ-ಬೇ-ಕಾದ
ಹೊಂದ-ಲಿ-ದ್ದರು
ಹೊಂದಲು
ಹೊಂದಾ-ಣಿಕೆ
ಹೊಂದಿ
ಹೊಂದಿ-ಕೆ-ಯಾಗ
ಹೊಂದಿ-ಕೊಂ-ಡಿವೆ
ಹೊಂದಿ-ಕೊಂಡು
ಹೊಂದಿ-ಕೊಂ-ಡು-ಬಿ-ಟ್ಟಿ-ದ್ದಾರೆ
ಹೊಂದಿ-ಕೊ-ಳ್ಳ-ಬೇ-ಕಾ-ಗು-ತ್ತದೆ
ಹೊಂದಿ-ಕೊಳ್ಳು
ಹೊಂದಿ-ಕೊ-ಳ್ಳುವ
ಹೊಂದಿ-ಕೊ-ಳ್ಳು-ವಂತೆ
ಹೊಂದಿ-ಕೊ-ಳ್ಳು-ವುದು
ಹೊಂದಿತ್ತು
ಹೊಂದಿದ
ಹೊಂದಿ-ದ-ವ-ನೆಂದು
ಹೊಂದಿ-ದ-ವರು
ಹೊಂದಿ-ದ-ವರೇ
ಹೊಂದಿ-ದಾಗ
ಹೊಂದಿದೆ
ಹೊಂದಿದ್ದ
ಹೊಂದಿ-ದ್ದರು
ಹೊಂದಿ-ದ್ದಳು
ಹೊಂದಿ-ದ್ದಾರೆ
ಹೊಂದಿಯೇ
ಹೊಂದಿ-ರ-ಬೇಕು
ಹೊಂದಿರಿ
ಹೊಂದಿ-ರು-ತ್ತಾನೋ
ಹೊಂದಿ-ರುವ
ಹೊಂದಿ-ರು-ವಂತೆ
ಹೊಂದಿ-ರು-ವ-ವನು
ಹೊಂದಿ-ರು-ವ-ವ-ರಿಗೆ
ಹೊಂದಿ-ರು-ವ-ವರೇ
ಹೊಂದಿ-ರು-ವು-ದಿಲ್ಲ
ಹೊಂದಿಲ್ಲ
ಹೊಂದಿ-ಸಿ-ಕೊಂಡು
ಹೊಂದಿ-ಸು-ವುದು
ಹೊಂದು
ಹೊಂದುತ್ತ
ಹೊಂದು-ತ್ತಿ-ದ್ದೇನೆ
ಹೊಂದು-ತ್ತೇನೆ
ಹೊಂದುವ
ಹೊಂದು-ವಂ-ತೆಯೇ
ಹೊಂದು-ವು-ದ-ಕ್ಕಾ-ಗಿಯೇ
ಹೊಂದು-ವು-ದಕ್ಕೆ
ಹೊಂದು-ವುದನ್ನು
ಹೊಕ್ಕು
ಹೊಗಲು
ಹೊಗ-ಳ-ಬೇಡಿ
ಹೊಗ-ಳಲಿ
ಹೊಗಳಿ
ಹೊಗ-ಳಿ-ದರೂ
ಹೊಗ-ಳಿ-ಬಿ-ಟ್ಟರೆ
ಹೊಗಳು
ಹೊಗೆ
ಹೊಚ್ಚ-ಹೊಸ
ಹೊಟ್ಟನ್ನು
ಹೊಟ್ಟೆ
ಹೊಟ್ಟೆ-ಗಳ
ಹೊಟ್ಟೆ-ಗ-ಳಿಗೆ
ಹೊಟ್ಟೆ-ಗಿ-ಲ್ಲ-ದಿ-ದ್ದರೂ
ಹೊಟ್ಟೆ-ಗಿ-ಲ್ಲದೆ
ಹೊಟ್ಟೆ-ಗಿ-ಲ್ಲವೋ
ಹೊಟ್ಟೆಗೂ
ಹೊಟ್ಟೆ-ತುಂಬ
ಹೊಟ್ಟೆ-ನೋವು
ಹೊಟ್ಟೆ-ಬಾಕ
ಹೊಟ್ಟೆಯ
ಹೊಟ್ಟೆ-ಯುರಿ
ಹೊಟ್ಟೆ-ಯೊಂ-ದಿಗೆ
ಹೊಡು-ತ್ತಾರೋ
ಹೊಡೆತ
ಹೊಡೆ-ತಕ್ಕೆ
ಹೊಡೆ-ತ-ಗ-ಳಂತೆ
ಹೊಡೆ-ತ-ಗಳು
ಹೊಡೆದ
ಹೊಡೆ-ದದ್ದು
ಹೊಡೆ-ದಾ-ಡು-ತ್ತಿದ್ದ
ಹೊಡೆ-ದಿತ್ತು
ಹೊಡೆದು
ಹೊಡೆ-ದು-ಬಿ-ಟ್ಟರು
ಹೊಡೆ-ದೋ-ಡಿ-ಸು-ತ್ತಿದೆ
ಹೊಡೆ-ಯಲು
ಹೊಡೆ-ಯಿತು
ಹೊಡೆ-ಯುವ
ಹೊಣೆ
ಹೊಣೆ-ಗಾ-ರನೂ
ಹೊಣೆ-ಗಾ-ರರು
ಹೊಣೆ-ಗಾ-ರಿ-ಕೆಯ
ಹೊಣೆ-ಗಾ-ರಿ-ಕೆ-ಯನ್ನು
ಹೊಣೆ-ಗಾ-ರಿ-ಕೆ-ಯಲ್ಲ
ಹೊಣೆ-ಗಾ-ರಿ-ಕೆ-ಯಿಂದ
ಹೊಣೆ-ಗಾ-ರಿ-ಕೆ-ಯಿತ್ತು
ಹೊಣೆ-ಯನ್ನು
ಹೊಣೆ-ಯಿತ್ತು
ಹೊತ್ತ
ಹೊತ್ತ-ವರು
ಹೊತ್ತ-ವ-ರೊಂ-ದಿಗೆ
ಹೊತ್ತಾ-ಗಿತ್ತು
ಹೊತ್ತಾ-ಗಿರ
ಹೊತ್ತಾ-ಗಿ-ರ-ಬ-ಹುದು
ಹೊತ್ತಾದ
ಹೊತ್ತಾ-ದರೂ
ಹೊತ್ತಿ-ಕೊಂಡು
ಹೊತ್ತಿ-ಗಾ-ಗಲೇ
ಹೊತ್ತಿಗೆ
ಹೊತ್ತಿ-ಗೆಲ್ಲ
ಹೊತ್ತಿ-ದ್ದಳು
ಹೊತ್ತಿನ
ಹೊತ್ತಿ-ನ-ಮೇಲೆ
ಹೊತ್ತಿ-ನಲ್ಲಿ
ಹೊತ್ತಿ-ನಲ್ಲೇ
ಹೊತ್ತಿ-ನ-ವ-ರೆಗೂ
ಹೊತ್ತಿ-ನ-ವ-ರೆಗೆ
ಹೊತ್ತಿ-ಸಿ-ಕೊಂಡು
ಹೊತ್ತಿ-ಸಿದ
ಹೊತ್ತಿ-ಸಿ-ದರೆ
ಹೊತ್ತಿ-ಸಿ-ದ್ದನ್ನು
ಹೊತ್ತಿ-ಸು-ತ್ತಿದ್ದ
ಹೊತ್ತು
ಹೊತ್ತು-ಕೊಂಡು
ಹೊತ್ತು-ಕೊ-ಳ್ಳ-ಲೇ-ಬೇಕು
ಹೊತ್ತು-ತಂದ
ಹೊತ್ತು-ಹೊ-ತ್ತಿಗೆ
ಹೊದಿಕೆ
ಹೊಮ್ಮಿ
ಹೊಮ್ಮಿತು
ಹೊಮ್ಮಿದ
ಹೊಮ್ಮಿ-ದುವು
ಹೊಮ್ಮಿ-ಬ-ರು-ತ್ತಿ-ದ್ದಂತೆ
ಹೊಮ್ಮಿ-ಸುವ
ಹೊಮ್ಮುತ್ತ
ಹೊಮ್ಮು-ತ್ತಿತ್ತು
ಹೊಮ್ಮು-ತ್ತಿದ್ದ
ಹೊಮ್ಮು-ತ್ತಿ-ರುವ
ಹೊಮ್ಮುವ
ಹೊಮ್ಮು-ವುದು
ಹೊರ
ಹೊರಕ್ಕೆ
ಹೊರ-ಗ-ಟ್ಟಲು
ಹೊರ-ಗ-ಟ್ಟು-ವಂತೆ
ಹೊರ-ಗಡೆ
ಹೊರ-ಗ-ಡೆಯ
ಹೊರ-ಗ-ಡೆ-ಯಿಂದ
ಹೊರ-ಗಿನ
ಹೊರ-ಗಿ-ನಿಂದ
ಹೊರ-ಗಿ-ನಿಂ-ದಲೇ
ಹೊರ-ಗು-ಳಿ-ದರು
ಹೊರಗೂ
ಹೊರಗೆ
ಹೊರ-ಗೆ-ಡ-ವಿದ
ಹೊರ-ಗೆ-ಡ-ವಿ-ದ್ದರು
ಹೊರ-ಗೆ-ಡ-ಹಿ-ದ-ರು-ನೋಡು
ಹೊರ-ಗೆ-ಡ-ಹು-ತ್ತಿ-ದ್ದರು
ಹೊರ-ಗೆ-ಲ್ಲೆಲ್ಲೂ
ಹೊರ-ಗೆ-ಳೆ-ಯು-ವಾಗ
ಹೊರಗೇ
ಹೊರ-ಗೋ-ಡೆಯ
ಹೊರ-ಚಾ-ಚಿ-ದುದು
ಹೊರ-ಜಿ-ಗಿ-ಯ-ಬ-ಲ್ಲಿರಿ
ಹೊರಟ
ಹೊರ-ಟದ್ದು
ಹೊರ-ಟ-ದ್ದೊಂದು
ಹೊರ-ಟರು
ಹೊರ-ಟರೆ
ಹೊರ-ಟಳು
ಹೊರ-ಟವ
ಹೊರ-ಟ-ವ-ರಲ್ಲ
ಹೊರ-ಟ-ವರು
ಹೊರ-ಟಾಗ
ಹೊರ-ಟಾ-ಗ-ಲೆಲ್ಲ
ಹೊರ-ಟಾಗಿ
ಹೊರ-ಟಿತು
ಹೊರ-ಟಿದ್ದ
ಹೊರ-ಟಿ-ದ್ದರು
ಹೊರ-ಟಿ-ದ್ದರೆ
ಹೊರ-ಟಿ-ದ್ದಳು
ಹೊರ-ಟಿ-ದ್ದಾರೆ
ಹೊರ-ಟಿ-ದ್ದಾ-ರೆಂಬ
ಹೊರ-ಟಿದ್ದು
ಹೊರ-ಟಿ-ದ್ದುದು
ಹೊರ-ಟಿ-ರು-ವ-ರೆಂಬ
ಹೊರಟು
ಹೊರ-ಟು-ನಿಂ-ತರು
ಹೊರ-ಟು-ಬಂದ
ಹೊರ-ಟು-ಬಂ-ದರು
ಹೊರ-ಟು-ಬಂ-ದಿ-ದ್ದಳು
ಹೊರ-ಟು-ಬ-ರ-ಲಿ-ದ್ದಳು
ಹೊರ-ಟು-ಬ-ರಲು
ಹೊರ-ಟು-ಬ-ರು-ವಂತೆ
ಹೊರ-ಟು-ಬ-ರು-ವು-ದಾ-ಗಿಯೂ
ಹೊರ-ಟು-ಬಿ-ಟ್ಟರು
ಹೊರ-ಟು-ಬಿ-ಟ್ಟಳು
ಹೊರ-ಟು-ಬಿ-ಟ್ಟಿ-ದ್ದರು
ಹೊರ-ಟು-ಬಿ-ಟ್ಟಿ-ದ್ದ-ರು-ಬ-ಹುಶಃ
ಹೊರ-ಟು-ಬಿ-ಟ್ಟಿ-ದ್ದಾ-ರಲ್ಲ
ಹೊರ-ಟು-ಬಿ-ಟ್ಟಿ-ದ್ದಾರೆ
ಹೊರ-ಟು-ಬಿ-ಟ್ಟಿ-ರಲ್ಲ
ಹೊರ-ಟು-ಬಿ-ಡ-ಬೇ-ಕಾ-ಗಿತ್ತು
ಹೊರ-ಟು-ಬಿ-ಡ-ಬೇಕು
ಹೊರ-ಟು-ಬಿ-ಡ-ಬೇ-ಕೆಂ-ಬುದೇ
ಹೊರ-ಟು-ಬಿ-ಡಲು
ಹೊರ-ಟು-ಬಿ-ಡು-ತ್ತಿ-ದ್ದರು
ಹೊರ-ಟು-ಬಿ-ಡು-ತ್ತಿದ್ದೆ
ಹೊರ-ಟು-ಹೋ-ಗಿ-ದ್ದರು
ಹೊರ-ಟು-ಹೋ-ಗಿ-ದ್ದಾನೆ
ಹೊರ-ಟು-ಹೋ-ಗು-ತ್ತವೆ
ಹೊರ-ಟು-ಹೋದ
ಹೊರ-ಟು-ಹೋ-ದ-ರೆಂಬ
ಹೊರ-ಟು-ಹೋ-ದಳು
ಹೊರ-ಟೇ-ಬಿ-ಟ್ಟರು
ಹೊರ-ಟೇ-ಬಿ-ಡೋ-ಣವೇ
ಹೊರ-ಡ-ದಂತಾ
ಹೊರ-ಡ-ಬ-ಹು-ದಿತ್ತು
ಹೊರ-ಡ-ಬಾ-ರದು
ಹೊರ-ಡ-ಬೇ-ಕ-ಲ್ಲವೆ
ಹೊರ-ಡ-ಬೇಕು
ಹೊರ-ಡ-ಬೇ-ಕೆಂದು
ಹೊರ-ಡ-ಲಾ-ಗ-ಲಿಲ್ಲ
ಹೊರ-ಡ-ಲಿ-ದ್ದರು
ಹೊರ-ಡ-ಲಿ-ದ್ದಾರೆ
ಹೊರ-ಡ-ಲಿ-ದ್ದೇನೆ
ಹೊರ-ಡ-ಲಿಲ್ಲ
ಹೊರ-ಡಲು
ಹೊರ-ಡ-ಲೇ-ಬೇ-ಕಾ-ಯಿತು
ಹೊರ-ಡಿ-ಸ-ಬೇಕು
ಹೊರ-ಡಿ-ಸ-ಬೇ-ಕೆಂದು
ಹೊರ-ಡಿ-ಸಿ-ಕೊಂಡು
ಹೊರ-ಡಿ-ಸಿದ
ಹೊರ-ಡಿ-ಸುವ
ಹೊರ-ಡಿ-ಸು-ವಷ್ಟ
ಹೊರ-ಡಿ-ಸು-ವು-ದೆಂದು
ಹೊರ-ಡುತ್ತ
ಹೊರ-ಡುತ್ತಿ
ಹೊರ-ಡು-ತ್ತಿದ್ದ
ಹೊರ-ಡು-ತ್ತಿ-ದ್ದಂತೆ
ಹೊರ-ಡು-ತ್ತಿ-ದ್ದಂ-ತೆಯೇ
ಹೊರ-ಡು-ತ್ತಿರು
ಹೊರ-ಡುವ
ಹೊರ-ಡು-ವಂತೆ
ಹೊರ-ಡು-ವಂದು
ಹೊರ-ಡು-ವ-ವ-ರೆಗೂ
ಹೊರ-ಡು-ವಾಗ
ಹೊರ-ಡು-ವು-ದಂತೂ
ಹೊರ-ಡು-ವು-ದಾ-ದರೂ
ಹೊರ-ಡು-ವು-ದಾ-ದ್ದ-ರಿಂದ
ಹೊರ-ಡು-ವು-ದೆಂದು
ಹೊರ-ಡು-ವು-ದೆಂದೇ
ಹೊರ-ಡು-ವುದೇ
ಹೊರ-ತಂದು
ಹೊರ-ತನ್ನಿ
ಹೊರ-ತ-ರ-ಬೇಕು
ಹೊರ-ತ-ರುವ
ಹೊರ-ತ-ರು-ವುದು
ಹೊರ-ತಾ-ಗಿ-ರ-ಲಿಲ್ಲ
ಹೊರತು
ಹೊರ-ತು-ಪ-ಡಿ-ಸಿ-ಕೊ-ಳ್ಳ-ಲಿಲ್ಲ
ಹೊರ-ತೆ-ಗೆ-ಸು-ತ್ತಿತ್ತು
ಹೊರ-ತೇ-ನ-ಲ್ಲ-ವಲ್ಲ
ಹೊರ-ದೂ-ಡಿ-ದರೂ
ಹೊರದೆ
ಹೊರ-ನ-ಡೆ-ದು-ಬಿ-ಟ್ಟರು
ಹೊರ-ನಾ-ಡು-ಗ-ಳಲ್ಲೂ
ಹೊರ-ನಾ-ಡು-ಗ-ಳಿಗೆ
ಹೊರ-ನೋ-ಟಕ್ಕೆ
ಹೊರ-ಬಂ-ದರು
ಹೊರ-ಬಂ-ದಾಗ
ಹೊರ-ಬಂ-ದಿತು
ಹೊರ-ಬಂ-ದಿ-ರ-ಬ-ಹುದು
ಹೊರ-ಬಂದು
ಹೊರ-ಬ-ರ-ದಿ-ದ್ದು-ದನ್ನು
ಹೊರ-ಬ-ರ-ಬೇ-ಕಾ-ದರೆ
ಹೊರ-ಬ-ರ-ಲಿಲ್ಲ
ಹೊರ-ಬ-ರು-ತ್ತಿತ್ತು
ಹೊರ-ಬ-ರುವ
ಹೊರ-ಬ-ರು-ವುದೇ
ಹೊರ-ಬಿ-ದ್ದಿ-ದ್ದರೆ
ಹೊರ-ಬಿ-ದ್ದುವು
ಹೊರ-ಬೀ-ಳ-ದಿ-ದ್ದರೂ
ಹೊರ-ಬೀ-ಳುವ
ಹೊರ-ಬೇ-ಕಾ-ದ-ವಳು
ಹೊರ-ಮೈ-ಯ-ನ್ನಾ-ದರೂ
ಹೊರಲು
ಹೊರ-ಳಾ-ಡು-ತ್ತಿ-ದ್ದರು
ಹೊರ-ಳು-ತ್ತಿತ್ತು
ಹೊರ-ವ-ಲ-ಯ-ದ-ಲ್ಲಿ-ರುವ
ಹೊರ-ವ-ಲ-ಯ-ದಲ್ಲೂ
ಹೊರ-ಸೂ-ಸು-ತ್ತಿತ್ತು
ಹೊರ-ಹಾ-ಕ-ಬೇ-ಕಾ-ಗು-ತ್ತದೆ
ಹೊರ-ಹಾ-ಕ-ಲಾ-ಗಿ-ದೆ-ಯೆಂಬ
ಹೊರ-ಹಾ-ಕ-ಲೇ-ಬೇ-ಕಾ-ದಲ್ಲಿ
ಹೊರ-ಹೊ-ಮ್ಮಲಿ
ಹೊರ-ಹೊಮ್ಮಿ
ಹೊರ-ಹೊ-ಮ್ಮಿದ
ಹೊರ-ಹೊ-ಮ್ಮುತ್ತ
ಹೊರ-ಹೊ-ಮ್ಮು-ತ್ತಿತ್ತು
ಹೊರ-ಹೊ-ಮ್ಮು-ತ್ತಿದ್ದ
ಹೊರ-ಹೊ-ಮ್ಮು-ತ್ತಿ-ರುವ
ಹೊರ-ಹೊ-ಮ್ಮುವ
ಹೊರ-ಹೋ-ಗಲು
ಹೊರಾಂ-ಗಣ
ಹೊರಿ
ಹೊರಿ-ಇ-ವರು
ಹೊರಿಸಿ
ಹೊರಿ-ಸಿ-ದರೆ
ಹೊರು-ವಂತೆ
ಹೊರು-ವ-ವ-ರನ್ನೂ
ಹೊರು-ವಷ್ಟು
ಹೊರೆ
ಹೊರೆ-ಯನ್ನು
ಹೊರೆ-ಯ-ಲ್ಲದ
ಹೊರೆ-ಯಾ-ಗ-ದಂತೆ
ಹೊಲಕ್ಕೆ
ಹೊಲ-ಗ-ದ್ದೆ-ಗಳು
ಹೊಲ-ಗಳಲ್ಲಿ
ಹೊಲ-ದಲ್ಲಿ
ಹೊಲಿಗೆ
ಹೊಲಿ-ದು-ಕೊಟ್ಟ
ಹೊಲಿ-ಯು-ವು-ದ-ರಲ್ಲಿ
ಹೊಲೆ-ಯನ
ಹೊಲೆ-ಯ-ರಿ-ಗಿಂ-ತಲೂ
ಹೊಲೆ-ಯರು
ಹೊಳ-ಪಿ-ರ-ಬ-ಹುದೆ
ಹೊಳಪು
ಹೊಳೆ
ಹೊಳೆ-ದಿ-ರ-ಲಿ-ಕ್ಕಿಲ್ಲ
ಹೊಳೆ-ದಿ-ರ-ಲಿಲ್ಲ
ಹೊಳೆಯ
ಹೊಳೆ-ಯ-ಲಿ-ಲ್ಲವೆ
ಹೊಳೆ-ಯಾಗಿ
ಹೊಳೆ-ಯಿ-ತು-ಅ-ನಾ-ಥ-ರಾದ
ಹೊಳೆ-ಯಿ-ತು-ಆಹ್
ಹೊಳೆ-ಯು-ತ್ತದೆ
ಹೊಳೆ-ಹೊಳೆ
ಹೊಸ
ಹೊಸ-ತ-ರಲ್ಲಿ
ಹೊಸತು
ಹೊಸ-ದಾಗಿ
ಹೊಸ-ದೊಂದು
ಹೊಸಲು
ಹೊಸ-ಹೊಸ
ಹೋ
ಹೋಂ
ಹೋಗ
ಹೋಗ-ದಂತೆ
ಹೋಗ-ದನ್ನು
ಹೋಗದೆ
ಹೋಗ-ಬ-ಹುದು
ಹೋಗ-ಬಾ-ರದು
ಹೋಗ-ಬೇ-ಕಾಗಿ
ಹೋಗ-ಬೇ-ಕಾ-ಗಿದ್ದ
ಹೋಗ-ಬೇ-ಕಾ-ಗಿ-ರು-ವುದೂ
ಹೋಗ-ಬೇ-ಕಾ-ಗಿ-ಲ್ಲ-ವಲ್ಲ
ಹೋಗ-ಬೇ-ಕಾ-ಗು-ತ್ತದೆ
ಹೋಗ-ಬೇ-ಕಾ-ದರೆ
ಹೋಗ-ಬೇ-ಕಾ-ದು-ದನ್ನು
ಹೋಗ-ಬೇ-ಕಾ-ಯಿತು
ಹೋಗ-ಬೇಕು
ಹೋಗ-ಬೇ-ಕೆಂದು
ಹೋಗ-ಬೇ-ಕೆಂಬ
ಹೋಗ-ಬೇ-ಕೆ-ನ್ನಿ-ಸಿತು
ಹೋಗ-ಲಾಡಿ
ಹೋಗ-ಲಾ-ಡಿ-ಸ-ಬೇ-ಕೆಂದು
ಹೋಗ-ಲಾ-ಡಿ-ಸಲು
ಹೋಗ-ಲಾ-ಡಿಸಿ
ಹೋಗ-ಲಾ-ಡಿ-ಸಿ-ಕೊಂಡು
ಹೋಗ-ಲಾ-ಡಿ-ಸು-ತ್ತದೆ
ಹೋಗ-ಲಾ-ಡಿ-ಸುವ
ಹೋಗ-ಲಾ-ರ-ವು-ಇವು
ಹೋಗ-ಲಾರೆ
ಹೋಗಲಿ
ಹೋಗ-ಲಿ-ಅಹೋ
ಹೋಗ-ಲಿಲ್ಲ
ಹೋಗಲು
ಹೋಗಲೂ
ಹೋಗ-ಲೇ-ಬಾ-ರ-ದೆಂದು
ಹೋಗ-ಲೇ-ಬೇಕು
ಹೋಗ-ಲೇ-ಬೇ-ಕೆನ್ನು
ಹೋಗಿ
ಹೋಗಿತ್ತು
ಹೋಗಿದೆ
ಹೋಗಿದ್ದ
ಹೋಗಿ-ದ್ದರು
ಹೋಗಿ-ದ್ದರೆ
ಹೋಗಿ-ದ್ದ-ರೆಂ-ದರೆ
ಹೋಗಿ-ದ್ದಳು
ಹೋಗಿ-ದ್ದಾಗ
ಹೋಗಿ-ದ್ದಾರೆ
ಹೋಗಿ-ದ್ದಾಳೆ
ಹೋಗಿ-ದ್ದುದು
ಹೋಗಿದ್ದೆ
ಹೋಗಿ-ಬಂದ
ಹೋಗಿ-ಬಂ-ದರು
ಹೋಗಿ-ಬಂ-ದರೂ
ಹೋಗಿ-ಬಂ-ದ-ವರ
ಹೋಗಿ-ಬ-ರಲು
ಹೋಗಿ-ಬರು
ಹೋಗಿ-ಬ-ರು-ತ್ತಿ-ದ್ದರು
ಹೋಗಿ-ಬ-ರುವ
ಹೋಗಿ-ಬಿ-ಟ್ಟಳು
ಹೋಗಿಯೂ
ಹೋಗಿ-ರ-ಬೇ-ಕೆಂಬ
ಹೋಗಿ-ರ-ಲಿಲ್ಲ
ಹೋಗಿ-ರಲು
ಹೋಗಿ-ರು-ತ್ತಾರೆ
ಹೋಗಿ-ರು-ತ್ತಿ-ದ್ದರು
ಹೋಗಿ-ರು-ವಂ-ತಿದೆ
ಹೋಗಿ-ರು-ವುದು
ಹೋಗಿಲ್ಲ
ಹೋಗಿವೆ
ಹೋಗು
ಹೋಗುತ್ತ
ಹೋಗು-ತ್ತದೆ
ಹೋಗು-ತ್ತಲೇ
ಹೋಗು-ತ್ತವೆ
ಹೋಗು-ತ್ತ-ವೆ-ಯ-ಲ್ಲವೆ
ಹೋಗು-ತ್ತಾ-ನೆಂ-ಬು-ದನ್ನು
ಹೋಗು-ತ್ತಾರೆ
ಹೋಗುತ್ತಿ
ಹೋಗು-ತ್ತಿದೆ
ಹೋಗು-ತ್ತಿದ್ದ
ಹೋಗು-ತ್ತಿ-ದ್ದಂತೆ
ಹೋಗು-ತ್ತಿ-ದ್ದಂ-ತೆಯೇ
ಹೋಗು-ತ್ತಿ-ದ್ದರು
ಹೋಗು-ತ್ತಿ-ದ್ದರೆ
ಹೋಗು-ತ್ತಿ-ದ್ದ-ವನು
ಹೋಗು-ತ್ತಿ-ದ್ದಾಗ
ಹೋಗು-ತ್ತಿ-ದ್ದಾನೆ
ಹೋಗು-ತ್ತಿ-ದ್ದಾರೆ
ಹೋಗು-ತ್ತಿದ್ದೆ
ಹೋಗು-ತ್ತಿ-ದ್ದೇವೆ
ಹೋಗು-ತ್ತಿ-ರುವ
ಹೋಗು-ತ್ತಿ-ರು-ವು-ದಾ-ಗಿಯೂ
ಹೋಗು-ತ್ತಿ-ರು-ವುದು
ಹೋಗು-ತ್ತಿ-ರು-ವೆ-ನೆಂಬ
ಹೋಗು-ತ್ತೇನೆ
ಹೋಗುವ
ಹೋಗು-ವಂ-ತಾ-ಗ-ಬೇಕು
ಹೋಗು-ವಂ-ತಾ-ದರೂ
ಹೋಗು-ವಂ-ತಿ-ರ-ಲಿಲ್ಲ
ಹೋಗು-ವಂತೆ
ಹೋಗು-ವ-ರೆಂದು
ಹೋಗು-ವ-ವ-ರೆಗೂ
ಹೋಗು-ವ-ಷ್ಟ-ರಲ್ಲಿ
ಹೋಗು-ವಾಗ
ಹೋಗುವು
ಹೋಗು-ವು-ದ-ಕ್ಕಿಂತ
ಹೋಗು-ವು-ದಕ್ಕೆ
ಹೋಗು-ವುದನ್ನು
ಹೋಗು-ವು-ದಿಲ್ಲ
ಹೋಗು-ವುದು
ಹೋಗು-ವು-ದುಂಟೆ
ಹೋಗು-ವು-ದೆಂದೂ
ಹೋಗು-ವುದೇ
ಹೋಗುವೆ
ಹೋಗೋಣ
ಹೋಗೋ-ಣ-ವೆಂದರೆ
ಹೋಟೆ-ಲಿ-ನಲ್ಲಿ
ಹೋಟೆ-ಲಿ-ನಲ್ಲೂ
ಹೋಟೆ-ಲೊಂ-ದರ
ಹೋಟೆ-ಲೊಂ-ದ-ರಲ್ಲಿ
ಹೋದ
ಹೋದಂ-ತಿತ್ತು
ಹೋದಂತೆ
ಹೋದಂ-ತೆಲ್ಲ
ಹೋದ-ದ್ದಲ್ಲ
ಹೋದದ್ದು
ಹೋದ-ಮೇಲೆ
ಹೋದರು
ಹೋದರೂ
ಹೋದರೆ
ಹೋದಲ್ಲಿ
ಹೋದಲ್ಲೆಲ್ಲ
ಹೋದಳು
ಹೋದ-ವನು
ಹೋದ-ವರೇ
ಹೋದಾಗ
ಹೋದು-ದ-ರಲ್ಲಿ
ಹೋದುವು
ಹೋದೆ
ಹೋದೆ-ನೆಂದೇ
ಹೋದೇನು
ಹೋಮ
ಹೋಮ್
ಹೋಮ್ನ
ಹೋಯಿತು
ಹೋಯಿತೇ
ಹೋರಾ
ಹೋರಾಟ
ಹೋರಾ-ಟ-ಪ್ರ-ತಿ-ಸ್ಪ-ರ್ಧೆ-ಗಳ
ಹೋರಾ-ಟ-ಗಳನ್ನು
ಹೋರಾ-ಟ-ಗಾರ
ಹೋರಾ-ಟದ
ಹೋರಾ-ಟ-ದಲ್ಲಿ
ಹೋರಾ-ಟ-ವನ್ನು
ಹೋರಾ-ಟ-ವೆಲ್ಲ
ಹೋರಾ-ಟವೇ
ಹೋರಾ-ಡ-ದಿರು
ಹೋರಾ-ಡ-ಲೇ-ಬೇಕು
ಹೋರಾಡಿ
ಹೋರಾ-ಡಿದ
ಹೋರಾ-ಡುತ್ತ
ಹೋರಾ-ಡು-ತ್ತಾರೆ
ಹೋರಾ-ಡು-ತ್ತಿದೆ
ಹೋರಾ-ಡು-ತ್ತಿ-ದ್ದೇನೆ
ಹೋರಾ-ಡು-ತ್ತಿ-ರು-ವು-ದನ್ನೇ
ಹೋರಾ-ಡು-ತ್ತೇನೆ
ಹೋರಾ-ಡುವ
ಹೋಲಿ-ಕೆ-ಗಳೇ
ಹೋಲಿ-ಕೆ-ಯಿ-ರು-ವಂ-ಥದು
ಹೋಲಿ-ಸ-ಲಾ-ಗದು
ಹೋಲಿಸಿ
ಹೋಲಿ-ಸಿ-ದರು
ಹೋಲಿ-ಸಿ-ದರೆ
ಹೋಲಿಸು
ಹೋಲಿ-ಸುತ್ತ
ಹೋಲಿ-ಸು-ತ್ತಿ-ದ್ದರು
ಹೌದ
ಹೌದು
ಹೌದು-ಆ-ರಾ-ಮ-ವಾಗಿ
ಹೌದೆ
ಹೌರಾದ
ಹೌಸ್
ಹ್ಯಾನ್ಸ್ಬ್ರೋ
ಹ್ಯಾನ್ಸ್ಬ-್ರೋಗೆ
ಹ್ಯಾನ್ಸ್ಬ-್ರೋಳ
ಹ್ಯಾನ್ಸ್ಬ-್ರೋ-ಳಂತೂ
ಹ್ಯಾನ್ಸ್ಬ-್ರೋ-ಳನ್ನು
ಹ್ಯಾನ್ಸ್ಬ-್ರೋ-ಳಿಗೂ
ಹ್ಯಾನ್ಸ್ಬ-್ರೋ-ಳಿಗೆ
ಹ್ಯಾನ್ಸ್ಬ-್ರೋಳೂ
ಹ್ಯಾಮಂ-ಡ್ಳಿಗೆ
ಹ್ಯಾರಿ-ಯೆಟ್
ಹ್ಯಾರಿ-ಯೆ-ಟ್ಟಳ
ಹ್ಯಾರಿ-ಸನ್
್ಣವಾಗಿ
್ತದ್ದಳು
}
