
\chapter{ಇಲ್ಲಿಗೂ ಬಂದನೆ ಆ ಮುದುಕ!}

\noindent

ಅದ್ವೈತಾಶ್ರಮದ ಭವಿಷ್ಯದ ಬಗ್ಗೆ ಸ್ವಾಮೀಜಿ ಸಾಕಷ್ಟು ಆಲೋಚಿಸಿದ್ದರು. ಒಂದು ದಿನ ಸ್ವಾಮಿ ಸ್ವರೂಪಾನಂದರೊಡನೆ ಮಾತನಾಡುತ್ತ, ಅಲ್ಲಿ ಆಗಬೇಕಾಗಿದ್ದ ಕೆಲಸಗಳ ಬಗ್ಗೆ ತಮ್ಮ ಯೋಜನೆ ಗಳನ್ನು ವಿವರಿಸಿ, ಅವುಗಳನ್ನೆಲ್ಲ ಶಕ್ತ್ಯುತ್ಸಾಹಗಳಿಂದ ಕೈಗೊಳ್ಳುವಂತೆ ಅವರನ್ನು ಹುರಿದುಂಬಿ ಸಿದರು. ಆಗ ಸ್ವರೂಪಾನಂದರು, “ಸ್ವಾಮೀಜಿ, ನಾನು ನನ್ನ ಕೈಯಲ್ಲಿ ಸಾಧ್ಯವಿರುವುದನ್ನೆಲ್ಲ ಖಂಡಿತ ಮಾಡುತ್ತೇನೆ. ಆದರೆ ಇತರ ಆಶ್ರಮವಾಸಿಗಳ ಸಹಕಾರವಿಲ್ಲದಿದ್ದರೆ ಏನೂ ಸಾಧ್ಯವಾಗು ವುದಿಲ್ಲ. ಪ್ರತಿಯೊಬ್ಬರೂ ಕನಿಷ್ಠಪಕ್ಷ ಮೂರು ವರ್ಷವಾದರೂ ನಿರಂತರವಾಗಿ ಇಲ್ಲಿಯೇ ನಿಲ್ಲಬೇಕಾಗುತ್ತದೆ. ಹಾಗಿದ್ದರೆ ಮಾತ್ರ ನಾವು ಏನಾದರೂ ಮಾಡಬಹುದು” ಎಂದರು. ಸ್ವಾಮೀಜಿ ಯವರಿಗೆ ಅರ್ಥವಾಯಿತು. ಅಂದು ಎಲ್ಲರೂ ಒಟ್ಟಾಗಿ ಸೇರಿದ್ದ ಸಮಯದಲ್ಲಿ ಅವರು ಈ ವಿಷಯವನ್ನು ಪ್ರಸ್ತಾಪಿಸಿ, “ಮೂರು ವರ್ಷ ಇಲ್ಲಿಯೇ ಇರಲು ಸಿದ್ಧನಿರುವೆಯಾ?” ಎಂದು ಎಲ್ಲರನ್ನೂ ಕೇಳುತ್ತ ಬಂದರು. ಪ್ರತಿಯೊಬ್ಬರೂ ಇದಕ್ಕೆ ಒಪ್ಪಿದರು–ವಿರಜಾನಂದರೊಬ್ಬರನ್ನು ಬಿಟ್ಟು. ತಮ್ಮ ಸರದಿ ಬಂದಾಗ ವಿರಜಾನಂದರು ವಿನಯದಿಂದ, ಆದರೆ ದೃಢವಾಗಿ ಹೇಳಿದರು, “ಸ್ವಾಮೀಜಿ, ನಾನು ಕೆಲಕಾಲ ಭಿಕ್ಷಾನ್ನವನ್ನವಲಂಬಿಸಿ ಏಕಾಂತದಲ್ಲಿದ್ದುಕೊಂಡು ಧ್ಯಾನ ಮಾಡ ಬೇಕೆಂದಿದ್ದೇನೆ” ಎಂದು. ಆಗ ಸ್ವಾಮೀಜಿ ಅದಕ್ಕೆ ವಿರುದ್ಧವಾಗಿ ಅವರ ಮನವೊಲಿಸಲು ಪ್ರಯತ್ನಿ ಸಿದರು: “ನೋಡು, ಸುಮ್ಮನೆ ಈ ಬಗೆಯ ಕಠಿಣ ತಪಶ್ಚರ್ಯೆಯಿಂದ ನಿನ್ನ ಆರೋಗ್ಯವನ್ನು ಕೆಡಿಸಿಕೊಳ್ಳಬೇಡ. ಬದಲಾಗಿ ನಮ್ಮ ಅನುಭವಗಳಿಂದ ಪಾಠ ಕಲಿತುಕೊ. ನಾವು ಉಗ್ರ ಸಾಧನೆಗಳಲ್ಲಿ ಮುಳುಗಿದೆವು. ಅದರ ಪರಿಣಾಮವನ್ನು ನೋಡು–ಜೀವನದ ಅತ್ಯುತ್ತಮ ಹಂತ ದಲ್ಲಿ ಆರೋಗ್ಯವನ್ನು ಕಳೆದುಕೊಂಡು, ಆ ದುರವಸ್ಥೆಯನ್ನು ಈಗಲೂ ಅನುಭವಿಸುತ್ತಿದ್ದೇವೆ. ಅಲ್ಲದೆ ನೀನು ಹೇಗೆ ತಾನೆ ಗಂಟೆಗಟ್ಟಲೆ ಧ್ಯಾನ ಮಾಡಬಲ್ಲೆ? ಮನಸ್ಸನ್ನು ಒಂದೈದು ನಿಮಿಷ ಅಥವಾ ಕೇವಲ ಒಂದು ನಿಮಿಷ ಏಕಾಗ್ರಗೊಳಿಸಲು ಸಾಧ್ಯವಾದರೆ ಸಾಕು. ಅಷ್ಟನ್ನು ಸಾಧಿಸಲು ಬೆಳಗಿನ ಹಾಗೂ ಸಂಜೆಯ ಕೆಲವು ಗಂಟೆಗಳನ್ನು ವಿನಿಯೋಗಿಸಿದರಾಯಿತು. ಉಳಿದ ಸಮಯ ವನ್ನೆಲ್ಲ ಅಧ್ಯಯನಾದಿಗಳಿಗೆ, ಇತರ ಒಳ್ಳೆಯ ಕೆಲಸಗಳಿಗೆ ಬಳಸಬೇಕು. ನನ್ನ ಶಿಷ್ಯರು ಸಾಧನೆ ಗಳಿಗಿಂತ ಕೆಲಸಕ್ಕೆ ಹೆಚ್ಚಿನ ಮಹತ್ವ ನೀಡಬೇಕು. ಕೆಲಸವೆಂಬುದು ಅವರ ಆಧ್ಯಾತ್ಮಿಕ ಸಾಧನೆಯ ಅಂಗವೇ ಆಗಬೇಕು.”

ವಿರಜಾನಂದರು ತಮ್ಮ ಗುರುವಿನ ಮಾತಿನ ಸತ್ಯವನ್ನು ಒಪ್ಪಿಕೊಂಡರಾದರೂ ನಮ್ರ ವಾಗಿಯೇ ಮತ್ತೆ ಹೇಳಿದರು, “ಆದರೆ, ನೈತಿಕ ಶಕ್ತಿಯನ್ನು ಗಳಿಸಿಕೊಳ್ಳಲೂ ಆಧ್ಯಾತ್ಮಿಕ ಶಕ್ತಿ ಯನ್ನು ಸಂರಕ್ಷಿಸಿಕೊಳ್ಳಲೂ ಸಾಧನೆ ಬೇಕೇಬೇಕಲ್ಲವೆ? ಮತ್ತು ನಿರ್ಲಿಪ್ತವಾಗಿ ಕರ್ಮ ಮಾಡಲೂ ಸಾಧನೆ ಆವಶ್ಯಕವಲ್ಲವೆ?” ಬಳಿಕ ಅವರು ಕಾರ್ಯನಿಮಿತ್ತವಾಗಿ ಎದ್ದು ಹೋದಾಗ ಸ್ವಾಮೀಜಿ ಸುತ್ತಲಿದ್ದವರ ಮುಂದೆ ಹೇಳಿದರು, “ಹೌದು, ಅವನು ಹೇಳುತ್ತಿರುವುದು ಸರಿ ಎನ್ನುವುದು ನನ್ನ ಹೃದಯಕ್ಕೆ ಗೊತ್ತಿದೆ. ಅವನ ಮಾತನ್ನು ನಾನು ಒಪ್ಪಲೇಬೇಕು. ಏಕೆಂದರೆ ನಾನೂ ಸಂನ್ಯಾಸಿಯೇ ಅಲ್ಲವೆ? ಸಂನ್ಯಾಸಿಯ ಸ್ವತಂತ್ರ ಮನೋವೃತ್ತಿಯನ್ನು ಹಾಗೂ ಧ್ಯಾನಮಯ ಜೀವನವನ್ನು ನಾನು ಮೆಚ್ಚುತ್ತೇನೆ, ಗೌರವಿಸುತ್ತೇನೆ. ಅದರ ಮಹತ್ವವನ್ನು ನಾನು ಅಲ್ಲಗಳೆಯಲಾರೆ.” ಬಳಿಕ ತಮ್ಮ ಪರಿವ್ರಾಜಕ ಜೀವನದ ಅವಧಿಯನ್ನು ಸ್ಮರಿಸುತ್ತ ನಡಿದರು, “ಮಧುಕರಿಯ ಮೇಲೆ ಜೀವಿಸುತ್ತ ಪ್ರಪಂಚದ ಆಲೋಚನೆಯೇ ಇಲ್ಲದೆ, ಮನಸ್ಸನ್ನು ಭಗವಂತನ ಮೇಲೆ ನಿಲ್ಲಿಸಿ ಅಲೆದಾಡುವ ಆ ಜೀವನ! ಆಹ್! ನಿಜಕ್ಕೂ ನನ್ನ ಜೀವನದಲ್ಲೇ ಅತ್ಯಂತ ಮಧುರವಾದ ಹಾಗೂ ಆನಂದಕರವಾದ ದಿನಗಳವು. ಈ ಸಾರ್ವಜನಿಕ ಜೀವನದ ಜಂಜಡಗಳಿಂದ ಬಿಡಿಸಿಕೊಂಡು ಏಕಾಂತದ ನಿರಾತಂಕ ಜೀವನವನ್ನು ನಡೆಸಲು ಏನನ್ನು ಬೇಕಾದರೂ ತ್ಯಾಗ ಮಾಡಬಹುದು.”

ಸ್ವಾಮೀಜಿಯವರ ಮಾತುಗಳೆಂದರೆ ನಿರಂತರ ಸ್ಫೂರ್ತಿಯ ಚಿಲುಮೆ. ಒಂದು ದಿನ ಹೀಗೆಯೇ ಮಾತನಾಡುತ್ತಿದ್ದ ಅವರು ಇದ್ದಕ್ಕಿದ್ದಂತೆ ಎದ್ದುನಿಂತು ಅತ್ತಿಂದಿತ್ತ ನಡೆಯಲಾರಂಭಿ ಸಿದರು. ಅವರ ಕಣ್ಣುಗಳು ಭಾವಾವೇಶದಿಂದ ಪ್ರಜ್ವಲಿಸತೊಡಗಿದುವು. ದೊಡ್ಡ ಸಭೆಯೊಂದ ನ್ನುದ್ದೇಶಿಸಿ ಮಾತನಾಡುತ್ತಿರುವಂತೆ ಅವರ ಧ್ವನಿ ಏರಿತು. ತಮ್ಮ ಪಾಶ್ಚಾತ್ಯ ಅನುಯಾಯಿಗಳನ್ನು ಕುರಿತು ಹೇಳತೊಡಗಿದರು, “ಓ! ನನ್ನ ಆ ಪಾಶ್ಚಾತ್ಯ ಶಿಷ್ಯರಿಗೆ ನನ್ನ ಮೇಲೆ ಅದೆಂಥ ಭಕ್ತಿ-ವಿಶ್ವಾಸ! ಅದೆಂಥ ಶ್ರದ್ಧೆ! ಅವರಲ್ಲಿ ನನ್ನ ಆಜ್ಞೆಯ ಮೇರೆಗೆ ಮೃತ್ಯುವಿನ ಬಾಯನ್ನೂ ಹೊಗಲು ಸಿದ್ಧರಾಗಿರುವವರು ಹಲವಾರು ಜನ! ಅವರೆಲ್ಲ ಎಂತಹ ನಿಷ್ಕಪಟ ಪ್ರೀತಿಯಿಂದ ಮೌನವಾಗಿ ನನ್ನ ಕಾರ್ಯ ಸಾಧಿಸಿದರು! ನನಗೆ ಸೇವೆ ಸಲ್ಲಿಸಿದರು! ನನ್ನ ಒಂದೇ ಒಂದು ಮಾತಿನಂತೆ ಎಲ್ಲವನ್ನೂ ತ್ಯಜಿಸಿ ಬರಲು ಅವರು ಸಿದ್ಧರಿದ್ದಾರೆ. ಕ್ಯಾಪ್ಟನ್ ಸೇವಿಯರ್​ರನ್ನು ನೋಡಿ! ಹೇಗೆ ಕಾರ್ಯಸಾಧನೆಯಲ್ಲಿ ಮುಳುಗಿ ಹುತಾತ್ಮನ ಮರಣವನ್ನಪ್ಪಿದರು!”

ಮಾಯಾವತಿಯ ಆ ಇಡೀ ಪರಿಸರವನ್ನು ಬಹಳವಾಗಿ ಮೆಚ್ಚಿಕೊಂಡವರು ಸ್ವಾಮೀಜಿ. ಅಲ್ಲಿದ್ದ ಒಂದು ಸರೋವರದ ಬದಿಯಲ್ಲಿ ಓಡಾಡುವುದು ಅವರಿಗೆ ತುಂಬ ಪ್ರಿಯವಾದದ್ದು. ತಮ್ಮ ಜೀವಿತಾವಧಿಯ ಕೊನೆಯ ದಿನಗಳಲ್ಲಿ ಎಲ್ಲ ಕೆಲಸಗಳನ್ನೂ ಬಿಟ್ಟು ಆ ಸರೋವರದ ಬಳಿ ಕುಳಿತು, ಪುಸ್ತಕಗಳನ್ನು ಬರೆಯುತ್ತ, ಹಾಡುಗಳನ್ನು ಹಾಡಿಕೊಂಡು ಆನಂದದಿಂದಿರುವ ಇಚ್ಛೆಯನ್ನು ಅವರೊಮ್ಮೆ ವ್ಯಕ್ತಪಡಿಸಿದರು. ಆ ದಿನಗಳಲ್ಲೇ ಒಮ್ಮೆ ಅವರು ತಮ್ಮ ಸಂಗಡಿಗ ರೊಂದಿಗೆ ಮಾಯಾವತಿಯಲ್ಲಿನ ಅತ್ಯುನ್ನತ ಶಿಖರವಾದ ಧರಮ್​ಘರವನ್ನ ಏರಿದರು. ಇದರ ಮೇಲಿನಿಂದ ಕಾಣುವ ವಿಹಂಗಮ ನೋಟವು ಅತ್ಯಂತ ಮನೋಹರವಾದದ್ದು. ಈ ನೋಟದಿಂದ ಸ್ವಾಮೀಜಿ ಎಷ್ಟು ಸಂತುಷ್ಟರಾದರೆಂದರೆ, ಅಲ್ಲೇ ಒಂದು ಜೋಪಡಿಯನ್ನು ಕಟ್ಟಿಕೊಂಡು ಧ್ಯಾನ ಮಗ್ನರಾಗಿದ್ದುಬಿಡುವ ಆಕಾಂಕ್ಷೆ ಅವರಲ್ಲಿ ತಾನೇತಾನಾಗಿ ಸ್ಫುರಿಸಿತು.

ಅದ್ವೈತಾಶ್ರಮದಲ್ಲಿ ಯಾವುದೇ ಬಗೆಯ ಬಾಹ್ಯಪೂಜೆಯಾಗಲಿ ದ್ವೈತಸಾಧನೆಯಾಗಲಿ ನಡೆಯಬಾರದೆಂಬುದು ಸ್ವಾಮೀಜಿಯವರ ಸ್ಪಷ್ಟ ಆದೇಶವಾಗಿತ್ತು. ಈ ಆಶ್ರಮದ ಮೂಲೋ ದ್ದೇಶವೇ ಅದ್ವೈತಭಾವವನ್ನು ಅದರ ಅತ್ಯಂತ ಪರಿಶುದ್ಧ ರೂಪದಲ್ಲಿ ಸಾಧನೆ ಮಾಡಿ ಸಾಕ್ಷಾತ್ಕರಿಸಿ ಕೊಳ್ಳಲು ನೆರವಾಗುವುದಾಗಿತ್ತು. ಎರಡು ವರ್ಷದ ಹಿಂದೆ ಮಠದ ಸ್ಥಾಪನೆಯಾದಾಗ ಸ್ವಾಮೀಜಿ ಅದರ ಮೇಲ್ವಿಚಾರಕರಿಗೆ ಈ ವಿಷಯವನ್ನು ದ್ವಂದ್ವಾರ್ಥಕ್ಕೆಡೆಯಿಲ್ಲದಂತೆ ಸ್ಪಷ್ಟಪಡಿಸಿದ್ದರು. ಅಲ್ಲದೆ, ಕಾಲಕ್ರಮದಲ್ಲಿ ಈ ಅದ್ವೈತ ಸಾಧನೆಯ ಬಿಗಿಯು ಸಡಿಲಗೊಂಡು ಇತರ ಭಾವನೆಗಳು ನುಸುಳಬಹುದಾದ್ದನ್ನೂ ಅವರು ಅಂದೇ ಮುಂಗಂಡು, ಆ ಬಗ್ಗೆ ಎಚ್ಚರಿಕೆ ನೀಡಿದ್ದರು. ಇಷ್ಟೆಲ್ಲ ಆದರೂ, ಕೆಲವು ಆಶ್ರಮವಾಸಿಗಳು ಬಹಳವಾಗಿ ಬಯಸಿದ್ದರಿಂದ, ಒಂದು ಕೋಣೆಯಲ್ಲಿ ಶ್ರೀರಾಮಕೃಷ್ಣರ ಭಾವಚಿತ್ರವನ್ನಿಟ್ಟು ಪೂಜಾಮಂದಿರವನ್ನು ಸ್ಥಾಪಿಸಿಲಾಗಿತ್ತು. ಒಂದು ದಿನ ಸ್ವಾಮೀಜಿ ಅಕಸ್ಮಾತ್ತಾಗಿ ಈ ಕೋಣೆಯೊಳಗೆ ಹೋದರು. ನೋಡುತ್ತಾರೆ–ಧೂಪ ದೀಪ ಪುಷ್ಪಾದಿಗಳಿಂದ ವಿಧ್ಯುಕ್ತ ಪೂಜೆ ನಡೆಯುತ್ತಿದೆ! ಸ್ವಾಮೀಜಿ ಚಕಿತರಾದರೂ ಆಗ ಏನೂ ಮಾತನಾಡಲಿಲ್ಲ. ಆದರೆ ಸಂಜೆ ಎಲ್ಲರೂ ಅಗ್ಗಿಷ್ಟಿಕೆಯ ಸುತ್ತ ಕುಳಿತಿದ್ದಾಗ, ಅದ್ವೈತಾಶ್ರಮದಲ್ಲಿ ದ್ವೈತಭಾವದ ಈ ವೈಧೀ ಪೂಜೆ ಮಾಡುವುದನ್ನು ತೀವ್ರವಾಗಿ ಖಂಡಿಸಿ ಮಾತನಾಡಿದರು. ಈ ಮಠದ ಸ್ಥಾಪನೆಯ ಸಂಬಂಧವಾಗಿ ತಾವು ಹಿಂದೆ ಹೇಳಿದ್ದನ್ನೇ ಮತ್ತೊಮ್ಮೆ ಹೇಳಿದರು–“ಇಲ್ಲಿ ಧರ್ಮದ ಅತ್ಯುನ್ನತವೂ ಸೂಕ್ಷ್ಮವೂ ಆದ ಅಂಶಕ್ಕೆ ಮಾತ್ರ ಗಮನ ಕೊಡಬೇಕು. ಎಂದರೆ, ಧ್ಯಾನ, ಶಾಸ್ತ್ರಾಧ್ಯಯನ ಮತ್ತು ದ್ವೈತಭಾವನೆಗಳೆಂಬ ದೌರ್ಬಲ್ಯದಿಂದ ಮುಕ್ತವಾದ ಅತ್ಯುನ್ನತ ಅದ್ವೈತ ಸಿದ್ಧಾಂತದ ಆಚರಣೆ ಹಾಗೂ ಬೋಧನೆ–ಇವುಗಳು ಮಾತ್ರ ಇಲ್ಲಿ ನಡೆಯಬೇಕಾದಂಥವು. ಈ ಆಶ್ರಮವು ಅದ್ವೈತಕ್ಕೆ–ಕೇವಲ ಅದ್ವೈತಕ್ಕೆ–ಮೀಸಲಾಗಿದೆ. ಆದ್ದರಿಂದ ನನಗೆ ಈ ರೀತಿ ಮಾತನಾಡುವ ಹಕ್ಕಿದೆ.”

ದ್ವೈತಭಾವದ ಈ ವೈಧೀ ಪೂಜೆಯನ್ನು ಸ್ವಾಮೀಜಿ ಖಡಕ್ಕಾಗಿ ಖಂಡಿಸಿದರಾದರೂ, ಕೂಡಲೇ ಆ ಪಟವನ್ನು ತೆಗೆದು ಪೂಜಾದಿಗಳನ್ನು ನಿಲ್ಲಿಸಿಬಿಡುವಂತೆ ಅವರು ಆಜ್ಞೆ ಮಾಡಲಿಲ್ಲ. ಪೂಜೆ ಮಾಡುತ್ತಿದ್ದವರ ಭಾವನೆಗಳನ್ನು ನೋಯಿಸುವುದು ಅವರ ಉದ್ದೇಶವಾಗಿರಲಿಲ್ಲ. ಹಾಗೆ ಮಾಡಿ ದ್ದರೆ ಅದು ಅಧಿಕಾರದ ಚಲಾವಣೆಯಾಗುತ್ತಿತ್ತು. ಆಶ್ರಮವಾಸಿಗಳು ತಮ್ಮ ತಪ್ಪನ್ನು ಕಂಡು ಕೊಂಡು ಸರಿಪಡಿಸಿಕೊಳ್ಳಬೇಕು ಎಂಬುದಷ್ಟೇ ಅವರ ಉದ್ದೇಶವಾಗಿತ್ತು.

ಅಂತೂ ಸ್ವಾಮೀಜಿಯವರ ಅಚಲ ನಿಲುವಿನಿಂದಾಗಿ ಪೂಜಾದಿಗಳು ನಿಂತು ಹೋದುವು. ಅಲ್ಲದೆ ಮರು ವರ್ಷ ಮಾರ್ಚಿಯಲ್ಲಿ ಆ ಪೂಜಾಗೃಹವನ್ನೇ ಖಾಲಿಮಾಡಲಾಯಿತು. ಬೇಲೂರು ಮಠಕ್ಕೆ ಹಿಂದಿರುಗಿದ ಮೇಲೆ ಈ ಘಟನೆಯ ಬಗ್ಗೆ ಮಾತನಾಡುತ್ತ ಸ್ವಾಮೀಜಿ ಹೇಳುತ್ತಾರೆ, “ಶ್ರೀರಾಮಕೃಷ್ಣರ ಬಾಹ್ಯಪೂಜೆಯಿಲ್ಲದಂತಹ ಒಂದು ಕೇಂದ್ರವಾದರೂ ಇರಲಿ ಎಂದು ನಾನು ಆಲೋಚಿಸಿದ್ದೆ. ಆದರೆ, ಅಲ್ಲಿಗೆ ಹೋಗಿ ನೋಡಿದರೆ ಈ ಮುದುಕ ಆಗಲೇ ಅಲ್ಲಿಗೂ ಹೋಗಿ ಸೇರಿಕೊಂಡು ಬಿಟ್ಟಿದ್ದಾನೆ! ಇರಲಿ, ಇರಲಿ.”

ಮುಂದೆ ಸ್ವಾಮೀಜಿ ತೀರಿಹೋದ ಮೇಲೆ, ಅದ್ವೈತಾಶ್ರಮದಲ್ಲಿದ್ದ ಸ್ವಾಮಿ ವಿಮಲಾನಂದರು ಈ ಬಗ್ಗೆ ಮತ್ತೊಮ್ಮೆ ಆಲೋಚಿಸಿದರು. ತಮ್ಮ ಒಲವು ದ್ವೈತದ ಕಡೆಗೇ ಇರುವಾಗ ತಾವು ಅದ್ವೈತಾಶ್ರಮದ ಸದಸ್ಯರೆಂದು ಕರೆದುಕೊಳ್ಳುವುದು ಸರಿಯೆ? ತಮ್ಮ ಈ ಸಂದೇಹವನ್ನು ಅವರು ತಮ್ಮೆಲ್ಲರ ಪಾಲಿನ ಜೀವಂತ ದೇವರಾದ ಶ್ರೀಮಾತೆ ಶಾರದಾದೇವಿಯರ ಮುಂದಿಟ್ಟರು. ಮತ್ತು, ಶ್ರೀಮಾತೆಯವರು ತಮ್ಮ ಅಭಿಪ್ರಾಯವನ್ನೇ ಸಮರ್ಥಿಸುತ್ತಾರೆಂದು ಅವರು ನಿರೀಕ್ಷಿಸಿದ್ದರು. ಆದರೆ ಜಯರಾಂಬಾಟಿಯಿಂದ ಬಂದ ಪತ್ರದಲ್ಲಿ ಅವರಿಗೊಂದು ಆಶ್ಚರ್ಯ ಕಾದಿತ್ತು. ಶ್ರೀಮಾತೆಯವರು ಬರೆದಿದ್ದರು:

“... ಯಾರು ನಮ್ಮ ಗುರುವಾಗಿದ್ದಾರೆಯೋ ಅವರು ಸ್ವಯಂ ಅದ್ವೈತವೇ ಆಗಿದ್ದಾರೆ. ನೀವೆಲ್ಲರೂ ಅವರ ಶಿಷ್ಯರಾದ್ದರಿಂದ ನೀವೂ ಅದ್ವೈತಿಗಳೇ. ನಾನು ಖಡಾಖಂಡಿತವಾಗಿ ಹೇಳು ತ್ತೇನೆ–ನೀವು ಅದ್ವೈತವಾದಿಗಳೆಂಬುದೇ ನಿಜ.

“ಶ್ರೀಮತಿ ಸೇವಿಯರ್​ಗೆ ನನ್ನ ಆಶೀರ್ವಾದಗಳನ್ನು ತಿಳಿಸಿ. ನಿಮ್ಮೆಲ್ಲರಿಗೂ ನನ್ನ ಆಶೀರ್ವಾದಗಳು... ”

ಮಾಯಾವತಿಯಲ್ಲಿ ಸ್ವಾಮೀಜಿಯವರ ದರ್ಶನಾರ್ಥಿಗಳಾಗಿ ಸುತ್ತಮುತ್ತಲ ಹಳ್ಳಿಗಳಿಂದ, ಊರುಗಳಿಂದ ಹಲವಾರು ಜನ ಬರುತ್ತಿದ್ದರು. ಜನವಸತಿಯಿಂದ ಅತಿ ದೂರದಲ್ಲಿದ್ದ ಈ ಸ್ಥಳಕ್ಕೂ ಜನ ಹುಡುಕಿಕೊಂಡು ಬರುತ್ತಿದ್ದುದೊಂದು ವಿಶೇಷವೇ.

ಸ್ವಾಮೀಜಿ ಮಾಯಾವತಿಯಲ್ಲಿದ್ದಾಗ ಅವರ ಹಳೆಯ ಸ್ನೇಹಿತರಾದ ಲಾಲಾ ಬದರೀ ಸಾಹರು ಅವರನ್ನು ಆಲ್ಮೋರಕ್ಕೆ ತಮ್ಮ ಅತಿಥಿಯಾಗಿ ಬರುವಂತೆ ಆಹ್ವಾನಿಸಿದರು. ಆದರೆ ಸ್ವಾಮೀಜಿ ಯವರಿಗೆ ಆ ಪ್ರಯಾಣದ ಶ್ರಮವನ್ನು ಸಹಿಸಿಕೊಳ್ಳಲು ಸಾಧ್ಯವಿಲ್ಲದ್ದರಿಂದ, ಸಾಹರನ್ನೇ ಮಾಯಾವತಿಗೆ ಆಹ್ವಾನಿಸಿದರು. ವಯೋವೃದ್ಧರಾದ ಬದರೀ ಸಾಹರು ಅದಕ್ಕೊಪ್ಪಿ ತಮ್ಮ ಕಿರಿಯ ಸೋದರನಾದ ಮೋಹನಲಾಲನೊಂದಿಗೆ ಮಾಯಾವತಿಗೆ ಬಂದರು. ಅವರ ಮತ್ತೊಬ್ಬ ಸೋದರ ಗೋವಿಂದ ಸಾಹನೂ ಇಲ್ಲಿಯೇ ಉಳಿದುಕೊಂಡಿದ್ದ.

ಮಾಯಾವತಿಯಲ್ಲಿ ಸ್ವಾಮೀಜಿಯವರ ಶಿಷ್ಯರು ಅವರ ಮೇಲಿನ ಅತಿಶಯ ಪ್ರೀತಿಯಿಂದ ಅವರಿಗೆ ಸಕಲ ವಿಧದಲ್ಲಿಯೂ ಸೇವೆ ಮಾಡುತ್ತಿದ್ದರು. ಒಂದು ದಿನ ಮಧ್ಯಾಹ್ನದ ಅಡಿಗೆ ತಯಾ ರಾಗುವುದು ತೀರಾ ತಡವಾಯಿತು. ಈ ಅಲಕ್ಷ್ಯವನ್ನು, ಸಮಯಪ್ರಜ್ಞೆಯಿಲ್ಲದಿರುವುದನ್ನು ಕಂಡು ಸ್ವಾಮೀಜಿ ಸಿಟ್ಟಿಗೆದ್ದರು. ಪ್ರತಿಯೊಬ್ಬನನ್ನೂ ಚೆನ್ನಾಗಿ ಬೈದರು. ಎಲ್ಲರಿಗೂ ಬೈದ ಮೇಲೆ, ಒಳಗೆ ಅಡಿಗೆ ಮಾಡುತ್ತಿದ್ದ ವಿರಜಾನಂದರಿಗೂ ಬೈಯಲು ಅಲ್ಲಿಗೇ ಹೋದರು. ಆದರೆ ಹೊಗೆ ತುಂಬಿದ್ದ ಅಡಿಗೆಮನೆಯಲ್ಲಿ ವಿರಜಾನಂದರು ಒಂದೇ ಸಮನೆ ಒಲೆ ಊದುತ್ತ ತಮ್ಮ ಕೈಲಾದ ಪ್ರಯತ್ನವನ್ನೆಲ್ಲ ಮಾಡುತ್ತಿದ್ದುದನ್ನು ಕಂಡು, ಒಂದು ಮಾತನ್ನೂ ಆಡದೆ ಹೊರಗೆ ಬಂದು ಬಿಟ್ಟರು. ಅವರು ಇನ್ನೂ ಸ್ವಲ್ಪ ಹೊತ್ತು ಅಲ್ಲೇ ನಿಂತಿದ್ದರೆ ಅವರ ಕೋಪದ ಕಾವಿಗೇ ಅಡಿಗೆಯಾಗುತ್ತಿತ್ತೇನೋ. ಅಂತೂ ಎಷ್ಟೋ ಹೊತ್ತಾದ ಮೇಲೆ ಅಡಿಗೆ ಸಿದ್ಧವಾಯಿತು. ವಿರಜಾನಂದರು ತಟ್ಟೆಯಲ್ಲಿ ಬಡಿಸಿಕೊಂಡು ಸ್ವಾಮೀಜಿಯವರ ಕೋಣೆಗೆ ತಂದರು. ಅದನ್ನು ಕಂಡು ಸ್ವಾಮೀಜಿ, “ತೆಗೆದುಕೊಂಡು ಹೋಗದನ್ನು; ನನಗೆಂಥದೂ ಬೇಕಾಗಿಲ್ಲ” ಎಂದರು. ಆದರೆ ತಮ್ಮ ಗುರುವಿನ ಸ್ವಭಾವವನ್ನರಿತಿದ್ದ ವಿರಜಾನಂದರು ತಟ್ಟೆಯನ್ನು ಅವರ ಮುಂದಿಟ್ಟು ಮೌನವಾಗಿ ನಿಂತುಕೊಂಡರು. ಆಗ ಸ್ವಾಮೀಜಿ ಪುಟ್ಟ ಬಾಲಕನಂತೆ ಊಟ ಮಾಡಲಾರಂಭಿ ಸಿದರು. ಅಡಿಗೆ ನಾಲಿಗೆಗೆ ತಗುಲಿತೋ ಇಲ್ಲವೋ, ಅವರ ಕೋಪವೆಲ್ಲ ಮಾಯವಾಯಿತು. ಮುಖ ಪ್ರಸನ್ನವಾಯಿತು. ಅಡಿಗೆಯನ್ನು ಬಾಯ್ತುಂಬ ಹೊಗಳಿ ಹೊಟ್ಟೆ ತುಂಬ ಊಟಮಾಡಿದರು. ಊಟ ಮಾಡುವಾಗ ನಗುತ್ತ ನುಡಿದರು, “ಆಗ ನನಗೆ ಅಷ್ಟೊಂದು ಸಿಟ್ಟು ಬಂದದ್ದು ಏಕೆ ಅಂತ ಈಗ ಅರ್ಥವಾಯಿತು. ನನಗೆ ಭಯಂಕರ ಹಸಿವಾಗಿಬಿಟ್ಟಿತ್ತು!”

ಶಿಷ್ಯರು ಗುರುವಿನ ಸೇವೆ ಮಾಡುವ ಹಿಂದೂ ಆದರ್ಶವನ್ನು ಅರ್ಥಮಾಡಿಕೊಳ್ಳಲು ಒಬ್ಬ ಪಾಶ್ಚಾತ್ಯನಿಗೆ ಎಷ್ಟು ಕಷ್ಟವಾಗಬಹುದೆಂದು ಊಹಿಸಿ ಸ್ವಾಮೀಜಿಯವರು ತಮ್ಮೊಂದಿಗಿದ್ದ ಅಮೆರಿಕನ್ ಶಿಷ್ಯರಾದ ಬ್ರಹ್ಮಚಾರಿ ಅಮೃತಾನಂದರಿಗೊಮ್ಮೆ ಹೇಳಿದರು, “ಇವರೆಲ್ಲ ನನಗೆ ಹೇಗೆ ಸೇವೆ ಮಾಡುತ್ತಾರೆ ನೋಡಿದೆಯಾ? ಒಬ್ಬ ಪಾಶ್ಚಾತ್ಯನಿಗೆ ಈ ಭಕ್ತಿಯೆಲ್ಲ ಗುಲಾಮಗಿರಿ ಯಂತೆ ಕಾಣಬಹುದು. ಅಲ್ಲದೆ ನಾನೂ ಕೂಡ ಅವರ ಸೇವೆಯನ್ನು ಹೇಗೆ ಯಾವ ಸಂಕೋಚವೂ ಇಲ್ಲದೆ ಸ್ವೀಕರಿಸುತ್ತೇನೆಂಬುದನ್ನು ಕಂಡು ನಿನಗೆ ಆಘಾತವಾಗಬಹುದು. ಆದರೆ ನೀನು ಭಾರ ತೀಯ ದೃಷ್ಟಿಕೋನವನ್ನು ಅರ್ಥಮಾಡಿಕೊಳ್ಳಬೇಕು. ಆಗ ನಿನಗೆ ಎಲ್ಲವೂ ಸ್ಪಷ್ಟವಾಗುತ್ತದೆ. ಗುರುವಿನ ಬಗ್ಗೆ ಶಿಷ್ಯನಿಗಿರುವ ಸ್ವಯಂಪ್ರೇರಿತ ಭಕ್ತಿಯೆಂದರೆ ಇದೇ. ಗುರುವಿನ ಸೇವೆಯೆನ್ನು ವುದು ಶಿಷ್ಯನು ಆಧ್ಯಾತ್ಮಿಕವಾಗಿ ಮುಂದುವರಿಯುವ ಒಂದು ವಿಧಾನ.”

ಭಯಂಕರ ಹಿಮಪಾತದಿಂದಾಗಿ ಹೆಚ್ಚಿನ ವೇಳೆಯನ್ನೆಲ್ಲ ಕೋಣೆಯೊಳಗೇ ಕಳೆಯಬೇಕಾ ಗಿದ್ದುದರಿಂದ ಸ್ವಾಮೀಜಿಯವರಿಗೆ ಮಾಯಾವತಿಯ ವಾಸ್ತವ್ಯ ಬೇಸರ ತರತೊಡಗಿತು. ಆಗ ಬೇಕಾದ ಕಾರ್ಯ ಬಹಳಷ್ಟಿರುವಾಗ ವೃಥಾ ಕುಳಿತಿರಲು ವ್ಯವಧಾನವೆಲ್ಲಿ? ಆದ್ದರಿಂದ ಅವರು ಜನವರಿ ೧೭ರಂದು ಅಲ್ಲಿಂದ ಹೊರಟು ನಿಂತರು. ಆದರೆ ಆ ಮಂಜಿನಲ್ಲಿ ಪ್ರಯಾಣ ಮಾಡಲು ಕೂಲಿಗಳು ಸಿದ್ಧರಿರಲಿಲ್ಲ. ಆದ್ದರಿಂದ ಕೂಲಿಗಳನ್ನು ಹೊಂದಿಸುವುದು ಬಹಳ ಕಷ್ಟವಾಯಿತು. ಸಂಜೆಯಾದರೂ ಕೂಲಿಗಳನ್ನು ಕಲೆಹಾಕುವುದು ಸಾಧ್ಯವಾಗದ್ದನ್ನು ಕಂಡು ಸ್ವಾಮೀಜಿ ಆತಂಕ ಗೊಂಡರು. ಆಗ ವಿರಜಾನಂದರು, “ನೀವೇನೂ ಚಿಂತಿಸಬೇಡಿ ಸ್ವಾಮೀಜಿ. ಕೂಲಿಗಳು ಸಿಗದಿದ್ದರೆ ಕಡೆಗೆ ನಾವೇ ನಿಮ್ಮನ್ನು ಹೇಗಾದರೂ ಮಾಡಿ ಬೇಲೂರು ಮಠಕ್ಕೆ ತಲುಪಿಸುತ್ತೇವೆ” ಎಂದರು. ಇದನ್ನು ಕೇಳಿ ಸ್ವಾಮೀಜಿ, “ಓಹೋ! ನೀವೆಲ್ಲ ಸೇರಿ ನನ್ನನ್ನು ಕಣಿವೆಯೊಳಗೆ ಎತ್ತಿಹಾಕಲು ಹೊಂಚಿಕೆ ಹಾಕುತ್ತಿದ್ದೀರಲ್ಲವೆ!” ಎಂದು ಗಟ್ಟಿಯಾಗಿ ನಕ್ಕರು.

ಅಂತೂ ಮರುದಿನ ಮಧ್ಯಾಹ್ನ ಸ್ವಾಮೀಜಿ ಹಾಗೂ ಅವರ ಸಂಗಡಿಗರು ಮಾಯಾವತಿಯಿಂದ ಹೊರಟರು. ಮರುಪ್ರಯಾಣವನ್ನು ಬೇರೊಂದು ಮಾರ್ಗವಾಗಿ ಯೋಜಿಸಲಾಗಿತ್ತು. ಮೊದಲ ನೆಯ ರಾತ್ರಿ ಚಂಪಾವತ್ ಎಂಬಲ್ಲಿನ ಪ್ರವಾಸಿ ಬಂಗಲೆಯಲ್ಲಿ ಇಳಿದುಕೊಂಡರು.

ಇಲ್ಲಿ ಸ್ವಾಮೀಜಿ ತಮ್ಮ ಸಂಗಡಿರೊಂದಿಗೆ ಮಾತನಾಡುವಾಗ ಶ್ರೀರಾಮಕೃಷ್ಣರ ಅದ್ಭುತ ಅಂತರ್ದೃಷ್ಟಿಯ ಪ್ರಸ್ತಾಪ ಬಂದಿತು. ವ್ಯಕ್ತಿಗಳನ್ನು ನೋಟಮಾತ್ರದಿಂದಲೇ ಅಳೆದುಬಿಡುವ ಶ್ರೀರಾಮಕೃಷ್ಣರ ಸಾಮರ್ಥ್ಯವನ್ನು ಬಣ್ಣಿಸುತ್ತ ಸ್ವಾಮೀಜಿ ಭಾವೋದ್ದೀಪ್ತರಾದರು. ಅವರ ಆ ದೈವೀಗುಣದ ಬಗ್ಗೆ ಸ್ವಾಮೀಜಿ ನುಡಿದರು–“ಗುರುಮಹಾರಾಜರು ಮುನ್ನುಡಿದಿದ್ದ ಅಥವಾ ಮುಂಗಂಡಿದ್ದ ಪ್ರತಿಯೊಂದು ಮಾತೂ ಸತ್ಯವಾಗಿದೆ. ವ್ಯಕ್ತಿಗಳ ವಿಚಾರದಲ್ಲೂ ಅಷ್ಟೆ. ಅವರು ಯಾರ್ಯಾರ ವಿಚಾರದಲ್ಲಿ ಏನೇನು ಹೇಳಿದ್ದರೋ ಅದೆಲ್ಲವೂ ನಿಜವಾಗಿರುವುದನ್ನು ನಾನು ಕಂಡಿದ್ದೇನೆ. ಆದ್ದರಿಂದ, ನನ್ನ ಗುರುಭಾಯಿಗಳೊಂದಿಗಿನ ನನ್ನ ಸಂಬಂಧವು, ಗುರುಮಹಾ ರಾಜರು ಅವರ ಬಗ್ಗೆ ಏನು ಹೇಳಿದ್ದರೋ ಅದರಿಂದ ತುಂಬ ಪ್ರಭಾವಿತವಾಗಿದೆ. ಯಾರನ್ನು ರಾಮಕೃಷ್ಣರು ಈಶ್ವರಕೋಟಿಗಳೆಂದು ಕರೆದಿದ್ದರೋ ಅವರೆಲ್ಲರ ಅಂತಸ್ಸತ್ವವನ್ನು ನಾನು ನನ್ನ ಸ್ವಂತ ಅಂತರ್ದೃಷ್ಟಿಯಿಂದ ಹಾಗೂ ಮತ್ತೆಮತ್ತೆ ನಡೆಸಿದ ಪರೀಕ್ಷೆಗಳಿಂದ ಮನಗಂಡಿದ್ದೇನೆ. ಆದರೆ ಈ ಈಶ್ವರಕೋಟಿಗಳೆಂಬವರ ಅಭಿಪ್ರಾಯಗಳನ್ನೂ ಕಾರ್ಯವಿಧಾನಗಳನ್ನೂ ನಾನು ಯಾವಾಗಲೂ ಒಪ್ಪದಿರಬಹುದು. ಅಥವಾ, ಕೆಲವೊಮ್ಮೆ ಅವರನ್ನು ನಿಷ್ಠುರದ ಮಾತುಗಳಿಂದ ಟೀಕಿಸಿರಬಹುದು, ಬೈದಿರಲೂಬಹುದು. ಆದರೆ ನನ್ನ ಹೃದಯದಲ್ಲಿ ಮಾತ್ರ ಅವರಿಗೆ ಬೇರೆಲ್ಲರಿ ಗಿಂತಲೂ ಉನ್ನತವಾದ ಸ್ಥಾನವನ್ನು ಕೊಟ್ಟಿದ್ದೇನೆ. ಏಕೆಂದರೆ ಯಾರ ನಿರ್ಣಯವನ್ನು ನಾನು ವೇದವಾಕ್ಯವೆಂದು ಒಪ್ಪಿಕೊಂಡಿದ್ದೇನೆಯೋ ಆ ರಾಮಕೃಷ್ಣರೇ ಅವರನ್ನೆಲ್ಲ ಆ ಸ್ಥಾನಕ್ಕೇರಿಸಿ ದ್ದಾರೆ.” ಹೀಗೆ ನುಡಿದ ಸ್ವಾಮೀಜಿ ಮತ್ತೆ ಮತ್ತೆ ಉದ್ಗರಿಸಿದರು, “ಎಲ್ಲಕ್ಕಿಂತ ಮಿಗಿಲಾಗಿ, ನಾನು (ಶ್ರೀರಾಮಕೃಷ್ಣರಿಗೆ) ನಿಷ್ಠಾವಂತನಾಗಿದ್ದೇನೆ! ನಾನು ನನ್ನ ಹೃದಯಾಂತರಾಳದವರೆಗೂ ಪ್ರಾಮಾಣಿಕನಾಗಿದ್ದೇನೆ!”

ಮರುದಿನ ಬೆಳಿಗ್ಗೆ ಸ್ವಾಮೀಜಿ ಹಾಗೂ ಅವರ ಸಂಗಡಿಗರಾದ ವಿರಜಾನಂದರು, ಶಿವಾ ನಂದರು, ಸದಾನಂದರು ಹಾಗೂ ಗೋವಿಂದ ಸಾಹ ಚಂಪಾವತ್​ನಿಂದ ಹೊರಟು ತನಕ್​ಪುರಕ್ಕೆ ಬಂದರು. ಇಲ್ಲಿಂದ ಮುಂದೆ ಬಯಲುಪ್ರದೇಶ ಪ್ರಾರಂಭ. ತನಕ್​ಪುರದಲ್ಲಿನ ಅಂಗಡಿ ಯೊಂದರಲ್ಲಿ ರಾತ್ರಿಯನ್ನು ಕಳೆದ ಪ್ರಯಾಣಿಕರು, ಕುದುರೆಗಳನ್ನೇರಿ ಪಿಲಿಭಿತ್ ಎಂಬಲ್ಲಿಗೆ ಹೊರಟರು. ದಾರಿಯಲ್ಲಿ ಸ್ವಾಮೀಜಿಯವರು ಶಿವಾನಂದರಿಗೆ ಆದೇಶಿಸಿದರು: “ನೀನು ಪಿಲಿಭಿತ್ ನಿಂದ ಏಕಾಂಗಿಯಾಗಿ ಹೊರಡಬೇಕು. ದಾರಿಯಲ್ಲಿನ ಊರುಗಳಲ್ಲಿ ಅಲ್ಲಲ್ಲಿ ಸಿಗಬಹುದಾದ ಧರ್ಮಿಷ್ಠರನ್ನು ಭೇಟಿಯಾಗಿ, ಅವರಿಂದ ಬೇಲೂರು ಮಠದ ನಿರ್ವಹಣೆಗಾಗಿ ಹಾಗೂ ಅಭಿವೃದ್ಧಿ ಗಾಗಿ ಧನಸಂಗ್ರಹ ಮಾಡುತ್ತ ಬಾ. ಈ ಕೆಲಸವನ್ನು ನೀನೊಬ್ಬನೇ ಅಲ್ಲದೆ ಮಠದ ಎಲ್ಲ ಸಂನ್ಯಾಸಿಗಳೂ ಮಾಡಬೇಕಾಗುತ್ತದೆ. ನೀವುಗಳು ಧರ್ಮಪ್ರಚಾರ ಮಾಡುತ್ತ, ಬೋಧನೆ ನೀಡುತ್ತ ಭಾರತದಲ್ಲೆಲ್ಲ ಸಂಚರಿಸಬೇಕು. ಕಡೆಯಲ್ಲಿ ಪ್ರತಿಯೊಬ್ಬರೂ ಮಠದ ನಿಧಿಗೆ ಕನಿಷ್ಠ ಪಕ್ಷ ಎರಡೆರಡು ಸಾವಿರ ರೂಪಾಯಿಗಳನ್ನು ತರಬೇಕು.” ತಮ್ಮ ನೆಚ್ಚಿನ ನಾಯಕನ ಮಾತಿಗೆ ಶಿವಾನಂದರು ತಕ್ಷಣ ಒಪ್ಪಿ ತಲೆಬಾಗಿದರು.

ಪಿಲಿಭಿತ್ತಿನಲ್ಲಿ ಸ್ವಾಮೀಜಿಯವರು ಸದಾನಂದರೊಡನೆ ಕಲ್ಕತ್ತಕ್ಕೆ ಹೋಗುವ ಟ್ರೈನನ್ನೇರಿ ದರು. ಅವರು ಪ್ರವೇಶಿಸಿದ ಎರಡನೆಯ ದರ್ಜೆಯ ಬೋಗಿಯಲ್ಲಿ ಅದಾಗಲೇ ಒಬ್ಬ ಆಂಗ್ಲ ಸೈನ್ಯಾಧಿಕಾರಿ ಕುಳಿತಿದ್ದ. ತಾನು ಕುಳಿತಿದ್ದ ಬೋಗಿಯೊಳಕ್ಕೆ ‘ಸ್ಥಳೀಯ’ರು (ಆಗಿನ ಕಾಲದಲ್ಲಿ ಭಾರತೀಯರನ್ನು ಆಂಗ್ಲರು ಕರೆಯುತ್ತಿದ್ದುದು \eng{natives} ಅಥವಾ ಸ್ಥಳೀಯರು ಎಂದು. ಈ ಶಬ್ದದಲ್ಲಿ ತಿರಸ್ಕಾರಾರ್ಥ ತಾನೇತಾನಾಗಿ ಸೇರಿಕೊಂಡಿತ್ತು.) ಪ್ರವೇಶಿಸಿದ್ದು ಆತನಿಗೆ ಸಹ್ಯವಾಗ ಲಿಲ್ಲ. ಆದರೆ ಅವರನ್ನು ಬೀಳ್ಕೊಡಲು ಅಲ್ಲಿನ ಡೆಪ್ಯುಟಿ ಕಲೆಕ್ಟರರೂ ಸೇರಿದಂತೆ ಬಹಳಷ್ಟು ಜನ ಸೇರಿಬಿಟ್ಟಿದ್ದರು. ಇವರನ್ನೆಲ್ಲ ನೋಡಿದ ಸೈನ್ಯಾಧಿಕಾರಿ ಒಳಗೊಳಗೇ ಕುದಿದ. ತಾನೀಗ ಮಾತನಾಡಿ ದರೆ ಕಷ್ಟಕ್ಕೆ ಬರುತ್ತದೆಂದು ಗೊತ್ತು. ಆದ್ದರಿಂದ ಸೀದಾ ಎದ್ದು ಸ್ಟೇಷನ್ ಮಾಸ್ಟರನ ಬಳಿಗೆ ಹೋಗಿ ಆ ‘ಅನಾಗರಿಕ ಸ್ಥಳೀಯ’ನನ್ನು ತನ್ನ ಬೋಗಿಯಿಂದ ಹೊರಗಟ್ಟುವಂತೆ ಆಜ್ಞೆ ಮಾಡಿದ. ಆ ಸ್ಟೇಷನ್ ಮಾಸ್ಟರೂ ಒಬ್ಬ ‘ಸ್ಥಳೀಯ’ನೇ. ಆದರೂ ಈ ಬಿಳಿಯನ ಜಬರದಸ್ತಿಗೆ ಹೆದರಿ ಆ ಬೋಗಿಗೆ ಹೋದ. ಸೈನ್ಯಾಧಿಕಾರಿ ದೂರದಲ್ಲೇ ಉಳಿದುಕೊಂಡ. ಸ್ಟೇಷನ್ ಮಾಸ್ಟರು ತನ್ನಿಂದ ಸಾಧ್ಯವಾದಷ್ಟು ವಿನಮ್ರತೆಯಿಂದ, ಬೇರೊಂದು ಬೋಗಿಗೆ ಹೋಗುವಂತೆ ಸ್ವಾಮೀಜಿಯವರನ್ನು ಪ್ರಾರ್ಥಿಸಿಕೊಂಡ. ಅವನಿನ್ನೂ ತನ್ನ ಮಾತನ್ನು ಮುಗಿಸಿರಲಿಲ್ಲ; ಅಷ್ಟರಲ್ಲೇ ಸ್ವಾಮೀಜಿ, “ನನಗೆ ಎದ್ದು ಹೋಗುವಂತೆ ಹೇಳಲು ಅದೆಷ್ಟು ಧೈರ್ಯ! ನಾಚಿಕೆಯಾಗುವುದಿಲ್ಲವೆ ನಿಮಗೆ?” ಎಂದು ಗರ್ಜಿಸಿದರು. ಸತ್ತೆನೋ ಕೆಟ್ಟೆನೋ ಎಂದು ಆತ ಅಲ್ಲಿಂದ ಪರಾರಿ. ಇತ್ತ ಸೈನ್ಯಾಧಿಕಾರಿ, ಇಷ್ಟುಹೊತ್ತಿಗೆ ತನ್ನ ಆಜ್ಞೆ ನೆರವೇರಿರುತ್ತದೆಂಬ ವಿಶ್ವಾಸದಿಂದ ಬಂದು ನೋಡುತ್ತಾನೆ– ಇಬ್ಬರೂ ಅಲ್ಲೇ ಕುಳಿತಿದ್ದಾರೆ! ಅವನಿಗೆ ಮೈಯೆಲ್ಲ ಉರಿದುಹೋಯಿತು. ಛಂಗನೆ ಬೋಗಿ ಯಿಂದ ಹಾರಿ ಸ್ಟೇಷನ್ ಮಾಸ್ಟರನ ಕೋಣೆಗೆ ನುಗ್ಗಿದ. ಆತ ಅಲ್ಲಿ ಇರದಿದ್ದುದನ್ನು ಕಂಡು, “ಸ್ಟೇಷನ್ ಮಾಸ್ಟರ್! ಏ ಸ್ಟೇಷನ್ ಮಾಸ್ಟರ್!” ಎಂದು ಅರಚುತ್ತ ಒಂದು ಪ್ಲಾಟ್​ಫಾರಂ ನಿಂದ ಮತ್ತೊಂದಕ್ಕೆ ಹಾರಾಡಿದ. ಆದರೆ ಈ ಸ್ಟೇಷನ್ ಮಾಸ್ಟರು ‘ಅತ್ತ ದರಿ ಇತ್ತ ಪುಲಿ’ ಎಂಬ ಪರಿಸ್ಥಿತಿಯಿಂದ ತಪ್ಪಿಸಿಕೊಂಡು ಎಲ್ಲೋ ತಲೆಮರೆಸಿಕೊಂಡಿದ್ದ. ಅಷ್ಟು ಹೊತ್ತಿಗೆ ನಿಲ್ದಾಣದ ಗಂಟೆ ಹೊಡೆಯಿತು. ಬೇರೆ ದಾರಿಗಾಣದೆ ಆ ಬಿಳಿಯ ಓಡಿ ಬಂದು ತನ್ನ ಗಂಟು ಮೂಟೆಯನ್ನೆತ್ತಿಕೊಂಡು ತಾನೇ ಬೇರೊಂದು ಬೋಗಿಗೆ ಹೋದ. ಕುಳಿತಲ್ಲಿಂದಲೇ ಎಲ್ಲವನ್ನೂ ನೋಡುತ್ತಿದ್ದ ಗುರುಶಿಷ್ಯರಿಬ್ಬರೂ ಹೊಟ್ಟೆತುಂಬ ನಕ್ಕರು.

ಕಲ್ಕತ್ತಕ್ಕೆ ಬರುವ ಹಾದಿಯಲ್ಲಿ ಸ್ವಾಮೀಜಿಯವರಿಗೆ, ತಮ್ಮ ಆಪ್ತ ಶಿಷ್ಯನಾದ ಖೇತ್ರಿಯ ಮಹಾರಾಜ ಅಜಿತ್​ಸಿಂಗ್ ಆಕಸ್ಮಿಕ ಮರಣವನ್ನಪ್ಪಿದನೆಂಬ ದಾರುಣ ವಾರ್ತೆ ತಿಳಿಯಿತು. ತನ್ನ ನೆರೆರಾಜ್ಯದ ಆಗ್ರಾದ ಸಮೀಪದಲ್ಲಿ ಅಕ್ಬರ್ ಕಟ್ಟಿಸಿದ್ದ ಎಂಬತ್ತಾರು ಅಡಿ ಎತ್ತರದ ಗೋಪುರದ ದುರಸ್ತಿ ಮಾಡಿಸುವುದಕ್ಕಾಗಿ ಅವನು ಮೇಲೆ ಹತ್ತಿದ್ದ. ಆದರೆ ಬಲವಾಗಿ ಗಾಳಿ ಬೀಸಿದ್ದರಿಂದಲೋ ಏನೋ ಆಯ ತಪ್ಪಿ ಬಿದ್ದು ತಕ್ಷಣ ಸಾವಿಗೀಡಾದ. ಸ್ವಾಮೀಜಿಯವರಿಗೆ ಇದು ದೊಡ್ಡ ಆಘಾತ ವನ್ನುಂಟುಮಾಡಿತು. ವೈಯಕ್ತಿಕವಾಗಿ ಅವರಿಗೆ ಇದೊಂದು ತುಂಬಲಾರದ ನಷ್ಟ. ಅದಕ್ಕಿಂತ ಹೆಚ್ಚಾಗಿ, ಸಂಘದ ಕಾರ್ಯಕ್ಕೂ ಇದರಿಂದ ಪೆಟ್ಟು ಬಿದ್ದಂತಾಯಿತು. ಆದರೆ ಸ್ವಾಮೀಜಿ ಬಹಳ ಕಷ್ಟಪಟ್ಟು ತಮ್ಮ ದುಃಖವನ್ನು ಹೃದಯದೊಳಗೆ ಅಡಗಿಸಿಕೊಂಡು ಕಲ್ಕತ್ತವನ್ನು ತಲುಪಿದರು.

