
\chapter{ಮಹಾಸಂಚಲನದ ಮೂಲಸ್ಥಾನ}

\noindent

೧೮೯೮ರ ಡಿಸೆಂಬರ್ ೯ನೇ ತಾರೀಕು, ಶುಕ್ರವಾರ, ಅದು ಶ್ರೀರಾಮಕೃಷ್ಣಸಂಘದ ಇತಿಹಾಸದಲ್ಲಿ ಸುವರ್ಣಾಕ್ಷರಗಳಿಂದ ಬರೆದಿಡಬೇಕಾದ ದಿನ. ಅಂದು ಬೇಲೂರಿನಲ್ಲಿ ಶ್ರೀರಾಮಕೃಷ್ಣ ಮಠದ ಪ್ರತಿಷ್ಠಾಪನೆ. ಸ್ವಾಮೀಜಿಯವರಿಗೂ ಅದು ಅತ್ಯಂತ ಆನಂದದ ದಿನ. ಏಕೆಂದರೆ ಅವರ ಬಹುವರ್ಷಗಳ ಕನಸು ಅಂದು ನನಸಾಗಲಿತ್ತು. ಈ ಶುಭ ಸಂದರ್ಭದಲ್ಲಿ ಸ್ವತಃ ಸ್ವಾಮೀಜಿ ಯವರೇ ತಮ್ಮ ಸಂನ್ಯಾಸೀ ಬಂಧುಗಳ ನೆರವಿನಿಂದ ಎಲ್ಲ ಪೂಜಾದಿ ವಿಧಿಗಳನ್ನು ನೆರವೇರಿಸಿ ದರು. ಉತ್ಸವ ತುಂಬ ಆಕರ್ಷಕವಾಗಿತ್ತು. ಸ್ವಾಮೀಜಿ ಗಂಗೆಯಲ್ಲಿ ಮಿಂದು, ನೂತನ ಕಾಷಾಯ ವಸ್ತ್ರವನ್ನು ಧರಿಸಿ, ಗರ್ಭಗುಡಿಯಲ್ಲಿಅರ್ಚಕ ಪೀಠದಲ್ಲಿ ಕುಳಿತು ಧ್ಯಾನಸ್ಥರಾದರು. ಅನಂತರ ಶ್ರೀರಾಮಕೃಷ್ಣರ ಅವಶೇಷಗಳನ್ನು ಪರಮ ಪೂಜ್ಯ ಭಾವದಿಂದ ಪೂಜಿಸಿ ಪುಷ್ಪ-ಬಿಲ್ವಪತ್ರಗಳಿಂದ ಅರ್ಚಿಸಿದರು. ಸ್ವಾಮಿ ಪ್ರೇಮಾನಂದರೇ ಮೊದಲಾದ ಸಾಧು ಸಂನ್ಯಾಸಿಗಳು ಬಾಗಿಲಿನಲ್ಲೇ ನಿಂತು ಸ್ವಾಮೀಜಿಯವರ ಪೂಜಾದಿಗಳನ್ನು ಭಕ್ತಿಭಾವದಿಂದ ವೀಕ್ಷಿಸುತ್ತಿದ್ದರು.

ಪೂಜಾದಿಗಳು ಮುಗಿದ ಮೇಲೆ ಸಂನ್ಯಾಸಿ-ಬ್ರಹ್ಮಚಾರಿಗಳೆಲ್ಲ ಸೇರಿ ಒಂದು ಮೆರವಣಿಗೆ ಯಲ್ಲಿ ಹೊರಟರು. ಮೆರವಣಿಗೆ ನೀಲಾಂಬರ ಮುಖರ್ಜಿಯವರ ತೋಟದ ಮನೆಯ ತಾತ್ಕಾಲಿಕ ಮಠದಿಂದ ಹೊರಟು, ಗಂಗೆಯ ದಡದ ಮೇಲಿನಿಂದ ಸಾಗಿ, ಹೊಸ ಮಠದ ನಿವೇಶನದ ಕಡೆಗೆ ಮುನ್ನಡೆಯಿತು. ಸ್ವಾಮೀಜಿ ತಮ್ಮ ಬಲಭುಜದ ಮೇಲೆ ಶ್ರೀರಾಮಕೃಷ್ಣದೇವರ ಅವಶೇಷಗಳ ನ್ನೊಳಗೊಂಡ ಪಾತ್ರೆಯನ್ನು ಹೊತ್ತು ಮೆರವಣಿಗೆಯ ನೇತೃತ್ವ ವಹಿಸಿದರು. ಶಂಖ ಜಾಗಟೆಗಳ ಮೊಳಗುವಿಕೆಯ ನಿನಾದ ಗಂಗಾನದಿಯಗಲಕ್ಕೂ ಪ್ರತಿಧ್ವನಿಸಿತು. ಸಾಧುಗಳೆಲ್ಲ ಆನಂದೋತ್ಸಾಹ ಭರಿತರಾಗಿದ್ದರು. ದಾರಿಯಲ್ಲಿ ಸ್ವಾಮೀಜಿ ಶರಚ್ಚಂದ್ರನಿಗೆ ಹೇಳಿದರು, “ಗುರುಮಹಾರಾಜರು ಒಮ್ಮೆ ನನಗೆ ಹೇಳಿದ್ದರು–‘ನರೇನ್, ನೀನು ನನ್ನನ್ನು ಹೆಗಲ ಮೇಲೆತ್ತಿಕೊಂಡು ಒಯ್ದು ಎಲ್ಲಿ ಬೇಕಾದರೂ ಪ್ರತಿಷ್ಠಾಪಿಸು–ಮರದ ಬುಡದಲ್ಲಾದರೂ ಸರಿಯೆ ಅಥವಾ ಒಂದು ಗುಡಿಸಲ ಲ್ಲಾದರೂ ಸರಿಯೆ, ನಾನು ಅಲ್ಲಿಯೇ ನೆಲಸುತ್ತೇನೆ’ ಎಂದು. ಅವರ ಈ ಪರಮ ಕೃಪೆಯ ಆಶ್ವಾಸನೆಯ ಮೇಲಿನ ನಂಬಿಕೆಯಿಂದಲೇ ಈಗ ನಾನು ಅವರನ್ನು ನಮ್ಮ ಹೊಸ ಮಠದ ಜಾಗಕ್ಕೆ ಕೊಂಡೊಯ್ಯುತ್ತಿರುವುದು. ಮಗೂ, ಇದನ್ನು ದೃಢವಾಗಿ ನಂಬು: ಎಲ್ಲಿಯವರೆಗೆ ಅವರ ದಿವ್ಯ ನಾಮವು ಅವರ ಅನುಯಾಯಿಗಳಲ್ಲಿ ಅವರ ಪರಿಶುದ್ಧತೆ-ದಿವ್ಯತೆ ಹಾಗೂ ಸಕಲ ಜೀವಿಗಳ ಬಗೆಗಿನ ಪ್ರೇಮದ ಮಹಾ ಆದರ್ಶಗಳನ್ನು ಸ್ಫುರಣೆಗೊಳಿಸುವುದೋ ಅಲ್ಲಿಯವರೆಗೆ ಗುರುಮಹಾ ರಾಜರು ಈ ಸ್ಥಳವನ್ನು ತಮ್ಮ ತೇಜೋಮಯವಾದ ಸನ್ನಿಧಿಯಿಂದ ಪವಿತ್ರಗೊಳಿಸುತ್ತಿರುತ್ತಾರೆ.” ಮೆರವಣಿಗೆ ಹೊಸ ನಿವೇಶನದ ಸಮೀಪಕ್ಕೆ ಬಂದಾಗ ಸ್ವಾಮೀಜಿಯವರು ಮಠದ ಭವ್ಯ ಭವಿಷ್ಯದ ಬಗೆಗಿನ ತಮ್ಮ ಕಲ್ಪನೆಯನ್ನು ಬಣ್ಣಿಸುತ್ತಾರೆ: “ಮುಂದೆ ಈ ಮಠದಲ್ಲಿ ಶ್ರೀರಾಮಕೃಷ್ಣರ ಜೀವನದಲ್ಲೇ ಆದರ್ಶಪ್ರಾಯವಾಗಿ ವ್ಯಕ್ತಗೊಂಡಂತಹ ಸಕಲ ಮತ-ಧರ್ಮಗಳ ಅದ್ಭುತ ಸಮನ್ವಯವನ್ನು ಅಭ್ಯಸಿಸಲಾಗುವುದು. ಧರ್ಮದ ಸಾರ್ವತ್ರಿಕ ಹಾಗೂ ಸಾರ್ವಕಾಲಿಕ ತತ್ತ್ವ ಗಳನ್ನು ಮಾತ್ರವೆ ಇಲ್ಲಿ ಬೋಧಿಸಲಾಗುವುದು. ಸರ್ವಧರ್ಮಸಹಿಷ್ಣುವಾದ ಈ ಕೇಂದ್ರದಿಂದ ಶಾಂತಿ ಸ್ನೇಹ ಸಮನ್ವಯಗಳ ಉಜ್ವಲ ಸಂದೇಶಗಳು ಹೊರಹೊಮ್ಮಿ ಸಮಸ್ತ ವಿಶ್ವವನ್ನೇ ಆವರಿಸಲಿವೆ.” ಆದರೆ ಈ ಕೇಂದ್ರದಲ್ಲಿಯೇ ಕಾಲಾಂತರದಲ್ಲಿ ಉದ್ಭವಿಸಬಹುದಾದಂತಹ ಒಳಪಂಗಡಗಳ ಅಪಾಯದ ಬಗ್ಗೆ ಎಚ್ಚರಿಸಲು ಸ್ವಾಮೀಜಿ ಮರೆಯಲಿಲ್ಲ.

ಈಗ ಮೆರವಣಿಗೆ ಹೊಸ ಮಠವನ್ನು ತಲುಪಿತು. ಅಲ್ಲಿ ನೆಲದ ಮೇಲೆ ಇರಿಸಿದ ವಿಶೇಷ ಪೀಠದ ಮೇಲೆ ಶ್ರೀರಾಮಕೃಷ್ಣರ ಪವಿತ್ರ ಅವಶೇಷಗಳನ್ನು ಇಟ್ಟು ಸ್ವಾಮೀಜಿ ಹಾಗೂ ಇತರೆಲ್ಲ ಸಂನ್ಯಾಸಿ-ಬ್ರಹ್ಮಚಾರಿಗಳು ದೀರ್ಘದಂಡಪ್ರಣಾಮ ಮಾಡಿದರು. ಬಳಿಕ ವಿಧ್ಯುಕ್ತ ಪೂಜಾದಿ ಗಳನ್ನು ನೇರವೇರಿಸಲಾಯಿತು. ಅನಂತರ ಅಗ್ನಿಕುಂಡದಲ್ಲಿ ಅಗ್ನಿಯನ್ನು ಸ್ಥಾಪಿಸಿ ಸಂನ್ಯಾಸಿಗಳು ಮಾತ್ರವೇ ಮಾಡಬಹುದಾದ ವಿರಜಾಹೋಮವನ್ನು ನೆರವೇರಿಸಿದರು. ತಮ್ಮ ಸೋದರ ಸಂನ್ಯಾಸಿ ಗಳ ಸಹಾಯದಿಂದ ಸ್ವಾಮೀಜಿ ತಾವೇ ಪಾಯಸಾನ್ನವನ್ನು ತಯಾರಿಸಿ ಅದನ್ನು ಶ್ರೀರಾಮಕೃಷ್ಣರಿಗೆ ನೈವೇದ್ಯಮಾಡಿದರು. ಇಲ್ಲಿಗೆ ಮಠದ ಪ್ರತಿಷ್ಠಾಪನಾ ಕಾರ್ಯ ಮುಕ್ತಾಯಗೊಂಡಿತು. ಆಗ ಸ್ವಾಮೀಜಿ ಅಲ್ಲಿ ನೆರೆದಿದ್ದವರನ್ನು ಉದ್ದೇಶಿಸಿ ಮಾತನಾಡಿದರು–“ಬಂಧುಗಳೇ, ಯುಗಾವತಾರ ನಾದ ಆ ಭಗವಂತನು ತನ್ನ ತೇಜೋಮಯ ದಿವ್ಯ ಸಾನ್ನಿಧ್ಯದಿಂದ ಈ ಸ್ಥಳದಲ್ಲಿ ಎಂದೆಂದಿಗೂ ನೆಲಸಿ ಬಹುಜನರ ಹಿತಕ್ಕಾಗಿ ಬಹುಜನರ ಸುಖಕ್ಕಾಗಿ ಇದನ್ನು ಸಕಲ ಧರ್ಮಗಳ ಸಮನ್ವಯದ ಒಂದು ವಿಶಿಷ್ಟ ಕೇಂದ್ರವನ್ನಾಗಿ–ಪುಣ್ಯಕ್ಷೇತ್ರವನ್ನಾಗಿ–ಮಾಡಲೆಂದು ಇಂದಿನ ಈ ಪವಿತ್ರ ದಿನದಂದು, ನಾವೆಲ್ಲ ನಮ್ಮ ಹೃದಯಾಂತರಾಳದಿಂದ ಪ್ರಾರ್ಥಿಸೋಣ.” ಕೂಡಲೇ ಎಲ್ಲರೂ ಅಂಜಲಿಬದ್ಧರಾಗಿ ನಿಂತು ಸ್ವಾಮೀಜಿಯವರ ಜೊತೆಗೆ ದನಿಗೂಡಿಸಿ ಭಗವಂತನಲ್ಲಿ ಪ್ರಾರ್ಥನೆ ಸಲ್ಲಿಸಿದರು. ಆ ಬಳಿಕ ಮೆರವಣಿಗೆ ವಾಪಸು ಹೊರಟಿತು. ಈಗ ಸ್ವಾಮೀಜಿಯವರ ಶಿಷ್ಯನಾದ ಶರಚ್ಚಂದ್ರನು ಗುರುವಿನ ಆದೇಶದಂತೆ ಶ್ರೀರಾಮಕೃಷ್ಣರ ಅವಶೇಷದ ಪಾತ್ರೆಯನ್ನು ತಲೆಯ ಮೇಲೆ ಹೊತ್ತು ತಂದ.

ಅಂದು ಮಠದ ಸಮಸ್ತ ವಾತಾವರಣವೇ ಆಧ್ಯಾತ್ಮಿಕ ಶಕ್ತಿಯಿಂದ ಸ್ಪಂದಿಸುತ್ತಿತ್ತು. ಸ್ವಾಮೀಜಿಯವರಂತೂ ವಿಜಯೋತ್ಸಾಹದಿಂದ ಆನಂದತುಂದಿಲರಾಗಿಬಿಟ್ಟಿದ್ದರು. ದುಸ್ಸಾಧ್ಯ ವಾದದ್ದನ್ನು ಸಾಧಿಸಿದ ಭಾವ ಅವರದಾಗಿತ್ತು. ಬಳಿಯಿದ್ದವರನ್ನು ಉದ್ದೇಶಿಸಿ ಸ್ವಾಮೀಜಿ ನುಡಿದರು: “ಭಗವಂತನ ಇಚ್ಛೆಯಿಂದ ಈ ದಿನ ಅವನದೇ ಆದ ಧರ್ಮಕ್ಷೇತ್ರವು ಸಂಸ್ಥಾಪಿಸ ಲ್ಪಟ್ಟಿದೆ. ನನ್ನ ತಲೆಯ ಮೇಲೆ ಹನ್ನೆರಡು ವರ್ಷಗಳಿಂದ ಕುಳಿತಿದ್ದ ದೊಡ್ಡ ಹೊಣೆಗಾರಿಕೆಯ ಭಾರದಿಂದ ಈ ದಿನ ಬಿಡುಗಡೆ ಹೊಂದಿರುವಂತೆ ನನಗನ್ನಿಸುತ್ತಿದೆ. ಈಗ ನನ್ನ ಮನದಲ್ಲಿ ದೃಶ್ಯವೊಂದು ಮೂಡುತ್ತಿದೆ: ಈ ಮಠವು ಮುಂದೆ ಶಾಸ್ತ್ರಾಧ್ಯಯನದ ಮತ್ತು ಆಧ್ಯಾತ್ಮಿಕ ಸಾಧನೆಯ ಒಂದು ಪ್ರಮುಖ ಕೇಂದ್ರವಾಗುತ್ತದೆ. ಮುಂದೆ ಉದಯಿಸಲಿರುವ ಈ ಧಾರ್ಮಿಕ ವಿಶ್ವವಿದ್ಯಾನಿಲಯದ ನಡುವೆ ಸಂನ್ಯಾಸಿಗಳು ವಾಸಿಸುತ್ತಾರೆ. ಸುತ್ತಲೂ ಧಾರ್ಮಿಕರಾದ ಸಜ್ಜನರು ಮನೆಗಳನ್ನು ಕಟ್ಟಿಕೊಂಡು ವಾಸವಾಗಿರುತ್ತಾರೆ. ದಕ್ಷಿಣದಿಕ್ಕಿನಲ್ಲಿ ಅಮೆರಿಕ-ಇಂಗ್ಲೆಂಡುಗಳಿಂದ ಬರಲಿರುವ ಭಗವಂತನ ಅನುಯಾಯಿಗಳು ನೆಲಸುತ್ತಾರೆ.” ಹೀಗೆ ಹೇಳಿದ ಸ್ವಾಮೀಜಿ ಶರ ಚ್ಚಂದ್ರನ ಕಡೆಗೆ ತಿರುಗಿ, “ಇದರ ಬಗ್ಗೆ ನಿನಗೇನೆನ್ನಿಸುತ್ತದೆ?” ಎಂದು ಕೇಳಿದರು. ಶಿಷ್ಯ ವಿನೀತನಾಗಿಯೇ ಸ್ವಾಮೀಜಿಯವರ ಈ ಅದ್ಭುತ ‘ಕನಸು’ ಎಂದಾದರೂ ನನಸಾಗಬಹು ದೆಂಬುದರ ಬಗ್ಗೆ ತನ್ನ ತೀವ್ರ ಸಂಶಯವನ್ನು ವ್ಯಕ್ತಪಡಿಸಿದ. ಆಗ ಸ್ವಾಮೀಜಿ ಗಟ್ಟಿ ಸ್ವರದಿಂದ ಹೇಳಿದರು: “ಏನು, ಇದನ್ನೆಲ್ಲ ಕನಸು ಎಂದು ಕರೆಯುವೆಯಾ ನೀನು? ಸಂಶಯಮತಿಯೇ, ಕೇಳು, ಕಾಲಾಂತರದಲ್ಲಿ ನನ್ನೆಲ್ಲ ನಿರೀಕ್ಷೆಗಳೂ ನಿಜವಾಗುತ್ತವೆ! ಈಗ ನಾನು ಕೇವಲ ಅಸ್ತಿಭಾರವನ್ನು ಮಾತ್ರ ಹಾಕುತ್ತಿದ್ದೇನೆ. ಮಹಾಕಾರ್ಯಗಳೆಲ್ಲ ಮುಂದೆ ನಡೆಯಲಿವೆ. ನಾನು ನನ್ನ ಪಾಲಿನ ಕಾರ್ಯವನ್ನು ಮಾಡಿ ಮುಗಿಸುತ್ತೇನೆ. ಮತ್ತು ನಿಮ್ಮ ಹೃದಯಗಳಲ್ಲಿ ಹಲವಾರು ವಿಚಾರಗಳನ್ನು ತುಂಬಿ ಹೋಗುತ್ತೇನೆ. ಅವುಗಳನ್ನೆಲ್ಲ ನೀವು ಮುಂದೆ ಕಾರ್ಯಗತಗೊಳಿಸಬೇಕಾ ಗುತ್ತದೆ. ಧರ್ಮದ ಅತ್ಯುನ್ನತ ತತ್ತ್ವಗಳನ್ನು ಹಾಗೂ ಆದರ್ಶಗಳನ್ನು ಕೇವಲ ಅಧ್ಯಯನ ಮಾಡಿ ಅರಿತುಕೊಂಡರೆ ಸಾಲದು; ಅವುಗಳನ್ನು ದೈನಂದಿನ ಜೀವನದಲ್ಲಿ ಅನುಷ್ಠಾನಕ್ಕೆ ತರಬೇಕು, ಅರ್ಥವಾಯಿತೇ?”

ಮತ್ತೆ ಕೆಲದಿನಗಳ ಬಳಿಕ, ಮಠದ ಆದರ್ಶಗಳ ಮತ್ತು ಕಾರ್ಯಕ್ಷೇತ್ರದ ಉದ್ದೇಶವ್ಯಾಪ್ತಿಯ ಬಗ್ಗೆ ಹಾಗೂ ಮುಂದೆ ಮಠದಲ್ಲಿ ಅನುಸರಿಸಲ್ಪಡಬೇಕಾದ ನೀತಿ ನಿಯಮಗಳ ಬಗ್ಗೆ ಸ್ವಾಮೀಜಿ ಯವರ ಅಭಿಪ್ರಾಯಗಳನ್ನು ತಿಳಿದುಕೊಳ್ಳುವ ಸದವಕಾಶ ಶರಚ್ಚಂದ್ರನಿಗೆ ಮತ್ತೆ ಸಿಕ್ಕಿತು. ಆತನ ಲೇಖನಗಳಿಂದ ರಾಷ್ಟ್ರೀಯ ಶಿಕ್ಷಣ ಹಾಗೂ ಭಾರತದಲ್ಲಿ ಜನಸೇವಾ ಕಾರ್ಯದ ವಿಷಯದಲ್ಲಿ ಸ್ವಾಮೀಜಿಯವರ ಅಭಿಪ್ರಾಯವೇನೆಂಬುದರ ಬಗ್ಗೆ ನಮಗೊಂದು ಇಣುಕು ನೋಟ ಸಿಗುತ್ತದೆ. ಸ್ವಾಮೀಜಿ ಹೊಸ ಮಠದ ವಠಾರದೊಳಗೆ ಅತ್ತಿಂದಿತ್ತ ಓಡಾಡುತ್ತ ಶಿಷ್ಯನಿಗೆ ಒಂದು ಹಳೆಯ ಕುಟೀರದ ಕಡೆಗೆ ಬೆರಳು ಮಾಡಿ ತೋರಿಸುತ್ತ ಹೇಳಿದರು, “ಆ ಸ್ಥಳ ಸಾಧುಗಳ ವಾಸಕ್ಕೆ. ಈ ಮಠವು ಧರ್ಮಾನುಷ್ಠಾನದ ಹಾಗೂ ಶಾಸ್ತ್ರಾಧ್ಯಯನ ಜ್ಞಾನಾರ್ಜನೆಗಳ ಪ್ರಧಾನ ಕೇಂದ್ರವಾಗು ವುದು. ಇಲ್ಲಿಂದ ಹೊರಹೊಮ್ಮುವ ಆಧ್ಯಾತ್ಮಿಕ ಶಕ್ತಿಯು ಮಾನವರ ಕಾರ್ಯಕಲಾಪಗಳನ್ನು ಹಾಗೂ ಉತ್ಕಾಂಕ್ಷೆಗಳನ್ನು ಒಂದು ನೂತನ ಪಥದಲ್ಲಿ ಮುನ್ನಡೆಸುತ್ತ ಸಮಸ್ತ ವಿಶ್ವವನ್ನೇ ವ್ಯಾಪಿಸುತ್ತದೆ. ಜ್ಞಾನಯೋಗ ಭಕ್ತಿಯೋಗ ರಾಜಯೋಗ ಕರ್ಮಯೋಗಗಳನ್ನು ಸಮನ್ವಯ ಗೊಳಿಸುವ ಆದರ್ಶಗಳೂ ಇಲ್ಲಿಂದಲೇ ಪ್ರಸಾರಗೊಳ್ಳುತ್ತವೆ. ಈ ಮಠದ ಸಂನ್ಯಾಸಿಗಳು ಇಚ್ಛಾಮಾತ್ರದಿಂದಲೇ, ಜೀವಚ್ಛವದಂತಿರುವ ಮನುಷ್ಯರಲ್ಲೂ ನವಜೀವನಚೇತನವನ್ನು ಸ್ಪಂದಿ ಸುವ ಕಾಲವೊಂದು ಬರಲಿದೆ. ಈ ಎಲ್ಲ ದೃಶ್ಯಗಳು ನನ್ನ ಕಣ್ಣಮುಂದೆ ಎದ್ದು ನಿಲ್ಲುತ್ತಿವೆ.

“ಅತ್ತ ದಕ್ಷಿಣ ದಿಕ್ಕಿನಲ್ಲಿರುವ ಆ ಜಾಗದಲ್ಲಿ, ಹಿಂದಿನ ಕಾಲದ ಸಂಸ್ಕೃತ ಪಾಠಶಾಲೆಯಂತಹ ಅಧ್ಯಯನ ಮಂದಿರವೊಂದು ಬರುತ್ತದೆ. ಅಲ್ಲಿ ವ್ಯಾಕರಣ ತತ್ತ್ವಶಾಸ್ತ್ರ ಕಲೆ ವಿಜ್ಞಾನ ಸಾಹಿತ್ಯ ಕಾವ್ಯ ನ್ಯಾಯಶಾಸ್ತ್ರ ಧರ್ಮಶಾಸ್ತ್ರಗಳು ಹಾಗೂ ಇಂಗ್ಲಿಷ್ ಭಾಷೆ–ಇವುಗಳನ್ನು ಕಲಿಸಲಾಗುತ್ತದೆ. ಅಲ್ಲಿ ಯುವಬ್ರಹ್ಮಚಾರಿಗಳು ವಾಸವಾಗಿದ್ದು ಶಾಸ್ತ್ರಾಧ್ಯಯನ ಮಾಡುತ್ತಾರೆ. ಅವರಿಗೆಲ್ಲ ಮಠ ದಿಂದಲೆ ಊಟ ಬಟ್ಟೆಗಳ ವ್ಯವಸ್ಥೆ. ಐದು ವರ್ಷಗಳ ತರಬೇತಿ ಮುಗಿದ ಬಳಿಕ ಅವರಿಗೆ ತಮ್ಮ ತಮ್ಮ ಮನೆಗಳಿಗೆ ಹಿಂದಿರುಗಿ ಗೃಹಸ್ಥ ಜೀವನವನ್ನು ಸ್ವೀಕರಿಸಲು ಸ್ವಾತಂತ್ರ್ಯವಿರುತ್ತದೆ. ಅಥವಾ ಅವರಿಗೆ ಇಚ್ಛೆಯಿದ್ದರೆ ಮಠದ ಹಿರಿಯ ಸಾಧುಗಳ ಅನುಮತಿಯ ಮೇರೆಗೆ ಸಂನ್ಯಾಸ ಸ್ವೀಕಾರ ಮಾಡಲೂ ಸ್ವಾತಂತ್ರ್ಯವಿರುತ್ತದೆ. ಈ ಬ್ರಹ್ಮಚಾರಿಗಳಲ್ಲಿ ಯಾರಾದರೂ ಅಶಿಸ್ತಿನಿಂದಿರುವುದೇ ಆಗಲಿ ಶೀಲಗೆಟ್ಟಿರುವುದೇ ಆಗಲಿ ಕಂಡುಬಂದರೆ ಮಠದ ಹಿರಿಯರಿಗೆ ಅವರನ್ನು ಮಠದಿಂದ ತೆಗೆದುಹಾಕಲು ಅಧಿಕಾರವಿದೆ. ಇಲ್ಲಿ ಜಾತಿಭೇದಗಳನ್ನು ಪರಿಗಣಿಸದೆ ಬ್ರಹ್ಮಚಾರಿಗಳಿಗೆ ಶಿಕ್ಷಣ ನೀಡಲಾಗುತ್ತದೆ. ಆದರೆ ಯಾರು ತಮ್ಮ ಜಾತಿ ಮತ ಸಂಪ್ರದಾಯಗಳಿಗೆ ಅನುಸಾರವಾದ ಅನುಷ್ಠಾನಗಳನ್ನು ಇಟ್ಟುಕೊಳ್ಳಬೇಕೆಂದು ಬಯಸುತ್ತಾರೊ ಅಂಥವರು ತಮ್ಮ ಆಹಾರಾದಿ ವ್ಯವಸ್ಥೆಗಳನ್ನು ತಾವೇ ಮಾಡಿಕೊಳ್ಳಬೇಕು. ಇಂಥವರು ತರಗತಿಗಳಲ್ಲಿ ಮಾತ್ರ ಎಲ್ಲರೊಂದಿಗೆ ಭಾಗವಹಿಸುತ್ತಾರೆ. ಮಠದ ಅಧಿಕಾರಿಗಳು ಇವರುಗಳ ಶೀಲ-ನಡತೆಗಳ ಮೇಲೂ ಕಣ್ಣಿಟ್ಟಿರು ತ್ತಾರೆ. ಇಲ್ಲಿ ತರಬೇತಿ ಪಡೆಯದಿರುವವರು ಸಂನ್ಯಾಸ ಸ್ವೀಕರಿಸಲು ಅರ್ಹರಾಗುವುದಿಲ್ಲ. ಹೀಗೆ ಮುಂದೆ ಕಾಲಕ್ರಮದಲ್ಲಿ ಈ ತರಬೇತಿ ಪಡೆದವರಿಂದಲೇ ಮಠದ ಕಾರ್ಯಕಲಾಪಗಳನ್ನು ನಿರ್ವಹಿಸಲಾಗುವುದು.”

ಶಿಷ್ಯ: “ಸ್ವಾಮೀಜಿ, ಹಾಗಾದರೆ ನೀವು ಪುರಾತನ ಕಾಲದ ಗುರುಕುಲ ಪದ್ಧತಿಯನ್ನು ಪುನಃ ಪ್ರಚಲಿತಗೊಳಿಸಲು ಉದ್ದೇಶಿಸಿರುವಿರೇನು?”

ಸ್ವಾಮೀಜಿ: “ಹೌದು, ಖಂಡಿತವಾಗಿಯೂ ಹೌದು. ಈಗಿನ ಆಧುನಿಕ ವಿದ್ಯಾಭ್ಯಾಸದ ಕ್ರಮದಲ್ಲಿ ಬ್ರಹ್ಮವಿದ್ಯೆಗೆ, ಬ್ರಹ್ಮಜ್ಞಾನದ ವಿಕಾಸಕ್ಕೆ ಅವಕಾಶವೇ ಇಲ್ಲ. ಹಿಂದಿನ ಬ್ರಹ್ಮಚರ್ಯ ಪದ್ಧತಿಯನ್ನು ಈಗ ಹೊಸದಾಗಿ ನೆಲೆಗೊಳಿಸಬೇಕಾಗಿದೆ. ಆದರೆ ಈ ಕಾರ್ಯಕ್ಕೆ ತಳಪಾಯವನ್ನು ಹಾಕುವಾಗ ಮಾತ್ರ ಸ್ವಲ್ಪ ವಿಶಾಲ ದೃಷ್ಟಿಯನ್ನು ಇಟ್ಟುಕೊಳ್ಳಬೇಕಾಗುತ್ತದೆ; ಕಾಲಧರ್ಮಕ್ಕನು ಸಾರವಾದ ಕೆಲವು ಬದಲಾವಣೆಗಳನ್ನು ಮಾಡಿಕೊಳ್ಳಬೇಕಾಗುತ್ತದೆ. ಈ ಸಂಬಂಧವಾದ ವಿವರ ಗಳನ್ನು ನಿನಗೆ ಮುಂದೆ ತಿಳಿಸುತ್ತೇನೆ.

“ಮಠದ ದಕ್ಷಿಣ ಭಾಗದಲ್ಲಿರುವ ಈ ಜಾಗವನ್ನು ಮುಂದೆ ಸಾಧ್ಯವಾದಾಗ ಕೊಂಡುಕೊಳ್ಳ ಬೇಕು. ಅಲ್ಲಿ ಶ್ರೀರಾಮಕೃಷ್ಣರ ಹೆಸರಿನಲ್ಲಿ ಅನ್ನಸತ್ರವನ್ನು ಕಟ್ಟಿಸಬೇಕು. ಮತ್ತು ಯಾರು ನಿಜವಾಗಿಯೂ ಬಡವರೊ, ನಿರ್ಗತಿಕರೊ ಅಂಥವರನ್ನು ಸಾಕ್ಷಾತ್ ನಾರಾಯಣ ಸ್ವರೂಪರೆಂದು ಭಾವಿಸಿ ಅವರಿಗೆ ಉಣಬಡಿಸಬೇಕು. ಮಠದ ಆದಾಯವನ್ನು ನೋಡಿಕೊಂಡು ಈ ಕಾರ್ಯವನ್ನು ನಡೆಸಬೇಕು. ಈ ಕೆಲಸವನ್ನು ಒಂದಿಬ್ಬರು-ಮೂವರಿಗೆ ಊಟ ಹಾಕುವುದರ ಮೂಲಕ ಕೂಡ ಪ್ರಾರಂಭಿಸಬಹುದು. ಉತ್ಸಾಹೀ ಬ್ರಹ್ಮಚಾರಿಗಳಿಗೆ ಈ ಅನ್ನಸತ್ರವನ್ನು ನಡೆಸಲು ತರಬೇತಿ ನೀಡಬೇಕಾಗುತ್ತದೆ. ಅನ್ನಸತ್ರದ ಖರ್ಚುವೆಚ್ಚವನ್ನೆಲ್ಲ ಅವರೇ ನೋಡಿಕೊಳ್ಳಬೇಕು–ಮನೆ ಮನೆಗೆ ಹೋಗಿ ಭಿಕ್ಷೆ ಬೇಡಿಯಾದರೂ ಸರಿಯೆ. ಈ ಸತ್ರಕ್ಕೆ ಮಠದಿಂದ ಯಾವ ಧನಸಹಾಯ ವನ್ನೂ ಒದಗಿಸಲಾಗುವುದಿಲ್ಲ. ಬ್ರಹ್ಮಚಾರಿಗಳು ಐದು ವರ್ಷಕಾಲ ಈ ಅನ್ನಸತ್ರದಲ್ಲಿ ಸೇವೆ ಸಲ್ಲಿಸಿದ ಮೇಲೆಯೇ ಅವರಿಗೆ ಮಠದ ‘ಅಧ್ಯಯನ ಮಂದಿರ’ದಲ್ಲಿ ಪ್ರವೇಶಾವಕಾಶ. ಹೀಗೆ ಹತ್ತುವರ್ಷಗಳ ತರಬೇತಿ ಮುಗಿದ ಮೇಲೆಯೇ ಅವರಿಗೆ ಸಂನ್ಯಾಸಾಧಿಕಾರ. ಆದರೆ ಸಂನ್ಯಾಸ ಸ್ವೀಕರಿಸಲು ಇಷ್ಟವಿಲ್ಲದಿರುವವರ ವಿಚಾರ ಬೇರೆ. ಇಷ್ಟವಿರುವವರ ವಿಚಾರದಲ್ಲೂ ಕೂಡ, ಮಠದ ಹಿರಿಯ ಸಾಧುಗಳು ಅವರಲ್ಲಿ ತಕ್ಕ ಅರ್ಹತೆಯಿರುವುದನ್ನು ಕಂಡರೆ ಮಾತ್ರ ದೀಕ್ಷೆ ನೀಡುವ ಮಾತು. ಆದರೆ ಈ ಮಠದ ಅಧ್ಯಕ್ಷರಾದವರು ಸಮರ್ಥರಾದ ಬ್ರಹ್ಮಚಾರಿಗಳ ವಿಷಯದಲ್ಲಿ ಈ ನಿಯಮಗಳನ್ನೆಲ್ಲ ಬದಿಗೆ ಸರಿಸಿ ಅವರಿಗೆ ಸಂನ್ಯಾಸ ನೀಡಬಹುದು. ನೋಡಿ ದೆಯಾ, ಇಷ್ಟೆಲ್ಲ ಆಲೋಚನೆಗಳು ನನ್ನ ತಲೆಯಲ್ಲಿವೆ.”

ಶಿಷ್ಯ: “ಸ್ವಾಮೀಜಿ, ಮಠದಲ್ಲಿ ಈ ಮೂರು ಬೇರೆಬೇರೆ ವಿಭಾಗಗಳನ್ನು ಸ್ಥಾಪಿಸುವ ಉದ್ದೇಶವಾದರೂ ಏನು?”

ಸ್ವಾಮೀಜಿ: “ಸ್ಪಷ್ಟವಾಗಿಯೇ ಇದೆಯಲ್ಲ! ಮೊದಲನೆಯದಾಗಿ ದೈಹಿಕ ಆವಶ್ಯಕತೆಗಳನ್ನು ಪೂರೈಸುವ ಅನ್ನದಾನಾದಿಗಳು ನಡೆಯಬೇಕು. ಎರಡನೆಯದಾಗಿ ವಿದ್ಯಾದಾನ ಅಥವಾ ಬೌದ್ಧಿಕ ಜ್ಞಾನದಾನ ನಡೆಯಬೇಕು. ಮೂರನೆಯದಾಗಿ ಆಧ್ಯಾತ್ಮಿಕಜ್ಞಾನದಾನ ನಡೆಯಬೇಕು. ಪರಿಪೂರ್ಣ ವ್ಯಕ್ತಿನಿರ್ಮಾಣಕ್ಕೆ ಅನುಕೂಲಕರವಾದ ಮತ್ತು ಆವಶ್ಯಕವಾದ ಈ ಮೂರಂಶಗಳನ್ನು ಸಮರಸ ವಾಗಿ ಸೇರಿಸಿಕೊಳ್ಳಬೇಕಾದದ್ದು ಮಠದ ಆದ್ಯ ಕರ್ತವ್ಯ. ನಾನು ಈಗ ತಾನೆ ಹೇಳಿದಂತೆ, ಬ್ರಹ್ಮಚಾರಿಗಳು, ಅನ್ನಸತ್ರದಲ್ಲಿ ಸೇವಾನಿರತರಾಗುವುದರಿಂದ ಪರಹಿತಕ್ಕಾಗಿ ಕೆಲಸ ಮಾಡುವುದು ಹೇಗೆ, ಪೂಜಾಭಾವದಿಂದ ಮಾನವಸೇವೆ ಮಾಡುವುದು ಹೇಗೆ ಎಂಬುದು ಅವರ ಮನದಲ್ಲಿ ಸ್ಥಿರವಾಗಿ ಬೇರೂರುವಂತಾಗುತ್ತದೆ. ಇದು ಕ್ರಮೇಣ ಬ್ರಹ್ಮಚಾರಿಗಳ ಮನಸ್ಸನ್ನು ಪರಿಶುದ್ಧ ಗೊಳಿಸಿ ಅವರನ್ನು ಸಾತ್ವಿಕತೆಯ ಕಡೆಗೆ, ಸಾತ್ವಿಕ ಧ್ಯೇಯೋದ್ದೇಶಗಳ ಕಡೆಗೆ ಮುನ್ನಡೆಸುತ್ತದೆ. ಇಂಥವರು ಮಾತ್ರವೇ ಅಪರಾವಿದ್ಯೆ ಮತ್ತು ಪರಾವಿದ್ಯೆಗಳನ್ನು ಸ್ವೀಕರಿಸಿ ಉಳಿಸಿಕೊಳ್ಳಲು ಸಮರ್ಥರಾಗುತ್ತಾರೆ.”

ಶಿಷ್ಯ: “ಸ್ವಾಮೀಜಿ, ನಿಮ್ಮ ಮಾತುಗಳನ್ನು ಕೇಳಿದಾಗ ಅನ್ನಸತ್ರ ಹಾಗೂ ಸೇವಾಶ್ರಮಗಳ ವಿಷಯದಲ್ಲಿ ನಿಮ್ಮ ಭಾವನೆಗಳನ್ನು ಇನ್ನೂ ಹೆಚ್ಚಾಗಿ ತಿಳಿದುಕೊಳ್ಳುವ ಆಸೆಯಾಗಿದೆ ನನಗೆ.”

ಸ್ವಾಮೀಜಿ: “ಈ ಸೇವಾಶ್ರಮದಲ್ಲಿ ಸಾಕಷ್ಟು ವಿಶಾಲವಾದ ಕಿಟಕಿ-ಗವಾಕ್ಷಗಳ ವ್ಯವಸ್ಥೆ ಯಿರುವ ಕೋಣೆಗಳಿರಬೇಕು. ಪ್ರತಿಯೊಂದು ಕೋಣೆಯಲ್ಲೂಇಬ್ಬರು ಅಥವಾ ಮೂವರು ಬಡವರು ಅಥವಾ ರೋಗಿಗಳು ಇರುತ್ತಾರೆ. ಅವರಿಗೆ ಹಿತಕರವಾದ ಹಾಸಿಗೆಗಳಿರಬೇಕು; ಶುಚಿ ಯಾದ ಬಟ್ಟೆಗಳಿರಬೇಕು. ಅವರಿಗೊಬ್ಬರು ಡಾಕ್ಟರು ಇರಬೇಕು; ಅವರು ವಾರಕ್ಕೆ ಒಂದೆರಡು ಸಲವಾದರೂ ಬಂದು ನೋಡಿಕೊಂಡು ಹೋಗಬೇಕು. ಈ ಸೇವಾಶ್ರಮ ಎನ್ನುವುದು ಅನ್ನಸತ್ರದ ಒಂದು ವಿಭಾಗ. ಇಲ್ಲಿ ರೋಗಿಗಳ ಶುಶ್ರೂಷೆ ನಡೆಯಬೇಕು. ಕಾಲಕ್ರಮದಲ್ಲಿ ಸಾಕಷ್ಟು ಹಣ ಸೇರಿದಾಗ ಒಂದು ದೊಡ್ಡ ಅಡಿಗೆಮನೆಯನ್ನು ಕಟ್ಟಬೇಕು. ಮತ್ತು ಎಷ್ಟು ಮಂದಿ ಬಂದರೂ ಸರಿಯೆ, ಅವರು ದಿನದ ಯಾವುದೇ ಸಮಯದಲ್ಲೇ ಬಂದರೂ ಹಸಿದು ಬಂದವರಿಗೆಲ್ಲ ತೃಪ್ತಿಯಾಗುವಷ್ಟು ಊಟಹಾಕಬೇಕು. ಯಾವುದೇ ಪರಿಸ್ಥಿತಿಯಲ್ಲೂ ಊಟ ಹಾಕದೆ ಅವರನ್ನು ಹಿಂದಿರುಗಿಸಬಾರದು. ಇಲ್ಲಿ ಅನ್ನ ಬಸಿದ ಗಂಜಿ ಗಂಗೆಗೆ ಹರಿದು ನದಿಯ ನೀರನ್ನೆಲ್ಲ ಬೆಳ್ಳಗಾಗಿಸಬೇಕು! ಓಹ್, ಈ ಅನ್ನಸತ್ರ ಇಷ್ಟು ಭಾರೀ ಪ್ರಮಾಣದಲ್ಲಿ ನಡೆಯುವುದನ್ನು ನೋಡಲು ನನಗೆಷ್ಟು ಆನಂದವಾಗುತ್ತದೆ ಗೊತ್ತೆ!”

ಹೀಗೆ ಹೇಳುತ್ತ ಸ್ವಾಮೀಜಿ ಎದ್ದು ನಿಂತು ಗಂಗೆಯತ್ತ ದಿಟ್ಟಿಸಿದರು. ತಮ್ಮ ಈ ಕನಸು ಎಂದಿಗೆ ನನಸಾಗುವುದೊ ಎಂಬುದನ್ನು ಭವಿಷ್ಯದ ಗರ್ಭದಲ್ಲಿ ಇಣಿಕಿ ನೋಡುವಂತಿತ್ತು ಅವರ ನೋಟ. ಬಳಿಕ ಅವರು ತುಂಬ ಆತ್ಮೀಯತೆಯ ದನಿಯಲ್ಲಿ ಹೇಳಿದರು:

“ಯಾರಿಗೆ ಗೊತ್ತು, ನಿಮ್ಮಲ್ಲೇ ಯಾರಾದರೊಬ್ಬರಲ್ಲಿ ಮಲಗಿರುವ ಸಿಂಹ ಯಾವಾಗ ಮೇಲೆದ್ದು ನಿಲ್ಲುತ್ತದೊ! ಜಗನ್ಮಾತೆ ನಿಮ್ಮಲ್ಲಿ ಯಾರಾದರೊಬ್ಬರ ಹೃದಯದಲ್ಲಿ ತನ್ನ ದೈವೀ ಶಕ್ತಿಯ ಕಿಡಿಯೊಂದನ್ನು ಹೊತ್ತಿಸಿದರೆ ಇಡೀ ದೇಶದಲ್ಲಿ ನೂರಾರು ಅನ್ನಸತ್ರಗಳು ತೆರೆದುಕೊಳ್ಳ ಲಾರವೇನು? ಪ್ರತಿಯೊಬ್ಬನ ಅಂತರಂಗದಲ್ಲೂ ಜ್ಞಾನ ಭಕ್ತಿ ಶಕ್ತಿ ಎಲ್ಲ ಅಡಗಿಕೊಂಡಿವೆ. ಆ ಶಕ್ತಿಯ ಅಭಿವ್ಯಕ್ತಿಯ ಮಟ್ಟದಲ್ಲಿರುವ ತಾರತಮ್ಯದಿಂದಾಗಿ ಒಬ್ಬ ವ್ಯಕ್ತಿ ಮಹಾತ್ಮನೆನಿಸುತ್ತಾನೆ, ಇನ್ನೊಬ್ಬ ಅಲ್ಪನೆನಿಸುತ್ತಾನೆ. ಆ ಪರಿಪೂರ್ಣ ದಿವ್ಯತೆಗೂ ನಮಗೂ ನಡುವೆ ಒಂದು ಪರದೆ ಇಳಿಬಿಟ್ಟಂತಿದೆ. ಆ ಪರದೆಯೊಂದು ಸರಿದುಬಿಟ್ಟರೆ ಸಮಸ್ತ ಪ್ರಕೃತಿಶಕ್ತಿಯೇ ನಮ್ಮ ಕಾಲ ಬುಡದಲ್ಲಿ ಬಿದ್ದಿರುತ್ತದೆ. ಆಗ ನಾವು ಏನನ್ನೇ ಬಯಸಲಿ, ಏನನ್ನೇ ಇಚ್ಛಿಸಲಿ, ಅದು ನಡೆದು ಹೋಗುತ್ತದೆ.

“ಭಗವಂತ ಇಚ್ಛಿಸಿದರೆ ಈ ಮಠವನ್ನು ನಾವು ಸರ್ವಧರ್ಮಸಮನ್ವಯದ ಮಹಾಕೇಂದ್ರ ವನ್ನಾಗಿ ಮಾಡಬಹುದು. ನಮ್ಮ ಭಗವಾನರು (ಶ್ರೀರಾಮಕೃಷ್ಣರು) ಸ್ವಯಂ ಸರ್ವಧರ್ಮ ಸಮನ್ವಯದ ಮೂರ್ತರೂಪರಾಗಿದ್ದಾರೆ. ಇಲ್ಲಿ ನಾವು ಆ ಸಮನ್ವಯದ ಆದರ್ಶವನ್ನು ಜೀವಂತ ವಾಗಿ ಇಟ್ಟದ್ದೇ ಆದರೆ, ಆಧ್ಯಾತ್ಮಿಕ ಸಾಮ್ರಾಜ್ಯದಲ್ಲಿ ಅವರ ಸಿಂಹಾಸನವು ಸುಭದ್ರವಾಗಿರುತ್ತದೆ. ಬ್ರಾಹ್ಮಣರಿಂದ ಚಂಡಾಲರವರೆಗಿನ ಪ್ರತಿಯೊಂದು ಮತ-ಪಂಥಗಳವರೂ ಈ ಕ್ಷೇತ್ರದಲ್ಲಿ ತಮ್ಮತಮ್ಮ ಆದರ್ಶಗಳು ಪರಿಪೂರ್ಣಗೊಳ್ಳುವುದನ್ನು ಕಾಣುವಂತಾಗಬೇಕು. ನಾನು ಈ ಮಠದ ಭೂಮಿಯಲ್ಲಿ ಶ್ರೀರಾಮಕೃಷ್ಣರ ಭಾವಚಿತ್ರವನ್ನು ಪ್ರತಿಷ್ಠಾಪಿಸುವ ಸಂದರ್ಭದಲ್ಲಿ, ಅವರ ಭಾವನೆಗಳೆಲ್ಲ ಇಲ್ಲಿಂದ ಹೊರಹೊಮ್ಮುತ್ತ ತಮ್ಮ ಪ್ರಖರ ಪ್ರಭೆಯಿಂದ ಇಡೀ ಜಗತ್ತನ್ನೇ ಆವರಿಸುತ್ತಿರುವುದನ್ನು ಕಂಡೆ. ನನ್ನ ಮಟ್ಟಿಗೆ ನಾನು ಅವರ ವಿಶಾಲ ಉದಾರ ಭಾವನೆಗಳ ಮರ್ಮವನ್ನು ತೆರೆದುತೋರಲು ಶಕ್ತಿಮೀರಿ ಪ್ರಯತ್ನಿಸುತ್ತೇನೆ. ನೀವೆಲ್ಲರೂ ಹಾಗೆಯೇ ಮಾಡ ಬೇಕು. ವೇದಾಂತವನ್ನು ಸುಮ್ಮನೆ ಓದಿಕೊಳ್ಳುವುದರಿಂದೇನು ಪ್ರಯೋಜನ? ಆ ಘನವಾದ ಅದ್ವೈತ ತತ್ತ್ವವನ್ನು ನಮ್ಮ ನಿತ್ಯಜೀವನದಲ್ಲಿ ಪ್ರಕಟಪಡಿಸುತ್ತ ಮಾದರಿಯಾಗಿರಬೇಕು. ಈ ಅದ್ವೈತವಾದವನ್ನೂ ಕೂಡ ಬಹಳ ಕಾಲದಿಂದಲೂ ಅರಣ್ಯಗಳಲ್ಲಿ ಪರ್ವತಗಹ್ವರಗಳಲ್ಲಿ ಅಡಗಿ ಸಿಡಲಾಗಿತ್ತು. ಈಗ ಅದನ್ನು ಆ ಸಂದಿಗೊಂದಿಗಳಿಂದ ಹೊರತಂದು ಸಮಾಜದಲ್ಲಿ, ದೈನಂದಿನ ಜೀವನದಲ್ಲಿ ಪ್ರಸಾರ ಮಾಡುವ ಕಾರ್ಯ ನನ್ನ ಪಾಲಿಗೆ ಮೀಸಲಾಗಿದೆ. ಅದ್ವೈತವೆಂಬ ಭೇರಿಯ ನಾದವು ಪ್ರತಿಯೊಂದು ಮನೆಯಲ್ಲಿ ಹಾಗೂ ಮಠದಲ್ಲಿ, ಮೈದಾನಗಳಲ್ಲಿ ಹಾಗೂ ಉದ್ಯಾನ ಗಳಲ್ಲಿ, ಪರ್ವತಗಳಲ್ಲಿ ಹಾಗೂ ಬಯಲುಗಳಲ್ಲಿ ಪ್ರತಿಧ್ವನಿಸುವಂತಾಗಬೇಕು. ಬನ್ನಿ ನೀವೆಲ್ಲ ನನ್ನ ನೆರವಿಗೆ! ಕಾರ್ಯರಂಗಕ್ಕಿಳಿಯಿರಿ!”

ಶಿಷ್ಯ: “ಆದರೆ ಸ್ವಾಮೀಜಿ, ನನ್ನ ಮನಸ್ಸು ಈ ಅದ್ವೈತವನ್ನು ಕೆಲಸದ ಮೂಲಕ ಪ್ರಕಟಿಸು ವುದಕ್ಕಿಂತ ಧ್ಯಾನದ ಮೂಲಕ ಸಾಕ್ಷಾತ್ಕರಿಸಿಕೊಳ್ಳಲು ಬಯಸುತ್ತದೆ...”

ಸ್ವಾಮೀಜಿ: “ಏನೆಂದೆ! ಜಡಸಮಾಧಿ ಹೊಂದಿ ಮರಗಟ್ಟಿಕೊಂಡು ಕುಳಿತುಕೊಳ್ಳುವುದ ರಿಂದೇನು ಪ್ರಯೋಜನ? ಅದ್ವೈತಾನುಭವದ ಸ್ಫೂರ್ತಿಯಿಂದ ಕೆಲವೊಮ್ಮೆ ನರ್ತನವನ್ನೇಕೆ ಮಾಡಬಾರದು! ಇನ್ನು ಕೆಲವೊಮ್ಮೆ ನಿರ್ವಿಕಲ್ಪ ಸ್ಥಿತಿಯಲ್ಲಿ ಮುಳುಗಿದ್ದರಾಯಿತು? ಸಿಹಿ ತಿಂಡಿಯ ಸವಿಯನ್ನು ಹೆಚ್ಚು ಆಸ್ವಾದಿಸಬಲ್ಲವರು ಯಾರು ಹೇಳು? ತಾವೊಬ್ಬರೇ ಕುಳಿತು ತಿನ್ನುವವರೋ ಅಥವಾ ಇತರರೊಂದಿಗೆ ಹಂಚಿಕೊಂಡು ತಿನ್ನುವವರೋ? ಆತ್ಮಸಾಕ್ಷಾತ್ಕಾರವನ್ನು ಮಾಡಿಕೊಳ್ಳುವುದರ ಮೂಲಕ ನೀನು ಮುಕ್ತಿಯನ್ನು ಗಳಿಸಿಕೊಂಡೆ ಎಂದೇ ಇಟ್ಟುಕೊ. ಆದರೆ ಅದರಿಂದ ಪ್ರಪಂಚಕ್ಕೇನು ಪ್ರಯೋಜನವಾದಂತಾಯಿತು? ನಾವು ಸಮಸ್ತ ವಿಶ್ವವನ್ನೇ ಮುಕ್ತಿ ಯೆಡೆಗೆ ಒಯ್ಯಬೇಕು. ನಾವಿನ್ನು ಈ ಮಹಾಮಾಯೆಯ ರಾಜ್ಯದಲ್ಲಿ ದಹನಕಾರ್ಯವನ್ನು ಪ್ರಾರಂಭಿಸೋಣ. ಆಗ ಮಾತ್ರವೇ ನಾವು ನಿತ್ಯ-ಶಾಶ್ವತ-ಸನಾತನ ಸತ್ಯದಲ್ಲಿ ನೆಲೆಗೊಳ್ಳುವೆವು. ಓ, ಆ ಆನಂದಕ್ಕೆ ಸರಿಸಾಟಿ ಯಾವುದಿದೆ? ಅದು ಅಗಾಧ, ಆಗಸದಂತೆ ಅನಂತ! ಆ ಸ್ಥಿತಿಯಲ್ಲಿ, ನಿನ್ನತನವನ್ನೇ ಮೀರಿದ ಆ ಸ್ಥಿತಿಯಲ್ಲಿ, ಪ್ರತಿಯೊಂದು ಜೀವಿಯಲ್ಲೂ–ವಿಶ್ವದ ಅಣುಅಣು ವಿನಲ್ಲೂ–ನಿನ್ನ ಆತ್ಮವನ್ನೇ ನೀನು ಕಾಣುತ್ತಿರುವ ಆ ಸ್ಥಿತಿಯಲ್ಲಿ ನೀನು ಮಾತಿಲ್ಲದವನಾಗುವೆ! ಈ ಸ್ಥಿತಿಯನ್ನು ಸಾಕ್ಷಾತ್ಕರಿಸಿಕೊಂಡಾಗ, ಪ್ರತಿಯೊಬ್ಬನನ್ನೂ ಪರಮ ಪ್ರೀತಿಯಿಂದ ಹಾಗೂ ಅತ್ಯಂತ ಅನುಕಂಪೆಯಿಂದ ನೋಡದೆ ಇರಲು ನಿನ್ನಿಂದ ಸಾಧ್ಯವೇ ಇಲ್ಲ. ನಿಜವಾದ ಅನುಷ್ಠಾನ ವೇದಾಂತವೆಂದರೆ ಇದೇ.”

