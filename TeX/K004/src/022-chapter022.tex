
\chapter{ಉದಾರದೃಷ್ಟಿಯ ಸುಧಾರಕ}

\noindent

ಸ್ವಾಮೀಜಿಯವರು ಬಲರಾಮ ಬಾಬುವಿನ ಮನೆಯಲ್ಲಿ ವಾಸವಾಗಿದ್ದ ಈ ದಿನಗಳಲ್ಲೇ ಒಮ್ಮೆ ಕಲ್ಕತ್ತದ ಪ್ರಾಣಿವನಕ್ಕೆ \eng{(Zoological Garden)} ಭೇಟಿಕೊಟ್ಟರು. ಅವರೊಂದಿಗೆ ಸ್ವಾಮಿ ಯೋಗಾನಂದರು, ಶರಚ್ಚಂದ್ರ ಚಕ್ರವರ್ತಿ ಹಾಗೂ ಸೋದರಿ ನಿವೇದಿತಾ ಇದ್ದರು. ಪ್ರಾಣಿವನದ ಮೇಲ್ವಿಚಾರಕರು ಸ್ವಾಮೀಜಿಯವರನ್ನು ಸ್ವಾಗತಿಸಿ ವನದ ಎಲ್ಲ ಪ್ರಾಣಿಗಳ ಬಳಿಗೂ ಕರೆ ದೊಯ್ದರು. ಹುಲಿ-ಸಿಂಹಗಳಿಗೆ ಆಹಾರ ಕೊಡುವುದನ್ನು ನೋಡಲು ಸ್ವಾಮೀಜಿ ಇಷ್ಟಪಟ್ಟರು. ಮೇಲ್ವಿಚಾರಕರು ಕೂಡಲೇ ಅದಕ್ಕೆ ವ್ಯವಸ್ಥೆ ಮಾಡಿಸಿದರು. ಇದಾದ ಬಳಿಕ ಅಲ್ಲಿನ ಬಗೆಬಗೆಯ ಹಾವುಗಳು ಅವರ ಗಮನ ಸೆಳೆದುವು. ಈ ಸರೀಸೃಪಗಳ (ಎಂದರೆ ಹಾವಿನ ವರ್ಗದ ಪ್ರಾಣಿಗಳ) ಆವಿರ್ಭಾವ-ವಿಕಾಸ ಹೇಗಾಯಿತು ಎಂಬುದರ ಕುರಿತಾಗಿ ಸ್ವಾಮೀಜಿ ಅಲ್ಲಿನ ಮೇಲಧಿಕಾರಿ ಯೊಂದಿಗೆ ಬಹಳ ಹೊತ್ತು ಸಂಭಾಷಿಸಿದರು. ಇದಾದ ನಂತರ ಕೋತಿಗಳು. ಈ ಕೋತಿಗಳಲ್ಲಿ ಕೆಲವು ಜಾತಿಗಳಂತೂ ಹೆಚ್ಚುಕಡಿಮೆ ಮನುಷ್ಯರಂತೆಯೇ ಕಂಡು ಬಂದುವು. ಸ್ವಾಮೀಜಿ ಅವುಗಳ ಹತ್ತಿರ ತಮಾಷೆಯ ದನಿಯಲ್ಲಿ ಮಾತನಾಡಿದರು, “ಈ ಶರೀರದೊಳಗೆ ಹೇಗೆ ಬಂದಿರಪ್ಪಾ! ಯಾವ ಪೂರ್ವಜನ್ಮದ ಕರ್ಮ ನಿಮ್ಮನ್ನು ಇಲ್ಲಿಗೆ ಕರೆತಂದಿತು!” ಆದರೆ ಈ ಪ್ರಶ್ನೆಗೆ ಉತ್ತರ ಕೊಡಲು ಮನುಷ್ಯರಿಗೇ ಸಾಧ್ಯವಿಲ್ಲದಿರುವಾಗ ಆ ಕೋತಿಗಳು ಏನು ಹೇಳಬಲ್ಲುವು?

ಸ್ವಾಮೀಜಿ ಹಾಗೂ ಅವರ ಸಂಗಡಿಗರಿಗೆ ಮೇಲಧಿಕಾರಿಗಳು ಲಘು ಉಪಾಹಾರವಿತ್ತು ಉಪಚರಿಸಿದರು. ಬಳಿಕ ಡಾರ್ವಿನ್ನನ ವಿಕಾಸವಾದದ ಬಗ್ಗೆ ದೀರ್ಘ ಸಂಭಾಷಣೆ ನಡೆಯಿತು. ಮೇಲಧಿಕಾರಿಗಳು ಪ್ರಾಣಿಶಾಸ್ತ್ರ ಸಸ್ಯಶಾಸ್ತ್ರಗಳಲ್ಲಿ ಪರಿಣತಿಯಿದ್ದವರು. ಅವರು ಡಾರ್ವಿನ್ನನ ವಾದವನ್ನು ಸಂಪೂರ್ಣವಾಗಿ ಬೆಂಬಲಿಸುತ್ತಿದ್ದರು. ಆದರೆ ಸ್ವಾಮೀಜಿ ಡಾರ್ವಿನ್ನನ ವಿಕಾಸವಾದ ವನ್ನು ಒಂದು ಹಂತದವರೆಗೆ ಒಪ್ಪಿಕೊಂಡರೂ, ಅದಕ್ಕಿಂತ ಸಮಗ್ರವಾದ ಹಾಗೂ ಸಮರ್ಪಕ ವಾದ ಸಿದ್ಧಾಂತವನ್ನು ಪತಂಜಲಿಗಳ ಯೋಗಸೂತ್ರಗಳಲ್ಲಿ ಮಾತ್ರ ಕಾಣಬಹುದಾಗಿದೆ ಎಂದು ಅಭಿಪ್ರಾಯಪಟ್ಟರು. ಪ್ರಾಣಿಜೀವನದಲ್ಲಿ ಕಂಡುಬರುವ ನಿರಂತರ ಹೋರಾಟ ಹಾಗೂ ಪ್ರತಿ ಸ್ಪರ್ಧೆಗಳು ವಿಕಾಸ ಹೊಂದುವುದಕ್ಕಾಗಿಯೇ ಎಂದು ಕಂಡುಬಂದರೂ ಕೂಡ ಅದು ಮನುಷ್ಯ ಜೀವನಕ್ಕೇರುವ ವಿಷಯದಲ್ಲಿ ಪ್ರತಿಬಂಧಕವೇ ಆಗಿದೆ. ಹಿಂದೂ ಪುಷಿಗಳ ಪ್ರಕಾರ ಪರಿ ಪೂರ್ಣತೆಯೇ ಮಾನವನ ಸಹಜಸ್ಥಿತಿ. ಆದರೆ ಕೆಲವು ಅಡೆತಡೆಗಳಿಂದಾಗಿ ಈ ಪರಿಪೂರ್ಣತೆಯು ಪ್ರಕಟವಾಗುತ್ತಿಲ್ಲ. ಈ ಅಡೆತಡೆಗಳು ಸರಿಯುತ್ತ ಬಂದಂತೆ ಪರಿಪೂರ್ಣತೆಯು ಪ್ರಕಾಶಕ್ಕೆ ಬರುತ್ತಿರುತ್ತದೆ. ಈ ಅಡೆತಡೆಗಳನ್ನು ಹೋಗಲಾಡಿಸಲು ಸಾಧನಗಳು ಯಾವುವೆಂದರೆ ವಿದ್ಯಾ ಭ್ಯಾಸ ಹಾಗೂ ಸಂಸ್ಕೃತಿ, ಏಕಾಗ್ರತೆ ಹಾಗೂ ಧ್ಯಾನ ಮತ್ತು ಇವೆಲ್ಲವುಗಳಿಗಿಂತ ಹೆಚ್ಚಾಗಿ ತ್ಯಾಗ-ವೈರಾಗ್ಯ. ಇಂತಹ ದಿವ್ಯವಾದ ಸಾಧನೆಗಳಲ್ಲಿ ನಿರತವಾದ ಉನ್ನತ ಮಟ್ಟದ ಮನುಷ್ಯ ವರ್ಗಕ್ಕೆ ಈ ಆಹಾರ ನಿದ್ರಾ ಭಯ ಮೈಥುನಾದಿಗಳ ನಿಯಮವು ಅನ್ವಯವಾಗುವುದಿಲ್ಲ ಎನ್ನುತ್ತಾರೆ ಸ್ವಾಮೀಜಿ. ಉದಾಹರಣೆಗೆ ಪುಷಿಮುನಿಗಳು ಈ ಪ್ರಕೃತಿಯ ಹಿಡಿತದಿಂದ ಪಾರಾಗು ವುದಕ್ಕಾಗಿ, ಪ್ರಾಣಿಜೀವನಕ್ಕೆ ಉಚಿತವಾದ ಸಹಜ ಪ್ರವೃತ್ತಿಗಳನ್ನು ಜಯಿಸುವುದಕ್ಕಾಗಿ, ಇವೆಲ್ಲ ಕ್ಕಿಂತ ಹೆಚ್ಚಾಗಿ ಈ ಮಾನವಸ್ವಭಾವವನ್ನೇ ಕರಗಿಸಿ ಪರಬ್ರಹ್ಮದಲ್ಲಿ ಲೀನಗೊಳಿಸುವುದಕ್ಕಾಗಿ ಹೋರಾಟ ನಡೆಸಿದವರು ಎಂದು ಸ್ವಾಮೀಜಿ ವಿವರಿಸಿದರು.

ಅವರ ಮಾತುಗಳನ್ನು ಕೇಳಿದ ಮೇಲಧಿಕಾರಿ ಬಹಳ ಆನಂದಗೊಂಡು “ಸ್ವಾಮೀಜಿ, ನಿಜಕ್ಕೂ ಇದೊಂದು ಅದ್ಭುತ ಸಿದ್ಧಾಂತ. ಇಂದಿನ ಕಾಲದಲ್ಲಿ ನಮ್ಮ ವಿದ್ಯಾವಂತ ವರ್ಗದವರಿಗೆ ಅವರ ಸಂಕುಚಿತತೆಯನ್ನು ತೋರಿಸಿಕೊಟ್ಟು ಅವರ ತಪ್ಪು ಕಲ್ಪನೆಗಳನ್ನು ತಿದ್ದಲು ಪ್ರಾಚ್ಯ ಪಾಶ್ಚಾತ್ಯ ತತ್ತ್ವಶಾಸ್ತ್ರಗಳಲ್ಲಿ ಪರಿಣತರಾದ ನಿಮ್ಮಂತಹ ವ್ಯಕ್ತಿಗಳು ಬೇಕಾಗಿದ್ದಾರೆ” ಎಂದು ಉದ್ಗರಿಸಿದರು. ಅಂದು ಸಂಜೆ ಬಲರಾಮ ಬಾಬುವಿನ ಮನೆಯಲ್ಲಿ ನೆರೆದಿದ್ದ ಸಂದರ್ಶಕರ, ವಿಶ್ವಾಸಿಗರ ಮುಂದೆ ಸ್ವಾಮೀಜಿಯವರು ಆಧುನಿಕ ಭಾರತದ ಆವಶ್ಯಕತೆಗಳಿಗೆ ಸಂಬಂಧಪಟ್ಟಂತೆ ಅದೇ ವಿಕಾಸವಾದದ ತತ್ತ್ವವನ್ನು ಇನ್ನಷ್ಟು ವಿವರವಾಗಿ ತಿಳಿಸಿದರು. ಅದನ್ನು ಸಂಗ್ರಹವಾಗಿ ಹೀಗೆ ಹೇಳಬಹುದು:

ಡಾರ್ವಿನ್ನನ ವಿಕಾಸವಾದವೆಂಬುದು ಪ್ರಾಣಿಸಾಮ್ರಾಜ್ಯ ಹಾಗೂ ಸಸ್ಯಸಾಮ್ರಾಜ್ಯಗಳಿಗೆ ಮಾತ್ರವೇ ಅನ್ವಯವಾಗಬಲ್ಲುದು, ಮಾನವ ಸಾಮ್ರಾಜ್ಯಕ್ಕಲ್ಲ. ಏಕೆಂದರೆ ಮಾನವನಲ್ಲಿ ವಿಚಾರ ಶಕ್ತಿ ಹಾಗೂ ಜ್ಞಾನಶಕ್ತಿ ಅತಿ ಎತ್ತರಕ್ಕೆ ಬೆಳೆದಿದೆ. ಸಾಧುಸಂತರಲ್ಲೂ ಆದರ್ಶ ಮಾನವರಲ್ಲೂ ಹೋರಾಟ-ಪ್ರತಿಸ್ಪರ್ಧೆಗಳ ಸುಳಿವೇ ಕಂಡುಬರುವುದಿಲ್ಲ; ಇತರರನ್ನು ತುಳಿದು ತಾವು ಮೇಲೇರ ಬೇಕೆಂಬ ಮನೋಭಾವ ಕಾಣಸಿಗುವುದಿಲ್ಲ. ಬದಲಾಗಿ, ಅವರಲ್ಲಿ ತ್ಯಾಗವನ್ನು, ಆತ್ಮಾರ್ಪಣೆ ಯನ್ನು ಕಾಣುತ್ತೇವೆ. ಮನುಷ್ಯ ತ್ಯಾಗಮಾಡಿದಷ್ಟೂ ಹೆಚ್ಚು ಮಹಾತ್ಮನೆನ್ನಿಸಿಕೊಳ್ಳುತ್ತಾನೆ. ವಿಚಾರವಂತ ಮಾನವನ ಹೋರಾಟವೆಲ್ಲ ಅವನ ಅಂತಃಸ್ವಭಾವದೊಂದಿಗೇ. ತನ್ನ ಮನಸ್ಸನ್ನು ಎಷ್ಟರಮಟ್ಟಿಗೆ ಸಂಯಮಗೊಳಿಸಬಲ್ಲನೋ ಅಷ್ಟರಮಟ್ಟಿಗೆ ಅವನು ಮಹಾತ್ಮನೆನ್ನಿಸುತ್ತಾನೆ.

ಸ್ವಾಮೀಜಿ ಈ ವಿಷಯವಾಗಿ ಮನಮುಟ್ಟುವಂತೆ ವಿವರಿಸಿದಾಗ ಅಲ್ಲಿದ್ದವರಲ್ಲೊಬ್ಬರು ಕೇಳಿದರು, “ಹಾಗಾದರೆ ನೀವು ನಮ್ಮ ಶಾರೀರಿಕ ಬೆಳವಣಿಗೆಗೆ ಅಷ್ಟೊಂದು ಪ್ರಾಮುಖ್ಯತೆ ಕೊಡುವುದೇಕೆ?” ಎಂದು. ಆಗ ಸ್ವಾಮೀಜಿ ಗುಡುಗಿದರು:

“ನೀವೆಲ್ಲ ಮನುಷ್ಯರೇನು?! ಆಹಾರ ನಿದ್ರಾ ಭಯ ಮೈಥುನಾದಿಗಳಷ್ಟರಿಂದಲೇ ತೃಪ್ತಿ ಹೊಂದುವ ಮೃಗಗಳಿಗಿಂತ ನೀವು ಹೆಚ್ಚೇನೂ ಅಲ್ಲ! ನಿಮ್ಮಲ್ಲಿ ಒಂದಿಷ್ಟು ಬುದ್ಧಿ ಎನ್ನುವುದು ಇಲ್ಲದೆಹೋಗಿದ್ದರೆ ನೀವೆಲ್ಲ ಈ ಹೊತ್ತಿಗೆ ಚತುಷ್ಪಾದಿಗಳೇ ಆಗಿಬಿಡುತ್ತಿದ್ದಿರಿ. ಆತ್ಮಗೌರವ ಶೂನ್ಯರಾದ ನೀವು ನಿಮ್ಮನಿಮ್ಮೊಳಗೇ ದ್ವೇಷಾಸೂಯೆಗಳನ್ನು ತಾಳಿದ್ದೀರಿ. ಆದ್ದರಿಂದಲೇ ನೀವು ಪಾಶ್ಚಾತ್ಯರ ತಿರಸ್ಕಾರಕ್ಕೆ ಗುರಿಯಾಗಿದ್ದೀರಿ. ನಿಮ್ಮ ಸಿದ್ಧಾಂತ, ನಿಮ್ಮ ಹೆಗ್ಗಳಿಕೆಗಳನ್ನೆಲ್ಲ ಎಸೆಯಿರಿ ಆ ಕಡೆಗೆ. ಮತ್ತು ಶಾಂತಚಿತ್ತದಿಂದ ನಿಮ್ಮ ದೈನಂದಿನ ಕಾರ್ಯಕಲಾಪಗಳ, ವ್ಯವಹಾರಗಳ ವಿಷಯವಾಗಿ ಯೋಚಿಸಿ ನೋಡಿ. ನೀವೆಲ್ಲ ಪ್ರಾಣಿಸಹಜವಾದ ಮನೋಭಾವದವರಾದ್ದರಿಂದ ಮೊದಲು ನೀವು ಈ ಪ್ರಪಂಚದಲ್ಲಿ ಚೆನ್ನಾಗಿ, ಯಶಸ್ವಿಯಾಗಿ ಬದುಕುವ ಮಾರ್ಗವನ್ನು ನಾನು ಬೋಧಿಸುತ್ತಿದ್ದೇನೆ. ನಿಮ್ಮ ಮನಸ್ಸಿನೊಂದಿಗೆ ಇನ್ನೂ ಸಮರ್ಥವಾಗಿ ಹೋರಾಟ ನಡೆಸಲು ಅನುಕೂಲವಾಗುವಂತೆ ನಿಮ್ಮ ಶಾರೀರಿಕ ಶಕ್ತಿಯನ್ನು ಬೆಳೆಸಿಕೊಳ್ಳಿ ಎಂದು ಬೋಧಿಸುತ್ತಿದ್ದೇನೆ. ನಾನು ಮತ್ತೆ ಮತ್ತೆ ಒತ್ತಿ ಹೇಳುತ್ತೇನೆ–ಶಾರೀರಿಕವಾಗಿ ದುರ್ಬಲರಾದವರು ಭಗವಂತನ ಸಾಕ್ಷಾತ್ಕಾರ ಮಾಡಿಕೊಳ್ಳಲು ಅನರ್ಹರು. ಆದರೆ ಒಮ್ಮೆ ಈ ಮನಸ್ಸು ಹತೋಟಿಗೆ ಸಿಕ್ಕಿ ಮನುಷ್ಯ ತನಗೆ ತಾನು ಪ್ರಭುವಾದ ಬಳಿಕ ಈ ಶರೀರ ಬಲವಾಗಿದ್ದರೂ ಸರಿಯೆ ಇಲ್ಲದಿದ್ದರೂ ಸರಿಯೆ, ಅದರಿಂದೇನೂ ಚಿಂತೆಯಿಲ್ಲ. ಏಕೆಂದರೆ ಶರೀರದ ಪ್ರಭುತ್ವ ಅವನ ಮೇಲೆ ಇನ್ನು ನಡೆಯದು.”

ಸ್ವತಃ ಸ್ವಾಮೀಜಿಯವರ ಶರೀರವೇ ಆ ದಿನಗಳಲ್ಲಿ ಅಷ್ಟೊಂದು ಆರೋಗ್ಯದಿಂದಿರಲಿಲ್ಲ. ನಿದ್ರೆಯೆಂಬುದು ಅವರ ಪಾಲಿಗೆ ಅಪರೂಪದ ವಸ್ತುವಾಗಿಬಿಟ್ಟಿತ್ತು. ಅವರ ಕಾಯಿಲೆಯು ಅವರ ಮೆದುಳನ್ನು ಸದಾ ಚಟುವಟಿಕೆಯಿಂದಿರುವಂತೆ ಮಾಡಿತ್ತು. ಒಂದಷ್ಟು ಸರಿಯಾದ ವಿಶ್ರಾಂತಿ ಯನ್ನು ಅವರು ಎಷ್ಟರ ಮಟ್ಟಿಗೆ ಬಯಸುತ್ತಿದ್ದರೆಂಬುದನ್ನು ಈ ಘಟನೆಯಿಂದ ತಿಳಿದುಕೊಳ್ಳ ಬಹುದು:

ಅಂದು ಪೂರ್ಣ ಸೂರ್ಯಗ್ರಹಣ. ಬಲರಾಮ ಬಾಬುವಿನ ಮನೆಯಲ್ಲಿದ್ದ ಸ್ವಾಮೀಜಿ ಊಟ ಮಾಡಿ ವಿಶ್ರಾಂತಿ ತೆಗೆದುಕೊಳ್ಳುತ್ತಿದ್ದರು. ಆಗ ಸೂರ್ಯಗ್ರಹಣ ಪ್ರಾರಂಭವಾಯಿತೆಂಬುದನ್ನು ಸೂಚಿಸಲು ಶಂಖ ಜಾಗಟೆಗಳನ್ನು ಬಾಜಿಸಿದ ಶಬ್ದ ಕೇಳಿ ಬಂದಿತು. ಆಗ ಸ್ವಾಮೀಜಿ “ಇದೀಗ ಸೂರ್ಯಗ್ರಹಣ ಪ್ರಾರಂಭವಾಗಿದೆ. ನಾನು ಸ್ವಲ್ಪ ವಿಶ್ರಾಂತಿ ತೆಗೆದುಕೊಳ್ಳುತ್ತೇನೆ” ಎಂದರು. ಬಳಿಕ ಸ್ವಲ್ಪ ಹೊತ್ತಿನ ಮೇಲೆ ಎಲ್ಲೆಲ್ಲೂ ಕತ್ತಲು ಆವರಿಸಿಬಿಟ್ಟಿತು. ಅದನ್ನು ಕಂಡು ಸ್ವಾಮೀಜಿ, “ನಿಜಕ್ಕೂ ಗ್ರಹಣವೆಂದರೆ ಇದು!” ಎಂದು ಉದ್ಗರಿಸಿ ನಿದ್ರೆಮಾಡುವುದಕ್ಕಾಗಿ ಒರಗಿಕೊಂಡರು. ಆದರೆ ಸ್ವಲ್ಪ ಹೊತ್ತಿನಲ್ಲೇ ಮೇಲೆದ್ದು ತಮ್ಮ ಪರಿಚರ್ಯ ಮಾಡುತ್ತಿದ್ದ ಬ್ರಹ್ಮಚಾರಿಗೆ ಹೇಳಿದರು, “ಗ್ರಹಣಕಾಲದಲ್ಲಿ ಏನನ್ನು ಬಯಸಿದರೂ ಅಥವಾ ಮಾಡಿದರೂ ಅದಕ್ಕೆ ನೂರರಷ್ಟು ಹೆಚ್ಚಿನ ಫಲ ಬರುತ್ತದೆ ಎಂಬ ಮಾತಿದೆ. ಆದ್ದರಿಂದ, ನನಗನಿಸಿತು, ಈಗ ನಾನು ಗಾಢವಾಗಿ ನಿದ್ರೆಮಾಡಿದ್ದೇ ಆದರೆ ಮುಂದೆ ಚೆನ್ನಾಗಿ ನಿದ್ರೆ ಮಾಡಲು ಸಾಧ್ಯವಾಗಬಹುದು, ಅಂತ. ಆದ್ದರಿಂದ ಮಲಗಿಕೊಂಡೆ. ಆದರೆ ನನಗೆ ನಿದ್ರೆ ಬಂದರೆ ತಾನೆ? ಕೇವಲ ಹದಿನೈದು ನಿಮಿಷ ನಿದ್ರೆ ಬಂದಿದ್ದರೆ ದೊಡ್ಡ ಮಾತು. ಜಗನ್ಮಾತೆ ಈ ಶರೀರಕ್ಕೆ ಸುಖನಿದ್ರೆಯ ಭಾಗ್ಯವನ್ನು ಬರೆಯಲಿಲ್ಲ.”

ಸ್ವಾಮೀಜಿಯವರು ಕಲ್ಕತ್ತದಲ್ಲೊಂದು ಬಂಗಾಳೀ ದಿನಪತ್ರಿಕೆಯನ್ನು ಇಲ್ಲವೆ ನಿಯತಕಾಲಿಕ ವನ್ನು ಹೊರಡಿಸಬೇಕೆಂದು ಇಚ್ಛಿಸಿದ್ದನ್ನು ಈ ಹಿಂದೆಯೇ ನೋಡಿದ್ದೇವೆ. ಆದರೆ ಅದಿನ್ನೂ ಕಾರ್ಯರೂಪಕ್ಕೆ ಬರಲು ಸಾಧ್ಯವಾಗಿರಲಿಲ್ಲ. ಕಾರಣ ಮತ್ತೇನೂ ಅಲ್ಲ–ಹಣದ ಅಭಾವ ಮತ್ತು ಚಂದಾದಾರರ ಅಭಾವ. ಅಲ್ಲದೆ ಬಂಗಾಳೀ ಭಾಷೆಯಲ್ಲಿ ಪತ್ರಿಕೆಯೊಂದನ್ನು ಹೇಗೋ ಪ್ರಾರಂಭಿ ಸಿದರೂ ನಷ್ಟವೇ ಕಟ್ಟಿಟ್ಟದ್ದು ಎಂಬ ಭಾವನೆ ಅನೇಕರಲ್ಲಿತ್ತು. ಆದ್ದರಿಂದ ಮನೆಮನೆಗೂ ಹೋಗಿ ಚಂದಾದಾರರನ್ನು ಹಿಡಿಯಬೇಕು ಎಂದು ಸ್ವಾಮೀಜಿ ತಮ್ಮ ಸೋದರ ಸಂನ್ಯಾಸಿಗಳಿಗೆ ತಿಳಿಸಿ ದ್ದರು. ಆದರೂ ಹಣದ ಅಭಾವದಿಂದಾಗಿ ಕೆಲಸ ಪ್ರಾರಂಭವಾಗಲೇ ಇಲ್ಲ. ಹೀಗಿರುವಾಗ ಶ್ರೀಮತಿ ಸಾರಾ ಬುಲ್ ಹಾಗೂ ಮಿಸ್ ಜೋಸೆಫಿನ್ ಉತ್ತರ ಬಾರತದ ಪ್ರವಾಸ ಮುಗಿಸಿ ಕೊಂಡು ಕಲ್ಕತ್ತಕ್ಕೆ ಬಂದರು. ಶ್ರೀಮತಿ ಬುಲ್ ಆಗಲೇ ಮಠದ ನಿರ್ಮಾಣಕ್ಕಾಗಿ ಹಲವು ಸಹಸ್ರ ಡಾಲರ್​ಗಳಷ್ಟು ಹಣವನ್ನು ನೀಡಿದ್ದಳು. ಆದರೆ ಜೋಸೆಫಿನ್ನಳ ಬಳಿ ಅಷ್ಟೊಂದು ಹಣವಿರಲಿಲ್ಲ. ಈಗ ಕಲ್ಕತ್ತಕ್ಕೆ ಬಂದ ಮೇಲೆ ಅವಳ ಕೈಯಲ್ಲಿ ಸ್ವಲ್ಪ ಹಣ ಉಳಿದಿತ್ತು. ಅದನ್ನವಳು ಸ್ವಾಮೀಜಿ ಯವರಿಗೆ ನೀಡಲು ಮುಂದಾದಳು. ಅವಳನ್ನು ಸ್ವಾಮೀಜಿ ಕೇಳಿದರು, “ಎಷ್ಟಿದೆ ನಿನ್ನ ಬಳಿ?” “ಎಂಟುನೂರು ಡಾಲರು.” ತಕ್ಷಣವೇ ಸ್ವಾಮೀಜಿ ತ್ರಿಗುಣಾತೀತಾನಂದರತ್ತ ತಿರುಗಿ “ಇಗೊ, ಈಗಲೇ ಹೋಗಿ ನಿನ್ನ ಮುದ್ರಣ ಯಂತ್ರವನ್ನು ಖರೀದಿಸು” ಎಂದರು. ಹೀಗೆ ‘ಉದ್ಬೋಧನ’ ಎಂಬ ಬಂಗಾಳೀ ಪಾಕ್ಷಿಕ ಪತ್ರಿಕೆ ಪ್ರಾರಂಭವಾಯಿತು. (ಮುಂದೆ ಇದು ಮಾಸಪತ್ರಿಕೆಯಾಯಿತು.)

‘ಉದ್ಬೋಧನ’ ಪತ್ರಿಕೆಗೆ ಸಂಪಾದಕರಾಗಿ ಸೇವೆ ಸಲ್ಲಿಸಲು ತ್ರಿಗುಣಾತೀತಾನಂದರು ಮುಂದಾ ದರು. ಆ ಪತ್ರಿಕೆಯಲ್ಲಿ ಯಾವಯಾವ ವಿಷಯಗಳು ಇರಬೇಕು ಎಂಬುದರ ಬಗ್ಗೆ ಸ್ವಾಮೀಜಿ ಯವರು ನಿರ್ದೇಶನ ನೀಡಿದರು–“ನಮ್ಮ ಜನತೆಯ ಶಾರೀರಿಕ ಬೌದ್ಧಿಕ ಆಧ್ಯಾತ್ಮಿಕ ಬೆಳವಣಿಗೆಗೆ ಸಂಬಂಧಪಟ್ಟ ರಚನಾತ್ಮಕ ವಿಷಯಗಳಲ್ಲದೆ ಇತರ ಯಾವ ವಿಷಯಗಳಿಗೂ ಇದರಲ್ಲಿ ಅವಕಾಶವಿರಬಾರದು. ವಿವಿಧ ಮತಪಂಥಗಳ ಶಾಸ್ತ್ರಗಳು ಸಾಹಿತ್ಯ ತತ್ತ್ವ ಕಾವ್ಯ ಕಲೆಯೇ ಮೊದಲಾದವುಗಳಲ್ಲಿ ಪ್ರಣೀತವಾದ ಭಾವನೆಗಳಲ್ಲಿ ಹಾಗೂ ಸಾಧನೆಗಳಲ್ಲಿ ದೋಷ ಹುಡುಕಿ ಟೀಕಿಸುವುದರ ಬದಲು ಅವುಗಳ ಪ್ರಗತಿಗೆ ಸಹಾಯವಾಗುವಂತೆ ಏನು ಮಾಡಬಹುದು ಎಂಬು ದನ್ನು ಪತ್ರಿಕೆ ತೋರಿಸಿಕೊಡಬೇಕು. ಈ ಪತ್ರಿಕೆಯು ಇತರರ ಭಾವನೆಗಳ ಮೇಲೆ ಆಕ್ರಮಣ ಮಾಡಲು ಹೋಗಬಾರದು. ವೇದವೇದಾಂತಗಳ ಅತ್ಯುನ್ನತ ತತ್ತ್ವಗಳನ್ನು ಅತ್ಯಂತ ಸರಳವಾದ ಭಾಷೆಯಲ್ಲಿ ಜನರ ಮುಂದಿಡಬೇಕು. ಹೀಗೆ ನಮ್ಮ ಸಂಸ್ಕೃತಿ ಹಾಗೂ ಜ್ಞಾನರಾಶಿಯನ್ನು ಜನಸಾಮಾನ್ಯರಿಗೂ ತಲುಪುವಂತೆ ಮಾಡುವುದರಿಂದಾಗಿ ಮುಂದೆ ಒಂದಲ್ಲ ಒಂದು ದಿನ ಚಂಡಾಲರನ್ನೂ ಬ್ರಾಹ್ಮಣರ ಮಟ್ಟಕ್ಕೆ ಏರಿಸಲು ಸಾಧ್ಯವಾಗಬಹುದು. ಈ ‘ಉದ್ಬೋಧನ’ ಪತ್ರಿಕೆಯು ಶ್ರೀರಾಮಕೃಷ್ಣರು ಬೋಧಿಸಿದ ಧರ್ಮ ಸಮನ್ವಯ ತತ್ತ್ವಗಳನ್ನು ಪ್ರತಿಪಾದಿಸಬೇಕು ಮತ್ತು ಅವರ ಪ್ರೀತಿ ಪವಿತ್ರತೆ ತ್ಯಾಗಗಳ ಆದರ್ಶವನ್ನು ಲೋಕದಾದ್ಯಂತ ಹರಡಬೇಕು.” ಈ ಪತ್ರಿಕೆಯ ಯಶಸ್ಸಿಗಾಗಿ ಸ್ವಾಮಿ ತ್ರಿಗುಣಾತೀತಾನಂದರು ತಮ್ಮನ್ನೇ ತಾವು ಮರೆತು ಅತ್ಯಂತ ಉತ್ಸಾಹದಿಂದ ಶ್ರಮವಹಿಸಿ ದುಡಿದರು. ಇದನ್ನು ಕಂಡು ಸ್ವಾಮೀಜಿ ನುಡಿದರು, “ಕೇವಲ ನಿಸ್ವಾರ್ಥಿ ಸಂನ್ಯಾಸಿಗಳಿಂದ ಮಾತ್ರ ಈ ಬಗೆಯ ಸಾಹಸಕಾರ್ಯ ಸಾಧ್ಯ!”

ಈ ದಿನಗಳಲ್ಲಿ ಸ್ವಾಮೀಜಿಯವರ ಮಾನಸಿಕ ಸ್ಥಿತಿ ಹೇಗಿತ್ತು ಎಂಬುದು ಅವರು ತಮ್ಮ ಶಿಷ್ಯೆ ಕ್ರಿಸ್ಟೀನಳಿಗೆ ಬರೆದ ಒಂದು ಪತ್ರದಲ್ಲಿ ವ್ಯಕ್ತವಾಗುತ್ತದೆ–“ನನ್ನಂತೆ ಎಲ್ಲ ಕಡೆಗಳಿಂದಲೂ ಕರ್ಮದಿಂದ ಬಂಧಿತನಾದವನು ಎಂದಿಗೂ ಇನ್ನೊಬ್ಬನಿದ್ದಿರಲಾರ; ಹಾಗೆಯೇ ಈ ಕರ್ಮ ಬಂಧನದಿಂದ ಬಿಡಿಸಿಕೊಳ್ಳಲು ನನಗಿಂತ ಹೆಚ್ಚು ಪ್ರಯತ್ನ ಪಟ್ಟವನೂ ಇನ್ನೊಬ್ಬನಿದ್ದಿರಲಾರ. ನನ್ನನ್ನು ಹೆಚ್ಚಾಗಿ ಮುನ್ನಡೆಸಿದ್ದು ಯಾವುದಿರಬಹುದೆಂದು ನಿನಗನ್ನಿಸುತ್ತದೆ?–ಬುದ್ಧಿಯೋ ಹೃದಯವೋ? ಜಗನ್ಮಾತೆಯೇ ನಮ್ಮ ಮಾರ್ಗದರ್ಶಕಿ. ಏನೇನು ನಡೆಯುತ್ತಿದೆಯೊ ಅಥವಾ ಏನೇನು ನಡೆಯಲಿದೆಯೊ ಅವೆಲ್ಲವೂ ಆಕೆಯ ಇಚ್ಛೆ.”

ಡಿಸೆಂಬರ್ ತಿಂಗಳಿನಲ್ಲಿ ಸ್ವಾಮೀಜಿ ಹವಾ ಬದಲಾವಣೆಗಾಗಿ ಕೆಲದಿನಗಳ ಮಟ್ಟಿಗೆ ವೈದ್ಯ ನಾಥಕ್ಕೆ ಹೋಗುವ ಇಂಗಿತವನ್ನು ವ್ಯಕ್ತಪಡಿಸಿದರು. ಮರುವರ್ಷದ ಬೇಸಿಗೆಯಲ್ಲಿ ಮತ್ತೊಮ್ಮೆ ಯೂರೋಪು ಅಮೆರಿಕಗಳಿಗೆ ಹೋಗುವ ಉದ್ದೇಶವೂ ಅವರಿಗಿತ್ತು. ಆ ತಿಂಗಳ ೧೯ರಂದು ಅವರು ನರೇಂದ್ರನಾಥ ಎಂಬೊಬ್ಬರು ಬ್ರಹ್ಮಚಾರಿಗಳೊಂದಿಗೆ ವೈದ್ಯನಾಥಕ್ಕೆ ಹೊರಟರು. ಅಲ್ಲಿ ಅವರು ಪ್ರಿಯನಾಥ ಮುಖರ್ಜಿ ಎಂಬವರ ಅತಿಥಿಗಳಾಗಿ ಉಳಿದುಕೊಂಡರು. ಈ ಸಂದರ್ಭದಲ್ಲಿ ಅವರು ಅಧ್ಯಯನದಲ್ಲಿ ಹಾಗೂ ಪತ್ರಗಳನ್ನು ಬರೆಯುವುದರಲ್ಲಿ ತಮ್ಮ ಹೆಚ್ಚಿನ ಸಮಯವನ್ನು ಕಳೆದರು. ಪ್ರತಿದಿನವೂ ಅನೇಕ ಗಂಟೆಗಳ ಕಾಲ ವಾಯುಸೇವನೆಗಾಗಿ ನಡೆದಾಡುತ್ತಿದ್ದರು. ಮತ್ತು ಬಹುತೇಕ ಅವರು ಏಕಾಂಗಿಯಾಗಿಯೇ ಇರುತ್ತಿದ್ದರು. ಈಗ ಅವರಿಗೆ ಯಾವುದೇ ಸಾರ್ವಜನಿಕ ಕಾರ್ಯ ಅಥವಾ ಸಂಘದ ಕಾರ್ಯಕಲಾಪಗಳ ಹೊಣೆ ಇಲ್ಲದಿದ್ದುದರಿಂದ ಅವರ ಮನಸ್ಸು ಯಾವಾಗಲೂ ಧ್ಯಾನಲೀನವಾಗಿರಲು ಹಾತೊರೆಯುತ್ತಿತ್ತು. ಆದ್ದರಿಂದ ಮನಸ್ಸಿಗೆ ವಿಶ್ರಾಂತಿ ಕೊಡಬೇಕೆಂದು ಎಷ್ಟು ಪ್ರಯತ್ನಿಸಿದರೂ ಸಾಧ್ಯವಾಗುತ್ತಿರಲಿಲ್ಲ. ಒಟ್ಟಿನಲ್ಲಿ ವೈದ್ಯನಾಥಕ್ಕೆ ಬಂದರೂ ಸ್ವಾಮೀಜಿಯವರ ಆರೋಗ್ಯ ಸುಧಾರಿಸಲಿಲ್ಲ. ಅಷ್ಟೇ ಅಲ್ಲ, ಕೆಲವೊಮ್ಮೆ ಪರಿಸ್ಥಿತಿ ಬಿಗಡಾಯಿಸಿದ್ದೂ ಉಂಟು. ಅವರ ಉಬ್ಬಸ ತೀವ್ರಗೊಂಡು ಬಹಳ ಬಾಧೆ ಕೊಟ್ಟಿತು. ಈ ರೋಗದ ಒಂದು ವೈಶಿಷ್ಟ್ಯವೇನೆಂದರೆ ಅದನ್ನು ಅನುಭವಿಸುವವರಿಗೂ ಹಿಂಸೆ, ಸಂಕಟ; ನೋಡುವವರಿಗೂ ಹಿಂಸೆ, ಸಂಕಟ. ಆದರೆ ಉಬ್ಬಸದ ಹೊಡೆತ ತಗುಲಿದಾಗ, ರೋಗಿ ಇನ್ನು ಉಳಿಯಲಾರ ಎಂದು ಎಲ್ಲರಿಗೂ ಅನ್ನಿಸಿದರೂ, ಆ ಕ್ಷಣಕ್ಕೆ ಪ್ರಾಣವಂತೂ ಹೋಗುವುದಿಲ್ಲ. ಆ ದಿನಗಳಲ್ಲಿ ಈಗಿನಂತಹ ಔಷಧಿಗಳೂ ಇರಲಿಲ್ಲ. ಒಮ್ಮೆಯಂತೂ ಸ್ವಾಮೀಜಿಯವರ ಉಸಿರು ಇನ್ನೇನು ನಿಂತೇಹೋದಂತೆ ಆಗಿಬಿಟ್ಟಿತು. ಒಂದು ಎತ್ತರದ ದಿಂಬಿನ ಮೇಲೆ ತಲೆಯಿರಿಸಿ ಸ್ವಾಮೀಜಿಯವರು–ಅವರೇ ಆಮೇಲೆ ಹೇಳಿದಂತೆ–ಸಾವನ್ನು ನಿರೀಕ್ಷಿಸುತ್ತ ಕುಳಿತರು; ಸುತ್ತಲಿರುವವರೆಲ್ಲ ನಿಸ್ಸಹಾಯಕರಾಗಿ ಸುಮ್ಮನೆ ನೋಡುತ್ತ ನಿಂತರು. ಆಗೊಮ್ಮೆ ಈಗೊಮ್ಮೆ ಮೇಲುಸಿರಿನ ಸುಂಯಿಗುಡುವ ಶಬ್ದ. ಇಂತಹ ಅಸಹನೀಯ ಸ್ಥಿತಿಯಲ್ಲಿ, ಆ ಶಬ್ದದ ನಡುವೆಯೇ ಸ್ವಾಮೀಜಿಯವರಿಗೆ ಮಧುರ ದನಿಯಲ್ಲಿ ವೇದವಾಣಿಯೊಂದು ಕೇಳಿ ಬಂದಿತು–‘ಸೋಽಹಂ, ಸೋಽಹಂ!’ ‘ನಾನು ಆತ್ಮ, ನಾನು ಆತ್ಮ’ ಎಂದು. ಬಳಿಕ ಸ್ವಲ್ಪ ಹೊತ್ತಿನಲ್ಲೇ ಸ್ವಾಮೀಜಿ ಚೇತರಿಸಿಕೊಂಡು ಎದ್ದು ಕುಳಿತರು.

ಜನವರಿ ಎರಡನೇ ವಾರದ ಹೊತ್ತಿಗೆ ಸ್ವಾಮೀಜಿಯವರ ಅನಾರೋಗ್ಯಸ್ಥಿತಿ ತುಂಬ ಗಂಭೀರ ವಾಯಿತು. ಆಗ ಅವರು ಕೂಡಲೇ ಹೊರಟುಬರುವಂತೆ ಶಾರದಾನಂದರಿಗೂ ಸದಾನಂದರಿಗೂ ತಂತಿ ಕಳಿಸಿದರು. ಅದಕ್ಕನುಸಾರವಾಗಿ ಅವರಿಬ್ಬರೂ ವೈದ್ಯನಾಥಕ್ಕೆ ಬಂದರು. ಸ್ವಾಮೀಜಿ ಯವರಿನ್ನೂ ಉಬ್ಬಸದಿಂದ ಬಳಲುತ್ತಿದ್ದುದರಿಂದ, ಅವರ ಆರೋಗ್ಯ ಸ್ವಲ್ಪ ಸುಧಾರಿಸಿದ ಕೂಡಲೇ ಕರೆದೊಯ್ಯುವುದೆಂದು ನಿರ್ಧರಿಸಿದರು.

ಸ್ವಾಮೀಜಿಯವರು ಸ್ವತಃ ತಾವೇ ಕಾಯಿಲೆಗಳಿಂದ ನರಳುತ್ತಿದ್ದರೂ ಇತರರ ಸಂಕಟವನ್ನು ಕಂಡರೆ ಅವರ ಹೃದಯ ಕರಗಿಹೋಗುತ್ತಿತ್ತು. ಅವರು ವೈದ್ಯನಾಥದಲ್ಲಿದ್ದಾಗ ಒಮ್ಮೆ ಸ್ವಾಮಿ ನಿರಂಜನಾನಂದರೊಂದಿಗೆ ವಾಯುಸೇವನೆಗಾಗಿ ಹೊರಟಿದ್ದಾರೆ. ದಾರಿಯಲ್ಲಿ ಒಬ್ಬ ಮನುಷ್ಯ ಅಸಹಾಯಕನಾಗಿ ಬಿದ್ದುಕೊಂಡಿದ್ದನು. ಅವನಿಗೆ ಭಯಂಕರ ಆಮಶಂಕೆ. ಅದು ಡಿಸೆಂಬರಾದ್ದ ರಿಂದ ಕೊರೆಯುವ ಚಳಿ ಬೇರೆ. ಆ ಮನುಷ್ಯನ ಮೈ ಮೇಲೆ ಇರುವುದೆಲ್ಲ ಒಂದು ಚಿಂದಿ ಬಟ್ಟೆ ಮಾತ್ರ; ಅದೂ ಕೂಡ ಹೇಸಿಗೆಯಾಗಿತ್ತು. ಅವನು ನೋವಿನಿಂದ ಕೂಗಿಕೊಳ್ಳುತ್ತಿದ್ದನು. ಇವನಿಗೆ ಯಾವ ರೀತಿಯಲ್ಲಿ ನೆರವಾಗಲಿ ಎಂಬುದು ಸ್ವಾಮೀಜಿಯವರ ಚಿಂತೆ. ಮನೆಗೆ ಕರೆದುಕೊಂಡು ಹೋಗೋಣವೆಂದರೆ ಅದು ಪರರ ಮನೆ. ತಾವಲ್ಲಿ ಕೇವಲ ಅತಿಥಿಗಳು ಮಾತ್ರ. ಮನೆಯವರ ಒಪ್ಪಿಗೆಯಿಲ್ಲದೆ ಈ ರೋಗಿಯನ್ನು ಮನೆಗೆ ಕರೆದೊಯ್ದು ಆರೈಕೆ ಮಾಡುವುದು ಸಾಧ್ಯವೆ? ಆದರೆ ಇಂತಹ ಸಂಕಟದಲ್ಲಿರುವ ಮನುಷ್ಯನಿಗೆ ಏನಾದರೊಂದು ವ್ಯವಸ್ಥೆ ಮಾಡದಿರುವುದಾದರೂ ಹೇಗೆ? ಕಡೆಗೆ ಸ್ವಾಮೀಜಿ ನಿರಂಜನಾನಂದರ ಸಹಾಯದಿಂದ ಆ ರೋಗಿಯನ್ನು ಮೆಲ್ಲಗೆ ಎಬ್ಬಿಸಿ ತಾವು ಉಳಿದುಕೊಂಡಿದ್ದ ಪ್ರಿಯನಾಥ ಮುಖರ್ಜಿಗಳ ಮನೆಗೆ ನಿಧಾನವಾಗಿ ನಡೆಸಿಕೊಂಡು ಬಂದರು. ಅಲ್ಲಿ ಅವನ ಶರೀರವನ್ನು ಶುಚಿಗೊಳಿಸಿ ಬೇರೆ ಬಟ್ಟೆಯನ್ನು ತೊಡಿಸಿದರು. ನೋವಿನ ಭಾಗಕ್ಕೆ ಶಾಖ ಕೊಟ್ಟರು. ಆ ರೋಗಿ ಗುಣಮುಖನಾಗುವವರೆಗೂ ಶುಶ್ರೂಷೆ ಮಾಡಿದರು. ಕಡೆಗೆ ಆತ ಅತ್ಯಂತ ಕೃತಜ್ಞತಾಭಾವದಿಂದ ಸ್ವಾಮಿಗಳಿಗೆ ಪ್ರಣಾಮ ಮಾಡಿ ಮನೆಯವರಿಗೂ ಧನ್ಯವಾದ ಗಳನ್ನರ್ಪಿಸಿ ಹೊರಟುಹೋದ. ಇಂತಹ ಗತಿಗೋತ್ರವಿಲ್ಲದ ರೋಗಿಯೊಬ್ಬನನ್ನು ತಮ್ಮ ಒಪ್ಪಿಗೆಯಿಲ್ಲದೆ ಮನೆಗೆ ಕರೆತಂದದ್ದರಿಂದ ಪ್ರಿಯನಾಥ ಮುಖರ್ಜಿಯವರು ಬೇಸರಿಸಿಕೊಳ್ಳು ವುದರ ಬದಲು ತುಂಬ ಮೆಚ್ಚಿಕೊಂಡರು. ಮತ್ತು ಸ್ವಾಮೀಜಿಯವರ ಬುದ್ಧಿಮತ್ತೆಯೆಂಬುದು ಎಷ್ಟು ಮಹತ್ತರವಾದದ್ದೋ ಅವರ ಹೃದಯವೂ ಅಷ್ಟೇ ಕರುಣಾಪೂರ್ಣವಾದದ್ದು ಎಂಬ ಸತ್ಯವನ್ನು ಮನಗಂಡರು.

ವೈದ್ಯನಾಥದಲ್ಲಿದ್ದ ದಿನಗಳಲ್ಲಿ ಸ್ವಾಮೀಜಿಯವರಿಗೆ ಬೇಲೂರು ಮಠದಿಂದ ಪ್ರತಿದಿನ ವೆಂಬಂತೆ ಪತ್ರಗಳು ಬರುತ್ತಿದ್ದುವು. ತನ್ಮೂಲಕ ಅಲ್ಲಿನ ಚಟುವಟಿಕೆಗಳ ಸುದ್ದಿಗಳು ಹಾಗೂ ಆಶ್ರಮವಾಸಿಗಳ ಯೋಗಕ್ಷೇಮದ ವಿವರಗಳು ತಿಳಿದುಬರುತ್ತಿದ್ದುವು. ಡಿಸೆಂಬರ್ ೨ಂರಂದು ಬೇಲೂರು ಮಠಕ್ಕೆ ಶ್ರೀಮಾತೆ ಶಾರದಾದೇವಿಯರು ಭೇಟಿ ನೀಡಿದ್ದರು. ಅಲ್ಲದೆ ಶ್ರೀಮತಿ ಸಾರಾಳ ಮನೆಗೂ ಅವರು ಭೇಟಿಯಿತ್ತಿದ್ದರು. ಈ ವಿಚಾರವನ್ನು ತಿಳಿದು ಸ್ವಾಮೀಜಿಯವರಿಗೆ ಅಪಾರ ಆನಂದ. ಸಾರಾ ಹಾಗೂ ಮಿಸ್ ಮೆಕ್​ಲಾಡ್ ಇಬ್ಬರೂ ಸದ್ಯದಲ್ಲೇ ತಮ್ಮ ದೇಶಕ್ಕೆ ಮರಳುವವ ರಿದ್ದುದರಿಂದ ಡಿಸೆಂಬರ್ ೩ಂರಂದು ಅವರಿಗೆ ಮಠದಲ್ಲಿ ಮಹಾಪ್ರಸಾದವನ್ನು ನೀಡಲಾಯಿತು. ೧೮೯೯ರ ಜನವರಿ ೨ರಂದು ನೀಲಾಂಬರ ಮುಖರ್ಜಿಯವರ ಮನೆಯಲ್ಲಿದ್ದ ಮಠವನ್ನು ಬೇಲೂರಿನ ನೂತನ ಕಟ್ಟಡಕ್ಕೆ ಸಂಪೂರ್ಣವಾಗಿ ಸ್ಥಳಾಂತರಿಸಲಾಯಿತು.

ಸ್ವಾಮೀಜಿ ಕಾಯಿಲೆಯಿಂದಿದ್ದರೂ ತುಂಬ ಚಟುವಟಿಕೆಯಿಂದ ಕೂಡಿದ್ದರು. ಅವರ ಮನಸ್ಸು ಬೃಹತ್ತಾದ ಆಲೋಚನೆಗಳಲ್ಲಿ ನಿರತವಾಗಿತ್ತು. ಅವರು ಸಮಾಜದಿಂದ ಮುಕ್ತರಾಗಿದ್ದರೂ ಸಾಮಾಜಿಕ ಕಟ್ಟುಪಾಡುಗಳ ವಿಷಯದಲ್ಲೇ ಆಗಲಿ, ಸಾಮಾಜಿಕ ಸ್ವಾತಂತ್ರ್ಯದ ವಿಷಯದಲ್ಲೇ ಆಗಲಿ ಅವರಿಗಿದ್ದ ಪ್ರಜ್ಞೆಯನ್ನು ನೋಡಿದರೆ ಅವರೊಬ್ಬ ಸಮಾಜಶಾಸ್ತ್ರಜ್ಞನಂತೆ ಕಂಡುಬರು ತ್ತಾರೆ. ಅವರು ವೈದ್ಯನಾಥದಿಂದ ಮೃಣಾಲಿನಿಬೋಸ್ ಎಂಬ ಗೃಹಸ್ಥಭಕ್ತೆಯೊಬ್ಬಳಿಗೆ ಬರೆದ ಪತ್ರದಿಂದ ಇದನ್ನು ತಿಳಿಯಬಹುದಾಗಿದೆ. ಆ ಪತ್ರ ಹೀಗಿದೆ:

ನಿನ್ನ ಪತ್ರದಲ್ಲಿ ಕೆಲವು ಬಹು ಮುಖ್ಯವಾದ ಪ್ರಶ್ನೆಗಳನ್ನು ಎತ್ತಿದ್ದೀಯೆ...

(೧) ಋಷಿಯಾಗಲಿ, ಮುನಿಯಾಗಲಿ, ದೇವರೇ ಆಗಲಿ ಯಾರಿಗೂ ತಮ್ಮದೇ ಆದ ಹೊಸ ಆಚಾರ ಪದ್ಧತಿಯೊಂದನ್ನು ಸಮಾಜದ ಮೇಲೆ ಬಲಾತ್ಕಾರವಾಗಿ ಹೇರುವ ಶಕ್ತಿಯಿಲ್ಲ. ಆಯಾ ಕಾಲಧರ್ಮಗಳ ಒತ್ತಡಕ್ಕನುಗುಣವಾಗಿ ಸಮಾಜವೇ ತನ್ನ ಉಳಿವಿಗಾಗಿ ಕೆಲವು ಸಂಪ್ರದಾಯ ಗಳನ್ನು ಹಾಕಿಕೊಳ್ಳುತ್ತದೆ. ಪುಷಿಗಳೆಲ್ಲ ಈ ಸಂಪ್ರದಾಯಗಳನ್ನು ಕೇವಲ ಬರೆದಿಟ್ಟಿದ್ದಾರೆ, ಅಷ್ಟೆ. ಹೇಗೆ ಕೆಲವೊಮ್ಮೆ ಒಬ್ಬ ಮನುಷ್ಯ, ಕಾಲಾಂತರದಲ್ಲಿ ಆತ್ಮಘಾತಕವಾಗಬಹುದಾದರೂ ತತ್​ಕ್ಷಣ ದಲ್ಲಿ ಆತ್ಮಸಂರಕ್ಷಣೆಗೆ ನೆರವಾಗಬಲ್ಲ ಕುಟಿಲೋಪಾಯಗಳನ್ನು ಹುಡುಕುತ್ತಾನೆಯೋ ಹಾಗೆಯೇ ಕೆಲವೊಮ್ಮೆ ಆಯಾ ಕಾಲಗಳಲ್ಲುಂಟಾದ ಬಿಕ್ಕಟ್ಟನ್ನು ಎದುರಿಸಲು ಸಮಾಜವು ಕೆಲವು ದಿಢೀರ್ ಉಪಾಯಗಳನ್ನು ಹುಡುಕಿಕೊಂಡರೂ ಮುಂದೆ ಅವು ಅತ್ಯಂತ ದುಷ್ಟ ಪದ್ಧತಿಗಳಾಗಬಹುದು.

ಉದಾಹರಣೆಗೆ ನಮ್ಮ ದೇಶದಲ್ಲೇ ಇರುವ ವಿಧವಾವಿವಾಹದ ಮೇಲಿನ ಬಹಿಷ್ಕಾರದ ವಿಷಯವನ್ನೇ ತೆಗೆದುಕೊ. ಇದಕ್ಕೆ ಸಂಬಂಧಿಸಿದ ನಿಯಮಗಳನ್ನು ಪುಷಿಗಳೋ ಅಥವಾ ಯಾರೋ ಕೆಲವು ದುರ್ಜನರೋ ಮಾಡಿಟ್ಟರೆಂದು ತಿಳಿಯಬೇಡ. ಹೆಂಗಸನ್ನು ಯಾವಾಗಲೂ ತನ್ನ ಕೈಕೆಳಗಿಟ್ಟುಕೊಳ್ಳುವ ಗಂಡಸಿನ ದಮನ ಪ್ರವೃತ್ತಿ ಎಂತೇ ಇದ್ದರೂ, ಈ ನಿಯಮಗಳನ್ನು ಮಾಡಿದವರು ಅವುಗಳನ್ನು ಆ ಕಾಲದ ಸಾಮಾಜಿಕ ಆವಶ್ಯಕತೆಗಳಿಗೆ ಹೊಂದುವಂತೆಯೇ ಮಾಡಿರಬೇಕು. ಈ ಒಂದು ಸಂಪ್ರದಾಯದ ವಿಷಯದಲ್ಲಿ ಎರಡು ಮುಖ್ಯ ಅಂಶಗಳನ್ನು ನಾವು ಗಮನಿಸಬೇಕು–

(ಅ)ಸಮಾಜದ ಕೆಳವರ್ಗದಲ್ಲಿ ವಿಧವಾ ವಿವಾಹದ ಪರಿಪಾಠವಿದೆ.

(ಆ) ಮೇಲ್ವರ್ಗಗಳಲ್ಲಿ ಸ್ತ್ರೀಯರ ಸಂಖ್ಯೆಯು ಪುರುಷರ ಸಂಖ್ಯೆಗಿಂತ ಹೆಚ್ಚು.

ಈಗ ಸಮಾಜದಲ್ಲಿ ಪ್ರತಿಯೊಬ್ಬ ಕನ್ಯೆಗೂ ವಿವಾಹವಾಗಲೇ ಬೇಕು ಎಂದಿಟ್ಟುಕೊಂಡರೆ ಒಬ್ಬಳಿಗೆ ಒಬ್ಬ ಗಂಡ ಸಿಗುವುದೇ ಕಷ್ಟ. ಹಾಗಿರುವಾಗ ಒಬ್ಬೊಬ್ಬಳಿಗೆ ಇಬ್ಬರು ಮೂವರನ್ನು ಎಲ್ಲಿಂದ ತರುವುದು? ಆದ್ದರಿಂದ ಸಮಾಜ ಒಂದು ಗುಂಪಿಗೆ ಅನುಕೂಲವಾಗುವಂತೆ ನಿಯಮವನ್ನು ಮಾಡಿಟ್ಟಿದೆ. ಇದರಿಂದ ಇನ್ನೊಂದು ಗುಂಪಿಗೆ ಸಹಜವಾಗಿಯೇ ಅನನುಕೂಲ ವಾಗುತ್ತದೆ. ಎಂದರೆ, ಈಗಾಗಲೇ ಒಬ್ಬ ಗಂಡನನ್ನು ಪಡೆದಿದ್ದವಳಿಗೆ ಮತ್ತೊಮ್ಮೆ ಮದುವೆ ಯಾಗುವ ಅನುಕೂಲವಿಲ್ಲ. ಆ ರೀತಿ ನಡೆಯಿತೆಂದರೆ ಮತ್ತೊಬ್ಬ ಕನ್ಯೆಗೆ ಗಂಡನಿಲ್ಲದಂತಾ ಗುತ್ತದೆ. ಆದರೆ ಹೆಂಗಸರಿಗಿಂತ ಗಂಡಸರು ಹೆಚ್ಚಾಗಿರುವ ವರ್ಗಗಳಲ್ಲಿ ವಿಧವಾ ವಿವಾಹ ನಡೆಯುತ್ತದೆ. ಕಾರಣವೇನೆಂದರೆ, ನಾನು ಮೇಲೆ ಹೇಳಿದ ಅನನುಕೂಲತೆ ಇಲ್ಲಿಲ್ಲ. ಪಾಶ್ಚಾತ್ಯ ರಾಷ್ಟ್ರಗಳಲ್ಲೂ ಕೂಡ ಬರಬರುತ್ತ ಅವಿವಾಹಿತ ಹೆಣ್ಣುಮಕ್ಕಳಿಗೂ ಗಂಡು ಸಿಗುವುದೇ ತುಂಬ ಕಷ್ಟವಾಗಿಬಿಟ್ಟಿದೆ.

ಜಾತಿಪದ್ಧತಿ ಹಾಗೂ ಇತರ ಸಂಪ್ರದಾಯಗಳ ವಿಷಯದಲ್ಲಿಯೂ ಹಾಗೆಯೇ.

ಆದ್ದರಿಂದ, ಯಾವುದೇ ಸಾಮಾಜಿಕ ನಿಯಮವನ್ನು ಬದಲಾಯಿಸಬೇಕಾದರೂ ಅದರ ಹಿಂದಿರುವ ಆವಶ್ಯಕತೆಗಳನ್ನು ಮೊದಲು ಕಂಡುಹಿಡಿಯಬೇಕು. ಈ ಆವಶ್ಯಕತೆಯನ್ನೇ ಬದಲಾ ಯಿಸುವಂತಾದರೆ, ನಿಯಮಗಳೆಲ್ಲ ತಾವಾಗಿಯೇ ಬಿದ್ದುಹೋಗುತ್ತವೆ. ಇಲ್ಲದಿದ್ದರೆ ಅವನ್ನು ಎತ್ತಿಹಿಡಿಯುವುದರಿಂದಾಗಲಿ ಟೀಕಿಸುವುದರಿಂದಾಗಲಿ ಏನೂ ಆಗುವುದಿಲ್ಲ.

(೨) ಈಗ ಪ್ರಶ್ನೆಯೇನೆಂದರೆ, ಸಮಾಜವಾಗಲಿ ಸಮಾಜದ ನಾನಾ ಕಟ್ಟುಕಟ್ಟಲೆಗಳಾಗಲಿ ಬಹು ಜನಹಿತಕ್ಕಾಗಿಯೇ ರೂಪಿಸಲ್ಪಟ್ಟಿವೆಯೆ? ಎಂಬುದು. ಹಲವರು ಇದಕ್ಕೆ ‘ಹೌದು’ ಎಂದುತ್ತರಿಸು ತ್ತಾರೆ. ಮತ್ತೆ ಕೆಲವರು ಅದನ್ನು ಒಪ್ಪದಿರಬಹುದು. ಸ್ವಾಭಾವಿಕವಾಗಿಯೇ, ಇತರರಿಗಿಂತ ಹೆಚ್ಚು ಶಕ್ತಿವಂತರಾದ ಕೆಲವರು ತಮ್ಮ ಬಲದಿಂದಲೋ, ಕುಯುಕ್ತಿಯಿಂದಲೋ, ಹೇಗೋ ಇತರರನ್ನು ನಿಧಾನವಾಗಿ ತಮ್ಮ ಹತೋಟಿಗೆ ತೆಗೆದುಕೊಳ್ಳುತ್ತಾರೆ. ಬಳಿಕ ತಮ್ಮ ಇಚ್ಛಾನುಸಾರ ಅವರನ್ನು ನಿಯಂತ್ರಿಸುತ್ತಾರೆ. ಇದು ನಿಜವೆನ್ನುವುದಾದರೆ, ತಿಳಿವಳಿಕೆಯಿಲ್ಲದವರಿಗೆ ಸ್ವಾತಂತ್ರ್ಯ ಕೊಡು ವುದು ಅಪಾಯಕರ ಎಂಬ ಮಾತಿನ ಅರ್ಥವಾದರೂ ಏನು?

‘ಸ್ವಾತಂತ್ರ್ಯ’ ಎಂದರೆ, ‘ಒಬ್ಬನು ತನ್ನ ಆಸ್ತಿಯನ್ನು ದುರುಪಯೋಗಪಡಿಸಿಕೊಳ್ಳುವುದೇ ಮೊದಲಾದ ಮಾರ್ಗಗಳಲ್ಲಿ ಅಡಚಣೆಗಳಿಲ್ಲದಿರುವುದು’ ಎಂದಂತೂ ಅಲ್ಲ. ಬದಲಾಗಿ, ಸ್ವಾತಂತ್ರ್ಯವೆಂದರೆ, ಒಬ್ಬನಿಗೆ ತನ್ನ ದೇಹ-ಬುದ್ಧಿ-ಐಶ್ವರ್ಯಗಳನ್ನು ಇತರರಿಗೆ ತೊಂದರೆ ಯಾಗದ ರೀತಿಯಲ್ಲಿ (ಕಾನೂನಿನ ಇತಿಮಿತಿಯೊಳಗೆ) ತನ್ನಿಚ್ಛೆಯಂತೆ ಬಳಸಿಕೊಳ್ಳುವ ಹಕ್ಕು. ಆದರೆ ಸಮಾಜದ, ದೇಶದ, ಪ್ರತಿಯೊಬ್ಬ ಪ್ರಜೆಗೂ ಹಣ, ವಿದ್ಯಾಭ್ಯಾಸ ಹಾಗೂ ಜ್ಞಾನವನ್ನು ಸಂಪಾದಿಸಲು ಸಮಾನ ಅವಕಾಶಗಳಿರಬೇಕು.

ಎರಡನೆಯ ಅಂಶವೆಂದರೆ, ಕೆಲವರು ಹೇಳುತ್ತಾರೆ–ತಿಳಿವಳಿಕೆ ಇಲ್ಲದವರಿಗೆ ಮತ್ತು ಬಡವ ರಿಗೆ ಸ್ವಾತಂತ್ರ್ಯ ಸಿಕ್ಕಿಬಿಟ್ಟರೆ, ಅರ್ಥಾತ್ ಅವರವರ ಶರೀರ-ಐಶ್ವರ್ಯಗಳ ಮೇಲೆ ಸಂಪೂರ್ಣ ಹಕ್ಕು ಇರುವಂತಾಗಿಬಿಟ್ಟರೆ, ಮತ್ತು ಅಂಥವರ ಮಕ್ಕಳಿಗೂ ಕೂಡ ಶ್ರೀಮಂತರ ಹಾಗೂ ಸಮಾಜದ ಮೇಲ್ವರ್ಗದವರ ಮಕ್ಕಳಿಗೆ ಸರಿಸಾಟಿಯಾದ ವಿದ್ಯಾಭ್ಯಾಸ ಹಾಗೂ ಇತರ ಅನು ಕೂಲತೆಗಳು ದೊರಕುವಂತಾಗಿಬಿಟ್ಟರೆ ಅವರೆಲ್ಲ ವಕ್ರಮಾರ್ಗ ಹಿಡಿಯುತ್ತಾರೆ, ದಾರಿ ತಪ್ಪು ತ್ತಾರೆ ಎಂದು. ಹೀಗೆ ಹೇಳುತ್ತಿರುವವರು ಅದನ್ನು ಸಮಾಜದ ಒಳಿತಿಗಾಗಿ ಹೇಳುತ್ತಿದ್ದಾರೆಯೊ ಅಥವಾ ತಮ್ಮ ಸ್ವಾರ್ಥದ ಭರದಲ್ಲಿ ಕುರುಡಾಗಿ ಹೇಳುತ್ತಿದ್ದಾರೆಯೊ? ಇಂಗ್ಲೆಂಡಿನಲ್ಲಿಯೂ ನಾನು ಇದೇ ಮಾತನ್ನು ಕೇಳಿದ್ದೇನೆ–‘ಕೆಳವರ್ಗದವರಿಗೆಲ್ಲ ವಿದ್ಯಾಭ್ಯಾಸ ಸಿಕ್ಕಿಬಿಟ್ಟರೆ ಆಮೇಲೆ ನಮ್ಮ ಸೇವೆ ಮಾಡುವವರು ಯಾರು?’

ಕೈ ಬೆರಳೆಣಿಕೆಯಷ್ಟು ಶ್ರೀಮಂತರ ಭೋಗಕ್ಕಾಗಿ ಲಕ್ಷಾಂತರ ಸ್ತ್ರೀಪುರುಷರು ದಾರಿದ್ರ್ಯದ ನರಕದಲ್ಲಿ, ಅಜ್ಞಾನದ ಕೂಪದಲ್ಲಿ ಮುಳುಗಿರಲಿ; ಏಕೆಂದರೆ ಅವರಿಗೆಲ್ಲ ವಿದ್ಯಾಭ್ಯಾಸ-ಐಶ್ವರ್ಯ ದೊರಕಿಬಿಟ್ಟರೆ ಸಮಾಜ ಏರುಪೇರಾಗಿಬಿಡುತ್ತದೆ!

ಸಮಾಜ ಯಾರಿಂದ ನಿರ್ಮಾಣಗೊಂಡಿದೆ? ಸಮಾಜವೆಂದರೆ ಯಾರು? ನಾನು, ನೀನು ಮತ್ತು ಮೇಲ್ವರ್ಗದ ಇನ್ನು ಕೆಲವರೊ, ಇತರ ಕೋಟ್ಯಂತರ ಜನರೊ?

ಹೋಗಲಿ; ಸಮಾಜವೆಂದರೆ ನಾವೇ ಎಂದಿಟ್ಟುಕೊಂಡರೂ ಇತರರಿಗೆಲ್ಲ ನಾವೇ ದಾರಿ ತೋರುವವರು ಎಂಬ ಪೊಳ್ಳು ನಂಬಿಕೆಗೆ ಆಧಾರವೇನು? ನಾವೇನು ಸರ್ವಜ್ಞರೇ? ‘ಉದ್ಧರೇ ದಾತ್ಮನಾತ್ಮಾನಂ’–ಅವರವರನ್ನು ಅವರವರೇ ಉದ್ಧರಿಸಿಕೊಳ್ಳಬೇಕು (ಭಗವದ್ಗೀತೆ), ಪ್ರತಿ ಯೊಬ್ಬನೂ ತನ್ನ ಮುಕ್ತಿಗೆ ದಾರಿಯನ್ನು ತಾನೇ ಕಂಡುಕೊಳ್ಳಲಿ. ಪ್ರತಿಯೊಂದು ಅರ್ಥದಲ್ಲಿಯೂ ಸ್ವಾತಂತ್ರ್ಯ, ಅರ್ಥಾತ್, ಮುಕ್ತಿಯೆಡೆಗೆ ಪಯಣ–ಇದೇ ಮಾನವನು ಗಳಿಸಬಹುದಾದ ಅತ್ಯು ತ್ಕೃಷ್ಟ ಲಾಭ. ಭೌತಿಕ ಮಾನಸಿಕ ಆಧ್ಯಾತ್ಮಿಕ ಸ್ವಾತಂತ್ರ್ಯದೆಡೆಗೆ ಮುನ್ನಡೆಯುವುದು ಮತ್ತು ಇತರರಿಗೂ ಈ ಕಾರ್ಯದಲ್ಲಿ ನೆರವಾಗುವುದು–ಇದು ಮಾನವನಿಗೆ ಅತ್ಯನರ್ಘ್ಯ ಬಹುಮಾನ. ಈ ಸ್ವಾತಂತ್ರ್ಯದ ಹಾದಿಯಲ್ಲಿ ಬರುವ ಯಾವುದೇ ಸಾಮಾಜಿಕ ಕಟ್ಟಲೆಯೂ ಹಾನಿಕರವಾದುದು. ಅಂಥವುಗಳನ್ನು ಸಾಧ್ಯವಾದಷ್ಟು ಬೇಗ ಕಿತ್ತೊಗೆಯುವ ಕ್ರಮಗಳನ್ನು ಕೈಗೊಳ್ಳಬೇಕು. ಮಾನವನು ಸ್ವಾತಂತ್ರ್ಯದ ಹಾದಿಯಲ್ಲಿ ಮುನ್ನಡೆಯಲು ಯಾವ ವಿಧಿವಿಧಾನಗಳು ನೆರವಾಗುತ್ತವೆಯೋ ಅಂಥವುಗಳನ್ನೆಲ್ಲ ಪ್ರೋತ್ಸಾಹಿಸಬೇಕು...”

ಸ್ವಾಮೀಜಿಯವರ ದೇಹಸ್ಥಿತಿ ಸ್ವಲ್ಪ ಸುಧಾರಿಸುತ್ತಿದ್ದಂತೆಯೇ ಅವರನ್ನು ಸ್ವಾಮಿ ಶಾರದಾ ನಂದರು ಹಾಗೂ ಸದಾನಂದರು ಕಲ್ಕತ್ತಕ್ಕೆ ಕರೆತಂದರು.

