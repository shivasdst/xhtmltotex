
\chapter{ಸಹೃದಯರ ಸಮ್ಮಿಲನ}

\noindent

ಸ್ವಾಮೀಜಿಯವರು ಆಲ್ಮೋರದಲ್ಲಿದ್ದ ವೇಳೆಯಲ್ಲಿ ಬಂಗಾಳದ ಪ್ರಸಿದ್ಧ ಸಂತ-ರಾಷ್ಟ್ರಭಕ್ತರೂ ಶ್ರೀರಾಮಕೃಷ್ಣರ ಭಕ್ತರೂ ಆದ ಅಶ್ವಿನೀಕುಮಾರ ದತ್ತ ಎಂಬವರು ಆ ಊರಿಗೆ ಬಂದಿದ್ದರು. ಇವರು ವಿವೇಕಾನಂದರಿಗಿಂತ ತುಂಬ ಹಿರಿಯರು. ಒಂದು ದಿನ ದತ್ತರ ಅಡಿಗೆಯವನು ಅವರಿಗೊಂದು ಸುದ್ದಿ ಕೊಟ್ಟ: “ಸ್ವಾಮಿ, ಇಲ್ಲೊಬ್ಬರು ವಿಚಿತ್ರ ಬಂಗಾಳೀ ಸಾಧುಗಳಿದ್ದಾರೆ. ಅವರು ಇಂಗ್ಲಿಷ್ ಮಾತನಾಡುತ್ತಾರೆ; ಕುದುರೆ ಸವಾರಿ ಮಾಡುತ್ತಾರೆ; ಒಳ್ಳೇ ರಾಜನ ಹಾಗೆ ಓಡಾಡುತ್ತಾರೆ!” ಎಂದು. ಯಾರಿರಬಹುದು ಅವರು? ಸ್ವಾಮಿ ವಿವೇಕಾನಂದರು ಆಲ್ಮೋರಕ್ಕೆ ಬಂದಿದ್ದಾರೆಂಬ ವಿಷಯವನ್ನು ಬಾಬುಗಳು ಆಗಲೇ ವೃತ್ತಪತ್ರಿಕೆಗಳ ಮೂಲಕ ತಿಳಿದಿದ್ದರು. ಆದ್ದರಿಂದ ಈ ಸಾಧುಗಳು ಅವರೇ–ಆ ‘ಯೋಧ ಸಂನ್ಯಾಸಿ’ಗಳೇ–ಆಗಿರಬೇಕು ಎಂದು ಊಹಿಸಲು ಅವರಿಗೇನೂ ಕಷ್ಟವಾಗಲಿಲ್ಲ.

ಶ್ರೀರಾಮಕೃಷ್ಣರು ಜೀವಿಸಿದ್ದಾಗ ಅಶ್ವಿನೀ ಬಾಬುಗಳೂ ನರೇಂದ್ರನೂ ಒಮ್ಮೆ ಮಾತ್ರ ಭೇಟಿ ಯಾಗಿದ್ದರು. ಬಳಿಕ ಅವರಿಬ್ಬರೂ ಪರಸ್ಪರರನ್ನು ಕಂಡಿರಲಿಲ್ಲ. ಆದರೆ ಪರಸ್ಪರರ ಬಗ್ಗೆ ಇಬ್ಬರಿಗೂ ಅತ್ಯಂತ ಗೌರವಾದರ ಭಾವವಿತ್ತು.

ಈಗ ಅಶ್ವಿನೀಕುಮಾರರು ಸ್ವಾಮೀಜಿಯವರನ್ನು ಹುಡುಕಿಕೊಂಡು ಹೊರಟರು. ಆದರೆ “ಸ್ವಾಮಿ ವಿವೇಕಾನಂದರು ಎಲ್ಲಿದ್ದಾರೆ?” ಎಂದು ಕೇಳಿದರೆ ಆಲ್ಮೋರದಲ್ಲಿ ಯಾರು ಹೇಳುವವರಿಲ್ಲ! ಕಡೆಗೆ ಅವರು ಸ್ವಲ್ಪ ತಲೆ ಓಡಿಸಿ, “ಇಲ್ಲೊಬ್ಬರು ಬಂಗಾಳೀ ಸಾಧುಗಳಿ ದ್ದಾರಂತಲ್ಲ, ಅವರೆಲ್ಲಿದ್ದಾರೆ?” ಎಂದು ಕೇಳಿದರು. ಆಗ ಒಬ್ಬರು “ಓ, ಸವಾರಿ ಸಾಧುಗಳೇ! ಅದೊ ನೋಡಿ, ಅಲ್ಲಿ ಕುದುರೆ ಮೇಲೆ ಬರುತ್ತಿದ್ದಾರೆ. ಇಲ್ಲೇ ಅವರು ಇರುವುದು” ಎಂದು ತೋರಿಸಿದರು. ಕಾಷಾಯ ವಸ್ತ್ರಧಾರಿಗಳೊಬ್ಬರು ಒಂದು ಬಂಗಲೆಯ ಮುಂದೆ ಕುದುರೆಯ ಮೇಲಿನಿಂದ ಇಳಿದರು; ಒಬ್ಬ ಇಂಗ್ಲಿಷ್ ಮನುಷ್ಯ ಕುದುರೆಯನ್ನು ಒಳಗೆ ನಡೆಸಿಕೊಂಡು ಹೋದ; ಆ ಸಂನ್ಯಾಸಿಗಳೂ ಒಳನಡೆದರು. ಇವೆಲ್ಲವನ್ನೂ ಅಶ್ವಿನೀಕುಮಾರರು ದೂರದಿಂದಲೇ ನೋಡಿದರು.

ಸ್ವಲ್ಪ ಸಮಯದ ಮೇಲೆ ಅಶ್ವಿನೀಕುಮಾರ ದತ್ತರು ಬಂಗಲೆಯೊಳಗೆ ಹೋಗಿ, “ಇಲ್ಲಿ ನರೇನ್​ದತ್ತ ಇದ್ದಾರೆಯೇ?” ಎಂದು ವಿಚಾರಿಸಿದರು. ಅಲ್ಲಿದ್ದ ಒಬ್ಬರು ತರುಣ ಸಂನ್ಯಾಸಿಗಳು ಬೇಸರದ ದನಿಯಲ್ಲಿ “ಆ ಹೆಸರಿನವರು ಇಲ್ಲಿ ಯಾರೂ ಇಲ್ಲ. ಅವರು ಯಾವಾಗಲೋ ಸತ್ತು ಹೋದರು. ಇಲ್ಲಿ ಸ್ವಾಮಿ ವಿವೇಕಾನಂದರು ಮಾತ್ರ ಇದ್ದಾರೆ” ಎಂದರು. ಅದಕ್ಕೆ ಅಶ್ವಿನೀ ಬಾಬುಗಳು “ನನಗೆ ಸ್ವಾಮಿ ವಿವೇಕಾನಂದರು ಬೇಕಾಗಿಲ್ಲ. ನನಗೆ ಬೇಕಾದದ್ದು ಪರಮಹಂಸ ದೇವರ ನರೇಂದ್ರ!” ಎಂದು ಮತ್ತೆ ಹೇಳಿದರು. ಹೊರಗಡೆ ಏನೋ ವಾಗ್ವಾದ ನಡೆಯು ತ್ತಿರುವುದು ಸ್ವಾಮೀಜಿಯವರ ಕಿವಿಗೆ ಬಿತ್ತು. ತಕ್ಷಣ ಅವರು ಆ ಸಂನ್ಯಾಸೀ ಶಿಷ್ಯರನ್ನು ಕರೆಸಿ ವಿಷಯವೇನೆಂದು ಕೇಳಿದರು. ಆ ಶಿಷ್ಯರು ತಿಳಿಸಿದರು, “ಯಾರೋ ಒಬ್ಬರು ಮಹಾಶಯರು ಬಂದಿದ್ದಾರೆ. ಅವರಿಗೆ ‘ಪರಮಹಂಸರ ನರೇಂದ್ರ’ ಬೇಕಂತೆ! ನಾನು ಅವರಿಗೆ ಹೇಳಿದೆ– ನರೇಂದ್ರ ತೀರಿಕೊಂಡು ಎಷ್ಟೋ ಕಾಲವಾಯಿತು, ಈಗ ಸ್ವಾಮಿ ವಿವೇಕಾನಂದರು ಬೇಕಾದರೆ ಇಲ್ಲಿದ್ದಾರೆ, ಅಂತ”. ಇದನ್ನು ಕೇಳಿ ಸ್ವಾಮೀಜಿ, “ಓ, ಇದೇನು ಮಾಡಿಬಿಟ್ಟೆ ನೀನು ಹೋಗು, ಈಗಲೇ ಅವರನ್ನು ಕರೆದು ತಾ!” ಎಂದುದ್ಗರಿಸಿದರು.

ಅಶ್ವಿನೀಬಾಬುಗಳನ್ನು ಒಳಗೆ ಕರೆತರಲಾಯಿತು. ಆರಾಮಕುರ್ಚಿಯ ಮೇಲೆ ಕುಳಿತಿದ್ದ ಸ್ವಾಮೀಜಿ, ಬಾಬುಗಳನ್ನು ನೋಡುತ್ತಲೇ ಎದ್ದು ಹೋಗಿ ಅವರನ್ನು ಅತ್ಯಂತ ಆದರದಿಂದ ಬರಮಾಡಿಕೊಂಡರು. ಆಗ ಅಶ್ವಿನೀಬಾಬುಗಳು ನುಡಿದರು, “ಹಿಂದೊಮ್ಮೆ ಶ್ರೀರಾಮಕೃಷ್ಣರು ತಮ್ಮ ಪ್ರೀತಿಯ ನರೇಂದ್ರನೊಂದಿಗೆ ಮಾತುಕತೆಯಾಡುವಂತೆ ನನಗೆ ಹೇಳಿದ್ದರು. ಆದರೆ ಆ ಸಂದರ್ಭದಲ್ಲಿ ಯಾವುದೋ ಕಾರಣದಿಂದ ನರೇಂದ್ರನಿಗೆ ನನ್ನೊಡನೆ ಹೆಚ್ಚು ಮಾತನಾಡಲು ಸಾಧ್ಯವಾಗಲಿಲ್ಲ. ಹದಿನಾಲ್ಕು ವರ್ಷಗಳು ಕಳೆದುಹೋದುವು. ಈಗ ನಾನು ಅವನನ್ನು ಮತ್ತೆ ನೋಡುತ್ತಿದ್ದೇನೆ! ಪರಮಹಂಸರ ನುಡಿ ಎಂದಿಗೂ ವ್ಯರ್ಥವಾಗುವಂತಿಲ್ಲ.” ಆಗ ಸ್ವಾಮೀಜಿ ಯವರು ಅಂದಿನ ಸಂದರ್ಭವನ್ನು ನೆನಪಿಸಿಕೊಂಡು, ಅಂದು ತಮಗೆ ಹೆಚ್ಚು ಮಾತನಾಡಲು ಸಾಧ್ಯವಾಗದಿದ್ದುದಕ್ಕೆ ಕ್ಷಮೆ ಕೋರಿದರು. ಇದನ್ನು ಕಂಡು ಅಶ್ವಿನೀ ಬಾಬುಗಳಿಗೆ ಅತ್ಯಾಶ್ಚರ್ಯ. ಅಷ್ಟು ದೀರ್ಘಕಾಲದ ಹಿಂದೆ ಭೇಟಿಯಾಗಿದ್ದ ತಮ್ಮನ್ನೂ ತಮ್ಮೊಂದಿಗಿನ ಕೇವಲ ಒಂದೆರಡು ನಿಮಿಷದ ಸಂಭಾಷಣೆಯನ್ನೂ ಸ್ವಾಮೀಜಿ ಇನ್ನೂ ನೆನಪಿಟ್ಟುಕೊಂಡಿರಬಹುದೆಂದು ಅವರು ಊಹಿಸಿರಲಿಲ್ಲ.

ಬಳಿಕ ಮಾತನಾಡುವಾಗ ಅಶ್ವಿನೀಬಾಬುಗಳು ತಮ್ಮನ್ನು “ಸ್ವಾಮೀಜಿ” ಎಂದು ಸಂಬೋಧಿಸಿ ದಾಗ ತಕ್ಷಣ ಸ್ವಾಮೀಜಿ ಅವರನ್ನು ತಡೆದು, “ಇದೇನಿದು! ನಾನು ಯಾವಾಗ ನಿಮಗೆ ‘ಸ್ವಾಮಿ’ಯಾದೆ? ಇಂದಿಗೂ ನಾನು ಅಂದಿನ ನರೇಂದ್ರನೇ. ಗುರು ಮಹಾರಾಜರು ನನ್ನನ್ನು ಯಾವ ಹೆಸರಿನಿಂದ ಕರೆದರೋ ಅದು ನನ್ನ ಪಾಲಿನ ಅನರ್ಘ್ಯ ನಿಧಿ. ನೀವು ನನ್ನನ್ನು ಅದೇ ಹೆಸರಿನಿಂದ ಕರೆಯಬೇಕು” ಎಂದರು. ಸ್ವಾಮೀಜಿಯವರ ಸರಳತೆಯನ್ನು ಗಮನಿಸಿದ ಬಾಬುಗಳು ಮನದಲ್ಲೇ ಅವರನ್ನು ಕೊಂಡಾಡಿದರು.

ಅನಂತರ ಗಂಭೀರ ವಿಚಾರಗಳ ಬಗ್ಗೆ ಸಂಭಾಷಿಸುತ್ತ, ಅಶ್ವಿನೀ ಕುಮಾರರು ಪ್ರಶ್ನಿಸಿದರು: “ನೀವು ಪ್ರಪಂಚದಾದ್ಯಂತ ಸಂಚರಿಸಿ, ಮಿಲಿಯಗಟ್ಟಲೆ ಜನರನ್ನು ನಿಮ್ಮ ಆಧ್ಯಾತ್ಮಿಕತೆಯಿಂದ ಸ್ಫೂರ್ತಿಗೊಳಿಸಿದ್ದೀರಿ. ಈಗ ನಮ್ಮ ಭಾರತದ ಉದ್ಧಾರಕ್ಕೆ ದಾರಿಯೇನೆಂದು ಹೇಳಬಲ್ಲಿರಾ?”

ಸ್ವಾಮೀಜಿ: “ನೀವು ಶ್ರೀರಾಮಕೃಷ್ಣರ ಮುಖಕಮಲದಿಂದ, ಹಿಂದೆಯೇ ಏನೇನು ಕೇಳಿ ದ್ದೀರೋ ಅದಕ್ಕಿಂತಲೂ ಹೆಚ್ಚು ನಾನೇನು ಹೇಳಿಯೇನು? ನಾನು ಹೇಳುವುದಿಷ್ಟೆ–ಧರ್ಮವೇ ನಮ್ಮ ಜೀವನಾಡಿ; ಯಾವುದೇ ಸುಧಾರಣೆಯನ್ನು ಜನಸಮುದಾಯವು ಸ್ವೀಕರಿಸುವಂತಾಗಬೇಕಾ ದರೆ, ಅದನ್ನು ಧರ್ಮದ ಮೂಲಕವೇ ತರಬೇಕು. ಅದಕ್ಕೆ ವ್ಯತಿರಿಕ್ತವಾಗಿ ಮಾಡುವುದೆಂದರೆ, ಹರಿಯುತ್ತಿರುವ ಗಂಗೆಯನ್ನು ಹಿಮಾಲಯದಲ್ಲಿನ ಅದರ ಉಗಮಸ್ಥಾನಕ್ಕೆ ತಳ್ಳಿ ಬೇರೆ ಮಾರ್ಗ ವಾಗಿ ಹರಿಯುವಂತೆ ಮಾಡುವಷ್ಟೇ ಅಸಾಧ್ಯವಾದುದು.”

ಅಶ್ವಿನೀಬಾಬು: “ಹಾಗಾದರೆ, ಭಾರತದ ರಾಷ್ಟ್ರೀಯ ಕಾಂಗ್ರೆಸ್ ಈಗ ಏನೇನು ಮಾಡು ತ್ತಿದೆಯೋ ಅದರಲ್ಲಿ ನಿಮಗೆ ವಿಶ್ವಾಸವಿಲ್ಲವೆ?” ಇವರು ಹೇಳುತ್ತಿರುವ ಕಾಂಗ್ರೆಸ್ಸು ಆಗಿನ ಕಾಲದ್ದು ಎಂಬುದನ್ನು ನೆನಪಿಡಬೇಕು. ಆಗ ಅದು, ಉನ್ನತ ವರ್ಗದ ಕೇವಲ ಕೆಲವೇ ಮಂದಿಯ ಸಂಘಟನೆಯಾಗಿತ್ತು. ಅಂದಿನ ಕಾಂಗ್ರೆಸ್ಸು ಹೆಚ್ಚು ಕಾರ್ಯಶೀಲವಾಗಿರಲಿಲ್ಲ. ಮತ್ತು ಅದಕ್ಕೆ ಸಾಮಾನ್ಯರ ಸಂಪರ್ಕವೇ ಇರಲಿಲ್ಲ.

ಸ್ವಾಮೀಜಿ: “ಇಲ್ಲ, ನನಗದರಲ್ಲಿ ವಿಶ್ವಾಸವಿಲ್ಲ. ಆದರೆ ಏನೂ ಆಗದಿರುವುದಕ್ಕಿಂತ ಸ್ವಲ್ಪವಾದರೂ ಆಗುವುದು ಒಳ್ಳೆಯದು. ಆದರೆ ಈ ಕಾಂಗ್ರೆಸ್ಸು ಸಾಮಾನ್ಯರಿಗಾಗಿ ಏನು ಮಾಡುತ್ತಿದೆಯೆಂದು ಹೇಳಿ ನೋಡೋಣ? ಕೇವಲ ಕೆಲವು ಠರಾವುಗಳನ್ನು ಹೊರಡಿಸುವಷ್ಟ ರಲ್ಲೇ ಸ್ವಾತಂತ್ರ್ಯವನ್ನು ಪಡೆಯಬಹುದೆಂದು ಭಾವಿಸಿದಿರಾ? ಮೊಟ್ಟಮೊದಲು ಜನಸಾಮಾನ್ಯ ರನ್ನು ಎಚ್ಚರಗೊಳಿಸಬೇಕು. ಮೊದಲು ಅವರಿಗೆ ಹೊಟ್ಟೆ ತುಂಬ ಅನ್ನ ಸಿಗಲಿ. ಆಮೇಲೆ ಅವರು ತಮ್ಮ ಸ್ವಾತಂತ್ರ್ಯವನ್ನು ತಾವೇ ಗಳಿಸಿಕೊಳ್ಳುತ್ತಾರೆ. ಆ ಬಗ್ಗೆ ಕಾಂಗ್ರೆಸ್ಸು ಏನಾದರೂ ಮಾಡುವು ದಾದರೆ ಅದಕ್ಕೆ ನನ್ನ ಬೆಂಬಲವಿದೆ. ಅಲ್ಲದೆ, ಇಂಗ್ಗಿಷರ ಸದ್ಗುಣಗಳನ್ನೂ ಕೂಡ ನಾವು ಸ್ವೀಕರಿಸಿ ಅರಗಿಸಿಕೊಳ್ಳಬೇಕೆಂಬುದು ನನ್ನ ಅಭಿಮತ.”

ಅಶ್ವಿನೀಬಾಬು: “ನೀವು ‘ಧರ್ಮ’ ಎಂದು ಹೇಳುವಾಗ ಯಾವುದಾದರೊಂದು ನಿರ್ದಿಷ್ಟವಾದ ಮತವನ್ನು ಉದ್ದೇಶಿಸಿ ಹೇಳುತ್ತೀರಾ?”

ಸ್ವಾಮೀಜಿ: “ಗುರುಮಹಾರಾಜರು ಯಾವುದಾದರೊಂದು ನಿರ್ದಿಷ್ಟ ಮತವನ್ನು ಬೋಧಿಸಿ ದರೆ? ಆದರೆ ಅವರು, ವೇದಾಂತವು ಸರ್ವವ್ಯಾಪಕವಾದ–ಸಕಲ ಮತಧರ್ಮಗಳನ್ನೂ ತನ್ನ ತೆಕ್ಕೆಯಲ್ಲಿ ಸ್ವೀಕರಿಸಬಲ್ಲ–ಧರ್ಮವೆಂದು ಹೇಳಿದ್ದಾರೆ. ಆದ್ದರಿಂದ ನಾನೂ ಅದನ್ನೇ ಬೋಧಿಸು ತ್ತೇನೆ. ಆದರೆ ನಾನು ಬೋಧಿಸುವ ಧರ್ಮದ ಸಾರವೆಂದರೆ ಅದು ಶಕ್ತಿ. ಅದು ಉಪನಿಷತ್ತೇ ಆಗಿರಲಿ, ಭಗವದ್ಗೀತೆಯೇ ಆಗಿರಲಿ, ಭಾಗವತವೇ ಆಗಿರಲಿ–ಅದರಲ್ಲಿ ಬೋಧಿಸಿದ ಧರ್ಮ ವೆಂಬುದು ನಮ್ಮ ಹೃದಯದಲ್ಲಿ ಶಕ್ತಿಯನ್ನು ಸ್ಫುರಿಸದಿದ್ದರೆ, ಅದು ನನ್ನ ಪಾಲಿಗೆ ಧರ್ಮವೇ ಅಲ್ಲ. ಶಕ್ತಿಯೇ ಧರ್ಮ; ಶಕ್ತಿಗಿಂತಲೂ ಮಿಗಿಲಾದದ್ದು ಯಾವುದೂ ಇಲ್ಲ.”

ಅಶ್ವಿನೀಬಾಬು: “ದಯವಿಟ್ಟು ನಾನೇನು ಮಾಡಬೇಕೆಂಬುದನ್ನು ತಿಳಿಸಿ.”

ಸ್ವಾಮೀಜಿ: “ನೀವು ಈಗಾಗಲೇ ಶಿಕ್ಷಣ ಕ್ಷೇತ್ರದಲ್ಲಿ ಕಾರ್ಯಮಗ್ನರಾಗಿದ್ದೀರೆಂದು ಕೇಳಿ ತಿಳಿದಿದ್ದೇನೆ. ನಿಶ್ಚಯವಾಗಿಯೂ ಅದೊಂದು ಬಹಳ ಒಳ್ಳೆಯ ಕೆಲಸವೇ ಸರಿ. ನಿಮ್ಮ ಮೂಲಕ ಬಹು ದೊಡ್ಡ ಶಕ್ತಿಯೊಂದು ಕೆಲಸ ಮಾಡುತ್ತಿದೆ. ಮತ್ತು ಈ ಜ್ಞಾನದಾನವು ದಾನಗಳಲ್ಲೆಲ್ಲ ಶ್ರೇಷ್ಠವಾದುದು. ಆದರೆ ಜನಸಾಮಾನ್ಯರಿಗೆಲ್ಲ ವ್ಯಕ್ತಿನಿರ್ಮಾಣಕಾರಿಯಾದ ವಿದ್ಯಾಭ್ಯಾಸ ದೊರಕು ವಂತೆ ನೀವು ನೋಡಿಕೊಳ್ಳಬೇಕು. ಇದರೊಂದಿಗೆ ಶೀಲನಿರ್ಮಾಣವಾಗಬೇಕು. ನಿಮ್ಮ ವಿದ್ಯಾರ್ಥಿ ಗಳ ಶೀಲವು ಸಿಡಿಲಿನಷ್ಟು ಶಕ್ತಿಯುತವಾಗಿರುವಂತೆ ಮಾಡಿ. ಬಂಗಾಳೀ ಯುವಕರ ಮೂಳೆಮೂಳೆ ಗಳೂ ಸಿಡಿಲಿನಂತೆ ಶಕ್ತವಾಗಿರಬೇಕು; ಮತ್ತು, ಅವರು ಭಾರತದ ಗುಲಾಮಗಿರಿಯನ್ನು ಉಚ್ಛಾಟಿಸಬೇಕು. ನೀವು ನನಗೆ ಕೆಲವು ಒಳ್ಳೇ ಶಕ್ತಿಶಾಲಿಗಳಾದ ಯುವಕರನ್ನು ಕೊಡಬಲ್ಲಿರಾ? ಆಗ ನಾನು ಇಡೀ ಜಗತ್ತನ್ನೇ ಅಲುಗಾಡಿಸಬಲ್ಲೆ.

“ಇನ್ನು, ನೀವು ಈ ರಾಧಾ-ಕೃಷ್ಣರ ಪ್ರೇಮದ ಗೀತೆಗಳನ್ನು ಎಲ್ಲಿಯಾದರೂ ಕೇಳಿದರೆ ಚಾವಟಿ ತೆಗೆದುಕೊಂಡು ಸರಿಯಾಗಿ ಬಾರಿಸಿ. ಇಂದ್ರಿಯ ಸಂಯಮವೇ ಇಲ್ಲದವರು ಇಂತಹ ಪ್ರೇಮ ಗೀತೆಗಳನ್ನು ಹಾಡುತ್ತಾರಂತೆ! ಈ ಬಗೆಯ ಹಾಡುಗಳಿಂದ ಇಡೀ ರಾಷ್ಟ್ರವೇ ವಿನಾಶದತ್ತ ಸಾಗುತ್ತಿದೆ. ರಾಧಾ-ಕೃಷ್ಣರ ಸಂಬಂಧದಂತಹ ಅತ್ಯುನ್ನತ ಆದರ್ಶಗಳನ್ನು ಅರಿಯುವಲ್ಲಿ, ಅತಿ ಸ್ವಲ್ಪ ಅಪವಿತ್ರತೆಯೂ ಬಹುದೊಡ್ಡ ಅಡ್ಡಿಯಾಗಿ ನಿಲ್ಲುತ್ತದೆ. ಅದೇನು ತಮಾಷೆಯೆಂದು ತಿಳಿದಿರಾ? ನಾವು ಸಾಕಷ್ಟು ಕಾಲ ಹಾಡಿ-ಕುಣಿದು ಮಾಡಿದ್ದೇವೆ. ಇನ್ನು ಕೆಲವು ಕಾಲ ಸ್ವಲ್ಪ ಸುಮ್ಮನಿರುವುದರಿಂದ ನಷ್ಟವೇನೂ ಆಗುವುದಿಲ್ಲ. ಅಷ್ಟರೊಳಗೆ ರಾಷ್ಟ್ರದಲ್ಲಿ ಒಂದಿಷ್ಟು ಶಕ್ತಿ ಬರಲಿ.

“ಮತ್ತು, ಅಸ್ಪೃಶ್ಯರ ಬಳಿಗೆ ಹೋಗಬೇಕು, ಜಾಡಮಾಲಿಗಳ ಬಳಿಗೆ ಹೋಗಬೇಕು, ಚಮ್ಮಾ ರರ ಬಳಿಗೆ ಹೋಗಬೇಕು. ಹೋಗಿ ಅವರಿಗೆ ಹೇಳಬೇಕು–‘ನೀವೇ ಈ ರಾಷ್ಟ್ರದ ಜೀವಾಳ; ನಿಮ್ಮಲ್ಲಿ ಅನಂತಶಕ್ತಿ ಅಡಗಿದೆ. ಆ ಶಕ್ತಿಯಿಂದ ನೀವು ಪ್ರಪಂಚವನ್ನೇ ಬದಲಿಸಬಲ್ಲಿರಿ. ಎದ್ದು ನಿಲ್ಲಿ, ಬಂಧನಗಳನ್ನು ಕಿತ್ತೊಗೆಯಿರಿ. ಇಡೀ ಜಗತ್ತೇ ನಿಮ್ಮನ್ನು ಕಂಡು ವಿಸ್ಮಯಗೊಳ್ಳುವುದು’ ಎಂದು. ಅವರುಗಳು ಇರುವಲ್ಲಿಗೇ ಹೋಗಿ ಅವರಿಗಾಗಿ ಶಾಲೆಗಳನ್ನು ತೆರೆಯಿರಿ. ಅವರಿಗೆ ಯಜ್ಞೋಪವೀತ ಧಾರಣೆ ಮಾಡಿಸಿ ಬ್ರಹ್ಮೋಪದೇಶ ನೀಡಿ.”

ಸ್ವಾಮೀಜಿಯವರ ಈ ಮಾತುಗಳನ್ನು ಕೇಳಿ ಅಶ್ವಿನೀಕುಮಾರ ದತ್ತರು ರೋಮಾಂಚಿತ ರಾದರು. ಹೀಗೆ ಸಂಭಾಷಣೆ ತುಂಬ ಹೊತ್ತು ನಡೆಯಿತು. ಈಗ ಅವರು ಹೊರಡಲು ಸಿದ್ಧ ರಾದರು. ಆದರೆ ಹೊರಡುವ ಮುನ್ನ ಒಂದೇ ಒಂದು ಪ್ರಶ್ನೆಯನ್ನು ಮುಂದಿಟ್ಟರು: “ಮದ್ರಾಸಿನ ಬ್ರಾಹ್ಮಣರು ನಿಮಗೆ, ‘ನೀವು ಶೂದ್ರರು; ನಿಮಗೆ ವೇದ ಪ್ರಚಾರ ಮಾಡಲು ಹಕ್ಕಿಲ್ಲ’ ಎಂದಾಗ ನೀವು ಹೇಳಿದಿರಂತೆ–‘ನಾನು ಶೂದ್ರನಾದರೆ, ನೀವಿದ್ದೀರಲ್ಲ ಮದ್ರಾಸೀ ಬ್ರಾಹ್ಮಣರು! ನೀವು ಹೊಲೆಯರಿಗಿಂತಲೂ ಹೊಲೆಯರು!’ ಎಂದು. ಇದು ನಿಜವೆ?”

ಸ್ವಾಮೀಜಿ: “ಹೌದು, ನಿಜ.”

ಅಶ್ವಿನೀಬಾಬು: “ನೀವು ಆಚಾರ್ಯರಾಗಿ, ಧರ್ಮಬೋಧಕರಾಗಿ, ಆತ್ಮಸಂಯಮವುಳ್ಳವ ರಾಗಿ ಹೀಗೆಲ್ಲ ಪ್ರತಿಕ್ರಿಯೆ ತೋರಿದ್ದು ಸರಿಯೇ?”

ಸ್ವಾಮೀಜಿ: “ಸರಿಯೆಂದವರು ಯಾರು! ನಾನು ಮಾಡಿದ್ದು ಸರಿ ಎಂದು ನಾನೆಂದೂ ಹೇಳಲಿಲ್ಲ. ಆ ಜನರ ಶುದ್ಧ ಅವಿವೇಕದ ಮಾತು ಕೇಳಿ ನನಗೆ ರೇಗಿತು. ಆದ್ದರಿಂದ ನನ್ನ ಬಾಯಿಂದ ಅಂತಹ ಮಾತುಗಳು ಹೊರಬಿದ್ದುವು. ನಾನೇನು ಮಾಡಲಿ? ಆದರೆ ನಾನು ಅದನ್ನೇ ಸರಿಯೆಂದು ಸಾಧಿಸುವವನಲ್ಲ.”

ಇದನ್ನು ಕೇಳಿ ಅಶ್ವಿನೀಬಾಬುಗಳು ಸಂಪೂರ್ಣ ಮಾರುಹೋದರು. ಅವರು ಸ್ವಾಮೀಜಿಯವ ರನ್ನು ಆಲಂಗಿಸಿಕೊಂಡು ಹೇಳಿದರು, “ನನ್ನ ದೃಷ್ಟಿಯಲ್ಲಿ ನೀವೀಗ ಹಿಂದೆಂದಿಗಿಂತಲೂ ಮೇಲೇರಿದಿರಿ. ಈಗ ನನಗೆ ಅರ್ಥವಾಯಿತು–ಶ್ರೀರಾಮಕೃಷ್ಣರು ನಿಮ್ಮನ್ನು ಅಷ್ಟೇಕೆ ಪ್ರೀತಿಸು ತ್ತಿದ್ದರು, ನೀವು ವಿಶ್ವವಿಜೇತರಾದದ್ದು ಹೇಗೆ, ಎಂದು!”

ಸ್ವಾಮೀಜಿ ತಮ್ಮ ತಪ್ಪನ್ನು ಸರಿಯೆಂದು ಸಾಧಿಸುವ ಪ್ರಯತ್ನ ಮಾಡದೆ ಒಪ್ಪಿಕೊಳ್ಳುವ ಔದಾರ್ಯವನ್ನು ತೋರಿದುದು ಅಶ್ವಿನೀಬಾಬುಗಳ ಹೃದಯವನ್ನು ಸೂರೆಗೊಂಡಿತು. ತಮ್ಮ ಹಳೆಯ ಪರಿಚಿತ ನರೇಂದ್ರನಿಂದು ನಿಜವಾದ ಮಹಾತ್ಮನಾಗಿ ಬೆಳಗುತ್ತಿರುವುದನ್ನು ಕಂಡು ಅವರು ಆನಂದದಿಂದ ಹಿಂದಿರುಗಿದರು.

