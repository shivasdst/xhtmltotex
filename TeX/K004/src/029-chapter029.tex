
\chapter{“ನಾನು ವಿಶಾಲವಾಗುತ್ತಿದ್ದೇನೆ”}

\noindent

೧೮೯೯ರ ಆಗಸ್ಟ್ ೨೮ ಸೋಮವಾರದಂದು ‘ನ್ಯುಮಿಡಿಯನ್​’ ಹಡಗು ನ್ಯೂಯಾರ್ಕ್ ತೀರ ದಲ್ಲಿ ಲಂಗರು ಹಾಕಿತು. ಮೂರೂವರೆ ವರ್ಷಗಳ ಬಳಿಕ ಸ್ವಾಮೀಜಿ ಮತ್ತೆ ಅಮೆರಿಕದ ನೆಲದ ಮೇಲೆ ಪದಾರ್ಪಣ ಮಾಡಿದ್ದರು. ಇದು ಅಮೆರಿಕೆಗೆ ಅವರ ಎರಡನೆಯ ಹಾಗೂ ಕೊನೆಯ ಭೇಟಿ. ಈ ಬಾರಿ ಅವರು ಮತ್ತೆ ಹಿಂದಿನಂತೆಯೇ ಅತ್ಯಂತ ತೀವ್ರ ಚಟುವಟಿಕೆಗಳಿಂದ ಕೂಡಿ ಹಲವಾರು ತಿಂಗಳ ದೀರ್ಘ ಕಾಲವನ್ನು ಕಳೆಯಲಿದ್ದರು–ಯಾರೂ ಊಹಿಸಿರದಿದ್ದ ರೀತಿಯಲ್ಲಿ, ಅವರೊಳಗಿನ ಮಹಾ ಚೇತನವು ಮತ್ತೊಮ್ಮೆ ಜಾಗೃತವಾಗಿ ನಿಲ್ಲುವುದನ್ನು ಮತ್ತು ಜಗತ್ತಿಗೆ ಅವರು ತಮ್ಮ ಸಂದೇಶಗಳನ್ನು ಸಾರಿ ಹೇಳುವುದನ್ನು ನಾವು ನೋಡಲಿದ್ದೇವೆ.

ಆದರೆ ಸ್ವಾಮೀಜಿ ನ್ಯೂಯಾರ್ಕಿಗೆ ಬಂದಾಗ ಅಲ್ಲಿಯೂ ಲಂಡನ್ನಿನಂತೆಯೇ ರಜಾದಿನಗಳ ಕಾಲ; ನಗರದ ಹೆಚ್ಚಿನ ಅನುಕೂಲಸ್ಥ ಜನರು ದೂರದ ಊರುಗಳಿಗೆ ಹೋಗಿರುತ್ತಾರೆ. ಆದ್ದ ರಿಂದ ಸ್ವಾಮೀಜಿಯವರು ನ್ಯೂಯಾರ್ಕಿನಲ್ಲಿ ನಿಲ್ಲದೆ, ಇಲ್ಲಿ ತಮ್ಮ ಆತಿಥೇಯರಾಗಲಿದ್ದ ಫ್ರಾನ್ಸಿಸ್ ಲೆಗೆಟ್ ದಂಪತಿಗಳಿಗೆ ಸೇರಿದ ‘ರಿಡ್ಜ್​ಲಿ ಮ್ಯಾನರ್​’ ನಿವಾಸಕ್ಕೆ ಸೀದಾ ಹೊರಟರು. ರಿಡ್ಜ್​ಲಿ ಇರುವುದು ನ್ಯೂಯಾರ್ಕಿನಿಂದ ಸುಮಾರು ತೊಂಬತ್ತು ಮೈಲಿ ದೂರದಲ್ಲಿ, ಹಡ್ಸನ್ ನದಿಯ ಕಣಿವೆಯಲ್ಲಿ. ಸುಮಾರು ಐವತ್ತು ಎಕರೆಯ ಈ ವಿಶಾಲ ಜಮೀನಿನಲ್ಲಿ ಹಲವಾರು ಕಟ್ಟಡಗಳಿದ್ದುವು. ಇಲ್ಲಿ ‘ಲಿಟ್ಲ್ ಕಾಟೇಜ್​’ ಎಂಬ ಸುಂದರವಾದ ಪುಟ್ಟ ಕಟ್ಟಡವೊಂದನ್ನು ಸ್ವಾಮೀಜಿ ಹಾಗೂ ತುರೀಯಾನಂದರಿಗೆ ಬಿಟ್ಟುಕೊಡಲಾಯಿತು. ಇಲ್ಲಿಗೆ ಬರುವಾಗ ಸ್ವಾಮೀಜಿ ಕೈಯಲ್ಲೊಂದು ದೊಡ್ಡ ಬಾಟಲಿಯನ್ನು ಎಚ್ಚರಿಕೆಯಿಂದ ತೆಗೆದುಕೊಂಡು ಬಂದಿದ್ದರು. ರಿಡ್ಜ್​ಲಿ ಮ್ಯಾನರನ್ನು ತಲುಪುವವರೆಗೂ ಅವರು ಅದನ್ನು ಯಾರ ಕೈಗೂ ಕೊಡದೆ ಬಹು ಜೋಪಾನವಾಗಿ ಹಿಡಿದುಕೊಂಡಿದ್ದರು. ಆ ಬಾಟಲಿಯಲ್ಲಿ ಅಂಥಾದ್ದು ಏನಿರಬಹುದು? ಏನೆಂದರೆ ಒಂದು ಬಗೆಯ ಗೊಜ್ಜು! ಸ್ವಾಮೀಜಿ ಅದನ್ನು ‘ಜೋ’ಳಿಗೋಸ್ಕರ ಭಾರತದಿಂದ ಹೊತ್ತು ತಂದಿದ್ದರು. ‘ಜೋ’ ಎಂದರೆ ಜೋಸೆಫಿನ್ ಮೆಕ್​ಲಾಡ್. ಇವಳು ಫ್ರಾನ್ಸಿಸ್ ಲೆಗೆಟ್ಟರ ಸೋದರ ಸೊಸೆ. ಅವಳಿಗೆ ಇಷ್ಟವಾದ ಗೊಜ್ಜನ್ನು ಸ್ವಾಮೀಜಿ ಭಾರತದಿಂದ ಮಾಡಿಸಿಕೊಂಡು ತಂದಿದ್ದಾರೆ! ಅವರ ಈ ವಿಶ್ವಾಸಕ್ಕೆ ‘ಜೋ’ ಮೂಕಳಾಗಿಬಿಟ್ಟಳು.

ಸ್ವಾಮೀಜಿ ವಾಸವಾಗಿದ್ದ ಆ ಸ್ಥಳ ಸಾಕಷ್ಟು ವಿಶಾಲವಾದದ್ದು. ಹಸಿರು ಹುಲ್ಲು ಬಯಲು ಹಾಸಿಗೆಯಂತೆ ಹರಡಿಕೊಂಡಿತ್ತು. ಸುಮಾರು ಇಪ್ಪತ್ತು ಮೈಲಿ ದೂರದಲ್ಲಿ ಗಂಭೀರವಾಗಿ ಎದ್ದು ನಿಂತಿದ್ದ ಕ್ಯಾಟ್​ಸ್ಕಿಲ್ ಪರ್ವತಗಳ ಸಾಲು ಮನಸ್ಸಿಗೊಂದು ಅನಿರ್ವಚನೀಯ ಅನುಭವ ಮಾಡಿ ಕೊಡುತ್ತಿತ್ತು. ಹೀಗೆ ಈ ತೋಟದ ಮನೆಯ ಸುತ್ತ ಪ್ರಸನ್ನ ವಾತಾವರಣ ಪಸರಿಸಿತ್ತು. ಅಲ್ಲಿ ಸ್ವಾಮೀಜಿ, ಮುಂದಿನ ಕಾರ್ಯಯೋಜನೆಯ ವಿಷಯದಲ್ಲಿ ಭಗವದಿಚ್ಛೆ ಏನಿದೆಯೊ ಎಂಬು ದನ್ನು ನಿರೀಕ್ಷಿಸುತ್ತ ಶಾಂತವಾಗಿ ಕಾದುಕುಳಿತರು. ಭಗವದಾಜ್ಞೆ ಬಂದೇ ಬರುತ್ತದೆಂಬುದರಲ್ಲಿ ಅವರಿಗೆ ದೃಢವಿಶ್ವಾಸವಿತ್ತು. ಈ ವೇಳೆಗೆ ಸ್ವಾಮಿ ಅಭೇದಾನಂದರು ಅಲ್ಲಿಗೆ ಆಗಮಿಸಿದ್ದರಿಂದ ಮೂವರು ಸ್ವಾಮಿಗಳ ವಸತಿಗಾಗಿ ಇನ್ನೊಂದು ವಿಶಾಲವಾದ ಕುಟೀರವನ್ನು ಬಿಟ್ಟುಕೊಡ ಲಾಯಿತು. ಮುಂದೆ ಆ ಕುಟೀರಕ್ಕೆ ‘ಸ್ವಾಮೀಜಿಯವರ ಕುಟೀರ’ ಎಂದೇ ಹೆಸರಾಯಿತು.

ಸ್ವಾಮೀಜಿಯವರ ವಾಸದಿಂದಾಗಿ ಲೆಗೆಟ್ ದಂಪತಿಗಳ ಮನೆವಾರ್ತೆಯು ವಿಶಾಲವಾಗಿ ಬೆಳೆಯುತ್ತ ಬಂದಿತು. ಅಲ್ಲಿ ಮಿಸ್ ಮೆಕ್​ಲಾಡಳಲ್ಲದೆ ಲೆಗೆಟ್ ದಂಪತಿಗಳ ಮಕ್ಕಳಾದ ಆಲ್ಬರ್ಟಾ ಮತ್ತು ಹಾಲಿಸ್ಟರ್ ಕೂಡ ಇದ್ದರು. ಲೆಗೆಟ್ಟರ ಸೋದರಳಿಯನಾದ ಥಿಯೊಡರ್ ವಿಟ್​ಮಾರ್ಶ್ ತನ್ನ ಸಂಸಾರಸಮೇತನಾಗಿ ಅಲ್ಲಿಯೇ ಇದ್ದ. ಇವರಲ್ಲದೆ ಮಾಡ್ ಸ್ಟಮ್ ಎಂಬೊಬ್ಬಳು ಕಲಾವಿದೆ ಹಾಗೂ ಇತರ ಕೆಲವರು, ಸ್ವಾಮೀಜಿಯವರಿದ್ದ ವೇಳೆಯಲ್ಲಿ ರಿಡ್ಜ್ ಲಿಯ ಖಾಯಂ ಅತಿಥಿಗಳಾಗಿ ಉಳಿದುಕೊಂಡಿದ್ದರು. ಸ್ವಾಮೀಜಿ ಅಲ್ಲಿ ವಾಸವಾಗಿದ್ದ ಹತ್ತು ವಾರಗಳ ಅವಧಿಯಲ್ಲಿ ಇನ್ನೂ ಹಲವಾರು ಅತಿಥಿಗಳು ಅವರ ದರ್ಶನಾರ್ಥಿಗಳಾಗಿ ಬಂದು ಬೇರೆಬೇರೆ ಅವಧಿಗಳ ಮಟ್ಟಿಗೆ ಇದ್ದರು. ಸೆಪ್ಟೆಂಬರ್ ೩ಂರಂದು ಸೋದರಿ ನಿವೇದಿತಾ ಲಂಡನ್ನಿನಿಂದ ಆಗಮಿಸಿ ಸ್ವಾಮೀಜಿ ಅಲ್ಲಿಂದ ಹೊರಡುವವರೆಗೂ ಇದ್ದಳು. ಈ ನಡುವೆ ಶ್ರೀಮತಿ ಸಾರಾ ಹಾಗೂ ಅವಳ ಮಗಳು ಓಲಿಯಾ ಕೂಡ ಬಂದು ಸ್ವಾಮೀಜಿಯವರನ್ನು ಕೂಡಿ ಕೊಂಡರು. ಶಿಕಾಗೋದಿಂದ ಹೇಲ್ ಕುಟುಂಬವರ್ಗಕ್ಕೆ ಸೇರಿದ ಇಸಾಬೆಲ್ ಹಾಗೂ ಹ್ಯಾರಿಯೆಟ್ ಸೋದರಿಯರು ಬಂದು ಕೆಲದಿನ ಉಳಿದುಕೊಂಡರು. ಸ್ವಾಮೀಜಿಯವರ ಅಗ್ರಗಣ್ಯ ನ್ಯೂಯಾರ್ಕ್ ಶಿಷ್ಯೆ ಶ್ರೀಮತಿ ಎಲೆನ್ ವಾಲ್ಡೊ ಕೂಡ ಬಂದಳು. ಅಲ್ಲದೆ, ಶಿಕಾಗೋದಿಂದ ಶ್ರೀಮತಿ ಫ್ಲಾರೆನ್ಸ್ ಮಿಲ್​ವರ್ಡ್ ಆ್ಯಡಮ್ಸ್ ಮತ್ತು ನ್ಯೂಯಾರ್ಕಿನಿಂದ ಮಿಸ್ ಫ್ಲಾರೆನ್ಸ್ ಗರ್ನ್​ಸೇ ಬಂದರು. ಫ್ಲಾರೆನ್ಸ್ ಗರ್ನ್​ಸೇ ಎಂಬವಳು ನ್ಯೂಯಾರ್ಕಿನ ಭಕ್ತರಾದ ಡಾ ॥ ಎಗ್​ಬರ್ಟ್ ಗರ್ನ್​ಸೇಯವರ ಮಗಳು. ಇವರಲ್ಲಿ ಪ್ರತಿಯೊಬ್ಬರೂ ಸ್ವಾಮೀಜಿಯವರಿಗೆ ಅತ್ಯಂತ ಆತ್ಮೀಯರು, ತುಂಬ ಬೇಕಾದವರು. ಇವರನ್ನೆಲ್ಲ ಭೇಟಿಯಾದದ್ದು ಅವರಿಗೆ ಅತ್ಯಂತ ಆನಂದವನ್ನುಂಟುಮಾಡಿತು. ಈ ನಡುವೆ ಲೆಗೆಟ್ ದಂಪತಿಗಳು ಸ್ವಾಮೀಜಿಯವರ ಆರೋಗ್ಯಸ್ಥಿತಿ ವೇಗವಾಗಿ ಸುಧಾರಿಸುವಂತಾಗಲು ನ್ಯೂಯಾರ್ಕಿನ ಪ್ರಸಿದ್ಧ ತಜ್ಞರಾದ ಡಾ ॥ ಹೆಲ್ಮರ್​ರನ್ನು ಕರೆಸಿದರು. ಹೀಗೆ ರಿಡ್ಜ್​ಲಿ ಮ್ಯಾನರ್ ಅತಿಥಿಗಳಿಂದ ವಿಶ್ವಾಸಿಗಳಿಂದ ತುಂಬಿ ತುಳುಕುತ್ತಿತ್ತು.

ಈ ಎಲ್ಲ ಅತಿಥಿ ಅಭ್ಯಾಗತರ ಪೈಕಿ ಸ್ವಾಮೀಜಿಯವರಿಗೆ ಬಹಳ ಪ್ರಿಯರಾದ ಮತ್ತೊಬ್ಬ ರೆಂದರೆ ಸ್ವಾಮಿ ಅಭೇದಾನಂದರು. ಸ್ವಾಮೀಜಿ ಇಲ್ಲಿಗೆ ಬರುವ ವೇಳೆಯಲ್ಲಿ ಅಭೇದಾನಂದರು ದೂರದ ಸ್ಥಳವೊಂದರಲ್ಲಿ ಉಪನ್ಯಾಸಗಳನ್ನು ನೀಡುತ್ತಿದ್ದುದರಿಂದ ಕೂಡಲೇ ಬಂದು ಸ್ವಾಮೀಜಿ ಯವರನ್ನು ಕೂಡಿಕೊಳ್ಳಲು ಸಾಧ್ಯವಾಗಿರಲಿಲ್ಲ. ಅಭೇದಾನಂದರು ನ್ಯೂಯಾರ್ಕಿನ ವೇದಾಂತ ಸೊಸೈಟಿಯ ಹೊಣೆಯನ್ನು ಹೊತ್ತು ಅಲ್ಲಿನ ಕಾರ್ಯಕಲಾಪಗಳನ್ನು ಯಶಸ್ವಿಯಾಗಿ ಮುಂದು ವರಿಸಿಕೊಂಡು ಬರುತ್ತಿದ್ದರು. ತಮ್ಮ ಗುರುಭಾಯಿಯ ಯಶಸ್ಸಿನ ವಿವರಗಳನ್ನೆಲ್ಲ ಸ್ವಾಮೀಜಿ ಬಹಳ ಸಂತೋಷದಿಂದ ಕೇಳಿ ತಿಳಿದುಕೊಂಡರು. ಅಭೇದಾನಂದರು ಹತ್ತು ದಿನಗಳ ಮಟ್ಟಿಗೆ ತಮ್ಮ ಪ್ರಿಯ ಗುರುಭಾಯಿಗಳೊಂದಿಗೆ ಆನಂದದಿಂದಿದ್ದು ಮತ್ತೆ ನ್ಯೂಯಾರ್ಕಿನ ತಮ್ಮ ಕಾರ್ಯಕ್ಷೇತ್ರದತ್ತ ಹೊರಟರು.

ಸ್ವಾಮೀಜಿಯವರು ರಿಡ್ಜ್​ಲಿಯ ವನ್ಯ ವಾತಾವರಣದಲ್ಲಿ ಒತ್ತಡದ ಕಾರ್ಯಕಲಾಪಗಳಾ ವುವೂ ಇಲ್ಲದೆ ಆರಾಮವಾಗಿರಲು ಸಾಧ್ಯವಾಯಿತು. ಸಂಜೆಯ ವೇಳೆಯಲ್ಲಿ ಸ್ನೇಹಿತರೊಂದಿಗೆ ಸಂತೋಷದಿಂದ ಸಂಭಾಷಿಸುತ್ತ ಆ ಹಳ್ಳಿಯ ಪರಿಸರದಲ್ಲಿ ವಾಯು ಸೇವನೆಗೆ ಹೋಗುತ್ತಿದ್ದರು. ಹೀಗೆ ಸ್ವಾಮೀಜಿಯವರ ದಿವ್ಯ ಸನ್ನಿಧಿಯಿಂದಾಗಿ ಲೆಗೆಟ್ ದಂಪತಿಗಳು, ಅವರ ಪರಿವಾರದವರು, ಅತಿಥಿಗಳು, ವಿಶ್ವಾಸಿಗಳು ಎಲ್ಲರೂ ಆನಂದದಿಂದ ದಿನಕಳೆದರು. ನಿಜಕ್ಕೂ ಆ ದಿನಗಳೇ ಅವರೆಲ್ಲರ ಜೀವನದ ಸಾರ್ಥಕದಿನಗಳು. ಸ್ವಾಮೀಜಿಯವರ ಮಟ್ಟಿಗಂತೂ, ಇಲ್ಲಿ ಕಳೆದ ಹತ್ತು ವಾರಗಳ ಅವಧಿಯು ಅತ್ಯಂತ ವಿಶ್ರಾಂತಿದಾಯಕವೂ ಉಲ್ಲಾಸಕರವೂ ಆಗಿತ್ತು. ಇದು ಅವರ ಜೀವನದಲ್ಲೇ ಒಂದು ಅಪೂರ್ವ ಕಾಲಾವಧಿ. ಲೆಗೆಟ್ ದಂಪತಿಗಳ ಪ್ರೀತಿಯ ಆರೈಕೆಯಿಂದ ಅವರ ಆರೋಗ್ಯವೂ ತುಂಬ ಸುಧಾರಿಸಿತು. ನವೆಂಬರ್ ೧ ರಂದು ಅವರು ಕ್ರಿಸ್ಟೀನಳಿಗೆ ಒಂದು ಪತ್ರದಲ್ಲಿ ಬರೆದರು:

“ಈಗ ನಾನು ತುಂಬ ಆರೋಗ್ಯವಂತನಾಗಿದ್ದೇನೆ, ತುಂಬ ಬಲಿಷ್ಠನಾಗಿದ್ದೇನೆ; ಎಷ್ಟೆಂದರೆ ನಾನೊಂದು ಸಿಂಹದಷ್ಟು ಬಲಿಷ್ಠನಾಗಿದ್ದೇನೆ ಎಂದೆನ್ನಿಸುತ್ತದೆ ನನಗೆ! ಈಗ ನಾನು ಯಾವ ಕಾರ್ಯವನ್ನಾದರೂ ಕೈಗೆತ್ತಿಕೊಳ್ಳಲು ಸಿದ್ಧ.”

ಈ ಪತ್ರವನ್ನು ಬರೆದ ಐದಾರು ದಿನಗಳಲ್ಲೇ ಸ್ವಾಮೀಜಿ ನ್ಯೂಯಾರ್ಕಿಗೆ ಹೊರಟುಬಿಟ್ಟರು. ಇಲ್ಲಿಗೆ ಅವರ ವಿಶ್ರಾಂತಿಯ ದಿನಗಳು ಮುಗಿದಂತಾಯಿತು.

ಸ್ವಾಮೀಜಿಯವರು ರಿಡ್ಜ್​ಲಿಯಿಂದ ಹೊರಡುವ ಮೊದಲೇ ಸ್ವಾಮಿ ತುರೀಯಾನಂದರು ನ್ಯೂಜೆರ್ಸಿಯಲ್ಲಿರುವ ಮಾಂಟ್ ಕ್ಲೇರ್ ಎಂಬಲ್ಲಿಗೆ ಹೊರಟುಬಿಟ್ಟಿದ್ದರು. ಇದು ನ್ಯೂಯಾರ್ಕಿ ನಿಂದ ಇಪ್ಪತ್ತು ಮೈಲಿ ದೂರದಲ್ಲಿದೆ. ಇಲ್ಲಿನ ಶ್ರೀಮತಿ ವೀಲರ್ ಎಂಬವಳು ಹಿಂದೆ ತನ್ನ ಮನೆ ಯಲ್ಲೇ ವೇದಾಂತ ಪ್ರಚಾರ ಕಾರ್ಯಕ್ಕೆ ಸ್ವಾಮಿ ಶಾರದಾನಂದರಿಗೆ ಅವಕಾಶ ಕಲ್ಪಿಸಿಕೊಟ್ಟಿದ್ದಳು. ಅದೇ ಸ್ಥಳದಲ್ಲಿ ಈಗ ತುರೀಯಾನಂದರು ಬಾಲಕಬಾಲಕಿಯರಿಗಾಗಿ ಕಥೆಗಳ ತರಗತಿಗಳನ್ನು ನಡೆಸಲಾರಂಭಿಸಿದರು. ಇದಲ್ಲದೆ ಅವರು ವಾರಕ್ಕೊಮ್ಮೆ ನ್ಯೂಯಾರ್ಕಿಗೆ ಬಂದು ಅಲ್ಲಿಯೂ ಕೂಡ ಎಳೆಯರಿಗಾಗಿ ಹಿತೋಪದೇಶವೇ ಮೊದಲಾದ ಪುಸ್ತಕಗಳಿಂದ ನೀತಿಪರವಾದ ಕಥೆಗಳನ್ನು ಹೇಳುತ್ತ, ಮಕ್ಕಳ ಮನೋಭಾವದಲ್ಲಿ ಸುಧಾರಣೆ ತರುವ ಪ್ರಯತ್ನ ಮಾಡಿದರು. ಬಳಿಕ ಅದೇ ವರ್ಷ ಡಿಸೆಂಬರಿನಲ್ಲಿ ಅವರು ಮಸಾಚುಸೆಟ್ಸ್ ರಾಜ್ಯದ ಕೇಂಬ್ರಿಡ್ಜಿಗೂ ಹೋಗಿ, ಅಲ್ಲಿನ ಕೇಂಬ್ರಿಡ್ಜ್ ಕಾನ್ಫರೆನ್ಸಸ್​ನ ಗಣ್ಯ ಸಭಿಕರಿಗಾಗಿ ‘ಶ್ರೀಶಂಕರಾಚಾರ್ಯ’ ಎಂಬ ವಿಷಯವಾಗಿ ಭಾಷಣ ಮಾಡಿದರು. ಅವರ ಈ ಮೊತ್ತ ಮೊದಲ ಭಾಷಣವು ಹಾರ್ವರ್ಡ್ ವಿಶ್ವವಿದ್ಯಾಲಯದ ಪ್ರೊಫೆಸರುಗಳ ಮೆಚ್ಚುಗೆ ಗಳಿಸಿತು. ಈ ಸಮಾಚಾರವನ್ನು ತಿಳಿದು ಸ್ವಾಮೀಜಿಯವರಿಗೆ ಸಂತೃಪ್ತಿ.

ಅಕ್ಟೋಬರ್ ೧೫ರಂದು ನ್ಯೂಯಾರ್ಕಿನ ವೇದಾಂತ ಸೊಸೈಟಿಯನ್ನು ಹೊಸ ವಿಶಾಲ ಸ್ಥಳ ವೊಂದಕ್ಕೆ ಬದಲಾಯಿಸಲಾಯಿತು. ಒಂದು ವಾರದ ಬಳಿಕ, ಆ ಪುತುವಿನ ತರಗತಿಗಳನ್ನು ಸ್ವಾಮಿ ಅಭೇದಾನಂದರು ಉದ್ಘಾಟಿಸಿದರು. ಈ ವೇಳೆಗಾಗಲೇ ಈ ವೇದಾಂತ ಸೊಸೈಟಿಯು ಅಭೇದಾ ನಂದರ ನೇತೃತ್ವ-ಮಾರ್ಗದರ್ಶನಗಳಲ್ಲಿ ಹೆಸರಾಂತ ಸಂಸ್ಥೆಯಾಗಿ ಬೆಳೆದು ನಿಂತಿತ್ತು. ಅಮೆರಿಕ ದಲ್ಲಿ ವೇದಾಂತದ ಪ್ರಭಾವವನ್ನು ಹರಡುವಲ್ಲಿ ಈ ಸಂಸ್ಥೆ ಬಹಳಷ್ಟು ಯಶಸ್ಸನ್ನು ಸಾಧಿಸಿತ್ತು. ಈ ಅವಧಿಯಲ್ಲಿ ಮತ್ತಷ್ಟು ಜನ ಈ ವೇದಾಂತ ಸೊಸೈಟಿಯ ಸಂಪರ್ಕಕ್ಕೆ ಬಂದು ಅದರ ಉತ್ಸಾಹೀ ಕಾರ್ಯಕರ್ತರಾಗಿದ್ದರು. ಸ್ವಾಮೀಜಿಯವರು ಅಮೆರಿಕೆಗೆ ಆಗಮಿಸಿರುವ ಸುದ್ದಿ ಕೇಳಿದ ಸಂಘದ ಹಳೆಯ ಮತ್ತು ಹೊಸ ಸದಸ್ಯರೆಲ್ಲರೂ ಅವರನ್ನು ಕಾಣಲು, ಅವರ ಮಾತುಗಳನ್ನು ಕೇಳಲು ಅತ್ಯಂತ ಕಾತರರಾಗಿದ್ದರು.

ನವೆಂಬರ್ ೭ರಂದು ರಿಡ್ಜ್​ಲಿಯಿಂದ ನ್ಯೂಯಾರ್ಕಿಗೆ ಬಂದ ಸ್ವಾಮೀಜಿಯವರು, ಅದೇ ದಿನ ಸಂಜೆ ವೇದಾಂತ ಸೊಸೈಟಿಗೆ ಭೇಟಿ ನೀಡಿದರು. ಅಂದಿನ ಸಭೆಯನ್ನು ಅಧ್ಯಕ್ಷಸ್ಥಾನದಿಂದ ನಡೆಸಿಕೊಡುವಂತೆ ಅವರನ್ನು ಕೇಳಿಕೊಳ್ಳಲಾಯಿತು. ಸ್ವಾಮಿ ಅಭೇದಾನಂದರು ಸ್ವಾಮೀಜಿ ಯವರನ್ನು “ವೇದಾಂತ ಸೊಸೈಟಿಯ ಸಂಸ್ಥಾಪಕರು ಹಾಗೂ ಅಮೆರಿಕದಲ್ಲಿ ವೇದಾಂತ ತತ್ತ್ವ ಗಳನ್ನು ಬಿತ್ತರಿಸಿದ ಮಹಾ ಪ್ರವಾದಿ” ಎಂದು ಸಭೆಗೆ ಪರಿಚಯಿಸಿಕೊಟ್ಟರು. ಬಳಿಕ ಸ್ವಾಮೀಜಿ ಪ್ರಶ್ನೋತ್ತರ ತರಗತಿಗಳನ್ನು ನಡೆಸಿಕೊಟ್ಟು, ಸಭಿಕರ ಸಂದೇಹಗಳಿಗೆ ಸಮಾಧಾನ ನೀಡಿದರು.

ನವೆಂಬರ್ ೧ಂರಂದು ಸ್ವಾಮೀಜಿಯವರಿಗೆ ಸೊಸೈಟಿಯ ಪರವಾಗಿ ಸತ್ಕಾರ ಸಮಾರಂಭ ವೊಂದನ್ನು ಏರ್ಪಡಿಸಲಾಯಿತು. ಅಂದು ಅವರ ಹಳೆಯ ಸ್ನೇಹಿತರು, ಶಿಷ್ಯರು, ಭಕ್ತರು, ವಿಶ್ವಾಸಿಗರು–ಎಲ್ಲರೂ ಆಗಮಿಸಿ ತಮ್ಮ ನೆಚ್ಚಿನ ಗುರುದೇವನನ್ನು ಕಣ್ತುಂಬ ಕಂಡು ಆನಂದಿಸಿ ದರು. ಇವರಲ್ಲದೆ, ಸ್ವಾಮೀಜಿಯವರ ವಿಷಯವನ್ನು ಕೇಳಿ, ಅವರ ಪುಸ್ತಕಗಳನ್ನು ಓದಿ ಪ್ರಭಾ ವಿತರಾಗಿದ್ದ ಇತರ ನೂರಾರು ಸ್ತ್ರೀಪುರುಷರು ಆಗಮಿಸಿದ್ದರು. ಅಂದು ನ್ಯೂಯಾರ್ಕಿನ ವೇದಾಂತ ವಿದ್ಯಾರ್ಥಿಗಳು ತಮ್ಮ ಗೌರವ-ಪ್ರೀತಿಗಳ ಸಂಕೇತವಾಗಿ ಸ್ವಾಮೀಜಿಯವರಿಗೆ ಒಂದು ಅಭಿ ನಂದನಾ ಪತ್ರವನ್ನು ಸಮರ್ಪಿಸಿದರು. ಅದಕ್ಕುತ್ತರವಾಗಿ ಸ್ವಾಮೀಜಿ, ಅವರೆಲ್ಲರ ಪ್ರೀತಿ ವಿಶ್ವಾಸಗಳು ತಮ್ಮ ಹೃದಯದಲ್ಲಿ ಎಂತಹ ಆನಂದದ ಲಹರಿಯನ್ನು ಹರಿಯಿಸಿವೆಯೆಂಬುದನ್ನು ಬಣ್ಣಿಸಿದರು.

ಅಲ್ಲಿನ ಜನಮನದಲ್ಲಿ ಸ್ವಾಮೀಜಿ ಬೀರಿದ ಪ್ರಭಾವ ಅಸದೃಶವಾದದ್ದು. ಅಂದಿನ ಆ ಸತ್ಕಾರಕೂಟದಲ್ಲಿದ್ದವರಲ್ಲಿ ಬ್ರಹ್ಮಚಾರಿ ಗುರುದಾಸ್ ಎಂಬ ಅಮೆರಿಕನ್ನರೂ ಒಬ್ಬರು. ಇವರು ಸ್ವಾಮೀಜಿಯವರನ್ನು ಕಂಡದ್ದು ಅದೇ ಮೊದಲ ಬಾರಿ. ಸ್ವಾಮಿ ಅಭೇದಾನಂದರ ಶಿಷ್ಯರಾದ ಇವರು ಮುಂದೆ ಸ್ವಾಮಿ ಅತುಲಾನಂದರೆಂದು ಪ್ರಸಿದ್ಧರಾದರು. \eng{\textit{With the Swamis in America}} (ಸ್ವಾಮಿಗಳೊಂದಿಗೆ ಅಮೆರಿಕದಲ್ಲಿ) ಎಂಬ ಗ್ರಂಥದಲ್ಲಿ ಅತುಲಾನಂದರು ಆ ಸಮಾರಂಭದ ದೃಶ್ಯವನ್ನು ಹೀಗೆ ಚಿತ್ರಿಸಿದ್ದಾರೆ:

“ಅದೊಂದು ಅತ್ಯಂತ ಸಂತೋಷದ ಒಕ್ಕೂಟ. ಅಂದಿನ ಸಮಾರಂಭ ತುಂಬ ಅನೌಪಚಾರಿಕ ವಾಗಿತ್ತು. ತಮ್ಮ ಪ್ರತಿಯೊಬ್ಬ ಸ್ನೇಹಿತನನ್ನು ಕಂಡಾಗಲೂ ಸ್ವಾಮೀಜಿ ಮುಗುಳ್ನಕ್ಕರು, ವಿಶ್ವಾಸದ ಮಾತನಾಡಿದರು, ಇಲ್ಲವೆ ಒಂದು ಚಟಾಕಿ ಹಾರಿಸಿದರು. ಸ್ವಲ್ಪ ಹೊತ್ತು ಅವರು ನೆಲದ ಮೇಲೆ ಸುಖಾಸನದಲ್ಲಿ ಕುಳಿತರು. ಅಲ್ಲಿ ಮಾತುಕತೆ ನಗು ಹರಟೆ ಬೇಕಾದಷ್ಟು ನಡೆಯಿತು. ತಮ್ಮ ಉದ್ಗಾರಗಳಿಂದ ಇಲ್ಲವೆ ಮುಖಭಾವದಿಂದ ಸ್ವಾಮೀಜಿ ತಾವು ತಮ್ಮ ಹಳೆಯ ಪರಿಚಿತರಲ್ಲಿ ಒಬ್ಬನನ್ನೂ ಮರೆತಿಲ್ಲವೆಂಬುದನ್ನು ತೋರಿಸಿದರು... 

“ಆದರೆ ನಾನು ಮೊದಲ ಬಾರಿಗೆ ಅವರನ್ನು ನೋಡುತ್ತಿದ್ದೆನಾದ್ದರಿಂದ, ನನಗೆ ಅವರ ನಡೆನುಡಿಗಳೆಲ್ಲ ತೀರಾ ಸಾಮಾನ್ಯವಾಗಿ ಕಂಡುಬಂದುವು. ಆದ್ದರಿಂದ ಅವರನ್ನು ಎಲ್ಲರಂತೆಯೇ ಒಬ್ಬ ಸಾಮಾನ್ಯ ವ್ಯಕ್ತಿ ಎಂದು ಭಾವಿಸಿದೆ. ಅವರನ್ನು ಎಷ್ಟೋ ಜನ ಕರೆದಿದ್ದಂತೆ ‘ವೇದಾಂತ ಸೊಸೈಟಿಯ ಸಿಂಹ’ ಎಂದು ಕರೆಯುವಷ್ಟು ವಿಶೇಷವೇನೂ ನನಗೆ ತೋರಲಿಲ್ಲ. ದೊಡ್ಡ ಮನುಷ್ಯ ರಾದವರು ಯಾವಾಗಲೂ ಇತರರಿಂದ ಸ್ವಲ್ಪ ಪ್ರತ್ಯೇಕವಾಗಿ ಕುಳಿತುಕೊಳ್ಳುವಂತೆ ಅವರೇನೂ ನಮ್ಮಿಂದ ದೂರವಿರಲಿಲ್ಲ. ಕೋಣೆಯಲ್ಲಿ ಸರಾಗವಾಗಿ ಓಡಾಡಿದರು. ನೆಲದ ಮೇಲೆಯೇ ಕುಳಿತರು, ಸಲ್ಲಾಪ ನಡೆಸಿದರು, ನಕ್ಕರು, ತಮಾಷೆ ಮಾಡಿದರು–ಯಾವುದರಲ್ಲೂ ಔಪಚಾರಿಕತೆ ಯಿಲ್ಲ. ಆದರೆ ಒಂದು ವಿಷಯ ಮಾತ್ರ ನಿಜ: ಅವರ ರಾಜಸಹಜವಾದ ಮೈಕಟ್ಟು-ನಿಲುವು ಗಳನ್ನು ಅವರ ವಿಶಾಲ-ತೇಜೋವಂತ ನಯನದ್ವಯವನ್ನು ನಾನು ಗಮನಿಸಿದ್ದೆ; ಇವುಗಳನ್ನು ಯಾವುದೂ ಮರೆ ಮಾಡುವಂತಿರಲಿಲ್ಲ. ಆದರೆ ಅವರು ಕೆಲವು ಕ್ಷಣ ಕಾಲ ವೇದಿಕೆಯ ಮೇಲೆ ಇತರರೊಂದಿಗೆ ನಿಂತಿದ್ದಾಗ ಮಾತ್ರ, ಮಿಂಚಿನಂತೆ ನನ್ನ ಮನಸ್ಸಿಗೆ ಹೊಳೆಯಿತು–‘ಆಹ್! ಇದೇನು ಭವ್ಯತೆ! ಇದೇನು ಧೀರತೆ! ಇದೇನು ಪುರುಷತ್ವ! ಅವರಿಗೆ ಹೋಲಿಸಿ ನೋಡಿದರೆ ಅವರ ಅಕ್ಕಪಕ್ಕದಲ್ಲಿರುವವರೆಲ್ಲ ತೀರ ಯಃಕಶ್ಚಿತ್ತುಗಳಂತೆ ಕಂಡುಬರುತ್ತಿದ್ದಾರಲ್ಲ!’ ಎಂದು. ಈ ಭಾವನೆ ನನ್ನ ಮನಸ್ಸಿಗೆ ಒಂದು ಆಘಾತದಂತೆ ಬಂದೆರಗಿತು; ನಾನು ಆಶ್ಚರ್ಯಚಕಿತನಾದೆ. ಸ್ವಾಮೀಜಿಯವರನ್ನು ಇತರರಿಂದೆಲ್ಲ ಪ್ರತ್ಯೇಕಿಸಿದ ಆ ಅಂಶ ಯಾವುದಿರಬಹುದು? ಅದು ಅವರ ಎತ್ತರವಿರಬಹುದೆ? ಅಲ್ಲ, ಅಲ್ಲಿ ಅವರಿಗಿಂತಲೂ ಎತ್ತರದ ವ್ಯಕ್ತಿಗಳಿದ್ದಾರೆ. ಹಾಗಾದರೆ ಅದು ಅವರ ಮೈಕಟ್ಟಿರಬಹುದೆ? ಅಲ್ಲ, ಅಲ್ಲಿ ಒಳ್ಳೇ ಕಟ್ಟುಮಸ್ತಾದ, ಅಮೆರಿಕೆಯ ಶ್ರೇಷ್ಠ ಮಾದರಿಯ ಮೈಕಟ್ಟಿನವರಿದ್ದಾರೆ. ಹಾಗಾದರೆ ಇನ್ನಾವ ವೈಶಿಷ್ಟ್ಯವಿರಬಹುದು? ಅವರ ಮುಖದಲ್ಲಿ ಎದ್ದು ಕಾಣುವ ಭಾವಪ್ರಕಾಶವೇ ಆ ವೈಶಿಷ್ಟ್ಯ! ಇದೇ ಅವರಲ್ಲಿ ಇತರರಿಗಿಂತ ಹೆಚ್ಚಾಗಿ ಕಂಡುಬರು ತ್ತಿದ್ದ ವಿಶೇಷತೆ! ಇದು ಅವರ ಪಾವಿತ್ರ್ಯದ ಹೊಳಪಿರಬಹುದೆ? ಏನಿರಬಹುದು? ನನಗದನ್ನು ವಿಶ್ಲೇಷಿಸಿ ತಿಳಿಯಲು ಸಾಧ್ಯವಾಗಲಿಲ್ಲ. ಆಗ ನನಗೆ ಬುದ್ಧನ ವಿಷಯದಲ್ಲಿ ಹೇಳಿರುವ ಮಾತೊಂದು ನೆನಪಿಗೆ ಬಂತು: ‘ಪುರುಷರೊಳಗಿನ ಸಿಂಹ–ಪುರುಷಸಿಂಹ!’ ನಿಜ, ಸ್ವಾಮೀಜಿ ಯವರೊಬ್ಬ ಪುರುಷಸಿಂಹ. ನನಗನ್ನಿಸಿತು–ಸ್ವಾಮೀಜಿಯವರಲ್ಲಿ ಅಪಾರ ಶಕ್ತಿ ತುಂಬಿ ಕೊಂಡಿದೆ; ಅವರು ಮನಸ್ಸು ಮಾಡಿದರೆ ಸ್ವರ್ಗ ಮರ್ತ್ಯ ಪಾತಾಳಗಳನ್ನು ಒಂದುಗೂಡಿಸಿಬಿಡ ಬಲ್ಲರು, ಎಂದು. ಇದೇ ಅವರ ದರ್ಶನದಿಂದ ನನ್ನ ಮನಸ್ಸಿನಲ್ಲುದ್ಭವಿಸಿದ ಬಲವಾದ ಅಭಿಪ್ರಾಯ.”

ನಿಜ, ಸ್ವಾಮೀಜಿಯವರಲ್ಲಿ ಎಣಿಕೆಗೆ ಮೀರಿದ ಶಕ್ತಿಯಿತ್ತು. ಸಮಸ್ತ ಜಗತ್ತನ್ನೇ ಬೆರಗಾಗಿ ಸುವಷ್ಟು ಶಕ್ತಿಯಿತ್ತು. ಆದರೆ ಅದನ್ನು ಅವರೆಂದೂ ಯಾರ ಮೇಲೂ ಪ್ರಭಾವ ಬೀರುವುದಕ್ಕಾಗಿ ಉಪಯೋಗಿಸಿದವರಲ್ಲ; ಬದಲಾಗಿ ಅವರು ಅಂತಹ ಸಂದರ್ಭಕ್ಕೆ ಅವಕಾಶವನ್ನೇ ಕೊಡುತ್ತಿರ ಲಿಲ್ಲ. ಇದಕ್ಕೊಂದು ಪ್ರಸಂಗವನ್ನು ಉದಾಹರಿಸಬಹುದು: ಈ ದಿನಗಳಲ್ಲೇ ಒಮ್ಮೆ ಅವರು ನ್ಯೂಯಾರ್ಕಿನ ವೇದಾಂತ ಸೊಸೈಟಿಯಲ್ಲಿ ತರಗತಿ ತೆಗೆದುಕೊಳ್ಳುತ್ತಿದ್ದರು. ಅವರ ಮಾತು ಗಳನ್ನೇ ಸ್ತಬ್ಧವಾಗಿ ಆಲಿಸುತ್ತ ಕಿಕ್ಕಿರಿದ ತರಗತಿ ಮೈಮರೆತಿತ್ತು. ಸ್ವಾಮೀಜಿಯವರೂ ತಮ್ಮ ಮಾತುಗಳಲ್ಲೇ ಲೀನರಾಗಿದ್ದರು. ಹೀಗೆ ತರಗತಿಯಲ್ಲಿ ಪ್ರತಿಯೊಬ್ಬರೂ ತನ್ಮಯರಾಗಿ ಕುಳಿತಿ ದ್ದಾರೆ, ಆಗ ಸ್ವಾಮೀಜಿಯವರಿಗೆ ಏನನ್ನಿಸಿತೋ ಏನೋ, ಹಠಾತ್ತಾಗಿ ತಮ್ಮ ಪ್ರವಚನವನ್ನು ನಿಲ್ಲಿಸಿ ಎದ್ದು ನಡೆದುಬಿಟ್ಟರು. ವಿಷಯವೇನೆಂದು ಯಾರಿಗೂ ಅರ್ಥವಾಗಲಿಲ್ಲ. ತರಗತಿ ಮುಗಿದು ಹೋಯಿತು. ಅಲ್ಲಿದ್ದವರೆಲ್ಲ ತೀವ್ರ ನಿರಾಶೆಯಿಂದ ಹಿಂದಿರುಗಬೇಕಾಯಿತು. ಬಳಿಕ ಒಬ್ಬ ಶಿಷ್ಯ ಕೇಳಿದ, “ಸ್ವಾಮೀಜಿ, ಎಲ್ಲರೂ ನಿಮ್ಮ ಮಾತಿನಲ್ಲೇ ಮನಸ್ಸಿಟ್ಟು ಕೇಳುತ್ತಿದ್ದಾಗ ನೀವು ಇದ್ದಕ್ಕಿದ್ದಂತೆ ಅರ್ಧಕ್ಕೆ ನಿಲ್ಲಿಸಿ ಹೊರಟುಬಿಟ್ಟಿರಲ್ಲ, ಏಕೆ?” ಅದಕ್ಕೆ ಸ್ವಾಮೀಜಿ ಉತ್ತರಿಸಿದರು, “ಅಲ್ಲಿ ಕುಳಿತಿದ್ದವರೆಲ್ಲರ ಮನಸ್ಸೂ ನನ್ನ ಕೈಯಲ್ಲಿನ ಒಂದು ಆವೆಮಣ್ಣಿನ ಮುದ್ದೆಯಂತೆ ಆಗಿಬಿಟ್ಟಿತ್ತು. ಇದ್ದಕ್ಕಿದ್ದಂತೆ ಇದು ನನ್ನ ಗಮನಕ್ಕೆ ಬಂದಿತು. ಆ ಸ್ಥಿತಿಯಲ್ಲಿ, ಅವರ ಮನಸ್ಸುಗಳನ್ನು ಯಾವ ಆಕಾರಕ್ಕೆ ಬೇಕಾದರೂ ತರಬಹುದಾದ ಶಕ್ತಿ ನನ್ನಲ್ಲಿ ಆವಿರ್ ಭವಿಸಿತ್ತು. ಆದರೆ ನಾನು ಹಾಗೆ ಮಾಡಲಾರೆ; ಅದು ನನ್ನ ನೀತಿಗೆ ವಿರುದ್ಧ. ಪ್ರತಿಯೊಬ್ಬ ವ್ಯಕ್ತಿಯೂ ಅವನವನ ಮನೋಭಾವಕ್ಕೆ, ಸ್ವಭಾವಕ್ಕೆ ಅನುಗುಣವಾಗಿಯೇ ಬೆಳೆಯಬೇಕು. ಆದ್ದರಿಂದ ನಾನು ನನ್ನ ಮಾತನ್ನು ಅಷ್ಟಕ್ಕೇ ನಿಲ್ಲಿಸಿದೆ.” ಆಗ ಆ ಶಿಷ್ಯ ಆಕ್ಷೇಪಿಸಿದ, “ಆದರೆ ಜನ ಭಾವಿಸುತ್ತಾರೆ–ನೀವು ಹೇಳಬೇಕಾದ್ದನ್ನು ಮರೆತುಬಿಟ್ಟಿರಿ ಎಂದು.” ಅದಕ್ಕೆ ಸ್ವಾಮೀಜಿ ಯವರ ಸ್ವಭಾವಸಹಜ ಉತ್ತರ: “ಜನ ಏನು ತಿಳಿದುಕೊಂಡರೆ ತಾನೆ ಏನು!”

ಈ ಸಂದರ್ಭದಲ್ಲಿ ನಾವು, ಹಿಂದೆಯೇ ಹೇಳಲಾದ ಒಂದು ಘಟನೆಯನ್ನು ಸ್ಮರಿಸಿಕೊಳ್ಳ ಬಹುದು. ಒಮ್ಮೆ ತರುಣ ನರೇಂದ್ರನಿಗೆ ತನ್ನಲ್ಲೊಂದು ಅಪೂರ್ವಶಕ್ತಿ ಆವಿರ್ಭವಿಸಿರುವಂತೆ ಭಾಸವಾಗಿತ್ತು. ಅವನಿಗೆ ಅದನ್ನು ಪರೀಕ್ಷಿಸಬೇಕೆಂಬ ಆಸೆಯಾಗಿ, ತಾನು ಧ್ಯಾನಕ್ಕೆ ಕುಳಿತಾಗ ತನ್ನನ್ನು ಮುಟ್ಟುವಂತೆ ಸ್ನೇಹಿತ ಕಾಳೀಪ್ರಸಾದನಿಗೆ (ಸ್ವಾಮೀ ಅಭೇದಾನಂದರಿಗೆ) ಹೇಳಿದ. ಅದರಂತೆ ನರೇಂದ್ರನನ್ನು ಸ್ಪರ್ಶಿಸಿದಾಗ ಕಾಳೀಪ್ರಸಾದನಿಗೆ ವಿದ್ಯುಚ್ಛಕ್ತಿ ಹರಿದ ಅನುಭವ ವಾಯಿತು. ಆದರೆ ಇದನ್ನೆಲ್ಲ ಕಂಡುಕೊಂಡ ಶ್ರೀರಾಮಕೃಷ್ಣರು ನರೇಂದ್ರನಿಗೆ ಚೆನ್ನಾಗಿ ಛೀಮಾರಿ ಹಾಕಿದರು. ‘ತನ್ನದೇ ಆದ ರೀತಿಯಲ್ಲಿ ಆಧ್ಯಾತ್ಮಿಕವಾಗಿ ಮುಂದುವರಿಯುತ್ತಿದ್ದ ಆತನ ಭಾವ ವನ್ನು ಕೆಡಿಸಿ ನಷ್ಟ ಮಾಡಿದೆಯಲ್ಲ! ಇನ್ನೆಂದೂ ಹಾಗೆ ಮಾಡಬೇಡ’ ಎಂದು ಬುದ್ಧಿ ಹೇಳಿದರು. ಇಂದು ಸ್ವಾಮೀಜಿಯವರು ಪ್ರಯತ್ನಪೂರ್ವಕವಾಗಿ ಅಂತಹ ಘಟನೆಯನ್ನು ತಡೆಗಟ್ಟುವುದನ್ನು ಇಲ್ಲಿ ಕಾಣುತ್ತೇವೆ.

ಸ್ವಾಮೀಜಿಯವರು ಎರಡು ವಾರದ ಅವಧಿಯಲ್ಲಿ, ತಮ್ಮ ವೇದಾಂತ ಸೊಸೈಟಿಯಲ್ಲಿ ಹಲವಾರು ಪ್ರಶ್ನೋತ್ತರ ತರಗತಿಗಳನ್ನು ತೆಗೆದುಕೊಂಡರು, ಸಂಭಾಷಣೆಗಳನ್ನು ನಡೆಸಿದರು, ಅಲ್ಲಿನ ಶಿಷ್ಯರಿಗೆ, ತಮ್ಮ ಆಧ್ಯಾತ್ಮಿಕ ಹಾಗೂ ತಾತ್ವಿಕ ಸಮಸ್ಯೆಗಳನ್ನು ಬಗೆಹರಿಸಿಕೊಳ್ಳಲು ಈ ತರಗತಿಗಳು ತುಂಬ ಉಪಯುಕ್ತವಾಗಿದ್ದುವು. ಆದರೆ ಕೆಲವೊಮ್ಮೆ ಸ್ವಾಮೀಜಿಯವರ ಉತ್ತರ ಗಳು ಕೇಳುಗರ ಮನಸ್ಸನ್ನು ತಬ್ಬಿಬ್ಬುಗೊಳಿಸುತ್ತಿದ್ದುವು; ಅನಿರೀಕ್ಷಿತ ಹೊಡೆತಗಳಂತೆ ಇರುತ್ತಿ ದ್ದುವು.

ಈ ತರಗತಿಗಳಲ್ಲಿ ಯಾರು ಯಾವ ಪ್ರಶ್ನೆಯನ್ನಾದರೂ ಕೇಳಬಹುದಾಗಿತ್ತು. ಒಮ್ಮೆ ತರಗತಿ ಯಲ್ಲಿ ಒಬ್ಬಳು ‘ಚರ್ಚಿನ’ ಮುದುಕಿ ಕೇಳಿದಳು, “ನೀವು ಪಾಪದ ವಿಷಯವಾಗಿ ಹೇಳುವುದೇ ಇಲ್ಲವಲ್ಲ. ಏಕೆ?” ಕ್ರೈಸ್ತ ಸಿದ್ಧಾಂತದಲ್ಲಿ ಪಾಪಕ್ಕೆ ಹಾಗೂ ಪಾಪಪರಿಹಾರಕ್ಕೆ ಅಗ್ರಸ್ಥಾನ. ಹಾಗಿರುವಾಗ ಪಾಪದ ವಿಷಯವನ್ನೇ ಪ್ರಸ್ತಾಪಿಸದೆ ಧರ್ಮವನ್ನು ಬೋಧಿಸಲು ಹೇಗೆ ಸಾಧ್ಯ ಎಂಬುದು ಆ ಮುದುಕಿಯ ಸಮಸ್ಯೆ. ಆದರೆ ಇದನ್ನು ಕೇಳಿ ಸ್ವಾಮೀಜಿಯವರ ಮುಖದಲ್ಲಿ ಅಚ್ಚರಿಯ ಮುದ್ರೆ ಮೂಡಿತು. ಅವರೆಂದರು, “ಮೇಡಂ, ನನ್ನ ಪಾಪಗಳಿಗೆ ಧನ್ಯವಾದಗಳು! ಈ ಪಾಪಗಳ ಮೂಲಕವೇ ಅಲ್ಲವೆ ನಾನು ಪುಣ್ಯ ಮಾಡುವುದನ್ನು ಕಲಿತದ್ದು! ಇಂದು ನಾನೇನಾಗಿರುವೆನೋ ಅದಕ್ಕೆ ನನ್ನ ಪುಣ್ಯಗಳಷ್ಟೇ ಪಾಪಗಳೂ ಕಾರಣ. ನಮ್ಮ ಮನಸ್ಸನ್ನು ಮನುಷ್ಯನ ದೋಷಗಳ ಮೇಲೆಯೇ ಏಕೆ ನೆಲೆಗೊಳಿಸಬೇಕು! ಪರಮ ಪಾಪಿ ಎನ್ನಿಸಿಕೊಂಡವ ನಲ್ಲೂ ಎಷ್ಟೋವೇಳೆ ಒಬ್ಬ ಸಂತನಲ್ಲಿ ಕಾಣಬರದ ಸದ್ಗುಣವೊಂದನ್ನು ಕಂಡಿಲ್ಲವೆ? ಇರುವ ಶಕ್ತಿ ಒಂದೇ. ಅದೇ ಒಳ್ಳೆಯದಾಗಿಯೂ ಕೆಟ್ಟದಾಗಿಯೂ ವ್ಯಕ್ತವಾಗುತ್ತದೆ. ದೇವರು ಮತ್ತು ಸೈತಾನ–ಇವರಿಬ್ಬರೂ ಎರಡು ದಿಕ್ಕುಗಳಲ್ಲಿ ಹರಿಯುವ ಒಂದೇ ನದಿ.”

ಈ ಉತ್ತರವನ್ನು ಕೇಳಿ ಮುದುಕಿ ಬೆಚ್ಚಿಬಿದ್ದಳು. ಆದರೆ ಉಳಿದವರು ಅರ್ಥ ಮಾಡಿಕೊಂಡರು. ಬಳಿಕ ಸ್ವಾಮೀಜಿ ಪ್ರತಿಯೊಬ್ಬ ಮಾನವನ ಅಂತರ್ಯದಲ್ಲಿ ಅಡಗಿರುವ ದೈವತ್ವದ ಕುರಿತಾಗಿ ಮಾತನಾಡಲಾರಂಭಿಸಿದರು. ಆತ್ಮ ಹೇಗೆ ಪರಿಪೂರ್ಣ, ಶಾಶ್ವತ ಹಾಗೂ ಅಮರ ಎಂಬುದನ್ನು ವಿವರಿಸಿದರು. ಈ ಬಗ್ಗೆ ಸ್ವಾಮಿ ಅತುಲಾನಂದರು ಹೇಳುತ್ತಾರೆ:

“ಇಲ್ಲಿ ಒಂದು ಭರವಸೆ ದೊರಕುತ್ತದೆ; ಮನುಷ್ಯ ತನ್ನ ದೈವತ್ವವನ್ನು ಸಾಕ್ಷಾತ್ಕರಿಸಿಕೊಂಡು ದೇವನೇ ಆಗಬಹುದು ಎಂಬ ಒಂದು ಸಮಾಧಾನ, ಒಂದು ಧೈರ್ಯ ಕಾಣುತ್ತದೆ. ಯಾರು ಭಗವಂತನಿಗಾಗಿ ಹುಡುಕುತ್ತಿದ್ದರೂ ಅವರಿಗಿನ್ನೂ ಅವನನ್ನು ಕಾಣಲು ಸಾಧ್ಯವಾಗಿರಲಿಲ್ಲವೊ, ಯಾರು ಅವನ ಬಾಗಿಲನ್ನು ತಟ್ಟಿಯೂ ಅವರಿಗಿನ್ನೂ ಅದು ತೆರೆದಿರಲಿಲ್ಲವೊ ಅಂಥವರ ಪಾಲಿಗೆ ಸ್ವಾಮೀಜಿಯವರ ಬೋಧನೆಗಳು ಅದೆಷ್ಟು ಸಮಾಧಾನ ನೀಡಿದುವೆಂದು ಊಹಿಸಬಲ್ಲಿರಾ? ಅವರ ಪಾಲಿಗೆ ಸ್ವಾಮೀಜಿಯವರು ಉದ್ಧಾರಕನಂತಿದ್ದರು. ಸ್ವಾಮೀಜಿ ಅವರ ಹೃದಯಗುಡಿಗೇ ಬಂದು ಬಾಗಿಲು ತಟ್ಟಿದರು! ಯಾರು ತಮ್ಮ ಹೃದಯದ ಬಾಗಿಲನ್ನು ತೆರೆದು ಸ್ವಾಮೀಜಿಯವರ ದಿವ್ಯ ಸನ್ನಿಧಿಯಿಂದ ಪ್ರಾಪ್ತವಾದ ಆಶೀರ್ವಾದವನ್ನು ತುಂಬಿಕೊಂಡರೋ ಅವರೇ ಧನ್ಯರು.”

ಸ್ವಾಮೀಜಿಯವರಿಗೆ ತಾವು ಈ ಮರ್ತ್ಯಲೋಕದಲ್ಲಿ ಇರಬೇಕಾದ ಅವಧಿಯ ಸಂಬಂಧವಾಗಿ ಆಗಾಗ ಅಪೂರ್ವ ಮುನ್ಸೂಚನೆಗಳು ದೊರೆಯುತ್ತಿದ್ದುವೆಂದು ತೋರುತ್ತದೆ. ಒಂದು ದಿನ ಅವರು ಸ್ವಾಮಿ ಅಭೇದಾನಂದರ ಹತ್ತಿರ ಇದ್ದಕ್ಕಿದ್ದಂತೆ ಹೇಳಿದರು, “ನೋಡು ಸೋದರ, ನನ್ನ ಆಯುಸ್ಸು ಮುಗಿಯುತ್ತ ಬರುತ್ತಿದೆ. ಇನ್ನು ನಾನು ಹೆಚ್ಚೆಂದರೆ ಕೇವಲ ಮೂರು-ನಾಲ್ಕು ವರ್ಷ ಬದುಕಬಹುದು.” ಸ್ವಾಮೀಜಿಯವರಿಗೆ ಆಗಿನ್ನೂ ಮೂವತ್ತಾರು ವರ್ಷ. ಇಂತಹ ಯೌವನ ದಲ್ಲಿರುವಾಗಲೇ ‘ನಾನಿನ್ನು ಹೆಚ್ಚೆಂದರೆ ಮೂರುನಾಲ್ಕು ವರ್ಷ ಇರಬಹುದು’ ಎಂದರೆ ಅಭೇದಾ ನಂದರಿಗೆ ಹೇಗೆನ್ನಿಸಬೇಡ! ಅವರು ಹೇಳಿದರು:

“ನರೇನ್, ದಯವಿಟ್ಟು ನೀನು ಆ ಮಾತನ್ನಾಡಬಾರದು. ನಿನ್ನ ಆರೋಗ್ಯ ಬಹಳ ಬೇಗ ಸುಧಾರಿಸುತ್ತಿದೆ. ನೀನಿನ್ನೂ ಕೆಲಸಮಯ ಇಲ್ಲೇ ಇದ್ದುಬಿಟ್ಟರೆ ನಿನ್ನ ಆರೋಗ್ಯ ಮತ್ತೆ ಹಿಂದಿ ನಂತೆಯೇ ಆಗುತ್ತದೆ, ನೀನು ಹಿಂದಿನಷ್ಟೇ ದೃಢಿಷ್ಠನಾಗುತ್ತೀ. ಅಲ್ಲದೆ, ನಾವಿನ್ನೂ ಎಷ್ಟೊಂದು ಕಾರ್ಯಗಳನ್ನು ಮಾಡಲಿಕ್ಕಿದೆ! ಈಗ ಆಗಿರುವುದೆಲ್ಲ ಕೇವಲ ಪ್ರಾರಂಭ ಅಷ್ಟೇ, ಅಲ್ಲವೆ!”

ಆಗ ಸ್ವಾಮೀಜಿ ಅರ್ಥಪೂರ್ಣವಾಗಿ ಒಂದು ಮಾತು ಹೇಳಿದರು, “ನನ್ನ ವಿಷಯ ನಿನಗೆ ಅರ್ಥವಾಗುತ್ತಿಲ್ಲ ಸೋದರ. ನಾನು ವಿಶಾಲವಾಗುತ್ತಿದ್ದೇನೆ; ಎಷ್ಟು ವಿಶಾಲವಾಗುತ್ತಿದ್ದೇನೆಂದರೆ ಕೆಲವೊಮ್ಮೆ ನನಗನ್ನಿಸುತ್ತದೆ, ಈ ಶರೀರದಲ್ಲಿ ನಾನು ಹಿಡಿಸಲಾರೆ ಎಂದು. ನನ್ನ ಶರೀರ ಇನ್ನೇನು ಬಿರಿಯುವುದರಲ್ಲಿದೆ. ಖಂಡಿತವಾಗಿಯೂ ಹೇಳುತ್ತೇನೆ ಕೇಳು, ಈ ರಕ್ತಮಾಂಸಗಳ ಪಂಜರವು ನನ್ನನ್ನು ಇನ್ನು ಹೆಚ್ಚು ಕಾಲ ಹಿಡಿದಿಡಲಾರದು.”

ಚಳಿಗಾಲ ಸಮೀಪಿಸುತ್ತಿದ್ದುದರಿಂದ ದಿನದಿನಕ್ಕೂ ಚಳಿ ಹೆಚ್ಚಲಾರಂಭಿಸಿತ್ತು. ಸ್ವಾಮೀಜಿ ಈಗಿದ್ದ ಪೂರ್ವ ಕರಾವಳಿ ಪ್ರದೇಶದಲ್ಲಿ ಚಳಿ ತುಂಬ ಹೆಚ್ಚು. ಇದರಿಂದ ಅವರಿಗೆ ಸ್ವಲ್ಪ ಅನನುಕೂಲವೇ ಆಯಿತು. ಆದ್ದರಿಂದ ಅವರು ತಮ್ಮ ವೈದ್ಯಸ್ನೇಹಿತರಾದ ಡಾ ॥ ಎಗ್ಬರ್ಟ್ ಗರ್ನ್​ಸೆಯವರ ಮನೆಗೆ ಬಂದಿಳಿದುಕೊಂಡರು. ಸ್ವಾಮೀಜಿಯವರ ಶರೀರವನ್ನು ಚೆನ್ನಾಗಿ ಪರೀಕ್ಷೆ ಮಾಡಿ ಅವರನ್ನು ಸಂಪೂರ್ಣ ಆರೋಗ್ಯವಂತರನ್ನಾಗಿ ಮಾಡಲು ಡಾ ॥ ಗರ್ನ್​ಸೇ ಉತ್ಸಾಹಿತ ರಾಗಿದ್ದರು. ಇವರಲ್ಲದೆ ಸ್ವಾಮೀಜಿಯವರ ಶುಶ್ರೂಷೆಗಾಗಿ ಡಾ ॥ ಹೆಲ್ಮರ್ ಕೂಡ ಮುಂದಾ ದರು. ಹೀಗೆ ಈ ಇಬ್ಬರು ವಿಖ್ಯಾತ ಡಾಕ್ಟರರೂ ಸೇರಿ ತುಂಬ ಮುತುವರ್ಜಿಯಿಂದ ನೋಡಿ ಕೊಳ್ಳುತ್ತಿದ್ದರೂ ಸ್ವಾಮೀಜಿಯವರಿಗೆ ಶೀತ ಮತ್ತು ಜ್ವರ ಹಿಡಿದುಕೊಂಡಿತು. ಇದಕ್ಕೆ ನ್ಯೂ ಯಾರ್ಕಿನ ಚಳಿಯೇ ಕಾರಣವಾಗಿರಬಹುದು. ಆದರೆ ಬಹುಶಃ ಅದಕ್ಕೆ ಬೇರೊಂದೂ ಕಾರಣವೂ ಇದ್ದಿರಬಹುದು–ಅದೇ ವೇಳೆಗೆ ಅವರಿಗೆ ಸ್ಟರ್ಡಿಯಿಂದ ಮೊದಲನೆಯ ದೂಷಣೆಯ ಪತ್ರ ಬಂದು ಸೇರಿತ್ತು. ಸ್ವಾಮೀಜಿಯವರ ಸೂಕ್ಷ್ಮ ಪ್ರಕೃತಿಗೆ ಅದೊಂದು ದೊಡ್ಡ ಆಘಾತವೇ ಸರಿ. ಅಲ್ಲದೆ ತಮ್ಮ ಶರೀರದ ಸೂಕ್ಷ್ಮತೆಯನ್ನು ಬಹುಶಃ ಅವರೊಬ್ಬರೇ ಚೆನ್ನಾಗಿ ಬಲ್ಲವರು. ಅವರು ನ್ಯೂಯಾರ್ಕಿನಿಂದ ತಮ್ಮ ಶಿಷ್ಯರೊಬ್ಬರಿಗೆ ಬರೆಯುತ್ತಾರೆ, “ಒಟ್ಟಿನಲ್ಲಿ ನನ್ನ ಶರೀರದ ಬಗ್ಗೆ ಅಷ್ಟೆಲ್ಲ ಕಳವಳಗೊಳ್ಳುವ ಅಗತ್ಯವೇನೂ ಇಲ್ಲ ಎಂದು ನನಗನ್ನಿಸುತ್ತದೆ. ಏಕೆಂದರೆ ಸೂಕ್ಷ್ಮ ಪ್ರಕೃತಿಯ ಈ ನನ್ನ ಶರೀರವು ಕೆಲವೊಮ್ಮೆ ಅದ್ಭುತ ಸಂಗೀತವನ್ನು ಹೊಮ್ಮಿಸುವ, ಇನ್ನು ಕೆಲವೊಮ್ಮೆ ಕತ್ತಲಲ್ಲಿ ನಿಡುಸುಯ್ಯುವ ಒಂದು ವಾದ್ಯ ಮಾತ್ರ.”

ಸ್ವಾಮೀಜಿಯವರಿಗೆ ಕೇಂಬ್ರಿಡ್ಜಿನಲ್ಲಿ ಶ್ರೀಮತಿ ಸಾರಾಳ ಮನೆಯಲ್ಲಿ ಚಳಿಗಾಲದ ದಿನಗಳನ್ನು ಕಳೆಯುವ ಅಭಿಪ್ರಾಯವಿತ್ತು. ಆದರೆ ಇದ್ದಕ್ಕಿದ್ದಂತೆ ಆ ಕಾರ್ಯಕ್ರಮ ಬದಲಾಯಿತು. ಅವರು ಅಮೆರಿಕದ ಪಶ್ಚಿಮ ತೀರದ ಕ್ಯಾಲಿಪೋರ್ನಿಯಾ ರಾಜ್ಯಕ್ಕೆ ಹೊರಡುವುದೆಂದು ನಿಶ್ಚಯವಾಯಿತು. ನ್ಯೂಯಾರ್ಕಿನಿಂದ ಎರಡು ಸಾವಿರ ಕಿಲೋಮೀಟರಿಗೂ ಹೆಚ್ಚು ದೂರವಿರುವ ಕ್ಯಾಲಿಫೋರ್ನಿಯಾ ರಾಜ್ಯಕ್ಕೆ ತೆರಳಬೇಕೆಂಬ ಸ್ವಾಮೀಜಿಯವರ ನಿರ್ಧಾರದ ಹಿಂದಿದ್ದ ಮುಖ್ಯ ಕಾರಣವೆಂದರೆ ಅವರ ಆರೋಗ್ಯ ಸುಧಾರಣೆ. ಅವರು ಚಳಿ ಕಡಿಮೆಯಿರುವ ಪಶ್ಚಿಮ ತೀರಕ್ಕೆ ಹೋದರೆ ಒಳ್ಳೆಯದೆಂದು ಡಾಕ್ಟರುಗಳು ಅಭಿಪ್ರಾಯಪಟ್ಟರು. ಆದರೆ ಇದು ಮೇಲ್ನೋಟಕ್ಕೆ ಕಂಡುಬರುವ ಕಾರಣ. ಇದರ ಹಿಂದಿನ ಮುಖ್ಯ ಉದ್ದೇಶವೆಂದರೆ ವೇದಾಂತ ಪ್ರಸಾರ ಮತ್ತು ಹಣ ಸಂಗ್ರಹ ಮಾಡುವುದೇ. ಈ ಆಲೋಚನೆ ಸ್ವಾಮೀಜಿಯವರ ಮನಸ್ಸಿನಲ್ಲಿ ಯಾವಾಗಲೂ ಸುಪ್ತವಾಗಿ ಇದ್ದೇ ಇತ್ತು. ಅಮೆರಿಕದ ಪಶ್ಚಿಮ ತೀರದ ಈ ಸ್ಥಳವನ್ನು ಅವರು ಹಿಂದೆಂದೂ ಸಂದರ್ಶಿಸಿರಲಿಲ್ಲ. ಆದ್ದರಿಂದ ಈಗ ಅಲ್ಲಿಗೆ ಹೋಗುವ ಅವಕಾಶ ಸಿಕ್ಕಿದ್ದರಿಂದ ತಮ್ಮ ಕಾರ್ಯಗಳಿಗೆ ಅಲ್ಲಿ ಎಷ್ಟರಮಟ್ಟಿಗೆ ಪ್ರೋತ್ಸಾಹ-ನೆರವು ದೊರಕೀತೆಂಬುದನ್ನು ನೋಡುವ ಇಚ್ಛೆ ಅವರಿಗಿತ್ತು.

ಇದೇ ಸಮಯಕ್ಕೆ ಸ್ವಾಮೀಜಿಯವರ ಆಪ್ತ ಶಿಷ್ಯೆ ಜೋಸೆಫಿನ್ನಳು ಕ್ಯಾಲಿಫೋರ್ನಿಯಾಕ್ಕೆ ತೆರಳಿದ್ದಳು. ಅಲ್ಲಿ ಮರಣಾಂತಿಕ ಕಾಯಿಲೆಗೆ ಗುರಿಯಾಗಿದ್ದ ತನ್ನ ಸೋದರ ಟೈಲರ್ ಮೆಕ್ ಲಾಡ್​ನ ಆರೈಕೆ ಮಾಡುವುದಕ್ಕಾಗಿ ಅವಳು ಅಲ್ಲಿಗೆ ಹೊರಟಿದ್ದು. ಅವಳು ರಿಡ್ಜ್​ಲಿ ಮ್ಯಾನರಿಂದ ಹೊರಡುವಾಗ ಸ್ವಾಮೀಜಿಯವರನ್ನೂ ಅಲ್ಲಿಗೆ ಬರುವಂತೆ ಆಹ್ವಾನಿಸಿದ್ದಳು. ಆಗ ಅವರು, ಕ್ಯಾಲಿಫೋರ್ನಿಯದಲ್ಲಿ ವೇದಾಂತದ ತರಗತಿಗಳನ್ನು ಏರ್ಪಾಡು ಮಾಡಲು ಸಾಧ್ಯವಾದರೆ ತಾವು ಅಲ್ಲಿಗೆ ಬರುವುದಾಗಿ ತಿಳಿಸಿದ್ದರು. ನವೆಂಬರಿನ ಮೊದಲ ವಾರದಲ್ಲಿ ಟೈಲರ್ ತೀರಿಕೊಂಡ. ಆದರೆ ಇದಾದ ಬಳಿಕವೂ ಜೋಸೆಫಿನ್ ಕ್ಯಾಲಿಫೋರ್ನಿಯದಲ್ಲೇ ಉಳಿದುಕೊಂಡಳಲ್ಲದೆ, ಅಲ್ಲಿಗೆ ಬರುವಂತೆ ಸ್ವಾಮೀಜಿಯವರನ್ನು ಒತ್ತಾಯಿಸಿ ಮತ್ತೆ ಪತ್ರ ಬರೆದಳು.

ಹೀಗೆ ಡಾಕ್ಟರುಗಳ ಸಲಹೆಯೊಂದಿಗೆ ಮೆಕ್​ಲಾಡಳ ಒತ್ತಾಯವೂ ಸೇರಿಕೊಂಡು, ಸ್ವಾಮೀಜಿ ಕ್ಯಾಲಿಫೋರ್ನಿಯಕ್ಕೆ ಹೊರಡಲು ನಿಶ್ಚಯಿಸಿದರು. ಆದರೆ ತಮ್ಮ ಹಳೆಯ ಯೋಜನೆಗಳೆಲ್ಲ ಇದ್ದ ಕ್ಕಿದ್ದಂತೆ ಬದಲಾದದ್ದರ ಬಗ್ಗೆ ಅವರಿಗೇ ಆಶ್ಚರ್ಯ. ಕ್ರಿಸ್ಟೀನಳಿಗೆ ಒಂದು ಪತ್ರದಲ್ಲಿ ಅವರು ಬರೆಯುತ್ತಾರೆ, “ಇರಲಿ, ನಾನು ಯೋಜನೆ ಹಾಕುವುದು, ಅದು ಮುರಿದುಬೀಳುವುದು ಹೀಗೆಯೆ!”

ಆದರೆ ಸ್ವಾಮೀಜಿಯವರ ಈ ನಿರ್ಧಾರ ಅತ್ಯಂತ ಮಹತ್ತರವಾಗಿತ್ತು. ಅವರ ಜೀವನೋ ದ್ದೇಶವು ಪೂರ್ಣವಾಗುವಲ್ಲಿ, ಈ ಪ್ರಯಾಣವು ಎಂತಹ ಪ್ರಮುಖ ಪಾತ್ರವಹಿಸಿತೆಂಬುದನ್ನು ನಾವು ನೋಡಲಿದ್ದೇವೆ. ನ್ಯೂಯಾರ್ಕಿನಿಂದ ಹೊರಡುವಾಗ ಅವರು ತಮ್ಮ ಉದ್ದೇಶಿತ ಕಾರ್ಯ ಗಳ ಬಗ್ಗೆ ಯಾರಿಗೂ ಏನೂ ಹೇಳಲಿಲ್ಲ. ಅಲ್ಲದೆ ಅವರ ಆಗಿನ ಸ್ಥಿತಿ ನೋಡಿದರೆ, ಯಾವುದೇ ಬಗೆಯ ಪರಿಶ್ರಮವನ್ನೂ ಇನ್ನು ಅವರು ತಡೆದುಕೊಳ್ಳಲಾರರು ಎಂಬಂತೆ ತೋರುತ್ತಿತ್ತು. ಆದರೆ ‘ಜಗನ್ಮಾತೆಯ ಕಾರ್ಯ’ಕ್ಕಾಗಿ ಅವರು ಕಟಿಬದ್ಧರಾಗಿ ನಿಂತಾಗ, ಅವರೊಳಗಿನ ಪ್ರಚಂಡ ಶಕ್ತಿ ಹೇಗೆ ಚಿಮ್ಮಿತು, ಅವರು ಮತ್ತೊಮ್ಮೆ ಹೇಗೆ ತ್ರಿವಿಕ್ರಮ ಗಾತ್ರಕ್ಕೆ ಬೆಳೆದು ನಿಂತರು ಎಂಬುದನ್ನು ಕಂಡು ನಾವು ವಿಸ್ಮಿತರಾಗದಿರುವಂತಿಲ್ಲ.

