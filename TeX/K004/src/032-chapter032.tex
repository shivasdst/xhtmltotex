
\chapter{ಬಾಂಬುಗಳ ಎಸೆತ!}

\noindent

ಲಾಸ್ ಏಂಜಲಿಸ್​ನಲ್ಲಿ ಕಳೆದ ಎರಡು ತಿಂಗಳ ಕಾಲದಲ್ಲಿ ಸ್ವಾಮೀಜಿಯವರ ದೇಹಾರೋಗ್ಯವು ಗಮನಾರ್ಹವಾಗಿ ಸುಧಾರಿಸಿತು. ಅಲ್ಲದೆ, ಅಮೆರಿಕದ ಪಶ್ಚಿಮ ಕರಾವಳಿಯ ಪ್ರದೇಶಗಳು ವೇದಾಂತದ ಪ್ರಸಾರಕ್ಕೆ ತುಂಬ ಸೂಕ್ತವಾದುವೆಂದು ಅವರು ಕಂಡುಕೊಂಡಿದ್ದರು. ಇದರಿಂದ ಉತ್ಸಾಹಗೊಂಡು ಅವರು ತಮ್ಮ ಕಾರ್ಯಕ್ಷೇತ್ರವನ್ನು ಮತ್ತಷ್ಟು ವಿಸ್ತರಿಸುವ ಬಗ್ಗೆ ಆಲೋಚಿಸ ಲಾರಂಭಿಸಿದರು. ಈ ಬಗ್ಗೆ ಅವರು ಬಹಳ ದಿನಗಳಿಂದಲೂ ಆಲೋಚಿಸುತ್ತಲೇ ಇದ್ದರಾದರೂ ಇನ್ನೂ ಯಾವ ನಿರ್ಧಾರಕ್ಕೂ ಬರಲಾಗಿರಲಿಲ್ಲ. ಈ ವೇಳೆಗೆ ಉತ್ತರ ಕ್ಯಾಲಿಫೋರ್ನಿಯದ ಓಕ್ ಲ್ಯಾಂಡಿನಲ್ಲಿ ನಡೆಯಲಿದ್ದ ‘ಕಾಂಗ್ರೆಸ್ ಆಫ್ ರಿಲಿಜನ್ಸ್​’ ಎಂಬ ಸಮ್ಮೇಳನದಲ್ಲಿ ಮಾತನಾಡುವಂತೆ ರೆವರೆಂಡ್ ಬೆಂಜಮಿನ್ ಫೆ ಮಿಲ್ಸ್ ಎಂಬವರಿಂದ ಒಂದು ಆಹ್ವಾನ ಬಂದಿತು. ಈ ಸಮ್ಮೇಳನವು ಪ್ರತಿ ಭಾನುವಾರ ಹಲವಾರು ವಾರಗಳವರೆಗೆ ನಡೆಯಲಿತ್ತು. ಸ್ವಾಮೀಜಿ ಇದರಲ್ಲಿ ಭಾಗವಹಿಸಲು ನಿರ್ಧರಿಸಿದರು. ಅಲ್ಲದೆ ಸ್ಯಾನ್​ಫ್ರಾನ್ಸಿಸ್ಕೋದಲ್ಲಿ ಸ್ವತಂತ್ರವಾಗಿಯೂ ತಮ್ಮ ಕಾರ್ಯವನ್ನು ಮುಂದುವರಿಸುವುದೆಂದು ನಿಶ್ಚಯಿಸಿದರು. ಅವರ ವಸತಿಗೆ ಸಿದ್ಧತೆಗಳನ್ನು ಮಾಡಲು ಶ್ರೀಮತಿ ಹ್ಯಾನ್ಸ್​ಬ್ರೋ ಮುಂದಾಗಿಯೇ ಹೊರಟಳು. ಅಲ್ಲಿ ಅವರು ಮಾಡಲಿದ್ದ ಮೊದಲ ಭಾಷಣಕ್ಕೆ ಬೇಕಾದ ಸಿದ್ಧತೆಗಳನ್ನೆಲ್ಲ ಮಾಡಿ ಮುಗಿಸಿದಳು.

ಸ್ವಾಮೀಜಿಯವರು ಫೆಬ್ರುವರಿ ೨೦ರಂದು ಪಸಾಡೆನದಿಂದ ಹೊರಟು ೪೦೦ ಮೈಲಿ ಉತ್ತರ ಕ್ಕಿರುವ ಸ್ಯಾನ್​ಫ್ರಾನ್ಸಿಸ್ಕೋ ಮಹಾನಗರಕ್ಕೆ ಬಂದು ತಲುಪಿದರು. ೨೩ರಂದು ಅವರು ಅಲ್ಲಿನ ‘ಗೋಲ್ಡನ್ ಗೇಟ್ ಹಾಲ್​’ನಲ್ಲಿ “ವಿಶ್ವಧರ್ಮವೊಂದರ ಆದರ್ಶ” ಎಂಬ ವಿಷಯವಾಗಿ ಮಾತನಾಡಿದರು. ಇದಾದ ಎರಡು ದಿನಗಳ ಬಳಿಕ ಸಮೀಪದ ಓಕ್​ಲ್ಯಾಂಡ್ ನಗರದಲ್ಲಿ ‘ಕಾಂಗ್ರೆಸ್ ಆಫ್ ರಿಲಿಜನ್ಸ್​’ನ ಅಂಗವಾಗಿ ಅವರ ಉಪನ್ಯಾಸ ಏರ್ಪಾಡಾಗಿತ್ತು. ಓಕ್​ಲ್ಯಾಂಡ್ ಇರುವುದು ಸ್ಯಾನ್​ಫ್ರಾನ್ಸಿಸ್ಕೋದಿಂದ ಸ್ವಲ್ಪ ದೂರದಲ್ಲಿ. ಇಲ್ಲಿಗೆ ದೋಣಿಯ ಮೂಲಕ ಕೊಲ್ಲಿಯನ್ನು ದಾಟಿ ಹೋಗಬೇಕು. ಸಮ್ಮೇಳನದಲ್ಲಿ ಸ್ವಾಮೀಜಿ ಮಾತನಾಡಿದ ವಿಷಯ ‘ಆಧುನಿಕ ಪ್ರಪಂಚದಲ್ಲಿ ಹಿಂದೂಧರ್ಮದ ವಾರಸುದಾರತ್ವ’. ಸ್ವಾಮೀಜಿಯವರ ಹೆಸರು ಆಗಲೇ ಆ ಪ್ರದೇಶದಲ್ಲೆಲ್ಲ ಪ್ರಸಿದ್ಧವಾಗಿತ್ತು. ಅಲ್ಲದೆ ಈ ಕಾರ್ಯಕ್ರಮಕ್ಕೆ ಸಾಕಷ್ಟು ಪ್ರಚಾರವೂ ದೊರಕಿದ್ದರಿಂದ ಸಭಾಂಗಣದಲ್ಲಿ ಹಾಗೂ ಅದರ ಪಕ್ಕದ ಹಜಾರದಲ್ಲಿ ಜನ ಕಿಕ್ಕಿರಿದಿದ್ದರು. ಅಲ್ಲದೆ ಸ್ಥಳಾವಕಾಶ ಸಾಲದೆ ಇನ್ನೂ ನೂರಾರು ಜನರು ನಿರಾಶೆಗೊಳ್ಳುವಂತಾಯಿತು. ಈ ಉಪನ್ಯಾಸ ಅತ್ಯಂತ ಯಶಸ್ವಿಯಾಗಿ, ವಿವೇಕಾನಂದರ ಹೆಸರು ಮನೆಮಾತಾಯಿತು. ಅಲ್ಲದೆ ಈ ಮೂಲಕ ಅವರಿಗೆ ಪ್ರಮುಖ ಕ್ರೈಸ್ತಧರ್ಮಾಧಿಕಾರಿಗಳ ಪರಿಚಯವುಂಟಾಯಿತು. ಸ್ವಾಮೀಜಿಯವರಿಂದ ಆಳವಾಗಿ ಪ್ರಭಾವಿತರಾದ ಫೆ ಮಿಲ್ಸರು ಶ್ರೀಮತಿ ಹ್ಯಾನ್ಸ್​ಬ್ರೋಗೆ ಹೇಳುತ್ತಾರೆ, “ಈ ವ್ಯಕ್ತಿ ನನ್ನ ಜೀವನದ ಗತಿಯನ್ನೇ ಬದಲಾಯಿಸಿದರು” ಎಂದು.

ಫೆಬ್ರವರಿ ೨೮ರಂದು ಇದೇ ಸ್ಥಳದಲ್ಲಿ ಮತ್ತೆ ಉಪನ್ಯಾಸ ನೀಡುವಂತೆ ಸ್ವಾಮೀಜಿಯವರಿಗೆ ಆಹ್ವಾನ ಬಂದಿತು. ಮತ್ತೆ ಜನ ಕಿಕ್ಕಿರಿದು, ಕೇಳಿ ಆನಂದಿಸಿದರು. ಸ್ವಾಮೀಜಿಯವರ ಆರೋಗ್ಯ ಎಷ್ಟು ಸುಧಾರಿಸಿತ್ತೆಂದರೆ ಅವರೀಗ ಸುಮಾರು ಮೂರು ಸಾವಿರ ಜನರಿಗೆ ಕೇಳುವಷ್ಟು ಗಟ್ಟಿಯಾಗಿ ಮಾತನಾಡಲು ಶಕ್ತರಾಗಿದ್ದರು.

ಸ್ವಾಮೀಜಿಯವರ ಮಾತುಗಳನ್ನು ಕೇಳಿದ ಸಾವಿರಾರು ಅದೃಷ್ಟವಂತರ ಪೈಕಿ ಕೆಲವು ನೂರು ಮಂದಿಗಾದರೂ ಅರಿವಾಗಿರಬೇಕು–ತಾವು ಕೇಳಿದುದು ಕೇವಲ ಒಬ್ಬ ಸಾಧಾರಣ ಉಪನ್ಯಾಸಕನ ಮಾತುಗಳನ್ನಲ್ಲ ಎಂದು. ಇಂತಹ ಒಬ್ಬ ವ್ಯಕ್ತಿ ಥಾಮಸ್ ಅಲನ್. ಈತನೊಬ್ಬ ಯುವಕ. ಇವನು ಸ್ವಾಮೀಜಿಯವರ ಮಾತುಗಳನ್ನು ಕೇಳಿದುದು ಅದೇ ಮೊದಲ ಸಲ. ಅಲ್ಲಿನ ಯೂನಿಟೇರಿಯನ್ ಚರ್ಚಿನಲ್ಲಿ ನೀಡಿದ ‘ವೇದಾಂತ ಹಾಗೂ ಕ್ರೈಸ್ತ ಧರ್ಮ’ ಎಂಬ ಉಪನ್ಯಾಸವನ್ನು ಕೇಳಿದ ಈತ ಅಂದಿನ ತನ್ನ ಅನುಭವವನ್ನು ಹೀಗೆ ವರ್ಣಿಸುತ್ತಾನೆ:

“ಅವರೊಬ್ಬ ಮಹಾ ವಾಗ್ಮಿ ಹಾಗೂ ಅದ್ಭುತ ವ್ಯಕ್ತಿ ಎಂದು ಕೇಳಿದ್ದ ನಾನು ಅಂದು ವಿಶೇಷ ಉತ್ಸಾಹ-ಕುತೂಹಲಗಳಿಂದ ಉಪನ್ಯಾಸಕ್ಕೆ ಹೋಗಿದ್ದೆ. ಆಗ ನನಗನ್ನಿಸಿದ್ದು ಇದು–‘ತಾವು ಮಾತನಾಡುತ್ತಿರುವ ವಿಷಯವನ್ನು ನಿಜವಾಗಿಯೂ ಅರಿತಿರುವ ವ್ಯಕ್ತಿ ಇವರು. ಯಾರೋ ಹೇಳಿ ದ್ದನ್ನು ಇವರು ಪುನರುಚ್ಚರಿಸುತ್ತಿಲ್ಲ. ಅಥವಾ ತಮ್ಮ ಮನಸ್ಸಿಗೆ ‘ಅನ್ನಿಸಿ’ದ್ದನ್ನೂ ಹೇಳುತ್ತಿಲ್ಲ. ತಮಗೆ ಏನು ‘ಅರಿವಾಗಿದೆ’ಯೋ ಅದನ್ನು ಹೇಳುತ್ತಿದ್ದಾರೆ’ ಎಂದು. ಅಂದು ಮನೆಗೆ ಹೋಗುವಾಗ ನಾನು ಗಾಳಿಯಲ್ಲಿ ತೇಲಿಕೊಂಡೇ ಹೋಗುತ್ತಿದ್ದೆ! ಮನೆಗೆ ಹೋದ ಮೇಲೂ ನಾನೊಬ್ಬ ಮರುಳನಂತೆಯೇ ವರ್ತಿಸುತ್ತಿದ್ದೆ. ‘ಎಂತಹ ಮನುಷ್ಯ ಅವರು?’ ಎಂದು ಯಾರೋ ಕೇಳಿದಾಗ ನಾನೆಂದೆ, ‘ಅವರೊಬ್ಬ ಮನುಷ್ಯ ಅಲ್ಲ, ದೇವರು!’ ಎಂದು. ನನ್ನ ಪಾಲಿಗೆ ನಿಜಕ್ಕೂ ಅವ ರೊಂದು ಅದ್ಭುತವೇ ಸರಿ. ಆ ಉಪನ್ಯಾಸವಾದ ನಂತರ, ಅವರು ಆ ಕರಾವಳಿ ಪ್ರದೇಶದಲ್ಲಿ ಉಪನ್ಯಾಸ ಕೊಟ್ಟಲ್ಲೆಲ್ಲ ನಾನವರನ್ನು ಹಿಂಬಾಲಿಸಿದೆ.”

ಥಾಮಸ್ ಅಲನ್ ಆ ಕ್ಷಣದಿಂದಲೇ ಮಾನಸಿಕವಾಗಿ ಸ್ವಾಮೀಜಿಯವರ ಅನುಯಾಯಿಯಾಗಿ ಬಿಟ್ಟ. ಅಲ್ಲದೆ, ವೇದಾಂತ ಧರ್ಮಕ್ಕೆ ತನ್ನನ್ನು ಸಮರ್ಪಿಸಿಕೊಂಡ. ಮುಂದೆ ೧೯೫೩ರವರೆಗೂ ಈತ ಸ್ಯಾನ್​ಫ್ರಾನ್ಸಿಸ್ಕೋದ ವೇದಾಂತ ಸೊಸೈಟಿಯ ಅಧ್ಯಕ್ಷನಾಗಿದ್ದ.

ಓಕ್​ಲ್ಯಾಂಡಿನ ನಾಗರಿಕರ ಬೇಡಿಕೆಯ ಮೇರೆಗೆ ಸ್ವಾಮೀಜಿಯವರು ಮೂರು ಉಪನ್ಯಾಸಗಳ ಎರಡು ಸರಣಿಗಳನ್ನು ನೀಡಿದರು. ಈ ಉಪನ್ಯಾಸಗಳೆಲ್ಲಕ್ಕೂ ಪ್ರವೇಶಶುಲ್ಕವಿದ್ದಿತಾದರೂ ಇವು ಹೆಚ್ಚಿನ ಜನಸಂದಣಿಯನ್ನು ಆಕರ್ಷಿಸಿದುವು. ಮೊದಲ ಸರಣಿಯ ವಿಷಯಗಳು–‘ಜನನ ಮರಣಗಳ ನಿಯಮಗಳು’, ‘ವಾಸ್ತವಿಕತೆ ಮತ್ತು ಅದರ ನೆರಳು ’ (‘ಪರಬ್ರಹ್ಮ ಮತ್ತು ಮಾಯೆ’) ಹಾಗೂ ‘ಮುಕ್ತಿಯ ದಾರಿ’. ಇವುಗಳಲ್ಲಿ ಮೊದಲನೆಯದು ಅಲ್ಲಿನ ಸಂಪ್ರದಾಯಸ್ಥ ಜನರಲ್ಲಿ ಒಂದು ಕೋಲಾಹಲವನ್ನೇ ಉಂಟುಮಾಡಿತು. ಇದರಲ್ಲಿ ಸ್ವಾಮೀಜಿಯು ಘೋಷಿಸಿದರು– “ಸ್ವರ್ಗಕ್ಕೆ ಹೋಗುವುದಕ್ಕೆ ಉಪಾಯವೇನು ಎಂಬುವುದಲ್ಲ, ಬದಲಾಗಿ, ಸ್ವರ್ಗಕ್ಕೆ ಮತ್ತೆಮತ್ತೆ ಹೋಗುವುದನ್ನು ತಡೆಯುವುದು ಹೇಗೆ–ಹಿಂದೂಗಳು ಕಂಡುಕೊಳ್ಳಬಯಸುವುದು ಇದನ್ನು.” ಕ್ರೈಸ್ತ ಧರ್ಮಕ್ಕೆ ಇದೊಂದು ಕ್ರಾಂತಿಕಾರೀ ಕಲ್ಪನೆ. ಏಕೆಂದರೆ ಸ್ವರ್ಗ ಗಳಿಸುವುದೇ ಸಾಮಾನ್ಯ ಕ್ರೈಸ್ತರ ಉಚ್ಚತಮ ಆದರ್ಶ. ಆದ್ದರಿಂದ ಈ ಭಾಷಣವು ಅಲ್ಲೊಂದು ಹೊಸ ವಿಚಾರಲಹರಿಯನ್ನೇ ಎಬ್ಬಿಸಿತು.

ಎರಡನೆಯ ಉಪನ್ಯಾಸಮಾಲಿಕೆಯು ಭಾರತದ ಸಂಸ್ಕೃತಿ ಸಂಪ್ರದಾಯ ಹಾಗೂ ಆದರ್ಶಗಳನ್ನು ಕುರಿತದ್ದು. ಅವುಗಳ ವಿಷಯಗಳು ಕ್ರಮವಾಗಿ ‘ಭಾರತದ ಸಂಪ್ರದಾಯಗಳು ಹಾಗೂ ರೀತಿನೀತಿಗಳು’, ‘ಭಾರತದ ವಿಜ್ಞಾನ ಮತ್ತು ಕಲೆ’ ಮತ್ತು ‘ಭಾರತದ ಆದರ್ಶಗಳು’. ಈ ಎರಡನೆಯ ಸರಣಿಯೂ ಮೊದಲದರಂತೆಯೇ ಜನರಿಗೆ ಹೊಸ ಭಾವನೆಗಳ ರಸದೌತಣವನ್ನುಣಿಸಿತು. ಭಾರತದ ಬಗ್ಗೆ ಪ್ರಚಲಿತವಾಗಿದ್ದ ವಿಕೃತ ಭಾವನೆಗಳನ್ನು, ನಂಬಿಕೆಗಳನ್ನು ಬುಡಮೇಲು ಮಾಡಿ, ಭವ್ಯ ಸಂಸ್ಕೃತಿ ಪರಂಪರೆಯ ಹಾಗೂ ಸಂಪ್ರದಾಯಗಳ ಸತ್ಯಚಿತ್ರಣವನ್ನು ಸ್ವಾಮೀಜಿಯವರು ಮುಂದಿಟ್ಟರು. ಉನ್ನತ ನೀತಿಗಳನ್ನೂ ಆಧ್ಯಾತ್ಮಿಕತೆಯನ್ನೂ ಎತ್ತಿಹಿಡಿಯುವ ಭಾರತೀಯ ಇತಿಹಾಸ-ಪುರಾಣಗಳ ಹಲವಾರು ಅದ್ಭುತ ಕಥೆಗಳನ್ನು ಜನರ ಮನಮುಟ್ಟುವಂತೆ ವರ್ಣಿಸಿದರು. ಕಥೆ ಹೇಳುವುದರಲ್ಲಿ ಸ್ವಾಮೀಜಿಯವರದು ಎತ್ತಿದ ಕೈ. ಅವರು ಹೇಳಿದ ಕಥೆಗಳನ್ನು ಕೇಳಿ ಆಸ್ವಾದಿಸಿದವರು, ಪರಿವರ್ತನೆ ಹೊಂದಿದವರು ಸಾವಿರಾರು ಜನ.

ಹೀಗೆ ಸ್ವಾಮೀಜಿ ಓಕ್​ಲ್ಯಾಂಡಿನಲ್ಲಿ ಹಲವಾರು ಭಾಷಣಗಳನ್ನು ಮಾಡಿದರಾದರೂ ಅವರು ನೆಲಸಿದ್ದುದು ಸ್ಯಾನ್ ಫ್ರಾನ್ಸಿಸ್ಕೋ ನಗರದಲ್ಲೇ. ಇಲ್ಲಿಂದ ಅವರು ಓಕ್​ಲ್ಯಾಂಡಿಗೆ ಪ್ರತಿಸಲವೂ ದೋಣಿಯ ಮುಖಾಂತರ ಓಡಾಡುತ್ತಿದ್ದರು. ಓಕ್​ಲ್ಯಾಂಡಿನಲ್ಲಿ ಉಪನ್ಯಾಸಗಳನ್ನು ನೀಡುವುದರೊಂದಿಗೆ, ಇತ್ತ ಸ್ಯಾನ್ ಫ್ರಾನ್ಸಿಸ್ಕೋದಲ್ಲೂ ಅವರು ತಮ್ಮ ಕಾರ್ಯವನ್ನು ನಿರಂತರವಾಗಿ ಹಾಗೂ ಯಶಸ್ವಿಯಾಗಿ ಮಾಡಿಕೊಂಡು ಬರುತ್ತಿದ್ದರು. ಈ ಮಧ್ಯೆ ಅವರು, ತಾವು ಮೊದಲು ಇಳಿದುಕೊಂಡಿದ್ದ ಸ್ನೇಹಿತರೊಬ್ಬರ ಮನೆಯನ್ನು ಬಿಟ್ಟು, ಟರ್ಕ್ ಸ್ಟ್ರೀಟ್ ಎಂಬಲ್ಲಿ ಮತ್ತೊಂದು ವಿಶಾಲವಾದ, ಸ್ವತಂತ್ರವಾದ ಮನೆಗೆ ಹೋಗಿ ನೆಲಸಿದರು. ಇಲ್ಲಿ ಅವರಿಗೆ ತಮ್ಮ ಇಚ್ಛಾನುಸಾರ ದರ್ಶನಾರ್ಥಿಗಳನ್ನು ಬರಮಾಡಿಕೊಳ್ಳಲು, ತರಗತಿಗಳನ್ನು ನಡೆಸಲು ಸಾಕಷ್ಟು ಸ್ಥಳಾವಕಾಶವಿತ್ತು. ಈ ಮನೆಯನ್ನು ಅವರಿಗಾಗಿ ಹುಡುಕಿಕೊಟ್ಟು ನೆರವಾದ ಶ್ರೀಮತಿ ಹ್ಯಾನ್ಸ್​ಬ್ರೋಳೂ ಶ್ರೀಮತಿ ಎಮಿಲಿ ಆ್ಯಸ್ಪಿನಾಲ್ ಎಂಬೊಬ್ಬ ಭಕ್ತೆಯೂ ಈ ಮನೆಗೇ ಬಂದಿದ್ದು ಅವರಿಗೆ ಹಲವಾರು ರೀತಿಯಲ್ಲಿ ನೆರವಾದರು. ಶ್ರೀಮತಿ ಹ್ಯಾನ್ಸ್​ಬ್ರೋಳಂತೂ ಅವರ ಕಾರ್ಯದರ್ಶಿ, ಮೇಲ್ವಿಚಾರಕಿ, ಅಡಿಗೆಯವಳು, ಗುಮಾಸ್ತೆ, ಪತ್ರಿಕಾ ಕಾರ್ಯದರ್ಶಿ ಎಲ್ಲವೂ ಆಗಿಬಿಟ್ಟಳು. ಇವಳ ಸೇವಾ ನಿಷ್ಠೆಯನ್ನು ಸ್ವಾಮೀಜಿ ಹೃತ್ಪೂರ್ವಕವಾಗಿ ಮೆಚ್ಚಿದರು. ನಿಜಕ್ಕೂ ಅವರು ಪಶ್ಚಿಮ ದೇಶಗಳಲ್ಲಿ ಎಲ್ಲೆಲ್ಲಿ ಹೋದರೂ ಅವರಿಗೆ ನೆರವಾಗಲು ಹೃದಯವಂತ, ನಿಷ್ಠಾವಂತ ವ್ಯಕ್ತಿಗಳು ತಾವಾಗಿಯೇ ಒದಗಿಬರುತ್ತಿದ್ದರು. ಅವರೇ ಒಮ್ಮೆ ಶ್ರೀಮತಿ ಹಾನ್ಸ್​ಬ್ರೋಗೆ ಹೇಳುತ್ತಾರೆ, “ಜಗನ್ಮಾತೆ ನನ್ನನ್ನೊಂದು ವಿಚಿತ್ರ ಜಗತ್ತಿನಲ್ಲಿ ಎತ್ತಿಹಾಕಿದ್ದಾಳೆ. ಅಲ್ಲಿಯವರನ್ನು ನಾನು ಅರ್ಥಮಾಡಿಕೊಳ್ಳ ಲಾರೆ–ನನ್ನನ್ನು ಅವರು ಅರ್ಥಮಾಡಿಕೊಳ್ಳಲಾರರು, ಎಂಬಂತಿದೆ. ಆದರೆ ನಾನು ಇಲ್ಲಿ ಹೆಚ್ಚೆಚ್ಚು ಕಾಲ ಇದ್ದಂತೆ ನನಗೊಂದು ಭಾವನೆ ಬರುತ್ತಿದೆ–ನಾನು ಸಂಧಿಸಿದ ಈ ಜನಗಳೆಲ್ಲ ನನಗೆ ಸೇರಿದವರೇ, ಮತ್ತು ನನ್ನ ಪಾಲಿನ ಕರ್ತವ್ಯದಲ್ಲಿ ಪಾಲ್ಗೊಳ್ಳಲೆಂದೇ ಅವರು ಇಲ್ಲಿಗೆ ಬಂದಿದ್ದಾರೆ, ಎಂದು.” ತಮಗೆ ತನ್ನಿಂದಾದ ಎಲ್ಲ ರೀತಿಯಲ್ಲೂ ನೆರವಾದ ಎಮಿಲಿ ಆ್ಯಸ್ಪಿನಾಲಳಿಗೆ ಅವ ರೊಮ್ಮೆ ಹೇಳುತ್ತಾರೆ, “ನೀನು ಒಂದು ಎತ್ತರದ ಪರ್ವತ ಶಿಖರದ ಮೇಲೆಯೇ ಜೀವಿಸಿದ್ದರೂ, ನನ್ನನ್ನು ನೋಡಿಕೊಳ್ಳುವುದಕ್ಕಾಗಿ ಕೆಳಗಿಳಿದು ಬರಲೇಬೇಕಾಗಿತ್ತು!”

ಟರ್ಕ್​ಸ್ಟ್ರೀಟಿನ ಹೊಸ ಮನೆಗೆ ಬಂದ ಮೇಲೆ ಸ್ವಾಮೀಜಿಯವರು ಸ್ಯಾನ್​ಫ್ರಾನ್ಸಿಸ್ಕೋದಲ್ಲಿ ಇನ್ನೊಂದು ತಿಂಗಳಾದರೂ ಉಳಿದುಕೊಳ್ಳುವುದೆಂದು ನಿಶ್ಚಯಿಸಿದರು. ಮತ್ತು ತಮ್ಮೆಲ್ಲ ಕಾರ್ಯಕ್ರಮಗಳನ್ನೂ ಪುನರ್​ರೂಪಿಸಿಕೊಂಡರು. ತಮ್ಮ ಆತ್ಮೀಯ ಶಿಷ್ಯರಿಗಾಗಿ ದಿನಕ್ಕೆರಡ ರಂತೆ ತರಗತಿಗಳನ್ನು ಪ್ರಾರಂಭಿಸಿದರು. ಆಧ್ಯಾತ್ಮಿಕ ಜೀವನದಲ್ಲಿ ನಿಜವಾದ ಆಸಕ್ತಿಯಿದ್ದವರಿಗೆ ಧ್ಯಾನದ ವಿಧಾನವನ್ನು ಬೋಧಿಸಿದರು. ಅಲ್ಲಿಯವರೆಗೂ ಅವರು ವೈವಿಧ್ಯಪೂರ್ಣ ವಿಷಯಗಳ ಮೇಲೆ ಉಪನ್ಯಾಸಗಳನ್ನು ಮಾಡುತ್ತಿದ್ದರು. ಆದರೆ ಇನ್ನು ಮುಂದೆ ಮಾಡಿದ ಉಪನ್ಯಾಸಗಳಲ್ಲಿ, ಮಾನವನಲ್ಲಿ ಸುಪ್ತವಾಗಿರುವ ದೈವತ್ವ ಹಾಗೂ ಅದನ್ನು ಸಾಕ್ಷಾತ್ಕರಿಸಿಕೊಳ್ಳುವ ಬಗೆ–ಇವುಗಳಿಗೆ ವಿಶೇಷ ಪ್ರಾಮುಖ್ಯ ನೀಡಿದರು. ಪ್ರಪಂಚದೆಲ್ಲೆಡೆಯಲ್ಲಿ ಆಗತಾನೆ ಉದಿಸುತ್ತಿದ್ದ, ಇಪ್ಪತ್ತನೆಯ ಶತಮಾನದ ಆಧುನಿಕ ಅಧ್ಯಾಯದ ಮೇಲೆ ಅದ್ವೈತ ವೇದಾಂತದ ಮಹಾತತ್ತ್ವಗಳನ್ನು ಅಚ್ಚೊತ್ತಿ ಬಿಡಲು ಅವರು ನಿರ್ಧರಿಸಿದ್ದಂತಿತ್ತು.

ಟರ್ಕ್ ಸ್ಟ್ರೀಟಿನ ಹೊಸ ಮನೆಯಲ್ಲಿ ತರಗತಿಗಳನ್ನು ನಡೆಸುತ್ತಿದ್ದ ಅವಧಿಯಲ್ಲೇ ಸ್ವಾಮೀಜಿ ಆ ಊರಿನ ‘ರೆಡ್ ಮೆನ್ಸ್ ಬಿಲ್ಡಿಂಗ್​’ ಎಂಬ ಸಭಾಂಗಣದಲ್ಲಿ ಮೂರು ಭಾಷಣಗಳನ್ನೂ ‘ಯೂನಿಯನ್ ಸ್ಕ್ವೇರ್ ಹಾಲ್​’ ಎಂಬಲ್ಲಿ ಭಾನುವಾರಗಳಂದು ಐದು ಭಾಷಣಗಳ ಸರಣಿಯನ್ನೂ ನಡೆಸಿದರು. ಈ ಸರಣಿಯ ವಿಷಯಗಳೆಂದರೆ, ‘ಜಗತ್ತಿಗೆ ಕ್ರಿಸ್ತನ ಸಂದೇಶ’, ‘ಜಗತ್ತಿಗೆ ಬುದ್ಧನ ಸಂದೇಶ’, ‘ಮಹಮದ್ ಪೈಗಂಬರ್​’, ‘ಕೃಷ್ಣ ಹಾಗೂ ಅವನ ಸಂದೇಶ’, ಮತ್ತು ‘ಭವಿಷ್ಯದ ಧರ್ಮವು ವೇದಾಂತವೆ?’ ಎಂಬವು. ಇವುಗಳಲ್ಲಿ ಕಡೆಯದಾದ ‘ಭವಿಷ್ಯದ ಧರ್ಮವು ವೇದಾಂತವೆ?’ ಎಂಬುದು ಆ ಉಪನ್ಯಾಸಗಳಲ್ಲೆಲ್ಲ ತುಂಬ ಪರಿಣಾಮಕಾರಿಯಾಗಿತ್ತು, ಭಾವಪ್ರಚೋದಕವಾಗಿತ್ತು. ಅವರು ಅಮೆರಿಕದ ಪಶ್ಚಿಮ ಕರಾವಳಿಯಲ್ಲಿ ಸಾರಿದ ನವಸಂದೇಶದ ಸಾರವೇ ಅದರಲ್ಲಿ ಅಡಕವಾಗಿತ್ತೆಂದು ಹೇಳಬಹುದು. ಅವರು ಅಮೆರಿಕದಲ್ಲಿ ಅದ್ವೈತ ವೇದಾಂತವನ್ನು ವಿಶೇಷವಾಗಿ ಬೋಧಿಸುತ್ತಿದ್ದರೆಂಬುದನ್ನು ನಾವು ಈಗಾಗಲೇ ನೋಡಿದ್ದೇವೆ. ವೇದಾಂತವು ಆಧುನಿಕ ಜಗತ್ತಿನ ಸಮಸ್ಯೆಗಳಿಗೆಲ್ಲ ಉತ್ತರವೆಂಬುದೇನೋ ನಿಜವೇ ಆಗಿದ್ದರೂ ಅದು ಅಷ್ಟೇನೂ ಸುಲಭದ ಉತ್ತರವಲ್ಲ ಎಂಬುದನ್ನೂ ಅವರೇ ಹೇಳಿದರು. ಈ ಮಹತ್ವಪೂರ್ಣ ಉಪ ನ್ಯಾಸದ ಕೆಲವು ವಾಕ್ಯಗಳು ಹೀಗಿವೆ:

“ಈ ದೇಶ(ಅಮೆರಿಕ)ದಲ್ಲಿ ರಾಜತ್ವವೆಂಬುದು ನಿಮ್ಮಲ್ಲಿ ಪ್ರತಿಯೊಬ್ಬರನ್ನೂ ಪ್ರವೇಶಿಸಿ ಬಿಟ್ಟಿದೆ. ಈ ದೇಶದಲ್ಲಿ ನೀವೆಲ್ಲರೂ ರಾಜರೇ. (ಪ್ರಜಾಪ್ರಭುತ್ವ ನೀತಿಯ ದೃಷ್ಟಿಯಿಂದಲೂ ಇದು ನಿಜ.) ಅಂತೆಯೇ ವೇದಾಂತಧರ್ಮದ ಪ್ರಕಾರವೂ ಕೂಡ. ಅದರ ಪ್ರಕಾರ ನೀವೆಲ್ಲರೂ ದೇವರೇ! ಒಬ್ಬನೇ ದೇವರು ಸಾಲದು. ಆದ್ದರಿಂದ ನೀವೆಲ್ಲರೂ ದೇವರೇ ಎನ್ನುತ್ತದೆ ವೇದಾಂತ.

“ಇದರಿಂದಾಗಿಯೇ ವೇದಾಂತವನ್ನು ಅರ್ಥಮಾಡಿಕೊಳ್ಳುವುದು ಬಹಳ ಕಷ್ಟ. ಅದು ದೇವ ರನ್ನು ಕುರಿತ ಹಳೆಯ ಕಲ್ಪನೆಗಳಾವುದನ್ನೂ ಬೋಧಿಸುವುದೇ ಇಲ್ಲ. ಮೇಲೆ ಆಕಾಶದಲ್ಲಿ ಮೋಡ ಗಳಾಚೆ ಎಲ್ಲೋ ಕುಳಿತುಕೊಂಡು, ನಮ್ಮ ಅನುಮತಿಯನ್ನು ಕೇಳದೆ ಈ ಜಗತ್ತಿನ ವ್ಯವಹಾರ ಗಳನ್ನೆಲ್ಲ ನೋಡಿಕೊಳ್ಳುತ್ತಿರುವ, ಮತ್ತು ಶೂನ್ಯದಿಂದ ನಮ್ಮನ್ನೆಲ್ಲ ಸೃಷ್ಟಿಸಿ, ತನಗದು ಇಷ್ಟವೆಂಬ ಒಂದೇ ಕಾರಣಕ್ಕಾಗಿ ನಮ್ಮನ್ನು ಇಷ್ಟೆಲ್ಲ ಕಷ್ಟ ಅನುಭವಿಸುವಂತೆ ಮಾಡುತ್ತಿರುವ ಈ ದೇವರಿಗೆ ಬದಲಾಗಿ ಪ್ರತಿಯೊಬ್ಬನಲ್ಲೂ ಇರುವ, ಪ್ರತಿಯೊಬ್ಬನೂ-ಪ್ರತಿಯೊಂದೂ ತಾನೇ ಆಗಿರುವಂತಹ ದೇವರ ಕಲ್ಪನೆಯನ್ನು ಬೋಧಿಸುತ್ತದೆ ವೇದಾಂತ. ಸರ್ವಾಧಿಕಾರಿಯಾದ ರಾಜ ಈ ದೇಶದ ರಾಜಕಾರಣದಿಂದ ಹೊರಟುಹೋಗಿದ್ದಾನೆ. ಅಂತೆಯೆ ‘ಸ್ವರ್ಗರಾಜ್ಯ’ದ ಕಲ್ಪನೆಯೂ ವೇದಾಂತದಿಂದ ಶತಮಾನಗಳ ಹಿಂದೆಯೇ ಮಾಯವಾಯಿತು... ನಿಮ್ಮದು ಪ್ರಜಾಪ್ರಭುತ್ವ ರಾಷ್ಟ್ರವಾದ್ದರಿಂದ, ವೇದಾಂತವು ನಿಮ್ಮ ರಾಷ್ಟ್ರಧರ್ಮವಾಗಲು ಸಾಧ್ಯವಿದೆ. ಆದರೆ ನೀವು ಅದನ್ನು ಸರಿಯಾಗಿ ಅರ್ಥಮಾಡಿಕೊಳ್ಳಲು ಸಮರ್ಥರಾಗಿದ್ದು ಅರಿತುಕೊಂಡರೆ, ಮೆದುಳಿನಲ್ಲಿ ಕೇವಲ ಅಸ್ಪಷ್ಟ ಭಾವನೆಗಳನ್ನೂ ಮೂಢನಂಬಿಕೆಗಳನ್ನೂ ತುಂಬಿಕೊಂಡಿರದ, ನಿಜವಾದ ಸ್ತ್ರೀ ಪುರುಷರು ನೀವಾದರೆ, ಮತ್ತು ನೀವು ನಿಜವಾದ ಆಧ್ಯಾತ್ಮಿಕ ವ್ಯಕ್ತಿಗಳಾಗಬಯಸಿದರೆ ಮಾತ್ರ ಇದು ಸಾಧ್ಯ...

“ವೇದಾಂತವು ಯಾವುದನ್ನು ಬೋಧಿಸುವುದಿಲ್ಲ ಎಂದರೆ ಇದನ್ನು: ಅದು ಯಾವುದೋ ಒಂದು ಪುಸ್ತಕವನ್ನು ಏಕಮಾತ್ರ ಧರ್ಮಗ್ರಂಥವೆಂದು ಸಾರುವುದಿಲ್ಲ; ಕೆಲವರನ್ನು ಮಾನವ ಸಮಾಜದಿಂದ ಪ್ರತ್ಯೇಕಿಸಿ ‘ನೀವೆಲ್ಲ ಹುಳುಗಳು, ನಾವೆಲ್ಲ ದೇವರುಗಳು’ ಎಂದು ಹೇಳುವುದನ್ನು ಅದು ಅನುಮೋದಿಸುವುದಿಲ್ಲ. ನೀನು ಸರ್ವಶಕ್ತ ದೇವರಾದರೆ ನಾನೂ ಅದೇ ದೇವರು! ಆದ್ದರಿಂದ ವೇದಾಂತವು ಪಾಪ ಎನ್ನುವುದನ್ನು ಬೋಧಿಸುವುದಿಲ್ಲ. ತಪ್ಪುಗಳು ಘಟಿಸಬಹುದೇ ಹೊರತು ಪಾಪಗಳಲ್ಲ. ಆದ್ದರಿಂದ ಕಾಲಕ್ರಮದಲ್ಲಿ ಎಲ್ಲವೂ ಸರಿಹೋಗುತ್ತದೆ. ‘ಸೈತಾನ’ ಎಂಬುದನ್ನಾಗಲಿ, ಅಂತಹ ಇತರ ಅರ್ಥಹೀನ ವಿಚಾರಗಳನ್ನಾಗಲಿ ಅದು ಬೋಧಿಸುವುದಿಲ್ಲ. ವೇದಾಂತವು ಪಾಪವೆಂದು ಸಾರುವುದು ಒಂದನ್ನೇ–ನೀವು ನಿಮ್ಮನ್ನಾಗಲಿ, ಇತರರನ್ನಾಗಲಿ ಪಾಪಿ ಎಂದು ಭಾವಿಸಿದರೆ ಅದೇ ಪಾಪ. ಈ ಒಂದು ಪಾಪದಿಂದಲೇ ನಾವು ಪಾಪ ಎಂದು ಕರೆಯುವ ಇತರ ಎಲ್ಲ ದೋಷಗಳೂ ಸಂಭವಿಸುತ್ತವೆ. ನಾವು ನಮ್ಮ ಜೀವನದಲ್ಲಿ ಎಷ್ಟೋ ಪಾಪಗಳನ್ನು ಮಾಡಿದ್ದೇವೆ; ಆದರೂ ನಾವು ಮುನ್ನಡೆಯುತ್ತಿದ್ದೇವೆ. ನಾವು ತಪ್ಪುಗಳನ್ನು ಮಾಡಿದ್ದೇವಲ್ಲ, ಆದ್ದರಿಂದ ನಮಗೆ ಜಯವಾಗಲಿ! ನಿಮ್ಮ ಹಿಂದಿನ ಜೀವನವನ್ನು ಒಮ್ಮೆ ಪರಿಶೀಲಿಸಿ ನೋಡಿ. ನಿಮ್ಮ ಇಂದಿನ ಸ್ಥಿತಿಯು ಉತ್ತಮವಾಗಿರುವುದಾದರೆ, ಅದು ಸಾಧ್ಯವಾಗಿರುವುದು ನಿಮ್ಮ ಎಲ್ಲ ತಪ್ಪುಗಳಿಂದ ಹಾಗೂ ಯಶಸ್ಸುಗಳಿಂದ ಎಂದು ಕಂಡುಕೊಳ್ಳುತ್ತೀರಿ. ನಮ್ಮ ಯಶಸ್ಸುಗಳಿಗೆ ಜಯವಾಗಲಿ! ಹಾಗೆಯೇ ತಪ್ಪುಗಳಿಗೂ ಜಯವಾಗಲಿ! ಹಿಂದೆ ಆಗಿಹೋದದ್ದನ್ನೇ ಚಿಂತಿಸುತ್ತ ಕುಳಿತಿರಬೇಡಿ. ಮುನ್ನಡೆಯಿರಿ!

“ಗ್ರಂಥಗಳಿಲ್ಲ, ವ್ಯಕ್ತಿಗಳಿಲ್ಲ. ಸಾಕಾರ ದೇವರೂ ಇಲ್ಲ–ಇವೆಲ್ಲ ಹೋಗಬೇಕಾಗುತ್ತದೆ. ಅಲ್ಲದೆ ಈ ಇಂದ್ರಿಯಗಳಿಗೂ ಅತೀತರಾಗಬೇಕಾಗುತ್ತದೆ. ತೇಲುತ್ತಿರುವ ಹಿಮಬಂಡೆಗಳ ಮೇಲೆ ಶೀತದಿಂದ ಮರಣೋನ್ಮುಖರಾಗಿರುವವರಂತೆ ನಾವು ಇಂದ್ರಿಯಗಳಿಗೆ ಕಟ್ಟಿಹಾಕಲ್ಪಟ್ಟಿದ್ದೇವೆ. ಹಿಮಬಂಡೆಯ ಮೇಲಿರುವವರಿಗೆ ನಿದ್ರಿಸಬೇಕೆಂದು ಎಂತಹ ಉತ್ಕಟ ಬಯಕೆಯಾಗುತ್ತದೆ ಯೆಂದರೆ, ಅವನ ಸ್ನೇಹಿತರು ಅವನಿಗೆ ‘ನೀನಿಲ್ಲಿ ನಿದ್ರಿಸಿದರೆ ಸತ್ತೇ ಹೋಗುವೆ! ಎದ್ದೇಳು, ಎದ್ದೇಳು!’ ಎಂದು ಎಚ್ಚರಿಸಿದರೂ ಅವನು ‘ಪರವಾಗಿಲ್ಲ, ನಾನು ಸತ್ತರೆ ಸಾಯಲೇಳು! ನಾನೀಗ ನಿದ್ರೆ ಮಾಡಬೇಕು. ನನ್ನನ್ನು ಸುಮ್ಮನೆ ಬಿಟ್ಟುಬಿಡು’ ಎನ್ನುತ್ತಾನೆ. ಹಾಗೆಯೇ ನಾವೂ ಈ ಕ್ಷುದ್ರ ಇಂದ್ರಿಯಗಳಿಗೆ, ನಾವು ಸರ್ವನಾಶವಾಗುತ್ತೇವೆಂದು ತಿಳಿದಿದ್ದರೂ, ಕಟ್ಟಿಕೊಂಡೇ ಇರುತ್ತೇವೆ. ಇದಕ್ಕಿಂತಲೂ ಉನ್ನತವಾದ ವಿಚಾರಗಳಿವೆ ಎಂಬುದನ್ನು ಮರೆತೇಬಿಡುತ್ತೇವೆ.”

ಇದು, ‘ಭವಿಷ್ಯದ ಧರ್ಮವು ವೇದಾಂತವೆ?’ ಎಂಬ ಉಪನ್ಯಾಸದ ಒಂದಂಶ.

ಈ ಸರಣಿಯ ಹಿಂದೆ ಸ್ವಾಮೀಜಿಯವರು ಭಾರತಕ್ಕೆ ಸಂಬಂಧಿಸಿದಂತೆ ಹಲವಾರು ಭಾಷಣ ಗಳನ್ನು ಮಾಡಿದುದನ್ನು ಹಿಂದೆ ನಾವು ನೋಡಿದ್ದೇವೆ. ಭಾರತವನ್ನು ಕುರಿತು ಮಾತನಾಡುವು ದೆಂದರೆ ಅವರಿಗೆ ಎಲ್ಲಿಲ್ಲದ ಆಸಕ್ತಿ, ಆನಂದ. ಒಮ್ಮೆ ಇಂತಹ ಭಾಷಣವೊಂದನ್ನು ಪ್ರಾರಂಭಿ ಸುವ ಮುನ್ನ ಅವರು ತಮ್ಮ ಸಹಾಯಕ ಥಾಮಸ್ ಅಲನ್​ನಿಗೆ ಹೇಳಿದ್ದರು, “ಭಾರತದ ವಿಷಯ ವಾಗಿ ಮಾತನಾಡಲಾರಂಭಿಸಿದರೆ ನನಗೆ ಯಾವಾಗ ನಿಲ್ಲಿಸಬೇಕೆಂದೇ ತಿಳಿಯುವುದಿಲ್ಲ. ಆದ್ದ ರಿಂದ ನಾನು ತುಂಬ ಹೊತ್ತು ಮಾತನಾಡುತ್ತ ಹೋದರೆ ನೀನು ಸ್ವಲ್ಪ ನನ್ನ ಗಮನ ಸೆಳೆ.” ಎಂಟು ಗಂಟೆಗೆ ಸರಿಯಾಗಿ ಉಪನ್ಯಾಸ ಶುರುವಾಯಿತು. ಸ್ವಾಮೀಜಿ ತುಂಬ ಸ್ಫೂರ್ತಿಯಿಂದ ಮಾತನಾಡತೊಡಗಿದರು. ಗಂಟೆ ಒಂಬತ್ತಾಯಿತು, ಹತ್ತೂ ಆಯಿತು. ಈಗ ಅವರ ಗಮನ ಸೆಳೆಯಬೇಕು ಎಂದು ಅಲನ್ ನಿಶ್ಚಯಿಸಿದ. ಅವನು ಸಭಾಂಗಣದ ಹಿಂದಿನ ಸಾಲಿಗೆ ಹೋಗಿ ಸ್ವಾಮೀಜಿಯವರಿಗೆ ಮಾತ್ರ ಕಾಣುವಂತೆ ತನ್ನ ಗಡಿಯಾರವನ್ನು ಹಿಡಿದು ಅತ್ತಿಂದಿತ್ತ ಅಲುಗಾಡಿ ಸಿದ. ಸ್ವಾಮೀಜಿ ತಕ್ಷಣ ಅದನ್ನು ಗಮನಿಸಿ ಹೇಳುತ್ತಾರೆ–“ಅಲ್ಲಿ ನೋಡಿ ಅವರನ್ನು! ನಾನು ಉಪನ್ಯಾಸವನ್ನು ಇನ್ನೇನು ಪ್ರಾರಂಭಿಸಬೇಕು ಎನ್ನುವಾಗ ನಿಲ್ಲಿಸಬೇಕು ಎಂದು ಗಡಿಯಾರ ತೋರಿಸುತ್ತಿದ್ದಾರೆ!”

ಇನ್ನೊಂದು ಸಲ, ಒಂದು ಉಪನ್ಯಾಸದ ನಂತರ ಶ್ರೋತೃಗಳು ಪ್ರಶ್ನೆಗಳನ್ನು ಕೇಳಲು ಪ್ರಾರಂಭಿಸಿದರು. ಆದರೆ ಪ್ರಶ್ನೆಗಳು ಮುಗಿಯುವಂತೆಯೇ ಕಾಣಲಿಲ್ಲ. ಸ್ವಲ್ಪ ಹೊತ್ತಾದ ಮೇಲೆ ಶ್ರೋತೃಗಳಲ್ಲೇ ಒಬ್ಬರು, ‘ಪ್ರಶ್ನೆಗಳನ್ನು ಕೇಳಿದ್ದು ಜಾಸ್ತಿಯಾಯಿತು; ಇನ್ನು ಸಾಕೆಂದು ತೋರುತ್ತದೆ’ ಎಂದು ಸೂಚನೆ ನೀಡಿದರು. ಆಗ ಸ್ವಾಮೀಜಿ ಹೇಳಿದರು, “ನಿಮಗಿಷ್ಟ ಬಂದಷ್ಟು ಪ್ರಶ್ನೆ ಕೇಳಿ. ಹೆಚ್ಚಾದಷ್ಟೂ ಒಳ್ಳೆಯದು! ನಾನಿಲ್ಲಿಗೆ ಬಂದಿರುವುದೇ ಅದಕ್ಕಾಗಿ. ಮತ್ತು ನಿಮಗೆ ಅರ್ಥವಾಗುವವರೆಗೂ ನಾನು ನಿಮ್ಮನ್ನು ಬಿಡುವವನಲ್ಲ!” ಇದಕ್ಕೆ ಜನರ ಕರತಾಡನ ಎಷ್ಟು ದೀರ್ಘವಾಗಿತ್ತೆಂದರೆ ಸ್ವಾಮೀಜಿಯವರಿಗೆ ಒಂದು ನಿಮಿಷ ಮಾತನ್ನು ಮುಂದುವರಿಸುವುದಕ್ಕೆ ಸಾಧ್ಯವಾಗಲಿಲ್ಲ.

ಕೆಲವೊಮ್ಮೆ ಅವರು ತಮ್ಮ ಉತ್ತರದಿಂದ ಜನರನ್ನು ಆಶ್ಚರ್ಯಚಕಿತರನ್ನಾಗಿಸುತ್ತಿದ್ದರು. “ಜನನ-ಮರಣಗಳ ನಿಯಮಗಳು’ ಎಂಬ ಉಪನ್ಯಾಸದ ಬಳಿಕ ಯಾರೋ ಕೇಳಿದರು, “ಸ್ವಾಮೀಜಿ, ನಿಮಗೆ ನಿಮ್ಮ ಪೂರ್ವಜನ್ಮದ ನೆನಪಿದೆಯೆ?” ತಕ್ಷಣ ಸ್ವಾಮೀಜಿ ಗಂಭೀರ ಭಾವ ದಲ್ಲಿ ಉತ್ತರಿಸಿದರು, “ಹೌದು, ಸ್ಪಷ್ಟವಾಗಿ; ನಾನು ಚಿಕ್ಕ ಹುಡುಗನಾಗಿದ್ದಾಗಿನಿಂದಲೂ.” ಇನ್ನು ಕೆಲವೊಮ್ಮೆ ಅವರು ತಮ್ಮ ಹಾಸ್ಯಮಯ ಉತ್ತರದಿಂದ ಸಭಾಂಗಣಕ್ಕೆ ಸಭಾಂಗಣವನ್ನೇ ನಗೆಗಡಲಿನಲ್ಲಿ ಮುಳುಗಿಸಿಬಿಡುತ್ತಿದ್ದರು. ಭಗವದ್​ಜ್ಞಾನದ ಕುರಿತಾದ ಉಪನ್ಯಾಸವೊಂದು ಮುಕ್ತಾಯಗೊಂಡ ಮೇಲೆ ಒಂದು ಪ್ರಶ್ನೆ ಬಂದಿತು, “ಸ್ವಾಮೀಜಿ, ನೀವು ದೇವರನ್ನು ನೋಡಿ ದ್ದೀರಾ?” ಸ್ವಾಮೀಜಿಯವರ ಮುಖದಲ್ಲಿ ತುಂಟತನದ ಕಿರುನಗೆಯೊಂದು ಮಿಂಚಿತು. “ಏನು! ನನ್ನನ್ನು ನೋಡಿದರೆ ಹಾಗೆನ್ನಿಸುತ್ತದೆಯೆ?–ನನ್ನಂತಹ ಸ್ಥೂಲಕಾಯದವನನ್ನು!”

ಉಪನ್ಯಾಸವನ್ನು ನೀಡಲು ಎದ್ದುನಿಂತರೆಂದರೆ ಸ್ವಾಮೀಜಿಯವರ ಪ್ರಶಾಂತ-ಪ್ರಚಂಡ ವ್ಯಕ್ತಿತ್ವವು ಅವರ ದಿವ್ಯಸಂದೇಶದ ಪ್ರಮಾಣಕ್ಕೆ ತಕ್ಕಂತೆ ಭವ್ಯವಾಗಿ ಕಂಡುಬರುತ್ತಿತ್ತು. ಸ್ವತಃ ಭಗವತ್ಸ್ವರೂಪಿಯಾದ ಅವರು, ತಾವು ಬೋಧಿಸುವುದನ್ನು ಸ್ವತಃ ಸಾಕ್ಷಾತ್ಕರಿಸಿಕೊಂಡಂತಹ ವ್ಯಕ್ತಿಯಾದ ಅವರು, ಇತರರನ್ನೂ, ಆ ದಿವ್ಯಜ್ಞಾನದ ಮಟ್ಟಕ್ಕೆ ಏರಿಸಬಲ್ಲವರಾಗಿ ಅಲ್ಲಿ ನಿಲ್ಲುತ್ತಿದ್ದರು. ಉಪನ್ಯಾಸಗಳಲ್ಲಿ ಅವರಿಗೆ ನೆರವಾಗುತ್ತಿದ್ದ ಅಲನ್ ಒಮ್ಮೆ ಹೇಳುತ್ತಾನೆ, “ನಾನು ಅವರನ್ನು ಸಭೆಗೆ ಪರಿಚಯಿಸಿಕೊಡುತ್ತಿದ್ದಂತಹ ಸಂದರ್ಭ ಅದು. ಆಗ ಇದ್ದಕ್ಕಿದ್ದಂತೆ ಅವರು ಬೃಹದಾಕಾರವಾಗಿ ಬೆಳೆದು ನಿಂತಿರುವಂತೆ ನಾನು ಕಂಡೆ! ಅವರ ಮುಂದೆ ನಾನೊಬ್ಬ ಕುಬ್ಜನಂತೆ ಭಾಸವಾಯಿತು. ಅಂದಿನಿಂದ ನಾನವರನ್ನು ಪರಿಚಯಿಸುವಾಗ ಅವರ ಪಕ್ಕದಲ್ಲಿ ನಿಲ್ಲುತ್ತಲೇ ಇರಲಿಲ್ಲ. ಬದಲಾಗಿ ವೇದಿಕೆಯ ಕೆಳಗೇ ನಿಂತು ಮಾತನಾಡುತ್ತಿದ್ದೆ.”

ಒಂದು ದಿನ ಸ್ವಾಮೀಜಿಯವರಿಗೆ ವಿಪರೀತ ಶೀತವಾಗಿತ್ತು. ಅದರಿಂದ ಅವರಿಗೆ ಸಾಕಷ್ಟು ತೊಂದರೆಯೇ ಆಗಿದ್ದಿರಬೇಕು. ಅಂದಿನ ಉಪನ್ಯಾಸವನ್ನು ವೀಕ್ಷಿಸಿದ ಮಿಸ್ ಆಲ್ಬರ್ಟ್ಸ್ ಎಂಬುವಳು ಬರೆಯುತ್ತಾಳೆ: “ಸ್ವಾಮೀಜಿಯವರು ವೇದಿಕೆಯ ಮೇಲೆ ಹೋಗಿ ನಿಂತಾಗ, ಅವರು ಪ್ರಯತ್ನಪೂರ್ವಕವಾಗಿ ನಿಂತಂತೆ ಕಂಡಿತು. ಅವರು ಭಾರವಾದ ಹೆಜ್ಜೆಗಳನ್ನಿಡುತ್ತ ನಡೆಯು ತ್ತಿದ್ದರು. ಅವರ ಕಣ್ರೆಪ್ಪೆಗಳು ಊದಿಕೊಂಡಿದ್ದುದನ್ನು ನಾನು ಗಮನಿಸಿದೆ. ಅವರು ನೋವು ಅನುಭವಿಸುತ್ತಿದ್ದಂತೆ ಕಂಡಿತು. ಮಾತನ್ನು ಪ್ರಾರಂಭಿಸುವ ಮೊದಲು ಅವರು ಒಂದು ನಿಮಿಷ ಮೌನವಾಗಿ ನಿಂತರು. ನೋಡನೋಡುತ್ತಿದ್ದಂತೇ ಅವರಲ್ಲೊಂದು ಅದ್ಭುತ ಬದಲಾವಣೆ ಕಂಡುಬಂದಿತು; ಅವರ ಮುಖಭಾವ ಬೆಳಗಿತು. ನಾನು ಅವರ ಮಾತುಗಳ ಮಹಾಶಕ್ತಿಯನ್ನು ಅನುಭವಿಸಿದೆ. ಆ ಮಾತುಗಳನ್ನು ಕೇಳುವುದಕ್ಕಿಂತ ಹೆಚ್ಚಾಗಿ ಅನುಭವಿಸಬಹುದಾಗಿತ್ತು. ನಾನು ಉನ್ನತ ಅಸ್ತಿತ್ವದ, ಭಾವಗಳ ಸಾಗರದೊಳಕ್ಕೆ ಸೆಳೆಯಲ್ಪಟ್ಟೆ. ಭಾಷಣ ಮುಗಿದ ಮೇಲೆ ಆ ಉನ್ನತ ಭಾವದಿಂದ ಕೆಳಗಿಳಿಯುವುದು ನೋವನ್ನುಂಟುಮಾಡಿತು. ಆ ಕಣ್ಣುಗಳು–ಆಹ್! ಎಷ್ಟು ಅದ್ಭುತ! ಅವು ನಕ್ಷತ್ರಗಳಂತೆ ಮಿನುಗುತ್ತಿದ್ದುವು. ಅವುಗಳಿಂದ ನಿರಂತರವಾಗಿ ಬೆಳಕು ಹೊಮ್ಮುತ್ತಿತ್ತು.”

ಜನರ ಮನಸ್ಸನ್ನು ಭದ್ರವಾಗಿ ಬಂಧಿಸಿಟ್ಟು ಕೊಳೆಸುವಂತಹ ಸಂಪ್ರದಾಯದ ಕಟ್ಟೆಗಳನ್ನು ಒಡೆದು ಚೂರು ಮಾಡುವ ಶಕ್ತಿ ಇತ್ತು ಸ್ವಾಮೀಜಿಯವರ ಮಾತಿನಲ್ಲಿ. ಹಳೆಯ ಸಿದ್ಧಾಂತಗಳನ್ನು ಬುಡಮೇಲು ಮಾಡುವ ಅವರ ಮಾತುಗಳು ಹಲವಾರು. ಇಂತಹ ಒಂದೊಂದು ಮಾತೂ ಸಿಡಿಲ ಆಸ್ಫೋಟದಂತೆ ಬಂದು ಅಪ್ಪಳಿಸಿ ಮೂಢನಂಬಿಕೆಗಳ ಕೋಟೆಯನ್ನು ಕುಟ್ಟಿ ಪುಡಿಗೈದು ಬಂಧಿತವಾದ ಆತ್ಮವನ್ನು ಮುಕ್ತಗೊಳಿಸುತ್ತಿತ್ತು. ಜನರು ತಾವು ಶರೀರ-ಮನಸ್ಸುಗಳಿಂದ ಕೂಡಿದ ಬಡಜೀವ ಎಂಬ ನಂಬಿಕೆಗೆ ಗಂಟುಬಿದ್ದಿದ್ದರೆ ಸ್ವಾಮೀಜಿ ಗುಡುಗುತ್ತಾರೆ, “ದಿವ್ಯಾತ್ಮರು ನೀವು! ನಿಮಗಿಲ್ಲ ಸಾವು!” ಎಂದು. ಒಂದು ಸಲವಂತೂ ಅವರು ತಮ್ಮ ಉಪನ್ಯಾಸವನ್ನು ಪ್ರಾರಂಭಿ ಸುವ ಮೊದಲು ಒಂದು ಕ್ಷಣ ಜನರನ್ನು ದಿಟ್ಟಿಸಿ ನೋಡಿ ಉದ್ಗರಿಸಿದರು, “ಏಳಿ, ಎಚ್ಚರಗೊಳ್ಳಿ, ಗುರಿಮುಟ್ಟುವವರೆಗೂ ನಿಲ್ಲದಿರಿ!” ಈ ನುಡಿಗಳು ಅಲ್ಲಿದ್ದವರಲ್ಲಿ ಒಂದು ವಿದ್ಯುದಾಘಾತವನ್ನೇ ಉಂಟುಮಾಡಿದುವು ಎಂದು ಒಬ್ಬಾಕೆ ಹೇಳುತ್ತಾಳೆ.

ಸ್ಯಾನ್​ಫ್ರಾನ್ಸಿಸ್ಕೋದಲ್ಲಿ ‘ಮುಕ್ತಿಯ ದಾರಿ’ ಎಂಬ ಉಪನ್ಯಾಸ ಮುಗಿದಿತ್ತು; ಮರುದಿನ ಹೊಸ ಉಪನ್ಯಾಸ ಮಾಲಿಕೆಯೊಂದು ಪ್ರಾರಂಭವಾಗಲಿತ್ತು. ಸ್ವಾಮೀಜಿಯವರು ಆ ಉಪನ್ಯಾಸದ ವಿಷಯವನ್ನು ಪ್ರಕಟಿಸುತ್ತ ಹೇಳಿದರು, “ನಾಳೆ ರಾತ್ರಿ ನಾನು ‘ಮನಸ್ಸು; ಅದರ ಶಕ್ತಿ ಹಾಗೂ ಸಾಧ್ಯತೆಗಳು’ ಎಂಬ ವಿಷಯವಾಗಿ ಮಾತನಾಡಲಿದ್ದೇನೆ. ಕೇಳಲು ಬನ್ನಿ; ನಾನು ಸ್ವಲ್ಪ ಬಾಂಬ್ ಎಸೆಯುವ ಕೆಲಸ ಮಾಡಲಿದ್ದೇನೆ!” ಬಳಿಕ ಶ್ರೋತೃಗಳತ್ತ ಮಂದಹಾಸದ ದೃಷ್ಟಿ ಹಾಯಿಸುತ್ತ ಬಲಗೈಯನ್ನು ಮೇಲೆತ್ತಿ ನುಡಿದರು, “ಬನ್ನಿ, ನಿಮಗೆ ಒಳಿತಾಗುತ್ತದೆ!”

ಮಾರನೆಯ ದಿನ ಸಂಜೆ ಸಭಾಂಗಣವು ಕಾಲಿಡಲೂ ಸ್ಥಳವಿಲ್ಲದಂತೆ ತುಂಬಿಹೋಯಿತು. ಸ್ವಾಮೀಜಿಯವರೂ ತಮ್ಮ ಮಾತಿಗೆ ತಪ್ಪಲಿಲ್ಲ. ಬಾಂಬುಗಳನ್ನು ಎಸೆದೇ ಎಸೆದರು. ಅಲ್ಲದೆ, ರೋಡ್​ಹ್ಯಾಮೆಲ್ ಎಂಬೊಬ್ಬ ವೇದಾಂತದ ವಿದ್ಯಾರ್ಥಿ ಹೇಳುವಂತೆ, ಬಾಂಬುಗಳನ್ನು ಅತಿ ಹೆಚ್ಚು ಪರಿಣಾಮಕಾರಿಯಾಗಿ ಎಸೆಯುವುದು ಹೇಗೆಂಬುದನ್ನು ಎಲ್ಲರಿಗಿಂತ ಚೆನ್ನಾಗಿ ತಿಳಿದವರು ಸ್ವಾಮೀಜಿ! ಅಂದಿನ ಉಪನ್ಯಾಸದಲ್ಲಿ ಅವರು ಪವಿತ್ರತೆ ಹಾಗೂ ಬ್ರಹ್ಮಚರ್ಯದ ಬಗ್ಗೆ ದೀರ್ಘ ವಾಗಿ ಮಾತನಾಡಿ, ಪ್ರತಿಯೊಬ್ಬ ಸ್ತ್ರೀಯನ್ನೂ ಮಾತೃಭಾವದಿಂದ ಕಾಣುವ ಕಲ್ಪನೆಯನ್ನು ಮುಂದಿಟ್ಟರು: “ನಾವು ನಮ್ಮ ಸ್ವಂತ ತಾಯಿಯನ್ನು ಇತರ ಸ್ತ್ರೀಯರಿಗಿಂತ ಭಿನ್ನವಾದ ದೃಷ್ಟಿ ಯಲ್ಲಿ ನೋಡುವುದು ತೀರ ಸಹಜವೇ. ಆದರೆ ಇತರ ಸ್ತ್ರೀಯರನ್ನೂ ಮಾತೃದೃಷ್ಟಿಯಿಂದ ನೋಡ ಬೇಕಾದುದು ಬಹಳ ಮುಖ್ಯ.” ಪವಿತ್ರತೆಯೆಂಬುದು ಸಂನ್ಯಾಸಿಗಳಿಗೂ ಗೃಹಸ್ಥರಿಗೂ ಸಮಾನ ವಾಗಿ ಪ್ರಮುಖವಾದದ್ದೆಂಬ ವಿಷಯವನ್ನು ಸ್ವಾಮೀಜಿ ಒತ್ತಿಹೇಳಿದರು. ಈ ವಿಷಯವಾಗಿ ಅವರೆಂದರು, “ಮೊನ್ನೆ ತಾನೆ ಒಬ್ಬ ಹಿಂದೂ ಯುವಕ ನನ್ನನ್ನು ನೋಡಲು ಬಂದಿದ್ದ. ಅವನು ಎರಡು ವರ್ಷದಿಂದ ಇಲ್ಲೇ ಇದ್ದಾನೆ; ಈಗ ಸ್ವಲ್ಪ ಕಾಲದಿಂದ ಏನೋ ಅನಾರೋಗ್ಯವಾಗಿದೆ. ಮಾತಿನ ಮಧ್ಯೆ ಅವನು ಹೇಳಿದ–‘ಈ ಬ್ರಹ್ಮಚರ್ಯದ ತತ್ತ್ವವೆಲ್ಲ ಸುಳ್ಳೆಂದು ಕಾಣುತ್ತದೆ. ಏಕೆಂದರೆ, ಇಲ್ಲಿನ ಡಾಕ್ಟರುಗಳೇ ನನಗೆ ಅದರ ವಿರುದ್ಧವಾಗಿ ಸಲಹೆ ನೀಡಿದ್ದಾರೆ. ಬ್ರಹ್ಮ ಚರ್ಯವು ಪ್ರಕೃತಿ ನಿಯಮಕ್ಕೆ ವಿರುದ್ಧ ಎಂದು ಹೇಳಿದ್ದಾರೆ’ ಅಂತ. ನಾನವನಿಗೆ ಹೇಳಿದೆ, ‘ನೀನು ಎಲ್ಲಿಂದ ಬಂದೆಯೋ ಆ ಭಾರತಕ್ಕೆ ವಾಪಸು ಹೋಗು! ಹೋಗಿ, ಸಾವಿರಾರು ವರ್ಷಗಳಿಂದಲೂ ಬ್ರಹ್ಮಚರ್ಯ-ಪಾವಿತ್ರ್ಯಗಳನ್ನು ಅಭ್ಯಾಸ ಮಾಡುತ್ತಿರುವ ನಿನ್ನ ಪೂರ್ವಿಕರ ಮಾತನ್ನೊಂದಿಷ್ಟು ಕೇಳಿಸಿಕೊಂಡು ಬಾ!’ ಅಂತ.” ಬಳಿಕ ಸ್ವಾಮೀಜಿಯವರು, ಮಾತಿನಲ್ಲಿ ಹೇಳಲಾಗದಂತಹ ಜುಗುಪ್ಸೆಯ ಮುಖಭಾವವನ್ನು ತಳೆದು, ‘ಬ್ರಹ್ಮಚರ್ಯವು ತೀರ ಅಸಹಜವಾದದ್ದು, ಪ್ರಕೃತಿ ನಿಯಮಕ್ಕೆ ವಿರುದ್ಧವಾದದ್ದು’ ಎಂಬ ಪಾಶ್ಚಾತ್ಯ ವೈದ್ಯರ ವಾದವನ್ನು ಉಗ್ರವಾಗಿ ಖಂಡಿಸುತ್ತ ನುಡಿದರು–“ಹೀಗೆ ಮಾತನಾಡುವ ಈ ದೇಶದ ವೈದ್ಯರಿದ್ದೀರಲ್ಲ, ನಿಮಗೆ ನೀವೇನು ಮಾತನಾಡು ತ್ತಿದ್ದೀರಿ ಎಂಬುದೇ ಗೊತ್ತಿಲ್ಲ! ನಿಮಗೆ ‘ಪಾವಿತ್ರ್ಯ’ ಎಂಬ ಶಬ್ದದ ಅರ್ಥವೇ ತಿಳಿದಿಲ್ಲ! ನೀವು ಮೃಗಗಳು! ಮೃಗಗಳು! ಪ್ರಾಣಿಧರ್ಮಕ್ಕನುಸಾರವಾಗಿ ಬದುಕುವ ನೀವು ಹೆಚ್ಚಿನದೇನನ್ನು ಹೇಳೀರಿ?” ಹೀಗೆ ಚುಚ್ಚಿ ನುಡಿದ ಸ್ವಾಮೀಜಿಯವರು, ವಿರೋಧವನ್ನು ಆಹ್ವಾನಿಸುವ ನೋಟ ದಿಂದ ಪ್ರೇಕ್ಷಕರತ್ತ ದೃಷ್ಟಿ ಹಾಯಿಸಿದರು, ಸಭೆಯಲ್ಲಿ ಹಲವಾರು ವೈದ್ಯರೂ ಆ ವೈದ್ಯರ ಅಭಿಪ್ರಾಯವನ್ನು ಸಮರ್ಥಿಸುವಂಥವರೂ ಇದ್ದಿರಲೇಬೇಕು. ಆದರೆ ಒಂದೇ ಒಂದು ಉಸಿರೂ ಮೇಲೇಳಲಿಲ್ಲ.

ಹೀಗೆ ಸ್ವಾಮೀಜಿ ಇಂತಹ ‘ಬಾಂಬು’ಗಳನ್ನು ಎಸೆದರೂ, ಉಪನ್ಯಾಸದ ಬಳಿಕ ತಮ್ಮ ಆಪ್ತರೊಂದಿಗೆ ಮನೆಗೆ ಹಿಂದಿರುವಾಗ ಹಿಂದಣ ಆನಂದಮಯರಾದ ಸರಳ ಸ್ನೇಹಜೀವಿಯೇ ಆಗಿರುತ್ತಿದ್ದರು.

ಅಂದಿನ ಸಭೆಯಲ್ಲಿ ಹಾಜರಿದ್ದ ಮಿಸ್ ಸಾರಾ ಫಾಕ್ಸ್ ಎನ್ನುವವಳು ಹೇಳುತ್ತಾಳೆ: “ಸ್ವಾಮೀಜಿಯವರು ತಮ್ಮ ಉಪನ್ಯಾಸದ ಮೊದಲರ್ಧ ಭಾಗದಲ್ಲಿ ಸಭಿಕರನ್ನು ಅವರ ಸ್ವೇಚ್ಛಾ ಚಾರ-ವಿಷಯಲಂಪಟತನಗಳಿಗಾಗಿ ಚೆನ್ನಾಗಿ ತರಾಟೆಗೆ ತೆಗೆದುಕೊಂಡರು. ಆದರೆ ನಂತರದ ಭಾಗದಲ್ಲಿ ಅವರ ಮೇಲೆ ತಮ್ಮ ವಾತ್ಯಲ್ಯಧಾರೆಯನ್ನೇ ಹರಿಯಿಸಿದರು” ಎಂದು. ಮಾನವನ ಮೇಲೆ ಅವರಲ್ಲಿ ಅಚಲವಾದ ವಾತ್ಸಲ್ಯ-ಪ್ರೇಮ ಯಾವಾಗಲೂ ಇದ್ದೇ ಇತ್ತು. ನಿಜಕ್ಕೂ ಅವರ ಅಂತರ್ದೃಷ್ಟಿ ಅಷ್ಟೊಂದು ನಿಖರವಾಗಿರಲೂ, ಅವರ ಮಾತುಗಳು ಅಷ್ಟೊಂದು ಹರಿತವಾ ಗಿರಲೂ ಅದೇ ಕಾರಣ. ಸತ್ಯವೆ ಅವರ ‘ಬಾಂಬು’ಗಳಲ್ಲಿನ ಸಿಡಿಮದ್ದಾದರೆ, ಅವರ ಪ್ರೀತಿಯೇ ಈ ಸಿಡಿಮದ್ದನ್ನು ಹೊತ್ತಿಸುತ್ತಿದ್ದ ಕಿಡಿ. ಆದರೆ ಅವುಗಳನ್ನು ತಡೆದುಕೊಳ್ಳಲು ಮಾತ್ರ ಎಲ್ಲರಿಗೂ ಸಾಧ್ಯವಾಗುತ್ತಿರಲಿಲ್ಲ.

ಈ ಉಪನ್ಯಾಸಕ್ಕೆ ಹಾಜರಿದ್ದವರಲ್ಲಿ ಹಲವಾರು ಮಂದಿ ಅನಂತರ ಮಾತನಾಡಿಕೊಳ್ಳು ತ್ತಿದ್ದರಂತೆ–‘ಅವರು ನನ್ನನ್ನು ಉದ್ದೇಶಿಸಿಯೇ ಹಾಗೆಲ್ಲ ಹೇಳಿರಬಹುದೆ?’ ಎಂದು. ಆದರೆ ಸ್ವಾಮೀಜಿ ಯಾರೊಬ್ಬರನ್ನೂ ಉದ್ದೇಶಿಸಿ ಆಡಿದ ಮಾತುಗಳಲ್ಲ ಅವು, ಅಥವಾ ಯಾರನ್ನಾದರೂ ನೋಯಿಸಬೇಕೆಂಬುದೂ ಅವರ ಉದ್ದೇಶವಾಗಿರಲಿಲ್ಲ. ಆದರೆ ಇದನ್ನು ಅರ್ಥಮಾಡಿಕೊಳ್ಳ ಲಾರದ ಕೆಲವರು ಉಪನ್ಯಾಸಗಳಿಗೆ ಬರುವುದನ್ನೇ ನಿಲ್ಲಿಸಿಬಿಟ್ಟರು.

ಇದನ್ನು ಸ್ವಾಮೀಜಿ ಗಮನಿಸಿದರೋ ಇಲ್ಲವೋ; ಒಂದು ವೇಳೆ ಗಮನಿಸಿದ್ದರೂ ಅವರಂತೂ ವಿಚಲಿತರಾಗುವವರಲ್ಲ. ಇಂದು ತಮ್ಮ ಮಾತುಗಳಿಂದ ಕಸಿವಿಸಿಗೊಂಡವರು ಮುಂದೆ ಅದೇ ಮಾತುಗಳಿಂದ ಧನ್ಯತೆಯನ್ನು ಹೊಂದಿಯೇ ತೀರುತ್ತಾರೆಂಬುದು ಅವರಿಗೆ ಗೊತ್ತಿತ್ತು. ಹಿಂದೊಮ್ಮೆ ಕೆಲವರು ಸ್ವಾಮೀಜಿಯವರಿಗೆ ಅವರ ಮಾತುಗಳನ್ನು ನಯಗೊಳಿಸಿಕೊಳ್ಳುವಂತೆ ಸಲಹೆ ನೀಡಿದಾಗ, ಅವರು ಮೇರಿಗೆ ಬರೆದ ಒಂದು ಪತ್ರದ ಈ ಸಾಲುಗಳನ್ನು ನೆನಪಿಸಿಕೊಳ್ಳ ಬಹುದು: “ಸತ್ಯವನ್ನು ನಾನು ಮಹಾ ಶಕ್ತಿಶಾಲಿಯಾದ ಒಂದು ಸುಡುವ ವಸ್ತುವಿಗೆ ಹೋಲಿಸು ತ್ತೇನೆ. ಅದು ಯಾವ ವಸ್ತುವಿನ ಮೇಲೆ ಬಿದ್ದರೂ ಅದನ್ನು ಸುಡುತ್ತದೆ. ಮೆದುವಸ್ತುವಾದರೆ ಬೇಗನೆ, ಕಠಿಣ ಶಿಲೆಯಾದರೆ ತಡವಾಗಿ–ಅದರೆ ಅದಂತೂ ಸುಡಲೇಬೇಕು. ನಾನೊಂದು ಸಂದೇಶವನ್ನು ನೀಡಬೇಕಾದದ್ದಿದೆ. ಜಗತ್ತು ಬಯಸುವಂತೆ ಸವಿಯಾಗಿರಲು ನನಗೆ ಸಮಯ ವಿಲ್ಲ; ಮತ್ತು ಹೀಗೆ ಸವಿಯಾಗಿರಲು ಮಾಡುವ ಪ್ರತಿಯೊಂದು ಪ್ರಯತ್ನವೂ ಒಬ್ಬನನ್ನು ಆಷಾಢಭೂತಿಯನ್ನಾಗಿಸುತ್ತದೆ.” ಸುಮಾರು ಮೂರ್ನಾಲ್ಕು ವರ್ಷಗಳ ಹಿಂದೆಯೇ ಹೀಗೆ ಬರೆದಿದ್ದ ಅವರ ನಿಲುವು ಈಗ ಮತ್ತಷ್ಟು ಸ್ಪಷ್ಟವಾಗಿತ್ತು, ಮತ್ತಷ್ಟು ದೃಢವಾಗಿತ್ತು. ಒಮ್ಮೆ ಶ್ರೀಮತಿ ಹ್ಯಾನ್ಸ್​ಬ್ರೋ ಕೂಡ ಸ್ವಾಮೀಜಿಯವರ ಕಾರ್ಯವಿಧಾನದ ಬಗ್ಗೆ ತನ್ನ ಶಂಕೆ ವ್ಯಕ್ತಪಡಿಸಿದಳು. ಆಗ ಅವರು ಅದಕ್ಕೆ ನೇರವಾಗಿ ಉತ್ತರಿಸದೆ ಹೀಗೆಂದರು, “ನೋಡು, ನನ್ನ ಮರಣಾನಂತರ ಹತ್ತು ವರ್ಷಗಳೊಳಗಾಗಿ ನಾನು ದೇವರೆಂದು ಪೂಜಿಸಲ್ಪಡುವೆ.”

ಕ್ಯಾಲಿಫೋರ್ನಿಯದಲ್ಲಿದ್ದ ಈ ದಿನಗಳಲ್ಲಿ ಸ್ವಾಮೀಜಿಯವರ ಆರೋಗ್ಯ ಬಹಳ ಮಟ್ಟಿಗೆ ಸುಧಾರಿಸುತ್ತ ಬಂದಿತು. ಇಲ್ಲಿ ಅವರಿಗೆ ನಿರಂತರವಾಗಿ ಕೆಲಸವಿರುತ್ತಿತ್ತಾದರೂ ಬೆಳಗಿನ ಹೊತ್ತಿನಲ್ಲಿ ಒಮ್ಮೊಮ್ಮೆ ಸುತ್ತಾಡಲು ಸಮಯ ಸಿಗುತ್ತಿತ್ತು. ಆಗ ಗೋಲ್ಡನ್ ಗೇಟ್ ಪಾರ್ಕಿಗೋ ಸಮುದ್ರ ತೀರಕ್ಕೋ ಇಲ್ಲವೆ ಅಲ್ಲಿನ ಪ್ರಸಿದ್ಧ ಚೈನಾ ಟೌನಿಗೋ ಹೋಗಿಬರು ತ್ತಿದ್ದರು. ಇದೀಗ ಅವರ ಜೀವನದ ಕೊನೆಯ ಕೆಲವು ವರ್ಷಗಳಲ್ಲಿ ವಿಶೇಷವಾಗಿ ಕಂಡುಬರು ತ್ತಿದ್ದ ಅಗಾಧ ಶಾಂತಿಯ ತಂಪನ್ನು ಅವರು ಆತ್ಮೀಯರಿಗೆ ಬರೆದ ಪತ್ರಗಳಲ್ಲಿ ಕಾಣಬಹುದಾ ಗಿತ್ತು. ಈ ಸಮಯದಲ್ಲಿ ಅವರು ಸೋದರಿ ನಿವೇದಿತೆಗೆ ಬರೆದ ಪತ್ರವೊಂದರಲ್ಲಿ ಹೇಳುತ್ತಾರೆ, “ಬಿತ್ತಿದ ಬೀಜವೊಂದು ಮರವಾಗಿ ಬೆಳೆಯುವ ಮುನ್ನ ತಾನು ನೆಲದಡಿಯಲ್ಲಿ ಕೊಳೆತು ಸಾಯಬೇಕು. ನನ್ನ ಜೀವನದ ಕಳೆದೆರಡು ವರ್ಷಗಳು ಇಂತಹ ಕೊಳೆತ ಎನ್ನಬಹುದು. ನಾನೆಂದಿಗೂ ಮೃತ್ಯುವಿಗೆ ಹೆದರಲಿಲ್ಲ. ಆದರೆ ನನ್ನ ಜೀವನದಲ್ಲೊಂದು ಉತ್ಕ್ರಾಂತಿಯೇ ಆಯಿತು. ಇಂತಹ ಒಂದು ಉತ್ಕ್ರಾಂತಿ ನನ್ನನ್ನು ರಾಮಕೃಷ್ಣರಲ್ಲಿಗೆ ಕರೆತಂದಿತು. ಇನ್ನೊಂದು ನನ್ನನ್ನು ಅಮೆರಿಕೆಗೆ ಕಳಿಸಿತು. ಆದರೆ ಇದು ಮಾತ್ರ ಅವುಗಳಲ್ಲೆಲ್ಲ ಪ್ರಬಲವಾದುದು. ಸದ್ಯ! ಅದು ಮುಗಿದುಹೋಯಿತು. ಈಗ ನಾನು ಎಷ್ಟು ಶಾಂತವಾಗಿದ್ದೇನೆಂದರೆ ಕೆಲವೊಮ್ಮೆ ನನಗೇ ಆಶ್ಚರ್ಯವಾಗುತ್ತದೆ.”

ಸ್ವಾಮೀಜಿಯವರ ಆಂತರ್ಯದಲ್ಲಿ ಈ ಶಾಂತಿ ಎಂಬುದು ಬೆಳೆಯುತ್ತಿದ್ದಂತೆ, ಅದಕ್ಕೆ ತದ್ವಿರುದ್ಧವಾಗಿ ಅವರ ಭಾಷಣಗಳೂ ತರಗತಿಗಳೂ ಅಸಾಧಾರಣ ರಭಸಪೂರ್ಣವಾಗಿದ್ದುವು. ಅವರ ಆಕರ್ಷಣ ಶಕ್ತಿ ಎಂದಿನಷ್ಟೇ ಪ್ರಬಲವಾಗಿತ್ತು. ಹೆಚ್ಚು ಹೆಚ್ಚು ಸ್ತ್ರೀಪುರುಷರು ಅವರ ನಿಕಟ ಸಂಪರ್ಕಕ್ಕೆ ಬಂದರು.

ಸ್ವಾಮೀಜಿಯವರ ವ್ಯಕ್ತಿತ್ವದಲ್ಲಿ ಹಾಸುಹೊಕ್ಕಾಗಿದ್ದ ಹಾಸ್ಯದ ಅಂಶವಂತೂ ಯಾವಾಗ ಬೇಕಾ ದರೂ ಸಿಡಿಯಬಹುದಾಗಿತ್ತು. ಕೆಲವು ಥಿಯಾಸೊಫಿಕ್ ಅನುಯಾಯಿಗಳಿಗೆ ತಂತ್ರಸಾಧನೆಯ ಬಗ್ಗೆ ವಿಶೇಷ ಆಸಕ್ತಿ. ಇವರೊಮ್ಮೆ ಕೇಳಿದರು, “ಸ್ವಾಮೀಜಿ, ನೀವು ಭೂತವನ್ನು ನೋಡಿದ್ದೀರಾ?” ತಕ್ಷಣ ಬಂದಿತು ಉತ್ತರ, “ಓ, ಖಂಡಿತ! ಭಾರತದಲ್ಲಿ ನಾವು ಬೆಳಗಿನ ತಿಂಡಿಗೆ ಉಪಯೋಗಿ ಸುವುದು ಅವುಗಳನ್ನೇ!”

ಸಾಧುಸಂತರ ಕಾಲಜ್ಞಾನದ ಬಗ್ಗೆ ಒಬ್ಬರು ಪ್ರಶ್ನಿಸಿದಾಗ ಸ್ವಾಮೀಜಿ ಹೇಳಿದರು, “ನಾನು ಚಿಕ್ಕ ಹುಡುಗನಾಗಿದ್ದಾಗ ಒಂದು ದಿನ ಬೀದಿಯಲ್ಲಿ ಆಟವಾಡುತ್ತಿದ್ದೆ. ಆಗ ಅಲ್ಲೇ ಹೋಗುತ್ತಿದ್ದ ಒಬ್ಬ ಸಾಧು ನನ್ನ ತಲೆಯ ಮೇಲೆ ಕೈಯಿಟ್ಟು, ‘ಮಗು, ನೀನು ಒಂದು ದಿನ ಒಬ್ಬ ಮಹಾ ಪುರುಷನಾಗುವೆ’ ಅಂತ ಹೇಳಿದ. ಈಗ ನೋಡಿ ನಾನು ಏನಾಗಿದ್ದೇನೆ!” ಇದನ್ನು ಹೇಳುತ್ತಿದ್ದಾಗ ಅವರ ಮುಖದಲ್ಲಿ ತುಂಟ ನಗು ಎದ್ದುಕಾಣುತ್ತಿತ್ತು. ಆದರೆ ಇಂತಹ ಮಾತುಗಳಲ್ಲಿ ಅಹಂ ಕಾರ ಲೇಶವೂ ಇರುತ್ತಿರಲಿಲ್ಲ.

ಒಂದು ದಿನ ಸ್ವಾಮೀಜಿ ಕೆಲವು ಸ್ನೇಹಿತರೊಂದಿಗೆ ಚಿತ್ರಕಲಾ ಪ್ರದರ್ಶನವೊಂದಕ್ಕೆ ಹೋದರು. ಅಲ್ಲಿ ಕೆಲವು ದಢೂತಿ ಸಾಧುಗಳ ಚಿತ್ರಗಳನ್ನು ನೋಡುತ್ತ ಅವರೆನ್ನುತ್ತಾರೆ, “ಆಧ್ಯಾತ್ಮಿಕ ವ್ಯಕ್ತಿಗಳು ಯಾವಾಗಲೂ ಸ್ಥೂಲಕಾಯರೇ. ಬೇಕಾದರೆ ನನ್ನನ್ನೇ ನೋಡಿ ಈಗ!”

ಮತ್ತೊಮ್ಮೆ ಭಾಷಣದ ಮಧ್ಯೆ ಹೇಳುತ್ತಾರೆ, “ಕ್ರೈಸ್ತರ ನರಕದ ಕಲ್ಪನೆ ನನ್ನ ಮಟ್ಟಿಗಂತೂ ಸ್ವಲ್ಪವೂ ಭಯಾನಕವಾಗಿಲ್ಲ. ನಾನು ದಾಂತೆಯ ‘ಇನ್​ಫರ್ನೋ’ ಎಂಬ ಪುಸ್ತಕವನ್ನು ಮೂರು ಸಲ ಓದಿದ್ದೇನೆ. ಆದರೆ ನಾನದರಲ್ಲಿ ಭಯಂಕರವಾದದ್ದೇನನ್ನೂ ಕಾಣಲಿಲ್ಲ ಎನ್ನಬೇಕಾಗುತ್ತದೆ. ನಮ್ಮ ಹಿಂದೂ ನರಕಗಳಲ್ಲಾದರೆ ಹಲವಾರು ವಿಧ. ಉದಾಹರಣೆಗೆ ಒಬ್ಬ ಹೊಟ್ಟೆಬಾಕ ಸತ್ತ ಎನ್ನಿ. ಅವನು ನರಕಕ್ಕೆ ಹೋದಾಗ ಅಲ್ಲಿ ಅವನ ಸುತ್ತಲೂ ಅತ್ಯುತ್ತಮವಾದ ತಿಂಡಿತೀರ್ಥಗಳ ರಾಶಿಯನ್ನೇ ಒಡ್ಡಲಾಗುತ್ತದೆ. ಅವನ ಹೊಟ್ಟೆ ಒಂದು ಸಾವಿರ ಮೈಲಿ ಉದ್ದವಿರುತ್ತದೆ. ಆದರೆ ಅವನ ಬಾಯಿ ಮಾತ್ರ ಸೂಜಿ ಮೊನೆಯಷ್ಟು ಮಾತ್ರವೇ! ಆಲೋಚಿಸಿ ನೋಡಿ!”

ಈ ಉಪನ್ಯಾಸದ ವೇಳೆಗೆ ಸಭಾಂಗಣದಲ್ಲಿ ಜನ ಕಿಕ್ಕಿರಿದು ತುಂಬಿಹೋಗಿದ್ದರು. ಜೊತೆಗೆ ಕಿಟಕಿಗಳು ಸಾಕಷ್ಟು ಇಲ್ಲದ್ದರಿಂದ ವಾತಾವರಣವು ಅಹಿತಕರವಾಗುವಷ್ಟು ಬಿಸಿಯಾಗಿತ್ತು. ಆದರೆ ಭಾಷಣ ಮುಗಿಸಿ ಆಚೆ ಬರುತ್ತಿದ್ದಂತೆ ಶೀತಗಾಳಿ ಸುಂಯ್ಯನೆ ಬೀಸಿತು. ತಕ್ಷಣ ಸ್ವಾಮೀಜಿ ತಮ್ಮ ಕೋಟನ್ನು ತಮ್ಮ ಸುತ್ತ ಬಿಗಿಯಾಗಿ ಸುತ್ತಿಕೊಳ್ಳುತ್ತ ಗಟ್ಟಿಯಾಗಿ ಉದ್ಗರಿಸಿದರು, “ಸರಿ ಸರಿ, ಇದು ನರಕವಲ್ಲ ಎನ್ನುವುದಾದರೆ ಮತ್ತೆ ಯಾವುದೋ ನನಗೆ ಗೊತ್ತಿಲ್ಲ.”

ಸಂನ್ಯಾಸಿಯ ಜೀವನವನ್ನು ಗೃಹಸ್ಥನ ಜೀವನದೊಂದಿಗೆ ಹೋಲಿಸುತ್ತ ಸ್ವಾಮೀಜಿ ಒಮ್ಮೆ ಹೇಳಿದರು, “ಯಾರೋ ನನ್ನನ್ನು ಕೇಳಿದರು, ನೀವು... ಹಿಂದೆ ಏನಾದರೂ ಮದುವೆಯಾಗಿ ದ್ದಿರಾ?–ಅಂತ.” ಹೀಗೆ ಹೇಳಿ ಅವರು ಮುಗುಳ್ನಗುತ್ತ ಒಂದು ಕ್ಷಣ ಮಾತನ್ನು ನಿಲ್ಲಿಸಿ ಪ್ರೇಕ್ಷಕರ ಕಡೆಗೆ ದೃಷ್ಟಿ ಹಾಯಿಸಿದರು. ಸಭೆಯಲ್ಲೆಲ್ಲ ಕುತೂಹಲದ ಗುಜುಗುಜು, ನಗು ಕೇಳಿಬಂದಿತು. ಇದ್ದಕ್ಕಿದ್ದಂತೆ ಸ್ವಾಮೀಜಿ ಮುಖದ ಮೇಲೊಂದು ವಿಪರೀತ ಗಾಬರಿಯ ಭಾವವನ್ನು ತಂದು ಕೊಂಡು ನುಡಿದರು, “ಅಯ್ಯಯ್ಯೊ! ನಾನಂತೂ ಎಂದೆಂದಿಗೂ ಮದುವೆಯಾಗಲಾರೆ. ಈ ಮದುವೆಯೊಂದು ದೆವ್ವದ ಆಟ!” ಹೀಗೆ ಹೇಳಿ ತಮ್ಮ ಮಾತುಗಳು ಪೂರ್ಣ ಪ್ರಭಾವ ಬೀರಲು ಮತ್ತೆ ಒಂದು ಕ್ಷಣ ಸುಮ್ಮನಾದರು. ಪ್ರೇಕ್ಷಕರ ಮೆಚ್ಚುಗೆಯ ದನಿ, ಕರತಾಡನ ಕೇಳಿಬರು ತ್ತಿದ್ದಂತೆ ಅದನ್ನು ನಿಲ್ಲಿಸುವಂತೆ ಸೂಚಿಸಿ ತಮ್ಮ ಕೈಯನ್ನು ಮೇಲೆತ್ತಿದರು. ಬಳಿಕ ಒಂದು ಬಗೆಯ ಗಂಭೀರ ಮುಖಮುದ್ರೆಯನ್ನು ತಂದುಕೊಂಡು ಹೇಳಿದರು, “ಆದರೆ, ಈ ಸಂನ್ಯಾಸ ಸಂಪ್ರದಾಯದ ವಿರುದ್ಧ ನನ್ನದು ಒಂದೇ ಒಂದು ಆರೋಪ. ಅದೇನೆಂದರೆ...ಅದು ಸಮಾಜದ ಅತ್ಯಂತ ಯೋಗ್ಯ ವ್ಯಕ್ತಿಗಳನ್ನು ಕಿತ್ತುಕೊಂಡು ಬಿಡುತ್ತದೆ.” ಈ ಮಾತನ್ನು ಕೇಳಿದಾಗ ಸಭೆಯಲ್ಲಿ ಕೋಲಾಹಲ ಉಕ್ಕಿ ಹರಿಯಿತು. ಸ್ವಾಮೀಜಿ ಅದನ್ನು ತಡೆಯುವ ಪ್ರಯತ್ನ ಮಾಡಲಿಲ್ಲ.

ಒಂದು ದಿನ ಖಾಸಗೀ ತರಗತಿಯೊಂದರಲ್ಲಿ ಸ್ವಾಮೀಜಿಯವರು ಅತ್ಯಂತ ಜ್ಞಾನಪ್ರಚೋದಕ ವಾಗಿ ಮಾತನಾಡುತ್ತಿದ್ದರು. ಇದ್ದಕ್ಕಿದ್ದಂತೆ ಭಾವ ತುಂಬಿ ಬಂದು ತಮ್ಮ ಮಾತನ್ನು ಅರ್ಧಕ್ಕೆ ನಿಲ್ಲಿಸಿ ನುಡಿದರು, “ತನ್ನ ಹೆಸರನ್ನೇ ಬರೆಯಲು ಬಾರದಿದ್ದಂತಹ ಒಬ್ಬ ಮನುಷ್ಯನ ಶಿಷ್ಯ ನಾನು. ನಾನು ಅವರ ಪಾದರಕ್ಷೆಗಳನ್ನು ಹಿಡಿಯಲೂ ಕೂಡ ಯೋಗ್ಯನಲ್ಲ. ನನ್ನ ಬುದ್ದಿವಂತಿಕೆಯನ್ನೆಲ್ಲ ತೆಗೆದುಕೊಂಡು ಹೋಗಿ ಗಂಗೆಗೆ ಎಸೆದುಬಿಡಬೇಕು ಎಂದು ನಾನೆಷ್ಟು ಸಲ ಆಶಿಸಿದ್ದೆ!” ಆಗ ಓರ್ವ ಮಹಿಳೆ ಹೇಳಿದಳು, “ಆದರೆ ಸ್ವಾಮೀಜಿ, ನಿಮ್ಮ ವ್ಯಕ್ತಿತ್ವದ ಆ ಅಂಶವನ್ನೇ (ಜ್ಞಾನ ಹಾಗೂ ಬುದ್ಧಿಮತ್ತೆಗಳನ್ನೇ) ನಾನು ಹೆಚ್ಚು ಮೆಚ್ಚಿಕೊಂಡಿರುವುದು!”

ಅದಕ್ಕೆ ಸ್ವಾಮೀಜಿಯವರ ಉತ್ತರ: “ಅದೇಕೆಂದರೆ ನೀವೊಬ್ಬರು ಮೂರ್ಖರು ಮೇಡಂ– ನನ್ನ ಹಾಗೆ!”

ಈ ‘ನನ್ನ ಹಾಗೆ’ ಎಂಬ ಒಂದು ಪ್ರಾಮಾಣಿಕ ಉದ್ಗಾರವೇ ಅವರನ್ನು ಎಲ್ಲರಿಗೂ ಪ್ರಿಯವಾಗಿಸುತ್ತಿದ್ದುದು, ಅವರ ಸ್ನೇಹವನ್ನು ದೃಢವಾಗಿಸುತ್ತಿದ್ದುದು.

ಉಪನ್ಯಾಸವಾದ ಬಳಿಕ ಕೆಲವೊಮ್ಮೆ ಸ್ವಾಮೀಜಿ ಪ್ರಶ್ನೆಗಳನ್ನು ಆಹ್ವಾನಿಸುತ್ತಿದ್ದರು. ಪ್ರಶ್ನಾರ್ಥಿ ಗಳಲ್ಲಿ ಎಲ್ಲ ಬಗೆಯವರೂ ಇರುತ್ತಿದ್ದರು. ಜಿಜ್ಞಾಸುಗಳು, ಕುತೂಹಲಿಗಳು, ವಿರೋಧಿಗಳು, ಪೂರ್ವಗ್ರಹಪೀಡಿತರು, ಭಾರತದ ಬಗ್ಗೆ ಹಾಗೂ ಸಂನ್ಯಾಸದ ಬಗ್ಗೆ ಚಿತ್ರವಿಚಿತ್ರ ಕಲ್ಪನೆಗಳಿರು ವವರು–ಹೀಗೆ ಹಲವಾರು ಬಗೆಯ ಜನ ಹಲವು ರೀತಿಯ ಪ್ರಶ್ನೆಗಳನ್ನು ಹಾಕುತ್ತಿದ್ದರು. ಸ್ವಾಮೀಜಿ ಆ ಎಲ್ಲ ಪ್ರಶ್ನೆಗಳಿಗೂ ಅವುಗಳ ಯೋಗ್ಯತೆಗೆ ತಕ್ಕಂತೆ ಉತ್ತರಿಸುತ್ತಿದ್ದರು.

ಲಾಸ್ ಏಂಜಲಿಸ್​ನಲ್ಲಿ ಒಂದು ಉಪನ್ಯಾಸದ ಬಳಿಕ, ಅವರನ್ನು ಅಮೆರಿಕದ ಆ ತುದಿಯಿಂದ ಈ ತುದಿಯವರೆಗೂ ಹಿಂಬಾಲಿಸಿದ್ದ ಪ್ರಶ್ನೆಯೊಂದು ಬಂದಿತು. ಒಬ್ಬಳು ಮಹಿಳೆ ಎದ್ದುನಿಂತು ಕೇಳಿದಳು, “ಸ್ವಾಮೀಜಿ, ನಿಮ್ಮ ದೇಶದಲ್ಲಿ ನವಜಾತ ಶಿಶುಗಳನ್ನು ಗಂಗೆಗೆ ಎಸೆಯುತ್ತೀರಂತೆ, ನಿಜವೆ?” ಸ್ವಾಮೀಜಿ ಬೇಸರಿಸಿಕೊಳ್ಳದೆ, ಅದಕ್ಕುತ್ತರವಾಗಿ ತಾವೇ ಒಂದು ಪ್ರಶ್ನೆ ಕೇಳಿದರು, “ನಿಮ್ಮ \eng{Thanks giving (}ದೇವರಿಗೆ ಕೃತಜ್ಞತೆಯನ್ನರ್ಪಿಸುವ) ಸಮಾರಂಭಗಳಲ್ಲಿ ನೀವು ನವಜಾತ ಶಿಶುಗಳನ್ನೇ ಅಡಿಗೆ ಮಾಡಿ ಬಡಿಸುತ್ತೀರಿ ಎಂದು ನಾವು ಕೇಳಿದ್ದೇವೆ. ಹೌದೆ?” ಅವಳ ಅರ್ಥವಿಲ್ಲದ ಪ್ರಶ್ನೆಗೆ ಇವರ ಅರ್ಥವಿಲ್ಲದ ಉತ್ತರ.

ಇದೇ ಪ್ರಶ್ನೆಗೆ ಸ್ವಾಮೀಜಿ ಅದೆಷ್ಟು ಸಲ ಉತ್ತರಿಸಬೇಕಾಯಿತೋ ತಿಳಿಯದು. ಈ ಬಗೆಯ ಪ್ರಶ್ನೆಗಳಿಗೆ ಅವರು ಪ್ರತಿಸಲವೂ ಬೇರೆ ಬೇರೆ ಉತ್ತರಗಳನ್ನು ಕೊಡುತ್ತಿದ್ದರೆಂದು ಕಾಣುತ್ತದೆ. ಹಿಂದೊಮ್ಮೆ ಹೀಗೆಯೇ ಒಬ್ಬಳು ಹೆಂಗಸು, “ನಿಮ್ಮಲ್ಲಿ ನವಜಾತ ಶಿಶುಗಳನ್ನೆಲ್ಲ ನದಿಗೆ ಎಸೆದು ಬಿಡುತ್ತಾರಂತಲ್ಲ!” ಎಂದು ಕೇಳಿದಾಗ ಸ್ವಾಮೀಜಿ ಉತ್ತರಿಸಿದ್ದರು, “ಹೌದು ಮೇಡಮ್, ನನ್ನನ್ನೂ ಹಾಗೇ ಎಸೆದುಬಿಟ್ಟಿದ್ದರು. ಆದರೆ ನಾನು ಹೇಗೋ ಮಾಡಿ ತಪ್ಪಿಸಿಕೊಂಡು ಬಂದು ಬಿಟ್ಟೆ, ಸದ್ಯ!” ಇದೇ ಪ್ರಶ್ನೆಗೆ ಇನ್ನೊಂದು ಸಲ ಹೀಗೆ ಉತ್ತರಿಸಿದ್ದರು, “ಹೌದು! ನನ್ನನ್ನೂ ಮೊಸಳೆಯ ಬಾಯಿಗೆ ಎಸೆದಿದ್ದರು. ಆದರೆ ನಾನು ಎಷ್ಟು ದಪ್ಪಗಿದ್ದೆನೆಂದರೆ, ಮೊಸಳೆಗಳು ನನ್ನನ್ನು ನುಂಗಲು ನಿರಾಕರಿಸಿಬಿಟ್ಟುವು! ಈಗಲೂ, ನಾನು ಇಷ್ಟೊಂದು ದಪ್ಪನೆಯ ಸಾಧುವಾ ಗಿರುವುದರ ಬಗ್ಗೆ ನನಗೇ ಒಮ್ಮೊಮ್ಮೆ ನಾಚಿಕೆಯಾಗುತ್ತದೆ. ಹಾಗಾದಾಗಲೆಲ್ಲ, ನಾನು ಹೇಗೆ ಬದುಕುಳಿದೆ ಎಂಬುದನ್ನು ನೆನಪಿಸಿಕೊಂಡು ಸಮಾಧಾನ ತಂದುಕೊಳ್ಳುತ್ತೇನೆ.” ಇನ್ನೊಬ್ಬರು ಕೇಳಿದ್ದರು: “ಭಾರತದಲ್ಲಿ ಹೆಣ್ಣುಮಕ್ಕಳನ್ನೆಲ್ಲ ಮೊಸಳೆಯ ಬಾಯಿಗೆ ಕೊಟ್ಟುಬಿಡುತ್ತಾರಂತೆ ನಿಜವೆ?” ಸ್ವಾಮೀಜಿ ಸಹಜವಾಗಿ ಉತ್ತರಿಸಿದ್ದರು, “ಹೌದು ಹೌದು; ಈಚೆಗೆ ಮಕ್ಕಳು ಹುಟ್ಟು ತ್ತಿರುವುದೆಲ್ಲ ಗಂಡಸರಿಗೇ!” ಡೆಟ್ರಾಯ್ಟಿನಲ್ಲಿ ಒಬ್ಬರಿಗೆ ಇನ್ನೊಂದು ಸಂದೇಹ ಬಂದಿತ್ತು: “ಸ್ವಾಮೀಜಿ, ನಿಮ್ಮ ದೇಶದಲ್ಲಿ ಹೆಣ್ಣು ಮಕ್ಕಳನ್ನು ಮಾತ್ರವೇ ಮೊಸಳೆಗೆ ಎಸೆಯುತ್ತಾರಂತಲ್ಲ, ಅದೇಕೆ?” “ಓ ಅದೇ! ಬಹುಶಃ ಹೆಣ್ಣು ಮಕ್ಕಳು ಹೆಚ್ಚು ನಾಜೂಕಾಗಿ ಮೃದುವಾಗಿ ಇರುತ್ತವೆ, ಮೊಸಳೆಗಳಿಗೆ ಅಗಿಯುವುದಕ್ಕೂ ಸುಲಭ ಎನ್ನುವ ಕಾರಣಕ್ಕಾಗಿ ಇರಬಹುದು.”

ಹೀಗೆ, ಇಂತಹ ಪ್ರಶ್ನೆಯನ್ನು ಕೇಳುವವರ ಮೌಢ್ಯವನ್ನು ಸ್ವಾಮೀಜಿ ಅವರಿಗೇ ತೋರಿಸಿಕೊಡುತ್ತಿದ್ದರು. ಲಾಸ್ ಏಂಜಲಿಸ್​ನಲ್ಲಿ ಈ ಬಗೆಯ ಪ್ರಶ್ನೆಯೊಂದನ್ನು ಕೇಳಿದ ಒಬ್ಬಳಿಗೆ ಸ್ವಾಮೀಜಿ ಇಂಥದೇ ಉತ್ತರ ಕೊಟ್ಟಾಗ, ಆಕೆ ಅವಮಾನವನ್ನು ತಡೆಯಲಾರದೆ ಕುರ್ಚಿಯ ಹಿಂದೆ ತನ್ನ ತಲೆಯನ್ನು ಹುದುಗಿಸಿಕೊಂಡಳು. ಆಗ ಸ್ವಾಮೀಜಿ ಮತ್ತೆ ಚುಚ್ಚಿದರು, “ಪಾಪ! ನಾನು ನಿಮ್ಮನ್ನು ದೂಷಿಸುವುದಿಲ್ಲ. ಅಂತಹ ಪ್ರಶ್ನೆಯನ್ನು ನಾನೇ ಕೇಳಿ ನನಗೇ ಅಂತಹ ಉತ್ತರ ಸಿಕ್ಕಿದ್ದರೆ, ನಾನಾದರೂ ಹಾಗೆಯೇ ಬಚ್ಚಿಟ್ಟುಕೊಳ್ಳುತ್ತಿದ್ದೆ!”

ಒಂದು ಭಾಷಣದ ಸಂದರ್ಭದಲ್ಲಿ ಸ್ವಾಮೀಜಿಯವರು ಹಿಂದೂಧರ್ಮದ ವಿವಿಧ ಅಂಶಗಳ ಬಗ್ಗೆ ಮಾತನಾಡುತ್ತ, ಯೋಗದ ಬಗ್ಗೆ ಪ್ರಸ್ತಾಪಿಸಿದರು. ಆದರೆ ಅಂದು ಅವರು ಯೋಗಗಳ ರೀತಿನೀತಿಗಳ ಬಗ್ಗೆ ವಿವರವಾಗಿ ತಿಳಿಸಲಿಲ್ಲ. ಸಭಿಕರ ಪೈಕಿ ಒಬ್ಬಳಿಗೆ ಯೋಗ, ಪ್ರಾಣಾಯಾಮ, ಸಮಾಧಿ–ಇವುಗಳ ಬಗ್ಗೆ ತುಂಬ ಕುತೂಹಲ, ಅವುಗಳ ಬಗ್ಗೆ ಮಾತನಾಡುವುದು ಆಗಿನ ಒಂದು ಫ್ಯಾಷನ್ನೇ ಆಗಿತ್ತು. ಆದ್ದರಿಂದ, ಉಪನ್ಯಾಸವಾದ ಮೇಲೆ ಆ ಮಹಿಳೆ ಎದ್ದುನಿಂತು ಕೇಳಿದಳು:

“ಸ್ವಾಮೀಜಿ, ತಿನ್ನುವುದು ಮತ್ತು ಉಸಿರಾಡುವುದು—ಇವುಗಳ ಬಗ್ಗೆ ನೀವೇನು ಹೇಳುತ್ತೀರಿ?”

ಆಹಾರಾದಿಗಳ ಹಾಗೂ ಪ್ರಾಣಾಯಾಮದ ವಿಷಯದಲ್ಲಿ ಅವರ ಅಭಿಪ್ರಾಯವೇನೆಂದು ತಿಳಿದುಕೊಳ್ಳುವುದು ಆಕೆಯ ಉದ್ದೇಶ. ಆದರೆ ಆ ಪ್ರಶ್ನೆಗೆ ಸ್ವಾಮೀಜಿ ತಕ್ಷಣ ಉತ್ತರಿಸಿದ್ದು ಹೀಗೆ–

“ನಾನು ನಿಮಗೆ ಆಶ್ವಾಸನೆ ನೀಡುತ್ತೇನೆ ಮೇಡಮ್, ನಾನು ಅವೆರಡನ್ನೂ (ತಿನ್ನುವುದನ್ನು ಮತ್ತು ಉಸಿರಾಡುವುದನ್ನು) ಬೆಂಬಲಿಸುವವನು!”

ಸ್ವಾಮೀಜಿಯವರು ತಮ್ಮೊಡನೆ ವಾಸವಾಗಿದ್ದ ಆಪ್ತ ಶಿಷ್ಯರೊಂದಿಗೆ ಅತ್ಯಂತ ಆತ್ಮೀಯತೆ ಯಿಂದ ವರ್ತಿಸುತ್ತಿದ್ದರು. ಕೆಲವೊಮ್ಮೆ ಅತಿ ರಹಸ್ಯ ವಿಚಾರಗಳನ್ನೂ ಅವರಿಗೆ ಹೇಳಿಬಿಡುತ್ತಿ ದ್ದರು. ಒಂದು ದಿನ ಅವರು ತಾವೇ ಅಡಿಗೆ ಮಾಡುತ್ತಿದ್ದರು; ಬಳಿಯಲ್ಲೇ ಶ್ರೀಮತಿ ಅಲನ್ ಕೂಡ ಇದ್ದಳು. ಆಗ ಅವರು, ಭಗವದ್ಗೀತೆಯ ಶ್ಲೋಕವೊಂದನ್ನು (೧೮-೬೧) ಸುಶ್ರಾವ್ಯವಾಗಿ ಹೇಳಿ ಅರ್ಥವನ್ನು ವಿವರಿಸಿದರು. ಬಳಿಕ ನುಡಿದರು, “ಜೀವನವೆಂಬುದು ದಾಳವನ್ನು ಎಸೆದಂತೆ. ಭಗವಂತನು ನಮ್ಮನ್ನು ಮತ್ತೆಮತ್ತೆ ಜೀವನಚಕ್ರದ ಮೇಲೆ ಕುಳ್ಳಿರಿಸುತ್ತಾನೆ... ಶ್ರೀ ಗುರು ಮಹಾರಾಜರು ಹೇಳಿದ್ದಾರೆ, ನಾನು ಅವರೊಂದಿಗೆ ಮತ್ತೆ ಹುಟ್ಟಿಬರಬೇಕಾಗುತ್ತದೆ ಎಂದು.” ಶ್ರೀಮತಿ ಅಲನ್ ಕೇಳಿದಳು, “ಶ್ರೀರಾಮಕೃಷ್ಣರು ಹೇಳಿದರು ಎಂದ ಮಾತ್ರಕ್ಕೆ ನೀವು ಮತ್ತೆ ಬರಬೇಕಾಗುತ್ತದೆಯೆ?” “ಅಂತಹ ವ್ಯಕ್ತಿಗಳಿಗೆ ಮಹಾ ಶಕ್ತಿಯಿರುತ್ತದೆ.”

ಪಸಾಡೆನದಲ್ಲಿಯೂ ಒಮ್ಮೆ ಅವರು ಹೀಗೆಯೇ ಹೇಳಿದ್ದರು. ಇದರ ಕುರಿತಾಗಿ ಶ್ರೀಮತಿ ಹ್ಯಾನ್ಸ್​ಬ್ರೋ ಹೇಳುತ್ತಾಳೆ, “ಊಟವಾದ ಬಳಿಕ ಸ್ವಾಮೀಜಿಯವರು ಕೋಣೆಯಲ್ಲಿ ಅತ್ತಿಂದಿತ್ತ ಓಡಾಡುತ್ತ ಅಲ್ಲಿದ್ದ ತಮ್ಮ ಆಪ್ತ ಶಿಷ್ಯರ ಮುಂದೆ ನುಡಿದರು–‘ತಾವು ಸುಮಾರು ಮುನ್ನೂರು ವರ್ಷಗಳಲ್ಲಿ ಮತ್ತೆ ಬರುವುದಾಗಿ ಶ್ರೀರಾಮಕೃಷ್ಣರು ತಿಳಿಸಿದ್ದರು; ಆಗ ಅವರೊಂದಿಗೆ ನಾನೂ ಬರುತ್ತೇನೆ. ಒಬ್ಬ ಮಹಾಗುರು ಅವತರಿಸಿ ಬಂದಾಗ ತನ್ನವರನ್ನೂ ಜೊತೆಗೆ ಕರೆತರುತ್ತಾನೆ’ ಎಂದು.”

ಸ್ವಾಮೀಜಿಯವರನ್ನು ಅತ್ಯಂತ ಸಮೀಪದಿಂದ ಕಂಡಿದ್ದ ಶ್ರೀಮತಿ ಎಡಿತ್ ಅಲನ್ ಬರೆಯು ತ್ತಾಳೆ: “ಅವರದು ಬಹುಮುಖ ವ್ಯಕ್ತಿತ್ವ; ವರ್ಣನೆಯನ್ನು ಮೀರಿದ್ದು. ಅವರು ನಾಲ್ಕು ಯೋಗ ಗಳ ಮೂರ್ತರೂಪ. ಕೆಲವೊಮ್ಮೆ ಅವರೊಬ್ಬ ವೇದಾಂತ ಕೇಸರಿ, ಮತ್ತೆ ಕೆಲವೊಮ್ಮೆ ಒಂದು ಶಿಶು. ನನಗಂತೂ ಅವರು ಯಾವಾಗಲೂ ಸಹನಾಪೂರ್ಣ, ಪ್ರೀತಿಯುತ ತಂದೆ. ನಮ್ಮ ಕಲ್ಪನೆಗೂ ಮೀರಿದ ಪ್ರೀತಿ ಅವರದ್ದು.”

ಸ್ವಾಮೀಜಿಯವರ ಜೊತೆಯಲ್ಲಿದ್ದವರಿಗಂತೂ ಕ್ಷಣಕ್ಷಣಕ್ಕೂ ಶಿಕ್ಷಣ ಸಿಗುತ್ತಿತ್ತು. ಒಮ್ಮೆ ಅವರು ಶ್ರೀಮತಿ ಅಲನ್​ಗೆ ಹೇಳಿದರು, “ಜೋಲು ಮೋರೆಯಿರುವ ಯಾರನ್ನಾದರೂ ನೋಡಿದರೆ ನೀನು ತಿಳಿದುಕೊ–ಆತನಲ್ಲಿರುವುದು ಧರ್ಮವಲ್ಲ, ಆದರೆ ಬಹುಶಃ ಆತನಿಗೆ ಹೊಟ್ಟೆನೋವು ಇರಬಹುದು ಅಂತ.” ಮತ್ತೊಮ್ಮೆ ಅಡಿಗೆ ಮಾಡುತ್ತ ಉದ್ಗರಿಸುತ್ತಾರೆ, “ನನ್ನನ್ನು ನಾನೊಂದು ಇರುವೆಗಿಂತಲೂ ಹೆಚ್ಚಿನವನೆಂದು ಭಾವಿಸುವುದಾದರೆ, ನನಗೆ ಏನೂ ತಿಳಿವಳಿಕೆಯಿಲ್ಲವೆಂದರ್ಥ... ನಾನು ಉನ್ನತ ಭಾವದಲ್ಲಿದ್ದಾಗ ‘ಶಿವೋ\eng{s}ಹಂ, ಶಿವೋ\eng{s}ಹಂ’ ಎನ್ನುತ್ತೇನೆ. ಹೊಟ್ಟೆನೋವು ಬಂದಾಗ ಹೇಳುತ್ತೇನೆ, ‘ಅಮ್ಮಾ, ನನ್ನ ಮೇಲೆ ಕರುಣೆಯಿಡು!’ ಎಂದು.”

ಶ್ರೀಮತಿ ಹ್ಯಾನ್ಸ್​ಬ್ರೋ ಯಾವುದಾದರೊಂದು ವಿಷಯದ ಬಗ್ಗೆ ಸ್ವಾಮೀಜಿಯವರೊಂದಿಗೆ ಆಗಾಗ ವಾಗ್ವಾದ ನಡೆಸುತ್ತಿದ್ದಳು. ಅವರ ಕಾರ್ಯವಿಧಾನದ ಬಗ್ಗೆ ಮತ್ತೆ ಮತ್ತೆ ಪ್ರಶ್ನಿಸುತ್ತಿ ದ್ದಳು. ಆದರೆ, ಅವಳು ಕೆಲಸ ಮಾಡುವ ರೀತಿಯನ್ನು ಸ್ವಾಮೀಜಿಯವರೇ ಎಷ್ಟೋ ಸಲ ಒಪ್ಪುತ್ತಿರಲಿಲ್ಲ. ಅವರ ದೃಷ್ಟಿಗೆ ಯಾವ ಕೆಲಸವೂ ಸಾಧಾರಣದ್ದು ಎನ್ನುವಂಥದ್ದಾಗಿರಲಿಲ್ಲ. ಸ್ನಾನದ ತೊಟ್ಟಿಯನ್ನು ಶುಚಿಗೊಳಿಸುವಂತಹ ಒಂದು ಸಣ್ಣ ಕೆಲಸದಲ್ಲಿಯೂ ಅವರೊಂದು ‘ಗತ್ತ’ನ್ನು ಅಥವಾ ಒಂದು ಬಗೆಯ ಗೌರವಭಾವವನ್ನು ನಿರೀಕ್ಷಿಸುತ್ತಿದ್ದರು.

ಟರ್ಕ್​ಸ್ಟ್ರೀಟಿನ ಮನೆಯಲ್ಲಿದ್ದಾಗ ಶ್ರೀಮತಿ ಹ್ಯಾನ್ಸ್​ಬ್ರೋ ಯಾವಾಗಲೂ ಸ್ವಾಮೀಜಿಯವರ ಟೀಕೆಗೆ, ಬೈಗುಳಕ್ಕೆ ಗುರಿಯಾಗುತ್ತಿದ್ದಳು. ಅವಳ ಕೆಲಸಗಳಲ್ಲಿ ಅವರು ಒಂದಲ್ಲ ಒಂದು ತಪ್ಪನ್ನು ಕಂಡುಹಿಡಿಯುತ್ತಲೇ ಇದ್ದರು. ಒಮ್ಮೊಮ್ಮೆ ತುಂಬ ಸಿಟ್ಟು ಬಂದಾಗ ಹೇಳುತ್ತಿದ್ದರು, “ನನ್ನೊಂ ದಿಗೆ ಕೆಲಸ ಮಾಡಲು ಇಂತಹ ಮೂರ್ಖರನ್ನು ಕಳಿಸಿಕೊಡುತ್ತಾಳೆ ಜಗನ್ಮಾತೆ!” ಆದರೆ ಆಕೆ ಇದಾವುದನ್ನೂ ಮನಸ್ಸಿಗೆ ಹಚ್ಚಿಕೊಳ್ಳುವವಳಲ್ಲ. ಒಮ್ಮೆ ಅವರು ಅವಳನ್ನು ಹೀಗೆಯೇ ತರಾಟೆಗೆ ತೆಗೆದುಕೊಂಡು ಚೆನ್ನಾಗಿ ಬೈಯುತ್ತಿದ್ದರು; ಆಗ ಶ್ರೀಮತಿ ಆ್ಯಸ್ಪಿನಾಲ್ ಎಂಬ ಶಿಷ್ಯೆ ಅಲ್ಲಿಗೆ ಬಂದುಬಿಟ್ಟಳು. ತಕ್ಷಣ ಸ್ವಾಮೀಜಿ ಸುಮ್ಮನಾಗಿಬಿಟ್ಟರು. ಆಗ ಶ್ರೀಮತಿ ಹ್ಯಾನ್ಸ್​ಬ್ರೋ ಹೇಳು ತ್ತಾಳೆ, “ಅವಳನ್ನೇನೂ ನೀವು ಲಕ್ಷಿಸಬೇಕಾಗಿಲ್ಲ, ಸ್ವಾಮೀಜಿ, ನಿಮ್ಮದಿನ್ನೂ ಮುಗಿದಿಲ್ಲದಿದ್ದರೆ ಮುಂದುವರಿಸಬಹುದು!”

ಸ್ವಾಮೀಜಿಯವರು ಆಗಾಗ ಹೇಳುತ್ತಿದ್ದರು, ‘ನಾನೆಂದೂ ಕ್ಷಮೆ ಕೇಳುವುದಿಲ್ಲ’ ಎಂದು. ಆದರೆ ಅವಳನ್ನು ಒಮ್ಮೆ ಚೆನ್ನಾಗಿ ಬೈದರೂ ಬಳಿಕ ತಾವಾಗಿಯೇ ಮೆಲ್ಲನೆ ಬಳಿ ಸಾರಿ, ಮೃದುವಾದ ದನಿಯಲ್ಲಿ ಕೇಳುತ್ತಿದ್ದರು–“ಏನ್ ಮಾಡ್ತಿದ್ದೀ... ” ಇದು ರಾಜಿ ಮಾಡಿಕೊಳ್ಳುವ ಪ್ರಯತ್ನ ವೆಂಬುದಂತೂ ಸ್ಪಷ್ಟ. ಅವರು ಆಗಾಗ ಹೇಳುತ್ತಿದ್ದರು, “ನಾನು ಯಾರನ್ನು ಅತಿ ಹೆಚ್ಚು ಪ್ರೀತಿಸುತ್ತೇನೆಯೋ ಅವರನ್ನು ಅತಿ ಹೆಚ್ಚು ಬೈಯುತ್ತೇನೆ” ಎಂದು. ಅದನ್ನು ಕೇಳಿ ಶ್ರೀಮತಿ ಹ್ಯಾನ್ಸ್​ಬ್ರೋ ಮನಸ್ಸಿನಲ್ಲೇ ‘ಇದೊಳ್ಳೆ ಚೆನ್ನಾಯಿತಲ್ಲ!’ ಎಂದುಕೊಳ್ಳುತ್ತಿದ್ದಳಂತೆ.

ಆಕೆಯನ್ನು ಸ್ವಾಮೀಜಿ ಅಷ್ಟೆಲ್ಲ ಬೈಯುತ್ತಿದ್ದರೂ ಅವಳ ಸೇವಾನಿಷ್ಠೆಯನ್ನು ಅವರು ಹೃತ್ಪೂರ್ ವಕವಾಗಿ ಮೆಚ್ಚಿಕೊಂಡಿದ್ದರು. ಬೆಸ್ಸಿ ಲೆಗೆಟ್​ಳಿಗೆ ಒಂದು ಪತ್ರದಲ್ಲಿ ಅವರು ಬರೆಯುತ್ತಾರೆ, “ಮೀಡ್ ಸಹೋದರಿಯರಲ್ಲಿ ಎರಡನೆಯವಳಾದ ಶ್ರೀಮತಿ ಹ್ಯಾನ್ಸ್​ಬ್ರೋ ಇಲ್ಲಿದ್ದಾಳೆ. ಅವಳು ಯಾವಾಗಲೂ ಮಾಡುತ್ತಿರುವುದೊಂದೇ–ಕೆಲಸ, ಕೆಲಸ, ಕೆಲಸ. ಭಗವಂತ ಅವಳನ್ನು ಚೆನ್ನಾ ಗಿಟ್ಟಿರಲಿ. ಆ ಮೂವರು ಸೋದರಿಯರೂ ಮೂವರು ದೇವತೆಗಳು, ಅಲ್ಲವೆ? ಅಲ್ಲಲ್ಲಿ ಇಂತಹ ಕೆಲವು ಧನ್ಯಾತ್ಮರನ್ನು ನೋಡಿದಾಗ ಈ ಬದುಕಿನ ಅರ್ಥಹೀನತೆಯೆಲ್ಲ ಮಾಯವಾಗಿ ಅದ ಕ್ಕೊಂದು ಅರ್ಥ ಬರುತ್ತದೆ.”

ಶ್ರೀಮತಿ ಹ್ಯಾನ್ಸ್​ಬ್ರೋಳಿಗೂ ಗೊತ್ತಿತ್ತು–ಸ್ವಾಮೀಜಿ ತನ್ನನ್ನು ಎಷ್ಟೇ ಬೈದರೂ ಅವರು ಕೃಪೆಯ ಮೂರ್ತರೂಪವೇ ಆಗಿದ್ದಾರೆ ಎಂದು. ಅವರು ಅಷ್ಟೊಂದು ಕೃಪಾಮಯರಾಗಿರ ದಿದ್ದಲ್ಲಿ, ತನ್ನೊಂದಿಗೆ ಅವರು ಅಷ್ಟೊಂದು ದಿನ ಇರುತ್ತಲೇ ಇರಲಿಲ್ಲ ಎಂದು ಮುಂದೊಮ್ಮೆ ಅವಳೇ ಹೇಳುತ್ತಾಳೆ. ಪಸಾಡೆನದವಳಾದ ಈಕೆ ತನ್ನ ಪುಟ್ಟ ಮಗು ಡೊರೋತಿಯನ್ನು ಬಿಟ್ಟು ಸ್ವಾಮೀಜಿಯವರಿಗಾಗಿ, ಅವರ ಸೇವೆಗಾಗಿ ಸ್ಯಾನ್​ಫ್ರಾನ್ಸಿಸ್ಕೋಗೆ ಬಂದುಬಿಟ್ಟಿದ್ದಳು. ಒಂದು ದಿನ ಆಕೆಗೆ ಮಗುವಿನ ಮೇಲಿನ ಸೆಳೆತ ತೀವ್ರವಾಗಿ, ಪಸಾಡೆನಕ್ಕೆ ಹಿಂದಿರುಗುವುದೆಂದು ನಿರ್ಧರಿಸಿದಳು. ಅದರಂತೆಯೇ ಮರುದಿನ ತನ್ನ ಬಟ್ಟೆ ಬರೆಗಳನ್ನೆಲ್ಲ ತುಂಬಿಕೊಂಡು ಹೊರಡಲು ಸಿದ್ಧಳಾದಳು. ಆಗ ಇದ್ದಕ್ಕಿದ್ದಂತೆ ಆಕೆಗೊಂದು ದನಿ ಕೇಳಿಸಿತು–‘ನೀನು ಹೋಗಲಾರೆ. ನೀನು ಪ್ರಯತ್ನಿಸದಿರುವುದೇ ಒಳ್ಳೆಯದು!’ ತಕ್ಷಣ ಆಗ ಒಂದು ಬಗೆಯ ಆಯಾಸದಿಂದ ನೆಲದ ಮೇಲೆ ಕುಸಿದು ಕುಳಿತಳು. ಏನಾದರೂ ಸ್ವಲ್ಪ ಆಹಾರ ತೆಗೆದುಕೊಳ್ಳುವ ಮನಸ್ಸಾಯಿತು; ಆದರೆ ಅಲುಗಾಡಲೂ ಸಾಧ್ಯವಾಗಲಿಲ್ಲ. ತನ್ನ ಸೂಟ್​ಕೇಸಿನ ಕಡೆಗೆ ನೋಡಲೂ ಅವಳಿಗೆ ಸಾಧ್ಯವಾಗ ಲಿಲ್ಲ. ಈಗ ಆಕೆ ತನ್ನ ನಿರ್ಧಾರವನ್ನು ಬದಲಿಸಬೇಕಾಯಿತು. ಆದರೆ ಅವಳಿಗೆ ಕೇಳಿಸಿದ್ದ ಆ ಧ್ವನಿ ಸ್ವಾಮೀಜಿಯವರದಲ್ಲ. ಅವರು ಏನೂ ಮಾತನಾಡಲೇ ಇಲ್ಲವೆಂಬುದು ಸ್ಪಷ್ಟವಾಗಿತ್ತು. ಹಾಗಾ ದರೆ ಅದು ಯಾರ ಧ್ವನಿ...? ಆಕೆಗದು ಗೊತ್ತಾಗಲೇ ಇಲ್ಲ.

ಆದರೆ ಶ್ರೀಮತಿ ಹ್ಯಾನ್ಸ್​ಬ್ರೋಳಿಗೆ ಮಗಳ ಮೇಲಿದ್ದ ಮೋಹವನ್ನು ಮಾತ್ರ ಸ್ವಾಮೀಜಿ ಯವರು ಸ್ವಲ್ಪ ಖಾರವಾಗಿಯೇ ಖಂಡಿಸಿದರು: “ನೀನು ತಿಳಿದಿರುವೆ–ನಿನ್ನ ಮಗಳನ್ನು ನೀನು ಪ್ರೀತಿಸುತ್ತಿದ್ದೀಯೆಂದು. ಆದರೆ ಅದು ಪ್ರೀತಿಯೇ ಅಲ್ಲ! ಅದು ತಾಯಿಕೋಳಿಗೆ ತನ್ನ ಮರಿಗಳ ಮೇಲಿರುವಂತಹ ಮಮತೆ ಅಷ್ಟೆ. ತಾಯಿಕೋಳಿ ತನ್ನ ಮರಿಗಳಿಗೆ ಆಹಾರ ತಿನ್ನಿಸುವುದಕ್ಕಾಗಿ ದಿನವಿಡೀ ನೆಲ ಕೆರೆಯುತ್ತದೆ. ಆದರೆ ಒಂದು ಅಪರಿಚಿತ ಕೋಳಿ ಮರಿ ಬರಲಿ! ಆಗ ಅದು ಏನು ಮಾಡುತ್ತದೆ ನೋಡು.”

ಸ್ವಾಮೀಜಿಯವರ ವ್ಯಕ್ತಿತ್ವದ ಮೇಲೆ ಬೆಳಕು ಬೀರುವಂತಹ ಘಟನೆಯೊಂದನ್ನು ಥಾಮಸ್ ಅಲನ್ ಹೇಳುತ್ತಾನೆ: ಸ್ಯಾನ್​ಫ್ರಾನ್ಸಿಸ್ಕೋದ ಹಡಗು ಕಟ್ಟೆಯಲ್ಲಿ ಒಂದು ಹೊಸದಾಗಿ ಕಟ್ಟಲ್ಪಟ್ಟ ಹಡಗನ್ನು ನೀರಿಗಿಳಿಸುತ್ತಾರೆ ಎಂಬ ವಿಷಯ ಸ್ವಾಮೀಜಿಯವರಿಗೆ ತಿಳಿದುಬಂತು. ಅವರು ಆ ದೃಶ್ಯವನ್ನು ಹಿಂದೆಂದೂ ನೋಡಿರಲಿಲ್ಲವಾದ್ದರಿಂದ ಅದನ್ನು ನೋಡಬೇಕೆಂದು ಆಸೆಪಟ್ಟರು. ಸ್ವಾಮೀಜಿ ಹಾಗೂ ಅವರ ಅನೇಕ ಶಿಷ್ಯರಿಗೆ ಅಲನ್ ಪಾಸ್​ಗಳನ್ನು ದೊರಕಿಸಿಕೊಟ್ಟ. ಈ ಪಾಸ್​ಗಳನ್ನು ಹೊಂದಿರುವವರಿಗೆ ಹಡಗುಕಟ್ಟೆಯಲ್ಲೊಂದು ಜಾಗ ಮಾಡಲಾಗಿತ್ತು. ಆದರೆ ಅದು ಹಡಗಿನಿಂದ ತುಂಬ ದೂರ. ಅಲ್ಲಿನ ಇನ್ನೊಂದು ವಿಶೇಷ ಕಟಕಟೆಯ ಮೇಲೆ ನಿಂತು ಕೊಂಡರೆ ಎಲ್ಲವನ್ನೂ ಸ್ಪಷ್ಟವಾಗಿ ನೋಡಬಹುದು ಎಂಬುದು ಸ್ವಾಮೀಜಿಯವರಿಗೆ ಕಂಡು ಬಂದಿತು. ಆದರೆ ಕೆಲವೇ ಮಂದಿ ಪ್ರಮುಖ ಆಹ್ವಾನಿತರಿಗೆ ಬಿಟ್ಟರೆ ಉಳಿದವರಿಗೆ ಅಲ್ಲಿ ಪ್ರವೇಶ ವಿರಲಿಲ್ಲ. ಅಲ್ಲಿಗೆ ಹೋಗಿ ನಿಂತುಕೊಳ್ಳುವ ಇಚ್ಛೆಯನ್ನು ಸ್ವಾಮೀಜಿ ವ್ಯಕ್ತಪಡಿಸಿದಾಗ, ಅದು ಸಾಧ್ಯವಾಗುವುದಿಲ್ಲ ಎಂದು ಇತರರು ಹೇಳಿದರು. ಅಲ್ಲದೆ, ಆ ಜಾಗದಲ್ಲಿ ಇಬ್ಬರು ಕಾವಲು ಗಾರರು ಬೇರೆ ನಿಂತಿದ್ದರು. ಆದರೂ ಸ್ವಾಮೀಜಿ ಅಲ್ಲಿಗೆ ಹೋದರು; ಇತರ ಆಹ್ವಾನಿತರೊಂದಿಗೆ ತಾವೂ ರಾಜಾರೋಷವಾಗಿ ಒಳಗೆ ನಡೆದೇಬಿಟ್ಟರು!ಬಾಗಿಲಲ್ಲಿ ನಿಂತಿದ್ದ ಪೇದೆಗಳಿಬ್ಬರೂ ಅವರನ್ನು ತಡೆಯುವ ಪ್ರಯತ್ನವನ್ನೇ ಮಾಡದೆ ಬೆಪ್ಪಾಗಿ ನಿಂತಿದ್ದರು! ಹಡಗನ್ನು ನೀರಿಗಿಳಿಸುವ ಕಾರ್ಯಕ್ರಮವಷ್ಟನ್ನೂ ವೀಕ್ಷಿಸಿದ ಸ್ವಾಮೀಜಿಯವರು ಹಿಂದಿರುಗಿ ಬಂದು ಸ್ನೇಹಿತರನ್ನು ಕೂಡಿ ಕೊಂಡರು. (ಬಾಲಕ ನರೇಂದ್ರ ಕಲ್ಕತ್ತದ ಬಂದರಿನಲ್ಲಿ ಹಡಗನ್ನು ನೋಡಲು ಹೋದ ಸಾಹಸ ನೆನಪಿಗೆ ಬರುತ್ತದೆಯಲ್ಲವೆ?)

ಸ್ವಾಮೀಜಿಯವರಿಗೆ ಸಂಬಂಧಿಸಿದ ಪ್ರತಿಯೊಂದು ಅಂಶವೂ ಅವರ ವ್ಯಕ್ತಿತ್ವದ ಮಾಹಾತ್ಮ್ಯ ವನ್ನು ಪ್ರತಿಬಿಂಬಿಸುವಂತಿತ್ತು. ಅವರ ವ್ಯಕ್ತಿತ್ವವು ಅವರ ಬೋಧನೆಗಳಷ್ಟೇ ಜ್ಞಾನಪ್ರದವಾಗಿತ್ತು. ಈ ಬಗ್ಗೆ ಅವರ ಶಿಷ್ಯೆ ಕ್ರಿಸ್ಟೀನ ಗ್ರೀನ್ ಸ್ಟೈಡಲ್ (ಮುಂದೆ ‘ಸೋದರಿ ಕ್ರಿಸ್ಟೀನ’) ಹೇಳುತ್ತಾಳೆ: “ಆಧ್ಯಾತ್ಮಿಕತೆಯ ಬಗೆಗಿನ ನಮ್ಮ ಕಲ್ಪನೆಯು ಸ್ವಾಮೀಜಿಯವರ ಸಂಪರ್ಕದಿಂದ ಸ್ಫುಟಗೊಂಡಿತು. ಅಷ್ಟೇ ಅಲ್ಲ, ಅದು ಸಾಮಾನ್ಯಮಟ್ಟವನ್ನು ಮೀರಿ ತುಂಬ ಮೇಲೇರಿತು. ಆಧ್ಯಾತ್ಮಿಕತೆ ಯೆಂಬುದು ನಮಗೆ ಜೀವ, ಶಕ್ತಿ, ಸಂತೋಷ ಕಾವು, ಪ್ರಕಾಶ, ಉತ್ಸಾಹ ಹಾಗೂ ಸುಂದರ ಸುಮಧುರ ಗುಣಗಳನ್ನೇ ತಂದುಕೊಡುತ್ತದೆ; ಸೋಮಾರಿತನ, ದೌರ್ಬಲ್ಯ, ಮಂಕುತನಗಳನ್ನಲ್ಲ. ಹಾಗಿರುವಾಗ, ಅಸಾಧಾರಣ ಶಕ್ತಿವಂತನಾದ ಒಬ್ಬ ದೇವಮಾನವನನ್ನು ಕಂಡಾಗ ಆಶ್ಚರ್ಯ ಪಡುವಂಥದೇನು? ಅದೇಕೆ ನಾವು ಪಾಶ್ಚಾತ್ಯರು ಯಾವಾಗಲೂ ನಿಸ್ತೇಜತೆಯನ್ನೂ ರಕ್ತ ಹೀನತೆಯನ್ನೂ ಆಧ್ಯಾತ್ಮಿಕತೆಯೊಂದಿಗೆ ಗಂಟುಹಾಕಿದ್ದೇವೆ? ಈಗ ಅದರ ಕುರಿತು ಆಲೋಚಿಸಿ ದರೆ, ನಾವು ಅಷ್ಟು ಅಸಂಬದ್ಧರಾಗಿರಲು ಅದು ಹೇಗೆ ಸಾಧ್ಯವಾಯಿತು ಎಂದು ಆಶ್ಚರ್ಯವಾಗುತ್ತದೆ.”

ನಿಜಕ್ಕೂ ಸ್ವಾಮೀಜಿಯವರನ್ನು, ಅವರ ವ್ಯಕ್ತಿತ್ವವನ್ನು ಎದ್ದು ಕಾಣುವಂತೆ ಮಾಡುತ್ತಿದ್ದುದೇ ಈ ದಿವ್ಯ ಗುಣಗಳು, ಈ ದಿವ್ಯೋತ್ಸಾಹ! ಅವರ ದೃಷ್ಟಿಕೋನದ ಮೂಲಕ ನೋಡಿದಾಗ ಕಾಣುವ ಭಗವಂತನ ಅಸದೃಶ ಆಕರ್ಷಣೆಯು ಪಾಶ್ಚಾತ್ಯರಿಗೊಂದು ಅಚ್ಚರಿಯ ಮೂಲವಾಗಿತ್ತು. ಅವರ ನಿತ್ಯನೂತನ ಭಗವದಾನಂದಮಯ ವ್ಯಕ್ತಿತ್ವವೇ ಅವರು ಬೋಧಿಸಿದ ಅತಿ ಮುಖ್ಯ ಅಂಶಗಳ ಲ್ಲೊಂದಾಗಿತ್ತು. ಅವರ ಈ ದೈವೀ ಶಕ್ತಿಯೇ ಅವರ ಪ್ರತಿಯೊಂದು ಮಾತನ್ನೂ ಒಂದೊಂದು ಸಿಡಿಲನ್ನಾಗಿಸುತ್ತಿತ್ತು.

ಒಂದು ಸಂಜೆ ಉಪನ್ಯಾಸದ ಬಳಿಕ ಸ್ವಾಮೀಜಿ ತಮ್ಮ ಶಿಷ್ಯರೊಂದಿಗೆ ಮನೆಯ ಕಡೆಗೆ ನಡೆ ಯುತ್ತಿದ್ದರು. ಯಾವುದೋ ವಿಷಯವಾಗಿ ಮಾತುಕತೆ ನಡೆಯುತ್ತಿತ್ತು. ಆಗ ಸಂದರ್ಭಾನುಸಾರ ಸ್ವಾಮೀಜಿ ಒಂದು ಮಾತನ್ನು ಹೇಳಿದರು: “ಕ್ರಿಸ್ತನ ನುಡಿಯನ್ನು ಕೇಳಿಲ್ಲವೆ–‘ನನ್ನ ಮಾತುಗಳು ಶಕ್ತಿಯುತ ಹಾಗೂ ಚೈತನ್ಯದಾಯಕ’ ಎಂದು; ಅದರಂತೆಯೇ ನನ್ನ ಮಾತುಗಳೂ ಶಕ್ತಿಯುತ ಹಾಗೂ ಚೈತನ್ಯದಾಯಕ. ಅವು ನಿಮ್ಮ ಮೆದುಳೊಳಗೆ ಬೆಂಕಿಯಂತೆ ಸುಡುತ್ತ ಪ್ರವೇಶಿಸುತ್ತವೆ. ನೀವು ಅವುಗಳಿಂದ ತಪ್ಪಿಸಿಕೊಳ್ಳಲಾರಿರಿ!”

ಮನಶ್ಶಾಸ್ತ್ರವನ್ನು ಕುರಿತ ಉಪನ್ಯಾಸಮಾಲಿಕೆಯೊಂದರ ಕೊನೆಯ ಭಾಷಣ ಮುಗಿದಿತ್ತು; ಆಗ ಮುಂದಿನ ಸರಣಿಯ ವಿಷಯ ಯಾವುದಿರಬೇಕು ಎನ್ನುವ ಪ್ರಶ್ನೆ ಎದ್ದಿತು. ಸಭಿಕರಲ್ಲೊಬ್ಬನು ತತ್ತ್ವಶಾಸ್ತ್ರದ ವಿಷಯವನ್ನು ತೆಗೆದುಕೊಳ್ಳುವಂತೆ ಕೇಳಿಕೊಂಡ. ಅದಕ್ಕೆ ಸಭಿಕರ ಕಡೆಯಿಂದ ಸಮ್ಮತಿಯ ದನಿಯೂ ಕೇಳಿಬಂತು. ಆಗ ಸ್ವಾಮೀಜಿ ಹೇಳಿದರು, “ಸರಿ, ನಿಮಗೆ ತತ್ತ್ವಶಾಸ್ತ್ರ ಬೇಕು? ಹಾಗಿದ್ದಲ್ಲಿ ನೀವು ಫಿರಂಗಿಯ ಗುಂಡುಗಳಿಗೆ ಸಿದ್ಧರಾಗಿರಿ!” ಶ್ರೀಮತಿ ಎಡಿತ್ ಅಲನ್ ಬರೆಯುತ್ತಾಳೆ, “ಅದು ನಮಗೆ ಸಿಕ್ಕಿತು ಕೂಡ!”

ಸ್ವಾಮೀಜಿ ತಮ್ಮ ಮಾತನ್ನು ಉಳಿಸಿಕೊಂಡರು–ಸಭಿಕರ ಮೇಲೆ ಫಿರಂಗಿ ಗುಂಡುಗಳ ವರ್ಷ ವನ್ನೇ ಕರೆದರು! ಏಕಾತ್ಮವಾದವನ್ನು ಪ್ರತಿಪಾದಿಸುವ ಅವರ ವಿಚಾರಲಹರಿಯು ಈ ಭಾಷಣ ಸರಣಿಯಲ್ಲಿ ಹಿಂದೆಂದಿಗಿಂತಲೂ ಹರಿತವೂ ಶಕ್ತಿಪೂರ್ಣವೂ ಆಗಿತ್ತು. ಈ ಪ್ರಖರ ವಿಚಾರಧಾರೆ ಯಿಂದ ಸಭಿಕರು ದಂಗಾದರು, ತಳಮಳಿಸಿದರು. ಈ ಉಪನ್ಯಾಸಗಳ ವಿಷಯಗಳು ‘ತತ್ತ್ವಮಸಿ’ (‘ಪ್ರಕೃತಿ ಮತ್ತು ಮಾನವ’), ‘ಜೀವಾತ್ಮ ಮತ್ತು ಪರಮಾತ್ಮ’ ಮತ್ತು ‘ಗುರಿ’. ಈ ಭಾಷಣ ಸರಣಿಯ ಬಗ್ಗೆ ರೋಡ್​ಹ್ಯಾಮೆಲ್ ಬರೆಯುತ್ತಾನೆ: “ಸಭಿಕರ ಪರಂಪರಾಗತ ಭಾವನೆಗಳೆಲ್ಲ ಅಡಿಮೇಲಾದುವು. ತಮ್ಮನ್ನು ‘ನವಚಿಂತನೆಯ ವಿದ್ಯಾರ್ಥಿಗಳು’ ಎಂದು ಕರೆದುಕೊಳ್ಳುತ್ತಿದ್ದವ ರಂತೂ ಆ ತೀಕ್ಷ್ಣ ನುಡಿಗಳಿಂದ ನಿರ್ದಯವಾಗಿ, ಆದರೆ ರಚನಾತ್ಮಕವಾಗಿ ದಂಡಿಸಲ್ಪಟ್ಟರು. ಕ್ರೈಸ್ತ ತತ್ತ್ವಗಳಿಗೆ ತದ್ವಿರುದ್ಧವಾದ ಪರಮಾದ್ಭುತ ವೇದಾಂತ ತತ್ತ್ವಗಳನ್ನು ಸ್ವಾಮೀಜಿ ಮುಗು ಳ್ನಗುತ್ತ ಘೋಷಿಸುತ್ತಿದ್ದರು. ಬಳಿಕ ಒಂದು ಕ್ಷಣ ಮಾತು ನಿಲ್ಲಿಸಿ, ಕೆಳತುಟಿ ಕಚ್ಚಿ ನಗುವನ್ನು ತಡೆಹಿಡಿದು, ತಮ್ಮ ಮಾತಿನ ಪರಿಣಾಮವನ್ನು ಗಮನಿಸುತ್ತ ನಿಲ್ಲುತ್ತಿದ್ದರು. ನಿಜಕ್ಕೂ ಅದೆಂತಹ ಪರಿಣಾಮ!”

‘ತತ್ತ್ವಮಸಿ’ ಎಂಬ ಮೊದಲ ಉಪನ್ಯಾಸ ಶಾಂತವಾಗಿಯೇ ಪ್ರಾರಂಭವಾಯಿತು. ಅದ್ವೈತ ವೇದಾಂತ ಸಾರಭೂತ ಬೋಧನೆಗಳನ್ನು ಸ್ವಾಮೀಜಿ ಸ್ಪಷ್ಟವಾಗಿ, ಸಂಗ್ರಹವಾಗಿ ತಿಳಿಸಿದರು. ಅಸ್ಥಿರವಾದ, ಚಲನಶೀಲವಾದ, ಸ್ಥಾವರವಾದ ಹಾಗೂ ಬಂಧಿತವಾದ ಆಂತರಿಕ ಹಾಗೂ ಬಾಹ್ಯ ಪ್ರಕೃತಿಯನ್ನು ಮುಕ್ತ, ಅಮರ, ಸ್ಥಿರ ಹಾಗೂ ಅನಂತ ಆತ್ಮದೊಂದಿಗೆ ಹೋಲಿಸಿದರು: “ಈ ಪರಮಾರ್ಥ ವಸ್ತುವೇ ತನ್ನನ್ನು ಸಾಕಾರಗೊಳಿಸಿಕೊಳ್ಳುತ್ತದೆ; ತನ್ನನ್ನು ತಾನೇ ನಾಮ ರೂಪಗಳ ಬಂಧನಕ್ಕೊಳಪಡಿಸಿಕೊಳ್ಳುತ್ತದೆ. ಹೀಗೆ ಮುಕ್ತಾತ್ಮವು ಕ್ಷಣಮಾತ್ರದಲ್ಲಿ ಬಂಧಿತವಾಗುತ್ತದೆ. ಅದರ ನಿಜಸ್ವರೂಪವು ಬದಲಾಗಿಲ್ಲ, ಬದಲಾಗುವುದೂ ಇಲ್ಲ. ಆದ್ದರಿಂದಲೇ ಅದು ಹೇಳುತ್ತದೆ –‘ನಾನು ನಿತ್ಯಮುಕ್ತ; ಈ ಎಲ್ಲ ಬಂಧನಗಳ ನಡುವೆಯೂ ನಾನು ಅನಿರ್ಬಂಧಿತ, ಮುಕ್ತ!’ ಆತ್ಮವು ತನ್ನ ಬಂಧನಗಳ ಬಗ್ಗೆ ಎಚ್ಚರಗೊಂಡು, ಅವುಗಳಿಂದ ಮುಕ್ತನಾಗಲು ನಡೆಸುವ ಪ್ರಯತ್ನವನ್ನೇ ಜೀವನ ಎನ್ನುವುದು. ಈ ಹೋರಾಟದಲ್ಲಿ ಸಿಗುವ ಯಶಸ್ಸಿಗೇ ವಿಕಾಸ ಎಂದು ಹೆಸರು. ಈ ದಾಸ್ಯ–ಈ ಬಂಧನ ಸಂಪೂರ್ಣವಾಗಿ ನಾಶವಾದಾಗ ಸಿಗುವ ಅಂತಿಮ ವಿಜಯ ವನ್ನೇ ಮುಕ್ತಿ, ನಿರ್ವಾಣ, ಸ್ವಾತಂತ್ರ್ಯ ಎನ್ನುವುದು. ಈ ವಿಶ್ವದಲ್ಲಿ ಪ್ರತಿಯೊಂದೂ ಸ್ವಾತಂತ್ರ್ಯ ಕ್ಕಾಗಿ ಹೋರಾಡುತ್ತಿದೆ. ನಾನು ಪ್ರಕೃತಿಯಿಂದ, ನಾಮ ರೂಪಗಳಿಂದ, ಕಾಲ ದೇಶ ಕಾರಣಗಳಿಂದ ಬದ್ಧನಾಗಿರುವವರೆಗೆ ನಾನು ನಿಜಕ್ಕೂ ಯಾರೆಂಬುದು ನನಗೆ ತಿಳಿದಿರುವುದಿಲ್ಲ. ಆದರೆ ಈ ಬಂಧನದಲ್ಲೂ ನಾನು ಸಂಪೂರ್ಣ ಮುಳುಗಿಹೋಗಿಲ್ಲ. ನಾನು ಬಂಧನಗಳ ವಿರುದ್ಧ ಹೋರಾ ಡುತ್ತೇನೆ. ಅವು ಒಂದೊಂದಾಗಿ ಕತ್ತರಿಸಿ ಬೀಳುತ್ತವೆ. ಆಗ ನಾನು ನನ್ನ ಸುಪ್ತ ಮಹಿಮೆಯನ್ನು ಅರಿಯುತ್ತೇನೆ; ಬಳಿಕ ಸಂಪೂರ್ಣ ವಿಮುಕ್ತಿಯನ್ನು ಹೊಂದುತ್ತೇನೆ. ನನ್ನ ಬಗೆಗಿನ ಅತ್ಯುನ್ನತ ಹಾಗೂ ಅತಿ ಸ್ಪಷ್ಟ ಅರಿವನ್ನು ಪಡೆಯುತ್ತೇನೆ. ಈಗ ನನಗೆ ತಿಳಿದಿದೆ–ನಾನು ಅನಂತ ಆತ್ಮ, ಪ್ರಕೃತಿಯ ಒಡೆಯ–ಅದರ ಗುಲಾಮನಲ್ಲ ಎಂದು.”

ಸ್ವಾಮೀಜಿಯವರು ವೇದಾಂತದ ಅತ್ಯುನ್ನತ ಸತ್ಯಗಳನ್ನು ವಿವರಿಸಿದಾಗಲೆಲ್ಲ ಅವರಿಗೊಂದು ಪ್ರಶ್ನೆ ಕಾದಿರುತ್ತಿತ್ತು–“ಹಾಗಾದರೆ, ಒಬ್ಬನು ಪರಮಾತ್ಮನೊಂದಿಗೆ ತಾದಾತ್ಮ್ಯವನ್ನು ಹೊಂದಿ ದಾಗ ಅವನ ವೈಯಕ್ತಿಕತೆ ಏನಾಗುತ್ತದೆ?” ಈ ಪ್ರಶ್ನೆಗೆ ಸ್ವಾಮೀಜಿ ನಕ್ಕು ಹಾಸ್ಯಮಾಡುತ್ತಿದ್ದರು –“ನಿಮ್ಮ ಈ ದೇಶದ ಜನಗಳಿಗೆ ನಿಮ್ಮ ‘ವೈ-ಯ-ಕ್ತಿ-ಕ-ತೆ-ಗ-ಳ-ನ್ನು\eng{’ (In-di-vi-du-al-i-ties) (}ಒಂದೊಂದು ಅಕ್ಷರವನ್ನೂ ಬಿಡಿಸಿ, ಎಳೆದು ಹೇಳುತ್ತ) ಕಳೆದುಕೊಳ್ಳುವ ಬಗ್ಗೆ ವಿಪರೀತ ಭೀತಿ!” ಬಳಿಕ ಚುಚ್ಚುವ ದನಿಯಲ್ಲಿ ಹೇಳುತ್ತಾರೆ. “ಯಾಕೆ! ನೀವಿನ್ನೂ ‘ವ್ಯಕ್ತಿ’ಗಳೇ ಆಗಿಲ್ಲ. (ನಿಮಗೆ ವೈಯಕ್ತಿಕತೆಯೆಂಬುದೇ ಇಲ್ಲ.) ನೀವು ಭಗವಂತನನ್ನು ಸಾಕ್ಷಾತ್ಕರಿಸಿಕೊಂಡಾಗ ಮಾತ್ರ ವ್ಯಕ್ತಿ ಎಂದಾಗುತ್ತೀರಿ. ನಿಮ್ಮ ಸ್ವರೂಪವನ್ನು ಅರಿತುಕೊಂಡಾಗ ಮಾತ್ರ ನೀವು ನಿಮ್ಮ ನಿಜವಾದ ವೈಯಕ್ತಿಕತೆಯನ್ನು ಗಳಿಸಿಕೊಳ್ಳುತ್ತೀರಿ; ಅದಕ್ಕೆ ಮುಂಚೆಯಲ್ಲ. ಭಗವಂತನನ್ನು ಅರಿಯುವುದ ರಿಂದ ಉಳಿಸಿಕೊಳ್ಳಲು ಯೋಗ್ಯವಾದಂಥದೇನನ್ನೂ ನೀವು ಕಳೆದುಕೊಳ್ಳುವುದಿಲ್ಲ...!ಅಲ್ಲದೆ ಈ ದೇಶದಲ್ಲಿ ನಾನು ಇನ್ನೊಂದು ಮಾತನ್ನು ಕೇಳುತ್ತಿದ್ದೇನೆ–‘ನಾವು ಪ್ರಕೃತಿಯೊಂದಿಗೆ, ಪ್ರಕೃತಿನಿಯಮಗಳೊಂದಿಗೆ ಸಾಮರಸ್ಯದಿಂದ ಜೀವಿಸಬೇಕು’ ಎಂದು.” ಈ ಮಾತನ್ನು ಗೇಲಿ ಮಾಡುತ್ತ ಹೇಳುತ್ತಾರೆ, “ಸಾ-ಮ-ರ-ಸ್ಯ \eng{! (Har-mo-ny!)} ಈ ಜಗತ್ತಿನಲ್ಲಿ ಯಾವಯಾವ ಪ್ರಗತಿಯು ಸಾಧ್ಯವಾಗಿದೆಯೋ ಅದೆಲ್ಲವೂ ಪ್ರಕೃತಿಯೊಂದಿಗೆ ಸೆಣಸುತ್ತ, ಪ್ರಕೃತಿಯನ್ನು ಜಯಿಸುತ್ತ ಸಾಧಿಸಲ್ಪಟ್ಟುವು ಎಂಬುದು ನಿಮಗೆ ತಿಳಿದಿದೆಯೆ? ಈ ನಿಯಮಕ್ಕೆ ಒಂದೇ ಒಂದು ಅಪವಾದವೂ ಇಲ್ಲ. ಮರಗಳು ಪ್ರಕೃತಿಯೊಂದಿಗೆ ಸಾಮರಸ್ಯದಿಂದಿರುತ್ತವೆ. ಹೌದು; ಅಲ್ಲಿ ಪರಿಪೂರ್ಣ ಸಾಮರಸ್ಯವಿರುತ್ತದೆ. ಅಲ್ಲಿ ಯಾವ ರೀತಿಯ ನಿರೋಧವೂ ಇಲ್ಲ. ಆದ್ದರಿಂದಲೇ ಅಲ್ಲಿ ಯಾವ ಮುನ್ನಡೆಯೂ ಇಲ್ಲ! ಕಡೆಗೂ ಮರವು ಮರವಾಗಿಯೇ ಉಳಿಯುತ್ತದೆ. ನಾವೇನಾದರೂ ಮುನ್ನಡೆದು ಅಭಿವೃದ್ಧಿ ಸಾಧಿಸಬೇಕಾದರೆ, ಒಳ-ಹೊರ ಪ್ರಕೃತಿಯನ್ನು ಪ್ರತಿ ಹಂತದಲ್ಲೂ ಎದುರಿಸಬೇಕು. ನಮ್ಮ ಜೀವನದಲ್ಲಿ ಆಗಾಗ ಏನೋ ಸ್ವಲ್ಪ ಹೆಚ್ಚು ಕಡಿಮೆಯಾಗು ತ್ತದೆ; ಆಗ ಪ್ರಕೃತಿ ಹೇಳುತ್ತದೆ–‘ಅಳು!’ ಎಂದು. ಸರಿ, ನಾವು ಅಳುತ್ತೇವೆ!”

ಅಷ್ಟರಲ್ಲಿ ಒಬ್ಬ ವೃದ್ಧ ಮಹಿಳೆ ಎದ್ದು ನಿಂತು ನುಡಿದಳು, “ಆದರೆ ಈಗ, ನಮಗೆ ಪ್ರಿಯರಾದವರೊಬ್ಬರು ನಮ್ಮನ್ನು ಅಗಲಿದರೆ, ಆ ಶೋಕ ಎಷ್ಟು ಅಸಹನೀಯವಾಗುತ್ತದೆ! ನನಗನ್ನಿಸುತ್ತದೆ–ಆಗ ಶೋಕಿಸದಿದ್ದರೆ ನಾವು ಕಟುಕರೇ ಸರಿ ಎಂದು.” ಸ್ವಾಮೀಜಿಯವರ ವಿಶಾಲ ನೇತ್ರಗಳು ಅಚ್ಚರಿಯ ಭಾವವನ್ನು ಸೂಚಿಸಿದುವು. ಆಕೆಯತ್ತ ತಿರುಗಿ ಅವರು ಹೇಳಿದರು, “ಓ, ಹೌದು ಮೇಡಮ್, ಅದು ನಿಜಕ್ಕೂ ಕಷ್ಟವೇ! ನಿಸ್ಸಂದೇಹವಾಗಿ. ಅಂದರೆ ಅದರಿಂದೇ ನಂತೆ? ಎಲ್ಲ ಮಹಾ ಸಾಧನೆಗಳೂ ಕಠಿಣವೇ. ಗಮನಾರ್ಹವಾದಂಥದು ಯಾವುದೂ ಅನಾ ಯಾಸವಾಗಿ ಆಗುವುದಿಲ್ಲ. ಆದರೆ ಆದರ್ಶವನ್ನು ಮುಟ್ಟುವುದು ಕಷ್ಟ ಎಂದು ಅದನ್ನೇ ಕೆಳಗೆಳೆಯಬಾರದು. ಸ್ವಾತಂತ್ರ್ಯದ ಧ್ವಜವನ್ನು ಎತ್ತಿಹಿಡಿಯಿರಿ. ಅಲ್ಲದೆ, ಮೇಡಮ್, ನೀವು ಅಳುವುದೇಕೆಂದರೆ, ಅಳುವುದು ನಿಮಗೆ ಇಷ್ಟ ಎಂದಲ್ಲ; ಪ್ರಕೃತಿ ನಿಮ್ಮನ್ನು ಹಾಗೆ ಮಾಡುವಂತೆ ಬಲಾತ್ಕರಿಸುತ್ತದೆ. ನಿಮಗೆ ಅಂತಹ ಸ್ವಾತಂತ್ರ್ಯವಿದ್ದರೆ, ಪ್ರಕೃತಿ ‘ಅಳು!’ ಎಂದಾಗ ‘ಇಲ್ಲ, ನಾನು ಅಳುವುದಿಲ್ಲ’ ಎಂದು ಹೇಳಿ ನೋಡೋಣ? ಶಕ್ತಿ! ಶಕ್ತಿ! ಶಕ್ತಿ! ಇದನ್ನು ನಿಮ್ಮಷ್ಟಕ್ಕೆ ನೀವು ಹಗಲಿರುಳು ಹೇಳಿಕೊಳ್ಳಿ. ನೀವು ಶಕ್ತಿವಂತರು, ಪವಿತ್ರಾತ್ಮರು, ಮುಕ್ತರು. ನಿಮ್ಮೊಳಗೆ ಯಾವ ದೌರ್ಬಲ್ಯವೂ ಇಲ್ಲ. ಯಾವ ಪಾಪವೂ ಇಲ್ಲ, ಯಾವ ದುಃಖವೂ ಇಲ್ಲ.”

ಸ್ವಾಮೀಜಿಯವರ ಅದ್ಭುತ ಪ್ರಚಂಡ ವ್ಯಕ್ತಿತ್ವದಿಂದ ಮತ್ತಷ್ಟು ಶೋಭೆಗೊಂಡ ಇಂತಹ “ಅಗ್ನಿಚಿಹ್ನೆ”ಯ ನುಡಿಗಳು ಸಭಿಕರ ದೃಷ್ಟಿಯಲ್ಲಿ ಸ್ವಾಮೀಜಿಯವರನ್ನು ಜಗತ್ತಿನ ಮಹೋನ್ನತ ಆಧ್ಯಾತ್ಮಿಕ ವ್ಯಕ್ತಿಗಳ ಸಾಲಿಗೆ ಸೇರಿಸಿದುವು. ರೋಡ್​ಹ್ಯಾಮೆಲ್ ಬರೆಯುತ್ತಾನೆ:

“ಈ ಜಗತ್ತು ಯಾರಿಗೆ ಕೇವಲ ಲೀಲೆಯಾಗಿದೆಯೋ, ತುರೀಯಾವಸ್ಥೆಯೊಂದೇ ಏಕಮಾತ್ರ ಸತ್ಯವಾಗಿದೆಯೋ ಅಂತಹ ಅತಿಮಾನುಷ ಚೇತನದ ಸಾನ್ನಿಧ್ಯದಲ್ಲಿ ಈ ಬಗೆಯ ಮಾತುಗಳ ನ್ನಾಲಿಸುತ್ತ ಕುಳಿತವರು ಅಧ್ಯಾತ್ಮವೆಂಬ ಅಂತರಿಕ್ಷದ ವೈಶಾಲ್ಯದಲ್ಲಿ ಆನಂದದಿಂದ ತೇಲಾಡು ತ್ತಿರುವ ಅನುಭವವನ್ನು ಪಡೆಯುತ್ತಿದ್ದರು.”

ಸ್ವಾಮೀಜಿಯವರ ಉಪನ್ಯಾಸಗಳು ಚಾಟೂಕ್ತಿಗಳಿಂದಲೂ ಬಗೆಬಗೆಯ ಉನ್ನತಮಟ್ಟದ ಹಾಸ್ಯದಿಂದಲೂ ತುಂಬಿ ಪ್ರತಿಯೊಬ್ಬರನ್ನೂ ಸುಳಿಯಂತೆ ಸೆಳೆದುಬಿಡುತ್ತಿದ್ದುವು. ಆದರೆ ಆ ಹಾಸ್ಯದಲ್ಲೂ ಒಂದು ಬಗೆಯ ಗಾಂಭೀರ್ಯವಿರುತ್ತಿತ್ತು. ಕ್ಷಣಮಾತ್ರದಲ್ಲಿ ಅವರು ಈ ಹಾಸ್ಯ ಭಾವದಿಂದ ಉನ್ನತ ಆಧ್ಯಾತ್ಮಿಕ ಭಾವಕ್ಕೇರಬಲ್ಲವರಾಗಿದ್ದರು. ಅವರು ಇಂತಹ ದಿವ್ಯಭಾವ ರಂಜಿತರಾಗಿ ಮಾತನಾಡುತ್ತಿರುವಾಗ ಅವರಿಂದ ಹೊರಹೊಮ್ಮಿದ ಪಾರಮಾರ್ಥಿಕ ಭಾವ ತರಂಗಗಳು ಸಭಿಕರಲ್ಲಿ ಅವರ ಬಗ್ಗೆ ಪೂಜ್ಯಭಾವವನ್ನು ಪ್ರಚೋದಿಸುತ್ತಿದ್ದುವು.

ನಿಜಕ್ಕೂ ಸ್ವಾಮೀಜಿಯವರ ಹಲವು ವಿಧದ ಆಕರ್ಷಣ ಶಕ್ತಿ, ಹಾಸ್ಯ, ಅವರ ನಡವಳಿಕೆ, ಅವರ ಆತ್ಮಶಕ್ತಿ, ಅವರ ಚೈತನ್ಯದಾಯಕ ಜೀವನ ಇವು ಅವರ ಮಾತುಗಳಷ್ಟೇ ಪ್ರಭಾವಶಾಲಿ ಯಾಗಿದ್ದುವು. ಅಷ್ಟೇಕೆ, ಅವರ ಶರೀರದ ಚಹರೆಚಹರೆಯೂ ಪ್ರಬಲ ಆಕರ್ಷಣ ಶಕ್ತಿಯನ್ನು ಹೊಂದಿತ್ತು. ಎರಡು ವರ್ಷಗಳಿಗೂ ಹೆಚ್ಚು ಅವಧಿಯ ಅನಾರೋಗ್ಯ ಹಾಗೂ ನಿರಂತರ ಚಟುವಟಿಕೆಯ ಬಳಿಕವೂ ಅವರು ತಮ್ಮ ಪ್ರಥಮ ಅಮೆರಿಕ ಭೇಟಿಯ ಸಮಯದಷ್ಟೇ ಮೋಹಕ ರಾಗಿದ್ದರು. ಶ್ರೀಮತಿ ಅಲನ್ ಒಮ್ಮೆ ಹೇಳುತ್ತಾಳೆ, “ಅವರ ಸೌಂದರ್ಯವನ್ನು ಯಾರೂ ಊಹಿಸಲಾರರು. ಕೆಲವೊಮ್ಮೆ ಅವರ ಮೈಬಣ್ಣ ಬದಲಾಗುವಂತೆ ಕಂಡುಬರುತ್ತಿತ್ತು. ಅದು ಕೆಲವು ದಿನ ಗಾಢವಾಗಿ ಮತ್ತೆ ಕೆಲವು ದಿನ ತಿಳಿಯಾಗಿರುವಂತೆ ಕಾಣುತ್ತಿತ್ತು. ಸಾಧಾರಣವಾಗಿ, ಅವರದು ಉಜ್ವಲ ಸುವರ್ಣವರ್ಣ ಎನ್ನಬಹುದಾದಂತಹ ಮೈಬಣ್ಣ; ಅದಕ್ಕೊಪ್ಪುವ ಕೇಸರಿ ವರ್ಣದ ಭರ್ಜರಿ ಪೇಟವನ್ನು ಸುತ್ತಿಕೊಂಡು ವೇದಿಕೆಯ ಮೇಲೆ ನಿಂತರೆ ಅವರ ನಿಲುವು ಸಾಕ್ಷಾತ್ ಪರಮಾತ್ಮನಂತೆಯೇ ಕಾಣುತ್ತಿತ್ತು.” ರೋಡ್​ಹ್ಯಾಮೆಲ್ ಬರೆಯುತ್ತಾನೆ: “ಆ ಮಾತನಾಡುವ ಕಂಗಳು, ಮುಖಭಾವದ ಹಾಗೂ ಭಾವಾಭಿನಯದ ವೈವಿಧ್ಯ, ಹಿತವಾದ ಸಮ್ಮೋಹಕವಾದ ಸಂಸ್ಕೃತ ಶ್ಲೋಕಗಳ ಪಠಣ, ಆತ್ಮವಿಶ್ವಾಸಭರಿತ ಮುಗುಳ್ನಗೆಯೊಡನೆ ಹೊಮ್ಮುವ ಮಾತುಗಳು ಮತ್ತು ಇವುಗಳಿಗೆ ಕಳೆಕೊಡುವಂತಿರುವ ಆ ಹಿಂದೂ ಸಂನ್ಯಾಸಿಯ ಉಡಿಗೆತೊಡಿಗೆ–ಇವು ಗಳನ್ನು ಯಾರು ತಾನೆ ಮರೆಯಬಲ್ಲರು?”

ಈ ಮಾತುಗಳೆಲ್ಲ ನಿಜಕ್ಕೂ ಎಷ್ಟು ಪ್ರಭಾವಶಾಲಿಯಾಗಿವೆ! ಆ ಪಾಶ್ಚಾತ್ಯ ವ್ಯಕ್ತಿಗಳು ಸ್ವಾಮೀಜಿಯವರ ಕುರಿತಾಗಿ ಇಂತಿಂತಹ ವರ್ಣನೆಯನ್ನು ಮಾಡಬೇಕಾದರೆ ಅವರೆಲ್ಲ ಎಂತಹ ಅಲೌಕಿಕತೆಯನ್ನು ಅನುಭವಿಸಿರಲಿಕ್ಕಿಲ್ಲ!

ಸ್ವಾಮೀಜಿಯವರ ಕಂಠಧ್ವನಿಯೂ ಅಷ್ಟೇ ಚಿತ್ತಾಕರ್ಷಕವಾದ ಮತ್ತೊಂದು ಅಂಶವಾಗಿತ್ತು. ಹಲವಾರು ಜನ ಅವರ ಕಂಠಶ್ರೀಯನ್ನು ಕೊಂಡಾಡಿದ್ದಾರೆ. ಅವರ ಧ್ವನಿಯನ್ನು “ಅತ್ಯಂತ ಅದ್ಭುತವಾದದ್ದು” “ಅತ್ಯಂತ ಸಂಗೀತಮಯವಾದದ್ದು” “ಆಳವಾದದ್ದು” “ಸುಸ್ಪಷ್ಟವಾದದ್ದು” “ಮಧುರವಾದದ್ದು” ಎಂದು ಒಬ್ಬೊಬ್ಬರು ಒಂದೊಂದು ರೀತಿಯಾಗಿ ವರ್ಣಿಸುತ್ತಾರೆ.

ಈ ದಿನಗಳಲ್ಲಿ ಮಾಡಿದ ಹಲವಾರು ಭಾಷಣಗಳಲ್ಲಿ ಸ್ವಾಮೀಜಿಯವರ ವಾಗ್ಝರಿ ಹಿಂದೆಂದಿ ಗಿಂತಲೂ ಪ್ರಬಲವಾಗಿತ್ತು, ಅಪ್ರತಿಹತವಾಗಿತ್ತು. ‘ತತ್ತ್ವಶಾಸ್ತ್ರ’ವನ್ನು ಕುರಿತ ಭಾಷಣಮಾಲಿಕೆ ಯಲ್ಲಿ, ಮತಾಂಧತೆಯನ್ನೂ ಸಂಕುಚಿತ ದೃಷ್ಟಿಯನ್ನೂ ನಿರ್ದಯವಾಗಿ ಖಂಡಿಸಿದ ಸ್ವಾಮೀಜಿ ಯವರು, ‘ಧ್ಯೇಯ’ ಎಂಬ ಕಡೆಯ ಭಾಷಣದಲ್ಲಿ ಗರ್ಜಿಸುತ್ತಾರೆ: “ಯಾವ ಚರ್ಚೂ (ಯಾವ ಒಂದು ಮತವೂ) ಒಬ್ಬ ವ್ಯಕ್ತಿಯನ್ನು ತಾನಾಗಿಯೇ ಎಂದಿಗೂ ರಕ್ಷಿಸಲಿಲ್ಲ. ಚರ್ಚಿನಲ್ಲಿ ಹುಟ್ಟು ವುದೇನೋ (ಸಾಂಪ್ರದಾಯಿಕ ನಂಬಿಕೆಗಳಿಂದ, ಆಚರಣೆಗಳಿಂದ ಧಾರ್ಮಿಕ ಜೀವನವನ್ನು ಪ್ರಾರಂಭಿಸುವುದೇನೋ) ಒಳ್ಳೆಯದೇ. ಆದರೆ ಅದರಲ್ಲೇ ಕೊನೆಯುಸಿರೆಳೆಯುವವನನ್ನು ನಾನು ಮೆಚ್ಚುವುದಿಲ್ಲ. ಅದರಿಂದಾಚೆಗೆ ಕ್ರಮಿಸಿ ಹೋಗಬೇಕು! ಈ ಚರ್ಚುಜೀವನವೊಂದು ಉತ್ತಮ ಪ್ರಾರಂಭವೇನೋ ನಿಜ. ಆದರೆ ಕೊನೆಗೊಂದು ದಿನ ಅದನ್ನು ಬಿಟ್ಟುಬಿಡಿ. ಅದು ಬಾಲ್ಯಾವಸ್ಥೆಯ ಸ್ಥಳ, ಅದು ಹಾಗೆಯೇ ಇದ್ದುಕೊಳ್ಳಲಿ. ನೀವು ಅದರಿಂದ ಹೊರಬಂದು, ನೇರವಾಗಿ ಭಗವಂತ ನಲ್ಲಿಗೆ ಸಾಗಿ.”

ತಮ್ಮ ಪಾಪಗಳ ಬಗ್ಗೆ ಕೊರಗುತ್ತ ಕುಳಿತವರಿಗೆ ಕೂಗಿ ಹೇಳುತ್ತಾರೆ: “ವಿಷಾದಿಸಬೇಡಿ! ಸಂಕಟ ಪಡಬೇಡಿ! ಆದದ್ದು ಆಗಿಹೋಯಿತು. ವಿವೇಕಿಗಳಾಗಿ! ನಾವು ತಪ್ಪು ಮಾಡುತ್ತೇವೆ ನಿಜ. ಆದರೇನೀಗ? ಅದೆಲ್ಲ ಒಂದು ತಮಾಷೆ, ಅಷ್ಟೆ. ಹಿಂದಿರುಗಿ ನೋಡಬೇಡಿ. ಹಾಗೆ ನೋಡುವುದರಿಂದ ಲಾಭವಾದರೂ ಏನು? ಏನೇನೂ ಇಲ್ಲ. ಹೆದರಬೇಡಿ, ನಿಮ್ಮನ್ನು ಯಾರೂ ಬೆಣ್ಣೆಯಂತೆ ಕರಗಿಸಿಬಿಡುವುದಿಲ್ಲ!... ಸ್ವರ್ಗ, ನರಕ, ಅವತಾರ–ಇವೆಲ್ಲ ಮೂರ್ಖತನ!”

ಸ್ವಾಮೀಜಿಯವರ ಈ ಮಾತುಗಳು ಆ ಪರಕೀಯರನ್ನಂತಿರಲಿ ಭಾರತೀಯರಾದ ನಮ್ಮನ್ನೇ ಕಂಗೆಡಿಸುವಂಥದು. ಸ್ವಾಮಿ ವಿವೇಕಾನಂದರ ಜೀವನದ ಬಗ್ಗೆ ಸಂಶೋಧನೆಗಳನ್ನು ನಡೆಸಿ \eng{\textit{Swami Vivekananda in the West–New Discoveries}} ಎಂಬ ಗ್ರಂಥಮಾಲೆಯನ್ನು ರಚಿಸುತ್ತಿರುವ ಮಿಸ್ ಲೂಯಿ ಬರ್ಕ್ (ಗಾರ್ಗಿ) ಬರೆಯುತ್ತಾರೆ: “ಸ್ವಾಮೀಜಿಯವರು ನಮ್ಮ ಮುಂದಿಟ್ಟ ವಿಚಾರಲಹರಿಯು ಇಂದು ನಮಗೆ ಅತ್ಯಾಶ್ಚರ್ಯಕರವಾಗಿ ಕಾಣಿಸಬಹುದು. ಆದರೆ ೧೯ಂಂರಲ್ಲಿ ಅವು ಸಿಡಿಲ ಬಡಿತಗಳಾಗಿದ್ದುವು. ಸ್ವಾಮೀಜಿಯವರ ವ್ಯಕ್ತಿತ್ವದ ಅಗಾಧ ಶಕ್ತಿಯಿಂದ ಚಿಮ್ಮಲ್ಪಟ್ಟ ಈ ಮಾತುಗಳು ಶ್ರೋತೃಗಳಲ್ಲಿ (ಕೇವಲ ಆಶ್ಚರ್ಯವನ್ನಲ್ಲ) ದೊಡ್ಡ ಅಲ್ಲೋಲ ಕಲ್ಲೋಲವನ್ನುಂಟುಮಾಡಿದುವು. ಆ ಮಾತುಗಳು ಕೇವಲ ಮಾತುಗಳಾಗಿರಲಿಲ್ಲ, ಅವು ಕೇಳುಗರ ಮೆದುಳೊಳಗೆ ಹೊಸ ದಾರಿಯನ್ನು ಮಾಡಿಕೊಂಡು ಪ್ರವೇಶಿಸುತ್ತಿದ್ದ ಅನುಭವಗಳಾಗಿದ್ದುವು. ಹಲವಾರು ಜನ ಅವರ ಮಾತಿನ ಸರಣಿಯನ್ನು ನಿರೋಧಿಸಲು ಪ್ರಯತ್ನಿಸಿದರು. ಆದರೆ ಅವರೆಲ್ಲ ಸ್ವಾಮೀಜಿ ತಮ್ಮ ಮುಂದಿಟ್ಟ ಬೆಳಕಿನ ಪ್ರಪಂಚದೊಳಗೆ ಮತ್ತಷ್ಟು ಆಳಕ್ಕೆ ಸೆಳೆಯಲ್ಪಡುತ್ತಿ ದ್ದರು.”

ರೋಡ್​ಹ್ಯಾಮೆಲ್ ಹೇಳುತ್ತಾನೆ, “ಪ್ರತಿಭಟಿಸುವವರ ನಿರೋಧಶಕ್ತಿ ಎನ್ನುವುದು ಆ ಮಹಾ ಗುರುವಿನ ಅಪ್ರತಿಹತ ತರ್ಕ, ತೀಕ್ಷ್ಣತೆ ಹಾಗೂ ಮಗುವಿನ ಸರಳತೆ ಇವುಗಳಿಂದ ತಣ್ಣಗಾಗಿಬಿಡು ತ್ತಿತ್ತು. ಆಕ್ಷೇಪಣೆಯೆತ್ತಿ ಎದ್ದುನಿಲ್ಲುತ್ತಿದ್ದವರೂ ಕೆಲವರಿದ್ದರು. ಆದರೆ ಅವರೂ ಕೂಡ ದಿಗ್ಬ್ರಮೆಗೊಂಡವರಾಗಿ ಅಲ್ಲೇ ಕುಕ್ಕರಿಸುತ್ತಿದ್ದರು.”

