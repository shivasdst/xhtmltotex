
\chapter{ಮತ್ತೊಮ್ಮೆ ಪಶ್ಚಿಮದತ್ತ}

\noindent

ಸುಮಾರು ಒಂದು ವರ್ಷದಿಂದಲೂ ಸ್ವಾಮೀಜಿ ಪಶ್ಚಿಮ ದೇಶಗಳಿಗೆ ಮತ್ತೆ ಹೋಗಬೇಕೆಂದು ಆಲೋಚಿಸುತ್ತಲೇ ಇದ್ದರು. ಆದರೆ ಅದು ಮುಂದುಮುಂದಕ್ಕೆ ಹೋಗುತ್ತಲೇ ಇತ್ತು. ಈಗ ಅವರಿಗೆ ಚಿಕಿತ್ಸೆ ಮಾಡುತ್ತಿದ್ದ ಮಹಾನಂದ ಪಂಡಿತ ಎಂಬವರು, ಸ್ವಾಮೀಜಿಯವರು ಆರೋಗ್ಯ ಸುಧಾರಣೆಗಾಗಿ ಸಮುದ್ರ ಪ್ರಯಾಣ ಕೈಗೊಂಡರೆ ಒಳ್ಳೆಯದು ಎಂದು ಸಲಹೆ ನೀಡಿದರು. ಅದರಲ್ಲೂ, ಹೆಚ್ಚು ಸಮಯ ತೆಗೆದುಕೊಳ್ಳುವ ಸರಕು ಸಾಗಣೆಯ ಹಡಗಾದರೆ ಉತ್ತಮ. ಈಗಾಗಲೇ ಅವರ ಆರೋಗ್ಯ ವಿಷಮ ಸ್ಥಿತಿಯಲ್ಲಿದ್ದುದರಿಂದ ಅವರು ಹವಾ ಬದಲಾವಣೆಗಾಗಿ ಕೂಡಲೇ ಹೊರಡಬೇಕೆಂದು ಬ್ರಹ್ಮಾನಂದರೇ ಮೊದಲಾದವರು ಒತ್ತಾಯಿಸಿದರು. ಆಗಿನ ತಮ್ಮ ಆರೋಗ್ಯದ ಬಗ್ಗೆ ಸ್ವಾಮೀಜಿಯವರು ಏಪ್ರಿಲ್ ಹನ್ನೊಂದರಂದು ಕ್ರಿಸ್ಟೀನಳಿಗೊಂದು ಪತ್ರ ಬರೆದರು–“ನನ್ನ ತೊಂದರೆ ನಿಜಕ್ಕೂ ಏನು...? ನನಗೆ ಗೊತ್ತಿಲ್ಲ. ಕೆಲವರು ಆಸ್ತಮಾ ಎನ್ನುತ್ತಾರೆ. ಇನ್ನು ಕೆಲವರು ಅತಿ ಪರಿಶ್ರಮದಿಂದ ಉಂಟಾದ ಹೃದಯ ದೌರ್ಬಲ್ಯ ಎನ್ನುತ್ತಾರೆ. ಆದರೂ, ಹಿಂದೆ ಒಮ್ಮೆ ಬಂದರೆ ಅನೇಕ ದಿನಗಳವರೆಗೆ ಒಂದೇ ಸಮನೆ ಕಾಡುತ್ತಿದ್ದ ಭಯಂಕರ ಆಸ್ತಮಾ ಕಳೆದೆರಡು ತಿಂಗಳಲ್ಲಿ ಮರುಕಳಿಸಿಲ್ಲ. ಆದರೆ ಇತರ ಆಸ್ತಮಾ ರೋಗಿಗಳಿಗೆ ಇರ ದಂತಹ ಹೃದಯದ ಒಂದು ಬಗೆಯ ದೌರ್ಬಲ್ಯ ಯಾವಾಗಲೂ ಇದ್ದೇ ಇರುತ್ತದೆ... ಒಮ್ಮೆ ಸಮುದ್ರಯಾನ ಮಾಡುವುದರಿಂದ ಖಂಡಿತವಾಗಿಯೂ ಅನುಕೂಲವಾಗುತ್ತದೆ. ಜೊತೆಗೆ ಕಲ್ಕತ್ತ ದಲ್ಲಿ ಪ್ಲೇಗ್ ಪರಿಹಾರಕ್ಕಾಗಿ ಸ್ವಲ್ಪ ಕೆಲಸ ಪ್ರಾರಂಭಿಸಿರುವುದರಿಂದ ನನ್ನ ಅಂತರಂಗಕ್ಕೂ ಸಮಾಧಾನವಾಗಿದೆ. ಒಟ್ಟಿನಲ್ಲಿ ತೀರ ಅನಿರೀಕ್ಷಿತವಾದದ್ದೇನೂ ಸಂಭವಿಸದಿದ್ದರೆ ಈ ಬೇಸಿಗೆ ಯಲ್ಲಿ ನಾನು ಇಂಗ್ಲೆಂಡಿನಲ್ಲಿರುವುದು ಖಚಿತ. ನೀನೂ ಆ ವೇಳೆಗೆ ಇಂಗ್ಲೆಂಡಿಗೆ ಬರಲು ಸಾಧ್ಯವಾಗಬಹುದೆ? ಸುಮ್ಮನೆ ಪ್ರವಾಸಕ್ಕಾಗಿ! ನಿನ್ನನ್ನು ನೋಡಲು ನನಗಂತೂ ತುಂಬ ಸಂತೋಷವಾಗುತ್ತದೆ. ನೀನು ಅಲ್ಲಿಗೆ ಬಂದರೆ ಇಂಗ್ಲೆಂಡನ್ನೂ ಪ್ರಪಂಚದ ಅತ್ಯುತ್ತಮ ವಸ್ತುವಾದ ನನ್ನನ್ನೂ ನೋಡಬಹುದು... ಆದರೆ ನಿನಗೆ ನನ್ನ ಗುರುತು ಹಿಡಿಯುವುದು ಬಹಳವೇ ಕಷ್ಟವಾಗುತ್ತದೆ. ನಾನು ಅಷ್ಟೊಂದು ಇಳಿದುಹೋಗಿದ್ದೇನೆ, ಅಷ್ಟೊಂದು ಮುದುಕ ನಾಗಿಬಿಟ್ಟಿದ್ದೇನೆ. ಎರಡು ವರ್ಷದ ಕಾಯಿಲೆ-ನರಳಾಟಗಳು ನನ್ನ ಆಯುಸ್ಸಿನಿಂದ ಇಪ್ಪತ್ತು ವರ್ಷಗಳನ್ನೇ ಕಿತ್ತುಕೊಂಡುಬಿಟ್ಟಿವೆ. ಇರಲಿ, ಆದರೆ ಒಳಗಿನ ಆತ್ಮವೇನೂ ಬದಲಾಗಿಲ್ಲ. ಅದೆಂದಾದರೂ ಬದಲಾದೀತೆ? ಅದು ಅಲ್ಲಿ ಇದ್ದೇ ಇದೆ–ಅದೇ ಹುಚ್ಚಪ್ಪ ಆತ್ಮ. ಒಂದೇ ಆಲೋಚನೆಯನ್ನು ಕುರಿತು ಬಲವಾದ ಹುಚ್ಚು ಹಿಡಿದಿರುವ ಆತ್ಮ...”

ಒಟ್ಟಿನಲ್ಲಿ ಸ್ವಾಮೀಜಿ ಪಶ್ಚಿಮ ದೇಶಗಳಿಗೆ ಮತ್ತೆ ಹೊರಡುವುದಂತೂ ಖಚಿತವಾಗಿತ್ತು. ಆದರೆ ಯಾವ ದಿನವೆಂಬುದು ಇನ್ನೂ ನಿರ್ಧಾರವಾಗಿರಲಿಲ್ಲ. ಧರ್ಮಪ್ರಚಾರ ಕಾರ್ಯಕ್ಕೆಂದು ಕಾಥೇವಾಡಕ್ಕೆ ತೆರಳಿದ್ದ ಶಾರದಾನಂದರೂ ತುರೀಯಾನಂದರೂ ಇನ್ನೂ ಹಿಂದಿರುಗಿರಲಿಲ್ಲ. ಸ್ವಾಮೀಜಿ ಇವರಿಬ್ಬರ ಆಗಮನವನ್ನೇ ನಿರೀಕ್ಷಿಸುತ್ತಿದ್ದರು. ಶಾರದಾನಂದರಿಗೆ ಮಠದ ನಿರ್ವಹಣೆ ಯನ್ನು ವಹಿಸಿಕೊಟ್ಟು, ತುರೀಯಾನಂದರನ್ನು ತಮ್ಮೊಂದಿಗೆ ಕರೆದೊಯ್ಯಬೇಕೆಂಬುದು ಅವರ ಉದ್ದೇಶವಾಗಿತ್ತು. ಅಂತೂ ಸ್ವಾಮೀಜಿಯವರ ಕರೆಯ ಮೇರೆಗೆ ಅವರಿಬ್ಬರೂ ಮೇ ಮೂರ ರಂದು ಹಿಂದಿರುಗಿದರು.

ಸ್ವಾಮಿ ತುರೀಯಾನಂದರು ಸ್ವಭಾವತಃ ಧ್ಯಾನಶೀಲ ವ್ಯಕ್ತಿ. ಆದ್ದರಿಂದ ಅವರಿಗೆ ಸಾರ್ವಜನಿಕ ಜೀವನವೆಂದರೆ ಅಷ್ಟಕ್ಕಷ್ಟೆ. ಅವರನ್ನು ಸಾರ್ವಜನಿಕ ಕಾರ್ಯರಂಗಕ್ಕೆ ಕರೆತರಲು ಸ್ವಾಮೀಜಿ ಬಹಳ ಕಾಲದಿಂದಲೂ ಪ್ರಯತ್ನ ಪಡುತ್ತಿದ್ದರು. ಆದರೇನೂ ಪ್ರಯೋಜನವಾಗಿರಲಿಲ್ಲ. ಕಡೆ ಗೊಂದು ದಿನ ಅವರಿಬ್ಬರಿಗೂ ಈ ವಿಷಯವಾಗಿ ವಾಗ್ವಾದ ಪ್ರಾರಂಭವಾಯಿತು. ತುರೀಯಾ ನಂದರು ತಮ್ಮ ಪಟ್ಟು ಬಿಡದೆ, ಆದರೆ ವಿನಯದಿಂದಲೇ, ಈ ಭಾಷಣ-ಪ್ರವಚನಗಳೆಲ್ಲ ತಮಗೆ ಹೇಳಿಸಿದ್ದಲ್ಲ ಎಂದು ವಾದಿಸಿದರು. ಕಡೆಗೆ ಸ್ವಾಮೀಜಿ, ತಮ್ಮ ಚತುರೋಪಾಯಗಳೂ ಮುಗಿದು ಹೋದಾಗ ತುರೀಯಾನಂದರ ಕೊರಳ ಸುತ್ತ ತಮ್ಮ ತೋಳುಗಳನ್ನು ಬಳಸಿ ತಮ್ಮ ತಲೆಯನ್ನು ಅವರ ಎದೆಯ ಮೇಲೊರಗಿಸಿ ಕಣ್ಣೀರ ಕೋಡಿ ಹರಿಸುತ್ತ ನುಡಿದರು, “ಪ್ರಿಯ ಹರಿಭಾಯ್, ನಿನಗೆ ಕಾಣಿಸುತ್ತಿಲ್ಲವೆ, ನಾನು ನಮ್ಮ ಗುರುದೇವರ ಕಾರ್ಯವನ್ನು ಪೂರೈಸುವುದಕ್ಕಾಗಿ ನನ್ನ ಜೀವವನ್ನು ಅಂಗುಲ ಅಂಗುಲವಾಗಿ ಕೊನೆಯುಸಿರಿನವರೆಗೂ ತೇಯುತ್ತಿರುವುದು? ಈ ನನ್ನ ಮಹಾಭಾರದಲ್ಲಿ ನೀನು ಒಂದಿಷ್ಟನ್ನಾದರೂ ಹೊರದೆ ಸುಮ್ಮನೆ ನೋಡುತ್ತ ನಿಂತಿರಬಲ್ಲೆಯಾ?” ಈ ಮಾತನ್ನು ಕೇಳಿದೊಡನೆಯೇ ತುರೀಯಾನಂದರು ಕರಗಿಹೋದರು. ಸ್ವಾಮೀಜಿಯವರ ಮೇಲಣ ಅವರ ಪ್ರೀತಿ ಅಂಥದು. ಅವರ ಸಂಕೋಚ-ಹಿಂಜರಿಕೆಯೆಲ್ಲ ಕ್ಷಣಮಾತ್ರದಲ್ಲಿ ಮಾಯವಾಯಿತು. ಸ್ವಾಮೀಜಿ ತಮಗೆ ಯಾವ ಕೆಲಸವನ್ನು ಒಪ್ಪಿಸುವರೋ ಅದನ್ನು ಮರುಮಾತಿ ಲ್ಲದೆ ಮಾಡುವುದಾಗಿ ಆಗಲೇ ಪ್ರತಿಜ್ಞೆ ಮಾಡಿದರು. ಅಂದಿನಿಂದಲೇ ಅವರು ತಮ್ಮ ಸೋದರ ಸಂನ್ಯಾಸಿಗಳೊಂದಿಗೆ ಕೆಲಸಕಾರ್ಯಗಳ ಹೊಣೆಗಾರಿಕೆಯನ್ನು ವಹಿಸಿಕೊಳ್ಳತೊಡಗಿದರು. ಅಲ್ಲದೆ ಈಗ ಸ್ವಾಮೀಜಿ ಅವರನ್ನು ತಮ್ಮೊಂದಿಗೆ ಪಶ್ಚಿಮ ದೇಶಗಳಿಗೆ ವೇದಾಂತ ಪ್ರಚಾರಕ್ಕಾಗಿ ಬರು ವಂತೆ ಕೇಳಿಕೊಂಡಾಗ ತುರೀಯಾನಂದರು ಅದನ್ನು ಜಗನ್ಮಾತೆಯ ಇಚ್ಛೆ ಎಂದು ಭಾವಿಸಿ ಮರು ಮಾತಿಲ್ಲದೆ ಒಪ್ಪಿಕೊಂಡರು.

ಸ್ವಾಮಿ ತುರೀಯಾನಂದರು ತಾವು ಬಾಲ್ಯದಿಂದಲೇ ನಡೆಸುತ್ತಿದ್ದಂತಹ ಕಟ್ಟುನಿಟ್ಟಿನ ಬ್ರಹ್ಮ ಚರ್ಯ ಹಾಗೂ ತಪಸ್ಸಿನ ಜೀವನ, ಪ್ರಜ್ವಲಿಸುತ್ತಿದ್ದ ತ್ಯಾಗ ಮನೋಭಾವ ಮತ್ತು ಉನ್ನತ ಮಟ್ಟದ ಆಧ್ಯಾತ್ಮಿಕ ಸ್ವಭಾವ–ಇವುಗಳಿಂದಾಗಿ ತಮ್ಮ ಸಹಸಂನ್ಯಾಸಿಗಳೆಲ್ಲರ ಪೂಜ್ಯತೆ-ಗೌರವ- ಪ್ರೀತಿಗಳನ್ನು ಗಳಿಸಿದ್ದರು. ಧ್ಯಾನಸಿದ್ಧರೂ ಸಂಸ್ಕೃತದಲ್ಲಿ ಪರಿಣತರೂ ಆಗಿದ್ದ ತುರೀಯಾ ನಂದರು ಆಲಂಬಜಾರ್ ಮಠದ ದಿನಗಳಿಂದಲೇ ತರಗತಿಗಳ ಹಾಗೂ ಸಂಭಾಷಣೆಗಳ ಮೂಲಕ, ಅದಕ್ಕೂ ಮಿಗಿಲಾಗಿ ತಮ್ಮ ಆದರ್ಶ ಜೀವನದ ಮೂಲಕ ಸಂಘದ ಕಿರಿಯ ಸದಸ್ಯರಿಗೆ ಮಾರ್ಗ ದರ್ಶಕರಾಗಿದ್ದರು. ತಮ್ಮೊಂದಿಗೆ ಅಮೆರಿಕೆಗೆ ಬರಬೇಕು ಎಂದು ಸ್ವಾಮೀಜಿ ಕರೆದಾಗ ಅವರು ತಮ್ಮೊಂದಿಗೆ ಕೆಲವು ವೇದಾಂತಕೃತಿಗಳನ್ನು ಕೊಂಡೊಯ್ಯಲು ಇಚ್ಛಿಸಿದರು. ಆಗ ಸ್ವಾಮೀಜಿ ಉದ್ಗರಿಸಿದರು, “ಓಹ್! ಅವರಿಗೆ (ಅಮೆರಿಕದ ಜನರಿಗೆ) ಪುಸ್ತಕಗಳೂ ಪುಸ್ತಕಜ್ಞಾನವೂ ಸಾಕುಸಾಕಾಗಿದೆ! ಅವರು ಇಲ್ಲಿಯವರಿಗೆ ಕ್ಷಾತ್ರಶಕ್ತಿಯನ್ನು ಕಂಡದ್ದಾಗಿದೆ. ನಾನವರಿಗೆ ಈಗ ಬ್ರಾಹ್ಮಣ ತೇಜಸ್ಸನ್ನು ತೋರಿಸಬೇಕಾಗಿದೆ!” ಎಂದರೆ, ಹೋರಾಡುವ ಮತ್ತು ರಕ್ಷಿಸುವ ಮನೋಭಾವವನ್ನು ಪಾಶ್ಚಾತ್ಯರು ತಮ್ಮಲ್ಲಿ ಈಗಾಗಲೇ ಕಂಡುಕೊಂಡಿದ್ದಾರೆ; ಈಗ ಅವರಿಗೆ ಬ್ರಾಹ್ಮಣತ್ವದ ಉಚ್ಚ ಸಂಪ್ರದಾಯಗಳಲ್ಲಿ ಮತ್ತು ಕಟ್ಟುನಿಟ್ಟಿನ ನಿಯಮಗಳಲ್ಲಿ ಹುಟ್ಟಿಬೆಳೆದ ಧ್ಯಾನಶೀಲ ವ್ಯಕ್ತಿಯೊಬ್ಬನನ್ನು ತೋರಿಸಿಕೊಡಬೇಕಾದ ಕಾಲ ಬಂದಿದೆ ಎಂಬುದು ಸ್ವಾಮೀಜಿಯವರ ಅಭಿಪ್ರಾಯ.

ಈ ನಡುವೆ ನಿವೇದಿತೆಯೂ ಸ್ವಾಮೀಜಿಯವರೊಂದಿಗೆ ಇಂಗ್ಲೆಂಡಿಗೆ ಹೋಗುವ ಪ್ರಸ್ತಾಪ ಬಂದಿತ್ತು. ಅದರ ಮುಖ್ಯ ಉದ್ದೇಶ ಅವಳ ಶಾಲೆಗಾಗಿ ಧನಸಂಗ್ರಹಣೆ ಮಾಡುವುದು. ಹೆಣ್ಣು ಮಕ್ಕಳಿಗಾಗಿ ಮತ್ತು ವಿಧವೆಯರಿಗಾಗಿ ಎಂದು ಅವಳು ಸ್ಥಾಪಿಸಿದ್ದ ಶಾಲೆಯ ಸಂಪೂರ್ಣ ಹೊಣೆ ಗಾರಿಕೆ ಅವಳದ್ದೇ ಆಗಿತ್ತು. ಆದರೆ ಕಲ್ಕತ್ತದಲ್ಲಿ ಆ ಶಾಲೆಗೆ ಉದಾರ ಸಹಾಯ ನೀಡಬಲ್ಲ ದಾನಿ ಗಳಾರೂ ಸಿಕ್ಕಿರಲಿಲ್ಲ. ಆದ್ದರಿಂದ ಶಾಲೆ ಹೇಗೋ ಕುಂಟಿಕೊಂಡು ಸಾಗುತ್ತಿತ್ತು. ಅಲ್ಲದೆ ಅಕ್ಷರಾಭ್ಯಾಸಕ್ಕೆಂದು ಬರುತ್ತಿದ್ದ ಹೆಣ್ಣು ಮಕ್ಕಳಿಗೆಲ್ಲ ಮನೆಯವರು ಮದುವೆ ಮಾಡಿ ಅರ್ಧಕ್ಕೆ ಶಾಲೆಯಿಂದ ಬಿಡಿಸಿಬಿಡುತ್ತಿದ್ದರು. ಇದರಿಂದ ನಿವೇದಿತೆಗೆ ತುಂಬ ನಿರಾಸೆ, ಬೇಸರ. ಇದನ್ನು ಕಂಡು ಸ್ವಾಮೀಜಿ ಆಕೆಗೆ ಹೇಳಿದರು, “ನೀನೂ ನನ್ನೊಂದಿಗೆ ಇಂಗ್ಲೆಂಡಿಗೆ ಬಂದುಬಿಡು. ಇಲ್ಲಿ ನೀನು ಎಷ್ಟೇ ಪ್ರಯತ್ನಪಟ್ಟರೂ ನಿನಗೆ ಬೇಕಾಗುವಷ್ಟು ಹಣ ಸಿಗಲಾರದು. ಆದ್ದರಿಂದ ಇಂಗ್ಲೆಂಡಿಗೆ ಬಂದು ಸ್ವಲ್ಪ ಹಣ ಮಾಡಿಕೊಂಡು ಹಿಂದಿರುಗಬಹುದು.” ಅಲ್ಲದೆ ಇದೇ ವೇಳೆಗೆ, ಕೆಲದಿನಗಳ ಮಟ್ಟಿಗೆ ಹಿಂದಿರುಗುವಂತೆ ಅವಳ ಮನೆಯವರಿಂದ ಅವಳಿಗೊಂದು ಪತ್ರ ಬಂದಿತು. ನಿವೇದಿತೆಯ ತಂಗಿ ಒಂದು ಶಾಲೆಯನ್ನು ನಡೆಸುತ್ತಿದ್ದು, ಅದಕ್ಕೆ ಸಂಬಂಧಿಸಿದ ಎಡೆಬಿಡದ ಕಾರ್ಯಗಳಿಂದಾಗಿ ಆಕೆಯ ಮದುವೆಯನ್ನು ನೆರವೇರಿಸಲು ಸಾಧ್ಯವಾಗಿರಲಿಲ್ಲ. ಆದ್ದರಿಂದ ಕೆಲ ಸಮಯದ ಮಟ್ಟಿಗೆ ನಿವೇದಿತಾ ಆ ಜವಾಬ್ದಾರಿಯನ್ನು ವಹಿಸಿಕೊಂಡರೆ ಅವಳ ತಂಗಿ ಮದುವೆ ಯಾಗಲು ಸಾಧ್ಯವಿತ್ತು. ಈ ವಿಷಯ ತಿಳಿದ ಮೇಲೆ ಸ್ವಾಮೀಜಿ, “ಹಾಗಾದರೆ ಬೇರೆ ದಾರಿಯೇ ಇಲ್ಲ. ನೀನು ನನ್ನೊಂದಿಗೆ ಬರಲೇಬೇಕಾಗುತ್ತದೆ” ಎಂದರು. ನಿವೇದಿತಾ ಈ ಬಗ್ಗೆ ಆಲೋಚಿಸಿ ಕಡೆಗೆ ಅದೇ ಸರಿಯೆಂದು ತೀರ್ಮಾನಿಸಿದಳು.

ಈ ಬಾರಿ ಸ್ವಾಮೀಜಿ ಇಂಗ್ಲೆಂಡಿನ ಮೂಲಕ ಅಮೆರಿಕೆಗೆ ಹೋಗುವುದೆಂದೂ, ಇಂಗ್ಲೆಂಡಿನಲ್ಲಿ ಕೆಲ ದಿನ ಮಾತ್ರ ಉಳಿದುಕೊಳ್ಳುವುದೆಂದೂ ನಿಶ್ಚಯವಾಯಿತು. ಸ್ವಾಮೀಜಿಯವರು ಪ್ರಯಾಣ ಹೊರಡಲು ಇನ್ನೂ ಒಂದು ತಿಂಗಳಿಗಿಂತ ಹೆಚ್ಚುಕಾಲ ಇರುವಾಗಲೇ ಭಕ್ತರಿಂದ, ದರ್ಶನಾರ್ಥಿ ಗಳಿಂದ ಮಠ ಗಿಜಿಗುಟ್ಟಲಾರಂಭಿಸಿತ್ತು. ಕಡೆಯ ಘಳಿಗೆಯವರೆಗೂ ಸ್ವಾಮೀಜಿ ನಿರಂತರ ಬೋಧನೆ ಸಲಹೆ ಸೂಚನೆಗಳನ್ನು ನೀಡುವುದರಲ್ಲಿ ನಿರತರಾಗಿದ್ದರು. ಕೆಲವೊಮ್ಮೆ ಅವರು ತಮ್ಮ ಹೃದಯದ ಭಾವವನ್ನು ಧಾರೆಹರಿಸುತ್ತ ಹಾಡುಗಳನ್ನು ಹಾಡುವುದು ಕೇಳಿಬರುತ್ತಿತ್ತು.

ಜೂನ್ ೧೯ರಂದು ಬೇಲೂರು ಮಠದಲ್ಲಿ ಸ್ವಾಮಿಗಳಿಬ್ಬರಿಗೂ ಬೀಳ್ಕೊಡುಗೆಯ ಸಮಾರಂಭ ವನ್ನು ನಡೆಸಲಾಯಿತು. ಸೋದರ ಸಂನ್ಯಾಸಿಗಳು ಸ್ವಾಮಿ ವಿವೇಕಾನಂದರಿಗೂ ತುರೀಯಾನಂದ ರಿಗೂ ಶುಭ ಹಾರೈಸಿದರು. ಸ್ವಾಮಿ ಅಖಂಡಾನಂದರು ಈ ಸಂದರ್ಭಕ್ಕಾಗಿ ಮಾಹುಲಾದಿಂದ ನಾಲ್ವರು ಅನಾಥ ಮಕ್ಕಳೊಂದಿಗೆ ಬಂದಿದ್ದರು. ಶಾರದಾನಂದರು, ಅಖಂಡಾನಂದರು ಹಾಗೂ ತ್ರಿಗುಣಾತೀತಾನಂದರು ಪುಟ್ಟ ಭಾಷಣಗಳನ್ನು ಮಾಡಿದರು. ಸ್ವಾಮಿ ತುರೀಯಾನಂದರು ಇವುಗಳಿಗೆ ಉತ್ತರವಾಗಿ ಅತ್ಯಂತ ಸೂಕ್ತವಾದ ಕೆಲವು ಮಾತುಗಳನ್ನಾಡಿದರು. ಬಳಿಕ ಸ್ವಾಮೀಜಿ ಯವರು ‘ಸಂನ್ಯಾಸ–ಅದರ ಆದರ್ಶ ಮತ್ತು ಅನುಷ್ಠಾನ’ ಎಂಬ ವಿಷಯವಾಗಿ ತುಂಬ ಉಜ್ಜಲ ವಾದ ಭಾಷಣ ಮಾಡಿದರು–“ಸಂನ್ಯಾಸಿಯಾದವನು ಮೃತ್ಯುವನ್ನು ಪ್ರೀತಿಸಬೇಕು. ಆಗ ಮಾತ್ರವೇ ಅವನ ಪ್ರತಿಯೊಂದು ಕರ್ಮವೂ ಇತರರ ಹಿತಕ್ಕಾಗಿಯೇ ಮಾಡಲ್ಪಡುತ್ತದೆ. ಮತ್ತು ಅದು ನಿಃಸ್ವಾರ್ಥವಾಗಿಯೇ ಇರುತ್ತದೆ. ಆದರೆ ತೀರಾ ಉನ್ನತವಾದ ಆದರ್ಶವನ್ನಿಟ್ಟುಕೊಳ್ಳುವುದು ಒಳ್ಳೆಯದಲ್ಲ. ಬೌದ್ಧ ಹಾಗೂ ಜೈನ ಸುಧಾರಕರು ಮಾಡಿದ ತಪ್ಪು ಅದೇ. ತೀರಾ ಸಾಧಾರಣ ಮಟ್ಟದ ಆದರ್ಶವೂ ಅಪೇಕ್ಷಣೀಯವಲ್ಲ. ಈ ಎರಡು ಅತಿಗಳನ್ನೂ ತಪ್ಪಿಸಬೇಕು. ನೀವು ನಿಮ್ಮ ಜೀವನದಲ್ಲಿ ಒಂದು ಅದ್ಭುತ ಆದರ್ಶದೊಂದಿಗೆ ತೀವ್ರ ಸಾಧನೆಯನ್ನು ಸಮ್ಮಿಳನಗೊಳಿಸಿ ಕೊಳ್ಳಲು ಪ್ರಯತ್ನಿಸಬೇಕು. ಈ ಕ್ಷಣಕ್ಕೆ ನೀವು ಗಾಢ ಧ್ಯಾನಮಗ್ನರಾಗಲು ಸಿದ್ಧರಿರಬೇಕು; ಮರು ಕ್ಷಣಕ್ಕೆ ಹೊಲಕ್ಕೆ ಹೋಗಿ ಉಳುವುದಕ್ಕೂ ತಯಾರಾಗಿರಬೇಕು. ಈ ಘಳಿಗೆಯಲ್ಲಿ ಶಾಸ್ತ್ರಗಳ ಸೂಕ್ಷ್ಮ ವಿಚಾರಗಳನ್ನು ವಿವರಿಸಲು ಸಿದ್ಧರಿರಬೇಕು; ಇನ್ನೊಂದು ಘಳಿಗೆಯಲ್ಲಿ ಹೊಲದಲ್ಲಿ ಬೆಳೆದ ಬೆಳೆಯನ್ನು ಮಾರುಕಟ್ಟೆಗೆ ಒಯ್ದು ಮಾರಿಕೊಂಡು ಬರಲೂ ಅಣಿಯಾಗಿರಬೇಕು. ವ್ಯಕ್ತಿ ನಿರ್ಮಾಣವೇ ಮಠದ ಗುರಿಯೆಂಬುದನ್ನು ಪ್ರತಿಯೊಬ್ಬನೂ ನೆನಪಿಡಬೇಕು. ಆಶ್ರಮವಾಸಿ ಗಳೆಲ್ಲರೂ ಪುಷಿಗಳೇ ಆಗಬೇಕು.” ಮತ್ತೆ ಮಾತನ್ನು ಮುಂದುವರಿಸುತ್ತ ನುಡಿದರು, “ಯಾವನು ಅತ್ಯಂತ ಶಕ್ತಿವಂತನಾಗಿದ್ದರೂ ಒಬ್ಬ ಹೆಂಗಸಿನ ಹೃದಯವನ್ನು ಹೊಂದಿರುತ್ತಾನೋ ಅವನೇ ನಿಜವಾದ ಪುರುಷ. ಇನ್ನೊಂದು ಬಹುಮುಖ್ಯವಾದ ಅಂಶವೇನೆಂದರೆ ಪ್ರತಿಯೊಬ್ಬ ಸದಸ್ಯನೂ ಸಂಘದ ಬಗ್ಗೆ ಅತ್ಯುನ್ನತ ಗೌರವಾದರಗಳನ್ನು ಹೊಂದಿರಬೇಕು. ಮತ್ತು ಸಂಘಕ್ಕೆ ವಿಧೇಯ ನಾಗಿರಬೇಕು.” ಹೀಗೆ ಕಡೆಯ ಸೂಚನೆಗಳನ್ನು ಕೊಟ್ಟ ಸ್ವಾಮೀಜಿಯವರು ತಂದೆ ತನ್ನ ಮಕ್ಕಳನ್ನು ನೋಡುವಂತೆ ಸಂಘದ ಸದಸ್ಯರನ್ನೆಲ್ಲ ಒಮ್ಮೆ ವಾತ್ಸಲ್ಯಭರಿತ ದೃಷ್ಟಿಯಿಂದ ವೀಕ್ಷಿಸಿ ಆಶೀರ್ವದಿಸಿದರು.

ಜೂನ್ ೧೭ರಂದು ಸ್ವಾಮೀಜಿ ಇನ್ನಿತರ ಸಂನ್ಯಾಸಿಗಳೊಂದಿಗೆ ಮಾಸ್ಟರ್ ಮಹಾಶಯರ ಆಹ್ವಾನವನ್ನು ಮನ್ನಿಸಿ ಅವರ ಮನೆಗೆ ಹೋದರು. ಅದೇ ದಿನ ಸಂಜೆ ಸರ್ ಜತೀಂದ್ರ ಮೋಹನ ಟಾಗೋರರನ್ನು ಭೇಟಿಯಾದರು. ಪ್ರಯಾಣ ಹೊರಡುವ ಹಿಂದಿನ ದಿನ ಸ್ವಾಮೀಜಿಯವರ ಹಾಗೂ ತುರೀಯಾನಂದರ ಛಾಯಾಚಿತ್ರಗಳನ್ನೂ ಎರಡು ಸಾಮೂಹಿಕ ಛಾಯಾಚಿತ್ರಗಳನ್ನೂ ತೆಗೆದುಕೊಳ್ಳಲಾಯಿತು.

ಸ್ವಾಮೀಜಿ ಪ್ರಯಾಣ ಹೊರಡುವ ದಿನದಂದು (ಜೂನ್ ೨ಂ) ಶ್ರೀಮಾತೆ ಶಾರದಾದೇವಿ ಯವರು ತಮ್ಮೆಲ್ಲ ಸಂನ್ಯಾಸೀ ಪುತ್ರರಿಗೂ ತಮ್ಮ ನಿವಾಸದಲ್ಲಿ ಔತಣವನ್ನೇರ್ಪಡಿಸಿದರು. ಸ್ವಾಮೀಜಿಯವರನ್ನೂ ತುರೀಯಾನಂದರನ್ನೂ ಅವರು ವಿಶೇಷ ವಾತ್ಸಲ್ಯದಿಂದ ಉಪಚರಿಸಿದರು. ಬಳಿಕ ಸ್ವಾಮೀಜಿಯವರು ಶ್ರೀಮಾತೆಯವರ ಆಶೀರ್ವಾದವನ್ನು ಪಡೆದುಕೊಂಡು ಮೂರು ಗಂಟೆಯ ವೇಳೆಗೆ ಕುದುರೆ ಗಾಡಿಯಲ್ಲಿ ಬಂದರಿನ ಕಡೆಗೆ ಹೊರಟರು. ಬಂದರಿನ ಬಳಿಯಲ್ಲಿ ಸ್ವಾಮಿಗಳಿಗೆ ವಿದಾಯ ಹೇಳಲು ಅನೇಕ ಜನ ಸೇರಿದ್ದರು. ಸ್ವಾಮೀಜಿಯವರಂತೂ ಬಹಳ ಆನಂದದ ಭಾವದಲ್ಲಿದ್ದರು. ಅಲ್ಲದೆ ಇತರರಿಗೂ ಹಸನ್ಮುಖಿಗಳಾಗಿರುವಂತೆ ಹೇಳಿದರು. ಸ್ವಾಮೀಜಿ ತಮ್ಮ ಹೃದಯದಲ್ಲಿ ಯಾವಾಗಲೂ ಇದ್ದೇ ಇರುತ್ತಾರೆ ಎಂಬುದು ಭಕ್ತರಿಗೆ ಹಾಗೂ ವಿಶ್ವಾಸಿಗರಿಗೆ ಗೊತ್ತಿದ್ದುದೇ ಆದರೂ ಹೊರಡುವ ವೇಳೆ ಸಮೀಪಿಸಿದಂತೆಲ್ಲ ಅವರ ಹೃದಯ ವನ್ನು ಒಂದು ಬಗೆಯ ಹೇಳಲಾರದ ದುಃಖ ಆವರಿಸಿಬಿಟ್ಟಿತ್ತು.

ಹಡಗು ಕಲ್ಕತ್ತದಿಂದ ಹೊರಡುವುದಾದ್ದರಿಂದ ಪ್ರಯಾಣಿಕರೆಲ್ಲರೂ ಕಟ್ಟುನಿಟ್ಟಿನ ವೈದ್ಯ ಕೀಯ ಪರೀಕ್ಷೆಗೆ ಒಳಗಾಗಬೇಕಾಗಿತ್ತು. ಭಯಂಕರ ಸಾಂಕ್ರಾಮಿಕ ರೋಗವಾದ ಪ್ಲೇಗು ಕಲ್ಕತ್ತ ದಲ್ಲಿ ಸುಳಿದಾಡಿದ್ದರಿಂದ ಆ ರೋಗ ಇತರ ಸ್ಥಳಗಳಿಗೂ ಹಬ್ಬದಂತೆ ತಡೆಯಲು ಈ ಮುನ್ನೆ ಚ್ಚರಿಕೆ. ಅಂತೂ ಪರೀಕ್ಷೆ-ನಿರೀಕ್ಷೆಗಳೆಲ್ಲ ಮುಗಿದ ಮೇಲೆ ಸ್ವಾಮಿಗಳು ಮತ್ತು ನಿವೇದಿತಾ ಹಡಗನ್ನೇರಿದರು. ಮೂವರೂ ಪ್ರಥಮ ದರ್ಜೆಯ ಪ್ರಯಾಣಿಕರು. ಸ್ವಾಮಿ ಶಾರದಾನಂದರ ಸೋದರನಾದ ಸತೀಶಚಂದ್ರನೂ ಇದೇ ಹಡಗಿನಲ್ಲಿ ಹೊರಟಿದ್ದ. ಅವನು ಬಾಸ್ಟನ್ನಿಗೆ ಹೋಗ ಲಿದ್ದವನು. ಜಹಜು ಸಂಜೆ ಐದು ಘಂಟೆಗೆ ಸರಿಯಾಗಿ ಬಂದರನ್ನು ಬಿಟ್ಟು ಹೊರಟಿತು. ಆಗ ಅಲ್ಲಿ ನೆರೆದಿದ್ದ ಭಕ್ತಾದಿಗಳು ತಮ್ಮ ಕಣ್ಣೀರನ್ನು ಹತ್ತಿಕ್ಕಲಾರದೆ ಹೋದರು. ಎಲ್ಲರೂ ಒಂದೇ ಸಲಕ್ಕೆ ದೀರ್ಘದಂಡ ಪ್ರಣಾಮ ಮಾಡಿ ತಮ್ಮ ವಿದಾಯ ಗೌರವವನ್ನು ಸೂಚಿಸಿದರು. ಅಲ್ಲಿದ್ದ ಇತರರೆಲ್ಲ ಇದನ್ನು ಕಂಡು ವಿಸ್ಮಯಮೂಕರಾದರು. ಹಡಗಿನಲ್ಲಿ ನಿಂತಿದ್ದ ಸ್ವಾಮೀಜಿ ಮತ್ತು ಅವರ ಸಂಗಡಿಗರು ಕಣ್ಣಿಗೆ ಕಾಣುತ್ತಿರುವವರೆಗೂ ಸಂನ್ಯಾಸಿಗಳು, ಸ್ನೇಹಿತರು ಮತ್ತು ಭಕ್ತರು ತಮ್ಮ ಕೈಗಳನ್ನು ಇಲ್ಲವೆ ಕರವಸ್ತ್ರಗಳನ್ನು ಬೀಸುತ್ತಲೇ ಇದ್ದರು.

