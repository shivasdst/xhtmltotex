
\chapter{ವಿನೂತನ ಉತ್ಸವಗಳು}

\noindent

ಮಹಾಶಿವರಾತ್ರಿಯ ದಿನ ಬಂದಿದೆ. ನೀಲಾಂಬರ ಮುಖರ್ಜಿಯ ಉದ್ಯಾನಗೃಹದಲ್ಲಿನ ರಾಮ ಕೃಷ್ಣ ಮಠದಲ್ಲೇ ಈ ಕಾರ್ಯಕ್ರಮ ನಡೆಯುವ ಏರ್ಪಾಡಾಗಿದೆ. ಶಿವರಾತ್ರಿ ಕಳೆದ ಮೂರು ದಿನಗಳಿಗೇ ಶ್ರೀರಾಮಕೃಷ್ಣರ ಜಯಂತ್ಯುತ್ಸವ. ಇವೆರಡೂ ಮಠದಲ್ಲಿ ಅತ್ಯಂತ ಸಂಭ್ರಮದಿಂದ ನಡೆಯುವ ಉತ್ಸವಗಳು. ಆದ್ದರಿಂದ ಮಠದಲ್ಲೆಲ್ಲ ವಿಶೇಷ ಉತ್ಸಾಹದ ವಾತಾವರಣ. ಸ್ವಾಮಿ ಶಾರದಾನಂದರು ಅಮೆರಿಕದಿಂದ ಹಿಂದಿರುಗಿ ಬಂದಿದ್ದಾರೆ. ಸಿಲೋನಿನಲ್ಲಿ ವೇದಾಂತ ಪ್ರಸಾರ ಕಾರ್ಯದಲ್ಲಿ ನಿರತರಾಗಿದ್ದ ಸ್ವಾಮಿ ಶಿವಾನಂದರೂ ಆಗಮಿಸಿದ್ದಾರೆ. ದಿನಜ್​ಪುರದಲ್ಲಿ ಬರಗಾಲ ಸಂತ್ರಸ್ತರ ಸೇವಾಕಾರ್ಯವನ್ನು ಪೂರೈಸಿ ಸ್ವಾಮಿ ತ್ರಿಗುಣಾತೀತಾನಂದರೂ ಹಿಂದಿರುಗಿದ್ದಾರೆ. ಇವರೆಲ್ಲ ತಮ್ಮತಮ್ಮ ಕೆಲಸಗಳನ್ನು ಅತ್ಯಂತ ಯಶಸ್ವಿಯಾಗಿ ಮಾಡಿ ಮುಗಿಸಿ ಬಂದದ್ದರಿಂದ ಸ್ವಾಮೀಜಿಯವರಿಗೆ ಬಹಳ ಆನಂದವಾಗಿಬಿಟ್ಟಿದೆ. ಕಲ್ಕತ್ತ ಕೇಂದ್ರದ ಅಧ್ಯಕ್ಷರಾಗಿ ಮಠದ ಕಾರ್ಯವನ್ನು ಕೌಶಲದಿಂದ ನಿರ್ವಹಿಸಿದುದಕ್ಕಾಗಿ ಬ್ರಹ್ಮಾನಂದರನ್ನು ಅವರು ಹಾರ್ದಿಕವಾಗಿ ಅಭಿನಂದಿಸಿದರು. ಮತ್ತು ಬ್ರಹ್ಮಾನಂದರ ಅನುಪಸ್ಥಿತಿಯಲ್ಲಿ ಮಠದ ಕಿರಿಯ ಸಂನ್ಯಾಸಿ-ಬ್ರಹ್ಮ ಚಾರಿಗಳಿಗೆ ತರಬೇತಿ ನೀಡಿದ ಸ್ವಾಮಿ ತುರೀಯಾನಂದರಿಗೂ ತಮ್ಮ ಅಭಿನಂದನೆ ಸಲ್ಲಿಸಿದರು. ಆದರೆ ಇಷ್ಟರಿಂದಲೇ ಸ್ವಾಮೀಜಿಯವರಿಗೆ ತೃಪ್ತಿಯಾಗಲಿಲ್ಲ. ಶಿವರಾತ್ರಿಯಂದು ಮಧ್ಯಾಹ್ನ ಎಲ್ಲ ಆಶ್ರಮವಾಸಿಗಳ ಸಭೆ ಸೇರಿಸಿದರು. ಪ್ರತಿಯೊಬ್ಬ ಹಿರಿಯ ಸಾಧುವಿಗೂ ಧನ್ಯವಾದಗಳ ನ್ನರ್ಪಿಸುವ ಸಲುವಾಗಿ ಇಂಗ್ಲಿಷಿನಲ್ಲಿ ಬಿನ್ನವತ್ತಳೆಗಳನ್ನು ತಯಾರಿಸುವಂತೆ ಕಿರಿಯ ಸಂನ್ಯಾಸಿ- ಬ್ರಹ್ಮಚಾರಿಗಳಿಗೆ ಸೂಚಿಸಿದರು. ಆಶ್ರಮವಾಸಿಗಳ ಒಂದು ಸಭೆಯಲ್ಲಿ ಈ ಬಿನ್ನವತ್ತಳೆಗಳನ್ನು ಸಮರ್ಪಿಸಲಾಯಿತು. ಅಂದಿನ ಸಭೆಯ ಅಧ್ಯಕ್ಷರು ಸ್ವತಃ ಸ್ವಾಮೀಜಿಯವರೇ. ಈಗ ಬಿನ್ನವತ್ತಳೆ ಗಳಿಗೆ ಉತ್ತರ ನೀಡುವಂತೆ ತಮ್ಮ ಸೋದರ ಸಂನ್ಯಾಸಿಗಳಲ್ಲಿ ಒಬ್ಬೊಬ್ಬರನ್ನೇ ಕರೆದರು. ಎಲ್ಲರೂ ತಮ್ಮ ಭಾಷಣಗಳನ್ನು ಮುಗಿಸಿದ ಮೇಲೆ ಸ್ವಾಮೀಜಿ ಅಧ್ಯಕ್ಷಸ್ಥಾನದಿಂದ ಮಾತ ನಾಡುತ್ತ, ತಮ್ಮ ಸೋದರ ಸಂನ್ಯಾಸಿಗಳಿಗೆ ಹಾಗೂ ತಮ್ಮ ಶಿಷ್ಯರಿಗೆ, ಅವರೆಲ್ಲ ವೈಯಕ್ತಿಕವಾಗಿ ಮತ್ತು ಸಾಮೂಹಿಕವಾಗಿ ಯಾವ ಯಾವ ಬಗೆಯ ಕಾರ್ಯಕಲಾಪಗಳನ್ನು ಕೈಗೊಳ್ಳಬೇಕು ಮತ್ತು ಅವುಗಳನ್ನು ಹೇಗೆ ನಿರ್ವಹಿಸಬೇಕು ಎಂಬುದನ್ನು ತಿಳಿಸಿಕೊಟ್ಟರು.

ಶಿವರಾತ್ರಿ ಕಳೆದ ಮೂರು ದಿನಕ್ಕೆ ಶ್ರೀರಾಮಕೃಷ್ಣರ ಜನ್ಮದಿನೋತ್ಸವ. ಇದು ಮುಖ್ಯವಾಗಿ ವಿಶೇಷ ಪೂಜೆ, ಅಲಂಕಾರ, ಹೋಮ, ಭಜನೆ ಮೊದಲಾದ ಧಾರ್ಮಿಕ ಆಚರಣೆಗಳಿಂದ ಕೂಡಿದ ಸಮಾರಂಭ. ಫೆಬ್ರವರಿ ೨೨ರಂದು ಸ್ವತಃ ಸ್ವಾಮೀಜಿಯವರ ಮೇಲ್ವಿಚಾರಣೆಯಲ್ಲಿ ಸಮಾರಂಭ ನಡೆಯಿತು. ಸಂಜೆ ಆರತಿಯ ವೇಳೆಯಲ್ಲಿ, ಸ್ವಾಮೀಜಿಯವರೇ ಹೊಸದಾಗಿ ಶ್ರೀರಾಮಕೃಷ್ಣರನ್ನು ಕುರಿತು ರಚಿಸಿದ್ದ “ಖಂಡನ ಭವಬಂಧನ ಜಗವಂದನ ವಂದಿ ತೋಮಾಯ್​” ಎಂಬ ಹಾಡನ್ನು ತಾಳ-ಮೃದಂಗಗಳೊಂದಿಗೆ ಹಾಡಲಾಯಿತು. ಈ ಹಾಡನ್ನು ಪ್ರತಿದಿನವೂ ಎಲ್ಲ ಶ್ರೀರಾಮಕೃಷ್ಣಾ ಶ್ರಮಗಳಲ್ಲಿ ಸಂಧ್ಯಾರತಿ ವೇಳೆಯಲ್ಲಿ ಹಾಡಲಾಗುತ್ತಿದೆ.

ಶ್ರೀರಾಮಕೃಷ್ಣ ಜಯಂತಿಯ ಆ ದಿನ ಬೆಳಿಗ್ಗೆ ಸುಮಾರು ನೂರು ಯಜ್ಞೋಪವೀತಗಳನ್ನು ತಂದಿಡುವಂತೆ ಸ್ವಾಮೀಜಿ ಹೇಳಿದ್ದರು. ಆದರೆ ಅಷ್ಟೊಂದು ಯಜ್ಞೋಪವೀತಗಳು ಏಕಿರಬಹು ದೆಂದು ಯಾರಿಗೂ ಗೊತ್ತಿರಲಿಲ್ಲ. ಸ್ವಾಮೀಜಿಯವರು ಶ್ರೀರಾಮಕೃಷ್ಣರ ಗೃಹಸ್ಥಶಿಷ್ಯರಲ್ಲಿ ಕ್ಷತ್ರಿಯರಿಗೂ ವೈಶ್ಯರಿಗೂ ಅಂದು ಯಜ್ಞೋಪವೀತಧಾರಣೆ ಮಾಡಿಸಿ ಗಾಯತ್ರೀ ಮಂತ್ರೋ ಪದೇಶ ನೀಡುವ ಯೋಜನೆ ಹಾಕಿಕೊಂಡಿದ್ದರು. ಈ ಸಮಾರಂಭ ಪ್ರಾರಂಭವಾಗುವ ಮುನ್ನ ವಷ್ಟೇ ತಮ್ಮ ಉದ್ದೇಶವನ್ನು ಹೊರಗೆಡವಿದ ಸ್ವಾಮೀಜಿ, ಬಳಿಕ ತಮ್ಮ ಬ್ರಾಹ್ಮಣಶಿಷ್ಯನಾದ ಶರಚ್ಚಂದ್ರನಿಗೆ ಆ ಕಾರ್ಯದ ಜವಾಬ್ದಾರಿಯನ್ನು ಒಪ್ಪಿಸುತ್ತ ಹೇಳಿದರು, “ನೋಡು, ನಮ್ಮ ಶ್ರೀರಾಮಕೃಷ್ಣದೇವರ ಮಕ್ಕಳೆಲ್ಲರೂ ನಿಜಕ್ಕೂ ಬ್ರಾಹ್ಮಣರೇ ಸರಿ. ಅಲ್ಲದೆ ಕ್ಷತ್ರಿಯ ಹಾಗೂ ವೈಶ್ಯರಿಗೂ ಬ್ರಹ್ಮೋಪದೇಶವಾಗಬೇಕೆಂದು ಶಾಸ್ತ್ರಗಳು ಹೇಳುತ್ತವೆ. ಆದರೆ ಕಾಲಾಂತರದಲ್ಲಿ ಅವರು ಕರ್ಮಭ್ರಷ್ಟರಾಗಿದ್ದಾರೆ, ನಿಜ. ಆದರೆ ವೇದಗಳ ಪ್ರಕಾರ ಅವರು ಅದಕ್ಕೆ ತಕ್ಕ ಪ್ರಾಯಶ್ಚಿತ್ತ ಮಾಡಿಕೊಂಡಲ್ಲಿ ಪುನಃ ಬ್ರಹ್ಮೋಪದೇಶಕ್ಕೆ ಅರ್ಹರಾಗಬಲ್ಲರು. ಇಂದು ಶ್ರೀ ರಾಮಕೃಷ್ಣರ ಜನ್ಮದಿನ. ಅವರ ನಾಮೋಪದೇಶವನ್ನು ಪಡೆದ ಪ್ರತಿಯೊಬ್ಬರೂ ಪವಿತ್ರರಾಗು ತ್ತಾರೆ. ಭಕ್ತರಿಗೆಲ್ಲ ಯಜ್ಞೋಪವೀತ ಧಾರಣೆ ಮಾಡಿಸಲು ಇಂದಿನ ದಿನವೇ ಅತ್ಯಂತ ಪ್ರಶಸ್ತ ವಾದುದು. ಆದ್ದರಿಂದ ಕ್ಷತ್ರಿಯ-ವೈಶ್ಯ ಶಿಷ್ಯರಿಗೆ ಇಂದು ಗಾಯತ್ರೀ ಮಂತ್ರೋಪದೇಶ ಮಾಡು. ಇವರೆಲ್ಲರನ್ನೂ ಕ್ರಮೇಣ ಬ್ರಾಹ್ಮಣತ್ವಕ್ಕೆ ಏರಿಸಬೇಕು. ಎಲ್ಲ ಹಿಂದೂಗಳೂ ಸೋದರರು. ನಮ್ಮ ಸೋದರರಲ್ಲಿ ಕೆಲವರನ್ನು ಕೆಳಗಿಳಿಸಿ ನೂರಾರು ವರ್ಷಗಳಿಂದ ಅವರಿಗೆ ‘ನಿಮ್ಮನ್ನು ನಾವು ಮುಟ್ಟುವುದಿಲ್ಲ’ ಎಂದು ಹೇಳಿ ಹೇಳಿ ಅವರನ್ನು ಅಸ್ಪೃಶ್ಯರನ್ನಾಗಿ ಮಾಡಿದವರು ನಾವೇ. ಹೀಗೆ ಮಾಡಿದ್ದರ ಫಲವಾಗಿ ಇಂದು ಇಡೀ ರಾಷ್ಟ್ರವೇ ಅವಹೇಳನಕ್ಕೆ ಅಧಃಪತನಕ್ಕೆ ಗುರಿಯಾಗಿರು ವುದರಲ್ಲಿ ಆಶ್ಚರ್ಯವೇನಿದೆ? ನೀವು ಅವರಿಗೆಲ್ಲ ಭರವಸೆ ಹೇಳಿ, ಅವರಲ್ಲಿ ಉತ್ಸಾಹ ತುಂಬಿ ಅವರನ್ನು ಮೇಲೆತ್ತಬೇಕು. ‘ನೀವೂ ನಮ್ಮಂತೆಯೆ ಮನುಷ್ಯರು; ನಮಗಿರುವ ಅಧಿಕಾರಗಳೆಲ್ಲ ನಿಮಗೂ ಇವೆ’ ಎಂದು ನೀವು ಅವರಿಗೆ ಹೇಳಬೇಕು.”

ಆ ದಿನ ಐವತ್ತಕ್ಕೂ ಹೆಚ್ಚು ಜನ ಗಂಗೆಯಲ್ಲಿ ಸ್ನಾನ ಮಾಡಿ ಬಂದು ಶ್ರೀರಾಮಕೃಷ್ಣರ ಭಾವಚಿತ್ರಕ್ಕೆ ನಮಿಸಿ ಯಜ್ಞೋಪವೀತ ಧಾರಣೆ ಮಾಡಿ ಗಾಯತ್ರೀ ಮಂತ್ರೋಪದೇಶವನ್ನು ಪಡೆದರು. ನಿಜಕ್ಕೂ ಇದು ಅಲ್ಲಿನ ಹಿಂದೂ ಸಂಪ್ರದಾಯಕ್ಕೆ ವಿರುದ್ಧವಾದ ಕಾರ್ಯ. ಸಂಪ್ರ ದಾಯಸ್ಥ ಹಿಂದೂಗಳು ಇದರ ಬಗ್ಗೆ ಆಡಿಕೊಂಡು ಮೂಗು ಮುರಿದದ್ದೂ ಉಂಟು. ಆದರೆ ಸ್ವಾಮೀಜಿ ಇವುಗಳಿಗೆಲ್ಲ ಗಮನ ಕೊಡುವವರಲ್ಲ. ಬದಲಾಗಿ ತಾವು ಯಾವ ನೂತನ ಆದರ್ಶ ವನ್ನು ಸರ್ವರೂ ಅನುಷ್ಠಾನಕ್ಕೆ ತರುವಂತೆ ಮಾಡಬೇಕೆಂದುಕೊಂಡಿದ್ದಾರೋ ಅದನ್ನು ಧೀರತೆ ಯಿಂದ ಮಾಡಿಯೇ ಬಿಡುವವರು.

ಅಂತೂ ಇದೊಂದು ಕಷ್ಟದ ಪರಿಸ್ಥಿತಿಯೇ ಸರಿ. ಸ್ವಾಮೀಜಿ ಮಾಡಿದ ಕಾರ್ಯವನ್ನು ಕಂಡು ಇತ್ತ ಸಂಪ್ರದಾಯಸ್ಥ ಹಿಂದೂಗಳು ಬಗೆಬಗೆಯಿಂದ ಟೀಕಿಸಿದರೆ ಅತ್ತ ಯಜ್ಞೋಪವೀತವನ್ನು ಧರಿಸಿದ ಬ್ರಾಹ್ಮಣೇತರರನ್ನು ಸಮಾಜದ ಜನ, ‘ಓಹೋ! ಇವರೆಲ್ಲ ಈಗ ಜನಿವಾರ ಸಿಕ್ಕಿಸಿ ಕೊಂಡ ಮಾತ್ರಕ್ಕೆ ಬ್ರಾಹ್ಮಣರಾಗಿಬಿಟ್ಟರೊ!’ ಎಂದು ಗೇಲಿ ಮಾಡಿದರು.

ಸ್ವಾಮೀಜಿಯವರು ಆಗಿನ ಕಾಲದಲ್ಲಿ ಪ್ರಚಲಿತವಿದ್ದ ಗೊಡ್ಡು ಸಂಪ್ರದಾಯಗಳನ್ನು ಉಗ್ರ ವಾಗಿ ಖಂಡಿಸುತ್ತಿದ್ದರಾದರೂ ಸಂಪ್ರದಾಯಗಳನ್ನೆಲ್ಲ ಇದ್ದಕ್ಕಿದ್ದಂತೆ ಮುರಿದುಹಾಕಿಬಿಡ ಬೇಕೆಂದು ಹೊರಟವರಲ್ಲ ಅವರು. ಸಂಪ್ರದಾಯ ಎನ್ನುವುದೂ ಕೂಡ ಬಹುಮಟ್ಟಿಗೆ ಶಾಸ್ತ್ರಗಳ ಆಧಾರದ ಮೇಲೆಯೇ ರಚಿತವಾದ್ದರಿಂದ ಈ ಸಂಪ್ರದಾಯಗಳಲ್ಲೂ ಕೆಲವು ಸತ್ವಪೂರ್ಣ ಅಂಶಗಳು ಇಲ್ಲವೆಂದಲ್ಲ. ಈ ಸತ್ವಗಳು ನಷ್ಟವಾಗದೆ ಇನ್ನಷ್ಟು ಬೆಳೆಯುವ ರೀತಿಯಲ್ಲಿ ಸುಧಾರಣೆಯಾಗಬೇಕೆಂಬವರು ಸ್ವಾಮೀಜಿ. ಆದ್ದರಿಂದ ಅವರು ಯಾವ ಸುಧಾರಣೆಗಳನ್ನು ಮಾಡಹೊರಟರೋ ಅವು ಶಾಸ್ತ್ರಗಳಿಗೆ ಅನುಗುಣವಾಗಿಯೇ ಇರುತ್ತಿದ್ದುವು ಮತ್ತು ಶಾಸ್ತ್ರವಾಕ್ಯ ಗಳಿಗೇ ಪುಷ್ಟಿಕೊಡುವಂತಿದ್ದುವು. ಸ್ವಾಮೀಜಿಯವರು ಶಾಸ್ತ್ರಗಳ ಆಳಕ್ಕಿಳಿದು ಅವುಗಳ ತಿರುಳನ್ನು ಅರಿತು ಅವನ್ನು ಧರ್ಮದ ಹಾಗೂ ಜನಾಂಗದ ಶ್ರೇಯಸ್ಸಿಗಾಗಿ ಸರಿಯಾದ ರೀತಿಯಲ್ಲಿ ವಿನಿಯೋಗವಾಗುವಂತೆ ಮಾಡುತ್ತಿದ್ದರು. ರಾಮಕೃಷ್ಣ ಮಹಾಸಂಘವು, ಅನೂಚಾನವಾಗಿ ಕಾಲ ಸಮ್ಮತವಾಗಿ ನಡೆದುಕೊಂಡು ಬಂದ ಧಾರ್ಮಿಕ ಆಚರಣೆಗಳನ್ನು ಕಟ್ಟುನಿಟ್ಟಾಗಿ ಪಾಲಿಸಿಕೊಂಡು ಬರಬೇಕೆಂಬುವವರು ಸ್ವಾಮೀಜಿ. ಉದಾಹರಣೆಗೆ ಶಿವರಾತ್ರಿಯಂದು ಉಪವಾಸ ಮಾಡಬೇಕು ಎನ್ನುವುದು ಅನೂಚಾನವಾಗಿ ನಡೆದುಕೊಂಡು ಬಂದ ಪದ್ಧತಿ. ಆದರೆ ದುರದೃಷ್ಟವಶಾತ್ ಆ ವರ್ಷ ಮಠದ ಯಾವ ಸಾಧು ಬ್ರಹ್ಮಚಾರಿಗಳೂ ಈ ಪದ್ಧತಿಯನ್ನು ಅನುಸರಿಸಿರಲಿಲ್ಲ. ಸ್ವಾಮೀಜಿಯವರಿಗೆ ಇದನ್ನು ಕಂಡು ಬಹಳ ವ್ಯಥೆಯಾಯಿತು. ಮುಂದಾದರೂ ಅದನ್ನು ಬಿಡದೆ ಆಚರಿಸುವಂತೆ ನಿರ್ದೇಶಿಸಿದರು.

ಉಪನಯನದ ಕಾರ್ಯಗಳೆಲ್ಲ ಮುಗಿದ ಬಳಿಕ ಸಾಧು-ಬ್ರಹ್ಮಚಾರಿಗಳೆಲ್ಲ ಶಿವಭಕ್ತಿಯಿಂದ ಆವೇಶಭರಿತರಾಗಿ ಸ್ವಾಮೀಜಿಯವರಿಗೆ ಶಿವನ ವೇಷ ತೊಡಿಸಿದರು. ಅವರ ಮೈಗೆಲ್ಲ ವಿಭೂತಿ ಯನ್ನು ಬಳಿದರು, ಕಿವಿಗಳಿಗೆ ಶಂಖದ ಒಂಟಿಯನ್ನಿಟ್ಟರು, ಮಂಡಿವರೆಗೂ ಜೋತಾಡುವ ಜಟೆಯನ್ನು ಕಟ್ಟಿದರು, ತೋಳುಗಳಿಗೆ ಕಡಗಗಳನ್ನು ತೊಡಿಸಿದರು. ಕೊರಳಿಗೆ ಮೂರು ಸುತ್ತು ಬರುವಂತೆ ರುದ್ರಾಕ್ಷಿ ಮಾಲೆ ಹಾಕಿದರು. ಅವರ ಎಡಗೈಗೆ ತ್ರಿಶೂಲವನ್ನು ಕೊಟ್ಟರು. ಬಳಿಕ ಎಲ್ಲ ಸಂನ್ಯಾಸಿಗಳೂ ತಮ್ಮ ತಮ್ಮ ಮೈಗಳಿಗೂ ವಿಭೂತಿ ಬಳಿದುಕೊಂಡರು. ಈ ದೃಶ್ಯವನ್ನು ಕಂಡು ಅನುಭವಿಸಿದ ಶರಚ್ಚಂದ್ರ ಬರೆಯುತ್ತಾನೆ, “ಅಂದು ಶಿವನ ವೇಷದಲ್ಲಿ ಕಾಣಿಸಿಕೊಂಡ ಸ್ವಾಮೀಜಿ ಯವರ ದಿವ್ಯ ಸೌಂದರ್ಯ ವರ್ಣನಾತೀತವಾಗಿತ್ತು. ಅದನ್ನು ಕಣ್ಣಾರೆ ಕಂಡೇ ಅನುಭವಿಸಬೇಕು! ಅಂದು ಅಲ್ಲಿ ನೆರೆದಿದ್ದವರೆಲ್ಲರೂ ಬಳಿಕ ಹೇಳಿದರು–ತರುಣ ತಪೋಮೂರ್ತಿ ಶಿವನ ಸಮ್ಮುಖ ದಲ್ಲೇ ನಿಂತಂತೆ ತಮಗೆ ಭಾಸವಾಯಿತು ಎಂದು. ತಮ್ಮ ಸುತ್ತ ಭೈರವಗಣಗಳಂತೆ ಕುಳಿತಿದ್ದ ಸಂನ್ಯಾಸಿಗಳ ನಡುವೆ ವಿರಾಜಿಸುತ್ತಿದ್ದ ಸ್ವಾಮೀಜಿ ಮಠದ ಪರಿಸರದಲ್ಲಿ ಪ್ರತ್ಯಕ್ಷ ಕೈಲಾಸದ ದಿವ್ಯ ವಾತಾವರಣವನ್ನೇ ನಿರ್ಮಾಣ ಮಾಡಿಬಿಟ್ಟರು.”

ಶಿವನ ವೇಷದಲ್ಲಿ ಕಂಗೊಳಿಸುತ್ತಿದ್ದ ಸ್ವಾಮೀಜಿ ಈಗ ಶ್ರೀರಾಮನ ಮೇಲೆ ಒಂದು ಸ್ತೋತ್ರವನ್ನು ಹಾಡಿದರು. ಶಿವನ ಇಷ್ಟದೇವತೆ ರಾಮ, ರಾಮನ ಇಷ್ಟದೇವತೆ ಶಿವ. ಆದ್ದ ರಿಂದಲೇ ಅವನು ‘ಶಿವರಾಮ’. ರಾಮನಾಮದಿಂದ ಮತ್ತರಾಗಿ ಸ್ವಾಮೀಜಿ, “ರಾಮ ರಾಮ ಶ್ರೀರಾಮ ರಾಮ, ರಾಮ ರಾಮ ಶ್ರೀರಾಮ ರಾಮ” ಎಂದು ಮತ್ತೆಮತ್ತೆ ಉದ್ಘೋಷಿಸಿದರು. ಸಾಕ್ಷಾತ್ ಶಿವನೇ ಅವರಲ್ಲಿ ಆವಿರ್ಭಾವಗೊಂಡಂತಿತ್ತು. ಅವರ ಮುಖದ ತೇಜಸ್ಸು ಮೊದಲಿ ಗಿಂತಲೂ ನೂರು ಪಟ್ಟು ವರ್ಧಿಸಿ ಪ್ರಕಾಶಿಸತೊಡಗಿತು. ಅವರ ನಯನಗಳು ಅರ್ಧನಿಮೀಲಿತ, ಅರ್ಧೋನ್ಮೀಲಿತ! ಈಗ ಸ್ವಾಮೀಜಿ ಪೀಠದ ಮೇಲೆ ಪದ್ಮಾಸನದಲ್ಲಿ ಕುಳಿತು ತಂಬುರ ನುಡಿಸುತ್ತ ಹಾಡಲಾರಂಭಿಸಿದರು. ಅಲ್ಲಿ ನೆರೆದಿದ್ದ ಸಮಸ್ತ ಸಂನ್ಯಾಸಿಗಳೂ ಗೃಹಸ್ಥರೂ ಸ್ಫೂರ್ತಿಗೊಂಡಿದ್ದರು, ಆಧ್ಯಾತ್ಮಿಕ ಆನಂದದಲ್ಲಿ ಮುಳುಗಿಹೋಗಿದ್ದರು. ಸ್ವಾಮೀಜಿಯವರ ಮುಖಕಮಲದಿಂದ ಹೊರಹೊಮ್ಮುತ್ತಿರುವ ರಾಮನಾಮಸಂಕೀರ್ತನೆಯ ನಾದಾಮೃತದಿಂದ ಸರ್ವರೂ ಮತ್ತರಾಗಿರುವಂತೆ ಕಾಣುತ್ತಿತ್ತು. ಹೀಗೆ ಅರ್ಧಗಂಟೆಗೂ ಹೆಚ್ಚುಕಾಲ ಇದೇ ಆನಂದ ದಲ್ಲಿ ಎಲ್ಲರೂ ಅವಿಚಲರಾಗಿ ಬೊಂಬೆಗಳಂತೆ ಕುಳಿತುಬಿಟ್ಟಿದ್ದರು. ಎಲ್ಲೆಲ್ಲೂ ಪರಮ ಶಾಂತಿ ತಾನೇ ತಾನಾಯಿತು.

ರಾಮನಾಮೋಚ್ಚಾರಣೆ ಮುಗಿದ ಬಳಿಕ ಸ್ವಾಮೀಜಿ ತುಲಸೀದಾಸರ ಕೀರ್ತನೆಯೊಂದನ್ನು ಹಾಗೂ ಶ್ರೀರಾಮಕೃಷ್ಣರಿಗೆ ಪ್ರಿಯವಾದ ಕೆಲವು ಹಾಡುಗಳನ್ನು ಹಾಡಿದರು. ಈಗ ಸ್ವಾಮಿ ಶಾರದಾನಂದರು, ಸ್ವಾಮೀಜಿಯವರೇ ಬರೆದ ಹಾಡೊಂದನ್ನು ಹಾಡಲಾರಂಭಿಸಿದರು. ಅದಕ್ಕೆ ಜೊತೆಯಾಗಿ ಸ್ವಾಮೀಜಿ ಖೋಲ್ ಬಾಜಿಸಿದರು. ಈ ಹೊತ್ತಿಗೆ ಸ್ವಾಮೀಜಿಯವರಿಗೆ ಏನೆನ್ನಿ ಸಿತೋ ಏನೋ, ತಮ್ಮ ಮೈಮೇಲಿದ್ದ ಶಿವನ ಉಡುಗೆ ತೊಡಿಗೆಗಳನ್ನೆಲ್ಲ ಒಂದೊಂದಾಗಿ ಕಳಚಿ ಅವುಗಳನ್ನು ಅಲ್ಲೇ ಇದ್ದ ಗಿರೀಶ್​ಚಂದ್ರನಿಗೆ ತೊಡಿಸಿದರು; ಅವನ ಮೈಗೆ ವಿಭೂತಿಯನ್ನು ಬಳಿದರು. ಎಲ್ಲರೂ ಆಶ್ಚರ್ಯದಿಂದ ನೋಡುತ್ತಿದ್ದರು. ಗಿರೀಶನೂ ಮಂತ್ರಮುಗ್ಧನಂತೆ ಸುಮ್ಮನೆ ನಿಂತಿದ್ದ! ಕೊನೆಯದಾಗಿ ಗಿರೀಶನಿಗೆ ಕಾವಿ ವಸ್ತ್ರವನ್ನು ತೊಡಿಸುತ್ತ ಸ್ವಾಮೀಜಿ ನುಡಿದರು, “ಶ್ರೀರಾಮಕೃಷ್ಣರು ಹೇಳುತ್ತಿದ್ದರು–ನಮ್ಮ ಜಿ.ಸಿ.ಯಲ್ಲಿ (ಗಿರೀಶಚಂದ್ರನಲ್ಲಿ) ಸ್ವಲ್ಪ ಭೈರವನ ಅಂಶವಿದೆ ಎಂದು. ನಿಜಕ್ಕೂ ಜಿ.ಸಿ.ಗೂ ನಮಗೂ (ಸಂನ್ಯಾಸಿಗಳಿಗೂ) ಭೇದವೇ ಇಲ್ಲ!” ಈ ಮಾತನ್ನು ಕೇಳಿ ಆ ಮಹಾಭಕ್ತನ ಕಣ್ಣಲ್ಲಿ ಅಶ್ರು ತುಂಬಿ ಬಂದಿತು. ಈಗ ಸ್ವಾಮೀಜಿ ಅವನಿಗೆ ಹೇಳಿದರು, “ಜಿ. ಸಿ., ನೀವೀಗ ನಮಗೆ ಶ್ರೀರಾಮಕೃಷ್ಣರ ಕುರಿತಾಗಿ ಏನಾದರೂ ಸ್ವಲ್ಪ ಹೇಳಿ.” ಬಳಿಕ ಅಲ್ಲಿ ನೆರೆದಿದ್ದವರೆಡೆಗೆ ತಿರುಗಿ, “ನಾವೆಲ್ಲ ಈಗ ಶಾಂತವಾಗಿ ಕುಳಿತು ಇವರು ಹೇಳುವುದನ್ನು ಆಲಿಸೋಣ” ಎಂದರು. ಆದರೆ ಗಿರೀಶ ಮಾತ್ರ ಬೊಂಬೆಯಂತೆ ಅಲುಗಾಡದೆ ಕುಳಿತುಬಿಟ್ಟ. ಅವನಿಂದ ಮಾತನಾಡುವುದಕ್ಕೇ ಸಾಧ್ಯವಾಗುತ್ತಿಲ್ಲ! ಕಾರಣ ಅವನ ಎದೆಯಲ್ಲಿ ಅಪೂರ್ವ ಆನಂದ ತುಂಬಿಕೊಂಡಿದೆ. ಸಾಧು-ಬ್ರಹ್ಮಚಾರಿಗಳೆಲ್ಲರ ಕಣ್ಣುಗಳು ಅವನಲ್ಲಿಯೇ ನೆಟ್ಟಿವೆ. ಕೊನೆಗೂ ಗಿರೀಶ ಬಾಯ್ತೆರೆದ. ಆದರೆ ಅವನು ಹೇಳಿದ್ದಿಷ್ಟೆ, “ಅಪಾರ ಕರುಣಾಸಾಗರ ರಾದ ಶ್ರೀರಾಮಕೃಷ್ಣರ ಬಗ್ಗೆ ದೀನನಾದ ನಾನು ಏನು ತಾನೆ ಹೇಳಬಲ್ಲೆ! ಹುಟ್ಟಿದಂದಿನಿಂದಲೇ ಕಾಮ-ಕಾಂಚನವನ್ನು ತ್ಯಾಗ ಮಾಡಿ, ಬಾಲ್ಯದಿಂದಲೂ ಪರಮಪರಿಶುದ್ಧ ಜೀವನ ನಡೆಸಿದ ಇಂತಹ ಸಂನ್ಯಾಸಿಗಳೊಂದಿಗೆ ಸರಿಸಮನಾಗಿ ಕುಳಿತು ಅವರೊಂದಿಗೆ ಬೆರೆಯುವ ಭಾಗ್ಯವನ್ನು ಆ ಶ್ರೀರಾಮಕೃಷ್ಣರು ನನ್ನಂತಹ ಕ್ಷುದ್ರ ಜೀವಿಗೆ ದಯಪಾಲಿಸಿದ್ದಾರಲ್ಲ! ನಾನು ಅವರ ಅಪಾರ ಕರುಣೆಯನ್ನು ಕಾಣುವುದು ಇಲ್ಲಿಯೇ!” ಹೀಗೆ ಹೇಳುತ್ತಿದ್ದಂತೆಯೇ ಆತನ ಕಂಠ ಬಿಗಿದುಬಂತು, ಸ್ವರ ಗದ್ಗದವಾಯಿತು. ಮುಂದೆ ಮಾತೇ ಹೊರಡಲಿಲ್ಲ.

ಬಳಿಕ ಸ್ವಾಮೀಜಿ ಒಂದು ಹಿಂದೀ ಹಾಡನ್ನು ಹಾಡಿದರು. ಆಮೇಲೆ ಪ್ರಸಾದ ವಿನಿಯೋಗ ವಾಯಿತು. ಈಗ ಸ್ವಾಮೀಜಿ ಕೆಳ ಅಂತಸ್ತಿಗೆ ಬಂದು ಕುಳಿತುಕೊಂಡರು. ಅವರನ್ನು ಅನುಸರಿಸಿ ಭಕ್ತಾದಿಗಳೂ ಬಂದು ಕುಳಿತರು. ಈ ಹೊತ್ತಿಗೆ ಸರಿಯಾಗಿ, ಮುರ್ಷಿದಾಬಾದಿನಲ್ಲಿ ಅನಾಥಾಶ್ರಮ ವನ್ನು ನಡೆಸುತ್ತಿದ್ದ ಸ್ವಾಮಿ ಅಖಂಡಾನಂದರು ಅಲ್ಲಿಗೆ ಆಗಮಿಸಿದರು. ಅವರನ್ನುದ್ದೇಶಿಸಿ ಸ್ವಾಮೀಜಿ ಅಲ್ಲಿ ನೆರೆದಿದ್ದವರಿಗೆ ಹೇಳಿದರು, “ನೋಡಿ ಇವರನ್ನು! ಎಂಥ ಕರ್ಮಯೋಗಿ ಇವರು! ಮರಣಭಯವೂ ಇಲ್ಲದ ನಿರ್ಭೀತರು! ಬಹುಜನರ ಹಿತಕ್ಕಾಗಿ, ಬಹುಜನರ ಸುಖ ಕ್ಕಾಗಿ ತಮ್ಮಷ್ಟಕ್ಕೆ ತಾವು ಕೆಲಸ ಮಾಡುತ್ತಲೇ ಹೋಗುತ್ತಿದ್ದಾರೆ.”

ಆಗ ಒಬ್ಬ ಶಿಷ್ಯ ಕೇಳಿದ: “ಸ್ವಾಮೀಜಿ, ಅವರು ಅಧಿಕವಾಗಿ ತಪಸ್ಸು ಮಾಡಿದ್ದರ ಫಲವಾಗಿ ಅವರಲ್ಲೀಗ ಬಹಳ ಶಕ್ತಿ ಬಂದಿರಬೇಕಲ್ಲವೆ?”

ಸ್ವಾಮೀಜಿ: “ಹೌದು, ತಪಸ್ಸಿನಿಂದ ಶಕ್ತಿ ಬರುತ್ತದೆ. ಆದರೆ ನೋಡು, ಇತರರ ಹಿತಕ್ಕಾಗಿ ಕೆಲಸ ಮಾಡುವುದೂ ಕೂಡ ತಪಸ್ಸೇ ಸರಿ. ಕರ್ಮಯೋಗಿಗಳು ಕರ್ಮವೂ ಕೂಡ ತಪಸ್ಸಿನ ಒಂದಂಶ ಎಂದು ಪರಿಗಣಿಸುತ್ತಾರೆ. ಒಂದು ವಿಧದಿಂದ ಈ ತಪಸ್ಸು ಎನ್ನುವುದು ಭಕ್ತನ ಮನಸ್ಸಿನಲ್ಲಿ ಲೋಕಹಿತಾರ್ಥವಾದ ಪರೋಪಕಾರ ಬುದ್ಧಿಯನ್ನು ಹುಟ್ಟಿಸಿ ಅವನನ್ನು ನಿಷ್ಕಾಮ ಕರ್ಮಕ್ಕೆ ಪ್ರಚೋದಿಸಿದರೆ, ಇನ್ನೊಂದು ವಿಧದಿಂದ ಕರ್ಮಯೋಗಿಯ ಕರ್ಮವು ತಪಸ್ಸಾಗಿ ಪರಿಣಮಿಸಿ, ಅದರಿಂದ ಮನಶ್ಶುದ್ಧಿ-ಹೃದಯಶುದ್ಧಿ ಪ್ರಾಪ್ತವಾಗುತ್ತದೆ; ಮತ್ತು ಈ ಶುದ್ಧ ಮನಸ್ಸೇ ಅವನನ್ನು ಆತ್ಮಸಾಕ್ಷಾತ್ಕಾರದೆಡೆಗೆ ಕರೆದೊಯ್ಯುತ್ತದೆ.”

ಶಿಷ್ಯ: “ಆದರೆ, ಸ್ವಾಮೀಜಿ, ಮೊದಲ ಮೆಟ್ಟಿಲಿಗೇ ಪರಹಿತ ದೃಷ್ಟಿಯಿಂದ ಕರ್ಮಮಾಡಬಲ್ಲ ವರು ನಮ್ಮಲ್ಲಿ ಎಷ್ಟು ಜನ ಸಿಕ್ಕಾರು? ಅಲ್ಲದೆ ತಮ್ಮ ಸ್ವಸುಖವನ್ನು ತ್ಯಾಗ ಮಾಡಿ ಪರರ ಹಿತಕ್ಕಾಗಿ ಶ್ರಮಿಸುವಂತಹ ಉದಾರಬುದ್ಧಿ ಉದಯವಾಗುವುದು ಅಷ್ಟೊಂದು ಸುಲಭವೆ?”

ಸ್ವಾಮೀಜಿ: “ಆದರೆ ಎಷ್ಟು ಜನರ ಮನಸ್ಸು ತಾನೆ ತಪಸ್ಸಿನ ಕಡೆಗೆ ಹರಿಯುತ್ತಿದೆ ಎಂದು ತಿಳಿದುಕೊಂಡೆ? ಈ ಕಾಮಕಾಂಚನ ಎನ್ನುವುದು ಮನಸ್ಸನ್ನು ಪ್ರಪಂಚದ ಕಡೆಗೆ ಎಳೆಯುತ್ತಿರು ವಾಗ ಎಷ್ಟು ಜನ ತಾನೆ ಭಗವಂತನ ಸಾಕ್ಷಾತ್ಕಾರಕ್ಕಾಗಿ ಹಂಬಲಿಸಿಯಾರೆಂದು ತಿಳಿದುಕೊಂಡೆ? ನಿಜಕ್ಕೂ ಈ ನಿಷ್ಕಾಮಕರ್ಮವೂ ತಪಸ್ಸಿನಷ್ಟೇ ಕಷ್ಟಕರವಾದದ್ದು. ಆದ್ದರಿಂದ ನೀನು ಪರಹಿತ ಕ್ಕಾಗಿ ಕರ್ಮಮಾಡುವವರ ವಿರುದ್ಧವಾಗಿ ಮಾತನಾಡುವಂತಿಲ್ಲ. ನಿನಗೆ ತಪಸ್ಸು ಮಾಡುವುದು ಇಷ್ಟವಾದರೆ ನೀನದನ್ನು ಮಾಡು. ಇನ್ನೊಬ್ಬರ ಪಾಲಿಗೆ ನಿಷ್ಕಾಮಕರ್ಮವೇ ಅನುಕೂಲವಾಗ ಬಹುದು; ಅದನ್ನು ತಡೆಯಲು ನಿನಗೆ ಅಧಿಕಾರವಿಲ್ಲ. ಒಟ್ಟಿನಲ್ಲಿ ನಿಷ್ಕಾಮಕರ್ಮ ಎನ್ನುವುದು ತಪಸ್ಸಲ್ಲ ಎಂಬ ಭಾವನೆ ನಿನ್ನ ತಲೆಯೊಳಗೆ ಗಟ್ಟಿಯಾಗಿ ನಿಂತುಬಿಟ್ಟಿದೆ.”

ಸ್ವಾಮೀಜಿಯವರ ಮಾತುಗಳನ್ನು ಕೇಳಿದ ಶಿಷ್ಯ ಈಗ ಹೇಳಿದ, “ನಿಜ ಸ್ವಾಮೀಜಿ, ಇಲ್ಲಿಯ ವರೆಗೂ ನಾನು ತಪಸ್ಸು ಎನ್ನುವುದನ್ನು ಬೇರೆಯೇ ರೀತಿಯಾಗಿ ಅರ್ಥಮಾಡಿಕೊಂಡಿದ್ದೆ.” ತಮ್ಮ ಮಾತನ್ನು ಮುಂದುವರಿಸುತ್ತ ಸ್ವಾಮೀಜಿ ಹೇಳಿದರು, “ನೋಡು, ನಾವು ನಮ್ಮ ಆಧ್ಯಾತ್ಮಿಕ ಸಾಧನೆಗಳನ್ನು ಸತತವಾಗಿ ಮಾಡಿಕೊಂಡು ಬರುತ್ತಿರುವಾಗ ಅವುಗಳ ವಿಷಯದಲ್ಲಿ ನಮ್ಮ ಲ್ಲೊಂದು ವಿಶೇಷ ಸಂಸ್ಕಾರವುಂಟಾಗುತ್ತದೆ, ಒಲವು ಬೆಳೆಯುತ್ತದೆ. ಹಾಗೆಯೇ, ನಿಷ್ಕಾಮಕರ್ಮ ವನ್ನು ಮಾಡುತ್ತ ಮಾಡುತ್ತ ಕ್ರಮೇಣ ನಮ್ಮ ವೈಯಕ್ತಿಕ ಇಚ್ಛೆ ಎನ್ನುವುದು ಅದರಲ್ಲಿ ಕರಗಿ ಹೋಗುವುದನ್ನು ಕಾಣುತ್ತೇವೆ. ಈ ರೀತಿಯಲ್ಲಿ ನಮಗೆ ನಿಷ್ಕಾಮಕರ್ಮದಲ್ಲಿ ಅಭಿರುಚಿ ಬೆಳೆಯುತ್ತದೆ, ಗೊತ್ತಾಯಿತೆ? ನೀನು ಕೂಡ ಯಾವುದಾದರೊಂದು ನಿಸ್ವಾರ್ಥಕರ್ಮವನ್ನು– ಅದು ನಿನಗೆ ಇಷ್ಟವಿಲ್ಲದಿದ್ದರೂ ಸರಿಯೆ–ಮಾಡಿಕೊಂಡು ಬರುತ್ತಿರು; ಆಮೇಲೆ ನೋಡು, ಅದು ನಿನಗೆ ನಿಜವಾದ ತಪಸ್ಸಿನ ಫಲವನ್ನು ತಂದುಕೊಡುತ್ತದೆಯೋ ಇಲ್ಲವೋ ಎಂದು. ಪರರಿಗಾಗಿ ಕರ್ಮ ಮಾಡುತ್ತ ಹೋಗುವಾಗ, ಮನಸ್ಸಿನ ಒರಟುತನವೆಲ್ಲ ಸವೆದು ಹೋಗಿ ಮೃದುವಾಗುತ್ತದೆ, ನಯವಾಗುತ್ತದೆ; ಮತ್ತು ಕ್ರಮೇಣ ಪ್ರಾಮಾಣಿಕವಾದ ಆತ್ಮಸಮರ್ಪಣ ಭಾವದಿಂದ ಪರಹಿತಕ್ಕಾಗಿ ಕರ್ಮಮಾಡುವ ಸಾಮರ್ಥ್ಯ ಬರುತ್ತದೆ.”

ಶಿಷ್ಯ: “ಆದರೆ, ಪರರಿಗೆ ಸಹಾಯ ಮಾಡುವುದರ, ಪರಹಿತಕ್ಕಾಗಿ ಕರ್ಮ ಮಾಡುವುದರ ಅಗತ್ಯವಾದರೂ ಏನಿದೆ?”

ಸ್ವಾಮೀಜಿ: “ನೋಡು, ಅದು ನಮ್ಮ ಒಳ್ಳೆಯದಕ್ಕಾಗಿಯೇ! ಇತರರಿಗೆ ಒಳಿತು ಮಾಡುವು ದರಿಂದ ನಿಜಕ್ಕೂ ಒಳ್ಳೆಯದಾಗುವುದು ನಮಗೇ. ನಾವು ನಮ್ಮ ಶರೀರದೊಂದಿಗೆ ತಾದಾತ್ಮ್ಯ ಭಾವ ಬೆಳೆಸಿಕೊಂಡುಬಿಟ್ಟಿದ್ದೇವೆ. ಅದು ಹೋಗಬೇಕಾದರೆ ಈ ಶರೀರವನ್ನು ಇತರರ ಹಿತಕ್ಕಾಗಿ- ಸೇವೆಗಾಗಿ ವಿನಿಯೋಗಿಸಬೇಕು. ಆಗ ನಾವು ನಮ್ಮ ಅಹಂಕಾರದೊಂದಿಗೆ ನಮ್ಮ ಶರೀರವನ್ನೂ ಮರೆತುಬಿಡುತ್ತೇವೆ. ಹೀಗೆಯೇ ಮಾಡುತ್ತ ಮಾಡುತ್ತ ಕಾಲಕ್ರಮದಲ್ಲಿ ಮುಂದೆ ಒಂದಲ್ಲ ಒಂದು ದಿನ ಜೀವನ್ಮುಕ್ತ ಸ್ಥಿತಿ ಪ್ರಾಪ್ತವಾಗುತ್ತದೆ. ನೀನು ಇತರರ ಹಿತಕ್ಕಾಗಿ ಎಷ್ಟೆಷ್ಟು ತೀವ್ರ ವಾಗಿ ಭಾವಿಸುತ್ತೀಯೋ ಅಷ್ಟಷ್ಟು ತೀವ್ರವಾಗಿ ನಿನ್ನ ಅಹಂಕಾರ ಕಣ್ಮರೆಯಾಗುತ್ತದೆ. ಈ ರೀತಿಯಾಗಿ, ಇಂತಹ ನಿಷ್ಕಾಮಕರ್ಮದ ಮೂಲಕ ನಿನ್ನ ಹೃದಯವು ಪರಿಶುದ್ಧವಾಗುವುದರಿಂದ, ಕಾಲಕ್ರಮದಲ್ಲಿ, ಸರ್ವಜೀವರಲ್ಲೂ ನೀನೇ ವ್ಯಾಪಿಸಿಕೊಂಡಿರುವಂತಹ ಒಂದು ದಿವ್ಯ ಸ್ಥಿತಿ ನಿನ್ನ ಅರಿವಿಗೆ ಬರುತ್ತದೆ. ಆದ್ದರಿಂದ ಈ ಬಗೆಯ ಕರ್ಮವು ಆತ್ಮಸಾಕ್ಷಾತ್ಕಾರಕ್ಕೆ ಒಂದು ಪ್ರಧಾನವಾದ ಸಾಧನ. ಮತ್ತು ಈ ನಿಷ್ಕಾಮ ಕರ್ಮದ ಉದ್ದೇಶವೇ ಅದು–ಆತ್ಮಸಾಕ್ಷಾತ್ಕಾರ. ಜ್ಞಾನಯೋಗ ಭಕ್ತಿಯೋಗವೇ ಮೊದಲಾದ ಮಾರ್ಗಗಳಿಂದ ಯಾವ ಆತ್ಮಸಾಕ್ಷಾತ್ಕಾರವೆಂಬ ಗುರಿಯನ್ನು ಮುಟ್ಟಬಹುದೋ, ಅದನ್ನು ಪರಹಿತಕ್ಕಾಗಿ ಮಾಡುವ ನಿಷ್ಕಾಮಕರ್ಮದಿಂದಲೂ ಮುಟ್ಟಲು ಸಾಧ್ಯ.”

ಶಿಷ್ಯ: “ಆದರೆ, ಸ್ವಾಮೀಜಿ, ನಾನು ಇತರರಿಗೆ ಹಿತವನ್ನುಂಟುಮಾಡುವ ಪ್ರಯತ್ನದಲ್ಲಿ, ಹಗಲಿರುಳೂ ಇತರರನ್ನು ಕುರಿತೇ ಆಲೋಚಿಸುವಂತಾದಾಗ ಆತ್ಮವನ್ನು ಕುರಿತು ಧ್ಯಾನ ಮಾಡುವುದು ಯಾವಾಗ? ನಾನು ಈ ಅನಿತ್ಯವಾದ ವಿಷಯಗಳಲ್ಲೇ ನಿರತನಾಗಿದ್ದುಬಿಟ್ಟರೆ, ನಿತ್ಯವಾದ ಆತ್ಮವನ್ನು ಸಾಕ್ಷಾತ್ಕರಿಸಿಕೊಳ್ಳಲು ಹೇಗೆ ಸಾಧ್ಯ?”

ಸ್ವಾಮೀಜಿ: “ಎಲ್ಲ ಆಧ್ಯಾತ್ಮಿಕ ಸಾಧನೆಗಳ, ಎಲ್ಲ ಧಾರ್ಮಿಕ ಕರ್ಮಗಳ ಪ್ರಧಾನ ಉದ್ದೇಶ ಯಾವುದೆಂದರೆ ಜ್ಞಾನಪ್ರಾಪ್ತಿ, ಆತ್ಮಸಾಕ್ಷಾತ್ಕಾರ. ನೀನೀಗ ಬಹುಜನ ಹಿತಕ್ಕಾಗಿ ಸೇವೆ ಸಲ್ಲಿಸುತ್ತ ಹೃದಯವನ್ನು ಶುದ್ಧಿಮಾಡಿಕೊಂಡಾಗ ನಿನಗೆ, ಸಕಲ ಜೀವರಲ್ಲೂ ಇರುವ ಆತ್ಮವು ನೀನೇ ಎಂಬ ಜ್ಞಾನವುಂಟಾಗುತ್ತದೆ. ಇಂತಹ ಮಹತ್ತರ ಜ್ಞಾನ ನಿನಗುಂಟಾದ ಮೇಲೆ ಇನ್ನು ನೀನು ಪಡೆಯ ಬೇಕಾದದ್ದೇನಿದೆ? ನಿನ್ನ ಅಭಿಪ್ರಾಯದಲ್ಲಿ ಆತ್ಮಸಾಕ್ಷಾತ್ಕಾರವೆಂದರೇನು? ಆ ಗೋಡೆಯಂತೆ ಅಥವಾ ಈ ಮರದ ಕೊರಡಿನಂತೆ ಜಡವಾಗಿ ಕುಳಿತುಕೊಂಡಿರುವುದು ಎಂದೆ?”

ಶಿಷ್ಯ: “ಹಾಗಲ್ಲ... ಆದರೂ ಶಾಸ್ತ್ರಗಳು ಹೇಳುವಂತೆ ಕೆಲಸ ಕಾರ್ಯಗಳನ್ನೆಲ್ಲ ಕನಿಷ್ಠಕ್ಕಿಳಿಸಿ, ಚಿತ್ತವೃತ್ತಿಯನ್ನು ನಿರೋಧಿಸಿ ಆತ್ಮವನ್ನು ಆತ್ಮದಲ್ಲಿ ನೆಲೆಗೊಳಿಸಬೇಕಲ್ಲವೆ?”

ಸ್ವಾಮೀಜಿ: “ನಿಜ. ಆದರೆ, ಶಾಸ್ತ್ರಗಳು ಹೇಳುವ ಈ ಸಮಾಧಿಸ್ಥಿತಿಯು ಸುಲಭದ ವಿಷಯ ವಲ್ಲ. ಒಂದು ವೇಳೆ ಅಪರೂಪಕ್ಕೆ ಯಾರಾದರೂ ಅದನ್ನು ಸಿದ್ಧಿಸಿಕೊಂಡರೂ ಕೂಡ, ಅದು ಬಹಳ ಹೊತ್ತು ಇರುವುದಿಲ್ಲ; ಮನಸ್ಸು ಕೆಳಕ್ಕೆ ಬಂದುಬಿಡುತ್ತದೆ. ಆಗ ಅವನು ಆ ಮನಸ್ಸನ್ನು ಎಲ್ಲಿಡುವುದು? ಆದ್ದರಿಂದ, ಅಂತಹ ಸಮಾಧಿಸ್ಥಿತಿಯನ್ನು ಪ್ರಾಪ್ತಿಮಾಡಿಕೊಂಡಂತಹ ಸಂತನಾ ದವನು, ಸರ್ವಜೀವರಲ್ಲೂ ತನ್ನ ಆತ್ಮವನ್ನೇ ಕಂಡು ಸರ್ವರ ಹಿತಕ್ಕಾಗಿ ನಿಷ್ಕಾಮಕರ್ಮದಲ್ಲಿ ನಿರತನಾಗುತ್ತಾನೆ. ಮತ್ತು ಆ ಮೂಲಕ, ತನ್ನ ಅಳಿದುಳಿದ ಪ್ರಾರಬ್ಧಕರ್ಮವನ್ನು ಸವೆಸುತ್ತಾನೆ. ಇದನ್ನೇ ಶಾಸ್ತ್ರಗಳು ಜೀವನ್ಮುಕ್ತಿ ಎಂದು ಕರೆಯುವುದು.”

ಶಿಷ್ಯ: “ಹಾಗಾದರೆ, ಈ ಜೀವನ್ಮುಕ್ತಿಯನ್ನು ಪಡೆಯದೆ ನಿಜವಾದ ಅರ್ಥದಲ್ಲಿ ಪರಹಿತಕ್ಕಾಗಿ ನಿಷ್ಕಾಮಸೇವೆ ಮಾಡಲು ಸಾಧ್ಯವಿಲ್ಲ ಎಂದಂತಾಯಿತು?”

ಸ್ವಾಮೀಜಿ: “ನಿಜ, ಶಾಸ್ತ್ರಗಳು ಹಾಗೆ ಹೇಳುತ್ತವೆ. ಆದರೆ ಅವು ಪರಹಿತಕ್ಕಾಗಿ ನಿಷ್ಕಾಮಕರ್ಮ ಮಾಡುವುದರ ಮೂಲಕ ಜೀವನ್ಮುಕ್ತಿ ಪ್ರಾಪ್ತವಾಗುತ್ತದೆ ಎಂಬುದನ್ನೂ ಹೇಳುತ್ತವೆ. ಇಲ್ಲದೆ ಹೋಗಿದ್ದರೆ, ಈ ಶಾಸ್ತ್ರಗಳು ನಮಗೆ ಕರ್ಮಯೋಗ ಎನ್ನುವ ಬೇರೆಯೇ ಆದ ಒಂದು ಸಾಧನೆ ಯನ್ನು ಕುರಿತು ಬೋಧಿಸಬೇಕಾಗಿಯೇ ಇರಲಿಲ್ಲ.”

ಸ್ವಾಮೀಜಿಯವರ ಅಭಿಪ್ರಾಯವನ್ನು ಅರಿತು ಶಿಷ್ಯ ಸುಮ್ಮನಾದ.

ಶ್ರೀರಾಮಕೃಷ್ಣರ ಜನ್ಮದಿನೋತ್ಸವಕ್ಕೆ ಸಂಬಂಧಿಸಿದಂತೆ ಬೇರೊಂದು ದಿನ ಒಂದು ಸಾರ್ವ ಜನಿಕ ಸಮಾರಂಭವನ್ನು ನಡೆಸುವ ಸಂಪ್ರದಾಯ ಬೆಳೆದುಬಂದಿದೆ. ಈ ಸಮಾರಂಭವನ್ನು ೧೮೮೧ರಲ್ಲೇ, ಎಂದರೆ ಶ್ರೀರಾಮಕೃಷ್ಣರು ಜೀವಿಸಿದ್ದಾಗಲೇ, ಸುರೇಂದ್ರನಾಥ ಮೊದಲಾದ ಭಕ್ತರು ಪ್ರಾರಂಭಿಸಿದ್ದರು. ಅಂದಿನಿಂದಲೂ ಅದು ನಡೆಯುತ್ತಿದ್ದುದು ದಕ್ಷಿಣೇಶ್ವರದಲ್ಲೇ. ಈ ಮೂಲಕ ಶ್ರೀರಾಮಕೃಷ್ಣರ ಭಕ್ತರೆಲ್ಲರೂ ಸಮಾನೋದ್ದೇಶದಿಂದ ಸೇರಿ, ಭಾವವಿನಿಮಯ ಮಾಡಿಕೊಳ್ಳುವುದಕ್ಕೊಂದು ಸದವಕಾಶವಾಗಿತ್ತು. ಆದರೆ ಈ ವರ್ಷ (೧೮೯೮) ದಕ್ಷಿಣೇಶ್ವರ ದೇವಾಲಯದ ಅಧಿಕಾರಿಯಾದ ತ್ರೈಲೋಕ್ಯ ಬಾಬು ವಿವೇಕಾನಂದರಿಗೆ ಹಾಗೂ ಅವರ ಸೋದರ ಸಂನ್ಯಾಸಿಗಳಿಗೆ ದೇವಾಲಯವನ್ನು ಪ್ರವೇಶಿಸದಂತೆ ಬಹಿಷ್ಕಾರ ಹಾಕಿದ್ದ. ಆದ್ದರಿಂದ ಆ ಉತ್ಸವವನ್ನು ತಮ್ಮದೇ ಆದಂತಹ ಬೇರೊಂದು ಸ್ಥಳದಲ್ಲಿ ನಡೆಸಬೇಕೆಂದು ಎಲ್ಲರೂ ಆಶಿಸಿ ದರು. ಹೊಸದಾಗಿ ಕೊಂಡಿದ್ದ ನಿವೇಶನದಲ್ಲಿ ಈ ಉತ್ಸವವನ್ನು ಮಾಡೋಣವೆಂದರೆ ಅದು ಹಳ್ಳತಿಟ್ಟುಗಳಿಂದ ಕೂಡಿತ್ತು. ಆದ್ದರಿಂದ, ಕೊನೆಗೆ ಈ ವರ್ಷದ ಉತ್ಸವವನ್ನು ಬೇಲೂರಿನಿಂದ ಕೆಲವೇ ಮೈಲಿ ದೂರದಲ್ಲಿರುವ, ಪೂಣಚಂದ್ರ ದಾವ್ ಎಂಬವರಿಗೆ ಸೇರಿದ ಸ್ಥಳದಲ್ಲಿ ಆಚರಿ ಸುವುದೆಂದು ತೀರ್ಮಾನವಾಯಿತು. ಈ ಸ್ಥಳವಿರುವುದು ಗಂಗೆಯ ದಡದ ಮೇಲಿನ ಬಾಲಿ ಎಂಬಲ್ಲಿ. ಎಲ್ಲರೂ ಬೇಲೂರಿನಿಂದ ಅಲ್ಲಿಗೆ ಹೋಗಲು ಸ್ಟೀಮರಿನ ವ್ಯವಸ್ಥೆ ಮಾಡಲಾಯಿತು. ಅಂದು ಭಾನುವಾರ. ಸಹಸ್ರಾರು ಜನ ಬಂದು ಅತ್ಯಂತ ವೈಭವಯುತವಾದ ಈ ಸಮಾರಂಭದಲ್ಲಿ ಪಾಲ್ಗೊಂಡರು. ಸಂಭ್ರಮದಿಂದ, ಭಕ್ತಿಶ್ರದ್ಧೆಗಳಿಂದ ವಾರ್ಷಿಕೋತ್ಸವವನ್ನು ಆಚರಿಸಲಾಯಿತು. ಸ್ವಾಮೀಜಿಯವರಿಗೆ ಬಹಿಷ್ಕಾರ ಹಾಕಿದ್ದ ತ್ರೈಲೋಕ್ಯಬಾಬುವಿಗೆ ಇದೊಂದು ದೊಡ್ಡ ಅಪಮಾನ. ಬಹುಶಃ ಅಲ್ಲಿ ಸೇರಿದ್ದ ಸಾವಿರಾರು ಜನರಲ್ಲಿ ಒಬ್ಬನಿಗೂ ಆತನ ಹೆಸರು ಕೂಡ ನೆನಪಿಗೆ ಬಂದಿರಲಾರದು. ಅಂದಿನ ಸಮಾರಂಭದ ಒಂದು ವೈಶಿಷ್ಟ್ಯವೆಂದರೆ ಪಾಶ್ಚಾತ್ಯ ಶಿಷ್ಯೆಯರಾದ ಮಿಸ್ ಮುಲ್ಲರ್, ಶ್ರೀಮತಿ ಓಲೆ ಬುಲ್, ಮಿಸ್ ಜೊಸೆಫಿನ್ ಮೆಕ್ಲಾಡ್ ಹಾಗೂ ಮಿಸ್ ಮಾರ್ಗರೆಟ್ ನೋಬೆಲ್ ಪಾಲ್ಗೊಂಡಿದ್ದು. ಅಲ್ಲಿ ಸೇರಿದ್ದ ಜನಸಂದಣಿ ಸ್ವಾಮೀಜಿಯವರನ್ನು ಒತ್ತಾಯಿಸಿದ್ದರಿಂದ ಅವರು ಸಂದರ್ಭೋಚಿತವಾಗಿ ಒಂದು ಪುಟ್ಟ ಭಾಷಣ ಮಾಡಿದರು. ಬಳಿಕ ನೂರಾರು ಜನರಿಗೆ ಭಕ್ಷ್ಯಸಮೇತವಾದ ಭೋಜನ ಕೊಡಲಾಯಿತು.

