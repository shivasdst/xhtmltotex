
\chapter{ಮೊಳಗಿತು ರಣಕಹಳೆ!}

\noindent

ಸ್ವಾಮಿ ವಿವೇಕಾನಂದರು ಮದ್ರಾಸಿಗೆ ಬರುವುದಕ್ಕಿಂತ ಎಷ್ಟೋ ವಾರಗಳಿಗೆ ಮೊದಲೇ ಅಲ್ಲಿ ಉತ್ಸಾಹ-ಕುತೂಹಲ-ಕಾತರ-ಸಂಭ್ರಮದ ವಿಶೇಷ ವಾತಾವರಣವೊಂದು ನಿರ್ಮಾಣಗೊಂಡಿತ್ತು. ಸ್ವಾಮೀಜಿಯವರಿಗೆ ಅತ್ಯಂತ ಭವ್ಯವಾದ ಸ್ವಾಗತ ನೀಡಲು ಆಗಲೇ ಸಿದ್ಧತೆಗಳು ಭರದಿಂದ ಪ್ರಾರಂಭವಾದುವು. ಜನವರಿ ತಿಂಗಳಲ್ಲೇ ‘ವಿವೇಕಾನಂದ ಸ್ವಾಗತ ಸಮಿತಿ’ ನಿರ್ಮಾಣವಾಯಿತು. ಜಸ್ಟಿಸ್ ಸುಬ್ರಹ್ಮಣ್ಯ ಅಯ್ಯರರು ಅದರ ಅಧ್ಯಕ್ಷರು. ನಗರದ ಅತ್ಯಂತ ಗಣ್ಯವ್ಯಕ್ತಿಗಳಾದ ಸರ್ ವಿ. ಭಾಷ್ಯಂ ಅಯ್ಯಂಗಾರ್, ವಿ. ಕೃಷ್ಣಸ್ವಾಮಿ ಅಯ್ಯರ್, ವಿ. ಸಿ. ಶೇಷಾಚಾರಿಯರ್, ಪ್ರೊ ॥ ಸುಂದರರಾಮ ಅಯ್ಯರ್, ಡಾ ॥ ನಂಜುಂಡರಾವ್ ಮೊದಲಾದವರಲ್ಲದೆ ಸ್ವಾಮೀಜಿಯವರ ಆಪ್ತ ಶಿಷ್ಯರಾದ ಅಳಸಿಂಗ ಪೆರುಮಾಳ್, ಸಿಂಗಾರವೇಲು ಮೊದಲಿಯಾರ್, ಬಾಲಾಜಿರಾವ್ ಮೊದಲಾದವರೆಲ್ಲ ಆ ಸಮಿತಿಯ ಉತ್ಸಾಹಯುತ ಕಾರ್ಯಕರ್ತರು. ಸ್ವಾಮೀಜಿ ಆಗಮಿಸು ತ್ತಿರುವ ಸುದ್ದಿಯನ್ನು ಕರಪತ್ರಗಳ ಮೂಲಕ ನಗರದಲ್ಲೆಲ್ಲ ಪ್ರಚಾರ ಮಾಡಿದರು. ಅಲ್ಲದೆ ಪಾಶ್ಚಾತ್ಯ ದೇಶಗಳಲ್ಲಿ ಸ್ವಾಮೀಜಿಯವರ ಸಾಧನೆಯ ಬಗ್ಗೆ ತಿಳಿಸುವ ಪತ್ರಗಳನ್ನೂ ಹಂಚಿದರು.

ಸ್ವಾಮೀಜಿಯವರ ಆಗಮನಕ್ಕೆ ಎಷ್ಟೋ ವಾರಗಳ ಹಿಂದಿನಿಂದಲೇ ಜನರು ಪತ್ರಿಕೆಗಳಲ್ಲಿ ಪ್ರತಿದಿನವೂ ಅವರ ಬಗ್ಗೆ ಓದಿ ತಿಳಿಯುತ್ತಿದ್ದರು. ಇತರ ಸ್ಥಳಗಳಲ್ಲಿ ಅವರಿಗೆ ನೀಡಲಾಗುತ್ತಿದ್ದ ಆದರದ ಸ್ವಾಗತದ ಬಗ್ಗೆ ಓದುತ್ತಿದ್ದರು. ಅಲ್ಲದೆ ಹಲವಾರು ಪತ್ರಿಕೆಗಳು ಸ್ವಾಮೀಜಿಯವರ ಆಗಮನಕ್ಕೆ ಮುಂಚೆಯೇ ಅವರ ಬಗ್ಗೆ ಸಂಪಾದಕೀಯಗಳನ್ನು ಬರೆದು ಸ್ವಾಗತಿಸಿದುವು. ಅವರ ಹಿರಿಮೆಯನ್ನು ಮೊದಲು ಗುರುತಿಸಿದವರೇ ಮದ್ರಾಸಿಗರು. ಇಲ್ಲಿನ ಉತ್ಸಾಹೀ ಯುವಕರೇ ಅಲ್ಲವೆ ಅವರನ್ನು ಅಮೆರಿಕೆಗೆ ಕಳಿಸಿಕೊಡಲು ಪಣತೊಟ್ಟು ಸಾಧಿಸಿದುದು? ಆದ್ದರಿಂದ ಮದ್ರಾಸಿನ ಜನ ‘ಸ್ವಾಮೀಜಿ ತಮ್ಮವರು’ ಎಂದೇ ಭಾವಿಸಿದ್ದರು. ಎಲ್ಲೆಲ್ಲಿ ನೋಡಿದರೂ ಅವರದೇ ವಿಷಯ. ಯಾರಿಬ್ಬರು ಭೇಟಿಯಾದರೂ ಕೇಳಿಬರುತ್ತಿದ್ದ ಪ್ರಶ್ನೆ ಇದು–“ಮದ್ರಾಸಿಗೆ ಸ್ವಾಮೀಜಿ ಯಾವತ್ತು ಬರುತ್ತಾರಂತೆ?” ಎಂದು. ಅಲ್ಲದೆ ‘ಸ್ವಾಮೀಜಿ ಮದ್ರಾಸಿನಲ್ಲಿ ಒಂದು ತಿಂಗಳು ಉಳಿದುಕೊಳ್ಳುತ್ತಾರಂತೆ!’ ಎಂಬಿತ್ಯಾದಿ ಮಾತುಗಳೂ ಕೇಳಿಬರುತ್ತಿದ್ದುವು. ತಮ್ಮ ನಗರದಲ್ಲಿ ಅವರೊಂದು ಶಾಶ್ವತ ಕೇಂದ್ರವನ್ನು ತೆರೆಯಲಿದ್ದಾರೆಂದೂ ಪ್ರಚಾರವಾಗಿತ್ತು. ಈ ಎಲ್ಲ ಲೆಕ್ಕಾಚಾರಗಳಿಂದಾಗಿ ಜನರ ಉತ್ಸಾಹ ಮತ್ತಷ್ಟು ಹೆಚ್ಚಾಗಿತ್ತು. ಹಿಂದೆ ಮದ್ರಾಸಿನಲ್ಲಿ ಯಾವುದೇ ವ್ಯಕ್ತಿಗೆ ನೀಡಲಾದ ಸ್ವಾಗತವನ್ನು ಸಪ್ಪೆಯಾಗಿಸುವಂತಹ ಸ್ವಾಗತವನ್ನು ನೀಡಲು ಅವರು ನಿರ್ಧರಿಸಿದರು. ಹಿಂದೂ ಧರ್ಮದ ಪುನರುತ್ಥಾನಕ್ಕಾಗಿ, ನವಭಾರತದ ನಿರ್ಮಾಣಕ್ಕಾಗಿ ಸ್ವಾಮೀಜಿ ಏನೇನು ಮಾಡಿದರು, ಹೇಗೆ ತಮ್ಮನ್ನೇ ಅರ್ಪಿಸಿಕೊಂಡರು ಎಂಬುದನ್ನು ಕಂಡು, ಭಾವಿಸಿ, ಜನ ಭಾವೋದ್ವೇಗಗೊಂಡರು; ಅವರ ರಾಷ್ಟ್ರಪ್ರೇಮ ಜಾಗೃತವಾಯಿತು. ಸ್ವಾಮೀಜಿ ಯವರ ಬಗೆಗಿನ ಅಭಿಮಾನ ಹುಚ್ಚು ಹೊಳೆಯಾಗಿ ಹರಿಯಿತು. ಆದರೆ ಅದೊಂದು ‘ಸಮೂಹ ಸನ್ನಿ’ ಅಲ್ಲ; ಪಾವಿತ್ರ್ಯಕ್ಕೆ, ಮಾಹಾತ್ಮ್ಯಕ್ಕೆ ಜನರ ಹೃದಯ ತೋರಿದ ಸ್ವಾಭಾವಿಕ ಪ್ರತಿಕ್ರಿಯೆ ಅದು.

ಸ್ವಾಮೀಜಿ ಮದ್ರಾಸಿಗೆ ಬರುವ ದಿನ ಸಮೀಪಿಸಿದಂತೆ ಸ್ವಾಗತದ ಸಿದ್ಧತೆಗಳು ಪೂರ್ಣ ರಭಸದಿಂದ ನಡೆದುವು. ರಸ್ತೆಗಳನ್ನೂ ರೈಲು ನಿಲ್ದಾಣವನ್ನೂ ವೈಭವಪೂರ್ಣವಾಗಿ ಅಲಂಕರಿಸ ಲಾಗಿತ್ತು. ಎಗ್​ಮೋರ್ ನಿಲ್ದಾಣದಿಂದ ಹಿಡಿದು, ಸ್ವಾಮೀಜಿಯವರ ವಾಸ್ತವ್ಯಕ್ಕಾಗಿ ಸಿದ್ಧಗೊಳಿಸ ಲಾಗಿದ್ದ ಬಂಗಲೆಯವರೆಗೂ ಹದಿನೇಳು ದಿಗ್ವಿಜಯದ ಕಮಾನುಗಳನ್ನು ರಚಿಸಲಾಗಿತ್ತು. ಜಯ ಘೋಷವನ್ನು ಸಾರುವ ಫಲಕಗಳನ್ನು ಎಲ್ಲೆಲ್ಲೂ ತೂಗು ಹಾಕಲಾಗಿತ್ತು. ವೈವಿಧ್ಯಮಯವಾದ ಈ ಫಲಕಗಳಲ್ಲಿ ಬರೆಯಲಾಗಿತ್ತು: “ಪರಮ ಪೂಜ್ಯರಾದ ಸ್ವಾಮಿ ವಿವೇಕಾನಂದರು ದೀರ್ಘಾಯು ಗಳಾಗಲಿ!” “ಭಗವಂತನ ಸೇವಕನಿಗೆ ಜಯವಾಗಲಿ!” “ಜಾಗೃತ ಭಾರತದ ಹೃತ್ಪೂರ್ವಕ ಸ್ವಾಗತ!” “ಶಾಂತಿದೂತನಿಗೆ ಜಯವಾಗಲಿ!” “ಮನುಕುಲದ ಸಾರ್ವಭೌಮನಿಗೆ ಸುಸ್ವಾಗತ!” “ಶ್ರೀರಾಮಕೃಷ್ಣರ ಸತ್ಪುತ್ರನಿಗೆ ಜಯವೆನ್ನಿ!” ಅಲ್ಲದೆ ಸಂಸ್ಕೃತದಲ್ಲಿ “ಏಕಂ ಸತ್ ವಿಪ್ರಾ ಬಹುಧಾ ವದಂತಿ” ಇತ್ಯಾದಿ ಫಲಕಗಳಿದ್ದುವು.

ಮದ್ರಾಸಿನಲ್ಲಿ ಸ್ವಾಮೀಜಿ ಹಾಗೂ ಅವರ ಸಂಗಡಿಗರಿಗಾಗಿ ಸಮುದ್ರತೀರದ ‘ಕ್ಯಾಸಲ್ ಕರ್ನನ್​’ ಎಂಬ ಭವ್ಯ ಕಟ್ಟಡದ ನೆಲಅಂತಸ್ತನ್ನು ಅನುವು ಮಾಡಲಾಗಿತ್ತು. ನಾಲ್ಕಂತಸ್ತುಗಳ ಈ ಕಟ್ಟಡದ ಮಾಲಿಕರಾದ ಶ್ರೀ ಬಿಳಿಗಿರಿ ಅಯ್ಯಂಗಾರರು ಪ್ರಸಿದ್ಧ ವಕೀಲರು. ಅವರು ಬಡ ವಿದ್ಯಾರ್ಥಿಗಳ ಹಾಗೂ ವಿಧವೆಯರ ಸಹಾಯಾರ್ಥವಾಗಿ ಈ ಕಟ್ಟಡವನ್ನು ಬಿಟ್ಟುಕೊಟ್ಟಿದ್ದರು. ನೆಲ ಅಂತಸ್ತಿನಲ್ಲಿ, ಆರು ಕೋಣೆಗಳಲ್ಲದೆ ಅತ್ಯಂತ ಸುಂದರವಾಗಿ ಅಲಂಕರಿಸಲ್ಪಟ್ಟ ಹಜಾರ ಹಾಗೂ ಸಮುದ್ರಕ್ಕೆದುರಾದ ಒಂದು ವರಾಂಡ—ಇವಿಷ್ಟು ಇದ್ದುವು. ದರ್ಶನಾರ್ಥಿಗಳು ಸ್ವಾಮೀಜಿ ಯವರನ್ನು ಭೇಟಿ ಮಾಡಲು ಕಟ್ಟಡದ ಮುಂದೆಯೇ ಒಂದು ಚಪ್ಪರವನ್ನು ಹಾಕಲಾಯಿತು. ಸ್ವಾಮೀಜಿಯವರ ಇರುವಿಕೆಯಿಂದ ಪವಿತ್ರವಾದ ಈ ಕಟ್ಟಡಕ್ಕೆ ಈಗ “ವಿವೇಕಾನಂದ ಹೌಸ್​” ಎಂದು ಹೆಸರಾಗಿದೆ.

೧೮೯೭ರ ಫೆಬ್ರುವರಿ ೬ನೇ ತಾರೀಕು ಬೆಳಗ್ಗೆ ಸ್ವಾಮೀಜಿಯವರ ಟ್ರೈನು ಮದ್ರಾಸಿನ ಎಗ್​ಮೋರ್ ನಿಲ್ದಾಣವನ್ನು ತಲುಪಿತು. ಎಷ್ಟೋ ಗಂಟೆಗಳ ಮೊದಲೇ ಪ್ಲಾಟ್​ಫಾರಂ ಟಿಕೆಟ್ಟನ್ನು ಕೊಂಡು ಅಲ್ಲಿ ನೆರೆದಿದ್ದ ಸಾವಿರಾರು ಜನ ಆನಂದೋತ್ಸಾಹದಿಂದ ಮಾಡಿದ ಕರತಾಡನವು ಸಿಡಿಲು ಬಡಿದಂತಿತ್ತು ಎಂದು ಆಗ ಅಲ್ಲಿದ್ದವರು ಹೇಳಿದ್ದಾರೆ. ಸ್ವಾಮೀಜಿ ಬೋಗಿ ಯಿಂದಾಚೆಗೆ ಬರುತ್ತಿದ್ದಂತೆಯೇ ಜೈಕಾರ-ಹರ್ಷೋದ್ಗಾರಗಳ ಕೂಗು ಮುಗಿಲು ಮುಟ್ಟಿತು. ಹಾರ-ತುರಾಯಿಗಳನ್ನರ್ಪಿಸಲು ಜನ ಮುನ್ನುಗ್ಗಿದರು. ಸ್ವಾಗತ ಸಮಿತಿಯ ಅಧ್ಯಕ್ಷರಾದ ಜಸ್ಟಿಸ್ ಸುಬ್ರಹ್ಮಣ್ಯ ಅಯ್ಯರರು ಸ್ವಾಮೀಜಿಯವರ ಕೊರಳಿಗೆ ಸುಂದರವಾದ ಮಲ್ಲಿಗೆಯ ಹಾರವನ್ನು ತೊಡಿಸಿದರು. ಹೇಗೋ ದಾರಿ ಮಾಡಿಕೊಂಡು ಮುಂದೆ ಬಂದಿದ್ದ ಪ್ರೊ ॥ ಸುಂದರರಾಮ ಅಯ್ಯರ್ ಸಾಷ್ಟಾಂಗ ಪ್ರಣಾಮ ಮಾಡಿ, “ಸ್ವಾಮೀಜಿ, ನನ್ನ ನೆನಪಿದೆಯೇ?” ಎಂದು ಕೇಳಿದರು. ಅದಕ್ಕೆ ಸ್ವಾಮೀಜಿ, “ನಾನು ಒಮ್ಮೆ ನೋಡಿದ ಮುಖವನ್ನು ಎಂದೂ ಮರೆಯುವುದಿಲ್ಲ!” ಎಂದುತ್ತರಿಸಿ, ತಿರುವನಂತಪುರದಲ್ಲಿ ತಾವು ಅವರ ಮನೆಯಲ್ಲಿ ಕಳೆದ ದಿನಗಳನ್ನು ಸ್ಮರಿಸಿ ಕೊಂಡರು. ಬಹಳ ಕಷ್ಟದಿಂದ ಸ್ವಾಮೀಜಿಯವರನ್ನು ಸಾರೋಟಿನ ಬಳಿಗೆ ಕರೆತರಲಾಯಿತು. ಅವರು ಶಿವಾನಂದರು, ನಿರಂಜನಾನಂದರು, ಗುಡ್​ವಿನ್ ಮತ್ತು ಜಸ್ಟಿಸ್ ಅಯ್ಯರರೊಂದಿಗೆ ಸಾರೋಟನ್ನೇರಿ ಕುಳಿತಾಗ ಅಲ್ಲಿಯೇ ಒಂದು ಸ್ವಾಗತವನ್ನು ಓದಲಾಯಿತು. ಕಣ್ಸೆಳೆಯುವಂತಿದ್ದ ಕೇಸರಿ ನಿಲುವಂಗಿ ಹಾಗೂ ರುಮಾಲಿನಿಂದ ಅತ್ಯಂತ ಮನಮೋಹಕರಾಗಿ ಕಾಣುತ್ತಿದ್ದ ಸ್ವಾಮೀಜಿ ಯವರನ್ನು ಕಂಡಕೂಡಲೇ ನಿಲ್ದಾಣದ ಹೊರಗೆ ನಿಂತಿದ್ದ ಇತರ ಸಾವಿರಾರು ಜನರಿಗಾದ ಆನಂದ ಅಷ್ಟಿಷ್ಟಲ್ಲ.

ಸ್ವಾಮೀಜಿಯವರ ಸಾರೋಟನ್ನು ಎರಡು ಸುಂದರ ಶ್ವೇತಾಶ್ವಗಳು ಎಳೆದುವು. ಅದನ್ನು ಹಿಂಬಾಲಿಸಿ ಸಾರೋಟುಗಳ ದಂಡೇ ನಡೆಯಿತು. ಮೊದಲೇ ನಿರ್ಧಾರಿತವಾಗಿದ್ದ ದಾರಿಯಲ್ಲಿ ಮೆರವಣಿಗೆ ನಿಧಾನವಾಗಿ ಸಾಗಿತು. ಮುನ್ನುಗ್ಗಿ ಬರುತ್ತಿದ್ದ ಜನಸಂದಣಿಯನ್ನು ತಡೆಗಟ್ಟಲು ಹಲವು ನೂರು ಮಂದಿ ಪೋಲೀಸರು ಶ್ರಮಿಸುತ್ತಿದ್ದರು. ಆದರೆ ಸ್ವಾಗತದ ಪತ್ರಗಳನ್ನು ಓದುವವರಿಗೆ ಅನುಕೂಲ ಮಾಡಿಕೊಡಲು ಮೆರವಣಿಗೆ ಮತ್ತೆ ಮತ್ತೆ ನಿಲ್ಲುತ್ತಿತ್ತು. ದಾರಿ ಯುದ್ದಕ್ಕೂ ನಿರಂತರವಾಗಿ ಪುಷ್ಪವೃಷ್ಟಿಯಾಗುತ್ತಲೇ ಇತ್ತು. ಅಲ್ಲದೆ ಉತ್ಸವಮೂರ್ತಿಯನ್ನು ಬರಮಾಡಿಕೊಳ್ಳುವಂತೆ, ಅವರಿಗೆ ದಾರಿಯುದ್ದಕ್ಕೂ ಅಲ್ಲಲ್ಲಿ ಆರತಿ ಬೆಳಗಿ, ಧೂಪ-ಫಲ-ಪುಷ್ಪ- ಗಂಧಗಳನ್ನು ಸಮರ್ಪಿಸಲಾಯಿತು. ಒಬ್ಬಳು ಉನ್ನತ ಕುಲದ ಮಹಿಳೆಯಂತೂ ಜನರನ್ನು ತಳ್ಳಿಕೊಂಡು ಬಂದು ಸ್ವಾಮೀಜಿಯವರನ್ನೊಮ್ಮೆ ಕಣ್ತುಂಬ ನೋಡಿ ಅತ್ಯಂತ ಭಕ್ತಿಭಾವದಿಂದ ನಮಸ್ಕರಿಸಿದಳು. ಅವರನ್ನು ಆಕೆ ತಮಿಳುನಾಡಿನ ಶೈವ ಸಂತನಾದ ಸಂಬಂಧಮೂರ್ತಿಯ ಅವತಾರವೆಂದೆ ನಂಬಿದ್ದಳು. ಅವರ ದರ್ಶನ ಮಾತ್ರದಿಂದಲೇ ತನ್ನೆಲ್ಲ ಪಾಪರಾಶಿಯೂ ನಾಶವಾಗುತ್ತದೆಂದು ಆಕೆ ತಿಳಿದಿದ್ದಳು. ಈ ಮಧ್ಯೆ ತಮ್ಮ ಉತ್ಸಾಹವನ್ನು ತಡೆದುಕೊಳ್ಳಲಾರದ ಕಾಲೇಜು ವಿದ್ಯಾರ್ಥಿಗಳು ಸ್ವಾಮೀಜಿಯವರ ಸಾರೋಟಿನ ಕುದುರೆಗಳನ್ನು ಬಿಡಿಸಿ, ಅದನ್ನು ತಾವೇ ಸಂಭ್ರಮದಿಂದ ಎಳೆತಂದರು.

ಮೆರವಣಿಗೆ ಕ್ಯಾಸಲ್ ಕರ್ನನ್ ತಲುಪುತ್ತಿದ್ದಂತೆ, ಅಲ್ಲಿಯೇ ಸಿದ್ಧಪಡಿಸಲಾಗಿದ್ದ ಚಪ್ಪರದ ವೇದಿಕೆಯ ಮೇಲಕ್ಕೆ ಸ್ವಾಮೀಜಿ ಹಾಗೂ ಅವರ ಸಂಗಡಿಗರನ್ನು ಕರೆತರಲಾಯಿತು. ಆಗ ‘ಮದ್ರಾಸ್ ವಿದ್ವನ್ ಮನೋರಂಜಿನೀ ಸಭೆ’ಯ ಪರವಾಗಿ ಒಬ್ಬರು ಸಂಸ್ಕೃತದಲ್ಲೊಂದು ಬಿನ್ನವತ್ತಳೆಯನ್ನು ಓದಿದರು. ಬಳಿಕ ಅಳಸಿಂಗ ಪೆರುಮಾಳರೇ ಮೊದಲಾದ ಉತ್ಸಾಹೀ ಕನ್ನಡಿ ಗರು ಒಂದು ಅಭಿನಂದನಾ ಪತ್ರವನ್ನು ಸಮರ್ಪಿಸಿದರು. ಸ್ವಾಮೀಜಿಯವರ ಆಶೀರ್ವಚನ ದೊಂದಿಗೆ ಸಭೆ ಮುಕ್ತಾಯಗೊಂಡಿತು. ಸ್ವಾಮೀಜಿ ವಿಶ್ರಾಂತಿ ಪಡೆದುಕೊಳ್ಳಲು ಸಾಧ್ಯವಾಗು ವಂತೆ ಜನರೆಲ್ಲ ಚದುರಬೇಕೆಂಬ ಸುಬ್ರಹ್ಮಣ್ಯ ಅಯ್ಯರರ ಮನವಿಯಂತೆ ಸಭೆ ವಿಸರ್ಜನೆಗೊಂಡಿತು.

ಸ್ವಾಮೀಜಿಯ ಊಟ ವಿಶ್ರಾಂತಿಗಳು ಮುಗಿದ ಮೇಲೆ ಪ್ರೊ ॥ ಸುಂದರರಾಮ ಅಯ್ಯರರು ಹಾಗೂ ಪ್ರೊ ॥ ರಂಗಾಚಾರ್ಯರು ಸ್ವಾಮೀಜಿಯವರ ಬಳಿಗೆ ಬಂದು ಮುಂದಿನ ಕಾರ್ಯಕ್ರಮ ಗಳ ಬಗ್ಗೆ ಸಮಾಲೋಚನೆ ನಡೆಸಿದರು. ಸ್ವಾಮೀಜಿಯವರು, “ನೀವು ಏನೇನು ಕಾರ್ಯಕ್ರಮ ಗಳನ್ನು ಹಾಕಿಕೊಳ್ಳಬೇಕೆಂದು ನಿರ್ಧರಿಸುತ್ತೀರೋ ಅದಕ್ಕೆಲ್ಲ ನನ್ನ ಒಪ್ಪಿಗೆಯಿದೆ. ನಾನು ಯಾವ ವಿಷಯದ ಮೇಲೆ ಮಾತನಾಡಬೇಕೋ ಅದರ ಬಗ್ಗೆ ನನಗೊಂದು ಮಾತನ್ನು ತಿಳಿಸಿ” ಎಂದರು. ಕೊನೆಗೆ, ಸ್ವಾಮೀಜಿಯವರು ಸ್ವಾಗತ ಸಮಿತಿಯ ಬಿನ್ನವತ್ತಳೆಗೆ ಮೊದಲು ಉತ್ತರ ಕೊಡುವುದು, ಆಮೇಲೆ ಭಾರತಕ್ಕೆ ಅವರು ನೀಡುವ ಸಂದೇಶದ ಕುರಿತಾಗಿ ನಾಲ್ಕು ಸಾರ್ವಜನಿಕ ಉಪನ್ಯಾಸ ಗಳನ್ನು ಮಾಡುವುದು ಎಂದು ತೀರ್ಮಾನವಾಯಿತು. ಅಲ್ಲದೆ ಈ ಉಪನ್ಯಾಸಗಳಲ್ಲಿ ನವಭಾರತದ ಪುನರ್ನಿರ್ಮಾಣಕ್ಕೆ ಹಾಗೂ ಆಧ್ಯಾತ್ಮಿಕ ಜೀವನಕ್ಕೆ ಸಂಬಂಧಿಸಿದ ವಿಷಯಗಳೂ ಇರಬೇಕೆಂದು ತೀರ್ಮಾನವಾಯಿತು. ಈ ನಾಲ್ಕು ಉಪನ್ಯಾಸಗಳ ವಿಷಯಗಳು: ‘ನನ್ನ ಸಮರನೀತಿ’, ‘ಭಾರತದ ಮಹಾಪುರುಷರು’, ‘ವೇದಾಂತ ಮತ್ತು ಭಾರತೀಯ ಜನಜೀವನದಲ್ಲಿ ಅದರ ಅನುಷ್ಠಾನ’ ಹಾಗೂ ‘ಭಾರತದ ಭವಿತವ್ಯ’. ಇವಲ್ಲದೆ ಟ್ರಿಪ್ಲಿಕೇನ್ ಸಾಹಿತ್ಯ ಸಂಘದಲ್ಲಿ ‘ನಮ್ಮ ಮುಂದಿ ರುವ ಕಾರ್ಯ’ ಎಂಬ ವಿಷಯವಾಗಿ ಮಾತನಾಡಬೇಕೆಂಬ ಅಳಸಿಂಗ ಪೆರುಮಾಳರ ಬಿನ್ನಹಕ್ಕೂ ಸ್ವಾಮೀಜಿ ಒಪ್ಪಿಕೊಂಡರು. ಈ ಉಪನ್ಯಾಸಗಳಲ್ಲದೆ ಎರಡು ದಿನ ಬೆಳಗ್ಗೆ ಕ್ಯಾಸಲ್ ಕರ್ನನ್ನಿನಲ್ಲಿ ಪ್ರಶ್ನೋತ್ತರಗಳ ಕಾರ್ಯಕ್ರಮವನ್ನು ಹಾಕಿಕೊಳ್ಳಲಾಯಿತು.

ಆದರೆ ಈ ಕಾರ್ಯಕ್ರಮಗಳ ದೀರ್ಘಪಟ್ಟಿಯನ್ನು ನೋಡಿದಾಗಲೂ ಕೂಡ ನಮಗೆ, ಮದ್ರಾಸಿನಲ್ಲಿ ಸ್ವಾಮೀಜಿಯವರ ಮುಂದೆ ಎಂತಹ ಘನತರವಾದ ಹೊಣೆಯಿತ್ತು ಹಾಗೂ ಅವರ ಮೇಲೆ ಅದೆಷ್ಟು ಕಾರ್ಯ ಒತ್ತಡವಿತ್ತು ಎಂಬುದರ ಪೂರ್ಣ ಕಲ್ಪನೆಯಾಗದು. ಅಲ್ಲಿನ ಜನರ ಪಾಲಿಗೇನೋ ಈ ಕಾರ್ಯಕ್ರಮಗಳೆಲ್ಲ ನವರಾತ್ರಿ ಉತ್ಸವದಂತಿದ್ದುವು. ಆದರೆ ಇದು ಸ್ವಾಮೀಜಿ ಯವರ ದೇಹಾರೋಗ್ಯವನ್ನು ಮಾತ್ರ ಸಂಪೂರ್ಣ ಹದಗೆಡಿಸಿತು. ಅವರು ರೈಲಿನಿಂದ ಇಳಿದ ಲಾಗಾಯ್ತಿನಿಂದ ಅವರಿಗೆ ಬಿನ್ನವತ್ತಳೆಗಳ ಸುರಿಮಳೆಯೇ ಆಯಿತು. ಸ್ವಾಮೀಜಿಯವರ ಪರಮಶಿಷ್ಯನಾದ ಖೇತ್ರಿಯ ಮಹಾರಾಜ ಅಜಿತ್​ಸಿಂಗ್ ತನ್ನ ಪರವಾಗಿ ಬಿನ್ನವತ್ತಳೆಯೊಂದನ್ನು ಸಮರ್ಪಿಸಲು ತನ್ನ ಆಪ್ತ ಕಾರ್ಯದರ್ಶಿಯಾದ ಜಗಮೋಹನಲಾಲನನ್ನು ಅಷ್ಟು ದೂರದಿಂದ ಮದ್ರಾಸಿಗೆ ಕಳಿಸಿಕೊಟ್ಟಿದ್ದ. ಸ್ವಾಮೀಜಿ ಮದ್ರಾಸಿನಲ್ಲಿದ್ದಷ್ಟು ದಿನವೂ ಯಾವಾಗ ನೋಡಿದರೂ ಸಂದರ್ಶಕರು ಅವರನ್ನು ಮುತ್ತಿಕೊಂಡೇ ಇರುತ್ತಿದ್ದರು. ಹಲವಾರು ಮಂದಿ ಮನೆತನಸ್ಥ ಮಹಿಳೆಯರು ದೇವರನ್ನು ನೋಡಲು ದೇವಸ್ಥಾನಗಳಿಗೆ ಬರುವಂತೆ ಅವರ ದರ್ಶನಕ್ಕೆ ಬರುತ್ತಿ ದ್ದರು; ಆರತಿಯೆತ್ತಿ ಹಣ್ಣುಕಾಯಿಗಳನ್ನು ಸಮರ್ಪಿಸುತ್ತಿದ್ದರು. ಸ್ವಾಮೀಜಿ ಹೋದಲ್ಲೆಲ್ಲ ಜನ ಗುಂಪುಗುಂಪಾಗಿ ಸಾಷ್ಟಾಂಗ ಪ್ರಣಾಮಗೈಯುತ್ತಿದ್ದರು.

ಸ್ವಾಮೀಜಿಯವರ ಕಂಠಮಾಧುರ್ಯದ ಬಗ್ಗೆ ಬಹಳವಾಗಿ ಕೇಳಿದ್ದ ಸುಂದರರಾಮ ಅಯ್ಯ ರರೂ ರಂಗಾಚಾರ್ಯರೂ ಒಮ್ಮೆ ಅವರನ್ನು ಹಾಡುವಂತೆ ವಿನಂತಿಸಿಕೊಂಡರು. ಆಗ ಅವರು ಜಯದೇವ ಕವಿಯ ಕವಿತೆಯೊಂದನ್ನು ಹಾಡಿದರು. ಆ ಧ್ವನಿ, ಆ ರಾಗ–ಅಲ್ಲಿನ ಜನ ಹಿಂದೆಂದೂ ಕೇಳಿರದಂಥದು! ಅದು ಅಲ್ಲಿನವರ ಮನಸ್ಸಿನ ಮೇಲೆ ಅಚ್ಚಳಿಯದ ಮುದ್ರೆಯನ್ನೊತ್ತಿತು.

ಏಳನೇ ತಾರೀಕಿನಂದು ಬೆಳಿಗ್ಗೆ ಪ್ರಶ್ನೋತ್ತರ ಕಾರ್ಯಕ್ರಮದಲ್ಲಿ ಭಾಗವಹಿಸಲು ಸುಮಾರು ಇನ್ನೂರು ಜನ ನೆರೆದಿದ್ದರು. ಕೆಲವರು ಮನಸ್ಸು ಮತ್ತು ಜಡವಸ್ತುವಿಗಿರುವ ವ್ಯತ್ಯಾಸವನ್ನು ವಿವರಿಸುವಂತೆ ಕೇಳಿದರು. ಮತ್ತೊಬ್ಬರು “ದೇವರು ಮನುಷ್ಯ ರೂಪ ತಾಳುವುದು ನಿಜವೆ?” ಎಂದು ಕೇಳಿದರು. ಇವುಗಳಿಗೆಲ್ಲ ಸ್ವಾಮೀಜಿ ಅತ್ಯಂತ ಸಹಾನುಭೂತಿಯಿಂದ ಸೂಕ್ತವಾಗಿ ಉತ್ತರಿಸಿದರು. ಅಂದು ಸಂಜೆ ಪ್ರಶ್ನೋತ್ತರಗಳ ಕಾರ್ಯಕ್ರಮ ಮುಂದುವರಿಯಿತು. ಒಬ್ಬರು ಸ್ವಾಮೀಜಿಯವರ ಗೌರವಾರ್ಥವಾಗಿ ರಚಿಸಿದ ಸಂಸ್ಕೃತ ಕವನವೊಂದನ್ನು ಓದಿದರು. ಬಳಿಕ ಒಬ್ಬರು “ಕರ್ಮ ಮತ್ತು ಅದೃಷ್ಟ ಇವುಗಳಿಗಿರುವ ವ್ಯತ್ಯಾಸವೇನು” ಎಂದು ಪ್ರಶ್ನೆ ಹಾಕಿದರು. ಅದಕ್ಕೆ ಸ್ವಾಮೀಜಿ ವಿವರವಾಗಿ ಉತ್ತರಿಸಿದ ಬಳಿಕ ಪ್ರೊ ॥ ಲಕ್ಷ್ಮೀನರಸು ಎಂಬ ಬೌದ್ಧಾನು ಯಾಯಿಯೊಬ್ಬರು ತಮ್ಮೊಂದಿಗೆ ವೇದಾಂತದ ಬಗ್ಗೆ ಚರ್ಚೆ ನಡೆಸಲು ಸ್ವಾಮೀಜಿಯವರನ್ನು ಆಹ್ವಾನಿಸಿದರು. ಅವರು ಸ್ವಾಮೀಜಿ ಬೋಧಿಸುವ ವೇದಾಂತ ತತ್ತ್ವಗಳನ್ನು ತೀವ್ರವಾಗಿ ವಿರೋಧಿಸುವವರು. ಪ್ರೊ ॥ ನರಸು ಸ್ವಾಮೀಜಿಯವರನ್ನು ವೇದಾಂತದ ಹಲವಾರು ಅಂಶಗಳ ಬಗ್ಗೆ ಆಳವಾಗಿ ಪ್ರಶ್ನಿಸಿದರು. ಅವುಗಳೆಲ್ಲದಕ್ಕೂ ಸ್ವಾಮೀಜಿಯವರು ಅತ್ಯಂತ ನಿರರ್ಗಳವಾಗಿ ಮತ್ತು ಸಹಜವಾಗಿ ಉತ್ತರಿಸಿ ಸಭಿಕರ ಮೆಚ್ಚುಗೆ ಗಳಿಸಿದರು.

ಅದೇ ದಿನವೇ ಸಂಜೆ ಸ್ವಾಮೀಜಿಯವರಿಗೆ ಪ್ರಧಾನ ಬಿನ್ನವತ್ತಳೆಯನ್ನು ಸಮರ್ಪಿಸುವ ಕಾರ್ಯಕ್ರಮವಿತ್ತು. ಸಂಜೆ ನಾಲ್ಕು ಗಂಟೆಗೆ ಸ್ವಾಮೀಜಿ ಕ್ಯಾಸಲ್ ಕರ್ನನ್ನಿನಿಂದ ಹೊರಟರು. ಅಂದು ಎಲ್ಲೆಲ್ಲೂ ಒಂದು ಉತ್ಸವದ ವಾತಾವರಣ. ಮದ್ರಾಸಿನ ವಿಕ್ಟೋರಿಯಾ ಭವನದಲ್ಲಿ ಸಭೆ ನಡೆಯಲಿತ್ತು. ಸಭಾಂಗಣದ ಹೊರಗೆ ಹತ್ತು ಸಾವಿರಕ್ಕೂ ಹೆಚ್ಚು ಜನ ಕಿಕ್ಕಿರಿದಿದ್ದರು. ಸ್ವಾಮೀಜಿಯವರ ವಾಹನ ಮುಂದುವರಿಯುವುದೇ ಕಷ್ಟವಾಯಿತು. ಅವರು ವಾಹನದಿಂದ ಇಳಿಯುತ್ತಿದ್ದಂತೆಯೇ ಜನರು “ಸಭೆ ಹೊರಗೇ ನಡೆಯಲಿ! ಸಭೆ ಹೊರಗೇ ನಡೆಯಲಿ! ಎಂದು ಕೂಗಿಕೊಂಡರು. ಸ್ವಾಮೀಜಿ ಸಭಾಂಗಣದ ಒಳಗೆ ವೇದಿಕೆಯಲ್ಲಿ ನಿಂತು ಮಾತನಾಡಿದರೆ ಒಳಗಿನ ಕೆಲವು ನೂರು ಮಂದಿಗೆ ಮಾತ್ರ ಪ್ರಯೋಜನವಾದೀತು. ಹೊರಗೆ ನಿಂತ ಸಾವಿರಾರು ಜನರ ಗತಿಯೇನು? ಆದರೆ ಅಂದು ಹೊರಗಡೆ ನಿಂತು ಮಾತನಾಡುವ ವ್ಯವಸ್ಥೆಯಿಲ್ಲದಿದ್ದ ಕಾರಣ ಸ್ವಾಮೀಜಿಯವರನ್ನು ಒಳಕ್ಕೆ ಕರೆದೊಯ್ಯಲಾಯಿತು. ನಗರದ ಗಣ್ಯವ್ಯಕ್ತಿಗಳೆಲ್ಲ ಸಭಾಂಗಣದಲ್ಲಿ ಸೇರಿದ್ದರು. ಅವರ ಪೈಕಿ ಕೆಲವರೆಂದರೆ ಜಸ್ಟಿಸ್ ಸುಬ್ರಹ್ಮಣ್ಯ ಅಯ್ಯರ್, ಆನರಬಲ್ ಸುಬ್ಬರಾವ್ ಪಂತುಲು, ಪಾರ್ಥಸಾರಥಿ ಅಯ್ಯಂಗಾರ್, ಕರ್ನಲ÷ ಹೆಚ್. ಎಸ್. ಓಲ್ಕಾಟ್ ಹಾಗೂ ರಾಜರತ್ನ ಮುದಲಿಯಾರ್. ಈ ಕರ್ನಲ್ ಓಲ್ಕಾಟ್ ಎಂಬುವನು ಮದ್ರಾಸಿನ ಥಿಯಾಸೊಫಿಸ್ಟ್ ಸಂಘದ ಅಧ್ಯಕ್ಷ; ಅತ್ಯಂತ ಪ್ರಭಾವಶಾಲೀ ವ್ಯಕ್ತಿ. ತಾವು ಅಮೆರಿಕೆಗೆ ಹೊರಡುವ ಮೊದಲು, ತಮಗೊಂದು ಪರಿಚಯ ಪತ್ರ ಕೊಡುವಂತೆ ಸ್ವಾಮೀಜಿಯವರು ಈತನನ್ನು ಕೇಳಿದ್ದನ್ನು ಇಲ್ಲಿ ನೆನಪಿಸಿಕೊಳ್ಳಬಹುದು. ಕಡ್ಡಿಮುರಿದಂತೆ ನಿರಾಕರಿಸಿದಾಗ, ತಾನು ಪರಿಚಯಪತ್ರ ಕೊಡಲೂ ಸಾಧ್ಯವಿಲ್ಲವೆಂದು ಈತ ಹೇಳಿಬಿಟ್ಟಿದ್ದ. ಆದರೆ ಈ ವಿಷಯಗಳಾವುವೂ ಆಗ ಬಹಿರಂಗವಾಗಿರಲಿಲ್ಲ. ಅತ್ಯಂತ ಕುತೂಹಲದ ಸಂಗತಿಯೇನೆಂದರೆ ಈಗ ಈ ಕರ್ನಲ್ ಓಲ್ಕಾಟ್ ಮದ್ರಾಸಿನ ಸ್ವಾಗತ ಸಮಿತಿಯ ಸದಸ್ಯರಲ್ಲೊಬ್ಬನಾಗಿದ್ದ!

ಅಂತೂ ಸಭೆ ಪ್ರಾರಂಭವಾಯಿತು. ಸ್ವಾಗತ ಸಮಿತಿಯ ಪರವಾಗಿ, ಖೇತ್ರಿಯ ಜನತೆಯ ಪರವಾಗಿ ಹಾಗೂ ಇನ್ನಿತರ ಸಂಸ್ಥೆಗಳ ಪರವಾಗಿ ಒಂದಾದ ಮೇಲೊಂದರಂತೆ ಬಿನ್ನವತ್ತಳೆಗಳನ್ನು ಸಮರ್ಪಿಸಲಾಯಿತು. ಆದರೆ ಸ್ವಾಮೀಜಿಯವರಿಗೆ ಏನೋ ಒಂದು ಬಗೆಯ ಮುಜುಗರ. ಏಕೆಂದರೆ ಸಭಾಂಗಣದಾಚೆಯಿಂದ ಒಂದೇ ಸಮನೆ ಕೂಗು ಕೇಳಿಬರುತ್ತಲೇ ಇದೆ–“ಸಭೆ ಹೊರಗೆ ನಡೆಯಲಿ!”ಎಂದು ಬುಹಶಃ ಒಳಗೆ ಸೇರಿಕೊಂಡಿದ್ದವರೆಲ್ಲ ತಾವೇ ಅದೃಷ್ಟವಂತರೆಂದು ಹಿಗ್ಗುತ್ತ ಸ್ವಾಮೀಜಿಯವರ ಮಾತುಗಳನ್ನು ಕೇಳುವ ಉತ್ಸಾಹದಿಂದ ಕುಳಿತಿದ್ದಾರೆ. ಆದರೆ ಹೊರಗಡೆಯ ಜನರ ಕೂಗನ್ನು ಕೇಳಿಯೂ ಸುಮ್ಮನೆ ಕುಳಿತಿರಲು ಸ್ವಾಮೀಜಿಯವರ ಮನಸ್ಸೊಪ್ಪುತ್ತಿಲ್ಲ. ಹೊರಗೆ ಸೇರಿರುವ ಅಸಂಖ್ಯಾತ ಯುವಜನರನ್ನು ಕಡೆಗಣಿಸಿ ತಾವು ಒಳಗೆ ಭಾಷಣ ಮಾಡಿ ಅವರನ್ನೆಲ್ಲ ನಿರಾಶೆಗೊಳಿಸುವುದು ಸ್ವಾಮೀಜಿಯವರಿಗೆ ಒಪ್ಪಿಗೆಯಾಗಲಿಲ್ಲ. ಆದ್ದರಿಂದ ಅವರು ಇದಕ್ಕಿದ್ದಂತೆ ಎದ್ದು ನಿಂತು, “ನಾನು ಜನತೆಗೆ ಸೇರಿದವನು. ಅವರೆಲ್ಲ ಅಲ್ಲಿ ಹೊರಗಡೆ ಇದ್ದಾರೆ. ನಾನು ಅವರನ್ನು ನೋಡಬೇಕು” ಎಂದು ಗುಡುಗಿ ಸೀದಾ ಹೊರನಡೆದುಬಿಟ್ಟರು! ಒಳಗೆ ಕುಳಿತಿದ್ದವರೆಲ್ಲ ಕ್ಷಣಕಾಲ ದಿಗ್ಭಾಂತರಾದರು. ಆದರೆ ಮಾಡುವುದೇನು? ವಿಧಿಯಿಲ್ಲದೆ ಇತಾವೂ ಹೊರಗೆ ಓಡಿಬಂದು ಕಾರ್ಯಕ್ರಮಕ್ಕೆ ಸಿದ್ಧತೆ ಮಾಡಲು ತೊಡಗಿದರು.

ಸ್ವಾಮೀಜಿ ಹೊರಗೆ ಬಂದು ಕಾಣಿಸಿಕೊಳ್ಳುತ್ತಿದ್ದಂತೆಯೇ ಸಿಡಿಲು ಬಡಿದಂತೆ ಕರತಾಡನದ ಧ್ವನಿ, ಹರ್ಷೋದ್ಗಾರ; ಬಳಿಕ ನೂಕುನುಗ್ಗಲು! ಆದರೆ ಆ ಬಯಲಲ್ಲಿ ನಿಂತು ಮಾತನಾಡಲು ಯಾವ ಅನುಕೂಲತೆಯೂ ಇರಲಿಲ್ಲವಾದ್ದರಿಂದ ಸ್ವಾಮೀಜಿ ಅಲ್ಲೇ ಇದ್ದ ಒಂದು ಸಾರೋಟನ್ನು ಹತ್ತಿ ಮಾತನಾಡಲು ಪ್ರಾರಂಭಿಸಿದರು. ಆದರೆ ಕಿವಿ ಕಿವುಡಾಗುವಷ್ಟು ಗದ್ದಲವಿದ್ದುದರಿಂದ ಅವರ ಮಾತು ಯಾರಿಗೂ ಕೇಳಿಸುವಂತಿರಲಿಲ್ಲ. ಸ್ವಾಮೀಜಿ ತಮ್ಮ ಮಾತನ್ನು ಹಾಸ್ಯವಾಗಿ ಪ್ರಾರಂಭಿಸಿದರು: “ಮನುಷ್ಯ ತಾನೊಂದು ಬಗೆದರೆ ದೈವ ಬೇರೊಂದು ಬಗೆಯಿತು. ನಾನು ನಿಮ್ಮನ್ನುದ್ದೇಶಿಸಿ ಆಂಗ್ಲ ಪದ್ಧತಿಯಲ್ಲಿ ಮಾತನಾಡಲು ಸಿದ್ಧತೆಯಾಗಿತ್ತು. ಆದರೆ ನಾನೀಗ ದೇವರ ಇಚ್ಛೆಯಂತೆ ಗಾಡಿಯೊಂದರ ಮೇಲೆ ನಿಂತು ‘ಗೀತಾಶೈಲಿ’ಯಲ್ಲಿ ಮಾತನಾಡುತ್ತಿದ್ದೇನೆ.” ಸ್ವಾಮೀಜಿ ಹೀಗೆ ಹೇಳುವಾಗ ಹಿಂದೆ ಶ್ರೀಕೃಷ್ಣನು ರಥದಲ್ಲಿ ನಿಂತು ಆ ಕುರುಕ್ಷೇತ್ರದಲ್ಲಿ ಅರ್ಜುನನಿಗೆ ಉಪದೇಶ ಮಾಡಿದ ದೃಶ್ಯ ಅವರ ಮನಸ್ಸಿನಲ್ಲಿತ್ತು. ಅಲ್ಲದೆ ಅಂದಿನ ಗೌಜು ಗದ್ದಲವನ್ನು ನೋಡಿದರೆ ಅದು ಕುರುಕ್ಷೇತ್ರದ ಕೋಲಾಹಲಕ್ಕೇನೂ ಕಡಿಮೆಯಿರಲಿಲ್ಲ. ಆ ಗಲಭೆಯ ನಡುವೆಯೇ ಸ್ವಾಮೀಜಿ ತುಂಬ ಕಷ್ಟಪಟ್ಟು ಸುಮಾರು ಹತ್ತುಹದಿನೈದು ನಿಮಿಷ ಮಾತನಾಡಿರಬಹುದು; ಆದರೆ ಆ ಬೃಹತ್ ಜನಸ್ತೋಮದ ಗಲಭೆಯ ನಡುವೆ ಸ್ವಾಮೀಜಿಯವರ ಧ್ವನಿ ಕೇಳಿಸದಂತಾಯಿತು. ಪರಿಣಾಮವಾಗಿ ಗಲಭೆ ಮತ್ತಷ್ಟು ಹೆಚ್ಚಿತು. ಕಡೆಗೆ ವಿಧಿಯಿಲ್ಲದೆ ಸ್ವಾಮೀಜಿ, “ಸ್ನೇಹಿತರೇ, ನಿಮ್ಮ ಉತ್ಸಾಹವನ್ನು ನೋಡಿ ನನಗೆ ಮಹದಾನಂದವಾಗಿದೆ. ನೀವಿಂದು ತೋರುತ್ತಿರುವ ಉತ್ಸಾಹವೇ ನಮಗಿಂದು ಬೇಕಾದುದು–ಪ್ರಚಂಡ ಉತ್ಸಾಹ! ಆದರೆ ಆ ಉತ್ಸಾಹವನ್ನು ಸ್ಥಿರವಾಗಿಟ್ಟಿರಿ; ಅದು ಯಾವಾಗಲೂ ಹೀಗೆಯೇ ಇರಲಿ. ಆ ಜ್ವಾಲೆ ಆರದಿರಲಿ. ಭಾರತದಲ್ಲಿ ಮಹತ್ಕಾರ್ಯಗಳನ್ನು ಸಾಧಿಸಬೇಕಾಗಿದೆ. ಅದಕ್ಕೆ ನಿಮ್ಮ ನೆರವು ಬೇಕು. ಇಂತಹ ಉತ್ಸಾಹವಂತೂ ಅಗತ್ಯವಾಗಿ ಬೇಕೇಬೇಕು. ಇಂದು ಈ ಸಭೆಯನ್ನು ಮುಂದುವರಿಸು ವುದು ಇನ್ನು ಸಾಧ್ಯವಿಲ್ಲ. ಇಂದು ನೀವು ನನ್ನನ್ನು ನೋಡುವಷ್ಟರಿಂದಲೇ ತೃಪ್ತರಾಗಬೇಕು. ದಯವಿಟ್ಟು ಅನ್ಯಥಾ ಭಾವಿಸಬಾರದು. ನಿಮ್ಮ ಉತ್ಸಾಹಕ್ಕೂ ಸ್ವಾಗತಕ್ಕೂ ನಾನು ಕೃತಜ್ಞ; ಇನ್ನೊಮ್ಮೆ ಶಾಂತ ಸನ್ನಿವೇಶದಲ್ಲಿ ನನ್ನ ಅಭಿಪ್ರಾಯಗಳನ್ನು ತಿಳಿಸುತ್ತೇನೆ. ಸದ್ಯಕ್ಕೆ ನಮಸ್ಕಾರ” ಎಂಬೀ ಮಾತುಗಳೊಂದಿಗೆ ತಮ್ಮ ಭಾಷಣವನ್ನು ಮುಗಿಸಿದರು.

ಹೀಗೆ ಅಂದಿನ ಸಭೆ ಅರ್ಧಕ್ಕೆ ನಿಲ್ಲಬೇಕಾಯಿತು. ಇದರಿಂದ ಜನಗಳಿಗೂ ಕಾರ್ಯಕರ್ತರಿಗೂ ತುಂಬ ನಿರಾಶೆಯಾಯಿತು. ಇನ್ನುಳಿದ ಕಾರ್ಯಕ್ರಮಗಳಾದರೂ ನಿರಾತಂಕವಾಗಿ ನಡೆಯ ಬೇಕೆಂದು ಇಚ್ಛಿಸಿದ ಸ್ವಾಗತ ಸಮಿತಿಯವರು ಅದಕ್ಕೊಂದು ಉಪಾಯ ಮಾಡಿದರು. ಅದೇ ನೆಂದರೆ ಕಾರ್ಯಕ್ರಮಗಳಿಗೆ ಟಿಕೆಟಿನ ವ್ಯವಸ್ಥೆ ಮಾಡುವುದು. ಪರಿಸ್ಥಿತಿಯನ್ನರಿತ ಸ್ವಾಮೀಜಿ ಅದಕ್ಕೆ ಒಪ್ಪಿಗೆ ನೀಡಿದರು. ಮರುದಿನವೇ ಪತ್ರಿಕೆಗಳಲ್ಲಿ ಕಾರ್ಯಕ್ರಮಗಳ ಪಟ್ಟಿಯನ್ನೂ ಟಿಕೆಟಿನ ವಿವರಗಳನ್ನೂ ಪ್ರಕಟಿಸಲಾಯಿತು. ವೇದಿಕೆಯ ಮೇಲೆ ಕುಳಿತುಕೊಳ್ಳಬಯಸುವವರಿಗೆ ಎರಡು ರೂಪಾಯಿ; ಸಭಾಂಗಣದಲ್ಲಿ ಕುಳಿತುಕೊಳ್ಳುವವರಿಗೆ ಒಂದು ರೂಪಾಯಿ. (ಸುಮಾರು ನೂರು ವರ್ಷಗಳ ಹಿಂದಿನ ಈ ಒಂದು ರೂಪಾಯಿಗೆ ಇಂದಿನ ಬೆಲೆಯೆಷ್ಟಾಗುತ್ತದೆ ಎಂಬುದನ್ನು ನೀವೇ ಲೆಕ್ಕಹಾಕಿ ನೋಡಿಕೊಳ್ಳಿ.) ಹೀಗೆ ಸಂಗ್ರಹವಾದ ಹಣವನ್ನು ಭಾರತದಲ್ಲಿ ಸ್ವಾಮೀಜಿಯವರ ಉದ್ದೇಶಿತ ಕಾರ್ಯಗಳಿಗೆ ಬಳಸಲಾಗುವುದೆಂದೂ ಪ್ರಕಟಿಸಲಾಯಿತು. ಆದರೆ, ಫೆಬ್ರವರಿ ೧೪ ರಂದು “ಭಾರತದ ಭವಿತವ್ಯ” ಎಂಬ ವಿಷಯವನ್ನು ಕುರಿತಾಗಿ ಮಾಡಲಿದ್ದ ಭಾಷಣವನ್ನು ಬಯಲಿನಲ್ಲಿ ನಡೆಸಲಾಗುವುದೆಂದೂ ಅದಕ್ಕೆ ಪ್ರವೇಶ ಉಚಿತವೆಂದೂ ತಿಳಿಸಲಾಯಿತು.

ಫೆಬ್ರವರಿ ೮ರಂದು ಮಧ್ಯಾಹ್ನ ತಿರುಪ್ಪತ್ತೂರಿನ ಶೈವರ ನಿಯೋಗವೊಂದು ಸ್ವಾಮೀಜಿಯವರ ಸಂದರ್ಶನಕ್ಕಾಗಿ ಬಂದಿತು. ಸ್ವಾಮೀಜಿಯವರೊಂದಿಗೆ ವಾದ ಹೂಡಿ, ಅವರ ಅದ್ವೈತ ಸಿದ್ಧಾಂತ ವನ್ನು ಬಗ್ಗುಬಡಿಯಲು ಈ ಶೈವರು ಸಕಲ ಸಿದ್ಧತೆಗಳನ್ನು ಮಾಡಿಕೊಂಡು ಬಂದಿದ್ದರು. ಅದಕ್ಕಾಗಿ ಕಠಿಣ ಪ್ರಶ್ನೆಗಳ ಒಂದು ದೊಡ್ಡ ಪಟ್ಟಿಯನ್ನೇ ತಯಾರಿಸಿಕೊಂಡಿದ್ದರು. ಈ ನಿಯೋಗದ ಮುಖ್ಯಸ್ಥ ತನ್ನ ಮೊದಲ ಪ್ರಶ್ನೆಯನ್ನು ಮುಂದಿಟ್ಟ:

“ಸ್ವಾಮಿ, ಅವ್ಯಕ್ತವು ಹೇಗೆ ವ್ಯಕ್ತವಾಗುತ್ತದೆ?”

ಈ ಪ್ರಶ್ನೆಯ ಅಭಿಪ್ರಾಯ ಇದು: ಅದ್ವೈತವೇದಾಂತದ ಪ್ರಕಾರ ಸಕಲ ವಸ್ತುಗಳೂ, ಅವ್ಯಕ್ತ ವಾದ ಪರಬ್ರಹ್ಮವಸ್ತುವಿನ ಅಭಿವ್ಯಕ್ತಿ; ಆದರೆ ಈ ಅನಂತ-ಅಖಂಡ-ಅವ್ಯಕ್ತ-ಸಚ್ಚಿದಾನಂದ ಪರಬ್ರಹ್ಮವಸ್ತುವು ವ್ಯಕ್ತವಾಗಲು ಹೇಗೆ ಸಾಧ್ಯವಾಯಿತು?–ಎಂದು.

ಹಿಂದೆ ಸ್ವಾಮೀಜಿಯವರಿಗೆ ಅಮೆರಿಕದಲ್ಲೂ ಹಾಗೂ ಇತ್ತೀಚೆಗೆ ಮಧುರೆಯಲ್ಲೂ ಹಲವಾರು ಜನ ಹಲವು ವಿಧದಲ್ಲಿ ಇದೇ ಪ್ರಶ್ನೆಯನ್ನೇ ಹಾಕಿದ್ದರು. ಅವೆಲ್ಲಕ್ಕೂ ಸ್ವಾಮೀಜಿಯವರ ಉತ್ತರ ಒಂದೇ ಬಗೆಯದಾಗಿತ್ತು. ಈಗ ಈ ಶೈವರ ಪ್ರಶ್ನೆಗೆ ಅವರ ಉತ್ತರ ಸಿಡಿಲಿನಂತೆ ಎರಗಿತು:

“ನೀವು ಕೇಳುವ ‘ಹೇಗೆ’ ‘ಏಕೆ’ ‘ಎಲ್ಲಿಂದ’ ಎನ್ನುವ ಈ ಪ್ರಶ್ನೆಗಳೆಲ್ಲ ವ್ಯಕ್ತ ಪ್ರಪಂಚಕ್ಕೆ ಮಾತ್ರ ಸಂಬಂಧಪಟ್ಟವುಗಳು; ಈ ವ್ಯಕ್ತಪ್ರಪಂಚದಲ್ಲಿ ಮಾತ್ರ ಕೇಳಬಹುದಾದ ಪ್ರಶ್ನೆಗಳು. ಈ ಪ್ರಶ್ನೆಗಳನ್ನು ಅವ್ಯಕ್ತ ಪರಬ್ರಹ್ಮದ ವಿಷಯವಾಗಿ ಕೇಳುವಂತೆಯೇ ಇಲ್ಲ. ಅದು ದೇಶ ಕಾಲ ಕಾರಣಗಳಿಗೆ ಅತೀತವಾದದ್ದು. ಆದ್ದರಿಂದ ಅದು ಈ ಪರಿವರ್ತನಶೀಲ ಜಗತ್ತಿಗೆ ವ್ಯತಿರಿಕ್ತ ವಾದದ್ದು. ಆದಕಾರಣ ನಿಮ್ಮ ಈ ಪ್ರಶ್ನೆ ಅಸಮಂಜಸವಾದದ್ದು. ಸಮಂಜಸವಾದ ಪ್ರಶ್ನೆಯನ್ನು ಕೇಳಿ, ನಾನು ಉತ್ತರಿಸುತ್ತೇನೆ.”

ಈ ಉತ್ತರವನ್ನು ಕೇಳಿದಾಗ ಪ್ರಾಶ್ನಿಕರಿಗೆ ಪಾರ್ಶ್ವವಾಯು ಬಡಿದಂತಾಯಿತು. ಏಕೆಂದರೆ ಅದು ತೀರ ಅನಿರೀಕ್ಷಿತವಾದ ಉತ್ತರ. ಈ ಪ್ರಶ್ನೆಯೇನೋ ತುಂಬ ಹಳೆಯದೇ ಆದರೂ ಅದಕ್ಕೆ ಆ ರೀತಿಯಲ್ಲಿ ಹಿಂದೆ ಯಾರೂ ಉತ್ತರಿಸಿರಲಿಲ್ಲ. ಆ ಉತ್ತರ ಎಷ್ಟು ಸಮಂಜಸವಾಗಿದೆಯೋ ಅಷ್ಟೇ ಸರಳವೂ ಆಗಿದೆ. ಈಗ ಆ ಶೈವರಿಗೆ ಅರಿವಾಯಿತು–ತಾವು ಕುಳಿತಿರುವುದು ಒಂದು ಸಿಂಹದ ಮುಂದೆ ಎಂದು. ಈ ಸಿಂಹವನ್ನು ತರ್ಕವಿತರ್ಕದ ಬಲೆಯಲ್ಲಿ ಹಿಡಿಯಲಾಗದು ಎಂದರಿತು ಅದಕ್ಕೆ ವಿನಯದಿಂದ ತಲೆಬಾಗುವುದೇ ವಿಹಿತವೆಂದು ಭಾವಿಸಿದರು. ಅವರ ಇನ್ನುಳಿದ ಪ್ರಶ್ನೆಗಳೆಲ್ಲ ಅಲ್ಲಲ್ಲೇ ಅಡಗಿಹೋದುವು. ಅವರೆಲ್ಲ ಮಂತ್ರವಾದಿಯ ದಂಡದ ಪ್ರಭಾವಕ್ಕೊಳ ಗಾದವರಂತೆ ಸ್ತಬ್ಧರಾಗಿ ಕುಳಿತುಬಿಟ್ಟರು. ಸ್ವಾಮೀಜಿಯವರ ವ್ಯಕ್ತಿತ್ವದ ಸಮ್ಮೋಹಿನೀ ಶಕ್ತಿ ಪ್ರಾಶ್ನಿಕರೆಲ್ಲರ ಹೃನ್ಮನಗಳನ್ನು ಸೂರೆಗೊಂಡಿತು. ವೇದಾಂತ ಕೇಸರಿಯಾದ, ಪ್ರತಿವಾದಿ ಭಯಂ ಕರರಾದ ಸ್ವಾಮೀಜಿ ಈಗ ಅಲ್ಲಿದ್ದವರನ್ನುದ್ದೇಶಿಸಿ ಅತ್ಯಂತ ಮಧುರ ದನಿಯಲ್ಲಿ ಮಾತನಾಡ ತೊಡಗಿದರು. ಅವರ ಮಾತಿನ ಸಾರಾಂಶ ಇದು: ‘ಭಗವಂತನ ಸಾಕ್ಷಾತ್ಕಾರಕ್ಕೆ ಹಾಗೂ ಭಗವಂತನ ಸೇವೆಗೆ ಅತ್ಯುತ್ತಮ ಮಾರ್ಗವೆಂದರೆ ದೀನದಲಿತರ ಸೇವೆ ಮಾಡುವುದು; ಹಸಿದವರಿಗೆ ಅನ್ನ ಕೊಡುವುದು; ನೊಂದವರಿಗೆ ಸಮಾಧಾನ ನೀಡುವುದು; ಅನಾಥರಿಗೆ-ಪತಿತರಿಗೆ ನೆರವಾಗು ವುದು.’ ಹೀಗೆ ಮಾನವತೆಯ ಸೇವೆಗಾಗಿ ಅವರು ನೀಡಿದ ಕರೆಯನ್ನು ಕೇಳಿದ ಆ ಶೈವ ಪ್ರತಿನಿಧಿಗಳ ಹೃದಯದಲ್ಲಿ ಒಂದು ಹೊಸ ಬೆಳಕು ತುಂಬಿತು. ಸ್ವಾಮೀಜಿಯವರ ಕಳಕಳಿಯ ನುಡಿಗಳು ಅವರ ಮನಕರಗಿಸಿದ್ದುದನ್ನು, ಅವರು ಸಾಷ್ಟಾಂಗ ಪ್ರಣಾಮ ಮಾಡಿ ಹೊರಟಾಗ ಅವರ ಮುಖಗಳಲ್ಲಿ ಕಾಣಬಹುದಾಗಿತ್ತು.

ಸ್ವಾಮೀಜಿಯವರ ಸಂದರ್ಶನವನ್ನು ಬಯಸಿ ಬರುತ್ತಿದ್ದವರೆಲ್ಲ ಪ್ರಾಮಾಣಿಕ ಜಿಜ್ಞಾಸುಗಳೇ ಆಗಿರಲಿಲ್ಲ. ಕೆಲವರು ಅವರ ಜ್ಞಾನದ ಆಳವನ್ನು ಪರೀಕ್ಷಿಸಿ ನೋಡಲು ಬಂದರೆ ಇನ್ನು ಕೆಲವರು ತಮ್ಮದನ್ನು ಪ್ರದರ್ಶಿಸಲು ಬರುತ್ತಿದ್ದರು. ಮತ್ತೆ ಕೆಲವರು ಅವರ ಅದ್ವೈತವಾದವನ್ನು ಖಂಡಿ ಸಲು ಅಥವಾ ಕುತರ್ಕ ಹೂಡಿ ಅವರ ಕಾಲು ತೊಡರಿಸಿ ಬೀಳಿಸಲು ಬರುತ್ತಿದ್ದರು. ಅವರೆಲ್ಲರ ಮನೋಭಾವಕ್ಕೆ ಅನುಗುಣವಾಗಿ ಸ್ವಾಮೀಜಿ ಉತ್ತರಿಸುತ್ತಿದ್ದ ಬಗೆ ಕುತೂಹಲಕರವಾಗಿತ್ತು; ಕೆಲವೊಮ್ಮೆ ರಂಜನೀಯವಾಗಿಯೂ ಇರುತ್ತಿತ್ತು. ಒಂದು ದಿನ ಒಬ್ಬ ಪಂಡಿತ, “ನೀವೊಬ್ಬ ಶೂದ್ರರಾಗಿದ್ದೂ ಹೇಗೆ ಸಂನ್ಯಾಸಿಗಳಾದಿರಿ? ಇದಕ್ಕೆ ನಮ್ಮ ಶಾಸ್ತ್ರಗಳು ಒಪ್ಪುವುದಿಲ್ಲವಲ್ಲ?” ಎಂದು ಪ್ರಶ್ನಿಸಿದ. ಆಗ ಸ್ವಾಮೀಜಿಯೆಂದರು, “ನೀವು ಬ್ರಾಹ್ಮಣರು ಪ್ರತಿದಿನವೂ ಯಾವ ಚಿತ್ರ ಗುಪ್ತರಿಗೆ ನಮಸ್ಕರಿಸುತ್ತೀರೋ ಆ ಚಿತ್ರಗುಪ್ತರ ಕುಲಕ್ಕೆ ಸೇರಿದವನು ನಾನು. ಆದ್ದರಿಂದ ಬ್ರಾಹ್ಮಣರಿಗೆ ಸಂನ್ಯಾಸ ಸ್ವೀಕರಿಸುವ ಹಕ್ಕಿರುವುದಾದರೆ ನನಗೆ ಅವರಿಗಿಂತ ಹೆಚ್ಚು ಹಕ್ಕಿದೆ!” ಹೀಗೆಂದ ಅವರು ಆ ಪಂಡಿತ ಮತ್ತೆ ತನ್ನ ಕುತರ್ಕವನ್ನು ಮುಂದುವರಿಸದಂತೆ, ಅವನಿಗೆ ಮತ್ತೊಂದು ಆಘಾತವನ್ನಿತ್ತರು–“ನೀವು ಕೇಳಿದ ಪ್ರಶ್ನೆಯಲ್ಲಿ ಒಂದು ಅಕ್ಷಮ್ಯವಾದ ಉಚ್ಚಾ ರಣಾ ದೋಷವಿತ್ತು. ಇಂತಹ ತಪ್ಪು ಉಚ್ಚಾರಣೆಯನ್ನು ಪಾಣಿನಿ ಮಹರ್ಷಿಗಳು ‘ನ ಮ್ಲೇಚ್ಛಿತಂ ವೈ, ನಾಪಭಾಷಿತಂ ವೈ’ (ಪದಗಳನ್ನು ಅಪಮಾನಗೊಳಿಸಬಾರದು, ಇಲ್ಲವೆ ತಪ್ಪಾಗಿ ಉಚ್ಚರಿಸ ಬಾರದು) ಎಂದು ಖಂಡಿಸುತ್ತಾರೆ. ಆದ್ದರಿಂದ ನಿಮಗೆ ಈ ವಾದವನ್ನು ಮುಂದುವರಿಸಲು ಹಕ್ಕಿಲ್ಲ.” ಪಂಡಿತ ಮತ್ತೆ ತುಟಿ ಬಿಚ್ಚಲಿಲ್ಲ.

ಮತ್ತೊಂದು ದಿನ ಒಬ್ಬ ವೈಷ್ಣವ ಪಂಡಿತ ಸ್ವಾಮೀಜಿಯವರೊಂದಿಗೆ ವೇದಾಂತದ ಬಗ್ಗೆ ಚರ್ಚಿಸಲು ಇಚ್ಛಿಸಿದ. ಆದರೆ ಸ್ವಾಮೀಜಿ, “ನನಗೆ ಕೇವಲ ವ್ಯರ್ಥ ಮಾತುಕತೆಯಲ್ಲಿ ತೊಡಗಿ ಸಮಯವನ್ನು ಹಾಳು ಮಾಡಲು ಸಾಧ್ಯವಿಲ್ಲ” ಎಂದು ಹೇಳಿ ವಾದದಲ್ಲಿ ತೊಡಗಲು ನಿರಾಕರಿಸಿ ದರು. ಆದರೆ ಪಂಡಿತ ಬಿಡಲಿಲ್ಲ. ಕಡೆಗೆ, “ನಾನು ಸಂಪೂರ್ಣ ಅದ್ವೈತಿ” ಎಂದಾದರೂ ಅವರ ಬಾಯಿಂದ ಹೇಳಿಸಬೇಕೆಂದು ಆತ ಪ್ರಯತ್ನ ಪಡುತ್ತಿದ್ದ. ಏಕೆಂದರೆ ಆತ ಅದ್ವೈತದ ಕಟ್ಟಾ ವಿರೋಧಿ. ಆ ಪಂಡಿತ ಸಂಸ್ಕೃತದಲ್ಲಿ ಮಾತನಾಡುತ್ತಿದ್ದರೆ ಸ್ವಾಮೀಜಿ ಬೇಕೆಂದೇ ಇಂಗ್ಲಿಷಿನಲ್ಲಿ ಉತ್ತರಿಸುತ್ತಿದ್ದರು. ಕಡೆಗೆ ಸ್ವಾಮೀಜಿ, “ಆ ಪಂಡಿತರಿಗೆ ಹೇಳಿ–ನಾನು ಈ ದೇಹದಲ್ಲಿರುವವರೆಗೆ ಮಾತ್ರ ದ್ವೈತಿ, ಆದರೆ ಆಮೇಲೆ ಅಲ್ಲ ಎಂದು” ಎಂದು ಬಳಿಯಲ್ಲಿದ್ದವರಿಗೆ ಇಂಗ್ಲಿಷಿನಲ್ಲಿ ಹೇಳಿದರು. ಅದನ್ನು ತಿಳಿದ ಪಂಡಿತ ತಮಿಳಿನಲ್ಲಿ, “ಎಂದರೆ ಅವರು ತಾವು ಅದ್ವೈತಿ ಎಂದೇ ಹೇಳಿದಂತಾಯಿತು!” ಎಂದು ಉದ್ಗರಿಸಿದ. “ಹಾಗೆಯೇ ಆಗಲಿ, ಬಿಡಿ” ಎಂದರು ಸ್ವಾಮೀಜಿ. ಆ ವಿಷಯ ಅಲ್ಲಿಗೇ ಮುಕ್ತಾಯವಾಯಿತು.

ದಿನಾಂಕ ೯ರಂದು ಮಂಗಳವಾರ ಬೆಳಿಗ್ಗೆ ಸ್ವಾಮೀಜಿ ಟ್ರಿಪ್ಲಿಕೇನಿನ ಸಾಹಿತ್ಯ ಸಂಘಕ್ಕೆ ಭೇಟಿ ನೀಡಿದರು. ಅವರು ಅಮೆರಿಕೆಗೆ ಹೊರಡುವ ಮುನ್ನ ತಮ್ಮ ಮೊಟ್ಟಮೊದಲ ಸಾರ್ವಜನಿಕ ಉಪನ್ಯಾಸವನ್ನು ನೀಡಿದ್ದು ಇಲ್ಲಿಯೇ. ಅವರು ಅಲ್ಲಿ ನೀಡಿದ ಉಪನ್ಯಾಸಗಳ ಮೂಲಕವೇ ಅವರ ಅಂತಃಸತ್ವ ಬೆಳಕಿಗೆ ಬರುವಂತಾದದ್ದು, ಮದ್ರಾಸಿನ ಜನ ಅವರನ್ನು ಗುರುತಿಸುವಂತಾದದ್ದು. ಅಂದು ಸ್ವಾಮೀಜಿ “ನಮ್ಮ ಮುಂದಿರುವ ಕಾರ್ಯ” ಎಂಬ ವಿಷಯವಾಗಿ ಒಂದು ಗಂಟೆಗೂ ಹೆಚ್ಚು ಕಾಲ ಮಾತನಾಡಿದರು. ಅದರ ಮುಖ್ಯಾಂಶಗಳು ಹೀಗಿದ್ದುವು:

“ಹಿಂದೂಗಳಿಗೆ ಹಿಂದೆ ತಮ್ಮದೇ ಆದ ಒಂದು ಸ್ವಂತಿಕೆಯಿತ್ತು. ಆದರೆ ಇಂದು ಅದು ಸಂಪೂರ್ಣ ನಶಿಸಿಹೋಗಿದೆ. ಅದು ಎಷ್ಟರ ಮಟ್ಟಿಗೆಂದರೆ ಜನರು ನೂರಾರು ವರ್ಷಗಳಿಂದಲೂ, ‘ನೀರನ್ನು ಬಲಗೈಯಿಂದ ಕುಡಿಯಬೇಕೆ ಅಥವಾ ಎಡಗೈಯಿಂದ ಕುಡಿಯಬಹುದೆ? ನಾನು ನಿನ್ನನ್ನು ಮುಟ್ಟಬಹುದೆ, ಅಥವಾ ನೀನು ನನ್ನನ್ನು ಮುಟ್ಟಬಹುದೆ? ಹಾಗೊಂದು ವೇಳೆ ಮುಟ್ಟಿ ದರೆ ಅದಕ್ಕೆ ಪ್ರಾಯಶ್ಚಿತ್ತವೇನು?’ ಎಂಬಂತಹ ಘನತರ ವಿಚಾರಗಳ ಚರ್ಚೆಯಲ್ಲಿ ಮುಳುಗಿ ಹೋಗಿದ್ದಾರೆ. ಅಲ್ಲದೆ, ನಾವು ನಮ್ಮನ್ನು ನಿರಂತರವಾಗಿ ಸಂಕುಚಿತಗೊಳಿಸಿಕೊಳ್ಳುತ್ತ ಇತರ ರಾಷ್ಟ್ರಗಳಿಂದ ಸಂಪೂರ್ಣ ಬೇರೆಯೇ ಆಗಿಬಿಟ್ಟಿದ್ದೇವೆ. ಈ ಜಗತ್ತಿನಲ್ಲಿ ಮನುಷ್ಯರೆಂದರೆ ಕೇವಲ ನಾವು ಮಾತ್ರವೇ ಎಂಬ ಈ ಭಾವನೆಯನ್ನು ಬಿಟ್ಟುಬಿಡಬೇಕು. ಇತರ ರಾಷ್ಟ್ರಗಳಿಗೆ ನಾವು ಕಲಿಸ ಬೇಕಾದದ್ದು ಬಹಳಷ್ಟು ಇದೆಯಾದರೂ ಇತರರಿಂದ ನಾವು ಕಲಿಯಬೇಕಾದದ್ದೂ ಸಾಕಷ್ಟಿದೆ. ಭಾರತವು ಇತರ ರಾಷ್ಟ್ರಗಳಿಗೆ ಕೊಡಬೇಕಾದದ್ದು, ಮತ್ತು ಯುಗಯುಗಾಂತರಗಳಿಂದಲೂ ಶಾಂತವಾಗಿ-ಅಗೋಚರವಾಗಿ ಕೊಡುತ್ತಿದ್ದುದು ಜ್ಞಾನದ ಉಡುಗೊರೆಯನ್ನು, ಅಧ್ಯಾತ್ಮದ ಉಡು ಗೊರೆಯನ್ನು... ಪಾಶ್ಚಾತ್ಯರು ಉಳಿಯಬೇಕಾದರೆ ತಮಗೆ ಅತ್ಯಂತ ಅವಶ್ಯವಾಗಿ ಬೇಕಾದದ್ದು ಆಧ್ಯಾತ್ಮಿಕತೆ ಎಂದು. ಇಂತಹ ಆಧ್ಯಾತ್ಮಿಕತೆಯ ಘನ ಸತ್ಯಗಳನ್ನೂ ವೇದಾಂತದ ಮಹಾತತ್ತ್ವ ಗಳನ್ನೂ ಜಗತ್ತಿನಾದ್ಯಂತ ಹರಡುವುದಕ್ಕಾಗಿ ಪೌರುಷವಂತರಾದ ವ್ಯಕ್ತಿಗಳು ಬೇಕಾಗಿದ್ದಾರೆ.... ನಾವು ಹೊರನಾಡುಗಳಿಗೆ ಹೋಗಬೇಕು; ನಮ್ಮ ಆಧ್ಯಾತ್ಮಿಕತೆ ಹಾಗೂ ವೇದಾಂತ ಸಿದ್ಧಾಂತ ಗಳಿಂದ ಇಡೀ ಜಗತ್ತನ್ನೇ ಗೆಲ್ಲಬೇಕು.”

ಅದೇ ದಿನ ಸಂಜೆ ಸ್ವಾಮೀಜಿ “ನನ್ನ ಸಮರ ನೀತಿ” ಎಂಬ ಸಾರ್ವಜನಿಕ ಭಾಷಣ ಮಾಡಿದರು. ಮದ್ರಾಸಿನಲ್ಲಿ ಇದು ಅವರ ಪ್ರಥಮ ಸಾರ್ವಜನಿಕ ಭಾಷಣ. ಭಾಷಣಕ್ಕೆ ಮುಂಚೆ ಸ್ವಾಮೀಜಿ ಪ್ರೊ ॥ ಸುಂದರರಾಮ ಅಯ್ಯರರ ಹತ್ತಿರ “ನಾನಿಂದು ಎಲ್ಲವನ್ನೂ ಬಯಲಿಗೆಳೆಯಲಿದ್ದೇನೆ” ಎಂದರು. ‘ಎಲ್ಲವನ್ನೂ’ ಎಂದರೆ ತಾವು ಅಮೆರಿಕದಲ್ಲಿದ್ದಾಗ ಥಿಯಾಸೊಫಿಸ್ಟರು, ಬ್ರಾಹ್ಮ ಸಮಾಜೀಯರು, ಕ್ರೈಸ್ತ ಮಿಷನರಿಗಳು–ಇವರೆಲ್ಲ ತಮಗೆ ನಿರಂತರವಾಗಿ ನೀಡಿದ ಕಿರುಕುಳದ ವಿಷಯವನ್ನು. ಇಷ್ಟು ವರ್ಷಗಳಲ್ಲಿ ಅವರು ಒಂದೇ ಒಂದು ಸಲವಾದರೂ ಯಾರನ್ನೂ ನಿಂದಿಸಿದವರಲ್ಲ, ಟೀಕಿಸಿದವರಲ್ಲ. ತೀರ ಅತ್ಯಗತ್ಯವಾದಾಗ ಮಾತ್ರ ಯಾವಾಗಲಾದರೊಮ್ಮೆ ತಮ್ಮ ಕಾರ್ಯೋದ್ದೇಶದ ಹಿತದೃಷ್ಟಿಯಿಂದ, ತಮ್ಮನ್ನು ಸಮರ್ಥಿಸಿಕೊಂಡಿದ್ದಿರಬಹುದು. ಸಂನ್ಯಾಸಿ ಎಂದಿಗೂ ನಿಂದೆ-ಟೀಕೆಗಳ ಆಕ್ರಮಣದಿಂದ ಆತ್ಮರಕ್ಷಣೆ ಮಾಡಿಕೊಳ್ಳಬಾರದೆಂಬ ತತ್ತ್ವಕ್ಕೆ ಅವರು ಬದ್ಧರಾಗಿದ್ದರೆಂಬುದಕ್ಕೆ ಹಿಂದೆ ಅನೇಕ ಜ್ವಲಂತ ನಿದರ್ಶನಗಳನ್ನು ಕಂಡಿದ್ದೇವೆ. ಅಮೆರಿಕದಲ್ಲಂತೂ ತಮ್ಮ ಮೇಲೆ ಅಪವಾದಗಳ ಭಯಂಕರ ಚಂಡಮಾರುತವೇ ಬೀಸಿದರೂ ಅವರು ಪರ್ವತದಂತೆ ಅಚಲರಾಗಿ ನಿಂತರು. ಹೀಗಿರುವಾಗ ಇಲ್ಲಿ ಮದ್ರಾಸಿನ ಸಹಸ್ರಾರು ಜನರ ಮುಂದೆ, ಅದೂ ಅಷ್ಟು ಪ್ರಬಲ ಶಕ್ತಿಗಳಾದ ಥಿಯಾಸೊಫಿಸ್ಟರೇ ಮೊದಲಾದವರನ್ನು ನೇರವಾಗಿ ಖಂಡಿಸಬೇಕೆಂದು ನಿಶ್ಚಯಿಸಿದರೆ, ಅದಕ್ಕೆ ಪ್ರಬಲ ಕಾರಣಗಳೇ ಇದ್ದುವು. ಇದೀಗ ಅವರ ಸಹನೆಯ ಕಟ್ಟೆ ಒಡೆದಿತ್ತು. ತಮ್ಮ ವಿರೋಧಿಗಳ ಕೂಗಾಟವನ್ನು ಕೇಳಿ ಇಲ್ಲಿನ ತಮ್ಮ ಶಿಷ್ಯರೆಲ್ಲ ಕಂಗಾಲಾದುದನ್ನು ಕಂಡು ಅವರಿಗೆ ರೋಸಿಹೋಗಿತ್ತು. ಇದಲ್ಲದೆ ಥಿಯಾಸೊಫಿಸ್ಟರ ದುಷ್ಟತನ ವನ್ನು ತೋರಿಸುವಂತಹ ಪತ್ರವೊಂದು ಅವರಿಗೆ ಅವರ ಗುರುಭಾಯಿಗಳೊಬ್ಬರಿಂದ ದೊರೆ ತಿತ್ತು. ಸ್ವಾಮೀಜಿ ಅಮೆರಿಕೆಗೆ ಹೋದ ಹೊಸತರಲ್ಲಿ ಕೆಲ ಕಾಲ ದಿಕ್ಕುಗಾಣದಂತಾಗಿ ತೊಳಲಾಡು ತ್ತಿದ್ದಾಗ, ಅವರ ಅವಸ್ಥೆಯನ್ನು ಕಂಡು ಪರಮಾನಂದಗೊಂಡ ಥಿಯಾಸೊಫಿಸ್ಟರೊಬ್ಬರು ಭಾರತದ ತಮ್ಮ ಒಬ್ಬ ಸ್ನೇಹಿತರಿಗೆ ಬರೆದಿದ್ದರು–“ಅಂತೂ ಪಿಶಾಚಿ ಇಲ್ಲಿ ಸಾಯುತ್ತ ಬಿದ್ದಿದೆ! ಸದ್ಯ ನಾವು ಬಚಾವಾದೆವು!” ಎಂದು. ಇಂಥ ದುರ್ಬುದ್ಧಿಯ ಥಿಯಾಸೋಫಿಸ್ಟರು, ಅಮೆರಿಕದಲ್ಲಿ ಸ್ವಾಮೀಜಿಯವರಿಗೆ ನೆರವಾದವರು ತಾವೇ ಎಂಬಂತೆ ಭಾರತದಲ್ಲಿ ನಾಟಕವಾಡಿದ್ದರು. ಇದನ್ನು ಕಂಡಾಗಲೇ ಸ್ವಾಮೀಜಿ ಎಲ್ಲವನ್ನು ಬಯಲಿಗೆಳೆಯುವ ನಿರ್ಧಾರ ಮಾಡಿದ್ದು. ಆದರೆ ಅವರ ಸ್ನೇಹಿತರೂ ಹಿತೈಷಿಗಳೂ “ಸ್ವಾಮೀಜಿ, ದಯವಿಟ್ಟು ನೀವು ಸಾರ್ವಜನಿಕ ಸಭೆಯಲ್ಲಿ ಈ ವಿಷಯಗಳನ್ನೆಲ್ಲ ಎತ್ತದಿದ್ದರೆ ಒಳ್ಳೆಯದು. ಅದರಲ್ಲೂ ಥಿಯಾಸೊಫಿಕಲ್ ಸೊಸೈಟಿಯ ಹೆಸರ ನ್ನಂತೂ ಪ್ರಸ್ತಾಪಿಸಲೇಬೇಡಿ. ಏಕೆಂದರೆ ಅದರಿಂದ ನಮ್ಮ ಮುಂದಿನ ಕಾರ್ಯಗಳೆಲ್ಲ ಕೆಡುವ ಸಂಭವವಿದೆ,” ಎಂದು ಕೇಳಿಕೊಂಡರು. ಆದರೆ ಸ್ವಾಮೀಜಿ ಅದಕ್ಕೊಪ್ಪಿಕೊಳ್ಳಲಿಲ್ಲ. ಈ ಮೂರು ವರ್ಷಗಳವರೆಗೆ ಅವರು ಬಾಯಿ ಬಿಗಿಹಿಡಿದಿಟ್ಟುಕೊಂಡಿದ್ದರು. ಇನ್ನು ತಾವು ಸುಮ್ಮನಿರುವುದು ಒಳ್ಳೆಯದಲ್ಲವೆಂದು ಅವರಿಗೆ ಮನದಟ್ಟಾಗಿತ್ತು.

ಮದ್ರಾಸಿನ ಜನರು ತಮಗೆ ಏಕಪ್ರಕಾರವಾಗಿ ತೋರಿಸಿದ ವಿಶ್ವಾಸಕ್ಕಾಗಿ ಕೃತಜ್ಞತೆಗಳನ್ನರ್ಪಿ ಸುತ್ತ ಸ್ವಾಮೀಜಿ ಅಂದಿನ ತಮ್ಮ ಉಪನ್ಯಾಸವನ್ನು ಪ್ರಾರಂಭಿಸಿದರು. ಅಲ್ಲದೆ ತಾವು ನಾಲ್ಕು ವರ್ಷಗಳ ಹಿಂದೆ ಯಾವ ದಂಡಕಮಂಡಲುಧಾರಿಯಾದ ಪರಿವ್ರಾಜಕ ಸಂನ್ಯಾಸಿಯಾಗಿ ಮದ್ರಾ ಸನ್ನು ಪ್ರವೇಶಿಸಿದ್ದರೋ ಈಗಲೂ ತಾವು ಅದೇ ಸರಳ ಸಂನ್ಯಾಸಿಯಾಗಿ ಅವರ ಮುಂದೆ ನಿಂತಿರುವುದಾಗಿ ನುಡಿದರು. ಬಳಿಕ, ಕಳೆದ ಮೂರು ವರ್ಷಗಳಿಂದಲೂ ತಮಗೆ ವಿರುದ್ಧವಾಗಿ ವರ್ತಿಸಿದ ಘಾತಕ ಶಕ್ತಿಗಳ ಕುತಂತ್ರ, ನೀಚತನ, ದ್ರೋಹಗಳನ್ನು ನಿರ್ದಾಕ್ಷಿಣ್ಯವಾಗಿ ಬಯಲಿ ಗೆಳೆದರು.

ಸ್ವಾಮೀಜಿ ಹೀಗೆ ಬಹಿರಂಗವಾಗಿ ಅನೇಕ ಮುಖಂಡರ ಹೆಸರುಗಳನ್ನು ಬಯಲಿಗೆಳೆದಾಗ, ಆ ಮಾತುಗಳು ದೇಶದಾದ್ಯಂತ ಪತ್ರಿಕೆಗಳಲ್ಲಿ ಪ್ರಕಟವಾದಾಗ, ಎಂತಹ ಪ್ರತಿಕ್ರಿಯೆ ಎದ್ದಿರಬಹು ದೆಂದು ನಾವು ಊಹಿಸಬಹುದು. ಏಕೆಂದರೆ ಈ ಥಿಯಾಸೊಫಿಸ್ಟರು, ಬ್ರಾಹ್ಮಸಮಾಜೀಯರೇ ಮೊದಲಾದವರೆಲ್ಲ ದೇಶದಲ್ಲಿ ಆಗಲೇ ಸಾಕಷ್ಟು ಆದ್ಯತೆ ಪಡೆದವರು, ಆಳವಾಗಿ ಬೇರು ಬಿಟ್ಟವರು. ಆದರೆ ವಿವೇಕಾನಂದರೋ ಆಗ ತಾನೆ ಪ್ರಸಿದ್ಧಿಗೆ ಬರುತ್ತಿದ್ದ ಏಕಾಂಗಿ ಸಂನ್ಯಾಸಿ. ನಿಜಕ್ಕೂ ಅವರ ಮಾತಿನ ಪರಿಣಾಮ ಕೂಡಲೆ ವ್ಯಕ್ತವಾಯಿತು. ‘ವಿವೇಕಾನಂದ ಸ್ವಾಗತ ಸಮಿತಿ’ಯ ಅಧ್ಯಕ್ಷರಾದ ಜಸ್ಟಿಸ್ ಸುಬ್ರಹ್ಮಣ್ಯ ಅಯ್ಯರರು ಸ್ವಾಮೀಜಿಯವರೊಂದಿಗೆ ತಮ್ಮ ಸಂಬಂಧವನ್ನು ಕಡಿದುಕೊಂಡು ಬಿಟ್ಟರು. ಏಕೆಂದರೆ ಅವರು ಥಿಯಾಸೊಫಿಕಲ್ ಸೊಸೈಟಿಯ ಒಬ್ಬ ಪ್ರಮುಖ ಸದಸ್ಯರು. ಆದರೆ ಅವರ ಬಗ್ಗೆ ಸ್ವಾಮೀಜಿಯವರ ವಿಶ್ವಾಸಾದರಗಳು ಮಾತ್ರ ಸ್ವಲ್ಪವೂ ತಗ್ಗಲಿಲ್ಲ.

ಅಂತೂ ಸ್ವಾಮೀಜಿಯವರ ಮಾತಿನ ಪರಿಣಾಮ ಮಾತ್ರ ಬಲವಾಗಿಯೇ ಇತ್ತು. ಆದರೆ ಸ್ವಾಮೀಜಿ ಇದಕ್ಕೆ ಹೆದರುವವರಲ್ಲ. ಸ್ವಾಮಿ ಬ್ರಹ್ಮಾನಂದರಿಗೆ ಬರೆದ ಪತ್ರವೊಂದರಲ್ಲಿ ಅವರು ಆ ವಿಷಯದ ಬಗ್ಗೆ ಹೇಳುತ್ತಾರೆ, “ಈ ಥಿಯಾಸೊಫಿಸ್ಟರು ಮತ್ತಿತರರು ನನಗೆ ಅಮೆರಿಕದಲ್ಲಿ ಕಿರುಕುಳ ಕೊಟ್ಟದ್ದಲ್ಲದೆ ಇಲ್ಲಿಯೂ ಅದನ್ನು ಪ್ರಾರಂಭಿಸಿಬಿಟ್ಟರು. ಆದ್ದರಿಂದ ನಾನು ಅವರಿಗೆ ಸ್ವಲ್ಪ ಬಿಸಿ ಮುಟ್ಟಿಸಬೇಕಾಯಿತು. ಇದರಿಂದ ನನ್ನ ಕಲ್ಕತ್ತದ ಸ್ನೇಹಿತರಿಗೆ ಬೇಸರವಾದರೆ ನಾನೇನೂ ಮಾಡಲು ಬರುವಂತಿಲ್ಲ. ಇರಲಿ. ನೀನೇನೂ ಹೆದರಬೇಕಾಗಿಲ್ಲ; ನಾನೇನೂ ಏಕಾಂಗಿ ಯಾಗಿ ಕೆಲಸ ಮಾಡುತ್ತಿಲ್ಲ. ಭಗವಂತ ಸದಾ ನನ್ನೊಂದಿಗೆ ಇದ್ದಾನೆ. ಅಲ್ಲದೆ ನಾನು ಇನ್ನೇನು ತಾನೆ ಮಾಡಲು ಸಾಧ್ಯ?”

ಸ್ವಾಮೀಜಿ ಅಂದಿನ ತಮ್ಮ ಉಪನ್ಯಾಸವನ್ನು ಮುಂದುವರಿಸುತ್ತಾ ಸಮಾಜ ಸುಧಾರಣಾ ಸಂಘಗಳ ಬಗ್ಗೆ ಪ್ರಸ್ತಾಪಿಸಿದರು–“ಅವರು ಸ್ವಲ್ಪಸ್ವಲ್ಪವನ್ನೇ ಸುಧಾರಣೆ ಮಾಡಲು ನೋಡು ತ್ತಾರೆ. ಆದರೆ ನಾನು ಆಮೂಲಾಗ್ರವಾಗಿ ಸುಧಾರಣೆಯಾಗಬೇಕೆನ್ನುತ್ತೇನೆ. ಅವರಿಗೂ ನನಗೂ ವ್ಯತ್ಯಾಸವಿರುವುದು ಕಾರ್ಯವಿಧಾನದಲ್ಲಿ. ಅವರದು ವಿಧ್ವಂಸಕ ವಿಧಾನ, ನನ್ನದು ರಚನಾತ್ಮಕ ವಿಧಾನ. ನಾನು ‘ಸುಧಾರಣೆ’ಎಂಬುದನ್ನು ಒಪ್ಪುವುದಿಲ್ಲ. ನಾನೊಪ್ಪುವುದು ಬೆಳವಣಿಗೆಯನ್ನು. ನಾನು ಭಗವಂತನ ಸ್ಥಾನದಲ್ಲಿ ನಿಂತುಕೊಂಡು ಸಮಾಜಕ್ಕೆ ‘ನೀನು ಹೀಗೆಯೇ ಮಾಡಬೇಕು, ಹೀಗೆ ಕೂಡದು’ ಎಂದು ಅಪ್ಪಣೆ ಮಾಡಲಾರೆ. ಈ ಅದ್ಭುತ ರಾಷ್ಟ್ರೀಯ ಜೀವನದಿಯು ಯುಗ ಯುಗಗಳಿಂದಲೂ ಹರಿಯುತ್ತಿದೆ. ಅದು ಆ ದಿಕ್ಕಿನಲ್ಲಿ ಹರಿಯುತ್ತಿರುವುದು ಒಳ್ಳೆಯದೋ ಅಲ್ಲವೋ ಎಂದು, ಅದು ಹೀಗೆಯೇ ಹರಿಯಬೇಕು ಎಂದು, ಯಾರು ಹೇಳಬಲ್ಲರು?... ರಾಷ್ಟ್ರೀಯ ಬೆಳವಣಿಗೆಗೆ ಬೇಕಾದ ಆಹಾರವನ್ನು ನೀಡುತ್ತ ಹೋಗಿ. ಅದು ತನ್ನಷ್ಟಕ್ಕೆ ತನ್ನದೇ ಆದ ನಿಯಮಗಳಿಗನುಸಾರವಾಗಿ ಬೆಳೆಯುತ್ತದೆ. ಬೆಳವಣಿಗೆಯನ್ನು ಮಾತ್ರ ಯಾರೂ ಬಲಾತ್ಕಾರ ದಿಂದ ಸಾಧಿಸುವಂತಿಲ್ಲ.”

ಅಂದಿನ ದಿನಗಳಲ್ಲಿ ವಿಗ್ರಹಾರಾಧನೆಯನ್ನು ಖಂಡಿಸುವುದು ಬುದ್ಧಿಜೀವಿಗಳೆನ್ನಿಸಿಕೊಂಡವರ ಫ್ಯಾಶನ್ ಆಗಿತ್ತು. ಈ ವಿಷಯವನ್ನು ಪ್ರಸ್ತಾಪಿಸುತ್ತ ಸ್ವಾಮೀಜಿ ಹೇಳಿದರು, “ನಾನೂ ಕೂಡ ವಿಗ್ರಹಾರಾಧನೆ ತಪ್ಪೆಂದೇ ತಿಳಿದುಕೊಂಡಿದ್ದವನು. ಅದಕ್ಕೆ ಪ್ರಾಯಶ್ಚಿತ್ತವೆಂಬಂತೆ, ಸಕಲವನ್ನೂ ವಿಗ್ರಹಾರಾಧನೆಯ ಮೂಲಕವೇ ಸಿದ್ಧಿಸಿಕೊಂಡ ಶ್ರೀರಾಮಕೃಷ್ಣ ಪರಮಹಂಸರ ಪದತಲದಲ್ಲಿ ಕುಳಿತು ನಾನು ಪಾಠ ಕಲಿಯಬೇಕಾಯಿತು. ವಿಗ್ರಹಾರಾಧನೆಯಿಂದ ರಾಮಕೃಷ್ಣ ಪರಮಹಂಸ ರಂಥವರು ನಿರ್ಮಾಣಗೊಳ್ಳುವಂತಿದ್ದರೆ ನೀವು ಯಾವುದನ್ನು ಆರಿಸಿಕೊಳ್ಳುತ್ತೀರಿ–ವಿಗ್ರಹ ಗಳನ್ನೋ ಅಥವಾ ಈ ಸುಧಾರಕರ ಮಾತನ್ನೋ? ನನಗಿದಕ್ಕೆ ಉತ್ತರ ಬೇಕು. ಶ್ರೀರಾಮಕೃಷ್ಣ ರಂಥವರೊಬ್ಬರನ್ನು ನೀವು ತಯಾರಿಸುವ ಹಾಗಿದ್ದರೆ, ಇಗೊ ತೆಗೆದುಕೊಳ್ಳಿ, ಇನ್ನೂ ಒಂದು ಸಾವಿರ ವಿಗ್ರಹಗಳನ್ನು!... ಭರತಖಂಡದಲ್ಲಿ ಸುಧಾರಕರಿಗೆ ಕೊರತೆಯೆ? ನೀವು ಭರತಖಂಡದ ಇತಿಹಾಸವನ್ನು ಓದಿರುವಿರಾ? ರಾಮಾನುಜರು ಯಾರು? ಶಂಕರರು ಯಾರು? ನಾನಕರು ಯಾರು? ಚೈತನ್ಯರು ಯಾರು? ಕಬೀರರು ಯಾರು? ದಾದೂ ದಯಾಲರು ಯಾರು? ಒಬ್ಬರಾದ ಮೇಲೊಬ್ಬರಂತೆ ಧರ್ಮದಿಗಂತದಲ್ಲಿ ಮೂಡಿದ ಈ ನಕ್ಷತ್ರ ಶ್ರೇಣಿಯಂತಿರುವವರೆಲ್ಲ ಯಾರು? ರಾಮಾನುಜರು ಅಂತ್ಯಜರಿಗಾಗಿ ಮರುಗಲಿಲ್ಲವೆ? ಚಂಡಾಲರನ್ನು ಕೂಡ ತಮ್ಮ ಪಂಗಡಕ್ಕೆ ಸೇರಿಸಿಕೊಳ್ಳಲು ತಮ್ಮ ಜೀವಮಾನವೆಲ್ಲ ಪ್ರಯತ್ನಪಡಲಿಲ್ಲವೆ? ನಾನಕರು ಹಿಂದೂಗಳೊಂದಿಗೆ ಮತ್ತು ಮಹಮ್ಮದೀಯರೊಂದಿಗೆ ಕಲೆತು ಹೊಸ ಸೌಹಾರ್ದಭಾವವನ್ನು ತರಲು ಪ್ರಯತ್ನಿಸ ಲಿಲ್ಲವೆ? ಹೌದು; ಅವರೆಲ್ಲರೂ ಪ್ರಯತ್ನಿಸಿದರು. ಈಗಲೂ ಆ ಕಾರ್ಯ ಮುಂದುವರಿಯುತ್ತಿದೆ. ಆದರೆ ವ್ಯತ್ಯಾಸವಿಷ್ಟೆ: ಅವರಲ್ಲಿ ಇಂದಿನ ಸಮಾಜ ಸುಧಾರಕರ ಅಟ್ಟಹಾಸವಿರಲಿಲ್ಲ. ಆಧುನಿಕ ಸುಧಾರಕರ ಬಾಯಿಂದ ಹೊರಬೀಳುವ ನಿಂದನೆಯ ಸುರಿಮಳೆ ಅವರಲ್ಲಿರಲಿಲ್ಲ. ಅವರ ನಾಲಿಗೆ ಯಾವಾಗಲೂ ಶುಭವನ್ನೇ ನುಡಿಯಿತು... ನಾವೂ ಇಂದು ಆ ವಿಧಾನವನ್ನು ಪ್ರಯತ್ನಿಸಿ ನೋಡೋಣ.”

ಹೀಗೆ ತಮ್ಮ ಕಾರ್ಯೋದ್ದೇಶಕ್ಕೆ ವಿರುದ್ಧವಾಗಿರುವ ಹಲವಾರು ಅಂಶಗಳನ್ನು ಪ್ರಸ್ತಾಪಿಸಿದ ಸ್ವಾಮೀಜಿ, ಈಗ ತಮ್ಮ ಯೋಜನೆಯೇನು, ತಮ್ಮ ಕಾರ್ಯವಿಧಾನವೆಂಥದು, ಎಂಬ ವಿಷಯಕ್ಕೆ ಬಂದರು. ತಾವು ಭವ್ಯವಾದ ಆತ್ಯಾಧುನಿಕ ಯೋಜನೆಯೇನನ್ನೋ ರೂಪಿಸಿರುವುದಾಗಿ ಅವರು ಘೋಷಿಸಲಿಲ್ಲ. ಬದಲಾಗಿ ಹೇಳಿದರು, “ನಮ್ಮ ಸನಾತನ ಮಹರ್ಷಿಗಳು-ಮಹಾತ್ಮರು ತೋರಿಸಿ ಕೊಟ್ಟ ಮಾರ್ಗವನ್ನು ಅನುಸರಿಸುವುದೇ ನನ್ನ ಕಾರ್ಯವಿಧಾನ... ಅವರು ಶಕ್ತಿ-ಪವಿತ್ರತೆ- ಸ್ಫೂರ್ತಿಗಳ ಚಿಲುಮೆಗಳಾಗಿದ್ದರು. ಅವರು ಅತ್ಯಂತ ಅದ್ಭುತ ಕಾರ್ಯಗಳನ್ನು ಮಾಡಿದರು. ಈಗ ನಾವೂ ಅತ್ಯಂತ ಅದ್ಭುತವಾದದ್ದನ್ನು ಸಾಧಿಸಬೇಕಾಗಿದೆ. ಅಂದಿಗಿಂತ ಇಂದಿನ ಪರಿಸ್ಥಿತಿ ಸ್ವಲ್ಪ ಭಿನ್ನವಾಗಿದೆ. ಅದಕ್ಕೆ ಸರಿಯಾಗಿ ನಾವು ನಮ್ಮ ಕಾರ್ಯವಿಧಾನವನ್ನು ಸ್ವಲ್ಪ ಬದಲಿಸಿಕೊಳ್ಳ ಬೇಕಾಗಿದೆ, ಅಷ್ಟೆ... ಪ್ರತಿಯೊಬ್ಬ ವ್ಯಕ್ತಿಗೂ ಒಂದು ಜೀವನೋದ್ದೇಶವಿರುವಂತೆ ಪ್ರತಿ ಯೊಂದು ರಾಷ್ಟ್ರಕ್ಕೂ ಒಂದು ಜೀವನಾದರ್ಶವಿದೆ. ಅದೇ ಅದರ ಜೀವಾಳ; ಅದೇ ಅದರ ಪ್ರಧಾನ ತಾನ. ಈ ಪ್ರಧಾನ ತಾನದಲ್ಲಿ ಮಿಕ್ಕೆಲ್ಲ ತಾನಗಳು ಮಿಲನಗೊಂಡು ಲೀನವಾಗುವುವು... ಧರ್ಮವೇ ಭರತಖಂಡದ ಜೀವಸ್ವರ. ರಾಷ್ಟ್ರಜೀವನದ ಉಳಿದ ಸ್ವರಗಳೆಲ್ಲ ಇದಕ್ಕೆ ಅಧೀನ. ನೀವು ನಿಮ್ಮ ಧರ್ಮವನ್ನು ಕಡೆಗಣಿಸಿ ರಾಜಕೀಯವನ್ನೋ ಇನ್ನಾವುದನ್ನೋ ನಿಮ್ಮ ರಾಷ್ಟ್ರದ ಜೀವಾಳವನ್ನಾಗಿ ಮಾಡಿಕೊಳ್ಳಲು ಪ್ರಯತ್ನಿಸಿದರೆ ನೀವು ನಿರ್ನಾಮವಾಗುವಿರಿ... ನಿಮ್ಮ ಬೆನ್ನೆಲುಬಿನಂತಿರುವ ಧರ್ಮವು ನಿಮ್ಮ ಸಮಸ್ತ ನರನಾಡಿಗಳಲ್ಲಿ ಮಿಡಿಯುತ್ತಿರಲಿ.

“ಆದ್ದರಿಂದ ಭಾರತದಲ್ಲಿ ಯಾವುದೇ ಬಗೆಯ ಪ್ರಗತಿಯಾಗಬೇಕಾದರೆ ಮೊದಲು ಧಾರ್ಮಿಕ ಆಂದೋಲನವಾಗಬೇಕು. ನಾವು ಗಮನ ಹರಿಸಬೇಕಾದ ಮೊತ್ತ ಮೊದಲ ಕಾರ್ಯವೆಂದರೆ, ನಮ್ಮ ವೇದೋಪನಿಷತ್ತುಗಳಲ್ಲಿ ಹುದುಗಿರುವ ಅದ್ಭುತ ಸತ್ಯಗಳನ್ನು ಹೊರತರುವುದು. ಆ ಸತ್ಯ ಗಳನ್ನು ಗ್ರಂಥಗಳೊಳಗಿನಿಂದ ಹೊರತರಬೇಕು, ಮಠಗಳೊಳಗಿನಿಂದ ಹೊರತರಬೇಕು, ಅರಣ್ಯ ಗಳಿಂದ ಹೊರತರಬೇಕು, ಕೆಲವೇ ಕೆಲವು ವ್ಯಕ್ತಿಗಳ ಕಪಿಮುಷ್ಟಿಯಿಂದ ಹೊರತರಬೇಕು. ಹೀಗೆ ಆ ಘನಸತ್ಯಗಳನ್ನು ಹೊರತಂದು ಪೂರ್ವದಿಂದ ಪಶ್ಚಿಮದವರೆಗೆ, ಉತ್ತರದಿಂದ ದಕ್ಷಿಣದ ವರೆಗೆ, ಸಿಂಧೂವಿನಿಂದ ಬ್ರಹ್ಮಪುತ್ರದವರೆಗೆ, ಹಿಮಾಲಯದಿಂದ ಕನ್ಯಾಕುಮಾರಿಯವರೆಗೆ ಕಾಳ್ಗಿಚ್ಚಿನಂತೆ ಹರಡಿ ಪ್ರಜ್ವಲಿಸುವಂತೆ ಮಾಡುವುದೇ ನಮ್ಮ ಪ್ರಥಮ ಕಾರ್ಯ.

“ಆದ್ದರಿಂದ ಸ್ನೇಹಿತರೇ, ನಮ್ಮ ಧರ್ಮಗ್ರಂಥಗಳಲ್ಲಿರುವ ಸತ್ಯಗಳನ್ನು ಭಾರತದಲ್ಲೂ ಹೊರನಾಡುಗಳಲ್ಲೂ ಪ್ರಚಾರ ಮಾಡಬಲ್ಲಂತಹ ಯುವಕರನ್ನು ತರಬೇತಿಗೊಳಿಸಲು ಸಂಸ್ಥೆ ಗಳನ್ನು ಸ್ಥಾಪಿಸುವುದು ನನ್ನ ಒಂದು ಕಾರ್ಯಯೋಜನೆ. ಮನುಷ್ಯರು! ಮನುಷ್ಯರು! ನಮ ಗಿಂದು ಬೇಕಾಗಿರುವುದು ಮನುಷ್ಯರು! ಮನುಷ್ಯರು ಸಿಕ್ಕರೆ ಮಿಕ್ಕಿದ್ದೆಲ್ಲವೂ ತಾನಾಗಿಯೇ ಬಂದೊದಗುವುದು. ಆದರೆ ಆಶಿಷ್ಠ-ದೃಢಿಷ್ಠ-ಬಲಿಷ್ಠರಾದ, ಅತ್ಯಂತ ಪ್ರಾಮಾಣಿಕರಾದ, ವೀರ್ಯವಂತ ಯುವಕರು ಬೇಕಾಗಿದ್ದಾರೆ. ಇಂತಹ ನೂರು ಜನ ಸಿಕ್ಕಿದರೆ ಇಡೀ ಪ್ರಪಂಚದ ವಿಚಾರಧಾರೆಯನ್ನೇ ಬದಲಿಸಬಹುದು. ಇಚ್ಛಾಶಕ್ತಿಯು ಎಲ್ಲಕ್ಕಿಂತ ಪ್ರಬಲವಾದದ್ದು. ಅದರ ಮುಂದೆ ಉಳಿದುದೆಲ್ಲವೂ ತಗ್ಗಿ ನಡೆಯಲೇಬೇಕು. ಏಕೆಂದರೆ ಆ ಇಚ್ಛಾಶಕ್ತಿ ಬರುವುದು ಭಗವಂತನಿಂದಲೇ. ಪ್ರಬಲ-ಪರಿಶುದ್ಧ ಇಚ್ಛಾಶಕ್ತಿಯು ಸರ್ವಶಕ್ತವಾದುದು. ಹೋಗಿ, ನಿಮ್ಮ ಧರ್ಮದ ಘನ ಸತ್ಯಗಳನ್ನು ಜಗತ್ತಿಗೆ ಬೋಧಿಸಿ. ಜಗತ್ತು ಅದಕ್ಕಾಗಿ ಎದುರುನೋಡುತ್ತಿದೆ. ಶತಶತಮಾನಗಳಿಂದಲೂ, ಜನರನ್ನು ಅಧೋಗತಿಗಿಳಿಸುವಂತಹ ಧರ್ಮವನ್ನು ಬೋಧಿಸಲಾಗಿದೆ; ಅವರೆಲ್ಲ ಕೆಲಸಕ್ಕೆ ಬಾರದ ನಿಷ್ಪ್ರಯೋಜಕರು ಎಂದು ಹೇಳಿಹೇಳಿ ಅವರನ್ನು ದುರ್ಬಲಗೊಳಿಸಲಾ ಗಿದೆ. ಜಗತ್ತಿನಲ್ಲಿ ಎಲ್ಲೆಲ್ಲಿಯೂ ಸಾಮಾನ್ಯ ಜನರಿಗೆ ‘ನೀವು ಮನುಷ್ಯರೇ ಅಲ್ಲ, ಕೇವಲ ಮೃಗಗಳು’ ಎಂದು ಹೇಳುತ್ತ ಬರಲಾಗಿದೆ. ಶತಶತಮಾನಗಳಿಂದ ಅವರನ್ನೆಲ್ಲ ಎಷ್ಟರಮಟ್ಟಿಗೆ ಹೆದರಿಸಿ ಬೆದರಿಸಲಾಗಿದೆಯೆಂದರೆ, ಬಹುಮಟ್ಟಿಗೆ ಅವರು ಮೃಗಗಳಂತೆಯೇ ಆಗಿಬಿಟ್ಟಿದ್ದಾರೆ. ತಾವು ಆತ್ಮಸ್ವರೂಪರು ಎಂದು ತಿಳಿಯಲು ಅವರಿಗೆ ಅವಕಾಶವೇ ಇರಲಿಲ್ಲ. ಆತ್ಮದ ವಿಚಾರ ವಾಗಿ ಕೇಳುವುದಕ್ಕೂ ಅವರಿಗೆ ಅವಕಾಶವಿರಲಿಲ್ಲ. ಕನಿಷ್ಠರಲ್ಲಿ ಕನಿಷ್ಠರಾದವರಲ್ಲಿಯೂ ಆತ್ಮ ನಿದ್ದಾನೆಂಬ ವಿಚಾರ ಅವರಿಗೆ ತಿಳಿಯಲಿ. ಈ ಆತ್ಮವನ್ನು ಖಡ್ಗ ಕತ್ತರಿಸಲಾರದು, ಬೆಂಕಿ ಸುಡ ಲಾರದು, ಗಾಳಿ ಒಣಗಿಸಲಾರದು; ಅದು ಆದ್ಯಂತರಹಿತ, ಅಮರ, ನಿತ್ಯ, ಶುದ್ಧ, ಸರ್ವಶಕ್ತ, ಸರ್ವಾಂತರ್ಯಾಮಿ ಎಂಬ ವಿಚಾರ ಅವರಿಗೆ ತಿಳಿಯಲಿ. ಅವರಿಗೆ ತಮ್ಮಲ್ಲಿ ತಮಗೆ ಶ್ರದ್ಧೆ ಯುಂಟಾಗಲಿ, ಆತ್ಮವಿಶ್ವಾಸವುಂಟಾಗಲಿ.

“ಇಂಗ್ಲಿಷರಿಗೂ ನಿಮಗೂ ವ್ಯತ್ಯಾಸ ಇರುವುದೆಲ್ಲಿ? ಅವರು ಆ ಬಗ್ಗೆ ಏನು ಬೇಕಾದರೂ ಹೇಳಿಕೊಳ್ಳಲಿ. ಆದರೆ ನನಗೆ ಆ ವ್ಯತ್ಯಾಸವೇನೆಂದು ಗೊತ್ತಾಗಿದೆ. ಆಂಗ್ಲೇಯನಲ್ಲಿ ಆತ್ಮ ವಿಶ್ವಾಸವಿದೆ, ನಿಮ್ಮಲ್ಲಿ ಅದು ಇಲ್ಲ, ಅಷ್ಟೆ. ಅವನು, ತಾನೊಬ್ಬ ಆಂಗ್ಲೇಯ, ಆದ್ದರಿಂದ ತಾನು ಏನು ಬೇಕಾದರೂ ಸಾಧಿಸಬಲ್ಲೆ ಎಂದು ನಂಬುತ್ತಾನೆ. ಅವನ ಈ ಆತ್ಮವಿಶ್ವಾಸವೇ ಅವನೊಳಗಿನ ದೈವೀಶಕ್ತಿಯನ್ನು ವ್ಯಕ್ತಗೊಳಿಸುತ್ತದೆ. ಮತ್ತು ಇಂತಹ ಆತ್ಮವಿಶ್ವಾಸದಿಂದಾಗಿ ಅವನು ತನ್ನಿಚ್ಛೆ ಯಂತೆ ಏನನ್ನಾದರೂ ಸಾಧಿಸಲೂ ಬಲ್ಲ. ಆದರೆ ನೀವು ಏನನ್ನೂ ಮಾಡಲಾರಿರಿ; ಏಕೆಂದರೆ ನೀವು ನಿಷ್ಪ್ರಯೋಜಕರಾಗುತ್ತ ಬರುತ್ತಿದ್ದೀರಿ; ವ್ಯಕ್ತಿತ್ವವೇ ಇಲ್ಲದವರಾಗುತ್ತಿದ್ದೀರಿ. ನಮಗಿಂದು ಬೇಕಾಗಿರುವುದು ಶಕ್ತಿ. ಆದ್ದರಿಂದ ಆತ್ಮವಿಶ್ವಾಸವಿಡಿ. ನಾವು ದುರ್ಬಲರಾಗಿರುವುದರಿಂದಲೇ ಈ ಮಾಯಾ-ಮಂತ್ರ-ಪವಾಡಗಳಂತಹ ಕ್ಷುದ್ರ ವಿಚಾರಗಳಿಗೆ ಮನಸ್ಸು ಕೊಡುತ್ತೇವೆ. ಅವುಗಳಲ್ಲಿ ಮಹಾಸತ್ವವೇ ಅಡಗಿರಬಹುದೇನೋ, ಆದರೆ ಅವು ನಮ್ಮನ್ನು ವಿನಾಶದತ್ತಲೇ ಎಳೆದೊಯ್ದಿವೆ. ಮೊದಲು ನಿಮ್ಮ ನರಮಂಡಲವನ್ನು ದೃಢಪಡಿಸಿಕೊಳ್ಳಿ. ನಮಗಿಂದು ಬೇಕಾಗಿರುವುದು ಕಬ್ಬಿಣದ ಮಾಂಸಖಂಡಗಳು, ಉಕ್ಕಿನ ನರಮಂಡಲ. ನಾವು ಸಾಕಷ್ಟು ಕಾಲ ಅತ್ತದ್ದಾಯಿತು. ಗೋಳಾಡಿದ್ದಾ ಯಿತು. ಇನ್ನು ಅಳುವುದು ಸಾಕು; ನಿಮ್ಮ ಕಾಲ ಮೇಲೆ ನಿಂತು ಪುರುಷಸಿಂಹರಾಗಿ. ನಮಗಿಂದು ಬೇಕಾಗಿರುವುದು ಪುರುಷಸಿಂಹರನ್ನು ನಿರ್ಮಾಣ ಮಾಡಬಲ್ಲ ಸಿದ್ಧಾಂತಗಳು. ನಮಗಿಂದು ಎಲ್ಲೆಡೆಯೂ ಬೇಕಾಗಿರುವುದು ಪುರುಷಸಿಂಹರನ್ನು ನಿರ್ಮಿಸಬಲ್ಲ ವಿದ್ಯಾಭ್ಯಾಸ. ಯಾವುದು ನಿಮ್ಮನ್ನು ದೈಹಿಕವಾಗಿ, ಬೌದ್ಧಿಕವಾಗಿ, ಆಧ್ಯಾತ್ಮಿಕವಾಗಿ ದುರ್ಬಲರನ್ನಾಗಿಸಿರುವುದೋ ಅವೆಲ್ಲ ವನ್ನೂ ವಿಷದಂತೆ ತಿರಸ್ಕರಿಸಿ. ಅವುಗಳಲ್ಲಿ ಯಾವುದೇ ಸತ್ವವಿರಲಾರದು; ಅವು ಸತ್ಯವಾಗಿರ ಲಾರವು. ಇದೇ ಸತ್ಯದ ನಿಜವಾದ ಪರೀಕ್ಷೆ; ಸತ್ಯವು ಶಕ್ತಿಪ್ರದವಾದುದು, ಸತ್ಯವು ಪವಿತ್ರ ವಾದುದು. ಯಾವುದು ಸತ್ಯವೋ ಅದು ಶಕ್ತಿಪ್ರದವಾಗಿರಲೇಬೇಕು, ಜ್ಞಾನಪ್ರದವಾಗಿರಲೇಬೇಕು, ಸ್ಫೂರ್ತಿ-ಉತ್ಸಾಹದಾಯಕವಾಗಿರಲೇಬೇಕು. ಮಾಯಮಂತ್ರಾದಿ ರಹಸ್ಯವಿದ್ಯೆಗಳಲ್ಲಿ ಏನಾದರೂ ಒಂದು ಹುಲ್ಲುಕಡ್ಡಿಯಷ್ಟು ಸತ್ಯವಿರಬಹುದು; ಆದರೆ ಅವು ಸಾಧಾರಣವಾಗಿ ನಮ್ಮನ್ನು ದುರ್ಬಲಗೊಳಿಸುವಂಥವು. ನನ್ನನ್ನು ನಂಬಿ, ನನಗೆ ಆ ಬಗ್ಗೆ ಒಂದು ಇಡೀ ಜೀವನದ ಅನುಭವ ವಿದೆ–ಈ ರಹಸ್ಯವಿದ್ಯೆಗಳು-ಪವಾಡಗಳೆಲ್ಲ ಮನುಷ್ಯನನ್ನು ದುರ್ಬಲಗೊಳಿಸುವುವೆಂಬುದೇ ನನ್ನ ತೀರ್ಮಾನ. ನಾನು ಭಾರತವನ್ನೆಲ್ಲ ಸಂಚರಿಸಿದ್ದೇನೆ, ಪ್ರತಿಯೊಂದು ಗಿರಿ ಗುಹೆಯನ್ನೂ ಹುಡುಕಿ ದ್ದೇನೆ, ಹಿಮಾಲಯದಲ್ಲಿ ವಾಸಿಸಿದ್ದೇನೆ. ತಮ್ಮ ಜೀವಮಾನವೆಲ್ಲ ಅಲ್ಲಿಯೇ ವಾಸಿಸಿದವರನ್ನೂ ನಾನು ಕಂಡಿದ್ದೇನೆ. ನಾನು ನನ್ನ ರಾಷ್ಟ್ರವನ್ನು ಪ್ರೀತಿಸುತ್ತೇನೆ. ಆದ್ದರಿಂದ ನೀವು ಇನ್ನಷ್ಟು ಮತ್ತಷ್ಟು ದುರ್ಬಲರಾಗುವುದನ್ನು, ಅಧೋಗತಿಗಿಳಿಯುವುದನ್ನು ನಾನು ಸಹಿಸಲಾರೆ. ಆದ ಕಾರಣವೇ ನಾನು ನಿಮಗಾಗಿ, ಸತ್ಯದ ಸಲುವಾಗಿ, ‘ನಿಲ್ಲಿ, ಇನ್ನು ಜಾರಬೇಡಿ!’ ಎಂದು ಕೂಗ ಬೇಕಾಗಿದೆ. ನನ್ನ ಜನಾಂಗದ ಈ ಅವನತಿಯ ವಿರುದ್ಧ ನನ್ನ ದನಿಯೆತ್ತಲೇಬೇಕಾಗಿದೆ. ನಿಮ್ಮನ್ನು ಸತ್ವಹೀನರನ್ನಾಗಿಸುವ ಈ ಕ್ಷುದ್ರವಾದ ರಹಸ್ಯವಿದ್ಯೆಗಳನ್ನು ತ್ಯಜಿಸಿರಿ, ಮತ್ತು ಶಕ್ತಿವಂತರಾಗಿ. ಜಾಜ್ವಲ್ಯಮಾನವಾದ, ಶಕ್ತಿದಾಯಕವಾದ ತತ್ತ್ವಗಳ ಗಣಿಯಾದ ಉಪನಿಷತ್ತುಗಳೆಡೆಗೆ ನಡೆಯಿರಿ. ಈ ಜಗತ್ತಿನಲ್ಲಿ ಅತ್ಯುನ್ನತ ಸತ್ಯಗಳೆಲ್ಲ ಅತ್ಯಂತ ಸರಳವಾದವುಗಳು; ನಿಮ್ಮ ಅಸ್ತಿತ್ವದಷ್ಟೇ ಸರಳ–ಈ ಸತ್ಯಗಳು. ಅಂತಹ ಉಪನಿಷತ್ತುಗಳ ಸತ್ಯಗಳು ನಿಮ್ಮ ಮುಂದಿವೆ. ಅವುಗಳನ್ನು ಕೈಗೆತ್ತಿಕೊಳ್ಳಿ, ಅವುಗಳಿಗನುಸಾರವಾಗಿ ಜೀವನ ನಡೆಸಿ. ಆಗ ಭಾರತದ ಉದ್ಧಾರ ಸ್ವತಸ್ಸಿದ್ಧ.

“ಇನ್ನೊಂದು ಮಾತು ಹೇಳಿ ಈ ಭಾಷಣವನ್ನು ಮುಗಿಸುತ್ತೇನೆ. ಕೆಲವರು ದೇಶಪ್ರೇಮದ ಮಾತನಾಡುತ್ತಾರೆ. ದೇಶಪ್ರೇಮದ ಆದರ್ಶವನ್ನು ನಾನು ಒಪ್ಪುತ್ತೇನೆ, ಮತ್ತು ನನ್ನದೇ ಆದ ದೇಶಪ್ರೇಮದ ಒಂದು ಆದರ್ಶವೂ ನನಗಿದೆ. ಯಾವುದೇ ಮಹಾಕಾರ್ಯವನ್ನು ಸಾಧಿಸಲು ಮೂರು ಅಂಶಗಳು ಅತ್ಯಗತ್ಯ. ಮೊದಲನೆಯದು, ಹೃದಯಾಂತರಾಳದಿಂದ ಹೊಮ್ಮುವ ಅನು ಕಂಪ, ಪ್ರೀತಿ. ಕೇವಲ ಬುದ್ಧಿವಂತಿಕೆಯಲ್ಲಿ, ತರ್ಕದಲ್ಲಿ ಏನಿದೆ? ಅವು ಕೆಲವು ಹೆಜ್ಜೆ ಮಾತ್ರ ಮುಂದುವರಿದು ಅಲ್ಲಿಯೇ ನಿಂತು ಬಿಡುತ್ತವೆ. ಆದರೆ ಸ್ಫೂರ್ತಿ ಚಿಮ್ಮುವುದು ಹೃದಯದಿಂದ. ಪ್ರೇಮವೆಂಬುದು ಅತ್ಯಂತ ದುರ್ಭೇದ್ಯವಾದ ಬಾಗಿಲುಗಳನ್ನೂ ತೆರೆಯಿಸುತ್ತದೆ. ಜಗತ್ತಿನ ರಹಸ್ಯಗಳಿಗೆಲ್ಲ ಈ ಪ್ರೇಮವೇ ಕೀಲಿಕೈ. ಆದ್ದರಿಂದ ನನ್ನ ಭಾವೀ ದೇಶಭಕ್ತರೇ, ಸಮಾಜ ಸುಧಾರಕರೇ, ಮೊದಲು ಹೃದಯವಂತರಾಗಿ. ನಿಮ್ಮಲ್ಲಿ ಅನುಕಂಪೆಯಿದೆಯೆ? ಕೋಟ್ಯನುಕೋಟಿ ದೇವಸಂತಾನರು, ಪುಷಿಸಂತಾನರು ಇಂದು ಕೇವಲ ಪಶುಸಮಾನರಾಗಿರುವುದನ್ನು ಕಂಡು ನಿಮ್ಮ ಹೃದಯ ಮರುಗುತ್ತದೆಯೆ? ಲಕ್ಷೋಪಲಕ್ಷ ಜನರು ಉಪವಾಸಬಿದ್ದು ನರಳುತ್ತಿರುವುದನ್ನು ಕಂಡು, ನೂರಾರು ವರ್ಷಗಳಿಂದಲೂ ಹಾಗೆಯೇ ನರಳುತ್ತಲೇ ಇರುವುದನ್ನು ಕಂಡು, ನಿಮ್ಮ ಹೃದಯ ಪರಿತಪಿಸುತ್ತದೆಯೆ? ಅಜ್ಞಾನವು ಭಾರತಭೂಮಿಯನ್ನು ಕಾರ್ಮೋಡದಂತೆ ಕವಿಯುತ್ತಿ ರುವುದು ನಿಮ್ಮ ಹೃದಯವನ್ನು ತಟ್ಟುತ್ತದೆಯೆ? ಅದು ನಿಮ್ಮನ್ನು ಅಶಾಂತರನ್ನಾಗಿಸಿದೆಯೆ? ನಿಮ್ಮನ್ನು ನಿದ್ರೆಗೆಡುವಂತೆ ಮಾಡಿದೆಯೆ?ಆ ಮರುಕವು ನಿಮ್ಮ ರಕ್ತದಲ್ಲಿ ಬೆರೆತು, ಧಮನಿಗಳಲ್ಲಿ ಹರಿದು, ನಿಮ್ಮ ಹೃದಯಸ್ಪಂದನದೊಂದಿಗೆ ಮಿಡಿಯುತ್ತಿದೆಯೆ? ಅದು ನಿಮ್ಮನ್ನು ಹುಚ್ಚರಂತಾ ಗಿಸಿದೆಯೆ? ನಮ್ಮ ದೇಶಕ್ಕೆ ಬಡಿದಿರುವ ದಾರಿದ್ರ್ಯದ ಚಿಂತೆಯು ನಿಮ್ಮ ಮನಸ್ಸು-ಬುದ್ಧಿಗಳನ್ನೆಲ್ಲ ಆವರಿಸಿದೆಯೆ? ಮತ್ತು ತತ್ಪರಿಣಾಮವಾಗಿ, ನೀವು ನಿಮ್ಮ ಹೆಸರು-ಕೀರ್ತಿ, ಮಡದಿ-ಮಕ್ಕಳು, ಆಸ್ತಿ-ಪಾಸ್ತಿ, ಕಡೆಗೆ ನಿಮ್ಮ ಸ್ವಂತ ದೇಹ–ಎಲ್ಲವನ್ನೂ ಮರೆತಿರುವಿರಾ? ನೀವು ನಿಜಕ್ಕೂ ಹಾಗೆ ಮಾಡಿರುವಿರಾ? ಹಾಗಿದ್ದರೆ ತಿಳಿದುಕೊಳ್ಳಿ–ಅದು ದೇಶಪ್ರೇಮದ ಮೊದಲ ಮೆಟ್ಟಿಲು, ಮೊಟ್ಟ ಮೊದಲ ಮೆಟ್ಟಿಲು! ನಾನು ಅಮೆರಿಕೆಗೆ ಹೋದದ್ದು ನಿಮ್ಮಲ್ಲಿ ಅನೇಕರು ಭಾವಿಸಿರುವಂತೆ ಸರ್ವಧರ್ಮ ಸಮ್ಮೇಳನಕ್ಕಾಗಿ ಅಲ್ಲ. ಈ ಮರುಕದ ಭೂತ, ನನ್ನ ಜನತೆಯ ಮೇಲಿನ ಅನು ಕಂಪೆಯ ಭೂತ ನನ್ನನ್ನು ಹಿಡಿದುಕೊಂಡಿತ್ತು. ನಾನು ಹಲವಾರು ವರ್ಷ ಭಾರತದಲ್ಲೆಲ್ಲ ಸಂಚರಿ ಸಿದೆ. ಆದರೆ ನನ್ನ ದೇಶಬಾಂಧವರನ್ನು ಮೇಲೆತ್ತುವ ದಾರಿ ಕಂಡುಬರಲಿಲ್ಲ. ಆದ್ದರಿಂದಲೇ ನಾನು ಅಮೆರಿಕೆಗೆ ಹೋದದ್ದು. ನನ್ನ ನಿಕಟವರ್ತಿಗಳಿಗೆಲ್ಲ ಇದು ತಿಳಿದಿತ್ತು. ಈ ಸರ್ವಧರ್ಮ ಸಮ್ಮೇಳನವೆಲ್ಲ ಯಾರಿಗೆ ಬೇಕಾಗಿತ್ತು! ಇತ್ತ ಭಾರತದಲ್ಲಿ ನನ್ನ ಪ್ರಾಣಸಮಾನರಾದ ದೇಶ ಬಾಂಧವರು ದಿನೇದಿನೇ ಮುಳುಗಿಹೋಗುತ್ತಿದ್ದಾಗ ಅವರನ್ನು ಕೇಳುವವರು ಯಾರಿದ್ದರು? ಆದ್ದರಿಂದ, ಅದು ನನ್ನ ಮೊದಲ ಹೆಜ್ಜೆಯಾಗಿತ್ತು.

“ಇನ್ನು ಎರಡನೆಯ ಅಂಶ. ನೀವು ಹೃದಯವಂತರಾಗಿರಬಹುದು. ಆದರೆ ಕೆಲಸಕ್ಕೆ ಬಾರದ ಹರಟೆಯಲ್ಲೇ ಶಕ್ತಿಹರಣ ಮಾಡುವುದಕ್ಕಿಂತ, ಏನಾದರೂ ಉಪಯುಕ್ತವಾದ, ಅನುಷ್ಠಾನ ಯೋಗ್ಯವಾದ ಮಾರ್ಗವನ್ನು ಹುಡುಕಿರುವಿರಾ? ಮರಣ ಸಮ ಸ್ಥಿತಿಯಿಂದ ನಿಮ್ಮ ದೇಶ ಬಾಂಧವರನ್ನು ಮೇಲೆತ್ತಲು, ಅವರನ್ನು ಸುಮ್ಮನೆ ನಿಂದಿಸುವ ಬದಲಾಗಿ ಏನಾದರೂ ನೆರವು ನೀಡಬಲ್ಲಿರಾ? ಅವರ ದುಃಖವನ್ನು ದೂರಮಾಡಲು ಒಂದೆರಡು ಒಳ್ಳೆಯ ಮಾತುಗಳನ್ನಾದರೂ ಆಡಬಲ್ಲಿರಾ?

“ಆದರೆ ಇದು ಕೂಡ ಸಾಲದು. ಪರ್ವತೋಪಮವಾದ ಅಡ್ಡಿ ಆತಂಕಗಳನ್ನು ಅತಿಕ್ರಮಿಸಬಲ್ಲ ಇಚ್ಛಾಶಕ್ತಿ ನಿಮ್ಮಲ್ಲಿದೆಯೇನು? ಸಮಸ್ತ ಜಗತ್ತೇ ಕತ್ತಿ ಹಿರಿದು ನಿಮಗೆದುರಾಗಿ ನಿಂತರೂ, ನಿಮಗೆ ಸರಿಯೆಂದು ತೋರಿದ್ದನ್ನು ಬಿಡದೆ ಸಾಧಿಸುವ ಛಲ ನಿಮ್ಮಲಿದೆಯೇನು? ನಿಮ್ಮ ಮಡದಿ ಮಕ್ಕಳೇ ನಿಮ್ಮನ್ನು ವಿರೋಧಿಸಿದರೂ, ನಿಮ್ಮ ಹೆಸರು ಕೀರ್ತಿ ಅಳಿದುಹೋದರೂ, ನಿಮ್ಮ ಸಂಪತ್ತು ಸೂರೆಯಾಗಿ ಹೋದರೂ, ಹಿಡಿದ ಕೆಲಸವನ್ನು ಬಿಡದೆ ಮಾಡುವ ಛಲ ನಿಮ್ಮಲ್ಲಿದೆ ಯೇನು? ಮಹಾರಾಜಾ ಭರ್ತೃಹರಿ ಹೇಳುವಂತೆ,

\begin{verse}
ನಿಂದಂತು ನೀತಿನಿಪುಣಾಃ ಯದಿ ವಾ ಸ್ತುವಂತು\\ಲಕ್ಷ್ಮೀಃ ಸಮಾವಿಶತು ಗಚ್ಛತು ವಾ ಯಥೇಷ್ಟಂ\\ಅದ್ಯೈವ ವಾ ಮರಣಮಸ್ತು ಯುಗಾಂತರೇ ವಾ\\ನ್ಯಾಯಾತ್ ಪಥಂ ಪ್ರವಿಚಲಂತಿ ಪದಂ ನ ಧೀರಾಃ ॥
\end{verse}

‘ನೀತಿನಿಪುಣರು ಹೊಗಳಲಿ ಅಥವಾ ತೆಗಳಲಿ, ಭಾಗ್ಯಲಕ್ಷ್ಮಿಯು ಬಂದು ಸೇರಲಿ ಅಥವಾ ಬಿಟ್ಟುಹೋಗಲಿ, ಮೃತ್ಯುವು ಇಂದೇ ಬರಲಿ ಅಥವಾ ಯುಗಾಂತರಗಳ ಮೇಲೆ ಬರಲಿ, ಧೀರರಾದ ವರು ಮಾತ್ರ ನ್ಯಾಯದ ಪಥದಿಂದ ಒಂದಿನಿತೂ ವಿಚಲಿತರಾಗುವುದಿಲ್ಲ’. ಈ ಬಗೆಯ ದೃಢತೆ ನಿಮ್ಮಲ್ಲಿದೆಯೇನು? ಇಂತಹ ನಿಷ್ಠೆ ನಿಮ್ಮಲ್ಲಿದೆಯೇನು?

“ಈ ಮೂರು ಗುಣಗಳು–ಹೃದಯದಲ್ಲಿ ಅಪರಿಮಿತ ಅನುಕಂಪೆ, ಭಾವನೆಯನ್ನು ಕಾರ್ಯ ಗತಗೊಳಿಸುವ ದಕ್ಷತೆ ಹಾಗೂ ಹಿಡಿದ ಕಾರ್ಯವನ್ನು ಬಿಡದೆ ಸಾಧಿಸಬಲ್ಲ ಎದೆಗಾರಿಕೆ–ಇವು ನಿಮ್ಮಲ್ಲಿದ್ದರೆ ನೀವು ಪ್ರತಿಯೊಬ್ಬರೂ ಒಂದೊಂದು ಅದ್ಭುತ ಪವಾಡವನ್ನೇ ಸಾಧಿಸಬಲ್ಲಿರಿ. ನೀವು ವೃತ್ತಪತ್ರಿಕೆಗಳಿಗೆ ಬರೆದು ನಿಮ್ಮ ಅಭಿಪ್ರಾಯಗಳನ್ನು ಪ್ರಕಟಿಸಿ ಪ್ರತಿಪಾದಿಸಬೇಕಾಗಿಲ್ಲ; ನೀವು ಭಾಷಣ ಕೊಡುತ್ತ ಸುತ್ತಾಡಬೇಕಿಲ್ಲ. ನಿಮ್ಮಲ್ಲಿ ಈ ಮೂರು ಗುಣಗಳಿದ್ದರೆ, ನಿಮ್ಮ ಮುಖದ ತೇಜಸ್ಸೇ ಅದನ್ನು ಪ್ರತಿಬಿಂಬಿಸುತ್ತದೆ. ನೀವೊಂದು ಗುಹೆಯಲ್ಲಿ ವಾಸಿಸುತ್ತಿದ್ದರೂ, ನಿಮ್ಮ ಆಲೋಚನೆ-ಭಾವನೆಗಳು ಶಿಲೆಯ ಗೋಡೆಗಳನ್ನೂ ತೂರಿಕೊಂಡು ಹೋಗಿ ಜಗತ್ತಿನಾ ದ್ಯಂತ ತರಂಗಗಳಾಗಿ ಹರಡಿ ನೂರಾರು ವರ್ಷಗಳವರೆಗೆ ಸ್ಪಂಧಿಸುತ್ತಿರುತ್ತವೆ. ಅವು ಯಾರೋ ಕೆಲವರ ಮೆದುಳನ್ನು ಹೊಕ್ಕು ಕಾರ್ಯಾರಂಭ ಮಾಡುವವರೆಗೂ ಹಾಗೆಯೇ ಸ್ಪಂದಿಸುತ್ತಿರ ಬಲ್ಲವು. ಶುದ್ಧ ಭಾವನೆಗಳ ಶಕ್ತಿ ಅಂಥದು; ಪ್ರಾಮಾಣಿಕತೆಯ ಶಕ್ತಿ ಅಂಥದು; ಶುದ್ಧ ಉದ್ದೇಶ ಗಳ ಶಕ್ತಿ ಅಂಥದು.”

ಕಡೆಯದಾಗಿ ಸ್ವಾಮೀಜಿ, ಮದ್ರಾಸಿನ ಜನರಲ್ಲಿ–ಅದರಲ್ಲೂ ಮುಖ್ಯವಾಗಿ ಯುವಜನತೆ ಯಲ್ಲಿ–ಅತ್ಯಂತ ಕಳಕಳಿಯಿಂದ ಒಂದು ಮನವಿ ಮಾಡಿಕೊಂಡರು; ತಮ್ಮ ಹೃದಯದ ಭಾವನೆ ಯನ್ನು ತೆರೆದಿಟ್ಟರು:

“ನನ್ನ ದೇಶಬಾಂಧವರೇ, ನನ್ನ ಸ್ನೇಹಿತರೇ, ನನ್ನ ಪುತ್ರರೇ, ಈ ನಮ್ಮ ಸನಾತನ ಧರ್ಮವೆಂಬ ಹಡಗು ಆದಿಕಾಲದಿಂದಲೂ ಕೋಟಿಗಟ್ಟಲೆ ಜೀವಾತ್ಮರನ್ನು ಬಾಳಕಡಲಿನಾಚೆಗೆ, ಅಮೃತತ್ವದ ದಡಕ್ಕೆ ತಲುಪಿಸಿದೆ. ಆದರೆ ಇಂದು, ಬಹುಶಃ ನಮ್ಮ ತಪ್ಪಿನಿಂದಲೇ, ಈ ಹಡಗು ಸ್ವಲ್ಪ ಹಾಳಾ ಗಿದೆ; ತೂತು ಬಿದ್ದಿದೆ. ಹಾಗೆಂದ ಮಾತ್ರಕ್ಕೆ ಅದನ್ನೀಗ ನಿಂದಿಸುವಿರಾ? ಶಪಿಸುವಿರಾ? ಪ್ರಪಂಚದ ಯಾವುದೇ ಇತರ ವಸ್ತುವಿಗಿಂತ ಹೆಚ್ಚಿನ ಮಹತ್ಕಾರ್ಯವನ್ನು ಸಾಧಿಸಿದ ಈ ಪವಿತ್ರ ತಮ ವಸ್ತುವಿನ ಮೇಲೆ ನಿಂದನೆಯ ಬಾಣಗಳನ್ನು ಕರೆಯುವುದು ತರವೆ? ಈ ನಮ್ಮ ಸಮಾಜ ವೆಂಬ, ಸನಾತನ ಧರ್ಮವೆಂಬ ಹಡಗು ತೂತು ಬಿದ್ದು ಕೆಟ್ಟಿದ್ದರೂ ನಾವು ಅದರ ಸಂತಾನ ರಲ್ಲವೆ? ನಾವು ಆ ರಂಧ್ರಗಳನ್ನು ಮುಚ್ಚೋಣ. ನಮ್ಮ ಹೃದಯದ ನೆತ್ತರನ್ನು ಬಸಿದಾದರೂ ಸಂತೋಷದಿಂದ ಅದನ್ನು ಸರಿಪಡಿಸೋಣ. ಒಂದು ವೇಳೆ ಸಾಧ್ಯವಾಗದಿದ್ದರೆ, ಅದರೊಂದಿಗೆ ನಾವೂ ಮುಳುಗಿ ಆತ್ಮಾರ್ಪಣೆ ಮಾಡೋಣ. ಆದರೆ ಎಂದೆಂದಿಗೂ ಅದನ್ನು ನಿಂದಿಸದಿರೋಣ. ಅದರ ಗತಕಾಲದ ಹಿರಿಮೆಗಾಗಿ ನಾನದನ್ನು ಹೃತ್ಪೂರ್ವಕವಾಗಿ ಪ್ರೀತಿಸುತ್ತೇನೆ. ನೀವೆಲ್ಲ ಭಗವಂತನ ಮಕ್ಕಳೆಂದು, ಆ ಮಹಾಮಹಿಮರಾದ ನಮ್ಮ ಪೂರ್ವಜರ ಸಂತತಿಗೆ ಸೇರಿದವ ರೆಂದು ನಾನು ನಿಮ್ಮನ್ನು ಪ್ರೀತಿಸುತ್ತೇನೆ. ನನ್ನ ಪ್ರಿಯ ಪುತ್ರರೆ, ನಿಮಗೆ ನನ್ನೆಲ್ಲ ಯೋಜನೆಗಳನ್ನು ಹೇಳಲು ನಾನಿಲ್ಲಿಗೆ ಬಂದಿದ್ದೇನೆ. ನೀವು ಅವುಗಳಿಗೆ ಕಿವಿಗೊಡುವುದಾದರೆ ನಾನು ನಿಮ್ಮೊಡನಿದ್ದು ಕೆಲಸ ಮಾಡಲು ಸಿದ್ಧನಿದ್ದೇನೆ. ಆದರೆ ನೀವು ಅದಕ್ಕೆ ಕಿವಿಗೊಡದಿದ್ದರೂ, ಅಥವಾ ನನ್ನನ್ನು ಭಾರತದಿಂದಲೇ ಹೊರದೂಡಿದರೂ ಕೂಡ ಮತ್ತೆ ನಾನು ಹಿಂದಿರುಗಿ ಬಂದು ಹೇಳುತ್ತೇನೆ– ‘ಎಚ್ಚರ! ನಾವು ಮುಳುಗಿಹೋಗುತ್ತಿದ್ದೇವೆ!’ ಎಂದು. ನಾನು ನಿಮ್ಮೊಡನಿರಲೆಂದೇ ಬಂದಿ ದ್ದೇನೆ. ನಾವು ಮುಳುಗುವುದಾದರೆ ಒಟ್ಟಾಗಿ ಮುಳುಗಿ ಜಲಸಮಾಧಿಯಾಗೋಣ. ಆದರೆ ಶಾಪ ಗಳು ಮಾತ್ರ ನಮ್ಮ ನಾಲಿಗೆಯಲ್ಲಿ ಬಾರದಿರಲಿ!”

ಇದು ಸ್ವಾಮೀಜಿ ಮದ್ರಾಸಿನಲ್ಲಿ ಮಾಡಿದ “ನನ್ನ ಸಮರನೀತಿ” ಎಂಬ ಪ್ರಥಮ ಸಾರ್ವಜನಿಕ ಭಾಷಣ. ಅವರು ಈ ಸುದೀರ್ಘ-ಭಾವಪೂರ್ಣ-ರೋಮಾಂಚಕ ಭಾಷಣವನ್ನು ಮುಗಿಸುತ್ತಿ ದ್ದಂತೆ, ನೆರೆದಿದ್ದ ಸಭಿಕರು ಕರತಾಡನ ಮಾಡಿ ಹರ್ಷೋದ್ಗಾರಗೈದರು. ಸಭೆ ವಿಸರ್ಜನೆಯಾದಾಗ ಕತ್ತಲಾಗಿತ್ತು. ಆದರೂ ಜನಸಮುದಾಯವು ಸ್ವಾಮೀಜಿಯವರ ದರ್ಶನಕ್ಕಾಗಿ ಸಭಾಂಗಣದಾಚೆ ಕಾದು ನಿಂತಿತ್ತು. ಸಭಾಧ್ಯಕ್ಷರಾಗಿದ್ದ ಸರ್ ಭಾಷ್ಯಂ ಅಯ್ಯಂಗಾರರ ಸಹಾಯದಿಂದ ಸ್ವಾಮೀಜಿ ಸಭಾಂಗಣದಿಂದ ಹೊರಬಂದಾಗ ಅವರ ದರ್ಶನಕ್ಕಾಗಿ ಜನ ಮುನ್ನುಗ್ಗಿ ಬಂದರು. ಬಳಿಕ ಅಲ್ಲಿದ್ದ ಗಾಡಿಯಲ್ಲಿ ಹೊರಟು ತಮ್ಮ ನಿವಾಸವಾದ ಕ್ಯಾಸಲ್ ಕರ್ನನ್ನಿಗೆ ಬಂದು ತಲುಪುವವ ರೆಗೂ ದಾರಿಯುದ್ದಕ್ಕೂ ಜನ ಜಯಘೋಷ ಮಾಡುತ್ತಿದ್ದರು.

ಸ್ವಾಮೀಜಿ ಮದ್ರಾಸಿಗೆ ಬಂದಂದಿನಿಂದ ಇಡೀ ನಗರದಲ್ಲೇ ನವಚೇತನವೊಂದು ತುಂಬಿ ಹರಿದಂತಿತ್ತು. ಹುಡುಗರು, ಯುವಕರು, ವೈದ್ಯರು, ವಕೀಲರು, ವರ್ತಕರು, ನೌಕರರು, ಹೆಂಗಸರು, ಗಂಡಸರು, ಮುದುಕರು, ಸಾಮಾನ್ಯರು, ಪ್ರತಿಷ್ಠಿತರು ಎಲ್ಲರೂ ವಿವೇಕಾನಂದರ ವಿಚಾರವನ್ನೇ ಮಾತನಾಡುತ್ತಿದ್ದರು. ಮಾರ್ಕೆಟ್ಟಿನಲ್ಲಿ, ಟ್ರಾಮುಗಳಲ್ಲಿ, ಬೀಚ್​ನಲ್ಲಿ, ಕಛೇರಿಗಳಲ್ಲಿ, ಕೋರ್ಟು ಗಳಲ್ಲಿ, ಮನೆಗಳಲ್ಲಿ, ಬೀದಿಗಳಲ್ಲಿ ಎಲ್ಲಿ ನೋಡಿದರೂ ಸ್ವಾಮೀಜಿಯವರ ಮಾತೇ, ಅವರ ವ್ಯಕ್ತಿತ್ವದ ಗುಣಗಾನವೇ! ಸ್ವಾಮೀಜಿ ರೈಲುನಿಲ್ದಾಣದಲ್ಲಿ ಬಂದಿಳಿದಾಗಿನಿಂದ ಅವರು ಮದ್ರಾಸಿ ನಲ್ಲಿ ಆಡಿದ ಮಾತು, ಮಾಡಿದ ಕೆಲಸ, ನೋಡಿದ ಸಂಗತಿಗಳನ್ನು ಕುರಿತು ಜನ ನಿಂತಲ್ಲಿ ಕುಳಿತಲ್ಲಿ ಮಾತನಾಡಿಕೊಳ್ಳುತ್ತಿದ್ದರು. ಸ್ವಾಮೀಜಿಯವರನ್ನು ಅತಿ ಹತ್ತಿರದಿಂದ ನೋಡಿದೆವೆಂಬ ಸಂತೋಷ ಕೆಲವರಿಗಾದರೆ, ಅವರ ಪಾದಸ್ಪರ್ಶ ಮಾಡಿ ಕೃತಾರ್ಥರಾದೆವೆಂಬ ಹರ್ಷ ಕೆಲವರಿಗೆ. ಇನ್ನು ಕೆಲವರದು ಅವರೊಡನೆ ಮಾತನಾಡಿದ ಸುದೈವ.

ಫೆಬ್ರವರಿ ೧೧ರಂದು ಸ್ವಾಮೀಜಿ ವಿಕ್ಟೋರಿಯಾ ಹಾಲ್​ನಲ್ಲಿ “ಭಾರತದ ಮಹಾಪುರುಷರು”\\ಎಂಬ ವಿಷಯವಾಗಿ ಮಾತನಾಡಲಿದ್ದರು. ನಿಗದಿತ ವೇಳೆಗೆ ಒಂದೆರಡು ಗಂಟೆ ಮೊದಲೇ ಜನರು ಬಂದು ಸೇರತೊಡಗಿದ್ದರು. ಸ್ವಾಗತದೊಂದಿಗೆ ಕಾರ್ಯಕ್ರಮ ಪ್ರಾರಂಭವಾಯಿತು. ತಮ್ಮ ಭಾಷಣವನ್ನು ಸುದೀರ್ಘ ಪೀಠಿಕೆಯೊಂದಿಗೆ ಆರಂಭಿಸಿದ ಸ್ವಾಮೀಜಿ, ಮೊದಲಿಗೆ ಹಿಂದೂ ಧರ್ಮದ ಶಾಸ್ತ್ರಗ್ರಂಥಗಳಲ್ಲಿ ಶ್ರುತಿಸ್ಮೃತಿಗಳೆಂಬ ಎರಡು ಪ್ರಭೇದಗಳಿರುವುದರತ್ತ ಜನರ ಗಮನಸೆಳೆದರು. ಶ್ರುತಿಗಳೆಂದರೆ ಜೀವಾತ್ಮ-ಪರಮಾತ್ಮರ ನಡುವಣ ಸಂಬಂಧವೇ ಮೊದಲಾದ ಸಾರ್ವಕಾಲಿಕ ಸತ್ಯಗಳನ್ನೊಳಗೊಂಡ ಗ್ರಂಥಗಳು, ವೇದಗಳು. ಸ್ಮೃತಿಗಳೆಂದರೆ ಪುರಾಣಗಳು, ತಂತ್ರಗಳು, ಮನು-ಯಾಜ್ಞವಲ್ಕ್ಯರೇ ಮೊದಲಾದ ಮಹರ್ಷಿಗಳು ರಚಿಸಿದ ಸಂವಿಧಾನಗಳು, ಮೊದಲಾದುವುಗಳು. ಇವು ಶ್ರುತಿಗಳಿಗೆ ಅಧೀನವಾದಂಥವು. ಆದ್ದರಿಂದ ಸ್ಮೃತಿಯ ಅಭಿ ಪ್ರಾಯವು ಶ್ರುತಿಯ ಅಭಿಪ್ರಾಯಕ್ಕೆ ವ್ಯತಿರಿಕ್ತವಾದಾಗಲೆಲ್ಲ ಶ್ರುತಿ ವಾಕ್ಯವೇ ಪ್ರಮಾಣವೆಂದು ಸ್ವೀಕರಿಸತಕ್ಕದ್ದು. ಈ ವಿಚಾರಗಳನ್ನು ಸ್ಪಷ್ಟಪಡಿಸಿದ ಸ್ವಾಮೀಜಿ, ಪುರಾತನ ವೇದಮಹರ್ಷಿಗಳ ಬಗ್ಗೆ ಪ್ರಸ್ತಾಪಿಸಿದರು. ‘ಆ ಪುಷಿಗಳ ವ್ಯಕ್ತಿತ್ವ-ಜೀವನಗಳ ಬಗ್ಗೆ ನಮಗೆ ಹೆಚ್ಚೇನೂ ತಿಳಿದಿರ ದಿದ್ದರೂ ಅವರ ಅತ್ಯದ್ಭುತ ಸಂಶೋಧನೆಗಳೆಲ್ಲ ವೇದಗಳಲ್ಲಿ ಸಂರಕ್ಷಿತವಾಗಿವೆ; ಮತ್ತು ಅವು ನಮಗಿಂದು ಲಭ್ಯವಾಗಿವೆ. ನಮ್ಮ ಧರ್ಮವು ನಿರಾಕಾರ ಹಾಗೂ ಸಾಕಾರ ತತ್ತ್ವಗಳೆರಡನ್ನೂ ಬೋಧಿಸಿದರೂ, ಶುದ್ಧ ನಿರಾಕಾರ ತತ್ತ್ವವೇ ಹಿಂದೂಧರ್ಮದ ಮೂಲತತ್ತ್ವ, ಮಹಾಅವತಾರ ಪುರುಷರೂ ದೇವತಾಸ್ವರೂಪಿಗಳೂ ನಮ್ಮ ಸ್ಮೃತಿ-ಪುರಾಣಗಳಲ್ಲಿ ಬರುವರಾದರೂ ಅವರಾ ರನ್ನೂ ನಮ್ಮ ಧರ್ಮದ ಮೂಲವೆಂದು ಹೇಳಲಾಗುವುದಿಲ್ಲ. ಏಕೆಂದರೆ ಜಗತ್ತಿನ ಇತರ ಧರ್ಮ ಗಳಂತೆ ಹಿಂದೂಧರ್ಮವು ವ್ಯಕ್ತಿಗಳನ್ನು ಅವಲಂಬಿಸಿ ನಿಂತಿಲ್ಲ, ತತ್ತ್ವಗಳನ್ನು ಮಾತ್ರ ಅವಲಂಬಿ ಸಿದೆ. ಆದರೆ ಮಾನವಕೋಟಿಯ ಬಹುಸಂಖ್ಯಾತ ಜನರು ನಿರಾಕಾರಬ್ರಹ್ಮದ ಕಲ್ಪನೆಯನ್ನು ಮಾಡಿ ಕೊಳ್ಳಲು ಅಸಮರ್ಥರು. ಅವರಿಗೆ ಒಂದು ಆಕಾರ ಬೇಕು; ಅವರಿಗೆ ಪೂಜಿಸಲು ಒಬ್ಬ ವ್ಯಕ್ತಿ ಬೇಕು. ಈ ಅಂಶವನ್ನು ನಮ್ಮ ಮಹರ್ಷಿಗಳು ಅರಿತಿದ್ದರು. ಆದ್ದರಿಂದ ಅವರು ಮಹಾಪುರುಷ ರನ್ನು, ವಿಭೂತಿಪುರುಷರನ್ನು, ಅವತಾರಗಳನ್ನು ಪೂಜಿಸುವ ಸ್ವಾತಂತ್ರ್ಯವನ್ನು ಎಲ್ಲರಿಗೂ ನೀಡಿದರು.’

ಜಗತ್ತನ್ನೇ ಅಲುಗಾಡಿಸಿದ ಮಹಾಪುರುಷರ ಬಗ್ಗೆ ಪ್ರಸ್ತಾಪಿಸಿದ ಸ್ವಾಮೀಜಿ ಶ್ರೀರಾಮನ ವಿಚಾರವಾಗಿ ಹೀಗೆ ಹೇಳುತ್ತಾರೆ: “ಶ್ರೀರಾಮಚಂದ್ರನು ಸತ್ಯ, ನೈತಿಕತೆಗಳ ಮೂರ್ತರೂಪ. ಅವನು ಆದರ್ಶ ಪುತ್ರ, ಆದರ್ಶ ಪತಿ, ಆದರ್ಶ ಪಿತ. ಎಲ್ಲಕ್ಕಿಂತ ಹೆಚ್ಚಾಗಿ ಆದರ್ಶ ರಾಜ.” ಆದರೆ ಸೀತಾದೇವಿಯ ಬಗ್ಗೆ ಹೇಳುತ್ತಾರೆ: “ನೂರಾರು ಜನ ರಾಮರು ಆಗಿಹೋಗಿರಬಹುದು;\\ಆದರೆ ಸೀತೆ ಮಾತ್ರ ಏಕಮೇವಾದ್ವಿತೀಯಳು! ಆರ್ಯಾವರ್ತದ ಉದ್ದಗಲಕ್ಕೂ ಸಕಲ ಸ್ತ್ರೀ ಪುರುಷರ ಪೂಜೆಗೆ ಅರ್ಹಳಾಗಿ ಸಾವಿರಾರು ವರ್ಷಗಳಿಂದ ನಿಂತಿರುವ ಆದರ್ಶನಾರಿ ಸೀತಾದೇವಿ. ಈ ಸೀತೆ ನಮ್ಮ ಜನಾಂಗದ ರಕ್ತದಲ್ಲಿ ಸೇರಿ ಹೋಗಿದ್ದಾಳೆ.”

ಹೀಗೆ ಸ್ವಾಮೀಜಿ ಭಾರತದ ಮಹಾನ್ ವ್ಯಕ್ತಿಗಳ ಕುರಿತಾಗಿ ಹೇಳುತ್ತ, ಶ್ರೀಕೃಷ್ಣ, ಬುದ್ಧ, ಶಂಕರ, ರಾಮಾನುಜ, ಶ್ರೀಚೈತನ್ಯರ ಬಗ್ಗೆ ಸಂಕ್ಷಿಪ್ತವಾಗಿಯಾದರೂ ಅದ್ಭುತವಾದ ವಿವರಣೆ ನೀಡಿದರು. ಕಡೆಯಲ್ಲಿ ಆಧುನಿಕ ಯುಗದ ಬಗ್ಗೆ ಮಾತನಾಡುತ್ತ ಶ್ರೀರಾಮಕೃಷ್ಣರ ಸಂಬಂಧವಾಗಿ ಆವೇಶಭರಿತ ದನಿಯಲ್ಲಿ ಹೀಗೆಂದರು:

“ಶ್ರೀಶಂಕರಾಚಾರ್ಯರಿಗೆ ಪ್ರಚಂಡ ಬುದ್ಧಿಮತ್ತೆಯಿತ್ತು; ಶ್ರೀಚೈತನ್ಯರಿಗೆ ಅದ್ಭುತ ಹೃದಯ ವೈಶಾಲ್ಯವಿತ್ತು. ಅವರ ಬುದ್ಧಿ, ಇವರ ಹೃದಯ–ಇವೆರಡೂ ಸಮ್ಮಿಳಿತಗೊಂಡ ವ್ಯಕ್ತಿ ಯೊಬ್ಬನು ಹುಟ್ಟಿ ಬರಲು ಕಾಲ ಕೂಡಿಬಂದಿತ್ತು. ಇದೀಗ ಎಲ್ಲ ಮತ-ಪಂಗಡಗಳಲ್ಲೂ ಒಂದೇ ದೈವವನ್ನು ಒಂದೇ ಚೈತನ್ಯವನ್ನು ಕಾಣುವ, ಪ್ರತಿಯೊಂದು ಜೀವಿಯಲ್ಲೂ ದೇವರನ್ನೇ ಕಾಣುವ ಒಬ್ಬ ಮಹಾತ್ಮ ಅವತರಿಸಿದ. ದರಿದ್ರರಿಗಾಗಿ ದುಃಖಿಗಳಿಗಾಗಿ ಪತಿತರಿಗಾಗಿ ಕಣ್ಣೀರು ಸುರಿಸುವ ಮಹಾತ್ಮನೊಬ್ಬ ಉದಿಸಿದ. ಇಲ್ಲಿಯವರೆಗೆ ಪರಸ್ಪರ ಕಾದಾಡುತ್ತಿದ್ದ ಈ ದೇಶದ ಹಾಗೂ ವಿದೇಶದ ವಿವಿಧ ಮತಪಂಗಡಗಳನ್ನು ಒಂದುಗೂಡಿಸುವ, ಉಜ್ವಲ ಜ್ಞಾನ-ಬುದ್ಧಿ-ಹೃದಯಗಳು ಒಂದುಗೂಡಿದ, ವಿಶ್ವಧರ್ಮವೊಂದನ್ನು ಸ್ಥಾಪಿಸುವಂತಹ ಓರ್ವ ಮಹಾಮಹಿಮ ಉದಿಸಿ ಬಂದ. ಆತನ ಪದತಲದಲ್ಲಿ ಅನೇಕ ವರ್ಷ ಕುಳಿತು ಕಲಿಯುವ ಸೌಭಾಗ್ಯ ನನ್ನ ಪಾಲಿನದಾಗಿತ್ತು. ಪುಸ್ತಕವಿದ್ಯೆ-ಅಕ್ಷರವಿದ್ಯೆಗಳೇನೇನೂ ಬಾರದ ಈ ಮಹಾಮೇಧಾವಿಯು ಸರಿಯಾಗಿ ತನ್ನ ಹೆಸರನ್ನು ಬರೆಯಲೂ ಕಲಿತಿರಲಿಲ್ಲ. ಆದರೂ ನಮ್ಮ ವಿಶ್ವವಿದ್ಯಾಲಯಗಳ ಅತ್ಯುನ್ನತ ಶ್ರೇಣಿಯ ಪದವೀ ಧರರೂ ಕೂಡ ಅವರೊಬ್ಬ ಅತ್ಯದ್ಭುತ ಮೇಧಾವಿಯೆಂಬುದನ್ನು ಗುರುತಿಸಿದರು. ಈ ವಿಶಿಷ್ಟ ವ್ಯಕ್ತಿಯೇ ಶ್ರೀರಾಮಕೃಷ್ಣ ಪರಮಹಂಸರು. ಅದೊಂದು ದೀರ್ಘ, ಬಹುದೀರ್ಘ ಕಥೆ. ಇಂದು ಅದನ್ನು ಬಣ್ಣಿಸಲು ಸಮಯವಿಲ್ಲ. ಆದರೆ ಯಾರು ಭಾರತೀಯ ಪುಷಿಗಳ ಪರಿಪೂರ್ಣ ರೂಪ ರಾಗಿರುವರೋ ಮತ್ತು ಯಾರ ಉಪದೇಶಾಮೃತವು ಈಗಿನ ಕಾಲಕ್ಕೆ ಅತ್ಯಂತ ಸೂಕ್ತವೂ ಉಪ ಯುಕ್ತವೂ ಆಗಿದೆಯೋ ಅಂತಹ ಪರಮಪಾವನರಾದ ಆ ಶ್ರೀರಾಮಕೃಷ್ಣ ಪರಮಹಂಸರ ನಾಮಸ್ಮರಣೆಯಿಂದಲೇ ತೃಪ್ತನಾಗುತ್ತೇನೆ. ಆ ವ್ಯಕ್ತಿಯ ಹಿಂದಿದ್ದ ದೈವೀ ಶಕ್ತಿಯನ್ನು ನೀವು ಗಮನಿಸಬೇಕು. ಅವರೊಬ್ಬ ಬಡ ಅರ್ಚಕನ ಮಗ. ಎಲ್ಲಿಯೋ ಒಂದು ಕುಗ್ರಾಮದಲ್ಲಿ ಹುಟ್ಟಿ ದರು. ಯಾರ ಕಣ್ಣಿಗೂ ಬೀಳದೆ, ಯಾರ ಗಮನಕ್ಕೂ ಬಾರದೆ ಅವರು ಬೆಳೆದರು. ಅಂಥವರನ್ನು ಇಂದು ಯೂರೋಪ್​-ಅಮೆರಿಕಗಳಲ್ಲಿ ಸಹಸ್ರಾರು ಜನ ಅಕ್ಷರಶಃ ಪೂಜಿಸುತ್ತಿದ್ದಾರೆ. ನಾಳೆ ಇನ್ನೂ ಹೆಚ್ಚು ಜನರಿಂದ ಪೂಜಿಸಲ್ಪಡಲಿದ್ದಾರೆ. ಭಗವಂತನ ಲೀಲೆಯನ್ನು ಬಲ್ಲವರಾರು!”

“ಸೋದರರೇ, ದೈವದ ಆ ಕೈಯನ್ನು ಇಂದು ನೀವು ಕಾಣಲಾರಿರಾದರೆ ಅದಕ್ಕೆ ಕಾರಣ, ನೀವು ಕುರುಡಾಗಿದ್ದೀರಿ; ಹೌದು, ಹುಟ್ಟುಕುರುಡರೆ ಆಗಿದ್ದೀರಿ! ಯಾವಾಗಲಾದರೂ ಕಾಲ ಕೂಡಿ ಬಂದರೆ ಅವರ ವಿಚಾರವಾಗಿ ವಿಶದವಾಗಿ ಹೇಳುತ್ತೇನೆ. ಆದರೆ ಒಂದು ಮಾತನ್ನು ಮಾತ್ರ ಈಗ ಹೇಳಬಯಸುತ್ತೇನೆ. ಇಂದು ನಾನು ಒಂದೇ ಒಂದಾದರೂ ಸತ್ಯವಾಕ್ಯವನ್ನು ನುಡಿದಿದ್ದರೆ ಅದಕ್ಕೆ ಅವರೇ, ಅವರೊಬ್ಬರೇ ಕಾರಣ. ಮತ್ತು ನಿಜವಲ್ಲದ ಹಾಗೂ ಮಾನವಕುಲಕ್ಕೆ ಉಪಯುಕ್ತವಲ್ಲದ ಮಾತುಗಳನ್ನಾಡಿದ್ದರೆ, ಅವೆಲ್ಲವೂ ನನ್ನವೇ. ಅದರ ಸಂಪೂರ್ಣ ಜವಾಬ್ದಾರಿ ನನ್ನದೇ.”

ಇದು “ಭಾರತದ ಮಹಾಪುರುಷರು” ಎಂಬ ವಿಷಯದ ಕುರಿತಾಗಿ ಸ್ವಾಮೀಜಿ ವಿಕ್ಟೋರಿಯಾ ಹಾಲ್​ನಲ್ಲಿ ಮಾಡಿದ ಭಾಷಣ.

ಮದ್ರಾಸಿನಲ್ಲಿ ಸ್ವಾಮೀಜಿಯವರ ಕಾರ್ಯಕ್ರಮಗಳ ಪಟ್ಟಿ ಅತ್ಯಂತ ಕಟ್ಟುನಿಟ್ಟಾಗಿತ್ತು. ಮುಂಜಾನೆಯಿಂದ ರಾತ್ರಿಯವರೆಗೂ ಬಿಡುವಿಲ್ಲದಂತೆ ಒಂದಾದ ಮೇಲೊಂದು ಕಾರ್ಯಕ್ರಮ. ಅವರ ದರ್ಶನಕ್ಕಾಗಿ ಬರುತ್ತಿದ್ದವರಿಗಂತೂ ಲೆಕ್ಕವೇ ಇಲ್ಲ. ಭಕ್ತರು, ಜಿಜ್ಞಾಸುಗಳು, ಪಂಡಿತರು, ಸಾಮಾನ್ಯರು–ಎಲ್ಲರೂ ತಂಡೋಪತಂಡವಾಗಿ ಬಂದು ಪ್ರಣಾಮ ಸಲ್ಲಿಸಿ, ಮಾತನಾಡಿಕೊಂಡು ಹೋಗುತ್ತಿದ್ದರು. ಇಷ್ಟು ಚಟುವಟಿಕೆಯಲ್ಲಿಯೂ ಸ್ವಾಮೀಜಿಯವರ ಆಹಾರ ಮಾತ್ರ ಅತ್ಯಂತ ಮಿತವಾಗಿತ್ತೆಂದು ತಿಳಿದುಬರುತ್ತದೆ–ಬೆಳಿಗ್ಗೆ ಸ್ವಲ್ಪವೇ ಊಟ, ರಾತ್ರಿ ಒಂದು ಬಟ್ಟಲು ಹಾಲು, ಇಷ್ಟೆ. ಸ್ವಾಮೀಜಿಯ ನಿಕಟವರ್ತಿಗಳಾಗಿದ್ದ ಪ್ರೊ ॥ ಸುಂದರರಾಮ ಅಯ್ಯರರಿಗೆ ಇದನ್ನು ಕಂಡು ಅತ್ಯಾಶ್ಚರ್ಯ. “ಸ್ವಾಮೀಜಿ, ಹೀಗೆ ಬಿಡುವಿಲ್ಲದೆ ಕೆಲಸ ಮಾಡಲು ನಿಮಗೆ ಶಕ್ತಿಯೆಲ್ಲಿಂದ ಬರುತ್ತದೆ?” ಎಂದೊಮ್ಮೆ ಅವರು ಕೇಳಿದರು. ಅದಕ್ಕೆ ಸ್ವಾಮೀಜಿ ಸರಳವಾಗಿ ಉತ್ತರಿಸಿದರು, “ಆಧ್ಯಾತ್ಮಿಕ ಕರ್ಮವು ಭಾರತದಲ್ಲಿ ಯಾರಿಗೂ ಆಯಾಸವನ್ನುಂಟುಮಾಡುವುದಿಲ್ಲ.”

೧೨ನೇ ತಾರೀಕು ಸಾರ್ವಜನಿಕ ಉಪನ್ಯಾಸದ ಕಾರ್ಯಕ್ರಮವಿರಲಿಲ್ಲ. ಅಂದು ಬೆಳಿಗ್ಗೆ ಸ್ವಾಮೀಜಿ ಮದ್ರಾಸಿನ ‘ಹಿಂದೂ ಥಿಯೊಲಾಜಿಕಲ್ ಹೈಸ್ಕೂಲಿ’ಗೆ ಭೇಟಿ ನೀಡಿದರು. ಅಂದು ಸಂಜೆ ಚನ್ನಪುರಿ ಅನ್ನದಾನ ಸಮಾಜದ ವಾರ್ಷಿಕೋತ್ಸವದಲ್ಲಿ ಭಾಗವಹಿಸಿದರು. ಅಲ್ಲಿ ಅವರು, ಇತರ ಸಾಂಪ್ರದಾಯಿಕ ಕಾರ್ಯಕ್ರಮಗಳಾದ ಮೇಲೆ ದಾನದ ಬಗ್ಗೆ ಕೆಲವು ಮಾತುಗಳನ್ನಾಡಿ, ದಾನದ ಮಹತ್ವವನ್ನು ಎತ್ತಿ ಹಿಡಿಯುತ್ತ ಕೆಲವು ಸ್ವಾರಸ್ಯಕರ ಅಭಿಪ್ರಾಯಗಳನ್ನು ವ್ಯಕ್ತಪಡಿಸಿ ದರು. ನಡುವೆ, ತಾವು ಪರಿವಾಜ್ರಕರಾಗಿದ್ದಾಗಿನ ದಿನಗಳನ್ನು ನೆನಪಿಸಿಕೊಂಡು ಒಂದು ಘಟನೆ ಯನ್ನು ತಿಳಿಸಿದರು: ಆಗ ಅವರೊಂದಿಗೆ ಒಬ್ಬ ಮುಸ್ಲಿಂ ಸಂಗಾತಿ ಇದ್ದ. ಇಬ್ಬರೂ ಭಿಕ್ಷೆ ಬೇಡಲು ಹೋದಾಗ ಒಬ್ಬರ ಮನೆಯಲ್ಲಿ ಸ್ವಾಮೀಜಿಯವರನ್ನು ಒಳಗೆ ಕರೆದು ಉಣಬಡಿಸಿದರು; ಆದರೆ ಅವರ ಮುಸ್ಲಿಂ ಸಂಗಾತಿಗೆ ಸ್ವಲ್ಪ ಹಣ ಕೊಟ್ಟು ಆಹಾರವನ್ನು ಎಲ್ಲಾದರೂ ಕೊಂಡು ತಿನ್ನುವಂತೆ ಹೇಳಿಕಳಿಸಿಬಿಟ್ಟರು. ಈ ಬಗೆಯ ಭೇದಭಾವವನ್ನು ಸ್ವಾಮೀಜಿ ಅನುಮೋದಿಸು ವವರಲ್ಲ. ಚನ್ನಪುರಿಯ ಈ ‘ಅನ್ನದಾನ ಸಮಾಜ’ವು ಜಾತಿ ಮತ ಭೇದಗಳನ್ನೆಣಿಸದೆ ಬೇಕೆಂಬವ ರೆಲ್ಲರಿಗೂ ಉಣಬಡಿಸುವ ಸಂಗತಿಯನ್ನು ತಿಳಿದ ಸ್ವಾಮೀಜಿ ಸಂತೋಷಗೊಂಡರು. ಶಾಸ್ತ್ರಗಳ ಪ್ರಕಾರ, ದಾನವನ್ನು ಪಡೆಯುವವನು ಕೊಡುವವನಿಗಿಂತ ಶ್ರೇಷ್ಠಸ್ಥಾನದಲ್ಲಿ ಇರುತ್ತಾನೆ; ಏಕೆಂ ದರೆ ದಾನವನ್ನು ಪಡೆಯುವವನು ಆ ಕ್ಷಣದಲ್ಲಿ ಸಾಕ್ಷಾತ್ ಭಗವಂತನ ಪ್ರತಿನಿಧಿಯಾಗಿರುತ್ತಾನೆ; ಆದರೆ ದಾನ ನೀಡುವವನು ಕೇವಲ ಈ ಭಗವಂತನ ಪೂಜಕನಾಗಿರುತ್ತಾನೆ ಎಂದರು ಸ್ವಾಮೀಜಿ. ಬಳಿಕ ಅಪಾತ್ರದಾನ ಹಾಗೂ ಸತ್ಪಾತ್ರದಾನಗಳ ಬಗ್ಗೆ ಪ್ರಸ್ತಾಪಿಸುತ್ತ ಹೇಳಿದರು, “ಭಾರತದಲ್ಲಿ ಅಪಾತ್ರದಾನವೆಂಬುದು ಬಹಳವಾಗಿ ಕಂಡುಬರುತ್ತದೆಂಬುದರಲ್ಲಿ ಸಂಶಯವಿಲ್ಲ. ಆದರೆ ಈ ಜಗತ್ತಿನಲ್ಲೆಲ್ಲೂ ಅಪಾತ್ರದಾನವನ್ನು ತಪ್ಪಿಸುವಂತಹ ವ್ಯವಸ್ಥೆಯಿಲ್ಲ. ಅಲ್ಲದೆ ಇದರಿಂದ ಕೆಡುಕಿ ಗಿಂತ ಹೆಚ್ಚಾಗಿ ಒಳಿತೇ ಆಗಿದೆಯೆಂದು ನಾನು ಹೇಳುತ್ತೇನೆ. ಈ ‘ನಾಗರಿಕತೆ’ ಎಂಬ ರೋಗವಿರುವವರೆಗೂ ಪ್ರತಿಯೊಂದು ದೇಶದಲ್ಲೂ ಒಂದು ನಿರ್ದಿಷ್ಟ ಪ್ರಮಾಣದಲ್ಲಿ ಭಿಕ್ಷುಕರು ಇರಲೇ ಬೇಕು. ಇಂಗ್ಲೆಂಡಿನಲ್ಲಿ ಭಿಕ್ಷಾಟನೆಯ ನಿರ್ಮೂಲನಕ್ಕಾಗಿ ಇರುವ ಕಾಯಿದೆಗಳು ಇಲ್ಲಿನ ನಿರ್ಗತಿಕ ರಿಗೆಲ್ಲ ಒಪ್ಪಿಗೆಯಾಗಲಾರವು. ಅಲ್ಲಿ ಕೆಲವರಿಗೆ ಭಿಕ್ಷುಕರ ಕಾಲೊನಿಗೆ ಹೋಗುವುದಕ್ಕಿಂತ ಬೀದಿಯಲ್ಲಿ ಉಪವಾಸವಿರುವುದೇ ಮೇಲೆನ್ನಿಸುತ್ತದೆ. ಅವರಿಗೆ ಸ್ವಾತಂತ್ರ್ಯವೇ ಹೆಚ್ಚು ಪ್ರಿಯ ವಾಗಿರುತ್ತದೆ. ಆದರೆ ಅವರು ಹಾಗೆ ಮಾಡಲು ಕಾನೂನು ಬಿಡುವುದಿಲ್ಲ... ಒಂದು ಚೂರು ರೊಟ್ಟಿಯನ್ನು ಭಿಕ್ಷೆ ಹಾಕಿದಾಗ ಒಬ್ಬ ತಿರುಕ ಅದನ್ನು ತೆಗೆದುಕೊಂಡ; ಅವನು ತಿರುಕನಾಗಿಯೇ ಉಳಿದ. ಅವನಲ್ಲೇನೂ ಮಾರ್ಪಾಡಾಗಲಿಲ್ಲ. ಆದರೆ ಮತ್ತೊಬ್ಬ ತಿರುಕ ಭಿಕ್ಷೆ ತೆಗೆದುಕೊಳ್ಳಲಿಲ್ಲ; ಅವನೊಬ್ಬ ಕಳ್ಳನಾದ!... ಈ ಕಳ್ಳರಿಗಾಗಿ ಹೆಚ್ಚು ಪೋಲಿಸರನ್ನು ನೇಮಕ ಮಾಡಿಕೊಳ್ಳಬೇಕಾ ಯಿತು, ಹೆಚ್ಚು ಜೈಲುಗಳನ್ನು ತೆರೆಯಬೇಕಾಯಿತು. ಆದ್ದರಿಂದ, ಈ ಭಿಕ್ಷಾಟನೆಯ ಹಾಗೂ ಅಪಾತ್ರದಾನದ ಸಮಸ್ಯೆಯನ್ನು ಕೇವಲ ಲೆಕ್ಕಾಚಾರದ ದೃಷ್ಟಿಯಿಂದ ನೋಡಿದಾಗಲೂ, ಐರೋಪ್ಯ ಕಾಯಿದೆಗಳ ವಿಧಾನಕ್ಕಿಂತ ನಮ್ಮ ಭಾರತದ ಅತಿಧಾರಾಳವಂತಿಕೆಯೇ ಮೇಲು ಎನ್ನಬಹುದು.” ಹೀಗೆ ಸ್ವಾಮೀಜಿ ಅನ್ನದಾನದ ಹಿರಿಮೆಯನ್ನು ಕೊಂಡಾಡಿದರು. ಇಲ್ಲಿ ಅವರ ಸ್ವತಂತ್ರ ಆಲೋ ಚನಾ ವಿಧಾನವನ್ನು, ಸಹೃದಯತೆಯನ್ನು, ಮಾತಿನ ತರ್ಕಬದ್ಧತೆಯನ್ನು ಕಂಡು ಬೆರಗಾಗದಿರುವಂತೆಯೇ ಇಲ್ಲ.

೧೩ನೇ ತಾರೀಕು ಶನಿವಾರ ಸಂಜೆ ಪಚ್ಚಯ್ಯಪ್ಪ ಹಾಲ್​ನಲ್ಲಿ ನೆರೆದಿದ್ದ ಭಾರೀ ಸಭೆಯನ್ನು ದ್ದೇಶಿಸಿ ಸ್ವಾಮೀಜಿ ತಮ್ಮ ಮೂರನೆಯ ಉಪನ್ಯಾಸವನ್ನು ನೀಡಿದರು. ವಿಷಯ: “ಭಾರತೀಯ ಜನಜೀವನದಲ್ಲಿ ವೇದಾಂತದ ಅನುಷ್ಠಾನ”. ವೇದಾಂತವೆಂದರೆ ಕೇವಲ ಕೆಲಸಕ್ಕೆ ಬಾರದ ಕಂತೆ ಎಂಬ ಅರ್ಥ ಪ್ರಚಲಿತವಾಗುತ್ತಿದ್ದ ಕಾಲ ಅದು. ವೇದ-ಶಾಸ್ತ್ರಗಳನ್ನೆಲ್ಲ ಹರಿದೊಗೆಯಬೇಕೆಂಬ ಪ್ರಚಾರ ನಡೆಯುತ್ತಿದ್ದ ಕಾಲ ಅದು. ಅಂತಹ ಸಂದರ್ಭದಲ್ಲಿ ವೇದಾಂತದ ಬಗ್ಗೆ ಸ್ವಾಮೀಜಿ ಮಾಡಿದ ಈ ಭಾಷಣ ಅತ್ಯಂತ ಪ್ರಭಾವಶಾಲಿಯಾಗಿತ್ತು.

ಮೊದಲಿಗೆ ಸ್ವಾಮೀಜಿ, ‘ಹಿಂದೂ’ ಎಂಬ ಶಬ್ದದ ವ್ಯುತ್ಪತ್ತಿಯನ್ನು ವಿವರಿಸಿದರು: ಸಿಂಧೂ ನದಿಯ ಬಲಭಾಗದಲ್ಲಿದ್ದವರನ್ನೆಲ್ಲ ಪುರಾತನ ಪರ್ಷಿಯನ್ನರು ‘ಹಿಂದೂಗಳು’ ಎಂದರು. ಗ್ರೀಕರು ಈ ‘ಹಿಂದೂ’ಗಳನ್ನು ‘ಇಂಡಿಯಾ’ದವರು ಎಂದು ಕರೆದರು. ಆದ್ದರಿಂದ, ‘ಹಿಂದೂ’\\ಎಂಬ ಶಬ್ದಕ್ಕೆ ಈಗ ಯಾವುದೇ ಅರ್ಥವಿಲ್ಲ ಎಂದರು ಸ್ವಾಮೀಜಿ. ಆದರೆ ಆ ಪದವು ಬಳಕೆಗೆ ಬಂದಾಗಿರುವುದರಿಂದ ಅದನ್ನೇನೂ ಈಗ ತಿರಸ್ಕರಿಸಬೇಕಾಗಿಲ್ಲ ಎಂದೂ ಅವರು ಸ್ಪಷ್ಟಪಡಿಸಿ ದರು. ಬಳಿಕ ಹಿಂದೂ ಧರ್ಮದ ಲಕ್ಷಣಗಳನ್ನು ವಿವರಿಸುತ್ತ ಹೇಳಿದರು, “ನಿಜ ಹೇಳಬೇಕೆಂದರೆ, ಹಿಂದೂ ಧರ್ಮವೆಂಬುದು ಹಲವಾರು ಧರ್ಮಗಳ, ಹಲವಾರು ಭಾವನೆಗಳ, ಹಲವಾರು ತತ್ತ್ವಶಾಸ್ತ್ರ-ಸಂಪ್ರದಾಯ-ಆಚರಣೆಗಳ ಸಮ್ಮಿಶ್ರಣ. ಇವೆಲ್ಲಕ್ಕೂ ಸಾಮಾನ್ಯವಾದ ಯಾವುದೇ ಒಂದು ಧಾರ್ಮಿಕ ಚೌಕಟ್ಟು, ಆಧಿಪತ್ಯ ಇಲ್ಲ. ಆದರೆ ಇವೆಲ್ಲವೂ ಅಧೀನವಾಗಿರುವುದು ವೇದಗಳ ಜ್ಞಾನಕಾಂಡಕ್ಕೆ. ಈ ಎಲ್ಲ ವಿಭಿನ್ನ ಮತಗಳೂ ಉಪನಿಷತ್ತುಗಳು ಹಾಗೂ ವೇದಾಂತವನ್ನೊಳ ಗೊಂಡ ಜ್ಞಾನಕಾಂಡವೇ ತಮ್ಮ ಅತ್ಯುನ್ನತ ಪ್ರಮಾಣವೆಂದು ಅಂಗೀಕರಿಸುತ್ತವೆ... ವೇದಾಂತ ವೆಂದರೆ ಅದ್ವೈತ ಮಾತ್ರವೇ ಎಂದು ಬಹಳ ಜನ ತಿಳಿದಿದ್ದಾರಾದರೂ ಅದು ನಿಜವಲ್ಲ; ದ್ವೈತ-ವಿಶಿಷ್ಟಾದ್ವೈತ-ಅದ್ವೈತಗಳೆಲ್ಲಕ್ಕೂ ವೇದಾಂತವೇ ಪ್ರಮಾಣ... ಅಷ್ಟೇ ಅಲ್ಲ; ನಮ್ಮ ಶಾಸ್ತ್ರ ಗ್ರಂಥಗಳಲ್ಲಿ ದ್ವೈತ-ಅದ್ವೈತಗಳೆರಡಕ್ಕೂ ಸೂಕ್ತಸ್ಥಾನವಿದೆ... 

“ಕಟ್ಟಾ ಅದ್ವೈತಿಯಾಗಿದ್ದರೂ ಅಷ್ಟೇ ನಿಷ್ಠಾವಂತ ದ್ವೈತಿಯಾಗಿದ್ದ, ಎಷ್ಟು ಜ್ಞಾನಿಯೋ ಅಷ್ಟೇ ದೊಡ್ಡ ಭಕ್ತನೂ ಆಗಿದ್ದ ದೇವಮಾನವನೊಬ್ಬನ ಸಂಪರ್ಕಕ್ಕೆ ಬರುವ ಸೌಭಾಗ್ಯ ನನ್ನದಾಗಿತ್ತು. ಈತನ ಸಹವಾಸ ಮಾಡಿದ ನಾನು, ಹಿಂದಿನ ಭಾಷ್ಯಕಾರರ ಟೀಕೆ-ಟಿಪ್ಪಣಿಗಳನ್ನು ಕುರುಡಾಗಿ ಓದಿಕೊಳ್ಳಲಿಲ್ಲ; ಬದಲಾಗಿ ಉಪನಿಷತ್ತುಗಳನ್ನು ಸ್ವತಂತ್ರವಾಗಿ, ಇನ್ನೂ ಉತ್ತಮ ವಿಧಾನದಲ್ಲಿ ಅಧ್ಯಯನ ಮಾಡಿದೆ. ನನ್ನ ಈ ಸಂಶೋಧನೆಯ ಫಲವಾಗಿ ನಾನು ತಳೆದ ಅಭಿಪ್ರಾಯವೇನೆಂದರೆ, ಈ ಎಲ್ಲ ಶಾಸ್ತ್ರಗಳಲ್ಲಿ ಪರಸ್ಪರ ವ್ಯತಿರಿಕ್ತತೆಯಾಗಲಿ ಅಸಂಬದ್ಧತೆಗಳಾಗಲಿ ಇಲ್ಲ ಎಂಬುದು... ಬದಲಾಗಿ ಅವು ಒಂದಕ್ಕೊಂದು ಅತ್ಯಂತ ಸುಂದರವಾಗಿ ಅದ್ಭುತವಾಗಿ ಹೊಂದಿಕೊಂಡಿವೆ. ಒಂದರಿಂದ ಇನ್ನೊಂದು ಸಹಜವಾಗಿ ತರ್ಕಬದ್ಧವಾಗಿ ಮೂಡುತ್ತದೆ. ಒಂದು ಭಾವ ಮತ್ತೊಂದು ಭಾವಕ್ಕೆ ಕರೆದೊಯ್ಯುತ್ತದೆ.”

ಭಾರತೀಯರನ್ನುದ್ದೇಶಿಸಿ ಸ್ವಾಮೀಜಿ ಹೇಳುತ್ತಾರೆ: “ನೀವು ಸುಧಾರಣೆಯ ಬಗ್ಗೆ ಮಾತನಾಡು ತ್ತೀರಿ; ಕಳೆದ ನೂರು ವರ್ಷಗಳಿಂದಲೂ ಮಾತನಾಡುತ್ತಲೇ ಇದ್ದೀರಿ. ಆದರೆ ಅನುಷ್ಠಾನದ ಪ್ರಶ್ನೆ ಬಂದಾಗ ಮಾತ್ರ ಅದರ ಸುಳಿವೇ ಕಾಣುವುದಿಲ್ಲ... ಅದಕ್ಕೆ ಕಾರಣವೇನು? ಕಾರಣ ವೊಂದೇ: ನೀವು ದುರ್ಬಲರು. ನಿಮ್ಮ ದೇಹ ದುರ್ಬಲ, ಮನಸ್ಸು ದುರ್ಬಲ, ಮತ್ತು ನಿಮ್ಮಲ್ಲಿ ನಿಮಗೆ ಶ್ರದ್ಧೆಯಿಲ್ಲ... ಶಕ್ತಿ! ಶಕ್ತಿಯೇ ನಮಗಿಂದು ಬೇಕಾಗಿರುವುದು. ಆದರೆ, ಆ ಶಕ್ತಿಯನ್ನು ನೀಡಬಲ್ಲವರಾರು? ನಮಗೆ ಆ ಶಕ್ತಿ ಎಲ್ಲಿ ದೊರೆತೀತು?–ಉಪನಿಷತ್ತುಗಳಲ್ಲಿ!....

“ಉಪನಿಷತ್ತುಗಳ ಪುಟಪುಟದಲ್ಲೂ ಶಕ್ತಿಯೇ ಎದ್ದುಕಾಣುತ್ತದೆ. ನಾನು ಕಲಿತ ಅತಿ ದೊಡ್ಡ ಪಾಠವೇ ಇದು. ‘ಓ ಮಾನವ, ಶಕ್ತಿವಂತನಾಗು, ದೌರ್ಬಲ್ಯವನ್ನು ತೊರೆ!’–ಇದೇ ಉಪನಿಷತ್ತು ಘೋಷಿಸುವ ಮಹಾವಾಣಿ. ಮಾನವನಲ್ಲಿ ಸಹಜವಾಗಿಯೇ ದೌರ್ಬಲ್ಯವಿಲ್ಲವೆ? ಹೌದು, ಇದೆ. ಆದರೆ ಆ ದೌರ್ಬಲ್ಯವನ್ನು ದೌರ್ಬಲ್ಯ ಕಳೆಯಬಲ್ಲುದೆ? ಕೆಸರಿನಿಂದ ಕೆಸರನ್ನು ತೊಳೆಯಬಹುದೆ? ಪಾಪವು ಪಾಪವನ್ನು ತೊಡೆಯಬಲ್ಲುದೆ? ಇಲ್ಲ! ‘ಓ ಮಾನವ, ಮೇಲೇಳು, ಶಕ್ತಿಶಾಲಿಯಾಗು!’ ಎಂದು ಮೊಳಗುತ್ತದೆ ಉಪನಿಷತ್ತು. ‘ಅಭೀಃ’ ‘ಅಭೀಃ’ ‘ನಿರ್ಭಯ’ ‘ನಿರ್ಭಯ’ ಎಂಬ ಶಬ್ದವನ್ನು ಮತ್ತೆ ಮತ್ತೆ ಬಳಸುವ ಗ್ರಂಥ ಈ ಜಗತ್ತಿನಲ್ಲಿ ಅದೊಂದೇ...

“ಒಬ್ಬ ಬೆಸ್ತರವನು ತನ್ನನ್ನು ಆತ್ಮನೆಂದು ಭಾವಿಸಿದರೆ ಅವನು ಇನ್ನೂ ಉತ್ತಮ ಬೆಸ್ತನಾಗು ತ್ತಾನೆ; ಒಬ್ಬ ವಿದ್ಯಾರ್ಥಿಯು ತನ್ನನ್ನು ಆತ್ಮನೆಂದರಿತರೆ ಅವನು ಇನ್ನೂ ಉತ್ತಮ ವಿದ್ಯಾರ್ಥಿ ಯಾಗುತ್ತಾನೆ... ಇದೇ ಮಾತು ಇತರ ವೃತ್ತಿಯವರಿಗೂ ಅನ್ವಯಿಸುತ್ತದೆ... 

“ಜಾತಿಯು ಪ್ರಕೃತಿವಿಹಿತ ಪದ್ಧತಿ... ಸಮಾಜದಲ್ಲಿ ಒಬ್ಬೊಬ್ಬರು ಒಂದೊಂದು ವೃತ್ತಿಯನ್ನು ಮಾಡುತ್ತಿರಬಹುದು. ನೀವು ರಾಜ್ಯಭಾರ ಮಾಡುತ್ತಿರಬಹುದು, ನಾನು ಜೋಡು ರಿಪೇರಿ ಮಾಡುತ್ತಿರಬಹುದು. ಹಾಗೆಂದ ಮಾತ್ರಕ್ಕೆ ನೀವು ನನಗಿಂತ ಶ್ರೇಷ್ಠರೆಂದೇನೂ ಆಗಲಿಲ್ಲ. ಅಲ್ಲದಿದ್ದರೆ, ನೀವು ಜೋಡು ರಿಪೇರಿ ಮಾಡಿ ನೋಡೋಣ! ಅಥವಾ ನಾನಾದರೂ ರಾಜ್ಯವಾಳ ಬಲ್ಲೆನೆ? ನಾನು ಜೋಡು ಹೊಲಿಯುವುದರಲ್ಲಿ ಜಾಣ, ನೀವು ವೇದವನ್ನೋದುವುದರಲ್ಲಿ ಜಾಣರು. ಆದರೆ ನೀವು ನನ್ನ ತಲೆಯ ಮೇಲೆ ಹತ್ತಿ ತುಳಿಯಲು ಅದು ಕಾರಣವಾಗಬೇಕಿಲ್ಲ.

“ಸಮಸ್ತ ಸ್ತ್ರೀಪುರುಷರನ್ನೂ ದೇವರೆಂದೇ ನೋಡಿ. ನೀವು ಯಾರಿಗೂ ಸಹಾಯ ಮಾಡಲಾರಿರಿ; ಆದರೆ ಸೇವೆ ಮಾಡಬಲ್ಲಿರಿ. ನಿಮಗೆ ಅಂತಹ ಭಾಗ್ಯವಿದ್ದರೆ ಭಗವಂತನ ಮಕ್ಕಳ ಸೇವೆ ಮಾಡಿ; ತನ್ಮೂಲಕ ಸಾಕ್ಷಾತ್ ಭಗವಂತನ ಸೇವೆಯನ್ನೇ ಮಾಡಿ. ತನ್ನ ಮಕ್ಕಳಲ್ಲಿ ಯಾರಾದರೊಬ್ಬರ ಸೇವೆ ಮಾಡುವ ಅವಕಾಶವನ್ನು ಭಗವಂತ ನಿಮಗೆ ಕರುಣಿಸಿದ್ದೇ ಆದರೆ ನೀವು ಧನ್ಯರೇ ಸರಿ... ಆದ್ದರಿಂದ ಇತರರ ಸೇವೆಯನ್ನು ಭಗವಂತನ ಪೂಜೆಯೆಂದು ಭಾವಿಸಿರಿ. ದರಿದ್ರರಲ್ಲಿ ಭಗವಂತ ನನ್ನು ಕಾಣಬೇಕು. ನಾವು ಅವರ ಸೇವೆ ಮಾಡುವುದಾದರೆ ಅದರಿಂದ ನಮಗೇ ಮುಕ್ತಿ. ನಾವು ಆತನ ಸೇವೆ ಮಾಡಿ ಉದ್ಧಾರವಾಗಲೆಂದು ಭಗವಂತನೇ ಕುಷ್ಠರೋಗಿಯಾಗಿ, ಮೂರ್ಖನಾಗಿ, ಪಾಪಿಯಾಗಿ ಬರುತ್ತಾನೆ! ನನ್ನ ಈ ಮಾತುಗಳು ಎದೆಗಾರಿಕೆಯ ಮಾತುಗಳು, ನಿಜ. ಅದನ್ನೇ ಮತ್ತೊಮ್ಮೆ ಎದೆ ತಟ್ಟಿ ಹೇಳುತ್ತೇನೆ... 

“ಈ ಲೋಕಕ್ಕೆ ಬೇಕಾದ ಬೆಳಕನ್ನೆಲ್ಲ ತನ್ನಿ. ಎಲ್ಲರಿಗೂ ಬೆಳಕು ದೊರಕಲಿ. ಪ್ರತಿಯೊಬ್ಬನೂ ಮುಕ್ತನಾಗುವವರೆಗೂ ನಮ್ಮ ಕರ್ತವ್ಯ ಮುಗಿಯದು. ಬಡವರಿಗೆ ಬೆಳಕು ಕೊಡಿ; ಶ್ರೀಮಂತರಿಗೆ ಹೆಚ್ಚು ಬೆಳಕು ಕೊಡಿ. ಏಕೆಂದರೆ ಬೆಳಕಿನ ಆವಶ್ಯಕತೆಯು ಬಡವರಿಗಿಂತ ಶ್ರೀಮಂತರಿಗೆ ಹೆಚ್ಚು. ಅವಿದ್ಯಾವಂತರಿಗೆ ಬೆಳಕು ಕೊಡಿ; ವಿದ್ಯಾವಂತರಿಗೆ ಹೆಚ್ಚು ಬೆಳಕು ಕೊಡಿ. ಏಕೆಂದರೆ ಈಗಿನ ವಿದ್ಯಾಭ್ಯಾಸ ಬಹುಮಟ್ಟಿಗೆ ನಿರರ್ಥಕವಾದದ್ದು, ನೀರಸವಾದದ್ದು. ಆದ್ದರಿಂದ ಎಲ್ಲರಿಗೂ ಬೆಳಕು ತೋರಿ. ಉಳಿದುದನ್ನು ದೇವರಿಗೆ ಬಿಟ್ಟುಬಿಡಿ. ‘ಕರ್ಮಣ್ಯೇವಾಧಿಕಾರಸ್ತೇ ಮಾ ಫಲೇಷು ಕದಾಚನ –ನಿನಗೆ ಕರ್ಮ ಮಾಡಲು ಮಾತ್ರ ಅಧಿಕಾರ, ಫಲಕ್ಕಲ್ಲ.’ಇದು ಭಗವಂತನ ವಾಣಿ.”

ದೀರ್ಘ ಕರತಾಡನ-ಹರ್ಷೋದ್ಗಾರಗಳ ನಡುವೆ ಸಭೆ ಮುಕ್ತಾಯಗೊಂಡಿತು. ಬಳಿಕ ಸ್ವಾಮೀಜಿ ಶ್ರೀ ಎಲ್. ಗೋವಿಂದ ದಾಸ್ ಎಂಬುವರ ಮನೆಗೆ ತೆರಳಿದರು. ಅಲ್ಲಿ ಅವರ ಸತ್ಕಾ ರಾರ್ಥವಾಗಿ ಸಾಂಸ್ಕೃತಿಕ ಕಾರ್ಯಕ್ರಮವೊಂದನ್ನು ಏರ್ಪಡಿಸಲಾಗಿತ್ತು. ಈ ವಿಷಯವು ಜನರಿಗೆ ಇದ್ದಕ್ಕಿದ್ದಂತೆ ತಿಳಿದುಬಂದಿದ್ದರೂ ಅಲ್ಲಿ ಬಹುದೊಡ್ಡ ಜನ ಸಂದಣಿಯೇ ನೆರೆದಿತ್ತು. ಅದರಲ್ಲಿ ಐರೋಪ್ಯರು ಬಹು ಸಂಖ್ಯೆಯಲ್ಲಿದ್ದರು. ಮೊದಲಿಗೆ, ಸ್ವಾಮೀಜಿಯವರಿಗೆ ಇಂಗ್ಲಿಷ್ ಹಾಗೂ ಸಂಸ್ಕೃತದಲ್ಲಿ ಒಂದೊಂದು ಬಿನ್ನವತ್ತಳೆಯನ್ನು ಅರ್ಪಿಸಲಾಯಿತು. ಬಳಿಕ ಇಬ್ಬರು ಪ್ರತಿಭಾ ವಂತ ಸಂಗೀತಗಾರರು ವೀಣೆ ಹಾಗೂ ಗಿಟಾರಿನ ಯುಗಳಬಂದಿ ಕಾರ್ಯಕ್ರಮವನ್ನು ನಡೆಸಿ ಜನಮನವನ್ನು ರಂಜಿಸಿದರು. ಅನಂತರ ಶ್ರೀ ಗೋವಿಂದ ದಾಸ್​ರವರು ಸ್ವಾಮೀಜಿಯವರಿಗೆ ರೇಶ್ಮೆಯ ಕಾಷಾಯ ವಸ್ತ್ರವನ್ನು ಸಮರ್ಪಿಸಿ ಮಾಲಾರ್ಪಣೆ ಮಾಡಿ ಗೌರವಿಸಿದರು.

ಮದ್ರಾಸಿನಲ್ಲಿ ಸ್ವಾಮೀಜಿಯವರ ನಾಲ್ಕನೆಯ (ಹಾಗೂ ಕಡೆಯ) ಸಾರ್ವಜನಿಕ ಉಪನ್ಯಾಸ ವನ್ನು ಫೆಬ್ರವರಿ ೧೪ರಂದು ಹಾರ್ಮ್​ಸ್ಟನ್ ಸರ್ಕಸ್ ಮೈದಾನದಲ್ಲಿ ಏರ್ಪಡಿಸಲಾಗಿತ್ತು. ಅಂದು ಭಾನುವಾರ. ಅಲ್ಲದೆ, ಈ ಉಪನ್ಯಾಸಕ್ಕೆ ಪ್ರವೇಶ ಶುಲ್ಕವಿರಲಿಲ್ಲ. ಕೆಲವು ಗಂಟೆಗಳ ಮೊದಲೇ ಮೂರು ಸಾವಿರಕ್ಕೂ ಹೆಚ್ಚು ಜನ ಅಲ್ಲಿ ನೆರೆದಿದ್ದರು; ಆದರೂ ಅತ್ಯಂತ ಶಿಸ್ತಿನಿಂದ ವರ್ತಿಸಿ ದ್ದರು. ಅಧ್ಯಕ್ಷ ಸ್ಥಾನದಲ್ಲಿ ಆನರಬಲ್ ಶ್ರೀ ಎನ್. ಸುಬ್ಬರಾವ್ ಪಂತುಲು ಇದ್ದರು. ಅಂದಿನ ಭಾಷಣದ ವಿಷಯ “ಭಾರತದ ಭವಿತವ್ಯ”. ಸ್ವಾಮೀಜಿ ತಮ್ಮ ಭಾಷಣವನ್ನು ಹಿಂದಿನ ಎಲ್ಲ ಭಾಷಣಗಳಿಗಿಂತ ಸ್ಫೂರ್ತಿಯುತವಾಗಿ ಹಾಗೂ ರೋಮಾಂಚಕರವಾಗಿ ಪ್ರಾರಂಭಿಸಿದರು: “ಅನಾದಿಯಿಂದಲೂ ಮನೆಮಾಡಿಕೊಂಡಿದ್ದ ಸನಾತನ ಜ್ಞಾನರಾಶಿಯು ಜಗತ್ತಿನ ಇತರೆಲ್ಲ ರಾಷ್ಟ್ರ ಗಳಿಗೆ ಪಸರಿಸಿದ್ದು ಈ ನಮ್ಮ ಭರತಖಂಡದಿಂದಲೇ. ಮಾನವನ ಸ್ವಭಾವದ ಬಗ್ಗೆ, ಮಾನವನ ಒಳಗೆಯೇ ಇರುವ ಅಂತರಾತ್ಮನ ಬಗ್ಗೆ ಜಿಜ್ಞಾಸೆ ಪುಟಿದೆದ್ದುದು ಪ್ರಥಮತಃ ಈ ನಮ್ಮ ಭರತ ಖಂಡದಿಂದಲೇ. ಮಹಾಮಹಾ ಪುಷಿಶ್ರೇಷ್ಠರ ಪಾದಧೂಳಿಯಿಂದ ಪುನೀತವಾದ ನಾಡು ಈ ನಮ್ಮ ಭರತಖಂಡ. ಆತ್ಮದ ಅಮೃತತ್ವ, ಸರ್ವಸಮೀಕ್ಷಕನಾದ ಪ್ರಭುವಿನ ಅಸ್ತಿತ್ವ, ಪ್ರಕೃತಿ ಯಲ್ಲೂ ಮಾನವನಲ್ಲೂ ಸರ್ವಾಂತರ್ಯಾಮಿಯಾದ ವಿಶ್ವರೂಪೀ ಈಶ್ವರತ್ವ, ಧರ್ಮ ಮತ್ತು ತತ್ತ್ವಜ್ಞಾನಗಳ ಶ್ರೇಷ್ಠತಮ ಆದರ್ಶ–ಇವುಗಳೆಲ್ಲ ಪ್ರಥಮತಃ ಮೂಡಿ ಮೇಲೆದ್ದು ತುತ್ತತುದಿ ಯನ್ನು ಮುಟ್ಟಿದ್ದು ಈ ನಮ್ಮ ಭರತಖಂಡದಲ್ಲೇ. ಆಧ್ಯಾತ್ಮಿಕತೆ ಮತ್ತು ತತ್ತ್ವಜ್ಞಾನಗಳು ಉಗ ಮಿಸಿ ಮತ್ತೆ ಮತ್ತೆ ಸಮುದ್ರದ ಅಲೆಯೋಪಾದಿಯಲ್ಲಿ ನುಗ್ಗಿ ಹೋಗಿ ಜಗತ್ತನ್ನೇ ಮುಳುಗಿಸು ವಂತಾದದ್ದು ಈ ನಮ್ಮ ಭರತಖಂಡದಿಂದಲೇ. ಅಲ್ಲದೆ, ಜೀರ್ಣವಾಗಿ ಬಿದ್ದು ಹೋಗುತ್ತಿರುವ ಮಾನವ ಜನಾಂಗಗಳಿಗೆ ಪ್ರಾಣವಾಯುವನ್ನು ಮತ್ತು ಶಕ್ತಿಯನ್ನು ತುಂಬಿಸುವ ಸಲುವಾಗಿ ಮತ್ತೆ ಅಂತಹ ಅಲೆಗಳು ನುಗ್ಗಿ ಹೋಗಬೇಕಾಗಿರುವುದೂ ಇದೇ ಭರತಖಂಡದಿಂದಲೇ. ಶತಶತಮಾನ ಗಳಿಂದ ಪರಕೀಯರ ನೂರಾರು ಆಕ್ರಮಣಗಳ ಆಘಾತವನ್ನು, ಮತ ಪದ್ಧತಿ ಹಾಗೂ ಸಂಪ್ರ ದಾಯಗಳ ಬುಡವನ್ನೇ ಅಡಿಮೇಲು ಮಾಡುವಂತಹ ಆಂದೋಲನಗಳ ಆಘಾತವನ್ನು ಸಹಿಸಿ ಬಾಳಿ ಬದುಕಿರುವ ಭೂಮಿ ಈ ಭರತಖಂಡ. ಅಳಿವಿಲ್ಲದ ಅಂತಸ್ಸತ್ತ್ವದಿಂದ ಕೂಡಿ ಎಂತಹ ಹೆಬ್ಬಂಡೆಗಿಂತಲೂ ಅಚಲವಾಗಿ ನಿಂತಿರುವ ಭೂಮಿ ಈ ಭರತಖಂಡ. ಆದ್ಯಂತರಹಿತವಾದ ಅಮರ ಆತ್ಮದಂತೆಯೇ ಇದೆ ಇದರ ಬಾಳು. ಇಂತಹ ರಾಷ್ಟ್ರದ ಅಮೃತಪುತ್ರರು ನಾವು!”

ಹೀಗೆ ಸ್ವಾಮೀಜಿ ನಮ್ಮ ದೇಶದ ಪುರಾತನ ವೈಭವವನ್ನು ಬಣ್ಣಿಸಿದರು. ಈ ಮೂಲಕ ಜನರಲ್ಲಿ ಸರಿಯಾದ ಪ್ರಜ್ಞೆ ಜಾಗೃತವಾಗಿ ಅವರು ಇನ್ನಷ್ಟು ಬಲಿಷ್ಠರಾಗಿ ಹಿಂದಿಗಿಂತಲೂ ಸಶಕ್ತವಾದ ರಾಷ್ಟ್ರವನ್ನು ನಿರ್ಮಾಣಮಾಡಲು ಸಮರ್ಥರಾಗಲಿ ಎಂಬುದು ಸ್ವಾಮೀಜಿಯವರ ಉದ್ದೇಶ. ಬಳಿಕ ತಮ್ಮ ಮಾತನ್ನು ಮುಂದುವರಿಸುತ್ತ ಹೇಳಿದರು–“ಭಾರತದ ಸಮಸ್ಯೆಗಳು ಬಹು ಕ್ಲಿಷ್ಟವಾದವುಗಳು. ಯಾವುದೇ ಇತರ ರಾಷ್ಟ್ರದ ಸಮಸ್ಯೆಗಳಿಗಿಂತಲೂ ಕ್ಲಿಷ್ಟವಾದವುಗಳು... ಈ ವೈವಿಧ್ಯಮಯ ಭಾರತದಲ್ಲಿ ಒಂದೇ ಒಂದು ಸಮಾನಾಂಶವೆಂದರೆ ನಮ್ಮ ಧರ್ಮ. ನಾವು ನಿರ್ಮಾಣಕಾರ್ಯವನ್ನು ಕೈಗೊಳ್ಳಲು ಇದೊಂದೇ ಸಮಾನ ನೆಲೆಗಟ್ಟು. ಆದ್ದರಿಂದ ಧಾರ್ಮಿಕ ಏಕೀಕರಣವು ಭವಿಷ್ಯ ಭಾರತದ ಅಭ್ಯುದಯದ ಮೊದಲ ಹೆಜ್ಜೆ... ಭಾರತೀಯರ ಮನೋ ಭಾವವು ಮೊದಲನೆಯದಾಗಿ ಧರ್ಮಪ್ರಧಾನವಾದುದು; ಆಮೇಲೆ ಇನ್ನೆಲ್ಲ. ಆದ್ದರಿಂದ ಈ ನೆಲೆಗಟ್ಟನ್ನೇ ಬಲಪಡಿಸಬೇಕಾಗಿದೆ. ಹೇಗೆ? ನಮ್ಮ ಸನಾತನ ಧರ್ಮಗ್ರಂಥಗಳಲ್ಲಡಗಿರುವ ಅಮೃತಮಯವಾದ ಆತ್ಮವಿದ್ಯೆಯನ್ನು ಹೊರತಂದು ಅದು ಸಕಲರ ಸೊತ್ತಾಗುವಂತೆ ಮಾಡ ಬೇಕೆಂಬುದೇ ನನ್ನ ಮೊದಲ ಯೋಜನೆ... ”

“ಜನರಿಗೆ ಅವರಾಡುವ ಭಾಷೆಯಲ್ಲಿಯೇ ಶಿಕ್ಷಣ ಕೊಡಿ. ಅವರಿಗೆ ಜ್ಞಾನವನ್ನು ಕೊಡಿ. ಆದರೆ ಅಷ್ಟೇ ಸಾಲದು, ಅದಕ್ಕಿಂತ ಹೆಚ್ಚಿನದೊಂದು ಬೇಕು. ಅದೇ ಸಂಸ್ಕೃತಿ, ಅವರಿಗೆ ಸಂಸ್ಕೃತಿಯನ್ನು ಕೊಡಿ. ಕೆಳದರ್ಜೆಯ ಜನರು ಶಾಶ್ವತವಾಗಿ ಮೇಲೆದ್ದು ನಿಲ್ಲುವಂತಾಗಬೇಕಾದರೆ, ಮಾನಸಿಕ ತುಮುಲವಿಲ್ಲದಂತಾಗಬೇಕಾದರೆ, ಮೊದಲು ಅವರು ಉಚ್ಚವರ್ಣದವರ ಸಂಸ್ಕೃತಿಯನ್ನು ಕಲಿತು ಮೈಗೂಡಿಸಿಕೊಳ್ಳಬೇಕು. ಉನ್ನತ ವರ್ಗದವರನ್ನು ಕೆಳಗೆಳೆಯುವುದಲ್ಲ ಪರಿಹಾರೋಪಾಯ, ಬದಲಾಗಿ, ನಿಮ್ಮ ವರ್ಗದವರನ್ನು ಉನ್ನತ ವರ್ಗದವರ ಮಟ್ಟಕ್ಕೆ ಎತ್ತುವುದು. ನಮ್ಮ ಗ್ರಂಥ ಗಳಲ್ಲೆಲ್ಲ ಹೇಳಿರುವುದು ಈ ಬಗೆಯ ಕಾರ್ಯಪ್ರಣಾಳಿಯನ್ನೇ. ಸಕಲರನ್ನೂ ಬ್ರಾಹ್ಮಣತ್ವಕ್ಕೆ ಏರಿಸುವುದೇ ಅವುಗಳ ಉದ್ದೇಶವಾಗಿದೆ... ಬ್ರಾಹ್ಮಣರಾದವರು ತಾವು ಶತಶತಮಾನಗಳಿಂದ ಸಂಪಾದಿಸಿದ ಸಂಸ್ಕೃತಿಯನ್ನು, ಜ್ಞಾನವನ್ನು ಭಾರತದ ಸಮಸ್ತ ಜನರಿಗೂ ಧಾರೆಯೆರೆದು ತನ್ಮೂ ಲಕ ಜನಸಾಮಾನ್ಯರನ್ನು ಮೇಲೆತ್ತುವ ಪ್ರಯತ್ನ ಮಾಡಬೇಕು... 

“ಭಾವೀ ಭಾರತದ ಹಿರಿಮೆ-ಗರಿಮೆಗಳ ರಹಸ್ಯವಡಗಿರುವುದು ಸಂಘಟನೆಯಲ್ಲಿ, ಶಕ್ತಿ ಸಂಚಯನದಲ್ಲಿ ಇಚ್ಛಾಶಕ್ತಿಯನ್ನು ಒಗ್ಗೂಡಿಸುವುದರಲ್ಲಿ. ಮೊದಲು ಜಗಳಗಳನ್ನೆಲ್ಲ ನಿಲ್ಲಸ ಬೇಕು. ಮತ್ತು, ಇನ್ನು ಐವತ್ತು ವರ್ಷಗಳವರೆಗೆ ‘ಭಾರತ’ ಎಂಬುದೊಂದೇ ನಮ್ಮ ಮಹಾ ಮಂತ್ರವಾಗಬೇಕು. ಅತ್ಯಗತ್ಯವಾದುದೇನೆಂದರೆ ಚಿತ್ತಶುದ್ಧಿ, ಹೃದಯ ಪರಿಶುದ್ಧತೆ. ಅದು ಹೇಗೆ ಸಿದ್ಧಿಸುತ್ತದೆ? ಮೊಟ್ಟಮೊದಲು ನಿಮ್ಮ ಸುತ್ತಮುತ್ತಲಿರುವ ಈ ವಿರಾಟ್ ಸ್ವರೂಪದ ದೇವರನ್ನು –ಭಾರತವನ್ನು–ಪೂಜಿಸಿ. ದೇಶಬಾಂಧವರೇ ನಮ್ಮ ಪಾಲಿನ ದೇವತೆಗಳು. ನಾವು ಪರಸ್ಪರ ಅಸೂಯೆ ತಾಳುವುದರ ಬದಲು, ಈ ದೇಶಬಾಂಧವರನ್ನು ಪೂಜಿಸಬೇಕು.”

ಕೊನೆಯದಾಗಿ ಸ್ವಾಮೀಜಿ, ಮದ್ರಾಸಿನಲ್ಲಿ ತಮ್ಮ ಉದ್ದೇಶಿತ ಕಾರ್ಯಗಳ ಬಗ್ಗೆ ಸಂಕ್ಷೇಪ ವಾಗಿ ತಿಳಿಸಿದರು. ಇದಕ್ಕೆ ಪೂರ್ವಭಾವಿಯಾಗಿ ಅವರು ವಿದ್ಯಾಭ್ಯಾಸ ಪದ್ಧತಿಯ ಬಗ್ಗೆ ಅತ್ಯಂತ ಮಹತ್ವಪೂರ್ಣವಾದ ತಮ್ಮ ಅಭಿಪ್ರಾಯಗಳನ್ನು ಹೀಗೆ ವ್ಯಕ್ತಪಡಿಸಿದರು:

“ಭಾರತದ ಲೌಕಿಕ ಹಾಗೂ ಆಧ್ಯಾತ್ಮಿಕ ಶಿಕ್ಷಣದ ಸಂಪೂರ್ಣ ಹಿಡಿತವು ಭಾರತೀಯರ ಕೈಯಲ್ಲಿರಬೇಕು. ಇದು ನಿಮಗೆ ಅರ್ಥವಾಗುತ್ತದೆಯೆ? ಅಲ್ಲಿಯವರೆಗೆ ನಮಗೆ ಉದ್ಧಾರವಿಲ್ಲ. ಈಗ ನಮಗೆ ದೊರಕುತ್ತಿರುವ ವಿದ್ಯಾಭ್ಯಾಸದಲ್ಲಿ ಕೆಲವು ಒಳ್ಳೆಯ ಅಂಶಗಳಿವೆ, ನಿಜ. ಆದರೆ ಅದರಲ್ಲಿರುವ ಭಯಂಕರ ದೋಷಗಳ ಭಾರ ಎಷ್ಟು ಅಧಿಕವಾಗಿದೆಯೆಂದರೆ ಅದರ ಉತ್ತಮ ಅಂಶಗಳೆಲ್ಲ ಮುಳುಗಿ ಹೋಗುತ್ತವೆ. ಅದು ವ್ಯಕ್ತಿನಿರ್ಮಾಣಕಾರಿಯಾದ ವಿದ್ಯಾಭ್ಯಾಸವಲ್ಲ. ಅದು ಸಂಪೂರ್ಣ ನಕಾರಾತ್ಮಕವಾದ ವಿದ್ಯಾಭ್ಯಾಸ ಪದ್ಧತಿ. ಇಂಥ ವಿದ್ಯಾಭ್ಯಾಸವನ್ನು ಪಡೆದು, ಕಳೆದ ಐವತ್ತು ವರ್ಷಗಳ ಅವಧಿಯಲ್ಲಿ ಮದ್ರಾಸು-ಮುಂಬಯಿ-ಬಂಗಾಳ ಪ್ರಾಂತ್ಯಗಳಲ್ಲಿ ಸ್ವಂತ ಪ್ರತಿಭೆಯುಳ್ಳ ಒಬ್ಬನೇ ಒಬ್ಬ ವ್ಯಕ್ತಿಯೂ ನಿರ್ಮಾಣಗೊಂಡಿಲ್ಲ... ವಿದ್ಯೆಯೆಂದರೆ ಅದು ನಿಮ್ಮ ತಲೆಯೊಳಕ್ಕೆ ತುರುಕಲ್ಪಟ್ಟ ಮಾಹಿತಿಯ ಸಂಗ್ರಹವಲ್ಲ... ಜೀವನವನ್ನು ರೂಪಿಸಬಲ್ಲ, ವ್ಯಕ್ತಿನಿರ್ಮಾಣಕಾರಿಯಾದ, ಪುರುಷ ನಿರ್ಮಾಣಕಾರಿಯಾದ ಭಾವನೆಗಳನ್ನು ನಾವು ಜೀರ್ಣಿಸಿಕೊಳ್ಳಬೇಕು.”

ಬಳಿಕ ಸ್ವಾಮೀಜಿ, ತಮ್ಮ ಕಾರ್ಯಯೋಜನೆಯ ಪ್ರಥಮ ಹೆಜ್ಜೆಯೆಂದರೆ, ಸಮಸ್ತ ಹಿಂದೂ ಗಳೂ ಜಾತಿ-ಮತಗಳ ಭೇದವನ್ನು ತೊರೆದು ಒಟ್ಟಾಗಿ ಸೇರಲು ಸಾಧ್ಯವಿರುವಂತಹ ದೇವಾಲಯ ವೊಂದರ ನಿರ್ಮಾಣ ಎಂದರು. ಅದು ಹಿಂದೂಗಳೆಲ್ಲರಿಗೂ ಪರಮ ಪವಿತ್ರವಾದ ಓಂಕಾರದ ದೇವಾಲಯ. ಧರ್ಮವೇ ಹಿಂದೂಗಳ ಪ್ರಥಮ ಪುರುಷಾರ್ಥವಾದ್ದರಿಂದ ಈ ದೇವಾಲಯವು ಇತರೆಲ್ಲ ಚಟುವಟಿಕೆಗಳ ಕೇಂದ್ರಸ್ಥಾನವಾಗಿರುತ್ತದೆ ಎಂದು ಸ್ವಾಮೀಜಿ ತಿಳಿಸಿದರು. ಈ ದೇವಾಲಯಕ್ಕೆ ಸಂಬಂಧಿಸಿದಂತೆ, ಲೌಕಿಕ ಹಾಗೂ ಆಧ್ಯಾತ್ಮಿಕ ಶಿಕ್ಷಣಗಳೆರಡನ್ನೂ ನೀಡಬಲ್ಲ ಬೋಧಕರ ತರಬೇತಿಗಾಗಿ ಒಂದು ಸಂಸ್ಥೆಯನ್ನು ನಿರ್ಮಿಸುವುದು ಎರಡನೆಯ ಹೆಜ್ಜೆ. ಇಂತಹ ಸಂಸ್ಥೆಗಳನ್ನು ಕ್ರಮೇಣ ಭಾರತದಾದ್ಯಂತ ಸ್ಥಾಪಿಸುವುದು ತಮ್ಮ ಉದ್ದೇಶವೆಂದು ಸ್ವಾಮೀಜಿ ತಿಳಿಸಿದರು. ಆದರೆ ಇದನ್ನೆಲ್ಲ ಮಾಡುವುದು ಆಡಿದಷ್ಟು ಸುಲಭವೇ? ಈ ಪ್ರಶ್ನೆಗೆ ಸ್ವಾಮೀಜಿ ಉತ್ತರಿಸುತ್ತಾರೆ: “ಇದೊಂದು ಭಾರೀ ಯೋಜನೆಯಾಗಿ ಕಾಣಬಹುದು. ಆದರಿದು ಆಗಲೇ ಬೇಕು. ಇದಕ್ಕೆಲ್ಲ ಹಣ ಎಲ್ಲಿದೆ ಎಂದು ನೀವು ಕೇಳಬಹುದು. ಆದರೆ ಇದಕ್ಕೆ ಹಣವೇ ಪ್ರಧಾನ ವಲ್ಲ. ನನಗೆ ಬೇಕಾದ ಹಣ ಮುಂತಾದುವೆಲ್ಲ ನನಗೆ ದೊರೆಯಲೇಬೇಕು, ಬಂದೇ ಬರುತ್ತವೆ. ಆದರೆ ಜನ ಎಲ್ಲಿದ್ದಾರೆ? ಅದೇ ಈಗ ನಮ್ಮ ಮುಂದಿರುವ ಪ್ರಶ್ನೆ... ಮದ್ರಾಸಿನ ಯುವಜನರೇ, ನನಗೆ ಭರವಸೆಯಿರುವುದು ನಿಮ್ಮಲ್ಲಿ.”

ಭಾರತದ ಯುವಜನತೆಗೆ ಸ್ವಾಮೀಜಿ ಅತ್ಯಂತ ಕಳಕಳಿಯ ಮನವಿ ಮಾಡಿಕೊಳ್ಳುತ್ತಾರೆ: “ನೀವು ನಮ್ಮ ರಾಷ್ಟ್ರದ ಈ ಕರೆಗೆ ಓಗೊಡುವಿರಾ? ನೀವು ಧೈರ್ಯ ಮಾಡಿ ನನ್ನನ್ನು ನಂಬಿದ್ದೇ ಆದರೆ ನಿಮ್ಮಲ್ಲಿ ಒಬ್ಬೊಬ್ಬರಿಗೂ ಭವ್ಯವಾದ ಭವಿಷ್ಯವಿದೆ. ನನ್ನ ತಾರುಣ್ಯದಲ್ಲಿ ನನಗಿದ್ದಂತಹ ಪ್ರಚಂಡವಾದ ಆತ್ಮವಿಶ್ವಾಸ ನಿಮಗೆ ನಿಮ್ಮಲ್ಲಿರಲಿ... ಮನಸ್ಸು ಮಾಡಿದರೆ ನೀವು ಇಡೀ ಭರತಖಂಡದ ಪುನರುತ್ಥಾನವನ್ನು ಸಾಧಿಸಬಲ್ಲಿರಿ. ಅಷ್ಟೇ ಅಲ್ಲ, ಈ ಜಗತ್ತಿನ ಪ್ರತಿಯೊಂದು ಜನಾಂಗದ ಜೀವನದಲ್ಲೂ ನಮ್ಮ ಅದ್ಭುತ ಭಾವನೆಗಳು ಪ್ರವೇಶ ಮಾಡಬೇಕು. ಈ ಕೆಲಸಕ್ಕೆ ಯುವಕರು ಬೇಕು. ‘ಶಕ್ತಿವಂತರೂ ಧೀಮಂತರೂ ದೃಢಕಾಯರೂ ಚತುರರೂ ಆದ ಯುವಕರು ಮಾತ್ರವೇ ದೇವರನ್ನು ಕಾಣಬಲ್ಲರು’ ಎನ್ನುತ್ತವೆ ವೇದಗಳು. ನಿಮ್ಮ ಭವಿಷ್ಯತ್ತನ್ನು ನಿರ್ಧರಿಸುವ ಕಾಲ ಇದು. ಯೌವನದ ಶಕ್ತಿ ಸಾಮರ್ಥ್ಯಗಳಿದ್ದಾಗಲೇ ನಿರ್ಧರಿಸಬೇಕು; ಮುದಿಯಾಗಿ ದೇಹ ಸವೆದುಹೋದಮೇಲಲ್ಲ... ಏಕೆಂದರೆ ಮುಟ್ಟದ, ಮೂಸದ, ಮಾಸದ, ಮೈಲಿಗೆಯಾಗದ, ತನಿಯಾದ ಕುಸುಮಗಳನ್ನು ಮಾತ್ರವೇ ಭಗವಂತನ ಪಾದಗಳಿಗೆ ಅರ್ಪಿಸಬೇಕು. ಅಂತಹ ನವಕುಸುಮಗಳನ್ನು ಮಾತ್ರವೇ ಅವನು ಸ್ವೀಕರಿಸುವುದು. ಆದ್ದರಿಂದ, ಯುವಕರೇ, ಏಳಿ! ಎದ್ದೇಳಿ! ಈ ಜೀವನವು ಕ್ಷಣಿಕ. ಆದರೆ ಆತ್ಮವು ಅಮರ ಮತ್ತು ನಿತ್ಯ. ಹೇಗಿದ್ದರೂ ಈ ಸಾವು ಎಂಬುದೊಂದು ಇದ್ದೇ ಇದೆ; ಅದಂತೂ ನಿಶ್ಚಿತ. ಆದ್ದರಿಂದ ಒಂದು ಶ್ರೇಷ್ಠ-ಉನ್ನತ ವಿಚಾರ ವನ್ನು ತೆಗೆದುಕೊಳ್ಳೋಣ ಮತ್ತು ಅದಕ್ಕಾಗಿ ನಮ್ಮ ಜೀವನವನ್ನೇ ಅರ್ಪಿಸೋಣ... ಭಗವಾನ್ ಶ್ರೀಕೃಷ್ಣನು ನಮ್ಮನ್ನು ಹರಸಲಿ, ನಮ್ಮ ಗುರಿಯನ್ನು ಸಿದ್ಧಿಸಿಕೊಳ್ಳಲು ನಮಗೆ ದಾರಿತೋರಲಿ.”

ಸ್ವಾಮೀಜಿಯವರು ತಮ್ಮ ಈ ಭಾಷಣವನ್ನು ಮುಕ್ತಾಯಗೊಳಿಸುತ್ತಿದ್ದಂತೆಯೇ, ಅಲ್ಲಿ ನೆರೆದಿದ್ದ ಸಹಸ್ರಾರು ಜನ ದೀರ್ಘ ಕರತಾಡನ ಮಾಡಿದರು. ಇದೊಂದು ಅತ್ಯಂತ ಪ್ರಭಾವ ಪೂರ್ಣವಾದ, ಸ್ಫೂರ್ತಿಯುತವಾದ ಭಾಷಣ. ಅಲ್ಲಿ ನೆರೆದಿದ್ದ ಸಹಸ್ರಾರು ಶ್ರೋತೃಗಳ ಮೇಲೆ ಅದು ಉಂಟುಮಾಡಿದ ಪರಿಣಾಮ ವರ್ಣಿಸಲಸಾಧ್ಯ. ಅಲ್ಲದೆ ಅಂದು ಸ್ವಾಮೀಜಿ ಉನ್ನತ ಭಾವಾವಸ್ಥೆಗೇರಿದ್ದರು. ಅವರಲ್ಲಿ ವಿಶೇಷ ಶಕ್ತಿಯ ಆವಿರ್ಭಾವವಾದಂತಿತ್ತು. ಅದಕ್ಕೆ ತಕ್ಕಂತೆ ಅತ್ಯಂತ ಉತ್ಸಾಹದಿಂದ ಕುಳಿತು ಆಲಿಸಿದ ಜನಸ್ತೋಮವು ಬಾರಿಬಾರಿಗೂ ಕರತಾಡನದ ಮೂಲಕ ತನ್ನ ಆನಂದವನ್ನು ವ್ಯಕ್ತಪಡಿಸಿತು. ಅಂದಿನ ಸಭೆಯ ವೈಶಿಷ್ಟ್ಯದ ಬಗ್ಗೆ ಪ್ರೊ ॥ ಸುಂದರ ರಾಮ ಅಯ್ಯರ್ ಹೇಳುತ್ತಾರೆ, “ಅಂದು ಕಿಕ್ಕಿರಿದು ನೆರೆದಿದ್ದಂತಹ ಜನಸ್ತೋಮವನ್ನು, ಅಷ್ಟು ಉತ್ಸಾಹಭರಿತವಾದ ಸಭೆಯನ್ನು ನಾನೆಂದೂ ಕಂಡಿರಲಿಲ್ಲ! ಅಂತೆಯೇ ಸ್ವಾಮೀಜಿಯವರ ವಾಗ್ಮಿತೆಯೂ ಅಂದು ಅತ್ಯುನ್ನತ ಮಟ್ಟದ್ದಾಗಿತ್ತು. ಅವರು ಮೃಗರಾಜನಂತೆ ವೇದಿಕೆಯ ಮೇಲೆ ಅತ್ತಿಂದಿತ್ತ ನಡೆದಾಡುತ್ತ ಮಾತನಾಡುತ್ತಿದ್ದರು. ಅವರ ಕಂಠದಿಂದ ಹೊರಡುತ್ತಿದ್ದ ಸಿಂಹ ಗರ್ಜನೆ ಎಲ್ಲೆಲ್ಲೂ ಅನುರಣಿತವಾಗುತ್ತಿತ್ತು; ಅದು ಅತ್ಯಂತ ಪರಿಣಾಮಕಾರಿಯಾಗಿತ್ತು.” ಅಂದಿನ ಉಪನ್ಯಾಸವು ನಿಜಕ್ಕೂ ಅತ್ಯಂತ ವೈಶಿಷ್ಟ್ಯಪೂರ್ಣವಾಗಿತ್ತೆಂಬುದು ಶ್ರೀ ಸಿ. ರಾಮಾನುಜಾಚಾರಿ ಎಂಬುವರ ಈ ಮಾತಿನಿಂದ ಮತ್ತಷ್ಟು ಸ್ಪಷ್ಟವಾಗುತ್ತದೆ:“ಅದೊಂದು ಅತ್ಯಂತ ಅದ್ಭುತವಾದ ಉಪನ್ಯಾಸ. ಸ್ವಾಮೀಜಿಯವರ ಧ್ವನಿಯು ಮೈದಾನದ ಮೂಲೆಮೂಲೆಗೂ ಕೇಳಿಸುವಂತಿತ್ತು; ಅದು ತುಂಬ ಸ್ಫುಟವಾಗಿತ್ತು. ಅವು ಧ್ವನಿವರ್ಧಕ ಯಂತ್ರಗಳಿಲ್ಲದಿದ್ದ ದಿನಗಳು. ಅವರ ಮಾತುಗಳ ಪರಿಣಾಮವು ಅತ್ಯಂತ ರೋಮಾಂಚಕಾರಿಯಾಗಿತ್ತು.”

ಹೀಗೆ ಸ್ವಾಮೀಜಿ ಮದ್ರಾಸಿನಲ್ಲಿ ತಮ್ಮ ಕಾರ್ಯಕ್ರಮವನ್ನು ಮುಕ್ತಾಯಗೊಳಿಸಿದರು. ಮರುದಿನವೇ ಅವರು ಮದ್ರಾಸಿನಿಂದ ಕಲ್ಕತ್ತಕ್ಕೆ ಹೊರಡಲಿದ್ದರು. ಈ ಕೆಲದಿನಗಳಲ್ಲಿ ಸ್ವಾಮೀಜಿ ಯವರ ಹಾಗೂ ಮದ್ರಾಸಿನ ನಾಗರಿಕರ ನಡುವೆ ಉಂಟಾಗಿದ್ದ ಬಾಂಧವ್ಯ ಅತ್ಯಂತ ನಿಕಟ ವಾದದ್ದು, ಗಾಢವಾದದ್ದು. ಸ್ವಾಮೀಜಿ ಮದ್ರಾಸಿಗೆ ಬಂದಾಗಿನಿಂದಲೂ ಅವರ ಶಿಷ್ಯರು-ಅಭಿ ಮಾನಿಗಳು ಅವರನ್ನು ಮದ್ರಾಸಿನಲ್ಲೇ ಉಳಿದುಕೊಂಡು ಒಂದು ಮಠವನ್ನು ಸ್ಥಾಪಿಸುವಂತೆ ಒತ್ತಾಯಿಸುತ್ತಿದ್ದರು. ಆದರೆ ಸ್ವಾಮೀಜಿಯವರ ದೇಹಾರೋಗ್ಯ ಸಂಪೂರ್ಣ ಹದಗೆಟ್ಟು ಹೋಗು ತ್ತಿತ್ತು. ಇದನ್ನೂ ಅವರ ಶಿಷ್ಯರು ಗಮನಿಸಲಿಲ್ಲವೆಂದಲ್ಲ. ಅಮೆರಿಕೆಗೆ ತೆರಳುವ ಮುನ್ನ ಮದ್ರಾಸಿಗೆ ಬಂದಿದ್ದ ವಜ್ರಕಾಯದ ಆ ಪರಿವ್ರಾಜಕ ಸಂನ್ಯಾಸಿಯ ಚಿತ್ರವನ್ನು ಕಂಡವರು, ಧರ್ಮಕ್ಕಾಗಿ–ಮಾನವತೆಗಾಗಿ ಸ್ವಾಮೀಜಿ ಹೇಗೆ ತಮ್ಮ ಶಕ್ತಿಯನ್ನೆಲ್ಲ ಧಾರೆಯೆರೆದಿದ್ದಾರೆಂಬು ದನ್ನು ಮನಗಂಡು ಸಂಕಟಪಟ್ಟರು. ಆದ್ದರಿಂದ ಸ್ವಾಮೀಜಿಯವರಿಗೆ ವಿಶ್ರಾಂತಿ ಅತ್ಯಗತ್ಯ ವಾಗಿದೆಯೆಂಬುದು ಅವರಿಗೇ ಗೊತ್ತಾಯಿತು. ಸ್ವಾಮೀಜಿ ಮದ್ರಾಸಿನಲ್ಲಿ ಹೆಚ್ಚು ದಿನ ಉಳಿದು ಕೊಳ್ಳಬೇಕೆಂದು ಒತ್ತಾಯಿಸುತ್ತಿದ್ದವರನ್ನು ಇತರರು ಸಮಾಧಾನ ಪಡಿಸಿದರು. ಕಡೆಗೆ ಸ್ವಾಮೀಜಿ ಮದ್ರಾಸಿನಲ್ಲಿ ಹತ್ತು ದಿನ ಕಳೆದು ಕಲ್ಕತ್ತಕ್ಕೆ ಹೊರಡಬೇಕೆಂದು ನಿಶ್ಚಯಿಸಲಾಯಿತು. ಅಲ್ಲದೆ ಸಾಧ್ಯವಾದಷ್ಟು ಬೇಗ ತಮ್ಮ ನೆಚ್ಚಿನ ಹಿಮಾಲಯಕ್ಕೆ ತೆರಳಿ ಕೆಲಕಾಲ ವಿಶ್ರಮಿಸಲು ಅವರು ಆಶಿಸಿದ್ದರು. ಶ್ರೀಮತಿ ಸಾರಾ ಬುಲ್ಲಳಿಗೆ ಬರೆದ ಒಂದು ಪತ್ರದಲ್ಲಿ ಸ್ವಾಮೀಜಿ ತಿಳಿಸಿದ್ದರು, “ನಾನು ಕೂಡಲೇ ಸ್ವಲ್ಪ ವಿಶ್ರಾಂತಿ ತೆಗೆದುಕೊಳ್ಳದಿದ್ದರೆ ಇನ್ನೊಂದು ಆರು ತಿಂಗಳೂ ಬದುಕುಳಿಯುವ ಸಂಭವ ಇರಲಾರದು” ಎಂದು.

ಸ್ವಾಮೀಜಿ ಮದ್ರಾಸಿನಿಂದ ಹೊರಡುವುದಾದರೂ, ಇಲ್ಲೊಂದು ಮಠದ ಸ್ಥಾಪನೆಗೆ ಏರ್ಪಾಡು ಮಾಡಬೇಕೆಂದು ಅವರ ಶಿಷ್ಯರೆಲ್ಲ ಒತ್ತಾಯಿಸುತ್ತಿದ್ದರು. ಭಾನುವಾರದಂದು ಸ್ವಾಮೀಜಿ ಸಭೆ ಯನ್ನು ಮುಗಿಸಿಕೊಂಡು ಹಿಂದಿರುಗಿದಾಗ ಈ ವಿಷಯ ಮತ್ತೆ ಪ್ರಸ್ತಾಪಕ್ಕೆ ಬಂದಿತು. ಎಸ್. ಸುಬ್ರಮಣ್ಯ ಅಯ್ಯರ್ ಮತ್ತಿತರರು ಮುಂದೆ ಆಗಬೇಕಾಗಿರುವ ಕಾರ್ಯವನ್ನು ಕುರಿತು ಸ್ವಾಮೀಜಿಯವರೊಂದಿಗೆ ಸಮಾಲೋಚನೆ ನಡೆಸಿದರು. ಆಗ ಸ್ವಾಮೀಜಿ ತಮ್ಮ ಪ್ರತಿನಿಧಿಯಾಗಿ ಸೋದರ ಸಂನ್ಯಾಸಿಯೊಬ್ಬರನ್ನು ಮದ್ರಾಸಿಗೆ ಖಂಡಿತವಾಗಿಯೂ ಕಳಿಸಿಕೊಡುವುದಾಗಿ ಭರವಸೆ ನೀಡಿದರು. ಬಳಿಕ ಸ್ವಲ್ಪ ತಮಾಷೆಯಾಗಿಯೇ ಹೇಳಿದರು, “ನಾನು ನಿಮ್ಮಲ್ಲಿಗೆ, ಧೂಮಪಾನ ಮಾಡದಿರುವ ಹಾಗೂ ನಿಮ್ಮಲ್ಲಿನ ಅತ್ಯಂತ ಆಚಾರವಂತರಿಗಿಂತಲೂ ಹೆಚ್ಚು ಆಚಾರವಂತರಾದ ಸ್ವಾಮಿಗಳೊಬ್ಬರನ್ನು ಕಳಿಸಿಕೊಡುತ್ತೇನೆ.” ಹೀಗೆ ಹೇಳುವಾಗ ಅವರ ಮನಸ್ಸಿನಲ್ಲಿದ್ದುದು ಸ್ವಾಮಿ ರಾಮಕೃಷ್ಣಾನಂದರು. ಮತ್ತು ಅದರಂತೆಯೇ ಅವರು ಮುಂದಿನ ತಿಂಗಳಲ್ಲೇ ರಾಮಕೃಷ್ಣಾನಂದ ರನ್ನು ಕಳಿಸಿಕೊಟ್ಟರು. ಇಲ್ಲಿ ಸ್ವಾಮೀಜಿ, ಧೂಮಪಾನ ಮಾಡದಿರುವ ಹಾಗೂ ಅತ್ಯಂತ ಆಚಾರವಂತ ಸ್ವಾಮಿಗಳೊಬ್ಬರನ್ನು ಕಳಿಸಿಕೊಡುವುದಾಗಿ ಹೇಳಿದ್ದರಲ್ಲಿ ಒಂದು ಸ್ವಾರಸ್ಯವಿದೆ. ದಕ್ಷಿಣ ಭಾರತೀಯರು–ಅದರಲ್ಲೂ ಮದ್ರಾಸಿನ ಬ್ರಾಹ್ಮಣರು–ತೀರಾ ಆಚಾರವಂತರು. ಮಡಿ-ಮೈಲಿಗೆ, ವಿಧಿ ನಿಯಮಗಳು ಉತ್ತರಭಾರತದಲ್ಲಿ ಇಲ್ಲಿನಷ್ಟು ಕಟ್ಟುನಿಟ್ಟಾಗಿಲ್ಲ. ಅಲ್ಲಿ ಸಾಧು ಸಂನ್ಯಾಸಿಗಳೂ ಬೀಡಿ, ಸಿಗರೇಟು ವಗೈರೆ ಸೇದುವುದುಂಟು. ಅಷ್ಟೇಕೆ ಸ್ವತಃ ವಿವೇಕಾ ನಂದರೇ ಸಿಗರೇಟು ಹೊತ್ತಿಸಿದ್ದನ್ನು ಆ ಮದ್ರಾಸು ಭಕ್ತರು ನೋಡಿರಲಿಲ್ಲವೆ? ದಕ್ಷಿಣ ಭಾರತ ದಲ್ಲಿ ಸಿಗರೇಟು ಸೇದುವ ಸಂನ್ಯಾಸಿಗಳಿಗೆ ಪೂಜ್ಯತೆ ಸಿಗುವ ಸಂಭವವುಂಟೆ? ಅಷ್ಟೇ ಅಲ್ಲ; ಸಂಪ್ರದಾಯನಿಷ್ಠರಿಂದ ತುಂಬಿರುವ ಮದ್ರಾಸಿನಂತಹ ನಗರದಲ್ಲಿ ಸ್ವಾಮಿಗಳೊಬ್ಬರು ಬಂದು ಎಲ್ಲರಿಂದಲೂ ಸೈ ಎನ್ನಿಸಿಕೊಳ್ಳಬೇಕಾದರೆ ಮಡಿ-ಮೈಲಿಗೆ, ವಿಧಿ-ನಿಷೇಧಗಳನ್ನೆಲ್ಲ ಪರಿಪಾಲಿಸ ಬಲ್ಲ ಸಂಪ್ರದಾಯನಿಷ್ಠರಾಗಿರಬೇಕು, ಸಂಪ್ರದಾಯ ನಿಷ್ಠುರರೂ ಆಗಿರಬೇಕು. ಆದ್ದರಿಂದ ಈ ಎಲ್ಲ ಗುಣಗಳನ್ನೂ ಹೊಂದಿರುವ ಸ್ವಾಮಿ ರಾಮಕೃಷ್ಣಾನಂದರನ್ನೇ ಸ್ವಾಮೀಜಿ ಆರಿಸಿದರು.

ಸ್ವಾಮೀಜಿ ಮದ್ರಾಸಿನಲ್ಲಿದ್ದ ಈ ಸಂದರ್ಭದಲ್ಲಿ ಅವರಿಗೆ ಅಮೆರಿಕದ ಹಾಗೂ ಇಂಗ್ಲೆಂಡಿನ ವಿಶ್ವಾಸಿಗರಿಂದ ಪತ್ರಗಳು ಬರುತ್ತಿದ್ದುವು. ಅಲ್ಲೆಲ್ಲ ಅವರು ಮಾಡಿದ ಕಾರ್ಯ ಎಂತಹ ಪರಿಣಾಮವನ್ನುಂಟುಮಾಡುತ್ತಿದೆ ಎಂಬುದು ಈ ಪತ್ರಗಳಲ್ಲಿ ವರ್ಣಿತವಾಗಿತ್ತು. ಇನ್ನು ಕೆಲವು ಸಂಸ್ಥೆಗಳವರು ಸ್ವಾಮೀಜಿಯವರ ಮಹತ್​ಕಾರ್ಯವನ್ನು ಶ್ಲಾಘಿಸಿ ಅಭಿನಂದನಾ ಪತ್ರಗಳನ್ನು ಕಳಿಸಿದರು. “ಕೇಂಬ್ರಿಡ್ಜ್ ಕಾನ್ಫರೆನ್ಸಸ್​” ಎಂಬ ಸಂಸ್ಥೆಯವರು ಬರೆದ ಪತ್ರ ಹೀಗಿತ್ತು:

ಪೂಜ್ಯ ಸ್ವಾಮೀ ವಿವೇಕಾನಂದರಿಗೆ,

ನೀವು ಅಮೇರಿಕೆಯಲ್ಲಿ ತತ್ತ್ವಶಾಸ್ತ್ರ ಹಾಗೂ ವೇದಾಂತಧರ್ಮದ ಮೇಲೆ ನೀಡಿದ ಸಮರ್ಥ ವ್ಯಾಖ್ಯಾನಗಳ ಮಹತ್ವವನ್ನು ಗುರುತಿಸಲು ಕೇಂಬ್ರಿಡ್ಜ್ ಕಾನ್ಫರೆನ್ಸಸ್ ಸಂಸ್ಥೆಯ ಸದಸ್ಯರಾದ ನಮಗೆ ತುಂಬ ಹರ್ಷವಾಗುತ್ತದೆ. ನಿಮ್ಮ ವ್ಯಾಖ್ಯಾನಗಳು ನಮ್ಮಲ್ಲಿನ ಬುದ್ಧಿ ಜೀವಿಗಳಲ್ಲಿ ತೀವ್ರ ಆಸಕ್ತಿಯನ್ನು ಕೆರಳಿಸಿವೆ. ನೀವು ನೀಡಿರುವ ವ್ಯಾಖ್ಯಾನಗಳು ಕೇವಲ ವೈಚಾರಿಕ ವಿಮರ್ಶೆಗಳಾಗಿ ರದೆ, ಅವು ದೂರದೂರದ ರಾಷ್ಟ್ರಗಳ ನಡುವೆ ಭ್ರಾತೃತ್ವ ಹಾಗೂ ಸ್ನೇಹವನ್ನು ಬೆಸೆಯಬಲ್ಲ ನೈತಿಕ ಬಲವನ್ನೊಳಗೊಂಡಿವೆ.

ನೀವೀಗ ಭಾರತದಲ್ಲಿ ಕೈಗೊಂಡಿರುವ ಕಾರ್ಯವು ಈ ನಿಮ್ಮ ಉದಾತ್ತ ಧ್ಯೇಯವನ್ನು ಈಡೇರಿ ಸುವಂತಾಗಲು ಭಗವಂತ ಕೃಪೆಮಾಡಲೆಂದು ನಾವೆಲ್ಲ ಹೃತ್ಪೂರ್ವಕವಾಗಿ ಹಾರೈಸುತ್ತೇವೆ. ಮತ್ತು ಆ ದೂರದ ಭಾರತೀಯ ಜನಾಂಗದ ಶುಭಾಶಯಗಳನ್ನು ಹೊತ್ತು ನೀವು ಮತ್ತೊಮ್ಮೆ ನಮ್ಮಲ್ಲಿಗೆ ಬರುವುದನ್ನೇ ನಿರೀಕ್ಷಿಸುತ್ತಿರುತ್ತೇವೆ.

ನಮ್ಮ ಕೇಂಬ್ರಿಡ್ಜ್ ಕಾನ್ಫರೆನ್ಸಸ್ಸಿನ ಕಾರ್ಯಕಲಾಪಗಳನ್ನು ಇನ್ನಷ್ಟು ಪರಿಣಾಮಕಾರಿಯಾಗಿ ಮಾಡಬೇಕಾಗಿದೆ. ಆದ್ದರಿಂದ ಮುಂದಿನ ವರ್ಷದ ನಿಮ್ಮ ಕಾರ್ಯ ಯೋಜನೆ ಏನೆಂಬುದನ್ನೂ ನೀವು ಇನ್ನೊಮ್ಮೆ ನಮ್ಮಲ್ಲಿಗೆ ಗುರುವಾಗಿ ಹಿಂದಿರುಗಿ ಬರುವಿರೆಂದು ನಾವು ನಿರೀಕ್ಷಿಸಬಹುದೇ ಎಂಬುದನ್ನೂ ತಿಳಿಯಲು ನಾವು ಕುತೂಹಲಿಗಳಾಗಿದ್ದೇವೆ. ನೀವು ಮತ್ತೊಮ್ಮೆ ನಮ್ಮಲ್ಲಿಗೆ ಬರಲು ಸಾಧ್ಯವಾದೀತೆಂಬುದು ನಮ್ಮೆಲ್ಲರ ಹಾರೈಕೆ. ನೀವು ಇಲ್ಲಿಗೆ ಬಂದಾಗ ನಾವೆಲ್ಲ ಹಿಂದಿನಂತೆಯೇ ಹೃತ್ಪೂರ್ವಕ ಆಸಕ್ತಿ ತೋರುವ ಭರವಸೆ ಕೊಡುತ್ತೇವೆ.

\begin{flushright}
ಇತಿ, ತಮ್ಮ ಪರಮ ವಿಶ್ವಾಸಿಗಳಾದ
\end{flushright}

ಲೂಯಿಸ್ ಜಿ. ಜೇನ್ಸ್, ಡೈರೆಕ್ಟರ್\\
 ಸಿ. ಸಿ ಎವರೆಟ್\\
 ವಿಲಿಯಮ್ ಜೇಮ್ಸ್ \\
 ಜಾನ್ ಹೆಚ್. ರೈಟ್\\
 ಜೊಸೈಯ ರಾಯ್

\begin{flushright}
ಜೆ. ಇ. ಲೋಫ್​\\
 ಎ. ಒ. ಲವ್ ಜಾಯ್​\\
 ರಾಖೇಲ್ ಕೆಂಟ್ ಟೈಲರ್​\\
 ಸಾರಾ ಸಿ. ಬುಲ್​\\
 ಜಾನ್ ಪಿ. ಫಾಕ್ಸ್
\end{flushright}

ಈ ಪತ್ರಕ್ಕೆ ಸಹಿ ಹಾಕಿದ್ದ ವ್ಯಕ್ತಿಗಳಲ್ಲಿ ಒಬ್ಬೊಬ್ಬರೂ ಸುವಿಖ್ಯಾತರು, ಗಣ್ಯರು. ಇವರ ಪೈಕಿ ಸಿ. ಸಿ. ಎವರೆಟ್, ವಿಲಿಯಮ್ ಜೇಮ್ಸ್, ಜಾನ್ ಹೆಚ್. ರೈಟ್,ಜಿ. ಇ. ಲೋಫ್, ಜೊಸೈಯ ರಾಯ್ಸ್ ಹಾಗೂ ಟೈಲರ್​–ಇವರುಗಳು ಹಾರ್ವರ್ಡ್ ವಿಶ್ವವಿದ್ಯಾಲಯದ ತತ್ತ್ವಶಾಸ್ತ್ರ ಮತ್ತು ಮನಃಶಾಸ್ತ್ರ ವಿಭಾಗಗಳಲ್ಲಿ ಪ್ರೊಫೆಸರರಾಗಿದ್ದರಲ್ಲದೆ, ಇತರ ಹಲವಾರು ಪ್ರಮುಖ ಸಂಘ ಸಂಸ್ಥೆಗಳ ಅಧ್ಯಕ್ಷರಾಗಿದ್ದರು. ಶ್ರೀಮತಿ ಸಾರಾ ಬುಲ್ ಕೇಂಬ್ರಿಡ್ಜ್ ಕಾನ್ಫರೆನ್ಸಸ್ಸಿನ ಪ್ರೋತ್ಸಾಹಕಿ, ಅಮೆರಿಕ-ನಾರ್ವೆ ದೇಶಗಳಲ್ಲಿ ಹೆಸರಾಂತ ಮಹಿಳೆ, ಸ್ವಾಮೀಜಿಯವರ ಆಪ್ತ ಶಿಷ್ಯೆ.

ಅಮೆರಿಕದಲ್ಲಿದ್ದಾಗ ಸ್ವಾಮೀಜಿ ಭೇಟಿ ನೀಡಿ ಕಾರ್ಯಕ್ರಮಗಳನ್ನು ನಡೆಸಿಕೊಟ್ಟಿದ್ದ ಮತ್ತೊಂದು ಸಂಸ್ಥೆಯೆಂದರೆ “ಬ್ರೂಕ್ಲಿನ್ ಎಥಿಕಲ್ ಅಸೋಸಿಯೇಶನ್​” ಈ ಸಂಸ್ಥೆಯ ಪರ ವಾಗಿ ಸ್ವಾಮೀಜಿಯವರಿಗೆ ಬಂದ ಒಂದು ಪತ್ರವನ್ನು “ಮಹಾ ಆರ್ಯ ಕುಟುಂಬಕ್ಕೆ ಸೇರಿದ ಭಾರತೀಯ ಬಂಧುಗಳಿಗೆ” ಎಂದು ಒಕ್ಕಣಿಸಲಾಗಿತ್ತು. (ಭಾರತೀಯರು ಹಾಗೂ ಐರೋಪ್ಯರು ಮೂಲತಃ ಒಂದೇ ‘ಮಹಾ ಆರ್ಯ’ ಬುಡಕಟ್ಟಿಗೆ ಸೇರಿದವರೆಂಬುದನ್ನು ಇಲ್ಲಿ ಸ್ಮರಿಸಬಹುದು.) ಸ್ವಾಮೀಜಿಯವರ ಶಿಷ್ಯರು ಈ ಪತ್ರವನ್ನು ಅಚ್ಚು ಹಾಕಿಸಿ ಅದರ ಪ್ರತಿಗಳನ್ನು ಮದ್ರಾಸಿನ ಉತ್ಸಾಹೀ ನಾಗರಿಕರಿಗೆ ಹಂಚಿದರು.

ಫೆಬ್ರುವರಿ ೧೫ರಂದು ಸೋಮವಾರ ಸ್ವಾಮೀಜಿ ತಮ್ಮ ಸಂಗಡಿಗರೊಂದಿಗೆ ಸ್ಟೀಮರಿನಲ್ಲಿ ಕಲ್ಕತ್ತಕ್ಕೆ ಹೊರಟರು. ಅಂದು ಹಡಗುಕಟ್ಟೆಯನ್ನು ಸುಂದರವಾಗಿ ಅಲಂಕರಿಸಲಾಗಿತ್ತು. ಬಿನ್ನಿ ಕಂಪನಿಯವರು ಸ್ವಾಮೀಜಿಯವರಿಗಾಗಿ ಬೀಳ್ಕೊಡುಗೆ ಸಮಾರಂಭವೊಂದನ್ನು ಏರ್ಪಡಿಸಿದ್ದರು. ಸ್ವಾಮೀಜಿ ಹಡಗುಕಟ್ಟೆಗೆ ಬರುವುದು ಸುಮಾರು ಎಂಟು ಗಂಟೆಯ ಹೊತ್ತಿಗೆ ಎಂದು ಗೊತ್ತಿ ದ್ದರೂ ಜನ ಬೆಳಗಿನ ಜಾವ ನಾಲ್ಕು ಗಂಟೆಯಿಂದಲೇ ನೆರೆಯಲಾರಂಭಿಸಿದ್ದರು. ನಿರ್ಗಮನದ ಸಮಯ ಸನ್ನಿಹಿತವಾದಂತೆ ಅಲ್ಲಿ ಸಹಸ್ರಾರು ಜನ ಬಂದು ಸೇರಿಬಿಟ್ಟಿದ್ದರು. ಸಮಾರಂಭಕ್ಕಾಗಿ ಒಂದು ಭಾರೀ ಶಾಮಿಯಾನವನ್ನು ನಿರ್ಮಿಸಲಾಗಿತ್ತು. ಏಳೂವರೆ ಗಂಟೆಯ ಹೊತ್ತಿಗೆ ಸ್ವಾಮೀಜಿ ಹಡಗುಕಟ್ಟೆಯನ್ನು ತಲುಪಿದಾಗ ಅವರನ್ನು ಗೌರವಾದರಗಳಿಂದ ಸ್ವಾಗತಿಸಿ ಶಾಮಿಯಾನದೊಳಕ್ಕೆ ಕರೆದೊಯ್ಯಲಾಯಿತು. ಅಲ್ಲಿ ಅವರನ್ನು ಸ್ವಾಗತ ಸಮಿತಿಯ ಸದಸ್ಯರು ಹಾಗೂ ಇತರ ಗಣ್ಯ ನಾಗರಿಕರು ಭೇಟಿಯಾದರು. ಸ್ವಾಮೀಜಿಯವರು ತಮ್ಮ ತಾಯ್ನಾಡಿಗೆ ಸಲ್ಲಿಸಿದ ಮಹತ್ತರ ಸೇವೆ ಗಾಗಿ ಅವರಿಗೊಂದು ಅಭಿನಂದನಾಪತ್ರವನ್ನು ಸಮರ್ಪಿಸಲಾಯಿತು. ಆನರಬಲ್ ಸುಬ್ಬರಾವ್ ರವರು ಮಾತನಾಡಿ, ಅಲ್ಲಿದ್ದವರೆಲ್ಲರ ಪರವಾಗಿ ಸ್ವಾಮೀಜಿಗೆ ಸುಖಪ್ರಯಾಣವನ್ನು ಹಾರೈಸಿ ದರು. ಸ್ವಾಮೀಜಿ ಅದಕ್ಕೆ ಕೃತಜ್ಞತಾಪೂರ್ವಕವಾಗಿ ತಲೆಬಾಗಿ, ತಮ್ಮ ಭಾವನೆಗಳನ್ನು ಮೌನವೇ ಹೆಚ್ಚು ಚೆನ್ನಾಗಿ ವ್ಯಕ್ತಪಡಿಸಬಲ್ಲುದು ಎಂದರು. ಕಿವಿಗಡಚಿಕ್ಕುವ ಚಪ್ಪಾಳೆಗಳ ನಡುವೆ ಸ್ವಾಮೀಜಿ ಹಡಗಿನ ಕಡೆಗೆ ನಡೆದರು.

ಹಡಗು ಚಲಿಸುವವರೆಗೂ ಸ್ವಾಮೀಜಿಯವರೊಂದಿಗಿದ್ದ ಆಪ್ತರಲ್ಲಿ ಒಬ್ಬರೆಂದರೆ ಪ್ರೊ ॥ ಸುಂದರರಾಮ ಅಯ್ಯರ್. ಅವರು ತಮಗಾಗಿ ಎರಡೇ ನಿಮಿಷದ ಸಂದರ್ಶನ ನೀಡುವಂತೆ ಸ್ವಾಮೀಜಿಯವರನ್ನು ಒಪ್ಪಿಸಿದರು. ಅವರು ಕೇಳಿದ ಮೊದಲ ಪ್ರಶ್ನೆ ಇದು: “ಸ್ವಾಮೀಜಿ, ನೀವು ಅಮೆರಿಕದಂತಹ ಭೋಗವಾದೀ ರಾಷ್ಟ್ರಗಳಿಗೆ ಭೇಟಿಕೊಟ್ಟು, ಅಲ್ಲಿ ನಿಮ್ಮ ಸಂದೇಶವನ್ನು ಪ್ರಸಾರ ಮಾಡಿದ್ದರಿಂದ ನಿಜಕ್ಕೂ ಶಾಶ್ವತವಾದ ಸತ್ಪರಿಣಾಮ ಆಗಿದೆಯೇ?” ಈ ಪ್ರಶ್ನೆಗೆ ಸ್ವಾಮೀಜಿ ಅತ್ಯಂತ ವಿನಮ್ರಭಾವದಿಂದ, ಆದರೆ ಮುಚ್ಚುಮರೆಯಿಲ್ಲದೆ ಉತ್ತರಿಸಿದರು, “ಬಹುಶಃ ಅಷ್ಟೇನೂ ಹೆಚ್ಚಿನ ಪರಿಣಾಮವಾಗಿಲ್ಲವೆನ್ನಬಹುದು. ಆದರೆ ನಾನು ಅಲ್ಲಲ್ಲಿ ಬೀಜವನ್ನಂತೂ ಬಿತ್ತಿದ್ದೇನೆ. ಅದು ಮುಂದೆ ಬೆಳೆದು ಮರವಾದರೆ ಕೆಲವರಿಗಾದರೂ ಅದರಿಂದ ಪ್ರಯೋಜನವಾಗ ಬಹುದು.” ಅಯ್ಯರರು ಕೇಳಿದ ಎರಡನೆಯ ಪ್ರಶ್ನೆ: “ಸ್ವಾಮೀಜಿ, ನೀವು ಇಲ್ಲಿಗೆ ಪುನಃ ಹಿಂದಿರುಗಿ ಬರುತ್ತೀರಾ? ಮತ್ತೊಮ್ಮೆ ಬಂದು ದಕ್ಷಿಣ ಭಾರತದಲ್ಲಿ ನಿಮ್ಮ ಕಾರ್ಯವನ್ನು ಮುಂದುವರಿಸು ತ್ತೀರಾ?” ಇದಕ್ಕೆ ಸ್ವಾಮೀಜಿ, “ಆ ಬಗ್ಗೆ ಯಾವ ಅನುಮಾನವೂ ಬೇಡ. ನಾನು ಹಿಮಾಲಯ ಪ್ರಾಂತ್ಯದಲ್ಲಿ ಕೆಲಕಾಲ ವಿಶ್ರಾಂತಿ ತೆಗೆದುಕೊಂಡು ಅನಂತರ ದೇಶದ ಮೇಲೆಲ್ಲ ಅಗ್ನಿಮುಖ ದಂತೆ ಸಿಡಿಯುತ್ತೇನೆ” ಎಂದುತ್ತರಿಸಿದರು. ಆದರೆ ಸ್ವಾಮೀಜಿಯವರ ಈ ಯೋಜನೆ ಕೈಗೂಡಲೇ ಇಲ್ಲ. ಪ್ರೊ ॥ ಅಯ್ಯರರಾಗಲಿ, ಇತರ ಸಹಸ್ರಾರು ಅಭಿಮಾನಿಗಳಾಗಲಿ ಮತ್ತೆಂದೂ ಸ್ವಾಮೀಜಿ ಯವರ ಮುಖದರ್ಶನ ಪಡೆಯಲು ಸಾಧ್ಯವಾಗಲಿಲ್ಲ.

ಹಡಗು ಲಂಗರು ಕೀಳುವ ಮುನ್ನ ಅಲ್ಲೊಂದು ತಮಾಷೆಯ ಘಟನೆ ನಡೆಯಿತು. ಮದ್ರಾಸಿನ ಭಕ್ತರು ಸ್ವಾಮೀಜಿಯವರ ಉಪಯೋಗಕ್ಕಾಗಿ ಎಳನೀರುಗಳನ್ನು ರಾಶಿರಾಶಿಯಾಗಿ ತುಂಬಿಸು ತ್ತಿದ್ದರು. ಇದನ್ನು ಕಂಡು ಶ್ರೀಮತಿ ಸೇವಿಯರ್ ಗಾಬರಿಕೊಂಡರು. ಆ ಎಳನೀರುಗಳು ಯಾರಿ ಗಾಗಿ ಎಂಬುದು ಆಕೆಗೆ ತಿಳಿಯದು. ಅಸಹನೆಯಿಂದ ಆಕೆ, “ಸ್ವಾಮೀಜಿ, ಇದೇನು ಸರಕು ಸಾಗಾಣಿಕೆಯ ಹಡಗಾಗಿ ಹೋಯಿತೇ? ಅಲ್ಲಿ ನೋಡಿ! ಎಳನೀರುಗಳನ್ನು ಹೇಗೆ ತುಂಬಿಸು ತ್ತಿದ್ದಾರೆ!” ಎಂದರು. ಅದನ್ನು ಕೇಳಿ ಸ್ವಾಮೀಜಿ ಗಟ್ಟಿಯಾಗಿ ನಕ್ಕು, ಅದೆಲ್ಲ ತಮಗಾಗಿಯೇ, ತಮ್ಮ ಭಕ್ತರು ಕಳಿಸುತ್ತಿರುವುದು ಎಂದು ವಿವರಿಸಿದರು. ಪ್ರಯಾಣಕಾಲದಲ್ಲಿ ನೀರಿನ ಬದಲಾಗಿ ಎಳನೀರನ್ನೇ ಕುಡಿಯುವಂತೆ ವೈದ್ಯರು ಸ್ವಾಮೀಜಿಯವರಿಗೆ ಹೇಳಿದ್ದರು. ಅದರಂತೆಯೇ ಅವರು ಎಳನೀರನ್ನು ತಾವು ಕುಡಿದುದಲ್ಲದೆ ಸಹಪ್ರಯಾಣಿಕರಿಗೆಲ್ಲ ಹಂಚಿದರು.

ದಕ್ಷಿಣ ಭಾರತದಲ್ಲಿ ಸ್ವಾಮೀಜಿ ಕೈಗೊಂಡ ವಿಜಯಯಾತ್ರೆ, ಅದರಲ್ಲೂ ಮುಖ್ಯವಾಗಿ ಮದ್ರಾಸಿನಲ್ಲಿ ಅವರು ಮಾಡಿದ ಭಾಷಣಗಳು ಭಾರತೀಯರ ಸುಪ್ತಚೈತನ್ಯವನ್ನು ಬಡಿದೆಬ್ಬಿಸಿ ದುವು. ಸ್ವಾಮೀಜಿ ಭಾರತೀಯರಿಗೆ ಅವರ ಘನತೆಯನ್ನು ತೋರಿಸಿಕೊಟ್ಟರು; ಹಾಗೆಯೇ ಅವರ ದೌರ್ಬಲ್ಯಗಳನ್ನು ಮನಗಾಣಿಸಿದರು. ವೈಭವಯುತ ಭಾರತೀಯ ಪರಂಪರೆಯತ್ತ ಅವರು ಬೆರಳು ಮಾಡಿ ತೋರಿಸಿದರು. ಜೊತೆಗೆ ಇನ್ನೂ ಹೆಚ್ಚಿನ ವೈಭವಯುತ ಭವಿಷ್ಯದ ಚಿತ್ರವನ್ನು ಮುಂದಿಟ್ಟರು. ಆ ಆದರ್ಶವನ್ನು ಸಿದ್ಧಿಸಿಕೊಳ್ಳಲು ಅನುವಾಗುವಂತೆ ಜನರಲ್ಲಿ ಸ್ಫೂರ್ತಿ ತುಂಬಿ ದರು. ಅವರು ಭಾರತೀಯರಲ್ಲಿ ರಾಷ್ಟ್ರಪ್ರಜ್ಞೆಯನ್ನುಂಟುಮಾಡಿದರು; ರಾಷ್ಟ್ರಾಭಿಮಾನವನ್ನು ತುಂಬಿದರು.

