
\chapter{ಅಮರನಾಥದಲ್ಲಿ ಶಿವಾನುಭವ}

\noindent

ಅಚಾಬಾಲ್​ನಲ್ಲಿ ಎರಡು-ಮೂರು ಸಾವಿರ ಅಮರನಾಥ ಯಾತ್ರಿಕರನ್ನು ಕಂಡಾಗ ಸ್ವಾಮೀಜಿ ಯವರಿಗೆ ತಾವೂ ಈ ಯಾತ್ರಿಕರ ತಂಡವನ್ನು ಸೇರಿಕೊಳ್ಳಬೇಕು, ಅಮರನಾಥನ ದರ್ಶನ ಮಾಡಬೇಕು ಎಂಬ ಸ್ಫೂರ್ತಿ ಮತ್ತೊಮ್ಮೆ ಉಕ್ಕಿ ಬಂದಿತು. ಆದ್ದರಿಂದ ಅವರು ತಮ್ಮ ಮನೋಭಿಲಾಷೆಯನ್ನು ಶಿಷ್ಯರ ಮುಂದಿಡುತ್ತ ಹೇಳಿದರು, “ನಾನೂ ಆ ಯಾತ್ರಿಕರ ತಂಡವನ್ನು ಸೇರಿಕೊಳ್ಳುತ್ತೇನೆ. ನನ್ನ ಜೊತೆಗೆ ನಿವೇದಿತಾ ಒಬ್ಬಳು ಬರಲಿ. ಅವಳು ಮುಂದೆ ಭಾರತದ ಸೇವೆಯಲ್ಲಿ ನಿರತಳಾಗುವವಳಾದ್ದರಿಂದ ಅವಳಿಗೆ ಈ ಪುಣ್ಯಕ್ಷೇತ್ರಗಳ ನೇರ ಪರಿಚಯವಿರುವುದು ಆವಶ್ಯಕ. ನೀವೆಲ್ಲ ಪಹಲ್​ಗಾಂವ್​ನವರೆಗೆ ನಮ್ಮೊಂದಿಗೆ ಬಂದು, ನಾನು ಹಿಂದಿರುಗಿ ಬರುವ ವರೆಗೆ ಅಲ್ಲಿ ನಮಗಾಗಿ ಕಾಯುತ್ತಿರಿ.” ಸ್ವಾಮೀಜಿಯವರ ಇಚ್ಛೆಯ ಮೇರೆಗೆ ಅದೇ ಸಂಜೆಯೇ ಎಲ್ಲರೂ ದೋಣಿಗಳಿಗೆ ಹಿಂದಿರುಗಿ ಪ್ರಯಾಣಕ್ಕೆ ಸಿದ್ಧರಾದರು. ಮರುದಿನ ಮಧ್ಯಾಹ್ನದ ವೇಳೆಗೆ ಮಾರ್ತಾಂಡದ ಕಡೆಗೆ ಪ್ರಯಾಣ ಸಾಗಿತು. ಅದು ಅಮರನಾಥದ ದಾರಿಯಲ್ಲಿ ಅವರ ಮೊದಲ ತಂಗುದಾಣ.

ಸ್ವಾಮೀಜಿ ಆಗ ತಾವು ಕೈಗೊಂಡಿರುವ ಯಾತ್ರೆಯ ಬಗ್ಗೆ ಒಂದು ವಿಶೇಷ ಉತ್ಸಾಹ ತಾಳಿದ್ದರು; ಒಪ್ಪೊತ್ತು ಮಾತ್ರ ಊಟ ಮಾಡುತ್ತಿದ್ದರು. ಸಾಧು ಸಂನ್ಯಾಸಿಗಳಲ್ಲದೆ ಇತರ ಯಾರೊಡನೆಯೂ ಮಾತನಾಡುವ ಇಚ್ಛೆಯಿರಲಿಲ್ಲ. ತುಂಬ ಅಂತರ್ಮುಖಿಗಳಾಗಿದ್ದರು. ಅವ ರಲ್ಲಿ ಶಿವಭಾವವೇ ತುಂಬಿತ್ತು. ಜುಲೈ ೨೭ರ ರಾತ್ರಿ ಎಲ್ಲರೂ ಮಾರ್ತಾಂಡಕ್ಕೆ ತಲುಪಿದರು. ಅಲ್ಲೊಂದು ಜನಜಾತ್ರೆಯೇ ಸೇರಿಬಿಟ್ಟಿತ್ತು. ಆದರೆ ಎಲ್ಲರೂ ಯಾತ್ರಿಕರೇ ಆದುದರಿಂದ ಅಲ್ಲಿ ಭಕ್ತಿ-ಶ್ರದ್ಧೆಯಿಂದ ಕೂಡಿದ ಧಾರ್ಮಿಕ ವಾತಾವರಣ ತಾನೇತಾನಾಗಿತ್ತು. ಒಮ್ಮೆ ಸ್ವಾಮೀಜಿ ಯವರ ಡೇರೆಯ ಬಳಿಗೆ ಸಾರಾ ಬುಲ್ ಮತ್ತು ಸೋದರಿ ನಿವೇದಿತಾ ಬಂದಾಗ ಅಲ್ಲಿ ಸಾಧುಗಳ ದೊಡ್ಡದೊಂದು ಗುಂಪು ಸ್ವಾಮೀಜಿಯವರನ್ನು ಆವರಿಸಿಕೊಂಡು ಬಗೆಬಗೆಯ ಪ್ರಶ್ನೆಗಳನ್ನು ಕೇಳುತ್ತಿದ್ದುದು ಕಂಡುಬಂತು.

ಸ್ವಾಮೀಜಿ ಮಾರ್ತಾಂಡದಿಂದ ಹೊರಟು ಮರುದಿನ ಪಹಲ್​ಗಾಂವ್ ತಲುಪಿದರು. ಇಲ್ಲಿ ಅವರು ನದಿಯ ಪಕ್ಕದಲ್ಲೇ ಡೇರೆ ಹಾಕಿಕೊಂಡರು. ಸಮೀಪದಲ್ಲೇ ಲಿಡ್ಡರ್ ಎಂಬ ಜಲಪಾತ ಭೋರ್ಗರೆಯುತ್ತಿತ್ತು. ಸ್ವಾಮೀಜಿ ಆ ಕುರುಬರ ಹಳ್ಳಿಯಲ್ಲಿ ಮರುದಿನ ಪೂರ್ತಿ ಇದ್ದು ಏಕಾದಶಿ ವ್ರತವನ್ನು ಆಚರಿಸಿದರು.

ಪ್ರತಿವರ್ಷ ಸಾವಿರಾರು ಯಾತ್ರಿಕರು ಕೈಗೊಳ್ಳುವ ಅಮರನಾಥ ತೀರ್ಥಯಾತ್ರೆಗೆ ಅದರದೇ ಆದ ಒಂದು ರೋಮಾಂಚಕಾರಕ ವೈಶಿಷ್ಟ್ಯವಿದೆ. ಅಮರನಾಥ ಗುಹೆಯಿರುವುದು ರುದ್ರ- ಪ್ರಶಾಂತ ಸೌಂದರ್ಯದ ಮಡಿಲಲ್ಲಿ, ಹಿಮವತ್ಪರ್ವತದ ಒಡಲಲ್ಲಿ. ಇಲ್ಲಿನ ಪ್ರಕೃತಿ ಸೌಂದರ್ಯ ಪ್ರಪಂಚದಲ್ಲೇ ಅಪೂರ್ವವಾದದ್ದು. ಈ ಯಾತ್ರೆಯ ದಾರಿಯಲ್ಲಿ ಪ್ರತಿಯೊಂದು ತಂಗುದಾಣ ದಲ್ಲಿಯೂ ಕಣ್ಮುಚ್ಚಿ ತರೆಯುವುದರೊಳಗಾಗಿ ಅತ್ಯಂತ ಶಿಸ್ತುಬದ್ಧವಾಗಿ ಉದಯಿಸುವ, ನಾನಾ ಆಕಾರ-ಗಾತ್ರ-ವರ್ಣಗಳಿಂದ ಕೂಡಿದ ಹಲವು ಬಗೆಯ ಡೇರೆಗಳ ಗುಂಪು ಹಾಗೂ ಅಗಲವಾದ ರಸ್ತೆಯಿಂದ ಕೂಡಿದ ಅಂಗಡಿಗಳ ಸಾಲು, ಮತ್ತು ಇವೆಲ್ಲವೂ ಮರುದಿನ ಮುಂಜಾನೆಯ ಮಸುಕಿನಲ್ಲಿಯೇ ಮಂಗಮಾಯವಾಗುವ ಪರಿ–ಇವು ಎಂಥವರನ್ನೂ ಆಶ್ಚರ್ಯಚಕಿತರನ್ನಾ ಗಿಸದಿರದು. ಉಷಃಕಾಲದಲ್ಲೇ ಯಾತ್ರಿಕರ ಸೈನ್ಯ ಎದ್ದು ಸಿದ್ಧವಾಗಿ ಮತ್ತೆ ಮುಂದಿನ ಗುರಿಯತ್ತ ವೇಗವಾಗಿ ಸಾಗುತ್ತದೆ. ಆ ತಂಗುದಾಣಗಳಲ್ಲಿ ಅಡಿಗೆಗಾಗಿ ಹೊತ್ತಿಸಿದ ಲೆಕ್ಕವಿಲ್ಲದಷ್ಟು ಒಲೆಗಳು; ಮೈಗೆಲ್ಲ ಬೂದಿ ಬಳಿದುಕೊಂಡ ಸಾಧುಗಳು ತಮ್ಮ ದೊಡ್ಡದೊಡ್ಡ ಕಾವಿಛತ್ರಿಗಳ ಕೆಳಗೆ ಕುಳಿತು ಧುನಿಯ ಮುಂದೆ ಧ್ಯಾನ ಮಾಡುತ್ತಲೂ ಆಧ್ಯಾತ್ಮಿಕ ವಿಷಯವಾಗಿ ಸಂಭಾಷಣೆ ನಡೆಸುತ್ತಲೂ ಇರುವ ದೃಶ್ಯ; ಅನೇಕ ಸಂಪ್ರದಾಯಗಳಿಗೆ ಸೇರಿದ ಸಂನ್ಯಾಸಿಗಳು, ಮತ್ತು ಅವರವರ ಸಂಪ್ರದಾಯಕ್ಕೆ ಅನುರೂಪವಾದ ವೇಷಭೂಷಣಗಳು; ರಾಷ್ಟ್ರದ ಎಲ್ಲ ಭಾಗಗಳಿಂದ ಬಂದಿ ರುವ, ಭಕ್ತಿ-ಪ್ರೇಮ-ಕಾತರತೆಗಳನ್ನು ಹೊಮ್ಮಿಸುವ, ಸ್ತ್ರೀಪುರುಷರ ಮುಖಗಳು ಹಾಗೂ ಅವರ ವಿಧವಿಧವಾದ ಉಡಿಗೆ ತೊಡಿಗೆಗಳು; ರಾತ್ರಿಯಲ್ಲಿ ಮಿನುಗುವ ದೀಪ-ಕೈದೀವಿಗೆಗಳು; ಶಂಖಜಾಗಟೆಗಳ ನಿನಾದ, ಭಜನೆ ಮಂತ್ರಘೋಷಗಳ ಸ್ವರಸಂಗಮ–ಅದೊಂದು ಬೇರೆಯೇ ಲೋಕ; ಬೇರೆಯೇ ಅನುಭವ.

ಯುಗಯುಗಗಳಿಂದ ಬಂದ ಹಾಗೂ ಕೋಟ್ಯಂತರ ಜನರ ಮನ್ನಣೆ ಪಡೆದ ಹಿಂದೂ ಸಂಪ್ರದಾಯಗಳನ್ನೆಲ್ಲ ಸ್ವಾಮೀಜಿಯವರು ಕಟ್ಟುನಿಟ್ಟಾಗಿ ಪಾಲಿಸುತ್ತಿದ್ದರು. ಯಾವುದೇ ತೀರ್ಥ ಯಾತ್ರೆಗೆ ತೊಡಗಿದಾಗಲೂ ಅವರು ಒಬ್ಬ ಸರಳ ಸಂಪ್ರದಾಯಸ್ಥ ಸ್ತ್ರೀಯಂತೆ ತುಂಬ ಭಕ್ತಿ- ಶ್ರದ್ಧೆಗಳಿಂದ ಎಲ್ಲ ಕಾರ್ಯಗಳನ್ನು ನಡೆಸುತ್ತಿದ್ದರು. ಅವರು ಎಲ್ಲ ತೀರ್ಥಗಳಲ್ಲೂ ಮಿಂದು ಅಲ್ಲಿನ ದೇವದೇವಿಯರಿಗೆ ಹಣ್ಣುಕಾಯಿ ಮಾಡಿಸಿ ಬಳಿಕ ಏನಾದರೂ ಸ್ವಲ್ಪ ಉಪಾಹಾರ ಮಾಡುತ್ತಿದ್ದರು. ಜಪತಪಗಳನ್ನು ಮುಗಿಸಿ ಪ್ರದಕ್ಷಿಣೆಗಳನ್ನು ಹಾಕಿ ದೀರ್ಘದಂಡ ಪ್ರಣಾಮ ಗಳನ್ನು ಸಲ್ಲಿಸುತ್ತಿದ್ದರು. ಈ ರೀತಿಯ ವಿಧಿವಿಧಾನಗಳನ್ನು ಅನುಸರಿಸುವಾಗ ಸ್ವಾಮೀಜಿ ಎಲ್ಲರೊಳಗೊಂದಾಗಿಬಿಡುತ್ತಿದ್ದರು. ಅಂತೆಯೇ ಈಗಲೂ ಕೂಡ ಅಮರನಾಥ ಯಾತ್ರೆಯ ಸಂದರ್ಭದಲ್ಲಿ ಅವರು ಪೂಜಾದಿಗಳನ್ನೆಲ್ಲ ಭಕ್ತಿ ಉತ್ಸಾಹಗಳಿಂದ ನಡೆಸಿಕೊಂಡು ಬಂದರು. ಮಡಿಯಲ್ಲಿ ತಯಾರಿಸಿದ ಅಡಿಗೆಯನ್ನು ಒಪ್ಪೊತ್ತು ಮಾತ್ರ ಉಣ್ಣುತ್ತ, ಸಾಧ್ಯವಾದಷ್ಟೂ ಏಕಾಂತದಲ್ಲಿರುತ್ತ ಬಹಳಷ್ಟು ಸಮಯವನ್ನು ಜಪಧ್ಯಾನಗಳಲ್ಲಿ ಕಳೆಯುತ್ತ ಯಾತ್ರೆಯನ್ನು ಮುಂದುವರಿಸಿದರು.

ಅಲ್ಲಿ ಪ್ರಯಾಣ ಮಾಡುತ್ತಿದ್ದ ನೂರಾರು ಸಾಧು ಸಂನ್ಯಾಸಿಗಳ ಮೇಲೆ ಉಂಟಾದ ಸ್ವಾಮೀಜಿ ಯವರ ವ್ಯಕ್ತಿತ್ವದ ಪ್ರಭಾವ ಅಗಾಧ. ಆದರೆ ಮೊದಮೊದಲು ತಮ್ಮೊಡನಿದ್ದ ಪಾಶ್ಚಾತ್ಯ ಶಿಷ್ಯೆಯ ಕಾರಣದಿಂದಾಗಿ ಅವರು ಸಂಪ್ರದಾಯಸ್ಥ ಸಾಧುಗಳಿಂದ ತೀವ್ರ ವಿರೋಧವನ್ನೇ ಎದುರಿಸ ಬೇಕಾಯಿತು. ಅವರು ತಮ್ಮ ಡೇರೆಯನ್ನು ಇತರ ಯಾತ್ರಿಕರ ಡೇರೆಗಳ ಬಳಿಯಲ್ಲಿಯೇ ಹೂಡಿದಾಗ ಅದನ್ನು ದೂರದಲ್ಲಿ ಹಾಕಿಕೊಳ್ಳುವಂತೆ ಈ ಸಾಧುಗಳು ಹೇಳಿದರು. ಆದರೆ ಸ್ವಾಮೀಜಿ ಅದಕ್ಕೆ ಕಿವಿಗೊಡಲಿಲ್ಲ. ಕಡೆಗೆ ಒಬ್ಬ ನಾಗಾ ಸಾಧು ಬಂದು ದೈನ್ಯದಿಂದ, “ಸ್ವಾಮೀಜಿ, ನೀವು ಮಹಾಶಕ್ತಿಶಾಲಿಗಳು; ಆದರೆ ದಯವಿಟ್ಟು ಆ ಶಕ್ತಿಯನ್ನು ವ್ಯಕ್ತಗೊಳಿಸಲು ಹೋಗ ಬಾರದು” ಎಂದು ಬೇಡಿಕೊಂಡಾಗ ಸ್ವಾಮೀಜಿ ಒಪ್ಪಿಕೊಂಡರು. ಮರುದಿನವೇ ಅವರು ತಮ್ಮ ಡೇರೆಯನ್ನು ಎಲ್ಲ ಡೇರೆಗಳಿಗಿಂತ ಮುಂದೆ–ಸುಂದರ ಪರ್ವತಶ್ರೇಣಿಯ ತಪ್ಪಲಿನ ಹೊಳೆಯ ದಡದಲ್ಲಿ ಹೂಡಿದರು. ಅಂದು ಮಧ್ಯಾಹ್ನ ಅವರು, ನಿವೇದಿತೆಯು ಎಲ್ಲರಿಂದಲೂ ಆಶೀರ್ವಾದ ಪಡೆಯಲೆಂದು, ಅರ್ಥಾತ್ ಎಲ್ಲ ಸಾಧುಗಳಿಗೂ ಭಿಕ್ಷೆಹಾಕಲೆಂದು, ಅವಳನ್ನು ಎಲ್ಲ ಡೇರೆಗಳಿಗೂ ಕರೆದೊಯ್ದರು.

ಅಲ್ಲಿಂದ ಮುಂದೆ ಪ್ರತಿಯೊಂದು ತಂಗುದಾಣದಲ್ಲಿಯೂ ಸ್ವಾಮೀಜಿಯವರ ಡೇರೆಯು ಜ್ಞಾನಪಿಪಾಸುಗಳಾಗಿ ಬಂದ ಸಾಧುಗಳಿಂದ ತುಂಬಿರುತ್ತಿತ್ತು. ಆದರೆ ಅವರಲ್ಲಿ ಅನೇಕರು ಧಾರ್ಮಿಕ ವಿಷಯಗಳ ಬಗೆಗಿನ ಸ್ವಾಮೀಜಿಯವರ ವಿಶಾಲ ಮನೋಭಾವವನ್ನು, ಇಸ್ಲಾಂ ಧರ್ಮದ ಬಗ್ಗೆ ಅವರಿಗಿದ್ದ ಆದರ ಸಹಾನುಭೂತಿಗಳನ್ನು ಅರ್ಥಮಾಡಿಕೊಳ್ಳಲಾರದೆ ಹೋದರು. ಈ ತೀರ್ಥಯಾತ್ರಿಕರ ತಂಡದಲ್ಲಿದ್ದ ಸರ್ಕಾರದ ಪ್ರತಿನಿಧಿಯಾದ ಒಬ್ಬ ಮುಸಲ್ಮಾನ ತಹಸೀ ಲ್ದಾರ ಮತ್ತು ಅವನ ಕೈಕೆಳಗಿನ ಅಧಿಕಾರಿಗಳು ಸ್ವಾಮೀಜಿಯವರಿಂದ ಎಷ್ಟೊಂದು ಆಕರ್ಷಿತ ರಾದರೆಂದರೆ ಅವರು ಪ್ರತಿದಿನವೂ ಸ್ವಾಮೀಜಿಯವರ ಮಾತುಗಳನ್ನು ಕೇಳಲು ಬರುತ್ತಿದ್ದರು. ಅಲ್ಲದೆ ಕಡೆಯಲ್ಲಿ ಅವರು ಸ್ವಾಮೀಜಿಯವರಿಂದ ಮಂತ್ರದೀಕ್ಷೆಯನ್ನು ಯಾಚಿಸಿ ಪಡೆದರು. ಸೋದರಿ ನಿವೇದಿತಾ ತನ್ನ ಸರಳ ವಿನಮ್ರ ವರ್ತನೆಯಿಂದ ಎಲ್ಲರ ವಿಶ್ವಾಸ ಆದರಗಳಿಗೆ ಪಾತ್ರಳಾದಳು.

ಜುಲೈ ೩ಂರಂದು ಮುಂಜಾನೆಯ ಉಪಾಹಾರದ ನಂತರ ಸ್ವಾಮೀಜಿಯವರು ನಿವೇದಿತೆ ಯೊಂದಿಗೆ ಚಂದನ್​ವಾರ್ ಎಂಬಲ್ಲಿಗೆ ಹೊರಟರು. ಅಲ್ಲಿ ಅವರು ಒಂದು ಕಣಿವೆಯ ಬದಿಯಲ್ಲೇ ಬೀಡುಬಿಟ್ಟರು. ಮಧ್ಯಾಹ್ನವೆಲ್ಲ ಮಳೆ ಸುರಿಯಿತು. ಮರುದಿನ ಪ್ರಯಾಣ ಮುಂದು ವರಿಸಿದಾಗ ಮಂಜಿನ ಸೇತುವೆಯೊಂದನ್ನು ದಾಟಬೇಕಾಯಿತು. ನಿವೇದಿತೆಗೆ ಅದೊಂದು ಹೊಸ ಅನುಭವ. ಮುಂದೆ ಅವರು ಕಡಿದಾದ ಗುಡ್ಡವನ್ನು ದಾಟಿ ಪಿಶು ಘಾಟ್​ನತ್ತ ಸಾಗಿದರು. ಮೇಲಿನಿಂದ, ಸುತ್ತಲ ನೆಲ ಹೂಗಳಿಂದ ಆವೃತವಾಗಿ ಸುಂದರವಾಗಿ ಕಾಣುತ್ತಿತ್ತು. ಈ ದಾರಿ ‘ಶೇಷನಾಗ’ದ ಸರೋವರದ ಮೂಲಕ ಹಾದುಹೋಗಿದೆ. ಅಂತೂ ಜುಲೈ ೩೧ರಂದು ಹನ್ನೆರ ಡೂವರೆ ಸಾವಿರ ಅಡಿ ಎತ್ತರದಲ್ಲಿರುವ ‘ವಾವ್ ಜಸ್​’ ಎಂಬಲ್ಲಿ ಬೀಡುಬಿಟ್ಟರು. ಸ್ವಾಮೀಜಿ, ನಿವೇದಿತಾ ಹಾಗೂ ತಹಸೀಲ್ದಾರರ ಡೇರೆಗಳನ್ನು ಹತ್ತಿರ ಹತ್ತಿರದಲ್ಲೆ ಹಾಕಲಾಯಿತು. ಚಳಿಯನ್ನು ನಿವಾರಿಸಲೋಸ್ಕರ ಒಂದು ದೊಡ್ಡ ಉರಿಯನ್ನು ಹಾಕಲಾಯಿತು. ಆದರೆ ಅದು ಚೆನ್ನಾಗಿ ಉರಿಯಲಿಲ್ಲ. ಬೆಂಕಿಯೇ ತಣ್ಣಗಾಗುವಷ್ಟು ಚಳಿ!

ಮರುದಿನ ಯಾತ್ರಿಕರು ಹದಿನಾಲ್ಕು ಸಾವಿರ ಅಡಿ ಎತ್ತರದ ‘ಮಹಾಗುನು’ ಶಿಖರವನ್ನು ದಾಟಿ ‘ಪಂಚತರಣಿ’ಯನ್ನು ತಲುಪಿದರು. ಅಲ್ಲಿನ ಶೀತದ ಒಣ ಹವೆ ಅತ್ಯಂತ ಪ್ರಯಾಸಕಾರಿ ಯಾಗಿತ್ತು. ಅವರ ತಂಗುದಾಣದ ಬಳಿಯೇ ಒಂದು ಬತ್ತಿ ಹೋದ ನದಿ ಇತ್ತು. ಅದರಲ್ಲಿ ಐದು ಸಣ್ಣ ಸಣ್ಣ ನೀರಿನ ಝರಿಗಳಿದ್ದು, ಯಾತ್ರಿಕರು ಒಂದಾದ ಮೇಲೊಂದರಂತೆ ಎಲ್ಲದರಲ್ಲೂ ಒದ್ದೆ ಬಟ್ಟೆ ಧರಿಸಿ ಸ್ನಾನ ಮಾಡಬೇಕಾಗಿತ್ತು. ಎಲ್ಲ ವಿಧಿಗಳನ್ನೂ ಬಿಡದೆ ಪಾಲಿಸುತ್ತಿದ್ದ ಸ್ವಾಮೀಜಿ ಇಲ್ಲಿಯೂ ನಿಯಮವನ್ನು ಅಕ್ಷರಶಃ ಪಾಲಿಸಿದರು. ಆದರೆ ತಮ್ಮ ಆಧ್ಯಾತ್ಮಿಕ ಪುತ್ರಿಗೆ ಅದರಿಂದ ವಿನಾಯತಿ ಕೊಟ್ಟರು.

ಆಗಸ್ಟ್ ೩ರಂದು ಅಮರನಾಥ ಯಾತ್ರೆಯ ಕೊನೆಯ ಹಾಗೂ ಅತ್ಯಂತ ರೋಮಾಂಚಕರ ವಾದ ಹಂತ. ಮೊದಲು ಕಡಿದಾದ ಶಿಖರದ ಆರೋಹಣ, ನಂತರ ಅವರೋಹಣ. ಒಂದು ಹೆಜ್ಜೆ ತಪ್ಪಿದರೂ ‘ಕೈಲಾಸಯಾತ್ರೆ’! ಮುಂದೆ ಯಾತ್ರಿಕರೆಲ್ಲ ಒಂದು ದೊಡ್ಡ ಹಿಮಬಂಡೆಯ ಮೂಲಕ ಹಾದುಹೋದರು. ಅಮರನಾಥ ಗುಹೆಯನ್ನು ಪ್ರವೇಶಿಸುವ ಮುನ್ನ ಎಲ್ಲರೂ ತಿಳಿನೀರಿನ ಝರಿಯಲ್ಲಿ ಸ್ನಾನ ಮಾಡಿದರು. ನಂತರ ಕಡಿದಾದ ಗುಡ್ಡವನ್ನೇರಿದರೆ ಗುಹೆಯ ದರ್ಶನ! ಎಲ್ಲರೂ ಆನಂದೋತ್ಸಾಹದಿಂದ ಮುನ್ನುಗ್ಗಿದರು. ಸ್ವಾಮೀಜಿ ಮಾತ್ರ ಯಾಕೋ ಸ್ವಲ್ಪ ಹಿಂದೆಬಿದ್ದರು. ಬಹುಶಃ ತಮ್ಮ ಆಲೋಚನೆಗಳೊಂದಿಗೆ ಏಕಾಂತದಿಂದಿರುವ ಉದ್ದೇಶ ದಿಂದಲೋ ಏನೋ. ತಮಗಾಗಿ ಕಾಯುತ್ತಿದ್ದ ಶಿಷ್ಯೆಯನ್ನು ಮುಂದೆ ಕಳುಹಿಸಿ ತಾವು ಝರಿಯಲ್ಲಿ ಸ್ನಾನ ಮುಗಿಸಿ ನಿಧಾನವಾಗಿ ಬಂದರು. ಗುಹೆಯನ್ನು ಪ್ರವೇಶಿಸುತ್ತಿದ್ದಂತೆ ಅವರ ಇಡಿಯ ಶರೀರ ಉದ್ವೇಗದಿಂದ ಕಂಪಿಸತೊಡಗಿತು. “ಒಂದು ದೊಡ್ಡ ದೇವಸ್ಥಾನವೇ ಹಿಡಿಸಬಹುದಾದಂತಹ ಮಹತ್ತರವಾದ ಗುಹೆಯಲ್ಲಿ ಸ್ಫಟಿಕವರ್ಣದ ಶಿವಲಿಂಗವು ಕಾಲವರ್ಣದ ನೆರಳೆಂಬ ಪೀಠದ ಮೇಲೆ ಕುಳಿತಿರುವಂತೆ ಭಾಸವಾಯಿತು” ಎನ್ನುತ್ತಾಳೆ ಸೋದರಿ ನಿವೇದಿತಾ. ಕೇವಲ ಕೌಪೀನಧಾರಿ ಯಾದ ಸ್ವಾಮೀಜಿಯವರು ಗುಹೆಯನ್ನು ಪ್ರವೇಶಿಸಿದರು. ಅವರ ಮುಖಮಂಡಲ ಭಕ್ತಿಭಾವ ದಿಂದ ಪ್ರಜ್ವಲಿಸುತ್ತಿತ್ತು. ಒಳಗೆ ಪ್ರವೇಶಿಸಿದೊಡನೆಯೇ ದೀರ್ಘದಂಡ ಪ್ರಣಾಮ ಮಾಡಿದರು. ನೂರಾರು ಕಂಠಗಳಿಂದ ಹೊಮ್ಮಿದ ಜೈಕಾರದಿಂದ ಅಮರನಾಥ ಗುಹೆ ಮಾರ್ದನಿಗೊಳ್ಳು ತ್ತಿದ್ದಂತೆಯೇ, ಸ್ಫಟಿಕವರ್ಣದ ಶಿವಲಿಂಗದಿಂದ ಹೊಮ್ಮುತ್ತಿದ್ದ ಪರಮ ಪಾವಿತ್ರ್ಯದ ದಿವ್ಯ ಅಲೆಗಳಲ್ಲಿ ಅವರು ಮೈಮರೆತರು; ಉದ್ವೇಗಭರದಲ್ಲಿ ಬಾಹ್ಯಪ್ರಜ್ಞೆಯನ್ನೇ ಕಳೆದುಕೊಂಡರು. ಆ ಸ್ಥಿತಿಯಲ್ಲಿ ಅವರಿಗೊಂದು ಪರಮಾದ್ಭುತ ಅಲೌಕಿಕ ಅನುಭವವಾಯಿತು. ಆದರೆ ಅವರು ಆ ವಿಷಯವಾಗಿ ಯಾರಿಗೂ ಏನೂ ಹೇಳಲಿಲ್ಲ. ತುಂಬ ಒತ್ತಾಯಿಸಿ ಕೇಳಿದಾಗ, “ನನಗೆ ಅಮರತ್ವದ ದೇವತೆಯಾದ ಅಮರನಾಥನ ಸಾಕ್ಷಾತ್ ದರ್ಶನವಾಯಿತು. ಅವನು ನನಗೆ ಇಚ್ಛಾಮರಣದ ವರದಾನ ಮಾಡಿದ” ಎಂದಷ್ಟೇ ಹೇಳಿದರು.

ಆದರೆ ಈ ಅಮರನಾಥ ಯಾತ್ರೆಯಿಂದ ಸ್ವಾಮೀಜಿಯವರ ಆರೋಗ್ಯವಂತೂ ಶಾಶ್ವತವಾಗಿ ಘಾಸಿಗೊಂಡಿತು. ಅದಕ್ಕೆ ಕಾರಣ ಅವರಿಗಾದ ಅಲೌಕಿಕ ಅನುಭವದ ಪರಿಣಾಮವೋ ಅಥವಾ ಅತಿ ಶೈತ್ಯದಲ್ಲಿ, ಪ್ರಾಣವಾಯು ಅತಿ ವಿರಳವಾಗಿರುವಲ್ಲಿ ದೇಹವನ್ನು ಬಹಳವಾಗಿ ದಂಡಿಸಿದ್ದರ ಪರಿಣಾಮವೋ, ಅಥವಾ ಈ ಎರಡರ ಒಟ್ಟು ಪರಿಣಾಮವೋ ಹೇಳಲಾಗದು. ಮುಂದೆ ಒಬ್ಬ ವೈದ್ಯರು ಸ್ವಾಮೀಜಿಯವರ ಅಮರನಾಥದ ಅನುಭವದ ಬಗ್ಗೆ ತಿಳಿಸಿದಾಗ ಹೇಳಿದರು, “ಸ್ವಾಮೀಜಿ, ಆ ಸ್ಥಿತಿ ಸಾವಿಗೆ ಅತಿ ಸಮೀಪವಾದದ್ದು! ನಿಮ್ಮ ಹೃದಯದ ಬಡಿತ ನಿಂತೇ ಹೋಗ ಬೇಕಾಗಿತ್ತು; ಬದಲಾಗಿ ಅದು ಶಾಶ್ವತವಾಗಿ ಹಿಗ್ಗಿಹೋಗಿದೆ” ಎಂದು.

ಹಿಂದೆ ಯಾವ ತೀರ್ಥಕ್ಷೇತ್ರವನ್ನು ಸಂದರ್ಶಿಸಿದಾಗಲೂ ಸ್ವಾಮೀಜಿಯವರಿಗೆ ಇಂತಹ ಉತ್ಕಟ ಅನುಭವವಾಗಿರಲಿಲ್ಲ. ಆಮೇಲೆ ಅವರು ನಿವೇದಿತೆಯ ಬಳಿ ತಮ್ಮ ಆ ಅನುಭವವನ್ನು ಹೇಳಿಕೊಂಡರು, “ಆ ಮೂರ್ತಿ ಸಾಕ್ಷಾತ್ ಭಗವಂತನೇ ಆಗಿತ್ತು. ಅಲ್ಲಿ ಕೇವಲ ಶಿವಪೂಜೆಯ ವಾತಾವರಣವೇ ತುಂಬಿಹೋಗಿತ್ತು. ಅಷ್ಟೊಂದು ಸುಂದರವಾದದ್ದನ್ನು, ಅಷ್ಟೊಂದು ಚೈತನ್ಯ ದಾಯಕವಾದದ್ದನ್ನು ನಾನು ಬೇರೆಲ್ಲೂ ಕಂಡಿಲ್ಲ.” ಮುಂದೊಮ್ಮೆ ಅವರು ತಮ್ಮ ಗುರುಭಾಯಿ ಗಳೊಂದಿಗೆ ಆ ಬಗ್ಗೆ ಮಾತನಾಡುತ್ತ ಹೇಳಿದರು, “ಅಮರನಾಥ ಗುಹೆಯ ಅಸ್ತಿತ್ವ ಹೇಗೆ ಬೆಳಕಿಗೆ ಬಂದಿರಬಹುದು ಎಂಬುದನ್ನು ನಾನು ಊಹಿಸಬಲ್ಲೆ. ಯಾವುದೋ ಒಂದು ಬೇಸಿಗೆಯ ದಿನ ತಮ್ಮ ಕುರಿಗಳನ್ನು ಕಳೆದುಕೊಂಡ ಕುರುಬರ ತಂಡವೊಂದು ಅದನ್ನು ಹುಡುಕಿಕೊಂಡು ಅಲೆಯುತ್ತ ಅಲ್ಲಿಗೆ ಬಂದಿರಬೇಕು; ಅನಿರೀಕ್ಷಿತವಾಗಿ ತಮ್ಮೆದುರು ನಿಂತಿದ್ದ ಈ ಮಹತ್ತರ ಗುಹೆ, ಅದರೊಳಗಿನ ಎಂದೂ ಕರಗದ ಬೃಹತ್ ಶುಭ್ರ ಧವಳ ಶಿವಲಿಂಗ, ಅದರ ಮೇಲೆ ತಾನಾಗಿಯೇ ನಿರಂತರವಾಗಿ ಅಭಿಷೇಕವಾಗುತ್ತಿರುವ ನೀರು–ಶತಮಾನಗಳಿಂದ ಮಾನವನ ಕಣ್ಣಿಗೇ ಬೀಳದಿರುವ ಈ ಅದ್ಭುತವನ್ನು ನೋಡಿ ಅವರು ಭಯಾಶ್ಚರ್ಯಚಕಿತರಾಗಿರಬೇಕು. ಕಾಲಾಂತರದಲ್ಲಿ ಈ ವಿಷಯ ಒಬ್ಬರಿಂದೊಬ್ಬರಿಗೆ ತಿಳಿದುಬಂದಿರಬೇಕು.” ಅದು ಹೇಗೇ ಇರಲಿ, ಸ್ವಾಮೀಜಿಯವರ ಮಟ್ಟಿಗೆ ಇದಂತೂ ನಿಜ; ಗುಹೆಯನ್ನು ಪ್ರವೇಶಿಸಿದೊಡನೆಯೇ ಅವರು ಮುಖಾಮುಖಿಯಾಗಿ ಭೇಟಿಯಾದದ್ದು ಸ್ವಯಂ ಪರಮೇಶ್ವರನನ್ನು!

ಅಮರನಾಥದಲ್ಲಿ ಸ್ವಾಮೀಜಿಯವರಿಗಾದದ್ದು ಅದ್ಭುತ ಅನುಭವವಾದರೆ, ಅವರಲ್ಲಿ ನಿವೇದಿತಾ ಕಂಡದ್ದು ಪರಮಾದ್ಭುತವಾದದ್ದನ್ನು! ಅವರ ತನುಮನಗಳೆಲ್ಲ ಶಿವಮಯವೇ ಆಗಿಬಿಟ್ಟಿದ್ದುವು. ಅನೇಕ ದಿನಗಳವರೆಗೆ ಅವರಿಗೆ ಅಮರನಾಥ ಕ್ಷೇತ್ರ, ಧ್ಯಾನಮಗ್ನ ವೈರಾಗ್ಯ ಮೂರ್ತಿ ಶಿವ, ಭೈರವಸ್ವರೂಪ ಶಿವ, ಇವುಗಳಲ್ಲದೆ ಬೇರೇನನ್ನು ಕುರಿತೂ ಮಾತನಾಡಲು ಸಾಧ್ಯವಾಗಲಿಲ್ಲ.

ಅಮರನಾಥನಿಂದ ಬೀಳ್ಕೊಂಡ ಸ್ವಾಮೀಜಿ ನಿವೇದಿತೆಯೊಂದಿಗೆ ಪಹಲ್ ಗಾಂವ್​ನತ್ತ ನಡೆದರು. ದಾರಿಯಲ್ಲೊಂದು ಪ್ರಸಿದ್ಧ ಸರೋವರ. ಅದರ ಹೆಸರು ಮೃತ್ಯುಸರೋವರ! ಒಮ್ಮೆ ಆ ಸ್ಥಳದಲ್ಲಿ ಸುಮಾರು ನಲವತ್ತು ಜನ ಯಾತ್ರಿಕರು ತಾರಸ್ವರದಲ್ಲಿ ವೇದಮಂತ್ರವನ್ನು ಘೋಷಿಸುತ್ತಿದ್ದರಂತೆ. ಅದರ ಪರಿಣಾಮವಾಗಿ ಪರ್ವತದ ಹಿಮಗಡ್ಡೆ ಕರಗಿ ಮಹಾರಭಸದಿಂದ ಹರಿದುಬಂದು ಯಾತ್ರಿಕರನ್ನೆಲ್ಲ ಕೊಚ್ಚಿಕೊಂಡು ಹೋಗಿ ಸರೋವರದಲ್ಲಿ ಮುಳುಗಿಸಿ ಕೈಲಾಸ ವಾಸಿಗಳನ್ನಾಗಿಸಿತಂತೆ. ಆದ್ದರಿಂದಲೇ ಅದಕ್ಕೆ ಮೃತ್ಯುಸರೋವರವೆಂಬ ಹೆಸರು ಬಂತೆನ್ನುತ್ತಾರೆ. ಇಲ್ಲಿಂದ ಮುಂದೆ ಸ್ವಾಮೀಜಿ ಇನ್ನು ಕೆಲವು ಯಾತ್ರಿಕರೊಂದಿಗೆ ಹತ್ತಿರದ ದಾರಿಯೊಂದರ ಮೂಲಕ ನಡೆದರು. ಈ ದಾರಿ ತುಂಬ ಇಕ್ಕಟ್ಟಿನದು. ಜೊತೆಗೆ ತುಂಬ ಇಳಿಜಾರಿನಿಂದ ಕೂಡಿದುದು. ಈ ದಾರಿಯಾಗಿ ಸ್ವಾಮೀಜಿ ಹಾಗೂ ನಿವೇದಿತಾ ಎಚ್ಚರಿಕೆಯಿಂದ ಸಾಗಿ ಪಹಲ್ ಗಾಂವನ್ನು ಮುಟ್ಟಿದರು. ಅಲ್ಲಿ ಅವರು ತಮಗಾಗಿ ಕಾದಿದ್ದ ಪಾಶ್ಚಾತ್ಯ ಶಿಷ್ಯೆಯರನ್ನು ಕೂಡಿ ಕೊಂಡಾಗ ಅಲ್ಲೊಂದು ಆನಂದದ ಸಂತೆಯೇ ನಿರ್ಮಾಣಗೊಂಡಿತು. ಅಲ್ಲಿ ಸ್ವಾಮೀಜಿಯವರ ಸಂಭಾಷಣೆಯಲ್ಲೆಲ್ಲ ಕೇವಲ ಅಮರನಾಥ ಶಿವ, ಅಮರನಾಥ ಗುಹೆ, ಅಲ್ಲಿನ ಅವರ ಅಪೂರ್ವ ಅನುಭವ–ಇವುಗಳೇ ತುಂಬಿಕೊಂಡಿದ್ದುವು.

ಪಹಲ್​ಗಾಂವ್​ನಿಂದ ಸ್ವಾಮೀಜಿಯವರ ತಂಡ ಅನಂತನಾಗಕ್ಕೆ ಹಿಂದಿರುಗಿತು. ಅಲ್ಲಿಂದ ಎಲ್ಲರೂ ದೋಣಿಗಳಲ್ಲಿ ಹೊರಟು ಆಗಸ್ಟ್ ೮ರಂದು ಶ್ರೀನಗರವನ್ನು ತಲುಪಿದರು. ಈ ದೋಣಿಯ ಪ್ರವಾಸದ ಸಮಯದಲ್ಲಿ ನಿವೇದಿತಾ, ಲಂಡನ್ನಿನಲ್ಲಿ ಸ್ವಾಮೀಜಿಯವರ ಶಿಷ್ಯೆ ಯಾಗಿದ್ದ ಶ್ರೀಮತಿ ಎರಿಕ್ ಹ್ಯಾಮಂಡ್​ಳಿಗೆ ಬರೆದ ಒಂದು ಪತ್ರ ಹೀಗಿದೆ:

“ಈಗ ನಾವು ಶ್ರೀನಗರಕ್ಕೆ ಹಿಂದಿರುಗುವ ದಾರಿಯಲ್ಲಿದ್ದೇವೆ. ನನ್ನೊಬ್ಬಳಿಗೇ ಒಂದು ಪ್ರತ್ಯೇಕವಾದ ದೋಣಿ ಇದೆ. ಎದುರಿನಲ್ಲೆ ಸ್ವಾಮೀಜಿಯವರ ದೋಣಿ ಸಾಗುತ್ತಿದೆ. ಅದರ ಹಿಂದೆಯೇ ಶ್ರೀಮತಿ ಬುಲ್ ಹಾಗೂ ಮಿಸ್ ಮೆಕ್​ಲಾಡ್​ರ ದೋಣಿ. ಆ ದೋಣಿಯಲ್ಲೇ ನಾವೆಲ್ಲ ಊಟ ಮಾಡುವುದು. (ಮುಂಜಾನೆ ಆರು ಗಂಟೆಗೆ ಉಪಾಹಾರ, ಮಧ್ಯಾಹ್ನ ಹನ್ನೆರಡಕ್ಕೆ ಭೋಜನ, ಸಂಜೆ ಐದಾರು ಗಂಟೆಗೆ ಏನಾದರೂ ತಿಂಡಿ.) ನದಿಯ ನೀರು ಗಾಜಿನಂತೆ ತಿಳಿಯಾಗಿದೆ. ದೋಣಿಗಳು ಮೆಲ್ಲನೆ ಚಲಿಸುತ್ತಿರುವಂತೆ, ಮಂದಮಾರುತ ಆಹ್ಲಾದಕರವಾಗಿ ಬೀಸುತ್ತಿದೆ. ನಿಜಕ್ಕೂ ಇದು ಸ್ವರ್ಗವೇ ಸರಿ! ಆದರೆ ಇನ್ನು ಕೆಲವೇ ವಾರಗಳಲ್ಲಿ ಇವೆಲ್ಲ ಕೇವಲ ಕನಸು...

“ಈಗ ನಾನು ನಿನಗೆ ನಿಜಕ್ಕೂ ಆಶ್ಚರ್ಯವನ್ನುಂಟುಮಾಡುವ ಸುದ್ದಿಯನ್ನು ಬರೆಯುತ್ತೇನೆ– ನಾನು ಒಂದು ವಾರದಿಂದಲೂ ಸ್ವಾಮೀಜಿಯವರೊಡನೆ ಹಿಮಾಲಯದ ಪ್ರವಾಸದಲ್ಲಿದ್ದೆ; ಸಮುದ್ರಮಟ್ಟದಿಂದ ೧೮ಂಂಂ ಅಡಿ ಎತ್ತರದ ಪರ್ವತ ಶಿಖರಗಳನ್ನೇರಿದೆ–ಇವಿಷ್ಟು ವಿಷಯ ವನ್ನು ನೀನು ಅಲ್ಲಿ ಯಾರಿಗೆ ಬೇಕಾದರೂ ಹೇಳಬಹುದು. ಆದರೆ ನಾನು ಈಗ ತಿಳಿಸುವುದನ್ನು ಮಾತ್ರ ಯಾರಿಗೂ ಹೇಳುವಂತಿಲ್ಲ–ನಾವು ನಿಜಕ್ಕೂ ಹೋಗಿದ್ದುದು ಅಮರನಾಥ ಯಾತ್ರೆಗೆ! ಸ್ವಾಮೀಜಿಯವರು ನನ್ನನ್ನು ಶಿವನಿಗೆ ಸಮರ್ಪಿಸಲು ಕಾತರರಾಗಿದ್ದರು; ಮತ್ತು ಅದರಂತೆಯೇ ನನ್ನನ್ನು ಶಿವನಿಗೆ ಸಮರ್ಪಿಸಿಬಿಟ್ಟರು!”

ಶ್ರೀನಗರದಲ್ಲಿ ಸ್ವಾಮೀಜಿ ತಮ್ಮ ಸಂಗಡಿಗರೊಂದಿಗೆ ಸೆಪ್ಟೆಂಬರ್ ೩ಂ ರವರೆಗೆ ಉಳಿದು ಕೊಂಡರು. ಆಗಾಗ ಅವರು ದೋಣಿಯಲ್ಲಿ ಒಬ್ಬರೇ ಹೋಗಿ ಕೆಲದಿನಗಳ ಕಾಲ ಏಕಾಂತದಲ್ಲಿದ್ದು ಹಿಂದಿರುಗುತ್ತಿದ್ದರು. ಧ್ಯಾನಾನಂದದಲ್ಲಿ ಮಗ್ನರಾಗಿರಬೇಕೆಂಬ ಅವರ ಆಸೆ ಈಗ ಹೆಚ್ಚು ಹೆಚ್ಚು ಸ್ಪಷ್ಟವಾಗತೊಡಗಿತ್ತು. ಆದರೆ ಇದರ ಮಧ್ಯದಲ್ಲೂ ಅವರು ತಮ್ಮ ಶಿಷ್ಯೆಯರಿಗೆ–ಮುಖ್ಯವಾಗಿ ನಿವೇದಿತೆಗೆ–ಶಿಕ್ಷಣ ನೀಡುವ ಕಾರ್ಯವನ್ನು ಮುಂದುವರಿಸಿದರು. ಅದರಲ್ಲೂ ವಿಶೇಷವಾಗಿ, ಹಿಂದೂ ಧರ್ಮದ ಅಸಹ್ಯಮಡಿವಂತಿಕೆಯನ್ನು ಹೋಗಲಾಡಿಸಿ, ಅದು ನವಚೇತನದಿಂದ ತುಂಬಿ ಮುನ್ನಡೆಯುವಂತೆ ಮಾಡುವ ತಮ್ಮ ಕನಸಿನ ಯೋಜನೆಗಳ ಬಗ್ಗೆ ಅವರು ಸುದೀರ್ಘವಾಗಿ ಸಮಾಲೋಚಿಸುತ್ತಿದ್ದರು. ಅತ್ಯುನ್ನತ ಆಧ್ಯಾತ್ಮಿಕತೆಯನ್ನು ಅತ್ಯಂತ ಕ್ರಿಯಾಶೀಲ ಜೀವನದಲ್ಲಿ ಅಳವಡಿಸಿಕೊಳ್ಳುವುದರ ಮಹತ್ವವನ್ನು ವಿವರಿಸುತ್ತಿದ್ದರು. ಶ್ರೀರಾಮಕೃಷ್ಣರ ಮಾತನ್ನು ಉದ್ಧರಿ ಸುತ್ತ ಅವರೆಂದರು, “ಆಗಸದ ವೈಶಾಲ್ಯದೊಂದಿಗೆ ಸಾಗರದ ಆಳ–ಇದು ನಮ್ಮ ಆದರ್ಶ ವಾಗಿರಬೇಕು,” ಎಂದು. ಬಳಿಕ ಹೇಳಿದರು, “ಶ್ರೀರಾಮಕೃಷ್ಣರು ತಮ್ಮ ಅಂತರಂಗದಾಳದಲ್ಲಿ ಬದುಕಿದ್ದವರು. ಹೀಗಿದ್ದರೂ ಕೂಡ ಅವರು ತಮ್ಮ ಹೊರವಲಯದಲ್ಲೂ ಸಂಪೂರ್ಣ ಸಕ್ರಿಯ ರಾಗಿದ್ದರು, ಸಮರ್ಥರಾಗಿದ್ದರು.”

ಸ್ವತಃ ಸ್ವಾಮೀಜಿಯವರೂ ಸಮಾಧಿಮಗ್ನರಾಗಿ ತಮ್ಮ ಅಂತರಂಗದಾಳದಲ್ಲಿ ಆನಂದದಿಂದಿರ ಬಲ್ಲವರು. ಆದರೆ ಅವರು ಪ್ರಪಂಚದ ಆಗುಹೋಗುಗಳ ವಿಚಾರದಲ್ಲಿ ಪ್ರಾಪಂಚಿಕರಿಗಿಂತಲೂ ಹೆಚ್ಚು ಎಚ್ಚತ್ತಿದ್ದರೆಂದರೆ ಅದು ಅತಿಶಯೋಕ್ತಿಯಲ್ಲ. ಒಮ್ಮೆ ಒಬ್ಬ ಅವರನ್ನು ಕೇಳಿದ, “ಬಲಿಷ್ಠ ರಾದವರು ದುರ್ಬಲರನ್ನು ತುಳಿಯುತ್ತಿರುವುದು ಕಂಡುಬಂದರೆ ನಾವೇನು ಮಾಡಬೇಕು?” ಅದಕ್ಕೆ ಸ್ವಾಮೀಜಿ ತಕ್ಷಣ ಉತ್ತರಿಸಿದರು, “ಅನುಮಾನವೇಕೆ? ನಾವು ಆ ಬಲಿಷ್ಠರನ್ನು ಬಗ್ಗು ಬಡಿಯಬೇಕು!” ಆದರೆ ಅವರು ಈ ಮಾತನ್ನು ಯಾರಿಗೆ ಹೇಳಿದರೆಂಬುದು ಇನ್ನೂ ಮುಖ್ಯವಾದ, ಸ್ವಾರಸ್ಯಕರವಾದ ಅಂಶ. ಸೋದರಿ ನಿವೇದಿತಾ ಹೇಳುತ್ತಾಳೆ, “ಆ ಪ್ರಶ್ನೆ ಕೇಳಿದವನು ಒಬ್ಬ ವೃದ್ಧ. ಅವನ ಮುಖದಲ್ಲಿ ಕರುಣಾಪೂರ್ಣವಾದ ದೌರ್ಬಲ್ಯ ಕಾಣುತ್ತಿತ್ತು” ಎಂದು. ಸ್ವಾಮೀಜಿ ದೌರ್ಬಲ್ಯವನ್ನು ದ್ವೇಷಿಸುವವರು; “ಶಕ್ತಿಯೇ ಜೀವನ, ದೌರ್ಬಲ್ಯವೇ ಮರಣ” ಎಂದು ಗರ್ಜಿಸಿದವರು. ಆದ್ದರಿಂದಲೇ ಅವರು ಆ ಮುದುಕನಿಗೆ ಹಾಗೆ ಹೇಳಿದುದು. ಇಂತಹದೇ ಇನ್ನೊಂದು ಸಂದರ್ಭದಲ್ಲಿ ಅವರು ಹೇಳಿದರು, “ದುರ್ಬಲನ ಕ್ಷಮೆಗೆ ಅರ್ಥವಿಲ್ಲ. ಅದಕ್ಕಿಂತ, ಪ್ರತಿಭಟಿಸುವುದೇ ಮೇಲು. ನಿಮಗೆ ದೇವತೆಗಳ ದಂಡನ್ನೇ ದಮನಗೊಳಿಸುವ ಸಾಮರ್ಥ್ಯ ಬಂದಾಗ ಕ್ಷಮಿಸುವಿರಂತೆ. ಈ ಜಗತ್ತು ಒಂದು ರಣರಂಗ; ಇಲ್ಲಿ ನೀವು ಹೋರಾಡಿ ನಿಮ್ಮ ಗುರಿ ಸೇರಿ.” ಮತ್ತೊಮ್ಮೆ ಒಬ್ಬರು ಕೇಳಿದರು, “ನ್ಯಾಯವನ್ನು ರಕ್ಷಿಸುವುದಕ್ಕಾಗಿ ಪ್ರಾಣವನ್ನೇ ತ್ಯಜಿಸುವುದು ಸರಿಯೆ, ಅಥವಾ ಯಾವುದೇ ಪ್ರತಿಕ್ರಿಯೆ ತೋರದೆ ಮೌನವಾಗಿದ್ದುಬಿಡುವುದು ಸರಿಯೆ?” ಅದಕ್ಕೆ ಸ್ವಾಮೀಜಿ ನಿಧಾನವಾಗಿ ಹೇಳಿದರು, “ಪ್ರತಿಕ್ರಿಯೆ ತೋರಬಾರದೆಂಬುದು ನನ್ನ ಅಭಿಪ್ರಾಯ.” ಹೀಗೆ ಹೇಳಿ ಸ್ವಲ್ಪ ಹೊತ್ತು ಸುಮ್ಮನಿದ್ದು ಬಳಿಕ ಮತ್ತೆ ಹೇಳಿದರು, “–ಅದು ಸಂನ್ಯಾಸಿಗೆ ಅನ್ವಯಿಸುವಂಥದು. ಆದರೆ ಗೃಹಸ್ಥರಾದರೆ, ಪ್ರತಿಭಟಿಸುವುದೇ ಯುಕ್ತವಾದದ್ದು.”

ಸ್ವಾಮೀಜಿ ಕಾಶ್ಮೀರದಲ್ಲಿದ್ದ ಈ ದಿನಗಳಲ್ಲಿ ಅವರನ್ನೂ ಅವರ ಶಿಷ್ಯೆಯರನ್ನೂ ಕಾಶ್ಮೀರದ ಮಹಾರಾಜ ಆದರದಿಂದ ನೋಡಿಕೊಂಡ. ಅಲ್ಲದೆ ಅನೇಕ ಉನ್ನತ ಅಧಿಕಾರಿಗಳು ಅವರ ದೋಣಿಮನೆಗೇ ಬಂದು ಅವರೊಡನೆ ನಾನಾ ವಿಷಯಗಳ ಬಗ್ಗೆ ಸಂಭಾಷಿಸಿದರು. ಈ ಸಂದರ್ಭ ದಲ್ಲೇ ಕಾಶ್ಮೀರದ ಮಹಾರಾಜ ಸ್ವಾಮೀಜಿಯವರಿಗೆ ಮಠ ನಿರ್ಮಾಣಕ್ಕಾಗಿ ಮತ್ತು ಒಂದು ಸಂಸ್ಕೃತ ಕಾಲೇಜನ್ನು ತೆರೆಯುವುದಕ್ಕಾಗಿ ಸರಿಯಾದ ಸ್ಥಳವೊಂದನ್ನು ಆರಿಸಿಕೊಳ್ಳುವಂತೆ ವಿನಂತಿಸಿಕೊಂಡ. ಅಲ್ಲಿನ ನದಿಯ ದಡದ ಮೇಲೊಂದು ಸುಂದರವಾದ ಸ್ಥಳವಿತ್ತು. ಅದು ಐರೋಪ್ಯರು ವಿಹಾರಕ್ಕಾಗಿ ಉಪಯೋಗಿಸಿಕೊಳ್ಳುತ್ತಿದ್ದ ಸ್ಥಳ. ಇದನ್ನು ಸ್ವಾಮೀಜಿ ತಮ್ಮ ಕಾರ್ಯೋದ್ದೇಶಕ್ಕಾಗಿ ಆರಿಸಿಕೊಂಡರು. ಇದಕ್ಕೆ ಮಹಾರಾಜ ಸಂತೋಷದಿಂದ ಒಪ್ಪಿಕೊಂಡು ಬಿಟ್ಟುಕೊಡಲು ಸಿದ್ಧನಾದ.

ಸ್ವಾಮೀಜಿ ಅಮರನಾಥದಿಂದ ಹಿಂದಿರುಗಿದ ಮೇಲೆ ಅವರ ಶಿಷ್ಯೆಯರ ಮನಸ್ಸಿನಲ್ಲಿ ಏಕಾಂತದಲ್ಲಿದ್ದುಕೊಂಡು ಧ್ಯಾನಾಭ್ಯಾಸ ಮಾಡುವ ಉತ್ಸಾಹದ ಅಲೆಯೆದ್ದಿತ್ತು. ಸಹವಾಸದ ಪರಿಣಾಮವೆಂದರೆ ಇದೇ. ಸ್ವಾಮೀಜಿ ಆಗಾಗ ಏಕಾಂತಕ್ಕೆ ಹೋಗಿ ಧ್ಯಾನಮಗ್ನರಾಗುತ್ತಿದ್ದುದನ್ನು ಕಂಡು ಆಗ ಅವರ ಶಿಷ್ಯೆಯರಿಗೂ ಆ ಹುಚ್ಚು ಹಿಡಿಯಿತು! ಸ್ವಾಮೀಜಿ ಅವರನ್ನು ಪ್ತೋತ್ಸಾಹಿ ಸುತ್ತ, “ಬಹಳ ಒಳ್ಳೆಯದು. ಏಕಾಂತವಾಸ ಬಹಳ ಸಹಾಯಕಾರಿ. ನಾವು ಈಗ ತಾನೆ ಆರಿಸಿ ಕೊಂಡಿರುವ ಆ ಹೊಸ ಸ್ಥಳಕ್ಕೆ ಹೋಗಿ. ಅಲ್ಲಿದ್ದುಕೊಂಡು ಧ್ಯಾನಾಭ್ಯಾಸ ಮಾಡಿ. ಅಲ್ಲದೆ ಹಿಂದೂ ಸಂಪ್ರದಾಯದ ಪ್ರಕಾರ ಹೊಸ ಮನೆ ಕಟ್ಟುವ ಜಾಗಕ್ಕೆ ಸ್ತ್ರೀಯರು ಹೋಗಿ ಪವಿತ್ರ ಗೊಳಿಸಬೇಕು” ಎಂದರು. ಹೀಗೆ ಆ ಶಿಷ್ಯೆಯರು ಆ ಹೊಸ ಸ್ಥಳದಲ್ಲಿ ವಾಸವಾಗಿರುವುದರ ಮೂಲಕ ಅಲ್ಲೊಂದು ಸ್ತ್ರೀ ಮಠವೇ ಪ್ರಾರಂಭವಾಯಿತೋ ಎಂಬಂತಿತ್ತು.

ಮಠದ ನಿರ್ಮಾಣಕ್ಕಾಗಿ, ಕಾಶ್ಮೀರದ ಮಹಾರಾಜನೇನೋ ನಿವೇಶನವನ್ನು ಬಿಟ್ಟುಕೊಡಲು ಸಿದ್ಧನಿದ್ದ. ಆದರೆ ಇದಕ್ಕೆ ಅಲ್ಲಿನ ಬ್ರಿಟಿಷ್ ಅಧಿಕಾರಿಯ ಅನುಮತಿ ಬೇಕಾಗಿತ್ತು. ಈ ವೇಳೆಗೆ ಕಾಶ್ಮೀರದಲ್ಲಿ ಸ್ವಾಮೀಜಿಯವರ ವಿರುದ್ಧ ಅಲ್ಲಿನ ಕ್ರೈಸ್ತಪಾದ್ರಿಗಳು ಹುಯಿಲೆಬ್ಬಿಸುತ್ತಿದ್ದರು. ಈಗ ಮಠ ನಿರ್ಮಾಣಕ್ಕೆ ಆ ಅಧಿಕಾರಿ ಅನುಮತಿ ನೀಡಿದ್ದರೆ ಅವನು ಕ್ರೈಸ್ತಮತೀಯರ ಆಕ್ರೋಶಕ್ಕೆ ತುತ್ತಾಗಬೇಕಾಗಿತ್ತು. ಆದ್ದರಿಂದ ಅವನು ಎರಡು ಸಲ ಮಹಾರಾಜನ ಮನವಿಯನ್ನು ತಳ್ಳಿಹಾಕಿದ. ನಮ್ಮ ದೇಶ ಗುಲಾಮಗಿರಿಯಲ್ಲಿದ್ದ ಕಾಲವಲ್ಲವೆ ಅದು! ಭಾರತೀಯ ರಾಜನಿಗೆ, ಭಾರತದ ಕಾರ್ಯಕ್ಕಾಗಿ, ಭಾರತೀಯರೊಬ್ಬರಿಗೆ ಭಾರತದ ಸ್ಥಳವೊಂದನ್ನು ಬಿಟ್ಟುಕೊಡುವುದಕ್ಕೆ ಸಾಧ್ಯವಾಗದೆ ಹೋಯಿತು. ಇದರಿಂದ ಸ್ವಾಮೀಜಿಯವರಿಗೆ ಮೊದಲು ಸ್ವಲ್ಪ ನಿರಾಸೆಯಾದರೂ ಆಮೇಲೆ ಆ ವಿಷಯವಾಗಿ ಸ್ವಲ್ಪ ಆಲೋಚಿಸಿದಾಗ ಅವರಿಗನ್ನಿಸಿತು–ತಮ್ಮ ಕಾರ್ಯಕ್ಕೆ ಈ ಜಾಗ ಅಷ್ಟೇನೂ ಸಮರ್ಪಕವಾದದ್ದಲ್ಲ ಎಂದು. ತಾವು ಯಾವ ಶೈಕ್ಷಣಿಕ ಕಾರ್ಯಯೋಜನೆ ಯನ್ನು ಹಾಕಿಕೊಂಡಿದ್ದರೊ ಅದನ್ನು ಕಾರ್ಯಗತಗೊಳಿಸಲು ಕಲ್ಕತ್ತವೇ ತಕ್ಕ ಸ್ಥಳವಲ್ಲದೆ ದೂರದ ಕಾಶ್ಮೀರವಲ್ಲ ಎಂದು ಅವರಿಗನ್ನಿಸಿತು. ಕಡೆಗೆ ಈ ಅಡೆತಡೆಗಳೆಲ್ಲವೂ ಜಗನ್ಮಾತೆಯ ಇಚ್ಛೆಯೇ ಎಂದು ಭಾವಿಸಿ ಸುಮ್ಮನಾದರು. ಆದರೆ ಸೇವಿಯರ್ ದಂಪತಿಗಳು ನೆಲೆ ನಿಲ್ಲಲು ತಂಪಾದ ಸ್ಥಳವೊಂದು ಬೇಕೇ ಬೇಕಾಗಿತ್ತು. ಅವರು ಸೂಕ್ತ ಸ್ಥಳವೊಂದನ್ನು ಹುಡುಕಲು ಹಿಮಾಲಯದ ಕುಮಾವೂ ಪರ್ವತ ಪ್ರದೇಶಗಳಲ್ಲಿ ತಪಾಸಣೆ ನಡೆಸಿದ್ದರು. ಆದ್ದರಿಂದ ಈಗ ಸ್ವಾಮೀಜಿ ಆ ಜವಾಬ್ದಾರಿಯನ್ನು ಅವರಿಗೇ ಒಪ್ಪಿಸಿಬಿಟ್ಟರು.

ಸ್ವಾಮೀಜಿ ತಮ್ಮ ಶಿಷ್ಯೆಯರೊಂದಿಗೆ ಝೇಲಂ ನದಿಯ ಮೇಲೆ ವಾಸವಾಗಿದ್ದ ಈ ದಿನಗಳಲ್ಲಿ ತಮ್ಮ ಅಮೆರಿಕನ್ ಶಿಷ್ಯೆ ಕ್ರಿಸ್ಟೀನಳಿಗೆ ಅವರು ಬರೆದ ಒಂದು ಪತ್ರ ಸ್ವಾರಸ್ಯಕರವಾಗಿದೆ:

“... ಈಗ ನಾನು ಕಳೆದ ಎರಡು ತಿಂಗಳಿನಿಂದ ನಡೆಸುತ್ತಿರುವುದು ಸೋಮಾರಿ ಜೀವನವನ್ನು. ಇಲ್ಲಿನ ಸುಂದರವಾದ ಝೇಲಮ್ಮಿನ ಜಲದ ಮೇಲೆ ನಾನು ದೋಣಿಯೊಂದರಲ್ಲಿ–ಈ ದೋಣಿ ನನ್ನ ಮನೆಯೂ ಹೌದು–ಆರಾಮವಾಗಿ ತೇಲುತ್ತಿದ್ದೇನೆ. ಈ ಪ್ರಕೃತಿಯ ಉದ್ಯಾನದಲ್ಲಿ ಭಗವಂತದನ ಸೃಷ್ಟಿಯು ಒದಗಿಸಬಹುದಾದ ಅತ್ಯದ್ಭುತ ದೃಶ್ಯಗಳಿವೆ, ಈ ಝೇಲಂ ನದಿಯ ಸುತ್ತ. ಇಲ್ಲಿನ ಗಾಳಿ-ಭೂಮಿ-ಹುಲ್ಲುಗಿಡ-ಮರ-ಪರ್ವತ-ಹಿಮ-ಮಾನವಾಕೃತಿ–ಎಲ್ಲವೂ ಕನಿಷ್ಠ ಪಕ್ಷ ಭಗವತ್ಸೌಂದರ್ಯದ ಹೊರಮೈಯನ್ನಾದರೂ ಅಭಿವ್ಯಕ್ತಿಗೊಳಿಸುತ್ತಿವೆ. ಈಗ ನನ್ನ ಹತ್ತಿರ ಏನೂ ಸಾಮಾನುಗಳೇ ಇಲ್ಲ ಎನ್ನಬಹುದು–ಒಂದು ಲೇಖನಿ ಅಥವಾ ಒಂದು ಮಸಿಕುಡಿಕೆ ಕೂಡ ಇದ್ದರಿತ್ತು ಇಲ್ಲದಿದ್ದರಿಲ್ಲ. ಎಲ್ಲೆಂದರಲ್ಲಿ ಯಾವಾಗೆಂದರಾವಾಗ ಒಂದು ಊಟವನ್ನು ದೊರಕಿಸಿಕೊಳ್ಳುವೆ... 

“ನೀನು ಅತಿಯಾಗಿ ದುಡಿದು ದೇಹದಂಡನೆ ಮಾಡಬೇಡ. ಅದರಿಂದೇನೂ ಪ್ರಯೋಜನವಿಲ್ಲ. ಯಾವಾಗಲೂ ನೆನಪಿಡು–‘ಕರ್ತವ್ಯವೆಂಬುದು ಮಾನವತೆಯ ಜೀವಾಳವನ್ನೇ ದಹಿಸುತ್ತಿರುವ ನಡುನೆತ್ತಿಯ ಸೂರ್ಯನ ಉಗ್ರ ಕಿರಣಗಳಿದ್ದಂತೆ.’ ಅದು ಒಂದು ಕಾಲದಲ್ಲಿ ಒಂದು ಶಿಸ್ತಾಗಿ ಮಾತ್ರ ಆವಶ್ಯಕ; ಅದರಿಂದಾಚೆಗೆ ಅದೊಂದು ಕೆಟ್ಟ ರೋಗ... ”

