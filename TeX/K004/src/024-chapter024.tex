
\chapter{ಎರಡು ಶಕ್ತಿಗಳ ಸಂಗಮ}

\noindent

ಮಠದಲ್ಲಿ ನಿರಂತರವಾಗಿ ಶಾಸ್ತ್ರಾಧ್ಯಯನ ನಡೆಯುವಂತಾಗಬೇಕೆಂಬುದು ಸ್ವಾಮೀಜಿಯವರ ಆಕಾಂಕ್ಷೆ. ಮಠವು ನೀಲಾಂಬರ ಮುಖರ್ಜಿಯವರ ತೋಟದ ಮನೆಯಲ್ಲಿದ್ದಾಗಲೇ ಅವರು ವೇದೋಪನಿಷತ್ತುಗಳು, ವೇದಾಂತಸೂತ್ರಗಳು, ಭಗವದ್ಗೀತೆ ಮೊದಲಾದವುಗಳ ಅಧ್ಯಯನ ತರಗತಿಗಳನ್ನು ಪ್ರಾರಂಭಿಸಿದ್ದರು. ಅಲ್ಲದೆ ಸ್ವತಃ ತಾವೇ ಕೆಲದಿನ ಪಾಣಿನಿಯ ‘ಅಷ್ಟಾಧ್ಯಾಯಿ’ ಯನ್ನು ಬೋಧಿಸಿದ್ದರು. ಈಗ ಅವರು ಮತ್ತೆ ಸಂಸ್ಕೃತ ಸಾಹಿತ್ಯದ ಹಾಗೂ ಶಾಸ್ತ್ರಗ್ರಂಥಗಳ ಸ್ವಾಧ್ಯಾಯ-ಪ್ರವಚನಗಳಲ್ಲಿ ತೊಡಗಿದರು. ಈ ಅವಧಿಯಲ್ಲೇ ಅವರು ಶ್ರೀರಾಮಕೃಷ್ಣರ ಕುರಿತ ಎರಡು ಪ್ರಸಿದ್ಧ ಶ್ಲೋಕಗಳನ್ನು ರಚಿಸಿದರು. ಇವುಗಳಲ್ಲಿ ಒಂದು ‘ಶ್ರೀರಾಮಕೃಷ್ಣ ಸ್ತುತಿ’. ಇದನ್ನು ಎಲ್ಲ ರಾಮಕೃಷ್ಣ ಮಠಗಳಲ್ಲೂ ಪ್ರತಿದಿನ ಸಂಜೆ ಆರತಿಯ ಬಳಿಕ ಹಾಡಲಾಗುತ್ತಿದೆ. ಮತ್ತೊಂದು, ‘ಆಚಂಡಾಲಾಪ್ರತಿಹತರಯೋ... ’ ಎಂದು ಪ್ರಾರಂಭವಾಗುವ ಶ್ಲೋಕ.

ಈ ದಿನಗಳಲ್ಲಿ ಸ್ವಾಮೀಜಿಯವರ ದರ್ಶನಕ್ಕಾಗಿ ಜನರು ದೂರದೂರದಿಂದ ಬರುತ್ತಿದ್ದರು. ಹಿಂದೆ ಅವರು ಗೋಪಾಲಲಾಲ್ ಸೀಲರ ಮನೆಯಲ್ಲಿದ್ದಾಗ ನಡೆಯುತ್ತಿದ್ದಂತೆಯೇ ಈಗ ಮತ್ತೆ ದರ್ಶನಾರ್ಥಿಗಳೊಂದಿಗೆ ಧರ್ಮ, ವೇದಾಂತ, ರಾಷ್ಟ್ರದ ಪುನರುತ್ಥಾನ ಮುಂತಾದ ವಿವಿಧ ವಿಷಯಗಳ ಬಗ್ಗೆ ಎಡೆಬಿಡದೆ ಮಾತುಕತೆ ನಡೆಯುತ್ತಿತ್ತು. ಒಮ್ಮೆ ಅವರು ಶ್ರೀರಾಮಕೃಷ್ಣರ ಪ್ರಮುಖ ಗೃಹೀಭಕ್ತರಲ್ಲೊಬ್ಬನಾದ ರಾಮಚಂದ್ರ ದತ್ತನ ಭೇಟಿಗಾಗಿ ಅವನ ಮನೆಗೆ ಹೋದರು. ಯುವಕ ನರೇಂದ್ರನನ್ನು ಶ್ರೀರಾಮಕೃಷ್ಣರ ಬಳಿಗೆ ಕರೆತಂದವನು ಆತನೇ ಎಂಬುದನ್ನು ನೆನಪಿಸಿ ಕೊಳ್ಳಬಹುದು. ಈಗ ಕೆಲಕಾಲದಿಂದಲೂ ಅವನು ಕಾಯಿಲೆಯಿಂದ ಹಾಸಿಗೆ ಹಿಡಿದಿದ್ದ. ಆದ್ದ ರಿಂದ ಒಮ್ಮೆ ಆತನನ್ನು ನೋಡಿಕೊಂಡು ಹೋಗಲು ಸ್ವಾಮೀಜಿ ಬಂದದ್ದು. ಆದರೆ ರಾಮಚಂದ್ರ ದತ್ತ ಮತ್ತು ಇನ್ನಿತರ ಕೆಲವು ಗೃಹೀಭಕ್ತರು ಈಚೆಗೆ ಈ ಸೋದರ ಸಂನ್ಯಾಸಿಗಳೊಂದಿಗೆ ಸಾಮರಸ್ಯದಿಂದಿರಲಿಲ್ಲ. ಏಕೆಂದರೆ, ಈ ಸಂನ್ಯಾಸಿಗಳು ಗೃಹಸ್ಥರಾದ ತಮ್ಮನ್ನೆಲ್ಲ ಕಡೆಗಣಿಸಿದ್ದಾರೆ ಎಂಬುದು ಅವರ ಭಾವನೆಯಾಗಿತ್ತು. ಈಗ ಸ್ವಾಮೀಜಿಯವರು ಆಗಮಿಸಿ ತುಂಬ ವಿಶ್ವಾಸದಿಂದ ರಾಮಚಂದ್ರನೊಡನೆ ಮಾತುಕತೆಯಾಡಿದರು. ಇದರಿಂದ ಆತನಿಗೆ ಎಷ್ಟೋ ಸಮಾಧಾನವಾ ಯಿತು. ಸ್ವಲ್ಪಹೊತ್ತಿನ ಮೇಲೆ ವೃದ್ಧ ರಾಮಚಂದ್ರ ಎದ್ದು ಶೌಚಕ್ಕೆಂದು ಹೊರಟ ತಕ್ಷಣ ಸ್ವಾಮೀಜಿ ಪಕ್ಕದಲ್ಲಿದ್ದ ಆತನ ಚಪ್ಪಲಿಗಳನ್ನು ಮುಂದಿಟ್ಟು ಅವನ್ನು ಹಾಕಿಕೊಳ್ಳುವಲ್ಲಿ ಆತನಿಗೆ ನೆರವಾದರು. ವಿಶ್ವವಿಖ್ಯಾತ ವಿವೇಕಾನಂದರು ತನ್ನಂತಹ ಒಬ್ಬ ಸಾಮಾನ್ಯ ಗೃಹಸ್ಥನ ಚಪ್ಪಲಿ ಯನ್ನು ಮುಟ್ಟಿದ್ದನ್ನು ಕಂಡ ರಾಮಚಂದ್ರನ ಹೃದಯ ತುಂಬಿಬಂತು, ಕಣ್ಣಲ್ಲಿ ಅಶ್ರು ಚಿಮ್ಮಿತು; ಗದ್ಗದ ಸ್ವರದಲ್ಲಿ ಉದ್ಗರಿಸಿದ: “ನರೇನ್, ನರೇನ್! ಇದೇನು ಮಾಡಿದೆಯಪ್ಪ? ನನ್ನಂಥವನ ಚಪ್ಪಲಿಯನ್ನು ಮುಟ್ಟಿಬಿಟ್ಟೆಯಲ್ಲ! ಜನ ನನ್ನ ಹತ್ತಿರ ಹೇಳುತ್ತಿದ್ದರು, ನೀನು ಅಮೆರಿಕಕ್ಕೆ ಹೋಗಿ ದೊಡ್ಡಮನುಷ್ಯನಾಗಿ ಬಂದ ಮೇಲೆ ನಮ್ಮನ್ನೆಲ್ಲ ಮರೆತು ಬಿಟ್ಟೆ ಅಂತ. ಆದರೆ ಈಗ ನೋಡಿದರೆ ನೀನು ಅದೇ ನರೇನನೇ! ನಾನು ನಿನ್ನನ್ನು ತಪ್ಪಾಗಿ ತಿಳಿದುಕೊಂಡಿದ್ದೆ. ಈಗ ನನಗೆ ಅರ್ಥವಾಗುತ್ತಿದೆ ಶ್ರೀರಾಮಕೃಷ್ಣರು ನಿನ್ನನ್ನು ಒಂದು ಅನರ್ಘ್ಯ ರತ್ನ ಅಂತ ಕರೆಯುತ್ತಿದ್ದುದು ಏಕೆ ಎನ್ನುವುದು. ದಯವಿಟ್ಟು ನನ್ನನ್ನು ಮನ್ನಿಸಪ್ಪ. ನೀನು ಹಾಗೆ ಮಾಡಿದರೆ ಮಾತ್ರ ಶ್ರೀಗುರು ಮಹಾರಾಜರು ನನ್ನನ್ನು ತಮ್ಮ ಬಳಿಗೆ ಕರೆದುಕೊಳ್ಳುವಂತಾದೀತು.” ಆಗ ಸ್ವಾಮೀಜಿಯವರು, “ರಾಮ್​ದಾದಾ, ಇದೇನು ಹೇಳುತ್ತಿದ್ದೀರಿ ನೀವು! ನೀವು ಶ್ರೀರಾಮಕೃಷ್ಣರ ಆಪ್ತಭಕ್ತರು. ನಿಮಗೆ ಅವರ ಬಳಿ ಸ್ಥಾನವಿಲ್ಲದೆಯಿರುತ್ತದೆಯೆ?” ಎಂದು ಸಂತೈಸಲು ನೋಡಿದರು. ಆದರೆ ರಾಮ ಚಂದ್ರ ಬಿಕ್ಕುತ್ತಲೇ “ಇಲ್ಲ, ಇಲ್ಲ, ನೀನು ನನ್ನನ್ನು ಕ್ಷಮಿಸುವವರೆಗೂ ಗುರುಮಹಾರಾಜರು ನನ್ನನ್ನು ಸ್ವೀಕರಿಸುವುದಿಲ್ಲ” ಎಂದ. ಆಗ ಸ್ವಾಮೀಜಿಯೆಂದರು, “ರಾಮ್​ದಾದಾ, ನೀವು ಹಾಗೆಲ್ಲ ಚಿಂತಿಸಬೇಕಾದ ಕಾರಣವೇ ಇಲ್ಲ. ನನಗೆ ಅವರ ಬಳಿ ಸ್ಥಾನವಿರುವುದಾದರೆ ನಿಮಗೂ ಖಂಡಿತ ಇರುತ್ತದೆ.” ಅಂತೂ ಕಡೆಗೆ ರಾಮಚಂದ್ರನಿಗೆ ಸಮಾಧಾನವಾಯಿತು. ಬಳಿಕ ಸ್ವಾಮೀಜಿ ಅವ ನಿಂದ ಬೀಳ್ಕೊಂಡು ಹೊರಟರು. ಇದಾದ ಕೆಲದಿನಗಳಲ್ಲೇ ರಾಮಚಂದ್ರ ಕೊನೆಯುಸಿರೆಳೆದ.

ಈ ದಿನಗಳಲ್ಲೇ ಸ್ವಾಮೀಜಿಯವರನ್ನು ಕಾಣಲು ಬಂದ ಪ್ರಮುಖ ವ್ಯಕ್ತಿಗಳಲ್ಲಿ ನಾಗಮಹಾ ಶಯರೊಬ್ಬರು. ಇವರಿಬ್ಬರ ನಡುವಿನ ಭೇಟಿಯು ಅತ್ಯಂತ ಸ್ಮರಣೀಯವಾದದ್ದು. ಯಾರು ಈ ನಾಗಮಹಾಶಯರು? ಇವರು ಶ್ರೀರಾಮಕೃಷ್ಣರ ಶ್ರೇಷ್ಠತಮ ಗೃಹಸ್ಥಭಕ್ತರು. ಸ್ವಾಮಿ ವಿವೇಕಾ ನಂದರು ಶ್ರೀರಾಮಕೃಷ್ಣರ ಸಂನ್ಯಾಸೀ ಶಿಷ್ಯರಲ್ಲಿ ಮಕುಟಮಣಿಯಾದರೆ ನಾಗಮಹಾಶಯರು ಗೃಹೀಭಕ್ತರಲ್ಲಿ ಚೂಡಾಮಣಿ. ಇವರು ದೂರದ ಢಾಕಾ ಜಿಲ್ಲೆಯ ತಮ್ಮ ಹುಟ್ಟೂರಾದ ದೇವಭೋಗದಿಂದ ಸ್ವಾಮೀಜಿಯವರ ದರ್ಶನಕ್ಕಾಗಿಯೇ ಬಂದಿದ್ದರು. ಆ ಸನ್ನಿವೇಶ ಎರಡು ಮಹಾಚೇತನಗಳ ಸಂಗಮದಂತಿತ್ತು. ಎರಡು ಮಹಾಶಕ್ತಿಗಳ ಸಮ್ಮಿಲನದಂತಿತ್ತು. ಒಂದು ಆದರ್ಶ ಗೃಹಸ್ಥಧರ್ಮದ ಶಕ್ತಿ, ಇನ್ನೊಂದು ಆದರ್ಶ ಸಂನ್ಯಾಸಧರ್ಮದ ಶಕ್ತಿ. ಅವರಲ್ಲಿ ಒಬ್ಬರು ಪುರಾತನ ಗೃಹಸ್ಥಧರ್ಮದ ಅತ್ಯುನ್ನತ ಆದರ್ಶವನ್ನು ಪ್ರತಿನಿಧಿಸುತ್ತಿದ್ದರೆ ಇನ್ನೊಬ್ಬರು ವಿನೂತನ ವಿಧಾನದ ಸಂನ್ಯಾಸದ ಆದರ್ಶವನ್ನು ಪ್ರತಿನಿಧಿಸುತ್ತಿದ್ದಾರೆ! ಒಬ್ಬರು ಭಗವದ್ಭಾವದಿಂದ ಹುಚ್ಚರಾಗಿದ್ದರೆ ಮತ್ತೊಬ್ಬರು ಮನುಷ್ಯನೊಳಗಿನ ದೈವತ್ವವನ್ನು ಹೊರತರುವ ಮಹಾಕಾರ್ಯ ದಲ್ಲಿ ಉನ್ಮತ್ತರಾಗಿದ್ದಾರೆ! ಆದರೆ ಆಂತರ್ಯದಲ್ಲಿ ಸಂನ್ಯಾಸ-ಸಾಕ್ಷಾತ್ಕಾರಗಳ ಸಮಾನ ಭಾವ ವನ್ನು ಹೊಂದಿದ್ದಾರೆ; ಎಂದರೆ, ಇಬ್ಬರೂ ಆಂತರ್ಯದಲ್ಲಿ ಸಂನ್ಯಾಸಿಗಳೇ ಮತ್ತು ಸಾಕ್ಷಾತ್ಕಾರ ಹೊಂದಿದವರೇ.

ನಾಗಮಹಾಶಯರು ಸ್ವಾಮೀಜಿಯವರನ್ನು ನೋಡಲು ಅವರಿದ್ದಲ್ಲಿಗೇ ಬಂದರು. ಆಗ ಉಪನಿಷತ್ತಿನ ತರಗತಿ ನಡೆಯುತ್ತಿತ್ತು. ನಾಗಮಹಾಶಯರನ್ನು ಕಂಡು ಸ್ವಾಮೀಜಿ ಆನಂದಾ ಶ್ಚರ್ಯದಿಂದ ಎದ್ದುನಿಂತು ನಮಸ್ಕರಿಸಿ “ದಯಮಾಡಿಸಬೇಕು!” ಎಂದು ಸ್ವಾಗತಿಸಿದರು. ನಾಗಮಹಾಶಯರು ಆವೇಶಭರಿತರಾಗಿ, “ಜೈ ಶಂಕರ! ಜೀವಂತ ಶಿವನನ್ನು ಕಣ್ಣೆದುರಿಗೇ ಕಾಣುತ್ತಿರುವ ನಾನೇ ಧನ್ಯ!” ಎಂದು ಉದ್ಗರಿಸುತ್ತ ಹಾಗೆಯೇ ಕೈಜೋಡಿಸಿ ನಿಂತುಬಿಟ್ಟರು. ಸ್ವಾಮೀಜಿಯವರು ನಾಗಮಹಾಶಯರ ಆರೋಗ್ಯದ ಕುರಿತಾಗಿ ವಿಚಾರಿಸಿದಾಗ ಅವರು “ಈ ಕೆಲಸಕ್ಕೆ ಬಾರದ ಮೂಳೆ ಮಾಂಸಗಳ ಕಂತೆಯ ವಿಷಯ ಅದೇನು ಕೇಳುತ್ತೀರಿ ಬಿಡಿ. ಸಾಕ್ಷಾತ್ ಶಿವನನ್ನೇ ಕಾಣುತ್ತ ನಾನು ಆನಂದಭರಿತನಾಗಿಬಿಟ್ಟಿದ್ದೇನೆ!” ಎನ್ನುತ್ತ ಸ್ವಾಮೀಜಿಯವರ ಮುಂದೆ ದೀರ್ಘದಂಡ ಪ್ರಣಾಮ ಮಾಡಿದರು. ತಕ್ಷಣ ಸ್ವಾಮೀಜಿ ಅವರನ್ನು ಮೇಲೆಬ್ಬಿಸಿದರು. ಬಳಿಕ ತಮ್ಮ ಶಿಷ್ಯರತ್ತ ತಿರುಗಿ, “ಇಂದಿನ ತರಗತಿ ಇಷ್ಟಕ್ಕೇ ಸಾಕು. ಎಲ್ಲರೂ ಬನ್ನಿ, ನಾಗಮಹಾ ಶಯರ ದರ್ಶನ ಮಾಡಿ” ಎಂದರು. ಈಗ ಎಲ್ಲರೂ ಬಂದು ಆ ಮಹಾಭಕ್ತನ ಸುತ್ತ ಕುಳಿತುಕೊಂಡರು.

ಸ್ವಾಮೀಜಿಯೆಂದರು, “ನೋಡಿ, ಇವರು ಹೆಸರಿಗೆ ಒಬ್ಬ ಗೃಹಸ್ಥರು. ಆದರೆ ಇವರಿಗೆ ತಮ ಗೊಂದು ಶರೀರ ಇದೆಯೋ ಇಲ್ಲವೋ ಎಂಬುದರ ಪ್ರಜ್ಞೆಯೂ ಇಲ್ಲ; ಈ ಪ್ರಪಂಚ ಎಂಬು ದೊಂದು ಇದೆಯೋ ಇಲ್ಲವೋ ಎನ್ನುವ ಪ್ರಜ್ಞೆಯೂ ಇಲ್ಲ! ಇವರು ಸದಾ ಭಗವಚ್ಚಿಂತನೆ ಯಲ್ಲೇ ಮುಳುಗಿರುವವರು. ಮನುಷ್ಯನು ಪರಾಭಕ್ತಿಯನ್ನು ಹೊಂದಿದಾಗ ಯಾವ ಅವಸ್ಥೆಗೆ ಏರುತ್ತಾನೆ ಎಂಬುದಕ್ಕೆ ಇವರೇ ಜ್ವಲಂತ ಸಾಕ್ಷ್ಯ.” ಅನಂತರ ನಾಗಮಹಾಶಯರತ್ತ ತಿರುಗಿ, “ನಮಗೆ ಶ್ರೀಗುರುಮಹಾರಾಜರ ವಿಷಯ ಏನಾದರೂ ಸ್ವಲ್ಪ ಹೇಳಿ” ಎಂದರು. ಆದರೆ ನಾಗ ಮಹಾಶಯರು ತಮಗೆ ಸಹಜವಾದ ವಿನಯದಿಂದ, ದೈನ್ಯದಿಂದ ಹೇಳಿದರು, “ನಾನು ಏನು ತಾನೆ ಹೇಳಲಿ! ಅವರ ಬಗ್ಗೆ ಮಾತನಾಡಲು ನಾನು ತೀರ ಅಯೋಗ್ಯ. ಆ ಮಹಾ ಲೀಲಾನಾಟಕದಲ್ಲಿ ಶ್ರೀರಾಮಕೃಷ್ಣರ ಪಾತ್ರಧರಿಸಿ ಬಂದ ಸ್ವಯಂ ಭಗವಂತನ ಲೀಲಾಸಹಚರನಾದ ಮಹಾವೀರ (ಆಂಜನೇಯನ) ದರ್ಶನದಿಂದ ಪುನೀತನಾಗಲು ಮಾತ್ರ ನಾನಿಲ್ಲಿಗೆ ಬಂದಿದ್ದೇನೆ. ಅವನಿಗೆ ಜಯವಾಗಲಿ! ಅವನಿಗೆ ಜಯವಾಗಲಿ!”

ಸ್ವಾಮೀಜಿ: “ಶ್ರೀಗುರುಮಹಾರಾಜರನ್ನು ಸರಿಯಾಗಿ ಅರ್ಥಮಾಡಿಕೊಂಡವರೆಂದರೆ ನೀವೇ. ನಾವೆಲ್ಲ ಸುಮ್ಮನೆ ಕತ್ತಲಲ್ಲಿ ತಡಕಾಡುತ್ತಿದ್ದೇವೆ, ಅಷ್ಟೆ.”

ನಾಗಮಹಾಶಯರು: “ದಯವಿಟ್ಟು ಇಂತಹ ಅರ್ಥಹೀನ ಮಾತುಗಳನ್ನಾಡಬಾರದು. ನೀವು ಶ್ರೀರಾಮಕೃಷ್ಣರ ಪ್ರತಿಬಿಂಬ. ನೀವಿಬ್ಬರೂ ಒಂದೇ ನಾಣ್ಯದ ಎರಡು ಮುಖಗಳು. ನೋಡುವ ಕಣ್ಣಿದ್ದವರು ಇದನ್ನು ನೋಡಲಿ!”

ನಾಗಮಹಾಶಯರೊಂದಿಗೆ ಮತ್ತೆ ಕೆಲಕಾಲ ಮಾತನಾಡಿದ ಬಳಿಕ ಸ್ವಾಮೀಜಿ ಅವರಿಗೆಂದರು, “ನೀವು ಮಠದಲ್ಲೇ ಬಂದಿರುವಂತಾದರೆ ನಿಜಕ್ಕೂ ಅದೆಷ್ಟು ಚೆನ್ನಾಗಿರುತ್ತಿತ್ತು! ಈ ಹುಡುಗರೆಲ್ಲ ತಮ್ಮ ಜೀವನವನ್ನು ರೂಪಿಸಿಕೊಳ್ಳಲು ಒಂದು ಜೀವಂತ ಉದಾಹರಣೆ ಸಿಕ್ಕಂತಾಗುತ್ತಿತ್ತು.” ಅದಕ್ಕೆ ಆ ಮಹಾಭಕ್ತ ಒಂದು ಬಗೆಯ ಶರಣಾಗತಿಯ ದನಿಯಲ್ಲಿ ನುಡಿದರು, “ಹಿಂದೊಮ್ಮೆ ನಾನು ಪ್ರಪಂಚವನ್ನು ತ್ಯಜಿಸಿಬಿಡಲು ಗುರುಮಹಾರಾಜರ ಅನುಜ್ಞೆಯನ್ನು ಬೇಡಿದ್ದೆ. ಆದರೆ ಅವರು ನನಗೆ, ‘ನೀನು ಪ್ರಪಂಚದಲ್ಲೇ ಇರು’ ಎಂದು ಆಜ್ಞಾಪಿಸಿದರು. ಆದ್ದರಿಂದ ಅವರ ಅನುಜ್ಞೆಯನ್ನು ಪಾಲಿಸುತ್ತಿದ್ದೇನೆ. ಆಗಾಗ ಅವರ ಮಕ್ಕಳಾದ ನಿಮ್ಮ ದರ್ಶನಾನುಗ್ರಹಕ್ಕಾಗಿ ಬರುತ್ತಿರುತ್ತೇನೆ.”

ಈಗ ಸ್ವಾಮೀಜಿ ಇತರ ವಿಷಯಗಳ ಬಗ್ಗೆ ಪ್ರಸ್ತಾಪಿಸಿದರು.

ಸ್ವಾಮೀಜಿ: “ಈಗ ನನ್ನ ಏಕಮಾತ್ರ ಅಭಿಲಾಷೆಯೆಂದರೆ ರಾಷ್ಟ್ರವನ್ನು ಜಾಗೃತಗೊಳಿಸುವುದು. ಈ ಮಹಾಚಂಡಿ ತನ್ನ ಶಕ್ತಿಯಲ್ಲಿಯೆ ನಂಬಿಕೆಯನ್ನು ಕಳೆದುಕೊಂಡು ನಿದ್ರಿಸುತ್ತಿದ್ದಾಳೆ. ಎಷ್ಟು ಗಾಢವಾಗಿ ನಿದ್ರಿಸುತ್ತಿದ್ದಾಳೆಂದರೆ ಮೇಲ್ನೋಟಕ್ಕೆ ಆಕೆ ಮೃತಳಾದಂತೆಯೇ ಕಾಣುತ್ತಿದ್ದಾಳೆ. ಆಕೆಯ ಸನಾತನಧರ್ಮದಲ್ಲಿ ಸುಪ್ತವಾಗಿರುವ ಅನಂತಶಕ್ತಿಯನ್ನು ಬಡಿದೆಬ್ಬಿಸುವಂತಾದರೆ ಗುರುಮಹಾರಾಜರೂ ನಾವೂ ಜನ್ಮವೆತ್ತಿದ್ದು ಸಾರ್ಥಕವಾಗುತ್ತದೆ. ಆ ಒಂದು ಆಸೆ ಮಾತ್ರ ನನ್ನಲ್ಲಿ ಉಳಿದುಕೊಂಡಿದೆ. ಮುಕ್ತಿ ಮುಂತಾದುವೆಲ್ಲ ಇದರ ಮುಂದೆ ಕಸವಾಗಿ ಕಾಣುತ್ತಿವೆ! ನಾನು ಇದರಲ್ಲಿ ಯಶಸ್ವಿಯಾಗುವಂತೆ ನನ್ನನ್ನು ಹರಸಿ.”

ನಾಗಮಹಾಶಯರು: “ಭಗವಂತನ ಕೃಪೆ ನಿಮ್ಮ ಮೇಲೆ ಸದಾ ಇದ್ದೇ ಇದೆ. ನಿಮ್ಮ ಇಚ್ಛೆಯನ್ನು ತಡೆಯಬಲ್ಲವರು ಯಾರು? ನಿಮ್ಮ ಇಚ್ಛೆಯೂ ಭಗವಂತನ ಇಚ್ಛೆಯೂ ಒಂದೇ. ಜೈ ರಾಮಕೃಷ್ಣ!”

ಸ್ವಾಮೀಜಿ: “ಓ! ನನಗೆ ಈ ಕೆಲಸಗಳಿಗೆಲ್ಲ ಅತ್ಯಗತ್ಯವಾದ ಬಲಿಷ್ಠ ಶರೀರವೊಂದಿದ್ದಿದ್ದರೆ!... ನೋಡಿ, ಭಾರತಕ್ಕೆ ಹಿಂದಿರುಗಿದ ಮೇಲೆ ನನ್ನ ಆರೋಗ್ಯ ಹೇಗೆ ಕೆಟ್ಟುಹೋಗಿ, ನನ್ನ ಯೋಜನೆಗಳನ್ನೆಲ್ಲ ಹಾಳುಗೆಡವುತ್ತಿದೆ! ಯೂರೋಪು ಅಮೆರಿಕಗಳಲ್ಲಿ ನಾನು ತುಂಬ ಚೆನ್ನಾ ಗಿದ್ದೆ.”

ನಾಗಮಹಾಶಯರು: “ಗುರುಮಹಾರಾಜರು ಹೇಳುತ್ತಿದ್ದಂತೆ ನಾವು ದೇಹವನ್ನು ಇಟ್ಟು ಕೊಂಡಿರುವವರೆಗೂ ಈ ರೋಗರುಜಿನಗಳ ರೂಪದಲ್ಲಿ ತೆರಿಗೆ ಕಟ್ಟುತ್ತಲೇ ಇರಬೇಕಾಗುತ್ತದೆ. ಆದರೆ ನಿಮ್ಮದು ಚಿನ್ನದ ನಾಣ್ಯಗಳನ್ನಿಟ್ಟಿರುವ ಪೆಟ್ಟಿಗೆ. ಆದ್ದರಿಂದ ಅದನ್ನು ಅತ್ಯಂತ ಎಚ್ಚರಿಕೆ ಯಿಂದ ಕಾಪಾಡಬೇಕು. ಅಯ್ಯೋ, ಅದನ್ನು ಯಾರು ಮಾಡಬಲ್ಲರು! ಪ್ರಪಂಚಕ್ಕೆ ಇದರಿಂದ ಎಂಥ ಉಪಕಾರವಾಗಬೇಕಾಗಿದೆ ಎಂಬುದನ್ನು ಯಾರು ಅರ್ಥಮಾಡಿಕೊಳ್ಳಬಲ್ಲರು!”

ಸ್ವಾಮೀಜಿ: “ಮಠದಲ್ಲಿ ಎಲ್ಲರೂ ನನ್ನನ್ನು ತುಂಬ ಪ್ರೀತಿ ವಿಶ್ವಾಸಗಳಿಂದಲೇ ನೋಡಿಕೊಳ್ಳು ತ್ತಿದ್ದಾರೆ.”

ನಾಗಮಹಾಶಯರು: “ಅವರೆಲ್ಲ ನಿಜಕ್ಕೂ ಧನ್ಯರೇ ಸರಿ. ಅವರು, ಅರಿತೋ ಅರಿಯದೆಯೋ ತಮಗೂ ಶುಭವನ್ನು ತಂದುಕೊಳ್ಳುವುದಲ್ಲದೆ ಸಮಸ್ತ ಜಗತ್ತಿಗೇ ಒಳ್ಳೆಯದು ಮಾಡುತ್ತಿದ್ದಾರೆ.”

ನಾಗಮಹಾಶಯರು ಸ್ವಾಮೀಜಿಯವರ ಬಗ್ಗೆ ಈ ಬಗೆಯ ಮೆಚ್ಚುಗೆಯ ಮಾತುಗಳನ್ನಾಡಿ ದರು, ಕಳಕಳಿಯ ಮಾತುಗಳನ್ನಾಡಿದರು. ಆದರೆ ಅವರು ಈ ಮಾತುಗಳನ್ನು ಯಾವ ರೀತಿಯಲ್ಲಿ ಮತ್ತು ಎಂತಹ ಭಾವದಿಂದ ಆಡಿದರು ಎಂಬುದು ಶಬ್ದಗಳಿಂದ ವರ್ಣಿಸಲಸಾಧ್ಯ. ನಾಗಮಹಾ ಶಯರ ವ್ಯಕ್ತಿತ್ವದ ಪರಿಚಯವಿಲ್ಲದವರಿಗೆ ಇವೆಲ್ಲ ಬರಿಯ ನಾಟಕೀಯವಾಗಿ ಕಾಣಬಹುದು. ಆದರೆ ಭಗವನ್ಮಯವಾದ ಆ ದಿವ್ಯ ಚೇತನವನ್ನು ಅರಿತುಕೊಂಡವರಿಗೆ ಮಾತ್ರವೇ, ಆ ಮಾತು ಗಳು ಅವರ ಹೃದಯಾಂತರಾಳದಿಂದಲೇ ಬರುತ್ತಿದ್ದುವೆಂಬುದು ಅರ್ಥವಾಗುತ್ತದೆ. ಭಕ್ತಾ ಗ್ರೇಸರ ನಾಗಮಹಾಶಯರ ಮತ್ತು ಮಹಾಸಂನ್ಯಾಸಿ ಸ್ವಾಮಿ ವಿವೇಕಾನಂದರ ಆ ದಿವ್ಯ ಸಮಾಗಮವನ್ನು ವೀಕ್ಷಿಸುತ್ತಿದ್ದವರಿಗೆ ಭಾವಾವೇಗದ ಕಣ್ಣೀರನ್ನು ತಡೆಯುವುದು ಅಸಾಧ್ಯ ವಾಯಿತು. ಏಕೆಂದರೆ, ನಾಗಮಹಾಶಯರು ಕೇವಲ ಕೆಲವು ಸರಳ ನುಡಿಗಳಿಂದ, ಅಷ್ಟೇಕೆ–ತಮ್ಮ ದೃಷ್ಟಿಮಾತ್ರದಿಂದ, ಬಳಿಯಿದ್ದವರ ಹೃದಯಗಳಲ್ಲಿ ಅತ್ಯಂತ ಸೂಕ್ಷ್ಮವಾದ ಭಾವತಂತುಗಳನ್ನು ಮಿಡಿಯಬಲ್ಲವರಾಗಿದ್ದರು.

