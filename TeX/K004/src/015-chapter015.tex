
\chapter{ವಿಯೋಗಗಳು}

\noindent

ಈ ನಡುವೆ ಸ್ವಾಮೀಜಿಯವರಿಗೆ ಪ್ರೀತಿವಿಶ್ವಾಸಪಾತ್ರರಾದ ಅನೇಕ ವ್ಯಕ್ತಿಗಳು ಇಹಲೋಕ ಯಾತ್ರೆ ಮುಗಿಸಿದರು. ಇವರಲ್ಲಿ ಮೊದಲನೆಯವರೆಂದರೆ, ಅವರ ಅತ್ಯಂತ ನಿಷ್ಠಾವಂತ ಬೆಂಬಲಿಗರೂ ಅಮೆರಿಕದ ಅಗ್ರಗಣ್ಯ ಮಹಿಳೆಯರಲ್ಲೊಬ್ಬರೂ ಆದ ಶ್ರೀಮತಿ ಬ್ಯಾಗ್​ಲೀ. ಸ್ವಾಮೀಜಿ ಕಲ್ಕತ್ತದಿಂದ ಹೊರಟು ಆಲ್ಮೋರಕ್ಕೆ ಬರುತ್ತಿದ್ದಂತೆಯೇ ಅವರಿಗೆ ಈ ದುಃಖದ ಸುದ್ದಿ ತಲುಪಿತು. ಶ್ರೀಮತಿ ಬ್ಯಾಗ್​ಲೀ ತಮಗೆ ನೀಡಿದ ಅಪಾರ ವಿಶ್ವಾಸ-ಬೆಂಬಲಗಳನ್ನು ನೆನೆದು ಸ್ವಾಮೀಜಿ, ತಮ್ಮ ಮೌನ ಶ್ರದ್ಧಾಂಜಲಿಯನ್ನರ್ಪಿಸಿದರು.

ಈಗ ಸ್ವಾಮೀಜಿಯವರು ಸೇವಿಯರ್ ದಂಪತಿಗಳೊಂದಿಗೆ ಆಲ್ಮೋರಕ್ಕೆ ಹಿಂದಿರುಗಿದಾಗ ಮತ್ತೆರಡು ಆಘಾತಕರ ಸುದ್ದಿಗಳು ಕಾದಿದ್ದುವು. ಒಂದು, ಅವರ ನೆಚ್ಚಿನ ಸಹಾಯಕ-ಶಿಷ್ಯ ಗುಡ್​ವಿನ್ನನ ನಿಧನ; ಮತ್ತೊಂದು, ಸಂತ ಪವಾಹಾರಿ ಬಾಬಾರ ನಿಧನ. ಸ್ವಾಮೀಜಿಯವರ ದೃಷ್ಟಿಯಲ್ಲಿ ಶ್ರೀರಾಮಕೃಷ್ಣರ ಅನಂತರದ ಎರಡನೆಯ ಸ್ಥಾನ ಪವಾಹಾರಿ ಬಾಬಾರಿಗೆ. ಬಾಬಾ ರೊಂದಿಗೆ ಸ್ವಾಮೀಜಿಯವರ ಸಂಬಂಧವನ್ನು ಮೊದಲ ಸಂಪುಟ(‘ಕಣ್ದೆರೆಸಿದ ಅನುಭವ’)ದಲ್ಲಿ ನೋಡಿದ್ದೇವೆ. ಬಾಬಾರ ನಿಧನದ ಸುದ್ದಿ ತಿಳಿಸುವ ಪತ್ರವೊಂದನ್ನು ಕಂಡು ಸ್ವಾಮೀಜಿ ತುಂಬ ದುಃಖಿತರಾದರು. ಅವರು ಆಲ್ಮೋರಕ್ಕೆ ಹಿಂದಿರುಗುವುದಕ್ಕೆ ಎರಡು ದಿನ ಹಿಂದೆಯೇ ಇನ್ನೊಂದು ತಂತಿ ವರ್ತಮಾನ ಬಂದಿತ್ತು. ಅದು ಅವರ ಆಪ್ತ ಶಿಷ್ಯನಾದ ಗುಡ್​ವಿನ್ನನ ಅಕಾಲ ಮರಣದ ದಾರುಣ ವಾರ್ತೆ. ಆದರೆ ಬಾಬಾರ ನಿಧನದ ಸುದ್ದಿ ಕೇಳಿ ದುಃಖತಪ್ತರಾಗಿದ್ದ ಸ್ವಾಮೀಜಿ ಯವರಿಗೆ ಈ ವಿಷಯವನ್ನು ಹೇಳಲು ಯಾರಿಗೂ ಧೈರ್ಯ ಸಾಲಲಿಲ್ಲ. ಮರುದಿನ ಅವರಿಗೆ ಈ ವಿಷಯವೂ ತಿಳಿಯಿತು. ಇದರಿಂದ ಅವರ ಮೇಲಾದ ಆಘಾತ ಪ್ರಚಂಡವಾದುದು. ಅದನ್ನು ಕೇಳಿದ ತಕ್ಷಣ ಅವರ ಬಾಯಿಂದ ಹೊರಟ ಉದ್ಗಾರ ಇದು: “ಇದೀಗ ನನ್ನ ಬಲಗೈ ಮುರಿ ದಂತಾಯಿತು. ನನ್ನ ಪಾಲಿಗಿದು ಭರಿಸಲಾಗದ ನಷ್ಟ!” ಆದರೆ ಸ್ವಾಮೀಜಿ ತಮ್ಮ ಅಪಾರ ದುಃಖ ವನ್ನು ಹತ್ತಿಕ್ಕಿ ಶಾಂತವಾಗಿರುವ ಪ್ರಯತ್ನ ಮಾಡಿದರು.

ಅಂದು ಅವರು ಕೆಲಕಾಲ ತಮ್ಮ ಶಿಷ್ಯರೊಂದಿಗೆ ಗಂಭೀರಭಾವದಲ್ಲಿ ಮಾತನಾಡಿದರು. ವ್ಯಕ್ತಿಯ ಮೇಲಣ ಮೋಹದಿಂದ ಬಿಡಿಸಿಕೊಳ್ಳಬೇಕಾದರೆ ಎಂತಹ ತೀವ್ರ ಪ್ರಯತ್ನ ಬೇಕಾಗು ತ್ತದೆ ಎಂಬುದನ್ನು ವಿವರಿಸಿದರು. ಹೀಗೆ ಕೆಲ ಗಂಟೆಗಳೇ ಕಳೆದುವು. ಸ್ವಾಮೀಜಿ ಮೌನವಾಗಿ ತಮ್ಮ ಶೋಕವನ್ನು ಸಹಿಸಿಕೊಳ್ಳುವ ಪ್ರಯತ್ನ ಮಾಡಿದರು. ಆದರೆ ತಮ್ಮ ಪ್ರಿಯ ಶಿಷ್ಯನ ಅಗಲುವಿಕೆಯನ್ನು ಎಷ್ಟೇ ಮರೆಯಲು ಪ್ರಯತ್ನಿಸಿದರೂ ಅವರ ಸುಕೋಮಲ ಹೃದಯಕ್ಕೆ ಸಂಕಟವಾಗದಿರಲಿಲ್ಲ. “ನನ್ನ ನಿಷ್ಠಾವಂತ ಗುಡ್​ವಿನ್!” ಎಂದು ಅವರು ಹೆಮ್ಮೆಯಿಂದ ಕರೆಯುತ್ತಿದ್ದ ನೆಚ್ಚಿನ ಶಿಷ್ಯನಲ್ಲವೆ ಆತ? ಕೆಲವು ಗಂಟೆಗಳು ಕಳೆದ ಬಳಿಕ ಅವರೆಂದರು, “ಗುಡ್​ವಿನ್ನನ ಚಿತ್ರವೇ ಮತ್ತೆ ಮತ್ತೆ ಮನಸ್ಸಿಗೆ ಬರುತ್ತಿದೆ. ಛೆ! ಇದೊಂದು ದೌರ್ಬಲ್ಯವೇ ಸರಿ. ಮನುಷ್ಯ ಈ ಭ್ರಮೆಯನ್ನು ಗೆಲ್ಲಬೇಕು. ಮರಣ ಹೊಂದಿದವರು ಮೊದಲಿನಂತೆಯೇ ಈಗಲೂ ನಮ್ಮ ನಡುವೆಯೇ ಇದ್ದಾರೆ ಎಂದು ತಿಳಿಯಬೇಕು. ಮೃತರಾದವರು ನಮ್ಮಿಂದ ದೂರವಾದರೆನ್ನುವುದೇ ಒಂದು ಮಿಥ್ಯೆ.” ಹೀಗೆ ಅವರು ಒಮ್ಮೆ ಸಮಾಧಾನ ತಂದುಕೊಳ್ಳು ವಂತಹ ಮಾತನಾಡಿದರೆ, ಮರುಕ್ಷಣವೇ ಅವರ ದುಃಖದ ಕಟ್ಟೆಯೊಡೆದು ಕ್ರೋಧದ ನುಡಿ ಹೊರಬರುತ್ತಿತ್ತು. ಆಗ ಅವರು ಗುಡುಗುತ್ತಿದ್ದರು, “ಒಬ್ಬ ಭಗವಂತನ ಇಷ್ಟಾನಿಷ್ಟಗಳನ್ನು ಅನುಸರಿಸಿ ಈ ಪ್ರಪಂಚ ನಡೆಯುತ್ತಿದೆ ಎನ್ನುವುದಾದರೆ ಅದೆಂತಹ ಭಯಂಕರ ಕಲ್ಪನೆ! ನನ್ನ ಗುಡ್​ವಿನ್ನನನ್ನು ಕೊಂದದ್ದು ಆ ಭಗವಂತನೇ ಆದರೆ, ಅವನೊಂದಿಗೆ ಯುದ್ಧ ಮಾಡಿ ಅವನನ್ನೂ ಕೊಂದರೆ ತಪ್ಪೇನಾದೀತು! ಓಹ್, ಗುಡ್​ವಿನ್! ಅವನಿನ್ನೂ ಬದುಕಿದ್ದರೆ ಎಷ್ಟೊಂದು ಮಾಡುತ್ತಿದ್ದ.” ಸ್ವಾಮಿ ವಿವೇಕಾನಂದರಂತಹ ಸಂನ್ಯಾಸಿಗಳೇ ಹೀಗೆ ದುಃಖಪಡಬಹುದೆ? ಹೀಗೆ ರೋಷಾವೇಶ ತಾಳಬಹುದೆ?–ಎನ್ನಿಸಬಹುದು. ಹಿಂದೆ ಬಲರಾಮ ಬಾಬು ತೀರಿಕೊಂಡ ಸುದ್ದಿ ಕೇಳಿ ಸ್ವಾಮೀಜಿ ದುಃಖಿಸಿದಾಗಲೂ, ಬಳಿಯಲ್ಲಿದ್ದ ಪ್ರಮದದಾಸ ಮಿತ್ರರು ಇದೇ ಪ್ರಶ್ನೆಯನ್ನೇ ಕೇಳಿದ್ದರು. ಅದಕ್ಕೆ ಸ್ವಾಮೀಜಿ ಉತ್ತರಿಸಿದ್ದರು, “ಏನು! ಸಂನ್ಯಾಸಿಗಳಾದ ಮಾತ್ರಕ್ಕೆ ನಮಗೆ ಹೃದಯವಿಲ್ಲವೆ?”

ಸ್ವಾಮೀಜಿಯವರು “ಬ್ರಹ್ಮವಾದಿನ್​” ಪತ್ರಿಕೆಯ ಕೆಲಸದಲ್ಲಿ ಅಳಸಿಂಗರಿಗೆ ನೆರವಾಗಲೆಂದು ಗುಡ್​ವಿನ್ನನನ್ನು ಮದ್ರಾಸಿಗೆ ಕಳಿಸಿಕೊಟ್ಟ ವಿಷಯವನ್ನು ಹಿಂದೆ ನೋಡಿದ್ದೆವು. ಅವನ ಮೂಲಕ ಒಂದು ಇಂಗ್ಲಿಷ್ ದಿನಪತ್ರಿಕೆಯನ್ನು ಹೊರಡಿಸುವ ಉದ್ದೇಶವೂ ಅವರಿಗಿತ್ತು. ಹೀಗೆ ಮದ್ರಾಸಿಗೆ ಬಂದ ಗುಡ್​ವಿನ್ ಸ್ವಾಮೀಜಿಯವರ ಕಾರ್ಯವನ್ನು ಮಾಡಿಕೊಂಡು ಬರುವುದರೊಂದಿಗೆ, ಜೀವನೋಪಾಯಕ್ಕಾಗಿ ಅಲ್ಲಿನ ಒಂದು ವೃತ್ತಪತ್ರಿಕೆಯಲ್ಲಿ ಕೆಲಸಕ್ಕೆ ಸೇರಿದ. ಕೆಲಸದ ನಿಮಿತ್ತ ಕೆಲದಿನಗಳ ಮಟ್ಟಿಗೆ ಊಟಿಗೆ ಹೋದವನು, ಅಲ್ಲಿ ಕಾಯಿಲೆ ಬಿದ್ದ. ನೋಡನೋಡುತ್ತಿದ್ದಂತೆಯೇ ಜ್ವರ ಉಲ್ಬಣಿಸಿತು. ಅವನನ್ನು ಪರೀಕ್ಷಿಸಿದ ವೈದ್ಯರು ಟೈಫಾಯ್ಡ್ ಆಗಿದೆ ಎಂದರು. ಈ ವಿಷಯ ತಿಳಿದು ಗಾಬರಿಗೊಂಡ ಅಳಸಿಂಗ ಮೊದಲಾದವರು, ಮದ್ರಾಸಿನಿಂದ ಊಟಿಗೆ ಹೊರಡಲು ಸಿದ್ಧರಾದರು. ಆದರೆ ಇನ್ನೂ ಸರಿಯಾಗಿ ಔಷಧೋಪಚಾರವನ್ನು ಪ್ರಾರಂಭಿಸುವ ಮೊದಲೇ ಪರಿಸ್ಥಿತಿ ಕೈಮೀರಿಹೋಯಿತು. ಸ್ವಾಮೀಜಿಯವರನ್ನೇ ಸ್ಮರಿಸುತ್ತ ಗುಡ್​ವಿನ್ ಕೊನೆಯುಸಿರೆಳೆದ. ಆಗಿನ್ನೂ ಅವನಿಗೆ ಇಪ್ಪತ್ತೇಳು ವರ್ಷ.

ಕೆಲ ತಿಂಗಳ ಹಿಂದೆಯಷ್ಟೇ ಗುಡ್​ವಿನ್ ತನ್ನ ಸೋದರಿಗೆ ಮದುವೆ ಮಾಡಿಸಿ, ತನ್ನ ತಾಯಿ ಯನ್ನು ಭಾರತಕ್ಕೆ ಕರೆಸಿಕೊಂಡಿದ್ದ. ಪುತ್ರಶೋಕಕ್ಕೆ ಗುರಿಯಾದ ಈ ತಾಯಿಗೆ ಸ್ವಾಮೀಜಿಯವರು ಸಂತಾಪ ಸೂಚಕ ಪತ್ರವೊಂದನ್ನು ಬರೆದು ಅದರೊಂದಿಗೆ ತಮ್ಮ ಪ್ರಿಯ ಗುಡ್​ವಿನ್ನನ ಮೇಲೊಂದುಸುಂದರವಾದ ಕವನವನ್ನು ರಚಿಸಿ ಕಳಿಸಿಕೊಟ್ಟರು. ಆ ಪದ್ಯದ ಅನುವಾದ ಹೀಗಿದೆ–

\begin{myquote}
ನುಗ್ಗಿ ನಡೆಯಲೆ ಆತ್ಮ! ಮುನ್ನುಗ್ಗಿ ನಡೆ ಜವದಿ\\ಚುಕ್ಕಿ ಚೆಲ್ಲಿದ ನಿನ್ನ ಹಾದಿಯಲ್ಲಿ;\\ನುಗ್ಗಿ ನಡೆ ಅಮೃತಾತ್ಮ, ಕಾಲ-ದೇಶಗಳೆಲ್ಲ\\ನಿನ್ನ ಕಾಣ್ಕೆಯದೆಂದು ಮಸುಳದಲ್ಲಿ–\\ಶಾಂತಿ-ಶುಭವೆಂದೆಂದು ನಿನಗಾಗಲಿ!
\end{myquote}

\begin{myquote}
ನಿನ್ನ ಸೇವೆಯು ಸತ್ಯ, ನಿನ್ನ ತ್ಯಾಗವು ಪೂರ್ಣ,\\ನಿನ್ನೊಲವಿನೆದೆ ತನ್ನ ಮನೆಯ ಕಾಣುವುದು;\\ಕಾಲದೇಶವ ಕೊಲ್ವ ಚಿರ ಮಧುರ ಸ್ಮೃತಿಗಳವು\\ದೇವಗರ್ಪಿತವಾದ ಕುಸುಮರಾಶಿಗಳಂತೆ\\ನೀ ನಡೆದ ಹಾದಿಯನು ಹಿಂದೆ ತುಂಬುವುದು!
\end{myquote}

\begin{myquote}
ಬಂಧಗಳ ಹರಿದೆಸೆದೆ, ಆನಂದವನು ಸವಿದೆ;\\ಜನನ-ಮರಣದ ರೂಪ ಧರಿಸಿ ಬಹ ಸತ್ಯದಲಿ\\ಎಂದೆಂದು ಒಂದಾಗಿ ಸ್ಥಿರದಿ ನಿಂದೆ;\\ನೆರವ ನೀಡುವ ನೀನು, ನಿಸ್ಸೀಮ ನಿಃಸ್ವಾರ್ಥಿ\\ಬವಣೆ ತುಂಬಿಹ ಜಗಕೆ ನಿನ್ನೊಲವ ಹರಿಸುತ್ತ\\ಎಂದೆಂದು ಮುನ್ನಡೆದು ನೆರವ ನೀಡು!
\end{myquote}

ಈ ಕವನದ ಜೊತೆಯಲ್ಲಿ ಸ್ವಾಮೀಜಿ ಬರೆದಿದ್ದ ಪತ್ರ ಇಂತಿತ್ತು–

“ನಾನವನ ಪುಣವನ್ನು ಎಂದೆಂದಿಗೂ ತೀರಿಸಲಾರೆ. ಅಲ್ಲದೆ ನನ್ನ ವಿಚಾರಧಾರೆಯಿಂದ ಉಪಕೃತರಾಗಿದ್ದೇವೆಂದು ಯಾರುಯಾರು ಭಾವಿಸುತ್ತಾರೋ ಅವರೆಲ್ಲ ತಿಳಿದಿರಬೇಕು–ಅವು ಗಳಲ್ಲಿ ಹೆಚ್ಚುಕಡಿಮೆ ಪ್ರತಿಯೊಂದು ಪದವೂ ಕೂಡ ಪ್ರಕಟವಾದದ್ದು ಗುಡ್​ವಿನ್ನನ ದಣಿವರಿ ಯದ ನಿಃಸ್ವಾರ್ಥ ಶ್ರಮದಿಂದ ಎಂದು. ಅವನ ಮರಣದಿಂದಾಗಿ ನಾನು ಉಕ್ಕಿನಂತಹ ಸ್ನೇಹಿತನೊಬ್ಬನನ್ನು, ಆರದ ಭಕ್ತಿಯ ಶಿಷ್ಯನೊಬ್ಬನನ್ನು, ಆಯಾಸವೆಂದರೇನೆಂದು ತಿಳಿಯದ ಕೆಲಸಗಾರನೊಬ್ಬನನ್ನು ಕಳೆದುಕೊಂಡಿದ್ದೇನೆ. ಇತರರಿಗಾಗಿಯೇ ಜನ್ಮವೆತ್ತುವ ಕೆಲವೇ ಕೆಲವರ ಲ್ಲೊಬ್ಬನನ್ನು ಕಳೆದುಕೊಂಡು ಜಗತ್ತು ಬಡವಾಗಿದೆ.”

ಈ ಪತ್ರ ತಲುಪಿದ ಮೇಲೆ ಗುಡ್​ವಿನ್ನನ ತಾಯಿ, ತನ್ನ ಮಗನ ಶೀಲವನ್ನು ರೂಪಿಸುವಲ್ಲಿ ಸ್ವಾಮೀಜಿಯವರು ಬೀರಿದ ಗಾಢ ಪ್ರಭಾವಕ್ಕಾಗಿ ಅವರಿಗೆ ತನ್ನ ಹೃತ್ಪೂರ್ವಕ ಕೃತಜ್ಞತೆಯನ್ನರ್ಪಿ ಸುವ ಪತ್ರವೊಂದನ್ನು ಬರೆದಳು.

ಗುಡ್​ವಿನ್ನನ ನಿಧನದ ಸುದ್ದಿ ಸ್ವಾಮೀಜಿಯವರಿಗೆ ತಿಳಿಯುವ ಹಿಂದಿನ ದಿನ ಅವರಿಗೆ ಪವಾಹಾರಿ ಬಾಬಾರ ದೇಹತ್ಯಾಗದ ವಿಷಯವು ಪತ್ರವೊಂದರ ಮೂಲಕ ತಿಳಿಯಿತು. ಆ ಪತ್ರವನ್ನು ಅವರು ತಮ್ಮ ಶಿಷ್ಯರಿಗೆ ಓದಿ ಹೇಳುತ್ತ “ಬಾಬಾಜಿಯವರು ತಮ್ಮ ಶರೀರವನ್ನೇ ಆಹುತಿಯಾಗಿಸಿಕೊಳ್ಳುವುದರ ಮೂಲಕ ತಮ್ಮ ಯಜ್ಞಗಳೆಲ್ಲವನ್ನೂ ಪರಿಸಮಾಪ್ತಿಗೊಳಿಸಿದರು. ಅವರು ಯಜ್ಞೇಶ್ವರನಿಗೆ ತಮ್ಮ ಶರೀರವನ್ನೇ ಸಮರ್ಪಿಸಿಬಿಟ್ಟರು” ಎಂದು ತಿಳಿಸಿದರು. ಆಗ ಅಲ್ಲಿದ್ದವರೊಬ್ಬರು, “ಆದರೆ ಅದು ತಪ್ಪಲ್ಲವೆ ಸ್ವಾಮೀಜಿ!” ಎಂದರು. ಅದು ಆತ್ಮಹತ್ಯೆ ಮಾಡಿಕೊಂಡಂತಾಗಲಿಲ್ಲವೆ, ಎಂಬುದು ಅವರ ಅಭಿಪ್ರಾಯ. ಅದಕ್ಕೆ ಸ್ವಾಮೀಜಿ ಉತ್ತರಿಸಿದರು, “ನಾನು ಹೇಗೆ ಹೇಳಲಿ ಅದನ್ನು? ನನ್ನ ಅಳತೆಗೆ ನಿಲುಕದಷ್ಟು ದೊಡ್ಡವರು ಅವರು. ತಾವು ಏನು ಮಾಡುತ್ತಿರುವೆವೆಂಬುದು ಅವರಿಗೆ ಚೆನ್ನಾಗಿ ತಿಳಿದಿತ್ತು.”

ಆಲ್ಮೋರಕ್ಕೆ ಬಂದ ಮೇಲೆ ಸ್ವಾಮೀಜಿಯವರಿಗೆ ಅತೀವ ಆಘಾತವನ್ನುಂಟುಮಾಡಿದ ಮತ್ತೊಂದು ಸುದ್ದಿಯೆಂದರೆ ‘ಪ್ರಬುದ್ಧ ಭಾರತ’ದ ಸಂಪಾದಕರಾದ ಶ್ರೀ ರಾಜಂ ಅಯ್ಯರ್ ರವರ ಅಕಾಲಿಕ ಮರಣ. ರಾಜಂ ಅಯ್ಯರರು ಕೇವಲ ಇಪ್ಪತ್ತಾರು ವಯಸ್ಸಿನ ಅತ್ಯಂತ ಪ್ರತಿಭಾವಂತ ಯುವಕ. ಅವರು ವೇದಾಂತದಲ್ಲಿ ನುರಿತವರು; ಸ್ವಾಮೀಜಿಯವರ ಕಟ್ಟಾ ಅನುಯಾಯಿ. ರಾಜಂ ಅಯ್ಯರರ ಸಂಪಾದಕತ್ವದಲ್ಲಿ ಪತ್ರಿಕೆ ಬಹಳ ಚೆನ್ನಾಗಿ ನಡೆದುಕೊಂಡು ಬರುತ್ತಿತ್ತು. ಈ ಪತ್ರಿಕೆಯ ಮೇಲೆ ಸ್ವಾಮೀಜಿಯವರಿಗೆ ತುಂಬ ವಿಶ್ವಾಸ-ಮಮತೆ. ಏಕೆಂದರೆ ಇದು ಅವರ ಶಿಷ್ಯರು ತಾವೇ ಧನಸಂಗ್ರಹಣೆ ಮಾಡಿ, ಸ್ವಯಂಸ್ಫೂರ್ತಿಯಿಂದ, ಶ್ರಮವಹಿಸಿ ಪ್ರಾರಂಭಿಸಿದುದು. ಇದನ್ನು ಮುಖ್ಯವಾಗಿ ಸಾಮಾನ್ಯ ಜನರನ್ನು ಹಾಗೂ ಎಳೆಯರನ್ನು ದೃಷ್ಟಿಯ ಲ್ಲಿಟ್ಟುಕೊಂಡು ಪ್ರಾರಂಭಿಸಲಾಗಿತ್ತಾದರೂ ಇದೊಂದು ಅತ್ಯುನ್ನತ ದರ್ಜೆಯ ಪತ್ರಿಕೆಯಾಗಿತ್ತು. ಆಗಲೇ ಅದಕ್ಕೆ ಮೂರು ಸಾವಿರ ಚಂದಾದಾರರಿದ್ದರು. ಇಂತಹ ಪತ್ರಿಕೆ ಈಗ ರಾಜಂ ಅಯ್ಯರರ ನಿಧನದೊಂದಿಗೆ ಇದ್ದಕ್ಕಿದ್ದಂತೆ ನಿಂತುಹೋಯಿತು. ಇದನ್ನು ಪುನರೂರ್ಜಿತಗೊಳಿಸುವ ಉದ್ದೇಶ ದಿಂದ ಸ್ವಾಮೀಜಿ, ತಮ್ಮ ಕಾರ್ಯದಲ್ಲಿ ನೆರವಾಗಲು ಬಂದಿದ್ದ ಕ್ಯಾಪ್ಟನ್ ಸೇವಿಯರ್​ರನ್ನು, ಆ ಪತ್ರಿಕೆಯ ಪ್ರಕಟಣೆಯ ಜವಾಬ್ದಾರಿ ಹೊರುವಂತೆ ಪ್ರೇರೇಪಿಸಿದರು. ಅತ್ಯಂತ ಸಮರ್ಥ ಯುವಸಂನ್ಯಾಸಿಗಳಾದ ಸ್ವಾಮಿ ಸ್ವರೂಪಾನಂದರನ್ನು ಅದರ ಸಂಪಾದಕರನ್ನಾಗಿ ನಿಯಮಿಸು ವುದಾಗಿಯೂ ಅವರು ಹೇಳಿದರು. ಅಲ್ಲದೆ ಸ್ವಾಮಿ ತುರೀಯಾನಂದರ ನೆರವಿನ ಭರವಸೆಯನ್ನೂ ನೀಡಿದರು.

ಸೇವಿಯರ್ ಈ ಸಲಹೆಗೆ ಒಪ್ಪಿದರು. ಆದರೆ ಅವರಿಗೆ ಭಾರತದ ಬಯಲು ಸೀಮೆಗಳ ಉಷ್ಣತೆಯನ್ನು ತಡೆದುಕೊಳ್ಳಲು ಸಾಧ್ಯವಿರಲಿಲ್ಲ. ಆದ್ದರಿಂದ ಪತ್ರಿಕೆಯನ್ನು ಹಿಮಾಲಯ ಪ್ರದೇಶದಿಂದಲೇ ಹೊರಡಿಸುವುದೆಂದು ನಿಶ್ಚಯವಾಯಿತು. ಕ್ಯಾಪ್ಟನ್ ಸೇವಿಯರರು ಪತ್ರಿಕೆಯ ಮ್ಯಾನೇಜರ್ ಆದರು; ಸ್ವರೂಪಾನಂದರು ಸಂಪಾದಕರಾದರು. ಆಲ್ಮೋರದಲ್ಲಿ ಸೇವಿಯರ್ ದಂಪತಿಗಳು ಇಳಿದುಕೊಂಡಿದ್ದ ಮನೆಗೇ ಅದರ ಆಫೀಸನ್ನು ವರ್ಗಾಯಿಸಲಾಯಿತು. ಕಲ್ಕತ್ತ ದಿಂದ ಮುದ್ರಣ ಯಂತ್ರಗಳನ್ನು ತರಿಸಲಾಯಿತು. ಈ ಯಂತ್ರಾದಿಗಳನ್ನು ಖರೀದಿಸಿ ಸಾಗಿಸುವ ಖರ್ಚನ್ನೆಲ್ಲ ಸಂಪೂರ್ಣವಾಗಿ ಸೇವಿಯರರೇ ವಹಿಸಿಕೊಂಡರು. ಅಂತೂ, ಜೂನ್ ತಿಂಗಳಲ್ಲಿ ನಿಂತುಹೋಗಿದ್ದ ಪತ್ರಿಕೆ, ಆಗಸ್ಟ್ ತಿಂಗಳಿನಿಂದಲೇ ಪುನರಾರಂಭವಾಯಿತು. ಇದೊಂದು ಮಹತ್ತರ ಸಾಧನೆಯೇ ಸರಿ. ಸ್ವಾಮೀಜಿ ಆಲ್ಮೋರದಿಂದ ಹೊರಡುವ ಮೊದಲೇ ಪ್ರಕಟಣೆಗೆ ಸಿದ್ಧತೆಗಳೆಲ್ಲ ಪ್ರಾರಂಭವಾದುವು. ಇತರ ಪತ್ರಿಕೆಗಳನ್ನು ಹೊರಡಿಸುವ ಅವರ ಯೋಜನೆಗಳಾ ವುವೂ ಕಾರ್ಯಗತವಾಗದಿದ್ದರೂ ‘ಪ್ರಬುದ್ಧ ಭಾರತ’ವಂತೂ ಪುನಃ ಸ್ಥಾಪಿತವಾದದ್ದು ಅವರಿಗೆ ತುಂಬ ಸಮಾಧಾನವನ್ನುಂಟುಮಾಡಿತು.

