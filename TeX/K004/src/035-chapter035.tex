
\chapter{ದೇವಮಾನವ! ದೇವಮಾನವ!}

\noindent

ಫ್ರಾನ್ಸಿನಿಂದ ಹೊರಡುವ ಮುನ್ನ ಸ್ವಾಮೀಜಿಯವರು ಜೂಲ್ಸ್ ಬ್ಯೂಸ್​ನೊಂದಿಗೆ ಮತ್ತೊಮ್ಮೆ ಬ್ರಿಟಾನಿಗೆ ಭೇಟಿ ನೀಡಿ ನಾಲ್ಕೈದು ದಿನಗಳನ್ನು ಕಳೆದು ಮೆಕ್​ಲಾಡಳೊಂದಿಗೆ ಪ್ಯಾರಿಸಿಗೆ ಮರಳಿದರು. ಇದೇ ವೇಳೆಗೆ, ಚಳಿಗಾಲದಲ್ಲಿ ವಿಶ್ರಮಿಸಲು ಈಜಿಪ್ಟಿಗೆ ಹೊರಟಿದ್ದ ಪ್ರಸಿದ್ಧ ಅಪೇರಾ (ಸಂಗೀತಮಯ ನಾಟಕ) ಕಲಾವಿದೆ ಮೇಡಂ ಎಮ್ಮಾ ಕಾಲ್ವೆ, ತಮ್ಮೊಂದಿಗೆ ಬರುವಂತೆ ಸ್ವಾಮೀಜಿಯವರನ್ನು ಆಹ್ವಾನಿಸಿದಳು. ಸ್ವಾಮೀಜಿ ಈ ಆಹ್ವಾನವನ್ನು ಅಂಗೀಕರಿಸಿದರು. ಹಿಂದೆ ಇಂಗ್ಲೆಂಡಿನೊಂದಿಗೆ ಫ್ರಾನ್ಸ್, ಜರ್ಮನಿ, ಹಾಲೆಂಡ್, ಇಟಲಿ ಹಾಗೂ ಸ್ವಿಟ್ಸರ್​ಲ್ಯಾಂಡ್​ಗಳನ್ನು ಸಂದರ್ಶಿಸಿದ್ದ ಅವರಿಗೀಗ ಆಸ್ಟ್ರಿಯಾ, ಬಾಲ್ಕನ್​ದೇಶಗಳು, ಏಷಿಯಾ ಮೈನರ್, ಗ್ರೀಸ್ ಹಾಗೂ ಈಜಿಪ್ಟ್​ಗಳನ್ನು ವೀಕ್ಷಿಸುವ ಅವಕಾಶ ಲಭ್ಯವಾಗಿತ್ತು. ಈ ಪ್ರವಾಸವನ್ನು ಮುಗಿಸಿಕೊಂಡು ಬಂದು ಅವರು ಪ್ಯಾರಿಸಿನಲ್ಲಿ ಮತ್ತೆ ಕೆಲವು ಭಾಷಣಗಳನ್ನು ಮಾಡಬೇಕೆಂಬ ಯೋಜನೆಯಿತ್ತು. ಆದರೆ ಅವರು ಇದ್ದಕ್ಕಿದ್ದಂತೆ ಮನಸ್ಸು ಬದಲಾಯಿಸಿ ಪ್ರವಾಸವನ್ನು ಮೊಟಕುಗೊಳಿಸಿ ನೇರವಾಗಿ ಭಾರತಕ್ಕೆ ಹಿಂದಿರುಗುವುದನ್ನು ನೋಡಲಿದ್ದೇವೆ.

ಅಕ್ಟೋಬರ್ ೨೪ರಂದು ಸ್ವಾಮೀಜಿ ಮತ್ತು ಅವರ ಸಂಗಡಿಗರು ಅಂದಿನ ಕಾಲದಲ್ಲಿ ಅತ್ಯಾಧುನಿಕವಾದ ‘ಓರಿಯೆಂಟಲ್ ಎಕ್ಸ್​ಪ್ರೆಸ್​’ ಎಂಬ ಖಂಡಾಂತರ \eng{(Inter-continental)} ಟ್ರೈನು ಹತ್ತಿ ಪ್ಯಾರಿಸಿನಿಂದ ಹೊರಟರು. ಈಗ ಅವರೊಂದಿಗಿದ್ದವರೆಂದರೆ ಮೇಡಂ ಕಾಲ್ವೆ, ಮಿಸ್ ಮೆಕ್​ಲಾಡ್, ಜೂಲ್ಸ್ ಬ್ಯೂಸ್ ಹಾಗೂ ಪಿಯರ್ ಹಯಸಿಂಥ್ ದಂಪತಿಗಳು. ಹಯಸಿಂಥರು ಪ್ಯಾಲೆಸ್ಟೀನಿನಲ್ಲಿ ಹೊಡೆದಾಡುತ್ತಿದ್ದ ಕ್ರೈಸ್ತರ ಹಾಗೂ ಮುಸಲ್ಮಾನರ ನಡುವೆ ಹೊಂದಾಣಿಕೆ ತರಿಸಿ, ಅಲ್ಲಿ ಉಂಟಾಗಿದ್ದ ಪ್ರಕ್ಷುಬ್ಧ ಪರಿಸ್ಥಿತಿಯನ್ನು ಶಾಂತಗೊಳಿಸುವ ಉದ್ದೇಶ ದಿಂದ ಶಾಂತಿದೂತನಾಗಿ ಹೊರಟಿದ್ದರು.

ಈ ಪ್ರಯಾಣಿಕರ ತಂಡದ ಸದಸ್ಯರೆಲ್ಲ ಹೀಗೆ ಒಟ್ಟಾಗಿ ಸೇರಿ ಪ್ರವಾಸ ಹೊರಟದ್ದೊಂದು ವಿಶೇಷವೆಂದೇ ಹೇಳಬೇಕು. ಆದರೆ ಇದರಲ್ಲಿ ವಿಚಿತ್ರವಾದದ್ದಾಗಲಿ ಅಸಹಜವಾದದ್ದಾಗಲಿ ಏನೂ ಇರಲಿಲ್ಲ. ಆದರೆ ಫ್ರಾನ್ಸಿನ ದಿನಪತ್ರಿಕೆಗಳಿಗೆಲ್ಲ ಇದೊಂದು ‘ಗರಮಾಗರಂ’ ಸುದ್ದಿ. ಪ್ರಸಿದ್ಧ ದಿನಪತ್ರಿಕೆಗಳೆಲ್ಲ ಇದನ್ನು ಮುಖಪುಟದಲ್ಲಿ ಚಿತ್ರಸಮೇತವಾಗಿ ದಪ್ಪಕ್ಷರಗಳಲ್ಲಿ ಪ್ರಕಟಿಸಿದುವು. ಪತ್ರಿಕೆಗಳ ಅಚ್ಚರಿಗೆ ಕಾರಣವೆಂದರೆ ಅಂದಿನ ಸುವಿಖ್ಯಾತ ನಟಿ ಎಮ್ಮಾ ಕಾಲ್ವೆ ಸ್ವಾಮೀಜಿಯವ ರೊಂದಿಗೆ ಹೊರಟಿದ್ದುದು. ಈ ಪ್ರಯಾಣದ ಉದ್ದೇಶವೇನಿರಬಹುದೆಂದು ‘ಊಹಿಸಿ’ ಪತ್ರಿಕೆ ಗಳು ಆ ಬಗ್ಗೆ ಪುಟಗಟ್ಟಲೆ ಬರೆದುವು!

೨೫ರ ಸಂಜೆ ಸ್ವಾಮೀಜಿಯವರ ತಂಡ ವಿಯೆನ್ನಾದಲ್ಲಿ ಇಳಿಯಿತು. ಇಲ್ಲಿ ಅವರೆಲ್ಲ ಹಲವಾರು ಪ್ರೇಕ್ಷಣೀಯ ಸ್ಥಳಗಳನ್ನು ನೋಡುತ್ತ ಮೂರು ದಿನ ಕಳೆದರು. ಇಲ್ಲಿನ ಪ್ರಸಿದ್ಧ ವಸ್ತು ಸಂಗ್ರಹಾಲಯವನ್ನೂ ಐತಿಹಾಸಿಕ ಅರಮನೆಯನ್ನೂ ವೀಕ್ಷಿಸಿ ಆನಂದಿಸಿದರು. ಪ್ಯಾರಿಸ್ಸನ್ನು ನೋಡಿದ ಮೇಲೆ ಸ್ವಾಮೀಜಿಯವರಿಗೆ ಯೂರೋಪಿನ ಬೇರಾವ ನಗರವೂ ರುಚಿಸಲಿಲ್ಲ. ಆದರೆ ಸಮಸ್ತ ಚರಿತ್ರೆಯನ್ನೂ ತಮ್ಮ ಮೆದುಳೊಳಗೆ ಅಡಗಿಸಿಕೊಂಡಿದ್ದ ಅವರಿಗೆ, ಮಾನವನಿಗೆ ಸಂಬಂಧಿಸಿದ ಯಾವೊಂದು ವಸ್ತುವೂ ಆಸಕ್ತಿಕರವಲ್ಲದಿರಲು ಸಾಧ್ಯವಿರಲಿಲ್ಲ. ವಸ್ತುಸಂಗ್ರಹಾ ಲಯದ ಒಂದೊಂದು ವಸ್ತುವೂ ಅವರ ನೆನಪಿನ ಚೀಲದಿಂದ ಹೊಸಹೊಸ ವಿಷಯಗಳನ್ನು ಹೊರತೆಗೆಸುತ್ತಿತ್ತು. ಅವರು ಅವುಗಳನ್ನು ಕುರಿತು ವರ್ಣಿಸುತ್ತಿದ್ದರೆ ಸುತ್ತಲಿದ್ದವರು ಆ ಅಲೌಕಿಕ ವಾಕ್ ಪ್ರವಾಹದಲ್ಲಿ ಮುಳುಗಿ ವಿಸ್ಮಯಮೂಕರಾಗುತ್ತಿದ್ದರು. ಆಗ ಆ ಇತರರಿಗೆ ಬೇರಾವ ವಿಷಯವೂ ಅನುಭವಕ್ಕೇ ಬರುತ್ತಿರಲಿಲ್ಲ! ಈ ಪ್ರವಾಸದ ಬಗ್ಗೆ ಮೇಡಂ ಕಾಲ್ವೆ ತನ್ನ ಆತ್ಮಚರಿತ್ರೆ ಯಲ್ಲಿ ಬರೆದಿದ್ದಾಳೆ:

“ಆಹಾ! ಅದೆಂತಹ ಅದ್ಭುತ ತೀರ್ಥಯಾತ್ರೆ! ವಿಜ್ಞಾನ, ತತ್ವಶಾಸ್ತ್ರ, ಚರಿತ್ರೆಗಳು ಸ್ವಾಮೀಜಿ ಯವರಿಂದ ಏನನ್ನೂ ಮುಚ್ಚಿಟ್ಟುಕೊಂಡಿರಲಿಲ್ಲ. ನಾನು ನನ್ನ ಸುತ್ತಲೂ ನಡೆಯುತ್ತಿದ್ದ ಪಾಂಡಿತ್ಯಪೂರ್ಣವೂ ಜ್ಞಾನಪ್ರದವೂ ಆದ ಸಂಭಾಷಣೆಗಳನ್ನು ಕಿವಿತೆರೆದು ಕೇಳುತ್ತಿದ್ದೆ. ಆದರೆ ನಾನು ಅವರ ಚರ್ಚೆಯಲ್ಲಿ ಸೇರಿಕೊಳ್ಳುವ ಪ್ರಯತ್ನ ಮಾಡಲಿಲ್ಲ. ಎಲ್ಲ ಸಂದರ್ಭಗಳಲ್ಲೂ ನಾನು ಹಾಡುತ್ತಿದ್ದೆ. ಸುಪ್ರಸಿದ್ಧ ಪಂಡಿತರೂ ತತ್ವಜ್ಞಾನಿಗಳೂ ಆದ ಫಾದರ್ ಪಿಯರ್ ಹಯಸಿಂಥರೊಂದಿಗೆ ಸ್ವಾಮೀಜಿ ಎಲ್ಲ ಬಗೆಯ ವಿಷಯಗಳನ್ನೂ ಚರ್ಚಿಸುತ್ತಿದ್ದರು. ಆದರೆ ಹಯಸಿಂಥರಿಗೇ ನೆನಪಿರದಿದ್ದ ಯಾವುದೋ ಚರ್ಚಿನ ಸಭೆಯ ದಿನಾಂಕವನ್ನೋ, ಯಾವುದೋ ದಾಖಲೆಯ ನಿಖರವಾದ ಪಾಠವನ್ನೋ ಅವರು ಹೇಳುವುದನ್ನು ಕೇಳಲು ಬಹಳ ಕುತೂಹಲಕರ ವಾಗಿತ್ತು.”

ಸ್ವಾಮೀಜಿಯವರಿಗೆ ಭೂತಕಾಲವು ವರ್ತಮಾನದ ಅಂಗವೇ ಆಗಿತ್ತು; ಭವಿಷ್ಯವು ತೆರೆದಿಟ್ಟ ಪುಸ್ತಕವಾಗಿತ್ತು. “ಯೂರೋಪು ಅಗ್ನಿಪರ್ವತದ ಮೇಲೆ ಕುಳಿತಿದೆ. ಅಧ್ಯಾತ್ಮದ ಪ್ರವಾಹದಿಂದ ಬೆಂಕಿಯು ನಂದಿಸಲ್ಪಡದಿದ್ದರೆ ಅದು ಆಸ್ಫೋಟಿಸುವುದು ಖಂಡಿತ” ಎಂದು ಆಗ ಐದು ವರ್ಷಗಳ ಹಿಂದೆಯೇ ಅವರು ನುಡಿದಿದ್ದರು. ಆದರೆ ಅಂದಿನ ಎಷ್ಟೋ ಜನ ಆದರ್ಶವಾದಿಗಳು ವಿಜ್ಞಾನದ ಮುನ್ನಡೆ ಹಾಗೂ ಸರ್ವತೋಮುಖ ಅಭಿವೃದ್ಧಿಯಿಂದ ಬಹಳ ಸಂತುಷ್ಟರಾಗಿದ್ದರು. ಇನ್ನು ಕೆಲವೇ ವರ್ಷಗಳಲ್ಲಿ ಮಹಾಯುದ್ಧವೊಂದು ಸಂಭವಿಸಬಹುದೆಂಬ ಕಲ್ಪನೆಯೇ ಅವರಿ ಗಿರಲಿಲ್ಲ. ಆದರೆ ಸ್ವಾಮೀಜಿಯವರು ಅದನ್ನು ಮುಂಗಂಡಿದ್ದರು. ಅಲ್ಲದೆ ಆ ಮಹಾಯುದ್ಧವು ಹೇಗೆ ಪ್ರಾರಂಭವಾಗಬಹುದೆಂದೂ ಮುನ್ನುಡಿದಿದ್ದರು. ಅವರ ಮಾತು ಮುಂದೆ ಅಕ್ಷರಶಃ ಸತ್ಯವಾಯಿತು (೧೯೧೪ರಿಂದ ೧೯೧೮ರವರೆಗೆ ನಡೆದ ಮೊದಲ ಮಹಾಯುದ್ಧ ನಮಗೆ ತಿಳಿದೇ ಇದೆ).

ಅಕ್ಟೋಬರ್ ೨೮ರಂದು ಪ್ರಯಾಣಿಕರ ತಂಡ ವಿಯೆನ್ನಾದಿಂದ ಕಾನ್​ಸ್ಟಾಂಟಿನೋಪಲ್ (ಇಸ್ತಾಂಬುಲ್​) ಕಡೆಗೆ ಹೊರಟಿತು. ಇದು ಹಲವಾರು ರಾಷ್ಟ್ರಗಳ ಮೂಲಕ ಸಾಗುವ ಹಲವು ನೂರು ಮೈಲಿಗಳ ಪ್ರಯಾಣ. ಸ್ವಾಮೀಜಿಯವರ ಸಾನ್ನಿಧ್ಯವೇ ಸಂಗಡಿಗರಿಗೊಂದು ಶಿಕ್ಷಣ. ಟ್ರೈನು ಹಾದುಹೋಗುತ್ತಿದ್ದ ರಾಷ್ಟ್ರಗಳ, ನಗರಗಳ ಭೂತ-ವರ್ತಮಾನಗಳನ್ನು ಅವರು ಬಣ್ಣಿಸು ತ್ತಿದ್ದಂತೆ, ಜೊತೆಯಲ್ಲಿದ್ದವರು ಮೈಯೆಲ್ಲ ಕಿವಿಯಾಗಿ ಆಲಿಸುತ್ತ ಕುಳಿತಿರುತ್ತಿದ್ದರು. ಸ್ವಾಮೀಜಿ ಯವರ ಕೊನೆಯಿಲ್ಲದ ಚಿಂತನ ಪ್ರವಾಹದ ಒಂದು ತುಣುಕನ್ನು ಅವರ ‘ಪರಿವ್ರಾಜಕ’ ಎಂಬ ಪುಸ್ತಕದಲ್ಲಿ ಕಾಣಬಹುದು. ಅಲ್ಲಲ್ಲಿ ಅವರ ಅಪೂರ್ವ ಮೊನಚು ಹಾಸ್ಯವನ್ನೂ ಕಾಣಬಹುದು. ಬಡ ಬಾಲ್ಕನ್ ರಾಷ್ಟ್ರಗಳಾದ ಸರ್ಬಿಯ, ಬಲ್ಗೇರಿಯಗಳ ಹಳ್ಳಿಗಾಡು ಪ್ರದೇಶಗಳನ್ನು ಕಂಡಾಗ ಅವರಿಗೆ ಭಾರತದ ನೆನಪಾಯಿತು. ಆದರೆ ಇವುಗಳನ್ನು ಭಾರತದೊಂದಿಗೆ ಹೋಲಿಸಲಾಗದು ಎಂಬ ಅರ್ಥದಲ್ಲಿ ಅವರು ಬರೆಯುತ್ತಾರೆ–“ಅವರೆಲ್ಲ ಕ್ರೈಸ್ತರಾದ್ದರಿಂದ ಅವರ ಬಳಿ ಹಂದಿ ಗಳಿರಲೇಬೇಕು. (ಮತ್ತು) ಒಂದು ಹಂದಿಯು ಯಾವುದೇ ಜಾಗವನ್ನು ಮುನ್ನೂರು ಜನ ಅನಾಗರಿಕರ ಕೈಯಲ್ಲಿ ಸಾಧ್ಯವಾಗುವುದಕ್ಕಿಂತ ಹೆಚ್ಚು ಕೊಳಕು ಮಾಡಬಲ್ಲದು... ” ಆದರೆ ಸ್ವಾತಂತ್ರ್ಯವೊಂದಿದ್ದರೆ ಬಡತನವಿದ್ದರೂ ಪರವಾಗಿಲ್ಲ ಎಂಬುದು ಅವರ ಅಭಿಪ್ರಾಯ. ಆದ್ದ ರಿಂದ ಮತ್ತೆ ಬರೆಯುತ್ತಾರೆ: “ಆದರೆ ಒಂದು ಹೊತ್ತಿನ ಊಟ, ಮೈ ಮುಚ್ಚುವಷ್ಟು ಬಟ್ಟೆ–ಇದು ಚಿನ್ನದ ಸರಪಣಿಯಲ್ಲಿ ಬಂಧಿತರಾಗಿರುವುದಕ್ಕಿಂತ ಲಕ್ಷ ಪಾಲು ಮೇಲು.”

ಅಕ್ಟೋಬರ್ ೩ಂರಂದು ಸ್ವಾಮೀಜಿ ಹಾಗೂ ಅವರ ಸಹಚರರು ಕಾನ್​ಸ್ಟಾಂಟಿನೋಪಲ್ಲಿಗೆ ಬಂದರು. ಇಲ್ಲಿ ಅವರು ಹಲವಾರು ಪುರಾತನ ಮಸೀದಿಗಳು, ಗೋಪುರಗಳು, ಪ್ರಾಚ್ಯವಸ್ತು ಸಂಗ್ರಹಾಲಯ, ಕೋಟೆ ಮೊದಲಾದವನ್ನು ವೀಕ್ಷಿಸಿದರು. ಸರ್ ಹಿರಂ ಮ್ಯಾಕ್ಸಿಂ ಕೊಟ್ಟಿದ್ದ ಪರಿಚಯ ಪತ್ರಗಳ ನೆರವಿನಿಂದ ಸ್ವಾಮೀಜಿ ಹಲವಾರು ಪ್ರಸಿದ್ಧ ವ್ಯಕ್ತಿಗಳ ಭೇಟಿ ಮಾಡಿದರು. ಕಾನ್​ಸ್ಟಾಂಟಿನೋಪಲ್​ನಲ್ಲಿ ಸುಮಾರು ಹತ್ತು ದಿನವಿದ್ದ ಸ್ವಾಮೀಜಿಯವರು ಒಮ್ಮೆ ಅಲ್ಲಿನ ‘ಅಮೆರಿಕನ್ ಕಾಲೇಜ್ ಫಾರ್ ಗರ್ಲ್ಸ್​’ ಎಂಬಲ್ಲಿ ಸಾಕಷ್ಟು ದೊಡ್ಡ ಸಭೆಯೊಂದನ್ನು ಉದ್ದೇಶಿಸಿ ಹಿಂದೂಧರ್ಮದ ಬಗ್ಗೆ ಮಾತನಾಡಿದರು.

ಇಲ್ಲಿಂದ ಅವರೆಲ್ಲ ಹಡಗಿನ ಮೂಲಕ ಗ್ರೀಸಿನ ರಾಜಧಾನಿಯಾದ ಅಥೆನ್ಸಿನತ್ತ ಹೊರಟರು. ದಾರಿಯಲ್ಲಿ ಇನ್ನೂ ಕೆಲವು ದ್ವೀಪಗಳಿಗೆ ಭೇಟಿಯಿತ್ತು ಅಲ್ಲಿನ ಪ್ರೇಕ್ಷಣೀಯ ಸ್ಥಳಗಳನ್ನು ವೀಕ್ಷಿಸಿದರು. ಗ್ರೀಸಿನಲ್ಲಿ ಪ್ರಾಚೀನ ಗ್ರೀಕ್ ಸಂಸ್ಕೃತಿಯ ಕುರುಹಾಗಿ ಉಳಿದಿರುವ ಶಿಥಿಲ ದೇವಾಲಯಗಳನ್ನು ವೀಕ್ಷಿಸುವಾಗ ಸ್ವಾಮೀಜಿ ತಮ್ಮ ಸಂಗಡಿಗರಿಗೆ ಅಲ್ಲಿನ ಸ್ಥಳಪುರಾಣವನ್ನು ಕಣ್ಣಿಗೆ ಕಟ್ಟುವಂತೆ ಬಣ್ಣಿಸಿದರು. ಅಲ್ಲದೆ ಅವರ ಪುರಾತನ ಪ್ರಾರ್ಥನೆಗಳನ್ನೂ ಸ್ವರಬದ್ಧವಾಗಿ ಹಾಡಿ ತೋರಿಸಿದರು.

ಇಲ್ಲಿ ಸುಮಾರು ಮೂರು ದಿನಗಳನ್ನು ಕಳೆದು ಅವರೀಗ ರಷ್ಯಾದ ಹಡಗಾದ ‘ಜ್ಸಾರ್​’ನ್ನು ಏರಿ ಈಜಿಪ್ಟಿನೆಡೆಗೆ ಸಾಗಿದರು. ಪ್ರಪಂಚದ ಅತಿ ಪ್ರಾಚೀನ ನಾಗರಿಕತೆಗಳಲ್ಲೊಂದಾದ ಈಜಿಪ್ಟಿನ ನಾಗರಿಕತೆಯ ಗತವೈಭವವನ್ನು ಅವರ ಮನಸ್ಸು ಮೆಲುಕು ಹಾಕುತ್ತಿತ್ತು. ಆ ಅದ್ಭುತ ನಾಗರಿ ಕತೆಯ ಪಳೆಯುಳಿಕೆಗಳನ್ನು ಹೊಂದಿರುವ ಕೈರೋದ ವಸ್ತುಸಂಗ್ರಹಾಲಯವನ್ನು ವೀಕ್ಷಿಸಲು ಅವರು ಅತ್ಯಂತ ಉತ್ಸುಕರಾಗಿದ್ದರು–ಅಥವಾ ಹಾಗೆ ಕಂಡುಬರುತ್ತಿತ್ತು. ಏಕೆಂದರೆ ತಮ್ಮ ಹೃದಯಾಂತರಾಳದಲ್ಲಿ ಅವರು ಎಲ್ಲದರಿಂದಲೂ ದೂರವಾಗುತ್ತಿದ್ದರು. ಈಜಿಪ್ಟಿನ ಪಿರಮಿಡ್ಡು ಗಳು, ಸ್ಫಿಂಕ್ಸ್, ದೇವಾಲಯಗಳು–ಯಾವುವೂ ಅವರಿಗೆ ಆನಂದ ನೀಡಲಿಲ್ಲ. ಸಕಲ ಭೋಗ ಸಂಪತ್ತು ಅಧಿಕಾರಗಳ ಕ್ಷಣಭಂಗುರತೆಯು ಅವರನ್ನು ಮಾಯೆಯ ಭಯಂಕರ ಬಂಧನದ ಬಗ್ಗೆ ತೀವ್ರವಾಗಿ ಆಲೋಚಿಸುವಂತೆ ಮಾಡುತ್ತಿತ್ತು. ಆ ಬಂಧನದಲ್ಲಿ ಸಿಲುಕಿಕೊಂಡವರ ಬಗ್ಗೆ ಅವರಿಗಿದ್ದ ಮರುಕವು ಕ್ರಿಸ್ತನ ಮರುಕಕ್ಕೆ ಸಮವಾಗಿತ್ತು. ಶಿಕಾಗೋದಿಂದ ನ್ಯೂಯಾರ್ಕಿಗೆ ಹೊರಡುವಂದು ಅವರು ಉದ್ಗರಿಸಿರಲಿಲ್ಲವೆ–‘ಓಹ್, ಈ ಮಾನವ ಬಂಧನಗಳನ್ನು ಕತ್ತರಿಸು ವುದು ಎಷ್ಟು ಕಷ್ಟ!’ ಎಂದು?

ಅವರ ಈ ಮರುಕವನ್ನು ತೋರಿಸುವಂತಹ ಘಟನೆಯೊಂದು ಕೈರೋದಲ್ಲಿ ನಡೆಯಿತು.

ಅವರು ಸುಮ್ಮನೆ ಲೋಕಾಭಿರಾಮವಾಗಿ ಮಾತನಾಡಿದರೂ ಕೇಳುಗರನ್ನು ಅದು ಮಂತ್ರ ಮುಗ್ಧರನ್ನಾಗಿಸುತ್ತಿತ್ತು. ಅವರ ಮಾತಿನ ಮೋಡಿಗೊಳಗಾಗಿ ಕುಳಿತುಬಿಟ್ಟರೆ ಯಾರಿಗೂ ಸಮಯದ ಅರಿವೇ ಇರುತ್ತಿರಲಿಲ್ಲ. ರೈಲುನಿಲ್ದಾಣದ ವಿಶ್ರಾಂತಿ ಕೋಣೆಯಲ್ಲಿ ಹೀಗೆ ಮಾತನಾ ಡುತ್ತ ಕುಳಿತುಬಿಟ್ಟು ಟ್ರೈನು ತಪ್ಪಿಹೋದದ್ದಂತೂ ಅದೆಷ್ಟೋ ಸಲ! ಈ ಗುಂಪಿನಲ್ಲಿ ಎಲ್ಲರಿ ಗಿಂತ ಜಾಗರೂಕಳೂ ಮೌನಿಯೂ ಆಗಿರುತ್ತಿದ್ದ ಮೆಕ್​ಲಾಡ್ ಕೂಡ ಆಗಾಗ ಮೋಸಹೋಗುತ್ತಿ ದ್ದಳು. ಇಂತಹ ಸಂದರ್ಭಗಳಲ್ಲಿ ಅವರು, ಹೋಗಬೇಕಾಗಿದ್ದ ಸ್ಥಳಕ್ಕಿಂತ ಎಷ್ಟೋ ದೂರದ ನಿಲ್ದಾಣದಲ್ಲಿ ಸಿಕ್ಕಿಹಾಕಿಕೊಂಡು ಪೇಚಿಗೀಡಾಗಬೇಕಾಗುತ್ತಿತ್ತು. ಹೀಗೆಯೇ ಒಂದು ದಿನ ಕೈರೋ ದಲ್ಲಿ ಮಾತಿನಲ್ಲಿ ಮುಳುಗಿಹೋಗಿ ಎಲ್ಲೋ ಉಳಿದುಕೊಂಡುಬಿಟ್ಟರು. ದಾರಿ ಬೇರೆ ತಪ್ಪಿ ಹೋಯಿತು. ಆಗ ನಡೆದ ಒಂದು ಹೃದಯಸ್ಪರ್ಶಿ ಘಟನೆಯನ್ನು ಮೇಡಂ ಕಾಲ್ವೆ ತನ್ನ ಆತ್ಮಚರಿತ್ರೆಯಲ್ಲಿ ಹೀಗೆ ಬಣ್ಣಿಸಿದ್ದಾಳೆ–

“... ದಾರಿ ತಪ್ಪಿದ ನಾವು ಯಾವುದೋ ಕುಪ್ರಸಿದ್ಧ ರಸ್ತೆಗೆ ಬಂದುಬಿಟ್ಟೆವು. ಅರ್ಧಂಬರ್ಧ ವಸ್ತ್ರಧಾರಿಗಳಾದ ಹೆಂಗಸರು ಕಿಟಕಿ ಬಾಗಿಲುಗಳಿಂದ ಕಾಣಿಸಿಕೊಳ್ಳುತ್ತಿದ್ದರು. ನಾವೆಲ್ಲ ಕಸಿವಿಸಿ ಪಟ್ಟುಕೊಳ್ಳುತ್ತಿದ್ದರೂ ಸ್ವಾಮೀಜಿ ಮಾತ್ರ ಇದಾವುದನ್ನೂ ಗಮನಿಸದೆ ನೆಟ್ಟಗೆ ಸಾವಧಾನವಾಗಿ ನಡೆದು ಹೋಗುತ್ತಿದ್ದರು. ಆಗ ಪಕ್ಕದಲ್ಲಿದ್ದ ಮುರುಕು ಕಟ್ಟಡವೊಂದರ ನೆರಳಲ್ಲಿ ಬೆಂಚಿನ ಮೇಲೆ ಕುಳಿತಿದ್ದ ಕೆಲವು ಹೆಂಗಸರು ಗಟ್ಟಿಯಾಗಿ ನಗುತ್ತ ಅವರನ್ನು ಕೂಗಿ ಕರೆದರು. ನಮ್ಮ ಗುಂಪಿನಲ್ಲಿದ್ದ ಮಹಿಳೆಯೊಬ್ಬರು ಸ್ವಾಮೀಜಿಯವರನ್ನು ಅವಸರ ಪಡಿಸುತ್ತ ಬೇಗ ಬೇಗ ಹೆಜ್ಜೆ ಹಾಕಿದರು. ಆದರೆ ಸ್ವಾಮೀಜಿ ಮೆಲ್ಲನೆ ಗುಂಪಿನಿಂದ ಬೇರ್ಪಟ್ಟು ಆ ಹೆಂಗಸರು ಕುಳಿತಿದ್ದ ಬೆಂಚಿನೆಡೆಗೆ ನಡೆದರು.

“ಹತ್ತಿರ ಹೋಗಿ ನಿಂತು ಸ್ವಾಮೀಜಿ ಉದ್ಗರಿಸಿದರು, ‘ಅಯ್ಯೋ, ದುರ್ದೈವವೆ! ನೋಡಾ ಮಗಳನ್ನು! ತಾನು ಯಾರೆಂಬುದನ್ನೇ ಮರೆತುಬಿಟ್ಟಿದ್ದಾಳೆ. ತನ್ನ ದೈವತ್ವವನ್ನು ದೇಹದೊಳಗೆ ಮರೆಮಾಚಿದ್ದಾಳೆ!’

“ಹೀಗೆಂದ ಸ್ವಾಮೀಜಿಯವರು, ವೇಶ್ಯಾವೃತ್ತಿಗಿಳಿದ ಹೆಂಗಸರನ್ನು ಕಂಡು ಕರುಣೆಯಿಂದ ಮರುಗಿದ ಕ್ರಿಸ್ತನಂತೆ, ಕಣ್ಣೀರು ಸುರಿಸತೊಡಗಿದರು. ಇದನ್ನು ಕಂಡ ಆ ಹೆಂಗಸರು ಆಶ್ಚರ್ಯ ನಾಚಿಕೆಗಳಿಂದ ಸ್ತಬ್ಧರಾದರು. ಒಬ್ಬಳು ಮುಂದೆ ಬಾಗಿ (ಭಕ್ತಿ ಗೌರವ ಸೂಚಕವಾಗಿ) ಅವರ ಬಟ್ಟೆಯ ಅಂಚನ್ನು ಮುಟ್ಟಿ ಮುತ್ತಿಟ್ಟು ಮುರುಕು ಸ್ಪ್ಯಾನಿಷ್ ಬಾಷೆಯಲ್ಲಿ ‘ದೇವಮಾನವ! ದೇವಮಾನವ!’ ಎಂದುದ್ಗರಿಸಿದಳು. ಮತ್ತೊಬ್ಬಳು ಅವರ ಪವಿತ್ರ ಕಂಗಳಿಂದ ತನ್ನ ಅಪವಿತ್ರ ವ್ಯಕ್ತಿತ್ವವನ್ನು ಮುಚ್ಚಿಕೊಳ್ಳಲೋ ಎಂಬಂತೆ, ಇದ್ದಕ್ಕಿದ್ದಂತೆ ತನ್ನ ಕೈಗಳಿಂದ ಮುಖವನ್ನು ಮರೆಮಾಚಿಕೊಂಡಳು.”

ಸ್ವಾಮೀಜಿಯವರು ಮತ್ತೆ ತಮ್ಮ ಗುಂಪನ್ನು ಸೇರಿಕೊಂಡು ಮುನ್ನಡೆದರು. ಅವರ ಮನಸ್ಸೀಗ ಹೆಚ್ಚು ಹೆಚ್ಚು ಅಂತರ್ಮುಖವಾಗತೊಡಗಿತ್ತು. ಅವರನ್ನು ಯಾವುದೋ ಚಿಂತೆ ಕಾಡುತ್ತಿದ್ದಂ ತಿತ್ತು. ಇತ್ತ ಭಾರತದಲ್ಲಿ ಅವರ ನೆಚ್ಚಿನ ಅನುಯಾಯಿ-ಸ್ನೇಹಿತ ಕ್ಯಾಪ್ಟನ್ ಸೇವಿಯರ್ ಮರಣೋನ್ಮುಖರಾಗಿದ್ದರು. ಇದು ಅವರ ಅಂತರ್ದೃಷ್ಟಿಗೆ ಗೋಚರವಾಗಿದ್ದಿರಬೇಕು. ಆದ್ದ ರಿಂದಲೋ ಏನೋ ಅವರು ಇದ್ದಕ್ಕಿದ್ದಂತೆ ಚಡಪಡಿಸಲಾರಂಭಿಸಿದರು. ಸೇವಿಯರರ ಮರಣ ಕಾಲದಲ್ಲಿ ಅವರ ಬಳಿಯಿರಬೇಕೆಂದು ಬಯಸಿರಬಹುದು. ಇದಲ್ಲದೆ ಇತರ ಕಾರಣಗಳೂ ಇದ್ದುವೆಂದು ತೋರುತ್ತದೆ. ಸ್ವಾಮೀಜಿಯವರ ಚಡಪಡಿಕೆಯನ್ನು ಕಂಡು ಮೇಡಂ ಕಾಲ್ವೆ ಕಾರಣ ವೇನೆಂದು ಕೇಳಿದಳು. ಆಗ ಮತ್ತೊಬ್ಬರು ಹೇಳಿದರು, “ಅವರು ಭಾರತಕ್ಕೆ ಹಿಂದಿರುಗಬಯಸು ತ್ತಾರೆ” ಎಂದು. ಕಾಲ್ವೆ ಸ್ವಾಮೀಜಿಯವರನ್ನು ಕೇಳಿದಳು, “ಸ್ವಾಮೀಜಿ, ನೀವು ಹೋಗಲೇಬೇಕೆನ್ನು ವುದಾದರೆ ದಾರಿಖರ್ಚನ್ನು ನಾನು ಕೊಡುತ್ತೇನೆ. ಅದೊಂದು ದೊಡ್ಡ ವಿಷಯವಲ್ಲ. ಆದರೆ ನೀವೇಕೆ ನಮ್ಮಿಂದ ದೂರವಾಗಬಯಸುತ್ತೀರಿ?”

ಕಾಲ್ವೆಯ ಔದಾರ್ಯದ ಮಾತನ್ನು ಕೇಳಿ ಇದ್ದಕ್ಕಿದ್ದಂತೆ ಸ್ವಾಮೀಜಿಯವರ ಕಣ್ಣು ಹನಿ ಗೂಡಿತು. ಅವರೆಂದರು, “ನಾನು ಸಾಯುವ ಮುನ್ನ ನನ್ನ ಸೋದರರೊಂದಿಗಿರಬೇಕು ಎಂಬುದು ನನ್ನಾಸೆ. ಆದ್ದರಿಂದ ನಾನೀಗ ಭಾರತಕ್ಕೆ ಹೋಗಬೇಕು.” ಸ್ವಾಮೀಜಿ ತಾವು ಸಾಯುವ ಮಾತನ್ನಾಡಿ ದಾಗ ಎಮ್ಮಾ ಕಾಲ್ವೆ ದಿಗ್ಭ್ರಾಂತಳಾಗಿ ಉದ್ಗರಿಸಿದಳು, “ಆದರೆ ಸ್ವಾಮೀಜಿ, ನೀವು ಸಾಯು ವಂತಿಲ್ಲ! ನೀವು ನಮಗೆ ಬೇಕು!... ” ಅವರು ಅವಳ ಮಾತನ್ನು ಪರಿಗಣಿಸಲೇ ಇಲ್ಲ. ಬದಲಾಗಿ, ತಾವು ಸಾಯುವ ದಿನಾಂಕವನ್ನು–೧೯ಂ೨ ಜುಲೈ ನಾಲ್ಕು ಎಂದು–ಆಕೆಗೆ ತಿಳಿಸಿ ದರು.\footnote{* ಸ್ವಾಮೀಜಿಯವರು ತಮ್ಮ ಮಹಾಸಮಾಧಿಯ ದಿನವನ್ನು ಮುನ್ನುಡಿದಿದ್ದರೆಂಬ ಈ ವಿಷಯವು ನಮಗೆ ತಿಳಿದುಬಂದಿರುವುದು ಮೇಡಂ ಪಾಲ್ ವರ್ಡಿಯರ್ ಎಂಬವಳ ಟಿಪ್ಪಣಿಗಳಿಂದ. ಈಕೆ ಎಮ್ಮಾ ಕಾಲ್ವೆಯ ಆತ್ಮೀಯ ಸ್ನೇಹಿತೆ ಮತ್ತು ಸ್ವಾಮೀಜಿಯವರ ಭಕ್ತೆ; ಮೆಕ್​ಲಾಡಳಿಗೂ ಈಕೆ ನಿಕಟ ಪರಿಚಿತೆ. ಚರ್ಚಾಸ್ಪದವಾದ ಈ ವಿಷಯಗಳನ್ನು ಮೇಡಂ ಕಾಲ್ವೆ ತನ್ನ ಆತ್ಮಚರಿತ್ರೆಯಲ್ಲಿ ಉಲ್ಲೇಖಿಸಿಲ್ಲವಾದರೂ ಪಾಲ್ ವರ್ಡಿಯರ್​ಳಿಗೆ ಹೇಳಿದಳಂತೆ. ಕಾಲ್ವೆ ಹಾಗೂ ಮೆಕ್​ಲಾಡ್ ಇಬ್ಬರೂ ತನಗೆ ತಿಳಿಸಿದ ವಿಷಯಗಳನ್ನು ಪಾಲ್ ವರ್ಡಿಯರ್ ಬರೆದಿಟ್ಟುಕೊಳ್ಳುತ್ತಿದ್ದಳು. ಆದರೆ, ಒಂದು ಅಂಶವೇನೆಂದರೆ, ಈ ಘಟನೆಯ ಬಗ್ಗೆ ಮಿಸ್ ಮೆಕ್​ಲಾಡ್ ಎಲ್ಲೂ ಪ್ರಸ್ತಾಪಿಸಿಲ್ಲ, ಅಥವಾ ಪಾಲ್ ವರ್ಡಿಯರ್​ಳಿಗೂ ಹೇಳಿರಲಿಲ್ಲ.} ಅವಳಿಗದರಲ್ಲಿ ನಂಬಿಕೆಯುಂಟಾಗಲಿಲ್ಲ. ಈ ಮಾತನ್ನು ಪರೀಕ್ಷಿಸಬೇಕೆಂದು ಅವಳಿಗನ್ನಿ ಸಿತು. ಮುಂದೆ ಅವಳು ಸಂಗೀತ ಪ್ರದರ್ಶನ ನೀಡಲು ವಿವಿಧ ದೇಶಗಳಲ್ಲಿ ಸಂಚಾರವನ್ನು ಕೈಗೊಂಡಾಗ ಭಾರತಕ್ಕೂ ಆಗಮಿಸಿ, ಸ್ವಾಮೀಜಿಯವರು ನಿಧನರಾದ ದಿನವನ್ನು ಕೇಳಿ ತಿಳಿದು ಕೊಂಡಳು. ಜುಲೈ ನಾಲ್ಕರಂದೇ ಅವರು ದೇಹತ್ಯಾಗ ಮಾಡಿದರೆಂಬುದು ದೃಢಪಟ್ಟಿತು.

ಹೀಗೆ ಸ್ವಾಮೀಜಿ ಇದ್ದಕ್ಕಿದ್ದಂತೆ ಭಾರತಕ್ಕೆ ಹೊರಟು ನಿಂತಾಗ ಅವರ ಶಿಷ್ಯರಿಗೆ, ಸ್ನೇಹಿತರಿಗೆ ದುಃಖವಾಯಿತಾದರೂ ಅವರ ಮಾತಿನಲ್ಲಿ ದೃಢತೆಯನ್ನು ಸ್ಪಷ್ಟವಾಗಿ ಕಂಡರು. ಅವರನ್ನು ತಮ್ಮೊಡನೆ ಉಳಿದುಕೊಳ್ಳುವಂತೆ ಅಥವಾ ಪ್ಯಾರಿಸಿಗೆ ಹಿಂದಿರುಗಿ ಬರುವಂತೆ ಒತ್ತಾಯಿಸಲು ಸಾಧ್ಯವಿಲ್ಲವೆಂಬುದು ಅವರಿಗೆಲ್ಲ ಮನದಟ್ಟಾಯಿತು. ಸ್ವಾಮೀಜಿಯವರಿಲ್ಲದಿದ್ದ ಮೇಲೆ ಇನ್ನು ಅವರ ಪ್ರವಾಸದಲ್ಲಿ ಸ್ವಾರಸ್ಯವೇನು ಉಳಿದೀತು? ಮೇಡಂ ಕಾಲ್ವೆಯ ಪಾಲಿಗೆ ಅವರೊಬ್ಬ ಮಹಾ ಸಂತ ಹಾಗೂ ಪ್ರಿಯ ತಂದೆ. ಮಿಸ್ ಮೆಕ್​ಲಾಡ್​ಳಿಗೆ ಅವರೊಬ್ಬ ಪ್ರವಾದಿ, ಸ್ನೇಹಿತ. ಜೂಲ್ಸ್ ಬ್ಯೂಸ್​ನ ದೃಷ್ಟಿಯಲ್ಲಿ ಸ್ವಾಮೀಜಿ ಒಬ್ಬ ಅದ್ಭುತ ಚಿಂತಕ, ಮಿಗಿಲಾಗಿ ಒಬ್ಬ ದೇವ ಮಾನವ. ಪ್ರತಿಯೊಬ್ಬರಿಗೂ ಅವರೊಬ್ಬ ಪ್ರಿಯತಮ ಸಂಗಾತಿ. ಇದ್ದಕ್ಕಿದ್ದಂತೆ ಹೀಗೆ ಅವರಿಂದ ಅಗಲಬೇಕಾದೀತೆಂದು ಯಾರೂ ನಿರೀಕ್ಷಿಸಿರಲಿಲ್ಲ. ಕಡೆಗೆ ನವೆಂಬರ್ ೨೬ರಂದು ಸ್ವಾಮೀಜಿ ಹೊರಟು ನಿಂತಾಗ ಅವರ ಸಹಚರರೆಲ್ಲ ಭಾರವಾದ ಹೃದಯದಿಂದ ಬೀಳ್ಕೊಟ್ಟರು. ಎಸ್. ಎಸ್. ರುಬ್ಯಾಟಿನೋ ಎಂಬ ಹಡಗನ್ನು ಹತ್ತಿ ಸ್ವಾಮೀಜಿ ಮುಂಬಯಿಯತ್ತ ಒಬ್ಬರೇ ಹೊರ ಟರು. ಅವರು ಭಾರತಕ್ಕೆ ಬರುತ್ತಿರುವ ಸುದ್ದಿ ಅವರಿಗಿಂತ ಮುಂಚೆ ಭಾರತವನ್ನು ಮುಟ್ಟಲಿಲ್ಲ.

ಕೈರೋದಿಂದ ಮುಂಬಯಿಯವರೆಗಿನ ಹತ್ತು ಹನ್ನೆರಡು ದಿನಗಳ ಪ್ರಯಾಣದಲ್ಲಿ ಸ್ವಾಮೀಜಿ ಯವರ ಪಾಲಿಗೆ ವಿಶೇಷವೇನೂ ಇದ್ದಿರಲಾರದು. ಆದರೆ ಅವರ ಸಹಪ್ರಯಾಣಿಕರ ಪೈಕಿ ಕಡೆಯ ಪಕ್ಷ ಒಬ್ಬನಿಗಾದರೂ ಈ ಪ್ರಯಾಣವು ಮಹತ್ವದ್ದಾಗಿ ಪರಿಣಮಿಸಿತು. ಅವನು ರೀವ್ಸ್ ಕ್ಯಾಲ್ಕಿನ್ಸ್ ಎಂಬೊಬ್ಬ ಅಮೆರಿಕನ್ ಪಾದ್ರಿ. ಇವನು ಸ್ವಾಮೀಜಿಯವರೊಂದಿಗಿನ ತನ್ನ ಈ ಭೇಟಿಯ ಕುರಿತಾಗಿ ಹತ್ತಾರು ವರ್ಷಗಳ ಬಳಿಕ ಯಾವುದೋ ಪತ್ರಿಕೆಯಲ್ಲಿ ಬರೆದ. ಆ ಲೇಖನವನ್ನು ಈಗ \eng{Reminiscences of Swami Vivekananda} ಎಂಬ ಪುಸ್ತಕದಲ್ಲಿ ಕಾಣಬಹುದಾಗಿದೆ.

ಕ್ಯಾಲ್ಕಿನ್ಸ್ ಎಂಬ ಈ ಪಾದ್ರಿ ಹಿಂದೆ ಶಿಕಾಗೋ ಸಮ್ಮೇಳನದಲ್ಲೇ ಸ್ವಾಮೀಜಿಯವರನ್ನು ಕಂಡಿದ್ದ. ಆದರೆ ಇವನು ಆಗತಾನೆ ಪರೀಕ್ಷೆಯನ್ನು ಪಾಸ್ ಮಾಡಿ ಒಬ್ಬ ಧರ್ಮಪ್ರಚಾರಕನಾಗಿ ನೇಮಕಗೊಂಡಿದ್ದವನು. ಆದ್ದರಿಂದ ಅವರ ರಾಜಠೀವಿಯನ್ನು, ಮಾತಿನ ಶೈಲಿಯನ್ನು ಕಂಡು ಇವನಿಗೆ ಸ್ವಲ್ಪ ಕಿರಿಕಿರಿಯೇ ಆಯಿತು. ಆಗಿನ ಅವರ ವ್ಯಕ್ತಿತ್ವದ ಬಗ್ಗೆ ಅವನು ಬರೆಯುತ್ತಾನೆ: “ತಾವೊಬ್ಬ ಅಸಾಧಾರಣ ವ್ಯಕ್ತಿಯೆಂದು ಅವರು ವಾದ ಮಾಡುತ್ತಿರಲಿಲ್ಲ. ಆದರೆ ಅವರು ಅದನ್ನು (ಪ್ರಶ್ನಾತೀತ ಸತ್ಯವೆಂಬಂತೆ) ಒಪ್ಪಿಕೊಂಡಿದ್ದರು.\eng{” ( He did not argue that he was a superior person, he admitted it.)} ಅವರು ಕೇವಲ ತಮ್ಮ ಶಕ್ತಿಯುತ ವ್ಯಕ್ತಿತ್ವದಿಂದ ಜನರನ್ನು ಮೋಡಿ ಮಾಡಿದ್ದರೇ ಹೊರತು ತಮ್ಮ ಭಾವನೆ-ಬೋಧನೆಗಳಿಂದಲ್ಲ ಎಂದು ಈತ ನಂಬಿದ್ದ. ಆದರೆ ಈಗ ಸ್ವಾಮೀಜಿಯವರನ್ನು ವೈಯಕ್ತಿಕವಾಗಿ ಭೇಟಿಯಾಗುತ್ತಿದ್ದಂತೆಯೇ ಅವನ ಪೂರ್ವಗ್ರಹ ಗಳೆಲ್ಲ ಪರಾರಿಯಾದುವು. ಈ ವೇಳೆಗೆ ರೀವ್ಸ್ ಕ್ಯಾಲ್ಕಿನ್ಸ್ ತನಗರಿವಿಲ್ಲದಂತೆಯೇ ಅವರೆಡೆಗೆ ಆಕರ್ಷಿತನಾಗಿದ್ದ. ಇದನ್ನು ಕಂಡುಕೊಂಡರೋ ಎಂಬಂತೆ ಅವನನ್ನು ಸ್ವಾಮೀಜಿ ತಾವಾಗಿಯೇ ಮಾತನಾಡಿಸಿದರು:

“ನೀವು ಅಮೆರಿಕನ್ನರೆ?”

“ಹೌದು.”

“ಮಿಷನರಿಗಳೆ?”

“ಹೌದು.”

“ನೀವೇಕೆ ನನ್ನ ರಾಷ್ಟ್ರದಲ್ಲಿ ನಿಮ್ಮ ಧರ್ಮವನ್ನು ಪ್ರಚಾರ ಮಾಡುವುದು?”

“ನೀವೇಕೆ ನನ್ನ ರಾಷ್ಟ್ರದಲ್ಲಿ ನಿಮ್ಮ ಧರ್ಮವನ್ನು ಪ್ರಚಾರ ಮಾಡುವುದು?”

ಅರೆಕ್ಷಣ ಒಬ್ಬರನ್ನೊಬ್ಬರು ಮೌನವಾಗಿ ದಿಟ್ಟಿಸಿದ ಇಬ್ಬರೂ ಗಟ್ಟಿಯಾಗಿ ನಕ್ಕುಬಿಟ್ಟರು. ತಕ್ಷಣದಿಂದ ಇಬ್ಬರೂ ಸ್ನೇಹಿತರಾದರು.

ಹಡಗಿನಲ್ಲಿದ್ದ ಇತರ ಐರೋಪ್ಯ ಪ್ರಯಾಣಿಕರು ಸ್ವಾಮೀಜಿಯವರನ್ನು ಕೆಣಕಿ, ವಾದಕ್ಕೆಳೆದು, ಕಾಲ್ತೊಡರಿಸುವ ಪ್ರಯತ್ನದಲ್ಲಿ ತೊಡಗಿದರು. ಆದರೆ ಈ ಪ್ರಯತ್ನದಲ್ಲಿ ವಿಫಲರಾಗಿ, ಮೆಲ್ಲಗೆ ಒಬ್ಬೊಬ್ಬರಾಗಿ ದೂರವಾದರು. ಈ ಸಂದರ್ಭಗಳಲ್ಲಿ ಕೆಲವೊಮ್ಮೆ ಸ್ವಾಮೀಜಿಯವರ ವಾಗ್ಝರಿ “ಮಹಾಪೂರ ಬಂದ ಗಂಗಾನದಿಯಂತಿರುತ್ತಿತ್ತು; ಅದನ್ನು ಯಾರೂ ತಡೆಯಲು ಶಕ್ಯವಿರಲಿಲ್ಲ” ಎನ್ನುತ್ತಾನೆ ಕ್ಯಾಲ್ಕಿನ್ಸ್. ಆದರೆ ಉಳಿದ ಸಮಯದಲ್ಲಿ, ವಿಶೇಷವಾಗಿ ರಾತ್ರಿಯ ವೇಳೆಯಲ್ಲಿ ಅವರೊಂದಿಗೆ ಹಡಗಿನ ಡೆಕ್ಕಿನ ಮೇಲೆ ಅಡ್ಡಾಡುವಾಗ ಆತ ಕಂಡ ವ್ಯಕ್ತಿತ್ವ ಸಂಪೂರ್ಣ ವಿಭಿನ್ನ. “ವಿವೇಕಾನಂದರ ನಿಗೂಢತೆಯು ಒಂದು ಅದ್ಭುತ, ಕೌತುಕಮಯ ವಸ್ತುವಾಗಿತ್ತು. ಏಕೆಂದರೆ ಅದು ಯಾವುದೇ ಬಾಹ್ಯಶಕ್ತಿಯ ಪ್ರಭಾವಕ್ಕೆ ಒಳಗಾಗುತ್ತಿರಲಿಲ್ಲ. ನಮ್ಮ ಸಂಭಾಷಣೆಯು ಆತ್ಮ-ಜೀವಗಳ ಕುರಿತಾಗಿ ತಿರುಗಿಕೊಂಡಾಗ ಅವರ ಭಾರವಾದ ಕಣ್​ರೆಪ್ಪೆಗಳು ನಿಧಾನವಾಗಿ ಕೆಳಮುಖವಾಗುತ್ತಿದ್ದುವು; ನನಗೆ ಪ್ರವೇಶವಿಲ್ಲದ ಯಾವುದೋ ಲೋಕದೊಳಕ್ಕೆ ಅವರು ತೇಲಿಹೋಗುತ್ತಿದ್ದರು” ಎಂದು ರೀವ್ಸ್ ಕ್ಯಾಲ್ಕಿನ್ಸ್ ಬರೆಯುತ್ತಾನೆ.

‘ಎಸ್. ಎಸ್. ರುಬ್ಯಾಟಿನೋ’ ಡಿಸೆಂಬರ್ ಆರರಂದು ಮುಂಬಯಿಗೆ ಬಂದು ಸೇರಿತು. ಮತ್ತೆ ತಮ್ಮ ಪ್ರಿಯ ಸಹಸಂನ್ಯಾಸಿಗಳನ್ನು ಕೂಡಿಕೊಳ್ಳುವ ಆಲೋಚನೆಯಿಂದ ಸ್ವಾಮೀಜಿ ಹರ್ಷಚಿತ್ತರಾಗಿದ್ದರು. ಐರೋಪ್ಯ ಪೋಷಾಕಿನಲ್ಲಿ ಒಬ್ಬ ಭಾರೀ ಅಧಿಕಾರಿಯಂತೆ ಕಾಣಿಸುತ್ತಿದ್ದ ಅವರನ್ನು ಮದ್ರಾಸಿನ ಒಬ್ಬರು ಪ್ರೊಫೆಸರ್ ಬಿಟ್ಟು ಬೇರೆ ಯಾರೂ ಗುರುತಿಸಲಿಲ್ಲ. ‘ಬಾಂಬೆ ಎಕ್ಸ್​ಪ್ರೆಸ್​’ ಟ್ರೈನು ಹತ್ತಿ ಸ್ವಾಮೀಜಿ ಸೀದಾ ಕಲ್ಕತ್ತದತ್ತ ಹೊರಟರು. ಇಲ್ಲಿ ಅವರು ತಮ್ಮ ಹಳೆಯ ಸ್ನೇಹಿತ ಮನ್ಮಥನಾಥ ಭಟ್ಟಾಚಾರ್ಯರನ್ನು ಆಕಸ್ಮಿಕವಾಗಿ ಸಂಧಿಸಿದರು. ಕ್ಷಣಕಾಲ ಇಬ್ಬರೂ ಒಬ್ಬರನ್ನೊಬ್ಬರು ಅಚ್ಚರಿಯಿಂದ ದಿಟ್ಟಿಸಿದರು; ಬಳಿಕ ಆನಂದದಿಂದ ಮಾತುಕತೆ ಯಲ್ಲಿ ತೊಡಗಿದರು.

ಡಿಸೆಂಬರ್ ಒಂಬತ್ತರಂದು ಸ್ವಾಮೀಜಿಯವರು ಕಲ್ಕತ್ತ ತಲುಪಿದರು. ಗಾಡಿಯಲ್ಲಿ ಬೇಲೂರು ಮಠಕ್ಕೆ ಬಂದಾಗ ರಾತ್ರಿಯಾಗಿತ್ತು. ಊಟದ ಗಂಟೆ ಹೊಡೆದಿತ್ತು. ಆಶ್ರಮವಾಸಿ ಗಳೆಲ್ಲ ಊಟದ ಮನೆಯಲ್ಲಿ ಸೇರಿದ್ದರು. ಆಗ ತೋಟದ ಮಾಲಿ ಓಡಿಬಂದು, “ಯಾರೋ ಸಾಹೇಬರು ಬಂದಿದ್ದಾರೆ. ಅವರು ಕಾಂಪೌಂಡು ಹಾರಿ ಸೀದಾ ಇಲ್ಲಿಗೇ ಬರುತ್ತಿದ್ದಾರೆ!” ಎಂದು ಒದರಿದ. ಊಟಕ್ಕೆ ಕುಳಿತ ಸಾಧುಗಳು ಒಬ್ಬರ ಮುಖವನ್ನೊಬ್ಬರು ನೋಡಿಕೊಂಡರು. ಸಾಹೇಬ? ಇಷ್ಟು ಹೊತ್ತಿನಲ್ಲಿ? ಕಾಂಪೌಂಡು ಹಾರಿಕೊಂಡು! ಯಾರಿರಬಹುದಪ್ಪ! ಏನಿರಬಹುದು ಅಂತಹ ಅವಸರದ ಕೆಲಸ? ನೋಡೋಣ ಎಂದು ಕೆಲವರು ಎದ್ದು ಹೊರಟು. ಹಿರಿಯ ಸಾಧು ಗಳು ಅಲ್ಲೇ ಕಾದು ಕುಳಿತರು. ಅಷ್ಟರಲ್ಲಿ ‘ಸಾಹೇಬ’ ಅಲ್ಲಿಗೇ ಸೀದಾ ನುಗ್ಗಬೇಕೆ! ನೋಡುತ್ತಾರೆ –ಸ್ವಾಮೀಜಿ! ಅವರಿಗೆ ತಮ್ಮ ಕಣ್ಣುಗಳನ್ನೇ ನಂಬಲಾಗಲಿಲ್ಲ. “ಓ, ಸ್ವಾಮೀಜಿ ಬಂದಿದ್ದಾರೆ, ಸ್ವಾಮೀಜಿ!” ಎಂಬ ಕೂಗು ಹೊರಟಿತು. ಸ್ವಾಮೀಜಿ ನಗುತ್ತ ಹೇಳಿದರು, “ಊಟದ ಗಂಟೆ ಹೊಡೆದದ್ದು ದೂರಕ್ಕೇ ಕೇಳಿಸಿತು. ತಡಮಾಡಿದರೆ ಎಲ್ಲಿ ತುತ್ತಿಗೆ ಕುತ್ತು ಬೀಳುತ್ತದೋ ಎಂದು ಹೆದರಿ ಓಡಿಬಂದೆ!” ತಕ್ಷಣ ಅವರಿಗೊಂದು ಜಾಗ ಮಾಡಿ, ಚಾಪೆ ಹಾಕಿ ಅವರಿಗೆ ಪ್ರಿಯವಾದ ಬಿಸಿಬಿಸಿ ಕಿಚುಡಿ (ತೊವ್ವೆ ಅನ್ನ) ಬಡಿಸಲಾಯಿತು. ಅವರು ಕಿಚುಡಿಯ ರುಚಿ ಕಂಡು ಎಷ್ಟು ದಿನ ವಾಗಿತ್ತೋ! ಈಗ ಮಠದಲ್ಲೊಂದು ಆನಂದದ ಅಲೆಯೇ ಎದ್ದಿತು. ಎಲ್ಲರಿಗೂ ಹಿಡಿಸಲಾರ ದಷ್ಟು ಸಂತೋಷ. ತಮ್ಮ ಸ್ವಾಮೀಜಿ ಮತ್ತೆ ತಮ್ಮ ಮಧ್ಯದಲ್ಲಿದ್ದಾರೆ! ಬಳಿಕ ಸ್ವಾಮೀಜಿ ತಮ್ಮ ರಸಾನುಭವಗಳನ್ನೆಲ್ಲ ವಿವರಿಸುತ್ತಿದ್ದಂತೆ ಎಲ್ಲರೂ ಅವರನ್ನು ಮುತ್ತಿಕೊಂಡರು. ಆದರೆ ಅವ ರೆಲ್ಲ ಎಷ್ಟು ಆನಂದಭರಿತರಾಗಿದ್ದರೆಂದರೆ, ಅದನ್ನೆಲ್ಲ ಬರೆದಿಟ್ಟುಕೊಳ್ಳಬೇಕೆಂದು ಯಾರಿಗೂ ತೋಚಲಿಲ್ಲ.

ಸ್ವಾಮೀಜಿಯವರಿಗೆ ತಾವು ಪಶ್ಚಿಮ ದೇಶಗಳಿಗೆ ನೀಡಿದ ಈ ಭೇಟಿಯು ಮೊದಲನೆಯದಕ್ಕಿಂತ ತುಂಬ ಭಿನ್ನವಾಗಿ ತೋರಿತ್ತು. ಮೊದಲ ಸಲ ಅವರು ಅಲ್ಲಿನ ಶಕ್ತಿ, ಸಂಘಟನೆ, ತಂತ್ರಜ್ಞಾನ, ಪ್ರಜಾಪ್ರಭುತ್ವಗಳನ್ನು ಕಂಡು ಮೆಚ್ಚಿಕೊಂಡಿದ್ದರು. ಆದರೆ ಈಗ ಆ ರಾಷ್ಟ್ರಗಳನ್ನು ಮತ್ತೂ ಆಳವಾಗಿ ಅಧ್ಯಯಿಸಿ ನೋಡಿದಾಗ, ಅದರ ಪ್ರಗತಿಯ ಆಕಾಂಕ್ಷೆಯ ಹಿಂದಿರುವುದೆಲ್ಲ ಕೇವಲ ದುರಾಸೆ, ಸ್ವಾರ್ಥ ಮತ್ತು ಅಧಿಕಾರ-ಸವಲತ್ತುಗಳಿಗಾಗಿ ಹೋರಾಟವೇ ವಿನಾ ಇನ್ನೇನೂ ಅಲ್ಲ ಎನ್ನುವುದು ಕಂಡು ಜುಗುಪ್ಸೆಗೊಂಡರು. “ತೋಳಗಳ ಹಿಂಡಿನಲ್ಲಿ ಎಂತಹ ಸೌಂದರ್ಯ?” ಎಂದು ಅವರೊಮ್ಮೆ ಉದ್ಗರಿಸಿದರು. ಆದರೆ ಅವರಿಗೆ ಹೀಗೆ ಅನ್ನಿಸಿದರೂ, ಸಾಧ್ಯವಾದರೆ ಮತ್ತೊಮ್ಮೆ ಇಂಗ್ಲೆಂಡ್, ಅಮೆರಿಕಗಳಿಗೆ ಹೋಗಿಬರುವ ಆಲೋಚನೆಯೂ ಅವರಿಗಿತ್ತು. ಅಲ್ಲದೆ ಜಪಾನ್ ದೇಶಕ್ಕೆ ಹೋಗುವ ಯೋಜನೆಯನ್ನು ಹಾಕಿಕೊಂಡಿದ್ದರು. ಆದರೆ ದಿನದಿನಕ್ಕೂ ಅವರ ಆರೋಗ್ಯ ಹದಗೆಡುತ್ತಲೇ ಬಂದದ್ದರಿಂದ ಈ ಯೋಜನೆಗಳಾವುವೂ ಕೈಗೂಡಲೇ ಇಲ್ಲ. ಅಥವಾ, ಅವರ ಮನಸ್ಸಿನಲ್ಲಿ ನಿಜಕ್ಕೂ ಏನಿತ್ತೋ ಬಲ್ಲವರಾರು! ತಾವು ತಮ್ಮ ನಲವತ್ತನೇ ವಯಸ್ಸನ್ನು ಕಾಣುವುದಿಲ್ಲ ಎಂದು ಅವರು ಎಷ್ಟೋ ಸಲ ಹೇಳಿದ್ದರು. ಈಗ ಅವರಿಗೆ ಮೂವತ್ತೆಂಟನೆಯ ವರ್ಷ ನಡೆಯುತ್ತಿದೆ. ಅವರ ಜೀವನದ ಕೊನೆಯ ಅಧ್ಯಾಯ ಪ್ರಾರಂಭವಾಗಿದೆ.

