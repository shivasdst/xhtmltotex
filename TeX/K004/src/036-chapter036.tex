
\chapter{ಮಾಯಾವತಿಯತ್ತ ಮುನ್ನಡೆದರು}

\noindent

ಇದೀಗ ಸ್ವಾಮೀಜಿ ತಮ್ಮದೇ ಆದ ಲೋಕಕ್ಕೆ–ತಮ್ಮ ಪ್ರಿಯ ಭಾರತಕ್ಕೆ, ತಮ್ಮ ಆಶ್ರಮ ಜೀವನಕ್ಕೆ ಹಿಂದಿರುಗಿದ್ದಾರೆ. ಅವರ ದೇಹಾರೋಗ್ಯ ಶಿಥಿಲವಾಗಿದ್ದರೂ, ಅವರು ಅದನ್ನು ಲೆಕ್ಕಿಸದೆ ತಮ್ಮ ‘ಭಾರತದ ಕೆಲಸ’ವನ್ನು ಹಾಗೂ ಆಶ್ರಮವಾಸಿಗಳಿಗೆ ತರಬೇತಿ ನೀಡುವ ಕೆಲಸವನ್ನು ಪುನರಾರಂಭಗೊಳಿಸಲು ಉದ್ಯುಕ್ತರಾದರು. ಮಠಕ್ಕೆ ಮರಳಿದ ಹತ್ತು ದಿನಗಳ ಬಳಿಕ ತಮ್ಮ ಪ್ರಿಯ ಶಿಷ್ಯೆ ಕ್ರಿಸ್ಟೀನಳಿಗೊಂದು ಪತ್ರ ಬರೆದರು:

“ಉಲ್ಲಾಸಭರಿತ ಚಟುವಟಿಕೆಯ ಪ್ಯಾರಿಸ್, ಗಂಭೀರ ಕಾನ್ಸ್ಟಾಂಟಿನೋಪಲ್, ಹೊಳೆಹೊಳೆ ಯುವ ಅಥೆನ್ಸ್, ಪಿರಮಿಡ್ಡುಗಳ ಕೈರೋ–ಎಲ್ಲವೂ ಹಿಂದುಳಿದು, ಇದೀಗ ನಾನು ಮಠದಲ್ಲಿ, ಗಂಗೆಯ ಬದಿಯ ಕೋಣೆಯಲ್ಲಿ ಕುಳಿತು ಬರೆಯುತ್ತಿದ್ದೇನೆ. ಸುತ್ತ ಎಲ್ಲವೂ ಶಾಂತ-ತಟಸ್ಥ. ವಿಶಾಲವಾದ ನದಿ ಉಜ್ವಲ ಬಿಸಿಲಿನಲ್ಲಿ ನರ್ತಿಸುತ್ತಿದೆ. ಎಲ್ಲೋ ಆಗೊಂದು ಈಗೊಂದು ಸರಕು ದೋಣಿಯ ಹುಟ್ಟುಗಳ ಸದ್ದು ಮೌನವನ್ನು ಮುರಿಯುತ್ತದೆ. ಈಗ ಇಲ್ಲಿ ಚಳಿಗಾಲ. ಆದರೆ ಮಧ್ಯಾಹ್ನವೆಲ್ಲ ಬೆಚ್ಚಗಿರುತ್ತದೆ... ಎಲ್ಲವೂ ಹರಿದ್ವರ್ಣ. ರತ್ನಗಂಬಳಿ ಹಾಸಿದಂಥ ಹಚ್ಚಹಸಿರು ಹುಲ್ಲು. ತಣ್ಣಗೆ ಲಘುವಾಗಿ ಆಹ್ಲಾದಕರವಾಗಿ ಬೀಸುತ್ತಿರುವ ಗಾಳಿ.”

“ಭಾರತದಲ್ಲಿ ಕೆಲತಿಂಗಳು ವಿಶ್ರಮಿಸಿ ಮುಂದಿನ ಬೇಸಿಗೆಯಲ್ಲಿ ಮತ್ತೆ ಇಂಗ್ಲೆಂಡಿಗೆ ಹೋಗುವ ಆಲೋಚನೆ...”

ಸ್ವಾಮೀಜಿಯವರು ಭಾರತಕ್ಕೆ ಇದ್ದಕ್ಕಿದ್ದಂತೆ ಹೊರಟು ಬಂದುದಕ್ಕೆ ಒಂದು ಕಾರಣವೆಂದರೆ ಸೇವಿಯರ ದೇಹಸ್ಥಿತಿಯ ಬಗ್ಗೆ ಅವರ ಅಂತರಂಗದಲ್ಲಿದ್ದ ಆತಂಕ ಎಂಬುದನ್ನು ನೋಡಿದ್ದೇವೆ. ಹೇಗಾದರಾಗಲಿ, ಸೇವಿಯರನ್ನೊಮ್ಮೆ ನೋಡಿಬಿಡಬೇಕು ಎಂದು ಅವರ ಮನಸ್ಸು ಕಾತರ ಗೊಂಡಿತ್ತು. ಮತ್ತೊಂದು ಕಾರಣವೆಂದರೆ, ಸ್ವತಃ ಅವರ ದೇಹಾರೋಗ್ಯವೂ ಕ್ಷೀಣವಾಗಿದ್ದುದು. ಅವರ ಹೃದಯದಲ್ಲಿ ಏನೋ ತೊಂದರೆ ಪ್ರಾರಂಭವಾದಂತಿತ್ತು. ಹೀಗಿದ್ದರೂ ಅವರು ಮಠಕ್ಕೆ ಮರಳಿ ಬಂದ ತಕ್ಷಣ ಮೊದಲು ವಿಚಾರಿಸಿದ್ದು ತಮ್ಮ ನೆಚ್ಚಿನ ಅನುಯಾಯಿ ಸೇವಿಯರ್ ಬಗ್ಗೆ. ಆದರೆ ಅವರು ಅಕ್ಟೋಬರ್ ೨೮ರಂದೇ ತೀರಿಕೊಂಡರೆಂದು ತಿಳಿದಾಗ ಸ್ವಾಮೀಜಿಯವರಿಗಾದ ದುಃಖ ಅಪಾರ. ಡಿಸೆಂಬರ್ ೧೧ರಂದು ಮಿಸ್ ಮೆಕ್​ಲಾಡಳಿಗೆ ಬರೆದರು–“ಅಯ್ಯೋ, ನಾನು ಅವಸರ ಪಟ್ಟು ಬಂದದ್ದು ಏನೂ ಪ್ರಯೋಜನಕ್ಕೆ ಬಾರದೆ ಹೋಯಿತು. ಸೇವಿಯರ್ ಕೆಲ ದಿನಗಳ ಹಿಂದೆಯೇ ತೀರಿಕೊಂಡರು. ಹೀಗೆ ಇಬ್ಬರು ಆಂಗ್ಲರು (ಮತ್ತೊಬ್ಬರು ಗುಡ್​ವಿನ್​) ಹಿಂದೂಗಳಾದ ನಮಗಾಗಿ ಪ್ರಾಣ ತೆತ್ತರು. ಹುತಾತ್ಮರಾಗುವುದೆಂದು ಯಾವುದನ್ನಾದರೂ ಕರೆಯಬಹುದಾದರೆ ಇದನ್ನೇ. (ಮಾಯಾವತಿಯಲ್ಲಿ ಉಳಿದುಕೊಳ್ಳುವ ವಿಷಯದಲ್ಲಿ) ಶ್ರೀಮತಿ ಸೇವಿಯರರ ನಿರ್ಧಾರವೇನೆಂದು ತಿಳಿಯಲು ಈಗತಾನೆ ಪತ್ರ ಬರೆದಿದ್ದೇನೆ.”

ಎರಡು ವಾರಗಳ ಬಳಿಕ ಮತ್ತೆ ಮೆಕ್​ಲಾಡಳಿಗೆ ಪತ್ರ ಬರೆದರು–

“ಸೇವಿಯರರ ಶರೀರವನ್ನು ಹೂಹಾರಗಳಿಂದ ಸಿಂಗರಿಸಿ, ಸಕಲ ಗೌರವಗಳೊಂದಿಗೆ ಮೆರ ವಣಿಗೆಯಲ್ಲಿ ಕರೆದೊಯ್ಯಲಾಗಿತ್ತು. ವೇದಘೋಷಗಳ ನಡುವೆ ಬ್ರಾಹ್ಮಣರೇ ಶವವನ್ನು ಹೊತ್ತು ನಡೆದರು. ಬಳಿಕ ಆಶ್ರಮದ ಸನಿಹದಲ್ಲೇ ಹರಿಯುವ ನದಿಯ ತೀರದಲ್ಲಿ ಹಿಂದೂ ವಿಧಿವಿಧಾನ ಗಳ ಪ್ರಕಾರವೇ ಸಂಸ್ಕಾರ ಮಾಡಲಾಯಿತು... ಪ್ರಿಯ ಶ್ರೀಮತಿ ಸೇವಿಯರರು ಶಾಂತ ಚಿತ್ತರಾಗಿದ್ದಾರೆ. ಅವರನ್ನು ನೋಡಲು ನಾನು ನಾಳೆ ಹೊರಡಲಿದ್ದೇನೆ. ಧೀರಾತ್ಮಳು! ಭಗವಂತ ಆಕೆಯ ಮೇಲೆ ಕೃಪೆಗೈಯಲಿ!”

ಸಾಧ್ಯವಾದಷ್ಟು ಬೇಗ ಶ್ರೀಮತಿ ಸೇವಿಯರರನ್ನು ಕಾಣಲು ಸ್ವಾಮೀಜಿ ಕಾತರರಾಗಿದ್ದರು. ತಾವು ಅಲ್ಲಿಗೆ ಹೊರಟುಬರುವುದಾಗಿಯೂ ಹೊರಡುವ ಮುನ್ನ ಮತ್ತೊಮ್ಮೆ ತಿಳಿಸುವು ದಾಗಿಯೂ ಮಾಯಾವತಿಗೆ ಅದಾಗಲೇ ತಂತಿ ಕಳಿಸಿದ್ದರು. ಆದರೆ ಮಾಯಾವತಿಯಲ್ಲಿ ಸಿದ್ಧತೆ ಗಳನ್ನು ಮಾಡುವುದಕ್ಕಾಗಿ, ಹೊರಡುವ ಎಂಟುದಿನ ಮುಂಚೆಯಾದರೂ ತಿಳಿಸುವಂತೆ ಮರು ತಂತಿ ಬಂದಿತು. ಆ ದಿನಗಳಲ್ಲಿ ಪ್ರಯಾಣ ಬಹಳ ಕಷ್ಟಕರವಾಗಿದ್ದು ಜನರನ್ನು ಬರಮಾಡಿಕೊಳ್ಳು ವುದಕ್ಕೂ ಇತರ ಸಿದ್ಧತೆಗಳನ್ನು ಮಾಡುವುದಕ್ಕೂ ವಿಶೇಷ ಶ್ರಮ ವಹಿಸಬೇಕಾಗಿತ್ತು. ಅಲ್ಲದೆ ಚಳಿಗಾಲವಾದ್ದರಿಂದ ಮತ್ತಷ್ಟು ತೊಂದರೆಗಳಿದ್ದುವು. ಆದರೆ ಸಿದ್ಧತೆಗಳನ್ನು ಮಾಡಲು ಎಷ್ಟು ಸಮಯ ಬೇಕೆಂಬುದನ್ನು ತಿಳಿದಿದ್ದ ಸ್ವಾಮೀಜಿಯವರು ಕೆಲಸಗಳನ್ನು ಚುರುಕುಗೊಳಿಸುವುದಕ್ಕಾಗಿ ತಾವು ಡಿಸೆಂಬರ್ ೨೭ರಂದು ಕಲ್ಕತ್ತ ಬಿಟ್ಟು ೨೯ರಂದೇ ಕಥಗೋಡಂ ತಲುಪುವುದಾಗಿ ತಂತಿ ಕಳಿಸಿದರು. ಈ ತಂತಿ ೨೫ರಂದು ಮಾಯಾವತಿಯನ್ನು ಮುಟ್ಟಿತು. ಮಾಯಾವತಿಯಿಂದ ತಮ್ಮನ್ನು ಎದುರ್ಗೊಳ್ಳಲು ಯಾರಾದರೂ ಬರುವುದು ತಡವಾಗಬಹುದು ಎಂದು ಆಲೋಚಿಸಿ ಸ್ವಾಮೀಜಿಯವರು ಆಲ್ಮೋರದ ತಮ್ಮ ಹಳೆಯ ಸ್ನೇಹಿತರಾದ ಲಾಲಾ ಬದರೀ ಸಾಹರಿಗೂ ತಂತಿ ಕಳಿಸಿದರು. ತಕ್ಷಣ ಸಾಹರು ತಮ್ಮ ಸೋದರ ಗೋವಿಂದನನ್ನು ಕಥಗೋಡಂಗೆ ಕಳಿಸಿ ಕೊಟ್ಟರು. ಇತ್ತ ಮಾಯಾವತಿಗೆ ತಂತಿ ತಲುಪುತ್ತಿದ್ದಂತೆಯೇ ಅಲ್ಲಿದ್ದ ಸ್ವಾಮಿ ವಿರಜಾನಂದರು ತ್ವರೆಯಿಂದ ಹೊರಟು, ಹಳ್ಳಿಯಿಂದ ಹಳ್ಳಿಗೆ ಹೋಗಿ ಕೂಲಿಗಳನ್ನೂ ದಂಡಿ ಹೊರುವವರನ್ನೂ ದುಬಾರಿ ಕೂಲಿಗೆ ಒಪ್ಪಿಸಿ, ಬಲವಂತವಾಗಿ ಹೊರಡಿಸಿಕೊಂಡು ನಡೆದರು. ಆ ಪರ್ವತ ಪ್ರದೇಶದಲ್ಲಿ ೬೫ ಮೈಲಿಗಳನ್ನು ಎರಡೇ ದಿನದಲ್ಲಿ ಕ್ರಮಿಸಿ (ಸಾಮಾನ್ಯವಾಗಿ ಇದಕ್ಕೆ ಮೂರು ದಿನವಾದರೂ ಬೇಕು) ಡಿಸೆಂಬರ್ ೨೮ರ ಮಧ್ಯರಾತ್ರಿ ಕಥಗೋಡಂ ತಲುಪಿದರು.

ಸ್ವಾಮೀಜಿಯವರ ಟ್ರೈನು ಮರುದಿನ ಬೆಳಿಗ್ಗೆ ಐದು ಗಂಟೆಗೆ ಬಂದಿತು. ಜೊತೆಯಲ್ಲಿ ಶಿವಾನಂದರೂ ಸದಾನಂದರೂ ಇದ್ದರು. ವಿರಜಾನಂದರ ಸಾಹಸವನ್ನು ತಿಳಿದ ಸ್ವಾಮೀಜಿ, “ನನ್ನ ಶಿಷ್ಯ ಅಂದಮೇಲೆ ಕೇಳಬೇಕೆ!” ಎಂದರು. ಲಾಲಾ ಗೋವಿಂದ ಸಾಹರು ಮೊದಲು ಆಲ್ಮೋರಕ್ಕೆ ಬಂದುಹೋಗುವಂತೆ ಸ್ವಾಮೀಜಿಯವರನ್ನು ಒತ್ತಾಯಿಸಿದರು. ಆದರೆ ವಿರಜಾನಂದರು ಅದ ಕ್ಕೊಪ್ಪದೆ, ನೇರವಾಗಿ ಮಾಯಾವತಿಗೆ ಹೊರಡುವಂತೆ ಅವರ ಮನವೊಲಿಸಿದರು. ಅಂದು ಕಥಗೋಡಂನಲ್ಲೇ ವಿಶ್ರಮಿಸಿ ಮರುದಿನ ಮಾಯಾವತಿಗೆ ಹೊರಡುವುದೆಂದು ತೀರ್ಮಾನ ವಾಯಿತು. ಮೊದಲೇ ಇದು ಕಷ್ಟಕರವಾದ ಪರ್ವತ ಪ್ರಯಾಣ. ಜೊತೆಗೆ ಚಳಿಗಾಲದ ನಡುಭಾಗ. ಸಾಲದ್ದಕ್ಕೆ ಆ ವರ್ಷ ಹಿಂದೆಂದಿಗಿಂತ ಹೆಚ್ಚು ಕೊರೆಯುವ ಚಳಿ. ದಾರಿಯುದ್ದಕ್ಕೂ ದಟ್ಟವಾಗಿ ಹಿಮ ಬಿದ್ದಿತ್ತು. ಆದರೆ ಸ್ವಾಮೀಜಿ ಯಾವುದನ್ನು ಕುರಿತೂ ಆಲೋಚಿಸದೆ, ಶ್ರೀಮತಿ ಸೇವಿಯರ ರನ್ನು ಕಾಣಬೇಕೆಂಬ ಒಂದೇ ಉದ್ದೇಶದಿಂದ ಹೊರಟುಬಿಟ್ಟಿದ್ದರು. ಈಗ ಮುಂದಿನ ಪ್ರಯಾಣದ ಸಂಪೂರ್ಣ ಜವಾಬ್ದಾರಿ ವಿರಜಾನಂದರದ್ದಾಗಿತ್ತು. ಜೊತೆಗೆ ಸ್ವಾಮೀಜಿಯವರಿಗೆ ಎಲ್ಲ ವೈಯ ಕ್ತಿಕ ಸೇವೆ ಸಲ್ಲಿಸುವ ಸೌಭಾಗ್ಯವೂ ಅವರದಾಗಿತ್ತು. ಹಿಂದೆಯೇ ಅವರು ತಮ್ಮ ಸೇವೆಯಿಂದ ಸ್ವಾಮೀಜಿಯವರ ಹೃದಯವನ್ನು ಗೆದ್ದುಕೊಂಡಿದ್ದರು. ಅವರ ಬಗ್ಗೆ ಸ್ವಾಮೀಜಿಯವರಿಗಿದ್ದ ವಾತ್ಸಲ್ಯ ಅಪಾರ. ಪ್ರಯಾಣಕಾಲದಲ್ಲಿ ಸ್ವಾಮೀಜಿ ಒಂದು ಹಸುಳೆಯಂತೆ ಉಲ್ಲಾಸಭರಿತರಾಗಿ ದ್ದರು. ಮೊದಲ ದಿನ ಸುಮಾರು ಹದಿನೇಳು ಮೈಲಿ ಪ್ರಯಾಣ ಮಾಡಿ ಸಂಜೆಯ ವೇಳೆಗೆ ಧಾರಿ ಎಂಬ ಸ್ಥಳವನ್ನು ತಲುಪಿದರು. ಆ ರಾತ್ರಿಯನ್ನು ಅಲ್ಲಿನ ಒಂದು ಪುಟ್ಟ ಛತ್ರದಲ್ಲಿ ಕಳೆದರು.

ಮರುದಿನ ಮುಂಜಾನೆಯಿಂದಲೇ ಮಳೆ ಹಿಡಿದುಕೊಂಡಿತ್ತು. ಜೊತೆಗೆ ಹಿಮಬೀಳುವ ಸೂಚನೆಯೂ ಇತ್ತು. ಆದರೂ ಸ್ವಲ್ಪ ತಡವಾಗಿಯಾದರೂ ಹೊರಟೇಬಿಟ್ಟರು. ಅಂದು ಹದಿ ನೈದು ಮೈಲಿ ಪ್ರಯಾಣ ಮಾಡಬೇಕಿತ್ತು. ವಿರಜಾನಂದರಿಗೆ ತುಂಬ ಆತಂಕ. ಬೇಗಬೇಗ ನಡೆಯ ದಿದ್ದರೆ ಸಂಜೆಗೆ ಮೊದಲೇ ಭಾರೀ ಹಿಮಪಾತಕ್ಕೆ ಸಿಲುಕಿಕೊಳ್ಳುವ ಸಂಭವವಿತ್ತು. ಆದರೆ ಎಲ್ಲಕ್ಕಿಂತ ಮುಖ್ಯವಾಗಿ ಅವರಿಗೆ ಸ್ವಾಮೀಜಿಯವರ ಆರೋಗ್ಯದ ಚಿಂತೆ. ಸ್ವಾಮೀಜಿಯವರ ದೇಹಸ್ಥಿತಿ ಮೊದಲೇ ಸೂಕ್ಷ್ಮ. ಆದ್ದರಿಂದ ಎಷ್ಟು ಎಚ್ಚರವಹಿಸಿದರೂ ಸಾಲದಾಗಿತ್ತು. ಆದರೆ ಎರಡು ಮೈಲಿ ನಡೆಯುವಷ್ಟರಲ್ಲಿ ಮತ್ತೆ ಮಳೆ ಪ್ರಾರಂಭವಾಯಿತು. ಬಳಿಕ ಹಿಮಪಾತವೂ ಶುರುವಾಯಿತು. ನೆಲವೆಲ್ಲ ಮಂಜಿನಿಂದ ಆವೃತವಾಗಿ ಮೇನೆಯನ್ನು ಹೊತ್ತವರು ಇಳಿಜಾರು ಗಳಲ್ಲಿ ಎಷ್ಟೋ ಸಲ ಜಾರಿದರು. ಆದರೂ ಸ್ವಾಮೀಜಿ ಸ್ವಲ್ಪವೂ ವಿಚಲಿತರಾಗದೆ ಎಲ್ಲವೂ ಒಂದು ತಮಾಷೆಯೋ ಎಂಬಂತಿದ್ದರು. ಅಲ್ಲದೆ ಮೇನೆ ಹೊತ್ತವರೊಂದಿಗೆ ತಮಾಷೆಯಾಗಿ ಮಾತನಾಡುತ್ತ ಖುಶಿ ಪಡಿಸುತ್ತಿದ್ದರು.

ಹೀಗೆ ಈ ಕೊರೆಯುವ ಚಳಿ ಹಾಗೂ ಹಿಮದಲ್ಲೇ ಸಾಗುತ್ತ ಅಪರಾಹ್ನ ಮೂರು ಗಂಟೆಗೆ ಪೌರ್ಹಾಪಾನಿ ಎಂಬಲ್ಲಿ ಊಟಕ್ಕಾಗಿ ನಿಂತರು. ಅಲ್ಲೊಂದು ‘ಅಂಗಡಿ’ ಇತ್ತು. ಅಲ್ಲಿ ಅಡಿಗೆ ಮಾಡಿಕೊಳ್ಳುವುದಕ್ಕೂ ಅವಕಾಶವಿತ್ತು. ಆಯಾಸಗೊಂಡಿದ್ದ ಮೇನೆಯವರು ಸ್ವಲ್ಪ ಚಹಾ ಕುಡಿಯಲು ಅನುಮತಿ ಕೇಳಿದರು. ಚಹಾ ಕುಡಿದು ಮತ್ತಷ್ಟು ಸ್ಫೂರ್ತಿಯಿಂದ ತಾವು ರಾತ್ರಿ ಯೊಳಗಾಗಿ ಗುರಿಯನ್ನು ತಲುಪುವುದಾಗಿ ಭರವಸೆ ಕೊಟ್ಟರು. ಆಗ ಸ್ವಾಮೀಜಿಯವರು ಅವರ ಮೇಲ್ಖರ್ಚನ್ನು ತಾವೇ ಕೊಡುವುದಾಗಿ ಹೇಳಿದರು. ಆದರೆ ಸ್ವಲ್ಪ ಹೊತ್ತಿನಲ್ಲಿ ವಿರಜಾನಂದರು ಬಂದು ನೋಡುತ್ತಾರೆ–ಕೂಲಿಗಳೆಲ್ಲ ಹುಕ್ಕಾ ಹೊತ್ತಿಸಿಕೊಂಡು ಆರಾಮವಾಗಿ ಒರಗಿಕೊಂಡಿ ದ್ದಾರೆ!ಅಗ್ಗಿಷ್ಟಿಕೆಯನ್ನು ಊದುತ್ತ ಅದರ ಪಕ್ಕದಲ್ಲಿ ಸಮಯದ ಪರಿವೆಯೇ ಇಲ್ಲದೆ ಆನಂದದಿಂ ದಿದ್ದುಬಿಟ್ಟಿದ್ದಾರೆ! ವಿರಜಾನಂದರಿಗೆ ಆತಂಕ ಹೆಚ್ಚಾಯಿತು. ಜೊತೆಗೆ ಎಲ್ಲೆಲ್ಲಿಯೂ ಅಗ್ಗಿ ಷ್ಟಿಕೆಯ ಹೊಗೆ. ಮೊದಲೇ ಸಣ್ಣ ಜಾಗ. ಅದರಲ್ಲೇ ಸಮಾರಾಧನೆ! ನೋಡನೋಡುತ್ತಿ ದ್ದಂತೆಯೇ ಗಂಟೆ ಐದಾಯಿತು. ಕತ್ತಲು ಆವರಿಸತೊಡಗಿತು. ಇನ್ನು ಪ್ರಯಾಣ ಸಾಧ್ಯವಿಲ್ಲ ಎಂಬುದು ಎಲ್ಲರಿಗೂ ಸ್ಪಷ್ಟವಾಯಿತು. ಮರುದಿನದವರೆಗೂ ಆ ಅಂಗಡಿ ಎಂಬ ಗೂಡೇ ಅವರ ತಂಗುದಾಣವಾಯಿತು.

ಈಗ ಸ್ವಾಮೀಜಿ ಸಣ್ಣ ಬಾಲಕನಂತೆ ಸಿಟ್ಟಿಗೆದ್ದರು. ಅಲ್ಲಿದ್ದ ಎಲ್ಲರ ಮೇಲೂ ಹರಿಹಾಯ್ದರು. ಹಿಮಪಾತದ ಅಪಾಯವಿದ್ದಾಗ ಪ್ರಯಾಣ ಹೊರಡಬಾರದು ಎಂದು ಯಾರೂ ಹೇಳಲಿಲ್ಲವೇಕೆ? ಅಷ್ಟೊಂದು ಜನ ಇದ್ದರಲ್ಲ, ಒಬ್ಬರಿಗೂ ಅದು ಹೊಳೆಯಲಿಲ್ಲವೆ? ಹೋಗಲಿ, ಗೋವಿಂದ ಸಾಹ ಕರೆದಾಗ ಮೊದಲು ಆಲ್ಮೋರಕ್ಕೆ ಹೋಗುವುದನ್ನು ಇತರರು ತಪ್ಪಿಸಿದ್ದೇಕೆ? “ಏಕೆ?” ಎಂದು ಗರ್ಜಿಸಿದರು ಸ್ವಾಮೀಜಿ. ಎಲ್ಲರ ಉಸಿರೂ ಅಡಗಿಹೋಯಿತು. ಸ್ವಾಮೀಜಿ ಸ್ವಲ್ಪ ಹೊತ್ತು ರೇಗಾಡಿ ಸುಮ್ಮನಾದರು. ಆದರೆ ಸಿಟ್ಟಿನ್ನೂ ಹೋಗಿರಲಿಲ್ಲ.

ಸ್ವಲ್ಪ ಹೊತ್ತು ಎಲ್ಲರೂ ಮೌನ. ಈಗ ವಿರಜಾನಂದರು ಮೆಲ್ಲನೆ ಮಾತನಾಡತೊಡಗಿದರು. “ಸ್ವಾಮೀಜಿ, ಕೂಲಿಗಳನ್ನು ಸಡಿಲ ಬಿಟ್ಟು, ಕಾಲ ಕಳೆಯಲು ಅವಕಾಶ ಕೊಟ್ಟಿದ್ದೆ ತಪ್ಪಾಯಿತು. ಸ್ವಲ್ಪ ಸಮಯ ಸಿಕ್ಕಿತೋ ಇಲ್ಲವೋ, ಇವರೆಲ್ಲ ಹುಕ್ಕಾ ಹಚ್ಚಿಸಿಕೊಂಡು ಒರಗಿಬಿಟ್ಟರು. ನಾವು ಸಮಯಕ್ಕೆ ಸರಿಯಾಗಿ ಹೊರಟಿದ್ದರೆ ರಾತ್ರಿಯೊಳಗಾಗಿ ಮೌರ್ನಾಲವನ್ನು ತಲುಪಬಹುದಾಗಿತ್ತು” ಎಂದು ಹೇಳಿ, ತಪ್ಪು ಸ್ವಾಮೀಜಿಯವರದೇ ಎಂಬುದನ್ನು ನೆನಪಿಸಿದರು. ಈಗ, ತಪ್ಪು ಮಾಡಿ ಸಿಕ್ಕಿಹಾಕಿಕೊಂಡ ಮಗುವಿನಂತೆ ಸ್ವಾಮೀಜಿ ಮರುಮಾತನಾಡದೆ ಕುಳಿತು ಅವರು ಹೇಳುವುದನ್ನೆಲ್ಲ ಕೇಳಿದರು. ಬಳಿಕ ನುಡಿದರು, “ಹೋಗಲಿ, ನಾನು ಹೇಳಿದ್ದನ್ನೆಲ್ಲ ಅಷ್ಟಕ್ಕೇ ಬಿಟ್ಟುಬಿಡು. ನಾವು ಸಂದರ್ಭಕ್ಕನುಸಾರವಾಗಿ ಹೊಂದಿಕೊಳ್ಳಬೇಕಾಗುತ್ತದೆ. ಈಗ ನೋಡು ನನ್ನ ಬೆನ್ನು ಬಹಳ ಕೊರೆಯುತ್ತಿದೆ. ಸ್ವಲ್ಪ ತಿಕ್ಕುತ್ತೀಯಾ?” ಸ್ವಾಮೀಜಿ ಮತ್ತೆ ಮೊದಲಿನಂತಾದರು. ಏನೋ ಒಂದು ತಮಾಷೆ ನಡೆದುಹೋಯಿತೋ ಎಂಬಂತೆ ನಗುತ್ತ, ತಮ್ಮ ಪ್ರಿಯ ಶಿಷ್ಯನೊಂದಿಗೆ ಆನಂದದಿಂದ ಮಾತನಾಡುತ್ತ ಕುಳಿತುಬಿಟ್ಟರು.

ಅದು ಹತ್ತೊಂಬತ್ತನೇ ಶತಮಾನದ ಕಟ್ಟಕಡೆಯ ದಿನ–ಡಿಸೆಂಬರ್ ೩೧, ೧೯ಂಂ. ಆ ತಂಡದವರಿಗೆಲ್ಲ ಅದು ಸುಲಭವಾಗಿ ಮರೆಯಲಾಗದಂತಹ ದಿನ–ಅನೇಕ ಕಾರಣಗಳಿಂದ. ಸ್ವಾಮೀಜಿಯವರೊಂದಿಗೆ ಪಯಣಿಸುವ, ಅವರೊಂದಿಗೆ ಮಾತನಾಡುವ, ಅವರ ವಿವಿಧ ಭಾವ ಗಳನ್ನು ಕಾಣುವ ಸೌಭಾಗ್ಯ. ಜೊತೆಗೆ ಒಂದಾದ ಮೇಲೊಂದು ನಡೆದ ವಿಚಿತ್ರ ಘಟನೆಗಳು. ಈ ‘ಪಂಚತಾರಾ’ ಗುಡಿಸಿಲಿನಲ್ಲಿ ರಾತ್ರಿಯನ್ನು ಕಳೆಯಬೇಕಲ್ಲ ಎಂದು ಚಿಂತಿಸುವಷ್ಟರಲ್ಲೇ ಮತ್ತೊಂದು ಆತಂಕ ಎದುರಾಯಿತು. ತಂಡದವರೆಲ್ಲ ಒಬ್ಬೊಬ್ಬರಾಗಿ, ಇಬ್ಬಿಬ್ಬರಾಗಿ ಬಂದು ಪೌರ್ಹಾಪಾನಿಯನ್ನು ತಲುಪಿದರು. ಆದರೆ ಗೋವಿಂದ ಸಾಹ ಹಾಗೂ ಸ್ವಾಮಿ ಸದಾನಂರು ಮಾತ್ರ ಬರಲೇ ಇಲ್ಲ. ಅಂಗಡಿಯವನನ್ನು ವಿಚಾರಿಸಿ ನೋಡಿದಾಗ ತಿಳಿಯಿತು–ಅವರಿಬ್ಬರೂ ಇತರರಿ ಗಿಂತ ಮುಂಚೆಯೇ ಪೌರ್ಹಾಪಾನಿಯನ್ನು ಹಾದು ಮುಂದೆ ಸಾಗಿದ್ದರು ಎಂದು. ಇತರರೂ ಕೂಡ ಬಂದುಬಿಡುತ್ತಾರೆಂದು ನಂಬಿ ಅವರು ಸೀದಾ ಹೊರಟುಬಿಟ್ಟಿದ್ದರು. ಆದರೆ ಈಗ ಸ್ವಾಮೀಜಿ ಅವರ ಬಗ್ಗೆ ಆತಂಕಗೊಂಡರು. ಇಷ್ಟು ಹೊತ್ತಿಗೆ ಅವರು ತಮ್ಮ ಮುಂದಿನ ಗುರಿಯಾದ ಮೌರ್ನಾಲವನ್ನು ತಲುಪಿರುತ್ತಾರೆ ಎಂದು ಇತರರು ಸಮಾಧಾನ ಹೇಳಿದರೂ ಸ್ವಾಮೀಜಿಯವರ ಆತಂಕ ನಿವಾರಣೆಯಾಗಲಿಲ್ಲ. ಯಾರಾದರೂ ಮೌರ್ನಾಲದವರೆಗೂ ಹೋಗಿ ನೋಡಿಕೊಂಡು ಬರಲು ಸಾಧ್ಯವೇ ಎಂದು ವಿಚಾರಿಸಿದರು. ಆಗ, ಆ ಅಂಗಡಿಯವನ ಸೋದರಳಿಯ ತಾನು ಓಡಿಹೋಗಿ ನೋಡಿಕೊಂಡು ಬರಲು ಸಿದ್ಧನಾದ. ಇದಕ್ಕೆ ಭಕ್ಷೀಸು ಎರಡು ರೂಪಾಯಿ ಎಂದು ನಿಶ್ಚಯವಾಯಿತು. ಆ ಕಾಲಕ್ಕೆ ಅದು ಬಹಳ ಜಾಸ್ತಿ. ಅಂತೂ, ಆತ ಹೋಗಿ ಬಂದು, ಇಬ್ಬರೂ ಮೌರ್ನಾಲವನ್ನು ಕ್ಷೇಮವಾಗಿ ತಲುಪಿದ್ದಾರೆ ಎನ್ನುವ ಸುದ್ದಿ ತಂದಮೇಲೆಯೇ ಸ್ವಾಮೀಜಿ ಯವರಿಗೆ ಸಮಾಧಾನವಾದದ್ದು.

ಇದರ ಜೊತೆಗೆ ಇಲ್ಲಿ ಮತ್ತೊಂದು ಅವಗಢ ಸಂಭವಿಸಿತ್ತು. ಸ್ವಾಮಿ ಶಿವಾನಂದರು ಪೌರ್ಹಾಪಾನಿಗೆ ಕುದುರೆಯ ಮೇಲೆ ಬಂದಿಳಿದರು; ಜೊತೆಯಲ್ಲಿ ಅದರ ಸವಾರನೂ ಇದ್ದ. ಈ ಕುದುರೆ ಹಿಂದೆ ಹಿಮವನ್ನೇ ಕಂಡಿರಲಿಲ್ಲವೋ ಏನೋ!–ಶಿವಾನಂದರು ಇಳಿದರೋ ಇಲ್ಲವೋ, ಅದು ವಾಪಸ್ಸು ತಿರುಗಿ ತನ್ನ ಯಜಮಾನನ ಸಮೇತ ವಾಯುವೇಗದಲ್ಲಿ ಓಡಿ ಕಣ್ಮರೆಯಾಯಿತು. ಆಮೇಲೆ ಆ ಕುದುರೆಯ ಸುಳಿವೂ ಇಲ್ಲ, ಅದರ ಯಜಮಾನನ ಸುದ್ದಿಯೂ ಇಲ್ಲ...! ಇಲ್ಲಿಂದ ಮುಂದೆ, ತಂಡದಲ್ಲಿ ಎಲ್ಲರಿಗಿಂತ ಕಿರಿಯರಾದ ವಿರಜಾನಂದರು ತಮ್ಮ ಕುದುರೆ ಯನ್ನು ಎಲ್ಲರಿಗಿಂತ ಹಿರಿಯರಾದ ಶಿವಾನಂದರಿಗೆ ಬಿಟ್ಟುಕೊಟ್ಟು, ತಾವು ಕಾಲ್ನಡಿಗೆಯಲ್ಲಿ ಬರಬೇಕಾಯಿತು.

ತಾವೆಲ್ಲ ಆ ರಾತ್ರಿಯನ್ನು ಅಲ್ಲೇ ಕಳೆಯುವುದಾಗಿಯೂ ತಮಗೆ ಊಟದ ವ್ಯವಸ್ಥೆ ಮಾಡ ಬೇಕು ಎಂದೂ ಸ್ವಾಮೀಜಿಯವರು ಅಂಗಡಿಯ ಯಜಮಾನನಿಗೆ ವಿನಂತಿಸಿಕೊಂಡರು. ಕೈತುಂಬ ಭಕ್ಷೀಸು ಕೊಡುವ ಆಶ್ವಾಸನೆಯನ್ನೂ ನೀಡಿದರು. ಅವನು ಒಪ್ಪಿಕೊಂಡ. ಆದರೆ ಆ ರಾತ್ರಿಯ ಊಟಕ್ಕೆ ಎಲ್ಲರಿಗೂ ಸಿಕ್ಕಿದ್ದು ಅರ್ಧ ಅಂಗುಲ ದಪ್ಪಗಿದ್ದ, ಅರೆಬರೆ ಬೆಂದಿದ್ದ ಚಪಾತಿ, ಅದಕ್ಕೆ ತಕ್ಕಂತಹ ಆಲೂಗಡ್ಡೆ ಪಲ್ಯ. ಆ ಚಪಾತಿಗೆ ‘ಕುದುರೆ ಚಪಾತಿ’ಎಂದು ಹೆಸರು ಕೊಡಲಾಯಿತು. ಅದನ್ನೇ ತಿಂದು ಎಲ್ಲರೂ ಮುದುರಿಕೊಂಡರು. ನಿದ್ರೆಯ ಪ್ರಶ್ನೆಯೇ ಇರಲಿಲ್ಲ. ಹಿಮ-ಮಳೆನೀರು ಎರಡೂ ಛಾವಣಿಯ ಬಿರುಕುಗಳ ಮೂಲಕ ಕುಂಭಾಭಿಷೇಕ ಮಾಡುತ್ತಿದ್ದುವು. ಜೊತೆಗೆ ಕೋಣೆಯನ್ನು ಬೆಚ್ಚಗಿಡಲೆಂದು ಉರಿಸುತ್ತಿದ್ದ ಅಗ್ಗಿಷ್ಟಿಕೆಗೆ ಒದ್ದೆ ಸೌದೆಯೇ ಗತಿಯಾದ್ದರಿಂದ ಎಲ್ಲೆಲ್ಲೂ ಉಸಿರುಗಟ್ಟಿಸುವ ಹೊಗೆ. ಇದೆಲ್ಲ ಸಾಲದೆಂಬಂತೆ ಆ ಅಂಗಡಿಯ ಯಜಮಾನ ತನ್ನ ಸಂಬಂಧಿಕನೊಡನೆ ಅಲ್ಲಿನ ಪಹಾಡೀ ಭಾಷೆಯಲ್ಲಿ ರಾತ್ರಿಯೆಲ್ಲ ಗೊರಗೊರ ಎಂದು ಅದೇನೋ ವಟಗುಟ್ಟುತ್ತಲೇ ಇದ್ದ. ಈ ಗುಂಪಿನಲ್ಲಿ ಯಾರಿಗೂ ಆ ಭಾಷೆ ಬರುವುದಿಲ್ಲವೆಂದು ಅವನು ಭಾವಿಸಿದ್ದ. ಆದರೆ ಸ್ವಾಮೀಜಿಯವರಿಗೆ ಅದು ಸ್ವಲ್ಪಸ್ವಲ್ಪ ಬರುತ್ತಿತ್ತು. ಆ ಅಂಗಡಿಯವನು ಹೇಳುತ್ತಿದ್ದ, “ನಾನು ಇವರಿಗೆಲ್ಲ ಇಲ್ಲಿ ಮಲಗಿಕೊಳ್ಳಲು ಅವಕಾಶ ಕೊಡಲೇಬಾರದಾಗಿತ್ತು. ಇದರಿಂದ ನನಗೆ ವಿಪರೀತ ತೊಂದರೆಯಾಯಿತು. ಬೆಳಗಾಗಲಿ, ಇವರನ್ನು ಮೊದಲು ಇಲ್ಲಿಂದ ಅಟ್ಟುತ್ತೇನೆ ನೋಡು.” ಹೀಗೆಯೇ ಮನಸ್ಸಿಗೆ ತೃಪ್ತಿಯಾಗುವವರೆಗೂ ಬೈದ. ಅವನ ಮಾತನ್ನೆಲ್ಲ ಕೇಳುತ್ತ ಸ್ವಾಮೀಜಿ ಸುಮ್ಮನೆ ಮಲಗಿದ್ದರು. ಅವನ ವರ್ತನೆಯನ್ನು ಕಂಡು ಅವರಿಗೆ ಜುಗುಪ್ಸೆ ಹುಟ್ಟಿತು. ಆದರೂ ಮರುದಿನ ಹೊರಡುವಾಗ ಅವನಿಗೆ ಕೊಡುವುದಾಗಿ ಹೇಳಿದ್ದ ಭಕ್ಷೀಸನ್ನು ಪೂರ್ತಿಯಾಗಿ ಸಲ್ಲಿಸಿಯೇ ಹೊರಟರು.

ಮರುದಿನ ಬೆಳಿಗ್ಗೆ ಮುಂಚೆ ಹೊರಟು, ಒಂದು ಅಡಿ ದಪ್ಪಕ್ಕೆ ಬಿದ್ದಿದ್ದ ಮಂಜಿನ ದಾರಿಯಾಗಿ ನಡೆದು ಮೌರ್ನಾಲವನ್ನು ಸೇರಿದರು. ಇಲ್ಲಿ ಸದಾನಂದರೂ ಗೋವಿಂದಸಾಹರೂ ಅವರಿಗಾಗಿ ಎಲ್ಲ ಸಿದ್ಧತೆಗಳನ್ನು ಮಾಡಿದ್ದರು. ಇವರಿಬ್ಬರನ್ನೂ ಕಂಡು ಸ್ವಾಮೀಜಿ ಆನಂದಗೊಂಡು ಅವರೊಂದಿಗೆ ಮಾತನಾಡುತ್ತ ಕುಳಿತರು. ವಿರಜಾನಂದರ ಬಗ್ಗೆಯಂತೂ ಸ್ವಾಮೀಜಿ ಎಷ್ಟು ಸಂತುಷ್ಟರಾಗಿದ್ದರೆಂದರೆ, ಅವರೆಂದರು–“ಎಂತಹ ಉದ್ರೇಕಗೊಳಿಸುವ ಪರಿಸ್ಥಿತಿಯಲ್ಲೂ ಕಾಳೀಕೃಷ್ಣ ತೋರಿಸುವ ಶಾಂತಚಿತ್ತತೆ, ಸಂಯಮ ಮತ್ತು ವಿವೇಚನಾ ಸಾಮರ್ಥ್ಯಗಳು ಎಷ್ಟು ಅದ್ಭುತವೆಂದರೆ, ಅವನನ್ನು ಎಷ್ಟು ಹೊಗಳಿದರೂ ಕಡಿಮೆಯೇ!”

ಮೌರ್ನಾಲದಲ್ಲಿ ಒಂದು ದಿನ ವಿರಮಿಸಿ ಮತ್ತೆ ಪ್ರಯಾಣವನ್ನು ಮುಂದುವರಿಸಿದರು. ಕಡೆ ಯಲ್ಲಿ ಸ್ವಲ್ಪ ದೂರ ಸ್ವಾಮೀಜಿ ನಡೆದುಕೊಂಡೇ ಬಂದರು. ಆದರೆ ಮಂಜಿನಿಂದಾವೃತವಾದ ನೆಲದಲ್ಲಿ ಹತ್ತುತ್ತ ಇಳಿಯುತ್ತ ನಡೆಯುವುದು ಅವರಿಗೆ ತೀರ ಪ್ರಯಾಸಕರವಾಗಿತ್ತು. ಒಂದು ಕೈಯಲ್ಲಿ ಊರುಗೋಲು ಹಿಡಿದು ಮತ್ತೊಂದು ಕೈಯಲ್ಲಿ ವಿರಜಾನಂದರ ಭುಜವನ್ನು ಆಸರೆ ಯಾಗಿ ಹಿಡಿದು ಮುನ್ನಡೆದರು. “ನಾನೀಗ ಎಷ್ಟು ಮುದುಕನಾಗಿದ್ದೇನೆ ನೋಡು!ಎಷ್ಟು ದುರ್ಬಲನಾಗಿದ್ದೇನೆ ನೋಡು! ಇಷ್ಟು ಸ್ವಲ್ಪ ದೂರ ನಡೆಯುವುದಕ್ಕೆ ಇಷ್ಟು ಕಷ್ಟವಾಗುತ್ತಿದೆ. ಹಿಂದೆಲ್ಲ ನಾನು ಇದೇ ಪರ್ವತ ಪ್ರದೇಶಗಳಲ್ಲಿ ದಿನಕ್ಕೆ ಇಪ್ಪತ್ತು-ಇಪ್ಪತ್ತೈದು ಮೈಲಿ ಲೀಲಾಜಾಲ ವಾಗಿ ನಡೆಯುತ್ತಿದ್ದೆ” ಎಂದರು ಸ್ವಾಮೀಜಿ. ವಿರಜಾನಂದರು ಖೇದಗೊಂಡರಾದರೂ ಮಾತ ನಾಡಲಿಲ್ಲ. ಆದರೆ ತಕ್ಷಣ ಸ್ವಾಮೀಜಿ, “ನೋಡು ಮಗು, ನಾನೀಗ ಅಂತ್ಯವನ್ನು ಸಮೀಪಿಸು ತ್ತಿದ್ದೇನೆ” ಎಂದಾಗ ವಿರಜಾನಂದರ ಮನಸ್ಸಿಗೆ ತುಂಬ ಯಾತನೆಯಾಯಿತು. ಸ್ವಾಮೀಜಿಯವರ ದೇಹಸ್ಥಿತಿ ನಿಜಕ್ಕೂ ಅಪಾಯದ ಅಂಚಿನಲ್ಲೇ ಇತ್ತು. ಯಾವಾಗ ಬೇಕಾದರೂ ಅಪಾಯದ ಗೆರೆಯನ್ನವರು ದಾಟಬಹುದಾಗಿತ್ತು.

ಜನವರಿ ಮೂರರಂದು ಸ್ವಾಮೀಜಿ ತಮ್ಮ ಪರಿವಾರಸಮೇತರಾಗಿ ಮಾಯಾವತಿಯನ್ನು ಮುಟ್ಟಿದರು. ಆಶ್ರಮದ ಕಟ್ಟಡ ಕಣ್ಣಿಗೆ ಬೀಳುತ್ತಿದ್ದಂತೆಯೇ ಅವರು ತುಂಬ ಸಂತಸಗೊಂಡರು. ಬೇಗ ಅಲ್ಲಿಗೆ ಹೋಗಿ ತಲುಪಬೇಕೆಂಬ ಉತ್ಸುಕತೆಯಿಂದ, ಒಂದು ಕುದುರೆಯ ಬೆನ್ನೇರಿ ಅದನ್ನು ಪೂರ್ಣವೇಗದಿಂದ ಓಡಿಸಿದರು. ಸ್ವಾಮೀಜಿಯವರ ಆಗಮನದ ಈ ವಿಶೇಷ ಸಂದರ್ಭಕ್ಕಾಗಿ ಆಶ್ರಮವನ್ನು ಮತ್ತಷ್ಟು ಸುಂದರವಾಗಿ ಸಿಂಗರಿಸಲಾಗಿತ್ತು. ದೀರ್ಘಕಾಲದ ಬಳಿಕ ಸ್ವಾಮೀಜಿ ಯವರನ್ನು ಕಂಡು ಅವರ ಶಿಷ್ಯರಿಗಾದ ಆನಂದ ಅಪರಿಮಿತ. ತಮ್ಮ ತೀವ್ರ ಅನಾರೋಗ್ಯದ ಸ್ಥಿತಿಯಲ್ಲೂ ಸ್ವಾಮೀಜಿ ಇಷ್ಟು ಕಷ್ಟಕರವಾದ ಪ್ರಯಾಣವನ್ನು ಕೈಗೊಂಡು ಕ್ಷೇಮವಾಗಿ ತಲುಪಿದರಲ್ಲ ಎಂದು ದುಗುಡದ ಎಷ್ಟೋ ಸಮಾಧಾನ. ಅಲ್ಲದೆ ಕ್ಯಾಪ್ಟನ್ ಸೇವಿಯರ್ ತೀರಿಹೋದ ಆ ದುಗುಡದ ಸಮಯದಲ್ಲಿ ಸ್ವಾಮೀಜಿಯವರ ಆಗಮನ ಅವರಲ್ಲೆಲ್ಲ ಒಂದು ನವೋತ್ಸಾಹವನ್ನು ಮೂಡಿಸಿತು.

ಇಲ್ಲಿ ಸ್ವಾಮೀಜಿ ಹದಿನೈದು ದಿನ ಇದ್ದರು. ಆದರೆ ಈ ಅವಧಿಯಲ್ಲಿ ಅವರು ಹೆಚ್ಚಿನ ವೇಳೆಯನ್ನೆಲ್ಲ ಆಶ್ರಮದ ಕಟ್ಟಡದೊಳಗೇ ಕಳೆಯಬೇಕಾಯಿತು. ಕೊರೆಯುವ ಚಳಿ, ಹಿಮಪಾತ, ಮಳೆ ಗಾಳಿಗಳಿಂದಾಗಿ ಹಾಗೂ ವಿಶೇಷತಃ ಅವರ ದೇಹಸ್ಥಿತಿಯು ತುಂಬ ಸೂಕ್ಷ್ಮವಾದದ್ದ ರಿಂದಾಗಿ ಹೆಚ್ಚು ಓಡಾಡುವುದು ಅಸಾಧ್ಯವಾಗಿತ್ತು. ಜೊತೆಗೆ ಅವರ ಅನಾರೋಗ್ಯ ಮತ್ತಷ್ಟು ಹದಗೆಟ್ಟು ಹಲವಾರು ಸಲ ಆಸ್ತಮಾ ಹೊಡೆತಕ್ಕೆ ಸಿಕ್ಕಿಕೊಂಡರು. ಅದೃಷ್ಟವಶಾತ್ ಆ ಹೊಡೆತ ಗಳು ಅಷ್ಟೊಂದು ತೀವ್ರವಾಗಿರಲಿಲ್ಲ. ಇಷ್ಟೆಲ್ಲ ಆದರೂ ಅವರ ಉತ್ಸಾಹ ಮಾತ್ರ ಅದಮ್ಯ ವಾಗಿತ್ತು, ಕೋಣೆಯೊಳಗೇ ಲೇಖನಗಳನ್ನೋ ಪತ್ರಗಳನ್ನೋ ಬರೆಯುತ್ತ ಆಶ್ರಮವಾಸಿಗ ಳೊಂದಿಗೆ ಮಾತನಾಡುತ್ತ, ಇಲ್ಲವೆ ಶ್ರೀಮತಿ ಸೇವಿಯರರೊಂದಿಗೆ ಆಶ್ರಮದ ಭವಿಷ್ಯದ ಯೋಜನೆಗಳ ಕುರಿತಾಗಿ ಚರ್ಚಿಸುತ್ತ ಯಾವಾಗಲೂ ಚಟುವಟಿಕೆಯಿಂದ ಕೂಡಿರುತ್ತಿದ್ದರು.

ಈ ಸಮಯದಲ್ಲಿ ಮಾಯಾವತಿಯ ಅದ್ವೈತಾಶ್ರಮದಲ್ಲಿ ನೆಲಸಿದ್ದವರೆಂದರೆ ಸ್ವಾಮಿಗಳಾದ ಸ್ವರೂಪಾನಂದರು, ವಿರಜಾನಂದರು, ವಿಮಲಾನಂದರು, ಸಚ್ಚಿದಾನಂದರು ಹಾಗೂ ಅಮೆರಿಕ ದಿಂದ ಬಂದಿದ್ದ ಚಾರ್ಲ್ಸ್ ಜಾನ್ಸ್​ಟನ್ (ಬ್ರಹ್ಮಚಾರಿ ಅಮೃತಾನಂದ). ಇವರೆಲ್ಲ ಶ್ರೀಮತಿ ಸೇವಿಯರರ ಸಹಾಯ-ಸಹಕಾರದಿಂದ ಆ ವಿಶಾಲವಾದ ಆಶ್ರಮವನ್ನು ನಡೆಸಿಕೊಂಡು ಬರು ತ್ತಿದ್ದರು. ಕ್ಯಾಪ್ಟನ್ ಸೇವಿಯರ್ ಕೊನೆಯುಸಿರಿನವರೆಗೂ ಅತ್ಯಂತ ಸರಳ, ತಪೋಮಯ ಜೀವನ ನಡೆಸಿದ್ದರು. ಅವರ ಕಟ್ಟುನಿಟ್ಟಿನ ವಿರಕ್ತ ಜೀವನವನ್ನು ಕಂಡು ಜನ ಆಶ್ಚರ್ಯಪಡುತ್ತಿದ್ದರು. ಆಶ್ರಮದ ಏಳ್ಗೆಗಾಗಿ ಅವರು ಬಹುವಾಗಿ ಶ್ರಮಿಸಿದ್ದರು. ಕೆಲಕಾಲದಿಂದ ಅವರು ಮೂತ್ರಪಿಂಡದ ತೊಂದರೆಗೊಳಗಾಗಿದ್ದರು. ಆದರೆ ಅದನ್ನವರು ಲೆಕ್ಕಿಸಲೇ ಇಲ್ಲ. ಆಲ್ಮೋರಕ್ಕೆ ಹೋಗಿ ಚೆನ್ನಾಗಿ ಪರೀಕ್ಷೆ ಮಾಡಿಸಿಕೊಂಡು ಔಷಧಿ ತೆಗೆದುಕೊಳ್ಳುವಂತೆ ಎಲ್ಲರೂ ಹೇಳಿದರೂ ಅವರು ಕೇಳಲೇ ಇಲ್ಲ. “ನಾನು ನನ್ನ ಜೀವಮಾನದಲ್ಲೇ ಆರು ತಿಂಗಳಿಗಿಂತ ಹೆಚ್ಚು ಕಾಲ ಒಂದು ಕಡೆ ಉಳಿದುಕೊಂಡವನಲ್ಲ. ಈಗ ನಾನು ಮಾಯಾವತಿಯನ್ನು ಬಿಡುವುದಿಲ್ಲವೆಂದು ತೀರ್ಮಾನಿಸಿ ದ್ದೇನೆ” ಎಂದುಬಿಟ್ಟರು. ಭಗವನ್ನಾಮವನ್ನು ಜಪಿಸುತ್ತ ನೋವನ್ನು ಸಹಿಸಿಕೊಂಡರು. ಕಡೆಗೆ ೧೯ಂಂರ ಅಕ್ಟೋಬರ್ ೨೮ರಂದು ಕೊನೆಯುಸಿರೆಳೆದರು. ತಮ್ಮ ಗುರುವಿನ ಕಾರ್ಯಕ್ಕಾಗಿ ಜೀವವನ್ನೇ ಸಮರ್ಪಿಸಿದರು.

