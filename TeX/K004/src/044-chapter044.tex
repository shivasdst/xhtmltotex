
\chapter{ನೆರಳು ಅಡಗಿತು, ಜ್ಯೋತಿ ಬೆಳಗಿತು}

\noindent

ಮಹಾಜೀವನವೊಂದರ ಮುಕ್ತಾಯಸಮಾರಂಭದ ದಿನ ಅತಿಶಯ ವೇಗದಿಂದ ಸಮೀಪಿಸುತ್ತಿದೆ. ಸಮಸ್ತ ಲೋಕದಲ್ಲಿ ಪ್ರಚಂಡ ಪ್ರಕಾಶ ಬೀರಿದ ಧರ್ಮಸೂರ್ಯನೀಗ ಅಸ್ತಂಗತನಾಗುವ ಸನ್ನಾಹದಲ್ಲಿದ್ದಾನೆ. ಸ್ವಾಮೀಜಿಯವರು ಈ ಭುವಿಯಲ್ಲಿ ಕಳೆದ ಅವರ ಕೊನೆಯ ಎರಡು ತಿಂಗಳಲ್ಲಿ ಅವರ ಅಂತ್ಯವನ್ನು ಸೂಚಿಸುವ ಹಲವಾರು ಘಟನೆಗಳು ನಡೆದುಹೋದುವು. ಆದರೆ ಅವರ ಬಳಿಯಿದ್ದವರಾರೂ ಆ ಸಂದರ್ಭಗಳಲ್ಲಿ ಅದರ ಸುಳಿವನ್ನು ಅರ್ಥಮಾಡಿಕೊಳ್ಳಲಾರದೆ ಹೋದರು. ಪ್ರತಿಯೊಂದು ಪುಟ್ಟ ಘಟನೆಯೂ ಏನೋ ಒಂದನ್ನು ಸೂಚಿಸುವ ಸಂಕೇತ ದಂತಿತ್ತು. ಒಂದೊಂದು ಘಟನೆಯಲ್ಲೂ ಏನೋ ಒಂದು ವಿಶೇಷವಾದ ಅರ್ಥವಡಗಿತ್ತು.

ಈ ದಿನಗಳಲ್ಲಿ ಜಗತ್ತಿನ ಹಲವೆಡೆಗಳಲ್ಲಿದ್ದ ಅವರ ಅನೇಕ ಪಾಶ್ಚಾತ್ಯ ಶಿಷ್ಯೆಯರು ಆಗಮಿಸಿ ಅವರನ್ನು ಭೇಟಿಯಾಗಿ ಅವರ ಆಶೀರ್ವಾದವನ್ನು ಪಡೆದುಕೊಂಡು ಬೀಳ್ಗೊಂಡರು. ಆಗ ಸ್ವಾಮೀಜಿಯವರು ಕಾಯಿಲೆಯಿಂದಿರುವುದನ್ನು ಕಂಡರೂ, ಅವರ ಅಂತ್ಯ ಸನ್ನಿಹಿತವಾಗಿದೆ ಯೆಂಬುದಾಗಲಿ, ಇದೇ ತಮ್ಮ ಅಂತಿಮ ಸಂದರ್ಶನವಾಗಬಹುದು ಎಂಬುದಾಗಲಿ ಯಾರಿಗೂ ಹೊಳೆದಿರಲಿಲ್ಲ. ಅವರಲ್ಲಿ ಹೆಚ್ಚಿನವರು ಅಲ್ಲಿಗೆ ಬರಬೇಕಾದರೆ ಅರ್ಧ ಪ್ರಪಂಚವನ್ನೇ ಸುತ್ತಿ ಕೊಂಡು ಬರಬೇಕಾಗಿತ್ತು ಎಂಬುದನ್ನು ಗಮನಿಸಬೇಕು.

ದಿನಕಳೆದಂತೆಲ್ಲ ಸ್ವಾಮೀಜಿಯವರು ಮಠದ ಕಾರ್ಯಕಲಾಪಗಳನ್ನು ನಿರ್ದೇಶಿಸುವ ಹೊಣೆ ಗಾರಿಕೆಯಿಂದ ತಮ್ಮನ್ನು ಹಿಂದೆಗೆದುಕೊಳ್ಳತೊಡಗಿದರು. ಮಠದ ಇತರ ಹಿರಿಯ ಸಾಧುಗಳಿಗೆ ಅದನ್ನು ಹಸ್ತಾಂತರಿಸುತ್ತಿದ್ದರು. ಅಲ್ಲದೆ ಅವರು ಸ್ಪಷ್ಟವಾಗಿ ಹೇಳುತ್ತಿದ್ದರು–“ಗುರುವಾದವನು ಸದಾ ತನ್ನ ಶಿಷ್ಯರ ಬಳಿಯಲ್ಲೇ ಇರುವುದರ ಮೂಲಕ ಶಿಷ್ಯರ ನಾಶಕ್ಕೆ ಕಾರಣನಾಗುತ್ತಾನೆ. ಶಿಷ್ಯರಿಗೆ ಸಾಕಷ್ಟು ಶಿಕ್ಷಣ ನೀಡಿದ ಬಳಿಕ ಗುರುವು ಅವರಿಂದ ದೂರವಾಗುವುದು ಶ್ರೇಯಸ್ಕರ. ಶಿಷ್ಯರು ಸರಿಯಾಗಿ ವಿಕಾಸಗೊಳ್ಳಬೇಕಾದರೆ ಇದು ಆವಶ್ಯಕ.” ಸ್ವಾಮೀಜಿಯವರ ಇಂತಹ ಮಾತುಗಳನ್ನು ಕೇಳಿದಾಗ ಮತ್ತು ಈ ಬಗೆಯ ವರ್ತನೆಯನ್ನು ಕಂಡಾಗ ಅವರ ಸಹಸಂನ್ಯಾಸಿ ಗಳಿಗೆಲ್ಲ ಅನಿರ್ವಚನೀಯ ಸಂಕಟವಾಗುತ್ತಿತ್ತು. ಎದೆಯಲ್ಲಿ ಏನೋ ಅಳುಕುಂಟಾಗುತ್ತಿತ್ತು.

ಆದರೆ ಕಡೆಯ ಒಂದು ತಿಂಗಳ ಅವಧಿಯಲ್ಲಿ ಸ್ವಾಮೀಜಿಯವರ ದೇಹಾರೋಗ್ಯ ತುಂಬ ಸುಧಾರಿಸಿತು. ಇದು ಅವರ ಬಳಿಯಿದ್ದವರಿಗೆಲ್ಲ ಸಂತಸವನ್ನುಂಟುಮಾಡಿತು, ಹೊಸ ಭರವಸೆ ಯನ್ನು ತಂದಿತು.

ಜೂನ್ ೨೬ರಂದು ನಿವೇದಿತೆ ಮಾಯಾವತಿಯಿಂದ ಕಲ್ಕತ್ತಕ್ಕೆ ಮರಳಿದಳು. ಸ್ವಾಮಿ ಶಾರದಾ ನಂದರು ಅವಳನ್ನು ಬರಮಾಡಿಕೊಂಡರು. ಮತ್ತು ಸ್ವಾಮೀಜಿಯವರು ಅವಳಿಗೆ ಉಡುಗೊರೆ ಯಾಗಿ ಕಳಿಸಿಕೊಟ್ಟಿದ್ದ ಕೃಷ್ಣಾಜಿನವನ್ನು ಆಕೆಗೆ ನೀಡಿದರು. ತನ್ನ ಆಗಮನದಿಂದ ಸ್ವಾಮೀಜಿ ಯವರಿಗೆ ಸಂತೋಷವಾಗಿದೆ ಎಂಬ ಮಾತನ್ನು ಕೇಳಿಯೇ ಅವಳಿಗೆ ಹಿಡಿಸಲಾರದಷ್ಟು ಹಿಗ್ಗುಂ ಟಾಯಿತು.

ನಿವೇದಿತೆ ಸ್ವಾಮೀಜಿಯವರನ್ನು ನೋಡುವುದಕ್ಕಾಗಿ ಬೇಲೂರಿಗೆ ಹೊರಡಲು ಸಿದ್ಧಳಾಗು ತ್ತಿದ್ದಾಗ, ಅವರಿಂದ ಆಕೆಗೊಂದು ಚೀಟಿ ಬಂದಿತು–ತಾವೇ ಕಲ್ಕತ್ತಕ್ಕೆ ಕಾರ್ಯಾರ್ಥವಾಗಿ ಬರುತ್ತಿದ್ದೇವೆ. ಆಗ ಅವಳನ್ನು ನೋಡುತ್ತೇವೆ ಎಂದು. ಅದರಂತೆಯೇ ೨೮ರಂದು ಬೆಳಿಗ್ಗೆ ಒಂಬತ್ತು ಗಂಟೆಯ ವೇಳೆಗೆ ಸ್ವಾಮೀಜಿ ಅಲ್ಲಿಗೆ ಆಗಮಿಸಿದರು. ಅಲ್ಲಿನ ಅವಳ ನಿವಾಸವನ್ನು ಸೂಕ್ಷ್ಮವಾಗಿ ಪರಿಶೀಲಿಸಿದರು; ಆಕೆಯ ಕಾರ್ಯದ ಸಂಬಂಧವಾಗಿ ಏನೇನು ನಡೆಯಬೇಕೆಂಬು ದನ್ನೆಲ್ಲ ವಿವರಿಸಿದರು. ನಿವೇದಿತೆ ತನ್ನ ಶಾಲೆಯ ಉಪಯೋಗಕ್ಕಾಗಿ ಸೂಕ್ಷ್ಮದರ್ಶಕ ಯಂತ್ರ, ಮಾಯಾ ಲಾಂದ್ರ, ಕ್ಯಾಮೆರಾ ಮೊದಲಾದುವನ್ನೆಲ್ಲ ತಂದಿದ್ದಳು. ಅವುಗಳನ್ನು ಕಂಡು ಸ್ವಾಮೀಜಿ ತುಂಬ ಹರ್ಷಗೊಂಡರು. ಸೂಕ್ಷ್ಮದರ್ಶಕವನ್ನು ಮಠಕ್ಕೆ ತಂದು ತೋರಿಸುವಂತೆ ಆಕೆಗೆ ತಿಳಿಸಿ ದರು. ಬಳಿಕ ಅವರು ಆಕೆಯ ಯೋಜನೆಗಳ ಬಗ್ಗೆ ವಿಚಾರಿಸಿದರು. ನಿವೇದಿತೆಯ ಮನಸ್ಸಿನಲ್ಲಿ, ಹೆಣ್ಣುಮಕ್ಕಳಿಗಾಗಿ ಒಂದು ಶಾಲೆ ಮಾತ್ರವಲ್ಲದೆ ಒಂದು ವಿಶ್ವವಿದ್ಯಾನಿಲಯವನ್ನೆ ಪ್ರಾರಂಭಿಸುವ ಇಚ್ಛೆಯಿತ್ತು. ಅದನ್ನು ಅವಳು ವ್ಯಕ್ತಪಡಿಸಿದಾಗ ಸ್ವಾಮೀಜಿ ತಮ್ಮ ಅನುಮೋದನೆ ನೀಡಿದರು. ಅವರು ಅಲ್ಲಿಂದ ಹೊರಡುವ ವೇಳೆಗೆ ನಿವೇದಿತೆ “ಸ್ವಾಮೀಜಿ, ನೀವು ಮತ್ತೊಮ್ಮೆ ಇಲ್ಲಿಗೆ ಬಂದು ನನ್ನನ್ನು ಆಶೀರ್ವದಿಸುತ್ತೀರಲ್ಲವೆ?” ಎಂದು ಯಾಚನೆಯ ದನಿಯಲ್ಲಿ ಕೇಳಿದಳು. ಆಗ ಸ್ವಾಮೀಜಿ ಅನ್ಯಮನಸ್ಕರಾಗಿ ಮುಗುಳ್ನಕ್ಕರು. ಬಳಿಕ ಅವಳ ಭುಜದ ಮೇಲೆ ತಮ್ಮ ಕೈಗಳನ್ನಿರಿಸಿ ನುಡಿದರು, “ನನ್ನ ಆಶೀರ್ವಾದ ನಿನ್ನ ಮೇಲೆ ಸದಾ ಇದ್ದೇ ಇದೆಯಲ್ಲ!”

ಅಂದು ಸ್ವಾಮೀಜಿಯವರನ್ನು ಕಲ್ಕತ್ತದಲ್ಲಿ ಹಲವಾರು ಜನ ಆಹ್ವಾನಿಸಿದ್ದರು; ಅಂತೆಯೇ ಅವರ ಸೋದರಿಯೂ ಕೂಡ. ತಮ್ಮ ಸೋದರಿಯ ಮನೆಗೆ ಹೋದ ಸ್ವಾಮೀಜಿಯವರು ಅಲ್ಲೊಂದು ಮುಖ್ಯವಾದ ಕಾರ್ಯವನ್ನು ಮಾಡಿ ಮುಗಿಸಿದರು. ಅದು ಅವರ ಪಿತ್ರಾರ್ಜಿತ ಆಸ್ತಿಗೆ ಸಂಬಂಧಿಸಿದ ಬಹಳ ಕಾಲದ ಮೊಕದ್ದಮೆಯನ್ನು ಒಂದು ವ್ಯವಸ್ಥೆಗೆ ತಂದದ್ದು. ಈ ಕುರಿತಾಗಿ ಅವರು ಎರಡು ಪಂಗಡಗಳವರನ್ನೂ ಸೇರಿಸಿ ಒಂದು ಒಪ್ಪಂದಕ್ಕೆ ಬರುವಂತೆ ಮಾಡಿದರು. ಈ ಒಪ್ಪಂದವು ಎರಡು ಕಡೆಯವರಿಗೂ ತೃಪ್ತಿಯುಂಟುಮಾಡಿತು. ಸ್ವಾಮೀಜಿಯವರಿಗೂ ಇದು ಸಮಾಧಾನ ತಂದಿತು.

ಮರುದಿನ ಬೆಳಿಗ್ಗೆ ನಿವೇದಿತೆ ಬೇಲೂರು ಮಠಕ್ಕೆ ಬಂದು ಸ್ವಾಮೀಜಿಯವರನ್ನು ಅವರ ಕೋಣೆಯಲ್ಲಿ ಭೇಟಿಯಾದಳು. ಅವರು ಲೋಕದ ಸುಖದುಃಖಗಳನ್ನೆಲ್ಲ ಮೀರಿದ ತುರೀಯಾವಸ್ಥೆ ಯಲ್ಲಿದ್ದಂತೆ ತೋರುತ್ತಿತ್ತು. ಬಳಿಕ ಆಕೆ ಮೆಲ್ಲನೆ ಮಾತಿಗಾರಂಭಿಸಿದಳು. ತಾನು ಹಾಕಿಕೊಂಡಿದ್ದ ಯೋಜನೆಗಳನ್ನು ವಿವರಿಸಿದಳು. ಸಂಭಾಷಣೆ ಬಹಳ ಹೊತ್ತು ನಡೆಯಿತು. ಕಡೆಗೆ ಸ್ವಾಮೀಜಿ ನುಡಿದರು, “ನನ್ನನ್ನೊಂದು ಮಹಾಧ್ಯಾನಭಾವ ಆವರಿಸುತ್ತಿದೆ. ಇಂದು ನೀನು ಬಂದಿರದಿದ್ದರೆ ನಾನಿನ್ನೂ ಧ್ಯಾನ ಮಾಡುತ್ತ ದೇವಸ್ಥಾನದಲ್ಲೇ ಕುಳಿತಿರುತ್ತಿದ್ದೆ. ನನಗನಿಸುತ್ತದೆ–ನನ್ನ ಅಂತ್ಯ ಸಮೀಪಿಸುತ್ತಿದೆ ಎಂದು.” ಅವರು ಹೀಗೆನ್ನುತ್ತಿದ್ದಂತೆಯೇ ಹಲ್ಲಿಯೊಂದು ಲೊಚಗುಟ್ಟಿತು.

ಮೂರು ದಿನಗಳ ಬಳಿಕ (ಜುಲೈ ೨, ಬುಧವಾರ) ನಿವೇದಿತೆಗೆ ಮತ್ತೊಮ್ಮೆ ಸ್ವಾಮೀಜಿ ಯವರನ್ನು ಕಾಣಬೇಕೆಂಬ ತೀವ್ರ ಹಂಬಲವುಂಟಾಯಿತು. ಅಂದು ಏಕಾದಶಿ, ಆದ್ದರಿಂದ ಸ್ವಾಮೀಜಿ ಕಟ್ಟುನಿಟ್ಟಾಗಿ ಉಪವಾಸ ಮಾಡುತ್ತಾರೆ, ಅವರಿಗೆ ಆಯಾಸವಾಗಿರುತ್ತದೆ ಎಂದು ಅವಳಿಗೆ ಗೊತ್ತಿತ್ತು. ಆದರೂ ಅವರ ದರ್ಶನ ಪಡೆಯಬೇಕೆಂಬ ಇಚ್ಛೆಯನ್ನು ತಡೆದುಕೊಳ್ಳ ಲಾರದೆ ಹೋದಳು. ನಿವೇದಿತೆ ಬಂದಿದ್ದಾಳೆಂಬ ವಿಷಯ ಕೇಳಿದೊಡನೆ ಸ್ವಾಮೀಜಿ ಆಕೆಯನ್ನು ತಮ್ಮ ಸನ್ನಿಧಿಗೆ ಕರೆಸಿಕೊಂಡರು. ಮತ್ತೆ ಆಕೆಯ ಶಾಲೆಯ ವಿಷಯ ಪ್ರಸ್ತಾಪಕ್ಕೆ ಬಂದಿತು. ಯಾವುದೋ ವಿಷಯಕ್ಕೆ ಸಂಬಂಧಿಸಿದಂತೆ ನಿವೇದಿತೆ ತನ್ನ ವಾದವನ್ನು ಮುಂದಿಟ್ಟಳು. ಅವಳು ಹೇಳುವುದನ್ನೆಲ್ಲ ಸಾವಧಾನದಿಂದ ಕೇಳಿದ ಸ್ವಾಮೀಜಿ ಕಡೆಗೆ ನುಡಿದರು, “ಬಹುಶಃ ನೀನೆನ್ನುವುದೇ ಸರಿ, ಮ್ಯಾರ್ಗಟ್. ಆದರೆ ಈಗ ನನ್ನ ಮನಸ್ಸು ಇತರ ಚಿಂತನೆಯಲ್ಲಿ ಮುಳುಗಿಹೋಗಿದೆ... ನಾನೀಗ ಸಾವಿಗೆ ಸಿದ್ಧನಾಗುತ್ತಿದ್ದೇನೆ.” ಇದನ್ನು ಕೇಳಿ ನಿವೇದಿತೆ ದುಃಖಿತಳಾದಳು. ಆದರೆ ಆ ಮಾತು ನಿಜವಾಗಬಹುದೆಂದು ಅವಳಿಗನ್ನಿಸಲಿಲ್ಲ.

ಸ್ವಾಮೀಜಿಯವರು ಹೀಗೆ ನುಡಿದರೂ, ಮತ್ತು ಅವರು ತುಂಬ ಆಯಾಸಗೊಂಡಿದ್ದರೂ ಅವರ ಸಹಜ ಉದಾರಬುದ್ಧಿ, ಸತ್ಕಾರಬುದ್ಧಿ ಮರೆಯಾಗಿರಲಿಲ್ಲ. ನಿವೇದಿತೆಗೆ ಅಂದು ಅಲ್ಲೇ ಊಟ ಮಾಡುವಂತೆ ಹೇಳಿದರು. ಅಲ್ಲದೆ ತಾವೇ ಆಕೆಗೆ ಕೈಯಾರೆ ಉಣಬಡಿಸಿದರು. ಜೊತೆಗೆ ಖುಷಿಯಾಗಿ ನಗುತ್ತ ಹರಟಿದರು. ಊಟ ಮುಗಿದ ಮೇಲೆ ಒತ್ತಾಯದಿಂದ ತಾವೇ ಆಕೆಗೆ ಕೈತೊಳೆಯಲು ನೀರು ಸುರಿದು, ಟವೆಲ್ಲಿನಿಂದ ಕೈಯನ್ನು ಒರೆಸಿದರು. ನಿವೇದಿತೆ ಸಂಕೋಚದಿಂದ ಕುಗ್ಗಿಹೋದಳು. “ಸ್ವಾಮೀಜಿ, ಈ ರೀತಿಯ ಸೇವೆಯನ್ನು ನಾನು ನಿಮಗೆ ಮಾಡಬೇಕೆ ಹೊರತು ನೀವು ನನಗೆ ಮಾಡಬಾರದು!” ಎಂದು ಆಕೆಯೆಂದಾಗ ಸ್ವಾಮೀಜಿ ಗಂಭೀರವಾಗಿ ನುಡಿದರು, “ಏಸುಕ್ರಿಸ್ತ ತನ್ನ ಶಿಷ್ಯರ ಪಾದವನ್ನೇ ತೊಳೆದ.” “ಆದರೆ ಅದು ಅವನ ಜೀವನದ ಕಟ್ಟಕಡೆಯ ಬಾರಿ...” ಎಂಬ ಉತ್ತರ ನಿವೇದಿತೆಯ ನಾಲಿಗೆಯ ತುದಿಗೆ ಬಂದಿತ್ತು. ಆದರೆ ಅದೇನೋ ಒತ್ತಿ ಹಿಡಿದಂತಾಗಿ ಆ ಮಾತು ಒಳಗೇ ಉಳಿದುಕೊಂಡಿತು.

ನಿವೇದಿತೆ ತನ್ನ ಗುರುದೇವನಿಗೆ ನಮಸ್ಕರಿಸಿ ಅವರ ಅನಂತ ಆಶೀರ್ವಾದವನ್ನು ಹೊತ್ತು ಮರಳಿದಳು.

ಹೀಗೆ ಸ್ವಾಮೀಜಿಯವರು ತಮ್ಮ ಮಹಾಪ್ರಸ್ಥಾನದ ದಿನ ಸಮೀಪಿಸುತ್ತಿರುವುದರ ಬಗ್ಗೆ ಹಲವಾರು ಸೂಚನೆಗಳನ್ನು ನೀಡುತ್ತಿದ್ದರು. ಅವರು ಮತ್ತೆಮತ್ತೆ ಗಾಢನೀರವ ಧ್ಯಾನದಲ್ಲಿ ಲೀನರಾಗುವುದನ್ನು ಕಂಡು ಆಶ್ರಮವಾಸಿಗಳು ಕಳವಳಗೊಳ್ಳುತ್ತಿದ್ದರು. “ನರೇಂದ್ರ ತನ್ನ ಕೆಲಸ ವನ್ನು ಮುಗಿಸಿದ ಮೇಲೆ, ತಾನು ಯಾರು, ಎಲ್ಲಿಂದ ಬಂದೆ ಎಂಬುದು ಅರಿವಾದಾಗ, ಅವನು ಶರೀರದಲ್ಲಿ ಒಂದು ಕ್ಷಣವೂ ಇರಬಯಸದೆ ನಿರ್ವಿಕಲ್ಪ ಸಮಾಧಿಯಲ್ಲಿ ಲೀನನಾಗಿಬಿಡುತ್ತಾನೆ’ ಎಂಬ ಶ್ರೀರಾಮಕೃಷ್ಣರ ಮಾತು ಅವರ ಗುರುಭಾಯಿಗಳ ಕಿವಿಯಲ್ಲಿ ಒಂದೇ ಸಮನೆ ಪ್ರತಿ ಧ್ವನಿಸುತ್ತಿತ್ತು. ಅವರು ನಿರ್ಯಾಣ ಹೊಂದುವುದಕ್ಕೆ ಕೆಲದಿನ ಮೊದಲು ಗುರುಭಾಯಿಗಳ ಲ್ಲೊಬ್ಬರು ಮಾತಿನ ಸಂದರ್ಭದಲ್ಲಿ ಸುಮ್ಮನೆ ಕೇಳಿದರು, “ಶ್ರೀರಾಮಕೃಷ್ಣರು ಹೀಗೆ ಹೇಳಿದ್ದ ರಲ್ಲ, ಈಗ ನಿಮಗೆ ನೀವು ಯಾರೆಂಬುದು ಗೊತ್ತಾಗಿದೆಯೆ!” ಅತ್ಯಂತ ಗಂಭೀರ ದನಿಯಲ್ಲಿ ನುಡಿದರು ಸ್ವಾಮೀಜಿ, “ಹೌದು, ಈಗ ನನಗೆ ಗೊತ್ತಾಗಿದೆ!” ನೀಲಾಕಾಶದಿಂದ ಬಂದೆರಗಿದ ಸಿಡಿಲಿನಂತೆ ಈ ಉತ್ತರ ಅಲ್ಲಿದ್ದವರನ್ನು ವಿಸ್ಮಯಮೂಕರನ್ನಾಗಿಸಿತು. ಮತ್ತೆ ಮಾತನ್ನು ಮುಂದುವರಿಸಲು ಯಾರಿಗೂ ಧೈರ್ಯ ಸಾಲಲಿಲ್ಲ.

ಜೂನ್ ತಿಂಗಳ ಕೊನೆಯ ವಾರದಲ್ಲೊಂದು ದಿನ ಸ್ವಾಮೀಜಿ ತಮ್ಮ ಶಿಷ್ಯರಾದ ಶುದ್ಧಾನಂದ ರನ್ನು ಕರೆದು ಬಂಗಾಳೀ ಪಂಚಾಂಗವನ್ನು ತರಲು ಹೇಳಿದರು. ಪಂಚಾಂಗ ಬಂದಿತು. ಸ್ವಾಮೀಜಿ ಪಂಚಾಂಗವನ್ನು ತೆರೆದು, ಅಂದಿನ ದಿನದಿಂದ ಪ್ರಾರಂಭಿಸಿ, ಗಮನವಿಟ್ಟು ನೋಡತೊಡಗಿದರು. ಅವರು ಯಾವುದೋ ಕಾರ್ಯಕ್ಕಾಗಿ ಶುಭಮುಹೂರ್ತವನ್ನು ಹುಡುಕುತ್ತಿದ್ದಂತೆ ತೋರಿತು. ಬಹಳ ಹೊತ್ತಾದರೂ ಅವರು ಯಾವ ನಿರ್ಧಾರಕ್ಕೂ ಬಂದಂತೆ ಕಾಣಲಿಲ್ಲ. ಅವರು ಆ ಪಂಚಾಂಗವನ್ನು ತಮ್ಮ ಬಳಿಯೇ ಇರಿಸಿಕೊಂಡರಲ್ಲದೆ, ಆಮೇಲೂ ಒಂದೆರಡು ದಿನ ಅದನ್ನು ತೆಗೆದು ನೋಡುತ್ತಿದ್ದರು. ಯಾವುದಿರಬಹುದು ಅಂತಹ ಕಾರ್ಯ? ಯಾರಿಗೂ ಏನೂ ತೋಚ ಲಿಲ್ಲ. ಅವರ ಮಹಾಸಮಾಧಿಯ ನಂತರವೇ ಸೋದರ ಸಂನ್ಯಾಸಿಗಳಿಗೆಲ್ಲ ಅವರು ಪಂಚಾಂಗ ವನ್ನು ನೋಡುತ್ತಿದ್ದುದರ ರಹಸ್ಯ ಅರಿವಾದದ್ದು! ಸ್ವಾಮೀಜಿಯವರು ತಮ್ಮ ಐಹಿಕ ಬಂಧನ ಗಳನ್ನು ಹರಿದು ಹೊರಬರುವ ಶುಭದಿನವನ್ನು ಹುಡುಕುತ್ತಿದ್ದರೆಂಬುದು ಅವರ ಅರಿವಿಗೆ ಬಂದದ್ದು ಆಗಲೇ. ಅವರು ಹುಡುಕಾಡಿ ಆರಿಸಿಕೊಂಡ ದಿನ–ಜುಲೈ ನಾಲ್ಕು, ಶುಕ್ರವಾರ.

ಸ್ವಾಮೀಜಿಯವರ ಮಹಾಪ್ರಸ್ಥಾನಕ್ಕೆ ಇನ್ನು ಮೂರು ದಿನಗಳಿವೆ; ಅಂದು ಮಧ್ಯಾಹ್ನ ಅವರು ಸ್ವಾಮಿ ಪ್ರೇಮಾನಂದರೊಡನೆ ಮಾತನಾಡುತ್ತ ಮಠದ ವಿಶಾಲವಾದ ಬಯಲಿನಲ್ಲಿ ಅತ್ತಿಂದಿತ್ತ ನಡೆದಾಡುತ್ತಿದ್ದರು. ಹೀಗೆಯೇ ಮಾತನಾಡುತ್ತಿದ್ದ ಸ್ವಾಮೀಜಿ ಇದ್ದಕ್ಕಿದ್ದಂತೆ ನಿಂತು, ಗಂಗೆಯ ದಡದ ಮೇಲೊಂದು ಸ್ಥಳದತ್ತ ಬೆರಳು ಮಾಡಿ ತೋರಿಸಿ ಗಂಭೀರವಾಗಿ ನುಡಿದರು: “ನಾನು ಶರೀರ ಬಿಟ್ಟ ಮೇಲೆ ಇಲ್ಲಿ ದಹನ ಮಾಡಿ.” ‘ಈಗೇಕೆ ಆ ವಿಷಯ...?’ ಎಂಬ ಪ್ರಶ್ನೆ ಪ್ರೇಮಾನಂದರ ಮನದಲ್ಲಿ ಮೂಡಿತು. ಆದರೆ ಸ್ವಾಮೀಜಿ ಆ ಭಾವದಲ್ಲಿದ್ದಾಗ ಮಾತನಾಡುವ ಧೈರ್ಯ ಯಾರಿಗಿದ್ದೀತು! ಈಗ ಅದೇ ಸ್ಥಳದಲ್ಲಿ ಸ್ವಾಮೀಜಿಯವರ ಸ್ಮಾರಕವಾಗಿ ಮಂದಿರ ವೊಂದು ನಿಂತಿದೆ.

ಅಂದು ೧೯೦೨ರ ಜುಲೈ ೪. ಅದೇ ಸ್ವಾಮೀಜಿಯವರ ಅಂತಿಮ ದಿನ. ಅಂದಿನ ಅವರ ನಡೆನುಡಿಗಳೆಲ್ಲ ಉದ್ದಿಶ್ಯಪೂರ್ಣವಾಗಿದ್ದುವು, ಅರ್ಥಪೂರ್ಣವಾಗಿದ್ದುವು. ಅವರು ಎಂದಿನಂತೆ ಬೇಗ ಎದ್ದರು. ಚಹಾದ ವೇಳೆಗೆ ತಮ್ಮ ಗುರುಭಾಯಿಗಳೊಂದಿಗೆ ಹಿಂದಿನ ದಿನಗಳ ಅನೇಕ ಮಧುರ ಸ್ಮೃತಿಗಳನ್ನು ಮೆಲುಕು ಹಾಕಿದರು. ಎಂಟು ಗಂಟೆಯ ವೇಳೆಗೆ ದೇವಾಲಯಕ್ಕೆ ಹೋಗಿ ಧ್ಯಾನಕ್ಕೆ ಕುಳಿತರು. ಹನ್ನೊಂದು ಗಂಟೆಯವರೆಗೆ, ಎಂದರೆ ಸುಮಾರು ಮೂರು ಗಂಟೆಗಳ ಕಾಲ ಅಪೂರ್ವ ಧ್ಯಾನದಲ್ಲಿ ಮುಳುಗಿಹೋದರು. ಅದೊಂದು ಗಮನೀಯ ಅಂಶ. ಅಲ್ಲದೆ, ಅವರು ಧ್ಯಾನಕ್ಕೆ ಕುಳಿತ ಸ್ವಲ್ಪ ಹೊತ್ತಿನ ಮೇಲೆ ಮಠದ ಸಂನ್ಯಾಸಿಗಳು ನೋಡುತ್ತಾರೆ, ಅವರು ಒಳಗಿನಿಂದ ಕಿಟಕಿ ಬಾಗಿಲುಗಳನ್ನೆಲ್ಲ ಮುಚ್ಚಿಕೊಂಡುಬಿಟ್ಟಿದ್ದಾರೆ! ಅಲ್ಲಿ ಒಳಗೆ ಏನು ನಡೆಯು ತ್ತಿದೆಯೆಂದು ಯಾರು ತಾನೆ ಅರಿಯಬಲ್ಲರು? ಅವರ ಆ ದಿವ್ಯಧ್ಯಾನದಲ್ಲಿ, ಅವರು ಯಾವ ಜಗನ್ಮಾತೆಯನ್ನೂ ಶ್ರೀರಾಮಕೃಷ್ಣರನ್ನೂ ಒಂದೇ ಶಕ್ತಿಯ ಎರಡು ಮುಖಗಳೆಂದು ಅರಿತಿ ದ್ದರೋ ಆ ಇಬ್ಬರೂ ಅಲ್ಲಿ ಜ್ಯೋತಿರೂಪರಾಗಿ ಕಾಣಿಸಿಕೊಂಡಿರಬೇಕು. ಏಕೆಂದರೆ, ಧ್ಯಾನ ದಿಂದೆದ್ದು ಹೊರಬಂದಾಗ ಅವರು, ಹೃದಯ ಮಿಡಿಯುವಂತಹ ದೇವೀಪರವಾದ ಹಾಡೊಂ ದನ್ನು ಹೇಳಿಕೊಳ್ಳುತ್ತಿದ್ದರು–“ಮಾ ಕಿ ಅಮಾರ್​ಕಾಲೊ?... ಕಾಲೊ ರೂಪೆ ಹೃದಿಪದ್ಮಕೊರೆ ಅಲೊ”–“ನನ್ನ ತಾಯಿ ಕಪ್ಪಗಿಹಳೆ? ಅವಳು ಕಪ್ಪಗಿದ್ದರೂ ನನ್ನ ಹೃದಯಪದ್ಮವನ್ನು ಮಾತ್ರ ಬೆಳಗುತ್ತಿರುವಳಲ್ಲ!” ಈ ಹಾಡಿನಲ್ಲಿ ಅತ್ಯುನ್ನತ ಭಕ್ತಿಯೊಂದಿಗೆ ಅತ್ಯುನ್ನತ ಜ್ಞಾನ ಸಮರಸವಾಗಿ ಮಿಳಿತಗೊಂಡಿದೆ.

ದೇವಮಂದಿರದಿಂದ ಕೆಳಗಿಳಿದು ಬಂದ ಸ್ವಾಮೀಜಿಯವರು ವಿಶಾಲವಾದ ಹುಲ್ಲುಹಾಸಿನ ಮೇಲೆ ಭಾರವಾದ ಹೆಜ್ಜೆಗಳನ್ನಿಡುತ್ತ ನಡೆದಾಡತೊಡಗಿದರು. ಅವರ ಮನಸ್ಸು ಸಂಪೂರ್ಣ ಅಂತರ್ಮುಖವಾಗಿದೆ. ಅಲ್ಲೇ ವರಾಂಡದಲ್ಲಿ ಸ್ವಾಮಿ ಪ್ರೇಮಾನಂದರು ಕಾತರದ ಕಣ್ಣಿನಿಂದ ಅವರನ್ನೇ ಈಕ್ಷಿಸುತ್ತ ನಿಂತಿದ್ದಾರೆ. ಇದ್ದಕ್ಕಿದ್ದಂತೆ ಸ್ವಾಮೀಜಿ ನಿಂತರು. ಅವರ ಅಂತರಾಳದ ಭಾವನೆಯೊಂದು ಮೌನವನ್ನು ಸೀಳಿಕೊಂಡು ಪಿಸುದನಿಯಾಗಿ ಕೇಳಿಬಂದಿತು: “ಇನ್ನೊಬ್ಬ ವಿವೇಕಾನಂದನಿದ್ದಿದ್ದರೆ ಅವನಿಗೆ ತಿಳಿಯುತ್ತಿತ್ತು–ಈ ವಿವೇಕಾನಂದ ಏನು ಮಾಡಿದ್ದಾನೆ ಎಂದು.. ಇರಲಿ, ಕಾಲಾಂತರದಲ್ಲಿ ಮತ್ತೆಷ್ಟು ಮಂದಿ ವಿವೇಕಾನಂದರು ಉದಿಸಲಿರುವರೋ!” ಇದನ್ನು ಕೇಳಿದ ಪ್ರೇಮಾನಂದರು ಸ್ತಂಭೀಭೂತರಾದರು. ಏನಿದಕ್ಕೆಲ್ಲ ಅರ್ಥ...?!

ಅಂದು ಇನ್ನೊಂದು ವಿಶೇಷ ಘಟನೆ ನಡೆಯಿತು. ಸ್ವಾಮೀಜಿಯವರು ಯಾವಾಗಲೂ ತಮ್ಮ ಕೋಣೆಯಲ್ಲೇ ಊಟ ಮಾಡುವ ಪರಿಪಾಠವಿಟ್ಟುಕೊಂಡಿದ್ದವರು. ಆದರೆ ಅಂದು ಅವರು ಆಶ್ರಮವಾಸಿಗಳೆಲ್ಲರೊಂದಿಗೆ ಕುಳಿತು ಉಲ್ಲಾಸಭರಿತರಾಗಿ ಮಾತನಾಡುತ್ತ ಊಟ ಮಾಡಿದರು. ಮತ್ತು ಅತ್ಯಂತ ಆನಂದದಿಂದ ಅಡಿಗೆಯನ್ನು ಆಸ್ವಾದಿಸಿದರು. “ನಾನೆಂದೂ ಇಷ್ಟು ತೃಪ್ತಿಯಾಗಿ ಉಂಡಿಲ್ಲ” ಎಂದರು.

ಈಗ ಅವರು ಶುದ್ಧಾನಂದರಿಗೆ ವಾಚನಾಲಯದಿಂದ ಶುಕ್ಲಯಜುರ್ವೇದ ಗ್ರಂಥವನ್ನು ತರುವಂತೆ ಹೇಳಿದರು, ಶುದ್ಧಾನಂದರು ಅದನ್ನು ತಂದೊಡನೆ ಅದರಲ್ಲಿ “ಸುಷಮ್ಣಃ ಸೂರ್ಯ ರಶ್ಮಿಃ” ಎಂದು ಪ್ರಾರಂಭವಾಗುವ ಶ್ಲೋಕವನ್ನೂ ಅದರ ಭಾಷ್ಯವನ್ನೂ ಓದುವಂತೆ ಹೇಳಿದರು. ಅದು ವಾಜಸನೇಯೀ ಸಂಹಿತೆ ಮಾಧ್ಯಂದಿನ ಶಾಖೆಯ ೧೮ನೇ ಆಧ್ಯಾಯದ ೪ಂನೇ ಶ್ಲೋಕ. ಆ ಶ್ಲೋಕ ಹೀಗಿದೆ:

\begin{verse}
ಸುಷುಮ್ಣಃ ಸೂರ್ಯರಶ್ಮಿಶ್ಚಂದ್ರಮಾ ಗಂಧರ್ವಸ್ತಸ್ಯ\\ನಕ್ಷತ್ರಾಣ್ಯಪ್ಸರಸೋ ಭೇಕುರಯೋ ನಾಮ ।\\ಸನ ಇದಂ ಬ್ರಹ್ಮಕ್ಷತ್ರಂ ಪಾತು\\ತಸ್ಮೈ ಸ್ವಾಹಾ ವಾಟ್ ತಾಭ್ಯಃ ಸ್ವಾಹಾ ॥
\end{verse}

ಇದಕ್ಕೆ ಮಹೀಧರನ ವ್ಯಾಖ್ಯೆ ಹೀಗಿದೆ: “ಯಾರು ಸುಷುಮ್ನವಾಗಿದ್ದಾನೆಯೋ (ಅರ್ಥಾತ್, ಯಜ್ಞಗಳನ್ನು ಮಾಡುವವರಿಗೆ ಅತ್ಯಾನಂದವನ್ನು ನೀಡುತ್ತಾನೆಯೋ), ಯಾರ ಕಿರಣಗಳು ಸೂರ್ಯಕಿರಣಗಳಿಗೆ ಸಮನಾಗಿವೆಯೊ, ಅಂತಹ ಗಂಧರ್ವರೂಪಿಯಾದ ಚಂದ್ರನು ನಮ್ಮನ್ನು ಪೊರೆಯಲಿ. ಅವನಿಗೆ ನಮ್ಮ ಭಕ್ತಿಪೂರ್ವಕ ನಮನಗಳು. ಮತ್ತು, ಆ ಚಂದ್ರನ ಅಪ್ಸರೆಯರೂ ಸ್ವಯಂ ಪ್ರಕಾಶದಿಂದ ವಿರಾಜಿಸುತ್ತಿರುವವರೂ ಆದ ನಕ್ಷತ್ರಗಳಿಗೆ ನಮ್ಮ ಭಕ್ತಿಪೂರ್ವಕ ನಮನಗಳು.”

ಸ್ವಾಮಿ ಶುದ್ಧಾನಂದರು ಇದನ್ನು ಓದಿ ಮುಗಿಸುತ್ತಿದ್ದಂತೆಯೇ ಸ್ವಾಮೀಜಿ ಹೇಳಿದರು –“ನೋಡು, ಈ ಮಂತ್ರದ ವ್ಯಾಖ್ಯಾನ ನನಗೆ ಒಪ್ಪಿಗೆಯಾಗಲಿಲ್ಲ. ಸುಷುಮ್ನದ ವಿಷಯದಲ್ಲಿ ಭಾಷ್ಯಕಾರನ ಅಭಿಪ್ರಾಯ ಹಾಗಿರಲಿ; ವೇದಗಳ ರಚನೆಯ ಎಷ್ಟೋ ಕಾಲದ ಬಳಿಕ ರಚಿತವಾದ ತಂತ್ರಗಳಲ್ಲಿ ಸುಷುಮ್ನಾ ನಾಡಿ ಎಂದು ಯಾವುದನ್ನು ವಿವರಿಸಲಾಗಿದೆಯೋ, ಅದರ ಭಾವನೆಯೇ ಈ ಶ್ಲೋಕದಲ್ಲಿ ಬೀಜರೂಪದಲ್ಲಿ ಅಡಕವಾಗಿದೆ. ನನ್ನ ಶಿಷ್ಯರಾದ ನೀವು ಈ ಮಂತ್ರಗಳ ನಿಜವಾದ ಅರ್ಥವನ್ನು ಸಂಶೋಧಿಸಿ ತೆಗೆದು ಇವುಗಳ ಮೇಲೆ ನಿಮ್ಮ ಅನುಭವಪೂರ್ಣವಾದ ಸ್ವಂತ ಭಾಷ್ಯಗಳನ್ನು ಬರೆಯಬೇಕು.”

ಯೋಗಿಗಳಿಗೆ ಈ ಸುಷುಮ್ನಾ ನಾಡಿ ಎಂಬುದು ತುಂಬ ಮುಖ್ಯವಾದದ್ದು. ಸ್ವಾಮೀಜಿಯವರು ಯೋಗದ ಮೂಲಕ ತಮ್ಮ ಶರೀರವನ್ನು ತ್ಯಜಿಸುವ ದಿನದಂದೇ ಸುಷುಮ್ನಾ ನಾಡಿಯ ವಿಷಯವಾಗಿ ಹೇಳಿದುದು ಒಂದು ಗಮನಾರ್ಹ ಅಂಶ.

ಅಂದು ಮಧ್ಯಾಹ್ನ ವಿಶ್ರಾಂತಿಯ ನಂತರ ಸ್ವಾಮೀಜಿ ಬ್ರಹ್ಮಚಾರಿಗಳ ಕೋಣೆಗಳಿಗೆ ಹೋಗಿ ತಾವೇ ಅವರನ್ನು ಅಂದಿನ ಸಂಸ್ಕೃತ ವ್ಯಾಕರಣದ ತರಗತಿಗೆ ಕರೆತಂದರು. ತರಗತಿ ಸುಮಾರು ಮೂರು ಗಂಟೆಗಳಷ್ಟು ದೀರ್ಘಕಾಲ ನಡೆಯಿತು. ಆದರೆ ವ್ಯಾಕರಣದಂತಹ ನೀರಸ ವಿಷಯವೂ ಬೇಸರ ತರುವಂತಿರಲಿಲ್ಲ. ಇದಕ್ಕೆ ಕಾರಣ ಸ್ವಾಮೀಜಿಯವರು ವಿಷಯವನ್ನು ನಿರೂಪಿಸುತ್ತಿದ್ದ ರೀತಿ. ವಿಷಯಕ್ಕೆ ಸಂಬಂಧಿಸಿದಂತೆ ಮಧ್ಯೆಮಧ್ಯೆ ಏನಾದರೂ ವಿನೋದದ ಕಥೆಗಳೋ ಚಾತುರ್ಯದ ನುಡಿಗಳೋ ಇದ್ದೇ ಇರುತ್ತಿದ್ದುವು. ಯಾವುದಾದರೊಂದು ಸೂತ್ರದ ಪದಲಾಲಿತ್ಯ ವನ್ನು ಉಪಯೋಗಿಸಿಕೊಂಡು ಅದನ್ನು ಅವರು ವಿಶಿಷ್ಟ ಹಾವಭಾವಗಳೊಂದಿಗೆ ವಿವರಿಸಿದರೆ ಎಂಥವನೂ ಅದನ್ನು ಮರೆಯುವಂತಿರಲಿಲ್ಲ.

ಸುಮಾರು ನಾಲ್ಕು ಗಂಟೆಯ ಹೊತ್ತಿಗೆ ಸ್ವಾಮೀಜಿಯವರು ಪ್ರೇಮಾನಂದರನ್ನೊಡಗೂಡಿ ಕೊಂಡು ಬೇಲೂರು ಬಜಾರಿನವರೆಗೆ ನಡೆದುಕೊಂಡು ಹೋದರು. ದಾರಿಯಲ್ಲಿ ಅವರು ಅನೇಕ ವಿಚಾರಗಳ ಕುರಿತಾಗಿ ಮಾತನಾಡಿದರು. ಮಠದಲ್ಲಿ ವೇದಾಂತ ಕಾಲೇಜೊಂದನ್ನು ಸ್ಥಾಪಿಸುವ ವಿಷಯ ಪ್ರಸ್ತಾಪಕ್ಕೆ ಬಂದಿತು. ಈ ವಿಷಯದಲ್ಲಿ ಅವರ ನಿಲುವನ್ನು ಸ್ಪಷ್ಟವಾಗಿ ಅರಿಯಲು ಪ್ರೇಮಾನಂದರು ಕೇಳಿದರು, “ಸ್ವಾಮೀಜಿ, ಈ ವೇದಗಳನ್ನು ಅಧ್ಯಯನ ಮಾಡುವುದರಿಂದ ಪ್ರಯೋಜನವೇನು?” ಸ್ವಾಮೀಜಿಯವರ ನೇರ ಗಂಭೀರ ಉತ್ತರ ಬಂದಿತು–“ಅದು ನಮ್ಮ ಬುದ್ಧಿಗಂಟಿದ ಮೂಢ ನಂಬಿಕೆಗಳನ್ನು ನಾಶಗೊಳಿಸುತ್ತದೆ.”

ನಡೆದಾಟ ಮುಗಿಸಿ ಸ್ವಾಮೀಜಿ ಮಠಕ್ಕೆ ಹಿಂದಿರುಗಿದರು. ಸ್ವಲ್ಪ ಹೊತ್ತು ಇತರ ಸಂನ್ಯಾಸಿ ಗಳೊಂದಿಗೆ ಮಾತನಾಡಿದರು. ಆಹ್! ಈ ಮಾತುಗಳೇ ಸ್ವಾಮೀಜಿಯವರ ಕೊನೆಯ ಮಾತು ಗಳು ಎನ್ನುವುದು ಆ ಸಂನ್ಯಾಸಿಗಳಿಗೇನಾದರೂ ಗೊತ್ತಿದ್ದರೆ ಅವರು ತಮ್ಮ ಪ್ರಿಯ ನಾಯಕನ ತುಟಿಗಳಿಂದ ಹೊರ ಹೊಮ್ಮುತ್ತಿದ್ದ ಆ ನುಡಿಗಳನ್ನು ಇನ್ನಷ್ಟು ಎಚ್ಚರಿಕೆಯಿಂದ ಕೇಳಿ ಅವು ಗಳನ್ನು ತಮ್ಮ ಹೃದಯಗಳಲ್ಲಿ ಕಾಪಾಡಿಕೊಳ್ಳುತ್ತಿದ್ದರೋ ಏನೋ!

ಸಂಜೆ ಸಮೀಪಿಸಿತು. ಸ್ವಾಮೀಜಿಯವರ ಮನಸ್ಸು ಹೆಚ್ಚು ಹೆಚ್ಚು ಅಂತರ್ಮುಖವಾಗುತ್ತ ಬಂದಿತು. ಸಂಜೆಯ ಆರತಿಯ ಘಂಟಾನಾದ ಕೇಳುತ್ತಿದ್ದಂತೆಯೇ ಅವರು ತಮ್ಮ ಕೋಣೆಗೆ ಹಿಂದಿರುಗಿದರು. ತಮ್ಮ ಪರಿಚರ್ಯೆ ಮಾಡುತ್ತಿದ್ದ ಬ್ರಹ್ಮಚಾರಿಗೆ ಜಪಮಾಲೆಯನ್ನು ತರುವಂತೆ ಹೇಳಿದರು. ಬ್ರಹ್ಮಚಾರಿಗೆ ಹೊರಗೆ ಕುಳಿತು ಧ್ಯಾನಮಾಡುವಂತೆ ಹೇಳಿ ಕಿಟಕಿಗಳನ್ನೆಲ್ಲ ಮುಚ್ಚಿ ಕೊಂಡು ತಾವೂ ಜಪಕ್ಕೆ ಕುಳಿತರು. ಅಂದು ಅವರು ಗಂಗೆಗೆ ಅಭಿಮುಖವಾಗಿ ಕುಳಿತದ್ದನ್ನು ಬ್ರಹ್ಮಚಾರಿ ಗಮನಿಸಿದರು. ಅವರು ಎಂದೂ ಆ ದಿಕ್ಕಿಗೆ ತಿರುಗಿಕೊಂಡು ಕುಳಿತಿರದವರು ಇಂದು ಹಾಗೆ ಕುಳಿತದ್ದು ಒಂದು ವಿಶೇಷ.

ಒಂದು ಗಂಟೆಯ ಬಳಿಕ ಸ್ವಾಮೀಜಿ ಬ್ರಹ್ಮಚಾರಿಯನ್ನು ಒಳಕರೆದು ಸ್ವಲ್ಪ ಗಾಳಿ ಹಾಕುವಂತೆ ಹೇಳಿ ಜಪಮಾಲೆ ಹಿಡಿದೇ ಮಲಗಿಕೊಂಡರು, ಸ್ವಲ್ಪ ಹೊತ್ತಾದ ಮೇಲೆ, “ಗಾಳಿ ಹಾಕಿದ್ದು ಸಾಕು, ಈ ಪಾದಗಳನ್ನು ಸ್ವಲ್ಪ ತಿಕ್ಕು” ಎಂದು ಹೇಳಿ ನಿದ್ರೆಯಲ್ಲಿ ಮುಳುಗಿದರು. ಹೀಗೆಯೇ ಮತ್ತೆ ಒಂದು ಗಂಟೆ ಕಳೆಯಿತು. ಬಳಿಕ ಅವರ ಕೈ ಸ್ವಲ್ಪ ಮಾತ್ರ ಅಲುಗಿತು. ಆಗ ಅವರು ಒಂದು ದೀರ್ಘವಾದ ಉಸಿರೆಳೆದರು. ಮತ್ತೆರಡು ನಿಮಿಷಗಳ ಕಾಲ ನಿಶ್ಶಬ್ದ. ಮತ್ತೊಮ್ಮೆ ಅವರು ದೀರ್ಘವಾಗಿ ಉಸಿರೆಳೆಯುತ್ತಿದ್ದಂತೆ ಅವರ ದೃಷ್ಟಿ ಎರಡೂ ಹುಬ್ಬುಗಳ ನಡುವೆ ಕೇಂದ್ರಿತ ವಾಯಿತು. ಮುಖದಲ್ಲಿ ಅಲೌಕಿಕ ದಿವ್ಯತೆಯೊಂದು ಮೂಡಿ ಬರುತ್ತಿದ್ದಂತೆಯೇ ಅವರು ಇನ್ನಿಲ್ಲವಾದರು. ಇದನ್ನು ಕಂಡ ಬ್ರಹ್ಮಚಾರಿ ವಿಭ್ರಾಂತರಾದರು. ಸ್ವಾಮೀಜಿ ಸಮಾಧಿಮಗ್ನ ರಾಗಿರಬಹುದೆ? ತಕ್ಷಣ ಓಡಿಹೋಗಿ ಸ್ವಾಮಿ ಅದ್ವೈತಾನಂದರಿಗೆ ಸುದ್ದಿ ಕೊಟ್ಟರು. ರಾತ್ರಿಯ ಊಟಕ್ಕೆ ಸಿದ್ಧರಾಗುತ್ತಿದ್ದ ಸಾಧು ಬ್ರಹ್ಮಚಾರಿಗಳೆಲ್ಲ ಓಡಿಬಂದರು. ಸ್ವಾಮೀಜಿಯವರು ಉನ್ನತ ಸಮಾಧಿಸ್ಥಿತಿಗೇರಿರಬೇಕೆಂದು ಊಹಿಸಿ ಎಲ್ಲರೂ ಗಟ್ಟಿಯಾಗಿ ರಾಮಕೃಷ್ಣನಾಮವನ್ನು ಉಚ್ಚರಿಸ ತೊಡಗಿದರು. ಈ ಉಪಾಯ ಹಿಂದೆ ಎಷ್ಟೋ ಸಲ ಕೆಲಸ ಮಾಡಿತ್ತು. ಆದರೆ ಇಂದಿನ ಅವರ ಈ ಸಮಾಧಿಯು, ಮತ್ತೆ ಹಿಂದಿರುಗಿ ಬರಲಾಗದ ಮಹಾಸಮಾಧಿಯಾಗಿತ್ತು. ಸ್ವಾಮಿ ಅದ್ವೈತಾ ನಂದರು ಬೋಧಾನಂದರಿಗೆ ನಾಡಿ ಪರೀಕ್ಷಿಸುವಂತೆ ಹೇಳಿದರು. ಸ್ವಲ್ಪ ಹೊತ್ತು ನಾಡಿ ಹಿಡಿಯಲು ಪ್ರಯತ್ನಿಸಿದ ಬೋಧಾನಂದರು ಕಡೆಗೆ ಎದ್ದು ನಿಂತು ರೋದಿಸಲಾರಂಭಿಸಿದರು. ಕೂಡಲೇ ಡಾ । ಮಹೇಂದ್ರನಾಥ ಮಜುಮ್ದಾರರಿಗೂ, ಗಂಗೆಯ ಆಚೆಯ ದಡದಲ್ಲಿದ್ದ ಬ್ರಹ್ಮಾನಂದರು ಮತ್ತು ಶಾರದಾನಂದರಿಗೂ ಸುದ್ದಿ ಕಳಿಸಲಾಯಿತು. ಗುರುಭಾಯಿಗಳು ಬರುವ ವೇಳೆಗೆ ರಾತ್ರಿ ಹತ್ತೂವರೆಯಾಗಿತ್ತು. ಸ್ವಾಮಿ ಬ್ರಹ್ಮಾನಂದರು ದುಃಖೋದ್ವೇಗದಿಂದ ತಮ್ಮ ಪ್ರಿಯ ನರೇಂದ್ರನ ಎದೆಯ ಮೇಲೆ ಮುಖವಿರಿಸಿ ಅಳತೊಡಗಿದರು. ಅವರನ್ನು ಇತರರು ಸಮಾಧಾನ ಪಡಿಸಿ ದೂರಕ್ಕೆ ಸೆಳೆಯಬೇಕಾಯಿತು. ಬ್ರಹ್ಮಾನಂದರು ಬಿಕ್ಕುತ್ತ ನುಡಿದರು, “ಓಹ್​! ಹಿಮಾಲಯವು ನಮ್ಮಿಂದ ಕಣ್ಮರೆಯಾಯಿತು!”

ಸ್ವಲ್ಪ ಹೊತ್ತಿಗೆಲ್ಲ ಡಾಕ್ಟರು ಬಂದರು. ಸ್ವಾಮೀಜಿಯವರನ್ನು ಸೂಕ್ಷ್ಮವಾಗಿ ಪರೀಕ್ಷಿಸಿದರು. ಉಸಿರಾಟವಿಲ್ಲದಿರುವುದನ್ನು ಗಮನಿಸಿ, ಕೃತಕ ಉಸಿರಾಟದ ಪ್ರಯೋಗ ನಡೆಸಿದರು. ಆದರೆ ಯಾವುದರಿಂದಲೂ ಏನೂ ಪ್ರಯೋಜನವಾಗಲಿಲ್ಲ. ಕಡೆಗೆ ಮಧ್ಯರಾತ್ರಿಯ ವೇಳೆಗೆ ಡಾಕ್ಟರು, ‘ಸ್ವಾಮೀಜಿ ಶರೀರವನ್ನು ಬಿಟ್ಟುಬಿಟ್ಟಿದ್ದಾರೆ’ ಎಂದು ಘೋಷಿಸಿದರು.

ಅಂದಿಗೆ ಸ್ವಾಮೀಜಿಯವರ ವಯಸ್ಸು ಕೇವಲ ಮೂವತ್ತೊಂಬತ್ತು ವರ್ಷ, ಐದು ತಿಂಗಳು, ಇಪ್ಪತ್ತನಾಲ್ಕು ದಿನ. “ನಾನು ನಲ್ವತ್ತನೆಯ ವರ್ಷವನ್ನು ಕಾಣುವುದಿಲ್ಲ” ಎಂಬ ಅವರ ಭವಿಷ್ಯ ವಾಣಿ ಸತ್ಯವಾಗಿತ್ತು.

ಆದರೆ ಈ ಭೀಕರ ವಾರ್ತೆಯನ್ನು ಎಲ್ಲಾದರೂ ನಂಬಲುಂಟೆ! ಬೇಲೂರು ಮಠದ ಸಂನ್ಯಾಸಿ ಗಳೆಲ್ಲ ದಿಙ್ಮೂಢರಾದರು. ಮಠ ಶೋಕಸಾಗರದಲ್ಲಿ ಮುಳುಗಿತು. ಬೆಳಗಾಗುತ್ತಿದ್ದಂತೆಯೇ ಎಲ್ಲೆಡೆಗಳಿಂದ ಜನ ಪ್ರವಾಹದೋಪಾದಿಯಲ್ಲಿ ಬರಲಾರಂಭಿಸಿದರು. ಎಲ್ಲೆಲ್ಲೂ ಶೋಕ, ಎಲ್ಲೆಲ್ಲೂ ಸಂತಾಪ. ಸ್ವಾಮೀಜಿಯವರ ಶರೀರ ಅವರ ಕೋಣೆಯಲ್ಲಿ ನಿಶ್ಚಲವಾಗಿ ಮಲಗಿತ್ತು. ಕೇವಲ ಎರಡು ದಿನಗಳ ಹಿಂದೆ ಅವರ ಸ್ಫೂರ್ತಿದಾಯಕ ಮಾತುಗಳನ್ನೂ ಅವರ ಭಾವಪೂರ್ಣ ನಡೆನುಡಿಗಳನ್ನೂ ಅವರ ಆನಂದದ ಅಟ್ಟಹಾಸವನ್ನೂ ಕೇಳಿದ್ದ ಜನರಿಗೆ ಅವರ ಮರಣದ ಸುದ್ದಿಯನ್ನು ನಂಬಲು ಸಾಧ್ಯವೇ ಆಗಲಿಲ್ಲ. ಅಲ್ಲಿಗೆ ಆಗಮಿಸಿದ ಸಾವಿರಾರು ಜನ ಸ್ವಾಮೀಜಿ ಯವರ ಶರೀರವನ್ನೇ ದಿಟ್ಟಿಸುತ್ತ ಅವರು ಮರಣ ಹೊಂದಿರುವ ವಿಷಯದಲ್ಲಿ ಇನ್ನೂ ಅನುಮಾನದಿಂದಲೇ ನಿಂತಿದ್ದರು. ಸ್ವಾಮೀಜಿಯವರ ಮುಖ ಅಷ್ಟು ಪ್ರಕಾಶಮಾನವಾಗಿತ್ತು! ಆ ವಿಶಾಲನಯನಗಳಲ್ಲಿ ಅದೇ ದಿವ್ಯಜ್ಯೋತಿ ಬೆಳಗುತ್ತಿತ್ತು! ಮಂಗಳಕರವಾದ ಸಾಕ್ಷಾತ್ ಸದಾಶಿವನೇ ಅಲ್ಲಿ ಒರಗಿಕೊಂಡಂತಿತ್ತು! ಆ ದಿವ್ಯ ಶರೀರವನ್ನು ದರ್ಶಿಸಿದ ಹಲವಾರು ಜನ ತಮ್ಮ ದುಃಖಗಳನ್ನು ಮರೆತು ಅಪೂರ್ವ ಶಾಂತಿಯನ್ನನುಭವಿಸಿದರು. ಇನ್ನು ಕೆಲವರಿಗೆ, ಸ್ವಾಮೀಜಿ ತಮ್ಮ ದೀರ್ಘ ಸಮಾಧಿಯಿಂದ ಇನ್ನೇನು ಇಳಿದುಬಂದಾರು ಎಂಬ ಭ್ರಮೆಯಿತ್ತು. ಆದ್ದರಿಂದ ಮರುದಿನ ಬೆಳಗ್ಗೆ ದೇಹವನ್ನು ಮೇಲಿನ ಕೋಣೆಯಲ್ಲೆ ತುಂಬ ಹೊತ್ತು ಬಿಡ ಲಾಗಿತ್ತು. ಆದರೆ ಹೊತ್ತು ಕಳೆದಂತೆ ಇನ್ನು ಯಾವ ಸಂಶಯವೂ ಉಳಿಯಲಿಲ್ಲ.

ಸುದ್ದಿ ತಿಳಿಯುತ್ತಿದ್ದಂತೆಯೇ ನಿವೇದಿತೆಯೂ ಧಾವಿಸಿ ಬಂದಳು. ಹಿಂದಿನ ದಿನ ತಾನೆ ಸ್ವಾಮೀಜಿಯವರ ದೇಹಸ್ಥಿತಿ ತುಂಬ ಚೆನ್ನಾಗಿದೆಯೆಂದು ಆಕೆ ಕೇಳಿದ್ದಳು. ಆ ರಾತ್ರಿ ಅವಳು ಮನೆಯ ತಾರಸಿಯ ಮೇಲೆ ಬೇಲೂರಿಗೆ ಅಭಿಮುಖವಾಗಿ ಕುಳಿತು ಹಲವು ಗಂಟೆಗಳ ಕಾಲ ಧ್ಯಾನ ಮಾಡಿದ್ದಳು. ಆದರೆ ಮಲಗಿ ನಿದ್ರಿಸುತ್ತಿದ್ದಂತೆಯೇ ಆಕೆಗೊಂದು ಕನಸಾಗಿತ್ತು. ಅದರ ಲ್ಲವಳು ಶ್ರೀರಾಮಕೃಷ್ಣರು ಮತ್ತೊಮ್ಮೆ ದೇಹತ್ಯಾಗ ಮಾಡಿದ್ದಂತೆ ಕಂಡಿದ್ದಳು. ಈಗ ಬೆಳಗಾ ಗುತ್ತಲೇ ಅವಳಿಗೆ ಸಿಕ್ಕಿದ ಸುದ್ದಿ ಇದು. ತಕ್ಷಣವೇ ಹೊರಟು ಬೇಲೂರು ಮಠಕ್ಕೆ ಬಂದಳು. ಅಲ್ಲಿ ಅವಳ ಪರಮ ಪ್ರೀತಿಯ ದಿವ್ಯಗುರುವಿನ ಶರೀರವು ಪುಷ್ಪಗಳಿಂದ ಅಲಂಕೃತವಾಗಿ ನೆಲದ ಮೇಲೆ ಮಲಗಿತ್ತು. ನಿವೇದಿತೆ ಅಲ್ಲೇ ಕುಳಿತು, ಸ್ವಾಮೀಜಿಯವರ ಶಿರವನ್ನು ತನ್ನ ಮಡಿಲಲ್ಲಿರಿಸಿ ಕೊಂಡು ಮೆಲ್ಲನೆ ಗಾಳಿ ಬೀಸಿದಳು.

ಗುರುದೇವ ಶ್ರೀರಾಮಕೃಷ್ಣರ ನಿರ್ಯಾಣದ ಬಳಿಕ ಅವರ ಶಿಷ್ಯರು ಇಂತಹ ಕಠೋರ ವಾರ್ತೆಯನ್ನೆಂದೂ ಕೇಳಿರಲಿಲ್ಲ. ಈ ಸಮಯದಲ್ಲಿ ಶಿಷ್ಯರ ಮನದಲ್ಲಿ ಗುರುದೇವನ ಅಂತ್ಯ ಕ್ರಿಯೆಯ ದೃಶ್ಯವೇ ಮತ್ತೆಮತ್ತೆ ಮೂಡಿಬರುತ್ತಿತ್ತು. ಶ್ರೀರಾಮಕೃಷ್ಣರು ತೀರಿಕೊಳ್ಳುವ ಮುನ್ನ ತಮ್ಮ ಶಕ್ತಿಯನ್ನು ಶಿಷ್ಯ ನರೇಂದ್ರನಿಗೆ ಧಾರೆಯೆರೆದಿದ್ದರು. ಈಗ ಅವರೀರ್ವರೂ ಇಹಲೋಕ ವನ್ನು ತ್ಯಜಿಸಿದ ಮೇಲೆ ತಾವೆಲ್ಲ ಅನಾಥರಾದಂತೆ ಆ ಸಂನ್ಯಾಸಿಗಳಿಗೆ ಭಾಸವಾಯಿತು.

ಸ್ವಾಮೀಜಿಯವರ ನಿಧನ ವಾರ್ತೆಯನ್ನು ಕಲ್ಕತ್ತದಲ್ಲೆಲ್ಲ ಪ್ರಸಾರ ಮಾಡುವ ವ್ಯವಸ್ಥೆ ಯಾಯಿತು. ತಂತಿಯ ಮೂಲಕ ಭಾರತದ ಹಾಗೂ ಜಗತ್ತಿನ ಇತರ ಭಾಗಗಳಿಗೆ ಈ ವಿಷಾದದ ಸುದ್ದಿಯನ್ನು ಕಳಿಸಲಾಯಿತು. ಗಂಧದ ಕೊರಡುಗಳು, ಸುಗಂಧದ್ರವ್ಯ, ಪುಷ್ಪಾದಿಗಳನ್ನು ತರಲು ವ್ಯವಸ್ಥೆ ಮಾಡಲಾಯಿತು. ಎಲ್ಲೆಡೆಗಳಲ್ಲಿ ಪರಿಮಳಭರಿತ ಗಂಧದ ಕಡ್ಡಿಗಳನ್ನು ಹಚ್ಚಿಡಲಾಯಿತು. ಆ ದಿವ್ಯ ಪುರುಷನ ಅಂತಿಮ ದರ್ಶನಕ್ಕಾಗಿ ಅನೇಕ ಸಹಸ್ರ ಜನ ಬಂದು ಮಠದಲ್ಲಿ ನೆರೆದರು.

ಮಧ್ಯಾಹ್ನದ ವೇಳೆಗೆ ಸ್ವಾಮೀಜಿಯವರ ಶರೀರವನ್ನು ಮಹಡಿಯ ಮೇಲಿಂದ ಕೆಳಗೆ ತಂದು ಪ್ರಾಂಗಣದಲ್ಲಿ ಇರಿಸಲಾಯಿತು. ಅವರ ಪಾದಗಳಿಗೆ ಚಂದನವನ್ನು ಲೇಪಿಸಿ ಮಸ್ಲಿನ್ ಬಟ್ಟೆ ಯೊಂದರ ಮೇಲೆ ಪಾದದ ಮುದ್ರೆಯನ್ನು ತೆಗೆದುಕೊಳ್ಳಲಾಯಿತು. ಬಳಿಕ ಸ್ವಾಮೀಜಿಯವರ ಶರೀರಕ್ಕೆ ಆರತಿ ಮಾಡಿದರು. ವೇದಮಂತ್ರಗಳ ಪಠಣ, ಶಂಖ-ಜಾಗಟೆ-ಗಂಟೆಗಳ ನಿನಾದ, ಮಂಗಳದ್ರವ್ಯಗಳ ಸುವಾಸನೆ–ಇವುಗಳ ನಡುವೆ ಅಸಂಖ್ಯಾತ ಜನ ಬಂದು ಸ್ವಾಮೀಜಿಯವರಿಗೆ ತಮ್ಮ ಅಶ್ರುತರ್ಪಣವನ್ನರ್ಪಿಸಿದರು.

ಇದೀಗ ಅಂತಿಮ ಯಾತ್ರೆ ಪ್ರಾರಂಭವಾಯಿತು. ಸ್ವಾಮೀಜಿಯವರ ಪರಮ ಪವಿತ್ರ ಶರೀರವನ್ನು ಅವರ ಗುರುಭಾಯಿಗಳು ಹೊತ್ತು ಸಾಗಿದರು. ಭಕ್ತರ ಅಂತರಾಳದಿಂದ ಮಹಾ ಉದ್ಘೋಷ ತಾನೇತಾನಾಗಿ ಮೂಡಿಬಂದಿತು.

\begin{verse}
ಜೈ ಶ್ರೀ ಗುರುಮಹಾರಾಜ್ ಜೀ ಕೀ ಜೈ!!!\\ಜೈ ಶ್ರೀ ಸ್ವಾಮೀಜಿ ಮಹಾರಾಜ್ ಜೀ ಕೀ ಜೈ!!!
\end{verse}

ಮೆರವಣಿಗೆ ಮಠದ ದಕ್ಷಿಣ ದಿಸೆಯತ್ತ ನಡೆಯಿತು. ಸ್ವಾಮೀಜಿಯವರೇ ನಿರ್ದೇಶಿಸಿದ್ದ ಸ್ಥಳದಲ್ಲಿ ಚಿತಾವೇದಿಕೆಯನ್ನು ನಿರ್ಮಿಸಲಾಗಿತ್ತು. ಶರೀರವನ್ನು ಮೆಲ್ಲನೆ ಚಿತೆಯ ಮೇಲಿರಿಸ ಲಾಯಿತು. ಬಳಿಕ ಆರ್ತನಾದ, ಜಯಘೋಷಗಳ ನಡುವೆ ಚಿತೆಗೆ ಅಗ್ನಿಸ್ಪರ್ಶ ಮಾಡಲಾಯಿತು.

ಅಗ್ನಿದೇವ ತನ್ನ ಧಗಧಗಿಸುವ ಜ್ವಾಲೆಗಳಿಂದ ಚಿತೆಯನ್ನಾವರಿಸಿದ. ಸನಿಹದಲ್ಲೇ ನಿಂತಿದ್ದ ನಿವೇದಿತೆ ಕೈಗಳಿಂದ ಮುಖವನ್ನು ಮುಚ್ಚಿಕೊಂಡಳು. ಅವಳ ಮನದಲ್ಲಿ ಒಂದೇ ಒಂದು ಆಸೆಯಿತ್ತು–ಅದು ಸ್ವಾಮೀಜಿಯವರ ಮಂಚವನ್ನು ಮುಚ್ಚಿದ್ದ ಕಾಷಾಯವಸ್ತ್ರದ ಒಂದು ತುಂಡನ್ನು ತೆಗೆದಿರಿಸಿಕೊಳ್ಳಬೇಕು ಎಂಬುದು. ಸ್ವಾಮಿ ಶಾರದಾನಂದರು ಅದನ್ನು ತೆಗೆದುಕೊಡಲು ಮುಂದಾಗಿದ್ದರು. ಆದರೆ ಅದೇಕೋ ಆಕೆ ಅದನ್ನು ತೆಗೆದುಕೊಳ್ಳಲು ಹಿಂಜರಿದಳು. ಈಗ ನಿವೇದಿತೆ ತನ್ನ ಮುಖವನ್ನು ಮುಚ್ಚಿಕೊಂಡು ನಿಂತಿದ್ದಾಗ, ಆಕೆಯ ತೋಳಿಗೆ ಏನೋ ತಗುಲಿತು, ಏನದು? ಅದು ಚಿತಾಗ್ನಿಯಿಂದ ಹಾರಿಬಂದ, ಅರ್ಧ ಸುಟ್ಟ ಕಾಷಾಯವಸ್ತ್ರದ ಒಂದು ತುಂಡು! ಅದನ್ನವಳು ಕಣ್ಣಿಗೊತ್ತಿಕೊಂಡು ಇಟ್ಟುಕೊಂಡಳು.

ಇದೊಂದು ಅರ್ಥಪೂರ್ಣವಾದ ಅದ್ಭುತ ಘಟನೆ. ಈ ಮೂಲಕ ಸ್ವಾಮೀಜಿ ತಮ್ಮ ಆಧ್ಯಾತ್ಮಿಕ ಪುತ್ರಿಗೆ ತಮ್ಮ ಅಂತಿಮ ಸಂದೇಶವನ್ನು ನೀಡಿದರೋ ಎಂಬಂತಿದೆ; ಅಥವಾ ತಾನು ಸಂನ್ಯಾಸಿ ಯಾಗುವ ಅವಳ ಉತ್ಕಟೇಚ್ಛೆಯನ್ನು ಅವರು ಸಾಂಕೇತಿಕವಾಗಿ ಈಡೇರಿಸಿದರೋ ಎಂಬಂತಿದೆ; ಅಥವಾ ‘ನೋಡು, ಸಂನ್ಯಾಸವೆಂದರೆ ಇದು–ತನ್ನತನವನ್ನು ನಿಶ್ಶೇಷವಾಗಿ ಆಹುತಿಯಾಗಿಸಿ ಕೊಳ್ಳುವುದು’ ಎಂದು ಅವಳಿಗೆ ತೋರಿಸಿಕೊಟ್ಟರೋ ಎಂಬಂತಿದೆ.

ಸಂಜೆಯ ವೇಳೆಗೆ ಅಗ್ನಿದೇವ ಸಂಪ್ರೀತನಾಗಿ ಶಾಂತನಾದ. ಅಂತೆಯೇ ಅಲ್ಲಿ ನೆರೆದಿದ್ದವ ರೆಲ್ಲರ ಮನದಲ್ಲೂ ಯಾವುದೋ ಅಪೂರ್ವ ಶಾಂತಿ ತಾನೇತಾನಾಗಿ ತುಂಬಿಕೊಂಡಿತು. ಅಗ್ನಿಯು ಪೂರ್ಣ ಶಮನವಾದಂತೆ ಸ್ವಾಮೀಜಿಯವರ ಶರೀರ ಪಂಚಭೂತಗಳಲ್ಲಿ ವಿಲೀನಗೊಂಡಿತು. ಬಳಿಕ ಅವರ ಸಂನ್ಯಾಸೀ ಸೋದರರು ತಂಪಾದ ಗಂಗಾಜಲವನ್ನು ಪ್ರೋಕ್ಷಿಸಿ ಚಿತಾಗ್ನಿಯನ್ನು ಶಮನಗೊಳಿಸಿದರು.

ಮರುದಿನ ಮಠದ ಸಂನ್ಯಾಸಿಗಳು ತಮಗಾಗಿ ಮತ್ತು ಮುಂದಿನ ಜನಾಂಗಕ್ಕಾಗಿ ಸ್ವಾಮೀಜಿ ಯವರ ಅಸ್ಥಿಯನ್ನು ಸಂಗ್ರಹಿಸಿಟ್ಟುಕೊಂಡರು. ಅವರ ಅಂತ್ಯಕ್ರಿಯೆ ನಡೆದ ಸ್ಥಳದಲ್ಲೀಗ ಸುಂದರ ಮಂದಿರವೊಂದು ತಲೆಯೆತ್ತಿ ನಿಂತಿದೆ. ಅದರ ಗರ್ಭಗುಡಿಯಲ್ಲಿ ಅಮೃತಶಿಲೆಯಿಂದ ನಿರ್ಮಿತವಾದ ಸ್ವಾಮೀಜಿಯವರ ಮೂರ್ತಿ ಕಂಗೊಳಿಸುತ್ತಿದೆ. ಸ್ವಾಮೀಜಿಯವರ ಅಸ್ಥಿಯ ಕೆಲಭಾಗವನ್ನು ಇಲ್ಲಿ ಇರಿಸಿದ್ದು, ಉಳಿದುದನ್ನು ಶ್ರೀರಾಮಕೃಷ್ಣರ ಗರ್ಭಗುಡಿಯಲ್ಲಿ ಇರಿಸ ಲಾಗಿದೆ.

ಜುಲೈ ಐದರಂದು, ಎಂದರೆ ಸ್ವಾಮೀಜಿಯವರು ಶರೀರವನ್ನು ತ್ಯಜಿಸಿದ ಮಾರನೆಯ ದಿನ ಬೆಳಿಗ್ಗೆ, ಇತ್ತ ಮದ್ರಾಸು ಮಠದಲ್ಲಿದ್ದ ಸ್ವಾಮಿ ರಾಮಕೃಷ್ಣಾನಂದರಿಗೆ ಬಾಗಿಲು ತಟ್ಟಿದ ಸದ್ದು ಕೇಳಿತು. “ಯಾರು?” ಎನ್ನುವಷ್ಟರಲ್ಲೇ ಒಂದು ಧ್ವನಿ ನುಡಿಯಿತು: “ಶಶಿ, ನಾನು ನನ್ನ ಶರೀರವನ್ನು ಉಗಿದುಬಿಟ್ಟೆ!” ರಾಮಕೃಷ್ಣಾನಂದರಿಗೆ ಈ ಧ್ವನಿಯ ಪರಿಚಯವಿದೆ–ಹೌದು! ಇದು ತಮ್ಮ ಪರಮಪ್ರಿಯ ಸೋದರ ವಿವೇಕಾನಂದರದೇ ಸರಿ! ಆದರೆ, ಏನಿರಬಹುದು ಇದರ ಅರ್ಥ?... ಕೆಲವು ಗಂಟೆಗಳಲ್ಲೇ ತಂತಿ ವಾರ್ತೆ ತಲುಪಿತು. ತಮ್ಮ ಪ್ರಿಯ ನಾಯಕ, ನೆಚ್ಚಿನ ನರೇನ್ ಇಹಲೋಕವನ್ನು ತ್ಯಜಿಸಿಬಿಟ್ಟಿದ್ದಾನೆ. ಇದ್ದಕ್ಕಿದ್ದಂತೆ ಈ ವಿಯೋಗ! ರಾಮಕೃಷ್ಣಾ ನಂದರು ಅಲ್ಲಿಂದಲೇ ತಮ್ಮ ಅಶ್ರುತರ್ಪಣವನ್ನು ಸಲ್ಲಿಸಿದರು.

ಕೆಲದಿನಗಳ ಬಳಿಕ ಸೋದರಿ ನಿವೇದಿತೆ ಜೋಸೆಫಿನ್ ಮೆಕ್​ಲಾಡಳಿಗೆ ಪತ್ರವೊಂದರಲ್ಲಿ ಬರೆಯುತ್ತಾಳೆ:

“ಈ ದಿನಗಳಲ್ಲಿ ಮಠವು ಶೋಕದ ವಾತಾವರಣದಿಂದ ಮತ್ತು ಪ್ರಾರ್ಥನೆಯಿಂದ ತುಂಬಿ ಹೋಗಿದೆ. ಸ್ವಾಮೀಜಿಯವರ ಅಂತ್ಯದ ದೃಶ್ಯ ಎಷ್ಟೊಂದು ದಿವ್ಯವಾಗಿತ್ತೆಂಬುದನ್ನು ಬಲ್ಲೆಯಾ? ಶತ್ರುಗಳೂ ಕೂಡ ಕಲೆತು ಪೂಜಿಸುವಷ್ಟು ದಿವ್ಯ! ಸಂಜೆಯ ದೀರ್ಘ ಧ್ಯಾನದ ನಂತರ ಶರೀರವನ್ನು ಅವರು ಜೀರ್ಣವಾದ ಬಟ್ಟೆಯಂತೆ ಎಸೆದ ರೀತಿಯು ಅದ್ಭುತವಲ್ಲವೆ? ‘ಹರ! ಹರ! ಹರ!–ಎನ್ನುತ್ತ ನಾನು ಮೃತ್ಯುವನ್ನಪ್ಪುತೇನೆ’ ಎಂದು ಅವರು ಹೇಳಿದ್ದ ಮಾತು ನನಗೆ ನೆನಪಿಗೆ ಬರುತ್ತದೆ. ಆ ಮಾತು ಸತ್ಯವೇ ಆಯಿತು. ಆದರೆ ಎಲ್ಲವೂ ವ್ಯವಸ್ಥಿತ ಗತಿಯಲ್ಲಿ ಸಾಗುತ್ತಿದ್ದಾಗ, ಅವರಿಗರ್ಪಿಸಿದ ಅಭಿನಂದನೆಗಳ ಮಾಲೆಯಿನ್ನೂ ಘಮಘಮಿಸುತ್ತಿರುವಾಗಲೇ, ಅವರು ನಿರ್ಗಮಿಸಿದರು. ಓ ಸ್ವಾಮೀಜಿ, ನನ್ನ ಪ್ರೀತಿಯ ಸ್ವಾಮೀಜಿ, ನಾನು ಕೇವಲ ನನ್ನ ಅನಿಸಿಕೆಗಳಿಗೇ ಅಂಟಿಕೊಳ್ಳದೆ ನಿಮ್ಮ ಅಂತರಂಗದ ಇಚ್ಛಾನುಸಾರ ನಡೆದುಕೊಳ್ಳುವಂತೆ ನನ್ನ ಮೇಲೆ ಅನುಗ್ರಹ ಮಾಡಿ!”

ಸ್ವಾಮೀಜಿಯವರು ಈ ಪ್ರಪಂಚವನ್ನು ಬಿಟ್ಟು ಹೊರಟುಹೋದರೆಂಬ ದುಃಖದ ಸಂಗತಿ ನಿಜವಾದರೂ, ಅವರು ತಮ್ಮ ಜೀವಿತಾವಧಿಯಲ್ಲೇ ನುಡಿದ ವಾಣಿಯೊಂದು ಘನಗಂಭೀರವಾಗಿ ಅನುರಣಿತವಾಗುತ್ತಿದೆ–‘ಈ ಶರೀರದ ಬಂಧನದಿಂದ ಹೊರಬರುವುದೇ ಮೇಲೆಂದು ನಾನು ಭಾವಿಸಬಹುದು. ಅಂತೆಯೇ ಅದನ್ನು ಜೀರ್ಣವಸ್ತ್ರವೆಂಬಂತೆ ಎಸೆಯಲೂ ಬಹುದು. ಆದರೆ ನಾನು ಹಿಡಿದ ಕಾರ್ಯವನ್ನು ಬಿಡುವವನಲ್ಲ. ಈ ಸಮಸ್ತ ಜಗತ್ತೂ ತಾನು ಪರಮಾತ್ಮನೊಡನೆ ಒಂದು ಎಂದು ಅರಿತುಕೊಳ್ಳುವವರೆಗೆ ನಾನು ಜಗದ ಜನರನ್ನು ಸ್ಫೂರ್ತಿಗೊಳಿಸುತ್ತ ಹೋಗು ತ್ತೇನೆ.’ ನಿಜ, ಇಂದು ನಾವು ಆ ಸ್ಫೂರ್ತಿಯನ್ನು ಕಾಣುತ್ತಿದ್ದೇವೆ. ಸಮಸ್ತ ಜಗತ್ತೂ ಪರಮಾತ್ಮ ನೊಡನೆ ಒಂದು ಎಂಬ ಸತ್ಯದ ಅರಿವಾಗುವವರೆಗೆ ಅವರ ಸ್ಫೂರ್ತಿಯು ಜನಮನದಲ್ಲಿ ಚಿರಸ್ಥಾಯಿಯಾಗಿ ಇರುತ್ತದೆ. ಅಷ್ಟೇ ಅಲ್ಲ, ತಮ್ಮ ವೈಯಕ್ತಿಕ ಮುಕ್ತಿಯನ್ನೇ ಕಡೆಗಣಿಸಿದವರು ವಿವೇಕಾನಂದರು. ವಿಶ್ವದ ಸಕಲ ಜೀವರನ್ನೂ ಭಗವಂತನ ಪಾದಪದ್ಮಗಳಲ್ಲಿ ಶರಣಾಗಿಸಿ ಎಲ್ಲರನ್ನೂ ಬಂಧನದಿಂದ ಮುಕ್ತರನ್ನಾಗಿಸುವ ವ್ರತ ತೊಟ್ಟವರು ವಿಶ್ವಮಾನವರಾದ ವಿವೇಕಾನಂದರು.

ರಾಮಕೃಷ್ಣ-ವಿವೇಕಾನಂದರು ಬಂದದ್ದು ಈ ಸಲಕ್ಕೇ ಮುಗಿಯಲಿಲ್ಲ. ಅವರು ಇನ್ನೂ ಆಗಾಗ ಬರುತ್ತಲೇ ಇರಬೇಕಾಗುತ್ತದೆ. ಬಂದು ಧರ್ಮವನ್ನು ಸುಸ್ಥಿತಿಗೆ ತಂದು, ಅದರ ಮರ್ಮವನ್ನು ಮನಗಾಣಿಸಬೇಕಾಗುತ್ತದೆ. ಮತ್ತು ಸಕಲರೂ ತಮ್ಮತಮ್ಮ ಆತ್ಮರಾಜ್ಯವನ್ನು ಪಡೆದುಕೊಳ್ಳು ವಂತೆ ಮಾಡಬೇಕಾಗುತ್ತದೆ. ಆ ಆತ್ಮರಾಜ್ಯ ಪ್ರಾಪ್ತಿಗೆ ಕೀಲಿಕೈ ಯಾವುದೆಂದರೆ–

\textbf{“ತ್ಯಾಗ ಮಾಡಿ! ತ್ಯಾಗ ಮಾಡಿ! ನಿಮ್ಮ ದಿವ್ಯ ಸ್ವರೂಪವನ್ನು ತಿಳಿಯಿರಿ! ಏಳಿ, ಎದ್ದೇಳಿ, ಗುರಿ ಮುಟ್ಟುವವರೆಗೂ ನಿಲ್ಲದಿರಿ!”}

\textbf{“ಉತ್ತಿಷ್ಠತ, ಜಾಗ್ರತ, ಪ್ರಾಪ್ಯವರಾನ್ನಿಬೋಧತ!”}

