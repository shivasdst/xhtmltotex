
\chapter{ಚಂದ್ರನಾಥ ಯಾತ್ರೆ}

\noindent

ಮಾಯಾವತಿಯ ಪ್ರವಾಸವು ಸ್ವಾಮೀಜಿಯವರಿಗೆ ಬಹಳ ಆನಂದವನ್ನು ಕೊಟ್ಟಿತು. ಮಠಕ್ಕೆ ಹಿಂದಿರುಗಿದ ಅವರು, ಅದ್ವೈತಾಶ್ರಮದ ಬಗ್ಗೆ ಮಾತನಾಡುತ್ತ ಅದನ್ನು ಬಾಯ್ತುಂಬ ಪ್ರಶಂಸಿಸಿ ದರು. ಆಶ್ರಮದ ಸುಂದರ ಪರಿಸರ, ಹಿಮಾಲಯದ ಪ್ರಶಾಂತ ಪರ್ವತಶ್ರೇಣಿಗಳಿಂದ ಆವೃತ ವಾದ ವಿಶೇಷ ನೋಟಗಳನ್ನು ಕೊಂಡಾಡಿದರು. ಶ್ರೀಮತಿ ಸೇವಿಯರರ ಸಹನೆ, ಶಕ್ತಿ ಹಾಗೂ ಶಾಂತಸ್ವಭಾವವನ್ನು ಮತ್ತು ಅವರು ತಮಗಿತ್ತ ಆದರದ ಆತಿಥ್ಯವನ್ನು ಬಹುವಾಗಿ ಶ್ಲಾಘಿಸಿದರು. ಅಲ್ಲದೆ ತಮ್ಮ ಶಿಷ್ಯರ ಪ್ರೀತಿಪೂರ್ವಕ ಸೇವೆಯಿಂದ ಅವರು ಸಂಪ್ರೀತರಾಗಿದ್ದರು.

ಆದರೆ ತಮ್ಮ ಆಪ್ತ ಶಿಷ್ಯನೂ ಸ್ನೇಹಿತನೂ ಆದ ಮಹಾರಾಜ ಅಜಿತ್​ಸಿಂಗನ ಆಕಸ್ಮಿಕ ನಿಧನದಿಂದಾಗಿ ಅವರ ಹೃದಯಕ್ಕಾಗಿದ್ದ ನೋವು ಬೇಗನೆ ಮಾಸಲಿಲ್ಲ. ಅದನ್ನು ಎಷ್ಟೇ ಮರೆಯಬೇಕೆಂದು ಪ್ರಯತ್ನಿಸಿದರೂ ನೆನಪು ಮತ್ತೆ ಮತ್ತೆ ಮರುಕಳಿಸುತ್ತಿತ್ತು. ಸುಮಾರು ನಾಲ್ಕು ತಿಂಗಳ ಬಳಿಕ ಮೇರಿಗೆ ಬರೆದ ಪತ್ರವೊಂದರಲ್ಲಿ ಅವರು ತಮ್ಮ ಶೋಕವನ್ನು ತೋಡಿಕೊಂಡಿ ದ್ದಾರೆ: “ಕೆಲ ತಿಂಗಳ ಹಿಂದೆ ಖೇತ್ರಿಯ ಮಹಾರಾಜ ಅಪಘಾತದಲ್ಲಿ ಮೃತನಾದ. ನೋಡು, ನನಗೀಗ ಎಲ್ಲವೂ ಮಬ್ಬುಗವಿದಂತಾಗಿದೆ... ”

ಈಗ ಸ್ವಾಮೀಜಿಯವರು ಮತ್ತೆ ಮಠದಲ್ಲಿ ತಮ್ಮ ಗುರುಭಾಯಿಗಳೊಂದಿಗೆ, ಶಿಷ್ಯರೊಂದಿಗೆ ಇರುವಂತಾಗಿದೆ. ಇದರಿಂದ ಸಹಜವಾಗಿಯೇ ಎಲ್ಲರಿಗೂ ಸಂತಸ, ಉತ್ಸಾಹ. ಸುಮಾರು ಒಂದೂವರೆ ವರ್ಷದ ಕಾಲ ಪಶ್ಚಿಮ ದೇಶಗಳ ವಾಸದಿಂದಾಗಿ ಮಠದಿಂದ ದೂರವಿದ್ದ ಅವರು ಭಾರತಕ್ಕೆ ಮರಳಿದ ಮೇಲೂ ಮಠದಲ್ಲಿದ್ದದ್ದು ಹದಿನೆಂಟೇ ದಿನಗಳು. ಅಷ್ಟರಲ್ಲೇ ಮಾಯಾ ವತಿಗೆ ತೆರಳಬೇಕಾಗಿ ಬಂದಿತು. ಈಗ ಅದೂ ಮುಗಿದು ಮಠದಲ್ಲಿದ್ದಾರೆ. ಆದರೆ ಅವರು ಮತ್ತಿನ್ನೆಲ್ಲಿಗೆ ಯಾವಾಗ ಹೊಡುತ್ತಾರೋ ಹೇಳಲು ಬರುವಂತಿರಲಿಲ್ಲ. ತಮ್ಮ ಮಧ್ಯೆ ಅವರನ್ನು ಸಾಧ್ಯವಾದಷ್ಟು ದಿನ ಉಳಿಸಿಕೊಳ್ಳಬೇಕೆಂಬುದೇ ಆಶ್ರಮವಾಸಿಗಳೆಲ್ಲರ ಇಚ್ಛೆ.

ಕಳೆದ ಒಂದೂವರೆ ವರ್ಷಗಳಲ್ಲಿ ಮಠವು ಸರ್ವಾಂಗೀಣ ಪ್ರಗತಿಯನ್ನು ಸಾಧಿಸಿತ್ತು. ಹಲವಾರು ಬಗೆಯ ತರಗತಿಗಳು ನಡೆಯುತ್ತಿದ್ದುವು. ಸ್ವಾಮೀಜಿಯವರಿಗೆ ಪ್ರಿಯವಾದ ವ್ಯಾಯಾಮವನ್ನು ಪ್ರಾರಂಭಿಸಲಾಗಿತ್ತು. ಧ್ಯಾನ ಜಪಗಳಿಗಾಗಿ ಸಮಯವನ್ನು ನಿಗದಿಗೊಳಿಸ ಲಾಗಿತ್ತು. ಮಠಕ್ಕೆ ಹೊಸ ಬ್ರಹ್ಮಚಾರಿಗಳು ಸೇರಿಕೊಂಡಿದ್ದರು. ಅಲ್ಲದೆ ಅವರ ಗುರುಭಾಯಿ ಗಳೂ ಶಿಷ್ಯರೂ ಅಧ್ಯಯನ, ಪ್ರವಚನ, ತರಬೇತಿ ಹಾಗೂ ಸೇವಾಕಾರ್ಯಗಳನ್ನು ಕೈಗೆತ್ತಿಕೊಂಡು ಚೆನ್ನಾಗಿ ನಡೆಸಿಕೊಂಡು ಬರುತ್ತಿದ್ದರು. ಇದನ್ನೆಲ್ಲ ಗಮನಿಸಿದ ಸ್ವಾಮೀಜಿ ಅತ್ಯಂತ ಸಂತುಷ್ಟ ರಾದರು.

ಕಲ್ಕತ್ತಕ್ಕೆ ಮರಳಿದ ಬಳಿಕ ಅವರು ಬಹುಕಾಲ ಯಾವ ಸಾರ್ವಜನಿಕ ಸಮಾರಂಭದಲ್ಲೂ ಭಾಗವಹಿಸಲಿಲ್ಲ. ಆದರೆ ಬೇಲೂರಿನ ಎಂ. ಇ. ಶಾಲೆಯಲ್ಲಿ ಬಹುಮಾನ ವಿತರಣಾ ಸಮಾ ರಂಭದ ಆಧ್ಯಕ್ಷತೆಯನ್ನು ವಹಿಸಲು ಒಪ್ಪಿಕೊಂಡರು. ಆ ಸಂದರ್ಭದಲ್ಲಿ ಮಾತನಾಡುತ್ತ ಅವರು ಶಿಕ್ಷಣಕ್ರಮದ ಕೆಲವು ದೋಷಗಳನ್ನು ಎತ್ತಿ ತೋರಿಸಿದರಲ್ಲದೆ, ಅವುಗಳ ಪರಿಹಾರೋಪಾಯ ವನ್ನೂ ಸೂಚಿಸಿದರು. ಅವರು ಹೇಳಿದರು: “ಆಧುನಿಕ ವಿದ್ಯಾರ್ಥಿಗಳು ಕಾರ್ಯಶೀಲರಾಗಿಲ್ಲ. ಅವರು ನಿಸ್ಸಹಾಯಕರು. ಇಂದಿನ ಇಂಗ್ಲಿಷ್ ವಿದ್ಯಾಭ್ಯಾಸ ಪದ್ಧತಿಯು ಕೇವಲ ಪುಸ್ತಕಜ್ಞಾನ ಪ್ರಧಾನವಾದದ್ದು. ನಮ್ಮ ವಿದ್ಯಾರ್ಥಿಗಳಿಗೆ ಕುಶಲಕಲೆಗಳನ್ನು, ಕೈಕೆಲಸವನ್ನು ಬೋಧಿಸಲಾಗು ತ್ತಿಲ್ಲ. ಆದ್ದರಿಂದ ಅವರಿಗೆ ತಮ್ಮ ಕಂಗಳನ್ನೂ ಕೈಗಳನ್ನೂ ಬಳಸುವುದು ಹೇಗೆಂಬುದೇ ತಿಳಿಯದು. ತಾವೇ ಸ್ವತಃ ಆಲೋಚಿಸುವುದು ಹೇಗೆ ಮತ್ತು ಅವರವರ ಕಾಲಿನ ಮೇಲೆ ನಿಂತು ಕೊಳ್ಳುವುದು ಹೇಗೆ ಎನ್ನುವುದನ್ನು ಅವರಿಗೆ ತಿಳಿಸಿಕೊಡಬೇಕು... 

“ಎರಡನೆಯದಾಗಿ, ಇಂದು ಲಕ್ಷಾಂತರ ವಿದ್ಯಾರ್ಥಿಗಳು ತುಂಬ ಕಳಪೆಯ ಆಹಾರದ ಮೇಲೆ ಜೀವಿಸುತ್ತಿದ್ದಾರೆ. ಇದು ಅತ್ಯಂತ ಶೋಚನೀಯ ಸಂಗತಿ. ವ್ಯಕ್ತಿಯ ಭವಿಷ್ಯದ ಆರೋಗ್ಯವು ಅವನ ಬಾಲ್ಯದ ಆರೋಗ್ಯವನ್ನು ಅವಲಂಬಿಸಿದೆ ಎನ್ನುವುದೇ ಜನರಿಗೆ ಮರೆತುಹೋಗಿದೆ. ಆದ್ದರಿಂದ, ಮಕ್ಕಳಿಗೆ ಸಾಧ್ಯವಾದಷ್ಟೂ ಪೌಷ್ಟಿಕಾಹಾರವನ್ನು ನೀಡುವ ಅಭ್ಯಾಸ ಬೆಳೆಯಬೇಕು.

“ಮೂರನೆಯದಾಗಿ, ನಮ್ಮಲ್ಲಿ ಶೀಲ-ಚಾರಿತ್ರ್ಯ ಎನ್ನುವುದು ಕಾಣೆಯಾಗುತ್ತಿದೆ. ಮಕ್ಕಳು ಸಭ್ಯನಡತೆ ಸದ್ವರ್ತನೆಗಳನ್ನೇ ಕಳೆದುಕೊಳ್ಳುತ್ತಿದ್ದಾರೆ. ಈಗೀಗ, ಸಭ್ಯವಾಗಿ ಮಾತನಾಡುವುದೇ ಗೌರವಕ್ಕೆ ಕಡಿಮೆ! ಹಿರಿಯರಿಗೆ ವಿಧೇಯರಾಗಿರುವುದೆಂದರೆ ಅವಮಾನಕರ! ಅವಿಧೇಯತೆಯೇ ಸ್ವಾತಂತ್ರ್ಯದ ಲಕ್ಷಣ!... ಇರಲಿ, ಆದರೂ ನಾನು ಹತಾಶನಾಗಿಲ್ಲ. ನನ್ನ ಜನಾಂಗದ ಬಗ್ಗೆ ನಾನು ಅಭಿಮಾನವನ್ನಿಟ್ಟುಕೊಂಡಿದ್ದೇನೆ. ಪ್ರತಿ ದಿನವೂ ಅದ್ಭುತ ಭವಿಷ್ಯವೊಂದರ ಚಿತ್ರವನ್ನು ಮುಂಗಾಣುತ್ತಿದ್ದೇನೆ....”

ಮಠಕ್ಕೆ ಮರಳಿದಾಗಿನಿಂದಲೂ ಸ್ವಾಮೀಜಿ ತಮ್ಮ ಮುಂದಿನ ಕಾರ್ಯಯೋಜನೆಗಳ ಬಗ್ಗೆ ಆಲೋಚಿಸುತ್ತಿದ್ದರು. ಬರುವ ಬೇಸಿಗೆಯಲ್ಲಿ ಶ್ರೀಮತಿ ಸೇವಿಯರರೊಂದಿಗೆ ಇಂಗ್ಲೆಂಡಿಗೆ ಹೋಗಿ, ವೇದಾಂತ ಪ್ರಚಾರ ಕಾರ್ಯಕ್ಕೆ ಹೊಸ ಶಕ್ತಿಯನ್ನು ತುಂಬಿ ಉದ್ದೀಪನಗೊಳಿಸುವ ಇಚ್ಛೆಯನ್ನು ಅವರು ಸಾರಾ ಬುಲ್​ಗೆ ಬರೆದ ಪತ್ರದಲ್ಲಿ ವ್ಯಕ್ತಗೊಳಿಸಿದರು. ದಕ್ಷಿಣ ಭಾರತ ಕ್ಕೊಮ್ಮೆ ಭೇಟಿ ನೀಡಬೇಕು ಎಂದು ಮುಂಬಯಿ ಹಾಗೂ ಇತರ ಸ್ಥಳಗಳಿಂದ ಒತ್ತಾಯಪೂರ್ವಕ ಆಹ್ವಾನಗಳು ಬರುತ್ತಿದ್ದುವು. ಅಲ್ಲದೆ ಪೂರ್ವಬಂಗಾಳದಿಂದಲೂ ಅವರಿಗೆ ಕರೆ ಬರುತ್ತಿತ್ತು. ಇವೆಲ್ಲದರೊಂದಿಗೆ, ಇನ್ನಾದರೂ ಕೆಲಕಾಲ ಅವರು ಮಠದಲ್ಲೇ ಉಳಿದುಕೊಳ್ಳಬೇಕು ಎಂಬ ಅವರ ಸಹಸಂನ್ಯಾಸಿಗಳ ಆಗ್ರಹಪೂರ್ವಕ ಬೇಡಿಕೆಯಂತೂ ಇದ್ದೇ ಇತ್ತು. ನಿಜಕ್ಕೂ ಕಾಯಿಲೆ ಗಳಿಂದಲೂ ಅತಿಯಾದ ಪರಿಶ್ರಮದಿಂದಲೂ ಜರ್ಜರಿತವಾಗಿದ್ದ ಅವರ ಶರೀರಕ್ಕೆ ದೀರ್ಘ ವಿಶ್ರಾಂತಿಯ ಅಗತ್ಯವಿತ್ತು. ಇದು ಅವರಿಗೂ ಚೆನ್ನಾಗಿಯೇ ತಿಳಿದಿತ್ತು. ಅಲ್ಲದೆ, ತಾವಿನ್ನು ನಿವೃತ್ತ ರಾಗುತ್ತೇವೆಂದು ಅವರೇ ಅದೆಷ್ಟು ಸಲ ತೀರ್ಮಾನ ಮಾಡಿದ್ದರೋ! ಆದರೂ ಅವರು ಈ ಎಲ್ಲ ಯೋಜನೆಗಳ ಬಗೆಗೂ ತೀವ್ರವಾಗಿ ಆಲೋಚಿಸುತ್ತಿದ್ದರು.

ಇವೆಲ್ಲದರೊಂದಿಗೆ ಅವರ ಮನಸ್ಸನ್ನು ಕೊರೆಯುತ್ತಿದ್ದ ಇನ್ನೊಂದು ಚಿಂತೆಯಿತ್ತು. ಅದು ತಮ್ಮ ವೃದ್ಧ ತಾಯಿಯನ್ನು ಕುರಿತದ್ದು. ಅವರು ಸಂನ್ಯಾಸಿಯಾಗಿ ಮನೆಯಿಂದ ಹೊರಟಾಗಿ ನಿಂದಲೂ, ತಾವು ತಮ್ಮ ಹೆತ್ತ ತಾಯಿಯ ಸುಖವನ್ನು ಬಲಿಯಾಗಿಸಿ ಈ ಕಾರ್ಯವನ್ನು ಕೈಗೊಂಡಿ ದ್ದೇವೆಂಬ ಅನಿಸಿಕೆ ಅವರ ಮನಸ್ಸಿನಲ್ಲಿ ಇದ್ದೇ ಇತ್ತು. ಆರ್ಥಿಕ ಸಂಕಷ್ಟದಲ್ಲಿ ಮುಳುಗಿದ್ದ ‘ತಮ್ಮ’ ಸಂಸಾರವು ಉಳಿದುಕೊಳ್ಳುವಂತಾಗಲು ಅವರು ಮಹಾರಾಜ ಅಜಿತ್​ಸಿಂಗನ ನೆರವನ್ನು ಪಡೆದಿದ್ದರು. ತನ್ಮೂಲಕ ಅವರ ಸೋದರರಾದ ಮಹೇಂದ್ರನಾಥ, ಭೂಪೇಂದ್ರನಾಥರಿಬ್ಬರೂ ವಿದ್ಯಾಭ್ಯಾಸವನ್ನು ಮುಂದುವರಿಸಲು ಸಾಧ್ಯವಾಗಿತ್ತು. ಆದರೆ ಬೆಳೆದುನಿಂತ ಹಿರಿಯ ಮಗನ ಆಸರೆ ಹಾಗೂ ಸಾಮೀಪ್ಯದ ಆನಂದ ಅತಿ ಆವಶ್ಯಕವಾಗಿದ್ದ ತನ್ನ ಇಳಿಗಾಲದಲ್ಲಿ ನರೇಂದ್ರ ತನ್ನಿಂದ ದೂರವಾಗಿದ್ದುದು ಭುವನೇಶ್ವರಿದೇವಿಗೊಂದು ದೊಡ್ಡ ಆಘಾತವೇ ಆಗಿತ್ತೆನ್ನಬೇಕು. ಲೋಕಕಲ್ಯಾಣಾರ್ಥವಾಗಿ ಹೊರಟ ತಮ್ಮ ವಿಷಯದಲ್ಲಿ ತಮ್ಮ ತಾಯಿ ಮಾಡಿದ ಈ ತ್ಯಾಗದ ಬೆಲೆ ಸ್ವಾಮೀಜಿಯವರಿಗೆ ಚೆನ್ನಾಗಿಯೇ ತಿಳಿದಿತ್ತು. ಆದ್ದರಿಂದ ತಾಯಿಯ ಕೊನೆಗಾಲದಲ್ಲಿ ತಾವು ಕೆಲಕಾಲವಾದರೂ ಜೊತೆಗಿದ್ದು ಆಕೆಯ ದಗ್ಧಹೃದಯಕ್ಕೆ ತಂಪನ್ನುಂಟುಮಾಡಬೇಕೆಂಬ ಇಚ್ಛೆ ಅವರ ಮನದಲ್ಲಿತ್ತು.

ಸ್ವಾಮೀಜಿಯವರ ಮನಸ್ಸಿನಲ್ಲಿ ಒಂದಾದಮೇಲೊಂದರಂತೆ ಏಳುತ್ತಿದ್ದ ತರಂಗಗಳ ರೂಪ ವನ್ನು, ಅವರು ತಮ್ಮ ಶಿಷ್ಯರಿಗೆ ಬರೆಯುತ್ತಿದ್ದ ಪತ್ರಗಳ ಮೂಲಕ ಕಲ್ಪಿಸಿಕೊಳ್ಳಬಹುದು. ತಮ್ಮ ಮುಂದಿದ್ದ ಹಲವಾರು ತುರ್ತು ಕಾರ್ಯಗಳ ಪೈಕಿ ಮೊದಲು ಯಾವುದನ್ನು ಕೈಗೆತ್ತಿಕೊಳ್ಳು ವುದೆಂದು ನಿರ್ಧರಿಸಲಾಗದಂತಹ ಪರಿಸ್ಥಿತಿ. ಒಂದು ಕಡೆ ತಮ್ಮ ಆರೋಗ್ಯದ, ಅಷ್ಟೇ ಅಲ್ಲ, ತಮ್ಮ ಉಳಿವು ಅಳಿವಿನ ಪ್ರಶ್ನೆ. ಇನ್ನೊಂದೆಡೆ ತಮ್ಮ ಕಾರ್ಯಭಾರದ ಸೆಳೆತ. ಮತ್ತೊಂದೆಡೆ ತಮ್ಮ ಮಾತೆಯನ್ನು ಕುರಿತ ಕರ್ತವ್ಯದ ಚಿಂತೆ. ಜೊತೆಗೆ ಆಪ್ತ ಶಿಷ್ಯ ಅಜಿತ್​ಸಿಂಗನ ಆಕಸ್ಮಿಕ ಸಾವಿನ ಆಘಾತ. ಇಷ್ಟು ಒತ್ತಡಗಳ ನಡುವೆಯೂ ಅವರ ಶಕ್ತಿ ಉತ್ಸಾಹಗಳು ಪುಟಿಯುವ ಪರಿ ಯನ್ನು ಕಂಡರೆ ಅತ್ಯಾಶ್ಚರ್ಯವಾಗುತ್ತದೆ.

೧೯ಂ೧ರ ಜನವರಿಯ ಅಂತ್ಯಕ್ಕೆ ಮಿಸ್ ಮೆಕ್​ಲಾಡ್ ಬೇಲೂರು ಮಠಕ್ಕೆ ಬಂದು ಸ್ವಾಮೀಜಿ ಯವರನ್ನು ಭೇಟಿಯಾದಳು. ಇವಳು ಹಾಗೂ ಇನ್ನಿತರರು ಈಜಿಪ್ಟಿನಲ್ಲಿ ಸ್ವಾಮೀಜಿಯವರನ್ನು ಬೀಳ್ಕೊಂಡದ್ದನ್ನು ಹಿಂದೆ ನೋಡಿದೆವು. ಈಜಿಪ್ಟಿನ ಪ್ರವಾಸವನ್ನು ಮುಗಿಸಿದ ಮಿಸ್ ಮೆಕ್ ಲಾಡ್, ಅಲ್ಲಿಂದ ತನ್ನ ಕೆಲವು ಆಂಗ್ಲ ಸ್ನೇಹಿತರೊಂದಿಗೆ ಜಪಾನ್ ದೇಶಕ್ಕೆ ಹೊರಟಿದ್ದಳು. ಜಪಾನಿಗೆ ಹೋಗುವ ಮಾರ್ಗದಲ್ಲಿ ಈಗ ಆಕೆ ಕಲ್ಕತ್ತಕ್ಕೆ ಬಂದಿದ್ದಳು. ಸ್ವಾಮೀಜಿಯವರನ್ನು ಭೇಟಿಯಾದ ಮೆಕ್​ಲಾಡಳು, ವೇದಾಂತ ಪ್ರಸಾರಕ್ಕಾಗಿ ಅವರು ಜಪಾನಿಗೆ ಹೋಗುವ ಸಾಧ್ಯತೆಯ ಬಗ್ಗೆ ಚರ್ಚಿಸಿದಳು. ಸ್ವಾಮೀಜಿಯವರಂತೂ ಆ ಬಗ್ಗೆ ತುಂಬ ಉತ್ಸಾಹ ತೋರಿದರು. ತಾನು ಜಪಾನಿಗೆ ಹೋದಮೇಲೆ ಮತ್ತೆ ಆ ಬಗ್ಗೆ ಪತ್ರ ಬರೆಯುವುದಾಗಿ ಹೇಳಿ ಮೆಕ್​ಲಾಡ್ ಹೊರಟಳು.

ಇತ್ತ ಬೇಲೂರು ಮಠದ ಆಡಳಿತವನ್ನು ವ್ಯವಸ್ಥಿತಗೊಳಿಸುವ ಕಾರ್ಯವಿನ್ನೂ ಬಾಕಿಯಿತ್ತು. ಸ್ವಾಮೀಜಿಯವರು ಜನವರಿ ೩ಂರಂದು ರಾಮಕೃಷ್ಣ ಮಠದ ಸಂಪೂರ್ಣ ಹಕ್ಕುಗಳನ್ನು ಹೊತ್ತ ಟ್ರಸ್ಟಿನ ರಚನೆಯನ್ನು ಪೂರ್ಣಗೊಳಿಸಿದರು. ಶ್ರೀರಾಮಕೃಷ್ಣರ ನೇರ ಶಿಷ್ಯರಾದ ಬ್ರಹ್ಮಾನಂದ, ಪ್ರೇಮಾನಂದ, ಶಿವಾನಂದ, ಶಾರದಾನಂದ, ಅಖಂಡಾನಂದ, ತ್ರಿಗುಣಾತೀತಾನಂದ, ರಾಮಕೃಷ್ಣಾ ನಂದ, ಅದ್ವೈತಾನಂದ, ಅಭೇದಾನಂದ, ಸುಬೋಧಾನಂದ ಹಾಗೂ ತುರೀಯಾನಂದ–ಇವರು ಗಳನ್ನು ಟ್ರಸ್ಟಿಗಳೆಂದು ನೇಮಿಸಲಾಯಿತು. ಸ್ವಾಮೀಜಿ ತಮ್ಮ ಹೊಣೆಗಾರಿಕೆಯನ್ನು ಸ್ವಲ್ಪ ಮಟ್ಟಿಗಾದರೂ ಇಳಿಸಿಕೊಳ್ಳುವ ಉದ್ದೇಶದಿಂದ ತಾವಾಗಿಯೇ ಈ ಟ್ರಸ್ಟಿನಿಂದ ಹೊರಗುಳಿದರು. ಅದ್ಭುತಾನಂದರನ್ನೂ ಟ್ರಸ್ಟಿಗಳಲ್ಲೊಬ್ಬರನ್ನಾಗಿ ಮಾಡಬೇಕೆಂದು ಸ್ವಾಮೀಜಿ ಇಚ್ಛಿಸಿದ್ದರು. ಆದರೆ ಅದ್ಭುತಾನಂದರು, “ನನಗದೆಲ್ಲ ಬೇಡವೇ ಬೇಡ; ದಯವಿಟ್ಟು ಆ ವಿಚಾರಗಳಲ್ಲಿ ನನ್ನ ಹೆಸರನ್ನು ತರಲೇಬೇಡಿ” ಎಂದುಬಿಟ್ಟರು. ಆಗ ಸ್ವಾಮೀಜಿ, “ನೀನೇನೂ ಮಾಡಬೇಕಾಗುವುದಿಲ್ಲ. ಸುಮ್ಮನೆ ನಿನ್ನ ಹೆಸರು ಮಾತ್ರ ಪಟ್ಟಿಯಲ್ಲಿರುತ್ತದೆ. ಆದ್ದರಿಂದ ನೀನು ಚಿಂತಿಸಬೇಕಾಗಿಲ್ಲ” ಎಂದರು. ಬ್ರಹ್ಮಾನಂದರೂ ಅವರನ್ನು ಒಪ್ಪಿಸಲು ನೋಡಿದರು. ಏನು ಮಾಡಿದರೂ ಅದ್ಭುತಾ ನಂದರು ಒಪ್ಪಲೇ ಇಲ್ಲ.

ಟ್ರಸ್ಟಿನ ದಸ್ತಾವೇಜಿನಲ್ಲಿ, ಟ್ರಸ್ಟಿಗಳು ತಮ್ಮಲ್ಲೇ ಒಬ್ಬರನ್ನು ಎರಡು ವರ್ಷದ ಅವಧಿಗೆ ಅಧ್ಯಕ್ಷರನ್ನಾಗಿ ಚುನಾಯಿಸುವಂತೆ ವ್ಯವಸ್ಥೆಯಾಗಿತ್ತು. ಫೆಬ್ರವರಿ ಹತ್ತರಂದು ಟ್ರಸ್ಟಿಗಳ ಸಭೆ ಯೊಂದು ನಡೆಯಿತು. ಅಧ್ಯಕ್ಷಸ್ಥಾನಕ್ಕೆ ಮೂರು ಹೆಸರುಗಳನ್ನು ಸೂಚಿಸಲಾಯಿತು; ಪ್ರೇಮಾ ನಂದರು ಬ್ರಹ್ಮಾನಂದರ ಹೆಸರನ್ನೂ ಶಾರದಾನಂದರು ರಾಮಕೃಷ್ಣಾನಂದರ ಹೆಸರನ್ನೂ ಬ್ರಹ್ಮಾ ನಂದರು ಶಾರದಾನಂದರ ಹೆಸರನ್ನೂ ಸೂಚಿಸಿದರು. ಈಗ ಮತಗಳನ್ನು ಹಾಕಲಾಯಿತು. ಸ್ವಾಮಿ ಬ್ರಹ್ಮಾನಂದರು ಅಧ್ಯಕ್ಷರಾಗಿ ಚುನಾಯಿತರಾದರು.

ಈ ಹಿಂದೆಯೇ ತಿಳಿಸಿದಂತೆ, ಬಾಲಿ ಟೌನಿನ ಪುರಸಭೆಯು ಮಠವನ್ನು ಒಂದು ಧಾರ್ಮಿಕ ಸಂಸ್ಥೆಯೆಂದು ಪರಿಗಣಿಸದೆ ದೊಡ್ಡ ಮೊತ್ತದ ತೆರಿಗೆ ಹಾಕಿತ್ತು. ೧೯ಂಂರ ಸೆಪ್ಟೆಂಬರಿನಲ್ಲಿ ಮಠದ ಪರವಾಗಿ ಮೊಕದ್ದಮೆ ಹೂಡಿ, ಮಠಕ್ಕೆ ತೆರಿಗೆಯಿಂದ ವಿನಾಯಿತಿ ನೀಡಬೇಕೆಂದು ಪ್ರಾರ್ಥಿಸಿಕೊಂಡಾಗ, ಜಿಲ್ಲಾ ನ್ಯಾಯಾಲಯವು ಮಠದ ಪರವಾಗಿ ತೀರ್ಪು ನೀಡಿತು. ಆದರೆ ಪುರಸಭೆಯು ಬಿಡದೆ, ಮೇಲಿನ ಕೋರ್ಟುಗಳಲ್ಲಿ ಮತ್ತೆಮತ್ತೆ ಮೊಕದ್ದಮೆ ಹೂಡಿತು. ಕಡೆಗೆ ೧೯ಂ೧ರ ಫೆಬ್ರುವರಿಯಲ್ಲಿ ಕಲ್ಕತ್ತ ಹೈಕೋರ್ಟು, ಮಠದ ವಾದವನ್ನೇ ಪುರಸ್ಕರಿಸಿದಾಗ ಆ ವಿಷಯ ಅಲ್ಲಿಗೆ ನಿಂತಿತು.

ಫೆಬ್ರುವರಿ ೨೪ ಭಾನುವಾರದಂದು ಬೇಲೂರು ಮಠದಲ್ಲಿ ಶ್ರೀರಾಮಕೃಷ್ಣರ ೬೪ನೆಯ ಜಯಂತ್ಯುತ್ಸವನ್ನು ಭಾರೀ ಪ್ರಮಾಣದಲ್ಲಿ ಆಚರಿಸಲಾಯಿತು. ಎಲ್ಲೆಡೆಗಳಿಂದ ಮೂವತ್ತು ಸಾವಿರಕ್ಕೂ ಹೆಚ್ಚು ಜನ ಸೇರಿದರು. ಹಲವಾರು ಭಜನ ಮಂಡಳಿಗಳು ಉತ್ಸವಕ್ಕೆ ವಿಶೇಷ ಕಳೆ ತಂದುಕೊಟ್ಟಿದ್ದುವು.

ಮಠದ ಆಡಳಿತಕ್ಕೆ ಸಂಬಂಧಿಸಿದ ತುರ್ತಿನ ಕೆಲಸಗಳನ್ನು ಮುಗಿಸಿದ ಸ್ವಾಮೀಜಿ ಈಗ ಮುಂದಿನ ಕಾರ್ಯಗಳನ್ನು ಕೈಗೊಳ್ಳಲು ಅಣಿಯಾದರು. ಆದರೆ ನಿಜಕ್ಕೂ ಅವರು ಹೇಗೆ ದಿನದಿನಕ್ಕೂ ಇಳಿದುಹೋಗುತ್ತಿದ್ದರೆಂಬುದನ್ನು ಅವರ ನಿಕಟವರ್ತಿಗಳೆಲ್ಲ ಸ್ಪಷ್ಟವಾಗಿ ಕಾಣು ತ್ತಿದ್ದರು. ದೈಹಿಕವಾಗಿಯೂ ಮಾನಸಿಕವಾಗಿಯೂ ದಣಿವುಂಟುಮಾಡುವಂತಹ ಯಾವುದೇ ಬಗೆಯ ಕೆಲಸವನ್ನೂ ಮಾಡಬಾರದೆಂದು ವೈದ್ಯರು ತೀವ್ರ ಎಚ್ಚರಿಕೆಯನ್ನಿತ್ತಿದ್ದರು. ಸ್ವಾಮೀಜಿ ಸಾಧ್ಯವಾದ ಮಟ್ಟಿಗೂ ಅವರ ಮಾತಿನಂತೆಯೇ ನಡೆಯುತ್ತಿದ್ದರು. ಆದರೆ ಈಗೀಗ ಸ್ವಲ್ಪ ಬಿಡುವು ಸಿಕ್ಕಿದಂತಾದ್ದರಿಂದ ತಮ್ಮ ದೇಹಸ್ಥಿತಿಯನ್ನು ಮರೆತು ಮತ್ತೆ ಕಾರ್ಯಮಗ್ನರಾದರು.

ದಕ್ಷಿಣಭಾರತಕ್ಕೂ ಇಂಗ್ಲೆಂಡಿಗೂ ಭೇಟಿ ನೀಡುವ ಕಾರ್ಯಕ್ರಮಗಳನ್ನು ಅವರು ತತ್ಕಾಲಕ್ಕೆ ರದ್ದುಪಡಿಸಿದರು. ಆದರೆ ಪೂರ್ವಬಂಗಾಳದಿಂದ ಬರುತ್ತಿದ್ದ ಆಹ್ವಾನವನ್ನು ನಿರಾಕರಿಸಲು ಅವರಿಗೆ ಸಾಧ್ಯವಾಗಲಿಲ್ಲ. ಆದ್ದರಿಂದ ಅವರು ಈ ಅವಕಾಶವನ್ನು ಉಪಯೋಗಿಸಿಕೊಂಡು ತಮ್ಮ ಪ್ರಿಯ ತಾಯಿಯನ್ನು ಪೂರ್ವಬಂಗಾಳದ ಹಾಗೂ ಅಸ್ಸಾಮಿನ ತೀರ್ಥಕ್ಷೇತ್ರಗಳಿಗೆ ಕರೆದೊಯ್ಯಲು ನಿಶ್ಚಯಿಸಿದರು. ಇತರ ಎಲ್ಲ ಕೆಲಸಗಳಿಗಿಂತಲೂ ತಮ್ಮ ಮಾತೆಯ ಮನಸ್ಸಿಗೆ ಸ್ವಲ್ಪ ಸಂತೋಷ ವನ್ನುಂಟುಮಾಡುವ ಕೆಲಸವೇ ಅವರ ಪಾಲಿನ ಆದ್ಯ ಕರ್ತವ್ಯವಾಯಿತು.

ಅವರ ಈ ತೀರ್ಥಯಾತ್ರೆಯಲ್ಲಿ ಮತ್ತೊಂದು ಉದ್ದೇಶವೂ ಅಡಗಿತ್ತು. ಪೂರ್ವ ಬಂಗಾಳವು (ಇಂದಿನ ಬಾಂಗ್ಲಾದೇಶ) ಬಂಗಾಳ ಪ್ರಾಂತದ ಒಂದು ಭಾಗವೇ ಆಗಿದ್ದರೂ ದಕ್ಷಿಣೇಶ್ವರದಲ್ಲಿ ಉದ್ಭವಿಸಿದ ಆಧ್ಯಾತ್ಮಿಕತೆಯ ಪ್ರವಾಹವು ಈ ಭಾಗವನ್ನಿನ್ನೂ ಮುಟ್ಟಿರಲಿಲ್ಲ. ಶ್ರೀರಾಮಕೃಷ್ಣರ ಹೆಸರು ಈ ಸ್ಥಳಗಳಲ್ಲಿ ಅಷ್ಟಾಗಿ ಪ್ರಚಾರವಾಗಿರಲಿಲ್ಲ. ಆದ್ದರಿಂದ ಆ ದಿವ್ಯ ಸಂದೇಶವನ್ನು ಪೂರ್ವ ಬಂಗಾಳದಲ್ಲಿ ಹರಡುವ ಉದ್ದೇಶ ಅವರದ್ದಾಗಿತ್ತು. ಅಲ್ಲದೆ, ಸ್ವತಃ ಸ್ವಾಮೀಜಿಯವರು ಭಾರತದ ಬಹುಭಾಗವನ್ನು ಸಂದರ್ಶಿಸಿದ್ದರೂ ಈ ಸಮೀಪದ ಸ್ಥಳವನ್ನೇ ವೀಕ್ಷಿಸಿರಲಿಲ್ಲ. ಕಡೆಯ ಬಾರಿ ನಾಗಮಹಾಶಯರನ್ನು ಭೇಟಿಯಾಗಿದ್ದಾಗ, ಅವರ ಊರಾದ ದೇವಭೋಗಕ್ಕೊಮ್ಮೆ ಬರುವುದಾಗಿ ಸ್ವಾಮೀಜಿ ಮಾತು ಕೊಟ್ಟಿದ್ದರು. (ನಾಗಮಹಾಶಯರು ಈಗ ತೀರಿಹೋಗಿದ್ದರೂ ಅವರ ಹೆಂಡತಿ ಅಲ್ಲಿಯೇ ಜೀವಿಸಿದ್ದರು.) ಈ ಎಲ್ಲ ಕಾರಣಗಳಿಂದಾಗಿ, ಅವರು ಪೂರ್ವ ಬಂಗಾಳ ಯಾತ್ರೆಗೆ ಉತ್ಸಾಹದಿಂದ ಸಿದ್ಧರಾಗಿ ನಿಂತರು.

ಮಾರ್ಚ್ ೧೮ರಂದು ಸ್ವಾಮೀಜಿ ತಮ್ಮ ಸಂನ್ಯಾಸೀ ಬಂಧುಗಳೊಡನೆ ಕಲ್ಕತ್ತವನ್ನು ಬಿಟ್ಟು ಹೊರಟರು. ಆದರೆ ಕಾರಣಾಂತರಗಳಿಂದ ಅವರ ತಾಯಿಗೆ ಅವರೊಂದಿಗೆ ಹೊರಡಲಾಗಲಿಲ್ಲ. ತನ್ನ ಸಂಗಡಿಗರೊಂದಿಗೆ ಒಂದು ವಾರದ ಬಳಿಕ ಕಲ್ಕತ್ತದಿಂದ ಹೊರಟು ಢಾಕಾದಲ್ಲಿ ಸ್ವಾಮೀಜಿ ಯವರ ತಂಡವನ್ನು ಕೂಡಿಕೊಂಡರು. ಕಲ್ಕತ್ತದಿಂದ ಹೊರಟ ಸ್ವಾಮೀಜಿ, ಗೋಯಲುಂಡೊ ಎಂಬಲ್ಲಿಗೆ ಟ್ರೈನಿನಲ್ಲಿ ಪ್ರಯಾಣಿಸಿ, ಅಲ್ಲಿಂದ ಸ್ಟೀಮರಿನಲ್ಲಿ ನಾರಾಯಣಗಂಜ್​ಗೆ ಹೋದರು. ಅಲ್ಲಿ ಅವರನ್ನು ಢಾಕಾದ ಸ್ವಾಗತ ಸಮಿತಿಯ ಕೆಲವು ಪ್ರತಿನಿಧಿಗಳು ಎದುರ್ಗೊಂಡು ಸ್ವಾಗತಿ ಸಿದರು. ಮಾರ್ಚ್ ೧೯ರ ಮಧ್ಯಾಹ್ನ ಟ್ರೈನು ಢಾಕಾ ತಲುಪಿದಾಗ ಅಲ್ಲಿ ನೆರೆದಿದ್ದ ಭಾರೀ ಜನಸ್ತೋಮವು “ಜೈ ರಾಮಕೃಷ್ಣದೇವ!” ಎಂಬ ಉದ್ಘೋಷದೊಂದಿಗೆ ಅವರನ್ನು ಬರಮಾಡಿ ಕೊಂಡಿತು. ನಗರದ ಹಲವಾರು ಕಾಲೇಜುಗಳ ವಿದ್ಯಾರ್ಥಿಗಳು ನಿಲ್ದಾಣದಲ್ಲಿ ನೆರೆದಿದ್ದರು. ನಾಗರಿಕರ ಪರವಾಗಿ ಪ್ರಸಿದ್ಧ ವಕೀಲರಾದ ಈಶ್ವರಚಂದ್ರ ಘೋಷ್ ಹಾಗೂ ಗಗನ ಚಂದ್ರ ಘೋಷ್ ಎಂಬ ಗಣ್ಯರು ಅವರನ್ನು ಆದರಿಂದ ಸ್ವಾಗತಿಸಿದರು. ಸುಸಜ್ಜಿತ ಸಾರೋಟಿನಲ್ಲಿ ಸ್ವಾಮೀಜಿಗಳನ್ನೊಳಗೊಂಡ ಬೃಹತ್ ಮೆರವಣಿಗೆಯೊಂದು ಢಾಕಾದ ರಸ್ತೆಗಳಲ್ಲಿ ಸಾಗಿ, ಬಾಬು ಮೋಹಿನೀಮೋಹನ ದಾಸ್ ಎಂಬ ಜಮೀನ್ದಾರರ ಬಂಗಲೆಯನ್ನು ಮುಟ್ಟಿತು. ಇಲ್ಲಿ ಸ್ವಾಮಿ ಗಳ ವಾಸ್ತವ್ಯಕ್ಕೆ ಏರ್ಪಾಡು ಮಾಡಲಾಗಿತ್ತು.

ಢಾಕಾದಲ್ಲಿ ಸ್ವಾಮೀಜಿಯವರನ್ನು ನೋಡಲು ಹಾಗೂ ಅವರ ಪ್ರವಚನಗಳನ್ನು ಆಲಿಸಲು ಪ್ರತಿದಿನವೂ ನೂರಾರು ಜನ ಸೇರುತ್ತಿದ್ದರು. ಮಾರ್ಚ್ ೨೫ರಂದು ಸ್ವಾಮೀಜಿಯವರ ತಾಯಿ, ತಂಗಿ, ಚಿಕ್ಕಮ್ಮ ಹಾಗೂ ಇನ್ನು ಕೆಲವರು ನಾರಾಯಣಗಂಜ್​ಗೆ ಬಂದರು. ಇಲ್ಲಿ ಅವರನ್ನು ಸ್ವಾಮೀಜಿಯ ಗುಂಪಿನವರು ಕೂಡಿಕೊಂಡರು. ಮರುದಿನ ಎಲ್ಲರೂ ಒಟ್ಟಿಗೆ ದೋಣಿಯಲ್ಲಿ ಹೊರಟು, ೨೭ರ ಮುಂಜಾನೆ ಲಂಗಲ್ ಬಂಧ ಕ್ಷೇತ್ರವನ್ನು ತಲುಪಿದರು. ಇದು ಬ್ರಹ್ಮಪುತ್ರ ನದಿಯ ದಡದಲ್ಲಿರುವ ಪುರಾಣ ಪ್ರಸಿದ್ಧ ಪುಣ್ಯಕ್ಷೇತ್ರ. ಚೈತ್ರಮಾಸದಲ್ಲಿ ಬರುವ ಅಶೋಕಾ ಷ್ಟಮಿಯ ದಿನದಂದು ಇಲ್ಲಿ ಮಿಂದರೆ, ಜಗತ್ತಿನ ಸಕಲ ತೀರ್ಥಗಳಲ್ಲೂ ಮಿಂದ ಫಲ ಎಂಬ ಪ್ರತೀತಿಯಿದೆ. ಅಂದು ಅಲ್ಲೊಂದು ಜಾತ್ರೆಯೂ ನಡೆಯುತ್ತದೆ. ಈ ಪುಣ್ಯಮುಹೂರ್ತದಲ್ಲಿ ನದೀಸ್ನಾನ ಮಾಡಲು ದೂರ ದೂರದ ಸ್ಥಳಗಳಿಂದಲೂ ಸಾವಿರಾರು ಜನ ಬಂದು ಸೇರುತ್ತಾರೆ. ಆ ವರ್ಷ ಅಲ್ಲಿ ಒಂದು ಲಕ್ಷಕ್ಕೂ ಹೆಚ್ಚು ಜನ ಸೇರಿದ್ದರು. ನದಿಯು ಮೈಲಿಗಟ್ಟಲೆ ಉದ್ದದ ವರೆಗೂ ದೋಣಿಗಳಿಂದ ತುಂಬಿಹೋಗಿತ್ತು. ಈ ಸ್ಥಳದಲ್ಲಿ ನದಿ ಸುಮಾರು ಒಂದು ಮೈಲಿ ಅಗಲವಿದೆ. ಆದರೂ ಇಡೀ ನದಿಯೇ ಕೆಸರಿನಿಂದ ರಾಡಿಯಾಗಿಬಿಟ್ಟಿತ್ತು.

ಆ ದಿನಗಳಲ್ಲಿ, ಅಸಂಖ್ಯಾತ ಜನ ಸೇರುವ ಇಂತಹ ಸಂದರ್ಭಗಳಲ್ಲೆಲ್ಲ ಅಂಟುಜಾಡ್ಯಗಳು ಹರಡುತ್ತಿದ್ದುದು ತೀರಾ ಸಾಮಾನ್ಯ. ಆಗೆಲ್ಲ ಚುಚ್ಚುಮದ್ದನ್ನು ಹಾಕಿಸಿಕೊಳ್ಳಬೇಕೆಂಬ ನಿಯಮವೂ ಇರಲಿಲ್ಲ. ಅಭ್ಯಾಸವೂ ಇರಲಿಲ್ಲ. ಹೀಗೆ ನೀರು ಕೆಸರಾಗಿದ್ದರೂ ಮತ್ತು ಅಂಟು ರೋಗಗಳ ಭಯವಿದ್ದರೂ ಸ್ವಾಮೀಜಿ ಅಂದು ನದಿಯಲ್ಲಿ ಸ್ನಾನ ಮಾಡಿದ್ದೊಂದು ಆಶ್ಚರ್ಯ ವೆಂಬಂತೆ ತೋರಬಹುದು. ಏಕೆಂದರೆ ಆರೋಗ್ಯ ಶುಚಿತ್ವಗಳ ವಿಷಯದಲ್ಲಿ ಅವರು ತುಂಬ ಕಟ್ಟುನಿಟ್ಟು. ಆದರೆ ಅವರಲ್ಲಿ ತೀರ್ಥಯಾತ್ರಿಕನ ಎಂತಹ ಸರಳ ಶ್ರದ್ಧೆಯಿತ್ತು ಎಂಬುದು ಸ್ವಾಮಿ ಶುದ್ಧಾನಂದರು ತಿಳಿಸುವ ಘಟನೆಯಿಂದ ಗೊತ್ತಾಗುತ್ತದೆ:

ಸ್ವಾಮೀಜಿ ತಮ್ಮ ತಂಡವರಿಗೆಲ್ಲ, ಸ್ನಾನದ ಸಮಯದಲ್ಲಿ ಒಂದು ಗುಟುಕು ನೀರನ್ನೂ ಕುಡಿಯಬಾರದೆಂದು ಮುನ್ನೆಚ್ಚರಿಕೆ ನೀಡಿದ್ದರು. ಬಳಿಕ ಢಾಕಾಗೆ ಹಿಂದಿರುಗಿದ ಮೇಲೆ ಕೇಳಿದರು, “ಯಾರಾದರೂ ನೀರನ್ನು ಕುಡಿದಿರಾ ಹೇಗೆ?” ಎಂದು. ಏಕೆಂದರೆ, ಎಷ್ಟೇ ಎಚ್ಚರಿಕೆ ಹೇಳಿದರೂ ಭಕ್ತಿಭಾವದ ವ್ಯಕ್ತಿಗಳು ಕದ್ದು ಮುಚ್ಚಿಯಾದರೂ ತೀರ್ಥಕ್ಷೇತ್ರದ ನೀರನ್ನು ಕುಡಿದಿರಬಹು ದಲ್ಲವೆ? ಅದೃಷ್ಟವಶಾತ್ ಯಾರೂ ಕುಡಿದಿರಲಿಲ್ಲ. ಸ್ವಾಮೀಜಿಯವರ ಮಾತನ್ನು ತಾವೆಲ್ಲ ಅಕ್ಷರಶಃ ಪಾಲಿಸಿದ್ದೇವೆಂಬ ಸಂತೃಪ್ತಿಯಿಂದ ಎಲ್ಲರೂ, “ಇಲ್ಲ, ಇಲ್ಲ; ನಾವ್ಯಾರೂ ನೀರು ಕುಡಿಯಲಿಲ್ಲ” ಎಂದರು. ಆಗ ಸ್ವಾಮೀಜಿ ನಸುನಗುತ್ತ ನುಡಿದರು, “ಆದರೆ ನಾನು ಮಾತ್ರ ಒಂದು ಗುಟುಕು ಕುಡಿದುಬಿಟ್ಟೆ!... ನೀರೊಳಗೆ ಮುಳುಗಿದ್ದಾಗ ನೀವೆಲ್ಲ ಕಾಣದ ಹಾಗೆ ಅಲ್ಲೇ ಒಂದು ಗುಟುಕು ಕುಡಿದೆ.” ಬಳಿಕ ನಗುತ್ತ ಹೇಳಿದರು, “ಇಲ್ಲದಿದ್ದರೆ ಪುಣ್ಯ ಸಿಕ್ಕೇ ಸಿಗುತ್ತದೆಂಬ ‘ಗ್ಯಾರಂಟಿ’ ಏನು?” ಇದನ್ನು ಕೇಳಿ ಎಲ್ಲರೂ ನಕ್ಕರು. ಜೊತೆಯಲ್ಲೇ ಅವರ ಸರಳ ಶ್ರದ್ಧೆಯನ್ನೂ ಗಮನಿಸಿದರು. ಇತರರ ಯೋಗಕ್ಷೇಮದ ಬಗ್ಗೆ ಅಷ್ಟೊಂದು ಎಚ್ಚರಿಕೆ ವಹಿಸುವ ಸ್ವಾಮೀಜಿ ತಾವು ಮಾತ್ರ ಆ ಅಪಾಯವನ್ನು ಎದುರಿಸಲು ಸಿದ್ಧರಾಗಿದ್ದರು! ಯಾವಾ ಗಲೂ ವೈಚಾರಿಕತೆಯ ಮಹತ್ವವನ್ನು ಎತ್ತಿ ಹಿಡಿಯುವ ಸ್ವಾಮೀಜಿಯವರು ಇಲ್ಲಿ ಶ್ರದ್ಧೆಯ ಮೂರ್ತಿಯಾಗಿ ಗೋಚರಿಸಿದರು.

ಪೂರ್ವ ಬಂಗಾಳದ ನಿಸರ್ಗ ಸೌಂದರ್ಯವನ್ನು ಕಂಡು ಸ್ವಾಮೀಜಿ ಉಲ್ಲಸಿತರಾದರು. ಸಾರಾಳಿಗೆ ಅವರೊಂದು ಪತ್ರದಲ್ಲಿ ಬರೆದರು–“ಅಂತೂ ಕಡೆಗೀಗ ನಾನು ಪೂರ್ವ ಬಂಗಾಳ ದಲ್ಲಿದ್ದೇನೆ. ನಾನಿಲ್ಲಿಗೆ ಬರುತ್ತಿರುವುದು ಇದೇ ಮೊದಲ ಸಲ. ಬಂಗಾಳ ಇಷ್ಟು ಸುಂದರ ವಾಗಿದೆಯೆಂದು ನನಗೆ ಈವರೆಗೂ ತಿಳಿದಿರಲಿಲ್ಲ. ನೀನು ಇಲ್ಲಿನ ನದಿಗಳನ್ನು ನೋಡಬೇಕು– ಶಾಂತವಾಗಿ ಉರುಳುವ ತಿಳಿ ನೀರಿನ ಸಾಗರಗಳಿವು! ಎಲ್ಲೆಲ್ಲೂ ಹಚ್ಚ ಹಸಿರು, ಪಚ್ಚೆಪಯಿರು....”

ಢಾಕಾದಲ್ಲಿ ಸ್ವಾಮೀಜಿಯವರು ಅನೇಕ ಪ್ರೇಕ್ಷಣೀಯ ಸ್ಥಳಗಳನ್ನೂ ಪುರಾತನ ದೇವಾ ಲಯಗಳನ್ನೂ ವೀಕ್ಷಿಸಿದರು. ಅವರು ಮನೆಯಲ್ಲೇ ಉಳಿದುಕೊಂಡಿದ್ದ ಅವಧಿಯಲ್ಲಿ ಅವರನ್ನು ನೋಡಲು ನೂರಾರು ಜನ ಬರುತ್ತಿದ್ದರು. ಒಂದು ದಿನ ಅಲ್ಲೊಂದು ಹೃದಯಸ್ಪರ್ಶಿ ಘಟನೆ ನಡೆಯಿತು. ಸರ್ವಾಭರಣಭೂಷಿತೆಯಾಗಿದ್ದ ವೇಶ್ಯೆಯೊಬ್ಬಳು ಸ್ವಾಮೀಜಿಯವರನ್ನು ನೋಡ ಲೆಂದು ತನ್ನ ತಾಯಿಯೊಂದಿಗೆ ಸಾರೋಟಿನಲ್ಲಿ ಬಂದಳು. ಸ್ವಾಮೀಜಿಯವರ ಆತಿಥೇಯನಾದ ಜತಿನ್​ಬಾಬು ಹಾಗೂ ಇತರರು ವಿಶೇಷ ಅತಿಥಿಗಳನ್ನು ಒಳಗೆ ಬಿಡಲು ಹಿಂಜರಿದರು. ಒಳಗಡೆಯಿದ್ದ ಸ್ವಾಮೀಜಿಯವರಿಗೆ ಈ ವಿಷಯ ತಿಳಿದಾಗ, ಅವರನ್ನು ತಮ್ಮ ಬಳಿಗೆ ಕಳಿಸುವಂತೆ ಹೇಳಿದರು. ತಾಯಿ ಮಗಳಿಬ್ಬರೂ ಸ್ವಾಮೀಜಿಯವರಿಗೆ ನಮಸ್ಕರಿಸಿ ಕುಳಿತುಕೊಂಡರು. ಬಳಿಕ ಆ ಯುವತಿ, “ಸ್ವಾಮೀಜಿ, ನಾನು ಉಬ್ಬಸರೋಗದಿಂದ ನರಳುತ್ತಿದ್ದೇನೆ. ದಯವಿಟ್ಟು ತಾವು ನನ್ನ ಈ ಕಾಯಿಲೆಗೆ ಏನಾದರೂ ಪರಿಹಾರ ಹೇಳಿ ನನ್ನನ್ನು ಕಾಪಾಡಬೇಕು” ಎಂದು ಬೇಡಿ ಕೊಂಡಳು. ಸ್ವಾಮೀಜಿ ಆಕೆಯ ಬಗ್ಗೆ ಮರುಕಗೊಂಡು, “ಅಮ್ಮಾ, ನಾನೂ ಕೂಡ ಅದೇ ಉಬ್ಬಸರೋಗದಿಂದ ನರಳುಳುತ್ತಿದ್ದೇನೆ! ನನಗೂ ಆ ಕಾಯಿಲೆಯನ್ನು ವಾಸಿ ಮಾಡಿಕೊಳ್ಳಲು ಸಾಧ್ಯವಾಗಿಲ್ಲ. ನಿನಗೆ ಯಾವ ರೀತಿಯಲ್ಲಾದರೂ ನೆರವಾಗಬೇಕೆಂಬ ಇಚ್ಛೆ ನನಗಿದೆ. ಆದರೆ ನಾನೇನು ಮಾಡಲಮ್ಮ” ಎಂದರು. ಅವರು ಹೃತ್ಪೂರ್ವಕ ವಿಶ್ವಾಸದಿಂದಾಡಿದ ಆ ಮಾತು ಆ ಇಬ್ಬರು ಮಹಿಳೆಯರಲ್ಲದೆ ಅಲ್ಲಿದ್ದವರೆಲ್ಲರ ಹೃದಯವನ್ನೂ ಮುಟ್ಟಿತು. ಬಳಿಕ ಇಬ್ಬರೂ ಸ್ವಾಮೀಜಿಯವರ ಆಶೀರ್ವಾದ ಪಡೆದುಕೊಂಡು ನಿರ್ಗಮಿಸಿದರು.

ಢಾಕಾದಲ್ಲಿ ಸ್ವಾಮೀಜಿಯವರನ್ನು ಭೇಟಿ ಮಾಡಿದವರಲ್ಲಿ ಕ್ರಾಂತಿಕಾರೀ ಮನೋಭಾವದ ಕೆಲವು ಯುವಕರಿದ್ದರು. ಇವರ ಪೈಕಿ, ಮುಂದೆ ಕ್ರಾಂತಿಕಾರಿಗಳ ನಾಯಕರಾದ ಹೇಮಚಂದ್ರ ಘೋಷ್, ಜೋಗೇಂದ್ರ ದತ್ತ ಮೊದಲಾದವರೂ ಇದ್ದರು. ಭಾರತವನ್ನು ಗುಲಾಮಗಿರಿಯ ನೊಗದಿಂದ ವಿಮುಕ್ತಗೊಳಿಸಲು ರಾಜಕೀಯ ಹತ್ಯೆಗಳನ್ನೂ ಮಾಡಲು ಸಿದ್ಧರಾಗಿದ್ದವರು ಇವರು. ವಿವೇಕಾನಂದರ ತಮ್ಮ ಡಾ ॥ ಭೂಪೇಂದ್ರನಾಥ ದತ್ತನೂ ಸಹ ಒಬ್ಬ ಕ್ರಾಂತಿಕಾರಿ. ಬಂಗಾಳದ ಕ್ರಾಂತಿಕಾರಿ ಚಳವಳಿಯ ಆಧಾರಪುರುಷರೆಂದರೆ ಸ್ವತಃ ವಿವೇಕಾನಂದರೇ ಎಂದು ಇವನು ಅಭಿಪ್ರಾಯಪಟ್ಟಿದ್ದಾನೆ. ಇವನು ತನ್ನ \eng{\textit{Swami Vivekananda: Patriot—Prophet}} ಎಂಬ ಪುಸ್ತಕದಲ್ಲಿ, ಈ ಕ್ರಾಂತಿಕಾರಿಗಳೆಲ್ಲ ವಿವೇಕಾನಂದರಿಂದ ಬಹಳಷ್ಟು ಪ್ರಭಾವಿತರಾಗಿದ್ದ ರೆಂದು ಹೇಳುತ್ತಾನೆ.

ಸ್ವಾಮೀಜಿಯವರು ಉಜ್ವಲ ರಾಷ್ಟ್ರಭಕ್ತರೂ ಸಮಾಜವಾದದ ಸಮರ್ಥಕರೂ ಆಗಿದ್ದ ರಾದರೂ, ನಿಜಕ್ಕೂ ಅವರು ಕ್ರಾಂತಿಕಾರಿಗಳ ಮಾರ್ಗವನ್ನು ಅನುಮೋದಿಸಿರಲಿಲ್ಲ. ಅವರ ಕಾರ್ಯಪ್ರಣಾಳಿಕೆಯು ತುಂಬ ಸಾತ್ವಿಕವೂ ವಿಭಿನ್ನವೂ ಆಗಿತ್ತು. ಅವರನ್ನು ಏಕಾಂತದಲ್ಲಿ ಭೇಟಿಯಾದ ಆ ಯುವಕರು, “ಸ್ವಾಮೀಜಿ, ಬಂಗಾಳದ ಉತ್ಸಾಹೀ ತರುಣರಿಂದ ನೀವೇನು ನಿರೀಕ್ಷಿಸುತ್ತೀರಿ?” ಎಂದು ಕೇಳಿದರು. ಆಗ ಸ್ವಾಮೀಜಿ ಅವರಿಗೆ ಹೇಳಿದರು: “ವ್ಯಕ್ತಿನಿರ್ಮಾಣವೇ ನನ್ನ ಧ್ಯೇಯ. ಬಂಕಿಮಚಂದ್ರರ ಕೃತಿಗಳನ್ನು ಮತ್ತೆ ಮತ್ತೆ ಓದಿ. ಅವರ ದೇಶಭಕ್ತಿಯನ್ನೂ ಸನಾತನ ಧರ್ಮದಲ್ಲಿ ಅವರಿಗಿರುವ ಶ್ರದ್ಧೆಯನ್ನೂ ಪಡೆದುಕೊಳ್ಳಿ. ನಿಮ್ಮೆಲ್ಲರಿಗೂ ನನ್ನ ಆಜ್ಞೆ ಇದು: ‘ಸಂಘಜೀವನ-ಸೇವಾವ್ರತ’. ನಿಮ್ಮ ಅಧ್ಯಯನಾದಿಗಳೊಂದಿಗೇ, ನಿಮ್ಮ ವಿದ್ಯಾಭ್ಯಾಸಕ್ಕೆ ಪೂರಕವಾಗಿ, ವಿನಯದಿಂದ ಭಕ್ತಿಯಿಂದ ಸಮಾಜಸೇವೆಯನ್ನು ಕೈಗೆತ್ತಿಕೊಳ್ಳಿ. ಏಕೆಂದರೆ ಜೀವನೇ ಶಿವ.” ಬಳಿಕ ಸ್ವಾಮೀಜಿ ದೈವೀಭಾವದೀಪ್ತಿಯಿಂದ ತೀವ್ರ ಕಳಕಳಿಯಿಂದ “ಮೃಗ ಗಳಂತೆ ಜೀವಿಸುತ್ತಿರುವ, ಮನುಷ್ಯನೆಂಬ ಹೆಸರನ್ನು ಮಾತ್ರ ಹೊತ್ತ ದೀನದಲಿತರ ಸೇವೆಯನ್ನು ಕೈಗೊಳ್ಳಿ” ಎಂದು ಆ ಯುವಕರನ್ನು ಒತ್ತಾಯಿಸಿದರು. ಮತ್ತೆ ಉಚ್ಚಸ್ವರದಲ್ಲಿ ಸಾರಿದರು: “ಎಲ್ಲ ಬಗೆಯ ಹಿಂದುಳಿದ ಬುದ್ಧಿ ಕೊನೆಗೊಳ್ಳಬೇಕು. ಅಸ್ಪೃಶ್ಯತೆಯೆಂಬ ಅತಿ ಘೋರ ಪಾಪವು ಮೊದಲು ತೊಲಗಬೇಕು. ಈ ಜಗತ್ತಿನಲ್ಲಿ ಅಸ್ಪೃಶ್ಯರೆಂಬವರು ಯಾರೂ ಇಲ್ಲ. ಅವರೆಲ್ಲ ನಾರಾಯಣರು.” ಈ ಯವಕರಿಗೆ ಸ್ವಾಮೀಜಿ ನಾಲ್ಕಂಶದ ಕಾರ್ಯಪ್ರಣಾಳಿಕೆಯನ್ನು ಹಾಕಿ ಕೊಟ್ಟರು–ಜನಸಾಮಾನ್ಯರೊಂದಿಗೆ ಬೆರೆಯುವಿಕೆ, ಅಸ್ಪೃಶ್ಯತಾ ನಿವಾರಣೆ, ವ್ಯಾಯಾಮಶಾಲೆಗಳ ನಿರ್ಮಾಣ ಮತ್ತು ವಾಚನಾಲಯ ಚಳವಳಿ.

ಸ್ವಾಮೀಜಿ ಢಾಕಾದಿಂದ ಹೊರಡುವ ದಿನ ಸಮೀಪಿಸಿದಂತೆ ಅಲ್ಲಿನ ನಾಗರಿಕರು, ವಿವೇಕಾ ನಂದರು ತಮ್ಮ ಪ್ರಾಂತಕ್ಕೆ ಭೇಟಿ ನೀಡಿದ ಈ ಅಪೂರ್ವ ಅವಕಾಶವನ್ನು ಕಳೆದುಕೊಳ್ಳದೆ ಅವರ ಭಾಷಣಗಳನ್ನು ಏರ್ಪಡಿಸಲು ಕಾತರರಾದರು. ಮಾರ್ಚ್ ೩ಂ ಹಾಗೂ ೩೧ರಂದು ಎರಡು ಸಮಾರಂಭಗಳು ನಡೆಯುವುದೆಂದು ನಿಶ್ಚಯವಾಯಿತು. ಮೊದಲನೆಯ ದಿನ ಜಗನ್ನಾಥ ಕಾಲೇಜಿ ನಲ್ಲಿ ಎರಡು ಸಾವಿರ ಜನರನ್ನುದ್ದೇಶಿಸಿ ಸ್ವಾಮೀಜಿಯವರು, “ನಾನೇನು ಕಲಿತಿದ್ದೇನೆ” ಎಂಬ ವಿಷಯವಾಗಿ ಇಂಗ್ಲಿಷಿನಲ್ಲಿ ಒಂದು ಗಂಟೆಯ ಕಾಲ ಮಾತನಾಡಿದರು. ಅದರ ಮುಖ್ಯಾಂಶಗಳು ಇಂತಿದ್ದುವು:

“ನಾನು ಈವರೆಗೂ ಕಣ್ಣಾರೆ ಕಂಡಿರದಿದ್ದ ಈ ಪೂರ್ವ ಬಂಗಾಳವನ್ನು ಸಂದರ್ಶಿಸುವ ಅವಕಾಶ ದೊರೆತದ್ದಕ್ಕಾಗಿ ನನಗೆ ತುಂಬ ಸಂತೋಷವಾಗಿದೆ. ಇಲ್ಲಿನ ಗಂಭೀರವಾಹಿನಿಗಳಾದ ನದಿಗಳು, ವಿಶಾಲವಾದ ಫಲವತ್ತಾದ ಬಯಲು ಪ್ರದೇಶಗಳು, ರಮಣೀಯವಾದ ಹಳ್ಳಿಗಳು– ಇವನ್ನೆಲ್ಲ ಕಂಡು ನನಗೆ ಮಹದಾನಂದವಾಗಿದೆ... 

“ಹಲವಾರು ವರ್ಷಗಳ ಹಿಂದೆಯೇ ನಾನೊಂದು ವಿಷಯವನ್ನು ಸ್ಪಷ್ಟವಾಗಿ ಕಂಡುಕೊಂಡೆ– ಅದೇನೆಂದರೆ, ಇಡೀ ಜಗತ್ತಿನಲ್ಲಿ ಹಿಂದೂಧರ್ಮವೇ ಅತ್ಯಂತ ಸಮಾಧಾನಕರವಾದ, ತೃಪ್ತಿಕರ ವಾದ, ಅತ್ಯುತ್ಕೃಷ್ಟವಾದ ಧರ್ಮ. ಆದರೆ ಈಗ ಭಾರತೀಯರಲ್ಲೇ ಎಲ್ಲೆಲ್ಲೂ ತಮ್ಮ ಧರ್ಮದ ಬಗ್ಗೆ ಅಸಡ್ಡೆಯುಂಟಾಗಿರುವುದು ಕಂಡುಬರುತ್ತಿದೆ. ಇದನ್ನು ಕಂಡು ನನಗೆ ನಾಚಿಕೆಯಾಗುತ್ತಿದೆ, ದುಃಖವಾಗುತ್ತಿದೆ. ಭಾರತೀಯರು ತಮ್ಮನ್ನು ಈ ಕಳಂಕದಿಂದ ಪಾರುಮಾಡಿಕೊಳ್ಳಬೇಕು. ಆದರೆ ಇತ್ತ ಸುಧಾರಕರೆನ್ನಿಸಿಕೊಂಡವರು ವಿವೇಚನೆಯಿಲ್ಲದೆ ಪಾಶ್ಚಾತ್ಯರನ್ನು ಅನುಕರಿಸಲು ನೋಡು ತ್ತಿದ್ದಾರೆ. ಹಿಂದೂಧರ್ಮವನ್ನು ತಲೆಕೆಳಗು ಮಾಡುವ ಯತ್ನದಲ್ಲಿದ್ದಾರೆ....

“ಇನ್ನು ಕೆಲವು ‘ಧರ್ಮರಕ್ಷಕ’ರೆನ್ನಿಸಿಕೊಂಡವರು ಪ್ರತಿಯೊಂದು ಹಿಂದೂ ಆಚಾರ-ಸಂಪ್ರ ದಾಯಕ್ಕೂ ವೈಜ್ಞಾನಿಕ ಸಮಜಾಯಿಷಿಯನ್ನು ಹುಡುಕುತ್ತಿದ್ದಾರೆ. ಇದಕ್ಕಿಂತ ಮೂರ್ಖತನ ಮತ್ತೊಂದಿರಲಾರದು. ಈ ‘ಧರ್ಮರಕ್ಷಕ’ರ ಬಾಯಲ್ಲಿ ಸದಾ ವಿದ್ಯುತ್ತು, ಅಯಸ್ಕಾಂತತ್ವ, ಕಂಪನಗಳು, ಅಲೆಗಳು–ಇಂಥದೇ ಶಬ್ದಗಳು! ಏನೋ, ಯಾರಿಗೆ ಗೊತ್ತು, ಒಂದು ದಿನ ಇವರು ದೇವರನ್ನೇ ‘ಕೇವಲ ವಿದ್ಯುತ್ ಕಂಪನಗಳ ಒಂದು ಮುದ್ದೆ’ ಎಂದು ವಿವರಣೆ ಕೊಟ್ಟರೂ ಆಶ್ಚರ್ಯವಿಲ್ಲ....

“ದೇವನನ್ನೂ ದೇಹವನ್ನೂ ಒಟ್ಟಿಗೆ ತೃಪ್ತಿಪಡಿಸಬೇಕು ಎಂದು ಮತ್ತೊಂದು ವರ್ಗವು ವಾದಿಸುತ್ತಿದೆ. ಆದರೆ ನೆನಪಿಡಿ: ‘ರಾಮನಿರುವೆಡೆ ಕಾಮನಿಲ್ಲ. ಕಾಮವಿರುವೆಡೆ ರಾಮನಿಲ್ಲ’.... ಹಿಂದೂಧರ್ಮದಿಂದ ನಾನು ಕಲಿತದ್ದು ಇದು:

\begin{verse}
ಮನುಷ್ಯತ್ವಂ ಮುಮುಕ್ಷುತ್ವಂ ಮಹಾಪುರುಷಸಂಶ್ರಯಃ ।\\ದುರ್ಲಭಂ ತ್ರಯಮೇವೈತತ್ ದೈವಾನುಗ್ರಹಹೇತುಕಮ್ ॥
\end{verse}

‘ಮನುಷ್ಯಜನ್ಮ, ಮುಕ್ತಿ ಗಳಿಸಬೇಕೆಂಬ ಆಕಾಂಕ್ಷೆ ಮತ್ತು ಮಹಾಪುರುಷರ ಸತ್ಸಂಗ– ಇವುಗಳು ಅತ್ಯಂತ ದುರ್ಲಭವಾದುವು. ಭಗವದನುಗ್ರಹದಿಂದ ಮಾತ್ರವೇ ಇವು ಮೂರೂ ಒಟ್ಟಿಗೆ ಲಭಿಸುತ್ತವೆ.’

“ಆಧ್ಯಾತ್ಮಿಕ ಸಾಧನೆಯಲ್ಲಿ ಗುರುವಿನ ಆವಶ್ಯಕತೆಯು ಮೂಲಭೂತವಾದದ್ದು, ಅನಿವಾರ್ಯ ವಾದದ್ದು. ಯಾರು ವೇದಗಳಲ್ಲಿ ಪಾರಂಗತನೋ, ಅಕಳಂಕನೋ, ಆಸೆಗಳಿಂದ ಪೀಡಿತನಾಗ ದವನೋ, ಬ್ರಹ್ಮಜ್ಞಾನಿಗಳಲ್ಲಿ ಅಗ್ರಗಣ್ಯನೋ, ಅವನೇ ನಿಜವಾದ ಗುರು ಎಂದು ಶ್ರುತಿಗಳು ಸಾರುತ್ತವೆ... 

“ಇಹಲೋಕವನ್ನೇ ನೆಚ್ಚಿಕೊಂಡಿರದೆ, ಒಳ್ಳೆಯದು ಕೆಟ್ಟದ್ದು–ಎರಡನ್ನೂ ಮೀರಿ ಸತ್​- ಚಿತ್​-ಆನಂದವನ್ನು ಸಾಕ್ಷಾತ್ಕರಿಸಿಕೊಳ್ಳುವುದೇ ಮಾನವಜನ್ಮದ ಗುರಿ.”

ಸ್ವಾಮೀಜಿಯವರ ಈ ಭಾಷಣವು ನಮಗಿಂದು ಪೂರ್ಣವಾಗಿ ದೊರೆತಿಲ್ಲ. ಆದರೂ ಇದರ ಧಾಟಿಯೂ ವಿಷಯವೂ ಅವರು ಭಾರತದಲ್ಲಿ ಮಾಡಿದ ಇತರೆಲ್ಲ ಭಾಷಣಗಳಿಗಿಂತ ವಿಭಿನ್ನ ವಾಗಿರುವುದು ಸ್ಪಷ್ಟವಾಗಿ ಗೋಚರವಾಗುತ್ತದೆ. ಇದರಲ್ಲಿ ಅವರು ಗುರುವಿನ ಮಹತ್ವದ ಬಗ್ಗೆ ಒತ್ತಿ ಹೇಳಿದ್ದಲ್ಲದೆ, ಗುರುವಿನಲ್ಲಿರಬೇಕಾದ ಗುಣಗಳ ಬಗ್ಗೆ ತಿಳಿಸುತ್ತಾರೆ. ಢಾಕಾದಲ್ಲಿ ಅವರಿ ಗಾಗಿದ್ದ ಕೆಲವು ಅನುಭವಗಳೇ ಇದಕ್ಕೆ ಕಾರಣವೆಂದು ಹೇಳಬಹುದು. ಪೂರ್ವ ಬಂಗಾಳದಲ್ಲಿ ಹಲವಾರು ಜನ ತಮ್ಮನ್ನು ತಾವು ಅವತಾರವೆಂದೂ ಬ್ರಹ್ಮಜ್ಞಾನಿಗಳೆಂದೂ ಘೋಷಿಸಿಕೊಳ್ಳ ತೊಡಗಿದ್ದರು. ಇವರಲ್ಲಿ ಒಬ್ಬೊಬ್ಬರಿಗೂ ಒಂದಷ್ಟು ಮಂದಿ ಅನುಯಾಯಿಗಳೂ ಇದ್ದರು. ಅಲ್ಲದೆ, ಇಲ್ಲಿನ ಬಹುತೇಕ ಜನ ಕಂದಾಚಾರಗಳಲ್ಲೇ ಮುಳುಗಿಹೋದವರು. ಇವರು ಸ್ವಾಮೀಜಿ ಯವರ ಸಂದೇಶಗಳನ್ನೂ ಮಾತುಕತೆ-ವರ್ತನೆಗಳನ್ನೂ ಅರ್ಥಮಾಡಿಕೊಳ್ಳಲು ಅಸಮರ್ಥರಾಗಿ ದ್ದರು. ಇಲ್ಲಿನ ಒಟ್ಟಾರೆ ಧಾರ್ಮಿಕ ಪರಿಸ್ಥಿತಿ ಸ್ವಾಮೀಜಿಯವರಿಗೆ ಸಮಾಧಾನವನ್ನುಂಟು ಮಾಡಲಿಲ್ಲ. ಒಂದು ದಿನ ಒಬ್ಬ ಯುವಕ ತನ್ನ ‘ಗುರು’ವಿನ ಚಿತ್ರವನ್ನು ಅವರಿಗೆ ತೋರಿಸಿ, “ಸ್ವಾಮೀಜಿ, ಇವರು ಅವತಾರಪುರುಷರೆ? ಇವರ ಬಗ್ಗೆ ನೀವೇನು ಹೇಳುತ್ತೀರಿ?” ಎಂದು ಕೇಳಿದ. ಆಗ ಸ್ವಾಮೀಜಿ, “ಇವರ್ಯಾರೋ ನನಗೆ ಗೊತ್ತಿಲ್ಲಪ್ಪ. ಇವರು ಅವತಾರ ಪುರುಷರೋ ಅಲ್ಲವೋ ಎನ್ನುವುದೂ ನನಗೆ ತಿಳಿಯದು” ಎಂದರು. ಆದರೆ ಆ ಯುವಕ ಬಿಡದೆ, ಮತ್ತೆ ಮತ್ತೆ ಅದೇ ಪ್ರಶ್ನೆಯನ್ನು ಹಾಕುತ್ತ ಕಾಡಲಾರಂಭಿಸಿದ. ಸ್ವಲ್ಪ ಹೊತ್ತು ಅವನ ಕಾಟವನ್ನು ಸಹಿಸಿಕೊಂಡ ಸ್ವಾಮೀಜಿ ಕಡೆಗೆ ಗಂಭೀರವಾಗಿ ಹೇಳಿದರು, “ಮಗು, ನೀನು ಇನ್ನುಮೇಲೆ ಸ್ವಲ್ಪ ಪೌಷ್ಟಿಕ ಆಹಾರವನ್ನು ತೆಗೆದುಕೊಂಡು ನಿನ್ನ ಮೆದುಳನ್ನು ಬಲಪಡಿಸಿಕೊಳ್ಳುವುದು ಒಳ್ಳೆಯದು. ಅದರ ಕೊರತೆಯಿಂದಾಗಿ ನಿನ್ನ ಮೆದುಳು ಒಣಗಿಹೋಗಿರುವಂತೆ ಕಾಣುತ್ತಿದೆ.” ಹೀಗೆ ಸ್ವಲ್ಪ ಖಾರ ವಾಗಿಯೇ ಹೇಳಿ ಅವನಿಂದ ಪಾರಾಗಬೇಕಾಯಿತು.

ಕಲ್ಕತ್ತಕ್ಕೆ ಹಿಂದಿರುಗಿದ ಮೇಲೆ ಅವರು ಢಾಕಾದಲ್ಲಿನ ತಮ್ಮ ಅನುಭವಗಳ ಬಗ್ಗೆ ಶರಚ್ಚಂದ್ರ ನಿಗೆ ತಿಳಿಸುತ್ತ ಹೇಳುತ್ತಾರೆ, “ಜನಗಳು ತಮ್ಮ ಗುರುವನ್ನು ಅವತಾರವೆಂದು ಕರೆಯಬಹುದು ಅಥವಾ ಬೇರೆ ಏನು ಬೇಕಾದರೂ ಕರೆಯಬಹುದು. ಆದರೆ ಭಗವಂತ ಎಲ್ಲೆಂದರಲ್ಲಿ ಯಾವಾ ಗೆಂದರಾವಾಗ ಅವತಾರವೆತ್ತುವುದಿಲ್ಲ. ಢಾಕಾ ನಗರದಲ್ಲೇ ಮೂರುನಾಲ್ಕು ಅವತಾರಗಳಿದ್ದಾರೆ ಎಂದು ನಾನು ಕೇಳಿದೆ!”

ಈ ಹಿನ್ನೆಲೆಯಲ್ಲಿ ನೋಡಿದಾಗ ಸ್ವಾಮೀಜಿಯವರ ಭಾಷಣದ ವಿಷಯವೂ ಅದರ ಧಾಟಿಯೂ ವಿಭಿನ್ನವಾಗಿದ್ದುದೇಕೆ ಎನ್ನುವುದು ತಿಳಿಯುತ್ತದೆ.

ಮರುದಿನ, ಎಂದರೆ ೩೧ನೇ ತಾರೀಕು ಭಾನುವಾರದಂದು ಸ್ವಾಮೀಜಿ ಪೋಗೋಸ್ ಹೈಸ್ಕೂ ಲಿನ ಮೈದಾನದಲ್ಲಿ ಭಾರೀ ಸಭೆಯನ್ನುದ್ದೇಶಿಸಿ ಸುಮಾರು ಎರಡು ಗಂಟೆಗಳ ಕಾಲ ಮಾತನಾಡಿ ದರು. ವಿಷಯ: “ನಾವು ಜನ್ಮತಳೆದ ಧರ್ಮ.” ಈ ಭಾಷಣದ ಬಗ್ಗೆಯೂ ನಮಗೆ ತಿಳಿದುಬಂದಿರು ವುದು ಎಲ್ಲೋ ಸ್ವಲ್ಪ ಮಾತ್ರವೇ. ಆದರೆ ಈ ಭಾಷಣದಲ್ಲಿ ಅವರು, ವೇದಾಧ್ಯಯನವನ್ನು ಕಡೆಗಣಿಸಿದ್ದರಿಂದ ಉಂಟಾದ ಅಧಃಪತನದ ಬಗ್ಗೆ ಖಾರವಾಗಿ ಮಾತನಾಡಿದರೆಂದು ತಿಳಿದು ಬರುತ್ತದೆ. ಅದರ ಕೆಲವು ಮುಖ್ಯಾಂಶಗಳು ಇವು:

“ಇದೇ ಭಾರತದಲ್ಲಿ ನಮ್ಮ ಪುರಾತನ ಮಹರ್ಷಿಗಳಲ್ಲನೇಕರು ಪರಮಸತ್ಯವನ್ನು ಸಾಕ್ಷಾ ತ್ಕರಿಸಿಕೊಂಡರೆಂಬುದನ್ನು ನೆನಪಿಡಿ. ಆದರೆ ಕಳೆದುಹೋದ ‘ರಾಮರಾಜ್ಯ’ವನ್ನೇ ಮೆಲುಕು ಹಾಕುತ್ತ ಕುಳಿತಿರಬೇಡಿ. ಬದಲಾಗಿ ನೀವೂ ಆ ಮಹರ್ಷಿಗಳಂತೆಯೇ ಆಗಲು ಪ್ರಯತ್ನಿಸಿ. ಯಾವನು ವಿಧಿಬದ್ಧ ಸಾಧನೆಗಳ ಮೂಲಕ ಧರ್ಮವನ್ನು ಸಾಕ್ಷಾತ್ಕರಿಸಿಕೊಳ್ಳಲು ಸಮರ್ಥನಾಗಿ ದ್ದಾನೆಯೋ ಅವನು ಹುಟ್ಟಿನಿಂದ ಮ್ಲೇಚ್ಛನೇ ಆಗಿದ್ದರೂ ಒಬ್ಬ ಪುಷಿಯೇ ಸರಿ. ಅನೈತಿಕ ಸಂಬಂಧದಿಂದ ಜನಿಸಿದ ವಸಿಷ್ಠರು, ಬೆಸ್ತರವಳ ಮಗನಾದ ವ್ಯಾಸರು, ದಾಸೀಪುತ್ರರಾದ ಮತ್ತು ತನ್ನ ತಂದೆ ಯಾರೆಂಬುದೇ ತಿಳಿಯದ ನಾರದರು–ಇವರೇ ಮೊದಲಾದವರೆಲ್ಲ ಪುಷಿಪದವಿ ಗೇರಿದ್ದೇ ಇದಕ್ಕೆ ಪುರಾವೆ... 

“ಹಿಂದೂಧರ್ಮದ ಏಕಮಾತ್ರ ಪ್ರಮಾಣಗ್ರಂಥವಾದ ವೇದವನ್ನು ಓದಲು ಪ್ರತಿಯೊಬ್ಬ ರಿಗೂ ಅಧಿಕಾರವಿದೆ. ವೇದಗಳ ಮೇಲೆ ಕೆಲವು ಜಾತಿಗಳವರಿಗೆ ಮಾತ್ರ ಸಂಪೂರ್ಣ ಸ್ವಾಮ್ಯವಿದೆ ಎಂದು ಕೆಲವು ಪುರಾಣಗಳಲ್ಲಿ ಹೇಳಿದೆಯಾದರೂ ಅವು ಪ್ರಮಾಣವಾಕ್ಯಗಳಲ್ಲ. ಆ ಮಾತುಗಳೇ ಅಂತಿಮವಲ್ಲ. ಏಕೆಂದರೆ ವೇದಗಳಲ್ಲಿ ಎಲ್ಲಿಯೂ ಹಾಗೆ ಹೇಳಿಲ್ಲ. ಸೇವಕನು ತನ್ನ ಯಜಮಾನ ನಿಗೇ ಆಜ್ಞೆ ಮಾಡಬಲ್ಲನೆ? (ಶ್ರುತಿಯೆಂದು ಕರೆಯಲ್ಪಡುವ ವೇದಗಳ ಮಾತೇ ಹಿಂದೂಧರ್ಮದ ಅತ್ಯುನ್ನತ ಪ್ರಮಾಣ, ಮತ್ತು ಇತರ ಯಾವುದೇ ಶಾಸ್ತ್ರಗ್ರಂಥವಾದರೂ ಶ್ರುತಿಗಳಿಗೆ ಅಧೀನ ವಾಗಿರುವವರೆಗೆ ಮಾತ್ರ ಸ್ವೀಕಾರಾರ್ಹವೆನಿಸುತ್ತದೆ.)

“ಬಂಗಾಳದಲ್ಲಿ ವೇದಗಳ ಅಧ್ಯಯನವೇ ನಿಂತುಹೋಗಿದೆ. ಎಷ್ಟೋ ಜನ ಪಂಡಿತರಿಗೂ ವೇದಗಳ ವಿಷಯಗಳು ಚೆನ್ನಾಗಿ ತಿಳಿದಿಲ್ಲ... ಸ್ತ್ರೀಯರೂ ಶೂದ್ರರೂ ವೇದಗಳನ್ನು ಓದು ವುದರಿಂದ ಬಹಿಷ್ಕೃತರಾಗಿಲ್ಲವೆಂಬುದನ್ನು ನಾನು ಸಾಬೀತುಮಾಡಿ ತೋರಿಸಕೊಡಬಲ್ಲೆ. ನಾನು ಪುರಾಣಾದಿಗಳನ್ನು ಹೀಗಳೆಯುತ್ತಿಲ್ಲ. ಆದರೆ ಅವು ವೇದಗಳಿಗೆ ಅಧೀನವೆಂಬುದನ್ನು ನೆನಪಿಸು ತ್ತಿದ್ದೇನೆ, ಅಷ್ಟೆ. ತಂತ್ರಶಾಸ್ತ್ರಗಳೂ ಕೂಡ ಅನೇಕರು ಭಾವಿಸಿರುವಷ್ಟು ಕೆಟ್ಟವೇನಲ್ಲ. ಆದರೆ ಅವು ವೇದವಾಕ್ಯಗಳಿಗೆ ಅಧೀನವಾಗಿರುವವರೆಗೆ ಮಾತ್ರ ಅವುಗಳಿಗೆ ಬೆಲೆ.”

ಬಳಿಕ ಅವತಾರಗಳ ಬಗ್ಗೆ ಪ್ರಸ್ತಾಪಿಸಿದ ಸ್ವಾಮೀಜಿ ನುಡಿದರು–“ಮಾನವರಲ್ಲಿ ದೇವರನ್ನು ನೋಡುವುದೇ ನಿಜವಾದ ಭಗವದ್ದರ್ಶನ. ಇದೇ ಅವತಾರವಾದದ ನಿಜಾರ್ಥ... ಮೂರ್ತಿ ಪೂಜೆಯು ಎಲ್ಲಕ್ಕಿಂತ ಕೆಳಹಂತದ ಪೂಜೆಯೆಂದು ಶಾಸ್ತ್ರಗಳೇ ತಿಳಿಸುವುವಾದರೂ ಮೂರ್ತಿ ಪೂಜೆ ತಪ್ಪಂತೂ ಅಲ್ಲ. ಆ ಪರಮಸಂಪ್ರದಾಯಸ್ಥ ವಿಗ್ರಹಾರಾಧಕ ಬ್ರಾಹ್ಮಣನ (ಶ್ರೀರಾಮ ಕೃಷ್ಣ ಪರಮಹಂಸರ) ಪವಿತ್ರ ಪಾದಧೂಳಿಯಿಂದ ಧನ್ಯನಾಗಿರದಿದ್ದಲ್ಲಿ ಈಗ ನಾನೆಲ್ಲಿರುತ್ತಿದ್ದೆ!... ”

ತಮ್ಮ ದೀರ್ಘ ಉಪನ್ಯಾಸದ ಕೊನೆಯಲ್ಲಿ ಸ್ವಾಮೀಜಿಯವರು ಸಮಾಜ ಸುಧಾರಕರೆನ್ನಿಸಿ ಕೊಂಡವರಿಗೆ, ಕ್ರಿಯಾತ್ಮಕವಾಗಿ-ಸಕಾರಾತ್ಮಕವಾಗಿ ಮುಂದುವರಿಯುವಂತೆ ಮತ್ತು ‘ವಿಧ್ವಂಸಕ’ ವಿಧಾನವನ್ನು ತ್ಯಜಿಸುವಂತೆ ಕರೆ ನೀಡಿದರು. ಅಲ್ಲದೆ ಬ್ರಾಹ್ಮಣರನ್ನುದ್ದೇಶಿಸಿ ಕಟುವಾಗಿ ನುಡಿದರು: “ನಾನು ಹೇಳುತ್ತೇನೆ–ನಿಮ್ಮ ಜಾತ್ಯಭಿಮಾನವೂ ಕುಲಾಭಿಮಾನವೂ ಕೇವಲ ವ್ಯರ್ಥ! ಆ ಅರ್ಥಹೀನ ದುರಭಿಮಾನವನ್ನಿನ್ನು ಕೊಡವಿಹಾಕಿ. ಶತಮಾನಗಳ ದೀರ್ಘಕಾಲ ಮ್ಲೇಚ್ಛರ ಆಡಳಿತಕ್ಕೊಳಪಟ್ಟಿರುವ ನಿಮ್ಮಲ್ಲಿ ನಿಮ್ಮ ಶಾಸ್ತ್ರಗಳ ಪ್ರಕಾರ, ಬ್ರಾಹ್ಮಣತ್ವವೆಂಬುದು ಇನ್ನೇನೂ ಉಳಿದಿಲ್ಲ... ನಿಮ್ಮ ಪೂರ್ವಿಕರ ಮಾತಿನಲ್ಲಿ ನಿಮಗೇನಾದರೂ ಶ್ರದ್ಧೆಯಿದ್ದುದೇ ಆದರೆ, ಈ ಕ್ಷಣವೇ ಹೋಗಿ; ಕುಮಾರಿಲಭಟ್ಟನಂತೆ ತುಷಾಗ್ನಿಯನ್ನು (ತುಷ=ಹೊಟ್ಟು; ತುಷಾಗ್ನಿ =ಸಣ್ಣ ಉರಿ) ಪ್ರವೇಶಿಸಿ ಪ್ರಾಯಶ್ಚಿತ್ತ ಮಾಡಿಕೊಳ್ಳಿ! ಅದನ್ನು ಮಾಡಲು ನಿಮಗೆ ಧೈರ್ಯ ಸಾಲದೆ? ಹಾಗಾದರೆ ನಿಮ್ಮ ದೌರ್ಬಲ್ಯವನ್ನು ಒಪ್ಪಿಕೊಳ್ಳಿ. ಮತ್ತು ಜ್ಞಾನಭಂಡಾರದ ಬಾಗಿಲನ್ನು ಪ್ರತಿ ಯೊಬ್ಬರಿಗೂ ತೆರೆಯಲು ನಿಮ್ಮ ಸಹಾಯಹಸ್ತವನ್ನು ಚಾಚಿ. ತುಳಿತಕ್ಕೊಳಗಾದ ಜನಸಮೂಹಕ್ಕೆ ನ್ಯಾಯಬದ್ಧವಾದ ಹಕ್ಕುಗಳನ್ನೂ ಸೌಲಭ್ಯಗಳನ್ನೂ ಮತ್ತೊಮ್ಮೆ ದೊರಕಿಸಿಕೊಡಿ.”

ಈ ಎರಡೂ ಭಾಷಣಗಳನ್ನು ಜನರು ಪ್ರಚಂಡ ಕರತಾಡನದಿಂದ ಸ್ವಾಗತಿಸಿ ಶ್ಲಾಘಿಸಿದರು. ಸ್ವಾಮೀಜಿಯವರ ಸಂದೇಶಗಳು ನೂರಾರು ಜನರ ಮೇಲೆ ಆಳವಾದ ಪರಿಣಾಮವನ್ನುಂಟು ಮಾಡಿದುವು. ಅದರಲ್ಲೂ ಯುವಜನರು, ಗುರುವಿನ ಅರ್ಹತೆಗಳ ಹಾಗೂ ಕರ್ತವ್ಯದ ಬಗ್ಗೆ ಅವರು ಹೇಳಿದ್ದನ್ನು ಕೇಳಿ ಅವರ ಮಾರ್ಗದರ್ಶನವನ್ನು ಕೋರಿ ಅವರನ್ನು ಸುತ್ತುವರಿಯ ಲಾರಂಭಿಸಿದರು. ಆದರೆ ಅವರ ಮಾತುಗಳ ತೀಕ್ಷ್ಣತೆಯನ್ನು ಸಹಿಸಲಾರದ ಕೆಲವರು, ಮುಖ್ಯವಾಗಿ ಸಂಪ್ರದಾಯಸ್ಥ ಪಂಡಿತರು, ಸಿಡಿಮಿಡಿಗೊಂಡರು. ಏನೇ ಆದರೂ ಒಟ್ಟಾರೆಯಾಗಿ ಅವರ ಮಾತುಗಳು ಹಿಂದೂ ಸಮಾಜದ ಮೇಲೊಂದು ಸತ್ಪರಿಣಾಮವನ್ನೇ ಉಂಟುಮಾಡಿದುವು.

ಢಾಕಾದಲ್ಲಿದ್ದ ದಿನಗಳಲ್ಲೇ ಸ್ವಾಮೀಜಿಯವರು ಸಂತ ನಾಗಮಹಾಶಯರ ಊರಾದ ದೇವ ಭೋಗಕ್ಕೆ ಭೇಟಿ ನೀಡಿದರು. ನಾಗಮಹಾಶಯರ ಬಗ್ಗೆ ಸ್ವಾಮೀಜಿಯವರಿಗಿದ್ದ ಭಕ್ತಿ-ಗೌರವ ಅಸದೃಶವಾದದ್ದು. ಅವರ ಬಗ್ಗೆ ಮಾತನಾಡುವ ಸಂದರ್ಭ ಬಂದಾಗಲೆಲ್ಲ ಭಾವಾವೇಶಭರಿತ ರಾಗದಿರಲು ಸ್ವಾಮೀಜಿಯವರಿಗೆ ಸಾಧ್ಯವೇ ಆಗುತ್ತಿರಲಿಲ್ಲ. ಒಮ್ಮೆ ಅವರು, “ನಾನು ಭಾಷಣ ಮಾಡಲು ಆ ದೇವಭೋಗದ ಪ್ರಾಂತಕ್ಕೆ ಮಾತ್ರ ಹೋಗುವುದಿಲ್ಲ. ಅದು ನಾಗಮಹಾಶಯರಿದ್ದ ಸ್ಥಳ” ಎಂದು ಉದ್ಗರಿಸಿದರು. ಆಗ ಅಲ್ಲಿದ್ದವರೊಬ್ಬರು, “ಆದರೆ ನಾಗಮಹಾಶಯರು ಮಾತನಾಡಿದ್ದೇ ತೀರ ಅಪರೂಪ!” ಎಂದರು. ಅದಕ್ಕೆ ಸ್ವಾಮೀಜಿ ಗಂಭೀರವಾಗಿ ಉತ್ತರಿಸಿ ದರು–“ನಾಗಮಹಾಶಯರಂಥವರು ಮಾತನ್ನೇ ಆಡಬೇಕಾಗಿಲ್ಲ. ಅವರ ಪರಮಪವಿತ್ರ ಚಿಂತನ ತರಂಗಗಳೇ ದೇಶದ ಆಲೋಚನಾ ವಿಧಾನದ ಮೇಲೆ ಎಷ್ಟೋ ಪ್ರಭಾವವನ್ನು ಬೀರಬಲ್ಲುವು.”

ನಾಗಮಹಾಶಯರ ಪತ್ನಿ ಠಾಕೂರಣಿಯವರು ಸ್ವಾಮೀಜಿಯವರನ್ನು ಸ್ವಾಗತಿಸಿದರು. ಕೈಯಾರೆ ತಯಾರಿಸಿದ ಭಕ್ಷ್ಯಗಳ ಭೋಜನವನ್ನಿಕ್ಕಿ ಉಪಚರಿಸಿದರು. ಅವುಗಳನ್ನು ತಿನ್ನುತ್ತ ಸ್ವಾಮೀಜಿ ಒಂದು ಪುಟ್ಟ ಮಗುವಿನಂತಾಗಿಬಿಟ್ಟರು. ಅಲ್ಲಿನ ವಾತಾವರಣವೇ ಅವರಲ್ಲೊಂದು ಅಪೂರ್ವ ಆನಂದವನ್ನುಂಟುಮಾಡಿತು. ಆ ಮನೆಯು ಅವರಿಗೊಂದು ‘ಶಾಂತಿ ಆಶ್ರಮ’ದಂತೆ ತೋರಿತು. ಅವರು ಊರ ಕೆರೆಯಲ್ಲಿ ಸ್ವೇಚ್ಛೆಯಾಗಿ ಈಜಾಡಿದರು. ಅನಂತರ ನಾಗಮಹಾಶಯರ ಮನೆಗೆ ಹಿಂದಿರುಗಿ ಬಂದು ಸ್ವಲ್ಪ ವಿಶ್ರಮಿಸಿಕೊಳ್ಳಲು ಮಲಗಿ, ಸುಮಾರು ಎರಡು ಗಂಟೆ ಕಾಲ ಸುಖವಾಗಿ ನಿದ್ರಿಸಿದರು. ಮುಂದೆ ಅವರು ಈ ಬಗ್ಗೆ ಮಾತನಾಡುತ್ತ, “ನಾನು ನನ್ನ ಜೀವನದಲ್ಲಿ ಗಾಢವಾಗಿ ನಿದ್ರಿಸಿದ ಕೆಲವೇ ದಿನಗಳಲ್ಲಿ ಅದೂ ಒಂದು” ಎನ್ನುತ್ತಾರೆ.

ಸ್ವಾಮೀಜಿ ಅಲ್ಲಿಂದ ಹೊರಟು ನಿಂತಾಗ ಠಾಕೂರಣಿಯವರು ಅವರಿಗೆ ಕಾಷಾಯವಸ್ತ್ರ ವೊಂದನ್ನು ಅರ್ಪಿಸಿದರು. ಅದನ್ನು ಸ್ವಾಮೀಜಿ ತಮ್ಮ ತಲೆಗೆ ಪೇಟದಂತೆ ಕಟ್ಟಿಕೊಂಡು ಢಾಕಾದ ಕಡೆಗೆ ಹೊರಟರು.

ನಾಗಮಹಾಶಯರ ಮನೆಯಲ್ಲಿ ಅವರ ಭಾವಚಿತ್ರವೊಂದನ್ನು ಇಟ್ಟು ಪೂಜಿಸಲಾಗುತ್ತಿತ್ತು. ಇದನ್ನು ಕಂಡು ಸ್ವಾಮೀಜಿ ಬಹಳ ಸಂತೋಷಪಟ್ಟರು. ಆದರೆ ಅವರಂತಹ ಮಹಾಮಹಿಮರು ಬದುಕಿದ್ದ ತಾಣವನ್ನು ಅಲ್ಲಿನವರು ಇನ್ನೂ ಚೊಕ್ಕಟವಾಗಿ, ಸುಂದರವಾಗಿ ಉಳಿಸಿಕೊಳ್ಳಬೇಕೆಂದು ಇಚ್ಛಿಸಿದರು. ಅಲ್ಲದೆ ಪೂರ್ವಬಂಗಾಳದ ಜನರು, ಆ ಸ್ಥಳವನ್ನೆಲ್ಲ ಪುನೀತಗೊಳಿಸಿದ ಈ ಸಂತನ ಜೀವನ-ಸಂದೇಶಗಳನ್ನು ಅಧ್ಯಯನ ಮಾಡಬೇಕು, ಅವರ ಮಾಹಾತ್ಮ್ಯವನ್ನು ಅರಿಯಬೇಕು ಎಂದೂ ಸ್ವಾಮೀಜಿ ಇಚ್ಛಿಸಿದರು.

ಸ್ವಾಮೀಜಿ ಢಾಕಾದಲ್ಲಿದ್ದಾಗ ರಮೇಶಚಂದ್ರ ದತ್ತ ಎಂಬವರಿಂದ ಒಂದು ಪತ್ರ ಬಂದಿತು. ಅವರು ಭಾರತದಲ್ಲಿ ಐ. ಸಿ. ಎಸ್. ಅಧಿಕಾರಿಯಾಗಿದ್ದು ನಿವೃತ್ತರಾದ ಮೇಲೆ ಇಂಗ್ಲೆಂಡಿನಲ್ಲಿ ನೆಲಸಿದ್ದರು. ಅವರು ಇಂಗ್ಲೆಂಡಿನಲ್ಲಿ ನಿವೇದಿತೆಯ ಕಾರ್ಯವನ್ನು ಶ್ಲಾಘಿಸಿ ಹಾಗೂ ಇನ್ನಷ್ಟು ಕಾಲ ಆಕೆ ಅಲ್ಲಿಯೇ ಉಳಿದುಕೊಳ್ಳುವಂತೆ ಸಲಹೆ ಮಾಡಿ ಆ ಪತ್ರ ಬರೆದಿದ್ದರು. ಅವರಿಗೆ ಕೃತಜ್ಞತೆಗಳನ್ನರ್ಪಿಸುತ್ತ ಸ್ವಾಮೀಜಿ ಹೀಗೆ ಬರೆದರು–

“ಸೋದರಿ ನಿವೇದಿತೆಯು ಇಂಗ್ಲೆಂಡಿನಲ್ಲಿ ಮಾಡುತ್ತಿರುವ ಸತ್ಕಾರ್ಯದ ಬಗ್ಗೆ ನಿಮ್ಮಂತಹ ಉಚ್ಚ ಸ್ಥಾನದ ವ್ಯಕ್ತಿಯೊಬ್ಬರಿಂದ ತಿಳಿಯಲು ನನಗೆ ಬಹಳ ಸಂತೋಷವಾಗುತ್ತದೆ. ಭಾರತಕ್ಕೆ ಆಕೆಯ ಮೂಲಕ ಆಗಬಹುದಾದ ಸೇವೆಯ ಬಗ್ಗೆ ನೀವು ವ್ಯಕ್ತಪಡಿಸಿರುವ ಆಶಯದಲ್ಲಿ ನಾನೂ ಸಹಭಾಗಿಯಾಗಿದ್ದೇನೆ... ಮಾನ್ಯರೇ, ನನ್ನ ಶಿಶುವಿಗೆ ನಿಮ್ಮ ಸ್ನೇಹವನ್ನು ನೀಡಿದ್ದಕ್ಕಾಗಿ ನಾನು ನಿಮಗೆ ಅತ್ಯಂತ ಪುಣಿಯಾಗಿದ್ದೇನೆ. ಆಕೆ ಎಲ್ಲಿಯವರೆಗೆ ಇಂಗ್ಲೆಂಡಿನಲ್ಲಿರಬೇಕು ಮತ್ತು ಆಕೆ ತನ್ನ ಕಾರ್ಯವನ್ನು ಹೇಗೆ ಮುಂದುವರಿಸಬೇಕು ಎಂಬುದರ ಬಗ್ಗೆ ನೀವು ಹೀಗೆಯೇ ಆಕೆಗೆ ಮಾರ್ಗದರ್ಶನ ನೀಡುವಿರೆಂದು ಆಶಿಸುತ್ತೇನೆ.”

ಅದೇ ದಿನ ಅವರು ನಿವೇದಿತೆಗೂ ಒಂದು ಪತ್ರ ಬರೆದರು–

“ನಿನ್ನನ್ನೂ, ಇಂಗ್ಲೆಂಡಿನಲ್ಲಿನ ನಿನ್ನ ಕೆಲಸವನ್ನೂ ತುಂಬ ಶ್ಲಾಘಿಸುವ ಶ್ರೀ ದತ್ತರ ಪತ್ರ ವೊಂದು ಈಗ ತಾನೆ ಬಂದಿತು. ಅಲ್ಲದೆ, ನೀನು ಮತ್ತಷ್ಟು ದಿನ ಅಲ್ಲಿಯೇ ಉಳಿದುಕೊಳ್ಳಬೇಕು ಎಂದು ನಿನಗೆ ತಿಳಿಸುವಂತೆ ನನಗವರು ಬರೆದಿದ್ದಾರೆ. ನಿನ್ನ ಕೆಲಸ ಚೆನ್ನಾಗಿ ಸಾಗುತ್ತಿದೆಯೆಂದು ನಿನಗೆ ಅನ್ನಿಸುತ್ತಿರುವವರೆಗೂ ನೀನು ಖಂಡಿತವಾಗಿ ಅಲ್ಲಿ ಉಳಿದುಕೊಳ್ಳಬಹುದು... ”

ಏಪ್ರಿಲ್ ೨೬ರಂದು ಸ್ವಾಮೀಜಿ ತಮ್ಮ ಅನುವರ್ತಿಗಳೊಂದಿಗೆ ಚಂದ್ರನಾಥ ಕ್ಷೇತ್ರಕ್ಕೆ ಹೊರಟರು. ಸ್ವಾತಂತ್ರ್ಯಪೂರ್ವದಲ್ಲಿ ಇದು ಅಖಿಲ ಭಾರತಮಟ್ಟದ ತೀರ್ಥಕ್ಷೇತ್ರಗಳಲ್ಲೊಂದು. ಕಲಿಯುಗದಲ್ಲಿ ತಾನು ಇಲ್ಲಿ ನೆಲಸುವುದಾಗಿ ಶಿವನು ವ್ಯಾಸಮಹರ್ಷಿಗಳಿಗೆ ಭರವಸೆ ನೀಡಿದ್ದ ನೆಂದು ಸ್ಥಳಪುರಾಣ. ರೈಲು ಹಾಗೂ ಸ್ಟೀಮರುಗಳಲ್ಲಿ ನೂರು ಮೈಲಿಗೂ ಹೆಚ್ಚಿನ ದೂರವನ್ನು ಕ್ರಮಿಸಿ ಬಂಗಾಳದ ತುತ್ತತುದಿಯಲ್ಲಿದ್ದ ಚಂದ್ರನಾಥಕ್ಕೆ ಬಂದರು. ಚಂದ್ರನಾಥ ದೇವಸ್ಥಾನ ಇಲ್ಲಿನ ಕಡಿದಾದ ಬೆಟ್ಟದ ಮೇಲಿದೆ. ಇಲ್ಲಿಗೆ ಹೋಗಲು ಏಳುನೂರಕ್ಕೂ ಹೆಚ್ಚು ಮೆಟ್ಟಿಲುಗಳನ್ನು ಹತ್ತಬೇಕಾಗಿತ್ತು. ತಮ್ಮ ಆಗಿನ ದೇಹಸ್ಥಿತಿಯಲ್ಲಿ ಸ್ವಾಮೀಜಿ ಅಷ್ಟೊಂದು ಶ್ರಮವನ್ನು ಹೇಗೆ ತಾಳಿಕೊಂಡರೋ ತಿಳಿಯದು. ದಾರಿಯಲ್ಲಿ ಹಲವಾರು ಕುಂಡಗಳಿದ್ದು ಅವರು ಅವುಗಳಲ್ಲೆಲ್ಲ ಸ್ನಾನ ಮಾಡಿದರು. ಅವರ ಯಾತ್ರಿಕನಿಷ್ಠೆಯನ್ನು ನಾವು ಹಿಂದೆಯೇ ನೋಡಿದ್ದೇವೆ. ಬೆಟ್ಟದ ಮೇಲಿನ ದೇವಾಲಯವನ್ನು ತಲುಪಿದಾಗ ಅಲ್ಲಿ ಮಹಾದೇವನ ದರ್ಶನದಿಂದ ಒಂದು ದಿವ್ಯ ಧನ್ಯತಾಭಾವ ಅವರಲ್ಲುದಿಸಿತು. ಆ ಎತ್ತರದಿಂದ ಸುತ್ತಲೂ ಕಣ್ಣು ಹಾಯಿಸಿದರೆ ಮನಮೋಹಕ ನೋಟ! ದೂರದಲ್ಲಿ ದಿಗಂತದವರೆಗೂ ಹರಡಿಕೊಂಡಿರುವ ಬಂಗಾಳ ಕೊಲ್ಲಿಯನ್ನೂ ನೋಡ ಬಹುದು.

ಚಂದ್ರನಾಥದಿಂದ ಸ್ವಾಮೀಜಿಯವರು ಅಸ್ಸಾಮಿನ ರಾಜಧಾನಿ ಗೌಹಾತಿಯ ಬಳಿಯ ಕಾಮಾಖ್ಯ ಕ್ಷೇತ್ರಕ್ಕೆ ಹೊರಟರು. ಇದೊಂದು ದೀರ್ಘ ಪ್ರಯಾಣ–ಸುಮಾರು ನಾನೂರು ಮೈಲಿ. ದಾರಿಯಲ್ಲಿ ಕೆಲವು ದಿನ ಗೋಯಲ್​ಪಾರಾದಲ್ಲಿ ವಿಶ್ರಮಿಸಿಕೊಂಡರು. ಗೌಹಾತಿಯು ಬ್ರಹ್ಮ ಪುತ್ರ ನದಿಯ ದಂಡೆಯ ಮೇಲಿದ್ದು ಬೆಟ್ಟಗಳಿಂದ ಸುತ್ತುವರಿದಿದೆ. ಇದು ಪ್ರಕೃತಿ ಸೌಂದರ್ಯದ ಖನಿ. ಇದರ ಸಮೀಪದಲ್ಲಿ ಅನೇಕ ತೀರ್ಥಕ್ಷೇತ್ರಗಳಿವೆ.

ಕಾಮಾಖ್ಯ ದೇವಿಯ ಪ್ರಸಿದ್ಧ ದೇವಾಲಯವಿರುವ ಈ ಕಾಮಾಖ್ಯ ಕ್ಷೇತ್ರವಿರುವುದು ಗೌಹಾತಿ ಯಿಂದ ಹನ್ನೆರಡು ಮೈಲಿ ದೂರದಲ್ಲಿ. ಇದು ಭಾರತದ ಮಹಾಪೀಠಗಳಲ್ಲೊಂದು. ಭಾರತದ ಹಲವೆಡೆಗಳಿಂದ ಇಲ್ಲಿಗೆ ಯಾತ್ರಿಕರು ಬರುತ್ತಾರೆ. ದುರ್ಗಾಷ್ಟಮಿ, ಅಶೋಕಾಷ್ಟಮಿಯೇ ಮೊದ ಲಾದ ದಿನಗಳಲ್ಲಿ ವಿಶೇಷ ಜನಸಂದಣಿಯಿರುತ್ತದೆ. ಈ ದೇವಾಲಯವನ್ನು ಕಟ್ಟಿಸಿದವನು ಕಾಮದೇವನೆಂದೂ ನರಕಾಸುರನೆಂದೂ ಎರಡು ಕಥೆಗಳಿವೆ. ೧೬ನೆಯ ಶತಮಾನದಲ್ಲಿ ಇದನ್ನು ಕೂಚ್​ಬಿಹಾರಿನ ರಾಜ ಪುನರುದ್ಧಾರ ಮಾಡಿದ. ಅನಾದಿಕಾಲದಿಂದಲೂ ಕಾಮರೂಪ-ಕಾಮಾಖ್ಯ ಗಳು ತಂತ್ರಸಾಧನೆ ಹಾಗೂ ವಾಮಾಚಾರಗಳ ಅತಿ ಮುಖ್ಯ ಸ್ಥಳಗಳಾಗಿವೆ. ಕಾಮಾಖ್ಯದೇವಿ ಯನ್ನು ದರ್ಶಿಸಿದ ಸ್ವಾಮೀಜಿಯವರು ಗೌಹಾತಿಯಲ್ಲಿ ಕೆಲದಿನ ಉಳಿದುಕೊಂಡರು. ಈ ಅವಧಿಯಲ್ಲಿ ಅಲ್ಲಿನ ನಾಗರಿಕರ ಬೇಡಿಕೆಯ ಮೇರೆಗೆ ಹಿಂದೂಧರ್ಮವನ್ನು ಕುರಿತು ಮೂರು ಉಪನ್ಯಾಸಗಳನ್ನು ನೀಡಿದರು. ಉಪನ್ಯಾಸಗಳ ಸಮಯದಲ್ಲಿ ಅವರು “ಅಗ್ನಿಮುಖಿ”ಯಾಗಿ ತೋರಿದರೆಂದು ಅವುಗಳನ್ನು ಕೇಳಿದವರು ಹೇಳಿದ್ದಾರೆ.

ಆದರೆ ಸ್ವಾಮೀಜಿಯವರಿಗಾದ ದೇಹಶ್ರಮ ಮಾತ್ರ ತೀವ್ರವಾಗಿತ್ತು. ಢಾಕಾದಲ್ಲೂ ಗೌಹಾತಿ ಯಲ್ಲೂ ಅವರ ಆರೋಗ್ಯ ಹದಗೆಡುತ್ತಲೇ ಬಂದಿತ್ತು. ಇದರಿಂದಾಗಿ ಚಿಂತೆಗೊಳಗಾಗಿದ್ದ ಅವರ ಸಂಗಡಿಗರು, ಚಿಕಿತ್ಸೆ ಮತ್ತು ಹವಾ ಬದಲಾವಣೆಗಾಗಿ ಅವರನ್ನು ಅಸ್ಸಾಮಿನ ಆಗಿನ ರಾಜಧಾನಿ ಯಾದ ಶಿಲ್ಲಾಂಗಿಗೆ ಕರೆದೊಯ್ದರು. ಇದೊಂದು ಅತಿ ಸುಂದರ ಗಿರಿಧಾಮ. ಇಲ್ಲಿನ ಹವೆ ತೇವರಹಿತವಾದ್ದರಿಂದ ಅವರ ಆರೋಗ್ಯ ಸುಧಾರಿಸಬಹುದೆಂದು ನಿರೀಕ್ಷಿಸಲಾಗಿತ್ತು. ಆದರೆ ಇಲ್ಲಿಯೂ ಅದು ಉತ್ತಮವಾಗುವುದರ ಬದಲು ದಿನದಿನಕ್ಕೆ ಕ್ಷೀಣಿಸುತ್ತ ಬಂದಿತು. ಡಯಾ ಬಿಟಿಸ್ ತೀವ್ರವಾಗಿದ್ದರ ಸೂಚನೆಗಳು ಕಂಡುಬಂದುವು. ಜೊತೆಗೆ ಒಂದು ದಿನವಂತೂ ಆಸ್ತಮಾ ಹೊಡೆತ ಭಯಂಕರವಾಗಿಯೇ ತಗುಲಿತು. ಒಂದು ಗಂಟೆಗೂ ಹೆಚ್ಚುಕಾಲ ಸ್ವಾಮೀಜಿ ಉಸಿರಿ ಗಾಗಿ ಚಡಪಡಿಸಿದರು. ಅದನ್ನು ಕಂಡವರಿಗೆ ಅವರಿನ್ನು ಉಳಿದುಕೊಂಡಾರೆಂಬ ಆಸೆಯೇ ತಪ್ಪಿಹೋಯಿತು. ಆಗ ಸ್ವಾಮೀಜಿಯವರು ಕನಸಿನಲ್ಲಿ ಕನವರಿಸುವಂತೆ, ಒಮ್ಮೆ ತಮ್ಮಷ್ಟಕ್ಕೆ ತಾವೇ ಗುನುಗುಟ್ಟಿದರು: “ಅದರಿಂದೇನಂತೆ? ನಾನವರಿಗೆ ಒಂದೂವರೆ ಸಾವಿರ ವರ್ಷಗಳಿಗೆ ಸಾಕಾಗುವಷ್ಟು ಕೊಟ್ಟಿದ್ದೇನೆ.” ತಮ್ಮ ಕಾರ್ಯೋದ್ದೇಶವು ಬಹುಮಟ್ಟಿಗೆ ನೆರವೇರಿರುವುದರಿಂದ ತಾವಿನ್ನು ನಿಶ್ಚಿಂತೆಯಿಂದ ದೇಹತ್ಯಾಗ ಮಾಡಬಹುದೆಂದು ಅವರು ನಿಶ್ಚಯಿಸಿದಂತಿತ್ತು.

ಶಿಲ್ಲಾಂಗಿನಲ್ಲಿ ಸ್ವಾಮೀಜಿಯವರು ಅಲ್ಲಿನ ಚೀಫ್ ಕಮೀಷನರನೂ ಸ್ವತಂತ್ರ ಭಾರತದ ಬೆಂಬಲಿಗನೂ ಆಗಿದ್ದ ಸರ್ ಹೆನ್ರಿ ಕಾಟನ್ ಎಂಬುವನನ್ನು ಸಂಧಿಸಿದರು. ಆತನ ಕೋರಿಕೆಯ ಮೇರೆಗೆ ಅವರು ಅಲ್ಲಿನ ಆಂಗ್ಲ ಅಧಿಕಾರಿಗಳು ಹಾಗೂ ಸ್ಥಳೀಯರನ್ನು ಉದ್ದೇಶಿಸಿ ಒಮ್ಮೆ ಭಾಷಣ ಮಾಡಿದರು. ಅವರ ಭಾಷಣವನ್ನು ತುಂಬ ಇಷ್ಟಪಟ್ಟ ಸರ್ ಕಾಟನ್, ಅವರೊಂದಿಗೆ ಭಾರತ ಮತ್ತು ಅದರ ಸಮಸ್ಯೆಗಳ ಬಗ್ಗೆ ಚರ್ಚಿಸಿ ಹೆಚ್ಚಿನ ಮಾಹಿತಿಯನ್ನು ಪಡೆದುಕೊಂಡ. ಸ್ವಾಮೀಜಿಯವರು ಅನಾರೋಗ್ಯಪೀಡಿತರಾಗಿದ್ದುದನ್ನು ಕಂಡು ಅವರಿಗೆ ಎಲ್ಲ ಬಗೆಯ ವೈದ್ಯ ಕೀಯ ನೆರವನ್ನು ನೀಡುವಂತೆ ಸರ್ಕಾರೀ ವೈದ್ಯರಿಗೆ ಆದೇಶಿಸಿದ. ಅವರು ಶಿಲ್ಲಾಂಗಿನಲ್ಲಿದ್ದಷ್ಟು ಕಾಲವೂ ಪ್ರತಿ ದಿನ ಬಂದು ಅವರ ಆರೋಗ್ಯವನ್ನು ವಿಚಾರಿಸಿಕೊಂಡ. ಅವನ ದಯಾರ್ದ್ರ ಹೃದಯವನ್ನು ಸ್ವಾಮೀಜಿ ಮೆಚ್ಚಿಕೊಂಡರು. ಅಲ್ಲದೆ ಅವನನ್ನು, “ಭಾರತದ ಆವಶ್ಯಕತೆಗಳನ್ನು ಅರಿತವನು ಹಾಗೂ ಅದರ ಉನ್ನತಿಗಾಗಿ ಶ್ರಮಿಸುತ್ತಿರುವವನು; ಭಾರತೀಯರ ಪ್ರೀತಿಗೆ ಪಾತ್ರನಾದವನು” ಎಂದು ಕೊಂಡಾಡಿದರು.

