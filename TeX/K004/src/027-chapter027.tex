
\chapter{ಪ್ರಯಾಣದೊಂದಿಗೆ ಶಿಕ್ಷಣ}

\noindent

ಸ್ವಾಮೀಜಿಯವರೀಗ ಮತ್ತೆ ಪಶ್ಚಿಮದತ್ತ ಹೊರಟಿದ್ದಾರೆ. ಮೇಲ್ನೋಟಕ್ಕೆ ಇದು ಅವರು ಆರೋಗ್ಯ ಸುಧಾರಣೆಗಾಗಿ ವೈದ್ಯರ ಸಲಹೆಯ ಮೇರೆಗೆ ಕೈಗೊಂಡ ಸಮುದ್ರ ಯಾನದಂತೆ ಕಂಡು ಬಂದರೂ ಇದರಲ್ಲಿ ಇನ್ನೂ ಹೆಚ್ಚಿನ ಮಹತ್ವದ ಅರ್ಥವೊಂದು ಅಡಗಿದೆ. ತಾವು ಮತ್ತೊಮ್ಮೆ ಪಶ್ಚಿಮರಾಷ್ಟ್ರಗಳಿಗೆ ಭೇಟಿ ನೀಡುವ ಇಚ್ಛೆಯನ್ನು ಅವರು ಒಂದು ವರ್ಷದಿಂದಲೂ ವ್ಯಕ್ತಪಡಿಸು ತ್ತಲೇ ಇದ್ದರು. ಅವರೇ ಹೇಳುತ್ತಿದ್ದರು, ‘ಪಶ್ಚಿಮದೇಶಗಳಿಗೆ ನಾನು ಕೊಡಬೇಕಾದ ಸಂದೇಶ ವನ್ನಿನ್ನೂ ಪೂರ್ತಿಯಾಗಿ ಕೊಟ್ಟಿಲ್ಲ’ ಎಂದು. ಎಂದಮೇಲೆ ಅವರು ಈ ಉದ್ದೇಶದಿಂದಲೇ ತಮ್ಮ ಪ್ರಯಾಣವನ್ನು ಯೋಜಿಸಿಕೊಂಡು ಹೊರಡಬಹುದಿತ್ತು. ಆದರೆ ಈಗ ಅವರು ಹೊರಡುತ್ತಿರು ವುದು ವೈದ್ಯನೊಬ್ಬನ ಸಲಹೆಯ ಮೇರೆಗೆ! ಈ ವೈದ್ಯನ ಸಲಹೆಯೆಂಬುದು ಕೇವಲ ನಿಮಿತ್ತ ಮಾತ್ರ. ಏಕೆಂದರೆ ಸ್ವಾಮೀಜಿಯವರು ಮುಂದೆ ಅಮೆರಿಕದಲ್ಲಿ ಸಾಧಿಸಲಿರುವ ಮಹಾಕಾರ್ಯ ಗಳನ್ನು ನೋಡಿದಾಗ ಇವೆಲ್ಲ ‘ರೋಗಿ’ಯೊಬ್ಬನಿಂದ ಸಾಧ್ಯವಾಗಬಹುದಾದ ಕೆಲಸವೆ ಎಂದು ನಾವು ಬೆಕ್ಕಸಬೆರಗಾಗದಿರಲಾರೆವು. ಹೀಗೆ ಅವರ ಪ್ರತಿಯೊಂದು ಕಾರ್ಯದ ಹಿಂದೆಯೂ ಅಗೋಚರ ಶಕ್ತಿಯೊಂದರ ಕೈವಾಡವನ್ನು ನಾವು ಕಾಣಬಹುದಾಗಿದೆ.

ಸ್ವಾಮೀಜಿಯವರನ್ನು ಹೊತ್ತ ಎಸ್.ಎಸ್. ಗೋಲ್ಕೊಂಡ ಎಂಬ ಹಡಗು ಕಲ್ಕತ್ತ ಬಂದರನ್ನು ಬಿಟ್ಟು ಹೊರಟಿತು. ಅದು ಹೂಗ್ಲಿ ನದಿಯ ಮೂಲಕ ಸಾಗಿ ಬಂಗಾಳ ಕೊಲ್ಲಿಯನ್ನು ಸೇರಬೇಕಾ ಗಿದೆ. ಆದರೆ ಈ ಹೂಗ್ಲಿ ನದಿಯ ಪ್ರಯಾಣ ಸ್ವಲ್ಪ ಅಪಾಯಕರ, ಆದ್ದರಿಂದ ಬಹಳ ನಿಧಾನ. ಬೆಳಗಿನ ಹೊತ್ತಿನಲ್ಲಿ ಮಾತ್ರವೇ, ಅದೂ ಬಹಳ ಎಚ್ಚರಿಕೆಯಿಂದ, ಇಲ್ಲಿ ಸಾಗಬಹುದು, ಆದ್ದರಿಂದ ಈ ಹಡುಗು ನದಿಯಿಂದ ಸಮುದ್ರಕ್ಕಿರುವ ಸ್ವಲ್ಪವೇ ದೂರವನ್ನು ಕ್ರಮಿಸಲು ಎರಡು ದಿನ ಬೇಕಾಯಿತು.

ಸ್ವಾಮೀಜಿಯವರು ಈ ಪಶ್ಚಿಮ ದೇಶಗಳ ಯಾತ್ರೆಯ ತಮ್ಮ ಅನುಭವಗಳನ್ನು ಬಂಗಾಳೀ ಭಾಷೆಯಲ್ಲಿ ಲೇಖನರೂಪದಲ್ಲಿ ಬರೆದು, ಹೊಸದಾಗಿ ಪ್ರಾರಂಭವಾಗಿದ್ದ ‘ಉದ್ಬೋಧನ’ ಪತ್ರಿಕೆಗೆ ಕಳಿಸಿಕೊಡಲು ಒಪ್ಪಿದರು. ಸ್ವಾಮಿ ತ್ರಿಗುಣಾತೀತಾನಂದರ ಕೋರಿಕೆಯಿಂದ ಇದು ಸಾಧ್ಯವಾಯಿತು. ಅವರ ಈ ಲೇಖನಗಳು ‘ಪರಿವ್ರಾಜಕ’ ಎಂಬ ಶೀರ್ಷಿಕೆಯಲ್ಲಿ ‘ಉದ್ಬೋಧನ’ ದಲ್ಲಿ ಧಾರಾವಾಹಿಯಾಗಿ ಪ್ರಕಟವಾದುವು. ಈಗ ಇವನ್ನು \eng{\textit{Memoirs of European Travel}} ಎಂಬ ಹೆಸರಿನಲ್ಲಿ ಇಂಗ್ಲಿಷಿಗೂ, ‘ಪರಿವ್ರಾಜಕ’ ಎಂಬ ಹೆಸರಿನಲ್ಲಿ ಕನ್ನಡಕ್ಕೂ ಅನುವಾದಿಸ ಲಾಗಿದ್ದು ವಿವೇಕಾನಂದರ ‘ಕೃತಿಶ್ರೇಣಿ’ಯಲ್ಲಿ ಇವನ್ನು ಕಾಣಬಹುದಾಗಿದೆ. ಸ್ವಾಮೀಜಿ ಬರೆದಿ ರುವ ಈ ಲೇಖನ ಮಾಲೆಯು ನಮ್ಮ ಪಾಲಿನ ಸುಕೃತಫಲವೇ ಸರಿ. ಇದರಿಂದ ಅವರ ರಸಮಯ ಪ್ರವಾಸದ ಸುಂದರ ಚಿತ್ರಣವೊಂದು ನಮಗೆ ಲಭ್ಯವಾಗಿದೆ. ಅಲ್ಲದೆ, ಇತರ ಹಲವಾರು ವೈವಿಧ್ಯಮಯ ವಿಷಯಗಳ ಬಗ್ಗೆ ಅವರು ಬರೆದಿರುವ ವಿವರಗಳು ಬಹುಶಃ ನಮಗೆ ಇನ್ನೆಲ್ಲೂ ಸಿಗಲಾರದಂಥವು. ಓದುಗರಿಗೆ ಅದು ಸ್ವಾಮೀಜಿಯವರ ಸಾನ್ನಿಧ್ಯದ ಅನುಭವವನ್ನೇ ಮಾಡಿಸಿ ಕೊಡುತ್ತದೆ. ಅವರು ತಮ್ಮ ಜೀವನದ ಯಾವುದೇ ಕಾಲದ ಬಗ್ಗೆಯೂ ಈ ರೀತಿಯಾಗಿ ಬರೆದದ್ದಿಲ್ಲ.

ಬಂಗಾಳಕೊಲ್ಲಿಯನ್ನು ಸೇರಿದ ಹಡಗು ಈಗ ಮದ್ರಾಸಿನತ್ತ ಹೊರಟಿತು. ಮತ್ತೆ ಎರಡು ದಿನಗಳ ಪ್ರಯಾಣ. ಆದರೆ ಚಂಡಮಾರುತದಿಂದಾಗಿ ಸಮುದ್ರದಲ್ಲಿ ಬೆಟ್ಟದೆತ್ತರದ ಅಲೆಗಳು ಏಳುತ್ತಿದ್ದುದರಿಂದ ಪ್ರಯಾಣ ಅಷ್ಟೇನೂ ಸುಖಕರವಾಗಿರಲಿಲ್ಲ. ಅಂತೂ ಹಡಗು ಮದ್ರಾಸಿಗೆ ಬಂದು ಸೇರಿತು. ಸ್ವಾಮೀಜಿ ಕಲ್ಕತ್ತದಿಂದ ಹೊರಟ ಸುದ್ದಿ ಆಗಲೇ ತಂತಿ ಮೂಲಕ ಮದ್ರಾಸಿಗೆ ಬಂದು ತಲುಪಿತ್ತು. ಈ ವರ್ತಮಾನ ಮಿಂಚಿನಂತೆ ನಗರದಲ್ಲೆಲ್ಲ ಹಬ್ಬಿ, ಸ್ವಾಮೀಜಿಯವರನ್ನು ಎದುರ್ಗೊಳ್ಳಲು ಇಡೀ ನಗರವೇ ಗರಿಗೆದರಿ ಸಿದ್ಧವಾಗಿತ್ತು. ಆದರೆ ಕಲ್ಕತ್ತವು ಪ್ಲೇಗ್ ಪೀಡಿತವಾದ್ದರಿಂದ ಅಲ್ಲಿನ ಭಾರತೀಯ ಪ್ರಯಾಣಿಕರು ಹಡಗಿನಿಂದ ಇಳಿಯುವಂತಿರಲಿಲ್ಲ. ಈ ವಿಷಯ ತಿಳಿದು ಬಂದಾಗ ಸ್ವಾಮೀಜಿಯವರ ಬರವಿಗಾಗಿಯೇ ಕಾದು ಕುಳಿತಿದ್ದ ಅಸಂಖ್ಯಾತ ಜನರಿಗೆ ಆದ ನಿರಾಶೆ ಅಷ್ಟಿಷ್ಟಲ್ಲ. ಆದರೆ ಸ್ವಾಮೀಜಿಯವರನ್ನು ಹಡಗಿನಿಂದ ಇಳಿಯಲು ಬಿಡಲಾರರೆಂದು ಈ ಮೊದಲೇ ತಿಳಿದಿದ್ದ ಕೆಲವು ಸಾರ್ವಜನಿಕರು ಕೆಲದಿನಗಳ ಹಿಂದೆಯೇ ಪ್ರಮುಖರನ್ನೆಲ್ಲ ಸೇರಿಸಿ ಸಭೆಗೂಡಿಸಿದ್ದರು. ಆ ಸಭೆಯಲ್ಲಿ ಒಂದು ನಿರ್ಣಯವನ್ನು ತಂದು, ಸ್ವಾಮಿ ವಿವೇಕಾನಂದರು ಕೆಲವು ಗಂಟೆಗಳ ಮಟ್ಟಿಗಾದರೂ ಮದ್ರಾಸಿನಲ್ಲಿ ಇಳಿಯಲು ಅನುಮತಿ ನೀಡಬೇಕು ಎಂದು ಸರ್ಕಾರಕ್ಕೆ ಮನವಿ ಸಲ್ಲಿಸಿದರು. ಆದರೆ ಸರ್ಕಾರದ ಕಿವುಡು ಕಿವಿಗೆ ಈ ಮೊರೆ ಕೇಳಿಸಲೇ ಇಲ್ಲ. ಹಡಗುಕಟ್ಟೆಯ ಆರೋಗ್ಯಾಧಿಕಾರಿಗೆ ‘ಮೇಲಿಂದ’ ಯಾವ ಸೂಚನೆಯೂ ಬಾರದಿದ್ದರಿಂದ ಅವನೂ ಕೈ ಅಲ್ಲಾಡಿಸಿ ಬಿಟ್ಟ. ಹಡಗಿನಿಂದ ಇಳಿದು ಹೋಗಿಬರಲು ಐರೋಪ್ಯ ರಿಗೆ ಅವಕಾಶವಿತ್ತು. ಭಾರತೀಯರಿಗೆ ಮಾತ್ರ ಈ ನಿಷೇಧ. ಭಾರತೀಯ ಎಂದರಾಯಿತು. ಅವನು ಎಂಥವನೇ ಆಗಿರಲಿ, ಅವನೊಬ್ಬ ಕೊಳಕು ಅಭ್ಯಾಸಗಳುಳ್ಳವನಾಗಿದ್ದು ಅವನ ಮೂಲಕ ಪ್ಲೇಗ್ ಕ್ರಿಮಿಗಳು ಖಂಡಿತವಾಗಿ ಹರಡುತ್ತವೆ ಎನ್ನುವುದು ಆ ಅಧಿಕಾರಿಗಳ ಅಭಿಪ್ರಾಯವಾಗಿರಬೇಕು. ಅಂತೂ ಸ್ವಾಮೀಜಿಯವರಿಗೆ ಹಡಗಿನಿಂದ ಇಳಿಯಲು ಅನುಮತಿ ಸಿಗಲೇ ಇಲ್ಲ.

ಸ್ವಾಮೀಜಿ ಇಳಿದು ಬರುವುದಿಲ್ಲವೆಂಬುದು ಖಚಿತವಾದ ಮೇಲೆ ಎಷ್ಟೋ ಜನ ನಿರಾಶರಾಗಿ ಹಿಂದಿರುಗಿದರು. ಆದರೆ, ಏನಾದರೂ ಸರಿಯೆ, ತಾವು ಸ್ವಾಮೀಜಿಯವರ ದರ್ಶನ ಮಾಡಲೇ ಬೇಕು ಎಂದು ತೀರ್ಮಾನಿಸಿದವರು ಸಣ್ಣ ಸಣ್ಣ ಗುಂಪುಗಳಾಗಿ ದೋಣಿಗಳ ಮೂಲಕ ಹಡಗಿನ ಬಳಿಗೆ ಸಾಗಿದರು. ಆದರೆ ಅವರು ಹಡಗಿನ ಒಳಗೆ ಹೋಗುವಂತಿರಲಿಲ್ಲ. ಸ್ವಾಮಿ ರಾಮಕೃಷ್ಣಾ ನಂದರೂ ಸ್ವಾಮಿ ನಿರ್ಭಯಾನಂರೂ ತಮ್ಮ ದೋಣಿಯಲ್ಲೇ ನಿಂತು ಸ್ವಾಮೀಜಿಯವರನ್ನು ಕಂಡು ನಮಸ್ಕರಿಸಿದರು. ಅನೇಕ ಹಳೆಯ ಭಕ್ತರು ಸ್ನೇಹಿತರು ಅಭಿಮಾನಿಗಳು ಹಡಗಿನ ಸಮೀಪ ಬಂದು, ತಮ್ಮ ದೋಣಿಗಳಲ್ಲಿ ತುಂಬಿಕೊಂಡು ತಂದ ಮಾವಿನ ಹಣ್ಣು, ಬಾಳೆಹಣ್ಣು, ಎಳನೀರು, ಮೊಸರನ್ನ ಮತ್ತು ರಾಶಿ ರಾಶಿ ಸಿಹಿ-ಖಾರದ ತಿಂಡಿ ಇವನ್ನೆಲ್ಲ ಸ್ವಾಮೀಜಿಯವರಿಗೆ ತಲುಪಿಸಿ ದರು. ಬರಬರುತ್ತ ಜನಸಂಖ್ಯೆ ವಿಪರೀತ ಹೆಚ್ಚಾಯಿತು. ಆದರೆ ಬಿಸಿಲೇರುತ್ತಿತ್ತು. ತೆರೆದ ದೋಣಿ ಗಳಲ್ಲಿ ತುಂಬ ಹೊತ್ತು ನಿಂತಿರುವುದು ಸಾಧ್ಯವಿರಲಿಲ್ಲ. ಆ ಬಿಸಿಲಿನಲ್ಲೂ ತಾವು ದಿನವಿಡೀ ಅಲ್ಲೇ ನಿಂತಿರುವುದಾಗಿ ರಾಮಕೃಷ್ಣಾನಂದರೂ ನಿರ್ಭಯಾನಂದರೂ ಹೇಳಿದಾಗ ಸ್ವಾಮೀಜಿ ಅವರನ್ನು ಸ್ವಲ್ಪ ಗದರಿಸಬೇಕಾಯಿತು. ಆಗ ರಾಮಕೃಷ್ಣಾನಂದರು ದೋಣಿಯಲ್ಲಿ ನಿಂತೇ ಹಡಗಿನ ಸುತ್ತ ಮೂರು ಬಾರಿ ಪ್ರದಕ್ಷಿಣೆ ಬಂದು ತಮ್ಮ ಪ್ರೀತಿಯ ಗುರುಭಾಯಿಯೂ ನಾಯಕನೂ ಆದ ಸ್ವಾಮೀಜಿಯವರಿಗೆ ನಮಸ್ಕರಿಸಿ ದಡದ ಕಡೆಗೆ ಹೊರಟರು. ಹಡಗಿನ ಕಟಕಟೆಯ ಮೇಲೆ ಬಾಗಿಕೊಂಡು ಬಹಳ ಹೊತ್ತು ನಿಂತಿದ್ದರಿಂದ ಸ್ವಾಮೀಜಿಯವರಿಗೂ ಆಯಾಸವಾಗಿತ್ತು. ಆದ್ದರಿಂದ ದೋಣಿಗಳಲ್ಲಿ ನೆರೆದಿದ್ದ ತಮ್ಮ ವಿಶ್ವಾಸಿಗರ, ಭಕ್ತರ ಕಡೆಗೆ ಕೈಬೀಸಿ ವಿದಾಯ ಹೇಳಿ ತಮ್ಮ ಕೋಣೆಯನ್ನು ಸೇರಿಕೊಂಡರು.

ಸ್ವಾಮೀಜಿಯವರ ಪ್ರಮುಖ ಮದ್ರಾಸೀ ಶಿಷ್ಯರಾದ ಅಳಸಿಂಗ ಪೆರುಮಾಳರಿಗೆ, ಮದ್ರಾಸಿನಲ್ಲಿ ನಡೆಯಬೇಕಾದ ಅನೇಕ ಕೆಲಸಗಳ ಬಗ್ಗೆ ಹಾಗೂ ‘ಬ್ರಹ್ಮವಾದಿನ್​’ ಪತ್ರಿಕೆಯ ಬಗ್ಗೆ ಅವರೊಂ ದಿಗೆ ಮಾತನಾಡಬೇಕಾದ ವಿಷಯ ಬಹಳವಿತ್ತು. ಆದರೆ ಅದಕ್ಕೆ ಅವಕಾಶವಾಗದ್ದರಿಂದ ಅವಸರ ವಸರವಾಗಿ ಒಂದು ಟಿಕೆಟ್ಟು ಕೊಂಡು ತಾವೂ ಅದೇ ಹಡಗನ್ನೇರಿ ಕೊಲಂಬೋವರೆಗೆ ಹೊರಟರು.

ಸಂಜೆಯ ಹೊತ್ತಿಗೆ ಹಡಗು ಬಂದರನ್ನು ಬಿಟ್ಟಿತು. ಅದು ಹೊರಡುತ್ತಿದ್ದಂತೆ ಜನಗಳ ಕಡೆ ಯಿಂದ ಗಟ್ಟಿಯಾದ ಕೂಗು ಕೇಳಿಬಂದಿತು. ಸ್ವಾಮೀಜಿ ತಮ್ಮ ಕೋಣೆಯ ಕಿಟಕಿಯಿಂದ ಇಣಿಕಿ ನೋಡುತ್ತಾರೆ–ಮದ್ರಾಸಿನ ಸಾವಿರಾರು ಮಂದಿ ಸ್ತ್ರೀಪುರುಷರು ದೂರದ ದಡದಲ್ಲಿ ನಿಂತು ಭಕ್ತ್ಯುತ್ಸಾಹದಿಂದ ಗಟ್ಟಿಯಾದ ದನಿಯಲ್ಲಿ ವಿದಾಯ ಕೋರುತ್ತಿದ್ದಾರೆ!

ಹಡಗು ಈಗ ಕೊಲಂಬೋ ಕಡೆಗೆ ಚಲಿಸುತ್ತಿದೆ. ಕಲ್ಕತ್ತವನ್ನು ಬಿಟ್ಟಾಗಿನಿಂದಲೂ ಅನನು ಕೂಲಕರವಾಗಿಯೇ ಇದ್ದ ಸಮುದ್ರಪ್ರಯಾಣ ಈಗ ಮತ್ತಷ್ಟು ಕಷ್ಟಕರವಾಗುತ್ತ ಬಂದಿತು. ಭಾರೀ ಗಾತ್ರದ ಅಲೆಗಳಿಂದಾಗಿ ಹಡಗು ವಿಪರೀತ ತೂರಾಡಲಾರಂಭಿಸಿತು. ಇದರಿಂದ ಪ್ರಯಾ ಣಿಕರಲ್ಲಿ ಅನೇಕರಿಗೆ ‘ಸಮುದ್ರರೋಗ’ವುಂಟಾಯಿತು. ಸ್ವಾಮಿ ತುರೀಯಾನಂದರೂ ಸ್ವಲ್ಪ ಬಳಲಿದರಾದರೂ ಬೇಗ ಚೇತರಿಸಿಕೊಂಡರು. ಆದರೆ ಇಬ್ಬರು ಬಂಗಾಳೀ ಹುಡುಗರು ಮಾತ್ರ ಬಹಳ ಒದ್ದಾಡಿದರು. ಒಬ್ಬನಂತೂ ತಾನು ಸತ್ತುಹೋಗುವುದೇ ಖಂಡಿತ ಎಂದು ತೀರ್ಮಾನಿಸಿ ಬಿಟ್ಟಿದ್ದ. ಸ್ವಾಮೀಜಿ ಅವನನ್ನು ಸಂತೈಸುತ್ತ ‘ಈ ತೊಂದರೆಗಳೆಲ್ಲ ಸಾಮಾನ್ಯವಾದವು; ಕೆಲವು ದಿನ ಇದ್ದು ಹೊರಟುಹೋಗುತ್ತವೆ. ಖಂಡಿತವಾಗಿಯೂ ಯಾರೂ ಸಾಯುವುದಿಲ್ಲ’ ಎಂದು ಧೈರ್ಯ ತುಂಬಿದರು. ಅಂತೂ ಆ ಇಬ್ಬರು ಹುಡುಗರೂ ಬೇಗ ಸುಧಾರಿಸಿಕೊಂಡರು. ಅವರಿಗೆ ತಿಂಡಿಯ ಕಡೆಗೆ ಸ್ವಲ್ಪ ಗಮನ ಹರಿಯುವಂತಾದಾಗ, ಸ್ವಾಮೀಜಿ ತಮಗೆ ಕಾಣಿಕೆಯಾಗಿ ಬಂದಿದ್ದ ಹಣ್ಣು ಹಂಪಲು, ಮೊಸರನ್ನ, ಸಿಹಿತಿನಿಸು–ಇವುಗಳಲ್ಲಿ ಸಿಂಹಪಾಲನ್ನು ಅವರಿಗೇ ಕೊಟ್ಟುಬಿಟ್ಟರು.

ಹಡಗಿನ ತೂರಾಟ ಹಾರಾಟಗಳಿಂದ ಸ್ವಲ್ಪವೂ ವಿಚಲಿತರಾಗದಿದ್ದವರಲ್ಲಿ ಅಳಸಿಂಗರೊ ಬ್ಬರು. ಇವರದು ಒಂದು ವಿಶಿಷ್ಟವಾದ ವ್ಯಕ್ತಿತ್ವ, ವಿಶಿಷ್ಟವಾದ ವೇಷಭೂಷಣ. ಶಿಖೆಯನ್ನು ಬಿಟ್ಟರೆ ನುಣ್ಣಗೆ ಮಾಡಿದ ತಲೆ; ಬರಿಗಾಲು; ಮೇಲುಕೋಟೆ ಪಂಚೆ; ಹಣೆಯ ಮೇಲೆ ಶೋಭಿ ಸುವ ತೆಂಗಲೆಯ ತಿರುನಾಮ. ಪೆರುಮಾಳರು ಕಟ್ಟಾ ಸಂಪ್ರದಾಯಸ್ಥರು. ಇವರು ತಮ್ಮ ಪ್ರಯಾಣ ಕಾಲದ ‘ಊಟ’ಕ್ಕಾಗಿ ಎರಡು ಗಂಟುಗಳನ್ನು ತಂದಿದ್ದರು. ಒಂದರಲ್ಲಿ ಅವಲಕ್ಕಿ, ಮತ್ತೊಂದರಲ್ಲಿ ಅರಳು ಮತ್ತು ಹುರಿದ ಬಟಾಣಿ. “ತನ್ನ ಜಾತಿ ಕೆಡಬಾರದೆಂದು ಈ ಸಿದ್ಧತೆ ಗಳು” ಎಂದು ಸ್ವಾಮೀಜಿ ಹಾಸ್ಯವಾಗಿ ಬರೆಯುತ್ತಾರೆ. ಆದರೆ ಸ್ವಾಮೀಜಿಯವರಿಗೆ ಪೆರುಮಾಳರ ಮೇಲೂ ಅವರ ಸಂಪ್ರದಾಯನಿಷ್ಠೆಯ ಮೇಲೂ ತುಂಬ ಗೌರವವಿತ್ತು. ಅವರೇ ಮತ್ತೆ ಬರೆಯುತ್ತಾರೆ, “ನಮ್ಮ ಅಳಸಿಂಗನಂತಹವರು ಕಾಣಸಿಗುವುದು ನಿಜಕ್ಕೂ ತುಂಬ ದುರ್ಲಭ. ಅವನಷ್ಟು ನಿಃಸ್ವಾರ್ಥಿಯಾದವರು, ಕಷ್ಟಪಟ್ಟು ದುಡಿಯುವವರು, ಮತ್ತು ಅಂತಹ ಗುರುಭಕ್ತಿ ಯನ್ನುಳ್ಳ ಶಿಷ್ಯರು ಈ ಭೂಮಿಯಲ್ಲಿ ತೀರ ಅಪರೂಪವೇ ಸರಿ.”

ಮೂರು ಹಗಲು ಎರಡು ಇರುಳಿನ ಬಳಿಕ ಹಡಗು ಕೊಲಂಬೋ ತಲುಪಿತು. ಅಲ್ಲಿನ ಕೆಲವು ಭಕ್ತರು, ಸ್ವಾಮೀಜಿ ಮತ್ತವರ ಜೊತೆಗಾರರು ಇಳಿಯಲು ವಿಶೇಷ ಅನುಮತಿ ಪಡೆದುಕೊಂಡಿದ್ದು ದರಿಂದ ಸ್ವಾಮೀಜಿ ಹಡಗಿನಿಂದ ಇಳಿಯಲು ಸಾಧ್ಯವಾಯಿತು. ಬಂದರಿನಲ್ಲಿ ನೆರೆದಿದ್ದ ನೂರಾರು ಜನ ಸ್ವಾಮೀಜಿಯವರನ್ನು ಹರ್ಷೋದ್ಗಾರಗಳೊಂದಿಗೆ ಬರಮಾಡಿಕೊಂಡರು. ಪಿ. ಕುಮಾರ ಸ್ವಾಮಿ ಮೊದಲಾದ ಹಲವಾರು ಗಣ್ಯ ವ್ಯಕ್ತಿಗಳು ಸ್ವಾಮೀಜಿಯವರನ್ನು ಎದುರುಗೊಂಡರು. ಕೊಲಂಬೋದಲ್ಲಿ ಅವರು ಹಲವಾರು ಸ್ಥಳಗಳನ್ನು ಸಂದರ್ಶಿಸಿದರು. ಹೀಗೆ ಅವರು ಸಂದರ್ಶಿ ಸಿದ ಸ್ಥಳಗಳಲ್ಲಿ ಅವರಿಗೆ ಪರಿಚಿತಳಾದ ಕೌಂಟೆಸ್ ಕೆನೋವರ ಎಂಬ ಅಮೆರಿಕನ್ ಮಹಿಳೆ ನಡೆಸುತ್ತಿದ್ದ ಹಲವಾರು ಶಾಲೆಗಳು ಮತ್ತು ಒಂದು ಅನಾಥಾಲಯ ಕೂಡ ಸೇರಿದ್ದುವು. ವಿದ್ಯಾ ಸಂಸ್ಥೆಗಳನ್ನು ನಡೆಸಿಕೊಂಡು ಬರುವ ಯೋಜನೆ ಹಾಕಿಕೊಂಡಿದ್ದ ನಿವೇದಿತಾ ಇವುಗಳನ್ನೆಲ್ಲ ತುಂಬ ಆಸಕ್ತಿಯಿಂದ ವೀಕ್ಷಿಸಿದಳು.

ಸ್ವಾಮೀಜಿಯವರು ಹಡಗಿಗೆ ಹಿಂದಿರುಗುವ ವೇಳೆಯಾಯಿತು. ಆಗ ಅಲ್ಲಿನ ಭಕ್ತರೊಬ್ಬರ ಮನೆಯಲ್ಲಿ ಸಂಭ್ರಮದ ಬೀಳ್ಕೊಡುಗೆಯ ಸಮಾರಂಭ ನಡೆಯಿತು. ಆ ಸಮಾರಂಭವನ್ನು ನಿವೇದಿತಾ ಮಿಸ್ ಮೆಕ್​ಲಾಡಳಿಗೆ ಬರೆದ ಪತ್ರದಲ್ಲಿ ಹೀಗೆ ಬಣ್ಣಿಸುತ್ತಾಳೆ:

“ಹಡಗಿಗೆ ಹಿಂದಿರುಗುವ ದಾರಿಯಲ್ಲಿ ನಮ್ಮನ್ನು ಒಂದು ಮನೆಗೆ ಬರಮಾಡಿಕೊಳ್ಳಲಾಯಿತು. ಮನೆಯ ಮುಂಭಾಗದಲ್ಲಿ ಬ್ಯಾಂಡು-ತುತ್ತೂರಿಗಳೊಂದಿಗೆ ಸ್ವಾಗತಿಸಿದರು. ಮನೆಯ ಒಳಗೂ ವಿಪರೀತ ಜನ. ಅಬ್ಬ, ಅದೇನು ಜನಸಂದಣಿ! ಅಲ್ಲದೆ ಅವರೆಲ್ಲ ಸ್ವಾಮೀಜಿಯವರನ್ನು ಹೇಗೆ ನೋಡುತ್ತಿದ್ದರು ಗೊತ್ತೆ? ಅವರ ಪಾಲಿಗೆ ಸ್ವಾಮೀಜಿ ಒಂದು ಅವತಾರವೇ ಸರಿ. ಸ್ವಾಮೀಜಿ ತಮ್ಮ ಐರೋಪ್ಯ ಉಡುಗೆಯತ್ತ ಬೆರಳು ಮಾಡಿ ತೋರಿಸಿದರು. ಆದರೆ ಜನ ಅದನ್ನೆಲ್ಲ ಗಮನಿಸುವವರಲ್ಲ. ಬಳಿಕ ಸ್ವಾಮೀಜಿಯವರು ಜನಗಳ ಸಂತೋಷಕ್ಕಾಗಿ ಒಂದು ಪುಟ್ಟ ಹಣ್ಣನ್ನೂ ತೊಟ್ಟು ಹಾಲನ್ನೂ ಸ್ವೀಕರಿಸಿದರು... ಅವರು ಹೊರಟು ನಿಲ್ಲುತ್ತಿದ್ದಂತೆಯೇ ಜನರೆಲ್ಲ ‘ಹರನಮಃ ಪಾರ್ವತೀಪತಯೇ ಹರಹರ ಮಹಾದೇವ!’ ಎಂದು ಉದ್ಘೋಷಿಸಿದರು. ಅದನ್ನು ನೀನು ಕೇಳಬೇಕಿತ್ತು...! ಕಿವಿ ಕಿವುಡಾಗುವಂತಿತ್ತು!! ನಾವು ಹೊರಗೆ ಬಂದು ನೋಡುತ್ತೇವೆ–ಭಾರೀ ಜನಜಾತ್ರೆ ಸೇರಿಬಿಟ್ಟಿದೆ!ಪುನಃ ಬಂದರಿನ ಬಳಿ ಜನಸಂದಣಿ! ಕೊಲಂಬೋದಲ್ಲಿ ನಮ್ಮ ಮೊದಲ ಆತಿಥೇಯರಾಗಿದ್ದ ಶ್ರೀಮತಿ ಮತ್ತು ಶ್ರೀ ಕುಮಾರಸ್ವಾಮಿ ಯವರು ನಮಗೆ ವಿದಾಯ ಹೇಳಲು, ಲೆಕ್ಕವಿಲ್ಲದಷ್ಟು ಉಡುಗೊರೆಗಳೊಂದಿಗೆ ಬಂದಿದ್ದರು.... ನಾವು ಹಡಗು ಹತ್ತಿ ಹೊರಡುತ್ತಿದ್ದಂತೆಯೇ ಮತ್ತೆ ಮೂರು ಸಲ ‘ಹರ ನಮಃ ಪಾರ್ವತಿಪತಯೇ ಹರಹರ ಮಹಾದೇವ!’ ಎಂದೂ, ಮೂರು ಸಲ ‘ಸ್ವಾಮಿ ವಿವೇಕಾನಂದಜೀ ಕೀ ನಮಸ್ಕಾರ್​!’ ಎಂದೂ ಘೋಷಿಸಿ ನಮಗೆ ವಿದಾಯ ಹೇಳಿದರು.”

ಸ್ವಾಮೀಜಿಯವರಿಂದ ಸಲಹೆ ಸೂಚನೆಗಳನ್ನು ಪಡೆದುಕೊಂಡು ಅಳಸಿಂಗ ಪೆರುಮಾಳರು ಮದ್ರಾಸಿಗೆ ಹಿಂದಿರುಗಿದರು. ಜೂನ್ ೨೮ರಂದು \eng{‘ S. S.}ಗೋಲ್ಕೊಂಡ’ ಹಡಗು ಕೊಲಂಬೋ ಬಿಟ್ಟು ದೀರ್ಘ ಪ್ರಯಾಣ ಹೊರಟಿತು. ಇಲ್ಲಿಂದ ಮುಂದಿನ ಪ್ರಯಾಣ ಮೊದಲಿಗಿಂತಲೂ ದುರ್ಭರವಾಗಿ ಕಂಡುಬರತೊಡಗಿತು. ಉದ್ದಕ್ಕೂ ಚಂಡಮಾರುತ, ಭಯಂಕರ ಮಳೆ, ಹಡಗು ಮುಂದೆಮುಂದೆ ಹೋದಂತೆಲ್ಲ ಪ್ರಕೃತಿಯ ಗರ್ಜನೆಯೂ ಜೋರಾಗುತ್ತಿದೆ! ವರ್ಷಾಕಾಲದ ಮಧ್ಯಬಾಗ. ಸುಂಯ್ಯನೆ ಬಿರುಗಾಳಿ ಬೀಸುತ್ತಿದೆ. ಶ್ಯಾಮಮೇಘಾಚ್ಛಾದಿತ ಗಗನದಿಂದ ಮುಸಲ ಧಾರೆ ಸುರಿಯುತ್ತಿದೆ. ಸಮುದ್ರರಾಜನೂ ತನ್ನ ದನಿಗೂಡಿಸುತ್ತ ವಿಕಟಾಟ್ಟಹಾಸಗೈಯುತ್ತಿದ್ದಾನೆ. ಪ್ರಕೃತಿಯ ಆರ್ಭಟವನ್ನು ತಾಳಲಾರದೆ ಬಡ ಹಡಗು ಗಡಗಡನೆ ನಡುಗುತ್ತ ಒಮ್ಮೆ ಆಗಸಕ್ಕೆ ನೆಗೆಯುತ್ತ ಮತ್ತೊಮ್ಮೆ ಪಾತಾಳಕ್ಕೆ ಕುಸಿಯುತ್ತ, ಹುಚ್ಚು ಕುದುರೆಯ ಮೇಲೇರಿ ಹೋಗುವಂತೆ ಮುನ್ನಡೆಯುತ್ತಿದೆ. ಪ್ರಯಾಣಿಕರಿಗೋ, ಬಲವಂತವಾಗಿ ತೊಟ್ಟಿಲಿಗೆ ಕಟ್ಟಿಹಾಕಿ ತೂಗಿದ ಅನು ಭವ. ಡೆಕ್ಕಿನ ಮೇಲೆ ನಿಂತರೆ ಹಡಗನ್ನೇ ಪುಡಿಗೈಯಲೆತ್ನಿಸುವ ಅಲೆಗಳು ಬಡಿಯುವ ಕಠೋರ ಶಬ್ದ. ಸೂರ್ಯನಿಗಂತೂ ಇಣಿಕಿ ನೋಡಲೂ ಅವಕಾಶವಿಲ್ಲ. ಹೋಗಲಿ, ಕೋಣೆಯೊಳಗೇ ಕುಳಿತು ಪ್ರಕೃತಿಯ ತಾಂಡವವನ್ನು ನೋಡೋಣ ಎಂದು ತುರೀಯಾನಂದರು ಮೆಲ್ಲನೆ ಕಿಟಕಿ ತೆಗೆದರೆ, ಸಮುದ್ರರಾಜ ಕೋಣೆಯೊಳಗೇ ನುಗ್ಗಿಬಿಡಬೇಕೆ! ಹಡಗಿನ ಅನುಭವೀ ಕ್ಯಾಪ್ಟನ್ನನೂ ಬೆರಗಾಗಿ “ಈ ವರ್ಷದ ಮಳೆಗಾಲ ಯಾಕೋ ಬಹಳ ಜೋರಾಗಿದೆ!” ಎಂದುದ್ಗರಿಸಿದ. ಆ ತೂಗಾಟದಲ್ಲಿ ಓದುವುದು-ಬರೆಯುವುದು ಸಾಧ್ಯವಿರಲಿಲ್ಲ. ಇಂಥ ಸ್ಥಿತಿಯಲ್ಲೂ ಸ್ವಾಮೀಜಿ ಆಗಾಗ ಕುಳಿತು ತಮ್ಮ ಪ್ರವಾಸಕಥನವನ್ನು ಬರೆದರು.

ಹೀಗೆ ಈ ಬಿರುಗಾಳಿ-ಮಳೆಗಳ ನಡುವೆಯೇ ಹಡಗು ಮುಂದುವರಿಯಿತು. ಏಡನ್​ಗೆ ೪೫ಂ ಮೈಲಿ ಪೂರ್ವಕ್ಕಿರುವ ಸುಕೋತ್ರ ದ್ವೀಪದ ಬಳಿಗೆ ಬಂದಾಗ ಪ್ರಕೃತಿಯ ಪ್ರತಾಪ ತುತ್ತತುದಿ ಯನ್ನು ಮುಟ್ಟಿತು. “ಇದೇ ಚಂಡಮಾರುತದ ಕೇಂದ್ರವಾಗಿರಬೇಕು” ಎಂದು ಕ್ಯಾಪ್ಟನ್ ನುಡಿದ. ಅವನೆಂದಂತೆಯೇ ಅಲ್ಲಿಂದ ಮುಂದೆ ಮಳೆಯ ರಭಸ ಕಡಿಮೆಯಾಗುತ್ತ ಬಂದು ಪ್ರಯಾಣ ಹೆಚ್ಚು ಸಹನೀಯವಾಯಿತು.

ಆದರೆ ಕೊಲಂಬೋದಿಂದ ಏಡನ್ನಿನವರೆಗಿನ ಪ್ರಯಾಣ ಮಾತ್ರ ಒಂದು ದೀರ್ಘ ದುಸ್ವಪ್ನವೇ ಸರಿ. ಈ ಪ್ರಯಾಣಕ್ಕೆ ಯಾವಾಗಲೂ ಆರು ದಿನ ಬೇಕಾಗುತ್ತಿದ್ದುದು ಈ ಸಲ ಹತ್ತು ದಿನ ತೆಗೆದು ಕೊಂಡಿತು. ಅಂತೂ ಹಡಗು ಏಡನ್ ತಲಪಿತು. ಆದರೆ ಇಲ್ಲಿ ಕರಿಯರಿಗೇ ಆಗಲಿ, ಬಿಳಿಯರಿಗೇ ಆಗಲಿ–ಯಾರಿಗೂ ಹಡಗಿನಿಂದಿಳಿಯಲು ಅವಕಾಶವಿರಲಿಲ್ಲ. ಅಲ್ಲದೆ ಇಲ್ಲಿ ನೋಡುವಂಥದೂ ಕೂಡ ಏನಿರಲಿಲ್ಲ. ಆದ್ದರಿಂದ ಪ್ರಯಾಣಿಕರೆಲ್ಲ ಹಡಗಿನಲ್ಲೇ ಸ್ವಸ್ಥವಾಗಿ ವಿರಮಿಸಿದರು.

ಸ್ವಾಮೀಜಿ ತಮ್ಮ ಹೆಚ್ಚಿನ ಸಮಯವೆಲ್ಲ ಓದುತ್ತಲೋ, ನಿವೇದಿತಾ ಅಥವಾ ತುರೀಯಾನಂದ ರೊಡನೆ ಮಾತನಾಡುತ್ತಲೋ, ಇಲ್ಲವೆ ‘ಉದ್ಬೋಧನ’ ಪತ್ರಿಕೆಗಾಗಿ ಬಂಗಾಳಿಯಲ್ಲಿ ಪ್ರವಾಸ ಕಥನವನ್ನು ಬರೆಯುತ್ತಲೋ ಇದ್ದರು. ತಮ್ಮ ಆರೋಗ್ಯದ ಬಗ್ಗೆ ಅವರು ವಿಶೇಷ ಎಚ್ಚರ ವಹಿಸುತ್ತಿದ್ದರು. ಪ್ರತಿದಿನ ವ್ಯಾಯಾಮ ಮಾಡಬೇಕು ಎಂದು ನಿರ್ಧರಿಸಿ ತುರೀಯಾನಂದರಿಗೆ ಹೇಳಿದರು, “ನೋಡು, ನಾನು ಪ್ರತಿ ದಿನ ವ್ಯಾಯಾಮ ಮಾಡಬೇಕು ಎಂದುಕೊಂಡಿದ್ದೇನೆ. ನಾನೇ ನಾದರೂ ಮರೆತುಬಿಟ್ಟರೆ ನೀನು ಜ್ಞಾಪಿಸಬೇಕು.” ತುರೀಯಾನಂದರು ಸಂತೋಷದಿಂದ ಒಪ್ಪಿ ಕೊಂಡರು. ಕೆಲದಿನ ವ್ಯಾಯಾಮ ಬಿಡದೆ ನಡೆದುಕೊಂಡುಬಂದಿತು. ಆದರೆ ಆಮೇಲೆ ಆಗಾಗ ಸ್ವಾಮೀಜಿಯವರು ನಿವೇದಿತೆಯೊಂದಿಗೆ ಸಂಭಾಷಣೆಯಲ್ಲಿ ಮುಳುಗಿ ವ್ಯಾಯಾಮದ ವಿಷಯ ವನ್ನು ಮರೆತುಬಿಡುತ್ತಿದ್ದರು. ತುರೀಯಾನಂದರು ಬಂದು ಜ್ಞಾಪಿಸಿದರೆ ಸ್ವಾಮೀಜಿ ಹೇಳುತ್ತಿ ದ್ದರು, “ಇವತ್ತು ಬೇಡ. ನಾನೀಗ ಸಾಕಷ್ಟು ಆರೋಗ್ಯವಾಗಿದ್ದೇನೆ. ಅಲ್ಲದೆ ಈಗ ನಿವೇದಿತೆ ಯೊಂದಿಗೆ ಮಾತನಾಡುತ್ತಿದ್ದೇನೆ. ಅವಳು ವಿದೇಶೀಯಳು. ನನ್ನಿಂದ ಈ ವಿಷಯಗಳನ್ನೆಲ್ಲ ತಿಳಿದುಕೊಳ್ಳಲೆಂದೇ ತನ್ನ ದೇಶವನ್ನು ಬಿಟ್ಟು ಬಂದಿದ್ದಾಳೆ. ಅಲ್ಲದೆ ಅವಳು ಬಹಳ ಜಾಣೆ. ಅವಳ ಜೊತೆಯಲ್ಲಿ ಮಾತನಾಡುವುದಕ್ಕೆ ಬಹಳ ಸಂತೋಷವಾಗುತ್ತದೆ.”

ಆದರೆ ವ್ಯಾಯಾಮ ಮಾಡಲಿ ಮಾಡದಿರಲಿ, ಸ್ವಾಮೀಜಿಯವರೇ ನಿರೀಕ್ಷಿಸಿದ್ದಂತೆ ಸಮುದ್ರ ಯಾನದಿಂದ ಅವರ ಆರೋಗ್ಯ ಬಹುಮಟ್ಟಿಗೆ ಸುಧಾರಿಸುತ್ತ ಬಂದಿತು. ಹಡಗು ಸೂಯೆಜ್ ಕಾಲುವೆಯ ಬಳಿ ಸಾಗುತ್ತಿರುವಾಗ (ಜುಲೈ ೧೪ರಂದು) ಅಮೆರಿಕನ್ ಶಿಷ್ಯೆ ಕ್ರಿಸ್ಟೀನಳಿಗೆ ಬರೆದ ಪತ್ರದಲ್ಲಿ ಸ್ವಾಮೀಜಿ ಅದನ್ನೆ ಹೇಳುತ್ತಾರೆ:

“ಭಾರತದಲ್ಲಿದ್ದಾಗ ನನ್ನ ಆರೋಗ್ಯ ಎಷ್ಟು ಹದಗೆಟ್ಟಿತ್ತೆಂದರೆ, ನನ್ನ ಹೃದಯದ ಬಡಿತವೇ ತಾಳತಪ್ಪಿಹೋಗಿತ್ತು. ಪರ್ವತಗಳನ್ನು ಏರಿದ್ದೇನು! ಹಿಮನದಿಯಲ್ಲಿ ಮಿಂದದ್ದೇನು! ಆ ಭಕ್ತ್ಯಾವೇಶದ ಭಾವವೇನು! ಇವುಗಳಿಂದೆಲ್ಲ ನನ್ನ ಶರೀರ ಜರ್ಜರಿತವಾಗಿಬಿಟ್ಟಿತ್ತು. ಆಗಾಗ ಆಸ್ತಮಾದ ಹೊಡೆತಗಳು ಬೀಳುತ್ತಲೇ ಇದ್ದುವು. ಅದು ಕಡೆಯ ಬಾರಿ ಬಡಿದಾಗಲಂತೂ ಏಳು ದಿನ ಏಳು ರಾತ್ರಿ ಹಿಡಿದುಕೊಂಡುಬಿಟ್ಟಿತ್ತು. ಉಸಿರಿಗಾಗಿ ಚಡಪಡಿಸುತ್ತ ನಾನು ಎದ್ದು ನಿಂತೇ ಇರುತ್ತಿದ್ದೆ.

“ಈ ಪ್ರವಾಸ ನನ್ನನ್ನು ಹೊಸ ಮನುಷ್ಯನನ್ನಾಗಿ ಮಾಡಿದೆ. ಈಗ ನಾನು ಎಷ್ಟೋ ಹಾಯಾಗಿ ದ್ದೇನೆ. ಇದು ಹೀಗೆಯೇ ಮುಂದುವರಿದರೆ ನಾನು ಅಮೆರಿಕೆಗೆ ಸೇರುವಷ್ಟರಲ್ಲಿ ಸಾಕಷ್ಟು ಶಕ್ತಿ ವಂತನಾಗಿರಬಲ್ಲೆ ಎಂದು ಭಾವಿಸುತ್ತೇನೆ.”

ಆದರೆ ಈ ಪತ್ರವು ಕ್ರಿಸ್ಟೀನಳ ಕೈಸೇರುವಷ್ಟರಲ್ಲೇ ಸ್ವಾಮೀಜಿ ಅವಳನ್ನು ಸಂಧಿಸಲಿದ್ದರು. ಈ ವಿಚಾರ ಆಗ ಅವರಿಗೆ ತಿಳಿದಿರಲಿಲ್ಲ. ಅವರು ಮೂರು ತಿಂಗಳ ಹಿಂದೆಯೇ, ಎಂದರೆ ಏಪ್ರಿಲ್ ೧೧ರಂದು ಆಕೆಗೆ ಬರೆದ ಪತ್ರದಲ್ಲಿ, ತಮ್ಮನ್ನು ಭೇಟಿಯಾಗಲು ಇಂಗ್ಲೆಂಡಿಗೆ ಬರುವಂತೆ ಆಹ್ವಾನಿಸಿರಲಿಲ್ಲವೆ? ಅದರ ಪ್ರಕಾರ ಕ್ರಿಸ್ಟೀನ ಮತ್ತು ಶ್ರೀಮತಿ ಮೇರಿ ಫಂಕೆ ಆಗಲೇ ಲಂಡನ್ನಿಗೆ ಬಂದು ಸೇರಿದ್ದರು. ಈ ವಿಷಯ ಸ್ವಾಮೀಜಿಯವರಿಗೆ ತಿಳಿದದ್ದು ಅವರು ಮಾರ್​ಸೆಲ್ಸ್ ತಲುಪಿದ ಮೇಲೆ ಪಡೆದ ತಂತಿಯಿಂದಲೇ.

ಈ ಸುದೀರ್ಘ ಪ್ರಯಾಣದಿಂದ ಸ್ವಾಮೀಜಿಯವರಿಗಾದ ಮತ್ತೊಂದು ಅನುಕೂಲವೇ ನೆಂದರೆ, ಅವರಿಗೆ ಸಾಕಷ್ಟು ಮಾನಸಿಕ ವಿಶ್ರಾಂತಿಯೂ ದೊರೆಯುವಂತಾದದ್ದು. ಅತ್ಯಧಿಕ ಪರಿಶ್ರಮದ ಕಾರ್ಯಕಲಾಪಗಳಿಂದ ಅವರಿಗೊಂದು ಬದಲಾವಣೆ ಸಿಕ್ಕಂತಾಗಿತ್ತು. ಸ್ವಾಮೀಜಿ ಯವರ ಲೇಖನಿಯಿಂದ ಮೂಡಿಬಂದ ‘ಪರಿವ್ರಾಜಕ’ ಎಂಬ ಪುಸ್ತಕವನ್ನು ಓದುತ್ತಿದ್ದಂತೆ, ಅದನ್ನವರು ನಿರಾಳವಾದ ಮನಸ್ಥಿತಿಯಲ್ಲಿ ಬರೆದಿರಬೇಕು ಎಂದು ನಮಗನ್ನಿಸುತ್ತದೆ. ಆದ್ದರಿಂದ, ಅವರು ಯಾವುದೇ ಗಂಭೀರ ಕಾರ್ಯಭಾರದಲ್ಲಿ ತೊಡಗಿರಲಿಲ್ಲ. ಮತ್ತು ಅವರು ಕೈಗೊಂಡದ್ದು ಕೇವಲ ಮನರಂಜನೆಯ ಉದ್ದೇಶವಿರಿಸಿಕೊಂಡ ಪ್ರವಾಸ ಮಾತ್ರ ಎಂಬ ಭಾವನೆ ನಮ್ಮಲ್ಲಿ ಮೂಡಿದರೆ ಅಚ್ಚರಿಯೇನಿಲ್ಲ. ಅವರ ಅಗಾಧ ಜ್ಞಾನ, ವೈವಿಧ್ಯಮಯ ಮಾಹಿತಿ, ತಮ್ಮ ಸುತ್ತಲಿನ ಪರಿಸರ-ಸನ್ನಿವೇಶಗಳನ್ನು ಸೂಕ್ಷ್ಮವಾಗಿ ಮತ್ತು ಸಂಕೀರ್ಣವಾಗಿ ಗಮನಿಸಬಲ್ಲ ಸಾಮರ್ಥ್ಯ ಮತ್ತು ಅವರ ಅಸಾಮಾನ್ಯ ಹಾಸ್ಯಪ್ರಜ್ಞೆ–ಇವು ಅವರ ಲೇಖನಿಯಿಂದ ಅತ್ಯಂತ ಸುಲಲಿತ, ಮೋಹಕ ಶೈಲಿಯಲ್ಲಿ ಪ್ರವಹಿಸಿವೆ; ಅಪೂರ್ವ ಹಾಸ್ಯ ಹಾಗೂ ವಾಕ್ಚಾತುರ್ಯದಿಂದ ತುಂಬಿವೆ. ಬಂಗಾಳೀ ಮೂಲವನ್ನು ಓದಿ ಆನಂದಿಸಬಲ್ಲವರಿಗೆ ಆ ಪುಸ್ತಕದ ಭಾಷೆಯ ಸೊಗಸು ತಿಳಿಯು ತ್ತದೆ. ಮನಸ್ಸು ಮಾಡಿದ್ದರೆ, ಅವರೊಬ್ಬ ಶ್ರೇಷ್ಠ ಬಂಗಾಳೀ ಸಾಹಿತಿಯಾಗಬಹುದಾಗಿತ್ತು ಎಂಬುದು ಅರಿವಾಗುತ್ತದೆ.

ಆದರೆ ಸ್ವಾಮೀಜಿಯವರು ಮೇಲ್ನೋಟಕ್ಕೆ ಕಾಣುವಷ್ಟು ನಿರಾಳವಾಗೇನೂ ಇರಲಿಲ್ಲ. ಅವರ ಅಂತರಂಗದಲ್ಲಿ ಅದೇ ಮಹಾ ಭಾವನೆಗಳೇ ತುಂಬಿದ್ದುವು ಹಾಗೂ ತಮ್ಮ ಮುಂದಿನ ಕಾರ್ಯ ಗಳ ಬಗೆಗಿನ ಗಂಭೀರವಾದ ಆಲೋಚನೆಗಳು ಸುಳಿದಾಡುತ್ತಿದ್ದುವು. ಇದು ಅವರು ಪೋರ್ಟ್ ಸೆಡ್​ನಿಂದ ಇಂಗ್ಲೆಂಡಿನ ತಮ್ಮ ಬೆಂಬಲಿಗ ಹಾಗೂ ಶಿಷ್ಯ ಇ. ಟಿ. ಸ್ಟರ್ಡಿಗೆ ಬರೆದ ಈ ಪತ್ರದಿಂದ ತಿಳಿದು ಬರುತ್ತದೆ: “ನಿನಗೆ ತಿಳಿದೇ ಇರುವಂತೆ, ಈಗ (ಬೇಸಿಗೆಯಲ್ಲಿ) ಲಂಡನ್ನಿನಲ್ಲಿ ನನ್ನ ಹೆಚ್ಚಿನ ಸ್ನೇಹಿತರಾರೂ ಇರುವುದಿಲ್ಲ. ಮಿಸ್ ಮೆಕ್​ಲಾಡ್ ಕೂಡ ನಾನು ಬೇಗನೆ ಅಮೆರಿಕೆಗೆ ಬರಬೇಕೆಂದು ಬಯಸುತ್ತಾಳೆ. ಆದ್ದರಿಂದ, ಇಂಗ್ಲೆಂಡಿನಲ್ಲಿ ಹೆಚ್ಚುದಿನ ಉಳಿದುಕೊಳ್ಳುವುದು ಅಷ್ಟೇನೂ ಅಪೇಕ್ಷಣೀಯವಲ್ಲ. ಅಲ್ಲದೆ, ನನ್ನ ಆಯುಸ್ಸಾದರೂ ಇನ್ನು ಹೆಚ್ಚೇನೂ ಉಳಿದಿಲ್ಲ. ಕಡೆಯ ಪಕ್ಷ ನಾನಂತೂ ಈ ದೃಷ್ಟಿಯಿಂದಲೇ ಮುಂದುವರಿಯಬೇಕಾಗಿದೆ. ಆದ್ದರಿಂದ ಅಮೆರಿಕದಲ್ಲಿ ನಾನೇನಾದರೂ ಸಾಧಿಸಬೇಕೆಂದಿದ್ದರೆ ಇದೇ ಸಮಯ.”

ಈ ಪ್ರಯಾಣದ ಪ್ರಾರಂಭದಿಂದಲೂ ಸ್ವಾಮೀಜಿಯವರ ವಿಚಾರಧಾರೆ ನಿರಂತರವಾಗಿ ಹರಿಯುತ್ತಿತ್ತು. ಅವರ ಚಿಂತನೆಯಲ್ಲಿ ಯಾವ ಕ್ಷಣದಲ್ಲಿ ಹೊಸ ಮಿಂಚೊಂದು ಬೆಳಗು ತ್ತದೆಯೋ ಹೇಳಲು ಬರುವಂತಿರಲಿಲ್ಲ. ಅವರ ಮುಖದಿಂದ ಇದ್ದಕ್ಕಿದ್ದಂತೆ ನವನೂತನ ಸತ್ಯ ವೊಂದು ಹೊಮ್ಮುತ್ತಿತ್ತು; ಬೆರಗಾಗಿಸುವಂತಹ ಉದ್ಗಾರವೊಂದು ಬರುತ್ತಿತ್ತು. ಒಂದು ಮಧ್ಯಾಹ್ನ ನಿವೇದಿತೆಯೊಂದಿಗೆ ಮಾತುಕತೆಯಾಡುತ್ತಿದ್ದಾಗ ಅವರು ಹಠಾತ್ತನೆ ಉದ್ಗರಿಸುತ್ತಾರೆ: “ಹೌದು; ದಿನ ಕಳೆದಂತೆಲ್ಲ ನನಗೆ ಹೆಚ್ಚುಹೆಚ್ಚಾಗಿ ಅನ್ನಿಸುತ್ತಿದೆ–ಎಲ್ಲವೂ ಇರುವುದು ಪೌರುಷ ದಲ್ಲಿ ಎಂದು. ಇದು ನನ್ನ ನೂತನ ಸಂದೇಶ: ಕೆಡಕನ್ನೂ ಕೂಡ ಪೌರುಷದಿಂದ ಮಾಡು! ನೀನು ದುಷ್ಟನಾಗಲೇಬೇಕಿದ್ದರೆ ದೊಡ್ಡ ಪ್ರಮಾಣದಲ್ಲಿ ದುಷ್ಟನಾಗು!” ಇದಕ್ಕೆ ಕೆಲದಿನ ಮುಂಚೆ ನಿವೇದಿತೆ, ಭಾರತದಲ್ಲಿ ಕ್ರೌರ್ಯ-ಅಪರಾಧ ಬಹಳ ಕಡಿಮೆ ಎಂಬ ಅಂಶವನ್ನು ಪ್ರಸ್ತಾಪಿಸಿ ಆ ಬಗ್ಗೆ ಮೆಚ್ಚಿಗೆ ಸೂಚಿಸುವ ರೀತಿಯಲ್ಲಿ ಮಾತನಾಡಿದಳು. ಆಗ ಸ್ವಾಮೀಜಿ ಉದ್ವಿಗ್ನರಾಗಿ ಹೇಳುತ್ತಾರೆ, “ದೇವರ ದಯದಿಂದ ನನ್ನ ನಾಡಿನಲ್ಲಿ ಈ ಪರಿಸ್ಥಿತಿ ತದ್ವಿರುದ್ಧವಾಗಲಿ! ಏಕೆಂದರೆ, ಇದು ಖಂಡಿತವಾಗಿಯೂ ಮೃತ್ಯುವಿಗೆ ಸಮೀಪವಾದಂತಹ ಸ್ಥಿತಿಯೇ ಸರಿ!” ಒಳ್ಳೆಯತನದ ಹೆಸರಿನಲ್ಲಿ ಜನರು ದುರ್ಬಲರೂ ತಮೋಗುಣಿಗಳೂ ಆಗಿರುವುದಕ್ಕಿಂತ ಅಪರಾಧಗಳನ್ನು ಮಾಡಬಲ್ಲ ಧೀರರಾದರೂ ಆಗಲಿ ಎಂಬುದು ಸ್ವಾಮೀಜಿಯವರ ಅಭಿಪ್ರಾಯ.

ಈ ಪ್ರಯಾಣದ ಸಮಯದಲ್ಲಿ ಅವರು ನಿವೇದಿತೆಗೆ ಶಿವರಾತ್ರಿ ಕತೆ, ವಿಕ್ರಮಾದಿತ್ಯನ ನ್ಯಾಯಸಿಂಹಾಸನದ ಕತೆ, ಬುದ್ಧ-ಯಶೋಧರೆಯರಿಗೆ ಸಂಬಂಧಿಸಿದ ಐತಿಹ್ಯಗಳು, ಇವೇ ಮೊದಲಾದ ನೂರಾರು ಕತೆಗಳನ್ನು ಹೇಳಿದರು. ಒಂದು ಗಮನಾರ್ಹ ಸಂಗತಿಯೇನೆಂದರೆ, ಒಂದೇ ಕಥೆಯನ್ನು ಅವರು ಎಂದೂ ಎರಡು ಬಾರಿ ಹೇಳುತ್ತಿರಲಿಲ್ಲ. ಅಲ್ಲದೆ ಅವರ ಸಂಭಾ ಷಣೆಗಳಲ್ಲಿ ಪ್ರಸ್ತಾಪಿಸಲ್ಪಡದ ವಿಷಯವೇ ಇರುತ್ತಿರಲಿಲ್ಲ. ಉದಾಹರಣೆಗೆ, ಅವರು ಚಾತುರ್ ವರ್ಣ್ಯದ ಸಮಸ್ಯೆಯನ್ನು ಮತ್ತೆ ಮತ್ತೆ ಪ್ರಸ್ತಾಪಿಸಿ ಅದನ್ನು ಹೊಸ ಹೊಸ ದೃಷ್ಟಿಕೋನಗಳಿಂದ ಪರಿಶೀಲಿಸುತ್ತಿದ್ದರು; ಆ ವಿಷಯದಲ್ಲಿ ಹಿಂದೆ ಆಗಿರುವ, ಮುಂದೆ ಆಗಬೇಕಾದ ಕಾರ್ಯಗಳ ಬಗ್ಗೆ ಚರ್ಚಿಸುತ್ತಿದ್ದರು.

ಸ್ವಾಮೀಜಿಯವರ ಭಾವಗಳೂ ವೈವಿಧ್ಯಪೂರ್ಣ, ಸಂಭಾಷಣೆಗಳೂ ವೈವಿಧ್ಯಪೂರ್ಣ. ಕೆಲ ವೊಮ್ಮೆ ಅವರು ಗಾಢ ಆಲೋಚನೆಯಲ್ಲಿ ಮುಳುಗಿ ದೀರ್ಘಕಾಲ ಮೌನವಾಗಿ ಕುಳಿತುಬಿಡು ತ್ತಿದ್ದರು. ಹಾಗೆ ಕುಳಿತಾಗ ಅವರು ನಿದ್ರೆಯ ಸೆಳೆತಕ್ಕೆ ಸಿಲುಕಿದ್ದಾರೋ ಎನ್ನುವಂತೆ ಕಾಣಬಹು ದಾಗಿತ್ತು. ಆದರೆ ಅವರು ಇದ್ದಕ್ಕಿದ್ದಂತೆ ಎದ್ದು ನಿಂತು ತಾವು ಆಲೋಚಿಸುತ್ತಿದ್ದ ವಿಷಯದ ಬಗ್ಗೆ ತಮ್ಮ ಹೊಸ ಸಂಶೋಧನೆಯನ್ನು ಹೊರಗೆಡಹುತ್ತಿದ್ದರು.

ಸಂಭಾಷಣೆಯ ಸಂದರ್ಭದಲ್ಲೊಮ್ಮೆ ಸ್ವಾಮೀಜಿ ನುಡಿದರು, “ನಾವು ಕಿತ್ತೊಗೆಯಲು ಪ್ರಯತ್ನಿಸಬೇಕಾದದ್ದು ಸ್ವಾರ್ಥಬುದ್ಧಿಯನ್ನು, ನಾನು ನನ್ನ ಜೀವನದಲ್ಲಿ ತಪ್ಪು ಮಾಡಿದಾಗಲೆಲ್ಲ ಕಂಡುಕೊಂಡಿದ್ದೇನೆ–ಅದಕ್ಕೆ ಕಾರಣ ನನ್ನ ಯೋಜನೆಯಲ್ಲಿ ಸ್ವಾರ್ಥ ಪ್ರವೇಶಿಸಿದ್ದೇ ಎಂದು. ಈ ಸ್ವಾರ್ಥ ಎಲ್ಲಿ ಪ್ರವೇಶಿಸಿರುವುದಿಲ್ಲವೋ ಅಲ್ಲೆಲ್ಲ ನನ್ನ ತೀರ್ಮಾನ ಸರಿಯಾಗಿರುತ್ತದೆ.”

ಹಡಗು ಸಿಸಿಲಿ ದ್ವೀಪದ ಬಳಿಗೆ ಬಂದಾಗ ಸೂರ್ಯ ಅಸ್ತಮಿಸುತ್ತಿದ್ದ. ದೂರದ ಎಟ್ನಾ ಅಗ್ನಿಪರ್ವತದಿಂದ ಸಣ್ಣಗೆ ಹೊಗೆ ಹೊಮ್ಮುತ್ತಿತ್ತು. ಆಕಾಶ ರಕ್ತವರ್ಣವನ್ನು ತಳೆದು ಶೋಭಿಸು ತ್ತಿತ್ತು. ಪ್ರಕೃತಿಯ ವೈಭವವನ್ನು ವೀಕ್ಷಿಸುತ್ತ ಸ್ವಾಮೀಜಿ ನಿವೇದಿತೆಯೊಂದಿಗೆ ಹಡಗಿನ ಅಂಗಳ ದಲ್ಲಿ ಅಡ್ಡಾಡುತ್ತಿದ್ದರು. ‘ಸೌಂದರ್ಯವಿರುವುದು ಅದನ್ನು ಆಸ್ವಾದಿಸುವ ಮನಸ್ಸಿನಲ್ಲೇ ಹೊರತು ಬಾಹ್ಯಪ್ರಪಂಚದಲ್ಲಲ್ಲ’ ಎಂಬ ವಿಷಯದ ಬಗ್ಗೆ ಅವರು ಮಾತನಾಡುತ್ತಿದ್ದರು. ಅಷ್ಟು ಹೊತ್ತಿಗೆ ದಿಗಂತದಲ್ಲಿ ಚಂದ್ರ ಉದಿಸಿದ. ಈಗ ಹಡಗು ಇಟಲಿಯ ಭೂಭಾಗವನ್ನೂ ಸಿಸಿಲಿ ದ್ವೀಪವನ್ನೂ ಪ್ರತ್ಯೇಕಿಸುವ ಮೆಸ್ಸಿನಾ ಜಲಸಂಧಿಯ ಬಳಿಗೆ ಬಂದಿತು. ಇತ್ತ ಇಟಲಿಯ ತೀರದ ಬೆಟ್ಟಗುಡ್ಡಗಳು ನಸುಗತ್ತಲೆಯಲ್ಲಿ ಕರಗುತ್ತಿದ್ದರೆ ಅತ್ತ ಸಿಸಿಲಿಯು ಚಂದ್ರನ ರಜತ ಕಿರಣಗಳನ್ನು ಪ್ರತಿಫಲಿಸುತ್ತಿತ್ತು. ಈ ರಮ್ಯನೋಟವನ್ನು ದಿಟ್ಟಿಸುತ್ತ ಸ್ವಾಮೀಜಿ ಉದ್ಗರಿಸಿದರು, “ಮೆಸ್ಸಿನಾ (ಪ್ರದೇಶವೇ) ನನಗೆ ಕೃತಜ್ಞವಾಗಿರಬೇಕು! ಏಕೆಂದರೆ ಅದಕ್ಕೆ ಈ ಸೌಂದರ್ಯವನ್ನು ಕೊಟ್ಟವನು ನಾನು!” ತಾವು ಗುರುತಿಸಿದ್ದರಿಂದ ತಾನೆ ಅದಕ್ಕೆ ಆ ಸೌಂದರ್ಯ ಉಂಟಾದದ್ದು?

ಈ ದಿನಗಳಲ್ಲಿ ಸ್ವಾಮೀಜಿಯವರು ತಮಗೆ ವೈಯಕ್ತಿಕವಾಗಿ ಪರಿಚಯವಿದ್ದ ಸಂತರ-ಮಹಾ ಪುರುಷರ ಕುರಿತಾಗಿ ಮಾತನಾಡುತ್ತಿದ್ದರು. ಇವರ ಪೈಕಿ ಸರ್ವಪ್ರಧಾನರಾದವರೆಂದರೆ ಶ್ರೀರಾಮ ಕೃಷ್ಣ ಪರಮಹಂಸರು. ಅವರ ಬಗ್ಗೆ ಸ್ವಾಮೀಜಿ ನಿವೇದಿತೆಗೆ ತಿಳಿಸಿದ ಅನೇಕ ವಿಷಯಗಳು ತುಂಬ ವಿಶೇಷವಾದವುಗಳೇ ಸರಿ.

ಮತ್ತೊಮ್ಮೆ ಅವರು ನಾಗಮಹಾಶಯರನ್ನು ಕುರಿತು ಉನ್ನತ ಭಾವದಿಂದ ಮಾತನಾಡಿದರು. ಅವರು ಕಲ್ಕತ್ತವನ್ನು ಬಿಟ್ಟು ಹೊರಡುವ ಕೆಲವೇ ವಾರಗಳ ಹಿಂದೆ ಅವರನ್ನು ನೋಡಲು ನಾಗ ಮಹಾಶಯರು ಬಂದಿದ್ದಾಗ ಅಲ್ಲಿ ಉಂಟಾಗಿದ್ದ ಹೃದಯಸ್ಪರ್ಶಿ ಸನ್ನಿವೇಶವನ್ನು ನಾವು ಈಗಾಗಲೇ ನೋಡಿದ್ದೇವೆ. ಆಗ ಸ್ವಾಮೀಜಿಯವರು ತಾವು ನಾಗಮಹಾಶಯರ ಸ್ವಗ್ರಾಮವಾದ ದೇವಭೋಗಕ್ಕೆ ಯಾವಾಗಲಾದರೊಮ್ಮೆ ಬರುವುದಾಗಿ ತಿಳಿಸಿ ಅವರಿಗೆ ಅತ್ಯಾನಂದವನ್ನುಂಟು ಮಾಡಿದ್ದರು. ಆದರೆ ಸ್ವಾಮೀಜಿಯವರು ಹಡಗಿನಲ್ಲಿ ಹೊರಟು ಹೂಗ್ಲಿ ನದಿಯ ಮೇಲೆ ಬರುತ್ತಿದ್ದಾಗಲೇ ಅವರಿಗೆ ನಾಗಮಹಾಶಯರು ತೀರಿಕೊಂಡರೆಂಬ ಸುದ್ದಿ ತಲುಪಿತು. ಸ್ವಾಮೀಜಿ ನಾಗಮಹಾಶಯರ ಅದ್ಭುತ ತ್ಯಾಗ-ವೈರಾಗ್ಯಗಳನ್ನು ಹೃತ್ಪೂರ್ವಕವಾಗಿ ಕೊಂಡಾಡಿದರು. ನಾಗ ಮಹಾಶಯರು ಶ್ರೀರಾಮಕೃಷ್ಣರ ಶ್ರೇಷ್ಠತಮ ಕೃತಿಗಳಲ್ಲೊಂದು ಎಂದು ಮತ್ತೆ ಮತ್ತೆ ಹೇಳಿದರು.

ಬಳಿಕ ಅವರು ಪವಾಹಾರಿ ಬಾಬ, ಮಹಾಜ್ಞಾನಿಗಳಾದ ತ್ರೈಲಿಂಗಸ್ವಾಮಿ, ಭಕ್ತವರೇಣ್ಯರಾದ ರಘುನಾಥ ದಾಸರು–ಇವರೇ ಮೊದಲಾದ ಅನೇಕ ಸಾಧುಸಂತರನ್ನು ಕುರಿತು ಹೇಳಿದರು. ಸ್ವಾಮೀಜಿ ಇವರೆಲ್ಲರ ವಿಚಾರಗಳನ್ನು ವಿವರಿಸುತ್ತಿದ್ದಂತೆ ಕೇಳುಗರಿಗೆ ‘ಅವರಿಗಿಂತ ಶ್ರೇಷ್ಠ ರಾದವರು ಇನ್ನಿರಲಾರರು’ ಎಂಬ ಭಾವನೆ ಮೂಡುತ್ತಿತ್ತು.

ಒಮ್ಮೆ ಮಾತುಕತೆಯ ಮಧ್ಯೆ, ಯೋಗಭ್ರಷ್ಟರಾದವರ ಗತಿ ಏನಾಗುತ್ತದೆ ಎಂಬ ಪ್ರಶ್ನೆ ಬಂದಿತು. ಇದಕ್ಕೆ ಉತ್ತರವಾಗಿ ಸ್ವಾಮೀಜಿ ಅತ್ಯಂತ ಮಧುರವಾದ ದನಿಯಲ್ಲಿ ಭಗವದ್ಗೀತೆಯ ಶ್ಲೋಕಗಳನ್ನು ಉದ್ಧರಿಸಿ ಹೇಳುತ್ತಾರೆ, “ಇಂಥವರು ಖಂಡಿತವಾಗಿಯೂ ನಾಶವಾಗುವುದಿಲ್ಲ. ಸತ್ಕರ್ಮಫಲವು ಎಂದಿಗೂ ಮನುಷ್ಯನ ಕೈಬಿಡುವುದಿಲ್ಲ. ಒಮ್ಮೆ ಯೋಗಭ್ರಷ್ಟನಾದವನು ಹೆಚ್ಚು ಅನುಕೂಲಕರವಾದ ಪರಿಸರದಲ್ಲಿ ಮತ್ತೆ ಜನ್ಮವೆತ್ತುತ್ತಾನೆ. ಆಗ ಅವನಿಗೆ ತನ್ನ ವ್ರತವನ್ನು ಪೂರ್ಣಗೊಳಿಸುವ ಅವಕಾಶ ದೊರೆಯುತ್ತದೆ. ಕೆಲವರಿಗೆ ಸ್ವಲ್ಪ ಮಟ್ಟಿಗೆ ಇವುಗಳ ನೆನಪಿರುತ್ತದೆ. ಅಕ್ಬರನಿಗೆ ಆಗಾಗ ಅನ್ನಿಸುತ್ತಿತ್ತು–ತಾನು ಪೂರ್ವಜನ್ಮದಲ್ಲಿ ವ್ರತಭ್ರಷ್ಟನಾದ ಬ್ರಹ್ಮಚಾರಿ ಯಾಗಿದ್ದೆ ಎಂದು. ಹೀಗೆ ರಾಜರುಗಳಿಗೆ ಕೆಲವೊಮ್ಮೆ ಪೂರ್ವಜನ್ಮದ ಧಾರ್ಮಿಕ ಅಭ್ಯಾಸಗಳು ಅಸ್ಪಷ್ಟವಾಗಿ ಸ್ಮರಣೆಗೆ ಬರುವುದುಂಟು.”

ಹೀಗೆನ್ನುತ್ತ ಸ್ವಾಮೀಜಿ ಇದ್ದಕ್ಕಿದ್ದಂತೆ ನಿವೇದಿತೆಯತ್ತ ತಿರುಗಿ ಹೇಳುತ್ತಾರೆ, “ನೀನು ಏನಾದರೂ ತಿಳಿದುಕೊ ಮ್ಯಾರ್ಗಟ್, ನನಗೆ ಅಂತಹ ನೆನಪಿದೆ!”

ಬಳಿಕ ಸ್ವಾಮೀಜಿ ದೀರ್ಘ ಮೌನದಲ್ಲಿ ಮುಳುಗಿದರು. ಇಬ್ಬರೂ ನಕ್ಷತ್ರಗಳ ನಸುಬೆಳಕಿನಲ್ಲಿ ವಿಶಾಲ ಸಾಗರವನ್ನು ದಿಟ್ಟಿಸುತ್ತ ಕುಳಿತರು. ಸ್ವಲ್ಪ ಹೊತ್ತಾದ ಮೇಲೆ ಸ್ವಾಮೀಜಿ ಮೌನದಿಂದೆದ್ದು ಮಾತಿನ ಸರಣಿಯನ್ನು ಮುಂದುವರಿಸುತ್ತ ಹೇಳಿದರು:

“ನಾನು ಬೆಳೆದು ದೊಡ್ಡವನಾದಂತೆಲ್ಲ, ಸಣ್ಣಸಣ್ಣ ವಿಷಯಗಳಲ್ಲಿ ಹೆಚ್ಚಿನ ಮಹತ್ವವನ್ನು ಕಾಣುತ್ತಿದ್ದೇನೆ. ಒಬ್ಬ ಮಹಾಪುರುಷನೆನಿಸಿಕೊಂಡವನು ಏನನ್ನು ತಿನ್ನುತ್ತಾನೆ, ಎಂತಹ ಉಡಿಗೆ ತೊಡಿಗೆ ಧರಿಸುತ್ತಾನೆ ಮತ್ತು ತನ್ನ ಸೇವಕರ ಜೊತೆಯಲ್ಲಿ ಹೇಗೆ ವರ್ತಿಸುತ್ತಾನೆ ಎಂಬುದನ್ನು ನಾನು ನೋಡಬಯಸುತ್ತೇನೆ. ಯಾವನು ಬೇಕಾದರೂ ದೊಡ್ಡ ಸ್ಥಾನದಲ್ಲಿದ್ದುಕೊಂಡು ದೊಡ್ಡ ಮನುಷ್ಯನೆನ್ನಿಸಿಕೊಳ್ಳಬಹುದು. ದೀಪಗಳ ಪ್ರಖರ ಬೆಳಕಿನಲ್ಲಿ ಎಂತಹ ಹೇಡಿಯೂ ಕೂಡ ಮಹಾ ಪರಾಕ್ರಮಶಾಲಿಯಾಗಿ ಬೆಳಗುತ್ತಾನೆ. ಇಡೀ ಜಗತ್ತು ತನ್ನನ್ನೇ ಬೆರಗುಗಣ್ಣಿನಿಂದ ನೋಡುತ್ತಿರು ವಾಗ ಯಾರ ಹೃದಯ ತಾನೆ ಉಬ್ಬದಿರದು! ಯಾರು ತಾನೆ ತಮ್ಮ ಅತ್ಯುನ್ನತ ಸಿದ್ಧಿಯ ಮಟ್ಟ ಕ್ಕೇರಲಾರರು! ಆದರೆ ನನ್ನ ದೃಷ್ಟಿಯಲ್ಲಿ, ತನ್ನ ಪಾಲಿನ ಕರ್ತವ್ಯವನ್ನು ಕ್ಷಣದಿಂದ ಕ್ಷಣಕ್ಕೆ, ಘಳಿಗೆಯಿಂದ ಘಳಿಗೆಗೆ ಮೌನವಾಗಿ, ನಿರಂತರವಾಗಿ ಮಾಡಿಕೊಂಡುಹೋಗುತ್ತಿರುವ ಒಂದು ಕೀಟವೇ ಹೆಚ್ಚಿನ ಮಹತ್ವವುಳ್ಳದ್ದಾಗಿ ಕಂಡುಬರುತ್ತದೆ.”

ಹಡಗು ಮುಂದೆಮುಂದೆ ಸಾಗುತ್ತಿದ್ದಂತೆ ಹೊಸಹೊಸ ಸ್ಥಳಗಳನ್ನು ಹಾದು ಹೋಗುತ್ತಿದ್ದಂತೆ ಸ್ವಾಮೀಜಿಯವರು ತಮ್ಮ ಅಗಾಧ ಜ್ಞಾನಭಂಡಾರದಿಂದ ಆಯಾಸ್ಥಳಗಳಿಗೆ ಸಂಬಂಧಿಸಿದ ಹಲವಾರು ವಿಚಾರಗಳನ್ನು ತೆಗೆದು ನಿವೇದಿತೆಯ ಮುಂದೆ ಧಾರೆಹರಿಸುತ್ತಿದ್ದರು. ಇದರಿಂದ ಆಕೆಯ ಪಾಲಿಗೆ ನಿಧಿಯೊಂದು ದೊರಕಿದಂತಾಯಿತು. ಭೂಪಟದ ಮೇಲಿನ ಒಂದೊಂದು ಸ್ಥಳದಲ್ಲೂ ಆಕೆ ನೂತನ ಸೌಂದರ್ಯವನ್ನು ಕಾಣುತ್ತಿದ್ದಳು. ಹಡಗು ಇಟಲಿಯ ತೀರದಲ್ಲಿ ಸಾಗುತ್ತಿದ್ದಂತೆ ಕ್ರೈಸ್ತ ಧರ್ಮದ ಹಾಗೂ ಚರ್ಚುಗಳ ವಿಷಯ ಬಂದಿತು. ಕೊರ್ಸಿಕಾ ದ್ವೀಪದ ತೀರದ ಬಳಿ ಸಾಗುತ್ತಿರುವಾಗ, ಯುದ್ಧವೀರನಾದ ನೆಪೋಲಿಯನ್ನನಿಗೆ ಜನ್ಮವಿತ್ತ ಆ ಸ್ಥಳವನ್ನು ಬಣ್ಣಿಸಿದರು ಸ್ವಾಮೀಜಿ.

ಸ್ವಾಮೀಜಿಯವರ ಈ ಮಾತುಗಳೆಲ್ಲವೂ ಕೇವಲ ಮನರಂಜನೆಗಾಗಿ ಅಥವಾ ಕಾಲಯಾಪನೆ ಗಾಗಿ ಆಡಿದವುಗಳಲ್ಲ; ಇಲ್ಲವೆ ನಿವೇದಿತೆಗೆ ತರಬೇತಿ ನೀಡುವುದಕ್ಕಾಗಿ ಮಾತ್ರವೇ ಹೇಳಿದಂ ಥವೂ ಆಗಿರಲಿಲ್ಲ. ಆಗಾಗ ಅವರ ಮಾತು ತಮ್ಮ ಜೀವಿತದ ಮಹೋದ್ದೇಶದ ಕಡೆಗೆ ಹೊರಳುತ್ತಿತ್ತು. ಇಂತಹ ಸಂದರ್ಭಗಳಲ್ಲಿ ನಿವೇದಿತೆ ಇನ್ನೂ ಹೆಚ್ಚಿನ ಗಮನವಿಟ್ಟು ಕಾತರತೆ ಯಿಂದ, ಅವರು ಉಚ್ಚರಿಸುವ ಪ್ರತಿಯೊಂದು ಶಬ್ದವನ್ನೂ ಹೆಕ್ಕಿ ಶೇಖರಿಸಿಕೊಳ್ಳಲು ಶ್ರಮಿಸು ತ್ತಿದ್ದಳು. ಬರಬರುತ್ತ ಆಕೆಗೆ ಒಂದು ವಿಷಯ ಮನವರಿಕೆಯಾಯಿತು–ಏನೆಂದರೆ, ಸ್ವಾಮೀಜಿ ನುಡಿಯುತ್ತಿರುವ ಹಲವಾರು ಅದ್ಭುತ ಅಮೂಲ್ಯ ನುಡಿಗಳನ್ನೆಲ್ಲ ಬರೆದಿಟ್ಟುಕೊಳ್ಳಲು, ಮುಂದೆ ಅವು ಅಸಂಖ್ಯಾತ ಜನರಿಗೆ ಚೇತೋದಾಯಕವಾಗುವಂತಾಗಲು, ತಾನೊಂದು ನಿಮಿತ್ತ ಮಾತ್ರ ಎಂದು. ಏಕೆಂದರೆ ಸ್ವಾಮೀಜಿ ನೀಡುತ್ತಿರುವ ಈ ಅಮೂಲ್ಯ ಸಂದೇಶಗಳನ್ನು ಕಾರ್ಯಗತಗೊಳಿ ಸಲು ತನ್ನೊಬ್ಬಳ ಜೀವಿತಾವಧಿಯಲ್ಲಂತೂ ಸಾಧ್ಯವೇ ಇಲ್ಲ! ಆದರೆ ಸ್ವಾಮೀಜಿಯವರ ಭವ್ಯ ಕನಸುಗಳನ್ನು ನನಸಾಗಿಸಲು ಅವರ ರಾಷ್ಟ್ರದಲ್ಲೇ ಹಲವಾರು ಪ್ರತಿಭಾನ್ವಿತರು ಉದಯಿಸಲಿ ದ್ದಾರೆ; ಅವರಿಗೆಲ್ಲ ಈ ಭವ್ಯ ಸಂದೇಶಗಳನ್ನು ತಲುಪಿಸಲು ತಾನು ಕೇವಲ ಒಬ್ಬ ಮಧ್ಯವರ್ತಿ, ಕೇವಲ ಒಂದು ಸೇತುವೆ ಎಂಬ ಭಾವ ಅವಳಲ್ಲುಂಟಾಯಿತು. ಈ ಭಾವದಿಂದಲೇ ಅವಳು ಆ ನುಡಿಮುತ್ತುಗಳನ್ನೆಲ್ಲ ಸಂಗ್ರಹಿಸಿದಳು.

ಆದರೆ ಸ್ವಾಮೀಜಿ ತಮ್ಮ ನುಡಿಮುತ್ತುಗಳ ಧಾರೆಯನ್ನು ಹರಿಯಿಸುತ್ತಿದ್ದರೆ ನಿವೇದಿತೆಗೆ ಅವನ್ನು ಆಯ್ದು ತುಂಬಿಕೊಳ್ಳಲೂ ವ್ಯವಧಾನವಿಲ್ಲ. ಆದ್ದರಿಂದ ಆಕೆ ಒಮ್ಮೊಮ್ಮೆ ಅವರನ್ನು ಬೇಡಿಕೊಳ್ಳುತ್ತಿದ್ದಳು–ತಮ್ಮ ವಾಕ್ ಪ್ರವಾಹವನ್ನು ಸ್ವಲ್ಪಕಾಲ ನಿಲ್ಲಿಸುವಂತೆ! ತನ್ನ ಮಸ್ತಕ ದಲ್ಲಿ ತುಂಬಿಕೊಂಡ ವಿಚಾರಗಳಲ್ಲಿ ಸಾಧ್ಯವಾದಷ್ಟನ್ನು ತನ್ನ ಪುಸ್ತಕಕ್ಕಿಳಿಸಿದ ಮೇಲೆಯೇ ಅವಳಿಗೆ ಸಮಾಧಾನ. ಬಳಿಕ ಆಕೆ ಮತ್ತೆ ಸಿದ್ಧಳಾಗುತ್ತಿದ್ದಳು.

ಒಂದು ದಿನ ಆಕೆ ಕೇಳಿದಳು, “ಸ್ವಾಮೀಜಿ, ಭಾರತದ ಉನ್ನತಿಗಾಗಿ ನೀವು ಆಲೋಚಿಸಿರುವ ವಿಧಾನಗಳಿಗೂ ಇತರ ಹಲವಾರು ಜನ ಬೋಧಿಸುವ ಮಾರ್ಗಗಳಿಗೂ ಇರುವ ಭೇದಗಳನ್ನು ಸ್ಥೂಲವಾಗಿ ವಿವರಿಸುವಿರಾ?” ಆದರೆ ಸ್ವಾಮೀಜಿಯವರು ಈ ವಿವಾದಾತ್ಮಕವಾದ ವಿಷಯದ ಬಗ್ಗೆ ಏನೂ ಹೇಳಲೇ ಇಲ್ಲ. ಬದಲಾಗಿ ಇತರ ಮಾರ್ಗಗಳಲ್ಲಿ ಮುಂದುವರಿಯುತ್ತಿರುವ ಹಲ ವಾರು ಮುಖಂಡರ ವೈಯಕ್ತಿಕ ಅಂಶಗಳ ಬಗ್ಗೆಯೂ ಅವರ ಕಾರ್ಯಧೋರಣೆಗಳ ಬಗ್ಗೆಯೂ ಮೆಚ್ಚುಗೆಯನ್ನೇ ವ್ಯಕ್ತಪಡಿಸಿದರು. ಸ್ವಾಮೀಜಿ ತನ್ನ ಪ್ರಶ್ನೆಯನ್ನು ತಳ್ಳಿಹಾಕಿದರು ಎಂದುಕೊಂಡು ನಿವೇದಿತೆ ಸುಮ್ಮನಾದಳು.

ಆದರೆ ಅಂದು ಸಂಜೆ ಸ್ವಾಮೀಜಿ ಇದ್ದಕ್ಕಿದ್ದಂತೆ ಅವಳ ಪ್ರಶ್ನೆಗೆ ಬಂದರು; ಹೇಳಿದರು, “ಭಾರತೀಯರಿಗೆ ಪುನಃ ಅವರ ಮೂಢನಂಬಿಕೆಗಳನ್ನೇ ಬೋಧಿಸಲು ಪ್ರಯತ್ನಿಸುವವರನ್ನು ನಾನು ಒಪ್ಪುವುದಿಲ್ಲ. ಗತಕಾಲದ ಈಜಿಪ್ಟಿನ ಬಗ್ಗೆ ಪುರಾತತ್ವ ಶಾಸ್ತ್ರಜ್ಞನೊಬ್ಬನಿಗೆ ಇರುವ ಆಸಕ್ತಿ ಯಂತೆ ಈ ಜನರಿಗೆ ಭಾರತದ ಮೇಲಿರುವ ಆಸಕ್ತಿ ಕೇವಲ ಸ್ವಾರ್ಥಪ್ರೇರಿತವಾದದ್ದು. ಕೆಲವರು ತಾವು ಪುಸ್ತಕಗಳಲ್ಲಿ ಓದಿಕೊಂಡ ಅಥವಾ ಕನಸುಗಳಲ್ಲಿ ಕಟ್ಟಿಕೊಂಡ ಭಾರತವನ್ನು ಮತ್ತೆ ನೋಡಬಯಸಬಹುದು. ಆದರೆ ನಾನು ಮಾತ್ರ ಹಿಂದಿನ ಕಾಲದ ಒಳ್ಳೆಯ ಅಂಶಗಳೊಂದಿಗೆ ಇಂದಿನ ಈ ಆಧುನಿಕ ಯುಗದ ಒಳ್ಳೆಯ ಅಂಶಗಳೂ ಸಹಜವಾಗಿ ಸಮ್ಮಿಳಿತಗೊಳ್ಳಬೇಕೆಂದು ಬಯಸುವವನು. ಈ ಹೊಸ ಮಾರ್ಗವು ಬೆಳವಣಿಗೆಯ, ಪ್ರಗತಿಯ ಮಾರ್ಗವಾಗಬೇಕು.

“ಆದ್ದರಿಂದಲೇ ನಾನು ನಮ್ಮ ಎಲ್ಲ ಶಾಸ್ತ್ರಗಳ ಪೈಕಿ ಉಪನಿಷತ್ತುಗಳನ್ನು ಮಾತ್ರವೇ ಬೋಧಿಸುವುದು. ಗಮನಿಸಿ ನೋಡಿದರೆ ನಿನಗೆ ತಿಳಿಯುತ್ತದೆ–ಉಪನಿಷತ್ತುಗಳಲ್ಲದೆ ಬೇರೆ ಯಾವ ಗ್ರಂಥವನ್ನೂ ನಾನು ಉದಾಹರಿಸಿಲ್ಲ. ಮತ್ತು ಆ ಉಪನಿಷತ್ತುಗಳಲ್ಲೂ ‘ಶಕ್ತಿ’ ಎಂಬ ಒಂದು ಭಾವನೆಯನ್ನು ಮಾತ್ರ ವಿಶೇಷವಾಗಿ ಆರಿಸಿಕೊಂಡಿದ್ದೇನೆ. ವೇದ ವೇದಾಂತಗಳ ಸಾರವೆಲ್ಲ ಆ ಒಂದು ಪದದಲ್ಲಿ ಅಡಕವಾಗಿದೆ. ಬುದ್ಧನು ಅಹಿಂಸೆಯನ್ನು ಬೋಧಿಸಿದ. ಆದರೆ ಅದೇ ಅಹಿಂಸಾತತ್ತ್ವವನ್ನೇ ಬೋಧಿಸುವುದಕ್ಕೆ ಈ ‘ಶಕ್ತಿ’ ಎಂಬುದು ಹೆಚ್ಚು ಉತ್ತಮ ಮಾರ್ಗವೆಂದು ನಾನು ಭಾವಿಸುತ್ತೇನೆ. ಏಕೆಂದರೆ ಬುದ್ಧ ಬೋಧಿಸಿದ ಅಹಿಂಸೆಯ ಹಿನ್ನೆಲೆಯಲ್ಲಿ ಭಯಂಕರ ದೌರ್ಬಲ್ಯವಿದೆ. ವಿರೋಧದ ಕಲ್ಪನೆ ಹುಟ್ಟುವುದೇ ದೌರ್ಬಲ್ಯದಿಂದ. ಒಂದು ಹನಿ ನೀರು ಮೈಮೇಲೆ ಸಿಡಿದರೆ ನಾನು ಅದರಿಂದ ತಪ್ಪಿಸಿಕೊಳ್ಳಬೇಕೆಂದೋ ಅಥವಾ ಅದನ್ನು ಶಿಕ್ಷಿಸಬೇಕೆಂದೋ ಆಲೋಚಿಸುವುದಿಲ್ಲ. ನಾನದನ್ನು ಗಣನೆಗೇ ತರುವುದಿಲ್ಲ. ಆದರೆ ಒಂದು ಸೊಳ್ಳೆಗೆ ಅದೇ ಒಂದು ಗಂಭೀರ ವಿಷಯ. ನಾನು ಎಲ್ಲ ಬಗೆಯ ಹಿಂಸೆಯನ್ನೂ ಇದೇ ದೃಷ್ಟಿಯಿಂದ ನೋಡುತ್ತೇನೆ. ಶಕ್ತಿ ಮತ್ತು ನಿರ್ಭಯತೆ–ಇದು ನನ್ನ ಸಂದೇಶ. ಸಿಪಾಯಿ ದಂಗೆಯ ಸಮಯದಲ್ಲಿ ಶತ್ರುಗಳ ಬಲಾತ್ಕಾರಕ್ಕೆ ತಲೆಬಾಗದೆ ಮೌನವಾಗಿಯೇ ಇದ್ದು, ಕಡೆಗೆ ತನ್ನನ್ನು ಇರಿದು ಕೊಂದವನನ್ನೂ ‘ನೀನೂ ಕೂಡ ಅವನೇ!’ ಎನ್ನುತ್ತ ಪ್ರಾಣಬಿಟ್ಟ ಆ ಮಹಾ ಸಾಧುಪುರುಷನೇ ನನ್ನ ಆದರ್ಶ.

“ಆದರೆ ಉಪನಿಷತ್ತಿನ ಆಧಾರದ ಮೇಲೆ ನಡೆಸುವ ನನ್ನ ಈ ಕಾರ್ಯವಿಧಾನದಲ್ಲಿ ಶ್ರೀರಾಮ ಕೃಷ್ಣರ ಪಾತ್ರವೇನು ಎಂದು ನೀನು ಕೇಳಬಹುದು. ಅವರೇ ನನ್ನ ದಾರಿದೀಪ, ಮಾರ್ಗದರ್ಶಕ! ನಮ್ಮ ಅರಿವಿಗೇ ಬಾರದೆ ಗೋಪ್ಯವಾಗಿ ಮಾರ್ಗದರ್ಶನ ನೀಡುತ್ತಿರುವವರು ಅವರೇ. ಅವರಿಗೆ ಇಂಗ್ಲೆಂಡಿನ ಬಗ್ಗೆಯಾಗಲಿ, ಇಂಗ್ಲಿಷರ ಬಗ್ಗೆಯಾಗಲಿ ಏನೇನೂ ತಿಳಿದಿರಲಿಲ್ಲ. ಆದರೆ ಅವರು ಅದ್ಭುತವಾದ ಜೀವನವನ್ನು ನಡೆಸಿದರು, ನಾನದರ ಅರ್ಥವನ್ನು ಅರಿತೆ. ಅವರು ಒಮ್ಮೆಯಾದರೂ ಒಬ್ಬರನ್ನು ಜರೆದವರಲ್ಲ, ಖಂಡಿಸಿದವರಲ್ಲ. ಒಮ್ಮೆ ನಾನು ವಾಮಾಚಾರವನ್ನು ಖಂಡಿಸುತ್ತ ಮೂರು ಗಂಟೆಗಳ ಕಾಲ ತೃಪ್ತಿಯಾಗುವಷ್ಟು ಬೈದೆ. ಅವರು ಎಲ್ಲವನ್ನೂ ಮಾತಿಲ್ಲದೆ ಕೇಳಿಸಿ ಕೊಂಡರು. ನಾನು ಮುಗಿಸಿದ ಮೇಲೆ ‘ಇರಲಿ ಇರಲಿ, ಪ್ರತಿಯೊಂದು ಮನೆಗೂ ಒಂದು ಹಿತ್ತಲ ಬಾಗಿಲು ಇರಬಹುದಲ್ಲವೆ?’ ಎಂದರು ಆ ಮುದುಕ!

“ಇಲ್ಲಿಯವರೆಗೆ ನಮ್ಮ ಹಿಂದೂಧರ್ಮಕ್ಕೆ ತಿಳಿದಿರುವುದು ಎರಡೇ ಶಬ್ದಗಳು–ತ್ಯಾಗ ಮತ್ತು ಮೋಕ್ಷ. ಇದೇ ಅದರ ದೊಡ್ಡ ದೋಷ. ಅಲ್ಲಿರುವುದು ಮೋಕ್ಷ ಮಾತ್ರ. ಗೃಹಸ್ಥರಿಗೆ ಏನೂ ದಾರಿಯೇ ಇಲ್ಲ! ಆದರೆ ನಾನು ನೆರವಾಗಬೇಕೆಂದಿರುವುದು ಈ ಗೃಹಸ್ಥರಿಗೇ. ಏಕೆಂದರೆ ಎಲ್ಲರ ಆತ್ಮವೂ ಒಂದೇ ಅಲ್ಲವೆ? ಪ್ರತಿಯೊಂದು ಜೀವಿಯ ಗುರಿಯೂ ಅದೇ ಅಲ್ಲವೆ?”

ಸ್ವಾಮೀಜಿಯವರ ಮಾತುಗಳನ್ನೆಲ್ಲ ಗಮನವಿಟ್ಟು ಕೇಳುತ್ತಿದ್ದ ನಿವೇದಿತಾ ಅದನ್ನು ಕುರಿತು ಬರೆಯುತ್ತಾಳೆ, “ಆ ಕ್ಷಣಕ್ಕೆ ನನಗೆ ಅನ್ನಿಸಿತು, ಮತ್ತು ಆಲೋಚಿಸದಂತೆಲ್ಲ ದೃಢವಾಯಿತು– ಏನೆಂದರೆ ನನ್ನ ಗುರುದೇವನ ಆ ಮಾತುಗಳನ್ನು ಕೇಳಿದ್ದೇ ಸಾಕು; ನನ್ನ ಇಡೀ ಪ್ರಯಾಣ ಸಾರ್ಥಕವಾಯಿತು” ಎಂದು.

ಹೀಗೆ ಸ್ವಾಮೀಜಿ ನಿವೇದಿತೆಗೆ ಅಸಂಖ್ಯಾತ ಅಮೂಲ್ಯ ವಿಷಯಗಳನ್ನು ತಿಳಿಸಿದರು. ಆದರೆ ಅವರು ಯಾವಾಗಲೂ ಇಂತಹ ಗಂಭೀರ ಚಿಂತನೆಯಲ್ಲೆ ತೊಡಗಿರುತ್ತಿದ್ದರು ಎಂದಲ್ಲ. ಸಾರ್ವ ಜನಿಕ ಜೀವನದ ಬಿಸಿಯಿಂದ ಸ್ವಲ್ಪ ಬಿಡುಗಡೆ ಹೊಂದಿದ್ದ ಅವರು ಆಗಾಗ ತಮ್ಮ ಗುರುಭಾಯಿ ಹಾಗೂ ಶಿಷ್ಯೆಯೊಡನೆ ಹಾಸ್ಯ ಚಟಾಕಿಗಳನ್ನು ಹಾರಿಸುತ್ತ ನಗುವಿನ ಲಹರಿಯಲ್ಲಿ ಈಜಾಡುತ್ತಿದ್ದರು.

