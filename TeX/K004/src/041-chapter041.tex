
\chapter{“ಇದೇ ಅತ್ಯುನ್ನತ ಸತ್ಯ”}

\noindent

೧೯೦೧ರ ಕೊನೆಯ ಭಾಗದಲ್ಲಿ ಮಠದ ನೆಲವನ್ನು ಸಮತಟ್ಟುಗೊಳಿಸಲು ಬಹಳ ಸಂಖ್ಯೆಯ ಸಂತಾಲ ಜನಾಂಗದ ಕೂಲಿಗಳನ್ನು ನೇಮಿಸಿಕೊಳ್ಳಲಾಗಿತ್ತು. ಇವರೆಲ್ಲ ಈಗಿನ ಬಿಹಾರ್ ರಾಜ್ಯದ ಸಂತಾಲ್ ಪರಗಣದ ಗುಡ್ಡಗಾಡು ಜನರು. ನಾಗರಿಕತೆಯ ಸಂಪರ್ಕಕ್ಕೆ ಬಂದಿದ್ದರೂ ಅದರ ಪ್ರಭಾವಕ್ಕೆ ಒಳಗಾಗದೆ ತಮ್ಮ ಸರಳತನವನ್ನು ಉಳಿಸಿಕೊಂಡಿದ್ದರು. ಸ್ವಾಮೀಜಿ ಅವರೊಂದಿಗೆ ಸಹಜವಾಗಿ ಬೆರೆತು ಮಾತುಕತೆಯಾಡುತ್ತ ಕುಳಿತುಬಿಡುತ್ತಿದ್ದರು. ಅವರ ಕಷ್ಟಸುಖಗಳನ್ನು ಕೇಳಿ ತಿಳಿದುಕೊಳ್ಳುತ್ತಿದ್ದರು. ಆ ಕೂಲಿಗಳೂ ಕೂಡ ತಮ್ಮನ್ನು ಇಷ್ಟೊಂದು ಆತ್ಮೀಯತೆಯಿಂದ ಮಾತನಾಡಿಸುವ ಈ ಸ್ವಾಮಿಗಳ ಬಗ್ಗೆ ತುಂಬ ಭಕ್ತಿವಿಶ್ವಾಸ ತಾಳಿದ್ದರು.

ಇವರ ಪೈಕಿ ಕೆಷ್ಟೊ (ಎಂದರೆ ಕೃಷ್ಣ) ಎಂಬವನ ಬಗ್ಗೆ ಸ್ವಾಮೀಜಿಯವರಿಗೆ ವಿಶೇಷ ಮಮತೆ. ಈತ ಮುದುಕ. ಸ್ವಾಮೀಜಿ ಇವರುಗಳೊಂದಿಗೆ ಮಾತನಾಡುತ್ತ ನಿಂತರೆ ಇವನು, “ಸ್ವಾಮಿ, ನಾವು ಕೆಲಸ ಮಾಡುವಾಗ ನೀವು ಮಾತಾಡಿಸಬೇಡಿ. ನಾವು ಮಾತಾಡುತ್ತ ನಿಂತರೆ ಕೆಲಸ ಸಾಗುವುದಿಲ್ಲ. ಆಮೇಲೆ ಮೇಸ್ತ್ರಿ ಸ್ವಾಮಿ (ಸ್ವಾಮಿ ಅದ್ವೈತಾನಂದರು) ನಮ್ಮನ್ನು ತರಾಟೆಗೆ ತಗೋತಾರೆ” ಎನ್ನುತ್ತಿದ್ದ. ಇದನ್ನು ಕೇಳಿದಾಗ ಅವರ ಅಸಹಾಯಕತೆಯನ್ನು ಕಂಡು ಸ್ವಾಮೀಜಿಯವರಿಗೆ ತುಂಬ ದುಃಖವಾಗುತ್ತಿತ್ತು. “ಆ ಸ್ವಾಮಿಗಳು ನಿಮ್ಮನ್ನೇನೂ ಬೈಯುವುದಿಲ್ಲ, ಚಿಂತಿಸಬೇಡಿ” ಎಂದು ಭರವಸೆ ನೀಡುತ್ತಿದ್ದರು.

ಒಂದು ಸಲ ಸ್ವಾಮೀಜಿ ಹೀಗೆಯೇ ಮಾತನಾಡುತ್ತಿದ್ದಾಗ ಅವರನ್ನು ನೋಡಲು ಯಾರೋ ದೊಡ್ಡಮನುಷ್ಯರು ಬಂದ ವಿಷಯ ತಿಳಿಯಿತು. ಆಗ ಸ್ವಾಮೀಜಿ, “ನಾನು ಈಗಲೇ ಬರಲು ಸಾಧ್ಯವಿಲ್ಲ. ನಾನೀಗ ಇವರೊಂದಿಗೆ ಮಾತನಾಡುತ್ತ ಬಹಳ ಆನಂದದಿಂದಿದ್ದೇನೆ. ಅವರು ಸ್ವಲ್ಪ ಕಾಯಲಿ” ಎಂದು ಹೇಳಿಕಳಿಸಿದರು.

ಈ ಜನರ ದುಃಖಸಂಕಟಗಳ ಕತೆಗಳನ್ನು ಕೇಳುತ್ತ ಕೆಲವೊಮ್ಮೆ ಸ್ವಾಮೀಜಿಯವರ ಕಣ್ಣಲ್ಲಿ ನೀರು ಚಿಮ್ಮುತ್ತಿತ್ತು. ಆಗ ಕೆಷ್ಟೊ ಹೇಳುತ್ತಿದ್ದ, “ಇನ್ನು ನೀವು ಹೋಗಿ ಸ್ವಾಮಿ. ನಾವಿನ್ನುಮೇಲೆ ನಿಮ್ಮಕೈಲಿ ನಮ್ಮ ಕಷ್ಟಗಳನ್ನು ಹೇಳಿಕೊಳ್ಳುವುದಿಲ್ಲ. ಅದನ್ನೆಲ್ಲ ಕೇಳುತ್ತ ನೀವು ಅತ್ತೇ ಬಿಡುತ್ತೀರಿ.”

ಸ್ವಾಮೀಜಿಯವರು ಈ ಸರಳಜೀವಿಗಳೊಂದಿಗೆ ಬೆರೆತು ಆನಂದದಿಂದ ಮಾತನಾಡುವುದನ್ನು ಕಂಡು ಅವರ ಸೋದರಸಂನ್ಯಾಸಿಗಳೂ ಶಿಷ್ಯರೂ ಸಂತೋಷಪಡುತ್ತಿದ್ದರು. ಅವಿಶ್ರಾಂತವಾಗಿ ದುಡಿದ ಸ್ವಾಮೀಜಿಯವರ ಮನಸ್ಸು ಬೇರೆಡೆಗೆ ಹರಿದು ಸ್ವಲ್ಪ ವಿಶ್ರಾಂತಿ ಸಿಗುವಂತಾಗುವುದಲ್ಲ ಎನ್ನುವುದು ಇವರ ಸಂತೋಷಕ್ಕೆ ಕಾರಣ. ಆದರೆ ಅಲ್ಲಿ ಅದಕ್ಕಿಂತ ಮಹತ್ವಪೂರ್ಣವಾದ, ಗಮನಾರ್ಹವಾದ ಮತ್ತೊಂದು ಅಂಶವಿತ್ತು–ಸ್ವಾಮೀಜಿಯವರು ಆ ಕೂಲಿಗಳೊಂದಿಗೆ ಕೇವಲ ಮಾನವೀಯತೆಯ ನೆಲೆಗಟ್ಟಿನಮೇಲೆ ಸರಿಸಮವಾಗಿ ನಿಂತು ಅವರೊಂದಿಗೆ ನಗುವ-ಅಳುವ ದೃಶ್ಯ ಪ್ರತಿಯೊಬ್ಬನಿಗೂ ಒಂದು ದೊಡ್ಡ ಪಾಠ. ಮಾನವನ ಭ್ರಾತೃತ್ವದ ಬೋಧನೆಯನ್ನು ಸ್ವಾಮೀಜಿ ಸ್ವಯಂ ಆಚರಿಸುವ ಪರಿಯನ್ನು ಕಣ್ಣಾರೆ ಕಂಡ ಶಿಷ್ಯರಿಗೆಲ್ಲ ಅದೊಂದು ಮಾರ್ಗದರ್ಶನ.

ಈ ಕೂಲಿಗಳಿಗೆಲ್ಲ ಒಂದು ದಿನ ಹೊಟ್ಟೆ ತುಂಬ ಹಬ್ಬದೂಟವನ್ನು ಬಡಿಸಬೇಕೆಂಬ ಇಚ್ಛೆ ಸ್ವಾಮೀಜಿಯವರಿಗೆ ಉಂಟಾಯಿತು. ಅವರು ಕೆಷ್ಟೊನನ್ನು ಕೇಳಿದರು, “ಏನಪ್ಪ, ನಿಮಗೆಲ್ಲ ಒಂದು ದಿನ ಒಳ್ಳೇ ಹಬ್ಬದೂಟ ಹಾಕಿಸಬೇಕು ಅಂತ ನನಗೆ ಆಸೆಯಾಗಿದೆ. ನೀನೇನು ಹೇಳುತ್ತೀ?” ಆತ ಇದಕ್ಕೆ ಸಂತೋಷದಿಂದ ಒಪ್ಪಿಕೊಂಡಿರಬೇಕಲ್ಲವೆ? ಇಲ್ಲ, ಹಾಗಾಗಲಿಲ್ಲ. ಅವನೆಂದ, “ಇಲ್ಲ ಬಾಬಾಜಿ, ಬೇರೆ ಜಾತಿಯವರು ಉಪ್ಪು ಹಾಕಿ ಮಾಡಿದ ಏನನ್ನಾದರೂ ಉಂಡರೆ ನಮಗೆ ಜಾತಿಯಿಂದ ಬಹಿಷ್ಕಾರ ಹಾಕಿಬಿಡುತ್ತಾರೆ!” ಈ ಸಮಸ್ಯೆಯನ್ನು ಸ್ವಾಮೀಜಿ ಸುಲಭವಾಗಿ ಪರಿ ಹರಿಸಿದರು–ಉಪ್ಪಿಲ್ಲದ ಅಡಿಗೆಯನ್ನೇ ಮಾಡುವುದು! ಆದರೆ ಉಪ್ಪಿಲ್ಲದೆ ಅದೆಂಥ ಅಡಿಗೆ? ಅದನ್ನು ಉಣ್ಣುವುದಾದರೂ ಹೇಗೆ? ಅದಕ್ಕೇನು, ಉಪ್ಪನ್ನು ಪ್ರತ್ಯೇಕವಾಗಿ ಬಡಿಸಿದರಾಯಿತು! ಇದಕ್ಕೆ ಕೆಷ್ಟೊನೂ ಅವನ ಸಂಗಡಿಗರೂ ಒಪ್ಪಿಕೊಂಡರು.

ಸರಿ, ಅವರಿಗಾಗಿ ಮಠದಲ್ಲಿ ಒಂದು ಭಾರೀ ಔತಣವನ್ನೇ ಏರ್ಪಡಿಸಲಾಯಿತು. ಪೂರಿ ಸಾಗು, ಸಿಹಿತಿಂಡಿಗಳು, ಸಿಹಿಮೊಸರು, ಮತ್ತಿತರ ಅನೇಕ ಭಕ್ಷ್ಯಭೋಜ್ಯಗಳನ್ನೆಲ್ಲ ಬಡಿಸಲಾಯಿತು. ಪಾಪ, ಗುಡ್ಡಗಾಡಿನ ಆ ಬಡಜನ ಅವನ್ನೆಲ್ಲ ಅಲ್ಲಿಯವರೆಗೆ ಕಣ್ಣಿಂದ ಕಂಡೂ ಇರಲಿಲ್ಲ. ಸ್ವಾಮೀಜಿಯವರು ಖುದ್ದಾಗಿ ನಿಂತು ಪ್ರತಿಯೊಬ್ಬನಿಗೂ ಯಾವುದು ಹೆಚ್ಚು ಇಷ್ಟವಾಯಿತೆಂದು ವಿಚಾರಿಸಿ, ಅದನ್ನವನಿಗೆ ಬೇಕೆಂಬಷ್ಟು ಬಡಿಸಿದರು. ಆ ಮುಗ್ಧ ಜೀವಿಗಳಿಗೆ ಹೊಟ್ಟೆಯೊಂದಿಗೆ ಹೃದಯವೂ ತುಂಬಿಹೋಯಿತು. ಅವರು ಮತ್ತೆಮತ್ತೆ ಆನಂದಾಶ್ಚರ್ಯಗಳಿಂದ, “ಸ್ವಾಮಿ, ಇವುಗಳನ್ನೆಲ್ಲ ಎಲ್ಲಿಂದ ತಂದಿರಪ್ಪ? ನಾವು ಇಂಥಾದ್ದನ್ನೆಲ್ಲ ಕಣ್ಣಲ್ಲಿ ಕಂಡೇ ಇಲ್ಲ!” ಎಂದು ಉದ್ಗರಿಸಿದರು. ಊಟವಾದ ಮೇಲೆ ಸ್ವಾಮೀಜಿ ಅವರಿಗೆ, “ನೀವೆಲ್ಲ ಸಾಕ್ಷಾತ್ ನಾರಾಯಣರು. ನಿಮಗೆ ಉಣಬಡಿಸುವುದರ ಮೂಲಕ ನಾನು ಭಗವಂತನಿಗೇ ನೈವೇದ್ಯ ಮಾಡಿದೆ” ಎಂದು ಹೇಳಿದರು. ಆನಂದಿತರಾದ ಸಂತಾಲರು ಸ್ವಾಮೀಜಿಯವರಿಗೆ ಕೈಮುಗಿದು ಬೀಳ್ಕೊಂಡರು.

ಅವರು ಹೋದಮೇಲೆ ಸ್ವಾಮೀಜಿ ಶರಚ್ಚಂದ್ರನ ಬಳಿ ನುಡಿದರು, “ಅವರಲ್ಲಿ ನಾನು ಸಾಕ್ಷಾತ್ ನಾರಾಯಣನನ್ನು ಪ್ರತ್ಯಕ್ಷ ಕಂಡೆ! ಅವರು ನಿಜಕ್ಕೂ ಎಂತಹ ಸರಳ-ಮುಗ್ಧ ಜೀವಿಗಳು! ಎಷ್ಟು ನಿಷ್ಕಪಟಿಗಳು!” ಬಳಿಕ ಆಶ್ರಮವಾಸಿಗಳೊಂದಿಗೆ ಅದೇ ಭಾವದಲ್ಲಿ ಮಾತನಾಡುತ್ತ ಹೇಳಿದರು, “ಅವರೆಷ್ಟು ಮುಗ್ಧರು ನೋಡಿದಿರಾ? ಅವರ ದುಃಖವನ್ನು ಕಿಂಚಿತ್ತಾದರೂ ಪರಿಹರಿಸಲು ನಿಮ್ಮಿಂದ ಸಾಧ್ಯವೆ? ಇಲ್ಲದಿದ್ದರೆ ಈ ಕಾಷಾಯವಸ್ತ್ರವನ್ನು ಧರಿಸಿಯಾದರೂ ಏನು ಪ್ರಯೋ ಜನ? ಇತರರ ಒಳಿತಿಗಾಗಿ ಸರ್ವಸ್ವವನ್ನೂ ತ್ಯಜಿಸುವುದೇ ನಿಜವಾದ ಸಂನ್ಯಾಸ. ಎಷ್ಟೋ ಸಲ ನಾನು ನನ್ನಷ್ಟಕ್ಕೇ ಆಲೋಚಿಸುತ್ತಿರುತ್ತೇನೆ–‘ಈ ಆಶ್ರಮ ಇತ್ಯಾದಿಗಳನ್ನೆಲ್ಲ ಕಟ್ಟುವುದರಿಂ ದೇನು ಪ್ರಯೋಜನ? ಅವುಗಳನ್ನೆಲ್ಲ ಮಾರಿ ಹಣವನ್ನೇಕೆ ಬಡವರಿಗೆ ಹಂಚಿಬಿಡಬಾರದು? ತರುತಲವಾಸಿಗಳಾದ ನಾವು ಈ ಕಟ್ಟಡಗಳನ್ನೆಲ್ಲ ಏಕೆ ಕಟ್ಟಿಕೊಳ್ಳಬೇಕು? ಅಯ್ಯೋ! ನಮ್ಮ ದೇಶಬಾಂಧವರಿಗೆ ಹೊಟ್ಟೆ ಬಟ್ಟೆಗೂ ಗತಿಯಿಲ್ಲದಿರುವಾಗ, ನಮ್ಮ ಬಾಯಿಗೆ ತುತ್ತನ್ನಿಡುವು ದಾದರೂ ಹೇಗೆ?’ ಅಂತ. ನಮ್ಮ ಈ ವಿದ್ಯೆಯ ಹಾಗೂ ಶಾಸ್ತ್ರಜ್ಞಾನದ ಅಹಂಕಾರವನ್ನೆಲ್ಲ ತ್ಯಜಿಸಿ, ಸ್ವಂತ ಮುಕ್ತಿಗಾಗಿ ಮಾಡುವ ಸಾಧನೆಗಳನ್ನೆಲ್ಲ ಬಿಟ್ಟು, ಹಳ್ಳಿಯಿಂದ ಹಳ್ಳಿಗೆ ಹೋಗುತ್ತ ಬಡಜನರ ಸೇವೆಗಾಗಿ ಜೀವನವನ್ನು ಮುಡಿಪಾಗಿಡೋಣ. ನಮ್ಮ ಶೀಲಬಲ, ಆತ್ಮಬಲಗಳಿಂದ ಮತ್ತು ತಪೋಮಯ ಜೀವನದ ಬಲದಿಂದ ನಾವು ಜನಸಾಮಾನ್ಯರ ಬಗ್ಗೆ ಶ್ರೀಮಂತನಿಗಿರುವ ಕರ್ತವ್ಯವನ್ನು ಮನದಟ್ಟು ಮಾಡಿಸೋಣ. ಮತ್ತು ದೀನಾರ್ತರ ಸೇವೆಗಾಗಿ ಅವನು ತನ್ನ ಹಣ ವನ್ನು ಸ್ವಲ್ಪವಾದರೂ ವಿನಿಯೋಗಿಸುವಂತೆ ಮಾಡೋಣ. ಅಯ್ಯೋ! ಈ ಕೆಳವರ್ಗದವರ, ಬಡ ವರ ಹಾಗೂ ಕಾರ್ಮಿಕರ ಬಗ್ಗೆ ನಮ್ಮ ದೇಶದಲ್ಲಿ ಯಾರೂ ಚಿಂತಿಸುವುದಿಲ್ಲವಲ್ಲ! ಈ ಅಸಂ ಖ್ಯಾತ ಜೀತದವರೇ ನಮ್ಮ ದೇಶದ ಬೆನ್ನೆಲುಬು, ನಮ್ಮ ಅನ್ನದಾತರು. ಇವರ ಬಗ್ಗೆ ಮರುಕಪಡು ವಂಥವನು, ಇವರ ಸುಖದುಃಖಗಳಲ್ಲಿ ಭಾಗಿಯಾಗುವಂಥವನು ನಮ್ಮ ದೇಶದಲ್ಲಿ ಯಾವನಿ ದ್ದಾನೆ? ಮದ್ರಾಸ್ ಪ್ರಾಂತದಲ್ಲಿ (ಈಗಿನ ಕೇರಳ) ಮೇಲ್ಜಾತಿಯ ಹಿಂದೂಗಳ ಅನಾದರಕ್ಕೆ ಗುರಿಯಾದ ಹೊಲೆಯರು ಹೇಗೆ ಸಾವಿರಗಟ್ಟಲೆ ಸಂಖ್ಯೆಯಲ್ಲಿ ಕ್ರೈಸ್ತರಾಗಿ ಮತಾಂತರ ಹೊಂದು ತ್ತಿದ್ದಾರೆ ನೋಡಿ! ಅವರು ಕ್ರೈಸ್ತಧರ್ಮವನ್ನು ಅಪ್ಪಿಕೊಳ್ಳುವಂತೆ ಮಾಡುವುದು ಹೊಟ್ಟೆಯ ಹಸಿವು ಮಾತ್ರವೇ ಎಂದು ತಿಳಿಯಬೇಡಿ. ನಿಜವಾದ ಕಾರಣವೇನೆಂದರೆ ಅವರಿಗೆ ನಿಮ್ಮ ಸಹಾನುಭೂತಿ ದೊರೆಯುತ್ತಿಲ್ಲ ಎಂಬುದು. ಈಗ ದೇಶದಲ್ಲಿ ಸಹೋದರತ್ವದ ಭಾವನೆಯಾಗಲಿ, ಧರ್ಮಬುದ್ಧಿಯಾಗಲಿ ಏನಾದರೂ ಉಳಿದಿದೆಯೆ? ಇರುವುದೆಲ್ಲ ಏನಿದ್ದರೂ ‘ಮುಟ್ಟಬೇಡ’ ಎಂಬ ಧರ್ಮ ಮಾತ್ರವೇ. ‘ಓ ಬಡಜನರೇ, ಓ ಹಿಂದುಳಿದವರೇ, ದುಃಖಾರ್ತರೇ, ಬನ್ನಿ! ಶ್ರೀರಾಮಕೃಷ್ಣರ ಹೆಸರಿನಲ್ಲಿ ನಾವೆಲ್ಲ ಒಂದು!’ ಎಂದು ಕೂಗಿ ಕರೆದು ಅವರೆಲ್ಲರನ್ನೂ ಒಂದಾ ಗಿಸಬೇಕು ಎಂದು ನನಗೆಷ್ಟು ಆಸೆಯಿದೆ ಗೊತ್ತೆ? ಅವರನ್ನೆಲ್ಲ ಮೇಲೆತ್ತಿದ ಹೊರತು ಈ ನಮ್ಮ ಮಾತೃಭೂಮಿ ಎಂದಿಗೂ ಎಚ್ಚರಗೊಳ್ಳಲಾರದು. ನಾವವರಿಗೆ ಹೊಟ್ಟೆಗೂ ಬಟ್ಟೆಗೂ ಕೊಡಲಾಗ ದಿದ್ದರೆ ನಮ್ಮಿಂದೇನು ಪ್ರಯೋಜನ? ಆಹಾ! ಈ ಹಿಂದುಳಿದ ಸೋದರರಿಗೆ ಜಗತ್ತಿನ ರೀತಿನೀತಿಗಳು ಗೊತ್ತಿಲ್ಲ. ಆದ್ದರಿಂದ ಅವರು ಹಗಲಿರುಳು ದುಡಿದರೂ ಸರಿಯಾಗಿ ಬದುಕಲೂ ಅಸಮರ್ಥರಾಗಿ ಉಳಿದಿದ್ದಾರೆ. ನೀವು ನಿಮ್ಮೆಲ್ಲ ಶಕ್ತಿಯನ್ನೂ ಒಗ್ಗೂಡಿಸಿಕೊಂಡು ಅವರ ಕಣ್ಣಿನ ಪರೆಯನ್ನು ಕಳಚಲು ಪ್ರಯತ್ನಿಸಿ. ನನ್ನಲ್ಲಿರುವ ಅದೇ ಪರಬ್ರಹ್ಮವಸ್ತುವೇ ಅವರಲ್ಲೂ ಇರುವು ದನ್ನು ನಾನು ಹಗಲಿನಷ್ಟು ಸ್ಫುಟವಾಗಿ ಕಾಣುತ್ತಿದ್ದೇನೆ. ಆದರೆ ಅದರ ವ್ಯಕ್ತತೆಯ ಪ್ರಮಾಣದಲ್ಲಿ ಮಾತ್ರ ಸ್ವಲ್ಪ ಅಂತರವಿದೆ ಅಷ್ಟೆ. ಇಡೀ ಜಗತ್ತಿನ ಚರಿತ್ರೆಯಲ್ಲಿ ಯಾವುದೇ ರಾಷ್ಟ್ರದ ನೆತ್ತರು ಅದರ ಶರೀರದಾದ್ಯಂತ ಸುಗಮವಾಗಿ ಹರಿಯದೆ ಆ ರಾಷ್ಟ್ರ ಮೇಲೆದ್ದು ನಿಂತದ್ದನ್ನು ಕಂಡಿರು ವಿರಾ? ಶರೀರದ ಒಂದು ಅಂಗ ಪಾರ್ಶ್ವವಾಯುವಿಗೀಡಾದರೆ, ಉಳಿದ ಅಂಗಗಳು ಆರೋಗ್ಯ ವಾಗಿದ್ದರೂ ಆ ಶರೀರದಿಂದ ಹೆಚ್ಚಿನ ಕೆಲಸ ಸಾಧ್ಯವಿಲ್ಲ. ಇದನ್ನು ನೀವು ಖಚಿತವಾಗಿ ತಿಳಿಯಿರಿ.”

ಸ್ವಾಮೀಜಿಯವರು ಹೀಗೆ ಆವೇಶಭರಿತರಾಗಿ ಮಾತನಾಡಿದ್ದನ್ನು ಕೇಳಿದ ಒಬ್ಬ ಭಕ್ತ ಒಡಕುದನಿಯೆತ್ತಿದ: “ಆದರೆ ಸ್ವಾಮೀಜಿ, ಈ ದೇಶದ ಭಿನ್ನಭಿನ್ನ ಜಾತಿಮತಗಳ ಜನರ ಮಧ್ಯೆ ಸಾಮರಸ್ಯವನ್ನೂ ಸಹಕಾರವನ್ನೂ ತಂದು, ಅವರು ಒಂದೇ ಉದ್ದೇಶಕ್ಕಾಗಿ ಒಟ್ಟಾಗಿ ದುಡಿಯು ವಂತೆ ಮಾಡುವುದು ತೀರಾ ಕಷ್ಟದ ಕೆಲಸ...” ಆಗ ಸ್ವಾಮೀಜಿ ರೇಗಿ ನುಡಿದರು, “ನಿನಗೆ ಯಾವುದಾದರೂ ಕೆಲಸ ದುಸ್ಸಾಧ್ಯವೆನಿಸಿದರೆ ಇಲ್ಲಿಗೆ ಬರಲೇಬೇಡ. ಭಗವಂತನ ಕೃಪೆಯಿಂದ ಪ್ರತಿಯೊಂದು ಕೆಲಸವೂ ಸುಲಭಸಾಧ್ಯವಾಗುತ್ತದೆ. ನಿನ್ನ ಕೆಲಸವೇನಿದ್ದರೂ ಜಾತಿಮತ ಭೇದ ಗಳನ್ನೆಣಿಸದೆ ಬಡವರ-ದೀನರ ಸೇವೆ ಮಾಡುವುದು. ನಿನ್ನ ಕರ್ಮದ ಫಲದ ಬಗ್ಗೆ ಆಲೋಚಿಸಲು ನಿನಗೇನು ಹಕ್ಕು? ಕೆಲಸ ಮಾಡುತ್ತ ಹೋಗುವುದೇ ನಿನ್ನ ಕರ್ತವ್ಯ. ಜಗತ್ತಿನ ಇತಿಹಾಸವನ್ನು ಓದಿನೋಡು–ಪ್ರತಿಯೊಂದು ದೇಶದಲ್ಲಿಯೂ ಯಾವುದೋ ಒಂದು ಕಾಲದಲ್ಲಿ ಕೆಲವು ಮಹಾ ವ್ಯಕ್ತಿಗಳು ಮೇಲೆದ್ದು ರಾಷ್ಟ್ರದ ಕೇಂದ್ರವಾಗಿ ನಿಂತು, ತಮ್ಮ ಆಲೋಚನೆಗಳಿಂದ ಜನಜೀವನ ವನ್ನು ಪ್ರಭಾವಿತಗೊಳಿಸಿದರೆಂಬುದನ್ನು ಖಂಡಿತ ಕಾಣುವೆ. ನೀವೆಲ್ಲ ಬುದ್ಧಿವಂತ ಹುಡುಗರು. ಅಲ್ಲದೆ ನನ್ನ ಶಿಷ್ಯರೆಂದು ಹೇಳಿಕೊಳ್ಳುತ್ತೀರಿ. ನೀವೇನು ಸಾಧಿಸಿದ್ದೀರಿ ಹೇಳಿ? ಇತರರಿಗಾಗಿ ನಿಮ್ಮ ಒಂದು ಜನ್ಮವನ್ನು ಕೊಡಲು ಸಾಧ್ಯವಿಲ್ಲವೆ? ವೇದಾಂತವನ್ನು ಓದುವುದು, ಧ್ಯಾನಾಭ್ಯಾಸ ಮಾಡುವುದು ಮುಂತಾದುವೆಲ್ಲ ಮುಂದಿನ ಜನ್ಮಕ್ಕಿರಲಿ. ಇಂದು ಈ ಶರೀರ ಇತರರ ಸೇವೆಗಾಗಿ ಸವೆದು ಹೋಗಲಿ. ಆಗ ನೀವು ನನ್ನಲ್ಲಿಗೆ ಬಂದದ್ದು ವ್ಯರ್ಥವಾಗಲಿಲ್ಲವೆಂದು ನಾನು ತಿಳಿಯುತ್ತೇನೆ.”

ಬಳಿಕ ಸ್ವಾಮೀಜಿ ಮತ್ತೆ ನುಡಿದರು, “ಅಷ್ಟೆಲ್ಲ ತಪಸ್ಸಿನ ನಂತರ ನಾನು ಇದೇ ಅತ್ಯುನ್ನತ ಸತ್ಯವೆಂದು ಅರಿತಿದ್ದೇನೆ–ಪ್ರತಿಯೊಬ್ಬನಲ್ಲೂ ದೇವರಿದ್ದಾನೆ. ಅದನ್ನು ಬಿಟ್ಟರೆ ಬೇರೆ ದೇವರಿಲ್ಲ. ಎಲ್ಲ ಜೀವರ ಸೇವೆ ಮಾಡುವವನು ಖಂಡಿತವಾಗಿಯೂ ದೇವರ ಸೇವೆಯನ್ನೇ ಮಾಡುತ್ತಾನೆ.” ಮತ್ತೆ ಒಂದು ಕ್ಷಣ ಮೌನವಾಗಿದ್ದು ಬಳಿಕ ಶರಚ್ಚಂದ್ರನಿಗೆ ಹೇಳಿದರು, “ನಾನಿಂದು ಏನು ಹೇಳಿ ದೆನೋ ಅದನ್ನು ನಿನ್ನ ಹೃದಯದಲ್ಲಿ ಬರೆದಿಟ್ಟುಕೊ, ಅದನ್ನೆಂದೂ ಮರೆಯದಂತೆ ನೋಡಿಕೊ.”

ಈ ಎರಡು ಘಟನೆಗಳು ಇಂತಹ ಎಷ್ಟೋ ಘಟನೆಗಳಿಗೆ ಉದಾಹರಣೆಗಳಷ್ಟೆ. ಸ್ವಾಮೀಜಿ ತಮ್ಮ ಅನಾರೋಗ್ಯ ಯಾತನೆಗಳ ನಡುವೆಯೂ ಹೀಗೆ ಆಗಾಗ, ತಮ್ಮಲ್ಲಡಗಿದ್ದ ಅದ್ಭುತ ಶಕ್ತಿಯನ್ನು ಮೆರೆಸುತ್ತ ತಮ್ಮ ಶಿಷ್ಯರಿಗೂ ದೇಶಬಾಂಧವರಿಗೂ ತಮ್ಮ ಸಂದೇಶಗಳನ್ನು ನೀಡಿದರು. ಇದರಿಂದಾಗಿ ಅವರಿಗೆ ಸಾಕಷ್ಟು ಶ್ರಮವೇ ಆಗುತ್ತಿತ್ತು. ಆದರೆ ಅವರಲ್ಲಿ ಬಡಬಾಗ್ನಿ ಯಂತೆ ಜ್ವಲಿಸುತ್ತಿದ್ದ ಶಕ್ತಿಯನ್ನು ಅದುಮಿ ಹಿಡಿಯಬಲ್ಲವರಾರು? ಅದು ಪ್ರಜ್ವಲಿಸಿ ಇತರರನ್ನೂ ಪ್ರಜ್ವಲಗೊಳಿಸಬೇಕು. ಇಲ್ಲವೆ ತನ್ನನ್ನೇ ಅದು ಸ್ವಾಹಾಕಾರ ಮಾಡಬೇಕು.

