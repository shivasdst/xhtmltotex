
\chapter{ಅಪೂರ್ವ ಅಶರೀರವಾಣಿ}

\noindent

ಅಮರನಾಥ ಯಾತ್ರೆಗೆ ಹೋಗಿಬಂದ ಮೇಲೆ ಸ್ವಾಮೀಜಿಯವರ ವ್ಯಕ್ತಿತ್ವದಲ್ಲಿ ಒಂದು ಗಮನಾರ್ಹ ಬದಲಾವಣೆಯುಂಟಾಯಿತು. ಈಗ ಅವರ ಭಕ್ತಿ ಜಗನ್ಮಾತೆಯ ಮೇಲೆ ಕೇಂದ್ರಿತ ವಾಯಿತು. ಮಹಾದೇವನ ಮೇಲಿನ ಧ್ಯಾನದಿಂದುಂಟಾದ ಶಕ್ತಿಯ ಜಾಗವನ್ನು ಈಗ ಭಕ್ತಿ ತುಂಬಿ ಕೊಂಡಿತು. ಜಗನ್ಮಾತೆಯ ಮೇಲಿನ ಭಕ್ತಿಯಿಂದ ಪರವಶರಾದ ಸ್ವಾಮೀಜಿ, ತಮ್ಮ ಮುಸಲ್ಮಾನ ಅಂಬಿಗನ ನಾಲ್ಕು ವರ್ಷದ ಮಗಳನ್ನು ಸಾಕ್ಷಾತ್ ಉಮಾ ಎಂದು ಭಾವಿಸಿ ಪೂಜಿಸಿದರು. ಆ ದೃಶ್ಯ ತುಂಬ ಹೃದಯಸ್ಪರ್ಶಿಯಾಗಿತ್ತು. ರಾಮಪ್ರಸಾದನೆಂಬ ಬಂಗಾಳೀ ಸಂತ ಬರೆದಿರುವ ದೇವೀಪರವಾದ ಹಾಡುಗಳು ಅವರ ಬಾಯಲ್ಲಿ ಸದಾ ನಲಿದಾಡುತ್ತಿದ್ದುವು. ಈ ದಿನಗಳಲ್ಲಿ ಅವರಿಗೆ ಎತ್ತ ತಿರುಗಿದರೂ ಜಗನ್ಮಾತೆಯ ಸಾನ್ನಿಧ್ಯದ ಅನುಭವವಾಗುತ್ತಿತ್ತು. ಅವಳು ತಮ್ಮ ಕೋಣೆಯಲ್ಲೇ ಸಾಕ್ಷಾತ್ಕಾಗಿ ಇರುವಂತೆ ಭಾಸವಾಗುತ್ತಿತ್ತು. ಸಾಕ್ಷಾತ್ ಜಗನ್ಮಾತೆ ಅಥವಾ ಶ್ರೀರಾಮಕೃಷ್ಣರು ತಮ್ಮ ಕೈಹಿಡಿದು ನಡೆಸುತ್ತಿದ್ದಂತೆ ಅವರಿಗೆ ಸ್ಪಷ್ಟ ಅನುಭವವಾಗುತ್ತಿತ್ತು. ಕೆಲಕಾಲದ ಹಿಂದೆ ಅವರ ತನುಮನಗಳು ಶಿವಮಯವೇ ಆಗಿಬಿಟ್ಟಿದ್ದವಂತೆ, ಈಗ ಅವರು ಜಗನ್ಮಾತೆಯ ಭಾವದಿಂದ ಆವೃತರಾಗಿಬಿಟ್ಟಿದ್ದರು.

ಕ್ರಮೇಣ ಅವರ ಮನಸ್ಸು ಜಗನ್ಮಾತೆಯ ಭಾವದಲ್ಲಿ ಎಷ್ಟು ತೀವ್ರವಾಗಿ ಮುಳುಗಿಬಿಟ್ಟಿ ತೆಂದರೆ ಅವರೀಗ ಬೇರೆ ಆಲೋಚನೆಯನ್ನೇ ಮಾಡಲು ಅಸಮರ್ಥರಾದರು. ಅವರೇ ಈ ವಿಷಯವಾಗಿ ಒಂದು ಬಗೆಯ ಅಸಹನೆಯ ದನಿಯಲ್ಲಿ ಹೇಳುತ್ತಾರೆ, “ಜಗನ್ಮಾತೆಯ ಭಾವ ನನಗೆ ಒಂದು ರೋಗದಂತೆ ಅಂಟಿಕೊಂಡಿದೆ. ಇದು ನನಗೆ ವಿಶ್ರಾಂತಿ-ನಿದ್ರೆಗಳಿಲ್ಲದಂತೆ ಮಾಡಿಬಿಟ್ಟಿದೆ. ಎಷ್ಟೋ ಸಲ ಈ ಭಾವವು ಒಬ್ಬ ಮನುಷ್ಯನ ದನಿಯಷ್ಟೇ ಸ್ಪಷ್ಟವಾಗಿ ನನ್ನ ಕಿವಿಯಲ್ಲಿ ಗುಂಯಿಗುಡುತ್ತಿರುತ್ತದೆ.” ಈ ದಿನಗಳಲ್ಲೇ ಅವರಿಗೆ ಮತ್ತೊಂದು ಅಲೌಕಿಕ ಅನುಭವವಾಯಿತು. ಹಿಂದೆ ಅವರು ನರೇಂದ್ರನಾಗಿದ್ದಾಗ ಒಮ್ಮೆ ಶ್ರೀರಾಮಕೃಷ್ಣರ ಆಣತಿ ಯಂತೆ ಕಾಳಿಯ ಮುಂದೆ ತನ್ನ ಕಷ್ಟಕ್ಕೇನಾದರೂ ಪರಿಹಾರ ತೋರುವಂತೆ ಬೇಡಲು ಹೋದಾಗ ಆದಂತಹ ಅದ್ಭುತ ಅನುಭವಕ್ಕೆ ಸಮನಾದದ್ದು ಅವರ ಇಂದಿನ ಈ ಅನುಭವ.

ಆ ಸಮಯದಲ್ಲಿ ಅವರು ತಮ್ಮ ದೋಣಿಯಲ್ಲಿ ಒಂದು ಏಕಾಂತ ಸ್ಥಳಕ್ಕೆ ಹೋಗಿದ್ದರು. ಆಗ ಅವರು ಭೇಟಿಯಾಗುತ್ತಿದ್ದ ಒಬ್ಬರೇ ವ್ಯಕ್ತಿಯೆಂದರೆ ಬ್ರಾಹ್ಮ ಸಮಾಜೀಯರಾದ ಒಬ್ಬ ವೈದ್ಯರು. ಆ ವೈದ್ಯರು ಸ್ವಾಮೀಜಿಯವರಲ್ಲಿ ವಿಶೇಷ ಪೂಜ್ಯ ಬುದ್ಧಿ ತಾಳಿದವರು. ಪ್ರತಿದಿನವೂ ಸ್ವಾಮೀಜಿಯವರನ್ನು ಭೇಟಿಯಾಗಿ ಅವರಿಗೆ ಬೇಕಾದುದನ್ನೆಲ್ಲ ಮಾಡಿಕೊಡುತ್ತಿದ್ದರು. ಆದರೆ ಅವರು ತಮ್ಮ ಭಾವದಲ್ಲೇ ತನ್ಮಯರಾಗಿದ್ದುದನ್ನು ಇಲ್ಲವೆ ಧ್ಯಾನಮಗ್ನರಾಗಿದ್ದುದನ್ನು ಕಂಡರೆ ಅವರಿಗೆ ತೊಂದರೆ ಕೊಡದೆ ಹಿಂದಿರುಗುತ್ತಿದ್ದರು. ಸ್ವಾಮೀಜಿಯವರು ಭಕ್ತಿಯ ಉತ್ಕಟಾವಸ್ಥೆ ಗೇರಿರುವುದನ್ನು ಈ ವೈದ್ಯರು ಗಮನಿಸಿದ್ದರು. ಜಗನ್ಮಾತೆಯ ಮೇಲಿನ ಪ್ರೇಮದ ಕಾವಿನಿಂದ ಸ್ವಾಮೀಜಿಯರ ಪ್ರಜ್ಞೆ ಉನ್ಮತ್ತವಾಗಿತ್ತು; ಮನಸ್ಸು ಭಾವಸಮಾಧಿಯ ಸ್ತರಕ್ಕೇರಿತ್ತು. ಈಗ ಜಗನ್ಮಾತೆಯ ದರ್ಶನವಾಗಲೇಬೇಕು!... 

ಒಂದು ಸಂಜೆ ಅವರಿಗೆ ಆ ದಿವ್ಯದರ್ಶನವಾಯಿತು. ಅಂದು ಅವರು ತಮ್ಮ ಮನಸ್ಸನ್ನು, ತಮ್ಮ ಸಂಪೂರ್ಣ ಅಂತಃಕರಣವನ್ನು, ತಾಯಿ ಕಾಳಿಯ ಮೇಲೆ, ಕಾಳಗತ್ತಲೆಯಂತಿರುವ ಕಾಳಿಯ ಮೆಲೆ, ರುದ್ರರೂಪಿಣಿಯಾದ-ರುದ್ರಾಣಿಯಾದ ಕಾಳಿಯ ಮೇಲೆ, ದೃಗ್ಗೋಚರವಾದ ಈ ಸಮಸ್ತ ವಿಶ್ವದ ಹಿನ್ನೆಲೆಯಲ್ಲಿ ಆಧಾರಭೂತೆಯಾಗಿ ನಿಂತಿರುವ ಕಾಳಿಯ ಮೇಲೆ ಕೇಂದ್ರೀಕರಿಸಿದ್ದರು. ಅಂದು ಹೊರಗೆಲ್ಲೆಲ್ಲೂ ಸ್ತಬ್ಧ-ನೀರವ-ಮೌನ. ಆದರೆ ಸ್ವಾಮೀಜಿಯವರ ಹೃದಯದೊಳಗೆ ಮಾತ್ರ ವಿಶ್ವವನ್ನೇ ಬಿರುಗಾಳಿಗೊಡ್ಡಿ ಧೂಳೀದೂಸರಗೈವ ತಾಯಿ ಕಾಳಿಯ ಭಾವ! ಇಂತಹ ಒಂದು ಭವ್ಯ ದರ್ಶನವನ್ನು ಅನುಭವಿಸುತ್ತಿದ್ದ ಸ್ವಾಮೀಜಿ \eng{Kali the Mother–“}ತಾಯಿ ಕಾಳಿ”\footnote{*ಅನುವಾದ: ಮುರಳೀಧರ} ಎಂಬ ಈ ದಿವ್ಯ ಕವನವನ್ನು ರಚಿಸಿದರು:

\begin{myquote}
ತಾರೆಗಳನು ಹೀರುತಿಹುದು\\ಕಾರಿರುಳಿನ ಗಗನ;\\ಮುಗಿಲು ಮುಗಿಲ ನುಂಗುತಿಹುದು;\\ಕತ್ತಲ ಕಂಪನನ!
\end{myquote}

\begin{myquote}
ಗರ್ಜರಿಸುತ ಗರ್ಗರಿಸುತ\\ಮುಗ್ಗರಿಸುತ ಗಾಳಿ\\ಮರಮರಗಳ ಮುರಿಮುರಿಯುತ\\ಹುಡಿಗೈಯುತ ಪಥದಿ\\ನುಗ್ಗುತಿಹುದು ಕಟ್ಟು ಕಡಿದು\\ಸೆರೆಮನೆಯಿಂ ಮುಕ್ತಿ ಪಡೆದ\\ಮರುಳರ ಪಡೆ ತೆರದಿ!
\end{myquote}

\begin{myquote}
ಆಗಸವನು ತಬ್ಬಲೆಂದು\\ಗಿರಿತರಂಗವೆಬ್ಬಿಸುತ್ತ\\ಸಾಗರಭೈರವನು\\ಕದನಕೆ ನಿಂತಿಹನು!
\end{myquote}

\begin{myquote}
ಮಸುಕು ಮಿಂಚು ತೋರುತಿಹುದು\\ಎಡೆಎಡೆಯೊಳು ಇಡಿಕಿರಿದಿಹ\\ಲಕ್ಷಲಕ್ಷ ಕರಿಕರಾಳ\\ಮೃತ್ಯುಛಾಯೆಯ!–
\end{myquote}

\begin{myquote}
ಹುಚ್ಚುವರಿದು ಕುಣಿಕುಣಿಯುತ\\ಸಂಕಟಗಳನೀಡಾಡುತ\\ಬಾ, ತಾಯಿ, ಬಾ!\\ನಲಿನಲಿಯುತ ಬಾ!
\end{myquote}

\begin{myquote}
ದುರ್ಗೆಯದುವೆ ನಿನ್ನ ಹೆಸರು,\\ಮೃಽತ್ಯುವೆ ನಿನ್ನುಸಿರು!\\ನಿನ್ನಡಿಗಳ ತಲ್ಲಣದಿಂ\\ನಿರ್ನಾಮವು ಜಗವು!\\ಕಾಲರೂಪಿ ತಾಯೆ, ನೀನು,\\ನಂಗುವೆ ಎಲ್ಲವನು!
\end{myquote}

\begin{myquote}
ಬಾ, ತಾಯೀ, ಬಾ!\\ನಲಿನಲಿಯುತ ಬಾ!
\end{myquote}

ಸ್ವಾಮೀಜಿಯವರು ರಚಿಸಿದ ಈ ಹಾಡಿನಲ್ಲಿ ವಿಶ್ವದ ಬಗೆಬಗೆಯ ಅಲ್ಲೋಲಕಲ್ಲೋಲಗಳ ಒಂದು ಮಿಣುಕು ನೋಟವಿದೆ. ವಿಕಟ ಅಟ್ಟಹಾಸದಿಂದ ಹುಚ್ಚೆದ್ದು ನರ್ತನಗೈಯುವ ಜಗ ನ್ಮಾತೆಯ ಚಿತ್ರಣವಿದೆ. ಇದೊಂದು ಸತ್ಯ-ದಿವ್ಯವಾದ ಭಾವನೆ;–ಅಲ್ಲ, ಇದು ಸ್ವಾಮೀಜಿಯವರಿ ಗಾದ ಕಾಳೀಮಾತೆಯ ದರ್ಶನಾನುಭವ! ಸ್ವಾಮೀಜಿ ಈ ಕವನದ ಕೊನೆಯ ಪದವನ್ನೂ ಬರೆದರು; ಅಷ್ಟರಲ್ಲಿ ಅವರ ಕೈಹಿಡಿತದಲ್ಲಿದ್ದ ಲೇಖನಿ ಜಾರಿಬಿದ್ದಿತು, ಜೊತೆಗೆ ಅವರೂ ನೆಲಕ್ಕುರುಳಿದರು. ಅವರ ಬಾಹ್ಯ ಪ್ರಜ್ಞೆ ತಪ್ಪಿತು. ಆದರೆ ಅವರ ಅಂತಃಪ್ರಜ್ಞೆಯು ಭಾವಸಮಾಧಿಯ ಅತ್ಯುನ್ನತ ಸ್ತರದಲ್ಲಿ ಲಯಗೊಂಡು ಸ್ತಬ್ಧವಾಯಿತು! ಯಾರು ತಮ್ಮ ವಿಚಾರಲಹರಿಯಲ್ಲಿ ಪಾಶ್ಚಾತ್ಯ ರಾಷ್ಟ್ರಗಳ ಸಹಸ್ರಾರು ಜನರನ್ನು ತೊನೆದಾಡಿಸಿದರೋ, ಯಾರು ಭಾರತದ ರಾಷ್ಟ್ರಕುಂಡಲಿನಿ ಯನ್ನೇ ಜಾಗೃತಗೊಳಿಸಿದರೋ ಅವರು ಇಲ್ಲೀಗ ರುದ್ರಮುಖೀ ಜಗನ್ಮಾತೆಯ ದರ್ಶನದಿಂದ ಆನಂದಾಶ್ಚರ್ಯವಿಮತ್ತರಾಗಿ ಮೃತರಾದವರಂತೆ ನಿಶ್ಚೇಷ್ಟಿತರಾಗಿ ಬಿದ್ದಿದ್ದಾರೆ!...

ಸ್ವಾಮೀಜಿಯವರ ಮನಸ್ಸಿನ ಮೇಲೆ ಕಾಳಿಯ ವ್ಯಕ್ತಿತ್ವವೆಂಬುದು ಎಷ್ಟು ಪ್ರಭಾವ ಬೀರಿದೆ ಯೆಂದರೆ ಈಗ ಅವರು ಮಾತೆತ್ತಿದರೆ ತಾಯಿ ಕಾಳಿಯ ವಿಷಯವಾಗಿಯೇ ಬಣ್ಣಿಸುತ್ತಾರೆ; ಕಾಳೀ ತತ್ತ್ವವನ್ನೇ ವಿವರಿಸುತ್ತಾರೆ; ತಾಯಿ ಕಾಳಿಯನ್ನೇ ಕರೆಯುತ್ತಾರೆ. ಅವರು ಆಗಾಗ ಉದ್ಗರಿಸು ತ್ತಾರೆ: “ಓ ತಾಯಿ, ನೀನು ನನ್ನನ್ನು ತುಂಡರಿಸಿದರೂ ಸರಿಯೆ, ನಾನು ಮಾತ್ರ ನಿನ್ನನ್ನು ಬಿಡುವವನಲ್ಲ!” “ಪ್ರತಿಯೊಬ್ಬನೂ ನಲಿವನ್ನೇ ಅರಸುತ್ತಾನೆಂಬ ಮಾತು ನಿಜವಲ್ಲ. ನೋವನ್ನೇ ಬಯಸುವವರು ಹಲವು ಮಂದಿ ಇದ್ದಾರೆ. ತಮಗೆತಾವೇ ಹಿಂಸೆಯನ್ನು ತಂದುಕೊಳ್ಳುವುದರಲ್ಲೂ ಆನಂದವಿರಬಲ್ಲದು. ನಾವು ಭೀಷಣತೆಯನ್ನು ಭೀಷಣತೆಗೋಸ್ಕರವೇ ಆರಾಧಿಸೋಣ!” ಮತ್ತೆ ಸ್ವಾಮೀಜಿ ಹೇಳುತ್ತಾರೆ, “ಯಾವ ತಾಯಿಯನ್ನು ನೀವು ಮಾಧುರ್ಯ-ಆನಂದಗಳಲ್ಲಿ ಗುರುತಿಸು ತ್ತೀರೋ, ಅಷ್ಟೇ ಸಹಜವಾಗಿ ಅವಳನ್ನು ದುಃಖ-ದಾರಿದ್ರ್ಯ-ದೌರ್ಜನ್ಯ-ವಿಧ್ವಂಸಕತೆಗಳಲ್ಲೂ ಕಾಣುವುದನ್ನು ಕಲಿಯಿರಿ.” ಇನ್ನು ಕೆಲವೊಮ್ಮೆ ಅವರು ಹೀಗೂ ಹೇಳುವುದುಂಟು–“ಓ ತಾಯಿ, ನಿಜ! ಜನ ನಿನ್ನನ್ನು ರುಂಡಮಾಲೆಯಿಂದ ಅಲಂಕರಿಸುತ್ತಾರೆ. ಆದರೆ ಮರುಕ್ಷಣವೇ ಭಯದಿಂದ ಕುಸಿದು ನಿನ್ನನ್ನು ‘ಓ ದಯಾಮಯಿ!’ ಎಂದು ಕರೆಯುತ್ತಾರೆ.” “ನಾನು ಹೇಳುತ್ತೇನೆ ಕೇಳಿ; ನಿಜಕ್ಕೂ ಭಯಂಕರವಾದುದನ್ನು ಪೂಜಿಸುವುದರಿಂದ ಮಾತ್ರವೇ ಭಯವನ್ನು ಗೆದ್ದು ಅಮೃತತ್ವ ವನ್ನು ಪಡೆಯಲು ಸಾಧ್ಯ. ಮೃತ್ಯುವನ್ನು ಧ್ಯಾನಿಸಿ! ಮೃತ್ಯುವನ್ನು ಧ್ಯಾನಿಸಿ! ಭೀಕರತೆಯನ್ನು ಆರಾಧಿಸಿ! ಜಗನ್ಮಾತೆಯೇ ಸ್ವತಃ ಪರಬ್ರಹ್ಮ ಸ್ವರೂಪಿಣಿಯಾಗಿದ್ದಾಳೆ. ಅವಳ ಶಾಪವೂ ಒಂದು ಅನುಗ್ರಹವೇ ಸರಿ. ನಮ್ಮ ಹೃದಯವೇ ಒಂದು ಸ್ಮಶಾನವಾಗಿ ನಮ್ಮ ಸ್ವಾರ್ಥ, ದುರಭಿಮಾನ, ದುರಾಸೆಗಳೆಲ್ಲ ಉರಿದು ಬೂದಿಯಾಗಬೇಕು. ಆಗ, ಆಗ ಮಾತ್ರವೇ, ತಾಯಿ ಬರುತ್ತಾಳೆ. ಆಕೆಯೇ ಸ್ವತಃ ಕಾಲರೂಪಿಣಿ. ಆಕೆಯೇ ಅನಂತ ಶಕ್ತಿರೂಪಿಣಿ. ಮತ್ತು ಜಗತ್ತಿನಲ್ಲಿ ಕಾಣುವ ವಿಭಿನ್ನತೆಗಳೆಲ್ಲ ಅವಳೇ.”

ಜನ ಭಗವಂತನನ್ನು ‘ಕರುಣಾಮಯ’ ಎಂದು ಕರೆಯುತ್ತಾರೆ; ‘ವರದಾಯಕ’ ಎಂದು ಕರೆಯುತ್ತಾರೆ; ‘ಅನುಗ್ರಹಕಾರಕ’ ಎಂದು ಕರೆಯುತ್ತಾರೆ. ಇದು ಜಗತ್ತಿನಲ್ಲಿ ಎಲ್ಲೆಲ್ಲಿಯೂ ಕಾಣಬರುವ ಸಂಗತಿ. ಆದರೆ ಜನ ಹೀಗೆ ಕರೆಯುವುದರ ಹಿಂದಿನ ಉದ್ದೇಶವೇನೆಂದು ಸ್ವಾಮೀಜಿ ಎತ್ತಿತೋರಿಸುತ್ತಾರೆ–ಅದು ಮನುಷ್ಯನ ಸ್ವಾರ್ಥತೆ ಎಂದು. ಇನ್ನೂ ಸ್ಪಷ್ಟವಾಗಿ ಹೇಳಬೇಕೆಂದರೆ, ಅದು ಅವನ ಭೀರುತೆಯನ್ನು ತೋರಿಸುತ್ತದೆ. ಬೆದರಿದ ಮನುಷ್ಯ ಭಗವಂತನ ಭೀಕರತೆಯನ್ನು ಬಯಸಿಯಾನೆ? ಕಲ್ಪಿಸಿಯಾನೆ? ಅವನಿಗೆ ಬೇಕು ಸಾಂತ್ವನ, ಅವನಿಗೆ ಬೇಕು ಕರುಣೆ, ಅವನಿಗೆ ಬೇಕು ಅನುಗ್ರಹ. ಆದ್ದರಿಂದ ಅವನು ಅನುಗ್ರಹಕಾರಕನಾದ ದಯಾಮಯ ಭಗವಂತನನ್ನೇ ಭಾವಿಸಿ ಆರಾಧಿಸಲು ಬಯಸುತ್ತಾನೆ. ಭಗವಂತನ ಭೀಕರತೆಯನ್ನು ಅವನು ಕನಸಿನಲ್ಲೂ ನೆನೆಸಿಕೊಳ್ಳಲಾರ. ಆದರೆ ಸ್ವಾಮೀಜಿ ಹೇಳುತ್ತಾರೆ, ನಾವು ಯಾವ ಭೂಕಂಪನವನ್ನು ನೋಡು ತ್ತೇವೆಯೋ, ಯಾವ ಜ್ವಾಲಾಮುಖಿಯನ್ನು ನೋಡುತ್ತೇವೆಯೋ ಅವು ಕೂಡ ಭಗವಂತನ ರೂಪಗಳೇ ಎಂದು. ಜಗತ್ತಿನಲ್ಲಿ ಏನೇನು ಒಳ್ಳೆಯದನ್ನು ಕಾಣುತ್ತಿದ್ದೇವೆಯೋ ಅದೂ ಭಗ ವಂತನೇ, ಏನೇನು ಕೆಟ್ಟದ್ದನ್ನು ಕಾಣುತ್ತಿದ್ದೇವೆಯೋ ಅದೂ ಭಗವಂತನೇ, ಎನ್ನುವ ಮಾತನ್ನು ಸ್ವಾಮೀಜಿ ಹೇಳುವಾಗ ನಮಗೆ ಅವರ ಅನುಭವಗಳ ಎತ್ತರ-ಬಿತ್ತರದ ಅರಿವಾಗದಿರದು; ಅವರ ಬೋಧನೆಗಳ ಹಿಂದಿರುವ ಸಾಹಸಮಯ ಅಂಶದ ಅರಿವಾಗದಿರದು. ನಿಜಕ್ಕೂ ನಾವು ಈ ಜಗತ್ತಿನ ರೀತಿನೀತಿಗಳನ್ನು ತೆರೆಮನಸ್ಸಿನಿಂದ ಪರೀಕ್ಷಿಸಿದರೆ ನಮಗೆ ಅಲ್ಲಿರುವ ಭೀಕರತೆಯ ದರ್ಶನವಾಗ ದಿರದು. ನಾವು ಭಯಗ್ರಸ್ತ ಮನಸ್ಕರಾದ್ದರಿಂದ ಸದಾ ಸುಂದರವಾದ ಹೂಗಳನ್ನು, ಹಿತಕರವಾದ ರೂಪಗಳನ್ನು ಮಾತ್ರವೇ ನೋಡುತ್ತ, ಜಗತ್ತೆಲ್ಲ ಶುಭಕರವಾಗಿದೆ, ಮಂಗಳಕರವಾಗಿದೆ ಎಂದು ಭ್ರಮಿಸುತ್ತಿರುತ್ತೇವೆ. ಆದರೆ ನಿಜಕ್ಕೂ ಇಲ್ಲಿ ಶುಭವೆಷ್ಟೋ ಅಷ್ಟೇ ಅಶುಭವೂ ಇದೆ; ಮಂಗಳವೆಷ್ಟೋ ಅಮಂಗಳವೂ ಅಷ್ಟೇ ಇದೆ; ಸೌಂದರ್ಯವೆಷ್ಟಿದೆಯೋ ರೌದ್ರವೂ ಅಷ್ಟೇ ಇದೆ. ತೂಗುವ ತೊಟ್ಟಿಲುಗಳೆಷ್ಟಿವೆಯೋ ಮಸಣದ ಗೋರಿಗಳೂ ಅಷ್ಟೇ ಇವೆ. ಆದರೆ ನಾವು ಈ ಅಮಂಗಳಕರ ರೌದ್ರ ಸ್ಮಶಾನದ ದೃಶ್ಯಗಳ ಕಡೆಗೆ ಕಣ್ಣುಮುಚ್ಚಿ ಕುಳಿತುಬಿಡುತ್ತೇವೆ. ಆದರೂ ಒಂದಲ್ಲ ಒಂದು ದಿನ ನಮಗೆ ಆ ಮುಖದ ದರ್ಶನವೂ ಆಗದಿರುವುದಿಲ್ಲ. ಕಾಳ್ಗಿಚ್ಚು ಹೊತ್ತಿಕೊಂಡು ದಹಿಸುತ್ತ ವೇಗವಾಗಿ ಬರುವಾಗ ಕಾಡುಕೋಳಿಯು ಅದರಿಂದ ತಪ್ಪಿಸಿಕೊಳ್ಳಲು ಸಾಧ್ಯವಾಗದಿರುವುದನ್ನು ಕಂಡು ಒಂದು ಉಪಾಯ ಹೂಡುತ್ತದೆ. ಏನೆಂದರೆ ಮಣ್ಣೊಳಗೆ ತನ್ನ ತಲೆಯನ್ನು ಅಡಗಿಸಿಟ್ಟುಕೊಳ್ಳುತ್ತದೆ; ಮತ್ತು ಭಾವಿಸುತ್ತದೆ, ‘ಈಗ ನಾನು ಸುರಕ್ಷಿತವಾಗಿದ್ದೇನೆ, ಈ ಕಾಳ್ಗಿಚ್ಚು ನನ್ನನ್ನೇನೂ ಮಾಡಲಾರದು’ ಎಂದು. ಆದರೆ ಕಾಳ್ಗಿಚ್ಚು ಬಂದೇಬಿಟ್ಟಿತು, ಅದರ ಶರೀರವನ್ನು ಸುಟ್ಟು ತಿಂದೇ ಬಿಟ್ಟಿತು. ಹಾಗೆಯೇ ರೋಗ ರುಜಿನ, ಕ್ಷಾಮ ಡಾಮರ, ಭೂಕಂಪ, ಅಪಘಾತ–ಇವೇ ಮೊದಲಾದ ಮೃತ್ಯುಮುಖಗಳು ಈ ಲೋಕದಲ್ಲಿ ರಭಸದಿಂದ ಸಂಚರಿಸು ತ್ತಿರುವುದನ್ನು ಕಂಡೂ ಕೂಡ, ನಾವು ಮಾತ್ರ ಸದಾ ಸುರಕ್ಷಿತವಾಗಿಯೇ ಇರುವವರು, ಸುಂದರ ರೂಪಿಯಾದ ನಮ್ಮ ದಯಾಮಯ ಭಗವಂತ ನಮ್ಮನ್ನು ಕಾಪಾಡುತ್ತಾನೆ ಎಂದು ಕಣ್ಣು ಮುಚ್ಚಿ ಕುಳಿತಿರುತ್ತೇವೆ. ಆದರೆ ಭಗವಂತನು ರಾಮನಾಗಿ ನಮ್ಮನ್ನು ಕಾಪಾಡುವುದು ಎಷ್ಟು ನಿಜವೋ ಯಮನಾಗಿ ಕೊಂಡೊಯ್ಯುವುದೂ ಅಷ್ಟೇ ನಿಜ ಎಂಬ ಕರಾಳಸತ್ಯವನ್ನು ಕಂಡುಕೊಳ್ಳಲು ನಾವು ಕಷ್ಟಪಡುತ್ತೇವೆ. ಆದ್ದರಿಂದಲೇ ಸ್ವಾಮೀಜಿ ಹೇಳುತ್ತಾರೆ, “ನಾವು ನಮ್ಮ ಭ್ರಮೆಯನ್ನು ಕಿತ್ತೊ ಗೆದು ಸತ್ಯಸ್ವರೂಪಿಯಾದ ಭಗವಂತನ ರೌದ್ರರೂಪದ ಸತ್ಯತೆಯನ್ನು ಗುರುತಿಸಿ ಆರಾಧಿಸಬೇಕು” ಎಂದು. “ಮೃತ್ಯುವನ್ನು ಧ್ಯಾನಿಸಿ! ಮೃತ್ಯುವನ್ನು ಧ್ಯಾನಿಸಿ!”–ಇದು ಅವರ ಧೀರವಾಣಿ.

ಸ್ವಾಮೀಜಿ ಎಷ್ಟೋ ಬಾರಿ ತೀವ್ರ ಕಾಯಿಲೆಯಿಂದ ನರಳುವಾಗ ಇಲ್ಲವೆ ಅಸಹನೀಯ ಯಾತನೆಯನ್ನು ಅನುಭವಿಸುವಾಗ ಉದ್ಗರಿಸುತ್ತಿದ್ದರು, “ನನ್ನ ಈ ಶರೀರದ ಅಂಗಾಂಗಗಳೂ ಜಗನ್ಮಾತೆಯೇ!ಅವು ಅನುಭವಿಸುತ್ತಿರುವ ನೋವೂ ಆಕೆಯೇ! ಮತ್ತು ನೋವನ್ನು ಕೊಡುತ್ತಿರು ವವಳೂ ಆಕೆಯೇ!ಕಾಳಿ! ಕಾಳಿ! ಕಾಳಿ!” ಎಂದು. ಇಂತಹ ಸಂದರ್ಭಗಳಲ್ಲಿ ತಮ್ಮ ಶಿಷ್ಯರಿಗೆ ಬೋಧನೆ ನೀಡುವಾಗ ಅವರು ಹೇಳುತ್ತಿದ್ದರು, “ಭಯವೆಂದಿಗೂ ಇರಬಾರದು. ಬೇಡುವುದಲ್ಲ, ಪಡೆಯುವುದು! ತಾಯಿಯ ನಿಜವಾದ ಭಕ್ತರು ಹಟಮಾರಿಗಳಂತೆ ದೃಢಚಿತ್ತರು; ಸಿಂಹದಂತೆ ನಿರ್ಭೀತರು. ಅವರು ಇಡೀ ವಿಶ್ವವೇ ತಮ್ಮ ಕಾಲ ಕೆಳಗಿನಿಂದ ಕುಸಿದುಬಿದ್ದರೂ ಲೆಕ್ಕಿಸುವವರಲ್ಲ. ಆಕೆಯು ನಿಮ್ಮ ಮಾತನ್ನು ಕೇಳಿಯೇ ತೀರುವಂತೆ ಮಾಡಿ. ತಾಯಿಯನ್ನು ಅತಿ ದೈನ್ಯದಿಂದ ಬೇಡಿಕೊಳ್ಳುವುದಲ್ಲ, ನೆನಪಿಡಿ!” “ಅವಳು ಸರ್ವಶಕ್ತಿಸ್ವರೂಪಿಣಿ. ಒಂದು ಮಣ್ಣಹೆಂಟೆಯಿಂದ ಕೂಡ ಆಕೆ ವೀರರನ್ನು ನಿರ್ಮಿಸಬಲ್ಲಳು.”

ಸ್ವಾಮೀಜಿ ಹೇಳುತ್ತಿದ್ದರು, “ಎಲ್ಲೆಲ್ಲಿ ಭಯವಿಲ್ಲವೋ, ಎಲ್ಲೆಲ್ಲಿ ತ್ಯಾಗವಿದೆಯೋ, ಎಲ್ಲೆಲ್ಲಿ ಈ ಅಹಮಿಕೆಯೆಂಬುದು ನಶಿಸಿಹೋಗಿದೆಯೋ ಮತ್ತು ಯಾರಿಗೆ ತಾನು ಮುಟ್ಟುವುದೆಲ್ಲ ಕೇವಲ ನೋವು ಎಂಬ ಅರಿವಿದೆಯೋ ಅಲ್ಲೆಲ್ಲ ಜಗನ್ಮಾತೆ ಇದ್ದಾಳೆ.” ಮತ್ತೆ ಹೇಳುತ್ತಾರೆ, “ಒಬ್ಬ ಮನುಷ್ಯ ತನ್ನ ಜೀವನದಲ್ಲಿ ಬಲವಾದ ಏಟುಗಳನ್ನು ತಿಂದು ಜೀವನದ ಕಹಿರಸವನ್ನು ಅನುಭವಿ ಸುವಾಗ ಅವನ ಆತ್ಮವೆಂಬ ಶಿಶುವು ಸಮಾಧಾನ-ಆಶ್ರಯಗಳಿಗಾಗಿ ಜಗನ್ಮಾತೆಯತ್ತಲೇ ತಿರುಗು ತ್ತದೆ.” ಪಾಶ್ಚಾತ್ಯ ಗೂಢಶಾಸ್ತ್ರದಲ್ಲಿ ಬರುವ ‘ತಲೆಬುರಡೆ-ಅಡ್ಡಮೂಳೆ’ಗಳ ಪೂಜೆಯ ಹಿಂದೆ ಕೂಡ ಸ್ವಾಮೀಜಿಯವರು ಅದೇ ಕಾಳೀ ಪೂಜೆಯ ಛಾಯೆಯನ್ನು ಗುರುತಿಸಿದ್ದರು. ಸಕಲ ಚರಾಚರಗಳ ಹಿಂದಿರುವ ಶಕ್ತಿಯಾದ ಜಗನ್ಮಾತೃತ್ವದ ಕಲ್ಪನೆ ಸ್ವಾಮೀಜಿಯವರಲ್ಲಿ ಕಾವ್ಯಮಯ ವಾಗಿ ಹೊರಸೂಸುತ್ತಿತ್ತು.

ತಮಗಾದ ಈ ಕಾಳೀದರ್ಶನದ ನಂತರ ಸ್ವಾಮೀಜಿಯವರು, ತಮ್ಮ ಶಿಷ್ಯೆಯರಿಗೆ ತಮ್ಮನ್ನು ಹಿಂಬಾಲಿಸಿ ಬರದಂತೆ ಸೂಚನೆಯಿತ್ತು ಇದ್ದಕ್ಕಿದ್ದಂತೆ ಕ್ಷೀರಭವಾನಿಯ ಕಡೆಗೆ ತೆರಳಿದರು. ಸೆಪ್ಟೆಂಬರ್ ೨ಂರಂದು ಹೀಗೆ ಹೊರಟವರು ಅಕ್ಟೋಬರ್ ೬ರವರೆಗೂ ಹಿಂದಿರುಗಲಿಲ್ಲ. ಕ್ಷೀರಭವಾನಿಯ ದೇವಾಲಯದಲ್ಲಿ ಅವರು ಪ್ರತಿದಿನವೂ ಹೋಮ ಮಾಡಿದರು. ಒಂದು ಮಣ ಹಾಲು, ಅಕ್ಕಿ, ಬಾದಾಮಿಗಳಿಂದ ತಯಾರಿಸಿದ ಖೀರನ್ನು ನೈವೇದ್ಯ ಮಾಡಿಸಿದರು. ಕ್ಷೀರಭವಾನಿ ಯಲ್ಲವೇ ಅವಳು! ಅಲ್ಲದೆ ವಿಶೇಷ ಸಾಧನೆಯಾಗಿ, ಪ್ರತಿದಿನ ಬೆಳಿಗ್ಗೆ ಅಲ್ಲಿನ ಒಬ್ಬ ಬ್ರಾಹ್ಮಣ ಪಂಡಿತನ ಪುಟ್ಟ ಮಗಳನ್ನು ದಿವ್ಯಕನ್ಯೆಯಾದ ಉಮಾಕುಮಾರಿಯ ರೂಪದಲ್ಲಿ ಪೂಜೆ ಮಾಡಿ ದರು. (ಬಂಗಾಳದಲ್ಲಿ ಈ ಪೂಜೆಯು ‘ಕುಮಾರೀಪೂಜೆ’ ಎಂಬ ಹೆಸರಿನಿಂದ ಪ್ರಸಿದ್ಧವಾಗಿದೆ.) ಇದಲ್ಲದೆ ಅವರು ಇನ್ನೂ ಕಠಿಣತರವಾದ ಸಾಧನೆಗಳನ್ನು ಕೈಗೊಂಡರು. ವರ್ಷಗಟ್ಟಲೆಯ ಕರ್ಮದ ಮೂಲಕ ಅವರು ತಮ್ಮ ಸುತ್ತ ತಾವೇ ಹೆಣೆದುಕೊಂಡಿದ್ದ ಸಕಲ ಬಂಧನಗಳನ್ನೂ ಹರಿದು ಹಾಕಿ, ಮಹಾತಾಯಿಯ ಮಡಿಲಲ್ಲಿ ಮತ್ತೊಮ್ಮೆ ಶಿಶುವಾಗಹೊರಟಿದ್ದಾರೋ ಎಂಬಂ ತಿತ್ತು ಅವರ ವರ್ತನೆ. ‘ನಾಯಕ’ ‘ಬೋಧಕ’ ಎಂಬ ಭಾವಗಳೆಲ್ಲ ಅವರ ಮನಸ್ಸಿನಿಂದ ಮಾಯವಾಗಿಹೋಗಿದ್ದುವು. ಈಗ ಅವರು ಕೇವಲ ಒಬ್ಬ ಸಂನ್ಯಾಸಿ–ಆ ಪದದ ನಿಜವಾದ ಅರ್ಥದಲ್ಲಿ.

ಸ್ವಾಮೀಜಿ ಶ್ರೀನಗರಕ್ಕೆ ಹಿಂದಿರುಗಿದ ಮೇಲೆ ಅವರ ಶಿಷ್ಯರಿಗೆ ಅವರು ಸಂಪೂರ್ಣ ಪರಿವರ್ತನೆಗೊಂಡ ವ್ಯಕ್ತಿಯಾಗಿ ಕಂಡುಬಂದರು. ಅವರು ತಮ್ಮ ಶಿಷ್ಯೆಯರಿದ್ದ ದೋಣಿಮನೆ ಯನ್ನು ಪ್ರವೇಶಿಸುವಾಗ ಎರಡೂ ಕೈಗಳನ್ನು ಆಶೀರ್ವಾದದ ಮುದ್ರೆಯಲ್ಲಿ ಮೇಲೆತ್ತಿ ಒಳ ಪ್ರವೇಶಿಸಿದರು. ಅಲ್ಲಿದ್ದ ಪ್ರತಿಯೊಬ್ಬರ ತಲೆಯ ಮೇಲೂ ಜಗನ್ಮಾತೆಗೆ ಅರ್ಪಿಸಿದ ಚೆಂಡುಹೂ ಗಳನ್ನಿರಿಸಿದರು. ಬಳಿಕ ಅಲ್ಲಿಯೇ ಕುಳಿತುಕೊಳ್ಳುತ್ತ ಉದ್ಗರಿಸಿದರು, “ಇನ್ನು ಮೇಲೆ ‘ಹರಿ ಓಂ’ ಅಲ್ಲ; ಏನಿದ್ದರೂ ‘ಅಮ್ಮಾ’ ಮಾತ್ರ!” ಬಳಿಕ ಮುಂದುವರಿಸುತ್ತ ಹೇಳಿದರು, “ಈಗ ನನ್ನ ರಾಷ್ಟ್ರಪ್ರೇಮ ಇತ್ಯಾದಿಯೆಲ್ಲ ಮಾಯವಾಗಿಬಿಟ್ಟಿದೆ. ಈಗಿರುವುದೆಲ್ಲ ‘ತಾಯಿ’ ‘ತಾಯಿ’ ಮಾತ್ರ.”

ಕ್ಷೀರಭವಾನಿಗೆ ಹೋಗಿದ್ದಾಗ ನಡೆದ ವೃತ್ತಾಂತವೊಂದನ್ನು ಅವರೀಗ ಬಣ್ಣಿಸ ತೊಡಗಿದರು: “... ನಿಜಕ್ಕೂ ನಾನೊಂದು ದೊಡ್ಡ ತಪ್ಪು ಮಾಡಿದೆ. ನಾನು ಒಮ್ಮೆ ಕ್ಷೀರಭವಾನಿಯಲ್ಲಿ ಕುಳಿತು ಚಿಂತಿಸುತ್ತಿದ್ದೆ–ನಮ್ಮ ದೇವಾಲಯಗಳು ಮುಸಲ್ಮಾನರ ಆಕ್ರಮಣದಿಂದ ಈ ರೀತಿ ಹಾಳುಗೆಡವಲ್ಪಟ್ಟು ಅಪವಿತ್ರಗೊಂಡಿವೆಯಲ್ಲ; ಇಂತಹ ವಿನಾಶವನ್ನು ನಮ್ಮ ಜನಗಳು ತಮ್ಮ ಶಕ್ತಿಮೀರಿ ತಡೆಗಟ್ಟದೆ ಹೇಗೆ ಸುಮ್ಮನಿದ್ದಿರಬಹುದು? ನಾನೇನಾದರೂ ಆಗ ಇಲ್ಲಿದ್ದಿದ್ದರೆ ಇಂಥದೆಲ್ಲ ನಡೆಯಲು ಖಂಡಿತ ಬಿಡುತ್ತಿರಲಿಲ್ಲ. ನಾನು ನನ್ನ ಪ್ರಾಣವನ್ನೇ ಪಣವಾಗೊಡ್ಡಿ ತಾಯಿಯನ್ನು ರಕ್ಷಿಸುತ್ತಿದ್ದೆ.’ ಆಗ ತಾಯಿ ನನಗೆ ಹೇಳಿದಳು, ‘ಏನು, ನಾಸ್ತಿಕರು ನನ್ನ ದೇವಾಲಯವನ್ನು ಪ್ರವೇಶಿಸಿದರೆ, ನನ್ನ ವಿಗ್ರಹಗಳನ್ನು ಹಾಳುಗೆಡವಿದರೆ ಅದರಿಂದ ನಿನಗೇನು? ನೀನು ನನ್ನನ್ನು ರಕ್ಷಿಸುವವನೋ ಅಥವಾ ನಾನು ನಿನ್ನನ್ನು ರಕ್ಷಿಸುವವಳೋ?’ ಆದ್ದರಿಂದ ಇನ್ನು ದೇಶಪ್ರೇಮ ಎನ್ನುವುದೆಲ್ಲ ಏನೂ ಉಳಿದಿಲ್ಲ. ಈಗ ನಾನೊಬ್ಬ ಹಸುಳೆ ಮಾತ್ರ.”

ಈ ವೃತ್ತಾಂತವನ್ನು ಕೇಳುತ್ತಿದ್ದ ಶಿಷ್ಯೆಯರೆಲ್ಲ ಸ್ತಬ್ಧರಾಗಿ ಕುಳಿತುಬಿಟ್ಟಿದ್ದರು. ಸ್ವಾಮೀಜಿ ಕುಳಿತಿದ್ದ ಆ ಸ್ಥಳದಲ್ಲಿ ಎಂತಹ ಒಂದು ಸಾಂದ್ರವಾದ ಘನ ಗಂಭೀರ ವಾತಾವರಣ ಏರ್ಪಟ್ಟಿ ತ್ತೆಂದರೆ ಅವರೆಲ್ಲ ತುಟಿಪಿಟಕ್ಕೆನ್ನದೆ ಮೂಗರಾಗಿದ್ದರು! ಈಗ ಸ್ವಾಮೀಜಿ ಮೆಲ್ಲನೆದ್ದು ಅಲ್ಲಿಂದ ಹೊರಡುತ್ತ ಉದ್ಗರಿಸಿದರು, “ಈಗಲೇ ನಾನು ಎಲ್ಲವನ್ನೂ ಹೇಳಿಬಿಡಲಾರೆ, ಹೇಳಬಾರದು... ಆದರೆ ಆಧ್ಯಾತ್ಮಿಕವಾಗಿ ನಾನು ಎಂದೂ ಕಟ್ಟಿಹಾಕಲ್ಪಟ್ಟಿರಲಿಲ್ಲ.”

ಈಗ ಸ್ವಾಮೀಜಿ ಮತ್ತೆ ತಮ್ಮ ಶಿಷ್ಯರೊಂದಿಗಿದ್ದರೆಂಬುದೇನೋ ನಿಜ. ಆದರೂ ಅವರು ಶಿಷ್ಯರ ಕಣ್ಣಿಗೆ ಬೀಳುತ್ತಿದ್ದುದೇ ಅಪರೂಪ. ಎಷ್ಟೋ ಸಲ ಅವರು ಏಕಾಂಗಿಯಾಗಿ ನದಿಯ ಪಕ್ಕದ ಕಾಡಿನಲ್ಲಿ ತಮ್ಮದೇ ಆದ ಆಲೋಚನೆಯಲ್ಲಿ ಮುಳುಗಿಹೋಗಿ ಗಂಟೆಗಟ್ಟಲೆ ಅಡ್ಡಾಡು ತ್ತಿದ್ದರು. ಆಗ ಅವರು, ಅಲ್ಲೇ ಸಮೀಪದಲ್ಲಿ ದೋಣಿಮನೆಯ ಛಾವಣಿಯ ಮೇಲೆ ಕುಳಿತಿರು ತ್ತಿದ್ದ ತಮ್ಮ ಶಿಷ್ಯರನ್ನೂ ಗಮನಿಸುತ್ತಿರಲಿಲ್ಲ. ಒಂದು ದಿನ ಅವರು ತಲೆಯನ್ನು ನುಣ್ಣಗೆ ಬೋಳಿಸಿಕೊಂಡು ಅತ್ಯಂತ ಸರಳ ಸಂನ್ಯಾಸಿಯಂತೆ ವಸ್ತ್ರಧರಿಸಿ ತಮ್ಮ ಶಿಷ್ಯರ ಮುಂದೆ ಕಾಣಿಸಿ ಕೊಂಡರು. ಆಗ ಅವರ ಮುಖದಲ್ಲಿ ಯಾರನ್ನೂ ಬಳಿಸಾರಲು ಬಿಡದಂತಹ ಗಂಭೀರಭಾವ ಕಾಣುತ್ತಿತ್ತು. ಅವರು ತಾವೇ ರಚಿಸಿದ \eng{Kali the Mother–‘}ತಾಯಿ ಕಾಳಿ’ ಎಂಬ ಕವನವನ್ನು ಹಾಡುತ್ತ ನಡುವೆ ನಿಲ್ಲಿಸಿ ಹೇಳಿದರು, “ಅವೆಲ್ಲವೂ, ಅದರ ಪ್ರತಿಯೊಂದು ಪದವೂ ನಿಜವಾಗಿ ಬಿಟ್ಟಿತು. ಅಲ್ಲದೆ ನಾನು ಅದನ್ನು ಸಾಬೀತುಗೊಳಿಸಿದ್ದೇನೆ. ಏಕೆಂದರೆ ನಾನು ಮೃತ್ಯುವಿನ ಸ್ವರೂಪವನ್ನು ಅಪ್ಪಿಕೊಂಡಿದ್ದೇನೆ.” ತಾವು ಕ್ಷೀರಭವಾನಿಯಲ್ಲಿ ಆಚರಿಸಿದ ಕಠಿಣ ವ್ರತಗಳು, ಉಪವಾಸ ಹಾಗೂ ತಮಗುಂಟಾದ ದಿವ್ಯ ದರ್ಶನಗಳು–ಇವುಗಳ ಬಗ್ಗೆ ಅವರು ಸೂಕ್ಷ್ಮವಾಗಿ ತಿಳಿಸಿದರು. ಇವಲ್ಲದೆ, ಕಾಳರಾತ್ರಿಯ ಕಗ್ಗತ್ತಲ ಸಮಯದಲ್ಲಿ ‘ರೌದ್ರ’ವನ್ನು ಕುರಿತು ಧ್ಯಾನಿಸು ವಾಗ ತಮಗುಂಟಾದ ಅನುಭವಗಳ ಕುರಿತಾಗಿ ಅವರು ಮುಂದೆ ತಮ್ಮ ಒಂದಿಬ್ಬರು ಗುರು ಭಾಯಿಗಳಿಗೆ ಮಾತ್ರ ಹೇಳಿದರು. ಏಕೆಂದರೆ ಅವು ಬೇರೆ ಯಾರೂ ತಿಳಿಯಬಾರದಷ್ಟು ‘ಪವಿತ್ರ ಹಾಗೂ ಗುಪ್ತ’ವಾಗಿದ್ದುವು.

ಒಮ್ಮೆ ಸ್ವಾಮೀಜಿ ಕ್ಷೀರಭವಾನಿಯಲ್ಲಿ ತಮಗಾದ ಇನ್ನೊಂದು ಅನುಭವವನ್ನು ಆಪ್ತರೊಬ್ಬರ ಮುಂದೆ ಹೀಗೆ ಬಣ್ಣಿಸುತ್ತಾರೆ–“ಒಂದು ದಿನ ಅಲ್ಲಿ ನಾನು ಹೀಗೇ ಆಲೋಚಿಸುತ್ತ ಕುಳಿತಿದ್ದೆ. ಆಗ ನನಗೆ ಆ ದೇವಸ್ಥಾನದ ಶಿಥಿಲಾವಸ್ಥೆಯನ್ನು ಕಂಡು ತುಂಬ ಸಂಕಟವಾಯಿತು. ನನಗನ್ನಿಸಿತು, ‘ನಮ್ಮ ದೇಶದಾದ್ಯಂತ ಶ್ರೀರಾಮಕೃಷ್ಣರಿಗಾಗಿ ಕಟ್ಟಬೇಕೆಂದಿರುವ ದೇವಸ್ಥಾನಗಳ ಹಾಗೂ ಆಶ್ರಮಗಳ ರೀತಿಯಲ್ಲಿ ಇಲ್ಲಿ ತಾಯಿ ಕ್ಷೀರಭವಾನಿಗೂ ಬಂದು ದೇವಸ್ಥಾನವನ್ನು ಕಟ್ಟಿಸಿದರೆ ಎಷ್ಟು ಚೆನ್ನಾಗಿರುತ್ತದೆ!’ ಹಾಗೆ ನಾನು ಕನಸು ಕಾಣುತ್ತ ಕುಳಿತಿದ್ದಾಗ ಇದ್ದಕ್ಕಿದ್ದಂತೆ ದೇವಿಯ ದಿವ್ಯ ಧ್ವನಿಯನ್ನು ಕೇಳಿ ದಿಗಿಲುಗೊಂಡೆ. ‘ಮಗು, ನಾನಿಚ್ಛಿಸಿದರೆ, ನನಗಾಗಿ ಅಸಂಖ್ಯಾತ ದೇವಸ್ಥಾನಗಳನ್ನು, ಅದ್ಭುತವಾದ ಆಶ್ರಮಗಳನ್ನು ಕಟ್ಟಿಸಿಕೊಳ್ಳಬಲ್ಲೆ. ಇಲ್ಲೇ, ಈ ಕ್ಷಣದಲ್ಲೇ ಏಳು ಅಂತಸ್ತಿನ ಸುವರ್ಣ ದೇವಾಲಯವನ್ನು ನಿರ್ಮಿಸಿಕೊಳ್ಳಬಲ್ಲೆ!’ ಎಂದಳು ದೇವಿ. ಆ ದನಿಯನ್ನು ಕೇಳಿದಾಗಿನಿಂದ ನಾನು ಯೋಜನೆಗಳನ್ನು ಹೂಡುವುದನ್ನೆಲ್ಲ ನಿಲ್ಲಿಸಿಬಿಟ್ಟಿದ್ದೇನೆ. ಎಲ್ಲವೂ ಆಕೆಯ ಇಚ್ಛೆಯಂತೆಯೇ ನಡೆಯಲಿ.”

ಕಾಶ್ಮೀರದಲ್ಲಿದ್ದ ಈ ದಿನಗಳಲ್ಲೇ, ಸ್ವಾಮೀಜಿಯವರಿಗೆ ಅವರ ಮನಶ್ಶಾಂತಿಯನ್ನು ಕದಡು ವಂಥದೊಂದು ಅನುಭವವಾಯಿತು. ಇದನ್ನು ಅವರು ತಮ್ಮ ಜೀವನದ ಒಂದು ‘ಸಂದಿಗ್ಧತೆ’ ಯೆಂದು ಕರೆಯುತ್ತಿದ್ದರು. ಒಬ್ಬ ಮುಸಲ್ಮಾನ ಫಕೀರನ ಶಿಷ್ಯ ಸ್ವಾಮೀಜಿಯವರ ವ್ಯಕ್ತಿತ್ವದಿಂದ ಆಕರ್ಷಿತನಾಗಿ ಅವರ ಬಳಿಗೆ ಆಗಾಗ ಬರುತ್ತಿದ್ದ. ಒಂದು ದಿನ ಆತ ಜ್ವರ ಮತ್ತು ವಿಪರೀತ ತಲೆನೋವಿಗೆ ಬಲಿಯಾಗಿದ್ದ. ಅದನ್ನು ಕೇಳಿ ಸ್ವಾಮೀಜಿ ಕರುಣೆಯಿಂದ ಆತನ ತಲೆಯನ್ನೊಮ್ಮೆ ಸ್ಪರ್ಶಿಸಿದರು. ಕೂಡಲೇ ಅವನು ಆಶ್ಚರ್ಯಕರವಾಗಿ ಗುಣಮುಖನಾದ. ಇದಾದ ಬಳಿಕ ಅವನು ಸ್ವಾಮೀಜಿಯವರಲ್ಲಿ ಇನ್ನೂ ಹೆಚ್ಚಿನ ಭಕ್ತಿಯಿಂದ ಹಿಂದಿಗಿಂತ ಹೆಚ್ಚಾಗಿ ಬರಲಾರಂಭಿಸಿದ. ಈ ವಿಷಯವನ್ನು ತಿಳಿದ ಅವನ ಫಕೀರಗುರುವಿಗೆ ಮಾತ್ರ ಸ್ವಾಮೀಜಿಯವರ ಮೇಲೆ ಹೊಟ್ಟೆಯುರಿ ಯುಂಟಾಯಿತು. ಎಲ್ಲಿ ಅವರು ತನ್ನ ಶಿಷ್ಯನನ್ನು ತನ್ನಿಂದ ಕಸಿದುಕೊಂಡುಬಿಡುತ್ತಾರೋ ಎಂದು ಹೆದರಿದ ಆ ಫಕೀರ, ತನ್ನ ಶಿಷ್ಯನ ಮುಂದೆ ಸ್ವಾಮೀಜಿಯವರ ಬಗ್ಗೆ ಹೀನಾಯವಾಗಿ ಮಾತನಾಡಿ, ಅವರ ಬಳಿಗೆ ಹೋಗಲೇಬಾರದೆಂದು ಎಚ್ಚರಿಕೆ ನೀಡಿದ. ಆದರೆ ಅದರಿಂದೇನೂ ಪ್ರಯೋಜನ ವಾಗಲಿಲ್ಲ. ಕೊನೆಗೆ ಅವನು ಸಿಟ್ಟಿಗೆದ್ದು ಸ್ವಾಮೀಜಿಯವರನ್ನು ಬಾಯಿಗೆ ಬಂದಂತೆ ಬೈದ. ಆ ಕೋಪದ ಭರದಲ್ಲಿ, ಮತ್ತು ಬಹುಶಃ ತನ್ನ ಶಕ್ತಿಯ ಬಗ್ಗೆ ಶಿಷ್ಯನಲ್ಲಿ ನಂಬಿಕೆಯುಂಟುಮಾಡಿಸುವ ಉದ್ದೇಶದಿಂದ, ತನ್ನ ಮಾಂತ್ರಿಕ ವಿದ್ಯೆಯನ್ನು ಪ್ರಯೋಗಿಸುವುದಾಗಿ ಬೆದರಿಕೆ ಹಾಕಿದ. ಸ್ವಾಮೀಜಿ ಕಾಶ್ಮೀರವನ್ನು ಬಿಡುವುದಕ್ಕೆ ಮುಂಚೆ ತಲೆ ತಿರುಗಿ ವಾಂತಿ ಮಾಡಿಕೊಳ್ಳುತ್ತಾರೆ ಎಂದು ಶಾಪ ಹಾಕಿಯೇಬಿಟ್ಟ. ಅದು ಹಾಗೆಯೇ ನಡೆದುಹೋಯಿತು ಕೂಡ! ಇದರಿಂದ ಸ್ವಾಮೀಜಿ ತೀವ್ರ ಮಾನಸಿಕ ಕ್ಷೋಭೆ ಹಾಗೂ ಆವೇಗಕ್ಕೊಳಗಾದರು. ಆದರೆ ಅವರು ಸಿಟ್ಟಿಗೆದ್ದದ್ದು ಆ ಫಕೀರನ ಮೇಲಲ್ಲ, ಬದಲಾಗಿ ತಮ್ಮ ಮೇಲೆಯೇ ಮತ್ತು ತಮ್ಮ ಗುರುದೇವನ ಮೇಲೆಯೇ! ಆಲೋಚಿಸಿ ದರು, ‘ಈಗ, ಶ್ರೀರಾಮಕೃಷ್ಣರಿಂದ ನನಗಾದದ್ದೇನು? ಕೇವಲ ಒಬ್ಬ ಯಃಕಿಶ್ಚಿತ್ ಮಾಂತ್ರಿಕನ ಶಕ್ತಿಯಿಂದ ನನ್ನನ್ನು ನಾನು ರಕ್ಷಿಸಿಕೊಳ್ಳಲು ಸಾಧ್ಯವಿಲ್ಲದಿದ್ದ ಮೇಲೆ, ನಾನು ಕಂಡ ದರ್ಶನಗಳು, ನಾನು ಬೋಧಿಸಿದ ವೇದಾಂತತತ್ತ್ವ ಇವುಗಳಿಂದೆಲ್ಲ ಏನು ಪ್ರಯೋಜನವಾಯಿತು?’ಎಂದು. ಈ ಒಂದು ಘಟನೆ ಸ್ವಾಮೀಜಿಯವರ ಮನಸ್ಸಿಗೆ ಎಷ್ಟೊಂದು ಶ್ರಮ ಕೊಟ್ಟಿತೆಂದರೆ, ಮೂರು ವಾರಗಳ ಬಳಿಕ ಅವರು ಕಲ್ಕತ್ತವನ್ನು ತಲುಪಿದ ಮೇಲೂ ಅದು ಅವರ ತಲೆಯನ್ನು ಕೊರೆಯು ತ್ತಿತ್ತು. ಅವರು ಶ್ರೀಮಾತೆ ಶಾರದಾದೇವಿಯವರ ಬಳಿಗೆ ಹೋಗಿ ದೂರುವ ದನಿಯಲ್ಲಿ, “ಏನಮ್ಮ, ನನಗೆ ಹೀಗೆಲ್ಲ ಆಯಿತಲ್ಲ; ನಿಮ್ಮ ರಾಮಕೃಷ್ಣರು ಅದನ್ನು ತಡೆಯಬಾರದಿತ್ತೆ?” ಎಂದರು. ಆಗ ಶ್ರೀಮಾತೆಯವರು ಅವರನ್ನು ಸಂತೈಸುತ್ತ ನುಡಿದರು, “ಏನು ಮಾಡುವುದಪ್ಪ? ಜಗತ್ತಿನಲ್ಲಿ ನಿಜವಾದ ಆಧ್ಯಾತ್ಮಿಕತೆಯಿರುವಂತೆಯೇ ಇಂತಹ ಮಾಯಾ ಮಂತ್ರಗಳೂ ಇವೆ. ಅಲ್ಲದೆ ಅವರು ಕೂಡ ಸಾಧನೆ ಮಾಡಿ ಗಳಿಸಿದ್ದೇ ಅಲ್ಲವೆ ಆ ಶಕ್ತಿ? ಅದು ವ್ಯರ್ಥವಾಗಬೇಕೆಂದರೆ ಹೇಗೆ ಸಾಧ್ಯ? ಅದೂ ಅಲ್ಲದೆ, ಹಿಂದೆ ಶ್ರೀರಾಮಕೃಷ್ಣರಿಗೇ ಹಲಧಾರಿ (ಶ್ರೀರಾಮಕೃಷ್ಣರ ಸಂಬಂಧಿ) ‘ನೀನು ರಕ್ತಕಾರುವಂತಾಗಲಿ’ ಎಂದು ಶಪಿಸಿಬಿಟ್ಟಿದ್ದ. ಅದು ನಿಜಕ್ಕೂ ಹಾಗೆಯೇ ಆಗಿಹೋಯಿತು. ಆದರೆ, ಆ ಸಮಯದಲ್ಲಿ ದಕ್ಷಿಣೇಶ್ವರದಲ್ಲಿದ್ದ ಒಬ್ಬ ಸಾಧು ಶ್ರೀರಾಮಕೃಷ್ಣ ರನ್ನು ಪರೀಕ್ಷೆ ಮಾಡಿ ಹೇಳಿದ–ಯಾವುದೋ ಒಂದು ವಿಶೇಷ ಸಾಧನೆಯಿಂದಾಗಿ ಅವರ ತಲೆಯೊಳಗೆ ರಕ್ತ ತುಂಬಿಕೊಂಡು ಪ್ರಾಣಾಪಾಯವುಂಟಾಗಿತ್ತು; ಈಗ ಆ ರಕ್ತ ವಾಂತಿಯಾದದ್ದ ರಿಂದ ಆ ಅಪಾಯ ತಪ್ಪಿತು, ಅಂದ. ಹಾಗೆಯೇ ಈಗ ಆ ಫಕೀರ ನಿನಗೆ ಶಾಪವಿತ್ತು ವಾಂತಿ ಮಾಡಿಸಿದ್ದರಿಂದ ಯಾವುದೋ ಅಪಾಯಕಾರಿಯಾದ ವಸ್ತು ಹೊರಬಂದಿರಬಹುದು. ಆದ್ದರಿಂದ ಅದೂ ಒಳ್ಳೆಯದಕ್ಕೇ ಅಂತ ತಿಳಿದುಕೊ.” ಆದರೆ ಈ ಉತ್ತರದಿಂದ ಸ್ವಾಮೀಜಿಯವರಿಗೇನೂ ಸಂಪೂರ್ಣ ಸಮಾಧಾನವಾಗಲಿಲ್ಲ. ಮುನಿಸಿಕೊಂಡ ಬಾಲಕ ತಾಯಿಯ ಮುಂದೆ ಹಟ ಮಾಡು ವಂತೆ ಹೇಳಿದರೆ, “ಒಟ್ಟಿನಲ್ಲಿ ನಿಮ್ಮ ರಾಮಕೃಷ್ಣರ ಕೈಯಲ್ಲಿ ಏನೂ ಆಗಲಿಲ್ಲ. ಹೋಗಿ, ನನಗೆ ಅವರ ಸಹವಾಸವೇ ಬೇಡ!” ಇದನ್ನು ಕೇಳಿ ಶ್ರೀಮಾತೆಯವರಿಗೆ ನಗುಬಂದಿತು. ಅವರೆಂದರು, “ಮಗು, ನೀನು ಏನೇ ಹೇಳಿದರೂ, ನಿನ್ನ ಜುಟ್ಟು ಅವರ ಕೈಯಲ್ಲಿ ಭದ್ರವಾಗಿದೆಯಲ್ಲ! ಏನು ಮಾಡುವುದು?” (ಈ ಎಲ್ಲ ಮಾತುಕತೆಯ ಸಂದರ್ಭದಲ್ಲಿ ಶ್ರೀಮಾತೆಯವರು ಸ್ವಾಮೀಜಿಯವ ರೊಂದಿಗೆ ನೇರವಾಗಿ ಮಾತನಾಡುತ್ತಿರಲಿಲ್ಲ. ಅವರು ಪಿಸುದನಿಯಲ್ಲಿ ಹೇಳಿದುದನ್ನು ಬ್ರಹ್ಮ ಚಾರಿಗಳೊಬ್ಬರು ಗಟ್ಟಿಯಾಗಿ ತಿಳಿಸುತ್ತಿದ್ದರು. ಶ್ರೀಮಾತೆಯವರು ಪುರುಷರೊಂದಿಗೆ ಮಾತ ನಾಡುವಾಗಲೆಲ್ಲ ತಮ್ಮ ಮುಖವನ್ನು ಸೆರಗಿನಿಂದ ಮುಚ್ಚಿಕೊಂಡೇ ಇರುತ್ತಿದ್ದರು.)

ಅಂತೂ ಈ ಅಪ್ರಿಯ ಘಟನೆಯಿಂದಾಗಿ ಸ್ವಾಮೀಜಿಯವರ ಮನಸ್ಥಿತಿ ಕದಡುವಂತಾಯಿತು. ಅಲ್ಲದೆ ಹೀಗೆ ಆರೋಗ್ಯ ಕೆಟ್ಟದ್ದರಿಂದ ಅವರು ತಮ್ಮ ಕಾಶ್ಮೀರ ಪ್ರವಾಸವನ್ನು ಅಲ್ಲಿಗೇ ಮುಕ್ತಾಯಗೊಳಿಸಿದರು. ಈಗ ಅವರು ಲಾಹೋರ್ ಮಾರ್ಗವಾಗಿ ಕಲ್ಕತ್ತಕ್ಕೆ ಹಿಂದಿರುಗಲಿದ್ದರು. ಅವರ ಪಾಶ್ಚಾತ್ಯ ಶಿಷ್ಯೆಯರು ಶ್ರೀನಗರದಿಂದ ಮುಂದಕ್ಕೆ ಉತ್ತರ ಭಾರತದ ಪ್ರವಾಸಕ್ಕೆ ಹೊರಡಲಿದ್ದರು. ಇವರಿಗೆ ಜೊತೆಗೊಡಲೆಂದು ಸ್ವಾಮೀಜಿಯವರು ಈಗಾಗಲೇ ಸ್ವಾಮಿ ಶಾರದಾ ನಂದರನ್ನು ಕಾಶ್ಮೀರಕ್ಕೆ ಆಹ್ವಾನಿಸಿದ್ದರು. ಅಲ್ಲದೆ ಅವರು ಸ್ವಾಮಿ ಸದಾನಂದರಿಗೂ ಒಂದು ತಂತಿ ಕಳಿಸಿ, ತಮ್ಮನ್ನು ಲಾಹೋರಿನಲ್ಲಿ ಕೂಡಿಕೊಳ್ಳುವಂತೆ ತಿಳಿಸಿದ್ದರು. ಅಲ್ಲಿಂದ ಮುಂದೆ ತಾವು ಹೋದಲ್ಲೆಲ್ಲ ಅವರನ್ನು ಕರೆದೊಯ್ಯುವುದು ಸ್ವಾಮೀಜಿಯವರ ಉದ್ದೇಶ. ಅಂತೆಯೇ ಸ್ವಾಮಿ ಶಾರದಾನಂದರು ಶ್ರೀನಗರಕ್ಕೆ ಬಂದು ತಲುಪಿದರು. ಈಗ ಸ್ವಾಮೀಜಿಯವರ ತಂಡ ಬರಾಮುಲ್ಲಾದ ಕಡೆಗೆ ದೋಣಿಯಲ್ಲಿ ಹೊರಟಿತು.

ಈ ಪ್ರಯಾಣದ ಸಂದರ್ಭದಲ್ಲಿ ಸ್ವಾಮೀಜಿ ಸಂಪೂರ್ಣ ಮೌನತಾಳಿ, ತಾವೇ ತಾವಾಗಿದ್ದು ಬಿಟ್ಟರು. ತಮಗಾಗಿದ್ದ ಬಗೆಬಗೆಯ ಅನುಭವಗಳಿಂದಾಗಿ ಅವರೆಷ್ಟು ಸೊರಗಿಹೋಗಿದ್ದರೆಂದರೆ, ಏನಾದರೂ ಹೆಚ್ಚುಕಡಮೆಯಾಗಬಹುದೆಂದು ಅವರ ಜೊತೆಗಾರರೆಲ್ಲ ಹೆದರಿದ್ದರು. ಕೆಲವೊಮ್ಮೆ ಸ್ವಾಮೀಜಿ ತಮ್ಮಷ್ಟಕ್ಕೆ ಹಾಡಿಕೊಳ್ಳುತ್ತಿದ್ದರು. ಅಪರೂಪಕ್ಕೊಮ್ಮೊಮ್ಮೆ ಮಾತನಾಡುವಾಗ ಹೇಳುತ್ತಿದ್ದರು, “ಇನ್ನು ನನ್ನಲ್ಲಿ ಯಾವ ಆಕಾಂಕ್ಷೆ-ನಿರೀಕ್ಷೆಗಳೂ ಉಳಿದಿಲ್ಲ. ಈಗ ನನ್ನ ಏಕಮಾತ್ರ ಬಯಕೆಯೆಂದರೆ ಗಂಗಾತೀರದಲ್ಲಿ ಮೌನವಾಗಿ, ದಿಗಂಬರ ಪರಿವ್ರಾಜಕನಾಗಿ ಅಲೆಯುವುದು, ಅಷ್ಟೆ–ನನಗಿನ್ನೇನೂ ಬೇಡ.” ಆದರೆ ಇದು ಕೇವಲ ನಿರ್ಲಿಪ್ತತೆಯ ಭಾವವಲ್ಲ. ಅವರನ್ನು ದಿವ್ಯ ಪ್ರೇಮದ ಭಾವವೊಂದು ಸಂಪೂರ್ಣವಾಗಿ ಆಕ್ರಮಿಸಿಬಿಟ್ಟಿತ್ತು.

ನಿವೇದಿತಾ ಬರೆಯುತ್ತಾಳೆ: “... ಅದೆಷ್ಟು ಬಲವತ್ತರವಾದ ಪ್ರೇಮವೆಂದರೆ ಅವರ ಕಡುಶತ್ರುವೂ ಆ ಪ್ರೇಮದ ರಭಸವನ್ನು ವಿರೋಧಿಸಲು ಸಾಧ್ಯವಿರಲಿಲ್ಲ... ಸ್ವಾಮೀಜಿಯವರ ಮಾತುಗಳ ಹಿಂದಿನ ಭಾವ ಅದೆಷ್ಟು ವಿಶಾಲವಾದದ್ದೆಂಬುದನ್ನು ವರ್ಣಿಸಲು ಅಸಾಧ್ಯ! –ಎಷ್ಟರಮಟ್ಟಿಗೆಂದರೆ ಈ ಜಗತ್ತಿನಲ್ಲಿ ಯಾವುದೇ ಜೀವಿಗೆ ಉಂಟಾದ ಸ್ವಲ್ಪ ನೋವೂ ಅವರನ್ನು ಮುಟ್ಟದಿರುವಂತಿರಲಿಲ್ಲ.

“ಸ್ವಾಮೀಜಿ ನಮಗೆ ವಸಿಷ್ಠ-ವಿಶ್ವಾಮಿತ್ರರ ಕಥೆ ಹೇಳಿದರು. ವಸಿಷ್ಠರ ನೂರು ಜನ ಪುತ್ರರು ಕೊಲ್ಲಲ್ಪಟ್ಟದ್ದು, ವಸಿಷ್ಠರು ಏಕಾಂಗಿಯಾಗಿ ನಿಸ್ಸಹಾಯಕರಾದದ್ದು–ಇವನ್ನೆಲ್ಲ ವಿವರಿಸಿದರು. ಬಳಿಕ ವಸಿಷ್ಠರು ಬೆಳ್ದಿಂಗಳ ರಾತ್ರಿಯಲ್ಲಿ ತಮ್ಮ ಕುಟೀರದಲ್ಲಿ ಕುಳಿತು ತಮ್ಮ ಪರಮ ವೈರಿಗಳಾದ ವಿಶ್ವಾಮಿತ್ರರು ಬರೆದಿದ್ದ ಯಾವುದೋ ಗ್ರಂಥವನ್ನು ಓದುತ್ತಿರುವ ಚಿತ್ರವನ್ನು ಬಣ್ಣಿಸಿದರು. ಆಗ ವಸಿಷ್ಠರ ಪತ್ನಿ ಅವರ ಬಳಿ ಸಾರಿ ಉದ್ಗರಿಸುತ್ತಾಳೆ, ‘ನೋಡಿ, ಚಂದ್ರಮ ಎಷ್ಟು ಪ್ರಕಾಶಮಾನನಾಗಿದ್ದಾನೆ!’ ಎಂದು. ಆಗ ವಸಿಷ್ಠರು ತಲೆಯೆತ್ತದೆ ಉತ್ತರಿಸುತ್ತಾರೆ, ‘ಆದರೆ ಅದಕ್ಕಿಂತ ಹತ್ತು ಸಾವಿರ ಪಾಲು ಪ್ರಕಾಶಮಾನವಾದದ್ದು, ಓ ಪ್ರಿಯೆ, ಈ ವಿಶ್ವಾಮಿತ್ರನ ಪ್ರಜ್ಞಾನ!’

“ವಸಿಷ್ಠರು ತಮ್ಮ ನೂರು ಜನ ಪುತ್ರರ ದುರ್ಮರಣ, ತಮ್ಮ ತಪ್ಪುಗಳು, ತಮ್ಮ ಕಷ್ಟಗಳು–ಎಲ್ಲವನ್ನೂ ಮರೆತುಬಿಟ್ಟಿದ್ದಾರೆ! ತಮ್ಮ ಶತ್ರುವಾದ ವಿಶ್ವಾಮಿತ್ರನ ಪ್ರಕಾಂಡ ಪಾಂಡಿತ್ಯವನ್ನು ಮನಗಂಡು ಆನಂದಿಸುತ್ತ ಮೈಮರೆತಿದ್ದಾರೆ! ಸ್ವಾಮೀಜಿ ನುಡಿದರು– ‘ಹಾಗೆಯೇ ನಮ್ಮ ಪ್ರೇಮವೆಂಬುದು ಕೂಡ ಎಲ್ಲ ವೈಯಕ್ತಿಕ ಪೂರ್ವ ಗ್ರಹಗಳನ್ನೂ ಮೀರಿ ನಿಂತಿರಬೇಕು’ ಎಂದು.”

ಸ್ವಾಮೀಜಿ ಮತ್ತು ಸಂಗಡಿಗರು ಅಕ್ಟೋಬರ್ ೧೧ರಂದು ಬರಾಮುಲ್ಲವನ್ನು ತಲುಪಿದರು. ಅಲ್ಲಿ ಕೆಲದಿನಗಳನ್ನು ಕಳೆದ ಮೇಲೆ ಅವರ ಪಾಶ್ಚಾತ್ಯ ಶಿಷ್ಯರು ದೆಹಲಿ, ಆಗ್ರಾ ಮುಂತಾದ ಪ್ರೇಕ್ಷ ಣೀಯ ಸ್ಥಳಗಳಿಗೆ ಹೋಗುವ ಏರ್ಪಾಡಾಗಿತ್ತು. ಸ್ವಾಮೀಜಿಯವರು ಲಾಹೋರ್, ರಾವಲ್ ಪಿಂಡಿಯ ಕಡೆಗೆ ಒಬ್ಬರೇ ಹೊರಟಿದ್ದರು. ಸ್ವಾಮೀಜಿ ಲಾಹೋರಿನ ಕಡೆಗೆ ಹೊರಟಾಗ ತಾವು ಅವರನ್ನು ಬೀಳ್ಕೊಟ್ಟ ಸಂದರ್ಭವನ್ನು ಸೋದರಿ ನಿವೇದಿತಾ ಹೀಗೆ ವರ್ಣಿಸುತ್ತಾಳೆ:

“ಸ್ವಾಮೀಜಿ ಹೊರಟುಬಿಟ್ಟಿದ್ದಾರೆ... ದೋಣಿಯವರು, ಅವರ ಹೆಂಡಿರು-ಮಕ್ಕಳು, ಆಳುಗಳು–ಎಲ್ಲರೂ ಸೇರಿ ನಾವೆಲ್ಲ ರಸ್ತೆಬದಿಯಲ್ಲಿ ಸಿದ್ಧವಾಗಿದ್ದ ಟಾಂಗಾದವರೆಗೆ ಅವರನ್ನು ಬೀಳ್ಕೊಡಲು ಬಂದೆವು. ಸ್ವಾಮೀಜಿಯವರು ತುಂಬ ಭಕ್ತಿ ಪ್ರೀತಿ ವಿಶ್ವಾಸಗಳನ್ನಿಟ್ಟುಕೊಂಡಿದ್ದ ಮುಖ್ಯ ಅಂಬಿಗನ ನಾಲ್ಕು ವರ್ಷದ ಪುಟ್ಟ ಮಗಳು ಅವರ ಪ್ರಯಾಣಕಾಲಕ್ಕೆಂದು ಹಣ್ಣುಗಳಿದ್ದ ಒಂದು ತಟ್ಟೆಯನ್ನು ತಲೆಯ ಮೇಲಿಟ್ಟುಕೊಂಡು ಅವರ ಜೊತೆಗೇ ಹೆಜ್ಜೆ ಹಾಕುತ್ತ ನಡೆದಳು. ಸ್ವಾಮೀಜಿ ಟಾಂಗಾ ಹತ್ತಿ ಹೊರಡುತ್ತಿದ್ದಂತೆ ಆ ಪುಟ್ಟ ಹುಡುಗಿ ನಗುತ್ತ ಕೈ ಬೀಸುತ್ತ ಬೀಳ್ಕೊಟ್ಟಳು. ನಮ್ಮ ಬಲಿತ ಬುದ್ಧಿ-ಹೃದಯಗಳ ಜಿಡುಕಿನಿಂದಾಗಿ ನಾವು ಆ ಮಗುವಿಗಿಂತ ಲಕ್ಷ ಪಾಲು ಕಡಿಮೆ ನಿಸ್ವಾರ್ಥಿಗಳಾಗಿದ್ದರೂ ಸ್ವಾಮೀಜಿಯವರ ಅಗಲುವಿಕೆಯಿಂದ ನಮಗಾದ ದುಃಖ ಆ ಮಗುವಿನ ದುಃಖಕ್ಕಿಂತೇನೂ ಕಡಿಮೆಯಿದ್ದಿರಲಾರದು.”

ಸಾರಾ ಬುಲ್ ಹಾಗೂ ಜೋಸೆಫಿನ್ನರೊಂದಿಗೆ ನಿವೇದಿತೆಯೂ ಉತ್ತರ ಭಾರತದ ಪ್ರವಾಸ ದಲ್ಲಿ ಜೊತೆಗೂಡಬೇಕೆಂದು ತೀರ್ಮಾನವಾಗಿತ್ತಾದರೂ, ಅವಳು ಬೇಗನೆ ಕಲ್ಕತ್ತಕ್ಕೆ ಹಿಂದಿರು ಗಲು ಬಯಸಿದಳು. ಕಾರಣ, ಸಾಧ್ಯವಾದಷ್ಟು ಬೇಗ ಕಲ್ಕತ್ತದಲ್ಲಿನ ತನ್ನ ಸೇವಾಕಾರ್ಯವನ್ನು ಪ್ರಾರಂಭಿಸಬೇಕೆಂಬ ತವಕ. ಆದ್ದರಿಂದ ಕೆಲದಿನಗಳಲ್ಲೇ ಅವಳು ತನ್ನ ಪಾಡಿಗೆ ತಾನು ಏಕಾಂಗಿ ಯಾಗಿ ಕಲ್ಕತ್ತದ ಕಡೆಗೆ ಹೊರಟಳು. ಅತ್ತ ಸ್ವಾಮೀಜಿಯವರೂ ತಮ್ಮ ಶಿಷ್ಯರಾದ ಸ್ವಾಮಿ ಸದಾನಂದರೊಂದಿಗೆ ಲಾಹೋರಿನಲ್ಲಿ ಟ್ರೈನು ಹತ್ತಿ ನೇರವಾಗಿ ಕಲ್ಕತ್ತಕ್ಕೆ ಬಂದು ತಲುಪಿದರು.

