
\chapter{ಪ್ರೇಮದ ಮೂರ್ತರೂಪ}

\noindent

ಸುಮಾರು ಆರು ತಿಂಗಳಷ್ಟು ದೀರ್ಘಕಾಲ ಮಠದಿಂದ ದೂರವಾಗಿದ್ದ ಸ್ವಾಮೀಜಿ, (೧೮೯೮) ಅಕ್ಟೋಬರ್ ೧೮ರಂದು ಮತ್ತೆ ಮಠಕ್ಕೆ ಮರಳಿದರು. ಅವರ ಆಗಮನ ಆಶ್ರಮವಾಸಿಗಳಿಗೆಲ್ಲ ಅತ್ಯಂತ ಅನಿರೀಕ್ಷಿತವಾದದ್ದು. ಸ್ವಾಮೀಜಿಯವರ ಓಡಾಟಗಳ ವಿವರಗಳನ್ನೆಲ್ಲ ಸೋದರಿ ನಿವೇದಿತಾ ತನ್ನ ಪತ್ರಗಳ ಮೂಲಕ ಮಠಕ್ಕೆ ತಿಳಿಸುತ್ತಿದ್ದಳು. ಅವುಗಳನ್ನು ವಿಶೇಷ ತರಗತಿಗಳ ಸಂದರ್ಭದಲ್ಲಿ ಎಲ್ಲರಿಗೂ ಓದಿ ಹೇಳಲಾಗುತ್ತಿತ್ತು. ಆದರೆ ಅವರು ಬರಾಮುಲ್ಲಾದಿಂದ ಹೊರಟ ಮೇಲೆ ಅವರ ಬಗ್ಗೆ ಯಾವ ವಿಷಯವೂ ತಿಳಿದುಬಂದಿರಲಿಲ್ಲ. ಈಗ ಇದ್ದಕ್ಕಿದ್ದಂತೆ ಆಗಮಿಸಿದ ಸ್ವಾಮೀಜಿಯವರನ್ನು ಕಂಡು ಮಠದಲ್ಲಿನ ಸಂನ್ಯಾಸಿಗಳಿಗೆಲ್ಲ ಎಲ್ಲಿಲ್ಲದ ಆನಂದ. ಆದರೆ ಅವರ ಸೊರಗಿಹೋದ ದುರ್ಬಲ-ರೋಗಿಷ್ಠ ಶರೀರವನ್ನು ಕಂಡು ಆ ಸಂತೋಷವೆಲ್ಲ ಮಾಯವಾಗಿ ಕಳವಳ, ಕಾತರ ತುಂಬಿಕೊಂಡಿತು.

ಸ್ವಾಮೀಜಿಯವರ ದೇಹಸ್ಥಿತಿ ನಿಜಕ್ಕೂ ಗಂಭೀರವಾಗಿತ್ತು. ಡಯಾಬಿಟೀಸ್ ಹಾಗೂ ಅನಿದ್ರೆ ಅವರಿಗೆ ಬಹಳ ಕಾಲದಿಂದ ಅಂಟಿಕೊಂಡಿದ್ದ ಕಾಯಿಲೆಗಳು. ಅವುಗಳೊಂದಿಗೆ ಈಗ ಉಬ್ಬಸದ ಸೂಚನೆಗಳು ಬೇರೆ ಕಂಡುಬಂದುವು. ಯಾವಾಗಲಾದರೊಮ್ಮೆ ಅವರಿಗೆ ಉಸಿರಾಟ ಸ್ವಲ್ಪ ಕಷ್ಟವಾಗುತ್ತಿದ್ದುದುಂಟು. ಆದರೆ ಈಗ ಅದು ತೀವ್ರವಾಗಿಬಿಟ್ಟಿತು. ಪರಿಣಾಮವಾಗಿ ದಣಿವು ತುಂಬ ಹೆಚ್ಚಾಯಿತು. ಮಠಕ್ಕೆ ಮರಳಿದೊಡನೆಯೇ ಹಾಸಿಗೆ ಹಿಡಿಯಬೇಕಾಯಿತು. ವಿಷಯ ತಿಳಿದು ಭಕ್ತಾದಿಗಳೆಲ್ಲ ಮಠಕ್ಕೆ ಧಾವಿಸಿ ಬಂದರು. ಗಿರೀಶ್ ಚಂದ್ರ ಘೋಷ್ ಮಠಕ್ಕೆ ಬಂದವನು ಸ್ವಾಮೀಜಿಯವರ ದರ್ಶನಕ್ಕಾಗಿ ಮಹಡಿಯ ಮೇಲೆ ಹೋಗುವಷ್ಟರಲ್ಲಿ ಅವರೇ ಕೆಳಗಿಳಿದು ಬಂದುಬಿಟ್ಟರು. ಇದನ್ನು ನೋಡಿ ಗಿರೀಶ ಅಚ್ಚರಿಯಿಂದ ಕೇಳಿದ, “ಇದೇನಿದು? ನಿಮಗೆ ತುಂಬ ಕಾಯಿಲೆಯಾಗಿದೆಯೆಂದು ಕೇಳಿದೆನಲ್ಲ, ನೀವೇಕೆ ಕೆಳಗೆ ಬಂದಿರಿ?” ಅದಕ್ಕೆ ಸ್ವಾಮೀಜಿ ತಮಾಷೆಯ ದನಿಯಲ್ಲಿ ಉತ್ತರಿಸಿದರು, “ರಾಜ (ಸ್ವಾಮಿ ಬ್ರಹ್ಮಾನಂದರು) ನನ್ನ ಬಗ್ಗೆ ಸುಮ್ಮನೆ ಗಾಬರಿಗೊಂಡಿದ್ದಾನೆ. ನಾನು ಸ್ವಲ್ಪ ನಿದ್ರೆ ಮಾಡೋಣ ಅಂತ ಕಣ್ಮುಚ್ಚಿದರೆ ಸಾಕು, ಅವನ ಕಾತರದ ಮುಖವೇ ಕಾಣುತ್ತದೆ! ಆದ್ದರಿಂದ ನಾನೀಗ ಓಡಾಡುತ್ತಿದ್ದೇನೆ–ಆಗಲಾದರೂ ಅವನು ಸ್ವಲ್ಪ ಸಮಾಧಾನಪಟ್ಟುಕೊಳ್ಳಬಹುದು! ಅವನು ನನ್ನನ್ನು ಕಾಯಿಲೆ ಮನುಷ್ಯನನ್ನಾಗಿ ಮಾಡಿಬಿಡುತ್ತಿದ್ದಾನೆ. ಆದರೆ ನಿಜಕ್ಕೂ ನಾನು ಚೆನ್ನಾಗಿಯೇ ಇದ್ದೇನೆ.” ಬಳಿಕ ಅವರು ಮಠದ ಪರಿಸ್ಥಿತಿಯ ಬಗ್ಗೆ ಮಾತನಾಡುತ್ತ ಬ್ರಹ್ಮಾನಂದರ ಬಗ್ಗೆ ಹೇಳುತ್ತಾರೆ, “ರಾಜನ ಕಾರ್ಯದಕ್ಷತೆ ಯನ್ನು ಕಂಡು ನಾನು ಬೆರಗಾಗಿಹೋದೆ. ಅವನು ಮಠವನ್ನು ಎಷ್ಟು ಚೆನ್ನಾಗಿ ನಡೆಸಿಕೊಂಡು ಹೋಗುತ್ತಿದ್ದಾನೆ! ಅವನ ಬಗ್ಗೆ ಶ್ರೀರಾಮಕೃಷ್ಣರೇ ಹೇಳುತ್ತಿದ್ದರು, ‘ಅವನು ಒಂದು ರಾಜ್ಯವನ್ನೇ ಆಳಬಲ್ಲ’ ಎಂದು.”

ಸ್ವಾಮೀಜಿ ತಾವು ಚೆನ್ನಾಗಿಯೇ ಇದ್ದೇವೆಂದು ಹೇಳುತ್ತಿದ್ದರಲ್ಲದೆ, ತಮ್ಮ ಕೆಲಸ ಕಾರ್ಯ ಗಳನ್ನೂ ಪುನರಾರಂಭಿಸಿದರು. ತರಗತಿಗಳನ್ನೂ ಪ್ರಾರಂಭಿಸಿದರು. ಆದರೆ ಅವರಲ್ಲಿ ಏನೋ ಸಂಪೂರ್ಣ ಬದಲಾವಣೆಯಾಗಿರುವುದು ಎಲ್ಲರಿಗೂ ಸ್ಪಷ್ಟವಾಗಿ ಗೋಚರವಾಯಿತು. ಯಾರೊ ಡನೆಯೂ ಅವರದು ಮಾತಿಲ್ಲ, ಕತೆಯಿಲ್ಲ. ಯಾವಾಗಲೂ ಅನ್ಯಮನಸ್ಕತೆ, ಮೌನ. ಯಾರಿಗೂ ಅವರೊಡನೆ ಮಾತನಾಡಲು ಧೈರ್ಯವಾಗಲಿಲ್ಲ. ಆರೋಗ್ಯಕ್ಕಿಂತ ಹೆಚ್ಚಾಗಿ ಅವರ ಈ ಭಾವ ಎಲ್ಲರನ್ನೂ ಕಂಗೆಡಿಸಿತು. ಅವರ ಮನಸ್ಸನ್ನು ಬಹಿರ್ಮುಖಗೊಳಿಸಲು ಬ್ರಹ್ಮಾನಂದರು ಬಹಳ ವಾಗಿ ಪ್ರಯತ್ನಿಸಿದರಾದರೂ ಪ್ರಯೋಜನವಾಗಲಿಲ್ಲ.

ಸ್ವಾಮೀಜಿ ಮಠಕ್ಕೆ ಹಿಂದಿರುಗಿದ ಎರಡು ಮೂರು ದಿನಗಳ ಮೇಲೆ, ಅವರ ಮೆಚ್ಚಿನ ಶಿಷ್ಯ ಶರಚ್ಚಂದ್ರ ಅವರನ್ನು ಕಾಣಲು ಬಂದ. ಅವನನ್ನು ಕಂಡು ಬ್ರಹ್ಮಾನಂದರು, “ನೋಡು, ಕಾಶ್ಮೀರದಿಂದ ಹಿಂದಿರುಗಿದಾಗಿನಿಂದ ಸ್ವಾಮೀಜಿ ಯಾರೊಡನೆಯೂ ಮಾತನಾಡುತ್ತಲೇ ಇಲ್ಲ. ನೀನು ಸ್ವಲ್ಪ ಅವರೊಂದಿಗೆ ಮಾತನಾಡುತ್ತ ಅವರ ಮನಸ್ಸನ್ನು ಈ ಕಡೆಗೆ ಸೆಳೆಯಲು ಪ್ರಯತ್ನಿಸು,” ಎಂದರು. ಶರಚ್ಚಂದ್ರನು ಸ್ವಾಮೀಜಿಯವರ ಕೋಣೆಯೊಳಗೆ ಪ್ರವೇಶಿಸಿದಾಗ ಅವರು ಧ್ಯಾನಮುದ್ರೆಯಲ್ಲಿ ಕುಳಿತಿದ್ದರು. ಅವರ ಮನಸ್ಸು ಯಾವುದೋ ಅತೀತದಲ್ಲಿ ತನ್ಮಯ ವಾಗಿರುವಂತೆ ಕಾಣುತ್ತಿತ್ತು. ಶರಚ್ಚಂದ್ರ ಮೆಲ್ಲನೆ ಒಳಗಡಿಯಿರಿಸಿದ. ಅವನನ್ನು ಕಂಡು ಸ್ವಾಮೀಜಿ ಪ್ರೀತಿಯ ದನಿಯಲ್ಲಿ “ಬಾ ಮಗು, ಕುಳಿತುಕೊ” ಎಂದರು; ಮತ್ತೆ ಮೌನದಲ್ಲಿ ಮುಳುಗಿದರು. ಶಿಷ್ಯ ನಮಸ್ಕರಿಸಿ ನೆಲದ ಮೇಲೆ ಕುಳಿತುಕೊಂಡ. ಗುರುವಿನ ಮುಖ ನೋಡಿದ– ಅವರ ಎಡಗಣ್ಣು ತುಂಬ ಕೆಂಪಗಿತ್ತು. “ಸ್ವಾಮೀಜಿ, ನಿಮ್ಮ ಎಡಗಣ್ಣು ಅಷ್ಟೊಂದು ಕೆಂಪಾಗಿದೆ ಯಲ್ಲ, ಯಾಕದು?”ಎಂದು ಕೆಳಿದ. ಅದಕ್ಕೆ ಸ್ವಾಮೀಜಿ, “ಓ ಅದೇ, ಅದೇನೂ ಇಲ್ಲ ಬಿಡು” ಎಂದು ಸುಮ್ಮನಾದರು. ಮತ್ತೆ ಮೌನ. ಹೀಗೆಯೇ ಬಹಳ ಹೊತ್ತಾದರೂ ಮೌನವನ್ನು ಮುರಿಯ ದಿದ್ದಾಗ ಶರಚ್ಚಂದ್ರ ಅವರ ಪಾದಗಳನ್ನು ಸ್ಪರ್ಶಿಸಿ, “ಸ್ವಾಮೀಜಿ, ನೀವು ಅಮರನಾಥದಲ್ಲಿ ಏನು ಕಂಡಿರೆಂಬುದನ್ನು ನನಗೆ ಹೇಳುವುದಿಲ್ಲವೆ?” ಎಂದ. ಶಿಷ್ಯನ ಉಪಾಯ ಫಲಿಸಿತು. ಈಗ ಅವರ ಗಾಢಮೌನ ಸ್ವಲ್ಪ ಭಂಗವಾಯಿತು; ಅವರ ಮನಸ್ಸು ಸ್ವಲ್ಪ ಹೊರಬಂದಿತು. ಮೆಲುದನಿಯಲ್ಲೇ ಹೇಳಿದರು, “ಅಮರನಾಥವನ್ನು ಸಂದರ್ಶಿಸಿದ ದಿನದಿಂದ, ಇಪ್ಪತ್ತನಾಲ್ಕು ಗಂಟೆಯೂ ಶಿವ ನನ್ನ ತಲೆಯ ಮೇಲೆಯೇ ಕುಳಿತಿರುವಂತಿದೆ; ಅವನೇಕೋ ಕೆಳಗಿಳಿದು ಬರಲೇ ಒಲ್ಲ!” ಇದನ್ನು ಕೇಳಿದ ಶಿಷ್ಯ ವಿಸ್ಮಯಮೂಕನಾಗಿ ಕುಳಿತು ಆಲೋಚಿಸಿದ–‘ಓ, ಸ್ವಾಮೀಜಿಯವರ ಮೌನಕ್ಕೆ ಇದೋ ಕಾರಣ!’

ಸ್ವಾಮೀಜಿ ತಮ್ಮ ಮಾತನ್ನು ಮುಂದುವರಿಸಿದರು, “ನಾನು ಅಮರನಾಥದಲ್ಲಿ ಮತ್ತು ಕ್ಷೀರಭವಾನಿಯಲ್ಲಿ ಬಹಳ ಕಠಿಣವಾದ ತಪಸ್ಸು ಮಾಡಿದೆ. ಅಮರನಾಥಕ್ಕೆ ಹೋಗುವಾಗ ನಾನು ಅತ್ಯಂತ ಕಡಿದಾದ ದಾರಿಯಲ್ಲಿ ಹೋದೆ. ಸಾಮಾನ್ಯವಾಗಿ ಯಾತ್ರಿಕರು ಆ ದಾರಿಯಾಗಿ ಹೋಗು ವುದಿಲ್ಲ. ಆದರೆ ನನಗೆ ಆ ದಾರಿಯಾಗೇ ಹೋಗಬೇಕೆನ್ನಿಸಿತು, ಹೋದೆ. ಆ ಶ್ರಮ ನನ್ನ ದೇಹಕ್ಕೆ ಅತಿಯಾಯಿತು. ಅಲ್ಲಿನ ಶೀತವೆಂದರೆ, ಮುಳ್ಳು ಚುಚ್ಚಿದಂತಿರುತ್ತದೆ. ನಾನು ಕೇವಲ ಕೌಪೀನ ಧಾರಿಯಾಗಿ, ಮೈಗೆ ವಿಭೂತಿ ಬಳಿದುಕೊಂಡು ಅಮರನಾಥ ಗುಹೆಯೊಳಗೆ ಪ್ರವೇಶಿಸಿದೆ. ಆಗ ನನಗೆ ಶೀತವೂ ಗೊತ್ತಾಗಲಿಲ್ಲ, ಉಷ್ಣವೂ ಗೊತ್ತಾಗಲಿಲ್ಲ; ಆದರೆ ನಾನು ಗುಹೆಯಿಂದ ಹೊರಗೆ ಬಂದಾಗ ಮಾತ್ರ ಥಂಡಿಯಿಂದ ನನ್ನ ಶರೀರ ಮರಗಟ್ಟಿಹೋಗಿತ್ತು... ”

ಬಳಿಕ ಅವರು, ಕ್ಷೀರಭವಾನಿಯಲ್ಲಿ ತಾವು ಕೇಳಿದ ಅಶರೀರವಾಣಿಯ ಬಗ್ಗೆ ತಿಳಿಸಿದರು. ಆಗ ಶರಚ್ಚಂದ್ರ ತನ್ನ ಸಂದೇಹವನ್ನು ಮುಂದಿಟ್ಟ, “ಆದರೆ ನೀವು ಅದನ್ನು ದೇವಿಯ ವಾಣಿಯೇ ಅಂತ ಹೇಗೆ ಹೇಳುವಿರಿ ಸ್ವಾಮೀಜಿ? ಅದು ನಿಮ್ಮ ಮನಸ್ಸಿನಲ್ಲಿ ಏಳುತ್ತಿದ್ದ ಪ್ರಬಲ ಭಾವನಾ ತರಂಗಗಳ ಪ್ರತಿಧ್ವನಿಯಷ್ಟೇ ಆಗಿರಬಹುದಲ್ಲವೆ?” ಸ್ವಾಮೀಜಿ ತುಂಬ ಗಂಭೀರ ದನಿಯಲ್ಲಿ ಉತ್ತರಿಸಿದರು, “ಅದು ನಿನ್ನೊಳಗಿನಿಂದಲೇ ಹೊರಹೊಮ್ಮಲಿ ಅಥವಾ ಹೊರಗಿನಿಂದಲೇ ಬಂದಿರಲಿ; ಆದರೆ ಆ ಧ್ವನಿ ಯಾವುದೋ ಅಶರೀರವಾಣಿಯಂತೆ ಕೇಳಿಬರದೆ, ಈಗ ನೀನು ನನ್ನ ಮಾತನ್ನು ಕೇಳುತ್ತಿರುವಷ್ಟೇ ಸ್ಪಷ್ಟವಾಗಿ ಕೇಳಿಬಂದರೆ, ಆಗಲೂ ನೀನು ಅದರ ಸತ್ಯತೆಯನ್ನು ಅನುಮಾನಿಸುವೆಯಾ?”. ಶರಚ್ಚಂದ್ರ ಮರುಮಾತನಾಡದೆ ಸುಮ್ಮನಾದ.

ಹೀಗೆ ಸ್ವಾಮೀಜಿ ಅಂದು ಬಹು ಹೊತ್ತು ಮಾತನಾಡಿದರೂ ಅವರು ಹೆಚ್ಚಾಗಿ ಮೌನ ವಾಗಿಯೇ ಇರುತ್ತಿದ್ದರು. ಎಷ್ಟೋ ಸಲ ಅವರ ಮನಸ್ಸು ಬಾಹ್ಯಪ್ರಪಂಚದಿಂದ ಎಷ್ಟರಮಟ್ಟಿಗೆ ವಿಮುಖವಾಗುತ್ತಿತ್ತೆಂದರೆ, ಅವರ ಕಿವಿಗೆ ಯಾವ ಮಾತೂ ಬೀಳುತ್ತಿರಲಿಲ್ಲ. ಒಮ್ಮೆ ಅವರು ತಮ್ಮ ಶಿಷ್ಯರಾದ ಸ್ವಾಮಿ ಶುದ್ಧಾನಂದರೊಂದಿಗೆ ಮಾತನಾಡುತ್ತ ಯಾವುದೋ ವಿಷಯದ ಬಗ್ಗೆ ಪ್ರಶ್ನಿಸಿದರು; ಶುದ್ಧಾನಂದರು ಅದಕ್ಕೆ ಉತ್ತರಿಸಿದರು. ಬಳಿಕ ಸ್ವಾಮೀಜಿ ಪುನಃ ಅದೇ ಪ್ರಶ್ನೆ ಯನ್ನು ಕೇಳಿದರು; ಶುದ್ಧಾನಂದರು ಮತ್ತೆ ಮೊದಲಿನಂತೆಯೇ ಉತ್ತರಿಸಿದರು. ಆದರೆ ಇದು ಅಲ್ಲಿಗೇ ನಿಲ್ಲಲಿಲ್ಲ. ಸ್ವಾಮೀಜಿ ಅದೇ ಪ್ರಶ್ನೆಯನ್ನು ಮತ್ತೆ ಮತ್ತೆ ಕೇಳತೊಡಗಿದರು. ಇಷ್ಟು ಹೊತ್ತಿಗಾಗಲೇ ಅವರ ಮನಸ್ಸು ಬೇರೆ ಯಾವುದೋ ವಿಷಯದ ಆಳಕ್ಕೆ ಪ್ರವೇಶಿಸಿಬಿಟ್ಟಿತ್ತು. ಅವರು ಸುಮ್ಮನೆ ಯಾಂತ್ರಿಕವಾಗಿ ಪ್ರಶ್ನೆ ಕೇಳುತ್ತಿದ್ದಾರೆ ಎಂಬುದು ಸ್ಪಷ್ಟವಾಗಿತ್ತು. ಇದರಿಂದ ಅವರ ಸೋದರಸಂನ್ಯಾಸಿಗಳೆಲ್ಲ ಕಳವಳಗೊಂಡರು. ಅವರ ಗಮನವನ್ನು ಬಾಹ್ಯಪ್ರಪಂಚದತ್ತ ಸೆಳೆಯಲು ಬಹುವಾಗಿ ಶ್ರಮಿಸಿದರು. ‘ನರೇಂದ್ರನಿಗೆ ತನ್ನ ನಿಜಸ್ವರೂಪ ಗೊತ್ತಾದೊಡನೆಯೇ ಅವನು ತನ್ನ ದೇಹವನ್ನು ತ್ಯಜಿಸಿಬಿಡುತ್ತಾನೆ’ ಎಂಬ ಶ್ರೀರಾಮಕೃಷ್ಣರ ಮಾತನ್ನು ನೆನಪಿಸಿ ಕೊಂಡು, ಹಾಗಾಗದಂತೆ ಮಾಡಲು ತೀವ್ರವಾಗಿ ಶ್ರಮಿಸಿದರು. ಆಧ್ಯಾತ್ಮಿಕ ವಿಚಾರವಾದ ಮಾತುಕತೆಗಳನ್ನೆಲ್ಲ ತಪ್ಪಿಸಿ ಕೇವಲ ಹಗುರವಾದ ವಿಷಯಗಳನ್ನೇ ಅವರ ಬಳಿ ಪ್ರಸ್ತಾಪಿಸು ತ್ತಿದ್ದರು. ತಮ್ಮ ಬಗ್ಗೆಯೇ ಸ್ವಾಮೀಜಿ ತಮಾಷೆ ಮಾಡುವಂತೆ ಅವಕಾಶ ಕಲ್ಪಿಸಿಕೊಡುತ್ತಿದ್ದರು. ಇನ್ನು ಕೆಲವೊಮ್ಮೆ ಅವರನ್ನು ಸಿಟ್ಟಿಗೆಬ್ಬಿಸಿಯಾದರೂ ಅವರ ಮನಸ್ಸನ್ನು ಬಾಹ್ಯ ಪ್ರಪಂಚದತ್ತ ಸೆಳೆಯುತ್ತಿದ್ದರು! ಪರಿಸ್ಥಿತಿಯನ್ನರಿತ ಸ್ವಾಮೀಜಿಯವರೂ ಸಹ ತಾವು ಸಮಾಧಿಸ್ಥಿತಿಗೇರದಂತೆ ಎಚ್ಚರಿಕೆ ವಹಿಸುತ್ತಿದ್ದರು. ಅದಕ್ಕಾಗಿ ಅವರು ಕೆಲವು ಆಪ್ತಭಕ್ತರ ಮನೆಗಳಿಗೆ ಹೋಗುತ್ತಿದ್ದರು. ಇಲ್ಲವೆ ಮಠಕ್ಕೆ ಬಂದ ಸಂದರ್ಶಕರನ್ನು ಮಾತನಾಡಿಸುತ್ತಿದ್ದರು.

ಸ್ವಾಮೀಜಿಯವರ ದೇಹಸ್ಥಿತಿ ಅವರ ಗುರುಭಾಯಿಗಳಿಗೆ ಬಹಳಷ್ಟು ಚಿಂತೆಯನ್ನುಂಟು ಮಾಡಿತು. ಆದ್ದರಿಂದ ಅವರು ಕಲ್ಕತ್ತಕ್ಕೆ ಮರಳಿದ ಕೆಲದಿನಗಳಲ್ಲೇ ಪ್ರಸಿದ್ಧ ವೈದ್ಯರಾದ ಆರ್. ಎಲ್. ದತ್ತ ಎಂಬವರಿಂದ ಎದೆಯನ್ನು ಪರೀಕ್ಷೆ ಮಾಡಿಸಿಕೊಳ್ಳಲು ಅವರನ್ನು ಒಪ್ಪಿಸಿದರು. ಡಾ. ದತ್ತರು ತಾವೂ ಪರೀಕ್ಷೆ ಮಾಡಿದ್ದಲ್ಲದೆ ಇನ್ನೂ ಕೆಲವು ಡಾಕ್ಟರುಗಳೊಂದಿಗೆ ಈ ವಿಷಯ ವಾಗಿ ಸಮಾಲೋಚಿಸಿ ಸ್ವಾಮೀಜಿ ತಮ್ಮ ದೇಹಾರೋಗ್ಯದ ಬಗ್ಗೆ ಹೆಚ್ಚಿನ ಎಚ್ಚರ ವಹಿಸಬೇಕು ಎಂಬ ಸಲಹೆ ನೀಡಿದರು. ಅತಿಯಾದ ಏಕಾಗ್ರತೆಯಿಂದಲೋ ಏನೋ ಅವರ ಎಡಗಣ್ಣಿನಲ್ಲಿ ಸ್ವಲ್ಪ ರಕ್ತ ಹೆಪ್ಪುಗಟ್ಟಿತ್ತು. ಆದ್ದರಿಂದ ಅವರು ಧ್ಯಾನದಲ್ಲಿ ತೊಡಗಬಾರದೆಂದು ಡಾ. ದತ್ತರು ಸೂಚಿಸಿದರು.

ಹೀಗೆ ತಮ್ಮ ಆರೋಗ್ಯಸ್ಥಿತಿ ಗಂಭೀರವಾಗಿರುವುದನ್ನು ಮನಗಂಡರೂ ಸ್ವಾಮೀಜಿ ಅದನ್ನು ಲೆಕ್ಕಿಸದೆ ಆಶ್ರಮವಾಸಿಗಳೊಂದಿಗೆ ತಮ್ಮ ಹಿಂದಿನ ದಿನಚರಿಯನ್ನು ಪುನರಾರಂಭಿಸಿದರು– ಪ್ರಶ್ನೋತ್ತರಗಳ ತರಗತಿಗಳು ನಡೆದುವು. ಆಧ್ಯಾತ್ಮಿಕ ಸಂಭಾಷಣೆಗಳು ನಡೆದುವು, ವೇದೋಪ ನಿಷತ್ತುಗಳನ್ನು ಓದಿ ಹೇಳಿ ಭಾಷ್ಯಗಳನ್ನೂ ವಿವರಣೆಗಳನ್ನೂ ನೀಡಲಾಯಿತು. ಮಠದ ನೂತನ ಬ್ರಹ್ಮಚಾರಿಗಳಿಗೆ ವಿಶೇಷ ತರಬೇತಿ ನೀಡುವುದಕ್ಕಾಗಿ ಸ್ವಾಮೀಜಿ ಹೆಚ್ಚಿನ ಸಮಯವನ್ನು ವಿನಿಯೋಗಿಸಿದರು. ಸಂಘದಲ್ಲಿ ಶಿಸ್ತು ನೆಲೆಗೊಳ್ಳುವುದಕ್ಕಾಗಿ ನೀತಿ ನಿಯಮಗಳನ್ನು ರೂಪಿಸಿ, ದಿನಕ್ಕೆ ಇಂತಿಷ್ಟು ಗಂಟೆಗಳ ಕಾಲ ಬೌದ್ಧಿಕ ಹಾಗೂ ಆಧ್ಯಾತ್ಮಿಕ ಸಾಧನೆಗಳಿಗಾಗಿ ಎಂದು ನಿಗದಿಪಡಿಸಿದರು. ಅವರು ಕಲ್ಕತ್ತಕ್ಕೆ ಮರಳಿದ ದಿನವೇ ತಾವು ಕಾಶ್ಮೀರದಲ್ಲಿದ್ದಾಗ ರಚಿಸಿದ ಕವನಗಳನ್ನು ಆಶ್ರಮವಾಸಿಗಳಿಗೆಲ್ಲ ತಮ್ಮ ವಿಶಿಷ್ಟವಾದ ಶೈಲಿಯಲ್ಲಿ ಭಾವಪೂರ್ಣವಾಗಿ ಓದಿ ಹೇಳಿದರು. ಅವರು ಉಚ್ಚರಿಸಿದ ಪ್ರತಿಯೊಂದು ಪದದಲ್ಲೂ ಅವರಿಗಾದ ಸಾಕ್ಷಾತ್ಕಾರಗಳೇ ಸಚೇತನವಾಗಿ ತುಂಬಿದ್ದಂತೆ ಭಾಸವಾಗುತ್ತಿತ್ತು.

ಸ್ವಾಮೀಜಿ ಕಲ್ಕತ್ತಕ್ಕೆ ಮರಳುವ ವೇಳೆಗೆ ಮಠದಲ್ಲಿ ಶ್ರೀದುರ್ಗಾಪೂಜೆಯ ಉತ್ಸವ ಪ್ರಾರಂಭ ವಾಯಿತು. ಸುಮಾರು ಹತ್ತು ವರ್ಷಗಳಿಂದಲೂ ಸ್ವಾಮೀಜಿಯವರಿಗೆ ಬಂಗಾಳದಲ್ಲಿ ದುರ್ಗಾ ಪೂಜೆಯ ಉತ್ಸವಗಳನ್ನು ನೋಡುವ ಅವಕಾಶವಾಗಿರಲಿಲ್ಲ. ಈ ವರ್ಷವಾದರೂ ಉತ್ಸವದಲ್ಲಿ ಭಾಗವಹಿಸಬೇಕೆಂಬ ಇಚ್ಛೆ ಅವರದ್ದಾಗಿತ್ತು. ಅವರು ಕಾಶ್ಮೀರದಿಂದ ಹೊರಟು ತರಾತುರಿ ಯಿಂದ ಕಲ್ಕತ್ತಕ್ಕೆ ಬಂದದ್ದಕ್ಕೆ ಇದೂ ಒಂದು ಕಾರಣ. ಅಕ್ಟೋಬರ್ ೧೯-೨ಂರಂದು ಸ್ವಾಮೀಜಿ ಶ್ರೀದುರ್ಗಾಹೋಮವನ್ನು ಮಾಡಿದರು. ಮುಂದಿನ ಮೂರು ದಿನಗಳು ವಿಜೃಂಭಣೆಯ ಉತ್ಸವ ಗಳು. ಇದರಲ್ಲಿ ಭಾಗವಹಿಸುವುದಕ್ಕಾಗಿ ಭಕ್ತಾದಿಗಳೆಲ್ಲ ನೆರೆದರು. ಆಲ್ಮೋರದಲ್ಲಿ ‘ಪ್ರಬುದ್ಧ ಭಾರತ’ದ ಕಾರ್ಯದಲ್ಲಿ ನೆರವಾಗುತ್ತಿದ್ದ ಸ್ವಾಮಿ ತುರೀಯಾನಂದರೂ ಆಗಮಿಸಿದರು. ಈಗ ಮಠದಲ್ಲಿ ಭಕ್ತಿ-ಆನಂದೋತ್ಸಾಹಗಳ ಸಂತೆಯೇ ನೆರೆಯಿತು. ಬಾರಾನಾಗೋರ್ ಮಠದ ದಿನ ಗಳು ಮತ್ತೆ ಮರುಕಳಿಸಿದಂತೆ ಕಂಡುಬರುತ್ತಿತ್ತು.

ಈ ವೇಳೆಗೆ ಮಠಕ್ಕೆ ಸೇರಿಕೊಂಡಿದ್ದ ಯುವ ಬ್ರಹ್ಮಚಾರಿಗಳಲ್ಲಿ ಕೆಲವರೆಂದರೆ ಖಗೇಂದ್ರ ಚಟರ್ಜಿ, ಹರಿಪದ ಚಟರ್ಜಿ, ಸುಧೀರ್​ಚಕ್ರವರ್ತಿ, ಆಂಧ್ರಪ್ರದೇಶದ ಕೃಷ್ಣಮೂರ್ತಿ ನಾಯ್ಡು, ಗೋವಿಂದ ಶುಕಲ್, ಮತ್ತು ದಕ್ಷಿಣೀರಂಜನ್ ಗುಹ. ಇವರೆಲ್ಲರೂ ಸ್ವಾಮಿ ವಿವೇಕಾನಂದರಿಂದ ಸಂನ್ಯಾಸದೀಕ್ಷೆಯನ್ನು ಸ್ವೀಕರಿಸಿ ಕ್ರಮವಾಗಿ ಸ್ವಾಮಿ ವಿಮಲಾನಂದ, ಸ್ವಾಮಿ ಬೋಧಾನಂದ, ಸ್ವಾಮಿ ಶುದ್ಧಾನಂದ, ಸ್ವಾಮಿ ಸೋಮಾನಂದ, ಸ್ವಾಮಿ ಆತ್ಮಾನಂದ ಹಾಗೂ ಸ್ವಾಮಿ ಕಲ್ಯಾಣಾ ನಂದ ಎಂಬ ಹೆಸರುಗಳನ್ನು ಪಡೆದರು. ಮುಂದೆ ಇವರಲ್ಲಿ ಒಬ್ಬೊಬ್ಬರೂ ರಾಮಕೃಷ್ಣ ಸಂಘದ ಸಮರ್ಥ ಸಾಧುಗಳಾಗಿ ಬೆಳಗಿದರು.

ನವೆಂಬರ್ ೧ನೇ ತಾರೀಕಿನ ನಂತರ ಸ್ವಾಮೀಜಿಯವರು ಆಗಾಗ ಬಾಗ್ ಬಜಾರಿನಲ್ಲಿದ್ದ ಬಲರಾಮ್ ಬಾಬುವಿನ ಮನೆಯಲ್ಲಿ ವಾಸವಾಗಿರಬೇಕಾಗಿ ಬಂದಿತು. ಕಾರಣ, ಅವರು ಔಷಧೋ ಪಚಾರ ಹಾಗೂ ಚಿಕಿತ್ಸೆ ಪಡೆಯುತ್ತಿದ್ದುದು ಕಲ್ಕತ್ತದಲ್ಲಿ. ಅವರಿಗೆ ವಿಶ್ರಾಂತಿ ಅತ್ಯಗತ್ಯವಾಗಿ ಬೇಕಾಗಿತ್ತು. ಆದರೂ ಅವರು ತಮ್ಮನ್ನು ನೋಡಬರುತ್ತಿದ್ದ ಜನಸಮೂಹಕ್ಕೆ ನಿರಾಸೆಯಾಗದಂತೆ ನೋಡಿಕೊಂಡರು. ಮುಂಜಾನೆಯಿಂದ ಹಿಡಿದು ರಾತ್ರಿ ಎಂಟು-ಒಂಬತ್ತು ಗಂಟೆಯವರೆಗೂ ಜನ ಗುಂಪುಗುಂಪಾಗಿ ಬರುತ್ತಿದ್ದರು. ಸಹಜವಾಗಿಯೇ ಇದರಿಂದ ಅವರ ಊಟ-ತಿಂಡಿಗಳ ಸಮಯವೆಲ್ಲ ಏರುಪೇರಾಯಿತು. ಇದರಿಂದ ಅವರ ಗುರುಭಾಯಿಗಳೂ ವಿಶ್ವಾಸಿಗರೂ ವ್ಯಾಕುಲ ಗೊಂಡರು. ಈ ಪರಿಸ್ಥಿತಿಯನ್ನು ಹೋಗಲಾಡಿಸಬೇಕೆಂದು ನಿಶ್ಚಯಿಸಿ ಅವರು, ನಿಗದಿಪಡಿಸಿದ ಅವಧಿಗಳಲ್ಲಲ್ಲದೆ ಬೇರೆ ಹೊತ್ತಿನಲ್ಲಿ ದರ್ಶನಾರ್ಥಿಗಳನ್ನು ಭೇಟಿಯಾಗಲೇಬಾರದೆಂದು ಸ್ವಾಮೀಜಿಯವರನ್ನು ಆಗ್ರಹ ಪಡಿಸಿದರು. ಆದರೆ ಅಧ್ಯಾತ್ಮದ ಅಮೃತವನ್ನರಸಿ ಬರುವ ಜನರಿಗಾಗಿ ಸ್ವಾಮೀಜಿಯವರ ಹೃದಯ ಕರಗದಿರಲು ಸಾಧ್ಯವೆ? ಅವರು ಹೇಳಿದ್ದಿಷ್ಟೆ: “ನನ್ನನ್ನು ನೋಡಬೇಕೆಂದು ಅವರೆಲ್ಲ ಎಷ್ಟೊಂದು ತೊಂದರೆ ಪಟ್ಟುಕೊಂಡು ದೂರದೂರದಿಂದ ಬರು ತ್ತಾರೆ. ಕೇವಲ ನನ್ನ ಅನಾರೋಗ್ಯಕ್ಕೆ ಹೆದರಿ ನಾನು ಅವರನ್ನೆಲ್ಲ ಮಾತನಾಡಿಸದಿರಲು ಸಾಧ್ಯವೆ?”

ಈ ದಿನಗಳಲ್ಲಿ ಸ್ವಾಮೀಜಿ ಪ್ರೇಮದ ಮೂರ್ತರೂಪವೇ ಆಗಿಬಿಟ್ಟಿದ್ದರು. ಅವರ ಪ್ರೇಮಸುಧೆ ಯೆಂಬುದು ಎಲ್ಲರೆಡೆಗೂ ಹರಿಯುತ್ತಿತ್ತು. ಸಜ್ಜನ-ದುರ್ಜನರೆನ್ನದೆ, ಪುಣ್ಯಾತ್ಮ-ಪಾಪಾತ್ಮರೆನ್ನದೆ ಸಕಲರೂ ಅವರ ಕೃಪೆಗೆ ಪಾತ್ರರಾಗುತ್ತಿದ್ದರು. ಜಗತ್ತಿನ ದುಃಖ ಅವರೆದೆಯನ್ನು ತೀವ್ರವಾಗಿ ತಟ್ಟುತ್ತಿತ್ತು. ತಮ್ಮ ಅವತಾರಸಮಾಪ್ತಿಕಾಲ ಸನ್ನಿಹಿತವಾಗುತ್ತಿದೆಯೆಂಬ ಅರಿವು ಅವರಲ್ಲಿ ಮೂಡಿದ್ದಿರಬೇಕು. ಆದ್ದರಿಂದ ಅವರು ಯಾರನ್ನೂ ಆಶೀರ್ವದಿಸದೆ ಕಳಿಸುತ್ತಲೇ ಇರಲಿಲ್ಲ. ಇದರ ಪರಿಣಾಮವಾಗಿ, ಬೇಕಾಬಿಟ್ಟಿ ಜೀವನ, ಪಾಪಿಷ್ಠ ಜೀವನ ನಡೆಸುತ್ತಿದ್ದ ಎಷ್ಟೋ ಜನರು ಆಧ್ಯಾತ್ಮಿಕ ಜಗತ್ತನ್ನು ಪ್ರವೇಶಿಸುವಂತಾಯಿತು. ಆದರೆ ಇದರಿಂದ ಕೆಲದಿನಗಳಲ್ಲೇ ವದಂತಿಗಳು ಹುಟ್ಟಿಕೊಂಡುವು–‘ಸ್ವಾಮೀಜಿಯವರನ್ನು ಸ್ವಲ್ಪ ಹೊಗಳಿಬಿಟ್ಟರೆ ಸಾಕು, ಯಾರು ಬೇಕಾದರೂ ಅವರನ್ನು ಸಂತೋಷಗೊಳಿಸಬಹುದು; ಏಕೆಂದರೆ ಮನುಷ್ಯನ ಅಂತರಾಳದೊಳಗಿನ ಹುಳುಕನ್ನು ಕಾಣುವ ಸಾಮರ್ಥ್ಯ ಅವರಿಗಿಲ್ಲ. ಹಾಗಲ್ಲದಿದ್ದರೆ, ಪಕ್ಕಾ ಪ್ರಾಪಂಚಿಕರ ಮೇಲೂ ಅವರು ಹೀಗೆ ಕೃಪೆ ಮಾಡುವುದೆಂದರೇನು?’ಎಂದು. ಈ ಮಾತುಗಳನ್ನೆಲ್ಲ ಕೇಳಿದ ಒಬ್ಬ ಶಿಷ್ಯನ ಮನಸ್ಸಿಗೆ ತುಂಬ ನೋವಾಯಿತು. ಈ ವಿಷಯವನ್ನು ಸ್ವಾಮೀಜಿಯವರಿಗೇ ನೇರವಾಗಿ ತಿಳಿಸಿಬಿಡಬೇಕೆಂದು ನಿಶ್ಚಯಿಸಿದ. ಒಂದು ದಿನ ಸ್ವಾಮೀಜಿ ಆಶ್ರಮದ ಆವರಣದಲ್ಲಿ ಒಬ್ಬರೇ ಅಡ್ಡಾಡುತ್ತಿದ್ದರು. ಆಗ ಈ ಶಿಷ್ಯ ಬಳಿಸಾರಿ ಹೇಳಿದ, “ಸ್ವಾಮೀಜಿ, ನಿಮ್ಮನ್ನು ನಾನೊಂದು ಪ್ರಶ್ನೆ ಕೇಳಬಹುದೆ?” ಸ್ವಾಮೀಜಿ ಕತ್ತೆತ್ತದೆ ನಡೆದಾಡುತ್ತಲೇ, “ಆಗಬಹುದು” ಎಂದರು. ಆಗ ಶಿಷ್ಯ ಹೇಳಿದ, “ಕೆಲವು ಜನ ಮಾತನಾಡಿಕೊಳ್ಳುತ್ತಿದ್ದಾರೆ–ಒಳ್ಳೆಯವರು ಯಾರು ಕೆಟ್ಟವರು ಯಾರು ಎಂಬುದನ್ನು ಗುರುತಿಸುವ ಸಾಮರ್ಥ್ಯವಿಲ್ಲ ನಿಮಗೆ ಅಂತ. ನೀವು ವ್ಯಕ್ತಿಗಳ ಪೂರ್ವಾಪರಗಳನ್ನರಿಯದೆ, ಅವರ ಸ್ವಭಾವಗಳನ್ನು ತಿಳಿದುಕೊಳ್ಳದೆ ಎಲ್ಲರ ಮೇಲೂ ನಿಮ್ಮ ಕೃಪೆಯನ್ನು ಹರಿಸುತ್ತೀರಿ. ಇದರಿಂದ ಎಷ್ಟೋ ಜನ ನಿಮ್ಮ ಶಿಷ್ಯರೆನ್ನಿಸಿಕೊಂಡರೂ ಸಹ ನಿಮ್ಮ ಆಶೀರ್ವಾದವನ್ನು ಪಡೆದೂ ತಮ್ಮ ಹಿಂದಿನ ಜೀವನವನ್ನೇ ಮುಂದುವರಿಸುತ್ತಿರುವುದನ್ನು ನಾವೇ ಕಾಣುತ್ತಿದ್ದೇವೆ.”

ಅವನು ಹೇಳುತ್ತಿದ್ದುದನ್ನೆಲ್ಲ ಸುಮ್ಮನೆ ಕೇಳುತ್ತಿದ್ದ ಸ್ವಾಮೀಜಿ ಇದ್ದಕ್ಕಿದ್ದಂತೆ ಅವನತ್ತ ತಿರುಗಿ ಭಾವಾವೇಶಭರಿತರಾಗಿ ನುಡಿದರು, “ಮಗು, ನಾನು ಒಬ್ಬ ಮನುಷ್ಯನನ್ನು ಅರ್ಥಮಾಡಿ ಕೊಳ್ಳಲಾರೆನೆಂದು ನೀನು ಹೇಳುವೆಯಾ?! ನಿಜ ಹೇಳಲೇನು? ನಾನು ಒಬ್ಬ ಮನುಷ್ಯನನ್ನು ನೋಡಿದಾಗ ಅವನ ಒಳಗಿರುವುದನ್ನೆಲ್ಲ ಅರಿಯಬಲ್ಲೆನಲ್ಲದೆ ಅವನ ಪೂರ್ವಜನ್ಮವನ್ನೂ ತಿಳಿಯಬಲ್ಲೆ. ಸ್ವತಃ ಅವನಿಗೇ ತನ್ನ ಸುಪ್ತ ಮನಸ್ಸಿನಲ್ಲಿ ಅಡಗಿರುವ ವಿಚಾರಗಳ ಅರಿವು ಇರುವುದಿಲ್ಲ. ಅವುಗಳನ್ನೂ ನಾನು ತಿಳಿದುಕೊಳ್ಳಬಲ್ಲೆ. ಹೀಗಿದ್ದರೂ ಕೂಡ ಅಂಥವರ ಮೇಲೆ ನಾನೇಕೆ ಕೃಪೆ ಮಾಡುತ್ತೇನೆ ಬಲ್ಲೆಯಾ? ಪಾಪ! ಆ ಬಡಜೀವಿಗಳು ಸ್ವಲ್ಪ ಮನಶ್ಶಾಂತಿಗಾಗಿ ಎಲ್ಲೆಡೆಗಳಲ್ಲಿ ಅಲೆದಾಡಿಬಿಟ್ಟಿದ್ದಾರೆ. ಆದರೆ ಎಲ್ಲರೂ ಅವರನ್ನು ತಿರಸ್ಕರಿಸಿದ್ದಾರೆ. ಆದ್ದರಿಂದ ಅವರು ಕಡೆಗೀಗ ನನ್ನ ಬಳಿಗೆ ಬಂದಿದ್ದಾರೆ. ನಾನೂ ಅವರನ್ನು ತಿರಸ್ಕರಿಸಿಬಿಟ್ಟರೆ ಇನ್ನು ಅವರಿಗೆ ದಿಕ್ಕು ತೋರುವವರು ಯಾರು? ಆದ್ದರಿಂದಲೇ ನಾನು ಭೇದ ಮಾಡಲಾರೆ. ಅಯ್ಯೋ, ಅವರೆಷ್ಟು ದುಃಖಿಗಳು! ಅವರೆಷ್ಟು ನಿರ್ಭಾಗ್ಯರು! ಈ ಪ್ರಪಂಚವೋ ಸಂಕಟಮಯವಾದುದು!” ಸಮಸ್ತ ವಿಶ್ವವನ್ನೇ ಪ್ರೀತಿಸುವ ವಿಶ್ವಮಾನವ ವಿವೇಕಾನಂದರ ಔದಾರ್ಯದ ಅಗಾಧತೆಯಿದು.

ಈ ಸಮಯದಲ್ಲೇ ಸೋದರಿ ನಿವೇದಿತಾ ತನ್ನ ಸಂಗಡಿಗರಿಂದ ಬೀಳ್ಗೊಂಡು ವಾರಾಣಸಿಯ ದಾರಿಯಾಗಿ ಕಲ್ಕತ್ತಕ್ಕೆ ಬಂದು ತಲುಪಿದಳು. ಮೊದಲು ಸ್ವಾಮೀಜಿಯವರನ್ನು ನೋಡುವ ಉದ್ದೇಶದಿಂದ ಅವಳು ನೇರವಾಗಿ ಬಲರಾಮಬಾಬುವಿನ ಮನೆಗೆ ಬಂದಳು. ಈ ಸಮಯದಲ್ಲಿ ಶ್ರೀಮಾತೆ ಶಾರದಾದೇವಿಯವರು ಬಾಗ್ ಬಜಾರಿನಲ್ಲೇ ಒಂದು ಮನೆಯಲ್ಲಿ ವಾಸವಾಗಿದ್ದು. ಸ್ವಾಮೀಜಿ ತಮ್ಮ ಶಿಷ್ಯೆಯನ್ನು ಆದರದಿಂದ ಸ್ವಾಗತಿಸಿ, ಬೇರೊಂದು ವ್ಯವಸ್ಥೆಯಾಗುವವರೆಗೆ ಅವಳು ಶ್ರೀಮಾತೆಯವರ ಜೊತೆಯಲ್ಲೇ ಇರುವ ವ್ಯವಸ್ಥೆ ಮಾಡಿದರು. ಇದಕ್ಕೆ ಶ್ರೀಮಾತೆ ಯವರಿಂದ ಯಾವ ಅಭ್ಯಂತರವೂ ಇರಲಿಲ್ಲ. ಏಕೆಂದರೆ ಅವರು ನಿವೇದಿತೆಯನ್ನು ಆಗಲೇ ತಮ್ಮ ಮಡಲಿಗೆ ತೆಗೆದುಕೊಂಡಾಗಿತ್ತು. ಆದರೆ ಶ್ರೀಮಾತೆಯವರೊಂದಿಗಿದ್ದ ಕೆಲವು ಮಡಿವಂತ ಹೆಂಗಸರು ಈ ‘ಮ್ಲೇಚ್ಛ ಯುವತಿ’ಯ ಆಗಮನದಿಂದಾಗಿ ಕಸಿವಿಸಿಗೊಂಡರು. ಪುಣ್ಯಕ್ಕೆ, ನಿವೇದಿತಾ ಆ ಮನೆಯಲ್ಲಿ ಹೆಚ್ಚು ದಿನ ಇರಬೇಕಾಗಿ ಬರಲಿಲ್ಲ. ಏಕೆಂದರೆ ಸುಮಾರು ಒಂದು ವಾರದಲ್ಲೇ ಆಕೆಗೆ ಬೋಸ್​ಪಾರಾ ಲೇನಿನಲ್ಲಿ ವಾಸಕ್ಕೊಂದು ಮನೆ ಸಿಕ್ಕಿತು. ಮುಂದಿನ ವರ್ಷ ಜೂನ್​ನಲ್ಲಿ ಮತ್ತೆ ಪಶ್ಚಿಮ ದೇಶಗಳಿಗೆ ತೆರಳುವವರೆಗೂ ಅವಳು ಈ ಮನೆಯಲ್ಲೇ ಸುಖವಾಗಿ ದ್ದಳು. ಆದರೆ ಸಾಧ್ಯವಾದಾಗಲೆಲ್ಲ ಮಧ್ಯಾಹ್ನದ ಹೊತ್ತಿಗೆ ಹಾಗೂ ಸೆಕೆಗಾಲದ ರಾತ್ರಿಗಳಲ್ಲಿ ಅವಳು ಶ್ರೀಮಾತೆಯವರ ಜೊತೆಯಲ್ಲಿ ಬಂದಿರುತ್ತಿದ್ದಳು.

ಕೆಲದಿನಗಳ ತರುವಾಯ, ಸಾರಾ ಬುಲ್ ಹಾಗೂ ಜೋಸೆಫಿನ್ನರು ಕಲ್ಕತ್ತಕ್ಕೆ ಬಂದು ತಲುಪಿ ದರು. ಅವರಿಬ್ಬರೂ ಕೆಲದಿನಗಳ ಮಟ್ಟಿಗೆ ನಿವೇದಿತೆಯ ಮನೆಯಲ್ಲಿ ಉಳಿದುಕೊಂಡರು. ಸಾರಾ ಬುಲ್ ಶ್ರೀಮಾತೆಯವರಿಗೆ ಹೇಳಿದಳು, “ಅಮ್ಮ, ನಾವು ನಿಮ್ಮ ಭಾವಚಿತ್ರವನ್ನು ತೆಗೆಸಿ ಅಮೆರಿಕೆಗೆ ತೆಗೆದುಕೊಂಡು ಹೋಗಿ ಅಲ್ಲಿ ಪೂಜೆ ಮಾಡುತ್ತೇವೆ” ಎಂದು. ಆದರೆ ಶ್ರೀಮಾತೆ ಯವರು ಅದಕ್ಕೆ ಒಪ್ಪಿಕೊಳ್ಳಲಿಲ್ಲ. ಲಜ್ಜಾಪಟಾವೃತೆಯಲ್ಲವೆ ಅವರು? ಸಾರಾ ಕೂಡ ಬಡಪಟ್ಟಿಗೆ ಬಿಡುವವಳಲ್ಲ. ಮತ್ತೆ ಮತ್ತೆ ಅವರನ್ನು ಕಾಡಿ ಬೇಡಿ ಒಪ್ಪಿಸಿಯೇಬಿಟ್ಟಳು. ಫೋಟೋಗ್ರಾಫರ್ ಬಂದು ಫೋಟೋ ತೆಗೆಯಲು ಸಿದ್ಧನಾದಾಗ ಶ್ರೀಮಾತೆಯವರು ನಿಮೀಲಿತ ನೇತ್ರರಾಗಿ ಧ್ಯಾನಭಾವದಲ್ಲಿ ಕುಳಿತುಬಿಟ್ಟರು. ಆಗ ಒಂದು ಫೋಟೋವನ್ನು ತೆಗೆಯಲಾಯಿತು. ಆದರೆ ಶ್ರೀಮಾತೆಯವರು ಕಣ್ಣು ತೆರೆದಿರುವ ಚಿತ್ರ ಸಿಕ್ಕಿದರೆ ಎಷ್ಟು ಚೆನ್ನ! ಹೀಗೆ ಆಲೋಚಿಸುತ್ತ ಸಾರಾ ಸಮಯ ಕಾಯುತ್ತಿದ್ದಳು. ಈಗ ಫೋಟೋ ತೆಗೆದಾಯಿತಲ್ಲ ಎಂದು ಶ್ರೀಮಾತೆಯವರು ಧ್ಯಾನಭಾವದಿಂದ ಸಹಜಸ್ಥಿತಿಗೆ ಬಂದು ಕಣ್ದೆರೆದರು. ಇದಕ್ಕಾಗಿಯೇ ಕಾಯುತ್ತಿದ್ದ ಫೋಟೋ ಗ್ರಾಫರ್ ತಕ್ಷಣ ಇನ್ನೊಂದು ಫೋಟೋವನ್ನು ತೆಗೆದುಕೊಂಡ. ಬಳಿಕ ನಿವೇದಿತೆ ಶ್ರೀಮಾತೆ ಯವರಿಗೆ ಅಭಿಮುಖಳಾಗಿ ಕುಳಿತಿರುವ ಮತ್ತೊಂದು ಫೋಟೋವನ್ನು ತೆಗೆಯಲಾಯಿತು. ಹೀಗೆ ಸಾರಾಳಿಂದ ನಮಗಿಂದು ಶ್ರೀಮಾತೆಯವರ ಕೆಲವು ಉತ್ಕೃಷ್ಟ ಛಾಯಾಚಿತ್ರಗಳು ಸಿಗುವಂತಾದುವು.

ನವೆಂಬರ್ ೧೩ರಂದು ಕಾಳೀ ಪೂಜೆ. ಅದರ ಹಿಂದಿನ ದಿನ ಬೇಲೂರು ಮಠದ ನೂತನ ನಿವೇಶನದ ವಿಶೇಷ ಪೂಜೆಯ ಕಾರ್ಯಕ್ರಮವಿತ್ತು. ಅಂದು ಶ್ರೀಮಾತೆಯವರು ತಮ್ಮ ಅನೇಕ ಭಕ್ತೆಯರಿಂದೊಡಗೂಡಿ ಅಲ್ಲಿಗೆ ಆಗಮಿಸಿದರು. ಸ್ವಾಮೀಜಿ ತಮ್ಮ ಪಾಶ್ಚಾತ್ಯ ಶಿಷ್ಯೆಯರನ್ನೂ ಆಹ್ವಾನಿಸಿದ್ದರು. ಮಠದ ಸಂನ್ಯಾಸಿಗಳೆಲ್ಲ ಅಲ್ಲಿ ಹಾಜರಿದ್ದರು. ಪೂಜೆಗಾಗಿ ಸುವಿಸ್ತಾರವಾದ ಏರ್ಪಾಡುಗಳಾಗಿದ್ದುವು. ಮಠದಲ್ಲಿ ಅಲ್ಲಿಯವರೆಗೂ ಪೂಜಿಸಲ್ಪಡುತ್ತಿದ್ದ ಶ್ರೀರಾಮಕೃಷ್ಣರ ಭಾವಚಿತ್ರವನ್ನು ನಿವೇಶನಕ್ಕೆ ತರಲಾಯಿತು. ಶ್ರೀಮಾತೆಯವರು ತಮ್ಮದೇ ಆದ ಶ್ರೀರಾಮಕೃಷ್ಣರ ಭಾವಚಿತ್ರವನ್ನೂ ತಂದಿದ್ದರು. ಅವರು ವಿಶೇಷ ಪೂಜೆ ಮಾಡಿ ಆ ಸ್ಥಳವನ್ನು ಆಶೀರ್ವದಿಸಿದರು. ಮಧ್ಯಾಹ್ನದ ಹೊತ್ತಿಗೆ ಶ್ರೀಮಾತೆಯವರು, ಸ್ವಾಮೀಜಿ, ಬ್ರಹ್ಮಾನಂದರು, ಶಾರದಾನಂದರು ಹಾಗೂ ಇನ್ನಿತರರು ಕಲ್ಕತ್ತಕ್ಕೆ ಹಿಂದಿರುಗಿದರು.

ಮರುದಿನ, ಎಂದರೆ ಕಾಳೀಪೂಜೆಯಂದು ಬಾಗ್​ಬಜಾರಿನಲ್ಲಿ ಸೋದರಿ ನಿವೇದಿತೆಯ ಹೆಣ್ಣು ಮಕ್ಕಳ ಶಾಲೆಯ ಪ್ರಾರಂಭೋತ್ಸವ. ಅಂದಿನ ಸಮಾರಂಭಕ್ಕೆ ಸ್ವಾಮೀಜಿ ಎಲ್ಲರನ್ನೂ ಆಹ್ವಾನಿಸಿ ದ್ದರು. ಶ್ರೀಮಾತೆಯವರೂ ಈ ಸಂದರ್ಭದಲ್ಲಿ ಹಾಜರಿದ್ದುದು ಸ್ವಾಮೀಜಿ ಹಾಗೂ ನಿವೇದಿತೆ– ಇಬ್ಬರಿಗೂ ಅತ್ಯಂತ ಸಂತೋಷವನ್ನುಂಟುಮಾಡಿತು. ಶ್ರೀಮಾತೆಯವರು ಮೊದಲು ಶ್ರೀರಾಮ ಕೃಷ್ಣರಿಗೆ ಪೂಜೆ ಸಲ್ಲಿಸಿ, “ಜಗದಂಬೆಯ ಕೃಪೆ ಈ ಶಾಲೆಯ ಮೇಲೆ ಸದಾಕಾಲಕ್ಕೂ ಇರಲಿ; ಮತ್ತು ಈ ಶಾಲೆಯಲ್ಲಿ ಶಿಕ್ಷಣ ಪಡೆದ ಹೆಣ್ಣು ಮಕ್ಕಳು ಆದರ್ಶ ಮಹಿಳೆಯರಾಗಿ ಬೆಳಗಲಿ” ಎಂದು ಹರಸಿದರು. ಶ್ರೀಮಾತೆಯವರ ಈ ವಿಶೇಷ ಕೃಪಾಶೀರ್ವಾದದ ಬಗ್ಗೆ ಸೋದರಿ ನಿವೇದಿತಾ ಬರೆಯುತ್ತಾಳೆ, “ಶ್ರೀಮಾತೆಯವರು ಭವಿಷ್ಯದ ವಿದ್ಯಾವಂತ ಹಿಂದೂ ಸ್ತ್ರೀತ್ವದ ಬಗ್ಗೆ ಆಡಿದ ಈ ಮಾತುಗಳಿಗಿಂತ ಶುಭಕರವಾದದ್ದೇನನ್ನೂ ನಾನು ಊಹಿಸಲಾರೆ!” ಎಂದು.

ನಾವು ಈ ಹಿಂದೆಯೇ ನೋಡಿದಂತೆ, ನಿವೇದಿತಾ ಇಂಗ್ಲೆಂಡಿನಲ್ಲಿ ಓರ್ವ ಉತ್ತಮ ಶಿಕ್ಷಕಿ ಯೆಂದು ಹಾಗೂ ಶಿಕ್ಷಣತಜ್ಞೆಯೆಂದು ಹೆಸರು ಗಳಿಸಿದ್ದವಳು. ವಿದ್ಯಾಭ್ಯಾಸದ ಬಗ್ಗೆ ಆಕೆ ತನ್ನದೇ ಆದ ಹಲವಾರು ಆಲೋಚನೆಗಳನ್ನು, ಅಭಿಪ್ರಾಯಗಳನ್ನು ಹೊಂದಿದ್ದಳು. ಅಲ್ಲದೆ ವಿದ್ಯಾಭ್ಯಾಸದ ಬಗೆಗಿನ ಸ್ವಾಮೀಜಿಯವರ ಅಭಿಪ್ರಾಯಗಳಿಂದ ಆಕೆ ತೀವ್ರವಾಗಿ ಪ್ರಭಾವಿತಳಾಗಿದ್ದಳು. ಅವಳು ಕಲ್ಕತ್ತದಲ್ಲಿ ಹೆಣ್ಣು ಮಕ್ಕಳಿಗಾಗಿ ಶಾಲೆಯೊಂದನ್ನು ಸ್ಥಾಪಿಸಬೇಕೆಂದು ತೀರ್ಮಾನವಾದಾಗಲೇ ಆ ಶಾಲೆಯ ರೂಪರೇಷೆಗಳು ಹೇಗಿರಬೇಕೆಂಬುದೂ ನಿಶ್ಚಯವಾಗಿತ್ತು. ಆ ಶಾಲೆಯ ಉದ್ದೇಶ ಕೇವಲ ಕೆಲವರಿಗೆ ಅಕ್ಷರದಾನ ಮಾಡುವುದಷ್ಟೇ ಆಗಿರಲಿಲ್ಲ. ಬದಲಾಗಿ ಸಮಸ್ತ ಭಾರತೀಯ ಸ್ತ್ರೀಯರಿಗೆ ಸಾರ್ವತ್ರಿಕವಾಗಿ ಅನ್ವಯವಾಗಬಹುದಾದ ಹಾಗೂ ಶ್ರೇಷ್ಠ ಗುಣಮಟ್ಟದ ಆಧುನಿಕ ಶಿಕ್ಷಣ ವಿಧಾನವೊಂದನ್ನು ಆವಿಷ್ಕರಿಸುವುದು ಈ ಶಾಲೆಯ ಉದ್ದೇಶವಾಗಿತ್ತು. ಸ್ವಾಮೀಜಿ ಕಲ್ಕತ್ತದಲ್ಲಿದ್ದಾಗಲೆಲ್ಲ ಅವಳನ್ನು ಭೇಟಿಯಾಗಿ ಭಾರತೀಯ ಮನೋಧರ್ಮದ ಬಗ್ಗೆ ಮತ್ತು ಆಕೆಯ ಕಾರ್ಯಕ್ಷೇತ್ರದ ಸ್ವರೂಪದ ಬಗ್ಗೆ ಹೆಚ್ಚಿನ ಅರಿವನ್ನು ನೀಡುತ್ತಿದ್ದರು. ಹೀಗೆ ತಾನು ಸ್ವಾಮೀಜಿಯವರಿಂದ ಪಡೆದುಕೊಂಡ ತಿಳಿವಳಿಕೆಯನ್ನು ಆಕೆ ತನ್ನ \eng{‘The Web of Indian Life’} (ಭಾರತೀಯ ಜೀವನದ ಹಾಸುಹೊಕ್ಕುಗಳು) ಎಂಬ ಪುಸ್ತಕದಲ್ಲಿ ನಿರೂಪಿಸಿದ್ದಾಳೆ. ಶ್ರೀಮಾತೆ ಯವರ ಮನೆಯಲ್ಲಿ ಆಕೆಗೆ ಅನೇಕ ಸಂಪ್ರದಾಯಸ್ಥ ಮಹಿಳೆಯರ ಸಂಪರ್ಕವಾಯಿತು. ಅವರಲ್ಲಿ ಅನೇಕರು ಪುರಾಣಗಳಲ್ಲಿ, ನಾಟಕಗಳಲ್ಲಿ ಹಾಗೂ ಹಿಂದೂ ಧರ್ಮದ ತತ್ತ್ವಗಳಲ್ಲಿ ಪರಿಣತ ರಾದವರು. ಅಲ್ಲದೆ ಹಿಂದೂಧರ್ಮವು ಬೋಧಿಸುವ ಮೌಲ್ಯಗಳಿಗೆ ಸಾಕ್ಷ್ಯದಂತಿದ್ದವರು. ಇಂತಹ ಸ್ತ್ರೀಯರ ಸಂಪರ್ಕವು ನಿವೇದಿತೆಯ ಈಗಿನ ಜೀವನಕ್ರಮಕ್ಕೆ ಒಂದು ವಿಶೇಷ ಅನುಕೂಲತೆಯನ್ನು ಕಲ್ಪಿಸಿಕೊಟ್ಟಿತು. ಏಕೆಂದರೆ ಸ್ವತಃ ಅವಳೂ ಒಬ್ಬ ಹಿಂದೂ ಬ್ರಹ್ಮಚಾರಿಣಿಯ ಜೀವನವನ್ನು ನಡೆಸುತ್ತಿದ್ದವಳಲ್ಲವೆ? ಶೀಘ್ರದಲ್ಲೇ ಅವಳು ಸಂಪೂರ್ಣ ಹಿಂದುವೇ ಆಗಿಬಿಟ್ಟಳು.

ಶಾಲೆಯ ಉದ್ಘಾಟನೆಯೊಂದಿಗೆ ಭಾರತದಲ್ಲಿ ನಿವೇದಿತೆಯ ಕಾರ್ಯ ಆರಂಭವಾಯಿತು. ಅವಳ ಈ ಕೆಲಸದಲ್ಲಿ ಸ್ವಾಮೀಜಿ ತೀವ್ರ ಆಸಕ್ತಿ ವಹಿಸಿದರು. ತನ್ನದೇ ಆದ ಆಲೋಚನೆಗಳನ್ನು ಕಾರ್ಯಗತಗೊಳಿಸಲು ಆಕೆಗೆ ಅವರು ಸಂಪೂರ್ಣ ಸ್ವಾತಂತ್ರ್ಯ ಕೊಟ್ಟರು. ಇಷ್ಟಪಟ್ಟರೆ ಆಕೆ ಇತರರ ಸಹಾಯವಿಲ್ಲದೆ ಏಕಾಂಗಿಯಾಗಿಯೇ ಕೆಲಸ ಮಾಡಬಹುದಾಗಿತ್ತು. ಎಲ್ಲಕ್ಕಿಂತ ಹೆಚ್ಚಾಗಿ, ಬೇಕಾದರೆ ಅವಳು ತನ್ನ ಕೆಲಸಕ್ಕೆ ಒಂದು ಸ್ಪಷ್ಟವಾದ ‘ಧಾರ್ಮಿಕ ಬಣ್ಣ’ವನ್ನೂ ಕೊಡಬಹು ದಾಗಿತ್ತು, ಅಥವಾ ಅದನ್ನು ಸಂಪೂರ್ಣವಾಗಿ ಒಂದು ಪಂಥವನ್ನಾಗಿಯೇ ಬೆಳೆಸಬಹುದಾಗಿತ್ತು. ಆದರೆ ಸ್ವಾಮೀಜಿ ಆಕೆಗೆ ಸೂಚಿಸಿದರು–“ಬೇರೆಲ್ಲ ಪಂಥಗಳನ್ನು ಮೀರಿ ಬೆಳೆಯುವ ಪಂಥ ನಿನ್ನದಾಗಲಿ” ಎಂದು. ಎಂದರೆ ಕಾಲಾಂತರದಲ್ಲಿ ಅವಳ ‘ಪಂಥ’ವು ಹಿಂದೂಧರ್ಮದೊಳಗಿನ ಹಲವು ಪಂಥಗಳನ್ನಲ್ಲದೆ ಹೊರಗಿನ ಇತರ ಎಲ್ಲ ಪಂಥಗಳನ್ನೂ ತನ್ನಲ್ಲಿ ಒಳಗೊಳ್ಳುವಂ ತಾಗಬೇಕು ಎಂಬುದು ಸ್ವಾಮೀಜಿಯವರ ಅಭಿಪ್ರಾಯ. ಒಮ್ಮೆ ಅವರು ಆಕೆಗೆ ಹೇಳಿದರು, “ಮುಂದಿನ ಜನಾಂಗದ ಸ್ತ್ರೀಯರು ತಮ್ಮ ಹೊಸ ಕಾರ್ಯಭಾರಗಳ ನಡುವೆ ಆಗಾಗ ‘ಶಿವ ಶಿವ!’ ಎನ್ನಲು ಸಾಧ್ಯವಾಗುವುದಾದರೆ ಅದೇ ಅವರ ಪಾಲಿಗೆ ಸಾಕಷ್ಟು ದೊಡ್ಡ ಪೂಜೆಯಾದಂತಾಗು ತ್ತದೆ” ಎಂದು. ಮಹಿಳೆಯರ ಉನ್ನತಿಗಾಗಿ ಶ್ರಮಿಸುವವರು ಹೇಗಿರಬೇಕು ಎಂಬುದರ ಬಗ್ಗೆ ತಮ್ಮ ಕಲ್ಪನೆಯನ್ನು ಆಕೆಗೆ ತಿಳಿಸುತ್ತ ಹೇಳುತ್ತಾರೆ–“ನಿಜ, ನಿನ್ನಲ್ಲಿ ಶ್ರದ್ಧೆ ಆಸಕ್ತಿ ವಿಶ್ವಾಸ ಎಲ್ಲ ಇದೆ. ಆದರೆ ಪ್ರಜ್ವಲಿಸುವ ಹುಮ್ಮಸ್ಸನ್ನು ಮಾತ್ರ ನಿನ್ನಲ್ಲಿ ನಾನು ಕಾಣುತ್ತಿಲ್ಲ. ನಿಜಕ್ಕೂ, ನಿನ್ನಲ್ಲಿ ಉಂಟಾದ ಶಕ್ತಿಯ ಜ್ವಾಲೆಯಲ್ಲಿ ನಿನ್ನನ್ನೇ ನೀನು ಆಹುತಿ ಮಾಡಿಕೊಂಡು ಬಿಡಬೇಕು.” ಹಾಗೆಯೇ ಅದು ನಡೆಯುವಂತೆ ಅವಳನ್ನು ಅವರು ಆಶೀರ್ವದಿಸಿದರು. ಮತ್ತು ಕಾಲಾಂತರದಲ್ಲಿ ಅವಳು ನಿಜಕ್ಕೂ ತನ್ನ ಉದ್ದೇಶಕ್ಕಾಗಿ ಆತ್ಮಾರ್ಪಣೆಯನ್ನೇ ಮಾಡಿಕೊಂಡಳು.

ನಿವೇದಿತಾ ಕಂಡುಕೊಂಡಿದ್ದ ತನ್ನ ಜೀವನದ ಪಥವನ್ನು ಸುಗಮಗೊಳಿಸಲು ಸ್ವಾಮೀಜಿ ತಮ್ಮಿಂದ ಸಾಧ್ಯವಾದದ್ದನ್ನೆಲ್ಲ ಮಾಡಿದರು. ಕೆಲವೊಮ್ಮೆ ಆಕೆಗೆ ತಮ್ಮ ಜೊತೆಯಲ್ಲೇ ಊಟ- ಉಪಾಹಾರಗಳನ್ನು ಸ್ವೀಕರಿಸುವಂತೆ ಹೇಳುತ್ತಿದ್ದರು. ಇನ್ನು ಕೆಲವೊಮ್ಮೆ ತಾವೇ ಆಕೆಗಾಗಿ ವಿಶೇಷ ಭಕ್ಷ್ಯಭೋಜ್ಯಗಳನ್ನು ತಯಾರಿಸುವುದಲ್ಲದೆ ತಮ್ಮೆದುರಿನಲ್ಲಿಯೇ ಅವುಗಳನ್ನು ತಿನ್ನುವಂತೆ ಹೇಳುತ್ತಿದ್ದರು. ಏಕೆಂದರೆ ಅವಳು ಕಟ್ಟುನಿಟ್ಟಾದ ವ್ರತನಿಷ್ಠೆಗಳನ್ನು ಆಚರಿಸುತ್ತಿದ್ದಳು. ಕೇವಲ ಹಾಲು-ಹಣ್ಣುಗಳ ಮೇಲೆ ಆಕೆ ಜೀವನ ನಡೆಸುತ್ತಿದ್ದಳು. ಅವಳು ಮಲಗುತ್ತಿದ್ದುದು ಒಂದು ಮರದ ಹಲಗೆಯ ಮೇಲೆ. ಇದೆಲ್ಲ ಸ್ವಾಮೀಜಿಯವರಿಗೆ ತಿಳಿದಿತ್ತು. ಆಗಾಗ ಅವರು ತಮಗಾಗಿ ಏನಾದರೂ ರುಚಿಕರವಾದ ಪದಾರ್ಥವನ್ನು ತಯಾರಿಸುವಂತೆ ಅವಳಿಗೆ ಹೇಳುತ್ತಿದ್ದರು. ಅದರಲ್ಲಿ ಅವಳೂ ಪಾಲ್ಗೊಳ್ಳಲಿ ಎನ್ನುವುದೇ ಅವರ ಉದ್ದೇಶ. ಅಲ್ಲದೆ, ಆಕೆ ತಯಾರಿಸಿದ ಆಹಾರವನ್ನು ಇತರರೂ ಸ್ವೀಕರಿಸುವಂತೆ ಮಾಡಿ ತನ್ಮೂಲಕ ತಮ್ಮವರಲ್ಲೇ ಅನೇಕ ಸಂಪ್ರದಾಯಸ್ಥರಿಗೆ ಆಕೆಯ ಬಗ್ಗೆ ಇದ್ದ ಮಡಿತನದ ತಿರಸ್ಕಾರ ದೃಷ್ಟಿಯನ್ನು ಹೋಗಲಾಡಿಸುವ ಪ್ರಯತ್ನ ಮಾಡಿದರು. ಹಿಂದೂ ಸಮಾಜವು ಆಕೆಯನ್ನು ತನ್ನವಳೆಂದೇ ಸ್ವೀಕರಿಸುವಂತೆ ಮಾಡಲು ಸ್ವಾಮೀಜಿ ಬಹಳ ವಾಗಿ ಪ್ರಯತ್ನಪಟ್ಟರು. ಯಾವುದಾದರೊಂದು ವಿಚಾರಗೋಷ್ಠಿ ನಡೆಯುತ್ತಿರುವಾಗ ಆಕೆಯ ಅಭಿಪ್ರಾಯವನ್ನು ಕೇಳಲು ಅವರು ಸದಾ ಸಿದ್ಧರಾಗಿದ್ದರು. ಅವಳು ತಮ್ಮ ಸಂದೇಶಗಳನ್ನು ಎಷ್ಟರಮಟ್ಟಿಗೆ ಮೈಗೂಡಿಸಿಕೊಂಡಿದ್ದಾಳೆಂಬುದನ್ನು ತಿಳಿದುಕೊಳ್ಳುವುದು ಅವರ ಒಂದು ಉದ್ದೇಶವಾದರೆ, ಅವಳ ಭಾವನೆಗಳಲ್ಲಿ ಕಂಡುಬರಬಹುದಾದ ಅಸಮಂಜಸತೆಗಳನ್ನು ತಿಳಿದು ತಿದ್ದುವುದು ಇನ್ನೊಂದು ಉದ್ದೇಶ. ಅವಳ ವಿಷಯದಲ್ಲಿ ಮಾತ್ರವಲ್ಲ, ತಮ್ಮ ಸ್ವಂತ ಸಂಪ್ರದಾ ಯಸ್ಥ ಶಿಷ್ಯರ ವಿಚಾರದಲ್ಲೂ ಸ್ವಾಮೀಜಿಯವರು ಅಷ್ಟೇ ನಿಷ್ಠುರರಾಗಿದ್ದರು. ಅವರೆಲ್ಲ ಬಾಲ್ಯದಿಂದ ಬೆಳೆಸಿಕೊಂಡು ಬಂದಿದ್ದ ಅರ್ಥಹೀನ ಕಂದಾಚಾರಗಳನ್ನು ಬುಡಮೇಲು ಮಾಡಿ ಬಿಡುತ್ತಿದ್ದರು. ಎಷ್ಟೋ ಸಲ ತಮ್ಮ ಬಗೆಗಿನ ಅವರ ನಿಷ್ಠೆಯನ್ನು ಪರೀಕ್ಷಿಸಲು, ಸಂಪ್ರದಾಯಸ್ಥ ರಿಂದ ಬಹಿಷ್ಕರಿಸಲ್ಪಟ್ಟ ಆಹಾರವನ್ನು ಮುಂದಿಟ್ಟು “ತೆಗೆದುಕೊಳ್ಳಿ, ಇದು ನನ್ನ ಪ್ರಸಾದ” ಎನ್ನುತ್ತಿದ್ದರು.

ಈ ಸಮಯದಲ್ಲಿ ಸ್ವಾಮೀಜಿಯವರಿಗೆ ಮುಂಬಯಿಯ ಸುಪ್ರಸಿದ್ಧ ಕೈಗಾರಿಕೋದ್ಯಮಿಗಳಾದ ಜೆಮ್ ಶೇಟ್​ಜಿ ಟಾಟಾರವರಿಂದ ಬಂದ ಪತ್ರವೊಂದು ಸ್ವಾರಸ್ಯಕರವಾಗಿದೆ. ಸ್ವಾಮೀಜಿ ಈ ಲೋಕೋಪಕಾರಿ ಕೋಟ್ಯಧಿಪತಿಯನ್ನು ಜಪಾನಿನಲ್ಲಿ ಈ ಹಿಂದೆಯೇ ಸಂಧಿಸಿದ್ದರು. ೧೮೯೩ರ ಜುಲೈನಲ್ಲಿ ಯೊಕೋಹಾಮಾದಿಂದ ವ್ಯಾಂಕೋವರ್​ಗೆ ಹೊರಟಿದ್ದ ಹಡಗಿನಲ್ಲೂ, ಅನಂತರ ವ್ಯಾಂಕೋವರ್​ನಿಂದ ಶಿಕಾಗೋದವರೆಗೆ ರೈಲಿನಲ್ಲೂ ಇಬ್ಬರೂ ಸಹಪ್ರಯಾಣಿಕರಾಗಿದ್ದು. ಸ್ವಾಮೀಜಿಯವರು ಜಪಾನಿನ ಕೈಗಾರಿಕಾ ಕ್ಷೇತ್ರದಲ್ಲಿನ ಮುನ್ನಡೆಯಿಂದ ತುಂಬ ಪ್ರಭಾವಿತರಾಗಿ ಅಂತಹ ಬೆಳವಣಿಗೆ ಭಾರತದಲ್ಲೂ ಸಾಧ್ಯವಾದೀತೇ ಎಂದು ಆಶಿಸುತ್ತಿದ್ದರು. ಬಹುಶಃ ಸರ್ ಜೆಮ್ ಶೇಟ್​ಜಿ ಟಾಟಾರೊಂದಿಗೂ ಸ್ವಾಮೀಜಿ ಅದೇ ಧಾಟಿಯಲ್ಲಿ ಮಾತನಾಡಿರಬೇಕು. ಆದ್ದರಿಂದ ಸ್ವಾಮೀಜಿಯವರ ದೃಷ್ಟಿಕೋನವನ್ನು ಅರಿತಿದ್ದ ಟಾಟಾರವರು ತಮ್ಮ ಹೊಸ ಕಾರ್ಯ ಕ್ರಮವೊಂದು ಅವರನ್ನು ಆಕರ್ಷಿಸಬಹುದೆಂದು ಭಾವಿಸಿ ಆ ಪತ್ರವನ್ನು ಬರೆದಿದ್ದಿರಬೇಕು. ಪತ್ರ ಈ ರೀತಿಯಾಗಿತ್ತು:

ಪ್ರಿಯ ಸ್ವಾಮಿ ವಿವೇಕಾನಂದರಿಗೆ,

ಜಪಾನಿನಿಂದ ಶಿಕಾಗೋದವರೆಗಿನ ನಿಮ್ಮ ಪ್ರಯಾಣದಲ್ಲಿ ನಾನು ನಿಮ್ಮ ಸಹಪ್ರಯಾಣಿಕ ನಾಗಿದ್ದುದು ನಿಮಗೆ ನೆನಪಿರಬಹುದೆಂದು ಭಾವಿಸುತ್ತೇನೆ. ಈ ಸಂದರ್ಭದಲ್ಲಿ, ಭಾರತೀಯ ಆಧ್ಯಾತ್ಮಿಕ ಪರಂಪರೆಯ ಚೇತನವನ್ನು ನಾಶಮಾಡದೆ ಅದನ್ನು ಸದುಪಯೋಗಪಡಿಸಿ ಕೊಳ್ಳುವ ನಮ್ಮ ಕರ್ತವ್ಯದ ಬಗೆಗಿನ ನಿಮ್ಮ ಅಭಿಪ್ರಾಯಗಳನ್ನು ನಾನು ಬಹಳವಾಗಿ ನೆನಪಿಸಿ ಕೊಳ್ಳುತ್ತಿದ್ದೇನೆ.

ನೀವು ಈಗಾಗಲೇ ಕೇಳಿ ಅಥವಾ ಓದಿ ತಿಳಿದಿರಬಹುದಾದ ‘ಭಾರತೀಯ ವಿಜ್ಞಾನ ಸಂಶೋ ಧನಾ ಸಂಸ್ಥೆ’ಯ \eng{(Research Institute of Science for India)} ಯೋಜನೆಗೆ ಸಂಬಂಧಿಸಿ ದಂತೆ, ನಾನು ನಿಮ್ಮ ಆ ಅಭಿಪ್ರಾಯಗಳನ್ನು ನೆನಪಿಸಿಕೊಳ್ಳುತ್ತಿದ್ದೇನೆ. ನಮ್ಮ ಆಧ್ಯಾತ್ಮಿಕ ಚೈತನ್ಯಭರಿತರಾದಂತಹ ವ್ಯಕ್ತಿಗಳಿಗೆ ಆಶ್ರಮಗಳನ್ನು ಕಟ್ಟಿಸಿಕೊಟ್ಟು ಅಲ್ಲಿ ಈ ವ್ಯಕ್ತಿಗಳು ಸರಳತೆಯಿಂದ ಜೀವಿಸುತ್ತ ನೈಸರ್ಗಿಕ ಹಾಗೂ ಮಾನವಿಕ ಶಾಸ್ತ್ರಗಳನ್ನು ಅಧ್ಯಯಿಸಲು ತಮ್ಮ ಜೀವನಗಳನ್ನು ಮುಡಿಪಾಗಿಡುವಂತಾದರೆ ಈ ಆಧ್ಯಾತ್ಮಿಕ ಸಂಸ್ಕೃತಿಯನ್ನು ಸದುಪಯೋಗ ಪಡಿಸಿಕೊಳ್ಳಲು ಅದಕ್ಕಿಂತ ಉತ್ತಮ ವಿಧಾನವಿರಲಾರದು ಎಂದು ನನಗನ್ನಿಸುತ್ತದೆ. ಇಂತಹ ತಪಶ್ಚರ್ಯೆಯ ಪರವಾಗಿ ಒಬ್ಬ ಸಮರ್ಥ ನಾಯಕನಿಂದ ಒಂದು ಮಹಾಕಾರ್ಯವೇನಾ ದರೂ ಕೈಗೆತ್ತಿಕೊಳ್ಳಲ್ಪಟ್ಟರೆ ಅದರಿಂದ ಆಧ್ಯಾತ್ಮಿಕತೆಗೂ ವಿಜ್ಞಾನಕ್ಕೂ ಮತ್ತು ನಮ್ಮ ಮಾತೃಭೂಮಿಯ ಕೀರ್ತಿ ಹೆಚ್ಚಲೂ ಬಹಳ ಸಹಾಯವಾಗುವುದೆಂದೂ ನಾನು ನಂಬುತ್ತೇನೆ. ಮತ್ತು ಇಂತಹ ಒಂದು ಚಳವಳಿಗೆ ವಿವೇಕಾನಂದರಿಗಿಂತ ಯೋಗ್ಯರಾದ ದಂಡನಾಯಕರನ್ನು ನಾನರಿಯೆ. ಹೀಗೆ ನಮ್ಮ ಪುರಾತನವಾದ ಭವ್ಯ ಸಂಪ್ರದಾಯಗಳನ್ನು ಪುನರುಜ್ಜೀವನಗೊಳಿ ಸುವ ಈ ಕಾರ್ಯಭಾರವನ್ನು ವಹಿಸಿಕೊಳ್ಳಲು ಕೃಪೆಮಾಡಿ ಒಪ್ಪುವಿರಾ? ಹಾಗೆ ನಿಮ್ಮ ಸಮ್ಮತಿ ಇರುವುದಾದರೆ, ಪ್ರಾಯಶಃ ಪ್ರಖರವಾದ ಕರಪತ್ರವೊಂದರಿಂದ ಜನಜಾಗೃತಿಯ ನ್ನುಂಟುಮಾಡುವುದರ ಮೂಲಕ ಈ ಕಾರ್ಯವನ್ನು ಆರಂಭಿಸಬಹುದು. ಅದರ ಪ್ರಕಟಣೆಯ ಸಂಬಂಧದಲ್ಲಿ ಎಲ್ಲ ಖರ್ಚುಗಳನ್ನೂ ವಹಿಸಿಕೊಳ್ಳಲು ನಾನು ಬಹಳ ಸಂತೋಷಿಸುತ್ತೇನೆ.

\textbf{೨೩ನೇ ನವೆಂಬರ್ ೧೮೯೮\\ಎಸ್​ಪ್ಲನೇಡ್ ಹೌಸ್, ಮುಂಬಯಿ}

\begin{flushright}
\textbf{ಶುಭ ಹಾರೈಕೆಗಳೊಂದಿಗೆ\\ನಿಮ್ಮ ವಿಶ್ವಾಸಿ\\ಜೆಮ್ ಶೇಟ್​ಜಿ ಎನ್. ಟಾಟಾ}
\end{flushright}

ಸ್ವಾಮೀಜಿಯವರು ಟಾಟಾರ ಕಾರ್ಯಯೋಜನೆಯ ಬಗ್ಗೆ ತಮ್ಮ ಮೆಚ್ಚುಗೆಯನ್ನು ವ್ಯಕ್ತ ಗೊಳಿಸಿದರಲ್ಲದೆ ಕೆಲವು ಅಮೂಲ್ಯ ಸಲಹೆಗಳನ್ನೂ ನೀಡಿದರು. ಬೆಂಗಳೂರಿನಲ್ಲಿ ನೆಲೆಗೊಂಡಿ ರುವ \eng{Indian Institute of Science–‘}ಭಾರತೀಯ ವಿಜ್ಞಾನ ಸಂಸ್ಥೆ’ ಎಂಬುದು ಅಸ್ತಿತ್ವಕ್ಕೆ ಬಂದಿರುವುದು ಸ್ವಾಮೀಜಿಯವರ ಸಲಹೆಯ ಮೇರೆಗೇ. ಅಲ್ಲಿನ ಆ ಪ್ರಶಾಂತವಾದ ಅರಣ್ಯ ಪ್ರದೇಶವನ್ನು ಸೂಚಿಸಿದವರೂ ಅವರೇ ಎಂದು ಕೂಡ ತಿಳಿದುಬಂದಿದೆ.

