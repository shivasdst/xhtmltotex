
\chapter{ಆತ್ಮೀಯರ ಆತಿಥ್ಯ}

\noindent

ಸ್ವಾಮೀಜಿ ಲಾಹೋರಿನಲ್ಲಿದ್ದದ್ದು ಒಟ್ಟು ಹತ್ತು ದಿನ. ಆದರೆ ಅದು ಲಾಹೋರ್​ನಗರದ ಚರಿತ್ರೆಯಲ್ಲೇ ಒಂದು ಪ್ರಮುಖ ಘಟನೆಯಾಗಿ ಪರಿಣಮಿಸಿತು. ಅವರನ್ನು ಇನ್ನಷ್ಟು ದಿನ ತಮ್ಮಲ್ಲಿಯೇ ಉಳಿಸಿಕೊಳ್ಳಲು ಲಾಹೋರಿನ ನಾಗರಿಕರು ಬಹಳವಾಗಿ ಪ್ರಯತ್ನಪಟ್ಟರೂ ಅದು ಸಾಧ್ಯವಾಗಲಿಲ್ಲ. ಏಕೆಂದರೆ ಆ ದಿನಗಳಲ್ಲಿ ವಿರಾಮವಿಲ್ಲದೆ ಶ್ರಮಿಸಿದ್ದರ ಪರಿಣಾಮವಾಗಿ ಸ್ವಾಮೀಜಿಯವರ ದೇಹಾರೋಗ್ಯ ಮತ್ತೆ ಕೆಟ್ಟಿತು. ಪರ್ವತ ಪ್ರದೇಶಗಳಲ್ಲಿ ಪಡೆದ ವಿಶ್ರಾಂತಿ ಯಿಂದಾಗಿ ಸುಧಾರಿಸಿದ್ದ ಅವರ ಆರೋಗ್ಯಸ್ಥಿತಿ ಮತ್ತೆ ಮೊದಲಿನಂತೆಯೇ ಆಯಿತು. ಹೀಗಾಗಿ ಬಯಲು ಸೀಮೆಯಲ್ಲಿ ಅವರು ಯಾವ ಭಾಷಣ ಪ್ರವಾಸದ ಕಾರ್ಯಕ್ರಮವನ್ನು ಹಾಕಿಕೊಂಡಿ ದ್ದರೋ ಅದನ್ನು ಅಲ್ಲಿಗೇ ನಿಲ್ಲಿಸಬೇಕಾಯಿತು. ಈಗ ಅವರು ಲಾಹೋರಿನಿಂದ ಮತ್ತೆ ಹಿಮಾ ಲಯದ ತಪ್ಪಲು ಪ್ರದೇಶವಾದ ಡೆಹರಾಡೂನಿಗೆ ಹೊರಟರು.

ಇಲ್ಲಿ ಕೆಲದಿನಗಳ ಕಾಲ ಶಾಂತವಾದ ಏಕಾಂತಜೀವನ ನಡೆಸಬೇಕೆಂಬುದು ಸ್ವಾಮೀಜಿಯವರ ಉದ್ದೇಶವಾಗಿತ್ತು. ಆದರೆ ಯಾವಾಗ ಅವರ ಆಗಮನದ ವಿಚಾರ ಅಲ್ಲಿನ ಜನರ ಕಿವಿಗೆ ಬಿದ್ದಿತೊ, ಜನ ಅವರನ್ನು ಮುತ್ತಲು ಪ್ರಾರಂಭವಾಯಿತು. ಆದ್ದರಿಂದ ಸಂಪೂರ್ಣ ವಿಶ್ರಾಂತಿ ತೆಗೆದು ಕೊಳ್ಳುವ ಇಚ್ಛೆ ಕೈಗೂಡಲಿಲ್ಲ.

ಡೆಹರಾಡೂನಿನಲ್ಲಿ ಸ್ವಾಮೀಜಿ ತಮ್ಮ ಪರಿವಾರದವರಿಗಾಗಿ ಶ್ರೀಭಾಷ್ಯದ ಮೇಲೆ ತರಗತಿ ಗಳನ್ನು ಪ್ರಾರಂಭಿಸಿದರು. ಇಲ್ಲಿ ಪ್ರಾರಂಭವಾದ ಈ ತರಗತಿಗಳು ಮುಂದೆ ಅವರು ಹೋದಲ್ಲೆಲ್ಲ ಮುಂದುವರಿದುವು. ಕೆಲದಿನಗಳ ಮೇಲೆ ಸ್ವಾಮೀಜಿ ಸಾಂಖ್ಯಶಾಸ್ತ್ರದ ಮೇಲೂ ತರಗತಿಯನ್ನು ಪ್ರಾರಂಭಿಸಿ, ತಮ್ಮ ಜೊತೆಯಲ್ಲಿದ್ದ ಆರ್ಯಸಮಾಜದ ಸ್ವಾಮಿ ಅಚ್ಯುತಾನಂದರಿಗೆ ಆ ತರ ಗತಿಯ ಜವಾಬ್ದಾರಿಯನ್ನು ಒಪ್ಪಿಸಿದರು. ಸ್ವಾಮೀ ಅಚ್ಯುತಾನಂದರು ಸಂಸ್ಕೃತದಲ್ಲಿ ಪ್ರಕಾಂಡ ಪಂಡಿತರಾದರೂ ಕೆಲಕೆಲವು ಶ್ಲೋಕಗಳನ್ನು ಸರಿಯಾದ ರೀತಿಯಲ್ಲಿ ವ್ಯಾಖ್ಯಾನಿಸಲು ಅವರಿಗೆ ಸಾಧ್ಯವಾಗುತ್ತಿರಲಿಲ್ಲ. ಅಂತಹ ಸಂದರ್ಭಗಳಲ್ಲಿ ಸ್ವಾಮೀಜಿ ಕೆಲವೇ ಮಾತುಗಳಲ್ಲಿ ಅವುಗಳ ಅರ್ಥವನ್ನು ಸ್ಪಷ್ಟಗೊಳಿಸುತ್ತಿದ್ದರು. ಡೆಹರಾಡೂನಿನಲ್ಲಿ ಸುಮಾರು ಹತ್ತು ದಿನಗಳನ್ನು ಕಳೆದು ಸ್ವಾಮೀಜಿ ತಮ್ಮ ಸಂಗಡಿಗರೊಂದಿಗೆ ಶಹರಾನ್​ಪುರದ ಮೂಲಕ ರಾಜಸ್ಥಾನದ ಕಡೆಗೆ ಪಯಣಿಸಿದರು. ಖೇತ್ರಿಯ ಕಡೆಯಿಂದ ಆಗಲೇ ಅವರಿಗೆ ಆಮಂತ್ರಣಗಳ ಮೇಲೆ ಆಮಂ ತ್ರಣಗಳು ಬರುತ್ತಿದ್ದುವು. ರಾಜಾ ಅಜಿತ್​ಸಿಂಗನಿಗಂತೂ ಯಾವಾಗ ಸ್ವಾಮೀಜಿಯವರನ್ನು ತನ್ನ ರಾಜ್ಯಕ್ಕೆ ಬರಮಾಡಿಕೊಂಡು ಉಪಚರಿಸಿಯೇನೆಂಬ ತವಕ. ಅಲ್ಲದೆ ತನ್ನ ಪ್ರಜೆಗಳೆಲ್ಲರಿಗೂ ಅವರ ಅಮೃತವಾಣಿಯನ್ನು ಪಾನ ಮಾಡಿಸಬೇಕೆಂಬ ಆತುರ. ಆದ್ದರಿಂದ ಅವರನ್ನು ಕರೆತರಲು ಆತ ತನ್ನ ದೂತರನ್ನು ಡೆಹರಾಡೂನಿಗೇ ಕಳಿಸಿಕೊಟ್ಟಿದ್ದ.

ಸ್ವಾಮೀಜಿ ತಮ್ಮ ಪ್ರಯಾಣವನ್ನು ಮುಂದುವರಿಸಿ ದೆಹಲಿಗೆ ಬಂದರು. ಇಲ್ಲಿ ಅವರು ನಟಕೃಷ್ಣ ಎಂಬ ತಮ್ಮ ಹಳೆಯ ಪರಿಚಯಸ್ಥನ ಮನೆಯಲ್ಲಿ ಇಳಿದುಕೊಂಡರು. ಹಿಂದೆ ಅವರು ಪರಿವ್ರಾಜಕರಾಗಿ ಸಂಚರಿಸುತ್ತಿದ್ದಾಗ ಈತನನ್ನು ಭೇಟಿಯಾಗಿದ್ದರು. ನಟಕೃಷ್ಣ ಒಬ್ಬ ಸಾಧಾರಣ ಗೃಹಸ್ಥ. ದೆಹಲಿಯ ಹಲವಾರು ಶ್ರೀಮಂತರು, ಗಣ್ಯರು ಸ್ವಾಮೀಜಿಯವರನ್ನು ತಮ್ಮ ಅತಿಥಿ ಯಾಗಿರುವಂತೆ ವಿನಂತಿಸಿಕೊಂಡರಾದರೂ ಅವರು ತಮ್ಮ ಈ ಹಳೆಯ ಸ್ನೇಹಿತನ ಮನೆಯಲ್ಲೇ ಇರಲು ಇಷ್ಟಪಟ್ಟರು.

ಡಿಸೆಂಬರ್ ೧ರಂದು ಸ್ವಾಮೀಜಿ ತಮ್ಮ ಪರಿವಾರದೊಂದಿಗೆ ಅಲ್ವರ್ ರಾಜ್ಯದ ಮೂಲಕ ಖೇತ್ರಿಯ ಕಡೆಗೆ ಹೊರಟರು. ಖೇತ್ರಿಯ ಮಹಾರಾಜನು ಸ್ವಾಮೀಜಿಯವರನ್ನು ಹತ್ತಿರದ ಬೇರೊಂದು ದಾರಿಯಲ್ಲಿ ಕರೆಸಿಕೊಳ್ಳಲು ಪಲ್ಲಕ್ಕಿಯನ್ನು ಸಿದ್ಧಪಡಿಸಿದ್ದ. ಆದರೆ ಅಲ್ವರಿಗೆ ಭೇಟಿ ನೀಡಿಯೇ ಮುಂದುವರಿಯಬೇಕೆಂದು ಸ್ವಾಮೀಜಿ ನಿರ್ಧರಿಸಿದ್ದುದರಿಂದ ನೇರವಾಗಿ ಖೇತ್ರಿಗೆ ಹೋಗಲು ಒಪ್ಪಲಿಲ್ಲ.

ಅಲ್ವರಿನಲ್ಲಿ ಸ್ವಾಮೀಜಿ ಹಾಗೂ ಅವರ ಸಂಗಡಿಗರಿಗೆ ವೈಭವದ ಸ್ವಾಗತ ದೊರಕಿತು. ಆ ಸಮಯದಲ್ಲಿ ಅಲ್ಲಿನ ಮಹಾರಾಜ ರಾಜಧಾನಿಯಲ್ಲಿರಲಿಲ್ಲ. ಸ್ವಾಮೀಜಿ ಹಿಂದೊಮ್ಮೆ ಅಲ್ವರಿಗೆ ಆಗಮಿಸಿದ್ದಾಗ ಅವರಿಗೆ ಇಲ್ಲಿ ಹಲವಾರು ಜನ ಸ್ನೇಹಿತರಾಗಿದ್ದರು, ಶಿಷ್ಯರಾಗಿದ್ದರು. ಅವರನ್ನೆಲ್ಲ ಮತ್ತೊಮ್ಮೆ ನೋಡುವ ಉದ್ದೇಶದಿಂದಲೇ ಸ್ವಾಮೀಜಿ ಇಲ್ಲಿಗೆ ಬಂದದ್ದು. ರೈಲುನಿಲ್ದಾಣದಲ್ಲಿ ಸ್ವಾಮೀಜಿಯವರಿಗೆ ಸಂಭ್ರಮದ ಸ್ವಾಗತ ಸತ್ಕಾರ ನಡೆಯುತ್ತಿತ್ತು; ನಗರದ ಗಣ್ಯ ವ್ಯಕ್ತಿಗಳೆಲ್ಲ ಅಲ್ಲಿ ನೆರೆದಿದ್ದರು. ಆ ಹೊತ್ತಿಗೆ ಸ್ವಲ್ಪ ದೂರದಲ್ಲಿ ನಿಂತಿದ್ದ ಒಬ್ಬ ಮನುಷ್ಯನು. ಸ್ವಾಮೀಜಿ ಯವರ ಕಣ್ಣಿಗೆ ಬಿದ್ದ. ಅವನೊಬ್ಬ ಬಡವ. ಸಾಮಾನ್ಯವಾದ ಉಡಿಗೆ, ತೊಡಿಗೆ. ಅವನನ್ನು ಕಂಡ ತಕ್ಷಣ ಸ್ವಾಮೀಜಿ ಅವನನ್ನು ಗುರುತಿಸಿ “ರಾಮ್​ಸ್ನೇಹಿ! ಓ ರಾಮ್ ಸ್ನೇಹಿ!” ಎಂದು ಕೂಗಿ ಕರೆದರು. ಈತ ಅವರ ಬಡಶಿಷ್ಯರಲ್ಲೊಬ್ಬ. ಹಿಂದೆ ಸ್ವಾಮೀಜಿ ಪರಿವ್ರಾಜಕರಾಗಿ ಬಂದಿದ್ದಾಗ ಅವರ ಸಂಪರ್ಕಕ್ಕೆ ಬಂದಿದ್ದವನು. ಅವರು ಅವನನ್ನು ನೋಡಿ ಮಾತನಾಡಿದುದು ನಾಲ್ಕೈದು ವರ್ಷಗಳ ಹಿಂದೆ! ಈ ಅವಧಿಯಲ್ಲಿ ಅವರು ಭಾರತದಲ್ಲಿ ಮಾತ್ರವಲ್ಲದೆ ಪಾಶ್ಚಾತ್ಯ ರಾಷ್ಟ್ರ ಗಳಲ್ಲೂ ಅಸಂಖ್ಯಾತ ಜನರನ್ನು ಭೇಟಿಯಾಗಿದ್ದಾರೆ. ಅಲ್ಲದೆ ಈಗ ಇಲ್ಲಿ ಗಣ್ಯವ್ಯಕ್ತಿಗಳ ಸಮೂಹ ದಲ್ಲಿ ನಿಂತು ಸತ್ಕಾರ ಸ್ವೀಕರಿಸುತ್ತಿದ್ದಾರೆ. ಇಂತಹ ಸಂದರ್ಭದಲ್ಲೂ ಅವರು ಯಾವ ಘನತೆ ಗೌರವಗಳನ್ನೂ ಲೆಕ್ಕಿಸದೆ ಆ ಬಡ ಶಿಷ್ಯನನ್ನು ಬಳಿಗೆ ಬರಮಾಡಿಕೊಂಡು ಅತ್ಯಂತ ವಿಶ್ವಾಸದಿಂದ, “ಏನಪ್ಪ, ಚೆನ್ನಾಗಿದ್ದೀಯಾ? ಮನೆಯವರು ಸ್ನೇಹಿತರು ಎಲ್ಲ ಹೇಗಿದ್ದಾರೆ?” ಎಂದು ವಿಚಾರಿಸಿಕೊಂಡರು ಸ್ವಾಮೀಜಿಯವರು. ಈ ಒಂದೆರಡು ಮಾತಿನಿಂದಲೇ ರಾಮ್​ಸ್ನೇಹಿಗಾದ ಆನಂದ ವರ್ಣನಾತೀತ.

ಅಲ್ವರಿನಲ್ಲಿ ಹಲವಾರು ಜನ ಸ್ವಾಮೀಜಿಯವರನ್ನು ತಮ್ಮ ಮನೆಗಳಿಗೆ ಆಹ್ವಾನಿಸಿ ಸತ್ಕರಿಸಿ ದರು. ಹಿಂದೆ ಅವರು ಇಲ್ಲಿಗೆ ಬಂದಿದ್ದಾಗ ಒಬ್ಬಳು ಬಡ ಮುದುಕಿಯ ಮನೆಯಲ್ಲಿ ಊಟ ಮಾಡಿದ್ದರು. ಈ ಸಲವೂ ಅವರನ್ನು ತನ್ನ ಮನೆಗೆ ಕರೆದು ಉಪಚರಿಸಬೇಕೆಂಬ ಆಸೆ ಆ ಮುದುಕಿಗೆ. ಆದರೆ ಈಗ ಅವರು ಹಿಂದಿನಂತೆ ಬಡ ಸಂನ್ಯಾಸಿಯಲ್ಲ; ಅವರ ಅಂತಸ್ತು ಬೆಳೆದು ಬಿಟ್ಟಿದೆ. ಜೊತೆಗೆ ನಗರದ ಭಾರೀ ವ್ಯಕ್ತಿಗಳೆಲ್ಲ ಅವರನ್ನು ಆಮಂತ್ರಿಸುತ್ತಿದ್ದಾರೆ. ಹೀಗಿರುವಾಗ ತಾನು ಅವರನ್ನು ಹೇಗೆ ಕರೆಯಬಲ್ಲೆ? ಒಂದು ವೇಳೆ ತಾನು ಕರೆದು ಅವರು ಒಪ್ಪಿ ಬಂದರೂ ತನ್ನ ಬಡ ಅಡಿಗೆ ಅವರಿಗೆ ಹಿಡಿಸೀತೆ? ಇದು ಆ ಮುದುಕಿಯ ಚಿಂತೆ. ಆದರೆ ಅವಳು ತಾನಾಗಿಯೇ ಅವರನ್ನು ಕರೆಯಬೇಕಾಗಿಬರಲಿಲ್ಲ. ಸ್ವತಃ ಸ್ವಾಮೀಜಿಯವರೇ ಅವಳಿಗೆ ಹೇಳಿಕಳಿಸಿದರು–ತಾವು ಅವಳ ಮನೆಗೆ ಬರುತ್ತಿದ್ದೇವೆ, ಹಿಂದಿನ ಸಲದಂತೆಯೇ ದಪ್ಪನೆಯ ಚಪಾತಿ ಮಾಡಿಹಾಕಬೇಕು ಎಂದು. ಇದನ್ನು ಕೇಳಿ ಮುದುಕಿಯ ಸಂಭ್ರಮಕ್ಕೆ ಪಾರವೇ ಇಲ್ಲ. ಸ್ವಾಮೀಜಿ ತಮ್ಮ ಸಂಗಡಿಗರೊಂದಿಗೆ ಬಂದರು. ಅವರಿಗೆಲ್ಲ ಚಪಾತಿಗಳನ್ನು ಬಡಿಸುತ್ತ ಆ ಹೆಂಗಸು, “ಮಗು, ನಿಮಗೆಲ್ಲ ಒಳ್ಳೆಯ ಅಡಿಗೆ ಮಾಡಿ ಬಡಿಸಬೇಕು ಅಂತ ನನ್ನ ಆಸೆ. ಆದರೆ ನಾನು ಬಡವಿ. ಅದೆಲ್ಲ ನನ್ನಿಂದೆಲ್ಲಾದೀತಪ್ಪ?” ಎಂದಳು. ಆದರೆ ಸ್ವಾಮೀಜಿ ತಮ್ಮ ಶಿಷ್ಯರ ಮುಂದೆ, “ನೋಡಿ, ಈಕೆಗೆ ಎಷ್ಟು ಭಕ್ತಿ, ಎಷ್ಟು ವಾತ್ಸಲ್ಯ! ಇವಳು ಮಾಡಿರುವ ಚಪಾತಿಗಳು ಎಷ್ಟು ಸಾತ್ವಿಕವಾಗಿವೆ ನೋಡಿ!” ಎಂದು ಮತ್ತೆ ಮತ್ತೆ ಹೇಳಿ ಕೊಂಡಾಡಿದರು. ಬಳಿಕ ಅಲ್ಲಿಂದ ಹೊರಡುವಾಗ ಅವಳಿಗೆ ಗೊತ್ತಾಗದಂತೆ ಮನೆಯ ಯಜಮಾನನ ಕೈಯಲ್ಲಿ ನೂರು ರೂಪಾಯಿ ನೋಟನ್ನು ತುರುಕಿ ಬಂದರು.

ಅಲ್ವರಿನಲ್ಲಿ ಕೆಲದಿನವಿದ್ದು, ಸ್ವಾಮೀಜಿ ಜೈಪುರಕ್ಕೆ ಬಂದರು. ಅಲ್ಲಿ ಅವರು ಖೇತ್ರಿ ರಾಜರಿಗೆ ಸೇರಿದ ಅತಿಥಿಗೃಹದಲ್ಲಿ ಇಳಿದುಕೊಂಡರು. ಹಿಂದೆ ಪರಿವ್ರಾಜಕರಾಗಿದ್ದಾಗಲೂ ಸ್ವಾಮೀಜಿ ಯವರಿಗೆ ಇಲ್ಲಿಯೇ ವಸತಿ ಕಲ್ಪಿಸಲಾಗಿತ್ತು. ಈ ಸಂದರ್ಭದಲ್ಲಿ ಅವರು ತಮ್ಮ ಶಿಷ್ಯರ ಮುಂದೆ ಹೇಳಿದರು, “ಹಿಂದೆ ನಾನು ಇಲ್ಲಿದ್ದಾಗ ಇಲ್ಲಿನ ಅಡಿಗೆಯವನು ನನಗೆ ದಿನಕ್ಕೆ ನಾಲ್ಕು ಚಪಾತಿ ಹಾಕುತ್ತಿದ್ದ–ಅದೂ ಒಲ್ಲದ ಮನಸ್ಸಿನಿಂದ, ಮುಖ ಸಿಂಡರಿಸಿಕೊಂಡು! ಆದರೆ ಈಗ ನಾನು ರಾಜನ ಮಂಚದ ಮೇಲೆಯೇ ಮಲಗುತ್ತಿದ್ದೇನೆ, ಹಲವಾರು ಜನ ಸೇವಕರು ನನ್ನ ಸೇವೆಗಾಗಿ ಸದಾ ಸಿದ್ಧರಾಗಿದ್ದಾರೆ. ಎಷ್ಟು ವ್ಯತ್ಯಾಸ, ನೋಡಿ! ಆದ್ದರಿಂದಲೇ ಭರ್ತೃಹರಿ ಹೇಳುತ್ತಾನೆ, ‘ಓ ರಾಜ, ಜನರು ಗೌರವ-ಮನ್ನಣೆ ಕೊಡುವುದು ಈ ಶರೀರಕ್ಕೂ ಅಲ್ಲ, ಒಳಗಿನ ಆತ್ಮಕ್ಕೂ ಅಲ್ಲ. ಅವರು ಮನ್ನಣೆ ಕೊಡುವುದು ಮನುಷ್ಯನ ಅಂತಸ್ತಿಗೆ ಮಾತ್ರ!’ ಅಂತ. ಈ ಮಾತು ಅಕ್ಷರಶಃ ನಿಜ.”

ಸ್ವಾಮೀಜಿ ತಮ್ಮ ಪರಿವಾರಸಮೇತರಾಗಿ ಜೈಪುರದಿಂದ ಖೇತ್ರಿಯ ಕಡೆಗೆ ಹೊರಟರು. ಖೇತ್ರಿಯಿಂದ ಹನ್ನೆರಡು ಮೈಲಿ ದೂರದಲ್ಲಿರುವ ಬಬೈ ಎಂಬಲ್ಲಿಗೆ ಮಹಾರಾಜ ಅಜಿತ್​ಸಿಂಗ್ ತನ್ನ ದಿವಾನನಾದ ಜಗಮೋಹನಲಾಲನೊಂದಿಗೆ ಬಂದು ಕಾದಿದ್ದ. ಅಲ್ಲಿಂದ ಎಲ್ಲರೂ ಒಟ್ಟಾಗಿ ಮೆರವಣಿಗೆಯಲ್ಲಿ ಹೊರಟು ಖೇತ್ರಿ ನಗರವನ್ನು ತಲುಪಿದರು. ನಗರದ ಗಡಿಯಲ್ಲಿ ನಾಗರಿಕರು ಸ್ವಾಮೀಜಿಯವರಿಗೆ ಆರತಿ ಬೆಳಗಿ ಸ್ವಾಗತಿಸಿದರು. ಅಲ್ಲಿಂದ ಮೆರವಣಿಗೆ ಮುಂದುವರಿದು ಅರಮನೆಯ ದೇವಾಲಯಕ್ಕೆ ತಲುಪಿತು. ಅಲ್ಲಿನ ಮುಖ್ಯ ಅರ್ಚಕರು ಸ್ವಾಮೀಜಿಯವರಿಗೆ ಆರತಿ ಬೆಳಗಿ ಕಾಣಿಕೆ ಸಮರ್ಪಿಸಿದರು. ಅನಂತರ ಇತರ ನಾಗರಿಕರೂ ತಮ್ಮತಮ್ಮ ಕಾಣಿಕೆಗಳನ್ನು ಅರ್ಪಿಸಿದರು. ದೇವಾಲಯದಿಂದ ಎಲ್ಲರೂ ಸ್ವಾಗತ ಸಮಾರಂಭ ನಡೆಯಲಿದ್ದ ಸ್ಥಳಕ್ಕೆ ಬಂದರು. ದಾರಿಯ ಉದ್ದಕ್ಕೂ ಸ್ವಾಮೀಜಿಯವರ ಗೌರವಾರ್ಥವಾಗಿ ಕೆಂಪು ಜಮಖಾನೆಯನ್ನು ಹಾಸ ಲಾಗಿತ್ತು. ಸ್ವಾಗತ ಸಮಾರಂಭದ ಸಂದರ್ಭದಲ್ಲಿ ಸಂಪ್ರದಾಯದ ಪ್ರಕಾರ ರಾಜನಿಗೆ ಐದು ಹರಿವಾಣಗಳ ತುಂಬ ಚಿನ್ನದ ಮೊಹರುಗಳನ್ನು ಅರ್ಪಿಸಲಾಯಿತು. ಅದರಲ್ಲಿ ಹೆಚ್ಚಿನಂಶವನ್ನು ರಾಜ ತನ್ನ ರಾಜ್ಯದ ವಿದ್ಯಾಸಂಸ್ಥೆಗಳಿಗೆ ಅಲ್ಲೇ ದಾನವಾಗಿ ಕೊಟ್ಟುಬಿಟ್ಟ. ಬಳಿಕ ಮುನ್ಷಿ ಜಗಮೋಹನ ಲಾಲ್ ರಾಜ್ಯದ ಪರವಾಗಿ ಸ್ವಾಮೀಜಿಯವರಿಗೆ ಬಿನ್ನವತ್ತಳೆಯನ್ನು ಸಮರ್ಪಿಸಿದ. ಉತ್ತರರೂಪವಾಗಿ ಸ್ವಾಮೀಜಿ ಒಂದು ಪುಟ್ಟ ಭಾಷಣ ಮಾಡಿದರು. ಖೇತ್ರಿಯ ರಾಜನಿಗೆ ತಮ್ಮ ಧನ್ಯವಾದಗಳನ್ನು ಸಮರ್ಪಿಸುತ್ತ, “ಭಾರತದ ಪುರೋಭಿವೃದ್ಧಿಗಾಗಿ ನನ್ನಿಂದ ಯಾವ ಕಿಂಚಿತ್ ಕಾರ್ಯ ಸಾಧ್ಯವಾಯಿತೋ ಅದಕ್ಕೆ ಖೇತ್ರಿಯ ಮಹಾರಾಜರೇ ಕಾರಣ” ಎಂದು ನುಡಿದರು.

ರಾತ್ರಿ ಎಂಟು ಗಂಟೆಯ ಹೊತ್ತಿಗೆ ಸ್ವಾಮೀಜಿಯವರನ್ನು ಅರಮನೆಯ ಸರೋವರದ ಬಳಿಗೆ ಕರೆದೊಯ್ಯಲಾಯಿತು. ಅಲ್ಲಿ ಅಂದು ಸುಡುಮದ್ದಿನ ಪ್ರದರ್ಶನ ಏರ್ಪಾಡಾಗಿತ್ತು. ಇಡೀ ಸರೋವರ ಹಾಗೂ ಅರಮನೆ ಅಸಂಖ್ಯಾತ ಹಣತೆಗಳಿಂದ ಕಂಗೊಳಿಸುತ್ತಿದ್ದುವು.

ಡಿಸೆಂಬರ್ ೨ಂರಂದು ಸ್ವಾಮೀಜಿ ಅರಮನೆಯ ಸಭಾಂಗಣದಲ್ಲಿ “ವೇದಾಂತ ಧರ್ಮ” ಎಂಬ ವಿಷಯವಾಗಿ ಮಾತನಾಡಿದರು. ನಗರದ ಎಲ್ಲ ಗಣ್ಯರೂ ಅನೇಕ ಐರೋಪ್ಯರೂ ಈ ಸಭೆಯಲ್ಲಿ ಉಪಸ್ಥಿತರಿದ್ದರು. ಸುಮಾರು ಒಂದೂವರೆ ಗಂಟೆಗಳ ಕಾಲ ಮಾತನಾಡಿದ ಸ್ವಾಮೀಜಿ, ‘ಭಾರತೀಯ ವಿಚಾರಧಾರೆಯು ಐರೋಪ್ಯ ನಾಗರಿಕತೆಯ ಮೇಲೆ ಹೇಗೆ ಪ್ರಭಾವ ಬೀರಿತು; ಸುಪ್ರಸಿದ್ಧ ತತ್ತ್ವಶಾಸ್ತ್ರಜ್ಞರಾದ ಪೈಥಾಗೊರಸ್, ಸಾಕ್ರೆಟಿಸ್, ಪ್ಲೇಟೋ ಮೊದಲಾದವರ ಮೇಲೆ ಹೇಗೆ ತನ್ನ ಪ್ರಭಾವ ಬೀರಿತು ಮತ್ತು ಸ್ಪೈನ್​-ಜರ್ಮನಿಯೇ ಮೊದಲಾದ ದೇಶಗಳ ಮೇಲೆಯೂ ಇತಿಹಾಸದ ವಿವಿಧ ಸಮಯಗಳಲ್ಲಿ ಹೇಗೆ ತನ್ನ ವರ್ಚಸ್ಸು ಬೀರಿತು’ ಎಂಬುದನ್ನು ವಿವರಿಸಿದರು. ತರುವಾಯ ವೇದಗಳ ಹಾಗೂ ಪುರಾಣಗಳ ತತ್ತ್ವಗಳನ್ನು ವಿವರಿಸಿ ಅವುಗಳಲ್ಲಿ ಕಂಡುಬರುವ ಆರಾಧನೆಯ ವಿವಿಧ ಭಾವನೆಗಳನ್ನು ಹಾಗೂ ವಿವಿಧ ಸ್ತರಗಳನ್ನು ತಿಳಿಯಪಡಿಸಿದರು. ಈ ಎಲ್ಲ ವಿವಿಧ ಭಾವನೆಗಳ ಹಿನ್ನೆಲೆಯಲ್ಲಿ ‘ಏಕಂ ಸತ್; ವಿಪ್ರಾ ಬಹುಧಾ ವದಂತಿ’ ಎಂಬ ತತ್ತ್ವವಡ ಗಿರುವುದನ್ನು ತೋರಿಸಿಕೊಟ್ಟರು. ‘ಗ್ರೀಕರು ಕೇವಲ ಬಾಹ್ಯ ಪ್ರಕೃತಿಯ ಕಡೆಗೇ ಧಾವಿಸುತ್ತಿದ್ದರೆ ಆರ್ಯರು ಬಾಹ್ಯಪ್ರಕೃತಿಯ ಬೆಡಗಿನಿಂದ ತೃಪ್ತರಾಗದೆ ಅಂತರಾತ್ಮದ ಕಡೆಗೆ ತಿರುಗಿ ಆತ್ಮ ಸಾಕ್ಷಾತ್ಕಾರವನ್ನು ಪಡೆದುಕೊಂಡರು, ಜನನ ಮರಣಗಳ ಪ್ರಶ್ನೆಯನ್ನು ಬಗೆಹರಿಸಿಕೊಂಡರು’ ಎಂಬ ಅಂಶವನ್ನು ಸ್ವಾಮೀಜಿ ಎತ್ತಿ ತೋರಿದರು.

ಬಳಿಕ ಅವರು ಹೇಳಿದರು, “ಈಗಿನ ಆಧುನಿಕ ಕಾಲದಲ್ಲಿ ಹಿಂದೂಗಳೂ ಇಲ್ಲ, ವೇದಾಂತಿ ಗಳೂ ಇಲ್ಲ. ಈಗಿರುವವರೆಲ್ಲ ಕೇವಲ ‘ಮುಟ್ಟಬೇಡಪ್ಪ’ಗಳು! ಅಡಿಗೆ ಮನೆಯೇ ಅವರ ದೇವಾಲಯ; ಅಡಿಗೆಪಾತ್ರೆಗಳೇ ಅವರ ಪೂಜಾಸಾಮಗ್ರಿಗಳು. ಈ ಪರಿಸ್ಥಿತಿ ದೂರವಾಗಬೇಕು. ಈ ಭಾವನೆಗಳು ಬೇಗ ದೂರವಾದಷ್ಟೂ ಧರ್ಮಕ್ಕೆ ಕ್ಷೇಮ. ಉಪನಿಷತ್ತುಗಳು ವೈಭವಪೂರ್ಣ ವಾಗಿ ಬೆಳಗುವಂತಾಗಲಿ; ಜೊತೆಗೆ ದ್ವೈತ, ವಿಶಿಷ್ಟಾದ್ವೈತ ಹಾಗೂ ಅದ್ವೈತಾವಲಂಬಿಗಳಲ್ಲಿ ಜಗಳವಿಲ್ಲದಿರಲಿ.”

ಹೀಗೆ ಸುಮಾರು ಒಂದು ಗಂಟೆ ಭಾಷಣ ಮಾಡುವಷ್ಟರಲ್ಲೇ ಸ್ವಾಮೀಜಿಯವರಿಗೆ ತುಂಬ ಬಳಲಿಕೆಯುಂಟಾಗಿ, ಮಾತನ್ನು ಅರ್ಧಕ್ಕೆ ನಿಲ್ಲಿಸಬೇಕಾಯಿತು. ಅವರ ದೇಹಸ್ಥಿತಿ ಅಷ್ಟೊಂದು ಹದಗೆಟ್ಟಿತ್ತು. ಸ್ವಾಮೀಜಿ ತಮ್ಮ ಮಾತನ್ನು ಮತ್ತೆ ಮುಂದುವರಿಸುತ್ತಾರೆಂದು ನಿರೀಕ್ಷಿಸುತ್ತ ಸಭಿಕರು ಮೌನವಾಗಿ ಕಾದರು. ಸ್ವಲ್ಪ ಹೊತ್ತು ಸುಧಾರಿಸಿಕೊಂಡು ಸ್ವಾಮೀಜಿ ಮತ್ತೆ ಅರ್ಧ ಗಂಟೆ ಮಾತನಾಡಿದರು. “ವಿವಿಧತೆಯಲ್ಲಿ ಏಕತೆಯನ್ನು ಕಾಣುವುದೇ ಜ್ಞಾನ. ಯಾವುದೇ ಸಿದ್ಧಾಂತದಲ್ಲಾಗಲಿ, ವಿವಿಧತೆಯ ಹಿನ್ನೆಲೆಯಲ್ಲಿರುವ ಏಕತೆಯನ್ನು ಕಂಡುಕೊಂಡಾಗಲೇ ಜ್ಞಾನದ ಪರಮಾವಧಿಯನ್ನು ಮುಟ್ಟಿದಂತಾಗುವುದು–ಈ ಮಾತು ವಿಜ್ಞಾನದ ವಿಷಯದಲ್ಲಿಯೂ ಸರಿಯೆ, ಅಧ್ಯಾತ್ಮದ ವಿಷಯದಲ್ಲಿಯೂ ಸರಿಯೆ” ಎಂದು ಅವರು ವಿವರಿಸಿದರು. ತಮ್ಮ ಮಾತು ಗಳನ್ನು ಮುಕ್ತಾಯಗೊಳಿಸುತ್ತ, “ಒಬ್ಬ ನಿಜವಾದ ಕ್ಷತ್ರಿಯರಾಜನಂತೆ ಖೇತ್ರಿಯ ಮಹಾರಾಜರು ನಮ್ಮ ಸನಾತನ ಹಿಂದೂಧರ್ಮದ ಅಮೂಲ್ಯ ತತ್ತ್ವಗಳನ್ನು ಪಾಶ್ಚಾತ್ಯ ರಾಷ್ಟ್ರಗಳಲ್ಲಿ ಪ್ರಸಾರ ಮಾಡುವಲ್ಲಿ ಆರ್ಥಿಕವಾಗಿ ಬಹಳಷ್ಟು ನೆರವಾಗಿದ್ದಾರೆ. ಸಮಸ್ತ ಹಿಂದೂಗಳ ಕೃತಜ್ಞತೆ-ಅಭಿ ನಂದನೆಗಳು ಅವರಿಗೆ ಸಲ್ಲುತ್ತವೆ” ಎಂದು ಅಜಿತ್ ಸಿಂಗನನ್ನು ಹೃತ್ಪೂರ್ವಕವಾಗಿ ಪ್ರಶಂಸಿಸಿ ದರು. ಅವರ ಅಂದಿನ ಭಾಷಣ ಖೇತ್ರಿಯ ಜನಮನದ ಮೇಲೆ ಅಚ್ಚಳಿಯದ ಪ್ರಭಾವ ಬೀರಿತು.

ಖೇತ್ರಿಯಲ್ಲಿ ಸ್ವಾಮೀಜಿ ತಮ್ಮ ಪ್ರಚಾರಕಾರ್ಯಗಳಲ್ಲಿ ಮಗ್ನರಾಗಿದ್ದರೂ ರಾಜನ ವಿಶ್ವಾಸ ಪೂರ್ಣ ಆತಿಥ್ಯದಿಂದಾಗಿ ಅವರಿಗೆ ಉಲ್ಲಾಸ ವಿಶ್ರಾಂತಿಗಳೂ ದೊರಕಿದ್ದುವು. ಅವರು ಅನೇಕ ಬಾರಿ ರಾಜನೊಂದಿಗೆ ಹಾಗೂ ತಮ್ಮ ಇತರ ಶಿಷ್ಯರೊಂದಿಗೆ ಗಾಡಿಗಳಲ್ಲಿ ಇಲ್ಲವೆ ಕುದುರೆಗಳ ನ್ನೇರಿ ಸುತ್ತಮುತ್ತಲ ಸ್ಥಳಗಳಿಗೆ ಹೋಗಿ ಬರುತ್ತಿದ್ದರು. ಒಂದು ದಿನ ಅವರು ಹೀಗೆಯೇ ವನವಿಹಾರಕ್ಕೆಂದು ಹೋಗಿದ್ದರು. ದಾರಿಯಲ್ಲೊಂದು ಇಕ್ಕಟ್ಟಿನ ಸ್ಥಳದಲ್ಲಿ ಒಂದು ಮರದ ಕೊಂಬೆ ಬಹಳ ಕೆಳಗೆ ಬಾಗಿಕೊಂಡಿತ್ತು. ಅದರಲ್ಲಿ ಮುಳ್ಳಿನ ಬಳ್ಳಿ ಬೇರೆ ಹರಡಿಕೊಂಡಿತ್ತು. ಈಗ ಅದನ್ನು ದಾಟಿ ಹೋಗಬೇಕಾದರೆ ಆ ಕೊಂಬೆಯನ್ನು ಪಕ್ಕಕ್ಕೆ ಸರಿಸಬೇಕು. ರಾಜನು ಬದಿಗೆ ನಿಂತು ಕೊಂಬೆಯನ್ನು ಎಳೆದು ಹಿಡಿದುಕೊಂಡ. ಸ್ವಾಮೀಜಿ ಸುರಕ್ಷಿತವಾಗಿ ದಾಟಿದರು; ರಾಜನೂ ಹಿಂಬಾಲಿಸಿದ. ಆದರೆ ಸ್ವಲ್ಪ ಹೊತ್ತಿನ ಮೇಲೆ ಸ್ವಾಮೀಜಿ ನೋಡುತ್ತಾರೆ–ರಾಜನ ಕೈಯಿಂದ ರಕ್ತ ಧಾರಾಕಾರವಾಗಿ ಸೋರುತ್ತಿದೆ! ಅದು ಮುಳ್ಳುಗಳು ತರಚಿದ್ದರಿಂದ ಆದ ಗಾಯ. ಇದನ್ನು ಕಂಡು ಸ್ವಾಮೀಜಿ “ಛೆ! ಹಾಗೇಕೆ ಮಾಡಿಕೊಂಡೆ?”ಎಂದು ಗದರಿಸಿದಾಗ ರಾಜ ನಕ್ಕು, “ಸ್ವಾಮೀಜಿ, ಧರ್ಮವನ್ನು ರಕ್ಷಣೆ ಮಾಡಬೇಕಾದ್ದು ಕ್ಷತ್ರಿಯನ ಕರ್ತವ್ಯವಲ್ಲವೆ?” ಎಂದ. ಸ್ವಾಮೀಜಿ ಸ್ವಲ್ಪ ಹೊತ್ತು ಮೌನವಾಗಿದ್ದು ಬಳಿಕ ಹೇಳಿದರು, “ಬಹುಶಃ ನೀನಂದದ್ದೇ ಸರಿ.”

ಸ್ವಾಮೀಜಿ ಖೇತ್ರಿಯಿಂದ ಹೊರಡುವ ಮುನ್ನ ಅವರಿಗೆ ಮಹಾರಾಜ ಮೂರು ಸಾವಿರ ರೂಪಾಯಿಗಳ ಕಾಣಿಕೆಯನ್ನು ಅರ್ಪಿಸಿದ. ಇದನ್ನು ಸ್ವಾಮೀಜಿ ತಮ್ಮ ಇಬ್ಬರು ಶಿಷ್ಯರ ಮೂಲಕ ಆಲಂಬಜಾರ್ ಮಠಕ್ಕೆ ಕಳಿಸಿಕೊಟ್ಟರು. ಡಿಸೆಂಬರ್ ೨೧ರಂದು ಸ್ವಾಮೀಜಿ ಇನ್ನಿತರರೊಂದಿಗೆ ಖೇತ್ರಿಯಿಂದ ಹೊರಟು ಜೈಪುರಕ್ಕೆ ಬಂದರು. ಅವರನ್ನು ಬೀಳ್ಕೊಡಲು ಸ್ವತಃ ಅಜಿತ್​ಸಿಂಗನೆ ಜೈಪುರದವರೆಗೂ ಬಂದಿದ್ದ. ಇಲ್ಲೊಂದು ಭಾರೀ ಸಮಾರಂಭವೇ ನಡೆಯಿತು. ಬಳಿಕ ಸ್ವಾಮೀಜಿ ತಮ್ಮ ಪರಿವಾರದವರ ಪೈಕಿ ಬ್ರಹ್ಮಚಾರಿ ಕೃಷ್ಣಲಾಲರೊಬ್ಬರನ್ನು ತಮ್ಮ ಸಹಾಯಕ್ಕೆಂದು ಇರಿಸಿಕೊಂಡು ಉಳಿದವರನ್ನೆಲ್ಲ ವಾಪಸು ಕಲ್ಕತ್ತಕ್ಕೆ ಕಳಿಸಿಕೊಟ್ಟರು. ಜೈಪುರದಲ್ಲಿ ಒಂದು ವಾರದ ಕಾಲ ಇದ್ದು ಬಳಿಕ ತಾವೂ ಅಲ್ಲಿಂದ ಹೊರಟರು. ಭಾರವಾದ ಹೃದಯದಿಂದ ಅಜಿತ್​ಸಿಂಗ್ ಹಾಗೂ ಮುನ್ಷಿ ಜಗಮೋಹನಲಾಲ್ ಸ್ವಾಮೀಜಿಯವರಿಗೆ ವಿದಾಯ ಹೇಳಿದರು.

ಈಗ ಸ್ವಾಮೀಜಿ ಶೀಘ್ರವಾಗಿ ಕಿಶನ್​ಘರ್, ಅಜ್ಮೀರ್, ಜೋಧ್​ಪುರ್, ಇಂದೋರ್ ಮಾರ್ಗವಾಗಿ ಖಾಂಡ್ವಾ ಕಡೆಗೆ ಪ್ರಯಾಣಿಸಿದರು. ಈ ಎಲ್ಲ ಸ್ಥಳಗಳಲ್ಲೂ ಎಂದಿನಂತೆ ವೈಭವದ ಸ್ವಾಗತ, ಗಣ್ಯವ್ಯಕ್ತಿಗಳ ಪರಿಚಯ. ಆದರೆ ಸ್ವಾಮೀಜಿಯವರ ದೇಹಾರೋಗ್ಯ ಮಾತ್ರ ಸುಸ್ಥಿತಿ ಯಲ್ಲಿರಲಿಲ್ಲ. ಖಾಂಡ್ವಾದಲ್ಲಿ ಅವರು ತಮ್ಮ ಹಳೆಯ ಪರಿಚಯಸ್ಥನಾದ ಹರಿದಾಸ ಚಟರ್ಜಿಯ ಮನೆಯಲ್ಲಿ ಇಳಿದುಕೊಂಡರು. ಇಲ್ಲಿಗೆ ಬಂದಾಗ ಅವರು ತೀವ್ರ ಜ್ವರದಿಂದ ನರಳುತ್ತಿದ್ದರು. ಆದರೆ ಚಟರ್ಜಿಯ ಮುತುವರ್ಜಿಯಿಂದ ಬೇಗ ಗುಣಹೊಂದಿ ಚೇತರಿಸಿಕೊಂಡರು.

ಖಾಂಡ್ವಾದಲ್ಲಿ ಸ್ವಾಮೀಜಿ ಸುಮಾರು ಒಂದು ವಾರ ಇದ್ದರು. ಅವರು ಅಲ್ಲಿಂದ ಹೊರಡುವ ರಾತ್ರಿ ಹರಿದಾಸ ಚಟರ್ಜಿಯು ಅವರನ್ನು ತನಗೆ ಮಂತ್ರೋಪದೇಶ ನೀಡುವಂತೆ ಕೇಳಿಕೊಂಡ. ಸ್ವಾಮೀಜಿ ಒಪ್ಪದಿದ್ದಾಗ ಅವನು ಅವರ ಪಾದಗಳನ್ನು ಬಿಗಿಯಾಗಿ ಹಿಡಿದುಕೊಂಡು ತನಗೆ ಮಂತ್ರೋಪದೇಶವನ್ನು ನೀಡಲೇಬೇಕೆಂದು ಅಂಗಲಾಚಿ ಕೇಳಿಕೊಂಡ. ಆದರೂ ಸ್ವಾಮೀಜಿ ಒಪ್ಪದೆ ಹೇಳಿದರು, “ನೋಡು, ನಾನು ಶಿಷ್ಯರನ್ನು ಮಾಡಿಕೊಂಡು ಗುರುಗಿರಿಯ ಪತಾಕೆಯೇರಿಸು ವವನಲ್ಲ. ಇತರರಂತೆಯೇ ನೀನೂ ಸಾಧನೆ ಮಾಡುತ್ತ ಬಾ. ಪ್ರತಿಯೊಬ್ಬನ ಅಂತರಂಗದಲ್ಲೂ ಅದೇ ಸರ್ವಶಕ್ತವಾದ ದಿವ್ಯತೆ ಅಡಗಿದೆ. ಇದನ್ನು ಸಾಕ್ಷಾತ್ಕರಿಸಿಕೊಳ್ಳಲು ಪ್ರಯತ್ನಿಸು.”

ಸ್ವಾಮೀಜಿಯವರು ಹರಿದಾಸನಿಗೆ ಮಂತ್ರೋಪದೇಶ ನೀಡಲು ಒಪ್ಪದಿದ್ದುದು ಅಚ್ಚರಿಯ ಸಂಗತಿಯೇ ಸರಿ. ಏಕೆಂದರೆ ಆತನೊಬ್ಬ ಸದ್ಗೃಹಸ್ಥ; ಅಲ್ಲದೆ ಸ್ವಾಮೀಜಿಯವರ ಸೇವೆ ಮಾಡಿದ ವನು; ಜೊತೆಗೆ ಹಿಂದೆಯೂ ಅವರ ಆತಿಥೇಯನಾಗಿದ್ದವನು. ಮಂತ್ರೋಪದೇಶವನ್ನು ಪಡೆದು ಕೊಳ್ಳಲು ಅವನು ಅತ್ಯಂತ ವ್ಯಾಕುಲನೂ ಆಗಿದ್ದ. ಅಲ್ಲದೆ, ಸ್ವಾಮೀಜಿ ಯಾರಿಗೂ ಮಂತ್ರದೀಕ್ಷೆ ನೀಡುತ್ತಿರಲಿಲ್ಲವೆಂದೇನೂ ಅಲ್ಲ; ಇದಕ್ಕೆ ಹಿಂದೆಯೂ ಅನಂತರವೂ ಅವರು ಹಲವಾರು ಜನರಿಗೆ ಮಂತ್ರ ದೀಕ್ಷೆ ನೀಡಿ ಅನುಗ್ರಹಿಸಿದ್ದುಂಟು. ಸ್ವಭಾವತಃ ಸ್ವಾಮೀಜಿ ಅತ್ಯಂತ ದಯಾ ಪೂರ್ಣ ಹೃದಯವುಳ್ಳವರು. ಹೀಗಿದ್ದೂ ಅವರು ಹರಿದಾಸನಿಗೇಕೆ ಮಂತ್ರದೀಕ್ಷೆಯನ್ನು ಕರುಣಿಸ ಲಿಲ್ಲ ಎಂದರೆ ಒಂದೇ ಒಂದು ಕಾರಣವನ್ನು ಊಹಿಸಬಹುದು. ಅದೇನೆಂದರೆ, ಅವನು ಸ್ವಾಮೀಜಿ ಯವರ ‘ವಲಯ’ಕ್ಕೆ ಸೇರಿದವನಲ್ಲ. ದಿವ್ಯ ಗುರುಗಳಾದವರು ತಮ್ಮ ವಲಯಕ್ಕೆ ಸೇರದವರನ್ನು ಮಂತ್ರೋಪದೇಶದ ಮೂಲಕ ಶಿಷ್ಯರನ್ನಾಗಿ ಸ್ವೀಕರಿಸುವುದಿಲ್ಲ. ಅವರು ತಮ್ಮ ಅಂತರ್ದೃಷ್ಟಿ ಯಿಂದ ಈ ವಿಚಾರವನ್ನು ಕಂಡುಕೊಳ್ಳಬಲ್ಲರು. ಉದಾಹರಣೆಗೆ ಸ್ವಾಮೀಜಿಯವರೇ ಇನ್ನೊಬ್ಬ ನಿಗೆ ಹೇಳಿದ್ದುಂಟು, ‘ನಿನ್ನ ಗುರು ನಾನಲ್ಲ, ನೀನು ಇಂಥವರಿಂದ ಮಂತ್ರೋಪದೇಶವನ್ನು ಪಡೆದುಕೊ’ ಎಂದು. ಹೀಗೆ ಗುರು-ಶಿಷ್ಯ ಸಂಬಂಧವೆನ್ನುವುದು ತುಂಬ ಸೂಕ್ಷ್ಮವಾದದ್ದು.

ಉತ್ತರ ಭಾರತದಲ್ಲಿ ಇನ್ನೂ ಹಲವಾರು ಸ್ಥಳಗಳಿಗೆ ಭೇಟಿ ನೀಡುವ ಉದ್ದೇಶ ಸ್ವಾಮೀಜಿ ಯವರಿಗಿತ್ತು. ಆ ವೇಳೆಗಾಗಲೇ ಲಿಂಬ್ಡಿಯ ರಾಜಾ ಠಾಕೂರ್ ಸಾಹೇಬನಿಂದ ಹಾಗೂ ಛತರ್​ಪುರದ ಮಹಾರಾಜನಿಂದ ಅವರಿಗೆ ಆಮಂತ್ರಣಗಳು ಬಂದಿದ್ದುವು. ಪಶ್ಚಿಮ ಭಾರತದ ಸಿಂಧ್, ಗುಜರಾತ್, ಕಾಥೇವಾಡ, ಬರೋಡ ಹಾಗೂ ಇನ್ನಿತರ ಸ್ಥಳಗಳಿಂದಲೂ ಅವರನ್ನು ಆಹ್ವಾನಿಸಿ ತಂತಿಗಳ ಹಾಗೂ ಪತ್ರಗಳ ಸುರಿಮಳೆಯಾಗುತ್ತಿತ್ತು. ಈ ಸ್ಥಳಗಳಲ್ಲೆಲ್ಲ ತಮ್ಮ ಸಂದೇಶಗಳನ್ನು ಪ್ರಸಾರ ಮಾಡಿ ಕಾರ್ಯ ಯೋಜನೆಯ ಬೀಜವನ್ನು ಬಿತ್ತಬೇಕೆಂದು ಸ್ವಾಮೀಜಿ ಯವರೂ ಇಚ್ಛಿಸಿದ್ದರು. ಏಕೆಂದರೆ ಅವರು ಕಲ್ಕತ್ತದಿಂದ ಇಷ್ಟು ದೂರ ಬಂದಿರುವಾಗ ಎಷ್ಟು ಕಾರ್ಯವನ್ನು ಮಾಡಿ ಮುಗಿಸಲು ಸಾಧ್ಯವಾಗಿದ್ದರೆ ಅಷ್ಟು ಒಳ್ಳೆಯದಿತ್ತಲ್ಲವೆ? ಇವುಗಳಲ್ಲದೆ ಕರಾಚಿ ಹಾಗೂ ಕಾಥೇವಾಡದಲ್ಲಿದ್ದ ತಮ್ಮ ಶಿಷ್ಯರನ್ನು ಭೇಟಿಯಾಗುವ ಮನಸ್ಸೂ ಅವರಿಗಿತ್ತು. ಆದ್ದರಿಂದ ತಮ್ಮ ಆರೋಗ್ಯ ಅಷ್ಟೇನೂ ಚೆನ್ನಾಗಿಲ್ಲದಿದ್ದರೂ ಈ ಸ್ಥಳಗಳಲ್ಲಿ ಕೆಲವನ್ನಾದರೂ ಸಂದರ್ಶಿಸುವ ಉದ್ದೇಶದಿಂದ ಖಾಂಡ್ವಾದಿಂದ ಹೊರಟು ರಟ್ಲಾಮ್ ಜಂಕ್ಷನ್ನಿನವರೆಗೂ ಬಂದರು. ಆದರೆ ದುರದೃಷ್ಟವಾತ್, ಅಷ್ಟು ಹೊತ್ತಿಗೆ ಅವರ ಆರೋಗ್ಯ ಸಂಪೂರ್ಣ ಹದಗೆಟ್ಟಿತು. ಇನ್ನು ಪ್ರಯಾಣ ಮುಂದುವರಿಸುವುದು ದುಸ್ಸಾಧ್ಯವಾಗಿ ತೋರಿತು. ಮತ್ತಿನ್ನೇನು ಮಾಡಲಾ ದೀತು? ಕೂಡಲೇ ಕಲ್ಕತ್ತಕ್ಕೆ ಹೊರಡುವುದೇ ಮೇಲೆಂದು ಅವರು ಆಲೋಚಿಸಿದರು. ಹೋಗಲಿ, ಕಲ್ಕತ್ತದಲ್ಲಿ ಸ್ವಲ್ಪ ವಿಶ್ರಾಂತಿಯಾದರೂ ಸಿಕ್ಕೀತು! ಅಲ್ಲದೆ ಕಲ್ಕತ್ತದಲ್ಲಾದರೂ ತಾವು ಮಾಡ ಬೇಕಾಗಿರುವ ಕಾರ್ಯವೇನು ಕಡಿಮೆಯೆ? ಆದ್ದರಿಂದ ಅವರು ಕಲ್ಕತ್ತಕ್ಕೆ ಹೋಗುವ ಟ್ರೈನು ಹತ್ತಿದರು. ದಾರಿಯಲ್ಲಿ ಸಿಗುವ ಜಬ್ಬಲ್​ಪುರದ ರೈಲು ನಿಲ್ದಾಣದಲ್ಲಿ ಅಲ್ಲಿನ ಜನ ಅವರನ್ನು ಅತ್ಯಂತ ಉತ್ಸಾಹದಿಂದ ಎದುರ್ಗೊಂಡು ಸನ್ಮಾನಿಸಿ ಬೀಳ್ಕೊಟ್ಟರು.

ಅಲ್ಲಿಂದ ಹೊರಟು ಸ್ವಾಮೀಜಿ ಸೀದಾ ಕಲ್ಕತ್ತಕ್ಕೆ ಬಂದು ತಲುಪಿದರು. ಇಲ್ಲಿಗೆ ಅವರು ಭಾರತದಲ್ಲಿ ಕೈಗೊಂಡ ಪ್ರಸಾರ-ಪ್ರವಾಸಕಾರ್ಯ ಹೆಚ್ಚುಕಡಿಮೆ ಮುಗಿದಂತೆಯೇ ಆಯಿತು. ಸಮಸ್ತ ಭಾರತದಲ್ಲಿ ಎಲ್ಲೆಲ್ಲೂ ಜನರು ಅವರನ್ನು ಬರಮಾಡಿಕೊಳ್ಳಲು, ಅವರ ಮಾತುಗಳನ್ನು ಆಲಿಸಲು ಅತ್ಯಂತ ಕುತೂಹಲಿಗಳಾಗಿದ್ದರು. ಆದರೆ, ದುರದೃಷ್ಟವಶಾತ್, ಕುಸಿಯುತ್ತಿದ್ದ ಆರೋಗ್ಯಸ್ಥಿತಿಯಿಂದಾಗಿ ಹಾಗೂ ಇನ್ನಿತರ ಕಾರ್ಯಗಳಿಂದಾಗಿ ಅವರಿಗೆ ಮತ್ತೆಂದೂ ಭಾರತದ ಪ್ರವಾಸವನ್ನು ಕೈಗೊಳ್ಳಲು ಕಾಲ ಕೂಡಿಬರಲೇ ಇಲ್ಲ. ಆದರೆ ಮತ್ತೊಂದು ವಿಷಯವೇನೆಂದರೆ, ಆಗಲೇ ಅವರು ಹಲವಾರು ಸ್ಥಳಗಳಲ್ಲಿ ಮಾಡಿದ್ದ ಭಾಷಣಗಳಲ್ಲಿ ಹಿಂದೂಧರ್ಮದ ಪುನರುತ್ಥಾನದ ಬಗೆಗಿನ ತಮ್ಮ ಆಲೋಚನೆಗಳನ್ನು ಹೊರಗೆಡವಿದ್ದರು. ಸಮಸ್ತ ಭಾರತವನ್ನು ಒಗ್ಗೂಡಿಸುವ ಅಂಶಗಳಾವುವು ಎಂಬುದನ್ನು ಎತ್ತಿ ತೋರಿಸಿ, ಹಿಂದೆಂದಿಗಿಂತಲೂ ಹೆಚ್ಚು ವೈಭವಯುತವಾದ ಭಾವೀಭಾರತವನ್ನು ಹೇಗೆ ನಿರ್ಮಾಣ ಮಾಡಬೇಕು ಎಂಬುದನ್ನು ಸ್ಪಷ್ಟವಾಗಿ ಸಾರಿದ್ದರು. ಭಾರತೀಯರಿಗೆ ತಮ್ಮ ಪೂರ್ವಜರಿಂದ ಪ್ರಾಪ್ತವಾದ ಧರ್ಮ-ಸಂಸ್ಕೃತಿಯ ಮೌಲ್ಯ ವೇನು, ಮಹತ್ವವೇನು ಎಂಬುದರ ಸ್ಪಷ್ಟಪರಿಚಯ ಮಾಡಿಕೊಟ್ಟರು. ಭಾರತದ ರಾಷ್ಟ್ರೀಯತೆ ಯನ್ನು ಅದರ ಸನಾತನ ಧರ್ಮ-ಸಂಸ್ಕೃತಿಯ ಆಧಾರದ ಮೇಲೆಯೇ ರೂಪಿಸಬೇಕು; ಆದರೆ ಹೊಸ ಹೊಸ ಭಾವನೆಗಳನ್ನು ಸ್ವೀಕರಿಸುತ್ತ, ಮೈಗೂಡಿಸಿಕೊಳ್ಳುತ್ತ ಅದು ಬೆಳೆಯಬೇಕು. ಭಾರತೀಯ ಜೀವನವಾಹಿನಿಯು ಹಿಂದಿನಿಂದಲೂ ಹರಿದುಬಂದದ್ದು ಧರ್ಮದ ಕಾಲುವೆಯಲ್ಲೇ, ಮತ್ತು ಜೀವನದ ಪ್ರತಿಯೊಂದು ಕ್ಷೇತ್ರದ ಆವಶ್ಯಕತೆಗಳನ್ನೂ ಪೂರೈಸಿಕೊಟ್ಟದ್ದು ಈ ಧರ್ಮದ ಕಾಲುವೆಯೇ ಎಂಬುದನ್ನು ಸ್ವಾಮೀಜಿ ತೋರಿಸಿಕೊಟ್ಟರು. ಅಲ್ಲದೆ, ಲೌಕಿಕ ಜೀವನದಲ್ಲಿ ಜನರು ಅಶಾಂತಿಯ ಜ್ವಾಲೆಗೆ ಗುರಿಯಾದಾಗ ಶಾಂತಿಜಲಸೇಚನ ಮಾಡಿದ್ದು ಈ ಧರ್ಮವೇ. ಭಾರತದಲ್ಲಿ ಲೌಕಿಕ ಜೀವನವು ಗಂಡಾಂತರವನ್ನೆದುರಿಸಿದಾಗಲೆಲ್ಲ ಅದರ ರಕ್ಷಣೆಗೆ ಒದಗಿಬಂದದ್ದು ಈ ಧರ್ಮವೇ. ಆದ್ದರಿಂದ ಸಮಸ್ತ ಭಾರತವನ್ನು ಧಾರ್ಮಿಕ-ಆಧ್ಯಾತ್ಮಿಕ ಆದರ್ಶಗಳ ಆಧಾರದ ಮೇಲೆ ಸಂಘಟಿಸುವಂತೆ ಮಾಡುವುದೇ ಪ್ರಧಾನ ಕಾರ್ಯ ಎಂದು ಸ್ವಾಮೀಜಿ ಸಾರಿದರು. ಆದರೆ ಧರ್ಮವೆಂದರೆ ಶಾಸ್ತ್ರಪ್ರಣೀತವಾದ ಸನಾತನ ತತ್ತ್ವಗಳೇ ಹೊರತು, ನಮ್ಮ ದೈನಂದಿನ ದೇಶಾ ಚಾರ-ಕಂದಾಚಾರಗಳಲ್ಲ; ಈ ಕಂದಾಚಾರಗಳು ಪೈರಿನೊಂದಿಗೇ ಬೆಳೆಯುವ ಕಳೆಗಳಂತೆ– ಅವುಗಳನ್ನು ಕಿತ್ತೊಗೆಯಬೇಕು ಎಂದು ಅವರು ಎಚ್ಚರಿಕೆ ನೀಡಿದರು. ಎಲ್ಲಕ್ಕಿಂತ ಹೆಚ್ಚಾಗಿ, ಒಂದು ರಾಷ್ಟ್ರದ ಸ್ಥಿತಿಗತಿ-ಸ್ಥಾನಮಾನಗಳು ಅವಲಂಬಿಸಿಕೊಂಡಿರುವುದು ಆ ರಾಷ್ಟ್ರದ ವ್ಯಕ್ತಿವ್ಯಕ್ತಿ ಗಳ ಶೀಲ-ಸ್ವಭಾವಗಳನ್ನು ಎನ್ನುವುದು ಸ್ವಾಮೀಜಿಯವರ ಪಲ್ಲವಿ. ವ್ಯಕ್ತಿಗಳ ಶಕ್ತಿಯೇ ರಾಷ್ಟ್ರದ ಶಕ್ತಿ. ಆದ್ದರಿಂದ ಶ್ರೇಷ್ಠ ರಾಷ್ಟ್ರವು ನಿರ್ಮಾಣವಾಗಬೇಕೆಂದು ಬಯಸುವ ಪ್ರತಿಯೊಬ್ಬನೂ– ಅವನು ಅಧಿಕಾರಿಯಾಗಿರಲಿ, ಕಾರಕೂನನಾಗಿರಲಿ, ವೈದ್ಯನಾಗಿರಲಿ, ಶಿಕ್ಷಕನಾಗಿರಲಿ, ಜಾಡಮಾಲಿ ಯಾಗಿರಲಿ–ಮೊದಲು ತನ್ನ ಶೀಲವನ್ನು ರೂಪಿಸಿಕೊಳ್ಳಬೇಕು; ಧೈರ್ಯ, ಶಕ್ತಿ, ಸ್ವಾಭಿಮಾನ, ಅನುಕಂಪೆ, ಸೇವಾಮನೋಭಾವ–ಇವುಗಳನ್ನು ಬೆಳೆಸಿಕೊಳ್ಳಬೇಕು ಎಂದು ಅವರು ಸಾರಿದರು. ಅದರಲ್ಲೂ ಮುಖ್ಯವಾಗಿ ಯುವ ಪೀಳಿಗೆಯು ತ್ಯಾಗ ಮತ್ತು ಸೇವೆಗಳನ್ನು ತನ್ನ ಅತ್ಯುಚ್ಚ ಆದರ್ಶವನ್ನಾಗಿ ಮಾಡಿಕೊಳ್ಳಬೇಕೆಂದು ಸ್ವಾಮೀಜಿ ಕಳಕಳಿಯಿಂದ ಮನವಿ ಮಾಡಿಕೊಂಡರು.

ಈಗ ಅವರು ಕಲ್ಕತ್ತದಲ್ಲಿ, ತಮ್ಮ ಧ್ಯೇಯೋದ್ದೇಶಗಳನ್ನು ಹಾಗೂ ಸಂದೇಶಗಳನ್ನು ಮೈಗೂಡಿಸಿಕೊಂಡು, ಮುಂದೆ ಅವುಗಳನ್ನು ಕಾರ್ಯರೂಪಕ್ಕೆ ತರಬಲ್ಲ ಹಾಗೂ ಪ್ರಚಾರ ಮಾಡಿ ಹರಡಬಲ್ಲ ಶೀಲವಂತ ಯುವಕಶಿಷ್ಯರ ನಿರ್ಮಾಣಕಾರ್ಯದಲ್ಲಿ ತೊಡಗಿದರು.

