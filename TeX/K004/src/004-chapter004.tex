
\chapter{ವಂಗಬಂಧುಗಳ ಸಂಗದಲ್ಲಿ}

\noindent

ಸ್ವಾಮೀ ವಿವೇಕಾನಂದರು ಭಾರತದ ನೆಲದಲ್ಲಿ ಕಾಲಿರಿಸಿದರೆಂಬ ವಾರ್ತೆ ಕೇಳಿದಾಗಿನಿಂದ ಅವರನ್ನು ಬರಮಾಡಿಕೊಳ್ಳಲು ಬಂಗಾಳಕ್ಕೆ ಬಂಗಾಳವೇ ತುದಿಗಾಲಿನಲ್ಲಿ ನಿಂತು ಕಾಯುತ್ತಿತ್ತು. ಅದರಲ್ಲೂ ಕಲ್ಕತ್ತವಂತೂ ಅವರು ಕೊಲಂಬೊದಿಂದ ಮದ್ರಾಸಿನವರೆಗೆ ಅಡಿಯಿಟ್ಟಲ್ಲೆಲ್ಲ ಅವರನ್ನು ಮಾನಸಿಕವಾಗಿ ಅನುಸರಿಸುತ್ತ ಬಂದಿತ್ತು. ಅವರು ನುಡಿದ ನುಡಿಗಳನ್ನು ಅತೀವ ಆಸಕ್ತಿ ಯಿಂದ ಕಿವಿದೆರೆದು ಆಲಿಸುತ್ತಿತ್ತು. ಕೋಮುವಾರು ಶಕ್ತಿಗಳಿಂದ, ಅಸೂಯಾಪರ ವ್ಯಕ್ತಿಗಳಿಂದ ಹಾಗೂ ಅತಿ ಮಡಿವಂತರಿಂದ ಪ್ರೇರಿತವಾದ ಕೆಲವು ಪತ್ರಿಕೆಗಳ ಒಡಕು ದನಿಯನ್ನು ಬಿಟ್ಟರೆ, ಸ್ವಾಮೀಜಿಯವರಿಗಾಗಿ ಸಮಸ್ತ ಬಂಗಾಳದ ಹೃದಯ ಹೆಮ್ಮೆಯಿಂದ ತುಡಿಯುತ್ತಿತ್ತು, ಅಭಿ ಮಾನದಿಂದ ಬೀಗುತ್ತಿತ್ತು. ಎಷ್ಟಾದರೂ ಸ್ವಾಮೀಜಿ ವಂಗಮಾತೆಯ ಹೆಮ್ಮೆಯ ಸುತನಲ್ಲವೆ? ತಮ್ಮಲ್ಲಿಯೇ ಹುಟ್ಟಿ ತಮ್ಮ ನಡುವೆಯೇ ಬೆಳೆದು ತಮ್ಮ ವಂಗನಾಡಿನ ಹೆಸರನ್ನು ಸಮಸ್ತ ಜಗ ತ್ತಿನ ಉನ್ನತ ಶೃಂಗಕ್ಕೇರಿಸಿದವನನ್ನು ಮರಳಿ ತಮ್ಮಲ್ಲಿಗೆ ಬರಮಾಡಿಕೊಳ್ಳುವೆವೆಂಬ ಭಾವನೆಯೇ ಅವರಲ್ಲಿ ಅನಿರ್ವಚನೀಯ ಆನಂದವನ್ನುಂಟು ಮಾಡುತ್ತಿತ್ತು.

ಸ್ವಾಮೀಜಿ ಕಲ್ಕತ್ತಕ್ಕೆ ಹಿಂದಿರುಗಿದಾಗ ಅವರಿಗೆ ವೈಭವದ ಸ್ವಾಗತ ನೀಡಿ ಸನ್ಮಾನಿಸಲು ಅಲ್ಲಿನ ಪ್ರಮುಖ ನಾಗರಿಕರು ಸ್ವಾಗತ ಸಮಿತಿಯೊಂದನ್ನು ರಚಿಸಿಕೊಂಡರು. ರಾಜಾ ವಿನಯಕೃಷ್ಣದೇವ್ ಈ ಸಮಿತಿಯ ಅಧ್ಯಕ್ಷರಾದರು. “ಇಂಡಿಯನ್ ಮಿರರ್​” ಪತ್ರಿಕೆಯ ಸಂಪಾದಕರಾದ ಬಾಬು ನರೇಂದ್ರನಾಥ ಸೇನ್ ಸಮಿತಿಯ ಗೌರವ ಕಾರ್ಯದರ್ಶಿಗಳಾದರು. ಅತ್ಯುನ್ನತ ಸ್ಥಾನಮಾನ ಗಳನ್ನು ಹೊಂದಿದ್ದ ಇತರ ಐವರು ಸಮಿತಿಯ ಉಪಾಧ್ಯಕ್ಷರುಗಳಾದರು.

ಇತ್ತ ಸ್ವಾಮೀಜಿ ಕೂಡ ತಾವು ಜನ್ಮವೆತ್ತಿದ ನಗರಕ್ಕೆ ಹಿಂದಿರುಗಿ ಬಂದು ತಮ್ಮ ಚಿರಪರಿಚಿತ ರನ್ನು ಭೇಟಿಯಾಗಲು ಉತ್ಕಂಠಿತರಾಗಿದ್ದರು. ಮದ್ರಾಸಿನಿಂದ ಕಲ್ಕತ್ತದವರೆಗೆ ಸಮುದ್ರಮಾರ್ಗ ವಾಗಿ ಪಯಣಿಸಲು ಅವರು ನಿರ್ಧರಿಸಿದ್ದರು. ಆಗಲೇ ನಿರಂತರ ಸಮಾರಂಭಗಳು–ಭಾಷಣ ಗಳು–ಸಂಭಾಷಣೆಗಳು ಇವುಗಳಿಂದ ಅವರ ಶರೀರ ಜರ್ಜರಗೊಂಡಿತ್ತು. ಈ ಬಗೆಯ ಸಮಾರಂಭಗಳಿಂದ ಪಾರಾಗುವುದಕ್ಕಾಗಿಯೇ ಅವರು ರೈಲು ಪ್ರಯಾಣದ ಬದಲಿಗೆ ಹಡಗು ಪ್ರಯಾಣವನ್ನು ಕೈಗೊಂಡದ್ದು. ಇದರಿಂದಾಗಿ ಒಂದು ವಾರದ ಕಾಲ ಅವರಿಗೆ ಅತ್ಯಗತ್ಯವಾಗಿದ್ದ ವಿಶ್ರಾಂತಿ ದೊರಕುವಂತಾಯಿತು. ಅವರ ದಣಿದ ನರಮಂಡಲಕ್ಕೆ ಇದೊಂದು ವರವೇ ಆಯಿತು.

ಈ ಪ್ರಯಾಣದ ಸಂದರ್ಭದಲ್ಲಿ ಒಂದು ಸ್ವಾರಸ್ಯದ ಘಟನೆ ನಡೆಯಿತು. ಪ್ರಯಾಣಿಕರ ಪೈಕಿ ಕೆಲವು ಅಮೆರಿಕನ್ ಪಾದ್ರಿಗಳೂ ಇದ್ದರು. ಸ್ವಾಮೀಜಿಯವರ ಜೊತೆಯಲ್ಲಿ ಸಾಗುತ್ತಿದ್ದ ಜೆ. ಜೆ. ಗುಡ್​ವಿನ್ ಹಾಗೂ ಸೇವಿಯರ್ ದಂಪತಿಗಳನ್ನು ಕಂಡು ಈ ಪಾದ್ರಿಗಳಿಗೆ ಆಶ್ಚರ್ಯವೋ ಆಶ್ಚರ್ಯ. ಜೊತೆಯಲ್ಲಿ ಸಂಕಟ ಕೂಡ. ತಮ್ಮವರಾದ ಈ ಕ್ರೈಸ್ತರು ಎಲ್ಲಾ ಬಿಟ್ಟು ಈ ಹಿಂದೂ ಸಂನ್ಯಾಸಿಯೊಂದಿಗೆ ಹೊರಟುಬಿಟ್ಟಿದ್ದಾರಲ್ಲ ಎಂದು ಮರುಗಿದರು. “ಹೋಗಲಿ, ಇನ್ನು ಮುಂದಾ ದರೂ ಬುದ್ಧಿ ಕಲಿತು ಜಗತ್ತಿನ ಶ್ರೇಷ್ಠತಮ ಮತವಾದ ಕ್ರೈಸ್ತ ಧರ್ಮಕ್ಕೆ ಹಿಂದಿರುಗಿ ಬನ್ನಿ” ಎಂದು ಅವರಿಗೆ ಬುದ್ಧಿವಾದ ಹೇಳಿದರು. ಅಲ್ಲದೆ ಅವರ ಪರವಾಗಿ ಭಗವಂತನಲ್ಲಿ ಪ್ರಾರ್ಥನೆ ಯನ್ನೂ ಸಲ್ಲಿಸಿದರು. ಆದರೆ ಆ ಪ್ರಾರ್ಥನೆಯಿಂದೇನೂ ಪ್ರಯೋಜನವಾದಂತೆ ಕಾಣಲಿಲ್ಲ.

ಫೆಬ್ರುವರಿ ೧೮ರಂದು ಸ್ವಾಮೀಜಿಯವರ ಹಡಗು ಬಜ್​ಬಜ್ ನಿಲ್ದಾಣವನ್ನು ತಲುಪಿತು. ಮರುದಿನ ಬೆಳಗ್ಗೆ ಸ್ವಾಮೀಜಿ ತಮ್ಮ ಸಂಗಡಿಗರೊಂದಿಗೆ ಒಂದು ವಿಶೇಷ ಟ್ರೈನಿನಲ್ಲಿ ಸಿಯಾ ಲ್ದಾಕ್ಕೆ ಹೊರಟರು. ಅವರ ದರ್ಶನ ಪಡೆಯಲು ಸಿಯಾಲ್ದಾ ರೈಲು ನಿಲ್ದಾಣದಲ್ಲಿ ಸುಮಾರು ಇಪ್ಪತ್ತು ಸಾವಿರ ಜನ ತುಂಬಿದ್ದರು. ನಗರದ ಪ್ರಮುಖರಿಂದ ಹಿಡಿದು ಸಾಮಾನ್ಯ ಪ್ರಜೆಗಳವರೆಗೆ ಎಲ್ಲ ವರ್ಗಗಳ ಜನರೂ ಅಲ್ಲಿದ್ದರು. ಜನರ ಸಂಭ್ರಮ ಹೇಳತೀರದು. ರೈಲು ನಿಲ್ದಾಣದಲ್ಲಿ ಮಾತ್ರವಲ್ಲದೆ ರಸ್ತೆಯ ಇಕ್ಕೆಲಗಳಲ್ಲಿ, ಮನೆಗಳ ಮಹಡಿಗಳ ಮೇಲೆ–ಹೀಗೆ ಎಲ್ಲೆಲ್ಲೂ ಜನ ಕಿಕ್ಕಿರಿದಿದ್ದರು. ರಸ್ತೆಯುದ್ದಕ್ಕೂ ತಳಿರುತೋರಣಗಳು. ಅಲ್ಲಲ್ಲಿ ಬೃಹತ್ ಕಮಾನುಗಳು. ಅವುಗಳ ಮೇಲೆ ‘ಸ್ವಾಮೀಜಿಯವರಿಗೆ ಜಯವಾಗಲಿ!’ ‘ಜೈ ಶ್ರೀರಾಮಕೃಷ್ಣ!’ ‘ಸುಸ್ವಾಗತ!’ ಎಂದೆಲ್ಲ ಬರೆಯಲಾಗಿತ್ತು. ಇವಲ್ಲದೆ ಅಲ್ಲಲ್ಲಿ ನಿರ್ಮಿಸಲಾಗಿದ್ದ ವಾದ್ಯಕುಟೀರಗಳಿಂದ ಮಂಗಳಧ್ವನಿ ಹೊರಹೊಮ್ಮುತ್ತಿತ್ತು. ಮೆರವಣಿಗೆ ಸಿಯಾಲ್ದಾ ನಿಲ್ದಾಣದಿಂದ ಹ್ಯಾರಿಸನ್ ರಸ್ತೆಯ ಮೂಲಕ ರಿಪ್ಪನ್ ಕಾಲೇಜಿಗೆ ಬರುವ ವ್ಯವಸ್ಥೆಯಾಗಿತ್ತು. ಸ್ವಾಮೀಜಿ ರೈಲಿನಿಂದಿಳಿಯುತ್ತಿದ್ದಂತೆ ಬಾಬು ನರೇಂದ್ರನಾಥ ಸೇನರು ಅವರನ್ನೂ ಅವರ ಸಂಗಡಿಗರನ್ನೂ ಸ್ವಾಗತಿಸಿದರು. ಬಳಿಕ ಅವರನ್ನೆಲ್ಲ ಸುಸಜ್ಜಿತವಾದ ಸಾರೋಟಿನಲ್ಲಿ ಕುಳ್ಳಿರಿಸಿ ಹಾರ ಸಮರ್ಪಣೆ ಮಾಡಿದರು.

ಸ್ವಾಮೀಜಿ ಹಾಗೂ ಅವರ ಸಂಗಡಿಗರನ್ನು ಹೊತ್ತ ಸಾರೋಟು ನಿಧಾನವಾಗಿ ರಿಪ್ಪನ್ ಕಾಲೇಜಿ ನತ್ತ ಮುಂಬರಿಯುತ್ತಿದ್ದಂತೆ ಅಲ್ಲಿ ನೆರೆದಿದ್ದ ಸಹಸ್ರ ಸಹಸ್ರ ಜನ ಏಕಕಂಠದಿಂದ ಜೈಕಾರ ಘೋಷಣೆ ಮಾಡಿದರು. ನೂರಾರು ಸಾರೋಟುಗಳು ಸ್ವಾಮೀಜಿಯವರ ಸಾರೋಟನ್ನು ಹಿಂಬಾಲಿ ಸಿದುವು. ದಾರಿಯುದ್ದಕ್ಕೂ ಉತ್ಸಾಹಭರಿತರಾದ ಜನರ ಹರ್ಷೋದ್ಗಾರ ನಿರಂತರವಾಗಿ ಕೇಳಿಬರು ತ್ತಲೇ ಇತ್ತು. ಜೊತೆಗೆ ಸಂಗೀತ, ವಾದ್ಯವೃಂದ, ಸಂಕೀರ್ತನೆಗಳೂ ನಡೆಯುತ್ತಿದ್ದುವು. ಈ ಮಧ್ಯೆ ಕೆಲವು ಯುವಕರು ತಮ್ಮ ಆವೇಶವನ್ನು ಹತ್ತಿಕ್ಕಲಾರದೆ ಕುದುರೆಗಳನ್ನು ಬಿಡಿಸಿ ಸಾರೋಟನ್ನು ತಾವೇ ಎಳೆದರು. ಸ್ವಾಮೀಜಿಯವರ ಮುಖ ಶಾಂತ-ಗಂಭೀರ ಭಾವವನ್ನು ಸೂಸುತ್ತಿತ್ತು. ಆದರೆ ಅದರ ನಡುವೆಯೇ ಒಂದು ದಿವ್ಯ ಮಂದಹಾಸವನ್ನು ಅಲ್ಲಿ ಗುರುತಿಸಬಹುದಾಗಿತ್ತು.

ರಿಪ್ಪನ್ ಕಾಲೇಜಿನ ಬಳಿ ಸ್ವಾಮೀಜಿ ಸಾರೋಟಿನಿಂದಿಳಿಯುತ್ತಿದ್ದಂತೆ ಅಲ್ಲಿ ಸೇರಿದ್ದ ಮತ್ತೊಂದು ಭಾರೀ ಜನಸಮೂಹ ಅವರಿಗೆ ಸ್ವಾಗತ ಕೋರಿತು. ಬಳಿಕ ಅವರನ್ನು ವೇದಿಕೆಯ ಮೇಲಕ್ಕೆ ಕರೆತರಲಾಯಿತು. ಸಾಂಪ್ರದಾಯಿಕ ಸ್ವಾಗತವಾದ ಮೇಲೆ ಕಿವಿಗಡಚಿಕ್ಕುವ ಹರ್ಷೋ ದ್ಗಾರಗಳ ನಡುವೆ ಸ್ವಾಮೀಜಿ ಎದ್ದುನಿಂತರು. ಈಗ ಅವರೊಂದು ಭವ್ಯವಾದ ಭಾಷಣ ಮಾಡುತ್ತಾರೆಂದು ಜನರೆಲ್ಲ ನಿರೀಕ್ಷಿಸುತ್ತಿದ್ದರು. ಆದರೆ ಸ್ವಾಮೀಜಿ ತುಂಬ ಆಯಾಸಗೊಂಡಿದ್ದ ರಿಂದ ಹೆಚ್ಚು ಮಾತನಾಡುವ ಸ್ಥಿತಿಯಲ್ಲಿರಲಿಲ್ಲ. ಆದರೂ ಜನರು ತಮ್ಮ ಮೇಲೆ ತೋರಿದ ಪ್ರೀತ್ಯಾದರಗಳನ್ನು ಕಂಡು ಅವರ ಹೃದಯ ತುಂಬಿ ಬಂದಿತು. ಜೊತೆಗೆ ಅಂದಿನ ಸಮಾರಂಭಕ್ಕೆ ಮೊದಲು ವಿಪರೀತ ನೂಕುನುಗ್ಗಲು ಉಂಟಾಗಿ ಗಲಭೆಯಾಗಿತ್ತು. ಇವುಗಳೊಂದಿಗೆ ಪ್ರಯಾಣದ ಆಯಾಸವೂ ಸೇರಿಕೊಂಡಿತ್ತು. ಆದ್ದರಿಂದ ಸ್ವಾಮೀಜಿ ಕೆಲವೇ ಮಾತುಗಳಲ್ಲಿ ಜನರ ವಿಶ್ವಾಸಕ್ಕಾಗಿ ಕೃತಜ್ಞತೆಗಳನ್ನು ಸಲ್ಲಿಸಿ ತಮ್ಮ ಭಾಷಣವನ್ನು ಮುಕ್ತಾಯಗೊಳಿಸಿದರು.

ಆದರೆ ಸಭೆ ಇಷ್ಟು ಬೇಗ ಮುಗಿದುಹೋದದ್ದನ್ನು ಕಂಡು ಜನಗಳಿಗೆ ಸ್ವಲ್ಪ ನಿರಾಶೆಯಾಯಿತು. ಈ ಎಲ್ಲ ಗದ್ದಲ-ನೂಕುನುಗ್ಗಲಿನಲ್ಲಿ ಸ್ವಾಮೀಜಿಯವರನ್ನು ಸರಿಯಾಗಿ ಸ್ವಾಗತಿಸಿದಂತಾಗಲಿಲ್ಲ ಎಂದು ಸ್ವಾಗತ ಸಮಿತಿಯವರಿಗೂ ಅನ್ನಿಸಿತು. ಆದ್ದರಿಂದ ಇನ್ನೊಂದು ವಾರದ ಬಳಿಕ ಸಾರ್ವ ಜನಿಕವಾಗಿ, ಕ್ರಮಾಗತವಾಗಿ ಸ್ವಾಗತದ ಕಾರ್ಯಕ್ರಮವನ್ನು ಏರ್ಪಡಿಸಬೇಕೆಂದು ನಿಶ್ಚಯಿಸಿದರು.

ಸ್ವಾಮೀಜಿ ಹಾಗೂ ಅವರ ಸಂಗಡಿಗರೆಲ್ಲ ತಮ್ಮ ಮನೆಗೆ ಭೇಟಿಯಿತ್ತು, ತಮ್ಮ ಆಹ್ವಾನವನ್ನು ಸ್ವೀಕರಿಸಬೇಕೆಂದು ರಾಯ್ ಪಶುಪತಿನಾಥ್ ಎಂಬ ಗಣ್ಯರು ವಿಶೇಷ ಕೋರಿಕೆ ಸಲ್ಲಿಸಿದ್ದರು. ಅದರಂತೆ ಸ್ವಾಮೀಜಿ ಅಂದು ಮಧ್ಯಾಹ್ನ ಅವರ ಆತಿಥ್ಯ ಸ್ವೀಕರಿಸಿದರು. ಬಳಿಕ ಅವರು ತಮ್ಮ ಸಂಗಡಿಗರೊಂದಿಗೆ ಕಾಶೀಪುರದ ಗೋಪಾಲ್ ಲಾಲ್ ಸೀಲ್ ಎಂಬುವರ ಬಂಗಲೆಗೆ ತೆರಳಿದರು. ನಗರದ ಪ್ರಮುಖ ವರ್ತಕರಾದ ಇವರು, ಗಂಗಾ ತೀರದಲ್ಲಿದ್ದ ತಮ್ಮ ಭಾರೀ ಬಂಗಲೆಯನ್ನು ಸ್ವಾಮೀಜಿ ಹಾಗೂ ಅವರ ಶಿಷ್ಯರ ಉಪಯೋಗಕ್ಕಾಗಿ ಬಿಟ್ಟುಕೊಟ್ಟಿದ್ದರು. ಮಾರ್ಚ್ ೮ರಂದು ಸ್ವಾಮೀಜಿಯವರು ಕಲ್ಕತ್ತದಿಂದ ಹೊರಡುವವರೆಗೂ ಅವರ ಪಾಶ್ಚಾತ್ಯ ಶಿಷ್ಯರು ಹಾಗೂ ಇನ್ನಿತ ರರು ಇಲ್ಲಿ ಇಳಿದುಕೊಂಡರು.

ಇತ್ತ ಬಡ ಆಲಂಬಜಾರ್ ಮಠದಲ್ಲಿ ಸಂನ್ಯಾಸೀ ಸೋದರರು ತಮ್ಮ ಪ್ರಿಯ ಗುರುಭಾಯಿ ಯನ್ನು ಬರಮಾಡಿಕೊಳ್ಳಲು ತಮ್ಮದೇ ಆದ ಸರಳ ಸಿದ್ಧತೆಗಳನ್ನು ಮಾಡಿದ್ದರು. ಮಠದ ಮುಂಬಾಗಿಲಿಗೆ ಮಾವಿನ ತೋರಣ-ಬಾಳೆಕಂಬಗಳನ್ನು ಕಟ್ಟಿ, ಒಂದು ಕಲಶವನ್ನಿರಿಸಿದರು. ಆದರೆ ಸ್ವಾಮೀಜಿಯವರನ್ನು ಸ್ವಾಗತಿಸಲು ಮಠದಲ್ಲಿ ಉಳಿದುಕೊಂಡವರು ರಾಮಕೃಷ್ಣಾನಂದರು, ಅಖಂಡಾನಂದರು ಇಬ್ಬರೇ. ಉಳಿದವರೆಲ್ಲ ಅವರನ್ನು ನೋಡಲು ರೈಲು ನಿಲ್ದಾಣಕ್ಕೆ ಓಡಿದ್ದರು. ತಮ್ಮ ನೆಚ್ಚಿನ ನರೇಂದ್ರ ವಿಶ್ವವಿಜೇತನಾಗಿ ಭಾರತಕ್ಕೆ ಹಿಂದಿರುಗುತ್ತಿರುವಾಗ ಅವರೆಲ್ಲ ಮಠ ದಲ್ಲೇ ಕುಳಿತಿರಲು ಸಾಧ್ಯವೆ? ಸಂಜೆಯ ಹೊತ್ತಿಗೆ ಸ್ವಾಮೀಜಿ ದೊಡ್ಡ ಪರಿವಾರದ ಸಮೇತ ಮಠಕ್ಕೆ ಬಂದು ತಲುಪಿದರು. ಈಗ ಆ ಮಠದಲ್ಲಿ ಅದೇನು ಸಂಭ್ರಮ! ಅದೇನು ಭಾವೋತ್ಕರ್ಷ! ಆ ಗುರುಭಾಯಿಗಳಿಗೆ ಎಷ್ಟು ಮಾತನಾಡಿದರೂ ತೃಪ್ತಿಯಿಲ್ಲ. ಎಷ್ಟೋ ವರ್ಷಗಳಿಂದ ಬಾಕಿ ಯಿದ್ದ ಮಾತೆಲ್ಲ ಅಷ್ಟು ಬೇಗ ಮುಗಿದುಹೋದೀತೇ?

ಸ್ವಾಮೀಜಿ ಕಲ್ಕತ್ತವನ್ನು ತಲುಪಿದ ಮೇಲೆ ಮಾಡಿದ ಮೊದಲ ಕೆಲಸಗಳಲ್ಲೊಂದೆಂದರೆ ತಮ್ಮ ಪರಮ ಪೂಜ್ಯ ತಾಯಿಯನ್ನು ಭೇಟಿಯಾದದ್ದು. ಪಾಶ್ಚಾತ್ಯ ದೇಶಗಳಲ್ಲಿನ ವಿಜಯಯಾತ್ರೆಯನ್ನು ಮುಗಿಸಿ ಬಂದ ಜಗತ್​ಪ್ರಸಿದ್ಧ ವಿವೇಕಾನಂದರು ಈಗ ಒಬ್ಬ ಸರಳ ಬಾಲಕನಂತೆ ಮಾತೆಯ ದರ್ಶನಕ್ಕಾಗಿ ಹೋದರು; ಹಿಂದಿನ ನರೇನನಾಗಿ ಓಡಿಹೋಗಿ ಆಕೆಯ ಪಾದಗಳಿಗೆ ಮಣಿದರು. ಆಗ ಆ ದಿವ್ಯಮಾತೆಯ ಹೃದಯದಲ್ಲಿ ಉಕ್ಕಿಹರಿದ ಆನಂದವನ್ನು ಅಳೆಯಬಲ್ಲವರಾರು!

ಸ್ವಾಮೀಜಿ ಆಲಂಬಜಾರಿನ ಮಠದಲ್ಲೇ ಉಳಿದುಕೊಂಡರೂ ಪ್ರತಿದಿನ ಬೆಳಗ್ಗೆ ಗೋಪಾಲ್ ಲಾಲ್ ಸೀಲರ ಉದ್ಯಾನಗೃಹಕ್ಕೆ ಬಂದುಬಿಡುತ್ತಿದ್ದರು. ಇಲ್ಲಿ ಅವರ ದರ್ಶನಕ್ಕಾಗಿ ಬೆಳಗಿನಿಂದ ರಾತ್ರಿಯವರೆಗೆ ಜನಪ್ರವಾಹ ಹರಿದುಬರತೊಡಗಿತು. ಬಗೆಬಗೆಯ ಪ್ರಶ್ನೋತ್ತರಗಳು, ವೇದಾಂ ತದ ವಿಚಾರವಾಗಿ ಸಂಭಾಷಣೆಗಳು ನಿರಂತರವಾಗಿ ನಡೆಯುತ್ತಲೇ ಹೋದುವು. ಈ ಮಧ್ಯೆ ಅವರನ್ನು ಅಭಿನಂದಿಸಿ ಬಂದ ತಂತಿಗಳು, ಬೇರೆಬೇರೆ ಊರುಗಳಿಂದ-ಸಂಘಸಂಸ್ಥೆಗಳಿಂದ ಬಂದ ಆಹ್ವಾನಗಳು ಲೆಕ್ಕವಿಲ್ಲದಷ್ಟು. “ನಮ್ಮಲ್ಲಿಗೆ ಬರಬೇಕು, ನಮ್ಮ ಸ್ವಾಗತವನ್ನು ಸ್ವೀಕರಿಸ ಬೇಕು” ಎಂದು ಕೇಳಿಕೊಳ್ಳುವವರೇ ಎಲ್ಲ. ಸ್ವಾಮೀಜಿ ಅಸಂಖ್ಯಾತ ಸಂದರ್ಶಕರೊಡನೆ ಸಂಭಾಷಣೆ ನಡೆಸಿದರು; ವಿದ್ಯಾವಂತ ವರ್ಗದವರೊಂದಿಗೆ ಶಾಸ್ತ್ರವಿಚಾರಗಳನ್ನು ಚರ್ಚಿಸಿದರು. ಇವುಗಳಿಂದೆಲ್ಲ ಸ್ವಾಮೀಜಿಯವರಿಗೆ ಸಾಕಷ್ಟು ಶ್ರಮವಾಯಿತು. ಇಷ್ಟಾದರೂ ಅವರ ಹೃದಯ ದಣಿಯಲಿಲ್ಲ, ಅವರು ಆನಂದಭರಿತರಾಗಿಯೇ ಇದ್ದರು. ಜನ ಅತ್ಯಂತ ಪ್ರಾಮಾಣಿಕಬುದ್ಧಿ ಯಿಂದ ಆಧ್ಯಾತ್ಮಿಕ ವಿಚಾರಗಳ ಕುರಿತಾಗಿ ತಿಳಿದುಕೊಳ್ಳಲು ಬರುವುದನ್ನು ಕಂಡು ಅವರಿಗೆ ಆನಂದ. ತಮ್ಮ ಭಾರತದ ಜನ ಇಷ್ಟೊಂದು ಧಾರ್ಮಿಕ ಶ್ರದ್ಧೆ ಉತ್ಸಾಹಗಳನ್ನು ತೋರುವುದನ್ನು ಕಂಡು ಆನಂದ. ಆದ್ದರಿಂದ ತಮಗೆಷ್ಟೇ ದಣಿವಾದರೂ ಅದನ್ನೆಲ್ಲ ಮರೆತು ಸರ್ವದಾ ಜನರನ್ನು ನಗುಮೊಗದಿಂದ ಸ್ವಾಗತಿಸಿದರು; ಅವರ ಧಾರ್ಮಿಕ-ಆಧ್ಯಾತ್ಮಿಕ ಸಮಸ್ಯೆಗಳಿಗೆ ಸಹನೆಯಿಂದ ಉತ್ತರ ನೀಡಿದರು.

ಫೆಬ್ರುವರಿ ೨೮ನೇ ತಾರೀಕು ಭಾನುವಾರ ಕಲ್ಕತ್ತದ ನಾಗರಿಕರ ಪರವಾಗಿ ವಿವೇಕಾನಂದರಿಗೆ ಸ್ವಾಗತ ಸಮಾರಂಭವನ್ನು ನಡೆಸಲಾಗುವುದೆಂದು ಘೋಷಿಸಲಾಯಿತು. ಸರ್ ರಾಧಾಕಾಂತ ದೇವ್ ಅವರ ಅರಮನೆಯಂತಹ ಬಂಗಲೆಯ ಮುಂದಿನ ಪ್ರಾಂಗಣದಲ್ಲಿ ಈ ಸಮಾರಂಭ. ಆದರೆ ಹಿಂದೆ ಆದಂತೆ ನೂಕುನುಗ್ಗಲು-ಗಲಭೆ ಸಂಭವಿಸದಂತೆ ಈ ಸಲ ಎಲ್ಲ ಏರ್ಪಾಡುಗಳನ್ನೂ ಮಾಡಲಾಗಿತ್ತು. ಪ್ರವೇಶ ಟಿಕೆಟ್ಟಿನ ಮೂಲಕ, ಆದರೆ ಟಿಕೆಟಿಗೆ ಹಣವಿಲ್ಲ, ಪುಕ್ಕಟೆ. ಬೇಕೆಂಬುವರ ಟಿಕೆಟ್ಟನ್ನು ಮೊದಲೇ ಪಡೆದಿರಬೇಕಾಗಿತ್ತು. ಭಾನುವಾರ ಸಂಜೆ ನಾಲ್ಕು ಗಂಟೆಗೆ ಕಾರ್ಯಕ್ರಮ ಆರಂಭವಾಯಿತು. ಸುಮಾರು ನಾಲ್ಕು ಸಾವಿರಕ್ಕೂ ಹೆಚ್ಚು ಜನ ಸೇರಿದ್ದರು. ವೇದಿಕೆಯ ಮೇಲೆ ಆಸೀನರಾಗಿದ್ದ ಗಣ್ಯ ವ್ಯಕ್ತಿಗಳಲ್ಲಿ ಕೆಲವರೆಂದರೆ ಜಸ್ಟಿಸ್ ಚಂದ್ರಮಾಧವಘೋಷ್, ಮಹಾ ರಾಜಾ ಜಗದಿಂದ್ರನಾಥ ರಾಯ್, ಪಂಡಿತ ರಾಮನರೇಂದ್ರನಾಥ ವಿದ್ಯಾನಿಧಿ, ಹಾಗೂ ವಿಶ್ವಕವಿ ಯಾದ ರವೀಂದ್ರನಾಥ ಟಾಗೋರ್. ರಾಜಾ ವಿನಯಕೃಷ್ಣದೇವ್ ಬಹಾದ್ದೂರರು ಅಧ್ಯಕ್ಷತೆ ವಹಿಸಿದ್ದರು. ಬ್ರಿಟಿಷ್ ಸರ್ಕಾರದ ಉನ್ನತ ಅಧಿಕಾರಿಯೊಬ್ಬರು ಸ್ವಾಮೀಜಿಯನ್ನು ಸ್ವಾಗತಿಸಿದರು. ಬಳಿಕ ಅಧ್ಯಕ್ಷರು ತಮ್ಮ ಸ್ಥಾನದಿಂದ ನಾಲ್ಕು ಮಾತುಗಳನ್ನಾಡಿ ಬಿನ್ನವತ್ತಳೆಯನ್ನು ಓದಿದರು. ಅನಂತರ ಆ ಬಿನ್ನವತ್ತಳೆಯನ್ನು ಬೆಳ್ಳಿಯ ಕರಂಡದಲ್ಲಿಟ್ಟು ಸ್ವಾಮೀಜಿಯವರಿಗೆ ಅರ್ಪಿಸ ಲಾಯಿತು. ಈಗ ಸ್ವಾಮೀಜಿ ಎದ್ದು ನಿಂತು ಮಾತಿಗಾರಂಭಿಸಿದರು. ಭಾಷಣಕಲೆಯ ದೃಷ್ಟಿಯಿಂದ ಮತ್ತು ಸ್ವಾಮೀಜಿ ವ್ಯಕ್ತಪಡಿಸಿದ ರಾಷ್ಟ್ರಪ್ರೇಮದ ಭಾವನೆಗಳ ಆಳದ ದೃಷ್ಟಿಯಿಂದ ಅವರ ಈ ಭಾಷಣ ತುಂಬ ಗಮನಾರ್ಹವಾದದ್ದು. ಅವರು ‘ನವಭಾರತದ ಪ್ರವಾದಿ’ ಎಂಬುದನ್ನು ಈ ಭಾಷಣವು ಮತ್ತೊಮ್ಮೆ ಸ್ಪಷ್ಟಪಡಿಸುವಂತಿತ್ತು.

ಕಲ್ಕತ್ತದ ನಾಗರಿಕರನ್ನುದ್ದೇಶಿಸಿ ಸ್ವಾಮೀಜಿ ನುಡಿದರು: “ಸೋದರರೆ, ನಾನಿಂದು ಇಲ್ಲಿ ನಿಂತಿ ರುವುದು ಒಬ್ಬ ಸಂನ್ಯಾಸಿಯಂತಲ್ಲ ಅಥವಾ ಒಬ್ಬ ಧರ್ಮಪ್ರಸಾರಕನಂತಲ್ಲ. ಬದಲಾಗಿ ನೀವು ಹಿಂದೆ ಕಂಡಿರುವ ಅದೇ ಕಲ್ಕತ್ತದ ಬಾಲಕನಾಗಿ ನಾನಿಲ್ಲಿ ನಿಂತಿದ್ದೇನೆ.” ಬಳಿಕ ಪಾಶ್ಚಾತ್ಯ ರಾಷ್ಟ್ರ ಗಳಲ್ಲಿನ ಕಾರ್ಯದ ಬಗ್ಗೆ ಪ್ರಸ್ತಾಪಿಸಿ, ಸರ್ವಧರ್ಮ ಸಮ್ಮೇಳನಕ್ಕಾಗಿಯೇ ತಾವು ಹೋದದ್ದಲ್ಲ, ಆದರೆ ಅದೊಂದು ಅವಕಾಶವಾಗಿತ್ತು ಅಷ್ಟೆ, ಎಂದು ಹೇಳಿದರು.

ಸ್ವಾಮೀಜಿಯವರಿಗೆ ನೀಡಲಾದ ಬಿನ್ನವತ್ತಳೆಯಲ್ಲಿ ಶ್ರೀರಾಮಕೃಷ್ಣರಿಗೆ ಶ್ರದ್ಧಾಂಜಲಿಯನ್ನು ಸಲ್ಲಿಸುವ ಮಾತುಗಳಿದ್ದುವು. ಆ ವಿಷಯವನ್ನು ಪ್ರಸ್ತಾಪಿಸಿ ಅವರು ಭಾವಭರಿತರಾಗಿ ನುಡಿದರು, “ಸೋದರರೆ, ನೀವು ನನ್ನ ಹೃದಯದ ಅತ್ಯಂತ ಸೂಕ್ಷ್ಮವಾದ ಭಾವತಂತುವೊಂದನ್ನು ಮಿಡಿದಿ ದ್ದೀರಿ. ಈಗ ತಾನೆ ನೀವು ನನ್ನ ಗುರುದೇವ, ನನ್ನ ಆಚಾರ್ಯ, ನನ್ನ ಜೀವನಾದರ್ಶ, ನನ್ನ ಇಷ್ಟ, ನನ್ನ ಪ್ರಾಣ ಹಾಗೂ ನನ್ನ ದೇವರೇ ಆದ ಶ್ರೀರಾಮಕೃಷ್ಣ ಪರಮಹಂಸರ ಪವಿತ್ರ ನಾಮೋಚ್ಚಾರಣೆ ಯನ್ನು ಮಾಡಿದಿರಿ. ನಾನು ಕಾಯಾ ವಾಚಾ ಮನಸಾ ಯಾವುದಾದರೊಂದು ಸತ್ಕರ್ಮವನ್ನು ಮಾಡಿದ್ದರೆ, ನನ್ನ ಬಾಯಿಂದ ಯಾರಿಗಾದರೂ ಸಹಾಯವಾಗುವಂತಹ ನುಡಿಯೊಂದು ಹೊರ ಬಿದ್ದಿದ್ದರೆ ಅದು ನನ್ನದಲ್ಲ, ಅದೆಲ್ಲ ಅವರದು. ಆದರೆ ನನ್ನ ಬಾಯಿಂದ ಏನಾದರೂ ನಿಂದೆಯ ನುಡಿ ಹೊರಬಿದ್ದಿದ್ದರೆ, ಖಂಡನೆಯ ಮಾತು ಹೊರ ಬಿದ್ದಿದ್ದರೆ ಅದೆಲ್ಲ ನನ್ನದೇ ಹೊರತು ಅವರ ದಲ್ಲ. ದೌರ್ಬಲ್ಯಕಾರಕವಾದದ್ದು ಏನಿದ್ದರೂ ಅದು ನನ್ನದು. ಆದರೆ ಯಾವುದು ಚೈತನ್ಯದಾಯಕ ವಾದುದೊ, ಪವಿತ್ರವಾದುದೊ ಅದೆಲ್ಲ ಅವರ ಶಕ್ತಿಲೀಲೆ, ಅವರ ವಾಣಿ, ಸ್ವಯಂ ಅವರೇ. ಹೌದು ಸ್ನೇಹಿತರೇ, ಆ ಮಹಾನುಭಾವರನ್ನು ಜಗತ್ತಿನ್ನೂ ಅರಿಯಬೇಕಾಗಿದೆ... ಅವರ ನೆರಳಲ್ಲಿ ಬೆಳೆದುಬಂದವನು ನಾನು. ನಾನು ಕಲಿತಿರುವುದೆಲ್ಲ ಅವರ ಅಡಿದಾವರೆಗಳಲ್ಲಿ ಕುಳಿತು.”

ತಮ್ಮ ಭಾಷಣದಲ್ಲಿ ಸ್ವಾಮೀಜಿಯವರು ಶ್ರೀರಾಮಕೃಷ್ಣರ ಬಗ್ಗೆ ಆಡಿದ ಈ ಮಾತುಗಳು ನಿಜಕ್ಕೂ ತುಂಬ ಸತ್ಪರಿಣಾಮ ಬೀರಿದುವು. ‘ಪಾಶ್ಚಾತ್ಯ ದೇಶಗಳಲ್ಲಿ ಸ್ವಾಮೀಜಿ ಕೇವಲ ತಮ್ಮ ಸ್ವಂತ ಭಾವನೆಗಳನ್ನಷ್ಟೇ ಪ್ರಚಾರ ಮಾಡಿದರು, ತಮ್ಮ ಗುರುದೇವನಾದ ಶ್ರೀರಾಮಕೃಷ್ಣರ ಬಗ್ಗೆ ಹಾಗೂ ಅವರ ಸಂದೇಶದ ಬಗ್ಗೆ ಪ್ರಸ್ತಾಪಿಸಲೇ ಇಲ್ಲ’ ಎಂದು ಕಟುವಾಗಿ ಟೀಕಿಸಿದ ಅನೇಕರು ಬಂಗಾಳದಲ್ಲಿದ್ದರು. ಆದರೆ ಅದಕ್ಕೆ ಕಾರಣಗಳಿವೆ ಎಂಬುದು ಇವರಿಗೆ ತಿಳಿಯದು. ಮೊದಲನೆ ಯದಾಗಿ, ಜನರನ್ನು ವ್ಯಕ್ತಿನಿಷ್ಠರನ್ನಾಗಿಸುವುದಕ್ಕಿಂತ ತತ್ತ್ವನಿಷ್ಠರನ್ನಾಗಿಸುವುದು ಉತ್ತಮ ಎಂಬುದು ಸ್ವಾಮೀಜಿಯವರ ನಿಶ್ಚಿತ ಅಭಿಮತ. ಎರಡನೆಯದಾಗಿ, ಶ್ರೀರಾಮಕೃಷ್ಣರ ಬಗ್ಗೆ ಪ್ರಚಾರ ಮಾಡಲು ಕಾಲವಿನ್ನೂ ಪಕ್ವವಾಗಿರಲಿಲ್ಲ. ಮೂರನೆಯದಾಗಿ, ಶ್ರೀರಾಮಕೃಷ್ಣರ ಬಗ್ಗೆ ಮಾತನಾಡ ಹೊರಟಾಗಲೆಲ್ಲ ಸ್ವಾಮೀಜಿಯವರಿಗೆ ಭಾವ ತುಂಬಿಬಂದು ಮಾತೇ ಹೊರಡದಂತಾ ಗುತ್ತಿತ್ತು. ಆದ್ದರಿಂದ ಜನ ತಿಳಿದಂತೆ ಸ್ವಾಮೀಜಿ ಶ್ರೀರಾಮಕೃಷ್ಣರ ಬಗ್ಗೆ ಹೆಚ್ಚಾಗಿ ಮಾತನಾಡ ದಿದ್ದುದು ಭಕ್ತಿಯ ಅಭಾವದಿಂದಾಗಿಯಲ್ಲ, ಭಕ್ತಿಯ ಆಧಿಕ್ಯದಿಂದಾಗಿ! ಆದರೆ ಮದ್ರಾಸು ಹಾಗೂ ಕಲ್ಕತ್ತಗಳಲ್ಲಿ ಅವರು ಮಾಡಿದ ಭಾಷಣಗಳು ಜನರ ತಪ್ಪು ಕಲ್ಪನೆಯನ್ನು ಹೋಗಲಾಡಿ ಸಿದುವು, ಟೀಕೆಗಾರರ ಬಾಯಿ ಮುಚ್ಚಿಸಿದುವು.

ಶ್ರೀರಾಮಕೃಷ್ಣರ ಬಗ್ಗೆ ಹೇಳಿದ ಬಳಿಕ ಸ್ವಾಮೀಜಿಯವರು ಕಲ್ಕತ್ತದ ಯುವಕರನ್ನುದ್ದೇಶಿಸಿ ನುಡಿದರು, “ಉತ್ತಿಷ್ಠತ, ಜಾಗ್ರತ, ಪ್ರಾಪ್ಯ ವರಾನ್ ನಿಬೋಧತ! ಕಲ್ಕತ್ತದ ಯುವಕರೇ, ಏಳಿ, ಎಚ್ಚರಗೊಳ್ಳಿ. ಸಮಯ ಸನ್ನಿಹಿತವಾಗುತ್ತಿದೆ. ನಿರ್ಭೀತರಾಗಿ, ಆಗ ಮಾತ್ರ ನಮ್ಮ ಕೆಲಸ ಪರಿ ಪೂರ್ಣವಾಗುತ್ತದೆ. ನಮ್ಮ ದೇಶಕ್ಕೆ ಅದ್ಭುತ ತ್ಯಾಗಿಗಳು ಬೇಕಾಗಿದ್ದಾರೆ. ಯುವಕರು ಮಾತ್ರ ಇದನ್ನು ಮಾಡಬಲ್ಲರು. ಆಶಿಷ್ಠ ದೃಢಿಷ್ಠ ಬಲಿಷ್ಠರಾದ ಮೇಧಾವಿಗಳಿಗೆ ಮಾತ್ರವೇ ಈ ಕೆಲಸ ಮೀಸಲಾಗಿದೆ. ಕಲ್ಕತ್ತದಲ್ಲಿ ಇಂತಹ ನೂರಾರು, ಸಾವಿರಾರು ಯುವಕರಿದ್ದೀರಿ. ಏಳಿ, ಜಾಗೃತರಾಗಿ!

“ನಾನಿನ್ನೂ ಏನನ್ನೂ ಮಾಡಿಲ್ಲ. ಕಾರ್ಯವನ್ನು ನೀವು ಸಾಧಿಸಬೇಕಾಗಿದೆ. ನಾಳೆ ನಾನು ಸತ್ತರೂ ಕಾರ್ಯ ನಿಲ್ಲುವುದಿಲ್ಲ. ಸಹಸ್ರಾರು ಜನರು ಕೆಲಸಕ್ಕೆ ಕೈಹಾಕಿ ನನ್ನ ಕಲ್ಪನೆಗೂ ಅತೀತವಾಗಿ ಕೆಲಸ ಮಾಡುತ್ತ ಹೋಗುವರೆಂದು ನಾನು ದೃಢವಾಗಿ ನಂಬಿದ್ದೇನೆ. ನನಗೆ ನನ್ನ ದೇಶದಲ್ಲಿ ವಿಶ್ವಾಸವಿದೆ. ಅದರಲ್ಲೂ ನಮ್ಮ ಯುವಕರಲ್ಲಿ ನನಗೆ ಹೆಚ್ಚಿನ ವಿಶ್ವಾಸವಿದೆ. ಕಳೆದ ಹತ್ತು ವರ್ಷಗಳ ಕಾಲ ಭರತಖಂಡವನ್ನೆಲ್ಲ ಸಂಚರಿಸಿದ್ದೇನೆ. ಭರತಖಂಡವನ್ನು ಉನ್ನತ ಆಧ್ಯಾತ್ಮಿಕ ಶಿಖರಕ್ಕೆ ಸಾಗಿಸುವ ಶಕ್ತಿ ಯುವಕರಿಂದಲೇ ಬರುವುದೆಂಬುದು ನನ್ನ ದೃಢ ನಂಬಿಕೆ. ಅನಂತ ಶ್ರದ್ಧೆ-ಉತ್ಸಾಹಗಳಿಂದ ತುಂಬಿ ತುಳುಕಾಡುವ ಯುವಜನಾಂಗದಿಂದ ನಮ್ಮ ಸನಾತನ ತತ್ತ್ವಗಳನ್ನು ಪ್ರಚಾರ ಮಾಡ ಬಲ್ಲವರು ಉದಿಸುತ್ತಾರೆ. ನಿಮ್ಮ ಮುಂದಿರುವ ಬೃಹತ್ತರ ಕಾರ್ಯವೇ ಇದು. ಏಳಿ, ಎಚ್ಚರ ಗೊಳ್ಳಿ, ಗುರಿ ಮುಟ್ಟುವವರೆಗೂ ನಿಲ್ಲದಿರಿ!

“ಧೈರ್ಯಗುಂದದಿರಿ; ಮಾನವ ಇತಿಹಾಸದಲ್ಲೆಲ್ಲ ಮಹಾಶಕ್ತಿ ಕಂಡುಬಂದದ್ದು ಜನ ಸಾಮಾನ್ಯರಲ್ಲೇ. ಜಗತ್ತಿನ ಮಹಾ ವಿಭೂತಿಗಳೆಲ್ಲ ಜನಿಸಿದುದು ಅಂಥವರ ನಡುವೆಯೇ. ಈಗಲೂ ಅಂತಹ ಮಹಾಪುರುಷರು ನಿಮ್ಮ ನಡುವೆ ಉದಿಸಬೇಕಾಗಿದೆ. ಯಾವುದಕ್ಕೂ ಅಂಜ ಬೇಡಿ. ನೀವು ಅದ್ಭುತ ಕಾರ್ಯಗಳನ್ನು ಸಾಧಿಸುವಿರಿ. ಆದರೆ ಅಂಜಿದೊಡನೆಯೇ ನೀವು ಕೆಲಸಕ್ಕೆ ಬಾರದವರಾಗುತ್ತೀರಿ. ಈ ಅಂಜಿಕೆ ಎಂಬುದೇ ಎಲ್ಲಕ್ಕಿಂತ ದೊಡ್ಡ ಮೂಢನಂಭಿಕೆ. ನಿಮ್ಮ ಕಷ್ಟಗಳಿಗೆಲ್ಲ ಕಾರಣ ಈ ಅಂಜಿಕೆ. ನಿರ್ಭಯತೆ ಎಂಬುದು ನಿಮಗೆ ಕ್ಷಣಮಾತ್ರದಲ್ಲಿ ಸ್ವರ್ಗವನ್ನೇ ತಂದುಕೊಡಬಲ್ಲುದು. ಆದ್ದರಿಂದ ಏಳಿ, ಎಚ್ಚರಗೊಳ್ಳಿ, ಗುರಿ ಮುಟ್ಟುವವರೆಗೂ ನಿಲ್ಲದಿರಿ!”

“ಮಹಾಜನರೇ, ನೀವು ನನ್ನ ಮೇಲೆ ತೋರಿದ ವಿಶ್ವಾಸಕ್ಕೆ ನಿಮಗೆ ಮತ್ತೊಮ್ಮೆ ನನ್ನ ಧನ್ಯ ವಾದಗಳು. ನಾನು ಈ ಜಗತ್ತಿಗೆ ಸ್ವಲ್ಪವಾದರೂ ಪ್ರಯೋಜನಕ್ಕೆ ಬರುವಂತಾಗಬೇಕು, ಎಲ್ಲಕ್ಕಿಂತ ಹೆಚ್ಚಾಗಿ ನನ್ನ ದೇಶಕ್ಕೆ ನನ್ನ ದೇಶಬಾಂಧವರಿಗೆ ಒಂದಿನಿತಾದರೂ ನೆರವಾಗಬೇಕು ಎಂಬುದೇ ನನ್ನ ಇಚ್ಛೆ; ಇದೇ ನನ್ನಂತರಂಗದ ತೀವ್ರ ಆಕಾಂಕ್ಷೆ.”

ಈ ಸಮಾರಂಭ ನಡೆದ ಕೆಲದಿನಗಳಲ್ಲೇ, ಎಂದರೆ ಮಾರ್ಚ್ ನಾಲ್ಕರಂದು ಗುರುವಾರ, ಸ್ವಾಮೀಜಿ ಕಲ್ಕತ್ತದಿಂದ ತಮ್ಮ ಎರಡನೇ ಸಾರ್ವಜನಿಕ ಭಾಷಣ ಮಾಡಿದರು. ಸ್ಟಾರ್ ಥಿಯೇಟರಿನಲ್ಲಿ ಅವರು ಮಾಡಿದ ಭಾಷಣದ ವಿಷಯ “ವೇದಾಂತದ ವಿವಿಧ ಮುಖಗಳು.” ಧಾರ್ಮಿಕ ವಿಚಾರಗಳ ಕುರಿತಾದ ಭಾಷಣಗಳನ್ನು ಮಾಡಲು ಅವರಿಗೀಗ ಸ್ವಲ್ಪವೂ ಇಷ್ಟವಿರ ಲಿಲ್ಲ. ಅವರ ಮನಸ್ಸೀಗ ಜನರ ದಾರಿದ್ರ್ಯವನ್ನು ಕುರಿತಾ ಚಿಂತೆಯಲ್ಲಿ ತೊಡಗಿತ್ತು. ಅಲ್ಲದೆ ಭೀಕರ ಕ್ಷಾಮ ಬೇರೆ ಆವರಿಸಿ ಉತ್ತರ ಹಾಗೂ ಮಧ್ಯ ಭಾರತದಲ್ಲಿ ಲಕ್ಷಾಂತರ ಮಾನವ ಜೀವಗಳನ್ನು ಆಹುತಿ ತೆಗೆದುಕೊಳ್ಳುತ್ತಿತ್ತು. “ಧರ್ಮಬೋಧನೆಯು ಬರಿಹೊಟ್ಟೆಗಲ್ಲ!” ಎಂಬ ಶ್ರೀರಾಮಕೃಷ್ಣರ ಮಾತು ಅವರ ಕಿವಿಯಲ್ಲೀಗ ಮಾರ್ದನಿಗೊಳ್ಳುತ್ತಿತ್ತು. ಆದರೂ ಸ್ವಾಮೀಜಿ ಕಲ್ಕತ್ತದ ನಾಗರಿಕರ ಬೇಡಿಕೆಗೆ ಮಣಿದು ಈ ಉಪನ್ಯಾಸವನ್ನು ನೀಡಲೊಪ್ಪಿದರು. ಅಲ್ಲದೆ, ಅವರು ಯಾವ ತಮ್ಮ ವಿನೂತನ ವ್ಯಾಖ್ಯಾನಗಳಿಂದ ಹಿಂದೂಧರ್ಮವನ್ನು, ವೇದಾಂತವನ್ನು ಜನಪ್ರಿಯ ಗೊಳಿಸಿದರೋ, ವಿಖ್ಯಾತಿಗೊಳಿಸಿದರೋ, ಆ ವ್ಯಾಖ್ಯಾನವನ್ನು ತಾವೂ ಕೇಳಬೇಕೆಂದು ಕಲ್ಕತ್ತದ ಜನ ಬಯಸಿದ್ದು ಸಹಜವೇ ಅಲ್ಲವೆ? ಅಲ್ಲದೆ, ಅವರು ಬೋಧಿಸಿದಂತಹ ವೇದಾಂತವೇ ಹಿಂದೂಧರ್ಮದ ಸಾರಸರ್ವಸ್ವವಲ್ಲ ಎಂಬ ಟೀಕೆ ಕೇಳಿ ಬಂದಿತ್ತು. ಅದಕ್ಕೀಗ ಉತ್ತರಿಸಲು ಹಾಗೂ ತಮ್ಮ ನಿಲುವನ್ನು ಸಮರ್ಥಿಸಿಕೊಳ್ಳಲು ಇದೊಂದು ಅವಕಾಶವೆಂದು ಸ್ವಾಮೀಜಿ ಭಾವಿಸಿದರು.

ಸಮಾರಂಭವನ್ನು ಶಿಸ್ತುಬದ್ಧವಾಗಿ ನಡೆಸುವ ಉದ್ದೇಶದಿಂದ ಈ ಭಾಷಣಕ್ಕೆ ಪ್ರವೇಶ ಶುಲ್ಕ ವನ್ನು ನಿಗದಿಪಡಿಸಲಾಗಿತ್ತು–ಬೆಲೆ ರೂ. ೨, ರೂ. ೧, ಮತ್ತು ಎಂಟಾಣೆ. ಸಮಾಜದ ಉನ್ನತ ವರ್ಗದ ಗೌರವಾನ್ವಿತ ಸಭಿಕರಿಂದ ಸಭಾಂಗಣ ತುಂಬಿಹೋಗಿತ್ತು. ಸುಮಾರು ಒಂದೂವರೆ ಗಂಟೆಗಳ ಕಾಲ ಸಾಗಿದ ಈ ಭಾಷಣವು ತತ್ತ್ವಶಾಸ್ತ್ರವನ್ನು ಕುರಿತ ಅವರ ಅತ್ಯುತ್ಕೃಷ್ಟ ಭಾಷಣ ಗಳಲ್ಲೊಂದಾಗಿತ್ತು. ಶ್ರೀಯುತ ನರೇಂದ್ರನಾಥ ಸೇನರು ಅಧ್ಯಕ್ಷತೆ ವಹಿಸಿದರು. ಇವರಲ್ಲದೆ ಇತರ ಹಲವಾರು ಗಣ್ಯವ್ಯಕ್ತಿಗಳು ಅಲ್ಲಿ ಉಪಸ್ಥಿತರಿದ್ದರು.

ಸ್ವಾಮೀಜಿಯವರ ಮೊದಲ ಕೆಲವು ಮಾತುಗಳೇ ಶ್ರೋತೃಗಳಲ್ಲಿ ವಿದ್ಯುತ್ ಸಂಚಾರವನ್ನು ಉಂಟುಮಾಡಿದುವು: “ಯಾವ ಅಗಾಧ ಕಾಲಗರ್ಭದೊಳಕ್ಕೆ–ಚಾರಿತ್ರಿಕ ದಾಖಲೆಗಳ ಮಾತು ಹಾಗಿರಲಿ–ಪರಂಪರಾಗತವಾದ ಅಂತೆಕಂತೆಯ ಕಥೆಗಳ ಮಂದಪ್ರಕಾಶವೂ ತೂರಿಹೋಗಲಾ ರದೋ, ಆ ಆಳದಿಂದಲೂ ಉಜ್ವಲ ಕಾಂತಿಯೊಂದು ನಿರಂತರವಾಗಿ ಹೊಮ್ಮಿ ಬರುತ್ತಿದೆ. ಆ ಕಾಂತಿಯು ಬಾಹ್ಯಪ್ರಕೃತಿಯ ಪ್ರಭಾವಕ್ಕನುಗುಣವಾಗಿ ಕೆಲವೊಮ್ಮೆ ಮಂದವಾದರೂ ಎಂದಿಗೂ ನಾಶವಾಗದೆ ಸ್ಥಿರವಾಗಿ ಪ್ರಕಾಶಿಸುತ್ತಿದೆ. ಯಾರ ಕಣ್ಣಿಗೂ ಕಾಣದಂತೆ, ಯಾರ ಅರಿವಿಗೂ ಬಾರ ದಂತೆ ಉದುರಿ, ಅತಿ ಸುಂದರ ಗುಲಾಬಿಯೊಂದನ್ನು ಅರಳಿಸುವ ಮುಂಜಾನೆಯ ಹಿಮಮಣಿ ಯಂತೆ ಈ ಜ್ಯೋತಿಕಿರಣಗಳು ಭರತಖಂಡವನ್ನು ಮಾತ್ರವಲ್ಲದೆ ಸಮಸ್ತ ಭಾವನಾಪ್ರಪಂಚ ವನ್ನೇ ಮೌನವಾಗಿ, ಮೃದುವಾಗಿ, ಆದರೆ ಅತಿ ಪ್ರಬಲವಾಗಿ ಆವರಿಸಿವೆ–ಉಪನಿಷತ್ತುಗಳ ಬೋಧನೆ ಇಂಥದು, ವೇದಾಂತದ ಬೋಧನೆ ಇಂಥದು.”

“ವೇದಾಂತ ದರ್ಶನವು ಆಧ್ಯಾತ್ಮಿಕ ರಾಜ್ಯದಲ್ಲಿ ಮಾನವನು ಮಾಡಿದ ಪ್ರಥಮ ಆವಿಷ್ಕಾರ ಮತ್ತು ಅದೇ ಪರಾಕಾಷ್ಠೆ ಕೂಡ ಎಂದು ನಾನು ಧೈರ್ಯವಾಗಿ ಹೇಳುತ್ತೇನೆ. ಈ ವೇದಾಂತವೆಂಬ ಮಹಾಸಾಗರದಿಂದ ಕಾಲಕಾಲಕ್ಕೆ ಅಲೆಗಳೆದ್ದು ಪೂರ್ವ ಪಶ್ಚಿಮ ದೇಶಗಳಿಗೆ ಪಸರಿಸಿವೆ. ಸಾಂಖ್ಯ ದರ್ಶನವು ಪುರಾತನ ಗ್ರೀಕರ ಮೇಲೆ ನಿಸ್ಸಂದೇಹವಾಗಿ ತನ್ನ ಪ್ರಭಾವವನ್ನು ಬೀರಿತು... ಸಾಂಖ್ಯ ಮತ್ತು ಇತರ ದರ್ಶನಗಳೆಲ್ಲ ಪ್ರತಿಷ್ಠಿತವಾಗಿರುವುದು ಉಪನಿಷತ್ತುಗಳ ಮೇಲೆಯೇ. ಭರತ ಖಂಡದಲ್ಲಿ ಹಿಂದೆ ಇದ್ದ ಹಾಗೂ ಈಗಲೂ ಇರುವ ಅಸಂಖ್ಯಾತ ಮತಪಂಗಡಗಳೆಲ್ಲಕ್ಕೂ ಒಂದು ಸಾಮಾನ್ಯ ಪ್ರಮಾಣ, ಒಂದು ಸಾಮಾನ್ಯ ನೆಲೆ ಎಂದರೆ ಅದು ಉಪನಿಷತ್ತು... ಯಾರು ಉಪನಿಷತ್ತನ್ನು ಅನುಸರಿಸುವುದಿಲ್ಲವೊ ಅವರನ್ನು ಹಿಂದೂಗಳೆನ್ನುವಂತಿಲ್ಲ. ಜೈನ ಹಾಗೂ ಬೌದ್ಧ ಮತಸ್ಥರೂ ಉಪನಿಷತ್ತುಗಳನ್ನು ಪ್ರಮಾಣವಾಗಿ ಸ್ವೀಕರಿಸದಿದ್ದುದರಿಂದಲೇ ಭರತ ಖಂಡದಲ್ಲಿ ಅವರಿಗೆ ಸ್ಥಳವಿಲ್ಲದಂತಾಯಿತು. ಆದ್ದರಿಂದ ನಮಗೆ ತಿಳಿದಿರಲಿ ತಿಳಿಯದಿರಲಿ, ವೇದಾಂತವು ಹಿಂದೂ ಧರ್ಮದ ಸಕಲ ಮತಪಂಗಡಗಳೊಳಕ್ಕೂ ಪ್ರವೇಶಿಸಿದೆ. ನಮಗೆ ತಿಳಿದೋ ತಿಳಿಯದೆಯೋ ನಾವು ವೇದಾಂತವನ್ನೇ ಆಲೋಚಿಸುತ್ತೇವೆ, ವೇದಾಂತದಲ್ಲೇ ಜೀವಿಸುತ್ತೇವೆ, ವೇದಾಂತವನ್ನೇ ಉಸಿರಾಡುತ್ತೇವೆ, ವೇದಾಂತದಲ್ಲೇ ಕೊನೆಯುಸಿರೆಳೆಯುತ್ತೇವೆ.

“ಹೀಗಿರುವಾಗ ಈ ಭಾರತದಲ್ಲಿ, ಅದರಲ್ಲೂ ಹಿಂದೂಗಳ ಮುಂದೆ ವೇದಾಂತವನ್ನು ಬೋಧಿ ಸುವುದು ಅಸಂಗತವಾಗಿ ಕಾಣಬಹುದು. ಆದರೆ ನಿಜವಾಗಿಯೂ ಬೋಧಿಸಬೇಕಾಗಿರುವುದೆಂದರೆ ಇದೊಂದನ್ನೇ. ಮತ್ತು ಈ ಕಾಲಕ್ಕೆ ಅದರ ಬೋಧನೆಯು ಅತ್ಯಾವಶ್ಯಕವಾಗಿದೆ. ನಾನು ಆಗಲೇ ಹೇಳಿದಂತೆ ಎಲ್ಲ ಹಿಂದೂ ಪಂಗಡಗಳೂ ಉಪನಿಷತ್ತಿಗೆ ಅಧೀನವಾಗಿರಬೇಕು. ಆದರೆ ಈ ಪಂಗಡಗಳಲ್ಲಿ ಎಷ್ಟೋ ಪರಸ್ಪರ ವಿರೋಧಾಭಿಪ್ರಾಯಗಳಿವೆ. ಎಷ್ಟೋ ವೇಳೆ ನಮ್ಮ ಪುರಾತನ ಪುಷಿಗಳಿಗೂ ಕೂಡ ಉಪನಿಷತ್ತುಗಳ ಹಿಂದಿನ ಏಕಸೂತ್ರತೆ ಅರಿವಿಗೆ ಬಾರದೆ ಹೋಯಿತು. ಅನೇಕ ವೇಳೆ ಈ ಪುಷಿಗಳ ನಡುವೆಯೂ ತಿಕ್ಕಾಟಗಳುಂಟಾದುವು. ಆದರೆ, ಉಪನಿಷತ್ತುಗಳ ಬೋಧನೆಗಳ ಹಿಂದಿನ ಏಕಸೂತ್ರತೆಯನ್ನು–ಅವು ದ್ವೈತವಾಗಿರಲಿ, ವಿಶಿಷ್ಟಾದ್ವೈತವಾಗಿರಲಿ ಅಥವಾ ಇನ್ನಾವುದೇ ಆಗಿರಲಿ–ಎತ್ತಿಹಿಡಿಯುವ ಸೂಕ್ತ ವ್ಯಾಖ್ಯಾನವೊಂದನ್ನು ನೀಡಲು ಕಾಲ ಸನ್ನಿಹಿತವಾಗಿದೆ. ಈ ಸತ್ಯವನ್ನು ಇಡೀ ಭಾರತಕ್ಕೆ ಮಾತ್ರವಲ್ಲ, ಸಮಸ್ತ ಜಗತ್ತಿಗೇ ತೋರಿಸಿಕೊಡಬೇಕಾಗಿದೆ.

“ಭಗವಂತನ ಕೃಪೆಯಿಂದ ಒಬ್ಬ ಮಹಾತ್ಮನ ಪದತಲದಲ್ಲಿ ಕುಳಿತು ಕಲಿಯುವ ಸೌಭಾಗ್ಯ ನನಗೆ ದೊರಕಿತ್ತು. ಅವರ ಇಡೀ ಜೀವನವೇ ಉಪನಿಷತ್ತಿನ ಹಿಂದಿರುವ ಏಕಸೂತ್ರತೆಯನ್ನು ತೋರಿಸಿಕೊಡುವ ವ್ಯಾಖ್ಯಾನವಾಗಿತ್ತು. ಅವರ ಉಪದೇಶಗಳಿಗಿಂತ ಸಾವಿರ ಪಾಲು ಹೆಚ್ಚಾಗಿ ಅವರ ಜೀವನವೇ ಉಪನಿಷತ್ತುಗಳಿಗೆ ಬರೆದ ಸಚೇತನ ಭಾಷ್ಯದಂತಿತ್ತು; ಉಪನಿಷತ್ತುಗಳ ತತ್ತ್ವವೇ ಮಾನವರೂಪದಿಂದ ಮೈದಳೆದಂತಿತ್ತು. ಬಹುಶಃ ಆ ಸಮನ್ವಯತೆಯು ನನ್ನಲ್ಲೂ ಸ್ವಲ್ಪ ಇರಬಹುದು. ಆದರೆ ಅದನ್ನು ವಿವರಿಸಲು ನನಗೆ ಸಾಧ್ಯವಾಗುವುದೋ ಇಲ್ಲವೋ ತಿಳಿ ಯದು. ಆದರೆ, ‘ವೇದಾಂತದ ವಿಭಿನ್ನ ಶಾಖೆಗಳು ಪರಸ್ಪರ ವಿರುದ್ಧವಾದವುಗಳಲ್ಲ, ಒಂದು ಮತ್ತೊಂದರ ಪೂರೈಕೆ; ಮತ್ತು ತತ್ತ್ವಮಸಿ ಎಂಬ ಅದ್ವೈತ ಶಿಖರವನ್ನು ಮುಟ್ಟುವವರೆಗೆ ಒಂದು ಮತ್ತೊಂದಕ್ಕೆ ಒಯ್ಯುವ ಮೆಟ್ಟಿಲು’ ಎಂಬುದನ್ನು ತೋರುವುದೇ ನನ್ನ ಉದ್ದೇಶ, ನನ್ನ ಜೀವನದ ಗುರಿ.”

ಹೀಗೆ ಸ್ವಾಮೀಜಿ ಅಂದಿನ ತಮ್ಮ ಒಂದೂವರೆ ಗಂಟೆಯ ಸುದೀರ್ಘ ಭಾಷಣದಲ್ಲಿ ಉಪನಿ ಷತ್ತುಗಳ ಪ್ರಮುಖ ಲಕ್ಷಣಗಳು, ಬ್ರಹ್ಮಸೂತ್ರಗಳು ಹಾಗೂ ಭಗವದ್ಗೀತೆಯ ಮಹತ್ವ–ಇವು ಗಳನ್ನು ವಿವರಿಸಿದರು. ಬಳಿಕ ಶಾಸ್ತ್ರಗಳಿಗೆ ವಿವಿಧ ಭಾಷ್ಯಕಾರರು ಬರೆದಿರುವ ಭಾಷ್ಯಗಳ ಲಕ್ಷಣ ಗಳನ್ನೂ ಅವುಗಳ ಯುಕ್ತಾಯುಕ್ತತೆಯನ್ನೂ ತಿಳಿಸಿದರು. ಭಾರತೀಯ ತತ್ತ್ವಶಾಸ್ತ್ರವನ್ನು ಪಾಶ್ಚಾತ್ಯ ತತ್ತ್ವಶಾಸ್ತ್ರದೊಂದಿಗೆ ಹೋಲಿಸಿ, ಪಾಶ್ಚಾತ್ಯ ತತ್ತ್ವಶಾಸ್ತ್ರದ ಅರ್ಥಹೀನತೆ-ಪೊಳ್ಳುತನಗಳನ್ನು ಬಯಲಿಗೆಳೆದರು. ತ್ಯಾಗದ ಮಹತ್ವವನ್ನು ಎತ್ತಿಹಿಡಿಯುತ್ತ ಸ್ವಾಮೀಜಿ ಹೇಳಿದರು, “ತ್ಯಾಗ! ಇದೇ ಭಾರತದ ಬಾವುಟ, ಹಿಂದೂ ಧರ್ಮದ ಲಾಂಛನ... ಹಿಂದೂಗಳೇ, ಆ ಬಾವುಟವು ನಿಮ್ಮ ಕೈಜಾರಲು ಬಿಡದಿರಿ! ಅದನ್ನು ಮೇಲೆತ್ತಿ ಹಿಡಿಯಿರಿ. ನೀವು ದುರ್ಬಲರಾಗಿದ್ದು, ತ್ಯಾಗ ಮಾಡಲು ನಿಮ್ಮಿಂದ ಸಾಧ್ಯವಿಲ್ಲವಾದರೂ, ಆದರ್ಶವನ್ನು ಮಾತ್ರ ಕೆಳಗಿಳಿಸಬೇಡಿ. ‘ನಾನು ದುರ್ಬಲ, ನಾನು ತ್ಯಾಗ ಮಾಡಲಾರೆ’ ಎಂದು ನೇರವಾಗಿ ಹೇಳಿಬಿಡಿ. ಅದನ್ನು ಬಿಟ್ಟು ಇಲ್ಲಸಲ್ಲದ ವಾದ ಹೂಡುತ್ತ, ಶಾಸ್ತ್ರಾರ್ಥವನ್ನು ತಿರಿಚುತ್ತ, ಅರಿಯದವರ ಕಣ್ಣಿಗೆ ಮಣ್ಣೆರಚುತ್ತ ಬೂಟಾಟಿಕೆ ಯಾಡಬೇಡಿ... ಏಕೆಂದರೆ ತ್ಯಾಗದ ಈ ಆದರ್ಶವು ಶ್ರೇಷ್ಠವಾದುದು, ಅತ್ಯುತ್ಕೃಷ್ಟವಾದುದು. ಆ ಪ್ರಯತ್ನದಲ್ಲಿ ಲಕ್ಷಾಂತರ ಜನ ಸೋತರೇನಂತೆ! ಹತ್ತು ಜನ ಅಥವಾ ಕೇವಲ ಇಬ್ಬರೇ ಸೈನಿ ಕರು ವಿಜಯಿಗಳಾಗಿ ಹಿಂದಿರುಗಿದರಾಯಿತಲ್ಲವೆ? ಅಲ್ಲದೆ, ಈ ಪ್ರಯತ್ನದಲ್ಲಿ ಮಡಿದ ಲಕ್ಷಾಂ ತರ ಜನರೂ ಧನ್ಯರೇ ಸರಿ! ಏಕೆಂದರೆ ಮಿಕ್ಕವರ ವಿಜಯಕ್ಕೆ ಕಾರಣವಾದುದು ಅವರ ಸಾವೇ ತಾನೆ?”

ಹೀಗೆ ಹಿಂದೂ ಧರ್ಮದ ಹಲವಾರು ಪ್ರಮುಖ ಅಂಶಗಳನ್ನು ಕ್ರೋಡೀಕರಿಸಿ ಸ್ವಾಮೀಜಿ ಅತ್ಯಂತ ವಿಶಿಷ್ಟವಾದ ಉಪನ್ಯಾಸವೊಂದನ್ನು ಮಾಡಿದರು. ಹಿಂದೂ ಧರ್ಮದ ಅತ್ಯುನ್ನತ ತತ್ತ್ವ ಗಳನ್ನು ತಮ್ಮ ವಿನೂತನ ದೃಷ್ಟಿಕೋನದಲ್ಲಿ, ಅತ್ಯಂತ ತರ್ಕಬದ್ಧವಾದ, ಶಕ್ತಿಯುತವಾದ ಶೈಲಿ ಯಲ್ಲಿ ವಿಶ್ಲೇಷಿಸಿ ವಿವರಿಸಿದ ರೀತಿಯು ಅತ್ಯಂತ ಪ್ರಭಾವಯುತವಾಗಿತ್ತು. “ಇಂಡಿಯನ್ ನೇಶನ್​” ಎಂಬ ಪತ್ರಿಕೆ ಹೀಗೆ ಬರೆಯಿತು: “ವೇದಾಂತದ ಕುರಿತಾದ ಈ ಉಪನ್ಯಾಸವು ಅತ್ಯಂತ ಸುಂದರವಾಗಿತ್ತು, ಶಕ್ತಿಶಾಲಿಯಾಗಿತ್ತು. ಬಹುಶಃ ಇದು ಅವರು ಇದುವರೆಗೆ ನೀಡಿರುವ ಎಲ್ಲ ಉಪನ್ಯಾಸಗಳಿಗಿಂತಲೂ ಉತ್ಕೃಷ್ಟವಾದದ್ದು...” ಅಲ್ಲಿ ನೆರೆದಿದ್ದ ಶ್ರೋತೃಗಳ ಮೇಲೆ ಮಾತ್ರ ವಲ್ಲದೆ ಇಡೀ ಬಂಗಾಳ ಪ್ರಾಂತ್ಯದಲ್ಲೇ ಈ ಉಪನ್ಯಾಸವು ಶಾಶ್ವತ ಸತ್ಪರಿಣಾಮವನ್ನುಂಟು ಮಾಡಿತು.

ಸ್ವಾಮೀಜಿಯವರು ಕಲ್ಕತ್ತದಲ್ಲಿದ್ದ ಈ ಅವಧಿಯಲ್ಲಿ ಅವರನ್ನು ಹಲವಾರು ಜನ ತಮ್ಮ ಮನೆಗಳಿಗೆ ಆಹ್ವಾನಿಸುತ್ತಿದ್ದರು. ಇವರಲ್ಲಿ ಕೆಲವರು ಅವರ ಹಳೆಯ ಪರಿಚಿತರು, ಶ್ರೀರಾಮ ಕೃಷ್ಣರ ಗೃಹೀಭಕ್ತರು. ಪ್ರಮುಖ ಗೃಹೀಭಕ್ತರಲ್ಲಿ ಒಬ್ಬರಾದ ದಿವಂಗತ ಬಲರಾಮ ಬೋಸರ ಮನೆಯಲ್ಲಿ ಸ್ವಾಮೀಜಿ ಒಮ್ಮೆ ಹಲವಾರು ದಿನಗಳವರೆಗೆ ಉಳಿದುಕೊಂಡರು. ಅವರು ಹೋದ ಲ್ಲೆಲ್ಲ ಸಂದರ್ಶಕರು ಅವರನ್ನು ಹಿಂಬಾಲಿಸುತ್ತಿದ್ದರು. ಈ ದಿನಗಳಲ್ಲಿ ಅವರನ್ನು ಭೇಟಿಯಾದ ವರ ಪೈಕಿ ಅನೇಕ ಯುವಕರು ಅವರ ಅಯಸ್ಕಾಂತೀಯ ವ್ಯಕ್ತಿತ್ವದಿಂದ ಸೆಳೆಯಲ್ಪಟ್ಟರು; ಅವರ ಸಂದೇಶಗಳನ್ನು ಕಾರ್ಯಗತಗೊಳಿಸಲು ಪಣತೊಟ್ಟರು. ಸುಧೀರ್ ಹಾಗೂ ಖಗೇನ್ ಈ ಯುವಕರ ಪೈಕಿ ಇಬ್ಬರು. ಮುಂದೆ ಇವರೇ ಸ್ವಾಮಿ ಶುದ್ಧಾನಂದ ಹಾಗೂ ಸ್ವಾಮಿ ವಿಮಲಾನಂದ.

ಸ್ವಾಮೀಜಿಯವರು ತಮ್ಮ ಬಳಿಗೆ ಬರುತ್ತಿದ್ದ ಪ್ರತಿಯೊಬ್ಬರನ್ನೂ ಆದರದಿಂದ ಮಾತನಾಡಿಸಿ, ಅವರ ಪ್ರಶ್ನೆಗಳಿಗೆಲ್ಲ ಸಹನೆಯಿಂದ ಉತ್ತರಿಸುತ್ತಿದ್ದುದೇನೋ ನಿಜ. ಆದರೆ ಅವರ ಹೃದಯದ ಒಲವು ಮಾತ್ರ ಯುವಕರ ಕಡೆಗೆ. ವಿದ್ಯಾವಂತರಾದ ಬುದ್ಧಿವಂತ ಯುವಕರನ್ನು ಕಂಡರೆ ಅವರಿಗೆ ಪರಮಾನಂದ. ಇಂತಹ ಯುವಕರೊಂದಿಗೆ ಮಾತನಾಡುವುದೆಂದರೆ ಅವರಿಗೆ ಸ್ವಲ್ಪವೂ ದಣಿವೇ ಆಗುತ್ತಿರಲಿಲ್ಲ. ತಮ್ಮ ಭಾವನಾತರಂಗಗಳನ್ನು, ತಮ್ಮ ಶಕ್ತ್ಯುತ್ಸಾಹಗಳನ್ನು ಯುವಜನರಿಗೆ ಧಾರೆ ಯೆರೆಯಲು ಅವರ ಹೃದಯ ಹಾತೊರೆಯುತ್ತಿತ್ತು. ಯುವಕರಲ್ಲಿ ಯಾರು ಪ್ರಾಮಾಣಿಕರೂ ಬಲಿಷ್ಠರೂ ಆದವರಿದ್ದಾರೋ ಅವರಿಗೆ ಹೆಚ್ಚಿನ ತರಬೇತಿ ನೀಡಲು ಬಯಸಿದರು. ಇಂತಹ ಯುವಕರು ಇತರ ಸಾಧಾರಣ ಜನಗಳಿಗಿಂತ ಮೇಲೇರಿ ಉನ್ನತ ಧ್ಯೇಯವೊಂದನ್ನು ಸಾಧಿಸಲು ಶ್ರಮಿಸುವಂತೆ ಮಾಡುವುದು ಅವರ ಉದ್ದೇಶವಾಗಿತ್ತು. ಈ ಯುವಕರು ಮುಂದೆ ತಮ್ಮ ವೈಯಕ್ತಿಕ ಮುಕ್ತಿಗಾಗಿ ಸಾಧನೆ ಮಾಡುವುದರೊಂದಿಗೆ ಇತರರ ಒಳಿತಿಗಾಗಿಯೂ ಶ್ರಮಿಸು ತ್ತಾರೆಂಬ ವಿಶ್ವಾಸ ಅವರಿಗಿತ್ತು. ಆದ್ದರಿಂದ ಸ್ವಾಮೀಜಿಯವರು ಇವರೊಂದಿಗೆ ಕೇವಲ ಆಧ್ಯಾತ್ಮಿಕ ವಿಚಾರವಾಗಿ ಮಾತ್ರವೇ ಮಾತನಾಡುತ್ತಿರಲಿಲ್ಲ; ಅವರಿಗೆ ವ್ಯಕ್ತಿನಿರ್ಮಾಣ ಹಾಗೂ ರಾಷ್ಟ್ರನಿರ್ಮಾಣಗಳ ಕುರಿತಾಗಿ ಸಲಹೆ ನೀಡುತ್ತಿದ್ದರು, ಮಾರ್ಗದರ್ಶನ ನೀಡುತ್ತಿದ್ದರು. ಸ್ವಾಮೀಜಿಯವರು ಯುವಕರಿಗೆ ಪ್ರಾಶಸ್ತ್ಯ ಕೊಡುತ್ತಿದ್ದ ಮಾತ್ರಕ್ಕೆ ಅವರನ್ನು ಕೇವಲ ಹೊಗಳು ತ್ತಿದ್ದರೆಂದಲ್ಲ. ಬದಲಾಗಿ ಯುವಕರ ಶಾರೀರಿಕ ದೌರ್ಬಲ್ಯವನ್ನು ಕಂಡು ಹಳಿಯುತ್ತಿದ್ದರು; ಚಿಕ್ಕ ವಯಸ್ಸಿನಲ್ಲೇ ಮದುವೆಯಾಗುವ ಪದ್ಧತಿಯನ್ನು ಕಟುವಾಗಿ ಟೀಕಿಸುತ್ತಿದ್ದರು; ಅವರಿಗೆ ತಮ್ಮ ಹಿಂದೂ ಸಂಸ್ಕೃತಿಯ ಮೇಲೆ ಶ್ರದ್ಧೆಯಿಲ್ಲದಿರುವುದನ್ನು ಕಂಡು ಟೀಕಿಸುತ್ತಿದ್ದರು. ಆದರೆ ಸ್ವಾಮೀಜಿಯವರು ಅಷ್ಟೆಲ್ಲ ಖಂಡಿಸಿ ಟೀಕಿಸಿದರೂ ಆ ಯುವಕರು ಅಸಮಾಧಾನಗೊಳ್ಳಲು ಸಾಧ್ಯವಿರಲಿಲ್ಲ. ಆ ಮಾತುಗಳ ಹಿಂದಿದ್ದ ಪ್ರೀತಿ-ಸಹಾನುಭೂತಿ-ಕಳಕಳಿಗಳನ್ನು ಅವರು ಸ್ಪಷ್ಟ ವಾಗಿ ಕಾಣುತ್ತಿದ್ದರು. ಆದ್ದರಿಂದ ಅವರೆಲ್ಲ ಸ್ವಾಮೀಜಿಯವರ ನಿಷ್ಠಾವಂತ ಶಿಷ್ಯರಾದರು.

ಫೆಬ್ರುವರಿಯ ಎರಡನೇ ವಾರದಲ್ಲಿ ಸ್ವಾಮೀಜಿಯವರು ಪ್ರಿಯನಾಥ ಮುಖರ್ಜಿ ಎಂಬವರ ಮನೆಗೆ ತೆರಳಿದರು. ಅವರು ಇಲ್ಲಿ ಇಳಿದುಕೊಂಡಿದ್ದಾಗ ‘ಇಂಡಿಯನ್ ಮಿರರ್​’ ಪತ್ರಿಕೆಯ ಸಂಪಾದಕರೂ ಸ್ವಾಮೀಜಿಯವರ ಅಭಿಮಾನಿಗಳೂ ಆದ ನರೇಂದ್ರನಾಥ ಸೇನರು ಒಮ್ಮೆ ಅವರ ಸಂದರ್ಶನಕ್ಕಾಗಿ ಬಂದರು. ಇವರಿಬ್ಬರ ಕಾರ್ಯರಂಗಗಳು ಮೂಲತಃ ವಿಭಿನ್ನವಾಗಿದ್ದರೂ ದೇಶ ಪ್ರೇಮವೆಂಬ ಸಾಮಾನ್ಯ ಅಂಶದಿಂದಾಗಿ ಇವರ ನಡುವಣ ಮಾತುಕತೆ ತುಂಬ ಸೌಹಾರ್ದಯುತ ವಾಗಿತ್ತು. ಪೂರ್ವಪಶ್ಚಿಮಗಳು ತಮ್ಮ ಉತ್ತಮ ಅಂಶಗಳನ್ನು ಪರಸ್ಪರ ವಿನಿಮಯಿಸಿಕೊಳ್ಳು ವುದರ ಮೂಲಕ ಸಕಲರಿಗೂ ಶ್ರೇಯಸ್ಸೆಂಬ ತಮ್ಮ ಅಭಿಪ್ರಾಯವನ್ನು ಸ್ವಾಮೀಜಿ ಪುನರುಚ್ಚರಿಸಿದರು.

ಸ್ವಾಮೀಜಿಯವರು ಪ್ರಿಯನಾಥ ಮುಖರ್ಜಿಗಳ ಮನೆಯಲ್ಲಿ ಇಳಿದುಕೊಂಡಿದ್ದ ಸಮಯ ದಲ್ಲಿ ಅವರನ್ನು ಭೇಟಿಯಾಗಲು ಹಲವಾರು ಬಗೆಯ ಜನ ಬರುತ್ತಿದ್ದರು. ಇವರಲ್ಲಿ ಶರಚ್ಚಂದ್ರ ಚಕ್ರವರ್ತಿ ಎಂಬ ಗೃಹಸ್ಥ ಯುವಕನೂ ಒಬ್ಬ. ಈತ ಪೂರ್ವ ಬಂಗಾಳ (ಈಗಿನ ಬಂಗ್ಲಾದೇಶ) ದವನು. ಶ್ರೀರಾಮಕೃಷ್ಣರ ಶ್ರೇಷ್ಠ ಗೃಹೀಭಕ್ತರಾದ ನಾಗಮಹಾಶಯರು ಇದ್ದದ್ದೂ ಪೂರ್ವ ಬಂಗಾಳದಲ್ಲಿಯೇ. ಶರಚ್ಚಂದ್ರನಿಗೆ ನಾಗಮಹಾಶಯರ ಪರಿಚಯ ಚೆನ್ನಾಗಿ ಇತ್ತು. ಅವನು ಸ್ವಾಮೀಜಿಯವರಿಗೆ ನಮಸ್ಕರಿಸಿ ತನ್ನ ಪರಿಚಯ ಹೇಳಿಕೊಂಡ. ಅವನು ನಾಗಮಹಾಶಯರನ್ನು ಬಲ್ಲವನೆಂದು ತಿಳಿಯುತ್ತಲೇ ಸ್ವಾಮೀಜಿ ಅವನೊಂದಿಗೆ ಸಂಸ್ಕೃತದಲ್ಲಿ ಮಾತನಾಡುತ್ತ ನಾಗ ಮಹಾಶಯರ ಯೋಗಕ್ಷೇಮವನ್ನು ವಿಚಾರಿಸಿದರು. ಶರಚ್ಚಂದ್ರ ಒಬ್ಬ ನಿಷ್ಠಾವಂತ ವೇದಾಂತಿ, ಬುದ್ಧಿವಂತ. ಬಹುಶಃ ಪ್ರಥಮ ನೋಟದಲ್ಲಿಯೇ ಸ್ವಾಮೀಜಿಯವರು ಶರಚ್ಚಂದ್ರನ ಗುಣ- ಸ್ವಭಾವಗಳನ್ನು ಅಳೆದುಬಿಟ್ಟರು. ಅವನೊಂದಿಗೆ ಮಾತನಾಡುತ್ತ ಸ್ವಾಮೀಜಿ ವಿವೇಕಚೂಡಾ ಮಣಿಯ ಒಂದು ಶ್ಲೋಕವನ್ನು ಹೇಳಿದರು. ಅದರ ಅರ್ಥ ಹೀಗಿದೆ: “ಓ ವಿವೇಕಿಯೇ, ಅಂಜ ಬೇಡ; ನಿನಗಿಲ್ಲ ಮರಣ, ಈ ಜನನ-ಮರಣದ ಸಾಗರವನ್ನು ದಾಟಲು ಉಪಾಯಗಳಿವೆ. ಹಲ ವಾರು ಪರಿಶುದ್ಧಾತ್ಮರು ಆ ಉಪಾಯಗಳನ್ನನುಸರಿಸಿ ಪಾರಾಗಿದ್ದಾರೆ. ಅದನ್ನು ನಾನು ನಿನಗೂ ತಿಳಿಸಿಕೊಡುತ್ತೇನೆ.”

ಕಾಲಕ್ರಮದಲ್ಲಿ ಈ ಶರಚ್ಚಂದ್ರನು ಸ್ವಾಮೀಜಿಯವರ ಪ್ರಿಯ ಶಿಷ್ಯನಾದ. ಅಥವಾ, ಮೊದಲ ದಿನದಿಂದಲೇ ಆತ ಅವರ ಶಿಷ್ಯನಾದನೆಂದೂ ಹೇಳಬಹುದು. ಏಕೆಂದರೆ, ಅಂದಿನಿಂದಲೇ ಅವನು ಸ್ವಾಮೀಜಿ, ಸಂಭಾಷಣೆಯ ಸಂದರ್ಭಗಳಲ್ಲಿ ಆಡಿದ ಮಾತುಗಳನ್ನೆಲ್ಲ ವಿವರವಾಗಿ ಬರೆದಿಟ್ಟು ಕೊಳ್ಳತೊಡಗಿದ. ಹೀಗಾಗಿ ಸ್ವಾಮೀಜಿಯವರ ವ್ಯಕ್ತಿತ್ವದ ಬಗ್ಗೆ, ಬೋಧನೆಗಳ ಬಗ್ಗೆ, ಅವರ ಜೀವನದ ಅನೇಕ ವಿಶೇಷ ಘಟನೆಗಳ ಬಗ್ಗೆ ತಿಳಿಸುವಂತಹ ಹಲವಾರು ಅಮೂಲ್ಯ ವಿಷಯಗಳು ನಮಗಿಂದು ಲಭ್ಯವಾಗಿವೆ. ಶರಚ್ಚಂದ್ರನ ಈ ಟಿಪ್ಪಣಿಗಳು ಮುಂದೆ “ಸ್ವಾಮಿ-ಶಿಷ್ಯ ಸಂವಾದ” ಎಂಬ ಹೆಸರಿನಲ್ಲಿ ಬಂಗಾಳಿಯಲ್ಲಿ ಪ್ರಕಟವಾದುವು. ಇವುಗಳನ್ನು ಈಗ ವಿವೇಕಾನಂದರ ಕೃತಿ ಶ್ರೇಣಿಯಲ್ಲೂ ಕಾಣಬಹುದು. ಶ್ರೀರಾಮಕೃಷ್ಣ-ವಿವೇಕಾನಂದ ಸಾಹಿತ್ಯಕ್ಕೆ ಇದೊಂದು ಅತ್ಯ ಮೂಲ್ಯ ಕೊಡುಗೆ.

ಸ್ವಾಮೀಜಿಯವರ ದರ್ಶನಕ್ಕಾಗಿ ಬರುತ್ತಿದ್ದವರಲ್ಲಿ ಬಗೆಬಗೆಯ ಜನರಿದ್ದರು. ಅದಕ್ಕೆ ತಕ್ಕಂತೆ, ಅವರೊಂದಿಗೆ ಮಾತನಾಡುವಾಗ ಸ್ವಾಮೀಜಿಯವರ ಭಾವವೂ ಬದಲಾಗುತ್ತಿತ್ತು. ಪ್ರಶ್ನೆಗಳಿಗೆ ತಕ್ಕಂತೆ, ಪ್ರಾಶ್ನಿಕರ ಉದ್ದೇಶಕ್ಕೆ ತಕ್ಕಂತೆ ಇರುತ್ತಿತ್ತು ಸ್ವಾಮೀಜಿಯವರ ಉತ್ತರ. ಶರಚ್ಚಂದ್ರನು ಸ್ವಾಮೀಜಿಯವರನ್ನು ಭೇಟಿಯಾಗಲು ಬಂದ ದಿನವೇ ‘ಗೋ ಸಂರಕ್ಷಣಾ ಸಮಿತಿ’ಯ ಪ್ರಚಾರಕ ನೊಬ್ಬ ಅವರ ದರ್ಶನಕ್ಕಾಗಿ ಬಂದಿದ್ದ. ಈತನೊಂದಿಗೆ ಸ್ವಾಮೀಜಿ ನಡೆಸಿದ ಸಂಭಾಷಣೆ ತುಂಬ ಕುತೂಹಲಕರವಾಗಿದೆ. ಸ್ವಾಮೀಜಿಯವರು ವ್ಯಕ್ತಪಡಿಸುವ ಭಾವನೆಗಳು ಅತ್ಯಂತ ಉದಾತ್ತ ವಾದವು; ಅವರ ಅಂತಃಕರಣದ ಆಳವನ್ನು ತೋರಿಸುವಂಥವು ಆ ಸಂಭಾಷಣೆ ಹೀಗಿತ್ತು:

ಸ್ವಾಮೀಜಿ: “ನಿಮ್ಮ ಸಂಘದ ಉದ್ದಿಶ್ಯವೇನು?”

ಪ್ರಚಾರಕ: “ನಾವು ಕಟುಕರ ಕೈಗೆ ಸೇರುವಂತಹ ಗೋವುಗಳನ್ನೆಲ್ಲ ತಡೆದು ರಕ್ಷಿಸುತ್ತೇವೆ. ಅಲ್ಲಲ್ಲಿ ಗೋಮಾಳಗಳನ್ನು ಸ್ಥಾಪಿಸಿದ್ದೇವೆ. ಕಾಯಿಲೆಯಾದ ಮತ್ತು ಮುದಿಯಾದ ಗೋಮಾತೆ ಯರನ್ನೂ ಅಲ್ಲಿಗೆ ಸೇರಿಸಿ ಸಂರಕ್ಷಿಸುತ್ತೇವೆ.”

ಸ್ವಾಮೀಜಿ: “ಹಾಗೇನು? ಬಹಳ ಒಳ್ಳೆಯ ಕೆಲಸವನ್ನೇ ಮಾಡುತ್ತಿದ್ದೀರಿ.. ನಿಮ್ಮ ಖರ್ಚು ವೆಚ್ಚಕ್ಕೆ ಏನು ವ್ಯವಸ್ಥೆ ಮಾಡಿಕೊಂಡಿದ್ದೀರಿ?”

ಪ್ರಚಾರಕ: “ಏನೋ ನಿಮ್ಮಂಥ ದೊಡ್ಡ ವ್ಯಕ್ತಿಗಳ ಉದಾರ ಸಹಾಯದಿಂದ ನಮ್ಮ ಸೇವಾ ಕಾರ್ಯ ಹೇಗೋ ನಡೆದುಕೊಂಡು ಹೋಗುತ್ತಿದೆ, ಸ್ವಾಮೀಜಿ.”

ಸ್ವಾಮೀಜಿ: “ಇದುವರೆಗೆ ಎಷ್ಟು ಹಣ ಸಂಗ್ರಹವಾಗಿದೆ?”

ಪ್ರಚಾರಕ: “ಆಗಿದೆ... ಮಾರ್ವಾಡಿಗಳೇ ನಮ್ಮ ಸಂಘದ ಮುಖ್ಯ ಆಧಾರ. ಅವರು ಈಗಾಗಲೇ ಬಹಳಷ್ಟು ಹಣವನ್ನು ನೀಡಿದ್ದಾರೆ... ”

ಸ್ವಾಮೀಜಿ: “ಇರಲಿ ಬಹಳ ಒಳ್ಳೇದು... ಅಂದಹಾಗೆ, ನಿಮಗೆ ಗೊತ್ತಿರಬೇಕಲ್ಲವೇ, ಈಗ ಮಧ್ಯಭಾರತದಲ್ಲಿ ಭಯಂಕರ ಕ್ಷಾಮ ಉಂಟಾಗಿದೆ. ಭಾರತ ಸರ್ಕಾರ ಪ್ರಕಟಿಸಿರುವಂತೆ ಈಗಾಗಲೇ ಒಂಬತ್ತು ಲಕ್ಷ ಜನ ಹಸಿವೆಯಿಂದ ಸತ್ತಿದ್ದಾರೆ. ಈ ಕ್ಷಾಮಸಂತ್ರಸ್ತರಿಗಾಗಿ ನಿಮ್ಮ ಸಮಿತಿ ಏನಾದರೂ ಮಾಡಿದೆಯೇನು?”

ಪ್ರಚಾರಕ: “ಸ್ವಾಮೀಜಿ, ನಮ್ಮ ಸಂಸ್ಥೆ ಸ್ಥಾಪನೆಯಾಗಿರುವುದು ಗೋಮಾತೆಯರನ್ನು ರಕ್ಷಿಸು ವುದಕ್ಕಾಗಿ ಮಾತ್ರ. ಈ ಕ್ಷಾಮಗೀಮದಲ್ಲಿ ನರಳುವವರಿಗೆಲ್ಲ ನಾವು ಏನೂ ಮಾಡಲಾಗುವುದಿಲ್ಲ.”

ಸ್ವಾಮೀಜಿಯವರ ಮುಖದಲ್ಲಿ ಅಚ್ಚರಿಯ ಗೆರೆಗಳು ಮೂಡಿದುವು.

ಸ್ವಾಮೀಜಿ: “ಏನು! ನಿಮ್ಮ ಸೋದರ ಸೋದರಿಯರು ಹೊಟ್ಟೆಗಿಲ್ಲದೆ ಲಕ್ಷಗಟ್ಟಲೆ ಸಂಖ್ಯೆ ಯಲ್ಲಿ ಸಾಯುತ್ತಿರುವಾಗ, ನಿಮ್ಮ ಕೈಯಲ್ಲಿ ಹಣವೂ ಇರುವಾಗ, ಅವರಿಗೆಲ್ಲ ನೆರವಾಗಬೇಕಾ ದದ್ದು ನಿಮ್ಮ ಕರ್ತವ್ಯ ಎಂದು ನಿಮಗೆ ಅನ್ನಿಸಲೇ ಇಲ್ಲವೆ?”

ಪ್ರಚಾರಕ: “ಹಾಗಲ್ಲ ಸ್ವಾಮೀಜಿ, ಬರಗಾಲ ಬಂದಿರುವುದು ಜನಗಳ ಕರ್ಮ, ಅವರವರ ಪಾಪದ ಫಲ ತಾನೆ? ಕರ್ಮದಂತೆ ಫಲ ಅಲ್ಲವೆ?–ಮಾಡಿದ್ದುಣ್ಣೋ ಮಹಾರಾಯ!”

ಈ ಮಾತುಗಳನ್ನು ಕೇಳಿದಾಗ ಸ್ವಾಮೀಜಿಯವರ ವಿಶಾಲವಾದ ಕಣ್ಣುಗಳು ಬೆಂಕಿಯುಂಡೆ ಗಳಾಗಿ ಕಿಡಿ ಕಾರಿದುವು. ಅವರ ಮುಖ ಕೋಪದಿಂದ ಕೆಂಪಾಯಿತು. ಆದರೆ ಅವರು ಉಕ್ಕಿಬರು ತ್ತಿದ್ದ ತಮ್ಮ ಭಾವೋದ್ರೇಕವನ್ನು ಬಲವಂತವಾಗಿ ತಡೆದುಕೊಳ್ಳುತ್ತ ಉದ್ಗರಿಸಿದರು:

“ಯಾವ ಸಂಘಸಂಸ್ಥೆಗಳು ತಮ್ಮ ಸಹಮಾನವರ ಬಗ್ಗೆ ಸಹಾನುಭೂತಿ ತೋರುವುದಿಲ್ಲವೋ, ತಮ್ಮ ಸೋದರರ ಸಾವನ್ನು ಕಂಡೂ ಮರುಗುವುದಿಲ್ಲವೋ, ಅವರ ಜೀವನವನ್ನು ಕಾಪಾಡಲು ಒಂದು ಹಿಡಿ ಅನ್ನವನ್ನೂ ಕೊಡದೆ ಪ್ರಾಣಿ ಪಕ್ಷಿಗಳ ಮುಂದೆ ರಾಶಿ ರಾಶಿ ಆಹಾರವನ್ನು ಒಡ್ಡುತ್ತವೆಯೋ ಅಂಥವುಗಳ ಬಗ್ಗೆ ನನಗೆ ಒಂದಿಷ್ಟೂ ಆದರವಿಲ್ಲ. ಮತ್ತು ಅಂತಹ ಸಂಸ್ಥೆ ಗಳಿಂದ ಸಮಾಜಕ್ಕೆ ಏನಾದರೂ ಪ್ರಯೋಜನವಾದೀತೆಂದೂ ನನಗನ್ನಿಸುವುದಿಲ್ಲ... ಮನುಷ್ಯರು ಸಾಯುವುದು ಅವರವರ ಕರ್ಮಕ್ಕನುಸಾರವಾಗಿ ಎಂದು ಈ ಕರ್ಮಸಿದ್ಧಾಂತವನ್ನು ತರುವುದಾ ದರೆ, ನಾವು ಈ ಜಗತ್ತಿನಲ್ಲಿ ಮಾಡುವ ಯಾವುದೇ ಪ್ರಯತ್ನವೂ ವ್ಯರ್ಥವೆಂದ ಹಾಗಾಯಿತು! ನಿಮ್ಮ ಗೋರಕ್ಷಣಾ ಕಾರ್ಯವೂ ಇದಕ್ಕೆ ಹೊರತೇನಲ್ಲವಲ್ಲ? ಬರಗಾಲದಿಂದ ಸಾಯುವುದು ಮನುಷ್ಯರ ಕರ್ಮವಾದರೆ ಕಟುಕರ ಕೈಯಲ್ಲಿ ಸಾಯುವುದು ನಿಮ್ಮ ಗೋಮಾತೆಯರ ಕರ್ಮವಲ್ಲವೆ? ಆದ್ದರಿಂದ ನಾವೇನೂ ಅವುಗಳ ರಕ್ಷಣೆಮಾಡಲು ಹೋಗಬೇಕಾಗಿಲ್ಲವಲ್ಲ?”

ಇದನ್ನು ಕೇಳಿ ಆ ಪ್ರಚಾರಕ ತಬ್ಬಿಬ್ಬಾದ. ಆದರೂ ತಡವರಿಸುತ್ತ ಹೇಳಿದ:

“ನೀವೆನ್ನುವುದೇನೋ ನಿಜವೇ. ಆದರೆ ಗೋವು ನಮ್ಮ ಮಾತೆಯೆಂದು ಶಾಸ್ತ್ರಗಳೇ ಹೇಳುತ್ತವೆ ಯಲ್ಲ?”

ಸ್ವಾಮೀಜಿ ಮುಗುಳ್ನಗುತ್ತ ವ್ಯಂಗ್ಯವಾಗಿ ನುಡಿದರು:

“ನಿಜ, ನಿಜ; ಗೋವು ನಿಮ್ಮ ಮಾತೆಯೇ ನಿಜ. ಇಲ್ಲದೆ ಹೋದರೆ, ನಿಮ್ಮಂತಹ ಪುತ್ರರತ್ನರಿಗೆ ಮತ್ತಾರು ಜನ್ಮವಿತ್ತಾರು?”

ಮತ್ತೆ ಆ ಪ್ರಚಾರಕ ಆ ವಿಷಯವಾಗಿ ಮಾತನಾಡಲಿಲ್ಲ. ಸ್ವಾಮೀಜಿಯವರ ಮಾತು ಅವನಿಗೆ ಅರ್ಥವಾದಂತೆ ಕಾಣಲಿಲ್ಲ. ಆದರೆ ಅವನು ಕೇವಲ ಅವರ ದರ್ಶನಕ್ಕಾಗಿ ಬಂದವನಲ್ಲ; ಅವನಿಗೆ ಬೇರೊಂದು ಮುಖ್ಯವಾದ ಉದ್ದೇಶವಿತ್ತು. ಈಗ ಆತ ಅದರ ಬಗ್ಗೆ ಮಾತನಾಡಿದ:

“ಸ್ವಾಮೀಜಿ, ನಾನೀಗ ನಮ್ಮ ಸಂಘಕ್ಕಾಗಿ ನಿಧಿ ಸಂಗ್ರಹಣೆ ಮಾಡಲು ಬಂದೆ. ತಾವು ದೊಡ್ಡ ಮನಸ್ಸು ಮಾಡಿ ಏನಾದರೂ ಸ್ವಲ್ಪ ಕೊಡಬೇಕು.. ”

ಸ್ವಾಮೀಜಿ ನಿಸ್ಸಂಕೋಚವಾಗಿ ಸತ್ಯವನ್ನೇ ಹೇಳಿದರು:

“ನೋಡಿ, ನಾನೊಬ್ಬ ಸಂನ್ಯಾಸಿ, ಬಡ ಫಕೀರ. ನಿಮ್ಮ ಸಂಸ್ಥೆಗೆ ನೀಡಲು ನನಗೆಲ್ಲಿಂದ ಬರ ಬೇಕು ಹಣ?... ಆದರೆ, ಒಂದು ವೇಳೆ ನನ್ನ ಕೈಗೇನಾದರೂ ಹಣ ಬಂದರೂ ಕೂಡ ನಾನದನ್ನು ಮೊದಲು ನನ್ನ ಸಹಮಾನವರ ಸೇವೆಗಾಗಿ ಬಳಸುತ್ತೇನೆ. ಮೊದಲು ಬದುಕಿಸಬೇಕಾದದ್ದು ಮನುಷ್ಯನನ್ನು. ಅವನಿಗೆ ಮೊದಲು ಅನ್ನ ಕೊಡಬೇಕು, ವಿದ್ಯೆ ಕೊಡಬೇಕು, ಆಧ್ಯಾತ್ಮಿಕತೆಯನ್ನು ಕೊಡಬೇಕು. ಇವುಗಳಿಗೆಲ್ಲ ಸಾಕಾಗಿ ಇನ್ನೂ ಹಣ ಮಿಕ್ಕರೆ ಆಗ ಮಾತ್ರ ನಿಮ್ಮ ಸಂಸ್ಥೆಗೆ ನನ್ನಿಂದೇನಾದರೂ ದೊರೆತೀತು.”

ಅಮೆರಿಕದಿಂದ ಈಗ ತಾನೆ ಹಿಂದಿರುಗಿರುವ ಈ ಸ್ವಾಮಿಗಳ ಹತ್ತಿರ ಸಾಕಷ್ಟು ಹಣವಿರಲೇ ಬೇಕು, ಅವು ಖರ್ಚಾಗಿಹೋಗುವ ಮೊದಲೇ ಒಂದಷ್ಟು ಪಡೆದುಕೊಂಡು ಹೋಗೋಣ ಎಂದು ಆ ಪ್ರಚಾರಕ ಹಂಚಿಕೆ ಹಾಕಿದ್ದ. ಆದರೆ ಸ್ವಾಮೀಜಿ ಹೀಗೆ ನಿಷ್ಠುರವಾಗಿ ಹೇಳಿಬಿಟ್ಟಮೇಲೆ ಅಲ್ಲಿ ಅವನಿಗಿನ್ನೇನು ಕೆಲಸ? ಅವರಿಗೆ ನಮಸ್ಕಾರ ಮಾಡಿ ಅಲ್ಲಿಂದ ಹೊರಟ.

ಆ ಪ್ರಚಾರಕ ಹೊರಟು ಹೋದ ಮೇಲೆ ಸ್ವಾಮೀಜಿ ವಿಷಾದದಿಂದ ಉದ್ಗರಿಸಿದರು: “ಆಹಾ, ಎಂಥಾ ಮಾತು! ಮನುಷ್ಯರು ಅನ್ನವಿಲ್ಲದೆ ಸಾಯುವುದು ಅವರ ಕರ್ಮವಂತೆ! ಆದ್ದರಿಂದ ಅವರ ಮೇಲೆ ಕನಿಕರ ತೋರಿ ಪ್ರಯೋಜನವೇನು ಎನ್ನುತ್ತಾನೆ ಆ ಮಹಾನುಭಾವ! ನಮ್ಮ ದೇಶ ಸಂಪೂರ್ಣ ಹಾಳಾಗಿ ಹೋಗಿದೆ ಎನ್ನುವುದಕ್ಕೆ ಇದಕ್ಕಿಂತ ಹೆಚ್ಚಿನ ಸಾಕ್ಷ್ಯ ಬೇಕೆ? ನಿಮ್ಮ ಹಿಂದೂಧರ್ಮದ ಕರ್ಮಸಿದ್ಧಾಂತ ಎನ್ನುವುದು ಎಲ್ಲಿಗೆ ಬಂದು ನಿಂತಿದೆ ನೋಡಿದಿರಾ? ಯಾರಿಗೆ ತಮ್ಮ ಸಹಮಾನವರ ಮೇಲೆ ಸಹಾನುಭೂತಿಯಿಲ್ಲವೋ, ಅಂಥವರನ್ನು ಮನುಷ್ಯರು ಎಂದು ಕರೆಯಬಹುದೆ?” ಈ ಮಾತುಗಳನ್ನಾಡುವಾಗ ಸ್ವಾಮೀಜಿಯವರ ಶರೀರ ಆಕ್ರೋಶದಿಂದ ನಖಶಿಖಾಂತ ಕಂಪಿಸುತ್ತಿತ್ತು.

ಮತ್ತೊಂದು ಸಂದರ್ಭ; ಯುವಕನೊಬ್ಬ ಸ್ವಾಮೀಜಿಯವರ ಮಾರ್ಗದರ್ಶನವನ್ನರಸಿ ಬಂದಿದ್ದಾನೆ. ಆತ ಹೇಳಿದ:

“ಸ್ವಾಮೀಜಿ, ನಾನು ಹಲವಾರು ಧಾರ್ಮಿಕ ಸಂಘ-ಸಂಸ್ಥೆಗಳಿಗೆ ಹೋಗುತ್ತೇನೆ; ಅವರ ತತ್ತ್ವ ಗಳನ್ನೆಲ್ಲ ಕೇಳುತ್ತೇನೆ. ಆದರೆ ಅವುಗಳ ಪೈಕಿ ಯಾವುದು ಸರಿ, ಯಾವುದು ಸತ್ಯ, ನಾನು ಯಾವು ದನ್ನು ಅನುಸರಿಸಬೇಕು ಎಂಬುದೇ ತಿಳಿಯುತ್ತಿಲ್ಲ. ನಾನು ತುಂಬ ಗೊಂದಲಕ್ಕೆ ಸಿಕ್ಕಿಕೊಂಡಿ ದ್ದೇನೆ.”

ಸ್ವಾಮೀಜಿ ತುಂಬ ಸಹಾನುಭೂತಿಯಿಂದ, ವಿಶ್ವಾಸಯುತ ದನಿಯಲ್ಲಿ ನುಡಿದರು: “ಮಗು, ಇದಕ್ಕಾಗಿ ನೀನೇನೂ ಕಳವಳಗೊಳ್ಳಬೇಕಾಗಿಲ್ಲ. ನಾನೂ ಒಂದು ಕಾಲದಲ್ಲಿ ಇಂಥ ತೊಳಲಾಟ ವನ್ನೆಲ್ಲ ಅನುಭವಿಸಿದ್ದೇನೆ. ಹೋಗಲಿ, ನಿನಗೆ ಯಾರ್ಯಾರು ಏನೇನು ಹೇಳಿಕೊಟ್ಟರು, ನೀನು ಏನೇನು ಮಾಡಿದೆ ಎಂಬುದನ್ನೆಲ್ಲ ಹೇಳು ನೋಡೋಣ?”

ಯುವಕ: “ಥಿಯೊಸಾಫಿಕಲ್ ಸೊಸೈಟಿಯ ಒಬ್ಬರು ವಿದ್ವಾಂಸರು ನನಗೆ ಮೂರ್ತಿಪೂಜೆಯ ಮಹತ್ವವನ್ನು ಚೆನ್ನಾಗಿ ಮನವರಿಕೆ ಮಾಡಿಸಿಕೊಟ್ಟರು. ಅದರಂತೆಯೇ ನಾನು ಪೂಜೆ-ಜಪಗಳನ್ನು ತುಂಬ ಭಕ್ತಿಯಿಂದ ಬಹಳ ಕಾಲ ಮಾಡಿದೆ. ಆದರೆ ಅದರಿಂದ ಮನಶ್ಶಾಂತಿ ದೊರಕಲಿಲ್ಲ. ಆಮೇಲೆ ನನಗೆ ಇನ್ನೊಬ್ಬರು ಹೇಳಿದರು–ಧ್ಯಾನಕಾಲದಲ್ಲಿ ಮನಸ್ಸನ್ನು ಸಂಪೂರ್ಣ ಶೂನ್ಯ ಗೊಳಿಸು ಎಂದು. ಆದರೆ ನಾನು ಎಷ್ಟೆಷ್ಟು ಪ್ರಯತ್ನ ಪಟ್ಟರೂ ಅದು ಸಾಧ್ಯವಾಗಲಿಲ್ಲ... ಸ್ವಾಮೀಜಿ, ಈಗಲೂ ನಾನು ಪ್ರತಿದಿನ ಕೋಣೆಯ ಕಿಟಕಿ ಬಾಗಿಲುಗಳನ್ನೆಲ್ಲ ಮುಚ್ಚಿ, ನನ್ನ ಕೈಯಲ್ಲಿ ಎಷ್ಟು ಹೊತ್ತು ಸಾಧ್ಯವೋ ಅಷ್ಟು ಹೊತ್ತು ಧ್ಯಾನ ಮಾಡಲು ಪ್ರಯತ್ನ ಪಡುತ್ತಿದ್ದೇನೆ. ಆದರೆ ಮನಸ್ಸಿಗೆ ಸ್ವಲ್ಪವೂ ಶಾಂತಿ-ನೆಮ್ಮದಿ ದೊರಕುತ್ತಲೇ ಇಲ್ಲ.”

ಅವನು ಹೇಳುವುದನ್ನೆಲ್ಲ ಸಾವಧಾನದಿಂದ ಕೇಳಿದ ಸ್ವಾಮೀಜಿ, ತುಂಬ ಮೃದುವಾಗಿ ಅವನನ್ನುದ್ದೇಶಿಸಿ ನುಡಿದರು:

“ಮಗು, ನನ್ನ ಮಾತನ್ನು ಕೇಳುವುದಾದರೆ, ನೀನು ಮೊದಲು ನಿನ್ನ ಕೋಣೆಯ ಕಿಟಕಿ ಬಾಗಿಲು ಗಳನ್ನು ತೆರೆಯಬೇಕು; ಮತ್ತು ಕಣ್ಣು ಮುಚ್ಚಿ ಕುಳಿತುಕೊಳ್ಳುವುದರ ಬದಲಾಗಿ ಕಣ್ದೆರೆದು ಸುತ್ತಲೂ ನೋಡಬೇಕು! ಆಗ ನೀನು ನಿನ್ನ ಮನೆಯ ಸುತ್ತಲೂ ನೂರಾರು-ಸಾವಿರಾರು ಬಡವ ರನ್ನು, ಅಸಹಾಯಕರನ್ನು ಕಾಣಬಹುದು. ಅವರಿಗೆಲ್ಲ ನಿನ್ನ ಕೈಯಲ್ಲಿ ಎಷ್ಟು ಸಾಧ್ಯವೋ ಅಷ್ಟು ಸೇವೆ ಮಾಡು. ಎಷ್ಟೋ ಜನ ಕಾಯಿಲೆ ಬಿದ್ದಿರುತ್ತಾರೆ, ಆದರೆ ಅವರನ್ನು ನೋಡಿಕೊಳ್ಳಲು ಯಾರೂ ಇರುವುದಿಲ್ಲ. ಅಂಥವರನ್ನು ಹುಡುಕು. ಅವರಿಗೆ ಔಷಧ ತಂದುಕೊಡು, ಶುಶ್ರೂಷೆ ಮಾಡು. ಯಾರಿಗೆ ಹೊಟ್ಟೆಗಿಲ್ಲವೋ ಅಂಥವರಿಗೆ ಅನ್ನ ಕೊಡು. ನೀನು ವಿದ್ಯಾವಂತ. ಆದ್ದರಿಂದ, ಅವಿದ್ಯಾವಂತರಿಗೆ ವಿದ್ಯೆ ಕೊಡು. ನಿನಗೆ ಮನಶ್ಶಾಂತಿ ಬೇಕಿದ್ದರೆ ಅದಕ್ಕೆ ನನ್ನ ಸಲಹೆ ಇಷ್ಟೆ– ನೀನು ಈ ರೀತಿಯಾಗಿ ಕಷ್ಟದಲ್ಲಿರುವವರ ಸೇವೆ ಮಾಡಬೇಕು. ಎಷ್ಟು ಚೆನ್ನಾಗಿ ಮಾಡಿದರೆ ಅಷ್ಟು ಒಳ್ಳೆಯದು.”

ಏಕಾಂತದಲ್ಲಿ ಕುಳಿತು ಧ್ಯಾನ ಮಾಡಿದರೆ, ಪ್ರಾರ್ಥನೆ ಮಾಡಿದರೆ ಮನಶ್ಶಾಂತಿ ಸಿಗುತ್ತದೆ ಎಂದು ಎಲ್ಲರೂ ಹೇಳಿದರೆ, ಸ್ವಾಮೀಜಿಯವರು ಅದಕ್ಕೆ ತದ್ವಿರುದ್ಧವಾದುದನ್ನು ಹೇಳುತ್ತಿದ್ದಾರೆ! –ಮನಶ್ಶಾಂತಿ ಬೇಕಿದ್ದರೆ ದುಃಖಿಗಳ ಸೇವೆ ಮಾಡುವಂತೆ! ಇದು ಹೇಗೆ ಸಾಧ್ಯ? ಇಲ್ಲಿ ಸ್ವಾಮೀಜಿಯವರ ಸಮಸ್ಯೆಯ ಮೂಲೋತ್ಪಾಟನೆಯ ಮಾರ್ಗವನ್ನು ತಿಳಿಸಿಕೊಡುತ್ತಿದ್ದಾರೆ. ಅವನನ್ನು ನೋಡಿದ ಕೂಡಲೇ ಅವರಿಗೆ ಅವನ ರಜೋಗುಣದ ದರ್ಶನವಾಗಿದೆ. ರಜೋಗುಣದ ಸ್ವಭಾವವೇ ಮನಸ್ಸನ್ನು ಚಂಚಲಗೊಳಿಸುವುದು. ಈ ರಜೋಗುಣವನ್ನು ಕರಗಿಸಿ ಸತ್ವಗುಣ ವನ್ನಾಗಿ ಪರಿವರ್ತಿಸಬೇಕಾಗುತ್ತದೆ. ಅದಕ್ಕೆ ನಿಷ್ಕಾಮಕರ್ಮವೇ ಶ್ರೇಷ್ಠ ಉಪಾಯ. ದೀನ ದಲಿತ ದುಃಖಿಗಳಲ್ಲಿ ಭಗವಂತನನ್ನು ಕಂಡು ಸೇವೆಗೈಯುವುದರಿಂದ ನಮ್ಮ ಹೃದಯ ಶುದ್ಧವಾಗುತ್ತದೆ, ಮನಸ್ಸು ಸಾತ್ವಿಕಗೊಳುತ್ತದೆ. ಇಂತಹ ಸಾತ್ವಿಕ ಮನಸ್ಸಿನಿಂದ ಧ್ಯಾನಕಾರ್ಯ ಅತ್ಯಂತ ಸುಲಭ ವಾಗುತ್ತದೆ. ಆದರೆ ಆ ಯುವಕನಿಗೆ ಸ್ವಾಮೀಜಿಯವರ ಮಾತನ್ನು ಕೇಳುತ್ತಿದ್ದಂತೆಯೇ ಗಾಬರಿ ಯಾಯಿತು. ಅವನು ತನ್ನ ಆತಂಕವನ್ನು ಮುಂದಿಟ್ಟು, “ಸ್ವಾಮೀಜಿ, ನೀವು ಹೇಳಿದಂತೆ ನಾನೀಗ ಒಬ್ಬ ರೋಗಿಯ ಸೇವೆಗೆ ಹೋದೆನೆಂದೇ ಇಟ್ಟುಕೊಳ್ಳಿ. ಆಗ ನನಗೆ ಹೊತ್ತುಹೊತ್ತಿಗೆ ಸರಿಯಾಗಿ ಊಟ ನಿದ್ರೆಯಿಲ್ಲದಂತಾಗಿ ನಾನೇ ಕಾಯಿಲೆ ಬಿದ್ದುಬಿಟ್ಟರೆ... ” ಅವನು ತನ್ನ ಮಾತನ್ನು ಮುಗಿಸುವಷ್ಟರಲ್ಲೇ ಸ್ವಾಮೀಜಿ ತೀಕ್ಷ್ಣವಾಗಿ ಹೇಳಿದರು, “ಸರಿ, ಸರಿ. ಅರ್ಥವಾಯಿತು ಬಿಡು. ನಿನ್ನಂತಹ ಐಷಾರಾಮಿಗಳು, ಮೈಸುಖದಲ್ಲೇ ಮನಸ್ಸಿರುವವರು ರೋಗಿಗಳ ಸೇವೆ ಮಾಡಲು ಹೋಗುವುದುಂಟೆ! ಹಾಗೆ ಮಾಡಹೋಗಿ, ನೀನೆಂದಂತೆ ಅಪಾಯಕ್ಕೆ ಗುರಿಯಾದರೆ!”

ಒಂದು ದಿನ ಮಾಸ್ಟರ್ ಮಹಾಶಯರು (ಮಹೇಂದ್ರನಾಥ ಗುಪ್ತ–‘ಶ್ರೀರಾಮಕೃಷ್ಣ ವಚನ ವೇದ’ದ ಕರ್ತೃ) ಸ್ವಾಮೀಜಿಯವರ ದರ್ಶನಕ್ಕಾಗಿ ಬಂದಿದ್ದರು. ಸ್ವಾಮೀಜಿಯವರ ‘ಲೋಕಹಿತ, ಧರ್ಮಪ್ರಚಾರ’ ಇತ್ಯಾದಿ ಬೋಧನೆಗಳೆಲ್ಲ ಇವರಿಗೆ ಅರ್ಥವಾಗಲೊಲ್ಲವು. ಸಾಧನೆಯ ಮೂಲಕ ಆತ್ಮಸಾಕ್ಷಾತ್ಕಾರ ಮಾಡಿಕೊಳ್ಳುವುದೊಂದನ್ನೇ ಶ್ರೀರಾಮಕೃಷ್ಣರು ಬೋಧಿಸಿದುದು ಎಂಬುದು ಮಾಸ್ಟರ್ ಮಹಾಶಯರ ದೃಢ ನಂಬಿಕೆ. ಸಂಭಾಷಣೆಯ ಸಂದರ್ಭದಲ್ಲಿ ಅವರು ಕೇಳಿದರು, “ಸ್ವಾಮೀಜಿ, ನೀವು ಸೇವೆಯ ವಿಚಾರವಾಗಿ, ದಾನಧರ್ಮದ ವಿಚಾರವಾಗಿ, ಲೋಕಹಿತದ ವಿಚಾರವಾಗಿ ಅಷ್ಟೆಲ್ಲ ಹೇಳುತ್ತೀರಲ್ಲ; ಎಷ್ಟಾದರೂ ಅವೆಲ್ಲ ಈ ಮಾಯಾಪ್ರಪಂಚಕ್ಕೆ ಸೇರಿದ ವಿಚಾರಗಳಲ್ಲವೆ? ಈ ಮಾಯೆಯ ತೆರೆಯನ್ನು ಹರಿದು ಮುಕ್ತರಾಗುವುದೇ ಮಾನವ ಜೀವನದ ಧ್ಯೇಯ ಎಂದು ವೇದಾಂತ ಬೋಧಿಸುತ್ತದೆ. ಎಂದ ಮೇಲೆ ನೀವು ನಮ್ಮ ಮನಸ್ಸನ್ನು ಈ ಮಾಯಾಪ್ರಪಂಚದ ಕಡೆಗೆ ತಿರುಗಿಸುವ ಸೇವಾಕಾರ್ಯಗಳನ್ನು ಕುರಿತು ಬೋಧನೆ ಮಾಡುವು ದೇಕೆ?” ಸ್ವಾಮೀಜಿಯವರ ಉತ್ತರ ಮಿಂಚಿನಂತೆ ಚಿಮ್ಮಿತು: “ಈಗ ನೀವು ಹೇಳುತ್ತಿರುವ ಆ ಮುಕ್ತಿಯೂ ಈ ಮಾಯೆಯ ರಾಜ್ಯಕ್ಕೇ ಸೇರಿದ ವಿಷಯವಲ್ಲವೆ? ಈ ನಮ್ಮ ಆತ್ಮವು ನಿತ್ಯಮುಕ್ತ ವಾದುದೆಂದು ವೇದಾಂತವು ಬೋಧಿಸುವುದಿಲ್ಲವೆ? ಇನ್ನು ಮುಕ್ತಿಗಾಗಿ ಪ್ರಯತ್ನ ಮಾಡುವುದ ರಲ್ಲಿ ಅರ್ಥವೇನು!” ಈ ಮಾತಿನಿಂದ ಮಾಸ್ಟರ್ ಮಹಾಶಯರಿಗೆ ಎಷ್ಟರ ಮಟ್ಟಿಗೆ ಸಮಾಧಾನ ವಾಯಿತೊ ತಿಳಿಯದು. ಆದರೆ ಈ ಮಾತು ತುಂಬ ಅರ್ಥಪೂರ್ಣವಾಗಿದೆ, ತರ್ಕಬದ್ಧವಾಗಿದೆ. ಅಲ್ಲದೆ ಶ್ರೀರಾಮಕೃಷ್ಣರ ಬೋಧನೆಗಳ ಮರ್ಮವನ್ನು ಸಂಪೂರ್ಣವಾಗಿ ಗ್ರಹಿಸಲು ಅಸಮರ್ಥ ರಾದವರೆಲ್ಲರೂ ಮಾಸ್ಟರ್ ಮಹಾಶಯರಂತೆಯೇ ಭಾವಿಸಿದ್ದರು. ಅವರೆಲ್ಲರಿಗೂ ಸ್ವಾಮೀಜಿ ಈ ಮೂಲಕ ಉತ್ತರಿಸಿದರು.

ಮತ್ತೊಂದು ದಿನ ಸ್ವಾಮೀಜಿ ಸಂದರ್ಶಕರೊಂದಿಗೆ ಮಾತನಾಡುತ್ತಿದ್ದಾಗ ಶ್ರೀರಾಮಕೃಷ್ಣರ ಅಣ್ಣನ ಮಗನಾದ ರಾಮಲಾಲ್ ದಾದಾ ಅವರ ಭೇಟಿಗಾಗಿ ಬಂದ. ತಕ್ಷಣವೇ ಸ್ವಾಮೀಜಿ ಎದ್ದು ನಿಂತು ರಾಮಲಾಲನಿಗೆ ತಮ್ಮ ಕುರ್ಚಿಯನ್ನು ಬಿಟ್ಟು ಕೊಟ್ಟರು. ರಾಮಲಾಲನಿಗೆ ತುಂಬ ಸಂಕೋಚವಾಯಿತು. ಅಷ್ಟೊಂದು ಜನರ ಮುಂದೆ ಇಂತಹ ಜಗತ್ಪ್ರಸಿದ್ಧ ವಿವೇಕಾನಂದರು ತನಗೆ ಈ ಬಗೆಯ ಗೌರವವನ್ನು ತೋರುವುದೆ! ಅವನು ಎಷ್ಟು ಒಲ್ಲೆನೆಂದರೂ ಕೇಳದೆ ಸ್ವಾಮೀಜಿ ಅವನನ್ನು ಬಲವಂತವಾಗಿ ಕುರ್ಚಿಯ ಮೇಲೆ ಕುಳ್ಳಿರಿಸಿದರು. ಬಳಿಕ ತಾವು ಅತ್ತಿಂದಿತ್ತ ನಡೆ ದಾಡುತ್ತ “ಗುರುವತ್ ಗುರುಪುತ್ರೇಷು” ಎಂದು ಭಾವಪೂರ್ಣವಾಗಿ ಮತ್ತೆ ಮತ್ತೆ ಉದ್ಗರಿಸಿ ದರು. ಎಂದರೆ ‘ಗುರುವಿನ ಪುತ್ರನನ್ನು ಗುರುವಿನಂತೆಯೇ ಕಾಣಬೇಕು’ ಎಂದರ್ಥ. ರಾಮಲಾಲ ಶ್ರೀರಾಮಕೃಷ್ಣರ ಸ್ವಂತ ಅಣ್ಣನ ಮಗನಲ್ಲವೆ? ಗುರುಭಕ್ತಿಯನ್ನು, ವಿನಯವನ್ನು ಅಭ್ಯಾಸ ಮಾಡ ಬೇಕೆಂಬುವರಿಗೆ ಇದೊಂದು ಪಾಠ.

ಈಗ ಸಂಭಾಷಣೆ \eng{Imitation of Christ (‘}ಕ್ರಿಸ್ತನ ಅನುಸರಣೆ’) ಎಂಬ ಪುಸ್ತಕದ ಕಡೆಗೆ ತಿರುಗಿತು. ಇದು ಸ್ವಾಮೀಜಿಯವರು ಬಹಳವಾಗಿ ಮೆಚ್ಚಿಕೊಂಡ ಪುಸ್ತಕ. ಈ ಪುಸ್ತಕದಲ್ಲಿ ಬೋಧಿಸಲ್ಪಟ್ಟಿರುವ ದೈನ್ಯತೆಯ ಬಗ್ಗೆ ಒಬ್ಬರು ಪ್ರಸ್ತಾಪಿಸಿ, “ನಮ್ಮನ್ನು ನಾವು ದೀನರಲ್ಲಿ ದೀನರೆಂದು ಭಾವಿಸಿದ ಹೊರತು ಆಧ್ಯಾತ್ಮಿಕ ಪ್ರಗತಿ ಅಸಾಧ್ಯ” ಎಂದು ಉದ್ಗರಿಸಿದರು. ಆದರೆ ಸ್ವಾಮೀಜಿ ಈ ಅಭಿಪ್ರಾಯವನ್ನು ಮೆಚ್ಚದೆ ಈ ರೀತಿ ಹೇಳಿದರು, “ನಮ್ಮನ್ನು ನಾವು ಕಳಪೆ ಎಂದೇಕೆ ತಿಳಿದುಕೊಳ್ಳಬೇಕು? ನಮ್ಮನ್ನು ನಾವು ಏಕೆ ನಿಂದಿಸಿಕೊಳ್ಳಬೇಕು? ನಾವು ದಿವ್ಯ ಜ್ಯೋತಿಯ ಪುತ್ರರು. ನಮಗೆಲ್ಲಿದೆ ಕತ್ತಲು! ಯಾವ ಜ್ಯೋತಿ ಈ ವಿಶ್ವವನ್ನೇ ಬೆಳಗುತ್ತಿದೆಯೋ ಆ ದಿವ್ಯ ಜ್ಯೋತಿಯಲ್ಲೇ ಇರುವವರು ನಾವು; ಆ ದಿವ್ಯ ಜ್ಯೋತಿಯ ಒಡನಾಡಿಗಳು; ಕೊನೆಗೆ ಆ ದಿವ್ಯಜ್ಯೋತಿಯನ್ನೇ ಸೇರಿಕೊಳ್ಳುವವರು!”

ಸ್ವಾಮೀಜಿಯವರಿಗೆ ಏಸುಕ್ರಿಸ್ತನ ಜೀವನ-ಬೋಧನೆಗಳು ತುಂಬ ಪ್ರಿಯವಾದುವು. ಸಾಧನೆ ಯಲ್ಲಿ ವಿನೀತಭಾವದ ಮಹತ್ವವನ್ನು ಅವರು ಚೆನ್ನಾಗಿ ಅರಿತವರು. ಅಲ್ಲದೆ ವಿನಯ ಮೂರುತಿ ಶ್ರೀರಾಮಕೃಷ್ಣರ ಶಿಷ್ಯರು ಅವರು. ಹೀಗಿರುವಾಗ ಅವರು ಆ ಪುಸ್ತಕದ ಬೋಧನೆಯನ್ನು ಸಮರ್ಥಿ ಸದಿದ್ದುದು ಏಕಿರಬಹುದು? ಏಕೆಂದರೆ, ಯಾವುದೇ ಬೋಧನೆಯನ್ನು ಸಂದರ್ಭಾನುಸಾರವಾಗಿ ಅರ್ಥ ಮಾಡಿಕೊಳ್ಳಬೇಕಾಗುತ್ತದೆ. ಮೊದಲೇ ಜಡ ಸ್ವಭಾವದವರಾದ, ಕುಗ್ಗಿ ಕುಗ್ಗಿ ಕುರಿಗಳಂತಾ ಗಿರುವ ಭಾರತೀಯರಿಗೆ ಮತ್ತಷ್ಟು ದೀನತೆಯನ್ನು ಬೋಧಿಸುವುದು ಅರ್ಥಹೀನವಲ್ಲವೆ? ಆದ್ದರಿಂದಲೇ ಸ್ವಾಮೀಜಿಯವರು ‘ನಾನು ಹೀನ, ನಾನು ದೀನ’ ಎಂಬ ಭಾವನೆಯ ಬದಲಾಗಿ ‘ನಿತ್ಯ-ಶುದ್ಧ-ಬುದ್ಧ-ಮುಕ್ತನಾದ ಅಮೃತಸ್ವರೂಪಿ ಆತ್ಮನೇ ನಾನು!’ ಎಂಬಂತಹ ಪುರುಷ ನಿರ್ಮಾಪಕ ಭಾವನೆಯನ್ನು ಪ್ರಚೋದಿಸುತ್ತಿದ್ದಾರೆ.

ಒಂದು ದಿನ ಸಂದರ್ಶಕನೊಬ್ಬ ಕೇಳಿದ, “ಅವತಾರಪುರುಷನಿಗೂ ಮುಕ್ತಪುರುಷನಿಗೂ ವ್ಯತ್ಯಾಸವೇನು?” ಇದಕ್ಕೆ ಸ್ವಾಮೀಜಿ ನೇರವಾಗಿ ಉತ್ತರಿಸದೆ ಹೀಗೆಂದರು: “ಹಿಂದೆ, ಮುಕ್ತಿ ಪಡೆಯುವುದೇ ಅತ್ಯುನ್ನತ ಧ್ಯೇಯವೆಂಬುದು ನನ್ನ ಅಭಿಮತವಾಗಿತ್ತು. ನಾನು ನನ್ನ ಸಾಧನೆಯ ಕಾಲದಲ್ಲಿ ಪರಿವ್ರಾಜಕನಾಗಿ ಸುತ್ತಾಡುತ್ತಿದ್ದಾಗ ದಿನಗಟ್ಟಲೆ ನಿರಂತರವಾಗಿ ನಿರ್ಜನ ಗುಹೆಗಳಲ್ಲಿ ಇದ್ದುಬಿಡುತ್ತಿದ್ದೆ. ಇನ್ನೂ ಮುಕ್ತಿ ದೊರಕಲಿಲ್ಲವಲ್ಲ ಎಂಬ ಕಾತರದಿಂದ ಕೆಲವೊಮ್ಮೆ ಪ್ರಾಯೋ ಪವೇಶ ಮಾಡಲೂ ನಿರ್ಧರಿಸಿಬಿಟ್ಟಿದ್ದೆ. ಆದರೆ ಈಗ ನನಗೆ ಮುಕ್ತಿಯ ಬಯಕೆಯೇ ಇಲ್ಲ. ಈ ಜಗತ್ತಿನ ಪ್ರತಿಯೊಂದು ಜೀವಿಯೂ ಮುಕ್ತವಾಗುವವರೆಗೆ ನಾನು ನನ್ನ ವೈಯಕ್ತಿಕ ಮುಕ್ತಿಯನ್ನು ಲೆಕ್ಕಿಸುವವನಲ್ಲ.” ಈ ಮಾತು ಅವರಿಗೆ ತಮ್ಮ ಸಹಮಾನವರ ಮೇಲಿದ್ದ ಅಪಾರ ಅನುಕಂಪೆ ಯನ್ನು ತೋರಿಸುತ್ತದೆ. ಪರಮ ಕಾರುಣಿಕನಾದ ಬುದ್ಧಭಗವಂತನೂ ಹೀಗೆಯೇ ಹೇಳಿದ. ಆದರೆ ಇಲ್ಲಿ ನಾವು ನೆನಪಿನಲ್ಲಿಡಬೇಕಾದ ಅಂಶವೇನೆಂದರೆ, ಬುದ್ಧನಾಗಲಿ ಸ್ವಾಮೀಜಿಯವರಾಗಲಿ ಈ ಮಾತನ್ನಾಡಿದ್ದು ಅವರು ದಿವ್ಯಜ್ಞಾನವನ್ನು ಪಡೆದುಕೊಂಡ ಮೇಲೆಯೇ, ಅವರು ಮುಕ್ತರಾದ ಮೇಲೆಯೇ! ಕೇವಲ ಜೀವನ್ಮುಕ್ತರು ಮತ್ತು ಅವತಾರಪುರುಷರು ಮಾತ್ರವೇ ಈ ಬಗೆಯಲ್ಲಿ ಮುಕ್ತಿಯನ್ನು ನಿರ್ಲಕ್ಷಿಸಬಲ್ಲರು. ಮುಕ್ತಾತ್ಮರಿಗೂ ಅವತಾರಪುರುಷರಿಗೂ ಇದೇ ವ್ಯತ್ಯಾಸ– ಸಾಮಾನ್ಯರು ಮುಕ್ತಿಗಾಗಿ ಹಂಬಲಿಸಿ ಹಂಬಲಿಸಿ, ಕಡೆಗೆ ಅದನ್ನು ಸಿದ್ಧಿಸಿಕೊಂಡು ಮುಕ್ತಾತ್ಮ ರೆನ್ನಿಸಿಕೊಳ್ಳುತ್ತಾರೆ. ಆದರೆ ಅವತಾರಪುರುಷರಿಗೆ ಮುಕ್ತಿ ಎನ್ನುವುದು ಅಂಗೈ ಮೇಲಿನ ನೆಲ್ಲಿಕಾಯಿ ಯಂತೆ ಸ್ವತಃಸಿದ್ಧವಾಗಿರುತ್ತದೆ. ಆದರೂ ಅವರು ಮುಕ್ತರಾಗಿ ಪರಬ್ರಹ್ಮದಲ್ಲಿ ಲೀನರಾಗಲು ನಿರಾಕರಿಸುತ್ತಾರೆ. ಏಕೆ? ಅವರು ಇತರರನ್ನೂ ಮುಕ್ತಿಯ ಆನಂದಕ್ಕೆ ಕರೆದೊಯ್ಯುವ ಇಚ್ಛೆಯುಳ್ಳವರಾಗಿರುತ್ತಾರೆ.

ಮತ್ತೊಮ್ಮೆ ಸ್ವಾಮೀಜಿ ಕೆಲವು ಯುವಕರೊಂದಿಗೆ ಸ್ಫೂರ್ತಿಭರಿತರಾಗಿ ಮಾತನಾಡುತ್ತ ಹೀಗೆಂದರು, “ಈಗ ಸಾವಿರಾರು ವರ್ಷಗಳಿಂದ ಪರಕೀಯರು ಭಾರತೀಯರಾದ ನಿಮ್ಮನ್ನು ಒಂದೇ ಸಮನೆ ‘ನೀವು ದುರ್ಬಲರು, ನೀವು ಅಪ್ರಯೋಜಕರು, ನಿಮಗೆ ವ್ಯಕ್ತಿತ್ವವಿಲ್ಲ’ ಎಂದು ಹೇಳಿ ಹೇಳಿ ಹೇಳಿ, ಅದನ್ನು ನೀವು ಕೇಳಿ ಕೇಳಿ ಕೇಳಿ, ಈಗ ನೀವು ನಿಜಕ್ಕೂ ದುರ್ಬಲರೇ ಹೌದು ಎಂದು ಒಪ್ಪಿಕೊಳ್ಳುವ ಸ್ಥಿತಿಗೆ ಬಂದಿದ್ದೀರಿ. ನನ್ನ ಈ ಶರೀರವೂ ಕೂಡ ಇದೇ ಭಾರತ ಭೂಮಿ ಯಲ್ಲೇ ಜನ್ಮ ತಳೆದು ಈ ಅನ್ನವನ್ನೇ ಉಂಡು ಬೆಳೆದಿದೆ. ಆದರೆ ನಾನು ಮಾತ್ರ ಎಂದೆಂದಿಗೂ ಕ್ಷಣಮಾತ್ರವಾದರೂ ‘ನಾನು ದುರ್ಬಲ’ ಎಂಬ ಭಾವನೆಯನ್ನು ನನ್ನ ಮನದ ಒಳಹೊಗಲು ಬಿಟ್ಟವನಲ್ಲ. ನನ್ನಲ್ಲಿ ನನಗೆ ಮೊದಲಿನಿಂದಲೂ ಅಪರಿಮಿತ ಆತ್ಮವಿಶ್ವಾಸವಿತ್ತು. ಈ ಪ್ರಬಲ ಆತ್ಮವಿಶ್ವಾಸದಿಂದಾಗಿ ಮತ್ತು ಭಗವತ್ಕೃಪೆಯಿಂದಾಗಿ, ಯಾವ ಪರರಾಷ್ಟ್ರೀಯರು ನಮ್ಮನ್ನು ದುರ್ಬಲರು-ದಲಿತರು ಎಂಬ ನಿಕೃಷ್ಟ ದೃಷ್ಟಿಯಿಂದ ನೋಡುತ್ತಾರೆಯೋ ಅವರೇ ನನ್ನನ್ನು ಇಂದು ಗುರುವಾಗಿ ಸ್ವೀಕರಿಸಿ ಗೌರವಿಸುತ್ತಿದ್ದಾರೆ. ನೀವೂ ಸಹ ನನ್ನಂತೆಯೇ ಆತ್ಮಶ್ರದ್ಧೆಯುಳ್ಳವ ರಾದರೆ, ನಿಮ್ಮೊಳಗೂ ಅನಂತ ಬಲವಿದೆ ಎಂದು ನೀವು ದೃಢವಾಗಿ ನಂಬಿದ್ದೇ ಆದರೆ, ನಿಮ್ಮೊಳಗೂ ಅಪರಿಮಿತ ಜ್ಞಾನ-ಅದಮ್ಯ ಶಕ್ತಿ ಅಡಗಿದೆಯೆಂದು ತಿಳಿದುದೇ ಆದರೆ, ಮತ್ತು ಆ ಶಕ್ತಿಯನ್ನು ಜಾಗೃತಗೊಳಿಸಿಕೊಂಡುದೇ ಆದರೆ ನೀವೂ ನನ್ನಂತೆಯೇ ಆಗುವಿರಿ, ಅದ್ಭುತಗಳನ್ನು ಸಾಧಿಸುವಿರಿ. ಆದರೆ ನೀವು, ‘ಆ ರೀತಿಯಲ್ಲಿ ಆಲೋಚಿಸುವ ಶಕ್ತಿ ನಮಗೆಲ್ಲಿ ಬರಬೇಕು? ನಮ್ಮೊಳಗೆ ಶಕ್ತಿಯನ್ನು ಸ್ಫುರಿಸಬಲ್ಲ ಗುರುಗಳು ನಮಗೆಲ್ಲಿ ಸಿಗಬೇಕು?’ ಎಂದು ಕೇಳಬಹುದು. ಹೆದರಬೇಡಿ, ನಾನಿದ್ದೇನೆ. ನಿಮಗೆ ಶಕ್ತಿ ಪಾಠವನ್ನು ಕಲಿಸುವುದಕ್ಕಾಗಿಯೇ, ನನ್ನ ಸ್ವಂತ ಜೀವನದ ಮೂಲಕ ನಿಮಗದನ್ನು ತೋರಿಸಿಕೊಡುವ ಸಲುವಾಗಿಯೇ ನಾನು ನಿಮ್ಮಲ್ಲಿಗೆ ಬಂದಿದ್ದೇನೆ. ನನ್ನಿಂದ ನೀವು ಆ ಪರಮತತ್ತ್ವವನ್ನು ಕಲಿಯಲೇಬೇಕು. ಬಳಿಕ ನೀವು ನಗರದಿಂದ ನಗರಕ್ಕೆ, ಗ್ರಾಮದಿಂದ ಗ್ರಾಮಕ್ಕೆ, ಮನೆಯಿಂದ ಮನೆಗೆ ಹೋಗಿ ಈ ದಿವ್ಯ ಸತ್ಯವನ್ನು ಪ್ರಸಾರ ಮಾಡ ಬೇಕು. ಹೋಗಿ ಪ್ರತಿಯೊಬ್ಬ ಭಾರತೀಯನಿಗೂ ಸಾರಿ ಹೇಳಿ–‘ಏಳಿ, ಎಚ್ಚರಗೊಳ್ಳಿ, ಕನಸು ಕಾಣುತ್ತ ಕುಳಿತಿರಬೇಡಿ. ಜಾಗೃತರಾಗಿ, ನಿಮ್ಮೊಳಗಣ ದಿವ್ಯತೆಯನ್ನು ಪ್ರಕಟಗೊಳಿಸಿ’ ಎಂದು. ಆತ್ಮಶಕ್ತಿಸಂಪನ್ನನಾದ ವ್ಯಕ್ತಿ ಪಡೆಯಲಾಗದ ವಸ್ತುವೇ ಇಲ್ಲ. ನನ್ನ ಈ ಮಾತನ್ನು ನಂಬಿ, ನೀವು ಸರ್ವಶಕ್ತರಾಗುವಿರಿ.”

ಸ್ವಾಮೀಜಿಯವರು ಗೋಪಾಲಲಾಲ್ ಸೀಲರ ಉದ್ಯಾನಗೃಹದಲ್ಲಿದ್ದಾಗ ಒಮ್ಮೆ ಕೆಲವು ಗುಜ ರಾತೀ ಪಂಡಿತರು ಅಲ್ಲಿಗೆ ಬಂದರು. ವೇದವೇದಾಂಗಗಳಲ್ಲಿ ಪಾರಂಗತರಾದ ಈ ಪಂಡಿತರು ಸ್ವಾಮೀಜಿಯವರೊಂದಿಗೆ ಶಾಸ್ತ್ರವಿಚಾರಗಳ ಚರ್ಚೆ ಮಾಡಲು ಬಂದಿದ್ದರು. ಇವರು ಸಂಸ್ಕೃತ ದಲ್ಲಿ ಚರ್ಚೆಯನ್ನು ಪ್ರಾರಂಭಿಸಿದರು. ಸ್ವಾಮೀಜಿ ತಮ್ಮೊಂದಿಗೆ ಸಂಸ್ಕೃತದಲ್ಲಿ ವ್ಯವಹರಿಸ ಬಲ್ಲರೇ ಎಂಬುದನ್ನು ಬಯಲಿಗೆಳೆದು ತಮಾಷೆ ನೋಡುವುದು ಇವರ ಒಳಉದ್ದೇಶ. ಆದರೆ ಸ್ವಾಮೀಜಿ ಶಾಂತವಾಗಿ ಸಂಸ್ಕೃತದಲ್ಲಿ ನಿರರ್ಗಳವಾಗಿ ಸಂಭಾಷಿಸತೊಡಗಿದರು. ಇದನ್ನು ಕಂಡು ಅಲ್ಲಿದ್ದ ಇತರ ಸಂನ್ಯಾಸಿಗಳಿಗೂ ಪರಮಾಶ್ಚರ್ಯ. ಸ್ವಾಮೀಜಿಯವರು ವಾದದಲ್ಲಿ ಸೋಲಬಾರ ದೆಂದು ಅಲ್ಲಿದ್ದ ಸ್ವಾಮಿ ರಾಮಕೃಷ್ಣಾನಂದರು ಮೌನವಾಗಿ ಪ್ರಾರ್ಥನೆ ಮಾಡುತ್ತ ಕುಳಿತರು! ಸ್ವಾಮೀಜಿಯವರ ಸಂಸ್ಕೃತ ಶೈಲಿ ಆ ಪಂಡಿತರ ಶೈಲಿಗಿಂತ ಹೆಚ್ಚು ಸಾಹಿತ್ಯ ಪೂರ್ಣವಾಗಿತ್ತು; ಮತ್ತು ಅವರ ಮಧುರ ಕಂಠದಿಂದಾಗಿ ಅದು ರಸಮಯವಾಗಿತ್ತು. ಪಂಡಿತರ ಪ್ರಶ್ನೆಗಳಿಗೆಲ್ಲ ಸ್ವಾಮೀಜಿ ಅಡೆತಡೆಯಿಲ್ಲದೆ ಉತ್ತರಿಸಿ, ತಮ್ಮ ಅಭಿಪ್ರಾಯಗಳನ್ನು ಸ್ಪಷ್ಟವಾಗಿ ಮಂಡಿಸಿದರು. ಆದರೆ ಮಧ್ಯೆ ಒಮ್ಮೆ ಮಾತ್ರ ಅವರು ‘ಅಸ್ತಿ’ ಎನ್ನುವುದರ ಬದಲು ಬಾಯಿತಪ್ಪಿ ‘ಸ್ವಸ್ತಿ’ ಎಂದು ಬಿಟ್ಟರು. ಇದೊಂದು ತೀರಾ ಕ್ಷುಲ್ಲಕವಾದ ತಪ್ಪು. ಆದರೆ ಇಂಥದಕ್ಕಾಗಿಯೆ ಕಾದಿದ್ದ ಆ ಪಂಡಿತರು ತಕ್ಷಣ ಚಪ್ಪಾಳೆ ತಟ್ಟಿ ಗಟ್ಟಿಯಾಗಿ ನಕ್ಕುಬಿಟ್ಟರು. ಆ ಪಂಡಿತರ ಅಲ್ಪತನವನ್ನು ಕಂಡೂ ಸ್ವಾಮೀಜಿ ಸಿಟ್ಟಿಗೇಳದೆ ತಕ್ಷಣ ತಮ್ಮ ದೋಷವನ್ನು ಸರಿಪಡಿಸಿಕೊಂಡರು. ಬಳಿಕ ಕೈ ಮುಗಿಯುತ್ತ, “ಪಂಡಿತಾನಾಂ ದಾಸೋ\eng{s}ಹಂ ಕ್ಷಂತವ್ಯಂ ಸ್ಖಲನಂ ಮಮ”, ಎಂದರೆ ‘ಪಂಡಿತರ ದಾಸ ನಾನು, ನನ್ನ ತಪ್ಪನ್ನು ಮನ್ನಿಸಬೇಕು’ ಎಂದರು. ಅವರ ವಿನಯವನ್ನು ಕಂಡು ಪಂಡಿತರಿಗೇ ಮುಖಭಂಗವಾದಂತಾಯಿತು.

ಅಂದಿನ ಚರ್ಚೆಯ ಮುಖ್ಯ ವಿಷಯ ಪೂರ್ವಮೀಮಾಂಸೆ ಮತ್ತು ಉತ್ತರ ಮೀಮಾಂಸೆಗಳ ಸ್ಥಾನಮಾನವನ್ನು ಕುರಿತದ್ದು. ಪೂರ್ವ ಮೀಮಾಂಸೆಯೆಂದರೆ ಕರ್ಮಕಾಂಡ. ಯಾವ ಬಗೆಯ ಸ್ವರ್ಗವನ್ನು ಪಡೆಯಲು ಎಂತಹ ಕರ್ಮವನ್ನು ಮಾಡಬೇಕು, ಯಾವ ಬಗೆಯ ಪ್ರಾಯಶ್ಚಿತ್ತಕ್ಕೆ ಯಾವ ಕರ್ಮವನ್ನು ಕೈಗೊಳ್ಳತಕ್ಕದ್ದು ಎಂಬಂತಹ ವಿವರಗಳಿಂದ ಕೂಡಿದ್ದು ಈ ಕರ್ಮಕಾಂಡ ಅಥವಾ ಪೂರ್ವಮೀಮಾಂಸೆ. ಉತ್ತರ ಮೀಮಾಂಸೆಯೆಂದರೆ ಜ್ಞಾನಕಾಂಡ; ಎಂದರೆ, ಆತ್ಮಜ್ಞಾನ ವನ್ನು ಬೋಧಿಸುವಂತಹ ಉಪನಿಷತ್ತುಗಳಿಂದ ಕೂಡಿದುದು. ಕರ್ಮಕಾಂಡವೇ ವೇದಗಳ ಅತಿ ಮುಖ್ಯಭಾಗ ಎಂಬುದು ಆ ಪಂಡಿತರ ವಾದ. ಇದೇ ರೀತಿಯ ಅಭಿಪ್ರಾಯವನ್ನು ಪ್ರೊ ॥ ಮ್ಯಾಕ್ಸ್ ಮುಲ್ಲರ್ ಕೂಡ ವ್ಯಕ್ತಪಡಿಸಿದ್ದರು. ಆದರೆ ಅತ್ಯಂತ ಶ್ರೇಷ್ಠವಾದ ತತ್ತ್ವಜ್ಞಾನವನ್ನೊಳಗೊಂಡಿ ರುವ ಜ್ಞಾನಕಾಂಡವೇ ವೇದಗಳ ತಿರುಳು; ಈ ಕರ್ಮಕಾಂಡವು ಕೇವಲ ಮೇಲಿನ ಹೊದಿಕೆ ಯಂಥದು ಎಂಬುದು ಸ್ವಾಮೀಜಿಯವರ ನಿಶ್ಚಿತ ಅಭಿಮತ. ಸ್ವಾಮೀಜಿ ತಮ್ಮ ಅಪಾರ ಜ್ಞಾನ ದಿಂದ ಮತ್ತು ತೀಕ್ಷ್ಣವಾದ ತರ್ಕಸಾಮರ್ಥ್ಯದಿಂದ ಜ್ಞಾನಕಾಂಡದ ಗರಿಮೆಯನ್ನು ಎತ್ತಿಹಿಡಿದಾಗ ಪಂಡಿತರು ಮಣಿಯಲೇಬೇಕಾಯಿತು. ಕಡೆಗೆ ಅಲ್ಲಿಂದ ಹೊರಡುವಾಗ ಪಂಡಿತರು ಅಲ್ಲಿದ್ದವ ರೊಬ್ಬರ ಬಳಿ, “ಸ್ವಾಮಿಗಳಿಗೆ ಸಂಸ್ಕೃತ ವ್ಯಾಕರಣದ ಮೇಲೆ ಸಂಪೂರ್ಣ ಹಿಡಿತವಿರುವಂತೆ ಕಾಣ ಲಿಲ್ಲ. ಆದರೆ ಅವರು ನಿಜಕ್ಕೂ ಮಹಾಜ್ಞಾನಿಗಳೇ ಸರಿ. ಶಾಸ್ತ್ರಗಳ ಮೇಲೆ ಅವರಿಗೆ ಅಸಾಧಾರಣ ವಾದ ಪ್ರಭುತ್ವವಿದೆ. ವಾದದಲ್ಲಂತೂ ಅವರು ಅಪ್ರತಿಮರು” ಎಂದು ಕೊಂಡಾಡಿದರು.

ಪಂಡಿತರು ನಿರ್ಗಮಿಸಿದ ಮೇಲೆ, ಅವರ ಅಸಭ್ಯ ನಡವಳಿಕೆಯ ಬಗ್ಗೆ ಪ್ರಸ್ತಾಪಿಸಿ ಸ್ವಾಮೀಜಿ ಹೇಳಿದರು, “ಸುಸಂಸ್ಕೃತ ಪಾಶ್ಚಾತ್ಯ ಸಮಾಜದಲ್ಲಿ ಇಂತಹ ವರ್ತನೆಯನ್ನು ಯಾರೂ ಸಹಿಸುವು ದಿಲ್ಲ. ಅಲ್ಲಿ ಪ್ರತಿವಾದಿಯ ಮಾತಿನ ಭಾವದ ಕಡೆಗೆ ಗಮನ ಕೊಡುತ್ತಾರೆಯೇ ಹೊರತು, ಹೀಗೆ ಮಾತಿನಲ್ಲಿ ವ್ಯಾಕರಣ ದೋಷಗಳನ್ನು ಹುಡುಕಿ ಅಪಹಾಸ್ಯ ಮಾಡುವುದಿಲ್ಲ. ನಮ್ಮ ಪಂಡಿತರು ಗಳು ಯಾವುದೋ ಅಕ್ಷರವ್ಯತ್ಯಾಸವನ್ನೇ ಹಿಡಿದು ಜಗ್ಗಾಡುತ್ತ ಮುಖ್ಯಾಂಶವನ್ನೇ ಮರೆತುಬಿಡು ತ್ತಾರೆ. ಹೊರಗಿನ ಹೊಟ್ಟನ್ನು ಹಿಡಿದು ಹೋರಾಡುತ್ತಾರೆ; ಒಳಗಿನ ತಿರುಳಿನ ಕಡೆಗೆ ಲಕ್ಷ್ಯವೇ ಇಲ್ಲ.”

ಇನ್ನೊಂದು ಸಂದರ್ಭದಲ್ಲಿ ಕೆಲವರು ಪ್ರಾಣಾಯಾಮದ ಕುರಿತಾಗಿ ತಿಳಿದುಕೊಳ್ಳಲು ಬಂದರು. ಆಗ ಸ್ವಾಮೀಜಿಯವರು ಮೊದಲೇ ಅಲ್ಲಿಗೆ ಬಂದಿದ್ದ ಕೆಲವು ಸಂದರ್ಶಕರೊಡನೆ ಮಾತನಾಡುತ್ತ ಕುಳಿತಿದ್ದರು. ಅವರ ಪ್ರಶ್ನೆಗಳಿಗೆಲ್ಲ ಉತ್ತರ ಕೊಟ್ಟಮೇಲೆ ಸ್ವಾಮೀಜಿ, ಪ್ರಾಣಾ ಯಾಮದ ವಿಷಯವಾಗಿ ತಾವಾಗಿಯೇ ಮಾತನಾಡಲಾರಂಭಿಸಿದರು. ಆ ವಿಷಯವಾಗಿಯೇ ಕೇಳಿ ತಿಳಿದುಕೊಳ್ಳಲೆಂದು ಬಂದಿದ್ದವರು ಪ್ರಶ್ನೆ ಕೇಳುವ ಅಗತ್ಯವೇ ಉಂಟಾಗಲಿಲ್ಲ! ಅಷ್ಟೇ ಅಲ್ಲ, ಪ್ರಾಣಾಯಾಮ ಹಾಗೂ ಅದಕ್ಕೆ ಸಂಬಂಧಿಸಿದ ವಿಷಯಗಳ ಬಗ್ಗೆ ಅಪರಾಹ್ನ ಮೂರು ಗಂಟೆಗೆ ಮಾತನಾಡಲು ತೊಡಗಿದವರು ಸಂಜೆ ಏಳು ಗಂಟೆಯವರೆಗೂ ಮಾತನಾಡುತ್ತಲೇ ಇದ್ದರು. ಆಗಲೇ ಪ್ರಕಟವಾಗಿದ್ದ ಅವರ ‘ರಾಜಯೋಗ’ ಎಂಬ ಗ್ರಂಥದಲ್ಲಿ ಈ ವಿಷಯಗಳೆಲ್ಲ ಇದ್ದುವು. ಆದರೆ ಯೋಗದ ಬಗ್ಗೆ ಸ್ವಾಮೀಜಿಯವರಿಗಿರುವ ಜ್ಞಾನದಲ್ಲಿ, ಪುಸ್ತಕದ ಮೂಲಕ ಪ್ರಕಟ ವಾಗಿರುವುದು ಕೇವಲ ಎಲ್ಲೋ ಒಂದಂಶ ಮಾತ್ರವೇ ಎಂಬುದು ಸುಸ್ಪಷ್ಟವಾಗಿತ್ತು. ಅಲ್ಲದೆ ಅವರಾಡುತ್ತಿದ್ದ ಮಾತುಗಳೊಂದೊಂದೂ ಅವರ ಅನುಭವದಿಂದಲೇ ಒಡಮೂಡಿದಂಥವು ಎಂಬುದರಲ್ಲಿ ಸಂಶಯವೇ ಇರಲಿಲ್ಲ. ಸಂದರ್ಶಕರಿಗೆ ಅತ್ಯಂತ ಅಚ್ಚರಿಯ ಸಂಗತಿ ಯೇನೆಂದರೆ, ತಮ್ಮ ಮನಸ್ಸಿನಲ್ಲಿದ್ದ ಪ್ರಶ್ನೆಯನ್ನು ಸ್ವಾಮೀಜಿ ಅರಿತುಕೊಂಡ ಬಗೆ ಹೇಗೆ ಎಂಬುದು. ಇದನ್ನು ಕುರಿತು ಒಬ್ಬ ಕೇಳಿಯೇಬಿಟ್ಟ. ಅದಕ್ಕೆ ಸ್ವಾಮೀಜಿ ನೇರವಾಗಿ ಬದಲು ಕೊಡದೆ ಹೀಗೆ ಹೇಳಿದರು, “ನಾನು ಪಾಶ್ಚಾತ್ಯ ದೇಶಗಳಲ್ಲಿದ್ದಾಗಲೂ ಇಂಥದೇ ಘಟನೆಗಳು ನಡೆಯುತ್ತಿದ್ದುವು. ಅಲ್ಲಿನ ಜನರೂ ನನ್ನನ್ನು ಕೇಳುತ್ತಿದ್ದರು. ‘ಅದು ಹೇಗೆ ಸ್ವಾಮೀಜಿ, ನಮ್ಮ ಮನಸ್ಸಿನಲ್ಲಿ ಕಾಡುತ್ತಿರುವ ಪ್ರಶ್ನೆಗಳನ್ನೆಲ್ಲ ನೀವು ತಿಳಿದುಕೊಂಡು ಬಿಡುತ್ತೀರಲ್ಲ!’ ಎಂದು.” ಇದನ್ನು ಕೇಳಿ ಅಲ್ಲಿದ್ದವರ ಕುತೂಹಲ ಇನ್ನಷ್ಟು ಕೆರಳಿತು. ಅಷ್ಟು ಹೊತ್ತಿಗೆ ಒಬ್ಬರು ಇನ್ನೊಂದು ಪ್ರಶ್ನೆ ಹಾಕಿದರು, “ಸ್ವಾಮೀಜಿ, ನಿಮಗೆ ನಿಮ್ಮ ಪೂರ್ವಜನ್ಮಗಳ ನೆನಪಿದೆಯೆ?” ತಕ್ಷಣ ಸ್ವಾಮೀಜಿ ಉತ್ತರಿಸಿದರು, “ಹೌದು, ನನಗೆ ನೆನಪಿದೆ!” ಎಂದು. ಆ ಬಗ್ಗೆ ತಮಗೆಲ್ಲ ತಿಳಿಸುವಂತೆ ಜನ ಒತ್ತಾಯಿಸಿದಾಗ ಸ್ವಾಮೀಜಿ, “ನಾನು ಅವುಗಳ ಬಗ್ಗೆ ತಿಳಿದುಕೊಳ್ಳಬಲ್ಲೆ; ತಿಳಿದುಕೊಂಡೂ ಇದ್ದೇನೆ. ಆದರೆ ನಾನು ಆ ಬಗ್ಗೆ ಏನೂ ಮಾತನಾಡದಿರಲು ಬಯಸುತ್ತೇನೆ” ಎಂದು ಹೇಳಿ ಆ ಮಾತನ್ನು ಅಲ್ಲಿಗೇ ಮುಕ್ತಾಯಗೊಳಿಸಿದರು.

ಒಂದಾನೊಂದು ಸಂಜೆ ಸ್ವಾಮೀಜಿಯವರು ಸೀಲರ ಉದ್ಯಾನಗೃಹದಲ್ಲಿ ಕುಳಿತಿದ್ದರು. ಜೊತೆಗೆ ಸ್ವಾಮಿ ಪ್ರೇಮಾನಂದರಿದ್ದರು. ಲೋಕಾಭಿರಾಮವಾಗಿ ಮಾತುಕತೆ ನಡೆಯುತ್ತಿತ್ತು. ಆಗ ಇದ್ದಕ್ಕಿದ್ದಂತೆ ಸ್ವಾಮೀಜಿ ಮೌನವಾಗಿಬಿಟ್ಟರು. ಅವರ ಮುಖಭಾವ ಗಂಭೀರವಾಯಿತು. ಅವರು ಏನನ್ನೋ ದಿಟ್ಟಿಸಿ ನೋಡುತ್ತಿರುವಂತೆ ಕಂಡಿತು. ಸ್ವಲ್ಪ ಹೊತ್ತಿನ ಬಳಿಕ ಸ್ವಾಮೀಜಿ ಪ್ರೇಮಾ ನಂದರನ್ನು, “ನೀನೀಗ ಏನಾದರೂ ನೋಡಿದೆಯಾ?” ಎಂದು ಕೇಳಿದರು. ಪ್ರೇಮಾನಂದರು ತುಸು ಆಶ್ಚರ್ಯದಿಂದ, “ಇಲ್ಲವಲ್ಲ!” ಎಂದರು. ಆಗ ಸ್ವಾಮೀಜಿ ಹೇಳಿದರು: “ನಾನು ಈಗತಾನೆ ರುಂಡವಿಲ್ಲದ ಪ್ರೇತವೊಂದನ್ನು ಕಂಡೆ. ಅದು ಕರುಣಾಜನಕಭಾವದಿಂದ ನನ್ನ ಮುಂದೆ ನಿಂತು ತನ್ನನ್ನು ಆ ಸ್ಥಿತಿಯಿಂದ ಪಾರು ಮಾಡುವಂತೆ ನನ್ನನ್ನು ಕೇಳಿಕೊಂಡಿತು.” ಮುಂದೆ ವಿಚಾರಿಸಿದಾಗ ತಿಳಿದುಬಂದ ವಿಷಯ ಇದು; ಹಲವಾರು ವರ್ಷಗಳ ಹಿಂದೆ ಆ ಮನೆಯಲ್ಲಿ ಒಬ್ಬ ಬ್ರಾಹ್ಮಣ ಇದ್ದ. ಅವನು ಬಡ್ಡಿಗೆ ಸಾಲ ಕೊಡುತ್ತಿದ್ದ. ಅವನ ಬಡ್ಡಿಯ ದರ ಚಾಲ್ತಿಯ ದರಕ್ಕಿಂತಲೂ ಎಷ್ಟೋ ಪಾಲು ಅಧಿಕ. ಕಡೆಗೊಂದು ದಿನ ಒಬ್ಬ ಸಾಲಗಾರ ಆ ಬಡ್ಡಿಗೆ ಅನುಸಾರವಾಗಿ ಸಾಲ ತೀರಿಸಲಾಗದೆ ಆ ಬ್ರಾಹ್ಮಣನ ಕತ್ತನ್ನೇ ಕತ್ತರಿಸಿ ಗಂಗೆಗೆ ಎಸೆದುಬಿಟ್ಟ. ಪರಿಣಾಮ ಈ ಪ್ರೇತ ಶರೀರ. ಈ ರೀತಿಯ ಅನುಭವ ಸ್ವಾಮೀಜಿಯವರಿಗೆ ಎಷ್ಟೋ ಸಲ ಆಗಿತ್ತು. ಮತ್ತು ಅವುಗಳು ಆ ದುಃಸ್ಥಿತಿಯಿಂದ ಪಾರಾಗುವಂತೆ ಅವರು ಹೃತ್ಪೂರ್ವಕವಾಗಿ ಪ್ರಾರ್ಥಿಸಿಕೊಳ್ಳುತ್ತಿದ್ದರು.

ಒಮ್ಮೆ ಮಠದ ಉಪಯೋಗಕ್ಕಾಗಿ ಪ್ರೊ । ಮ್ಯಾಕ್ಸ್ ಮುಲ್ಲರರು ಅನುವಾದಿಸಿದ ಪುಗ್ವೇದದ ಒಂದು ಪ್ರತಿಯನ್ನು ಶ್ರೀಮಂತರೊಬ್ಬರ ಮನೆಯಿಂದ ತರಲಾಗಿತ್ತು. ಅದನ್ನೋದುವಲ್ಲಿ ಶರಚ್ಚಂದ್ರನಿಗೆ ಸ್ವಾಮೀಜಿ ನೆರವಾಗುತ್ತಿದ್ದರು. ಒಂದು ದಿನ ಅದನ್ನು ವಿವರಿಸುವ ಸಂದರ್ಭದಲ್ಲಿ ಸ್ವಾಮೀಜಿ ನುಡಿದರು, “ನನಗೆ ಒಮ್ಮೊಮ್ಮೆ ಅನಿಸುತ್ತದೆ, ಸಾಯಣಾಚಾರ್ಯರೇ ಈಗ ಮ್ಯಾಕ್ಸ್ ಮುಲ್ಲರರಾಗಿ ಜನ್ಮತಾಳಿದ್ದಾರೆ ಎಂದು.” ಆದರೆ ಶರಚ್ಚಂದ್ರನಿಗೇಕೋ ಇದು ಒಪ್ಪಿಗೆಯಾಗದೆ ಹೇಳಿದ, “ಸ್ವಾಮೀಜಿ, ಸಾಯಣಾಚಾರ್ಯರು ಮತ್ತೊಮ್ಮೆ ಜನ್ಮವೆತ್ತಿ ಬಂದರೂ, ಈ ಪವಿತ್ರ ಭಾರತದಲ್ಲಿ ಒಬ್ಬ ಬ್ರಾಹ್ಮಣನಾಗಿ ಹುಟ್ಟಿಯಾರೇ ಹೊರತು ಒಬ್ಬ ಮ್ಲೇಚ್ಛನಾಗಿ ಹುಟ್ಟುವು ದುಂಟೆ!” ಆಗ ಸ್ವಾಮೀಜಿಯೆಂದರು, “ನೋಡು ‘ಅವನು ಮ್ಲೇಚ್ಛ, ಇವನು ಆರ್ಯ’ ಎಂದು ಭೇದ ಮಾಡುವುದು ಅಜ್ಞಾನಿಗಳು ಮಾತ್ರ. ಯಾರು ನಿಜವಾಗಿಯೂ ವೇದಗಳನ್ನು ಅರಿತು ವ್ಯಾಖ್ಯಾನ ಬರೆಯಬಲ್ಲರೋ ಮತ್ತು ಯಾರು ಅತ್ಯುನ್ನತ ಜ್ಞಾನವನ್ನು ಪಡೆದುಕೊಂಡಿದ್ದಾರೆಯೋ ಅವರು ಈ ಭೇದಗಳನ್ನು ಮೀರಿರುತ್ತಾರೆ. ಜಗದ್ಧಿತಕ್ಕಾಗಿ ಅಂತಹ ವ್ಯಕ್ತಿಗಳು ಜಗತ್ತಿನ ಯಾವುದೇ ಭಾಗದಲ್ಲಿ ಹುಟ್ಟಿಬರಬಲ್ಲರು. ಅಲ್ಲದೆ ಈಗ ಅವರು ಪಶ್ಚಿಮ ದೇಶದಲ್ಲಿ ಹುಟ್ಟಿರದಿದ್ದರೆ ಈ ವೇದಗಳ ಪ್ರಕಟಣೆಗೆ ತಗುಲಿದ ಭಾರೀ ವೆಚ್ಚವನ್ನು ಯಾರು ವಹಿಸಿಕೊಳ್ಳುತ್ತಿದ್ದರು?”

ಹಾಗೆಯೇ ಸ್ವಲ್ಪ ಹೊತ್ತು ಪಾಠ ಮುಂದುವರಿಯಿತು. ಆಗ ಗಿರೀಶ್​ಚಂದ್ರ ಘೋಷ್ ಅಲ್ಲಿಗೆ ಬಂದು ಸ್ವಾಮೀಜಿಯವರ ಮಾತುಗಳನ್ನು ಆಲಿಸುತ್ತ ಕುಳಿತ. ಅವನನ್ನು ಕಂಡು ಸ್ವಾಮೀಜಿ ಸ್ವಲ್ಪ ತಮಾಷೆಯಾಗಿ ಹೇಳಿದರು, “ಅದೆಲ್ಲ ಸರಿ ಜಿ. ಸಿ., (ಜಿ.ಸಿ. ಎಂದರೆ ಗಿರೀಶ್ ಚಂದ್ರ) ಆದರೆ ನೀವು ಮಾತ್ರ ಈ ವೇದಗೀದಗಳನ್ನು ಓದುವವರೇ ಅಲ್ಲ. ಏನಿದ್ದರೂ ನಿಮ್ಮ ಕೃಷ್ಣ-ವಿಷ್ಣು ಇಷ್ಟರಲ್ಲೇ ಇದ್ದುಬಿಡುತ್ತೀರಿ ನೀವು.”–ಎಂದರೆ, ಭಕ್ತಿ-ಶ್ರದ್ಧೆಗಳಿದ್ದರೆ ಬೇಕಾದಷ್ಟಾಯಿತು, ಈ ಶಾಸ್ತ್ರ ವಿಚಾರಗಳೆಲ್ಲ ನಮ್ಮಂಥವರಿಗಲ್ಲ ಎಂಬುದು ಗಿರೀಶನ ಭಾವನೆ ಎಂದು ಸ್ವಾಮೀಜಿ ಯವರು ಸೂಚಿಸುತ್ತಿದ್ದಾರೆ.

ಇದಕ್ಕೆ ಮಹಾ ನಾಟಕಕರ್ತನೂ ಭಕ್ತವರೇಣ್ಯನೂ ಆದ ಗಿರೀಶ ತಕ್ಷಣ ನುಡಿದ, “ಹೌದು; ನನಗೆ ಅವುಗಳನ್ನೆಲ್ಲ ಓದಲು ಸಮಯವೂ ಇಲ್ಲ, ಅದಕ್ಕೆ ಬೇಕಾದ ಬುದ್ಧಿಮತ್ತೆಯೂ ಇಲ್ಲ. ನಿನ್ನ ವಿಷಯವಾದರೆ ಹಾಗಲ್ಲ. ನಿನ್ನ ಮೂಲಕ ಶ್ರೀರಾಮಕೃಷ್ಣರು ಬಹಳಷ್ಟು ಪ್ರಚಾರಕಾರ್ಯ ಬೋಧನಾಕಾರ್ಯಗಳನ್ನು ಮಾಡಿಸಬೇಕಾಗಿತ್ತು. ಆದ್ದರಿಂದಲೇ ಅವರು ನಿನಗೆ ಅವುಗಳನ್ನೆಲ್ಲ ಓದುವಂತೆ ಪ್ರೋತ್ಸಾಹಿಸಿದರು. ನನ್ನ ವಿಷಯ ಹೇಳುವುದಾದರೆ, ಈ ವೇದಶಾಸ್ತ್ರಗಳಿಗೆಲ್ಲ ದೂರದಿಂದಲೇ ನಮಸ್ಕಾರ! ಶ್ರೀರಾಮಕೃಷ್ಣರ ಕೃಪೆಯೊಂದರಿಂದಲೇ ನಾನು ಈ ಭವಸಾಗರ ವನ್ನು ದಾಟಿಯೇನು.” ಹೀಗೆಂದವನೇ ಗಿರೀಶ ಭಕ್ತಿಭಾವದಿಂದ, “ವೇದಸ್ವರೂಪರಾದ ಶ್ರೀರಾಮ ಕೃಷ್ಣರಿಗೆ ಜಯವಾಗಲಿ!” ಎಂದು ಉದ್ಘೋಷಿಸುತ್ತ ವೇದದ ಪ್ರತಿಯ ಮುಂದೆ ಮತ್ತೆ ಮತ್ತೆ ಬಾಗಿ ನಮಸ್ಕರಿಸಿದ.

ಇದನ್ನು ಕಂಡು ಸ್ವಾಮೀಜಿ ಮೂಕರಾಗಿ ಕುಳಿತುಬಿಟ್ಟರು. ಅವರು ಯಾವುದೋ ಗಾಢ ಆಲೋಚನೆಯಲ್ಲಿ ಮಗ್ನರಾಗಿದ್ದುದನ್ನು ಅವರ ಮುಖಭಾವ ಪ್ರತಿಬಿಂಬಿಸುತ್ತಿತ್ತು. ಆಗ ಗಿರೀಶ ಇದ್ದಕ್ಕಿದ್ದಂತೆ ಸ್ವಾಮೀಜಿಯವರಿಗೆ ಸವಾಲೆಸೆಯುವಂತೆ ತನ್ನ ಕವಿಸಹಜ ಶೈಲಿಯಿಂದ ಕೇಳಿದ, “ಸರಿ ನರೇನ್, ಈಗ ನಿನ್ನನ್ನು ನಾನೊಂದು ವಿಷಯ ಕೇಳುತ್ತೇನೆ. ನೀನು ವೇದವೇದಾಂತಗಳನ್ನು ಸಾಕಷ್ಟು ಓದಿಕೊಂಡವನು. ಆದರೆ ಈಗ ಹೇಳು, ಈ ದೀನದಲಿತರ ದಯನೀಯ ದುಃಸ್ಥಿತಿಗೆ, ದುಃಖಾರ್ತ ಜನರ ಆಕ್ರಂದನಕ್ಕೆ, ಅಸಂಖ್ಯಾತ ಹಸಿದ ಹೊಟ್ಟೆಗಳ ಕೂಗಿಗೆ, ವ್ಯಭಿಚಾರವೇ ಮೊದಲಾದ ಹೀನ ಪಾತಕಗಳಿಗೆ, ಹಾಗೂ ನಾವು ದಿನಬೆಳಗಾದರೆ ಕಾಣುವ ಇತರ ಅನೇಕ ಬಗೆಯ ದುಷ್ಕೃತ್ಯ ದುರಾಚಾರಗಳಿಗೆ ಆ ನಿಮ್ಮ ವೇದಗಳಲ್ಲಿ ಪರಿಹಾರವೇನಾದರೂ ಇದೆಯೇ? ಅಗೋ ಅಲ್ಲಿ ನೋಡು, ಆ ಮನೆಯ ಹೆಂಗಸು ಒಂದು ಕಾಲದಲ್ಲಿ ಪ್ರತಿ ದಿನವೂ ಐವತ್ತು ಹೊಟ್ಟೆಗಳಿಗೆ ಅನ್ನ ಮಾಡಿ ಹಾಕುತ್ತಿದ್ದಳು; ಈಗ ಮೂರು ದಿನದಿಂದ ಅವಳಿಗೂ ಅವಳ ಮಕ್ಕಳಿಗೂ ತುತ್ತಿಗೇ ಗತಿಯಿಲ್ಲದಂತಾಗಿದೆ! ಇನ್ನೊಂದು ಮನೆಯ ಹೆಂಗಸಿನ ಕತೆ ಇನ್ನೂ ದಾರುಣ; ಕೆಲವು ಗೂಂಡಾ ಗಳು ಸೇರಿ ಅವಳ ಮೇಲೆ ಅತ್ಯಾಚಾರವೆಸಗಿ ಕೊಂದು ಹಾಕಿದರು! ಮತ್ತೊಂದು ಮನೆಯ ಯುವ ವಿಧವೆ ಗರ್ಭಿಣಿಯಾಗಿ ಬಳಿಕ ಮಾನವುಳಿಸಿಕೊಳ್ಳಲು ಗರ್ಭಪಾತ ಮಾಡಿಸಿಕೊಳ್ಳಲು ಹೋಗಿ ಪ್ರಾಣಬಿಟ್ಟಳು!... ಈಗ ನಾನು ಕೇಳುತ್ತೇನೆ, ನರೇನ್, ಇವುಗಳನ್ನೆಲ್ಲ ತಡೆಯಲು ವೇದಗಳಲ್ಲಿ ಉಪಾಯವೇನಾದರೂ ನಿನಗೆ ಕಾಣಿಸಿದೆಯೆ?”

ಗಿರೀಶನ ಮಾತುಗಳನ್ನು ಕೇಳಕೇಳುತ್ತಿದ್ದಂತೆಯೇ ಸ್ವಾಮೀಜಿಯವರ ವಿಶಾಲ ಕಮಲನಯನ ಗಳು ಹನಿಗೂಡಿದುವು! ಇತರರು ಅವುಗಳನ್ನು ಗಮನಿಸುವಷ್ಟರಲ್ಲೇ ಅವರು ತಮ್ಮ ಭಾವನೆ ಗಳನ್ನು ಬಲವಂತದಿಂದ ತಡೆಹಿಡಿದು ವೇಗವಾಗಿ ಅಲ್ಲಿಂದ ಹೊರಗೆ ನಡೆದರು. ಆದರೆ ಸ್ವಾಮೀಜಿಯವರ ವರ್ತನೆಯಿಂದಲೇ ಅವರ ಭಾವನೆಗಳ ತೀವ್ರತೆಯನ್ನು ಅರಿಯಬಹುದಾಗಿತ್ತು. ಈಗ ಗಿರೀಶನು ಶರಚ್ಚಂದ್ರನತ್ತ ತಿರುಗಿ ಹೇಳಿದ, “ನೋಡಿದೆಯಾ, ಬಂಗಾಳಿ? ಎಂಥಾ ಪ್ರೇಮ ಭರಿತ ಹೃದಯ! ಎಂಥಾ ದಯಾಪೂರಿತ ಹೃದಯ! ನಾನು ನಿನ್ನ ಸ್ವಾಮೀಜಿಯನ್ನು ಗೌರವಿಸು ವುದು, ಅವರು ಮಹಾ ವೇದಪಾರಂಗತರೆಂಬ ಕಾರಣಕ್ಕಲ್ಲ. ಆದರೆ ತಮ್ಮ ಸಹಮಾನವರ ದುಃಖದ ಬಗ್ಗೆ ಮಾತನಾಡಿದೊಡನೆಯೇ ಅವರ ಕಣ್ಣಲ್ಲಿ ನೀರುಕ್ಕುವಂತೆ ಮಾಡಿತಲ್ಲ, ಅವರ ಆ ಮಹಾಹೃದಯಕ್ಕಾಗಿ ನಾನವರನ್ನು ಗೌರವಿಸುವುದು!”

ಸ್ವಾಮೀಜಿ ಕಲ್ಕತ್ತದಲ್ಲಿದ್ದಾಗ ರಾತ್ರಿಯ ವೇಳೆಯನ್ನು ಹೆಚ್ಚಾಗಿ ಆಲಂಬಜಾರ್ ಮಠದಲ್ಲಿ ತಮ್ಮ ಸೋದರ ಸಂನ್ಯಾಸಿಗಳೊಂದಿಗೆ ಕಳೆಯುತ್ತಿದ್ದರು. ಅವರ ಸಾನ್ನಿಧ್ಯವೆಂದರೆ ಆ ಸಂನ್ಯಾಸಿ ಗಳಿಗೆಲ್ಲ ಅದೊಂದು ರೋಚಕ ಅನುಭವ. ಅವರೆಲ್ಲ ಒಟ್ಟಾಗಿ ಕುಳಿತು ತಾವು ಶ್ರೀರಾಮಕೃಷ್ಣ ರೊಂದಿಗೆ ಕಳೆದ ಚಿರಸ್ಮರಣೀಯ ದಿನಗಳನ್ನು ಮೆಲುಕು ಹಾಕುತ್ತಿದ್ದರು; ತಮ್ಮತಮ್ಮ ಪರಿ ವ್ರಾಜಕ ದಿನಗಳಲ್ಲಿ ತಮಗಾದ ಬಗೆಬಗೆಯ ಅನುಭವಗಳನ್ನು ಹೇಳಿಕೊಳ್ಳುತ್ತಿದ್ದರು. ಸ್ವಾಮೀಜಿ ಯವರು ತಮ್ಮ ಪಾಶ್ಚಾತ್ಯ ಪ್ರವಾಸದ ಅನುಭವಗಳ ಕುರಿತಾಗಿಯೋ ಇನ್ನಾವುದಾದರೂ ವಿಷಯದ ಬಗ್ಗೆಯೋ ಹೇಳತೊಡಗಿದರೆ ಎಲ್ಲರೂ ಅದನ್ನು ತೆರೆದ ಕಿವಿಗಳಿಂದ ಆಲಿಸುತ್ತಿದ್ದರು. ಸ್ವಾಮೀಜಿ ತಮ್ಮ ಗುರುಭಾಯಿಗಳೊಂದಿಗೆ ಹಿಂದಿನಷ್ಟೇ ಆತ್ಮೀಯತೆಯಿಂದ, ಸಲಿಗೆಯಿಂದ ಮಾತನಾಡುತ್ತಿದ್ದರು. ಅವರಿಗೆಲ್ಲ ಸ್ವಾಮೀಜಿ ಹಿಂದಿನ ನರೇಂದ್ರನಾಗಿಯೇ ತೋರಿದರೂ ಈಗ ಅವರ ನಡೆನುಡಿಯಲ್ಲಿ ಹೆಚ್ಚಿನ ಭಕ್ತಿ-ಗೌರವಗಳನ್ನು ಕಾಣಬಹುದಾಗಿತ್ತು. ಸ್ವಾಮೀಜಿ ತಮ್ಮೊಂ ದಿಗೆ ಎಷ್ಟೇ ಸಲಿಗೆಯಿಂದ ನಡೆದುಕೊಂಡರೂ, ಕಳೆದ ಕೆಲ ವರ್ಷಗಳಲ್ಲಿ ಅವರಲ್ಲಾಗಿದ್ದ ಬದಲಾ ವಣೆಯನ್ನೂ ಈ ಗುರುಭಾಯಿಗಳು ಗುರುತಿಸದಿರುವಂತಿರಲಿಲ್ಲ. ಹಿಂದಿನಿಂದಲೂ ನರೇಂದ್ರನೇ ಅವರೆಲ್ಲರ ನಾಯಕ, ಹಿರಿಯ ಅಣ್ಣ, ಮಾರ್ಗದರ್ಶಕ. ಆದರೆ ಹಿಂದೆ ಅವರು ಕಂಡಿದ್ದ ನರೇಂದ್ರ ಒಂದು ಪುಟ್ಟ ಮರವಾದರೆ ಈಗ ಕಾಣುತ್ತಿರುವ ಸ್ವಾಮೀಜಿ ಒಂದು ಮಹಾ ವಟವೃಕ್ಷ. ಹೀಗೆ ತಮ್ಮೆಲ್ಲರ ಊಹೆಗೂ ಮೀರಿ ಬೆಳೆದಿದ್ದ ವಿವೇಕಾನಂದರನ್ನು ಕಂಡು ಅವರೆಲ್ಲ ಬೆಕ್ಕಸ ಬೆರಗಾಗುತ್ತಿದ್ದರು.

ಆದರೆ ಆಗಿನ್ನೂ ಈ ಸಂನ್ಯಾಸಿಗಳಲ್ಲಿ ಯಾರೂ ವಿವೇಕಾನಂದರ ವಿಚಾರಗಳನ್ನು ಸಂಪೂರ್ಣ ವಾಗಿ ಅರಿತು, ಮೆಚ್ಚಲು ಸಮರ್ಥರಾಗಿರಲಿಲ್ಲ. ತಮ್ಮ ತಮ್ಮ ಸಂಪ್ರದಾಯ, ನಂಬಿಕೆ, ಆಚಾರ ಗಳ ಚೌಕಟ್ಟನ್ನು ಭೇದಿಸಿ ಹೊರಬರುವ ಶಕ್ತಿ ಅವರಲ್ಲಿರಲಿಲ್ಲ. ಅದರಲ್ಲೂ ಪಂಡಿತರು-ಬುದ್ಧಿ ವಂತರು-ಹಿರಿಯರು ಎನ್ನಿಸಿಕೊಂಡವರ ಟೀಕೆಯನ್ನು ಎದುರಿಸಿ, ದೃಢವಾಗಿ ನಿಲ್ಲಬಲ್ಲ ಎದೆಕೆಚ್ಚು ಇವರಲ್ಲಿ ಕಡಿಮೆಯೇ. ಆದ್ದರಿಂದ, ಸ್ವಾಮೀಜಿಯವರು ಸಮುದ್ರಪ್ರಯಾಣ ಮಾಡಿದ್ದನ್ನು ಸಂಪ್ರದಾಯಸ್ಥ ಪಂಡಿತರು-ಸಂನ್ಯಾಸಿಗಳು ಖಂಡಿಸಿದಾಗ, ಈ ಗುರುಭಾಯಿಗಳಿಗೂ ‘ಹೌದ ಲ್ಲವೆ!’ ಎನ್ನಿಸಿಬಿಟ್ಟಿತ್ತು. ಇದು ಸ್ವಾಮೀಜಿಯವರಿಗೂ ಗೊತ್ತಾಗದಿರಲಿಲ್ಲ. ಅವರು ಹಲವಾರು ಶಾಸ್ತ್ರವಾಕ್ಯಗಳನ್ನು ಉದಾಹರಿಸಿ ಸೋದರಸಂನ್ಯಾಸಿಗಳ ಆತಂಕವನ್ನು ಕ್ರಮೇಣ ಹೋಗಲಾಡಿಸಿ ದರು. ಅಲ್ಲದೆ ಸ್ವಾಮೀಜಿಯವರ ಜೊತೆಯಲ್ಲಿ ಬಂದಿದ್ದ ಪಾಶ್ಚಾತ್ಯ ಶಿಷ್ಯರೊಂದಿಗೆಲ್ಲ ಹೇಗೆ ವ್ಯವಹರಿಸುವುದು ಎಂಬ ಅನುಮಾನ ಆ ಸಂನ್ಯಾಸಿಗಳಿಗೆ. ಈ ಬಗೆಯ ಸಂಕೋಚ-ಹಿಂಜರಿತ ಗಳನ್ನೂ ನಿಧಾನವಾಗಿ ಹೋಗಲಾಡಿಸಿ, ಪರಸ್ಪರ ಅರಿವನ್ನುಂಟುಮಾಡಿಸುವಲ್ಲಿ ಸ್ವಾಮೀಜಿ ಯಶಸ್ವಿಯಾದರು.

ಕಲ್ಕತ್ತಕ್ಕೆ ಹಿಂದಿರುಗಿ ಬರುತ್ತಿದ್ದಂತೆಯೇ ಸ್ವಾಮೀಜಿ ಕೈಗೆತ್ತಿಕೊಂಡ ಮೊದಲ ಕೆಲಸಗಳ ಲ್ಲೊಂದೆಂದರೆ ಮಠದ ಕಾರ್ಯಕಲಾಪಗಳನ್ನು ವ್ಯವಸ್ಥಿತಗೊಳಿಸುವುದು. ಬಾರಾನಾಗೋರ್ ಮಠದ ದಿನಗಳಿಗಿಂತ ಅಂದಿನ ಆಲಂಬಜಾರ್ ಮಠದ ದಿನಗಳು ಹೆಚ್ಚೇನೂ ವಿಭಿನ್ನವಾಗಿರಲಿಲ್ಲ. ಈಗಲೂ ಆಧ್ಯಾತ್ಮಿಕ ಸಾಧನೆಯೊಂದೇ ಈ ಸಂನ್ಯಾಸಿಗಳ ಮುಖ್ಯ ಕಾರ್ಯಕ್ರಮವಾಗಿತ್ತು. ಆದರೆ ಸ್ವಾಮೀಜಿ ಇದನ್ನು ಒಪ್ಪುವವರಲ್ಲ. ಆಶ್ರಮವಾಸಿಗಳು ಸರ್ವತೋಮುಖ ಬೆಳವಣಿಗೆಯತ್ತ ಗಮನ ಹರಿಸಬೇಕೆಂದು ಅವರು ಒತ್ತಿ ಹೇಳಿದರು. ಪ್ರತಿಯೊಬ್ಬನೂ ಪ್ರತಿದಿನವೂ ನಿಯತವಾಗಿ ವ್ಯಾಯಾಮ ಮಾಡಬೇಕೆಂದು ಹೇಳಿದರಲ್ಲದೆ ಕೆಲವು ವ್ಯಾಯಾಮಗಳನ್ನು ತಾವೇ ಹೇಳಿಕೊಟ್ಟರು. ಶುಚಿತ್ವದ ಬಗ್ಗೆ ವಿಶೇಷ ಗಮನ ಕೊಡಬೇಕೆಂದು ಎಚ್ಚರಿಸಿದರು. ಆ ದಿನಗಳಲ್ಲಿ ನದಿಯ ಅಶುದ್ಧ ನೀರನ್ನು ಕುಡಿದು ಜನ ಹಲವಾರು ಕಾಯಿಲೆಗಳಿಗೆ ಬಲಿಯಾಗುತ್ತಿದ್ದರು. ಆದ್ದರಿಂದ ನೀರನ್ನು ಶೋಧಿಸಿ, ಕಾಯಿಸಿ, ಬಳಿಕವೇ ಕುಡಿಯಬೇಕೆಂದು ತಾಕೀತು ಮಾಡಿದರು.

ಬೆಳಿಗ್ಗೆ ಎಂಟರಿಂದ ಒಂಬತ್ತು ಗಂಟೆಯವರೆಗೆ ಸ್ವಾಮೀಜಿಯವರು, ಗೀತೆ, ಉಪನಿಷತ್ತು ಇಲ್ಲವೆ ಪ್ರಶ್ನೋತ್ತರಗಳ ತರಗತಿಯನ್ನು ನಡೆಸುತ್ತಿದ್ದರು. ಆಶ್ರಮವಾಸಿಗಳು ಇತರ ವೇಳೆಗಳಲ್ಲಿ ಜಪ ಧ್ಯಾನಾದಿಗಳನ್ನು ನಿಯತವಾಗಿ ಮಾಡಿಕೊಂಡು ಬರಬೇಕಾಗಿತ್ತು. ಚಂಡಿಯೇ ಮೊದಲಾದ ಸ್ತೋತ್ರಗಳ ಪಾರಾಯಣ-ಪಠಣಗಳಿಗೂ ಪ್ರೋತ್ಸಾಹವಿತ್ತು. ಸ್ವತಃ ಸ್ವಾಮೀಜಿ ಚಂಡೀ ಪಾರಾ ಯಣದ ಅಭ್ಯಾಸವಿಟ್ಟುಕೊಂಡಿದ್ದರು. ಒಮ್ಮೊಮ್ಮೆ ‘ಗೀತಗೋವಿಂದ’ದ ಶ್ಲೋಕಗಳನ್ನು ಹೇಳಿ ಕೊಳ್ಳುತ್ತ ಅವರು ದೈವೀಭಾವೋನ್ಮತ್ತರಾಗಿಬಿಡುತ್ತಿದ್ದರು. ಪಾಶ್ಚಾತ್ಯ ದೇಶಗಳಲ್ಲಿ ಮೂರು ವರ್ಷಗಳಿಗೂ ಹೆಚ್ಚು ಕಾಲ ಕಳೆದು ಹಿಂದಿರುಗಿದ ಸ್ವಾಮೀಜಿ ಹೀಗೆ ಭಕ್ತಿ ಪ್ರೇಮಭಾವರಂಜಿತ ರಾಗುವುದನ್ನು ಕಂಡು ಅವರ ಸ್ವಂತ ಗುರುಭಾಯಿಗಳಿಗೂ ವಿಸ್ಮಯ!

ಆದರೆ ಸ್ವಾಮೀಜಿಯವರ ಅತಿ ಮುಖ್ಯ ಉದ್ದೇಶವೊಂದು ಇನ್ನೂ ಈಡೇರಿರಲಿಲ್ಲ. ಅದೇ ಶ್ರೀರಾಮಕೃಷ್ಣ ಮಹಾಸಂಘದ ನಿರ್ಮಾಣ. ಅದರ ಬಗ್ಗೆ ಅವರು ಆಗಲೇ ಸಾಕಷ್ಟು ದೀರ್ಘ ಕಾಲ ಆಲೋಚಿಸಿದ್ದರು. ಇದೀಗ ಅದನ್ನು ಕಾರ್ಯಗತಗೊಳಿಸುವ ಸಮಯ ಸನ್ನಿಹಿತವಾಗಿತ್ತು. ಈಗ ಸ್ವಾಮೀಜಿಯವರ ಮುಂದಿದ್ದ ಪ್ರಥಮ ಕಾರ್ಯವೆಂದರೆ ತಮ್ಮೆಲ್ಲ ಸೋದರಸಂನ್ಯಾಸಿ ಗಳಿಗೆ ಈ ಸಂಘದ ಬಗ್ಗೆ, ಅದರ ಕಾರ್ಯೋದ್ದೇಶಗಳ ಬಗ್ಗೆ ತಿಳಿಸಿ ಅವರ ಮನವೊಲಿಸುವುದು; ಅವರ ಸಂಪೂರ್ಣ ಬೆಂಬಲ ಪಡೆದುಕೊಳ್ಳುವುದು. ಈ ಯುವ ಸಂನ್ಯಾಸಿಗಳನ್ನೆಲ್ಲ ಒಗ್ಗೂಡಿಸಲು ಹಿಂದೆ ಸ್ವಾಮೀಜಿ ಎಂತಹ ಅಸಾಧಾರಣ ಶಕ್ತಿಯನ್ನು ಪ್ರಯೋಗಿಸಬೇಕಾಯಿತೆಂಬುದನ್ನು ಈಗಾಗಲೇ ನೋಡಿದ್ದೇವೆ. ಇವರಲ್ಲಿ ಒಬ್ಬೊಬ್ಬರೂ ಒಂದೊಂದು ಶಕ್ತಿ. ಆದರೆ ಇವರೆಲ್ಲ ಸಮಾ ನೋದ್ದೇಶದಿಂದ ಸಂಘಟಿತರಾಗಿ ಶ್ರಮಿಸಿದಾಗ ಮಾತ್ರವೇ ಸಂಘದ ಉಳಿವು, ಬೆಳವಣಿಗೆ. ಆದ್ದರಿಂದಲೇ ಅಮೆರಿಕದಲ್ಲಿದ್ದಾಗಿನಿಂದಲೂ ಸ್ವಾಮೀಜಿ ಪತ್ರಗಳ ಮೂಲಕ ಈ ಕಾರ್ಯೋದ್ದೇಶ ಗಳ ಬಗ್ಗೆ ತಿಳಿಸುತ್ತ, ತಮ್ಮ ಸೋದರ ಸಂನ್ಯಾಸಿಗಳನ್ನು ಹುರಿದುಂಬಿಸುತ್ತಿದ್ದರು. ಆದರೆ ಅವರಿಗೆ ಈ ಯೋಜನೆಗಳೆಲ್ಲ ಅಷ್ಟೇನೂ ರುಚಿಸಲಿಲ್ಲ. ಶ್ರೀರಾಮಕೃಷ್ಣರ ಬೋಧನೆಗಳಿಗೂ ಈ ಯೋಜನೆ ಗಳಿಗೂ ಸಂಬಂಧವೇ ಇಲ್ಲ ಎಂದು ಅವರಿಗನ್ನಿಸಿತ್ತು. ಈಗ ಮಠಕ್ಕೆ ಮರಳಿದ ಸ್ವಾಮೀಜಿಯವ ರಿಗೆ ತಮ್ಮ ಸೋದರ ಸಂನ್ಯಾಸಿಗಳ ಈ ಭಾವನೆ ಅರ್ಥವಾಯಿತು. ತಾವು ಉದ್ದೇಶಿಸಿರುವ ಸಂಘದ ಕಾರ್ಯಯೋಜನೆಗಳು ನಿಜಕ್ಕೂ ಶ್ರೀರಾಮಕೃಷ್ಣರ ಬೋಧನೆಗಳಿಗೆ ಅನುಗುಣವಾಗಿವೆ, ಹಾಗೂ ಈಗ ತಾವದನ್ನು ಕೈಗೆತ್ತಿಕೊಳ್ಳದಿದ್ದರೆ ಅದು ಅಕ್ಷಮ್ಯ ದೋಷವಾಗುತ್ತದೆ ಎಂಬುದನ್ನು ತಮ್ಮ ಸೋದರ ಸಂನ್ಯಾಸಿಗಳಿಗೆ ಮನದಟ್ಟು ಮಾಡಿಸಲು ಅವರು ಬಹಳವಾಗಿಯೇ ಶ್ರಮಿಸಬೇಕಾಯಿತು.

ಈ ಸೋದರ ಸಂನ್ಯಾಸಿಗಳನ್ನು ಅವರ ಧೋರಣೆಗಾಗಿ ಒಮ್ಮೆಗೇ ಟೀಕಿಸಿ ಬಿಡುವಂತೆಯೂ ಇರಲಿಲ್ಲ. ಏಕೆಂದರೆ ಭಾರತದಲ್ಲಿ ವೇದಕಾಲದಿಂದಲೂ ಸಂನ್ಯಾಸಿಗಳ ಆದರ್ಶವೆಂದರೆ ಧ್ಯಾನ ಜಪ ತಪಾದಿಗಳ ಮೂಲಕ ಆತ್ಮಸಾಕ್ಷಾತ್ಕಾರಕ್ಕಾಗಿ ಶ್ರಮಿಸುವುದು; ಅದಕ್ಕಾಗಿ ತಮ್ಮೆಲ್ಲ ಶಕ್ತಿ ಯನ್ನೂ ವಿನಿಯೋಗಿಸುವುದು. ಹೀಗೆ ಸಾಧನೆಯಲ್ಲಿ ತೊಡಗಬೇಕಾದರೆ ಸಾಧಕ ತನ್ನನ್ನು ಈ ಪ್ರಪಂಚದ ಎಲ್ಲ ಸಂಬಂಧಗಳಿಂದಲೂ ಸಾಧ್ಯವಾದ ಮಟ್ಟಿಗೂ ಬಿಡಿಸಿಕೊಳ್ಳಬೇಕಾಗುತ್ತದೆ. ಆದರೆ ಸ್ವಾಮೀಜಿಯವರು ಈ ಸಂನ್ಯಾಸಿಗಳ ಮುಂದಿಟ್ಟ ಯೋಜನೆಯಲ್ಲಿ ಲೋಕಸೇವೆಗೆ ಪ್ರಮುಖ ಸ್ಥಾನ. ಸಂಪ್ರದಾಯವಿರುದ್ಧವಾದ ಈ ಧ್ಯೇಯವನ್ನು ಸ್ವೀಕರಿಸಲು ಈ ಸಂನ್ಯಾಸಿಗಳು ಒಮ್ಮೆಗೇ ಒಪ್ಪದಿದ್ದುದರಲ್ಲಿ ಅಚ್ಚರಿಯೇನೂ ಇಲ್ಲ. ಆದರೆ ಈಗ ಸ್ವಾಮೀಜಿಯವರು ಕಾರ್ಯ ಗತಗೊಳಿಸಲಿದ್ದುದು ಒಂದು ಕ್ರಾಂತಿಕಾರೀ ಯೋಜನೆಯನ್ನು. ಇಂಥದೊಂದನ್ನು ಚರಿತ್ರೆ ಕಂಡರಿಯದು. ಸ್ವಾಮೀಜಿ ಸಮಗ್ರಭಾರತವನ್ನು ಕಣ್ಣಾರೆ ಕಂಡವರು; ಜನರು ತಮಸ್ಸಿನ ಅಂಧಕಾರದಲ್ಲಿ ಮುಳುಗಿ ಮನುಷ್ಯತ್ವವನ್ನೇ ಕಳೆದುಕೊಂಡಿರುವುದನ್ನು ಕಂಡವರು; ಸಮಸ್ತ ಜಗತ್ತಿಗೇ ಬೆಳಕು ನೀಡಬಲ್ಲ ಸನಾತನ ಹಿಂದೂ ಧರ್ಮವು ತನ್ನೆಲ್ಲ ವೈಭವವನ್ನೂ ಕಳೆದುಕೊಂಡು ಘೋರಾಂಧಕಾರದಲ್ಲಿ ಮುಳುಗಿರುವುದನ್ನು ಕಂಡವರು. ಇಂತಹ ರಾಷ್ಟ್ರವನ್ನು ಮೇಲೆತ್ತುವವರು ಯಾರು? ಇಂತಹ ಜನರನ್ನು ಎಚ್ಚರಿಸುವವರು ಯಾರು? ಸ್ವಾಮೀಜಿಯವರು ಇದನ್ನೆಲ್ಲ ತಮ್ಮ ಸೋದರರ ಮುಂದೆ ಮನಮುಟ್ಟುವಂತೆ ಬಣ್ಣಿಸುತ್ತ, ಕಾರ್ಯಪ್ರವೃತ್ತರಾಗುವಂತೆ ಅವರನ್ನು ಅಗ್ನಿಸದೃಶ ಮಾತುಗಳಿಂದ ಪ್ರಚೋದಿಸಿದರು–“ನಿಮಗೆ ಇನ್ನೂ ನಿಮ್ಮಲ್ಲಿಯೇ ಆತ್ಮವಿಶ್ವಾಸ ಮೂಡಿಲ್ಲ; ಶ್ರೀರಾಮಕೃಷ್ಣರ ಕಾರ್ಯದಲ್ಲಿಯೇ ಶ್ರದ್ಧೆ ಮೂಡಿಲ್ಲ! ಮಹಾಕಾರ್ಯವೊಂದನ್ನು ಸಾಧಿಸಲು ಸಮರ್ಥವಾದ ಸಂಘವನ್ನು ರಚಿಸಿಕೊಳ್ಳಲು ಕೂಡ ನಿಮಗಿನ್ನೂ ಸಾಧ್ಯವಾಗಿಲ್ಲ!... ತಿಳಿದುಕೊಳ್ಳಿ–ನಿಮ್ಮಲ್ಲಿ ಒಬ್ಬೊಬ್ಬರೂ ಶಕ್ತಿ ಸಂಪನ್ನರು, ಪುರುಷಸಿಂಹರು. ನಿಮ್ಮ ಆ ಅಂತ ಶ್ಶಕ್ತಿಯನ್ನು ಪ್ರಕಟಗೊಳಿಸಿದ್ದೇ ಆದರೆ ನೀವು ಪ್ರಪಂಚವನ್ನೇ ಅಲುಗಾಡಿಸಬಲ್ಲಿರಿ. ನೀವೀಗ ಜಗತ್ತಿಗೆ ಹೊಸ ಬೆಳಕನ್ನು ನೀಡುವ ಕಾಲ ಸನ್ನಿಹಿತವಾಗಿದೆ. ದೀನ-ದಲಿತ-ದುಃಖಿಗಳಲ್ಲಿ ನೀವು ಸಾಕ್ಷಾತ್ ನಾರಾಯಣನನ್ನು ಕಂಡು ಸೇವೆಗೈದು, ಇತರರಿಗೆ ಮಾರ್ಗದರ್ಶನ ನೀಡುವಂತಾಗಬೇಕು. ಹೀಗೆ ಜನರಲ್ಲಿ ಸೇವಾ ಸ್ಫೂರ್ತಿಯನ್ನು ತುಂಬಬೇಕು. ಇತರರಿಗಾಗಿ ತಮ್ಮ ಜೀವನವನ್ನೇ ಮುಡಿಪಾಗಿಡುವ ಸಂನ್ಯಾಸಿಗಳ ಸಂಘವನ್ನು ಸ್ಥಾಪಿಸುವುದೇ ನನ್ನ ಜೀವನದ ಮುಖ್ಯ ಗುರಿ.”

ಸ್ವಾಮೀಜಿ ತಮ್ಮ ಮುಂದಿಟ್ಟ ಈ ವಿಚಾರಧಾರೆಯು ಆ ಸಂನ್ಯಾಸಿಗಳಿಗೆ ಸ್ಫೂರ್ತಿದಾಯಕ ವಾಗಿಯೇ ಇತ್ತು; ಅತ್ಯಂತ ಆಕರ್ಷಣೀಯವಾಗಿಯೇ ಇತ್ತು. ಆದರೆ ಅದು ಅಷ್ಟೇ ಕ್ರಾಂತಿಕಾರಕ ವಾಗಿಯೂ ಇತ್ತು. ಅವರ ಪಾಲಿಗೆ ಅದು ಬಿಸಿ ತುಪ್ಪ. ಆದರೆ ವಿವೇಕಾನಂದರ ಇಚ್ಛೆಗೆ ವಿರುದ್ಧ ವಾಗಿ ನಿಲ್ಲಬಲ್ಲವರಾರು? ವಿವೇಕಾನಂದರ ವಾಕ್ ಪ್ರವಾಹವನ್ನು ಎದುರಿಸಬಲ್ಲವರಾರು? ಆ ತರ್ಕವನ್ನು ಅಲ್ಲಗಳೆಯಲು ಯಾರಿಂದ ಸಾಧ್ಯ? ಅಲ್ಲದೆ ಸ್ವಾಮೀಜಿಯವರು ಶ್ರೀರಾಮಕೃಷ್ಣರ ಅವತಾರೋದ್ದೇಶವನ್ನು ಪೂರ್ಣಗೊಳಿಸಲೆಂದು ಅವರಿಂದಲೇ ನಿಯೋಜಿಸಲ್ಪಟ್ಟವರಲ್ಲವೆ? ಸ್ವಾಮೀಜಿಯವರು ಸ್ವಯಂ ಶ್ರೀರಾಮಕೃಷ್ಣರ ಪ್ರತಿನಿಧಿ ಎಂಬುದು ಮತ್ತೆ ಮತ್ತೆ ಸ್ಪಷ್ಟವಾಗು ತ್ತಿದೆ. ಹಾಗಿರುವಾಗ ಆ ಗುರುಭಾಯಿಗಳು ಸ್ವಾಮೀಜಿಯವರ ಮಾತನ್ನು ತಳ್ಳಿಹಾಕಲು ಹೇಗೆ ತಾನೆ ಸಾಧ್ಯ! ಇಷ್ಟೆಲ್ಲ ಆದರೂ ಸ್ವಾಮೀಜಿಯವರಿಗೆ ತಮ್ಮ ಗುರುಭಾಯಿಗಳ ಮನವೊಲಿಸು ವುದು ಅಷ್ಟೇನೂ ಸುಲಭವಾಗಲಿಲ್ಲ. ಅವರು ಸಾಕಷ್ಟು ಪ್ರತಿರೋಧವನ್ನೇ ಎದುರಿಸಬೇಕಾಯಿತು. ತಮ್ಮ ತೇಜಃಪುಂಜ ಬುದ್ಧಿ ಪ್ರಕಾಶದಿಂದ, ನಿಶಿತವಾದ ವಿಚಾರಸರಣಿಯಿಂದ ಅವರು ನಿಧಾನ ವಾಗಿ ತಮ್ಮ ಗುರುಭಾಯಿಗಳ ಮನವೊಲಿಸಿಕೊಂಡರು. ಶ್ರೀರಾಮಕೃಷ್ಣರ ದಿವ್ಯ ಸಂದೇಶಗಳನ್ನು ನೂತನ ದೃಷ್ಟಿಕೋನದಿಂದ ವ್ಯಾಖ್ಯಾನಿಸಿ ಹೇಳಿದರು. ಹೀಗೆ ತಮ್ಮ ಸೋದರ ಸಂನ್ಯಾಸಿಗಳ ದೃಷ್ಟಿಗೆ ಒಂದು ಹೊಸ ಬೆಳಕನ್ನು ನೀಡಿ, ಶ್ರೀರಾಮಕೃಷ್ಣರ ಅವತಾರೋದ್ದೇಶದ ಪರಿಪೂರ್ಣತೆ ಗಾಗಿ ಶ್ರಮಿಸುವುದೇ ಅವರ ಪಾಲಿನ ಆದ್ಯ ಕರ್ತವ್ಯ ಎಂಬ ಭಾವನೆಯನ್ನು ಅವರ ಮನದಲ್ಲಿ ಮುದ್ರೆಯೊತ್ತಿದರು. ನಿಷ್ಠಾಯುತ ಪ್ರೀತಿಪೂರ್ವಕ ಸೇವೆಯ ಮೂಲಕ ಭಾರತದ ಲಕ್ಷೋಪಲಕ್ಷ ಜನಸಾಮಾನ್ಯರನ್ನು ಮೇಲೆತ್ತಲು ಕಂಕಣಬದ್ಧರಾಗುವಂತೆ ಅವರನ್ನು ಪ್ರಚೋದಿಸಿದರು. ಸಮಸ್ತ ಜಗತ್ತಿನಲ್ಲಿ ಶ್ರೀರಾಮಕೃಷ್ಣರ ಉದ್ಬೋಧಕ ಸಂದೇಶಗಳನ್ನು ಬಿತ್ತರಿಸುವುದರ ಮೂಲಕ ಧಾರ್ಮಿಕ ಜಾಗೃತಿಯನ್ನುಂಟುಮಾಡುವಂತೆ ಪ್ರೋತ್ಸಾಹಿಸಿದರು.

ಹಿಂದೊಮ್ಮೆ ನರೇಂದ್ರ, ‘ನನಗೆ ಯಾವಾಗಲೂ ಸಮಾಧಿಯ ಸುಖದಲ್ಲೇ ಮುಳುಗಿದ್ದು ಬಿಡುವ ಆಸೆ’ ಎಂದು ಹೇಳಿದಾಗ ಶ್ರೀರಾಮಕೃಷ್ಣರು ಅವನಿಗೆ ಛೀಮಾರಿ ಹಾಕಿ, ‘ನೀನು ಸಮಸ್ತ ಜಗತ್ತಿಗೇ ಆಶ್ರಯ ನೀಡಬಲ್ಲ ಮಹಾ ವಟವೃಕ್ಷವಾಗಬೇಕು’ ಎಂದು ತಿಳಿಸಲಿಲ್ಲವೆ? ಅಂತೆಯೇ ಈಗ ಸ್ವಾಮೀಜಿ ತಮ್ಮ ಸೋದರ ಸಂನ್ಯಾಸಿಗಳನ್ನು ಪ್ರಚೋದಿಸುತ್ತ ಹೇಳುತ್ತಿದ್ದರು, “ಶ್ರೀರಾಮ ಕೃಷ್ಣರ ಶಿಷ್ಯರಾದ ನಿಮಗೆ ಕೇವಲ ನಿಮ್ಮ ವೈಯಕ್ತಿಕ ಮುಕ್ತಿಗಾಗಿ ಶ್ರಮಿಸುವುದು ಎಂದಿಗೂ ತಕ್ಕದ್ದಲ್ಲ. ಅಲ್ಲದೆ, ನೀವು ಶ್ರೀರಾಮಕೃಷ್ಣರ ಅಂತರಂಗ ಶಿಷ್ಯರು ಎಂಬ ಒಂದು ಕಾರಣ ಕ್ಕಾಗಿಯೇ ನಿಮ್ಮ ಮುಕ್ತಿ ಈಗಾಗಲೇ ನಿಶ್ಚಿತವಾಗಿದೆ. ಹೀಗಿರುವಾಗ ನೀವು ಕೇವಲ ನಿಮ್ಮ ಮುಕ್ತಿ ಗಾಗಿ ಸಾಧನೆ ಮಾಡುತ್ತ ಕುಳಿತುಕೊಳ್ಳದೆ ಇತರರ ಉದ್ಧಾರಕ್ಕಾಗಿ ಶ್ರಮಿಸಬೇಕು.”

ಅವರು ಇಷ್ಟೊಂದು ಹೇಳಿದ ಮೇಲೆ ಆ ಸೋದರ ಸಂನ್ಯಾಸಿಗಳಿಗೆ ಶ್ರೀರಾಮಕೃಷ್ಣರ ವಾಣಿಯೇ ವಿವೇಕಾನಂದರ ಮೂಲಕ ಪ್ರತಿಧ್ವನಿಸುತ್ತಿದೆ ಎಂಬುದು ದೃಢವಾಗುತ್ತ ಬಂತು; ಕ್ರಮೇಣ ಅವರ ಮಾತನ್ನು ಒಪ್ಪಿಕೊಳ್ಳಲಾರಂಭಿಸಿದರು. ಈಗ ಅವರು ತಮ್ಮ ಸಹಮಾನವರ ಹಿತಸಾಧನೆಗಾಗಿ ಸ್ವಾಮೀಜಿಯವರ ಆಜ್ಞೆಯ ಮೇರೆಗೆ ಯಾವ ಕಾರ್ಯವನ್ನಾದರೂ ಮಾಡಲು ಸಿದ್ಧರಾದರು, ಎಲ್ಲಿಗೆ ಹೋಗಲೂ ತಯಾರಾದರು.

ಈ ಸಂದರ್ಭದಲ್ಲಿ ಸ್ವಾಮೀಜಿಯವರ ಆಜ್ಞೆಯನ್ನು ಮನ್ನಿಸಿದವರಲ್ಲಿ, ಅಭೇದಾನಂದರು ಹಾಗೂ ಶಾರದಾನಂದರ ತರುವಾಯ, ಮೊದಲಿಗರೇ ಸ್ವಾಮಿ ರಾಮಕೃಷ್ಣಾನಂದರು. ಅವರು ತಮ್ಮ ಸಂನ್ಯಾಸಜೀವನದ ಹತ್ತು ವರ್ಷಗಳ ದೀರ್ಘ ಕಾಲ ತನುಮನ ಸಮರ್ಪಣಾಪೂರ್ವಕವಾಗಿ ಶ್ರೀರಾಮಕೃಷ್ಣರ ದೈನಂದಿನ ಪೂಜೆಯಲ್ಲೇ ಮುಳುಗಿದ್ದವರು. ಅವರು ಮಠವನ್ನು ಬಿಟ್ಟು ಹೊರಗೆ ಹೋದವರೇ ಅಲ್ಲ. ಶ್ರೀರಾಮಕೃಷ್ಣರ ಪೂಜೆಯೇ ತಮ್ಮ ಜೀವನದ ಪರಮ-ಚರಮ ಗುರಿ ಎಂಬ ಭಾವನೆಯಿಂದ ದೃಢವಾಗಿ ಕುಳಿತುಬಿಟ್ಟಿದ್ದವರು ಅವರು. ಅವರನ್ನು ಈಗ ಮದ್ರಾಸಿನಲ್ಲಿ ಮಠವೊಂದನ್ನು ಸ್ಥಾಪಿಸುವುದಕ್ಕಾಗಿ ಕಳಿಸಿಕೊಡಲು ಸ್ವಾಮೀಜಿ ಬಯಸಿದರು. ಆದರೆ ಪೂಜಾಗೃಹವನ್ನು ತೊರೆದು ಒಂದು ದಿನವೂ ಇರಲಿಚ್ಛಿಸದ ರಾಮಕೃಷ್ಣಾನಂದರು ಮದ್ರಾಸಿಗೆ ಹೋಗಲು ಒಪ್ಪುವರೇ ಎಂಬ ಶಂಕೆ ಸ್ವಾಮೀಜಿಯವರಿಗೆ. ಆದ್ದರಿಂದ, ಅವರಿಗೆ ತಮ್ಮ ಮೇಲಿನ ನಿಷ್ಠೆಯನ್ನು ಒರೆಹಚ್ಚಿ ನೋಡಲು ಒಂದು ಪರೀಕ್ಷೆಯಿಟ್ಟರು.

ಒಂದು ದಿನ ಸ್ವಾಮೀಜಿ ರಾಮಕೃಷ್ಣಾನಂದರನ್ನು ಬಳಿಗೆ ಕರೆದು, “ನೋಡು ಶಶಿ, ನನ್ನ ಮೇಲೆ ನಿನಗೆ ತುಂಬ ಪ್ರೀತಿ, ಅಲ್ಲವೆ?” ಎಂದು ಕೇಳಿದರು. “ಖಂಡಿತವಾಗಿಯೂ!” ಎಂಬ ಉತ್ತರ ಬರುತ್ತಲೇ ಸ್ವಾಮೀಜಿ ಹೇಳಿದರು, “ಸರಿ, ಹಾಗಾದರೆ ಈಗಲೇ ಹೋಗಿ ಪೌಜ್​ದಾರೀ ಬಲಖಾನಾ ದಿಂದ ನನಗಾಗಿ ಒಂದು ಪೌಂಡು ಒಳ್ಳೆ ಮೃದುವಾದ ಬ್ರೆಡ್ಡನ್ನು ತೆಗೆದುಕೊಂಡು ಬರುತ್ತೀಯಾ?” ಆ ಅಂಗಡಿ ಮುಸಲ್ಮಾನರದ್ದು. ಸಂಪ್ರದಾಯಸ್ಥ ಹಿಂದೂಗಳು ಆ ಅಂಗಡಿಯಿಂದ ಬ್ರೆಡ್ಡನ್ನು ಕೊಂಡಾರು ಎಂಬುದನ್ನು ಊಹಿಸಲೂ ಸಾಧ್ಯವಿರಲಿಲ್ಲ. ಇನ್ನು ರಾಮಕೃಷ್ಣಾನಂದರೋ ಆ ಸಂನ್ಯಾಸಿಗಳಲ್ಲೂ ಅತ್ಯಂತ ಸಂಪ್ರದಾಯಸ್ಥರು, ಮಡಿವಂತರು. ಇಂತಹ ಅವರು ಸ್ವಾಮೀಜಿ ಆಡಿದ ಮಾತನ್ನು ಕುಚೋದ್ಯವೆಂದು ತಳ್ಳಿ ಹಾಕದೆ, ಅಥವಾ ಹೀಗೆ ಹೇಳಿದರಲ್ಲ ಎಂದು ಕೋಪಿಸಿಕೊಳ್ಳದೆ ಮರುಕ್ಷಣವೇ ಅಂಗಡಿಗೆ ಹೊರಟುಬಿಟ್ಟರು! ಆಲಂಬಜಾರಿನಿಂದ ಐದಾರು ಮೈಲಿ ದೂರವಿದ್ದ ಆ ಸ್ಥಳಕ್ಕೆ ಹಾಡಹಗಲಲ್ಲೇ ನಡೆದುಕೊಂಡು ಹೋಗಿ ಬ್ರೆಡ್ಡನ್ನು ಕೊಂಡು ತಂದು ಸ್ವಾಮೀಜಿಯವರಿಗೆ ಒಪ್ಪಿಸಿದರು. ಅತ್ಯಾನಂದಿತರಾದ ಸ್ವಾಮೀಜಿ ಈಗ ಅವರಿಗೆ ತಮ್ಮ ಮನದ ಇಂಗಿತವನ್ನು ಮುಂದಿಟ್ಟರು, “ಸೋದರ, ನೀನು ಮದ್ರಾಸಿಗೆ ಹೋಗಿ ಒಂದು ಮಠವನ್ನು ಸ್ಥಾಪಿಸಬೇಕು. ನನಗಾಗಿ ನೀನಿದನ್ನು ಮಾಡಬಲ್ಲೆಯಾ?” ಇದಕ್ಕೆ ರಾಮಕೃಷ್ಣಾನಂದರು ಮರು ಮಾತಿಲ್ಲದೆ ಒಪ್ಪಿಕೊಂಡುಬಿಟ್ಟರು. ಆಗ ಸ್ವಾಮೀಜಿಯವರಿಗಾದ ಸಂತಸವೆಷ್ಟೆಂದು ಹೇಳೋಣ!

ಮಾರ್ಚ್ ಕಡೆಯ ವಾರದಲ್ಲಿ ರಾಮಕೃಷ್ಣಾನಂದರು ಮದ್ರಾಸಿಗೆ ತೆರಳಿದರು. ಅಷ್ಟು ಹೊತ್ತಿಗೆ ಅಮೆರಿಕದಿಂದ ಹಿಂದಿರುಗಿದ್ದ ಶಾರದಾನಂದರೂ ಅವರೊಂದಿಗೆ ಹೊರಟರು. ಅಲ್ಲಿ ಅವರು ಅವಿಶ್ರಾಂತ ಶ್ರಮದಿಂದ, ತಮ್ಮ ತಪಶ್ಶಕ್ತಿಯಿಂದ ರಾಮಕೃಷ್ಣ ಮಠವನ್ನು ಸ್ಥಾಪಿಸಲು ಸಮರ್ಥ ರಾದರು. ತನ್ಮೂಲಕ ಸಮಸ್ತ ದಕ್ಷಿಣ ಭಾರತದಲ್ಲಿ ರಾಮಕೃಷ್ಣ ಚಳುವಳಿಯು ಹಬ್ಬಲು ಕಾರಣರಾದರು.

ವಿವೇಕಾನಂದರ ಸೋದರ ಸಂನ್ಯಾಸಿಗಳ ಪೈಕಿ, ಅವರ ಜನಸೇವೆಯ ಆದರ್ಶವನ್ನು ಮೊದಲು ಕೈಗೆತ್ತಿಕೊಂಡು ಕಾರ್ಯಗತಗೊಳಿಸಲು ದುಡಿದ ಕೀರ್ತಿ ಸಲ್ಲುವುದು ಸ್ವಾಮಿ ಅಖಂಡಾನಂದರಿಗೆ. ೧೮೯೪ರಲ್ಲೇ, ಎಂದರೆ ಸ್ವಾಮೀಜಿ ಅಮೆರಿಕದಲ್ಲಿದ್ದಾಗಲೇ ಇವರು ಖೇತ್ರಿಯ ದೀನದಲಿತರಿಗಾಗಿ ಹಲವು ಬಗೆಯ ಸೇವಾಕಾರ್ಯಗಳನ್ನು ಪ್ರಾರಂಭಿಸಿದ್ದರು. ಈಗ ಇವರು ಮುರ್ಶಿದಾಬಾದ್ ಜಿಲ್ಲೆಯಲ್ಲಿ ಬರಗಾಲ ಪೀಡಿತರ ನೆರವಿಗಾಗಿ ಹೊಸ ಸೇವಾಕಾರ್ಯವನ್ನು ಕೈಗೊಳ್ಳಲು ತೆರಳಿದರು. ಹೀಗೆಯೇ ಇತರ ಸಂನ್ಯಾಸಿಗಳೂ ಸ್ವಾಮೀಜಿಯವರ ಆಜ್ಞೆಯನ್ನು ಶಿರಸಾವಹಿಸಲು, ಅವರು ತಮಗೊಪ್ಪಿಸುವ ಯಾವುದೇ ಜವಾಬ್ದಾರಿಯನ್ನು ನಿರ್ವಹಿಸಲು ಸಿದ್ಧರಾಗಿ ನಿಂತರು.

ಈ ವೇಳೆಯಲ್ಲಿ, ಸ್ವಾಮೀಜಿಯವರ ಮನಸ್ಸನ್ನು ಆವರಿಸಿಕೊಂಡಿದ್ದ ಒಂದು ವಿಷಯವೆಂದರೆ ಗಂಗೆಯ ತೀರದಲ್ಲಿ ಶಾಶ್ವತವಾದ ಮಠವೊಂದನ್ನು ನಿರ್ಮಿಸುವುದು. ಇದು ಅವರ ಬಹಳ ಕಾಲದ ಹೆಬ್ಬಯಕೆ. ಈಗಿನ ಆಲಂಬಜಾರಿನ ಮಠವನ್ನು ಬಿಟ್ಟು, ಒಳ್ಳೆಯ ಆರೋಗ್ಯಕರವಾದ ಸ್ಥಳದಲ್ಲಿ ವಿಶಾಲವಾದ ಭವ್ಯ ಕಟ್ಟಡವನ್ನು ಕಟ್ಟಲು ಅವರು ಯೋಜನೆ ಹಾಕುತ್ತಿದ್ದರು. ಆ ಮಠದ ಮೂಲಕ ಯಾವ ಯಾವ ಕಾರ್ಯಗಳೆಲ್ಲ ನಡೆಯಬೇಕಾಗಿದೆಯೆಂದು ಅವರು ಈಗಾಗಲೇ ಆಲೋಚಿಸಿ ದ್ದರು. ಮುಖ್ಯವಾಗಿ, ಹೊಸದಾಗಿ ಮಠಕ್ಕೆ ಸೇರುವ ಬ್ರಹ್ಮಚಾರಿಗಳಿಗೆ ಸರ್ವತೋಮುಖ ಶಿಕ್ಷಣ –ಅದರಲ್ಲೂ ಮುಖ್ಯವಾಗಿ ಆಧ್ಯಾತ್ಮಿಕ ಹಾಗೂ ನೈತಿಕ ಶಿಕ್ಷಣ ನೀಡಿ, ಅವರನ್ನು ಸಮರ್ಥ ಸಂನ್ಯಾಸಿಗಳನ್ನಾಗಿ ಸಿದ್ಧಪಡಿಸಬೇಕಾಗಿದೆ. ಮಾರ್ಗದರ್ಶನವನ್ನರಸಿ ಬರುವ ಭಕ್ತಾದಿಗಳಿಗೆ ಹಾಗೂ ಸಂದರ್ಶಕರಿಗೆ ಸರಿಯಾದ ವಿಚಾರಗಳನ್ನು ತಿಳಿಸಿಕೊಟ್ಟು ನೆರವಾಗಬೇಕಾಗಿದೆ. ಇನ್ನು ಸೇವಾ ಕಾರ್ಯಗಳಂತೂ ಇದ್ದೇ ಇವೆ. ಈ ನೂತನ ಮಠವನ್ನು ಕೇಂದ್ರಸ್ಥಾನವಾಗಿಟ್ಟುಕೊಂಡು ಮಹಾ ಸಂಸ್ಥೆಯೊಂದು ನಿರ್ಮಾಣಗೊಳ್ಳಬೇಕಾಗಿದೆ.

ಇವುಗಳೆಲ್ಲದರ ಬಗ್ಗೆ ಸ್ವಾಮೀಜಿಯವರ ಮನಸ್ಸಿನಲ್ಲಿ ಸ್ಪಷ್ಟ ಆಲೋಚನೆಗಳು ರೂಪತಾಳು ತ್ತಿದ್ದುವು. ಇವುಗಳಲ್ಲದೆ, ಹಿಮಾಲಯದ ನಿರ್ಜನ-ಪ್ರಶಾಂತ ಸ್ಥಳದಲ್ಲಿ ಸ್ಥಾಪಿಸಲು ಉದ್ದೇಶಿಸ ಲಾಗಿದ್ದ ಆಶ್ರಮದ ಸ್ವರೂಪದ ಬಗ್ಗೆಯೂ ಅವರು ದೀರ್ಘವಾಗಿ ಆಲೋಚಿಸಿದ್ದರು. ಜೊತೆಗೆ ಅಮೆರಿಕ ಹಾಗೂ ಇಂಗ್ಲೆಂಡಿನ ಕಾರ್ಯಗಳ ಕಡೆಗೂ ಅವರೇ ಗಮನ ಹರಿಸಬೇಕಾಗಿತ್ತು. ಆ ದೇಶಗಳಿಂದ ಅವರಿಗೆ ಪತ್ರಗಳ ಮೇಲೆ ಪತ್ರಗಳು ಬರುತ್ತಲೆ ಇದ್ದುವು. ಅವುಗಳ ಪಲ್ಲವಿ ಒಂದೇ: “ಇಲ್ಲಿ ಇನ್ನೂ ಹೆಚ್ಚಿನ ಕಾರ್ಯಾವಕಾಶಗಳು ಉಂಟಾಗುತ್ತಿವೆ. ನೀವು ಸಾಧ್ಯವಾದಷ್ಟು ಬೇಗ ಹಿಂದಿರುಗಿಬರಬೇಕು.”

ಹೀಗೆ ಕಲ್ಕತ್ತಕ್ಕೆ ಬಂದಂದಿನಿಂದ ಸ್ವಾಮೀಜಿಯವರಿಗೆ ಕೈತುಂಬ ಕೆಲಸ, ತಲೆ ತುಂಬ ಕೆಲಸ, ವಿಶ್ರಾಂತಿಯೇ ಇಲ್ಲದಷ್ಟು ಕೆಲಸ. ಆಗಲೇ ಸಂಪೂರ್ಣ ಜರ್ಜರಿತವಾಗಿದ್ದ ಅವರ ಶರೀರಕ್ಕೆ ಈ ಕೆಲಸಗಳೆಲ್ಲ ಅತಿಯಾಯಿತು. ಅವರು ತಕ್ಷಣವೇ ಸಂಪೂರ್ಣ ವಿಶ್ರಾಂತಿ ತೆಗೆದುಕೊಳ್ಳಬೇಕೆಂದು ವೈದ್ಯರು ಸಲಹೆ ಮಾಡಿದರು. ಇತ್ತ ಕಲ್ಕತ್ತದ ಸ್ವಾಗತ ಸಮಿತಿಯವರು ಸ್ವಾಮೀಜಿಯವರಿಂದ ಹಲವಾರು ಭಾಷಣಗಳನ್ನು ಮಾಡಿಸಬೇಕೆಂದು ಯೋಜನೆ ಹಾಕಿಕೊಂಡಿದ್ದರು. ಆದರೆ ವೈದ್ಯರ ಸಲಹೆಯನ್ನು ಸ್ವಾಮೀಜಿ ಮನ್ನಿಸಲೇಬೇಕಾಗಿತ್ತು. ಆದ್ದರಿಂದ ಕೆಲಕಾಲದ ಮಟ್ಟಿಗೆ ವಿಶ್ರಾಂತಿ ತೆಗೆದುಕೊಳ್ಳಲು ಅವರು ಮಾರ್ಚ್ ೮ರಂದು ಡಾರ್ಜಿಲಿಂಗಿಗೆ ತೆರಳುವುದೆಂದು ನಿಶ್ಚಯ ವಾಯಿತು.

ಮಾರ್ಚ್ ೭ರಂದು ಭಾನುವಾರ ಶ್ರೀರಾಮಕೃಷ್ಣರ ಜನ್ಮದಿನ. ಪ್ರತಿ ವರ್ಷದಂತೆ ದಕ್ಷಿಣೇ ಶ್ವರದ ಕಾಳೀ ದೇವಾಲಯದಲ್ಲೇ ಜಯಂತಿಯನ್ನು ನಡೆಸುವ ಸಿದ್ಧತೆ ನಡೆದಿತ್ತು. ಸ್ವಾಮಿ ವಿವೇಕಾನಂದರು ಈ ಸಂದರ್ಭದಲ್ಲಿ ಹಾಜರಿದ್ದುದು ಆ ವರ್ಷದ ವಿಶೇಷ. ಆದ್ದರಿಂದ ಅಂದಿನ ಜಯಂತ್ಯುತ್ಸವಕ್ಕೆ ಅಸಂಖ್ಯಾತ ಜನ ಆಗಮಿಸಿದರು. ಸುಮಾರು ೬ಂ,0ಂಂ ಜನ ಬಂದಿದ್ದರೆಂದು ಒಂದು ಅಂದಾಜು. ಬೆಳಿಗ್ಗೆ ಒಂಬತ್ತು ಗಂಟೆಯ ಹೊತ್ತಿಗೆ ಸ್ವಾಮೀಜಿ ತಮ್ಮ ಸೋದರಸಂನ್ಯಾಸಿ ಗಳೊಂದಿಗೆ ಕಾಳೀ ದೇವಾಲಯದ ಉದ್ಯಾನಕ್ಕೆ ಬಂದು ತಲುಪಿದರು. ಅಲ್ಲಿ ನೆರೆದಿದ್ದ ಅಪಾರ ಜನಸಾಗರವು ಸ್ವಾಮೀಜಿಯವರನ್ನು ನೋಡುತ್ತಲೇ ಉಚ್ಚಕಂಠದಿಂದ “ರಾಮಕೃಷ್ಣರಿಗೆ ಜಯ ವಾಗಲಿ! ವಿವೇಕಾನಂದರಿಗೆ ಜಯವಾಗಲಿ!” ಎಂದು ಮತ್ತೆ ಮತ್ತೆ ಜಯಕಾರ ಘೋಷಿಸಿತು. ಸ್ವಾಮೀಜಿಯವರನ್ನು ಹತ್ತಿರದಿಂದ ನೋಡಲು ಮತ್ತು ಅವರ ಪವಿತ್ರ ಪಾದಧೂಳಿಯನ್ನು ಸ್ವೀಕರಿಸಲು ಜನ ಅವರ ಬಳಿಗೆ ನುಗ್ಗುತ್ತಿದ್ದರು. ಸ್ವಾಮೀಜಿ ಕಾಳೀದೇವಾಲಯಕ್ಕೆ ಹೋಗಿ ಸಾಷ್ಟಾಂಗ ಪ್ರಣಾಮ ಮಾಡಿದರು. ಸ್ವಲ್ಪ ಹೊತ್ತಿನ ಮೌನ ಪ್ರಾರ್ಥನೆಯ ನಂತರ ಹೊರಬಂದು ತಮ್ಮ ಶಿಷ್ಯರೊಂದಿಗೆ ಯಾತ್ರಿಕರಂತೆ ಅಲ್ಲಿನ ಪ್ರತಿಯೊಂದು ಸ್ಥಳಕ್ಕೂ ನಡೆದುಹೋದರು. “ಇದು ಶ್ರೀರಾಮಕೃಷ್ಣರ ಕೋಣೆ” “ಇದು ನಹಬತ್ ಖಾನೆ” “ಇದು ಪಂಚವಟಿ” ಎಂದು ಅಲ್ಲಿನ ಸ್ಥಳಗಳನ್ನೆಲ್ಲ ಒಂದೊಂದಾಗಿ ತಮ್ಮ ಶಿಷ್ಯರಿಗೆ ತೋರಿಸಿದರು. ಅವರು ಹೋದಲ್ಲೆಲ್ಲ ಜನ ಹಿಂಬಾಲಿಸುತ್ತಿದ್ದರು. ಅಲ್ಲಲ್ಲಿ ಹಲವಾರು ಸಂಕೀರ್ತನೆಯ ತಂಡದವರು ಭಕ್ತ್ಯಾವೇಶಭರಿತರಾಗಿ ಭಗವನ್ನಾಮ ಸಂಕೀರ್ತನೆ ಮಾಡುತ್ತಿದ್ದರು.

ಭಗವಾನ್ ಶ್ರೀರಾಮಕೃಷ್ಣರ ಜನ್ಮದಿನದ ಈ ಶುಭ ಸಂದರ್ಭದಲ್ಲಿ ಅವರನ್ನು ಕುರಿತು ಸ್ವಾಮೀಜಿಯವರ ಬಾಯಲ್ಲಿ ನಾಲ್ಕು ಮಾತುಗಳನ್ನು ಕೇಳಬೇಕೆಂದು ಅಲ್ಲಿ ನೆರೆದಿದ್ದ ಸಾವಿರಾರು ಜನ ಅಪೇಕ್ಷಿಸಿದರು. ಜನರಿಂದ ಬೇಡಿಕೆ ಮತ್ತೆ ಮತ್ತೆ ಕೇಳಿ ಬಂದಾಗ, ಸ್ವಾಮೀಜಿ ಅದಕ್ಕೊಪ್ಪಿ ದರು. ನಿಜಕ್ಕೂ ಅಂದು ಅವರು ಮಾತನಾಡುವ ಸ್ಥಿತಿಯಲ್ಲಿರಲಿಲ್ಲ; ಆದರೆ ಬೇರೆ ಉಪಾಯ ವಿರಲಿಲ್ಲ. ಆದರೆ ಮಾತನಾಡಲು ಎದ್ದು ನಿಂತರೆ ಅದೇನು ಜನಸಂದಣಿ, ಅದೇನು ಗೌಜುಗದ್ದಲ! ಎಷ್ಟು ಗಟ್ಟಿಯಾಗಿ ಮಾತನಾಡಿದರೂ ಅದು ಜನರಿಗೆ ಕೇಳಿಸುವಂತೆಯೇ ಇಲ್ಲ! ಕಡೆಗೆ ಸ್ವಾಮೀಜಿ ಆ ಪ್ರಯತ್ನವನ್ನು ಕೈಬಿಡಬೇಕಾಯಿತು. ಬದಲಾಗಿ ಅವರು ಜನರೊಂದಿಗೆ ಬೆರೆತು, ಜನರ ನಮಸ್ಕಾರವನ್ನು ಸ್ವೀಕರಿಸಿದರು; ಪರಿಚಯಸ್ಥರನ್ನೆಲ್ಲ ಆತ್ಮೀಯವಾಗಿ ಮಾತನಾಡಿಸಿದರು; ತಮ್ಮ ಐರೋಪ್ಯ ಶಿಷ್ಯರನ್ನು ಇತರರಿಗೆ ಪರಿಚಯಿಸಿಕೊಟ್ಟರು. ಅಪರಾಹ್ನದ ವೇಳೆಗೆ ಜನಸಂದಣಿ ಕರಗಿತು. ತರುವಾಯ ಸ್ವಾಮೀಜಿ ತಮ್ಮ ಸಂಗಡಿಗರೊಂದಿಗೆ ಮಠಕ್ಕೆ ಹೊರಟರು.

ಗಾಡಿಯಲ್ಲಿ ಕುಳಿತು ಬರುವಾಗ ಸ್ವಾಮೀಜಿ ಈ ಬಗೆಯ ಉತ್ಸವಗಳ ವೈಶಿಷ್ಟ್ಯದ ಬಗ್ಗೆ ಹೇಳಿದರು. ಶಾಸ್ತ್ರಗಳಲ್ಲಿ ಪ್ರಣೀತವಾದ ಸೂಕ್ಷ್ಮ ತತ್ತ್ವಗಳನ್ನು ಅಧ್ಯಯನ ಮಾಡದ-ಮಾಡ ಲಾಗದ ಸಾಮಾನ್ಯರಿಗೆ ಈ ಬಗೆಯ ಉತ್ಸವಗಳು ಎಷ್ಟು ಸಹಾಯಕಾರಿ ಎಂಬುದನ್ನು ವಿವರಿಸಿ ದರು. ಆಗ ಅಲ್ಲಿದ್ದ ಶರಚ್ಚಂದ್ರ ಚಕ್ರವರ್ತಿ ಈ ಮಾತನ್ನು ಸಂಪೂರ್ಣವಾಗಿ ಒಪ್ಪದೆ, ತನ್ನ ಶಂಕೆಯನ್ನು ಮುಂದಿಟ್ಟ: “ಸ್ವಾಮೀಜಿ, ಹೀಗೆ ಶ್ರೀರಾಮಕೃಷ್ಣರ ಹೆಸರಿನಲ್ಲಿ ಉತ್ಸವಾದಿಗಳನ್ನು ಆಚರಿಸುವುದರ ಮೂಲಕ ನೀವು ಮತ್ತೊಂದು ಮತವನ್ನೇ ಹುಟ್ಟುಹಾಕಬಹುದಲ್ಲವೆ?” ಸ್ವತಃ ಶ್ರೀರಾಮಕೃಷ್ಣರು ಸರ್ವಧರ್ಮಗಳನ್ನೂ ಸಮನ್ವಯಗೊಳಿಸಲೆಂದೆ ಬಂದಿರುವಾಗ, ಅವರ ಹೆಸರಿ ನಲ್ಲೇ ಮತ್ತೊಂದು ಮತ ಹುಟ್ಟಿಕೊಂಡರೆ ಅದೊಂದು ವಿಪರ್ಯಾಸವಾಗುತ್ತದೆಯೆಂದು ಶರಚ್ಚಂದ್ರನ ಅಭಿಪ್ರಾಯ. ಇದನ್ನು ಕೇಳಿ ಸ್ವಾಮೀಜಿಯವರಿಗೆ ತಮಾಷೆಯೆನಿಸಿತು. ಬಳಿಯ ಲ್ಲಿದ್ದ ನಿರಂಜನಾನಂದರತ್ತ ತಿರುಗಿ “ನೋಡು, ಈ ‘ಬಂಗಾಳಿ’\footnote{*‘ಬಂಗಾಳಿ’ ಎಂಬ ಪದಪ್ರಯೋಗವು ಪೂರ್ವ ಬಂಗಾಳದ (ಈಗಿನ ಬಾಂಗ್ಲಾದೇಶದ) ಜನರಿಗೆ ಹೆಚ್ಚಾಗಿ ಅನ್ವಯವಾಗುತ್ತದೆ. ಪೂರ್ವ ಬಂಗಾಳದ ಜನ ಸರಳರು ಪ್ರಾಮಾಣಿಕರು ಶ್ರಮಜೀವಿಗಳು; ಆದರೆ ನಯ ನಾಜೂಕಿಲ್ಲದವರು. ಇವರನ್ನು ಒರಟರೆಂದು ಛೇಡಿಸಲು ಪಶ್ಚಿಮ ಬಂಗಾಳದವರು ‘ಬಂಗಾಳಿ’ ಎಂದು ಸಂಬೋಧಿ ಸುತ್ತಾರೆ–ನಮ್ಮಲ್ಲಿ ಹಳ್ಳಿಗರನ್ನು ‘ಗಮಾರ’ ಎನ್ನುವುದಿಲ್ಲವೆ? ಹಾಗೆ.} ಏನು ಹೇಳುತ್ತಿದ್ದಾನೆ!” ಎಂದರು. ಬಳಿಕ ಶರಚ್ಚಂದ್ರನ ಶಂಕೆಯನ್ನು ದೂರಗೊಳಿಸುತ್ತ ಹೇಳಿದರು, “ನೋಡು, ಅದು ಹಾಗೆಂದಿಗೂ ಆಗಲಾರದು. ನಾನು ಎಲ್ಲೆಲ್ಲೂ ಬೋಧಿಸಿದುದು ಕೇವಲ ಪರಿಶುದ್ಧವಾದ ಉಪನಿಷತ್ ಧರ್ಮವನ್ನು ಮಾತ್ರವೇ.”

ಕಲ್ಕತ್ತದ ಉರಿಬಿಸಿಲಿನಿಂದ ಪಾರಾಗಿ ಕೆಲಕಾಲ ವಿಶ್ರಾಂತಿ ಪಡೆಯುವ ಸಲುವಾಗಿ ಸ್ವಾಮೀಜಿ ಯವರು ಮಾರ್ಚ್ ಎಂಟರಂದು ಡಾರ್ಜಿಲಿಂಗಿಗೆ ಹೊರಟರು. ಅವರೊಂದಿಗೆ ಅವರ ಶಿಷ್ಯರ ಹಾಗೂ ಸಂನ್ಯಾಸೀ ಬಂಧುಗಳ ದೊಡ್ಡ ತಂಡವೇ ಹೊರಟಿತು. ಹೀಗೆ ಅವರೊಂದಿಗೆ ಹೊರಟವ ರೆಂದರೆ ಬ್ರಹ್ಮಾನಂದರು, ತುರೀಯಾನಂದರು, ತ್ರಿಗುಣಾತೀತಾನಂದರು, ಜ್ಞಾನಾನಂದರು, ಸೇವಿ ಯರ್ ದಂಪತಿಗಳು, ಗುಡ್​ವಿನ್, ಗಿರೀಶ್​ಚಂದ್ರ ಘೋಷ್, ಡಾ. ಟರ್ನ್​ಬುಲ್ ಹಾಗೂ ಮದ್ರಾಸೀ ಶಿಷ್ಯರಾದ ಅಳಸಿಂಗ ಪೆರುಮಾಳ್, ಜಿ. ಜಿ. ನರಸಿಂಹಾಚಾರ್ಯ ಮತ್ತು ಸಿಂಗಾರ ವೇಲು ಮೊದಲಿಯಾರ್. ‘ಗಿರಿಧಾಮಗಳ ರಾಣಿ’ ಎಂದು ಹೆಸರಾದ ಈ ಡಾರ್ಜಿಲಿಂಗಿನಲ್ಲಿ ಸ್ವಾಮೀಜಿ ಹಾಗೂ ಅವರ ಸಂಗಡಿಗರು ಶ್ರೀ ಎಂ. ಎನ್. ಮುಖರ್ಜಿ ಎಂಬವರ ಅತಿಥಿಗಳಾಗಿ ಉಳಿದುಕೊಂಡರು. ಅಲ್ಲದೆ, ಸ್ವಾಮೀಜಿಯವರ ಭಕ್ತನಾದ ಬರ್ದ್ವಾನಿನ ಮಹಾರಾಜ ತನ್ನ ಅರಮನೆಯ ಒಂದು ಭಾಗವನ್ನು ಅವರಿಗಾಗಿ ಬಿಟ್ಟುಕೊಟ್ಟ.

ಡಾರ್ಜಿಲಿಂಗಿನ ಸುಂದರ-ಪ್ರಶಾಂತ ವಾತಾವರಣದಲ್ಲಿ ಸ್ವಾಮೀಜಿ ವಿಶ್ರಾಂತಿ ಪಡೆಯ ಲಾರಂಭಿಸಿದರು. ಕಲ್ಕತ್ತದಲ್ಲಿ ಅವರನ್ನು ಪರೀಕ್ಷಿಸಿದ ವೈದ್ಯರು ಅವರಿಗೆ ಡಯಾಬಿಟೀಸ್ (ಸಿಹಿಮೂತ್ರ ರೋಗ) ಉಂಟಾಗಿದೆ ಎಂದು ಕಂಡುಕೊಂಡಿದ್ದರು. ಆದ್ದರಿಂದ ಈಗ ಸ್ವಾಮೀಜಿ ಕಟ್ಟುನಿಟ್ಟಾದ ಪಥ್ಯಪಾನದಲ್ಲಿರಬೇಕಾಯಿತು. ಡಯಾಬಿಟೀಸಿಗೆ ಪರಿಹಾರವಾದ ಇನ್ಸುಲಿನ್ನನ್ನು ಆಗಿನ್ನೂ ಕಂಡು ಹಿಡಿದಿರಲಿಲ್ಲ. ಈ ಕಾಯಿಲೆಯನ್ನು ನಿಯಂತ್ರಿಸದಿದ್ದರೆ ಅದು ರೋಗಿಯನ್ನು ಹಣ್ಣು ಮಾಡಿಬಿಡುತ್ತದೆ. ಅಲ್ಲದೆ ರೋಗಿಯ ಆಯುಸ್ಸೂ ಕಡಿಮೆಯಾಗುತ್ತದೆ. ಆಗಲೇ ಈ ಕಾಯಿಲೆ ಸ್ವಾಮೀಜಿಯವರ ಶರೀರಕ್ಕೆ ಸಾಕಷ್ಟು ಹಾನಿಯುಂಟುಮಾಡಿತ್ತು. ನಿಜ ಹೇಳಬೇಕೆಂದರೆ, ಅದು ಸರಿಪಡಿಸಲಾಗದಂತಹ ಹಾನಿ. ಅವರ ಇಹಜೀವನದ ಆಯುಸ್ಸು ಸರಿಯಾಗಿ ಇನ್ನೈದು ವರ್ಷಗಳಷ್ಟೆ. ತಾವಿನ್ನು ಹೆಚ್ಚು ಕಾಲ ಜೀವಿಸುವುದಿಲ್ಲವೆಂದು ಸ್ವಾಮೀಜಿಯವರಿಗೂ ತಿಳಿದಿತ್ತು. ಏಪ್ರಿಲ್ ೨೮ರಂದು ತಮ್ಮ ಪ್ರಿಯ ‘ಸೋದರಿ’ ಮೇರಿಗೆ ಬರೆದ ಒಂದು ಪತ್ರದಲ್ಲಿ ಸ್ವಾಮೀಜಿ, ಡಾರ್ಜಿಲಿಂಗಿನಲ್ಲಿ ತಾವು ಕಳೆದ ದಿನಗಳ ಬಗ್ಗೆ ತಿಳಿಸಿದರು. ಬಳಿಕ ತಮ್ಮ ದೇಹಸ್ಥಿತಿಯ ಬಗ್ಗೆ ಹಾಸ್ಯವಾಗಿ ಬರೆದರು: “ನನ್ನ ತಲೆಗೂದಲು ಒಟ್ಟೊಟ್ಟಾಗಿ ಬೂದು ಬಣ್ಣಕ್ಕೆ ತಿರುಗುತ್ತಿದೆ; ಮುಖದಲ್ಲಿ ಗೆರೆಗಳು ಕಾಣಿಸಿಕೊಳ್ಳುತ್ತಿವೆ... ಆದರೆ ಈ ಬೆಟ್ಟ ಪ್ರದೇಶದಲ್ಲಿ ನಾನು ಬಹಳ ಚೆನ್ನಾಗಿದ್ದೇನೆ... ಈಗ ನಾನು ದೊಡ್ಡ ಬಿಳಿಯಗಡ್ಡ ಬೆಳೆಸಬೇಕೆಂದಿದ್ದೇನೆ. ಅದರಿಂದ ಒಳ್ಳೇ ಗಂಭೀರವಾದ, ಪೂಜ್ಯ ಬುದ್ಧಿಯನ್ನು ಹುಟ್ಟಿಸುವಂತಹ ಕಳೆ ಬರುತ್ತದೆ! ಆಗ ಅಮೆರಿಕದ ಅಪನಿಂದೆ ಹರಡುವ ಜನರಿಂದ ಬಚಾವಾಗಬಹುದು! ‘ಓ ನರೆಗೂದಲೆ, ನೀನು ಎಷ್ಟೊಂದನ್ನು ಮರೆಮಾಚಬಲ್ಲೆ! ನಿನಗೆ ಜಯವಾಗಲಿ!’ ಆದರೆ ನಿಜಕ್ಕೂ ಅವರ ಕೂದಲು ಬೆಳ್ಳಗಾಗಿರಲಿಲ್ಲ ವೆಂದು ಕಾಣುತ್ತದೆ. ಆದ್ದರಿಂದ ಸುಮಾರು ಎರಡು ವರ್ಷಗಳ ಬಳಿಕ ಬರೆದ ಮತ್ತೊಂದು ಪತ್ರದಲ್ಲಿ ಅವರು ಆ ಬಗ್ಗೆ ‘ವಿಷಾದ’ದಿಂದ ಪ್ರಸ್ತಾಪಿಸಿದರು: “ನನ್ನ ತಲೆಗೂದಲು ಬಹಳ ವೇಗವಾಗಿ ಬೆಳ್ಳಗಾಗತೊಡಗಿತ್ತು. ಆದರೆ ಹೇಗೊ ಏನೋ ಅದು ನಿಂತುಹೋಯಿತು. ಅಯ್ಯೋ, ಈಗೆಲ್ಲೋ ಕೆಲವು ಬಿಳಿಕೂದಲುಗಳು ಮಾತ್ರ ಉಳಿದುಕೊಂಡಿವೆ. ಆದರೆ ಸರಿಯಾಗಿ ಸಂಶೋಧನೆ ನಡೆಸಿದರೆ ಮತ್ತಷ್ಟು ಕಾಣಸಿಕ್ಕಾವು!”

ಡಾರ್ಜಿಲಿಂಗಿನಲ್ಲಿ ಸ್ವಾಮೀಜಿ ತಮ್ಮ ಸಂಗಡಿಗರೊಂದಿಗೆ ಸಂಭಾಷಿಸುತ್ತ, ಧ್ಯಾನಮಗ್ನ ರಾಗಿರುತ್ತ, ಕಾಲಯಾಪನೆ ಮಾಡುತ್ತಿದ್ದರು. ಆಗಾಗ ಆ ಹಿಮಾಚಲದ ರಮ್ಯ ಪ್ರದೇಶಗಳಲ್ಲಿ ಓಡಾಡುತ್ತಿದ್ದರು. ಒಮ್ಮೆ ಅಲ್ಲಿನ ಒಂದು ಬೌದ್ಧ ವಿಹಾರವನ್ನು ಸಂದರ್ಶಿಸಿದರು. ಹೀಗೆ ಅವರ ದಣಿದ ದೇಹ ಮನಸ್ಸುಗಳಿಗೆ ಸ್ವಲ್ಪ ವಿಶ್ರಾಂತಿ ಸಿಗುವಂತಾಯಿತು.

ಸ್ವಾಮೀಜಿಯವರ ಆತಿಥೇಯರಾದ ಎಂ. ಎನ್. ಮುಖರ್ಜಿಯವರ ಮನೆಯಲ್ಲಿ ಮೋತೀ ಲಾಲ್ ಎಂಬುವರೊಬ್ಬರು ಇದ್ದರು. ಆ ಸಮಯದಲ್ಲಿ ಅವರು ತೀವ್ರಜ್ವರದಿಂದ ನರಳು ತ್ತಿದ್ದರು. ಜ್ವರ ಉಲ್ಬಣಿಸಿ ಸನ್ನಿಗೆ ತಿರುಗಿಕೊಂಡಿತು; ಬುದ್ಧಿಗೆಟ್ಟಂತಾಗಿ ಬಡಬಡಿಸಲಾರಂಭಿ ಸಿದರು. ವೈದ್ಯಕೀಯ ಚಿಕಿತ್ಸೆಯೇನೋ ನಡೆಯುತ್ತಿತ್ತು. ಆದರೆ ಏನೂ ಸುಧಾರಣೆ ಉಂಟಾಗಲಿಲ್ಲ. ಇದನ್ನು ಕಂಡ ಸ್ವಾಮೀಜಿ ತುಂಬ ಕರುಣೆಯಿಂದ ಮೋತೀಲಾಲರ ಹಣೆಯನ್ನೊಮ್ಮೆ ಸವರಿದರು. ಆಶ್ಚರ್ಯ! ಶೀಘ್ರವೇ ಜ್ವರ ಬಿಟ್ಟು ಹೋಯಿತು; ರೋಗಿ ಸುಧಾರಿಸಿಕೊಂಡರು. ಇದನ್ನು ಕಂಡು ಅಲ್ಲಿದ್ದವರೆಲ್ಲ ಅತ್ಯಾಶ್ಚರ್ಯಪಟ್ಟರು.

ಮೋತೀಲಾಲ್ ಮುಖರ್ಜಿಯವರು ಸ್ವಭಾವತಃ ಭಾವಾವೇಶದ ವ್ಯಕ್ತಿ. ಸಂಕೀರ್ತನೆಯ ಸಮಯದಲ್ಲಿ ಇವರು ಭಕ್ತ್ಯಾಧಿಕ್ಯದಿಂದ ಕಣ್ಣೀರು ಸುರಿಸುತ್ತ ನೆಲದ ಮೇಲೆ ಹೊರಳಾಡುತ್ತಿದ್ದರು. ಕೈಕಾಲುಗಳನ್ನು ನೆಲಕ್ಕೆ ಅಪ್ಪಳಿಸುತ್ತಿದ್ದರು. ಇಲ್ಲಿಯವರೆಗಿತ್ತು ಅವರ ಭಾವಾವೇಶದ ಭರಾಟೆ. ಸಂಕೀರ್ತನೆಯಲ್ಲಿ ಶಾಂತವಾಗಿ ಆನಂದದಿಂದ ಭಾಗವಹಿಸುತ್ತಿರುವವರಿಗೆಲ್ಲ ಇದೊಂದು ರಗಳೆ. ಇದನ್ನು ಗಮಿನಿಸಿದ ಸ್ವಾಮೀಜಿ ಒಂದು ದಿನ ಅವರ ಎದೆಯನ್ನು ಸ್ಪರ್ಶಿಸಿದರು. ಆಗಿನಿಂದ ಅವರ ಆಧ್ಯಾತ್ಮಿಕ ಭಾವದಲ್ಲೇ ಒಂದು ದೊಡ್ಡ ಬದಲಾವಣೆಯಾಗಿಬಿಟ್ಟಿತು. ಅಂದಿನಿಂದಲೇ ಅವರ ಭಾವಾವೇಶದ ಅಬ್ಬರವೆಲ್ಲ ನಿಂತುಹೋಯಿತು. ಭಕ್ತಿಸ್ವಭಾವದವರಾಗಿದ್ದ ಮೋತೀಲಾಲರು ಅದ್ವೈತಿಯಾಗಿ ಜ್ಞಾನಯೋಗದ ಅಭ್ಯಾಸದಲ್ಲಿ ನಿರತರಾದರು. ಅಷ್ಟೇ ಅಲ್ಲ, ಮುಂದೆ ಅವರು ಸಂನ್ಯಾಸ ಸ್ವೀಕರಿಸಿ ‘ಸ್ವಾಮಿ ಸಚ್ಚಿದಾನಂದ’ರಾದರು.

ಸ್ವಾಮೀಜಿ ಡಾರ್ಜಿಲಿಂಗಿಗೆ ಬಂದು ಎಂಟು ದಿನಗಳಾಗಿದ್ದುವಷ್ಟೇ. ಆಗ ಖೇತ್ರಿ ಮಹಾರಾಜ ಅಜಿತ್​ಸಿಂಗನು ಅವರನ್ನು ಕಾಣಲು ಕಲ್ಕತ್ತಕ್ಕೆ ಬರುತ್ತಿದ್ದಾನೆಂಬ ತಂತಿ ವರ್ತಮಾನ ಬಂದಿತು. ಸ್ವಾಮೀಜಿಯವರ ಪರಮಶಿಷ್ಯನಾದ ಈತ ಅವರ ದರ್ಶನಕ್ಕಾಗಿ ಉತ್ಕಂಠಿತನಾಗಿದ್ದ. ಸ್ವಾಮೀಜಿ ಯವರನ್ನು ತನ್ನ ರಾಜ್ಯಕ್ಕೆ ಬರಮಾಡಿಕೊಂಡು ವೈಭವದಿಂದ ಸತ್ಕರಿಸಬೇಕೆಂಬ ಆಸೆಯೇನೋ ಅವನಿಗಿತ್ತು. ಆದರೆ ತಮ್ಮ ಆಗಿನ ದೇಹಸ್ಥಿತಿಯಲ್ಲಿ ಸ್ವಾಮೀಜಿ ಪ್ರಯಾಣ ಮಾಡುವಂತಿರಲಿಲ್ಲ. ಆದ್ದರಿಂದ ಅವರ ದರ್ಶನಕ್ಕಾಗಿ ಅಜಿತ್​ಸಿಂಗ್ ಪರಿವಾರ ಸಮೇತನಾಗಿ ತಾನೇ ಹೊರಟು ಬಂದಿದ್ದ. ಮಠದ ಪರವಾಗಿ ಸ್ವಾಮಿ ಶಿವಾನಂದರು ಹಾಗೂ ಸ್ವಾಮಿ ತ್ರಿಗುಣಾತೀತಾನಂದರು ಮತ್ತು ಬಹು ಸಂಖ್ಯೆಯಲ್ಲಿ ಸೇರಿದ್ದ ಮಾರ್ವಾಡಿಗಳು, ಉತ್ತರ ಭಾರತೀಯರು ಹಾಗೂ ಕಲ್ಕತ್ತದ ಪ್ರಮುಖ ನಾಗರಿಕರು ರೈಲು ನಿಲ್ದಾಣದಲ್ಲಿ ಮಹಾರಾಜನನ್ನು ಸ್ವಾಗತಿಸಿದರು. ಬಳಿಕ ಅವನನ್ನು ಮೆರವಣಿಗೆಯಲ್ಲಿ ಕರೆತರಲಾಯಿತು.

ರಾಜಾ ಅಜಿತ್​ಸಿಂಗ್ ಕಲ್ಕತ್ತಕ್ಕೆ ಬಂದ ವರ್ತಮಾನ ತಲುಪುತ್ತಲೇ ಸ್ವಾಮೀಜಿ ಡಾರ್ಜಿಲಿಂಗಿ ನಲ್ಲಿನ ತಮ್ಮ ವಿಶ್ರಾಂತಿಯ ಕಾರ್ಯಕ್ರಮವನ್ನು ಮೊಟಕುಗೊಳಿಸಿ ಕೂಡಲೇ ಹೊರಟುಬಂದರು. ಅವರನ್ನು ಎದುರ್ಗೊಳ್ಳಲು ಮಹಾರಾಜ ಹಾಗೂ ಅವನ ಪರಿವಾರದವರಲ್ಲದೆ, ಕಲ್ಕತ್ತದ ಸಮಸ್ತ ಮಾರ್ವಾಡಿಗಳೂ ನಿಲ್ದಾಣದಲ್ಲಿ ಸೇರಿಬಿಟ್ಟಿದ್ದರು! ಟ್ರೈನು ಬಂದು ನಿಲ್ಲುತ್ತಲೇ ಮಹಾರಾಜನು ಸ್ವಾಮೀಜಿಯವರಿದ್ದ ಬೋಗಿಯೊಳಗೆ ಪ್ರವೇಶಿಸಿ ಅವರಿಗೆ ಸಾಷ್ಟಾಂಗ ಪ್ರಣಾಮ ಮಾಡಿದ. ಅವನ ಗುರುಭಕ್ತಿಯ ತೀವ್ರತೆಯನ್ನು ಕಂಡು ಸ್ವಾಮೀಜಿಯವರ ಹೃದಯ ತುಂಬಿಬಂದಿತು. ಬಳಿಕ ಆತ ಅವರ ಪಾದಗಳನ್ನು ತೊಳೆದು ಹೂವುಗಳಿಂದ ಪೂಜಿಸಿ ಹಾರವನ್ನು ಸಮರ್ಪಿಸಿದ. ಸ್ವಾಮೀಜಿ ಹೊರಗೆ ಬಂದಕೂಡಲೇ ಅವರಿಗೆ ಅಲ್ಲಿದ್ದ ಲೊಹರು ಸಂಸ್ಥಾನದ ನವಾಬನನ್ನು ಹಾಗೂ ಇತರ ಗಣ್ಯ ವ್ಯಕ್ತಿಗಳನ್ನು ಪರಿಚಯಿಸಿಕೊಡಲಾಯಿತು.

ಬಳಿಕ ಅಲ್ಲಿ ನೆರೆದಿದ್ದವರ ಸಮ್ಮುಖದಲ್ಲಿ ಮಹಾರಾಜ ತನ್ನ ಕಡೆಯಿಂದ ಒಂದು ಬಿನ್ನವತ್ತಳೆ ಯನ್ನು ಓದಿ ಸಮರ್ಪಿಸಿದ. ತರುವಾಯ ಅವರನ್ನು ತನ್ನ ವಿಶೇಷ ಸಾರೋಟಿಗೆ ಕರೆದೊಯ್ದ. ನಿಲ್ದಾಣದಿಂದ ಮಹಾರಾಜನ ನಿವಾಸದವರೆಗೆ ಸುಮಾರು ಇನ್ನೂರು ಸಾರೋಟುಗಳ ಭಾರೀ ಮೆರವಣಿಗೆ ಸಾಗಿತು.

ಸ್ವಲ್ಪ ಹೊತ್ತಿನ ಬಳಿಕ ಮಹಾರಾಜ ಸ್ವಾಮೀಜಿಯವರ ಗೌರವಾರ್ಥವಾಗಿ ವಿಶೇಷ ದರ್ಬಾರನ್ನು ನಡೆಸಿದ. ಸ್ವಾಮೀಜಿಯವರನ್ನು ತನ್ನ ಸಿಂಹಾಸನದ ಪಕ್ಕದಲ್ಲಿಯೇ ಒಂದು ಆಸನದಲ್ಲಿ ಕುಳ್ಳಿರಿಸಿ ‘ನಜರ್​’ ಸಮರ್ಪಿಸಿದ. (ನಜರ್ ಎಂದರೆ ಚಿನ್ನದ ನಾಣ್ಯವನ್ನು ಕೈಗಿತ್ತು ನಮಸ್ಕರಿಸುವ ಸಂಪ್ರದಾಯ) ಅನಂತರ ಇತರರೊಂದಿಗೆ ತಾನೂ ಜಮಖಾನೆಯ ಮೇಲೆ ಕುಳಿತುಕೊಂಡ. ಬಳಿಕ ಮಹಾರಾಜನ ಸೂಚನೆಯಂತೆ, ಅಲ್ಲಿದ್ದ ಇತರ ಗಣ್ಯರು ಒಬ್ಬೊಬ್ಬರಾಗಿ ಬಂದು ಸ್ವಾಮೀಜಿಯವರಿಗೆ ನಜರ್ ಸಮರ್ಪಿಸಿದರು.

ಅಂದು ಸಂಜೆ ಸ್ವಾಮೀಜಿ ಮಹಾರಾಜನೊಂದಿಗೆ ದಕ್ಷಿಣೇಶ್ವರಕ್ಕೆ ತೆರಳಿದರು. ಅಲ್ಲಿ ಅವರನ್ನು ಕಾಳೀ ದೇವಾಲಯದ ಖಜಾಂಚಿಗಳು ಎದುರ್ಗೊಂಡು ಸ್ವಾಗತಿಸಿದರು. ಶ್ರೀರಾಮಕೃಷ್ಣರ ಲೀಲಾ ಸ್ಥಾನವನ್ನು ಕಂಡು ಮಹಾರಾಜನಿಗೆ ಬಹಳ ಆನಂದವಾಯಿತು. ಬಳಿಕ ಅಲ್ಲಿಂದ ಹಿಂದಿರುಗಿ ಬರುವಾಗ ಮಹಾರಾಜ ಆಲಂಬಜಾರ್ ಮಠವನ್ನು ಸಂದರ್ಶಿಸಿದ. ಪೂಜಾಗೃಹದಲ್ಲಿ ವಿರಾಜ ಮಾನರಾದ ಶ್ರೀರಾಮಕೃಷ್ಣರ ಭಾವಚಿತ್ರಕ್ಕೆ ಭಕ್ತಿಯಿಂದ ಸಾಷ್ಟಾಂಗ ಪ್ರಣಾಮ ಮಾಡಿದ. ಅನಂತರ ಸ್ವಾಮೀಜಿಯವರ ಮುಂದೆ ಮಂಡಿಯೂರಿ ಕುಳಿತು ಬಹಳ ಹೊತ್ತಿನವರೆಗೆ ಸಂಭಾಷಿ ಸಿದ. ಅಜಿತ್​ಸಿಂಗನ ಸರಳ ಉಡಿಗೆತೊಡಿಗೆ ಸರಳ ನಡೆನುಡಿಗಳನ್ನು ಕಂಡು ಆಶ್ರಮವಾಸಿಗಳಿಗೆಲ್ಲ ಬಹಳ ಸಂತೋಷವಾಯಿತು. ಅಂದು ಶ್ರೀರಾಮಕೃಷ್ಣರಿಗೆ ವಿಶೇಷ ನೈವೇದ್ಯ ಮಾಡಿಸಲಾಯಿತು. ಮಹಾರಾಜ ಅಲ್ಲೇ ಪ್ರಸಾದ ಸ್ವೀಕರಿಸಿದ. ಅಂದಿನ ರಾತ್ರಿಯನ್ನು ಸ್ವಾಮೀಜಿ ಅಜಿತ್​ಸಿಂಗ ನೊಂದಿಗೆ ಅವನ ಬಂಗಲೆಯಲ್ಲೇ ಕಳೆದರು. ಸ್ವಾಮೀಜಿಯವರ ಸಂದರ್ಶನದಿಂದ ಮಹಾರಾಜ ಸಂಪ್ರೀತನಾದ. ಮಾರ್ಚ್ ೨೩ರಂದು ಸ್ವಾಮೀಜಿ ಮತ್ತೆ ಡಾರ್ಜಿಲಿಂಗಿನತ್ತ ಪಯಣಿಸಿದರು. ಮೂರು ದಿನಗಳ ಬಳಿಕ ಅಜಿತ್​ಸಿಂಗನೂ ಖೇತ್ರಿಯತ್ತ ಹೊರಟ.

ಡಾರ್ಜಿಲಿಂಗಿನಲ್ಲಿ ಸ್ವಾಮೀಜಿಯವರ ಆರೋಗ್ಯ ಅಷ್ಟೇನೂ ವೇಗವಾಗಿ ಸುಧಾರಿಸಲಿಲ್ಲ. ಆಗಾಗ ಅವರು ತಾವೀಗ ಆರೋಗ್ಯವಂತರಾಗಿದ್ದೇವೆ ಎಂದು ಭಾವಿಸಿದರೂ ಡಾಕ್ಟರರು ಅದನ್ನು ಒಪ್ಪಲು ಸಿದ್ಧರಿರಲಿಲ್ಲ. ಡಾಕ್ಟರುಗಳು ಎಷ್ಟರ ಮಟ್ಟಿಗೆ ಎಚ್ಚರಿಕೆ ಹೇಳಿದರೆಂದರೆ ಸ್ವಾಮೀಜಿ ಯಾವ ಪುಸ್ತಕವನ್ನೂ ಓದುವಂತಿರಲಿಲ್ಲ. ಅಷ್ಟೇ ಅಲ್ಲ, ಗಾಢವಾಗಿ ಆಲೋಚನೆಯನ್ನೂ ಕೂಡ ಮಾಡುವಂತಿರಲಿಲ್ಲ. ಈ ವಿಷಯವಾಗಿ ಅವರು ಶ್ರೀಮತಿ ಸಾರಾ ಬುಲ್ಲಳಿಗೆ ಒಂದು ಪತ್ರದಲ್ಲಿ ಬರೆದರು: “ಇಲ್ಲಿನ ಉತ್ಸವ ಸಂಭ್ರಮಗಳೆಲ್ಲ ಮುಗಿದುವು. ಅಥವಾ ನನ್ನ ಆರೋಗ್ಯ ಸಂಪೂರ್ಣ ಹದಗೆಟ್ಟದ್ದರಿಂದ ನಾನೇ ಅವುಗಳನ್ನು ಮೊಟಕುಗೊಳಿಸಿದೆ ಎನ್ನಬೇಕು. ಪಾಶ್ಚಾತ್ಯ ರಾಷ್ಟ್ರಗಳಲ್ಲಿ ಮಾಡಿದ ನಿರಂತರ ಕಾರ್ಯಗಳಿಂದ ಹಾಗೂ ಭಾರತದಲ್ಲಿ ಈ ಒಂದು ತಿಂಗಳ ಕಾಲ ಮಾಡಿದ ನಿರಂತರ ಪ್ರಯಾಣ-ಭಾಷಣ-ಸಂಭಾಷಣೆಗಳಿಂದ ಉಂಟಾದ ಅತೀವ ಶ್ರಮದ ಫಲವಾಗಿ ನನ್ನ ಬಂಗಾಳೀ ಶರೀರಕ್ಕೆ ಸಿಕ್ಕಿದ ಫಲ ಡಯಾಬಿಟೀಸ್​–ಇದು ನನ್ನ ಮನೆತನದ ಶತ್ರು. ಇನ್ನು ಕೆಲವು ವರ್ಷಗಳಲ್ಲೇ ಇದು ನನ್ನನ್ನು ಆಹುತಿ ತೆಗೆದುಕೊಳ್ಳದೆ ಬಿಡುವುದಿಲ್ಲ. ಪಥ್ಯಪಾನ, ಮಿದುಳಿಗೆ ಸಂಪೂರ್ಣ ವಿಶ್ರಾಂತಿ–ಇವುಗಳಿಂದ ಮಾತ್ರವೇ ನನ್ನ ಕೊನೆಯ ದಿನವನ್ನು ಸ್ವಲ್ಪ ಮುಂದೂಡ ಬಹುದಾಗಿದೆ. ಈಗ ಡಾರ್ಜಿಲಿಂಗಿನಲ್ಲಿ ಮಿದುಳಿಗೆ ಈ ವಿಶ್ರಾಂತಿಯನ್ನು ಸ್ವಲ್ಪ ಕೊಡಿಸು ತ್ತಿದ್ದೇನೆ.”

ಈ ವಿಶ್ರಾಂತಿಯಿಂದ ಸ್ವಾಮೀಜಿಯವರ ಶರೀರಕ್ಕೆ ಎಷ್ಟೋ ಅನುಕೂಲವಾಯಿತಾದರೂ ಅದು ಅವರಿಗೆ ಅಸಹನೀಯವಾಗಿ ಪರಿಣಮಿಸಿತ್ತು. ಪ್ರಚಂಡ ಕ್ರಿಯಾಶೀಲರಾದ ಅವರಿಗೆ ಹೀಗೆ ಮೈಗಳ್ಳರಂತೆ ಕುಳಿತುಕೊಳ್ಳುವುದೆಂದರೆ ಮರಣಕ್ಕಿಂತಲೂ ಕಡೆ. ಡಾಕ್ಟರು ಏನೇ ಹೇಳಿದರೂ ಎಷ್ಟೇ ಹೇಳಿದರೂ ಸ್ವಾಮೀಜಿ ಗಾಢವಾಗಿ ಆಲೋಚನೆ ಮಾಡದೆ ಶಾಂತವಾಗಿ ಕುಳಿತಿರಲು ಸಾಧ್ಯವೇ ಇರಲಿಲ್ಲ. ಭಾರತದ ಅಸಂಖ್ಯಾತ ದೀನದಲಿತರಿಗಾಗಿ ಅವರ ಹೃದಯ ಹೇಗೆ ಮರುಗುತ್ತಿತ್ತು ಎಂಬುದು ಈ ಸಮಯದಲ್ಲಿ ಅವರು ಇಂಗ್ಲೆಂಡಿನಲ್ಲಿದ್ದ ಮಿಸ್ ಮಾರ್ಗರೆಟ್ ನೋಬೆಲ್ಲಳಿಗೆ (ಇವಳೇ ಮುಂದೆ ಸೋದರಿ ನಿವೇದಿತಾ) ಬರೆದ ಪತ್ರದಿಂದ ವ್ಯಕ್ತವಾಗುತ್ತದೆ:

“ನಾನು ನಿನಗಾಗಿ ಒಂದು ಸಣ್ಣ ಆದರೆ ಅತ್ಯಂತ ಮಹತ್ವದ ಕೆಲಸವೊಂದನ್ನು ಹುಡುಕಿಟ್ಟಿ ದ್ದೇನೆ. ಅದು, ಭಾರತದ ದೀನದಲಿತರ ಸೇವೆಗೆ ಸಂಬಂಧಿಸಿದುದು.

“ನಾನು ಈಗ ನಿನಗೆ ಪರಿಚಯಿಸಿಕೊಡುತ್ತಿರುವ ಮಹನೀಯರು ಕೇರಳದ ಈಡಿಗ ಕೋಮಿನ ಪ್ರತಿನಿಧಿಯಾಗಿ ಇಂಗ್ಲೆಂಡಿಗೆ ಬಂದಿದ್ದಾರೆ. ಈಡಿಗರದ್ದು ಅಲ್ಲಿನ ಅತ್ಯಂತ ಹಿಂದುಳಿದ ಕೋಮು. ಮೇಲ್ಜಾತಿಯ ಹಿಂದೂಗಳಿಂದ ಇವರು ನಿರಂತರ ಶೋಷಣೆಗೊಳಗಾಗಿದ್ದಾರೆ. ಈ ವಿಷಯ ವನ್ನೆಲ್ಲ ನಿನಗೆ ಆ ಪ್ರತಿನಿಧಿಗಳು ತಿಳಿಸುತ್ತಾರೆ. ಈಗ ಇವರಿಗೆ ನೆರವಾಗುವವರು ಯಾರೂ ಇಲ್ಲ. ಇಲ್ಲಿನ ರಾಜರ ಒಳಾಡಳಿತದ ವಿಷಯದಲ್ಲಿ ತಾನು ತಲೆಹಾಕುವಂತಿಲ್ಲ ಎಂಬ ನೆಪವೊಡ್ಡಿ ಭಾರತ ಸರ್ಕಾರವೂ ಸುಮ್ಮನಿದೆ. ಈಗ ಈ ಜನರ ಒಂದೇ ಒಂದು ಆಶಾಕಿರಣವೆಂದರೆ ಇಂಗ್ಲೆಂಡಿನ ಪಾರ್ಲಿಮೆಂಟು. ಆದ್ದರಿಂದ ಈ ವಿಷಯವು ಬ್ರಿಟಿಷ್ ಸಾರ್ವಜನಿಕರ ಗಮನಕ್ಕೆ ಬರುವಂತೆ ದಯವಿಟ್ಟು ನೀನು ನಿನ್ನ ಕೈಲಾದುದನ್ನೆಲ್ಲ ಮಾಡು.”

ಈ ಸಮಯದಲ್ಲಿ ಹಗಲಿರುಳೂ ಸ್ವಾಮೀಜಿಯವರ ತಲೆಯಲ್ಲಿ ತುಂಬಿದ್ದ ಇನ್ನೊಂದು ವಿಚಾರವೆಂದರೆ, ಅವರು ಸ್ಥಾಪಿಸಲು ಉದ್ದೇಶಿಸಿದ್ದ ರಾಮಕೃಷ್ಣ ಮಹಾಸಂಘದ ವಿಷಯ. ಈ ವಿಷಯವಾಗಿ ಅವರು ತಮ್ಮ ಸೋದರ ಸಂನ್ಯಾಸಿಗಳೊಂದಿಗೆ ವಿವರವಾಗಿ ಚರ್ಚಿಸಿದರು. ಅವರು ಗಿರೀಶ್​ಚಂದ್ರನನ್ನು ಡಾರ್ಜಿಲಿಂಗಿಗೆ ಕರೆದೊಯ್ದದ್ದು ಆ ಉದ್ದೇಶದಿಂದಲೇ. ಬಹಳಷ್ಟು ಚರ್ಚೆಯ ಬಳಿಕ ಸ್ವಾಮೀಜಿ, ಸಂಘದ ಒಂದು ಸ್ಥೂಲ ರೂಪುರೇಷೆಯನ್ನು ರಚಿಸಿದರು. ಕಲ್ಕತ್ತಕ್ಕೆ ತೆರಳಿದ ಬಳಿಕ ಅದನ್ನು ಪೂರ್ಣಗೊಳಿಸುವುದೆಂಬ ನಿರ್ಧಾರಕ್ಕೆ ಬರಲಾಯಿತು. ಸಂನ್ಯಾಸ ಧರ್ಮಕ್ಕೆ, ಸಂನ್ಯಾಸ ಸಂಪ್ರದಾಯಕ್ಕೆ ಒಂದು ಹೊಸ ತಿರುವನ್ನು ಕೊಡಬೇಕೆಂಬುದು ಅವರ ಉದ್ದೇಶವಾಗಿತ್ತು. ಅವರೊಮ್ಮೆ ತುರೀಯಾನಂದರಿಗೆ ಹೇಳಿದರು, “ನಾನೊಂದು ಹೊಸ ಮಾದರಿಯ ಸಂನ್ಯಾಸಿಗಳ ಸಂಘವನ್ನು ನಿರ್ಮಾಣ ಮಾಡುವವನಿದ್ದೇನೆ. ಈ ದಿಸೆಯಲ್ಲಿ ನಾನೊಂದು ಕ್ರಾಂತಿಯನ್ನೇ ಎಬ್ಬಿಸಲಿದ್ದೇನೆ” ಎಂದು. ನಿಜಕ್ಕೂ ಸ್ವಾಮೀಜಿಯವರು ನವಯುಗದ ಧರ್ಮಕ್ಕೆ ಅನುಸಾರವಾಗಿ ಸಂನ್ಯಾಸ ಸಂಪ್ರದಾಯಕ್ಕೆ ಹೊಸ ರೂಪವೊಂದನ್ನು ಕೊಡಲು ಸಮರ್ಥರಾದರು.

