
\chapter{ನಾಯಕತ್ವದ ದಾರಿ ನಯವಲ್ಲ}

\noindent

ಡಾರ್ಜಿಲಿಂಗಿನಲ್ಲಿ ಸುಮಾರು ಒಂದು ತಿಂಗಳನ್ನು ಕಳೆದು ಸ್ವಾಮೀಜಿಯವರು ತಮ್ಮ ಪರಿವಾರ ದೊಂದಿಗೆ ಕಲ್ಕತ್ತಕ್ಕೆ ಹಿಂದಿರುಗಿದರು. ಆದರೆ ಆಗ ಕಲ್ಕತ್ತದಲ್ಲಿ ಉರಿಬಿಸಿಲು. ಈಗಲಾದರೂ ಸ್ವಾಮೀಜಿ ಕೆಲವು ಭಾಷಣಗಳನ್ನು ಮಾಡಬಹುದೆಂದು ಕಲ್ಕತ್ತದ ಜನ ಆಸೆಯಿಂದ ನಿರೀಕ್ಷಿಸಿ ಕೊಂಡಿದ್ದರು. ಆದರೆ ಕಲ್ಕತ್ತದ ಆ ಉರಿಬಿಸಿಲಿನಲ್ಲಿ ಸ್ವಾಮೀಜಿ ಭಾಷಣ ಮಾಡುವ ಪ್ರಶ್ನೆಯೇ ಇರಲಿಲ್ಲ. ಮಠದಲ್ಲಿ ಕೆಲದಿನಗಳ ಮಟ್ಟಿಗಷ್ಟೇ ಇದ್ದು ಮತ್ತೆ ಹಿಮಾಲಯ ಪ್ರದೇಶಕ್ಕೆ ಹೊರಟುಬಿಡಬೇಕೆಂಬುದೇ ಅವರ ಯೋಜನೆಯಾಗಿತ್ತು.

ಸ್ವಾಮೀಜಿಯವರು ಮಠದಲ್ಲಿದ್ದಾಗ ಬೇರೆಲ್ಲ ಕಡೆಗಳಿಗಿಂತಲೂ ಆನಂದದಿಂದಿದ್ದರು. ತಮ್ಮ ಪ್ರೀತಿಯ ಸೋದರ ಸಂನ್ಯಾಸಿಗಳೊಂದಿಗೆ ಹಾಗೂ ನೆಚ್ಚಿನ ಶಿಷ್ಯರೊಂದಿಗೆ ಒಡನಾಡುವುದು ಅವರಿಗೆ ಬಹಳ ಪ್ರಿಯವಾದದ್ದು. ಇಂತಹ ಶುದ್ಧ ಹೃದಯದ ತ್ಯಾಗಿಗಳೊಂದಿಗೆ ನಾಗರಿಕತೆಯ ಎಲ್ಲ ಶಿಷ್ಟಾಚಾರಗಳನ್ನು ಬದಿಗಿಟ್ಟು ಸರಳವಾಗಿ ಸರಸವಾಗಿ ಇರುವುದೆಂದರೆ ಅವರಿಗೆ ತುಂಬ ಇಷ್ಟ. ಸ್ವಾಮೀಜಿಯವರ ವೈರಾಗ್ಯ ಪ್ರಚೋದಕವಾದ, ಸ್ಫೂರ್ತಿದಾಯಕವಾದ ಮಾತುಗಳನ್ನು ಕೇಳಿದ ಹಲವಾರು ಯುವಕರು ಮಠಕ್ಕೆ ಸೇರಿಕೊಂಡಿದ್ದರು. ಇವರಿಗೆಲ್ಲ ಸ್ವಾಮೀಜಿ ವಿವಿಧ ವಿಷಯಗಳಲ್ಲಿ ತರಬೇತಿ ನೀಡಿದರು; ಭಗವದ್ಗೀತೆ-ವೇದಾಂತಗಳ ಮೇಲೆ ತರಗತಿಗಳನ್ನು ತೆಗೆದು ಕೊಂಡರು. ಸ್ವಾಮೀಜಿ ಅಮೆರಿಕದಲ್ಲಿದ್ದಾಗಲೇ ನಾಲ್ವರು ಬ್ರಹ್ಮಚಾರಿಗಳು ಮಠಕ್ಕೆ ಸೇರಿಕೊಂಡಿ ದ್ದರು. ಅವರು ಇತರ ಹಿರಿಯ ಸಂನ್ಯಾಸಿಗಳಿಂದ ಅನೇಕ ವರ್ಷಗಳ ಕಾಲ ಆಧ್ಯಾತ್ಮಿಕ ಶಿಕ್ಷಣವನ್ನು ಪಡೆದಿದ್ದರು. ಈ ಬ್ರಹ್ಮಚಾರಿಗಳು ಸ್ವಾಮಿ ವಿವೇಕಾನಂದರಿಂದಲೇ ಸಂನ್ಯಾಸ ಸ್ವೀಕರಿಸಲು ಬಯಸಿದ್ದರು. ಅವರು ಸಂನ್ಯಾಸ ಸ್ವೀಕಾರಕ್ಕೆ ತಕ್ಕ ಅರ್ಹತೆಯನ್ನು ಪಡೆದಿರುವುದನ್ನು ಗಮನಿಸಿದ ಸ್ವಾಮೀಜಿ, ಅವರಿಗೆ ಸಂನ್ಯಾಸ ನೀಡಿ ತಮ್ಮ ಶಿಷ್ಯರನ್ನಾಗಿ ಸ್ವೀಕರಿಸಲು ಸಂತೋಷದಿಂದ ಒಪ್ಪಿಕೊಂಡರು.

ಆದರೆ ಈ ನಾಲ್ವರು ಶಿಷ್ಯರ ಪೈಕಿ ಒಬ್ಬನಿಗೆ ಮಾತ್ರ ಸಂನ್ಯಾಸ ಕೊಡಕೂಡದು ಎಂದು ಕೆಲವು ಹಿರಿಯ ಸಂನ್ಯಾಸಿಗಳು ಕಟುವಾಗಿ ಹೇಳಿದರು. ಅವನ ಹಿಂದಿನ ಚರಿತ್ರೆ ಚೆನ್ನಾಗಿಲ್ಲವೆಂಬುದು ಅವರ ದೂರು. ಆ ಸಂನ್ಯಾಸಿಗಳ ಮಾತನ್ನು ಕೇಳಿ ಸ್ವಾಮೀಜಿ ವ್ಯಗ್ರಗೊಂಡು ಹೇಳಿದರು, “ಏನಿದೆಲ್ಲ! ಪತಿತರನ್ನು ನಾವೇ ದೂರ ಮಾಡಿದರೆ ಅವರನ್ನು ಕಾಪಾಡುವವರು ಯಾರು? ಅಲ್ಲದೆ ಒಬ್ಬ ಮನುಷ್ಯ ತಾನು ಇನ್ನು ಮುಂದೆ ಉತ್ತಮ ಜೀವನ ನಡೆಸಬೇಕೆಂಬ ಇಚ್ಛೆಯಿಂದ ಈ ಮಠಕ್ಕೆ ಸೇರಬೇಕೆಂದು ಬಂದಾಗಲೇ ಸ್ಪಷ್ಟವಾಗುತ್ತದೆ–ಅವನ ಉದ್ದೇಶ ಒಳ್ಳೆಯದೇ ಆಗಿದೆ ಎಂದು. ಆದ್ದರಿಂದ ನಾವು ಅವನ ಉದ್ಧಾರಕ್ಕೆ ನೆರವಾಗಲೇಬೇಕು. ಒಂದು ವೇಳೆ ಅವನ ಶೀಲ ಚೆನ್ನಾಗಿಲ್ಲವೆಂದೇ ಇಟ್ಟುಕೊಳ್ಳಿ; ಆದರೂ ನಿಮಗೆ ಅವನನ್ನು ತಿದ್ದಿ ಪರಿವರ್ತನೆಗೊಳಿಸಲು ಸಾಧ್ಯ ವಿಲ್ಲದಿದ್ದರೆ ನೀವು ಈ ಕಾಷಾಯವನ್ನು ಧರಿಸಿದ್ದೇಕೆ? ಈ ಗುರುಸ್ಥಾನಕ್ಕೆ ಬಂದದ್ದಾದರೂ ಏಕೆ?” ಹೀಗೆ ಸ್ವಾಮೀಜಿ ಆ ಶಿಷ್ಯನ ಸಂಪೂರ್ಣ ಜವಾಬ್ದಾರಿಯನ್ನು ತಾವೇ ವಹಿಸಿಕೊಂಡರು. (ಆದರೆ ಸ್ವಾಮಿ ವಿವೇಕಾನಂದರು ಮಾತ್ರ ಹಾಗೆ ಹೇಳಬಹುದು, ಹಾಗೆ ಅಂಥವರಿಗೆಲ್ಲ ಸಂನ್ಯಾಸ ಕೊಡಬಹುದು. ಈಗ ಅದೆಲ್ಲ ಸಾಧ್ಯವಿಲ್ಲ. ಶೀಲದಲ್ಲಿ ಸ್ವಲ್ಪ ಕುಂದು-ಕೊರತೆ ಕಂಡುಬಂದರೂ ರಾಮಕೃಷ್ಣ ಮಹಾಸಂಘದಲ್ಲಿ ಅಂಥವರಿಗೆ ಪ್ರವೇಶ ಸಿಗಲಾರದು. ಚೊಕ್ಕ ಬಂಗಾರವನ್ನು ಮಾತ್ರವೇ ಸ್ವೀಕರಿಸುವ ಪ್ರಯತ್ನ ನಡೆಯುತ್ತಿದೆ ಇಂದು.)

ಅಂತೂ ಸ್ವಾಮೀಜಿ ಆ ನಾಲ್ವರಿಗೆ ಸಂನ್ಯಾಸ ಕೊಡಲು ನಿರ್ಧರಿಸಿದರು. ಸಂನ್ಯಾಸ ದೀಕ್ಷೆಯ ದಿನ ಬಂದಿತು. ಅಂದು ಸ್ವಾಮೀಜಿ ಆ ದೀಕ್ಷಾರ್ಥಿಗಳ ಮುಂದೆ ತ್ಯಾಗ ಜೀವನದ ವೈಭವವನ್ನು ಉಜ್ವಲವಾಗಿ ವರ್ಣಿಸಿದರು. ಆಗ ಅವರ ಕಣ್ಣುಗಳಲ್ಲಿ ಅಗ್ನಿ ಪ್ರಜ್ವಲಿಸುತ್ತಿತ್ತು. ಅವರಾಡಿದ ಶಬ್ದಗಳು ಶಿಷ್ಯರ ಹೃದಯದಲ್ಲಿ ನವಸ್ಫೂರ್ತಿಯನ್ನು ತುಂಬುತ್ತಿದ್ದುವು. ತ್ಯಾಗಜೀವನದ ಮಹತ್ವದ ಬಗ್ಗೆ ಸ್ವಾಮೀಜಿ ಸುದೀರ್ಘವಾಗಿಯೇ ಮಾತನಾಡಿದರು. ಮಾತಿನ ಕೊನೆಯಲ್ಲಿ ತಮ್ಮ ಭಾವನೆಯ ಸಾರವನ್ನೇ ಸುರಿಸಿದರು: “ನೆನಪಿಡಿ, ಸಂನ್ಯಾಸಿಗಳ ಜೀವನವೆಂಬುದು ಕೇವಲ ತಮ್ಮ ಸ್ವಂತ ಮುಕ್ತಿಗಲ್ಲದೆ ಇತರರ ಶ್ರೇಯಸ್ಸಿಗಾಗಿಯೂ ಮುಡಿಪಾಗಿರಬೇಕು–ಆತ್ಮನೋ ಮೋಕ್ಷಾರ್ಥಂ ಜಗದ್ಧಿತಾಯ ಚ –ಸಂನ್ಯಾಸಿಯ ಜನ್ಮವೇ ಜಗತ್ತಿನ ಹಿತಕ್ಕಾಗಿ. ಸಂನ್ಯಾಸಿಯ ತ್ಯಾಗ ಜೀವನ ಕೋಟಿಗಟ್ಟಲೆ ನಿರ್ಭಾಗ್ಯರ ದುಃಖವನ್ನು ದೂರಗೊಳಿಸುವುದಕ್ಕಾಗಿ. ಸಂನ್ಯಾಸಿಯ ತ್ಯಾಗಜೀವನವು ವಿಧವೆಯರ ಕಣ್ಣೀರನ್ನು ಒರಸುವುದಕ್ಕಾಗಿ. ಸಂನ್ಯಾಸಿಯ ಜೀವವು ಪುತ್ರಶೋಕ ದಿಂದ ಪ್ರಲಾಪಿಸುತ್ತಿರುವ ಮಾತೆಯರ ಹೃದಯಕ್ಕೆ ಸಾಂತ್ವನ ನೀಡುವುದಕ್ಕಾಗಿ. ಸಂನ್ಯಾಸಿಯ ಜೀವನವು ಅಜ್ಞಾನಿಗಳಿಗೆ-ದೀನದಲಿತ ಅಸಹಾಯಕರಿಗೆ ಜೀವನೋಪಾಯವನ್ನು ಕಲ್ಪಿಸಿ ಅವರನ್ನು ಸ್ವಾವಲಂಬಿಗಳನ್ನಾಗಿಸುವುದಕ್ಕಾಗಿ. ಸಂನ್ಯಾಸಿಯ ಜೀವನವು ಮೇಲು-ಕೀಳು, ಬಡವ-ಬಲ್ಲಿದ ಎಂಬ ತಾರತಮ್ಯವಿಲ್ಲದೆ ಶಾಸ್ತ್ರಗಳ ಬೋಧನೆಯನ್ನು ಪ್ರಸಾರ ಮಾಡುವುದಕ್ಕಾಗಿ. ಸಂನ್ಯಾಸಿಯ ಜೀವನವು ಸಕಲ ಜೀವರಲ್ಲಿ ಪರಬ್ರಹ್ಮವೆಂಬ ಸಿಂಹವನ್ನು ಜಾಗೃತಗೊಳಿಸುವುದಕ್ಕಾಗಿ! ಸಂನ್ಯಾಸಿಯು ಈ ಜಗತ್ತಿನಲ್ಲಿ ಜನ್ಮವೆತ್ತಿರುವುದು ಇದಕ್ಕಾಗಿಯೇ!” ಅನಂತರ ಸ್ವಾಮೀಜಿ ತಮ್ಮ ಸೋದರ ಸಂನ್ಯಾಸಿಗಳತ್ತ ತಿರುಗಿ ಉದ್ಗರಿಸಿದರು: “ನೆನಪಿಡಿ, ಈ ಎಲ್ಲ ಧ್ಯೇಯೋದ್ದೇಶ ಗಳನ್ನು ಪರಿಪೂರ್ಣಗೊಳಿಸುವುದಕ್ಕಾಗಿಯೇ ನಾವು ಹುಟ್ಟಿರುವುದು. ಇದಕ್ಕಾಗಿ ನಾವು ನಮ್ಮ ಪ್ರಾಣ ಗಳನ್ನೇ ಬಲಿಯಾಗಿಸೋಣ. ಏಳಿ, ಎದ್ದೇಳಿ; ಮತ್ತು ಇತರರನ್ನೂ ಎಬ್ಬಿಸಿ ಎಚ್ಚರಗೊಳಿಸಿ ಸಂನ್ಯಾಸ ಜೀವನದ ಮಹಾದರ್ಶವನ್ನು ಈಡೇರಿಸಿ; ನೀವು ಅತ್ಯುನ್ನತ ಗುರಿಯನ್ನು ಮುಟ್ಟು ವುದು ಖಂಡಿತ.”

ಈಗ ಸ್ವಾಮೀಜಿ ಸಂನ್ಯಾಸ ಸ್ವೀಕರಿಸುತ್ತಿದ್ದ ಶಿಷ್ಯರತ್ತ ತಿರುಗಿ ಮತ್ತೆ ಹೀಗೆಂದರು: “ನೀವು ನಿಮ್ಮ ಸರ್ವಸ್ವವನ್ನೂ ತ್ಯಾಗ ಮಾಡಬೇಕು. ನಿಮ್ಮ ಸ್ವಂತ ಸುಖ ಸೌಲಭ್ಯಗಳ ಕಡೆಗೆ ಗಮನ ಕೊಡಲೇಬಾರದು. ಕಾಮಿನೀ-ಕಾಂಚನವನ್ನು ವಿಷದಂತೆ ನೋಡಬೇಕು. ಹೆಸರು ಕೀರ್ತಿಗಳನ್ನು ಕೊಳಕು ಅಮೇಧ್ಯದಂತೆ ಭಾವಿಸಬೇಕು. ಸಮ್ಮಾನ ವೈಭವಗಳನ್ನು ರೌರವ ನರಕವೆಂದು ಭಾವಿಸ ಬೇಕು. ‘ನಾವು ಉಚ್ಚ ಕುಲದವರು, ಉನ್ನತ ಸ್ಥಾನದಲ್ಲಿರುವವರು’ ಎಂಬ ಭಾವನೆಯನ್ನು ಮದ್ಯಪಾನಕ್ಕೆ ಸಮನಾದ ಪಾಪದಂತೆ ಭಾವಿಸಬೇಕು. ಇನ್ನು ಮೇಲೆ ನೀವು ಲೋಕಗುರುಗಳಾದ್ದ ರಿಂದ ಮತ್ತು ಆತ್ಮಜ್ಞಾನಕ್ಕಾಗಿ ನಿಮ್ಮ ಜೀವನವನ್ನು ಮುಡಿಪಾಗಿಟ್ಟವರಾದ್ದರಿಂದ ನೀವು ನಿಮ್ಮ ಮುಕ್ತಿಗಾಗಿ ಮತ್ತು ಜಗತ್ತಿನ ಹಿತಕ್ಕಾಗಿ ಮಾತ್ರವೇ ಜೀವಿಸಬೇಕು. ಇವುಗಳನ್ನೆಲ್ಲ ಮಾಡಲು ನೀವು ನಿಮ್ಮ ಸಮಸ್ತ ಶಕ್ತಿಯನ್ನು ಉಪಯೋಗಿಸಿ ಪ್ರಯತ್ನಿಸಬಲ್ಲಿರೇನು? ಚೆನ್ನಾಗಿ ಪರ್ಯಾ ಲೋಚಿಸಿದ ಮೇಲೆ ಮಾತ್ರವೇ ಈ ಮಾರ್ಗಕ್ಕೆ ಬನ್ನಿ. ನಿಮ್ಮ ಹಿಂದಿನ ಪ್ರಾಪಂಚಿಕ ಜೀವನಕ್ಕೆ ಹಿಂದಿರುಗಲು ಈಗಲೂ ಸಮಯವಿದೆ. ನನ್ನ ಆಜ್ಞೆಗಳನ್ನು ಚಾಚೂ ತಪ್ಪದೆ ಪರಿಪಾಲಿಸಲು ಸಿದ್ಧರಾಗಿರುವಿರೇನು? ನಿಮ್ಮನ್ನೀಗ ಒಂದು ಭಯಂಕರ ವ್ಯಾಘ್ರದ ಮುಂದೆ ಅಥವಾ ಘಟ ಸರ್ಪದ ಮುಂದೆ ನಿಲ್ಲುವಂತೆ, ಇಲ್ಲವೆ ಗಂಗೆಗೆ ಧುಮುಕಿ ಒಂದು ಮೊಸಳೆಯನ್ನು ಹಿಡಿದು ತರುವಂತೆ ನಾನು ಹೇಳಿದರೆ, ಅದರಿಂದ ನಿಮಗೇ ಒಳಿತು ಎಂದು ತಿಳಿದು ತಕ್ಷಣ ವಿಧೇಯತೆ ಯಿಂದ ನಾನು ಹೇಳಿದಂತೆ ಮಾಡಬಲ್ಲಿರೇನು? ನಿಮ್ಮನ್ನೆಲ್ಲ ನಾನೀಗ ಅಸ್ಸಾಮಿನ ಟೀ ತೋಟದ ಮಾಲೀಕರಿಗೆ ಮಾರಿ ನೀವು ನಿಮ್ಮ ಜೀವಮಾನ ಪರ್ಯಂತ ಅಲ್ಲೇ ಕೂಲಿ ಮಾಡಿ ಸಾಯಿರಿ ಎಂದು ಹೇಳಿದರೆ, ಅಥವಾ ನೀವೆಲ್ಲ ಉಪವಾಸ ಮಾಡಿ ಮರಣವನ್ನಪ್ಪುವಂತೆ ಆಜ್ಞೆ ಮಾಡಿದರೆ, ಅಥವಾ ಸಣ್ಣ ಉರಿಯಲ್ಲಿ ಬಿದ್ದು ನಿಧಾನವಾಗಿ ಪ್ರಾಣತ್ಯಾಗ ಮಾಡುವಂತೆ ಹೇಳಿದರೆ, ಅದು ನಿಮ್ಮ ಒಳ್ಳೆಯದಕ್ಕೇ ಎಂದು ಭಾವಿಸಿ ತಕ್ಷಣ ಮರುಮಾತಿಲ್ಲದೆ ನಾನು ಹೇಳಿದಂತೆ ಮಾಡಬಲ್ಲಿರೇನು?”

ಆ ನಾಲ್ವರು ಬ್ರಹ್ಮಚಾರಿಗಳು ಶಿರಬಾಗಿ ಈ ಮಾತುಗಳಿಗೆ ಸಮ್ಮತಿ ಸೂಚಿಸಿದರು. ಆದರೆ ಅವರು, ಸ್ವಾಮೀಜಿ ಸುಮ್ಮನೆ ತಮ್ಮನ್ನು ಪರೀಕ್ಷಿಸುವುದಕ್ಕಾಗಿ ಹಾಗೆ ಹೇಳಿದರೇ ಹೊರತು ನಿಜವಾಗಿಯೂ ಗಂಗೆಗೆ ಹಾರುವಂತಾಗಲಿ, ಬೆಂಕಿಗೆ ಬೀಳುವಂತಾಗಲಿ ಹೇಳಲಾರರೆಂಬ ಭರವಸೆ ಯಿದ್ದುದರಿಂದ ಹಾಗೆ ತಲೆಯಾಡಿಸಿದ್ದಲ್ಲ. ಅವರು ಸ್ವಾಮೀಜಿಯವರ ಆಜ್ಞೆಯನ್ನು ನಿಶ್ಚಯ ವಾಗಿಯೂ ಪಾಲಿಸಬಲ್ಲ ಧೀರರೇ ಆಗಿದ್ದರು. ಅಲ್ಲದೆ ಸ್ವಾಮೀಜಿ ಕೂಡ ಸುಮ್ಮನೆ ಮಾತಿಗಾಗಿ ಆಡಿದ ಮಾತಲ್ಲ ಅದು. ಸಂದರ್ಭ ಬಂದರೆ ಹಾಗೆ ಮಾಡುವಂತೆ ಹೇಳಿಬಿಡುವವರೆ! ಮಹಾ ಮಡಿವಂತರಾದ, ಅದರಲ್ಲೂ ತಮ್ಮ ಸೋದರ ಸಂನ್ಯಾಸಿಗಳಾದ ಸ್ವಾಮಿ ರಾಮಕೃಷ್ಣಾನಂದರಿಗೇ ಅವರು ಮುಸಲ್ಮಾನರ ಅಂಗಡಿಯಿಂದ ಬ್ರೆಡ್ಡು ತರುವಂತೆ ಹೇಳಲಿಲ್ಲವೇ!

ಅಂದು ಸ್ವಾಮೀಜಿಯವರಿಂದ ಸಂನ್ಯಾಸ ದೀಕ್ಷೆಯನ್ನು ಪಡೆದುಕೊಂಡ ಆ ಅದೃಷ್ಟಶಾಲೀ ಯುವಕರೆಂದರೆ ಕಾಳೀಕೃಷ್ಣ ಬಸು, ಕನ್ಹಾಯ್ ಸೇನ್, ಸುಶೀಲ್ ಚಕ್ರವರ್ತಿ ಹಾಗೂ ಜೋಗೇಂದ್ರನಾಥ ಚಟರ್ಜಿ. ಇವರ ಸಂನ್ಯಾಸನಾಮಗಳು ಕ್ರಮವಾಗಿ ಸ್ವಾಮಿ ವಿರಜಾನಂದ, ಸ್ವಾಮಿ ನಿರ್ಭಯಾನಂದ, ಸ್ವಾಮಿ ಪ್ರಕಾಶಾನಂದ ಹಾಗೂ ಸ್ವಾಮಿ ನಿತ್ಯಾನಂದ. ಇವರ ಪೈಕಿ ಜೋಗೇಂದ್ರನಾಥ ಸ್ವಾಮೀಜಿಯವರಿಗಿಂತಲೂ ದೊಡ್ಡವನು. ಈತ ಬಾರಾನಾಗೋರ್ ಮಠದ ದಿನಗಳಿಂದಲೂ ಆ ಗುರುಭಾಯಿಗಳಿಗೆಲ್ಲ ಪರಿಚಿತನಾದವನು. ಆ ಸಂನ್ಯಾಸಿಗಳಿಗೆ ಆಹಾರಕ್ಕೆ ಏನೇನೂ ಇಲ್ಲದಿದ್ದ ಸಂದರ್ಭಗಳಲ್ಲೆಲ್ಲ ಈತ ಅವರ ನೆರವಿಗೆ ಬಂದು ಭಿಕ್ಷೆಯೊದಗಿಸುತ್ತಿದ್ದ. ಕಾಳೀಕೃಷ್ಣ ಕಲ್ಕತ್ತದ ಸಿಮ್ಲಾ ಪ್ರದೇಶದವನು. ಒಂದು ಸ್ವಾರಸ್ಯದ ಸಂಗತಿಯೇನೆಂದರೆ, ಬಾಲಕ ನರೇಂದ್ರ ಮರಕೋತಿಯಾಡುತ್ತಿದ್ದಾಗ ‘ಬ್ರಹ್ಮರಾಕ್ಷಸ’ನ ಹೆಸರು ಹೇಳಿ ಅವನನ್ನು ಹೆದರಿಸಲು ನೋಡಿದ ವೃದ್ಧರು ಕಾಳೀಕೃಷ್ಣನ ತಾತನೇ.

ಈ ಸಂನ್ಯಾಸದೀಕ್ಷೆ ಸಮಾರಂಭವು ಸ್ವಾಮೀಜಿಯವರ ಪಾಲಿಗೆ ತಮ್ಮ ಗೌರವಾರ್ಥವಾಗಿ ನಡೆದ ಎಲ್ಲ ಸಮಾರಂಭಗಳಿಗಿಂತಲೂ ಹೆಚ್ಚು ಸಂತೋಷಪ್ರದವಾಗಿತ್ತು, ಹೆಚ್ಚು ಹೃದಯಸ್ಪರ್ಶಿ ಯಾಗಿತ್ತು. ಏಕಿರಬಹುದು? ಇತರ ಸಮಾರಂಭಗಳಿಗೆ ಸಾವಿರಾರು ಜನ-ಲಕ್ಷಗಟ್ಟಲೆ ಜನ ಬಂದರು, ಗೌರವ ತೋರಿದರು, ಚಪ್ಪಾಳೆ ತಟ್ಟಿದರು, ಬಿನ್ನವತ್ತಳೆಯೋದಿದರು, ಹೊರಟು ಬಿಟ್ಟರು. ಆದರೆ ಯಾರು ತಂದೆತಾಯಿ, ಬಂಧು ಬಳಗ, ಇಷ್ಟಮಿತ್ರ, ಕಾಮ ಕಾಂಚನಗಳನ್ನು ತ್ಯಾಗ ಮಾಡಿ ಆತ್ಮಸಾಕ್ಷಾತ್ಕಾರಕ್ಕಾಗಿ, ಜಗತ್ತಿನ ಹಿತಕ್ಕಾಗಿ ಮುಂದೆ ಬಂದಿದ್ದಾರೋ ಅಂತಹ ಈ ನಾಲ್ಕು ಜನರೇ ಸರ್ವಶ್ರೇಷ್ಠರಲ್ಲವೆ! ಅಲ್ಲದೆ ಸ್ವಾಮೀಜಿಯವರ ಕೆಲಸವೇನಾದರೂ ಕೈಗೂಡ ಬೇಕಾದರೆ ತ್ಯಾಗಿಗಳಾದ ಜನರಿಂದ ಆಗಬೇಕಾಗಿದೆಯೇ ಹೊರತು ಹಾರ ಹಾಕಿ ಚಪ್ಪಾಳೆ ತಟ್ಟುವವರಿಂದಲ್ಲ.

ಆಲಂಬಜಾರ್ ಮಠದಲ್ಲಿ ಮೇ ತಿಂಗಳ ಮೊದಲ ವಾರದಲ್ಲಿ ಇನ್ನೊಂದು ದೀಕ್ಷೆ ನೀಡುವ ಕಾರ್ಯಕ್ರಮ ನಡೆಯಿತು. ಸ್ವಾಮೀಜಿಯವರು ಶರಚ್ಚಂದ್ರ ಚಕ್ರವರ್ತಿ ಹಾಗೂ ಸುಧೀರ್ ಚಕ್ರವರ್ತಿ ಎಂಬಿಬ್ಬರಿಗೆ ಮಂತ್ರದೀಕ್ಷೆ ನೀಡಿದರು. ಸುಧೀರ್ ಹಿಂದಿನ ತಿಂಗಳಷ್ಟೇ ಮಠಕ್ಕೆ ಸೇರಿದ್ದರು. ಇವರು ಸುಶೀಲ್ ಚಕ್ರವರ್ತಿಯ (ಪ್ರಕಾಶಾನಂದರ) ಅಣ್ಣ. ಮುಂದೆ ಇವರು ಸ್ವಾಮಿ ಶುದ್ಧಾನಂದರೆಂಬ ಹೆಸರನ್ನು ಪಡೆದರು. ಸ್ವಾಮೀಜಿಯವರ ಭಾಷಣಗಳನ್ನು ಕೇಳಿ ಪ್ರಭಾವಿತ ರಾದ ಇವರು ಇನ್ನು ಲೌಕಿಕದಲ್ಲಿ ನಿಲ್ಲಲು ಸಾಧ್ಯವಾಗದೆ ಕಾಲೇಜು ವಿದ್ಯಾಭ್ಯಾಸವನ್ನು ಬದಿಗೊತ್ತಿ ಮಠಕ್ಕೆ ಧಾವಿಸಿ ಬಂದು ಸ್ವಾಮೀಜಿಯವರ ಪದತಲದಲ್ಲಿ ಶರಣಾದವರು. ಅವರ ಶ್ರದ್ಧೆಯನ್ನು ಕಂಡು ಸ್ವಾಮಿಜಿಯವರಿಗೆ ಬಲು ಹಿಗ್ಗು. ಮಂತ್ರೋಪದೇಶ ಪಡೆದ ಶರಚ್ಚಂದ್ರನಿಗೆ ಸ್ವಾಮೀಜಿ ಹೇಳುತ್ತಾರೆ: “ನಿನ್ನೊಳಗೆ ಶ್ರದ್ಧೆಯನ್ನು ಜಾಗೃತಗೊಳಿಸಿಕೊ. ಮತ್ತು ನಿನ್ನ ದೇಶ ಬಾಂಧವರಲ್ಲೂ ಶ್ರದ್ಧೆಯನ್ನು ಜಾಗೃತಗೊಳಿಸು. ಸತ್ಯವನ್ನರಿಯಲು, ಎಲ್ಲ ಬಂಧನಗಳಿಂದ ಪಾರಾಗಿ ಮುಕ್ತಿಯನ್ನು ಹೊಂದಲು ಯತ್ನಿಸು; ಆವಶ್ಯಕತೆ ಬಿದ್ದರೆ ನಚಿಕೇತನಂತೆ ಯಮನ ಬಳಿಗೂ ಹೋಗು... ಎಲ್ಲ ಬಗೆಯ ಭಯವೇ ಮೃತ್ಯು. ನಿನ್ನ ಮುಕ್ತಿಗಾಗಿ ಹಾಗೂ ಜಗದ್ಧಿತಕ್ಕಾಗಿ ಸರ್ವಸಮರ್ಪಣ ಮಾಡಿಕೊಳ್ಳಲು ನೀನು ಇಂದಿನಿಂದಲೇ ಸಿದ್ಧನಾಗು. ಇಲ್ಲದಿದ್ದರೆ ಈ ಅಸ್ಥಿ-ಮಾಂಸಗಳ ಗೂಡನ್ನು ಸುಮ್ಮನೆ ಹೊತ್ತುಕೊಂಡು ತಿರುಗಾಡುವುದರಿಂದೇನು ಪ್ರಯೋಜನ? ವೃತ್ರಾಸುರ ನನ್ನು ಸಂಹರಿಸಲು ದಧೀಚಿಯ ಬೆನ್ನೆಲುಬಿನಿಂದ ತಯಾರಿಸಲಾದ ವಜ್ರಾಯುಧದಿಂದ ಮಾತ್ರವೇ ಸಾಧ್ಯವೆಂದು ತಿಳಿದ ದೇವತೆಗಳು ಅವನನ್ನು ಅದಕ್ಕಾಗಿ ಪ್ರಾರ್ಥಿಸಿಕೊಂಡಾಗ ಅವನು ತನ್ನ ಬೆನ್ನೆಲುಬನ್ನೇ ತೆಗೆದುಕೊಟ್ಟಂತೆ, ಅಗ್ನಿಮಂತ್ರೋಪದೇಶವನ್ನು ಪಡೆದ ನೀವೂ ಕೂಡ ಭಗವಂತ ನಿಗಾಗಿ ಮಹಾತ್ಯಾಗಕ್ಕೆ ಸಿದ್ಧರಾಗಿರಬೇಕು, ನಿಮ್ಮ ಶರೀರವನ್ನು ಜಗತ್ತಿನ ಹಿತಕ್ಕಾಗಿ ಮೀಸಲಾ ಗಿಡಲು ಸಿದ್ಧರಾಗಿರಬೇಕು.”

ಸ್ವಾಮೀಜಿಯವರಿಗೆ ನಿಃಸ್ವಾರ್ಥ ಭಾವನೆಯಿಂದ ಇತರರ ಒಳಿತಿಗಾಗಿ ಶ್ರಮಿಸುವವರ ಮೇಲೆ ವಿಶೇಷ ಆದರ. ಮೇ ಮೊದಲ ವಾರದಲ್ಲಿ ಒಂದು ದಿನ ಅವರು ಗೌರೀ ಮಾ ಸ್ಥಾಪಿಸಿ ದಂತಹ‘ಮಹಾಕಾಳೀ ಪಾಠಶಾಲೆಗೆ’ಗೆ ಭೇಟಿ ನೀಡಿದರು. ಶಾಲೆಯನ್ನು ಸಂಪೂರ್ಣವಾಗಿ ನೋಡಿ ಅವರು ವೀಕ್ಷಕರ ಪುಸ್ತಕದಲ್ಲಿ ಬರೆದರು, “ಮಾತಾಜಿಯವರು ನಮ್ಮ ನಗರದಲ್ಲಿ ಪ್ರಾರಂಭಿಸಿದ ಈ ಘನವಾದ ಕಾರ್ಯವನ್ನು ನೋಡಿ ನನಗೆ ಬಹಳ ಆನಂದವಾಯಿತು. ನಿಜಕ್ಕೂ ಈ ಕಾರ್ಯ ಸರಿಯಾದ ಮಾರ್ಗದಲ್ಲಿ ಮುನ್ನಡೆಯುತ್ತಿದೆ ಎಂದು ನನಗನಿಸುತ್ತದೆ. ಯಾವ ತಂದೆತಾಯಂ ದಿರು ತಮ್ಮ ಹೆಣ್ಣು ಮಕ್ಕಳಿಗೆ ಭಾರತೀಯ ಸಂಸ್ಕೃತಿಗನುಗುಣವಾಗಿ ವಿದ್ಯಾಭ್ಯಾಸ ಕೊಡಿಸಲು ಇಚ್ಛಿಸುತ್ತಾರೋ ಅವರೆಲ್ಲರ ಸಂಪೂರ್ಣ ಬೆಂಬಲವು ಈ ಸಂಸ್ಥೆಗೆ ಸಲ್ಲಬೇಕು.”

ಅಲ್ಲಿಂದ ವಾಪಸು ಬರುವಾಗ ಸ್ವಾಮೀಜಿ ಶರಚ್ಚಂದ್ರನ ಬಳಿ ಹೇಳಿದರು, “ನಮ್ಮ ಸ್ತ್ರೀಯರಿಗೆ ಧರ್ಮ, ಕಲೆ, ವಿಜ್ಞಾನ, ಗೃಹಕೃತ್ಯ, ಅಡಿಗೆ, ಹೊಲಿಗೆ, ಕಸೂತಿ, ಆರೋಗ್ಯಶಾಸ್ತ್ರ–ಇವುಗಳ ಮುಖ್ಯಾಂಶಗಳನ್ನು ಕಲಿಸಬೇಕು. ಇವರೆಲ್ಲ ಕಾದಂಬರಿ, ಪತ್ತೇದಾರಿಗಳನ್ನು ಮುಟ್ಟಲು ಬಿಡ ಬಾರದು. ಈ ದಿಸೆಯಲ್ಲಿ ಮಹಾಕಾಳೀ ಪಾಠಶಾಲೆ ಒಳ್ಳೆಯ ಕೆಲಸ ಮಾಡುತ್ತಿದೆ. ಆದರೆ ಕೇವಲ ಪೂಜಾ ವಿಧಾನವನ್ನು ಕಲಿಸಿದರೆ ಸಾಲದು. ಅವರಿಗೆ ಸಿಗುವ ವಿದ್ಯಾಭ್ಯಾಸವು ಎಲ್ಲ ವಿಷಯ ಗಳಲ್ಲೂ ಅವರಿಗೆ ಬೆಳಕು ನೀಡುವಂತಿರಬೇಕು. ತರುಣಿಯರ ಮುಂದೆ ಯಾವಾಗಲೂ ಆದರ್ಶ ಸ್ತ್ರೀಯರ ವ್ಯಕ್ತಿತ್ವವನ್ನಿರಿಸಬೇಕು. ಈ ಆದರ್ಶ ವ್ಯಕ್ತಿತ್ವಗಳು ಸ್ತ್ರೀಯರಲ್ಲಿ ತೀವ್ರ ನಿಃಸ್ವಾರ್ಥತೆಯ ಭಾವವನ್ನು ಪ್ರಚೋದಿಸುವಂತಿರಬೇಕು. ಭಾರತೀಯ ಮಹಿಳೆಯರು ಸೀತಾ, ಸಾವಿತ್ರಿ, ದಮ ಯಂತಿ, ಲೀಲಾವತಿ, ಮೀರಾ–ಇವರೇ ಮೊದಲಾದ ಆದರ್ಶ ಸ್ತ್ರೀಯರ ಜೀವನದ ಬೆಳಕಿನಲ್ಲಿ ತಮ್ಮ ಜೀವನವನ್ನು ರೂಪಿಸಿಕೊಳ್ಳುವಂತಿರಬೇಕು.”

ಸ್ವಾಮೀಜಿ ಕಲ್ಕತ್ತಕ್ಕೆ ಬಂದಾಗಲೆಲ್ಲ ಬಲರಾಮ ಬಾಬುವಿನ ಮನೆಯಲ್ಲಿ ಕೆಲಕಾಲ ನಿಲ್ಲು ತ್ತಿದ್ದರು. ಅಲ್ಲಿ ಶ್ರೀರಾಮಕೃಷ್ಣರ ಸಂನ್ಯಾಸೀ ಶಿಷ್ಯರು ಹಾಗೂ ಗೃಹಸ್ಥ ಶಿಷ್ಯರು ಒಟ್ಟಾಗಿ ಕಲೆತು ವಿಶ್ವಾಸದಿಂದ ಭಾವವಿನಿಮಯ ಮಾಡಿಕೊಳ್ಳುತ್ತಿದ್ದರು. ಈಗ ಡಾರ್ಜಿಲಿಂಗಿನಿಂದ ಹಿಂದಿರುಗಿದ ಸ್ವಾಮೀಜಿ, ಎಲ್ಲ ಸಂನ್ಯಾಸೀ ಶಿಷ್ಯರ ಹಾಗೂ ಗೃಹೀಭಕ್ತರ ಸಭೆಯೊಂದನ್ನು ಕರೆದರು. ಹಿಂದೆಯೇ ನೋಡಿದಂತೆ, ಆಧ್ಯಾತ್ಮಿಕ ಹಾಗೂ ಲೌಕಿಕ ಕ್ಷೇತ್ರಗಳಲ್ಲಿ ಬೇರೆ ಬೇರೆ ಸ್ತರಗಳಲ್ಲಿ ಜನಸೇವೆ ಸಲ್ಲಿಸುವುದಕ್ಕಾಗಿ ಒಂದು ಸಂಘವನ್ನು ಸ್ಥಾಪಿಸಬೇಕು ಮತ್ತು ಆ ಸಂಘದ ಸೇವಾಕಾರ್ಯದಲ್ಲಿ ಸಂನ್ಯಾಸಿಗಳು, ಭಕ್ತರು ಹಾಗೂ ಸಾರ್ವಜನಿಕರು ಭಾಗವಹಿಸುವಂತಿರಬೇಕು ಎಂಬುದು ಸ್ವಾಮೀಜಿಯವರ ಮಹದಾಸೆಯಾಗಿತ್ತು. ಡಾರ್ಜಿಲಿಂಗಿನಲ್ಲಿ ನಡೆದ ಸಭೆಯಲ್ಲಿ ಈ ಅಭಿಪ್ರಾಯ ವನ್ನು ಎಲ್ಲರೂ ಅನುಮೋದಿಸಿದ್ದರು. ಈಗ ಅದನ್ನು ಕಲ್ಕತ್ತದ ಸಮಸ್ತ ಭಕ್ತವೃಂದದ ಮುಂದಿಟ್ಟು ಅವರ ಬೆಂಬಲವನ್ನು ಪಡೆದುಕೊಳ್ಳುವ ಉದ್ದೇಶದಿಂದ ಈ ಸಭೆಯನ್ನು ಕರೆಯಲಾಗಿತ್ತು.

ಮೇ ೧ನೇ ತಾರೀಕು ಶನಿವಾರ ಬಲರಾಮ ಬಾಬುವಿನ ಮನೆಯಲ್ಲಿ ಸಭೆ ಸೇರಿತು. ಅಲ್ಲಿ ನೆರೆ ದಿದ್ದವರನ್ನುದ್ದೇಶಿಸಿ ಸ್ವಾಮೀಜಿ ಮಾತನಾಡಲಾರಂಭಿಸಿದರು:

“ದೇಶದೇಶಾಂತರಗಳನ್ನು ಸಂಚರಿಸಿ ಬಂದ ನಾನು ಈಗ ಒಂದು ತೀರ್ಮಾನಕ್ಕೆ ಬಂದಿದ್ದೇನೆ. ಏನೆಂದರೆ, ಸರಿಯಾದ ಸಂಘಟನೆಯಿಲ್ಲದೆ ನಾವು ಯಾವುದೇ ಘನವಾದ ಹಾಗೂ ಶಾಶ್ವತವಾದ ಕಾರ್ಯವನ್ನು ಸಾಧಿಸಲಾರೆವು. ಆದರೆ ಭಾರತದಂತಹ ದೇಶದಲ್ಲಿ, ಅದರಲ್ಲೂ ನಾವು ಈಗಿರುವ ಪರಿಸ್ಥಿತಿಯಲ್ಲಿ, ಎಲ್ಲರಿಗೂ ಸಮಾನವಾದ ಹಕ್ಕುಬಾಧ್ಯತೆಗಳಿರುವಂತಹ, ಬಹುಮತದಿಂದ ನಿರ್ಧಾರಗಳನ್ನು ಕೈಗೊಳ್ಳುವಂತಹ ಪ್ರಜಾಪ್ರಭುತ್ವದ ರೀತಿಯ ಸಂಘವನ್ನು ಸ್ಥಾಪಿಸುವುದು ತರವಲ್ಲವೆಂದು ನನಗನ್ನಿಸುತ್ತದೆ. ಪಾಶ್ಚಾತ್ಯ ರಾಷ್ಟ್ರಗಳ ವಿಷಯವೇ ಬೇರೆ... ನಮ್ಮ ರಾಷ್ಟ್ರ ದಲ್ಲೂ ಕೂಡ ವಿದ್ಯಾಭ್ಯಾಸವು ಸಾಕಷ್ಟು ಹರಡಿ, ನಾವು ನಮ್ಮ ಸ್ವಾರ್ಥವನ್ನು ತ್ಯಾಗ ಮಾಡಲು ಕಲಿತಾಗ ಮತ್ತು ನಾವು ನಮ್ಮ ವೈಯಕ್ತಿಕ ಆಶೋತ್ತರಗಳನ್ನು ಮೀರಿ ವಿಶಾಲ ಭಾರತದ ಹಿತದ ಕಡೆಗೆ ಗಮನ ಹರಿಸಬಲ್ಲವರಾದಾಗ ನಾವೂ ಕೂಡ ಪ್ರಜಾಪ್ರಭುತ್ವ ತತ್ತ್ವದ ಮೇಲೆ ಕಾರ್ಯ ನಿರ್ವಹಿಸಲು ಸಾಧ್ಯವಾಗಬಹುದು. ಆದ್ದರಿಂದ ಈಗ ನಮ್ಮ ಸಂಘಕ್ಕೆ ಒಬ್ಬರು ನಿರ್ದೇಶಕರಿರ ಬೇಕು ಮತ್ತು ಆ ನಿರ್ದೇಶಕರ ಆಜ್ಞೆಗೆ ಸರ್ವರೂ ವಿಧೇಯರಾಗಿರಬೇಕು. ಬಳಿಕ ಸಕಾಲದಲ್ಲಿ ಈ ಸಂಘವು ಅದರ ಸಮಸ್ತ ಸದಸ್ಯರ ಅಭಿಪ್ರಾಯ ಹಾಗೂ ಸಮ್ಮತಿಯ ಮೇರೆಗೆ ನಡೆದುಕೊಂಡು ಬರುವಂತಾಗುತ್ತದೆ.

“ಈ ಸಂಘವು, ನಾವು ಯಾರ ಹೆಸರಿನಲ್ಲಿ ಸರ್ವಸಂಗ ಪರಿತ್ಯಾಗ ಮಾಡಿ ಸಂನ್ಯಾಸಿಗಳಾ ಗಿರುವೆವೋ, ಯಾರನ್ನು ನೀವು (ಗೃಹಸ್ಥರು) ನಿಮ್ಮ ಜೀವನಾದರ್ಶವನ್ನಾಗಿ ಮಾಡಿಕೊಂಡು ಈ ಜಗತ್ತಿನಲ್ಲಿ ಗೃಹಸ್ಥಧರ್ಮವನ್ನು ನಿರ್ವಹಿಸುತ್ತಿರುವಿರೋ, ಯಾರ ಪವಿತ್ರ ನಾಮವು ಮತ್ತು ಯಾರ ಅಸದೃಶ-ಅದ್ಭುತ ಜೀವನದ ಪ್ರಭಾವವು ಅವರು ಪರಿನಿರ್ಯಾಣ ಹೊಂದಿದ ಕೇವಲ ಹನ್ನೆರಡೇ ವರ್ಷಗಳಲ್ಲಿ ಪ್ರಾಚ್ಯ ಪಾಶ್ಚಾತ್ಯ ರಾಷ್ಟ್ರಗಳಲ್ಲಿ ಯಾರೂ ಊಹಿಸಿರದಿದ್ದ ರೀತಿಯಲ್ಲಿ ಹರಡಿತೋ ಅಂಥವರ ಹೆಸರನ್ನು ಧರಿಸುತ್ತದೆ. ಆದ್ದರಿಂದ ಈ ಸಂಘಕ್ಕೆ ‘ರಾಮಕೃಷ್ಣ ಮಿಷನ್​’ ಎಂಬ ಹೆಸರಿರಲಿ. ನಾವು ಆ ಪ್ರಭುವಿನ ಕೇವಲ ಸೇವಕರು ಮಾತ್ರ. ಈ ಉದ್ದೇಶದತ್ತ ನಿಮ್ಮೆಲ್ಲರ ನಲ್ಮೆಯ ಸಹಕಾರ ಹರಿದುಬರಲಿ!”

ಸ್ವಾಮೀಜಿಯವರ ಈ ಸಲಹೆಯನ್ನು ಸರ್ವರೂ ಅತ್ಯುತ್ಸಾಹದಿಂದ ಸ್ವಾಗತಿಸಿದರು. ಈ ನಿರ್ಣಯ ಸರ್ವಾನುಮತದಿಂದ ಅಂಗೀಕೃತವಾಯಿತು. ಬಳಿಕ ಮುಂದಿನ ಕಾರ್ಯವಿಧಾನದ ಬಗ್ಗೆ ಸಾಕಷ್ಟು ವಿಚಾರ ವಿನಿಮಯ ನಡೆಯಿತು. ಅಂತೂ ಹೀಗೆ ೧೮೯೭ರ ಮೇ ೧ರಂದು ರಾಮಕೃಷ್ಣ ಮಿಷನ್ ಸಂಸ್ಥೆ ಜನ್ಮತಾಳಿತೆಂದು ಹೇಳಬಹುದು. ಮೇ ೫ರಂದು ಇನ್ನೊಂದು ಸಭೆ ಸೇರಿತು. ಅದರಲ್ಲಿ ಈ ರಾಮಕೃಷ್ಣ ಮಹಾಸಂಚಲನವನ್ನು ಹೇಗೆ ನಡೆಸಬೇಕು, ಅದರ ಉದ್ದೇಶಗಳು- ಆದರ್ಶಗಳು ಏನೇನು ಎಂಬ ವಿಷಯಗಳನ್ನೆಲ್ಲ ನಿರ್ಧರಿಸಲಾಯಿತು. ಸ್ವಾಮೀಜಿಯವರ ಆದೇಶ ದಂತೆ ರೂಪುಗೊಂಡ ಆ ಗೊತ್ತುವಳಿ ಹೀಗಿದೆ:

ಈ ಸಂಘವು ‘ರಾಮಕೃಷ್ಣ ಮಿಷನ್​’ ಎಂದು ಕರೆಯಲ್ಪಡುವುದು. ಈ ಸಂಘದ ಉದ್ದೇಶ ವೇನೆಂದರೆ, ಮಾನವತೆಯ ಒಳಿತಿಗಾಗಿ ಶ್ರೀರಾಮಕೃಷ್ಣರು ಬೋಧಿಸಿದ ಮತ್ತು ಸ್ವಯಂ ತಮ್ಮ ಜೀವನದಲ್ಲಿ ಸಿದ್ದಮಾಡಿ ತೋರಿಸಿದ ಸತ್ಯಗಳನ್ನು ಪ್ರಸಾರ ಮಾಡುವುದು ಮತ್ತು ಈ ಸತ್ಯಗಳನ್ನು ಜನರು ತಮ್ಮ ಲೌಕಿಕ, ಬೌದ್ಧಿಕ ಹಾಗೂ ಆಧ್ಯಾತ್ಮಿಕ ಪ್ರಗತಿಗಾಗಿ ತಮ್ಮ ಜೀವನದಲ್ಲಿ ಅಳವಡಿಸಿಕೊಳ್ಳಲು ಸಹಾಯ ಮಾಡುವುದು.

ಈ ಮಿಷನ್ನಿನ ಕರ್ತವ್ಯವೇನೆಂದರೆ ವಿಭಿನ್ನ ಮತಧರ್ಮಗಳ ನಡುವೆ ಭ್ರಾತೃತ್ವವನ್ನು ನೆಲೆ ಗೊಳಿಸಲು ಶ್ರೀರಾಮಕೃಷ್ಣರು ಅಂಕುರಾರ್ಪಣ ಮಾಡಿದ ಸಂಚಲನದ ಕಾರ್ಯ ಚಟುವಟಿಕೆ ಗಳನ್ನು, ಆ ಎಲ್ಲ ಮತ ಧರ್ಮಗಳೂ ಅವಿನಾಶಿಯಾದ ಏಕೈಕ ಸನಾತನ ಧರ್ಮದ ಬೇರೆಬೇರೆ ರೂಪಗಳೆಂದು ತಿಳಿದು, ಯಥಾರ್ಥವಾಗಿ ನಡೆಸಿಕೊಂಡು ಹೋಗುವುದು.

\section{ಈ ಮಿಷನ್ನಿನ ಕಾರ್ಯವಿಧಾನಗಳು:}

(ಅ) ಜನಸಮುದಾಯದ ಐಹಿಕ ಹಾಗೂ ಆಧ್ಯಾತ್ಮಿಕ ಅಭ್ಯುದಯಕ್ಕೆ ಸಹಾಯಕವಾಗಬಲ್ಲ ಜ್ಞಾನವನ್ನು ಅಥವಾ ವಿಜ್ಞಾನವನ್ನು ಬೋಧಿಸಲು ಸಮರ್ಥರಾಗುವಂತೆ ವ್ಯಕ್ತಿಗಳಿಗೆ ತರಬೇತಿ ಕೊಡುವುದು; (ಆ) ಕಲೆ ಮತ್ತು ಕೈಗಾರಿಕೆಗಳಿಗೆ ಉತ್ತೇಜನ ನೀಡುವುದು; (ಇ) ವೇದಾಂತ ಹಾಗೂ ಇನ್ನಿತರ ಧಾರ್ಮಿಕ ಭಾವನೆಗಳನ್ನು—ಅವು ಶ್ರೀರಾಮಕೃಷ್ಣರ ಜೀವನದಲ್ಲಿ ಪ್ರಕಟ ಗೊಂಡಿರುವ ರೀತಿಯಲ್ಲೇ—ಜನಸಾಮಾನ್ಯರ ನಡುವೆ ಪ್ರಸಾರ ಮಾಡುವುದು.


\section{ಭಾರತದಲ್ಲಿನ ಕಾರ್ಯಗಳ ವಿಭಾಗ:}

ಭಾರತದ ಬೇರೆ ಬೇರೆ ಭಾಗಗಳಲ್ಲಿ ಮಠ-ಆಶ್ರಮಗಳನ್ನು ಸ್ಥಾಪಿಸುವ ದಿಸೆಯಲ್ಲಿ ರಾಮಕೃಷ್ಣ ಮಿಷನ್ ಕಾರ್ಯಪ್ರವೃತ್ತವಾಗಬೇಕು. ಈ ಮಠಗಳು ಮತ್ತು ಆಶ್ರಮಗಳು ಸಂನ್ಯಾಸಿಗಳನ್ನು, ಹಾಗೂ ಇತರರಿಗೆ ಶಿಕ್ಷಣ ನೀಡಲು ತಮ್ಮ ಜೀವನವನ್ನು ಮುಡಿಪಾಗಿಡಲು ಸಿದ್ಧರಿರುವಂತಹ ಗೃಹಸ್ಥರನ್ನು, ತರಬೇತಿಗೊಳಿಸಬೇಕು. ಈ ಸಂನ್ಯಾಸಿಗಳು ಮತ್ತು ಗೃಹಸ್ಥರು ಒಂದು ಪ್ರಾಂತ ದಿಂದ ಮತ್ತೊಂದು ಪ್ರಾಂತಕ್ಕೆ ಹೋಗುತ್ತ ಜನರಿಗೆ ಶಿಕ್ಷಣ ನೀಡಲು ಬೇಕಾದ ನೆರವನ್ನು ಪಡೆದುಕೊಳ್ಳುವುದೂ ರಾಮಕೃಷ್ಣ ಮಿಷನ್ನಿನ ಕರ್ತವ್ಯವಾಗಿರಬೇಕು.


\section{ವಿದೇಶಗಳಲ್ಲಿನ ಕಾರ್ಯಗಳ ವಿಭಾಗ:}

ಈ ವಿಭಾಗದ ಕಾರ್ಯವೇನೆಂದರೆ ಭಾರತದ ಹಾಗೂ ವಿದೇಶಗಳ ನಡುವೆ ಪರಸ್ಪರ ಹತ್ತಿರದ ಸಂಬಂಧವುಂಟಾಗುವ ಉದ್ದೇಶದಿಂದ ಮತ್ತು ಈ ಜನಗಳ ನಡುವೆ ಹೆಚ್ಚಿನ ಭಾವಗ್ರಾಹಕತೆ ಬೆಳೆಯುವ ಉದ್ದೇಶದಿಂದ, ವೇದಾಂತ ಪ್ರಸಾರ ಮಾಡುವುದಕ್ಕಾಗಿ ವಿದೇಶಗಳಲ್ಲಿ ಕೇಂದ್ರಗಳನ್ನು ಸ್ಥಾಪಿಸಲು ಸಂಘದ ತರಬೇತುಗೊಂಡ ಸದಸ್ಯರನ್ನು ಆ ದೇಶಗಳಿಗೆ ಕಳಿಸುವುದು.

ಈ ಮಿಷನ್ನಿನ ಉದ್ದೇಶ ಹಾಗೂ ಆದರ್ಶಗಳು ಸಂಪೂರ್ಣವಾಗಿ ಆಧ್ಯಾತ್ಮಿಕವೂ ಮಾನವತೆಯ ಒಳಿತಿಗೆ ಸಂಬಂಧಿಸಿದಂಥವೂ ಆಗಿದ್ದು, ರಾಜಕಾರಣದೊಂದಿಗೆ ಅದು ಯಾವ ಸಂಬಂಧವನ್ನೂ ಹೊಂದಿರುವುದಿಲ್ಲ.

ಶ್ರೀರಾಮಕೃಷ್ಣರ ಕಾರ್ಯೋದ್ದೇಶಗಳಲ್ಲಿ ನಂಬಿಕೆಯಿರುವ, ಅಥವಾ ಈ ಮೇಲೆ ಹೇಳಿರುವ ಸಂಘದ ಉದ್ದೇಶಗಳಲ್ಲಿ ಸಹಾನುಭೂತಿಯನ್ನು ಹೊಂದಿರುವ ಇಲ್ಲವೆ ಸಹಕರಿಸಲು ಇಚ್ಛಿಸುವ ಯಾರೇ ಆಗಲಿ ಇದರ ಸದಸ್ಯತ್ವಕ್ಕೆ ಅರ್ಹರಾಗಿರುತ್ತಾರೆ.

ಹೀಗೆ ರಾಮಕೃಷ್ಣ ಮಹಾಸಂಘದ ನಿರ್ಣಯಗಳು ಸಿದ್ಧವಾದುವು ಮತ್ತು ಅಂಗೀಕೃತ ವಾದುವು. ಈಗ ಸಂಘದ ಅಧಿಕಾರಿವರ್ಗದವರ ನೇಮಕವಾಗಬೇಕಾಗಿದೆ. ಸ್ವಾಮಿ ವಿವೇಕಾ ನಂದರೇ ಸಂಘದ ಮಹಾಧ್ಯಕ್ಷರಾಗಿ ನೇಮಕಗೊಂಡರು. ಸ್ವಾಮಿ ಬ್ರಹ್ಮಾನಂದರು ಕಲ್ಕತ್ತ ಕೇಂದ್ರದ ಅಧ್ಯಕ್ಷರಾದರು, ಸ್ವಾಮಿ ಯೋಗಾನಂದರು ಉಪಾಧ್ಯಕ್ಷರಾದರು. ಶ್ರೀರಾಮಕೃಷ್ಣರ ಗೃಹೀ ಶಿಷ್ಯರೂ ಸಂಘದ ಹಿತೈಷಿಗಳೂ ಆದ ನರೇಂದ್ರನಾಥ ಮಿತ್ರರನ್ನು ಕಾರ್ಯದರ್ಶಿಗಳನ್ನಾಗಿ ನೇಮಿಸಲಾಯಿತು. ಡಾ ॥ ಶಶಿಭೂಷಣ ಘೋಷ್ ಹಾಗೂ ಶರಚ್ಚಂದ್ರ ಸರ್ಕಾರ್​–ಇವರು ಉಪಕಾರ್ಯದರ್ಶಿಗಳಾದರು. ಶರಚ್ಚಂದ್ರ ಚಕ್ರವರ್ತಿಯನ್ನು ಶಾಸ್ತ್ರಗಳ ಉಪನ್ಯಾಸಕರನ್ನಾಗಿ ನೇಮಿಸಲಾಯಿತು. ಸಂಘದ ಕಾರ್ಯಕಲಾಪಗಳ ಬಗ್ಗೆ ಈ ನಿರ್ಣಯಗಳನ್ನು ತೆಗೆದುಕೊಳ್ಳ ಲಾಯಿತು, ಏನೆಂದರೆ–ಪ್ರತಿ ಭಾನುವಾರ ಮಧ್ಯಾಹ್ನ ಬಲರಾಮ ಬಾಬುವಿನ ಮನೆಯಲ್ಲಿ ಎಲ್ಲರೂ ಸೇರಬೇಕು; ಅಲ್ಲಿ ಭಗವದ್ಗೀತೆ, ಉಪನಿಷತ್ತು ಅಥವಾ ವೇದಾಂತ ಶಾಸ್ತ್ರಗಳಿಂದ ಶ್ಲೋಕಗಳ ಪಠಣ ಮತ್ತು ಅವುಗಳ ಕುರಿತಾಗಿ ವ್ಯಾಖ್ಯಾನ ನಡೆಯಬೇಕು; ಉಪನ್ಯಾಸ-ಚರ್ಚೆಗಳು ನಡೆಯಬೇಕು.

ಹೀಗೆ ಏನೇನು ಮಾಡಬೇಕೆಂಬುದೆಲ್ಲ ಒಂದು ಸ್ಪಷ್ಟ ರೂಪಕ್ಕೆ ಬಂದಿತು. ಅಂದಿನ ಸಭೆ ಮುಕ್ತಾಯಗೊಂಡು ಸ್ವಲ್ಪ ಹೊತ್ತಾಗಿರಬಹುದು; ಆಗ ಸ್ವಾಮಿ ಯೋಗಾನಂದರೊಂದಿಗೆ ಮಾತ ನಾಡುತ್ತ ಸ್ವಾಮೀಜಿ ಹೇಳಿದರು, “ಅಂತೂ ನಾವು ಮಾಡಬೇಕಾದ ಕಾರ್ಯ ಈ ರೀತಿಯಲ್ಲಿ ಪ್ರಾರಂಭವಾಗಿದೆ. ಇನ್ನು ಇದು ಶ್ರೀರಾಮಕೃಷ್ಣರ ಇಚ್ಛೆಯಿಂದ ಹೇಗೆ ಮುಂದುವರಿಯುವುದೋ ನೋಡೋಣ.”

ಯೋಗಾನಂದರು: “ನೀನು ಇದನ್ನೆಲ್ಲ ನಿನ್ನ ಪಾಶ್ಚಾತ್ಯ ರೀತಿ ನೀತಿಗೆ ಅನುಸಾರವಾಗಿ ಮಾಡು ತ್ತಿದ್ದೀಯೆ. ಇದನ್ನೆಲ್ಲ ಹೀಗೆಯೇ ಮಾಡಬೇಕೆಂದು ಶ್ರೀರಾಮಕೃಷ್ಣರೇನಾದರೂ ಹೇಳಿ ಹೋಗಿ ದ್ದಾರೆಯೆ?”

ಸ್ವಾಮೀಜಿ: “ಈ ಕಾರ್ಯವಿಧಾನಗಳೆಲ್ಲ ಶ್ರೀರಾಮಕೃಷ್ಣರ ಅಭಿಪ್ರಾಯಗಳಿಗೆ ಅನುಸಾರ ವಾಗಿಲ್ಲ ಅಂತ ನಿನಗೆ ಹೇಗೆ ಗೊತ್ತು? ಅವರು ಅನಂತ ಭಾವಗಳ ಸಾಕಾರಮೂರ್ತಿ. ನೀನು ಅವರನ್ನು ನಿನ್ನ ಪರಿಮಿತಿಯೊಳಗೆ ಬಂಧಿಸಿಡುವ ಪ್ರಯತ್ನ ಮಾಡುತ್ತಿರುವಂತಿದೆ! ನಾನು ನಿನ್ನ ಆ ಎಲ್ಲ ಪರಿಮಿತಿಗಳನ್ನು ಹರಿದೊಗೆದು ಅವರ ದಿವ್ಯ ಭಾವನೆಗಳನ್ನು ಜಗತ್ತಿನಾದ್ಯಂತ ಪ್ರಸಾರ ಗೈಯುವವನಿದ್ದೇನೆ. ತಮ್ಮನ್ನು ಪ್ರತಿಷ್ಠಾಪಿಸಿ ಪೂಜೆ ಮಾಡಬೇಕು, ಅದು ಮಾಡಬೇಕು, ಇದು ಮಾಡಬೇಕು ಅಂತ ಅವರೇನೂ ನನಗೆ ಹೇಳಲಿಲ್ಲ. ಆಧ್ಯಾತ್ಮಿಕ ಸಾಧನೆ-ಧ್ಯಾನಾಭ್ಯಾಸಗಳನ್ನು ಹಾಗೂ ಧರ್ಮದ ಉನ್ನತ ಆದರ್ಶಗಳನ್ನು ಅವರು ನಮಗೆ ಹೇಗೆ ಬೋಧಿಸಿರುವರೋ ಅವನ್ನು ನಾವೂ ಸಾಕ್ಷಾತ್ಕರಿಸಿಕೊಳ್ಳಬೇಕು, ಇತರರಿಗೂ ಬೋಧಿಸಬೇಕು. ಭಗವಂತನನ್ನು ಸೇರಲು ಮತ ಗಳೂ ಅಸಂಖ್ಯಾತ, ಪಂಥಗಳೂ ಅಸಂಖ್ಯಾತ. ಈಗಾಗಲೇ ಸಾಕುಸಾಕೆನಿಸುವಷ್ಟು ಮತಗಳಿಂದ ತುಂಬಿಹೋಗಿರುವ ಈ ಜಗತ್ತಿನಲ್ಲಿ ಇನ್ನೊಂದು ಹೊಸ ಮತವನ್ನು ಸೃಷ್ಟಿಸುವುದಕ್ಕಾಗಿ ಬಂದ ವನಲ್ಲ ನಾನು. ಶ್ರೀರಾಮಕೃಷ್ಣರ ಅಡಿದಾವರೆಗಳಲ್ಲಿ ಆಶ್ರಯವನ್ನು ಪಡೆದ ನಾವೇ ನಿಜಕ್ಕೂ ಧನ್ಯರು. ಅವರು ನಮಗೆ ಕೃಪೆಮಾಡಿ ನೀಡಿದ ನೂತನ ಭಾವನೆಗಳನ್ನೂ ದಿವ್ಯ ಸಂದೇಶಗಳನ್ನೂ ಇಡೀ ಜಗತ್ತಿಗೆ ಕೊಡಬೇಕಾದದ್ದು ನಮ್ಮ ಧರ್ಮ.”

ಯೋಗಾನಂದರು ಒಂದು ಮಾತನ್ನೂ ಆಡದೆ ಮೌನವಾಗಿ ಆಲಿಸುತ್ತಿದ್ದರು. ಅದೇ ಭಾವದಲ್ಲಿ ಸ್ವಾಮೀಜಿ ತಮ್ಮ ಮಾತನ್ನು ಮುಂದುವರಿಸಿದರು:

“ಈ ನನ್ನ ಜೀವಿತಾವಧಿಯಲ್ಲಿ ನಾನು ಮತ್ತೆ ಮತ್ತೆ ಹಲವಾರು ಸಂದರ್ಭಗಳಲ್ಲಿ ಶ್ರೀರಾಮ ಕೃಷ್ಣರ ಕೃಪೆಯನ್ನು ಸ್ಪಷ್ಟವಾಗಿ ಕಂಡಿದ್ದೇನೆ. ಸ್ವತಃ ಅವರೇ ನನ್ನ ಬೆನ್ನ ಹಿಂದಿದ್ದು ನನ್ನ ಮೂಲಕ ಈ ರೀತಿಯಲ್ಲೆಲ್ಲ ತಮ್ಮ ಕೆಲಸವನ್ನು ಮಾಡಿಸುತ್ತಿದ್ದಾರೆ. ನಾನು ಬಳಲಿ ಬೆಂಡಾಗಿ ಮರದ ಬುಡದಲ್ಲಿ ಬಿದ್ದುಕೊಂಡಿದ್ದಾಗಲೇ ಆಗಲಿ, ಹಸಿವಿನಿಂದ ಕಂಗಾಲಾಗಿದ್ದಾಗಲೇ ಆಗಲಿ, ಇಲ್ಲವೆ ನನ್ನ ಕೌಪೀನಕ್ಕೂ ಒಂದು ಚೂರು ಬಟ್ಟೆಯಿಲ್ಲದಿದ್ದಾಗಲೇ ಆಗಲಿ, ಸಂಚಾರದ ಕಾಲದಲ್ಲಿ ಹಣವನ್ನು ಮುಟ್ಟಬಾರದೆಂಬ ವ್ರತವನ್ನು ಕೈಗೊಂಡಾಗಲೇ ಆಗಲಿ–ಯಾವಾಗಲೂ ಅವರ ಕೃಪೆ ಯಿಂದಲೇ ನನಗೆ ಎಲ್ಲ ರೀತಿಯ ಸಹಾಯವೂ ಒದಗಿ ಬಂದಿತು. ಶಿಕಾಗೋದ ಬೀದಿಗಳಲ್ಲಿ ಈ ವಿವೇಕಾನಂದನನ್ನು ಒಮ್ಮೆಯಾದರೂ ಹತ್ತಿರದಿಂದ ನೋಡಬೇಕೆಂದು ಜನ ಒಬ್ಬರನ್ನೊಬ್ಬರು ತಳ್ಳಿಕೊಂಡು ಬರುತ್ತಿದ್ದಾಗ–ಯಾವ ಗೌರವ-ಸನ್ಮಾನದ ನೂರನೇ ಒಂದು ಭಾಗವೇ ಸಾಮಾನ್ಯ ಮನುಷ್ಯನ ತಲೆ ತಿರುಗುವಂತೆ ಮಾಡಬಲ್ಲುದೋ ಅಂಥದನ್ನು ನಾನು ಶ್ರೀರಾಮಕೃಷ್ಣರ ಕೃಪಾ ವಿಶೇಷದಿಂದ ಸುಲಭವಾಗಿ ಜೀರ್ಣಿಸಿಕೊಳ್ಳಲು ಸಮರ್ಥನಾದೆ. ಅವರ ಕೃಪೆಯಿಂದ, ಎಲ್ಲೆ ಲ್ಲಿಯೂ ವಿಜಯಲಕ್ಷ್ಮಿ ನನಗೊಲಿದಿದ್ದಾಳೆ. ಈಗ ನಾನು ಈ ದೇಶಕ್ಕಾಗಿ ಏನಾದರೂ ಮಾಡ ಬೇಕೆಂದು ಸಂಕಲ್ಪಿಸಿದ್ದೇನೆ. ನೀವೆಲ್ಲರೂ ನಿಮ್ಮ ಸಂಶಯಗಳನ್ನು, ತಪ್ಪು ಕಲ್ಪನೆಗಳನ್ನು ಬಿಟ್ಟು ಈ ಕಾರ್ಯದಲ್ಲಿ ನೆರವಾಗಿ. ಆಮೇಲೆ ಶ್ರೀರಾಮಕೃಷ್ಣರ ಕೃಪೆಯಿಂದ ಏನೇನು ಅದ್ಭುತಗಳೆಲ್ಲ ನಡೆಯಲಿವೆ ಎಂಬುದನ್ನು ನೀವೇ ನೋಡುವಿರಿ!”

ಸ್ವಾಮೀಜಿ ಹೇಳುವುದನ್ನೆಲ್ಲ ಕೇಳುತ್ತಿದ್ದ ಯೋಗಾನಂದರು ಈಗ ಹೇಳಿದರು, “ನಿಜ, ನೀನೇನು ಸಂಕಲ್ಪ ಮಾಡಿದರೂ ಅದು ನಡೆದೇ ನಡೆಯುತ್ತದೆ. ನಾವೆಲ್ಲರೂ ನಿನ್ನ ನಾಯಕತ್ವವನ್ನು ಸ್ವೀಕರಿಸಿ ನಿನ್ನನ್ನು ಅನುಸರಿಸಲು ಸಿದ್ಧರಾಗಿದ್ದೇವೆ. ಶ್ರೀರಾಮಕೃಷ್ಣರೇ ನಿನ್ನ ಮೂಲಕ ಕೆಲಸ ಮಾಡುತ್ತಿರು ವುದನ್ನು ನಾನು ಸ್ಪಷ್ಟವಾಗಿ ಕಾಣುತ್ತಿದ್ದೇನೆ. ಆದರೂ ಕೂಡ ಒಂದೊಂದು ಸಲ ಮನಸ್ಸಿನಲ್ಲಿ ಸಂಶಯಗಳು ಏಳುವುದುಂಟು ಎಂಬುದನ್ನು ನಾನು ಒಪ್ಪಿಕೊಳ್ಳಬೇಕಾಗುತ್ತದೆ. ಏಕೆಂದರೆ ನಾವೆಲ್ಲ ನೋಡಿದಂತೆ, ಶ್ರೀರಾಮಕೃಷ್ಣರ ಕಾರ್ಯವಿಧಾನ ಸಂಪೂರ್ಣ ವಿಭಿನ್ನವಾದುದು. ಆದ್ದರಿಂದ ನಾವೆಲ್ಲ ಶ್ರೀರಾಮಕೃಷ್ಣರ ಬೋಧನೆಗೆ ವ್ಯತಿರಿಕ್ತವಾಗಿ ನಡೆಯುತ್ತಿಲ್ಲ ತಾನೆ?– ಎಂಬ ಪ್ರಶ್ನೆಯನ್ನು ನನಗೆ ನಾನೇ ಕೇಳಿಕೊಳ್ಳುವಂತಾಗುತ್ತದೆ.”

ಸ್ವಾಮೀಜಿ: “ನೋಡು, ವಿಷಯ ಇದು–ಏನೆಂದರೆ, ಶ್ರೀರಾಮಕೃಷ್ಣರನ್ನು ಅವರ ಶಿಷ್ಯರು ಎಷ್ಟರ ಮಟ್ಟಿಗೆ ಅರ್ಥಮಾಡಿಕೊಂಡಿದ್ದಾರೋ ಅದಕ್ಕಿಂತಲೂ ಎಷ್ಟೋ ಪಾಲು ಅವರು ಹೆಚ್ಚು ಮಹಿಮಾವಂತರು. ಅನಂತ ಬಗೆಗಳಲ್ಲಿ ವ್ಯಕ್ತವಾಗಬಲ್ಲ ಅನಂತ ಆಧ್ಯಾತ್ಮಿಕ ಭಾವಗಳ ಸಾಕಾರ ಮೂರ್ತಿ ಅವರು. ಬ್ರಹ್ಮಜ್ಞಾನಕ್ಕೂ ಒಂದು ಮಿತಿಯನ್ನು ಕಾಣಬಹುದೇನೋ; ಆದರೆ ನಮ್ಮ ಗುರುದೇವನ ಮನದಾಳವನ್ನು ಅಳೆದು ನೋಡಬಲ್ಲವರಿಲ್ಲ! ಅವರ ಒಂದೇ ಒಂದು ಕೃಪಾ ಕಟಾಕ್ಷವು ಈ ಕ್ಷಣದಲ್ಲಿ ಸಹಸ್ರಾರು ವಿವೇಕಾನಂದರನ್ನು ಸೃಷ್ಟಿಸಬಲ್ಲುದು. ಆದರೆ ಅದಕ್ಕೆ ಬದಲಾಗಿ ಅವರು ಈ ಬಾರಿ ನನ್ನನ್ನು ತಮ್ಮ ಉಪಕರಣವನ್ನಾಗಿ ಮಾಡಿಕೊಂಡು ನನ್ನ ಮೂಲಕ ಕೆಲಸ ಮಾಡಬೇಕೆಂದು ಸಂಕಲ್ಪಿಸಿದರೆ ಅವರ ಇಚ್ಛೆಗೆ ಶಿರಬಾಗುವವನು ನಾನು.”

ಶ್ರೀರಾಮಕೃಷ್ಣರು ಕೇವಲ ಒಬ್ಬ ವ್ಯಕ್ತಿಯಲ್ಲ, ಅವರೊಂದು ಮಹಾತತ್ತ್ವ. ಶ್ರೀರಾಮಕೃಷ್ಣರ ಶಿಷ್ಯರಲ್ಲಿ ಇದನ್ನು ಕಂಡುಕೊಂಡವರೆಂದರೆ ಬಹುಶಃ ಸ್ವಾಮಿ ವಿವೇಕಾನಂದರೊಬ್ಬರೇ. ಶ್ರೀರಾಮ ಕೃಷ್ಣರು ಮಹತ್ತರವಾದ ತ್ಯಾಗವನ್ನು ಮಾಡಿ ಮಹತ್ತರವಾದ ಸಾಕ್ಷಾತ್ಕಾರಗಳನ್ನು ಪಡೆದುಕೊಂಡ ಮಹಾಮಹಿಮರಷ್ಟೇ ಅಲ್ಲ; ಅವರು ಜೀವರೂಪದ ಶಿವಸೇವೆಯ, ನರರೂಪೀ ನಾರಾಯಣನ ಸೇವೆಯ ಮರ್ಮವನ್ನೂ ತಿಳಿಸಿಕೊಟ್ಟವರು. ಇದನ್ನು ಅರಿತವರು ಸ್ವಾಮಿ ವಿವೇಕಾನಂದ ರೊಬ್ಬರೇ. ತಮ್ಮಿಂದ ಜನಕೋಟಿಗೆ ಸಹಾಯವಾಗಲೆಂದು ಶ್ರೀರಾಮಕೃಷ್ಣರು ಅತ್ಯುನ್ನತ ಸಮಾ ಧಿಯ ಆನಂದವನ್ನೂ ಬದಿಗೊತ್ತಿ, ಭಾವಮುಖದಲ್ಲಿ ಜೀವಿಸಿದವರಲ್ಲವೆ? ತಮ್ಮ ಜೀವಿತದ ಪ್ರತಿಕ್ಷಣದಲ್ಲೂ ಸರ್ವ ಜೀವರನ್ನೂ ನಾರಾಯಣ ಸ್ವರೂಪಿಗಳಾಗಿ ಕಂಡು ಪೂಜಿಸಿದವರಲ್ಲವೆ? ಶ್ರೀರಾಮಕೃಷ್ಣರು ಜನರ ಬಡತನದ ಬವಣೆಯನ್ನು ಕಂಡು ಕಣ್ಣೀರು ಸುರಿಸಿದ್ದನ್ನು ಅವರ ಶಿಷ್ಯರು ಕಂಡಿಲ್ಲವೆ? ನಿಜ. ಆದರೆ ಶ್ರೀರಾಮಕೃಷ್ಣರ ದಿವ್ಯ ಸಾಕ್ಷಾತ್ಕಾರಗಳ ಪ್ರಭೆ ಎಂಬುದು, ಅವರ ಭಾವಸಮಾಧಿಗಳ ತುರೀಯಾನುಭವ ಎಂಬುದು ಅವರ ದೀನದಯಾಪರತೆಯ ವ್ಯಕ್ತಿತ್ವವನ್ನು ಮರೆಮಾಡಿಬಿಟ್ಚಿತ್ತು. ಜೊತೆಗೆ ಅವರು ಪ್ರತಿಯೊಬ್ಬರನ್ನೂ ಭಗವತ್ಸಾಕ್ಷಾತ್ಕಾರ ಮಾಡಿಕೊಳ್ಳು ವಂತೆ ಪ್ರೋತ್ಸಾಹಿಸುತ್ತಿದ್ದವರು. ಆದ್ದರಿಂದ ಶ್ರೀರಾಮಕೃಷ್ಣರ ಜೀವನ-ಸಂದೇಶಗಳನ್ನು ವಿಮರ್ಶಿಸಿ ವ್ಯಾಖ್ಯಾನಿಸುವ ಅಧಿಕಾರ ಎಲ್ಲರಿಗೂ ದಕ್ಕುವಂಥದಲ್ಲ; ಅದು ಸ್ವಾಮಿ ವಿವೇಕಾನಂದ ರಂತಹ ಅಸಾಮಾನ್ಯ ಪ್ರತಿಭಾನ್ವಿತರಿಗೆ ಮಾತ್ರವೇ ಮೀಸಲು. ಶ್ರೀರಾಮಕೃಷ್ಣರ ಜೀವನ-ಸಂದೇಶ ಗಳಲ್ಲಿನ ಮಾನವೀಯ ಅಂಶವನ್ನು ಎತ್ತಿ ತೋರಿಸುವ ಕಾರ್ಯವನ್ನು ಅವರ ಈ ಮಹಾ ಶಿಷ್ಯನೇ ಮಾಡಬೇಕಾಗಿತ್ತು. ತ್ಯಾಗ ಮತ್ತು ಸೇವೆ–ಇವೆರಡೂ ಪರಸ್ಪರ ವಿರುದ್ಧವಾದ ಆದರ್ಶಗಳು ಎಂಬ ಭಾವನೆ ಆ ಕಾಲದಲ್ಲಿ ಆಳವಾಗಿ ಬೇರೂರಿತ್ತು. ಆದರೆ, ಅದು ತಪ್ಪು; ಅವೆರಡೂ ಪರ ಸ್ಪರ ಪೂರಕವಾಗಬಲ್ಲ ಆದರ್ಶಗಳು ಎಂಬುದನ್ನು ಸಾಧಿಸಿ ತೋರಿಸಿದ ಹೆಗ್ಗಳಿಕೆ ಸ್ವಾಮೀಜಿಯವ ರದು. ಅಲ್ಲದೆ ‘ರಾಮಕೃಷ್ಣ ಮಿಷನ್ನ’ನ್ನು ಸ್ಥಾಪಿಸುವುದರ ಮೂಲಕ ದಿವ್ಯವಾದ ಈ ಅವಳಿ ಆದರ್ಶಗಳಿಗೆ ಒಂದು ಮೂರ್ತ ಸ್ವರೂಪವನ್ನು ಕೊಟ್ಟ ಸಾಧನೆ ಸ್ವಾಮೀಜಿಯವರದು.

ಸ್ವಾಮಿ ಯೋಗಾನಂದರೊಂದಿಗಿನ ಅಂದಿನ ಈ ವಾಗ್ವಾದ ಅಲ್ಲಿಗೆ ಮುಕ್ತಾಯವಾಯಿತು. ಆದರೆ ಇತರ ಸೋದರ ಸಂನ್ಯಾಸಿಗಳಲ್ಲಿ ಸ್ವಾಮೀಜಿಯವರ ಕಾರ್ಯೋದ್ದೇಶಗಳ ಹಾಗೂ ಕಾರ್ಯ ವಿಧಾನಗಳ ಬಗ್ಗೆ ಶಂಕೆ-ಅಸಮಾಧಾನ ಇದ್ದೇ ಇತ್ತು. ಇದು ಮತ್ತೆ ಕೆಲವು ಸಂದರ್ಭಗಳಲ್ಲಿ ಹೊರಬಂದಿತು. ತನ್ಮೂಲಕ ಅವರ ಈ ಶಂಕೆ-ಅಸಮಾಧಾನ ಪರಿಹಾರವಾಗಲೂ ಸಾಧ್ಯವಾಯಿತು. ಮತ್ತೊಂದು ದಿನ ಈ ಸೋದರ ಸಂನ್ಯಾಸಿಗಳು ಹಾಗೂ ಕೆಲವು ಗೃಹೀ ಭಕ್ತರು ಬಲರಾಮ ಬಾಬುವಿನ ಮನೆಯಲ್ಲಿ ಕಲೆತಿದ್ದರು. ಇವರೊಂದಿಗೆ ಸ್ವಾಮೀಜಿ ಸರಸ ಸಂಭಾಷಣೆಯಲ್ಲಿ ನಿರತ ರಾಗಿದ್ದರು. ಸ್ವಲ್ಪ ತಮಾಷೆಯಾಗಿಯೇ ಕೆಲವು ಗುರುಭಾಯಿಗಳು ಸ್ವಾಮೀಜಿಯವರನ್ನು ಛೇಡಿಸು ತ್ತಿದ್ದರು; ಅದಕ್ಕೆ ಸ್ವಾಮೀಜಿ ಅಷ್ಟೇ ತಮಾಷೆಯಾಗಿ ಉತ್ತರಿಸುತ್ತಿದ್ದರು. ಆಗ ಸ್ವಾಮಿ ಅದ್ಭುತಾ ನಂದರು\footnote{*ಸ್ವಾಮಿ ಅದ್ಭುತಾನಂದರ ಹಿಂದಿನ ಹೆಸರು ಲಾಟು ಎಂದು. ಈತ ಶ್ರೀರಾಮಕೃಷ್ಣರ ಗೃಹೀಭಕ್ತನಾದ ರಾಮಚಂದ್ರದತ್ತನ ಮನೆಯಲ್ಲಿ ಕೆಲಸಕ್ಕಿದ್ದವನು. ಶ್ರೀರಾಮಕೃಷ್ಣರ ಸೇವೆಗೆ ನಿಯೋಜಿತನಾದ ಲಾಟು, ಮುಂದೆ ಅವರ ಸತ್ಸಹವಾಸದ ಪರಿಣಾಮವಾಗಿ ಸಂನ್ಯಾಸಿಯಾದ. ಇವನಿಗೆ ಶ್ರೀರಾಮಕೃಷ್ಣರಲ್ಲಿದ್ದ ಅದ್ಭುತ ಶ್ರದ್ಧಾಭಕ್ತಿ ಗಳನ್ನು ಕಂಡು ಹಾಗೂ ಅದರಿಂದಾಗಿ ಇವನ ವ್ಯಕ್ತಿತ್ವದಲ್ಲಾದ ಪರಿವರ್ತನೆಯನ್ನು ಕಂಡು ಸ್ವಾಮೀಜಿಯವರೇ ಇವನಿಗೆ ಸ್ವಾಮಿ ಅದ್ಭುತಾನಂದ ಎಂಬ ಹೆಸರನ್ನು ಕೊಟ್ಟರು. ಆದರೆ ಲಾಟು ನಿರಕ್ಷರಕುಕ್ಷಿ. ಶ್ರೀರಾಮಕೃಷ್ಣರೇ ಈತನಿಗೆ ಅಕ್ಷರಾಭ್ಯಾಸ ಮಾಡಿಸಲು ಹೊರಟು ಸೋತುಹೋಗಿದ್ದರು.} ಅವರಿಗೊಂದು ಪ್ರಶ್ನೆ ಹಾಕಿದರು: “ಅಲ್ಲ ನರೇನ್, ನಮ್ಮ ಗುರುಮಹಾರಾಜರು ಎಲ್ಲಕ್ಕಿಂತ ಹೆಚ್ಚಾಗಿ ಭಕ್ತಿ ಹಾಗೂ ಸಾಕ್ಷಾತ್ಕಾರದ ಬಗ್ಗೆ ಒತ್ತಿ ಹೇಳುತ್ತಿದ್ದರು. ಆದರೆ ನೀನು ಮಾತ್ರ ಯಾವಾಗ ನೋಡಿದರೂ ಧರ್ಮಪ್ರಸಾರ ಮಾಡಬೇಕು, ಬಡವರನ್ನು ಮೇಲೆತ್ತಬೇಕು, ರೋಗಿಗಳ ಸೇವೆ ಮಾಡಬೇಕು ಅಂತ ಹೇಳುತ್ತೀಯೆ. ಇವುಗಳೆಲ್ಲ ನಮ್ಮ ಮನಸ್ಸನ್ನು ಅಂತರ್ ಮುಖಗೊಳ್ಳಲು ಬಿಡದೆ ಹೊರಕ್ಕೆ ಎಳೆಯುವುದಿಲ್ಲವೆ? ಸಾಧನೆಗೆ ಇದೊಂದು ದೊಡ್ಡ ಆತಂಕವಲ್ಲವೆ? ಅಲ್ಲದೆ ಮಠಗಳನ್ನು-ಸೇವಾಶ್ರಮಗಳನ್ನು ಸ್ಥಾಪಿಸಬೇಕು, ರಾಷ್ಟ್ರಭಕ್ತಿಯನ್ನು ಬೆಳೆಸಬೇಕು, ಸಂಘಗಳನ್ನು ಕಟ್ಟಬೇಕು, ಅಂತೆಲ್ಲ ಎನ್ನುವೆಯಲ್ಲ, ಇವೆಲ್ಲ ಇಂಗ್ಲಿಷಿನವರ ರೀತಿ ನೀತಿಗಳು. ಇವು ಏನಿದ್ದರೂ ಅವರಿಗೆ ಸರಿ. ಮತ್ತು ನೀನು ಹೊಸ ರೀತಿಯ ಸಂನ್ಯಾಸಿಗಳ ಗುಂಪನ್ನು ಕಟ್ಟಬೇಕು, ಆ ಸಂನ್ಯಾಸಿಗಳು ಇಂದಿನ ಸಂನ್ಯಾಸಸಂಪ್ರದಾಯಕ್ಕಿಂತಲೂ ಉದಾರ ಮನೋಭಾವದವರಾಗಿರಬೇಕು ಅಂತ ಏನೇನೋ ಹೇಳುತ್ತೀಯಲ್ಲ, ಇವೆಲ್ಲ ಶ್ರೀರಾಮಕೃಷ್ಣರು ನಮ್ಮ ಮುಂದಿಟ್ಟಂತಹ ಆದರ್ಶಕ್ಕಿಂತ ತೀರ ವ್ಯತಿರಿಕ್ತವಾದದ್ದು. ಅದನ್ನೆಲ್ಲ ನಾನು ಒಪ್ಪಿಕೊಳ್ಳು ವುದಕ್ಕಾಗುವುದಿಲ್ಲ.” ಈ ಮಾತುಗಳನ್ನೆಲ್ಲ ಸ್ವಾಮೀಜಿ ತಮಾಷೆಯಾಗಿಯೇ ತೆಗೆದುಕೊಂಡರು. ಮತ್ತು ಅವರೂ ತಮಾಷೆಯಾಗಿಯೇ ಹೇಳಿದರು, “ನಿನಗವೆಲ್ಲ ಏನು ಗೊತ್ತೋ? ನೀನೊಬ್ಬ ತಿಳಿವಳಿಕೆಯಿಲ್ಲದ ಮನುಷ್ಯ! ಅಂತೂ ಶ್ರೀರಾಮಕೃಷ್ಣರಿಗೆ ತಕ್ಕ ಶಿಷ್ಯ ಅಂದರೆ ನೀನೆ ಕಣೊ! ಗುರುವಿನಂತೆ ಶಿಷ್ಯ!ನಿನ್ನ ವಿದ್ಯಾಭ್ಯಾಸವೆಲ್ಲ ‘ಕ’ ಬರೆಯುವಷ್ಟಕ್ಕೇ ಮುಗಿದು ಹೋಯಿತು. ನೀವೆಲ್ಲ ‘ಭಕ್ತರು’! ಅರ್ಥಾತ್ ಭಾವಾವೇಶದ ಮೂರ್ಖರು! ನಿಮಗೆಲ್ಲ ಧರ್ಮದ ವಿಷಯ ಏನು ಗೊತ್ತೋ? ನೀವೆಲ್ಲ ಮಕ್ಕಳು! ನೀವೇನಿದ್ದರೂ ಕೈಜೋಡಿಸಿಕೊಂಡು ‘ಹೇ ಭಗವಂತ, ನಿನ್ನ ಮೂಗು ಎಷ್ಟು ಸುಂದರವಾಗಿದೆ! ನಿನ್ನ ಕಣ್ಣುಗಳು ಎಷ್ಟು ಮನೋಹರವಾಗಿವೆ!’ ಎಂದು ಹೀಗೆ ಅಸಂಬದ್ಧ ಪ್ರಲಾಪ ಮಾಡುವುದಕ್ಕೇ ಲಾಯಕ್ಖು! ಅಲ್ಲದೆ, ನಿಮ್ಮ ಮುಕ್ತಿ ಅಲ್ಲಿ ನಿಮಗಾಗಿ ಕಾದು ಕುಳಿತಿದೆ, ನಿಮ್ಮ ಪ್ರಾಣ ಹೋಗುವ ಘಳಿಗೆಯಲ್ಲಿ ಶ್ರೀರಾಮಕೃಷ್ಣರೇ ಸ್ವತಃ ಬಂದು ನಿಮ್ಮ ಕೈಹಿಡಿದು ಕರೆದೊಯ್ಯುತ್ತಾರೆ ಅಂತ ನೀವು ನಂಬಿಕೊಂಡು ಕುಳಿತಿದ್ದೀರಿ. ನಿಮ್ಮ ದೃಷ್ಟಿಯಲ್ಲಿ ಈ ಅಧ್ಯಯನ, ಧರ್ಮಪ್ರಚಾರ, ಮಾನವ ಕಲ್ಯಾಣಕಾರ್ಯ ಇವೆಲ್ಲ ಕೇವಲ ಮಾಯೆ, ಅಲ್ಲವೆ? ಏಕೆಂದರೆ ಸ್ವತಃ ಶ್ರೀರಾಮಕೃಷ್ಣರು ಇದನ್ನೆಲ್ಲ ಮಾಡಲಿಲ್ಲವಲ್ಲ! ಅಲ್ಲದೆ ಅವರು ಯಾರಿಗೋ ಹೇಳಿದರಂತಲ್ಲ, ‘ಮೊದಲು ಸಾಧನೆ ಮಾಡಿ ದೇವರ ಸಾಕ್ಷಾತ್ಕಾರ ಮಾಡಿಕೊಳ್ಳಿ ಜಗತ್ಕಲ್ಯಾಣದ ಕಾರ್ಯವೆಲ್ಲ ಕೇವಲ ದುರಹಂಕಾರದ ಮಾತು’ ಅಂತ! ಆದ್ದರಿಂದ ನೀವೆಲ್ಲ ಕೇವಲ ಭಗವಂತನ ಸಾಕ್ಷಾತ್ಕಾರ ಮಾಡಿಕೊಳ್ಳುವುದಕ್ಕಾಗಿ ಕುಳಿತಿದ್ದೀರಿ! ಭಗವಂತನ ಸಾಕ್ಷಾತ್ಕಾರ ವೆಂದರೆ ಬಾಳೆಯ ಹಣ್ಣು ನುಂಗಿದಂತೆ ಅಂತ ತಿಳಿದುಕೊಂಡಿದ್ದೀರಿ ನೀವು! ದೇವರು ಎಂದರೆ ಮಕ್ಕಳ ಕೈಯಲ್ಲಿರುವ ಆಟದ ಬೊಂಬೆ ಅಂತ ತಿಳಿದುಕೊಂಡಿದ್ದೀರಿ ನೀವು!... ”

ಹೀಗೆ ಸ್ವಾಮೀಜಿ ಮೊದಲು ತಮಾಷೆಯಾಗಿಯೇ ಮಾತನಾಡತೊಡಗಿದರೂ ಬರಬರುತ್ತ ಅವರ ಮಾತಿನ ದನಿ ಹರಿತವಾಯಿತು; ವಾತಾವರಣ ಗಂಭೀರವಾಯಿತು. ಸ್ವಾಮೀಜಿಯವರ ವಾಕ್ ಪ್ರಹಾರವನ್ನು ತಿಂದು ಅದ್ಭುತಾನಂದರು ದಂಗಾಗಿ ಕುಳಿತರು. ಆಗ ಅಲ್ಲಿದ್ದ ಮತ್ತೊಬ್ಬರು ಸಂನ್ಯಾಸಿಗಳು ಅದ್ಭುತಾನಂದರ ಪರವಾಗಿ ಮಾತನಾಡಲು ಪ್ರಯತ್ನಿಸಿದರು. ಇದರಿಂದ ಸ್ವಾಮೀಜಿ ಮತ್ತಷ್ಟು ಸಿಟ್ಟಿಗೆದ್ದು ಗುಡುಗಿದರು–

“ಏನು! ಶ್ರೀರಾಮಕೃಷ್ಣರನ್ನು ನನಗಿಂತಲೂ ಚೆನ್ನಾಗಿ ಅರ್ಥಮಾಡಿಕೊಂಡವರೊ ನೀವು? ಜ್ಞಾನ ಅನ್ನುವುದು ಕೇವಲ ಶುಷ್ಕ, ಅದನ್ನು ಪಡೆಯುವ ಮಾರ್ಗವೂ ಅಷ್ಟೇ ಶುಷ್ಕ ಮತ್ತು ಅದು ನಮ್ಮ ಹೃದಯದ ಮಧುರ ಭಾವನೆಗಳನ್ನೆಲ್ಲ ನಾಶಗೊಳಿಸಿಬಿಡುತ್ತದೆ ಎಂದಲ್ಲವೆ ನಿಮ್ಮ ಅಭಿಪ್ರಾಯ? ನಿಮ್ಮ ಭಕ್ತಿ ಎನ್ನುವುದೆಲ್ಲ ಕೇವಲ ಅಸಂಬದ್ಧ ಭಾವಾವೇಗ! ಅದು ಕೊನೆಗೆ ನಿಮ್ಮನ್ನು ಕೆಲಸಕ್ಕೆ ಬಾರದ ಷಂಡರನ್ನಾಗಿ ಮಾಡಿಬಿಟ್ಟಿರುತ್ತದೆ. ಶ್ರೀರಾಮಕೃಷ್ಣರ ಬಗ್ಗೆ ನೀವು ತಿಳಿದ ನಿಮ್ಮ ಅಲ್ಪ ಜ್ಞಾನವನ್ನು ಪ್ರಚಾರ ಮಾಡಹೊರಟವರಲ್ಲವೆ ನೀವು? ಬಾಯ್ಮುಚ್ಚಿ! ಯಾರು ಕೇಳುತ್ತಾರೆ ನಿಮ್ಮ ಶ್ರೀರಾಮಕೃಷ್ಣರನ್ನು! ನಿಮ್ಮ ಭಕ್ತಿ ಮುಕ್ತಿಯನ್ನೆಲ್ಲ ಯಾರು ಕೇಳುತ್ತಾರೆ ಅಂತ ತಿಳಿದುಕೊಂಡಿರಿ? ನಿಮ್ಮ ಶಾಸ್ತ್ರಗಳು ಏನು ಹೇಳುತ್ತವೆ ಎನ್ನುವುದು ಯಾರಿಗೆ ಬೇಕಾಗಿದೆ? ಘೋರ ತಮಸ್ಸಿನಲ್ಲಿ ಮುಳುಗಿರುವ ನನ್ನ ದೇಶಬಾಂಧವರನ್ನು ಎಚ್ಚರಿಸಿ ಮೇಲೆತ್ತಲು ನನ್ನಿಂದ ಸಾಧ್ಯವಾಗುವುದಾದರೆ, ಅವರೆಲ್ಲರನ್ನೂ ಅವರವರ ಕಾಲ ಮೇಲೆ ನಿಲ್ಲಿಸಿ ಅವರನ್ನು ಪುರುಷಸಿಂಹರನ್ನಾಗಿ ಮಾಡಲು ನನ್ನಿಂದ ಸಾಧ್ಯವಾಗುವುದಾದರೆ, ನಾನು ಸಾವಿರ ಸಲ ಬೇಕಾದರೂ ಸಂತೋಷದಿಂದ ನರಕಕ್ಕೆ ಹೋದೇನು. ನಾನು ಶ್ರೀರಾಮಕೃಷ್ಣರ ಅನುಯಾ ಯಿಯೂ ಅಲ್ಲ, ಯಾರ ಅನುಯಾಯಿಯೂ ಅಲ್ಲ! ಆದರೆ ಯಾರು ನನ್ನ ಯೋಜನೆಗಳನ್ನು ಕಾರ್ಯಗತಗೊಳಿಸುತ್ತಾರೊ ಅವರ ಅನುಯಾಯಿ ನಾನು. ನಾನು ರಾಮಕೃಷ್ಣರ ಸೇವಕನೂ ಅಲ್ಲ, ಯಾರ ಸೇವಕನೂ ಅಲ್ಲ. ಆದರೆ ಯಾರು ತನ್ನ ವೈಯಕ್ತಿಕ ಮುಕ್ತಿಯ ಕಡೆಗೂ ಲಕ್ಷ್ಯಕೊಡದೆ ಇತರರಿಗೆ ನೆರವಾಗುವನೋ ಇತರರ ಸೇವೆ ಮಾಡುವನೋ ಅವರ ಸೇವಕ ನಾನು.”

ಹೀಗೆ ಹೇಳುತ್ತ ಹೇಳುತ್ತ ಸ್ವಾಮೀಜಿಯವರ ಸ್ವರ ಗದ್ಗದವಾಯಿತು. ಭಾವಾವೇಶದಿಂದ ಅವರ ಇಡೀ ಶರೀರ ಕಂಪಿಸತೊಡಗಿತು. ಅವರಿಂದ ಇನ್ನು ತಡೆದುಕೊಳ್ಳಲು ಸಾಧ್ಯವಾಗಲಿಲ್ಲ. ಕಣ್ಣುಗಳಿಂದ ಧಾರಾಕಾರವಾಗಿ ಅಶ್ರುಪ್ರವಾಹ! ಕ್ಷಣಾರ್ಧದಲ್ಲಿ ಅವರು ಕುಳಿತಲ್ಲಿಂದೆದ್ದು ತಮ್ಮ ಮಲಗುವ ಕೋಣೆಗೆ ಓಡಿದರು. ಈಗ ಅವರ ಸೋದರ ಸಂನ್ಯಾಸಿಗಳೆಲ್ಲ ಭಯಗ್ರಸ್ತರಾದರು. ತಾವು ಆ ರೀತಿಯಲ್ಲಿ ಸ್ವಾಮೀಜಿಯವರನ್ನು ಟೀಕಿಸಿದ್ದಕ್ಕಾಗಿ ತಮ್ಮನ್ನೇ ಹಳಿದುಕೊಂಡರು. ಸ್ವಲ್ಪ ಹೊತ್ತಿನ ಬಳಿಕ ಅವರಲ್ಲಿ ಕೆಲವರು ಭಯಪಡುತ್ತಲೇ ಸ್ವಾಮೀಜಿಯವರ ಕೋಣೆಗೆ ಹೋದರು. ಮೆಲ್ಲಗೆ ಒಳಗೆ ಹೋಗಿ ನೋಡುತ್ತಾರೆ–ಸ್ವಾಮೀಜಿಯವರು ಧ್ಯಾನಸ್ಥರಾಗಿ ಕುಳಿತಿದ್ದಾರೆ! ಅವರ ಅರೆಮುಚ್ಚಿದ ಕಣ್ಣುಗಳಿಂದ ಅಶ್ರುಧಾರೆ ಹರಿಯುತ್ತಿದೆ! ಮೈ ರೋಮಗಳು ಸೆಟೆದು ನಿಂತಿವೆ! ಸ್ವಾಮೀಜಿ ಭಾವಸಮಾಧಿಯಲ್ಲಿ ಲೀನರಾಗಿರುವಂತೆ ಕಾಣುತ್ತಿದೆ! ಈಗ ಯಾರೂ ಅವರನ್ನು ಮಾತನಾಡಿಸುವ ಸಾಹಸ ಮಾಡಲಿಲ್ಲ. ಸುಮಾರು ಒಂದು ಗಂಟೆಯ ಬಳಿಕ ಸ್ವಾಮೀಜಿ ಧ್ಯಾನ ಸ್ಥಿತಿಯಿಂದ ಹೊರಬಂದರು. ಎದ್ದು ಹೋಗಿ ಮುಖ ತೊಳೆದುಕೊಂಡರು. ಅನಂತರ ಸೋದರ ಸಂನ್ಯಾಸಿಗಳೂ ಗೃಹೀಭಕ್ತರೂ ಕುಳಿತಿದ್ದ ಕೋಣೆಗೆ ಬಂದರು. ಎಲ್ಲರೂ ಅವರ ಬರವಿಗಾಗಿಯೇ ಕಾಯುತ್ತ ಕುಳಿತಿದ್ದರು. ಇಡೀ ವಾತಾವರಣ ನೀರವ, ಗಂಭೀರ! ಬಳಿಕ ಸ್ವಾಮೀಜಿಯವರೇ ಮೌನವನ್ನು ಮುರಿದು ಮಾತನಾಡಿದರು, “ನಿಜವಾದ ಭಕ್ತಿಯನ್ನು ಪ್ರಾಪ್ತ ಮಾಡಿಕೊಂಡವರ ಹೃದಯ-ನರಮಂಡಲ ಎಲ್ಲ ಅತ್ಯಂತ ಸುಕೋಮಲವಾಗಿಬಿಡುತ್ತದೆ. ಆಗ ಅವರ ಮೇಲೊಂದು ಹೂವನ್ನು ಎಸೆದರೂ ಅವರು ತಡೆದುಕೊಳ್ಳಲಾರರು. ನಾನು ಬಹಳ ಹೊತ್ತು ಶ್ರೀರಾಮಕೃಷ್ಣರ ವಿಚಾರವಾಗಿ ಮಾತನಾಡಲೂ ಆರೆ, ಆಲೋಚಿಸಲೂ ಆರೆ. ಏಕೆಂದರೆ ಭಾವದಿಂದ ನನ್ನೆದೆ ತುಂಬಿ ಬಿಡುತ್ತದೆ. ಆದ್ದರಿಂದ ನನ್ನ ಹೃದಯದೊಳಗೆ ಉಕ್ಕಿಹರಿಯುತ್ತಿರುವ ಭಕ್ತಿಯ ಆವೇಗವನ್ನು ತಡೆಹಿಡಿದಿಡಲು ನಾನು ಯಾವಾಗಲೂ ಪ್ರಯತ್ನ ಮಾಡುತ್ತಲೇ ಇರುತ್ತೇನೆ. ನಾನು ಜ್ಞಾನವೆಂಬ ಕಬ್ಬಿಣದ ಸರಪಣಿಯಿಂದ ನನ್ನನ್ನು ಚೆನ್ನಾಗಿ ಬಿಗಿದಿಟ್ಟುಕೊಳ್ಳಲು ಪ್ರಯತ್ನಿಸುತ್ತಿದ್ದೇನೆ. ಏಕೆಂದರೆ ನನ್ನ ಮಾತೃಭೂಮಿಗಾಗಿ ನಾನು ಮಾಡಬೇಕಾಗಿರುವ ಕಾರ್ಯವಿನ್ನೂ ಮುಗಿದಿಲ್ಲ; ನಾನು ಪಾಶ್ಚಾತ್ಯ ರಾಷ್ಟ್ರಗಳಿಗೆ ನೀಡಬೇಕಾದ ಸಂದೇಶವನ್ನಿನ್ನೂ ಪೂರ್ಣವಾಗಿ ನೀಡಿಲ್ಲ. ಆದ್ದರಿಂದಲೇ ಯಾವಾಗ ಈ ಭಕ್ತಿಭಾವವು ನನ್ನಲ್ಲಿ ಉದಿಸಿ ನನ್ನನ್ನು ಆವರಿಸಿಕೊಳ್ಳುವಂತೆ ಕಂಡುಬರು ತ್ತದೆಯೋ ಆಗ ತಪೋಮಯ ಜ್ಞಾನವನ್ನು ಜಾಗೃತಗೊಳಿಸಿಕೊಂಡು ಆ ಭಕ್ತಿಭಾವಕ್ಕೆ ಹೊಡೆತ ಕೊಡುತ್ತೇನೆ, ಬಂಡೆಯಂತಾಗಿಬಿಡುತ್ತೇನೆ. ಓ! ನಾನು ಮಾಡಬೇಕಾದ ಕಾರ್ಯ ಇನ್ನೂ ಬಹಳಷ್ಟಿದೆ. ಶ್ರೀರಾಮಕೃಷ್ಣರ ದಾಸ ನಾನು. ಅವರೇ ನನಗೆ ಕೆಲಸವನ್ನು ಕೊಟ್ಟು ಹೋಗಿದ್ದಾರೆ. ಇದು ಮುಗಿಯುವವರೆಗೆ ಅವರು ನನಗೆ ವಿಶ್ರಾಂತಿ ಕೊಡುವು ದಿಲ್ಲ. ಆಹ್! ಅವರ ವಿಷಯವಾಗಿ ನಾನೇನೆಂದು ಹೇಳಲಿ! ಆಹ್! ಅವರಿಗೆ ನನ್ನ ಮೇಲಿರುವ ಪ್ರೀತಿಯನ್ನು ಏನೆಂದು ಹೇಳಲಿ!”

ಈಗ ಪುನಃ ಸ್ವಾಮೀಜಿಯವರಲ್ಲಿ ಭಕ್ತಿಯ ಕಾವು ಏಳುತ್ತಿರುವುದನ್ನು ಗಮನಿಸಿದ ಯೋಗಾ ನಂದರು ಮತ್ತಿತರರು ಜಾಗೃತರಾದರು. ಸ್ವಾಮೀಜಿಯವರ ಗಮನವನ್ನು ಬೇರೆಡೆಗೆ ಸೆಳೆಯಲು ತಕ್ಷಣ ವಿಷಯ ಬದಲಾಯಿಸಿ, “ಕೋಣೆಯಲ್ಲಿ ಬಹಳ ಸೆಖೆ; ಸ್ವಲ್ಪ ಹೊತ್ತು ತಾರಸಿಯ ಮೇಲೆ ಗಾಳಿಯಲ್ಲಿ ಅಡ್ಡಾಡೋಣ” ಎನ್ನುತ್ತ ಅವರನ್ನು ಹೊರಗೆ ಕರೆದುಕೊಂಡು ಹೋದರು. ಸ್ವಾಮೀಜಿ ಯವರ ಮನಸ್ಸು ಸಂಪೂರ್ಣ ಶಾಂತವಾಗಿ ಸಾಧಾರಣ ಸ್ಥಿತಿಗೆ ಬರಬೇಕಾದರೆ ರಾತ್ರಿ ಬಹಳ ಹೊತ್ತಾಗಿತ್ತು.

ನಿಜಕ್ಕೂ ಈ ಘಟನೆ ತುಂಬ ಮಹತ್ವಪೂರ್ಣವಾದದ್ದು. ಇಲ್ಲಿ ಸ್ವಾಮೀಜಿಯವರ ಭಕ್ತಿಯ ಆಳ ಎಷ್ಟು ಅಗಾಧವಾದದ್ದು ಎಂಬುದು ವ್ಯಕ್ತವಾಗುತ್ತದೆ. ಅಲ್ಲದೆ ಅವರು ತಮ್ಮ ಯಾವ ಅಮೂಲ್ಯ ವಸ್ತುವನ್ನು ಬಲಿಯಾಗಿಸಿ ಲೋಕಸೇವೆಯಲ್ಲಿ ನಿರತರಾಗಿದ್ದಾರೆ ಎಂಬುದು ತಿಳಿದು ಬರುತ್ತದೆ. ಅಂತಹ ಅಪಾರವಾದ ಭಕ್ತಿಪ್ರೇಮಾನಂದವನ್ನು ಅನುಭವಿಸುವ ಸಾಮರ್ಥ್ಯವಿದ್ದೂ ಇತರರ ಸೇವೆಗಾಗಿ ಅದನ್ನು ಬದಿಗೊತ್ತಬೇಕಾದರೆ ಅದೆಂತಹ ತ್ಯಾಗ! ಈ ಅಂಶವನ್ನು ಕ್ರಮೇಣ ಅವರ ಗುರುಭಾಯಿಗಳು ಗಮನಿಸಲು ಸಮರ್ಥರಾದರು.

ಸ್ವಾಮೀಜಿಯವರ ಕುರಿತಾಗಿ ಶರಚ್ಚಂದ್ರನ ಜೊತೆಯಲ್ಲಿ ಮಾತನಾಡುತ್ತ ಒಮ್ಮೆ ಸ್ವಾಮಿ ಯೋಗಾನಂದರು ಹೇಳಿದರು, “ಶ್ರೀರಾಮಕೃಷ್ಣರ ಮೇಲೆ ಅವನಲ್ಲಿರುವ ಭಕ್ತಿಯಲ್ಲಿ ನೂರರ ಲ್ಲೊಂದು ಭಾಗ ನನಗಿದ್ದಿದ್ದರೂ ನಾನು ಧನ್ಯನಾಗುತ್ತಿದ್ದೆ... ನರೇಂದ್ರನಲ್ಲಿ ವೇದ ಪುಷಿಗಳ ಜ್ಞಾನ, ಶಂಕರರ ವೈರಾಗ್ಯ, ಬುದ್ಧನ ಹೃದಯ ಹಾಗೂ ಶುಕದೇವನ ಬ್ರಹ್ಮಜ್ಞಾನ ಇವೆಲ್ಲ ಆವಿರ್ಭವಿಸಿರುವುದನ್ನು ಕಾಣಬಹುದು.”

ಆ ದಿನ ಬಲರಾಮ ಬಾಬುವಿನ ಮನೆಯಲ್ಲಿ ನಡೆದ ಘಟನೆ, ಆ ಸೋದರ ಸಂನ್ಯಾಸಿಗಳಿಗೆಲ್ಲ ಒಂದು ಮಹತ್ವದ ಸಂಗತಿಯನ್ನು ಅರಿವು ಮಾಡಿಸಿಕೊಟ್ಟಿತು. ಸ್ವಾಮಿ ಅದ್ಭುತಾನಂದರಂತೆಯೇ ಇತರ ಸಂನ್ಯಾಸಿಗಳೂ ಭಾವಿಸಿದ್ದರು–ಸ್ವಾಮೀಜಿ ಕೇವಲ ಜ್ಞಾನಿಗಳು ಎಂದು. ಆದರೆ ಅವರ ಒಳ ವ್ಯಕ್ತಿತ್ವವು ಪರಿಪೂರ್ಣ ಭಕ್ತಿಯಿಂದ ಕೂಡಿದ್ದು ಎಂಬುದು ಅಂದು ಸ್ಪಷ್ಟವಾಯಿತು. ಅಲ್ಲಿಯವರೆಗೂ, ಸ್ವಾಮೀಜಿ ಪ್ರಚಾರ ಮಾಡುತ್ತಿರುವ ಲೋಕಕಲ್ಯಾಣಕಾರ್ಯದ ಆದರ್ಶ ಸರಿಯೆ ತಪ್ಪೆ ಎಂಬ ಸಂಶಯದ ಮೋಡ ಈ ಸಂನ್ಯಾಸಿಗಳ ಮನಸ್ಸಿನಲ್ಲಿ ಕವಿದಿತ್ತು. ಆದರೆ ಆ ಘಟನೆಯಿಂದ ಅದು ಚದುರಿಹೋಯಿತು. ಅಂದಿನಿಂದಲೇ ಅವರೆಲ್ಲ ಸ್ವಾಮೀಜಿಯವರ ಕಾರ್ಯೋದ್ದೇಶಗಳನ್ನು ಟೀಕಿಸುವುದು-ಪ್ರತಿಭಟಿಸುವುದು ನಿಂತುಹೋಯಿತು. ಸ್ವಾಮೀಜಿಯವರ ಕಾರ್ಯದಲ್ಲಿ ಎಲ್ಲರೂ ಸಂಪೂರ್ಣ ಶ್ರದ್ಧೆಯಿಂದ ಪಾಲ್ಗೊಂಡರು.

