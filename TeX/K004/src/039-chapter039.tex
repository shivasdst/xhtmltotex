
\chapter{ಅದ್ಭುತ ಇಚ್ಛಾಶಕ್ತಿ}

\noindent

ಮೇ ೧೨ರಂದು ಸ್ವಾಮೀಜಿಯವರು ಪರಿವಾರ ಸಮೇತರಾಗಿ ಕಲ್ಕತ್ತಕ್ಕೆ ಹಿಂದಿರುಗಿದರು. ಇದು ಅವರ ಕಟ್ಟಕಡೆಯ ಸಾರ್ವಜನಿಕ ಪ್ರವಾಸ. ಇದಾದ ಮೇಲೆ ಅವರ ಆರೋಗ್ಯ ಮತ್ತಷ್ಟು ಕ್ಷೀಣವಾಗುತ್ತ ಬಂದಿತು. ಸೋದರ ಸಂನ್ಯಾಸಿಗಳೆಲ್ಲ ಅವರನ್ನು ಸಂಪೂರ್ಣ ವಿಶ್ರಮಿಸುವಂತೆ ಬಲಾತ್ಕರಿಸಿದರು. ಆರೋಗ್ಯ ಸುಧಾರಿಸುವವರೆಗೂ ಕೆಲಸದ ಯೋಚನೆಯನ್ನೇ ತೊಡೆದುಹಾಕು ವಂತೆ ಬೇಡಿಕೊಂಡರು. ಕಡೆಗೆ ಅವರೆಲ್ಲರ ಸಂತೋಷಕ್ಕಾಗಿ ಸ್ವಾಮೀಜಿ ತಮ್ಮ ಯೋಜನೆಗಳನ್ನು ರದ್ದುಪಡಿಸಬೇಕಾಯಿತು.

ಆದರೆ ಈ ಸೋದರ ಸಂನ್ಯಾಸಿಗಳ ಕೆಲಸ ಅಷ್ಟು ಸುಲಭದ್ದೇನೂ ಆಗಿರಲಿಲ್ಲ. ಸ್ವಲ್ಪ ಶಕ್ತಿ ಬಂದಂತೆ ತೋರಿದರೆ ಸಾಕು, ತಕ್ಷಣ ಸ್ವಾಮೀಜಿ ಯಾವುದಾದರೊಂದು ಕೆಲಸದಲ್ಲಿ ತೊಡಗು ತ್ತಿದ್ದರು. ಪ್ರವಾಸದಿಂದ ಹಿಂದಿರುಗಿದ ಮೇಲೆ ಕೆಲದಿನಗಳಂತೂ ಅವರ ಶರೀರವು ಎಷ್ಟು ಸೂಕ್ಷ್ಮವಾಗಿತ್ತೆಂದರೆ, ಸ್ವಲ್ಪ ಮುಟ್ಟಿದರೂ ಸಾಕು ಯಾತನೆಯಾಗುತ್ತಿತ್ತು. ಆದರೆ ಇಂತಹ ಸಮಯದಲ್ಲೂ ಅವರು ಶರಚ್ಚಂದ್ರನಿಗೆ ಹೇಳಿದರು, “ನಾನು ಕೆಲಸ ಮಾಡುತ್ತಲೇ ಇದ್ದು ಕೆಲಸದಲ್ಲೇ ಸಾಯಬೇಕೆಂದು ನಿರ್ಧರಿಸಿಬಿಟ್ಟಿದ್ದೇನೆ.” ಸ್ವಲ್ಪ ವಿಶ್ರಮಿಸಿಕೊಳ್ಳಬೇಕು ಎಂದು ಇಚ್ಛಿಸಿದರೂ ಅದು ಸಾಧ್ಯವಿಲ್ಲ. ಶ್ರೀರಾಮಕೃಷ್ಣರು ದೇಹತ್ಯಾಗ ಮಾಡುವ ಎರಡು-ಮೂರು ದಿನ ಮೊದಲು, ಅವರು ಯಾರನ್ನು ಕಾಳಿ ಎಂದು ಕರೆಯುತ್ತಿದ್ದರೋ ಅವಳು ನನ್ನ ದೇಹದೊಳಕ್ಕೆ ಪ್ರವೇಶಿಸಿಬಿಟ್ಟಳು. ಅವಳು ನನ್ನನ್ನು ಸುಮ್ಮನಿರಲು ಬಿಡುವುದೇ ಇಲ್ಲ!” ಬಳಿಕ, ಶ್ರೀರಾಮ ಕೃಷ್ಣರು ತಮಗೆ ಅವರ ಶಕ್ತಿಯನ್ನು ಧಾರೆಯೆರೆದ ಕಥೆಯನ್ನು ಸ್ವಾಮೀಜಿಯವರು ಶರಚ್ಚಂದ್ರನಿಗೆ ಅರುಹಿದರು.

ಸ್ವಾಮೀಜಿಯವರ ದೇಹಸ್ಥಿತಿಯೊಂದಿಗೆ ಅವರ ಮಾನಸಿಕಸ್ಥಿತಿಯೂ ಕೂಡ ಸಹಸಂನ್ಯಾಸಿಗಳ ಆತಂಕಕ್ಕೆ ಕಾರಣವಾಯಿತು. ಚಂದ್ರನಾಥ-ಕಾಮಾಖ್ಯಗಳ ತೀರ್ಥಯಾತ್ರೆಯಿಂದ ಮರಳಿದ ಮೇಲೆ, ಅವರ ಬಳಿ ಯಾವಾಗಲೂ ಒಂದು ದೈವೀ ಸಾನ್ನಿಧ್ಯವಿರುವಂತೆ ಭಾಸವಾಗುತ್ತಿತ್ತು. ಅವರ ಮನಸ್ಸನ್ನು ಹಗುರ ವಿಷಯಗಳತ್ತ ಹರಿಯುವಂತೆ ಮಾಡಲು ಇತರರು ಎಷ್ಟೇ ಪ್ರಯತ್ನಿಸಿದರೂ ಅದು ಬೇರಾವುದೋ ವಸ್ತುವಿನ ಮೇಲೆ ಬಲು ಬೇಗ ನೆಲೆಗೊಂಡುಬಿಡುತ್ತಿತ್ತು. ನೀರನ್ನೋ ಹುಕ್ಕಾವನ್ನೋ ತರಲು ಹೇಳಿದವರು, ಅದು ಬರುವಷ್ಟರಲ್ಲಿ ಆ ವಿಷಯವನ್ನೇ ಮರೆತುಬಿಡು ತ್ತಿದ್ದರು. ಅಥವಾ, ಆಗ ಇತರರು ಹೇಳಿದ್ದು ಕೇಳಿಸುತ್ತಲೇ ಇರಲಿಲ್ಲ.

ಸ್ವಾಮೀಜಿಯವರು ತಾವೇ ಕೆಲಸಗಳಲ್ಲಿ ತೊಡಗದಿದ್ದಾಗಲೂ ಮಠದ ಕಾರ್ಯಕಲಾಪಗಳ ಪ್ರತಿಯೊಂದು ವಿವರವನ್ನೂ ತಿಳಿದುಕೊಳ್ಳುತ್ತಿದ್ದರು. ಸೇವಕರನ್ನು ಅವರು ತಮ್ಮ ಸ್ವಂತದವ ರಂತೆ ಕಾಣುತ್ತಿದ್ದರು. ಈ ಸೇವಕರು ಸ್ವಾಮೀಜಿಯವರಿಗೆ ಅತಿ ಸಣ್ಣ ಸೇವೆ ಮಾಡಲೂ ನಾ ಮುಂದು ತಾ ಮುಂದೆಂದು ಧಾವಿಸುತ್ತಿದ್ದರು. ಕೆಲವೊಮ್ಮೆ ಸ್ವಾಮೀಜಿ ಪರಿವ್ರಾಜಕ ಸಂನ್ಯಾಸಿ ಗಳಂತಹ ನಿಲುವಂಗಿಯನ್ನು ಧರಿಸಿ, ಕೈಯಲ್ಲಿ ಊರುಗೋಲು ಹಿಡಿದು ಆಲೋಚನಾಮಗ್ನರಾಗಿ, ಮುಖ್ಯ ರಸ್ತೆಗೆ ಸೇರುವ ಕಾಲುದಾರಿಯಲ್ಲಿ ಒಬ್ಬರೇ ನಡೆದಾಡುತ್ತಿದ್ದರು. ಇಲ್ಲವೆ ಕೇವಲ ಕೌಪೀನವನ್ನು ಧರಿಸಿ ಆಶ್ರಮದೊಳಗೆ ಅಡ್ಡಾಡುತ್ತಿದ್ದರು. ಕೆಲವೊಮ್ಮೆ ಗಂಗೆಯ ದಡದಲ್ಲೋ, ತಂಪಾದ ಮರದ ನೆರಳಲ್ಲೋ ಧ್ಯಾನಕ್ಕೆ ಕುಳಿತುಬಿಡುತ್ತಿದ್ದರು. ಮತ್ತೆ ಕೆಲವು ಸಲ ತಮ್ಮ ಕೋಣೆಯಲ್ಲಿ ಕುಳಿತು ಪುಸ್ತಕವೊಂದರಲ್ಲಿ ಮುಳುಗಿಹೋಗುತ್ತಿದ್ದರು. ಆಗಾಗ ಅವರು ತಮ್ಮ ಹಿಂದಿನ ಜ್ವಲಂತ ಭಾವಗಳಿಗೆ ಮರಳಿ, ಇಡೀ ಆಶ್ರಮದಲ್ಲೇ ಆಧ್ಯಾತ್ಮಿಕ ಪ್ರಜ್ಞೆಯನ್ನು ಪ್ರಜ್ವಲನಗೊಳಿಸುತ್ತಿದ್ದರು. ವಿಶ್ರಮಿಸಿಕೊಳ್ಳುವಂತೆ ಅವರ ಗುರುಭಾಯಿಗಳು ಒತ್ತಾಯಿಸಿದಾಗ ಅವರು ಗುರುಭಾಯಿಗಳ ಮಾತನ್ನು ಕೇಳುತ್ತಿದ್ದುದು ಸ್ವಲ್ಪ ಹೊತ್ತು ಮಾತ್ರ. ಮತ್ತೆ ಮಾತಿ ಗಾರಂಭ! ಜಗತ್ತನ್ನೇ ಅಲುಗಾಡಿಸಿದ ಸ್ವಾಮೀಜಿಯವರನ್ನು ತಡೆದು ನಿಲ್ಲಿಸುವುದು ಸುಲಭ ವೇನು? ಅದು ಸಮುದ್ರದ ಅಲೆಗಳನ್ನು ತಡೆಯಹೊರಟಂತೆ. ಅಲ್ಲದೆ ಅವರಿಗೆ ತಮ್ಮ ಆರೋಗ್ಯ ವನ್ನೂ ಜೀವವನ್ನೂ ಕಾಪಾಡಿಕೊಳ್ಳಬೇಕೆಂಬ ಇಚ್ಛೆಯೇನೂ ಇರಲಿಲ್ಲ. ಅವರು ಮೊದಲಿ ನಿಂದಲೂ ಹೇಳುತ್ತಿದ್ದ, “ಸದ್ಯ, ಈ ಜೀವನ ಶಾಶ್ವತವಲ್ಲವಲ್ಲ! ಅದಕ್ಕೆ ನಾವು ಆಭಾರಿಯಾಗಿರ ಬೇಕು!” ಎಂಬ ಮಾತು ಅವರ ಶಿಷ್ಯರಿಗೆ ಈಗ ಮತ್ತೆ ಮತ್ತೆ ನೆನಪಿಗೆ ಬರುತ್ತಿತ್ತು.

ಸ್ವಾಮೀಜಿಯವರು ತಮ್ಮ ಕಾಯಿಲೆಗಳಿಗೆ ಬಂಗಾಳದ ಕವಿರಾಜರಿಂದ (ಅಳಲೆಕಾಯಿ ಪಂಡಿತ ರಿಂದ) ಚಿಕಿತ್ಸೆ ಪಡೆಯುವ ಬದಲು ಇಂಗ್ಲಿಷ್ ಔಷಧಿಯನ್ನೇ ಹೆಚ್ಚು ಇಷ್ಟಪಟ್ಟರು. ಈ ‘ಪಂಡಿತ’ರ ಔಷಧಿ ತೆಗೆದುಕೊಂಡು, ಇಂದು ವಾಸಿಯಾದೀತು ನಾಳೆ ವಾಸಿಯಾದೀತು ಎಂದು ಕುಳಿತಿರುವುದಕ್ಕಿಂತ ಇಂಗ್ಲಿಷ್ ಡಾಕ್ಟರರ ಚಿಕಿತ್ಸೆಯಲ್ಲಿ ಬೇಗ ಸಾಯುವುದೇ ಮೇಲು ಎನ್ನುವುದು ಅವರ ಅಂಬೋಣ! ಆದರೂ ನಿರಂಜನಾನಂದರು ಹಾಗೂ ಇತರ ಸಂನ್ಯಾಸಿಗಳು ಕಲ್ಕತ್ತದ ಪ್ರಸಿದ್ದ ಕವಿರಾಜರೊಬ್ಬರಿಂದ ಚಿಕಿತ್ಸೆ ಪಡೆಯುವಂತೆ ಅವರನ್ನು ಒಪ್ಪಿಸಿದರು. ಆ ವೈದ್ಯರ ಚಿಕಿತ್ಸೆಯೋ ಅತ್ಯಂತ ಕಠಿಣ–ಒಂದು ಹನಿ ನೀರನ್ನೂ ಕುಡಿಯುವಂತಿಲ್ಲ. ಉಪ್ಪನ್ನೂ ಮುಟ್ಟು ವಂತಿಲ್ಲ. ನೀರಿನ ಬದಲಿಗೆ ಹಾಲು ಕುಡಿಯಬಹುದು. ಒಮ್ಮೆ ಇದಕ್ಕೆ ಒಪ್ಪಿಕೊಂಡ ಮೇಲೆ ಮಾತ್ರ ಈ ಎಲ್ಲ ನಿಯಮಗಳನ್ನೂ ಸ್ವಾಮೀಜಿ ಕಟ್ಟುನಿಟ್ಟಾಗಿ ಪಾಲಿಸಿದರು. ಅವರು ಇಂತಹ ಪಥ್ಯವನ್ನು ಒಪ್ಪಿಕೊಂಡದ್ದಕ್ಕೆ ಹಲವಾರು ಕಾರಣಗಳಿದ್ದುವು. ಮೊದಲನೆಯದಾಗಿ, ತಮ್ಮ ಇಚ್ಛಾ ಶಕ್ತಿಯು ತಮ್ಮ ದೇಹದ ಮೇಲೆ ಹೇಗೆ ಪರಿಣಾಮ ಬೀರಬಲ್ಲುದೆಂದು ನೋಡಿಕೊಳ್ಳುವ ಕುತೂಹಲ ಅವರಿಗೆ. ಎರಡನೆಯದಾಗಿ, ತಮ್ಮ ಸಂನ್ಯಾಸೀ ಬಂಧುಗಳ ಅಪೇಕ್ಷೆಯಂತೆ ನಡೆದು ಕೊಂಡು ಸಂತೋಷವನ್ನುಂಟುಮಾಡುವ ಇಚ್ಛೆ. ಕೊನೆಯದಾಗಿ, ತಮ್ಮ ಮುಂದೆ ಮಹತ್ತರವಾಗಿ ನಿಂತಿದ್ದ ಕಾರ್ಯಬಾಹುಳ್ಯಗಳ ಕಾರಣಕ್ಕಾಗಿ. ಆದರೆ ತಮ್ಮ ಕಾಯಿಲೆ ಗುಣವಾಗುವ ವಿಷಯದಲ್ಲಿ ಅವರಿಗೇ ನಂಬಿಕೆಯಿರಲಿಲ್ಲ. ಹಲವಾರು ದಿನಗಳ ಮೇಲೆ ಅವರು ಶರಚ್ಚಂದ್ರನ ಹತ್ತಿರ ಹೇಳಿದರು, “ನಾನು ನನ್ನ ಗುರುಭಾಯಿಗಳ ಆಜ್ಞೆಗೆ ವಿಧೇಯನಾಗಿರುವುದನ್ನು ನೋಡಿದೆಯಾ? ಅವರ ಕೋರಿಕೆಯನ್ನು ತಳ್ಳಿಹಾಕಲು ನನ್ನಿಂದ ಸಾಧ್ಯವಿಲ್ಲ. ಏಕೆಂದರೆ ಅವರೆಲ್ಲ ನನ್ನನ್ನು ಅದೆಷ್ಟು ಪ್ರೀತಿಸುತ್ತಾರೆ!”

ಆಗ ಶರಚ್ಚಂದ್ರ ಕೇಳಿದ–“ಸ್ವಾಮೀಜಿ, ನೀವು ಗಂಟೆಗೊಮ್ಮೆ ನೀರು ಕುಡಿಯುತ್ತಿದ್ದವರು. ಅದರಲ್ಲೂ ಈಗ ಸೆಕೆಕಾಲ, ಹೀಗಿದ್ದರೂ ಅದು ಹೇಗೆ ನೀವು ನೀರನ್ನೇ ಕುಡಿಯದೆ ಇದ್ದು ಬಿಟ್ಟಿರಿ?”

ಸ್ವಾಮೀಜಿ: “ಈ ಚಿಕಿತ್ಸೆಗೆ ಒಪ್ಪಿಕೊಂಡಾಗಲೇ ನಾನು ನೀರು ಕುಡಿಯುವುದಿಲ್ಲ ಎಂದು ಶಪಥ ಮಾಡಿಬಿಟ್ಟೆ. ಈಗ ನನ್ನ ಗಂಟಲೊಳಗೆ ನೀರು ಇಳಿಯುವುದೇ ಇಲ್ಲ. ಇಪ್ಪತ್ತೊಂದು ದಿನಗಳಿಂದಲೂ ನಾನು ನೀರು ಕುಡಿದೇ ಇಲ್ಲ. ಈಗ, ನಾನು ಬಾಯಿಮುಕ್ಕಳಿಸಲು ಹೊರಟರೆ, ಗಂಟಲೊಳಗೆ ಒಂದು ತೊಟ್ಟೂ ನೀರಿಳಿಯದಂತೆ ಗಂಟಲಿನ ಮಾಂಸಖಂಡಗಳು ತಾವಾಗಿಯೇ ಮುದುಡಿಕೊಳ್ಳುತ್ತವೆ. ಎಷ್ಟಾದರೂ ಈ ಶರೀರವು ಮನಸ್ಸಿನ ಅಡಿಯಾಳಲ್ಲವೆ? ಮನಸ್ಸು ಆಜ್ಞೆ ಮಾಡಿದ್ದನ್ನು ಶರೀರ ವಿಧೇಯವಾಗಿ ಪಾಲಿಸಲೇಬೇಕು!”

ಇದನ್ನು ಕೇಳಿ ಶರಚ್ಚಂದ್ರ ದಂಗಾಗಿ ಕುಳಿತ. ಕೆಲದಿನಗಳ ಚಿಕಿತ್ಸೆ ಮುಗಿದ ಮೇಲೆ ಸ್ವಾಮೀಜಿ ತಮ್ಮ ಗುರುಭಾಯಿಗಳಿಗೆ ಹೇಳಿದರು, “ನನಗೀಗ ನೀರಿನ ಯೋಚನೆಯೇ ಬರುವುದಿಲ್ಲ. ನನಗದು ಬೇಕೆಂದು ಅನ್ನಿಸುವುದೇ ಇಲ್ಲ!”

ಈ ಚಿಕಿತ್ಸೆಯನ್ನು ಪ್ರಾರಂಭಿಸಿದ ಮೇಲೆ ಸುಧಾರಣೆಯೇನೂ ಕಂಡುಬರಲಿಲ್ಲವಾದರೂ ಅವರು ಪಥ್ಯವನ್ನು ಮಾತ್ರ ಬಿಡಲಿಲ್ಲ.

ಒಟ್ಟಿನಲ್ಲಿ ಯಾವ ಚಿಕಿತ್ಸೆಯಿಂದಲೂ ಏನೂ ಪ್ರಯೋಜನ ಕಂಡುಬರಲಿಲ್ಲ. ಆದರೆ ಅವರ ಶರೀರ ಮಾತ್ರ ಮೇಲ್ನೋಟಕ್ಕೆ ಏನೋ ಸ್ವಲ್ಪ ತುಂಬಿಕೊಂಡಂತೆ ಕಾಣುತ್ತಿತ್ತು. ಕೆಲವು ವಾರ ಹೀಗೆ ಉತ್ತಮವಾದಂತೆ ಕಂಡುಬಂದರೂ ಇನ್ನು ಕೆಲವು ವಾರ ಅವರ ಆರೋಗ್ಯ ಮತ್ತೆ ಇಳಿಮುಖವಾಗುತ್ತಿತ್ತು. ಡಯಾಬಿಟಿಸಿನ ಸ್ವಭಾವವೇ ಹೀಗೆ. ಈ ನಡುವೆ ಅವರ ದೃಷ್ಟಿಶಕ್ತಿಯೂ ಹೆಚ್ಚುಕಡಿಮೆಯಾದಂತೆ ತೋರಿತು. ‘ಸಕ್ಕರೆಕಾಯಿಲೆ’ಯಲ್ಲಿ ಇದೊಂದು ಅಪಾಯಕಾರಿ ಬೆಳ ವಣಿಗೆಯೇ ಸರಿ. ಅವರ ಒಂದು ಕಣ್ಣಿನ ಸೂಕ್ಷ್ಮ ನರತಂತುವೊಂದು ಘಾಸಿಯಾಗಿದ್ದಂತಿತ್ತು. ಇದರಿಂದಾಗಿ ಆ ಕಣ್ಣಿನ ದೃಷ್ಟಿಶಕ್ತಿ ಸ್ವಲ್ಪ ಮಂದವಾಯಿತು. ಆದರೆ ಸ್ವಾಮೀಜಿ ಇದನ್ನೊಂದು ತಮಾಷೆಯಂತೆ ಪರಿಗಣಿಸಿದರು. ಒಂದು ದಿನ ಸ್ವಾಮಿ ಶಿವಾನಂದರೊಡನೆ ಹರಟುತ್ತ, “ಏನಂ ತೀಯ ಮಹಾಪುರುಷ್? ನನ್ನನ್ನು ರಾಕ್ಷಸರ ಗುರು ಶುಕ್ರಾಚಾರ್ಯ ಎನ್ನುತ್ತೀಯಾ?” ಎಂದು ಹೇಳಿ ನಕ್ಕರು. ಶುಕ್ರಾಚಾರ್ಯನು ಒಕ್ಕಣ್ಣನೆಂಬುದು ಒಂದು ಅಂಶವಾದರೆ, ತಾವು ಅವನಂತೆ ರಾಕ್ಷಸರ ಗುರು–ಅರ್ಥಾತ್, ‘ಮ್ಲೇಚ್ಛ’ರಾದ ಪಾಶ್ಚಾತ್ಯರ ಗುರು ಎನ್ನುವುದು ಇನ್ನೊಂದು ಅಂಶ.

ಮಠದಲ್ಲಿನ ತಮ್ಮ ದಿನಚರಿಯ ಬಗ್ಗೆ ಸ್ವಾಮೀಜಿಯವರು ಜುಲೈ ೬ರಂದು ಕ್ರಿಸ್ಟೀನಳಿಗೆ ಬರೆದರು:

“ನನಗೆ ಬರುವುದೆಲ್ಲವೂ ಒಟ್ಟೊಟ್ಟಾಗಿಯೇ. ಇವತ್ತು ನಾನು ಬರೆಯುವ ಉತ್ಸಾಹದಲ್ಲಿ ದ್ದೇನೆ. ಆದ್ದರಿಂದ ಈಗ ನಾನು ಮಾಡಬೇಕಾದ ಮೊದಲ ಕೆಲಸವೆಂದರೆ ನಿನಗೆ ಕೆಲವು ಸಾಲುಗಳನ್ನು ಬರೆಯುವುದು... ಮಿಸ್ ಮೆಕ್​ಲಾಡ್ ಜಪಾನಿನಲ್ಲಿದ್ದಾಳೆ. ಅವಳಿಗದು ತುಂಬ ಮೆಚ್ಚಿಗೆಯಾಗಿದೆ. ನಾನೂ ಅವಳೊಂದಿಗೆ ಹೊರಟುಬಿಡುತ್ತಿದ್ದೆ. ಆದರೇನು ಮಾಡುವುದು. ಈ ಅನಾರೋಗ್ಯ ದೂರ ಸಮುದ್ರಪ್ರಯಾಣದ ತೊಂದರೆ–ಇವುಗಳನ್ನೆಲ್ಲ ನೋಡಿ ಸುಮ್ಮನಾದೆ.

“ನಾನು ಸಾಧ್ಯವಾದಷ್ಟೂ ತೆಪ್ಪಗಿರಲು ಪ್ರಯತ್ನಿಸುತ್ತಿದ್ದೇನೆ. ಇಲ್ಲಿ ನನ್ನ ಬಳಿ ಕೆಲವು ಮೇಕೆ- ಆಡುಗಳು, ಒಂದು ಜಿಂಕೆ ಮತ್ತು ಕೆಲವು ಹಸುಗಳಿವೆ. ಜೊತೆಗೆ ಹೂದೋಟ, ಮೀನಿನ ಕೆರೆಗಳು, ತರಕಾರಿ ತೋಟಗಳು ಎಲ್ಲ ಇವೆ.

“ನಾನು ಬಹಳ ಮುಂಚೆ ಏಳುತ್ತೇನೆ. ಬಳಿಕ ಮೇಕೆಗಳಿಗೆ ಹುಲ್ಲು ಹಾಕಿ ಹಾಲು ಕರೆಯುತ್ತೇನೆ. ಒಂದು ನಾಯಿಮರಿ, ಒಂದು ಸುಂದರ ಕರೀ ಆಡಿನ ಮರಿ–ಇವು ನನ್ನ ಮೆಚ್ಚಿನ ಸಂಗಾತಿಗಳು. ಸ್ವಲ್ಪ ಹೊತ್ತು ಡಂಬೆಲ್ಲುಗಳಿಂದ ವ್ಯಾಯಾಮ ಮಾಡಿ ಬಳಿಕ ಬಿಸಿಲೇರಿದಾಗ ಹತ್ತು ಗಂಟೆಯ ವರೆಗೂ ಹಾಸಿಗೆಯ ಮೇಲೆ ಉರುಳಿಕೊಳ್ಳುತ್ತೇನೆ... ”

ಇನ್ನು ಕೆಲವೊಮ್ಮೆ ಅವರು ತಮ್ಮ ಗುರುಭಾಯಿಗಳೊಂದಿಗೆ ತಮಾಷೆ ಮಾಡುತ್ತ ಅಥವಾ ಅವರನ್ನೇ ಲೇವಡಿ ಮಾಡುತ್ತ ನಕ್ಕುನಗಿಸುವುದುಂಟು. ಒಂದು ದಿನ ಅದ್ವೈತಾನಂದರೊಂದಿಗೆ ಮಾತನಾಡುತ್ತಿದ್ದವರು ಇದ್ದಕ್ಕಿದ್ದಂತೆ ಹೇಳಿದರು, “ಗೊತ್ತಾಯ್ತೆ ಗೋಪಾಲ್ ದಾ, ಈ ವರ್ಷ ಗುಡ್ ಫ್ರೈಡೇ ಒಂದು ಸೋಮವಾರ ಬೀಳುತ್ತಂತೆ.” ಸರಳ ಸ್ವಭಾವದ ಅದ್ವೈತಾನಂದರು, ಇದ್ದರೂ ಇರಬಹುದೇನೋ ಎಂದುಕೊಂಡು “ಓ ಹಾಗೇನು?” ಎಂದರು. ತಕ್ಷಣ ಸುತ್ತಲಿ ದ್ದವರೆಲ್ಲ ಎಷ್ಟು ಗಟ್ಟಿಯಾಗಿ ನಕ್ಕರೆಂದರೆ ಸ್ವಾಮೀಜಿಯವರಿಗೇ ಸ್ವಲ್ಪ ಕಸಿವಿಸಿಯಾಯಿತಂತೆ.

ಸ್ವಾಮೀಜಿ ಯಾವಾಗಲೂ ಬ್ರಾಹ್ಮೀ ಮುಹೂರ್ತದಲ್ಲೇ ಏಳುವ ಅಭ್ಯಾಸವಿದ್ದವರು. ಆಶ್ರಮ ವಾಸಿಗಳೆಲ್ಲರೂ ಉಷಃಕಾಲದಲ್ಲಿ ಏಳಲೇಬೇಕೆಂಬ ನಿಯಮವನ್ನು ಅವರು ಜಾರಿಗೆ ತಂದಿದ್ದರು. ಮಠದ ನಿಯಮಗಳನ್ನು ಪ್ರತಿಯೊಬ್ಬರೂ ಕಟ್ಟುನಿಟ್ಟಾಗಿ ಪಾಲಿಸಲೇಬೇಕೆಂಬುದು ಅವರ ಅಪೇಕ್ಷೆ. ಯಾರಾದರೂ ಅದಕ್ಕೆ ತಪ್ಪಿದರೆ ತುಂಬ ಅಸಮಾಧಾನ. ಆಧ್ಯಾತ್ಮಿಕ ಸಾಧನೆಗಳನ್ನು ಎಲ್ಲರೂ ನಿಯತವಾಗಿ ಮಾಡಿಕೊಂಡು ಬರುವಂತೆ ನೋಡಿಕೊಳ್ಳುತ್ತಿದ್ದರು. ಆದರೆ ಯಾರೂ ಅತಿಯಾಗಿ ಮಾಡದಂತೆ ತಡೆಯುತ್ತಿದ್ದರು. ಏಕೆಂದರೆ ಎಷ್ಟೋ ಶರೀರಗಳಿಗೆ ಆಧ್ಯಾತ್ಮಿಕ ಸಾಧನೆಯ ರಭಸವನ್ನು ತಾಳಿಕೊಳ್ಳುವ ಶಕ್ತಿಯಿರುವುದಿಲ್ಲ. ಸಾಧನೆ ಮಾಡಬೇಕು. ಅಪಾಯ ಸಂಭವಿಸದಂತೆ ನೋಡಿಕೊಳ್ಳಬೇಕು–ಇದು ಸ್ವಾಮೀಜಿಯವರ ಎಚ್ಚರ.

ಮಠದ ಕಟ್ಟಡಗಳೆಲ್ಲ ಚೊಕ್ಕಟವಾಗಿರಬೇಕೆಂಬುದು ಅವರ ಇಚ್ಛೆ. ಕೆಲಸಗಾರರು ರಜಾ ಹಾಕಿದ್ದರಿಂದಲೋ ಇನ್ನಿತರ ಕಾರಣಗಳಿಂದಲೋ ಯಾವುದಾದರೂ ಕೋಣೆಯಲ್ಲಿ ಧೂಳು ತುಂಬಿದ್ದುದು ಕಂಡುಬಂದರೆ ತಾವೇ ಕಸಪೊರಕೆಯನ್ನು ಕೈಗೆತ್ತಿಕೊಳ್ಳುತ್ತಿದ್ದರು. ಇತರರು ಅವರ ಕೈಯಿಂದ ಅದನ್ನು ಕಸಿದುಕೊಳ್ಳಲು ಎಷ್ಟೇ ಪ್ರಯತ್ನಿಸಿದರೂ ಕೊಡದೆ ತಾವೇ ಶುಚಿಗೊಳಿಸು ತ್ತಿದ್ದರು. ಕೆಲವೊಮ್ಮೆ ಅವರು ಆಶ್ರಮವಾಸಿಗಳ ಕೋಣೆಗಳಿಗೆ ಹೋಗಿ ಹಾಸಿಗೆಗಳನ್ನು ಶುಚಿಯಾ ಗಿಟ್ಟುಕೊಂಡಿದ್ದಾರೆಯೇ, ಚೆನ್ನಾಗಿ ಬಿಸಿಲಿಗೆ ಹಾಕಿದ್ದಾರೆಯೇ ಎಂಬುದನ್ನು ಪರೀಕ್ಷಿಸಿ ನೋಡು ತ್ತಿದ್ದರು. ಆಶ್ರಮವಾಸಿಗಳ ಕಾಯಿಲೆಗಳಿಗೆ ಮುಖ್ಯ ಕಾರಣ ಅವರು ಕುಡಿಯುವ ನೀರು. ಏಕೆಂದರೆ ಅಲ್ಲಿ ಹರಿಯುವ ಗಂಗೆಯು ಬೇಸಿಗೆಯಲ್ಲಿ ಜನರ ಅವಿವೇಕದಿಂದ ಕೊಳಕಾಗಿದ್ದರೆ ಮಳೆಗಾಲ ದಲ್ಲಿ ಕೆಸರಿನಿಂದಾಗಿ ಕೊಳಕಾಗಿರುತ್ತದೆ. ಈ ಸಮಸ್ಯೆಯನ್ನು ಹೋಗಲಾಡಿಸಲು ಅವರು ಗಂಗೆಯ ಬದಿಯಲ್ಲೇ ಒಂದು ಬಾವಿ ತೋಡಿಸಿದರು.

ಮಠದ ತೋಟದ ಮೇಲೆ ಸ್ವಾಮಿ ಬ್ರಹ್ಮಾನಂದರಿಗೆ ವಿಶೇಷ ಮಮತೆ. ಅವರ ಹೂವು- ತರಕಾರಿಗಳ ತೋಟವಿದ್ದದ್ದು ಹುಲ್ಲುಹಾಸಿನ ಪಕ್ಕದಲ್ಲಿ. ಈ ಹಾಸಿನ ಮೇಲೆ ಸ್ವಾಮೀಜಿಯವರ ಮುದ್ದಿನ ಪ್ರಾಣಿಗಳು ಓಡಾಡಿಕೊಂಡಿದ್ದುವು. ಇವೆರಡರ ಮಧ್ಯೆ ಬೇಲಿಯೂ ಇತ್ತು. ಆದರೆ ಒಂದೊಂದು ಸಲ ‘ಮಾತೃ’ ಎಂಬ ಮೇಕೆಮರಿಯೋ ಜಿಂಕೆಯೋ ಬೇಲಿ ಹಾರಿ, ಬ್ರಹ್ಮಾನಂದರು ಎಚ್ಚರಿಕೆಯಿಂದ ಬೆಳೆಸುತ್ತಿದ್ದ ಪುಟ್ಟ ಸಸಿಯನ್ನು ಮೇಯ್ದಬಿಡುತ್ತಿತ್ತು. ಅಥವಾ ಆ ತೋಟದ ಸೌತೆಯ ಬಳ್ಳಿ ಹುಲ್ಲುಹಾಸಿನೊಳಗೆ ನುಸುಳಿ, ಈ ಮುದ್ದುಪ್ರಾಣಿಗಳಿಗೆ ಓಡಾಡಲು ‘ತೊಂದರೆ’ ಯಾಗುತ್ತಿತ್ತು. ಇಂತಹ ಅತಿಕ್ರಮ ಪ್ರವೇಶಗಳಾದಾಗ ಇಬ್ಬರು ಮಹಾಸ್ವಾಮಿಗಳ ನಡುವೆ ಯುದ್ಧ ಪ್ರಾರಂಭವಾಗುತ್ತಿತ್ತು. ಇಬ್ಬರೂ ತಮ್ಮತಮ್ಮವರನ್ನು ನಿರ್ದೋಷಿಗಳೆಂದು ಸಮರ್ಥಿಸಿ ಕೊಳ್ಳುತ್ತಿದ್ದರು. ಆದರೆ ನ್ಯಾಯಶಾಸ್ತ್ರ-ತರ್ಕಶಾಸ್ತ್ರಗಳಲ್ಲಿ ತಮ್ಮ ಸೋದರನಷ್ಟು ನುರಿತವರಲ್ಲದ ಬ್ರಹ್ಮಾನಂದರಿಗೆ ಸ್ವಾಮೀಜಿಯವರೊಂದಿಗೆ ವಾದ ಹೂಡಿ ಗೆಲ್ಲುವುದೆಂದರೆ ಕಷ್ಟವೇ. ಇಬ್ಬರೂ ಸಿಟ್ಟಿನಿಂದ ಕೂಗಾಡುತ್ತಿದ್ದರು. “ನೋಡಿಕೊ, ಇನ್ನೊಂದು ಸಲ ನಿನ್ನ ಜಿಂಕೆಯೇನಾದರೂ ನನ್ನ ತೋಟದೊಳಕ್ಕೆ ನುಗ್ಗಿದರೆ ಅದರ ಕಾಲು ನೆಟ್ಟಗಿರಲಿಕ್ಕಿಲ್ಲ” ಎಂದು ಬ್ರಹ್ಮಾನಂದರೆಂದರೆ, “ನಿನ್ನ ಸೌತೆಯ ಬಳ್ಳಿ ನನ್ನ ಹುಲ್ಲಿನ ಮೇಲಕ್ಕೆ ಬರಲಿ, ಅದನ್ನು ಬೇರುಸಹಿತ ಕಿತ್ತು ಹಾಕುತ್ತೇನೆಯೋ ಇಲ್ಲವೋ ನೋಡು!” ಎಂದು ಸ್ವಾಮೀಜಿ ಆರ್ಭಟಿಸುತ್ತಿದ್ದರು. ಇವರಿಬ್ಬರಿಗೆ ಜಗಳ ಹತ್ತಿ ಕೊಂಡಿತೆಂದರೆ ಉಳಿದವರಿಗೆಲ್ಲ ಮೋಜು. ಸ್ವಲ್ಪ ದೂರದಲ್ಲಿ ನಿಂತು, ಇಬ್ಬರು ಸ್ವಾಮಿಗಳು ಜಗಳವಾಡುವ ಪರಿಯನ್ನು ಕಂಡು ನಗುತ್ತಿದ್ದರು. ಆದರೆ ಈ ಜಗಳ ಬಹಳ ಹೊತ್ತು ನಡೆಯುತ್ತಿರಲಿಲ್ಲ. ಇತರರು ತಮ್ಮ ಜಗಳವನ್ನು ನೋಡುತ್ತ ಮನರಂಜನೆ ಪಡೆಯುತ್ತಿರುವು ದನ್ನು ಕಂಡ ತಕ್ಷಣ ಇಬ್ಬರೂ ತಮ್ಮ ಕೂಗಾಟವನ್ನು ನಿಲ್ಲಿಸಿ ಗಟ್ಟಿಯಾಗಿ ನಕ್ಕು ತಮ್ಮ ಕೋಣೆ ಗಳಿಗೆ ಹೊರಟುಬಿಡುತ್ತಿದ್ದರು. ಅಂತೂ ಅವರ ಜಗಳ ಎಂದಿಗೂ ಒಂದು ಇತ್ಯರ್ಥಕ್ಕೆ ಬಂದದ್ದೇ ಇಲ್ಲ. ಮತ್ತೆ ಯಾವಾಗಲಾದರೂ ಜಗಳದ ‘ಮೂಡ್​’ ಬಂದರೆ ಅದು ಅಂದು ನಿಂತಲ್ಲಿಂದ ಪುನರಾರಂಭವಾಗುತ್ತಿತ್ತು.

೧೯ಂ೧ರ ಬೇಸಿಗೆಯಲ್ಲಿ ಸ್ವಾಮೀಜಿಯವರಿಗೆ ಸಂಪೂರ್ಣ ವಿಶ್ರಾಂತಿ. ಆದರೆ ಅವರ ಮನದಲ್ಲಿ ಯೋಜನೆಗಳು ರೂಪುಗೊಳ್ಳುತ್ತಲೇ ಇದ್ದುವು. ದೂರಸ್ಥಳಗಳಲ್ಲಿನ ತಮ್ಮ ಸೋದರ ಸಂನ್ಯಾಸಿಗಳೊಂದಿಗೆ, ಶಿಷ್ಯರೊಂದಿಗೆ ಹಾಗೂ ವಿಶ್ವಾಸಿಗರೊಂದಿಗೆ ಪತ್ರ ವ್ಯವಹಾರವನ್ನು ಇಟ್ಟುಕೊಂಡೇ ಇದ್ದರು. ಜೂನ್ ಮೂರರಂದು ಅವರು ಮದ್ರಾಸಿನಲ್ಲಿದ್ದ ರಾಮಕೃಷ್ಣಾನಂದರಿಗೆ ಬರೆದರು–“ಈಚೆಗೆ ನನ್ನ ಆರೋಗ್ಯ ಸ್ವಲ್ಪ ಸುಧಾರಿಸುತ್ತಿದೆ. ಮದ್ರಾಸಿನಲ್ಲಿ ಮಳೆ ಪ್ರಾರಂಭ ವಾಯಿತೆ? ದಕ್ಷಿಣದಲ್ಲಿ ಸ್ವಲ್ಪ ಮಳೆ ಪ್ರಾರಂಭವಾದರೆ ಬಿಸಿಲಿನ ಧಗೆ ಇಳಿಯುತ್ತದೆ. ಆಗ ನಾನು ಮುಂಬಯಿ, ಪೂನಾ ಮಾರ್ಗವಾಗಿ ಮದ್ರಾಸಿಗೆ ಬರಬಹುದು.... ನೀನು ನಿನ್ನ ಕೆಲಸವನ್ನು ಸ್ವಲ್ಪ ದಿನ ನಿಲ್ಲಿಸಿ ಇಲ್ಲಿಗೆ ಬಂದರೆ ನಾನೂ ನೀನೂ ಗುಜರಾತ್, ಮುಂಬಯಿ, ಪೂನಾ, ಹೈದರಾಬಾದ್, ಮೈಸೂರು ಹಾಗೂ ಮದ್ರಾಸ್ ಮೂಲಕ ಒಂದು ಭಾರೀ ಯಾತ್ರೆ ಕೈಗೊಳ್ಳ ಬಹುದು. ಅದು ಭಾರಿಯಾಗೇ ಇರಬೇಕಲ್ಲವೆ...?”

ಅದೇ ದಿನವೇ ಅವರು ಅಮೆರಿಕದ ಶಿಷ್ಯೆ ಶ್ರೀಮತಿ ಹ್ಯಾನ್ಸ್​ಬ್ರೋಗೆ ಪತ್ರ ಬರೆದು ಅವಳ ಹಾಗೂ ಇನ್ನಿತರರ ಯೋಗಕ್ಷೇಮವನ್ನು ವಿಚಾರಿಸಿಕೊಂಡರು.

ಇತ್ತ ಜಪಾನಿನಲ್ಲಿ ಮಿಸ್ ಮೆಕ್​ಲಾಡ್ ಅಲ್ಲಿನ ಸುಂದರ ಪರಿಸರದಲ್ಲಿ ತುಂಬ ಸಂತೋಷ ದಿಂದಿದ್ದಳು. ಅಲ್ಲದೆ ತಾನು ಕ್ಯಾಲಿಫೋರ್ನಿಯದಲ್ಲಿ ಮಾಡಿದ್ದಂತೆ, ಸ್ವಾಮೀಜಿಯವರು ತಮ್ಮ ಕಾರ್ಯಕ್ಷೇತ್ರವನ್ನು ಪ್ರಾರಂಭಿಸಲು ಅನುವು ಮಾಡುತ್ತಿದ್ದಳು. ಜಪಾನ್ ಹಾಗೂ ಚೀನಾಗಳಿಗೆ ಭೇಟಿ ನೀಡಬೇಕು, ಅಲ್ಲಿ ವೇದಾಂತ ಪ್ರಸಾರ ಮಾಡಬೇಕು ಎಂಬ ಆಸೆ ಸ್ವಾಮೀಜಿಯವರಿಗೆ ಹಿಂದಿನಿಂದಲೂ ಇತ್ತು. ಅಲ್ಲದೆ ಜಪಾನಿನಿಂದ ಭಾರತವು ಕಲಿತುಕೊಳ್ಳಬೇಕಾದದ್ದು ಬಹಳಷ್ಟಿದೆ ಎನ್ನುವುದು ಅವರಿಗೆ ತಿಳಿದಿತ್ತು. ಆದ್ದರಿಂದ ಮಿಸ್ ಮೆಕ್​ಲಾಡಳಿಗೆ ಜಪಾನಿಗೆ ಹೋಗುವ ಮುನ್ನ ತಮ್ಮನ್ನು ಕಂಡಾಗ, ಅಲ್ಲಿಂದ ಆಕೆ ಪತ್ರ ಬರೆದರೆ ತಾವು ಬರುವುದಾಗಿ ಹೇಳಿದ್ದರು. ಜಪಾನಿನಲ್ಲಿ ಮೆಕ್​ಲಾಡಳಿಗೆ ಕಕುಸೋ ಒಕಾಕುರ ಎಂಬವನ ಪರಿಚಯವಾಯಿತು. ಇವನು ಸ್ವಾಮೀಜಿ ಯವರನ್ನು ಜಪಾನಿಗೆ ಆಹ್ವಾನಿಸಲು ಉತ್ಸುಕನಾಗಿ, ಅವರಿಗೆ ಪ್ರಯಾಣದ ಖರ್ಚಿಗೆ ಮುಂಗಡ ವಾಗಿ ಮೂನ್ನೂರು ರೂಪಾಯಿಯ ಚೆಕ್ ಕಳಿಸಿದ. ಆದರೆ ಕೂಡಲೇ ಹೊರಡಲು ಅವರಿಗೆ ಸಾಧ್ಯವೆ ಇರಲಿಲ್ಲ. ತೀವ್ರ ಅನಾರೋಗ್ಯದೊಂದಿಗೆ ಇತರ ಅಡಚಣೆಗಳೂ ಇದ್ದುವು. ಅವರೊಬ್ಬರೇ ದೀರ್ಘ ಪ್ರಯಾಣ ಮಾಡುವುದಂತೂ ಅಸಾಧ್ಯವಾಗಿತ್ತು. ಆದರೆ ಹೋಗಬೇಕೆಂಬ ಇಚ್ಛೆಯೂ ಪ್ರಬಲವಾಗಿಯೇ ಇತ್ತು. ಆದ್ದರಿಂದ ಕೆಲವು ದಿನಗಳಾದಮೇಲೆ ಬರುವುದಾಗಿ ಬರೆದರು. ಇಷ್ಟರ ಮೇಲೆ, ಎಲ್ಲವೂ ಜಗನ್ಮಾತೆಯ ಇಚ್ಛೆ ಎಂದೂ ಸ್ಪಷ್ಟವಾಗಿ ತಿಳಿಸಿದರು.

ಬಹುಶಃ ಮೆಕ್​ಲಾಡಳಿಗೆ ಸ್ವಾಮೀಜಿಯವರ ಆರೋಗ್ಯದ ಸ್ಪಷ್ಟ ಅರಿವು ಇರಲಿಲ್ಲವೆಂದು ತೋರುತ್ತದೆ. ಅವರು ಬರಲು ಸಾಧ್ಯವಾಗುವುದೆಂದೇ ಆಕೆ ನಿರೀಕ್ಷಿಸಿರಬೇಕು. ಆದ್ದರಿಂದ ಆಕೆ ಮತ್ತೆ ಮತ್ತೆ ಒತ್ತಾಯಿಸಿ ಪತ್ರಗಳನ್ನು ಬರೆಯುತ್ತಲೇ ಇದ್ದಳು. ಸ್ವಾಮೀಜಿ ಕೂಡ ಇನ್ನು ತಾವು ಹೊರಡುವುದೆಂದೇ ನಿಶ್ಚಯಿಸಿಬಿಟ್ಟರು. ಈ ವಿಷಯವಾಗಿ ಅವರು ಕ್ರಿಸ್ಟೀನಳಿಗೆ ಹೀಗೆ ಬರೆದರು– “ಮಿಸ್ ಮೆಕ್​ಲಾಡಳು ಅಷ್ಟೊಂದು ಒತ್ತಾಯಿಸುತ್ತಿರುವುದರಿಂದ ನಾನು ಜಪಾನಿಗೆ ಹೊರಟು ಬಿಡಲೇ ಎಂದು ಆಲೋಚಿಸುತ್ತಿದ್ದೇನೆ. ಏನೋ, ಯಾರಿಗೆ ಗೊತ್ತು, ಆದರೂ ಆಗಬಹುದು. ಜಪಾನಿಗೆ ಹೋದಮೇಲೆ ಅಲ್ಲಿಂದ ಅಮೆರಿಕಕ್ಕೂ ಒಂದು ಭೇಟಿ ಕೊಡದಿರಲು ಸಾಧ್ಯವಿಲ್ಲವೆಂದು ಕಾಣುತ್ತದೆ... ”

ಅವರು ಈ ಪತ್ರವನ್ನು ಬರೆಯುತ್ತಿರುವಾಗಲೇ ಮೆಕ್​ಲಾಡಳಿಂದ ಅವರಿಗೊಂದು ತಂತಿ ಬಂದಿತು. ಆದ್ದರಿಂದ ಪತ್ರದ ಕೊನೆಯಲ್ಲಿ ಸೇರಿಸಿದರು–“ಅವಳು ಎಷ್ಟು ಪಟ್ಟು ಹಿಡಿದಿದ್ದಾ ಳೆಂದರೆ ಹೊರಟೇಬಿಡೋಣವೇ ಎನ್ನಿಸುತ್ತಿದೆ. ಹಾಗೇನಾದರೂ ಆದರೆ ಈ ಚಳಿಗಾಲದಲ್ಲಿ ಅಮೆರಿಕೆಗೆ ಹೋಗಿ ಅಲ್ಲಿಂದ ಇಂಗ್ಲೆಂಡಿಗೆ ಹೋಗುತ್ತೇನೆ.”

ಆದರೆ ಕಡೆಗೂ ಸ್ವಾಮೀಜಿಯವರಿಗೆ ಹೊರಡಲು ಸಾಧ್ಯವಾಗಲೇ ಇಲ್ಲ. ಆದ್ದರಿಂದ ತಾವು ಬರಲಾಗದ್ದಕ್ಕೆ ಕ್ಷಮಿಸುವಂತೆ ಮೆಕ್​ಲಾಡಳಿಗೂ ಒಕಾಕುರನಿಗೂ ಪತ್ರ ಬರೆದರು. ಆದರೂ ಅವರಿಬ್ಬರೂ ಪೂರ್ಣ ನಿರಾಶರಾಗಲಿಲ್ಲ. ಬಳಿಕ ಡಿಸೆಂಬರಿನಲ್ಲಿ ಒಕಾಕುರ ತನ್ನ ಸ್ನೇಹಿತನೊಂದಿಗೆ ಭಾರತಕ್ಕೆ ಬಂದು ಅವರನ್ನು ಖುದ್ದಾಗಿ ಆಹ್ವಾನಿಸುವುದನ್ನು ನಾವು ನೋಡಲಿದ್ದೇವೆ.

ಬಂಗಾಳದಲ್ಲಿ ಕವಿಗಳು ಹಾಡಿದ ಮಳೆಗಾಲವು ಸಮಯಕ್ಕೆ ಸರಿಯಾಗಿ ಪ್ರಾರಂಭವಾಯಿತು. ಆಗಸ್ಟ್​-ಸೆಪ್ಟೆಂಬರ್ ಹೊತ್ತಿಗೆ ಮಳೆ ಪೂರ್ಣ ರಭಸದಿಂದ ಸುರಿಯತೊಡಗಿತು. ಗಂಗಾನದಿ ಯಲ್ಲಿ ಪ್ರವಾಹ ಏರಿತು. ಈ ಸಮಯದಲ್ಲಿ ಸ್ವಾಮೀಜಿ ತಮ್ಮ ಆತ್ಮೀಯ ಶಿಷ್ಯೆ ಕ್ರಿಸ್ಟೀನಳಿಗೆ ಬರೆದರು–

“ಮಳೆ ಹಗಲಿರುಳೆನ್ನದೆ ಜರ್ರೋ ಎಂದು ಸುರಿಯುತ್ತಿದೆ. ಎಲ್ಲೆಲ್ಲಿ ನೋಡಿದರೂ ನೀರೇ ನೀರು. ನದಿ ಏರುತ್ತಿದೆ. ದಡವನ್ನು ಮೀರಿ ಹರಿಯುತ್ತಿದೆ. ಕೆರೆ ಕುಂಟೆಗಳೆಲ್ಲ ಭರ್ತಿಯಾಗಿವೆ. ಈಗತಾನೆ ನಾನು ಮಠದ ಆವರಣದಿಂದ ನೀರು ಹರಿದು ಹೋಗಲು ತೋಡುತ್ತಿರುವ ಕಾಲುವೆಯ ಕೆಲಸದಲ್ಲಿ ಸ್ವಲ್ಪ ನೆರವಾಗಿ ಹಿಂದಿರುಗುತ್ತಿದ್ದೇನೆ. ಕೆಲವು ಜಾಗಗಳಲ್ಲಂತೂ ನೀರು ಹಲವಾರು ಅಡಿ ಆಳ ನಿಲ್ಲುತ್ತದೆ. ನನ್ನ ಕೊಕ್ಕರೆಗಂತೂ ಬಹಳ ಖುಷಿಯಾಗಿದೆ. ಇತರ ಬಾತುಕೋಳಿಗಳಿಗೂ ಅಷ್ಟೆ. ಆದರೆ ನನ್ನ ಚಿಗರೆ ಮರಿ ತಪ್ಪಿಸಿಕೊಂಡು ಕೆಲವು ದಿನ ತಲೆನೋವು ಕೊಟ್ಟಿತು. ಆಮೇಲೆ ಸಿಕ್ಕಿತೆನ್ನು. ದುರದೃಷ್ಟಕ್ಕೆ ನಿನ್ನೆ ದಿನ ನನ್ನ ಒಂದು ಬಾತುಕೋಳಿ ಸತ್ತುಹೋಯಿತು. ಒಂದು ವಾರದಿಂದಲೂ ಅದು ಉಸಿರಿಗಾಗಿ ಚಡಪಡಿಸುತ್ತಿತ್ತು. ನಮ್ಮಲ್ಲೊಬ್ಬರು ಚೇಷ್ಟೆ ಸ್ವಭಾವದ ವೃದ್ಧ ಸಂನ್ಯಾಸಿಗಳು ಹೇಳುತ್ತಾರೆ, ‘ಅಯ್ಯೊ ಬಿಡಿ ಸ್ವಾಮಿ, ಈ ಕಲಿಯುಗದಲ್ಲಿ ಬದುಕಿ ಪ್ರಯೋಜನವಿಲ್ಲ. ಕಲಿಗಾಲದಲ್ಲಿ ಮಳೆ ಬಂದರೆ ಬಾತುಗಳಿಗೆ ನೆಗಡಿಯಾಗುತ್ತದೆ, ಕಪ್ಪೆಗಳು ಸೀನುತ್ತವೆ’ ಎಂದು. ನಮ್ಮ ಇನ್ನೊಂದು ಬಾತುಕೋಳಿಗೆ ಪುಕ್ಕ ಉದುರಲಾರಂಭಿಸಿತ್ತು. ನನಗೆ ಬೇರೇನು ಮಾಡಲೂ ತೋಚಲಿಲ್ಲ. ಕಡೆಗೆ, ಅದು ಬದುಕಿದರೆ ಬದುಕಲಿ ಸತ್ತರೆ ಸಾಯಲಿ ಎಂದು ತೀರ್ಮಾನಿಸಿ ಕಾರ್ಬಾಲಿಕ್ ಆಮ್ಲ ಹಾಕಿದ ನೀರಿನಲ್ಲಿ ಸ್ವಲ್ಪ ಹೊತ್ತು ಬಿಟ್ಟೆ. ಈಗ ಅದು ಚೆನ್ನಾಗಿದೆ.”

ವಿಶ್ವವಿಖ್ಯಾತರಾದ ವಿವೇಕಾನಂದರು ಬರೆದ ಈ ಪತ್ರ ನಿಜಕ್ಕೂ ತುಂಬ ಸ್ವಾರಸ್ಯಕರವಾಗಿದೆಯಲ್ಲವೆ?

ಮಠದಲ್ಲಿ ಈ ಎಲ್ಲ ಪ್ರಾಣಿಗಳೊಂದಿಗೆ ಎರಡು ನಾಯಿಗಳಿದ್ದುವು–ಒಂದು ಬಾಘಾ (ವ್ಯಾಘ್ರ), ಇನ್ನೊಂದು ಲಯನ್. ಇವೆಲ್ಲವುಗಳ ಪೈಕಿ ಬಾಘಾಗೆ ಒಂದು ವಿಶೇಷ ಸ್ಥಾನ. ಅದಕ್ಕೂ ಈ ವಿಷಯ ಚೆನ್ನಾಗಿ ಗೊತ್ತು. ಮಠವು ನ್ಯಾಯಬದ್ಧವಾಗಿ ತನಗೇ ಸೇರಿದ್ದು ಎಂಬಂತೆ ಅದು ವರ್ತಿಸುತ್ತಿತ್ತು. ತನಗಿಷ್ಟ ಬಂದಂತೆ ಓಡಾಡಿಕೊಂಡಿತ್ತು. ಆದರೆ ಒಂದು ದಿನ ಅದು ಶ್ರೀರಾಮ ಕೃಷ್ಣರ ಪೂಜೆಗೆಂದು ತಂದಿದ್ದ ನೀರನ್ನು ಮೈಲಿಗೆ ಮಾಡಿಬಿಟ್ಟಿತು. ಇದಕ್ಕೆ ಚಿಕ್ಕ ಸ್ವಾಮಿಗಳೊಬ್ಬರ ಅಜಾಗರೂಕತೆಯೂ ಕಾರಣವಾಗಿತ್ತು. ಈ ವಿಷಯ ತಿಳಿದಾಗ ಅವರಿಗೆ ಸ್ವಾಮೀಜಿ ಚೆನ್ನಾಗಿ ಛೀಮಾರಿ ಹಾಕಿದರು. ಆದರೆ ಈ ದುರ್ವರ್ತನೆಗಾಗಿ ಬಾಘಾಗೂ ಶಿಕ್ಷೆಯಾಗಬೇಕೆಂದು ಇತರರೆಲ್ಲ ಅಭಿಪ್ರಾಯಪಟ್ಟರು. ಸರಿ, ಅದಕ್ಕೆ ಗಡೀಪಾರು ಶಿಕ್ಷೆ ವಿಧಿಸಬೇಕೆಂದು ತೀರ್ಮಾನವಾಯಿತು. ಅದು ಮರಳಿ ಬಾರದಂತೆ ಗಂಗಾನದಿಯ ಆಚೆ ತೀರಕ್ಕೆ ತೆಗೆದುಕೊಂಡು ಹೋಗಿ ಬಿಟ್ಟು ಬಂದರು.

ಆ ದಿನವೆಲ್ಲ ಅದು ಹೇಗೆ ಕಳೆಯಿತೊ ಗೊತ್ತಿಲ್ಲ. ಆದರೆ ಆ ರಾತ್ರಿ, ದಡದಿಂದ ದಡಕ್ಕೆ ಪ್ರಯಾಣಿಕರನ್ನು ಸಾಗಿಸುವ ಬಾಡಿಗೆ ದೋಣಿ ಕೊನೆಯ ಪ್ರಯಾಣ ಹೊರಟಾಗ ಬಾಘಾ ಒಳಗೆ ಹಾರಿ ಕುಳಿತುಬಿಟ್ಟಿತು. ‘ಇದ್ಯಾರಿದು ಟಿಕೆಟ್ ರಹಿತ ಪ್ರಯಾಣಿಕ!’ ಎಂದು ದೋಣಿಯವನೂ ಇತರರೂ ಅದನ್ನು ಹೊರಗಟ್ಟಲು ನೋಡಿದರು. ಆದರೆ ಬಾಘಾ ಹುಲಿಯಂತೆಯೇ ಗರ್ಜಿಸಿ ಗುರುಗುಟ್ಟಿತು. ಬಳಿಕ ಯಾರೂ ಅದರ ಹಕ್ಕನ್ನು ಪ್ರಶ್ನಿಸಲು ಹೋಗದೆ ಸುಮ್ಮನಾದರು. ದೋಣಿಯವನು ತನ್ನ ಹುಟ್ಟುಗೋಲಿನಿಂದ ಅದನ್ನು ಮಟ್ಟಹಾಕದಿದ್ದುದಕ್ಕೆ ಕಾರಣ ಅದರ ಧೈರ್ಯವೋ ಅಥವಾ ಅದರ ಪುಣ್ಯವೋ ಎನ್ನುವುದು ಬೇರೆ ವಿಚಾರ. ಅಂತೂ ಜನ ಸುಮ್ಮನಾದ ಕೂಡಲೆ ತಾನೂ ಸುಮ್ಮನೆ ಕುಳಿತುಕೊಂಡಿತು. ಸಹಪ್ರಯಾಣಿಕರಿಗಿಂತ ಸಭ್ಯವಾಗಿತ್ತೆನ್ನಲೂ ಬಹುದು. ದೋಣಿ ಈಚೆಯ ದಡಕ್ಕೆ ತಲಪುತ್ತಿದ್ದಂತೆಯೇ ಬಾಘಾ ಛಂಗನೆ ನೆಗೆದು ಕಣ್ಮರೆಯಾಯಿತು.

ಬೆಳಿಗ್ಗೆ ನಾಲ್ಕು ಗಂಟೆಗೆ ಸ್ವಾಮೀಜಿ ಎಂದಿನಂತೆ ಬಾಗಿಲು ತೆರೆದುಕೊಂಡು ಹೊರಗೆ ಬಂದಾಗ ಕಾಲಿಗೆ ಏನೋ ತಾಗಿತು. ನೋಡಿದರೆ, ಬಾಘಾ! ಅರೆ! ನಿನ್ನೆ ತಾನೆ ನದಿಯಾಚೆಗೆ ಬಿಟ್ಟ ನಾಯಿ ಅದು ಹೇಗೆ ಬಂದುಬಿಟ್ಟಿತು! ‘ನೀವು ನನ್ನನ್ನು ದೂರ ಮಾಡಿದರೂ ನಾನು ಮಾತ್ರ ನಿಮ್ಮ ಪಾದ ಬಿಡುವವನಲ್ಲ!’ ಎನ್ನುವಂತೆ ಅವರ ಕಾಲಿನ ಸುತ್ತ ಸುತ್ತಿಕೊಂಡು ಬಾಲ ಅಲ್ಲಾಡಿಸುತ್ತ ಕುಂಯ್​ಗುಟ್ಟಿತು. ತಕ್ಷಣ ಸ್ವಾಮೀಜಿ ಅದರ ಮೈದಡವಿ ಸಂತೈಸಿದರು. ಬಳಿಕ ಈ ವಿಷಯವನ್ನು ಇತರರಿಗೂ ತಿಳಿಸಿ ಇನ್ನು ಮೇಲೆ ಅದು ಏನೇ ಮಾಡಿದರೂ ಅದನ್ನು ಓಡಿಸಬಾರದು ಎಂದು ತಾಕೀತು ಮಾಡಿದರು. ಆದರೆ ಆ ಮಠದಲ್ಲಿ ಅತ್ಯುಚ್ಚ ನ್ಯಾಯಾಧೀಶರು ಯಾರು ಎನ್ನುವುದಾಗಲಿ, ಆ ನ್ಯಾಯಾಧೀಶರು ಎಲ್ಲರ ವಿಷಯದಲ್ಲೂ ದಯಾಪರರು, ಕ್ಷಮಾಶೀಲರು ಎನ್ನುವುದಾಗಲಿ ಈ ನಾಯಿಗೆ ಹೇಗೆ ಗೊತ್ತಾಯಿತೆಂಬುದೇ ಎಲ್ಲರ ಆಶ್ಚರ್ಯ.

