
\chapter{ನಿವೇದಿತೆಯ ನಿರ್ಮಾಣ}

\noindent

ಸ್ವಾಮೀಜಿಯವರು ಕಲ್ಕತ್ತದಲ್ಲಿದ್ದ ಈ ಅವಧಿಯಲ್ಲಿ ಸಾರ್ವಜನಿಕವಾಗಿ ಕಾಣಿಸಿಕೊಂಡದ್ದು ಕೆಲವೇ ಬಾರಿ. ಇವುಗಳಲ್ಲೊಂದೆಂದರೆ, ಮಾರ್ಚ್ ೧೧ರಂದು ಸ್ಟಾರ್ ಥಿಯೇಟರಿನಲ್ಲಿ ನಡೆದ ಸಭೆ. ಅಂದಿನ ಮುಖ್ಯ ಭಾಷಣಕರ್ತೆ ಮಿಸ್ ಮಾರ್ಗರೆಟ್ ನೋಬೆಲ್; ವಿಷಯ: “ಇಂಗ್ಲೆಂಡಿ ನಲ್ಲಿ ಭಾರತೀಯ ಆಧ್ಯಾತ್ಮಿಕ ಚಿಂತನೆಯ ಪ್ರಭಾವ.” ಶ್ರೀಮತಿ ಸಾರಾ ಬುಲ್ ಹಾಗೂ ಮಿಸ್ ಹೆನ್ರಿಟಾ ಮುಲ್ಲರ್ ವೇದಿಕೆಯ ಮೇಲೆ ಉಪಸ್ಥಿತರಿದ್ದರು. ತಮ್ಮ ಶಿಷ್ಯೆಯನ್ನು ಸಭಿಕರಿಗೆ ಪರಿಚಯಿಸಿಕೊಡುತ್ತ ಸ್ವಾಮೀಜಿಯವರು, “ಈಕೆ ಭಾರತಕ್ಕೆ ಇಂಗ್ಲೆಂಡಿನ ಇನ್ನೊಂದು ಕೊಡುಗೆ” ಎಂದು ಬಣ್ಣಿಸಿದರು. ಉಳಿದಿಬ್ಬರಾದ ಶ್ರೀಮತಿ ಆ್ಯನಿ ಬೆಸೆಂಟ್ ಹಾಗೂ ಮಿಸ್ ಹೆನ್ರಿಟಾ ಮುಲ್ಲರ್​ರಂತೆ ಮಿಸ್ ನೋಬೆಲ್ಲಳೂ ಭಾರತಕ್ಕಾಗಿ ತನ್ನ ಜೀವನವನ್ನು ಸಮರ್ಪಿಸಿದ್ದಾಳೆ ಎಂದರು ಸ್ವಾಮೀಜಿ.

ಮಾರ್ಗರೆಟ್ಟಳು ತನ್ನ ಸುಲಲಿತ ವಾಗ್ಝರಿಯಿಂದ ಸಭಿಕರನ್ನು ಸಂಪೂರ್ಣ ಗೆದ್ದುಕೊಳ್ಳಲು ಸಮರ್ಥಳಾದಳು. ಆಕೆಯ ಭಾಷಣದ ಮಧ್ಯೆ ಸಭಿಕರು ಹಲವಾರು ಬಾರಿ ಕರತಾಡನ ಮಾಡಿ ತಮ್ಮ ಮೆಚ್ಚುಗೆ ಸೂಚಿಸಿದರು. ಮಾರ್ಗರೆಟ್ಟಳ ಭಾಷಣದ ಬಳಿಕ ಸ್ವಾಮೀಜಿಯವರು ಶ್ರೀಮತಿ ಸಾರಾ ಬುಲ್ ಹಾಗೂ ಮಿಸ್ ಮುಲ್ಲರ್ ಇವರಿಬ್ಬರಿಗೂ ಕೆಲವು ಮಾತುಗಳನ್ನಾಡುವಂತೆ ಸೂಚಿಸಿದರು. ಶ್ರೀಮತಿ ಸಾರಾ ತನ್ನ ಭಾಷಣದಲ್ಲಿ, “ಇದೀಗ ಪಾಶ್ಚಾತ್ಯ ಬುದ್ಧಿಗೆ ಭಾರತೀಯ ಸಾಹಿತ್ಯವು ಜೀವಂತವಾಗಿ-ಜ್ವಲಂತವಾಗಿ ಭಾಸವಾಗುತ್ತಿದೆ. ಅದರಲ್ಲೂ ಸ್ವಾಮಿ ವಿವೇಕಾನಂದರ ಗ್ರಂಥಗಳಂತೂ ಅಮೆರಿಕದಲ್ಲಿ ಮನೆಮಾತಾಗಿವೆ” ಎಂದಳು. ಮಿಸ್ ಮುಲ್ಲರ್ ಎದ್ದು ನಿಂತು ಸಭಿಕರನ್ನು “ನನ್ನ ಪ್ರಿಯ ಸ್ನೇಹಿತರೇ ಮತ್ತು ದೇಶಬಾಂಧವರೇ” ಎಂದು ಸಂಬೋಧಿಸಿದಾಗ ಪ್ರಚಂಡ ಕರತಾಡನ ಮತ್ತು ಹರ್ಷೋದ್ಗಾರ! ತಾನು ಹೀಗೆ ಕರೆದುದಕ್ಕೆ ಕಾರಣವನ್ನು ವಿವರಿಸುತ್ತ ಮಿಸ್ ಮುಲ್ಲರ್ ಹೇಳಿದಳು, “ನಾನು ಮತ್ತು ಸ್ವಾಮೀಜಿಯವರ ಇತರ ಪಾಶ್ಚಾತ್ಯ ಅನುಯಾಯಿ ಗಳು ಇಲ್ಲಿಗೆ (ಭಾರತಕ್ಕೆ) ಬಂದಾಗ, ನಮಗೆ ನಮ್ಮ ಮಾತೃಭೂಮಿಗೇ ಬಂದಂತೆನ್ನಿಸಿತು” ಎಂದು. ಈ ಕಾರ್ಯಕ್ರಮದ ಯಶಸ್ಸು–ಅದರಲ್ಲೂ ಮುಖ್ಯವಾಗಿ ಮಾರ್ಗರೆಟ್ಟಳ ಯಶಸ್ಸು– ಸ್ವಾಮೀಜಿಯವರಿಗೆ ತುಂಬ ಸಂತೋಷವನ್ನುಂಟುಮಾಡಿತು. ಅವಳ ವಾಕ್ ಸಾಮರ್ಥ್ಯವನ್ನು ಕಂಡು ಅವರು, ಅವಳು ಈ ಬಗೆಯ ಭಾಷಣಗಳ ಮೂಲಕ ತಮ್ಮ ಕಾರ್ಯಕ್ಕೆ ವಿಶೇಷವಾಗಿ ನೆರವಾಗಬಲ್ಲಳು ಎಂದು ಊಹಿಸಿದರು.

ಈಗ ಸ್ವಾಮೀಜಿ ಮಾರ್ಗರೆಟ್ಟಳಿಗೆ ವಿಧಿಬದ್ಧವಾಗಿ ಬ್ರಹ್ಮಚರ್ಯದೀಕ್ಷೆ ನೀಡಿ ಅವಳನ್ನು ಕಾರ್ಯರಂಗಕ್ಕೆ ತೊಡಗಿಸಲು ನಿರ್ಧರಿಸಿದರು. ಆಕೆಯ ಪಾಲಿಗೆ ಇದೊಂದು ಅವಿಸ್ಮರಣೀಯ ಘಟನೆ. ಮಾರ್ಚ್ ೨೫ರಂದು ದೀಕ್ಷೆ ಕೊಡುವುದೆಂದು ನಿರ್ಧಾರವಾಯಿತು. ಅಂದು ಸ್ವಾಮೀಜಿ ಯವರು ಮಾರ್ಗರೆಟ್ಟಳನ್ನು ಶ್ರೀರಾಮಕೃಷ್ಣ ಪೂಜಾ ಗೃಹಕ್ಕೆ ಬರಮಾಡಿಕೊಂಡು ಆಸನದ ಮೇಲೆ ತಮಗೆದುರಾಗಿ ಕುಳ್ಳಿರಿಸಿಕೊಂಡರು. ಆಕೆಗೆ ಅವರು ಮೊದಲಿಗೆ ಶಿವಪೂಜೆಯನ್ನು ಹೇಳಿಕೊಟ್ಟರು. ಶಿವನು ಸಂನ್ಯಾಸಿಗಳ-ತ್ಯಾಗಿಗಳ ದೇವರಲ್ಲವೆ? ಅನಂತರ ಅವರು ಆಕೆಗೆ ಮಂತ್ರದೀಕ್ಷೆ ನೀಡಿದರು. ಆ ಶುಭ ಸಂದರ್ಭದಲ್ಲೇ ಬ್ರಹ್ಮಚರ್ಯದೀಕ್ಷೆಯನ್ನೂ ಅನುಗ್ರಹಿಸಿ, “ನಿವೇದಿತಾ” ಎಂಬ ನೂತನ ನಾಮಧೇಯವನ್ನು ದಯಪಾಲಿಸಿದರು. ನಿವೇದಿತಾ–ಭಾರತದ ಸೇವೆಗಾಗಿ ನಿವೇದಿಸಲ್ಪಟ್ಟವಳು, ಸಮರ್ಪಿತಳಾದವಳು! ಭಾರತಕ್ಕಾಗಿ ಹಾಗೂ ಸ್ವಾಮೀಜಿಯವರ ಕಾರ್ಯ ಕ್ಕಾಗಿ ತನ್ನನ್ನು ಸಮರ್ಪಿಸಿಕೊಳ್ಳಲು ಮುಂದಾದ ಆಕೆಗೆ ಅತ್ಯಂತ ಸೂಕ್ತ ನಾಮಧೇಯ–ಸೋದರಿ ನಿವೇದಿತಾ.

ದೀಕ್ಷಾಕಾರ್ಯ ಮುಗಿದ ಬಳಿಕ ಸ್ವಾಮೀಜಿ ತಮ್ಮ ಪಾಶ್ಚಾತ್ಯ ಶಿಷ್ಯೆಯರನ್ನೆಲ್ಲ ಮಹಡಿಯ ಮೇಲಕ್ಕೆ ಕರೆದೊಯ್ದರು. ಸ್ವಾಮೀಜಿ ತಮ್ಮ ಅಂಗಾಂಗಗಳಿಗೆ ವಿಭೂತಿಯನ್ನು ಬಳಿದು ಕೊಂಡರು, ಜಟೆಯನ್ನು ಧರಿಸಿ ಶಿವನಂತೆ ಕಂಗೊಳಿಸುತ್ತಿದ್ದರು! ತಮ್ಮ ಶಿಷ್ಯರಿಗಾಗಿ ಅವರು ಒಂದು ಗಂಟೆಯ ಕಾಲ ತಮ್ಮ ಸುಮಧುರ ಕಂಠದಿಂದ ಹಾಡುಗಳನ್ನು ಹಾಡಿದರು. ತಮ್ಮ ಧರ್ಮಪ್ರಸಾರ ಕಾರ್ಯಕ್ಕಾಗಿ ಪರಿಗ್ರಹಿಸಿದ ನಿವೇದಿತೆಯನ್ನುದ್ದೇಶಿಸಿ ಸ್ವಾಮೀಜಿ ಹೇಳಿದ ಮಾತು ಇದು: “ಮುನ್ನಡೆ; ಬುದ್ಧತ್ವವನ್ನು ಹೊಂದುವ ಮುನ್ನ ಇತರರ ಹಿತಕ್ಕಾಗಿ ಐದುನೂರು ಸಲ ಜನ್ಮವೆತ್ತಿ ತನ್ನ ಜೀವನವನ್ನು ಸಮರ್ಪಿಸಿದ ಬುದ್ಧಭಗವಂತನ ಹೆಜ್ಜೆಯಲ್ಲಿ ಹೆಜ್ಜೆಯನ್ನಿಡುತ್ತ ಮುನ್ನಡೆ!”

ನಿವೇದಿತಾಳ ಪಾಲಿಗೆ ಈ ದೀಕ್ಷಾಕಾರ್ಯವು ಅತ್ಯಂತ ಮಹತ್ವಪೂರ್ಣವಾದದ್ದು. ಇಂಥದೇ ಇನ್ನೊಂದು ಪ್ರಮುಖವಾದ ಘಟನೆಯೆಂದರೆ ಆಕೆ ಶ್ರೀಮಾತೆ ಶಾರದಾದೇವಿಯವರನ್ನು ಭೇಟಿ ಯಾದದ್ದು. ದೀಕ್ಷಾಕಾರ್ಯಕ್ಕೆ ಒಂದು ವಾರದ ಹಿಂದೆ ಈ ಭೇಟಿ ನಡೆಯಿತು. ಹಳ್ಳಿಯಲ್ಲೇ ಹುಟ್ಟಿ ಬೆಳೆದ, ಅತ್ಯಂತ ಸಂಪ್ರದಾಯಸ್ಥ ಬ್ರಾಹ್ಮಣ ಕುಟುಂಬದ, ಅನಕ್ಷರಸ್ಥ ಮಹಿಳೆ–ಶ್ರೀಮಾತೆ ಯವರು. ನಿವೇದಿತೆಯಾದರೋ, ಆಳರಸರ ನಾಡಾದ ಇಂಗ್ಲೆಂಡಿನ ಸುಶಿಕ್ಷಿತ ಯುವತಿ (ಈಕೆ ಮೂಲತಃ ಐರ್ಲೆಂಡಿನ ಜನಾಂಗಕ್ಕೆ ಸೇರಿದವಳು), ಕ್ರೈಸ್ತ ಪುರೋಹಿತರ ವಂಶದವಳು. ಹೀಗೆ ಶ್ರೀಮಾತೆ ಹಾಗೂ ನಿವೇದಿತಾ–ಇವರ ನಡುವೆ ಧ್ರುವಗಳಷ್ಟು ಅಂತರ! ಶ್ರೀಮಾತೆಯವರು ಪುರಾತನ ಭಾರತದ ಪ್ರತಿನಿಧಿಯಾದರೆ, ನಿವೇದಿತಾ ಆಧುನಿಕ ಪಾಶ್ಚಾತ್ಯ ಜಗತ್ತಿನ ಪ್ರತಿನಿಧಿ. ನಿವೇದಿತಾ ಶ್ರೀಮಾತೆಯವರ ದರ್ಶನವನ್ನು ಪಡೆಯಲು ಹೊರಟಾಗ ಆಕೆಯ ಪಾಶ್ಚಾತ್ಯ ಸ್ನೇಹಿತೆಯರಿಬ್ಬರೂ ಜೊತೆಯಲ್ಲಿ ಬಂದರು. ಎದೆಯಲ್ಲಿ ಏನೋ ಅಳುಕು, ಏನೋ ಕಾತರ. ಸ್ವಾಮೀಜಿಯವರ ಪಾಲಿಗೋ ಶ್ರೀಮಾತೆಯವರು ಕೇವಲ ಗುರುಪತ್ನಿಯಲ್ಲ; ಸಾಕ್ಷಾತ್ ಭಗವತಿ! ಅವರು ಶ್ರೀರಾಮಕೃಷ್ಣರ ಪ್ರತಿನಿಧಿಯೆಂದೇ ಸ್ವಾಮೀಜಿಯವರ ನಂಬಿಕೆ. ಇಂತಹ ಶ್ರೀಮಾತೆ ಯವರು ತನ್ನನ್ನು ಹೇಗೆ ಸ್ವೀಕರಿಸಬಹುದು? ತನ್ನನ್ನು ಕಂಡು ಏನೆನ್ನಬಹುದು? ಮ್ಲೇಚ್ಛಳೆಂದು ದೂರ ಮಾಡಿಯಾರೆ? ಇಲ್ಲ! ಹಾಗೆಂದಿಗೂ ಆಗಲಾರದು!.... ನಿವೇದಿತೆಯ ಮನಸ್ಸಿನಲ್ಲಿ ಭಾವಗಳ ತುಮುಲ! ಸ್ವಾಮೀಜಿಯವರಿಗೂ ಒಂದು ಬಗೆಯ ಕುತೂಹಲ–ತಮ್ಮ ಶಿಷ್ಯೆಯ ರನ್ನು ಕಂಡು ಮಾತೆಯವರಿಗೇನೆನ್ನಿಸೀತು? ಎಂದು. ಸ್ವಾಮೀಜಿಯವರ ಕರೆಯ ಮೇರೆಗೆ ಶ್ರೀಮಾತೆಯವರು ಜಯರಾಂಬಾಟಿಯಿಂದ ಕಲ್ಕತ್ತಕ್ಕೆ ಬಂದಿದ್ದರು. ಸ್ವಾಮೀಜಿ ತಮ್ಮ ಶಿಷ್ಯೆಯ ರನ್ನು ಕರೆದೊಯ್ದು ಶ್ರೀಮಾತೆಯವರಿಗೆ ಪರಿಚಯ ಮಾಡಿಸಿದರು. ಅವರೆಲ್ಲರನ್ನೂ ಶ್ರೀಮಾತೆ ಯವರು ಅತ್ಯಂತ ಪ್ರೀತಿಯಿಂದ ಬರಮಾಡಿಕೊಂಡು, ಸ್ವಂತ ಮಕ್ಕಳಂತೆ ಆದರಿಸಿದರು; ಅವರನ್ನು ತಮ್ಮ ಮಕ್ಕಳೆಂದೇ ಕರೆದರು. ಶ್ರೀಮಾತೆಯವರಿಗೆ ಇಂಗ್ಲಿಷ್ ಬಾರದು, ಈ ಪಾಶ್ಚಾತ್ಯರಿಗೆ ಬಂಗಾಳಿ ಬಾರದು. ಆದರೇನಂತೆ? ಹೃದಯದ ಭಾಷೆಯಿದೆಯಲ್ಲ! ಅದಕ್ಕೆ ಶಬ್ದಗಳು ಬೇಕಿಲ್ಲ. ವಿಶ್ವವನ್ನೇ ತಬ್ಬುವ ಶ್ರೀಮಾತೆಯವರ ವಾತ್ಸಲ್ಯದ ಬಂಧನದಲ್ಲಿ ಸಿಲುಕಿ ಈ ಶಿಷ್ಯೆಯರು ಕರಗಿ ನೀರಾದರು! ಅಷ್ಟೇ ಅಲ್ಲ, ಶ್ರೀಶಾರದಾದೇವಿಯವರು ಅವರ ಜೊತೆಯಲ್ಲಿ ಸ್ವಲ್ಪ ಉಪಾಹಾರವನ್ನೂ ಸ್ವೀಕರಿಸಿದರು. ಇದನ್ನು ಕಂಡು ಸ್ವಾಮೀಜಿಯವರಿಗಾದ ಆನಂದ ಅಷ್ಟಿಷ್ಟಲ್ಲ. ಕೆಲದಿನಗಳ ಬಳಿಕ ಅವರು ಸ್ವಾಮಿ ರಾಮಕೃಷ್ಣಾನಂದರಿಗೆ ಒಂದು ಪತ್ರದಲ್ಲಿ ಬರೆಯುತ್ತಾರೆ, “ಅಮೆರಿಕ ಹಾಗೂ ಯೂರೋಪಿನಿಂದ ಬಂದ ಮಹಿಳೆಯರು ಶ್ರೀಮಾತೆಯವ ರನ್ನು ನೋಡಲು ಹೋಗಿದ್ದರು. ಏನು ಹೇಳುತ್ತಿ! ಮಾತೆಯವರು ಅವರ ಜೊತೆಯಲ್ಲಿ ಉಪಾಹಾರ ತೆಗೆದುಕೊಂಡರು! ಅದೊಂದು ಅದ್ಭುತವಲ್ಲವೆ?”

ಶ್ರೀಮಾತೆಯವರ ದರ್ಶನದ ಬಳಿಕ ಸ್ವಾಮೀಜಿ ತಮ್ಮ ಶಿಷ್ಯೆಯರನ್ನು ಗೋಪಾಲೇರ್ ಮಾಳ ಬಳಿಗೆ ಕರೆದೊಯ್ದರು. ಶ್ರೀರಾಮಕೃಷ್ಣರ ಪ್ರಸಿದ್ಧ ಬ್ರಾಹ್ಮಣ ಶಿಷ್ಯೆಯಾದ ಈಕೆ ಮಹಾಭಕ್ತೆ; ಬಾಲಗೋಪಾಲನನ್ನು ಸಾಕ್ಷಾತ್ಕರಿಸಿಕೊಂಡವಳು. ಈಕೆ ಕಟ್ಟುನಿಟ್ಟಿನ ಸಂಪ್ರದಾಯಸ್ಥ ಪರಿಸರ ದಲ್ಲೇ ಹುಟ್ಟಿ ಬೆಳೆದವಳು. ಇಂತಹ ಗೋಪಾಲೇರ್ ಮಾಳೂ ಆ ಪಾಶ್ಚಾತ್ಯ ಮಹಿಳೆಯರನ್ನು ಪ್ರೀತಿಯಿಂದ ಆದರಿಸಿದಳು. “ಅವರ ಪವಿತ್ರ ಸ್ಮೃತಿಯು ಅಂದಿನ ದಿನವನ್ನು ಬದುಕಿನ ದೊಡ್ಡ ಹಬ್ಬಗಳಲ್ಲೊಂದಾಗಿ ಮಾಡುತ್ತದೆ” ಎಂದು ನಿವೇದಿತಾ ಬರೆಯುತ್ತಾಳೆ. ಕೆಲದಿನಗಳ ಬಳಿಕ ನಿವೇದಿತಾ ಮತ್ತೊಮ್ಮೆ ಗೋಪಾಲೇರ್ ಮಾಳ ಮನೆಗೆ ಹೋಗಿ ಮೂರು ದಿನ ಆಕೆಯ ಅತಿಥಿಯಾಗಿ ಇಳಿದುಕೊಂಡಳು. ಈ ವಿಷಯಗಳೆಲ್ಲ ಸ್ವಾಮೀಜಿಯವರಿಗೆ ಅತ್ಯಂತ ಸಂತಸ ತಂದಂಥವು. ಸಂಪ್ರದಾಯಸ್ಥ ಸಮಾಜವು ತಮ್ಮ ಪಾಶ್ಚಾತ್ಯ ಶಿಷ್ಯರನ್ನು ಸ್ವೀಕರಿಸುವುದು ಅವರ ಪಾಲಿಗೆ ತುಂಬ ಮಹತ್ವಪೂರ್ಣವಾಗಿತ್ತು. ನಿಜಕ್ಕೂ ಇದೊಂದು ಪರಮಾದ್ಭುತವಲ್ಲದೆ ಮತ್ತೇನು! ಶ್ರೀಮಾತೆಯವರಿರಲಿ, ಗೋಪಾಲೇರ್ ಮಾಳಂತಹ ಪರಮ ಸಂಪ್ರದಾಯನಿಷ್ಠ ಬ್ರಾಹ್ಮಣ ಹೆಂಗಸೇ ಈ ‘ಮ್ಲೇಚ್ಛ’ರೊಂದಿಗೆ ನಿಸ್ಸಂಕೋಚವಾಗಿ ಬೆರೆತ ಮೇಲೆ ಇನ್ಯಾರ ಕುಹಕ ತಾನೆ ನಡೆದೀತು? (ದೀಕ್ಷೆ ನೀಡುವ ತಮ್ಮ ನಿರ್ಧಾರವನ್ನು ಮಾರ್ಗರೆಟ್ಟಳಿಗೆ ಸ್ವಾಮೀಜಿ ತಿಳಿಸಿದುದೂ ಶ್ರೀಮಾತೆಯವರ ದರ್ಶನದ ಬಳಿಕವೇ.)

ಮಾರ್ಗರೆಟ್ಟಳಿಗೆ ದೀಕ್ಷೆ ನೀಡಿ, ಆಕೆಯನ್ನು ‘ನಿವೇದಿತ’ಳನ್ನಾಗಿಸಿದ ನಾಲ್ಕು ದಿನಗಳ ನಂತರ ಸ್ವಾಮೀಜಿ ತಮ್ಮ ಶಿಷ್ಯರಾದ ಸುರೇಂದ್ರನಾಥ ಬೋಸ್ ಹಾಗೂ ಅಜಯ ಹರಿ ಬ್ಯಾನರ್ಜಿ ಇವರಿಗೆ ಸಂನ್ಯಾಸ ದೀಕ್ಷೆ ನೀಡಿದರು. ಇವರ ಸಂನ್ಯಾಸನಾಮಗಳು ಸ್ವಾಮಿ ಸುರೇಶ್ವರಾನಂದ ಹಾಗೂ ಸ್ವಾಮಿ ಸ್ವರೂಪಾನಂದ. ಅಜಯಹರಿ ಸಂನ್ಯಾಸವನ್ನು ಸ್ವೀಕರಿಸಿದ್ದರ ಹಿನ್ನೆಲೆ ತುಂಬ ಸ್ವಾರಸ್ಯಕರ ವಾಗಿದೆ. ಬಾಲ್ಯದಿಂದಲೂ ಈತ ಉಳಿದವರಂತಲ್ಲ. ಬದುಕಿನ ಗಂಭೀರ ಪ್ರಶ್ನೆಗಳು ಆಗಲೇ ಅವನ ಮನಸ್ಸಿನಲ್ಲಿ ಸುತ್ತುತ್ತಿದ್ದುವು. ಒಂದು ದಿನ ಬಾಲಕ ಅಜಯ ಬೀದಿಯಲ್ಲಿ ಹೋಗುತ್ತಿ ದ್ದನು; ಒಬ್ಬಳು ಮುದಿ ಭಿಕ್ಷುಕಿ ಅಂದು ತಾನು ಸಂಪಾದಿಸಿದ ಅನ್ನವನ್ನು ಹಿಡಿದು ತರುತ್ತಿದ್ದಾಗ ಅಕಸ್ಮಾತ್ತಾಗಿ ಆ ಪಾತ್ರೆ ಕೆಳಗೆ ಬಿದ್ದು, ಅದರಲ್ಲಿದ್ದ ಅನ್ನವಷ್ಟೂ ಮಣ್ಣುಪಾಲಾಯಿತು. ಆಗ ಅವಳ ಮುಖದಲ್ಲಿ ಮೂಡಿದ ವೇದನೆಯನ್ನು ಕಂಡ ಈ ಬಾಲಕ ಕೂಗಿಕೊಂಡನಂತೆ–“ದೇವರು ಎನ್ನುವವನೇನಾದರೂ ಇದ್ದರೆ ಅವನು ನಿದ್ರೆ ಮಾಡುತ್ತಿದ್ದಾನೆಯೆ? ಇಂಥದನ್ನೆಲ್ಲ ಅವನೇಕೆ ತಡೆಯುವುದಿಲ್ಲ!” ಎಂದು. ಹೀಗೆ ಪರರಿಗಾಗಿ ಮರುಗುವ, ಪರರಿಗೆ ನೆರವಾಗುವ ಬುದ್ಧಿ ಈತನಲ್ಲಿ ಬಾಲ್ಯದಿಂದಲೂ ಬೆಳೆಯುತ್ತ ಬಂದಿತು. ಈತ ಸಂಸ್ಕೃತ ಸಾಹಿತ್ಯದಲ್ಲೂ ಶಾಸ್ತ್ರಗಳಲ್ಲೂ ಪ್ರವೀಣನಾದ. ಅಜಯನಿಗೆ ಮದುವೆಯೂ ಆಯಿತು. ಆದರೆ ಅವನು ದಾಂಪತ್ಯ ಜೀವನ ನಡೆಸದೆ ಕಟ್ಟುನಿಟ್ಟಾದ ಬ್ರಹ್ಮಚರ್ಯವನ್ನು ಪಾಲಿಸಿಕೊಂಡು ಬಂದ. ದಿನದಿನಕ್ಕೂ ಸಂಸಾರದಿಂದ ಮತ್ತಷ್ಟು ವಿಮುಖನಾದ. ತನ್ನ ಸಹಮಾನವರಿಗೆ ಹೇಗೆ ನೆರವಾದೇನು ಎಂಬ ಒಂದೇ ಚಿಂತೆ ಅವನ ಮನಸ್ಸನ್ನಾವರಿಸಿತು. ಈ ಸಂದರ್ಭದಲ್ಲೇ ಅವನು ಆಗಾಗ ಮಠಕ್ಕೆ ಬರಲಾರಂಭಿಸಿದ. ಸುಮಾರು ನಾಲ್ಕನೆಯ ಸಲ ಬಂದಾಗ ಸ್ವಾಮೀಜಿಯವರ ಭೇಟಿಯಾಯಿತು;ಅವರೊಡನೆ ಬಹಳ ಹೊತ್ತು ಮಾತುಕತೆ ನಡೆಯಿತು. ಸ್ವಾಮೀಜಿಯವರ ಮಾತುಗಳಿಂದ ಅಜಯ ಎಷ್ಟು ಪ್ರಭಾವಿತ ನಾದನೆಂದರೆ ಅಂದೇ ಅವನು ಸರ್ವಸಂಗಪರಿತ್ಯಾಗ ಮಾಡಿ ಆಧ್ಯಾತ್ಮಿಕ ಜೀವನವನ್ನು ಸ್ವೀಕರಿಸು ವುದಾಗಿ ನಿರ್ಧರಿಸಿಬಿಟ್ಟ. ತನ್ನೊಂದಿಗೆ ಬಂದಿದ್ದ ಸ್ನೇಹಿತರ ಹತ್ತಿರ, “ನಾನು ಇಂದಿನಿಂದ ಮಠದಲ್ಲೇ ಇದ್ದು ಬಿಡುತ್ತೇನೆ. ಇನ್ನು ನಾನು ಮನೆಗೆ ಬರುವವನಲ್ಲ ಎಂದು ಮನೆಯವರಿಗೆ ಹೇಳಿಬಿಡಿ” ಎಂದು ತಿಳಿಸಿದ. ಅಜಯನ ಈ ನಿರ್ಧಾರವನ್ನು ಕಂಡು ಅವನ ಸ್ನೇಹಿತರು ದಿಗ್ಭ್ರಾಂತರಾದರು. ಅವನು ತನ್ನ ನಿರ್ಧಾರವನ್ನು ಬದಲಾಯಿಸಲೇ ಇಲ್ಲ. ಮಠದಲ್ಲಿ ಆತ ಕೆಲದಿನಗಳು ಉಳಿದುಕೊಂಡಿರಬಹುದು ಅಷ್ಟೆ; ಆಗಲೇ ಸ್ವಾಮೀಜಿ ಆತನಿಗೆ ತಾವು ಸಂನ್ಯಾಸ ನೀಡುವುದಾಗಿ ತಿಳಿಸಿದರು. ಈ ತರುಣನ ಮೇಲೆ ಅವರಿಗಿದ್ದ ವಿಶ್ವಾಸ ಎಷ್ಟು ಬಲವಾಗಿತ್ತೆಂದರೆ ಎಲ್ಲ ನಿಯಮಗಳನ್ನೂ ಒತ್ತಟ್ಟಿಗಿಟ್ಟು ಈತನಿಗೆ ಸಂನ್ಯಾಸ ದೀಕ್ಷೆಯನ್ನು ನೀಡಿದರು. ಈತನಿಗೆ ಸಂನ್ಯಾಸ ನೀಡಿದ ದಿನ ಸ್ವಾಮೀಜಿ ಅತ್ಯಂತ ಸಂತುಷ್ಟಿಯಿಂದ ತಮ್ಮ ಸೋದರ ಸಂನ್ಯಾಸಿಗಳ ಬಳಿ, “ಇಂದು ನಮಗೊಂದು ದೊಡ್ಡ ಲಾಭವಾಯಿತು” ಎಂದು ಉದ್ಗರಿಸಿದರು. ಮತ್ತೊಮ್ಮೆ ಸ್ವಾಮೀಜಿ ಅವನನ್ನು ಕುರಿತು ಹೇಳುತ್ತಾರೆ, “ಸ್ವರೂಪನಂತಹ ಕರ್ಮಕುಶಲಿಯನ್ನು ಪಡೆಯು ವುದೆಂದರೆ ಸಾವಿರಾರು ಚಿನ್ನದ ಮೊಹರನ್ನು ಪಡೆಯುವುದಕ್ಕಿಂತಲೂ ಹೆಚ್ಚಿನ ಲಾಭವಾದಂತೆ” ಎಂದು. ಕೆಲವೇ ತಿಂಗಳಲ್ಲಿ ಅವರು ಸ್ವರೂಪಾನಂದರನ್ನು ‘ಪ್ರಬುದ್ಧ ಭಾರತ’ ಪತ್ರಿಕೆಯ ಸಂಪಾದಕರನ್ನಾಗಿ ನೇಮಕ ಮಾಡಿದರು. ಅಲ್ಲದೆ ಮರು ವರ್ಷ ಸೇವಿಯರ್ ದಂಪತಿಗಳ ನೆರವಿನಿಂದ ಹಿಮಾಲಯದ ಮಾಯಾವತಿ ಎಂಬಲ್ಲಿ ಅದ್ವೈತಾಶ್ರಮವನ್ನು ಸ್ಥಾಪಿಸಿದಾಗ ಸ್ವರೂ ಪಾನಂದರನ್ನು ಅದರ ಅಧ್ಯಕ್ಷರನ್ನಾಗಿ ನೇಮಿಸಿದರು. ಅವರ ಮೇಲೆ ಸ್ವಾಮೀಜಿಯವರಿಗಿದ್ದ ವಿಶ್ವಾಸ ಅಂಥದು.

ಈ ದಿನಗಳಲ್ಲಿ ಸ್ವಾಮೀಜಿಯವರನ್ನು ಭೇಟಿಯಾದ ಅನೇಕ ಗಣ್ಯವ್ಯಕ್ತಿಗಳಲ್ಲಿ ಒಬ್ಬರೆಂದರೆ ಬೌದ್ಧ ಮಿಷನರಿಗಳಾದ ಧರ್ಮಪಾಲರು. ಇವರು ಅಮೆರಿಕದಿಂದ ಕಲ್ಕತ್ತಕ್ಕೆ ಹಿಂದಿರುಗಿದ ಮೇಲೆ, ವಿವೇಕಾನಂದರು ಅಮೆರಿಕದಲ್ಲಿಸಾಧಿಸಿದ ವಿಜಯವನ್ನು ಪ್ರಶಂಸಿಸಿ ಹಲವೆಡೆ ಭಾಷಣಗಳನ್ನು ನೀಡಿದ್ದರು. ಒಂದು ದಿನ ಇವರು ಸ್ವಾಮೀಜಿಯವರನ್ನು ನೋಡಲು ಮಠಕ್ಕೆ ಬಂದರು. ಸ್ವಲ್ಪ ಹೊತ್ತು ಆತ್ಮೀಯವಾಗಿ ಮಾತುಕತೆ ನಡೆಯಿತು. ಬಳಿಕ ಧರ್ಮಪಾಲರು ತಮಗೆ ಪರಿಚಿತಳಾಗಿದ್ದ ಶ್ರೀಮತಿ ಸಾರಾಳನ್ನು ಭೇಟಿಯಾಗಲಿಚ್ಛಿಸಿದರು. ನಾವು ಹಿಂದೆಯೇ ನೋಡಿದಂತೆ ಆಕೆಯ ಜೋಪಡಿಯಿದ್ದುದು ನೂತನ ಮಠದ ನಿರ್ಮಾಣಕ್ಕಾಗಿ ಕೊಂಡುಕೊಂಡ ಹೊಸ ಜಾಗದಲ್ಲಿ. ಈಗ ಸ್ವಾಮೀಜಿ ಧರ್ಮಪಾಲರನ್ನು ಆ ಜೋಪಡಿಗೆ ಕರೆದೊಯ್ಯಬೇಕು. ಆದರೆ ಹೊರಗೆ ಜಡಿಮಳೆ. ಒಂದು ಗಂಟೆ ಕಾದ ಬಳಿಕ ಸ್ವಾಮೀಜಿ, ಇನ್ನೂ ಕೆಲವು ಸಾಧುಗಳೊಂದಿಗೆ ಧರ್ಮಪಾಲರನ್ನು ಜೋಪಡಿಗೆ ಕರೆದುಕೊಂಡು ಹೊರಟರು. ಆದರೆ ಮಳೆ ಪೂರ್ತಿ ಬಿಟ್ಟಿಲ್ಲ; ಜೊತೆಗೆ ದಾರಿಯೆಲ್ಲ ಕೆಸರು. ಸ್ವಾಮೀಜಿ ಆ ಕೆಸರು ದಾರಿಯಲ್ಲಿ ಹುಡುಗರಂತೆ ಜಾರುತ್ತ ಆಟವಾಡಿಕೊಂಡು ನಡೆದರು. ಅವರಲ್ಲೆಲ್ಲ ಪಾದರಕ್ಷೆಯಿದ್ದವರೆಂದರೆ ಧರ್ಮಪಾಲರೊಬ್ಬರೇ. ಮಠದ ಆವರಣದಲ್ಲಂತೂ ಇನ್ನೂ ಕೆಸರು. ಏಕೆಂದರೆ ಅಲ್ಲಿನ ನೆಲವನ್ನು ಸಮತಟ್ಟುಗೊಳಿಸುವುದಕ್ಕಾಗಿ ಅಗೆತ ನಡೆಯು ತ್ತಿತ್ತು. ಮಳೆಯಲ್ಲಿ ನೆನೆಯುತ್ತ ಆ ಕೆಸರಿನಲ್ಲಿ ಜಾರುತ್ತ ಆಗಾಗ ಸ್ವಾಮೀಜಿ ಮಕ್ಕಳಂತೆ ಗಟ್ಟಿಯಾಗಿ ನಗುತ್ತಿದ್ದರು. ಆದರೆ ಒಂದೆಡೆ ಧರ್ಮಪಾಲರ ಒಂದು ಕಾಲು ಮೆತ್ತನೆಯ ಮಣ್ಣಿ ನೊಳಗೆ ಹೂತುಹೋಗಿಬಿಟ್ಟಿತು. ಎಷ್ಟು ಎಳೆದರೂ ಕಾಲು ಹೊರಬರಲಿಲ್ಲ. ಅವರ ಅವಸ್ಥೆಯನ್ನು ಕಂಡು ಸ್ವಾಮೀಜಿ ತಮ್ಮ ಭುಜವನ್ನು ಅವರಿಗೆ ಆಸರೆಯಾಗಿ ಕೊಟ್ಟರು. ಮತ್ತು ಅವರನ್ನು ಹಿಡಿದುಕೊಂಡು ಮೆಲ್ಲನೆ ಮೇಲಕ್ಕೆತ್ತಿದರು. ಬಳಿಕ ಉಳಿದ ದಾರಿಯನ್ನು ಇಬ್ಬರೂ ಕೈಕೈ ಹಿಡಿದುಕೊಂಡು ನಗುತ್ತ ಮುನ್ನಡೆದರು.

ಜೋಪಡಿಯನ್ನು ತಲುಪಿದ ಮೇಲೆ ಎಲ್ಲರೂ ತಮ್ಮ ಕಾಲುಗಳನ್ನು ತೊಳೆದುಕೊಳ್ಳಲು ಹೋದರು. ಧರ್ಮಪಾಲರು ಒಂದು ತಂಬಿಗೆಯನ್ನು ತೆಗೆದುಕೊಂಡು ತಮ್ಮ ಕಾಲಿಗೆ ನೀರು ಸುರಿದುಕೊಳ್ಳಲು ಹೊರಟರು; ಆಗ ಸ್ವಾಮೀಜಿ ಬಂದು ತಕ್ಷಣ ಅವರ ಕೈಯಿಂದ ತಂಬಿಗೆಯನ್ನು ಕಿತ್ತುಕೊಂಡು, “ನೋಡಿ, ನೀವು ನನ್ನ ಅತಿಥಿಗಳು. ಆದ್ದರಿಂದ ನಾನೇ ನಿಮ್ಮ ಸೇವೆ ಮಾಡಬೇಕು” ಎನ್ನುತ್ತ ಅವರ ಕಾಲುಗಳನ್ನು ತೊಳೆಯಲು ಸಿದ್ಧರಾದರು. ಆಗ ಧರ್ಮಪಾಲರು ಗಟ್ಟಿಯಾಗಿ ಕೂಗುತ್ತ “ಅಯ್ಯಯ್ಯೋ! ನೀವು ನನ್ನ ಕಾಲುಗಳನ್ನು ತೊಳೆಯುವುದೆಂದರೇನು! ದಯವಿಟ್ಟು ಹಾಗೆ ಮಾಡಬೇಡಿ” ಎಂದು ಪ್ರತಿಭಟಿಸಿದರು. ಆದರೆ ಸ್ವಾಮೀಜಿ ಅದನ್ನು ಕಿವಿಯ ಮೇಲೆ ಹಾಕಿಕೊಂಡರೆ ತಾನೆ! ಭಾರತದಲ್ಲಿ ಇನ್ನೊಬ್ಬರ ಪಾದಗಳನ್ನು ತೊಳೆಯುವುದೆಂದರೆ ಅದು ಅತ್ಯಂತ ನಮ್ರತೆಯನ್ನು ಸೂಚಿಸುವ ಸೇವೆ. ಇಲ್ಲಿ ಸ್ವಾಮೀಜಿ ಈ ಬಗೆಯ ಪಾದಸೇವೆಯಲ್ಲಿ ತೊಡಗಿದ್ದುದನ್ನು ನೋಡಿದವರಿಗೆಲ್ಲ ಅವರ ಈ ವಿನಯವನ್ನು ಕಂಡು ಅತ್ಯಾಶ್ಚರ್ಯ!

ಮಾರ್ಗರೆಟ್ಟಳು ಭಾರತಕ್ಕೆ ಬಂದಾಗಿನಿಂದಲೂ ಸ್ವಾಮೀಜಿಯವರ ಮನಸ್ಸಿನಲ್ಲಿ ಒಂದು ಮುಖ್ಯ ಆಲೋಚನೆ ಕೆಲಸ ಮಾಡುತ್ತಿತ್ತು–ಅವಳು ಯಾವ ಭಾರತವನ್ನು ತನ್ನ ನಾಡಾಗಿ ಸ್ವೀಕರಿಸಿದ್ದಾಳೋ ಆ ನಾಡಿಗೆ ಅವಳಿಂದ ಸೇವೆ ಸಲ್ಲುವಂತೆ ಅವಳಿಗೆ ತರಬೇತಿ ನೀಡಬೇಕು ಎಂಬುದು. ಅವಳು ಭಾರತಕ್ಕೆ ತಲುಪಿದ ಕೆಲದಿನಗಳಲ್ಲೇ ಅವಳಿಗೆ ಬಂಗಾಳೀ ಭಾಷೆಯನ್ನು ಕಲಿಸುವುದಕ್ಕಾಗಿ ಶ್ರೀರಾಮಕೃಷ್ಣರ ಶಿಷ್ಯರಾದ ಮಹೇಂದ್ರನಾಥ ಗುಪ್ತರನ್ನು (ಮಾಸ್ಟರ್ ಮಹಾ ಶಯ) ನೇಮಕ ಮಾಡಿದರು. ಸಂಪೂರ್ಣ ಐರೋಪ್ಯ ನಾಗರಿಕತೆಯಿಂದ ಕೂಡಿದ ಮಾರ್ಗರೆಟ್ ಈಗ ಭಾರತದಲ್ಲಿ ಭಾರತೀಯ ರೀತಿಯಲ್ಲೇ ಸೇವೆ ಸಲ್ಲಿಸಬೇಕಾದರೆ ಅವಳು ಭಾರತೀಯ ರೀತಿನೀತಿಗಳನ್ನು, ಸಂಪ್ರದಾಯಗಳನ್ನು, ನಡೆನುಡಿಗಳನ್ನು ವಿವರವಾಗಿ ಅರಿತಿರಬೇಕಾಗುತ್ತದೆ. ಅಷ್ಟೇ ಅಲ್ಲ, ಅವುಗಳೆಲ್ಲವನ್ನೂ ಅವಳು ಮೈಗೂಡಿಸಿಕೊಳ್ಳಬೇಕಾಗುತ್ತದೆ. ಒಬ್ಬ ನಿಷ್ಠಾವಂತ ಹಿಂದೂ ಸ್ತ್ರೀಯಂತೆಯೇ ಜೀವನ ನಡೆಸಬೇಕಾಗುತ್ತದೆ. ಆದ್ದರಿಂದ ಸ್ವಾಮೀಜಿ ಅವಳೊಂದಿಗೆ ಮಾತನಾಡುವಾಗಲೆಲ್ಲ ಈ ವಿಷಯಗಳನ್ನು ಅವಳ ಬುದ್ಧಿಗೆ ತುಂಬುತ್ತಿದ್ದರು. ನಿವೇದಿತಾ ಹಿಂದೂ ಮಹಿಳೆಯರ ಶಿಕ್ಷಣದ ಹೊಣೆಯನ್ನು ಹೊರಬೇಕಾದವಳು. ಆದ್ದರಿಂದ ಅವಳು ಅತ್ಯಂತ ಪರಿಶುದ್ಧವಾದ ಬ್ರಹ್ಮಚರ್ಯ ಜೀವನವನ್ನು ನಡೆಸುತ್ತ, ಈ ಮಹಿಳೆಯರ ಪಾಲಿಗೆ ಆದರ್ಶವಾಗಿ ಪರಿಣಮಿಸಬೇಕೆಂದು ಸ್ವಾಮೀಜಿ ಆಕೆಗೆ ತಿಳಿಸಿದರು. ಬ್ರಾಹ್ಮಣ ವಿಧವೆಯೊಬ್ಬಳು ಪಾಲಿಸು ವಂತಹ ಕಟ್ಟು ನಿಟ್ಟಾದ ನಿಯಮಗಳನ್ನು ಪಾಲಿಸಬೇಕೆಂದು ಅವಳಿಗೆ ತಿಳಿಸಿದರು. ಆದರೆ ಒಬ್ಬಳು ವಿಧವೆ ಕೇವಲ ತನ್ನ ಕುಟುಂಬದವರ ಸೇವೆಯಲ್ಲಿ ನಿರತಳಾಗಿದ್ದರೆ, ನಿವೇದಿತೆ ಸಮಸ್ತ ಭಾರತೀ ಯರ ಸೇವೆಯಲ್ಲಿ ನಿರತಳಾಗಬೇಕು. ಆದರೆ ಇದು ಅಷ್ಟೊಂದು ಸುಲಭವಾಗಿ ಸಾಧ್ಯವಾಗುವ ಮಾತೇನಲ್ಲ. ಆದ್ದರಿಂದ ಸ್ವಾಮೀಜಿ ಆಕೆಗೆ ಹೇಳಿದರು, “ನೀನು ನಿನ್ನ ಆಲೋಚನೆ-ಭಾವನೆ- ಅಭ್ಯಾಸಗಳನ್ನು ಹಿಂದೂ ಸಂಪ್ರದಾಯಕ್ಕೆ ಅನುಗುಣವಾಗಿ ರೂಪಿಸಿಕೊಳ್ಳಲು ಸಿದ್ಧಳಾಗಬೇಕು. ನಿನ್ನ ಒಳ-ಹೊರ ಜೀವನವು ಭಾರತದ ಓರ್ವ ಸಂಪ್ರದಾಯಸ್ಥ ಬ್ರಹ್ಮಚಾರಿಣಿಯ ಜೀವನ ದಂತಿರಬೇಕು. ನೀನು ಸಾಕಷ್ಟು ಪ್ರಾಮಾಣಿಕಳಾಗಿದ್ದ ಪಕ್ಷದಲ್ಲಿ ಇದನ್ನೆಲ್ಲ ಸಾಧಿಸುವ ಕ್ರಮ ನಿನಗೇ ಹೊಳೆಯುತ್ತದೆ. ಆದರೆ ನೀನು ನಿನ್ನ ಹಳೆಯದನ್ನೆಲ್ಲ ಮರೆಯುವ ಪ್ರಯತ್ನ ಮಾಡಬೇಕು, ಮರೆತುಬಿಡಬೇಕು. ನಿನ್ನ ಹಿಂದಿನ ಅಭ್ಯಾಸಗಳ ನೆನಪನ್ನೇ ಅಳಿಸಿಕೊಂಡುಬಿಡಬೇಕು.” ನಿವೇದಿತಾ ಯಾವ ಸ್ಥಾನಕ್ಕೆ ಏರಬೇಕಾಗಿದೆಯೋ, ಯಾವ ಮನಃಸ್ಥಿತಿಯನ್ನು ಬೆಳೆಸಿಕೊಳ್ಳಬೇಕಾಗಿದೆಯೋ, ಅದಕ್ಕೆ ಈ ಬಗೆಯ ಶಿಸ್ತು, ಈ ಬಗೆಯ ಅಭ್ಯಾಸ ಅತ್ಯಾವಶ್ಯಕ. ಅಲ್ಲದೆ ಸ್ವಾಮೀಜಿ ಆಕೆಗೆ ಇನ್ನೊಂದು ಮಾತು ಹೇಳುತ್ತಾರೆ, “ಇಲ್ಲಿನ ಜನಗಳ ಕೆಲವು ಭಾವನೆಗಳು-ಆಚರಣೆಗಳು ನಿನಗೆ ಎಷ್ಟೇ ಒರಟಾಗಿ, ವಿಚಿತ್ರವಾಗಿ ಕಾಣಬಹುದು. ಆದರೂ ನೀನು ಅವುಗಳನ್ನು ತುಚ್ಛೀಕರಿಸು ವಂತಿಲ್ಲ ಅಥವಾ ಕಡೆಗಣಿಸುವಂತಿಲ್ಲ. ಅವುಗಳನ್ನೂ ನೀನು ಎಚ್ಚರಿಕೆಯಿಂದ ಗಮನಿಸಿ ಅರಿತು ಕೊಳ್ಳಬೇಕು, ಅರಿತು ಗೌರವಿಸಬೇಕು. ನಾವು ಪ್ರತಿಯೊಬ್ಬರಿಗೂ ಅವರವರ ಸಂಪ್ರದಾಯದ ಧಾಟಿಯಲ್ಲೇ ತಿಳಿಸಿ ಹೇಳಬೇಕು.”

ಸ್ವಾಮೀಜಿಯವರು ತಮ್ಮ ನಾಡಿನ, ತಮ್ಮ ಜನರ ಸಂಸ್ಕೃತಿ-ಸಂಪ್ರದಾಯಗಳ ಗರಿಮೆಯನ್ನು ಯಾವುದೇ ಅಕಾರಣ ಟೀಕೆಯಿಂದ ರಕ್ಷಿಸಲು ಸದಾ ಸಿದ್ಧ. ತಮ್ಮ ದೇಶಬಾಂಧವರಿಗೆ ಸಂಬಂಧಿ ಸಿದ ಯಾವುದನ್ನೇ ಆಗಲಿ–ತೆರೆದ ಮನಸ್ಸಿನಿಂದ, ಸಹೃದಯತೆಯಿಂದ ಸ್ವೀಕರಿಸಿದರೋ ಸರಿ;ಇಲ್ಲದೆ ಹೋದರೆ, ಅದಕ್ಕೆ ವಿರುದ್ಧವಾಡಿದವರನ್ನು ಸದೆಬಡಿಯಲು ಸದಾ ಸಿದ್ಧ. ಈ ವಿಷಯ ದಲ್ಲಿ ಸ್ವತಃ ಅವರ ಶಿಷ್ಯರಿಗೂ ವಿನಾಯಿತಿಯಿಲ್ಲ. ಭಾರತವನ್ನು ಆಧ್ಯಾತ್ಮಿಕ ದೃಷ್ಟಿಕೋನದಿಂದ ನೋಡಬೇಕೆಂಬುದು ಸ್ವಾಮೀಜಿಯವರ ವಾದ. ಭಾರತವು ಅವನತಿ ಹೊಂದಿರುವ ವಿಷಯ ಅವರಿಗೆ ಚೆನ್ನಾಗಿ ತಿಳಿದಿರುವಂಥದೇ; ಆದರೆ ಹಿಂದೂ ಧರ್ಮವು ಸತ್ತುಹೋಗಿದೆ ಎನ್ನುವ ಧಾಟಿಯಲ್ಲಿ ಯಾರಾದರೂ ಮಾತನಾಡಿದರೆ ಅದನ್ನು ಅವರು ಉಗ್ರವಾಗಿ ಖಂಡಿಸುತ್ತಿದ್ದರು. ಏಕೆಂದರೆ, ‘ನವತಾರುಣ್ಯದಿಂದ ಕೂಡಿದ ಪ್ರಗತಿಶೀಲ ಜನಾಂಗ ಮಾತ್ರವೇ ಪರಧರ್ಮಗಳ- ಪರರಾಷ್ಟ್ರಗಳ ವಿನೂತನ ಭಾವನೆಗಳನ್ನು ಸ್ವೀಕರಿಸಬಲ್ಲುದು, ಅರಗಿಸಿಕೊಳ್ಳಬಲ್ಲುದು. ಭಾರತವು ತನ್ನಲ್ಲಿ ಈ ಸಾಮರ್ಥ್ಯವಿರುವುದನ್ನು ತೋರಿಸಿಕೊಟ್ಟಿದೆ. ಆದ್ದರಿಂದ ಅದು ನಿರ್ವೀರ್ಯವಾಗಿಲ್ಲ, ನಿರ್ಜೀವವಾಗಿಲ್ಲ’ ಎಂಬುದು ಸ್ವಾಮೀಜಿಯವರ ವಾದ. ಹಿಂದೂಧರ್ಮದ ಆದರ್ಶಗಳನ್ನು- ಮೌಲ್ಯಗಳನ್ನು ಅಧ್ಯಯನ ಮಾಡಿದಾಗ, ಅದು ತಾರುಣ್ಯಭರಿತವಾದ, ವೀರ್ಯವತ್ತರವಾದ ಹಾಗೂ ಅದ್ಭುತ ಸಾಧ್ಯತೆಗಳಿಂದ ಕೂಡಿದ ಧರ್ಮವೆಂಬುದು ಸ್ಪಷ್ಟವಾಗುತ್ತದೆ ಎನ್ನುತ್ತಾರೆ ಸ್ವಾಮೀಜಿ. ತಮ್ಮ ಪಾಶ್ಚಾತ್ಯ ಶಿಷ್ಯೆಯರಿಗೆ ಭಾರತದ ಪ್ರತಿಯೊಂದು ಸಂಪ್ರದಾಯದ ಹಿನ್ನೆಲೆಯ ಲ್ಲಿರುವ ಅರ್ಥವನ್ನು ತೆರೆದು ತೋರಿಸುತ್ತಾರೆ. ಭಾರತವು ಬಡರಾಷ್ಟ್ರವಾಗಿರಬಹುದು; ಆದರೆ ಹಿಂದೂಧರ್ಮದಲ್ಲಿ ತ್ಯಾಗವೇ ಧರ್ಮದ ಮೊದಲ ಮೆಟ್ಟಿಲು. ಆದ್ದರಿಂದ ಇಲ್ಲಿ ಬಡತನ ವೊಂದು ಪಾಪವಲ್ಲ. ಅಲ್ಲದೆ ಭಾರತದಲ್ಲಿ ಇತರ ರಾಷ್ಟ್ರಗಳಂತೆ ಬಡತನದೊಂದಿಗೆ ದುಷ್ಟತನ ಸೇರಿಕೊಂಡಿಲ್ಲ ಎಂಬುದನ್ನು ಮನಗಾಣಿಸುತ್ತಿದ್ದರು. ತಮ್ಮ ಪಾಶ್ಚಾತ್ಯ ಅನುಯಾಯಿಗಳು ಭಾರತದ ಸಮಸ್ಯೆಗಳನ್ನು ನಿಷ್ಪಕ್ಷಪಾತ ದೃಷ್ಟಿಯಿಂದ ಅಧ್ಯಯನ ಮಾಡಬೇಕೆಂಬುದು ಅವರ ಪ್ರಧಾನವಾದ ಕಳಕಳಿ. ಆ ಪಾಶ್ಚಾತ್ಯ ಶಿಷ್ಯೆಯರು–ಮುಖ್ಯವಾಗಿ ನಿವೇದಿತಾ–ಕೇವಲ ಭಾರತದ ಗತಕಾಲದ ವೈಭವವನ್ನಾಗಲಿ, ತತ್ತ್ವಶಾಸ್ತ್ರದ ಔನ್ನತ್ಯವನ್ನಾಗಲಿ ಕಾಣಬೇಕೆಂದಲ್ಲ; ಬದಲಾಗಿ ಅದರ ಪ್ರಸ್ತುತ ಸಮಸ್ಯೆಗಳೆಡೆಗೆ ಗಮನ ಹರಿಸಿ, ಪಾಶ್ಚಾತ್ಯ ವಿಜ್ಞಾನದ ಹಾಗೂ ಪಾಶ್ಚಾತ್ಯ ವಿಧಾನಗಳ ನೆರವಿನಿಂದ ಈ ಸಮಸ್ಯೆಗಳಿಗೆ ಪರಿಹಾರವನ್ನು ಕಂಡುಹಿಡಿಯುವಂತಾಗಬೇಕು ಎಂಬುದು ಅವರ ಇಂಗಿತ.

ತಮ್ಮ ಪಾಶ್ಚಾತ್ಯ-ಭಾರತೀಯ ಶಿಷ್ಯರ ನಡುವೆ ಮಧುರ-ಸಹಜ ಸಂಬಂಧವನ್ನು ಬೆಸೆಯುವು ದಕ್ಕಾಗಿ ಕೆಲವೊಮ್ಮೆ ಸ್ವಾಮೀಜಿಯವರು ಅತ್ಯಂತ ಅಸಾಂಪ್ರದಾಯಿಕವಾದ ಏನನ್ನಾದರೂ ಮಾಡಿ ತೋರಿಸುತ್ತಿದ್ದರು. ಆ ರಾಷ್ಟ್ರದವರನ್ನು–ಈ ರಾಷ್ಟ್ರದವರನ್ನು ಬೇರ್ಪಡಿಸುವುದು ಯಾವುದು ಎಂದರೆ ಅನೂಚಾನವಾಗಿ ಬಂದ ಅಭ್ಯಾಸ-ಸಂಪ್ರದಾಯಗಳೇ ತಾನೆ? ಮತ್ತು ಈ ಕಾರಣ ದಿಂದಲೇ ಅಲ್ಲವೆ ಹಿಂದೂಗಳು ಪಾಶ್ಚಾತ್ಯರನ್ನು ಮ್ಲೇಚ್ಛರೆಂದು ಕರೆದು ಅವರೊಂದಿಗೆ ಬೆರೆಯ ದಿರುವುದು? ಆದ್ದರಿಂದ ಸ್ವಾಮೀಜಿ ತಮ್ಮ ಪಾಶ್ಚಾತ್ಯ ಶಿಷ್ಯೆಯರೊಂದಿಗೆ ಭಾರತೀಯ ಶಿಷ್ಯರು ಕೂಡಿ ಬಾಳಲು ಅನುಕೂಲವಾಗುವಂತೆ ಹಳೆಯ ಸಂಪ್ರದಾಯದ ಕಟ್ಟನ್ನು ಸ್ವಲ್ಪ ಸಡಿಲಿಸು ವುದಿತ್ತು. ಉದಾಹರಣೆಗೆ, ಅವರು ಎಲ್ಲರೆದುರಿಗೆ “ನನ್ನ ಪಾಶ್ಚಾತ್ಯ ಶಿಷ್ಯರು ನಿಜವಾದ ಬ್ರಾಹ್ಮಣರು ಮತ್ತು ಕ್ಷತ್ರಿಯರು!” ಎಂದು ಸಾರುತ್ತಿದ್ದರು; ಆ ಪಾಶ್ಚಾತ್ಯ ಮಹಿಳೆಯರಿಂದಲೇ ಅಡಿಗೆ ಮಾಡಿಸಿ ಅವರೊಂದಿಗೇ ಕುಳಿತು ಅದನ್ನು ಊಟ ಮಾಡುತ್ತಿದ್ದರು, ಅಲ್ಲದೆ ತಮ್ಮ ಸಂನ್ಯಾಸೀ ಸೋದರರೂ ಅದನ್ನು ಸ್ವೀಕರಿಸುವಂತೆ ಮಾಡುತ್ತಿದ್ದರು. ಇದನ್ನೆಲ್ಲ ಕೇಳಿದರೆ, ಇಂದಿನ ಆಧುನಿಕ ನಾಗರಿಕರಾದ ನಮಗೆ ಏನೂ ಅನ್ನಿಸದಿರಬಹುದು. ಆದರೆ ಅಂದಿನ ಕಾಲದ ಸಂಪ್ರ ದಾಯಪರ ಭಾರತೀಯ ಪರಿಸ್ಥಿತಿಯನ್ನು ಭಾವಿಸಿ ನೋಡಬೇಕು. ಇಂದಿನ ಕಾಲದಲ್ಲೂ ಅಂತಹ ಮಡಿವಂತ ಜನರು, ಮತ್ತು ಅವರಿಗಿಂತ ಮಡಿವಂತರಾದ ಸಂನ್ಯಾಸಿಗಳು ಇಲ್ಲವೆಂದಲ್ಲ. ಹೀಗಿರುವಾಗ ಸ್ವಾಮಿ ವಿವೇಕಾನಂದರು ತಮ್ಮ ಪಾಶ್ಚಾತ್ಯ ಶಿಷ್ಯೆಯರಿಂದ ಅಡಿಗೆ ಮಾಡಿಸಿ ತಾವೂ ಉಂಡದ್ದಲ್ಲದೆ ತಮ್ಮ ಸೋದರ ಸಂನ್ಯಾಸಿಗಳಿಗೂ ಉಣಬಡಿಸಿದರೆಂದರೆ ಅಂದಿನ ಕಾಲದ ಸಂಪ್ರದಾಯಸ್ಥ ಸಂನ್ಯಾಸಿಗಳು ಹಾಗೂ ಬ್ರಾಹ್ಮಣರು ಎಷ್ಟು ಮೂಗು ಮುರಿದಿರಬಹುದೆಂಬು ದನ್ನು ಊಹಿಸಬಹುದು. ಆದರೆ ಸ್ವಾಮೀಜಿಯವರು ಯುಗಪ್ರವರ್ತಕರು; ನೂತನ ಜನಾಂಗ ವೊಂದನ್ನು ಸೃಷ್ಟಿಸಹೊರಟವರು. ಆದ್ದರಿಂದಲೇ ಅವರು ಎಂತಹ ವಾಗ್ಬಾಣಗಳು ತೂರಿ ಬಂದರೂ ಅಲಕ್ಷಿಸಿದರು. ಲೋಕಹಿತವನ್ನು ಸಾಧಿಸಬೇಕೆಂಬ ಪ್ರಾಮಾಣಿಕ-ಧೀರ ಸತ್ಸಂಕಲ್ಪವು ಅವರ ಹೃದಯವನ್ನು ತುಂಬಿದ್ದರಿಂದಲೇ ಇಂತಹ ಅತಿಕಷ್ಟದ ಕಾರ್ಯವನ್ನು ಹಿಡಿದು ಮಾಡಲು ಅವರು ಸಮರ್ಥರಾದರು.

ಆದರೆ ತಮ್ಮ ಪಾಶ್ಚಾತ್ಯ ಶಿಷ್ಯೆಯರನ್ನು ಭಾರತೀಯ ರೀತಿನೀತಿಗಳಿಗೆ ಒಗ್ಗಿಸುವ ಪ್ರಯತ್ನದಲ್ಲಿ ಸ್ವಾಮೀಜಿಯವರು ಅವರ ವೈಯಕ್ತಿಕ ಸಂಸ್ಕಾರಗಳನ್ನು ಹಾಗೂ ಮನೋಭಾವಗಳನ್ನು ಮರೆತಿರ ಲಿಲ್ಲವೆಂಬುದನ್ನು ಹೇಳಬೇಕಾಗಿಯೇ ಇಲ್ಲ. ಏಕೆಂದರೆ ಹೀಗೆ ಅವರವರ ವ್ಯಕ್ತಿತ್ವ-ಸ್ವಭಾವಗಳಿಗೆ ಧಕ್ಕೆಯಾಗುವಷ್ಟು ವಿರುದ್ಧವಾಗಿ ಹೋದಲ್ಲಿ ಅಪಾಯ ಅನಿವಾರ್ಯವೆಂಬುದು ಸ್ವಾಮೀಜಿ ಯವರಿಗೆ ತಿಳಿಯದ ವಿಷಯವಲ್ಲ. ಅಲ್ಲದೆ ತಮ್ಮ ಶಿಷ್ಯರ ವೈಯಕ್ತಿಕ ಸ್ವಾತಂತ್ರ್ಯದಲ್ಲಿ ಅನಾ ವಶ್ಯಕ ಪ್ರವೇಶ ಮಾಡುವುದು ಅವರ ರೀತಿಯಲ್ಲ. ತಾವಾಗಿಯೇ ವಿಷಯಗಳನ್ನೆಲ್ಲ ಗಮನಿಸು ವಂತೆ,–ತಪ್ಪು ಮಾಡಿಯಾದರೂ ಸರಿಯೆ–ಕೆಲವೊಂದು ಅನುಭವಗಳನ್ನು ಪಡೆದುಕೊಳ್ಳು ವಂತೆ, ಅವರಿಗೆಲ್ಲ ಸ್ವಾಮೀಜಿ ಹೇಳುತ್ತಿದ್ದರು. ಆದರೆ ಎಷ್ಟೇ ಸ್ವಾತಂತ್ರ್ಯ ಕೊಟ್ಟರೂ ಕೆಲವೊಮ್ಮೆ ಅವರ ಹಿತಕ್ಕಾಗಿಯೇ ಕೆಲವು ನಿರ್ಬಂಧಗಳನ್ನು ಹಾಕಬೇಕಾಗುತ್ತಿತ್ತು. ಸ್ವಾಮೀಜಿ ತಮ್ಮ ಶಿಷ್ಯೆ ಯರಿಗೆ ಹೇಳುತ್ತಿದ್ದರು, “ನೀವೇ ಹೋರಾಡಿ, ಕಷ್ಟಪಟ್ಟು ಕಲಿತುಕೊಳ್ಳಿ. ಆದರೆ ಯಾವುದೇ ಬಗೆಯ ಭಾವೋದ್ವೇಗಕ್ಕೆ ಬಲಿಯಾಗದೆ ಹೋರಾಡಿ. ನೆನಪಿಡಿ, ಇನ್ನು ನಿಮ್ಮ ಪಾಲಿಗೆ ಯಾವ ಸ್ಥಾನಮಾನವೂ ಇಲ್ಲ! ಪ್ರಪಂಚದ ಯಾವ ವಿಲಾಸವೂ ಇಲ್ಲ! ಅವನ್ನೆಲ್ಲ ಬೇರುಸಮೇತ ಕಿತ್ತು ಬಿಸುಡಬೇಕು. ಅವೆಲ್ಲ ಕೇವಲ ಇಂದ್ರಿಯ ಚಾಪಲ್ಯದಿಂದ ಹುಟ್ಟಿಕೊಂಡಂಥವು. ಶಬ್ದ ಸ್ಪರ್ಶ ರೂಪ ರಸ ಗಂಧಗಳ ಆಕಾರಗಳಿಂದ ಅವು ನಿಮ್ಮ ಬಳಿಗೆ ಬರುತ್ತವೆ. ಕತ್ತರಿಸಿ ಹಾಕಿ ಅವನ್ನು! ಅವುಗಳ ಮೇಲೆ ಜುಗುಪ್ಸೆ ತಾಳಲು ಕಲಿಯಿರಿ. ಅವು ಹಾಲಾಹಲ ವಿಷ!”

ಹೀಗೆ ಸ್ವಾಮೀಜಿ ೧೮೯೮ ರ ವರ್ಷವಿಡೀ ತಮ್ಮ ಪಾಶ್ಚಾತ್ಯ ಶಿಷ್ಯೆಯರಿಗೆ ತೀವ್ರತರ ಶಿಕ್ಷಣ ನೀಡಿದರು. ಈ ಶಿಕ್ಷಣದ ಅವಧಿಯು ಹಾಸ್ಯದ ಕ್ಷಣಗಳಿಂದಲೂ ಕೂಡಿತ್ತು. ಗಾಂಭೀರ್ಯದ ಮುಹೂರ್ತಗಳಿಂದಲೂ ಕೂಡಿತ್ತು. ಆದರೆ ಪೂರ್ಣಕ್ಕೆ ಪೂರ್ಣ ಬೇರೆಯೇ ಸಂಸ್ಕಾರಗಳುಳ್ಳ ಆ ಪಾಶ್ಚಾತ್ಯ ಶಿಷ್ಯೆಯರನ್ನು ಸಂಪೂರ್ಣ ನೂತನ ಸಂಸ್ಕೃತಿ ಸಂಪ್ರದಾಯಗಳಿಗೆ ಹೊಂದಿಕೊಳ್ಳು ವಂತೆ ಶಿಕ್ಷಣ ನೀಡುವ ಕೆಲಸ ಸುಲಭ ಸಾಧಾರಣದ್ದಾಗಿರಲಿಲ್ಲ. ಹಾಗೆಯೇ, ಭಾರತೀಯ ನಡವಳಿಕೆಗಳಿಗೆ ಹಾಗೂ ಆಹಾರ ಕ್ರಮಕ್ಕೆ ಹೊಂದಿಕೊಳ್ಳುವ ಕೆಲಸ ಆ ಶಿಷ್ಯೆಯರಿಗೂ ಕಷ್ಟಕರ ವಾಗಿತ್ತು, ಮುಜುಗರದ್ದಾಗಿತ್ತು. ಆದ್ದರಿಂದ ಆ ಪ್ರಯತ್ನದಲ್ಲಿ ಅನೇಕ ತಪ್ಪುಗಳೂ ಘಟಿಸಿ ದುವು. ಆದರೆ ಸ್ವಾಮೀಜಿ ಯಾವಾಗಲೂ ತಾಳ್ಮೆಯಿಂದ ಅವುಗಳನ್ನು ತಿದ್ದಿ ಸರಿಪಡಿಸಿದರು. ಅಂತೂ ಸ್ವಾಮೀಜಿ ಅವರನ್ನೆಲ್ಲ ಹಿಂದೂಧರ್ಮದ ಹಾಗೂ ಭಾರತದ ಘನತೆ ಹಿರಿಮೆಗಳನ್ನು ಪ್ರಸಾರ ಮಾಡಬಲ್ಲ ಧರ್ಮದೂತರನ್ನಾಗಿ ರೂಪಿಸುವಲ್ಲಿ ಸಮರ್ಥರಾದರೆನ್ನಬಹುದು. ಇವರಲ್ಲಿ ಕೆಲವರು ಸಾರ್ವಜನಿಕವಾಗಿ ಅದನ್ನು ಮಾಡಬಲ್ಲವರಾದರೆ ಇನ್ನು ಕೆಲವರು ಶಾಂತರೀತಿಯಲ್ಲಿ ಮಾಡಬಲ್ಲವರಾದರು. ಒಮ್ಮೆ ಮಿಸ್ ಮೆಕ್​ಲಾಡ್ ಕೇಳಿದಳು, “ಸ್ವಾಮೀಜಿ, ನಾನು ನಿಮಗೆ ಅತ್ಯುತ್ಕೃಷ್ಟವಾದ ರೀತಿಯಲ್ಲಿ ಹೇಗೆ ನೆರವಾಗಬಲ್ಲೆ?” ಸ್ವಾಮೀಜಿ ತಕ್ಷಣ ಉತ್ತರಿಸಿದರು, “ಭಾರತವನ್ನು ಪ್ರೀತಿಸು.” ಸ್ವಾಮೀಜಿಯವರ ಈ ಮಾತಿನಲ್ಲಿ, ಸ್ವಾರ್ಥದ ಕಿಲುಬು ಲವಲೇಶವೂ ಇಲ್ಲದ ರಾಷ್ಟ್ರಪ್ರೇಮವನ್ನು ಕಾಣಬಹುದಾಗಿದೆ. ತಮ್ಮ ಪ್ರೀತಿಪಾತ್ರರಾದ ಶಿಷ್ಯರಲ್ಲಿ ಸ್ವಾಮೀಜಿ ಯವರ ಕಳಕಳಿಯ ಕೋರಿಕೆ–“ಭಾರತವನ್ನು ಪ್ರೀತಿಸು.” ಭಾರತವನ್ನು ಪ್ರೀತಿಸಿದರೆ ತಮ್ಮನ್ನು ಪ್ರೀತಿಸಿದಂತೆ, ಭಾರತಕ್ಕೆ ನೆರವಾದರೆ ತಮಗೆ ನೆರವಾದಂತೆ ಎಂಬುದು ಸ್ವಾಮೀಜಿಯವರ ಭಾವ. ಇದು, ಅವರು ಭಾರತದೊಂದಿಗೆ ತಮ್ಮನ್ನು ತಾವು ತಾದಾತ್ಮ್ಯಗೊಳಿಸಿಕೊಂಡದ್ದರ ಸ್ಪಷ್ಟಲಕ್ಷಣ.

ನಿರಂತರ ಕಾರ್ಯಚಟುವಟಿಕೆಗಳಿಂದಾಗಿ ಸ್ವಾಮೀಜಿಯವರ ದೇಹಾರೋಗ್ಯ ಮತ್ತೆ ಹದಗೆಡ ತೊಡಗಿತ್ತು. ಇದನ್ನು ಕಂಡು ಅವರ ಸೋದರ ಸಂನ್ಯಾಸಿಗಳು ಮತ್ತು ಸ್ನೇಹಿತರು ಮತ್ತೊಮ್ಮೆ ವಿಶ್ರಾಂತಿಗಾಗಿ ಡಾರ್ಜಿಲಿಂಗಿಗೆ ಹೋಗುವಂತೆ ಅವರನ್ನು ಒತ್ತಾಯಪಡಿಸಿದರು. ಏಕೆಂದರೆ, ಹಿಂದಿನ ಸಲ ಡಾರ್ಜಿಲಿಂಗಿಗೆ ಹೋಗಿದ್ದಾಗ ಅವರ ಆರೋಗ್ಯ ಎಷ್ಟೋ ಸುಧಾರಿಸಿತ್ತು. ಆದ್ದರಿಂದ ಅವರು ಮಾರ್ಚ್ ೩ಂರಂದು ಸ್ವಾಮಿ ನಿರ್ಭಯಾನಂದ ಹಾಗೂ ನವಗೋಪಾಲ ಬೋಸ್ ಅವರೊಂದಿಗೆ ಮತ್ತೆ ಡಾರ್ಜಿಲಿಂಗಿಗೆ ಹೊರಟರು. ಇಲ್ಲಿ ಅವರು ಹಿಂದಿನ ಸಲವೂ ತಮ್ಮ ಆತಿಥೇಯರಾಗಿದ್ದ ಶ್ರೀ ಎಮ್. ಎನ್. ಬ್ಯಾನರ್ಜಿಯವರ ಮನೆಯಲ್ಲಿ ಇಳಿದುಕೊಂಡರು. ಸುಮಾರು ಒಂದು ತಿಂಗಳ ಕಾಲ ಇಲ್ಲಿನ ಶಾಂತ-ಸ್ವತಂತ್ರ ವಾತಾವರಣದಲ್ಲಿ ವಿಶ್ರಮಿಸಿದರು. ಯಾವ ಗಹನವಾದ ಆಲೋಚನೆಯನ್ನೂ ಮಾಡಬಾರದು, ಮೆದುಳಿಗೆ ವಿಶ್ರಾಂತಿ ಕೊಡಬೇಕು ಎಂಬ ಡಾಕ್ಟರರ ಸಲಹೆಯನ್ನು ಅವರು ಸಾಧ್ಯವಾದಷ್ಟೂ ಪಾಲಿಸಿದರು. ಆದರೆ ನಾಗರಿಕರ ಅಪೇಕ್ಷೆಯ ಮೇರೆಗೆ ಏಪ್ರಿಲ್ ೩ರಂದು ಹಿಂದೂ ಧರ್ಮದ ಕುರಿತಾಗಿ ಒಂದು ಭಾಷಣ ಮಾಡಿದರು.

ಶುಭ್ರವಾದ ತುಷಾರ ರಾಶಿಯನ್ನು ವೀಕ್ಷಿಸುವುದೆಂದರೆ ಸ್ವಾಮೀಜಿಯವರಿಗೆ ತುಂಬ ಇಷ್ಟ. ಈ ಹಿಮದ ದೃಶ್ಯವನ್ನು ನೋಡುವುದಕ್ಕಾಗಿಯೇ ಅವರು ಸಮೀಪದ ಸಂದುಕ್​ಫೂ ಹಾಗೂ ಇತರ ಸ್ಥಳಗಳಿಗೆ ಹೋಗಿಬಂದರು. ಆದರೆ ವಾಪಸು ಬಂದ ಮೇಲೆ, ಜ್ವರ-ನೆಗಡಿ-ಕೆಮ್ಮು ಹಿಡಿದು ಕೊಂಡಿತು. ಇದೆಲ್ಲ ಪರ್ವತಾರೋಹಣ ಮಾಡಿದ ಆಯಾಸದ ಫಲ. ಸ್ವಾಮೀಜಿ ಸ್ವಲ್ಪ ಆರೋಗ್ಯ ಹೊಂದುವ ಹೊತ್ತಿಗೆ ಸರಿಯಾಗಿ ಒಂದು ಸುದ್ದಿ ಬಂದಿತು–ಕಲ್ಕತ್ತದಲ್ಲಿ ಪ್ಲೇಗ್ ಮಾರಿ ಹರಡಿದೆ ಎಂದು. ಇದನ್ನು ಕೇಳಿ ಅವರ ಮನಸ್ಥಿತಿ ಕದಡಿತು. ಆರೋಗ್ಯ ಮತ್ತೆ ಹದಗೆಟ್ಟಿತು. ಅಂದಿನವರೆಗೂ ನಗುನಗುತ್ತ ಆನಂದದಿಂದಿದ್ದ ಸ್ವಾಮೀಜಿ ಇದ್ದಕ್ಕಿದ್ದಂತೆ ಗಂಭೀರ ಭಾವ ತಾಳಿ ಕುಳಿತುಬಿಟ್ಟರು. ಅಂದಿನ ದಿನವೆಲ್ಲ ಅವರು ಆಹಾರವನ್ನೇ ತೆಗೆದುಕೊಳ್ಳಲಿಲ್ಲ ಮತ್ತು ಯಾರೊಡ ನೆಯೂ ಮಾತನಾಡಲೂ ಇಲ್ಲ. ಯಾರಿಗೂ ಇದಕ್ಕೆ ಕಾರಣವೇನೆಂದು ಊಹಿಸಲಾಗಲಿಲ್ಲ. ಕೂಡಲೇ ಡಾಕ್ಟರರನ್ನು ಕರೆಸಲಾಯಿತು. ಡಾಕ್ಟರರು ಬಂದು ಪರೀಕ್ಷೆ ಮಾಡಿ ನೋಡಿದರು. ಅವರಿಗೂ ಏನೂ ಗೊತ್ತಾಗಲಿಲ್ಲ. ಕಡೆಗೆ ಎಲ್ಲರಿಗೂ ತಿಳಿಯಿತು–ಕಲ್ಕತ್ತದಲ್ಲಿ ಪ್ಲೇಗ್ ಹರಡಿ ಜನ ಸಂಕಟಕ್ಕೆ ಗುರಿಯಾದ ಸುದ್ದಿ ಕೇಳಿ ಸ್ವಾಮೀಜಿ ಇಷ್ಟೊಂದು ಕಳವಳಗೊಂಡಿದ್ದಾರೆ ಎಂದು. ಈ ಉಪದ್ರವದ ಬಗ್ಗೆ ತಾವು ಏನಾದರೂ ಮಾಡಲೇಬೇಕೆಂದು ನಿಶ್ಚಯಿಸಿದ ಸ್ವಾಮೀಜಿ ತಮ್ಮ ಸೋದರ ಸಂನ್ಯಾಸಿಗಳಿಗೆ ಹೇಳಿದರು, “ನಾವು ಅವರ ನೆರವಿಗೆ ಧಾವಿಸಬೇಕು–ಅದಕ್ಕಾಗಿ ನಮ್ಮದೆಲ್ಲವನ್ನೂ ಮಾರಬೇಕಾಗಿ ಬಂದರೂ ಸರಿಯೆ. ನಾವೆಲ್ಲ ಪರಿವ್ರಾಜಕ ಸಂನ್ಯಾಸಿಗಳಲ್ಲವೆ? ಬೇಕಾದರೆ ನಾವು ಮರದ ಬುಡದಲ್ಲಿ ವಾಸವಾಗಿರೋಣ!”

ಇನ್ನು ಸ್ವಾಮೀಜಿ ಸುಮ್ಮನೆ ಕುಳಿತಿರಲಾರದೆ, ತಮ್ಮ ಆರೋಗ್ಯವನ್ನೂ ಲೆಕ್ಕಿಸದೆ ಕಲ್ಕತ್ತಕ್ಕೆ ಧಾವಿಸಿದರು. ಬಂದ ಕೂಡಲೇ ಅಲ್ಲಿನ ಪರಿಸ್ಥಿತಿಯನ್ನು ಪರಿಶೀಲಿಸಿದರು. ಪರಿಸ್ಥಿತಿ ನಿಜಕ್ಕೂ ಗಂಭೀರವಾಗಿತ್ತು. ಪ್ಲೇಗು ಆವರಿಸಿಕೊಂಡರೆ ಜನ ಊರು ಬಿಟ್ಟು ಹೊರಡಬೇಕಲ್ಲವೆ? ಅದೊಂದು ಬಗೆಯ ಗಡೀಪಾರಿನ ಶಿಕ್ಷೆ. ಕಲ್ಕತ್ತದಲ್ಲಿ ಇನ್ನೇನು ಒಂದು ಚಂಡಮಾರುತ ಬೀಸುತ್ತದೆಯೊ ಎಂಬಂತಿತ್ತು. ಜನ ಕಂಗಾಲಾಗಿ ಚೆಲ್ಲಾಪಿಲ್ಲಿಯಾಗಿದ್ದರೆ, ಮನೆಗಳನ್ನು ಲೂಟಿ ಮಾಡಲು ದುಷ್ಟರಿಗೊಂದು ಸದವಕಾಶ. ಅವರನ್ನೆಲ್ಲ ಹತ್ತಿಕ್ಕಲು ಪೋಲೀಸು ವ್ಯವಸ್ಥೆ ಮಾಡ ಬೇಕಾಯಿತು. ಸ್ವಾಮೀಜಿ ಪರಿಸ್ಥಿತಿಯ ಭೀಕರತೆಯನ್ನು ತಕ್ಷಣ ಮನಗಂಡರು. ಅವರು ಮಠಕ್ಕೆ ತಲುಪಿದವರೇ ಸೋದರಿ ನಿವೇದಿತಾ ಹಾಗೂ ಸ್ವಾಮಿ ಸದಾನಂದರ ಸಹಾಯದಿಂದ, ಪ್ಲೇಗ್ ನಿವಾರಣೆಗಾಗಿ ಮತ್ತು ರೋಗಿಗಳ ಉಪಚಾರಕ್ಕಾಗಿ ತಾವು ಕೈಗೊಳ್ಳಲಿರುವ ಪರಿಹಾರಕಾರ್ಯದ ಒಂದು ಪ್ರಕಟಣೆಯನ್ನು ಸಿದ್ಧಪಡಿಸಿದರು. ಅದನ್ನು ಕಲ್ಕತ್ತದ ಪ್ರತಿಯೊಂದು ಮನೆಗೂ ತಲುಪಿಸ ಲಾಯಿತು. ತಕ್ಷಣವೇ ಸೇವಾಕೇಂದ್ರವೊಂದನ್ನು ಸ್ಥಾಪಿಸಲು ಸ್ವಾಮೀಜಿ ಸನ್ನದ್ಧರಾದರು. ಅವರ ಉತ್ಸಾಹವನ್ನು ಕಂಡು ಸೋದರ ಸಂನ್ಯಾಸಿಗಳೊಬ್ಬರು, “ಸ್ವಾಮೀಜಿ, ಸೇವೆ ಮಾಡಲು ಅಷ್ಟೆಲ್ಲ ಹಣವನ್ನು ಎಲ್ಲಿಂದ ತರುವುದು?” ಎಂದರು. ತಕ್ಷಣ ಅವರು ದೃಢ ಸ್ವರದಲ್ಲಿ ಹೇಳಿದರು, “ಏಕೆ, ನಾವು ಹೊಸದಾಗಿ ಮಠಕ್ಕಾಗಿ ಕೊಂಡುಕೊಂಡ ಜಾಗ ಇಲ್ಲವೆ? ಸಂದರ್ಭ ಬಂದರೆ ಅದನ್ನೇ ಮಾರಿಬಿಡುವುದು! ನಾವು ಸಂನ್ಯಾಸಿಗಳು; ನಾವು ಹಿಂದೆ ಇದ್ದಂತೆಯೇ ಭಿಕ್ಷಾನ್ನವನ್ನು ತಿಂದುಕೊಂಡು ಮರದ ಬುಡದಲ್ಲಿ ಮಲಗಲು ಸದಾ ಸಿದ್ಧರಿರಬೇಕು. ಏನು! ಈ ಭೂಮಿ-ಆಸ್ತಿ ಯನ್ನೆಲ್ಲ ಮಾರಿ ನಮ್ಮ ಕಣ್ಣೆದುರಿಗೇ ನರಳುತ್ತಿರುವ ಸಹಸ್ರಾರು ಜನರ ಸಂಕಟವನ್ನು ಹೋಗ ಲಾಡಿಸಲು ಸಾಧ್ಯವಾಗುವುದಾದರೆ, ಇನ್ನು ನಮಗೆ ಈ ಆಸ್ತಿಪಾಸ್ತಿಯನ್ನೆಲ್ಲ ಕಟ್ಟಿಕೊಂಡು ಏನಾಗಬೇಕು!” ಆದರೆ ಭಗವತ್ಕೃಪೆಯಿಂದ ಮಠವನ್ನು ಮಾರುವ ಪರಿಸ್ಥಿತಿ ಬರಲಿಲ್ಲ. ಅವರು ಕೈಗೊಂಡಿದ್ದ ಕಾರ್ಯಕ್ಕೆ ಬೇರೆಬೇರೆ ಕಡೆಗಳಿಂದ ಸಾಕಷ್ಟು ಧನಸಹಾಯ ಒದಗಿಬಂತು. ಆದ್ದ ರಿಂದ ಕೂಡಲೇ ಒಂದು ವಿಶಾಲವಾದ ಜಾಗವನ್ನು ಬಾಡಿಗೆಗೆ ತೆಗೆದುಕೊಂಡು ಅಲ್ಲಿ ವ್ಯಾಧಿಗ್ರಸ್ತ ರನ್ನು ಇರಿಸಿ ಶುಶ್ರೂಷೆ ನಡೆಸುವ ನಿರ್ಧಾರವಾಯಿತು. ಸ್ವಾಮೀಜಿಯವರ ಶಿಷ್ಯರು ರೋಗಿಗಳ ಸೇವೆಗೆ ಟೊಂಕಕಟ್ಟಿ ನಿಂತಾಗ ಅವರೊಂದಿಗೆ ಸಹಕರಿಸಲು ಹಲವಾರು ಸ್ವಯಂಸೇವಕರು ಮುಂದೆ ಬಂದರು. ಜನರಿಗೆ ಶುಚಿತ್ವದ ಕಡೆಗೆ ಹೆಚ್ಚಿನ ಗಮನ ಕೊಡುವಂತೆ ಬೋಧಿಸಬೇಕೆಂದು ಇವರಿಗೆಲ್ಲ ಸ್ವಾಮೀಜಿ ಸಲಹೆ ನೀಡಿದರು. ಅಲ್ಲದೆ ಅಲ್ಲಿನ ಗಲ್ಲಿಗಲ್ಲಿಗಳನ್ನೂ ಮನೆಮನೆಗಳನ್ನೂ ಅವರೇ ಸ್ವತಃ ಶುಚಿಗೊಳಿಸುವಂತೆ ನಿರ್ದೇಶಿಸಿದರು. ಇವರ ನಿಷ್ಠಾಯುತ ಸೇವೆಯ ಮೂಲಕ ಪ್ಲೇಗ್ ರೋಗಿಗಳಿಗೆ ಸಿಕ್ಕಿದ ಸಮಾಧಾನ ಅಪರಿಮಿತ. ಅಲ್ಲದೆ ಸ್ವಾಮೀಜಿ ರೋಗಪರಿಹಾರಕ್ಕಾಗಿ ಕೈಗೊಂಡ ಈ ಕಾರ್ಯಕ್ರಮದಿಂದಾಗಿ ಜನರಿಗೆ ಅವರಲ್ಲಿ ಹೊಸ ವಿಶ್ವಾಸವುಂಟಾಯಿತು. ವಿವೇಕಾನಂದರು ಮತ್ತು ಅವರ ಸಂಘದವರು ಕೇವಲ ಒಣವೇದಾಂತಿಗಳಲ್ಲ, ಅನುಷ್ಠಾನ ವೇದಾಂತಿಗಳು ಎಂಬುದನ್ನು ಜನರು ಮನಗಂಡರು. ಏಕೆಂದರೆ, ಪ್ರತಿಯೊಬ್ಬ ವ್ಯಕ್ತಿಯೂ ಆತ್ಮಸ್ವರೂಪಿಯೆಂದು ಹೇಳುತ್ತಿದ್ದ ಸ್ವಾಮೀಜಿಯವರು, ಜನಸಾಮಾನ್ಯರೆಲ್ಲ ಕಷ್ಟಕ್ಕೆ ಗುರಿಯಾ ದಾಗ ಅದೇ ಆತ್ಮಭಾವನೆಯಿಂದ, ಆತ್ಮೀಯ ಭಾವನೆಯಿಂದ ಪೂಜಾರೂಪದ ಸೇವೆಯನ್ನು ಮಾಡಲು ಮುಂದಾದರಲ್ಲ!

ಕಲ್ಕತ್ತದಲ್ಲಿ ಪ್ಲೇಗ್ ಮಾರಿ ಶಾಂತವಾಗಿ, ನಗರವನ್ನು ಬಿಟ್ಟುಹೋಗಿದ್ದ ಜನರೆಲ್ಲ ಮನೆಗಳಿಗೆ ಹಿಂದಿರುಗುವಂತಾಗುವವರೆಗೆ ಸ್ವಾಮೀಜಿ ಕಲ್ಕತ್ತದಲ್ಲೇ ಉಳಿದುಕೊಂಡರು.

