
\chapter{ಮಾನವಪ್ರೇಮಿ}

\noindent

ಸ್ವಾಮೀಜಿಯವರು ಸ್ಯಾನ್ ಫ್ರಾನ್ಸಿಸ್ಕೋದಲ್ಲಿದ್ದಾಗ ಎಲ್ಲ ಬಗೆಯ ಜನರನ್ನೂ ಭೇಟಿ ಮಾಡುತ್ತಿದ್ದರು, ಎಲ್ಲರ ಬಗ್ಗೆಯೂ ಸಮಾನವಾದ ಸ್ನೇಹದಿಂದ ಇರುತ್ತಿದ್ದರು. ಶ್ರೀಮತಿ ಹ್ಯಾನ್ಸ್​ಬ್ರೋ ಹೇಳುತ್ತಾಳೆ, “ಅವರು ಪ್ರತಿಯೊಬ್ಬರನ್ನೂ ಇಷ್ಟಪಡುವಂತೆ ಕಾಣುತ್ತಿತ್ತು. ಅವರು ಅತ್ಯಂತ ಅನುಕಂಪೆಯುಳ್ಳವರಾಗಿದ್ದರು. ವ್ಯಕ್ತಿವ್ಯಕ್ತಿಗಳ ನಡುವೆ ಅವರು ಯಾವ ಭೇದವನ್ನೂ ಕಾಣು ತ್ತಿರಲಿಲ್ಲವೆಂಬಂತೆ–ಎಷ್ಟರ ಮಟ್ಟಿಗೆಂದರೆ, ಒಂದು ಬಾತುಕೋಳಿಗೂ ಮನುಷ್ಯನಿಗೂ ನಡುವೆ ಭೇದವನ್ನು ಕಾಣದಿರುವಂತೆ–ಭಾಸವಾಗುತ್ತಿತ್ತು... ಸ್ವಾಮೀಜಿಯವರು ಯಾವ ರೀತಿಯಾಗಿ ವರ್ತಿಸುತ್ತಿದ್ದರು, ಮಾತನಾಡುತ್ತಿದ್ದರು ಹಾಗೂ ನಡೆದಾಡುತ್ತಿದ್ದರೆಂದರೆ, ಅವರು ಯಾವಾ ಗಲೂ ಪ್ರತಿಯೊಂದು ಜೀವಿಯನ್ನೂ ಭಗವನ್ಮಯವಾಗಿ ಕಾಣುತ್ತಿದ್ದರೆಂಬುದು ಸ್ಪಷ್ಟವಾಗಿ ಗೋಚರಿಸುತ್ತಿತ್ತು.”

‘ಸಕಲವೂ ಭಗವಂತನೇ ಆಗಿದ್ದಾನೆ’ ಎಂಬ ಉನ್ನತ ದೃಷ್ಟಿಯಿಂದಲ್ಲದೆ, ಇನ್ನೂ ಕೆಳಗಿನ ಹಂತದಿಂದಲೂ ಸ್ವಾಮೀಜಿಯವರು ಆ ತತ್ವವನ್ನು ತಿಳಿಯಪಡಿಸುತ್ತಿದ್ದರು. ಅವರು ಆಗಾಗ ಹೇಳುತ್ತಿದ್ದ ಮಾತು ಇದು: “ನಾವು ಮುಂದುವರಿಯುತ್ತಿರುವುದು ಅಸತ್ಯದಿಂದ ಸತ್ಯದೆಡೆಗಲ್ಲ, ಸತ್ಯದಿಂದ ಸತ್ಯದೆಡೆಗೆ.” ಒಮ್ಮೆ ತಮ್ಮ ಆಪ್ತ ಅಲನ್​ನಿಗೆ ಹೇಳುತ್ತಾರೆ, “ಯಾರನ್ನೂ ನಾವು ಈ ಕ್ಷಣದಲ್ಲಿ ಅವರೇನು (ತಪ್ಪು) ಮಾಡುತ್ತಿದ್ದಾರೆಯೋ ಅದಕ್ಕಾಗಿ ಅವರನ್ನು ದೂಷಿಸುವಂತಿಲ್ಲ. ಏಕೆಂದರೆ, ಅವರು ತತ್​ಕ್ಷಣದಲ್ಲಿ ತಮ್ಮಿಂದ ಸಾಧ್ಯವಾಗುವ ಅತ್ಯುತ್ತಮವಾದದ್ದನ್ನೇ ಮಾಡುತ್ತಿ ದ್ದಾರೆ. ಒಂದು ಮಗುವು ಕೈಯಲ್ಲಿ ಹರಿತವಾದ ಚಾಕುವನ್ನು ಹಿಡಿದಿದ್ದರೆ ಅದನ್ನು ಕಿತ್ತುಕೊಳ್ಳಲು ಪ್ರಯತ್ನಿಸಬೇಡಿ. ಬದಲಾಗಿ ಅದಕ್ಕೆ ಒಂದು ಒಳ್ಳೆಯ ಆಟಿಕೆಯನ್ನೋ ಒಂದು ಕೆಂಪು ಸೇಬನ್ನೋ ಕೊಡಿ. ಆಗ ಅದು ಚಾಕುವನ್ನು ಕೊಟ್ಟುಬಿಡುತ್ತದೆ. ಆದರೆ ಯಾರು ಬೆಂಕಿಯಲ್ಲಿ ಕೈಹಾಕುತ್ತಾರೋ ಅವರು ಕೈ ಸುಟ್ಟುಕೊಳ್ಳದೆ ವಿಧಿಯಿಲ್ಲ. ನಾವು ಪಾಠ ಕಲಿಯುವುದು ಅನುಭವದಿಂದ ಮಾತ್ರವೇ”

ಇನ್ನೂ ಸರಿಯಾಗಿ ಹೇಳಬೇಕೆಂದರೆ, ಸ್ವಾಮೀಜಿಯವರಿಗೆ ಯಾರ ಮೇಲಾದರೂ ಹೆಚ್ಚಿನ ಮಮತೆಯಿತ್ತು ಎನ್ನುವುದಾದರೆ ಅದು ಅನುಭವಗಳಿಂದ ಪಾಠ ಕಲಿಯುತ್ತಿರುವ ‘ಪಾಪಿ’ಯ ಮೇಲೆಯೇ ಹೊರತು, ತನ್ನ ಸಂಪ್ರದಾಯದ ಕಟ್ಟಿನಿಂದಾಚೆ ಎಂದೂ ಹೆಜ್ಜೆಯನ್ನಿಡದ ‘ಒಳ್ಳೆಯ ಮನುಷ್ಯ’ನ ಮೇಲಲ್ಲ. ಮನುಷ್ಯ ತನ್ನ ಪಾಡಿಗೆ ತಾನು ಒಳ್ಳೆಯವನಾಗಿದ್ದುಕೊಂಡು ತನ್ನ ಬದುಕನ್ನು ತಾನು ಸಾಗಿಸುತ್ತಿದ್ದರೆ, ಯಾರಿಗೂ ಅವನಿಂದ ತೊಂದರೆಯಾಗಲಿಲ್ಲ ಎಂಬುದನ್ನು ಬಿಟ್ಟರೆ ಅಂಥವನಿಂದ ಪ್ರಪಂಚಕ್ಕೆ ಪ್ರಯೋಜನವೇನಾಯಿತು? ಅಲನ್ ದಂಪತಿಗಳು ಹೇಳು ತ್ತಾರೆ, “ನಾವು ಅರ್ಥಮಾಡಿಕೊಂಡ ಪ್ರಕಾರ, ಸ್ವಾಮೀಜಿಯವರು ಯಾರನ್ನಾದರೂ ಇಷ್ಟಪಡುತ್ತಿರಲಿಲ್ಲ ಎನ್ನುವುದಾದರೆ ಅದು ‘ಪಾಪದವನ’ನ್ನು \eng{(Goody-goody man!)}. ಅವರ ಪಾಲಿಗೆ ಅಂತಹ ಮನುಷ್ಯನಿಂದ ಏನೂ ಪ್ರಯೋಜನವಿರಲಿಲ್ಲ. ಅವರೆನ್ನುತ್ತಿದ್ದರು, ‘ಏನನ್ನಾದರೂ ಮಾಡುತ್ತಿರು–ಕೆಟ್ಟದ್ದನ್ನಾದರೂ ಸರಿಯೆ–ಆದರೆ ಏನನ್ನಾದರೂ ಮಾಡುತ್ತಿರು! ಎಲ್ಲರೂ ಹೇಳುತ್ತಾರೆ–ಒಳ್ಳೆಯವನಾಗು, ಒಳ್ಳೆಯವನಾಗು ಅಂತ. ನಾನೇಕೆ ಒಳ್ಳೆಯವನಾಗಬೇಕು?’”

ತಮಸ್ಸಿನ ಲಕ್ಷಣವಾದ ಆಲಸ್ಯದಲ್ಲಿ ಮುಳುಗಿರುವುದಕ್ಕಿಂತ, ರಜೋಗುಣದ ಲಕ್ಷಣವಾದ ಕ್ರಿಯಾಕಲಾಪಗಳಲ್ಲಿ ನಿರತನಾಗಿರುವುದು ಶ್ರೇಯಸ್ಕರ, ಮತ್ತು ಒಂದು ದಿನ ಈ ರಜೋಗುಣವನ್ನೂ ದಾಟಿ ಸತ್ವಗುಣಕ್ಕೇರಬಹುದು ಎನ್ನುವುದು ಸ್ವಾಮೀಜಿಯವರ ಅಭಿಪ್ರಾಯ.

ಸ್ಯಾನ್​ಫ್ರಾನ್ಸಿಸ್ಕೋದ ಟರ್ಕ್​ಸ್ಟ್ರೀಟಿನ ಮನೆಯಲ್ಲಿದ್ದಾಗ ಸ್ವಾಮೀಜಿಯವರ ಆರೋಗ್ಯ ಸಾಮಾನ್ಯವಾಗಿ ಸುಧಾರಿಸಿತು. ಈ ಸಮಯದಲ್ಲಿ ಅವರು ನಿವೇದಿತೆಗೊಂದು ಪತ್ರದಲ್ಲಿ ಬರೆಯು ತ್ತಾರೆ: “ನಾನು ಪ್ರತಿದಿನ ಬೆಳಗಿನಿಂದ ಸಂಜೆಯವರೆಗೆ ದುಡಿಯುತ್ತೇನೆ. ಅನುಕೂಲವಾದಾಗ ಏನಾದರೂ ಒಂದಿಷ್ಟು ತಿನ್ನುತ್ತೇನೆ. ಮತ್ತೆ ರಾತ್ರಿ ಹನ್ನೆರಡಕ್ಕೆ ಸರಿಯಾಗಿ ಮಲಗಿಬಿಡುತ್ತೇನೆ. ಆಹ್! ಎಂಥ ಸುಖನಿದ್ರೆ! ಹಿಂದೆಂದೂ ನನ್ನಲ್ಲಿ ಅಂತಹ ‘ನಿದ್ರಾಶಕ್ತಿ’ಯಿರಲಿಲ್ಲ!”

ಆದರೆ ನಿಜಕ್ಕೂ ಸ್ವಾಮೀಜಿ ಎಷ್ಟು ಗಂಟೆ ನಿದ್ರಿಸುತ್ತಿದ್ದರೋ ತಿಳಿಯದು. ಬಹುಶಃ ಬಹಳ ಕಡಿಮೆಯೇ ಇರಬೇಕು. ಏಕೆಂದರೆ ಅವರು ತಮ್ಮ ಜೀವಮಾನದಾದ್ಯಂತ ಬಹಳ ಕಡಿಮೆ ನಿದ್ರಿಸುವ ಅಭ್ಯಾಸವಿಟ್ಟುಕೊಂಡಿದ್ದರು. ನಡುರಾತ್ರಿಯಲ್ಲಿ ಅಥವಾ ನಸುಕಿನಲ್ಲಿ ಎದ್ದು ಧ್ಯಾನಾ ನಂದಲೀನರಾಗುತ್ತಿದ್ದರು. ಆ ಏಕಾಂತ-ಪ್ರಶಾಂತ ಸಮಯದಲ್ಲಿ ಅವರ ಮನಸ್ಸು ತನ್ನ ಅತ್ಯಂತ ಸಹಜಸ್ಥಿತಿಯನ್ನು ಮುಟ್ಟುತ್ತಿತ್ತು.

ಇಹಲೋಕದಲ್ಲಿ ತಮ್ಮ ಕಾರ್ಯ ಪರಿಸಮಾಪ್ತವಾಗುವ ಸಮಯ ಸನ್ನಿಹಿತವಾದಂತೆ ಅವರು ಹೆಚ್ಚುಹೆಚ್ಚು ಅಂತರ್ಮುಖಿಗಳಾಗತೊಡಗಿದ್ದರು. ಹೊರಗಿನ ಬಂಧನಗಳಿಂದೆಲ್ಲ ಅವರು ವಿಮುಕ್ತರಾಗತೊಡಗಿದ್ದರು. ಇದನ್ನು ಅವರು ಈ ಅವಧಿಯಲ್ಲಿ ಬರೆದ ಪತ್ರಗಳಲ್ಲಿ ಗುರುತಿಸ ಬಹುದಾಗಿದೆ. ಮಾರ್ಚ್ ೨೫ರಂದು ನಿವೇದಿತೆಗೆ ಒಕ್ಕಣಿಸಿದ ಪತ್ರದಲ್ಲಿ ಅವರು ಬರೆಯುತ್ತಾರೆ–

“... ನಾನು ಅನಂತ ನೀಲಾಕಾಶ, ಮೋಡಗಳು ನನ್ನನ್ನು ಮುತ್ತಿ ಆವರಿಸಬಹುದು; ಆದರೆ ನಾನು ಅದೇ ಅನಂತ ನೀಲ, ನನ್ನ ಹಾಗೂ ಪ್ರತಿಯೊಬ್ಬರ ನಿಜ ಸ್ವಭಾವವೆಂದು ನಾನು ತಿಳಿದಿರುವ ಆ ಶಾಂತಿಯ ಸವಿಯನ್ನು ಪಡೆದುಕೊಳ್ಳಲು ನಾನು ಪ್ರಯತ್ನಿಸುತ್ತಿದ್ದೇನೆ. ಈ ಮಣ್ಣಿನ ಮಡಕೆಯಂತಹ ದೇಹಗಳು, ಸುಖ ದುಃಖದ ಹುಚ್ಚು ಕನಸುಗಳು–ಇವೆಲ್ಲ ಎಂಥದು! ನನ್ನ ಕನಸುಗಳೆಲ್ಲ ಮುರಿದು ಬೀಳುತ್ತಿವೆ. ಓಂ ತತ್ ಸತ್!”

ಮೂರು ದಿನಗಳ ನಂತರ ಮೇರಿ ಹೇಲ್ ಗೆ ಬರೆಯುತ್ತಾರೆ–

“... ನಾನು ದುಃಖ ಸಂತೋಷಗಳಿಲ್ಲದ–ಆದರೆ ಅವೆರಡನ್ನೂ ಮೀರಿದ–ಬುದ್ಧಿಗ್ರಾಹ್ಯ ವಲ್ಲದ ಆ ಶಾಂತಿಯನ್ನು ಹೊಂದುತ್ತಿದ್ದೇನೆ. ತಾಯಿಗೂ ಅದನ್ನು ತಿಳಿಸು. (ಮೇರಿಯ ತಾಯಿ ಶ್ರೀಮತಿ ಬೆಲ್ ಹೇಲ್​ಳನ್ನು ಸ್ವಾಮೀಜಿ ‘ತಾಯಿ’ ಎಂದೇ ಕರೆಯುತ್ತಿದ್ದುದು.) ಕಳೆದೆರಡು ವರ್ಷಗಳಲ್ಲಿ ದೈಹಿಕ-ಮಾನಸಿಕ ಸಾವಿನ ಕಣಿವೆಯಲ್ಲಿ ಸಂಚರಿಸಿದ್ದು ನನಗೆ ಸಹಾಯವಾಗಿದೆ. ಈಗ ನಾನು ಶಾಂತಿಯನ್ನು, ಅನಂತಮೌನವನ್ನು ಸಮೀಪಿಸುತ್ತಿದ್ದೇನೆ. ನಾನು ಹಿಂದೆಯೂ ಮುಕ್ತನಾಗಿದ್ದೆ, ಈಗಲೂ ಮುಕ್ತನಾಗಿದ್ದೇನೆ, ಮುಂದೆಯೂ ಮುಕ್ತನಾಗಿರುತ್ತೇನೆ–ಇದೇ ವೇದಾಂತ ತತ್ತ್ವ. ಇದನ್ನು ನಾನು ಬಹಳ ಕಾಲ ಬೋಧಿಸಿದೆ. ಆದರೆ, ಓಹ್ ಅದೇನಾನಂದ! ಮೇರಿ, ನನ್ನ ಪ್ರಿಯ ಸೋದರಿ, ನಾನೀಗ ಅದನ್ನು ಪ್ರತಿದಿನ ಅನುಭವಿಸುತ್ತಿದ್ದೇನೆ. ಹೌದು, ನಾನು ಮುಕ್ತ, ಏಕಾಂಗಿ, ನಾನು ಅದ್ವಿತೀಯ.”

ಮೇರಿಗೆ ಬರೆದ ಈ ಪತ್ರದ ಕೊನೆಯ ಮಾತು (ವಿ. ಸೂ.) ಕೂಡ ಅಷ್ಟೇ ಭಾವಾವೇಶಭರಿತವಾಗಿದೆ:

“ನಾನು ನಿಜವಾಗಿ ವಿವೇಕಾನಂದನಾಗಲಿದ್ದೇನೆ. ನೀನೆಂದಾದರೂ ಕೆಟ್ಟದ್ದನ್ನು ಆನಂದಿಸಿ ದ್ದೀಯಾ? ಹಹ್ಹ! ಹುಚ್ಚು ಹುಡುಗಿ! ಎಲ್ಲವೂ ಒಳ್ಳೆಯದೇ. ನಾನು ಒಳ್ಳೆಯದನ್ನು ಆನಂದಿಸು ತ್ತೇನೆ, ಕೆಟ್ಟದ್ದನ್ನೂ ಆನಂದಿಸುತ್ತೇನೆ. ನಾನು ಏಸುವಾಗಿದ್ದೆ, ನಾನೇ ಜುದಾಸನೂ ಆಗಿದ್ದೆ. ಎರಡೂ ನನ್ನ ಆಟ, ನನ್ನ ತಮಾಷೆ. ಉಷ್ಟ್ರಪಕ್ಷಿಯು ಮಣ್ಣೊಳಗೆ ತಲೆ ಮರೆಸಿಟ್ಟುಕೊಂಡು ತಾನಿನ್ನು ಸುರಕ್ಷಿತವಾಗಿದ್ದೇನೆಂದು ಭಾವಿಸುತ್ತದೆ. ಧೈರ್ಯವಂತಳಾಗು, ಎಲ್ಲವನ್ನೂ ಎದುರಿಸು. ಒಳ್ಳೆಯದೂ ಬರಲಿ, ಕೆಟ್ಟದ್ದೂ ಬರಲಿ. ಎರಡಕ್ಕೂ ಸ್ವಾಗತ. ನಾನು ಹೊಂದಬೇಕಾದ ಶ್ರೇಯ ಸ್ಸಿಲ್ಲ, ಹಿಡಿದಿಟ್ಟುಕೊಳ್ಳಬೇಕಾದ ಆದರ್ಶವಿಲ್ಲ, ಈಡೇರಿಸಿಕೊಳ್ಳಬೇಕಾದ ಆಕಾಂಕ್ಷೆಯಿಲ್ಲ! ವಜ್ರದ ಖನಿಯಾದ ನಾನು, ಒಳ್ಳೆಯದು-ಕೆಟ್ಟದ್ದು ಎಂಬ ಎರಡು ಕಲ್ಗುಂಡುಗಳೊಡನೆ ಆಟ ವಾಡುತ್ತಿದ್ದೇನೆ. ಓ ಒಳಿತೇ, ನಿನಗೆ ಶುಭವಾಗಲಿ, ಬಾ! ಓ ಕೆಡುಕೇ, ನಿನಗೆ ಶುಭವಾಗಲಿ, ನೀನೂ ಬಾ! ಸಮಸ್ತ ವಿಶ್ವವೇ ಉರುಳಿಬಿದ್ದರೂ ಅದರಿಂದ ನನಗೇನು? ನಾನು ಬುದ್ಧಿಯನ್ನೂ ಮೀರಿ ನಿಂತಿರುವ ಶಾಂತಿಸ್ವರೂಪ! ಬುದ್ಧಿಯು ಕೇವಲ ಸುಖವನ್ನು ಇಲ್ಲವೆ ದುಃಖವನ್ನು ಕೊಡುತ್ತದೆ. ಆದರೆ ನಾನು ಅತೀತ, ನಾನು ಶಾಂತಿ.”

ಈ ಮನಸ್ಥಿತಿ\eng{(mood)}ಯಲ್ಲಿ, ಎಂದರೆ ಇತರ ಯಾವ ಪರಿಸ್ಥಿತಿಯೂ ಕದಲಿಸಲಾರದಂತಹ ಮನಸ್ಥಿತಿಯಲ್ಲಿ ಸ್ವಾಮೀಜಿಯವರು ಉಪನ್ಯಾಸಗಳನ್ನು ಮಾಡಿದರು, ತರಗತಿಗಳನ್ನು ನಡೆಸಿದರು; ತನ್ಮೂಲಕ ತಮ್ಮ ಅಂತಿಮ ಹಾಗೂ–ಅವರೇ ಹೇಳಿದಂತೆ–ತಮ್ಮ ಅತ್ಯುನ್ನತ ಸಂದೇಶಗಳನ್ನು ನೀಡಿದರು.

ಏಪ್ರಿಲಿನ ಎರಡನೇ ವಾರದಲ್ಲಿ ಸ್ವಾಮೀಜಿಯವರು ಟರ್ಕ್​ಸ್ಟ್ರೀಟಿನ ತಮ್ಮ ಮನೆಯಲ್ಲಿ ನಡೆಸುತ್ತಿದ್ದ ತರಗತಿಗಳನ್ನು ಮುಕ್ತಾಯಗೊಳಿಸಿ, ಓಕ್​ಲ್ಯಾಂಡಿನ ಸಮೀಪದ ಅಲಮೇಡ ಎಂಬ ಸಣ್ಣ ಊರಿಗೆ ಹೋದರು. ಇಲ್ಲಿ ಅವರು ‘ಹೋಂ ಆಫ್ ಟ್ರೂತ್​’ನ ಶಾಖೆಯಲ್ಲಿ ಇಳಿದು ಕೊಂಡರು. ಅದಾಗಲೇ ಅವರು ಈ ಊರಿನಲ್ಲಿ ಹಲವಾರು ಸಲ ಉಪನ್ಯಾಸ ಮಾಡಿದ್ದರು. ಆದ್ದ ರಿಂದ ಈ ಊರೂ, ಊರಿನ ಶ್ರೋತೃಗಳೂ ಅವರಿಗೆ ಸುಪರಿಚಿತರಾಗಿದ್ದರು.

ಆ ದಿನಗಳಲ್ಲಿ ಅಮೆರಿಕದಾದ್ಯಂತ ಪ್ರಚಲಿತವಾಗುತ್ತಿದ್ದ ಘೋಷಣೆಯೆಂದರೆ ‘ಕ್ರೈಸ್ತೀಕರಣ’ ಅಥವಾ ‘ನವ ಸಾಮಾಜಿಕ ಧರ್ಮದ ಅನುಷ್ಠಾನ’. ಕೈಗಾರಿಕಾ ಕ್ರಾಂತಿಯೆಂಬುದು ಅತ್ಯಂತ ಸುಖೀ ಸಮಾಜವನ್ನುಂಟುಮಾಡಬಲ್ಲುದು ಎಂದು ಹಿಂದೆ ಸಾರ್ವತ್ರಿಕವಾಗಿ ನಂಬಲಾಗಿತ್ತು. ಆದರೆ ಕೈಗಾರಿಕಾ ಕ್ರಾಂತಿಯ ಪರಿಣಾಮವಾಗಿ ಹೊಸಹೊಸ ಸಮಸ್ಯೆಗಳು ಹುಟ್ಟಿಕೊಳ್ಳತೊಡಗಿ ದುವು. ಇದನ್ನು ಮನಗಂಡ ಅಂದಿನ ಅನೇಕ ಚಿಂತಿನಶೀಲ ವ್ಯಕ್ತಿಗಳು ಈ ಹೊಸ ಘೋಷಣೆ ಯನ್ನು ಸೃಷ್ಟಿ ಮಾಡಿದ್ದರು. ಇದು ಜನಗಳ ಕಷ್ಟಗಳನ್ನೆಲ್ಲ ದೂರ ಮಾಡಿಬಿಡುತ್ತದೆಯೆಂಬ ನಂಬಿಕೆ ಅವರದ್ದಾಗಿತ್ತು. ಜೀವನದ ಸಕಲ ರಂಗಗಳನ್ನೂ ‘ಕ್ರೈಸ್ತೀಕರಣ’ಗೊಳಿಸಿಬಿಟ್ಟರೆ ಸಮಸ್ಯೆ ಗಳೆಲ್ಲ ತಾವೇ ತಾವಾಗಿ ಪರಿಹಾರವಾಗಿಬಿಡುತ್ತವೆ ಎನ್ನುವ ಈ ಘೋಷಣೆ ಕೆಲಕಾಲದಲ್ಲೇ ಅತ್ಯಂತ ಜನಪ್ರಿಯವಾಗಿಬಿಟ್ಟಿತು. ಜನತಾ ಸೇವೆಯೇ ಕ್ರಿಸ್ತನ ಬೋಧನೆಗಳ ಸಾರಸರ್ವಸ್ವ ಎಂದು ಈ ಪ್ರಚಾರಕರು ಬೋಧಿಸಿದರು. ‘ಯಾರು ನೈತಿಕವಾಗಿ ಅತ್ಯಂತ ಶಕ್ತಿವಂತರೋ, ಯಾರು ಶ್ರಮ ಜೀವಿಗಳೋ, ಅಂಥವರು ಮಾತ್ರ ಶ್ರೀಮಂತರಾಗಬಲ್ಲರು; ಆದ್ದರಿಂದ ಶ್ರೀಮಂತರೇ ಐಶ್ವರ್ಯದ ಮೇಲ್ವಿಚಾರಕರಾಗಿರಬೇಕಾದುದು ನ್ಯಾಯ’ ಎನ್ನುವುದು ಇವರ ಬೋಧನೆಗಳ ಲ್ಲೊಂದು. ಆಸ್ಪತ್ರೆಗಳನ್ನು ಕಟ್ಟಿಸುವುದೇ ಮೊದಲಾದ ಜನೋಪಕಾರಿ ಕಾರ್ಯಗಳೇ ಧರ್ಮದ ಅರ್ಥ ಎನ್ನುವುದೂ ಅವರ ಘೋಷಣೆಯಾಗಿತ್ತು.

ಸಾರ್ವತ್ರಿಕವಾಗಿ ಒಪ್ಪಿಕೊಳ್ಳಲಾಗುತ್ತಿದ್ದ ಈ ನಂಬಿಕೆಯನ್ನು ಸ್ವಾಮೀಜಿಯವರು ‘ಧರ್ಮದ ಅನುಷ್ಠಾನ’ ಎಂಬ ಉಪನ್ಯಾಸದಲ್ಲಿ ಉಗ್ರವಾಗಿ ಖಂಡಿಸಿ ಆ ವಾದವನ್ನು ಮೂದಲಿಸಿದರು –“ನೀವು ಹೇಳುತ್ತಿರುವ ‘ಅನುಷ್ಠಾನಧರ್ಮ’ ಯಾವುದದು? ರಸ್ತೆಗಳನ್ನು ಶುಚಿಗೊಳಿಸುವುದು, ಆಸ್ಪತ್ರೆಗಳನ್ನು ಕಟ್ಟಿಸುವುದು–ಇವೇ ಅಲ್ಲವೆ? ಬಡವರ, ದೀನರ, ಆರ್ತರ ಸೇವೆಯನ್ನು ಸಾಕ್ಷಾತ್ ಭಗವಂತನ ಪೂಜೆಯೇ ಎಂಬ ರೀತಿಯಲ್ಲಿ ಮಾಡಿ. ಅದನ್ನೇ ಶುದ್ಧ ಹೃದಯದಿಂದ ಮಾಡಿ, ಅಷ್ಟೆ. ಅದರ ಫಲದ ಮಾತು ಅಷ್ಟು ಮುಖ್ಯವಲ್ಲ. ಹೀಗೆ ಫಲದ ವಿಷಯದಲ್ಲಿ ಯಾವ ನಿರೀಕ್ಷೆಯನ್ನೂ ಇಟ್ಟುಕೊಳ್ಳದೆ ಮಾಡಿದ ಕರ್ಮವು ನಿಮ್ಮ ಆತ್ಮಕ್ಕೆ ಒಳಿತನ್ನುಂಟುಮಾಡುತ್ತದೆ. ಸ್ವರ್ಗರಾಜ್ಯವೆಂದರೆ ಇದೇ.

“ನಾವು ಎಲ್ಲೆಲ್ಲೂ ಅನುಷ್ಠಾನಧರ್ಮದ ಬಗ್ಗೆ ಕೇಳುತ್ತಿದ್ದೇವೆ. ಅದನ್ನು ವಿಶ್ಲೇಷಿಸಿ ನೋಡಿ ದಾಗ, ಅದರ ಬೋಧನೆಯನ್ನೆಲ್ಲ ಒಂದೇ ಮಾತಿನಲ್ಲಿ ಹೇಳಬಹುದು; ಅದು–ಸಹಮಾನವರಿಗೆ ಸೇವೆ ಸಲ್ಲಿಸುವುದು. ಧರ್ಮವೆಂದರೆ ಅಷ್ಟೇ ಏನು? ಪ್ರತಿದಿನ ನಾವು ಈ ‘ಅನುಷ್ಠಾನ ಕ್ರೈಸ್ತಧರ್ಮ’ದ ಬಗ್ಗೆ ಕೇಳುತ್ತೇವೆ–ಒಬ್ಬನು ಕೆಲವು ಜನರಿಗೆ ಏನೋ ಸ್ವಲ್ಪ ಉಪಕಾರ ಮಾಡಿದ್ದಾನೆ ಎಂದು. ಅಷ್ಟೇ ಏನು? ಹಾಗಾದರೆ ಜೀವನದ ಗುರಿಯೇನು? ಈ ಇಹ ಜೀವನವೇ ನಮ್ಮ ಗುರಿಯೆ? ಮತ್ತೇನೂ ಇಲ್ಲವೆ? ನಾವು ಈಗಿರುವಷ್ಟಲ್ಲದೆ ಮತ್ತೇನೂ ಆಗುವುದಕ್ಕಿಲ್ಲವೆ? ಈ ಪ್ರಪಂಚವೇ ಎಲ್ಲ ಧರ್ಮಗಳ ಅತ್ಯುನ್ನತ ಆದರ್ಶವೆ? ಬಹುತೇಕ ಜನ ಈ ಜಗತ್ತಿನಲ್ಲಿ ಇನ್ನು ಯಾವ ರೋಗವೂ ಬಡತನವೂ ಕಷ್ಟವೂ ಇರದಂತಹ ದಿನವೊಂದನ್ನು ಪ್ರತೀಕ್ಷಿಸು ತ್ತಿದ್ದಾರೆ. ಆದ್ದರಿಂದ ಅನುಷ್ಠಾನಧರ್ಮ ಎಂಬುದರ ಅರ್ಥವಿಷ್ಟೆ–‘ರಸ್ತೆಯನ್ನು ಗುಡಿಸು, ಅದನ್ನು ಶುಚಿಗೊಳಿಸು!’–ಅಲ್ಲವೆ? ಆದರೆ ಇದನ್ನೇ ಮಾಡುತ್ತ ಎಲ್ಲರೂ ಹೇಗೆ ಸಂತೋಷ ಪಡುತ್ತಿದ್ದಾರೆಂದು ನಾವು ನೋಡುತ್ತಿಲ್ಲವೆ? ಹಾಗಾದರೆ ಈ ಸುಖವೇ ಜೀವನದ ಗುರಿಯೇನು? ಹಾಗೇನಾದರೂ ಆದರೆ, ಮನುಷ್ಯನಾಗುವುದು ಎಂಥಾ ದೊಡ್ಡ ತಪ್ಪು! ಹಾಗಾದರೆ ಕೇವಲ ಇಂದ್ರಿಯಸುಖವನ್ನು ಅರಸುವುದರಲ್ಲಿಯೇ ನಾವು ಈ ಮನುಷ್ಯ ಜನ್ಮದ ನೂರಾರು ವರ್ಷ ಗಳನ್ನು ವ್ಯರ್ಥಗೊಳಿಸಿದೆವೆಂದಾಯಿತಲ್ಲ!

“ಆದ್ದರಿಂದ, ಅನುಷ್ಠಾನಧರ್ಮದ ಈ ವಿವರಣೆಯು ಎಲ್ಲಿಗೆ ಕೊಂಡೊಯ್ಯುತ್ತದೆಂಬುದನ್ನು ನೋಡಿ! ಲೋಕೋಪಕಾರವು ಒಳ್ಳೆಯದೇ; ಆದರೆ ಅದೇ ಗುರಿಯೆನ್ನುವುದಾದರೆ ನೀವು ಚಾರ್ವಾಕರಾಗುವ ಅಪಾಯದಲ್ಲಿರುತ್ತೀರಿ. ಅದು ಧರ್ಮವಲ್ಲ. ಅದು ನಾಸ್ತಿಕವಾದಕ್ಕಿಂತಲೂ ಸ್ವಲ್ಪ ಮೇಲು ಅಷ್ಟೆ. ಸಹಜೀವಿಗಳಿಗಾಗಿ ಕೆಲಸ ಮಾಡುವುದು, ಆಸ್ಪತ್ರೆ ಕಟ್ಟಿಸುವುದು–ಇವು ಗಳನ್ನು ಬಿಟ್ಟು ನೀವು ಬೈಬಲ್ಲಿನಲ್ಲಿ ಬೇರೇನನ್ನಾದರೂ ಕಂಡಿರಾ? ಇಂತಹ ಧರ್ಮವೂ ಕೆಟ್ಟದ್ದೇನಲ್ಲ. ಒಳ್ಳೆಯದೇ. ಆದರದು ಬಾಲವಾಡಿಯ ಧರ್ಮ ಅಷ್ಟೆ. ಅದು ನಿಮ್ಮನ್ನೆಲ್ಲಿಗೂ ಮುಟ್ಟಿಸಲಾರದು. ಈ ಪ್ರಪಂಚದ ಒಟ್ಟು ಒಳಿತು ಕೆಡಕುಗಳನ್ನು ಎಣಿಸಿ ನೋಡಿ. ಯುಗಯುಗಾಂ ತರಗಳಿಂದಲೂ ಅದೇನಾದರೂ ಬದಲಾಗಿದೆಯೇ? ಪ್ರತಿಸಲವೂ ಈ ಜಗತ್ತು ಆಲೋಚಿಸಿತು– ‘ಓ, ಇನ್ನು ಈ ಸಮಸ್ಯೆಗಳೆಲ್ಲ ಪರಿಹಾರವಾದುವು’ ಎಂದು. ಆದರೆ ಅವೆಲ್ಲ ಹಾಗೆಯೇ ಉಳಿದಿವೆ. ಹೆಚ್ಚೆಂದರೆ ಅವುಗಳ ರೂಪ ಬದಲಾಗಿರಬಹುದಷ್ಟೆ. ಒಂದು ರಾಷ್ಟ್ರ ಶ್ರೀಮಂತವಾದರೆ ಮತ್ತೊಂದು ಬಡವಾಗುತ್ತದೆ. ಅಲೆಯ ಒಂದು ಭಾಗ ಉಬ್ಬಾದರೆ ಮತ್ತೊಂದು ತಗ್ಗಾಗಿರುತ್ತದೆ. ಆ ಬಗ್ಗೆ ನಾವು ಏನು ತಾನೇ ಮಾಡಬಲ್ಲೆವು? ಏನೇನೂ ಇಲ್ಲ...”

ಹಾಗಾದರೆ ನಾವು ಒಳ್ಳೆಯದನ್ನು ಮಾಡಲು ಪ್ರಯತ್ನಿಸಬಾರದೆ? ಸ್ವಾಮೀಜಿ ಹೇಳುತ್ತಾರೆ: “ಖಂಡಿತ ಮಾಡಿ; ಒಳ್ಳೆಯದಕ್ಕೋಸ್ಕರ ಒಳ್ಳೆಯದನ್ನು ಮಾಡಿ.” ಈ ಪ್ರಶ್ನೆಗೆ ಒಮ್ಮೆ ಅವರು ಭಾರತದಲ್ಲಿ ಇನ್ನೂ ಸ್ಪಷ್ಟವಾಗಿ ಉತ್ತರಿಸಿದ್ದರು. ಅವರ ಶಿಷ್ಯ ಶರಚ್ಚಂದ್ರ ಕೇಳಿದ್ದ: “ಈ ಕ್ಷಣಿಕ ಜಗತ್ತಿನಲ್ಲಿ ಏನು ಮಾಡಿದರೆ ತಾನೆ ಏನು ಪ್ರಯೋಜನ?” ಸ್ವಾಮೀಜಿ ಹೇಳಿದ್ದರು: “ಮಗು, ಸಾವು ತಪ್ಪಿದ್ದಲ್ಲವೆಂದ ಮೇಲೆ ಕ್ರಿಮಿಕೀಟಗಳಂತೆ ಸಾಯುವುದಕ್ಕಿಂತ ವೀರರಂತೆ ಸಾಯುವುದು ಒಳ್ಳೆಯದಲ್ಲವೆ? ಈ ನಶ್ವರ ವಿಶ್ವದಲ್ಲಿ ಒಂದೆರಡು ದಿನ ಹೆಚ್ಚಿಗೆ ಬದುಕಿದ್ದರೂ ಪ್ರಯೋಜನ ವೇನು? ತುಕ್ಕು ಹಿಡಿದು ಹಾಳಾಗುವುದಕ್ಕಿಂತ ಸವೆಸಿ ಮುಗಿಸುವುದು ಎಷ್ಟೋ ಮೇಲು– ಅದರಲ್ಲೂ ಇತರರಿಗಾಗಿ ತೃಣಮಾತ್ರವಾದರೂ ಒಳಿತನ್ನು ಮಾಡುತ್ತ.” ಮತ್ತೊಮ್ಮೆ ಇಂಥದೇ ಪ್ರಶ್ನೆಗೆ ಉತ್ತರಿಸುತ್ತಾರೆ: “ತನ್ನ ಒಳಿತನ್ನು ಸಾಧಿಸುವುದಕ್ಕೇ ಆದರೂ ಇತರರಿಗೆ ಒಳಿತನ್ನು ಮಾಡಬೇಕಾಗುತ್ತದೆ. ಇತರರ ಸೇವೆಗಾಗಿ ನಾವು ಈ ನಮ್ಮ ದೇಹವನ್ನು ಸಮರ್ಪಿಸಿದ್ದೇವೆಂದು ಭಾವಿಸಿದಾಗ ನಾವು ನಮ್ಮ ಅಹಮಿಕೆಯನ್ನು ಮರೆಯಲು ಸಾಧ್ಯವಾಗುತ್ತದೆ... ಎಂದರೆ ಇತರರ ಸೇವೆ ಮಾಡುವುದು ನಮ್ಮ ಆತ್ಮಸಾಕ್ಷಾತ್ಕಾರಕ್ಕೆ ಒಂದು ದಾರಿ.”

ನಿಜಕ್ಕೂ ಸ್ವಾಮೀಜಿಯವರ ಹೃದಯವನ್ನು ಮುಟ್ಟದ, ಮಿಡಿಯದ ಮಾನವನ ಮೊರೆ ಯಾವುದೂ ಇರಲಿಲ್ಲ. ಮಾನವನ ಉನ್ನತಿಗೆ ಸಂಬಂಧಿಸಿದ ಯಾವುದೇ ಅಂಶವನ್ನು ಅವರು ತಮಗೆ ಸಂಬಂಧಿಸಿದ್ದಲ್ಲ ಅಥವಾ ತಮ್ಮ ಹೊಣೆಗಾರಿಕೆಯಲ್ಲ ಎಂದು ಭಾವಿಸಿರಲಿಲ್ಲ. ಒಮ್ಮೆ ಅವರು ಶ್ರೀಮತಿ ಹ್ಯಾನ್ಸ್​ಬ್ರೋಗೆ ಹೇಳುತ್ತಾರೆ, “ನೋಡು, ನಾನು ಮತ್ತೊಮ್ಮೆ ಹುಟ್ಟಿ ಬರಬೇಕಾಗಬಹುದು. ಏಕೆಂದರೆ, ನಾನು ಮಾನವಪ್ರೇಮಕ್ಕೆ ಸಿಲುಕಿಕೊಂಡಿದ್ದೇನೆ \eng{!” “ I have fallen in love with man!”} ಆದರೆ ಮಾನವ ದುಃಖದ ಪರಿಹಾರಕ್ಕೆ ಒಂದೇ ಉಪಾಯ ವೆಂಬುದು ಅವರಿಗೆ ತಿಳಿದಿತ್ತು–ಜನರು ಆಧಾತ್ಮಿಕ ವ್ಯಕ್ತಿಗಳಾಗುವವರೆಗೆ, ಜನರ ಭಾವನೆ-ಕೃತಿ ಗಳು ಆಧ್ಯಾತ್ಮಿಕ ಆದರ್ಶದಿಂದ ಆವೃತವಾಗುವವರೆಗೆ ಮತ್ತು ಈ ಆಧ್ಯಾತ್ಮಿಕ ಆದರ್ಶವು ಆಧ್ಯಾತ್ಮಿಕಾನಂದದಿಂದ ನೈಜವಾಗಿ ಪರಿಣಮಿಸುವವರೆಗೆ ಭೂಮಿಯು ಸ್ವರ್ಗವಾಗಲಾರದೆಂ ಬುದು ಅವರಿಗೆ ತಿಳಿದಿತ್ತು.

ಹಾಗಾದರೆ ಅನುಷ್ಠಾನಧರ್ಮವೆಂದರೆ ನಿಜಕ್ಕೂ ಏನು? “ಅನುಷ್ಠಾನ ಧರ್ಮವೆಂದರೆ, ನನ್ನನ್ನು ನಾನು ಆತ್ಮನೆಂದರಿಯುವುದು. ನೀವು ನಿಮ್ಮನ್ನು ಲೌಕಿಕ ವಸ್ತುವಿನೊಂದಿಗೆ ತಾದಾತ್ಮ್ಯ ಗೊಳಿಸಿಕೊಳ್ಳುವ ತಪ್ಪು ಮಾಡುವುದನ್ನು ನಿಲ್ಲಿಸಿ. ನಿಮ್ಮನ್ನು ನೀವು ಆತ್ಮ ಎಂದರಿತುಕೊಳ್ಳ ದಿದ್ದರೆ, ನೀವು ಸಾವಿರ ಆಸ್ಪತ್ರೆಗಳನ್ನು ಕಟ್ಟಿಸಿದರೇನು? ಐವತ್ತು ಸಾವಿರ ರಸ್ತೆಗಳನ್ನು ನಿರ್ಮಿ ಸಿದರೇನು?... ನೀವು ದೇವರನ್ನು ನೋಡಲೇಬೇಕು. ಅದೇ ನಿಜವಾದ ಅನುಷ್ಠಾನ ಧರ್ಮ. ನೀವು ಯಾವುದನ್ನು ಅನುಷ್ಠಾನಧರ್ಮ ಎಂದು ಕರೆಯುವಿರೋ ಅದನ್ನಲ್ಲ ಕ್ರಿಸ್ತ ಬೋಧಿಸಿದ್ದು.... ಆತ್ಮ ಮಾತ್ರವೇ ನಾಶವಾಗದಿರುವುದು. ಎಲ್ಲ ಕೆಲಸಗಳೂ ಈ ಭಾವನೆಯತ್ತ ಅಭಿಮುಖ ವಾಗಿರುವವರೆಗೆ ಒಳ್ಳೆಯದೇ. ‘ನನಗೆ ಈ ಲೌಕಿಕ ಜಗತ್ತು ಬೇಕಾಗಿಲ್ಲ, ಇಂದ್ರಿಯಜೀವನ ಬೇಕಾಗಿಲ್ಲ. ನನಗೆ ಉನ್ನತವಾದದ್ದೇನೋ ಬೇಕು’–ಇದೇ ತ್ಯಾಗ.”

ಪಾಶ್ಚಾತ್ಯ ಜಗತ್ತು ಸ್ವಾಮೀಜಿಯವರ ಈ ಕರೆಗೆ ಇನ್ನೂ ಓಗೊಡಬೇಕಾಗಿದೆ. ಆದರೆ ಅವರು ಮಾತನಾಡಿದ್ದು ಒಂದು ದಿನವನ್ನು ದೃಷ್ಟಿಯಲ್ಲಿಟ್ಟುಕೊಂಡಲ್ಲ, ಒಂದು ಯುಗವನ್ನು ಪರಿಗಣಿಸಿ. ಒಂದು ಸಂದರ್ಭದಲ್ಲಿ ಅವರೇ ಹೇಳುತ್ತಾರೆ–“ನಾನು ಹೇಳುತ್ತಿರುವುದು ನಿಮಗೆ ಇಷ್ಟವಾಗ ದಿರಬಹುದು. ಇಂದು ನೀವು ನನ್ನನ್ನು ಶಪಿಸಬಹುದು. ಆದರೆ ಒಂದು ದಿನ ನೀವೇ ನನಗೆ ಶುಭ ಹಾರೈಸುತ್ತೀರಿ.”

ಈಸ್ಟರ್ ದಿನದ ಸಂಜೆ ಸ್ವಾಮೀಜಿ ಕೆಲವು ಶಿಷ್ಯರೊಂದಿಗೆ ತ್ಯಾಗ ಹಾಗೂ ಶಿಷ್ಯತ್ವದ ಬಗ್ಗೆ ಮತ್ತು ಗುರುವಿನಡಿಯಲ್ಲಿ ಶಿಷ್ಯನು ಸಂಪೂರ್ಣ ಶರಣಾಗಿರಬೇಕಾದುದರ ಬಗ್ಗೆ ಮಾತನಾಡಿದರು. ಈ ವಿಚಾರಗಳೆಲ್ಲ ಆ ಪಾಶ್ಚಾತ್ಯರಿಗೆ ತೀರ ಹೊಸತು, ನುಂಗಲಾರದ ತುತ್ತು. ಸ್ವಾಮೀಜಿ ಹೇಳಿ ದರು, “ನೀವು ನನ್ನ ಶಿಷ್ಯರಾಗಬಯಸುವುದಾದರೆ, ನಾನು ನಿಮಗೆ ಫಿರಂಗಿಯ ಬಾಯನ್ನು ಪ್ರವೇಶಿಸುವಂತೆ ಹೇಳಿದರೂ ನೀವು ಮರುಮಾತಿಲ್ಲದೆ ಅದನ್ನು ಪಾಲಿಸಬೇಕು.”

ಮತ್ತೊಂದು ದಿನ, ಊಟದ ಕೋಣೆಯಲ್ಲಿ ಸ್ವಾಮೀಜಿ ಹೀಗೆಯೇ ಸ್ಫೂರ್ತಿಭರಿತರಾಗಿ ಮಾತನಾಡಲಾರಂಭಿಸಿದರು. ಗಂಟೆಗಳೇ ಉರುಳಿದರೂ ಯಾರಿಗೂ ಸಮಯದ ಪರಿವೆಯೇ ಇಲ್ಲ. ಸ್ವಾಮೀಜಿ ಮಂಡಿಸಿದ ಆಧ್ಯಾತ್ಮಿಕ ವಿಚಾರಲಹರಿಯಲ್ಲಿ ಅಲ್ಲಿದ್ದವರೆಲ್ಲ ಹೇಗೆ ಮುಳುಗಿ ಹೋಗಿದ್ದರೆಂದರೆ ಬೆಳಗಿನಿಂದ ಅಪರಾಹ್ನ ಬಹಳ ಹೊತ್ತಿನವರೆಗೂ ಒಬ್ಬರೂ ಮಿಸುಕಾಡಲಿಲ್ಲ! ಇಂತಹ ಹಲವಾರು ಅನುಭವಗಳಲ್ಲಿ ಪಾಲ್ಗೊಂಡವನೊಬ್ಬ ಹೇಳುತ್ತಾನೆ, “ಗಂಟೆಗಟ್ಟಲೆ ಸ್ವಾಮೀಜಿ ಹೀಗೆ ಸತತವಾಗಿ ಮಾತನಾಡುತ್ತಿದ್ದಾಗ, ಎಲ್ಲಿ ಆ ಆಧ್ಯಾತ್ಮಿಕ ಪ್ರವಾಹಕ್ಕೆ ಅಡ್ಡಿ ಉಂಟಾಗುತ್ತದೆಯೋ ಎಂಬ ಭಯದಿಂದ ಎಲ್ಲರೂ ಮಿಸುಕಾಡದೆ ಕುಳಿತಲ್ಲೇ ಕುಳಿತಿರುತ್ತಿದ್ದರು; ಉಸಿರು ಬಿಗಿಹಿಡಿದು ಅವರ ಮಾತುಗಳನ್ನು ಆಲಿಸುತ್ತಿದ್ದರು. ಸ್ವಾಮೀಜಿಯವರ ವಾಕ್ ಪ್ರವಾಹ ದಲ್ಲಿ ಅವರೆಲ್ಲ ಕೊಚ್ಚಿಹೋಗುತ್ತಿದ್ದರು. ಅತ್ಯುನ್ನತ ಸ್ಥಳವೊಂದರಲ್ಲಿ ತೇಲಾಡುತ್ತಿರುವ ಅನು ಭವವಾಗುತ್ತಿತ್ತು ಅವರಿಗೆ. ಸ್ವಾಮೀಜಿ ಮೌನವಾದಾಗಲೇ ಮತ್ತೆ ಅವರಿಗೆಲ್ಲ ತಾವು ಈ ಲೋಕಕ್ಕೆ ಬಂಧಿತರಾಗಿರುವ ನೆನಪು.”

ಈಗಾಗಲೇ ಸ್ವಾಮೀಜಿಯವರು ಕ್ಯಾಲಿಫೋರ್ನಿಯದಲ್ಲಿ ಶಕ್ತಿಮೀರಿ ದುಡಿದಿದ್ದರು. ನಾಲ್ಕೂ ವರೆ ತಿಂಗಳ ಅವಿರತ ಪರಿಶ್ರಮದಿಂದ ಅವರ ದೇಹ ಬಳಲಿತ್ತು. ಅವರು ಅಮೆರಿಕೆಗೆ ಬಂದದ್ದರ ಮುಖ್ಯ ಉದ್ದೇಶವೇ ತಮ್ಮ ಆರೋಗ್ಯವನ್ನು ಸುಧಾರಿಸಿಕೊಳ್ಳುವುದು. ಆದರೆ ಇಲ್ಲಿ ತಾವು ನೆಟ್ಟಿದ್ದ ವೇದಾಂತದ ಸಸಿಯ ಬೆಳವಣಿಗೆಯನ್ನು ಗಮನಿಸಿ ಅದಕ್ಕೆ ನೀರೆರೆಯುವುದೂ ಅವರ ಮತ್ತೊಂದು ಉದ್ದೇಶವಾಗಿತ್ತು. ಇದಲ್ಲದೆ ಭಾರತದಲ್ಲಿ ಮಠಸ್ಥಾಪನೆಯ ಉದ್ದೇಶಕ್ಕಾಗಿ ಹಣ ಸಂಗ್ರಹದ ಕಾರ್ಯ ನಿಧಾನವಾಗಿ ಸಾಗುತ್ತಿತ್ತು. ಅಲಮೇಡದಲ್ಲಿ ‘ಧರ್ಮದ ಅನುಷ್ಠಾನ’ ಎಂಬ ಉಪನ್ಯಾಸದೊಂದಿಗೆ ಅಮೆರಿಕದ ತಮ್ಮ ಕೆಲಸಗಳನ್ನು ಮುಗಿಸಿಕೊಂಡು ಹಿಂದಿರುಗುವುದೆಂದು ಅವರು ನಿಶ್ಚಯಿಸಿದ್ದರು. ಆದರೆ ಕಡೇ ಘಳಿಗೆಯಲ್ಲಿ ತಮ್ಮ ನಿರ್ಧಾರವನ್ನು ಬದಲಿಸಬೇಕಾಯಿತು. ಏಕೆಂದರೆ ಅವರೀಗ ಮತ್ತೆ ಆರ್ಥಿಕ ಮುಗ್ಗಟ್ಟಿಗೆ ಸಿಲುಕಿದ್ದರು. ಬಹುಕಾಲದಿಂದಲೂ ಖೇತ್ರಿಯ ಮಹಾರಾಜ ಅಜಿತ್​ಸಿಂಗ್ ಅವರ ವೈಯಕ್ತಿಕ ಖರ್ಚಿಗಾಗಿ ಪ್ರತಿ ತಿಂಗಳೂ ಸ್ವಲ್ಪ ಹಣವನ್ನು ಕಳಿಸಿಕೊಡುತ್ತಿದ್ದ.\footnote{*ಇದಲ್ಲದೆ ಸ್ವಾಮೀಜಿಯವರ ವಿಶೇಷ ಕೋರಿಕೆಯಂತೆ ಅಜಿತ್​ಸಿಂಗನು ಅವರ ತಾಯಿ ಹಾಗೂ ತಮ್ಮಂದಿ ರಾದ ಮಹೇಂದ್ರನಾಥ ಮತ್ತು ಭೂಪೇಂದ್ರನಾಥರ ಊಟ ವಸತಿ ವಿದ್ಯಾಭ್ಯಾಸಗಳ ಖರ್ಚಿಗಾಗಿ ಅವರ ಮನೆಗೆ ಪ್ರತಿ ತಿಂಗಳೂ ನೂರು ರೂಪಾಯಿಗಳನ್ನು ಕಳಿಸುತ್ತಿದ್ದ. ಅಲ್ಲದೆ ಇವರಿಬ್ಬರ ವಿಷಯದಲ್ಲಿ ಸ್ವಂತ ಅಣ್ಣನಂತೆ ವಿಶ್ವಾಸವಿರಿಸಿದ್ದ. ‘ಸ್ವಾಮೀಜಿಯವರ’ ಸಂಸಾರ ಬೀದಿಗಿಳಿಯದಂತಾಗಲು ಇದರಿಂದ ಸಾಧ್ಯವಾಯಿತು. ‘ಅವರಿಗೆ ಸಾಧಾರಣ ಅನ್ನ ಬಟ್ಟೆಗೆ ಕೊರತೆಯಿರುವುದಿಲ್ಲ’ ಎಂಬ ಶ್ರೀರಾಮಕೃಷ್ಣರ ಮಾತು ಈ ಮೂಲಕ ಅಕ್ಷರಶಃ ಸತ್ಯವಾಯಿತು. ಮುಂದೆ ಮಹೇಂದ್ರನಾಥ ದತ್ತ ಹಾಗೂ ಭೂಪೇಂದ್ರನಾಥ ದತ್ತ ಇಬ್ಬರೂ ಉನ್ನತ ವಿದ್ಯಾಭ್ಯಾಸ ಪಡೆದರಲ್ಲದೆ ಸ್ವಾಮಿ ವಿವೇಕಾನಂದರಿಗೆ ತಕ್ಕ ಸೋದರರಾಗಿ ಕೀರ್ತಿವಂತರಾದರು.} ಈಗ ಅವನು ಅದನ್ನು ನಿಲ್ಲಿಸಿಬಿಟ್ಟಿದ್ದ. (ಈ ಹಣದಲ್ಲೂ ಬಹು ಪಾಲನ್ನು ಸ್ವಾಮೀಜಿ ಮಠಕ್ಕೆ ಕಳಿಸಿಬಿಡುತ್ತಿದ್ದರು. ತಮ್ಮ ಖರ್ಚಿಗಾಗಿ ಸಾಧ್ಯವಾದಷ್ಟು ಕಡಿಮೆ ಹಣವನ್ನು ಬಳಸುತ್ತಿದ್ದರು). ಬಹುಶಃ ಅಮೆರಿಕದ ಶ್ರೀಮಂತಭಕ್ತರು ಸ್ವಾಮೀಜಿಯವರಿಗೆ ಬೇಕೆಂಬಷ್ಟು ಹಣ ಕೊಡುತ್ತಿರಬಹುದೆಂದು ಆತ ಭಾವಿಸಿರಬೇಕು. ಅಂತೂ ಇದರಿಂದಾಗಿ ಸ್ವಾಮೀಜಿಯವರ ಮೇಲೆ ಆರ್ಥಿಕ ಒತ್ತಡ ಹೆಚ್ಚಾಯಿತು. ಆದರೆ ಇದಕ್ಕಾಗಿ ಅವರು ಅಜಿತ್​ಸಿಂಗನ ಮೇಲೆ ಒಂದಿನಿತೂ ಬೇಸರಿಸಿಕೊಳ್ಳಲಿಲ್ಲ. ಬದಲಾಗಿ ಸಾರಾಳಿಗೆ ಬರೆದ ಒಂದು ಪತ್ರದಲ್ಲಿ ಅವರು ಮಹಾರಾಜ ತಮಗೆ ಇದುವರೆಗೆ ಮಾಡಿದ ಉಪಕಾರವನ್ನು ಸ್ಮರಿಸಿ ಅವನ ಶ್ರೇಯಸ್ಸಿಗಾಗಿ ಮನಸಾರೆ ಹರಸಿದರು.

ಈಗ ಹಣ ಸಂಗ್ರಹಣೆಯ ಒಂದು ಗುರಿಯನ್ನು ಮುಟ್ಟಬೇಕೆಂಬ ಉದ್ದೇಶದಿಂದ ಅವರು ಅಲಮೇಡದಲ್ಲಿ ಮತ್ತೊಂದು ಉಪನ್ಯಾಸ ಮಾಲಿಕೆಯನ್ನು ಆರಂಭಿಸಿದರು. ಇದು ಭಕ್ತಿಯೋಗ ವನ್ನು ಕುರಿತದ್ದು. ವಿಷಯಗಳು–‘ಪೂಜ್ಯ ಮತ್ತು ಪೂಜಕ’, ‘ವಿಧ್ಯುಕ್ತ ಪೂಜೆ’ ಮತ್ತು ‘ಭಕ್ತಿ ಹಾಗೂ ಪ್ರೇಮ’.

ಶ್ರೀಮತಿ ಸಾರಾಳಿಗೆ ಬರೆದ ಪತ್ರದಲ್ಲಿ ಸ್ವಾಮೀಜಿ ತಾವು ಮತ್ತೊಂದು ಸರಣಿಯನ್ನು ಪ್ರಾರಂಭಿಸಬೇಕಾದುದರ ಬಗ್ಗೆ ತಿಳಿಸುತ್ತ ಹೇಳುತ್ತಾರೆ, “ನೀನು ಅದರ ಬಗ್ಗೆ (ಎಂದರೆ ತಮ ಗಾಗಬಹುದಾದ ಶ್ರಮದ ಬಗ್ಗೆ) ಸ್ವಲ್ಪವೂ ತಲೆ ಕೆಡಿಸಿಕೊಳ್ಳಬೇಡ. ನಾನೀಗ ನನ್ನ ಕಾಲ ಮೇಲೆ ನಿಂತಿದ್ದೇನೆ. ನಾನೀಗ ಬೆಳಕು ಕಾಣುತ್ತಿದ್ದೇನೆ. ನನ್ನ ಯಶಸ್ಸುಗಳು ನನ್ನನ್ನು ದಾರಿತಪ್ಪಿಸಿಬಿಡು ತ್ತಿದ್ದುವು. ಮತ್ತು ನಾನೊಬ್ಬ ಸಂನ್ಯಾಸಿಯೆಂಬ ವಾಸ್ತವಾಂಶವನ್ನು ನಾನು ಮರೆತುಬಿಡುತ್ತಿದ್ದೆ. ಆದ್ದರಿಂದಲೇ ಜಗನ್ಮಾತೆ ನನಗೆ ಈ ಅನುಭವವನ್ನು ಮಾಡಿಸುತ್ತಿದ್ದಾಳೆ.”

ಭಕ್ತಿಯೋಗವನ್ನು ಕುರಿತ ಈ ಉಪನ್ಯಾಸಗಳು ಅವರ ಯಾವುದೇ ಇತರ ಉಪನ್ಯಾಸದಂತೆಯೇ ಅತ್ಯಂತ ವಿಶಿಷ್ಟವಾಗಿದ್ದುವು. ಪಾಶ್ಚಾತ್ಯರಿಗೆ ಸುಪರಿಚಿತವಾದ ದ್ವೈತ ಸಿದ್ಧಾಂತವನ್ನೂ ಅವರು ತಮ್ಮದೇ ಆದ ರೀತಿಯಲ್ಲಿ ಮಂಡಿಸಿದರು.

ನಿಜಕ್ಕೂ ಈ ಉಪನ್ಯಾಸಗಳ ಕೆಲಸ, ವೇದಿಕೆಯನ್ನೇರುವ ಕೆಲಸ ಅವರಿಗೆ ಅಷ್ಟೇನೂ ಪ್ರಿಯವಾದದ್ದಾಗಿರಲಿಲ್ಲ ಎಂದು ತೋರುತ್ತದೆ. ಆದರೆ ಕಾರ್ಯಕ್ರಮ ನಿಗದಿಯಾಗಿತ್ತು; ಬೇರೆ ಉಪಾಯವಿಲ್ಲ. ಒಮ್ಮೆ ಶ್ರೀಮತಿ ಅಲನ್​ಗೆ ಅವರು ಹೇಳುತ್ತಾರೆ, “ನಾನು ಸಂಜೆ ಎಂಟು ಗಂಟೆಯ ವೇಳೆಗೆ ‘ಪ್ರೇಮ’ದ (‘ಭಗವತ್ಪ್ರೇಮ’ದ) ಬಗ್ಗೆ ಮಾತನಾಡಬೇಕಾಗಿದೆ. ಆದರೆ ಎಂಟು ಗಂಟೆಗೆ ನನ್ನಲ್ಲಿ ಪ್ರೇಮದ ಭಾವನೆಯೇ ಬರುವುದಿಲ್ಲ!” ಅವರು ಬಾಯಲ್ಲಿ ಹೀಗೆ ಹೇಳಿದರೂ ಒಮ್ಮೆ ಮಾತನಾಡಲಾರಂಭಿಸಿದರೆಂದರೆ ಆ ಮಾತುಗಳೆಲ್ಲ ಭಗವತ್​ಪ್ರೇಮಭರಿತ ಹೃದಯ ದಿಂದಲೇ ಹೊಮ್ಮಿಬರುತ್ತಿದ್ದಂತೆ ಕಾಣುತ್ತಿತ್ತು.

ಈ ದಿನಗಳಲ್ಲಿ ನಡೆದ ಒಂದು ಅತಿ ಮುಖ್ಯ ಘಟನೆಯೆಂದರೆ, ಸ್ಯಾನ್​ಫ್ರಾನ್ಸಿಸ್ಕೋದಲ್ಲಿ ‘ವೇದಾಂತ ಸೊಸೈಟಿ’ಯೊಂದರ ಉದಯ. ಪಸಾಡೆನ ಹಾಗೂ ಸ್ಯಾನ್​ಫ್ರಾನ್ಸಿಸ್ಕೋಗಳಲ್ಲಿ ಇಂತಹ ಒಂದೊಂದು ಸಂಘಗಳು ಹುಟ್ಟಿಕೊಂಡುವು. ಇವುಗಳ ಉಳಿವಿನ ಬಗ್ಗೆ ಸ್ವಾಮೀಜಿ ಯವರಿಗೇ ತೀವ್ರ ಶಂಕೆಯಿತ್ತು. ಆದರೆ ಅವರು ತಮ್ಮ ಆಧ್ಯಾತ್ಮಿಕ ಶಕ್ತಿಯನ್ನು ಬಸಿದು ಹರಿಸಿದ್ದ ಈ ಸ್ಥಳಗಳಲ್ಲಿ ಅದು ವ್ಯರ್ಥವಾಗಿ ಹೋಗದಂತೆ ಸಂಘಗಳನ್ನು ನಿರ್ಮಿಸುವುದು ಅವರ ಮಹಾಕಾರ್ಯದ ಒಂದು ಅಂಗವೇ ಆಗಿತ್ತು.

ಹೀಗೆ ಎಡೆಬಿಡದೆ ನಡೆದ ಕೆಲಸಗಳೂ ಈಗ ಕೊನೆಗೊಳ್ಳುವ ಸೂಚನೆ ಕಂಡುಬಂದಾಗ, ಸ್ವಾಮೀಜಿಯವರು ಸ್ವಲ್ಪ ವಿಶ್ರಾಂತಿಗಾಗಿ ತಮ್ಮ ಆಪ್ತ ಶಿಷ್ಯರೊಂದಿಗೆ ಸಮೀಪದ ಕ್ಯಾಂಪ್ ಟೈಲರ್ ಎಂಬ ಸ್ಥಳಕ್ಕೆ ಭೇಟಿ ನೀಡಿದರು. ಇಲ್ಲಿ ಅವರು ಮೇ ೨ರಿಂದ ಸುಮಾರು ಹದಿನೈದು ದಿನ ಉಳಿದುಕೊಂಡರು. ಇಲ್ಲೊಂದು ಹೊಳೆ ಹರಿಯುತ್ತಿದೆ; ವಾತಾವರಣ ಪರಮ ಪ್ರಶಾಂತ ವಾಗಿದೆ. ಸ್ವಾಮೀಜಿ ಇಲ್ಲಿನ ಜಗತ್ಪ್ರಸಿದ್ಧವಾದ ದೈತ್ಯಾಕಾರದ ರೆಡ್​ವುಡ್ ವೃಕ್ಷಗಳ ನೆರಳಿನಲ್ಲಿ ತಮ್ಮ ಶಿಷ್ಯರೊಂದಿಗೆ ಹರಟುತ್ತ, ಕೆಲವೊಮ್ಮೆ ಗಂಭೀರವಾದ ಮಾತುಕತೆಗಳಲ್ಲಿ ತೊಡಗುತ್ತ, ಇಲ್ಲವೆ ಧ್ಯಾನ ಮಾಡುತ್ತ ಆನಂದದಿಂದ ದಿನ ಕಳೆದರು.

ಒಂದು ದಿನ ಆ ಹೊಳೆಯ ಬದಿಯಲ್ಲಿ ಕೆಲವು ಹುಡುಗರು ಆಟವಾಡುತ್ತಿರುವುದನ್ನು ಸ್ವಾಮೀಜಿ ನೋಡಿದರು. ಆ ಹುಡುಗರು ಮೊಟ್ಟೆಯ ಚಿಪ್ಪುಗಳನ್ನು ಹರಿಯುವ ನೀರಿನ ಮೇಲೆ ತೇಲಿಬಿಟ್ಟು ಅವುಗಳಿಗೆ ಬಂದೂಕಿನಿಂದ ಗುರಿಯಿಟ್ಟು ಗುಂಡು ಹೊಡೆಯುವ ಪ್ರಯತ್ನದಲ್ಲಿ ದ್ದರು. ಆದರೆ, ಪಾಪ, ಆ ಪ್ರಯತ್ನದಲ್ಲಿ ಯಾರೂ ಯಶಸ್ವಿಗಳಾದಂತೆ ಕಾಣಲಿಲ್ಲ. ಸ್ವಾಮೀಜಿ ಇದನ್ನು ಮುಗುಳ್ನಗುತ್ತ ಗಮನಿಸುತ್ತಿದ್ದರು. ಇದನ್ನು ಕಂಡ ಒಬ್ಬ ಹುಡುಗ, “ಅದು ನೀವಂದು ಕೊಂಡಷ್ಟು ಸುಲಭವಲ್ಲ ಸ್ವಾಮಿ! ಬೇಕಾದರೆ ಹೊಡೆದು ನೋಡಿ” ಎಂದು ಪಂಥವೊಡ್ಡಿದ. ಸ್ವಾಮೀಜಿ ಪಂಥಕ್ಕೆ ನಿಂತೇಬಿಟ್ಟರು. ಅವರು ಆ ಹುಡುಗನ ಕೈಯಿಂದ ಬಂದೂಕವನ್ನು ತೆಗೆದುಕೊಂಡು ಒಂದೇ ಸಮನೆ ಸುಮಾರು ಹತ್ತು ಹನ್ನೆರಡು ಚಿಪ್ಪುಗಳನ್ನು ಹೊಡೆದುಬಿಟ್ಟರು! ಹುಡುಗರಿಗೆ ಆಶ್ಚರ್ಯವೋ ಆಶ್ಚರ್ಯ. ಆದರೆ ಅವರೆಲ್ಲ ಮಾತನಾಡಿಕೊಂಡರು, “ಓ, ಇವರಿ ಗೆಲ್ಲೋ ಬಂದೂಕು ಹಿಡಿದು ಚೆನ್ನಾಗಿ ಅಭ್ಯಾಸವಿರಬೇಕು” ಎಂದು. ಇದನ್ನು ಕೇಳಿದ ಸ್ವಾಮೀಜಿ ತಮ್ಮ ಗುಟ್ಟನ್ನು ಬಿಟ್ಟುಕೊಟ್ಟರು: “ನೋಡಿ ಮಕ್ಕಳೆ, ನಾನು ಇಂದಿನವರೆಗೂ ಬಂದೂಕು ಹಿಡಿದವನೇ ಅಲ್ಲ. ನಾನು ನನ್ನ ಮನಸ್ಸನ್ನು ಏಕಾಗ್ರಗೊಳಿಸಿದೆ; ಏಕಾಗ್ರತೆಯಿಂದ ಹೊಡೆದ ಗುಂಡುಗಳೆಲ್ಲ ನೇರವಾಗಿ ಗುರಿಮುಟ್ಟಿದುವು. ಅಷ್ಟೆ. ಮನಸ್ಸಿನ ಏಕಾಗ್ರತೆಯೊಂದಿದ್ದುಬಿಟ್ಟರೆ ಏನನ್ನು ಬೇಕಾದರೂ ಸಾಧಿಸಬಹುದು. ಈ ಏಕಾಗ್ರತೆಯೇ ನನ್ನ ಯಶಸ್ಸಿನ ಗುಟ್ಟು.”

 ಸುಮಾರು ಹದಿನೈದು ದಿಗಳ ಬಳಿಕ ಸ್ವಾಮೀಜಿ ಸ್ಯಾನ್​ಫ್ರಾನ್ಸಿಸ್ಕೋಗೆ ಹಿಂದಿರುಗಿದರು. ಆದರೆ ಅವರ ಆರೋಗ್ಯ ಹೆಚ್ಚೇನೂ ಸುಧಾರಿಸಿರಲಿಲ್ಲ. ಆದ್ದರಿಂದ ಅವರು ತಮ್ಮ ಶಿಷ್ಯನಾದ ಡಾ ॥ ಲೋಗನ್ ಎಂಬವನ ಮನೆಯಲ್ಲಿ ಉಳಿದುಕೊಂಡರು. ಒಂದು ವಾರದ ಕಾಲ ಅವರು ಯಾವ ಕಾರ್ಯಕ್ರಮದಲ್ಲೂ ಭಾಗವಹಿಸಲಿಲ್ಲ. ಆದರೆ ಮೇ ೨೪ರಂದು ಅಲ್ಲಿನ ವೇದಾಂತ ಸೊಸೈಟಿಯಲ್ಲಿ ಭಗವದ್ಗೀತೆಯ ಬಗ್ಗೆ ಮಾತನಾಡಿದರು. ಶ್ರೀಕೃಷ್ಣನ ಸಂದೇಶದ ಬಗ್ಗೆ ಮಾತನಾ ಡುತ್ತ ಅವರೆಂದರು:

“ಎದ್ದು ನಿಲ್ಲಿ, ಹೋರಾಡಿ. ಒಂದು ಹೆಜ್ಜೆಯನ್ನೂ ಹಿಂದಿಡಬೇಡಿ. ನಕ್ಷತ್ರಗಳೇ ಉದುರಿ ಬೀಳಲಿ! ಇಡೀ ಪ್ರಪಂಚವೇ ನಿಮಗೆ ಎದುರಾಗಿ ನಿಲ್ಲಲಿ! ಸಾವೆಂದರೆ ಒಂದು ಬಟ್ಟೆಯನ್ನು ಬದಲಿಸಿದಂತಷ್ಟೇ. ಅದರಿಂದೇನು? ಆದ್ದರಿಂದ ಹೋರಾಡಿ. ಹೇಡಿಗಳಾಗುವುದರಿಂದ ನಿಮ ಗೇನೂ ಲಾಭವಿಲ್ಲ. ನೀವು ಜಗತ್ತಿನ ಎಲ್ಲ ದೇವರಿಗೂ ಮೊರೆಯಿಟ್ಟಾಯಿತು. ಕಷ್ಟಗಳೇನಾದರೂ ದೂರವಾದುವೆ? ಭಾರತದ ಜನ ಅರವತ್ತು ಕೋಟಿ ದೇವತೆಗಳಿಗೆ ಮೊರೆಯಿಡುತ್ತಾರೆ. ಆದರೂ ನಾಯಿಗಳಂತೆ ಸಾಯುತ್ತಿರುತ್ತಾರೆ. ಎಲ್ಲಿದ್ದಾರೆ ಆ ದೇವತೆಗಳೆಲ್ಲ? ನಿಮ್ಮದೆಲ್ಲ ಮುಗಿದ ಮೇಲೆ ಆ ದೇವತೆಗಳು ನಿಮ್ಮ ಸಹಾಯಕ್ಕೆ ಬರುತ್ತಾರೆ. ಅದರಿಂದೇನು ಪ್ರಯೋಜನ! ವೀರಮರಣ ವನ್ನಪ್ಪಿ. ಮೂಢನಂಬಿಕೆಗಳಿಗೆ ತಲೆಬಾಗುವುದು! ನಿಮ್ಮನ್ನೇ ನಿಮ್ಮ ಆಸೆಗೆ ಮಾರಿಕೊಳ್ಳುವುದು! –ಇದು ನಿಮಗೆ ತಕ್ಕುದಲ್ಲ. ನೀವು ಅನಂತಸ್ವರೂಪರು, ಜನನವಿಲ್ಲದವರು, ಮರಣವಿಲ್ಲ ದವರು. ಗುಲಾಮರಾಗಿರುವುದು ನಿಮಗೆ ಹೇಳಿಸಿದ್ದಲ್ಲ... ಏಳಿ! ಎಚ್ಚರಗೊಳ್ಳಿ! ಎದ್ದು ನಿಂತು ಹೋರಾಡಿ. ಸಾಯಲೇಬೇಕಾಗಿ ಬಂದರೆ ಸಾಯಿರಿ. ನಿಮಗೆ ನೆರವಾಗಬಲ್ಲವರು ಯಾರೂ ಇಲ್ಲ. ನೀವೇ ಸಕಲ ಜಗತ್ತಾಗಿರುವಾಗ ನಿಮಗೆ ನೆರವಾಗಲು ಯಾರಿಂದಾದೀತು?”

ಮೇ ತಿಂಗಳ ವೇಳೆಗೆ ಕ್ಯಾಲಿಫೋರ್ನಿಯದಲ್ಲಿ ಸ್ವಾಮೀಜಿಯವರ ಕೆಲಸ ಒಂದು ಹಂತವನ್ನು ಮುಟ್ಟಿ ತೃಪ್ತಿಕರವಾಗಿ ಸಾಗತೊಡಗಿತ್ತು. ಅತ್ತ ಲಾಸ್ ಏಂಜಲಿಸ್ ಹಾಗೂ ಪಸಾಡೆನಗಳ ವೇದಾಂತ ಸೊಸೈಟಿಗಳ ಸದಸ್ಯರು ನಿಯತವಾಗಿ ಸಭೆ ಸೇರುತ್ತಿದ್ದರು. ಇಲ್ಲಿಗೆ ಮತ್ತೊಮ್ಮೆ ಭೇಟಿ ಕೊಡಬೇಕು ಎಂದು ಸ್ವಾಮೀಜಿಯವರನ್ನು ಒತ್ತಾಯಿಸುವ ಪತ್ರಗಳು ಬರಲಾರಂಭಿಸಿದ್ದುವು. ಆದರೆ ಇತ್ತ ಉತ್ತರ ಕ್ಯಾಲಿಫೋರ್ನಿಯದಲ್ಲಿ ತುಂಬ ಕಾರ್ಯಚಟುವಟಿಕೆಗಳಲ್ಲಿ ಮುಳುಗಿದ್ದ ಅವರಿಗೆ ಹೊರಟುಬರಲು ಸಾಧ್ಯವಿರಲಿಲ್ಲ. ಇಲ್ಲಿಯ ಓಕ್​ಲ್ಯಾಂಡ್ ಹಾಗೂ ಅಲಮೇಡಗಳಲ್ಲೂಅನೌಪಚಾರಿಕ ಅಧ್ಯಯನ ಕೇಂದ್ರಗಳು ಪ್ರಾರಂಭವಾಗಿದ್ದುವು.

\noindent

ವೇದಾಂತ ಸೊಸೈಟಿಗಳ ಸದಸ್ಯರು ನಿಯತವಾಗಿ ಸಭೆ ಸೇರುತ್ತಿದ್ದರು. ಇಲ್ಲಿಗೆ ಮತ್ತೊಮ್ಮೆ ಭೇಟಿ ಕೊಡಬೇಕು ಎಂದು ಸ್ವಾಮೀಜಿಯವರನ್ನು ಒತ್ತಾಯಿಸುವ ಪತ್ರಗಳು ಬರಲಾರಂಭಿಸಿದ್ದುವು. ಆದರೆ ಇತ್ತ ಉತ್ತರ ಕ್ಯಾಲಿಫೋರ್ನಿಯದಲ್ಲಿ ತುಂಬ ಕಾರ್ಯಚಟುವಟಿಕೆಗಳಲ್ಲಿ ಮುಳುಗಿದ್ದ ಅವರಿಗೆ ಹೊರಟುಬರಲು ಸಾಧ್ಯವಿರಲಿಲ್ಲ. ಇಲ್ಲಿಯ ಓಕ್​ಲ್ಯಾಂಡ್ ಹಾಗೂ ಅಲಮೇಡ ಗಳಲ್ಲೂಅನೌಪಚಾರಿಕ ಅಧ್ಯಯನ ಕೇಂದ್ರಗಳು ಪ್ರಾರಂಭವಾಗಿದ್ದುವು.

ಈ ವೇಳೆಗೆ, ಜುಲೈ ತಿಂಗಳಲ್ಲಿ ತಮ್ಮನ್ನು ಪ್ಯಾರಿಸಿನಲ್ಲಿ ಕೂಡಿಕೊಳ್ಳುವಂತೆ ಸ್ವಾಮೀಜಿಯವ ರಿಗೆ ಆಗ ಲಂಡನ್ನಿನಲ್ಲಿದ್ದ ಫ್ರಾನ್ಸಿಸ್ ಲೆಗಟ್ ದಂಪತಿಗಳಿಂದ ಒತ್ತಾಯಪೂರ್ವಕ ಆಹ್ವಾನ ಬಂದಿತು. ಅದೇ ಸಮಯದಲ್ಲಿ ಪ್ಯಾರಿಸಿನಲ್ಲಿ ಅಂತರರಾಷ್ಟ್ರೀಯ ಪ್ರದರ್ಶನವೊಂದು ನಡೆಯು ತ್ತಿದ್ದು ಇದಕ್ಕೆ ಸಂಬಂಧಿಸಿದಂತೆ ‘ವಿಶ್ವಧರ್ಮಚರಿತ ಮೇಳ’(ವಿಶ್ವದ ಸಕಲ ಧರ್ಮಗಳ ಪ್ರಾರಂಭ-ಬೆಳವಣಿಗೆಗಳ ಅಧ್ಯಯನಕ್ಕೆ ಸಂಬಂಧಿಸಿದ ಸಮ್ಮೇಳನ)ವೊಂದನ್ನು ಏರ್ಪಡಿಸಲಾ ಗಿತ್ತು. ಜಗತ್ತಿನ ಪ್ರಮುಖ ವಿದ್ವಾಂಸರು, ಪಂಡಿತರು ಭಾಗವಹಿಸಲಿದ್ದ ಈ ಸಮ್ಮೇಳನದಲ್ಲಿ ಹಿಂದೂಧರ್ಮವನ್ನು ಪ್ರತಿನಿಧಿಸುವಂತೆ ಅದರ ಸಂಘಟಕರು ಸ್ವಾಮೀಜಿಯವರನ್ನು ವಿನಂತಿಸಿ ಕೊಂಡರು. ಇವರೆಡು ಆಹ್ವಾನಗಳಿಗೂ ಸ್ವಾಮೀಜಿ ಸಮ್ಮತಿಸಿದರು. ಆದರೆ ಯೂರೋಪಿಗೆ ತೆರಳುವುದಕ್ಕೆ ಮೊದಲು ನ್ಯೂಯಾರ್ಕಿನಲ್ಲಿ ಕೆಲವು ವಾರಗಳನ್ನು ಕಳೆಯಲು ಅವರು ನಿಶ್ಚಯಿಸಿ ದರು. ಅದರಂತೆ ಮೇ ೨೯ರಂದು ಡಾ ॥ ಲೋಗನ್ನರ ಮನೆಯಲ್ಲಿ ತಮ್ಮ ಕೊನೆಯ ಉಪನ್ಯಾಸವನ್ನು ನೀಡಿ ನ್ಯೂಯಾರ್ಕಿಗೆ ಹೊರಟುನಿಂತರು. ಹೊರಡುವ ಮುನ್ನ ಅವರು ತಮ್ಮ ಹಲವಾರು ಶಿಷ್ಯರು-ಸ್ನೇಹಿತರಿಂದ ಬೀಳ್ಕೊಳ್ಳುತ್ತ, “ನಾನು ನಿಮ್ಮಲ್ಲಿಗೆ ಮತ್ತೊಬ್ಬರನ್ನು–ನನ ಗಿಂತ ಉತ್ತಮ ವ್ಯಕ್ತಿಯೊಬ್ಬರನ್ನು ಕಳಿಸಿಕೊಡುತ್ತೇನೆ. ಅವರ ಹೆಸರು ಸ್ವಾಮಿ ತುರೀಯಾನಂದ ಎಂದು. ನಾನು ಏನು ಮಾತನಾಡುತ್ತೇನೋ ಅವರದನ್ನು ಮಾಡುತ್ತಾರೆ” ಎಂದು ಹೇಳಿದರು.

ಸ್ವಾಮೀಜಿ ಈಗ ತಮ್ಮಿಂದ ಕಟ್ಟಕಡೆಯದಾಗಿ ಬೀಳ್ಗೊಳ್ಳುತ್ತಿದ್ದಾರೆ ಎಂದು ಅಲ್ಲಿ ನೆರೆದಿ ದ್ದವರಲ್ಲಿ ಯಾರೂ ಭಾವಿಸಿರಲಿಲ್ಲ. ನಿಜಕ್ಕೂ ಸ್ವಾಮೀಜಿ ಸ್ಯಾನ್​ಫ್ರಾನ್ಸಿಸ್ಕೋ ಪ್ರದೇಶಕ್ಕೆ– ಅಲ್ಲಿನ ಜನರಿಗೆ–ಮನಸೋತಿದ್ದರು. ವೇದಾಂತದ ಬೀಜವನ್ನು ಬಿತ್ತಲು ಅಮೆರಿಕದ ಪಶ್ಚಿಮ ಕರಾವಳಿಯ ಈ ಭಾಗ ಅತ್ಯಂತ ಫಲವತ್ತಾದದ್ದು ಎಂದು ಅವರು ಕಂಡುಕೊಂಡಿದ್ದರು.

ಮೇ ೩ಂರಂದು ಸ್ವಾಮೀಜಿಯವರು ಓಕ್​ಲ್ಯಾಂಡಿನಲ್ಲಿ ಟ್ರೈನು ಹತ್ತಿ ಅಮೆರಿಕದ ಪೂರ್ವ ತೀರಕ್ಕೆ ಪ್ರಯಾಣ ಬೆಳೆಸಿದರು. ಮೂರು ದಿನಗಳ ಸುದೀರ್ಘ ಪ್ರಯಾಣ ತುಂಬ ಪ್ರಯಾಸಕರ ವಾಗಿತ್ತು. ಈಗ ಅವರು ಶಿಕಾಗೋ ನಗರಕ್ಕೆ ಬಂದು ತಲುಪಿದರು. ಇಲ್ಲಿ ತಮ್ಮ ಹಳೆಯ ಸ್ನೇಹಿತರನ್ನೆಲ್ಲ ಮತ್ತೆ ಭೇಟಿಯಾದರು. ಪಾಶ್ಚಾತ್ಯ ಶಿಷ್ಯರಲ್ಲೆಲ್ಲ ತಮಗೆ ಅತ್ಯಂತ ಪ್ರಿಯಳಾದ ಮೇರಿಯನ್ನೂ ಕೊನೆಯ ಬಾರಿಗೆ ಸಂಧಿಸಿದರು.

ಈ ಅವಧಿಯಲ್ಲಿ ಒಂದು ಹೃದಯಸ್ಪರ್ಶಿ ಘಟನೆ ನಡೆಯಿತು. ಸ್ವಾಮೀಜಿಯವರು ಶಿಕಾಗೋ ದಿಂದ ಹೊರಡುವ ದಿನ ಬೆಳಿಗ್ಗೆ ಮೇರಿ ಅವರ ಕೋಣೆಗೆ ಬಂದು ನೋಡುತ್ತಾಳೆ–ಅವರು ಅದೇಕೋ ದುಃಖಿತರಾಗಿರುವಂತೆ ಕಂಡುಬರುತ್ತಿದೆ; ಅವರ ಹಾಸಿಗೆ ಸ್ವಲ್ಪವೂ ನಲುಗಿದಂತೆ ಕಾಣುತ್ತಿಲ್ಲ... ಆ ಬಗ್ಗೆ ಮೇರಿ ಕೆದಕಿ ಕೇಳಿದಾಗ, ತಾವು ರಾತ್ರಿಯಿಡೀ ನಿದ್ರೆ ಮಾಡಲೇ ಇಲ್ಲ ಎಂದು ಸ್ವಾಮೀಜಿ ಒಪ್ಪಿಕೊಂಡರು. ಬಳಿಕ ಪಿಸುದನಿಯಲ್ಲಿ ಸ್ವಗತವೋ ಎಂಬಂತೆ ಉಸುರಿದರು –“ಓಹ್! ಈ ಮಾನವ ಬಂಧನಗಳನ್ನು ಕತ್ತರಿಸುವುದು ಎಷ್ಟು ಕಷ್ಟ!”

ಶಿಕಾಗೋದಿಂದ ಅವರು ಮತ್ತೆ ರೈಲಿನಲ್ಲಿ ಪಯಣಿಸಿ ಜೂನ್ ಏಳರಂದು ನ್ಯೂಯಾರ್ಕನ್ನು ತಲುಪಿದರು. ಆರು ತಿಂಗಳ ಬಳಿಕ ಈಗ ತಮ್ಮ ಪ್ರಿಯ ಗುರುಭಾಯಿಗಳಾದ ಅಭೇದಾನಂದರನ್ನು ಹಾಗೂ ತುರೀಯಾನಂದರನ್ನು ಮತ್ತು ಶಿಷ್ಯೆ ನಿವೇದಿತೆಯನ್ನು ಭೇಟಿ ಮಾಡಿ ಆನಂದಿತರಾದರು. ಅವರ ಇತರ ಅನೇಕ ಹಳೆಯ ಶಿಷ್ಯರೂ ವೇದಾಂತ ಸೊಸೈಟಿಯ ಸದಸ್ಯರೂ ಅವರನ್ನು ಸಂಭ್ರಮದಿಂದ ಬರಮಾಡಿಕೊಂಡರು.

ನ್ಯೂಯಾರ್ಕಿನಲ್ಲಿ ಸ್ವಾಮೀಜಿ ಸಾರ್ವಜನಿಕ ಉಪನ್ಯಾಸದ ಕೆಲಸವನ್ನು ಹಮ್ಮಿಕೊಳ್ಳಲಿಲ್ಲ. ನಿರಂತರ ಶ್ರಮದಿಂದ ಬಳಲಿದ್ದ ಅವರು ಸ್ವಲ್ಪ ವಿಶ್ರಮಿಸಿಕೊಳ್ಳಲು ಇಷ್ಟಪಟ್ಟರು. ಆದರೆ ಒಂದು ತಿಂಗಳ ಅವಧಿಯಲ್ಲಿ ಭಗವದ್ಗೀತೆಯ ವಿಷಯವಾಗಿ ಕೆಲವು ತರಗತಿಗಳನ್ನು ತೆಗೆದುಕೊಂಡರು. ಮತ್ತು ವೇದಾಂತ ಸೊಸೈಟಿಯ ಕಟ್ಟಡದಲ್ಲೇ ಭಾನುವಾರಗಳಂದು ಉಪನ್ಯಾಸಗಳನ್ನು ನೀಡಿದರು. ಉಪನ್ಯಾಸಗಳಿಗೆ ಜನರು ಕಿಕ್ಕಿರಿದು ಸೇರುತ್ತಿದ್ದರು. ಈ ಎಲ್ಲ ಉಪನ್ಯಾಸಗಳಿಗೂ ತಪ್ಪದೆ ಹಾಜರಿದ್ದ ನಿವೇದಿತೆ ತನ್ನ ಗುರುದೇವನ ಒಂದೊಂದು ಮಾತನ್ನೂ ಆಸ್ವಾದಿಸಿದಳು. ಅವರ ಮಾತುಗಳನ್ನು ಆಕೆ ಈ ಹಿಂದೆ ಅದೆಷ್ಟು ಸಲ ಕೇಳಿದ್ದಳೋ! ಅವರ ಸಾನ್ನಿಧ್ಯದ ಸೌಭಾಗ್ಯವನ್ನು ಅದೆಷ್ಟು ಸಲ ಪಡೆದು ಆನಂದಿಸಿದ್ದಳೋ! ಆದರೆ ಈಗ ಅವರ ಮಾತಿನಲ್ಲಿ, ಅವರ ವ್ಯಕ್ತಿತ್ವದಲ್ಲಿ ಅದೇ ಆಕರ್ಷಣೆಯನ್ನು ಅಥವಾ ಬಹುಶಃ ಅದಕ್ಕಿಂತ ಹೆಚ್ಚಿನದನ್ನು ಕಾಣುತ್ತಿದ್ದಾಳೆ. ದಿನದಿನಕ್ಕೂ ಕ್ಷಣಕ್ಷಣಕ್ಕೂ ಆಕೆಯ ಪಾಲಿಗೆ ಸ್ವಾಮೀಜಿ ಮತ್ತಷ್ಟು ಹಿರಿದಾಗಿ, ಮತ್ತಷ್ಟು ನಿಗೂಢವಾಗಿ ಕಂಡುಬಂದರು. ತನ್ನ ಈ ಆನಂದವನ್ನು ಹಿಡಿದಿಟ್ಟುಕೊಳ್ಳಲಾರದೆ ಆಕೆ ಜೋಸೆಫಿನ್ನಳಿಗೆ ಬರೆಯುತ್ತಾಳೆ–

“... ನಾನು ಸ್ವಾಮೀಜಿಯವರ ಉಪನ್ಯಾಸವನ್ನು ಆಲಿಸಲು ಕಾತರಳಾಗಿ ನನ್ನ ಎಂದಿನ ಜಾಗದಲ್ಲಿ ಕುಳಿತಿದ್ದೆ... ಅವರು ಪ್ರವೇಶಿಸಿದರು. ಅವರು ಪ್ರವೇಶಿಸಿದ್ದು, ಬಂದು ಮೌನವಾಗಿ ನಿಂತದ್ದು, ಮಾತನಾಡುವ ಮುನ್ನ ಒಂದು ಕ್ಷಣ ತಡೆದದ್ದು–ಆಹ್! ಅದೇ ಒಂದು ಮಂತ್ರ, ಅದೇ ಒಂದು ಪೂಜೆ!

“ಅಂತೂ ಕಡೆಗೆ ಅವರು ಪ್ರಾರಂಭಿಸಿದರು–ಇದ್ದಕ್ಕಿದ್ದಂತೆ ಅವರ ಮುಖದಲ್ಲಿ ಪ್ರಕಾಶ ಹೊಮ್ಮಿತು. ‘ಇಂದಿನ ಭಾಷಣದ ವಿಷಯವೇನು?’ ಎಂದು ಕೇಳಿದರು. ‘ವೇದಾಂತ ತತ್ತ್ವ’ ಎಂದು ಯಾರೋ ಸಲಹೆ ಮಾಡಿದರು. ಸರಿ, ಸ್ವಾಮೀಜಿ ಪ್ರಾರಂಭಿಸಿದರು.

“... ಆ ಅದ್ಭುತ ವಾಕ್ಯಗಳು ಒಂದಾದ ಮೇಲೊಂದು ಉರುಳಿದುವು. ಅನಂತತೆಗೆ ಮೇಲೆತ್ತಲ್ಪಟ್ಟಿದ್ದ ನಾವು, ನಮ್ಮ ಎಂದಿನ ‘ನಾನು’ಗಳ ಬಗ್ಗೆ ಆಲೋಚಿಸಿದಾಗ, ಸೂರ್ಯನ ಕಡೆಗೋ ಚಂದ್ರದ ಕಡೆಗೋ ಅವುಗಳನ್ನು ಆಟಿಕೆಗಳೆಂದು ಭಾವಿಸಿ ಕೈಚಾಚುವ ಮಕ್ಕಳು ನಾವೆಂಬಂತೆ ತೋರಿತು. ಸ್ವಾಮೀಜಿಯವರ ಆ ಆಶ್ಚರ್ಯಕರ ಧ್ವನಿ ಹಾಗೆಯೇ ಮುಂದುವರಿಯಿತು... ”

ಆದರೆ ಈ ಅನುಭವ ನಿವೇದಿತೆಗೆ ಮಾತ್ರವೇ ಆದದ್ದಾಗಿರಲಿಲ್ಲ. ಇತರ ಜನಸಾಮಾನ್ಯರಿರಲಿ, ಸ್ವತಃ ಸ್ವಾಮೀಜಿಯವರ ಗುರುಭಾಯಿಗಳಿಗೂ ಇಂತಹ ಅನುಭವವಾಗಿತ್ತು. ಒಮ್ಮೆ ತರಗತಿ ಯೊಂದರಲ್ಲಿ ಅವರು ಮಾತನಾಡುವುದನ್ನು ಕೇಳುತ್ತಿದ್ದ ತುರೀಯಾನಂದರು ಅದರ ಅದ್ಭುತ ಪರಿಣಾಮವನ್ನು ಕಂಡು ಅತ್ಯಾಶ್ಚರ್ಯಪಟ್ಟರು. ಮತ್ತೊಮ್ಮೆ ಅಭೇದಾನಂದರೆನ್ನುತ್ತಾರೆ: “ಅವರ ಭಾಷಣವನ್ನು ಕೇಳುತ್ತಿದ್ದಂತೆ, ಧ್ಯಾನ ಮಾಡುವಾಗ ಹೇಗೋ ಹಾಗೆಯೇ ಯಾವುದೋ ಒಂದು ಶಕ್ತಿ ನನ್ನ ಕುಂಡಲಿನಿಯನ್ನು ಮೇಲೆತ್ತುತ್ತಿರುವಂತೆ ಭಾಸವಾಯಿತು.” ಸ್ವತಃ ಸ್ವಾಮೀಜಿಯವ ರಿಗೂ ತಮ್ಮ ಮಾತಿನ ಶಕ್ತಿಯ ಅರಿವಿತ್ತು. ಅವರೊಮ್ಮೆ ತುರೀಯಾನಂದರೊಡನೆ ಹೇಳಿದರು, “ಏನು, ನಾನು ಸುಮ್ಮನೆ ಭಾಷಣ ಮಾಡುತ್ತೇನೆಂದು ಭಾವಿಸಿದೆಯಾ? ನಾನು ಅವರಿಗೆ ಘನವಾದದ್ದೇನನ್ನೋ ಕೊಡುತ್ತೇನೆ. ಅದನ್ನು ಪಡೆಯುವಂತಹ ಅವರಿಗೂ ಅದು ಗೊತ್ತಿದೆ.”

೧೯ಂಂ ಜೂನ್​ನಲ್ಲಿ ನ್ಯೂಯಾರ್ಕ್ ವೇದಾಂತ ಸೊಸೈಟಿಯ ಸದಸ್ಯೆಯೂ ಸ್ವಾಮಿ ಅಭೇದಾ ನಂದರ ಶಿಷ್ಯೆಯೂ ಆಗಿದ್ದ ಮಿಸ್ ಮಿನ್ನಿ ಬೂಕ್ ಎಂಬಾಕೆ ಕ್ಯಾಲಿಫೋರ್ನಿಯದ ನಿರ್ಜನ ಪ್ರದೇಶವಾದ ಸ್ಯಾನ್ ಆ್ಯಂಟೋನ್ ಎಂಬಲ್ಲಿ ೧೬0 ಎಕರೆಯಷ್ಟು ವಿಸ್ತಾರವಾದ ಭೂಮಿಯನ್ನು ವೇದಾಂತ ಪ್ರಸಾರಕ್ಕಾಗಿ ಸಮರ್ಪಿಸಲು ಮುಂದಾದಳು. ಅತಿ ಹತ್ತಿರದ ರೈಲುನಿಲ್ದಾಣದಿಂದ ಐವತ್ತು ಮೈಲಿಯಷ್ಟು ದೂರದಲ್ಲಿತ್ತು ಈ ಪ್ರದೇಶ. ಜನವಸತಿಯ ಸ್ಥಳವಿದ್ದುದು ಹನ್ನೆರಡು ಮೈಲಿ ದೂರದಲ್ಲಿ. ಆದ್ದರಿಂದ ನಾಗರಿಕತೆಯ ಗದ್ದಲದಿಂದ ಸಾಕಷ್ಟು ದೂರದಲ್ಲೇ ಇತ್ತು; ಧ್ಯಾನಜೀವನಕ್ಕೆ ಅತ್ಯಂತ ಸೂಕ್ತವಾಗಿತ್ತು. ಆದರೆ ಅನುಕೂಲತೆಗಳೊಂದಿಗೆ ಅನನುಕೂಲತೆಗಳೂ ಇರಲೇಬೇಕಲ್ಲವೆ? ಹಾಗೆಯೇ ಇಲ್ಲಿಯೂ ತೊಂದರೆಗಳಿಗೇನೂ ಕೊರತೆಯಿರಲಿಲ್ಲ. ಉದಾ ಹರಣೆಗೆ ನೀರಿನ ಸೌಲಭ್ಯವಿರಲಿಲ್ಲ; ಯಾವುದೇ ಜೀವನೋಪಯೋಗೀ ವಸ್ತುಗಳು ಸಿಗುತ್ತಿರ ಲಿಲ್ಲ; ಆ ಸ್ಥಳದಿಂದ ಸಂತೆಕಟ್ಟೆಗೆ ಹೋಗಬೇಕಾದರೆ ಒಂದು ದಿನದ ದೀರ್ಘ ಪ್ರಯಾಣ. ಜೊತೆಗೆ ಸೆಕೆಗಾಲವೆಂದರೆ ಸುಡುಬಿಸಿಲು, ಚಳಿಗಾಲವೆಂದರೆ ಹೆಪ್ಪುಗಟ್ಟಿಸುವ ಶೀತ. ಆದರೆ ಧರ್ಮಕಾರ್ಯಕ್ಕೆ ಆ ಪರಧರ್ಮೀಯರ ನಾಡಿನಲ್ಲಿ ಇಷ್ಟೊಂದು ವಿಶಾಲವಾದ ಸ್ಥಳ ಸಿಕ್ಕಿ ದ್ದೊಂದು ಭಾಗ್ಯವೇ ಸರಿ. ಸ್ವಾಮೀಜಿಯವರು ಈ ಬಳುವಳಿಯನ್ನು ಸಂತೋಷದಿಂದ ಸ್ವೀಕರಿಸಿ ದರು. ಕ್ಯಾಲಿಫೋರ್ನಿಯದಲ್ಲಿ ಸ್ಥಾಪಿತವಾಗಿದ್ದ ಇತರ ವೇದಾಂತ ಸೊಸೈಟಿಗಳನ್ನು ನಡೆಸಿಕೊಂಡು ಬರುವುದಕ್ಕೂ ಸ್ಯಾನ್ ಆ್ಯಂಟೋನ್​ನಲ್ಲಿ ಆಶ್ರಮವೊಂದನ್ನು ಸ್ಥಾಪಿಸುವುದಕ್ಕೂ ಅವರು ತುರೀಯಾನಂದರನ್ನು ನೇಮಿಸಿದರು. ಆದರೆ ಧ್ಯಾನಶೀಲ ವ್ಯಕ್ತಿಯಾದ ತುರೀಯಾನಂದರು ಕಾರ್ಯಕ್ಷೇತ್ರಕ್ಕೆ ಇಳಿಯಲು ಯಾವಾಗಲೂ ಹಿಂದೇಟು ಹಾಕುತ್ತಿದ್ದವರೇ. ಅವರನ್ನು ಅಮೆರಿಕೆಗೆ ಕರೆಸಬೇಕಾದರೇ ಸ್ವಾಮೀಜಿ ಸಾಕಷ್ಟು ಕಷ್ಟಪಡಬೇಕಾಗಿತ್ತು. ಈಗ ಹೊಸ ಆಶ್ರಮವೊಂದನ್ನು ಸ್ಥಾಪಿಸುವ ಕೆಲಸ ಕೊಟ್ಟಾಗ ಮತ್ತೆ ಅವರು, “ಇಲ್ಲ, ನನಗದೆಲ್ಲ ಬೇಡ” ಎನ್ನಲು ಶುರುಮಾಡಿ ದರು. ವಾದವಿವಾದಗಳಾವುವೂ ಕೆಲಸ ಮಾಡದಿದ್ದಾಗ ಸ್ವಾಮೀಜಿ ಕೊನೆಗೆ ಹೇಳಿದರು, “ನೋಡು ಹರಿ, ನೀನು ಹೋಗಿ ಆ ಕೆಲಸದ ಜವಾಬ್ದಾರಿಯನ್ನು ಹೊತ್ತುಕೊಳ್ಳಲೇಬೇಕು. ಇದು ಜಗ ನ್ಮಾತೆಯ ಇಚ್ಛೆ.” ಅದಕ್ಕೆ ಸ್ವಾಮೀ ತುರೀಯಾನಂದರು ನಸುನಕ್ಕು, “ಜಗನ್ಮಾತೆಯ ಇಚ್ಛೆ!? ನಿನ್ನ ಇಚ್ಛೆ ಎನ್ನು! ಜಗನ್ಮಾತೆ ಈ ವಿಚಾರದಲ್ಲಿ ತನ್ನ ಇಚ್ಛೆಯೇನೆಂದು ಹೇಳಿದ್ದನ್ನು ನೀನೇನೂ ಕೇಳಿಲ್ಲವಲ್ಲ!” ಎಂದರು. ಅದಕ್ಕೆ ಸ್ವಾಮೀಜಿ ಗಂಭೀರವಾಗಿ ಉತ್ತರಿಸಿದರು, “ಹಾಗೆನ್ನದಿರು ಸೋದರ, ನಾವು ಇತರರ ಮಾತುಗಳನ್ನು ಕೇಳುವಷ್ಟೇ ಸ್ಪಷ್ಟವಾಗಿ ಜಗನ್ಮಾತೆಯ ಮಾತುಗಳನ್ನೂ ಕೇಳಬಹುದು. ಆದರೆ ಅದನ್ನು ಆಲಿಸಬೇಕಾದರೆ ಹೃದಯನಾಡಿ ತುಂಬ ಸೂಕ್ಷ್ಮವಾಗಿರಬೇಕಾಗು ತ್ತದೆ.” ಈಗ ತುರೀಯಾನಂದರ ಸಂದೇಹಗಳೆಲ್ಲ ಮಾಯವಾದುವು. ಅವರು ಕ್ಯಾಲಿಫೋರ್ನಿಯಕ್ಕೆ ಹೋಗಲು ಒಪ್ಪಿದರು.

ಜುಲೈ ೩ರಂದು ಸ್ವಾಮಿ ತುರೀಯಾನಂದರು ಕ್ಯಾಲಿಫೋರ್ನಿಯದೆಡೆಗೆ ಹೊರಟು ನಿಂತರು. ಅದೇ ದಿನ ಸ್ವಾಮೀಜಿ ತಮ್ಮ ಹಳೆಯ ಸ್ನೇಹಿತರನ್ನು ಭೇಟಿ ಮಾಡಲು ಡೆಟ್ರಾಯ್ಟಿಗೆ ಹೊರಟರು. ಹೊರಡುವ ಮುನ್ನ ತುರೀಯಾನಂದರು ತಾವು ಹೇಗೆ ಮುಂದುವರಿಯಬೇಕು ಎಂಬ ವಿಷಯ ವಾಗಿ ಸ್ವಾಮೀಜಿಯವರ ಸಲಹೆಯನ್ನು ಕೇಳಿದಾಗ ಅವರೆಂದರು–“ಹೋಗಿ ಕ್ಯಾಲಿಫೋರ್ನಿಯ ದಲ್ಲಿ ಆಶ್ರಮವನ್ನು ಸ್ಥಾಪಿಸು; ಅಲ್ಲಿ ವೇದಾಂತದ ಧ್ವಜವನ್ನು ಹಾರಿಸು. ಈ ಕ್ಷಣದಿಂದ ಭಾರತದ ನೆನಪನ್ನೇ ಸುಟ್ಟುಹಾಕು. ಎಲ್ಲಕ್ಕಿಂತ ಹೆಚ್ಚಾಗಿ ಆದರ್ಶ ಜೀವನವನ್ನು ನಡೆಸು. ಉಳಿದುದನ್ನು ಜಗನ್ಮಾತೆ ನೋಡಿಕೊಳ್ಳುತ್ತಾಳೆ.”

ಮಿಸ್ ಮಿನ್ನಿ ಬೂಕಳೊಂದಿಗೆ ಹೊರಟ ತುರೀಯಾನಂದರು ಲಾಸ್ ಏಂಜಲಿಸ್, ಪಸಾಡೆನ ಹಾಗೂ ಸ್ಯಾನ್​ಫ್ರಾನ್ಸಿಸ್ಕೋ ನಗರಗಳ ವೇದಾಂತ ಸೊಸೈಟಿಗಳನ್ನು ಸಂದರ್ಶಿಸಿ ವಿದ್ಯಾರ್ಥಿಗಳ ಪರಿಚಯ ಮಾಡಿಕೊಂಡರು. ಬಳಿಕ ಹನ್ನೆರಡು ಮಂದಿ ವಿದ್ಯಾರ್ಥಿಗಳೊಂದಿಗೆ ಸ್ಯಾನ್ ಆ್ಯಂಟೋನ್ ಕಣಿವೆಯ ಪ್ರದೇಶವನ್ನು ಪ್ರವೇಶಿಸಿದರು. ಇಲ್ಲಿ ಅವರು ಶಾಂತಿ ಆಶ್ರಮವನ್ನು ಸ್ಥಾಪಿಸಿ ಅಧ್ಯಾತ್ಮ ಪಿಪಾಸುಗಳಾಗಿ ಬಂದವರನ್ನು ತರಬೇತುಗೊಳಿಸುತ್ತ ಅನವರತ ಶ್ರಮಿಸಿದರು. ನಿರಂತರವಾಗಿ ಸುಮಾರು ಎರಡು ವರ್ಷಗಳ ಕಾಲ ಸ್ವಾಮೀಜಿ ಹೇಳಿದ್ದಂತೆ ಆದರ್ಶದ ಬದುಕನ್ನು ಬಾಳಿದರು. ತಮ್ಮ ತನುಮನಗಳನ್ನೇ ತೇದು, ಎಂದೆಂದಿಗೂ ಪ್ರಭಾವವಳಿಯದ ಆಧ್ಯಾತ್ಮಿಕ ವಾತಾವರಣವೊಂದನ್ನು ನಿರ್ಮಾಣ ಮಾಡಿದರು.

ಡೆಟ್ರಾಯ್ಟಿನಲ್ಲಿ ಸ್ವಾಮೀಜಿಯವರು ತಮ್ಮ ಪ್ರಿಯ ಶಿಷ್ಯೆ ಕ್ರಿಸ್ಟೀನಳ ಮನೆಯಲ್ಲಿ ಇಳಿದು ಕೊಂಡರು. ತಮ್ಮ ಪ್ರಮುಖ ಅಮೆರಿಕನ್ ಶಿಷ್ಯೆ ಶ್ರೀಮತಿ ಮೇರಿ ಫಂಕೆಯನ್ನು ಭೇಟಿ ಮಾಡಿ ಮಾತುಕತೆ ನಡೆಸಿದರು. ಅವರ ಕಟ್ಟಾ ಬೆಂಬಲಿಗಳಾಗಿದ್ದ ಶ್ರೀಮತಿ ಬ್ಯಾಗ್​ಲೀ ಇದ್ದ ಊರು ಡೆಟ್ರಾಯ್ಟ್. ಆದರೆ ಅವಳು ಎರಡು ವರ್ಷಗಳ ಕೆಳಗೆ ತೀರಿಕೊಂಡಿದ್ದಳು. ಅವಳಿಲ್ಲದಿರುವುದು ಈಗ ಎದ್ದು ಕಾಣುತ್ತಿತ್ತು. ಡೆಟ್ರಾಯ್ಟಿನಲ್ಲಿ ಅವರು ನಾಲ್ಕೈದು ದಿನ ಉಳಿದುಕೊಂಡು ತಮ್ಮ ಹಳೆಯ ಸ್ನೇಹಿತರನ್ನೆಲ್ಲ ಭೇಟಿಯಾದರು. ಸ್ವಾಮೀಜಿಯವರನ್ನು ಕಂಡು ಇವರಿಗೆಲ್ಲ ಬಹಳ ಸಂತೋಷವಾಯಿತಾದರೂ, ಅವರು ಇಳಿದುಹೋಗಿದ್ದುದನ್ನು ಕಂಡಾಗ ಅದಕ್ಕಿಂತ ಹೆಚ್ಚು ದುಃಖವಾಯಿತು.

ನ್ಯೂಯಾರ್ಕಿಗೆ ಹಿಂದಿರುಗಿದ ಸ್ವಾಮೀಜಿ ಈಗ ಹೆಚ್ಚುಕಡಿಮೆ ಏಕಾಂಗಿಯಾಗಿಯೇ ಇರತೊಡ ಗಿದರು. ಸ್ವಾಮಿ ಅಭೇದಾನಂದರು ಸಮೀಪದ ಗಿರಿಧಾಮವೊಂದಕ್ಕೆ ಹೋಗಿದ್ದರು; ಸೋದರಿ ನಿವೇದಿತಾ ಯಾವುದೋ ಕೆಲಸಕ್ಕಾಗಿ ಪ್ಯಾರಿಸಿಗೆ ಹೋಗಿದ್ದಳು. ಈ ಅವಧಿಯಲ್ಲಿ ಸ್ವಾಮೀಜಿ ಮಾಡಿ ಮುಗಿಸಿದ ಒಂದು ಮುಖ್ಯ ಕೆಲಸವೆಂದರೆ ಶ್ರೀರಾಮಕೃಷ್ಣ ಮಠ ಮತ್ತು ಮಿಷನ್ನಿನ ಲಾಂಛನವನ್ನು ರೂಪಿಸಿದ್ದು. ಅವರು ಇದನ್ನು ಹೇಗೆ ರೂಪಿಸಿದರು ಎಂಬುದು ಬಹಳ ಕುತೂಹಲ ಕಾರಿಯಾಗಿದೆ:

ಒಂದು ದಿನ ಸ್ವಾಮೀಜಿ ತಮ್ಮ ಉಪಾಹಾರದ ಮೇಜಿನ ಮುಂದೆ ಕುಳಿತಿದ್ದಾಗ ಅಲ್ಲಿಗೆ ಹೆನ್ರಿ ವಾನ್ ಹಾಗೆನ್ ಎಂಬಾತ ಬಂದ. ಈತ ವೇದಾಂತ ಸೊಸೈಟಿಯ ಸದಸ್ಯ ಹಾಗೂ ಸ್ವಾಮೀಜಿಯವ ರಿಗೆ ಸುಪರಿಚಿತ. ವೃತ್ತಿಯಿಂದ ಇವನೊಬ್ಬ ನಕ್ಷೆ ಬರೆಯುವವನು ಮತ್ತು ಮುದ್ರಕ. ವೇದಾಂತ ಸೊಸೈಟಿಯ ಕೆಲವು ಕರಪತ್ರಗಳನ್ನು ಈತ ಅಚ್ಚು ಹಾಕುತ್ತಿದ್ದ. ಈ ಕರಪತ್ರಗಳ ಮೇಲೆ ಸೊಸೈಟಿಯ ಲಾಂಛನವೊಂದನ್ನು ಮುದ್ರಿಸಿದರೆ ಚೆನ್ನಾಗಿರುತ್ತದೆ ಎಂದು ಅವನಿಗನ್ನಿಸಿತು. ಈ ಬಗ್ಗೆ ಸ್ವಾಮೀಜಿಯವರ ಸಲಹೆ ಸೂಚನೆಗಳನ್ನು ಕೇಳಲು ಅವನು ಬಂದಿದ್ದ.

ಸ್ವಾಮೀಜಿ ಮಾತೇನೂ ಆಡಲಿಲ್ಲ. ಆಗತಾನೆ ತಮಗೆ ಬಂದಿದ್ದ ಒಂದು ಪತ್ರದ ಖಾಲಿ ಲಕೋಟೆಯನ್ನು ಹರಿದರು. ಅದರ ಒಳಭಾಗ ಬೆಳ್ಳಗೆ ಶುಭ್ರವಾಗಿತ್ತು. ಒಂದು ಪೆನ್ನಿನಿಂದ ಅವರು ಅದರ ಮೇಲೆ ಕೆಲವು ರೇಖೆಗಳನ್ನೆಳೆದರು–ಅಲೆಗಳು, ಕಮಲ, ಸೂರ್ಯ, ಹಂಸ ಮತ್ತು ಇವುಗಳನ್ನೆಲ್ಲ ಸುತ್ತಿರುವ ಸರ್ಪ. ಚಿತ್ರವನ್ನು ಗೀಚಿ ವಾನ್​ಹಾಗೆನ್ನನ ಕಡೆಗೆ ತಳ್ಳಿ ಸ್ವಾಮೀಜಿ, “ಇಕೊ, ಇದನ್ನು ಅಳತೆಗೆ ಸರಿಯಾಗಿ ಬರೆ” ಎಂದರು. ವಾನ್ ಹಾಗೆನ್ ಚೆನ್ನಾಗಿ ಗಮನ ಕೊಟ್ಟು ಬರೆದು ಅದನ್ನು ಸುಂದರ ಚಿತ್ರವಾಗಿ ರೂಪಿಸಿದ.

ಈ ಚಿತ್ರ ಮುದ್ರಿತವಾಗಿ ಬಂದ ಕಾಗದದಲ್ಲಿ ಸ್ವಾಮೀಜಿ ಜುಲೈ ೨೪ರಂದು ಮಿಸ್ ಮೆಕ್​ಲಾಡಳಿಗೊಂದು ಪತ್ರ ಬರೆದು ಚಿತ್ರದ ಅರ್ಥವನ್ನು ವಿವರಿಸಿದರು.\footnote{\textbf{*ಶ್ರೀ ರಾಮಕೃಷ್ಣ ಮಠ ಮತ್ತು ಮಿಷನ್ನಿನ ಲಾಂಛನ}

ಈ ಚಿತ್ರದಲ್ಲಿ ಕಾಣುವ ಸರೋವರದ ಅಲೆಗಳು ಕರ್ಮವನ್ನೂ, ಕಮಲವು ಭಕ್ತಿ ಯನ್ನೂ, ಉದಯಿಸುತ್ತಿರುವ ಸೂರ್ಯನು ಜ್ಞಾನವನ್ನೂ, ಸುತ್ತಿಕೊಂಡಿರುವ ಸರ್ಪವು ಯೋಗ ಹಾಗೂ ಜಾಗೃತ ಕುಂಡಲಿನೀ ಶಕ್ತಿಯನ್ನೂ ಪ್ರತಿನಿಧಿಸುತ್ತವೆ. ಹಂಸವು ಪರ ಮಾತ್ಮನ ಪ್ರತೀಕವಾಗಿರುತ್ತದೆ. ಕರ್ಮ, ಭಕ್ತಿ, ಜ್ಞಾನ ಮತ್ತು ಯೋಗ–ಇವುಗಳ ಸಮ ನ್ವಯದ ಮೂಲಕ ಪರಮಾತ್ಮನ ದರ್ಶನಲಾಭವಾಗುತ್ತದೆ ಎಂಬುವುದು ಈ ಚಿತ್ರದ ಭಾವಾರ್ಥ.

ದೇವನಾಗರಿ ಲಿಪಿಯಲ್ಲಿ ಬರೆದಿರುವ ‘ತನ್ನೋ ಹಂಸಃ ಪ್ರಚೋದಯಾತ್​’ ಎಂಬು ದರ ಅರ್ಥ: ‘ಹಂಸರೂಪಿಯಾದ ಪರಮಾತ್ಮನು ನಮ್ಮನ್ನು ಪ್ರಚೋದಿಸಲಿ’ ಎಂದು.}

ನ್ಯೂಯಾರ್ಕಿನಲ್ಲಿ ನಡೆದ ಮತ್ತೊಂದು ಘಟನೆಯನ್ನು ಮಿಸ್ ಮೆಕ್​ಲಾಡ್ ತಿಳಿಸುತ್ತಾಳೆ:

“ಇಲ್ಲಿ ಸ್ವಾಮೀಜಿಯವರ ಸುತ್ತ ಯಾವಾಗಲೂ ನಾಲ್ಕಾರು ವಿಚಿತ್ರ ವ್ಯಕ್ತಿಗಳ ಒಂದು ಗುಂಪು ಸುತ್ತಾಡುತ್ತಿತ್ತು. ನೋಡಿದವರಿಗೆ ಕರೆಕರೆಯಾಗುವಂತಹ ವೇಷ, ವರ್ತನೆ. ಸ್ವಾಮೀಜಿ ಹೋದ ಲ್ಲೆಲ್ಲ ಇವರು ಹಿಂಬಾಲಿಸುತ್ತಿದ್ದರು. ಇವರನ್ನು ಸ್ವಾಮೀಜಿ ತಾವಾಗಿಯೇ ಪ್ರೋತ್ಸಾಹಿಸಿರಲಿಲ್ಲ. ಅಥವಾ ತಾವಾಗಿಯೇ ದೂರ ಅಟ್ಟಲೂ ಇಲ್ಲ. ಒಂದು ದಿನ ಸ್ವಾಮೀಜಿ ಸ್ವಲ್ಪ ಅಡ್ಡಾಡಿಕೊಂಡು ಬರಲು ಹೋದಾಗ ಇವರಲ್ಲಿಬ್ಬರು ಗಂಟುಬಿದ್ದರು. ಸ್ವಾಮೀಜಿ ಹಿಂದಿರುಗಿ ಬಂದು ವೇದಾಂತ ಸೊಸೈಟಿಯ ಮೆಟ್ಟಲು ಹತ್ತುತ್ತಿದ್ದಾಗ ಅದನ್ನು ಕಂಡು ಅಲ್ಲಿದ್ದವನೊಬ್ಬ ಆಲೋಚಿಸಿದ, ‘ಅಲ್ಲಾ, ಎಲ್ಲಾ ಬಿಟ್ಟು ಸ್ವಾಮೀಜಿ ಇಂತಹ ವಿಚಿತ್ರ ಜನಗಳನ್ನೇಕೆ ಆಕರ್ಷಿಸಿದ್ದಾರೆ!’ ಎಂದು. ಅವನ ಈ ಆಲೋಚನೆ ಮಾತಿನ ಮೂಲಕ ಹೊರಬೀಳದಿದ್ದರೂ ಅದಕ್ಕೆ ಉತ್ತರ ತಕ್ಷಣವೇ ಸಿಕ್ಕಿತು. ಸರ್ರನೆ ಅವನತ್ತ ತಿರುಗಿ ಸ್ವಾಮೀಜಿ ಹೇಳಿದರು, ‘ನೋಡು, ಅವರೆಲ್ಲ ಶಿವನ ಭೂತಗಣಗಳು.’”

ಡೆಟ್ರಾಯ್ಟಿನಿಂದ ನ್ಯೂಯಾರ್ಕಿಗೆ ಹಿಂದಿರುಗಿದ ಮೇಲೆ ಸ್ವಾಮೀಜಿ ಮತ್ತೆ ಎರಡು ವಾರಗಳ ಕಾಲ ವಿಶ್ರಾಂತಿ ಪಡೆದುಕೊಂಡು, ಬಳಿಕ ಪ್ಯಾರಿಸ್ ಸಮ್ಮೇಳನವನ್ನು ಮುಗಿಸಿಕೊಂಡು ಭಾರತಕ್ಕೆ ಮರಳಲು ಅಣಿಯಾದರು. ಹೀಗೆ ಅಮೆರಿಕೆಗೆ ಅವರ ಎರಡನೆಯ ಹಾಗೂ ಕೊನೆಯ ಭೇಟಿ ಮುಕ್ತಾಯಗೊಂಡಿತು. ೧೯ಂಂ ಜುಲೈ ೨೬ರಂದು ‘ಲಾ ಶಾಂಪೇನ್​’ ಎಂಬ ಹಡಗು ಹತ್ತಿ ನ್ಯೂಯಾರ್ಕಿನಿಂದ ಫ್ರಾನ್ಸಿಗೆ ಹೊರಟರು.

