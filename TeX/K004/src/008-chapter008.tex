
\chapter{ಕಾಶ್ಮೀರಪುರವಾಸ}

\noindent

ಸ್ವಾಮೀಜಿಯವರು ಆಲ್ಮೋರದಲ್ಲಿದ್ದು ಆಗಲೇ ಮೂರು ತಿಂಗಳು ಕಳೆದುವು. ಕುಸಿದು ಹೋಗಿದ್ದ ಅವರ ಆರೋಗ್ಯ ಈಗ ತಕ್ಕಮಟ್ಟಿಗೆ ಸುಧಾರಿಸಿದೆ. ಈಗ ಅವರು ತಮ್ಮ ಕಾರ್ಯೋ ದ್ದೇಶದತ್ತ ಮುನ್ನಡಿಯಿಡಬೇಕಾಗಿದೆ. ಅಂತೆಯೇ, ಅವರಿಗೆ ಪಂಜಾಬ್ ಮತ್ತು ಕಾಶ್ಮೀರಗಳ ಕಡೆಯಿಂದ ತ್ವರಿತದ ಹಾಗೂ ಆದರದ ಕರೆಯೂ ಬರುತ್ತಿದೆ. ಆದ್ದರಿಂದ ಸ್ವಾಮೀಜಿ ಪರಿವಾರ ಸಮೇತರಾಗಿ ಆಗಸ್ಟ್ ಎರಡರಂದು ಆಲ್ಮೋರದಿಂದ ಹೊರಟರು. ಒಂಬತ್ತರಂದು ಅವರೆಲ್ಲ ಬರೇಲಿಗೆ ಬಂದರು. ಅಲ್ಲಿನ ನಾಗರಿಕರು ಅವರನ್ನು ಆದರಿಂದ ಬರಮಾಡಿಕೊಂಡರು. ಅಲ್ಲಿ ಅವರು ನಾಲ್ಕು ದಿನ ಉಳಿದುಕೊಂಡರು. ಈ ಅವಧಿಯಲ್ಲಿ ಒಮ್ಮೆ ಸ್ವಾಮೀಜಿಯವರು ಅಲ್ಲಿನ ಕೆಲವು ಕಾಲೇಜು ವಿದ್ಯಾರ್ಥಿಗಳೊಂದಿಗೆ ಸಂಭಾಷಿಸುತ್ತ, ಯುವಕರೆಲ್ಲ ಸೇರಿ ಸಂಘ ಕಟ್ಟಿಕೊಳ್ಳು ವುದರ ಮಹತ್ವವನ್ನು ವಿವರಿಸಿದರು. ಹೀಗೆ ಸಂಘಟಿತರಾದ ಯುವಕರು ವೇದಾಂತದ ಬೋಧನೆ ಗಳನ್ನು ಜೀವನದಲ್ಲಿ ಅಳವಡಿಸಿಕೊಂಡು ಆಚರಣೆಗೆ ತರಬೇಕು ಮತ್ತು ಇತರರಿಗಾಗಿ ಶ್ರಮಿಸ ಬೇಕು ಎಂದು ಬೋಧಿಸಿದರು. ಸ್ವಾಮೀಜಿಯವರ ಮಾತಿನಿಂದ ಪ್ರಭಾವಿತರಾದ ಆ ಯುವಕರು ಆಗಲೇ ಅವರ ಸನ್ನಿಧಿಯಲ್ಲೇ ಒಂದು ಸಂಘವನ್ನು ರಚಿಸಿಕೊಂಡರು.

ಸ್ವಾಮೀಜಿ ಬರೇಲಿಯಿಂದ ಹೊರಟು ೨೨೫ ಮೈಲಿ ದೂರದ ಅಂಬಾಲಾಗೆ ಬಂದರು. ಯಥಾ ಪ್ರಕಾರ ಅಲ್ಲಿಯೂ ಅದ್ಧೂರಿಯ ಸ್ವಾಗತ. ಇಲ್ಲಿ ಸೇವಿಯರ್ ದಂಪತಿಗಳೂ ಸ್ವಾಮೀಜಿಯವರ ತಂಡವನ್ನು ಕೂಡಿಕೊಂಡರು. ನಾಲ್ಕು ದಿನಗಳ ಕಾಲ ಇಲ್ಲಿಇಳಿದುಕೊಂಡ ಸ್ವಾಮೀಜಿ ಇಲ್ಲಿನ ವಿಭಿನ್ನ ವರ್ಗಗಳ ಜನರೊಂದಿಗೆ ಸಂಭಾಷಿಸಿ ಆಧುನಿಕ ಯುಗಕ್ಕೆ ತಾವು ನೀಡಲಿರುವ ಸಂದೇಶ ಗಳನ್ನು ಮನಮುಟ್ಟಿಸಿದರು. ಅವರೊಮ್ಮೆ ಈ ಊರಿನ ‘ಹಿಂದೂ-ಮುಸ್ಲಿಮ್ ಶಾಲೆ’ ಎಂಬ ಒಂದು ವಿದ್ಯಾಸಂಸ್ಥೆಗೆ ಭೇಟಿಯಿತ್ತು, ಅಲ್ಲಿನ ಚಟುವಟಿಕೆಗಳನ್ನು ಅತ್ಯಂತ ಆಸಕ್ತಿಯಿಂದ ವೀಕ್ಷಿಸಿ ದರು. ಭಾರತದ ಎರಡು ಪ್ರಮುಖ ಪಂಥಗಳು ಈ ರೀತಿ ಒಂದಾಗಿ ಸೇರಿರುವುದು ಒಂದು ಶುಭ ಸಂಕೇತವೇ ಸರಿ ಎಂದು ಅವರು ನುಡಿದರು.

ಆಗಸ್ಟ್ ೨ಂರಂದು ಸ್ವಾಮೀಜಿ ತಮ್ಮ ಸಂಗಡಿಗರೊಂದಿಗೆ ಸಿಕ್ಖರ ಪವಿತ್ರ ಕ್ಷೇತ್ರವೂ ಅಂದಿನ ಪಂಜಾಬ್ ಪ್ರಾಂತ್ಯದ ಅತ್ಯಂತ ಸಂಪದ್ಭರಿತ ನಗರವೂ ಆದ ಅಮೃತಸರಕ್ಕೆ ಬಂದಾಗ, ನಿಲ್ದಾಣ ದಲ್ಲಿ ಅವರಿಗೆ ವೈಭವಯುತ ಸ್ವಾಗತ ದೊರಕಿತು. ಇಲ್ಲಿ ಅವರು ಎರಡು ದಿನ ಇಳಿದುಕೊಂಡು ಸುಪ್ರಸಿದ್ಧ ಸ್ವರ್ಣದೇವಾಲಯವನ್ನೂ ಇತರ ಸ್ಥಳಗಳನ್ನೂ ಸಂದರ್ಶಿಸಿದರು. ಅಲ್ಲದೆ ಸಮೀಪದ ಧರ್ಮಶಾಲಾ ಎಂಬ ಗಿರಿಧಾಮದಲ್ಲಿ ಹತ್ತು ದಿನ ವಿಶ್ರಮಿಸಿದರು. ಪಂಜಾಬಿನ ಇತರ ಪ್ರಮುಖ ನಗರಗಳಾದ ಲಾಹೋರ್, ರಾವಲ್ಪಿಂಡಿ ಮೊದಲಾದವುಗಳನ್ನು ಸಂದರ್ಶಿಸುವುದು ಸ್ವಾಮೀಜಿ ಯವರ ಉದ್ದೇಶವಾಗಿತ್ತು. ಆದರೆ ಮೊದಲು ತಮ್ಮೊಂದಿಗೆ ಕಾಶ್ಮೀರಕ್ಕೆ ಬರಬೇಕೆಂದು ಸೇವಿ ಯರ್ ದಂಪತಿಗಳು ಆಗ್ರಹಪಡಿಸಿದ್ದರಿಂದ ಅವರು ರಾವಲ್ಪಿಂಡಿ ಹಾಗೂ ಮುರ್ರಿಯ ಮುಖಾಂತರ ಕಾಶ್ಮೀರಕ್ಕೆ ಹೊರಟರು.

ಸೆಪ್ಟೆಂಬರ್ ೧ಂರಂದು ಸ್ವಾಮೀಜಿಯವರು ನಿರಂಜನಾನಂದರು, ಅದ್ಭುತಾನಂದರು, ಸೇವಿ ಯರ್ ದಂಪತಿಗಳು, ಬ್ರಹ್ಮಚಾರಿ ಕೃಷ್ಣಲಾಲ್ ಹಾಗೂ ಇನ್ನು ಕೆಲವರನ್ನು ಕೂಡಿಕೊಂಡು ಶ್ರೀನಗರಕ್ಕೆ ಬಂದರು. ಶ್ರೀನಗರವು ಪ್ರಪಂಚದ ಸೌಂದರ್ಯದ ಖನಿಗಳಲ್ಲೊಂದು. ಇಲ್ಲಿನ ಸೌಂದರ್ಯ ಸ್ವಾಮೀಜಿಯವರ ಮನಸೂರೆಗೊಂಡಿತು.

ಕಾಶ್ಮೀರದ ಮಹಾರಾಜನ ಪರವಾಗಿ ಆತನ ಸೋದರ ರಾಜಾ ರಾಮಸಿಂಗ್, ಸ್ವಾಮೀಜಿಯವ ರನ್ನು ಅರಮನೆಗೆ ಬರಮಾಡಿಕೊಂಡು ಹೃತ್ಪೂರ್ವಕ ಸ್ವಾಗತ ನೀಡಿದ. (ಕಾಶ್ಮೀರದ ಮಹಾರಾಜ ಆಗ ನಗರದಲ್ಲಿರಲಿಲ್ಲ) ಬಳಿಕ ಅವರನ್ನು ಯೋಗ್ಯ ಆಸನದ ಮೇಲೆ ಕುಳ್ಳಿರಿಸಿ ತಾನು ಇತರ ಅಧಿ ಕಾರಿಗಳೊಂದಿಗೆ ನೆಲದ ಮೇಲೆ ಕುಳಿತು ಅವರೊಂದಿಗೆ ಮಾತುಕತೆ ನಡೆಸಿದ. ಈ ಸಂಭಾಷಣೆ ಸುಮಾರು ಎರಡು ಗಂಟೆಯ ಕಾಲ ನಡೆಯಿತು. ಸ್ವಾಮೀಜಿಯವರು ಹಿಂದೂಧರ್ಮದ ವಿಚಾರ ವಾಗಿ ಮಾತನಾಡಿದರಲ್ಲದೆ ಬಡ ಜನರನ್ನು ಮೇಲೆತ್ತುವ ಉಪಾಯಗಳ ಬಗ್ಗೆ ಚರ್ಚಿಸಿದರು. ಸ್ವಾಮೀಜಿಯವರ ಮಾತುಗಳಿಂದ ತುಂಬ ಪ್ರಭಾವಿತನಾದ ರಾಜಾ ರಾಮಸಿಂಗ್, ಅವರ ಕಾರ್ಯ ದಲ್ಲಿ ತನ್ನ ಸಂಪೂರ್ಣ ನೆರವು ನೀಡಲು ಮುಂದಾದ.

ಸ್ವಾಮೀಜಿಯವರು ಕಾಶ್ಮೀರದಲ್ಲಿ ಸುಮಾರು ಒಂದು ತಿಂಗಳು ಉಳಿದುಕೊಂಡು, ಸುತ್ತ ಮುತ್ತಲ ಐತಿಹಾಸಿಕ ಕ್ಷೇತ್ರಗಳನ್ನು ಸಂದರ್ಶಿಸಿದರು. ಶ್ರೀನಗರದಲ್ಲಿ ಅವರ ದರ್ಶನಕ್ಕಾಗಿ ಸಾಧು ಗಳು ಪಂಡಿತರು ವಿದ್ಯಾರ್ಥಿಗಳು ಉನ್ನತ ಸರ್ಕಾರೀ ಅಧಿಕಾರಿಗಳು ಹಾಗೂ ಇತರ ನೂರಾರು ಜನರು ಬರುತ್ತಿದ್ದರು. ಒಂದು ದಿನ ಸ್ವಾಮೀಜಿ ವಿಶಾಲವಾದ ಪಂಡಿತಸಭೆಯನ್ನುದ್ದೇಶಿಸಿ ಮಾತ ನಾಡಿದರು. ಅವರ ಆಧ್ಯಾತ್ಮಿಕ ಅನುಭವದಿಂದೊಡಗೂಡಿದ ಮಾತುಗಳು ಪಂಡಿತರಿಗೆ ಹೊಸ ಬೆಳಕು ನೀಡಿದುವು. ಕಾಶ್ಮೀರವಾಸದಿಂದ ಸ್ವಾಮೀಜಿಯವರ ಆರೋಗ್ಯ ಬಹುಮಟ್ಟಿಗೆ ಸುಧಾರಿಸಿತು.

ಶ್ರೀನಗರದಿಂದ ಹೊರಟ ಸ್ವಾಮೀಜಿ ಅಕ್ಟೋಬರ್ ೧೬ರಂದು ಮತ್ತೆ ರಾವಲ್ಪಿಂಡಿಗೆ ಬಂದರು. ಇಲ್ಲಿ ಅವರಿಗೆ ವೈಭವದ ಸ್ವಾಗತ ನೀಡಲಾಯಿತು. ಇಲ್ಲಿನ ಆರ್ಯ ಸಮಾಜದ ಸ್ವಾಮಿ ಪ್ರಕಾಶಾನಂದರು, ಜಸ್ಟಿಸ್ ನಾರಾಯಣ ದಾಸ್ ಮತ್ತಿತರರು ಅವರನ್ನು ಭೇಟಿಯಾಗಿ ಮಾತುಕತೆ ನಡೆಸಿದರು. ಮರುದಿನ ಸರ್ದಾರ್ ಸುಜನ್​ಸಿಂಗ್ ಎಂಬವರ ಸುಂದರ ಉದ್ಯಾನದಲ್ಲಿ ಬಹು ದೊಡ್ಡ ಸಭೆಯನ್ನುದ್ದೇಶಿಸಿ ಸ್ವಾಮೀಜಿ ಹಿಂದೂ ಧರ್ಮವನ್ನು ಕುರಿತು ಮಾತನಾಡಿದರು. ಸುಮಾರು ಎರಡು ಗಂಟೆಗಳ ಕಾಲ ಅವರು ಹಲವಾರು ವೇದೋಕ್ತಿಗಳನ್ನು ಉದ್ಧರಿಸುತ್ತ ಸುಲಲಿತವಾಗಿ ಮಾತನಾಡಿದರು. ಅಂದಿನ ಕಾರ್ಯಕ್ರಮದ ದೃಶ್ಯ ಅತ್ಯಂತ ಮನೋಹರವಾಗಿತ್ತು. ಅದನ್ನು ಕಂಡ ಪಾಶ್ಚಾತ್ಯ ಶಿಷ್ಯರೊಬ್ಬರು ಹೇಳುತ್ತಾರೆ, “ಭಾಷಣ ಕಾಲದಲ್ಲಿ ಸ್ವಾಮೀಜಿಯವರು, ತಮ್ಮ ಅಭ್ಯಾಸದಂತೆ, ವೇದಿಕೆಯ ಮೇಲೆ ಅತ್ತಿಂದಿತ್ತ ನಡೆದಾಡುತ್ತಿದ್ದರು. ಆಗಾಗ ಅವರು ತಳಿರುತೋರಣ ಹೂಮಾಲೆಗಳಿಂದ ಅಲಂಕೃತವಾದ ಕಂಬಕ್ಕೆ ಒರಗಿಕೊಳ್ಳುತ್ತಿದ್ದರು. ಅವರ ಕೊರಳಿಗೂ ಸುಂದರವಾದ ಪುಷ್ಪಹಾರವನ್ನು ಹಾಕಲಾಗಿತ್ತು; ತಲೆಗೆ ವಿಧವಿಧದ ಪುಷ್ಪಗಳ ಕಿರೀಟ ವನ್ನು ತೊಡಿಸಲಾಗಿತ್ತು. ಕಾಷಾಯವಸ್ತ್ರಧಾರಿಯಾದ ಅವರು ಈ ಅಲಂಕಾರದಲ್ಲಿ ಗ್ರೀಕ್ ದೇವತೆ ಯಂತೆ ಕಂಗೊಳಿಸುತ್ತಿದ್ದರು. ಅಲ್ಲದೆ, ಇವೆಲ್ಲಕ್ಕೂ ಹಿನ್ನೆಲೆಯಾಗಿ, ಅಲ್ಲಿನ ಹಚ್ಚಹಸಿರು ಹುಲ್ಲಿನ ಮೇಲೆ ಬಣ್ಣಬಣ್ಣದ ಪೇಟಗಳನ್ನು ಧರಿಸಿದ್ದ ಸಭಿಕರು ಕಿಕ್ಕಿರಿದಿದ್ದುದು, ದಿಗಂತದಲ್ಲಿ ಸಂಧ್ಯಾ ಸೂರ್ಯನು ರಕ್ತವರ್ಣವನ್ನು ಬೀರುತ್ತ ಅಸ್ತಮಿಸುತ್ತಿದ್ದುದು–ಇವೆಲ್ಲ ಸೇರಿ ಅದ್ಭುತ-ರಮ್ಯ ನೋಟವೇರ್ಪಟ್ಟಿತ್ತು.”

ರಾವಲ್ಪಿಂಡಿಯಲ್ಲಿ ಮುಸಲ್ಮಾನರ ಪ್ರಾಬಲ್ಯ ಹೆಚ್ಚು. ಅಲ್ಲಿ ಮುಸಲ್ಮಾನರ ಹಾಗೂ ಹಿಂದೂ ಸಂಸ್ಥೆಯಾದ ಆರ್ಯಸಮಾಜದವರ ನಡುವೆ ತೀವ್ರ ದ್ವೇಷದ ವಾತಾವರಣವಿತ್ತು. ಆಗಾಗ ಇದು ಸ್ಫೋಟಿಸಿ ಗಲಭೆಗೆ ಕಾರಣವಾಗುತ್ತಿತ್ತು. ಸ್ವಾಮೀಜಿಯವರನ್ನು ಭೇಟಿ ಮಾಡಿದ ಆರ್ಯ ಸಮಾಜದ ಸ್ವಾಮಿ ಪ್ರಕಾಶಾನಂದರೇ ಮೊದಲಾದವರು ಅವರ ಮುಂದೆ ಈ ಸಮಸ್ಯೆಯನ್ನಿಟ್ಟರು. ಆಗ ಸ್ವಾಮೀಜಿ ಈ ಮತೀಯ ವಿದ್ವೇಷದ ಪರಿಹಾರಕ್ಕೆ ತುಂಬ ಸಮಾಧಾನಕರವಾದ ಕೆಲವು ಸಲಹೆಗಳನ್ನು ನೀಡಿದರು.

ಸ್ವಾಮೀಜಿಯವರು ಕಾಶ್ಮೀರ ರಾಜ್ಯಕ್ಕೆ ಆಗಮಿಸಿರುವ ಸುದ್ದಿ ತಿಳಿದು, ಆಗ ಜಮ್ಮುವಿನಲ್ಲಿದ್ದ ಕಾಶ್ಮೀರದ ಮಹಾರಾಜ ಅವರನ್ನು ತನ್ನಲ್ಲಿಗೆ ಆಹ್ವಾನಿಸಿದ. ಈ ಆಹ್ವಾನವನ್ನು ಮನ್ನಿಸಿ ಸ್ವಾಮೀಜಿ ತಮ್ಮ ಅನುಚರರೊಂದಿಗೆ ಅಕ್ಟೋಬರ್ ೨೧ರಂದು ಜಮ್ಮುವಿಗೆ ಬಂದಾಗ ಅವರನ್ನು ರಾಜ ವೈಭವದಿಂದ ಸ್ವಾಗತಿಸಲಾಯಿತು. ಮಹಾರಾಜನ ಅತಿಥಿಗೃಹದಲ್ಲಿ ಅವರೆಲ್ಲ ಇಳಿದುಕೊಂಡರು. ನಿಸರ್ಗ ಸೌಂದರ್ಯದ ನೆಲೆವೀಡಾದ ಕಾಶ್ಮೀರದಲ್ಲಿ ಮಠದ ಒಂದು ಕೇಂದ್ರವನ್ನು ಸ್ಥಾಪಿಸ ಬೇಕೆಂಬ ಪ್ರಬಲ ಇಚ್ಛೆ ಸ್ವಾಮೀಜಿಯವರಿಗಿತ್ತು. ಆದ್ದರಿಂದ ಅವರು ಮರುದಿನ, ರಾಜ್ಯದ ಅಧಿಕಾರಿಯಾದ ಮಹೇಶಚಂದ್ರ ಭಟ್ಟಾಚಾರ್ಯರೊಂದಿಗೆ ಈ ವಿಷಯವಾಗಿ ಚರ್ಚಿಸಿದರು.

ಮತ್ತೊಂದು ದಿನ ಕಾಶ್ಮೀರದ ರಾಜ ಸ್ವಾಮೀಜಿಯವರೊಂದಿಗೆ ಬಹಳ ಹೊತ್ತು ಸಂಭಾಷಣೆ ನಡೆಸಿದ. ಜೊತೆಗೆ ರಾಜನ ಸೋದರರೂ ಕೆಲವು ಉನ್ನತಾಧಿಕಾರಿಗಳೂ ಕುಳಿತಿದ್ದರು. ರಾಜರು ಗಳೊಂದಿಗೆ ನಡೆಸುವ ಸಂಭಾಷಣೆಗಳಿಗೆ ಸ್ವಾಮೀಜಿ ವಿಶೇಷ ಮಹತ್ವ ನೀಡುತ್ತಿದ್ದರು. ಏಕೆಂದರೆ ಅವರ ಅಭಿಪ್ರಾಯದಂತೆ, ಒಬ್ಬ ರಾಜನ ತಲೆಯನ್ನು ‘ಸರಿ’ಪಡಿಸಿದರೆ ಅವನ ಮೂಲಕ ಅವನ ಪ್ರಜೆಗಳೆಲ್ಲರಿಗೂ ಸನ್ಮಾರ್ಗವನ್ನು ತೋರಿಸಬಹುದು, ಸತ್ಪ್ರಜೆಗಳನ್ನಾಗಿಸಬಹುದು. ಆದ್ದರಿಂದ ಹಿಂದೂಧರ್ಮದ ಮರ್ಮ-ಮಹಿಮೆಗಳ ವಿಷಯವಾಗಿ ಮಾತನಾಡುತ್ತ ಅವರೆಂದರು: “ಅರ್ಥ ವಿಲ್ಲದ ಕಂದಾಚಾರಗಳಿಗೆ ಕಟ್ಟುಬಿದ್ದಿರುವುದಾಗಲಿ, ಬಾಹ್ಯಾಚರಣೆಗಳನ್ನೇ ಧರ್ಮವೆಂದು ಭಾವಿ ಸುವುದಾಗಲಿ ಕೇವಲ ಮೂರ್ಖತನ. ಈಗ ಏಳ್ನೂರು ವರ್ಷಗಳಿಂದ ನಾವು ದಾಸ್ಯಕ್ಕೆ ಗುರಿಯಾಗಿ ರುವುದು, ಈ ಕಾರಣದಿಂದಲೇ ಮತ್ತು ಧರ್ಮವನ್ನು ಯಥಾರ್ಥವಾಗಿ ಗ್ರಹಿಸದೆ ವಿಕೃತವಾಗಿ ಅರ್ಥೈಸಿದ್ದರಿಂದಲೇ. ಇಂದಿನ ದಿನಗಳಲ್ಲಿ ವ್ಯಭಿಚಾರವೇ ಮೊದಲಾದ ಮಹಾಪರಾಧಗಳನ್ನು ಮಾಡಿದರೂ ಅವನು ಜಾತಿಭ್ರಷ್ಟನಾಗುವುದಿಲ್ಲ; ಆದರೆ ಇತರ ಜಾತಿಯವನ ಕೈಯಿಂದ ಏನನ್ನಾ ದರೂ ತಿಂದರೆ ಮಾತ್ರ ಅವನು ಜಾತಿಭ್ರಷ್ಟನಂತೆ! ಇಂದಿನ ಪಾಪಕಲ್ಪನೆಯೆಲ್ಲ ಕೇವಲ ಆಹಾರಕ್ಕೆ ಮಾತ್ರ ಸಂಬಂಧಪಟ್ಟದ್ದಾಗಿದೆ.” ತಾವು ಸಮುದ್ರ ಪ್ರಯಾಣ ಮಾಡಿದುದನ್ನು ಅವರು ಬಲವಾಗಿ ಸಮರ್ಥಿಸಿಕೊಂಡರು. ಅಲ್ಲದೆ, “ಪರರಾಷ್ಟ್ರಗಳನ್ನು ಸಂದರ್ಶಿಸದಿದ್ದರೆ ನಮ್ಮ ವಿದ್ಯಾಭ್ಯಾಸವು ಪರಿಪೂರ್ಣವಾಗಲಾರದು” ಎಂದು ಅಭಿಪ್ರಾಯಪಟ್ಟರು. ಅನಂತರ ಅವರು ತಾವು ಅಮೆರಿಕದಲ್ಲಿ ಕೈಗೊಂಡ ಕಾರ್ಯಗಳ ಪ್ರಾಶಸ್ತ್ಯವನ್ನು ಕುರಿತು ಹಾಗೂ ಭಾರತದಲ್ಲಿನ ತಮ್ಮ ಕಾರ್ಯಯೋಜನೆ ಗಳ ಸ್ವರೂಪದ ಬಗ್ಗೆ ವಿವರಿಸಿದರು. ತಮ್ಮ ಮಾತನ್ನು ಮುಕ್ತಾಯಗೊಳಿಸುತ್ತ ಅವರು, “ನಾನು ನನ್ನ ರಾಷ್ಟ್ರದ ಶ್ರೇಯೋಭಿವೃದ್ಧಿಗಾಗಿ ದುಡಿಯುವ ಪ್ರಯತ್ನದಲ್ಲಿ ನರಕಕ್ಕೆ ಹೋಗಬೇಕಾಗಿ ಬಂದರೂ ಅದನ್ನು ನನ್ನ ಪಾಲಿನ ಪರಮಭಾಗ್ಯವೆಂದು ಭಾವಿಸುತ್ತೇನೆ” ಎಂದು ಘೋಷಿಸಿದರು. ಹೀಗೆ ಮಹಾರಾಜ ಹಾಗೂ ಸ್ವಾಮೀಜಿಯವರ ನಡುವಣ ಸಂಭಾಷಣೆ ಸುಮಾರು ನಾಲ್ಕು ಗಂಟೆ ಗಳ ಕಾಲ ನಡೆಯಿತು. ಸ್ವಾಮೀಜಿಯವರ ಮಾತುಗಳು ಅಲ್ಲಿದ್ದವರೆಲ್ಲರ ಮೇಲೆ ಆಳವಾದ ಪ್ರಭಾವ ಬೀರಿದುವು.

ಮರುದಿನ ಸ್ವಾಮೀಜಿ ಮಹಾರಾಜನ ಕೋರಿಕೆಯ ಮೇರೆಗೆ ಒಂದು ಸಾರ್ವಜನಿಕ ಭಾಷಣ ಮಾಡಿದರು. ಅದನ್ನು ಕೇಳಿದ ಮಹಾರಾಜನಿಗೆ ಎಷ್ಟು ಸಂತೋಷವಾಯಿತೆಂದರೆ ಅದರ ಮರು ದಿನವೂ ಅವರ ಭಾಷಣವನ್ನಿರಿಸಿದ. ಆದರೆ ಅದರಿಂದಲೂ ತೃಪ್ತಿಯಾಗದೆ, ಸ್ವಾಮೀಜಿ ಅಲ್ಲಿ ಹತ್ತುಹನ್ನೆರಡು ದಿನಗಳಾದರೂ ಉಳಿದುಕೊಳ್ಳಬೇಕು ಮತ್ತು ಒಂದು ಐದಾರು ಭಾಷಣಗಳನ್ನಾ ದರೂ ಮಾಡಬೇಕು ಎಂದು ಕೇಳಿಕೊಂಡ. ಅವನ ಅರಿಕೆಯನ್ನು ಮನ್ನಿಸಿ ಸ್ವಾಮೀಜಿ ಜಮ್ಮುವಿ ನಲ್ಲಿ ಸುಮಾರು ಎಂಟು ದಿನ ಉಳಿದುಕೊಂಡರು.

ಅಲ್ಲಿಯೂ ಯಥಾಪ್ರಕಾರ ಹಲವಾರು ಚರ್ಚೆಗಳು ಸಂಭಾಷಣೆಗಳು ನಡೆದುವು. ಈ ಸಂದರ್ಭ ಗಳಲ್ಲಿ ಒಮ್ಮೆ ಆರ್ಯಸಮಾಜದ ಸ್ವಾಮಿ ಅಚ್ಯುತಾನಂದರೊಂದಿಗೆ ಮಾತನಾಡುತ್ತ ಸ್ವಾಮೀಜಿ ಅವರಿಗೆ ಅವರ ಸಮಾಜದ ಕೆಲವು ಲೋಪದೋಷಗಳನ್ನು ತೋರಿಸಿಕೊಟ್ಟರು. ಆದರೆ ಅವುಗಳ ನ್ನೆಲ್ಲ ಸ್ವಾಮೀಜಿ ಅತ್ಯಂತ ಸಹೃದಯತೆಯಿಂದ ಹೇಳಿದ್ದರಿಂದ ಅಪಾರ್ಥಕ್ಕೆ ಎಡೆಯಿರಲಿಲ್ಲ.

ಜಮ್ಮುವಿನಲ್ಲಿ ಸ್ವಾಮೀಜಿ ಸಂಭಾಷಣೆ-ಪ್ರವಚನಗಳನ್ನು ನಡೆಸಿದುದು ಹೆಚ್ಚಾಗಿ ಹಿಂದಿಯಲ್ಲಿ. ಆ ಹಿಂದೀ ಭಾಷೆಯಲ್ಲೂ ಅವರು ಎಷ್ಟೊಂದು ರಸವತ್ತಾಗಿ ಹಾಗೂ ಶಕ್ತಿಯುತವಾಗಿ ಮಾತ ನಾಡಿದರೆಂದರೆ, ಅದು ಕಾಶ್ಮೀರದ ಮಹಾರಾಜನಿಗೆ ತುಂಬ ಸಂತೋಷವನ್ನುಂಟುಮಾಡಿತು. ಅವರು ಅಷ್ಟು ಚೆನ್ನಾಗಿ ಹಿಂದಿಯಲ್ಲಿ ಮಾತನಾಡಬಲ್ಲರೆಂದು ಯಾರೂ ಊಹಿಸಿರಲಿಲ್ಲ. ಏಕೆಂ ದರೆ, ಬಂಗಾಳಿಗಳು ಹಿಂದಿಯಲ್ಲಿ ಮಾತನಾಡಿದರೆ ಅದರಲ್ಲಿ ಬಂಗಾಳಿಯ ‘ಸೋಂಕು’ಇದ್ದೇ ಇರುತ್ತದೆ. ಆದ್ದರಿಂದ ಹಿಂದಿಯ ಸಹಜ ಮಾಧುರ್ಯ ಅದರಲ್ಲಿರಲಾರದು. ಅಲ್ಲದೆ ಹಿಂದೀ ಭಾಷೆಯಲ್ಲಿ ಇಂಗ್ಲಿಷಿನ ಅನೇಕ ಸೌಲಭ್ಯಗಳಿಲ್ಲ. ಸ್ವಾಮೀಜಿಯವರ ಹಿಂದಿಯ ಶೈಲಿಯನ್ನು ಕೇಳಿ ಪ್ರಭಾವಿತನಾದ ಮಹಾರಾಜ, ಹಿಂದಿಯಲ್ಲಿ ಕೆಲವು ಲೇಖನಗಳನ್ನು ಬರೆದುಕೊಡುವಂತೆ ಕೇಳಿ ಕೊಂಡ. ಅದಕ್ಕೊಪ್ಪಿ ಅವರು ಬರೆದುಕೊಟ್ಟ ಲೇಖನಗಳು ಎಲ್ಲರಿಂದಲೂ ಮೆಚ್ಚುಗೆ ಗಳಿಸಿದುವು.

ಸ್ವಾಮೀಜಿ ಹಾಗೂ ಅವರ ಸಂಗಡಿಗರನ್ನು ಸಿಯಾಲ್​ಕೋಟಿಗೆ ಆಹ್ವಾನಿಸಿ, ಬಳಿಕ ತಮ್ಮೊಂ ದಿಗೆ ಕರೆದೊಯ್ಯಲು ಅಲ್ಲಿನ ಕೆಲವು ನಾಗರಿಕರು ಜಮ್ಮುವಿಗೇ ಬಂದು ಒಂದು ವಾರದಿಂದ ಕಾದಿ ದ್ದರು. ಅವರ ಒತ್ತಾಯದ ಆಮಂತ್ರಣವನ್ನು ಮನ್ನಿಸಿ ಸ್ವಾಮೀಜಿ ಸಿಯಾಲ್​ಕೋಟಿಗೆ ಹೊರ ಟಾಗ, ಅವರನ್ನು ಮಹಾರಾಜ ಮನಸ್ಸಿಲ್ಲದ ಮನಸ್ಸಿನಿಂದ ಬೀಳ್ಕೊಡಬೇಕಾಯಿತು. ಅವರು ಕಾಶ್ಮೀರಕ್ಕೆ ಬಂದಾಗಲೆಲ್ಲ ತನ್ನ ಅತಿಥಿಯಾಗಿ ಉಳಿದುಕೊಳ್ಳಬೇಕೆಂದು ಆತ ಪ್ರಾರ್ಥಿಸಿಕೊಂಡ.

