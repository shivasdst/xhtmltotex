
\chapter{ವೈಭವದ ದುರ್ಗಾಪೂಜೆ}

\noindent

ಬೇಲೂರಿನಲ್ಲಿ ಮಠ ಸ್ಥಾಪನೆಯಾದ ಮೇಲೆ ಸುತ್ತಮುತ್ತಲ ಹಳ್ಳಿಗಳ ಜನರು, ಸ್ವಾಮೀಜಿ ಹಾಗೂ ಇತರ ಸಂನ್ಯಾಸಿಗಳ ವ್ಯವಹಾರ-ಧೋರಣೆಗಳನ್ನು ಇಷ್ಟಪಡದೆ ಬಾಯಿಗೆ ಬಂದಂತೆ ಟೀಕಿಸಲಾರಂಭಿಸಿದ್ದರು. ಈ ಸಂನ್ಯಾಸಿಗಳು ಸಂಪ್ರದಾಯಸಿದ್ಧವಾದ ಕಂದಾಚಾರಗಳನ್ನು ಪರಿ ಪಾಲಿಸುತ್ತಿಲ್ಲ ಎನ್ನುವುದೇ ಆ ಜನರ ಟೀಕೆಗೆ ಮುಖ್ಯ ಕಾರಣ. ಆದ್ದರಿಂದ ಆಶ್ರಮವಾಸಿಗಳ ಮೇಲೆ ಸುಳ್ಳು ವದಂತಿಗಳನ್ನು ಹರಡಿ ಅವರ ಚಾರಿತ್ರ್ಯದ ಬಗ್ಗೆಯೂ ಕಥೆ ಕಟ್ಟಲಾರಂಭಿಸಿದ್ದರು. ಅವರ ಆ ಕಥೆಗಳನ್ನು ಮಠಕ್ಕೆ ಬರುವ ಭಕ್ತರ ಕಿವಿಗೆ ತಲುಪಿಸುವ ಪ್ರಸಾರ ಕೇಂದ್ರವೆಂದರೆ ದೋಣಿಗಳು. ಈ ವಿಷಯ ಸ್ವಾಮೀಜಿಯವರ ಕಿವಿಗೆ ಬಿದ್ದಾಗ ಅವರು ಶಾಂತವಾಗಿ ಹೇಳಿದರು, “ನಿಮಗೆ ಗಾದೆ ಗೊತ್ತಿಲ್ಲವೆ?–ರಸ್ತೆಯಲ್ಲಿ ಆನೆ ಹೋಗುವಾಗ ಅದರ ಹಿಂದೆ ನಾಯಿಗಳೂ ಬೊಗಳುತ್ತ ಹೋಗುತ್ತವೆ. ಜಗತ್ತು ಟೀಕಿಸಿದರೆ ಅದರಿಂದ ಸಂನ್ಯಾಸಿಗೇನೂ ಬಾಧಕವಿಲ್ಲ.” ಅಥವಾ ಅವರು ಹೀಗೂ ಹೇಳುತ್ತಿದ್ದುದುಂಟು: “ಯಾವುದೇ ದೇಶದಲ್ಲಿ ಹೊಸ ಭಾವನೆಗಳನ್ನು ಬೋಧಿಸಿದಾಗ, ಹಳೆಯ ನಂಬಿಕೆಗಳಿಗೆ ಅಂಟಿಕೊಂಡವರು ಅದನ್ನು ಪ್ರತಿಭಟಿಸುವುದು ಸಹಜವೇ. ಪ್ರತಿಯೊಬ್ಬ ಧರ್ಮಸಂಸ್ಥಾಪಕನೂ ಈ ಪರೀಕ್ಷೆಯನ್ನು ಎದುರಿಸಲೇಬೇಕು. ಇಂತಹ ವಿರೋಧ ಗಳಿಲ್ಲದೆ ಉನ್ನತ ಭಾವನೆಗಳೆಂದೂ ಸಮಾಜದ ಒಳಭಾಗವನ್ನು ತಲುಪಲಾರವು.”

ಹೀಗೆ ಸ್ವಾಮೀಜಿಯವರು ಈ ವಿರೋಧ ಹಾಗೂ ಕಟು ಟೀಕೆಗಳು ತಮ್ಮ ಸಂದೇಶಗಳ ಪ್ರಸಾರಕ್ಕೆ ಒಂದು ಬಗೆಯ ‘ಸಹಾಯಕ ಅಡ್ಡಿಗಳು’ ಎಂದು ಪರಿಗಣಿಸಿದ್ದರು. ಆದ್ದರಿಂದ ಇಂತಹ ಸಂದರ್ಭಗಳಲ್ಲಿ ತಮ್ಮನ್ನು ತಾವು ಹೆಚ್ಚಾಗಿ ಸಮರ್ಥಿಸಿಕೊಳ್ಳುವ ಶ್ರಮ ತೆಗೆದುಕೊಳ್ಳು ತ್ತಿರಲಿಲ್ಲ. ಇತರರೂ ಹಾಗೆ ಮಾಡಲು ಬಿಡುತ್ತಿರಲಿಲ್ಲ. “ನೀವು ನಿಮ್ಮ ಕರ್ತವ್ಯಗಳನ್ನು ಅನಾಸಕ್ತಿಭಾವದಿಂದ ಮಾಡಿಕೊಂಡು ಹೋಗಿ. ಇಂದಲ್ಲ ನಾಳೆ ಅದು ಖಂಡಿತ ಫಲಪ್ರದವಾಗು ತ್ತದೆ” ಎಂದು ಅವರು ಎಲ್ಲರಿಗೂ ಒತ್ತಿಒತ್ತಿ ಹೇಳುತ್ತಿದ್ದರು. ಮತ್ತೆ ಕೆಲವೊಮ್ಮೆ “ಒಳ್ಳೆಯ ದನ್ನು ಮಾಡುವವನು ಎಂದಿಗೂ ಹಾಳಾಗುವುದಿಲ್ಲ” ಎಂದು ಹೇಳಿ ಭರವಸೆ ತುಂಬುತ್ತಿದ್ದರು. ಅಂತೂ ಬರಬರುತ್ತ ಸ್ವಾಮೀಜಿಯವರ ಕಾರ್ಯಗಳ ಮೇಲಣ ಹಾಗೂ ಇತರ ಸಾಧುಗಳ ಬಗೆಗಿನ ಟೀಕೆಗಳು ಕಡಿಯೆಯಾಗುತ್ತ ಬಂದು ಸ್ವಾಮೀಜಿ ತೀರಿಹೋಗುವುದಕ್ಕೆ ಮೊದಲೇ ಅದು ನಿಂತು ಹೋಯಿತು. ಇದಕ್ಕೆ ಒಂದು ಮುಖ್ಯ ಕಾರಣವೆಂದರೆ ಮಠದಲ್ಲಿ ಅತ್ಯಂತ ವಿಧಿಬದ್ಧವಾಗಿ, ಸಂಪ್ರದಾಯಬದ್ಧವಾಗಿ ದುರ್ಗಾಪೂಜೆಯನ್ನು ಆಚರಿಸಿದ್ದು.

ಮಠದಲ್ಲಿ ದುರ್ಗಾಪೂಜೆಯನ್ನು ಆಚರಿಸಬೇಕೆಂಬ ಆಲೋಚನೆ ಸ್ವಾಮೀಜಿಯವರ ಮನಸ್ಸಿ ನಲ್ಲಿ ಅನೇಕ ತಿಂಗಳಿನಿಂದಲೂ ಇತ್ತು. ಆದರೆ ಅವರು ಅದನ್ನು ಯಾರ ಮುಂದೂ ಹೇಳಿರಲಿಲ್ಲ. ೧೯ಂ೧ರ ಮೇ ಅಥವಾ ಜೂನಿನಲ್ಲೇ ಅವರು ‘ಅಷ್ಟಾವಿಂಶತಿ ತತ್ತ್ವ’ ಎಂಬ ಗ್ರಂಥವನ್ನು ತರುವಂತೆ ಶರಚ್ಚಂದ್ರ ಚಕ್ರವರ್ತಿಗೆ ಹೇಳಿದ್ದರು. ಇದು ದುರ್ಗಾಪೂಜೆಯ ಸಕಲ ವಿಧಿ ನಿಯಮಗಳನ್ನೂ ವಿವರವಾಗಿ ತಿಳಿಸುವ ಆಧಾರಗ್ರಂಥ. ಅಂದಿನ ಬಂಗಾಳದ ವಿದ್ಯಾವಂತ ವರ್ಗವು ಆ ಪುಸ್ತಕವನ್ನು ‘ಮೂಢನಂಬಿಕೆಗಳ ಕಂತೆ’ ಎಂದು ತುಚ್ಛೀಕರಿಸಿತ್ತು. ಇಂತಹ ಪುಸ್ತಕವನ್ನು ತರಲು ಹೇಳಿದ್ದರ ಉದ್ದೇಶವೇನೆಂಬುದು ಶರಚ್ಚಂದ್ರನಿಗೆ ಅರ್ಥವಾಗಲಿಲ್ಲ. ಆಗ ಸ್ವಾಮೀಜಿ, ಆ ವರ್ಷ ದುರ್ಗಾಪೂಜೆಯನ್ನು ಮಠದಲ್ಲಿ ಕಟ್ಟುನಿಟ್ಟಾಗಿ ಸಂಪ್ರದಾಯದ ಪ್ರಕಾರವೇ ನಡೆಸಬೇಕೆಂಬ ತಮ್ಮ ಮನದಿಂಗಿತವನ್ನು ಅರುಹಿದರು. ಆದರೆ ನವರಾತ್ರಿಗೆ ಇನ್ನು ಕೆಲವೇ ದಿನಗಳಿವೆ ಎನ್ನುವವರೆಗೂ ಅವರು ಅದನ್ನು ಬಹಿರಂಗಪಡಿಸಿರಲಿಲ್ಲ.

ದುರ್ಗಾಪೂಜೆಗೆ ಇನ್ನು ನಾಲ್ಕೈದು ದಿನಗಳಿವೆ; ಆಗೊಂದು ದಿನ ಯಾವ ಕಾರಣಕ್ಕೋ ಕಲ್ಕತ್ತಕ್ಕೆ ಹೋಗಿದ್ದ ಸ್ವಾಮೀಜಿ ದೋಣಿಯಲ್ಲಿ ಮಠಕ್ಕೆ ಬಂದಿಳಿದರು. ಬರುತ್ತಿದ್ದಂತೆಯೇ ಕೇಳಿದರು “ರಾಜಾ (ಸ್ವಾಮಿ ಬ್ರಹ್ಮಾನಂದರು) ಎಲ್ಲಿ?” ಅದೇ ಸಮಯಕ್ಕೆ ಸ್ವಾಮಿ ಬ್ರಹ್ಮಾ ನಂದರು ಗಂಗೆಗೆ ಅಭಿಮುಖವಾಗಿ ಕುಳಿತು ಏನನ್ನೋ ದಿಟ್ಟಿಸುತ್ತಿದ್ದರು. ಅವರನ್ನು ಕಂಡು ಸ್ವಾಮೀಜಿ ಬೇಗಬೇಗ ಅವರ ಬಳಿಗೆ ಬಂದು ಭಾವಭರಿತ ದನಿಯಲ್ಲಿ ಹೇಳಿದರು, “ರಾಜಾ, ಈ ವರ್ಷ ಮಠದಲ್ಲಿ ಪ್ರತಿಮೆಯನ್ನಿಟ್ಟು ದುರ್ಗಾಪೂಜೆ ಮಾಡಬೇಕು! ಬೇಗ ಅದಕ್ಕೆ ಸಿದ್ಧತೆ ಮಾಡು.” ಸ್ವಾಮೀಜಿಯವರ ಈ ಹಠಾತ್ ನಿರ್ಧಾರವನ್ನು ಕೇಳಿ ಬ್ರಹ್ಮಾನಂದರು ಸ್ವಲ್ಪ ಅಚ್ಚರಿಗೊಂಡರಾದರೂ ಹೇಳಿದರು: “ಆಗಲಿ; ಆದರೆ ನನಗೆ ಎರಡು ದಿನ ಅವಕಾಶ ಕೊಡು. ಈ ಕೊನೆಯ ಘಳಿಗೆಯಲ್ಲಿ ಹೋದರೆ ಮಾರುಕಟ್ಟೆಯಲ್ಲಿ ಪ್ರತಿಮೆ ಸಿಗುತ್ತದೊ ಇಲ್ಲವೊ ಹೇಳುವುದು ಕಷ್ಟ. ವಿಚಾರಿಸಿನೋಡಿ ನಿನಗೆ ಹೇಳುತ್ತೇನೆ.” ಆಗ ಸ್ವಾಮೀಜಿ ತಮಗಾದ ಒಂದು ಅದ್ಭುತ ದರ್ಶನದ ವಿಷಯವನ್ನು ಹೊರಗೆಡಹಿದರು–“ನೋಡು, ಈ ದಿನ ನಾನೇನು ಕಂಡೆ ಬಲ್ಲೆಯಾ? ನಾನು ದೋಣಿಯಲ್ಲಿ ಬರುತ್ತಿರುವಾಗ, ಮಠದಲ್ಲಿ ದೇವಿಯ ಪ್ರತಿಮೆಯನ್ನು ಪ್ರತಿಷ್ಠಾಪಿಸಿ ವೈಭವದಿಂದ ದುರ್ಗಾಪೂಜೆಯನ್ನು ನಡೆಸುತ್ತಿರುವಂತೆ ಕಂಡೆ!” ಆಗ ಬ್ರಹ್ಮಾ ನಂದರೆಂದರು, “ಏನಾಶ್ಚರ್ಯ! ನನಗೂ ಅಂಥದೇ ಒಂದು ದರ್ಶನವಾಯಿತು. ಈಗತಾನೆ ನಾನು, ದಶಭುಜಧಾರಿಣಿಯಾದ ದುರ್ಗಾಮಾತೆ, ಅಗೋ ಆ ದಕ್ಷಿಣೇಶ್ವರದ ಕಡೆಯಿಂದ ಗಂಗೆಯ ಮೇಲೆ ನಡೆದುಕೊಂಡು ಬಂದು ಇಲ್ಲೇ ಈ ಬಿಲ್ವವೃಕ್ಷದ ಬಳಿ ನಿಂತಂತೆ ಕಂಡೆ!” ಪರಸ್ಪರರ ಈ ಅನುಭವಗಳನ್ನು ಕೇಳಿ ಇಬ್ಬರೂ ಆನಂದಾಶ್ಚರ್ಯಗೊಂಡರು. ದುರ್ಗಾಪೂಜೆಯನ್ನು ನಡೆಸಬೇಕೆನ್ನುವುದರ ಬಗ್ಗೆ ಇನ್ನು ಸಂಶಯವೇನಿದೆ?

ಈ ಇಬ್ಬರು ಸ್ವಾಮಿಗಳಿಗಾದ ದರ್ಶನಗಳ ವಿಚಾರ ಮಠದಲ್ಲೆಲ್ಲ ಕಾಳ್ಗಿಚ್ಚಿನಂತೆ ಹರಡಿತು. ಎಲ್ಲೆಲ್ಲೂ ಹೊಸ ಉತ್ಸಾಹ, ಸಂಭ್ರಮ, ಸಡಗರಗಳ ವಾತಾವರಣವೇರ್ಪಟ್ಟಿತು. ಪೂಜೆಗೀಗ ದುರ್ಗೆಯ ಪ್ರತಿಮೆ ಬೇಕು. ಆದರೆ, ಪ್ರತಿಮೆ ಬೇಕಾದರೆ ಗುಡಿಗಾರರಿಗೆ ಮೊದಲೇ ಹೇಳಿರಬೇಕಾ ಗುತ್ತದೆ. ಏಕೆಂದರೆ ಅದು ಗಣೇಶನ ವಿಗ್ರಹದಂತೆ ಸಿದ್ಧವಾಗಿರುವುದಿಲ್ಲ. ಈ ಕೊನೆಯ ಘಳಿಗೆಯಲ್ಲಿ ಗುಡಿಗಾರರೆಲ್ಲ ಮೊದಲೇ ಹೇಳಿಟ್ಟವರಿಗಾಗಿ ವಿಗ್ರಹಗಳನ್ನು ತಯಾರಿಸುತ್ತಿರು ತ್ತಾರೆಯೇ ಹೊರತು ಹೊಸದಾಗಿ ಬೇಡಿಕೆ ಸಲ್ಲಿಸಿದವರಿಗೆ ಮಾಡಿಕೊಡಲು ಸಿದ್ಧರಿರುವುದಿಲ್ಲ. ಆದರೂ ದುರ್ಗಾಪೂಜೆ ನಡೆಸುವ ವಿಚಾರದಲ್ಲಿ ಶ್ರೀಮಾತೆಯವರ ಅನುಮತಿ ಪಡೆದು ಪ್ರತಿಮೆ ಯನ್ನು ಹುಡುಕಲು ಜನರನ್ನು ಕಳಿಸಲಾಯಿತು.

ಜಗನ್ಮಾತೆಯ ಕೃಪೆಯಿಂದ ವಿಗ್ರಹ ಸಿಕ್ಕಿಯೇ ಬಿಟ್ಟಿತು. ಹೇಗೆಂದರೆ ವಿಗ್ರಹ ತಯಾರಿಸಲು ಬೇಡಿಕೆ ಸಲ್ಲಿಸಿದ್ದವರೊಬ್ಬರು ಅದೇಕೋ ಬೇಡಿಕೆಯನ್ನು ಹಿಂತೆಗೆದುಕೊಂಡುಬಿಟ್ಟಿದ್ದರು. ಹೀಗೆ ಒಂದೇ ಒಂದು ವಿಗ್ರಹ ಉಳಿದುಹೋಗಿದ್ದು ಅದು ಅವರಿಗೆ ಸಿಕ್ಕಿತು. ತಕ್ಷಣವೇ ಸ್ವಾಮೀಜಿ ಯವರು ಪ್ರೇಮಾನಂದರನ್ನು ಕರೆದುಕೊಂಡು ಶ್ರೀಮಾತೆಯವರನ್ನು ಕಾಣಲು ಕಲ್ಕತ್ತಕ್ಕೆ ಧಾವಿಸಿ ದರು. ಈಗ ಪೂಜೆಯ ಸಂಕಲ್ಪವನ್ನು ಮಾಡಬೇಕಾಗಿದ್ದುದು ಶ್ರೀಮಾತೆಯವರ ಹೆಸರಿನಲ್ಲಿ. ಅದಕ್ಕೆ ಅವರ ಅನುಮತಿ ಬೇಡಲೆಂದೇ ಸ್ವಾಮೀಜಿ ಅಲ್ಲಿಗೆ ಹೊರಟದ್ದು. ಸಂಕಲ್ಪವೆಂದರೆ ಶಾಸ್ತ್ರವಿಧಿಯನ್ನು ಆಚರಿಸುವವನು ತನ್ನ ಬೇಡಿಕೆಯನ್ನು–ಅದು ಪ್ರಾಪಂಚಿಕವಾಗಿರಬಹುದು ಅಥವಾ ಪಾರಮಾರ್ಥಿಕವಾಗಿರಬಹುದು–ವಿಧ್ಯುಕ್ತವಾಗಿ ಘೋಷಿಸುವುದು. ಇದನ್ನು ಮಾಡಿದ ಹೊರತು ಕರ್ಮಕಾಂಡದ ಯಾವ ಕಾರ್ಯವೂ ಫಲಿಸುವುದಿಲ್ಲ. ಆದರೆ ಸಂನ್ಯಾಸಿಗಳು ಸಂಕಲ್ಪ ದಿಂದ ಕೂಡಿದ ಯಾವ ಶಾಸ್ತ್ರವಿಧಿಯನ್ನೂ ಆಚರಿಸುವಂತಿಲ್ಲ. ಶ್ರೀಮಾತೆಯವರು ದುರ್ಗಾಪೂಜೆ ಯನ್ನು ನಡೆಸಲು ಸಂತೋಷದಿಂದ ಅನುಮತಿ ನೀಡಿದರು.

“ಮಠದಲ್ಲಿ ಈ ವರ್ಷ ಪ್ರತಿಮೆಯ ಮೂಲಕ ಪೂಜೆ ನಡೆಯುತ್ತದೆಯಂತೆ!” ಎಂಬ ಸುದ್ದಿ ನಗರದಲ್ಲೆಲ್ಲ ಹರಡಿತು. ಇದನ್ನು ಕೇಳಿ ಆನಂದಿತರಾದ ಗೃಹೀ ಭಕ್ತರೆಲ್ಲ ಕಾರ್ಯಕ್ರಮವನ್ನು ಯಶಸ್ವಿಯಾಗಿಸಲು ಬಹುಸಂಖ್ಯೆಯಲ್ಲಿ ಬಂದು ಸಹಾಯಹಸ್ತ ನೀಡಿದರು.

ಶ್ರೀದುರ್ಗಾಪೂಜೆಯನ್ನು ಹೀಗೆ ವಿಧಿಯುಕ್ತವಾಗಿ ನಡೆಸುವುದರ ಹಿಂದೆ ಸ್ವಾಮೀಜಿಯವ ರೊಂದು ವಿಶೇಷ ಉದ್ದೇಶವಿಟ್ಟುಕೊಂಡಿದ್ದರು. ಈ ಹಿಂದೆಯೇ ಹೇಳಿದಂತೆ, ಮಠದ ರೀತಿನೀತಿ ಗಳನ್ನು ಸುತ್ತಲಿನ ಸಂಪ್ರದಾಯಸ್ಥ ಜನ ಕಟುವಾಗಿ ಟೀಕಿಸುತ್ತಿದ್ದರು. ಮ್ಲೇಚ್ಛರ ದೇಶಕ್ಕೆ ಹೋಗಿ, ಅವರ ಆಹಾರವನ್ನೇ ತಿಂದವನ ನಾಯಕತ್ವದಲ್ಲಿ ನಡೆಯುವ ಈ ಮಠದ ಸಂನ್ಯಾಸಿಗಳು ಹಿಂದೂಗಳೇ ಅಲ್ಲ ಎಂದು ಜನ ತಿರಸ್ಕಾರದಿಂದ ಮಾತನಾಡುತ್ತಿದ್ದರು. ಅದರ ಬಗ್ಗೆ ಸ್ವಾಮೀಜಿ ಮೊದಲು ನಿರ್ಲಿಪ್ತಭಾವ ತಾಳಿದ್ದರೂ ಈಗ ಅವರ ಭಾವನೆ ಬದಲಾಗಿತ್ತು. ಈ ಆರೋಪಗಳು ಆಧಾರರಹಿತವೆಂದು ತೋರಿಸಿಕೊಡಲು ಮತ್ತು ರಾಮಕೃಷ್ಣ ಮಹಾಸಂಘವು ಹಿಂದೂ ಪುನರು ತ್ಥಾನ ಚಳವಳಿಯ ಮುಂದಾಳ್ತನ ವಹಿಸುವಂತೆ ಮಾಡಲು ಏನಾದರೊಂದು ಸ್ಪಷ್ಟ ಕ್ರಮ ಕೈಗೊಳ್ಳಬೇಕೆಂದು ಅವರಿಗನ್ನಿಸಿತು. ಭಾರತದಾದ್ಯಂತ ನವರಾತ್ರಿಯನ್ನು ಸಂಭ್ರಮದಿಂದ ಆಚರಿ ಸುತ್ತಾರಾದರೂ ಬಂಗಾಳದಲ್ಲಿ ಅದಕ್ಕೊಂದು ವಿಶೇಷ ಸ್ಥಾನ. ಅಲ್ಲಿನ ಜನಜೀವನಕ್ಕೂ ದುರ್ಗಾ ಪೂಜೆಗೂ ಬಿಡಿಸಲಾಗದ ನಂಟು. ಆದ್ದರಿಂದ ಈ ದುರ್ಗಾಪೂಜೆಯನ್ನು ಪ್ರತಿಮಾಸಹಿತವಾಗಿ ಮಾಡುವುದರಿಂದ ಜನರ ಅಪನಂಬಿಕೆಗಳೆಲ್ಲ ದೂರವಾಗುವುವೆಂದು ಸ್ವಾಮೀಜಿ ಭಾವಿಸಿದರು.

ಮಠದ ಆವರಣದ ಹುಲ್ಲು ಹಾಸಿನ ಮೇಲೆ ಒಂದು ಚಪ್ಪರ ಕಟ್ಟಿ ದೇವಿಯ ಪ್ರತಿಷ್ಠಾಪನೆಗೆ ಸಿದ್ಧಗೊಳಿಸಲಾಯಿತು. ಅತ್ಯಂತ ಸುಂದರವಾಗಿ ಅಲಂಕೃತವಾಗಿದ್ದ ದೇವಿಯ ಪ್ರತಿಮೆಯನ್ನು ಷಷ್ಠಿಗೆ ಎರಡು ದಿನ ಮುಂಚಿತವಾಗಿ ತರಲಾಯಿತು. ಶ್ರೀಮಾತೆ ಹಾಗೂ ಅವರ ಪರಿವಾರದವರಿ ಗಾಗಿ ಸಮೀಪದ ನೀಲಾಂಬರ ಬಾಬುವಿನ ಮನೆಯನ್ನು ಸಜ್ಜುಗೊಳಿಸಲಾಯಿತು.

ಶ್ರೀಮಾತೆಯವರ ಅಪ್ಪಣೆ ಪಡೆದು ಬ್ರಹ್ಮಚಾರಿ ಕೃಷ್ಣಲಾಲರು ಅರ್ಚಕ ಪೀಠವನ್ನೇರಿದರು. ಸ್ವಾಮಿ ರಾಮಕೃಷ್ಣಾನಂದರ ತಂದೆ ಶ್ರೀ ಈಶ್ವರಚಂದ್ರ ಭಟ್ಟಾಚಾರ್ಯರು ‘ತಂತ್ರಧಾರಕ’ ರಾದರು. (ತಂತ್ರಧಾರಕರೆಂದರೆ ಅರ್ಚಕರು ಮಾಡುವ ಪೂಜಾ ವಿಧಾನಗಳಲ್ಲಿ ಲೋಪದೋಷ ಗಳಾಗದಂತೆ ಸಕಾಲಿಕ ನಿರ್ದೇಶನ ನೀಡುವವರು; ಮಂತ್ರ ಲೋಪವಾಗದಂತೆ ಮಂತ್ರವನ್ನು ಎತ್ತಿಕೊಡುವವರು.) ಇವರು ನಿಷ್ಠಾವಂತ ವೈದಿಕರಲ್ಲದೆ ಮಂತ್ರಗಳಲ್ಲೂ ಪೂಜಾವಿಧಿಗಳಲ್ಲೂ ಪಾರಂಗತರು. ಎಲ್ಲ ವಿಧ್ಯುಕ್ತ ಕರ್ಮಗಳನ್ನೂ ಕಟ್ಟುನಿಟ್ಟಾಗಿ ಆಚರಿಸಲಾಯಿತು. ಆದರೆ ಬಂಗಾಳದಲ್ಲಿ ರೂಢಿಯಲ್ಲಿರುವ ಆಡಿನ ಬಲಿಯನ್ನು ಮಾತ್ರ ಶ್ರೀಮಾತೆಯವರ ಆದೇಶದಂತೆ ಕೈಬಿಡಲಾಯಿತು.

ಈ ಪೂಜೆಯ ಮುಖ್ಯ ಅಂಶವೆಂದರೆ ಸಾಮೂಹಿಕ ಅನ್ನದಾನ. ಪೂಜೆಯ ಮೂರು ದಿನವೂ ಬಂದಬಂದವರಿಗೆಲ್ಲ ಪುಷ್ಕಳವಾಗಿ ಪ್ರಸಾದ ನೀಡಲಾಯಿತು. ಸಹಸ್ರಾರು ದರಿದ್ರನಾರಾಯಣ ರನ್ನು ಸತ್ಕರಿಸಲಾಯಿತು. ಸ್ವಾಮಿಗಳು ಖುದ್ದಾಗಿ ನಿಂತು ಜನರನ್ನು ಸ್ವಾಗತಿಸಿದರು. ಅಲ್ಲದೆ ದುರ್ಗಾಪೂಜೆಗಾಗಿ ವಿಶೇಷವಾಗಿ ಆಹ್ವಾನಿಸಲಾಗಿದ್ದ ಬೇಲೂರು, ದಕ್ಷಿಣೇಶ್ವರಗಳ ಬ್ರಾಹ್ಮಣರೂ ಪಂಡಿತರೂ ಆಹ್ವಾನವನ್ನು ಮನ್ನಿಸಿ ಬಂದು ಪೂಜಾದಿಗಳಲ್ಲಿ ಭಾಗವಹಿಸಿದರು. ಜೊತೆಗೆ ಸಾಮೂಹಿಕ ಭೋಜನದಲ್ಲೂ ಪಾಲ್ಗೊಂಡು ಸಂತೃಪ್ತರಾಗಿ ಹಿಂದಿರುಗಿದರು. ಇದಾದಮೇಲೆ ಸುತ್ತಮುತ್ತಲಿನ ಸಂಪ್ರದಾಯಸ್ಥ ಹಿಂದುಗಳು ಮಠದ ಮೇಲೆ ವಿಷ ಕಾರುವುದನ್ನು ಬಿಟ್ಟು ಬಿಟ್ಟರು.

ಇತ್ತ ಪೂಜಾದಿಗಳೆಲ್ಲ ಸಂಭ್ರಮದಿಂದ ನಡೆಯುತ್ತಿದ್ದಾಗ ಸ್ವಾಮೀಜಿಯವರಿಗೆ ಮಾತ್ರ ಅದರಲ್ಲಿ ಭಾಗವಹಿಸುವ ಭಾಗ್ಯವಿರಲಿಲ್ಲ. ಪೂಜೆಯ ಮೊದಲನೆಯ ದಿನ, ಎಂದರೆ ಸಪ್ತಮಿ ಯಂದು ಅವರು ಪೂಜೆಯನ್ನು ವೀಕ್ಷಿಸಿದರು. ಆದರೆ ಆ ರಾತ್ರಿಯೇ ಅವರಿಗೆ ಜ್ವರ ಹಿಡಿದದ್ದ ರಿಂದ ಮರುದಿನ ಬೆಳಿಗ್ಗೆ ಎದ್ದು ಬರಲೂ ಸಾಧ್ಯವಾಗಲಿಲ್ಲ. ಜ್ವರ ನೂರೈದು ಡಿಗ್ರಿಯವರೆಗೂ ಏರಿತು. ಇದರೊಂದಿಗೆ ಆಸ್ತಮಾ ಕೂಡ ಕೆರಳಿತು. ಆದರೆ ಮರುದಿನ ಪೂಜೆಯ ಅತಿ ಮುಖ್ಯ ಭಾಗವಾದ ಸಂಧಿಪೂಜೆಯ ವೇಳೆ ನಿಧಾನವಾಗಿ ಪೂಜಾಸ್ಥಾನಕ್ಕೆ ನಡೆದುಬಂದರು. (ಸಂಧಿಪೂಜೆ ಯೆಂದರೆ ಅಷ್ಟಮಿ ಮತ್ತು ನವಮಿ ತಿಥಿಗಳು ಸಂಧಿಸುವ ಕಾಲದಲ್ಲಿ ನಡೆಸುವ ಪೂಜೆ. ಈ ಸಂಧಿಕಾಲದಲ್ಲೇ ಜಗನ್ಮಾತೆ ಶ್ರೀದುರ್ಗೆಯು ಮಹಿಷಾಸುರನನ್ನು ಸಂಹರಿಸಿದ್ದರಿಂದ ಇದು ಪೂಜೆಗೆ ಅತ್ಯಂತ ಪ್ರಶಸ್ತ ಕಾಲವೆಂದು ಪರಿಗಣಿಸಲಾಗಿದೆ.) ಸ್ವಾಮೀಜಿಯವರು ದೇವಿಯ ಪಾದಪದ್ಮಗಳಿಗೆ ಮೂರು ಸಲ ಪುಷ್ಪಾಂಜಲಿಯನ್ನರ್ಪಿಸಿ ನಮಸ್ಕರಿಸಿದರು. ಮರುದಿನ, ಎಂದರೆ ನವಮಿಯಂದು ಅವರು ಸ್ವಲ್ಪ ಸುಧಾರಿಸಿಕೊಂಡರು. ಅಂದು ರಾತ್ರಿ ಅವರು ಶ್ರೀರಾಮಕೃಷ್ಣರು ಹೇಳುತ್ತಿದ್ದ ಕೆಲವು ದೇವೀ ಕೀರ್ತನೆಗಳನ್ನು ಹಾಡಿದರು. ವಿಜಯದಶಮಿಯಂದು ಶ್ರೀಮಾತೆ ಯವರು ಯಜ್ಞದಕ್ಷಿಣೆಯನ್ನು ಸಲ್ಲಿಸಿ ಪೂಜಾಫಲವನ್ನು ಸ್ವೀಕರಿಸಿದರು. ರಾತ್ರಿಯ ವೇಳೆಗೆ ಪ್ರತಿಮೆಯನ್ನು ವಿಧಿವತ್ತಾಗಿ ಗಂಗೆಯಲ್ಲಿ ವಿಸರ್ಜಿಸಲಾಯಿತು. ಪೂಜಾದಿಗಳು ಅತ್ಯಂತ ಯಶಸ್ವಿ ಯಾಗಿ ನಡೆದ ಪರಿಯನ್ನು ಕಂಡು ಆನಂದಿತರಾದ ಶ್ರೀಮಾತೆಯವರು ಆಶ್ರಮವಾಸಿಗಳನ್ನೆಲ್ಲ ಆಶೀರ್ವದಿಸಿ ಬಾಘ್​ಬಜಾರಿಗೆ ಹಿಂದಿರುಗಿದರು.

ಅದೇ ವರ್ಷ ಮಠದಲ್ಲಿ ಲಕ್ಷ್ಮೀಪೂಜೆಯನ್ನೂ ಕಾಳೀಪೂಜೆಯನ್ನೂ ಆಚರಿಸಲಾಯಿತು. ಕಾಳೀಪೂಜೆಯಂದು ಸ್ವಾಮೀಜಿ ವಿಶೇಷ ಆನಂದೋತ್ಸಾಹಭರಿತರಾಗಿದ್ದರು. ಅಂದು ಪೂಜೆ ಪ್ರಾರಂಭವಾಗುವ ಮುನ್ನ ಅವರು ಮಂಟಪದಲ್ಲಿ ಕುಳಿತು ಧ್ಯಾನಮಗ್ನರಾದರು; ಕೆಲವೇ ನಿಮಿಷಗಳಲ್ಲಿ ಗಾಢ ಸಮಾಧಿಯಲ್ಲಿ ಮುಳುಗಿಹೋದರು. ಬಾಹ್ಯನೋಟಕ್ಕೆ ಶರೀರದಲ್ಲಿ ಜೀವದ ಲಕ್ಷಣಗಳೇ ಕಾಣುತ್ತಿರಲಿಲ್ಲ. ಬಹಳ ಹೊತ್ತಿನವರೆಗೂ ಅವರು ಅದೇ ಸ್ಥಿತಿಯಲ್ಲಿದ್ದರು. ಕಡೆಗೆ ಸ್ವಾಮಿ ಪ್ರೇಮಾನಂದರು ಅವರ ಕಿವಿಯಲ್ಲಿ ಗಟ್ಟಿಯಾಗಿ ಶ್ರೀರಾಮಕೃಷ್ಣನಾಮವನ್ನು ಉಚ್ಚರಿಸಿದ ಮೇಲೆಯೇ ಅವರು ಬಾಹ್ಯ ಪ್ರಪಂಚಕ್ಕೆ ಮರಳಿದ್ದು.

ಮುಂದಿನ ಮುಖ್ಯ ಹಬ್ಬವೇ ಜಗದ್ಧಾತ್ರೀ ಪೂಜೆ. ಅಂದು ಸ್ವಾಮೀಜಿಯವರ ತಾಯಿ ಭುವನೇಶ್ವರಿದೇವಿ ಸ್ವಾಮಿಗಳನ್ನೆಲ್ಲ ತನ್ನ ಮನೆಗೇ ಆಹ್ವಾನಿಸಿದಳು. ಅಂದಿನ ಏರ್ಪಾಡುಗಳನ್ನೆಲ್ಲ ಸ್ವಾಮೀಜಿಯವರೇ ಸ್ವತಃ ನಿಂತು ನೋಡಿಕೊಂಡರು.

ಕೆಲದಿನಗಳ ಬಳಿಕ ಸ್ವಾಮೀಜಿ ತಮ್ಮ ತಾಯಿಯ ಆದೇಶದಂತೆ ಕಾಳೀಘಾಟಿನ ಕಾಳೀ ದೇವಾಲಯವನ್ನು ದರ್ಶಿಸಿದರು. ಅವರಿನ್ನೂ ಶಿಶು ನರೇಂದ್ರನಾಗಿದ್ದಾಗ ಒಮ್ಮೆ ತೀವ್ರ ಕಾಯಿಲೆಗೆ ಗುರಿಯಾಗಿದ್ದರು. ಆಗ ತಾಯಿ ಭುವನೇಶ್ವರಿ, ಮಗುವಿಗೆ ಗುಣವಾದರೆ ಕಾಳಿಗೆ ವಿಶೇಷ ಪೂಜೆ ಮಾಡಿಸುವುದಾಗಿಯೂ ಮಗು ದೊಡ್ಡವನಾದ ಮೇಲೆ ಅವನಿಂದ ‘ಉರುಳುಸೇವೆ’ ಮಾಡಿಸುವುದಾಗಿಯೂ ಹರಕೆ ಹೊತ್ತಿದ್ದಳು. ಆದರೆ ಮಗನಿಗೆ ಕಾಯಿಲೆ ಬಿಟ್ಟುಹೋದ ಮೇಲೆ ಹರಕೆ ಮರೆತುಹೋಗಿತ್ತು. ಇಷ್ಟು ವರ್ಷಗಳಲ್ಲಿ ಅದು ಮತ್ತೆ ನೆನಪಿಗೆ ಬಂದೇ ಇರಲಿಲ್ಲ. ಆದರೆ ಈಗ ಅವರು ಶರೀರ ತ್ಯಜಿಸಲು ಒಂದು ವರ್ಷವೂ ಇಲ್ಲದಿರುವಾಗ ಆಕೆಗೆ ತನ್ನ ಆ ಹರಕೆ ನೆನಪಾಯಿತು! ಅದು ಹೇಗೆ ಈಗ ಇದ್ದಕ್ಕಿದ್ದಂತೆ ನೆನಪಿಗೆ ಬಂದಿತು? ಅದಕ್ಕೆ ಕಾರಣವಿದೆ. ಸ್ವಾಮೀಜಿ ಈಗ ಮತ್ತೆ ಮತ್ತೆ ಕಾಯಿಲೆ ಬೀಳುತ್ತಿರುವುದು ಕಾಣುತ್ತಿದೆಯಲ್ಲ! ಅವರು ಲೋಕದ ದೃಷ್ಟಿಗೆ ಸಂನ್ಯಾಸಿಯಿರಬಹುದು, ವೇದಾಂತಿಯಿರಬಹುದು. ಆದರೆ ಊರಿಗೆ ಅರಸಾದರೂ ತಾಯಿಗೆ ಮಗನಲ್ಲವೆ! ಮಗನ ಕಾಯಿಲೆಯನ್ನು ನೋಡಿ ಹೆತ್ತ ಕರುಳಿಗೆ ನೋವಾಗುವುದಿಲ್ಲವೆ? ನೋವಾದಾಗ ದೇವರ ನೆನಪಾಗುವುದಿಲ್ಲವೆ? ಹೀಗೆ ಭುವನೇಶ್ವರಿಗೆ ತಾನು ಹಲವು ವರ್ಷಗಳ ಹಿಂದೆ ಹೇಳಿಕೊಂಡಿದ್ದ ಹರಕೆ ನೆನಪಿಗೆ ಬಂದಿತು. ತಡವಾಯಿತು, ನಿಜ. ಆದರೇನಂತೆ? ಮಗನನ್ನು ಕಲ್ಕತ್ತದ ಕಾಳೀದೇವಾಲಯಕ್ಕೆ ಬರಮಾಡಿಕೊಂಡಳು.

ಆಗ ಸ್ವಾಮೀಜಿಯವರಿಗೆ ಸ್ವಲ್ಪ ಅಸೌಖ್ಯವಿತ್ತಾದರೂ ತಾಯಿಯ ಸಂತೋಷಕ್ಕಾಗಿ ಅಲ್ಲಿಗೆ ಹೋದರು. ಯಾವ ಜಗನ್ಮಾತೆ ತನ್ನ ದಿವ್ಯ ಪುತ್ರನ ಇಹಲೋಕ ಜೀವನವನ್ನು ಇನ್ನು ಕೆಲವೇ ತಿಂಗಳಲ್ಲಿ ಕೊನೆಗೊಳಿಸಲಿರುವಳೋ ಆ ಮಾತೆಯ ಬಳಿಗೆ ತಮ್ಮ ಜನ್ಮದಾತೆಯ ಮಾತಿ ಗನುಸಾರವಾಗಿ ಹೋದರು. ಅಲ್ಲಿ ‘ಆದಿಗಂಗೆ’ಯಲ್ಲಿ ಮಿಂದು ಒದ್ದೆ ಬಟ್ಟೆಯಲ್ಲೇ ದೇವಸ್ಥಾನದ ವರೆಗೂ ನಡೆದುಕೊಂಡು ಹೋಗಿ, ದೇವಿಯ ಮುಂದೆ ಮೂರು ಸಲ ಉರುಳಿದರು. ಬಳಿಕ ಪೂಜೆ ಸಲ್ಲಿಸಿ ದೇವಸ್ಥಾನದ ಸುತ್ತ ಏಳು ಸಲ ಪ್ರದಕ್ಷಿಣೆ ಬಂದರು. ಆಮೇಲೆ ಅಲ್ಲಿನ ನಾಟ್ಯ ಮಂದಿರದಲ್ಲಿ ತಾವೇ ಹೋಮ ಮಾಡಿದರು. ಸ್ವಾಮೀಜಿಯವರು ಸಮುದ್ರವನ್ನು ದಾಟಿದ ಸಂನ್ಯಾಸಿಯೆಂದು ತಿಳಿದಿದ್ದರೂ ಅಲ್ಲಿನ ಅರ್ಚಕರು ಅವರನ್ನು ಆದರದಿಂದ ಸ್ವಾಗತಿಸಿದರಲ್ಲದೆ, ಅವರಿಗೆ ಬೇಕಾದಂತೆ ಪೂಜೆ ಸಲ್ಲಿಸಲು ನೆರವಾದರು.

ಸ್ವಾಮೀಜಿಯವರು ದುರ್ಗೆ, ಕಾಳಿ, ಲಕ್ಷ್ಮಿ ಜಗದ್ಧಾತ್ರೀ ದೇವಿಯರನ್ನು ವಿಗ್ರಹದ ಮೂಲಕ ಆರಾಧಿಸಿದ್ದನ್ನು ಕಂಡಾಗ ಅಲ್ಲಿನ ಎಷ್ಟೋ ಜನಕ್ಕೆ ಅದೊಂದು ವಿಪರ್ಯಾಸವಾಗಿ ಕಂಡಿತು. ‘ಪಕ್ಕಾ ಅದ್ವೈತಿಯಾದ ವಿವೇಕಾನಂದರು ಇಂತಹ ವೈಧೀ ಪೂಜೆಯನ್ನು–ಅದೂ ಪ್ರತಿಮೆಯಲ್ಲಿ! –ಅದು ಹೇಗೆ ಮಾಡಿದರು?’ ಎಂದೂ ಎಷ್ಟೋ ಜನ ತಲೆಕೆಡಿಸಿಕೊಂಡದ್ದುಂಟು. ಆದರೆ ಎಂದಿನಿಂದಲೂ ಅವರು ಈ ಬಗ್ಗೆ ತಮ್ಮ ನಿಲುವನ್ನು ಸ್ಪಷ್ಟವಾಗಿಯೇ ತಿಳಿಸಿದ್ದರಲ್ಲವೆ? ಶಿಕಾಗೋ ಉಪನ್ಯಾಸಮಾಲೆಯಿಂದ ಹಿಡಿದು ಇತ್ತೀಚಿನ ಢಾಕಾ ಭಾಷಣದವರೆಗೂ ಯಾವಾಗಲೂ ಅವರು, ಎಲ್ಲ ರೀತಿಯ ಪೂಜೆಯೂ ಅದೇ ಭಗವಂತನಲ್ಲಿಗೇ ಕರೆದೊಯ್ಯುವುದೆಂದು ಸಾರಿದ್ದವ ರಲ್ಲವೆ? ಮೇಲಾಗಿ, ಎಷ್ಟಾದರೂ ಅವರು ಆ ದಕ್ಷಿಣೇಶ್ವರದ ‘ಹುಚ್ಚುಪೂಜಾರಿ’ಯ ಶಿಷ್ಯಾಗ್ರಣಿ ಯಲ್ಲವೆ? ಢಾಕಾದಲ್ಲಿ ಅವರು ಮೂರ್ತಿಪೂಜೆಯ ಬಗ್ಗೆ ಮಾತನಾಡುತ್ತ ಹೇಳಿದ್ದುಂಟು– ‘ನಾನು ಆ ವಿಗ್ರಹಾರಾಧಕ ಬ್ರಾಹ್ಮಣನ ಪವಿತ್ರ ಪಾದಧೂಳಿಯಿಂದ ಪುನೀತನಾಗದಿದ್ದಲ್ಲಿ ಈಗೆಲ್ಲಿರುತ್ತಿದ್ದೆ?’ ಎಂದು.

ಮಠದಲ್ಲಿ ಜಗನ್ಮಾತೆಯ ಉತ್ಸವಗಳೆಲ್ಲ ಒಂದಾದಮೇಲೊಂದರಂತೆ ಸಂಭ್ರಮದಿಂದ ನಡೆದವು. ಈಗ ಮಠದ ವಾತಾವರಣದಲ್ಲಿ ಒಂದು ಬಗೆಯ ಹೊಸ ಹುರುಪು ಕಂಡುಬರ ಲಾರಂಭಿಸಿದೆ. ಆದರೆ ಸಕಲರ ಹೃದಯದಲ್ಲೂ ಒಂದೇ ಚಿಂತೆ, ಆತಂಕ–ಅದೇ ಕುಂದುತ್ತಿರುವ ಸ್ವಾಮೀಜಿಯವರ ಆರೋಗ್ಯದ ಬಗ್ಗೆ. ಅವರ ಸಾಮಾನ್ಯ ದೇಹಸ್ಥಿತಿಯೊಂದಿಗೆ, ಆತಂಕಕ್ಕೆ ಕಾರಣವಾಗಿದ್ದ ಮತ್ತೊಂದು ಅಂಶವೆಂದರೆ, ಅವರ ಬಲಗಣ್ಣಿನ ದೃಷ್ಟಿಶಕ್ತಿ ಕುಗ್ಗುತ್ತಿದ್ದುದು. ಈ ನಡುವೆ ನೆಗಡಿ, ಆಸ್ತಮಾಗಳೂ ಮತ್ತೆಮತ್ತೆ ಮುತ್ತಲಾರಂಭಿಸಿದ್ದುವು. ಮತ್ತು ರಕ್ತದಲ್ಲಿ ‘ಆಲ್ಬುಮಿನ್​’ನ ಅಂಶ ಹೆಚ್ಚಾದಾಗ ಇತರೆಲ್ಲ ತೊಂದರೆಗಳೂ ಉಲ್ಬಣಿಸುತ್ತಿದ್ದುವು. ಅವರ ದೇಹಸ್ಥಿತಿಯ ಬಗ್ಗೆ ವಿವರವಾಗಿ ತಿಳಿಸಬೇಕೆಂದು ಕ್ರಿಸ್ಟೀನ ಒತ್ತಾಯಪಡಿಸುತ್ತಿದ್ದರಿಂದ, ಇವರಿಗೆ ಇಷ್ಟವಿಲ್ಲದಿದ್ದರೂ ಆ ಬಗ್ಗೆ ಬರೆದರು–

“... ನಾನೀಗ ಡಾಕ್ಟರರ ಸಲಹೆಯ ಮೇರೆಗೆ ಹೀಗೆ ವಿಶ್ರಾಂತಿಯಲ್ಲಿರುವುದರಿಂದ ಈಗಾಗಲೇ ಸ್ವಲ್ಪ ಅನುಕೂಲ ಕಾಣಿಸುತ್ತಿದೆ. ನಾನು ಬಹಳ ಹೊತ್ತು ನಿದ್ರಿಸುತ್ತಿದ್ದೇನೆ, ಹಸಿವೂ ಹೆಚ್ಚಾಗಿದೆ ಮತ್ತು ಉಂಡ ಆಹಾರವೂ ಚೆನ್ನಾಗಿ ಅರಗುತ್ತಿದೆ. ಹೀಗೆ ಯಾವಾಗಲೂ ಮಲಗಿಕೊಂಡಿದ್ದರೂ ಕೂಡ ನಿದ್ರೆ-ಹಸಿವುಗಳು ಹೆಚ್ಚಾಗಿರುವುದು ಆಶ್ಚರ್ಯವಲ್ಲವೆ?... ಡಾಕ್ಟರ್ ಹೇಳುತ್ತಾರೆ– ನೀವು ಮೂರು ತಿಂಗಳು ಹಾಸಿಗೆಗೇ ಅಂಟಿಕೊಂಡಿದ್ದರೆ ಸಂಪೂರ್ಣ ಗುಣವಾಗುತ್ತೀರಿ, ಎಂದು.”

ಆದರೆ ಅವರಿಗೆ ಡಾಕ್ಟರರ ಮಾತನ್ನು ಅಕ್ಷರಶಃ ಪಾಲಿಸಲು ಸಾಧ್ಯವೇ ಇರಲಿಲ್ಲ. ಆದ್ದರಿಂದ ಮತ್ತೆ ಬರೆಯುತ್ತಾರೆ:

“ನಾನಿನ್ನು ಪತ್ರವನ್ನು ಮುಗಿಸಬೇಕು. ಈಗ ನಾನು ಬಾತುಗಳನ್ನೂ ವರಟೆ\footnote{*ಬಾತುಕೋಳಿಗಳಿಗೂ ಹಂಸಕ್ಕೂ ನಡುವಿನ ಒಂದು ಬಗೆಯ ಪಕ್ಷಿ.}ಗಳನ್ನೂ ನೋಡಿ ಕೊಳ್ಳಲು–ಐದೇ ನಿಮಿಷದ ಮಟ್ಟಿಗೆ–ಹೋಗಬೇಕು. ಯಾವಾಗಲೂ ಮಲಗಿಕೊಂಡೇ ಇರ ಬೇಕೆಂಬ ಡಾಕ್ಟರರ ಆಜ್ಞೆಯನ್ನು ಉಲ್ಲಂಘಿಸಲೇಬೇಕು. ಆ ಬಾತುಗಳ ಪೈಕಿ ಒಂದು ಯಾವಾ ಗಲೂ ಭಯಪಡುತ್ತಿರುವ ಪುಕ್ಕಲು ಹಕ್ಕಿ. ಅದಕ್ಕೆ ಯಾವಾಗಲೂ ಏನೋ ಗಾಬರಿ, ಏನೋ ಕಳವಳ. ಅದು ನನಗೆ ಗೊತ್ತಿರುವ–ಬೇರೊಂದು ಕಡೆಯಿರುವ–ಮತ್ತೊಂದು ಹಕ್ಕಿಯಂತೆಯೇ (ಎಂದರೆ ಕ್ರಿಸ್ಟೀನಳಂತೆಯೇ ಎಂಬುದು ಸ್ವಾಮೀಜಿಯವರ ಇಂಗಿತ) ಯಾವಾಗಲೂ ಹೆದರಿ ಕೊಳ್ಳುತ್ತ ತಾನೊಂದೇ ಇರಲು ಬಯಸುತ್ತದೆ!”

ಸ್ವಾಮೀಜಿ ತಮಗೆ ಎಷ್ಟೇ ಅನಾರೋಗ್ಯವಿದ್ದರೂ ತಮ್ಮ ಮುದ್ದು ಪ್ರಾಣಿಗಳನ್ನು ಹಾಗೂ ಇತರ ವಿಚಾರಗಳನ್ನು ಮರೆತಿರಲಿಲ್ಲ. ಅಲ್ಲದೆ ಸಂದರ್ಭ ಸಿಕ್ಕಾಗಲೆಲ್ಲ ಒಂದು ಚಟಾಕಿ ಹಾರಿಸಲೂ ಸಿದ್ಧ.

ಇಷ್ಟೆಲ್ಲ ಕಾಯಿಲೆಗಳಿಂದ ಬಳಲುತ್ತಿದ್ದರೂ ಸ್ವಾಮೀಜಿಯವರ ಇಚ್ಛಾಶಕ್ತಿ-ಯೋಗಶಕ್ತಿಗಳು ಎಷ್ಟು ಬಲವಾಗಿದ್ದುವೆಂಬುದಕ್ಕೆ ನಿದರ್ಶನದಂತಿದ್ದ ಒಂದು ಘಟನೆ ನಡೆಯಿತು. ಅವರ ಶಿಷ್ಯರಾದ ಸ್ವಾಮಿ ನಿರ್ಭಯಾನಂದರು ಒಮ್ಮೆ ಕಾಯಿಲೆಬಿದ್ದರು. ಜ್ವರ ಒಂದೇ ಸಮನೆ ಜಾಸ್ತಿಯಾಗಿ ನೂರೇಳು ಡಿಗ್ರಿ ಮುಟ್ಟಿತು. ಕಡೆಗೆ ಜ್ವರ ಸನ್ನಿಪಾತಕ್ಕೆ ತಿರುಗಿಕೊಂಡಿತು. ಪರಿಸ್ಥಿತಿ ಅತ್ಯಂತ ನಿರಾಶಾದಾಯಕವಾಗಿ ಕಂಡಿತು. ವ್ಯಾಕುಲಿತರಾದ ಸ್ವಾಮೀಜಿಯವರು ಇದ್ದಕ್ಕಿದ್ದಂತೆ ಏನೋ ಆಲೋಚಿಸುತ್ತ ಎದ್ದು ನಿಂತರು. ನೇರವಾಗಿ ಶ್ರೀರಾಮಕೃಷ್ಣರ ಗರ್ಭಗುಡಿಗೆ ನಡೆದರು. ಶ್ರೀರಾಮಕೃಷ್ಣರ ಅಸ್ಥಿಯನ್ನಿಟ್ಟಿದ್ದ ಪಾತ್ರೆಯನ್ನು ಗಂಗೋದಕದಿಂದ ತೊಳೆದು ಆ ಪವಿತ್ರ ಜಲವನ್ನು ತಂದು ಜ್ವರಪೀಡಿತರಾದ ನಿರ್ಭಯಾನಂದರಿಗೆ ಕುಡಿಯಲು ಕೊಟ್ಟರು. ನಿರ್ಭಯಾ ನಂದರು ಅದನ್ನು ಕುಡಿದರು. ಒಂದು ಕ್ಷಣಕಾಲ, ಜ್ವರ ಮತ್ತೂ ಹೆಚ್ಚಿದಂತೆ ಕಂಡಿತು. ಆದರೆ ಜ್ವರ ಇದ್ದಕ್ಕಿದ್ದಂತೆ ಪೂರ್ತಿ ಇಳಿದುಹೋಯಿತು! ಅತ್ಯಂತ ಆನಂದಿತರಾದ ಸ್ವಾಮೀಜಿ ಸೋದರ ಸಂನ್ಯಾಸಿಗಳತ್ತ ತಿರುಗಿ, “ಅಗೋ ನೋಡಿ–ಶ್ರೀರಾಮಕೃಷ್ಣರ ಶಕ್ತಿಯನ್ನು! ಅವರಿಗೆ ಸಾಧ್ಯವಾಗ ದ್ದೇನಿದೆ?” ಎಂದು ಉದ್ಗರಿಸಿದರು. ಆಶ್ರಮವಾಸಿಗಳ ಆನಂದಾಶ್ಚರ್ಯಗಳಿಗೆ ಪಾರವೇ ಇಲ್ಲ.

ಸ್ವಾಮೀಜಿಯವರು ಈ ಘಟನೆಯ ಮೂಲಕ ಶ್ರೀರಾಮಕೃಷ್ಣರ ಶಕ್ತಿಯನ್ನು ವ್ಯಕ್ತಪಡಿಸಿದಂ ತಾಯಿತು. ಅವರು ಯಾವಾಗಲೂ ಶ್ರೀರಾಮಕೃಷ್ಣರನ್ನು ಒರೆಹಚ್ಚಿ ನೋಡುವವರೇ. ನರೇಂದ್ರ ನಾಗಿದ್ದಾಗಿನಿಂದಲೂ ಈ ಕೆಲಸ ನಡೆದುಬಂದಿತ್ತು. ಹಿಂದೆ ಶ್ರೀರಾಮಕೃಷ್ಣರನ್ನು ಪರೀಕ್ಷೆ ಮಾಡಿ ನೋಡಿದರೆ ಇಂದು ಅವರ ಅಸ್ಥಿಶಕ್ತಿಯನ್ನೇ ಪರೀಕ್ಷಿಸಿ ನೋಡುತ್ತಿದ್ದಾರೆ! ಈ ಪರೀಕ್ಷೆಯಲ್ಲೂ ಅವರು ಉತ್ತೀರ್ಣರಾದಂತಾಯಿತು. ಆದರೆ ಸ್ವಾಮೀಜಿಯವರ ಪರೀಕ್ಷೆ ಮಾತ್ರ ನಿಲ್ಲಲಿಲ್ಲ. ಇನ್ನೊಂದು ಪರೀಕ್ಷೆಯಿಡಲು ಹೊರಟರು:

ಗ್ವಾಲಿಯರಿನ ಮಹಾರಾಜ ಕಲ್ಕತ್ತಕ್ಕೆ ಒಂದು ಸಣ್ಣ ಭೇಟಿಯ ಮೇಲೆ ಬಂದಿದ್ದ. ಇದು ಸ್ವಾಮೀಜಿಯವರಿಗೆ ತಿಳಿದುಬಂದಿತು. ಆದರೆ ಮಹಾರಾಜನಿಗೆ ಬೇಲೂರು ಮಠಕ್ಕೆ ಬರುವ ಆಲೋಚನೆಯೂ ಇರಲಿಲ್ಲ, ಕಾರ್ಯಕ್ರಮವೂ ಇರಲಿಲ್ಲ. ಇದೂ ಅವರಿಗೆ ತಿಳಿದೇ ಇತ್ತು. ಆದರೂ ಆತ ಮಠಕ್ಕೆ ಬಂದು ಹೋಗುವಂತಾಗಬೇಕು ಎಂಬ ಒಂದು ಭಾವನೆ ಅವರಲ್ಲಿ ಮೂಡಿತು. ಸರಿ, ಈಗ ಶ್ರೀರಾಮಕೃಷ್ಣರ ಶಕ್ತಿಯನ್ನು ಪರೀಕ್ಷಿಸಲಾರಂಭ–ಗರ್ಭಗುಡಿಯಲ್ಲಿ ಅಸ್ಥಿಯ ಪೆಟ್ಟಿಗೆಯ ಮುಂದೆ ಕುಳಿತು ಮೌನವಾಗಿ ಪ್ರಾರ್ಥಿಸಿದರು: “ಗುರುದೇವ! ನೀನಿಲ್ಲಿ ನಿಜಕ್ಕೂ ನೆಲಸಿರುವುದಾದರೆ ಗ್ವಾಲಿಯರ್ ಮಹಾರಾಜ ಮೂರು ದಿನಗಳೊಳಗಾಗಿ ಇಲ್ಲಿಗೆ ಬರುವಂತೆ ಮಾಡು!” ಎಂದು. ತಮ್ಮ ಈ ಪ್ರಾರ್ಥನೆಯನ್ನು ಅವರು ಯಾರ ಮುಂದೂ ಹೇಳಲಿಲ್ಲ. ಅಷ್ಟೇ ಅಲ್ಲ, ಅವರೇ ಅದನ್ನು ಪೂರ್ತಿ ಮರೆತುಬಿಟ್ಟರು.

ಮರುದಿನ ಅವರು ಯಾವುದೋ ಕಾರ್ಯನಿಮಿತ್ತವಾಗಿ ಕಲ್ಕತ್ತಕ್ಕೆ ಹೋಗಿದ್ದರು. ಆಗ ಬೇಲೂರು ಮಠದ ಬಳಿಯ ಗ್ರ್ಯಾಂಡ್ ಟ್ರಂಕ್ ರಸ್ತೆಯಲ್ಲಿ ಮಹಾರಾಜ ತನ್ನ ಕಾರಿನಲ್ಲಿ ಕುಳಿತು ಹೋಗುತ್ತಿದ್ದವನು, ಕಾರನ್ನು ನಿಲ್ಲಿಸಿ ತನ್ನ ತಮ್ಮನನ್ನು ಮಠಕ್ಕೆ ಕಳಿಸಿದ–‘ಮಠದಲ್ಲಿ ವಿವೇಕಾ ನಂದರಿದ್ದಾರೆಯೇ ಎಂಬುದನ್ನುನೋಡಿಕೊಂಡು ಬಾ’ ಎಂದು. ಅವನು ಬಂದು ವಿಚಾರಿಸಲಾಗಿ, ಸ್ವಾಮೀಜಿ ಕಲ್ಕತ್ತಕ್ಕೆ ಹೋಗಿರುವುದು ತಿಳಿದು ಬಂದಿತು. ಆಗ ಆತ ಅಲ್ಲಿದ್ದ ಸ್ವಾಮಿಗಳಿಗೆ, “ಗ್ವಾಲಿಯರಿನ ಮಹಾರಾಜರು ಸ್ವಾಮೀಜಿಯವರನ್ನು ಭೇಟಿಯಾಗುವ ಉದ್ದೇಶದಿಂದ ಅತ್ಯಂತ ಉತ್ಸುಕರಾಗಿ ಇಲ್ಲಿಯವರೆಗೆ ಬಂದಿದ್ದಾರೆ. ಆದರೆ ಸ್ವಾಮೀಜಿಯವರು ಕಲ್ಕತ್ತದಿಂದ ಹಿಂದಿರು ಗುವವರೆಗೂ ಕಾಯಲು ಅವರಿಗೆ ಕಾಲಾವಕಾಶವಿಲ್ಲ. ಏಕೆಂದರೆ ಅವರು ನಾಳೆಯೇ ವಾಪಸು ಹೊರಡಲಿದ್ದಾರೆ. ಮುಂದೆ ಎಂದಾದರೊಮ್ಮೆ ಅವರು ಸ್ವಾಮೀಜಿಯವರ ದರ್ಶನ ಮಾಡಲು ಬರುತ್ತಾರೆ ಎಂದು ದಯವಿಟ್ಟು ತಿಳಿಸಿಬಿಡಿ” ಎಂದು ಹೇಳಿ ಹೊರಟುಹೋದ.

ಸ್ವಾಮೀಜಿ ಮಠಕ್ಕೆ ಹಿಂದಿರುಗಿದಾಗ ಸುದ್ದಿ ತಿಳಿಯಿತು. ಆಗ ಇದ್ದಕ್ಕಿದ್ದಂತೆ ಅವರಿಗೆ ತಾವಿಟ್ಟ ‘ಪರೀಕ್ಷೆ’ ನೆನಪಾಯಿತು. ತಕ್ಷಣ ಅವರು ಉದ್ವೇಗಭರಿತರಾಗಿ ಓಡೋಡುತ್ತ ಮೆಟ್ಟಲುಗಳನ್ನೇರಿ ಗರ್ಭಗುಡಿಗೆ ಧಾವಿಸಿದರು; ಅಸ್ಥಿಸಂಪುಟವನ್ನು ಮತ್ತೆ ಮತ್ತೆ ಹಣೆಗೆ ಒತ್ತಿಕೊಂಡರು. ಅಲ್ಲೇ ಧ್ಯಾನಮಗ್ನರಾಗಿ ಕುಳಿತಿದ್ದ ಸ್ವಾಮಿ ಪ್ರೇಮಾನಂದರು ಈ ದೃಶ್ಯವನ್ನು ಕಂಡು ಸ್ತಂಭೀಭೂತ ರಾದರು. ಆಗ ಸ್ವಾಮೀಜಿ ಪ್ರೇಮಾನಂದರಿಗೂ ಇತರ ಸಾಧುಗಳಿಗೂ ತಾವಿಟ್ಟ ಪರೀಕ್ಷೆಯ ವಿಷಯವನ್ನು ಹೇಳಿದರು. ಅದನ್ನು ಕೇಳಿದವರೆಲ್ಲ, ಗರ್ಭಗುಡಿಯಲ್ಲಿ ಶ್ರೀರಾಮಕೃಷ್ಣರ ಸಾನ್ನಿಧ್ಯದ ಪುರಾವೆಯನ್ನು ಕಂಡು ಅತ್ಯಾಶ್ಚರ್ಯಭರಿತರಾದರು.

೧೯ಂ೧ ಡಿಸೆಂಬರಿನಲ್ಲಿ ಸ್ವಾಮೀಜಿಯವರ ಹಲವಾರು ಪಾಶ್ಚಾತ್ಯ ಶಿಷ್ಯೆಯರು ಕಲ್ಕತ್ತಕ್ಕೆ ಬರುವ ಸೂಚನೆ ಕಂಡಿತು. ಕೆಲಕಾಲ ಇಂಗ್ಲೆಂಡಿಗೆ ಹೋಗಿದ್ದ ಶ್ರೀಮತಿ ಸೇವಿಯರ್ ಡಿಸೆಂಬರ್ ಒಂಬತ್ತರಂದು ಹಿಂದಿರುಗಿದರು. ಆ ತಿಂಗಳ ಕೊನೆಯಲ್ಲಿ ಮಿಸ್ ಜೋಸೆಫಿನ್ ಮೆಕ್​ಲಾಡ್ ತನ್ನ ಜಪಾನೀ ಸ್ನೇಹಿತರೊಂದಿಗೆ ಹೊರಟುಬರಲಿದ್ದಳು. ಸೋದರಿ ನಿವೇದಿತೆಯೂ ಯೂರೋಪಿ ನಲ್ಲಿ ತನ್ನ ಕೆಲಸವನ್ನು ಮುಗಿಸಿಕೊಂಡು, ಶ್ರೀಮತಿ ಸಾರಾ ಬುಲ್​ಳೊಂದಿಗೆ ಬರಲು ಅನುವಾಗು ತ್ತಿದ್ದಳು. ಇದೇ ವೇಳೆಗೆ ತಮ್ಮ ಆತ್ಮೀಯ ಶಿಷ್ಯೆ ಕ್ರಿಸ್ಟೀನಳೂ ಬರುವಂತಾದರೆ ಚೆನ್ನಾಗಿರು ತ್ತದೆಂದು ಆಶಿಸಿ, ಸ್ವಾಮೀಜಿ ಅವಳಿಗೊಂದು ಪತ್ರ ಬರೆದರು. ಆದರೆ ಅವಳು ಆ ಪ್ರಯಾಣದ ಖರ್ಚನ್ನು ಹೊರುವಷ್ಟು ಶ್ರೀಮಂತಳಲ್ಲ. ಆದ್ದರಿಂದ ಅವರು ಆಕೆಯ ಖರ್ಚಿಗೆ ಹಣವನ್ನೂ ಕಳಿಸಿಕೊಟ್ಟರು.

ಆ ವರ್ಷದ ಡಿಸೆಂಬರ್ ೨೮ರಿಂದ ೩೧ರವರೆಗೆ ಕಲ್ಕತ್ತದಲ್ಲಿ ಭಾರತದ ರಾಷ್ಟ್ರೀಯ ಕಾಂಗ್ರೆಸ್ಸಿನ ಹದಿನೇಳನೇ ಮಹಾ ಅಧಿವೇಶನ ನಡೆಯಿತು. ಭಾರತದ ಎಲ್ಲೆಡೆಯಿಂದ ಪ್ರತಿನಿಧಿ ಗಳು ಆಗಮಿಸಿದ್ದರು. ಇವರಲ್ಲಿ ಸಮಾಜ ಸುಧಾರಕರು, ಪ್ರಾಧ್ಯಾಪಕರು, ವಕೀಲರು, ಪತ್ರಿಕೋ ದ್ಯಮಿಗಳು ಹಾಗೂ ಇತರ ಧುರೀಣರು ಸೇರಿದ್ದರು. ಆದರೆ, ಮೊದಲನೆಯದಾಗಿ, ಸ್ವಾಮೀಜಿ ಯವರಿಗೆ ರಾಜಕಾರಣದಲ್ಲಿ ಆಸಕ್ತಿಯಿರಲಿಲ್ಲ. ಏಕೆಂದರೆ ಅವರ ಕಾರ್ಯವಿಧಾನ ಸಂಪೂರ್ಣ ವಿಭಿನ್ನವಾಗಿತ್ತು. ಅಲ್ಲದೆ, ಅಂದಿನ ಕಾಂಗ್ರೆಸ್ಸಿನ ರೀತಿನೀತಿಗಳು ಅವರಿಗೆ ಒಪ್ಪಿಗೆಯಾಗಿರಲಿಲ್ಲ. ಎಲ್ಲೆಲ್ಲೂ ಕ್ಷಾಮಡಾಮರಗಳು, ರೋಗರುಜಿನಗಳು, ದಾರಿದ್ರ್ಯ-ಅನಕ್ಷರತೆಗಳು ತಾಂಡವವಾಡು ತ್ತಿರುವಾಗ, ಈ ಕಾಂಗ್ರೆಸಿಗರು ಅದನ್ನೆಲ್ಲ ನೋಡಿಕೊಂಡು ಸುಮ್ಮನಿದ್ದಾರಲ್ಲ ಎಂದು ಅವರಿಗೆ ಅಸಮಾಧಾನವಿತ್ತು. ಅವರು ತಮ್ಮ ಈ ಅಭಿಪ್ರಾಯಗಳನ್ನು ಸಾರ್ವಜನಿಕವಾಗಿ ಎಂದೂ ವ್ಯಕ್ತಪಡಿಸಿರಲಿಲ್ಲ. ಆದರೂ ಕಾಂಗ್ರೆಸ್ಸಿಗರಲ್ಲಿ ಸ್ವಾಮೀಜಿಯವರು ಅಷ್ಟೊಂದು ಜನಪ್ರಿಯರಾ ಗಿರಲಿಲ್ಲ. ಹೀಗಿದ್ದರೂ, ಅವರೊಬ್ಬ ದೊಡ್ಡ ದೇಶಭಕ್ತ ಸಂತನೆಂದು ಅರಿತು ಬಹಳಷ್ಟು ಜನ ಪೂಜ್ಯಭಾವವನ್ನಿಟ್ಟುಕೊಂಡಿದ್ದರು. ಅಧಿವೇಶನಕ್ಕೆ ಬಂದಿದ್ದ ಹಲವಾರು ಪ್ರತಿನಿಧಿಗಳು ಆ ಅವಕಾಶವನ್ನು ಉಪಯೋಗಿಸಿಕೊಂಡು, ಸಮಯವಾದಾಗಲೆಲ್ಲ ಗುಂಪುಗುಂಪಾಗಿ ಬಂದು ಸ್ವಾಮೀಜಿಯವರ ದರ್ಶನ ಮಾಡಿದರು. ಇವರೊಂದಿಗೆ ಸ್ವಾಮೀಜಿ ಅನೇಕ ವೇಳೆ ಶುದ್ಧ ಹಿಂದಿಯಲ್ಲಿ ಸಂಭಾಷಿಸಿ ತಮ್ಮ ಕಾರ್ಯಯೋಜನೆಗಳ ಭವ್ಯ ಕಲ್ಪನೆಯೊಂದನ್ನು ಅವರ ಮುಂದಿಟ್ಟರು. ಇಂತಹ ಕೆಲವು ಮಾತುಕತೆಗಳಲ್ಲಿ ಪಾಲ್ಗೊಂಡು ಸ್ಫೂರ್ತಿಗೊಂಡ ಲಕ್ನೋದ “ಅಡ್ವೊಕೇಟ್​” ಎಂಬ ಪತ್ರಿಕೆಯ ಸಂಪಾದಕರು ಬರೆಯುತ್ತಾರೆ:

“ನಾವು ಸ್ವಾಮಿ ವಿವೇಕಾನಂದರನ್ನು ಕಳೆದ ಸಲ ಕಾಂಗ್ರೆಸ್ ಅಧಿವೇಶನದ ಸಮಯದಲ್ಲಿ ನೋಡಿದಾಗ ಅವರು ಸ್ವಚ್ಛವಾದ ಹಿಂದಿಯಲ್ಲಿ ಭಾರತ ದೇಶದ ಪುನರುತ್ಥಾನದ ಬಗ್ಗೆ ತಮ್ಮ ಯೋಜನೆಗಳನ್ನು ವಿವರಿಸುತ್ತಿದ್ದರು. ಎಂಥವರಿಗಾದರೂ ಗೌರವ ತರಬಲ್ಲ ಭಾಷೆ ಅವರದ್ದು. ಮಾತನಾಡುವಾಗ ಅವರ ಮುಖ ಉತ್ಸಾಹದಿಂದ ಬೆಳಗುತ್ತಿತ್ತು... ” (ಸ್ವಾಮೀಜಿಯವರಿಗೆ ಹಿಂದಿಯಲ್ಲಿ ಮಾತನಾಡುವ ಅವಕಾಶವಿದ್ದದ್ದು ತೀರ ಕಡಿಮೆ ಎಂಬುದನ್ನು ಗಮನಿಸಬೇಕು.)

ಈ ಸಂದರ್ಶನದ ಬಗ್ಗೆ ಒಬ್ಬ ಕಾಂಗ್ರೆಸಿಗರು ಮುಂದೊಮ್ಮೆ ಬರೆದರು:

“ತಮ್ಮನ್ನು ಭೇಟಿಯಾಗಲು ಬಂದವರಿಗೆ ವಿವೇಕಾನಂದರು ಸಾಮಾಜಿಕ, ರಾಜಕೀಯ, ಧಾರ್ಮಿಕ ಹಾಗೂ ಇತರ ಹಲವಾರು ವಿಷಯಗಳ ಬಗ್ಗೆ ಹೊಸ ವಿಚಾರಗಳನ್ನು ತಿಳಿಸಿಕೊಡು ತ್ತಿದ್ದರು. ನಿಜ ಹೇಳಬೇಕೆಂದರೆ, ಒಂದು ರೀತಿಯಲ್ಲಿ ಈ ಸಂದರ್ಶನಗಳೇ ನಿಜವಾದ ಅಧಿವೇಶನ ಗಳಿಗಿಂತಲೂ ಉತ್ತಮವೂ ಹೆಚ್ಚು ಉಪಯುಕ್ತವೂ ಆದ ಅಧಿವೇಶನಗಳಾಗಿದ್ದುವು... ”

ಕಾಂಗ್ರೆಸ್ಸಿನ ಧುರೀಣರೊಂದಿಗೆ ಸ್ವಾಮೀಜಿ ಚರ್ಚಿಸಿದ ಯೋಜನೆಗಳ ಪೈಕಿ ವೇದಾಧ್ಯಯನ ಕೇಂದ್ರವೊಂದರ ಸ್ಥಾಪನೆಯೂ ಒಂದು. ವೇದಗಳ ವಿಚಾರದಲ್ಲಿ ಶಿಕ್ಷಣ-ತರಬೇತಿ ನೀಡಿ ಸಮರ್ಥ ಅಧ್ಯಾಪಕರನ್ನು ಸೃಷ್ಟಿಸುವುದು ಮತ್ತು ಸತಾತನ ಆರ್ಯಸಂಸ್ಕೃತಿಯನ್ನೂ ಸಂಸ್ಕೃತಾ ಧ್ಯಯನ ಪರಂಪರೆಯನ್ನೂ ಉಳಿಸಿ ಬೆಳೆಸುವುದು–ಇವೇ ಈ ಉದ್ದೇಶಿತ ಕೇಂದ್ರದ ಧ್ಯೇಯಗಳಾ ಗಿದ್ದುವು. ಈ ಯೋಜನೆಗೆ ಕಾಂಗ್ರೆಸ್ ಧುರೀಣರೆಲ್ಲರೂ ಸಂಪೂರ್ಣ ಬೆಂಬಲವನ್ನು ವ್ಯಕ್ತಪಡಿಸಿ ದರು.

ಈ ಸಂದರ್ಭದಲ್ಲಿ ಬೇಲೂರು ಮಠವನ್ನು ಸಂದರ್ಶಿಸಿದ ಧುರೀಣರಲ್ಲಿ ಬಾಲಗಂಗಾಧರ ತಿಲಕರೂ ಮಹಾತ್ಮಾ ಗಾಂಧಿಯವರೂ ಇದ್ದರು. ಹಿಂದೆ ಸ್ವಾಮೀಜಿ ಪರಿವ್ರಾಜಕರಾಗಿದ್ದಾಗಲೇ ತಿಲಕರನ್ನು ಭೇಟಿಯಾಗಿದ್ದರು. ಅಲ್ಲದೆ ಅವರ ಅತಿಥಿಗಳಾಗಿ ಒಂದು ವಾರ ಇದ್ದರು. (ನೋಡಿ: “ವಿಶ್ವವಿಜೇತ ವಿವೇಕಾನಂದ” ಪುಟ ೭೪-೭೬) ಮತ್ತೆ ಸ್ವಾಮೀಜಿಯವರನ್ನು ಭೇಟಿಯಾಗಲು ತುಂಬ ಉತ್ಸುಕರಾಗಿದ್ದ ತಿಲಕರು, ಅವರು ಭಾರತಕ್ಕೆ ಮರಳುತ್ತಿದ್ದಂತೆಯೇ ಒಂದು ಪತ್ರ ಬರೆದು ತಮ್ಮಲ್ಲಿಗೆ ಆಹ್ವಾನಿಸಿದ್ದರು. ಆದರೆ ಆಗ ಸ್ವಾಮೀಜಿಯವರಿಗೆ ಆಹ್ವಾನವನ್ನು ಅಂಗೀಕರಿಸಲು ಸಾಧ್ಯವಾಗಿರಲಿಲ್ಲ. ಇದೀಗ ಅವರನ್ನು ಭೇಟಿಯಾಗುವ ಅವಕಾಶವನ್ನು ತಿಲಕರು ಕಳೆದುಕೊಳ್ಳ ಲಿಲ್ಲ. ಇವರು ಅಧಿವೇಶನದ ಸಮಯದಲ್ಲಿ ಅನೇಕ ಬಾರಿ ಬಂದು ಸ್ವಾಮೀಜಿಯವರನ್ನು ಭೇಟಿ ಯಾಗಿ ದೀರ್ಘವಾಗಿ ಚರ್ಚಿಸಿದರು. ಒಂದು ಸಲ ಅವರಿಬ್ಬರೇ ಮಠದ ಹುಲ್ಲು ಹಾಸಿನ ಮೇಲೆ ಅತ್ತಿಂದಿತ್ತ ಅಡ್ಡಾಡುತ್ತ ಸುಮಾರು ಒಂದೂವರೆ ಗಂಟೆಯ ಕಾಲ ವಿವಿಧ ವಿಷಯಗಳ ಬಗ್ಗೆ ಉತ್ಸಾಹದಿಂದ ಸಂಭಾಷಿಸುತ್ತಿದ್ದರೆಂದು ತಿಳಿದುಬರುತ್ತದೆ. ಮಹಾತ್ಮಾ ಗಾಂಧಿಯವರು–ಆಗ ಅವರಿನ್ನೂ ಎಮ್. ಕೆ. ಗಾಂಧಿ–ಕಲ್ಕತ್ತದಲ್ಲಿ ನಡೆದ ಈ ಮಹಾ ಅಧಿವೇಶನದಲ್ಲಿ ಭಾಗವಹಿಸಲು ದಕ್ಷಿಣ ಆಫ್ರಿಕದಿಂದ ಬಂದಿದ್ದರು. ಇವರು ದಕ್ಷಿಣ ಆಫ್ರಿಕದ ಸ್ವಾತಂತ್ರ್ಯ ಚಳವಳಿಗೆ ಕಾಂಗ್ರೆಸ್ಸಿನ ಬೆಂಬಲವನ್ನು ಕೋರುವುದಕ್ಕಾಗಿ ಬಂದರು. ಸ್ವಾಮೀಜಿಯವರನ್ನು ನೋಡಬೇಕೆಂದು ಅವರಿಗೆ ತೀವ್ರವಾದ ಇಚ್ಛೆಯಿತ್ತು. ಆದ್ದರಿಂದ ಅವರು ಅಧಿವೇಶನ ಮುಗಿದ ಮೇಲೆ ಒಂದು ದಿನ, ಬಹಳ ದೂರ ಕಾಲ್ನಡಿಗೆಯಲ್ಲಿ ಪ್ರಯಾಣ ಮಾಡಿ ಬೇಲೂರು ಮಠಕ್ಕೆ ಬಂದರು, ಆದರೆ ಆಗ ಸ್ವಾಮೀಜಿ ಮಠದಲ್ಲಿರಲಿಲ್ಲ. ಅನಾರೋಗ್ಯ ಪೀಡಿತರಾಗಿ ಕಲ್ಕತ್ತದಲ್ಲೇ ಉಳಿದುಕೊಂಡಿದ್ದರು. ಮತ್ತು ಯಾರಿಗೂ ಅವರನ್ನು ನೋಡಲು ಅವಕಾಶವಿರಲಿಲ್ಲ. ಇದನ್ನು ಕೇಳಿದಾಗ ಗಾಂಧೀಜಿಯವರಿಗಾದ ನಿರಾಶೆ ಅಷ್ಟಿಷ್ಟಲ್ಲ. ಮತ್ತು ಮುಂದೆ ಅವರಿಬ್ಬರೂ ಪರಸ್ಪರರನ್ನು ಭೇಟಿಯಾಗುವ ಸಂದರ್ಭ ಒದಗಿಬರಲೇ ಇಲ್ಲ.

ಇದಾದ ಕೆಲಕಾಲದಲ್ಲೇ ಗಾಂಧೀಜಿಯವರು ಸೋದರಿ ನಿವೇದಿತೆಯನ್ನು ಭೇಟಿಮಾಡಿದರು. ಅವಳ ಅಗಾಧ ಭಾರತಪ್ರೇಮವನ್ನು ಅವರು ಮೆಚ್ಚಿಕೊಂಡರಾದರೂ, ತಮ್ಮಿಬ್ಬರ ನಡುವೆ ಅವರಿಗೆ ಯಾವುದೇ ಸಮಾನ ಅಂಶ ಕಂಡುಬರಲಿಲ್ಲ.

ಸ್ವಾಮಿ ವಿವೇಕಾನಂದರ ಬಗ್ಗೆ ಮಹಾತ್ಮಾ ಗಾಂಧೀಜಿಯವರು ಅಪಾರ ಗೌರವ ಹೊಂದಿ ದ್ದರು. ಅಲ್ಲದೆ ಅವರಿಂದ ಆಳವಾಗಿ ಪ್ರಭಾವಿತರಾಗಿದ್ದರು. ಮುಂದೆ ೧೯೨೧ರ ಫೆಬ್ರುವರಿ ೬ರಂದು ವಿವೇಕಾನಂದರ ಜನ್ಮದಿನೋತ್ಸವದ ಸಂದರ್ಭದಲ್ಲಿ ಅವರು ಬೇಲೂರು ಮಠಕ್ಕೆ ಭೇಟಿ ನೀಡಿದರು. ಸ್ವಾಮೀಜಿಯವರ ಬಗ್ಗೆ ಏನಾದರೂ ಹೇಳುವಂತೆ ಅವರನ್ನು ಕೇಳಿಕೊಳ್ಳಲಾಯಿತು. ಆಗ ಅವರು ಗಂಭೀರವಾಗಿ ಎದ್ದು ನಿಂತು, ಅಲ್ಲಿ ನೆರೆದಿದ್ದ ಜನರನ್ನುದ್ದೇಶಿಸಿ ಕೆಲವು ಮಾತುಗಳನ್ನಾಡಿದರು. ಅದರ ಸಾರಾಂಶ ಹೀಗಿದೆ: “ನಾನು ಇಂದು ಸ್ವಾಮಿ ವಿವೇಕಾನಂದರಿಗೆ ನನ್ನ ಗೌರವವನ್ನೂ ಶ್ರದ್ಧಾಂಜಲಿಯನ್ನೂ ಅರ್ಪಿಸಲು ಬಂದಿದ್ದೇನೆ. ನಾನು ಅವರ ಕೃತಿಗಳನ್ನು ಆಮೂಲಾಗ್ರವಾಗಿ ಅಧ್ಯಯನ ಮಾಡಿದ್ದೇನೆ. ಅವುಗಳನ್ನು ಓದಿದ ಮೇಲೆ ದೇಶದ ಮೇಲಿನ ನನ್ನ ಪ್ರೀತಿ ಸಾವಿರ ಪಟ್ಟು ಹೆಚ್ಚಾಯಿತು.”

