
\chapter{ಶಿಷ್ಯರಿಗೆ ಸ್ಫೂರ್ತಿಧಾರೆ}

\noindent

ಸ್ವಾಮೀಜಿ ಕಲ್ಕತ್ತದ ಆಲಂಬಜಾರ್ ಮಠದಲ್ಲಿದ್ದ ಈ ಸಂದರ್ಭದಲ್ಲಿ ದಿನದ ಪ್ರತಿ ಮುಹೂರ್ತ ದಲ್ಲೂ ಅವರು ಒಂದಲ್ಲ ಒಂದು ಕಾರ್ಯದಲ್ಲಿ ನಿಮಗ್ನರಾಗಿರುವುದನ್ನು ಕಾಣಬಹುದಾಗಿತ್ತು. ಮಠದ ದಿನಚರಿ ದಾಖಲೆಗಳಲ್ಲಿ ಅವರು ಕೈಗೊಂಡ ವೈವಿಧ್ಯಪೂರ್ಣ ಕಾರ್ಯಕಲಾಪಗಳ ವಿವರಗಳು ದಾಖಲಾಗಿವೆ. ಅವರು ಒಂದೋ ಭಕ್ತರಿಂದ ಆಹ್ವಾನಿತರಾಗಿ ಅವರ ಮನೆಗಳಿಗೆ ಹೋಗಿರುತ್ತಿದ್ದರು; ಅಥವಾ ಮಠದಲ್ಲೋ, ಬಲರಾಮ ಬಾಬುವಿನ ಮನೆಯಲ್ಲೋ ಸಂದರ್ಶಕ ರೊಂದಿಗೆ ಸಂಭಾಷಣೆಯಲ್ಲಿ ತೊಡಗಿರುತ್ತಿದ್ದರು; ಅಥವಾ ತಮಗೆ ಬಂದ ಪತ್ರಗಳಿಗೆ ಉತ್ತರ ಬರೆಯುತ್ತಿದ್ದರು; ಇಲ್ಲವೆ ಯಾವುದಾದರೂ ಲೇಖನ ಅಥವಾ ಗ್ರಂಥವನ್ನು ಬರೆಯುತ್ತ ಕುಳಿತಿ ರುತ್ತಿದ್ದರು. ಆದರೆ ಆ ಅವಧಿಯಲ್ಲಿ ಅವರ ಪ್ರಧಾನ ಕಾರ್ಯ ಯಾವುದಾಗಿತ್ತೆಂದರೆ ಯುವ ಸಂನ್ಯಾಸೀ ಬ್ರಹ್ಮಚಾರಿಗಳನ್ನು ತರಬೇತಿಗೊಳಿಸುವುದು, ಈ ಯುವ ಸಾಧಕರಿಂದ ಅಧ್ಯಯನ ಧ್ಯಾನ, ಭಜನೆಗಳನ್ನು ಮಾಡಿಸುವುದು ಅಥವಾ ಅವರಿಗೆ ಯೋಗದ ನಾನಾ ಪ್ರಕ್ರಿಯೆಗಳನ್ನು ತಿಳಿಸಿಕೊಡುವುದು–ಇವುಗಳಿಗಾಗಿ ಸ್ವಾಮೀಜಿ ತಮ್ಮ ಹೆಚ್ಚಿನ ಸಮಯವನ್ನು ವಿನಿಯೋಗಿಸು ತ್ತಿದ್ದರು. ಪ್ರತಿದಿನವೂ ಅವರು ಈ ಯುವ ಸಾಧಕರಿಗಾಗಿ ನಿಯತವಾಗಿ ಶಾಸ್ತ್ರಗಳ ಮೇಲೆ ತರಗತಿಗಳನ್ನು ತೆಗೆದುಕೊಳ್ಳುತ್ತಿದ್ದರು; ಉಪನಿಷತ್ತು, ಗೀತೆ, ವಿಜ್ಞಾನ, ದೇಶವಿದೇಶಗಳ ಇತಿಹಾಸ ಇವುಗಳ ಮೇಲೆ ಮಾತನಾಡುತ್ತಿದ್ದರು, ಅಥವಾ ಅವರ ಪ್ರಶ್ನೆಗಳಿಗೆ ಉತ್ತರಿಸುತ್ತಿದ್ದರು.

೧೮೯೮ರ ಜನವರಿ ೨೮ರಂದು, ಎಂದರೆ, ಸ್ವಾಮೀಜಿ ಕಲ್ಕತ್ತಕ್ಕೆ ಹಿಂದಿರುಗಿದ ಸುಮಾರು ಒಂದು ವಾರದಲ್ಲಿ ಮಿಸ್ ಮಾರ್ಗರೆಟ್ ನೋಬೆಲ್ ಇಂಗ್ಲೆಂಡಿನಿಂದ ಕಲ್ಕತ್ತಕ್ಕೆ ಬಂದು ತಲುಪಿದಳು. ಮಾರ್ಗರೆಟ್ಟಳು ವಿವೇಕಾನಂದರನ್ನು ತನ್ನ ಗುರುವಾಗಿ ಸ್ವೀಕರಿಸಿದ ಕಥೆಯನ್ನು ಹಿಂದಿನ ಸಂಪುಟದಲ್ಲಿ ನೋಡಿದ್ದೇವೆ. ಅಲ್ಲದೆ ಅವಳು ಭಾರತಕ್ಕೆ ಬರಬೇಕೆಂಬ ತನ್ನ ಇಚ್ಛೆಯನ್ನು ಸ್ವಾಮೀಜಿಯವರಿಗೆ ತಿಳಿಸಿ, ಕಡೆಗೆ ಅವರನ್ನು ಒಪ್ಪಿಸಿದ ವಿಷಯವನ್ನೂ ನೋಡಿದ್ದೇವೆ. ಹೀಗೆ ಸ್ವಾಮೀಜಿಯವರ ಒಪ್ಪಿಗೆಯನ್ನು ಪಡೆದ ಮಾರ್ಗರೆಟ್, ಅವರ ಮೇಲೆ ಸಂಪೂರ್ಣ ಭರವಸೆ ಯಿಟ್ಟು, ದೃಢನಿಶ್ಚಯಮಾಡಿ ಹೊರಟುಬಿಟ್ಟಳು. ಆದರೆ ಇಂಗ್ಲೆಂಡನ್ನು ಬಿಟ್ಟು ಹೊರಡುತ್ತಿ ದ್ದಂತೆ ಅವಳ ಮನಸ್ಸನ್ನು ಏನೋ ಒಂದು ಬಗೆಯ ದುಗುಡ, ಕಳವಳ ಆವರಿಸಿಕೊಂಡಿತು. ಪ್ರಯಾಣದ ಅವಧಿಯ ಉದ್ದಕ್ಕೂ ಅದೇ ಕಾತರ. ಒಂದೆಡೆ ತನ್ನ ಬಯಕೆ ಈಡೇರುತ್ತಿದೆಯೆಂಬ ಹಿಗ್ಗು; ಇನ್ನೊಂದೆಡೆ ಮನೆಮಂದಿಯನ್ನೆಲ್ಲ ತ್ಯಜಿಸಿ ಹೋಗುತ್ತಿರುವೆನೆಂಬ ದುಗುಡ. ತನ್ನ ಕಾರ್ಯದಲ್ಲಿ ತಾನು ಎಷ್ಟರಮಟ್ಟಿಗೆ ಯಶಸ್ವಿಯಾದೇನು ಎಂಬ ಚಿಂತೆ; ಆದರೆ ಸ್ವಾಮೀಜಿ ಯವರನ್ನು ನೆನಪಿಸಿಕೊಂಡೊಡನೆಯೇ ಒಂದು ಅಪೂರ್ವ ಸಮಾಧಾನ-ಹರ್ಷ! ತಾನೀಗ ಹೋಗುತ್ತಿರುವುದು ತನ್ನ ಗುರುದೇವನ ನಾಡಿಗೆ, ತಾನು ಸೇರಲಿರುವುದು ತನ್ನ ಗುರುವಿನ ಸನ್ನಿಧಿ ಯನ್ನು ಎಂಬ ಆಲೋಚನೆ ಮೂಡಿದೊಡನೆಯೇ ಮನದಲ್ಲೊಂದು ಪುಳಕ! ಹಡಗು ಮದ್ರಾಸಿನ ಮೂಲಕ ಹಾದು ಕಲ್ಕತ್ತವನ್ನು ತಲುಪುತ್ತಿದ್ದಂತೆಯೇ ಅವಳ ಭಾವತರಂಗ ಹುಚ್ಚೆದ್ದು ಹರಿ ಯಿತು. ಹಡಗಿನಿಂದಿಳಿದೊಡನೆ ಮಾರ್ಗರೆಟ್ಟಳ ಕಾತರದ ಕಂಗಳು ಅಲ್ಲಿ ನೆರೆದಿದ್ದವರಲ್ಲಿ ತನಗೆ ಬೇಕಾದ ಒಬ್ಬರನ್ನು ಹುಡುಕಾಡಿತು. ಹೌದು! ಅವರು ಅಲ್ಲಿದ್ದರು! ಮಾರ್ಗರೆಟ್ಟಳನ್ನು ಸ್ವಾಗತಿಸಲು ಸ್ವಾಮೀಜಿ ನಗುಮುಖದಿಂದ ಕಾದುನಿಂತಿದ್ದರು! ತಮ್ಮ ಪಾಶ್ಚಾತ್ಯ ಶಿಷ್ಯೆಯನ್ನು ಅವರು ಅತ್ಯಂತ ಆದರದಿಂದ ಬರಮಾಡಿಕೊಂಡರು. ಅವರ ಮುಖದರ್ಶನವಾಗುತ್ತಿದ್ದಂತೆ ಮಾರ್ಗರೆಟ್ಟಳ ಚಿಂತೆ ದುಗುಡಗಳೆಲ್ಲ ಓಡಿಹೋದುವು.

ಮಾರ್ಗರೆಟ್ಟಳಿಗೆ ಚೌರಂಘೀ ರಸ್ತೆಯ ಹೋಟೆಲೊಂದರಲ್ಲಿ ತಾತ್ಕಾಲಿಕವಾಗಿ ವಸತಿ ಕಲ್ಪಿಸಲಾ ಯಿತು. ಕೆಲದಿನಗಳಲ್ಲೇ ಆಕೆಯ ದೇಶದವಳೇ ಆದ ಮಿಸ್ ಹೆನ್ರಿಟಾ ಮುಲ್ಲರ್ ಆಲ್ಮೋರದಿಂದ ಹಿಂದಿರುಗಿದಳು. ಬಳಿಕ ಅವರಿಬ್ಬರೂ ಬೇರೊಂದು ಮನೆಯಲ್ಲಿ ಇರಲಾರಂಭಿಸಿದರು.

ಮಾರ್ಗರೆಟ್ಟಳನ್ನು ತಮ್ಮ ಮುಂದಿನ ಕಾರ್ಯೋದ್ದೇಶಗಳಿಗೆ ಅನುಗುಣವಾಗಿ ರೂಪಿಸುವ ಕಾರ್ಯದಲ್ಲಿ ಈಗ ಸ್ವಾಮೀಜಿ ನಿರತರಾದರು. ಮೊಟ್ಟಮೊದಲನೆಯದಾಗಿ, ಹಿಂದೂ ಧರ್ಮ- ಸಂಸ್ಕೃತಿ-ಪರಂಪರೆಗಳ ಬಗ್ಗೆ ಅವಳಲ್ಲಿ ಸ್ಪಷ್ಟವಾದ ಅರಿವನ್ನುಂಟುಮಾಡಿಸುವ ಉದ್ದೇಶದಿಂದ ಅವರು ಬಹುವಾಗಿ ಶ್ರಮಿಸಿದರು, ಅವಳಿಗೆ ನಿಖರವಾದ ಶಿಕ್ಷಣ ನೀಡಿದರು. ಯೋಗ್ಯರಾದ ಬುದ್ಧಿ ವಂತ ವಿದ್ಯಾರ್ಥಿಗಳಿಗೆ ದೊರಕಿದ ಯಾವುದೇ ಶಿಕ್ಷಣ-ತರಬೇತಿಯೂ ಅವರಲ್ಲೇ ಮುಳುಗಿ ವ್ಯರ್ಥ ವಾಗುವುದಿಲ್ಲ, ಬದಲಾಗಿ ಅದು ಹತ್ತಾರು-ನೂರಾರು ಜನರಿಗೆ ತಲುಪುತ್ತದೆ. ಅಂತೆಯೇ ತುಂಬ ಸೂಕ್ಷ್ಮಮತಿಯೂ ಬುದ್ಧಿವಂತಳೂ ಆದ ಮಾರ್ಗರೆಟ್ ಸ್ವಾಮೀಜಿಯವರ ಅದ್ಭುತ ಶಿಕ್ಷಣದಿಂದ ಏನೇನನ್ನು ಕಲಿತು ಅರಗಿಸಿಕೊಂಡಳೋ, ಯಾವ ಯಾವ ವಿಶೇಷ ಅಂಶಗಳನ್ನೆಲ್ಲ ಗಮನಿಸಿದಳೋ ಅವುಗಳೆಲ್ಲ ಕಾಲಾಂತರದಲ್ಲಿ ಅವಳ ಶಕ್ತಿಯುತ ಲೇಖನಿಯ ಮೂಲಕ ಅಸಂಖ್ಯಾತ ಪ್ರಾಚ್ಯ- ಪಾಶ್ಚಾತ್ಯ ಓದುಗರಿಗೆ ತಲುಪುವಂತಾಯಿತು. ಹಿಗೆ ಸ್ವಾಮೀಜಿಯವರು ಅವಳೊಬ್ಬಳಿಗೆ ನೀಡಿದ ಶಿಕ್ಷಣವು ಬೃಹತ್ ಸಭೆಯೊಂದನ್ನು ಉದ್ದೇಶಿಸಿ ಮಾತನಾಡಿದಷ್ಟು ಪರಿಣಾಮವನ್ನು ಬೀರಿತು. ಅಲ್ಲದೆ ಸ್ವಾಮೀಜಿ ಅವಳ ಮೂಲಕ ಪ್ರಸಾರ ಮಾಡಿದ ಭಾವನೆಗಳು-ವಿಚಾರಗಳು ಮುಂದೆ ಜನರಲ್ಲಿ ರಾಷ್ಟ್ರೀಯ ಪ್ರಜ್ಞೆ ಬೆಳೆಯಲು ಪ್ರಚಂಡ ಪ್ರಚೋದನೆ ನೀಡಿದುವು.

ಈ ಸಮಯದಲ್ಲಿ ಸ್ವಾಮೀಜಿ ಕಲ್ಕತ್ತದಲ್ಲಿ ಹಲವಾರು ಕಾರ್ಯಕ್ರಮಗಳಲ್ಲಿ ಪಾಲ್ಗೊಂಡರು. ಇವುಗಳಲ್ಲಿ ಒಂದು ಗಮನೀಯವಾದ ಕಾರ್ಯಕ್ರಮವೆಂದರೆ ನವಗೋಪಾಲ ಘೋಷನ ಮನೆ ಯಲ್ಲಿ ಶ್ರೀರಾಮಕೃಷ್ಣರ ಪ್ರತಿಷ್ಠಾಪನೆಯ ಸಮಾರಂಭ. ಶ್ರೀರಾಮಕೃಷ್ಣರ ಗೃಹೀಭಕ್ತನಾದ ನವಗೋಪಾಲ, ಹೌರಾದ ರಾಮಕೃಷ್ಣಪುರದಲ್ಲಿ ಹೊಸ ಮನೆಯೊಂದನ್ನು ಕಟ್ಟಿಸಿಕೊಂಡಿದ್ದ. ಆ ಮನೆಯ ಪೂಜಾಗೃಹದಲ್ಲಿ ಶ್ರೀರಾಮಕೃಷ್ಣರ ಪ್ರತಿಮೆಯನ್ನು ಪ್ರತಿಷ್ಠಾಪಿಸುವ ಅಭಿಲಾಷೆ ಅವನದ್ದು. ಈ ಕಾರ್ಯಕ್ರಮದಲ್ಲಿ ಭಾಗವಹಿಸಬೇಕೆಂದು ಅವನು ಸ್ವಾಮೀಜಿ ಹಾಗೂ ಅವರ ಸಂನ್ಯಾಸೀಬಂಧುಗಳಲ್ಲಿ ವಿನಂತಿಸಿಕೊಂಡ. ಅದಕ್ಕೆ ಸ್ವಾಮೀಜಿ ಸಮ್ಮತಿಸಿದಾಗ ನವಗೋಪಾಲ ನಿಗಾದ ಆನಂದ ಅಪಾರ. ಅಂದು ಫೆಬ್ರುವರಿ ೧೬, ಪೂರ್ಣಿಮೆ. ಎಲ್ಲ ಸಾಧುಗಳು ಹಾಗೂ ಶಿಷ್ಯರೊಂದಿಗೆ ಸ್ವಾಮೀಜಿ ದೋಣಿಯಲ್ಲಿ ಗಂಗೆಯನ್ನು ದಾಟಿ ರಾಮಕೃಷ್ಣಪುರದ ಘಾಟಿನಲ್ಲಿ ಬಂದಿಳಿದರು. ಅಲ್ಲಿಂದಲೇ ಸಂಕೀರ್ತನೆಯ ತಂಡವೊಂದು ಹೊರಟಿತು. ಅದು ರಸ್ತೆಯಲ್ಲಿ ಮುಂದುವರಿದಂತೆಲ್ಲ ಹಲವಾರು ಭಕ್ತರು ಅದರಲ್ಲಿ ಸೇರಿಕೊಂಡರು. ಕೀರ್ತನೋತ್ಸಾಹ ಅಪರಿ ಮಿತವಾಗಿತ್ತು. ಸ್ವಾಮೀಜಿಯವರು ಅತ್ಯಂತ ಉತ್ಸಾಹಭರಿತರಾಗಿದ್ದರು! ಅವರ ಕೊರಳಲ್ಲಿ ಖೋಲ್ (ಮೃದಂಗದಂತಹ ವಾದ್ಯ) ಜೋತಾಡುತ್ತಿತ್ತು; ಅದನ್ನು ಬಾಜಿಸುತ್ತ ಅವರು ಶ್ರೀ ರಾಮಕೃಷ್ಣರ ಬಾಲ್ಯವನ್ನು ಬಣ್ಣಿಸುವ ದುಃಖಿನೀ ಬ್ರಾಹ್ಮಣಿಕೋಲೆ\footnote{*ಈ ಹಾಡಿನ ಕರ್ತೃ ಗಿರೀಶ್​ಚಂದ್ರ ಘೋಷ್. ಶ್ರೀರಾಮಕೃಷ್ಣರ ಬಾಲ್ಯದ ಕುರಿತಾದ ಈ ಹಾಡಿನ ಪಲ್ಲವಿಯ ಅರ್ಥ ಹೀಗಿದೆ: ‘ಬಡ ಬ್ರಾಹ್ಮಣ ಮಹಿಳೆಯ ಮಡಿಲಲ್ಲಿ ಆಟವಾಡುತ್ತಿರುವ ತೇಜೋಮಯನೆ, ಯಾರು ನೀನು?’ ಈ ಹಾಡು ‘ವಿಪ್ರವನಿತೆ ಅಂಕದಲ್ಲಿ... ’ ಎಂದು ಕನ್ನಡಕ್ಕೆ ಅನುವಾದವಾಗಿದೆ.} ಎಂಬ ಹಾಡನ್ನು ಸುಶ್ರಾವ್ಯ ವಾಗಿ ಹಾಡುತ್ತ ಬರುತ್ತಿದ್ದರು; ಇತರರು ಅದನ್ನು ಅನುಸರಿಸುತ್ತಿದ್ದರು. ಸಂಕೀರ್ತನೆಯ ತಂಡ ದೊಂದಿಗೆ ಸ್ವಾಮೀಜಿ ರಸ್ತೆಯಲ್ಲಿ ನಡೆದುಬರುತ್ತಿದ್ದಂತೆ, ಆ ದೃಶ್ಯವನ್ನು ನೋಡಲು ರಸ್ತೆಯ ಇಕ್ಕೆಲಗಳಲ್ಲಿ ಸಾವಿರಾರು ಜನ ಸೇರಿಬಿಟ್ಟರು. ಇತರೆಲ್ಲ ಸಾಧುಗಳಂತೆಯೇ ಸ್ವಾಮೀಜಿ ಸಾಧಾರಣ ವಾದ ಕಾವಿ ವಸ್ತ್ರ ಧರಿಸಿ, ಖೋಲನ್ನು ಬಾಜಿಸುತ್ತ, ಭಾವೋನ್ಮತ್ತರಾಗಿ ಹಾಡುತ್ತ ಬರಿಗಾಲಿನಲ್ಲೇ ದಾರಿಯಲ್ಲಿ ಬರುತ್ತಿರುವಾಗ, ಅವರ ಸರಳವಾದ ಹಾಗೂ ರಾಜಗಾಂಭೀರ್ಯದಿಂದ ಕೂಡಿದ ವ್ಯಕ್ತಿತ್ವವನ್ನು ಕಂಡ ಜನ ಹರ್ಷೋದ್ಗಾರ ಮಾಡಿ ಜೈಕಾರ ಹಾಕಿದರು. ಪಾಶ್ಚಾತ್ಯದೇಶಗಳಲ್ಲಿ ಕೀರ್ತಿಶಿಖರವನ್ನೇರಿದ ವಿಶ್ವವಿಜೇತ ವಿವೇಕಾನಂದರು ಇವರೇ...? ಜನರಿಗೆ ನಂಬುವುದಕ್ಕೇ ಸಾಧ್ಯವಾಗುತ್ತಿಲ್ಲ!

ಸ್ವಾಮೀಜಿಯವರು ಸಂಕೀರ್ತನೆಯ ತಂಡದ ಸಮೇತರಾಗಿ ನವಗೋಪಾಲ ಘೋಷನ ಮನೆಗೆ ತಲುಪಿದಾಗ ಅಲ್ಲಿ ಅವರನ್ನು ಶಂಖ-ಜಾಗಟೆಗಳ ನಿನಾದದ ನಡುವೆ ಅತ್ಯಂತ ಪೂಜ್ಯಭಾವದಿಂದ ಆದರಿಸಿ ಸತ್ಕರಿಸಲಾಯಿತು. ಸ್ವಲ್ಪಹೊತ್ತಿನ ಬಳಿಕ ಅವರನ್ನು ಪೂಜಾಗೃಹಕ್ಕೆ ಕರೆದೊಯ್ಯ ಲಾಯಿತು. ಚಂದ್ರಕಾಂತ ಶಿಲೆ ಹಾಸಿದ ಆ ಕೋಣೆಯನ್ನು ಅಂದವಾಗಿ ಅಲಂಕರಿಸಲಾಗಿತ್ತು. ಪೀಠದ ಮೇಲೆ ಶ್ರೀರಾಮಕೃಷ್ಣರ ಸುಂದರ ವಿಗ್ರಹ ವಿರಾಜಿಸುತ್ತಿತ್ತು. ಆ ಪೂಜಾಗೃಹವನ್ನೂ ಪೂಜಾ ಸಿದ್ಧತೆಗಳನ್ನೂ ನೋಡಿ ಸ್ವಾಮೀಜಿಯವರಿಗೆ ಅತ್ಯಾನಂದವಾಯಿತು. ಮನೆಯ ಯಜ ಮಾನಿಯ ಕಾರ್ಯಕೌಶಲವನ್ನು ಹಾಗೂ ಕುಟುಂಬದವರೆಲ್ಲರ ಭಕ್ತಿ-ಶ್ರದ್ಧೆಗಳನ್ನು ಅವರು ಕೊಂಡಾಡಿದರು. ಆಗ ಮನೆಯ ಯಜಮಾನಿ ವಿನಯದಿಂದ ನುಡಿದಳು, “ಸ್ವಾಮೀಜಿ, ನಾವು ಬಡವರು; ಭಗವಂತನನ್ನು ಯೋಗ್ಯರೀತಿಯಲ್ಲಿ ಪೂಜಿಸುವ ಸಾಮರ್ಥ್ಯ ನಮಗಿಲ್ಲ. ದಯವಿಟ್ಟು ತಾವು ನಮ್ಮ ಮೇಲೆ ಕೃಪೆಯಿಟ್ಟು ಆಶೀರ್ವದಿಸಬೇಕು.” ಅದಕ್ಕೆ ಸ್ವಾಮೀಜಿ ಹೇಳಿದರು, “ಅಮ್ಮಾ, ಶ್ರೀರಾಮಕೃಷ್ಣರ ಹದಿನಾಲ್ಕು ತಲೆಮಾರಿನವರು ಇಂತಹ ಚಂದ್ರಕಾಂತ ಶಿಲೆ ಹಾಸಿರುವ ಕೋಣೆಯಲ್ಲಿ ವಾಸ ಮಾಡಲಿಲ್ಲ! ಸಣ್ಣ ಹಳ್ಳಿಯ ಬಡ ಜೋಪಡಿಯಲ್ಲಿ ಜನಿಸಿದ ಠಾಕೂರರು ಯಾವಾಗಲೂ ಸಾದಾ ಜೀವನ ನಡೆಸಿದವರು. ಅವರು ಇಂಥಾ ಭಕ್ತಿಪೂರ್ಣ ಪೂಜೆ ಪುರಸ್ಕಾರ ಗಳಿರುವ ಈ ಸ್ಥಳದಲ್ಲೇ ನೆಲಸಿರದಿದ್ದರೆ ಬೇರೆ ಇನ್ನೆಲ್ಲಿ ತಾನೆ ಇರಬಲ್ಲರೋ ನಾ ಕಾಣೆ!” ಈ ಮಾತುಗಳನ್ನು ಕೇಳಿ ಅಲ್ಲಿದ್ದವರೆಲ್ಲ ಗಟ್ಟಿಯಾಗಿ ನಗುತ್ತಿದ್ದಂತೆ ಸ್ವಾಮೀಜಿ ಮೇಲೆದ್ದರು. ಮೈಗೆ ವಿಭೂತಿ ಹಚ್ಚಿಕೊಂಡು ಪೂಜಾಸನದ ಮೇಲೆ ಕುಳಿತರು. ಭಕ್ತಿಭಾವಲೀನರಾಗಿ ಶ್ರೀರಾಮ ಕೃಷ್ಣರನ್ನು ವಿಗ್ರಹದಲ್ಲಿ ಆವಾಹನೆ ಮಾಡಿ ಪೂಜೆಗೈದರು. ಸ್ವಾಮಿ ಪ್ರಕಾಶಾನಂದರು ಪೂಜೆಗೆ ಸಂಬಂಧಿಸಿದ ಮಂತ್ರಗಳನ್ನು ಪಠಿಸಿದರು. ವಿಧಿವತ್ತಾಗಿ ವಿಗ್ರಹವನ್ನು ಪ್ರತಿಷ್ಠಾಪಿಸಿದ ಮೇಲೆ ಸ್ವಾಮೀಜಿಯವರು ವಿಗ್ರಹದ ಮುಂದೆ ಧ್ಯಾನಮುದ್ರೆಯಲ್ಲಿ ಕುಳಿತುಕೊಂಡು ಭಾವಪರವಶರಾಗಿ ಅಲ್ಲಿಯೇ ಶ್ಲೋಕವೊಂದನ್ನು ರಚಿಸಿ ಪಠಿಸಿದರು:

\begin{verse}
ಸ್ಥಾಪಕಾಯ ಚ ಧರ್ಮಸ್ಯ ಸರ್ವಧರ್ಮಸ್ವರೂಪಿಣೇ ।\\ಅವತಾರವರಿಷ್ಠಾಯ ರಾಮಕೃಷ್ಣಾಯ ತೇ ನಮಃ ॥
\end{verse}

\noindent

ಎಂದರೆ, ‘ಧರ್ಮವನ್ನು ಸಂಸ್ಥಾಪನೆ ಮಾಡಿದವನೂ ಸರ್ವಧರ್ಮಸ್ವರೂಪಿಯೂ, ಅವತಾರ ವರಿಷ್ಠನೂ ಆದ ರಾಮಕೃಷ್ಣ! ನಿನಗೆ ನಮಸ್ಕಾರ.’

ನವಗೋಪಾಲನ ಭಕ್ತಿಪೂರಿತ ಉಪಚಾರವನ್ನು ಸ್ವೀಕರಿಸಿ ಸ್ವಾಮೀಜಿ ತಮ್ಮ ಸಂಗಡಿಗ ರೊಂದಿಗೆ ಮಠಕ್ಕೆ ಹಿಂದಿರುಗಿದರು. ಮಠದಲ್ಲಿ ಅವರಿಗೆ ಬಿಡುವಿಲ್ಲದಷ್ಟು ಕೆಲಸ. ಮುಖ್ಯವಾಗಿ ಸಂಘದ ಸಂನ್ಯಾಸಿ-ಬ್ರಹ್ಮಚಾರಿಗಳಿಗೆ ತರಬೇತಿ ನೀಡುವ ಕೆಲಸ; ಎಲ್ಲಿಯವರೆಗೆಂದರೆ ತಮ್ಮ ಭಾವನೆಗಳು, ಸಂದೇಶಗಳು ಆ ಯುವಸಾಧಕರಲ್ಲಿ ರಕ್ತಗತವಾಗುವವರೆಗೆ. ಸ್ವಾಮೀಜಿಯವರ ಅನುಭವದ ಆಧಾರದ ಮೇಲೆ ಅವರೆಲ್ಲ ಧಾರ್ಮಿಕ-ಆಧ್ಯಾತ್ಮಿಕ ಜೀವನಕ್ಕೊಂದು ನೂತನ ಅರ್ಥವನ್ನು ಕಂಡುಕೊಂಡರು; ಹೊಸ ಬೆಳಕನ್ನು ಕಂಡರು. ಸ್ವಾಮೀಜಿಯವರಿಂದ ಪಡೆದ ಪ್ರಬಲ ಸ್ಫೂರ್ತಿಯಿಂದಾಗಿ ಕೆಲವು ಯುವಸಾಧಕರಲ್ಲಿ ತೀವ್ರವಾದ ಆಧ್ಯಾತ್ಮಿಕ ಸಾಧನೆ ತಪ ಶ್ಚರ್ಯೆಗಳಲ್ಲಿ ನಿರತರಾಗಿರಬೇಕೆಂಬ ಹಂಬಲ ಹುಟ್ಟಿಕೊಂಡಿತು; ಇನ್ನು ಕೆಲವರಲ್ಲಿ ರೋಗಿಗಳ ಹಾಗೂ ದೀನದಲಿತರ ಸೇವೆ ಮಾಡಬೇಕೆಂಬ ಆಸಕ್ತಿಯುಂಟಾಯಿತು; ಮತ್ತೆ ಕೆಲವರಲ್ಲಿ ವೇದಾಂತ ತತ್ತ್ವಗಳ ಪ್ರಸಾರ ಮಾಡಬೇಕೆಂಬ ಉತ್ಸಾಹ ಉದಿಸಿತು. ಎಲ್ಲರ ಮನಸ್ಸಿನಲ್ಲೂ ಸ್ವಾಮೀಜಿಯವರ ವಿಚಾರಲಹರಿಯೇ ತುಂಬಿತ್ತು; ಅವರ ರಾಷ್ಟ್ರಪ್ರೇಮವೇ ತುಂಬಿತ್ತು. ಈ ಸಮಯದಲ್ಲಿ ಸ್ವಾಮೀಜಿಯವರ ವ್ಯಕ್ತಿತ್ವದಲ್ಲಿ ಪ್ರಚಂಡ ಭಾವನೆಗಳ, ಪ್ರಚಂಡ ಆತ್ಮಶಕ್ತಿಯ ಜ್ವಾಲೆ ಧಗಧಗಿಸುತ್ತಿತ್ತು. ವೇದಾಂತ, ಭಗವದ್ಗೀತೆ ಹಾಗೂ ಹಿಂದೂಧರ್ಮದ ವಿವಿಧ ಮತಗಳು ನಿರಂತರ ವಿಚಾರವಿನಿಮಯದ ವಿಷಯಗಳಾಗಿದ್ದುವು. ಆದರೆ ಅವರೆಲ್ಲ ಸದಾಕಾಲದಲ್ಲಿಯೂ ಶ್ರೀರಾಮಕೃಷ್ಣರ ಬೋಧನೆಗಳನ್ನೇ ಕಣ್ಣಮುಂದಿರಿಸಿಕೊಂಡಿದ್ದರು. ಬಾರಾನಾಗೋರ್ ಮಠದ ದಿನಗಳೇ ಮತ್ತೊಮ್ಮೆ ಮೈದಾಳಿದುವು–ಅದೇ ಸಾಧನಾ ಉತ್ಸಾಹ, ಅದೇ ಜ್ಞಾನಪಿಪಾಸೆ, ಅದೇ ಆಧ್ಯಾತ್ಮಿಕತೆಯ ಕಾವು!

ಸ್ವಾಮೀಜಿ ಈ ಸಂದರ್ಭದಲ್ಲೇ (೧೮೯೮ರ ಮಾರ್ಚ್​) ಕಲ್ಕತ್ತದ ಬಳಿಯ ಬೇಲೂರು ಎಂಬಲ್ಲಿ ಗಂಗಾತೀರದಲ್ಲಿನ ಸುಮಾರು ಏಳು ಎಕರೆ ಜಮೀನನ್ನು ನೂತನ ಮಠಸ್ಥಾಪನೆಗಾಗಿ ಕೊಂಡುಕೊಂಡರು. ಅದರ ಬೆಲೆ ಮೂವತ್ತೊಂಬತ್ತು ಸಾವಿರ ರೂಪಾಯಿ. ಈ ಹಣವನ್ನು ನೀಡಿ ದವಳು ಮಿಸ್ ಹೆನ್ರಿಟಾ ಮುಲ್ಲರ್. ನಾವು ಆಗಾಗಲೇ ನೋಡಿದಂತೆ ಈಕೆ ಸ್ವಾಮೀಜಿಯವರ ಶ್ರದ್ಧಾವಂತ ಭಕ್ತೆ. ಸ್ವಾಮೀಜಿ ಇಂಗ್ಲೆಂಡಿನಲ್ಲಿ ಮಾಡಿದ ವೇದಾಂತಪ್ರಸಾರಕಾರ್ಯಕ್ಕೆ ನೆರವಾಗಿ ದ್ದವಳು ಇವಳು. ಇವಳೊಬ್ಬ ಭಾರೀ ಶ್ರೀಮಂತೆಯಾದರೂ ನಡೆಸುತ್ತಿದ್ದುದು ಸರಳ ತಪೋ ಮಯ ಜೀವನವನ್ನು. ಮಿಸ್ ಮುಲ್ಲರ್ ಕ್ರೈಸ್ತಮತೀಯಳಾದರೂ ಉದಾರಿಯಾಗಿದ್ದುದರಿಂದ ಹಾಗೂ ಆಧ್ಯಾತ್ಮಶೀಲಳಾಗಿದ್ದುದರಿಂದ, ಸ್ವಾಮೀಜಿಯವರ ವ್ಯಕ್ತಿತ್ವದಲ್ಲಿ ಹಾಗೂ ಅವರ ಬೋಧನೆಗಳಲ್ಲಿ ಆಧ್ಯಾತ್ಮಿಕ ಸಾಧನೆಗೆ ಬೇಕಾದ ಎಲ್ಲ ಆವಶ್ಯಕ ಅಂಶಗಳನ್ನು ಕಂಡುಕೊಂಡಿ ದ್ದಳು. ಒಮ್ಮೆ ಅವಳು ಪ್ರಪಂಚವನ್ನು ತ್ಯಜಿಸಿ ಸಂನ್ಯಾಸಿನಿಯಾಗಬೇಕೆಂದೂ ಅಪೇಕ್ಷಿಸಿದ್ದಳು. ಆದರೆ ಸ್ವಾಮೀಜಿಯವರೇ ಅವಳನ್ನು ತಡೆದರು. ಸಂನ್ಯಾಸ ತೆಗೆದುಕೊಳ್ಳುವುದಕ್ಕಿಂತ ಪ್ರಪಂಚ ದಲ್ಲೇ ಇದ್ದುಕೊಂಡು ನಿಃಸ್ವಾರ್ಥದಿಂದ ಸೇವೆ ಸಲ್ಲಿಸುವುದೇ ಅವಳ ಪಾಲಿಗೆ ಶ್ರೇಷ್ಠ ಎಂದು ಅವಳಿಗೆ ಬೋಧಿಸಿದರು.

ಅಂತೂ ಮಿಸ್ ಮುಲ್ಲರಳ ಸಹಾಯದಿಂದ ಏಳು ಎಕರೆ ವಿಸ್ತಾರದ ಭೂಮಿಯೇನೋ ಸಿಕ್ಕಿತು. ಆದರೆ ಆ ಹೊಸ ಕಚ್ಚಾ ಸ್ಥಳವನ್ನು ವಾಸಯೋಗ್ಯವಾಗಿಸಲು ಮಾಡಬೇಕಾದ ಕೆಲಸಗಳು ಬಹಳಷ್ಟಿ ದ್ದುವು. ಮುಖ್ಯವಾಗಿ ಅಲ್ಲಿನ ನೆಲವನ್ನು ಸಮಗೊಳಿಸಿ ಕಟ್ಟಡವನ್ನು ನಿರ್ಮಿಸಬೇಕಾಗಿತ್ತು. ಅದೃಷ್ಟಕ್ಕೆ ಈ ವೇಳೆಗೆ ಸರಿಯಾಗಿ ಶ್ರೀರಾಮಕೃಷ್ಣರ ನೇರ ಶಿಷ್ಯರಲ್ಲೊಬ್ಬರಾದ ಹರಿಪ್ರಸನ್ನ ಎಂಬವರು ಬ್ರಹ್ಮಚಾರಿಗಳಾಗಿ ಮಠಕ್ಕೆ ಸೇರಿದರು. ಇವರು ಉತ್ತರ ಪ್ರದೇಶದಲ್ಲಿ ಎಗ್ಸಿಕ್ಯೂಟಿವ್ ಎಂಜಿನಿಯರಾಗಿ ಉದ್ಯೋಗದಲ್ಲಿದ್ದರು. ಇವರು ಇತರ ಗುರುಭಾಯಿಗಳೊಂದಿಗೆ ಹಿಂದೆಯೇ ಸಂನ್ಯಾಸ ಸ್ವೀಕಾರ ಮಾಡಬೇಕಾಗಿದ್ದವರು. ಆದರೆ ಕಾರಣಾಂತರಗಳಿಂದ ಕೆಲಕಾಲ ಉದ್ಯೋಗಕ್ಕೆ ಸೇರಿಕೊಳ್ಳಬೇಕಾಯಿತು. ಈಗ ಸಕಾಲ ಒದಗಿಬಂದಿದೆ. ಆದ್ದರಿಂದ ತಮ್ಮ ಉದ್ಯೋಗವನ್ನು ಬಿಟ್ಟು ಮಠಕ್ಕೆ ಸೇರಿದರು. ಮುಂದೆ ಇವರು ಸಂನ್ಯಾಸ ಸ್ವೀಕರಿಸಿ ಸ್ವಾಮಿ ವಿಜ್ಞಾನಾನಂದರಾದರು. ಇವರು ಎಂಜಿನಿಯರಾದ್ದರಿಂದ ನೂತನ ನಿವೇಶನದಲ್ಲಿ ಕಟ್ಟಡ ಕಟ್ಟುವ ಕೆಲಸ ಸಹಜವಾಗಿಯೇ ಇವರ ಪಾಲಿಗೆ ಬಿತ್ತು. ಸ್ವಾಮಿ ಅದ್ವೈತಾನಂದರು ಇವರಿಗೆ ಸಹಾಯಕರಾಗಿ ನಿಂತರು. ಅಂತೂ ಸಕಾಲಕ್ಕೆ ಆಗಮಿಸಿದ ಸ್ವಾಮಿ ವಿಜ್ಞಾನಾನಂದರ ಮುತುವರ್ಜಿಯಿಂದ ಕಟ್ಟಡ ನಿರ್ಮಾಣ ವಾಯಿತು. ಶ್ರೀರಾಮಕೃಷ್ಣ ಪೂಜಾಗೃಹ, ಪ್ರಾರ್ಥನಾ ಮಂದಿರ, ಸಾಧು ಬ್ರಹ್ಮಚಾರಿಗಳ ವಾಸಕ್ಕೆ ಕೋಣೆಗಳು, ಅಡಿಗೆಮನೆ, ಊಟದ ಮನೆ, ಉಗ್ರಾಣ–ಎಲ್ಲವೂ ಸಿದ್ಧವಾದುವು. ಇವೆಲ್ಲಕ್ಕೂ ಸುಮಾರು ಒಂದು ವರ್ಷ ಹಿಡಿಯಿತು. ಪೂಜಾಗೃಹದ ನಿರ್ಮಾಣಕ್ಕೆ ತಗುಲಿದ ಹೆಚ್ಚಿನ ಖರ್ಚನ್ನು ಶ್ರೀಮತಿ ಸಾರಾಬುಲ್ ವಹಿಸಿಕೊಂಡಳು. ಅಲ್ಲದೆ ಇವಳು ಮಠಕ್ಕೊಂದು ಶಾಶ್ವತ ವರಮಾನ ಬರುವಂತೆ ಒಂದು ದತ್ತಿಯ ವ್ಯವಸ್ಥೆಯನ್ನು ಮಾಡಿದಳು. ಇವಳು ಮಾಡಿದ ಧನ ಸಹಾಯವೇ ಮಠದ ಆರ್ಥಿಕ ಸ್ಥಿತಿಯನ್ನು ಸುಧಾರಿಸಿತು, ಒಂದು ಭದ್ರ ತಳಹದಿಯಾಯಿತು. ಮಠದ ನಿರ್ಮಾಣಕ್ಕಾಗಿ ಮತ್ತು ಶಾಶ್ವತ ನಿಧಿಗಾಗಿ ಒಂದು ಲಕ್ಷ ರೂಪಾಯಿಗೂ ಹೆಚ್ಚಿನ ವೆಚ್ಚ ತಗುಲಿತು. (ಇದು ಅಂದಿನ ಒಂದು ಲಕ್ಷ ಎಂಬುದನ್ನು ಗಮನಿಸಬೇಕು.) ಈ ಮಠವು ಬೇಲೂರು ಮಠ ಎಂಬ ಹೆಸರಿನಿಂದ ಪ್ರಸಿದ್ಧವಾಯಿತು.

ಬೇಲೂರು ಮಠದ ನಿವೇಶನವನ್ನು ಕೊಂಡುಕೊಂಡಾಗ ಹಿಂದೆಯೇ ಒಮ್ಮೆ ಸ್ವಾಮೀಜಿ ನುಡಿದಿದ್ದ ಭವಿಷ್ಯವಾಣಿ ನಿಜವಾದಂತಾಯಿತು. ಅವರು ಪಾಶ್ಚಾತ್ಯ ರಾಷ್ಟ್ರಗಳಿಗೆ ತೆರಳುವುದಕ್ಕೆ ಬಹಳ ಹಿಂದೆಯೇ ಒಂದು ದಿನ ಬಾರಾನಾಗೋರ್ ಮಠದ ಬಳಿಯಿರುವ ಗಂಗೆಯ ಸ್ನಾನ ಘಟ್ಟದ ಮೇಲೆ ನಿಂತಿದ್ದಾಗ ನದಿಯ ಎದುರು ದಡದತ್ತ ಕೈತೋರಿಸುತ್ತ ತಮ್ಮ ಸೋದರ ಸಂನ್ಯಾಸಿಗಳ ಹತ್ತಿರ ಹೇಳಿದ್ದರು, “ನಮ್ಮ ಶಾಶ್ವತ ಮಠವು ಅಸ್ತಿತ್ವಕ್ಕೆ ಬರುವುದು ಅಲ್ಲೇ ಆಸುಪಾಸಿನಲ್ಲಿ ಅಂತ ನನ್ನ ಮನಸ್ಸು ಹೇಳುತ್ತಿದೆ” ಎಂದು. ಈಗ ಆ ಮಾತು ನಿಜವಾಗಿದೆ. ಸ್ಥಳವನ್ನು ಕೊಂಡುಕೊಳ್ಳುವುದರಿಂದ ಹಿಡಿದು ಕಟ್ಟಡ ನಿರ್ಮಾಣದವರೆಗಿನ ಎಲ್ಲ ಕಾರ್ಯಗಳೂ ಸುಸೂತ್ರವಾಗಿ ಮುಗಿದುವು. ಮಠದ ನಿವೇಶನವು ಕಲ್ಕತ್ತ ನಗರದಿಂದ ನಾಲ್ಕು ಮೈಲಿ ದೂರದ ಗಂಗಾತೀರದಲ್ಲಿದ್ದುದರಿಂದ ತುಂಬ ಪ್ರಶಾಂತವಾಗಿತ್ತು.

ಫೆಬ್ರುವರಿ ತಿಂಗಳಲ್ಲಿ ಮಠವನ್ನು ಆಲಂಬಜಾರಿನ ಹಳೆಯ ಕಟ್ಟಡದಿಂದ ನೀಲಾಂಬರ ಮುಖರ್ಜಿ ಎಂಬವರ ತೋಟದ ಮನೆಗೆ ವರ್ಗಾಯಿಸಲಾಯಿತು. ಇದು ಬೇಲೂರು ಮಠದ ನಿವೇಶನದ ಬಳಿಯಿರುವ ಸ್ಥಳ. ಮಠವನ್ನು ಇಲ್ಲಿಗೆ ವರ್ಗಾಯಿಸಿದ್ದು ಎರಡು ಕಾರಣಕ್ಕಾಗಿ: ಮೊದಲನೆಯದಾಗಿ, ಹಿಂದಿನ ವರ್ಷ ಸಂಭವಿಸಿದ ಒಂದು ಭಾರೀ ಭೂಕಂಪದಿಂದಾಗಿ, ಮೊದಲೇ ಜೀರ್ಣವಾಗಿದ್ದ ಆಲಂಬಜಾರಿನ ಮಠದ ಕಟ್ಟಡ ಬಲವಾಗಿ ಅದುರಿ ಮತ್ತಷ್ಟು ಜಖಂ ಆಗಿತ್ತು. ಇನ್ನು ಅಲ್ಲಿ ವಾಸ ಮಾಡುವುದು ಅಪಾಯಕರವಾಗಿತ್ತು. ಎರಡನೆಯ ಉದ್ದೇಶವೇನೆಂದರೆ ನೀಲಾಂಬರ ಮುಖರ್ಜಿಯವರ ಮನೆಯು ಹೊಸದಾಗಿ ಕಟ್ಟಲ್ಪಡುತ್ತಿದ್ದ ಮಠದ ಸಮೀಪ ದಲ್ಲಿಯೇ ಇದ್ದುದರಿಂದ ನಿರ್ಮಾಣ ಕಾರ್ಯದ ಮೇಲ್ವಿಚಾರಣೆಯನ್ನು ನೋಡಿಕೊಳ್ಳುವುದು ಸುಲಭವಾಗಿತ್ತು.

ಬೆಳೆಯುತ್ತಿರುವ ತಮ್ಮ ಸಂಘದ ಜವಾಬ್ದಾರಿಯನ್ನು ಹೊರಲು ಸ್ವಾಮೀಜಿಯವರು, ಅಮೆರಿಕದಲ್ಲಿ ಕಾರ್ಯನಿರತರಾಗಿದ್ದ ಸ್ವಾಮಿ ಶಾರದಾನಂದರಿಗೆ ಕರೆ ಕಳಿಸಿದರು. ಅವರೊಂದಿ ಗಿದ್ದ ಸ್ವಾಮಿ ಅಭೇದಾನಂದರು ಅಲ್ಲಿನ ಕಾರ್ಯಗಳನ್ನು ಒಬ್ಬರೇ ಮುಂದುವರಿಸಿಕೊಂಡು ಬರಲು ಸಮರ್ಥರಾಗಿದ್ದರು. ಇದೇ ವೇಳೆಗೆ ಸ್ವಾಮೀಜಿಯವರ ಆಪ್ತಶಿಷ್ಯೆಯರಾದ ಶ್ರೀಮತಿ ಸಾರಾ ಬುಲ್ ಹಾಗೂ ಮಿಸ್ ಜೋಸೆಫಿನ್ ಮೆಕ್​ಲಾಡರು ಭಾರತದ ಸಂದರ್ಶನಕ್ಕಾಗಿ ಹೊರಟುಬರಲು ಬಯಸಿದರು. ಅವರಿಗೆ ಸ್ವಾಮೀಜಿ ಪತ್ರ ಬರೆದು ಹೃತ್ಪೂರ್ವಕ ಸ್ವಾಗತ ಕೋರಿದರಲ್ಲದೆ, ಭಾರತದಲ್ಲಿ ಅವರು ಯಾವ ಬಗೆಯ ದೃಶ್ಯಗಳನ್ನು ಕಾಣಲು ನಿರೀಕ್ಷಿಸಬಹುದು ಎಂಬುದರ ಬಗ್ಗೆ ಮುನ್ಸೂಚನೆಯನ್ನೂ ಕೊಟ್ಟರು. ಫೆಬ್ರುವರಿ ತಿಂಗಳಲ್ಲಿ ಅವರಿಬ್ಬರೂ ಸ್ವಾಮಿ ಶಾರದಾನಂದರೊಡನೆ ಕಲ್ಕತ್ತಕ್ಕೆ ಆಗಮಿಸಿದಾಗ ಸ್ವಾಮೀಜಿ ಅವರನ್ನೆಲ್ಲ ಆದರದಿಂದ ಬರಮಾಡಿ ಕೊಂಡರು. ಸಾರಾ ಬುಲ್ ಹಾಗೂ ಜೋಸೆಫಿನ್ನರು ಕಲ್ಕತ್ತದ ಹೋಟೆಲೊಂದರಲ್ಲಿ ಇಳಿದುಕೊಂಡರು.

ಒಂದು ದಿನ ಸ್ವಾಮೀಜಿ ತಮ್ಮ ಶಿಷ್ಯೆಯರನ್ನು ಹೊಸ ಮಠದ ನಿವೇಶನಕ್ಕೆ ಕರೆದೊಯ್ದರು. ಅಲ್ಲಿ ನದಿಯ ದಂಡೆಯ ಮೇಲಿದ್ದ ಹಳೆಯದಾದ ಪುಟ್ಟ ಜೋಪಡಿಯೊಂದು ಈ ಮಹಿಳೆಯರ ಕಣ್ಣಿಗೆ ಬಿದ್ದಿತು. ಆಗ ಅವರಿಗೆ ಇದನ್ನೇಕೆ ತಮ್ಮ ವಾಸಕ್ಕೆ ಉಪಯೋಗಿಸಿಕೊಳ್ಳಬಾರದು ಎಂಬ ಆಲೋಚನೆ ಬಂದಿತು. ಏಕೆಂದರೆ ಅವರು ಸ್ವಾಮೀಜಿಯವರನ್ನು ನೋಡಬೇಕಾದರೆ ಅಷ್ಟು ದೂರದ ಕಲ್ಕತ್ತದಿಂದ ಬರಬೇಕಾಗಿತ್ತು. ಅವರು ಆ ಜೋಪಡಿಯನ್ನು ತೋರಿಸುತ್ತ, “ಸ್ವಾಮೀಜಿ, ಈ ಮನೆಯನ್ನು ನಾವು ನಮ್ಮ ವಾಸಕ್ಕೆ ಉಪಯೋಗಿಸಿಕೊಳ್ಳಬಹುದಲ್ಲವೆ?” ಎಂದು ಕೇಳಿದರು. “ಆದರೆ ಅದು ಸ್ವಲ್ಪವೂ ಚೆನ್ನಾಗಿಲ್ಲವಲ್ಲ!” ಎಂದರು ಸ್ವಾಮೀಜಿ. ತಕ್ಷಣ ಅವರಿಬ್ಬರೂ ಹೇಳಿದರು, “ಪರವಾಗಿಲ್ಲ! ನಾವು ಅದನ್ನು ಸರಿಪಡಿಸಿಕೊಳ್ಳುತ್ತೇವೆ.” ಇದಕ್ಕೆ ಸ್ವಾಮೀಜಿಯವ ರೇನೋ ಸಮ್ಮತಿಸಿದರು. ಆದರೆ ಈ ಶ್ರೀಮಂತ ಮಹಿಳೆಯರಿಗೆ ಇಲ್ಲಿರಲು ಸಾಧ್ಯವೆ? ಎಂಬ ಶಂಕೆ ಅವರದು. ಆ ಜೋಪಡಿಯನ್ನು ಶ್ರೀಮತಿ ಬುಲ್ ಮತ್ತು ಮೆಕ್​ಲಾಡ್ ತಮ್ಮ ಖರ್ಚಿನಲ್ಲೇ ದುರಸ್ತಿ ಮಾಡಿಸಿಕೊಂಡರು, ಸುಣ್ಣ ಬಣ್ಣ ಹಾಕಿಸಿದರು, ಸಾಮಾನು-ಸಲಕರಣೆಗಳನ್ನು ಜೋಡಿಸಿ ಕೊಂಡರು. ‘ಗೃಹಪ್ರವೇಶ’ವನ್ನೂ ಮಾಡಿಯೇ ಬಿಟ್ಟರು. ಆ ಜೋಪಡಿಯಲ್ಲೇ ಅವರು ಆರಾಮ ವಾಗಿ ಇದ್ದುಬಿಟ್ಟದ್ದನ್ನು ಕಂಡು ಸ್ವಾಮೀಜಿಯವರಿಗೆ ಅತ್ಯಾಶ್ಚರ್ಯ. ಎಷ್ಟೇ ರಿಪೇರಿ ಮಾಡಿಸಿ ದರೂ ಜೋಪಡಿ ಜೋಪಡಿಯೇ ಅಲ್ಲವೆ? ತಮ್ಮ ಅಮೆರಿಕದ ಶಿಷ್ಯೆ ಕ್ರಿಸ್ಟೀನಳಿಗೆ ಒಂದು ಪತ್ರದಲ್ಲಿ ಸ್ವಾಮೀಜಿ ಬರೆದರು, “ಶ್ರೀಮತಿ ಬುಲ್ ಹಾಗೂ ಮಿಸ್ ಮೆಕ್​ಲಾಡ್ ಇಲ್ಲಿದ್ದಾರೆ... ಅವರು ನಮ್ಮ ಭಾರತೀಯ ಜೀವನದ ಅನನುಕೂಲತೆಗಳಿಗೆ ಹೇಗೆ ಹೊಂದಿಕೊಂಡುಬಿಟ್ಟಿದ್ದಾರೆ ಎಂಬುದು ನಿಜಕ್ಕೂ ಆಶ್ಚರ್ಯಕರ. ಎಲೆಲಾ! ಈ ಯಾಂಕಿಗಳು (ಅಮೆರಿಕನ್ನರು) ಏನು ಬೇಕಾದರೂ ಮಾಡಬಲ್ಲರು! ಆ ಬಾಸ್ಟನ್​-ನ್ಯೂಯಾರ್ಕುಗಳ ಭೋಗವಿಲಾಸದಲ್ಲಿ ಬದುಕು ನಡೆಸಿದ ಇವರು ಈಗ ಈ ಬಡಜೋಪಡಿಯಲ್ಲಿ ಖುಷಿಯಾಗಿದ್ದಾರಲ್ಲ!... ”

ಕೆಲದಿನಗಳಲ್ಲಿ ಮಾರ್ಗರೆಟ್ಟಳೂ ಸಾರಾ ಬುಲ್ ಹಾಗೂ ಮಿಸ್ ಮೆಕ್​ಲಾಡ್​ರನ್ನು ಕೂಡಿ ಕೊಂಡಳು. ಸ್ವಾಮೀಜಿಯವರು ಈ ಜೋಪಡಿಯ ಬಳಿಯಲ್ಲಿ ಪ್ರತಿದಿನವೂ ಕೆಲವು ಗಂಟೆಗಳ ಕಾಲ ತಮ್ಮ ಪಾಶ್ಚಾತ್ಯ ಶಿಷ್ಯೆಯರೊಂದಿಗೆ ಇರತೊಡಗಿದರು. ನದಿಯ ಪಕ್ಕದ ಮರಗಳ ಬುಡದಲ್ಲಿ ಕುಳಿತು ಅವರಿಗೆ ಭಾರತೀಯತೆಯ ವಿಶ್ವವನ್ನೇ ತೆರೆದಿಡುತ್ತಿದ್ದರು. ಭಾರತೀಯ ಇತಿಹಾಸ, ಅದರ ಸಂಸ್ಕೃತಿ ಸಂಪ್ರದಾಯಗಳು, ಜಾತಿ ಮತ ಪದ್ಧತಿಗಳು–ಇವುಗಳನ್ನೆಲ್ಲ ಅವರು ಕಣ್ಣಿಗೆ ಕಟ್ಟುವಂತೆ ವರ್ಣಿಸುತ್ತಿದ್ದರು. ಭಾರತದಲ್ಲಿ ಜನಿಸಿದ ವಿವಿಧ ಧರ್ಮಗಳ ಆದರ್ಶಗಳನ್ನೂ ಅವುಗಳ ವಾಸ್ತವಿಕತೆಗಳನ್ನೂ ಸ್ವಾಮೀಜಿ ಎಷ್ಟು ಕಾವ್ಯಾತ್ಮಕವಾಗಿ, ವಿಶದವಾಗಿ ವಿವರಿಸುತ್ತಿದ್ದ ರೆಂದರೆ ಅದರ ಕುರಿತಾಗಿ ಮೆಕ್​ಲಾಡ್ ಹೇಳುತ್ತಾಳೆ, “ಸಮಗ್ರ ಭಾರತವೇ ಒಂದು ಭವ್ಯ ಪುರಾಣವಾಗಿ ಅವರ ತುಟಿಗಳಿಂದ ತಾನಾಗಿಯೇ ಹರಿದು ಬರುತ್ತಿದ್ದಂತಿತ್ತು.” ತಮ್ಮ ಈ ಶಿಷ್ಯೆಯರು ಭಾರತದ ಬಗ್ಗೆ ಇಟ್ಟುಕೊಂಡಿದ್ದ ತಪ್ಪು ಕಲ್ಪನೆಗಳನ್ನು, ಪೂರ್ವಗ್ರಹಗಳನ್ನು ಸ್ವಾಮೀಜಿ ನಿರ್ದಾಕ್ಷಿಣ್ಯವಾಗಿ ಖಂಡಿಸುತ್ತಿದ್ದರು. ಪಾಶ್ಚಾತ್ಯರ ದೃಷ್ಟಿಗೆ ಹಿಂದೂ ಧರ್ಮದ ಯಾವಯಾವ ಅಂಶಗಳು ಮೊದಲ ನೋಟಕ್ಕೆ ವಿಚಿತ್ರವಾಗಿ ಅಥವಾ ಜುಗುಪ್ಸಾಕಾರಕವಾಗಿ ಕಂಡುಬರಬಹುದೋ ಆ ಅಂಶಗಳನ್ನೆಲ್ಲ ಮರೆಮಾಚಲು ಅವರೆಂದೂ ಪ್ರಯತ್ನಿಸಲಿಲ್ಲ. ಬದ ಲಾಗಿ ಅದೇ ಅಂಶಗಳ ನಗ್ನರೂಪವನ್ನು ಅವರ ಮುಂದಿಟ್ಟು ಅವುಗಳನ್ನು ಯಥಾವತ್ತಾಗಿ ಅರಿತುಕೊಳ್ಳುವಂತೆ ಪ್ರೋತ್ಸಾಹಿಸುತ್ತಿದ್ದರು. ಆ ಪಾಶ್ಚಾತ್ಯರ ಪ್ರಧಾನವಾದ ಕಷ್ಟ ಯಾವುದೆಂದರೆ ಹಿಂದೂ ಸಂಪ್ರದಾಯದ ಧಾರ್ಮಿಕ ಆದರ್ಶಗಳನ್ನು ಹಾಗೂ ಪೂಜಾ ವಿಧಾನಗಳ ಭಾವವನ್ನು ಗ್ರಹಿಸುವುದು. ಸ್ವಾಮೀಜಿ ಇವುಗಳನ್ನೆಲ್ಲ ಆ ಶಿಷ್ಯೆಯರಿಗೆ ತಿಳಿಯಪಡಿಸಲು ಭಗೀರಥ ಪ್ರಯತ್ನ ವನ್ನೇ ಮಾಡಿದರು. ಅವರ ಅಪಾರ ಉತ್ಸಾಹದ ಫಲವಾಗಿ ಆ ಪಾಶ್ಚಾತ್ಯರು ಹಿಂದೂ ಧರ್ಮದ ಸಂಕೇತಗಳ ಅರ್ಥವನ್ನು ಸಾವಕಾಶವಾಗಿ ಗ್ರಹಿಸುತ್ತ ಬಂದರು; ಹಿಂದೂ ಆದರ್ಶಗಳನ್ನು ಮೈಗೂಡಿಸಿಕೊಂಡು ತಮ್ಮದಾಗಿಸಿಕೊಳ್ಳುವವರೆಗೆ ಕಲಿಯುತ್ತ ಬಂದರು. ನಿಜಕ್ಕೂ ಸ್ವಾಮೀಜಿ ಯವರಲ್ಲಿ ಪೂರ್ವ-ಪಶ್ಚಿಮಗಳೆರಡೂ ಸಮೀಕರಣಗೊಂಡಿದ್ದುವು. ಆದ್ದರಿಂದಲೇ ಅವರ ಪ್ರಾಚ್ಯ-ಪಾಶ್ಚಾತ್ಯ ಶಿಷ್ಯರು ಅವರಲ್ಲಿ, ಅವರ ಭಾವದಲ್ಲಿ, ಅವರ ಜೀವನದಲ್ಲಿ ಬೆರೆತು ಒಂದಾ ಗಲು ಸಾಧ್ಯವಾಯಿತು. ಆದರೆ ಇದು ಅಷ್ಟೊಂದು ಸುಲಭದಲ್ಲಿ ಸಾಧ್ಯವಾಗುವ ಕೆಲಸವೇನಾಗಿರ ಲಿಲ್ಲ. ಪಾಶ್ಚಾತ್ಯರಲ್ಲಿ ಪರಂಪರಾಗತವಾಗಿ ರೂಢಮೂಲವಾಗಿರುವ ಚಿಂತನ ಪ್ರವಾಹವನ್ನು ಭಾರತೀಯ ಚಿಂತನಧಾರೆಯೊಂದಿಗೆ ಸೇರಿಸಿ ಪ್ರವಹಿಸುವಂತೆ ಮಾಡುವ ಕಾರ್ಯ ಸುಲಭ ಹೇಗಾದೀತು? ಈ ಕಾರ್ಯಕ್ಕೆ ಪ್ರಬಲ ವ್ಯಕ್ತಿಯೊಬ್ಬನ ಆವಶ್ಯಕತೆಯಿತ್ತು. ಸ್ವಾಮೀಜಿ ಅಂತಹ ಒಬ್ಬ ವ್ಯಕ್ತಿ ಅಥವಾ ಒಂದು ಶಕ್ತಿಯಾಗಿದ್ದರು. ತಮ್ಮ ಪಾಶ್ಚಾತ್ಯ ಶಿಷ್ಯೆಯರಿಗೆ ಭಾರತೀಯ ಭಾವನೆಗಳನ್ನು ಮನಗಾಣಿಸುವಲ್ಲಿ ಸ್ವಾಮೀಜಿಯವರು ತಳೆದ ಸಹನೆ ಅಪರಿಮಿತ. ತಾವು ನಿರರ್ಗಳವಾಗಿ ವ್ಯಾಖ್ಯಾನಿಸುತ್ತಿರುವಾಗ, ವಿಷಯವನ್ನು ನಿರೂಪಿಸುತ್ತಿರುವಾಗ ಆ ಶಿಷ್ಯೆಯರು ಮಧ್ಯೆಮಧ್ಯೆ ಎಷ್ಟೇ ಅಸಂಬದ್ಧ ಪ್ರಶ್ನೆಗಳನ್ನು ಕೇಳಿದರೂ ಅದಕ್ಕೆ ಸ್ವಾಮೀಜಿ ಸ್ವಲ್ಪವೂ ಬೇಸರಿಸದೆ ಉತ್ತರಿಸುತ್ತಿದ್ದರು. ಏಕೆಂದರೆ ಭಾರತೀಯ ವಿಚಾರಗಳನ್ನು ಅರ್ಥಮಾಡಿಕೊಳ್ಳುವಲ್ಲಿ ಪಾಶ್ಚಾತ್ಯ ರಿಗೆ ಯಾವ ಕಷ್ಟವಿದೆ ಎಂಬುದು ಅವರಿಗೆ ಚೆನ್ನಾಗಿ ಗೊತ್ತಿತ್ತು.

ಓರ್ವ ಗುರುವಾಗಿ ಹಾಗೂ ವಿಶಾಲದೃಷ್ಟಿಯ ಹಿಂದೂವಾಗಿ, ಸ್ವಾಮೀಜಿಯವರಿಗೆ ಭಾರತಕ್ಕೆ ಆಗಮಿಸಿದ ತಮ್ಮ ಪಾಶ್ಚಾತ್ಯ ಶಿಷ್ಯೆಯರನ್ನು ತರಬೇತಿಗೊಳಿಸುವುದೊಂದು ಬಹಳ ದೊಡ್ಡ ಕಳಕಳಿಯ ವಿಷಯವಾಗಿತ್ತು. ಮತ್ತು ಇದೊಂದು ಬಹುದೊಡ್ಡ ಹೊಣೆಗಾರಿಕೆಯ ಕಾರ್ಯವೂ ಕೂಡ. ಅವರು ತಮ್ಮ ಪಾಶ್ಚಾತ್ಯ ಶಿಷ್ಯರಿಗೆ ವೇದಾಂತವನ್ನು ವಿವರಿಸಿ ಹೇಳಿದ್ದುದು ಎಷ್ಟರ ಮಟ್ಟಿಗೆ ಸಾರ್ಥಕವಾಗಿದೆ ಎಂಬುದು ಈಗ ಪರೀಕ್ಷೆಗೆ ಗುರಿಯಾಗಲಿತ್ತು. ಇದು ಕೂಡ ಸ್ವಾಮೀಜಿ ಯವರಿಗೆ ತಿಳಿದಿತ್ತು. ಈ ಪಾಶ್ಚಾತ್ಯರು ಈಗ ಭಾರತದಲ್ಲೇ ಇದ್ದುದರಿಂದ ಅವರಿಗೆ ಹಿಂದೂಗಳ ಭಾವನೆಗಳು, ಆಲೋಚನಾ ವಿಧಾನಗಳು, ರೀತಿನೀತಿಗಳು, ಆಹಾರಾಭ್ಯಾಸಗಳು, ಲೋಪದೋಷ ಗಳು–ಇವುಗಳೆಲ್ಲದರ ಸೂಕ್ಷ್ಮಪರಿಚಯವಾಗುತ್ತಿತ್ತು. ಆದ್ದರಿಂದ ತಮ್ಮ ದೇಶದ ಭೋಗಭರಿತ ಜೀವನಕ್ರಮಕ್ಕೂ ಈ ಭಾರತೀಯರ ಬಡಜೀವನ ಕ್ರಮಕ್ಕೂ ಇರುವ ತಾರತಮ್ಯವು ಅವರ ಅರಿವಿಗೆ ಬರುತ್ತಿತ್ತು. ಇವುಗಳನ್ನೆಲ್ಲ ಕಂಡೂ ವೇದಾಂತದಲ್ಲಿ ಅವರ ಶ್ರದ್ಧೆ ಎಷ್ಟು ದೃಢವಾಗಿ ನೆಲೆನಿಂತಿದೆ ಎಂಬುದನ್ನು ಒರೆಹಚ್ಚಿ ನೋಡಲು ಇದೊಂದು ಸುಸಂದರ್ಭ ಎಂದು ಸ್ವಾಮೀಜಿ ಭಾವಿಸಿದರು. ಆದರೆ ಈ ಪಾಶ್ಚಾತ್ಯ ಶಿಷ್ಯರಿಗೆ ಭಾರತದಲ್ಲಿ ಕಂಡು ಬಂದ ಅತ್ಯಂತ ಆಶ್ಚರ್ಯಕರ ವಿಷಯವೆಂದರೆ ಅದು ಸ್ವತಃ ಸ್ವಾಮೀಜಿಯವರೇ ಆಗಿದ್ದರೆಂಬುದು ಬಹುಶಃ ಸ್ವಾಮೀಜಿಯವರಿಗೆ ಹೊಳೆದಿರಲಿಕ್ಕಿಲ್ಲ. ಈ ಪಾಶ್ಚಾತ್ಯರು ಅವರನ್ನು ತಮ್ಮ ದೇಶಗಳಲ್ಲಿ ಕೇವಲ ಧರ್ಮಗುರುವನ್ನಾಗಿ ಕಂಡವರು, ಆಚಾರ್ಯನನ್ನಾಗಿ ಕಂಡವರು, ಹಿಂದೂಧರ್ಮದ ಪ್ರತಿನಿಧಿಯನ್ನಾಗಿ ಕಂಡವರು. ಆಗ ಸ್ವಾಮೀಜಿಯವರ ಏಕಮಾತ್ರ ಉದ್ದೇಶ ಮಾನವ ಜನಾಂಗದ ಅಜ್ಞಾನದ ಪೊರೆಯನ್ನು ಹರಿದು, ಪ್ರಪಂಚದ ರಾಷ್ಟ್ರರಾಷ್ಟ್ರಗಳಲ್ಲಿ-ಮತಪಂಥಗಳಲ್ಲಿ ಸಹೋದರತೆಯ ಸೌಹಾರ್ದವನ್ನು ನೆಲೆಗೊಳಿಸುವುದಾಗಿತ್ತು. ಆ ಶಿಷ್ಯೆಯರು ಹಿಂದೆ ಸ್ವಾಮೀಜಿಯವರನ್ನು ಕಂಡ ಬಗ್ಗೆ ಈ ರೀತಿಯದಾಗಿತ್ತು. ಆದರೆ ಈಗ ಅವರೇ ಸ್ವಾಮೀಜಿಯವರನ್ನು ಭಾರತದಲ್ಲಿ ಬೇರೆಯೇ ರೀತಿ ಯಾಗಿ ಕಾಣುತ್ತಿದ್ದಾರೆ–ಇಲ್ಲಿ ಸ್ವಾಮೀಜಿ ಮುಖ್ಯವಾಗಿ ಒಬ್ಬ ರಾಷ್ಟ್ರಪ್ರೇಮಿಯಾಗಿ ಕಂಡುಬರು ತ್ತಿದ್ದಾರೆ; ಮಾತೃಭೂಮಿಯ ಪುನರುದ್ಧಾರ ಕಾರ್ಯದಲ್ಲಿ ನಿರತನಾದ ನಿಷ್ಠಾವಂತ ಸೇವಕನಾಗಿ ಕಂಡುಬರುತ್ತಿದ್ದಾರೆ; ಕೆಲವೊಮ್ಮೆ ಅವರು ಪಂಜರದ ಸಿಂಹದಂತೆ ಚಡಪಡಿಸುತ್ತಿರುವಂತೆ ಕಂಡುಬರುತ್ತಿದ್ದಾರೆ.

ಸ್ವಾಮೀಜಿಯವರು ತಮ್ಮ ಕಾರ್ಯನಿರ್ವಹಣೆಯ ಸಮಯದಲ್ಲಿ ಎದುರಾಗುವ ಯಾವುದೇ ಅಡ್ಡಿ ಆತಂಕಗಳನ್ನು ಲೆಕ್ಕಿಸದೆ, ಅಂಜದೆ ಅಳುಕದೆ ಮುಂದೆ ಸಾಗುವ ಕ್ಷಾತ್ರವನ್ನು ತೋರಿದರೂ, ಕ್ಷೀಣಿಸುತ್ತಿದ್ದ ತಮ್ಮ ದೇಹಾರೋಗ್ಯವೊಂದೇ ಕೆಲವೊಮ್ಮೆ ಅವರಲ್ಲಿ ನಿರಾಶಾಭಾವವನ್ನುಂಟು ಮಾಡುತ್ತಿತ್ತು. ಆದರೆ ಅವರು ಆ ನಿರಾಶಾಭಾವವನ್ನು ದೂರ ತಳ್ಳಿ, ತಮ್ಮ ಕಾರ್ಯವು ಮುನ್ನಡೆಯುವಂತೆ ಮಾಡಲು ಅತಿಮಾನುಷ ಶಕ್ತಿಯನ್ನೇ ಪ್ರಯೋಗಿಸುತ್ತಿದ್ದರು. ತಮ್ಮ ಶಾರೀರಿಕ ಸ್ಥಿತಿಯು ತಮ್ಮನ್ನು ನಿವೃತ್ತಿ ಜೀವನದತ್ತ ಸೆಳೆಯುತ್ತಿದೆಯೆಂಬುದನ್ನು ಕಂಡು ಅವರು ತಮ್ಮ ಜೀವಿತೋದ್ದೇಶಗಳನ್ನು ಕಾರ್ಯಗತಗೊಳಿಸಬಲ್ಲ ವ್ಯಕ್ತಿಗಳ ನಿರ್ಮಾಣಕ್ಕಾಗಿ ತಮ್ಮ ತನುಮನಗಳನ್ನೇ ತೇಯ್ದರು. ಈ ಉದ್ದೇಶದಿಂದಲೇ ಅವರು ತಮ್ಮ ಪಾಶ್ಚಾತ್ಯ ಅನುಯಾಯಿ ಗಳಿಗೂ ಅಷ್ಟೊಂದು ತರಬೇತಿ ನೀಡಿದುದು. ಅದರಲ್ಲೂ ಮುಖ್ಯವಾಗಿ, ಮಿಸ್ ಮಾರ್ಗರೆಟ್ ನೋಬೆಲ್ಲಳ ಮೇಲೆ ಅವರು ಹೆಚ್ಚಿನ ಭರವಸೆಯನ್ನಿಟ್ಟಿದ್ದರು. ಸ್ವಾಮೀಜಿಯವರ ಮಾತುಕತೆ ಗಳೆಲ್ಲ ಮುಖ್ಯವಾಗಿ ಅವಳನ್ನು ಸಿದ್ಧಗೊಳಿಸುವುದಕ್ಕಾಗಿಯೇ. ಅವರಾಡಿದ ಈ ಎಲ್ಲ ಮಾತುಕತೆ ಗಳು, ಅವರು ಪಟ್ಟ ಶ್ರಮಗಳು ಮಾರ್ಗರೆಟ್ಟಳನ್ನು ‘ಸೋದರಿ ನಿವೇದಿತೆ’ಯನ್ನಾಗಿ ರೂಪಿಸುವಲ್ಲಿ ಮಾತ್ರ ಯಶಸ್ವಿಯಾಗಿದ್ದರೂ, ಅವರ ಆ ಎಲ್ಲ ಶ್ರಮಕ್ಕೆ ಅದೇ ಒಂದು ದೊಡ್ಡ ಪ್ರತಿಫಲ ಎನ್ನ ಬಹುದು.

ತಮ್ಮ ಪಾಶ್ಚಾತ್ಯ ಶಿಷ್ಯೆಯರಿಗೆಲ್ಲ ಇದೊಂದು ಪರೀಕ್ಷಾ ಕಾಲವೆಂದು ಸ್ವಾಮೀಜಿ ತಿಳಿದಿದ್ದರು. ಅವರೆಲ್ಲ ಭಾರತವನ್ನು ಕಣ್ಣಾರೆ ಕಂಡ ಮೇಲೆ ಈಗ ಮನಸ್ಸು ಬದಲಾಯಿಸಿ ತಮ್ಮಿಂದ ದೂರ ವಾದರೂ ಸ್ವಾಮೀಜಿಯವರಿಗೆ ಆಶ್ಚರ್ಯವಾಗುತ್ತಿರಲಿಲ್ಲ. ಅಲ್ಲದೆ ಒಂದು ವೇಳೆ ಆ ಶಿಷ್ಯೆಯರು ಹಾಗೆ ದೂರವಾಗಿದ್ದರೂ ಸ್ವಾಮೀಜಿ ಮಾತ್ರ ಅವರ ವಿಷಯದಲ್ಲಿ ಎಂದಿಗೂ ಅಸಮಾಧಾನ ಗೊಳ್ಳಲು ಸಾಧ್ಯವಿರಲಿಲ್ಲ. ಅವರು ಅದುವರೆವಿಗೂ ತಮಗೆ ನೀಡಿದ ಸಹಾಯ-ಸಹಾನು ಭೂತಿ-ಬೆಂಬಲಗಳಿಗಾಗಿಯೇ ಸ್ವಾಮೀಜಿ ಅವರಿಗೆಲ್ಲ ಚಿರಋಣಿಯಾಗಿದ್ದರು.

