
\chapter{ಪಂಚನದಿಗಳ ನಾಡಿನಲ್ಲಿ}

ಪಂಜಾಬಿಗೆ ಭೇಟಿ ನೀಡಬೇಕೆಂಬುದು ಸ್ವಾಮೀಜಿಯವರ ಬಹುಕಾಲದ ಇಚ್ಛೆಯಾಗಿತ್ತು. ಪಂಜಾಬಿನ ಬಗ್ಗೆ ಅವರಿಗೊಂದು ವಿಶೇಷ ಅಭಿಮಾನ, ಆದರ, ಗೌರವ. ಕಾರಣ ಅದು ಧೀರರ ನಾಡು, ಸಾಹಸಿಗಳ ನೆಲೆವೀಡು ಎಂಬುದು. ಧೀರತೆಯನ್ನು ಯಾವಾಗಲೂ ಮೆಚ್ಚುವವರು ಸ್ವಾಮೀಜಿ. ವಿಶ್ವವನ್ನೇ ಜಯಿಸಿದ ಅಲೆಗ್ಸಾಂಡರ್​ನಂಥವನನ್ನೇ ಕೆಚ್ಚೆದೆಯಿಂದ ಎದುರಿಸಿದ ಪೌರವನಿಂದ ಹಿಡಿದು, ಅಲ್ಲಿ ಜನ್ಮವೆತ್ತಿದ ಪರಾಕ್ರಮಿಗಳು ಅಸಂಖ್ಯಾತ. ಶತಶತಮಾನಗಳ ಅವಧಿಯಲ್ಲಿ ಭಾರತದ ಮೇಲೆ ದಂಡೆತ್ತಿ ಬಂದ ಪರ್ಷಿಯನ್ನರು, ಗ್ರೀಕರು, ಕುಷಾಣರು, ಹೂಣರು, ಮಂಗೋಲರೇ ಮೊದಲಾದವರ ದುರಾಕ್ರಮಣಕ್ಕೆ ಎದೆಯೊಡ್ಡಿ ನಿಂತ ನಾಡು ಈ ಪಂಜಾಬ್. ಅದರಲ್ಲೂ ಸಿಕ್ಖರ ಬಗ್ಗೆ ಸ್ವಾಮೀಜಿಯವರಿಗೆ ಹೆಚ್ಚಿನ ಅಭಿಮಾನ. ಗುರು ನಾನಕರನ್ನು ಅವರು ಅತ್ಯಂತ ಪೂಜ್ಯ ದೃಷ್ಟಿಯಿಂದ ಕಾಣುತ್ತಿದ್ದರು. ಮತ್ತೊಬ್ಬ ಸಿಕ್ಖ್ ಗುರುವಾದ ಗೋವಿಂದ ಸಿಂಗ್​ನನ್ನು ಅವರು “ನಮ್ಮ ಜನಾಂಗದ ಗಂಡುಗಲಿಗಳಲ್ಲೊಬ್ಬ” ಎಂದು ಬಣ್ಣಿಸುತ್ತಾರೆ. ಇಂತಹ ಪಂಜಾಬಿಗೆ ಬರುವುದೆಂದರೆ ಅವರಿಗೆ ಅದೊಂದು ಮೈನವಿರೆಬ್ಬಿಸುವ ಅನುಭವ.

ಅಂದಿನ ಪಂಜಾಬಿನ ಜನಸಂಖ್ಯೆಯಲ್ಲಿ ಮುಸಲ್ಮಾನರ ಪ್ರಮಾಣ ಸುಮಾರು ೫೧% ಮತ್ತು ಸಿಕ್ಖರ ಪ್ರಮಾಣ ೧೨%. ಆದರೆ ಪಂಜಾಬಿಗರೆಲ್ಲರೂ ಹಿಂದೂಗಳೇ ಸರಿ ಎಂಬುದು ಸ್ವಾಮೀಜಿ ಯವರ ಅಭಿಪ್ರಾಯ. ಏಕೆಂದರೆ ಸಿಕ್ಖರಾಗಲಿ ಮುಸಲ್ಮಾನರಾಗಲಿ ಅಲ್ಲಿನ ಮೂಲ ನಿವಾಸಿಗಳಾದ ಆರ್ಯರ ವಂಶಸ್ಥರೇ ಮತ್ತು ಆರ್ಯರ ಸಂಸ್ಕೃತಿಯನ್ನು ಹೊಂದಿರುವವರೇ ಅಲ್ಲವೆ? ಆದ್ದ ರಿಂದ ಅವರೆಲ್ಲರೂ ಪವಿತ್ರ ಆರ್ಯಾವರ್ತಕ್ಕೆ ಸೇರಿದವರೇ\footnote{*ಸ್ವಾತಂತ್ರ್ಯಾನಂತರ, ಅಂದಿನ ಪಂಜಾಬ್ ಹಾಗೂ ಕಾಶ್ಮೀರಗಳಿಗೆ ಸೇರಿದ್ದ ಕರಾಚಿ, ಲಾಹೋರ್, ಸಿಯಾಲ್ ಕೋಟ್ ಹಾಗೂ ಇಸ್ಲಾಮಾಬಾದಿನಂತಹ ಪ್ರಮುಖ ನಗರಗಳ ಜೊತೆಯಲ್ಲಿ ಬಹುದೊಡ್ಡ ಭಾಗವೊಂದು ಭಾರತದಿಂದ ಬೇರ್ಪಟ್ಟು ಪಾಕಿಸ್ತಾನ ಎನ್ನಿಸಿಕೊಂಡಿತು. ಆದರೆ ಇಂಥದೊಂದು ವಿಭಜನೆಯನ್ನು ಅಂದಿನವರು ಊಹಿಸಿಯೂ ಇರಲಿಲ್ಲ ಮತ್ತು ‘ಪಾಕಿಸ್ತಾನ’ವೆಂಬ ಹೊಸ ರಾಷ್ಟ್ರದ ಕಲ್ಪನೆಯೇ ಜನರಿಗಿರಲಿಲ್ಲ ಎಂಬುದನ್ನು ನಾವಿಂದು ಮರೆಯಬಾರದು.}. ಆದರೆ ಈ ಪಂಜಾಬಿನಲ್ಲಿ ಧರ್ಮ ನದಿಯು ಬತ್ತಿಹೋದಂತಿತ್ತೆಂಬುದೇ ಸ್ವಾಮೀಜಿಯವರ ದುಃಖ. ಮತ್ತು ಅಲ್ಲಿ ಆ ಧರ್ಮ ನದಿಯು ಮತ್ತೊಮ್ಮೆ ಹರಿದು ಪಂಜಾಬಿಗೆ ಜೀವಕಳೆಯನ್ನೀಯುವಂತೆ ಮಾಡುವುದೇ ಅವರ ಭೇಟಿಯ ಮೂಲೋದ್ದೇಶವಾಗಿತ್ತು.

ಅಕ್ಟೋಬರ್ ೩೧ರಂದು ಸ್ವಾಮೀಜಿ ಹಾಗೂ ಅವರ ಸಂಗಡಿಗರು ಸಿಯಾಲ್ ಕೋಟನ್ನು ತಲುಪಿದಾಗ ಅವರಿಗೆ ಅಲ್ಲಿನ ನಾಗರಿಕರಿಂದ ಆದರದ ಸ್ವಾಗತ ದೊರಕಿತು. ಅಂದು ಸಂಜೆ ನಡೆದ ಭಾರೀ ಸಭೆಯನ್ನುದ್ದೇಶಿಸಿ ಸ್ವಾಮೀಜಿ ‘ಧರ್ಮ’ ಎಂಬ ವಿಷಯವಾಗಿ ಇಂಗ್ಲಿಷಿನಲ್ಲಿ ಮಾತನಾಡಿ ದರು. ಮುಂದೆ ಕೆಲವೇ ದಿನಗಳ ಬಳಿಕ ಅವರು ಲಾಹೋರಿನಲ್ಲಿ ಮಾಡಿದ ಭಾಷಣಗಳಂತೆಯೇ ಈ ಭಾಷಣವೂ ಕೂಡ ಅತ್ಯಂತ ಪ್ರಭಾವಪೂರ್ಣವೂ ಸ್ಫೂರ್ತಿದಾಯಕವೂ ಆಗಿತ್ತೆಂದು ತಿಳಿದು ಬರುತ್ತದೆ. ಆದರೆ ದುರದೃಷ್ಟದಿಂದ ಈ ಭಾಷಣದ ಬಗ್ಗೆ ವಿವರವಾದ ವರದಿಗಳು ಸಿಕ್ಕಿಲ್ಲ. ಮೊದಲಿಗೆ ಸ್ವಾಮೀಜಿ ಇಂಗ್ಲಿಷಿನಲ್ಲಿ ಮಾತನಾಡಿದರಾದರೂ, ಬಳಿಕ ತಾವೇ ಅದನ್ನು ಇಂಗ್ಲಿಷ್ ಬಾರದ ಸಾಮಾನ್ಯಜನಗಳಿಗಾಗಿ ಹಿಂದಿಯಲ್ಲಿ ಅನುವಾದಿಸಿ ಹೇಳಿದರು. ಇದಲ್ಲದೆ ಅವರು ಸಿಯಾಲ್​ಕೋಟಿನಲ್ಲಿ ‘ಭಕ್ತಿ’ಎಂಬ ವಿಷಯವಾಗಿ ಹಿಂದಿಯಲ್ಲಿ ಮತ್ತೊಂದು ಭಾಷಣ ಮಾಡಿ ದರು. ಭಕ್ತಿ-ಭಾವ ವಿಹೀನವಾಗಿದ್ದ ಪಂಜಾಬಿನಲ್ಲಿ ಭಕ್ತಿರಸವನ್ನು ಹರಿಸುವುದೇ ಸ್ವಾಮೀಜಿ ಯವರ ಉದ್ದೇಶವಾಗಿತ್ತು.

ಈ ಊರಿನಲ್ಲಿ ಸ್ವಾಮೀಜಿಯವರ ದರ್ಶನ-ಸಂದರ್ಶನಗಳನ್ನರಸಿ ಬಂದವರ ಪೈಕಿ ಇಬ್ಬರು ಸಂನ್ಯಾಸಿನಿಯರಿದ್ದರು. ಈ ತ್ಯಾಗಿಗಳಾದ ಸಂನ್ಯಾಸಿನಿಯರೊಂದಿಗೆ ಮಾತನಾಡುತ್ತ ಸ್ವಾಮೀಜಿ, ಹುಡುಗಿಯರಿಗಾಗಿಯೇ ಒಂದು ಶಾಲೆಯನ್ನು ತೆರೆಯುವಂತೆ ಸೂಚಿಸಿದರು. ಈ ಸೂಚನೆಯನ್ನು ಅಲ್ಲಿನ ನಾಗರಿಕರು ಸಂತೋಷದಿಂದ ಅಂಗೀಕರಿಸಿದರು. ಬಳಿಕ ಆ ಉದ್ದೇಶಕ್ಕಾಗಿ ಒಂದು ಸಮಿತಿ ರಚಿತವಾಯಿತು. ವಿದ್ಯಾರ್ಥಿನಿಯರಿಗೆ ಉಪಾಧ್ಯಾಯಿನಿಯರೇ ಶಿಕ್ಷಣ ನೀಡುವಂತಾಗಬೇಕು ಎಂದು ಸ್ವಾಮೀಜಿ ಅಭಿಪ್ರಾಯಪಟ್ಟರು. ಈ ಸಂದರ್ಭದಲ್ಲೇ ಅವರಿಗೆ ಮತ್ತೊಂದು ಆಲೋಚನೆ ಹೊಳೆಯಿತು–ಅನಾಥರಾದ ವಿಧವೆಯರಿಗೆ ಆಶ್ರಯ ಕೊಡಲು ಈ ಹೆಣ್ಣುಮಕ್ಕಳ ಶಾಲೆಗಳೇ ಯಾಕೆ ಒಂದು ಸಾಧನವಾಗಬಾರದು, ಎಂದು. ವಿದ್ಯಾವತಿಯರಾದ ವಿಧವೆಯರಿದ್ದರೆ ಸರಿಯೇ ಸರಿ; ವಿದ್ಯೆಯಿಲ್ಲದ ವಿಧವೆಯರಿಗೂ ಕೂಡ ವಿದ್ಯಾಭ್ಯಾಸವನ್ನು ಕೊಟ್ಟು ಅವರನ್ನು ಶಿಕ್ಷಕಿಯರ ನ್ನಾಗಿ ಬಳಸಿಕೊಳ್ಳಬಹುದು ಎಂಬ ಆಲೋಚನೆ ಅವರಲ್ಲುಂಟಾಯಿತು. (ಆ ದಿನಗಳಲ್ಲಿ ಸಮಾಜ ದಲ್ಲಿ ಬಾಲವಿಧವೆಯರ ಪ್ರಮಾಣ ಬಹಳಷ್ಟಿದ್ದು ಅದೊಂದು ದೊಡ್ಡ ಸಮಸ್ಯೆಯೇ ಆಗಿತ್ತೆಂಬು ದನ್ನು ಇಲ್ಲಿ ಸ್ಮರಿಸಬಹುದು.)

ಸಿಯಾಲ್​ಕೋಟಿನಲ್ಲಿ ತಮ್ಮ ಕಾರ್ಯಯೋಜನೆಗಳ ಬೀಜವನ್ನು ಬಿತ್ತಿ ಸ್ವಾಮೀಜಿ ಲಾಹೋ ರಿಗೆ ಹೊರಟರು. ಅಲ್ಲಿ ರೈಲುನಿಲ್ದಾಣದಲ್ಲಿ ಭಾರೀ ಸಂಖ್ಯೆಯಲ್ಲಿ ನೆರೆದಿದ್ದ ನಾಗರಿಕರು ಅವರನ್ನು ಸ್ವಾಗತಿಸಿದರು. ಬಳಿಕ ಅವರನ್ನು ರಾಜಾ ಧ್ಯಾನ್ ಸಿಂಗನ ಅರಮನೆಗೆ ಮೆರವಣಿಗೆಯಲ್ಲಿ ಕರೆ ದೊಯ್ಯಲಾಯಿತು. ಅವರ ಮಾತನ್ನು ಆಲಿಸಲು ಅಲ್ಲಿ ಆಗಲೇ ಸುಮಾರು ನಾಲ್ಕು ಸಾವಿರ ಜನ ತುಂಬಿಹೋಗಿದ್ದರು. ಇನ್ನೂ ಸುಮಾರು ಎರಡು ಸಾವಿರ ಜನರು ನಿಲ್ಲಲೂ ಜಾಗ ಸಿಗದೆ ನಿರಾಶರಾಗಿ ಹಿಂದಿರುಗಬೇಕಾಯಿತು. ಮದರಾಸಿನ ಸಮಾರಂಭದ ವೇಳೆಯಲ್ಲಿ ಆದಂತೆಯೇ ಇಲ್ಲೂ ಜನಗಳ ಪ್ರಚಂಡ ಉತ್ಸಾಹದಿಂದ ಭಾರೀ ಕೋಲಾಹಲವೇರ್ಪಟ್ಟಿತ್ತು. ಆದರೂ ಸ್ವಾಮೀಜಿ ಬಹಳ ಪ್ರಯತ್ನಪಟ್ಟು ಒಂದೂವರೆ ಗಂಟೆಯ ಕಾಲ ಮಾತನಾಡಿ ಜನರಿಗೆ ಸಂತೋಷವನ್ನುಂಟುಮಾಡಿದರು.

ಸಮಾರಂಭದ ಬಳಿಕ ಸ್ವಾಮೀಜಿ ಕೆಲಕಾಲ ಅರಮನೆಯಲ್ಲಿ ವಿಶ್ರಮಿಸಿ ಗಣ್ಯವ್ಯಕ್ತಿಗಳಿಗೆ ಸಂದರ್ಶನ ನೀಡಿದರು. ಅನಂತರ ಅವರು ಅಲ್ಲಿನ ‘ಟ್ರಿಬ್ಯೂನ್​’ಎಂಬ ಪತ್ರಿಕೆಯ ಸಂಪಾದಕ ರಾದ ಶ್ರೀನಾಗೇಂದ್ರನಾಥ ಗುಪ್ತರ ಮನೆಗೆ ತೆರಳಿದರು. ಈ ಊರಿನ ಆರ್ಯಸಮಾಜೀಯರು ಸ್ವಾಮೀಜಿಯವರಿಗೆ ವಿಶೇಷ ಸತ್ಕಾರ ನೀಡಿದರು. ಸ್ವಾಮೀಜಿಯವರು ಪ್ರತಿದಿನವೂ ಅರಮನೆ ಯಲ್ಲಿ ಸಿಯಾಲ್​ಕೋಟಿನ ಪಂಜಾಬೀ ಹಾಗೂ ಬಂಗಾಳೀ ನಿವಾಸಿಗಳನ್ನುದ್ದೇಶಿಸಿ ಮಾತನಾಡುತ್ತಿ ದ್ದರು. ಒಂದು ದಿನ ಸ್ವಾಮೀಜಿ ಎಲ್ಲರೆದುರಿನಲ್ಲಿ ಯಾರೋ ಒಬ್ಬ ವ್ಯಕ್ತಿಯನ್ನು ಮನಸಾರೆ ಪ್ರಶಂಸಿಸಿದರು. ಅದನ್ನೆಲ್ಲ ಕೇಳಿದವನೊಬ್ಬ, “ಸ್ವಾಮೀಜಿ, ನೀವೇನೋ ಅವನ ಬಗ್ಗೆ ಅಷ್ಟೆಲ್ಲ ಒಳ್ಳೆಯ ಮಾತನಾಡುತ್ತಿದ್ದೀರಿ. ಆದರೆ ಅವನಿಗೆ ನಿಮ್ಮ ಬಗ್ಗೆ ಗೌರವವಿಲ್ಲ” ಎಂದ. ಅದಕ್ಕೆ ಸ್ವಾಮೀಜಿ ತಕ್ಷಣ ಉತ್ತರಿಸಿದರು, “ಅವನು ಒಳ್ಳೆಯವನೆನ್ನಿಸಬೇಕಾದರೆ ನನ್ನನ್ನು ಗೌರವಿಸಲೇ ಬೇಕೆ?”

ಲಾಹೋರಿನಲ್ಲಿ ಸ್ವಾಮೀಜಿ ಮೂರು ಸಾರ್ವಜನಿಕ ಉಪನ್ಯಾಸಗಳನ್ನು ನೀಡಿದರು. ಜಮ್ಮುವಿ ನಲ್ಲಿ ಸ್ವಾಮೀಜಿಯವರ ತಂಡವನ್ನು ಕೂಡಿಕೊಂಡಿದ್ದ ಅವರ ನೆಚ್ಚಿನ ಶೀಘ್ರಲಿಪಿಕಾರ ಗುಡ್ ವಿನ್ನನ ಕೃಪೆಯಿಂದಾಗಿ ಈ ಮೂರು ಉಪನ್ಯಾಸಗಳು ನಮಗಿಂದು ಪೂರ್ಣವಾಗಿ ಲಭಿಸುವಂತಾ ಗಿದೆ. ಇವುಗಳಲ್ಲಿ ಮೊದಲನೆಯದು “ಹಿಂದೂ ಧರ್ಮದ ಸಾಮಾನ್ಯ ತತ್ತ್ವಗಳು\eng{” (The Common Bases of Hinduism)}. ಮಾತನಾಡಲು ಎದ್ದುನಿಂತೊಡನೆ ಅವರಲ್ಲೊಂದು ಸ್ಫೂರ್ತಿ ಚಿಮ್ಮಿತು. ಅತ್ಯಂತ ರೋಮಾಂಚಕಾರಿಯಾದ ತಮ್ಮ ಈ ಭಾಷಣವನ್ನು ಸ್ವಾಮೀಜಿ ಹೀಗೆ ಪ್ರಾರಂಭಿಸಿದರು:

“ಪವಿತ್ರ ನಾಡಾದ ಆರ್ಯಾವರ್ತದಲ್ಲಿಯೂ ಅತಿ ಪವಿತ್ರ ಸ್ಥಳವೆಂದು ಖ್ಯಾತಿಗೊಂಡಿರು ವಂಥದು–ಈ ಪಂಜಾಬು!ಮನು ಮಹರ್ಷಿಯು ‘ಬ್ರಹ್ಮಾವರ್ತ’ವೆಂದು ಕರೆದುದು ಈ ಕ್ಷೇತ್ರವನ್ನೇ. ಅಗಾಧ ಆಧ್ಯಾತ್ಮಿಕ ಜಿಜ್ಞಾಸೆಯ ಅಲೆಯು ಉದಿಸಿದುದು–ಮತ್ತು ಮುಂದೆ ಅದು ಇಡಿಯ ಜಗತ್ತನ್ನೇ ಆವರಿಸಲಿರುವುದು–ಈ ಕ್ಷೇತ್ರ(ಪಂಜಾಬ್​)ದಲ್ಲಿಯೇ. ಪ್ರಪಂಚದ ಮೂಲೆ ಮೂಲೆಗಳಿಗೆ ನುಗ್ಗಿ ಮೇಘಗರ್ಜನೆಗೈಯುತ್ತಿರುವ ಆಧ್ಯಾತ್ಮಿಕ ಉತ್ಕಾಂಕ್ಷೆಯ ಪ್ರವಾಹಗಳು ಜನ್ಮತಳೆದು ಒಂದುಗೂಡಿ ಹರಿದುದು ಈ ನಾಡಿನಿಂದಲೇ. ಭಾರತ ದೇಶದ ಮೇಲೆ ತಂಡೋಪ ತಂಡವಾಗಿ ದಂಡೆತ್ತಿ ಬಂದ ಬರ್ಬರ ಜನರಿಗೆ ಮೊದಲು ತನ್ನೆದೆಯನ್ನು ಅಡ್ಡವಾಗಿ ಒಡ್ಡಿದುದು ಈ ನಾಡು. ಈ ಎಲ್ಲ ದುರಾಕ್ರಮಣಗಳನ್ನೆದುರಿಸಿಯೂ ಈ ನಾಡು ತನ್ನ ವೈಭವವನ್ನು, ಶಕ್ತಿಯನ್ನು ಇನ್ನೂ ಉಳಿಸಿಕೊಂಡಿದೆ. ಈಚಿನ ಶತಮಾನಗಳಲ್ಲಿ, ಸಾಧು ನಾನಕರು ತಮ್ಮ ಅದ್ಭುತವಾದ ವಿಶ್ವ ಪ್ರೇಮವನ್ನು ಬೋಧಿಸಿದುದು ಇಲ್ಲಿಯೆ. ಅವರ ವಿಶಾಲ ಹೃದಯವು ವಿಕಸಿತವಾದುದು, ಸಮಸ್ತ ಹಿಂದೂಗಳನ್ನು ಮಾತ್ರವಲ್ಲದೆ ಮಹಮ್ಮದೀಯರೂ ಸೇರಿದಂತೆ ಸಕಲ ಜಗತ್ತನ್ನೇ ಆಲಿಂಗಿಸಲು ಅವರು ತಮ್ಮ ಬಾಹುಗಳನ್ನು ಹೊರಚಾಚಿದುದು ಇಲ್ಲಿಯೇ. ನಮ್ಮ ಜನಾಂಗದ ಧೀರಾಗ್ರಣಿಗಳ ಲ್ಲೊಬ್ಬರೂ ಇತ್ತೀಚಿನವರೂ ಆದ ಗುರು ಗೋವಿಂದ ಸಿಂಗರು ಧರ್ಮಕ್ಕಾಗಿ, ತಮ್ಮ ಹಾಗೂ ತಮ್ಮಿಷ್ಟಮಿತ್ರರೆಲ್ಲರ ರಕ್ತವನ್ನು ಚೆಲ್ಲಿದುದು ಈ ನಾಡಿನಲ್ಲೇ ಅಲ್ಲವೆ?

“ಪಂಚನದಿಗಳ ಈ ನಾಡಿನಲ್ಲಿ\footnote{*ಪಾಂಚ್ ಆಬ್​(ಐದು ನದಿಗಳು) ಎಂಬುದರಿಂದ ‘ಪಂಜಾಬ್​’ ಪದ ಬಂದಿದೆ. ಇಲ್ಲಿ ಹರಿಯುವ ಈ ಐದು ನದಿಗಳೆಂದರೆ ಝೇಲಂ, ಚೀನಾಬ್, ರಾವಿ, ಬಿಯಾಸ್ ಹಾಗೂ ಸಟ್ಲೆಜ್.}, ಈ ನಮ್ಮ ಪುರಾತನ ದೇಶದಲ್ಲಿ, ನಾನೀಗ ನಿಮ್ಮ ಮುಂದೆ ಗುರುವಿನಂತೆ ನಿಂತಿಲ್ಲ; ಏಕೆಂದರೆ ನಾನು ಇತರರಿಗೆ ಬೋಧಿಸುವಷ್ಟು ಪ್ರಾಜ್ಞನಲ್ಲ. ಬದಲಾಗಿ, ನಮ್ಮ ಪಶ್ಚಿಮದ ಬಂಧುಗಳಾದ ನಿಮ್ಮೊಂದಿಗೆ ಕುಶಲೋಪರಿಯ ಮಾತುಗಳನ್ನಾಡಿ, ಅಭಿಪ್ರಾಯ ವಿನಿಮಯ ಮಾಡಿಕೊಳ್ಳಲು ಪೂರ್ವ ಪ್ರಾಂತ್ಯದಿಂದ ಬಂದು ಇಲ್ಲಿ ನಿಂತಿದ್ದೇನೆ...”

ಬಳಿಕ ಸ್ವಾಮೀಜಿ, ಹಿಂದೂ ಧರ್ಮಕ್ಕೆ ಸೇರಿದ ವಿವಿಧ ಮತಪಂಥಗಳೆಲ್ಲದರ ಹಿಂದಿರುವ ಮೂಲಭೂತ ತತ್ತ್ವಗಳ ಬಗ್ಗೆ ಪ್ರಸ್ತಾಪಿಸುತ್ತ ಹೀಗೆಂದರು:

“ಮಹನೀಯರೇ, ನೀವು ನನ್ನನ್ನು ನಿಮ್ಮಲ್ಲಿಗೆ ಬರಮಾಡಿಕೊಂಡಿರುವುದು ಹಿಂದೂಧರ್ಮದ ಪುನಃಸ್ಥಾಪನೆಯ ಉದ್ದೇಶದಿಂದ... ಇಂದು ನಿಮ್ಮ ಮುಂದೆ ನಾವೆಲ್ಲರೂ ಒಪ್ಪಿರುವ ವಿಷಯ ಗಳನ್ನಿಟ್ಟು, ಸಾಧ್ಯವಾದರೆ, ನಾವೆಲ್ಲರೂ ಒಪ್ಪಬಹುದಾದ ಹಿಂದೂಧರ್ಮದ ಸಾಮಾನ್ಯ ಆಧಾರ ವೊಂದನ್ನು ಕಂಡುಕೊಳ್ಳುವುದು ನನ್ನ ಉದ್ದೇಶ... ಇದುವರೆಗೆ ಜನ್ಮವೆತ್ತಿ ಆತ್ಮಗೌರವದಿಂದ ನಡೆದ ಮಾನವರಲ್ಲಿ ನಾನೂ ಒಬ್ಬ. ಆದರೆ, ನಾನು ಹೆಮ್ಮೆ ಪಡುವುದು ಖಂಡಿತವಾಗಿಯೂ ನನ್ನ ಬಗ್ಗೆ ಅಲ್ಲ, ನನ್ನ ಪೂರ್ವಜರ ಬಗ್ಗೆ. ನಾನು ಚರಿತ್ರೆಯನ್ನು ಹೆಚ್ಚುಹೆಚ್ಚಾಗಿ ಓದಿದಂತೆಲ್ಲ, ಇನ್ನಷ್ಟು ಮತ್ತಷ್ಟು ಹಿಂದಿನ ಕಾಲದೊಳಕ್ಕೆ ಇಣಿಕಿ ನೋಡಿದಂತೆಲ್ಲ, ನಮ್ಮ ಪೂರ್ವಿಕರ ಬಗೆಗಿನ ಅಭಿಮಾನ ನನ್ನಲ್ಲಿ ಹೆಚ್ಚುಹೆಚ್ಚಾಗುತ್ತ ಬಂದಿದೆ. ನನ್ನ ನಂಬಿಕೆಗನುಸಾರವಾಗಿಯೇ ಕೆಲಸ ಮಾಡುವ ಶ್ರದ್ಧೆ-ಶಕ್ತಿಗಳು ನನ್ನಲ್ಲಿ ಉಂಟಾಗಿ, ಮಣ್ಣಧೂಳಿನಲ್ಲಿ ಬಿದ್ದಿದ್ದ ನನ್ನನ್ನು ಅವು ಮೇಲೆತ್ತಿ ನಿಲ್ಲಿಸಿವೆ; ಮಹಾತ್ಮರಾದ ಆ ನಮ್ಮ ಪೂರ್ವಿಕರು ಹಾಕಿಕೊಟ್ಟ ಮಾರ್ಗದಲ್ಲಿ ಮುಂಬರಿ ಯುವಂತೆ ನನ್ನನ್ನು ಪ್ರೇರೇಪಿಸಿವೆ, ಓ ಪ್ರಾಚೀನ ಆರ್ಯರ ಪುತ್ರರಿರಾ, ಭಗವಂತನ ಕೃಪೆಯಿಂದ ನೀವೂ ಈ ಅಭಿಮಾನವನ್ನು ಹೊಂದಿರಿ! ನಿಮ್ಮ ಪೂರ್ವಿಕರ ಕುರಿತಾದ ಶ್ರದ್ಧೆಯು ನಿಮಗೆ ರಕ್ತ ಗತವಾಗಲಿ; ಅದು ನಿಮ್ಮ ಜೀವನದ ಅವಿಚ್ಛಿನ್ನ ಅಂಗವಾಗಲಿ! ಅದು ಇಡಿಯ ಜಗತ್ತಿಗೆ ಸನ್ಮಂಗಳವನ್ನುಂಟುಮಾಡಲಿ!”

ಹಿಂದೂಧರ್ಮದ ವಿವಿಧ ಪಂಥಗಳ ಸಾಮಾನ್ಯ ತತ್ತ್ವಗಳಾವುವು ಎಂಬ ವಿಷಯಕ್ಕೆ ಬರುವ ಮುನ್ನ ಸ್ವಾಮೀಜಿ, ತಾವು ಎಲ್ಲ ಭಾಷಣಗಳಲ್ಲೂ ಹೇಳುತ್ತ ಬಂದಿದ್ದಂತೆ, ಧರ್ಮವೇ ಭಾರತದ ಜೀವನಾಡಿ ಎಂಬ ಅಂಶವನ್ನು ಎತ್ತಿಹಿಡಿದರು:

“ಪ್ರತಿಯೊಬ್ಬ ಮನುಷ್ಯನಿಗೂ ಅವನದೇ ಆದ ವ್ಯಕ್ತಿತ್ವವಿರುವಂತೆ ಪ್ರತಿಯೊಂದು ಜನಾಂ ಗಕ್ಕೂ ಒಂದು ವ್ಯಕ್ತಿತ್ವವಿದೆ. ಪ್ರತಿಯೊಂದು ಜನಾಂಗಕ್ಕೂ ತಾನು ನೆರವೇರಿಸಬೇಕಾದ ಉದ್ದೇಶ ವೊಂದಿರುತ್ತದೆ. ಇತರರಿಗೆ ಅದು ಕೊಡಬೇಕಾದ ಸಂದೇಶವೊಂದಿರುತ್ತದೆ. ನಾವು ನಮ್ಮ ಕಥೆ ಗಳಲ್ಲಿ ಕೇಳಿಲ್ಲವೆ–ಕೆಲವು ರಾಕ್ಷಸರ ಜೀವಗಳು ಪಕ್ಷಿಗಳಲ್ಲಿ ಅಡಗಿರುತ್ತವೆ; ಈ ಪಕ್ಷಿಗಳನ್ನು ಕೊಂದಲ್ಲದೆ ಆ ರಾಕ್ಷಸರಿಗೆ ಬೇರಾವ ವಿಧದಲ್ಲೂ ಸಾವಿಲ್ಲ, ಎಂದು? ಜನಾಂಗಗಳ ವಿಷಯ ದಲ್ಲಿಯೂ ಹೀಗೆಯೇ. ಪ್ರತಿಯೊಂದು ಜನಾಂಗಜೀವನಕ್ಕೂ ಮೂಲ ಒಂದಿರುತ್ತದೆ. ಅದನ್ನು ಮುಟ್ಟಿದ್ದಲ್ಲದೆ ಆ ಜನಾಂಗಕ್ಕೆ ನಾಶವಿಲ್ಲ. ಈ ದೃಷ್ಟಿಯಿಂದ ನೋಡಿದಾಗ, ಚರಿತ್ರೆಯಲ್ಲಿ ಕಂಡು ಬರುವ ಪರಮಾದ್ಭುತ ಸಂಗತಿಯೊಂದನ್ನು ನಾವು ಅರ್ಥಮಾಡಿಕೊಳ್ಳಬಲ್ಲೆವು. ಈ ನಮ್ಮ ಪವಿತ್ರ ನಾಡಿನ ಮೇಲೆ ಅನಾಗರಿಕ ತಂಡಗಳು ಒಂದಾದಮೇಲೊಂದರಂತೆ ದುರಾಕ್ರಮಣ ನಡೆಸಿದುವು. ‘ಅಲ್ಲಾ ಹೋ ಅಕ್ಬರ್​’ ಎಂಬ ಧ್ವನಿ ನೂರಾರು ವರ್ಷಗಳ ಕಾಲ ಆಕಾಶದಲ್ಲೆಲ್ಲ ಪ್ರತಿಧ್ವನಿಸಿತು. ಯಾವ ಹಿಂದೂವಿಗೂ ತನ್ನ ಕೊನೆಯ ಗಳಿಗೆ ಯಾವಾಗ ಸನ್ನಿಹಿತವಾಗುವುದೆಂದು ಗೊತ್ತಿರಲಿಲ್ಲ. ಜಗತ್ತಿನ ಚರಿತ್ರೆಯಲ್ಲಿ, ಎಲ್ಲಕ್ಕಿಂತ ಹೆಚ್ಚಿನ ಸಂಕಟವನ್ನೂ ಪರಾಧೀನತೆಯನ್ನೂ ಅನುಭವಿಸಿದ ರಾಷ್ಟ್ರವೆಂದರೆ ನಮ್ಮದೇ. ಆದರೂ ನಾವು ವಿನಾಶ ಹೊಂದಿಲ್ಲ; ಮತ್ತು ಅಗತ್ಯವಾದಲ್ಲಿ, ಪುನಃ ಪುನಃ ಕಷ್ಟಗಳನ್ನು ಎದುರಿಸಲು ಸಿದ್ಧರಿದ್ದೇವೆ.”

 ಹಿಂದೂಧರ್ಮದಲ್ಲಿ ಇಷ್ಟೊಂದು ಬಗೆಯ, ಇಷ್ಟೊಂದು ಸಂಖ್ಯೆಯ ಮತಪಂಥಗಳೇಕೆ? ಎಂಬ ಪ್ರಶ್ನೆಯ ಬಗ್ಗೆ ಪ್ರಸ್ತಾಪಿಸಿದ ಸ್ವಾಮೀಜಿ ಹೀಗೆ ನುಡಿದರು:

“ಈ ದೇಶದಲ್ಲಿ ಅಸಂಖ್ಯಾತ ಪಂಥಗಳು ಆಗಿಹೋಗಿವೆ; ಈಗಲೂ ಅವು ಸಾಕಷ್ಟು ಸಂಖ್ಯೆ ಯಲ್ಲಿವೆ, ಮುಂದೆಯೂ ಇರುತ್ತವೆ. ಏಕೆಂದರೆ ಇದು ನಮ್ಮ ಧರ್ಮದ ಒಂದು ವೈಶಿಷ್ಟ್ಯ. ಹಿಂದೂಧರ್ಮದ ಮೂಲತತ್ತ್ವಗಳನ್ನು, ಹಲವಾರು ದೃಷ್ಟಿಗಳಿಂದ ಕಂಡು ಅರ್ಥೈಸಲು ಸಾಧ್ಯ ವಿರುವಂತೆ ರಚಿಸಲಾಗಿದೆ. ಆದ್ದರಿಂದ ಕ್ರಮೇಣ ಇವುಗಳ ಆಧಾರದ ಮೇಲೆ ನೂರಾರು ವಿವರ ಗಳು ದಿನಂಪ್ರತಿಯ ವ್ಯವಹಾರದಲ್ಲಿ ಆಚರಣೆಗೆ ತರಲು ಸಾಧ್ಯವಾಗುವಂತೆ ಸುವಿಸ್ತಾರವಾಗಿ ರಚಿ ತವಾಗಿವೆ. ಈ ವಿವರಗಳ ಹಿಂದಿನ ಮೂಲತತ್ತ್ವಗಳು ಮಾತ್ರ ಆಕಾಶದಂತೆ ವಿಶಾಲವೂ ಪ್ರಕೃತಿ ಯಂತೆ ಶಾಶ್ವತವೂ ಆಗಿವೆ. ಆದ್ದರಿಂದ ಭಿನ್ನಭಿನ್ನ ಮತಗಳು-ಸಂಪ್ರದಾಯಗಳು ಇರಬೇಕಾದ್ದು ಸಹಜವೇ. ಆದರೆ ಯಾವುದು ಇರಬಾರದು ಎಂದರೆ ಗುಂಪುಗಾರಿಕೆ-ಮತಾಂಧತೆಗಳು.”

“ಸಮಸ್ತ ಹಿಂದೂಗಳಿಗೂ ಪವಿತ್ರವಾದ ಗ್ರಂಥಗಳೆಂದರೆ ವೇದಗಳು. ಮತ್ತು ಎಲ್ಲ ಪಂಥ ಗಳಿಗೂ ವೇದಗಳೇ ಅಂತಿಮ ಪ್ರಮಾಣ. ಆದ್ದರಿಂದ, ಸ್ವಾಮೀಜಿ ಹೇಳುತ್ತಾರೆ–ಮೊದಲನೆಯ ದಾಗಿ, ನಾವೆಲ್ಲರೂ “ವೇದಗಳ ಹೆಸರಿನಲ್ಲಿ” ಸಹೋದರರು. ಎರಡನೆಯದಾಗಿ, ನಾವು ಪ್ರತಿಯೊ ಬ್ಬರೂ ದೇವರನ್ನು ಒಪ್ಪುತ್ತೇವೆ. ಆ ದೇವರ ಕಲ್ಪನೆಗಳು ಪರಸ್ಪರ ಎಷ್ಟೇ ಭಿನ್ನವಾಗಿರಬಹುದು. ಮತ್ತು, ಇವುಗಳಲ್ಲಿ ಒಂದು ಕಲ್ಪನೆ ಮತ್ತೊಂದಕ್ಕಿಂತ ಉತ್ತಮವಾಗಿರಬಹುದು; ಆದರೆ ಯಾವುದೂ ತುಚ್ಛವಾದದ್ದಲ್ಲ, ನೆನಪಿಡಿ.” ಮೂರನೆಯದಾಗಿ, ಇತರ ಎಲ್ಲ ಧರ್ಮಗಳೂ ಸಾರುವುದೇನೆಂದರೆ ಈ ಜಗತ್ತು ಇಂತಿಷ್ಟು ಸಾವಿರ ವರ್ಷಗಳ ಹಿಂದೆ ಸೃಷ್ಟಿಸಲ್ಪಟ್ಟಿತು ಮತ್ತು ಇಂತಿಷ್ಟು ವರ್ಷಗಳ ಬಳಿಕ ಅದು ಶಾಶ್ವತವಾಗಿ ನಾಶವಾಗುತ್ತದೆ ಎಂದು. ಆದರೆ ಸಮಸ್ತ ಹಿಂದೂಗಳು ನಂಬುವುದೇನೆಂದರೆ, ಸೃಷ್ಟಿಯು ಆದ್ಯಂತರಹಿತವಾದುದು ಎಂದು. ಹಿಂದೂ ಗಳೆಲ್ಲರೂ ಒಪ್ಪುವ ಇನ್ನೊಂದು ಅಂಶವೇನೆಂದರೆ, ಮನುಷ್ಯನು ಕೇವಲ ಅವನ ದೇಹವಲ್ಲ ಅಥವಾ ಮನಸ್ಸೂ ಅಲ್ಲ; ಅವನು ಇವೆರಡನ್ನೂ ಮೀರಿದ ಆತ್ಮವೆಂಬ ವಸ್ತು, ಎಂದು. ಅಲ್ಲದೆ ಹಿಂದೂಧರ್ಮದಲ್ಲಿ ಮಾತ್ರವೇ ಕಂಡುಬರುವ ಒಂದು ತತ್ತ್ವವೆಂದರೆ, ಮುಕ್ತಿಯ ಕಲ್ಪನೆ. ಇತರೆಲ್ಲ ಧರ್ಮಗಳು ಬೋಧಿಸುವ ಅತ್ಯುನ್ನತ ಆದರ್ಶವೆಂದರೆ ಸ್ವರ್ಗ. ಸ್ವರ್ಗವೆಂದರೆ ಮತ್ತಷ್ಟು ಭೋಗ. ಆದರೆ ಮುಕ್ತಿಯೆಂದರೆ ಇಂದ್ರಿಯಭೋಗಗಳೆಲ್ಲವನ್ನೂ ಮೀರಿದ ಅತ್ಯುನ್ನತ ಸ್ಥಿತಿ. ಅಲ್ಲದೆ, ‘ಆತ್ಮವು ಸಹಜವಾಗಿಯೇ ಪರಿಶುದ್ಧವಾದುದು, ಅನಂತ ಶಕ್ತಿಯುಳ್ಳದ್ದು’ ಎಂಬ ತತ್ತ್ವವನ್ನು ಹಿಂದೂಧರ್ಮವು ಮಾತ್ರವೆ ಬೋಧಿಸುವುದು. ನಾವು ಪ್ರತಿಯೊಬ್ಬರೂ ಸರ್ವಶಕ್ತ ನಾದ ಭಗವಂತನ ಮಕ್ಕಳು, ಅನಂತ ಜ್ಯೋತಿಯ ಕಿಡಿಗಳು ಎಂಬುದನ್ನರಿಯಬೇಕು” ಎಂದು ಸ್ವಾಮೀಜಿ ಸಾರಿದರು. “ನಾನು ಕೇವಲ ನೀರಿನ ಮೇಲಿನ ಒಂದು ಗುಳ್ಳೆಯಾಗಿರಬಹುದು, ನೀನು ಪರ್ವತೋಪಮವಾದ ಅಲೆಯಾಗಿರಬಹುದು. ಆದರೇನಂತೆ? ನನ್ನ ಮತ್ತು ನಿನ್ನ–ಇಬ್ಬರ ಹಿಂದಿರುವುದೂ ಅದೇ ಅನಂತ ಸಾಗರ!”

ಹೀಗೆ, ಹಿಂದೂಧರ್ಮದ ಸಕಲ ಮತಪಂಥಗಳಿಗೂ ಸಾಮಾನವಾದ ಹಲವಾರು ಮೂಲ ಭೂತ ತತ್ತ್ವಗಳಿವೆ. ಈ ತತ್ತ್ವಗಳು ಸಂಪ್ರದಾಯವಾದಿಗಳಿಗೂ ಸುಧಾರಣೆಯನ್ನು ಬೆಂಬಲಿಸು ವವರಿಗೂ ಸಮಾನವಾಗಿ ಒಪ್ಪಿಗೆಯಾಗುವಂತಿವೆ. ಅಲ್ಲದೆ, ಧರ್ಮವಿರುವುದು ಬೋಧನೆಯ ಲ್ಲಲ್ಲ, ಅನುಷ್ಠಾನದಲ್ಲಿ–ಸಾಕ್ಷಾತ್ಕಾರದಲ್ಲಿ. ಭಗವಂತನನ್ನು ಸಾಕ್ಷಾತ್ಕರಿಸಿಕೊಳ್ಳಲೇಬೇಕು ಎಂದು ಮತ್ತೆ ಮತ್ತೆ ಸಾರುವುದು ನಮ್ಮ ಹಿಂದೂ ಗ್ರಂಥಗಳು ಮಾತ್ರವೇ. ದೇವರಿದ್ದಾನೆಂಬುದಕ್ಕೆ ಅತಿ ದೊಡ್ಡ ಪುರಾವೆಯೆಂದರೆ ನಮ್ಮ ತರ್ಕಶಕ್ತಿಯಲ್ಲ, ಬದಲಾಗಿ, ಅವನನ್ನು ಹಿಂದಿನವರೂ ಕಂಡಿದ್ದರು, ಇಂದಿನವರೂ ಕಂಡಿದ್ದಾರೆ ಎಂಬುದು. ಯಾವನು ಭಗವಂತನನ್ನು ಕಂಡಿದ್ದಾ ನೆಯೋ ಅವನನ್ನು ಮಾತ್ರವೇ ‘ಧಾರ್ಮಿಕ’ ವ್ಯಕ್ತಿ ಎನ್ನಬಹುದು. ಇದನ್ನು ಅರ್ಥಮಾಡಿ ಕೊಂಡಾಗ, ಮತಾಂಧತೆ ತಾನಾಗಿಯೇ ತೊಲಗಿಹೋಗುತ್ತದೆ. ಆದ್ದರಿಂದ, ಯಾವನಾದರೂ ಮತದ ಹೆಸರಿನಲ್ಲಿ ಹೊಸದೊಂದು ಗಲಭೆಯನ್ನೆಬ್ಬಿಸಲು ಹೊರಟರೆ ಅವನನ್ನು ಕೇಳಿ–‘ನೀನು ದೇವರನ್ನು ನೋಡಿದ್ದೀಯಾ? ನೋಡಿರದಿದ್ದರೆ, ದೇವರ ಹೆಸರಿನಲ್ಲಿ ಬೋಧನೆ ಮಾಡಲು ನಿನಗೇನು ಅಧಿಕಾರವಿದೆ? ಕುರುಡನು ಕುರುಡನಿಗೆ ದಾರಿ ತೋರಿಸಿದಂತೆ!’... ನೀವು ಅಧ್ಯಾತ್ಮಶೀಲರಾಗುವವರೆಗೆ ಭಾರತದ ಪುನರುತ್ಥಾನವಿಲ್ಲ. ಅಲ್ಲದೆ ಇಡೀ ವಿಶ್ವದ ಕಲ್ಯಾಣಕ್ಕೆ ಅದು ಅತ್ಯಾವಶ್ಯಕ. ಪಾಶ್ಚಾತ್ಯ ನಾಗರಿಕತೆಯ ಅಸ್ತಿಭಾರವೇ ಅಲುಗಾಡುತ್ತಿದೆ. ಪರಕೀಯರನ್ನು ಅನುಕರಿಸಲು ಪರದಾಡಬೇಡಿ; ಅನುಕರಣೆಯು ಸಂಸ್ಕೃತಿಯಲ್ಲ. ಇತರರಿಂದ ನಾವು ಕಲಿಯ ಬೇಕಾದದ್ದು ಖಂಡಿತವಾಗಿಯೂ ಸಾಕಷ್ಟಿದೆ; ಯಾರು ಹೊಸದಾಗಿ ಕಲಿಯಲು ನಿರಾಕರಿಸು ತ್ತಾರೆಯೋ ಅವರು ಸತ್ತಂತೆಯೇ ಸರಿ. ಆದ್ದರಿಂದ ಇತರರಲ್ಲಿ ಒಳ್ಳೆಯ ಅಂಶಗಳು ಏನೇ ನಿವೆಯೋ ಅವೆಲ್ಲವನ್ನೂ ಕಲಿತುಕೊಳ್ಳಿ. ರಕ್ತಗತಮಾಡಿಕೊಳ್ಳಿ; ಆದರೆ ನೀವೇ ಅವರಾಗಬೇಡಿ. ಭಾರತೀಯತೆಯನ್ನು ತ್ಯಜಿಸಬೇಡಿ. ಕಡೆಯದಾಗಿ ನಾನು ಹೇಳಬಯಸುವ ಮಾತೆಂದರೆ ಇದು– ಭಾರತದಲ್ಲಿ ಧರ್ಮವೆಂಬುದು ಬಹುಕಾಲದಿಂದಲೂ ಸ್ಥಗಿತಗೊಂಡಿದೆ, ಮೃತಪ್ರಾಯವಾಗಿದೆ. ಅದನ್ನು ನಾವೀಗ ಪುನಶ್ಚೇತನಗೊಳಿಸಬೇಕಾಗಿದೆ. ಅದು ಸಕಲ ಭಾರತೀಯರ ಜೀವನದಲ್ಲೂ ನೆಲಸುವಂತೆ ಮಾಡಲು ನಾನು ನಿರ್ಧರಿಸಿದ್ದೇನೆ.”

ಇದು ಲಾಹೋರಿನಲ್ಲಿ ಸ್ವಾಮೀಜಿ “ಹಿಂದೂಧರ್ಮದ ಮೂಲಭೂತ ತತ್ತ್ವಗಳು” ಎಂಬ ವಿಷಯವನ್ನು ಕುರಿತು ನೀಡಿದ ಪ್ರಥಮ ಉಪನ್ಯಾಸದ ಸಾರಾಂಶ. ಇದಾದ ಮೇಲೆ ಅವರು ಲಾಹೋರಿನಲ್ಲಿ “ಭಕ್ತಿ” ಹಾಗೂ “ವೇದಾಂತ” ಎಂಬ ವಿಷಯಗಳ ಮೇಲೆ ಮತ್ತೆರಡು ಉಪನ್ಯಾಸಗಳನ್ನು ಮಾಡಿದರು. ವೇದಾಂತವನ್ನು ಕುರಿತು ಮಾಡಿದ ಉಪನ್ಯಾಸದಲ್ಲಿ ಸ್ವಾಮೀಜಿ ಯವರು ಅದ್ವೈತ ವೇದಾಂತವನ್ನು ಅತ್ಯಂತ ಸ್ಪಷ್ಟವಾಗಿ ಹಾಗೂ ಪರಿಣಾಮಕಾರಿಯಾಗಿ ಪ್ರತಿಪಾದಿಸಿದರು. “ಹಿಂದೂಧರ್ಮದ ಮನೋವೈಜ್ಞಾನಿಕ ಹಾಗೂ ತಾತ್ತ್ವಿಕ ನೆಲೆಗಟ್ಟನ್ನು ಸ್ವಾಮೀಜಿ ವಿವರಿಸಿದ ರೀತಿ ಅದ್ಭುತವಾಗಿತ್ತು... ಉಪನ್ಯಾಸದ ಮೊದಲಿನಿಂದ ಕಡೆಯವರೆಗೂ ಅವರು ಬೋಧಿಸಿದ್ದು ಶಕ್ತಿಯನ್ನು; ಮತ್ತು, ಮನುಷ್ಯನಲ್ಲಿ, ಮನುಷ್ಯತ್ವದಲ್ಲಿ ನಂಬಿಕೆಯಿದ್ದರೆ ಭಗವಂತನಲ್ಲಿ ನಂಬಿಕೆ ತಾನಾಗಿಯೇ ಉಂಟಾಗುತ್ತದೆ ಎಂಬುದನ್ನು. ಭಾರತದಲ್ಲಿ ಅವರು ನೀಡಿದ ಉಪನ್ಯಾಸಗಳಲ್ಲೆಲ್ಲ ಶ್ರೇಷ್ಠತಮವಾದ ಈ ಉಪನ್ಯಾಸದ ಪ್ರತಿಯೊಂದು ಪದವೂ ಶಕ್ತಿಪ್ರದವಾಗಿತ್ತು, ಶಕ್ತಿಪೂರ್ಣವಾಗಿತ್ತು” ಎಂದು ಗುಡ್​ವಿನ್ ಬರೆದಿದ್ದಾರೆ.

ವೇದಾಂತದ ತತ್ತ್ವಗಳನ್ನು ಹಾಗೂ ಅದರ ತರ್ಕಬದ್ಧತೆಯನ್ನು ವಿವರಿಸಲು ಸ್ವಾಮೀಜಿ ತಮ್ಮ ಈ ಉಪನ್ಯಾಸದ ಬಹುಪಾಲನ್ನು ವಿನಿಯೋಗಿಸಿದರು. ಕಡೆಯಲ್ಲಿ ಅವರು ಆ ವೇದಾಂತವು ಕೇವಲ ನೀರಸವಾದ—ನಿರರ್ಥಕವಾದ ತತ್ತ್ವವಾದವಲ್ಲವೆಂಬುದನ್ನು ಮನವರಿಕೆ ಮಾಡಿಕೊಡುತ್ತ, ವೇದಾಂತದ ಅನುಷ್ಠಾನದ ಬಗ್ಗೆ ಅತ್ಯಂತ ಉಜ್ವಲವಾಗಿ ಮಾತನಾಡಿದರು–

“... ಇದೇ ಅದ್ವೈತ ವೇದಾಂತದ ಅನುಷ್ಠಾನದ ಅಂಶ: ನಿಮ್ಮಲ್ಲಿ ನೀವು ಶ್ರದ್ಧೆಯಿಡಿ. ನಿಮಗೆ ಲೌಕಿಕ ಸಂಪತ್ತು ಬೇಕಾದರೆ ಕಷ್ಟಪಟ್ಟು ದುಡಿಯಿರಿ; ಅದು ನಿಮಗೆ ಲಭಿಸಿಯೇ ಲಭಿಸುತ್ತದೆ. ಬೇಕಾದರೆ ಬೌದ್ಧಿಕ ಭೂಮಿಕೆಯಲ್ಲಿ ಸಾಧನೆ ಮಾಡಿ; ನೀವು ಪ್ರಚಂಡ ಮೇಧಾಶಾಲಿಗಳಾಗುತ್ತೀರಿ. ನಿಮಗೆ ಮುಕ್ತಿ ಬೇಕಾದರೆ ಆಧ್ಯಾತ್ಮಿಕ ಭೂಮಿಕೆಯಲ್ಲಿ ಸಾಧನೆ ಮಾಡಿ; ನೀವು ಮುಕ್ತರಾಗುತ್ತೀರಿ, ನಿರ್ವಾಣ ಸುಖವನ್ನು ಪಡೆಯುತ್ತೀರಿ. ಆದರೆ ಈ ವೇದಾಂತ ವಾದದಲ್ಲಿ ಒಂದೇ ಒಂದು ದೋಷವೆಂದರೆ, ಇದುವರೆವಿಗೂ ಅದು ಆಧ್ಯಾತ್ಮಿಕ ಭೂಮಿಕೆಯಲ್ಲೇ ಸ್ಥಗಿತವಾಗಿಬಿಟ್ಟಿತ್ತು ಎನ್ನುವುದು. ಈಗ ಅದನ್ನು ಕರ್ಮಭೂಮಿಕೆಗೆ ಎಳೆದು ತರುವ ಸಮಯ ಸನ್ನಿಹಿತವಾಗಿದೆ. ಇನ್ನು ಮುಂದೆ ಅದೊಂದು ರಹಸ್ಯವಾಗಿ ಉಳಿಯಬಾರದು. ಅದು ಹಿಮಾ ಲಯದ ಗುಹಾಂತರಗಳಲ್ಲಿರುವ ಸಾಧುಸಂನ್ಯಾಸಿಗಳ ಆಸ್ತಿಯಾಗಿ ಉಳಿಯಬಾರದು. ಅದು ಸಾಮಾನ್ಯ ಜನರ ದಿನನಿತ್ಯದ ಉಪಯೋಗದ ವಸ್ತುವಾಗಬೇಕು. ಅದು ಅರಸರ ಅರಮನೆಗಳಲ್ಲೂ ಪುಷಿಗಳ ಪರ್ಣಕುಟಿಗಳಲ್ಲೂ ಬಡವರ ಗುಡಿಸಲುಗಳಲ್ಲೂ ಕಾರ್ಯರೂಪಕ್ಕೆ ಬರಬೇಕು. ಎಲ್ಲೆಲ್ಲಿಯೂ ಅದನ್ನು ಕಾರ್ಯಗತಗೊಳಿಸಲು ಸಾಧ್ಯವಿದೆ. ಆದ್ದರಿಂದ, ನೀವು ಸ್ತ್ರೀಯಾಗಿರ ಬಹುದು ಇಲ್ಲವೆ ಶೂದ್ರರಾಗಿರಬಹುದು, ನೀವು ಹೆದರಬೇಕಾದ ಕಾರಣವಿಲ್ಲ. ಗೀತೆಯಲ್ಲಿ ಶ್ರೀಕೃಷ್ಣ ಹೇಳುವಂತೆ–‘ಸ್ವಲ್ಪಮಪ್ಯಸ್ಯ ಧರ್ಮ ಸ್ಯತ್ರಾಯತೇ ಮಹತೋ ಭಯಾತ್​’–ಈ ತತ್ತ್ವದ ಸ್ವಲ್ಪಮಾತ್ರ ಅನುಷ್ಠಾನವೂ ನಮ್ಮನ್ನು ಮಹತ್ತಾದ ಭಯದಿಂದ ಪಾರುಮಾಡುತ್ತದೆ...”

“ಪಂಜಾಬಿನ ತರುಣರಿರಾ, ಇದನ್ನು ತಿಳಿಯಿರಿ. ವಂಶಪಾರಂಪರ್ಯವಾಗಿ ಬಂದ, ರಾಷ್ಟ್ರದ ಸಮಸ್ತ ಮಹಾಪಾಪವು ನಮ್ಮ ತಲೆಯ ಮೇಲಿದೆ. ನೀವು ಸಾವಿರ ಸುಧಾರಕ ಸಂಘಗಳನ್ನೂ ಇಪ್ಪತ್ತು ಸಾವಿರ ರಾಜಕೀಯ ಪಂಗಡಗಳನ್ನೂ ಸೇರಿಸಬಹುದು. ಆದರೆ ಅವುಗಳಿಂದಲೇ ನಾವು ಉದ್ಧಾರವಾಗಲಾರೆವು. ಎಲ್ಲಿಯವರೆಗೆ ನಿಜವಾದ ಸಹಾನುಭೂತಿಯೂ, ಪ್ರೀತಿಯೂ, ಎಲ್ಲರಿ ಗಾಗಿ ಮರುಗುವ ಅಂತಃಕರಣವೂ ನಮ್ಮಲ್ಲಿ ಉಂಟಾಗುವುದಿಲ್ಲವೋ, ಎಲ್ಲಿಯವರೆಗೆ ಬುದ್ಧನ ಹೃದಯವು ಮತ್ತೆ ಭರತಖಂಡದಲ್ಲಿ ಆವಿರ್ಭವಿಸುವುದಿಲ್ಲವೋ, ಎಲ್ಲಿಯವರೆಗೆ ಶ್ರೀಕೃಷ್ಣನ ಬೋಧನೆಗಳು ಅನುಷ್ಠಾನಕ್ಕೆ ಬರುವುದಿಲ್ಲವೋ ಅಲ್ಲಿಯವರೆಗೂ ನಮಗೆ ಯಾವುದೇ ಭವಿಷ್ಯ ವಿಲ್ಲ... ಪ್ರತಿಯೊಂದು ಪಾಶ್ಚಾತ್ಯ ರಾಷ್ಟ್ರದಲ್ಲಿಯೂ ನನ್ನನ್ನು ಎಷ್ಟು ಆದರದಿಂದ ಸ್ವಾಗತಿಸಿ ಸ್ವೀಕರಿಸಿದರೆಂದು ನಾನಿಲ್ಲಿ ಹೇಳದಿದ್ದರೆ ನಾನು ಮಹಾ ಕೃತಘ್ನನಾಗುತ್ತೇನೆ. ರಾಷ್ಟ್ರನಿರ್ಮಾಣಕ್ಕೆ ಅಂತಹ ಹೃದಯದ ಆಧಾರ ಇಲ್ಲಿ ಎಲ್ಲಿದೆ? ನಾಲ್ಕು ಜನ ಸೇರಿ ಒಂದು ಸಹಕಾರ ಸಂಘವನ್ನು ಮಾಡಿಕೊಳ್ಳುವುದೇ ತಡ, ಒಬ್ಬರನ್ನೊಬ್ಬರು ವಂಚಿಸಲು ಪ್ರಯತ್ನಿಸುತ್ತೇವೆ; ಸಂಘವು ಕುಸಿದು ಬಿದ್ದು ಚೂರಾಗುತ್ತದೆ. ಇಂಗ್ಲಿಷರನ್ನು ಅನುಕರಿಸಿ ಅವರಂತೆಯೇ ಮಹಾಸಾಮ್ರಾಜ್ಯವನ್ನು ನಿರ್ಮಾಣ ಮಾಡುವ ಮಾತನಾಡುತ್ತೀರಿ. ಆದರೆ ಅದಕ್ಕೆ ತಳಪಾಯವೆಲ್ಲಿದೆ? ನಮ್ಮ ತಳಪಾಯ ವೇನಿದ್ದರೂ ಕೇವಲ ಮರಳು. ಆದ್ದರಿಂದಲೆ ಅದರ ಮೇಲೆ ಕಟ್ಟಿದ್ದೆಲ್ಲವೂ ಕ್ಷಣಮಾತ್ರದಲ್ಲಿ ಕುಸಿದು ಭೂಗತವಾಗುತ್ತದೆ. ಆದ್ದರಿಂದ, ಲಾಹೋರಿನ ಯುವಕರೇ, ಅದ್ವೈತವೇದಾಂತದ ಅದ್ಭುತ ವೈಜಯಂತಿಯನ್ನು ಮತ್ತೊಮ್ಮೆ ಗಗನದೆತ್ತರಕ್ಕೆ ಎತ್ತಿಹಿಡಿಯಿರಿ. ಈ ವೇದಾಂತದ ಆಧಾರದ ಮೇಲೆ ನಿಂತಲ್ಲದೆ, ಎಲ್ಲರಲ್ಲೂ ಯಾವಾಗಲೂ ಏಕಮಾತ್ರ ಈಶ್ವರನನ್ನು ಕಾಣುವ ಸಮ ದರ್ಶಿತ್ವವೂ ಪ್ರೀತಿಯೂ ಉಂಟಾಗಲಾರದು. ಪ್ರೀತಿಯ ಧ್ವಜವನ್ನು ಬಿಚ್ಚಿ ಎತ್ತಿಹಿಡಿಯಿರಿ. ಉತ್ತಿಷ್ಠತ! ಜಾಗ್ರತ! ಪ್ರಾಪ್ಯವರಾನ್ನಿಬೋಧತ!... ಮೊದಲು ಅನ್ನ, ಆಮೇಲೆ ಅಧ್ಯಾತ್ಮ. ನಮ್ಮವರಿಗೆ ಹೊಟ್ಟೆಗಿಲ್ಲದಿದ್ದರೂ ಧರ್ಮವಂತೂ ಅಜೀರ್ಣವಾಗುವಷ್ಟಿದೆ. ನಮ್ಮಲ್ಲಿ ಎರಡು ಮಹಾ ಪಾತಕಗಳಿವೆ–ಒಂದು, ದೌರ್ಬಲ್ಯ; ಮತ್ತೊಂದು, ದ್ವೇಷ–ಶುಷ್ಕಹೃದಯತೆ. ನೀವು ಸಾವಿರ ಸಿದ್ಧಾಂತಗಳನ್ನು ಹರಡಬಹುದು, ಲಕ್ಷ ಪಂಥಗಳನ್ನು ಸೃಜಿಸಬಹುದು. ಆದರೆ ಎಲ್ಲಿಯ ವರೆಗೆ ನಿಮ್ಮಲ್ಲಿ ಇತರರಿಗಾಗಿ ಮರುಗುವ ಹೃದಯವಿಲ್ಲವೋ, ಅಲ್ಲಿಯವರೆಗೆ ನಿಮ್ಮ ಸಿದ್ಧಾಂತ ಬೋಧನೆಯೂ ಮತಪ್ರಚಾರವೂ ಕೆಲಸಕ್ಕೆ ಬರುವುದಿಲ್ಲ.”

“ಮಹನೀಯರೆ, ಅದ್ವೈತವೇದಾಂತದ ಅತ್ಯಮೂಲ್ಯ ಅಂಶಗಳಲ್ಲಿ ಕೆಲವನ್ನು ನಿಮ್ಮ ಮುಂದಿ ಡಲು ಪ್ರಯತ್ನಿಸಿದ್ದೇನೆ. ಅವುಗಳನ್ನು ಕಾರ್ಯತಃ ಪ್ರಯೋಗಿಸುವ ಸಮಯವೀಗ ಬಂದಿದೆ. ಈ ದೇಶದಲ್ಲಿ ಮಾತ್ರವಲ್ಲ, ಇತರ ದೇಶಗಳಲ್ಲಿಯೂ ಅದನ್ನೀಗ ಕಾರ್ಯಗತಗೊಳಿಸಬೇಕಾಗಿದೆ. ಆಧುನಿಕ ವಿಜ್ಞಾನದ ಪ್ರಚಂಡ ಆಘಾತಗಳಿಂದಾಗಿ, ದ್ವೈತವಾದದ ಪಿಂಗಾಣಿಯ ಅಸ್ತಿಭಾರದ ಮೇಲೆ ನಿಂತ ಧರ್ಮಗಳು ಎಲ್ಲೆಲ್ಲಿಯೂ ತತ್ತರಿಸುತ್ತಿವೆ. ದ್ವೈತಿಗಳು ಶಾಸ್ತ್ರಪಾಠಗಳನ್ನು ಜಗ್ಗಿ ತಿರುಚಿ ವಕ್ರವ್ಯಾಖ್ಯಾನಗಳನ್ನು ಹೇಳಲು ಯತ್ನಿಸುತ್ತಿರುವುದು ಈ ದೇಶದಲ್ಲಿ ಮಾತ್ರವೇ ಅಲ್ಲ. ಎಳೆದತ್ತ ಹಿಗ್ಗಲು ಶಾಸ್ತ್ರಗಳೇನು ರಬ್ಬರೇ? ಸ್ವಸಂರಕ್ಷಣಾರ್ಥವಾಗಿ ಈ ಮತಪ್ರಮುಖರು ವಿವಿಧೋಪಾಯಗಳನ್ನು ಕಂಡುಹಿಡಿಯುತ್ತಿರುವುದು ಇಲ್ಲಿ ಮಾತ್ರವೇ ಅಲ್ಲ. ಯೂರೋಪು ಅಮೆರಿಕಗಳಲ್ಲಿ ಆ ಕಾರ್ಯವು ಇನ್ನೂ ಭರದಿಂದ ಸಾಗುತ್ತಿದೆ. ಈ ಅದ್ವೈತವೇದಾಂತವು ಭರತ ಖಂಡದಿಂದ ಅಲ್ಲಿಗೂ ದಾಳಿಯಿಡಬೇಕು; ಈಗಾ.ಗಲೇ ಅದು ಅಲ್ಲಿ ಕಾಲಿಟ್ಟಿದೆ. ಆದರೆ ನಾವು ಹಿಡಿದ ಕೆಲಸವನ್ನು ಇನ್ನೂ ಮುಂದುವರಿಸಬೇಕು; ಅವರ ನಾಗರಿಕತೆಗಳನ್ನೂ ಉಳಿಸಬೇಕು. ಏಕೆಂದರೆ ಈಗ ಪಾಶ್ಚಾತ್ಯ ದೇಶಗಳಲ್ಲಿ ಪ್ರಾಚೀನ ಸಂಪ್ರದಾಯಗಳೆಲ್ಲ ಸಡಿಲವಾಗಿ ನವೀನ ಜೀವನ ರೀತಿಗಳು ಬೇರೂರುತ್ತಿವೆ. ಯಾವ ಜನಾಂಗವೇ ಆಗಲಿ ಅಂತಹ ತಳಹದಿಯ ಮೇಲೆ ಬಹಳ ಕಾಲ ನಿಲ್ಲಲಾರದು. ಅಂತಹ ತಳಹದಿಯ ಮೇಲೆ ಕಟ್ಟಿದ ನಾಗರಿಕತೆಗಳೆಲ್ಲ ಮಣ್ಣುಪಾಲಾ ಗಿವೆ ಎಂಬುದನ್ನು ಜಗತ್ತಿನ ಇತಿಹಾಸವು ಸಾರಿ ಹೇಳುತ್ತಿದೆ... ಪಾಶ್ಚಾತ್ಯ ರಾಷ್ಟ್ರಗಳ ವಿಲಾಸ- ವಿಭ್ರಮಗಳು ನಮ್ಮ ದೇಶಕ್ಕೆ ನುಗ್ಗದಂತೆ ತಡೆದು ಆಧುನಿಕ ವಿಜ್ಞಾನಶಾಸ್ತ್ರದ ಆಘಾತವನ್ನು ಎದುರಿಸಬಲ್ಲ ಅದ್ವೈತವೇದಾಂತವನ್ನು ಎಲ್ಲರಿಗೂ ಬೋಧಿಸಬೇಕು. ಅಷ್ಟೇ ಅಲ್ಲ, ಇತರ ರಾಷ್ಟ್ರ ಗಳಿಗೂ ನಾವು ನೆರವಾಗಬೇಕು. ಆದರೆ ಎಲ್ಲಕ್ಕಿಂತ ಹೆಚ್ಚಾಗಿ ಇಲ್ಲಿ ನಾವು ಮಾಡಬೇಕಾದ ಕೆಲಸ ಅಗಾಧವಾಗಿದೆ ಎಂಬುದನ್ನು ನಿಮಗೆ ನೆನಪಿಸಿಕೊಡುತ್ತಿದ್ದೇನೆ. ಮೊಟ್ಟಮೊದಲನೆಯದಾಗಿ, ಅಧೋಗತಿಗಿಳಿಯುತ್ತಿರುವ ಲಕ್ಷಾಂತರ ಭಾರತೀಯರನ್ನು ನಾವು ಕೈಹಿಡಿದು ಮೇಲೆತ್ತಬೇಕಾಗಿದೆ.”

ವೇದಾಂತ ಹಾಗೂ ಅದರ ಅನುಷ್ಠಾನವನ್ನು ಕುರಿತ ಈ ಸಾರ್ವಜನಿಕ ಭಾಷಣವು ಅತ್ಯಂತ ಪ್ರಭಾವಶಾಲಿಯಾಗಿತ್ತು. ಇದರ ಪರಿಣಾಮವಾಗಿ ಲಾಹೋರಿನ ಯುವಕರು ದೀನರಲ್ಲಿ ನಾರಾಯಣನನ್ನು ಕಂಡು ಸೇವೆ ಮಾಡುವುದಕ್ಕಾಗಿ ಕೂಡಲೇ ಒಂದು ಸಂಘವನ್ನು ನಿರ್ಮಾಣ ಮಾಡಿಕೊಂಡರು.

ಸ್ವಾಮೀಜಿಯವರು ಹಿಂದೂಧರ್ಮ ಹಿಂದೂಸಂಸ್ಕೃತಿಗಳನ್ನು ಪುನಸ್ಸಂಸ್ಥಾಪನೆ ಮಾಡಲು ಕಾರ್ಯಯೋಜನೆಗಳನ್ನು ಹಾಕಿಕೊಂಡಿದ್ದುದೇನೋ ನಿಜವೆ. ಆದರೆ ಆ ಕಾರಣದಿಂದಾಗಿ ಇತರ ಮತಗಳವರ, ಇತರ ಸಂಪ್ರದಾಯಗಳವರ ಭಾವನೆಗೆ ಧಕ್ಕೆಯುಂಟಾಗುವಂತೆ ಎಂದಿಗೂ ಮಾತ ನಾಡುತ್ತಿರಲಿಲ್ಲ. ಒಮ್ಮೆ ಕೆಲವು ಸಂಪ್ರದಾಯಸ್ಥ ಹಿಂದೂಗಳು ಆರ್ಯಸಮಾಜದ ವಿರುದ್ಧವಾಗಿ ಸಾರ್ವಜನಿಕ ಸಭೆಯಲ್ಲಿ ಮಾತನಾಡಬೇಕೆಂದು ಸ್ವಾಮೀಜಿಯವರನ್ನು ಆಗ್ರಹಪೂರ್ವಕವಾಗಿ ಕೇಳಿ ಕೊಂಡರು. ಸ್ವಾಮೀಜಿ ಅದಕ್ಕೆ ಸುತರಾಂ ಒಪ್ಪಲಿಲ್ಲ. ಆದರೆ ಅವರು ಶ್ರಾದ್ಧಕರ್ಮದ ವಿಷಯ ವಾಗಿ ಮಾತನಾಡಲು ಒಪ್ಪಿಕೊಂಡರು. ಆರ್ಯಸಮಾಜದವರು ಈ ಶ್ರಾದ್ಧಕರ್ಮಾದಿಗಳನ್ನೆಲ್ಲ ನಂಬುವವರಲ್ಲ. ಸ್ವಾಮೀಜಿ ಆ ವಿಷಯವಾಗೇನೋ ಮಾತನಾಡಿದರು; ಆದರೆ ಮಾತಿನ ಸಂದರ್ಭದಲ್ಲಿ ಆರ್ಯಸಮಾಜದವರನ್ನು ಸ್ವಲ್ಪವೂ ಟೀಕಿಸಲು ಹೋಗಲಿಲ್ಲ. ಅಂದಿನ ಭಾಷಣಕ್ಕೆ ಹಲವಾರು ಜನ ಆರ್ಯಸಮಾಜದ ಮುಖಂಡರು ಆಗಮಿಸಿದ್ದರು. ಈ ಸಂಪ್ರದಾಯಗಳ ಬಗ್ಗೆ ಸ್ವಾಮೀಜಿಯವರ ಅಭಿಪ್ರಾಯವೇನೆಂಬುದನ್ನು ತಿಳಿಯುವುದು ಅವರ ಉದ್ದೇಶ. ಸ್ವಾಮೀಜಿ ಶ್ರಾದ್ಧಕರ್ಮದ ಆವಶ್ಯಕತೆಯ ಬಗ್ಗೆ ವಿವರವಾಗಿ ಮಾತನಾಡಿದರು. ಆಗ ಆರ್ಯಸಮಾಜದ ಮುಖಂಡರು ಸ್ವಾಮೀಜಿಯವರ ಮಾತುಗಳನ್ನು ಒಪ್ಪದೆ ಶ್ರಾದ್ಧಕರ್ಮಗಳೆಲ್ಲ ಮೂಢನಂಬಿಕೆ ಯೆಂದು ವಾದಿಸಿದರು. ಆದರೆ ಸ್ವಾಮೀಜಿ ಸ್ವಲ್ಪವೂ ತಾಳ್ಮೆಗೆಡದೆ ಗೌರವಪೂರ್ಣವಾಗಿಯೇ ಆರ್ಯಸಮಾಜದವರ ವಾದವನ್ನು ಖಂಡಿಸಿದರು. ಅನಾದಿಕಾಲದಿಂದ ನಡೆದುಕೊಂಡು ಬರು ತ್ತಿರುವ ಈ ಶ್ರಾದ್ಧಕಾರ್ಯ ಅಥವಾ ಪಿತೃಪೂಜೆ ಎನ್ನುವುದೇ ಹಿಂದೂಧರ್ಮದ ಮೂಲ, ಧರ್ಮ ಶ್ರದ್ಧೆ ಪ್ರಾರಂಭವಾದದ್ದೇ ಅಲ್ಲಿಂದ ಎಂದು ಅವರು ವಿವರಿಸಿದರು. ‘ಮೊಟ್ಟಮೊದಲು ಹಿಂದೂ ಗಳು ತಮ್ಮ ಪಿತೃಗಳನ್ನು ಒಬ್ಬ ಮನುಷ್ಯನ ಮೇಲೆ ಆವಾಹನೆ ಮಾಡಿ ಆ ಮನುಷ್ಯನ ಮೂಲಕ ಪಿತೃಗಳನ್ನು ಪೂಜಿಸುತ್ತಿದ್ದರು, ಪಿಂಡಪ್ರದಾನ ಮಾಡುತ್ತಿದ್ದರು. ಆದರೆ ಈ ಮಧ್ಯವರ್ತಿಯಾದ ಮನುಷ್ಯನು ಶಾರೀರಿಕವಾಗಿ ಬಹಳ ಯಾತನೆಗಳನ್ನು ಅನುಭವಿಸುತ್ತಿದ್ದ. ಮೊದಮೊದಲು ಜನರಿಗೆ ಇದೇಕೆ ಹೀಗಾಗುತ್ತಿದೆ ಎನ್ನುವುದು ಅರ್ಥವಾಗಲಿಲ್ಲ. ಆದರೆ ಬರಬರುತ್ತ ಅವರಿಗೆ ಅರಿವಾಯಿತು –ಪಿತೃಗಳನ್ನು ಆವಾಹನೆ ಮಾಡಿದ್ದರಿಂದಲೇ ಈ ಮನುಷ್ಯನು ಯಾತನೆಗೆ ಗುರಿಯಾಗುತ್ತಿರು ವುದು ಎಂದು. ಆದ್ದರಿಂದ ಅವರು ದರ್ಭೆಯಿಂದ ಒಂದು ಆಕೃತಿಯನ್ನು ಮಾಡಿ (ಇದಕ್ಕೆ ‘ಕುಶ ಪುತ್ಥಳಿ’ ಎನ್ನುತ್ತಾರೆ) ಪಿತೃಗಳನ್ನು ಆವಾಹನೆ ಮಾಡಿ ಪೂಜೆ-ಪಿಂಡಪ್ರದಾನ ಮಾಡಲು ಪ್ರಾರಂಭಿ ಸಿದರು. ಈ ಪಿತೃಪೂಜೆಯೇ ಮುಂದೆ ವೈದಿಕವಾದಂತಹ ದೇವತಾರಾಧನೆಯ ರೂಪವನ್ನು ಪಡೆದು ಕೊಂಡಿತು’ ಎಂದು ಸ್ವಾಮೀಜಿ ವಿವರಿಸಿದರು.

ಪಂಜಾಬಿನಲ್ಲಿ ಸನಾತನ ಹಿಂದೂ ಸಂಪ್ರದಾಯದವರು ಎಷ್ಟು ಪ್ರಬಲರೋ ಆರ್ಯಸಮಾಜ ದವರೂ ಅಷ್ಟೇ ಬಲಿಷ್ಠರು. ಸನಾತನಿಗಳಾದ ಹಿಂದೂಗಳು ಅನಾದಿಯಿಂದ ಬಂದ ಹಿಂದೂ ಸಂಪ್ರದಾಯಗಳನ್ನು ಬಲವಾಗಿ ಹಿಡಿದುಕೊಂಡವರಾದರೆ ಆರ್ಯ ಸಮಾಜದವರು ಅದೇ ಹಿಂದೂ ತತ್ತ್ವಗಳಿಗೆ ಹೊಸ ವ್ಯಾಖ್ಯಾನ ಬರೆದು ಅದನ್ನು ಹೊಸ ರೀತಿಯಿಂದ ಆಚರಿಸುವವರು. ಈ ಎರಡು ಬಣಗಳ ನಡುವೆ ಅಲ್ಲಿ ನಿತ್ಯಕಲಹ. ಸ್ವಾಮೀಜಿ ಪಂಜಾಬಿನಲ್ಲಿದ್ದಾಗ ಈ ಕದನ ಕುತೂಹಲಿಗಳ ನಡುವೆ ರಾಜಿ ಮಾಡಿಸುವುದೇ ಅವರ ಮುಖ್ಯ ಕಾರ್ಯವಾಗಿತ್ತು. ಅವರು ಈ ಪ್ರಯತ್ನದಲ್ಲಿ ಸಾಕಷ್ಟು ಯಶಸ್ವಿಗಳೂ ಆದರು–ತತ್ಕಾಲಕ್ಕೆ! ಎರಡು ಪಂಗಡದವರೂ ಅವ ರನ್ನು ತುಂಬ ಗೌರವಾದರಗಳಿಂದ ನೋಡಿಕೊಂಡರು. ಎರಡು ಪಂಗಡದವರೂ ಅವರ ವಾಣಿ ಯನ್ನು ಕೇಳಲು ಬರುತ್ತಿದ್ದರು. ಸ್ವಾಮೀಜಿಯವರು ಆರ್ಯಸಮಾಜದವರ ಧ್ಯೇಯ ಧೋರಣೆಗಳ ವಿಷಯದಲ್ಲಿ ತುಂಬ ಉದಾರಭಾವವನ್ನು ತಾಳಿದ್ದರು. ಅಂತೆಯೇ ಆರ್ಯಸಮಾಜದವರೂ ಸ್ವಾಮೀಜಿಯವರನ್ನು ತುಂಬ ಪೂಜ್ಯಭಾವದಿಂದ ಕಾಣುತ್ತಿದ್ದರು. ಈ ಬಗೆಯ ಸಂಬಂಧ ಎಷ್ಟು ನಿಕಟವಾಗಿತ್ತೆಂದರೆ ಆರ್ಯಸಮಾಜದ ಹಲವಾರು ಮುಖಂಡರು ಸ್ವಾಮೀಜಿಯವರನ್ನು ತಮ್ಮ ಸಮಾಜದ ಪ್ರಧಾನ ನಾಯಕರಾಗಬೇಕೆಂದು ಬಯಸಿದ್ದಾರೆಂಬ ಬಲವಾದ ವದಂತಿ ಹರಡಿಕೊಂಡಿತು!

ಸ್ವಾಮೀಜಿಯವರು ಹಲವಾರು ಸಂದರ್ಭಗಳಲ್ಲಿ ಪಂಜಾಬಿಗಳ ಶುಷ್ಕ ಹೃದಯತೆಯನ್ನು ಟೀಕಿಸಿದರು–“ನಿಮ್ಮ ಹೃದಯದಲ್ಲಿ ಭಾವನೆ ಎನ್ನುವುದು ಬತ್ತಿಯೇ ಹೋಗಿರುವಂತಿದೆ. ಐದು ನದಿಗಳ ನಾಡಾದ ಪಂಜಾಬಿನಲ್ಲಿ ಆಧ್ಯಾತ್ಮಿಕತೆಯೆಂಬುದು ತುಂಬ ಶುಷ್ಕವಾಗಿದೆ. ಧರ್ಮದಲ್ಲಿ ಭಕ್ತಿಭಾವ ಎಂಬ ಒಂದು ಅಂಶವಿದೆ. ಈ ಭಕ್ತಿಭಾವವನ್ನು ನೀವು ಚೆನ್ನಾಗಿ ಬೆಳೆಸಿಕೊಳ್ಳಬೇಕು” ಎಂದು. ಪಂಜಾಬಿನಲ್ಲಿ ಬಂಗಾಳದ ಶ್ರೀಚೈತನ್ಯರ ಸಂಕೀರ್ತನೆಯನ್ನೊಂದಿಷ್ಟು ಕಲಿಸಿ ಪ್ರಚಲಿತ ಗೊಳಿಸಿದರೆ ಜನರ ಹೃದಯದಲ್ಲಿ ಭಕ್ತಿಭಾವ ಉದಿಸಬಹುದು ಎಂದು ಸ್ವಾಮೀಜಿ ಭಾವಿಸಿದರು.

ಸ್ವಾಮಿ ರಾಮತೀರ್ಥರೆಂದು ಮುಂದೆ ಪ್ರಸಿದ್ಧರಾದ ಶ್ರೀ ತೀರ್ಥರಾಮ ಗೋಸ್ವಾಮಿಯವರು ಲಾಹೋರಿನ ನಾಗರಿಕರು. ಇವರು ಅಲ್ಲಿನ ಒಂದು ಕಾಲೇಜಿನಲ್ಲಿ ಗಣಿತಶಾಸ್ತ್ರದ ಪ್ರಾಧ್ಯಾಪಕರಾಗಿ ದ್ದರು. ತೀರ್ಥರಾಮರು ಆಗಲೇ ಸ್ವಾಮೀಜಿಯವರ ಕೃತಿಗಳೆಲ್ಲವನ್ನೂ ಅಧ್ಯಯನ ಮಾಡಿದ್ದರು. ಅಲ್ಲದೆ ಲಾಹೋರಿನಲ್ಲಿ ಅವರ ಭಾಷಣಗಳನ್ನೆಲ್ಲ ಉತ್ಸಾಹದಿಂದ ಕೇಳಿದರು. ಸ್ವಾಮೀಜಿಯವರ ವ್ಯಕ್ತಿತ್ವದಿಂದ ಗಾಢವಾಗಿ ಪ್ರಭಾವಿತರಾಗಿದ್ದ ತೀರ್ಥರಾಮರು ತಮ್ಮ ಕಾಲೇಜಿಗೆ ಬರಮಾಡಿ ಕೊಂಡು ಒಂದು ಭಾಷಣವನ್ನೇರ್ಪಡಿಸಿದರು. ಅಲ್ಲದೆ ಸ್ವಾಮೀಜಿ ಹಾಗೂ ಅವರ ಶಿಷ್ಯವರ್ಗದವ ರನ್ನು ತಮ್ಮ ಮನೆಗೆ ಭೋಜನಕ್ಕೆ ಆಹ್ವಾನಿಸಿದರು. ಭೋಜನವಾದ ಮೇಲೆ ಸ್ವಾಮೀಜಿ ಒಂದು ಹಾಡನ್ನು ಹಾಡಿದರು–

\begin{verse}
ಜಹಾಂ ರಾಮ್ ವಹಾಂ ನಹಿ ಕಾಮ್ ।\\ಜಹಾಂ ಕಾಮ್ ತಹಾಂ ನಹಿ ರಾಮ್ ॥
\end{verse}

\noindent

ಎಂದರೆ–

\begin{verse}
ಎಲ್ಲಿ ರಾಮನಿರುವನೊ ಅಲ್ಲಿ ಕಾಮವಿಲ್ಲ,\\ಎಲ್ಲಿ ಕಾಮವಿರುವುದೊ ಅಲ್ಲಿ ರಾಮನಿಲ್ಲ.
\end{verse}

ಸ್ವಾಮೀಜಿ ಹಾಡಿದ ರೀತಿಯನ್ನು ಬಣ್ಣಿಸುತ್ತ ತೀರ್ಥರಾಮರು “ಅವರು ತಮ್ಮ ಸುಶ್ರಾವ್ಯವಾದ ಮಧುರ ಕಂಠದಿಂದ ಹಾಡಿದಾಗ ಆ ಹಾಡಿನ ಅರ್ಥವು ಅಲ್ಲಿದ್ದವರ ಹೃದಯದಲ್ಲಿ ಸ್ಫುರಿಸಿ ರೋಮಾಂಚನವನ್ನುಂಟುಮಾಡಿತು” ಎಂದು ಹೇಳಿದ್ದಾರೆ.

ಸ್ವಾಮೀಜಿಯವರು ತೀರ್ಥರಾಮ ಗೋಸ್ವಾಮಿಯವರ ವೈಯಕ್ತಿಕ ಗ್ರಂಥ ಭಂಡಾರವನ್ನು ಕಂಡು ಬಹಳ ಸಂತೋಷಪಟ್ಟರು. ತಮ್ಮಲ್ಲಿದ್ದ ಅಸಂಖ್ಯಾತ ಗ್ರಂಥಗಳ ಪೈಕಿ ಯಾವುದ ನ್ನಾದರೂ ತೆಗೆದುಕೊಳ್ಳುವಂತೆ ತೀರ್ಥರಾಮರು ಅವರನ್ನು ಕೇಳಿಕೊಂಡರು. ಆಗ ಸ್ವಾಮೀಜಿ, ವಾಲ್ಟ್ ವಿಟ್​ಮನ್ ಎಂಬವನು ಬರೆದ \eng{Leaves of Grass} ಎಂಬ ಗ್ರಂಥವನ್ನು ಮಾತ್ರ ಆರಿಸಿಕೊಂಡರು. ಈ ವಾಲ್ಟ್ ವಿಟ್​ಮನ್ನನನ್ನು ಸ್ವಾಮೀಜಿ ‘ಅಮೆರಿಕದ ಸಂನ್ಯಾಸಿ’ ಎಂದು ಕರೆಯುತ್ತಿದ್ದರು.

ಒಂದು ಸಂಜೆ ಸ್ವಾಮೀಜಿ ತಮ್ಮ ಸಂಗಡಿಗರೊಂದಿಗೆ ಲಾಹೋರಿನ ರಸ್ತೆಯಲ್ಲಿ ನಡೆದು ಕೊಂಡು ಹೋಗುತ್ತಿದ್ದರು. ಇವರ ಪೈಕಿ ಅವರ ಅನೇಕ ಸೋದರಸಂನ್ಯಾಸಿಗಳೂ ನಗರದ ಯುವಕರೂ ಪ್ರೊ ॥ ತೀರ್ಥರಾಮರೂ ಇದ್ದರು. ಇವರೆಲ್ಲ ಹಲವಾರು ಸಣ್ಣ ಪುಟ್ಟ ಗುಂಪು ಗಳಾಗಿ, ಬೇರೆ ಬೇರೆ ವಿಷಯಗಳ ಬಗ್ಗೆ ಮಾತನಾಡಿಕೊಳ್ಳುತ್ತ ಸಾಗುತ್ತಿದ್ದರು. ತೀರ್ಥರಾಮರಿದ್ದ ಗುಂಪಿನಲ್ಲಿ ಒಬ್ಬರು ಅವರನ್ನು ಸಾಂದರ್ಭಿಕವಾಗಿ ಕೇಳಿದರು, “ನಿಜವಾದ ಮಹಾತ್ಮರು ಎಂದರೆ ಯಾರು? ಅಂಥವರ ಲಕ್ಷಣವೇನು?” ಎಂದು. ಅದಕ್ಕುತ್ತರವಾಗಿ ತೀರ್ಥರಾಮರು ತಮ್ಮ ಅಭಿಪ್ರಾಯವನ್ನು ಹೇಳಿದರು, “ನೋಡಿ, ಯಾವನು ತನ್ನ ಅಹಮಿಕೆಯನ್ನು ಕರಗಿಸಿಕೊಳ್ಳುವುದರ ಮೂಲಕ ತನ್ನ ವೈಯಕ್ತಿಕತೆಯನ್ನೆ ಹೋಗಲಾಡಿಸಿಕೊಂಡು ಸರ್ವರ ಆತ್ಮವಾಗಿ ಬದುಕು ತ್ತಾನೆಯೋ ಅವನೇ ಆದರ್ಶ ಮಹಾತ್ಮ. ಯಾವುದೇ ಸ್ಥಳದಲ್ಲಿ ಬಿಸಿಲು ಹೆಚ್ಚಾಗಿ ಬಿದ್ದಿತೆಂದರೆ ಅಲ್ಲಿನ ಗಾಳಿ ಆ ಶಾಖವನ್ನು ಹೀರಿಕೊಂಡು ಬಿಸಿಯಾಗುತ್ತದೆ, ಹಗುರವಾಗುತ್ತದೆ; ತತ್ಪರಿ ಣಾಮವಾಗಿ ಅದು ಮೇಲೇರುತ್ತದೆ. ಹೀಗೆ ಗಾಳಿ ಮೇಲೇರಿದಾಗ ಅಲ್ಲಿ ಉಂಟಾದ ನಿರ್ವಾತವನ್ನು ತುಂಬಲು ಸುತ್ತಲಿನ ಗಾಳಿ ಅಲ್ಲಿಗೆ ನುಗ್ಗುತ್ತದೆ. ಹೀಗೆ ಇಡೀ ವಾತಾವರಣದಲ್ಲೆಲ್ಲ ಚಲನೆ ಯುಂಟಾಗುತ್ತದೆ. ಅಂತೆಯೇ ಒಬ್ಬ ಮಹಾತ್ಮನಾದವನು ತನ್ನ ಅಸಾಧಾರಣ ವ್ಯಕ್ತಿತ್ವದ ಪ್ರಭಾವ ದಿಂದ ಸಮಸ್ತ ರಾಷ್ಟ್ರದಲ್ಲೇ ಹೊಸ ಶಕ್ತಿ-ಸ್ಫೂರ್ತಿಗಳನ್ನು ಜಾಗೃತಗೊಳಿಸುತ್ತಾನೆ.”

ತೀರ್ಥರಾಮರ ಈ ಮಾತುಗಳನ್ನು ಗಮನಿಸಿದಾಗ, ಅವರು ಸ್ವಾಮಿ ವಿವೇಕಾನಂದರನ್ನು ಮನಸ್ಸಿನಲ್ಲಿಟ್ಟುಕೊಂಡೇ ಈ ಮಾತುಗಳನ್ನಾಡಿದ್ದಾರೆಂಬುದು ವ್ಯಕ್ತವಾಗುತ್ತದೆ. ಈ ಸಮಯ ದಲ್ಲಿ ಅನತಿ ದೂರದಲ್ಲಿ ಸಾಗುತ್ತಿದ್ದ ಸ್ವಾಮೀಜಿಯವರ ಗುಂಪಿನವರು ಮೌನವಾಗಿದ್ದುದರಿಂದ ಅವರ ಕಿವಿಗೂ ಈ ಮಾತುಗಳು ಬಿದ್ದುವು. ತಕ್ಷಣ ಸ್ವಾಮೀಜಿ ಇದ್ದಕ್ಕಿದ್ದಂತೆ ಅಲ್ಲಿಯೇ ನಿಂತು ದೃಢ ಸ್ವರದಲ್ಲಿ ಹೇಳಿದರು:

“ನನ್ನ ಗುರುಗಳಾದ ಶ್ರೀರಾಮಕೃಷ್ಣದೇವರು ಅಂಥವರಾಗಿದ್ದರು!”

ಇದೊಂದು ಸ್ವಾರಸ್ಯಕರ ಪ್ರಸಂಗ. ಸ್ವಾಮೀಜಿಯವರು ಶ್ರೀರಾಮಕೃಷ್ಣರನ್ನು ಕುರಿತು ಆಡಿದ ಮಾತು ಅವರ ಗುರುಭಕ್ತಿಯ ದ್ಯೋತಕ ಎಂದು ನಾವು ತಿಳಿಯಬಹುದು. ಆದರೆ ಅಷ್ಟೇ ಅಲ್ಲ; ಇಲ್ಲಿ ಗಮನಾರ್ಹವಾದ ಮತ್ತೊಂದು ಅಂಶವಿದೆ: ತೀರ್ಥರಾಮರು ಮಹಾತ್ಮರ ವಿಷಯವಾಗಿ ಹೇಳಿದುದು ಸ್ವಾಮೀಜಿಯವರನ್ನು ಉದ್ದೇಶಿಸಿ. ಅಲ್ಲದೆ, ಅಹಮಿಕೆಯನ್ನು ಸಂಪೂರ್ಣವಾಗಿ ಕರಗಿಸಿಕೊಂಡವರೇ ನಿಜವಾದ ಮಹಾತ್ಮರೆಂದೂ ಅವರು ಹೇಳುತ್ತಾರೆ. ಆದರೆ ಆ ಮಾತು ತಮ್ಮ ಗುರುದೇವನಿಗೆ ಅನ್ವಯವಾಗುತ್ತದೆಂದು ಹೇಳುತ್ತಿದ್ದಾರೆ ಸ್ವಾಮೀಜಿ! ಸ್ವಾಮೀಜಿಯವರ ಅಹಂ ಕಾರರಾಹಿತ್ಯವು ಇದರಿಂದ ಮತ್ತಷ್ಟು ಸ್ಪಷ್ಟವಾಯಿತು; ಅಲ್ಲದೆ ಅವರನ್ನು ಮಹಾತ್ಮರೆಂದು ಕರೆದ ತೀರ್ಥರಾಮರ ಮಾತಿಗೆ ಮತ್ತಷ್ಟು ಪುಷ್ಟಿ ದೊರಕಿದಂತಾಯಿತು!

ಸ್ವಾಮೀಜಿಯವರು ತಮ್ಮ ಆತಿಥ್ಯ ಸ್ವೀಕರಿಸಿ ಹೊರಟಾಗ ತೀರ್ಥರಾಮರು ತಮ್ಮ ವಿಶ್ವಾಸದ ಸಂಕೇತವಾಗಿ ಅವರಿಗೊಂದು ಸುಂದರವಾದ ಚಿನ್ನದ ಕೈಗಡಿಯಾರವನ್ನು ಉಡುಗೊರೆಯಾಗಿ ನೀಡಿದರು. ಅದನ್ನು ಸ್ವಾಮೀಜಿ ಸಂತೋಷದಿಂದ ಸ್ವೀಕರಿಸಿದರು; ಆದರೆ ಮರುಕ್ಷಣದಲ್ಲೇ ಆ ಕೈ ಗಡಿಯಾರವನ್ನು ತೀರ್ಥರಾಮರ ಕೋಟಿನ ಜೇಬಿನಲ್ಲಿಟ್ಟು, “ನನ್ನ ಈ ಕೈಗಡಿಯಾರವನ್ನು ನನ್ನದೇ ಆದ ಈ ಜೇಬಿನಲ್ಲಿ ಇಟ್ಟುಕೊಂಡಿರುತ್ತೇನೆ” ಎಂದರು. ಇದೊಂದು ಅತ್ಯಂತ ಅರ್ಥ ಪೂರ್ಣವಾದ ಘಟನೆ. ತೀರ್ಥರಾಮರು ಸ್ವಭಾವತಃ ಅದ್ವೈತ ಮನೋಭಾವದ ವ್ಯಕ್ತಿ. ಆದ್ದರಿಂದ ಸ್ವಾಮೀಜಿ ಈ ಸಂದರ್ಭದಲ್ಲಿ ಅದ್ವೈತವನ್ನು ಆಚರಿಸಿಯೇ ತೋರಿಸಿಕೊಟ್ಟರು.

ಸ್ವಾಮೀಜಿಯವರ ವ್ಯಕ್ತಿತ್ವದಿಂದ ತೀರ್ಥರಾಮರು ಎಷ್ಟು ಪ್ರಭಾವಿತರಾದರೆಂಬುದು ಅವರು ತಮ್ಮ ಪರಿಚಿತರಾದ ಪಂಡಿತ ದೀನದಯಾಳು ವ್ಯಾಖ್ಯಾನವಾಚಸ್ಪತಿ ಎಂಬವರಿಗೆ ಬರೆದ ಈ ಪತ್ರದಿಂದ ಮನವರಿಕೆಯಾಗುತ್ತದೆ:

\noindent

“ಶ್ರೀ ಮಹಾರಾಜ್ ಜೀ,

“ವಂದನೆಗಳು. ಇಲ್ಲಿ ಹತ್ತು ದಿನಗಳನ್ನು ಕಳೆದು ಸ್ವಾಮಿ ವಿವೇಕಾನಂದಜಿಯವರು ಡೆಹರಾ ಡೂನಿಗೆ ಹೊರಟರು. ಅವರು ಇಲ್ಲಿ ಮೂರು ಭಾಷಣಗಳನ್ನು ಮಾಡಿದರು. ಮೊದಲನೆಯದು ‘ಸಕಲ ಹಿಂದೂಗಳಿಗೂ ಸಮಾನವಾಗಿ ಅನ್ವಯಿಸುವ ತತ್ತ್ವಗಳು’ ಹಾಗೂ ಎರಡನೆಯದು ‘ಭಕ್ತಿ’ ಎಂಬ ವಿಷಯದ ಮೇಲೆ... ಮೂರನೆಯ ಭಾಷಣದ ವಿಷಯ ‘ವೇದಾಂತ’. ಇದರ ಮೇಲೆ ಅವರು ಎರಡೂವರೆ ಗಂಟೆಗಳ ಕಾಲ ಮಾತನಾಡಿದರು. ಸಭಿಕರು ಎಷ್ಟು ತನ್ಮಯರಾಗಿದ್ದರೆಂದರೆ ಮತ್ತು ಅಲ್ಲಿ ಎಂತಹ ವಾತಾವರಣ ನಿರ್ಮಾಣವಾಗಿತ್ತೆಂದರೆ, ಕಾಲದೇಶಗಳ ಪ್ರಜ್ಞೆಯೇ ಅಡಗಿಹೋಗಿತ್ತು! ಕೆಲವು ಸಲವಂತೂ ಶ್ರೋತೃಗಳು ತಮ್ಮನ್ನು ಪರಬ್ರಹ್ಮವಸ್ತುವಿನೊಂದಿಗೆ ತಾದಾತ್ಮ್ಯಗೊಳಿಸಿಕೊಂಡೇ ಆ ಮಾತುಗಳನ್ನು ಅರ್ಥಮಾಡಿಕೊಳ್ಳಬೇಕಾಗಿತ್ತು. ಆ ಮಾತುಗಳು ವ್ಯಕ್ತಿಯ ಅಭಿಮಾನ ಅಹಂಕಾರಗಳನ್ನು ಬೇರುಸಹಿತ ಕಿತ್ತೊಗೆಯುವಂತಿದ್ದುವು. ಒಟ್ಟಿನಲ್ಲಿ ಅದೊಂದು ಅಭೂತಪೂರ್ವ ಯಶಸ್ಸನ್ನು ಗಳಿಸಿದ ಭಾಷಣ. ಆ ಉಪನ್ಯಾಸವನ್ನು ಯಾರೇ ಕೇಳಿರಲಿ–ಅವರು ಇಂಗ್ಲಿಷರಾಗಿರಬಹುದು, ಕ್ರೈಸ್ತರಾಗಿರಬಹುದು, ಮುಸಲ್ಮಾನರಾಗಿರ ಬಹುದು, ಬ್ರಾಹ್ಮಸಮಾಜದವರಾಗಿರಬಹುದು ಅಥವಾ ಆರ್ಯಸಮಾಜದವರಾಗಿರಬಹುದು– ಅದು ಅವರೆಲ್ಲರ ಕಣ್ಣು ತೆರೆಸುವಂತಿತ್ತು.

“ಸ್ವಾಮೀಜಿಯವರು ಸಾರ್ವಜನಿಕ ಭಾಷಣಗಳನ್ನೇನೋ ಮಾಡಿದರು, ನಿಜ. ಆದರೆ ಅವರ ಜ್ಞಾನರಾಶಿಯ ಅಗಾಧತೆಯು ಅವರ ಸಂಭಾಷಣೆಗಳಲ್ಲಿ ವ್ಯಕ್ತವಾಗುವಂತೆ ಭಾಷಣಗಳಲ್ಲಿ ಪ್ರತಿಫಲಿಸುವುದಿಲ್ಲ. ಸ್ವಾಮೀಜಿಯವರು ಆರ್ಯಸಮಾಜದ ಹಾಗೂ ಬ್ರಾಹ್ಮ ಸಮಾಜದ ಮುಖಂಡರೊಂದಿಗೆ ಖಾಸಗಿಯಾಗಿ ಸಂಭಾಷಣೆ ನಡೆಸಿದ ಸಂದರ್ಭದಲ್ಲಿ ನಾನೂ ಹಾಜರಿದ್ದೆ. ಅವರುಗಳ ಪ್ರಶ್ನೆಗಳಿಗೆಲ್ಲ ಸ್ವಾಮೀಜಿ ಎಂತಹ ರಭಸದಿಂದ ಉತ್ತರಿಸುತ್ತಿದ್ದರೆಂದರೆ ಮತ್ತು ಅವರ ವಾದಗಳ ದೋಷಗಳನ್ನು ಅವರ ಮುಂದೆಯೆ ಹೇಗೆ ಬಗೆದು ತೋರಿಸುತ್ತಿದ್ದರೆಂದರೆ ಅವರು ಸುಮ್ಮನೆ ತಲೆ ತಗ್ಗಿಸಿಕೊಂಡು ಹಿಂದಿರುಗುತ್ತಿದ್ದರು. ಆದರೆ ಅತ್ಯಂತ ಸ್ವಾರಸ್ಯಕರ ವಿಷಯವೇನೆಂದರೆ ಸ್ವಾಮೀಜಿಯವರ ಮಾತುಗಳು ಯಾರ ಭಾವಕ್ಕೂ ಘಾಸಿಯುಂಟುಮಾಡು ತ್ತಿರಲಿಲ್ಲ. ಅತ್ಯಲ್ಪ ಸಮಯದಲ್ಲಿ ಸ್ವಾಮೀಜಿಯವರು ಅವರವರು ಪ್ರತಿಪಾದಿಸುವ ತತ್ತ್ವಗಳ ಅಸಮಂಜಸತೆಯನ್ನು ಅವರವರೇ ಒಪ್ಪಿಕೊಳ್ಳುವಂತೆ ಮಾಡಿಬಿಡುತ್ತಿದ್ದರು. ಸ್ವಾಮೀಜಿಯವರು ಪುರಾಣಗಳನ್ನೂ ಶ್ರಾದ್ಧಕರ್ಮಗಳನ್ನೂ ಮೂರ್ತಿಪೂಜೆಯನ್ನೂ ಸಾರ್ವಜನಿಕವಾಗಿ ಎತ್ತಿಹಿಡಿ ದರು. ಅಲ್ಲದೆ ಅವರು ಒಳ್ಳೆಯ ಪಂಡಿತರು ಸಹ. ವೇದಗಳ ಹೆಚ್ಚಿನ ಶ್ಲೋಕಗಳೆಲ್ಲ ಅವರಿಗೆ ಕಂಠಸ್ಥವಾಗಿವೆ. ಅವರು ಶಾರೀರಕ ಸೂತ್ರಗಳ ಮೇಲಿನ ಶಾಂಕರಭಾಷ್ಯ, ಶ್ರೀಭಾಷ್ಯ, ಮಾಧ್ವ ಭಾಷ್ಯಗಳನ್ನು ಅಧ್ಯಯನ ಮಾಡಿದ್ದಾರೆ. ಅವರು ವಲ್ಲಭಾಚಾರ್ಯರ ಅಣುಭಾಷ್ಯವನ್ನು ಅಧ್ಯ ಯನ ಮಾಡುವವರಿದ್ದಾರೆ. ಸಾಂಖ್ಯ ಮತ್ತು ಯೋಗಗಳಲ್ಲಂತೂ ಅವರಿಗೆ ಸಂಪೂರ್ಣ ಪ್ರಭುತ್ವವಿದೆ. ಇನ್ನು ಭಗವದ್ಗೀತೆಯ ಮೇಲೆ ವ್ಯಾಖ್ಯಾನ ನೀಡುವಲ್ಲಿ ಅವರು ಅತ್ಯಂತ ಸಮರ್ಥರು. ಜೊತೆಗೆ ಅವರು ತುಂಬ ಮಧುರವಾಗಿ ಹಾಡುತ್ತಾರೆ... ”

ಹೀಗೆ ಸ್ವಾಮಿ ವಿವೇಕಾನಂದರಿಂದ ಹೊಸ ಸ್ಫೂರ್ತಿಯೊಂದನ್ನು ಪಡೆದುಕೊಂಡ ತೀರ್ಥ ರಾಮರು ಸಕಾಲದಲ್ಲಿ ಸಂನ್ಯಾಸವನ್ನು ಸ್ವೀಕರಿಸಿ ಸ್ವಾಮಿ ರಾಮತೀರ್ಥರೆಂಬ ನಾಮಧೇಯದಿಂದ ಪ್ರಸಿದ್ಧರಾಗಿ, ವೇದಾಂತಪ್ರಸಾರ ಕಾರ್ಯದಲ್ಲಿ ನಿರತರಾದರು.

ಸ್ವಾಮೀಜಿ ಲಾಹೋರಿನಲ್ಲಿದ್ದಾಗ ಹೃದಯಸ್ಪರ್ಶಿಯಾದ ಒಂದು ಘಟನೆ ನಡೆಯಿತು. ಅವರ ಬಾಲ್ಯ ಸ್ನೇಹಿತನಾದ ಮೋತಿಲಾಲ್ ಬೋಸ್ ಎಂಬವನು ಸರ್ಕಸ್ ಕಂಪನಿಯೊಂದರ ಮಾಲಿಕ ನಾಗಿದ್ದ. ಸ್ವಾಮೀಜಿ ಲಾಹೋರಿಗೆ ಬಂದಿದ್ದಾಗ ಅಲ್ಲಿ ಮೋತಿಲಾಲನ ಸರ್ಕಸ್ ಪ್ರದರ್ಶನಗಳು ನಡೆಯುತ್ತಿದ್ದುವು. ಒಂದು ದಿನ ಈತ ಸ್ವಾಮೀಜಿಯವರನ್ನು ನೋಡಲು ಬಂದ. ಆದರೆ ಈಗ ಅವರನ್ನು ಹೇಗೆ ಮಾತನಾಡಿಸುವುದು ಎಂದು ಅವನಿಗೆ ತಿಳಿಯಲಿಲ್ಲ. ಏಕೆಂದರೆ, ಸ್ವಾಮೀಜಿ ಯವರೇನೋ ಅವನಿಗೆ ಚಿರಪರಿಚಿತರು, ಇಬ್ಬರೂ ಹಲವಾರು ವರ್ಷ ಜೊತೆಯಾಗಿ ಆಟವಾಡಿ ಕೊಂಡಿದ್ದವರು. ಆದರೆ ಇಡೀ ಜಗತ್ತಿಗೆ ಅವರೀಗ ಸ್ವಾಮಿ ವಿವೇಕಾನಂದರು! ಪ್ರತಿಯೊಬ್ಬರೂ ಅವರಿಗೆ ಪೂಜ್ಯತೆಯಿಂದ ಬಾಗಿ ನಮಿಸುತ್ತಿದ್ದಾರೆ! ಆದರೆ ತನಗೆ ಸುಪರಿಚಿತನಾದ ‘ನರೇಂದ್ರ’ ನಿಗೆ ತಾನು ನಮಸ್ಕರಿಸಿ “ಸ್ವಾಮೀಜಿ” ಎಂದು ಸಂಬೋಧಿಸಿದರೆ ಸರಿಯಾದೀತೆ ಎಂದು ಆಲೋಚಿಸಿದ ಮೋತಿ. ಕಡೆಗೆ ಅವನು ಹಿಂಜರಿಯುತ್ತ ಅವರನ್ನೇ ಕೇಳಿದ, “ನಾನೀಗ ನಿಮ್ಮನ್ನು ಹೇಗೆ ಕರೆಯಬೇಕು? ನರೇನ್ ಅಂತಲೇ? ಅಥವಾ ಸ್ವಾಮೀಜಿ ಅಂತಲೇ?” ಎಂದು. ಅದಕ್ಕೆ ಸ್ವಾಮೀಜಿ ತಕ್ಷಣ, “ಇದೇನು ಮೋತಿ, ನಿನಗೆ ತಲೆಗಿಲೆ ಕೆಟ್ಟಿಲ್ಲತಾನೆ? ನಾನು ನಿನ್ನ ಅದೇ ನರೇನ್, ನೀನು ನನ್ನ ಅದೇ ಮೋತಿ, ಅಲ್ಲವೆ!” ಎನ್ನುತ್ತಾ ವಿಶ್ವಾಸದಿಂದ ತಬ್ಬಿಕೊಂಡರು.

ಹೀಗೆ ಸ್ವಾಮೀಜಿ ವಿಶ್ವವಿಖ್ಯಾತರಾಗಿ ಪಾಶ್ಚಾತ್ಯ ರಾಷ್ಟ್ರಗಳಿಂದ ಹಿಂದಿರುಗಿದ್ದರೂ, ಮತ್ತು ಭಾರತದಲ್ಲೂ ಅವರಿಗೆ ವೈಭವದ ಸತ್ಕಾರ ಸಿಗುತ್ತಿದ್ದರೂ ತಮ್ಮ ಬಾಲ್ಯದ ಸ್ನೇಹಿತರನ್ನೂ ಹಳೆಯ ಪರಿಚಿತರನ್ನೂ ಅದೇ ವಿಶ್ವಾಸದಿಂದಲೇ ನೋಡಿಕೊಳ್ಳುತ್ತಿದ್ದರು; ಅದೇ ನರೇನನಾಗಿ ವ್ಯವಹರಿಸುತ್ತಿದ್ದರು. ಸ್ವಾಮೀಜಿಯವರ ಸ್ನೇಹಿತರು ಅವರ ವಿಶ್ವಾಸ-ಸ್ನೇಹಗಳಲ್ಲಿ ಯಾವುದೇ ಬಗೆಯ ವ್ಯತ್ಯಾಸವನ್ನೂ ಕಾಣುತ್ತಿರಲಿಲ್ಲ.

