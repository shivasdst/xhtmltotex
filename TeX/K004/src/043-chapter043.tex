
\chapter{ಅವತಾರಸಮಾಪ್ತಿ}

\noindent

ಮಾರ್ಚ್ ೧೧ರಂದು ಬೇಲೂರು ಮಠದಲ್ಲಿ ಶ್ರೀರಾಮಕೃಷ್ಣರ ಜಯಂತಿಯನ್ನು ವೈಭವದಿಂದ ಆಚರಿಸಲಾಯಿತು. ಇದರ ಅಂಗವಾಗಿ ಆಚರಿಸಲಾಗುವ ಸಾರ್ವಜನಿಕ ಸಮಾರಂಭ ನಡೆದದ್ದು ಮಾರ್ಚ್ ೧೬ರಂದು. ಆದರೆ ಸ್ವಾಮೀಜಿ ಮಾತ್ರ ತಮ್ಮ ಕೋಣೆಯಿಂದ ಹೊರಬರುವ ಸ್ಥಿತಿಯ ಲ್ಲಿರಲಿಲ್ಲ. ಮಠಕ್ಕೆ ಹಿಂದಿರುಗಿದಂದಿನಿಂದಲೂ ಅವರು ಹಾಸಿಗೆ ಹಿಡಿದಿದ್ದರು. ಅವರ ಕಾಲುಗಳು ಬಾತುಕೊಂಡಿದ್ದುವು. ಸ್ವಾಮೀಜಿಯವರ ಪರಿಸ್ಥಿತಿ ಗಂಭೀರವಾಗಿದೆ ಎನ್ನುವ ಸುದ್ದಿಯನ್ನು ಕೇಳಿ ಎಲ್ಲರ ಮನಸ್ಸೂ ಮಂಕಾಗಿತ್ತು. ಉತ್ಸವಕ್ಕೆ ಆಗಮಿಸಿದ ಸಹಸ್ರಾರು ಜನ ಸ್ವಾಮೀಜಿಯವರ ದರ್ಶನ ಮಾಡಬೇಕು, ಅವರ ವಾಣಿಯ ಶ್ರವಣ ಮಾಡಬೇಕು ಎಂದು ಬಯಸಿದ್ದರು. ಆದರೆ ಎಲ್ಲರಿಗೂ ನಿರಾಶೆಯೇ ಗತಿಯಾಯಿತು. ಅಂದು ಸ್ವಾಮೀಜಿ ಹಲವು ಸಲ ಆಲೋಚಿಸಿದ್ದರು– ಸಾರ್ವಜನಿಕರಿಗೆ ದರ್ಶನ ಕೊಡಬೇಕು ಎಂದು. ಆದರೆ ಅಂದು ಬೆಳಿಗ್ಗೆ ಬಂದ ನಾಲ್ಕಾರು ಭಕ್ತರೊಂದಿಗೆ ಮಾತನಾಡುವುದರಲ್ಲೇ ಬಳಲಿ ಹೋದರು. ಆದ್ದರಿಂದ ವಿಶ್ರಾಂತಿ ಪಡೆಯಲು ಇಚ್ಛಿಸಿ, ಬಾಗಿಲಲ್ಲಿ ಕಾವಲು ನಿಂತು ಯಾರನ್ನೂ ಒಳಗೆ ಬಿಡದಂತೆ ನೋಡಿಕೊಳ್ಳಲು ನಿರಂಜನಾ ನಂದರಿಗೆ ಹೇಳಿದರು. ಒಳಗೆ ಶರಚ್ಚಂದ್ರ ಮಾತ್ರ ಅವರ ಪರಿಚರ್ಯೆಗೆ ನಿಂತಿದ್ದ. ಇಂತಹ ಉತ್ಸವದ ದಿನದಲ್ಲಿ ಸ್ವಾಮೀಜಿ ಹೀಗೆ ಹಾಸಿಗೆ ಹಿಡಿಯುವಂತಾಯಿತಲ್ಲ ಎಂದು ಅವನ ಎದೆ ಮರುಗಿತು. ಸ್ವಾಮೀಜಿಯವರ ಸೂಕ್ಷ್ಮದೃಷ್ಟಿ ಅದನ್ನು ಗಮನಿಸಿತು. ಅವನಿಗೆ ಮೃದುದನಿಯಲ್ಲಿ ಹೇಳಿದರು, “ಚಿಂತೆ ಮಾಡುವುದರಿಂದೇನು ಪ್ರಯೋಜನ, ಮಗು? ಈ ಶರೀರ ಸವೆದು ಹೋಯಿತು. ಇದಿನ್ನು ಸಾಯಲೇಬೇಕು. ಆದರೆ, ನಿಮ್ಮೆಲ್ಲರಲ್ಲಿ ನಾನು ಕಿಂಚಿತ್ತಾದರೂ ನನ್ನ ಭಾವನೆಗಳನ್ನು ತುಂಬಲು ಸಮರ್ಥನಾಗಿದ್ದರೆ ನನ್ನ ಜೀವನ ವಿಫಲವಾಗಲಿಲ್ಲ ಎಂದುಕೊಳ್ಳು ತ್ತೇನೆ.”

ಈ ವೇಳೆಗೆ ಹೊರಗೆ ಉತ್ಸವ ಒಳ್ಳೇ ಉತ್ಸಾಹದಿಂದ ನಡೆದಿತ್ತು. ಸಂಕೀರ್ತನೆಯ ತಂಡದವರು ಮಧುರವಾಗಿ ಗಾನ ಮಾಡುತ್ತಿದ್ದರು. ಸ್ವಾಮೀಜಿ ಮೆಲ್ಲನೆದ್ದು ಕಿಟಕಿಯ ಬಳಿಗೆ ಬಂದು, ಕಿಟಕಿಯ ಸರಳನ್ನು ಹಿಡಿದುಕೊಂಡು ಹೊರಗೆ ನಿಂತಿದ್ದ ಸಹಸ್ರಾರು ಭಕ್ತರನ್ನು ಪ್ರೇಮಮಯ ನಯನ ಗಳಿಂದ ವೀಕ್ಷಿಸಿದರು. ಕೆಲ ನಿಮಿಷಗಳಲ್ಲೇ ಕುಳಿತುಕೊಳ್ಳಬೇಕಾಯಿತು–ಬಹಳ ಹೊತ್ತು ನಿಲ್ಲಲೂ ಶಕ್ತಿಯಿಲ್ಲ. ಬಳಿಕ ಅವರು ತಮ್ಮ ಶಿಷ್ಯನೊಡನೆ, ಯುಗಯುಗಗಳಲ್ಲಿ ಅವತಾರವೆತ್ತುವ ಭಗವಂತನ ಬಗ್ಗೆ ಮತ್ತು ಅವನ ಮೇಲೆ ಭಕ್ತಿ ತಾಳುವುದರಿಂದ ಆತ್ಮಜ್ಞಾನ ಪ್ರಾಪ್ತವಾಗುವ ಬಗ್ಗೆ ಮಾತನಾಡಿದರು; ಅವತಾರದ ಮಹಿಮೆಯೇನು, ಲಕ್ಷಗಟ್ಟಲೆ ಜೀವಿಗಳ ಅಜ್ಞಾನವನ್ನು ಹೋಗಲಾಡಿಸಿ ಅವರಿಗೆ ಮುಕ್ತಿ ದೊರಕಿಸಿಕೊಡಬಲ್ಲ ಅವತಾರ ಪುರುಷನ ಶಕ್ತಿಯೇನು ಎಂಬ ವಿಷಯವಾಗಿ ಹೇಳಿದರು. ಕೃಪೆಯ ಬಗ್ಗೆ ಅವರು ನೀಡಿದ ವಿವರಣೆ ತುಂಬ ಅದ್ಭುತವಾಗಿದೆ –“ಆತ್ಮಸಾಕ್ಷಾತ್ಕಾರವನ್ನು ಹೊಂದಿರುವವನು ಶಕ್ತಿಯ ಉಗ್ರಾಣದಂತಿರುತ್ತಾನೆ. ಅವನೊಂದು ಕೇಂದ್ರಬಿಂದುವಾಗಿ, ಅವನ ಸುತ್ತಮುತ್ತಲಿನ ವಲಯದಲ್ಲೆಲ್ಲ ಆ ಶಕ್ತಿ ಸ್ಪಂದಿಸುತ್ತಿರುತ್ತದೆ; ಯಾರು ಯಾರು ಆ ವಲಯದೊಳಗೆ ಬರುತ್ತಾರೋ ಅವರೆಲ್ಲ ಆ ಶಕ್ತಿಸ್ಪರ್ಶದಿಂದ ಪ್ರಭಾವಿತ ರಾಗುತ್ತಾರೆ. ಹೀಗೆ ಸಾಕಷ್ಟು ಆಧ್ಯಾತ್ಮಿಕ ಸಾಧನೆ ಮಾಡದಿರುವವರೂ ಕೂಡ ಆ ಮಹಾಪುರುಷನ ಅದ್ಭುತ ಆಧ್ಯಾತ್ಮಿಕ ಶಕ್ತಿಯನ್ನು ಹೀರಿಕೊಳ್ಳುವಂತಾಗುತ್ತದೆ. ಕೃಪೆಯೆಂದರೆ ಇದೇ.”

ಬಳಿಕ ಮಾತನ್ನು ಮುಂದುವರಿಸುತ್ತ ನುಡಿದರು–“ಶ್ರೀರಾಮಕೃಷ್ಣರನ್ನು ನೋಡಿದವರು ಧನ್ಯರೇ ಸರಿ. ನೀವೆಲ್ಲರೂ ಮುಂದೆ ಅವರ ದರ್ಶನವನ್ನು ಪಡೆಯುವಿರಿ. ಇಲ್ಲಿಯವರೆಗೆ ಬಂದಿ ರುವ ನೀವೆಲ್ಲರೂ ಅವರಿಗೆ ಅತಿ ಸಮೀಪದವರೇ ಸರಿ. ಶ್ರೀರಾಮಕೃಷ್ಣರೂಪದಿಂದ ಆಗಮಿಸಿದ ಅವರು ಯಾರೆಂಬುದನ್ನು ತಿಳಿಯಲು ಒಬ್ಬರೂ ಸಮರ್ಥರಾಗಿಲ್ಲ. ಅವರ ಅತಿ ಸಮೀಪದ ಭಕ್ತರಿಗೂ ಕೂಡ ಅವರು ಯಾರೆಂಬುದರ ಸರಿಯಾದ ಸುಳಿವು ಸಿಗಲಿಲ್ಲ. ಎಲ್ಲೋ ಕೆಲವರಿಗೆ ಮಾತ್ರ ಸ್ವಲ್ಪ ಸುಳಿವು ಸಿಕ್ಕಿದೆ. ಮಿಕ್ಕವರೆಲ್ಲ ಮುಂದೆ ಸಕಾಲದಲ್ಲಿ ತಿಳಿಯುತ್ತಾರೆ.”

ಕಾಯಿಲೆಯಿಂದಾಗಿ ಸ್ವಾಮೀಜಿಯವರು ಊಟಮಾಡುತ್ತಿದ್ದುದು ಅತ್ಯಲ್ಪ; ಅವರ ನಿದ್ರೆಯೂ ಬಹಳ ಕಡಿಮೆ. ಅಲ್ಲದೆ ಚಿಕಿತ್ಸೆಯ ಸಂಬಂಧವಾಗಿ ಇನ್ನಷ್ಟು ಕಟ್ಟುನಿಟ್ಟಿನ ಪಥ್ಯ. ಹೀಗಿದ್ದರೂ ಕೂಡ ಸ್ವಾಮೀಜಿಯವರ ಮುಖಮಂಡಲದ ತೇಜಸ್ಸು, ಕಣ್ಣುಗಳ ಹೊಳಪು ಸ್ವಲ್ಪವೂ ತಗ್ಗಿರ ಲಿಲ್ಲ. ಜೊತೆಗೆ ಅವರು ತಮ್ಮ ಕೆಲಸಕಾರ್ಯಗಳನ್ನೇನೂ ಕಡಿಮೆ ಮಾಡಲಿಲ್ಲ. ಅಲ್ಲದೆ ಅವರ ಬುದ್ಧಿಶಕ್ತಿಯಾಗಲಿ ಸ್ಮರಣಶಕ್ತಿಯಾಗಲಿ ಒಂದಿನಿತೂ ಕುಂದಿರಲಿಲ್ಲ. ಈ ದಿನಗಳಲ್ಲೇ ಒಮ್ಮೆ ಅವರ ಪ್ರಚಂಡ ಗ್ರಹಣಶಕ್ತಿಯನ್ನು ತೋರಿಸುವ ಒಂದು ಘಟನೆ ನಡೆಯಿತು. ಒಂದು ದಿನ ಅವರು ಶರಚ್ಚಂದ್ರನೊಂದಿಗೆ ಮಠದ ಗ್ರಂಥಾಲಯದಲ್ಲಿ ಅಡ್ಡಾಡುವಾಗ ಅಲ್ಲಿದ್ದ ಸುಪ್ರಸಿದ್ಧ ವಿಶ್ವಕೋಶ ‘ಎನ್​ಸೈಕ್ಲೋಪೀಡಿಯ ಬ್ರಿಟಾನಿಕ’ದ ಬೃಹತ್ ಸಂಪುಟಗಳನ್ನು ಕಂಡು ಶರಚ್ಚಂದ್ರ ಅಚ್ಚರಿಯಿಂದ, “ಇವುಗಳನ್ನು ಓದಿ ಮುಗಿಸಲು ಒಂದು ಜನ್ಮಸಾಲದು!” ಎಂದುದ್ಗರಿಸಿದ. ಆದರೆ ಸ್ವಾಮೀಜಿಯವರು ಅದಾಗಲೇ ಹತ್ತು ಸಂಪುಟಗಳನ್ನು ಓದಿ ಮುಗಿಸಿ ಈಗ ಹನ್ನೊಂದನೆ ಯದನ್ನು ಓದುತ್ತಿದ್ದ ವಿಷಯ ಅವನಿಗೆ ತಿಳಿಯದು. ಆದ್ದರಿಂದ ಅವರು, “ಏಕೆ ಸಾಧ್ಯವಿಲ್ಲ? ಈಗ ಮೊದಲ ಹತ್ತು ಸಂಪುಟಗಳಲ್ಲಿ ಯಾವ ವಿಷಯದ ಮೇಲಾದರೂ ಪ್ರಶ್ನೆ ಕೇಳು” ಎಂದು ಸವಾಲೊಡ್ಡಿದರು. ಶರಚ್ಚಂದ್ರ ದಂಗಾದ. ಆದರೂ ತನ್ನ ಕುತೂಹಲದ ಸಮಾಧಾನಕ್ಕಾಗಿ ಆ ಹತ್ತು ಸಂಪುಟಗಳಲ್ಲಿ ಅತ್ಯಂತ ಕಠಿಣ ವಿಷಯಗಳಿಗೆ ಸಂಬಂಧಿಸಿದಂತೆ ಕೆಲವು ಪ್ರಶ್ನೆಗಳನ್ನು ಕೇಳಿದ. ಸ್ವಾಮೀಜಿ ಅವುಗಳಿಗೆಲ್ಲ ಕರಾರುವಾಕ್ಕಾಗಿ ಉತ್ತರಿಸಿದರಲ್ಲದೆ ಕೆಲವೊಮ್ಮೆ ಪಂಕ್ತಿಪಂಕ್ತಿ ಗಳನ್ನೇ, ಪುಟ ಪುಟಗಳನ್ನೇ ಬಾಯಿಪಾಠ ಮಾಡಿದಂತೆ ಒಪ್ಪಿಸಿಬಿಟ್ಟರು. ತನ್ನ ಗುರುವಿನ ಈ ವಿಸ್ಮಯಾದ್ಭುತ ಗ್ರಹಣ ಶಕ್ತಿ ಸ್ಮರಣಶಕ್ತಿಗಳನ್ನು ಕಂಡು ದಿಗ್ಬ್ರಾಂತನಾದ ಶಿಷ್ಯ, “ಇದು ಮಾತ್ರ ಮನುಷ್ಯಶಕ್ತಿಗೆ ಮೀರಿದ್ದು” ಎಂದುದ್ಗರಿಸಿದ. ಆಗ ಅವರೆಂದರು, “ಇದರಲ್ಲಿ ಅಂತಹ ಪವಾಡ ವಾದದ್ದೇನೂ ಇಲ್ಲ. ಬ್ರಹ್ಮಚರ್ಯವನ್ನು ಕಟ್ಟುನಿಟ್ಟಾಗಿ ಪಾಲಿಸಿದಲ್ಲಿ ಒಮ್ಮೆ ಕೇಳಿದ್ದನ್ನು ಅಥವಾ ಓದಿದ್ದನ್ನು ನೆನಪಿಟ್ಟುಕೊಂಡು ವರ್ಷಾಂತರಗಳ ಮೇಲೆಯೂ ಹಾಗೆಹಾಗೆಯೇ ಹೇಳಲು ಸಾಧ್ಯ ವಾಗುತ್ತದೆ. ಈ ಬ್ರಹ್ಮಚರ್ಯದ ಅಭಾವದಿಂದಾಗಿ ನಮ್ಮ ಇಡೀ ರಾಷ್ಟ್ರವೇ ಶಕ್ತಿಯಲ್ಲಿ ಹಾಗೂ ಬುದ್ಧಿಯಲ್ಲಿ ಬಡವಾಗುತ್ತಿದೆ; ತನ್ನ ಪುರುಷತ್ವವನ್ನೇ ಕಳೆದುಕೊಳ್ಳುತ್ತಿದೆ.”

ಇನ್ನೊಂದು ದಿನ ಅವರು ಶರಚ್ಚಂದ್ರನೊಂದಿಗೆ ಸಂಭಾಷಿಸುತ್ತ ಸಾಂದರ್ಭಿಕವಾಗಿ ನುಡಿ ದರು, “ಸ್ತ್ರೀಯರಿಗಾಗಿ ಒಂದು ಮಠ ಸ್ಥಾಪನೆಯಾಗಬೇಕು ಎನ್ನುವುದು ನನ್ನ ಮನದಾಸೆ. ಇದು ಕಲ್ಕತ್ತದಲ್ಲೇ ಎಲ್ಲಾದರೂ ಗಂಗೆಯ ತೀರದಲ್ಲಿ ನಿರ್ಮಾಣಗೊಳ್ಳಬೇಕು. ಇದು ರಾಮಕೃಷ್ಣರ ಮಠದ ರೀತಿಯಲ್ಲೇ ನಡೆದುಕೊಂಡು ಬರಬೇಕು. ಶ್ರೀಮಾತೆಯವರು ಈ ಮಠದ ಮಾರ್ಗ ದರ್ಶಕರಾಗಿರಬೇಕು, ಹಾಗೂ ಅದರ ಆಧಾರಸ್ತಂಭದಂತಿರಬೇಕು. ಇಲ್ಲಿ ಬ್ರಹ್ಮಚಾರಿಣಿಯರಿಗೆ ಮತ್ತು ಶಿಕ್ಷಕಿಯರಿಗೆ ತರಬೇತಿ ನೀಡಿ ಅವರನ್ನು ಭಾರತದ ಸ್ತ್ರೀಸಮುದಾಯದ ಪುನರುದ್ಧಾರ ಕಾರ್ಯಕ್ಕೆ ನಿಯೋಜಿಸಬೇಕು. ಬಳಿಕ, ಇಂತಹ ಬ್ರಹ್ಮಚಾರಿಣಿಯರನ್ನೂ ಸೇವಾವ್ರತಿಯರನ್ನೂ ತರಬೇತಿಗೊಳಿಸುವಂತಹ ಸ್ತ್ರೀಮಠಗಳು ಭಾರತದ ಹಲವೆಡೆಗಳಲ್ಲಿ ನಿರ್ಮಾಣಗೊಳ್ಳಬೇಕು ಮತ್ತು ತನ್ಮೂಲಕ ರಾಷ್ಟ್ರದ ಎಲ್ಲೆಡೆಯಲ್ಲೂ ಸ್ತ್ರೀವಿದ್ಯಾಭ್ಯಾಸ ಕೇಂದ್ರಗಳ ಉಗಮವಾಗಬೇಕು. ಹೀಗೆ ಮಾಡುತ್ತ ಹೋದರೆ ಕಾಲಕ್ರಮದಲ್ಲಿ ಎಂತಹ ಅದ್ಭುತಗಳು ನಡೆಯುವುವು ಬಲ್ಲೆಯಾ?”

ಶರಚ್ಚಂದ್ರನೊಂದಿಗೆ ಅವರು ಈ ಧಾಟಿಯಲ್ಲಿ ಮಾತನಾಡಿದರೆ, ಮಠದ ನೂತನ ಸಾಧು- ಬ್ರಹ್ಮಚಾರಿಗಳಿಗೆ ಆಧ್ಯಾತ್ಮಿಕ ಜೀವನದ ಸ್ವರೂಪವನ್ನು ಬಣ್ಣಿಸುತ್ತಿದ್ದರು. ಶಾಸ್ತ್ರಗ್ರಂಥಗಳ ಮೇಲೆ ತರಗತಿಗಳನ್ನು ನಡೆಸುತ್ತಿದ್ದರು. ಮುಮುಕ್ಷುಗಳಾದ ಸಂದರ್ಶಕರು ಬಂದರೆ ಸಾಧನಾ ವಿಧಾನಗಳನ್ನು ಬೋಧಿಸುತ್ತಿದ್ದರು. ಆದರೆ ಕ್ಷೀಣಗೊಂಡ ಅವರ ಶರೀರ ಇಷ್ಟೊಂದು ಪ್ರಯಾಸ ವನ್ನು ತಡೆದುಕೊಂಡೀತೇ?–ಇದು ಅವರ ಸಹಸಂನ್ಯಾಸಿಗಳ ಚಿಂತೆ, ಆತಂಕ. ಆದ್ದರಿಂದ ಇವರು ಸಂದರ್ಶಕರ ಭೇಟಿಯನ್ನು ಸೀಮಿತಗೊಳಿಸಿದರು. ಆದರೆ ಕೆಲವು ಪ್ರಾಮಾಣಿಕ ಸಾಧಕ ರನ್ನೂ ತಮ್ಮ ಬಳಿಗೆ ಬರದಂತೆ ತಡೆಯುತ್ತಿದ್ದಾರೆಂದು ತಿಳಿದುಬಂದಾಗ ಸ್ವಾಮೀಜಿ ತುಂಬ ಕಳಕಳಿಯಿಂದ ಹೇಳಿದರು, “ಇಲ್ಲಿ ನೋಡಿ, ಈ ಶರೀರದಿಂದ ಏನು ಪ್ರಯೋಜನವಿದೆಯೆಂದು ತಿಳಿದುಕೊಂಡಿರಿ? ಇತರರಿಗೆ ನೆರವಾಗುತ್ತಲೇ ಅದು ಹೋಗಲಿ, ನಮ್ಮ ಗುರು ಶ್ರೀರಾಮಕೃಷ್ಣರು ಕೊನೆಯುಸಿರಿನವರೆಗೂ ಬೋಧಿಸಲಿಲ್ಲವೆ? ಈಗ ನಾನೂ ಹಾಗೆಯೆ ಮಾಡಬೇಡವೆ? ಈ ಶರೀರ ಹೋದರೆ ಹೋಗಲಿ ಬಿಡಿ. ಅದಕ್ಕೆ ಯಾರು ಚಿಂತಿಸುತ್ತಾರೆ! ಪ್ರಾಮಾಣಿಕ ಸಾಧಕರು ಸಿಕ್ಕಾಗ ನನಗೆಷ್ಟು ಸಂತೋಷವಾಗುತ್ತದೆ ಎಂಬುದನ್ನು ನೀವು ಊಹಿಸಲೂ ಆರಿರಿ. ಸಹಮಾನವರಲ್ಲಿ ಆತ್ಮಜಾಗೃತಿಯನ್ನುಂಟುಮಾಡಲು ನಾನು ಎಷ್ಟು ಸಲ ಬೇಕಾದರೂ ಸಂತೋಷದಿಂದ ಪ್ರಾಣ ಕೊಡಬಲ್ಲೆ.”

ಮಾರ್ಚ್ ಅಂತ್ಯದಲ್ಲಿ ಕ್ರಿಸ್ಟೀನ ಮುಂಬಯಿಗೆ ಬಂದು ತಲುಪಿದಳು. ಸ್ವಾಮೀಜಿಯವರು ಅವಳಿಗೆ ಪತ್ರ ಬರೆದು ಅವಳನ್ನು ಸ್ವಾಗತಿಸಿದ ಬಗೆಯಲ್ಲಿ ಆತ್ಮೀಯತೆಯ ಅತಿಶಯತೆಯನ್ನು ಕಾಣಬಹುದು: “ನಮ್ಮ ಕಡೆಯಿಂದ ನಿನಗೆಂಥ ಸ್ವಾಗತವಿದೆ ಎನ್ನುವುದನ್ನು ನಾನು ಪ್ರತ್ಯೇಕವಾಗಿ ಬರೆಯಬೇಕಾಗಿಲ್ಲ. ನೇರವಾಗಿ ಬಂದುಬಿಡು. ಆದರೆ ಬಿಸಿಲು ಬಹಳವಾದ್ದರಿಂದ ಎಚ್ಚರಿಕೆ ಯಿಂದಿರು; ತಲೆಯ ಹಿಂಭಾಗವನ್ನು ಮುಚ್ಚಿಕೊಳ್ಳಬೇಕು. ಆಯಾಸವಾಗಿದ್ದರೆ ಮುಂಬಯಿ ಯಲ್ಲಿ ಸ್ವಲ್ಪ ವಿಶ್ರಮಿಸಿಕೊ. ಶ್ರೀಮತಿ ಸಾರಾಬುಲ್, ಜೋ (ಮಿಸ್ ಮೆಕ್​ಲಾಡ್​) ಮತ್ತು ಮ್ಯಾರ್ಗಟ್ (ನಿವೇದಿತಾ) ನಿನ್ನ ಬರವಿಗಾಗಿ ಕಾತರರಾಗಿದ್ದಾರೆ; ಹಾಗೆಯೇ ವಿವೇಕಾನಂದ ಕೂಡ.”

ಏಪ್ರಿಲ್ ೭ರಂದು ಕ್ರಿಸ್ಟೀನ ಕಲ್ಕತ್ತವನ್ನು ತಲುಪಿ ಇತರ ಪಾಶ್ಚಾತ್ಯ ಶಿಷ್ಯೆಯರೊಂದಿಗೆ ಉಳಿದುಕೊಂಡಳು. ಮರುದಿನ ಬೆಳಿಗ್ಗೆ ಶ್ರೀಮತಿ ಸಾರಾ ಬುಲ್​ಳೊಂದಿಗೆ ಸ್ವಾಮೀಜಿಯವರ ದರ್ಶನಕ್ಕಾಗಿ ಮಠಕ್ಕೆ ಬಂದು ಇಡೀ ದಿನ ಅಲ್ಲೇ ಉಳಿದಳು. ಅವಳ ವಿಷಯದಲ್ಲಿ ಸ್ವಾಮೀಜಿ ಯವರು ಅಷ್ಟೊಂದು ಆಸಕ್ತಿ ವಹಿಸಿದ್ದಕ್ಕೆ ಒಂದು ಕಾರಣವೆಂದರೆ ಅವಳ ಸೌಜನ್ಯಪೂರ್ಣ ಸ್ವಭಾವ ಹಾಗೂ ಸ್ಫಟಿಕದಷ್ಟು ಶುದ್ಧ ಶೀಲ. ಸೋದರಿ ನಿವೇದಿತಾ ಹೇಳುವಂತೆ \eng{‘Her character is radiantly beautiful’}.

ಮೇ ತಿಂಗಳ ೫ರಂದು ಒಕಾಕುರ, ನಿವೇದಿತಾ, ಕ್ರಿಸ್ಟೀನ, ಸ್ವಾಮಿ ಸದಾನಂದರು, ಮತ್ತಿತರರು ಮಾಯಾವತಿಗೆ ಹೊರಟರು. ಕ್ರಿಸ್ಟೀನಳ ಎಲ್ಲ ದಾರಿ ಖರ್ಚನ್ನು ವಹಿಸಿಕೊಂಡವರು ಸ್ವಾಮೀಜಿ.

ಸ್ವಾಮೀಜಿಯವರು ಮಠದಲ್ಲಿ ಬಗೆಬಗೆಯ ಕಾರ್ಯಕಲಾಪಗಳಲ್ಲಿ ನಿರತರಾಗಿರುತ್ತಿದ್ದರು. ಸೌಖ್ಯವೋ ಅಸೌಖ್ಯವೋ, ಏನಾದರೂ ಮಾಡುತ್ತಲೇ ಇರುತ್ತಿದ್ದರು. ಅಷ್ಟೇಕೆ, ಅವರು ಶರೀರ ತ್ಯಾಗ ಮಾಡುವ ದಿವಸದವರೆಗೂ ಮಠದಲ್ಲಿ ಅನೇಕ ತರಗತಿಗಳನ್ನು ತೆಗೆದುಕೊಂಡು ಶಾಸ್ತ್ರ ಗಳನ್ನು ವ್ಯಾಖ್ಯಾನಿಸುತ್ತಿದ್ದರು. ಮತ್ತು ಮಠದ ಬ್ರಹ್ಮಚಾರಿಗಳು ಮಾತ್ರವಲ್ಲದೆ ಅವರ ಗುರು ಭಾಯಿಗಳೂ ಕೂಡ ಸ್ವಾಮೀಜಿಯವರಿಂದ ಆಧ್ಯಾತ್ಮಿಕ ಜೀವನದ ಬಗ್ಗೆ ಸಲಹೆಗಳನ್ನು ಪಡೆಯು ತ್ತಿದ್ದರು. ಸ್ವಾಮೀಜಿಯವರು ಧ್ಯಾನದ ಹಲವಾರು ಪ್ರಕ್ರಿಯೆಗಳ ಕುರಿತಾಗಿ, ಹಲವಾರು ಧ್ಯಾನ ಕ್ರಮಗಳ ಕುರಿತಾಗಿ ವಿವರಣೆ ನೀಡುತ್ತಿದ್ದರು. ಮತ್ತು ಯಾರು ಯೋಗಾಭ್ಯಾಸದಲ್ಲಿ ಹಿಂದೆ ಬಿದ್ದಿರುವರೋ ಅಂಥವರಿಗೆ ವಿಶೇಷ ನೆರವು ನೀಡುತ್ತಿದ್ದರು. ಪತ್ರವ್ಯವಹಾರದಲ್ಲೋ, ಓದುವು ದರಲ್ಲೋ, ಹಿಂದೂ ತತ್ತ್ವಶಾಸ್ತ್ರಗಳ ವಿಷಯವಾಗಿ ಟಿಪ್ಪಣಿ ಮಾಡಿಕೊಳ್ಳುತ್ತಲೋ ಭಾರತೀಯ ಇತಿಹಾಸವನ್ನು ಪ್ರಕಾಶನಕ್ಕೆ ತರುವ ಕಾರ್ಯದಲ್ಲೋ ಅವರು ಗಂಟೆಗಟ್ಟಲೆ ಕಳೆಯುತ್ತಿದ್ದರು. ನಡುನಡುವೆ ಸ್ವಸಂತೋಷಕ್ಕಾಗಿ ಹಾಡುತ್ತಿದ್ದರು. ಅಥವಾ ಗುರುಭಾಯಿಗಳೊಂದಿಗೆ ಸಲ್ಲಾಪ ನಡೆಸುತ್ತ ನಗುವಿನ ಹೊಳೆ ಹರಿಸುತ್ತಿದ್ದರು. ಹೆಚ್ಚಿನ ಸಂದರ್ಭಗಳಲ್ಲಿ ಅವರು ತಮ್ಮ ಸಲ್ಲಾಪಗಳ ನಡುವೆ ಇದ್ದಕ್ಕಿದ್ದಂತೆ ಗಂಭೀರವಾಗಿಬಿಡುತ್ತಿದ್ದರು; ಅವರ ದೃಷ್ಟಿ ದೂರದ ಭವಿಷ್ಯದಲ್ಲಿ ಕೇಂದ್ರಿತವಾದಂತೆ ಕಂಡುಬರುತ್ತಿತ್ತು. ‘ಈಗ ಅವರು ಏಕಾಂಗಿಯಾಗಿರಲು ಇಷ್ಟಪಡುತ್ತಾರೆ’ ಎನ್ನುವುದನ್ನು ಮನಗಂಡು ಗುರುಭಾಯಿಗಳು ಅಲ್ಲಿಂದ ಮೆಲ್ಲಗೆ ಸರಿದುಕೊಳ್ಳುತ್ತಿದ್ದರು.

ಮಠದಲ್ಲಿನ ಆಗುಹೋಗುಗಳನ್ನು ಹಾಗೂ ಆಶ್ರಮವಾಸಿಗಳ ಚಲನವಲನಗಳನ್ನು ಸ್ವಾಮೀಜಿ ಯವರ ಕಣ್ಣುಗಳು ಸೂಕ್ಷ್ಮವಾಗಿ ಗಮನಿಸುತ್ತಿದ್ದುವು. ಈ ದಿನಗಳಲ್ಲಿ ಅವರು ಶಿಸ್ತಿನ ವಿಷಯ ದಲ್ಲಿ ಅತ್ಯಂತ ಕಟ್ಟುನಿಟ್ಟಿನಿಂದಿದ್ದುಬಿಟ್ಟರು. ಮಠದ ಕಾರ್ಯಗಳಲ್ಲಿ ಯಾವುದೇ ಆಗಲಿ ‘ಅತಿ’ ಯಾಗಬಾರದು ಎನ್ನುವುದು ಅವರ ಭಾವನೆ. ಮಠದಲ್ಲಿ ಪ್ರತಿದಿನದ ಪೂಜೆಯ ಕೆಲಸವನ್ನು ಅತಿ ಯಾಗಿ ಬೆಳೆಸಿಕೊಂಡು ಹೋಗುವುದನ್ನು ಅವರು ಖಡಕ್ಕಾಗಿ ಪ್ರತಿಭಟಿಸುತ್ತಿದ್ದರು. ತಮ್ಮ ಶಿಷ್ಯರೆಲ್ಲ ಹೆಚ್ಚು ಹೆಚ್ಚು ಸಮಯವನ್ನು ಶಾಸ್ತ್ರಾಧ್ಯಯನಕ್ಕೆ, ಸ್ವಾಧ್ಯಾಯ-ಪ್ರವಚನಗಳಿಗೆ ಹಾಗೂ ಧ್ಯಾನಾಭ್ಯಾಸಕ್ಕೆ ವಿನಿಯೋಗಿಸುವಂತಾಗಬೇಕು ಎನ್ನುವುದು ಸ್ವಾಮೀಜಿಯವರ ಅಪೇಕ್ಷೆ. ಈ ಮೂಲಕ ಅವರೆಲ್ಲ ತಮ್ಮತಮ್ಮ ಆಧ್ಯಾತ್ಮಿಕ ಜೀವನವನ್ನು ರೂಪಿಸಿಕೊಳ್ಳಬೇಕು ಮತ್ತು ಶ್ರೀರಾಮಕೃಷ್ಣರ ಜೀವನ-ಸಂದೇಶಗಳನ್ನು ಯಥಾರ್ಥವಾಗಿ ಅರಿಯುವ ಯತ್ನ ಮಾಡಬೇಕು; ಅದರ ಬದಲಾಗಿ, ಪೂಜಾಕಾರ್ಯದ ಸಣ್ಣಪುಟ್ಟ ವಿವರಗಳ ಕಡೆಗೆಲ್ಲ ಗಮನ ಕೊಡುತ್ತ ಸಮಯ ನಷ್ಟ ಮಾಡಬಾರದು ಎನ್ನುವುದು ಅವರ ಅಭಿಮತ. ಪೂಜೆ ಎನ್ನುವುದು ಭಕ್ತಿ-ಶ್ರದ್ಧೆಗಳಿಂದ ಕೂಡಿರಬೇಕೆ ಹೊರತು ಸಂನ್ಯಾಸಿಗಳ ಸಮಯವನ್ನೆಲ್ಲ ನುಂಗಿಹಾಕುವಷ್ಟು ಸಂಭ್ರಮಕ್ಕೆ ಹೋಗ ಬಾರದು; ಅಲ್ಲದೆ ಪೂಜೆಯೊಂದಿಗೆ ಶಾಸ್ತ್ರಾಧ್ಯಯನ ಹಾಗೂ ಜಪಧ್ಯಾನಗಳೂ ನಡೆಯಬೇಕು ಎನ್ನುವುದು ಸ್ವಾಮೀಜಿಯವರ ನಿಶ್ಚಿತ ನಿಲುವು. ಇದನ್ನು ಜಾರಿಗೆ ತರಲೋಸುಗ ಗಂಟೆ ಬಾರಿಸುವ ವ್ಯವಸ್ಥೆ ಮಾಡಿಸಿದ್ದರು. ಗಂಟೆ ಬಾರಿಸಿದ ಕೂಡಲೇ, ಯಾರ್ಯಾರು ಯಾವಯಾವ ಕೆಲಸದಲ್ಲಿದ್ದರೆ ಅದನ್ನೆಲ್ಲ ಹಾಗೆಹಾಗೆಯೇ ಬಿಟ್ಟು ಶಾಸ್ತ್ರಾಭ್ಯಾಸದ ತರಗತಿಗೋ, ಚರ್ಚಾಗೋಷ್ಠಿಗೋ ಅಥವಾ ಧ್ಯಾನಕ್ಕೋ ಹೊರಟುಬಿಡಬೇಕಾಗಿತ್ತು. ಈ ನಿಯಮವನ್ನು ಮೀರಿದವರಿಗೆ ಕಠಿಣ ಶಿಕ್ಷೆ ಕಾದಿರು ತ್ತಿತ್ತು. ಆದ್ದರಿಂದಲೇ ಸ್ವಾಮೀಜಿಯವರು ಶಿಷ್ಯರ ಪಾಲಿಗೆ ಅತ್ಯಂತ ಮೃದು, ಆದರೆ ಅಷ್ಟೇ ಕಠಿಣ ಗುರು. ಆದ್ದರಿಂದಲೇ ಅವರನ್ನು ಕಂಡರೆ ಪ್ರೀತಿ, ಭಕ್ತಿ; ಹಾಗೆಯೇ ಅವರನ್ನು ಕಂಡರೆ ನಡುಕ, ಭಯ ಕೂಡ. ಸ್ವಾಮೀಜಿಯವರು ಮಠದಲ್ಲಿ ಇದ್ದಾಗಲೆಲ್ಲ, ಅದರಲ್ಲೂ ತಮ್ಮ ಜೀವಿತದ ಕೊನೆಯ ಕೆಲವು ತಿಂಗಳುಗಳಲ್ಲಿ, ಧ್ಯಾನಕ್ಕೆ ವಿಶೇಷ ಒತ್ತು ಕೊಡುತ್ತಿದ್ದರು. ಅವರು ಶರೀರ ಬಿಡುವುದಕ್ಕೆ ಮೂರು ತಿಂಗಳಿರುವಾಗ ಒಂದು ನಿಯಮ ಮಾಡಿದರು–ಆ ಪ್ರಕಾರ ಬೆಳಿಗ್ಗೆ ನಾಲ್ಕು ಗಂಟೆಗೆ ಸರಿಯಾಗಿ ಪ್ರತಿಯೊಬ್ಬನ ಕೋಣೆಯ ಮುಂದೆಯೂ ಕೈಗಂಟೆ ಬಾರಿಸ ಲಾಗುತ್ತಿತ್ತು; ಇದಾದ ಅರ್ಧಗಂಟೆಗೆಲ್ಲ ಎಲ್ಲರೂ ಧ್ಯಾನಕ್ಕಾಗಿ ಪ್ರಾರ್ಥನಾ ಮಂದಿರದಲ್ಲಿ ನೆರೆಯಬೇಕಾಗಿತ್ತು. ಇವೆಲ್ಲದರೊಂದಿಗೆ ಅವರು ತಮ್ಮ ಶಿಷ್ಯರನ್ನು ತಪಸ್ಸನ್ನಾಚರಿಸುವಂತೆ ಪ್ರೋತ್ಸಾಹಿಸುತ್ತಿದ್ದರು. ಆಶ್ರಮವಾಸಿಗಳಿಗೆ ಎಂದೆಂದಿಗೂ ಮಾರ್ಗದರ್ಶನ ನೀಡವುದಕ್ಕಾಗಿ ಅವರು ಹಲವಾರು ನೀತಿನಿಯಮಗಳನ್ನೊಳಗೊಂಡ ಪುಟ್ಟ ಪುಸ್ತಕವೊಂದನ್ನು ರಚಿಸಿದರು. ಅದರಲ್ಲಿ ಅವರ ಪ್ರಧಾನ ಭಾವನೆಗಳು ಹಾಗೂ ಬೋಧನೆಗಳು ಮತ್ತು ಅವುಗಳನ್ನು ಕಾರ್ಯ ರೂಪಕ್ಕೆ ತರಲು ಬೇಕಾದ ಸೂಚನೆಗಳಿವೆ. ಭ್ರಾತೃತ್ವದ ಆದರ್ಶವನ್ನು ಅನುಷ್ಠಾನದಲ್ಲಿ ತಂದು ನೆಲೆಗೊಳಿಸುವುದಕ್ಕಾಗಿಯೇ ಮೀಸಲಾದ ಪುಸ್ತಕ ಇದು. ‘ಯಾವ ಸಂನ್ಯಾಸಿಗಳ ಸಂಸ್ಥೆಗೇ ಆಗಲಿ, ಒಂದು ನಿಶ್ಚಿತವಾದ ಧ್ಯೇಯವಿರಬೇಕು; ಅದನ್ನು ಸಿದ್ಧಿಸಿಕೊಳ್ಳಲು ಯೋಗ್ಯ ಸಾಧನೆ, ತಪಸ್ಸು, ವ್ರತನಿಯಮಗಳಿರಬೇಕು; ಹಾಗೂ ಸಂಸ್ಕೃತಿ, ಶಿಕ್ಷಣ ಹಾಗೂ ಶಾಸ್ತ್ರಾಧ್ಯಯನಗಳು ರೂಢಿ ಗೊಂಡಿರಬೇಕು–ಇವುಗಳಿಲ್ಲದೆ ಆ ಸಂಸ್ಥೆಯು ಶುದ್ಧ ರೂಪದಲ್ಲಿ ಉಳಿದುಕೊಳ್ಳಲು ಮತ್ತು ತನ್ನ ಪ್ರಾರಂಭದ ಹುರುಪನ್ನು ಉಳಿಸಿಕೊಳ್ಳಲು ಸಾಧ್ಯವಿಲ್ಲ’–ಇದು ಸ್ವಾಮೀಜಿಯವರು ತಮ್ಮ ಶಿಷ್ಯರಿಗೆ ಪದೇಪದೇ ಹೇಳುತ್ತಿದ್ದ ಮಾತು. ಒಮ್ಮೆ ಅವರು ತಮ್ಮ ಶಿಷ್ಯರಿಗೆ ಹೇಳಿದರು, “ನಾನು-ನನ್ನ ಗುರುಭಾಯಿಗಳು ಶ್ರೀರಾಮಕೃಷ್ಣರ ಜೀವಿತಾವಧಿಯಲ್ಲೂ ಬಳಿಕ ಅವರ ನಿರ್ಯಾಣಾನಂತರವೂ ಅತ್ಯಂತ ಕಠಿಣ ತಪಶ್ಚರ್ಯೆಯನ್ನು ಮಾಡದೆ ಹೋಗಿದ್ದರೆ, ಮತ್ತು ಶ್ರೀರಾಮಕೃಷ್ಣರ ದಿವ್ಯ ಜೀವನಾದರ್ಶವು ನಮ್ಮ ಕಣ್ಣೆದುರಿಗೆ ಇಲ್ಲದೆ ಹೋಗಿದ್ದರೆ ನಾವಿಂದು ಏನಾಗಿದ್ದೇವೋ ಅದಾಗುವುದು ಸಾಧ್ಯವಿರಲಿಲ್ಲ.”

ಮಠದಲ್ಲಿ ಧ್ಯಾನಾದಿಗಳು ಉತ್ಸಾಹದಿಂದ ನಡೆದುಕೊಂಡು ಬರುವುದನ್ನು ಕಂಡಾಗ ಸ್ವಾಮೀಜಿ ಯವರಿಗಾಗುತ್ತಿದ್ದ ಆನಂದ ಅಷ್ಟಿಷ್ಟಲ್ಲ. ತಮ್ಮ ಹಳೆಯ ವಿಶ್ವಾಸಿಗರ ಹಾಗೂ ಗೃಹೀಭಕ್ತರ ಮುಂದೆ ಹೆಮ್ಮೆಯಿಂದ ಹೇಳುತ್ತಿದ್ದರು, “ಇಲ್ಲಿ ಸಾಧುಗಳೆಲ್ಲ ಹೇಗೆ ತಪೋಮಗ್ನರಾಗಿದ್ದಾರೆ ನೋಡಿದಿರಾ? ಇದ್ದರೆ ಹೀಗಿರಬೇಕು! ಶ್ರೀರಾಮಕೃಷ್ಣರು ಹೇಳುತ್ತಿದ್ದಂತೆ, ಪ್ರಾತಃಕಾಲ ಸಾಯಂ ಕಾಲಗಳಲ್ಲಿ ಮನಸ್ಸು ಸಹಜವಾಗಿಯೇ ಆಧ್ಯಾತ್ಮಿಕ ಭಾವಕ್ಕೆ ಏರಬಲ್ಲುದಾದ್ದರಿಂದ ಈ ಸಂಧಿಕಾಲ ದಲ್ಲಿ ಅದನ್ನು ಹತೋಟಿಯಲ್ಲಿಟ್ಟುಕೊಳ್ಳುವುದು ಸುಲಭಸಾಧ್ಯವಾಗುತ್ತದೆ. ಆದ್ದರಿಂದ ಸಾಧಕ ನಾದವನು ಈ ಕಾಲದಲ್ಲಿ ಧ್ಯಾನಾಭ್ಯಾಸ ಮಾಡಬೇಕು.” ಸ್ವಾಮೀಜಿಯವರು ಕೇವಲ ಆಡಿ ಮುಗಿಸುವವರಲ್ಲ, ಮಾಡಿ ತೋರಿಸುವವರೂ ಕೂಡ. ಉಷಃಕಾಲದಲ್ಲಿ ಅವರೂ ಇತರರೊಂದಿಗೆ ಧ್ಯಾನಕ್ಕೆ ಕುಳಿತುಬಿಡುತ್ತಿದ್ದರು. ಪ್ರಾರ್ಥನಾಮಂದಿರದ ಮುಖ್ಯವಾದ ಸ್ಥಳವೊಂದರಲ್ಲಿ ಅವರಿ ಗಾಗಿ ಒಂದು ಆಸನವನ್ನು ಹಾಕಿರುತ್ತಿತ್ತು. ಅವರು ತಮ್ಮ ಆಸನದಿಂದ ಮೇಲೇಳುವವರೆಗೆ ಯಾರಿಗೂ ಎದ್ದು ಹೋಗಲು ಅನುಮತಿಯಿರಲಿಲ್ಲ. ಎಷ್ಟೋ ಸಲ ಅವರ ಧ್ಯಾನ ಎರಡು ಗಂಟೆಗೂ ಹೆಚ್ಚುಹೊತ್ತು ನಡೆಯುತ್ತಿತ್ತು, ಬಳಿಕ ಅವರು “ಶಿವ! ಶಿವ!” ಎನ್ನುತ್ತ ಮೆಲ್ಲನೆ ಏಳುತ್ತಿದ್ದರು;ಶ್ರೀರಾಮಕೃಷ್ಣರಿಗೆ ಪ್ರಣಾಮ ಸಲ್ಲಿಸಿ ಮೆಟ್ಟಲಿಳಿದು ಕೆಳಕ್ಕೆ ಬಂದು ನಿಧಾನವಾಗಿ ಅತ್ತಿಂದಿತ್ತ ಅಡ್ಡಾಡುತ್ತ ದೇವಿಯ ಮೇಲೋ ಶಿವನ ಮೇಲೋ ಹಾಡೊಂದನ್ನು ಗುನುಗಿಕೊಳ್ಳು ತ್ತಿದ್ದರು. ಪ್ರಾರ್ಥನಾ ಮಂದಿರದಲ್ಲಿ ಸ್ವಾಮೀಜಿಯವರ ಸನ್ನಿಧಿಯು ಅವರ ಸುತ್ತ ಕುಳಿತವರಿಗೆಲ್ಲ ಹೆಚ್ಚಿನ ಸ್ಫೂರ್ತಿಯನ್ನೂ ಶಕ್ತಿಯನ್ನೂ ನೀಡುತ್ತಿತ್ತು. ಸ್ವತಃ ಧ್ಯಾನಸಿದ್ಧರಾದ ಸ್ವಾಮಿ ಬ್ರಹ್ಮಾ ನಂದರು ಒಮ್ಮೆ ಹೇಳುತ್ತಾರೆ, “ಆಹ್!ನರೇಂದ್ರನ ಜೊತೆಯಲ್ಲಿ ಕುಳಿತರೆ ಮನಸ್ಸು ಕೂಡಲೇ ಧ್ಯಾನದ ಸ್ಥಿತಿಗೇರುತ್ತದೆ. ನಾನೊಬ್ಬನೇ ಕುಳಿತಾಗ ಹಾಗಾಗುವುದಿಲ್ಲ” ಎಂದು.

ಹೀಗೆ ಸ್ವಾಮೀಜಿಯವರು ಒಂದು ಬಿಗಿಯಾದ ದಿನಚರಿಯನ್ನು ಹಾಕಿಕೊಟ್ಟದ್ದರಿಂದ ಊಟ, ವಿಶ್ರಾಂತಿ, ಪೂಜೆ, ಕೆಲಸಕಾರ್ಯಗಳು, ಅಧ್ಯಯನ, ಧ್ಯಾನ–ಇವೆಲ್ಲವೂ ಆ ದಿನಚರಿಯ ಕಟ್ಟಿನಲ್ಲಿ ಸೇರಿಕೊಂಡುವು. ಆಶ್ರಮವಾಸಿಗಳಿಗಿದ್ದಂತೆಯೇ ಭಕ್ತರಿಗೂ ನಿಯಮಾವಳಿಯನ್ನು ರೂಪಿಸ ಲಾಗಿತ್ತು. ಭಕ್ತರ ಸಂದರ್ಶನಗಳಿಂದಾಗಿ ಆಶ್ರಮವಾಸಿಗಳ ದಿನಚರಿಗೆ ಬಾಧಕವಾಗಬಾರದಲ್ಲವೆ? ಸಂಘದ ಹಿತದೃಷ್ಟಿಯಿಂದ ಸ್ವಾಮೀಜಿಯವರು ಕೆಲವೊಮ್ಮೆ ಕಠಿಣವಾಗಿರಬೇಕಾಗಿ ಬರುತ್ತಿತ್ತು. ಆಶ್ರಮವಾಸಿಗಳು ದಿನಚರಿಯನ್ನು ಕಟ್ಟುನಿಟ್ಟಾಗಿ ಪರಿಪಾಲಿಸಿಕೊಂಡು ಬರುವಂತೆ ಮಾಡಲು ಕೆಲವೊಮ್ಮೆ ಬಲಪ್ರಯೋಗ ಮಾಡಬೇಕಾಗಿ ಬರುತ್ತಿತ್ತು. ಇದರಿಂದಾಗಿ ಅವರು ಆಗಾಗ ಅಪ್ರಿಯರೂ ಆಗಬೇಕಾಗುತ್ತಿತ್ತು.

ಎಲ್ಲರೂ ಸೇರಿ ಮಾಡುವ ಧ್ಯಾನದಲ್ಲಿ ತಮಗೆ ಭಾಗವಹಿಸಲು ಸಾಧ್ಯವಾಗದಿದ್ದಾಗ ಸ್ವಾಮೀಜಿ ಯವರು ಅಂದಿನ ಹಾಜರಿಯನ್ನು ವಿಚಾರಿಸಿಯಾದರೂ ತಿಳಿದುಕೊಳ್ಳುತ್ತಿದ್ದರು. ಒಮ್ಮೆ ಕೆಲದಿನ ಗಳವರೆಗೆ ಅವರು ಧ್ಯಾನದ ಕಾರ್ಯಕ್ರಮಕ್ಕೆ ಹೋಗಿರಲಿಲ್ಲ. ಒಂದು ದಿನ ಇದ್ದಕ್ಕಿದ್ದಂತೆ ಆ ಹೊತ್ತಿಗೆ ಸರಿಯಾಗಿ ಪ್ರಾರ್ಥನಾ ಮಂದಿರದೊಳಗೆ ಪ್ರವೇಶಿಸಿ ನೋಡಿದರೆ ಆ ದಿನವೇ ಬಹಳ ಜನ ಗೈರುಹಾಜರಾಗಿರಬೇಕೆ! ಈ ಲೋಪವನ್ನು ಕಂಡು ಅವರಿಗೆ ಬಹಳ ದುಃಖವಾಯಿತು. ಕೂಡಲೇ ಕೆಳಗಿಳಿದು ಬಂದು ಎಲ್ಲರನ್ನೂ ಬಳಿಗೆ ಕರೆಸಿಕೊಂಡರು; ಅಂದು ಧ್ಯಾನಕ್ಕೆ ಬರದಿದ್ದವರನ್ನು ಕಾರಣ ಕೇಳಿದರು. ಒಂದಿಬ್ಬರು ಕಾಯಿಲೆಯಾದವರನ್ನು ಬಿಟ್ಟು, ಯಾರದೂ ಸರಿಯಾದ ಕಾರಣವಿರಲಿಲ್ಲ. ಆದ್ದರಿಂದ ಸ್ವಾಮೀಜಿ, ಮೊದಲೇ ಮಾಡಿಟ್ಟ ನಿಯಮದಂತೆ ಎಲ್ಲರಿಗೂ ಶಿಕ್ಷೆ ವಿಧಿಸಿಯೇ ಬಿಟ್ಟರು–“ನೀವೆಲ್ಲ ಇಂದು ಮಧುಕರಿ ಭಿಕ್ಷೆಗೆ ಹೋಗಬೇಕು. ಅಡಿಗೆ ಪದಾರ್ಥಗಳನ್ನು ಬೇಡಿ ತಂದು ಮಠದ ಮರದಡಿಯಲ್ಲಿ ಬೇಯಿಸಿಕೊಂಡು ಊಟ ಮಾಡಬೇಕು. ಆದರೆ ನೀವು ನಿಮ್ಮ ಸ್ನೇಹಿತರ ಮನೆಗಳಿಗೆ ಹೋಗಿ ಔತಣ ಮಾಡಿಕೊಂಡು ಬರುವಂತಿಲ್ಲ, ಜೋಪಾನ!” ಅಲ್ಲದೆ ಮಠದಿಂದ ಇವರ್ಯಾರಿಗೂ ಅಡಿಗೆ ಪದಾರ್ಥಗಳನ್ನು ಕೊಡಬಾರದು ಎಂದು ಅಡಿಗೆಮನೆ ನೋಡಿಕೊಳ್ಳುವ ಬ್ರಹ್ಮಚಾರಿಗಳಿಗೂ ಎಚ್ಚರಿಕೆ ಹೇಳಿದರು! ಮಧುಕರಿ ಭಿಕ್ಷೆಯ ಈ ಶಿಕ್ಷೆಗೆ ಅವರ ಸೋದರಸಂನ್ಯಾಸಿಗಳೂ ಹೊರತಾಗಿರಲಿಲ್ಲ. ಇತರ ಸಮಯದಲ್ಲಿ ಇವರನ್ನೆಲ್ಲ ಸ್ವಾಮೀಜಿಯವರು ಗೌರವಾದರಗಳಿಂದ ನಡೆಸಿಕೊಂಡರೂ ತಪ್ಪು ಮಾಡಿದ್ದಕ್ಕೆ ಶಿಕ್ಷೆ ವಿಧಿಸುವಾಗ ಮಾತ್ರ ಇವರನ್ನೂ ಎಲ್ಲರೊಂದಿಗೆ ಸೇರಿಸಿಬಿಡುತ್ತಿದ್ದರು.

ಈಗ ಎಲ್ಲರೂ ಭಿಕ್ಷೆಗೆ ಹೊರಡಲೇಬೇಕಾಯಿತು. ಆದರೆ ಸ್ವಾಮೀಜಿಯವರು ಎಲ್ಲರಿಗೂ ಸಮಾನ ಶಿಕ್ಷೆಯನ್ನೇನೋ ವಿಧಿಸಿದರಾದರೂ ತಾವು ಗೌರವಸ್ಥಾನದಲ್ಲಿಟ್ಟಿದ್ದ ಹಾಗೂ ತಮ್ಮ ಪ್ರೀತಿಪಾತ್ರರಾದ ಗುರುಭಾಯಿಗಳು ಭಿಕ್ಷೆಗೆ ಹೋಗಬೇಕಾದುದನ್ನು ಕಂಡು ಅವರ ಹೃದಯ ತಳಮಳಿಸಿತು. ಆದ್ದರಿಂದ ಅವರು ಕೆಲಸದ ನೆವ ಮಾಡಿಕೊಂಡು ಕಲ್ಕತ್ತಕ್ಕೆ ಹೋಗಿ ಆ ದಿನವಿಡೀ ಅಲ್ಲೇ ಇದ್ದು ಮರುದಿನ ಹಿಂದಿರುಗಿದರು. ಹಿಂದಿರುಗಿ ಬಂದಾಗ ಮಾತ್ರ ಎಲ್ಲರನ್ನೂ ಅತ್ಯಂತ ಕರುಣೆಯಿಂದ, ಪ್ರೀತಿಯಿಂದ ನೋಡಿಕೊಂಡರು. ಭಿಕ್ಷೆಗೆ ಹೋಗಿಬಂದವರ ವಿಚಿತ್ರ ಅನುಭವ ಗಳನ್ನು ಕೇಳಿ ನಕ್ಕರು. ಕೆಲವರಿಗೆ ಸುಖಭೋಜನದ ಅದೃಷ್ಟವಾದರೆ ಇನ್ನು ಕೆಲವರಿಗೆ ಬೈಯಿಸಿ ಕೊಳ್ಳುವ ಅದೃಷ್ಟ. ಇನ್ನು ಕೆಲವರಿಗೆ ಮಾತ್ರ ಬೇಲೂರಿನಿಂದ ಮೂರು ಮೈಲಿ ದೂರದ ಸಲ್ಕಿಯಾ ಎಂಬಲ್ಲಿನ ಮಾರ್ವಾಡಿಗಳ ರಸಕವಳದ ಸತ್ಕಾರ!

ಈ ನಡುವೆ ಸ್ವಾಮೀಜಿ ಮಾಯಾವತಿಗೆ ಹೋಗಿದ್ದ ತಮ್ಮ ಶಿಷ್ಯೆಯರನ್ನು ಮರೆತಿರಲಿಲ್ಲ. ಅದರಲ್ಲೂ ಸಾರಾ ಬುಲ್, ಕ್ರಿಸ್ಟೀನ, ನಿವೇದಿತಾ ಇವರೆಲ್ಲ ಅವರ ಪರಮ ವಿಶ್ವಾಸಪಾತ್ರರಲ್ಲವೆ? ಅವರು ಕ್ರಿಸ್ಟೀನಳಿಗೆ ಬರೆದ ಒಂದು ಪತ್ರ ಬಹಳ ಹಾಸ್ಯಪ್ರದವಾಗಿದೆ. ತುಂಬ ಆತ್ಮೀಯತೆ ಯಿದ್ದಾಗ ಮಾತ್ರವೇ ಈ ಬಗೆಯ ತಮಾಷೆಯ ಪತ್ರವನ್ನು ಬರೆಯಲು ಸಾಧ್ಯ. ಆ ಪತ್ರ ಹೀಗಿದೆ:

“ನನ್ನ ಆರೋಗ್ಯ ನಾನು ಬಯಸಿದಷ್ಟು ಸುಧಾರಿಸದಿದ್ದರೂ ತೀರ ಕೆಟ್ಟಿಲ್ಲ. ನನ್ನ ಯಕೃತ್ತು (ಲಿವರ್​) ಸುಧಾರಿಸಿದೆ. ಇದೊಂದು ಶುಭ ಸೂಚನೆ... ನಿನಗೆ ಆ ಪರ್ವತ ಪ್ರದೇಶದಿಂದ ಪ್ರಯೋಜನವಾಗಿದೆಯೆಂದು ತಿಳಿದು ಬಹಳ ಸಂತೋಷವಾಯಿತು.

“ಚೆನ್ನಾಗಿ ಊಟ ಮಾಡು, ಬೇಕೆಂಬಷ್ಟು ನಿದ್ರೆ ಮಾಡು. ಚೆನ್ನಾಗಿ ಕೊಬ್ಬಿ ಊದಿಕೊಳ್ಳುವವರೆಗೆ ಅಥವಾ ಕೊಬ್ಬಿ ಸಿಡಿದು ಹೋಗುವವರೆಗೂ ತಿನ್ನು, ನಿವೇದಿತಾ ಹೇಗಿದ್ದಾಳೆ?... 

ಈ ದಿನಗಳಲ್ಲೇ ಮೇಡಂ ಎಮ್ಮಾ ಕಾಲ್ವೆಯ ತಂದೆ ತೀರಿಕೊಂಡಿದ್ದರಿಂದ ಸ್ವಾಮೀಜಿ ಅವಳಿಗೊಂದು ಸಾಂತ್ವನದ ಪತ್ರ ಬರೆದರು:

“ನಿನಗೊದಗಿದ ದುಃಖವನ್ನು ತಿಳಿದು ನನಗೆ ಅತ್ಯಂತ ಖೇದವಾಯಿತು. ಈ ಬಗೆಯ ಪೆಟ್ಟುಗಳು ನಮಗೆಲ್ಲರಿಗೂ ಬರಬೇಕಾದದ್ದೇ. ಇವೆಲ್ಲ ತೀರ ಸಹಜವಾದದ್ದೇ. ಆದರೂ ಸಹಿಸಿ ಕೊಳ್ಳಲು ಮಾತ್ರ ಬಹಳ ಕಷ್ಟ. ನಾವು ವ್ಯಕ್ತಿಗಳೊಂದಿಗೆ ಬೆಳೆಸುವ ಸಂಬಂಧದಿಂದಾಗಿ ಆ ಅಸತ್ಯವಾದ ಜಗತ್ತೂ ಸತ್ಯವಾಗಿ ಕಾಣುತ್ತದೆ. ಈ ಸಂಬಂಧವು ನಿಕಟವಾದಷ್ಟೂ ಮತ್ತು ದೀರ್ಘವಾದಷ್ಟೂ, ನೆರಳೇ ಹೆಚ್ಚು ಸತ್ಯವಾಗಿ ಕಾಣುತ್ತದೆ. ಅದೊಂದು ದಿನ ಬರುತ್ತದೆ–ಆಗ ಅಸತ್ಯವು ಸತ್ಯದಲ್ಲಿ ಸೇರಿಕೊಂಡು ಕಣ್ಮರೆಯಾಗುತ್ತದೆ. ಆದರೆ ಅಯ್ಯೋ, ಅದನ್ನು ಸಹಿಸಲು ಎಷ್ಟು ಕಷ್ಟ! ಆದರೂ ಯಾವುದು ಸತ್ಯವಾದುದೋ ಸರ್ವವ್ಯಾಪಿಯಾದುದೋ ಅಂತಹ ಆತ್ಮವು ಸದಾ ನಮ್ಮೊಂದಿಗೆ ಇದೆ. ಈ ಮಾಯವಾಗುವ ನೆರಳುಗಳ ಪ್ರಪಂಚದಲ್ಲಿ ಸತ್ಯವನ್ನು ಕಂಡವನೇ ನಿಜಕ್ಕೂ ಧನ್ಯ!... ಭಗವಂತ ನಿನ್ನ ಮೇಲೆ ಕೃಪೆಯ ಮಳೆಗರೆಯಲಿ ಎಂಬುದೇ ಈ ವಿವೇಕಾನಂದನ ಅವಿರತ ಪ್ರಾರ್ಥನೆ.”

ಬೇಲೂರು ಮಠದಲ್ಲಿ ಈಗ ಒಂದೊಂದು ದಿನವೂ ಒಂದೊಂದು ಗಂಟೆಯೆಂಬಂತೆ ಕಳೆಯುತ್ತಿದೆ. ಆಧ್ಯಾತ್ಮಿಕಾನುಭವದಿಂದ ಹಾಗೂ ಲೋಕಾನುಭವದಿಂದ ಪರಿಪಕ್ವಗೊಂಡ ಸ್ವಾಮೀಜಿಯವರ ಸಾನ್ನಿಧ್ಯವು ಅವರ ಗುರುಭಾಯಿಗಳಿಗೆ, ಶಿಷ್ಯರಿಗೆ, ಭಕ್ತಜನರಿಗೆ ಮಹದಾನಂದ ಕಾರಿಯಾಗಿದೆ. ಕೆಲವೊಮ್ಮೆ ಅವರು ಅಸಹನೆ ತಾಳಬಹುದು, ಇನ್ನು ಕೆಲವೊಮ್ಮೆ ಸಹನಾಮೂರ್ತಿ ಯಂತೆ ಕಾಣಬಹುದು; ಒಮ್ಮೆ ಅವರು ಬೈಯುವುದು ಸರಿಯಿರಬಹುದು, ಮತ್ತೊಮ್ಮೆ ಅದು ಸರಿಯಿಲ್ಲದಿರಬಹುದು; ಕೆಲವೊಮ್ಮೆ ಅವರು ಬೋಧಕನಂತೆ ಕಾಣಿಸಿಕೊಳ್ಳಬಹುದು, ಇನ್ನು ಕೆಲವೊಮ್ಮೆ ಬ್ರಹ್ಮರ್ಷಿಯಂತೆ ಬೆಳಗಬಹುದು; ಒಮ್ಮೆ ಅವರು ಹಾಸ್ಯ ಚಟಾಕಿಗಳನ್ನು ಹಾರಿಸಿ ನಗೆಗಡಲಲ್ಲಿ ಮುಳುಗಿಸಬಹುದು, ಮತ್ತೊಮ್ಮೆ ಗಂಭೀರವದನರಾಗಿ ಕುಳಿತಿರಬಹುದು–ಏನೇ ಆದರೂ ಅವರು ತಮ್ಮ ಗುರುಭಾಯಿಗಳ ಪಾಲಿಗೆ ಅದೇ ಪ್ರಿಯ ನರೇನ್, ಅವರ ಶಿಷ್ಯರ ಪಾಲಿಗೆ ಅದೇ ಅದ್ಭುತ ಗುರು.

ಸ್ವಾಮೀಜಿಯವರಿಗೆ ಇದೀಗ ತಮ್ಮ ಕೆಲಸ ಮುಗಿದಿದೆ ಎಂಬುದು ವೇದ್ಯವಾಗಿತ್ತು. ಅವ ರೊಮ್ಮೆ ಮಿಸ್ ಮೆಕ್​ಲಾಡಳಿಗೆ ಹೇಳಿದರು, “ನಾನು ನನ್ನ ನಲವತ್ತನ್ನು ಮುಟ್ಟುವುದಿಲ್ಲ.” ಈಗ ಅವರಿಗೆ ಮೂವತ್ತೊಂಬತ್ತು ನಡೆಯುತ್ತಿತ್ತು. ಮಿಸ್ ಮೆಕ್​ಲಾಡ್ ತನ್ನ ತರ್ಕದಿಂದಾದರೂ ಅವರನ್ನು ಕಟ್ಟಿಹಾಕಲು ನೋಡಿದಳು–

“ಆದರೆ, ಸ್ವಾಮೀಜಿ, ಬುದ್ಧ ತನ್ನ ಮಹಾಕಾರ್ಯವನ್ನು ಸಾಧಿಸಿದ್ದು ತನ್ನ ನಲವತ್ತು-ಎಂಬ ತ್ತರ ನಡುವಿನ ಅವಧಿಯಲ್ಲಲ್ಲವೆ?”

“ನಾನು ನನ್ನ ಸಂದೇಶವನ್ನು ನೀಡಿಯಾಯಿತು. ನಾನಿನ್ನು ಹೋಗಬೇಕು.”

“ಏಕೆ ಹೋಗಬೇಕು ಎನ್ನುತ್ತೀರಿ?”

“ದೊಡ್ಡ ಮರದ ನೆರಳಲ್ಲಿ ಇತರ ಗಿಡಗಳು ಬೆಳೆದು ದೊಡ್ಡವಾಗಲಾರವು. ಚಿಕ್ಕವರಿಗೆ ಬೆಳೆಯಲು ಅವಕಾಶ ಮಾಡಿಕೊಡಬೇಕಾದರೆ ನಾನು ಹೋಗಬೇಕು.”

ಮೆಕ್​ಲಾಡಳ ಬಾಯಿ ಕಟ್ಟಿತು. ಅವಳಿನ್ನೇನು ತಾನೆ ಹೇಳಬಲ್ಲಳು?

