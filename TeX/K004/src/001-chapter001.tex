
\chapter{ಮರಳಿ ಮಾತೃಭೂಮಿಯೆಡೆಗೆ}

\noindent

ಪಶ್ಚಿಮ ದೇಶಗಳಲ್ಲಿ ಜಯಭೇರಿ ಬಾರಿಸಿ, ತಮ್ಮ ಕಾರ್ಯೋದ್ದೇಶದ ಬಹು ಮುಖ್ಯ ಅಂಶವನ್ನು ಯಶಸ್ವಿಯಾಗಿ ಮುಗಿಸಿದ ಸಂತೃಪ್ತಿಯಿಂದ ಸ್ವಾಮಿ ವಿವೇಕಾನಂದರು ೧೮೯೬ರ ಡಿಸೆಂಬರ್ ೧೬ರಂದು ಲಂಡನ್ನಿಗೆ ವಿದಾಯ ಹೇಳಿ ಟ್ರೈನಿನಲ್ಲಿ ಹೊರಟರು. ಅವರೊಂದಿಗೆ ಸೇವಿಯರ್ ದಂಪತಿಗಳೂ ಇದ್ದರು. ಸ್ವಾಮೀಜಿ ಮತ್ತು ಅವರ ಅನುಯಾಯಿಗಳು ಯೂರೋಪಿನ ಮಾರ್ಗ ವಾಗಿ ಸಾಗಿ, ಬಳಿಕ ನೇಪಲ್ಸ್ ನಗರದಿಂದ ಸಮುದ್ರಮಾರ್ಗವಾಗಿ ಪಯಣಿಸಿ ಭಾರತವನ್ನು ಮುಟ್ಟುವ ಕಾರ್ಯಕ್ರಮವಿತ್ತು. ಸ್ವಾಮೀಜಿಯವರ ಶೀಘ್ರಲಿಪಿಕಾರ-ಶಿಷ್ಯ ಗುಡ್​ವಿನ್, ನೇಪಲ್ಸಿ ನಲ್ಲಿ ಅವರನ್ನು ಕೂಡಿಕೊಳ್ಳಲಿದ್ದ. ಇಟಲಿಯ ಹಲವಾರು ಪುರಾತನ ನಗರಗಳನ್ನು ಸಂದರ್ಶಿಸ ಬೇಕೆಂಬುದು ಅವರ ಬಹುಕಾಲದ ಕನಸು. ಅವರು ಅವುಗಳನ್ನೆಲ್ಲ ಕಂಡು ಆನಂದಿಸಲಾಗುವಂತೆ ಅವರ ಸ್ನೇಹಿತರು ಪ್ರಯಾಣವನ್ನು ರೂಪಿಸಿದ್ದರು. ಅಲ್ಲದೆ ಆಯಾಸಕರವೂ ನೀರಸವೂ ಆದ ಸಮುದ್ರ ಪ್ರಯಾಣದ ದೂರವೂ ಇದರಿಂದಾಗಿ ಕಡಿಮೆಯಾಗಲಿತ್ತು. ಕಣ್ಣಿಗೆ ಹಬ್ಬವುಂಟು ಮಾಡುವ ಇಟಲಿಯ ಹಲವಾರು ಸುಂದರ ಸ್ಥಳಗಳನ್ನು ವೀಕ್ಷಿಸುವ ನಿರೀಕ್ಷೆಯಿಂದ ಸ್ವಾಮೀಜಿ ಬಾಲಕನಂತೆ ಉಲ್ಲಸಿತರಾಗಿದ್ದರು. ಟ್ರೈನಿನಲ್ಲಿ ಲಂಡನ್ನಿನಿಂದ ಡೋವರ್​ವರೆಗೂ ಪಯಣಿಸಿದ ಸ್ವಾಮೀಜಿಯವರ ತಂಡದವರು, ಅಲ್ಲಿ ಸ್ಟೀಮರ್ ಹತ್ತಿ ಇಂಗ್ಲಿಷ್ ಕಾಲುವೆಯನ್ನು ದಾಟಿ ಫ್ರಾನ್ಸಿನ ಕೆಲೇಸ್ ನಗರವನ್ನು ಸೇರಿದರು. ಇಲ್ಲಿ ಅವರು ಅಂತರರಾಷ್ಟ್ರೀಯ ಟ್ರೈನು ಹತ್ತಿ, ತಮ್ಮ ಮುಂದಿನ ಗುರಿಯಾದ ಮಿಲಾನ್ ನಗರದೆಡೆಗೆ ಸಾಗಿದರು. ಇದು ಹಲವು ನೂರು ಮೈಲಿಗಳ ಸುದೀರ್ಘ ಪ್ರಯಾಣ. ಆದರೆ ಸ್ವಾಮೀಜಿ ತುಂಬ ಉಲ್ಲಾಸಯುತರಾಗಿದ್ದರಿಂದ ಪ್ರಯಾಣ ಪ್ರಯಾಸವಾಗಿ ಪರಿಣಮಿಸಲಿಲ್ಲ. ಅಲ್ಲದೆ ಸೇವಿಯರ್ ದಂಪತಿಗಳೂ ತಮ್ಮ ಲವಲವಿಕೆಯ ಮಾತುಕತೆಯಿಂದ ಸ್ವಾಮೀಜಿಯವರಿಗೆ ಸಂತಸವುಂಟುಮಾಡಿದರು. ಅವರು ಟ್ರೈನಿನ ಕಿಟಕಿಯ ಮೂಲಕ ಕಾಣುವ ಪ್ರತಿಯೊಂದು ಊರನ್ನು, ಹಿಮಾವೃತ ಪರ್ವತಗಳನ್ನು, ಮನೋಹರ ವನರಾಶಿಯನ್ನು, ನದಿ-ಸರೋವರಗಳನ್ನು ವೀಕ್ಷಿಸುತ್ತ ಆನಂದಿಸಿದರು.

ಟ್ರೈನು ಫ್ರಾನ್ಸನ್ನು ದಾಟಿ ಇಟಲಿಯನ್ನು ಪ್ರವೇಶಿಸಿತು; ಕೊನೆಗೆ ಮಿಲಾನ್ ನಗರವನ್ನು ತಲುಪಿತು. ಇಲ್ಲಿ ಸ್ವಾಮೀಜಿ ಮತ್ತವರ ಸಂಗಾತಿಗಳು ಒಂದು ಹೋಟೆಲಿನಲ್ಲಿ ಇಳಿದುಕೊಂಡರು. ಈ ಹಿಂದೆ ಸ್ವಾಮೀಜಿ ಪಶ್ಚಿಮ ಯೂರೋಪಿನ ರಾಷ್ಟ್ರಗಳ ಪೈಕಿ ಫ್ರಾನ್ಸ್, ಜರ್ಮನಿ, ಸ್ವಿಟ್ಸರ್ ಲ್ಯಾಂಡ್ ಹಾಗೂ ಹಾಲೆಂಡ್​ಗಳನ್ನು ಸಂದರ್ಶಿಸಿದ್ದರು. ಇಟಲಿಗೆ ಇದು ಅವರ ಪ್ರಥಮ ಭೇಟಿ. ಮಿಲಾನ್ ನಗರದಲ್ಲಿ ಅವರು ಹಲವಾರು ಪ್ರಸಿದ್ಧ ಚರ್ಚುಗಳನ್ನು, ಕಲಾ ಶಾಲೆಗಳನ್ನು ಹಾಗೂ ಪ್ರದರ್ಶನಾಲಯಗಳನ್ನು ವೀಕ್ಷಿಸಿದರು. ಲೇನಾರ್ದೊ ದ ವಿಂಚಿಯ ಜಗತ್ಪ್ರಸಿದ್ಧ ಕಲಾಕೃತಿಯಾದ \eng{Last Supper (}ಕೊನೆಯ ಭೋಜನ) ಎಂಬ ವರ್ಣಚಿತ್ರವು ಸ್ವಾಮೀಜಿಯವರ ಮನಸೂರೆ ಗೊಂಡಿತು. ಇಲ್ಲಿಂದ ಅವರು ಪೀಸಾ ನಗರಕ್ಕೆ ಹೋದರು. ಪೀಸಾ ನಗರದಲ್ಲಿ ಅವರು ವಿಶ್ವವಿಖ್ಯಾತ \eng{Leaning Tower (}ಓರೆ ಗೋಪುರವನ್ನು) ಮತ್ತು ಕಪ್ಪು ಹಾಗೂ ಬಿಳುಪು ಅಮೃತಶಿಲೆಯ ಸುಂದರ ಕೆತ್ತನೆ ಕೆಲಸವನ್ನು ವೀಕ್ಷಿಸಿ ಆನಂದಿಸಿದರು. ಇಲ್ಲಿನ ಪ್ರಸಿದ್ಧ ‘ಕ್ಯಾಂಪೊಸ್ಯಾಂಟೊ’ ಇಗರ್ಜಿಗೂ ಅವರು ಭೇಟಿಯಿತ್ತರು.

ಪೀಸಾದಿಂದ ಸ್ವಾಮೀಜಿ ತಮ್ಮ ತಂಡದೊಡನೆ ಫ್ಲಾರೆನ್ಸ್ ನಗರಕ್ಕೆ ಆಗಮಿಸಿದರು. ಸುಂದರ ಪರ್ವತಗಳ ಹಿನ್ನೆಲೆಯಲ್ಲಿರುವ ಈ ಫ್ಲಾರೆನ್ಸ್ ನಗರವು ಅತ್ಯಂತ ಮನೋಹರವಾದುದು. ಸ್ವಾಮೀಜಿಯವರೂ ಅವರ ಸಂಗಾತಿಗಳೂ ಇಲ್ಲಿನ ಹಲವಾರು ವಿಖ್ಯಾತ ವಸ್ತುಸಂಗ್ರಹಾಲಯ ಗಳನ್ನು, ಕಲಾ ಪ್ರದರ್ಶನಾಲಯಗಳನ್ನು, ವಾಸ್ತುಶಿಲ್ಪಗಳನ್ನು ವೀಕ್ಷಿಸಿದರು. ಅಲ್ಲದೆ, ಇಲ್ಲಿನ ಹಲವಾರು ಸುಂದರ ಉದ್ಯಾನಗಳಲ್ಲಿ ಓಡಾಡಿದರು. ಈ ಸಂದರ್ಭದಲ್ಲಿ ಸ್ವಾಮೀಜಿಯವರಿಗೆ ಅನಿರೀಕ್ಷಿತ ಆನಂದವೊಂದು ಎದುರಾಯಿತು. ಅವರು ಸೇವಿಯರ್ ದಂಪತಿಗಳೊಂದಿಗೆ ಉದ್ಯಾನವೊಂದರಲ್ಲಿ ಓಡಾಡುತ್ತಿರುವಾಗ ತಮ್ಮ ಪರಮ ಆಪ್ತ ಅಮೆರಿಕನ್ ಸ್ನೇಹಿತರಾದ ಶ್ರೀಮತಿ ಮತ್ತು ಶ್ರೀ ಜಾರ್ಜ್ ಹೇಲ್​ರನ್ನು ಸಂಧಿಸಿದರು. ಪರಸ್ಪರರನ್ನು ಕಂಡಾಗ ಅವರಿಗಾದ ಆನಂದ ವರ್ಣನಾತೀತವಾದದ್ದು. (ಸರ್ವಧರ್ಮ ಸಮ್ಮೇಳನಕ್ಕೆ ಮೊದಲು, ಸ್ವಾಮೀಜಿ ತಮ್ಮ ಗುರುತಿನ ಚೀಟಿಯನ್ನು ಕಳೆದುಕೊಂಡು ನಿಸ್ಸಹಾಯಕರಾಗಿ ಕುಳಿತಿದ್ದಾಗ, ದೇವತೆಯಂತೆ ಅವರ ನೆರವಿಗೆ ಬಂದ ಮಹಿಳೆಯೇ ಶ್ರೀಮತಿ ಜಾರ್ಜ್ ಹೇಲ್ ಎಂಬುದನ್ನಿಲ್ಲಿ ನೆನಪಿಸಿಕೊಳ್ಳಬಹುದು. ಈ ದಂಪತಿಗಳೂ ಇವರ ಕುಟುಂಬಕ್ಕೆ ಸೇರಿದ ನಾಲ್ವರು ಹೆಣ್ಣುಮಕ್ಕಳೂ ಸ್ವಾಮೀಜಿಯವರ ಅತ್ಯಂತ ಆತ್ಮೀಯ ವರ್ಗಕ್ಕೆ ಸೇರಿದ್ದವರು.) ಕ್ಯಾಪ್ಟನ್ ಮತ್ತು ಶ್ರೀಮತಿ ಸೇವಿಯರ್​ರನ್ನೂ ಹೇಲ್ ದಂಪತಿಗಳನ್ನೂ ಸ್ವಾಮೀಜಿ ಪರಸ್ಪರರಿಗೆ ಪರಿಚಯಿಸಿಕೊಟ್ಟರು. ಹೇಲ್ ದಂಪತಿಗಳು ಆಗ ಇಟಲಿಯ ಪ್ರವಾಸದಲ್ಲಿದ್ದರು. ಇವರೊಂದಿಗೆ ಸ್ವಾಮೀಜಿ ಲೋಕಾಭಿರಾಮವಾಗಿ ಮಾತ ನಾಡುತ್ತ, ಅಮೆರಿಕದಲ್ಲಿನ ತಮ್ಮ ಹಲವಾರು ಮಧುರ ಸ್ಮರಣೆಗಳನ್ನು ಮೆಲುಕು ಹಾಕಿದರು. ಅಲ್ಲದೆ, ಭಾರತದಲ್ಲಿನ ತಮ್ಮ ಉದ್ದೇಶಿತ ಕಾರ್ಯಯೋಜನೆಗಳ ಬಗ್ಗೆ ಅವರಿಗೆ ವಿವರಿಸಿದರು. ಹೀಗೆ ಅತ್ಯಂತ ಆನಂದಕರವಾದ ಕೆಲವು ಗಂಟೆಗಳನ್ನು ಕಳೆದು ಸ್ವಾಮೀಜಿ ಅವರಿಂದ ಬೀಳ್ಕೊಂಡರು.

ಟ್ರೈನು ಫ್ಲಾರೆನ್ಸಿನಿಂದ ರೋಮ್ ನಗರದೆಡೆಗೆ ಸಾಗಿದಂತೆ ಸ್ವಾಮೀಜಿ ಭಾವೋದ್ವೇಗಭರಿತ ರಾದರು. ಈ ನಗರದ ಬಗ್ಗೆ ಅವರಿಗೊಂದು ವಿಶೇಷ ಆಕರ್ಷಣೆ, ಅಭಿಮಾನ. ತಮ್ಮ ಕಾಲೇಜು ದಿನಗಳಲ್ಲಿ ರೋಮನ್ ಚರಿತ್ರೆಯನ್ನು ಅಧ್ಯಯನ ಮಾಡಿ ಅಲ್ಲಿನ ವ್ಯಕ್ತಿಗಳ ಹಾಗೂ ವಿಷಯಗಳ ಬಗ್ಗೆ ಅರಿತುಕೊಂಡಾಗಿನಿಂದಲೂ ಸ್ವಾಮೀಜಿ ಎಷ್ಟೋ ಬಾರಿ ತಮ್ಮ ಭಾವನಾ ಪ್ರಪಂಚದಲ್ಲೇ ರೋಮ್​ಗೆ ಭೇಟಿ ನೀಡಿದ್ದರು; ಅಲ್ಲಿ ವಿಹರಿಸಿದ್ದರು. ಅವರ ಪಾಲಿಗೆ ರೋಮ್ ಪಾಶ್ಚಾತ್ಯ ನಾಗರಿಕತೆಯ ಕೇಂದ್ರ ಸ್ಥಾನ. ಇಲ್ಲಿಗೆ ಬರುವುದಕ್ಕೆ ಬಹಳ ಹಿಂದಿನಿಂದಲೇ ಸ್ವಾಮೀಜಿ ತಮ್ಮ ಜೊತೆಯವರೊಂದಿಗೆ ರೋಮ್ ಜಗತ್ತಿನ ವೈಭವದ ಕುರಿತಾಗಿ ಸಂಭಾಷಿಸುತ್ತಿದರು. ಅವರ ಆ ಸ್ಪೂರ್ತಿಯುತ ಮಾತುಗಳನ್ನು ಕೇಳುತ್ತಿದ್ದವರಿಗೆ ಅಲ್ಲಿನ ಸಮಸ್ತ ಇತಿಹಾಸವೇ ಜೀವಂತವಾಗಿ ಕಣ್ಣೆದುರು ಬಂದು ನಿಂತಂತೆ ಭಾಸವಾಗುತ್ತಿತ್ತು. ಸ್ವಾಮೀಜಿಯವರ ಮನಸೆಳೆದಿದ್ದ ಅಂಶ ಗಳೆಂದರೆ ಪುರಾತನ ರೋಮ್​ನ ಅದ್ಭುತ ಅವಶೇಷಗಳು ಮಾತ್ರವೇ ಅಲ್ಲ; ರೋಮ್ ಅನ್ನು ಪ್ರತಿನಿಧಿಸುವ ವಾಸ್ತುಶಿಲ್ಪ, ಚಿತ್ರಕಲೆ, ಶಿಲ್ಪಕಲೆ ಎಲ್ಲವೂ ಸ್ವಾಮೀಜಿಯವರ ಪಾಲಿಗೆ ಅತ್ಯಂತ ಆಸಕ್ತಿಕರ ವಿಷಯಗಳಾಗಿದ್ದುವು.

ಡಿಸೆಂಬರ್ ೨೧ರಂದು ಸ್ವಾಮೀಜಿ ತಮ್ಮ ಸಂಗಡಿರೊಂದಿಗೆ ರೋಮ್ ತಲುಪಿದರು. ಅಲ್ಲಿ ಅವರು ಸುಮಾರು ಒಂದು ವಾರ ಕಾಲ ಇದ್ದು ಪ್ರತಿದಿನವೂ ಹೊಸ ಹೊಸ ಸ್ಥಳಗಳನ್ನು ಸಂದರ್ಶಿಸಿದರು. ಮಿಸ್ ಮೆಕ್​ಲಾಡಳ ಸೋದರ ಸೊಸೆ ಆಲ್ಬರ್ಟಾ ಸ್ಟರ್ಜಿಸ್ ಈ ಸಮಯದಲ್ಲಿ ರೋಮ್ ನಗರದಲ್ಲಿದ್ದಳು. ಈಕೆ ಸ್ವಾಮೀಜಿಯವರಿಗೆ ಚೆನ್ನಾಗಿ ಪರಿಚಯವಿದ್ದವಳು. ಇವಳು ಆ ನಗರದಲ್ಲಿ ತನ್ನ ಸ್ನೇಹಿತೆ ಮಿಸ್ ಎಡ್ವರ್ಡ್ಸ್​ಳೊಂದಿಗಿದ್ದಳು. ಮಿಸ್ ಮೆಕ್​ಲಾಡ್ ಕೊಟ್ಟಿದ್ದ ವಿಳಾಸದ ಸಹಾಯದಿಂದ ಸ್ವಾಮೀಜಿ ತಮ್ಮ ಸಂಗಡಿರೊಂದಿಗೆ ಇವರನ್ನು ಭೇಟಿಯಾದರು. ಸ್ವಾಮೀಜಿಯವರನ್ನು ಕಂಡು ಆನಂದಿತರಾದ ಆಲ್ಬರ್ಟಾ ಹಾಗೂ ಮಿಸ್ ಎಡ್ವರ್ಡ್ಸ್, ಅವರ ನಗರಪ್ರವಾಸದಲ್ಲಿ ಜೊತೆಗೂಡಿಕೊಂಡರು. ಸ್ವಾಮೀಜಿಯವರ ಸಂಪರ್ಕಕ್ಕೆ ಬಂದ ಮಿಸ್ ಎಡ್ವರ್ಡ್ಸ್, ಅವರ ಸಿದ್ಧಾಂತದಿಂದ ತೀವ್ರವಾಗಿ ಆಕರ್ಷಿತಳಾದಳು. ಅಲ್ಲದೆ ರೋಮನ್ ಇತಿಹಾಸದ ಬಗ್ಗೆ ಮತ್ತು ವಿವಿಧ ಸಂಸ್ಕೃತಿಗಳ ಬಗ್ಗೆ ಅವರಿಗಿದ್ದ ಅಗಾಧ ಪರಿಜ್ಞಾನವನ್ನು ಕಂಡು ವಿಸ್ಮಿತಳಾದಳು.

ರೋಮ್ ನಗರದಲ್ಲಿ ಈ ಯಾತ್ರಿಕರು, ಈಗ ಕೇವಲ ಅವಶೇಷಗಳಾಗಿರುವ ಸೀಸರ್​ನ ಮನೆತನಕ್ಕೆ ಸೇರಿದ ಅರಮನೆಗಳು, ಪ್ಯಾಲ್​ಟೀನ್ ಬೆಟ್ಟ, ಕ್ಯಾಪಿಟಲೀನ್ ಬೆಟ್ಟ, ಪ್ರಾಚೀನ ರೋಮನ್ನರ ಸಾರ್ವಜನಿಕ ಸ್ನಾನಗೃಹಗಳು, ಕ್ರೀಡಾಂಗಣಗಳು, ಬೃಹತ್ ದೇವಸ್ಥಾನಗಳು– ಇವುಗಳನ್ನೆಲ್ಲ ವೀಕ್ಷಿಸಿದರು. ಇವುಗಳಲ್ಲಿ ಒಂದೊಂದೂ ಅದ್ಭುತ ನಿರ್ಮಿತಿಗಳು. ಪ್ರಾಚೀನ ರೋಮನ್ ನಾಗರಿಕತೆಯ ವೈಭವವನ್ನು ಇವು ಸಾರುತ್ತಿವೆ. \eng{Forum of Trajan} ಎಂಬ ಸ್ಥಳವು ಹಿಂದೆ ಹಲವಾರು ಸುಂದರ ಕಟ್ಟಡಗಳಿಂದ ಕೂಡಿತ್ತು. ಈಗ \eng{Trajan’s Pillar} ಎಂಬ ಬೃಹದ್ ಗಾತ್ರದ ಸ್ತಂಭವೊಂದೇ ತಕ್ಕಮಟ್ಟಿಗೆ ಉಳಿದುಕೊಂಡಿದೆ. ಇದು ಇಡೀ ರೋಮ್ ನಗರದಲ್ಲೇ ಅತ್ಯಂತ ಅದ್ಭುತವಾದುದು. ಅಮೃತ ಶಿಲೆಯಿಂದ ನಿರ್ಮಿತವಾದ ಈ ಸ್ತಂಭವು, ಅದರ ಪೀಠವೂ ಸೇರಿದಂತೆ ೧೨೫ ಅಡಿ (ಸುಮಾರು ೧೨ ಅಂತಸ್ತಿನ ಕಟ್ಟಡದಷ್ಟು) ಎತ್ತರವಿದೆ. ಇದರ ಮೇಲೆ ಎರಡು ಸಾವಿರಕ್ಕೂ ಹೆಚ್ಚು ಮನುಷ್ಯರ ಆಕೃತಿಗಳಿವೆ. ಇವನ್ನು ಸ್ವಾಮೀಜಿ ವಿಶೇಷ ಆಸಕ್ತಿಯಿಂದ ಸೂಕ್ಷ್ಮವಾಗಿ ಗಮನಿಸಿದರು.

ರೋಮ್ ನಗರದ ‘ಪಳೆಯುಳಿಕೆ’ಗಳನ್ನು ವೀಕ್ಷಿಸುವಾಗ ಮೊದಮೊದಲಿಗೆ ಸ್ವಾಮೀಜಿ ತುಂಬ ಗಂಭೀರ ಭಾವದಿಂದ ಮೌನವಾಗಿರುತ್ತಿದ್ದರು. ಆದರೆ ಅವರನ್ನು ಗಮನಿಸಿ ನೋಡಿದಾಗ ತಿಳಿಯುತ್ತಿತ್ತು–ಅವರು ಅವುಗಳೆಲ್ಲದರ ಬಗ್ಗೆ ಎಷ್ಟು ಆಳವಾಗಿ ಆಸಕ್ತರಾಗಿದ್ದಾರೆ, ಮತ್ತು ಅವರ ಮನಸ್ಸಿನಲ್ಲಿ ಎಂತಹ ಪ್ರತಿಕ್ರಿಯೆ ನಡೆಯುತ್ತಿದೆ ಎಂದು. ಆದರೆ ಸ್ಥಳದಿಂದ ಸ್ಥಳಕ್ಕೆ ಹೋದಂತೆ ಸ್ವಾಮೀಜಿ ತಾವು ವೀಕ್ಷಿಸುತ್ತಿದ್ದ ವಿಷಯಗಳ ಬಗ್ಗೆ ಮಾತನಾಡತೊಡಗಿದರು. ರೋಮನ್ ನಾಗರಿಕತೆಯು ಇಂದಿನ ಆಧುನಿಕ ನಾಗರಿಕತೆಯ ಮೇಲೆ ಎಷ್ಟರಮಟ್ಟಿಗೆ ಪ್ರಭಾವ ಬೀರಿದೆಯೆಂಬುದನ್ನು ಅವರು ಉಜ್ವಲವಾಗಿ ವಿವರಿಸಿದರು. ಸ್ವಾಮೀಜಿಯವರ ಇತಿಹಾಸ ಜ್ಞಾನ ವನ್ನು ಕಂಡು ಅವರಲ್ಲೊಬ್ಬರು ಉದ್ಗರಿಸಿದರು, “ಸ್ವಾಮೀಜಿ, ಇದು ನಿಜಕ್ಕೂ ಅದ್ಭುತವೇ ಸರಿ; ನಿಮಗೆ ರೋಮ್ ನಗರದ ಪ್ರತಿಯೊಂದು ಕಲ್ಲೂ ಪರಿಚಿತವಾಗಿರುವಂತೆ ತೋರುತ್ತದಲ್ಲ!” ಸುಶಿಕ್ಷಿತರೂ ಮೇಧಾವಿಗಳೂ ಆದ, ಮತ್ತು ತಮ್ಮ ನಾಗರಿಕತೆ-ಚರಿತ್ರೆಗಳ ಬಗ್ಗೆ ಸಾಕಷ್ಟು ಅರಿವಿರುವ ಆ ಪಾಶ್ಚಾತ್ಯರಿಗೇ ಅಷ್ಟು ಆಶ್ಚರ್ಯವಾಗಿರಬೇಕಾದರೆ ಸ್ವಾಮೀಜಿಯವರ ಜ್ಞಾನ ಭಂಡಾರ ಎಷ್ಟು ಅಗಾಧವಾಗಿದ್ದಿರಬೇಕೆಂದು ಊಹಿಸಬಹುದು.

ರೋಮನ್ ನಾಗರೀಕತೆಯ ಪ್ರಾಚೀನ ಅವಶೇಷಗಳಿಗಿಂತ ಹೆಚ್ಚಾಗಿ ಸ್ವಾಮೀಜಿ ಆಸಕ್ತರಾಗಿ ದ್ದುದು ‘ಕ್ರೈಸ್ತ ರೋಮ್​’ನಲ್ಲಿ–ಎಂದರೆ, ಕ್ರೈಸ್ತ ಧರ್ಮಕ್ಕೆ ಸಂಬಂಧಿಸಿದ ಐತಿಹಾಸಿಕ ಸ್ಥಳಗಳಲ್ಲಿ. ಕ್ರೈಸ್ತರ ಅತ್ಯುನ್ನತ ಧರ್ಮಗುರುಗಳಾದ ಪೋಪರು ವಾಸಿಸುವುದು ಇಲ್ಲಿನ ವ್ಯಾಟಿ ಕನ್ ಎಂಬಲ್ಲಿಯೇ. ಜಗತ್ತಿನ ಅತಿದೊಡ್ಡ ಹಾಗೂ ಅತ್ಯಂತ ಪವಿತ್ರ ಇಗರ್ಜಿಯಾದ ‘ಸೈಂಟ್ ಪೀಟರ್ಸ್ ಚರ್ಚ್​’ ಇರುವುದು ಇಲ್ಲಿಯೇ. ಇದಲ್ಲದೆ ರೋಮ್ ನಗರದ ಸಮೀಪದಲ್ಲಿ ಇತರ ಅಸಂಖ್ಯಾತ ಚರ್ಚುಗಳೂ ಪೂಜಾಗೃಹಗಳೂ ಇವೆ. ಸ್ವಾಮೀಜಿ ತಮ್ಮ ಸಂಗಡಿಗರೊಂದಿಗೆ ಇಲ್ಲಿನ ಪ್ರಾಚೀನ ‘ಸಮಾಧಿ ಸುರಂಗ’ಗಳನ್ನು ವೀಕ್ಷಿಸಿದರು. ಇವುಗಳಲ್ಲಿ ಪ್ರಮುಖ ವ್ಯಕ್ತಿಗಳ ಗೋರಿಗಳಿವೆ. ಮಧ್ಯ ಯುಗದಲ್ಲಿ ನಿರ್ಮಿತವಾದ ‘ಬೆಸಿಲಿಕಾ’ಗಳೆಂದು ಕರೆಯಲ್ಪಡುವ ಸುಂದರ ದೇವಾಲಯಗಳಿಗೆ ಭೇಟಿಯಿತ್ತಾಗ ಸ್ವಾಮೀಜಿ, ಅವುಗಳಲ್ಲಿ ರಚಿತವಾದ ಅಭೂತಪೂರ್ವ ಕಲಾಕೃತಿಗಳನ್ನು ಕಲಾವಿನ್ಯಾಸಗಳನ್ನು ಕಂಡು ಆಶ್ಚರ್ಯಚಕಿತರಾದರು; ಅಂದಿನ ಕಾಲದ ಕ್ರೈಸ್ತ ಮಿಷನರಿಗಳ ಸಂಘಟನಾ ಪ್ರತಿಭೆಯನ್ನು ಭಾವಿಸಿ ಬೆರಗಾದರು. ಮಹಾ ವೈಭವೋಪೇತವಾದ ವ್ಯಾಟಿಕನ್, ಅಲ್ಲಿನ ಬೃಹತ್ ಗುಡಿಗೋಪುರಗಳು ಇವುಗಳನ್ನೆಲ್ಲ ಕಂಡು ಗಾಢವಾಗಿ ಪ್ರಭಾವಿತ ರಾದರು. ಸೈಂಟ್ ಪೀಟರ್ಸ್ ಚರ್ಚಿನ ಬೃಹದ್ ಛಾವಣಿಯು ಮಹಾನ್ ಕಲಾಕಾರ ಮೈಕಲೇಂ ಜೆಲೋನ ವರ್ಣಚಿತ್ರಗಳಿಂದ ಅಲಂಕೃತವಾಗಿದೆ. ಇವುಗಳನ್ನು ತನ್ಮಯತೆಯಿಂದ ದಿಟ್ಟಿಸುತ್ತ ಸ್ವಾಮೀಜಿ ಭಾವಪರವಶರಾದರು. ಸಂತ ಪಾಲನೂ ಸಂತ ಪೀಟರನೂ ಧರ್ಮಬೋಧನೆ ಮಾಡಿದ ಸ್ಥಳದಲ್ಲಿ ನಿಂತಾಗ ಸ್ವಾಮೀಜಿ, ಆ ಸಂತರು ಭಗವಂತನ ಕಾರ್ಯಕ್ಕಾಗಿ ಹೇಗೆ ತಮ್ಮನ್ನೇ ಸಮರ್ಪಿಸಿಕೊಂಡರೆಂಬುದನ್ನು ಸ್ಮರಿಸುತ್ತ ಧ್ಯಾನಮಗ್ನರಾದರು. ಅಲ್ಲಿನ ಚರ್ಚುಗಳ ವೈಶಾಲ್ಯ ವನ್ನು, ಅವುಗಳ ವಾಸ್ತುಶಿಲ್ಪದ ವೈಭವಗಳನ್ನು ನೋಡುತ್ತ ಸಾಗುವಾಗ ಅವರ ಜೊತೆಗಾರ ರಲ್ಲೊಬ್ಬರು ಉದ್ಗರಿಸಿದರು, “ಸ್ವಾಮೀಜಿ, ಭಗವಂತನ ಹೆಸರಿನಲ್ಲಿ ನಡೆಯುತ್ತಿರುವ ಈ ಎಲ್ಲ ಅಬ್ಬರ-ಆಡಂಬರಗಳ ಬಗ್ಗೆ ನಿಮ್ಮ ಅಭಿಪ್ರಾಯವೇನು? ಜಗತ್ತಿನಲ್ಲಿ ಲಕ್ಷಾಂತರ ಜನ ಹೊಟ್ಟೆಗಿಲ್ಲದೆ ನರಳುತ್ತಿರುವಾಗ ಅದ್ಧೂರಿಯ ಪೂಜಾವಿಧಿಗಳಿಗಾಗಿ ಹಣವನ್ನು ಚೆಲ್ಲುತ್ತಿದ್ದಾ ರಲ್ಲ, ಇದು ಅನ್ಯಾಯವಲ್ಲವೆ?” ಬ್ರಹ್ಮಜ್ಞಾನಿಗಳೂ ಅದ್ವೈತ ವೇದಾಂತಿಗಳೂ ಸಂನ್ಯಾಸಿಗಳೂ ಆದ ಸ್ವಾಮೀಜಿ ಈ ಪ್ರಶ್ನೆಗೆ ಏನೆಂದು ಉತ್ತರಿಸಬಹುದು? ಸ್ವಲ್ಪವೂ ಅನುಮಾನಿಸದೆ ಅವರು ತಕ್ಷಣ ಹೇಳಿದರು, “ಏನು! ಭಗವಂತನಿಗೆ ಎಷ್ಟು ಕೊಟ್ಟರೆ ತಾನೆ ‘ಅತಿ’ಯಾದೀತು! ಈ ಎಲ್ಲ ಅದ್ಧೂರಿ ವೈಭವಗಳ ಮೂಲಕ ಜನರಿಗೆ, ಕ್ರಿಸ್ತನಂತಹ ಒಬ್ಬ ವ್ಯಕ್ತಿಯ ಶಕ್ತಿಯೆಂಥದು ಎಂಬುದರ ಅರಿವಾಗುತ್ತದೆ. ಯಾವ ಭಗವಾನ್ ಏಸುಕ್ರಿಸ್ತನು ಸ್ವತಃ ಒಬ್ಬ ನಿರ್ಗತಿಕನಾಗಿದ್ದನೋ ಅಂತಹ ಅವನು ತನ್ನ ಪರಮ ಪರಿಶುದ್ಧ ವ್ಯಕ್ತಿತ್ವದಿಂದ ಮನುಷ್ಯನ ಭಾವನಾಶಕ್ತಿಯನ್ನು ಎಷ್ಟರಮಟ್ಟಿಗೆ ಪ್ರಚೋದಿಸಿದ್ದಾನೆ ಎಂಬುದು ಜನರಿಗೆ ಗೊತ್ತಾಗುತ್ತದೆ.” ಬಳಿಕ ತಮ್ಮ ಮಾತನ್ನು ಮುಂದುವರಿಸುತ್ತ ಸ್ವಾಮೀಜಿ ಹೇಳಿದರು, “ಆದರೆ ಇದರ ಜೊತೆಯಲ್ಲೇ ನಾವು ಇನ್ನೊಂದು ವಿಷಯವನ್ನು ನೆನಪಿಟ್ಟುಕೊಳ್ಳಬೇಕು; ಏನೆಂದರೆ, ನಮ್ಮ ಅಂತರಂಗದ ಪರಿ ಶುದ್ಧತೆಗೆ ನೆರವಾಗುವಂತಿದ್ದರೆ ಮಾತ್ರ ಈ ಬಾಹ್ಯಾಚರಣೆಗಳಿಗೆಲ್ಲ ಬೆಲೆಯಿದೆ, ಅರ್ಥವಿದೆ. ಆದರೆ ಪವಿತ್ರಜೀವನವನ್ನು ನಡೆಸುವಂತೆ ಮಾಡುವಲ್ಲಿ ಅವು ನಿಷ್ಫಲವಾದರೆ, ಅವುಗಳನ್ನು ನಿರ್ದಾಕ್ಷಿಣ್ಯವಾಗಿ ಮೆಟ್ಟಿಹಾಕುವುದು ಒಳ್ಳೆಯದು.”

ಆದರೆ ಸ್ವಾಮೀಜಿ ಹೀಗೆ ಹೇಳಿದುದು ಒಂದು ವಿಶಾಲ ದೃಷ್ಟಿಕೋನದಿಂದ. ಅವರ ಈ ಮಾತು ಜಗತ್ತಿಗೆ ಹೆಚ್ಚಿನ ಜನಗಳಿಗೆ ಹಾಗೂ ಸಂದರ್ಭಗಳಿಗೆ ಅನ್ವಯಿಸುವಂಥದು. ಪ್ರತಿಯೊಂದು ವಿಷಯಕ್ಕೂ ಒಂದಕ್ಕಿಂತ ಹೆಚ್ಚು ಮುಖಗಳಿರುತ್ತವೆ. ಒಂದು ದೃಷ್ಟಿಯಿಂದ ಅದು ಒಳ್ಳೆಯದಾಗಿ ತೋರಿದರೆ ಮತ್ತೊಂದು ದೃಷ್ಟಿಯಿಂದ ಅದೇ ಕೆಟ್ಟದ್ದಾಗಿ ತೋರಬಹುದು. ಒಂದು ದೃಷ್ಟಿಯಿಂದ ಸಮಂಜಸವಾಗಿದ್ದರೆ ಮತ್ತೊಂದು ದೃಷ್ಟಿಯಿಂದ ಅಸಮಂಜಸವಾಗಬಹುದು. ಪ್ರತಿಯೊಂದು ವಿಷಯದ ಹಿನ್ನೆಲೆಯಲ್ಲೂ ಒಳ್ಳೆಯದನ್ನು ಕಂಡುಕೊಳ್ಳುವುದು ಸ್ವಾಮೀಜಿಯವರ ಸ್ವಭಾವ, ಮತ್ತು ಅದು ಅವರ ಸಂದೇಶವೂ ಹೌದು. ಆದ್ದರಿಂದಲೇ ಅವರು ಪೂಜಾದಿಗಳ ವಿಷಯದಲ್ಲಿ ಬಾಹ್ಯಾಚರಣೆಗಳ ಹಾಗೂ ಬಾಹ್ಯಾಡಂಬರಗಳ ಪ್ರಶ್ನೆ ಬಂದಾಗ ಅದನ್ನು ಸಮರ್ಥಿಸಿದುದು. ಇದೇ ವಿಷಯದ ಬಗ್ಗೆ ಅವರ ಮತ್ತೊಂದು ಅಭಿಪ್ರಾಯವು ವ್ಯಕ್ತವಾಗುವಂತಹ ಸಂದರ್ಭವೂ ಶೀಘ್ರದಲ್ಲೇ ಒದಗಿಬಂದಿತು.

ಅದು ಡಿಸೆಂಬರ್ ತಿಂಗಳು; ಎಲ್ಲೆಲ್ಲೂ ಏಸುಕ್ರಿಸ್ತನನ್ನು ಆರಾಧಿಸುವ ಕ್ರಿಸ್​ಮಸ್ ಹಬ್ಬದ ಸಂಭ್ರಮ. ಸೈಂಟ್ ಪೀಟರ್ಸ್ ಚರ್ಚಿನಲ್ಲಿ ಕ್ರಿಸ್​ಮಸ್ ದಿನದಂದು ನಡೆಯುವ ಆರಾಧನೆಯು ಒಂದು ಅಪೂರ್ವ-ಅದ್ಭುತ ದೃಶ್ಯ. ಆ ಕಾರ್ಯಕ್ರಮದಲ್ಲಿ ಲಕ್ಷಾಂತರ ಜನ ಭಾಗವಹಿಸುತ್ತಾರೆ. ಸೇವಿಯರ್ ದಂಪತಿಗಳೊಂದಿಗೆ ಸ್ವಾಮೀಜಿಯವರೂ ಅಂದು ಅದರಲ್ಲಿ ಭಾಗವಹಿಸಿದರು. ಕಾರ್ಯಕ್ರಮ ವಿಜೃಂಭಣೆಯಿಂದ ನಡೆಯುತ್ತಿತ್ತು. ಆದರೆ ಸ್ವಲ್ಪ ಹೊತ್ತಿನಲ್ಲೇ ಇವೆಲ್ಲ ಸ್ವಾಮೀಜಿಯವರಿಗೆ ಬೇಸರವುಂಟುಮಾಡಿತು. ಸೇವಿಯರ್ ದಂಪತಿಗಳ ಕಿವಿಯಲ್ಲಿ ಅವರು ಪಿಸುಗುಟ್ಟಿದರು, “ಈ ಬೆಡಗು, ಈ ವೈಭವ, ಈ ಡಾಂಭಿಕ ಪ್ರದರ್ಶನ–ಇವೆಲ್ಲ ಏಕೆ? ಇಂತಹ ಆಡಂಬರ-ಅಲಂಕಾರಗಳನ್ನು ಪ್ರದರ್ಶಿಸುವ ಚರ್ಚು, ದೀನ ಏಸುಕ್ರಿಸ್ತನ ನಿಜವಾದ ಅನು ಯಾಯಿಯಾಗಲು ಹೇಗೆ ಸಾಧ್ಯ?” ಕ್ರಿಸ್ತನು ಬೋಧಿಸಿದ ಸಂನ್ಯಾಸದ ಮಹಾ ಆದರ್ಶಕ್ಕೂ ಸೈಂಟ್ ಪೀಟರ್ಸ್ ಚರ್ಚಿನ ಬಾಹ್ಯಾಡಂಬರಕ್ಕೂ ನಡುವೆ ಮುಚ್ಚಲಾಗದ ಕಂದಕವಿರುವಂತೆ ಸ್ವಾಮೀಜಿಯವರಿಗೆ ತೋರಿತು.

ಕ್ರಿಸ್​ಮಸ್ ಹಬ್ಬದ ಸಂಭ್ರಮದ ವಾತಾವರಣದಿಂದ ಆವೃತರಾಗಿದ್ದ ಸ್ವಾಮೀಜಿ, ಏಸುಕ್ರಿಸ್ತನ ಸ್ಮರಣೆಯಲ್ಲಿ ತೊಡಗಿ, ಕ್ರಿಸ್ತಭಾವದಲ್ಲಿ ಮುಳುಗಿದ್ದರು. ಕ್ರೈಸ್ತಧರ್ಮೀಯರ ಪರಮ ಪವಿತ್ರ ಕ್ಷೇತ್ರವಾದ ವ್ಯಾಟಿಕನ್, ಕ್ರಿಸ್​ಮಸ್​ನ ಸಮಯದಲ್ಲಿ ಹೇಗೆ ಕಂಗೊಳಿಸುತ್ತಿರುತ್ತದೆ ಎಂಬುದು ಯಾರಿಗೆ ತಿಳಿದಿದೆಯೋ ಅವರಿಗೆ ಸ್ವಾಮೀಜಿಯವರ ಸಂತೋಷವೆಷ್ಟಿದ್ದಿರಬಹುದೆಂಬುದರ ಕಲ್ಪನೆಯಾದೀತು. ಕೆಲವೊಮ್ಮೆ ಅವರು ಬಾಲಕ್ರಿಸ್ತನ ಬಗ್ಗೆ ತುಂಬ ಭಾವಭರಿತರಾಗಿ ಮಾತನಾಡು ತ್ತಿದ್ದರು. ಕ್ರಿಸ್ತನ ಜನನದ ಕುರಿತಾದ ಕಥೆಗಳೊಂದಿಗೆ ಹಿಂದೂಗಳ ಪ್ರಿಯ ದೇವನಾದ ಶ್ರೀಕೃಷ್ಣನ ಜನನದ ಕುರಿತಾದ ಕಥೆಗಳನ್ನು ಹೋಲಿಸುತ್ತಿದ್ದರು.

ಹೀಗೆ, ರೋಮ್ ನಗರದ ಭೇಟಿಯು ಸ್ವಾಮೀಜಿಯವರ ಪಾಲಿಗೊಂದು ಅತ್ಯಾನಂದದ, ಅವಿಸ್ಮರಣೀಯವಾದ ಅನುಭವವಾಗಿ ಪರಿಣಮಿಸಿತು. ಅಂತೆಯೇ ಅವರ ಜೊತೆಯಲ್ಲಿದ್ದ ವರಿಗೂ ಕೂಡ. ಒಂದು ವಾರದ ಕಾಲ ಈ ರಸಾನುಭವವನ್ನು ಸವಿದ ಸ್ವಾಮೀಜಿ, ತಮ್ಮ ಸಂಗಡಿಗರಾದ ಸೇವಿಯರ್ ದಂಪತಿಗಳೊಂದಿಗೆ ರೋಮ್ ನಗರಕ್ಕೆ ವಿದಾಯ ಹೇಳಿ ಹೊರಟರು.

ಆದರೆ ರೋಮ್​ನಿಂದ ಹೊರಡುವಾಗ ಸ್ವಾಮೀಜಿಯವರಿಗೆ ದುಃಖವೇನೂ ಆಗಲಿಲ್ಲ. ಏಕೆಂದರೆ ಒಂದೊಂದು ದಿನ ಕಳೆದಂತೆಯೂ ತಾವು ತಮ್ಮ ಅತ್ಯಂತ ಪ್ರಿಯ ವಸ್ತುವಿಗೆ– ಭಾರತಕ್ಕೆ–ಸಮೀಪವಾಗುತ್ತಿದ್ದೇವೆಂಬ ಭಾವನೆ ಅವರಿಗೆ ಸಂತಸವುಂಟುಮಾಡುತ್ತಿತ್ತು. ಯೂರೋಪ್ ಪ್ರವಾಸದ ಕೊನೆಯ ಹಂತವಾಗಿ ಅವರು ನೇಪಲ್ಸ್ ತಲುಪಿದರು. ಅವರು ಭಾರತಕ್ಕೆ ಹೊರಡುವ ಹಡಗನ್ನೇರಬೇಕಾಗಿದ್ದುದು ಇಲ್ಲಿಯೇ. ಆದರೆ ನಿರ್ಗಮನಕ್ಕೆ ಇನ್ನೂ ಹಲವಾರು ದಿಗಳಿದ್ದುದರಿಂದ ನೇಪಲ್ಸ್ ನಗರವನ್ನೂ ಅದರ ಆಸುಪಾಸಿನ ಸ್ಥಳಗಳನ್ನೂ ನಿಧಾನವಾಗಿ ವೀಕ್ಷಿಸಲು ಸಮಯವಿತ್ತು. ನೇಪಲ್ಸ್ ಅತಿ ಸುಂದರ ನಗರಗಳಲ್ಲೊಂದು. “ಸಾಯುವ ಮೊದಲು ನೇಪಲ್ಸ್ ನೋಡು” ಎಂಬ ಗಾದೆಯೇ ಇದೆ. ಈ ನಗರದ ರಮಣೀಯ ಉದ್ಯಾನಗಳನ್ನೂ ಪುರಾತನ ಅವಶೇಷಗಳನ್ನೊಳಗೊಂಡ ಮ್ಯೂಸಿಯಮ್ಮನ್ನೂ ಸ್ವಾಮೀಜಿ ಸಂದರ್ ಶಿಸಿದರು. ಪ್ರಾಚೀನ ನಗರವಾದ ಪಾಂಪೆ ಹಾಗೂ ಅದರ ವಿನಾಶಕ್ಕೆ ಕಾರಣವಾದ ವೆಸೂವಿಯಸ್ ಅಗ್ನಿಪರ್ವತವಿರುವುದು ಇಲ್ಲಿಗೆ ಸಮೀಪದಲ್ಲಿಯೇ. ಈ ಅಗ್ನಿಪರ್ವತವನ್ನು ವೀಕ್ಷಿಸಲು ಸುತ್ತು ಬಳಸಿನ ರೈಲು ಮಾರ್ಗವನ್ನು ನಿರ್ಮಿಸಲಾಗಿದೆ. ಸ್ವಾಮೀಜಿ ತಮ್ಮ ಅನುಚರರೊಂದಿಗೆ ಪರ್ವತವ ನ್ನೇರಿದರು. ಅನೇಕ ಶತಮಾನಗಳಿಂದಲೂ ಈ ಪರ್ವತವು ಸುಪ್ತಾವಸ್ಥೆಯಲ್ಲಿದೆ. ಆದರೆ ಅಪರೂಪಕ್ಕೆ ಯಾವಾಗಲಾದರೊಮ್ಮೆ ‘ಕೆಮ್ಮು’ವುದುಂಟು. ಸ್ವಾಮೀಜಿ ಭೇಟಿ ನೀಡಿದಾಗ ಈ ವೆಸೂವಿಯಸ್ ಪರ್ವತವು, ಅವರನ್ನು ಸ್ವಾಗತಿಸಲೋ ಎಂಬಂತೆ ಒಮ್ಮೆ ಕಲ್ಲು ಮಣ್ಣುಗಳ ಪುಟ್ಟರಾಶಿಯನ್ನು ಆಗಸಕ್ಕೆ ಚಿಮ್ಮಿಸಿತು! ಮರುದಿನ ಅವರು ಪಾಂಪೆ ನಗರವಿದ್ದ ಜಾಗಕ್ಕೆ ಬೇಟಿ ಯಿತ್ತರು. ಆ ಇಡೀ ಪ್ರದೇಶವೇ ಒಂದು ಮ್ಯೂಸಿಯಂ ಎನ್ನಬಹುದು. ಅಳಿದು ಹೋದ ಪಾಂಪೆ ನಗರದ ಮನೆಗಳು, ಕಟ್ಟಡಗಳು ಮೊದಲಾದುವನ್ನೆಲ್ಲ ಈಗ ಯಥಾಸ್ಥಿತಿಯಲ್ಲಿ ಉಳಿಸಿಡಲಾಗಿದೆ. ಮಣ್ಣು-ಬೂದಿಯ ರಾಶಿಯಡಿಯಿಂದ ಇತರ ನೂರಾರು ವಸ್ತುಗಳನ್ನು ತೆಗೆದು ಅಲ್ಲಿಯೇ ಸಂರಕ್ಷಿಸಿಟ್ಟಿದ್ದಾರೆ. ಇವುಗಳನ್ನೆಲ್ಲ ಸ್ವಾಮೀಜಿ ಅತ್ಯಾಸಕ್ತಿಯಿಂದ ವೀಕ್ಷಿಸಿದರು. 

ಆದರೆ ಈ ವೇಳೆಗೆ ಅವರ ಮನಸ್ಸು ಭಾರತದತ್ತ ಹರಿಯತೊಡಗಿತ್ತು. ಹಡಗು ಎಂದು ಆಗಮಿಸೀತು, ತಾವು ಅದನ್ನೇರಿ ಹೊರಡುವುದೆಂದು?–ಎಂದು ಸ್ವಾಮೀಜಿ ಚಿಂತಿಸುತ್ತಿದ್ದರು. ಇಂಗ್ಲೆಂಡಿನ ಸೌಥಾಂಪ್ಟನ್ನಿನಿಂದ ಬರಬೇಕಾಗಿದ್ದ ಆ ಹಡಗಿನಲ್ಲೇ ಗುಡ್​ವಿನ್ನನೂ ಬರಲಿದ್ದ. ಕಡೆಗೂ ಈ ಹಡಗು ಆಗಮಿಸಿದಾಗ ಸ್ವಾಮೀಜಿ ಆನಂದದಿಂದ ಉದ್ಗರಿಸಿದರು, “ಅಹ್! ಇನ್ನೀಗ ಭಾರತದ ಕಡೆಗೆ! ನನ್ನ ಭಾರತದ ಕಡೆಗೆ!”

ಸ್ವಾಮೀಜಿ ಹಾಗೂ ಅವರ ಸಂಗಡಿಗರು ಏರಿದ ‘ಪ್ರಿನ್ಸ್ ರೆಜೆಂಟ್ ಲುಯ್ಟ್​ಪೋಲ್ಡ್​’ ಎಂಬ ಉಗಿಹಡಗು ೧೮೯೬ನೇ ಡಿಸೆಂಬರ್ ೨ಂರಂದು ನೇಪಲ್ಸಿನಿಂದ ಹೊರಟಿತು. ಅಲ್ಲಿಂದ ಕೊಲಂಬೋಗೆ ಸುಮಾರು ಹದಿನೈದು ದಿನದ ಪ್ರಯಾಣ. ಈ ಅವಧಿಯಲ್ಲಿ ಕೆಲವು ದಿನಗಳಂತೂ ಪ್ರಯಾಣ ತುಂಬ ಕಷ್ಟಕರವಾಗಿತ್ತು. ಆದರೆ ಸ್ವಾಮೀಜಿ, ಹೋದ ಸಲದಂತೆ ಈ ಸಲ ಹೆಚ್ಚಿನ ಅನಾರೋಗ್ಯಕ್ಕೆ ಗುರಿಯಾಗಲಿಲ್ಲ. ಬದಲಾಗಿ ಅವರು ಹರ್ಷಚಿತ್ತರಾಗಿಯೇ ಇದ್ದರು. ಹಡಗಿನಿಂದ ಅವರು ತಮ್ಮ ಪ್ರಿಯ ಶಿಷ್ಯೆ ಮೇರಿಗೆ ಬರೆದ ಒಂದು ಪತ್ರ ಹೀಗಿತ್ತು: “ನೇಪಲ್ಸಿನಿಂದ ಹೊರಟು ಪೋರ್ಟ್​ಸೆಡ್ ಅನ್ನು ತಲುಪುವ ಈ ನಾಲ್ಕು ದಿನಗಳ ಪ್ರಯಾಣ ಭಯಾನಕವಾಗಿತ್ತು. ಈಗಲೂ ಹಡಗು ಬಹಳ ಪ್ರಯಾಸದಿಂದ ಮುಂಬರಿಯುತ್ತಿದೆ. ಇಂತಹ ಪರಿಸ್ಥಿತಿಯಲ್ಲಿ ನಾನೀ ಪತ್ರವನ್ನು ಗೀಚುತ್ತಿದ್ದೇನೆ. ಆದ್ದರಿಂದ ನನ್ನ ಬರವಣಿಗೆಯನ್ನು ನೋಡಿ ನೀನು ಬೇಸರಿಸಿಕೊಳ್ಳಬಾರದು. ಸ್ಯೂಯೆಜ್ ಕಾಲುವೆಯಿಂದ ಏಷಿಯಾ ಪ್ರಾರಂಭ. ಅಂತೂ ಈಗ ನಾನು ಮತ್ತೆ ಏಷಿಯಾ ದಲ್ಲಿದ್ದೇನೆ. ನಾನು ಯಾರು? ಏಷಿಯಾದವನೊ, ಯೂರೋಪಿನವನೊ, ಅಮೆರಿಕದವನೊ! ನನ್ನೊಳಗೆ ಹಲವಾರು ವ್ಯಕ್ತಿತ್ವಗಳ ವಿಚಿತ್ರ ಸಮ್ಮಿಶ್ರಣವಾಗಿರುವಂತೆ ನನಗನ್ನಿಸುತ್ತದೆ.”

ಈ ಸಂದರ್ಭದಲ್ಲಿ ನಡೆದ ಒಂದು ಅದ್ಭುತ ಘಟನೆಯಿಂದಾಗಿ ಈ ಪ್ರಯಾಣವು ಚಿರ ಸ್ಮರಣೀಯವಾಗುವಂತಾಯಿತು. ಒಂದು ರಾತ್ರಿ; ಸ್ವಾಮೀಜಿ ಮಲಗಿ ಸ್ವಲ್ಪ ಹೊತ್ತಾಗಿರ ಬಹುದು. ಆಗ ಅವರಿಗೊಂದು ವಿಚಿತ್ರವಾದ ಕನಸಾಯಿತು. ಗಡ್ಡಧಾರಿಯಾದ ಪುಷಿಸದೃಶ ವೃದ್ದನೊಬ್ಬ ಅವರೆದುರಿಗೆ ನಿಂತು ಹೇಳಿದ, “ನಾನು ನಿನಗೀಗ ತೋರಿಸುತ್ತಿರುವ ಸ್ಥಳವನ್ನು ಚೆನ್ನಾಗಿ ಗಮನವಿಟ್ಟು ನೋಡು. ಈಗ ನೀನು ಕ್ರೀಟ್ ದ್ವೀಪದ ಸಮೀಪದಲ್ಲಿದ್ದೀಯೆ. ಕ್ರೈಸ್ತ ಧರ್ಮ ಪ್ರಾರಂಭವಾದದ್ದು ಈ ನಾಡಿನಲ್ಲಿಯೇ.” ಸ್ವಾಮೀಜಿ ಅಚ್ಚರಿಯಿಂದ ನೋಡುತ್ತಿದ್ದಂತೆ ಆ ವೃದ್ಧ ಹೇಳಿದ, “ನೀನು ಇಲ್ಲಿಗೆ ಬಂದು ನಮ್ಮ ಧರ್ಮವನ್ನು ಪುನಸ್ಸಂಸ್ಥಾಪಿಸು. ನಾನು ಪುರಾತನ ತೇರಪುತ್ತರಲ್ಲಿ ಒಬ್ಬ. ತೇರಪುತ್ತರ ಧರ್ಮವು ಭಾರತೀಯ ಪುಷಿಗಳ ಬೋಧನೆಗಳ ನ್ನಾಧರಿಸಿ ಜನಿಸಿದ್ದು. ನಾವು ಬೋಧಿಸಿದ ಜೀವನಾದರ್ಶಗಳು, ಸತ್ಯದರ್ಶನಗಳು ಏಸುಕ್ರಿಸ್ತನು ಬೋಧಿಸಿದ್ದವೆಂದು ಪ್ರಚಲಿತವಾಗಿವೆ. ಆದರೆ ನಿಜ ಹೇಳಬೇಕೆಂದರೆ ಏಸುಕ್ರಿಸ್ತ ಎಂಬ ಹೆಸರಿನ ವ್ಯಕ್ತಿಯೊಬ್ಬ ಹುಟ್ಟಿಯೇ ಇರಲಿಲ್ಲ. ಈ ಜಾಗವನ್ನು ಅಗೆದು ನೋಡಿದರೆ ನನ್ನ ಮಾತಿನ ಸತ್ಯಾಂಶವನ್ನು ಸಮರ್ಥಿಸುವ ಅನೇಕ ಆಧಾರಗಳು ಲಭಿಸುತ್ತವೆ.” ತಕ್ಷಣ ಸ್ವಾಮೀಜಿ ಕೇಳಿದರು: “ಅದು ಯಾವ ಜಾಗ?” ಆಗ ನೆರೆಗೂದಲಿನ ವೃದ್ಧ ಟರ್ಕಿ ದೇಶದ ಪಕ್ಕದ ಒಂದು ಪ್ರದೇಶದತ್ತ ಕೈದೋರಿ, “ಅಗೋ ನೋಡು, ಅಲ್ಲಿ” ಎಂದ. ಆ ಕ್ಷಣದಲ್ಲಿ ಸ್ವಾಮೀಜಿಯವರಿಗೆ ಎಚ್ಚರ ವಾಯಿತು. ತಕ್ಷಣವೇ ಅವರು ಹಡಗಿನ ಡೆಕ್ಕಿನ ಬಳಿಗೆ ಓಡಿದರು. ಅಲ್ಲಿ ಸಿಕ್ಕಿದ ಒಬ್ಬ ನೌಕಾಧಿಕಾರಿಯನ್ನು, “ಈಗ ವೇಳೆ ಎಷ್ಟಾಯಿತು?” ಎಂದು ಕೇಳಿದರು. “ಮಧ್ಯರಾತ್ರಿ” ಎಂದು ಉತ್ತರ ಬಂದಿತು. ಆಗ ಸ್ವಾಮೀಜಿ ಕೇಳಿದರು, “ನಾವೀಗ ಯಾವ ಪ್ರದೇಶದಲ್ಲಿದ್ದೇವೆ?” “ಕ್ರೀಟ್ ದ್ವೀಪದಿಂದ ಐವತ್ತುಮೈಲಿ ದೂರದಲ್ಲಿ.”

ಸ್ವಾಮೀಜಿ ಆ ಸ್ವಪ್ನದಲ್ಲಿ ಕಂಡದ್ದು ಒಬ್ಬ ಬೌದ್ಧ ಸಂತನನ್ನು. ಅವನೊಬ್ಬ ‘ತೇರಪುತ್ತ’; ಎಂದರೆ ಸಂಸ್ಕೃತದಲ್ಲಿ‘ಸ್ಥವಿರಪುತ್ರ’. ಸ್ಥವಿರ ಎಂದರೆ ಹಿರಿಯ ಎಂದರ್ಥ. ಆದ್ದರಿಂದ ಇಲ್ಲಿ ‘ತೇರಪುತ್ತ’ರು ಎಂದರೆ ಹಿರಿಯ ಬೌದ್ಧ ಸಂನ್ಯಾಸಿಯೋರ್ವನ ಶಿಷ್ಯರು ಎಂದರ್ಥ.

ಸ್ವಾಮೀಜಿಗೆ ಈ ಬಗೆಯ ಕನಸು ಬೀಳಲು ಕಾರಣವೇನಿದ್ದಿರಬಹುದು? ಕಳೆದ ಅನೇಕ ವಾರಗಳಿಂದ ಅವರು ಕ್ರೈಸ್ತಧರ್ಮದ ಕುರಿತಾದ ಆಲೋಚನೆ-ಮಾತುಕತೆಗಳಲ್ಲೇ ಮಗ್ನರಾಗಿ ದ್ದರು. ರೋಮ್ ನಗರದಲ್ಲಿ ಕಂಡಿದ್ದ ಹಲವಾರು ದೃಶ್ಯಾವಳಿಗಳು ಅವರ ಮೇಲೆ ಗಾಢ ಪ್ರಭಾವ ಬೀರಿದ್ದುವು. ಅಲ್ಲದೆ ಅಲ್ಲಿನ ಧಾರ್ಮಿಕ ವಿಧಿ ಹಾಗೂ ಬಾಹ್ಯಾಚರಣೆಗಳಲ್ಲಿ ಹಿಂದೂ ಆಚರಣೆ ಗಳ ಲಕ್ಷಣಗಳನ್ನು ಗುರುತಿಸಿ ಆಶ್ಚರ್ಯಚಕಿತರಾಗಿದ್ದರು. ಈ ಅಂಶಗಳೇ ಅವರ ಕನಸಿಗೆ ಕಾರಣವಾಗಿರಬಹುದಾದರೂ ಈ ಕನಸು ಅವರನ್ನು ಏಸುಕ್ರಿಸ್ತನ ಐತಿಹಾಸಿಕತೆಯ ಬಗ್ಗೆ ಆಲೋ ಚಿಸುವಂತೆ ಮಾಡಿತು. ಆದರೆ ಇದರಿಂದ ಅವರಿಗೆ ಏಸುಕ್ರಿಸ್ತನ ಮೇಲಿನ ಪೂಜ್ಯಬುದ್ಧಿಯಾಗಲಿ ಗೌರವವಾಗಲಿ ಕಿಂಚಿತ್ತೂ ಕಡಮೆಯಾಗಲಿಲ್ಲ. ಅಲ್ಲದೆ ಅವರ ಗುರು ಶ್ರೀರಾಮಕೃಷ್ಣರು ಏಸುಕ್ರಿಸ್ತನನ್ನು ಪ್ರತ್ಯಕ್ಷವಾಗಿ ಕಂಡವರಲ್ಲವೆ?

ಹಡಗು ಏಡನ್ ಬಂದರನ್ನು ತಲುಪಿತು. ಇಲ್ಲಿ ಸ್ವಾಮೀಜಿ ಗುಡ್​ವಿನ್ ಹಾಗೂ ಸೇವಿಯರ್ ದಂಪತಿಗಳೊಂದಿಗೆ ನಗರ ದರ್ಶನಕ್ಕಾಗಿ ಒಳಗೆ ಬಂದರು. ಅಲ್ಲಿ ಮಳೆ ನೀರನ್ನು ಸಂಗ್ರಹಿಸುವ ಒಂದು ವಿಶಾಲ ಜಲಾಶಯವನ್ನು ಕಂಡರು. ಅವರು ಅಲ್ಲೇ ಅಡ್ಡಾಡುತ್ತಿದ್ದಾಗ ಒಂದು ಮೋಜಿನ ಘಟನೆ ನಡೆಯಿತು. ತಮ್ಮ ಸ್ನೇಹಿತರೊಂದಿಗೆ ಮಾತನಾಡುತ್ತ ನಿಂತಿದ್ದ ಸ್ವಾಮೀಜಿಗೆ ದೂರದಲ್ಲಿ ಏನೋ ಒಂದು ದೃಶ್ಯ ಕಂಡಿತು. ತಕ್ಷಣ ಅವರು “ಒಂದು ನಿಮಿಷ, ಬಂದುಬಿಡುತ್ತೇನೆ” ಎನ್ನುತ್ತ ಅವಸರವಸರವಾಗಿ ಆ ಕಡೆಗೆ ನಡೆದುಬಿಟ್ಟರು. ಅವರು ಸೀದಾ ಅಲ್ಲಿನ ಒಂದು ಪಾನ್ ಬೀಡಾ ಅಂಗಡಿಗೆ ಹೋದರು. ಆ ಅಂಗಡಿಯವನು ಒಬ್ಬ ಹಿಂದೂಸ್ತಾನೀ ಮನುಷ್ಯ; ತದೇಕಚಿತ್ತದಿಂದ ಹುಕ್ಕ ಸೇದುತ್ತ ಕುಳಿತಿದ್ದ. ಬಹಳ ಕಾಲದ ಮೇಲೆ ಭಾರತೀಯ ಮುಖವನ್ನು ಕಂಡು ಸ್ವಾಮೀಜಿಗೆ ಬಹಳ ಸಂತೋಷವಾಗಿಬಿಟ್ಟಿತ್ತು. “ಕ್ಯಾ ಭಯ್ಯಾ!” ಎನ್ನುತ್ತ ಸ್ವಾಮೀಜಿ ಅವನೊಂದಿಗೆ ಆತ್ಮೀಯ ಸಂಭಾಷಣೆಯಲ್ಲಿ ತೊಡಗಿದರು. ಅಷ್ಟು ಹೊತ್ತಿಗೆ, ‘ಅಲ್ಲೇನಿರಬಹುದು ಅಂಥ ವಿಶೇಷ?’ ಎಂದು ಅಚ್ಚರಿ ಪಡುತ್ತ ಈ ಆಂಗ್ಲ ಸ್ನೇಹಿತರೂ ಅಲ್ಲಿಗೆ ಬಂದರು. ಅಂಗಡಿಯವನೊಂದಿಗೆ ಸಲಿಗೆಯಿಂದ ಮಾತನಾಡುತ್ತ ಸ್ವಾಮೀಜಿ, “ಅಣ್ಣಾ, ನನಗೆ ಸ್ವಲ್ಪ ನಿನ್ನ ಹುಕ್ಕಾ ಕೊಡುತ್ತೀಯಾ? ಎರಡು ದಂ ಎಳೆದು ಕೊಡುತ್ತೇನೆ” ಎಂದು ಕೇಳಿ ತೆಗೆದುಕೊಂಡು ಆನಂದರಿಂದ ಸೇದುತ್ತ ಕುಳಿತರು. ಇದನ್ನು ಕಂಡು ಆ ಪಾಶ್ಚಾತ್ಯ ಶಿಷ್ಯರಿಗೆ ತುಂಬ ತಮಾಷೆಯೆನಿಸಿತು. ಕ್ಯಾಪ್ಟನ್ ಸೇವಿಯರ್ ಗಟ್ಟಿಯಾಗಿ ನಗುತ್ತ ಹೇಳಿದರು, ”ಓಹೋ, ಈಗ ಗೊತ್ತಾಯಿತು! ನೀವು ನಮ್ಮನ್ನು ಬಿಟ್ಟು ಇದ್ದಕ್ಕಿದ್ದಂತೆ ಓಡಿ ಬಂದದ್ದು ಇದಕ್ಕಾಗಿಯೇ ಅಲ್ಲವೆ!” ಈ ಹೊತ್ತಿಗೆ, ಅಂಗಡಿಯವನಿಗೆ ಸ್ವಾಮೀಜಿ ಯಾರೆಂಬುದು ಅವರ ಸಂಗಡಿಗರಿಂದ ಗೊತ್ತಾಯಿತು.

ತಕ್ಷಣ ಅವನು ಅವರ ಪಾದಗಳಿಗೆ ಬಿದ್ದು ನಮಸ್ಕರಿಸಿದ. ಈ ಘಟನೆಯ ಬಗ್ಗೆ ಮುಂದೆ ಶ್ರೀಮತಿ ಸೇವಿಯರ್ ಹೇಳುತ್ತಾರೆ, “ಸ್ವಾಮೀಜಿ ಹುಕ್ಕಾ ಕೇಳಿದ ಬಗೆ ಹೇಗಿತ್ತೆಂದರೆ, ಆ ಅಂಗಡಿಯವನು ಅದಕ್ಕೆ ನಿರಾಕರಿಸಲು ಸಾಧ್ಯವೇ ಇರಲಿಲ್ಲ. ಅವರು ಶಿಶುಸಹಜ ಸರಳತೆಯಿಂದ, ‘ಅಣ್ಣಾ, ನನಗೆ ನಿನ್ನ ಹುಕ್ಕಾ ಕೊಡುತ್ತೀಯಾ?’ ಎಂದು ಕೇಳುವಾಗ ಅವರ ಮುಖದಲ್ಲೆದ್ದುಕಾಣು ತ್ತಿದ್ದ ನಿಷ್ಕಪಟತೆಯನ್ನು ನಾವೆಂದೂ ಮರೆಯಲಾರೆವು.”

ಈಗ ಹಡಗು ಏಡನ್ನನ್ನು ಬಿಟ್ಟು ಕೊಲಂಬೋದತ್ತ ಹೊರಟಿತು. ಇದೇ ಹಡಗಿನಲ್ಲಿ ಇಬ್ಬರು ಕ್ರೈಸ್ತ ಮಿಷನರಿಗಳು ಪ್ರಯಾಣಿಸುತ್ತಿದ್ದರು. ಹಿಂದೂ ಧರ್ಮದ ಪ್ರಬಲ ಸಮರ್ಥಕನೊಬ್ಬನು ಜೊತೆಯಲ್ಲಿರುವುದು ಅವರಿಗೆ ಸಹನೀಯವಾಗಲಿಲ್ಲವೆಂದು ತೋರುತ್ತದೆ. ಇವರು ಹಿಂದೂ ಹಾಗೂ ಕ್ರೈಸ್ತ ಧರ್ಮಗಳ ನಡುವಿನ ಭೇದಗಳ ಬಗ್ಗೆ ತಮ್ಮೊಂದಿಗೆ ಚರ್ಚಿಸುವಂತೆ ಸ್ವಾಮೀಜಿ ಯವರನ್ನು ಒತ್ತಾಯಿಸಿದರು. ಮತಾಂಧ ವ್ಯಕ್ತಿಗಳೊಂದಿಗೆ ಚರ್ಚೆಮಾಡುತ್ತ ಕಾಲಹರಣ ಮಾಡುವುದು ಸ್ವಾಮೀಜಿಯವರಿಗೆ ಸ್ವಲ್ಪವೂ ಇಷ್ಟವಿಲ್ಲದ ಕೆಲಸ. ಆದರೂ ಸೌಜನ್ಯಕ್ಕಾಗಿ ಅವರೊಂದಿಗೆ ಚರ್ಚೆಯಲ್ಲಿ ತೊಡಗಿದರು. ಆ ಮಿಷನರಿಗಳ ವಾದದ ರೀತಿ ತುಂಬ ಕರಕರೆಯಾಗು ವಂತಿತ್ತು. ಆದರೆ ಸ್ವಾಮೀಜಿ ಶಾಂತವಾಗಿಯೇ ಇದ್ದು, ಅವರ ಪ್ರತಿಯೊಂದು ವಾದವನ್ನೂ ಖಂಡಿಸಿದರು. ಇದರಿಂದ ಅವರ ಕೋಪ ಕೆರಳಿತು; ಬೇರೇನೂ ಹೇಳಲು ತೋರದೆ ಹಿಂದೂ ಗಳನ್ನೂ ಹಿಂದೂ ಧರ್ಮವನ್ನೂ ನೀಚಾವಾಚಾ ಬೈಯಲು ಶುರುಮಾಡಿದರು. ಸ್ವಾಮೀಜಿ ಮೊದಮೊದಲು ಸಾಧ್ಯವಾದಷ್ಟು ಸಹಿಸಿಕೊಂಡರು. ಅವರು ಬೈಗಳನ್ನು ನಿಲ್ಲಿಸುವ ಸೂಚನೆ ಕಾಣದಿದ್ದಾಗ ಸ್ವಾಮೀಜಿ ಶಾಂತವಾಗಿ ಅವರಲ್ಲೊಬ್ಬನ ಬಳಿಗೆ ಹೋಗಿ, ಇದ್ದಕ್ಕಿದಂತೆ ಅವನ ಕೊರಳ ಪಟ್ಟಿಯನ್ನು ಬಲವಾಗಿ ಹಿಡಿದುಕೊಂಡರು. ಬಳಿಕ ಅರ್ಧ ತಮಾಷೆಯ ಅರ್ಧ ಗಂಭೀರದ ದನಿಯಲ್ಲಿ, “ಇನ್ನು ನೀವು ನಮ್ಮ ಧರ್ಮವನ್ನು ಬೈದದ್ದೇ ಆದರೆ ನಿಮ್ಮನ್ನು ಸಮುದ್ರಕ್ಕೆ ಎಸೆದುಬಿಡುತ್ತೇನೆ!” ಎಂದರು. ಆ ಮನುಷ್ಯ ಹೆದರಿ ಗಡಗಡ ನಡುಗಿದ; ಆತನ ಗಂಟಲು ಒಣಗಿಹೋಯಿತು. “ದಯವಿಟ್ಟು ನನ್ನನ್ನು ಬಿಟ್ಟುಬಿಡು ಸ್ವಾಮಿ! ಇನ್ನು ಮೇಲೆ ಎಂದೂ ಹಾಗೆ ಮಾಡುವುದಿಲ್ಲ” ಎಂದು ಗೋಗರೆದ. ಅಂದಿನಿಂದ ಆತ ತನ್ನ ವರ್ತನೆಯನ್ನು ಬದಲಾಯಿಸಿ ಕೊಂಡ. ತಮ್ಮ ಅದುವರೆಗಿನ ದುರ್ವರ್ತನೆಗೆ ಪ್ರಾಯಶ್ಚಿತ್ತ ಮಾಡಿಕೊಳ್ಳಲೋ ಎಂಬಂತೆ ಸ್ವಾಮೀಜಿಯವರೊಂದಿಗೆ ಅತಿವಿನಯದಿಂದ ನಡೆದುಕೊಳ್ಳಲಾರಂಭಿಸಿದ!

‘ದುಷ್ಟರು ಕೆಣಕಿದಾಗ ಬುಸುಗುಟ್ಟಬೇಕು’ ಎಂಬುದು ಶ್ರೀರಾಮಕೃಷ್ಣರ ಬೋಧನೆ. ಅದಕ್ಕೆ ಅನುಗುಣವಾಗಿದೆ ಇಲ್ಲಿ ಸ್ವಾಮೀಜಿಯವರ ವರ್ತನೆ. ಇದು ಇತರರಿಗೆ ಅವರ ಸಂದೇಶ ಕೂಡ. ಬಹುಶಃ ಒಬ್ಬ ಹಿಂದೂ, ಅದರಲ್ಲೂ ಒಬ್ಬ ಸಂನ್ಯಾಸಿ, ಈ ರೀತಿಯಾಗಿ ಪ್ರತಿಕ್ರಿಯಿಸಬಹುದೆಂದು ಆ ಮಿಷನರಿಗಳು ಎಂದೂ ಊಹಿಸಿದ್ದಿರಲಾರರು! ಸಹಿಷ್ಣುತೆಯೊಂದು ಶ್ರೇಷ್ಠ ಗುಣವೇ ಆದರೂ, ಅದು ದೌರ್ಬಲ್ಯದಿಂದ ಒಡಮೂಡಿದ್ದಾದರೆ ಅದಕ್ಕೇನೂ ಬೆಲೆಯಿಲ್ಲ. ಸಹಿಷ್ಣುತೆ ಅತಿಯಾದರೆ ಅದು ದೌರ್ಬಲ್ಯವೆನಿಸಿಕೊಳ್ಳುತ್ತದೆ. ಭಾರತೀಯರಲ್ಲಿ ಕಂಡುಬರುವಂತಹ ಸಹನೆ ಹೆಚ್ಚಾಗಿ ಈ ಬಗೆಯದು. ಆದ್ದರಿಂದಲೇ ಸ್ವಾಮೀಜಿ ಕ್ಷಾತ್ರವೀರ್ಯವನ್ನು ಮೈಗೂಡಿಸಿಕೊಳ್ಳು ವಂತೆ ಭಾರತೀಯರಿಗೆ ಮತ್ತೆಮತ್ತೆ ಕರೆ ನೀಡಿದರು. ಅಭಿಮಾನಶೂನ್ಯರಾಗಿದ್ದ ಭಾರತೀಯರನ್ನು ಚುಚ್ಚಿ ಮೇಲೆಬ್ಬಿಸಲು ಪ್ರಯತ್ನಿಸಿದರು. ಅಲ್ಲದೇ ರಾಷ್ಟ್ರಾಭಿಮಾನಕ್ಕೆ ಹಾಗೂ ಸ್ವಧರ್ಮಾಭಿ ಮಾನಕ್ಕೆ ತಾವೇ ಸ್ವತಃ ಒಂದು ಜ್ವಲಂತ ಉದಾಹರಣೆಯಾದರು.

ಜನವರಿ ೧೫ ರ ಮುಂಜಾನೆಯ ಹೊತ್ತಿಗೆ, ಹಡಗು ಕೊಲಂಬೋ ನಗರವನ್ನು ಸಮೀಪಿಸಿತು. ಅರುಣೋದಯದ ಪ್ರಭೆಯಲ್ಲಿ ಕಂಗೊಳಿಸುತ್ತಿದ್ದ ಸುಂದರ ತೀರವನ್ನು ದೂರದಿಂದಲೇ ಕಾಣ ಬಹುದಾಗಿತ್ತು. (ಈಗಿನ ಶ್ರೀಲಂಕಾ ದೇಶವು ಆಗ ಬ್ರಿಟಿಷರ ಆಳ್ವಿಕೆಯಲ್ಲೇ ಇದ್ದು ಭಾರತದ ಅಂಗವೆಂದು ಪರಿಗಣಿಸಲ್ಪಟ್ಟಿತ್ತು.) ಭಾರತದ ತೀರವನ್ನು ಕಂಡು ಸ್ವಾಮೀಜಿಯವರಿಗಾದ ಆನಂದ ವರ್ಣನಾತೀತ. ಕೊಲಂಬೋದಲ್ಲೇ ಇಳಿಯುವ ಯೋಜನೆ ಅವರದ್ದಾಗಿತ್ತು. ಕೆಲವು ತಿಂಗಳ ಹಿಂದೆ ಅವರು ತಮ್ಮ ಮದ್ರಾಸಿನ ಶಿಷ್ಯರಿಗೆ ಒಂದು ಪತ್ರದಲ್ಲಿ, ತಾವು ಭಾರತಕ್ಕೆ ಮರಳಬೇಕೆಂದಿರುವ ವಿಷಯವನ್ನು ಸಾಂದರ್ಭಿಕವಾಗಿ ತಿಳಿಸಿ, ಸಿಲೋನ್ (ಶ್ರೀಲಂಕಾ) ದ್ವೀಪವ ನ್ನೊಮ್ಮೆ ಸಂದರ್ಶಿಸಿ ಮದ್ರಾಸಿಗೆ ಬರುವುದಾಗಿ ಬರೆದಿದ್ದರು. ಇಷ್ಟನ್ನು ಬಿಟ್ಟರೆ ಬೇರೆ ಯಾವ ವಿವರವನ್ನೂ ತಿಳಿಸಿರಲಿಲ್ಲ. ಅಂತೆಯೇ ಅವರೀಗ ಕೊಲಂಬೋದಲ್ಲಿ ಇಳಿಯುವ ಸಿದ್ಧತೆಯಲ್ಲಿ ದ್ದರು.

ಹಡಗು ಕೊಲಂಬೋ ಬಂದರಿನಲ್ಲಿ ಲಂಗರು ಹಾಕಿತು. ಪ್ರಯಾಣಿಕರನ್ನು ದೋಣಿಗಳಲ್ಲಿ ತೀರಕ್ಕೆ ಸಾಗಿಸುವ ಕಾರ್ಯ ಕಾರಣಾಂತರಗಳಿಂದ ತುಂಬ ತಡವಾಯಿತು. ಕಡೆಗೂ ಪ್ರಯಾಣಿಕರು ತೀರಕ್ಕೆ ಬರುವಷ್ಟರಲ್ಲಿ ಸಂಜೆಯಾಗುತ್ತ ಬಂದಿತ್ತು. ಆದರೆ ಹಡಗು ಬಂದು ನಿಂತ ಸ್ವಲ್ಪ ಹೊತ್ತಿನಲ್ಲೇ ದೋಣಿಯೊಂದರಲ್ಲಿ ಕೆಲವರು ತೀರದಿಂದ ಹಡಗಿಗೆ ಬಂದರು. ಇವರು ಸ್ವಾಮೀಜಿ ಯವರನ್ನು ಎದುರ್ಗೊಳ್ಳಲೆಂದು ಬಂದವರು. ಅವರಲ್ಲಿ ಕೊಲಂಬೋ ನಗರದ ಹಿಂದೂ ಪ್ರಮುಖರೊಂದಿಗೆ ಅವರ ಗುರುಭಾಯಿಗಳಾದ ಸ್ವಾಮಿ ನಿರಂಜನಾನಂದರೂ ಇದ್ದರು. ಇವರ ನ್ನೆಲ್ಲ ಕಂಡು ಸ್ವಾಮೀಜಿಯವರಿಗೆ ಪರಮಾಶ್ಚರ್ಯ. ತಮ್ಮನ್ನು ಎದುರ್ಗೊಳ್ಳಲು ಇಲ್ಲಿಗೆ ಯಾರಾದರೂ ಬಂದಾರೆಂದು ಅವರು ನಿರೀಕ್ಷಿಸಿಯೇ ಇರಲಿಲ್ಲ. ಅದರಲ್ಲೂ ನಿರಂಜನಾನಂದರು ಅಷ್ಟು ದೂರದ ಕಲ್ಕತ್ತದಿಂದ ಇಲ್ಲಿಯವರೆಗೂ ಬಂದುದನ್ನು ಕಂಡು ಅವರಿಗೆ ಸಂತೋಷ ಕೂಡ. ಸ್ವಾಮೀಜಿ ಹಡಗಿನ ಡೆಕ್ಕಿನಲ್ಲಿ ನಿಂತು ನೋಡುತ್ತಾರೆ–ಅವರನ್ನು ಸ್ವಾಗತಿಸಲು ಅಲ್ಲಿ ಭಾರೀ ಜನಸ್ತೋಮವೇ ಸೇರಿಬಿಟ್ಟಿದೆ!

ತಮಗೆ ದೊರಕಿದ ಈ ಸ್ವಾಗತವನ್ನು ಕಂಡ ಅವರಿಗೆ ಆಶ್ಚರ್ಯವಾದದ್ದು ತೀರ ಸಹಜವೇ. ಏಕೆಂದರೆ, ಅವರಿಗೆ ಆ ಬಗ್ಗೆ ಯಾವುದೇ ಸುಳಿವು ಸಿಕ್ಕಿರಲಿಲ್ಲ. ನಿಜ ಹೇಳಬೇಕೆಂದರೆ, ತಮ್ಮನ್ನು ಭಾರತೀಯರು ಆದರದಿಂದ ಸ್ವೀಕರಿಸಿಯಾರೆಂಬ ನಂಬಿಕೆಯೂ ಅವರಿಗಿರಲಿಲ್ಲ. ಆದರೆ ಸ್ವಾಮೀಜಿ ಭಾರತಕ್ಕೆ ಹಿಂದಿರುಗುತ್ತಿರುವ ಸುದ್ದಿ ಕಾಳ್ಗಿಚ್ಚಿನಂತೆ ದಕ್ಷಿಣ ಭಾರತದಲ್ಲೆಲ್ಲ ಹರಡಿ ದೂರದ ಸಿಲೋನನ್ನೂ ಮುಟ್ಟಿತ್ತು. ದಕ್ಷಿಣ ಭಾರತದ ಹಲವಾರು ನಗರ-ಪಟ್ಟಣಗಳ ನಾಗರಿಕರು ಅವರಿಗೆ ಅಭೂತಪೂರ್ವ ಸ್ವಾಗತ ನೀಡಿ, ಬಿನ್ನವತ್ತಳೆಗಳನ್ನು ಸಮರ್ಪಿಸಲು ಅದ್ಧೂರಿಯ ತಯಾರಿ ನಡೆಸಿದ್ದರು. ಮದ್ರಾಸು-ಕಲ್ಕತ್ತಗಳಲ್ಲಂತೂ ಜನ ಭಾವಾವೇಗಭರಿತರಾಗಿದ್ದರು. ಈಗ ಸ್ವಾಮೀಜಿ ಭಾರತದ ಜನಮನನಾಯಕರಾಗಿದ್ದರು! ಸಿಲೋನಿನಲ್ಲಿ ಅವರು ಪಡೆಯಲಿದ್ದಂತಹ ವೈಭವಪೂರ್ಣ ಸ್ವಾಗತವು, ಉತ್ತರದ ತುತ್ತ ತುದಿಯ ಆಲ್ಮೋರದವರೆಗೂ ಅವರು ನಡೆಸಲಿದ್ದ ಜೈತ್ರಯಾತ್ರೆಗೆ ನಾಂದಿಯಾಗಲಿತ್ತು.

ಸ್ವಾಮಿ ವಿವೇಕಾನಂದರು ಭಾರತಕ್ಕೆ ಮರಳಿದ ಘಟನೆಯು ಆಧುನಿಕ ಭಾರತದ ಇತಿಹಾಸ ದಲ್ಲೊಂದು ಮೈಲಿಗಲ್ಲು. ಏಕೆಂದರೆ ಸಮಗ್ರ ಭಾರತದ ಸಮಸ್ತ ಜನತೆಯೂ, ತಾವು ಆರ್ಯರು- ದ್ರಾವಿಡರು, ಉತ್ತರ ಭಾರತೀಯರು-ದಕ್ಷಿಣ ಭಾರತೀಯರು, ಹಿಂದೂಸ್ತಾನಿಗಳು-ಬಂಗಾಳಿಗಳು, ಕನ್ನಡಿಗರು-ತಮಿಳರು ಎಂಬೆಲ್ಲ ಭೇದಭಾವಗಳನ್ನೂ ಬದಿಗೊತ್ತಿ ಏಕಭಾವದಿಂದ ಅವರನ್ನು ಸ್ವಾಗತಿಸಿತು, ಅಭಿನಂದಿಸಿತು, ಗೌರವಿಸಿತು. \textbf{ಭಾರತದ ರಾಷ್ಟ್ರೀಯ ನಾಯಕನೆಂದು ಸಮಸ್ತ ಭಾರತೀಯರಿಂದ ಸಮ್ಮಾನಿತರಾದ ಪ್ರಪ್ರಥಮ ವ್ಯಕ್ತಿಯೆಂದರೆ ಸ್ವಾಮಿ ವಿವೇಕಾನಂದರು.} ಹಿಂದೂಧರ್ಮದ ನವನೂತನ ವ್ಯಾಖ್ಯಾನಕಾರರಾಗಿ, ಹಿಂದೂ ಧರ್ಮದಲ್ಲಿ ನವಚೇತನ ತುಂಬಿದ ವರಾಗಿ ಆಗಮಿಸಿದ್ದರು ಸ್ವಾಮೀಜಿ. ಭಾರತವು ಅವರಲ್ಲಿ ಯುಗಪ್ರವರ್ತನಾಚಾರ್ಯನನ್ನು ಕಂಡುಕೊಂಡಿತ್ತು. ಹೀಗಿರುವಾಗ, ಭಾರತದ ಕೋಟಿಗಟ್ಟಲೆ ಜನ ಅವರ ಆಗಮನಕ್ಕಾಗಿ ಉತ್ಕಂ ಠಿತರಾಗಿದ್ದರೆಂದರೆ ಅಚ್ಚರಿಯೇನಿದೆ? ಸರ್ವಧರ್ಮ ಸಮ್ಮೇಳನದ ಅನಂತರದ ಸುಮಾರು ಮೂರು ವರ್ಷಗಳಿಂದಲೂ ಜನರು ವೃತ್ತಪತ್ರಿಕೆಗಳ ಮೂಲಕ, ವಿವೇಕಾನಂದರು ಪಾಶ್ಚಾತ್ಯ ರಾಷ್ಟ್ರಗಳಲ್ಲಿ ಗಳಿಸಿದ ಯಶಸ್ಸಿನ ಬಗ್ಗೆ ತಿಳಿದಿದ್ದರು. ಅವಹೇಳನದ ವಸ್ತುಗಳಾಗಿದ್ದ ಭಾರತ ಹಾಗೂ ಹಿಂದೂಧರ್ಮಗಳ ಮೇಲಿದ್ದ ಕಳಂಕವನ್ನು ತೊಡೆದುಹಾಕಿ, ಅವುಗಳ ಸ್ಥಾನಮಾನವನ್ನು ಮರಳಿ ದೊರಕಿಸಿಕೊಟ್ಟ ಮಹಾನುಭಾವನನ್ನು ಜನ ಗುರುತಿಸಿದ್ದರು. ಪಾಶ್ಚಾತ್ಯ ರಾಷ್ಟ್ರಗಳಲ್ಲಿ ಆರ್ಯಧರ್ಮದ ಕೀರ್ತಿಪತಾಕೆಯನ್ನು ನೆಟ್ಟು ಹಿಂದಿರುಗಿದ ವಿಶ್ವವಿಜೇತ ವೀರಪುತ್ರನನ್ನು ಸ್ವಾಗತಿಸಲು ಭಾರತಾಂಬೆ ಕಾತರಳಾಗಿದ್ದಳು. ಸ್ವಾಮೀಜಿಯವರ ಉಪನ್ಯಾಸಗಳ ಹಾಗೂ ವೀರ ವಾಣಿಯ ಕುರಿತಾಗಿ ಓದಿ ತಿಳಿದ ವಿದ್ಯಾವಂತ ಜನತೆಯ ಕಣ್ಣಿಗೆ ಈಗ ಹಿಂದೂ ಧರ್ಮದಲ್ಲಿ ಅಡಗಿದ್ದ ಅನರ್ಘ್ಯ ರತ್ನಗಳು ಗೋಚರಿಸಿದ್ದುವು. ವೇದಾಂತವೊಂದೇ ವಿಶ್ವಾತ್ಮಕ ಧರ್ಮವಾಗಿ ನಿಲ್ಲಬಲ್ಲುದೆಂಬುದನ್ನು ವಿದ್ಯಾವಂತ ಜನರು ಹೆಚ್ಚುಹೆಚ್ಚಾಗಿ ಅರಿತುಕೊಳ್ಳತೊಡಗಿದ್ದರು. ಸ್ವಾಮೀಜಿ ಕೇವಲ ಒಬ್ಬ ಆಧ್ಯಾತ್ಮಿಕ ವ್ಯಕ್ತಿಯಲ್ಲ, ಅವರು ಪರಮ ರಾಷ್ಟ್ರಪ್ರೇಮಿಯೂ ಹೌದು ಎನ್ನುವುದು ಜನರ ಅರಿವಿಗೆ ಬಂದಿತ್ತು. ಇಂತಹ ವಿವೇಕಾನಂದರು ಈಗ ಪಾಶ್ಚಾತ್ಯ ರಾಷ್ಟ್ರಗಳಿಂದ ಹಿಂದಿರುಗಿ ಭಾರತಕ್ಕೆ ಆಗಮಿಸುತ್ತಿದ್ದಾರೆಂದು ತಿಳಿದ ಜನ ಅವರ ದರ್ಶನ ಮಾಡಿ, ಅವರ ದಿವ್ಯ ಸಂದೇಶವನ್ನು ಆಲಿಸಲು ಆತುರಗೊಂಡು ಕಾದಿದ್ದರು.

ಸ್ವಾಮೀಜಿ ಯೂರೋಪಿನಿಂದ ಭಾರತದತ್ತ ಹೊರಟಿದ್ದಾರೆಂಬ ವರ್ತಮಾನ ತಿಳಿದ ಕೂಡಲೇ ಭಾರತದ ಪ್ರಮುಖ ನಗರಗಳಲ್ಲಿ, ಅವರನ್ನು ಬರಮಾಡಿಕೊಳ್ಳಲು ಸ್ವಾಗತ ಸಮಿತಿಗಳನ್ನು ರಚಿಸಲಾಯಿತು. ಇಡೀ ದೇಶದಲ್ಲಿನ ವೃತ್ತಪತ್ರಿಕೆಗಳು ಅವರ ಸಾಧನೆಗಳನ್ನು ಶ್ಲಾಘಿಸಿ ಲೇಖನ ಗಳನ್ನು ಹಾಗೂ ಸಂಪಾದಕೀಯಗಳನ್ನು ಬರೆಯಲಾರಂಭಿಸಿದುವು. ಆದ್ದರಿಂದ ಜನಕೋಟಿ ಸ್ವಾಮೀಜಿಯವರ ಆಗಮನವನ್ನು ನಿರೀಕ್ಷಿಸುವುದು ಇನ್ನಷ್ಟು ತೀವ್ರವಾಯಿತು.

ಆದರೆ ಹಿಂದೆ ಹೇಳಿದಂತೆ, ಸ್ವಾಮೀಜಿಯವರಿಗೆ ಮಾತ್ರ ತಮ್ಮ ಗೌರವಾರ್ಥವಾಗಿ ಭಾರತ ದಲ್ಲಿ ಇಷ್ಟೆಲ್ಲ ಸಿದ್ಧತೆಗಳು ನಡೆಯುತ್ತಿವೆಯೆಂಬುದರ ಅರಿವೇ ಇರಲಿಲ್ಲ. ಅವರು ಭಾರತಕ್ಕೆ ಹೊರಡುವ ಮುನ್ನ ಶ್ರೀಮತಿ ಓಲ್ ಬುಲ್​ಳಿಗೆ ಗುಡ್​ವಿನ್ ಹೀಗೊಂದು ಪತ್ರ ಬರೆದಿದ್ದ: “... ಭಾರತದಲ್ಲಿ ತಮ್ಮನ್ನು ಹೇಗೆ ಸ್ವೀಕರಿಸುವರೋ ಎಂಬುದರ ಬಗ್ಗೆ ಸ್ವಾಮೀಜಿ ಸಂಶಯ ಹೊಂದಿದ್ದಾರೆ. ಹೆಚ್ಚಿನ ಪಕ್ಷ ತಮಗೆ ತಣ್ಣನೆಯ ಸ್ವಾಗತ ಸಿಗಬಹುದೆಂದು ಭಾವಿಸಿದ್ದಾರೆ. ಅದು ನಿಜಕ್ಕೂ ಹಾಗೆಯೇ ಆದಲ್ಲಿ ತಾವು ಕೆಲಕಾಲ ಭಾರತದಲ್ಲಿ ವಿಶ್ರಾಂತಿ ಪಡೆದು ಅಮೆರಿಕ- ಇಂಗ್ಲೆಂಡುಗಳಿಗೆ ಮತ್ತೊಮ್ಮೆ ಭೇಟಿ ನೀಡುವುದೇ ಉತ್ತಮವೆಂದು ಅವರು ಹೇಳುತ್ತಿದ್ದಾರೆ.” ಒಂದು ಬಗೆಯ ‘ವೈರಾಗ್ಯಶೀಲ’ರಾದ ಭಾರತೀಯರು ತಮ್ಮನ್ನು ಹೇಗೆ ಸ್ವೀಕರಿಸಿಯಾರೆಂಬುದರ ಬಗ್ಗೆ ಸ್ವಾಮೀಜಿಯವರಿಗೆ ಅನುಮಾನವಿದ್ದದ್ದು ಸಹಜವೇ ಎನ್ನಬೇಕು. ಆದರೆ ಮತ್ತೊಂದು ಸಂದರ್ಭದಲ್ಲಿ ಅವರು ತಮ್ಮ ಆಪ್ತ ಪಾಶ್ಚಾತ್ಯ ಶಿಷ್ಯರೊಂದಿಗೆ ಸಂಭಾಷಿಸುತ್ತ, ಭಾರತದ ಬಗ್ಗೆ ಹಾಗೂ ಹಿಂದೂ ಧರ್ಮದ ಬಗ್ಗೆ ಇಂಗ್ಲೆಂಡ್ ಅಮೆರಿಕಗಳಲ್ಲಿ ಗೌರವದ ಸ್ಥಾನವನ್ನು ದೊರಕಿಸಿ ಕೊಡಬೇಕಾದರೆ ತಾವು ಅದೆಷ್ಟು ಶ್ರಮಿಸಬೇಕಾಯಿತು ಎಂಬುದನ್ನು ವಿವರಿಸುತ್ತಿದ್ದರು. ಆಗ ಇದ್ದಕಿದ್ದಂತೆ ಭಾವಾವೇಗದಿಂದ ಅವರ ಮೈ ಕಂಪಿಸಲಾರಂಭಿಸಿತು. ಅವರು ಗಟ್ಟಿಯಾಗಿ ಹೇಳಿದರು, “ಭಾರತವು ನನ್ನ ಮಾತನ್ನು ಕೇಳಲೇಬೇಕು! ನಾನು ಸಮಗ್ರ ಭರತಖಂಡವನ್ನು ಬುಡಸಮೇತ ಅಲುಗಾಡಿಸಿಬಿಡುತ್ತೇನೆ; ರಾಷ್ಟ್ರಧಮನಿಗಳಲ್ಲಿ ವಿದ್ಯುತ್ಸಂಚಾರ ಮಾಡಿಸಿಬಿಡು ತ್ತೇನೆ. ಕಾದು ನೋಡಿ! ಭಾರತವು ನನ್ನನ್ನು ಹೇಗೆ ಸ್ವಾಗತಿಸುವುದೆಂಬುದನ್ನು ನೀವೇ ನೋಡುವಿರಿ! ನಾನಿಲ್ಲಿ ನನ್ನೆದೆಯ ರಕ್ತವನ್ನೇ ಬಸಿದು ವೇದಾಂತದ ಸಾರವನ್ನು ಧಾರಾಳವಾಗಿ ನೀಡಿರುವೆ. ಇದನ್ನು ಆ ಭಾರತ–ನನ್ನ ಭಾರತ–ಮಾತ್ರವೇ ಸರಿಯಾಗಿ ಅರಿತು ಗೌರವಿಸಬಲ್ಲದು. ಭಾರತವು ನನ್ನನ್ನು ಜಯಘೋಷದೊಂದಿಗೆ ಸ್ವಾಗತಿಸುತ್ತದೆ.” ಈ ಮಾತುಗಳು ಒಬ್ಬ ಪ್ರವಾದಿಯ ಮಾತುಗಳಾಗಿದ್ದುವು. ಅವು ಸತ್ಯವಾಗುವುದನ್ನು ನಾವು ಶೀಘ್ರದಲ್ಲೇ ನೋಡಲಿದ್ದೇವೆ.

