
\chapter{ಬಗೆಬಗೆಯ ಬೋಧನೆಗಳು}

\noindent

ಪಸಾಡೆನದಲ್ಲಿ ಸ್ವಾಮೀಜಿಯವರ ಕಾರ್ಯವು ಆಗ ಪ್ರಸಿದ್ಧವಾದ ಹಾಗೂ ವೈಭವಪೂರ್ಣವಾದ ಹೋಟೆಲೊಂದರ ಸಭಾಂಗಣದಲ್ಲಿ ಒಂದು ಉಪನ್ಯಾಸದೊಂದಿಗೆ ಪ್ರಾರಂಭವಾಯಿತು. ಅಂದಿನ ವಿಷಯ ‘ಭಕ್ತಿಯೋಗ’. ಈ ಊರಿನಲ್ಲಿ ಅವರು ಅತಿ ಹೆಚ್ಚಿನ ಸಂಖ್ಯೆ ಹಾಗೂ ಅಪೂರ್ವವಾದ ಉಪನ್ಯಾಸಗಳನ್ನು ನೀಡಿದ್ದು ಶೇಕ್ಸ್​ಪಿಯರ್ ಕ್ಲಬ್ ಎಂಬ ಮಹಿಳಾ ಸಂಘದಲ್ಲಿ. ಇಲ್ಲಿ ಅವರು ಜನವರಿ-ಫೆಬ್ರುವರಿ ತಿಂಗಳಲ್ಲಿ ಕಡೆಯಪಕ್ಷ ಹನ್ನೆರಡು ಉಪನ್ಯಾಸಗಳನ್ನು ನೀಡಿದರೆಂದು ತಿಳಿದುಬರುತ್ತದೆ. ಉಪನ್ಯಾಸದ ವಿಷಯಗಳಂತೂ ವೈವಿಧ್ಯಮಯವಾದವು. ಅವುಗಳಲ್ಲಿ ಕೆಲ ವೆಂದರೆ, ‘ಭಾರತದ ಸ್ತ್ರೀಯರು’, ‘ಪರ್ಶಿಯದ ಕಲೆ ’ (ಪುರಾತನ ಜಗತ್ತಿನಲ್ಲಿ ಪರ್ಶಿಯದ ಕಲೆ ಎಂಬುದು ಅತಿ ಶ್ರೇಷ್ಠಮಟ್ಟದ್ದು ಎಂದು ಪ್ರಸಿದ್ಧವಾಗಿತ್ತು. ಇದು ಭಾರತದ ತಾತ್ತ್ವಿಕ-ಆಧ್ಯಾ ತ್ಮಿಕ ಜ್ಞಾನಕ್ಕೆ ಹೋಲಿಕೆಯಿರುವಂಥದು), ‘ವಿಶ್ವಧರ್ಮವೊಂದರ ಆದರ್ಶ’, ‘ಯೋಗ ವಿಜ್ಞಾನ’, ‘ನನ್ನ ಜೀವನ ಮತ್ತು ಧ್ಯೇಯ ’, ‘ಆರ್ಯಜನಾಂಗ’, ‘ಬೌದ್ಧಕಾಲೀನ ಭಾರತ’ ಮತ್ತು ‘ಜಗತ್ತಿನ ಮಹಾ ಗುರುಗಳು’. ಇವುಗಳಲ್ಲದೆ ಜಡಭರತನ ಹಾಗೂ ಪ್ರಹ್ಲಾದನ ಕಥೆಗಳನ್ನೊಳಗೊಂಡಂತೆ ‘ಧಾರ್ಮಿಕ ಕಥೆಗಳು’ ಎಂಬ ವಿಷಯವಾಗಿ ಹಾಗೂ ‘ರಾಮಾಯಣ’ ಮತ್ತು ‘ಮಹಾಭಾರತ’ ಎಂಬ ವಿಷಯಗಳನ್ನೊಳಗೊಂಡು ಹಿಂದೂ ಪುರಾಣಗಳ ಕುರಿತಾಗಿ ಹಲವಾರು ಭಾಷಣಗಳನ್ನು ಮಾಡಿದರು. ಈ ಉಪನ್ಯಾಸಗಳ ಬಗ್ಗೆ ಪಸಾಡೆನದಲ್ಲಲ್ಲದೆ ಲಾಸ್​ಎಂಜಲಿಸ್ ನಗರದಲ್ಲೂ ವೃತ್ತಪತ್ರಿಕೆಗಳ ಮೂಲಕ ಪ್ರಕಟಣೆಯನ್ನು ಕೊಟ್ಟಿದ್ದರಿಂದ ದೂರದ ಸ್ಥಳಗಳಿಂದಲೂ ಜನರು ಬಂದು ಭಾಗವಹಿಸಿದರು. ಜನವರಿಯ ಅಂತ್ಯಭಾಗದಲ್ಲಿ ಸ್ವಾಮೀಜಿಯವರು ಪ್ರತಿದಿನವೆಂಬಂತೆ ಉಪನ್ಯಾಸಗಳನ್ನು ನೀಡಿದ್ದರೂ ತಮ್ಮ ಕಾರ್ಯದ ಪರಿಮಾಣದ ಬಗ್ಗೆ ಅವರಿಗೇ ತೃಪ್ತಿಯಿರಲಿಲ್ಲ. ಕ್ರಿಸ್ಟೀನಳಿಗೆ ಬರೆದ ಪತ್ರವೊಂದರಲ್ಲಿ ಹೇಳುತ್ತಾರೆ, “ಸದ್ಯಕ್ಕಂತೂ ನಾನು ಸಾಕಷ್ಟು ಪರಿಶ್ರಮ ವಹಿಸಿ ಕೆಲಸ ಮಾಡುತ್ತಿಲ್ಲ. ಏಕೆಂದರೆ, ಮಾಡಲು ನನಗೆ ಕೈತುಂಬ ಕೆಲಸವೇ ಇಲ್ಲ. ಉಪನ್ಯಾಸಕ್ಕೆ, ಮೊದಲಿನ ನೂಕುನುಗ್ಗಲು ಈಗಿಲ್ಲ. ಕಾರಣ–ಜನಗಳಿಗೆ ಹಣ ತೆತ್ತು ಬರುವ ಮನಸ್ಸಿಲ್ಲ. ಆದ್ದರಿಂದ ನಾನು ಸ್ಯಾನ್ ಫ್ರಾನ್ಸಿಸ್ಕೋಗೆ ಹೋಗುವ ಬಗ್ಗೆ ಆಲೋಚಿಸುತ್ತಿದ್ದೇನೆ. ಅದೊಂದು ಹೊಸ ಕ್ಷೇತ್ರ. ನಾನು ಬಹಳ ಬಳಲಿದ್ದೇನೆ. ಕೆಲಸದ ಬಗ್ಗೆ ಹಿಂದಿನ ಉತ್ಸಾಹ ಈಗಿಲ್ಲ.” ಇಲ್ಲಿ ಸ್ವಾಮೀಜಿಯವರು ನಿವೃತ್ತ ಜೀವನದ ಮನಸ್ಥಿತಿಯಲ್ಲಿರುವಂತೆ ಕಂಡುಬರುತ್ತದೆ. ಆದರೆ ಈ ಸ್ಥಿತಿಯಲ್ಲೂ ಅವರಿಂದ ಹೊಮ್ಮುವುದು ಮಾತ್ರ ಶಕ್ತಿ-ಉತ್ಸಾಹದ ಮಾತುಗಳೇ. ಅದೇ ಪತ್ರದಲ್ಲಿ ಅವರು ಬರೆಯುತ್ತಾರೆ:

“ನಾನು ಯಾವ ವಿಶ್ರಾಂತಿಯನ್ನೂ ಶಾಂತಿಯನ್ನೂ ಅರಸುತ್ತಿದ್ದೇನೆಯೋ ಅದು ನನಗೆಂದಿಗೂ ದೊರೆಯಲಾರದೆಂದು ನನಗನ್ನಿಸುತ್ತದೆ. ಆದರೆ ಜಗನ್ಮಾತೆ ನನ್ನ ಮೂಲಕ ಇತರರಿಗೆ ಒಳ್ಳೆಯ ದನ್ನೇ ಮಾಡುತ್ತಾಳೆ–ಕಡೇ ಪಕ್ಷ ನನ್ನ ಮಾತೃಭೂಮಿಯ ಮಟ್ಟಿಗಾದರೂ. ನಮ್ಮ ವಿಧಿಯನ್ನೇ ‘ತ್ಯಾಗ’ ಎಂದುಕೊಳ್ಳುವುದೇ ಸುಲಭವಾದುದು! ನಾವು ಪ್ರತಿಯೊಬ್ಬರೂ ಅವರವರದೇ ಆದ ರೀತಿಯಲ್ಲಿ ಬಲಿಪಶುಗಳೇ. ಒಂದು ಮಹಾ ಯಜ್ಞ ನಡೆಯುತ್ತಿದೆ; ಆದರೆ ಅದನ್ನು ಮಹಾ ತ್ಯಾಗವೆಂಬಂತಲ್ಲದೆ ಬೇರಾವ ದೃಷ್ಟಿಯಿಂದಲೂ ಅರ್ಥಮಾಡಿಕೊಳ್ಳಲಾಗದು. ಯಾರು ಸ್ವಇಚ್ಛೆ ಯಿಂದ ಇದರಲ್ಲಿ ಪಾಲ್ಗೊಳ್ಳುತ್ತಾರೋ ಅವರು ಬಹಳಷ್ಟು ಯಾತನೆಯಿಂದ ಪಾರಾಗುತ್ತಾರೆ. ಯಾರು ಅದನ್ನು ವಿರೋಧಿಸಲು ಪ್ರಯತ್ನಿಸುತ್ತಾರೋ ಅವರು ಬಲಾತ್ಕಾರವಾಗಿ ಕೆಳಕ್ಕೆ ತಳ್ಳ ಲ್ಪಟ್ಟು ಹೆಚ್ಚಿನ ಯಾತನೆಗೆ ಗುರಿಯಾಗುತ್ತಾರೆ. ನಾನಂತೂ ಸ್ವಇಚ್ಛೆಯಿಂದಲೇ ಬಲಿಯಾಗಲು ಸಿದ್ಧನಾಗಿದ್ದೇನೆ. ನಾನು ಕರ್ಮ ಮಾಡಲೆಂದೇ ಹುಟ್ಟಿದ್ದೇನೆ, ಮತ್ತು ಈ ದೇಹ ಬಿದ್ದುಹೋಗು ವವರೆಗೂ ದುಡಿಯುತ್ತಲೇ ಇರುತ್ತೇನೆ... ಕ್ರಿಸ್ಟೀನಾ, ಉತ್ಸಾಹ ತಂದುಕೊ. ಈ ಪ್ರಪಂಚದಲ್ಲಿ ನಿರಾಶೆಗಾಗಲಿ ದೌರ್ಬಲ್ಯಕ್ಕಾಗಲಿ ಸಮಯವಿಲ್ಲ. ಒಂದೋ ಶಕ್ತಿವಂತನಾಗಿರಬೇಕು, ಇಲ್ಲವೆ ಇಲ್ಲಿಂದ ಹೊರಟುಬಿಡಬೇಕು. ಇದೇ ನಿಯಮ.”

ಸ್ವಾಮೀಜಿಯವರು ‘ತಮಗೆ ಹಿಂದಿನ ಉತ್ಸಾಹ ಈಗಿಲ್ಲ’ವೆಂದು ಹೇಳಿದರೂ ಅವರ ಭಾಷಣ ಗಳಲ್ಲಿ ಮಾತ್ರ ಅದರ ಕುರುಹು ಇರಲಿಲ್ಲ. ಅವು, ಅವರು ಹಿಂದೆ ಅಮೆರಿಕೆಗೆ ಬಂದಿದ್ದಾಗ ಮಾಡಿದ ಭಾಷಣಗಳಷ್ಟೇ ಶಕ್ತಿಯುತವೂ ರಸಪೂರ್ಣವೂ ಸ್ಫೂರ್ತಿದಾಯಕವೂ ಆಗಿದ್ದುವು. ಅವುಗಳ ಪೈಕಿ ಹಲವು ಶೀಘ್ರಲಿಪಿಯ ಮೂಲಕ ಕಾಗದಕ್ಕಿಳಿದು, ನಮಗಿಂದು ತಮ್ಮ ಮೂಲ- ಸುಲಲಿತ ಶೈಲಿಯಲ್ಲಿ ಲಭ್ಯವಿದೆ. ಅವುಗಳಲ್ಲಿ ಕೆಲವಂತೂ ತಮ್ಮ ವಿಷಯ-ವಿಚಾರಗಳ ದೃಷ್ಟಿ ಯಿಂದ ಅತ್ಯಂತ ವಿಶಿಷ್ಟವಾದವು. ಅವರು ಪಸಾಡೆನದಲ್ಲಿ ನೀಡಿದ ‘ಭಾರತದ ಸ್ತ್ರೀಯರು’, ‘ನನ್ನ ಜೀವನ ಮತ್ತು ಧ್ಯೇಯ’ ಮತ್ತು ‘ಜಗತ್ತಿನ ಮಹಾಗುರುಗಳು’ ಇವುಗಳಲ್ಲಿ ಮುಖ್ಯವಾದುವೆನ್ನ ಬಹುದು. ‘ಜಗತ್ತಿನ ಮಹಾಗುರುಗಳು’ ಎಂಬುದರ ಒಂದೆರಡು ಪಂಕ್ತಿಗಳನ್ನು ಇಲ್ಲಿ ಉದಾ ಹರಿಸಬಹುದು:

“ಈ ಮಹಾಗುರುಗಳು ಭೂಮಿಯ ಮೇಲಿನ ಜೀವಂತ ದೇವರುಗಳೇ ಸರಿ. ಅವರನ್ನಲ್ಲದೆ ನಾವು ಇನ್ನಾರನ್ನು ಪೂಜಿಸಬೇಕು? ನಾನು ನನ್ನ ದೇವರನ್ನು ಮನಸ್ಸಿನಲ್ಲೇ ಕಲ್ಪಿಸಿಕೊಳ್ಳಲು ಪ್ರಯತ್ನಿಸುತ್ತೇನೆ. ಆದರೆ ಆಗ ನನಗನ್ನಿಸುತ್ತದೆ, ಆ ನನ್ನ ಕಲ್ಪನೆ ಎಷ್ಟು ಕ್ಷುದ್ರವಾದದ್ದು ಎಂದು. ನನ್ನ ಕಲ್ಪನೆಯ ಆ ದೇವರನ್ನು ಪೂಜಿಸುವುದೊಂದು ಪಾಪವೇ ಸರಿ! ಬಳಿಕ ನಾನು ಕಣ್ತೆರೆದು ಈ ಮಹಾಪುರುಷರ ನಿಜಜೀವನವನ್ನು ನೋಡುತ್ತೇನೆ; ಅದು ನನ್ನ ಕಲ್ಪನೆಯ ಯಾವುದೇ ದೇವರಿಗಿಂತಲೂ ಉನ್ನತವಾಗಿದೆ! ಏಕೆಂದರೆ, ಒಬ್ಬ ಮನುಷ್ಯ ಏನನ್ನಾದರೂ ಕದ್ದರೆ, ಅವನನ್ನು ಬೆನ್ನಟ್ಟಿ ಹಿಡಿದು ಸೆರೆಮನೆಗೆ ದೂಡುವ ನನ್ನಂಥವನೊಬ್ಬನ ದಯೆಯ ಕಲ್ಪನೆಯಾದರೂ ಎಂಥ ದಿರಬಹುದು? ಅಥವಾ ಕ್ಷಮೆಯನ್ನು ಕುರಿತ ನನ್ನ ಅತ್ಯಂತ ಉದಾತ್ತ ಕಲ್ಪನೆಯಾದರೂ ಏನಿರ ಬಹುದು? ಅದು ನನ್ನತನವನ್ನು ಮೀರಿದ್ದಂತೂ ಅಲ್ಲ! ನಿಮ್ಮಲ್ಲಿ ಯಾರು ‘ನಿಮ್ಮತನ’ವನ್ನು ಮೀರಿ ಹೊರಜಿಗಿಯಬಲ್ಲಿರಿ? ಯಾರು ನಿಮ್ಮ ಮನಸ್ಸನ್ನು ಅತಿಕ್ರಮಿಸಿ ನಿಲ್ಲಬಲ್ಲಿರಿ? ಒಬ್ಬರೂ ಇಲ್ಲ! ನೀವು ಇತರರಿಗೆ ಕೊಡಬಲ್ಲ ಪ್ರೇಮಕ್ಕಿಂತ ಹೆಚ್ಚಿನದಾದ ದೈವೀ ಪ್ರೇಮವನ್ನು ಊಹಿಸಿ ಕೊಳ್ಳಬಲ್ಲಿರಾ? ನಾವು ಯಾವುದನ್ನು ಎಂದೂ ಅನುಭವಿಸಿಲ್ಲವೋ ಅದರ ಕಲ್ಪನೆಯನ್ನು ಎಂದಿಗೂ ಮಾಡಿಕೊಳ್ಳಲಾರೆವು. ಆದ್ದರಿಂದ ಭಗವಂತನ ಒಂದು ಕಲ್ಪನೆಯನ್ನು ಮಾಡಿಕೊಳ್ಳ ಬಯಸುವ ನಮ್ಮೆಲ್ಲ ಪ್ರಯತ್ನಗಳೂ ನಿಷ್ಫಲವಾಗುತ್ತವೆ.

“ನಾವು ಈ ಮಹಾತ್ಮರಲ್ಲಿ ಕಾಣುವುದು ಗೊಡ್ಡು ಆದರ್ಶಗಳನ್ನಲ್ಲ, ಸರಳ ಸತ್ಯಗಳನ್ನು. ನಾವು ಊಹಿಸಲೂ ಸಾಧ್ಯವಿಲ್ಲದಂತಹ ಪ್ರೀತಿ, ದಯೆ, ಪರಿಶುದ್ಧತೆಗಳ ವಾಸ್ತವಿಕತೆಯನ್ನು ಇವರಲ್ಲಿ ಕಾಣಬಹುದಾಗಿದೆ. ಆದ್ದರಿಂದ ನಾವು ಈ ಮಹಾಪುರುಷರ ಪಾದಗಳಲ್ಲಿ ಶರಣಾಗಿ ಅವರನ್ನು ಭಗವಂತನೆಂದು ಆರಾಧಿಸಿದರೆ ಅದರಲ್ಲಿ ಆಶ್ಚರ್ಯವೇನು? ಅಲ್ಲದೆ, ಇನ್ನಾರೇ ಆದರೂ ಹೀಗಲ್ಲದೆ ಮತ್ತೇನು ಮಾಡಲು ಸಾಧ್ಯ? ಒಬ್ಬ ಮನುಷ್ಯ ಎಷ್ಟೇ ಮಾತನಾಡಬಹುದು; ಆದರೆ ಅವನು ಅದನ್ನು ಬಿಟ್ಟು ಬೇರೇನು ಮಾಡುತ್ತಾನೆಂದು ನಾನು ನೋಡಬೇಕು. ಮಾತನಾಡುವುದೆಲ್ಲ ವಾಸ್ತವಿಕತೆಯಲ್ಲ. ಭಗವಂತ, ಅವನ ನಿರಾಕಾರ ಸ್ವರೂಪ, ಅದು-ಇದು ಎಂದೆಲ್ಲ ಮಾತನಾಡು ವುದು ಬಹಳ ಒಳ್ಳೆಯದೇ. ಆದರೆ ಈ ಮನುಷ್ಯ ದೇವರೇ ಎಲ್ಲ ರಾಷ್ಟ್ರಗಳ, ಎಲ್ಲ ಜನಾಂಗಗಳ ನಿಜವಾದ ದೇವರುಗಳು. ಈ ದೈವೀ ವ್ಯಕ್ತಿಗಳು ಹಿಂದೆಯೂ ಪೂಜಿಸಲ್ಪಟ್ಟಿದ್ದಾರೆ, ಮತ್ತು ಮನುಷ್ಯನು ಮನುಷ್ಯನಾಗಿರುವವರೆಗೂ ಹಾಗೆಯೇ ಪೂಜಿಸಲ್ಪಡುತ್ತಾರೆ. ಸತ್ಯದ ಮೇಲಿನ ನಮ್ಮ ಭರವಸೆ-ವಿಶ್ವಾಸ ಅಲ್ಲಿಯೇ ಕೇಂದ್ರೀಕೃತವಾಗಿದೆ. ಕೈಗೆಟುಕದ ತತ್ತ್ವಗಳಿಂದ ಏನು ತಾನೆ ಪ್ರಯೋಜನ?

“ನಾನು ನಿಮಗೆ ಹೇಳಬೇಕೆಂದಿರುವುದು ಇದನ್ನು–ಏನೆಂದರೆ, ನಾನು ಸ್ವತಃ ನನ್ನ ಜೀವನ ದಲ್ಲೇ ಅವರೆಲ್ಲರನ್ನೂ ಪೂಜಿಸಲು ಸಮರ್ಥನಾಗಿದ್ದೇನೆ; ಮತ್ತು ಮುಂದೆ ಹುಟ್ಟಿಬರಲಿರು ವವರನ್ನೂ ಪೂಜಿಸಲು ಸಿದ್ಧನೆಂಬುದನ್ನು ಅರಿತಿದ್ದೇನೆ. ಒಬ್ಬ ತಾಯಿ, ತನ್ನ ಮಗ ಯಾವುದೇ ವೇಷದಲ್ಲಿ ಕಾಣಿಸಿಕೊಳ್ಳಲಿ, ಅವನನ್ನು ಗುರುತು ಹಿಡಿಯುತ್ತಾಳೆ. ಹಾಗೇನಾದರೂ ಅವಳು ಗುರುತು ಹಿಡಿಯದಿದ್ದಲ್ಲಿ ಅವಳು ಅವನ ತಾಯಿಯಲ್ಲವೆಂಬುದು ನಿಶ್ಚಯ. ಮತ್ತು ನಿಮ್ಮಲ್ಲಿ ಯಾರು ಸತ್ಯವನ್ನೂ ಭಗವಂತನನ್ನೂ ದಿವ್ಯತೆಯನ್ನೂ ಯಾರೋ ಒಬ್ಬ ವ್ಯಕ್ತಿಯಲ್ಲಿ ಹೊರತು ಬೇರಾರಲ್ಲೂ ಕಾಣಲು ಸಾಧ್ಯವಿಲ್ಲವೆಂದು ತಿಳಿದಿರುವಿರೋ ಅಂಥವರ ಬಗ್ಗೆ ನನ್ನ ತೀರ್ಮಾನ ಇದು–ನಿಮ್ಮಿಂದ ಯಾರೊಬ್ಬರಲ್ಲೂ ದಿವ್ಯತೆಯನ್ನು ಕಾಣಲು ಸಾಧ್ಯವಿಲ್ಲ ಎಂದು. ನೀವು ಕೇವಲ ಕೆಲವು ಮಾತುಗಳನ್ನು ನುಂಗಿಕೊಂಡು, ರಾಜಕೀಯ ಪಕ್ಷಗಳವರಂತೆ ನಿಮ್ಮನ್ನು ನೀವು ಯಾವುದೋ ಮತಸ್ಥರು ಎಂದು ನಂಬಿಕೊಂಡಿದ್ದೀರಿ–ಅದು ಕೇವಲ ಒಬ್ಬನ ಅಭಿಪ್ರಾಯದ ಪ್ರಶ್ನೆಯೋ ಎಂಬಂತೆ. ಆದರೆ ಧರ್ಮವೆಂದರೆ ಅದಲ್ಲವೇ ಅಲ್ಲ...

“ಈ ಭೂಮಿಯಲ್ಲಿ ಇನ್ನೂ ಇತರ ಹೆಚ್ಚು ಮಹಾತ್ಮೆಯುಳ್ಳ ವ್ಯಕ್ತಿಗಳು ಉದಿಸಲಿದ್ದಾರೆಯೆ? ಹೌದು, ಖಂಡಿತವಾಗಿಯೂ ಅಂಥವರು ಉದಿಸಲಿದ್ದಾರೆ, ಆದರೆ ಅದನ್ನೇ ನಿರೀಕ್ಷಿಸುತ್ತ ಕುಳಿತಿರ ಬೇಡಿ. ನಿಜ ಹೇಳಬೇಕೆಂದರೆ, ನಾನು ನಿಮ್ಮಲ್ಲೇ ಪ್ರತಿಯೊಬ್ಬರೂ, ಎಲ್ಲ ಪೂರ್ವ ಸಿದ್ಧಾಂತ ಗಳೂ ಸೇರಿ ಮಾಡಲ್ಪಟ್ಟ ಈ ಹೊಸ ಹಾಗೂ ನಿಜ ತತ್ತ್ವಗಳನ್ನು ಬೋಧಿಸುವ ಪ್ರವಾದಿ ಗಳಾಗಬೇಕೆಂದು ಬಯಸುತ್ತೇನೆ. ಹಿಂದಿನ ಎಲ್ಲ ಸಂದೇಶಗಳನ್ನೂ ತೆಗೆದುಕೊಳ್ಳಿ, ಅವುಗಳಿಗೆ ನಿಮ್ಮದೇ ಆದ ಸಾಕ್ಷಾತ್ಕಾರಗಳ ಅನುಭವಗಳನ್ನು ಸೇರಿಸಿ, ಮತ್ತು ಇತರರ ಪಾಲಿಗೆ ನೀವೇ ಪ್ರವಾದಿಗಳಾಗಿ. ಈ ಜಗತ್ತಿನಲ್ಲಿ ಉದಿಸಿದ ಗುರುಗಳಲ್ಲಿ ಪ್ರತಿಯೊಬ್ಬರೂ ಮಹಾತ್ಮರೇ ಸರಿ. ಅವರೆಲ್ಲರೂ ನಮಗಾಗಿ ಏನಾದರೊಂದನ್ನು ಉಳಿಸಿ ಹೋಗಿದ್ದಾರೆ; ಅವರೇ ನಮ್ಮ ಪಾಲಿಗೆ ದೇವರಾಗಿದ್ದಾರೆ. ನಾವು ಅವರಿಗೆ ಶಿರಬಾಗಿ ನಮಿಸೋಣ. ನಾವು ಅವರ ಕಿಂಕರರು. ಜೊತೆಗೆ ನಾವು ನಮಗೂ ನಮಸ್ಕರಿಸಿಕೊಳ್ಳೋಣ! ಏಕೆಂದರೆ ಅವರನ್ನು ಪ್ರವಾದಿಗಳು, ಭಗವಂತನ ಪುತ್ರರು ಎನ್ನುವುದಾದರೆ ನಾವೂ ಅದೇ ಆಗಿದ್ದೇವೆ! ಅವರು ತಮ್ಮದೇ ಆದ ರೀತಿಯಲ್ಲಿ ಪರಿಪೂರ್ಣತೆಯನ್ನು ಸಾಧಿಸಿದರು; ನಾವೀಗ ನಮ್ಮ ರೀತಿಯಲ್ಲಿ ಅದನ್ನು ಸಾಧಿಸಲಿದ್ದೇವೆ. ಈ ಕ್ಷಣದಲ್ಲಿಯೇ ನಾವು ದೃಢ ನಿರ್ಧಾರವನ್ನು ಕೈಗೊಳ್ಳೋಣ: ‘ನಾನೊಬ್ಬ ಪ್ರವಾದಿಯಾಗುತ್ತೇನೆ, ನಾನು ದಿವ್ಯಜ್ಯೋತಿಯ ದೂತನಾಗುತ್ತೇನೆ. ನಾನು ಭಗವಂತನ ಪುತ್ರನಾಗುತ್ತೇನೆ, ಅಲ್ಲ, ನಾನು ಭಗವಂತನೇ ಆಗುತ್ತೇನೆ!’ ಎಂದು.”

ಪಸಾಡೆನದಲ್ಲಿ ಸ್ವಾಮೀಜಿಯವರು ಸಾರ್ವಜನಿಕ ಭಾಷಣಗಳನ್ನು ನೀಡಿದ್ದಲ್ಲದೆ, ಆಸಕ್ತ ವಿದ್ಯಾರ್ಥಿಗಳಿಗಾಗಿ ಹಲವಾರು ಹೊರಾಂಗಣ ತರಗತಿಗಳನ್ನು ನಡೆಸಿದರು. ಸಾಧಾರಣವಾಗಿ ಇವು ನಡೆದದ್ದು ಮೀಡ್ ಸೋದರಿಯರ ಮನೆಯ ಬಳಿಯಿದ್ದ ಗುಡ್ಡದ ಮೇಲೆ. ಅದೃಷ್ಟವಶಾತ್ ಆ ದಿನಗಳ ಹವೆ ಬೆಚ್ಚಗಿದ್ದು ಅವರಿಗೆ ತುಂಬ ಅನುಕೂಲಕರವಾಗಿತ್ತು. ತಮ್ಮ ನಿರಂತರ ಕಾರ್ಯಗಳ ನಡುವೆಯೂ ಅವರು ಓದುತ್ತಲೋ ಬರೆಯುತ್ತಲೋ ತೋಟದಲ್ಲಿ ಅಡ್ಡಾಡುತ್ತಲೋ ನಿರಾಳವಾ ಗಿದ್ದುಬಿಟ್ಟರು. ನಿವೇದಿತೆಯ ಉಪಯೋಗಕ್ಕಾಗಿ ಕೆಲವೊಮ್ಮೆ ಹಿಂದೂ ಪುರಾಣಗಳ ಕಥೆಗಳನ್ನು ಬರೆಯುತ್ತ ಕುಳಿತಿರುತ್ತಿದ್ದರು. ಈ ದಿನಗಳಲ್ಲೇ ಅವರು, \eng{\textit{“Who knows How Mother Plays?”}} “ತಾಯಿ ಲೀಲೆಯಾಡುವ ಪರಿಯನ್ನು ಬಲ್ಲವರಾರು?” ಎಂಬ ಪ್ರಸಿದ್ಧ ಕವನವನ್ನು ರಚಿಸಿದರು. ಕೆಲವೊಮ್ಮೆ ತಮ್ಮ ಶಿಷ್ಯರೊಂದಿಗೆ ಸುದೀರ್ಘ ಸಂಭಾಷಣೆಯಲ್ಲಿ ತೊಡಗು ತ್ತಿದ್ದರು. ಆಗಾಗ ಅವರುಗಳೊಂದಿಗೆ ಸಣ್ಣಪುಟ್ಟ ಪ್ರವಾಸಗಳಿಗೂ ಹೋಗಿಬರುತ್ತಿದ್ದರು.

ದಕ್ಷಿಣ ಕ್ಯಾಲಿಫೋರ್ನಿಯದ ಜನರು ಸ್ವಭಾವತಃ ಪ್ರವಾಸಪ್ರಿಯರು. ಸ್ವಾಮೀಜಿಯವರ ಆತಿಥೇಯರೂ ಇದಕ್ಕೆ ಅಪವಾದವಾಗಿರಲಿಲ್ಲ. ಆದ್ದರಿಂದ ಅವರು ತಮ್ಮೊಡನೆ ಪ್ರವಾಸದಲ್ಲಿ ಭಾಗವಹಿಸುವಂತೆ ಸ್ವಾಮೀಜಿಯವರ ಮನವೊಲಿಸುವ ಪ್ರಯತ್ನ ಮಾಡಿದರು. ಆದರೆ ಸ್ವಾಮೀಜಿ ಯವರಿಗೆ ಈ ಪ್ರವಾಸಗಳೆಲ್ಲ ಅಷ್ಟೇನೂ ಪ್ರಿಯವಾದುವಲ್ಲ. ಶ್ರೀಮತಿ ಹ್ಯಾನ್ಸ್​ಬ್ರೋ ಒಂದು ಕುತೂಹಲಕರ ಘಟನೆಯನ್ನು ತಿಳಿಸುತ್ತಾಳೆ: ಒಮ್ಮೆ ಅವರೆಲ್ಲ ಒಂದು ಗುಡ್ಡದ ಮೇಲೆ ನಿಂತಿ ದ್ದಾಗ ಆಕೆಯ ಸೋದರಿ ಹೆಲೆನ್, ವಿವಿಧ ದೃಶ್ಯಗಳತ್ತ ಸ್ವಾಮೀಜಿಯವರ ಗಮನ ಸೆಳೆಯು ತ್ತಿದ್ದಳು. ಆಗ ಅವರೆನ್ನುತ್ತಾರೆ, “ಸೋದರಿ ಹೆಲೆನ್, ನನಗೆ ಈ ದೃಶ್ಯಗಳನ್ನೆಲ್ಲ ತೋರಿಸಬೇಡ. ನಾನು ಹಿಮಾಲಯವನ್ನೇ ನೋಡಿದವನು! ಅಲ್ಲದೆ ಪ್ರಕೃತಿಸೌಂದರ್ಯವನ್ನು ವೀಕ್ಷಿಸಲು ನಾನು ನಾಲ್ಕು ಹೆಜ್ಜೆಯನ್ನು ಇಡಲಾರೆ. ಆದರೆ ಒಬ್ಬ ಶ್ರೇಷ್ಠ ಮಾನವನನ್ನು ನೋಡಲು ಸಾವಿರ ಮೈಲಿ ಬೇಕಾದರೂ ನಡೆದೇನು!”

ಆದರೂ ತಮ್ಮ ವಿಶ್ವಾಸಿಗರ ಒತ್ತಾಯಕ್ಕೆ ಮಣಿದು ಸ್ವಾಮೀಜಿ ಕೆಲವು ಪ್ರವಾಸಗಳಿಗೆ ಹೋಗಿ ಬರಬೇಕಾಯಿತು. ಜನವರಿಯಲ್ಲಿ ಒಮ್ಮೆ ಅವರು ಪಸಾಡೆನದ ಬಳಿಯ ಉನ್ನತ ಗಿರಿಶೃಂಗವಾದ ಮೌಂಟ್ ಲೋವ್​ಗೆ ಪ್ರವಾಸ ಹೋದರು. ಅಂದು ಅವರ ಜೊತೆಯಲ್ಲಿದ್ದವರೆಂದರೆ ಬಾಮ್ ಗಾರ್ಟ್ ದಂಪತಿಗಳು, ಮಿಸ್ ಮೆಕ್​ಲಾಡ್ ಮತ್ತು ಶ್ರೀಮತಿ ಲೆಗೆಟ್. ಪ್ರಯಾಣದ ದಾರಿ ಯಲ್ಲಿ ಅವರು ರಾತ್ರಿಯನ್ನು ಹೋಟೆಲೊಂದರಲ್ಲಿ ಕಳೆದರು. ಮರುದಿನ ಪ್ರಯಾಣವನ್ನು ಮುಂದುವರಿಸಿ ಸುಂದರ ದೃಶ್ಯಗಳಿಂದ ಕೂಡಿದ ರೈಲುಮಾರ್ಗದಲ್ಲಿ ಪ್ರಯಾಣ ಮಾಡಿ ಬಳಿಕ ಟ್ರಾಲಿಯಲ್ಲಿ ಬೆಟ್ಟದ ತುದಿಯನ್ನು ತಲುಪಿದರು. ಅಂದು ರಾತ್ರಿ ಅಲ್ಲಿನ ವೀಕ್ಷಣಾಲಯಕ್ಕೆ ಭೇಟಿಯಿತ್ತು ದೂರದರ್ಶಕದ ಮೂಲಕ ನಕ್ಷತ್ರಗಳನ್ನು ಕಂಡು ಆನಂದಿಸಿ, ಮರುದಿನ ಬೆಳಿಗ್ಗೆ ಹಿಂದಿರುಗಿದರು.

ಮೀಡ್ ಕುಟುಂಬವರ್ಗದಲ್ಲಿದ್ದ ಒಟ್ಟು ಏಳು ಮಂದಿ ಸದಸ್ಯರಲ್ಲಿ ಇಬ್ಬರು ಕಿರಿಯರು– ಶ್ರೀಮತಿ ವೈಕಾಫ್​ಳ ಹದಿನೇಳು ವರ್ಷದ ಮಗ ರಾಲ್ಫ್ ಹಾಗೂ ಶ್ರೀಮತಿ ಹ್ಯಾನ್ಸ್​ಬ್ರೋಳ ನಾಲ್ಕು ವರ್ಷದ ಮಗಳು ಡೊರೋತಿ. ಇವರಿಬ್ಬರೂ ಸ್ವಾಮೀಜಿಯವರ ಆತ್ಮೀಯ ಗೆಳೆಯರಾಗಿ ಬಿಟ್ಟರು. ಬೆಳಗಿನ ತರಗತಿಗಳು ಇಲ್ಲದಿದ್ದ ದಿನಗಳಂದು ಸ್ವಾಮೀಜಿಯವರು ಡೊರೋತಿ, ರಾಲ್ಫ್ ಹಾಗೂ ಅವರ ಸ್ನೇಹಿತರೊಂದಿಗೆ ತೋಟದಲ್ಲಿ ಆಟವಾಡುತ್ತಿದ್ದರು. ಅವರು ಬಾಲ ಸಹಜವಾದ ರೀತಿಯಲ್ಲಿ ಮುಗ್ಧ ಮಕ್ಕಳಂತೆಯೇ ಸಂತೋಷ ಪಡುವುದಲ್ಲದೆ, ಮಕ್ಕಳ ವಿಧಾನಗಳನ್ನೂ ಭಾವನೆಗಳನ್ನೂ ಗಮನಿಸುತ್ತ ಆನಂದಿಸುತ್ತಿದ್ದರು. ಮಕ್ಕಳ ವಿದ್ಯಾಭ್ಯಾಸದ ಬಗ್ಗೆ ಅವರು ವಿಶೇಷ ಆಸಕ್ತಿ ವಹಿಸುತ್ತಿದ್ದರು. ಮಕ್ಕಳಿಗೆ ಶಿಕ್ಷೆ ವಿಧಿಸುವುದನ್ನು, ಯಾವುದೇ ರೀತಿಯಲ್ಲಿ ಹೆದರಿಕೆ ಹುಟ್ಟಿಸುವುದನ್ನು ಅವರು ವಿರೋಧಿಸುತ್ತಿದ್ದರು. ಕೆಲವೊಮ್ಮೆ ಅವರು ಮಕ್ಕಳೊಂದಿಗೆ ತೋಟದ ಷೆಡ್ಡಿನಲ್ಲಿ ಕುಳಿತು ಮಕ್ಕಳ ಚಿತ್ರಪುಸ್ತಕಗಳನ್ನು ನೋಡುತ್ತಿದ್ದರು. \eng{\textit{Alice in Wonderland}} ಮತ್ತು \eng{\textit{Alice Through the Looking Glass}} ಎಂಬ ಮಕ್ಕಳ ಪುಸ್ತಕಗಳು ಅವರಿಗೆ ತುಂಬ ಪ್ರಿಯವಾದುವು.

ರಾಲ್ಫ್​ನಿಗೆ ಸ್ವಾಮೀಜಿಯವರ ಮೇಲೆ ವಿಶೇಷ ಪ್ರೀತಿ. ಯಾವ ರೀತಿಯಲ್ಲಾದರೂ ತಾನು ಅವರಿಗೆ ನೆರವಾಗಬೇಕು ಎಂದು ಪ್ರಯತ್ನಿಸುತ್ತಿದ್ದ. ಸ್ವಾಮೀಜಿ ಅವನನ್ನು ಎಷ್ಟೋ ಸಲ ಆತ್ಮೀಯವಾಗಿ ಮಾತನಾಡಿಸುತ್ತಿದ್ದರು. ಒಮ್ಮೆ ಅವರು ಅವನನ್ನು ಕೇಳಿದರು, “ಏನು ರಾಲ್ಫ್, ನೀನು ನಿನ್ನ ಕಣ್ಣುಗಳನ್ನೇ ಕಾಣಬಲ್ಲೆಯಾ?” ರಾಲ್ಫ್ ಹೇಳಿದ, “ಇಲ್ಲ ಸ್ವಾಮೀಜಿ; ಆದರೆ ಕನ್ನಡಿಯಲ್ಲಿ ಬೇಕಾದರೆ ನೋಡಿಕೊಳ್ಳಬಹುದು.” ಆಗ ಸ್ವಾಮೀಜಿ ನುಡಿದರು, “ಹಾಗೆಯೇ ಭಗವಂತನೂ ಕೂಡ. ಅವನು ನಿನ್ನ ಕಣ್ಣುಗಳಷ್ಟೇ ಸಮೀಪದಲ್ಲಿದ್ದಾನೆ. ನೀನು ಅವನನ್ನು ಕಾಣಲಾರೆಯಾದರೂ ಅವನು ನಿನ್ನವನೇ.”

ಅವರ ಜೊತೆಯಲ್ಲಿದ್ದವರಾರಿಗೂ ‘ಇವರೊಬ್ಬರು ದೊಡ್ಡಮನುಷ್ಯರು’ ಎಂಬ ಭಾವನೆಯೇ ಉಂಟಾಗದಂತಿತ್ತು ಸ್ವಾಮೀಜಿಯವರ ನಡವಳಿಕೆ. ಮೀಡ್ ಕುಟುಂಬ ವರ್ಗದ ಹಿರಿಯರೆಲ್ಲರೂ ಅವರೊಂದಿಗೆ ಸಲಿಗೆಯಿಂದಲೇ ಇದ್ದರಾದರೂ ‘ಸ್ವತಃ ಕ್ರಿಸ್ತಭಗವಂತನೇ ನಮ್ಮ ನಡುವೆ ಇದ್ದಾನೆ’ ಎಂಬ ಭಾವನೆ ಅವರಲ್ಲಿ ಅಪ್ರಯತ್ನವಾಗಿ ಉಂಟಾಗುತ್ತಿತ್ತು. ಸ್ವಾಮೀಜಿಯವರ ಮಹಾ ಸಾನ್ನಿಧ್ಯವೂ ಅವರ ಮಾತುಗಳೂ ಅವರ ಸುತ್ತ ಒಂದು ಆಧ್ಯಾತ್ಮಿಕ ವಾತಾವರಣವನ್ನು ತಾವೇತಾವಾಗಿ ನಿರ್ಮಿಸುತ್ತಿದ್ದುವು. ಅವರ ಸಾಗರೋಪಮವಾದ ಅನುಕಂಪೆಯ ಬಳಿಗೆ ಬಂದ ವರ ದುಃಖಸಂಕಟಗಳನ್ನು ತನ್ನೊಡಲೊಳಗೆ ಸ್ವೀಕರಿಸಿ ಮರೆಯಾಗಿಸುತ್ತಿತ್ತು.

ಸ್ವಾಮೀಜಿಯವರು ಪಸಾಡೆನದಿಂದ ಹೊರಡುವ ಮೊದಲು ಒಂದು ದಿನ ಮೀಡ್ ಸೋದರಿ ಯರಿಗೆ ಹೇಳಿದರು, “ಯಾವಾಗಲೂ ಹೋಗುವಾಗ ನನ್ನ ನೆನಪಿಗಾಗಿ ಒಂದು ವಸ್ತುವನ್ನು ಬಿಟ್ಟು ಹೋಗುವುದು ನನ್ನ ಕ್ರಮ. ನಾನು ಸ್ಯಾನ್ ಫ್ರಾನ್ಸಿಸ್ಕೋಗೆ ಹೋಗುವಾಗ ಈ ನನ್ನ ಚುಟ್ಟಾ ಕೊಳವೆಯನ್ನು ಇಲ್ಲಿಯೇ ಬಿಟ್ಟು ಹೋಗುತ್ತೇನೆ” ಎಂದು. ಅದರಂತೆಯೇ ಅವರು ಅದನ್ನು ಅಲ್ಲಿಯೇ ಬಿಟ್ಟುಹೋದರು. ಅದನ್ನು ಮನೆಯವರೆಲ್ಲ ವಿಶೇಷ ಪ್ರೀತಿಯಿಂದ ಒಂದು ಅಲಂ ಕಾರದ ವಸ್ತುವಿನಂತೆ ಇಟ್ಟಿದ್ದರು. ಮುಂದೆ ಎಷ್ಟೋ ವರ್ಷಗಳ ಬಳಿಕ ಶ್ರೀಮತಿ ವೈಕಾಫ್ ಒಮ್ಮೆ ಅದನ್ನು ಅಕಾರಣವಾಗಿ ಕೈಗೆತ್ತಿಕೊಂಡಳು. ಬಹಳ ಕಾಲದಿಂದ ಆಕೆ ಯಾವುದೋ ನರಮಂಡಲದ ತೊಂದರೆಯಿಂದ ನರಳುತ್ತಿದ್ದಳು. ಅಲ್ಲದೆ ಹಲವಾರು ವೈಯಕ್ತಿಕ ಸಮಸ್ಯೆಗಳೂ ಅವಳನ್ನು ಮುತ್ತಿಕೊಂಡಿದ್ದುವು. ಕೆಲವು ದಿನಗಳಿಂದಂತೂ ಆ ತೊಂದರೆಗಳೆಲ್ಲ ವಿಪರೀತ ಹೆಚ್ಚಾಗಿ ಆಕೆ ಅತ್ಯಂತ ಅಸಹಾಯಕ ಸ್ಥಿತಿಯನ್ನು ಮುಟ್ಟಿದ್ದಳು. ಈಗ ಅವಳು ಕೊಳವೆಯನ್ನು ಕೈಗೆತ್ತಿಕೊಂಡ ತಕ್ಷಣ ಅವಳಿಗೆ ಸ್ವಾಮೀಜಿಯವರ ದನಿ ಕೇಳಿಸಿತು, \eng{“Madam, is it so hard!”} “ಮೇಡಮ್, ಅದು ನಿಜಕ್ಕೂ ಅಷ್ಟು ಕಷ್ಟವಾಗಿದೆಯೇ!” ಇದನ್ನು ಕೇಳಿ ಚಕಿತಳಾದ ಆಕೆ ತನಗರಿವಿಲ್ಲದಂತೆಯೇ ಆ ಕೊಳವೆಯನ್ನು ತನ್ನ ಹಣೆಗೆ ಒತ್ತಿಕೊಂಡಳು. ಮರುಕ್ಷಣಕ್ಕೆ ಅವಳ ಎಲ್ಲ ತೊಂದರೆಗಳೂ ಮಾಯವಾದುವು! ಅವಳಲ್ಲಿ ಒಂದು ಬಗೆಯ ಅಪೂರ್ವ ಶಾಂತಿ ಸಮಾಧಾನ ನೆಲಸಿತು! ಈ ವಿಷಯವನ್ನು ತಿಳಿದ ಆಕೆಯ ಕುಟುಂಬದವರೆಲ್ಲ ಆ ಕೊಳವೆಯು ಅವಳಿಗೇ ಸೇರಬೇಕಾದ್ದು ಎಂದು ನಿರ್ಧರಿಸಿ ಅವಳಿಗೆ ಅದನ್ನು ಕೊಟ್ಟುಬಿಟ್ಟರು. ಈಗ ಅದು ದಕ್ಷಿಣ ಕ್ಯಾಲಿಫೋರ್ನಿಯದ ವೇದಾಂತ ಸೊಸೈಟಿಯ ಅಧೀನದಲ್ಲಿದೆ.

ಸ್ವಾಮೀಜಿಯವರಿಗೆ ಈ ಮೂವರು ಮೀಡ್ ಸೋದರಿಯರೂ ಅವರ ಒಬ್ಬ ಸೋದರನೂ ತಮ್ಮಿಂದಾದ ಎಲ್ಲ ಸೇವೆಗಳನ್ನು ಸಲ್ಲಿಸಿದರು. ಅದರಲ್ಲೂ ಶ್ರೀಮತಿ ಹ್ಯಾನ್ಸ್​ಬ್ರೋ ತರಗತಿ ಗಳನ್ನು ಸೇರಿಸುವಲ್ಲಿ ಹಾಗೂ ಸ್ವಾಮೀಜಿಯವರಿಗೆ ಬೇಕಾಗುವಂತಹ ಇನ್ನಾವುದೇ ಅನುಕೂಲತೆ ಗಳನ್ನು ಒದಗಿಸುವಲ್ಲಿ ವಿಶೇಷ ಪರಿಶ್ರಮ ವಹಿಸಿದಳು.

ಈ ಅವಧಿಯಲ್ಲಿ ತಮ್ಮ ಉಪನ್ಯಾಸಗಳಲ್ಲೂ ತರಗತಿಗಳಲ್ಲೂ ಸ್ವಾಮೀಜಿಯವರು ಅನು ಷ್ಠಾನ ವೇದಾಂತ ಹಾಗೂ ಯೋಗದ ವಿಚಾರಗಳಿಗೆ ವಿಶೇಷ ಪ್ರಾಧಾನ್ಯ ನೀಡುತ್ತಿದ್ದರು. ಪ್ರತಿಯೊಬ್ಬ ಸ್ತ್ರೀಪುರುಷರಿಗೂ ತಮ್ಮ ಮನಸ್ಸಿನ ಮೇಲೆ ಸಂಪೂರ್ಣ ಹತೋಟಿಯನ್ನು ಹೊಂದುವ, ತನ್ಮೂಲಕ ಮುಕ್ತರಾಗುವ ಮಾರ್ಗವನ್ನು ತೋರಿಸಿಕೊಡಬೇಕು ಎನ್ನುವುದು ಸ್ವಾಮೀಜಿಯವರ ಅಂತರಂಗದ ಅಭಿಲಾಷೆಯಾಗಿತ್ತು. “ನಿಮ್ಮ ಮೇಲೆ ನೀವು ಹಿಡಿತವನ್ನು ಸಾಧಿಸಿಕೊಳ್ಳಿ” ಎನ್ನುವುದು ಅವರು ಮತ್ತೆ ಮತ್ತೆ ನೀಡುತ್ತಿದ್ದ ಸಂದೇಶವಾಗಿತ್ತು. ಏಕೆಂದರೆ ಅವರು ಬೋಧಿಸಲು ಬಯಸಿದ್ದು ‘ಪುರುಷ ನಿರ್ಮಾಣಕಾರಿ’ಯಾದ ಧರ್ಮವನ್ನು. ಆದ್ದರಿಂದ ಅವರು ತಮ್ಮ ಶ್ರೋತೃಗಳನ್ನು ಶಿಶುಗಳಂತೆ ನೋಡುತ್ತಿರಲಿಲ್ಲ; ಎಂದರೆ ಮಕ್ಕಳೊಡನೆ ಮಾತನಾಡುವಂತೆ ಮೃದುವಾದ ಮುಗ್ಧ ನುಡಿಗಳನ್ನೇ ಯಾವಾಗಲೂ ಬಳಸುತ್ತಿರಲಿಲ್ಲ. ಕೆಲ ವೊಮ್ಮೆ ಅಪ್ರಿಯವಾದ ಕಹಿ ಸತ್ಯವನ್ನು ಹೇಳುವುದರಿಂದ ಕೇಳುಗರಿಗೆ ಒಳಿತಾಗುತ್ತದೆ ಎಂದೆನ್ನಿಸಿ ದಾಗ, ತಮ್ಮ ಆ ಬಗೆಯ ಎದೆಗಾರಿಕೆಯ ಮಾತುಗಳಿಂದ ಕೇಳುಗರ ಮೇಲೆ ಎಂತಹ ಪರಿಣಾಮವುಂಟಾಗಬಹುದೆಂಬುದನ್ನೂ ಲೆಕ್ಕಿಸದೆ, ಆ ಸತ್ಯಗಳನ್ನು ಹೇಳಿಯೇ ಬಿಡುತ್ತಿದ್ದರು. ಬಹಳ ಹಿಂದೆಯೇ–ಅವರಿನ್ನೂ ನರೇಂದ್ರನಾಗಿದ್ದಾಗಲೇ–ಶ್ರೀರಾಮಕೃಷ್ಣರು ಅವರನ್ನು ಒರೆ ಯಿಂದ ತೆಗೆದ ಖಡ್ಗ ಎಂದು ಬಣ್ಣಿಸಿದ್ದರು. ಸ್ವಾಮೀಜಿ ತಮ್ಮನ್ನು ಎಂದೆಂದೂ ಒರೆಯೊಳಗೆ ಇರಿಸಿಕೊಂಡವರೇ ಅಲ್ಲ. ಪಸಾಡೆನದ ತಮ್ಮ ಉಪನ್ಯಾಸಗಳಲ್ಲಿ ಅವರು, ಸರಿಯಾದ ತಿಳಿವಳಿಕೆ ಯಿಲ್ಲದೆ ಭಾರತದ ಸನಾತನ ನಿತ್ಯನೂತನ ಸಂಪ್ರದಾಯಗಳನ್ನೂ ಆಚಾರ-ಪದ್ಧತಿಗಳನ್ನೂ ಅನ್ಯಾಯವಾಗಿ ಟೀಕಿಸುತ್ತಿದ್ದವರಿಗೆ ಮಾತಿನ ಚಾಟಿಯೇಟುಗಳನ್ನು ಮುಟ್ಟಿಸಿದರು. ಪಾಶ್ಚಾತ್ಯರ ಆತ್ಮಘಾತಕವಾದ ಪೊಳ್ಳು ದುರಹಂಕಾರಕ್ಕಾಗಿ ಅವರನ್ನು ಕಟುವಾಗಿ ಟೀಕಿಸಿದರು. ತಮಗೆ ಸರಿ ಯೆಂದು ತೋರಿದ್ದನ್ನು ಯಾವ ಮುಚ್ಚುಮರೆಯೂ ಇಲ್ಲದೆ ನಿರ್ದಾಕ್ಷಿಣ್ಯವಾಗಿ ಹೇಳಿಬಿಟ್ಟರು. ಒಂದು ಸಂಜೆ ಉಪನ್ಯಾಸದ ಕಾರ್ಯಕ್ರಮವಾದ ಮೇಲೆ ಮನೆಗೆ ಹಿಂದಿರುಗುವಾಗ ಅವರು ಶ್ರೀಮತಿ ಹ್ಯಾನ್ಸ್​ಬ್ರೋಳನ್ನು ಕೇಳಿದರು, “ಹೇಗಿತ್ತು ಉಪನ್ಯಾಸ?” ಆಕೆಯೆಂದಳು, “ಸ್ವಾಮೀಜಿ, ನಾನೇನೋ ಚೆನ್ನಾಗಿ ಆನಂದಿಸಿದೆ. ಆದರೆ ಕೆಲವು ಸಲ ನಿಮ್ಮ ಮಾತುಗಳಿಂದ ನೀವು ಶ್ರೋತೃ ಗಳನ್ನು ಎದುರುಹಾಕಿಕೊಳ್ಳುತ್ತೀರೇನೊ ಎನ್ನುವ ಭಯ ನನಗೆ.” ಇದಕ್ಕೆ ಸ್ವಾಮೀಜಿ ನಿರ್ಲಕ್ಷ್ಯ ದಿಂದ ನಸುನಕ್ಕು ಹೇಳುತ್ತಾರೆ, “ಮೇಡಮ್, ನ್ಯೂಯಾರ್ಕಿನಲ್ಲಿ ನಾನು (ಈ ಬಗೆಯ ಮಾತು ಗಳಿಂದ) ಸಭಾಂಗಣಕ್ಕೆ ಸಭಾಂಗಣಗಳನ್ನೇ ಖಾಲಿ ಮಾಡಿಸಿಬಿಟ್ಟಿದ್ದೆ!”

ಒಟ್ಟಿನಲ್ಲಿ ಸ್ವಾಮೀಜಿ ತಮ್ಮ ನಿರ್ಭಿಡೆಯ ಮಾತುಗಳಿಂದ ಕೆಲವರ ವೈರ ಕಟ್ಟಿಕೊಳ್ಳ ಬೇಕಾಯಿತು. ಪ್ರತಿಯಾಗಿ ಇವರೆಲ್ಲ ಸೇರಿಕೊಂಡು ಸ್ವಾಮೀಜಿಯವರ ಮೇಲೆ ಅಪವಾದದ ಮಾತುಗಳನ್ನು ಹಬ್ಬಿಸುವ ಕಾರ್ಯದಲ್ಲಿ ತೊಡಗಿದರು. ಆದರೆ ಈ ಸಲ ಅದು ಅವರ ಪ್ರಥಮ ಭೇಟಿಯ ಸಮಯದಲ್ಲಿದ್ದಷ್ಟು ತೀವ್ರವಾಗಿರಲಿಲ್ಲ. ಸ್ವಾಮೀಜಿಯವರಂತೂ ಇದರಿಂದೆಲ್ಲ ವಿಚಲಿತರಾಗಲಿಲ್ಲ. ಒಂದು ದಿನ ಬೆಳಿಗ್ಗೆ ಮೀಡ್ ಸೋದರಿಯರ ಮನೆಯಲ್ಲಿ ಅವರ ಕೆಲವು ನಿಷ್ಠಾವಂತ ಅನುಯಾಯಿಗಳು ಇಂತಹ ಒಂದು ಸುದ್ದಿಯನ್ನು ಚರ್ಚಿಸುತ್ತಿದ್ದರು. ಆಗ ಸ್ವಾಮೀಜಿ ಮೌನವಾಗಿ ಕೋಣೆಯಲ್ಲಿ ಅತ್ತಿಂದಿತ್ತ ನಡೆದಾಡುತ್ತಿದ್ದರು. ಕಡೆಗೊಮ್ಮೆ ನುಡಿದರು, “ನೋಡಿ, ನಾನು ಯಾರು–ಏನು ಎನ್ನುವುದು ನನ್ನ ಹಣೆಯ ಮೇಲೆ ಸ್ಪಷ್ಟವಾಗಿ ಬರೆದಿದೆ. ನಿಮಗದನ್ನು ಓದುವುದಕ್ಕೆ ಸಾಧ್ಯವಾದರೆ ನೀವು ಧನ್ಯರು. ಇಲ್ಲದಿದ್ದರೆ, ನಷ್ಟವೇನಿದ್ದರೂ ನಿಮ್ಮದೇ ಹೊರತು ನನ್ನದಲ್ಲ.”

೧೯೦೦ರ ಫೆಬ್ರುವರಿ ೩ರಂದು ನೀಡಿದ ‘ಜಗತ್ತಿನ ಮಹಾಗುರುಗಳು’ ಎಂಬ ಉಪನ್ಯಾಸ ದೊಂದಿಗೆ ದಕ್ಷಿಣ ಕ್ಯಾಲಿಫೋರ್ನಿಯದಲ್ಲಿ ಸ್ವಾಮೀಜಿಯವರ ಭಾಷಣಗಳ ಕಾರ್ಯಕ್ರಮ ಕೊನೆಗೊಂಡಿತು. ಸುಮಾರು ಎರಡು ತಿಂಗಳ ಅವಧಿಯಲ್ಲಿ ಅವರು ಪಸಾಡೆನ ನಗರದಲ್ಲಿ ಹಲವಾರು ಖಾಸಗಿ ತರಗತಿಗಳಲ್ಲದೆ ಸುಮಾರು ಮೂವತ್ತೆಂಟು ಉಪನ್ಯಾಸಗಳನ್ನು ನೀಡಿ ದ್ದರು!ಇದೇ ಅವಧಿಯಲ್ಲಿ ಲಾಸ್ ಏಂಜಲಿಸ್ ಹಾಗೂ ಪಸಾಡೆನಗಳಲ್ಲಿ ಅವರ ಉತ್ಸಾಹೀ ಶಿಷ್ಯರು ವೇದಾಂತ ಸೊಸೈಟಿಗಳನ್ನು ಸ್ಥಾಪಿಸಿಕೊಂಡಿದ್ದರು. ಈ ಸಂಘಗಳು ಹೆಚ್ಚು ಕಾಲ ಉಳಿದುಕೊಳ್ಳಲಿಲ್ಲವಾದರೂ ಸ್ವಾಮೀಜಿಯವರು ಬಿತ್ತಿದ್ದ ಬೋಧನೆಗಳು ಬೀಜರೂಪದಲ್ಲಿ ಉಳಿದುಕೊಂಡುವು. ಮುಂದೆ ೧೯೩ಂರಲ್ಲಿ ಸ್ವಾಮಿ ಪ್ರಭವಾನಂದರಿಂದ ‘ದಕ್ಷಿಣ ಕ್ಯಾಲಿ ಫೋರ್ನಿಯ ವೇದಾಂತ ಸಂಘ’ವು ಸ್ಥಾಪಿಸಲ್ಪಟ್ಟಿತು. ಆ ಮೂಲಕ, ಸ್ವಾಮೀಜಿಯವರು ಬಿತ್ತಿದ್ದ ಬೀಜಗಳು ಮೊಳೆತು ಬೆಳೆಯತೊಡಗಿದುವು.

