
\chapter{ಪರಮೇಶ್ವರನ ಅವತಾರ!}

\noindent

ಈ ಮಧ್ಯೆ, ಜಪಾನಿಗೆ ಬರುವಂತೆ ಸ್ವಾಮೀಜಿಯವರನ್ನು ಒತ್ತಾಯಿಸುತ್ತಿದ್ದ ಜೋಸೆಫಿನ್ ಮೆಕ್​ಲಾಡ್ ಅಲ್ಲಿ ಅವರಿಗಾಗಿ ಸಿದ್ಧತೆ ಮಾಡುತ್ತಿದ್ದಳು. ಆದರೆ ಅನಾರೋಗ್ಯದ ಕಾರಣದಿಂದಾಗಿ ಅವರಿಗೆ ಹೊರಡಲು ಸಾಧ್ಯವಾಗಲೇ ಇಲ್ಲ ಎಂಬುದನ್ನು ಹಿಂದೆಯೇ ನೋಡಿದೆವು. ಆದ್ದರಿಂದ ಮಿಸ್ ಮೆಕ್​ಲಾಡ್ ಹಾಗೂ ಆಕೆಯ ಜಪಾನೀ ಸ್ನೇಹಿತರಾದ ಒಕಾಕುರ ಮತ್ತು ಹೊರಿ–ಇವರು ಸ್ವಾಮೀಜಿಯವರನ್ನು ಭೇಟಿ ಮಾಡಲು ಹಾಗೂ ಜಪಾನಿನಲ್ಲಿ ನಡೆಸಲುದ್ದೇಶಿಸಲಾಗಿದ್ದ ಸರ್ವ ಧರ್ಮ ಸಮ್ಮೇಳನದಲ್ಲಿ ಭಾಗವಹಿಸುವಂತೆ ಅವರನ್ನು ಕೇಳಿಕೊಳ್ಳಲು ಖುದ್ದಾಗಿ ಬರಲು ನಿಶ್ಚಯಿಸಿದ್ದರು. ಒಕಾಕುರ ಜಪಾನಿನ ಪುರಾತನ ದೇವಾಲಯಗಳ ಜೀರ್ಣೋದ್ದಾರ ಸಮಿತಿಯ ಅಧ್ಯಕ್ಷ ಮತ್ತು ಕಲಾಶಾಲೆಯೊಂದರ ಮುಖ್ಯಸ್ಥ. ಹೊರಿ ಒಬ್ಬ ತರುಣ ಬೌದ್ಧಸಂನ್ಯಾಸಿ, ತುಂಬ ಉತ್ಸಾಹಿ. ಇವರಿಬ್ಬರೂ ೧೯೦೨ರ ಜನವರಿ ೬ರಂದು ಸ್ವಾಮೀಜಿಯವರನ್ನು ಭೇಟಿಯಾದರು. ಪ್ರಥಮ ಭೇಟಿಯಲ್ಲಿ ಒಕಾಕುರ ಹಾಗೂ ಸ್ವಾಮೀಜಿ ಪರಸ್ಪರ ಆಕರ್ಷಿತರಾದರು. ಕಲೆ ಹಾಗೂ ಧರ್ಮದ ವಿಚಾರಗಳಲ್ಲಿ ಇಬ್ಬರದೂ ಸಮಾನಾಭಿಪ್ರಾಯ. ಬೌದ್ಧಧರ್ಮದ ಬಗ್ಗೆ ಸ್ವಾಮೀಜಿ ಯವರಿಗಿದ್ದ ಅಭಿಮಾನ ಸರ್ವವಿದಿತ. ಒಕಾಕುರನಿಗೆ ಅವರು ಹೇಳಿದರು, “ನಾವಿಬ್ಬರೂ ಜಗತ್ತಿನ ಎರಡು ಮೂಲೆಗಳಿಂದ ಬಂದು ಭೇಟಿಯಾಗುತ್ತಿರುವ ಸಹೋದರರು” ಎಂದು. ಒಕಾಕುರ ತನ್ನ ಜಪಾನೀ ಸ್ನೇಹಿತನಾದ ರೆವರೆಂಡ್ ಒಡಾ ಟೊಕುನೋಗೆ ಬರೆದ ಪತ್ರದಲ್ಲಿ ತಿಳಿಸುತ್ತಾನೆ–“ನಾವು ಸ್ವಾಮಿ ವಿವೇಕಾನಂದರನ್ನು ಭೇಟಿಯಾದೆವು. ಅವರು ಅತ್ಯಂತ ಉತ್ಸಾಹಿಗಳು, ಮಹಾಪಂಡಿತರು. ಅವರೆಂತಹ ಗಣ್ಯವ್ಯಕ್ತಿಯೆಂದರೆ ಇಡೀ ಜಗತ್ತಿನ ಜನ ಅವರನ್ನು ಗೌರವಿಸುತ್ತಾರೆ.” ಒಮ್ಮೆ ಒಕಾಕುರ ಮಿಸ್ ಮೆಕ್​ಲಾಡಳಿಗೆ ಸ್ವಲ್ಪ ಕಠಿಣವಾಗಿಯೇ ಹೇಳಿದ, “ವಿವೇಕಾನಂದರು ನಮ್ಮವರು (ಪೌರ್ವಾತ್ಯರು ); ಅವರು ನಿಮ್ಮವರಲ್ಲ!” ಆತನಿಗೆ ಸ್ವಾಮೀಜಿಯವರ ಮೇಲಿದ್ದ ಗಾಢ ಅಭಿಮಾನವನ್ನು ಆ ಮಾತು ಸೂಚಿಸುತ್ತದೆ. ಆದರೆ ನಿಜಕ್ಕೂ ಸ್ವಾಮೀಜಿಯವರು ಕೇವಲ ಪೌರ್ವಾತ್ಯರೂ ಅಲ್ಲ, ಪಾಶ್ಚಾತ್ಯರೂ ಅಲ್ಲ; ಅವರೇ ಹೇಳಿದಂತೆ, ಅವರು ಸಮಸ್ತ ವಿಶ್ವಕ್ಕೇ ಸೇರಿದವರು–ವಿಶ್ವಮಾನವರು.

ಒಕಾಕುರ ಭಾರತಕ್ಕೆ ಬಂದದ್ದರ ಉದ್ದೇಶ ಸ್ವಾಮೀಜಿಯವರನ್ನು ಧರ್ಮಸಮ್ಮೇಳನಕ್ಕೆ ಆಹ್ವಾನಿಸುವುದು ಮಾತ್ರವಲ್ಲದೆ, ಜಪಾನಿನ ಧರ್ಮ-ಸಂಸ್ಕೃತಿಗಳ ಉಗಮಸ್ಥಾನವಾದ ಭಾರತದ ತೀರ್ಥಕ್ಷೇತ್ರಗಳನ್ನು ಸಂದರ್ಶಿಸುವುದೂ ಆಗಿತ್ತು. ಇವುಗಳಲ್ಲದೆ ಬುದ್ಧಗಯೆಯ ಮಹಾಬೋಧಿ ದೇವಸ್ಥಾನದ ಆಡಳಿತವನ್ನು ಹಿಂದೂಗಳ ಕೈಯಿಂದ ತಾವು ಪಡೆದುಕೊಳ್ಳಬೇಕೆಂಬ ಇಚ್ಛೆಯೂ ಅವನಿಗಿತ್ತು. ಈ ವೇಳೆಗೆ ಸ್ವಾಮೀಜಿ ತಮ್ಮ ಆರೋಗ್ಯಸುಧಾರಣೆಗಾಗಿ ವಾರಾಣಸಿಗೆ ಹೋಗಿರಲು ನಿಶ್ಚಯಿಸಿದ್ದರು. ಬುದ್ಧಗಯೆ ವಾರಾಣಸಿಗೆ ಸಮೀಪ. ಆದ್ದರಿಂದ ಅವನು ತನ್ನೊಡನೆ ಬುದ್ಧ ಗಯೆಗೆ ಬರುವಂತೆ ವಿನಂತಿಸಿಕೊಂಡಾಗ ಸ್ವಾಮೀಜಿ ಸಂತೋಷದಿಂದ ಸಮ್ಮತಿಸಿದರು. ಅಲ್ಲದೆ ಬುದ್ಧನು ನಿರ್ಯಾಣವನ್ನು ಹೊಂದಿದ ಪರಮ ಪವಿತ್ರ ಸ್ಥಳ ಬುದ್ಧಗಯೆ; ಅವನು ತನ್ನ ಬೋಧನೆಯನ್ನು ಮೊಟ್ಟಮೊದಲು ನೀಡಿದ ಸ್ಥಳ ವಾರಾಣಸಿ. ಆದ್ದರಿಂದ ಈ ಸ್ಥಳಗಳಿಗೆ ಭೇಟಿ ನೀಡುವುದು ಅವರಿಗೆ ವಿಶೇಷ ಆನಂದದ ಸಂಗತಿಯಾಗಿತ್ತು.

ಮಠದಲ್ಲಿ ಮೂರು ವಾರಗಳನ್ನು ಕಳೆದನಂತರ ಜನವರಿ ೨೭ರಂದು ಒಕಾಕುರನು ಸ್ವಾಮೀಜಿ ಹಾಗೂ ಮಿಸ್ ಮೆಕ್​ಲಾಡರೊಂದಿಗೆ ಬುದ್ಧಗಯೆಗೆ ಹೊರಟ. ಜೊತೆಯಲ್ಲಿ ಕೆಲವು ಕಿರಿಯ ಸ್ವಾಮಿಗಳಿದ್ದರು. ಪ್ರಯಾಣ ಅತ್ಯಂತ ಆನಂದಕರವಾಗಿತ್ತು. ಮರುದಿನ ಗಯೆಯನ್ನು ತಲುಪಿ ದರು. ಜನವರಿ ೨೯–ಸ್ವಾಮೀಜಿಯವರ ಹುಟ್ಟುಹಬ್ಬ. ಅವರೇ ಮುನ್ನುಡಿದಿದ್ದಂತೆ ಅವರ ಜೀವಿತಾವಧಿಯ ಕಡೆಯ ವರ್ಷ.

ಬಾರತದ ಅಂದಿನ ವೈಸ್​ರಾಯನಾದ ಲಾರ್ಡ್ ಕರ್ಜನನಿಂದ ಒಕಾಕುರ ಒಂದು ಪತ್ರವನ್ನು ತಂದಿದ್ದನಲ್ಲದೆ ಮುಂಚಿತವಾಗಿಯೇ ಗಯೆಯ ಸರ್ಕಾರೀ ಅಧಿಕಾರಿಗಳಿಗೆ ತಂತಿ ಕಳಿಸಿದ್ದ. ಆದ್ದರಿಂದ ರೈಲು ನಿಲ್ದಾಣಕ್ಕೆ ಕೆಲವು ಅಧಿಕಾರಿಗಳು ಬಂದಿದ್ದು ಅವರನ್ನು ಆದರದಿಂದ ಸ್ವಾಗತಿಸಿ ಕರೆದೊಯ್ದರು. ಅಲ್ಲಿನ ಪ್ರವಾಸೀ ಮಂದಿರದಲ್ಲಿ ಅವರಿಗೆಲ್ಲ ವಸತಿ ಕಲ್ಪಿಸಲಾಯಿತು.

ಗಯೆಯಲ್ಲಿ ಸ್ವಾಮೀಜಿ ವಿಷ್ಣುಪಾದಪದ್ಮವನ್ನು ದರ್ಶಿಸಿದರು. ಬಳಿಕ ಎಲ್ಲರೂ ಸಾರೋಟಿ ನಲ್ಲಿ ಕುಳಿತು ಬುದ್ಧಗಯೆಗೆ ತೆರಳಿದರು. ಅಲ್ಲಿನ ದೇವಾಲಯದ ಮಹಾದ್ವಾರದ ಬಳಿಗೆ ಸಾರೋಟು ಹೋಗುತ್ತಿದ್ದಂತೆಯೇ ಅಲ್ಲಿನ ಮಹಂತರು ತಮ್ಮ ಅನುಯಾಯಿಗಳೊಂದಿಗೆ ಬಂದು ಸ್ವಾಮೀಜಿಯವರಿಗೆ ಸಾಷ್ಟಾಂಗ ಪ್ರಣಾಮ ಮಾಡಿ ಸ್ವಾಗತಿಸಿದರು.

ಬುದ್ಧಗಯೆಯಲ್ಲಿ ಸ್ವಾಮೀಜಿ ಸುಮಾರು ಒಂದು ವಾರ ಇದ್ದರು. ಪ್ರತಿದಿನವೂ ಮಹಂತರು ಬಂದು ಅವರೊಂದಿಗೆ ಸುಮಾರು ಎರಡು ಗಂಟೆಗಳ ಕಾಲ ವಿವಿಧ ಧಾರ್ಮಿಕ ವಿಚಾರಗಳ ಬಗ್ಗೆ ಚರ್ಚಿಸಿದರು. ಅಲ್ಲದೆ ಸ್ವಾಮೀಜಿಯವರು ತಮ್ಮ ಸಂಗಡಿಗರೊಂದಿಗೆ ಅನೇಕ ಬಾರಿ ಅಲ್ಲಿನ ದೇವಸ್ಥಾನಕ್ಕೆ ಹೋದರು; ಅಲ್ಲಿನ ಪ್ರತಿಯೊಂದು ವಿಗ್ರಹದ ವಾಸ್ತುಶಿಲ್ಪವನ್ನೂ ಚರಿತ್ರೆಯನ್ನೂ ಅವರಿಗೆಲ್ಲ ವಿವರಿಸಿದರು.

ಈ ಅವಧಿಯಲ್ಲಿ ಸ್ವಾಮೀಜಿ ಬೌದ್ಧಧರ್ಮದ ಇತಿಹಾಸದ ಬಗ್ಗೆ ಹೊಸ ಅಧ್ಯಯನವನ್ನು ಕೈಗೊಂಡರು. ಆಗ ಅವರಿಗೆ ಅನೇಕಾನೇಕ ಹೊಸ ವಿಷಯಗಳು ಗೋಚರವಾದುವು. ಹೀಗೆ ಬುದ್ಧ ಗಯೆಯಲ್ಲಿ ಒಂದು ವಾರವನ್ನು ಕಳೆದು ಸ್ವಾಮೀಜಿ ತಮ್ಮ ಸಂಗಡಿಗರೊಂದಿಗೆ ವಾರಾಣಸಿಗೆ ತೆರಳಿದರು.

ಕಾಶೀ ಕ್ಷೇತ್ರದ ರೈಲು ನಿಲ್ದಾಣದಲ್ಲಿ ಸುಮಾರು ಐನೂರು ಜನ ಅವರನ್ನು ಜಯಘೋಷ ದೊಂದಿಗೆ ಬರಮಾಡಿಕೊಂಡರು. ಯಾಮಿನೀ ರಂಜನ್ ಮಜುಮ್ದಾರ್ ಮತ್ತು ಚಾರುಚಂದ್ರ ದಾಸ್​–ಇವರು ಜನರ ಮುಂದಾಳ್ತನ ವಹಿಸಿದ್ದರು. ವಾರಾಣಸಿಯಲ್ಲಿ ‘ಗೋಪಾಲ್ ಲಾಲ್ ವಿಲ್ಲಾ’ ಎಂಬ ಭವ್ಯವಾದ ಬಂಗಲೆಯಲ್ಲಿ ಸ್ವಾಮೀಜಿ ಇಳಿದುಕೊಳ್ಳಲು ಏರ್ಪಾಡಾಗಿತ್ತು. ಈ ಊರಿನ ಒಣಹವೆಯಿಂದ ತಮ್ಮ ಆರೋಗ್ಯ ಸುಧಾರಿಸಬಹುದೆಂಬ ಆಶಯ ಸ್ವಾಮೀಜಿಯವರ ದಾಗಿತ್ತು.

ಇಲ್ಲಿ ಅವರನ್ನು ಭೇಟಿಯಾಗಲು ಪ್ರತಿದಿನವೂ ಬಹಳ ಜನ ಬರುತ್ತಿದ್ದರು. ಊರಿನ ಅನೇಕ ಮಹಂತರು, ಸಂನ್ಯಾಸಿಗಳು, ಸಂಪ್ರದಾಯಸ್ಥ ಪಂಡಿತರು ಬಂದು ಅವರೊಂದಿಗೆ ಶಾಸ್ತ್ರವಿಚಾರ ವಾಗಿ ಚರ್ಚಿಸಿದರು. ಹಿಂದೂಧರ್ಮದ ಹಲವಾರು ಸಂಪ್ರದಾಯಗಳಿಗೆ ಸ್ವಾಮೀಜಿ ವಿರುದ್ಧ ವಾಗಿದ್ದು ಅನೇಕ ಪ್ರಗತಿಪರ ಧೋರಣೆಗಳ ಪ್ರತಿಪಾದಕರಾಗಿದ್ದರೂ ಮತ್ತು ಮ್ಲೇಚ್ಛರೊಂದಿಗೆ ಬೆರೆತವರಾಗಿದ್ದರೂ ವಾರಾಣಸಿಯ ಪಂಡಿತರು ಅವರ ಬಗ್ಗೆ ಪೂಜ್ಯ ಭಾವನೆಯನ್ನೇ ಇಟ್ಟು ಕೊಂಡದ್ದು ಸ್ಪಷ್ಟವಾಗಿತ್ತು. ಸ್ವತಃ ಸ್ವಾಮೀಜಿಯವರಿಗೇ ಇದು ಅಚ್ಚರಿಯನ್ನುಂಟುಮಾಡಿತು. ಬಹುಶಃ ಈ ಸಂಪ್ರದಾಯವಾದಿಗಳ ಧೋರಣೆಯೂ ಬದಲಾಗುತ್ತಿದ್ದಿರಬೇಕು. ಅಲ್ಲದೆ ಕಾಶಿಯ ವಿಶ್ವನಾಥ ದೇವಾಲಯದಲ್ಲಿ ಸ್ವಾಮೀಜಿಯವರಿಗೆ ಮಾತ್ರವಲ್ಲದೆ ಅವರ ವಿದೇಶೀ ಸ್ನೇಹಿತನಾದ ಒಕಾಕುರನಿಗೂ ಸ್ವಾಗತ ದೊರೆಯಿತು. ಅಷ್ಟೇ ಅಲ್ಲದೆ, ಶಿವಲಿಂಗವನ್ನು ಸ್ಪರ್ಶಿಸಿ ಪೂಜೆ ಮಾಡಲೂ ಅವನಿಗೆ ಅವಕಾಶ ದೊರೆಯಿತು. ವಿದೇಶೀಯನೂ ಬೌದ್ಧನೂ ಆದ ಆತನಿಗೆ ದೊರಕಿದ ಈ ಅವಕಾಶ ಅಪೂರ್ವವಾದುದೆನ್ನಬೇಕು.

ಒಂದು ದಿನ ಅಲ್ಲಿನ ಕೇದಾರನಾಥ ದೇವಾಲಯದ ಮಹಂತರು ಸ್ವಾಮೀಜಿಯವರ ದರ್ಶನ ಕ್ಕಾಗಿ ಬಂದರು. ಅವರಿಗೆ ಸುಮಾರು ಎಂಬತ್ತು ವರ್ಷ ವಯಸ್ಸು. ಅವರು ವಯೋವೃದ್ಧರಷ್ಟೇ ಅಲ್ಲದೆ ಜ್ಞಾನವೃದ್ಧರು ಕೂಡ. ಅವರು ಬಂದಿದ್ದಾರೆಂಬ ವಿಷಯ ತಿಳಿದ ಕೂಡಲೇ ಸ್ವಾಮೀಜಿ ಯವರು ಅತ್ಯಂತ ಪೂಜ್ಯಭಾವದಿಂದ ಅವರನ್ನು ಭೇಟಿಯಾಗಿ “ನಮೋ ನಾರಾಯಣಾಯ!” ಎಂದು ಅಭಿವಾದಿಸಿದರು. ಆ ಮಹಂತರು ಹಾಗೆಯೇ ಪ್ರತಿನಮಸ್ಕಾರ ಮಾಡಿ ಸಂಸ್ಕೃತದ ಒಂದು ಶ್ಲೋಕವನ್ನು ಪಠಿಸಿದರು. ಬಳಿಕ ತಮಿಳಿನಲ್ಲಿ (ಅದು ಅವರ ಮಾತೃಭಾಷೆ) ಸ್ವಾಮೀಜಿಯವರನ್ನು ಸ್ತುತಿಸಲಾರಂಭಿಸಿದರು. ಅವರ ಮಾತುಗಳನ್ನು ಒಬ್ಬರು ಸಿಂಹಳೀ ಸಂನ್ಯಾಸಿಗಳು ಭಾಷಾಂತರಿಸಿ ಹೇಳಿದರು:

“ಸ್ವಾಮೀಜಿ, ನೀವು ಸಾಕ್ಷಾತ್ ಪರಮೇಶ್ವರನ ಅವತಾರ! ನೀವು ಮಾನವಕೋಟಿಯ ಉದ್ಧಾರಕ್ಕಾಗಿ ಜನ್ಮವೆತ್ತಿದ್ದೀರಿ. ಯೂರೋಪು ಅಮೆರಿಕಗಳಲ್ಲಿ ಪ್ರಕಟವಾದ ನಿಮ್ಮ ಶಕ್ತಿಯು ನಾಗರಿಕತೆಯ ಇತಿಹಾಸದಲ್ಲೇ ಅಭೂತಪೂರ್ವವಾದುದು. ನೀವು ಪೌರ್ವಾತ್ಯಧ್ವಜವನ್ನು ಪಾಶ್ಚಾತ್ಯರ ಮುಂದೆ ಎತ್ತಿಹಿಡಿದು ಪ್ರತಿಯೊಬ್ಬ ಹಿಂದುವೂ, ಪ್ರತಿಯೊಬ್ಬ ಸಂನ್ಯಾಸಿಯೂ ಹೆಮ್ಮೆ ಪಡುವಂತೆ ಮಾಡಿದ್ದೀರಿ. ವೇದಗಳ ಬಗ್ಗೆ ನೀವು ನೀಡಿದ ವ್ಯಾಖ್ಯಾನಗಳಿಗಾಗಿ ಮತ್ತು ಅವುಗಳ ವಿಶ್ವವ್ಯಾಪಕತೆಯನ್ನು ತೋರಿಸಿಕೊಟ್ಟುದಕ್ಕಾಗಿ ಹಿಂದೂಗಳು ಮತ್ತು ವಿಶೇಷವಾಗಿ ಸಂನ್ಯಾಸಿಗಳು ನಿಮಗೆ ಅತ್ಯಂತ ಆಭಾರಿಗಳಾಗಿದ್ದೇವೆ... ”

ಆ ವೃದ್ಧ ಸಂನ್ಯಾಸಿಗಳು ಹೀಗೆ ಭಾವಾವೇಶಭರಿತರಾಗಿ ತಮ್ಮ ಗುಣಗಾನ ಮಾಡತೊಡಗಿದಾಗ ಸ್ವಾಮೀಜಿ ತುಂಬ ಸಂಕೋಚಪಟ್ಟುಕೊಂಡು “ಮಹಾರಾಜ್, ನಾನು ಹೆಚ್ಚಿನದೇನನ್ನೂ ಮಾಡಿಲ್ಲ. ನಾನು ಏನಾದರೂ ಸ್ವಲ್ಪ ಕೆಲಸ ಮಾಡಿದ್ದರೆ ಅದು ಭಗವಂತನ ಕೃಪೆಯಿಂದ ಅಷ್ಟೆ. ಅವೆಲ್ಲವೂ ಅವನದೇ ಇಚ್ಛೆ. ಅವನದೇ ಕಾರ್ಯ. ಅದರಲ್ಲೇನಾದರೂ ಹೆಚ್ಚುಗಾರಿಕೆಯಿದ್ದರೆ ಅದರ ಕೀರ್ತಿಗೆ ನಾನು ಎಳ್ಳಷ್ಟೂ ಪಾತ್ರನಲ್ಲ. ಈ ನನ್ನ ದೇಹವೇನಿದ್ದರೂ ಅವನ ಉಪಕರಣ, ವಾಹನ ಅಷ್ಟೆ. ನೀವು ಮಹಾಜ್ಞಾನವನ್ನು ಪಡೆದುಕೊಂಡವರು. ನಿಮ್ಮಂಥವರು ನನ್ನ ಮೇಲೆ ಕೃಪೆದೋರಿ ಆಶೀರ್ವದಿಸಿದರೆ ಅಂತಹ ಎಷ್ಟೋ ಕೆಲಸಗಳು ಲೀಲಾಜಾಲವಾಗಿ ನಡೆದುಹೋಗುತ್ತವೆ. ಅಲ್ಲದೆ ನೀವು ಕೇದಾರನಾಥನ ಹಿರಿಯ ಅರ್ಚಕರು. ನಿಜಕ್ಕೂ ನೀವು ಶಿವನ ಅವತಾರ. ನಾನೊಬ್ಬ ನರ ಮಾತ್ರ” ಎಂದು ವಿನಯದಿಂದ ನುಡಿದರು.

ಆದರೆ ಆ ಮಹಂತರು ಸ್ವಾಮೀಜಿಯವರ ಮಾತುಗಳನ್ನು ಲಕ್ಷಿಸದೆ ಮತ್ತೆ ನುಡಿದರು, “ನೀವು ಭಾರತಕ್ಕೆ ಮರಳಿದ ಮೇಲೆ ರಾಮೇಶ್ವರದಿಂದ ಉತ್ತರಾಭಿಮುಖವಾಗಿ ಹೋಗುತ್ತಿದ್ದಾಗ ಅಲ್ಲಿನ ನಮ್ಮ ಮಠದ ಮಹಂತರು ನಿಮ್ಮನ್ನು ಮಠಕ್ಕೆ ಬರಮಾಡಿಕೊಳ್ಳಲು ಪಲ್ಲಕ್ಕಿಯನ್ನು ಕಳಿಸಿಕೊಟ್ಟಿ ದ್ದರು. ಆದರೆ ಆಗ ಜನಸಾಗರವೇ ನಿಮ್ಮ ದರ್ಶನಕ್ಕಾಗಿ ಕಾದಿತ್ತು. ಅಲ್ಲದೆ ನೀವೂ ತುಂಬ ದಣಿದಿದ್ದಿರಿ. ಆದ್ದರಿಂದ ನಿಮಗೆ ಬರಲು ಸಾಧ್ಯವಾಗಲಿಲ್ಲ. ಆಗ ನಮ್ಮ ಗುರುಗಳಿಗೆ ತುಂಬ ದುಃಖವಾಯಿತು. ಅದನ್ನೀಗ ಸರಿಪಡಿಸಲು ಈಗ ನಿಮ್ಮನ್ನು ಇಲ್ಲಿನ ಮಠಕ್ಕೆ ಬರಮಾಡಿಕೊಳ್ಳ ಬೇಕು ಎಂದು ಅಲ್ಲಿಂದ ನನಗೆ ತಂತಿ ಬಂದಿದೆ. ಆದ್ದರಿಂದ ಈಗ ನೀವು ನಿಮ್ಮ ಸಹಸಂನ್ಯಾಸಿ ಗಳೊಂದಿಗೆ, ಕಡೆಯ ಪಕ್ಷ ಒಂದು ದಿನವಾದರೂ ನಮ್ಮಲ್ಲಿಗೆ ಬಂದು ಭಿಕ್ಷೆಯನ್ನು ಸ್ವೀಕರಿಸ ಬೇಕೆಂದು ಪ್ರಾರ್ಥಿಸುತ್ತೇನೆ.”

ಇಂತಹ ವೃದ್ಧರಾದ ಸಾಧುಗಳು ತಮ್ಮನ್ನು ಅಷ್ಟೊಂದು ವಿನಯದಿಂದ ಪ್ರಾರ್ಥಿಸಿಕೊಂಡ ದ್ದನ್ನು ಕಂಡಾಗ ಸ್ವಾಮೀಜಿಯವರಿಗೆ ತುಂಬ ನಾಚಿಕೆಯಾಯಿತು. ಅವರೊಬ್ಬ ಬಾಲಕನಂತೆ ನಮ್ರಭಾವದಿಂದ, “ನೀವು ಹಾಗೆ ಬಯಸಿದ್ದರೆ, ನನಗೆ ಒಂದು ಮಾತನ್ನು ಯಾರ ಮೂಲಕ ವಾದರೂ ಹೇಳಿಕಳಿಸಿದ್ದರೂ ಆಗುತ್ತಿತ್ತು. ನಾನು ಸಂತೋಷದಿಂದ ಬಂದುಬಿಡುತ್ತಿದ್ದೆ. ನೀವು ಇಷ್ಟೆಲ್ಲ ತೊಂದರೆ ತೆಗೆದುಕೊಳ್ಳಬೇಕಾಗಿಯೇ ಇರಲಿಲ್ಲವಲ್ಲ!... ಆಗಲಿ ಮಹಾರಾಜ್, ನಾನು ನಾಳೆ ದಿನ ಬರುತ್ತೇನೆ” ಎಂದರು. ಅದರಂತೆಯೇ ಮರುದಿನ ತಮ್ಮ ಸಹಸಂನ್ಯಾಸಿ ಗಳೊಂದಿಗೆ ಮಠಕ್ಕೆ ಹೋಗಿ ಭಿಕ್ಷೆ ಸ್ವೀಕರಿಸಿದರು.

ಸ್ವಾಮೀಜಿಯವರ ಬುದ್ಧಗಯೆಯ ಹಾಗೂ ವಾರಾಣಸಿಯ ಈ ಭೇಟಿ ಅತ್ಯಂತ ವಿಶಿಷ್ಟವೂ ಮಹತ್ವಪೂರ್ಣವೂ ಆಗಿತ್ತು. ಸೋದರಿ ನಿವೇದಿತೆ ಹೇಳುವಂತೆ, ಅವರ ಎಲ್ಲ ಪರಿವ್ರಜನಕ್ಕೂ ಇದೊಂದು ಅತ್ಯಂತ ಸೂಕ್ತ ಮುಕ್ತಾಯವಾಗಿತ್ತು. ಬುದ್ಧಗಯೆಯಲ್ಲೂ ವಾರಾಣಸಿಯಲ್ಲೂ ಸಂಪ್ರದಾಯಸ್ಥ ಜನರು ಅವರ ಬಗ್ಗೆ ತೋರಿದ ಪ್ರೀತಿ ವಿಶ್ವಾಸ ಎಷ್ಟು ಆಳವಾಗಿತ್ತೆಂದರೆ ಸ್ವತಃ ಸ್ವಾಮೀಜಿಯವರೂ ಅದನ್ನು ಕಂಡು ದಂಗಾದರು. ಸ್ವಾಮೀಜಿ ಹಿಂದೆ ನರೇಂದ್ರನಾಗಿದ್ದಾಗ ತಾರಕ ಹಾಗೂ ಕಾಳೀಪ್ರಸಾದರೊಂದಿಗೆ ಗಯೆಗೆ ಬಂದಿದ್ದರು; ಈಗ ವಿವೇಕಾನಂದರಾಗಿ ಮತ್ತೊಮ್ಮೆ ಬಂದಿದ್ದಾರೆ. ಹೀಗೆ, ಅವರು ಸಂದರ್ಶಿಸಿದ್ದ ಪ್ರಥಮ ತೀರ್ಥಕ್ಷೇತ್ರ ಬುದ್ಧಗಯೆ, ಈಗ ಅವರು ಕೈಗೊಂಡ ಕಡೆಯ ತೀರ್ಥಯಾತ್ರೆಯೂ ಬುದ್ಧಗಯೆಗೇ ಎಂಬುದು ವಿಶೇಷವೇ ಸರಿ.

ಇದಲ್ಲದೆ ಇಲ್ಲಿ ನಾವು ಮತ್ತೊಂದು ವಿಷಯವನ್ನು ಸ್ಮರಿಸಬಹುದು. ಹಿಂದೆ ಅವರಿನ್ನೂ ಒಬ್ಬ ಅನಾಮಧೇಯ ಸಂನ್ಯಾಸಿಯಾಗಿದ್ದಾಗ, ದೀರ್ಘ ಪರಿವ್ರಜನವನ್ನು ಕೈಗೊಳ್ಳುವ ಮುನ್ನ ವಾರಾಣಸಿಗೆ ಬಂದಿದ್ದರು. ತಮ್ಮ ಉದಾತ್ತ-ಭವ್ಯ ಯೋಜನೆಗಳಿಗೆ ಸಂಪ್ರದಾಯಬದ್ಧ ಸಮಾಜದ ಪ್ರೋತ್ಸಾಹವಾಗಲಿ ನೆರವಾಗಲಿ ಕಿಂಚಿತ್ತೂ ದೊರಕಲಿಲ್ಲವೆಂಬ ಆಕ್ರೋಶದಿಂದ ಅವರು, ಅಲ್ಲಿಂದ ಹೊರಟು ನಿಂತಾಗ ಪ್ರಮದದಾಸ ಮಿತ್ರ ಮತ್ತಿತರರ ಮುಂದೆ ‘ನಾನು ಈ ಸಮಾಜದ ಮೇಲೆ ಒಂದು ಬಾಂಬಿನಂತೆ ಸಿಡಿದೆರಗುವವರೆಗೂ ಇಲ್ಲಿಗೆ ಮತ್ತೆ ಬರುವುದಿಲ್ಲ’ ಎಂದು ಉದ್ಗರಿಸಿದ್ದರು. ಅವರ ಆ ಮಾತು ಇಂದು ನಿಜವಾಗಿರುವುದನ್ನು ಕಾಣುತ್ತಿದ್ದೇವೆ.

ಸ್ವಾಮೀಜಿಯವರು ಯುವಕರ ಮುಂದಿಟ್ಟಂತಹ ‘ತ್ಯಾಗ ಮತ್ತು ಸೇವೆ’ ಹಾಗೂ ‘ಜೀವರಲ್ಲಿ ಶಿವನ ಕಾಣು’ ಎಂಬ ಆದರ್ಶಗಳಿಂದ ಉತ್ಸಾಹಿತರಾದ ಕೆಲವು ಬಂಗಾಳೀ ಯುವಕರು ವಾರಾ ಣಸಿಯಲ್ಲಿ ಒಂದು ಸಂಘವನ್ನು ನಿರ್ಮಿಸಿಕೊಂಡಿದ್ದರು. ಇದರ ಮುಖಂಡರೆಂದರೆ ಚಾರುಚಂದ್ರ ದಾಸ್ (ಮುಂದೆ ಸ್ವಾಮಿ ಶುಭಾನಂದ), ಕೇದಾರನಾಥ ಮೌಲಿಕ್ (ಮುಂದೆ ಸ್ವಾಮಿ ಅಚಲಾನಂದ) ಹಾಗೂ ಯಾಮಿನೀ ರಂಜನ್ ಮಜುಮ್​ದಾರ್. ಇದು ನಡೆದದ್ದು ೧೯ಂಂರ ಜೂನಿನಲ್ಲಿ. ಇವರ ಮುಖ್ಯ ಉದ್ದೇಶವೆಂದರೆ ಕಾಶೀ ಕ್ಷೇತ್ರಕ್ಕೆ ಧಾವಿಸುವ ಸಹಸ್ರಾರು ಬಡಯಾತ್ರಿಕರಲ್ಲಿ ಕೆಲವರಿ ಗಾದರೂ ಆಶ್ರಯವೊದಗಿಸುವುದು. ಈ ಉದ್ದೇಶಕ್ಕಾಗಿ ಇವರು ಒಂದು ಪುಟ್ಟ ಮನೆಯನ್ನು ಬಾಡಿಗೆಗೆ ತೆಗೆದುಕೊಂಡು ತಮ್ಮ ಕೈಯಲ್ಲಿದ್ದ ಅಲ್ಪಸ್ವಲ್ಪ ಹಣದಿಂದಲೇ ಬಡಜನರ ಸೇವೆಯಲ್ಲಿ ತೊಡಗಿದರು. ನಿರಾಶ್ರಿತ ಯಾತ್ರಿಕರಿಗೆ, ನಿರ್ಗತಿಕ ವಿಧವೆಯರಿಗೆ ಹಾಗೂ ಆ ಊರಿನ ಬೀದಿಗಳಲ್ಲಿ ರೋಗಪೀಡಿತರಾಗಿ ಮಲಗಿರುವ ವೃದ್ಧರಿಗೆ ಆಹಾರ, ಆಶ್ರಯ ಹಾಗೂ ವೈದ್ಯಕೀಯ ನೆರವನ್ನು ನೀಡಿದರು. ತಮ್ಮ ಸಂಘಕ್ಕೆ ಈ ಯುವಕರು “ಬಡಜನರ ಪರಿಹಾರ ಸಂಘ” ಎಂಬ ಹೆಸರಿಟ್ಟಿ ದ್ದರು. ಇವರು ತಮ್ಮ ಸೇವಾಕಾರ್ಯದೊಂದಿಗೆ ಸ್ವಾಮೀಜಿಯವರ ‘ಜ್ಞಾನಯೋಗ’, ‘ಕರ್ಮ ಯೋಗ’ಗಳನ್ನು ಓದುತ್ತ ನವಸ್ಫೂರ್ತಿಯನ್ನು ಪಡೆದುಕೊಂಡರು. ಅತಿ ಸಣ್ಣ ಪ್ರಮಾಣದಲ್ಲಿ ಪ್ರಾರಂಭವಾದ ಈ ಸೇವಾಸಂಸ್ಥೆಯು ಎರಡು ವರ್ಷಗಳಲ್ಲಿ ಬಹುಮಟ್ಟಿಗೆ ಬೆಳೆಯಿತು. ಚಾರುಚಂದ್ರನ ಬುದ್ಧಿ ಮತ್ತೆ, ಯಾಮಿನೀರಂಜನನ ಉತ್ಸಾಹ, ಕೇದಾರನಾಥನ ಸೇವಾ ಮನೋ ಭಾವ ಹಾಗೂ ಇತರ ಸ್ನೇಹಿತರ ಸಹಾಯ ಸಹಕಾರಗಳು ಸೇರಿದುವು. ಇದರ ಬಗ್ಗೆ ಕೇಳಿ ತಿಳಿದ ಸ್ವಾಮೀಜಿ ಈ ಹಿಂದೆಯೇ ತಮ್ಮ ಆನಂದವನ್ನು ವ್ಯಕ್ತಪಡಿಸಿದ್ದರು. ಈಗ ಅವರ ಕಾರ್ಯಕಲಾಪ ಗಳನ್ನು ಕಣ್ಣಾರೆ ಕಂಡಾಗ ಸ್ವಾಮೀಜಿಯವರ ಹೃದಯಕ್ಕಾದ ಸಂತಸ ಸಮಾಧಾನ ವರ್ಣನಾತೀತ.

ಸ್ವಾಮೀಜಿಯವರು ಬೇಲೂರು ಮಠದಲ್ಲಿದ್ದಾಗ ಒಮ್ಮೆ ಯಾಮಿನೀರಂಜನ ಅಲ್ಲಿಗೆ ಬಂದು ತಾವು ನಡೆಸುತ್ತಿದ್ದ ಸಂಸ್ಥೆಯ ವಿಚಾರವನ್ನು ವಿವರಿಸಿದ. ಸ್ವಾಮೀಜಿ ಸಂತುಷ್ಟರಾಗಿ, ತನಗೆ ಮಂತ್ರದೀಕ್ಷೆ ನೀಡಬೇಕೆಂಬ ಅವನ ಪ್ರಾರ್ಥನೆಯನ್ನು ಮನ್ನಿಸಿದರು. ಕೆಲದಿನಗಳ ಮೇಲೆ ಅವರು ಈ ಸೇವಾಸಂಸ್ಥೆಯ ಬಗ್ಗೆ ಮತ್ತಷ್ಟು ವಿವರಗಳನ್ನು ತಿಳಿದುಕೊಂಡು ಬರಲು ಇನ್ನೊಬ್ಬರು ಸ್ವಾಮಿಗಳನ್ನು ವಾರಾಣಸಿಗೆ ಕಳಿಸಿದರು. ಅಲ್ಲದೆ ೧೯ಂ೧ರ ಆರಂಭದಲ್ಲಿ ಸ್ವಾಮಿ ಕಲ್ಯಾಣಾ ನಂದರು ಬೇಲೂರಿಗೆ ಬಂದು ತಮ್ಮನ್ನು ಭೇಟಿಯಾದಾಗ ಅವರು, ಕಾಶಿಯ ಸೇವಾಸಂಸ್ಥೆಯ ರೀತಿಯಲ್ಲೇ ಹರಿದ್ವಾರ-ಹೃಷೀಕೇಶ ಪ್ರದೇಶದಲ್ಲಿ ದುರ್ಬಲ, ವೃದ್ಧ ಸಾಧುಗಳಿಗಾಗಿ ಏನಾದರೂ ಮಾಡುವಂತೆ ಪ್ರೋತ್ಸಾಹಿಸಿದರು. ಅದರಂತೆ ಕೆಲವು ತಿಂಗಳಲ್ಲಿ ಸ್ವಾಮಿ ಕಲ್ಯಾಣಾನಂದರು ಹರಿದ್ವಾರಕ್ಕೆ ಐದು ಮೈಲಿ ದೂರದಲ್ಲಿರುವ ಕಂಖಲಿನಲ್ಲಿ ಸೇವಾಶ್ರಮವೊಂದನ್ನು ಸ್ಥಾಪಿಸಿದರು.

ಒಂದು ದಿನ ಬಾಲಗಂಗಾಧರ ತಿಲಕರ ಸಹಕಾರ್ಯಕರ್ತರೂ ‘ಕೇಸರಿ’ ಪತ್ರಿಕೆಯ ಸಂಪಾ ದಕರೂ ಆದ ಶ್ರೀ ನರಸಿಂಹ ಕೇಲ್ಕರರು ಸ್ವಾಮೀಜಿಯವರನ್ನು ಭೇಟಿಯಾಗಲು ಬಂದರು. ಅಂದು ಸ್ವಾಮೀಜಿ ಅನಾರೋಗ್ಯದಿಂದಾಗಿ ಮಲಗಿದ್ದರು. ಕೇಲ್ಕರರು ಅತ್ಯಂತ ಭಕ್ತಿಭಾವದಿಂದ ಕೋಣೆ ಯನ್ನು ಪ್ರವೇಶಿಸಿ, ಸ್ವಾಮೀಜಿಯವರಿಗೆ ನಮಸ್ಕರಿಸಿ ಜಮಖಾನದ ಮೇಲೆ ಕುಳಿತುಕೊಂಡರು. ಸ್ವಾಮೀಜಿ ಮಲಗಿಕೊಂಡೇ ಮಾತಿಗಾರಂಭಿಸಿದರು. ಮೊದಮೊದಲು ಮೆಲುದನಿಯಲ್ಲಿ ನಿಧಾನ ವಾಗಿ ಮಾತನಾಡಿದರು. ಮಾತುಕತೆ ಭಾರತದ ಸಾಮಾಜಿಕ ಹಾಗೂ ರಾಜಕೀಯ ಸ್ಥಿತಿಗತಿಯ ಬಗ್ಗೆ ನಡೆಯುತ್ತಿತ್ತು. ಸ್ವಲ್ಪ ಹೊತ್ತಿನಲ್ಲೇ ಸಂಭಾಷಣೆ ಚುರುಕಾಯಿತು. ಆಗ ಸ್ವಾಮೀಜಿ ಮೇಲೆದ್ದು ಕುಳಿತರು. ಮೆದುವಾಗಿದ್ದ ಅವರ ದನಿ ಕ್ರಮೇಣ ಗಟ್ಟಿಯಾಯಿತು. ಅವರು ತಮ್ಮ ಕಾಯಿಲೆಯನ್ನು ಮರೆತು ಆರೋಗ್ಯವಂತರಂತೆ ಮಾತನಾಡತೊಡಗಿದರು. ಭಾರತವನ್ನು ಮೇಲೆ ತ್ತುವ ವಿಷಯದಲ್ಲಿ ಅವರು ತಮ್ಮ ಅಭಿಪ್ರಾಯಗಳನ್ನು ಸ್ಪಷ್ಟವಾಗಿ ವಿವರಿಸಿದರು; ಭಾರತದ ಅವನತಿಯ ಬಗ್ಗೆ ಪ್ರಸ್ತಾಪಿಸಿ ತೀವ್ರ ದುಃಖವನ್ನು ವ್ಯಕ್ತಪಡಿಸಿದರು. ಕೇವಲ ರಾಜಕಾರಣ ಮತ್ತು ಪಾಶ್ಚಾತ್ಯರ ಅಂಧಾನುಕರಣೆಯಿಂದ ಏನೂ ಪ್ರಯೋಜನವಾಗಲಾರದು. ಸನಾತನ ಧಾರ್ಮಿಕ ಚೌಕಟ್ಟಿನ ಒಳಗೆ ಆಗುವ ಆಂತರಂಗಿಕ ಬೆಳವಣಿಗೆಯಿಂದ ಮಾತ್ರ ಭಾರತವು ಉನ್ನತಿಗೇರಲು ಸಾಧ್ಯ ಎಂದು ಸ್ಪಷ್ಟಪಡಿಸಿದರು. ಅವರ ಈ ವಿಚಾರಧಾರೆಯನ್ನು ಕೇಳಿದ ಕೇಲ್ಕರರು ಆಳವಾಗಿ ಪ್ರಭಾವಿತರಾದರಲ್ಲದೆ ಅವರ ಬಗ್ಗೆ ಮತ್ತೂ ಹೆಚ್ಚಿನ ಪೂಜ್ಯಭಾವವನ್ನು ಹೊತ್ತು ಮರಳಿದರು.

ಸ್ವಾಮೀಜಿಯವರು ವಾರಾಣಸಿಯಲ್ಲಿದ್ದ ಸಮಯದಲ್ಲಿ ಒಮ್ಮೆ ‘ಭಿಂಗಾದ ರಾಜ’ನನ್ನು ನೋಡಲು ಅವನ ಮನೆಗೆ ಹೋದರು. ಈತನ ಹೆಸರು ಉದಯ ಪ್ರತಾಪಸಿಂಗ್. ಇವನು ಉತ್ತರಪ್ರದೇಶದ ಒಬ್ಬ ಧನಿಕ ಜಮೀನ್ದಾರನಾಗಿದ್ದವನು. ಈಗ ವಾರಾಣಸಿಯಲ್ಲಿ ಒಂದು ಮನೆ ಮಾಡಿಕೊಂಡು ಸಂನ್ಯಾಸಿಯಂತೆ ವಿರಕ್ತ ಜೀವನವನ್ನು ನಡೆಸುತ್ತಿದ್ದ. ಇವನು ತನ್ನ ಮನೆಯನ್ನು ಬಿಟ್ಟು ಆಚೆಗೆ ಬರುವುದಿಲ್ಲವೆಂಬ ವ್ರತ ತೊಟ್ಟಿದ್ದ. ಇವನು ಇಂಗ್ಲಿಷ್, ಸಂಸ್ಕೃತಗಳಲ್ಲಿ ಪಂಡಿತ. ಸ್ವಾಮೀಜಿಯವರ ಬಗ್ಗೆ ಬಹಳಷ್ಟು ಕೇಳಿದ್ದ ಈತ ಅವರನ್ನು ಭೇಟಿಯಾಗಲು ಕಾತರನಾಗಿದ್ದ. ಆದರೆ ಅವನು ಕೈಗೊಂಡಿದ್ದ ವ್ರತದಿಂದಾಗಿ ಈ ಸಂದರ್ಶನ ಸಾಧ್ಯವಾಗಿರಲಿಲ್ಲ. ಕಡೆಗೊಮ್ಮೆ ಈಗ ಸ್ವಾಮೀಜಿ ವಾರಾಣಸಿಗೇ ಬಂದಿದ್ದಾರೆಂಬ ವಿಷಯ ತಿಳಿದು ಹರ್ಷಗೊಂಡ. ಒಂದು ದಿನ ಈತ ಸ್ವಾಮಿ ಗೋವಿಂದಾನಂದಜಿ ಎಂಬ ಸಾಧುಗಳನ್ನು ಫಲಪುಷ್ಪ ಕಾಣಿಕೆಗಳೊಂದಿಗೆ ಸ್ವಾಮೀಜಿಯವರ ಬಳಿಗೆ ಕಳಿಸಿ, ಸಂದರ್ಶನಕ್ಕೆ ಅವಕಾಶ ಮಾಡಿಕೊಡಬೇಕೆಂದು ಪ್ರಾರ್ಥಿಸಿಕೊಂಡ. ಮನೆಯಿಂದಾಚೆಗೆ ಬರುವುದಿಲ್ಲವೆಂಬ ತನ್ನ ವ್ರತವನ್ನು ಮುರಿಯುವುದಕ್ಕೂ ರಾಜ ಸಿದ್ಧನಿದ್ದಾನೆಂದು ಕೇಳಿದಾಗ ಸ್ವಾಮೀಜಿಯವರು ಹಾಗೆ ಮಾಡುವುದು ತರವಲ್ಲ ಎಂದು ಹೇಳಿ ತಾವೇ ಅಲ್ಲಿಗೆ ಬರುವುದಾಗಿ ತಿಳಿಸಿದರು.

ಮರುದಿನ ಅವರು ಸ್ವಾಮಿ ಗೋವಿಂದಾನಂದರು ಹಾಗೂ ಸ್ವಾಮಿ ಶಿವಾನಂದರೊಡನೆ ಆತನ ಮನೆಗೆ ಹೋದರು. ರಾಜ ಅವರನ್ನು ಅತ್ಯಂತ ಗೌರವಾದರದಿಂದ ಬರಮಾಡಿಕೊಂಡ. ಸ್ವಾಮೀಜಿ ಯವರನ್ನು ಕಣ್ಣಾರೆ ಕಂಡೊಡನೆ ಭಾವಾವೇಶಭರಿತನಾಗಿ, “ಸ್ವಾಮೀಜಿ, ನೀವು ಬುದ್ಧ-ಶಂಕರ ರಂತಹ ಮಹಾತ್ಮರು!” ಎಂದು ಉದ್ಗರಿಸಿದ. ವಾರಾಣಸಿಯಲ್ಲಿ ರಾಮಕೃಷ್ಣ ಮಠದ ಒಂದು ಶಾಖೆಯನ್ನು ತೆರೆಯಬೇಕು ಎಂದು ಇವನು ಸ್ವಾಮೀಜಿಯವರನ್ನು ಒತ್ತಾಯಪೂರ್ವಕವಾಗಿ ವಿನಂತಿಸಿಕೊಂಡ. ಒಂದು ವರ್ಷದವರೆಗೂ ಅದರ ಖರ್ಚನ್ನೆಲ್ಲ ತಾನು ವಹಿಸಿಕೊಳ್ಳುವುದಾಗಿ ವಾಗ್ದಾನ ಮಾಡಿದ. ಆದರೆ, ತೀವ್ರ ಅನಾರೋಗ್ಯಗ್ರಸ್ತರಾಗಿದ್ದ ಸ್ವಾಮೀಜಿ, ಹೊಸ ಕೇಂದ್ರದ ಬಗ್ಗೆ ಮಾತುಕೊಡುವ ಸ್ಥಿತಿಯಲ್ಲಿರಲಿಲ್ಲ. ಆದ್ದರಿಂದ ಅವರು, “ನಾನೀಗ ಸದ್ಯಕ್ಕೆ ಬೇಲೂರು ಮಠಕ್ಕೆ ಹಿಂದಿರುಗುತ್ತೇನೆ. ಆರೋಗ್ಯ ಸುಧಾರಿಸಿದ ಮೇಲೆ ಮತ್ತೆ ಕೆಲಸದ ಕಡೆಗೆ ಗಮನ ಕೊಡುತ್ತೇನೆ. ಈಗಂತೂ ನನ್ನಿಂದ ಏನೂ ಸಾಧ್ಯವಿಲ್ಲ” ಎಂದರು.

ಮರುದಿನ ಭಿಂಗಾದ ರಾಜ ಸ್ವಾಮೀಜಿಯವರಿಗೆ ೫೦೦ ರೂಪಾಯಿಯನ್ನು ಕಾಣಿಕೆಯಾಗಿ ಕಳಿಸಿಕೊಟ್ಟ. ತಕ್ಷಣ ಅದನ್ನು ಅವರು ಶಿವಾನಂದರ ಕೈಗಿತ್ತು, “ನೋಡು, ಇದರಿಂದ ವೇದಾಂತ ಪ್ರಸಾರಕ್ಕಾಗಿ ಏನಾದರೂ ಕೆಲಸ ಪ್ರಾರಂಭಿಸು” ಎಂದರು. ಶಿವಾನಂದರು ಅದನ್ನು ತೆಗೆದು ಕೊಂಡರಾದರೂ ಕೂಡಲೇ ಏನೂ ಮಾಡಲಿಲ್ಲ. ಕೆಲದಿನಗಳಲ್ಲಿ ಸ್ವಾಮೀಜಿ ಹಿಂದಿರುಗಿದರು. ಅನಾರೋಗ್ಯ ಹಾಗೂ ಇತರ ಕಾರ್ಯಕಲಾಪಗಳ ನಡುವೆಯೂ ಅವರು ವಾರಾಣಸಿಯ ಉದ್ದೇಶಿತ ಕಾರ್ಯವನ್ನು ಮರೆತಿರಲಿಲ್ಲ. ಈ ಬಗ್ಗೆ ಅವರು ತಮ್ಮ ಸೋದರರೊಂದಿಗೆ ಚರ್ಚಿಸಿದರು. ಈ ಕಷ್ಟದ ಕೆಲಸವನ್ನು ಕೈಗೆತ್ತಿಕೊಳ್ಳಲು ಯಾರೂ ಮುಂದಾಗುವಂತೆ ಕಾಣಲಿಲ್ಲ. ಶಿವಾನಂದರೇ ಮೊದಲಾದ ಹಿರಿಯ ಸಾಧುಗಳು ಸ್ವಾಮೀಜಿಯವರ ದೇಹಸ್ಥಿತಿ ಗಂಭೀರವಾಗಿರುವಾಗ ಮಠ ವನ್ನು ಬಿಟ್ಟು ಹೋಗಲು ಹಿಂಜರಿದರು. ಆ ವೇಳೆಗೆ ಕಾಶಿಯಲ್ಲಿ ಸೇವಾಕೇಂದ್ರವನ್ನು ನಡೆಸುತ್ತಿದ್ದ ತರುಣ ಶಿಷ್ಯರಿಂದ ಒಂದು ಪತ್ರ ಬಂದಿತು–“ನಮಗೆ ಆಧ್ಯಾತ್ಮಿಕ ಮಾರ್ಗದರ್ಶನವನ್ನು ನೀಡಲು ಸ್ವಾಮಿ ಶಿವಾನಂದರಂತಹ ಯಾರಾದರೂ ಹಿರಿಯ ಸಂನ್ಯಾಸಿಗಳನ್ನು ಕಳಿಸಿಕೊಡಬೇಕು” ಎಂದು. ಶಿವಾನಂದರು ವಾರಾಣಸಿಯಲ್ಲಿದ್ದಾಗ ಆ ಯುವಕರಿಗೆ ತುಂಬ ಪ್ರಯೋಜನವಾಗಿತ್ತು. ಅವರು ರೋಗಿಗಳ ಉಪಚಾರದಿಂದ ಹಿಡಿದು ಆಧ್ಯಾತ್ಮಿಕ ತರಬೇತಿ ಕೊಡುವುದರವರೆಗೆ ಪ್ರತಿಯೊಂದು ಕೆಲಸವನ್ನೂ ನಿರ್ವಹಿಸಿ ಅತ್ಯುನ್ನತ ಆದರ್ಶವನ್ನು ತೋರಿಸಿಕೊಟ್ಟಿದ್ದರು; ತನ್ಮೂಲಕ ಆ ತರುಣ ರೆಲ್ಲರ ಹೃನ್ಮನಗಳನ್ನು ಗೆದ್ದುಕೊಂಡಿದ್ದರು. ಆದ್ದರಿಂದ ಅವರಂಥವರು–ಎಂದರೆ ಅವರೇ– ತಮ್ಮೊಂದಿಗಿದ್ದರೆ ಬಹಳ ಚೆನ್ನಾಗಿರುತ್ತದೆಂಬುದು ಆ ತರುಣರ ಆಶಯ. ಈ ಪತ್ರ ಬಂದ ಕೂಡಲೇ ವಿಷಯ ಇತ್ಯರ್ಥವಾಯಿತು. ಕಡೆಗೆ ಶಿವಾನಂದರು ವಾರಾಣಸಿಯಲ್ಲಿ ಆಶ್ರಮವನ್ನು ಸ್ಥಾಪಿಸಲು ಒಪ್ಪಿಕೊಂಡರು. ಜೂನ್ ೨೬ರಂದು ಅವರು ಕೇದಾರನಾಥ ಮೌಲಿಕನೊಂದಿಗೆ ಕಾಶಿಗೆ ಬಂದರು. ಹತ್ತು ರೂಪಾಯಿ ಬಾಡಿಗೆಗೆ ದೊಡ್ಡದೊಂದು ತೋಟದಲ್ಲಿದ್ದ ಒಂದು ಸಣ್ಣ ಮನೆಯನ್ನು ಗೊತ್ತುಮಾಡಿಕೊಂಡು, “ರಾಮಕೃಷ್ಣ ಅದ್ವೈತಾಶ್ರಮ” ಎಂದು ಹೆಸರಿಟ್ಟರು. ೧೯ಂ೨ರ ಜುಲೈ ನಾಲ್ಕರಂದು ಶ್ರೀರಾಮಕೃಷ್ಣರ ಭಾವಚಿತ್ರವನ್ನು ಪ್ರತಿಷ್ಠಾಪಿಸಲಾಯಿತು. ಮರುದಿನವೇ ತಂತಿ ಬಂದಿತು–‘ಸ್ವಾಮೀಜಿ ದೇಹತ್ಯಾಗ ಮಾಡಿದರು’ ಎಂದು.

ಸ್ವಾಮೀಜಿ ವಾರಾಣಸಿಯಲ್ಲಿದ್ದಾಗ, ಥಿಯಾಸೊಫಿಸ್ಟನಾದ ಒಬ್ಬ ಡಾಕ್ಟರು ಅವರಿಗೆ ಚಿಕಿತ್ಸೆ ಮಾಡಲು ಬರುತ್ತಿದ್ದ. ಸ್ವಲ್ಪ ಸಮಯ ಆತನ ಚಿಕಿತ್ಸೆಯಲ್ಲಿ ಗುಣ ಕಂಡಿತ್ತು. ಅವನು ಸ್ವಾಮೀಜಿ ಯವರನ್ನು ಒಬ್ಬ ಸಾಧಾರಣ ರೋಗಿ ಎಂಬಂತೆಯೇ ನೋಡುತ್ತಿದ್ದ. ಪಥ್ಯಪಾನದ ವಿಷಯದಲ್ಲಿ ಮಾತ್ರವಲ್ಲದೆ ಧರ್ಮದ ವಿಚಾರದಲ್ಲೂ ಅವರು ತನ್ನ ಮಾತನ್ನೇ ಒಪ್ಪಿಕೊಳ್ಳಬೇಕು ಎನ್ನುವುದು ಅವನ ಧಿಮಾಕು. ಒಂದು ದಿನ ಆತ ಥಿಯಾಸೊಫಿಸ್ಟರು ಭಾರತಕ್ಕೆ ಮಾಡಿದ ಉಪಕಾರದ ಬಗ್ಗೆ ಹೇಳಲಾರಂಭಿಸಿದ. ಥಿಯಾಸೊಫಿಸ್ಟರು, ಅವರು ಮಾಡಿದ ‘ಉಪಕಾರ’–ಇವೆಲ್ಲ ಸ್ವಾಮೀಜಿ ಯವರಿಗೆ ತಿಳಿಯದ್ದೆ? ಆದರೂ ಆ ಡಾಕ್ಟರು ತನ್ನ ಭಾಷಣ ಮಾಡಿಕೊಂಡು ಹೋಗಲಿ ಎಂದು ಸುಮ್ಮನಿದ್ದರು. ಥಿಯಾಸೊಫಿಯ ತತ್ತ್ವವನ್ನು ಹೊಗಳಿ ಸಮಾಧಾನವಾಗದೆ ಅವನು ವ್ಯಕ್ತಿಗಳ ಹೆಸರು ತೆಗೆದ. ಆ ವೇಳೆಗೆ ಶ್ರೀಮತಿ ಆ್ಯನಿ ಬೆಸಂಟರು ಕಾಶಿಯಲ್ಲಿದ್ದರಾದರೂ ಸ್ವಾಮೀಜಿಯವ ರನ್ನು ನೋಡಲು ಬಂದಿರಲಿಲ್ಲ. ಆ ಡಾಕ್ಟರು ಈ ವಿಷಯವನ್ನು ಬೇಕುಬೇಕೆಂದೇ ಎತ್ತಿ ಆಡಿದ. ಆಗಲೂ ಸ್ವಾಮೀಜಿ ಸುಮ್ಮನಿದ್ದರು. ತನ್ನ ಮಾತು ಗುರಿ ಮುಟ್ಟಿದೆ ಎಂದು ಭಾವಿಸಿ ಆತ ಮತ್ತೆ ಕುಹಕವಾಡಿದ–“ಶ್ರೀಮತಿ ಆ್ಯನಿ ಬೆಸೆಂಟರು ನಿಮ್ಮನ್ನು ನೋಡಲು ಬಾರದಿರುವುದು ನಿಜಕ್ಕೂ ತುಂಬ ಖೇದದ ವಿಷಯ, ಅಲ್ಲವೆ?”

ಇನ್ನು ಸ್ವಾಮೀಜಿಯವರಿಂದ ಸುಮ್ಮನಿರಲಾಗಲಿಲ್ಲ. ಅವರ ಮುಖಭಾವ ಸಂಪೂರ್ಣ ಬದಲಾಯಿತು. ಇಡಿಯ ಶರೀರ ಬಿರುಸಾಯಿತು. ಇದ್ದಕ್ಕಿದ್ದಂತೆ ಅವರ ಬಾಯಿಂದ ಶಬ್ದಗಳು ಪ್ರವಾಹದಂತೆ ಹೊಮ್ಮಿದುವು. ಥಿಯಾಸೊಫಿಸ್ಟರ ವಿಧಾನಗಳನ್ನೂ ತತ್ತ್ವಗಳನ್ನೂ ಅವರು ಛಿದ್ರಛಿದ್ರಗೊಳಿಸಿದರು. “ಭಾರತೀಯರು ಐರೋಪ್ಯರ ಮುಂದೆ ಕುಳಿತು ಧರ್ಮವನ್ನು ಕಲಿಯ ಬೇಕಾಗಿಲ್ಲ. ಏಕೆಂದರೆ ಧರ್ಮದ ವಿಚಾರದಲ್ಲಿ ಭಾರತವೇ ಜಗತ್ತಿನಲ್ಲಿ ಅತ್ಯುತ್ಕೃಷ್ಟವಾದುದು” ಎಂದು ಘೋಷಿಸಿದರು. ಅಷ್ಟೇ ಅಲ್ಲ, ತಮ್ಮ ಸ್ವಭಾವಕ್ಕೆ ಸಹಜವಲ್ಲದ ಶಕ್ತಿಪ್ರದರ್ಶನವನ್ನು ಮಾಡಿದರು. ಅವರು ಡಾಕ್ಟರಿಗೆ ಹೇಳಿದರು: “ನೋಡಿ, ನಾನು ಮನಸ್ಸು ಮಾಡಿದರೆ ಶ್ರೀಮತಿ ಬೆಸೆಂಟ್ ಮಾತ್ರವಲ್ಲ, ನಿಮ್ಮ ಯಾವುದೇ ನಾಯಕನನ್ನೂ ನನ್ನ ಕಾಲಡಿಗೆ ಕರೆಸಬಲ್ಲೆ, ತಿಳಿದಿರಿ!” ಇದನ್ನು ಕೇಳಿ ಆ ಡಾಕ್ಟರು ಕಣ್ಕಣ್ಣು ಬಿಟ್ಟ. ಇವರು ಸಾಧಾರಣರಲ್ಲ ಎನ್ನುವುದು ಆಗ ಅವನಿಗೆ ಗೋಚರವಾಯಿತು. ಮೆಲ್ಲನೆ ವಿಷಯವನ್ನು ಬದಲಾಯಿಸಿ, ಅಲ್ಲಿಂದ ಕಾಲುಕಿತ್ತ.

ಇದೇ ಅವಧಿಯಲ್ಲಿ ಶ್ರೀಮತಿ ಸಾರಾ ಬುಲ್ ಮತ್ತು ಸೋದರಿ ನಿವೇದಿತೆ ಯೂರೋಪಿನಿಂದ ಕಲ್ಕತ್ತಕ್ಕೆ ಮರಳಿದರು. ಆಗ ಸ್ವಾಮೀಜಿ ನಿವೇದಿತೆಗೆ ಬರೆದ ಪತ್ರ ತುಂಬ ಸ್ವಾರಸ್ಯಕರವಾಗಿದೆ:

\noindent

“ಪ್ರಿಯ ನಿವೇದಿತಾ,

“ನಿನ್ನ ಪತ್ರವನ್ನು ನೋಡಿ ಅತ್ಯಾನಂದವಾಯಿತು. ಅದರಲ್ಲೂ ನೀನು ಅದಮ್ಯ ಇಚ್ಛಾಶಕ್ತಿ ಹಾಗೂ ಸುಧಾರಿಸಿದ ಆರೋಗ್ಯದೊಂದಿಗೆ ಹಿಂದಿರುಗಿರುವೆ ಎಂದು ತಿಳಿದು ಇನ್ನೂ ಆನಂದ. ನಾನು ನಿನಗೆ ಹೇಳಬೇಕೆಂದಿರುವುದನ್ನೆಲ್ಲ ನನ್ನ ಹಿಂದಿನ ಪತ್ರದಲ್ಲೇ ಬರೆದಿರುವೆ. ನೀನು ಅಲ್ಲಿ ಬ್ರಹ್ಮಾನಂದರನ್ನು ಬಿಟ್ಟು ಬೇರಾರ ಸಲಹೆಯನ್ನೂ ಕೇಳಲು ಹೋಗಬಾರದು. ಆ ಮುದುಕನ (ಶ್ರೀರಾಮಕೃಷ್ಣರ) ಮಾತು ಎಂದಿಗೂ ಸುಳ್ಳಾಗುವುದಿಲ್ಲ. ನನ್ನದಾಗುತ್ತದೆ. ನೀನು ನನ್ನ ಸಲಹೆ ಕೇಳುವುದಾದರೆ, ಯಾರಾದರೂ ನಿನ್ನ ಕೆಲಸದಲ್ಲಿ ನೆರವಾಗಬೇಕೆಂದು ನೀನು ಬಯಸುವೆಯಾದರೆ, ನಾನು ನಿನಗೆ ಸೂಚಿಸುವುದು ಕೇವಲ ಬ್ರಹ್ಮಾನಂದರ ಹೆಸರನ್ನು ಮಾತ್ರವೇ. ಇನ್ನಾರದ್ದೂ ಅಲ್ಲ, ಈ ವಿಷಯದಲ್ಲಿ ಇದೇ ನನ್ನ ಖಚಿತ ಅಭಿಪ್ರಾಯ.

“ಜಗನ್ಮಾತೆ ನಡೆಸಿದಂತೆ ನಡೆ. ನನಗೆ ಸಾಧ್ಯವಿದ್ದಿದ್ದರೆ ನಾನೂ ನಿನ್ನ ನೆರವಿಗೆ ಬರುತ್ತಿದ್ದೆ. ಆದರೆ ನನ್ನ ಶರೀರ ಚಿಂದಿಚಿಂದಿಯಾಗಿದೆ. ಜೊತೆಗೆ ಒಂದೇ ಕಣ್ಣು. ಆದರೆ ನಿನ್ನ ಮೇಲೆ ನನ್ನ ಸಂಪೂರ್ಣ ಆಶೀರ್ವಾದವಿದೆ. ಮತ್ತು ನನ್ನ ಬಳಿ ಇನ್ನೂ ಹೆಚ್ಚು ಇದ್ದರೆ ಅದೂ ನಿನಗೆ.

“ನನ್ನೆಲ್ಲ ಶಕ್ತಿಗಳೂ ನಿನಗೆ ಬರಲಿ. ನಿನ್ನ ಹೃನ್ಮನಗಳಲ್ಲಿ ಜಗನ್ಮಾತೆಯೆ ತುಂಬಿರಲಿ. ನಾನು ಪ್ರಾರ್ಥಿಸುವುದು ನಿನಗೆ ಅದಮ್ಯ ಶಕ್ತಿ ಪ್ರಾಪ್ತವಾಗಲಿ ಎಂದು; ಸಾಧ್ಯವಾದರೆ, ಅದರ ಜೊತೆಗೆ ಅನಂತ ಶಾಂತಿಯೂ ಕೂಡ. ಶ್ರೀರಾಮಕೃಷ್ಣರಲ್ಲಿ ಏನಾದರೂ ಸತ್ಯವಿರುವುದಾದರೆ ಅವರು ನಿನ್ನನ್ನು ಕೈಹಿಡಿದು ಮುನ್ನಡೆಸಲಿ; ನನ್ನನ್ನು ಮುನ್ನಡೆಸಿದಂತೆಯೇ–ಅಲ್ಲ, ಅದಕ್ಕಿಂತಲೂ ಸಾಸಿರಮಡಿ ಹೆಚ್ಚಾಗಿ.”

ವಾರಾಣಸಿಯಲ್ಲಿ ಸ್ವಾಮೀಜಿಯವರ ದೇಹಾರೋಗ್ಯ ಸುಧಾರಿಸಲಿಲ್ಲ; ಬದಲಾಗಿ ಇನ್ನಷ್ಟು ಕೆಟ್ಟಿತು. ಒಮ್ಮೆಯಂತೂ ಅವರ ಕಾಯಿಲೆ ಅದೆಷ್ಟು ಉಲ್ಬಣಿಸಿತೆಂದರೆ ಮೂವರು ಪರಿಚಾರಕರು ಸರದಿ ಪ್ರಕಾರ ರಾತ್ರಿಯಿಡೀ ಗಾಳಿ ಬೀಸಬೇಕಾಯಿತು. ಸ್ವಾಮೀಜಿ ಶಿವಾನಂದರಿಗೆ ಹೇಳಿದರು, “ಈ ರೋಗಗ್ರಸ್ತ ಶರೀರಕ್ಕೆ ತೇಪೆ ಹಚ್ಚುತ್ತ ಎಷ್ಟು ದಿನ ಅಂತ ಉಳಿಸಿಕೊಳ್ಳಲು ಸಾಧ್ಯ? ಒಂದು ವೇಳೆ ನಾನೀಗ ತೀರಿಹೋದರೂ ನಿವೇದಿತಾ, ಶಶಿ ಮತ್ತಿತರರು ನನ್ನ ಕಾರ್ಯವನ್ನು ಮುಂದು ವರಿಸಿಕೊಂಡು ಹೋಗುತ್ತಾರೆ. ಅವರು ತಮ್ಮ ಕೊನೆಯುಸಿರಿರುವವರೆಗೂ ಶ್ರೀರಾಮಕೃಷ್ಣರ ಕಾರ್ಯವನ್ನು ಮಾಡುತ್ತಿರುತ್ತಾರೆ. ಅವರೆಂದಿಗೂ ಆದರ್ಶದಿಂದ ಕದಲುವುದಿಲ್ಲ. ನನ್ನ ಭರವಸೆ ಯೆಲ್ಲ ಅವರ ಮೇಲಿದೆ.”

ಇದೇ ಮನಸ್ಥಿತಿಯಲ್ಲಿ ಅವರು ನಿವೇದಿತೆಗೆ ಇನ್ನೊಂದು ಪತ್ರ ಬರೆದರು:

“ಇದೀಗ ರಾತ್ರಿ ವೇಳೆ. ನಾನೀಗ ಕುಳಿತುಕೊಳ್ಳಲೂ ಆರೆ, ಬರೆಯಲೂ ಆರೆ. ಆದರೂ ಕರ್ತವ್ಯಬುದ್ಧಿಯಿಂದ ಮತ್ತು ಈ ಪತ್ರವೇ ಎಲ್ಲಿ ಕೊನೆಯ ಪತ್ರವಾಗಿಬಿಡುವುದೋ ಎಂಬ ಅಂಜಿಕೆಯಿಂದ, ನಿನಗೆ ಬರೆಯುತ್ತಿದ್ದೇನೆ. ಈಗ ನನ್ನ ಸ್ಥಿತಿಯೇನೂ ಅಷ್ಟೊಂದು ಕೆಟ್ಟು ಹೋಗಿಲ್ಲ. ಆದರೆ ಯಾವ ಕ್ಷಣದಲ್ಲಾದರೂ ಕೆಟ್ಟುಹೋಗಬಲ್ಲುದು. ಅದೇನೋ ಒಂದು ಬಗೆಯ ಸಣ್ಣ ಜ್ವರವಂತೆ–ಹಾಗೆಂದರೇನೋ ನನಗೆ ತಿಳಿಯದು–ಅದು ನನ್ನನ್ನು ಬಿಡುವುದೇ ಇಲ್ಲ. ಜೊತೆಗೆ ಉಸಿರಾಟದ ತೊಂದರೆ ಬೇರೆ.

“ಕ್ರಿಸ್ಟೀನಳು ಭಾರತಕ್ಕೆ ಬರಲು ಇಚ್ಛಿಸಿದ್ದರಿಂದ ನಾನವಳಿಗೆ ಶ್ರೀಮತಿ ಸೇವಿಯರ್​ರಿಂದ ನೂರು ಪೌಂಡುಗಳನ್ನು ಕೊಡಿಸಿದ್ದೆ. ಅವಳ ಕೊನೆಯ ಪತ್ರದ ಪ್ರಕಾರ, ಅವಳು ಅಲ್ಲಿಂದ ಹಡಗನ್ನೇರುವುದು ಸೆಪ್ಟೆಂಬರ್ ೧೫ರಂದು. ಎಂದರೆ ಅವಳು ಇಲ್ಲಿಗೆ ತಲಪುವ ದಿನ ತೀರ ಸನಿಹದಲ್ಲೇ ಇದೆ. ಒಂದು ವೇಳೆ ನಾನಿಲ್ಲಿ ಪ್ರಾಣತ್ಯಾಗ ಮಾಡಿದರೆ–ಈ ಶಿವನಗರಿಯಾದ ವಾರಾಣಸಿಯಲ್ಲಿ ಪ್ರಾಣ ಬಿಡಲು ನನಗೆ ತುಂಬ ಇಷ್ಟ–ನನಗವಳು ಬರೆದ ಪತ್ರವನ್ನು ನೀನು ಒಡೆದು ಓದಿ, ಅವಳನ್ನು ಬರಮಾಡಿಕೊಂಡು, ವಾಪಸು ಕಳಿಸಿಕೊಡುವೆಯಾ? ಹಿಂದಿರುಗಲು ಅವಳ ಕೈಯಲ್ಲಿ ಹಣವಿಲ್ಲದಿದ್ದರೆ ನೀನು ಭಿಕ್ಷೆ ಬೇಡಿಯಾದರೂ ಅವಳ ದಾರಿ ಖರ್ಚನ್ನು ಒದಗಿಸಿಕೊಡು... ”

“ನಾನು ಕಾಶಿಗೆ ಬರುವುದಕ್ಕೆ ಕೆಲವು ವಾರಗಳ ಮೊದಲೇ ರಾಮಕೃಷ್ಣಾನಂದರು ಬಂದು ನನ್ನನ್ನು ಭೇಟಿಯಾದರು. ಆಗ ಅವರು ಮಾಡಿದ ಮೊದಲ ಕೆಲಸವೆಂದರೆ ಅವರು ಬಹಳ ವರ್ಷ ಶ್ರಮಪಟ್ಟು ಸಂಗ್ರಹಿಸಿದ ನಾನ್ನೂರು ರೂಪಾಯಿಗಳನ್ನು ನನ್ನ ಮುಂದಿಟ್ಟದ್ದು. ಇಂಥದೊಂದು ಘಟನೆ ನನ್ನ ಜೀವನದಲ್ಲಿ ನಡೆದದ್ದು ಅದೇ ಮೊದಲ ಸಲ. ಒತ್ತಿ ಬರುತ್ತಿದ್ದ ಕಣ್ಣೀರನ್ನು ತಡೆಯುವುದೇ ಕಷ್ಟವಾಯಿತು ನನಗೆ. ಓ ತಾಯಿ! ಓ ಜಗನ್ಮಾತೆ! ಕೃತಜ್ಞತೆ, ಪ್ರೀತಿ, ಮನುಷ್ಯತ್ವ ಇವೆಲ್ಲ ಇನ್ನೂ ಪೂರ್ತಿ ಸತ್ತಿಲ್ಲ. ನನ್ನ ಪ್ರೀತಿಯ ಮಗಳೇ, ಒಂದೇ ಒಂದು ಬೀಜ ಸಾಕು–ಈ ಇಡೀ ಪ್ರಪಂಚವನ್ನು ವೃಕ್ಷಸಂಪತ್ತಿನಿಂದ ತುಂಬಲು ಒಂದೇ ಒಂದು ಬೀಜ ಸಾಕು.

“ಸರಿ; ಅವರು ಕೊಟ್ಟ ಆ ಹಣ ಮಠದ ಠೇವಣಾತಿಯಲ್ಲಿದೆ. ನಾನದರ ಒಂದು ಕಾಸನ್ನೂ ಮುಟ್ಟಿಲ್ಲ. ನಾನು ತೀರಿಹೋದರೆ ನೀನು ಎಲ್ಲ ಹಣವನ್ನೂ ಅವರಿಗೇ ಹಿಂದಿರುಗಿಸಿಬಿಡು. ಭಗವಂತ ನಿನ್ನನ್ನೂ ರಾಮಕೃಷ್ಣಾನಂದರನ್ನೂ ಹರಸಲಿ. ನನ್ನ ಕಾರ್ಯದಿಂದ ನಾನು ಸಂತೃಪ್ತ ನಾಗಿದ್ದೇನೆ. ನಿಮ್ಮಂತಹ ಇಬ್ಬರು ಪ್ರಾಮಾಣಿಕ ವ್ಯಕ್ತಿಗಳನ್ನು ಬಿಟ್ಟುಹೋಗಲು ಸಾಧ್ಯವಾದದ್ದು ಒಬ್ಬ ಮಹಾತ್ಮನ ಮಹತ್ವಾಕಾಂಕ್ಷೆಗೂ ಮೀರಿದ್ದು.

\begin{flushright}
ಎಂದೆಂದಿಗೂ ನಿನ್ನ ಪ್ರಿಯ ತಂದೆ\\ವಿವೇಕಾನಂದ.
\end{flushright}

ವಾರಾಣಸಿಯಲ್ಲಿ ಸ್ವಾಮೀಜಿಯವರ ಆರೋಗ್ಯ ಚೆನ್ನಾಗಿಲ್ಲದಿದ್ದರೂ ಅಲ್ಲಿನ ಜೀವನ ಅವರಿಗೆ ಪ್ರಿಯವಾಗಿತ್ತು, ಸುಖಮಯವಾಗಿತ್ತು. ಅಲ್ಲಿನ ದೇವಾಲಯಗಳ ಹಾಗೂ ಸಾಧುಸಜ್ಜನರ ನಡುವೆ ವಾಸ ಮಾಡಿದ ಅವರಿಗೆ ಪರಮಾತ್ಮನಲ್ಲೇ ವಿಹರಿಸಿದ ಅನುಭವ. ಮಾರ್ಚ್ ೮ರಂದು ಅವರು ಶಿವಾನಂದರು, ನಿರಂಜನಾನಂದರು ಹಾಗೂ ಇತರರೊಂದಿಗೆ ಬೇಲೂರು ಮಠಕ್ಕೆ ಮರಳಿದರು –ಮಾರ್ಚ್ ೧೧ರಂದು ನಡೆಯಲಿದ್ದ ಶ್ರೀರಾಮಕೃಷ್ಣರ ಜಯಂತ್ಯುತ್ಸವದಲ್ಲಿ ಪಾಲ್ಗೊಳ್ಳಲು.

