
\chapter{ಮಹಾಸಂಘದ ಮುನ್ನಡೆ}

\noindent

ಧರ್ಮಪ್ರಸಾರಕ್ಕಾಗಿ ಸ್ವಾಮೀಜಿಯವರಿಂದ ನಿಯೋಜಿಸಲ್ಪಟ್ಟಿದ್ದ ಸ್ವಾಮಿ ಶಾರದಾನಂದರಾದಿ ಯಾಗಿ ನಾಲ್ವರು ತಮ್ಮ ಪಾಲಿನ ಕಾರ್ಯವನ್ನು ಅತ್ಯುತ್ತಮ ರೀತಿಯಲ್ಲಿ ನಿರ್ವಹಿಸಿದರು. ಶ್ರೀರಾಮಕೃಷ್ಣರ ದಿವ್ಯ ಬೋಧನೆಗಳು ಅತ್ಯಂತ ಸರಳ ಹಾಗೂ ನೇರವಾಗಿದ್ದು ಎಲ್ಲರ ಹೃದಯ ವನ್ನೂ ಸುಲಭವಾಗಿ ಮುಟ್ಟುವಂಥವಾದ್ದರಿಂದ, ಅವನ್ನು ಪ್ರಸಾರ ಮಾಡಲು ಅವರಿಗೆ ಸಾಕಷ್ಟು ಅವಕಾಶಗಳು ದೊರೆಯುತ್ತಿದ್ದುವು. ಶಾರದಾನಂದರೂ ತುರೀಯಾನಂದರೂ ಕಾಥೇವಾಡದ ಪಟ್ಟಣಗಳಲ್ಲಿ ಸಂಚರಿಸುತ್ತಿದ್ದಾಗ, ಅಲ್ಲಿ ಅವರನ್ನು ವಿವೇಕಾನಂದರ ಭಕ್ತರೂ ಅಭಿಮಾನಿಗಳೂ ಅತ್ಯಾದರದಿಂದ ಬರಮಾಡಿಕೊಂಡರು. ಸ್ವಾಮಿಗಳು ಆ ದೂರದ ಪ್ರಾಂತದಲ್ಲಿಯೂ ಕೂಡ ವೇದಾಂತದ ಕುರಿತಾದ ತಮ್ಮ ಭಾಷಣ-ಸಂಭಾಷಣೆಗಳ ಮೂಲಕ ಜನಮನದ ಮೇಲೆ ಶಾಶ್ವತ ಮುದ್ರೆಯನ್ನೊತ್ತಲು ಸಮರ್ಥರಾದರು. ಇತ್ತ ಪೂರ್ವಬಂಗಾಳಕ್ಕೆ ಹೋಗಿದ್ದ ವಿರಜಾನಂದರೂ ಪ್ರಕಾಶಾನಂದರೂ ಢಾಕಾದ ನಾಗರಿಕರ ಕೋರಿಕೆಯ ಮೇರೆಗೆ ಅಲ್ಲೊಂದು ಮಠದ ಶಾಖೆಯನ್ನು ತೆರೆದರು. ಹೀಗೆ ಆ ನಾಲ್ವರು ಸ್ವಾಮಿಗಳೂ ಮೂರು ತಿಂಗಳ ಕಾಲ ಬೋಧನಾಕಾರ್ಯ ಪ್ರಸಾರಕಾರ್ಯಗಳನ್ನು ಯಶಸ್ವಿಯಾಗಿ ನಡೆಸಿ ಹಿಂದಿರುಗಿದರು. ಅವರೆಲ್ಲರ ಯಶಸ್ಸನ್ನು ಕಂಡು ಸ್ವಾಮೀಜಿ ಅತ್ಯಂತ ಆನಂದಿತರಾದರು.

ಸ್ವಾಮೀಜಿಯವರ ಮನಸ್ಸಿನಲ್ಲಿ ಇನ್ನೊಂದು ಬೃಹತ್ ಯೋಜನೆಯು ಕಾರ್ಯರೂಪಕ್ಕೆ ಬರಲು ಕಾದು ಕುಳಿತಿತ್ತು. ಹಿಮಾಲಯ ಪ್ರದೇಶದ ಯಾವುದಾದರೊಂದು ಶಾಂತ ನಿರ್ಜನ ಸ್ಥಳದಲ್ಲಿ ಮಠವೊಂದನ್ನು ಸ್ಥಾಪಿಸಬೇಕು; ಅಲ್ಲಿ ಪ್ರಾಚ್ಯ ಪಾಶ್ಚಾತ್ಯ ಮುಮುಕ್ಷುಗಳು ಸಮಾನೋ ದ್ದೇಶಗಳ ಮೇಲೆ, ಪ್ರೀತಿಯ ತಳಹದಿಯ ಮೇಲೆ ಸಮ್ಮಿಲನಗೊಂಡು ವಿಚಾರವಿನಿಮಯ ಮಾಡಿಕೊಳ್ಳುವಂತಾಗಬೇಕು; ಅದ್ವೈತ ವಿಚಾರಗಳನ್ನು ಅನುಷ್ಠಾನ ಮಾಡುವಂತಾಗಬೇಕು ಎಂಬುದೇ ಆ ಬೃಹದ್ವಿಚಾರ. ಈ ಮಠದ ನಿವೇಶನವು ಸಮುದ್ರಮಟ್ಟದಿಂದ ಏಳುಸಾವಿರ ಅಡಿ ಗಳಷ್ಟಾದರೂ ಎತ್ತರದಲ್ಲಿರಬೇಕು ಎಂಬುದು ಅವರ ಇಚ್ಛೆ. ಏಕೆಂದರೆ, ಮುಂದೆ ಭಾರತಕ್ಕೆ ಬರಲಿರುವ ತಮ್ಮ ಪಾಶ್ಚಾತ್ಯ ಶಿಷ್ಯರನ್ನು ಭಾರತೀಯ ಜೀವನಕ್ರಮಕ್ಕೆ ಹೊಂದಿಕೊಳ್ಳುವಂತೆ ಮಾಡುವ ಉತ್ಸಾಹದಲ್ಲಿ, ಅವರು ಬಯಲುಸೀಮೆಯ ಬೇಗೆಯಲ್ಲಿ ಬೆಂದುಹೋಗುವಂತೆ ಮಾಡ ಬಾರದೆಂಬುದು ಸ್ವಾಮೀಜಿಯವರ ಆಲೋಚನೆ. ಮುಖ್ಯವಾಗಿ ಅಂಥವರಿಗಾಗಿಯೇ ನಿರ್ಮಿತ ವಾಗಲಿದ್ದುದು ಈ ಹಿಮಾಲಯದ ಆಶ್ರಮ. ಅವರು ಈಗಾಗಲೇ ಸಂಚಾರ ಕಾಲದಲ್ಲಿ ಶ್ರೀನಗರ, ಡೆಹರಾಡೂನ್, ಆಲ್ಮೋರಗಳಲ್ಲಿ ಇಂತಹ ಸ್ಥಳವೊಂದಕ್ಕಾಗಿ ಸಂಶೋಧನೆ ನಡೆಸಿದ್ದರು. ಆದರೆ ಅವರ ಮನಸ್ಸಿಗೆ ಒಪ್ಪಿಗೆಯಾಗುವಂತಹ, ಅವರ ಉದ್ದೇಶಕ್ಕೆ ತಕ್ಕುದಾದಂತಹ ಒಂದು ಸ್ಥಳವೂ ಸಿಗಲಿಲ್ಲ. ಕಡೆಗೆ ಅವರು ಇಂತಹ ಒಂದು ಸ್ಥಳವನ್ನು ಸಂಶೋಧಿಸುವ ಕೆಲಸವನ್ನು ಸೇವಿಯರ್ ದಂಪತಿಗಳಿಗೇ ಬಿಟ್ಟುಬಿಟ್ಟರು. ಬಹಳಷ್ಟು ಹುಡುಕಾಟದ ಬಳಿಕ ಸೇವಿಯರ್ ದಂಪತಿಗಳು ಅತ್ಯಂತ ಮನೋಹರವಾದ ಸ್ಥಳವೊಂದನ್ನು ಹುಡುಕಿದರು. ಇದರ ಹೆಸರು ಮಾಯಾವತಿ. ಇದು ದಟ್ಟವಾದ ಮರಗಳಿಂದ ಕೂಡಿದ ಪರ್ವತ ಪ್ರದೇಶ; ಸಮುದ್ರಮಟ್ಟದಿಂದ ೬,೪ಂಂ ಅಡಿ ಎತ್ತರ. ಈ ಸ್ಥಳದಿಂದ ದೂರದಲ್ಲಿ ಭವ್ಯವಾದ ಹಿಮವತ್ಪರ್ವತಶ್ರೇಣಿಯ ಶಿಖರಗಳನ್ನು ಕಾಣಬಹುದು. ಸೇವಿಯರ್ ದಂಪತಿಗಳು ಈ ಸುಂದರ ಸ್ಥಳವನ್ನು ಕಂಡೊಡನೆಯೇ ಸ್ವಾಮೀಜಿ ಉದ್ದೇಶಿಸಿರುವ ಅದ್ವೈತಾಶ್ರಮಕ್ಕೆ ಹಾಗೂ “ಪ್ರಬುದ್ಧ ಭಾರತ” ಪತ್ರಿಕೆಯ ಕಾರ್ಯಕ್ಕೆ ಈ ಸ್ಥಳವೇ ಸರ್ವಸಮರ್ಪಕ ಎಂದು ನಿಶ್ಚಯಿಸಿ, ಕೂಡಲೇ ಆ ಸ್ಥಳವನ್ನು ಕೊಂಡುಕೊಂಡರು. ಮತ್ತು ೧೮೯೯ರ ಮಾರ್ಚ್ ೧೯ರಿಂದ ಅಲ್ಲಿ ತಮ್ಮ ವಸತಿಯನ್ನು ಪ್ರಾರಂಭಿಸಿದರು. ಸ್ವಾಮೀಜಿ ಯವರ ಹೃತ್ಪೂರ್ವಕ ಆಶೀರ್ವಾದದಿಂದ ಅಲ್ಲಿ ಅದ್ವೈತಾಶ್ರಮವನ್ನು ಸ್ಥಾಪಿಸಲಾಯಿತು. ಬಳಿಕ “ಪ್ರಬುದ್ಧ ಭಾರತ” ಪತ್ರಿಕೆಯ ಮುದ್ರಣಾಲಯವನ್ನೂ ಅಲ್ಲಿಗೆ ಸಾಗಿಸಲಾಯಿತು. (ಆದರೆ ಕಾಲಾಂತರದಲ್ಲಿ ಅದನ್ನು ಕಲ್ಕತ್ತಕ್ಕೆ ಸ್ಥಳಾಂತರಿಸಲಾಯಿತು.)

ಸ್ವಾಮೀಜಿಯವರ ಸ್ಫೂರ್ತಿ ಪ್ರೇರಣೆಗಳಿಂದ ಸ್ಥಾಪಿತವಾದ ಸಂಸ್ಥೆಗಳ ಪೈಕಿ ಈ ಅದ್ವೈತ ಆಶ್ರಮವು ಅನೇಕ ವಿಷಯಗಳಲ್ಲಿ ಅಪೂರ್ವ ಮಹತ್ವದಿಂದ ಕೂಡಿದ್ದೆನ್ನಬಹುದು. ಈ ಆಶ್ರಮದ ತತ್ತ್ವ-ಆದರ್ಶಗಳನ್ನು ಅರಿಯಬೇಕಾದರೆ ಅವರು ಇದರ ಸಂಸ್ಥಾಪಕರಾದ ಶ್ರೀ ಸೇವಿಯರ್​ರಿಗೆ ಬರೆದ ಒಂದು ಪತ್ರವನ್ನು ನೋಡಬೇಕು:

“... ಯಾರೊಳಗೆ ಈ ವಿಶ್ವವಡಗಿದೆಯೋ, ಯಾರು ಈ ವಿಶ್ವದೊಳಗೆಲ್ಲೆಲ್ಲೂ ಇರುವನೋ, ಯಾರು ಈ ವಿಶ್ವವೇ ಆಗಿರುವನೋ (ಅಂತಹ ಪರಮಾತ್ಮನನ್ನು) ಮತ್ತು ಯಾರೊಳಗೆ ಆತ್ಮವಡಗಿದೆಯೋ, ಯಾರು ಆತ್ಮದೊಳಗೆಯೇ ಇದ್ದಾನೆಯೋ ಮತ್ತು ಜೀವರೊಳಗಿನ ಆತ್ಮವೇ ತಾನಾಗಿದ್ದಾನೆಯೋ ಅಂತಹ ಪರಮಾತ್ಮನೇ ನಾನು, ಆದ್ದರಿಂದ ಈ ಇಡೀ ವಿಶ್ವವೇ ನಾನು, ಎಂಬುದನ್ನು ಅರಿಯುವುದರಿಂದ ಮಾತ್ರವೇ ನಮ್ಮ ಸಕಲ ಭಯವೂ ಸಕಲ ದುಃಖವೂ ನಾಶ ವಾಗುತ್ತದೆ; ಪರಿಪೂರ್ಣ ಮುಕ್ತಿಯ ಬಾಗಿಲು ತೆರೆದುಕೊಳ್ಳುತ್ತದೆ. ಎಲ್ಲೆಲ್ಲಿ ಮನುಷ್ಯರ ನಡು ವಣ ಪ್ರೀತಿ ಬಾಂಧವ್ಯಗಳು ವಿಕಸಿತಗೊಂಡಿವೆಯೋ, ಎಲ್ಲೆಲ್ಲಿ ಜನಜೀವನ ಹೆಚ್ಚು ಸುಖಶಾಂತಿ ಭರಿತವಾಗಿದೆಯೋ ಅವೆಲ್ಲ ಸಾಧ್ಯವಾಗಿರುವುದು ಆ ಸನಾತನ ತತ್ತ್ವದ ಅನುಷ್ಠಾನ, ದರ್ಶನ ಹಾಗೂ ಅನುಭವಗಳ ಮೂಲಕ–ಅದೇ ಸಕಲ ಜೀವಿಗಳಲ್ಲೂ ಅಂತರ್ಯಾಮಿಯಾಗಿರುವ ಅಖಂಡ ಆತ್ಮದ ಅಮರ ತತ್ತ್ವ. ‘ಸರ್ವಂ ಸ್ವಾವಲಂಬನಂ ಹಿ ಸುಖಂ ಸರ್ವಂ ಪರಾವಲಂಬನಂ ಹಿ ದುಃಖಂ’–ಸ್ವಾವಲಂಬನೆಯೇ ಸುಖ, ಪರಾವಲಂಬನೆಯೇ ದುಃಖ. ಅದ್ವೈತವೊಂದೇ ಮಾನವನಿಗೆ ತನ್ನ ಮೇಲೆ ಸಂಪೂರ್ಣ ಸ್ವಾಧೀನತೆಯನ್ನು ದೊರಕಿಸಿಕೊಡಬಲ್ಲ ಮಾರ್ಗ. ಅದ್ವೈತ ಮಾತ್ರವೇ ಪರಾಧೀನತೆಯನ್ನು, ಮತ್ತು ಅದರೊಂದಿಗೇ ಸೇರಿಕೊಂಡಿರುವ ಮೌಢ್ಯವನ್ನು ಹೋಗಲಾಡಿಸುತ್ತದೆ. ಹೀಗೆ ಈ ಅದ್ವೈತವೆಂಬುದು ಕಷ್ಟಗಳನ್ನು ಧೈರ್ಯದಿಂದ ಎದುರಿಸಲು, ಮಾಡಬೇಕಾದ ಕಾರ್ಯಗಳನ್ನು ಧೈರ್ಯದಿಂದ ಮಾಡಿಮುಗಿಸಲು ಮತ್ತು ಕಾಲಾಂತರದಲ್ಲಿ ಮುಕ್ತಿಯನ್ನು ಸಿದ್ಧಿಸಿಕೊಳ್ಳಲು ನಮ್ಮನ್ನು ಅರ್ಹರನ್ನಾಗಿಸುತ್ತದೆ, ಸತ್ತ್ವಶಾಲಿಗಳನ್ನಾಗಿಸುತ್ತದೆ.

“ಆದರೆ ಇಂದಿನವರೆಗೂ, ಈ ಅದ್ವೈತವೆಂಬ ಪರಶ್ರೇಷ್ಠ ತತ್ತ್ವವನ್ನು ಭೇದದೃಷ್ಟಿಯೆಂಬ ದೌರ್ಬಲ್ಯದಿಂದ ಬೇರ್ಪಡಿಸಿ ಅದರ ಶುದ್ಧರೂಪದಿಂದ ಪ್ರಸಾರ ಮಾಡಲು ಸಾಧ್ಯವಾಗಿಲ್ಲ. ಈ ಅದ್ವೈತವೆಂಬುದು ಈವರೆಗೆ ಏಕೆ ಇನ್ನೂ ಹೆಚ್ಚಾಗಿ ಕಾರ್ಯಗತವಾಗಿಲ್ಲ ಮತ್ತು ಮಾನವ ಜನಾಂಗಕ್ಕೆ ಅಷ್ಟೊಂದು ಉಪಯುಕ್ತವಾಗಿ ಪರಿಣಮಿಸಿಲ್ಲ ಎಂಬುದಕ್ಕೆ ಇದೇ ಕಾರಣ. ವ್ಯಕ್ತಿ ವ್ಯಕ್ತಿಯ ಜೀವನವನ್ನು ಉತ್ಕರ್ಷಗೊಳಿಸಲು ಮತ್ತು ಮಾನವಸಮೂಹದ ಮೇಲೆ ಸತ್ಪ್ರಭಾವ ಬೀರಲು ಸಾಧ್ಯವಾಗುವಂತೆ ಈ ಮಹಾಸತ್ಯದ ಪ್ರಸಾರಕ್ಕೆ ಇನ್ನೂ ಹೆಚ್ಚಿನ ಅನಿರ್ಬಂಧಿತ ಅವಕಾಶವನ್ನು ಕೊಡುವ ಮಹೋದ್ದೇಶದಿಂದ, ಅದ್ವೈತದ ಉಸಿರು ಪ್ರಥಮತಃ ಕೇಳಿಬಂದ ಹಿಮಾಲಯ ಶೃಂಗಗಳಲ್ಲಿ ಈ ಅದ್ವೈತಾಶ್ರಮವನ್ನು ಸ್ಥಾಪಿಸುತ್ತಿದ್ದೇವೆ.

“ಈ ಸ್ಥಳದಲ್ಲಿ ಅದ್ವೈತದ ಪರಮಸತ್ಯವನ್ನು, ಅದನ್ನು ದುರ್ಬಲಗೊಳಿಸುವ ಎಲ್ಲ ಆಚಾರ ವಿಚಾರಗಳಿಂದ ಹಾಗೂ ಮೂಢನಂಬಿಕೆಗಳಿಂದ ದೂರವಿಟ್ಟು, ಅದನ್ನು ಶುದ್ಧರೂಪದಲ್ಲಿ ಉಳಿಸಿ ಕೊಂಡು ಬರುವ ಆಕಾಂಕ್ಷೆ ನಿರೀಕ್ಷೆಗಳನ್ನು ಇಟ್ಟುಕೊಂಡಿದ್ದೇವೆ. ಇಲ್ಲಿ ಶುದ್ಧವಾದ ಏಕತೆಯ ತತ್ತ್ವವನ್ನಲ್ಲದೆ ಬೇರೇನನ್ನೂ ಬೋಧಿಸಲಾಗುವುದಿಲ್ಲ ಅಥವಾ ಆಚರಿಸಲಾಗುವುದಿಲ್ಲ. ಇತರ ಎಲ್ಲ ಪಥಗಳ ಮತ್ತು ನಂಬಿಕೆಗಳ ವಿಷಯದಲ್ಲಿ ಸಂಪೂರ್ಣ ವಿಶ್ವಾಸ-ಸಹಾನುಭೂತಿಗಳನ್ನು ಉಳ್ಳದ್ದಾದರೂ ಈ ಆಶ್ರಮವು ಅದ್ವೈತಕ್ಕೆ ಮಾತ್ರವೇ ಮೀಸಲಾಗಿದೆ.”

ಮಾಯಾವತಿಯಲ್ಲಿ ಸ್ಥಾಪಿತವಾದ ಈ ಅದ್ವೈತಾಶ್ರಮದಲ್ಲಿ ಪ್ರತಿಮೆ-ಚಿತ್ರ-ಸಂಕೇತಗಳ ಮೂಲಕ ಯಾವುದೇ ಬಾಹ್ಯಪೂಜೆಯಾಗಲಿ, ಅಥವಾ–ವಿರಜಾಹೋಮವೊಂದನ್ನು ಬಿಟ್ಟು– ಇನ್ನಾವುದೇ ವಿಧ್ಯುಕ್ತ ಧಾರ್ಮಿಕ ಆಚಾರಗಳಾಗಲಿ ಇಲ್ಲ. ಅಷ್ಟೇ ಅಲ್ಲ, ರಾಮಕೃಷ್ಣ ಮಹಾ ಸಂಘದ ಇನ್ನಿತರ ಕೇಂದ್ರಗಳಲ್ಲಿನ ಅತ್ಯಂತ ಪ್ರಮುಖ ಅಂಗವಾಗಿರುವ ಶ್ರೀರಾಮಕೃಷ್ಣರ ಪೂಜೆ ಕೂಡ ಇಲ್ಲಿಲ್ಲ.

ಅದ್ವೈತಾಶ್ರಮದ ಸಂಸ್ಥಾಪಕರಾದ ಸೇವಿಯರ್ ದಂಪತಿಗಳ ಕೋರಿಕೆಯ ಮೇರೆಗೆ ಸ್ವಾಮೀಜಿ ತಮ್ಮ ನಾಲ್ವರು ಶಿಷ್ಯರನ್ನು ಅಲ್ಲಿಗೆ ಕಳಿಸಿಕೊಡಲು ಒಪ್ಪಿದರು. ಅದರಂತೆ ಸ್ವಾಮಿ ಸಚ್ಚಿದಾನಂದ, ಸ್ವಾಮಿ ವಿರಜಾನಂದ, ಸ್ವಾಮಿ ವಿಮಲಾನಂದ ಹಾಗೂ ಬ್ರಹ್ಮಚಾರಿ ಹರೇಂದ್ರನಾಥರು ಕೆಲಕಾಲ ದಲ್ಲೇ ಅದ್ವೈತಾಶ್ರಮದಲ್ಲಿ ತಮ್ಮ ಹೊಸ ಜವಾಬ್ದಾರಿಗಳನ್ನು ವಹಿಸಿಕೊಳ್ಳಲು ಹೊರಟರು. ಜನವಸತಿಯಿಂದ ಮಾಯಾವತಿಯು ಬಹಳ ದೂರದಲ್ಲಿದ್ದುದರಿಂದ ಅಲ್ಲಿ ಯಾವ ಸೌಲಭ್ಯಗಳೂ ಇರಲಿಲ್ಲ. ಈಗ ಅಲ್ಲಿ ರಸ್ತೆ ನಿರ್ಮಾಣದಿಂದ ಹಿಡಿದು ಎಲ್ಲ ಕೆಲಸಗಳೂ ಪ್ರಾರಂಭವಾಗಬೇಕಿತ್ತು. ಕಟ್ಟಡ ನಿರ್ಮಾಣ, ತೋಟಗಾರಿಕೆ ಹಾಗೂ ‘ಪ್ರಬುದ್ಧ ಭಾರತ’ದ ಪ್ರಕಟಣೆಯಲ್ಲಿ ನೆರವಾಗು ವುದು–ಇವಿಷ್ಟು ಅಲ್ಲಿ ಅವರ ಪ್ರಮುಖ ಕಾರ್ಯಗಳಾಗಿದ್ದುವು.

ಈಗ ಪಾಶ್ಚಾತ್ಯ ರಾಷ್ಟ್ರಗಳಲ್ಲಿನ ವೇದಾಂತಪ್ರಸಾರದ ಕುರಿತಾಗಿ ನೋಡುವುದಾದರೆ, ಸ್ವಾಮಿ ಅಭೇದಾನಂದರು ಲಂಡನ್ನಿನಲ್ಲಿ ಸುಮಾರು ಹತ್ತು ತಿಂಗಳ ಕಾಲ ತರಗತಿಗಳನ್ನು ನಡೆಸಿದ ಮೇಲೆ, ೧೮೯೭ರ ಜುಲೈ ತಿಂಗಳಲ್ಲಿ ಅಮೆರಿಕೆಗೆ ತೆರಳಬೇಕಾಯಿತು. ನ್ಯೂಯಾರ್ಕಿನ ವೇದಾಂತ ಸೊಸೈಟಿಯ ಉಸ್ತುವಾರಿಯನ್ನು ನೋಡಿಕೊಳ್ಳಬಲ್ಲ ಸ್ವಾಮಿಗಳೊಬ್ಬರಿಗಾಗಿ ಅಲ್ಲಿಂದ ಮತ್ತೆ ಮತ್ತೆ ಕರೆಗಳು ಬಂದದ್ದೆ ಇದಕ್ಕೆ ಕಾರಣ. ಇದರಿಂದಾಗಿ ಅವರು ಲಂಡನ್ನಿನಲ್ಲಿ ನಡೆಸುತ್ತಿದ್ದ ತರಗತಿಗಳನ್ನು ತಾತ್ಕಾಲಿಕವಾಗಿ ನಿಲ್ಲಿಸಬೇಕಾಯಿತಾದರೂ ಅಲ್ಲಿನ ಕೆಲಸ ಕಾರ್ಯಗಳು ಪೂರ್ಣ ಸ್ಥಗಿತಗೊಳ್ಳಲಿಲ್ಲ. ಸ್ವಾಮಿ ಅಭೇದಾನಂದರ ಶಿಷ್ಯರು ಮತ್ತು ವೇದಾಂತದಲ್ಲಿ ಆಸಕ್ತರಾಗಿದ್ದ ಇನ್ನಿತರ ವಿದ್ಯಾರ್ಥಿಗಳು ಚಿಕ್ಕಚಿಕ್ಕ ಗುಂಪುಗಳಲ್ಲಿ ಆಗಾಗ ಸೇರುತ್ತಿದ್ದರು. ಅವರು ಹಿಂದಿನ ಉತ್ಸಾಹದಿಂದಲೇ ಓದು, ಸಂವಾದ, ಚರ್ಚೆಗಳಲ್ಲಿ ಪಾಲ್ಗೊಳ್ಳುವುದರ ಮೂಲಕ ಪರಸ್ಪರ ನೆರವಾಗುತ್ತ ಸ್ವಾಮಿ ವಿವೇಕಾನಂದರು ತಮ್ಮಲ್ಲಿಗೆ ಹಿಂದಿರುಗಿ ಬರುವುದನ್ನೇ ಇದಿರು ನೋಡುತ್ತಿದ್ದರು.

ಅಮೆರಿಕದಲ್ಲಿ ಸ್ವಾಮಿ ಅಭೇದಾನಂದರು ತಮ್ಮ ಪ್ರತಿಭಾಪೂರ್ಣ ವ್ಯಕ್ತಿತ್ವದಿಂದ ಹಾಗೂ ಆಳವಾದ ಪಾಂಡಿತ್ಯದಿಂದ ಅಲ್ಲಿನ ಜನಮನದಲ್ಲಿ ವೇದಾಂತಭಾವನೆಗಳನ್ನು ಮುದ್ರೆಯೊತ್ತಲು ಸಮರ್ಥರಾದರು. ಹೀಗೆ ಸ್ವಾಮೀಜಿಯವರು ಅಮೆರಿಕದ ನೆಲದಲ್ಲಿ ಬಿತ್ತಿದ್ದ ಬೀಜಗಳು ದಿನ ಕಳೆದಂತೆ ಮೊಳಕೆಯೊಡೆದು ಚಿಗುರಿ ಸೊಂಪಾಗಿ ಬೆಳೆದು ಬೇರೂರಿ ನಿಂತವು. ಸ್ವಾಮೀಜಿಯವರು ೧೮೯೩ರಲ್ಲಿ ಮೊದಲ ಸಲ ಅಮೆರಿಕದಲ್ಲಿ ಕಾಲಿಟ್ಟಾಗಿನಿಂದ, ಎರಡನೆಯ ಸಲ, ಎಂದರೆ ೧೮೯೯ರಲ್ಲಿ ಭೇಟಿ ಕೊಡುವವರೆಗಿನ ಅವಧಿಯಲ್ಲಿ ಅಮೆರಿಕದ ಮೇಲೆ ಉಂಟಾಗಿದ್ದ ಭಾರತೀಯ ತತ್ತ್ವಶಾಸ್ತ್ರದ ಪ್ರಭಾವವನ್ನು ಸ್ಪಷ್ಟವಾಗಿ ಕಾಣಬಹುದಾಗಿತ್ತು. ಈ ಅವಧಿಯಲ್ಲಿ ಅಲ್ಲಿ ಹಲವಾರು ‘ನವಚಿಂತನೆ’ಯ ಸಂಘಗಳು ಹುಟ್ಟಿಕೊಂಡದ್ದಲ್ಲದೆ ವೇದಾಂತದ ತತ್ತ್ವಗಳೂ ಅದರ ಆಚ ರಣೆಯೂ ಬೇರೆ ಬೇರೆ ಹೆಸರುಗಳಲ್ಲಿ ಮತ್ತು ಬೇರೆ ಬೇರೆ ರೂಪಗಳಲ್ಲಿ ಪ್ರಚಾರಕ್ಕೆ ಬರುವಂ ತಾದುವು. ಆದ್ದರಿಂದ ಸ್ವಾಮೀಜಿ ಎರಡನೆಯ ಸಲ ಪಶ್ಚಿಮ ರಾಷ್ಟ್ರಗಳ ಭೇಟಿಗಾಗಿ ಭಾರತದಿಂದ ಹೊರಟಾಗ ತಾವೀಗಾಗಲೇ ಚಾಲನೆಯಲ್ಲಿ ತೊಡಗಿಸಿದ್ದ ಕಾರ್ಯಕ್ಕೆ ಒಂದು ಉಜ್ವಲ ಭವಿಷ್ಯವು ತೆರೆಯಲಿರುವುದನ್ನು ಮನಗಂಡರು. ಈ ಸಲ ಅವರು ಪಾಶ್ಚಾತ್ಯ ರಾಷ್ಟ್ರಗಳಿಗೆ ಹೋಗುವ ಮುಖ್ಯ ಉದ್ದೇಶ ದೇಹಾರೋಗ್ಯ ಸುಧಾರಣೆಯೇ ಆಗಿದ್ದರೂ ಅಲ್ಲಿಗೆ ಹೋದಮೇಲೆ ಮಾತ್ರ ಮತ್ತೆ ತೀವ್ರ ಚಟುವಟಿಕೆಯಲ್ಲಿ ಸಿಲುಕಿಕೊಳ್ಳಲಿರುವುದನ್ನು ನಾವು ನೋಡಲಿದ್ದೇವೆ.

ಇತ್ತ ಭಾರತದಲ್ಲಿ ಸ್ವಾಮೀಜಿಯವರ ಧ್ಯೇಯೋದ್ದೇಶಗಳಲ್ಲಿ ಒಂದಾದ ಮಾನವ ಕಲ್ಯಾಣ ಕಾರ್ಯವೂ ರಭಸಪೂರ್ಣವಾಗಿ ನಡೆದುಕೊಂಡು ಬರತೊಡಗಿತ್ತು. ಉದಾಹರಣೆಗೆ, ಖೇತ್ರಿಯಲ್ಲಿ ಸ್ವಾಮಿ ಅಖಂಡಾನಂದರು ಅಲ್ಲಿನ ಬಡಬಗ್ಗರ ಮೇಲ್ಮೆಗಾಗಿ, ಮಕ್ಕಳ ವಿದ್ಯಾಭ್ಯಾಸಕ್ಕಾಗಿ ಕೈಗೊಂಡ ಸೇವಾಕಾರ್ಯವು ಒಂದು ಇತಿಹಾಸವನ್ನೇ ನಿರ್ಮಾಣ ಮಾಡಿತು. ಇನ್ನೊಬ್ಬ ಗುರು ಭಾಯಿಗಳಾದ ಸ್ವಾಮಿ ತ್ರಿಗುಣಾತೀತಾನಂದರು ದಿನಜ್​ಪುರದ ಸುಮಾರು ಎಂಬತ್ತನಾಲ್ಕು ಹಳ್ಳಿ ಗಳಲ್ಲಿ ಕ್ಷಾಮಪರಿಹಾರ ಕಾರ್ಯವನ್ನು ಪ್ರಚಂಡ ಸಾಮರ್ಥ್ಯದಿಂದ ನಿರ್ವಹಿಸಿ ಹಳ್ಳಿಗರೆಲ್ಲರ ಪ್ರೀತ್ಯಾದರಗಳಿಗೆ ಪಾತ್ರರಾದರು. ಇದೇ ರೀತಿ ದೇವಘರ್ ಎಂಬಲ್ಲಿ ಸ್ವಾಮಿ ವಿರಜಾನಂದರ ನೇತೃತ್ವದಲ್ಲಿ ಮೂರನೆಯ ಕೇಂದ್ರವೊಂದು ಪ್ರಾರಂಭವಾಯಿತು. ಇವುಗಳಲ್ಲದೆ ದಕ್ಷಿಣೇಶ್ವರ ಹಾಗೂ ಕಲ್ಕತ್ತಗಳಲ್ಲೂ ಪರಿಹಾರ ಕೇಂದ್ರಗಳು ತೆರೆಯಲ್ಪಟ್ಟುವು. ಅಮೆರಿಕ ಇಂಗ್ಲೆಂಡುಗಳಲ್ಲಿನ ಸ್ವಾಮೀಜಿಯವರ ಶಿಷ್ಯರು ಜನರ ಕಷ್ಟಸಂಕಟಗಳ ವರದಿಗಳನ್ನು ಕೇಳಿ ಸಂತಾಪಗೊಂಡು ಸಭೆಗಳನ್ನು ಸೇರಿಸಿ ಕ್ಷಾಮಪರಿಹಾರ ನಿಧಿಗೆ ಧಾರಾಳವಾಗಿ ಧನ ಸಹಾಯ ಮಾಡಿದರು.

ಹೀಗೆ ಶ್ರೀರಾಮಕೃಷ್ಣ ಮಹಾಸಂಘವು ಕ್ಷಾಮಪರಿಹಾರ ಕಾರ್ಯಗಳನ್ನಲ್ಲದೆ ಇನ್ನೊಂದು ಜನಸೇವಾ ಪರವಾದ ಮಹಾಕಾರ್ಯವನ್ನು ಕೈಗೊಂಡಿತು. ೧೮೯೮ರ ಮೇ ತಿಂಗಳಲ್ಲಿ ಕಲ್ಕತ್ತದಲ್ಲಿ ಪ್ಲೇಗ್ ಮಾರಿ ತಾಂಡವವಾಡಿದಾಗ ಗ್ರಾಮಾಂತರಗಳಲ್ಲಿ ಮನೆಗಳಿದ್ದವರೇನೋ ನಗರವನ್ನು ಬಿಟ್ಟು ಅಲ್ಲಿಗೆ ಓಡಿಹೋದರು. ಆದರೆ ಮಿಕ್ಕವರೆಲ್ಲ ಕಲ್ಕತ್ತದಲ್ಲೇ ಉಳಿಯಬೇಕಾಯಿತು. ಇಂತಹ ಬಡಜನರಿಗಾಗಿ ಸೇವಾಕಾರ್ಯವನ್ನು ಮೊಟ್ಟಮೊದಲಿಗೆ ಪ್ರಾರಂಭಿಸಿದ್ದವರೂ ಒಬ್ಬ ಸಂನ್ಯಾಸಿಯೇ. ಅದು ಮತ್ತಾರೂ ಅಲ್ಲ, ಸ್ವಾಮಿ ವಿವೇಕಾನಂದರೇ. ಆದರೆ ಮತ್ತೆ ಮರು ವರ್ಷ ಪ್ಲೇಗ್ ಮರುಕಳಿಸಿತು. ಆಗ ಸ್ವಾಮೀಜಿಯವರ ಆದೇಶದಂತೆ ‘ಶ್ರೀರಾಮಕೃಷ್ಣ ಮಿಷನ್ ಪ್ಲೇಗ್ ಸರ್ವಿಸ್​’ ಎಂಬ ಕಾರ್ಯವನ್ನು ಪ್ರಾರಂಭಿಸಲಾಯಿತು. ತನ್ಮೂಲಕ ವ್ಯವಸ್ಥಿತ ರೀತಿಯಲ್ಲಿ ಸಾಕಷ್ಟು ಕೆಲಸವನ್ನು ಸಾಧಿಸಲು ಸಾಧ್ಯವಾಯಿತು. ಸ್ವಾಮೀಜಿಯವರು ಅಲ್ಲಿನ ಸೇವಾಕರ್ತರಲ್ಲಿ ಉತ್ಸಾಹ ತುಂಬುವ ಮತ್ತು ಬಡಜನರಿಗೆ ಧೈರ್ಯ ಕೊಡುವ ಉದ್ದೇಶದಿಂದ ಸ್ವತಃ ತಾವೇ ಹೋಗಿ ಅಲ್ಲಿನ ಗುಡಿಸಲುಗಳಲ್ಲಿ ಕೆಲಕಾಲ ಉಳಿದುಕೊಂಡರು. ಈ ಸೇವಾಕಾರ್ಯದ ಕಾರ್ಯ ದರ್ಶಿಯನ್ನಾಗಿ ಸೋದರಿ ನಿವೇದಿತೆಯನ್ನು ಹಾಗೂ ಮುಖ್ಯ ಅಧಿಕಾರಿಯನ್ನಾಗಿ ಸ್ವಾಮಿ ಸದಾ ನಂದರನ್ನು ನೇಮಿಸಲಾಯಿತು. ಸ್ವಾಮಿಗಳಾದ ಶಿವಾನಂದರು, ನಿತ್ಯಾನಂದರು ಮತ್ತು ಆತ್ಮಾ ನಂದರು ಈ ಕಾರ್ಯದಲ್ಲಿ ಸಹಾಯಕರಾಗಿದ್ದು, ಇವರೆಲ್ಲರ ಮೇಲ್ವಿಚಾರಣೆಯಲ್ಲಿ ನಗರದ ಹಲವಾರು ಕೊಳೆಗೇರಿಗಳನ್ನು ಶುಚಿಗೊಳಿಸಲಾಯಿತು. ಮತ್ತು ಗಾಡಿಗಟ್ಟಲೆ ಕಸವನ್ನು ತೆಗೆದು ಕ್ರಿಮಿನಾಶಕಗಳನ್ನು ಚೆನ್ನಾಗಿ ಸಿಂಪಡಿಸಲಾಯಿತು. ಹೀಗೆ ಕಲ್ಕತ್ತದಲ್ಲಿ ಪ್ಲೇಗ್ ರೋಗವನ್ನು ಬಹುಮಟ್ಟಿಗೆ ನಿರ್ಮೂಲನಗೊಳಿಸಲು ಸಾಧ್ಯವಾಯಿತು. ಅದರ ಮರುವರ್ಷವೂ ಅಲ್ಲಿ ಪ್ಲೇಗ್ ಕಾಣಿಸಿಕೊಂಡಿತಾದರೂ ಅದು ಬೇಗ ಹತೋಟಿಗೆ ಬಂದಿತು. ಮತ್ತು ಅದಾದನಂತರ ಕಲ್ಕತ್ತದಲ್ಲಿ ಪ್ಲೇಗ್ ಮಾರಿ ತಲೆ ಹಾಕಿಯೇ ಇಲ್ಲವೆನ್ನಬಹುದು.

ಸ್ವಾಮೀಜಿಯವರು ಪಾಶ್ಚಾತ್ಯ ರಾಷ್ಟ್ರಗಳಿಂದ ಹಿಂದಿರುಗಿದ ಮೇಲೆ ಪ್ರಾರಂಭಿಸಿ ಪ್ರಚಾರ ಗೊಳಿಸಿದ ಇನ್ನೊಂದು ಪ್ರಮುಖ ಕಾರ್ಯವೆಂದರೆ ಶ್ರೀರಾಮಕೃಷ್ಣರ ಜಯಂತ್ಯುತ್ಸವ. ಇದು ಕೇವಲ ಒಂದು ಸಾಂಪ್ರದಾಯಿಕ ಉತ್ಸವವಾಗಿರದೆ ಸಹಸ್ರಾರು ಬಡಜನರಿಗೆ ಈ ಸಂದರ್ಭದಲ್ಲಿ ಅನ್ನದಾನ ಮಾಡುವ ಪದ್ಧತಿ ಬೆಳೆದುಬಂದಿತು.

ಇವಿಷ್ಟು ಶ್ರೀರಾಮಕೃಷ್ಣ ಮಹಾಸಂಘವು ಜನ್ಮತಾಳಿದ ಸುಮಾರು ಎರಡೂವರೆ ವರ್ಷಗಳಲ್ಲಿ ಸಾಧಿಸಿದ ಜನಹಿತ ಕಾರ್ಯಗಳ ಪಕ್ಷಿನೋಟ. ಈ ಕಾರ್ಯಗಳ ಪ್ರಮಾಣವು ಸಾಕಷ್ಟು ಗಣ ನೀಯವೇ ಆದರೂ ಇದರ ಪರಿಣಾಮವಾಗಿ ಇತರರಲ್ಲೂ ಉಂಟಾದ ಸೇವಾ ಮನೋಭಾವ, ಭ್ರಾತೃತ್ವ ಮತ್ತು ಸಹಕಾರ ಬುದ್ಧಿ ಇವು ಇನ್ನೂ ಹೆಚ್ಚು ಪ್ರಮುಖವಾದವು.

ಕೆಲವೊಮ್ಮೆ ಸ್ವಾಮೀಜಿಯವರೊಂದಿಗೆ ತತ್ತ್ವ-ಶಾಸ್ತ್ರ-ಸಿದ್ಧಾಂತಗಳ ಕುರಿತಾಗಿ ಸಂಭಾಷಿಸ ಲೆಂದು ಅನೇಕ ಪಂಡಿತರು, ಗಣ್ಯರು ಬರುತ್ತಿದ್ದರು. ಆದರೆ ಸ್ವಾಮೀಜಿಯವರ ಕರುಣೆಯ ಮಹಾಪೂರದಲ್ಲಿ ಆ ವಿಷಯಗಳೆಲ್ಲ ಕೊಚ್ಚಿಹೋಗುತ್ತಿದ್ದವು. ಬದಲಾಗಿ ಈ ಪಂಡಿತರು ಯಾವ ವಿಷಯವನ್ನು ದೈನಂದಿನ ಜೀವನಕ್ಕೆ ಸಂಬಂಧಪಟ್ಟ ಅನಿತ್ಯ ವಿಷಯಗಳೆಂದು ಭಾವಿಸಿದ್ದರೋ ಅಂತಹ ವಿಷಯಗಳ ದಿಕ್ಕಿನಲ್ಲೇ ಸಂಭಾಷಣೆ ಸಾಗುತ್ತಿತ್ತು. ಒಮ್ಮೆ ಮಹಾರಾಷ್ಟ್ರದ ಪ್ರಸಿದ್ಧ ಪತ್ರಕರ್ತರಾದ ಪಂಡಿತ ಸಖಾರಾಂ ಗಣೇಶ್ ದೇವಸ್ಕರ್ ಎಂಬುವರು ತಮ್ಮಿಬ್ಬರು ಸ್ನೇಹಿತ ರೊಂದಿಗೆ ಸ್ವಾಮೀಜಿಯವರ ಭೇಟಿಗಾಗಿ ಬಂದರು. ಅವರಲ್ಲೊಬ್ಬರು ಪಂಜಾಬಿನವರೆಂಬುದನ್ನು ತಿಳಿದ ಸ್ವಾಮೀಜಿ ಅವರೊಂದಿಗೆ ಪಂಜಾಬಿನ ಕುರಿತಾದ ಮಾತುಕತೆಯನ್ನು ಪ್ರಾರಂಭಿಸಿದರು. ಆ ಸಮಯದಲ್ಲಿ ಪಂಜಾಬು ಎದುರಿಸುತ್ತಿದ್ದ ಕ್ಷಾಮ ಪರಿಸ್ಥಿತಿ, ಅಲ್ಲಿನ ಜನರ ಇತರ ಆವಶ್ಯಕತೆ ಗಳು, ಅವುಗಳನ್ನು ಈಡೇರಿಸುವ ಬಗೆ–ಈ ವಿಚಾರಗಳ ಕುರಿತಾಗಿ ಮಾತುಕತೆ ನಡೆಯಿತು. ಬಳಿಕ ಮಾತಿನ ವಿಷಯ ಕೆಳವರ್ಗಗಳ ಜನರ ಕುರಿತಾಗಿ ತಿರುಗಿತು. ಈ ಜನಗಳ ಆರ್ಥಿಕ ಮತ್ತು ಸಾಮಾಜಿಕ ಸ್ಥಿತಿಗತಿಗಳ ಸುಧಾರಣೆ, ಅವರಿಗೆ ವಿದ್ಯಾಭ್ಯಾಸವನ್ನು ಒದಗಿಸುವಲ್ಲಿ ಮೇಲ್ವರ್ಗದವರ ಕರ್ತವ್ಯ–ಇವುಗಳನ್ನು ಅವರು ಚರ್ಚಿಸಿದರು. ಆದರೆ ಸಂಭಾಷಣೆಯೆಲ್ಲ ಮುಗಿದ ಬಳಿಕ ಸ್ವಾಮೀಜಿಯವರಿಂದ ಬೀಳ್ಗೊಳ್ಳುವ ಮುನ್ನ ಪಂಜಾಬೀ ಸಂದರ್ಶಕ ವಿನಯದಿಂದಲೇ ತನ್ನ ವಿಷಾದವನ್ನು ವ್ಯಕ್ತಪಡಿಸುತ್ತ ಹೇಳಿದ, “ಸ್ವಾಮೀಜಿ, ಧರ್ಮದ ಕುರಿತಾಗಿ ನಿಮ್ಮಿಂದ ಅನೇಕ ವಿಷಯಗಳನ್ನು ಕೇಳಬಹುದೆಂಬ ಮಹದಾಸೆಯಿಂದ ನಿಮ್ಮನ್ನು ನೋಡಲು ಬಂದೆವು. ಆದರೆ ದುರದೃಷ್ಟವಾಶಾತ್ ನಮ್ಮ ಮಾತುಕತೆ ತೀರ ಸಾಧಾರಣ ವಿಷಯಗಳತ್ತ ತಿರುಗಿತು. ಇಂದಿನ ದಿನ ವ್ಯರ್ಥವಾಗಿ ಕಳೆದುಹೋಯಿತು!” ಆದರೆ ಆ ವ್ಯಕ್ತಿ ತಿಳಿದುಕೊಂಡಿದ್ದಂತೆ ಸ್ವಾಮೀಜಿ ಅಕಸ್ಮಾತ್ತಾಗಿ ಪಂಜಾಬಿನ ವಿಷಯವನ್ನು ಪ್ರಸ್ತಾಪಿಸಿದ್ದಲ್ಲ. ಅದು ಅವರು ನೀಡುವ ಉತ್ತರ ದಿಂದಲೇ ಗೊತ್ತಾಗುತ್ತದೆ. ಹೇಳಿದರು, “ನೋಡಿ, ಎಲ್ಲಿಯವರೆಗೆ ನನ್ನ ದೇಶದ ಒಂದು ನಾಯಿಯೂ ಕೂಡ ಅನ್ನವಿಲ್ಲದೆ ಇರುತ್ತದೆಯೋ ಅಲ್ಲಿಯವರೆಗೆ ಅದಕ್ಕೆ ಆಹಾರವಿತ್ತು ಪೂಜಿಸು ವುದೇ ನನ್ನ ಪಾಲಿಗೆ ಧರ್ಮ; ಉಳಿದುದೆಲ್ಲ ಅಧರ್ಮ, ಇಲ್ಲವೆ ಸುಳ್ಳುಧರ್ಮ!” ಈ ಮಾತು ಕೇಳಿ ಆ ಮೂವರು ಸಂದರ್ಶಕರೂ ದಂಗುಬಡಿದುಹೋದರು. ಸ್ವಾಮೀಜಿಯವರು ಮಹಾ ಸಮಾಧಿ ಹೊಂದಿದ ಎಷ್ಟೋ ವರ್ಷಗಳ ಬಳಿಕ ಈ ಘಟನೆಯ ಬಗ್ಗೆ ಮಾತನಾಡುತ್ತ ಪಂಡಿತ ದೇವಸ್ಕರ್ ಹೇಳುತ್ತಾರೆ: “ಸ್ವಾಮೀಜಿಯವರ ಈ ಮಾತುಗಳು ನನ್ನ ಆಂತರ್ಯವನ್ನೇ ಇರಿದುವು. ನಿಜವಾದ ರಾಷ್ಟ್ರಪ್ರೇಮವೆಂದರೆ ಏನೆಂಬುದನ್ನು ಹಿಂದೆಂದೂ ಊಹಿಸದ ರೀತಿಯಲ್ಲಿ ಅಂದು ನಾನು ಕಂಡುಕೊಂಡೆ.”

ಇನ್ನೊಮ್ಮೆ ಉತ್ತರ ಭಾರತದ ಒಬ್ಬ ಪಂಡಿತನು ಸ್ವಾಮೀಜಿಯವರೊಂದಿಗೆ ವೇದಾಂತದ ಬಗ್ಗೆ ವಾದಮಾಡಲು ಬಂದ. ಆದರೆ ಆಗ ಸ್ವಾಮೀಜಿ ದೇಶದ ಹಲವೆಡೆಗಳಲ್ಲಿ ಉಂಟಾಗಿದ್ದ ಕ್ಷಾಮ ಪರಿಸ್ಥಿತಿಯ ನಿವಾರಣೆಗಾಗಿ ತಮ್ಮಿಂದ ಏನೂ ಮಾಡಲು ಸಾಧ್ಯವಾಗುತ್ತಿಲ್ಲವಲ್ಲ ಎಂದು ಬಹಳ ದುಃಖಕ್ಕೀಡಾಗಿದ್ದರು. ಆದ್ದರಿಂದ ಅವರು ಆ ಪಂಡಿತನಿಗೆ ವೇದಾಂತದ ಬಗ್ಗೆ ಚರ್ಚಿಸಲು ಅವಕಾಶವನ್ನೇ ಕೊಡದೆ ಹೇಳಿದರು, “ಪಂಡಿತ್​ಜೀ, ಮೊಟ್ಟಮೊದಲಿಗೆ, ಎಲ್ಲೆಲ್ಲೂ ಕಂಡುಬರು ತ್ತಿರುವ ದುಃಖದಾರಿದ್ರ್ಯದ ಪರಿಸ್ಥಿತಿಯನ್ನು ಸುಧಾರಿಸಲು ಪ್ರಯತ್ನಪಡಿ. ತುತ್ತು ಕೂಳಿಗಾಗಿ ಗೋಳಿಡುತ್ತಿರುವ, ಹಸಿದು ಕಂಗಾಲಾಗಿರುವ ಜನರ ಹೃದಯವಿದ್ರಾವಕ ರೋದನವನ್ನು ಶಾಂತ ಗೊಳಿಸಿ. ಆಮೇಲೆ ಬಂದು ನೀವು ನನ್ನೊಂದಿಗೆ ವೇದಾಂತದ ಬಗ್ಗೆ ಚರ್ಚೆ ಮಾಡಬಹುದು. ಉಪವಾಸದಿಂದ ಸಾವಿರಾರು ಜನರನ್ನು ಬದುಕಿಸಲು ತನ್ನ ಸಮಸ್ತ ಜೀವನ ಮತ್ತು ಸಂಪೂರ್ಣ ಶಕ್ತಿಯನ್ನು ಪಣವಾಗಿಡುವುದೇ ವೇದಾಂತದ ಸಾರ, ಧರ್ಮದ ತಿರುಳು.”

ಸ್ವಾಮೀಜಿಯವರು ಹೀಗೆ ಕೆಲವೊಮ್ಮೆ ವೇದಾಂತ ಅಥವಾ ತಾತ್ತ್ವಿಕ ವಿಚಾರಗಳ ಕುರಿತಾಗಿ ಮಾತನಾಡದೆ ಲೌಕಿಕ ಯೋಗಕ್ಷೇಮದ ಕುರಿತಾಗಿಯೇ ಮಾತನಾಡುತ್ತಿದ್ದರು. ಆದರೆ ಹಾಗೆಂದ ಮಾತ್ರಕ್ಕೆ ವೇದಾಂತವು ಎಲ್ಲರಿಗೂ ಹೇಳಿಸಿದ್ದಲ್ಲ ಅಥವಾ ತಾತ್ತ್ವಿಕ ವಿಚಾರಗಳು ಎಲ್ಲರಿಗೂ ಆವಶ್ಯಕವಲ್ಲ ಎಂದು ಸ್ವಾಮೀಜಿ ಭಾವಿಸಿದ್ದಿರಬಹುದೆ ಎಂಬ ಶಂಕೆ ಬೇಡ. ಏಕೆಂದರೆ ಅವರು ಇಂತಹ ಉನ್ನತ ವಿಷಯಗಳ ಬಗ್ಗೆಯೂ ಹಲವಾರು ಸಲ ಚರ್ಚಿಸುತ್ತಿದ್ದುದುಂಟು. ಉದಾ ಹರಣೆಗೆ ಶರಚ್ಚಂದ್ರ ಚಕ್ರವರ್ತಿಯಂತಹ ಸತ್ಯಾನ್ವೇಷಿಗಳಾಗಲಿ ಅಥವಾ ಇತರ ಭಕ್ತಾದಿಗಳೇ ಆಗಲಿ ಬಂದು ಪ್ರಶ್ನಿಸಿದಾಗ ಅವರು ಅತ್ಯುನ್ನತ ತತ್ತ್ವಶಾಸ್ತ್ರದಿಂದ ಹಿಡಿದು ತೀರ ಸಾಮಾನ್ಯವೆನ್ನಿಸ ಬಹುದಾದ ಯಾವುದೇ ಲೌಕಿಕ ವಿಷಯದ ಬಗ್ಗೆಯೂ ಸಹಜವಾಗಿಯೇ ಉತ್ತರಿಸುತ್ತಿದ್ದರು. ಅಷ್ಟೇ ಅಲ್ಲ, ಪ್ರಶ್ನೆಗಳು ಅವರಿಗೆ ಮುಜುಗರವುಂಟುಮಾಡುವಂಥದಾಗಿದ್ದರೂ ಬೇಸರಿಸು ತ್ತಿರಲಿಲ್ಲ. ಉದಾಹರಣೆಗೆ ಅವರು ಪಶ್ಚಿಮ ದೇಶಗಳಿಂದ ಕಲ್ಕತ್ತಕ್ಕೆ ಮರಳಿದ ಕೆಲದಿನಗಳಲ್ಲೇ ಅವರ ಬಾಲ್ಯಸ್ನೇಹಿತನಾದ ಪ್ರಿಯನಾಥ ಸಿನ್ಹ ಅವರ ಭೇಟಿಗಾಗಿ ಬಂದ. ಕಲ್ಕತ್ತದಲ್ಲಿ ಸ್ವಾಮೀಜಿ ಯವರ ಗೌರವಾರ್ಥವಾಗಿ ನಡೆದ ಭಾರೀ ಸಮಾರಂಭದ ಬಗ್ಗೆ ಪ್ರಸ್ತಾಪಿಸಿ ಅವನೆಂದ, “ಈಗ ಬಂಗಾಳದಲ್ಲೆಲ್ಲ ಇಂತಹ ಭಾರೀ ಬರಗಾಲ ಬಡಿದಿದೆ. ಹಾಗಿರುವಾಗ ನಿನ್ನ ಸನ್ಮಾನಕ್ಕಾಗಿ ಅಷ್ಟೆಲ್ಲ ಅದ್ಧೂರಿಯ ಸಮಾರಂಭವನ್ನು ಮಾಡಿದರಲ್ಲ! ನಿಜ ಹೇಳಬೇಕೆಂದರೆ, ನಾನೆಣಿಸಿದ್ದೆ, ‘ಸಮಾ ರಂಭಕ್ಕಾಗಿ ಸಂಗ್ರಹವಾದ ಹಣವನ್ನೆಲ್ಲ ಬರಪರಿಹಾರಕ್ಕಾಗಿ ಉಪಯೋಗಿಸಿಬಿಡಿ’ ಅಂತ ನೀನು ಮೊದಲೇ ತಂತಿ ಕಳಿಸುತ್ತೀ ಅಂತ.” ಅದಕ್ಕೆ ಸ್ವಾಮೀಜಿ ತಕ್ಷಣ ನುಡಿದರು, “ನೋಡು, ಈ ರೀತಿ ಸ್ವಲ್ಪ ಗಲಭೆ-ಗದ್ದಲ ಆಗಬೇಕು ಎನ್ನುವುದೇ ನನ್ನ ಉದ್ದೇಶವಾಗಿತ್ತು. ಏಕೆ ಎನ್ನುವಿಯೊ? ಹಾಗಾದರೂ ಜನಗಳಿಗೆ ಶ್ರೀರಾಮಕೃಷ್ಣರ ವಿಷಯ ಸ್ವಲ್ಪ ತಿಳಿಯಲಿ, ಅವರ ದಿವ್ಯ ಜೀವಿತೋ ದ್ದೇಶ ಎಲ್ಲರ ಗಮನ ಸೆಳೆಯುವಂತಾಗಲಿ ಎಂದು. ಅಷ್ಟಲ್ಲದೆ, ಈ ಸನ್ಮಾನ-ಗಿನ್ಮಾನಗಳಿಂದೆಲ್ಲ ನನಗೆ ಬಂದ ಭಾಗ್ಯವೇನು? ಹಿಂದೆ ನಾನೂ ನೀನೂ ನಿನ್ನ ಮನೆಯಲ್ಲಿ ಆಟವಾಡುತ್ತಿದ್ದಾಗ್ಗಿಂತ ಈಗ ನಾನೇನಾದರೂ ದೊಡ್ಡಮನುಷ್ಯನಾಗಿಬಿಟ್ಟೆ ಅಂತ ನಿನಗನಿಸುತ್ತದೆಯೆ?” ಇದನ್ನು ಕೇಳಿ ಪ್ರಿಯನಾಥ ಅಪ್ರತಿಭನಾದರೂ ಸಾವರಿಸಿಕೊಂಡು ಹೇಳಿದ, “ಅದು ಸರಿ, ಅಂಥಾ ವ್ಯತ್ಯಾಸವೇನೂ ನನಗೆ ಕಾಣುವುದಿಲ್ಲ.” ಈ ಘಟನೆಯನ್ನು ಮುಂದೆ ಅವನು ಇತರರಿಗೆ ತಿಳಿಸುವಾಗ ಹೇಳುತ್ತಾನೆ, “ನಾನಂದು ಬಾಯಲ್ಲೇನೋ ಹಾಗೆ ಹೇಳಿದೆ. ಆದರೆ ಮನಸ್ಸಿನಲ್ಲೇ ಅಂದುಕೊಂಡೆ, ‘ನರೇನ್, ನಿಜಕ್ಕೂ ನೀನಿಂದು ದೇವರೇ ಆಗಿದ್ದೀಯೆ!’ ಎಂದು.”

ಬಳಿಕ ಪ್ರಿಯನಾಥನೊಂದಿಗೆ ಕ್ಷಾಮಡಾಮರಗಳ ಬಗ್ಗೆ ಮಾತನಾಡುತ್ತ ಸ್ವಾಮೀಜಿ ಹೇಳಿದರು, “ಇತರ ದೇಶಗಳಲ್ಲಿ ಕ್ಷಾಮಗಳು ತುಂಬ ಅಪರೂಪ. ಆದರೆ ಭಾರತದಲ್ಲಿ ಮಾತ್ರ ಇವು ಮರು ಕಳಿಸುತ್ತಲೇ ಇರುತ್ತವೆ. ಅದಕ್ಕೆ ಕಾರಣವೇನು ಗೊತ್ತೆ? ಇತರ ದೇಶಗಳಲ್ಲಿನ ಜನ ಮನುಷ್ಯರಾಗಿ ದ್ದಾರೆ. ಆದರೆ ಇಲ್ಲಿಯವರಿನ್ನೂ ಮೃತಪ್ರಾಯರಾಗಿಯೇ ಇದ್ದಾರೆ. ಇಲ್ಲಿ ಚಟುವಟಿಕೆಯೆ ಕಾಣುವುದಿಲ್ಲ. ಮೊದಲು ಭಾರತೀಯರು ಶ್ರೀರಾಮಕೃಷ್ಣರ ವಿಷಯವನ್ನು ತಿಳಿದುಕೊಂಡು ತನ್ಮೂಲಕ ತಮ್ಮ ಸ್ವಾರ್ಥಬುದ್ಧಿಯನ್ನು ಕಳೆದುಕೊಳ್ಳಲಿ, ಪರಸ್ಪರ ಸಹಕಾರದಿಂದ ಬದುಕುವು ದನ್ನು ಕಲಿಯಲಿ. ಆಗ ಹೀಗೆ ಪದೇಪದೇ ಮರುಕಳಿಸುವ ಕ್ಷಾಮಡಾಮರಗಳನ್ನೇ ತಡೆಗಟ್ಟುವಂತಹ ಪ್ರಯತ್ನ ಮಾಡಲು ಜನರಿಗೆ ಸಾಧ್ಯವಾಗುತ್ತದೆ. ನಾನು ಆ ದಿಸೆಯಲ್ಲೇ ಕಾರ್ಯಮಗ್ನನಾಗು ವುದನ್ನು ನೀನು ನೋಡಲಿದ್ದೀಯೆ.”

ಟಾಗೋರರ ಮೊಮ್ಮಗಳೂ ‘ಭಾರತಿ’ ಎಂಬ ಪತ್ರಿಕೆಯ ಸಂಪಾದಕಿಯೂ ಆದ ಸರಲಾ ಘೋಷಾಲಳು, ಸ್ವಾಮೀಜಿ ಬಹು ಚೆನ್ನಾಗಿ ಅಡಿಗೆ ಮಾಡಬಲ್ಲರೆಂದು ಕೇಳಿ ಕುತೂಹಲಗೊಂಡು, “ಇದು ನಿಜವೆ?” ಎಂದು ನಿವೇದಿತೆಯನ್ನು ವಿಚಾರಿಸಿದಳು. ಈ ವಿಷಯವನ್ನು ತಿಳಿದ ಸ್ವಾಮೀಜಿ ಇಬ್ಬರನ್ನೂ ಒಂದು ದಿನ ಮಠಕ್ಕೆ ಆಹ್ವಾನಿಸಿದರು. ತಮ್ಮ ಕೈಯಿಂದಲೇ ಕೆಲವು ಭಕ್ಷ್ಯಗಳನ್ನು ತಯಾರಿಸಿ ಬಡಿಸಿದರು. ಅನಂತರ ಅವರಿಬ್ಬರೊಡನೆ ಮಾತನಾಡುವ ಸಂದರ್ಭದಲ್ಲಿ, ತಮ್ಮ ಇತರ ಶಿಷ್ಯರಿಗೆ ಹೇಳುತ್ತಿದ್ದಂತೆಯೇ, ತಮಗೆ ಗುಡುಗುಡಿ ತಯಾರಿಸಿ ಕೊಡುವಂತೆ ನಿವೇದಿತೆಗೆ ಹೇಳಿದರು. ತಕ್ಷಣ ನಿವೇದಿತೆ ಮೇಲೆದ್ದು ತನ್ನ ಗುರುದೇವನ ಸೇವೆ ಮಾಡುವ ಸೌಭಾಗ್ಯ ದೊರೆತ ದ್ದಕ್ಕಾಗಿ ಸಂತೋಷಿಸುತ್ತ ಗುಡುಗುಡಿ ತಯಾರಿಸಿ ಕೊಟ್ಟಳು. ಬಳಿಕ ಸ್ವಲ್ಪ ಹೊತ್ತಿನಲ್ಲೇ ನಿವೇದಿತಾ ಮತ್ತು ಸರಳಾ ಘೋಷಾಲರಿಬ್ಬರೂ ಸ್ವಾಮೀಜಿಯವರಿಂದ ಬೀಳ್ಕೊಂಡು ತೆರಳಿದರು. ಆಗ ಸ್ವಾಮೀಜಿ ತಮ್ಮ ಸಹಸಂನ್ಯಾಸಿಗಳಿಗೆ ಹೇಳಿದರು, “ನಾನು ಗುಡುಗುಡಿ ತಯಾರಿಸಿ ಕೊಡು ವಂತೆ ನಿವೇದಿತೆಗೆ ಹೇಳಿದ್ದು ಏಕೆ ಗೊತ್ತೇನು? ಬಂಗಾಳದ ವಿದ್ಯಾವಂತ ವರ್ಗದ ಕೆಲವರು ಮಾತನಾಡಿಕೊಳ್ಳುತ್ತಾರಂತೆ, ‘ಸ್ವಾಮೀಜಿಯವರು ಈ ಪಾಶ್ಚಾತ್ಯ ಶಿಷ್ಯರನ್ನೆಲ್ಲ, ಒಳ್ಳೇ ಬೆಣ್ಣೆಯ ಮಾತುಗಳನ್ನಾಡುತ್ತ ಮತ್ತು ಅವರ ಒಲವನ್ನು ನೋಡಿಕೊಂಡು ಅದಕ್ಕೆ ತಕ್ಕ ಹಾಗೆ ನಡೆದು ಕೊಳ್ಳುತ್ತ ಆಕರ್ಷಿಸಿ ಹಿಡಿದಿಟ್ಟುಕೊಂಡಿದ್ದಾರೆ’ ಅಂತ. ಆದ್ದರಿಂದ ನಾನು ಬೇಕೆಂದೇ ನಿವೇದಿತೆಗೆ ಆ ಕೆಲಸ ಹೇಳಿದೆ.”

೧೮೯೯ರಮಾರ್ಚ್ ವೇಳೆಗೆ ಸ್ವಾಮೀಜಿಯವರ ಆರೋಗ್ಯಸ್ಥಿತಿ ಮತ್ತೆ ಇಳಿಮುಖವಾಯಿತು. ಈ ಸಮಯದಲ್ಲಿ ಗಂಗಾನದಿಯ ಮೇಲೆ ಬೀಸುವ ತಂಗಾಳಿಯಲ್ಲಿ ವಿರಮಿಸಲು ಸಾಧ್ಯವಾಗು ವಂತೆ ಅವರ ಅಭಿಮಾನಿಗಳಾದ ಕೆಲವು ಶ್ರೀಮಂತರು ಅವರಿಗಾಗಿ ಒಂದು ದೋಣಿಮನೆಯನ್ನು ಬಿಟ್ಟುಕೊಟ್ಟರು. ಈ ದೋಣಿಯಲ್ಲಿ ಕುಳಿತು ಸ್ವಾಮೀಜಿ ಮುಂಜಾವಿನಲ್ಲಿ ಮತ್ತು ಸಂಜೆಯ ವೇಳೆಯಲ್ಲಿ ವಾಯುಸಂಚಾರ ಮಾಡಲು ಅನುಕೂಲವಾಯಿತು. ಅನೇಕ ವೇಳೆ ಅವರು ದೋಣಿ ಮನೆಯ ಛಾವಣಿಯ ಮೇಲೆ ಕುಳಿತು ಧ್ಯಾನಮಗ್ನರಾಗುತ್ತಿದ್ದರು. ಇನ್ನು ಕೆಲವೊಮ್ಮೆ ಅವ ರೊಂದು ‘ಮಹಾಶಿಶು’ವಿನ ಭಾವದಲ್ಲಿರುತ್ತಿದ್ದರು. ಮಂದಹಾಸ ತುಂಬಿದ ಮುಖದಿಂದ, ಮೃದುಮಧುರ ಭಾವವನ್ನು ಸೂಸುವ ಕಣ್ಣುಗಳಿಂದ ಕೂಡಿದ, ತನ್ನತನವನ್ನೇ ಮರೆತ ಅವರ ಪ್ರತಿಯೊಂದು ಚಲನವಲನವೂ ಅವರ ಜಿತೇಂದ್ರಿಯತ್ವವನ್ನು ಸಾರಿಹೇಳುತ್ತಿತ್ತು. ದೋಣಿಯು ಸಾಮಾನ್ಯವಾಗಿ ಉತ್ತರ ದಿಕ್ಕಿನಲ್ಲಿ ದಕ್ಷಿಣೇಶ್ವರದತ್ತ ನಿಧಾನವಾಗಿ ಸಾಗುತ್ತಿತ್ತು. ಮುಂಜಾನೆಯ ಅಥವಾ ಸಂಜೆಯ ನಸುಬೆಳಕಿನಲ್ಲಿ ಸ್ವಾಮೀಜಿ ಆಗಾಗ ಗಾಢ ಚಿಂತನೆಯಲ್ಲಿ ಮುಳುಗಿರುತ್ತಿದ್ದರು. ತಮ್ಮ ಶಿಷ್ಯರೊಂದಿಗೆ ಮತ್ತು ತಮ್ಮ ದರ್ಶನಾರ್ಥಿಗಳಾಗಿ ಬಂದು ಮುತ್ತುತ್ತಿದ್ದ ಇತರರೊಂದಿಗೆ ಇಡೀ ದಿನವನ್ನು ಪಾಠ-ಪ್ರವಚನ-ಸಂಭಾಷಣೆಗಳಲ್ಲಿ ಕಳೆದ ಬಳಿಕ ಸಂಜೆಯ ಜಲಸಂಚಾರದ ಕಾರ್ಯಕ್ರಮ ಅವರಲ್ಲಿ ನವೋತ್ಸಾಹವನ್ನುಂಟುಮಾಡುತ್ತಿತ್ತು.

ತಮ್ಮ ಶಿಷ್ಯರಲ್ಲಿ ಆತ್ಮವಿಶ್ವಾಸವನ್ನು ತುಂಬುವುದು ಸ್ವಾಮೀಜಿಯವರು ನೀಡುತ್ತಿದ್ದ ಶಿಕ್ಷಣದ ಒಂದು ಬಹು ಮುಖ್ಯ ಅಂಶವಾಗಿತ್ತು. ಈಗ ಅವರು ನಿವೇದಿತೆಯಲ್ಲೂ ಇಂತಹ ಆತ್ಮ ವಿಶ್ವಾಸವನ್ನು ಜಾಗೃತಗೊಳಿಸುವಂತೆ ಮತ್ತು ಜನಸಮೂಹದ ಟೀಕೆಗಳ ಬಗ್ಗೆ ಆಕೆ ತಲೆ ಕೆಡಿಸಿಕೊಳ್ಳದಿರುವಂತೆ ತರಬೇತಿ ನೀಡತೊಡಗಿದರು. ಶ್ರೀರಾಮಕೃಷ್ಣರು ತಮಗೆ ಹೇಳಿಕೊಟ್ಟ ಒಂದು ಅಮೂಲ್ಯ ಪಾಠವನ್ನು ಆಕೆಗೆ ಅವರು ಮತ್ತೆ ಮತ್ತೆ ಹೇಳಿಕೊಟ್ಟರು. ಶ್ರೀರಾಮಕೃಷ್ಣರು ಆಗಾಗ ಹೇಳುತ್ತಿದ್ದ ಮಾತು ಇದು–‘ಲೋಕ್ ನಾ, ಪೋಕ್​’, ‘ಜನ ಅಲ್ಲ, ಹುಳ’; ಅರ್ಥಾತ್, ಜನಗಳೆಂದರೆ ಹುಳುಗಳು. ಸ್ವಾಮೀಜಿ ಶಿಕಾಗೋದಲ್ಲಿದ್ದಾಗ ಒಮ್ಮೆ ತೀವ್ರ ಆತಂಕ-ಕಾತರ ಗಳಿಂದ ಬಳಲಿ ನೆಲದ ಮೇಲೆ ಮಲಗಿದ್ದರು. ಆಗ ಶ್ರೀರಾಮಕೃಷ್ಣರು ಕೋಣೆಯಲ್ಲಿ ಕಾಣಿಸಿ ಕೊಂಡು ಅವರನ್ನು ಮುಟ್ಟಿ ಹೇಳಿದರು, “ಏ, ಕಿ? ಉಠೋ ಗೋ, ಲೋಕ್ ನಾ ಪೋಕ್!” “ಛೆ, ಇದೇನಯ್ಯ? ಎದ್ದೇಳು, ಇವರೆಲ್ಲ ಜನ ಅಲ್ಲ, ಹುಳ!” ಸ್ವಾಮೀಜಿ ತಮ್ಮ ಈ ಅನು ಭವವನ್ನು ನಿವೇದಿತೆಗೆ ಹೇಳಿ ಅವಳಲ್ಲಿ ಆತ್ಮವಿಶ್ವಾಸವುಂಟುಮಾಡಲು ಪ್ರಯತ್ನಿಸಿದರು.

ಈ ದಿನಗಳಲ್ಲಿ ಸ್ವಾಮೀಜಿಯವರು, ತಮ್ಮ ಆರೋಗ್ಯ ಚೆನ್ನಾಗಿರಲಿ ಇಲ್ಲದಿರಲಿ, ಯಾವಾ ಗಲೂ ಬಹಳ ಚಟುವಟಿಕೆಯಿಂದ ಕೂಡಿರುತ್ತಿದ್ದರು. ವೈದ್ಯರ ಆದೇಶದಂತೆ ಅವರು ಭಾಷಣ ಗಳನ್ನು ಮಾಡುವಂತಿರಲಿಲ್ಲ. ಆದರೂ ಅವರು ಸೋದರಿ ನಿವೇದಿತಾ ನೀಡಿದ “ಯುವಭಾರತ ಚಳವಳಿ” ಎಂಬ ಉಪನ್ಯಾಸದ ಸಂದರ್ಭದಲ್ಲಿ ಭಾಷಣ ಮಾಡಿದರು. ಭಾನುವಾರಗಳಂದು ಬೇಲೂರು ಮಠದಲ್ಲಿ ನಡೆಯುತ್ತಿದ್ದ ಸಭೆಗಳಲ್ಲಿ ಅವರೇ ಪ್ರಮುಖ ವ್ಯಕ್ತಿ. ಇವುಗಳಲ್ಲದೆ ಅವರು ಇನ್ನೂ ಅನೇಕ ಸಾರ್ವಜನಿಕ ಸಮಾರಂಭಗಳಲ್ಲಿ ಭಾಗವಹಿಸಿ ಜನರನ್ನು ಸ್ಫೂರ್ತಿಗೊಳಿಸಿದರು.

ಈ ನಡುವೆ ನಿವೇದಿತಾ ತನ್ನನ್ನು ರಾಮಕೃಷ್ಣ ಮಹಾಸಂಘದ ‘ಆಜೀವ ಸದಸ್ಯೆ’ಯನ್ನಾಗಿ ಮಾಡಿಕೊಳ್ಳಬೇಕೆಂದು ಸ್ವಾಮೀಜಿಯವರನ್ನು ಪ್ರಾರ್ಥಿಸಿಕೊಂಡಳು. ಇದಕ್ಕೆ ಸ್ವಾಮೀಜಿ ಸಮ್ಮತಿಸಿ ದರು. ಅವರ ಸ್ಫೂರ್ತಿದಾಯಕ ತರಬೇತಿಯ ಪರಿಣಾಮವಾಗಿ ಆಕೆಯಲ್ಲಿ ತ್ಯಾಗದ ಅಗ್ನಿ ಪ್ರಜ್ವಲಿಸ ಲಾರಂಭಿಸಿತ್ತು. ಆದ್ದರಿಂದ ಅವಳೀಗ ತಾನು ಸಂಪ್ರದಾಯ ರೀತ್ಯಾ ಸಂನ್ಯಾಸ ಸ್ವೀಕರಿಸಬೇಕೆಂಬ ಇಚ್ಛೆಯನ್ನು ವ್ಯಕ್ತಪಡಿಸಿದಳು. ಈಗ ತಾನು ಸಂನ್ಯಾಸ ಪಡೆದುಕೊಳ್ಳಲು ತಕ್ಕ ಕಾಲ ಬಂದಿದೆ ಯೆಂದು ಬಹುಶಃ ಆಕೆಗೆ ಅನ್ನಿಸಿತ್ತು.

ಆದರೆ ಸ್ವಾಮೀಜಿಯವರ ಆಲೋಚನೆಯ ದಿಕ್ಕು ಬೇರೆಯೇ ಆಗಿತ್ತು. ಅವರು ನಿವೇದಿತೆಗೆ ‘ನೈಷ್ಠಿಕ ಬ್ರಹ್ಮಚರ್ಯ’ದ ದೀಕ್ಷೆಯನ್ನು ನೀಡಲು ನಿರ್ಧರಿಸಿದ್ದರೇ ಹೊರತು ಸಂನ್ಯಾಸದೀಕ್ಷೆಯ ನ್ನಲ್ಲ. ೧೮೯೯ರ ಮಾರ್ಚ್ ೨೫ರಂದು–ಎಂದರೆ, ಅವಳು ಮಂತ್ರದೀಕ್ಷೆಯನ್ನು ಪಡೆದ ಒಂದು ವರ್ಷಕ್ಕೆ ಸರಿಯಾಗಿ–ನೈಷ್ಠಿಕ ಬ್ರಹ್ಮಚರ್ಯ ದೀಕ್ಷೆಯನ್ನು ನೀಡಿದರು. ಆ ಸಂದರ್ಭದಲ್ಲಿ ಅನೇಕ ಸಂನ್ಯಾಸಿಗಳು ಉಪಸ್ಥಿತರಿದ್ದರು. ಅಗ್ನಿಕುಂಡದ ಮುಂದೆ ನಿವೇದಿತೆ ಆಜೀವ ಬ್ರಹ್ಮಚರ್ಯ, ವಿಧೇಯತೆ ಮತ್ತು ಸರಳತೆಯ ವ್ರತವನ್ನು ಕೈಗೊಂಡಳು. ವೇದಮಂತ್ರಗಳ ಘೋಷದ ನಡುವೆ ಅಗ್ನಿಗೆ ಆಹುತಿಗಳನ್ನು ನೀಡಲಾಯಿತು. ವಿಧ್ಯುಕ್ತ ಕಾರ್ಯಗಳೆಲ್ಲ ಮುಗಿದ ಮೇಲೆ ನಿವೇದಿತೆ ತನ್ನ ಗುರುದೇವನ ಪಾದಗಳಿಗೆ ಎರಗಿದಳು. ಸ್ವಾಮೀಜಿ ಆಕೆಯ ಹಣೆಯ ಮೇಲೆ ವಿಭೂತಿಯನ್ನು ಹಚ್ಚಿದರು. ಈ ವಿಭೂತಿಯು ವೈರಾಗ್ಯಸೂಚಕ. ನಾವು ಯಾವ ಈ ಶರೀರವನ್ನು ತುಂಬ ಅಭಿಮಾನದಿಂದ ಕೊಬ್ಬಿಸಿ ಅಲಂಕರಿಸಿ ಮೆರೆದಾಡಿಸುತ್ತಿರುತ್ತೇವೊ ಅದು ಒಂದಲ್ಲ ಒಂದು ದಿನ ಹಿಡಿ ಬೂದಿಯಾಗಲಿದೆ ಎಂಬ ಸತ್ಯವನ್ನು ಸದಾ ನೆನಪಿಸಿಕೊಡುತ್ತದೆ ಈ ವಿಭೂತಿ. ಆದ್ದರಿಂದಲೇ ಭೋಗಭೂಮಿಯಿಂದ ಬಂದ ನಿವೇದಿತೆಗೆ ಸ್ವಾಮೀಜಿಯವರು ತ್ಯಾಗಸೂಚಕವಾದ ವಿಭೂತಿ ಯನ್ನು ಹಚ್ಚಿದರು. ಸಂನ್ಯಾಸಿಗಳೊಬ್ಬರು ಯುಕ್ತ ಮಂತ್ರಗಳನ್ನು ಪಠಿಸುವುದರೊಂದಿಗೆ ಸಮಾರಂಭ ಮುಕ್ತಾಯವಾಯಿತು. ಆ ದಿನ ನಿವೇದಿತಾ ಮಠದಲ್ಲೇ ಉಳಿದುಕೊಂಡಳು. ಮಧ್ಯಾಹ್ನ ಊಟವಾದ ಬಳಿಕ ಸ್ವಾಮೀಜಿ ಅವಳಿಗೊಂದು ರುದ್ರಾಕ್ಷಿ ಮಾಲೆಯನ್ನು ನೀಡಿದರು. ಅಂದಿನಿಂದ ಅವಳು ಶ್ವೇತವಸ್ತ್ರವನ್ನು ಧರಿಸಿ ಕೊರಳಿನಲ್ಲಿ ರುದ್ರಾಕ್ಷಿ ಮಾಲೆಯನ್ನು ಧರಿಸಲಾರಂಭಿಸಿದಳು.

ಈ ಸಮಾರಂಭದ ಬಗ್ಗೆ ನಿವೇದಿತೆ ಮಿಸ್ ಮೆಕ್​ಲಾಡಳಿಗೆ ಬರೆಯುತ್ತಾಳೆ:

“ನಾನು ಬೆಳಿಗ್ಗೆ ಎಂಟು ಗಂಟೆಯ ವೇಳೆಗೆ ಮಠಕ್ಕೆ ಹೋದೆ. ಅಲ್ಲಿನ ಪ್ರಾರ್ಥನಾ ಮಂದಿರ ದಲ್ಲಿ ಎಲ್ಲರೂ ನೆಲದ ಮೇಲೆ ಕುಳಿತೆವು. ಪೂಜೆಗಾಗಿ ಹೂವುಗಳು ಬರುವವರೆಗೂ ಸ್ವಾಮೀಜಿ ನನ್ನೊಡನೆ ಬುದ್ಧನ ವಿಷಯವಾಗಿ ಮಾತನಾಡುತ್ತಿದ್ದರು. ಜಾತಕ ಕತೆಗಳ ಪ್ರಕಾರ ಒಬ್ಬಾತ ಇತರರಿಗಾಗಿ ತನ್ನ ಜೀವವನ್ನು ಐನೂರು ಸಲ ಸಮರ್ಪಿಸಿದ ಮೇಲೆ ಆತ ಬೋಧಿಯ ಬೆಳಕನ್ನು ಕಂಡ ಬುದ್ಧನಾಗುತ್ತಾನೆ ಎಂಬ ವಿಷಯವನ್ನು ಸ್ವಾಮೀಜಿ ನನಗೆ ತಿಳಿಸಿದರು. ಎಂತಹ ಸಂದರ್ಭೋಚಿತ ಮಾತಲ್ಲವೆ ಅದು? ಅನಂತರ ಪೂಜಾ ಸಾಮಗ್ರಿಗಳನ್ನು ತಂದಿರಿಸಲಾಯಿತು. ಆಗ ಸ್ವಾಮೀಜಿಯವರು ನನಗೆ ಪೂಜಾ ವಿಧಾನವನ್ನು ಅರ್ಥಸಹಿತವಾಗಿ ಮಧುರ ದನಿಯಲ್ಲಿ ವಿವರಿಸುತ್ತ ತೋರಿಸಿಕೊಟ್ಟರು. ನನ್ನ ನೆತ್ತಿಯ ಮೇಲಿರಿಸಿಕೊಳ್ಳಲು ಅವರು ಕೊಟ್ಟ ಶ್ವೇತ ಪುಷ್ಪವನ್ನು ನಾನು ನಿನಗೆ ಕಳಿಸುತ್ತಿದ್ದೇನೆ.”

ನಿವೇದಿತೆ ಈಗ ನೈಷ್ಠಿಕ ಬ್ರಹ್ಮಚಾರಿಣಿಯಾಗಿದ್ದಾಳೆ. ಆದರೆ ತನ್ನನ್ನು ಸ್ವಾಮೀಜಿಯವರು ಸಂನ್ಯಾಸಿನಿಯನ್ನಾಗಿ ಮಾಡದಿರುವುದು ಅವಳಿಗೆ ತುಂಬ ನಿರಾಶೆಯನ್ನುಂಟುಮಾಡಿತ್ತು. ಅವಳು ಶ್ರೀಮತಿ ಸಾರಾ ಹಾಗೂ ಮಿಸ್ ಮೆಕ್​ಲಾಡ್​ಳಿಗೆ ಬರೆದ ಪತ್ರಗಳಿಂದ ಇದು ವ್ಯಕ್ತವಾಗುತ್ತದೆ. ಶ್ರೀಮತಿ ಸಾರಾಳಿಗೆ ಆಕೆ ಬರೆಯುತ್ತಾಳೆ: “ಸ್ವಾಮೀಜಿಯವರು ನನ್ನನ್ನು ಆಜನ್ಮಬ್ರಹ್ಮಚಾರಿಣಿ ಯನ್ನಾಗಿ ಮಾಡಿದ್ದರಲ್ಲಿ ಎರಡು ಉದ್ದೇಶಗಳಿರಬಹುದೆಂದು ನನಗನ್ನಿಸುತ್ತದೆ–ಮೊದಲನೆಯ ದಾಗಿ ನೈಷ್ಠಿಕ ಬ್ರಹ್ಮಚಾರಿಣಿಯರ ಸಂಪ್ರದಾಯವನ್ನು ಪುನಃ ಸ್ಥಾಪಿಸುವುದು; ಎರಡನೆಯದಾಗಿ, ಪ್ರಾಯಶಃ ಅವರ ದೃಷ್ಟಿಯಲ್ಲಿ ನಾನಿನ್ನೂ ಇದಕ್ಕಿಂತ ಉತ್ತಮವಾದದ್ದಕ್ಕೇನೂ ಅರ್ಹಳಾಗಿಲ್ಲದಿರುವುದು.”

ಈ ವಿಷಯವಾಗಿ ನಿವೇದಿತಾ ಸ್ವಾಮೀಜಿಯವರೊಂದಿಗಿನ್ನೂ ಮಾತನಾಡಿರಲಿಲ್ಲ. ಕೆಲದಿನ ಗಳಲ್ಲೇ ಆಕೆ ಮಠಕ್ಕೆ ಹೋಗಿದ್ದಾಗ ಅದಕ್ಕೊಂದು ಅವಕಾಶ ಸಿಕ್ಕಿತು. ಆಕೆ ಕೇಳಿದಳು, “ನನ್ನಲ್ಲಿ ಸಂನ್ಯಾಸಿನಿಯಾಗುವ ಯೋಗ್ಯತೆ ಉಂಟಾಗಬೇಕಾದರೆ ನಾನು ಯಾವ ಪರಿಪೂರ್ಣತೆಯ ಮಟ್ಟ ಕ್ಕೇರಬೇಕಾಗಿದೆ ಸ್ವಾಮೀಜಿ?” ಸ್ವಾಮೀಜಿಯವರು ಅದಕ್ಕೊಂದು ಮಾರ್ಗವನ್ನು ಸೂಚಿಸಿದರೆ ಸಾಕು, ಎಷ್ಟೇ ಕಷ್ಟವಾದರೂ ತಾನು ಅದಕ್ಕನುಗುಣವಾಗಿ ನಡೆದುಕೊಂಡು ಸಂನ್ಯಾಸವನ್ನು ಸ್ವೀಕರಿಸುವಂತಾಗಬೇಕು ಎಂಬುದು ನಿವೇದಿತೆಯ ಹಂಬಲ. ಆದರೆ ಸ್ವಾಮೀಜಿ ಒಂದೇ ಮಾತಿನಲ್ಲಿ ಆ ವಿಚಾರಕ್ಕೆ ಪೂರ್ಣವಿರಾಮ ಹಾಕಿದರು, “ನೀನು ಈಗಿರುವಂತೆಯೇ ಇರು.” ನಿವೇದಿತಾ ಆ ವಿಚಾರವನ್ನು ಮತ್ತೆಂದೂ ಪ್ರಸ್ತಾಪಿಸಲಿಲ್ಲ.

೧೮೯೯ರ ಮಾರ್ಚ್ ೨೮; ಶ್ರೀರಾಮಕೃಷ್ಣ ಮಹಾಸಂಘದ ಸದಸ್ಯರಿಗೆಲ್ಲ ಅದೊಂದು ದುಃಖದ ದಿನ. ಅಂದು ಶ್ರೀರಾಮಕೃಷ್ಣರ ಸಂನ್ಯಾಸೀ ಶಿಷ್ಯರಲ್ಲೊಬ್ಬರಾದ ಸ್ವಾಮಿ ಯೋಗಾ ನಂದರು ಮಹಾಸಮಾಧಿಯನ್ನೈದಿದರು. ಅವರು ಕೆಲಕಾಲದಿಂದ ತೀವ್ರ ರೋಗಗ್ರಸ್ತರಾಗಿದ್ದರು. ಅವರ ಮರಣದ ಬಗ್ಗೆ ಸ್ವಾಮೀಜಿ ವಿಷಾದದಿಂದ ನುಡಿದರು, “ನಮ್ಮ ಕಟ್ಟಡದ ಇಟ್ಟಿಗೆ ಗಳಲ್ಲೊಂದು ಬಿದ್ದುಹೋಯಿತು” ಎಂದು.

