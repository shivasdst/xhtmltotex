
\chapter{ಪೌರ್ವಾತ್ಯ ಪ್ರವಾದಿ}

\noindent

ನವೆಂಬರ್ ೨೨ರಂದು ನ್ಯೂಯಾರ್ಕಿನಿಂದ ಹೊರಟ ಸ್ವಾಮೀಜಿಯವರು, ತಮ್ಮ ಆಪ್ತ ಶಿಷ್ಯರ ಬೇಡಿಕೆಯನ್ನು ಮನ್ನಿಸಿ, ದಾರಿಯಲ್ಲಿನ ಶಿಕಾಗೋ ನಗರಕ್ಕೆ ಬಂದರು. ಅವರಿಗೆ ಪರಮ ಪ್ರಿಯ ರಾದ ಹೇಲ್ ಕುಟುಂಬದವರು ಇರುವುದು ಇಲ್ಲಿಯೇ. ಅಲ್ಲದೆ ರಿಡ್ಜ್​ಲಿ ಮ್ಯಾನರಿಂದ ಹೊರಟ ನಿವೇದಿತೆ, ತನ್ನ ಭಾರತದ ಯೋಜನೆಗಳಿಗಾಗಿ ಧನಸಂಗ್ರಹ ಮಾಡಲು ಇಲ್ಲಿಗೆ ಬಂದಿದ್ದಳು. ಇವರನ್ನೆಲ್ಲ ಭೇಟಿಯಾಗುವುದು ಸ್ವಾಮೀಜಿಯವರ ಒಂದು ಉದ್ದೇಶ. ಇದರೊಂದಿಗೆ, ನ್ಯೂಯಾರ್ ಕಿನಿಂದ ಕ್ಯಾಲಿಫೋರ್ನಿಯವವರೆಗಿನ ಎರಡು ಸಾವಿರ ಕಿಲೋಮೀಟರಿನಷ್ಟು ದೀರ್ಘ ಪ್ರಯಾಣದ ನಡುವೆ ಕೆಲಕಾಲ ವಿರಮಿಸುವುದೂ ಅವರ ಇನ್ನೊಂದು ಉದ್ದೇಶ.

ಶಿಕಾಗೋದ ರೈಲುನಿಲ್ದಾಣದಲ್ಲಿ ಮೇರಿ ಹಾಗೂ ಇನ್ನಿತರ ಆಪ್ತರು ಸ್ವಾಮೀಜಿಯವರನ್ನು ಎದುರುಗೊಂಡರು. ಬಳಿಕ ಸ್ವಾಮೀಜಿ ಹೇಲ್ ಕುಟುಂಬದವರ ಮನೆಗೆ ತೆರಳಿದರು. ಸ್ವಾಮೀಜಿ ಯವರನ್ನು ಅಮೆರಿಕೆಗೆ, ಅಷ್ಟೇಕೆ ಇಡೀ ವಿಶ್ವಕ್ಕೇ ಪರಿಚಯಿಸಿಕೊಟ್ಟ ನಗರ ಈ ಶಿಕಾಗೋ. ಅವರು ಹಿಂದೆ ಇಲ್ಲಿಗೆ ಅನೇಕ ಬಾರಿ ಬಂದಿದ್ದರು, ಅಸಂಖ್ಯಾತ ಬಾಷಣಗಳನ್ನು ಮಾಡಿದ್ದರು, ತರಗತಿಗಳನ್ನು ನಡೆಸಿದ್ದರು. ಆದ್ದರಿಂದ ಇಲ್ಲಿ ಅವರ ಶಿಷ್ಯರು, ಭಕ್ತರು, ಅಭಿಮಾನಿಗಳು ಅಸಂಖ್ಯಾತ. ಹಿಂದೆ ಅವರ ಆತಿಥೇಯರಾಗಿದ್ದ ಲಿಯಾನ್ ದಂಪತಿಗಳು, ಸುಪ್ರಸಿದ್ಧ ಅಪೇರಾ ನಟಿ ಎಮ್ಮಾ ಕಾಲ್ವೆ ಮುಂತಾದ ಹಳೆಯ ಪರಿಚಯಸ್ಥರು ಸೇರಿದಂತೆ ಹಲವಾರು ಜನ ಸ್ವಾಮೀಜಿ ಯವರ ದರ್ಶನ ಪಡೆದು, ಆನಂದದಿಂದ ಮಾತುಕತೆಯಾಡಿದರು.

ಇವರುಗಳಲ್ಲದೆ ಸ್ವಾಮೀಜಿ ಶಿಕಾಗೋದಲ್ಲಿ ಹೊಸ ಸ್ನೇಹಿತರನ್ನು, ಅನುಯಾಯಿಗಳನ್ನು ಸಂಪಾದಿಸಿಕೊಂಡರು. ಅವರು ಬರೆದ ರಾಜಯೋಗವೇ ಮೊದಲಾದ ಗ್ರಂಥಗಳು ಅಲ್ಲಿನ ಸಮಾಜದ ಮೇಲೆ ಆಳವಾದ ಪ್ರಭಾವ ಬೀರಿದ್ದುವು. ಅವರ ಸ್ಫೂರ್ತಿದಾಯಕ ಬೋಧನೆಗಳಿಂದ ಹಲವಾರು ಜನ ಆಕರ್ಷಿತರಾಗಿದ್ದರು. ಅಲ್ಲದೆ, ವಿದ್ಯಾವಂತ ಜನವರ್ಗದಲ್ಲಿ ಭಾರತ ಹಾಗೂ ಭಾರತೀಯ ಸಂಸ್ಕೃತಿಯ ಬಗ್ಗೆ ಗೌರವ ಮೂಡಿಸುವಲ್ಲಿ ಈ ಪುಸ್ತಕಗಳು ಯಶಸ್ವಿಯಾಗಿದ್ದುವು. ಈ ದಿನಗಳಲ್ಲಿ ಸ್ವಾಮೀಜಿ ತಮ್ಮ ಹಲವಾರು ಪರಿಚಿತರ ಮನೆಗಳಿಗೆ ಹೋಗಿ ಅವರ ಆತಿಥ್ಯ ಸ್ವೀಕರಿಸಿದರು. ನವೆಂಬರ್ ೨೭ರಂದು ಹೇಲ್ ಸೋದರಿಯರು ಏರ್ಪಡಿಸಿದ್ದ ಚಹಾ ಕೂಟದಲ್ಲಿ ತಮ್ಮ ಹಳೆಯ ಹಾಗೂ ಹೊಸ ಪರಿಚಿತರೆಲ್ಲರನ್ನೂ ಉದ್ದೇಶಿಸಿ ಮಾತನಾಡಿದರು.

ಈ ಮಾತುಕತೆ-ಓಡಾಟಗಳ ನಡುವೆಯೇ ಸ್ವಾಮೀಜಿ ಕೆಲಕಾಲ ವಿಶ್ರಾಂತಿ ಪಡೆಯಲೂ ಅವಕಾಶ ಮಾಡಿಕೊಂಡರು. ಅವರ ಅತ್ಯಂತ ಪ್ರೀತಿಪಾತ್ರರಾದ ಹೇಲ್ ಸೋದರಿಯರ ಸಾನ್ನಿ ಧ್ಯವೇ ಅವರಿಗೊಂದು ಆನಂದದ ಸಂಗತಿ. ಇವರ ಮನೆಯೆಂದರೆ ಸ್ವಾಮೀಜಿಯವರಿಗೆ ಸ್ವಂತ ಮನೆಗಿಂತ ಹೆಚ್ಚು. ಕೆಲವೊಮ್ಮೆ ಅವರು, ಮೇರಿ ಮತ್ತು ಹ್ಯಾರಿಯೆಟ್ಟಳ ದ್ವಂದ್ವ ಪಿಯಾನೋ ವಾದನವನ್ನು ಆಲಿಸುತ್ತಿದ್ದರು. ಅವರಿಗಾಗಿ ತಮ್ಮ ಧ್ವನಿಯನ್ನು ಮುದ್ರಿಸಿಕೊಳ್ಳಲೂ ಅವಕಾಶ ಮಾಡಿಕೊಟ್ಟರು. ಆದರೆ ಮೇಣ ಸವರಿದ ತಟ್ಟೆಗಳ ಮೇಲೆ ಮಾಡಿದ ಆ ಧ್ವನಿಮುದ್ರಿಕೆಗಳು ಇಂದು ಲಭ್ಯವಿಲ್ಲ. ಅಲ್ಲದೆ ಆ ಮುದ್ರಿಕೆಗಳು ದೀರ್ಘಕಾಲ ಬಾಳಿಕೆ ಬರುವಂಥವೂ ಅಲ್ಲ.

ಹೀಗೆ, ಕ್ಯಾಲಿಫೋರ್ನಿಯದತ್ತ ಹೊರಟ ಸ್ವಾಮೀಜಿಯವರು ಶಿಕಾಗೋದಲ್ಲಿ ಕೆಲದಿನಗಳನ್ನು ಆನಂದದಿಂದ ಕಳೆದು, ನವೆಂಬರ್ ೩ಂರಂದು ತಮ್ಮ ಪ್ರಯಾಣವನ್ನು ಮುಂದುವರಿಸಿದರು. ಡಿಸೆಂಬರ್ ೩ರಂದು ಅವರು ದಕ್ಷಿಣ ಕ್ಯಾಲಿಫೋರ್ನಿಯದ ಪ್ರಮುಖ ನಗರವಾದ ಲಾಸ್ ಏಂಜೆಲಿಸ್​ಗೆ (‘ದೇವತೆಗಳ ನಗರ’) ಬಂದರು. ಮಿಸ್ ಮೆಕ್​ಲಾಡಳು ಇದ್ದದ್ದು ಇಲ್ಲಿಯೇ. ಇವಳಲ್ಲದೆ, ಸ್ವಾಮೀಜಿಯವರನ್ನು ಇಲ್ಲಿಗೆ ಆಹ್ವಾನಿಸಿದ್ದ ಮತ್ತೊಬ್ಬಳೆಂದರೆ ಶ್ರೀಮತಿ ಬ್ಲಾಜೆಟ್. ಈಕೆ ಲಾಸ್ ಏಂಜೆಲಿಸ್​ನಲ್ಲಿ ಅವರ ಆತಿಥೇಯಳಾಗಲಿದ್ದಳು.

ಸ್ವಾಮೀಜಿಯವರು ಶ್ರೀಮತಿ ಬ್ಲಾಜೆಟ್ಟಳ ಮನೆಗೆ ತೆರಳುವ ಮುನ್ನ ಒಂದು ವಾರ ಕಾಲ ಜೋಸೆಫಿನ್ನಳೊಂದಿಗೆ ಮಿಸ್ ಸ್ಪೆನ್ಸರ್ ಎಂಬವಳ ಮನೆಯಲ್ಲಿ ಉಳಿದುಕೊಂಡರು. ಶೀಘ್ರವಾಗಿ ಇವಳು ಸ್ವಾಮೀಜಿಯವರ ಕಟ್ಟಾ ಅನುಯಾಯಿಗಳಲ್ಲೊಬ್ಬಳಾದಳು. ಮಿಸ್ ಸ್ಪೆನ್ಸರ್ ಹತ್ತು ವರ್ಷಗಳಿಂದಲೂ ತನ್ನ ಕುರುಡಿ ತಾಯಿಯ ನಿರಂತರ ಪ್ರೇಮಪೂರ್ಣ ಸೇವೆಯನ್ನು ಮಾಡಿ ಕೊಂಡು ಬಂದಿದ್ದವಳು. ಈಗ ಆ ತಾಯಿ ಮರುಣೋನ್ಮುಖಳಾಗಿದ್ದಳು. ಕೆಲವೊಮ್ಮೆ ಸ್ವಾಮೀಜಿ ಆಕೆಯ ಹಾಸಿಗೆಯ ಬಳಿಯಲ್ಲೇ ಕುಳಿತು ಆಕೆಯ ಸ್ಥಿತಿಯನ್ನು ಗಮನಿಸುತ್ತಿದ್ದರು. ಇದನ್ನು ಕಂಡು ಒಮ್ಮೆ ಮಿಸ್ ಸ್ಪೆನ್ಸರ್ ಕೇಳಿದಳು, “ಸ್ವಾಮೀಜಿ, ನನ್ನ ತಾಯಿಯ ಬಗ್ಗೆ ಅಷ್ಟೊಂದು ಆಸಕ್ತಿ ತೋರುತ್ತೀರಲ್ಲ, ಏಕೆ?” ಎಂದು. ಅದಕ್ಕೆ ಅವರು ಹೇಳಿದರು, “ಈ ಮರಣ ಎಂಬುದು ಜನನದಂತೆಯೇ ಒಂದು ರಹಸ್ಯ; ಅದೊಂದು ಕುತೂಹಲದ ಸಂಗತಿ. ಮರಣಕಾಲ ಸನ್ನಿಹಿತ ವಾದಂತೆ ಜೀವಾತ್ಮನು ಶರೀರದಿಂದ ಹೊರಹೋಗಲು ಅನುವಾಗುತ್ತಾನೆ. ಆಗ ಇಂದ್ರಿಯಗಳೆಲ್ಲ ಸ್ತಬ್ಧವಾಗುತ್ತವೆ. ಸೂಕ್ಷ್ಮವಾಗಿ ಗಮನಿಸಿದರೆ ಇವುಗಳನ್ನು ಗುರುತಿಸಬಹುದು.” ಸಾಮಾನ್ಯರ ಪಾಲಿಗೆ ಇದೊಂದು ಹೃದಯವಿದ್ರಾವಕ ನೋಟ; ಅಸಹನೀಯ ಅನುಭವ. ಆದರೆ ಸ್ವಾಮೀಜಿ ಯಂತಹ ಯೋಗಿಗಳ ದೃಷ್ಟಿಯೇ ಬೇರೆ. ಅವರಿಗೆ ಇದೊಂದು ಅರ್ಥಪೂರ್ಣವಾದ, ಮಹತ್ವ ಪೂರ್ಣವಾದ ವಿಷಯ.

ಸ್ವಾಮೀಜಿಯವರು ಲಾಸ್ ಏಂಜೆಲಿಸ್​ಗೆ ಬಂದ ಎರಡು-ಮೂರು ದಿನ ಪ್ರಮುಖವಾದ ದ್ದೇನೂ ನಡೆಯಲಿಲ್ಲ. ಆದರೆ ಡಿಸೆಂಬರ್ ಆರನೇ ತಾರೀಕಿನಿಂದ ಅವರ ಜೀವನದಲ್ಲಿ ಅದ್ಭುತ ಘಟನೆಗಳು ಪ್ರಾರಂಭವಾಗುತ್ತವೆಂದು ನಿವೇದಿತಾ ಮುನ್ನುಡಿದಿದ್ದಳು. ಆದ್ದರಿಂದ ಸ್ವಾಮೀಜಿ ಸ್ವಲ್ಪ ಹಾಸ್ಯವಾಗಿ ಅವಳಿಗೊಂದು ಪತ್ರ ಬರೆದರು: “ನಿನ್ನ ಆರನೇ ತಾರೀಕೇನೋ ಬಂದಿತು. ಆದರೆ ನನ್ನ ಅದೃಷ್ಟವೇನೂ ಖುಲಾಯಿಸಲಿಲ್ಲ!... ಇರಲಿ, ಕೆಲವರು ಕಷ್ಟವನ್ನು ಪ್ರೀತಿಸಲೆಂದೇ ಹುಟ್ಟಿರುತ್ತಾರೆ. (ನಾನು ಅಂಥವರಲ್ಲೊಬ್ಬ). ನಾನು ಯಾರ ನಡುವೆ ಹುಟ್ಟಿದೆನೋ ಆ ದೇಶಬಾಂಧವರಿಗಾಗಿ ನನ್ನ ರಕ್ತವನ್ನು ಬಸಿದಿರದಿದ್ದರೆ, ಬೇರೆ ಇನ್ಯಾರಿಗೋಸ್ಕರವಾದರೂ ಬಸಿಯುತ್ತಿದ್ದೆ. ಇದು ಖಂಡಿತ.”

ಸ್ವಾಮೀಜಿ ಕ್ಯಾಲಿಫೋರ್ನಿಯಕ್ಕೆ ಬರುವುದಕ್ಕೆ ಮೊದಲೇ ಜೋಸೆಫಿನ್ನಳು ಅಲ್ಲಿನ ತನ್ನ ಹಲ ವಾರು ಪರಿಚಿತರಿಗೆ ಸ್ವಾಮೀಜಿಯವರ ಬಗ್ಗೆ ತಿಳಿಸಿದ್ದಳು. ಅಲ್ಲದೆ ಆ ದೂರದ ರಾಜ್ಯದಲ್ಲೂ ಅವರ ಪುಸ್ತಕಗಳ ಹಾಗೂ ಭಾಷಣಗಳ ವರದಿಗಳ ಮೂಲಕ ಅವರ ಪರಿಚಯ ಮಾಡಿಕೊಂಡಿ ದ್ದವರು ಅಸಂಖ್ಯಾತ ಜನರಿದ್ದರು. ಇವರೆಲ್ಲ ಸ್ವಾಮೀಜಿಯವರ ಬರವನ್ನೇ ಕುತೂಹಲದಿಂದ ಎದುರು ನೋಡುತ್ತಿದ್ದರು. ಆದ್ದರಿಂದ ಸ್ವಾಮೀಜಿ ಇಲ್ಲಿಗೆ ಬರುತ್ತಿದ್ದಂತೆಯೇ ಈ ಜನರೆಲ್ಲ ಅವರ ದರ್ಶನಾರ್ಥಿಗಳಾಗಿ ಧಾವಿಸತೊಡಗಿದರು. ಹೀಗೆ ಬಂದವರಲ್ಲಿ ಬರ್ನ್​ಹಾರ್ಡ್ ಎಂಬವ ನೊಬ್ಬ. ಈತ ‘ಸದರ್ನ್ ಕ್ಯಾಲಿಫೋರ್ನಿಯ ಅಕಾಡೆಮಿ ಆಫ್ ಸೈನ್ಸಸ್​’ ಎಂಬ ಸಂಸ್ಥೆಯ ಕಾರ್ಯದರ್ಶಿ. ಇವನು ಬಹಳ ಮೇಧಾವಿ, ಹಲವಾರು ವಿಷಯಗಳಲ್ಲಿ ಪ್ರತಿಭಾವಂತ. ಈತ ಸ್ವಾಮೀಜಿಯವರ ಹಾಗೂ ಅವರ ಕಾರ್ಯೋದ್ದೇಶಗಳ ವಿಷಯದಲ್ಲಿ ತೀವ್ರ ಆಸಕ್ತಿ ತೋರಿದ. ಅಲ್ಲದೆ, ಸ್ವಾಮೀಜಿ ಕ್ಯಾಲಿಫೋರ್ನಿಯಕ್ಕೆ ಬರುವ ಮೊದಲೇ ಈತ ಲಾಸ್ ಏಂಜೆಲಿಸ್​ನಲ್ಲಿ ಅವರ ಪ್ರಥಮ ಸಾರ್ವಜನಿಕ ಉಪನ್ಯಾಸವನ್ನು ಏರ್ಪಡಿಸಿದ್ದ. ನಗರದಲ್ಲೇ ಅತ್ಯುತ್ತಮವಾದ ‘ಬ್ಲಾಂಕರ್ಡ್ ಹಾಲ್​’ ಎಂಬ ಸಭಾಂಗಣದಲ್ಲಿ ಡಿಸೆಂಬರ್ ೮ರ ಸಂಜೆ ಉಪನ್ಯಾಸವೆಂದು ಪ್ರಕಟ ವಾಗಿತ್ತು.

ಈ ವೇಳೆಗಾಗಲೇ ನಗರದ ಪತ್ರಿಕೆಗಳಲ್ಲಿ ವಿವೇಕಾನಂದರನ್ನು ಕುರಿತು ವರದಿಗಳು, ಲೇಖನ ಗಳು ಪ್ರಕಟವಾಗತೊಡಗಿದ್ದುವು. ಅವರು ಅಲ್ಲಿಗೆ ಬರುವ ಹಿಂದಿನ ದಿನವೇ ಅಲ್ಲಿನ ‘ಕ್ಯಾಪಿಟಲ್​’ ಎಂಬ ವಾರಪತ್ರಿಕೆ ಅವರ ಛಾಯಾಚಿತ್ರವನ್ನು ಮುದ್ರಿಸಿ, \eng{A Prince From India} (‘ಭಾರತದಿಂದ ಬಂದ ರಾಜ ಈತ!’) ಎಂಬ ಶೀರ್ಷಿಕೆಯಡಿಯಲ್ಲಿ ಸುದೀರ್ಘ ವರದಿಯೊಂದನ್ನು ಪ್ರಕಟಿಸಿತು. ಅದರ ಕೆಲವಂಶಗಳು ಹೀಗಿದ್ದುವು:

“ಕೆಲದಿನಗಳಲ್ಲೇ ಲಾಸ್ ಏಂಜೆಲಿಸ್ ನಗರವು ಸ್ವಾಮಿ ವಿವೇಕಾನಂದರ ಭೇಟಿಯ ಸೌಭಾಗ್ಯಕ್ಕೆ ಪಾತ್ರವಾಗಲಿದೆ. ಅವರು ಅಕ್ಷರಶಃ ಒಬ್ಬ ರಾಜನೇ ಸರಿ–ಬುದ್ಧಿಮತ್ತೆಯಲ್ಲಿ, ಜ್ಞಾನದಲ್ಲಿ ಹಾಗೂ ಅಧ್ಯಾತ್ಮವೆಂದು ನಾವು ಅಸ್ಪಷ್ಟವಾಗಿ ಕರೆಯುವ ರಾಜ್ಯದಲ್ಲಿ. ಈ ನಗರದ ಸ್ತ್ರೀಪುರುಷರಿಗೆ ಆ ಪೌರ್ವಾತ್ಯ ಪ್ರವಾದಿಯ ಅಪೂರ್ವ ಹೃದಯದ ಚಿಲುಮೆಯ ತೀರ್ಥವನ್ನು ಪಾನಮಾಡುವ ಸದವಕಾಶ ಲಭ್ಯವಾಗಲಿದೆ. ಖಂಡಿತವಾಗಿಯೂ ನಮಗೆಲ್ಲ ಅವರು ಹೊಸ ರಸಾನುಭವ ವೊಂದನ್ನು ಮಾಡಿಸಿ ಕೊಡುತ್ತಾರೆಂದು ನಿರೀಕ್ಷಿಸಬಹುದು.

“ವಿವೇಕಾನಂದರು ಶಿಕಾಗೋದ ಸರ್ವಧರ್ಮ ಸಮ್ಮೇಳನದಲ್ಲಿ ಮಾತನಾಡಿದವರು... ಪೌರ್ವಾತ್ಯ ರಾಷ್ಟ್ರಗಳಿಂದ ಬರುವ ಹಲವಾರು ಪಂಡಿತರಂತೆ ಅವರು ಪಾಶ್ಚಾತ್ಯ ಬುದ್ಧಿಜೀವಿ ಯೊಬ್ಬನಿಗೆ ಸಲಾಮು ಹೊಡೆಯಲು ಬರುತ್ತಿಲ್ಲ; ಬದಲಾಗಿ ಇಲ್ಲಿನ ಜನಗಳಲ್ಲಿ ತಮ್ಮ ಜನಾಂಗದ ಸನಾತನ ಜ್ಞಾನವನ್ನು ಪಸರಿಸಲು ಬರುತ್ತಿದ್ದಾರೆ...

“ಸರ್ವಧರ್ಮ ಸಮ್ಮೇಳನವಾಗಿ ಆಗಲೇ ಆರು ವರ್ಷಗಳು ಕಳೆದುಹೋಗಿವೆ. ಈ ಅವಧಿಯಲ್ಲಿ ಸ್ವಾಮಿಗಳ ವಿಚಿತ್ರ ವ್ಯಕ್ತಿತ್ವವು ಅನೇಕ ಬದಲಾವಣೆಗಳನ್ನು ಕಂಡಿರಬಹುದು. ಅಲ್ಲದೆ ಅವರು ಹಲವಾರು ಹೊಸ ದೇಶಗಳಿಗೆ ಭೇಟಿಯಿತ್ತಿದ್ದಾರೆ. ಈಗಲೂ ಅವರು ತಮ್ಮ ಆಲೋಚನೆಗಳಲ್ಲಿ ಹಿಂದಿನಷ್ಟೇ ಪೌರ್ವಾತ್ಯರಾಗಿದ್ದಾರೆಯೇ? ಅಥವಾ, ಸ್ವಲ್ಪ ‘ಸಂಮಿಶ್ರಣ’ಗೊಂಡಿದ್ದಾರೆಯೇ? ಈ ಪ್ರಚಂಡ ಬುದ್ಧಿಯು ಪಾಶ್ಚಾತ್ಯ ಪ್ರಪಂಚದಿಂದ ಏನನ್ನು ಪಡೆದುಕೊಂಡಿದೆಯೆಂಬುದನ್ನು ಗಮನಿಸಲು ಸ್ವಾರಸ್ಯವಾಗಿರುತ್ತದೆ... ”

ಬ್ಲಾಂಕರ್ಡ್ ಹಾಲಿನಲ್ಲಿ ಸ್ವಾಮೀಜಿ ಮಾಡಿದ ಮೊದಲ ಉಪನ್ಯಾಸ ‘ವೇದಾಂತ ತತ್ತ್ವ’. ಇದಕ್ಕೆ ೫ಂ ಸೆಂಟ್ (ಅರ್ಧ ಡಾಲರ್​) ಪ್ರವೇಶ ಶುಲ್ಕವಿದ್ದು ಸುಮಾರು ಆರುನೂರು ಸ್ತ್ರೀ ಪುರುಷರು ಸೇರಿದ್ದರು. ಅವರ ಮೊದಲ ಉಪನ್ಯಾಸವೇ ಭಾರೀ ಯಶಸ್ಸು ಗಳಿಸಿತು. ಸಭಿಕರೆಲ್ಲ ಮಂತ್ರಮುಗ್ಧರಂತೆ ಕುಳಿತು ಆಲಿಸಿದರು. ಹಾಗೂ ಸೂಜಿ ಬಿದ್ದರೂ ಕೇಳಿಸುವಂತಹ ನೀರವತೆ ತುಂಬಿತ್ತು ಎಂದು ವರದಿ ತಿಳಿಸುತ್ತದೆ. ಎಷ್ಟೇ ಅನಾರೋಗ್ಯಕ್ಕೆ ಬಲಿಯಾದರೂ ಸ್ವಾಮೀಜಿ ಯವರ ವಾಕ್ಸಾಮರ್ಥ್ಯವಾಗಲಿ ಆತ್ಮಶಕ್ತಿಯಾಗಲಿ ಕಿಂಚಿತ್ತಾದರೂ ಕುಂದಿರಲಿಲ್ಲವೆಂಬುದಕ್ಕೆ ಇದು ಪ್ರಥಮ ಕುರುಹು.

ಅಂದು ಅವರ ಭಾಷಣವನ್ನು ಮೈಮರೆತು ಕೇಳಿದವರಲ್ಲಿ ಮೂವರು ಸೋದರಿಯರಿದ್ದರು. ಇವರು ಮುಂದೆ ಅಮೆರಿಕದ ಪಶ್ಚಿಮ ಕರಾವಳಿಯಲ್ಲಿ ಅವರ ಧರ್ಮಪ್ರಸಾರ ಕಾರ್ಯದಲ್ಲಿ ಬಹು ಮುಖ್ಯ ಪಾತ್ರವನ್ನು ವಹಿಸಲಿದ್ದರು. ಇವರೇ ದಕ್ಷಿಣ ಪಸಾಡೆನದ ‘ಮೀಡ್ ಸೋದರಿ ಯರು’. ಇವರ ಹೆಸರುಗಳು ಕ್ರಮವಾಗಿ ಶ್ರೀಮತಿ ಕ್ಯಾರೀ ಮೀಡ್ ವೈಕಾಫ್, ಶ್ರೀಮತಿ ಆ್ಯಲಿಸ್ ಮೀಡ್ ಹ್ಯಾನ್ಸ್​ಬ್ರೋ ಮತ್ತು ಕುಮಾರಿ ಹೆಲೆನ್ ಮೀಡ್.

ಸ್ವಾಮೀಜಿಯವರು ಲಾಸ್​ಏಂಜಲಿಸ್​ನ ಯೂನಿಟಿ ಚರ್ಚಿನಲ್ಲಿ ದಕ್ಷಿಣ ಕ್ಯಾಲಿಫೋರ್ನಿಯದ ಅಕಾಡೆಮಿ ಆಫ್ ಸೈನ್ಸಸ್​ನ ಆಶ್ರಯದಲ್ಲಿ ಎರಡನೆಯ ಭಾಷಣ ಮಾಡಿದರು. ಎರಡು ಸಾವಿರಕ್ಕೂ ಹೆಚ್ಚು ಜನ ಕಿಕ್ಕಿರಿದು ಅವರ ಮಾತುಗಳನ್ನಾಲಿಸಿದರು. ಈ ಭಾಷಣವು ಜನರಲ್ಲಿ ಅತಿ ಹೆಚ್ಚಿನ ಆಸಕ್ತಿಯನ್ನು ಕೆರಳಿಸಿತು. ಈ ಕುರಿತಾಗಿ ಇನ್ನೂ ಕೇಳಬೇಕೆಂಬ ತವಕವನ್ನು ಜನ ವ್ಯಕ್ತಪಡಿಸಿದರು. ಈ ವೇಳೆಗೆ ಸ್ವಾಮೀಜಿಯವರ ದೇಹಾರೋಗ್ಯವೂ ಆಶ್ಚರ್ಯಕರವಾಗಿ ಸುಧಾರಿಸಿದ್ದರಿಂದ ಅವರು ಇನ್ನಷ್ಟು ಕೆಲಸಗಳನ್ನು ಕೈಗೆತ್ತಿಕೊಳ್ಳಲು ಉತ್ಸುಕರಾದರು.

ಈ ಸಂದರ್ಭದಲ್ಲೇ ಅವರು ಮಿಸ್ ಸ್ಪೆನ್ಸರಳ ಮನೆಯಿಂದ ಶ್ರೀಮತಿ ಬ್ಲಾಜೆಟ್ಟಳ ಮನೆಗೆ ಅತಿಥಿಯಾಗಿ ಆಗಮಿಸಿದರು. ಶ್ರೀಮತಿ ಬ್ಲಾಜೆಟ್ ವೃದ್ಧ ವಿಧವೆ. ಹಾಸ್ಯ ಪ್ರವೃತ್ತಿಯವಳು, ತುಂಬ ವಾತ್ಸಲ್ಯದ ಸ್ವಭಾವ. ಈಕೆ ಹಿಂದೆ ಶಿಕಾಗೋದಲ್ಲಿ ನಡೆದ ಧರ್ಮಸಮ್ಮೇಳನದಲ್ಲಿ ಭಾಗವಹಿಸಿ ದ್ದಳು, ಸ್ವಾಮೀಜಿಯವರ ಮೊತ್ತಮೊದಲ ಭಾಷಣವನ್ನು ಕೇಳಿದ್ದಳು. ಹೇಗೋ ಮಾಡಿ ಸ್ವಾಮೀಜಿ ಯವರ ದೊಡ್ಡದೊಂದು ವರ್ಣಚಿತ್ರವನ್ನು ಸಂಪಾದಿಸಿ ಮನೆಗೆ ತಂದು ತೂಗುಹಾಕಿದ್ದಳು. ಅವಳು ಆ ಚಿತ್ರವನ್ನು ಮನೆಗೆ ತರುವಾಗ, ಸ್ವತಃ ಆ ಮಹಾಪುರುಷನೇ ಒಂದು ದಿನ ತನ್ನ ಮನೆಗೆ ಬರಬಹುದು, ಬಂದು ತನ್ನ ಅತಿಥಿಯಾಗಿ ಇರಬಹುದು ಎಂದು ಕನಸುಮನಸ್ಸಿನಲ್ಲೂ ಎಣಿಸಿದವ ಳಲ್ಲ. ಅವಳು ಅಂದು ತಾಳಿದ ಭಕ್ತಿ ಇಂದು ಫಲ ಕೊಟ್ಟಿದೆ!

ಸ್ವಾಮೀಜಿ ತಮ್ಮ ಎರಡನೆಯ ಉಪನ್ಯಾಸವನ್ನು ಮಾಡಿದ ಮರುದಿನ ಇಬ್ಬರು ಮೀಡ್ ಸೋದರಿಯರು–ಶ್ರೀಮತಿ ಆ್ಯಲಿಸ್ ಮತ್ತು ಮಿಸ್ ಹೆಲೆನ್​–ಅವರ ದರ್ಶನಕ್ಕಾಗಿ ಬಂದರು. ‘ಕರ್ಮಯೋಗ’ ಮತ್ತು ‘ರಾಜಯೋಗ’ಗಳನ್ನು ಓದಿ ಇವರು ತೀವ್ರವಾಗಿ ಪ್ರಭಾವಿತರಾಗಿದ್ದರು. ಮತ್ತು ಆ ಗ್ರಂಥದ ಕರ್ತೃವನ್ನು ಎಂದಾದರೂ ನೋಡಲು ಸಾಧ್ಯವಾದೀತೆ ಎಂದು ಹಂಬಲಿಸಿ ದ್ದರು. ಅಂದು ಮಾತಿನ ಮಧ್ಯೆ ಸ್ವಾಮೀಜಿ ಈ ಸೋದರಿಯರಿಗೆ, “ನೀವೇನಾದರೂ ತರಗತಿಗಳನ್ನು ಏರ್ಪಡಿಸುವ ಹಾಗಿದ್ದರೆ ನಾನು ಬರಲು ಸಿದ್ಧ” ಎಂದರು. ಈ ಸೋದರಿಯರಿಗೆ ಅಷ್ಟೇ ಬೇಕಿತ್ತು. ಸ್ವಲ್ಪವೂ ಸಮಯ ನಷ್ಟ ಮಾಡದೆ ಬ್ಲಾಂಕರ್ಡ್ ಕಟ್ಟಡದ ಕೋಣೆಗಳಲ್ಲಿ ತರಗತಿಗಳನ್ನು ಏರ್ಪಡಿಸಿ ಆಸಕ್ತಿಯುಳ್ಳ ಜನರಿಗೆ ಆಮಂತ್ರಣ ನೀಡಿದರು. ಡಿಸೆಂಬರ್ ೧೯, ೨೧, ೨೨ರಂದು ಮೂರು ತರಗತಿಗಳು ಏರ್ಪಾಡಾದುವು. ಹೀಗೆ ಸ್ವಾಮೀಜಿಯವರಿಗೆ ಪೆಸಿಫಿಕ್ ಸಾಗರದ ತೀರಪ್ರದೇಶದಲ್ಲಿ ಧರ್ಮಪ್ರಸಾರಕ್ಕೆ ದಾರಿ ತೆರೆದುಕೊಂಡಿತು.

ಅವರು ತಮ್ಮ ಮೊದಲ ತರಗತಿಯಲ್ಲಿ ವಿವರಿಸಿದ ವಿಷಯ ‘ಬಳಕೆಯಲ್ಲಿ ಮನಶ್ಶಾಸ್ತ್ರ’. ಈ ತರಗತಿಯು ಎಷ್ಟು ಜನಪ್ರಿಯವಾಯಿತೆಂದರೆ ತರಗತಿಗಳನ್ನು ನಡೆಸಲು ಇನ್ನೂ ದೊಡ್ಡ ಸ್ಥಳ ವನ್ನು ಗೊತ್ತುಪಡಿಸಬೇಕಾಯಿತು. ಅವರು ‘ಹೋಂ ಆಫ್ ಟ್ರೂತ್​’ ಎಂಬ ಧಾರ್ಮಿಕ ಸಂಘ ದಲ್ಲಿ ತರಗತಿಗಳನ್ನು ಮುಂದುವರಿಸಿದರು.

ಈ ಮೊದಲ ಮೂರು ತರಗತಿಗಳು ಸಂಪೂರ್ಣ ಯಶಸ್ವಿಯಾದ್ದರಿಂದ ಇವು ತಾವಾಗಿಯೇ ಮುಂದುವರಿಯತೊಡಗಿದುವು. ಈ ಸಂಸ್ಥೆಯ ಹಲವಾರು ಸದಸ್ಯರು ಸ್ವಾಮೀಜಿಯವರ ಕಟ್ಟಾ ಅನುಯಾಯಿಗಳಾದರು. ಅವರ ವಿಚಾರಧಾರೆಯಿಂದ ಪ್ರಭಾವಿತರಾದ ಈ ಸಂಸ್ಥೆಯ ನಾಯಕರು ಅಲ್ಲಿನ ಪತ್ರಿಕೆಯೊಂದರಲ್ಲಿ ಬರೆಯುತ್ತಾರೆ, “... ಸ್ವಾಮಿ ವಿವೇಕಾನಂದರ ವ್ಯಕ್ತಿತ್ವದಲ್ಲಿ ವಿಶ್ವವಿದ್ಯಾಲಯದ ಕುಲಪತಿಗಳ ವಿದ್ಯೆ, ಆರ್ಚ್ ಬಿಷಪ್ಪರ ಗಾಂಭೀರ್ಯ ಮತ್ತುಅತ್ಯಂತ ಸಹಜ ಸ್ವಾತಂತ್ರ್ಯದಿಂದ ಓಡಾಡುವ ಸರಳ-ಮುಗ್ಧ ಶಿಶುವಿನ ಶೋಭೆ–ಇವು ಸೇರಿಕೊಂಡಿವೆ... ಇವರು ಒಂದು ನಿಮಿಷದ ಪೂರ್ವತಯಾರಿಯೂ ಇಲ್ಲದೆ ವೇದಿಕೆಯನ್ನೇರಿ ವಿಷಯದಾಳದಲ್ಲಿ ಮುಳುಗಿಬಿಡುತ್ತಾರೆ.”

ಸ್ವಾಮೀಜಿಯವರು ತಮ್ಮ ಸಂದೇಶಗಳನ್ನು ನೇರವಾಗಿ ನೀಡುತ್ತಿದ್ದರು. ಅವರು ಅದಕ್ಕೆ ಸಕ್ಕರೆಯ ಲೇಪವನ್ನು ಸವರುತ್ತಲೂ ಇರಲಿಲ್ಲ, ಈ ಜನರಿಗೆ ಅನುಕೂಲವಾಗಲೆಂಬ ದೃಷ್ಟಿಯಿಂದ ಅವುಗಳ ಸತ್ಯತೆಯನ್ನು ಮರೆಮಾಚುತ್ತಲೂ ಇರಲಿಲ್ಲ. ಅಲ್ಲದೆ, ಆಯ್ದ ಕೆಲವೇ ಮಂದಿ ಬುದ್ಧಿವಂತರಿಗೆ ಮಾತ್ರವೇ ಅರ್ಥವಾಗುವಂತಹ ಕಠಿಣ ಶೈಲಿಯನ್ನು ಬಳಸಿ ಬೆರಗುಗೊಳಿಸಲೂ ಇಲ್ಲ. ಸ್ವಾಮೀಜಿಯವರಿಗೆ ಪರಮ ಸತ್ಯದ–ಆತ್ಮವಿಚಾರದ–ಭಯನಿವಾರಕ ಶಕ್ತಿಯಲ್ಲಿ ಅಪಾರ ನಂಬಿಕೆ. ಆದ್ದರಿಂದ ಇಂತಹ ಮಹಾಸತ್ಯವನ್ನು ಮನುಷ್ಯನಿಂದ ಮುಚ್ಚಿಡಬಾರದು ಎಂಬುದು ಅವರ ವ್ರತ. ಆದ್ದರಿಂದಲೇ ಅವರು ಗರ್ಜಿಸುತ್ತಾರೆ, “ಮಾನವನಲ್ಲಿ ಸುಪ್ತವಾಗಿರುವ ಮಹಿಮೆಯನ್ನು ಮರೆಯದಿರಿ. ಅವನಿಗೆ ಅತ್ಯುತ್ತಮ ಹಾಗೂ ಶ್ರೇಷ್ಠತಮವಾದ ತತ್ತ್ವವನ್ನೇ ನೀಡಿ; ಅವನ ಮಹಿಮೆಯನ್ನು ಅವನಿಗೆ ತಿಳಿಸಿಕೊಡಿ. ಸುಮ್ಮನೆ ಅವನನ್ನು ಹೊಗಳಬೇಡಿ, ಸುಮ್ಮನೆ ಕಣ್ಣೊರೆಸುವ ಮಾತನಾಡಬೇಡಿ. ಮತ್ತು ಸತ್ಯವನ್ನು ತಿಳಿಸಿಕೊಡಲು ಹಿಂಜರಿಯಬೇಡಿ.” ಜೀವನದಲ್ಲಿ ಶಿವನನ್ನು ಕಾಣುವ ವಿಧಾನವನ್ನು ಅವರು ಬೋಧಿಸಿದುದು ಹೀಗೆ.

ಈ ದಿವ್ಯವಾಣಿಯನ್ನು ಕೇಳುತ್ತಿದ್ದವರಲ್ಲಿ ಯಾರು ಚರ್ಚುಗಳ ಸಿದ್ಧಾಂತಗಳಿಗೆ ಕಟ್ಟುಬಿದ್ದಿರ ಲಿಲ್ಲವೋ, ಯಾರ ಮನಸ್ಸು ಪೂರ್ವಗ್ರಹಪೀಡಿತವಾಗಿರಲಿಲ್ಲವೋ ಅವರೆಲ್ಲರ ಮನೋರಂಗ ದಲ್ಲಿ ಸ್ವಾಮೀಜಿಯವರ ನೂತನ ವಿಚಾರಧಾರೆಯು ಶಾಶ್ವತ ಜ್ಯೋತಿಯನ್ನು ಬೆಳಗಿತು, ಸ್ಫೂರ್ತಿ ಆನಂದಗಳನ್ನು ನೀಡಿತು. ಆದರೆ ಇನ್ನುಳಿದವರು ಮಾತ್ರ ಸ್ವಾಮೀಜಿಯವರ ಸಾಹಸಮಯ ಸಂದೇಶಗಳನ್ನು ಕೇಳಿ ಬೆಕ್ಕಸಬೆರಗಾಗದಿರಲಿಲ್ಲ. ಅವರು ನೀಡಿದ “ಅಧ್ಯಾತ್ಮದ ಅನುಷ್ಠಾನದ ವಿಷಯದಲ್ಲಿ ಕೆಲವು ಸಲಹೆಗಳು” ಎಂಬ ಉಪನ್ಯಾಸವನ್ನು ಕೇಳಿದ ಯಾರೇ ಆದರೂ ಒಂದಲ್ಲ ಒಂದು ರೀತಿಯಲ್ಲಿ ತೀವ್ರವಾಗಿ ಪ್ರಭಾವಿತರಾಗದಿರಲು ಸಾಧ್ಯವೇ ಇರಲಿಲ್ಲ. ಆ ಉಪನ್ಯಾಸದಲ್ಲಿ ಸ್ವಾಮೀಜಿ ನುಡಿದರು, “... ಮನುಷ್ಯನನ್ನು ನಾವು ಅತ್ಯಂತ ಉದಾರ ದೃಷ್ಟಿಯಿಂದ ನೋಡಬೇಕು. ನಿಜಕ್ಕೂ ಮನುಷ್ಯ ಒಳ್ಳೆಯವನಾಗಿರುವುದು ಅಷ್ಟೇನೂ ಸುಲಭದ ಮಾತಲ್ಲ. ನಿಜವಾದ ಅರ್ಥದಲ್ಲಿ ಸ್ವತಂತ್ರರಾಗುವವರೆಗೂ ನೀವು ಕೇವಲ ಯಂತ್ರಗಳಲ್ಲದೆ ಮತ್ತೇನು? ನೀವು ಒಳ್ಳೆಯವರಾಗಿರುವುದರ ಬಗ್ಗೆ ಜಂಬ ಪಟ್ಟುಕೊಳ್ಳುವುದೆ? ಖಂಡಿತ ಸಲ್ಲದು. ನೀವು ಒಳ್ಳೆಯವರಾಗಿರುವುದೇಕೆಂದರೆ ಬೇರೆ ದಾರಿಯಿಲ್ಲದಿರುವುದರಿಂದ! ಮತ್ತೊಬ್ಬನು ಕೆಟ್ಟವನಾ ಗಿರುವುದೇಕೆಂದರೆ ಹಾಗಿರದೆ ಅವನಿಗೂ ಬೇರೆ ಗತ್ಯಂತರವಿಲ್ಲ! ನೀವೇ ಅವನ ಸ್ಥಾನದಲ್ಲಿದ್ದರೆ ನೀವು ಏನಾಗಿರುತ್ತಿದ್ದಿರೋ ಯಾರಿಗೆ ಗೊತ್ತು? ವೇಶ್ಯಾವಾಟಿಕೆಯಲ್ಲಿರುವ ಒಬ್ಬಳು ವಾರಾಂಗನೆ ಯಾಗಲಿ, ಸೆರೆಮನೆಯಲ್ಲಿರುವ ಒಬ್ಬ ಕಳ್ಳನಾಗಲಿ, ಅವರೆಲ್ಲ ನೀವು ಒಳ್ಳೆಯವರಾಗಿರಲು ಸಾಧ್ಯವಾಗುವಂತೆ ದಿನದಿನವೂ ಬಲಿಯಾಗುತ್ತಿರುವ ಕ್ರಿಸ್ತರೇ! ಇದೇ ಸಮತೋಲನದ ನಿಯಮ. ಇಲ್ಲಿ ದುರ್ಜನರೂ ಕಳ್ಳರೂ ಕೊಲೆಗಾರರೂ ಅನ್ಯಾಯಗಾರರೂ ದುರ್ಬಲರೂ ನೀಚರೂ ಪಿಶಾಚಿಗಳೂ–ಇವರೆಲ್ಲರೂ ನನ್ನ ಪಾಲಿನ ಕ್ರಿಸ್ತರೇ! ಭಗವಾನ್ ಕ್ರಿಸ್ತನಂತೆಯೇ ಈ ಸೈತಾನ ರೂಪೀ ಕ್ರಿಸ್ತನಿಗೂ ನನ್ನ ಪೂಜೆ ಸಲ್ಲುತ್ತದೆ! ಇದೇ ನನ್ನ ಸಿದ್ಧಾಂತ, ಈ ಬಗ್ಗೆ ನಾನು ಇನ್ನೇನೂ ಮಾಡಲಾರೆ. ಸಜ್ಜನರ, ಸಂತರ ಪಾದಗಳಿಗೆ ನನ್ನ ಪ್ರಣಾಮಗಳು. ಅಂತೆಯೇ ದುಷ್ಟರ, ಸೈತಾನರ ಪಾದಗಳಿಗೂ ನನ್ನ ಪ್ರಣಾಮಗಳು. ಅವರೆಲ್ಲ ನನ್ನ ಗುರುಗಳು, ನನ್ನ ಆಧ್ಯಾತ್ಮಿಕ ಪಿತರು, ನನ್ನ ರಕ್ಷಕರು. ನಾನು ಒಬ್ಬನನ್ನು ನಿಂದಿಸಬಹುದು, ಆದರೆ ಅವನ ಪತನದಿಂದಲೇ ನಾನು ಪಾಠ ಕಲಿಯಬಹುದು. ನಾನು ಇನ್ನೊಬ್ಬನನ್ನು ಕೊಂಡಾಡಬಹುದು ಮತ್ತು ಅವನ ಸತ್ಕಾರ್ಯಗಳಿಂದ ಪ್ರಯೋಜನವನ್ನೂ ಪಡೆಯಬಹುದು. ಈ ಮಾತು ಈಗ ನಾನಿಲ್ಲಿ ನಿಂತಿರುವಷ್ಟೇ ಸತ್ಯ. ಹೊಸಲು ದಾಟಿದ ಹೆಣ್ಣೊಬ್ಬಳನ್ನು ನಾನು ತಿರಸ್ಕಾರದಿಂದ ನೋಡಬೇಕಂತೆ, ಏಕೆಂದರೆ ನಾನು ಹಾಗೆ ಮಾಡಬೇಕೆಂದು ಸಮಾಜ ಬಯಸುತ್ತದೆ. ಆದರೆ ಆ ವ್ಯಭಿಚಾರಿಣಿಯೇ ನಮ್ಮ ರಕ್ಷಕಿ. ಏಕೆಂದರೆ ನಮ್ಮ ಇತರ ಸ್ತ್ರೀಯರು ತಮ್ಮ ಪಾವಿತ್ರ್ಯವನ್ನು ಕಾಪಾಡಿಕೊಂಡಿರುವುದಕ್ಕೆ ಈಕೆ ಹೊಸಲು ದಾಟಿದ್ದೇ ಕಾರಣ. ಈ ಬಗ್ಗೆ ನೀವು ಆಲೋಚಿಸಿ ನೋಡಿ. ಓ ಸ್ತ್ರೀಪುರುಷರೇ, ನಿಮ್ಮೊಳಗೇ ಚಿಂತಿಸಿ ನೋಡಿ. ಇದೊಂದು ಸತ್ಯ; ನಗ್ನ ನಿರ್ಭೀತ ಸತ್ಯ. ನಾನೀ ಜಗತ್ತನ್ನು ಹೆಚ್ಚು ಹೆಚ್ಚಾಗಿ ಕಂಡಂತೆ, ಹೆಚ್ಚು ಹೆಚ್ಚು ಸ್ತ್ರೀಪುರುಷರನ್ನು ಕಂಡಂತೆ ಈ ನನ್ನ ನಂಬಿಕೆ ದೃಢವಾಗುತ್ತಿದೆ. ನಾನು ಯಾರನ್ನು ಹೊಗಳಲಿ, ಯಾರನ್ನು ತೆಗಳಲಿ? ನಾಣ್ಯದ ಇನ್ನೊಂದು ಮುಖವನ್ನೂ ನಾವು ನೋಡಬೇಕಲ್ಲವೆ?”

ಹೀಗೆ ಸ್ವಾಮೀಜಿಯವರು ಆಧ್ಯಾತ್ಮಿಕತೆಯ ಸ್ಪಷ್ಟ ಹಾಗೂ ಪುರೋಗಾಮಿ ಭಾವನೆಗಳನ್ನು ನೀಡಿದಾಗ ‘ಹೋಮ್ ಆಫ್ ಟ್ರೂತ್​’ನ ಹಲವು ಸದಸ್ಯರು ಅವರ ವಿಚಾರಧಾರೆಗೆ ವಶರಾದರು. ಆದರೆ ಇನ್ನುಳಿದ ಕೆಲವರಿಗೆ ಅವುಗಳನ್ನು ಜೀರ್ಣಿಸಿಕೊಳ್ಳಲು ಸಾಧ್ಯವಾಗಲಿಲ್ಲ. ಮೊದಲಿಗೆ ಸ್ವಾಮೀಜಿಯವರ ಬೋಧನೆಗಳನ್ನು ಬಹುವಾಗಿ ಮೆಚ್ಚಿಕೊಂಡಿದ್ದ ಬ್ರಾನ್ಸ್​ಬಿ ಎಂಬವನು ಕೂಡ ಅವರ ಈ ನೂತನ ವಿಚಾರಧಾರೆಯನ್ನು ತಾಳಲಾರದೆ ಒಮ್ಮೆ ಅವರನ್ನು ವ್ಯಂಗ್ಯವಾಗಿ ಕೇಳಿದ, “ಸ್ವಾಮೀಜಿ, ಎಲ್ಲರೂ ಒಂದೇ, ಎಲ್ಲವೂ ಒಂದೇ ಎನ್ನುವಿರಲ್ಲ, ಹಾಗಾದರೆ ಒಂದು ಎಲೆ ಕೋಸಿಗೂ ಮನುಷ್ಯನಿಗೂ ಏನು ವ್ಯತ್ಯಾಸ?” ಅದಕ್ಕೆ ಸ್ವಾಮೀಜಿಯವರ ಉತ್ತರ ಮರುಕ್ಷಣ ದಲ್ಲೇ ಚಿಮ್ಮಿತು, “ಒಂದು ಚಾಕುವನ್ನು ತೆಗೆದುಕೊಂಡು ನಿನ್ನ ಕಾಲಿಗೆ ಚುಚ್ಚಿ ನೋಡಿಕೊ. ಆಗ ನಿನಗೆ ವ್ಯತ್ಯಾಸ ಗೊತ್ತಾಗುತ್ತದೆ!”

ಶ್ರೀಮತಿ ಆ್ಯಲಿಸ್ ಹೇಳುತ್ತಾಳೆ, “ಸ್ವಾಮೀಜಿಯವರ ನೂತನ ವಿಚಾರಧಾರೆಯನ್ನು ಪ್ರತಿಭಟಿಸಿ ದವರು ಕ್ರೈಸ್ತ ಪಾದ್ರಿಗಳು ಮಾತ್ರವಲ್ಲ. ಅಲ್ಲಿನ ಇತರ ಹಲವಾರು ತತ್ತ್ವಬೋಧಕರು ಹಾಗೂ ಗುರುಗಳೆನ್ನಿಸಿಕೊಂಡವರು ಅವರನ್ನು ವಿರೋಧಿಸಿ ನಿಂದಿಸತೊಡಗಿದರು. ಏಕೆಂದರೆ, ಅವರೆಲ್ಲ ಅಲ್ಲಿಯವರೆಗೆ ಬೋಧಿಸುತ್ತಿದ್ದ ಆಧ್ಯಾತ್ಮಿಕತೆಗಳಿಂದ ಸ್ವಾಮೀಜಿಯವರ ವಿಚಾರಧಾರೆ ಅತ್ಯಂತ ಶ್ರೇಷ್ಠಮಟ್ಟದ್ದಾಗಿತ್ತು. ಅಲ್ಲದೆ, ಸ್ವಾಮೀಜಿ ಅವರೆಲ್ಲರ ಪೊಳ್ಳುತನವನ್ನು ಬಹಿರಂಗಪಡಿಸಿ ದ್ದರಿಂದ ಅವರ ‘ಆಧ್ಯಾತ್ಮಿಕ ವ್ಯಾಪಾರ’ಕ್ಕೆ ಕಲ್ಲು ಬೀಳುವಂತಾಗಿತ್ತು.

ಹೀಗೆ ಸ್ವಾಮೀಜಿಯವರನ್ನು ಒಂದೇ ಸಮನೆ ಟೀಕಿಸುತ್ತಿದ್ದವರಲ್ಲಿ ಬ್ರಾನ್ಸ್​ಬಿ ಒಬ್ಬ. ಏಕೆಂದರೆ ಅವನೂ ಒಬ್ಬ ತತ್ತ್ವಬೋಧಕನೆ ಆಗಿದ್ದ. ಅವನ ಒಂದು ಟೀಕೆಯೇನೆಂದರೆ ಸ್ವಾಮೀಜಿಯವರು ತಮ್ಮ ಸಂನ್ಯಾಸದ ನಿಯಮಕ್ಕೆ ವಿರುದ್ಧವಾಗಿ ಧನ ಸಂಗ್ರಹ ಮಾಡುತ್ತಿದ್ದಾರೆ ಎಂಬುದು. ಒಮ್ಮೆ ಶ್ರೀಮತಿ ಆ್ಯಲಿಸ್ ಹ್ಯಾನ್ಸ್​ಬ್ರೋ ಈ ವಿಷಯವನ್ನು ಅವರಿಗೆ ತಿಳಿಸಿದಳು. ಯಾವುದೋ ಶ್ಲೋಕವನ್ನು ಗುನುಗಿಕೊಳ್ಳುತ್ತಿದ್ದ ಸ್ವಾಮೀಜಿ, ತಕ್ಷಣ ಸುಮ್ಮನಾಗಿ ಮುಗು ಳ್ನಕ್ಕರು. ಬಳಿಕ ಹೇಳಿದರು, “ಹೌದು, ಅದು ನಿಜ. ಆದರೆ ನಿಯಮಗಳು ನನಗೆ ಹೊಂದಿಕೆಯಾಗ ದಿದ್ದರೆ, ನಾನವುಗಳನ್ನು ಬದಲಾಯಿಸುತ್ತೇನೆ.”

ಸ್ವಾಮೀಜಿಯವರು ನಿಜವಾದ ಆಧ್ಯಾತ್ಮಿಕತೆಯ ಕುರಿತಾಗಿ ಕೇವಲ ‘ಭಾಷಣ’ ಮಾಡುತ್ತಿರ ಲಿಲ್ಲ; ಅವರು ತಮ್ಮ ಶ್ರೋತೃಗಳ ಮೇಲೆ ಆಧ್ಯಾತ್ಮಿಕ ಪ್ರಭೆಯನ್ನೇ ಹರಿಸುತ್ತಿದ್ದರು. ಕಾಣುವ ಕಣ್ಣಿದ್ದವರಿಗೆ ಅವರು ಆಧ್ಯಾತ್ಮಿಕ ಜ್ಯೋತಿಯ ಮೂರ್ತಸ್ವರೂಪವೇ ಆಗಿದ್ದರು. ಕ್ರಿಸ್​ಮಸ್ ದಿನದಂದು ಸ್ವಾಮೀಜಿ ‘ಜಗತ್ತಿಗೆ ಕ್ರಿಸ್ತನ ಸಂದೇಶ’ ಎಂಬುದನ್ನು ಕುರಿತು ಮಾತನಾಡಿದರು. ಇದು ಆ ದಿನಗಳಲ್ಲಿ ಅವರು ನೀಡಿದ ಉಪನ್ಯಾಸಗಳಲ್ಲೆಲ್ಲ ಅತ್ಯಪೂರ್ವವಾದುದೆನ್ನಬಹುದು. ಆ ಉಪನ್ಯಾಸಕ್ಕೆ ಹಾಜರಿದ್ದ ಮಿಸ್ ಮೆಕ್​ಲಾಡ್ ಬರೆಯುತ್ತಾಳೆ, “ಅವರು ಅಂದು ಮಾತನಾಡು ತ್ತಿದ್ದಾಗ ಕ್ರಿಸ್ತನ ಅದ್ಭುತ ಲೀಲೆಯಲ್ಲಿ ಅದೆಷ್ಟು ತನ್ಮಯರಾಗಿದ್ದರೆಂದರೆ, ನಖಶಿಖಾಂತ ಅವರ ಶರೀರವು ಶ್ವೇತಜ್ಯೋತಿಯಿಂದ ಪ್ರಜ್ವಲಿಸುತ್ತಿತ್ತು.” ಅವರ ಸುತ್ತ ಹರಡಿಕೊಂಡಿದ್ದ ಆ ದಿವ್ಯ ಪ್ರಭೆಯಿಂದ ಜೋಸೆಫಿನ್ ಎಷ್ಟು ಪ್ರಭಾವಿತಳಾಗಿದ್ದಳೆಂದರೆ, ಅವರೊಂದಿಗೆ ಮನೆಗೆ ಹಿಂದಿರು ಗುವಾಗಲೂ ಆಕೆ ಮೌನವಾಗಿಯೇ ಇದ್ದಳು. ಸ್ವಾಮೀಜಿಯವರು ಅದೇ ಭಾವದಲ್ಲಿರಬಹುದು, ತಾನದಕ್ಕೆ ಭಂಗ ತರಬಾರದು ಎಂದು ಎಚ್ಚರದಿಂದ ಅವರ ಹಿಂದೆ ಸದ್ದಿಲ್ಲದೆ ನಡೆದುಬರುತ್ತಿದ್ದಳು.

ಸ್ವಲ್ಪಹೊತ್ತು ಮೌನವಾಗಿಯೇ ಸಾಗಿದ ಸ್ವಾಮೀಜಿ ಇದ್ದಕ್ಕಿದ್ದಂತೆ ನಿಂತು ಉದ್ಗರಿಸಿದರು, “ಅದನ್ನು ಹೇಗೆ ಮಾಡುತ್ತಾರೆಂದು ನನಗೆ ಗೊತ್ತು!”

ಜೋಸೆಫಿನ್ ತಕ್ಷಣ ಕೇಳಿದಳು, “ಅದನ್ನು ಎಂದರೆ ಯಾವುದನ್ನು, ಸ್ವಾಮೀಜಿ?”

“ಮಿಳುಹು ತಣ್ಣಿಯನ್ನು (ಇಂಗ್ಲಿಷಿನಲ್ಲಿ \eng{Mulligatawny soup–}ಮೆಣಸಿನ ಸಾರು) ಅದಕ್ಕೆ ಆ ಘಮಲು ಬರಲು ಅವರು ‘ಬೇ’ ಎಲೆಯನ್ನು ಕಡೆಯಲ್ಲಿ ಹಾಕುತ್ತಾರೆ–ಮೊದಲೇ ಅಲ್ಲ!”

ಸ್ವಾಮೀಜಿಯವರು ಯಾವುದೋ ಗಹನ ವಿಚಾರದ ಬಗ್ಗೆ ಹೇಳಬಹುದೆಂದು ನಿರೀಕ್ಷಿಸಿದ್ದ ಜೋಸೆಫಿನ್ ತಬ್ಬಿಬ್ಬು! “ಇದು ಅವರ ನಿರಹಂಭಾವ ಸ್ಥಿತಿ. ಅವರಿಗೆ ತಾವೊಬ್ಬ ದೊಡ್ಡ ಗಣ್ಯ ವ್ಯಕ್ತಿ ಎಂಬ ಭಾವನೆಯೇ ಇರಲಿಲ್ಲ. ಇದು ಅವರ ಹಲವಾರು ಶ್ರೇಷ್ಠ ಗುಣಗಳಲ್ಲಿ ಎದ್ದುಕಾಣು ವಂತಿದ್ದ ಒಂದು ಅಂಶ” ಎಂದಾಕೆ ಬರೆಯುತ್ತಾಳೆ.

ಯೋಗಮಾರ್ಗದಲ್ಲಿ ಸಿದ್ಧಿಯನ್ನು ತಲುಪಿ ಪರಮಹಂಸಾವಸ್ಥೆಗೇರಿದ ಮಹಾತ್ಮರು ಬಾಲ ವತ್, ಎಂದರೆ ಮಕ್ಕಳಂತೆ ಇರುತ್ತಾರೆ ಎಂಬುದು ಶಾಸ್ತ್ರವಾಕ್ಯ. ಸ್ವಾಮೀಜಿಯವರು ಅತ್ಯುನ್ನತ ಆಧ್ಯಾತ್ಮಿಕ ಪ್ರಜ್ಞೆಯಲ್ಲಿದ್ದಾಗಲೂ ಮಿಳುಹು ತಣ್ಣಿಯಂತಹ ಪ್ರಾಪಂಚಿಕ ವಿಷಯದ ಬಗ್ಗೆ ಆಲೋಚಿಸಬಲ್ಲವರಾಗಿದ್ದರು. ಸಕಲವನ್ನೂ ಭಗವನ್​ಮಯವಾಗಿ ಕಾಣುವವರಿಗೆ ಯಾವ ವಸ್ತು ತಾನೆ ಭಗವಚ್ಚಿಂತನೆಗೆ ಅಡ್ಡಿಯುಂಟುಮಾಡಬಲ್ಲುದು?

ಲಾಸ್ ಏಂಜಲಿಸ್​ನಲ್ಲಿ ಸ್ವಾಮೀಜಿಯವರ ಆತಿಥೇಯಳಾಗಿದ್ದ ಶ್ರೀಮತಿ ಬ್ಲಾಜೆಟ್ಟಳಿಗೆ ಅವರನ್ನು ಅವರ ವರ್ತನೆಗಳನ್ನು ಎಲ್ಲ ಕಾಲಗಳಲ್ಲೂ ಕಾಣುವ ಅವಕಾಶ ಲಭಿಸಿತ್ತು. ಮುಂದೆ ಸ್ವಾಮೀಜಿ ನಿರ್ಯಾಣಹೊಂದಿದ ಮೇಲೆ ಮಿಸ್ ಮೆಕ್​ಲಾಡಳಿಗೆ ಬರೆದ ಪತ್ರವೊಂದರಲ್ಲಿ ಅವರ ಕುರಿತಾದ ತನ್ನ ಮಧುರ ಸ್ಮೃತಿಗಳನ್ನು ಹೀಗೆ ತಿಳಿಸುತ್ತಾಳೆ:

“... ಅವರನ್ನು ನಾನು ವೈಯಕ್ತಿಕವಾಗಿ ಕಂಡದ್ದು ಎಲ್ಲೋ ಸ್ವಲ್ಪ ಸಮಯ ಮಾತ್ರ. ಆದರೆ ಆ ಅಲ್ಪಾವಧಿಯಲ್ಲೇ ನಾನು ಅವರ ವ್ಯಕ್ತಿತ್ವದ ಶಿಶುಸಹಜ ಅಂಶವನ್ನು ನೂರಾರು ವಿಧದಲ್ಲಿ ಕಂಡೆ. ಅವರ ಈ ಶಿಶುಭಾವವು ಸ್ತ್ರೀಯರ ಸಹಜ ಮಾತೃಭಾವಕ್ಕೆ ಒಂದು ನಿರಂತರ ಪ್ರಚೋದನೆ ಯಾಗಿತ್ತು. ಸ್ವಾಮೀಜಿ ತಮ್ಮ ಬಳಿಯಿದ್ದವರ ಮೇಲೆ ಹೇಗೆ ಅವಲಂಬಿತರಾಗುತ್ತಿದ್ದರೆಂದರೆ ಅವರುಗಳ ಹೃದಯಕ್ಕೆ ಅತಿ ಸಮೀಪದವರಾಗುತ್ತಿದ್ದರು. ಕೆಲವೊಮ್ಮೆ ಉಪನ್ಯಾಸ ಮುಗಿದ ಬಳಿಕ ತಮ್ಮನ್ನು ಮುತ್ತಿಕೊಳ್ಳುತ್ತಿದ್ದ ಶ್ರೋತೃಗಳಿಂದ ಬಲವಂತವಾಗಿ ಬಿಡಿಸಿಕೊಂಡು, ತಾವಿಳಿದು ಕೊಂಡಿದ್ದವರ ಮನೆಗೆ ಓಡಿ ಬಂದು, ಶಾಲೆ ಬಿಟ್ಟೊಡನೆಯೇ ಓಡಿಬರುವ ಹುಡುಗನಂತೆ ಅಡಿಗೆಮನೆಯೊಳಗೆ ನುಗ್ಗಿ ಹೇಳುತ್ತಿದ್ದರು, ‘ಈಗ ಸ್ವಲ್ಪ ಏನಾದರೂ ತಿಂಡಿ ಮಾಡೋಣ!’ ಎಂದು. ಆಗ ಅವರಲ್ಲಿನ ಪ್ರವಾದಿಯೂ ಪುಷಿಯೂ ಮಾಯವಾಗಿ, ಶಿಶುಸಹಜ ಸರಳ ವ್ಯಕ್ತಿತ್ವ ಪ್ರಕಟವಾಗುತ್ತಿತ್ತು.”

ಲಾಸ್​ಏಂಜಲಿಸ್​ನ ‘ಹೋಮ್ ಆಫ್ ಟ್ರೂತ್​’ನಲ್ಲಿಸ್ವಾಮೀಜಿ ನೀಡುತ್ತಿದ್ದ ಪ್ರವಚನಗಳು ತುಂಬ ಜನಪ್ರಿಯವಾದುವು. ಆದರೆ ಅವರ ಕ್ರಾಂತಿಕಾರೀ ಬೋಧನೆಗಳನ್ನು ‘ಹೋಮ್​’ನ ಹಲವರು ಮೆಚ್ಚಿಕೊಂಡರೂ ಇನ್ನು ಕೆಲವರಿಗೆ ಅವು ಸಹ್ಯವಾಗಲಿಲ್ಲ. ಆದ್ದರಿಂದ ಆ ಸಂಸ್ಥೆಯ ಆಶ್ರಯದಲ್ಲಿ ಮಾತನಾಡಲು ಸ್ವಾಮೀಜಿ ಇಷ್ಟಪಡಲಿಲ್ಲ. ಅಲ್ಲಿನ ಸದಸ್ಯರ ದಾಕ್ಷಿಣ್ಯಕ್ಕೆ ಕಟ್ಟು ಬಿದ್ದು, ಅವರಿಗೆ ಹಿತವಾಗುವಂತೆ ಮಾತನಾಡುವುದು ಸ್ವಾಮೀಜಿಯವರಿಗೆಂತು ಸಾಧ್ಯ? ಆದ್ದ ರಿಂದ ಅವರು ೧೯ಂಂರ ಪ್ರಾರಂಭದಲ್ಲಿ ತಮ್ಮ ತರಗತಿಗಳ ಸ್ಥಳವನ್ನು ಪೇನ್ಸ್ ಹಾಲ್ ಎಂಬಲ್ಲಿಗೆ ಬದಲಾಯಿಸಿದರು. ಇಲ್ಲಿ ಅವರು ನಡೆಸಿದ ಹಲವಾರು ತರಗತಿಗಳಲ್ಲಿ ಎರಡರ ವಿಷಯಗಳೆಂದರೆ ‘ಯಶಸ್ಸಿನ ರಹಸ್ಯ’ ಮತ್ತು ‘ನಾವು ನಾವೇ’. ‘ಯಶಸ್ಸಿನ ರಹಸ್ಯ’ ಎಂಬುದು ಕರ್ಮ ಯೋಗವನ್ನು ಕುರಿತದ್ದಾಗಿತ್ತು, ಇದರಲ್ಲಿ ನಿರ್ಲಿಪ್ತತೆಯ ಅಂಶದ ಬಗ್ಗೆ ಸ್ವಾಮೀಜಿ ಹೇಳುತ್ತಾರೆ:

“ನಮ್ಮ ಮನಸ್ಸನ್ನು ನಾವು ಇಚ್ಛಾಮಾತ್ರದಿಂದಲೇ ಯಾವುದೇ ವಸ್ತುವಿನಿಂದ ಹಿಂದೆಗೆದು ಕೊಳ್ಳಬಲ್ಲೆವಾದರೆ ಆಗ ಯಾವ ಕಷ್ಟವೂ ಇರುವುದೇ ಇಲ್ಲ. ಯಾವ ಮನುಷ್ಯನು ಒಂದು ವಸ್ತು ವಿಗೆ ಬಲವಾಗಿ ಅಂಟಿಕೊಂಡಿದ್ದರೂ, ಸಮಯವೊದಗಿದಾಗ ತಕ್ಷಣ ಅದರಿಂದ ದೂರವಾಗಬಲ್ಲ ಸಾಮರ್ಥ್ಯ ಹೊಂದಿರುತ್ತಾನೋ ಅವನೇ ಈ ಪ್ರಕೃತಿಯಿಂದ ಅತಿ ಹೆಚ್ಚಿನ ಲಾಭ ಪಡೆಯುತ್ತಾನೆ. ನಿಜಕ್ಕೂ ಇದು ತುಂಬ ಕಷ್ಟದ ಮಾತೇ ಸರಿ... ಗೋಡೆಗೆ ಎಂದೂ ಕಷ್ಟವಾಗುವುದಿಲ್ಲ, ಗೋಡೆ ಯೆಂದೂ ಪ್ರೀತಿಸುವುದಿಲ್ಲ, ಗೋಡೆಗೆ ದುಃಖವೂ ಆಗುವುದಿಲ್ಲ. ಆದರೆ ಅದು ಎಂದೆಂದೂ ಗೋಡೆಯೇ. ಇಂತಹ ಗೋಡೆಯಾಗಿರುವುದಕ್ಕಿಂತ, ವ್ಯಾಮೋಹ ಬೆಳೆಸಿಕೊಂಡು ಅದರಲ್ಲಿ ಸಿಕ್ಕಿಹಾಕಿಕೊಳ್ಳುವುದೇ ಮೇಲು! ಆದ್ದರಿಂದ ಯಾವನು ಎಂದೂ ಯಾವುದನ್ನೂ ಪ್ರೀತಿಸುವು ದಿಲ್ಲವೋ, ಕಲ್ಲಿನಂತೆ ಬರಡಾಗಿರುತ್ತಾನೋ ಅವನು ಜೀವನದ ಆನಂದಗಳಿಂದಲೂ ವಂಚಿತ ನಾಗುತ್ತಾನೆ. ನಮಗದು ಬೇಕಿಲ್ಲ. ಅದು ದುರ್ಬಲತೆ, ಅದು ಮರಣ.”

‘ನಾವು ನಾವೇ!’ ಎಂಬುದು ಅತ್ಯುನ್ನತ ಸಾಕ್ಷಾತ್ಕಾರ ಮಾಡಿಕೊಳ್ಳುವಂತೆ ಪ್ರೇರೇಪಿಸುವ ಒಂದು ಸ್ಫೂರ್ತಿಯುತ ಭಾಷಣವಾಗಿತ್ತು. ಇದೇ ಅವಧಿಯಲ್ಲಿ ಅವರು ಬ್ಲಾಂಕರ್ಡ್ ಸಭಾಂಗಣ ದಲ್ಲಿ ಎರಡು ದಿನ ‘ಭಾರತ ಮತ್ತು ಭಾರತೀಯರು’ ಹಾಗೂ ‘ಭಾರತದ ಇತಿಹಾಸ’ ಎಂಬ ವಿಷಯಗಳ ಬಗ್ಗೆ ಮಾತನಾಡಿದರು. ಜನವರಿ ೭ರ ಭಾನುವಾರದಂದು ಅವರು ಪೇನ್ಸ್ ಹಾಲಿನಲ್ಲಿ ಮಾಡಿದ ‘ದೇವದೂತ ಏಸುಕ್ರಿಸ್ತ’ ಎಂಬ ಭಾಷಣವು ಅತ್ಯುತ್ಕೃಷ್ಟವಾದುದು. ಕ್ರಿಸ್ತನ ಕುರಿತಾಗಿ ಅವರು ಮಾಡಿದ ಭಾಷಣಗಳಲ್ಲೆಲ್ಲ ಪ್ರಕಟವಾಗಿರುವುದು ಇದೊಂದೇ. ಏಸುಕ್ರಿಸ್ತನ ಜೀವನ- ವ್ಯಕ್ತಿತ್ವಗಳಲ್ಲಿ ಪ್ರಕಟವಾಗುವ ಕ್ರೈಸ್ತಧರ್ಮದ ಅತ್ಯುನ್ನತ ಆದರ್ಶಗಳ ಬಗ್ಗೆ ಸ್ವಾಮೀಜಿ ಎಷ್ಟು ಪೂಜ್ಯಭಾವ ಹೊಂದಿದ್ದರು ಎಂಬುದು ಇದರಲ್ಲಿ ಸುವ್ಯಕ್ತವಾಗಿದೆ. ಅದರಲ್ಲಿ ಅವರು ಘೋಷಿಸಿ ದರು, “ಹೊಸ ಒಡಂಬಡಿಕೆಯು (ಬೈಬಲಿನ ನ್ಯೂ ಟೆಸ್ಟಮೆಂಟ್​) ಕ್ರಿಸ್ತನ ಜನನವಾದ ಐನೂರು ವರ್ಷಗಳೊಳಗಾಗಿ ಬರೆಯಲ್ಪಟ್ಟಿತೇ ಇಲ್ಲವೇ ಎಂಬುದು ಮುಖ್ಯವಲ್ಲ. ಅಥವಾ ಅದರಲ್ಲಿ ಬರೆಯಲ್ಪಟ್ಟಿರುವ ಅವನ ಜೀವನ ಕಥೆಯು ಸಂಪೂರ್ಣ ಸತ್ಯವೇ ಎಂಬುದೂ ಕೂಡ ಮುಖ್ಯ ವಿಚಾರವಲ್ಲ. ಆದರೆ ಅದರ ಹಿನ್ನೆಲೆಯಲ್ಲಿ ಏನೋ ಒಂದು ಇದೆ; ನಾವು ಅನುಕರಿಸಬೇಕಾದ್ದು ಏನೋ ಇದೆ... ಅಲ್ಲಿ ಪ್ರಚಂಡ ಆಧ್ಯಾತ್ಮಿಕ ಶಕ್ತಿಯೊಂದು ಪ್ರಕಟಗೊಂಡು ಕೇಂದ್ರಬಿಂದುವಿ ನಂತೆ ಇದ್ದಿರಲೇಬೇಕು. ಅದು ಮಾತ್ರವೇ ಇಂದಿಗೂ ಉಳಿದುಕೊಂಡಿದೆ. ನಾವಿಂದು ಆಸಕ್ತರಾಗಿ ರುವುದು ಅದರಲ್ಲಿ.”

ಜನವರಿ ೮ರಂದು ಸ್ವಾಮೀಜಿಯವರು ‘ಮನಸ್ಸಿನ ಶಕ್ತಿಗಳು’ ಎಂಬ ಭಾಷಣದೊಂದಿಗೆ ಲಾಸ್ ಏಂಜೆಲಿಸ್ ನಗರದಲ್ಲಿ ತಮ್ಮ ಉಪನ್ಯಾಸ-ತರಗತಿಗಳನ್ನು ಮುಕ್ತಾಯಗೊಳಿಸಿದರು. ಈ ವೇಳೆಗೆ ಅವರಿಗೆ ಸಮೀಪದ ದಕ್ಷಿಣ ಪಸಾಡೆನ ನಗರಕ್ಕೆ ಬಂದು ಉಪನ್ಯಾಸಗಳನ್ನು ನೀಡುವಂತೆ ಆಹ್ವಾನ ಬಂದಿತ್ತು. ಅಲ್ಲಿ ಅವರ ಆತಿಥೇಯರಾಗಲಿದ್ದವರು ಮೂವರು ಮೀಡ್ ಸೋದರಿ ಯರು. ಇವರು ಲಾಸ್ ಏಂಜೆಲಿಸ್​ನಲ್ಲಿ ಸ್ವಾಮೀಜಿಯವರ ಪ್ರತಿಯೊಂದು ಕಾರ್ಯಕ್ರಮದಲ್ಲೂ ಭಾಗವಹಿಸಿದ್ದರು; ಅವರಿಗೆ ಹಲವು ವಿಧದಲ್ಲಿ ನೆರವಾಗಿ ಅವರ ಆಪ್ತ ಶಿಷ್ಯೆಯರಾಗಿದ್ದರು. ದಕ್ಷಿಣ ಪಸಾಡೆನಕ್ಕೆ ತೆರಳಿದ ಸ್ವಾಮೀಜಿ, ಜನವರಿ ೧೫ರಿಂದ ಅಲ್ಲಿ ತಮ್ಮ ಕಾರ್ಯಕ್ರಮಗಳನ್ನು ಪ್ರಾರಂಭಿಸಿದರು.

ಕ್ಯಾಲಿಫೋರ್ನಿಯದಲ್ಲಿ ಈವರೆಗಿನ ವಾಸ ಸ್ವಾಮೀಜಿಯವರಿಗೆ ತುಂಬ ಅನುಕೂಲಕರವಾ ಗಿತ್ತು. ಅವರ ದೇಹಾರೋಗ್ಯದ ಸುಧಾರಣೆಯೊಂದಿಗೆ, ಅವರ ಇನ್ನೆರಡು ಮುಖ್ಯೋದ್ದೇಶಗಳಾದ ವೇದಾಂತ ಪ್ರಸಾರ ಹಾಗೂ ಧನ ಸಂಗ್ರಹಣೆಯೂ ತುಂಬ ಯಶಸ್ವಿಯಾಗಿ ಸಾಗಿತ್ತು. ಅದರಲ್ಲೂ ಧನಸಂಗ್ರಹಣೆಯಂತೂ ಅವರ ಪಾಲಿಗೆ ಆಗ ಅತಿ ಮುಖ್ಯವಾಗಿ ಪರಿಣಮಿಸಿತ್ತು. ಏಕೆಂದರೆ ಬೇಲೂರು ಮಠದ ಹೊಸ ನಿವೇಶನದ ಮೇಲೆ ಅಲ್ಲಿನ ಪುರಸಭೆ ಭಾರೀ ಮೊತ್ತದ ತೆರಿಗೆ ಹಾಕಿದ್ದು, ಮಠ ತೀವ್ರ ಆರ್ಥಿಕ ಒತ್ತಡಕ್ಕೆ ತುತ್ತಾಗಿತ್ತು. ರಾಮಕೃಷ್ಣ ಮಠದ ನಿರ್ವಹಣೆಯನ್ನು ಟ್ರಸ್ಟ್ ಒಂದಕ್ಕೆ ಒಪ್ಪಿಸಿರಲಿಲ್ಲವಾದ್ದರಿಂದ ಆ ನಿವೇಶನವನ್ನೆಲ್ಲ ಒಬ್ಬರ ವೈಯಕ್ತಿಕ ಆಸ್ತಿ ಎಂದು ಪರಿಗಣಿಸಿ ಆ ತೆರಿಗೆ ಹಾಕಲಾಗಿತ್ತು. ಇದಕ್ಕೆ ವಿರುದ್ಧವಾಗಿ ರಾಮಕೃಷ್ಣ ಮಠ ಮನವಿ ಸಲ್ಲಿಸಿ ದಾಗಲೂ ಪುರಸಭೆ ಅದನ್ನು ತಳ್ಳಿಹಾಕಿತು. ಪುರಸಭೆ ಹಾಕಿದಷ್ಟು ತೆರಿಗೆ ನೀಡಿದರೆ ಮಠ ಉಳಿಯುವಂತೆಯೇ ಇರಲಿಲ್ಲ! ಆದ್ದರಿಂದ ಪುರಸಭೆಯ ಆಜ್ಞೆಯ ವಿರುದ್ಧ ನ್ಯಾಯಾಲಯದಲ್ಲಿ ಅರ್ಜಿ ಸಲ್ಲಿಸಲಾಯಿತು. ಆದರೆ ಆ ಸಂಬಂಧವಾಗಿ ವಕೀಲರ ಖರ್ಚು ಹಾಗೂ ಕೋರ್ಟು ಫೀಸಿಗೇ ಸಾವಿರಾರು ರೂಪಾಯಿ ಬೇಕಾಯಿತು. ಇದರಿಂದಾಗಿ ಆಶ್ರಮವಾಸಿಗಳಿಗೆ ಮತ್ತೆ ಅನ್ನಾಹಾರಗಳಿಗೇ ಕುತ್ತು ಬರುವ ಪರಿಸ್ಥಿತಿ ಒದಗಿತು. ಈ ವಿಷಯವನ್ನು ತಿಳಿದು ಸ್ವಾಮೀಜಿ ತುಂಬ ಆತಂಕ ಗೊಂಡರು. ಆದರೆ ಅದೃಷ್ಟವಶಾತ್ ಉಪನ್ಯಾಸಗಳಿಂದ ಸಾಕಷ್ಟು ಹಣ ಸಂಗ್ರಹವಾಗುತ್ತಿತ್ತು. ಆಗ ತಮ್ಮ ಕೈಯಲ್ಲಿದ್ದ ಸುಮಾರು ೨ಂಂಂ ರೂಪಾಯಿಯನ್ನು ಅವರು ತಕ್ಷಣ ಮಠಕ್ಕೆ ಕಳಿಸಿ ಕೊಟ್ಟರು. ಇದು ಕೂಡ ಸಾಲದೆ ಬಂದಿತು. ಆಗ ಮಿಸ್ ಮೆಕ್​ಲಾಡ್ ಹಾಗೂ ಅವಳ ತಂಗಿ ಬೆಸ್ಸಿ ಧಾರಾಳವಾಗಿ ನೆರವು ನೀಡಿದರು. ಮೆಕ್​ಲಾಡಳೊಬ್ಬಳೇ ತಾನು ಕೂಡಿಟ್ಟುಕೊಂಡಿದ್ದ ೮ಂಂ ಡಾಲರುಗಳನ್ನು (ಆಗ ಸುಮಾರು ೨೫ಂಂ ರೂಪಾಯಿ) ಸ್ವಾಮೀಜಿಯವರಿಗೆ ಕೊಟ್ಟುಬಿಟ್ಟಳು. ಹೀಗೆ ಸಮಯಕ್ಕೆ ಸರಿಯಾಗಿ ನೆರವು ಒದಗಿಬಂದದ್ದರಿಂದ ಮಠ ದೊಡ್ಡದೊಂದು ತೊಂದರೆ ಯಿಂದ ಪಾರಾಯಿತು; ಸ್ವಾಮೀಜಿಯವರ ಮೇಲಿದ್ದ ಭಾರವೂ ದೂರವಾಯಿತು.

