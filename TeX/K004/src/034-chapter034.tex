
\chapter{“ಮೇಲಿನಿಂದ ಕರೆ ಬಂದಿದೆ”}

\noindent

ಮಳೆಗಾಳಿಗಳಿಂದ ಕೂಡಿದ ಸಮುದ್ರ ಪ್ರಯಾಣವನ್ನು ಮುಗಿಸಿ ಆಗಸ್ಟ್ ಮೂರರಂದು ಸ್ವಾಮೀಜಿ ಪ್ಯಾರಿಸ್ ತಲುಪಿದರು. ಇದು ಫ್ರಾನ್ಸಿನ ರಾಜಧಾನಿಯಷ್ಟೇ ಅಲ್ಲ; ಸ್ವಾಮೀಜಿಯವರೇ ಹೇಳುವಂತೆ ಇದು ‘ಆಧುನಿಕ ನಾಗರಿಕತೆಯ ರಾಜಧಾನಿ; ಭೂಮಿಯ ಮೇಲಿನ ಸುಖಭೋಗ ಆಮೋದ ಪ್ರಮೋದಗಳ ಸ್ವರ್ಗ; ಕಲೆ ವಿಜ್ಞಾನಗಳ ಕೇಂದ್ರಬಿಂದು.’ ಈ ಮಾತು ಅಂದಿಗೂ ನಿಜ, ಇಂದಿಗೂ ನಿಜ. ಪ್ಯಾರಿಸಿನಲ್ಲಿ ಸ್ವಾಮೀಜಿಯವರು ಮೊದಲು ಗೆರಾಲ್ಡ್ ನೋಬೆಲ್ ಎಂಬ ಗಣ್ಯ ವ್ಯಕ್ತಿಯ ಅತಿಥಿಯಾಗಿ ಇಳಿದುಕೊಂಡರು. ಇವರು ಸ್ವಾಮೀಜಿಯವರನ್ನು ಫ್ರಾನ್ಸಿಗೆ ಆಹ್ವಾನಿಸಿದ್ದ ಲೆಗೆಟ್ ದಂಪತಿಗಳ ಸ್ನೇಹಿತರು; ಅತ್ಯಂತ ಸುಸಂಸ್ಕೃತ, ಬುದ್ಧಿವಂತ, ಸ್ನೇಹಮಯ ವ್ಯಕ್ತಿ. ಹಿಂದೆಯೇ ಸ್ವಾಮೀಜಿಯವರಿಗೆ ಇವರ ಪರಿಚಯವಾಗಿತ್ತು. ಪ್ಯಾರಿಸಿನ ಧರ್ಮ ಚರಿತ ಮೇಳದಲ್ಲಿ ಸ್ವಾಮೀಜಿಯವರು ಭಾಗವಹಿಸುವಂತೆ ಏರ್ಪಾಡು ಮಾಡಿದ್ದವರು ಇವರೇ. ಇವರ ಸ್ನೇಹಪರತೆಯನ್ನು ಮೆಚ್ಚಿಕೊಂಡು ಸ್ವಾಮೀಜಿ ಒಮ್ಮೆ ಉದ್ಗರಿಸಿದರು, “ಶ್ರೀ ನೋಬೆಲ್ಲರಂತಹ ಒಬ್ಬ ಸ್ನೇಹಿತರನ್ನು ಪಡೆದುಕೊಂಡರೆ ನಾವು ಹುಟ್ಟಿದ್ದು ಸಾರ್ಥಕವಾಯಿತು ಎಂದುಕೊಳ್ಳ ಬಹುದು.”

ಕೆಲವು ದಿನಗಳ ಮೇಲೆ ಸ್ವಾಮೀಜಿಯವರು ಲೆಗೆಟ್ ದಂಪತಿಗಳ ಮನೆಗೆ ತೆರಳಿದರು. ಈ ದಂಪತಿಗಳು ಪ್ಯಾರಿಸಿನ ಉನ್ನತ ವರ್ಗಕ್ಕೆ ಸೇರಿದ್ದ ಸುಸಂಸ್ಕೃತರು. ಇವರು ತಮ್ಮ ಮನೆಯಲ್ಲಿ ಆಗಾಗ ಅದ್ಧೂರಿಯ ‘ಪಾರ್ಟಿ’ (ಚಹಾಕೂಟ)ಗಳನ್ನು ಏರ್ಪಡಿಸುತ್ತಿದ್ದರು. ಎಲ್ಲ ಕ್ಷೇತ್ರಗಳಲ್ಲೂ ಅತ್ಯುಚ್ಚ ಸ್ಥಾನವನ್ನಲಂಕರಿಸಿದ್ದ ಸ್ತ್ರೀಪುರುಷರು ಇದರಲ್ಲಿ ಭಾಗವಹಿಸುತ್ತಿದ್ದರು. ವಿಜ್ಞಾನಿಗಳು, ಕವಿಗಳು, ತತ್ವವಾದಿಗಳು, ಬರಹಗಾರರು, ನಾಟಕಕಾರರು, ನಟನಟಿಯರು, ವಾಸ್ತುಶಿಲ್ಪಿಗಳು, ಚಿತ್ರಕಾರರು –ಹೀಗೆ ಸಮಾಜದ ಅತ್ಯುನ್ನತ ವರ್ಗದ ವ್ಯಕ್ತಿಗಳು ಒಟ್ಟಾಗಿ ಸೇರುತ್ತಿದ್ದರು. ಈ ಚಹಾಕೂಟಗಳ ಮೂಲಕ ಸ್ವಾಮೀಜಿಯವರಿಗೆ ಪಶ್ಚಿಮದ ಪ್ರಮುಖ ಚಿಂತಕರೊಡನೆ ಬೆರೆತು ತಮ್ಮ ಅಭಿಪ್ರಾಯಗಳನ್ನು ವಿನಿಮಯ ಮಾಡಿಕೊಳ್ಳುವ ಅವಕಾಶ ಲಭ್ಯವಾಯಿತು. ಇಂತಹ ಒಂದು ಕೂಟದಲ್ಲಿ ಆಗಸ್ಟ್ ೨೪ರಂದು ಅವರು “ಹಿಂದೂ ಧರ್ಮ ಹಾಗೂ ತತ್ವ” ಎಂಬ ವಿಷಯವಾಗಿ ಫ್ರೆಂಚ್ ಭಾಷೆಯಲ್ಲಿ ಮಾತನಾಡಿದರೆಂದು ತಿಳಿದುಬಂದಿದೆ.

‘ಪರಿವ್ರಾಜಕ’ ಎಂಬ ತಮ್ಮ ಪುಸ್ತಕದಲ್ಲಿ (ಅವರು ‘ಉದ್ಬೋಧನ’ ಪತ್ರಿಕೆಗೆ ಕಳಿಸಿಕೊಟ್ಟ ಪ್ರವಾಸ ಲೇಖನಗಳ ಸಂಕಲನ) ಸ್ವಾಮೀಜಿಯವರು ಈ ಒಕ್ಕೂಟಗಳ ವರ್ಣಮಯ ಚಿತ್ರವನ್ನು ನಿರೂಪಿಸಿದ್ದಾರೆ–“ಪರ್ವತದ ಝರಿಯಂತೆ ತಿಳಿಯಾದ-ನಿರ್ಮಲವಾದ ಮಾತುಗಳ ನಿರಂತರ ಧಾರೆ, ಎಲ್ಲೆಡೆಗಳಿಂದಲೂ ಹೊಮ್ಮುತ್ತಿರುವ ಭಾವಪೂರ್ಣ ಮಾತುಗಳು, ಒಂದು ಕಡೆಯಿಂದ ಮಂಕು ಕವಿಸುವ ಸಂಗೀತ, ಅದ್ವಿತೀಯ ಮೆದುಳುಗಳಿಂದ ಹೊಮ್ಮುತ್ತ ಒಂದನ್ನೊಂದು ಸಂಘರ್ಷಿಸುತ್ತಿರುವ ಚಿಂತನಲಹರಿಗಳು–ಇವು ಎಲ್ಲರನ್ನೂ ಮಂತ್ರಮುಗ್ಧವಾಗಿಸಿ ದೇಶಕಾಲ ಗಳನ್ನೇ ಮರೆಯುವಂತೆ ಮಾಡುತ್ತಿದ್ದುವು.”

ಪ್ಯಾರಿಸಿನಲ್ಲಿ ಸ್ವಾಮೀಜಿ ಭೇಟಿಯಾದ ಪ್ರಮುಖರಲ್ಲಿ ಕೆಲವರೆಂದರೆ ಸುಪ್ರಸಿದ್ಧ ಸಮಾಜ ವಿಜ್ಞಾನಿಯಾದ ಪ್ರೊ ॥ ಪ್ಯಾಟ್ರಿಕ್ ಗಿಡ್ಡಿಸ್, ಖ್ಯಾತ ನಟಿ ಸಾರಾ ಬರ್ನ್​ಹಾರ್ಟ್, ಹೆಸರಾಂತ ಗಾಯಕಿ ಎಮ್ಮಾ ಥರ್ಸ್​ಬಿ, ಮುಂದೆ ಸ್ವಾಮೀಜಿಯವರ ಸ್ಮೃತಿ ಲೇಖನ ಬರೆದ ಮಹಾ ಗಾಯಕಿ ಹಾಗೂ ನಟಿ ಎಮ್ಮಾ ಕಾಲ್ವೆ, ಅಮೆರಿಕದ ಸಮಾಜ ಸೇವಕಿ ಜೇನ್ ಆ್ಯಡಮ್ಸ್, ಶಿಕಾಗೊ ಸಮಾಜದ ‘ರಾಣಿ’ ಎಂದು ಪರಿಗಣಿಸಲ್ಪಟ್ಟಿದ್ದ ಶ್ರೀಮತಿ ಪಾಟರ್ ಪಾಮರ್, ನ್ಯೂಕ್ಯಾಸಲ್​ನ ರಾಜಕುವರಿ ಡೆಮಿಡಾಫ್, ಪ್ರಸಿದ್ಧ ಕ್ರೈಸ್ತ ಪಾದರಿಯಾದ ಪಿಯರ್ ಹಯಸಿಂತ್ ಹಾಗೂ ಪ್ರಸಿದ್ಧ ಅನ್ವೇಷಕ- ಲೇಖಕ ಸರ್ ಹಿರಮ್ ಮ್ಯಾಕ್ಸಿಮ್. ಇವರುಗಳಲ್ಲದೆ ಪ್ಯಾರಿಸ್ ಮೇಳದ ವಿಜ್ಞಾನ ಸಮ್ಮೇಳನ ದಲ್ಲಿ ಪಾಲ್ಗೊಳ್ಳಲು ಬಂದಿದ್ದ ಬಂಗಾಳದ ಪ್ರಸಿದ್ಧ ಸಸ್ಯವಿಜ್ಞಾನಿ ಸರ್ ಜಗದೀಶ್​ಚಂದ್ರಬೋಸ ರನ್ನು ಸ್ವಾಮೀಜಿ ಆಗಾಗ ಭೇಟಿಯಾಗುತ್ತಿದ್ದರು. ಸಸ್ಯಗಳ ಸೂಕ್ಷ್ಮ ಪ್ರತಿಕ್ರಿಯೆಗಳ ಬಗ್ಗೆ ಹಲವಾರು ಸಂಶೋಧನೆಗಳನ್ನು ಮಾಡಿದ್ದ ಬೋಸರ ಬಗ್ಗೆ ಸ್ವಾಮೀಜಿ ತುಂಬ ಅಭಿಮಾನ ಹೊಂದಿದ್ದರು. ಅವರ ಬಗ್ಗೆ ಇತರರ ಮುಂದೆಯೂ ಹೇಳುತ್ತಿದ್ದರು. ಒಮ್ಮೆ ಹಲವಾರು ಪ್ರಮುಖ ವ್ಯಕ್ತಿಗಳು ಒಟ್ಟಾಗಿದ್ದಾಗ ಓರ್ವ ಇಂಗ್ಲಿಷ್ ವಿಜ್ಞಾನಿಯ ವಿದ್ಯಾರ್ಥಿನಿಯೊಬ್ಬಳು, ಬೆಳವಣಿಗೆ ಕುಂಠಿತಗೊಂಡ ಲಿಲ್ಲಿ ಸಸ್ಯದ ಬೆಳವಣಿಗೆಯ ಬಗ್ಗೆ ತನ್ನ ಪ್ರಾಧ್ಯಾಪಕರು ಪ್ರಯೋಗ ನಡೆಸುತ್ತಿದ್ದಾರೆಂದು ಹೇಳಿದಳು. ಆಗ ಸ್ವಾಮೀಜಿ ಹಾಸ್ಯವಾಗಿ ನುಡಿದರು, “ಅದೇನು ಮಹಾ, ಆ ಲಿಲ್ಲಿ ಸಸ್ಯ ಬೆಳೆಯುವ ಕುಂಡವನ್ನೇ ಪ್ರತಿಸ್ಪಂದಿಸುವಂತೆ ಮಾಡಬಲ್ಲರು ನಮ್ಮ ಬೋಸ್!”

ಪ್ಯಾರಿಸ್ ಸಮ್ಮೇಳನದಲ್ಲಿ ಸ್ವಾಮೀಜಿಯವರು ಫ್ರೆಂಚ್ ಭಾಷೆಯಲ್ಲೇ ಮಾತನಾಡಲು ಸಿದ್ಧರಾಗುತ್ತಿದ್ದರು. ಬಹಳ ವರ್ಷಗಳಿಂದಲೂ ಅವರಿಗೆ ಆ ಭಾಷೆಯ ಪರಿಚಯ ಇದ್ದೇ ಇತ್ತು. ಅವರು ಪರಿವ್ರಾಜಕ ಸಂನ್ಯಾಸಿಯಾಗಿದ್ದಾಗಲೇ ಜೈಪುರದಲ್ಲಿ ಪಂಡಿತ ಶಂಕರ ಪಾಂಡುರಂಗ ಎಂಬವರಿಂದ ಫ್ರೆಂಚ್ ಭಾಷೆಯ ಅಭ್ಯಾಸವನ್ನು ಪ್ರಾರಂಭಿಸಿದ್ದನ್ನು ನೋಡಿದ್ದೇವೆ. ಈಗ ಒಂದು ವರ್ಷದಿಂದ, ಅದರಲ್ಲೂ ಅಮೆರಿಕೆಗೆ ಬಂದಾಗಿನಿಂದ, ಫ್ರೆಂಚಿನ ಅಭ್ಯಾಸವನ್ನು ತೀವ್ರಗೊಳಿಸಿ ದ್ದರು. ಕ್ಯಾಲಿಫೋರ್ನಿಯದಿಂದ ಮೇರಿಗೆ ಬರೆದ ಒಂದು ಪತ್ರದಲ್ಲಿ, “... ನಾನು ಫ್ರೆಂಚ್ ಶಬ್ದಕೋಶವನ್ನೇ ಬಾಯಿಪಾಠ ಮಾಡುತ್ತಿದ್ದೇನೆ” ಎಂದು ತಿಳಿಸಿದ್ದರು. ಈಗ ಪ್ಯಾರಿಸಿನಲ್ಲಿ ಅವರಿಗೆ ಫ್ರೆಂಚಿನಲ್ಲೇ ಮಾತನಾಡಿ ಆಡುಭಾಷೆಯ ಮೇಲಿನ ಜ್ಞಾನವನ್ನು ಹೆಚ್ಚಿಸಿಕೊಳ್ಳುವ ಅವಕಾಶ ಸಿಕ್ಕಿತ್ತು. ಬಹುಶಃ ಈ ಕಾರಣಕ್ಕಾಗಿಯೇ ಅವರು ಲೆಗೆಟ್ಟರ ಮನೆಯಲ್ಲಿ ನಡೆಯುತ್ತಿದ್ದ ಒಕ್ಕೂಟಗಳಲ್ಲಿ ಬೆರೆಯುತ್ತಿದ್ದರೆಂದೂ ಊಹಿಸಬಹುದು. ಅಲ್ಲದೆ ಅವರು ಈ ಒಕ್ಕೂಟಗಳಿಂದ ದೂರ ಉಳಿಯಬೇಕಾದ ಕಾರಣವೂ ಇರಲಿಲ್ಲ. ಮಾನವನ ಹಿರಿಮೆಯು ಯಾವುದೇ ರೂಪದಲ್ಲಿ ವ್ಯಕ್ತವಾದರೂ ಅದನ್ನು ಗುರುತಿಸುವಲ್ಲಿ ಸ್ವಾಮೀಜಿ ಎಂದೂ ಹಿಂದೆ ಬೀಳುತ್ತಿರಲಿಲ್ಲ. ಅತ್ಯಂತ ಸುಸಂಸ್ಕೃತ ಪ್ಯಾರಿಸ್ ನಾಗರಿಕರಿಗೆ ಸಮನಾಗಿ ಸಂಭಾಷಿಸಬಲ್ಲವರು ಅವರು. ಅತ್ಯಂತ ಚುರುಕಾ ದವರಿಗೂ ಚುರುಕು ಮುಟ್ಟಿಸಬಲ್ಲ ಚಾತುರ್ಯ ಅವರದು. ಸರ್ವಶ್ರೇಷ್ಠರಾದ ಬುದ್ಧಿಜೀವಿ ಗಳನ್ನೂ ವಾದದಲ್ಲಿ ಸೋಲಿಸಬಲ್ಲವರು ಅವರು. ಹೀಗಾಗಿ ದಿನದಿನಕ್ಕೂ ಅವರ ಫ್ರೆಂಚ್ ಭಾಷಾಜ್ಞಾನ ಹೆಚ್ಚುತ್ತ ಬಂದಿತು. ಇಷ್ಟೆಲ್ಲ ಆದರೂ, ಬಹುಶಃ ಸಮ್ಮೇಳನದ ಅತ್ಯಂತ ಶ್ರೇಷ್ಠ ವ್ಯಕ್ತಿಗಳ ಮುಂದೆ ತಮ್ಮ ಘನತೆಗೆ ತಕ್ಕಂತೆ ಮಾತನಾಡುವುದಕ್ಕೆ ಬೇಕಾದ ನಿರರ್ಗಳತೆಯನ್ನು ಗಳಿಸಿಕೊಳ್ಳುವ ಉದ್ದೇಶದಿಂದ ಅವರು ತಮ್ಮ ವಾಸಸ್ಥಳವನ್ನು ಬದಲಾಯಿಸಿ ಜೂಲ್ಸ್ ಬ್ಯೂಸ್ ಎಂಬ ಫ್ರೆಂಚ್ ತರುಣನ ಮನೆಗೆ ತೆರಳಿದರು. ಇವನಿಗೆ ಇಂಗ್ಲಿಷ್ ಬರುತ್ತಿರಲಿಲ್ಲ. ಆದ್ದರಿಂದ ಏನೇ ಆದರೂ ಫ್ರೆಂಚಿನಲ್ಲೇ ಮಾತನಾಡಬೇಕಲ್ಲ! ಹೀಗಾಗಿ ಕೆಲದಿನಗಳಲ್ಲೇ ತಮ್ಮ ಫ್ರೆಂಚ್ ಭಾಷೆ ಬಹಳಷ್ಟು ಸುಧಾರಿಸಬಹುದೆಂಬ ಭರವಸೆ ಸ್ವಾಮೀಜಿಯವರಿಗಿತ್ತು.

ಈ ದಿನಗಳಲ್ಲಿ ಅವರು ಫ್ರೆಂಚ್ ಸಂಸ್ಕೃತಿಯನ್ನು ಸೂಕ್ಷ್ಮವಾಗಿ ಗಮನಿಸುತ್ತ ಅದರ ಲಕ್ಷಣಗಳನ್ನೆಲ್ಲ ಗುರುತಿಸುತ್ತಿದ್ದರು. ಪ್ಯಾರಿಸಿನಲ್ಲಿ ಪೂರ್ಣ ರಭಸದಿಂದ ನಡೆಯುತ್ತಿದ್ದ ಜಾಗತಿಕ ಪ್ರದರ್ಶನಕ್ಕೆ ಪ್ರತಿದಿನ ಭೇಟಿ ನೀಡುತ್ತಿದ್ದರು. ಈ ಪ್ರದರ್ಶನದ ಅಂಗವಾದ ಧರ್ಮಚರಿತ ಮೇಳವು ಸೆಪ್ಟೆಂಬರ್ ಮೂರರಿಂದ ಒಂಬತ್ತರವರೆಗೆ ನಡೆಯಲಿತ್ತು. ಆದರೆ ಇದು ಅಂಥಾ ದೊಡ್ಡ ಸಮಾರಂಭವೇನೂ ಆಗಿರಲಿಲ್ಲ. ಅಲ್ಲದೆ ಶಿಕಾಗೋದಲ್ಲಿ ನಡೆದಿದ್ದಂತೆ ಎಲ್ಲ ಬಗೆಯ ಜನರೂ ಆಸಕ್ತಿಯಿಂದ ಪಾಲ್ಗೊಳ್ಳುವಂಥದೂ ಆಗಿರಲಿಲ್ಲ. ಏಕೆಂದರೆ ಇಲ್ಲಿ ಧರ್ಮಗಳ ಮೂಲಭೂತ ತತ್ತ್ವಗಳ, ಆಚಾರ ವಿಚಾರಗಳ ಕುರಿತಾಗಿ ಚರ್ಚಿಸಲು ಅವಕಾಶವಿರಲಿಲ್ಲ. ಇಲ್ಲಿ ಪ್ರಸ್ತಾಪಿಸಲ್ಪಟ್ಟದ್ದು ಕೇವಲ ಪಾಂಡಿತ್ಯದ ವಿಚಾರಗಳು ಮಾತ್ರ. ಅದರಲ್ಲೂ ವಿವಿಧ ಧರ್ಮಗಳ ಹಾಗೂ ಸಂಪ್ರದಾಯಗಳ ಉಗಮ ಮತ್ತು ಬೆಳವಣಿಗೆಯ ವಿಷಯದಲ್ಲಿ ಪರಿಣತರು ಮಾತ್ರ ಮಾತನಾಡುವ ಏರ್ಪಾಡಾಗಿತ್ತು. ಆದರೆ, ಮೊದಲಿಗೆ ಇಲ್ಲಿ ಮತ್ತೊಂದು ವಿಶ್ವಧರ್ಮ ಸಮ್ಮೇಳನ ವನ್ನೇ ನಡೆಸುವ ಯೋಜನೆ ಇತ್ತೆಂದೂ ರೋಮನ್ ಕ್ಯಾಥೋಲಿಕರ ತೀವ್ರ ವಿರೋಧದಿಂದಾಗಿ ಅದನ್ನು ರದ್ದುಗೊಳಿಸಬೇಕಾಯಿತೆಂದೂ ಪ್ರತಿಯೊಬ್ಬರೂ ಗುಟ್ಟಾಗಿ ಮಾತನಾಡಿಕೊಳ್ಳುತ್ತಿದ್ದರು! ಏಕೆಂದರೆ ಒಮ್ಮೆ ಅಮೆರಿಕದಲ್ಲಿ ಆಗಿದ್ದ ಅಪಮಾನದಿಂದಲೇ ಅವರಿನ್ನೂ ಚೇತರಿಸಿಕೊಂಡಿರ ಲಿಲ್ಲ. ಕ್ರೈಸ್ತಧರ್ಮವೇ ಸರ್ವಶ್ರೇಷ್ಠವಾದ ಹಾಗೂ ವಿಶ್ವಧರ್ಮವೆನಿಸಬಲ್ಲ ಏಕೈಕ ಧರ್ಮವೆಂದು ಮಂಡಿಸಿ ಜಗತ್ಪ್ರಸಿದ್ಧವನ್ನಾಗಿ ಮಾಡಬೇಕೆಂದು ಶಿಕಾಗೋದ ಆ ಸಮ್ಮೇಳನವನ್ನು ಭಾರೀ ಕಷ್ಟಪಟ್ಟು ಸಂಯೋಜಿಸಿದ್ದರೆ, ಮೂಢ ನಂಬಿಕೆಗಳ ನಾಡಾದ ಭಾರತದಿಂದ ಯಾವನೋ ಸಂನ್ಯಾಸಿಯೊಬ್ಬ ಬಂದು, ಹಿಂದೂ ಧರ್ಮವೇ ಸರ್ವಶ್ರೇಷ್ಠವೆಂದು ಜಗದ್ವಿಖ್ಯಾತಗೊಳಿಸಿಬಿಡ ಬೇಕೆ? ಆದ್ದರಿಂದ ಇನ್ನೊಮ್ಮೆ ಅಂಥದೇ ಸಮ್ಮೇಳನ ಮಾಡಿ ‘ಅಂಗೈ ತೋರಿಸಿ ಅವಲಕ್ಷಣ ಎನ್ನಿಸಿಕೊಂಡರು’ ಎಂದಾಗುವುದು ಬೇಡ ಎಂದು ತೀರ್ಮಾನಿಸಿದ್ದರು. ಅಲ್ಲದೆ ಶಿಕಾಗೋ ಸಮ್ಮೇಳನದ ಬಳಿಕ ಕ್ರಿಶ್ಚಿಯನ್ ಮಿಷನರಿಗಳಿಗೆ ಬರುತ್ತಿದ್ದ ಆದಾಯದಲ್ಲಿ ಲಕ್ಷಗಟ್ಟಲೆ ಡಾಲರಿನಷ್ಟು ಇಳಿತಾಯವಾಗಿತ್ತು. (ಸ್ವಾಮೀಜಿಯವರ ಆತಿಥೇಯರಾದ ಫ್ರಾನ್ಸಿಸ್ ಲೆಗೆಟ್ ಕೂಡ ವರ್ಷಕ್ಕೆ ಹತ್ತು ಸಾವಿರ ಡಾಲರ್ ಕೊಡುತ್ತಿದ್ದುದನ್ನು ನಿಲ್ಲಿಸಿಬಿಟ್ಟಿದ್ದರಂತೆ. ಸ್ವಾಮೀಜಿ ಈ ವಿಷಯವಾಗಿ ಶ್ರೀಮತಿ ಹ್ಯಾನ್ಸ್​ಬ್ರೋಗೆ ನಗುತ್ತ ಹೇಳುತ್ತಾರೆ, “ಆದರೆ ಅವರು ಆ ಹಣವನ್ನು ನನಗೇನೂ ಕೊಡಲಿಲ್ಲ!” ಎಂದು.) ಈ ಪ್ಯಾರಿಸ್ ಸಮ್ಮೇಳನವನ್ನು ಅವರೊಂದು ಪತ್ರದಲ್ಲಿ ತಮಾಷೆಗೆ “ಹುಚ್ಚಪ್ಪಗಳ ಸಮ್ಮೇಳನ” ಎಂದು ಉಲ್ಲೇಖಿಸಿದ್ದಾರೆ.

ಸ್ವಾಮೀಜಿಯವರು ಈ ಸಮ್ಮೇಳನದ ಹಲವಾರು ಸಭೆಗಳಲ್ಲಿ ಭಾಗವಹಿಸಿದರು. ಆದರೆ ಅನಾರೋಗ್ಯದ ಕಾರಣ ಎರಡು ಸಭೆಗಳಲ್ಲಿ ಮಾತ್ರ ಮಾತನಾಡಿದರು. ಪ್ರಕೃತಿ ಪೂಜೆಯೇ ಹಿಂದೂಧರ್ಮದ ಮೂಲ ಎಂಬುದರ ಸತ್ಯಾಸತ್ಯತೆಯನ್ನು ಪಾಶ್ಚಾತ್ಯ ವಿದ್ವಾಂಸರೊಂದಿಗೆ ಚರ್ಚಿಸುವಂತೆ ಅವರಿಗೆ ಸಮ್ಮೇಳನಾಧ್ಯಕ್ಷರು ಸೂಚಿಸಿದರು.

ಸ್ವಾಮೀಜಿಯವರು ವಿಶ್ವಧರ್ಮಚರಿತ ಮೇಳದ ಸಭೆಗಳಲ್ಲಿ ಭಾರತದ ಧಾರ್ಮಿಕ ಚಿಂತನೆಗಳ ಬೆಳವಣಿಗೆಯ ಬಗ್ಗೆ ಮಾತನಾಡಿದರು. ಹಿಂದೂ ಧರ್ಮದ ಎಲ್ಲ ಹಂತಗಳಿಗೂ ವೇದಗಳೇ ಮೂಲ, ಆಧಾರ ಎಂದು ಹೇಳಿದ ಅವರು, ಬೌದ್ಧ ಧರ್ಮವೂ ಸೇರಿದಂತೆ ಭಾರತದ ಪ್ರತಿಯೊಂದು ಧಾರ್ಮಿಕ ನಂಬಿಕೆಯ ಆಧಾರವೂ ವೇದಗಳೇ ಎಂದು ಪ್ರತಿಪಾದಿಸಿದರು.

ಬುದ್ಧನಿಗಿಂತ ಶ್ರೀಕೃಷ್ಣನು ಹೇಗೆ ಶ್ರೇಷ್ಠನಾದವನು ಎಂಬುದನ್ನು ವಿಶ್ಲೇಷಿಸಿ ಶ್ರೀಕೃಷ್ಣನ ಪೂಜೆಯು ಬುದ್ಧನ ಕಾಲಕ್ಕಿಂತ ಪುರಾತನವಾದುದು ಎಂದು ವಿವರಿಸಿದರು. ಅಲ್ಲದೆ, ‘ಭಗವ ದ್ಗೀತೆಯು ಮಹಾಭಾರತದ ಅಂಗವಲ್ಲ, ಅದನ್ನು ಬೇರೆಡೆಯಿಂದ ತಂದು ಮಹಾಭಾರತದೊಳಕ್ಕೆ ತುರುಕಲಾಗಿದೆ’ ಎಂಬ ಕೆಲವರ ವಾದವನ್ನು ತಮ್ಮ ತರ್ಕಸರಣಿಯಿಂದ ಖಂಡಿಸಿದರು.

ಈ ಸಂದರ್ಭದಲ್ಲಿ ಅವರು ಭಾರತದ ಬಗ್ಗೆ ಪಾಶ್ಚಾತ್ಯ ಸಂಶೋಧಕರ ತೀರ್ಮಾನಗಳ ಪೊಳ್ಳುತನವನ್ನು ಬಯಲಿಗೆಳೆದರು. ಪುರಾತನ ಭಾರತದ ವಿಜ್ಞಾನ, ಕಲೆ, ಗಣಿತ ಎಲ್ಲವೂ ಗ್ರೀಕ್ ಸಂಸ್ಕೃತಿಯ ಪ್ರತಿಫಲನವಷ್ಟೆ ಎಂಬ ತರ್ಕವನ್ನು ಮುರಿದುಹಾಕಿದರು. ಗ್ರೀಕ್ ಭಾಷೆಯ ಒಂದೊಂದು ಪದವನ್ನು ಕುರಿತೂ ಆಳವಾದ ಸಂಶೋಧನೆ ಮಾಡುವ ಪಾಶ್ಚಾತ್ಯ ವಿಜ್ಞಾನಿ ಗಳಿಗೆ,ಸಂಸ್ಕೃತ ಭಾಷೆಯನ್ನು ಕುರಿತೂ ಅಷ್ಟೇ ಪರಿಶ್ರಮ ವಹಿಸಿ ಸಂಶೋಧನೆ ಮಾಡುವಂತೆ ಕರೆ ಕೊಟ್ಟರು.

ಹೀಗೆ, ಅಜ್ಞಾನದಿಂದಲೂ ಉದ್ದೇಶಪೂರ್ವಕವಾಗಿಯೂ ಭಾರತದ ಭವ್ಯ ಸಂಸ್ಕೃತಿಯನ್ನು ವಿಚಿತ್ರವಾಗಿ ತಿರಿಚುವ ಪಾಶ್ಚಾತ್ಯ ಪಂಡಿತರ ಪ್ರಯತ್ನವನ್ನು ಬಯಲಿಗೆಳೆಯುವಲ್ಲಿ ಈ ‘ಧರ್ಮ ಚರಿತ ಮೇಳ’ ನೆರವಾಯಿತು. ಸಮ್ಮೇಳನದಲ್ಲಿ ಭಾಗವಹಿಸಿದ ಹಲವಾರು ವಿದ್ವಾಂಸರು ಸ್ವಾಮೀಜಿಯವರ ವಾದಗಳನ್ನು ಅನುಮೋದಿಸಿದರು. ಒಟ್ಟಿನಲ್ಲಿ ಈ ಸಮ್ಮೇಳನದಿಂದ ಹೆಚ್ಚೇನೂ ಸಾಧಿಸಲ್ಪಡದಿದ್ದರೂ ಸ್ವಾಮೀಜಿಯವರ ಅಸಾಮಾನ್ಯ ಪಾಂಡಿತ್ಯವೂ, ಹಿಂದೂ ಹಾಗೂ ಇತರ ಧರ್ಮಗಳ ಬಗೆಗಿನ ಅವರ ಜ್ಞಾನವೂ ಸ್ವಂತಿಕೆಯೂ ಮತ್ತಷ್ಟು ಪ್ರಕಾಶಕ್ಕೆ ಬಂದುವು. ಅಲ್ಲದೆ ಹಿಂದೂ ಧರ್ಮ-ಸಂಸ್ಕೃತಿಗಳ ಬಗ್ಗೆ ಇನ್ನಷ್ಟು ಆಳವಾಗಿ, ಹೊಸದಾಗಿ ಸಂಶೋಧನೆ ನಡೆಯಬೇಕಾದ ಮತ್ತು ಅದರ ಚರಿತ್ರೆಯನ್ನು ಭಾರತೀಯರೇ ಪುನರ್​ರಚಿಸ ಬೇಕಾದ ಆವಶ್ಯಕತೆಯೂ ಸ್ಪಷ್ಟವಾಗಿ ಎದ್ದುಕಂಡಿತು.

ಪ್ಯಾರಿಸಿನಲ್ಲಿ ಸ್ವಾಮೀಜಿ ಜಾಗತಿಕ ಪ್ರದರ್ಶನವನ್ನು ವೀಕ್ಷಿಸುವುದರಲ್ಲಿ ಹಾಗೂ ಸಮ್ಮೇಳನ ದಲ್ಲಿ ಭಾಗವಹಿಸುವುದರಲ್ಲಿ ನಿರತರಾಗಿದ್ದರು. ಬಿಡುವಿನ ವೇಳೆಯನ್ನೆಲ್ಲ ತಮ್ಮ ಆತಿಥೇಯನಾದ ಜೂಲ್ಸ್ ಬ್ಯೂಸ್​ನ ಮನೆಯಲ್ಲಿ ಅಧ್ಯಯನದಲ್ಲೂ ಧ್ಯಾನದಲ್ಲೂ ಕಳೆಯುತ್ತಿದ್ದರು. ಆದರೆ ಇವೆಲ್ಲದರೊಂದಿಗೆ ಅವರ ಗಮನವನ್ನು ಬಹಳವಾಗಿ ಸೆಳೆದಿದ್ದ ಎರಡು ವಿಚಾರಗಳೆಂದರೆ, ಬೇಲೂರು ಮಠದ ವ್ಯವಹಾರದ ಮೇಲ್ವಿಚಾರಣೆಯ ಕಾರ್ಯ ಹಾಗೂ ತಮ್ಮ ಶಿಷ್ಯೆ ನಿವೇದಿತೆ ಯನ್ನು ನಿಜಾರ್ಥದಲ್ಲಿ ‘ನಿವೇದಿಸಲ್ಪಟ್ಟವ’ಳನ್ನಾಗಿ ತಯಾರು ಮಾಡುವ ಕಾರ್ಯ.

ಇಲ್ಲಿಯವರೆಗೆ ಬೇಲೂರು ಮಠದ ಆಸ್ತಿಯೆಲ್ಲವೂ ಸ್ವಾಮೀಜಿಯವರ ಹೆಸರಿನಲ್ಲಿತ್ತು. ಸಾಧ್ಯ ವಾದಷ್ಟು ಬೇಗನೆ ಈ ಹೊಣೆಗಾರಿಕೆಯನ್ನು ತಮ್ಮ ಗುರುಭಾಯಿಗಳಿಗೆ ದಾಟಿಸಲು ಅವರು ಕಾತರರಾಗಿದ್ದರು. ಅದಕ್ಕಾಗಿ ಒಂದು ಟ್ರಸ್ಟನ್ನು ರಚಿಸಿ, ಅದರ ಹೆಸರಿಗೆ ಈ ಆಸ್ತಿಯನ್ನೆಲ್ಲ ಬರೆಯುವ ಕಾನೂನು ರೀತ್ಯಾ ಕ್ರಮಗಳು ಒಂದು ವರ್ಷದ ಹಿಂದೆಯೇ ಪ್ರಾರಂಭವಾಗಿದ್ದುವು. ಅಲ್ಲದೆ, ಹಿಂದೆಯೇ ಹೇಳಿದಂತೆ, ಇಲ್ಲಿಯವರೆಗೆ ಅದು ಸಾಧ್ಯವಾಗದ್ದರಿಂದ ಬಾಲಿಯ ಪುರಸಭೆ ಭಾರೀ ತೆರಿಗೆ ಹಾಕಲು ಸನ್ನದ್ಧವಾಗಿತ್ತು. ಅಲ್ಲದೆ ಮಠಕ್ಕೆ ಏನಾದರೂ ತೊಂದರೆಯಾಗುವು ದಿದ್ದಲ್ಲಿ ಅದನ್ನು ತಂದೊಡ್ಡಲು ಭಾರತದಲ್ಲಿ ಅನೇಕರು ಸಿದ್ಧರಾಗಿದ್ದರು. ಆದ್ದರಿಂದ ಟ್ರಸ್ಟ್ ರಚಿಸಿ ಒಪ್ಪಂದಪತ್ರವನ್ನು ಸಾಧ್ಯವಾದಷ್ಟು ಬೇಗ ಸಿದ್ಧಪಡಿಸಿ ಸಹಿ ಹಾಕಲು ಸ್ವಾಮೀಜಿ ಕಾತರ ರಾಗಿದ್ದರು. ಅಂತೂ ಕಡೆಗೆ ೧೯ಂಂರ ಆಗಸ್ಟಿನಲ್ಲಿ ಮಠದಿಂದ ಕಳಿಸಿ ಕೊಡಲಾದ ಒಪ್ಪಂದಪತ್ರದ ಕರಡಿಗೆ ಅವರು ಸಹಿ ಹಾಕಿ ಹಿಂದಿರುಗಿಸಿದರು. ಅದರಲ್ಲಿ ಅವರು ಮಠದ ಅಧ್ಯಕ್ಷಸ್ಥಾನಕ್ಕೆ ರಾಜೀನಾಮೆ ನೀಡಿ, ಸ್ವಾಮಿ ಬ್ರಹ್ಮಾನಂದರನ್ನು ಆ ಸ್ಥಾನಕ್ಕೆ ನೇಮಿಸಿದ್ದರು. ಈಗ ಅವರಿಗೆ ಒಂದು ದೊಡ್ಡ ಭಾರವನ್ನು ಕಳೆದುಕೊಂಡಂತೆ ನೆಮ್ಮದಿಯೆನಿಸಿತು.

ಆದರೆ ಮಠದ ಕೆಲಸಕಾರ್ಯಗಳು ತೀರಾ ನಿಧಾನವಾಗಿ ಸಾಗುತ್ತಿವೆಯೆಂದು ಸ್ವಾಮೀಜಿ ಬಹಳವಾಗಿ ಸಿಟ್ಟಾಗಿದ್ದರು. ಗಂಗಾನದಿಯಲ್ಲಿ ಪ್ರವಾಹ ಬಂದಾಗ ನೀರು ಮಠದ ಆವರಣ ದೊಳಗೆ ನುಗ್ಗುವುದನ್ನು ತಡೆಯಲು ಅಡ್ಡಗೋಡೆಯೊಂದರ ನಿರ್ಮಾಣ ನಡೆಯುತ್ತಿತ್ತು. ಇದು ಬಹಳ ಪ್ರಾಮುಖ್ಯವಾದದ್ದಾಗಿತ್ತು. ಆದರೆ ತಾಂತ್ರಿಕವಾಗಿ ಅಷ್ಟೇ ಕಷ್ಟದ್ದಾಗಿತ್ತು. ಇದನ್ನು ಹಿಂದಿನ ವರ್ಷದೊಳಗಾಗಿ (೧೮೯೯) ಮುಗಿಸಲೇಬೇಕೆಂದು ಅವರು ಬ್ರಹ್ಮಾನಂದರಿಗೆ ತಿಳಿಸಿ ದ್ದರು. ಆದರೆ ಕೆಲಸ ತುಂಬ ಮಂದಗತಿಯಲ್ಲಿ ಸಾಗುತ್ತಿದೆಯೆಂದು ತಿಳಿದಾಗ ಸ್ವಾಮೀಜಿ ಸಿಟ್ಟಿಗೆದ್ದು ಕಟು ಶಬ್ದಗಳಿಂದ ಪತ್ರ ಬರೆದರು. ತಕ್ಷಣ ಕೆಲಸ ಚುರುಕಾಯಿತು. ಆದರೆ ಕೆಲಸ ನಿಧಾನವಾದುದಕ್ಕೆ ಕಾರಣಗಳಿದ್ದುವೆಂಬುದು ಅವರಿಗೆ ಆಮೇಲೆ ತಿಳಿಯಿತು. ಅಂತೂ ಭಾರತಕ್ಕೆ ಮರಳುವ ವೇಳೆಗೆ ಸಾಕಷ್ಟು ಪ್ರಗತಿಯಾಗಿದ್ದುದನ್ನು ಕಂಡು ಅವರು ಸಮಾಧಾನ ತಾಳಿದರು.

ಇದೇ ವೇಳೆಯಲ್ಲಿ ಅವರು ನಿವೇದಿತೆಯನ್ನು ತಯಾರುಗೊಳಿಸುವ ಕಾರ್ಯದಲ್ಲಿ ನಿರತರಾಗಿ ದ್ದರು. ನಾವೀಗಾಗಲೇ ನೋಡಿರುವಂತೆ, ನಿವೇದಿತೆ ತನ್ನ ಸ್ವಂತ ಇಚ್ಛೆಯಿಂದ ಅವರ ಶಿಷ್ಯೆಯಾಗಿ ಬಂದಿದ್ದವಳು. ಮುಂದೆ ಆಕೆ ಎದುರಿಸಬೇಕಾಗುವ ಸಮಸ್ಯೆಗಳನ್ನೆಲ್ಲ ಅವರು ಮೊದಲೇ ವಿವರಿಸಿ, ತನ್ನ ನಿರ್ಧಾರವನ್ನು ಕೈಗೊಳ್ಳುವಾಗ ಸಾಕಷ್ಟು ಎಚ್ಚರ ವಹಿಸುವಂತೆ ಮುನ್ಸೂಚನೆ ನೀಡಿದ್ದರು. ನಿವೇದಿತೆಯೂ ಆ ವಿಷಯವಾಗಿ ಮತ್ತೆ ಮತ್ತೆ ಆಲೋಚಿಸಿ, ಕಡೆಗೆ ದೃಢನಿರ್ಧಾರದಿಂದ, ಆತ್ಮ ವಿಶ್ವಾಸದಿಂದ, ತಾನು ಆ ಕಾರ್ಯಕ್ಕೆ ಸಿದ್ಧಳಾಗಿದ್ದೇನೆಂದು ಮುಂದೆ ಬಂದಿದ್ದಳು. ತಾನು ಕೈಗೊಂಡ ನಿರ್ಧಾರಕ್ಕಾಗಿ ಆಕೆ ಇದುವರೆಗೂ ಪಶ್ಚಾತ್ತಾಪ ಪಡಬೇಕಾಗಿ ಬಂದಿರಲಿಲ್ಲ.

ಆದರೆ ನಿವೇದಿತೆಗಿನ್ನೂ ನಿಜವಾದ ಶಿಷ್ಯತ್ವದ ಅರ್ಥವ್ಯಾಪ್ತಿ ಅರಿವಾಗಿರಲಿಲ್ಲ. ಅವಳಿಗೆ ತನ್ನ ಹೊಣೆಗಾರಿಕೆಯ ಅರಿವಿತ್ತು. ತನ್ನ ಕಾರ್ಯದ ಮೇಲೆ ಪ್ರೀತಿಯಿತ್ತು. ಆದರೆ ತನ್ನನ್ನು ಆ ಕಾರ್ಯದಲ್ಲಿ ಹೇಗೆ ತೊಡಗಿಸಿಕೊಳ್ಳಬೇಕು, ತನ್ನಲ್ಲಿ ಯಾವ ರಭಸ ಉತ್ಪನ್ನವಾಗಬೇಕೆಂದು ಸ್ವಾಮೀಜಿ ನಿರೀಕ್ಷಿಸುತ್ತಿದ್ದಾರೆ ಎಂಬುದನ್ನು ಅವಳಿನ್ನೂ ಅರ್ಥಮಾಡಿಕೊಂಡಿರಲಿಲ್ಲ. ಅವಳ ಮನಸ್ಸನ್ನು ಇತರ ಶಕ್ತಿಗಳು ಈಗಲೂ ವಿಚಲಿತಗೊಳಿಸಬಹುದಾಗಿತ್ತು ಎಂಬ ಅಂಶ ನ್ಯೂಯಾರ್ಕಿ ನಲ್ಲಿ ವ್ಯಕ್ತವಾಯಿತು.

ಭಾರತದಲ್ಲಿನ ತನ್ನ ಕಾರ್ಯಗಳಿಗಾಗಿ ಧನಸಹಾಯವನ್ನು ಪಡೆಯಲು ಸ್ವಾಮೀಜಿಯವ ರೊಂದಿಗೆ ಬಂದಿದ್ದ ನಿವೇದಿತೆ, ತನಗರಿವಿಲ್ಲದಂತೆಯೇ ತನ್ನ ಕೆಲಸದಿಂದ ವಿಮುಖಳಾಗತೊಡಗಿ ದ್ದಳು. ನ್ಯೂಯಾರ್ಕಿನಲ್ಲಿ ಆಕೆಗೆ ಅಂದಿನ ಖ್ಯಾತ ಸಮಾಜ ಶಾಸ್ತ್ರಜ್ಞರಾದ ಪ್ರೊ ॥ ಪ್ಯಾಟ್ರಿಕ್ ಗಿಡ್ಡಿಸ್ ಎಂಬವರ ಪರಿಚಯವಾಯಿತು. ನಿವೇದಿತೆ ತನ್ನ ಕಾರ್ಯಕ್ಕಾಗಿ ನೆರವನ್ನು ಬೇಡಲು ಬಂದದ್ದು. ಆದರೆ ಪರಿಸ್ಥಿತಿ ತದ್ವಿರುದ್ಧವಾಯಿತು. ಗಿಡ್ಡಿಸರ ಸಮಾಜಶಾಸ್ತ್ರದ ವ್ಯಾಖ್ಯಾನಗಳು ಅವಳನ್ನು ಬಲವಾಗಿ ಆಕರ್ಷಿಸಿದುವು. ಇವರ ಮತ್ತು ಇವರ ಗುರುವಾದ ಹರ್ಬರ್ಟ್ ಸ್ಪೆನ್ಸರನ ಸಮಾಜಶಾಸ್ತ್ರದ ವ್ಯಾಖ್ಯಾನಗಳನ್ನೆಲ್ಲ ಕುಡಿದು ಅರಗಿಸಿಕೊಂಡವರು ಸ್ವಾಮೀಜಿ ಎಂಬುದು ಅವಳಿಗಿನ್ನೂ ತಿಳಿದಿರಲಿಲ್ಲವೆಂದು ತೋರುತ್ತದೆ. ತಾರುಣ್ಯದ ಅಪರಿಪಕ್ವ ಮನಃಸ್ಥಿತಿಯಲ್ಲಿ ಅವಳಿಗೆ ಎಲ್ಲವೂ ಅದ್ಭುತವಾಗಿ ತೋರಿರಬೇಕು. ಈ ಸಮಾಜಶಾಸ್ತ್ರದ ಅಧ್ಯಯನದಿಂದ ತಾನು ಕೈಗೊಂಡಿದ್ದ ಕಾರ್ಯಕ್ಕೆ ಬಹಳ ಸಹಾಯವಾಗುತ್ತದೆ ಎಂದು ಆಕೆ ನಂಬಿದಳು. ತನ್ನ ಈ ಹೊಸ ಸ್ನೇಹಿತರ ವ್ಯಕ್ತಿತ್ವವನ್ನು ಬಣ್ಣಿಸಿ, ಆಗ ಕ್ಯಾಲಿಫೋರ್ನಿಯದಲ್ಲಿದ್ದ ಸ್ವಾಮೀಜಿಯವರಿಗೆ ದೀರ್ಘ ಪತ್ರವೊಂದನ್ನು ಬರೆದಳು. ಅವಳ ಅಂತರಂಗವನ್ನರಿಯಲು ಅವರಿಗೆ ಅದು ಬೇಕಾದಷ್ಟಾಯಿತು. ಆದರೂ ಅವರು ಅವಳನ್ನು ಎಚ್ಚರಿಸುವ ಗೋಜಿಗೆ ಹೋಗದೆ, ಆ ಬಗ್ಗೆ ತಮ್ಮ ಸಂತೋಷವನ್ನೇ ವ್ಯಕ್ತಪಡಿಸಿ ಪತ್ರ ಬರೆದರು.

ಇತ್ತ ನಿವೇದಿತೆ ಪ್ರೊ ॥ ಗಿಡ್ಡಿಸರ ಕೆಲಸದಲ್ಲಿ ಹೆಚ್ಚು ಹೆಚ್ಚು ಆಸಕ್ತಿ ವಹಿಸುತ್ತ ಬಂದಳು. ಗಿಡ್ಡಿಸರಿಗೆ ಪ್ಯಾರಿಸಿನ ಜಾಗತಿಕ ಮೇಳದ ಸಂಬಂಧವಾಗಿ ಬಹಳಷ್ಟು ಹೊಣೆಗಾರಿಕೆಯಿತ್ತು. ಈ ಕೆಲಸದಲ್ಲಿ ಅವರಿಗೆ ನೆರವಾಗಲು ನಿವೇದಿತೆ ಒಪ್ಪಿದಳು; ಮತ್ತು ಆ ಕೆಲಸದ ಪ್ರಯುಕ್ತವಾಗಿಯೇ ಪ್ಯಾರಿಸಿಗೆ ಬಂದಿದ್ದಳು. ಇದನ್ನೆಲ್ಲ ಸೂಕ್ಷ್ಮವಾಗಿ ಗಮನಿಸುತ್ತಲೇ ಇದ್ದ ಸ್ವಾಮೀಜಿ, ನಿಧಾನವಾಗಿ ಅವಳನ್ನು ಅವಳಷ್ಟಕ್ಕೆ ಬಿಟ್ಟುಬಿಡಲಾರಂಭಿಸಿದರು. ಇದರಿಂದ ನಿವೇದಿತೆಗೆ ತನ್ನ ತಪ್ಪಿನ ಅರಿವಾ ಗದೆ, ಸ್ವಾಮೀಜಿಯವರ ಮೇಲೆಯೇ ಅಸಮಾಧಾನವಾಯಿತು. ಅವರು ತನ್ನನ್ನು ಕಡೆಗಣಿಸು ತ್ತಿದ್ದಾರೆ, ನಡುದಾರಿಯಲ್ಲಿ ಕೈಬಿಡುತ್ತಿದ್ದಾರೆ ಎಂದು ದುಃಖಿಸಿದಳು.

ಈ ವಿಚಾರವನ್ನೆಲ್ಲ ಅರಿತಿದ್ದ ಶ್ರೀಮತಿ ಸಾರಾ, ನಿವೇದಿತೆಯನ್ನು ಸಮಾಧಾನ ಪಡಿಸುವುದಕ್ಕಾಗಿ ಆಕೆಯನ್ನು ಫ್ರಾನ್ಸಿನ ಸಮುದ್ರತೀರದಲ್ಲಿದ್ದ ತನ್ನ ಹಳ್ಳಿಯ ಮನೆಗೆ ಆಹ್ವಾನಿಸಿದಳು. ಇಲ್ಲಿಂದ ನಿವೇದಿತೆ ತನ್ನ ದುಃಖವನ್ನು ತೋಡಿಕೊಂಡು ಸ್ವಾಮೀಜಿಯವರಿಗೊಂದು ಪತ್ರ ಬರೆದಳು. ಅದಕ್ಕೆ ಅವರು ಬರೆದ ಉತ್ತರ ಹೀಗಿದೆ:

“ನಿನ್ನ ಪತ್ರ ಈಗ ತಾನೆ ತಲುಪಿತು. ನೀನು ವ್ಯಕ್ತಪಡಿಸಿರುವ ಸದ್ಭಾವನೆಗಳಿಗಾಗಿ ಕೃತಜ್ಞತೆಗಳು.... ಈಗ ನಾನು ಸಂಪೂರ್ಣ ಸ್ವತಂತ್ರನಾಗಿದ್ದೇನೆ. ಸಂಘದ ಕೆಲಸದಲ್ಲಿ ನಾನು ಯಾವ ಅಧಿಕಾರ ವನ್ನೂ ಸ್ಥಾನವನ್ನೂ ಇಟ್ಟುಕೊಂಡಿಲ್ಲ. ರಾಮಕೃಷ್ಣ ಮಿಷನ್ನಿನ ಅಧ್ಯಕ್ಷ ಸ್ಥಾನಕ್ಕೂ ನಾನು ರಾಜೀನಾಮೆ ನೀಡಿದ್ದೇನೆ. ಈಗ ಮಠ ಇತ್ಯಾದಿಗಳೆಲ್ಲ ರಾಮಕೃಷ್ಣರ ಇತರ ಶಿಷ್ಯರಿಗೆ ಸೇರಿದವು. ಈಗ ಅಧ್ಯಕ್ಷಸ್ಥಾನ ಬ್ರಹ್ಮಾನಂದರದು. ಮುಂದೆ ಅದು ಕ್ರಮವಾಗಿ ಪ್ರೇಮಾನಂದರೇ ಮೊದಲಾ ದವರದ್ದಾಗಲಿದೆ.

“ಈ ಕ್ಷಣದಿಂದ ನಾನು ಯಾರನ್ನೂ ಪ್ರತಿನಿಧಿಸುವುದಿಲ್ಲ. ಅಥವಾ ನಾನು ಯಾರಿಗೂ ಹೊಣೆಗಾರನೂ ಅಲ್ಲ. ನನ್ನ ಸ್ನೇಹಿತರ ಬಗ್ಗೆ ನನಗೊಂದು ಹಂಗು ಇತ್ತು. ಈಗ ನಾನು ಚೆನ್ನಾಗಿ ಆಲೋಚಿಸಿ ಒಂದು ತೀರ್ಮಾನಕ್ಕೆ ಬಂದಿದ್ದೇನೆ–ನಾನು ಯಾರಲ್ಲಿಯೂ ಯಾವುದೇ ಪುಣವ ನ್ನಿಟ್ಟುಕೊಂಡಿಲ್ಲ. ನಾನು ನನ್ನ ಶಕ್ತಿಯನ್ನೆಲ್ಲ ಬಸಿದಿದ್ದೇನೆ; ಅಷ್ಟೇಕೆ, ಜೀವವನ್ನೇ ಕೊಟ್ಟಿದ್ದೇನೆ. ಆದರೆ ನನಗೆ ಪ್ರತಿಯಾಗಿ ಸಿಕ್ಕಿದ್ದು ಕೇವಲ ತಲೆಹರಟೆ, ತರಲೆ, ತಲೆನೋವು.

“ನಾನು ನಿನ್ನ ಹೊಸ ಸ್ನೇಹಿತರ ಬಗ್ಗೆ ಅಸೂಯೆ ತಾಳಿದ್ದೇನೆಂದು ನಿನ್ನ ಪತ್ರ ಸೂಚಿಸುತ್ತದೆ. ನಾನು ಯಾವುದೇ ದೋಷಗಳಿಂದೊಡಗೂಡಿ ಹುಟ್ಟಿರಬಹುದು, ಆದರೆ ಒಂದು ವಿಷಯವನ್ನು ಮಾತ್ರ ಖಂಡಿತವಾಗಿ ತಿಳಿದುಕೊ–ಜನ್ಮತಃ ನಾನು ಮತ್ಸರದಿಂದ, ದ್ವೇಷದಿಂದ ಮತ್ತು ಯಜಮಾನಿಕೆಯ ಆಸೆಯಿಂದ ದೂರವಾಗಿದ್ದೇನೆ. 

“ನಾನು ಹಿಂದೆಂದೂ ನಿನಗೆ ಹೀಗೆಯೇ ಮಾಡಬೇಕು ಎಂದು ಆದೇಶ ನೀಡಿದವನಲ್ಲ. ಈಗಲಂತೂ ನಾನು ಏನೂ ಅಲ್ಲ–ಇನ್ನು ನಾನು ನಿರ್ದೇಶಿಸುವಂತೆಯೂ ಇಲ್ಲ. ನಾನು ಇಷ್ಟನ್ನು ಮಾತ್ರ ಹೇಳಬಲ್ಲೆ–ಎಲ್ಲಿಯವರೆಗೆ ನೀನು ಹೃತ್ಪೂರ್ವಕವಾಗಿ ಜಗನ್ಮಾತೆಯ ಸೇವೆ ಮಾಡು ತ್ತಿರುತ್ತೀಯೋ ಅಲ್ಲಿಯವರೆಗೆ ಅವಳೇ ನಿನ್ನ ಮಾರ್ಗದರ್ಶಕಿಯಾಗಿರುತ್ತಾಳೆ.

“ನೀನು ಯಾವ ಸ್ನೇಹಿತರನ್ನು ಪಡೆದುಕೊಂಡರೂ ಅದರ ಬಗ್ಗೆ ನನಗೆಂದೂ ಅಸೂಯೆ ಯಿರಲಿಲ್ಲ... ಆದರೆ ನನಗನ್ನಿಸುತ್ತದೆ–ಪಾಶ್ಚಾತ್ಯರದ್ದೊಂದು ವಿಚಿತ್ರ ಸ್ವಭಾವವಿದೆ; ಏನೆಂದರೆ, ತಮಗೆ ಯಾವುದು ಸರಿಯೆಂದು ತೋರುತ್ತದೆಯೋ ಅದನ್ನೇ ಇತರರ ಮೇಲೂ ಹೇರುವುದು. ತಮಗೆ ಯಾವುದು ಒಳ್ಳೆಯದೋ ಅದು ಇತರರಿಗೆ ಪಥ್ಯವಾಗದಿರಬಹುದು ಎಂಬುದನ್ನು ಅವರು ಮರೆತುಬಿಡುತ್ತಾರೆ. ಅಂತೆಯೇ, ನೀನು ಹೊಸ ಸ್ನೇಹಿತರ ಸಂಪರ್ಕಕ್ಕೆ ಬಂದಾಗ ನಿನ್ನ ಮನಸ್ಸು ಯಾವ ಯಾವ ರೀತಿಯಲ್ಲಿ ತಿರುಗುತ್ತದೆಯೋ ಅದೇ ಆಲೋಚನೆ ಗಳನ್ನೇ ನೀನು ಇತರರ ಮೇಲೂ ಹೇರಬಹುದು ಎಂಬುದು ನನ್ನ ಶಂಕೆ. ಆದ್ದರಿಂದಲೇ ನಾನು ನಿನ್ನ ಮೇಲೆ ಆಗಬಹುದಾದ ಯಾವುದೇ ಹೊಸ ಪ್ರಭಾವವನ್ನು ತಡೆಗಟ್ಟಲು ಕೆಲವೊಮ್ಮೆ ಪ್ರಯತ್ನಿಸಿದೆನಷ್ಟೇ ಅಲ್ಲದೆ ಬೇರೇನಿಲ್ಲ... ನೀನು ಸ್ವತಂತ್ರಳು. ನಿನ್ನದೇ ಆದ ಆಯ್ಕೆಯನ್ನು ಮಾಡಿಕೊ; ನಿನಗಿಷ್ಟ ಬಂದ ಕೆಲಸವನ್ನು ಮಾಡು.

“ಮಿತ್ರರಾಗಲಿ ಶತ್ರುಗಳೇ ಆಗಲಿ, ಅವರೆಲ್ಲರೂ ನಾವು ನಮ್ಮ ಕರ್ಮಗಳನ್ನು ಸಂತೋಷ ದಿಂದಲೋ ದುಃಖದಿಂದಲೋ ಸವೆಸಲು ಜಗನ್ಮಾತೆಯ ಕೈಯಲ್ಲಿನ ಕರಣಗಳಷ್ಟೆ. ಆದ್ದರಿಂದ ಆಕೆ ಎಲ್ಲರನ್ನೂ ಹರಸಲಿ.”

ಸ್ವಾಮೀಜಿಯವರ ಉತ್ತರವನ್ನು ಕಂಡು ನಿವೇದಿತೆ ಸಂಪೂರ್ಣ ಕುಸಿದು ಹೋಗಿಬಿಟ್ಟಳು. ಅವಳನ್ನು ತನ್ನ ಮಗಳಂತೆ ಕಾಣುತ್ತಿದ್ದ ಸಾರಾ ಬುಲ್ ಸ್ವಾಮೀಜಿಯವರು ತನ್ನ ಮನೆಗೇ ಬಂದಿರಲು ಸಾಧ್ಯವಾದರೆ ಇಬ್ಬರ ನಡುವಣ ಭಿನ್ನಾಭಿಪ್ರಾಯ ಸರಿಹೋಗಬಹುದೆಂದು ಭಾವಿಸಿ ಅವರನ್ನು ತನ್ನಲ್ಲಿಗೆ ಆಹ್ವಾನಿಸಿದಳು. ಇವರ ಜೊತೆಯಲ್ಲೇ ಇದ್ದ ಜೋಸೆಫಿನ್ನಳೂ ಸ್ವಾಮೀಜಿ ಯವರನ್ನು ಆಹ್ವಾನಿಸಿ ಒಂದು ಪತ್ರ ಬರೆದಳು. ಇಷ್ಟು ಹೊತ್ತಿಗೆ ಸಮ್ಮೇಳನದ ಭರಾಟೆಯೆಲ್ಲ ಕೊನೆಗೊಂಡಿದ್ದರಿಂದ, ಈ ಆಹ್ವಾನಗಳಿಗೆ ಸ್ವಾಮೀಜಿ ಸಮ್ಮತಿಸಿದರು.

ಸಾರಾ ಇದ್ದದ್ದು ಪ್ಯಾರಿಸಿನಿಂದ ೩ಂಂ ಮೈಲಿ ದೂರದಲ್ಲಿ; ಇಂಗ್ಲಿಷ್ ಕಾಲುವೆಯ ತೀರದ ಬ್ರಿಟಾನಿ ಎಂಬಲ್ಲಿ. ತಮ್ಮ ಆತಿಥೇಯನಾದ ಜೂಲ್ಸ್ ಬ್ಯೂಸನೊಂದಿಗೆ ಸ್ವಾಮೀಜಿ ಸೆಪ್ಟೆಂಬರ್ ೧೭ರಂದು ಇಲ್ಲಿಗೆ ಬಂದರು. ಇದೊಂದು ಪ್ರಶಾಂತ, ಸುಂದರ ತಾಣ. ಇಲ್ಲಿ ಸ್ವಾಮೀಜಿ ತಮ್ಮ ಆಪ್ತರೊಂದಿಗೆ ಸಂಭಾಷಿಸುತ್ತ, ವಿಶ್ರಾಂತಿ ಪಡೆಯುತ್ತ ತಿಂಗಳ ಕೊನೆಯವರೆಗೂ ಉಳಿದು ಕೊಂಡರು.

ಆದರೆ ಸ್ವಾಮೀಜಿಯವರ ಮನೋಭಾವದಲ್ಲಿ ನಿವೇದಿತೆ ಯಾವ ಬದಲಾವಣೆಯನ್ನೂ ಕಾಣಲಿಲ್ಲ. ಆಕೆಗೆ ತನ್ನ ತಪ್ಪು ನಿಧಾನವಾಗಿ ಅರಿವಾಗತೊಡಿಗಿತ್ತು. ಅಲ್ಲದೆ ಅವಳು ಪ್ರೊ ॥ ಗಿಡ್ಡಿಸರೊಂದಿಗೆ ಕೆಲಸ ಮಾಡಲೂ ಅಸಮರ್ಥಳಾಗಿ ಅವರಿಗೆ ವಿದಾಯ ಹೇಳಿದ್ದಳು. ಈಗಲಾ ದರೂ ಸ್ವಾಮೀಜಿ ತನ್ನನ್ನು ಕ್ಷಮಿಸಿ ಸಂತೈಸಲಿ ಎಂದು ಹಾರೈಸಿದಳು. ಸ್ವಾಮೀಜಿ ಮಾತ್ರ ತಾವಾಡಿದ ಮಾತನ್ನು ಹಿಂದೆಗೆದುಕೊಳ್ಳಲಿಲ್ಲ. ಆದರೆ ತಮ್ಮ ಆಧ್ಯಾತ್ಮಿಕ ಪುತ್ರಿಯ ಮೇಲಿನ ಅವರ ಪ್ರೀತಿಯೇನೂ ಕಡಿಮೆಯಾಗಿರಲಿಲ್ಲ. ಅವರು ಅದನ್ನು ಬಹಿರಂಗವಾಗಿ ವ್ಯಕ್ತಪಡಿಸಲಿಲ್ಲ ಅಷ್ಟೆ. ನಿವೇದಿತೆ ಇದನ್ನು ಸರಿಯಾಗಿ ಅರ್ಥ ಮಾಡಿಕೊಳ್ಳಲಾರದೆಹೋದಳು. ತನ್ನ ಕಾರ್ಯದಲ್ಲಿ ತನ್ನನ್ನು ಮತ್ತೆ ತೊಡಗಿಸಿಕೊಳ್ಳಲು ಅವಳ ಮನಸ್ಸು ಕಾತರಗೊಂಡಿತ್ತು. ಈ ಸಂದರ್ಭದಲ್ಲಿ ಸ್ವಾಮೀಜಿ ಅವಳಿಗೊಂದು ಕವನವನ್ನು ಬರೆದರು–

\begin{verse}
ತಾಯ ಹೃದಯದ ವಾತ್ಸಲ್ಯಶಕ್ತಿ,\\ಧೀರನೆದೆಯ ಸಂಕಲ್ಪಶಕ್ತಿ,\\ಮಲಯ ಮಾರುತದ ಮಾಧುರ್ಯ,\\ಆರ್ಯ ಪೂಜಾಗೃಹದಿ ನೆಲೆಗೊಂಡು, ಸ್ವತಂತ್ರವಾಗಿ\\ಜ್ವಲಿಸುವ ಪಾವಿತ್ರ್ಯದ ಶಕ್ತಿ, ಸೌಂದರ್ಯ–\\ಇವೆಲ್ಲವೂ ನಿನ್ನವಾಗಲಿ;\\ಹಿಂದೆಂದೂ ಯಾರೂ ಕನಸಿನಲ್ಲೂ ನೆನೆಸಿರದ\\ಇನ್ನೆಷ್ಟೋ ನಿನ್ನವಾಗಲಿ!\\ಭಾರತದ ಭವಿಷ್ಯಪುತ್ರನಿಗೆ ನೀನಾಗು\\ಒಡತಿ, ಸೇವಕಿ, ಪ್ರಿಯಸಖಿ.
\end{verse}

ಇಷ್ಟಾದರೂ ನಿವೇದಿತೆಗೆ ಪೂರ್ಣ ಸಮಾಧಾನವಾಗಲಿಲ್ಲ. ಏಕೆಂದರೆ ಸ್ವಾಮೀಜಿಯವರು ಈ ಅವಧಿಯಲ್ಲಿ ಮತ್ತೆ ಅವಳ ಕೆಲಸದ ಬಗ್ಗೆ ಪ್ರಸ್ತಾಪಿಸಲೇ ಇಲ್ಲ. ಅಲ್ಲದೆ ಆಕೆ ಇತರ ಆಸಕ್ತಿ ಗಳನ್ನೆಲ್ಲ ಮರೆತು ತನ್ನ ಕಾರ್ಯದಲ್ಲಿ ಮತ್ತೆ ಹೊಸ ಉತ್ಸಾಹದಿಂದ ತೊಡಗುವ ಬಗ್ಗೆ ಅವರಿ ಗಿನ್ನೂ ಭರವಸೆಯುಂಟಾಗಿರಲಿಲ್ಲ. ಅವರ ಅಂತರಂಗದ ಈ ಭಾವ ನಿವೇದಿತೆಯ ಎದೆಗೂ ತಟ್ಟಿತು. ಅವಳು ಮುಂದೆ ೧೯೧೦ರಲ್ಲಿ ಬರೆದು ಮುಗಿಸಿದ “ನಾ ಕಂಡಂತೆ ನನ್ನ ಗುರುದೇವ” ಎಂಬ ಪುಸ್ತಕದಲ್ಲಿ ಬರೆಯುತ್ತಾಳೆ–“ಸ್ವಾಮೀಜಿ ನನ್ನ ಬಗ್ಗೆ ಎಷ್ಟು ನಿರ್ಲಿಪ್ತರಾಗಿದ್ದರೆಂದರೆ, ಹಲವಾರು ವಿಶ್ವಾಸ ದ್ರೋಹಗಳನ್ನು ಎದುರಿಸಿದ್ದ ಅವರು, ಅಂತಹ ಮತ್ತೊಂದು ಘಟನೆಗೆ ಸಿದ್ಧರಾಗಿಯೇ ಇದ್ದಂತಿತ್ತು.”

ಆದರೆ ಸ್ವಾಮೀಜಿಯವರು ಅವಳ ಬಗ್ಗೆ ವಿಶೇಷವಾಗಿ ಬೇಸರಿಸಿಕೊಂಡಂತೆ ಅಥವಾ ಆ ಬಗ್ಗೆ ಚಿಂತಿಸುತ್ತಿದ್ದಂತೆಯೂ ಕಾಣುತ್ತಿರಲಿಲ್ಲ. ಅವರು ಎಲ್ಲರೊಂದಿಗೂ ಆನಂದದಿಂದ ಸಹಜವಾಗಿ ಮಾತನಾಡಿಕೊಂಡಿದ್ದರು. ಅವರ ಹೃದಯಾಂತರಾಳದಲ್ಲಿ ಏನಿದೆಯೆಂಬುದು ಯಾರಿಗೂ ಗೊತ್ತಾ ಗುವಂತಿರಲಿಲ್ಲ. ಆದರೆ ಅವರು ತೋರಾಣಿಕೆಗೆ ಮಾತ್ರ ಹಾಗೆ ನಡೆದುಕೊಳ್ಳುತ್ತಿದ್ದರೆಂದಲ್ಲ. ಅವರ ಅಂತರಂಗವು ವಿಶಾಲ ಸಾಗರದಂತೆ. ಒಂದು ಮೂಲೆಯಲ್ಲಿ ಅದು ಚಂಡಮಾರುತ ದಿಂದೊಡಗೂಡಿ ಭೋರ್ಗರೆಯುತ್ತಿದ್ದರೆ, ಮತ್ತೊಂದೆಡೆಯಲ್ಲಿ ಶಾಂತ ಸರೋವರದಂತೆ ಪ್ರಸನ್ನ-ಗಂಭೀರ.

ನಿವೇದಿತೆ ತನ್ನ ಮಾನಸಿಕ ತುಮುಲದ ನಡುವೆಯೂ ಸ್ವಾಮೀಜಿಯವರ ಪ್ರತಿಯೊಂದು ಮಾತುಕತೆಯನ್ನೂ ಆಸ್ವಾದಿಸುತ್ತಿದ್ದಳೆಂಬುದು ಆಕೆಯ ಗ್ರಂಥದಿಂದ ತಿಳಿದುಬರುತ್ತದೆ. ಒಮ್ಮೆ ಬ್ರಿಟಾನಿಯಲ್ಲಿ ಸ್ವಾಮೀಜಿ ತಮ್ಮ ಆಪ್ತರೊಂದಿಗೆ ಮಾತನಾಡುತ್ತ ಕುಳಿತಿದ್ದರು; ವಿಷಯ ಬುದ್ಧನನ್ನು ಕುರಿತದ್ದು. ಅವರು ಬೌದ್ಧಧರ್ಮಗ್ರಂಥಗಳ ಸಾಲುಗಳನ್ನು ಉದ್ಧರಿಸುತ್ತ, ಬುದ್ಧನ ದೈವೀ ಗುಣಗಳನ್ನೂ ಬೌದ್ಧಧರ್ಮದ ತತ್ತ್ವಗಳನ್ನೂ ವರ್ಣಿಸಿದರು; ಬೌದ್ಧತತ್ತ್ವಗಳೊಂದಿಗೆ ವೇದಾಂತವನ್ನು ಹೋಲಿಸಿದರು; ಅವೆರಡರ ನಡುವಣ ಸಾಮ್ಯವನ್ನು ಹಾಗೂ ವಿರೋಧವನ್ನು ಎತ್ತಿತೋರಿಸಿದರು. ಬಳಿಕ ಬುದ್ಧ ಮತ್ತು ಶಂಕರಾಚಾರ್ಯರ ಬೋಧನೆಗಳನ್ನು ಹಾಗೂ ವ್ಯಕ್ತಿತ್ವ ಗಳನ್ನು ಹೋಲಿಸಿದರು. ಬುದ್ಧನ ಹೃದಯವೂ ಶಂಕರಾಚಾರ್ಯರ ಬುದ್ಧಿಮತ್ತೆಯೂ ಸೇರಿ ದಂತಹ ವ್ಯಕ್ತಿತ್ವವೇ ಮಾನವತೆಯ ಅತ್ಯುಚ್ಚ ಸ್ಥಿತಿ ಎಂದು ಅವರೆನ್ನುತ್ತಿದ್ದರು. ಮತ್ತು ಮಾನವನ ಇತಿಹಾಸದಲ್ಲಿ ಕಂಡುಬರುವ ಇಂತಹ ಏಕಮಾತ್ರ ವ್ಯಕ್ತಿಯೆಂದರೆ ಶ್ರೀರಾಮಕೃಷ್ಣರು ಎಂಬುದು ಅವರ ತೀರ್ಮಾನ.

ಬುದ್ಧನ ವಿಷಯದಲ್ಲಿ ಅವರಿಗೆ ತಾರುಣ್ಯದಿಂದಲೂ ವಿಶೇಷ ನಂಟು. ಆದರೆ ಅವರಿಗೆ ಈಗ ಬುದ್ಧನ ಮೇಲಿದ್ದ ಪರಮಭಕ್ತಿಗೆ ಮತ್ತೊಂದು ಕಾರಣವನ್ನು ಊಹಿಸುತ್ತಾಳೆ ನಿವೇದಿತಾ–ತಮ್ಮ ಕಣ್ಣೆದುರಿನಲ್ಲೇ ಬದುಕಿದ್ದ ತಮ್ಮ ಗುರುದೇವನ ಜೀವನವನ್ನು ಅವರು ಇಪ್ಪತ್ತೈದು ಶತಮಾನಗಳ ಹಿಂದೆ ನಡೆದುಹೋದ ಇತಿಹಾಸಪ್ರಸಿದ್ಧ ಕಥೆಯೊಂದಿಗೆ ನಿರಂತರವಾಗಿ ಹೋಲಿಸಿ ನೋಡು ತ್ತಿದ್ದರು. ಬುದ್ಧನಲ್ಲಿ ಅವರು ರಾಮಕೃಷ್ಣರನ್ನು ನೋಡುತ್ತಿದ್ದರು; ರಾಮಕೃಷ್ಣರಲ್ಲಿ ಬುದ್ಧನನ್ನು ಕಾಣುತ್ತಿದ್ದರು. ಒಂದು ದಿನ ಅವರು ಬುದ್ಧನ ನಿರ್ಯಾಣದ ದೃಶ್ಯವನ್ನು ಬಣ್ಣಿಸುವಾಗ ಈ ಅಂಶ ಇದ್ದಕ್ಕಿದ್ದಂತೆ ಬಹಿರಂಗವಾಯಿತು–

ಬುದ್ಧಭಗವಂತನನ್ನು ವೃಕ್ಷದ ನೆರಳಲ್ಲಿ ವಸ್ತ್ರವನ್ನು ಹಾಸಿ ಮಲಗಿಸಲಾಗಿದೆ. ಬುದ್ಧ, ಸಾವಿಗೆ ಸಿದ್ಧನಾಗಿ ಬಲ ಮಗ್ಗುಲಲ್ಲಿ ಶಾಂತವಾಗಿ ಮಲಗಿದ್ದಾನೆ. ಶಿಷ್ಯರು ದುಃಖದಿಂದ ಅವನನ್ನೇ ನಿಟ್ಟಿಸುತ್ತ ಮೌನವಾಗಿ ಸುತ್ತಲೂ ಕುಳಿತಿದ್ದಾರೆ. ಆಗ ಇದ್ದಕ್ಕಿದ್ದಂತೆ ಒಬ್ಬ ಅಪರಿಚಿತ ವ್ಯಕ್ತಿ ಬುದ್ಧನ ವಚನಾಮೃತವನ್ನಾಲಿಸಲು ಓಡೋಡಿ ಬಂದ. ‘ಈ ಘಳಿಗೆಯಲ್ಲಿ ಶಾಂತಿಯನ್ನು ಕದಡಲು ಇವನೆಲ್ಲಿಂದ ಬಂದ!’ ಎಂದು ಶಿಷ್ಯರು ಅವನನ್ನು ನಿವಾರಿಸಲು ನೋಡಿದರು. ಆದರೆ ಅವರ ಮಾತುಗಳನ್ನು ಆಲಿಸಿದ ಬುದ್ಧ ತಕ್ಷಣ ಮಧ್ಯೆ ಪ್ರವೇಶಿಸಿ ನುಡಿದ: “ಕೂಡದು, ಕೂಡದು, ಬುದ್ಧನು ಭಗವಂತನಿಂದ ಕಳಿಸಲ್ಪಟ್ಟವನು–ಭಗವಂತನ ದೂತ. ಅವನು ಸದಾ ಸಿದ್ಧ.” ಹೀಗೆನ್ನುತ್ತ ಮೇಲೆದ್ದು ಕುಳಿತು ಅವನಿಗೆ ಉಪದೇಶಿಸಿದ. ಹೀಗೆಯೇ ನಾಲ್ಕು ಬಾರಿ ಪುನರಾವರ್ತನೆ ಯಾಯಿತು–ಎಂದರೆ ಒಬ್ಬರಾದ ಮೇಲೊಬ್ಬರಂತೆ ನಾಲ್ಕು ಜನ ಬಂದರು. ಪ್ರತಿ ಸಲವೂ ಬುದ್ಧ, ಅಪಾರ ಕರುಣೆಯಿಂದ ಬಂದವರಿಗೆ ಬೋಧಿಸಿ ಉಪದೇಶಿಸಿದ. ಅನಂತರವೇ ಅವನು ತನ್ನ ಪ್ರಾಣವನ್ನು ತ್ಯಜಿಸಲು ಅನುವಾದ!

ಈ ಅರ್ಥಪೂರ್ಣ ಘಟನೆಯನ್ನು ವರ್ಣಿಸುತ್ತಿದ್ದ ಸ್ವಾಮೀಜಿಯವರು, ‘ಬುದ್ಧನು ಮೇಲೆದ್ದು ಕುಳಿತ’ ಎನ್ನುವಾಗ ಒಂದು ಮಾತನ್ನು ಸೇರಿಸುತ್ತಾರೆ–“ನಾನೂ ಇದನ್ನು ಕಂಡೆ, ಶ್ರೀರಾಮಕೃಷ್ಣರ ಜೀವನದಲ್ಲಿ” ಎಂದು. ತಕ್ಷಣ ಕೇಳುಗರ ಮನಃಪಟಲದ ಮೇಲೆ ಆ ಚಿತ್ರ ಮೂಡಿನಿಂತಿತು– ಕಾಶೀಪುರದ ತೋಟದ ಮನೆ; ಶ್ರೀರಾಮಕೃಷ್ಣರು ಮರಣಶಯ್ಯೆಯಲ್ಲಿದ್ದಾರೆ, ವಾಕ್ ಶಕ್ತಿ ಉಡುಗಿಹೋಗಿದೆ. ಆಗ ಅವರನ್ನು ಕಂಡು ಅವರಿಂದ ಉಪದೇಶ ಪಡೆಯಲು, ನೂರು ಮೈಲಿಗಳ ದೂರದಿಂದ ಒಬ್ಬಾತ ಬಂದ. ಶಿಷ್ಯರು ಅವನನ್ನು ಇನ್ನೇನು ಹಿಂದಕ್ಕೆ ಕಳಿಸಬೇಕು, ಅಷ್ಟರಲ್ಲಿ ಶ್ರೀರಾಮಕೃಷ್ಣರು ಅವರನ್ನು ತಡೆದರು. ಆ ವ್ಯಕ್ತಿಯನ್ನು ಕರೆದು, ಪ್ರೀತಿಯಿಂದ ಉಪದೇಶ ನೀಡಿ ಕಳಿಸಿಕೊಟ್ಟರು.

ಸ್ವಾಮೀಜಿಯವರ ಮನಸ್ಸು ಹೇಗೆ ಆಲೋಚಿಸುತ್ತದೆಯೆಂದು ಯಾರೂ ಊಹಿಸಲು ಸಾಧ್ಯ ವಿರಲಿಲ್ಲ. ಈ ದಿನಗಳಲ್ಲೊಮ್ಮೆ ಅವರು ಸಮೀಪದ ದ್ವೀಪಕ್ಕೆ ಭೇಟಿ ನೀಡಿದರು. ಇಲ್ಲಿ ಮಧ್ಯ ಯುಗಕ್ಕೆ ಸೇರಿದ ಕತ್ತಲೆಯ ಗುಹೆಯಂತಹ ಕಾರಾಗೃಹಗಳಿವೆ. ಇವುಗಳಲ್ಲಿ ಎಂಟಡಿ ಉದ್ದ- ಅಗಲ-ಎತ್ತರದ ಮರದ ಬೋನುಗಳನ್ನು ತೂಗುಹಾಕಿ ಅಪರಾಧಿಗಳನ್ನು ಏಕಾಂಗಿಗಳಾಗಿ ಕೂಡಿ ಹಾಕುತ್ತಿದ್ದರು. ಇವುಗಳನ್ನು ನೋಡಿದರೆ ಮೈ ಜುಮ್ಮೆನ್ನುವಂತಿದೆ. ಆದರೆ ಈ ಬೋನುಗಳನ್ನು ಕಂಡಾಗ ಸ್ವಾಮೀಜಿ ತಮ್ಮಷ್ಟಕ್ಕೇ ಹೇಳಿಕೊಳ್ಳುವುದು ಕೇಳಿಸಿತು–“ಧ್ಯಾನಕ್ಕೆ ಎಂತಹ ಅದ್ಭುತ ಸ್ಥಳ!”

ಸೆಪ್ಟೆಂಬರ್ ೨೯ರಂದು ಅವರು ಪ್ಯಾರಿಸಿಗೆ ವಾಪಸಾಗಲಿದ್ದರು. ಅದಕ್ಕೆ ಮುಂಚೆಯೇ ನಿವೇದಿತೆ ತನ್ನ ಧನಸಂಗ್ರಹಣೆಯ ಕಾರ್ಯವನ್ನು ಮುಂದುವರಿಸುವಲ್ಲಿ, ಇಂಗ್ಲೆಂಡಿನಲ್ಲಿ ತನ್ನ ಅದೃಷ್ಟ ವನ್ನು ಪರೀಕ್ಷೆ ಮಾಡಿ ನೋಡುವುದಕ್ಕಾಗಿ ಹೊರಟಳು. ಹಿಂದಿನ ರಾತ್ರಿ ಸ್ವಾಮೀಜಿ ಆಕೆಯನ್ನು ಕರೆದು ನುಡಿದರು, “ನೋಡು, ಮುಸಲ್ಮಾನರಲ್ಲಿ ಒಂದು ಪಂಗಡವಿದೆ. ಅವರೆಷ್ಟು ಮತಾಂಧ ರೆಂದರೆ, ಅವರು ನವಜಾತ ಶಿಶುವನ್ನು ಕೈಯಲ್ಲೆತ್ತಿ ಹಿಡಿದು ಹೇಳುತ್ತಾರಂತೆ–‘ನಿನ್ನನ್ನು ಹುಟ್ಟಿಸಿದವನು ಮಹಮ್ಮದನಾದರೆ ನೀನು ಉಳಿದುಕೊ; ನಿನ್ನನ್ನು ಹುಟ್ಟಿಸಿದವನು ವಿಧರ್ಮೀಯ ದೇವರಾದರೆ ನಾಶವಾಗು!’ ಎಂದು. ಆ ಜನ ತಮ್ಮ ಮಕ್ಕಳಿಗೆ ಹೇಳುವುದನ್ನೇ, ಆದರೆ ವಿರುದ್ಧಾರ್ಥದಲ್ಲಿ ನಾನೀಗ ನಿನಗೆ ಹೇಳುತ್ತೇನೆ: ಹೋಗು, ಜಗತ್ತಿನಲ್ಲಿ ಧುಮುಕು. ನೀನು ನನ್ನ ಕೃತಿಯಾದರೆ ನಾಶವಾಗು; ಜಗನ್ಮಾತೆಯ ಕೃತಿಯಾದರೆ ಉಳಿದುಕೊ!”

ಮರುದಿನ ಬೆಳಿಗ್ಗೆ ಗಾಡಿಯಲ್ಲಿ ಕುಳಿತು ನಿವೇದಿತಾ ಹೊರಡುತ್ತಿದ್ದಂತೆ, ಸ್ವಾಮೀಜಿ ತಮ್ಮೆ ರಡೂ ಕೈಗಳನ್ನು ಮೇಲೆತ್ತಿ ಜೋಡಿಸಿ ಆಶೀರ್ವದಿಸುತ್ತ, ಜೊತೆಗೇ ನಮಸ್ಕರಿಸುತ್ತ ವಿದಾಯ ಹೇಳಿದರು.

ಮತ್ತೆ ಅವರು ತಮ್ಮ ಆಧ್ಯಾತ್ಮಿಕ ಪುತ್ರಿಯನ್ನು ಭೇಟಿ ಮಾಡಿದ್ದು ೧೯ಂ೨ರ ಮಾರ್ಚಿಯಲ್ಲೇ.

ಸೆಪ್ಟೆಂಬರಿನ ಕೊನೆಯ ವಾರದಲ್ಲಿ ಸ್ವಾಮೀಜಿಯವರು ಜೂಲ್ಸ್ ಬ್ಯೂಸ್ ಹಾಗೂ ಮಿಸ್ ಮೆಕ್​ಲಾಡರೊಂದಿಗೆ ಬ್ರಿಟಾನಿಯಿಂದ ಪ್ಯಾರಿಸಿಗೆ ಮರಳಿದರು. ಇಲ್ಲಿ ಅವರು ಮತ್ತೆ ಹಲವಾರು ಪ್ರತಿಭಾನ್ವಿತ ವ್ಯಕ್ತಿಗಳ ಸಂಪರ್ಕಕ್ಕೆ ಬಂದರು. ಪಿಯರ್ ಹಯಸಿಂಥ್ ಎಂಬವರು ಯೂರೋಪಿ ನಲ್ಲೆಲ್ಲ ಪ್ರಸಿದ್ಧರಾಗಿದ್ದ ಜನಪ್ರಿಯ ಪಾದ್ರಿಗಳು. ಆದರೆ ಇವರ ವಿನೂತನ ಕ್ರಾಂತಿಕಾರಕ ಬೋಧನೆಗಳು ವ್ಯಾಟಿಕನ್ನಿಗೆ ಸಹ್ಯವಾಗದ್ದರಿಂದ ತಮ್ಮ ಸ್ಥಾನಕ್ಕೆ ರಾಜೀನಾಮೆಯಿತ್ತಿದ್ದರು. ಆದರೂ ಇವರು ತಮ್ಮ ಹಿಂದಿನ ಬೋಧನೆಗಳನ್ನೇ ಮತ್ತಷ್ಟು ನಿರ್ಭಿಡೆಯಿಂದ ಪ್ರಸಾರ ಮಾಡಿ ದರು. ಇದರಿಂದ ಇವರು ಪ್ರಾಟೆಸ್ಟೆಂಟರ ಹಾಗೂ ಇತರ ವಿಶಾಲ ದೃಷ್ಟಿಯ ಕ್ರೈಸ್ತರ ಆದರಕ್ಕೆ ಪಾತ್ರರಾಗಿದ್ದರು. ಸ್ವಾಮೀಜಿಯವರು ಹಯಸಿಂಥರನ್ನು ಹಲವಾರುಬಾರಿ ಭೇಟಿಯಾದರು. ಇವರ ಮೃದು ಮಧುರ ವ್ಯಕ್ತಿತ್ವವನ್ನು ಬಹುವಾಗಿ ಮೆಚ್ಚಿಕೊಂಡರು. ಭಕ್ತಿಸಿದ್ಧಾಂತಗಳ ವಿಷಯವಾಗಿ ಇಬ್ಬರ ನಡುವೆ ಅನೇಕ ಬಾರಿ ಸಂಭಾಷಣೆ ನಡೆಯಿತು.

ನಾವು ಹಿಂದೆಯೇ ನೋಡಿದಂತೆ ಪ್ರೊ ॥ ಪ್ಯಾಟ್ರಿಕ್ ಗಿಡ್ಡಿಸ್ ಆ ಕಾಲದ ಖ್ಯಾತ ಸಮಾಜ ವಿಜ್ಞಾನಿ. ಇವರೊಂದಿಗೆ ಸ್ವಾಮೀಜಿ ಪ್ರಾಚೀನ ಗ್ರೀಕ್ ನಾಗರಿಕತೆ, ವಿವಿಧ ಜನಾಂಗಗಳ ಉದ್ಭವ, ಆಧುನಿಕ ಯೂರೋಪಿನ ಚರಿತ್ರೆ ಮೊದಲಾದ ಅಸಂಖ್ಯಾತ ವಿಚಾರಗಳ ಬಗ್ಗೆ ಚರ್ಚಿಸಿದರು. ಸರ್ ಹಿರಂ ಮ್ಯಾಕ್ಸಿಂ ಎಂಬವನು ಪ್ರಸಿದ್ಧ ವಿಜ್ಞಾನಿ; ಅಲ್ಲದೆ ಧರ್ಮ ಹಾಗೂ ತತ್ತ್ವಶಾಸ್ತ್ರಗಳಲ್ಲಿ ವಿಶಾಲವಾದ ಪಾಂಡಿತ್ಯವಿದ್ದವನು. ಗುಂಡುಗಳನ್ನು ನಿರಂತರವಾಗಿ ಮಳೆಗರೆಯಬಲ್ಲ ಮೆಷಿನ್ ಗನ್ನಿನ ಅನ್ವೇಷಕ ಈತನೇ. ಇವನು ಶಿಕಾಗೋ ಸಮ್ಮೇಳನದಲ್ಲೇ ಸ್ವಾಮೀಜಿಯವರನ್ನು ಕಂಡಿದ್ದ; ಅವರಿಂದ ಗಾಢವಾಗಿ ಪ್ರಭಾವಿತನಾಗಿದ್ದ. ಅವರೊಂದಿಗೆ ಮಾತನಾಡಬೇಕೆಂಬ ಇಚ್ಛೆ ಇವನಿಗೆ ಆಗಿನಿಂದಲೂ ಇತ್ತು. ಈಗ ಅದಕ್ಕೊಂದು ಅವಕಾಶ ದೊರೆಯಿತು. ಇವನು ತನ್ನ ಒಂದು ಪುಸ್ತಕದ ಮುನ್ನುಡಿಯಲ್ಲಿ ಸ್ವಾಮೀಜಿಯವರ ಬಗ್ಗೆ ಪ್ರಸ್ತಾಪಿಸಿ, ಅವರನ್ನು ಹಾರ್ದಿಕವಾಗಿ ಕೊಂಡಾಡಿದ್ದಾನೆ. (ನೋಡಿ: “ವಿಶ್ವವಿಜೇತ ವಿವೇಕಾನಂದ” ಪುಟ ೨೫೨)

ಮೇಡಮ್ ಸಾರಾ ಬರ್ನ್ ಹಾರ್ಟ್ ಎಂಬವಳು ಇಡೀ ಯೂರೋಪು-ಅಮೆರಿಕಗಳಲ್ಲೇ ಅತ್ಯಂತ ಪ್ರಸಿದ್ಧಳಾದ ರಂಗನಟಿ. ಇವಳೂ ಹಿಂದೆಯೇ ನ್ಯೂಯಾರ್ಕಿನಲ್ಲಿ ಸ್ವಾಮೀಜಿಯವರನ್ನು ಭೇಟಿಯಾಗಿದ್ದಳು. ಭಾರತದ ಬಗ್ಗೆ ಈಕೆಗೆ ವಿಶೇಷ ಆಸಕ್ತಿ, ಅಭಿಮಾನ. ಈ ದಿನಗಳಲ್ಲೇ ಸ್ವಾಮೀಜಿ ಇವಳು ಅಭಿನಯಿಸಿದ್ದ ಒಂದು ಪ್ರಸಿದ್ಧ ಚಾರಿತ್ರಿಕ ನಾಟಕವನ್ನು ನೋಡಿ ಆನಂದಿಸಿದರು.

ಪ್ಯಾರಿಸಿನಲ್ಲಿ ಸ್ವಾಮೀಜಿಯವರು ತಮ್ಮ ಆತ್ಮೀಯ ಪರಿಚಯಸ್ಥಳಾದ ಮೇಡಂ ಎಮ್ಮಾ ಕಾಲ್ವೆಯನ್ನು ಭೇಟಿಯಾದರು. ಇವಳು ಪಶ್ಚಿಮ ದೇಶಗಳಲ್ಲೆಲ್ಲ ವಿಖ್ಯಾತಳಾದ ವಿಶಿಷ್ಟ ಗಾಯಕಿ- ನಟಿ; ಅತ್ಯಂತ ಪ್ರತಿಭಾನ್ವಿತೆ. ಹಿಂದೆ ಶಿಕಾಗೋದಲ್ಲಿ ಇವಳು ಸ್ವಾಮೀಜಿಯವರ ಸಂಪರ್ಕಕ್ಕೆ ಬಂದು ಅವರಿಂದ ಬಹಳಷ್ಟು ಪ್ರಭಾವಿತಳಾದ ಕಥೆಯನ್ನು ಅದಾಗಲೇ ನೋಡಿದ್ದೇವೆ.

ಇವರೆಲ್ಲರ ಸಹವಾಸದ ನಡುವೆಯೂ ಮತ್ತು ಈ ಎಲ್ಲ ಕಾರ್ಯಕ್ರಮಗಳಲ್ಲಿ ಭಾಗವಹಿಸು ತ್ತಲೇ ಸ್ವಾಮೀಜಿಯವರು ಆಂತರ್ಯದಲ್ಲಿ ಅವುಗಳೆಲ್ಲದರಿಂದಲೂ ನಿರ್ಲಿಪ್ತರಾಗಿದ್ದರು. ಹೆಚ್ಚು ಹೆಚ್ಚು ಸಮಯವನ್ನು ಅವರು ಅಧ್ಯಯನ ಹಾಗೂ ಧ್ಯಾನದಲ್ಲಿ ಕಳೆಯಲಾರಂಭಿಸಿದ್ದರು. ಧನಾರ್ಜನೆಗಾಗಲಿ ಧರ್ಮಪ್ರಚಾರಕ್ಕಾಗಲಿ ತಾವಿನ್ನು ಉಪನ್ಯಾಸಗಳನ್ನು ಮಾಡಬೇಕಾದ ಆವಶ್ಯ ಕತೆಯಿಲ್ಲ ಎಂದು ಅವರಿಗೆ ಅನ್ನಿಸತೊಡಗಿತ್ತು. ಚಂಡಮಾರುತದಂತೆ ಅವರ ಅಂತರಂಗದಲ್ಲಿ ಬೀಸುತ್ತಿದ್ದ ಭಾವಾವೇಶದ ರಭಸವು ನಿಧಾನವಾಗಿ ಶಾಂತವಾಗುತ್ತ ಬಂದಿತ್ತು. ಸ್ವಾಮೀಜಿಯವರ ಭಾವಗಳನ್ನು ಅರಿತುಕೊಳ್ಳುವಲ್ಲಿ ಎಲ್ಲರಿಗಿಂತ ಸೂಕ್ಷ್ಮಗ್ರಾಹಿಯಾಗಿದ್ದ ನಿವೇದಿತೆ ಬರೆಯುತ್ತಾಳೆ:

“ಅಮೆರಿಕದ ಹಾಗೂ ಯುರೋಪಿನ ಯಾತ್ರೆಯ ಈ ದಿನಗಳಲ್ಲೆಲ್ಲ ಸ್ವಾಮೀಜಿಯವರ ನಿಲುವಿ ನಲ್ಲಿ ಎದ್ದುಕಾಣುವಂತಿದ್ದ ಒಂದು ಅಂಶವೆಂದರೆ ಸುತ್ತಮುತ್ತಲಿನ ಎಲ್ಲ ವಿಷಯಗಳ ಬಗ್ಗೆ ಅವರಿಗಿದ್ದ ಸಂಪೂರ್ಣ ಅನಾಸಕ್ತಿ, ನಿರ್ಲಿಪ್ತತೆ. ತಮ್ಮ ಮೂಲಕ ಕೆಲಸ ಮಾಡುತ್ತಿದ್ದ ಶಕ್ತಿಯ ಅಗಾಧತೆಯನ್ನು ಮನಗಂಡಿದ್ದ ಅವರು ತಮಗೆ ಬಂದ ಯಶಸ್ಸನ್ನು ಕಂಡು ಯಾವ ರೀತಿಯಲ್ಲೂ ಹಿಗ್ಗುತ್ತಿರಲಿಲ್ಲ. ಅಥವಾ ಅಪಯಶಸ್ಸಿನಂತೆ ತೋರಿದ ಘಟನೆಗಳಿಂದ ಕುಗ್ಗುತ್ತಿರಲೂ ಇಲ್ಲ. ಇಲ್ಲವೆ ಅಧೈರ್ಯ ತಾಳುತ್ತಲೂ ಇರಲಿಲ್ಲ. ಜಯಾಪಜಯಗಳಿಗೆ ಅವರು ಸಾಕ್ಷಿಯಂತಿದ್ದು ಬಿಟ್ಟರು. ಅವರೊಮ್ಮೆ ಹೇಳಿದರು: ‘ಈ ಪ್ರಪಂಚವೇ ಕಣ್ಮರೆಯಾದರೂ ಅದರಿಂದ ನನಗೇ ನಂತೆ? ನನ್ನ ತತ್ತ್ವದ ಪ್ರಕಾರ–ನಿನಗೆ ಗೊತ್ತಿರುವಂತೆ–ಹಾಗೇನಾದರೂ ಆದರೆ ಒಳ್ಳೆಯದೇ!’ ಬಳಿಕ ಇದ್ದಕ್ಕಿದ್ದಂತೆ ಗಂಭೀರ ದನಿಯಲ್ಲಿ ನುಡಿದರು: ‘ಆದರೆ, ಈಗ ನನಗೆ ವಿರುದ್ಧವಾಗಿರುವು ದೆಲ್ಲವೂ ಕಡೆಯಲ್ಲಿ ನನ್ನ ಪರವಾಗಿಯೇ ಆಗಬೇಕು. ನಾನು ‘ಆಕೆ’ಯ ಸೈನಿಕನಲ್ಲವೆ.’”

ಹೀಗೆ ಅನೇಕ ಕಾರ್ಯಕಲಾಪಗಳಿಂದ ಪ್ರಕ್ಷುಬ್ಧಗೊಂಡ ಸ್ವಾಮೀಜಿಯವರ ಮನಸ್ಸು ಕ್ರಮೇಣ ಶಾಂತವಾಗುತ್ತ ಬಂದು ಪ್ಯಾರಿಸಿನಲ್ಲಿ ಅದು, ನೀಲಾಕಾಶವನ್ನು ಸ್ಫುಟವಾಗಿ ಪ್ರತಿಬಿಂಬಿಸುವ ಪ್ರಶಾಂತ ಸರೋವರದಂತಾಗಿತ್ತು. ತುರೀಯಾನಂದರಿಗೆ ಅವರು ಹೇಳಿದ್ದರು: “ಮೇಲಿನಿಂದ ಕರೆ ಬಂದಿದೆ–‘ಬಂದುಬಿಡು, ಸುಮ್ಮನೆ ಬಂದುಬಿಡು. ಇತರರಿಗೆ ಉಪದೇಶಿಸುವುದಕ್ಕಾಗಿ ನಿನ್ನ ತಲೆಯನ್ನು ಕೆಡಿಸಿಕೊಳ್ಳುವ ಅಗತ್ಯವಿಲ್ಲ’ ಅಂತ. ಈಗ ಆಟ ಮುಗಿದುಹೋಗಬೇಕೆಂದು ‘ಅಜ್ಜಿ’ಯ ಇಚ್ಛೆ.” (ಕಣ್ಣಾಮುಚ್ಚಾಲೆ ಆಟದಲ್ಲಿ ಅಜ್ಜಿಯನ್ನು ಮುಟ್ಟಿದವನ ಆಟ ಮುಗಿದು ಹೋಗುತ್ತದೆ.)

ಅಕ್ಟೋಬರ್ ೧೪ರಂದು ಸ್ವಾಮೀಜಿ ಕ್ರಿಸ್ಟೀನಳಿಗೆ ಫ್ರೆಂಚ್ ಭಾಷೆಯಲ್ಲಿ ಪತ್ರವೊಂದನ್ನು ಬರೆದರು: “ನಾನೀಗ ಎಲ್ಲ ಹೊಣೆಗಾರಿಕೆಯಿಂದ ದೂರವಾದೆ. ನಾನೀಗ ಸ್ವತಂತ್ರ! ಅರುಣೋ ದಯವಾಗುತ್ತಿದ್ದಂತೆಯೇ ಮರದ ಕೊಂಬೆಗಳ ಮೇಲೆ ಮಲಗಿದ್ದ ಪಕ್ಷಿಗಳೆಲ್ಲ ಎದ್ದು ಹಾಡುತ್ತ ನೀಲಾಕಾಶದೆಡೆಗೆ ಹಾರಿಹೋಗುವಂತೆ ನಾನೂ ಈಗ ಚಿದಾಕಾಶದೆಡೆಗೆ ಹಾರಿಹೋಗುತ್ತಿದ್ದೇನೆ.

“ನಾನು ಹಲವಾರು ಕಷ್ಟಗಳನ್ನು ಪಟ್ಟಿದ್ದೇನೆ. ಅನೇಕ ಅದ್ಭುತ ಯಶಸ್ಸುಗಳನ್ನು ಕಂಡಿದ್ದೇನೆ. ಆದರೆ ನಾನೀಗ ಯಶಸ್ವಿಯಾಗಿರುವುದರಿಂದ ಹಿಂದೆ ಪಟ್ಟ ಕಷ್ಟಗಳೂ ಯಾತನೆಗಳೂ ಈಗ ಲೆಕ್ಕಕ್ಕೆ ಬರುವುದಿಲ್ಲ. ನಾನು ನನ್ನ ಗುರಿಯನ್ನು ಮುಟ್ಟಿದ್ದೇನೆ. ಯಾವ ಮುತ್ತಿಗಾಗಿ ಜೀವನಸಾಗರದೊಳಗೆ ಧುಮುಕಿದೆನೊ ಅದನ್ನು ಪಡೆದುಕೊಂಡಿದ್ದೇನೆ. ನಾನೀಗ ಸಂತೃಪ್ತನಾಗಿದ್ದೇನೆ.

“ಈಗ ನನ್ನ ಜೀವನದಲ್ಲಿ ಹೊಸ ಅಧ್ಯಾಯವೊಂದು ಪ್ರಾರಂಭವಾಗುತ್ತಿದೆ ಎಂದು ಅನ್ನಿಸು ತ್ತಿದೆ. ಜಗನ್ಮಾತೆ ನನ್ನನ್ನು ಕೈಹಿಡಿದು ನಿಧಾನವಾಗಿ ನಡೆಸುತ್ತಿದ್ದಾಳೆಂದು ಅನ್ನಿಸುತ್ತಿದೆ. ಅಡಚಣೆ ಗಳಿಂದ ತುಂಬಿದ ದಾರಿಯಲ್ಲಿ ಹೋರಾಡುತ್ತ ಮುಂದುವರಿಯುವುದೆಲ್ಲ ಆಗಿಹೋಯಿತು. ಇನ್ನು ಮುಂದೆ ಕೇವಲ ಹಂಸತೂಲಿಕಾತಲ್ಪ! ನಿನಗಿದು ಅರ್ಥವಾಗುತ್ತದೆಯೇ? ನನ್ನನ್ನು ನಂಬು. ನಾನಂತೂ ಹಾಗೆಯೇ ಭಾವಿಸುತ್ತೇನೆ.”

ಇಂತಹ ಮುಕ್ತಾತ್ಮರಾದ ಸ್ವಾಮೀಜಿಯವರನ್ನು ಲೋಕಕ್ಕೆ ಕಟ್ಟಿಹಾಕಿದ ಸರಪಣಿ ಯಾವುದಿದ್ದಿರಬಹುದು? ಅವರು ಸ್ಥಾಪಿಸಿದ ಸಂಘವೆ? ಅಥವಾ ಆ ಸಂಘದ ಅಧ್ಯಕ್ಷಸ್ಥಾನವೆ? ಅಲ್ಲ. ಅದಾವುದೂ ಅಲ್ಲ. ಅವರನ್ನು ಈ ಲೋಕಕ್ಕೆ ಬಂಧಿಸಿದ್ದುದು ಅವರ ಅನುಕಂಪೆ; ವ್ಯಕ್ತಿವ್ಯಕ್ತಿಗಳ ಮೇಲೆ, ಸಮಸ್ತ ಮಾನವಜನಾಂಗದ ಮೇಲೆ ಅವರಿಗಿದ್ದ ಅಹೈತುಕ ಅನುಕಂಪೆ.

