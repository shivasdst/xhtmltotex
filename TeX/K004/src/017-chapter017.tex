
\chapter{ಝೇಲಮ್ಮಿನ ಜಲದ ಮೇಲೆ}

\noindent

ಆಲ್ಮೋರದಲ್ಲಿ ಸ್ವಾಮೀಜಿ ತಮ್ಮ ಶಿಷ್ಯೆಯರ ತರಬೇತಿಯಲ್ಲಿ ನಿರತರಾಗಿದ್ದರು, ಬಂದವ ರೊಡನೆ ಮಾತನಾಡಿದರು ಎಂಬುದೇನೋ ನಿಜವೆ. ಆದರೆ ತಮ್ಮ ನಿಷ್ಠಾವಂತ ಶಿಷ್ಯನಾದ ಗುಡ್​ವಿನ್ನನ ಮರಣದ ದುಃಖವನ್ನು ಮಾತ್ರ ಮರೆಯಲು ಅವರಿಗೆ ಸಾಧ್ಯವಾಗಲಿಲ್ಲ. ಆ ಸುದ್ದಿ ತಲುಪಿದ ಸ್ಥಳದಲ್ಲಿ ಇರುವುದು ಅವರಿಗೆ ಅಸಹನೀಯವಾಗಿತ್ತು. ಆದ್ದರಿಂದ ಅವರು ಗುಡ್ ವಿನ್ನನ ನೆನಪು ಮರುಕಳಿಸುವಂತೆ ಮಾಡುತ್ತಿದ್ದ ಆಲ್ಮೋರದಿಂದ ಹೊರಟುಬಿಡಲು ನಿಶ್ಚಯಿ ಸಿದರು. ಆದರೆ ಅವರೀಗ ಕಲ್ಕತ್ತಕ್ಕೆ ಹಿಂದಿರುಗುವಂತಿರಲಿಲ್ಲ. ಭಾರತದ ಸಂದರ್ಶನಕ್ಕಾಗಿ ಬಂದಿದ್ದ ಅವರ ಪಾಶ್ಚಾತ್ಯ ಶಿಷ್ಯೆಯರು ಈಗ ಕಾಶ್ಮೀರ ಪ್ರವಾಸಕ್ಕೆ ಹೊರಟಿದ್ದರು. ಅವರಿಗೆ ತಾವು ಜೊತೆಗೊಡುವುದಾಗಿ ಸ್ವಾಮೀಜಿ ಮಾತುಕೊಟ್ಟಿದ್ದರು. ಅದರಂತೆ ಅವರು ಜೂನ್ ೧೧ರಂದು ತಮ್ಮ ಶಿಷ್ಯೆಯರೊಂದಿಗೆ ಕಾಶ್ಮೀರದತ್ತ ಹೊರಟರು.

ಪರ್ವತ ಪ್ರದೇಶದಿಂದ ಬಯಲುಸೀಮೆಗೆ ಬರುವ ದಾರಿಯಲ್ಲಿ ದಟ್ಟವಾದ ಹಸಿರು ಅರಣ್ಯ ಗಳಿವೆ. ಇವುಗಳ ಮೂಲಕ ಹಾದುಹೋಗುವುದೊಂದು ಸುಂದರ ಅನುಭವ. ಎರಡೂವರೆ ದಿನಗಳ ಸುಖಕರ ಪ್ರಯಾಣದ ಬಳಿಕ ಪ್ರವಾಸಿಗರ ತಂಡ ಈಗ ಕಥಗೋಡಂ ತಲುಪಿತು. ಹಿಮಾಲಯದ ದೇವದಾರು-ಪೀತದಾರು ವೃಕ್ಷಗಳು, ಪರ್ವತಶಿಖರಗಳೆಲ್ಲ ಮಾಯವಾದುವು. ಬಳಿಕ ಅವರು ಲೂಧಿಯಾನ, ಲಾಹೋರ್​ಗಳ ಮೂಲಕ ರೈಲಿನಲ್ಲಿ ಪ್ರಯಾಣ ಮಾಡಿ ರಾವಲ್ಪಿಂಡಿಗೆ ಬಂದರು. ಮತ್ತೆ ಇಲ್ಲಿಂದ ಮುರ್ರೀ ಎಂಬಲ್ಲಿಯವರೆಗೆ ಟಾಂಗಾಗಳಲ್ಲಿ ಪ್ರಯಾಣ ಮಾಡಿ, ಅಲ್ಲಿ ಮೂರು ದಿನ ಉಳಿದುಕೊಂಡರು. ದಾರಿಯುದ್ದಕ್ಕೂ ನಿರಂತರ ಮಾತುಕತೆ, ಶಿಕ್ಷಣ. ಒಂದೇ ಟಾಂಗಾದಲ್ಲಿ ಎಲ್ಲರೂ ಕುಳಿತುಕೊಳ್ಳಲು ಸಾಧ್ಯವಿಲ್ಲದುದರಿಂದ, ಶಿಷ್ಯೆಯರು ಸರದಿಯ ಪ್ರಕಾರ ಸ್ವಾಮೀಜಿ ಯವರೊಂದಿಗೆ ಪ್ರಯಾಣ ಮಾಡುತ್ತಿದ್ದರು. ತನ್ಮೂಲಕ ಎಲ್ಲರಿಗೂ ಅವರ ಮಾತುಗಳನ್ನು ಕೇಳುವ ಸೌಲಭ್ಯ-ಸೌಭಾಗ್ಯ! ಅವರ ಈ ಅಮೂಲ್ಯ ಮಾತುಕತೆಗಳಲ್ಲಿ ಕೆಲವನ್ನು ನಾವು ಸೋದರಿ ನಿವೇದಿತೆಯ ಗ್ರಂಥಗಳಲ್ಲಿ ನೋಡಬಹುದು.

ಪ್ರಯಾಣಿಕರ ತಂಡ ಈಗ ಶ್ರೀನಗರಕ್ಕೆ ಹೋಗುವ ದಾರಿಯಲ್ಲಿರುವ ಬರಾಮುಲ್ಲಾ ಎಂಬ ಲ್ಲಿಗೆ ಸಾಗಿತು. ಮತ್ತೆ ಟಾಂಗಾಗಳಲ್ಲಿ ಪ್ರಯಾಣ. ಅತ್ಯಂತ ಕಡಿದಾದ ದಾರಿ. ನದಿಯ ದಡದ ಮೇಲೆ, ಪರ್ವತಗಳನ್ನು ಬಳಸಿಕೊಂಡು, ಅಗಲ ಕಿರಿದಾದ ರಸ್ತೆಗಳಲ್ಲಿ ಸಾಗಬೇಕು. ಪ್ರಯಾಣದ ಸಂದರ್ಭದಲ್ಲಿ ಸ್ವಾಮೀಜಿ ತಮ್ಮ ಅಪೂರ್ವ ಸಂಭಾಷಣಾಸಾಮರ್ಥ್ಯದಿಂದ ತಮ್ಮ ಸಂಗಡಿಗರ ಮನಸ್ಸನ್ನು ಯಾವಾಗಲೂ ಹರ್ಷಪೂರ್ಣವಾಗಿಡುತ್ತಿದ್ದರು. ಈ ದಿನಗಳಲ್ಲೇ ಒಮ್ಮೆ ಅವರು ಸಾಂದರ್ಭಿಕವಾಗಿ ಒಂದು ಪುಟ್ಟ ಘಟನೆಯನ್ನು ತಿಳಿಸುತ್ತಾರೆ:

“ನನಗೊಬ್ಬ ಬಾಲ್ಯಸ್ನೇಹಿತನಿದ್ದ–ಅವನು ನನ್ನ ಸಹಪಾಠಿಯೂ ಕೂಡ. ಮುಂದೆ ಅವನು ಭಾರೀ ಶ್ರೀಮಂತನಾದ. ಆದರೆ ಅವನು ಸಂಪೂರ್ಣ ಆರೋಗ್ಯ ಕಳೆದುಕೊಂಡು ಹಾಸಿಗೆ ಹಿಡಿದ. ಆ ಕಾಯಿಲೆ ಯಾವ ಚಿಕಿತ್ಸೆಗೂ ಜಗ್ಗದಿರುವುದನ್ನು ಕಂಡು ಡಾಕ್ಟರುಗಳೆಲ್ಲ ಕೈಚೆಲ್ಲಿ ಕುಳಿತರು. ಇನ್ನು ಅವನಿಗೂ ತನ್ನ ಕಾಯಿಲೆ ವಾಸಿಯಾಗುವುದೆಂಬ ಆಸೆ ಉಳಿಯಲಿಲ್ಲ; ಜೀವನದಲ್ಲಿ ಉತ್ಸಾಹವೇ ಬತ್ತಿಹೋಯಿತು. ಆದ್ದರಿಂದ ಅವನು ಧರ್ಮದ ಕಡೆಗೆ ತಿರುಗಿಕೊಂಡ. ಈ ಸಂದರ್ಭದಲ್ಲಿ ಅವನಿಗೆ, ನಾನು ಸಂನ್ಯಾಸಿಯಾಗಿದ್ದೇನೆ, ಯೋಗಶಕ್ತಿ ಪಡೆದಿದ್ದೇನೆ ಎನ್ನುವ ವಿಷಯ ಹೇಗೋ ಕಿವಿಗೆ ಬಿತ್ತು. ಸರಿ, ನನಗೆ ಹೇಳಿಕಳಿಸಿದ. ಒಂದು ಸಲವಾದರೂ ಬಂದು ಹೋಗಲೇಬೇಕು ಎಂದು ಬೇಡಿಕೊಂಡ. ನಾನು ಹೋದೆ; ಅವನ ಹಾಸಿಗೆಯ ಬಳಿ ಕುಳಿತು ಕೊಂಡೆ. ಆಗ ಇದ್ದಕ್ಕಿದ್ದಂತೆ ನನಗೆ ಉಪನಿಷತ್ತಿನ ಒಂದು ಮಾತು ನೆನಪಾಯಿತು: ‘ಯಾರು ತನ್ನನ್ನು ತಾನು ಬ್ರಾಹ್ಮಣನಿಗಿಂತ ಪ್ರತ್ಯೇಕ ಎಂದು ತಿಳಿಯುವನೊ ಅವನನ್ನು ಬ್ರಾಹ್ಮಣನು ಗೆಲ್ಲುವನು; ಯಾರು ತನ್ನನ್ನು ತಾನು ಕ್ಷತ್ರಿಯನಿಗಿಂತ ಪ್ರತ್ಯೇಕವೆಂದು ತಿಳಿಯುವನೊ ಅವನನ್ನು ಕ್ಷತ್ರಿಯನು ಗೆಲ್ಲುವನು; ಯಾರು ತನ್ನನ್ನು ತಾನು ಜಗತ್ತಿಗಿಂತ ಪ್ರತ್ಯೇಕವೆಂದು ತಿಳಿಯುವನೊ ಅವನನ್ನು ಜಗತ್ತು ಗೆಲ್ಲುವುದು.’ ನಾನು ಈ ಶ್ಲೋಕವನ್ನು ಪಠಿಸಿದ ಕೂಡಲೇ ಆಶ್ಚರ್ಯಕರವಾದ ರೀತಿಯಲ್ಲಿ ಅದು ಆ ಮನುಷ್ಯನ ಮೇಲೆ ಪರಿಣಾಮ ಬೀರಿತು. ಅವನು ತಕ್ಷಣವೇ ಆ ಶ್ಲೋಕದ ಅರ್ಥವನ್ನು ಗ್ರಹಿಸಿದ; ಗುಣ ಹೊಂದುತ್ತ ಬಂದ! ಅವನ ಶರೀರಕ್ಕೆ ಹೊಸ ಶಕ್ತಿ ಬಂದಿತು, ಕಾಯಿಲೆ ಬೇಗ ವಾಸಿಯಾಗುವಂತಾಯಿತು.

“ಹೀಗೆ ನಾನು ಒಂದೊಂದು ಸಲ ಏನಾದರೂ ವಿಚಿತ್ರವಾದ ಮಾತನ್ನು ಹೇಳಿಬಿಡುತ್ತೇನೆ, ಇಲ್ಲವೆ ಕೋಪದ ಮಾತುಗಳನ್ನು ಆಡಿಬಿಡುತ್ತೇನೆ. ಆದರೆ ತಿಳಿದಿರಿ, ನನ್ನ ಹೃದಯದಲ್ಲಿ ಮಾತ್ರ ಸ್ವಲ್ಪವೂ ಕೋಪವಿಲ್ಲ. ಅಲ್ಲಿರುವುದೆಲ್ಲ ಬರಿಯ ಪ್ರೀತಿ ಮಾತ್ರ. ನಾನು ಬೋಧಿಸುವುದು ಪ್ರೀತಿಯನ್ನು ಮಾತ್ರ. ನಮ್ಮಲ್ಲಿ ಪರಸ್ಪರ ಪ್ರೀತಿಯೊಂದು ಇದ್ದುಬಿಟ್ಟರೆ ಎಲ್ಲ ತೊಂದರೆಗಳೂ ತಾವಾಗಿಯೇ ಸರಿಹೋಗುತ್ತವೆ.”

ಈಗ ಸ್ವಾಮೀಜಿಯವರ ತಂಡದವರು ಕಾಶ್ಮೀರವನ್ನು ಪ್ರವೇಶಿಸಿದರು. ಪ್ರಕೃತಿ ಸೌಂದರ್ಯದ ತವರೂರು ಈ ಕಾಶ್ಮೀರ. ಯಾವ ಸ್ಥಳಕ್ಕೆ ಹೋದರೂ ಅಲ್ಲಿನವರ ಜೊತೆಗೆ ಹೊಂದಿಕೊಳ್ಳುವುದು ಸ್ವಾಮೀಜಿಯವರದೊಂದು ವೈಶಿಷ್ಟ್ಯ. ಅಲ್ಲಿನ ಆಹಾರ ವಿಹಾರ ನಡೆನುಡಿಗಳನ್ನು ಅನುಸರಿಸು ವಲ್ಲಿ ಅವರಿಗೊಂದು ಸಂತೋಷ. ಸ್ವಾಮೀಜಿ ತಮ್ಮೊಂದಿಗೆ ಪರಿಚಾರಕರನ್ನು ಕರೆತಂದಿರಲಿಲ್ಲ ವಾದ್ದರಿಂದ ತಮ್ಮ ಪ್ರತಿಯೊಂದು ಸಣ್ಣಪುಟ್ಟ ಕೆಲಸವನ್ನೂ ಅವರು ತಾವೇ ಮಾಡಿಕೊಳ್ಳ ಬೇಕಾಗಿತ್ತು. ಅಲ್ಲದೆ ಗಾಡಿಯ ವ್ಯವಸ್ಥೆಯನ್ನೋ ನದಿಯಲ್ಲಿ ಪಯಣಿಸಲು ದೋಣಿಮನೆಯ ವ್ಯವಸ್ಥೆಯನ್ನೋ ಮಾಡಬೇಕೆಂದರೆ ತಾವೇ ಓಡಾಡಬೇಕಾಗಿತ್ತು. ಆದರೆ ಅವರ ಅದೃಷ್ಟಕ್ಕೆ ಬರಾಮುಲ್ಲಾದಲ್ಲಿ ಅವರಿಗೊಬ್ಬ ಮನುಷ್ಯ ಸಿಕ್ಕಿದ. ಅವನು ಸ್ವಾಮೀಜಿಯವರ ಹೆಸರನ್ನು ಕೇಳು ತ್ತಲೇ ಅವರ ಪ್ರಯಾಣದ ಎಲ್ಲ ವ್ಯವಸ್ಥೆಗಳನ್ನೂ ತಾನೇ ವಹಿಸಿಕೊಳ್ಳಲು ಮುಂದಾದ. ಈಗ ಅವರಿಗೆ ಎಷ್ಟೋ ಹಗುರವೆನಿಸಿತು.

ಸಂಜೆ ನಾಲ್ಕು ಗಂಟೆಯ ಹೊತ್ತಿಗೆ ಎಲ್ಲರೂ ಮೂರು ದೋಣಿಮನೆಗಳಲ್ಲಿ ಕುಳಿತು ಶ್ರೀನಗರದ ಕಡೆಗೆ ಹೊರಟರು. ಮರುದಿನ ಒಂದು ಅತ್ಯಂತ ಮನೋಹರವಾದ ಕಣಿವೆಯ ಪ್ರದೇಶವನ್ನು ತಲುಪಿದರು. ಅಲ್ಲಿನ ಬೆಟ್ಟಗಳ ಶಿಖರಗಳು ಹಿಮಾವೃತವಾಗಿದ್ದು ಒಂದು ಅದ್ಭುತ ದೃಶ್ಯವನ್ನೇ ನಿರ್ಮಿಸಿದ್ದುವು. ಕೆಲವೊಮ್ಮೆ ದೋಣಿಗಳು ಕಮಲದ ಎಲೆಗಳ ನಡುವೆ ಸಾಗು ತ್ತಿದ್ದುವು. ಎಲೆಗಳ ಮಧ್ಯೆ ಅಲ್ಲಲ್ಲಿ ಕಮಲಗಳು ಕಂಗೊಳಿಸುತ್ತಿದ್ದುವು. ನದಿಯ ಇಕ್ಕೆಲಗಳಲ್ಲೂ ವಿಶಾಲವಾಗಿ ಹರಡಿಕೊಂಡಿರುವ ಮೈದಾನಗಳು ಮನಸೆಳೆಯುವಂತಿದ್ದುವು. ಸೋದರಿ ನಿವೇದಿತಾ ಹೇಳುವಂತೆ, ಆಗಸದ ನೀಲವರ್ಣ, ವನಸ್ಪತಿಗಳ ಹರಿದ್ವರ್ಣ, ಹಿಮರಾಶಿಯ ಶ್ವೇತವರ್ಣಗಳು ಸಮರಸವಾಗಿ ಸೇರಿ ಸೃಷ್ಟಿಸೌಂದರ್ಯ ಸಾಸಿರಮಡಿಯಾಗಿತ್ತು.

ಮರುದಿನ ದೋಣಿಗಳು ಝೇಲಂ ನದಿಯ ಮೇಲ್ಭಾಗದಲ್ಲಿನ ಒಂದು ಹಳ್ಳಿಯ ಬಳಿಗೆ ತಲುಪಿದುವು. ಅಲ್ಲಿ ಸ್ವಾಮೀಜಿ ತಮ್ಮ ಸಂಗಡಿಗರನ್ನು ಹಸಿರು ಹೊಲಗಳಲ್ಲಿ ತಿರುಗಾಡಿಕೊಂಡು ಬರಲು ಕರೆದೊಯ್ದರು. ಅಲ್ಲಿನ ಒಬ್ಬಳು ವೃದ್ಧೆಯನ್ನು ತಮ್ಮ ಸಂಗಡಿಗರಿಗೆ ಪರಿಚಯಿಸಿಕೊಡು ವುದು ಅವರ ಉದ್ದೇಶ. ಆ ವೃದ್ಧೆಯ ಶ್ರದ್ಧೆ ಆತ್ಮವಿಶ್ವಾಸಗಳ ಬಗ್ಗೆ ಅವರು ತಮ್ಮ ಶಿಷ್ಯರಿಗೆ ತಿಳಿಸಿದ್ದರಲ್ಲದೆ, ತಮ್ಮ ಒಂದು ಭಾಷಣದಲ್ಲೂ ಅದನ್ನು ಪ್ರಸ್ತಾಪಿಸಿದ್ದರು. ಹಿಂದಿನ ವರ್ಷ ಅವರು ಇಲ್ಲಿಗೆ ಬಂದಿದ್ದಾಗ ಬಾಯಾರಿಕೆಯನ್ನು ತಣಿಸಿಕೊಳ್ಳಲು ಇವಳ ಮನೆಗೆ ಬಂದಿದ್ದರು. ನೀರನ್ನು ಕುಡಿದ ಬಳಿಕ ಸ್ವಾಮೀಜಿ ಆಕೆಯನ್ನು “ತಾಯಿ, ಅಂದ ಹಾಗೆ ನೀನು ಯಾವ ಮತಸ್ಥಳು?” ಎಂದು ಕೇಳಿದ್ದರು. ಅದಕ್ಕೆ ಆ ಹೆಂಗಸು ಹೆಮ್ಮೆ ತುಂಬಿದ ದನಿಯಲ್ಲಿ ಉತ್ತರಿಸಿದ್ದಳು, “ಸ್ವಾಮಿ, ಅಲ್ಲಾನ ದಯದಿಂದ ನಾನೊಬ್ಬ ಮುಸಲ್ಮಾನಳು!” ಈ ಸಲವೂ ಅವಳು ಸ್ವಾಮೀಜಿ ಹಾಗೂ ಅವರ ಸಂಗಡಿಗರನ್ನು ತುಂಬ ವಿಶ್ವಾಸದಿಂದ ಸತ್ಕರಿಸಿದಳು. ಹೃದಯದ ಔದಾರ್ಯವೊಂದಿದ್ದರೆ ಯಾವ ಜಾತಿಯಾದರೇನು, ಯಾವ ಮತವಾದರೇನು?

ಪ್ರಯಾಣ ಮುಂದುವರಿಯಿತು. ಎರಡುಮೂರು ದಿನಗಳ ಬಳಿಕ ಎಲ್ಲರೂ ಕಾಶ್ಮೀರದ ರಾಜಧಾನಿಯಾದ ಶ್ರೀನಗರಕ್ಕೆ ಬಂದು ತಲುಪಿದರು. ಇಲ್ಲಿ ಅವರಿಗಾಗಿ ಪತ್ರಗಳ ಕಂತೆಯೇ ಕಾದಿತ್ತು. ಮೊದಲು ನಗರದರ್ಶನ ಮಾಡಿ ಬಂದು ಬಳಿಕ ಉಳಿದವುಗಳ ಕಡೆ ಗಮನ ಕೊಡು ವುದೆಂದು ನಿರ್ಧರಿಸಿದರು. ಜುಲೈ ೧೫ರವರೆಗೆ ಅವರೆಲ್ಲ ಆ ಝೇಲಂ ನದಿಯ ಮೇಲೆ ದೋಣಿಮನೆಗಳಲ್ಲೇ ವಾಸವಾಗಿದ್ದು ಶ್ರೀನಗರದ ಆಸುಪಾಸಿನ ಸ್ಥಳಗಳನ್ನೆಲ್ಲ ನೋಡಿಕೊಂಡು ಬಂದರು. ಆ ಶಿಷ್ಯೆಯರ ಪಾಲಿಗೆ ಅದೊಂದು ಅತ್ಯಮೂಲ್ಯ ಅನುಭವ. ಆ ದಿನಗಳಲ್ಲಿ ಸ್ವಾಮೀಜಿ ಮಾತನಾಡದ ವಿಷಯವೇ ಇಲ್ಲವೇನೋ ಎಂಬಂತಿತ್ತು. ಕೆಲವೊಮ್ಮೆ ಅವರು ಕಾನಿಷ್ಕನ ಕಾಲದ ಕಾಶ್ಮೀರದ ಬಗ್ಗೆ ಹೇಳಿದರೆ ಮತ್ತೊಮ್ಮೆ ಬೌದ್ಧಧರ್ಮವು ಬೋಧಿಸುವಂತಹ ನೈತಿಕ ಮೌಲ್ಯಗಳ ಬಗ್ಗೆ, ಇಲ್ಲವೆ ಅಶೋಕನ ರಾಜ್ಯಪಾಲನೆಯ ಬಗ್ಗೆ ಹೇಳುತ್ತಿದ್ದರು. ಚೆಂಗೀಸ್​ಖಾನನ ಬಗ್ಗೆ ಒಮ್ಮೆ ಪ್ರಸ್ತಾಪಿಸಿದ ಸ್ವಾಮೀಜಿ, ಜನ ತಿಳಿದಿರುವಂತೆ ಅವನೊಬ್ಬ ಕ್ರೂರ ದುರಾಕ್ರಮಣಕಾರಿ ಯಲ್ಲ; ಅವನಿಗೆ ಇಡೀ ಪ್ರಪಂಚವನ್ನು ಒಗ್ಗೂಡಿಸುವ ಬಯಕೆಯಿತ್ತು ಎಂದರು. ನೆಪೋ ಲಿಯನ್ ಮತ್ತು ಅಲೆಕ್ಸಾಂಡರರೂ ಸಹ ಅವನಂತೆಯೇ ಎಂದೂ ಅವರು ಅಭಿಪ್ರಾಯಪಟ್ಟರು. ಸ್ವಾಮೀಜಿ ಮತ್ತೆ ಮತ್ತೆ ಭಗವದ್ಗೀತೆಯ ವಿಷಯವಾಗಿ ಮಾತನಾಡುತ್ತಿದ್ದರು. “ಷಂಡತನ ಹಾಗೂ ದೌರ್ಬಲ್ಯಗಳ ಸುಳಿವೇ ಇಲ್ಲದಂತಹ ಅದ್ಭುತಕಾವ್ಯ”ವಾದ ಭಗವದ್ಗೀತೆಯ ಬಗ್ಗೆ ಮಾತನಾಡುವುದೆಂದರೆ ಅವರಿಗೆ ವಿಶೇಷ ಸ್ಫೂರ್ತಿ-ಹುಮ್ಮಸ್ಸು.

ಇಲ್ಲಿಯೂ ಅವರಿಗೆ ತಮ್ಮ ಪ್ರಿಯ ಪತ್ರಿಕೆಯಾದ ‘ಪ್ರಬುದ್ಧ ಭಾರತ’ದ ನೆನಪು. ಹೊಸ ಸಂಪಾದಕರಾದ ಸ್ವಾಮಿ ಸ್ವರೂಪಾನಂದರಿಗೆ ಅವರು ತಮ್ಮ ಹೊಸದೊಂದು ಕವನವನ್ನು ಕಳಿಸಿಕೊಟ್ಟರು. ಕವನದ ಶೀರ್ಷಿಕೆ\eng{–To the Awakened India} (“ಎಚ್ಚತ್ತ ಭಾರತಕ್ಕೆ”). ಈ ಕವನವು ರೂಪತಾಳಿದ ಬಗೆಯನ್ನು ನಿವೇದಿತಾ ತಿಳಿಸುತ್ತಾಳೆ: “ಒಂದು ಮಧ್ಯಾಹ್ನ ನಾವೆಲ್ಲ ಒಟ್ಟಿಗೆ ಕುಳಿತಿದ್ದಾಗ ಸ್ವಾಮೀಜಿ ಒಂದು ಕಾಗದವನ್ನು ತಂದು ತೋರಿಸಿದರು. ಅವರೊಂದು ಪತ್ರವನ್ನು ಬರೆಯಲೆಂದು ಹೊರಟಿದ್ದರು; ಆದರೆ ಆ ಪತ್ರ ಕವನದ ರೂಪತಾಳಿತ್ತು!”

ಒಂದು ದಿನ ಸ್ವಾಮೀಜಿಯವರಿಗೆ ಇದ್ದಕ್ಕಿದ್ದಂತೆ ಏಕಾಂತಕ್ಕೆ ಹೋಗಿರಬೇಕೆಂಬ ಇಚ್ಛೆ ಯುಂಟಾಯಿತು. ಸರಿ, ತಾವೊಬ್ಬರೇ ಅಲ್ಲಿಂದ ದೋಣಿಯಲ್ಲಿ ಹೊರಟರು. ಆದರೆ ಅವರ ಮನದ ಇಂಗಿತವನ್ನರಿಯದೆ ಅವರ ಶಿಷ್ಯೆಯರು ತಾವೂ ಅವರನ್ನು ಹಿಂಬಾಲಿಸಿದರು. ಈಗ ಅವರು ಹೊರಟದ್ದು ಕ್ಷೀರಭವಾನಿ ಎಂಬಲ್ಲಿಗೆ. ಆಶ್ಚರ್ಯದ ಸಂಗತಿಯೇನೆಂದರೆ, ಇವರನ್ನು ಕರೆತಂದ ಮುಸಲ್ಮಾನ ಅಂಬಿಗ, ಇವರ್ಯಾರನ್ನೂ ಪಾದರಕ್ಷೆಗಳನ್ನು ಧರಿಸಿ ಆ ಸ್ಥಳದಲ್ಲಿ ಕಾಲಿಡಲು ಬಿಡಲಿಲ್ಲ! ಅಲ್ಲಿನ ಮುಸಲ್ಮಾನರೂ ಕೂಡ ಹಿಂದೂ ದೇವದೇವಿಯರಲ್ಲಿ ಅಷ್ಟು ಭಕ್ತಿ-ಶ್ರದ್ಧೆಗಳ ನ್ನಿಟ್ಟುಕೊಂಡಿದ್ದಾರೆ. ಅಲ್ಲದೆ ಅವರು ಹಿಂದೂಗಳ ಹಬ್ಬ-ಹರಿದಿನಗಳಲ್ಲಿ ಉಪವಾಸ ವ್ರತವನ್ನೂ ಆಚರಿಸುತ್ತಾರೆ. ಇವೆಲ್ಲ ಆ ಶಿಷ್ಯೆಯರಿಗೊಂದು ವಿಶೇಷ ಅನುಭವ. ಮತ್ತೊಂದು ವಿಶೇಷ ಸಂಗತಿಯೇನೆಂದರೆ, ಆ ಕ್ಷೀರಭವಾನಿ ಕ್ಷೇತ್ರಕ್ಕೆ ಮುಸಲ್ಮಾನರಾಗಲಿ ಕ್ರೈಸ್ತರಾಗಲಿ ಕಾಲಿಟ್ಟದ್ದು ಪ್ರಾಯಶಃ ಅದೇ ಮೊದಲನೆಯ ಸಲ. ಆದರೆ ಸ್ವಾಮೀಜಿಯವರಿಗೆ ಇಲ್ಲಿಗೆ ಒಬ್ಬರೇ ಬರುವ ಅಭಿಪ್ರಾಯವಿದ್ದುದರಿಂದ, ಈ ಸಲ ಅವರು ಹೆಚ್ಚು ಕಾಲ ನಿಲ್ಲದೆ ಹಿಂದಿರುಗಿಬಿಟ್ಟರು.

ಸ್ವಾಮೀಜಿಯವರಿಗೆ ಏಕಾಂತಕ್ಕೆ ಹೋಗಿರಬೇಕೆಂಬ ಇಚ್ಛೆಯುಂಟಾದಾಗಲೆಲ್ಲ ಇದ್ದಕ್ಕಿದ್ದಂತೆ ತಮ್ಮ ಸಂಗಡಿಗರಿಂದ ತಪ್ಪಿಸಿಕೊಂಡು ಹೊರಟುಬಿಡುತ್ತಿದ್ದರು. ಹಾಗೆ ಹೋಗಿ ಹಿಂದಿರುಗಿ ಬಂದಾಗಲೆಲ್ಲ ಅವರ ಮುಖದಲ್ಲಿ ದಿವ್ಯ ಕಾಂತಿಯೊಂದು ಪ್ರಕಾಶಿಸುವುದನ್ನು ಅವರ ಸಂಗಡಿ ಗರು ಗುರುತಿಸುತ್ತಿದ್ದರು. ಅಂತಹ ಸ್ಥಿತಿಯಲ್ಲಿ ಅವರು ಮತ್ತೆ ಮತ್ತೆ ಉದ್ಗರಿಸುತ್ತಿದ್ದರು–“ಈ ಶರೀರದ ಬಗ್ಗೆ ಆಲೋಚಿಸುವುದೂ ಸಹ ಒಂದು ಪಾಪವೇ ಸರಿ!” ಇಲ್ಲವೆ, “ನಮ್ಮ ಪರಿಸರ ಸುಧಾರಿಸುವುದೆಂಬ ಮಾತು ಸುಳ್ಳು. ಅದು ಇದ್ದಂತೆಯೇ ಇರುತ್ತದೆ. ಆದರೆ ನಾವು ಸುಧಾರಿಸಿ ದಾಗ, ಅದೇ ಪರಿಸರವನ್ನೇ ಬೇರೆ ದೃಷ್ಟಿಯಿಂದ ಕಾಣುತ್ತೇವೆ, ಅಷ್ಟೆ” ಎಂದು. ಮಾನವಜೀವನವು ಭಗವಂತನ ಅಭಿವ್ಯಕ್ತಿ ಎಂಬ ಅಂಶವನ್ನು ಅವರು ಮತ್ತೆ ಮತ್ತೆ ವಿಶ್ಲೇಷಿಸಿ ವಿವರಿಸುತ್ತಿದ್ದರು. ಏಕಾಂತವಾಸದ ಸುಖವನ್ನು ಅನುಭವಿಸಿ ಬಂದಮೇಲೆ ಅವರು ಆಗಾಗ ಹೇಳುತ್ತಿದ್ದರು, “ಸಮಾಜಜೀವನವೆಂಬುದು ನಿಜಕ್ಕೂ ತುಂಬ ದುಃಖಕರ; ಜೀವನದ ಅತ್ಯುನ್ನತ ಆದರ್ಶವನ್ನು ಸಾಕ್ಷಾತ್ಕರಿಸಿಕೊಳ್ಳುವಲ್ಲಿ ಅದೊಂದು ದೊಡ್ಡ ಅಡಚಣೆಯೇ ಸರಿ” ಎಂದು. ಪೂರ್ವಕಾಲದ ಸಂನ್ಯಾಸಿಗಳಂತೆ ಗುಪ್ತವಾಗಿ, ನಿರ್ಜನ ಪ್ರದೇಶದಲ್ಲಿ ವಾಸವಾಗಿರಬೇಕೆಂಬ ವ್ಯಾಕುಲತೆ ಅವರನ್ನು ಆವರಿಸಿಕೊಂಡೇ ಇರುತ್ತಿತ್ತು. ಸ್ವಾಮೀಜಿಯವರ ಭವ್ಯ ವ್ಯಕ್ತಿತ್ವದ ನೆರಳಲ್ಲಿ ನಿಂತ ಅವರ ಪಾಶ್ಚಾತ್ಯ ಶಿಷ್ಯರು, ಅಂತರಂಗದ ವಿಕಾಸಕ್ಕೆ ಇಂತಹ ಏಕಾಂತಜೀವನದ ಆವಶ್ಯಕತೆಯೆಷ್ಟೆಂಬು ದನ್ನು ಮನಗಂಡರು. ಈ ವಿಷಯವಾಗಿ ಪಾಶ್ಚಾತ್ಯರ ಹಾಗೂ ಭಾರತೀಯರ ವಿಚಾರಗಳು ಬೇರೆಬೇರೆ. ಇದರ ಬಗ್ಗೆ ಸ್ವಾಮೀಜಿ ಹೇಳುತ್ತಿದ್ದರು–“ಒಬ್ಬ ಮನುಷ್ಯ ಇಪ್ಪತ್ತು ವರ್ಷ ಏಕಾಂತದಲ್ಲಿ ಜೀವಿಸಿದ್ದೇ ಆದರೆ ಅವನ ಬುದ್ಧಿ ಸಂಪೂರ್ಣ ಸ್ವಸ್ಥವಾಗಿ ಉಳಿಯಲಾರದು ಎಂಬುದು ಪಾಶ್ಚಾತ್ಯರ ಅಭಿಮತ. ಆದರೆ ಒಬ್ಬ ಮನುಷ್ಯ ಇಪ್ಪತ್ತು ವರ್ಷಗಳ ಕಾಲ ಏಕಾಂಗಿಯಾಗಿರದೆ ಹೋದರೆ ಅವನು ಪರಿಪೂರ್ಣಜ್ಞಾನಿಯಾಗಲು ಸಾಧ್ಯವೇ ಇಲ್ಲ ಎಂಬುದು ಭಾರತೀಯರ ಅಭಿಮತ. ಬಹುಶಃ, ಪಾಶ್ಚಾತ್ಯ-ಭಾರತೀಯ ಚಿಂತನೆಗಳ ನಡುವಿನ ವ್ಯತ್ಯಾಸವನ್ನು ತೋರಿಸಲು ಇದಕ್ಕಿಂತ ಉತ್ತಮ ಉದಾಹರಣೆ ಸಿಗಲಾರದು.”

ಝೇಲಂ ನದಿಯ ಮೇಲೆ ದೋಣಿಮನೆಗಳಲ್ಲಿ ಇಳಿದುಕೊಂಡಿದ್ದ ಸ್ವಾಮೀಜಿ ಹಾಗೂ ಅವರ ಶಿಷ್ಯೆಯರು ಶ್ರೀನಗರದ ಆಸುಪಾಸಿನ ಅನೇಕ ಸ್ಥಳಗಳಿಗೆ ಭೇಟಿಯಿತ್ತರು. ಇವುಗಳಲ್ಲಿ ಒಂದೆಂದರೆ ‘ಶಂಕರಾಚಾರ್ಯ ಬೆಟ್ಟ’ದ ಮೇಲಿರುವ ಶಿವದೇವಾಲಯ. ನೆಲಮಟ್ಟದಿಂದ ಸಾವಿರ ಅಡಿ ಎತ್ತರ ದಲ್ಲಿರುವ ಈ ಬೆಟ್ಟದ ಮೇಲೆ ನಿಂತು ನೋಡಿದರೆ ಕೆಳಗೆ ಮೈಲಿಗಟ್ಟಲೆ ಹರಡಿಕೊಂಡಿರುವ ‘ತೇಲು ಉದ್ಯಾನ’ಗಳನ್ನು ನೋಡಬಹುದು. ಅದೊಂದು ಅದ್ಭುತ-ರಮ್ಯ ನೋಟ.

ಜುಲೈ ನಾಲ್ಕು ಅಮೆರಿಕದ ಸ್ವಾತಂತ್ರ್ಯ ದಿನ. ಅಂದು ತಮ್ಮ ಅಮೆರಿಕನ್ ಶಿಷ್ಯೆಯರಾದ ಶ್ರೀಮತಿ ಸಾರಾ ಬುಲ್ ಹಾಗೂ ಮಿಸ್ ಜೋಸೆಫಿನ್ನರಿಗೆ ಅನಿರೀಕ್ಷಿತ ಬೆರಗನ್ನುಂಟುಮಾಡಲು ಸ್ವಾಮೀಜಿ ಆಲೋಚಿಸಿದರು. ಅದರ ಹಿಂದಿನ ಸಂಜೆ ಅವರು ತಮ್ಮ ಇಂಗ್ಲಿಷ್ ಶಿಷ್ಯೆ ನಿವೇದಿತೆ ಯನ್ನು ಕರೆದುಕೊಂಡು ಒಬ್ಬ ದರ್ಜಿಯ ಬಳಿಗೆ ಹೋದರು. ಅಮೆರಿಕದ ರಾಷ್ಟ್ರಧ್ವಜವನ್ನು ರಚಿಸುವುದು ಹೇಗೆಂಬುದನ್ನು ಆಕೆ ದರ್ಜಿಗೆ ವಿವರಿಸಿದಳು. ಅವನು ಒಂದು ಬಟ್ಟೆಯ ಮೇಲೆ ಒಡ್ಡೊಡ್ಡಾಗಿ ಅಮೆರಿಕದ ಧ್ವಜದಲ್ಲಿರುವ ನಕ್ಷತ್ರಗಳನ್ನೂ ಅಡ್ಡಪಟ್ಟಿಗಳನ್ನೂ ಹೊಲಿದುಕೊಟ್ಟ. ಈ ‘ಧ್ವಜ’ವನ್ನು ತಂದು ಊಟದ ಮೇಜಿನ ಮೇಲೆ ತೂಗು ಹಾಕಲಾಯಿತು. ಸುತ್ತಲೂ ಹಸಿರೆಲೆಗಳಿಂದ ಕೂಡಿದ ಸಣ್ಣ ಟೊಂಗೆಗಳನ್ನು ಜೋಡಿಸಲಾಯಿತು. ಮರುದಿನ ಮುಂಜಾನೆ ಶ್ರೀಮತಿ ಸಾರಾ ಹಾಗೂ ಜೋಸೆಫಿನ್ ಚಹಾಕ್ಕೆಂದು ಬಂದಾಗ ಅವರಿಗೆ ಅತ್ಯಾಶ್ಚರ್ಯ! ಇದ ಲ್ಲದೆ ಸ್ವಾಮೀಜಿ ತಮ್ಮ ವೈಯಕ್ತಿಕ ಕಾಣಿಕೆಯಾಗಿ, ತಾವು ರಚಿಸಿದ \eng{To the Fourth of July} (“ಜುಲೈ ನಾಲ್ಕನೇ ದಿನಕ್ಕೆ”)ಎಂಬ ಕವನವನ್ನು ಓದಿದರು. ಅದರ ಕನ್ನಡ ರೂಪ ಇಲ್ಲಿದೆ:

\begin{verse}
ಕತ್ತಲಲಿ ದಟ್ಟೈಸಿ ಇಳೆಗೆ ಮಬ್ಬನು ಕವಿದ\\ಕಾಳಮೇಘಗಳೆಲ್ಲ ಕರಗುತಿವೆ ನೋಡು!\\ನಿನ್ನ ಮಂತ್ರಸ್ಪರ್ಶ ಜಗವನೆಚ್ಚರಿಸುತಿದೆ,\\ಹಕ್ಕಿಗಳ ಇಂಪುದನಿ ಗುಂಪಾಗಿದೆ;\\ಹಿಮಮಣಿಗಳಿಂದಿಡಿದ ನಕ್ಷತ್ರಮಕುಟಗಳ\\ಹೂವುಗಳು ಮೇಲೆತ್ತಿ ತೂಗುತ್ತಿವೆ;\\ನಿನಗೆ ಸುಸ್ವಾಗತ ಹಾರೈಸಿವೆ!\\ನೂರುಸಾಸಿರ ದಳದ ಕಮಲನಯನಗಳಿಂದ\\ಒಲುಮೆಯಾಳವ ಸರಸಿ ತೆರೆಯುತ್ತಿವೆ;\\ನಿನಗೆ ಸುಸ್ವಾಗತವ ಹಾರೈಸಿವೆ!\\ಎಲ್ಲವೂ ನಿನಗೆ ಜಯಜಯವೆಂದಿವೆ!
\end{verse}

\begin{verse}
ಓವೊ, ಬೆಳಕಿನ ಪ್ರಭುವೆ,\\ನಿನಗೆ ನವಸ್ವಾಗತವು;\\ನೀನಿಂದು ಮುಕ್ತಿಯನು ಪಸರಿಸಿರುವೆ!\\ಎನಿತೊ ನಾಡುಗಳಲ್ಲಿ ಎನಿತೊ ಕಾಲಗಳಲ್ಲಿ\\ನಿನಗಾಗಿ ಜಗವೆನಿತು ಕಾಯುತಿತ್ತು!\\ನಿನ್ನನ್ನೆ ಹಗಲಿರುಳು ಹುಡುಕುತಿತ್ತು.\\ಮನೆಯ ತೊರೆದರು ಕೆಲರು, ನೇಹಿಗರನುಳಿದವರು,\\ನಿನ್ನನರಸುತ ಜೀವ ತೇದರವರು;\\ಕಡಲಿನಲೆಗಳ ದಾಟಿ, ಕಾನನವ ಬೆನ್ನಟ್ಟಿ,\\ಅಡಿಯ ಕಿತ್ತಡಿಯಿಡುತ ನಡೆದರವರು–\\ಸಾವು-ಬದುಕಿನ ನಡುವೆ ಹೆಣಗಿದವರು!\\ಕೊನೆಗೊಂದು ದಿನ ಬಂತು, ಕೆಲಸ ಹಣ್ಣಾಯಿತು,\\ತ್ಯಾಗ-ಪೂಜೆಗಳೆಲ್ಲ ಪೂರ್ಣವಾಯ್ತು.
\end{verse}

\begin{verse}
ಅಂದು ಕರುಣಾನಿಧಿಯೆ, ಮೇಲೆದ್ದು ನಿಂತು ನೀ\\ಮುಕ್ತಿ ಪ್ರಭೆಯನು ಸುರಿದೆ ಜಗದ ಮೇಲೆ!\\ನಡೆ ಮುಂದೆ, ಓ ಪ್ರಭುವೆ, ತಡೆಯದಲೆ ನಡೆ ಮುಂದೆ,\\ನಾಡದೆಲ್ಲವು ಬೆಳಕ ಪ್ರತಿಫಲಿಸುವರೆಗೂ,\\ನಿನ್ನ ಬೆಳಕದು ಜಗವ ತಬ್ಬುವರೆಗೂ,\\ಜನವೆಲ್ಲ ತಲೆಯೆತ್ತಿ, ಸಂಕೋಲೆಗಳ ಕಡಿದು\\ಸಂತಸದಿ ಮರುಜೀವ ಪಡೆವವರೆಗೂ!
\end{verse}

ಇದು ಅಮೆರಿಕದ ಸ್ವಾತಂತ್ರ್ಯದಿನದ ಕುರಿತಾಗಿ ಬರೆದದ್ದಾದರೂ, ಇದರಲ್ಲಿ ಆತ್ಮವು ಮುಕ್ತಿ ಯನ್ನು ಪಡೆದು ಅನಂತ ಪರಬ್ರಹ್ಮದಲ್ಲಿ ಒಂದಾಗುವುದರ ಮಾರ್ಮಿಕ ವರ್ಣನೆಯಿದೆ. ಈ ಕವನ ದಲ್ಲಿ ಸ್ವಾಮೀಜಿ ತಮ್ಮ ಅಂತರಂಗದ ಆಶಯವನ್ನೇ ವ್ಯಕ್ತಪಡಿಸಿದ್ದಾರೆಂದೂ ಅರ್ಥೈಸಬಹುದು. ಇದು ಕಾಲಜ್ಞಾನಿಯೊಬ್ಬನ ಭವಿಷ್ಯವಾಣಿಯಂತಿದೆಯೆಂದು ಕೂಡ ಸಕಾಲದಲ್ಲಿ ಸಿದ್ಧವಾಗಲಿದೆ; ಏಕೆಂದರೆ, ಇನ್ನು ಕೇವಲ ನಾಲ್ಕೇ ವರ್ಷಗಳ ಬಳಿಕ, ಅದೇ ಜುಲೈ ನಾಲ್ಕರಂದು, ಅವರು ತಾವೇ ಬಿಗಿದುಕೊಂಡಿದ್ದ ಬಂಧನಗಳನ್ನು ಹರಿದು, ಅಂತಿಮ ಸ್ವಾತಂತ್ರ್ಯವನ್ನು ಹೊಂದಲಿದ್ದರು.

ಒಂದು ದಿನ ಮಿಸ್ ಮೆಕ್​ಲಾಡ್ ವಿನೋದಕ್ಕಾಗಿ, ಊಟದ ಬಳಿಕ ತನ್ನ ತಟ್ಟೆಯಲ್ಲಿ ಉಳಿದಿದ್ದ ಚೆರ್ರಿ ಹಣ್ಣಿನ ಬೀಜಗಳನ್ನು ಎಣಿಸುತ್ತ, ತಾನು ಮದುವೆಯಾಗುವುದು ಯಾವಾಗ ಎಂದು ಲೆಕ್ಕ ಹಾಕುತ್ತಿದ್ದಳು. ಅವಳ ಸ್ನೇಹಿತೆಯರೂ ಈ ವಿನೋದದಲ್ಲಿ ಸೇರಿಕೊಂಡಿದ್ದರು. ಇದು ಸ್ವಾಮೀಜಿ ಯವರ ಕಣ್ಣಿಗೆ ಬಿತ್ತು. ಅವರು ಆಗೇನೂ ಮಾತನಾಡಲಿಲ್ಲ. ಆದರೆ ಅವರು ಅದನ್ನು ಗಂಭೀರವಾಗಿ ಪರಿಗಣಿಸಿದರು. ಮರುದಿನ ಶಿಷ್ಯೆಯರೆಲ್ಲ ಒಟ್ಟಾಗಿದ್ದಾಗ ಸ್ವಾಮೀಜಿ ಇದ್ದಕ್ಕಿ ದ್ದಂತೆ ಸ್ಫೂರ್ತಿಭರಿತರಾಗಿ ತ್ಯಾಗ ಜೀವನದ ಹಿರಿಮೆಯನ್ನು ಬಣ್ಣಿಸತೊಡಗಿದರು. ಮದುವೆಯ ವಿಷಯದಲ್ಲಿ ಸಂನ್ಯಾಸಿಗಳ ಭಾವನೆಯೇ ಬೇರೆ, ಪ್ರಾಪಂಚಿಕರ ಭಾವನೆಯೇ ಬೇರೆ. ‘ಮದುವೆ ಎನ್ನುವುದು ಸಂನ್ಯಾಸಿಗಳು ಭಾವಿಸುವಷ್ಟು ತುಚ್ಛವಾದದ್ದೇನಲ್ಲ; ಉದಾಹರಣೆಗೆ ಜನಕ ಮಹಾರಾಜ ಗೃಹಸ್ಥನಾಗಿದ್ದರೂ ರಾಜರ್ಷಿ ಎನ್ನಿಸಿಕೊಳ್ಳಲಿಲ್ಲವೆ?’–ಇದು ಗೃಹಸ್ಥರ ವಾದ. ಈ ಬಗೆಯ ವಾದವನ್ನು ಸ್ವಾಮೀಜಿ ತೀಕ್ಷ ್ಣವಾಗಿ ಖಂಡಿಸುತ್ತ ಹೇಳಿದರು, “ಏನು, ಜನಕನಂತೆ ಇರುವುದೆಂದರೆ ಹುಡುಗಾಟವೆಂದು ಭಾವಿಸಿದಿರಾ? ಸಂಪೂರ್ಣ ನಿರ್ಲಿಪ್ತತೆಯಿಂದ ಸಿಂಹಾ ಸನದ ಮೇಲೆ ಕುಳಿತಿರುವುದು ಅಷ್ಟು ಸುಲಭವೆಂದು ತಿಳಿದಿರಾ? ಮಡದಿ-ಮಕ್ಕಳು, ಧನ-ಮಾನ ಗಳನ್ನು ಲಕ್ಷ್ಯಕ್ಕೆ ತಾರದೆ ಇರಬಲ್ಲಿರೇನು? ಪಾಶ್ಚಾತ್ಯ ರಾಷ್ಟ್ರಗಳಲ್ಲಿ ಒಬ್ಬರಾದ ಮೇಲೊಬ್ಬರು ಬಂದು ತಾವು ಇಂತಹ ಸ್ಥಿತಿಯನ್ನು ಮುಟ್ಟಿರುವುದಾಗಿ ಹೇಳಿದರು. ಅದಕ್ಕೆ ನಾನು, ‘ಅಯ್ಯೋ, ಅಂತಹ ಮಹಾತ್ಮರು ಭಾರತದಲ್ಲಿನ್ನೂ ಹುಟ್ಟಲಿಲ್ಲವಲ್ಲ!’ ಎಂದಷ್ಟೇ ಹೇಳಿದೆ.” ಬಳಿಕ ಅವರು ಮತ್ತೆ ಹೇಳಿದರು, “ಈ ಸತ್ಯವನ್ನು ನೀವೂ ಅರಿತುಕೊಂಡು ನಿಮ್ಮ ಮಕ್ಕಳಿಗೂ ಬೋಧಿಸಲು ಮರೆಯಬೇಡಿ–ಗೃಹಸ್ಥನಿಗೂ ಸಂನ್ಯಾಸಿಗೂ ಇರುವ ಅಂತರ ಬಹಳ ದೊಡ್ಡದು. ಅದು ಮಿಣುಕುಹುಳುವಿಗೂ ಸೂರ್ಯನಿಗೂ ಇರುವ ಅಂತರ; ಸಣ್ಣಕೊಳಕ್ಕೂ ವಿಶಾಲ ಸಾಗರಕ್ಕೂ ಇರುವ ಅಂತರ; ಸಾಸಿವೆ ಕಾಳಿಗೂ ಮೇರುಪರ್ವತಕ್ಕೂ ಇರುವ ಅಂತರ. ಈ ಜಗತ್ತಿನಲ್ಲಿ ಪ್ರತಿಯೊಂದು ವಿಷಯವೂ ಭಯದಿಂದ ಕೂಡಿದೆ. ಆದರೆ ವೈರಾಗ್ಯದಲ್ಲಿ ಮಾತ್ರ ಭಯದ ಸೋಂಕೇ ಇಲ್ಲ–‘ವೈರಾಗ್ಯಮೇವಾಭಯಮ್​’. ಒಬ್ಬನು ಕಳ್ಳಸಂನ್ಯಾಸಿಯಿರಬಹುದು ಅಥವಾ ಸಂನ್ಯಾಸಧರ್ಮವನ್ನು ಪಾಲಿಸುವಲ್ಲಿ ವಿಫಲನಾದವನಿರಬಹುದು–ಆದರೆ ಅಂಥವನು ಕೂಡ ಧನ್ಯನೇ ಸರಿ! ಏಕೆಂದರೆ ಅವನೂ ಕೂಡ ಸಂನ್ಯಾಸವೆಂಬ ಮಹಾ ಆದರ್ಶವೊಂದಿದೆ ಎಂಬುದ ನ್ನಾದರೂ ಕಂಡುಕೊಂಡಿದ್ದಾನಲ್ಲ! ಅಲ್ಲದೆ ಅಂಥವರು ಇತರ ಸಾಧು-ಸಂನ್ಯಾಸಿಗಳ ಯಶಸ್ಸಿಗೆ ಪರೋಕ್ಷವಾಗಿ ಕೆಲಮಟ್ಟಿಗೆ ಕಾರಣರಾಗಿದ್ದಾರೆ. ಈ ಕಾಷಾಯವಸ್ತ್ರವೊಂದು ಇಲ್ಲದೆ ಹೋಗಿದ್ದರೆ, ಭೋಗ ಮತ್ತು ಪ್ರಾಪಂಚಿಕತೆ ಎಂಬುದು ಮನುಷ್ಯನ ಮನುಷ್ಯತ್ವವನ್ನೇ ಸಂಪೂರ್ಣವಾಗಿ ತಿಂದು ಹಾಕುತ್ತಿತ್ತು.”

ಮರುದಿನ ಶ್ರೀಮತಿ ಬುಲ್ ಹಾಗೂ ಮಿಸ್ ಮೆಕ್​ಲಾಡ್ ಯಾವುದೋ ವೈಯಕ್ತಿಕ ಕೆಲಸಕ್ಕಾಗಿ ಗುಲ್​ಮಾರ್ಗಿಗೆ ಹೊರಟರು. ಈ ಅವಕಾಶವನ್ನು ಉಪಯೋಗಿಸಿಕೊಳ್ಳಲು ನಿಶ್ಚಯಿಸಿದ ಸ್ವಾಮೀಜಿ ತಮ್ಮ ಮನದಾಸೆಯಂತೆ ಏಕಾಂತದೆಡೆಗೆ ಹೊರಡಲು ಅನುವಾದರು. ಈ ವಿಷಯ ವನ್ನು ಯಾರಿಗೂ ತಿಳಿಯಗೊಡದೆ ಅವರು ಅಮರನಾಥದ ಕಡೆಗೆ ಹೊರಟೇಬಿಟ್ಟರು. ಅಮರ ನಾಥವು ಹಿಮಾಲಯದ ಅತಿ ಎತ್ತರದ ಸ್ಥಳದಲ್ಲಿರುವ ಒಂದು ಪವಿತ್ರ ಶಿವಕ್ಷೇತ್ರ. ಅವರು ಕೈಯಲ್ಲಿ ಬಿಡಿಗಾಸನ್ನೂ ಇಟ್ಟುಕೊಳ್ಳದೆ ಏಕಾಂಗಿಯಾಗಿ ಹೊರಟುಬಿಟ್ಟಿದ್ದರು. ಆದರೆ ದಾರಿಯ ಲ್ಲಿನ ಹಿಮನದಿಯೊಂದು ಬೇಸಿಗೆಯ ಬಿಸಿಲಿಗೆ ಕರಗಿದ್ದುದರಿಂದ ರಸ್ತೆ ಮುಚ್ಚಿಹೋಗಿತ್ತು. ಆದ್ದರಿಂದ ಅವರು ಹಿಂದಿರುಗಬೇಕಾಯಿತು.

ಜುಲೈ ೧೯ರಂದು ಸ್ವಾಮೀಜಿಯವರ ತಂಡ ಅನಂತನಾಗದ ಕಡೆಗೆ ಹೊರಟಿತು. ಪ್ರಯಾಣ ಝೇ|ಲಂ ನದಿಯ ಮೂಲಕ. ಮೊದಲನೆಯ ದಿನ ಮಧ್ಯಾಹ್ನ ಅವರು ನದಿಯ ಪಕ್ಕದ ಕಾಡಿನ ಒಂದು ಕೆರೆಯಲ್ಲಿ ‘ಪಂದ್ರೇಥಾನ’ ದೇವಸ್ಥಾನವನ್ನು ದರ್ಶಿಸಿದರು. ‘ಪಂದ್ರೇಥಾನ’ ಎಂದರೆ ಪುರಾಣಾಧಿಷ್ಠಾನ–ಅರ್ಥಾತ್, ಪುರಾತನ ರಾಜಧಾನಿ. ಸ್ವಾಮೀಜಿ ತಮ್ಮ ಶಿಷ್ಯರಿಗೆ ಈ ದೇವ ಸ್ಥಾನದ ವಿಶಿಷ್ಟ ವಾಸ್ತುಶಿಲ್ಪದ ಹಾಗೂ ಶಿಲ್ಪಶಾಸ್ತ್ರದ ವಿಷಯವಾಗಿ ವಿವರಿಸಿದರು. ದೇವಸ್ಥಾನ ಅಷ್ಟೇನೂ ದೊಡ್ಡದಲ್ಲ. ಹೊರಗಿನಿಂದ ನೋಡುವಾಗ ಇದು ನಾಲ್ಕು ಕಂಬಗಳ ಮೇಲೆ ನಿಂತ ಪಿರಮಿಡ್ಡಿನಂತಿದೆ. ನಾಲ್ಕು ದಿಕ್ಕುಗಳಲ್ಲಿ ನಾಲ್ಕು ದ್ವಾರಗಳಿವೆ. ಇದರ ತುದಿಭಾಗದ ಮೇಲೆ ಪೊದೆ ಬೆಳೆಯಲು ಅನುಕೂಲಿಸಲಾಗಿದೆ. ದೇವಾಲಯದೊಳಗೆ ಸೂರ್ಯಬಿಂಬವನ್ನು ಕೆತ್ತಿರುವ ಕಲ್ಲಿನ ಫಲಕವಿದೆ. ಸರ್ಪಗಳಿಂದ ಆವೃತರಾದ ಸ್ತ್ರೀಪುರುಷರ ವಿಗ್ರಹಗಳಿವೆ. ಇನ್ನೂ ಅನೇಕ ಸೂಕ್ಷ್ಮ ಕೆತ್ತನೆಯ ಕೆಲಸಗಳಿವೆ. ಸ್ವಾಮೀಜಿ ಇವುಗಳೆಲ್ಲದರ ಕಡೆಗೆ ತಮ್ಮ ಶಿಷ್ಯರ ಗಮನ ಸೆಳೆದರು. ದೇವಸ್ಥಾನದ ಹೊರವಲಯದಲ್ಲೂ ಸುಂದರ ಶಿಲ್ಪಗಳಿವೆ. ಪೂರ್ವ ದಿಸೆಯ ದ್ವಾರದ ಕಮಾನಿನ ಮೇಲೆ ಸುಂದರವಾದ ಬುದ್ಧನ ವಿಗ್ರಹವಿದೆ. ಬುದ್ಧನ ಕೈಗಳು ಆಶೀರ್ವಾದದ ಭಂಗಿಯಲ್ಲಿ ಮೇಲೆತ್ತಿಕೊಂಡಿವೆ. ದೇವಸ್ಥಾನದ ಹೊರಗೋಡೆಯ ಮೇಲೆ ಒಬ್ಬಳು ಕುಳಿತಿರುವ ಸ್ತ್ರೀ ಹಾಗೂ ಒಂದು ಮರ–ಇವುಗಳ ಕೆತ್ತನೆ ಇದೆ. ಈ ಸ್ತ್ರೀ ವಿಗ್ರಹದ ಮುಖ ಸ್ವಲ್ಪ ಭಗ್ನವಾಗಿದೆ. ಅದು ಬುದ್ಧನ ತಾಯಿಯಾದ ಮಾಯಾದೇವಿ ಎಂಬುದರಲ್ಲಿ ಸಂಶಯವಿಲ್ಲ. ಈ ದೇವಾಲಯ ಸುಮಾರು ಹತ್ತನೇ ಶತಮಾನದ್ದು. ಸೋದರಿ ನಿವೇದಿತಾ ಹೇಳುವಂತೆ ಈ ಸ್ಥಳ ಸ್ವಾಮೀಜಿಯವರ ಪಾಲಿಗೆ ‘ಸುಂದರ ಹಾಗೂ ಸೂಚ್ಯ’ವಾಗಿತ್ತು.

ಸ್ವಾಮೀಜಿಯವರ ತಂಡ ಅಲ್ಲಿಂದ ತಮ್ಮ ದೋಣಿಮನೆಗಳಿಗೆ ಹಿಂದಿರುಗುವಾಗ ಸಂಜೆಯಾ ಗಿತ್ತು. ಆ ನೀರವ ಪ್ರಶಾಂತ ಅರಣ್ಯದ ನಡುವೆ ಇರುವ ದೇವಾಲಯ, ಶಾಂತಮುದ್ರೆಯ ಬುದ್ಧನ ಸನ್ನಿಧಿ ಇವುಗಳೆಲ್ಲ ಸ್ವಾಮೀಜಿಯವರ ಮನದ ಮೇಲೆ ಬಹಳವಾಗಿಯೇ ಪ್ರಭಾವ ಬೀರಿರಬೇಕು. ಅಂದು ಸಂಜೆಯೆಲ್ಲ ಅವರ ಮಾತಿನಲ್ಲಿ ಐತಿಹಾಸಿಕ ಹೋಲಿಕೆಗಳೇ ತುಂಬಿಕೊಂಡಿದ್ದುವು. ವೈದಿಕ ಹಾಗೂ ರೋಮನ್ ಕ್ಯಾಥೋಲಿಕ್ ಸಂಪ್ರದಾಯಗಳನ್ನು ಹೋಲಿಸುತ್ತ ಅವರು ಹೇಳುತ್ತಾರೆ, “ವೇದಧರ್ಮದಿಂದಲೇ ಹುಟ್ಟಿಬಂದ ಬೌದ್ಧಧರ್ಮದ ಹಲವಾರು ವಿಧಿನಿಯಮಗಳು ರೋಮನ್ ಕ್ಯಾಥೋಲಿಕ್ ಸಂಪ್ರದಾಯಗಳಲ್ಲೂ ಹರಿದುಬಂದಿವೆ. ಆದ್ದರಿಂದ ವೈದಿಕ ಹಾಗೂ ರೋಮನ್ ಕ್ಯಾಥೋಲಿಕ್ ವಿಧಿನಿಯಮಗಳಲ್ಲಿ ಅನೇಕ ಸಾಮ್ಯಗಳನ್ನು ಕಾಣಬಹುದು. ಕ್ರೈಸ್ತಧರ್ಮದಲ್ಲಿ ಯಾವುದನ್ನು \eng{Mass} (ಪ್ರಭುಭೋಜನ) ಎನ್ನುತ್ತಾರೋ ಅದನ್ನು ವೈದಿಕ ಸಂಪ್ರದಾಯದಲ್ಲಿ ನೈವೇದ್ಯ ಎನ್ನುತ್ತಾರೆ. ಆದರೆ ನೀವು ಮಂಡಿಯೂರಿ ನೈವೇದ್ಯ ಮಾಡುತ್ತೀರಿ, ವೈದಿಕರು ಸುಖಾಸನದಲ್ಲಿ ಕುಳಿತು ನೈವೇದ್ಯ ಮಾಡುತ್ತಾರೆ. ನಾವು ಯಾವುದನ್ನು ಪ್ರಸಾದ ಎಂದು ಕರೆಯುತ್ತೇವೆಯೋ ಅದನ್ನೇ ನೀವು ಸೇಕ್ರಮೆಂಟ್ \eng{(Sacrament)} ಎನ್ನುತ್ತೀರಿ. ಟೆಬೆಟ್ಟಿನ ಬೌದ್ಧರು ಮಂಡಿಯೂರಿ ಕುಳಿತುಕೊಳ್ಳುತ್ತಾರೆ. ಅಲ್ಲದೆ ನಿಮ್ಮಲ್ಲಿರುವಂತೆಯೇ ನಮ್ಮಲ್ಲಿಯೂ ಧೂಪ-ದೀಪ-ಭಜನೆಗಳ ಕ್ರಮವಿದೆ.” ಬಳಿಕ ಬುದ್ಧನ ವಿಷಯವಾಗಿ ಮಾತನಾಡುತ್ತ ಹೇಳಿದರು, “ಬುದ್ಧ! ನಿಸ್ಸಂಶಯವಾಗಿ ಈ ಭೂಮಿಯ ಮೇಲೆ ಅವತರಿಸಿದ ಮಹಾತ್ಮರಲ್ಲೆಲ್ಲ ಶ್ರೇಷ್ಠ ನಾದವನು! ಅವನು ತನ್ನ ಒಂದು ಉಸಿರನ್ನೂ ತನಗಾಗಿ ವ್ಯಯಿಸಿದವನಲ್ಲ! ಬುದ್ಧ ಒಬ್ಬ ವ್ಯಕ್ತಿಯಲ್ಲ, ಅದೊಂದು ಸ್ಥಿತಿ. ಅವನು ಹೇಳಿದ, ‘ನಾನು ಮಾರ್ಗವನ್ನು ಹುಡುಕಿದ್ದೇನೆ; ಓ ಮಾನವರೇ ನೀವೆಲ್ಲ ಅದರ ಮೂಲಕ ಪ್ರವೇಶಿಸಿ’ ಎಂದು.”

ಈಗ ಸ್ವಾಮೀಜಿಯವರ ತಂಡ ಝೇಲಂ ನದಿಯ ಮೂಲಕ ಸಾಗುತ್ತ, ಇಕ್ಕೆಲಗಳಲ್ಲಿ ಕಾಣುವ ಪ್ರಕೃತಿ ಸೌಂದರ್ಯವನ್ನು ಆಸ್ವಾದಿಸುತ್ತ, ಮರುದಿನ ಅವಂತಿಪುರದ ಎರಡು ಪಾಳುಬಿದ್ದ ದೇವಸ್ಥಾನಗಳ ಬಳಿಗೆ ಬಂದಿತು. ಇಲ್ಲಿ ನದಿಯ ದಂಡೆಯ ಮೇಲೆ ಸ್ವಾಮೀಜಿ ತಮ್ಮ ಶಿಷ್ಯೆಯರೊಂದಿಗೆ ಸುಮಾರು ಮೂರು ಮೈಲಿ ದೂರ ನಡೆಯುತ್ತ ಹೋದರು. ದಾರಿಯಲ್ಲಿ ‘ಪಾಪ’ ಎಂಬ ವಿಷಯವಾಗಿ ಮಾತನಾಡುತ್ತ ಭಾರತೀಯ, ಪಾಶ್ಚಾತ್ಯ ಹಾಗೂ ಈಜಿಪ್ಷಿಯನ್ ಧರ್ಮಗಳಲ್ಲಿನ ಪಾಪದ ಕಲ್ಪನೆಯನ್ನು ತುಲನಾತ್ಮಕವಾಗಿ ವಿವರಿಸಿದರು. ಬಳಿಕ ಭಾರತದ ಭವಿತವ್ಯದ ಕುರಿತು ಹೇಳಿದರು, “ಭಾರತದ ಜನಜೀವನವನ್ನು ಬಲಪಡಿಸಬೇಕಾದರೆ, ಅದರದೇ ಆದ ಸಂಸ್ಕೃತಿ-ಧ್ಯೇಯಗಳನ್ನು ದೃಢಪಡಿಸಬೇಕು. ಉದಾಹರಣೆಗೆ ಬುದ್ಧ ಉಪದೇಶಿಸಿದ ತ್ಯಾಗದ ಆದರ್ಶವನ್ನು ಭಾರತ ಸ್ವೀಕರಿಸಿತು. ಮತ್ತು ಒಂದು ಸಾವಿರ ವರ್ಷಗಳೊಳಗೆ ಅದು ತನ್ನ ಅತ್ಯುನ್ನತ ಸಮೃದ್ಧ ಸ್ಥಿತಿಯನ್ನು ಮುಟ್ಟಿತು. ಏಕೆಂದರೆ ಭಾರತೀಯ ಜನಜೀವನದ ಶಕ್ತಿಮೂಲ ವಿರುವುದು ತ್ಯಾಗದಲ್ಲಿ. ಸೇವೆ ಮತ್ತು ಮುಕ್ತಿಗಳೇ ಅದರ ಅತ್ಯುನ್ನತ ಆದರ್ಶಗಳು.” ಜುಲೈ ೨೨ರಂದು ಯಾತ್ರಿಕರ ತಂಡ ಬಿಜ್​ಬೆಹರ ಎಂಬಲ್ಲಿನ ದೇವಸ್ಥಾನದ ಬಳಿಗೆ ತಲುಪಿತು. ಆ ದೇವಸ್ಥಾನ ಆಗಲೇ ಅಮರನಾಥ ಯಾತ್ರಿಕರಿಂದ ಕಿಕ್ಕಿರಿದಿತ್ತು. ಸ್ವಾಮೀಜಿ ಸ್ವಲ್ಪ ಹೊತ್ತು ಆ ಯಾತ್ರಿಕರೊಂದಿಗಿದ್ದು ದೋಣಿಗೆ ಹಿಂದಿರುಗಿ ಪ್ರಯಾಣವನ್ನು ಮುಂದುವರಿಸಿದರು. ಮರುದಿನ ಮಧ್ಯಾಹ್ನ ಅನಂತನಾಗವನ್ನು ತಲುಪಿದರು.

ಅನಂತನಾಗದಲ್ಲಿ ಸ್ವಾಮೀಜಿ ತಮ್ಮ ಶಿಷ್ಯೆಯರೊಂದಿಗೆ ಸೇಬಿನ ತೋಟವೊಂದರಲ್ಲಿ ಕುಳಿತು ಮಾತನಾಡುತ್ತಿದ್ದರು. ವಿಷಯ–ಅತ್ಯಂತ ಅಪರೂಪವಾಗಿ ಪ್ರಸ್ತಾಪಕ್ಕೆ ಬರುವಂಥದು, ಅತ್ಯಂತ ವೈಯಕ್ತಿಕವಾದದ್ದು. ಸ್ವಾಮೀಜಿ ತಮ್ಮ ಕೈಯಲ್ಲೆರಡು ನುಣುಪಾದ ಬೆಣಚುಕಲ್ಲುಗಳನ್ನು ಹಿಡಿದುಕೊಂಡು ಆಟವಾಡುತ್ತ ಹೇಳಿದರು: “ಸಾವು ನನ್ನ ಬಳಿ ಸಾರಿದಾಗಲೆಲ್ಲ ನನ್ನೆಲ್ಲ ದೌರ್ಬಲ್ಯಗಳೂ ಮಾಯವಾಗಿಬಿಡುತ್ತವೆ. ಆಗ ನನಗೆ ಯಾವುದೇ ಭಯ-ಸಂಶಯಗಳಾಗಲಿ ಬಾಹ್ಯ ಪರಿಸರದ ಕಡೆಗಿನ ಗಮನವಾಗಲಿ ಇರುವುದಿಲ್ಲ. ನಾನಾಗ ಕೇವಲ ಸಾವಿನ ಸಿದ್ಧತೆಯಲ್ಲಿ ಮಗ್ನನಾಗಿದ್ದುಬಿಡುತ್ತೇನೆ. ನಾನು ಈ ಕಲ್ಲಿನಷ್ಟು ಗಟ್ಟಿಮನಸ್ಸಿನವನಾಗಿದ್ದೇನೆ–” ತಮ್ಮ ಕೈಯಲ್ಲಿದ್ದ ಬೆಣಚುಕಲ್ಲುಗಳನ್ನು ಒಂದಕ್ಕೊಂದು ಕುಟ್ಟಿ ಉದ್ಗರಿಸಿದರು, “ಏಕೆಂದರೆ, ನಾನು ಭಗವಂತನ ಪಾದವನ್ನು ಮುಟ್ಟಿದ್ದೇನೆ!” ಅನಂತರ ಸಂನ್ಯಾಸದ ವಿಷಯವಾಗಿ ಹೇಳಿದರು, “ಕಾಂಚನವನ್ನು ಬಯಸುವ ಅಥವಾ ಕಾಂಚನದ ಬಗ್ಗೆ ಚಿಂತಿಸುವ ಸಂನ್ಯಾಸಿಯೊಬ್ಬ ತನ್ನ ಗೋರಿಯನ್ನು ತಾನೇ ತೋಡಿಕೊಂಡನೆಂದೇ ಅರ್ಥ” ಎಂದು. ವಿವಾಹದ ಬಗ್ಗೆ ಮಾತನಾಡುತ್ತ ಸ್ವಾಮೀಜಿ ಹೇಳಿದರು, “ಹಿಂದೂಧರ್ಮದಲ್ಲಿ ವಿವಾಹವೆಂಬುದು ಕೇವಲ ವೈಯಕ್ತಿಕ ಸುಖಕ್ಕಾಗಿ ಮಾತ್ರವಲ್ಲ; ಅದು ಸಮಾಜದ ಹಾಗೂ ರಾಷ್ಟ್ರದ ಒಳಿತಿಗಾಗಿ ರೂಪಿಸಲ್ಪಟ್ಟಿದೆ” ಎಂದು. ಬಳಿಕ ಅವರು ಹೇಳಿದರು:

“ನೀವು ಪಾಶ್ಚಾತ್ಯರಿದ್ದೀರಲ್ಲ–ತುಂಬ ಗೀಳಿನವರು, ವಿಕೃತ ಮನಸ್ಸಿನವರು! ನೀವು ದುಃಖ ವನ್ನು ಆರಾಧಿಸುತ್ತೀರಿ! ನಿಮ್ಮ ದೇಶಗಳಲ್ಲೆಲ್ಲ ನಾನು ಅದನ್ನೇ ಕಂಡೆ. ಹೊರನೋಟಕ್ಕೆ ನಿಮ್ಮ ಸಾಮಾಜಿಕ ಜೀವನವು ಅಟ್ಟಹಾಸದ ನಗುವಿನಂತೆ ಕಾಣುತ್ತದೆ. ಆದರೆ ಅದರ ಒಳಗೆಲ್ಲ ಬರಿಯ ರೋದನ. ಕಡೆಯಲ್ಲಿ ಅದು ಕಣ್ಣೀರಿನಲ್ಲಿ ಪರ್ಯವಸಾನಗೊಳ್ಳುತ್ತದೆ. ನಿಮ್ಮ ಹುಡುಗಾಟಿಕೆ ಚೆಲ್ಲಾಟಗಳೆಲ್ಲ ಕೇವಲ ಮೇಲ್ನೋಟಕ್ಕೆ. ಆದರೆ ಆಂತರ್ಯದಲ್ಲಿ ಅದೊಂದು ದುರಂತ ದುರ್ಭರ ಕಥೆ. ಇಲ್ಲಿ ಭಾರತದಲ್ಲಿ ಹೊರನೋಟಕ್ಕೆ ದುಃಖದ ಮಬ್ಬು ಕವಿದಂತೆ ಕಾಣುತ್ತದೆ. ಆದರೆ ಒಳಗಡೆ ಅನಾಸಕ್ತಿ, ಶಾಂತಿ, ತೃಪ್ತಿ ತುಂಬಿದೆ.” ಅಮೆರಿಕದಲ್ಲಿ ನಡೆದ ಒಂದು ಘಟನೆಯನ್ನು ನೆನಪಿಸಿ ಕೊಂಡು ಸ್ವಾಮೀಜಿ ಹೇಳಿದರು, “ನಾನು ನಗುನಗುತ್ತ ಮಾತನಾಡುವುದನ್ನು, ತಮಾಷೆ ಮಾಡಿ ನಗಿಸುವುದನ್ನು ಕಂಡ ಅಲ್ಲಿನ ಪಾದ್ರಿಗಳಿಗೆ ಆಶ್ಚರ್ಯವೋ ಆಶ್ಚರ್ಯ. ಅವರೊಮ್ಮೆ ನನಗೆ ಹೇಳಿದರು, ‘ಸ್ವಾಮಿ, ನೀವೊಬ್ಬ ಧಾರ್ಮಿಕ ಬೋಧಕರು. ನೀವು ಸಾಮಾನ್ಯ ಜನರಂತೆ ಈ ರೀತಿ ನಗುವುದು, ಹುಡುಗಾಟವಾಡುವುದು ಸರಿಯಲ್ಲ. ಅದು ನಿಮಗೆ ಭೂಷಣವಲ್ಲ’ ಎಂದು. ಅದಕ್ಕೆ ನಾನೆಂದೆ, ‘ನೋಡಿ, ನಾವು ಅಮೃತಪುತ್ರರು! ಸಚ್ಚಿದಾನಂದನ ಸಂತಾನರು! ಹಾಗಿರುವಾಗ ನಾನೇಕೆ ಜೋಲು ಮೋರೆ ಹಾಕಿಕೊಂಡು ಕುಳಿತಿರಬೇಕು?’ ಎಂದು.”

ಈಗ ಸ್ವಾಮೀಜಿ ಇನ್ನೊಂದು ವಿಚಾರ ಸರಣಿಯನ್ನು ಮುಂದಿಟ್ಟರು, “ಕೇವಲ ಒಂದೇ ಜನ್ಮದಲ್ಲಿ ನಾಯಕನೊಬ್ಬ ತಯಾರಾಗಲಾರ. ಅವನು ಅದಕ್ಕಾಗಿಯೇ ಜನ್ಮವೆತ್ತಿರಬೇಕಾಗುತ್ತದೆ. ನಾಯಕನಾಗುವುದರಲ್ಲಿರುವ ದೊಡ್ಡ ಕಷ್ಟವೆಂದರೆ, ಯೋಜನೆಗಳನ್ನು ಸಿದ್ಧಪಡಿಸುವುದರಲ್ಲೂ ಅಲ್ಲ, ಅಥವಾ ಗುಂಪುಗೂಡಿಸಿ ಸಂಘಟಿಸುವುದರಲ್ಲೂ ಅಲ್ಲ. ನಾಯಕನ ಯೋಗ್ಯತೆ ಗೊತ್ತಾಗು ವುದು ಎಲ್ಲೆಂದರೆ, ವಿಭಿನ್ನ ಸ್ವಭಾವಗಳ ಜನರನ್ನು ಅವರಲ್ಲಿರುವ ಸಮಾನ ಗುಣಗಳ ಆಧಾರದ ಮೇಲೆ ಒಂದುಗೂಡಿಸುವ ಕಾರ್ಯದಲ್ಲಿ. ಇದನ್ನು ಯಾರೆಂದರವರು ಪ್ರಯತ್ನಪೂರ್ವಕವಾಗಿ ಮಾಡಲು ಸಾಧ್ಯವಿಲ್ಲ. ಅದು ತಾನಾಗಿಯೇ, ಒಳಗಿನಿಂದಲೇ ಬರಬೇಕಾಗುತ್ತದೆ.” ಬಳಿಕ ಅವರು ತಮ್ಮ ಪರಿವ್ರಾಜಕ ದಿನಗಳ ಕೆಲವು ಅದ್ಭುತ ಘಟನಾವಳಿಗಳನ್ನು ಹೇಳಲಾರಂಭಿಸಿದರು. ಆದರೆ ಅದು ಹಠಾತ್ತನೆ ನಿಲ್ಲಬೇಕಾಯಿತು. ಏಕೆಂದರೆ ಕೆಲವು ಹಳ್ಳಿಗರು ಕೈಗೆ ತೀವ್ರ ಗಾಯವಾಗಿದ್ದ ಮಗುವೊಂದನ್ನು ಅಲ್ಲಿಗೆ ಕರೆತಂದರು. ರಕ್ತ ಹರಿಯುತ್ತಿರುವುದನ್ನು ಕಂಡ ಸ್ವಾಮೀಜಿ ಸ್ವತಃ ತಾವೇ ಗಾಯವನ್ನು ತೊಳೆದು ಬಟ್ಟೆ ಸುಟ್ಟ ಬೂದಿಯನ್ನು ಲೇಪಿಸಿ ರಕ್ತಸ್ರಾವವನ್ನು ನಿಲ್ಲಿಸಿದರು.

ಜುಲೈ ೨೩ರಂದು ಸ್ವಾಮೀಜಿ ಮತ್ತು ಅವರ ಶಿಷ್ಯರು ಮಾರ್ತಾಂಡದ ಅವಶೇಷಗಳನ್ನು ನೋಡಲು ಹೋದರು. ಈ ವಿಷಯವಾಗಿ ನಿವೇದಿತೆ ಬರೆಯುತ್ತಾಳೆ, “ಅದೊಂದು ಅದ್ಭುತವಾದ ಪುರಾತನ ದೇವಾಲಯವಾಗಿತ್ತು. ವಿಭಿನ್ನ ಕಾಲಗಳ ವಿಭಿನ್ನ ಶೈಲಿಗಳ ಆ ಸಂಗಮ ಬಹಳ ಕುತೂಹಲಕಾರಿಯಾಗಿ ಕಂಡುಬರುತ್ತಿತ್ತು.”

ಸ್ವಾಮೀಜಿ ಮತ್ತು ಅವರ ಶಿಷ್ಯೆಯರು ಜುಲೈ ೨೫ರಂದು ಅಚಾಬಾಲ್ ಎಂಬಲ್ಲಿಗೆ ಪ್ರಯಾಣ ಬೆಳೆಸಿದರು. ದಾರಿಯುದ್ದಕ್ಕೂ ಪ್ರಕೃತಿದೇವಿ ತನ್ನ ಅಲೌಕಿಕ ಸೌಂದರ್ಯದಿಂದ ವಿರಾಜಿಸುತಿ ್ತದ್ದಳು. ಆ ದೃಶ್ಯವನ್ನು ಕಂಡು ಅವರಿಗೆಲ್ಲ ಮೇರೆಯಿಲ್ಲದ ಆನಂದ. ಅವರು ಅಚಾಬಾಲ್​ನಲ್ಲಿನ ಉದ್ಯಾನಗಳಲ್ಲಿ ನಡೆದಾಡಿ, ಪುಷ್ಕರಿಣಿಯೊಂದರಲ್ಲಿ ಸ್ನಾನ ಮಾಡಿದರು. ಬಳಿಕ ಅಲ್ಲೇ ವನ ಭೋಜನ ಮಾಡಿ ಸಂಜೆಯ ಹೊತ್ತಿಗೆ ಅನಂತನಾಗಕ್ಕೆ ಹಿಂದಿರುಗಿದರು.

