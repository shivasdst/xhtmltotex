
\chapter{ವಿಚಾರ-ವಿಹಾರ}

\noindent

ಈಗ ಸ್ವಾಮೀಜಿ ತಮ್ಮ ಮುಂದಿನ ಕಾರ್ಯದತ್ತ ಗಮನ ಹರಿಸಬೇಕಾಗಿತ್ತು. ಅವರು ತಮ್ಮ ಪಾಶ್ಚಾತ್ಯ ಶಿಷ್ಯರನ್ನೆಲ್ಲ ಕರೆದುಕೊಂಡು ಹಿಮಾಲಯದ ಆಲ್ಮೋರಕ್ಕೆ ಹೋಗುವ ಯೋಜನೆ ಆಗಲೇ ಸಿದ್ಧವಾಗಿತ್ತು. ಈ ವೇಳೆಗೆ ಸೇವಿಯರ್ ದಂಪತಿಗಳು ಭಾರತದ ಪ್ರವಾಸವನ್ನು ಮುಗಿಸಿಕೊಂಡು ಆಲ್ಮೋರದಲ್ಲಿ ನೆಲಸಿದ್ದರು. ಅವರು ಸ್ವಾಮೀಜಿಯವರನ್ನು ತಮ್ಮಲ್ಲಿಗೆ ಬರುವಂತೆ ಆಹ್ವಾನಿಸುತ್ತಲೇ ಇದ್ದರು. ಅಲ್ಲದೆ ಶ್ರೀಮತಿ ಸಾರಾ ಬುಲ್ ಹಾಗೂ ಮಿಸ್ ಮೆಕ್​ಲಾಡ್​ರ ಕಾಶ್ಮೀರ ಪ್ರವಾಸದಲ್ಲಿ ತಾವು ಜೊತೆಗೊಡುವುದಾಗಿ ಸ್ವಾಮೀಜಿ ಮಾತುಕೊಟ್ಟಿ ದ್ದರು. ಆದ್ದರಿಂದ ಅವರು ೧೮೯೮ರ ಮೇ ೧೧ರಂದು ತಮ್ಮ ಬಹುದೊಡ್ಡ ಪರಿವಾರದೊಂದಿಗೆ ಆಲ್ಮೋರದತ್ತ ಹೊರಟರು. ಆ ಪರಿವಾರದಲ್ಲಿ ಇದ್ದವರೆಂದರೆ ತುರೀಯಾನಂದರು, ನಿರಂಜನಾ ನಂದರು, ಸದಾನಂದರು, ಸ್ವರೂಪಾನಂದರು, ಶ್ರೀಮತಿ ಬುಲ್, ಸೋದರಿ ನಿವೇದಿತಾ, ಮಿಸ್ ಮೆಕ್​ಲಾಡ್ ಹಾಗೂ ಶ್ರೀಮತಿ ಪ್ಯಾಟರ್​ಸನ್. ಶ್ರೀಮತಿ ಪ್ಯಾಟರ್​ಸನ್ ಕಲ್ಕತ್ತದಲ್ಲಿನ ಅಮೆರಿಕನ್ ರಾಯಭಾರಿಯ ಪತ್ನಿ. ಹಿಂದೆ ಸ್ವಾಮೀಜಿ ಅಮೆರಿಕದ ಬಾಲ್ಟಿಮೋರ್ ನಗರಕ್ಕೆ ಬಂದಿದ್ದಾಗ ಅವರು ಕಪ್ಪು ಬಣ್ಣದವರೆಂಬ ಕಾರಣದಿಂದ ಅಲ್ಲಿನ ಯಾವ ಹೋಟೆಲಿನಲ್ಲೂ ಅವರಿಗೆ ಪ್ರವೇಶ ಸಿಕ್ಕಿರಲಿಲ್ಲ. ಇದನ್ನು ತಿಳಿದು ಆಗ ಆ ನಗರದ ಓರ್ವ ಗಣ್ಯ ಮಹಿಳೆಯಾದ ಶ್ರೀಮತಿ ಪ್ಯಾಟರ್​ಸನ್ ಅವರನ್ನು ತನ್ನ ಮನೆಗೆ ಕರೆತಂದು ಸತ್ಕರಿಸಿದ್ದಳು. ಅಂದಿನಿಂದ ಅವಳು ಅವರ ವಿಶ್ವಾಸಿಗರಲ್ಲೊಬ್ಬಳಾಗಿದ್ದಳು. ಈಗ ಸ್ವಾಮೀಜಿ ಆಲ್ಮೋರದತ್ತ ಹೊರಟಿರುವರೆಂಬ ವರ್ತಮಾನ ಸಿಕ್ಕಿದೊಡನೆಯೇ ಈಕೆ ಅವರ ಪರಿವಾರವನ್ನು ಸೇರಿಕೊಂಡಳು. ಈಗಲೂ ಕೂಡ ಅವಳ ಈ ಕೃತ್ಯ ಕಲ್ಕತ್ತದ ಇಂಗ್ಲಿಷ್ ವಲಯಗಳಲ್ಲಿ ತೀವ್ರ ಟೀಕೆಗೆ ಗುರಿಯಾಯಿತು. ಆದರೆ ಸ್ವಾಮೀಜಿಯವರೊಂದಿಗೆ ಪ್ರಯಾಣ ಮಾಡುವುದೆಂದರೆ ಅವಳ ಪಾಲಿಗೊಂದು ಪರಮ ಸೌಭಾಗ್ಯದ ವಿಷಯ. ಆದ್ದರಿಂದ ಆಕೆ ಆ ಟೀಕೆಯನ್ನೆಲ್ಲ ಲೆಕ್ಕಿಸಲಿಲ್ಲ.

ಕಲ್ಕತ್ತದಿಂದ ಕಥಗೋಡಂವರೆಗೆ ಎರಡು ದಿನದ ರೈಲು ಪ್ರಯಾಣ; ಬಳಿಕ ಅಲ್ಲಿಂದ ನೈನಿ ತಾಲ್​ಗೆ ಗಾಡಿಯ ಮೂಲಕ ಪ್ರಯಾಣ. ಈ ಇಡೀ ಪ್ರಯಾಣ ಕಾಲದಲ್ಲಿ ಸ್ವಾಮೀಜಿಯವರು ಅಲ್ಲಲ್ಲಿನ ಸ್ಥಳಗಳ ವರ್ಣನೆ ಮಾಡುತ್ತಿದ್ದಾಗ ಅವರ ಅಸಾಧಾರಣ ಇತಿಹಾಸ ಪ್ರಜ್ಞೆ ಮತ್ತು ಅಗಾಧ ರಾಷ್ಟ್ರಪ್ರೇಮ ಸುವ್ಯಕ್ತವಾಗುತ್ತಿತ್ತು. ಇದರಿಂದ ಅವರ ಪರಿವಾರದವರ ಪಾಲಿಗೆ ಶೈಕ್ಷಣಿಕ ಅನುಭವವೂ ಆಯಿತು. ಮನರಂಜನೆಯೂ ಆಯಿತು. ನಿಲ್ದಾಣದಿಂದ ನಿಲ್ದಾಣಕ್ಕೆ ರೈಲು ಮುಂದುವರಿದಂತೆಲ್ಲ ಸ್ವಾಮೀಜಿಯವರು ದಾರಿಯಲ್ಲಿ ಸಿಗುವ ಕಾಶೀಕ್ಷೇತ್ರದ ಮಹಿಮೆ, ಪಾಟ್ನಾ- ಲಕ್ನೋ ನಗರಗಳ ನವಾಬರ ದರ್ಬಾರು–ಇವುಗಳನ್ನೆಲ್ಲ ಸ್ವಾರಸ್ಯವಾಗಿ ಬಣ್ಣಿಸಿದರು. ಅದನ್ನು ಕೇಳುತ್ತಿದ್ದವರಿಗೆಲ್ಲ ಅನ್ನಿಸಿತು, ಆ ವೈಭವಗಳನ್ನೆಲ್ಲ ಸ್ವಾಮೀಜಿ ಕಣ್ಣಾರೆ ಕಂಡಿದ್ದಾರೆ ಮತ್ತು ಈಗ ಈ ಕ್ಷಣದಲ್ಲೂ ಕಾಣುತ್ತಿದ್ದಾರೆ ಎಂದು. ಸ್ವಾಮೀಜಿಯವರಿಗೆ ಗೊತ್ತಿಲ್ಲದ ಸ್ಥಳವೇ ಇಲ್ಲ! ಅಷ್ಟೇ ಅಲ್ಲ, ಪ್ರತಿಯೊಂದು ಸ್ಥಳದ ಬಗ್ಗೆಯೂ ಅವರಲ್ಲೊಂದು ವಿಶೇಷ ಪ್ರೀತ್ಯಾದರ ಭಾವ! ಟ್ರೈನು ‘ತೆರೈ’ ಎಂಬ ಸ್ಥಳವನ್ನು ಹಾದುಹೋಗುತ್ತಿದ್ದಾಗ–ಅದು ಬುದ್ಧ ಭಗವಂತನು ಜೀವಿಸಿದ ಮತ್ತು ಸಾಧನೆ ಮಾಡಿದ ಸ್ಥಳವಾದ್ದರಿಂದ–ಆ ಕ್ಷೇತ್ರವನ್ನು ಸ್ವಾಮೀಜಿ ಹೇಗೆ ಬಣ್ಣಿಸಿದರೆಂದರೆ, ಅವರ ಸಂಗಡಿಗರ ಕಣ್ಮುಂದೆ ಬುದ್ಧನ ಆ ದಿನಗಳು ಜೀವಂತವಾಗಿ ಮೂಡಿನಿಂತುವು! ಅಲ್ಲಲ್ಲಿ ಗರಿಗೆದರಿ ಹಾರುವ ನವಿಲುಗಳನ್ನು ಕಂಡಾಗ, ಸ್ವಾಮೀಜಿಯವರಿಗೆ ವೀರ ರಜಪೂತರ ನೆನಪು ಒತ್ತರಿಸಿ ಬರುತ್ತಿತ್ತು! ನಾಲ್ಕಾರು ಒಂಟೆಗಳು ನಡೆದು ಹೋಗುತ್ತಿದ್ದರೆ ಗತಕಾಲದ ಕಾರವಾನ್ ಗಳ, ವರ್ತಕರ, ವ್ಯಾಪಾರಿಗಳ ಕತೆಗಳು ನೆನಪಾಗುತ್ತಿದ್ದವು! ದೂರದಲ್ಲೊಂದು ಆನೆ ಕಣ್ಣಿಗೆ ಬಿದ್ದರೆ, ಅಂದಿನ ರಾಜರ ಅಥವಾ ಮೊಘಲರ ಯಾವುದೋ ಯುದ್ಧದ ರಣಕಹಳೆಯ ಮೊಳಗುವಿಕೆ ಕೇಳಿಬಂದಂತಾಗುತ್ತಿತ್ತು! ಕೆಲವೊಮ್ಮೆ ಕ್ಷಾಮ ಡಾಮರಕ್ಕೆ ತುತ್ತಾದ ನತದೃಷ್ಟ ಪ್ರದೇಶಗಳೂ ಕಣ್ಣಿಗೆ ಬೀಳುತ್ತಿದ್ದವು. ಅಲ್ಲಲ್ಲಿ ಕಂಡುಬರುವ ವಿಶಾಲವಾದ ಹೊಲಗದ್ದೆಗಳು, ಪೈರು ಪಚ್ಚೆಗಳು, ಹಳ್ಳಿಗಳು–ಇವುಗಳನ್ನೆಲ್ಲ ನೋಡಿದಾಗ ಅಲ್ಲಿನ ಗೃಹಿಣಿಯರ ದೈನಂದಿನ ಜೀವನವೊ, ವ್ಯವಸಾಯ ಪದ್ಧತಿಗಳೊ ಅಥವಾ ಹಳ್ಳಿಯ ಮುಗ್ಧ ಜನರು ದಾರಿಹೋಕ ಸಾಧುಗಳ ಸತ್ಕಾರ ಮಾಡುವ ದೃಶ್ಯವೊ ಅವರ ಮನಃಪಟಲದ ಮೇಲೆ ಮೂಡುತ್ತಿದ್ದವು. ಸ್ವತಃ ಸ್ವಾಮೀಜಿ ಓರ್ವ ಬಡಸಂನ್ಯಾಸಿಯಾಗಿ ಈ ಎಲ್ಲ ಸ್ಥಳಗಳಲ್ಲಿ ಪಾದಯಾತ್ರೆ ಮಾಡಿದವರಲ್ಲವೆ? ಆ ಸಂದರ್ಭಗಳಲ್ಲಿ ಎಷ್ಟೋ ಹಳ್ಳಿಗಳ ಮುಗ್ಧ ಜನರು ಅವರನ್ನು ಆದರಿಸಿ ಸತ್ಕರಿಸಿರಲಿಲ್ಲವೆ? ಈಗ ಅಂತಹ ಹಳ್ಳಿಗಳನ್ನು ಕಂಡು ಆ ಜನರ ಸತ್ಕಾರಬುದ್ಧಿಯನ್ನು ಬಣ್ಣಿಸುವಾಗ ಸ್ವಾಮೀಜಿ ಯವರ ಕಣ್ಣುಗಳು ಹನಿಗೂಡಿ ಮಂಜಾದವು, ಸ್ವರ ಗದ್ಗದವಾಯಿತು!

ದಾರಿಯಲ್ಲಿ ಕಂಡುಬರುವ ವಿಶಾಲವಾದ ನದಿಗಳನ್ನು, ವಿಸ್ತಾರವಾದ ಅರಣ್ಯಗಳನ್ನು, ಉನ್ನತ ಪರ್ವತಗಳನ್ನು ಸ್ವಾಮೀಜಿ ತಮ್ಮ ಮಾತಿನ ಕುಂಚದಿಂದ ವರ್ಣಮಯಗೊಳಿಸುತ್ತಿದ್ದರು. ಭಾರತೀಯ ಸಂಸ್ಕೃತಿಯಲ್ಲಿ ಈ ನದಿ ಅರಣ್ಯ ಪರ್ವತಗಳದು ಬಹು ಮುಖ್ಯ ಪಾತ್ರ. ಅಂತೆಯೇ ಬಿಸಿಲಿಗೆ ಬೆಂದ ನೆಲ, ಮರುಭೂಮಿಯ ಸುಡುಮರಳು, ಒಣಗಿ ಬತ್ತಿಹೋದ ನದೀತಲ– ಇವುಗಳೂ ಅವರ ನೆನಪನ್ನು, ಆಲೋಚನೆಯನ್ನು ಪ್ರಚೋದಿಸುತ್ತಿದ್ದುವು. ಕೇಳುವವರು ಕುತೂ ಹಲಿಗಳಾಗಿದ್ದರೆ ಹೇಳುವವರ ಕೌಶಲ ದ್ವಿಗುಣವಾಗುವುದಲ್ಲವೆ? ಅವರ ಸಂಗಡಿಗರು– ಅದರಲ್ಲೂ ಮುಖ್ಯವಾಗಿ ಅವರ ಪಾಶ್ಚಾತ್ಯ ಶಿಷ್ಯೆಯರು–ಅವರಾಡುವ ಒಂದೊಂದು ಶಬ್ದ ವನ್ನೂ ಕಿವಿಯರಳಿಸಿಕೊಂಡು ತದೇಕಚಿತ್ತದಿಂದ ಕೇಳುತ್ತಿದ್ದರೆ, ಸ್ವಾಮೀಜಿಯವರ ಭಾವದಾಳ ದಿಂದ ಜ್ಞಾನವಾಹಿನಿ ಉಕ್ಕಿ ಹರಿಯುತ್ತಿತ್ತು; ಭಾರತದ ಮೇಲಿನ ಅವರ ಅನುರಾಗ ಅಧಿಕವಾಗಿ ವ್ಯಕ್ತಗೊಳ್ಳುತ್ತಿತ್ತು. ಅವರ ವರ್ಣನಾ ಶೈಲಿಯಲ್ಲಿ ಕವಿತ್ವ ಸ್ಫುರಿಸುತ್ತಿತ್ತು. ಭಾರತದಲ್ಲಿ ಹೇಗೆ ಸಂಸ್ಕೃತಿ, ಸಂಪ್ರದಾಯ ಹಾಗೂ ಧರ್ಮ–ಇವು ಮೂರೂ ಒಂದೇ ಆಗಿವೆಯೆಂಬುದನ್ನು ಸ್ವಾಮೀಜಿ ನಿರೂಪಿಸಿದರು. ಅವರ ಮಾತುಕತೆಯ ಸಂದರ್ಭದಲ್ಲಿ ಪ್ರಸ್ತಾಪಕ್ಕೆ ಬರುತ್ತಿದ್ದ ವಿಷಯಗಳು ಅಸಂಖ್ಯಾತ. ಸ್ಮಶಾನವೆಂಬುದು ಏಕೆ ಮೈಲಿಗೆಯ ಸ್ಥಳ; ಊಟಕ್ಕೆ ಪೂಜೆಗೆ ಜಪಕ್ಕೆ ಬಲಗೈಯನ್ನೇ ಏಕೆ ಉಪಯೋಗಿಸಬೇಕು; ಹಿಂದೂ ವಿಧವೆಯರು ಉಪವಾಸ-ಜಾಗರಣೆಯೇ ಮೊದಲಾದ ವ್ರತಗಳನ್ನಾಚರಿಸುತ್ತ ಸಂನ್ಯಾಸಿನಿಯರಂತೆ ಜೀವಿಸುವುದರ ಅರ್ಥವೇನು, ವರ್ಣಾ ಶ್ರಮ ಧರ್ಮವೆಂದರೇನು, ಇತ್ಯಾದಿಯಾಗಿ ಅನೇಕಾನೇಕ ವಿಷಯಗಳನ್ನು ಸ್ವಾಮೀಜಿ ವಿವರಿಸಿ ದರು. ಹೀಗೆ ಈ ಪ್ರಯಾಣಕಾಲವು ಶಿಷ್ಯೆಯರಿಗೆ ಅತ್ಯಂತ ಬೋಧಪ್ರದವಾಗಿತ್ತು. ಇವರ ಪೈಕಿ ಸ್ವಾಮೀಜಿ ಅತಿ ಹೆಚ್ಚಿನ ಗಮನ ನೀಡುತ್ತಿದ್ದುದು ಸೋದರಿ ನಿವೇದಿತೆಯ ಕಡೆಗೆ, ಅವರ ಕರೆಗೆ ಓಗೊಟ್ಟು ಭಾರತದ ಸೇವೆಗಾಗಿ ತನ್ನ ಸರ್ವಸ್ವವನ್ನರ್ಪಿಸುವ ಸಲುವಾಗಿ ತನ್ನ ದೇಶ ಮನೆ ಬಂಧುಗಳೆಲ್ಲರನ್ನೂ ತ್ಯಜಿಸಿ ಬಂದವಳಲ್ಲವೆ ಅವಳು? ಆದ್ದರಿಂದ ಅವಳಿಗೆ ‘ಮರುಶಿಕ್ಷಣ’ ನೀಡಿ, ಅವಳನ್ನು ಅತ್ಯಂತ ಸಮರ್ಥ ವ್ಯಕ್ತಿಯನ್ನಾಗಿ ರೂಪಿಸುವುದು ಅವರ ಉದ್ದೇಶ. ಆದ್ದ ರಿಂದಲೇ ಈ ಎಲ್ಲ ಸಂಭಾಷಣೆ ವಿವರಣೆಗಳು.

ಸ್ವಾಮೀಜಿಯವರ ತಂಡದವರು ಮೇ ೧೩ರಂದು ನೈನಿತಾಲಿಗೆ ತಲುಪಿದರು. ನೈನಿತಾಲ್ ಅತ್ಯಂತ ರಮ್ಯ ಪ್ರದೇಶ, ಸೆಕೆಗಾಲದ ಸ್ವರ್ಗ, ಹಣವಂತರ ವಿಶ್ರಾಂತಿ ಧಾಮ. ಖೇತ್ರಿಯ ಮಹಾರಾಜನೂ ಆ ವೇಳೆಯಲ್ಲಿ ನೈನಿತಾಲಿನಲ್ಲೇ ಇದ್ದ. ಆತನ ಅಪೇಕ್ಷೆಯಂತೆ ಸ್ವಾಮೀಜಿ ಇಲ್ಲಿ ಕೆಲದಿನ ನಿಲ್ಲಲು ನಿರ್ಧರಿಸಿದ್ದರು. ಅಲ್ಲಿನ ನಾಗರಿಕರು ಸ್ವಾಮೀಜಿಯವರನ್ನು ಅತ್ಯುತ್ಸಾಹದಿಂದ ಬರಮಾಡಿಕೊಂಡರು. ಅವರನ್ನು ಬೆಟ್ಟದ ಕುದುರೆಯ ಮೇಲೆ ಕುಳ್ಳಿರಿಸಿ, ಹೂಗಳನ್ನೂ ತಾಳೆಯ ಗರಿಗಳನ್ನೂ ಹರಡಿದ ದಾರಿಯಲ್ಲಿ ಕರೆದೊಯ್ದರು. (ಹಿಂದೆ ಏಸುಕ್ರಿಸ್ತನನ್ನು ಜೆರೂಸಲೆಮ್ಮಿಗೆ ಕರೆದೊಯ್ದಾಗಲೂ ಇಂಥದೇ ಸ್ವಾಗತ ನೀಡಲಾಗಿತ್ತು.) ಸ್ವಾಮೀಜಿ ತಮ್ಮ ಪಾಶ್ಚಾತ್ಯ ಶಿಷ್ಯೆಯ ರನ್ನು ಖೇತ್ರಿಯ ಮಹಾರಾಜನಿಗೆ ಪರಿಚಯಿಸಿಕೊಟ್ಟರು. ಬಳಿಕ ಸ್ವಾಮೀಜಿ ತಮ್ಮ ಸಂಗಡಿಗರನ್ನು ಅವರಷ್ಟಕ್ಕೆ ಸ್ವತಂತ್ರವಾಗಿರಲು ಬಿಟ್ಟು, ತಾವು ಏಕಾಂತದಲ್ಲಿರಲು ಇಷ್ಟಪಟ್ಟರು. ಅವರ ಪರಿವಾರದವರೂ ಕೂಡ ತಾವು ಸ್ವಾಮೀಜಿಯವರಿಗೆ ಹೊರೆಯಾಗದಂತೆ ಬೇರೊಂದೆಡೆಯಲ್ಲಿ ಇಳಿದುಕೊಂಡರು. ಮೂರು ದಿನಗಳ ಬಳಿಕ ಸ್ವಾಮೀಜಿ ತಮ್ಮ ಸಂಗಡಿಗರನ್ನು ತಾವಿರುವಲ್ಲಿಗೆ ಕರೆಸಿಕೊಂಡರು.

ನೈನಿತಾಲಿನಲ್ಲಿ ಒಂದು ಸ್ವಾರಸ್ಯಕರ ಘಟನೆ ನಡೆಯಿತು. ಇಲ್ಲಿನ ಮೊಹಮ್ಮದ್ ಸರ್ಫ್​ರಾಜ್ ಹುಸೇನ್ ಎಂಬ ಮುಸಲ್ಮಾನ ಸ್ವಾಮೀಜಿಯವರ ದರ್ಶನಕ್ಕಾಗಿ ಬಂದ. ಈತನಿಗೆ ಅದ್ವೈತ ವೇದಾಂತದಲ್ಲಿ ವಿಶೇಷ ಒಲವು. ಸ್ವಾಮೀಜಿಯವರನ್ನು ಕಂಡು ಮಾತನಾಡಿದ ಈತ, ಅವರ ವ್ಯಕ್ತಿತ್ವದಿಂದ ಹಾಗೂ ಆಧ್ಯಾತ್ಮಿಕ ಶಕ್ತಿಯ ತೇಜಸ್ಸಿನಿಂದ ಗಾಢವಾಗಿ ಪ್ರಭಾವಿತನಾದ. ಎಷ್ಟರ ಮಟ್ಟಿಗೆಂದರೆ, ಅವನು ಉದ್ಗರಿಸುತ್ತಾನೆ, “ಸ್ವಾಮೀಜಿ, ಮುಂದೆ ನಿಮ್ಮ ಕಾಲಾನಂತರ ಯಾರಾ ದರೂ ನಿಮ್ಮನ್ನು ಅವತಾರಪುರುಷರೆಂದು ಕರೆಯುವುದಾದರೆ–ನೆನಪಿರಲಿ ಸ್ವಾಮೀಜಿ–ಹಾಗೆ ಕರೆಯುವವರಲ್ಲಿ ಮುಸಲ್ಮಾನನಾದ ನಾನೇ ಮೊದಲಿಗ!” ಈ ಮುಸಲ್ಮಾನ ವೇದಾಂತಿ, ಅಂದಿನಿಂದ ಅವರನ್ನು ತನ್ನ ಗುರುವೆಂದು ಭಾವಿಸಿದ. ತನ್ನನ್ನು ಅವರ ಶಿಷ್ಯನೆಂದು ಹೇಳಿಕೊಳ್ಳು ತ್ತಿದ್ದನಲ್ಲದೆ ತನ್ನ ಹೆಸರನ್ನು ಮೊಹಮ್ಮದಾನಂದ ಎಂದು ಬದಲಾಯಿಸಿಕೊಂಡ. ಕೆಲದಿನಗಳಲ್ಲಿ ನೈನಿತಾಲಿನಿಂದ ಆಲ್ಮೋರಕ್ಕೆ ಹೋದ ಸ್ವಾಮೀಜಿಯವರು ಅವನ ಒಂದು ಪತ್ರಕ್ಕೆ ಉತ್ತರವಾಗಿ ಹೀಗೆ ಬರೆಯುತ್ತಾರೆ:

“... ನಾವದನ್ನು ವೇದಾಂತ ತತ್ತ್ವ ಎಂದಾದರೂ ಕರೆಯಬಹುದು, ಇನ್ನಾವುದೇ ತತ್ತ್ವ ವೆಂದಾದರೂ ಕರೆಯಬಹುದು; ಆದರೆ ಸತ್ಯಸಂಗತಿಯೇನೆಂದರೆ ಅದ್ವೈತವೇ ಧರ್ಮ ಹಾಗೂ ತತ್ತ್ವಶಾಸ್ತ್ರಗಳ ಕೊನೆಯ ಮಾತು. ಮತ್ತು ಈ ಅದ್ವೈತದ ದೃಷ್ಟಿಕೋನದಿಂದ ಮಾತ್ರವೇ ನಾವು ಜಗತ್ತಿನ ಸಮಸ್ತ ಮತಧರ್ಮಗಳನ್ನೂ ಪ್ರೀತಿ ವಿಶ್ವಾಸಗಳಿಂದ ಕಾಣಲು ಸಾಧ್ಯ... ನನಗೆ ಚೆನ್ನಾಗಿ ಮನವರಿಕೆಯಾಗಿದೆ, ಏನೆಂದರೆ–ಈ ವೇದಾಂತಸಿದ್ಧಾಂತಗಳೆಂಬುದು ಎಷ್ಟೇ ಉತ್ಕೃಷ್ಟ ವಾಗಿರಬಹುದು, ಅದ್ಭುತವಾಗಿರಬಹುದು; ಆದರೆ ಇಸ್ಲಾಂ ಧರ್ಮದ ತತ್ತ್ವಗಳನ್ನು ಬಳಸಿ ಕೊಂಡು ಅಳವಡಿಸಿಕೊಳ್ಳದೆ ಹೋದರೆ ಈ ಸಿದ್ಧಾಂತಗಳಿಂದ ಮಾನವ ಜನಾಂಗಕ್ಕೆ ಯಾವ ಪ್ರಯೋಜನವೂ ಇಲ್ಲ, ಎಂದು... ಅಲ್ಲದೆ ನಮ್ಮ ಮಾತೃಭೂಮಿಯ ಶ್ರೇಯಸ್ಸಿಗೂ ಈ ಎರಡು ಮಹಾ ತತ್ತ್ವಗಳಾದ ಹಿಂದೂ ಧರ್ಮ ಮತ್ತು ಇಸ್ಲಾಂ ಧರ್ಮ–ವೇದಾಂತದ ಬುದ್ಧಿಮತ್ತೆ ಮತ್ತುಇಸ್ಲಾಂ ಸಹೋದರತೆ–ಇವೆರಡೂ ಕೂಡಿಕೊಳ್ಳಬೇಕು. ಇದೊಂದೇ ದಾರಿ....”

ಸ್ವಾಮೀಜಿಯವರ ಆಂತರ್ಯದಲ್ಲಿ ರಾಷ್ಟ್ರಪ್ರೇಮದ ಭಾವ ನಿರಂತರವಾಗಿ ತುಡಿಯುತ್ತಿತ್ತು. ಯಾವುದಾದರೊಂದು ಸಣ್ಣ ಪ್ರಚೋದನೆ ದೊರೆತರೂ ಅದು ಅಪ್ರಯತ್ನವಾಗಿ ವ್ಯಕ್ತವಾಗು ತ್ತಿತ್ತು; ಬಳಿಯಿದ್ದವರನ್ನು ವಿಸ್ಮಿತರನ್ನಾಗಿಸುತ್ತಿತ್ತು. ಅವರು ನೈನಿತಾಲಿನಲ್ಲಿದ್ದಾಗ ಇಂಥ ದೊಂದು ಘಟನೆ ನಡೆಯಿತು. ಸ್ವಾಮೀಜಿಯವರ ಶಾಲಾ ಸಹಪಾಠಿಯಾಗಿದ್ದ ಜೋಗೇಶ್​ಚಂದ್ರ ದತ್ತ ಎಂಬವರು ಒಮ್ಮೆ ಅವರನ್ನು ಭೇಟಿಯಾಗಲು ಬಂದರು. ಮಾತಿನ ಸಂದರ್ಭದಲ್ಲಿ ಇವರು ಹೇಳಿದರು–ಭಾರತದ ಬುದ್ಧಿವಂತ ಪದವೀಧರರು ಇಂಗ್ಲೆಂಡಿಗೆ ಹೋಗಿ ಸಿವಿಲ್ ಸರ್ವಿಸ್ ಪರೀಕ್ಷೆಗೆ ಶಿಕ್ಷಣ ಪಡೆಯಲು ಸಾಧ್ಯವಾಗುವಂತೆ ಅವರಿಗಾಗಿ ನಿಧಿ ಸಂಗ್ರಹ ಮಾಡಬೇಕು; ಹೀಗೆ ಶಿಕ್ಷಣ ಪಡೆದು ಬಂದ ಯುವಕರು ಉನ್ನತ ಅಧಿಕಾರಿಗಳಾಗಿ ಭಾರತದ ಆರ್ಥಿಕ ಉನ್ನತಿಗಾಗಿ ಶ್ರಮಿಸಬಲ್ಲರು, ಎಂದು. ಆದರೆ ಸ್ವಾಮೀಜಿ ತಕ್ಷಣ ಈ ಸಲಹೆಯನ್ನು ತಳ್ಳಿಹಾಕಿ ಬಿಟ್ಟರು. ಮೇಲ್ನೋಟಕ್ಕೇನೋ ಈ ಯೋಜನೆ ಬಹಳ ಅದ್ಭುತವಾಗಿ ತೋರಬಹುದು; ಆದರೆ ಹಾಗೆ ಶಿಕ್ಷಣ ಪಡೆದು ಅಧಿಕಾರಿಗಳಾದವರು ಭಾರತಕ್ಕಾಗಿ ಶ್ರಮಿಸುವುದಿರಲಿ, ಭಾರತೀಯರ ಹತ್ತಿರವೂ ಬರದೆ ಐರೋಪ್ಯರೊಂದಿಗೆ ಬೆರೆತು ಐರೋಪ್ಯರಂತೆಯೇ ಆಗಿಬಿಡುತ್ತಾರೆ; ಅಲ್ಲದೆ ಅವರು ಸಂಪೂರ್ಣ ಸ್ವಾರ್ಥಿಗಳೇ ಆಗುವುದು ಖಂಡಿತ ಎಂದು ಸ್ವಾಮೀಜಿ ನುಡಿದರು. ಬಳಿಕ ಆರ್ಥಿಕ-ಔದ್ಯೋಗಿಕ- ಕೈಗಾರಿಕಾ ಕ್ಷೇತ್ರಗಳಲ್ಲಿ ಭಾರತವು ಅತ್ಯಂತ ಹಿಂದುಳಿದಿದ್ದರ ವಿಷಯವಾಗಿ ಮಾತನಾಡುತ್ತ ಸ್ವಾಮೀಜಿ ತುಂಬ ದುಃಖದಿಂದ ಹೇಳುತ್ತಾರೆ, “ಭಾರತದ ಐಹಿಕ ಅಭಿವೃದ್ಧಿಯ ಬಗ್ಗೆ ಭಾರತೀ ಯರೇ ಸಂಪೂರ್ಣ ಉದಾಸೀನರಾಗಿ ಕುಳಿತಿದ್ದಾರೆ. ಅವರಲ್ಲಿ ಉತ್ಸಾಹಪೂರ್ಣವಾದ ಉದ್ಯಮ ಶೀಲತೆಯೇ ಕಂಡುಬರುತ್ತಿಲ್ಲ. ಅಯ್ಯೋ! ಭಾರತೀಯರಿಗೆ ತಾವು ಮೇಲೇಳಬೇಕು, ಇತರ ರಂತೆಯೇ ಮುನ್ನಡೆಯಬೇಕು ಎಂಬ ಆಸೆಯೇ ಇಲ್ಲವಲ್ಲ!” ಎಂದು. ಇದು ಯಾರಾದರೂ ಹೇಳಬಹುದಾದ ಮಾತೇ ಸರಿ. ಆದರೆ ಆಶ್ಚರ್ಯದ ಸಂಗತಿಯೇನೆಂದರೆ, ಹೀಗೆ ಹೇಳುವಾಗ ಸ್ವಾಮೀಜಿಯವರ ಕಣ್ಣಿನಿಂದ ನೀರು ಹರಿಯುತ್ತಿತ್ತು! ಇದನ್ನು ಕಂಡವರ ಹೃದಯ ಕರಗಿ ಹೋಯಿತು. ಈ ವಿಷಯವಾಗಿ ಜೋಗೇಶ್​ಚಂದ್ರ ದತ್ತ ಬರೆಯುತ್ತಾರೆ, “ನಾನು ಆ ದೃಶ್ಯವನ್ನು ನನ್ನ ಜನ್ಮದಲ್ಲೇ ಮರೆಯಲಾರೆ! ಅವರಾದರೋ ತ್ಯಾಗಿಗಳು; ಪ್ರಪಂಚವನ್ನು ತ್ಯಜಿಸಿದವರು. ಆದರೆ ಭಾರತವು ಅವರ ಪ್ರಾಣದ ಪ್ರಾಣವಾಗಿತ್ತು. ಭಾರತ ಅವರ ಪ್ರೇಮದ ವಸ್ತು. ಭಾರತಕ್ಕಾಗಿ ಅವರು ಭಾವಿಸಿದರು, ಭಾರತಕ್ಕಾಗಿ ಕಣ್ಣೀರ್ಗರೆದರು, ಭಾರತಕ್ಕಾಗಿ ಮಡಿದರು. ಅವರ ಹೃದಯದ ಬಡಿತದಲ್ಲಿ ಭಾರತ, ನಾಡಿಗಳ ಮಿಡಿತದಲ್ಲಿ ಭಾರತ! ಅವರ ಜೀವಿತದಲ್ಲಿ ಭಾರತವು ಬಿಡಿಸಲಾರದಂತೆ ಬೆರೆತಿತ್ತು.”

ರಾಷ್ಟ್ರಪ್ರೇಮವೆಂದರೆ ರಾಷ್ಟ್ರದಲ್ಲಿ ವಾಸಿಸುವ ಮಾನವರ ಮೇಲಿನ ಪ್ರೇಮವೇ ತಾನೆ? ಸ್ವಾಮೀಜಿಯವರ ಅಗಾಧ ಮಾನವಪ್ರೇಮವನ್ನು ಅಥವಾ ಮಾನವತಾ ಪ್ರೇಮವನ್ನು ಅವರ ಜೀವನದಲ್ಲಿ ನಾವು ಮತ್ತೆಮತ್ತೆ ಕಾಣಬಹುದು. ಅವರ ವ್ಯಕ್ತಿತ್ವದ ಈ ಅಂಶವನ್ನು ತೋರು ವಂತಹ ಒಂದು ಘಟನೆ ಇಲ್ಲಿಯೂ ನಡೆಯಿತು. ಒಮ್ಮೆ ಅವರ ದರ್ಶನಕ್ಕಾಗಿ ಅಲ್ಲಿನ ಇಬ್ಬರು ಮಹಿಳೆಯರು ಬಂದರು. ಆದರೆ ಅವರಿಬ್ಬರೂ ವೇಶ್ಯೆಯರು. ಇದನ್ನು ತಿಳಿದಿದ್ದ ಇತರ ಸಂದರ್ ಶಕರು ಅವರನ್ನು ಕಂಡೊಡನೆ ಅವರಿಗೆ ಛೀಮಾರಿ ಹಾಕಿ ಓಡಿಸಲು ನೋಡಿದರು. ಆದರೆ, ವಿಷಯವೇನೆಂದು ಕೇಳಿ ತಿಳಿದ ಸ್ವಾಮೀಜಿಯವರು ಕೂಡಲೇ ಅವರನ್ನು ಒಳಗೆ ಬರಮಾಡಿ ಕೊಂಡರು. ಇದರಿಂದ ಅಲ್ಲೊಂದು ಪ್ರತಿಭಟನೆಯ ಬಿರುಗಾಳಿಯೇ ಎದ್ದುಬಿಟ್ಟಿತು. ಸ್ವಾಮೀಜಿ ಯವರು ವೇಶ್ಯೆಯರಿಗೆ ದರ್ಶನಾವಕಾಶ ಕೊಟ್ಟದ್ದು ಸರಿಯಲ್ಲವೇ ಅಲ್ಲ ಎಂದು ಆ ಮಡಿ ವಂತರು ವಾದಿಸಿದರು. ಆದರೆ ಸ್ವಾಮೀಜಿ ಇದಾವುದಕ್ಕೂ ಗಮನ ಕೊಡದೆ ಆ ಸ್ತ್ರೀಯರೊಡನೆ ಅತ್ಯಂತ ಕರುಣೆಯಿಂದ ಮಾತನಾಡಿ, ಅವರಲ್ಲಿ ನವಶಕ್ತಿ ತುಂಬುವಂತಹ ಸಾಂತ್ವನದ ಸಂದೇಶ ನೀಡಿದರು. ಸ್ವಾಮೀಜಿಯವರ ಮಾತಿನಲ್ಲಿ ಒಂದೇ ಒಂದು ಬೈಗುಳದ ಪದವಿಲ್ಲ, ಅಸಹನೆಯ ದನಿಯಿಲ್ಲ. ಇಬ್ಬರನ್ನೂ ಅವರು ಹೃತ್ಪೂರ್ವಕವಾಗಿ ಆಶೀರ್ವದಿಸಿ ಕಳಿಸಿಕೊಟ್ಟರು. ಅವರಾಡಿದ ಸಾಂತ್ವನದ ಮಾತುಗಳನ್ನು ಕೇಳಿ ಅಲ್ಲಿ ಕುಳಿತಿದ್ದವರ ಎದೆ ಕರಗಿಹೋಯಿತು.

ನೈನಿತಾಲಿನಲ್ಲಿ ಮೂರು ದಿನಗಳನ್ನು ಕಳೆದ ಸ್ವಾಮೀಜಿ, ಪರಿವಾರಸಹಿತರಾಗಿ ಆಲ್ಮೋರಕ್ಕೆ ಹೊರಟರು. ಇದು ಒಟ್ಟು ೩೨ ಮೈಲಿಗಳ ಕಾಲ್ನಡಿಗೆಯ ಪ್ರಯಾಣ. ಪರ್ವತಗಳ ತಪ್ಪಲಿನ ಬಳಸು ದಾರಿಯನ್ನು ಸುತ್ತಿಕೊಂಡು ಹೋಗಬೇಕು. ಸುತ್ತಲೂ ಹಿಮಾವೃತ ಶಿಖರಗಳು, ದಟ್ಟ ವನರಾಶಿ, ಅಲ್ಲಲ್ಲಿ ಜುಳುಜುಳು ಹರಿಯುವ ಒರತೆಗಳು, ಎಲ್ಲಿಂದಲೋ ಆಗಾಗ ಕೇಳಿಬರುವ ಪ್ರಾಣಿಪಕ್ಷಿ ಗಳ ಕೂಗು. ಅದೊಂದು ರೋಮಾಂಚಕಾರಿ ಅನುಭವ. ಎರಡನೆಯ ದಿನ ಸಂಜೆ ಆಲ್ಮೋರವನ್ನು ತಲುಪಬೇಕಾಗಿತ್ತು. ಆದರೆ ದಟ್ಟವಾದ ಅರಣ್ಯದಲ್ಲಿ ಸಾಗುತ್ತಿದ್ದಾಗಲೇ ಸೂರ್ಯ ಮುಳುಗಿದ; ಸ್ವಲ್ಪಹೊತ್ತಿನಲ್ಲೇ ಕತ್ತಲಾವರಿಸಿತು. ಆದರೆ ಸುರಕ್ಷಿತ ಸ್ಥಳವೊಂದನ್ನು ತಲುಪುವವರೆಗೂ ನಿಲ್ಲುವಂತಿಲ್ಲವಲ್ಲ? ಆದ್ದರಿಂದ ಲಾಟೀನು ಪಂಜು ಹೊತ್ತಿಸಿಕೊಂಡು ಕತ್ತಲಲ್ಲೇ ಮುನ್ನಡೆ ದರು. ಮುಖ್ಯವಾಗಿ ಕರಡಿ ಹುಲಿಗಳನ್ನು ದೂರವಿಡಲು ಈ ವ್ಯವಸ್ಥೆ. ಮಿನುಗುವ ಚುಕ್ಕೆಗಳಿಂದ ಕೂಡಿದ ಶುಭ್ರ ನೀಲಾಕಾಶ. ಆ ಬೆಳಕಿನಲ್ಲಿ ರುದ್ರ-ಗಂಭೀರ ಭಾವದಿಂದ ತಲೆಯೆತ್ತಿ ನಿಂತಿರುವ ಹಿಮವಂತನ ಶಿಖರಗಳ ವೈಭವವನ್ನು ಕಂಡು ಎಲ್ಲರೂ ಮೈ ಮರೆತರು. ಆ ಪರ್ವತದ ಮೌನವನ್ನು ಮುರಿಯುವಂತೆ ಆಗಾಗ ಮಾತು-ನಗು. ಸ್ವಾಮೀಜಿಯವರು ತಮ್ಮ ಹಾಸ್ಯ-ತಮಾಷೆ ಗಳಿಂದ ಎಲ್ಲರನ್ನೂ ಸಂತೋಷವಾಗಿ ಇಟ್ಟಿದ್ದರು. ಹೀಗೆಯೇ ಬಹಳ ದೂರ ಸಾಗಿ ಕಡೆಗೊಂದು ಪ್ರವಾಸೀ ಮಂದಿರವನ್ನು ತಲುಪಿದರು. ತಾವು ಹೊತ್ತುತಂದ ಆಹಾರ ಸಾಮಗ್ರಿಗಳಿಂದ ತಾವೇ ಅಡಿಗೆ ಮಾಡಿಕೊಂಡರು. ಅಡಿಗೆಯ ಮೇಲ್ವಿಚಾರಣೆಗೆ ಸ್ವಾಮೀಜಿಯವರೇ ನಿಂತರು. ಆ ರಾತ್ರಿ ಯನ್ನು ಅಲ್ಲೇ ಕಳೆದು, ಮರುದಿನ ಹೊರಟು ಆಲ್ಮೋರವನ್ನು ತಲುಪಿದರು.

ಆಲ್ಮೋರದಲ್ಲಿ ಸೇವಿಯರ್ ದಂಪತಿಗಳು ಎಲ್ಲರನ್ನೂ ಬರಮಾಡಿಕೊಂಡರು. ಸ್ವಾಮೀಜಿ ಯವರೂ ಇತರ ಸಂನ್ಯಾಸಿಗಳೂ ಸೇವಿಯರ್ ದಂಪತಿಗಳೊಂದಿಗೆ ಅವರಿದ್ದ ಮನೆಯಲ್ಲೇ ಇಳಿದುಕೊಂಡರು. ಪಾಶ್ಚಾತ್ಯ ಶಿಷ್ಯೆಯರಾದ ಶ್ರೀಮತಿ ಸಾರಾ ಬುಲ್, ಮಿಸ್ ಮೆಕ್​ಲಾಡ್, ಸೋದರಿ ನಿವೇದಿತಾ ಹಾಗೂ ಶ್ರೀಮತಿ ಪ್ಯಾಟರ್​ಸನ್ನರು ಬಳಿಯಲ್ಲಿನ ಇನ್ನೊಂದು ಮನೆಯಲ್ಲಿ ಇಳಿದುಕೊಂಡರು. ಸ್ವಾಮೀಜಿ ಇಲ್ಲಿಗೆ ಬಂದದ್ದು ಮುಖ್ಯವಾಗಿ ಸ್ವಲ್ಪ ವಿಶ್ರಾಂತಿ ಪಡೆದು, ತಮ್ಮ ಹದಗೆಡುತ್ತಿದ್ದ ಆರೋಗ್ಯವನ್ನು ಸುಧಾರಿಸಿಕೊಳ್ಳಲು. ಆದರೆ ಅದಕ್ಕಿಂತ ಹೆಚ್ಚಾಗಿ ಅವರು ಇಲ್ಲಿದ್ದ ಒಂದು ತಿಂಗಳ ಅವಧಿಯಲ್ಲಿ ನಡೆಸಿದ ಅತಿ ಮುಖ್ಯ ಕಾರ್ಯವೆಂದರೆ ತಮ್ಮ ಶಿಷ್ಯೆಯರಿಗೆ– ಅದರಲ್ಲೂ ಸೋದರಿ ನಿವೇದಿತೆಗೆ–ತೀವ್ರ ತರಬೇತಿಯನ್ನು ನೀಡಿದ್ದು. ವಿವೇಕಾನಂದರ ಆಧ್ಯಾ ತ್ಮಿಕ ಪುತ್ರಿಯೆಂಬ ಹೆಸರು ಗಳಿಸಿದ ನಿವೇದಿತೆ ನಿರ್ಮಾಣಗೊಂಡದ್ದು ಇಲ್ಲಿಯೇ ಎಂದೂ ಹೇಳಬಹುದು. ಸ್ವಾಮೀಜಿಯವರು ಪ್ರತಿದಿನವೂ ಬೆಳಿಗ್ಗೆ ತಮ್ಮ ಶಿಷ್ಯೆಯರೊಂದಿಗೆ ಬಹಳ ಹೊತ್ತು ಮಾತುಕತೆ ನಡೆಸುತ್ತಿದ್ದರು–ತನ್ಮೂಲಕ ಶಿಕ್ಷಣ ನೀಡುತ್ತಿದ್ದರು. ಅವರ ಈ ಎಲ್ಲ ಮಾತುಗಳನ್ನೂ ಎಚ್ಚರಿಕೆಯಿಂದ ಬರೆದಿಟ್ಟುಕೊಂಡ ನಿವೇದಿತಾ, ಮುಂದೆ ಇವುಗಳನ್ನು ತನ್ನ ಭಾಷಣಗಳ ಹಾಗೂ ಕೃತಿಗಳ ಮೂಲಕ ಪ್ರಕಟಪಡಿಸಿದಳು. ಇವುಗಳ ಪೈಕಿ ಮುಖ್ಯವಾದುದೆಂದರೆ \eng{\textit{Notes of Some Wanderings With Swami Vivekananda}} (ಸ್ವಾಮಿ ವಿವೇಕಾನಂದ ರೊಂದಿಗಿನ ಕೆಲವು ಪ್ರಯಾಣಗಳ ಟಿಪ್ಪಣಿಗಳು). ತತ್ಪರಿಣಾಮವಾಗಿ ಸ್ವಾಮೀಜಿಯವರ ಬೋಧನೆಯ ಹಾಗೂ ವ್ಯಕ್ತಿತ್ವದ ಹಲವಾರು ಅಂಶಗಳು ಭಾರತದಾದ್ಯಂತ ಪ್ರಸಾರವಾದುವಲ್ಲದೆ, ಭಾರತೀಯರಲ್ಲಿ ರಾಷ್ಟ್ರಪ್ರಜ್ಞೆ ಜಾಗೃತವಾಗುವಲ್ಲಿ ತುಂಬ ಸಹಾಯಕವಾದುವು.

ನಿವೇದಿತೆಗೆ ಸ್ವಾಮೀಜಿ ಸ್ವತಃ ತಮ್ಮ ಮಾತುಗಳ ಮೂಲಕ ಶಿಕ್ಷಣ ನೀಡುತ್ತಿದ್ದರಲ್ಲದೆ ಇತರ ವಿಷಯಗಳಲ್ಲೂ ಆಕೆಗೆ ತರಬೇತಿ ಸಿಗುವ ವ್ಯವಸ್ಥೆ ಮಾಡಿದ್ದರು. ಆಕೆ ಕಲ್ಕತ್ತಕ್ಕೆ ಬಂದ ಕೆಲದಿನಗಳಲ್ಲೇ, ಆಕೆಗೆ ಬಂಗಾಳೀ ಭಾಷೆಯನ್ನೂ ಲಿಪಿಯನ್ನೂ ಕಲಿಸಲು ಮೊದಲಿಗೆ ಮಾಸ್ಟರ್ ಮಹಾಶಯರನ್ನು, ಬಳಿಕ ತಮ್ಮ ಹೊಸ ಶಿಷ್ಯರಾದ ಸ್ವಾಮಿ ಸ್ವರೂಪಾನಂದರನ್ನು ನೇಮಿಸಿದ್ದರು. ಈಗ ಆಲ್ಮೋರದಲ್ಲಿ ಸ್ವಾಮೀಜಿಯವರ ಆಣತಿಯ ಮೇರೆಗೆ ಸ್ವಾಮಿ ಸ್ವರೂಪಾನಂದರು ನಿವೇದಿತೆಗೆ ಹಿಂದೂ ಶಾಸ್ತ್ರಗಳನ್ನೂ ಬೋಧಿಸಲಾರಂಭಿಸಿದರು. ಅವಳಿಗೆ ಧ್ಯಾನದ ವಿಧಾನವನ್ನು ತಿಳಿಸಿಕೊಟ್ಟು ಸಾಧನೆಯಲ್ಲಿ ನೆರವಾದರು. ಅಲ್ಲದೆ ಭಗವದ್ಗೀತೆಯ ಅಧ್ಯಯನದಲ್ಲೂ ನೆರವಾ ದರು. ಇವೆಲ್ಲವೂ ಆಕೆಗೆ ಅತ್ಯಂತ ಸಹಾಯಕಾರಿಯಾಗಿ ಪರಿಣಮಿಸಿದುವು.

ಆದರೆ ಅವಳ ಪಾಲಿಗೆ ಈ ಶಿಕ್ಷಣದ ಅವಧಿ ಆಹ್ಲಾದಕರವೇನಾಗಿರಲಿಲ್ಲ. ಅವಳಿಗೆ ಅದೊಂದು ತೀವ್ರ ಘರ್ಷಣೆ-ತೊಳಲಾಟಗಳ ಅವಧಿ; ತರಬೇತಿಗೊಳಗಾದ ಕಾಡಾನೆಯಂತಹ ಅನುಭವ! ಭಾರತಕ್ಕೆ ತನ್ನನ್ನು ತಾನು ಸಂಪೂರ್ಣವಾಗಿ ಸಮರ್ಪಿಸಿಕೊಳ್ಳುವೆನೆಂದು ಆಕೆ ಎಷ್ಟೇ ಬಯಸ ಬಹುದು. ಆದರೆ ಅದು ಅಷ್ಟೊಂದು ಸುಲಭವೆ? ತನ್ನ ವ್ಯಕ್ತಿತ್ವವನ್ನು ಆಮೂಲಾಗ್ರವಾಗಿ ಬದಲಾಯಿಸಿಕೊಳ್ಳುವುದೇನು ಹುಡುಗಾಟವೆ? ಅಲ್ಲದೆ ನಿವೇದಿತೆಯಾದರೂ ಸಾಮಾನ್ಯಳೆ! ಅವಳಿಗೆ ಅವಳದೇ ಆದ ಹಲವಾರು ಕಲ್ಪನೆಗಳಿದ್ದುವು, ನಂಬಿಕೆಗಳಿದ್ದುವು, ದೃಷ್ಟಿಕೋನವಿತ್ತು. ಅವುಗಳನ್ನೆಲ್ಲ ಬಿಟ್ಟುಕೊಡಲು ಅವಳು ಸರ್ವಥಾ ಸಿದ್ಧಳಿರಲಿಲ್ಲ. ಇಂಗ್ಲಿಷರಿಗೆಲ್ಲ ಸಾಮಾನ್ಯವಾದ ಹಲವಾರು ನಂಬಿಕೆಗಳು-ಪೂರ್ವಗ್ರಹಗಳು ಅವಳಲ್ಲೂ ಆಳವಾಗಿ ಬೇರೂರಿದ್ದುವು. ಇದನ್ನೆಲ್ಲ ಸ್ವಾಮೀಜಿ ಆಗಲೇ ಗುರುತಿಸಿದ್ದರು. ತಾನು ಕಲಿಯಬೇಕಾದ್ದು ಬಹಳಷ್ಟಿದೆಯೆಂಬ ಅರಿವು ನಿವೇದಿತೆಗೂ ಇತ್ತು. ಆದರೆ ಮುಂದೆ ತಾನು ಪಡೆಯಲಿರುವ, ಪಡೆಯಬೇಕಾದ ಶಿಕ್ಷಣ ಎಂತಹದೆಂಬ ಕಲ್ಪನೆ ಅವಳಿಗಿರಲಿಲ್ಲ. ಹಲವಾರು ವರ್ಷಗಳ ಮೇಲೆ ಅವಳೇ ಈ ಅವಧಿಯ ಬಗ್ಗೆ ಬರೆಯುತ್ತಾಳೆ: “ತಮ್ಮ ಈ ಶಿಷ್ಯೆಯು (ನಿವೇದಿತಾ) ತಮ್ಮವರೊಡನೆ ಒಂದಾಗಿ ಬೆರೆತಿರು ವುದು ಕೇವಲ ಮೇಲ್ನೋಟಕ್ಕೆ ಮಾತ್ರವೇ ಎಂಬುದನ್ನು ಸ್ವಾಮೀಜಿ ಅರಿತಿದ್ದರಾದರೂ ಅವಳ ವಿಷಯದಲ್ಲಿ ಅವರ ವಿಶ್ವಾಸವಾಗಲಿ ಸಹೃದಯ ವರ್ತನೆಯಾಗಲಿ ಕಿಂಚಿತ್ತೂ ಬದಲಾಗಲಿಲ್ಲ. ಆದರೆ ಆಲ್ಮೋರದೊಂದಿಗೆ ಹೊಸ ಕತೆಯೇ ಪ್ರಾರಂಭವಾಯಿತು–ಆಕೆಗೆ ಹೊಸದಾಗಿ ಶಾಲೆಗೆ ಸೇರಿದ ಅನುಭವವಾಯಿತು.” ಹೊಸದಾಗಿ ಶಾಲೆಗೆ ಸೇರಿದ ಮಕ್ಕಳೆಲ್ಲ ಕೆಲವು ದಿನವಾದರೂ ರಚ್ಚೆ ಹಿಡಿದು ಅತ್ತುಕೊಂಡೇ ಹೋಗುತ್ತವೆಯಲ್ಲವೆ? ಇಲ್ಲಿಂದ ಮುಂದೆ ಸ್ವಾಮೀಜಿಯವರ ಮಾತು ಗಳು ಆಕೆಯ ಮನದಲ್ಲಿ ಬೇರೂರಿದ್ದ ನೂರಾರು ಪೂರ್ವಗ್ರಹಗಳಿಗೆ ಮರ್ಮಾಘಾತಕವಾಗಿ ಪರಿಣಮಿಸಿದುವು. ಆಳವಾಗಿ ಚುಚ್ಚಿಕೊಂಡ ಮುಳ್ಳನ್ನು ಹೊರಗೆಳೆಯುವಾಗ ನೋವು ಅಸಹ ನೀಯವಾಗುತ್ತದೆ. ಆದರೆ ಸ್ವಾಮೀಜಿಯವರೆಂದಿಗೂ ತಮ್ಮ ನಂಬಿಕೆಯನ್ನಾಗಲಿ ತಮ್ಮದೇ ಆದ ಮತವನ್ನಾಗಲಿ ತಮ್ಮ ಶಿಷ್ಯೆಯ ಮೇಲೆ ಹೇರಲು ಪ್ರಯತ್ನಿಸಿದವರಲ್ಲ. ಮುಂದೆ ಇದನ್ನು ಮನ ಗಂಡು ಆಕೆಯೇ ಬರೆಯುತ್ತಾಳೆ: “... ಕೇವಲ ಅರ್ಧ ದೃಶ್ಯವನ್ನು ಮಾತ್ರ ನೋಡುವ ‘ಅರೆ ಕುರುಡುತನ’ವನ್ನು ಸರಿಪಡಿಸುವುದಷ್ಟೇ ಸ್ವಾಮೀಜಿಯವರ ಉದ್ದೇಶವಾಗಿತ್ತು... ಒಬ್ಬನ ಮನಸ್ಸನ್ನು ಸರಿಪಡಿಸಬೇಕಾದರೆ ಅದರ ಗುರುತ್ವಕೇಂದ್ರ\eng{(centre of gravity)}ವನ್ನು ಬದಲಾ ಯಿಸಬೇಕು; ಅರ್ಥಾತ್, ವ್ಯಕ್ತಿಯ ದೃಷ್ಟಿಕೋನವನ್ನು ಬದಲಾಯಿಸಬೇಕು, ಅಷ್ಟೆ. ಸ್ವಾಮೀಜಿ ಮಾಡಹೊರಟದ್ದಾದರೂ ಅದನ್ನೇ.”

ನಿವೇದಿತೆಗೆ ತನ್ನ ಜನಾಂಗದ ಮೇಲೆ ವಿಪರೀತ ಅಭಿಮಾನ. ಸ್ವಾಭಿಮಾನವನ್ನು ಎಂದಿಗೂ ಗೌರವಿಸುವವರು ಸ್ವಾಮೀಜಿ. ಆದರೆ ಅವಳ ಈ ಅಭಿಮಾನವು ದುರಭಿಮಾನಕ್ಕೆ ತಿರುಗಿತ್ತು. ಆಂಗ್ಲರೆಲ್ಲರೂ ತುಂಬ ಸತ್ಯಶೀಲರು, ಪ್ರಾಮಾಣಿಕರು, ಶುಚಿಯಾಗಿರುವವರು, ವಿಶ್ವಾಸಾರ್ಹರು ಎಂಬಿತ್ಯಾದಿ ನಂಬಿಕೆಗಳಿಗೆ ಅವಳು ಬಲವಾಗಿ ಅಂಟಿಕೊಂಡಿದ್ದಳು. ಅದರೊಂದಿಗೇ ಪೌರ್ವಾತ್ಯರ ಚರಿತ್ರೆ-ಸಂಪ್ರದಾಯಗಳು-ಸ್ವಭಾವಗಳು ಮೊದಲಾದವುಗಳಿಗೆ ಸಂಬಂಧಿಸಿದಂತೆ ಅನೇಕ ತಪ್ಪು ಕಲ್ಪನೆಗಳೂ ಅವಳಲ್ಲಿ ಸಾಕಷ್ಟು ಆಳವಾಗಿಯೇ ಇದ್ದುವು. ಇವುಗಳನ್ನು ಕಿತ್ತೊಗೆಯಬೇಕಾದರೆ ಸ್ವಾಮೀಜಿಯವರಿಗೆ ಅದೆಷ್ಟು ಶ್ರಮವಾಗಿರಬಹುದೆಂಬುದು ಊಹಿಸಲಸಾಧ್ಯ!

ಸ್ವಾಮೀಜಿ ತಮ್ಮ ಶಿಷ್ಯೆಯರೊಂದಿಗೆ ಮಾತನಾಡುತ್ತಿರುವಾಗ ಅಲ್ಲಿ ಆಗಾಗ ಒಂದು ಪ್ರತಿ ರೋಧದ ದನಿ ಕೇಳುವುದುಂಟು. ಅದು ನಿವೇದಿತೆಯದೆಂಬುದು ಮಾತ್ರ ನಿಶ್ಚಿತ. ಒಮ್ಮೆ ಸ್ವಾಮೀಜಿ ಬುದ್ಧನನ್ನು ತುಂಬ ಭಾವಭರಿತವಾಗಿ ಕೊಂಡಾಡುತ್ತಿದ್ದರು. ಆಗ ನಿವೇದಿತಾ ಇದ್ದಕ್ಕಿ ದ್ದಂತೆ, “ಸ್ವಾಮೀಜಿ! ನೀವೊಬ್ಬ ಬೌದ್ಧರೆಂದು ನನಗೆ ತಿಳಿದಿರಲೇ ಇಲ್ಲ!” ಎಂದು ಉದ್ಗರಿಸಿದಳು. ಬುದ್ಧನು ಅಷ್ಟೆಲ್ಲ ಪ್ರಶಂಸೆಗೆ ಪಾತ್ರನಲ್ಲವೆಂಬುದು ಅವಳ ಈ ತಮಾಷೆಯ ಮಾತಿನ ಭಾವ. ಆದರೆ ಈ ಮಾತು ಸ್ವಾಮೀಜಿಯವರನ್ನು ಕೆರಳಿಸಿತು. ತಮ್ಮನ್ನು ಅತ್ಯಂತ ಪ್ರಬಲವಾಗಿ ಸಮರ್ಥಿಸಿಕೊಳ್ಳುತ್ತ ಅವರು ಘೋಷಿಸುತ್ತಾರೆ, “ನಾನು ಬುದ್ಧಭಗವಂತನ ದಾಸರ ದಾಸರ ದಾಸ!” ಎಂದು. ಮತ್ತೊಂದು ಸಂದರ್ಭದಲ್ಲಿ ಯಾವುದೋ ವಿಷಯಕ್ಕೆ ಸಂಬಂಧಿಸಿ ದಂತೆ ನಿವೇದಿತಾ ಆಂಗ್ಲರನ್ನು ಬಲವಾಗಿ ಸಮರ್ಥಿಸುತ್ತಿದ್ದಳು. ಅವಳೊಂದಿಗೆ ಮಾತನಾಡಿ ಕಡೆಗೆ ರೋಸಿಹೋಗಿ ಸ್ವಾಮೀಜಿ ಹೇಳಿದರು, “ನಿಜವಾಗಿಯೂ, ನಿನಗಿರುವಂತಹ ದೇಶಪ್ರೇಮವು ಒಂದು ಪಾಪವೇ ಸರಿ!... ಈಗ ನಾನು ಹೇಳುತ್ತಿರುವುದಿಷ್ಟೆ–ಬಹುತೇಕ ಜನರ ವರ್ತನೆಯು ಮೂಲತಃ ಸ್ವಾರ್ಥೋದ್ದೇಶ ಪ್ರೇರಿತವಾಗಿರುತ್ತದೆ, ಎಂದು. ಇದನ್ನು ಒಪ್ಪದೆ ನೀನು ಹೇಳು ತ್ತಿದ್ದೀಯೆ–ಯಾವುದೋ ಒಂದು ಜನಾಂಗದಲ್ಲಿ ಪ್ರತಿಯೊಬ್ಬರೂ ದೇವತಾ ಸ್ವರೂಪಿಗಳು, ಎಂದು. ಇಷ್ಟು ಬಲವಾದ ಅಜ್ಞಾನವನ್ನು ದುಷ್ಟತನವೆಂದರೂ ತಪ್ಪಾಗಲಾರದು!”

ಹೀಗೆ ನಿವೇದಿತಾ ಹೆಜ್ಜೆಹೆಜ್ಜೆಗೂ ತನ್ನ ಗುರುವಿನೊಂದಿಗೆ ಸೆಣಸುತ್ತಿದ್ದಳು. ಒಂದೊಂದು ಕ್ಷಣವೂ ಅವಳಿಗೊಂದು ಹೊಸ ಪಾಠವನ್ನು ಕಲಿಸುತ್ತಿತ್ತು. ದಿನದಿನಕ್ಕೂ ಆಕೆ ತನ್ನ ಲೋಪದೋಷ ಗಳನ್ನು ಸ್ಪಷ್ಟವಾಗಿ ಅರಿತುಕೊಳ್ಳಲು ಸಮರ್ಥಳಾದಳು. ಅಂತೆಯೇ ಸ್ವಾಮೀಜಿಯವರ ಮಾತು ಗಳ, ಬೋಧನೆಗಳ ತಥ್ಯವನ್ನು ಅರಿತಳು. ಅವರ ವ್ಯಕ್ತಿತ್ವದ ಹಾಗೂ ಬೋಧನೆಗಳ ಔನ್ನತ್ಯವನ್ನು ಧೀಮಂತಿಕೆಯನ್ನು ಹೆಚ್ಚು ಹೆಚ್ಚಾಗಿ ಅರಿತಂತೆಲ್ಲ ನಿವೇದಿತಾ ತನ್ನ ಗುರುವಿನಡಿಯಲ್ಲಿ ತನ್ನನ್ನು ಸಂಪೂರ್ಣವಾಗಿ ಸಮರ್ಪಿಸಿಕೊಳ್ಳಲು ಸಮರ್ಥಳಾದಳು. ಆದರೆ ಸ್ವಾಮೀಜಿಯವರು ನಿವೇ ದಿತೆಯ ವಿಷಯದಲ್ಲಿ ಎಷ್ಟೇ ಕಟುವಾಗಿ ವರ್ತಿಸಿದರೂ, ಎಷ್ಟೇ ಖಾರವಾಗಿ ಮಾತನಾಡಿದರೂ ಅವಳ ಬಗ್ಗೆ ಅವರಲ್ಲಿ ಲವಲೇಶವಾದರೂ ಅಸಮಾಧಾನ ಅಥವಾ ಬೇಸರವಿರಲಿಲ್ಲ. ಅವಳ ಹೋರಾಟವನ್ನು, ಪ್ರತಿರೋಧವನ್ನು ಅವರು ಹೃತ್ಪೂರ್ವಕವಾಗಿ ಮೆಚ್ಚಿದ್ದರು. ಈ ಅವಧಿ ಕಳೆದು ದೀರ್ಘಕಾಲವಾದ ನಂತರ ಒಮ್ಮೆ ಅವಳು ಕ್ರಿಸ್ಟೀನಳೊಂದಿಗೆ ಮಾತನಾಡುತ್ತಿದ್ದಳು. ಸ್ವಾಮೀಜಿ ಸನಿಹದಲ್ಲೇ ಇದ್ದರು. ಸಾಂದರ್ಭಿಕವಾಗಿ ಕ್ರಿಸ್ಟೀನ ಹೇಳಿದಳು, “ನೋಡು! ನಾನು ಸ್ವಾಮೀಜಿ ಯವರಾಡಿದ ಪ್ರತಿಯೊಂದು ಮಾತನ್ನೂ ಸ್ವೀಕರಿಸಲು ಸಮರ್ಥಳಾಗಿದ್ದೇನೆ” ಎಂದು. ನಿವೇ ದಿತೆಯ ವಿಷಯ ಹಾಗಲ್ಲ ಎಂಬುದು ಕ್ರಿಸ್ಟೀನಳಿಗೆ ಗೊತ್ತು. ಈ ಮಾತು ಸ್ವಾಮೀಜಿಯವರ ಕಿವಿಗೂ ಬಿದ್ದಿತು. ಆಗ ಅವರು ಅದು ತಮಗೆ ಕೇಳಿಸಲೇ ಇಲ್ಲವೋ ಎಂಬಂತಿದ್ದುಬಿಟ್ಟರು. ಆದರೆ ಮತ್ತೊಂದು ಸಲ ಸರಿಯಾದ ಸಂದರ್ಭ ಬಂದಾಗ ಸ್ವಾಮೀಜಿ ಹೇಳಿದರು, “ಯಾರೂ ಕೂಡ ‘ತಮ್ಮನ್ನು ಒಪ್ಪಿಸುವುದು ಇತರರಿಗೆ ಅಷ್ಟೊಂದು ಕಷ್ಟವಾಯಿತಲ್ಲಾ!’ ಎಂದು ಕೊರಗದಿರಲಿ. ನಾನು ನನ್ನ ಗುರುವಿನೊಂದಿಗೆ ಆರು ವರ್ಷಗಳಷ್ಟು ಸುದೀರ್ಘಕಾಲ ಸೆಣಸಿದ್ದೇನೆ. ಆದ್ದರಿಂದಲೇ ನಾನು ನನ್ನ ಪಥದ ಪ್ರತಿಯೊಂದು ಅಂಗುಲವನ್ನೂ ಅರಿತುಕೊಳ್ಳಲು ಸಾಧ್ಯ ವಾಗಿದೆ!” ಈ ಸಂದರ್ಭದಲ್ಲಿ ಮಾತ್ರವಲ್ಲದೆ ನಿವೇದಿತಾ ಇನ್ನೂ ಹೋರಾಡುವ ಸ್ಥಿತಿಯ ಲ್ಲಿದ್ದಾಗಲೂ ಸ್ವಾಮೀಜಿ ಅನೇಕ ಬಾರಿ ಅವಳಿಗೆ ಸಮಾಧಾನ ಹೇಳುತ್ತಿದ್ದರು, ಉತ್ತೇಜನ ನೀಡುತ್ತಿದ್ದರು.

ಹೀಗೆ ನಿವೇದಿತಾ ಸ್ವಾಮೀಜಿಯವರ ಸಂಕೀರ್ಣ-ಬಹುಮುಖ ವ್ಯಕ್ತಿತ್ವವನ್ನು ಅರ್ಥಮಾಡಿ ಕೊಳ್ಳುತ್ತ ಬಂದಳು. ತನ್ನ ಬಗ್ಗೆ ಅವರಿಗಿರುವ ವಾತ್ಸಲ್ಯವನ್ನು ಕಂಡು ಮೂಕಳಾದಳು. ಸ್ವಾಮೀಜಿ ಕಣ್ಮರೆಯಾದ ಮೇಲೆ ಒಮ್ಮೆ ಅವಳು ತನ್ನೊಂದಿಗೆ ಮಾತನಾಡುತ್ತಿರುವಾಗ ಜೋಸೆಫಿನ್ ಹೇಳುತ್ತಾಳೆ, “ಸ್ವಾಮೀಜಿ ಶಕ್ತಿಸ್ವರೂಪಿಯಾಗಿದ್ದರು.”

ಅದಕ್ಕೆ ನಿವೇದಿತಳ ಪ್ರತಿಕ್ರಿಯೆ–“ಅವರು ಮಾಧುರ್ಯವೇ ಮೈವೆತ್ತಂತಿದ್ದರು.”

“ಹೌದೆ! ನನಗೆ ಎಂದೂ ಹಾಗನ್ನಿಸಲೇ ಇಲ್ಲವಲ್ಲ?”

“ಅದೇಕೆಂದರೆ ಅವರು ನಿನಗದನ್ನು ತೋರಿಸಲಿಲ್ಲ!”

ನಿವೇದಿತಾ ತನ್ನನ್ನು ಸ್ವಾಮೀಜಿಯವರ ಆಧ್ಯಾತ್ಮಿಕ ಪುತ್ರಿಯೆಂದು ಕರೆದುಕೊಳ್ಳಲು ಅತ್ಯಂತ ಹೆಮ್ಮೆ ಪಡುತ್ತಿದ್ದಳು. (ಸ್ವಾಮೀಜಿ ‘ತಾಯಿ’ ಎಂದು ಕರೆಯುತ್ತಿದ್ದ ಶ್ರೀಮತಿ ಸಾರಾ ಬುಲ್ಲಳನ್ನು ನಿವೇದಿತಾ ‘ಅಜ್ಜಿ’ ಎನ್ನುತ್ತಿದ್ದಳು, ಹಾಗೂ ಸ್ವಾಮೀಜಿಯವರ ‘ತಂಗಿ’ ಮೇರಿ ಹೇಲ್​ಳನ್ನು ‘ಚಿಕ್ಕಮ್ಮ\eng{’—aunt—}ಎನ್ನುತ್ತಿದ್ದಳು.) ಸ್ವಾಮೀಜಿ ಹಾಗೂ ಅವರ ಶಿಷ್ಯೆಯರ ನಡುವಣ ಸಂಬಂಧ ಇಷ್ಟು ಆಪ್ತವಾದುದು. ಪ್ರತಿಯೊಬ್ಬ ಸ್ತ್ರೀಯೂ ಅವರ ಪಾಲಿಗೆ ಸಾಕ್ಷಾತ್ ಜಗನ್ಮಾತೆಯೇ ಆಗಿ ದ್ದಳು ಎಂಬ ಮಾತನ್ನು ಹೇಳಬೇಕಾಗಿಯೇ ಇಲ್ಲ. ಆದರೂ ಸಹ, ಆಶ್ರಮವಾಸಿಗಳು ಸ್ತ್ರೀಯ ರೊಂದಿಗೆ ಹೇಗೆ ವರ್ತಿಸಬೇಕೆಂಬುದರ ಬಗ್ಗೆ ಅವರೇ ರೂಪಿಸಿದ್ದ ನೀತಿನಿಯಮಗಳಿದ್ದುವಲ್ಲವೆ? ಸ್ವಾಮೀಜಿ ಆ ನಿಯಮಗಳೆಲ್ಲಕ್ಕೂ ಅತೀತರಾಗಿದ್ದರೆಂಬುದೇನೋ ನಿಜವೇ. ಆದರೂ ಅವರು ಈ ನಿಯಮಗಳಿಂದ ತಮ್ಮನ್ನು ಹೊರತುಪಡಿಸಿಕೊಳ್ಳಲಿಲ್ಲ. ಈ ವಿಷಯವನ್ನು ಅವರು ಎಷ್ಟು ಗಂಭೀರವಾಗಿ ಪರಿಗಣಿಸಿದ್ದರೆಂಬುದು ಅವರು ತಮ್ಮ ನೆಚ್ಚಿನ ಸೋದರ ಸ್ವಾಮಿ ರಾಮಕೃಷ್ಣಾ ನಂದರಿಗೆ ಬರೆದ ಒಂದು ಪತ್ರದಲ್ಲಿ ವ್ಯಕ್ತವಾಗುತ್ತದೆ. ಆಗ ರಾಮಕೃಷ್ಣಾನಂದರು ಮದ್ರಾಸಿನಲ್ಲಿ ಮಠಸ್ಥಾಪನೆಯ ಸಂಬಂಧವಾಗಿ ಶ್ರೀ ಬಿಳಿಗಿರಿ ಅಯ್ಯಂಗಾರರ ಸಂಪರ್ಕದಲ್ಲಿದ್ದರು. (ಹಿಂದೆಯೇ ನೋಡಿದಂತೆ ಇವರು ಮದ್ರಾಸಿನ ಸುವಿಖ್ಯಾತ ಸಮಾಜಸೇವಕರು ಹಾಗೂ ವಿವೇಕಾನಂದರ ಭಕ್ತರು.) ಬಿಳಿಗಿರಿ ಅಯ್ಯಂಗಾರರಿಗೆ, ವಿಧವೆಯರಾದ ಇಬ್ಬರು ಹೆಣ್ಣು ಮಕ್ಕಳು; ಈಗ ಸ್ವಾಮಿ ಗಳು ಈ ಹೆಣ್ಣು ಮಕ್ಕಳ ಸಂಪರ್ಕಕ್ಕೂ ಬರುವ ಸಂದರ್ಭವೊದಗಬಹುದು. ಆದ್ದರಿಂದ ಸ್ವಾಮೀಜಿ ಬರೆಯುತ್ತಾರೆ, “ವಯಸ್ಸಿಗೆ ಬಂದ ಹೆಣ್ಣು ಮಕ್ಕಳ ಬಳಿಯಲ್ಲಿರುವಾಗ ಎಂಥವ ರಾದರೂ ಅತ್ಯಂತ ಜಾಗ್ರತೆಯಿಂದಿರಬೇಕಾಗುತ್ತದೆ. ಒಮ್ಮೆ ಕಾಲು ಜಾರಿತೆಂದರೆ ಮತ್ತೆ ಬೇರೆ ದಾರಿಯೇ ಇಲ್ಲ. ಮತ್ತು ಆ ಅಪರಾಧವೋ ಅಕ್ಷಮ್ಯ!”

ಸ್ವಾಮೀಜಿ ಆಲ್ಮೋರದಲ್ಲಿ ಹಲವಾರು ಸ್ಥಳೀಯ ನಿವಾಸಿಗಳಿಗೆ ಸಂದರ್ಶನ ನೀಡಿದರು. ಅಲ್ಲದೆ ಆ ಸಮಯದಲ್ಲಿ ಬೇಸಿಗೆಯನ್ನು ಕಳೆಯಲೆಂದು ನೈನಿತಾಲಿಗೆ ಬಂದಿದ್ದ ಅನೇಕ ಗಣ್ಯವ್ಯಕ್ತಿಗಳನ್ನು ಭೇಟಿಯಾದರು. ಅವರೆಲ್ಲರೊಂದಿಗೆ ಸ್ವಾಮೀಜಿ ಆಧ್ಯಾತ್ಮಿಕ ಹಾಗೂ ಧಾರ್ಮಿಕ ವಿಷಯಗಳಿಗೆ ಸಂಬಂಧಿಸಿದಂತೆ ಮಾತನಾಡಿ ಬೋಧನೆ ನೀಡಿದರು. ಈ ಸಂದರ್ಭದಲ್ಲಿ ಅವರು ಶ್ರೀಮತಿ ಆ್ಯನಿ ಬೆಸೆಂಟರನ್ನು ಎರಡು ಬಾರಿ ಸಂಧಿಸಿದರು. ಒಮ್ಮೆ ಅವರು ಶ್ರೀಮತಿ ಬೆಸೆಂಟರ ಆಹ್ವಾನದ ಮೇರೆಗೆ ಅವರಿಳಿದುಕೊಂಡಿದ್ದಲ್ಲಿಗೆ ಹೋಗಿ ದೀರ್ಘಕಾಲ ಸಂಭಾಷಣೆ ನಡೆಸಿದರು. ಶ್ರೀಮತಿ ಬೆಸೆಂಟರು ತಮ್ಮ ಥಿಯೊಸಾಫಿಕಲ್ ಸೊಸೈಟಿ ಹಾಗೂ ಸ್ವಾಮೀಜಿಯವರ ಸಂಸ್ಥೆಯ ನಡುವೆ ಅನ್ಯೋನ್ಯತೆ ಬೆಳೆಯಬೇಕೆಂದು ಕೇಳಿಕೊಂಡರು. ಮತ್ತೊಮ್ಮೆ ಶ್ರೀಮತಿ ಬೆಸೆಂಟರನ್ನು ಸ್ವಾಮೀಜಿಯವರೇ ತಮ್ಮಲ್ಲಿಗೆ ಚಹಾಕ್ಕೆ ಆಹ್ವಾನಿಸಿದರು.

ಸ್ವಾಮೀಜಿ ಆಲ್ಮೋರದಲ್ಲಿದ್ದಾಗ ಬ್ರಿಟಿಷ್ ಸರಕಾರದ ಗುಪ್ತ ಪೋಲಿಸರು ಅವರ ಮೇಲೆ ಕಣ್ಣಿಟ್ಟಿದ್ದರು. ಇದಕ್ಕೆ ಕಾರಣವೇನೆಂದರೆ ಅವರು ಅನೇಕ ಪಾಶ್ಚಾತ್ಯ ಮಹಿಳೆಯರೊಂದಿಗೆ ಓಡಾಡುತ್ತಿದ್ದುದು. ಅದರಲ್ಲೂ ನಿವೇದಿತೆಯೊಂದಿಗಿನ ಅವರ ಸಂಪರ್ಕ ಪೋಲಿಸರ ಅನು ಮಾನಕ್ಕೆ ಕಾರಣವಾಗಿತ್ತು. ಏಕೆಂದರೆ ಜನ್ಮತಃ ಐರಿಶ್ ಜನಾಂಗಕ್ಕೆ ಸೇರಿದ ಮಾರ್ಗರೆಟ್ ನೋಬೆಲ್ ಹಿಂದೆ ಐರಿಶ್ ಚಳವಳಿಯಲ್ಲಿ ಭಾಗವಹಿಸಿದ್ದಳು. ಅಲ್ಲದೆ ಸ್ವಾಮೀಜಿಯವರೇ ಹೇಳುತ್ತಿದ್ದಂತೆ, ಹಿಂದೊಮ್ಮೆ ಶಿವಾಜಿ ಮಹಾರಾಜ ಸಂನ್ಯಾಸಿಯ ವೇಷದಿಂದ ತಲೆತಪ್ಪಿಸಿಕೊಂಡ ಕಥೆಯನ್ನು ತಿಳಿದಿದ್ದ ಬ್ರಿಟಿಷರಿಗೆ, ಯಾವ ಕಾಷಾಯ ವಸ್ತ್ರದೊಳಗೆ ಯಾವ ಶಿವಾಜಿ ಅಡಗಿರು ವನೋ ಎಂಬ ಅನುಮಾನ ಇದ್ದೇ ಇತ್ತು! ಪೋಲಿಸರು ತಮ್ಮ ಮೇಲೆ ಕಣ್ಣಿಟ್ಟಿರುವ ವಿಷಯ ವನ್ನು ನಿವೇದಿತಾ ಗಂಭೀರವಾಗಿ ಪರಿಗಣಿಸಿದರೂ ಸ್ವಾಮೀಜಿ ನಕ್ಕು ಸುಮ್ಮನಾದರು.

ಸ್ವಾಮೀಜಿ ಆಲ್ಮೋರದಲ್ಲಿ ಹಾಸ್ಯ ಮಾಡುತ್ತ ನಕ್ಕು ನಲಿಯುತ್ತಿದ್ದರಾದರೂ ಮತ್ತೆ ಕೆಲ ವೊಮ್ಮೆ ಜೀವನದ ದಾರುಣ ದುಃಖ-ದುಮ್ಮಾನಗಳ ಬಗ್ಗೆಯೂ ಮಾತನಾಡುತ್ತ ಅಂತರ್ಮುಖಿ ಗಳಾಗಿಬಿಡುತ್ತಿದ್ದರು. ಆ ದಿನಗಳಲ್ಲಿ ಅವರಿಗೆ ಏಕಾಂತದಲ್ಲಿ ಶಾಂತವಾಗಿದ್ದುಬಿಡಬೇಕೆಂಬ ಪ್ರಬಲ ಇಚ್ಛೆಯುಂಟಾಗುತ್ತಿತ್ತು. ಇಂತಹ ಭಾವದಲ್ಲಿ ಒಂದು ದಿನ ಅವರು ತಮ್ಮ ಸ್ನೇಹಿತ ರನ್ನು-ಶಿಷ್ಯರನ್ನು ಬಿಟ್ಟು ಆಲ್ಮೋರದಿಂದ ಸ್ವಲ್ಪ ದೂರದಲ್ಲಿರುವ ಶಿಯಾದೇವಿ ಬೆಟ್ಟಕ್ಕೆ ಹೋಗಿ ಅಲ್ಲೊಂದು ಡೇರೆ ಹಾಕಿಕೊಂಡು ಇದ್ದುಬಿಟ್ಟರು. ನಿತ್ಯವೂ ಅರಣ್ಯದ ನೀರವ ವಾತಾವರಣದಲ್ಲಿ ಹತ್ತುಗಂಟೆಗಳ ಕಾಲ ಧ್ಯಾನನಿರತರಾಗಿದ್ದು ಡೇರೆಗೆ ಹಿಂದಿರುಗುತ್ತಿದ್ದರು. ಇದು ಹೀಗೆ ಹಲವಾರು ದಿನ ನಡೆಯಿತು. ಆದರೆ ಸ್ವಾಮೀಜಿ ಡೇರೆಗೆ ಹಿಂದಿರುಗುವ ವೇಳೆಗೆ ಅಲ್ಲಿಯೂ ಕೂಡ ಜನ ಬರಲಾರಂಭಿಸಿದರು. ಆದ್ದರಿಂದ ಅವರು ಆಲ್ಮೋರಕ್ಕೆ ಹಿಂದಿರುಗಿದರು. ಆದರೆ ಈ ಕೆಲವು ದಿನಗಳ ಅನುಭವ ಅವರಿಗೆ ತುಂಬ ಹರ್ಷದಾಯಕವಾಗಿತ್ತು. ತಾವು ಹಿಂದಿನ ಪರಿವ್ರಾಜಕ ಸಂನ್ಯಾಸಿಯಂತೆಯೇ ಈಗಲೂ ಬರಿಗಾಲಿನಲ್ಲಿ ನಡೆಯುತ್ತ ಚಳಿಬಿಸಿಲುಗಳನ್ನೆಲ್ಲ ಎದುರಿಸ ಬಲ್ಲೆವು, ಪಾಶ್ಚಾತ್ಯದೇಶಗಳಿಗೆ ಹೋಗಿಬಂದರೂ ತಾವೇನೂ ಕೆಟ್ಟಿಲ್ಲ ಎಂದು ಸ್ವಾಮೀಜಿ ಸಮಾಧಾನಗೊಂಡರು.

ಮೇ ೩ಂ ರಂದು ಅವರು ಸೇವಿಯರ್ ದಂಪತಿಗಳೊಂದಿಗೆ ಪ್ರಶಾಂತವಾದ ಏಕಾಂತ ಸ್ಥಳವೊಂದನ್ನರಸಿ ಹೊರಟರು. ಮತ್ತೆ ಕೆಲವು ದಿನ ಏಕಾಂತದಲ್ಲಿರುವುದು ಒಂದು ಉದ್ದೇಶ ವಾದರೆ ತಾವು ಸ್ಥಾಪಿಸಲಿರುವ ಮಠಕ್ಕಾಗಿ ಸೂಕ್ತ ನಿವೇಶನವನ್ನು ಗೊತ್ತುಪಡಿಸುವುದು ಮತ್ತೊಂದು ಉದ್ದೇಶ. ಆದರೆ ಅಷ್ಟೆಲ್ಲ ಹುಡುಕಿದರೂ ಅವರಿಗೆ ತೃಪ್ತಿಯಾಗುವ ಸ್ಥಳ ಆಗ ಎಲ್ಲಿಯೂ ಸಿಗಲಿಲ್ಲ.

