
\chapter{“ಕ್ರಿಸ್ತನಂತಹ ವ್ಯಕ್ತಿ!”}

\noindent

ನಲವತ್ತೆರಡು ದಿನಗಳ ದೀರ್ಘ ಪ್ರಯಾಣದ ನಂತರ ಜುಲೈ ೩೧ರಂದು ಸ್ವಾಮೀಜಿಯವರ ಹಡಗು ಲಂಡನ್ನಿಗೆ ಬಂದು ಸೇರಿತು. ಆಗ ಅದು ಬೇಸಿಗೆಯ ಸಮಯ; ಆದ್ದರಿಂದ ಅವರ ಭಕ್ತರು ಹಾಗೂ ವಿಶ್ವಾಸಿಗರಲ್ಲಿ ಬಹಳ ಮಂದಿ ವಿಹಾರಾರ್ಥವಾಗಿ ದೂರದೂರದ ಸ್ಥಳಗಳಿಗೆ ಹೊರಟುಹೋಗಿದ್ದರು. ಸ್ವಾಮೀಜಿಯವರನ್ನು ಸ್ವಾಗತಿಸಲು ಬಂದರಿನಲ್ಲಿದ್ದ ಕೆಲವೇ ಜನ ರೆಂದರೆ ನಿವೇದಿತೆಯ ತಾಯಿ ಮತ್ತು ತಂಗಿ, ಮಿಸ್ ಪ್ಯಾಸ್ಟನ್ ಎಂಬೊಬ್ಬ ಭಕ್ತೆ ಮತ್ತು ಅಮೆರಿಕದಿಂದ ಅವರನ್ನು ಭೇಟಿ ಮಾಡಲೆಂದೇ ಬಂದಿದ್ದ ಕುಮಾರಿ ಕ್ರಿಸ್ಟೀನ ಗ್ರೀನ್​ಸ್ಟೈಡಲ್ ಮತ್ತು ಶ್ರೀಮತಿ ಮೇರಿ ಫಂಕೆ. ಸ್ವಾಮೀಜಿಯವರನ್ನು ಕಂಡ ಶ್ರೀಮತಿ ಫಂಕೆ ಆಮೇಲೆ ಹೇಳುತ್ತಾಳೆ, “ಸ್ವಾಮೀಜಿ ತುಂಬ ಇಳಿದುಹೋಗಿದ್ದರು, ಒಬ್ಬ ಹುಡುಗನಂತಾಗಿಬಿಟ್ಟಿದ್ದರು” ಎಂದು. ಆದರೆ ದೀರ್ಘ ಸಮುದ್ರ ಪ್ರಯಾಣದಿಂದ ಅವರ ಆರೋಗ್ಯ ಎಷ್ಟೋ ಸುಧಾರಿಸಿತ್ತು. ಶಕ್ತಿ ಉತ್ಸಾಹಗಳು ಬಹುಮಟ್ಟಿಗೆ ಹಿಂದಿರುಗಿದ್ದುವು.

ಬಂದರಿನ ಬಳಿ ಸ್ವಾಮೀಜಿಯವರನ್ನು ಎದುರುಗೊಳ್ಳಲು ಅವರ ಹಳೆಯ ಸ್ನೇಹಿತ ಇ. ಟಿ. ಸ್ಟರ್ಡಿ ಇಲ್ಲದಿದ್ದುದು ಎದ್ದು ಕಾಣುತ್ತಿತ್ತು. ಆದರೆ ಅವನು ಈ ಹಿಂದೆಯೇ ಪತ್ರ ಬರೆದು ತಾನು ವೇಲ್ಸ್ ಪ್ರದೇಶಕ್ಕೆ ಹೋಗುತ್ತಿರುವುದಾಗಿಯೂ, ಆದ್ದರಿಂದ ಅವರು ಲಂಡನ್ನಿಗೆ ಬಂದಾಗ ಎದುರುಗೊಳ್ಳಲು ತಾನು ಇರುವುದಿಲ್ಲವೆಂದೂ ತಿಳಿಸಿದ್ದ. ಈ ಪತ್ರ ತಲುಪಿದಾಗ ಸ್ವಾಮೀಜಿ ಯವರಿಗೆ ಸ್ವಲ್ಪ ನಿರಾಶೆಯೇನೋ ಆಗಿತ್ತಾದರೂ ಅವರಿಗೆ ಸ್ಟರ್ಡಿಯ ಮೇಲಿದ್ದ ವಿಶ್ವಾಸವೇನೂ ಕುಂದಿರಲಿಲ್ಲ. ತಮ್ಮ ಕಾರ್ಯೋದ್ದೇಶದಲ್ಲಿ ಅವನು ಎಂದಿನ ನಿಷ್ಠೆ-ಉತ್ಸಾಹಗಳನ್ನೇ ಹೊಂದಿ ದ್ದಾನೆಂದು ಭಾವಿಸಿದ್ದರು. ಆದರೆ ಸ್ಟರ್ಡಿ ಸಂಪೂರ್ಣ ಬದಲಾಗಿದ್ದ. ಕೆಲದಿನಗಳಲ್ಲೇ ಸ್ವಾಮೀಜಿ ಯವರಿಗೆ ವಾಸ್ತವಾಂಶ ತಿಳಿದುಬಂದಾಗ ಅವರಿಗೆ ಮೊದಲು ಉಂಟಾಗಿದ್ದ ನಿರಾಶೆಯು ತೀವ್ರ ಆಘಾತವಾಗಿ ಪರಿಣಮಿಸಿತು.

ಸ್ವಾಮೀಜಿ ಲಂಡನ್ನಿನಲ್ಲಿ ಸುಮಾರು ಹದಿನೈದು ದಿನ ಇದ್ದರು. ಈ ಅವಧಿಯಲ್ಲಿ ಸ್ಟರ್ಡಿ ವೇಲ್ಸ್​ನಿಂದ ಮೂರು ದಿನಗಳ ಮಟ್ಟಿಗೆ ಬಂದುಹೋದ. ಆದರೆ ತನ್ನ ಗುರುವಿನ ಮುಂದೆ ಮುಖಕ್ಕೆ ಮುಖ ಕೊಟ್ಟು ಮನಸ್ಸಿನಲ್ಲಿರುವುದನ್ನೆಲ್ಲ ಹೇಳಲಾರದೆಹೋದ. ಆದರೆ ಏನೋ ಹದಗೆಟ್ಟಿದೆ ಎಂಬುದು ಸ್ವಾಮೀಜಿಯವರಿಗೆ ಅರ್ಥವಾಗುತ್ತಿತ್ತು. ಮುಂದೆ ಅವರು ಅಮೆರಿಕೆಗೆ ತೆರಳಿದ ಮೇಲೆ ಸ್ಟರ್ಡಿ ಅವರಿಗೊಂದು ಪತ್ರ ಬರೆದು ತನ್ನೆಲ್ಲ ಸಿಟ್ಟನ್ನೂ ಕಾರಿದ. ವಿವೇಕಾನಂದರ ಹಾಗೂ ಅವರ ಸಹ ಸಂನ್ಯಾಸಿಗಳ ಜೀವನಕ್ರಮವನ್ನೂ ಅವರ ಬೋಧನೆಗಳನ್ನೂ ತೀವ್ರವಾಗಿ ಖಂಡಿಸಿದ. ಈ ವಿಷಯವಾಗಿ ಆಗಲೆ ನಿವೇದಿತಾ, ಸ್ಟರ್ಡಿ ಹಾಗೂ ಇತರ ಪಾಶ್ಚಾತ್ಯ ಶಿಷ್ಯರ ನಡುವೆ ಬೇಕಾದಷ್ಟು ಪತ್ರವ್ಯವಹಾರ ನಡೆದಿತ್ತು. ಮತ್ತು ಆಮೇಲೂ ಮುಂದುವರಿಯಿತು. ಸ್ವಾಮೀಜಿ ಹಾಗೂ ಅವರ ಗುರುಭಾಯಿಗಳ ಸಂನ್ಯಾಸದ ಹಾಗೂ ವೈರಾಗ್ಯದ ಪುಜುತ್ವವನ್ನೇ ಸ್ಟರ್ಡಿ ಶಂಕಿಸಿದ್ದ. ಅವನು ಇಂತಹ ತೀವ್ರ ಆರೋಪಗಳನ್ನು ಮಾಡಿದರೂ ಸ್ವಾಮೀಜಿ ಶಾಂತ ವಾಗಿಯೇ ಅವುಗಳಿಗೆಲ್ಲ ಉತ್ತರಿಸಿದರು. ಆದರೆ ಸ್ಟರ್ಡಿ ಸಂಪೂರ್ಣ ಕುರುಡನಾಗಿಬಿಟ್ಟಿದ್ದ. ಕಡೆಗೆ ಸ್ವಾಮೀಜಿ ಈ ಎಲ್ಲ ಅಹಿತಕರ ವ್ಯವಹಾರಗಳಿಗೂ ಮಂಗಳ ಹಾಡುವ ಪತ್ರವೊಂದನ್ನು ಬರೆದರು, “ಸ್ಟರ್ಡಿ,... ಅಮರಪ್ರೇಮದ, ಅಳಿವಿಲ್ಲದ ವಿಶ್ವಾಸದ ಆ ನನ್ನ ಭಾರತವಿನ್ನೂ ಜೀವಂತವಾಗಿದೆ. ಸಂಪ್ರದಾಯಗಳು ಮಾತ್ರವಲ್ಲದೆ ಪ್ರೀತಿ ವಿಶ್ವಾಸ ಸ್ನೇಹಗಳು ಎಂದೆಂದೂ ಬದಲಾಗದ ಆ ಸನಾತನ ಭಾರತವಿನ್ನೂ ಉಳಿದುಕೊಂಡಿದೆ. ಆ ಭಾರತದ ಕೇವಲ ಒಬ್ಬ ಶಿಶುವಾದ ನಾನು ನಿನ್ನನ್ನು ಪ್ರೀತಿಸುತ್ತೇನೆ,–ಭಾರತೀಯ ಹೃದಯದಿಂದ ಪ್ರೀತಿಸುತ್ತೇನೆ, ಈಗ ಆವರಿಸಿಕೊಂಡಿರುವ ಭ್ರಮೆ ಯಿಂದ ನಿನ್ನನ್ನು ಮುಕ್ತನನ್ನಾಗಿಸಲು ನಾನು ಸಾವಿರ ಬಾರಿ ಜನ್ಮವೆತ್ತಲೂ ಸಿದ್ಧನಿದ್ದೇನೆ.”

ಹೀಗೆ ಸ್ವಾಮೀಜಿಯವರಿಂದ ದೂರವಾದ ಇಂಗ್ಲಿಷ್ ಸ್ನೇಹಿತರು ಸ್ಟರ್ಡಿಯಲ್ಲದೆ ಇನ್ನೂ ಕೆಲವರಿದ್ದರು. ಆ ವರ್ಷದ ಮೊದಲ ಭಾಗದಲ್ಲಿ ಸ್ವಾಮೀಜಿಯವರನ್ನು ಕಾಣಲೆಂದೇ ಭಾರತಕ್ಕೆ ಬಂದು, ಬೇಲೂರು ಮಠದ ಸ್ಥಾಪನಾಕಾರ್ಯದಲ್ಲಿ ಆರ್ಥಿಕವಾಗಿ ಬಹಳಷ್ಟು ನೆರವಾದ ಮಿಸ್ ಮುಲ್ಲರಳೂ ಈಗ ಅವರಿಗೆ ವಿರುದ್ಧವಾಗಿದ್ದಳು. ಅವಳು ಭಾರತದಿಂದ ಇಂಗ್ಲೆಂಡಿಗೆ ಮರಳು ವಾಗಲೇ ಅವರೊಂದಿಗಿನ ತನ್ನ ಸಂಬಂಧವನ್ನು ಕಡಿದುಕೊಂಡುಬಿಟ್ಟಳು. ಅಲ್ಲದೆ ಅದನ್ನು ಪತ್ರಿಕೆಗಳಲ್ಲೂ ಘೋಷಿಸಿದಳು. ಇದನ್ನು ಕಂಡು ಶ್ರೀ ಸೇವಿಯರರು ಪತ್ರವೊಂದರಲ್ಲಿ ಬರೆಯು ತ್ತಾರೆ–“ಮಿಸ್ ಮುಲ್ಲರ್ ತನ್ನ ಮೂರ್ಖತನವನ್ನು ಪತ್ರಿಕೆಗಳಲ್ಲಿ ಜಾಹೀರಾತು ಮಾಡಬೇಕಾ ಗಿತ್ತೆ!” ಹೀಗೆ ತಮ್ಮ ಕಾರ್ಯದಲ್ಲಿ ಬಹಳಷ್ಟು ನೆರವಾಗಿದ್ದ ವ್ಯಕ್ತಿಯೊಬ್ಬಳು ತಮ್ಮಿಂದ ದೂರವಾದುದು ಸ್ವಾಮೀಜಿಯವರಿಗೆ ತುಂಬ ನಿರಾಶೆಯನ್ನೇ ಉಂಟುಮಾಡಿತೆನ್ನಬೇಕು.

ಸ್ವಾಮೀಜಿಯವರಿಂದ ಪ್ರಭಾವಿತಳಾದ ಮಿಸ್ ಮುಲ್ಲರ್ ಎಷ್ಟು ರಭಸದಿಂದ ಅನುಸರಿಸಿ ಬಂದಳೋ ಅಷ್ಟೇ ರಭಸದಿಂದ ಹೊರಟುಹೋದಳು. ವಿಶ್ವಾಸದ ಮನೋಭಾವವನ್ನು ಶಾಶ್ವತ ವಾಗಿ ಉಳಿಸಿಕೊಳ್ಳುವುದು ಸುಲಭದ ಮಾತೇನಲ್ಲ. ಹೀಗೆಯೇ ಸ್ವಾಮೀಜಿಯವರ ಮೇಲಣ ಭಕ್ತಿವಿಶ್ವಾಸಗಳು ತಣ್ಣಗಾದ ಆಂಗ್ಲ ಅನುಯಾಯಿಗಳಲ್ಲಿ ಇನ್ನೊಬ್ಬಳೆಂದರೆ ಶ್ರೀಮತಿ ಆ್ಯಷ್ಟನ್ ಜಾನ್​ಸನ್ ಎಂಬವಳು. ಇವಳ ದೃಷ್ಟಿಯಲ್ಲಿ, ಆಧ್ಯಾತ್ಮಿಕ ವ್ಯಕ್ತಿಗಳೆನ್ನಿಸಿಕೊಂಡವರು ಕಾಯಿಲೆ ಬೀಳಲೇಬಾರದು. ಇದು ಆಗಿನ ಕಾಲದಲ್ಲಿ ವಿಶೇಷವಾಗಿ ಪ್ರಚಲಿತವಿದ್ದ ಭಾವನೆ. ಈಗ ಸ್ವಾಮೀಜಿಯವರಲ್ಲಿ ಬಗೆಬಗೆಯ ಕಾಯಿಲೆಗಳಿಂದಾಗಿ ಮೊದಲಿನ ಶರೀರಸೌಷ್ಠವವನ್ನು ಕಾಣದ ಶ್ರೀಮತಿ ಜಾನ್​ಸನ್ ಅವರನ್ನು ದೂರೀಕರಿಸಿಬಿಟ್ಟಳು. ಅವರು ಹಿಂದೆ ಅಮೆರಿಕದಲ್ಲಿ ಬಳಿಕ ಇಂಗ್ಲೆಂಡಿನಲ್ಲಿ, ಮತ್ತೆ ಭಾರತದಲ್ಲಿ ನಡೆಸಿದ ಕಠಿಣತಮ ಚಟುವಟಿಕೆಗಳನ್ನೆಲ್ಲ ಆಕೆ ಪರಿಗಣಿಸಲೇ ಇಲ್ಲ. ಸ್ವಾಮೀಜಿಯವರು ಕಾರ್ಯವೆಸಗಿದ ಪರಿಯಲ್ಲಿ ಎಂತಹ ಭೀಮಕಾಯನೂ ಬಳಲಿ ಬೆಂಡಾಗಿ ಬಿದ್ದು ಹೋಗುವುದೇ ಖಂಡಿತ. ಅವರದೂ ಮನುಷ್ಯ ಶರೀರವಲ್ಲವೆ! ಅಲ್ಲದೆ ಸ್ವಾಮೀಜಿಯವರೇ ಹೇಳುತ್ತಿದ್ದರು, ತಾವು ನಲವತ್ತನ್ನು ದಾಟುವುದಿಲ್ಲ ಎಂದು. ಆದರೆ ಶ್ರೀಮತಿ ಜಾನ್​ಸನ್​ರಂಥವರು ಭಾವಿಸುತ್ತಾರೆ, ಆಧ್ಯಾತ್ಮಿಕ ವ್ಯಕ್ತಿ ಕಾಯಿಲೆ ಬೀಳಬಹುದೆ? ಎಂದು. ಇಂಥದನ್ನೆಲ್ಲ ಮನಗಂಡೇ ಸ್ವಾಮೀಜಿ ಮುಂದೆ ಹೇಳುತ್ತಾರೆ, “ಆಲದ ಮರವೊಂದು ಆರೋಗ್ಯ ವಾಗಿ ಸಾವಿರ ವರ್ಷ ಬದುಕಬಹುದು. ಆದರೆ ಕೊನೆಗೂ ಅದೊಂದು ಆಲದ ಮರ ಮಾತ್ರ!”

ಲಂಡನ್ನಿನಲ್ಲಿ ತಮ್ಮ ಕಾರ್ಯವು ಹೆಚ್ಚು ಕಡಿಮೆ ಸ್ಥಗಿತಗೊಂಡಿದೆ ಎಂಬುದನ್ನು ಸ್ವಾಮೀಜಿ ಹಿಂದೆಯೇ ಮನಗಂಡಿದ್ದರು. ಈಗ ತಮ್ಮ ಅನೇಕ ವಿಶ್ವಾಸೀ ಸ್ನೇಹಿತರು ಸಂಪೂರ್ಣ ದೂರವಾ ಗಿರುವುದನ್ನು ಕಂಡ ಸ್ವಾಮೀಜಿ ಇಂಗ್ಲೆಂಡಿನಿಂದ ಹೊರಟ ಮೇಲೆ ತಮ್ಮೊಬ್ಬ ಶಿಷ್ಯೆಗೆ ಬರೆಯು ತ್ತಾರೆ, “ಬಹುಶಃ ಇಂಗ್ಲೆಂಡಿನ ಕಾರ್ಯಕಲಾಪಗಳೆಲ್ಲ ಮುರಿದುಬಿದ್ದಂತೆಯೇ! ಅಲ್ಲಿನ ನನ್ನ ಸ್ನೇಹಿತರೆಲ್ಲ \eng{‘Shaky’} (ಅಸ್ಥಿರ) ಆಗಿದ್ದಾರೆ–ಸ್ಟರ್ಡಿಯೂ ಸಹ.” (ಇಂಗ್ಲಿಷಿನಲ್ಲಿ ಸ್ಟರ್ಡಿ ಎಂದರೆ ಸ್ಥಿರ-ದೃಢ ಎಂದರ್ಥ.)

ಹೀಗೆ ಸ್ವಾಮೀಜಿಯವರಿಂದ ದೂರವಾದವರೆಂದರೆ ಸ್ಟರ್ಡಿ, ಮಿಸ್ ಮುಲ್ಲರ್ ಇವರು ಮಾತ್ರವಲ್ಲ, ಮುಂದೆ ಅವರ ಅಮೆರಿಕನ್ ಅನುಯಾಯಿಗಳಾದ ಸ್ವಾಮಿ ಅಭಯಾನಂದಾ (ಮಿಸ್ ಮೇರಿ ಲೂಯಿಸ್​) ಮೊದಲಾದವರು ಬೇರೆಬೇರೆ ಕಾರಣಗಳಿಂದಾಗಿ ಅವರಿಂದ ದೂರವಾದರು. ಇದೆಲ್ಲವೂ ಸ್ವಾಮೀಜಿಯವರಿಗೆ ನಿರಾಶೆಯನ್ನುಂಟುಮಾಡಿತು ಎಂಬುದೇನೋ ನಿಜವೆ. ಆದರೆ ಒಂದು ಮುಖ್ಯ ಅಂಶವನ್ನು ನಾವಿಲ್ಲಿ ಗಮನಿಸಬೇಕು. ಏನೆಂದರೆ, ತಮ್ಮ ಈ ಎಲ್ಲ ಅನುಯಾಯಿ ಗಳಲ್ಲೂ ಅವರು ಇಂತಹ ದೌರ್ಬಲ್ಯಗಳನ್ನು ಬಹಳ ಹಿಂದೆಯೇ ಗುರುತಿಸಿದ್ದರು ಎಂಬುದು. ಆಯಾ ಶಿಷ್ಯರಿಗೂ ಇತರರಿಗೂ (ಅವರು ತಿರುಗಿಬೀಳುವುದಕ್ಕೆ ಬಹಳಷ್ಟು ಮೊದಲೇ) ಸ್ವಾಮೀಜಿ ಬರೆದ ಕೆಲವು ಪತ್ರಗಳಿಂದ ಇದು ಸ್ಪಷ್ಟವಾಗುತ್ತದೆ. ಉದಾಹರಣೆಗೆ, ಮಿಸ್ ಹೆನ್ರಿಟಾ ಮುಲ್ಲರಳು ಮಹಾ ಹಟಮಾರಿಯೂ ಗರ್ವಿಷ್ಠಳೂ ಆಗಿದ್ದಳು. ಅಲ್ಲದೆ ಅವಳು ತನ್ನ ಹಣದಿಂದ ಏನನ್ನು ಬೇಕಾದರೂ ಕೊಳ್ಳಬಹುದು ಎಂದು ನಂಬಿದ್ದಳು. ಅವಳಲ್ಲಿ ಇತರ ಸದ್ಗುಣಗಳೇನೇ ಇದ್ದರೂ ಈ ಅಹಂಕಾರದಿಂದಾಗಿ ಅವಳೊಂದಿಗೆ ಯಾರೂ ಇರುವಂತಿರಲಿಲ್ಲ. ಆದ್ದರಿಂದಲೇ ಸ್ವಾಮೀಜಿ ಯವರು ನಿವೇದಿತೆ ಭಾರತಕ್ಕೆ ಬರುವ ಮೊದಲೇ ಅವಳಿಗೆ ಎಚ್ಚರಿಕೆ ನೀಡಿದ್ದರು, “ನೀನು ಮಿಸ್ ಮುಲ್ಲರಳ ರೆಕ್ಕೆಯ ಕೆಳಗೆ ನಿಲ್ಲುವಂತಿಲ್ಲ!” ಎಂದು. ಅಂತೆಯೇ ಸ್ಟರ್ಡಿ, ಲ್ಯಾಂಡ್ಸ್​ಬರ್ಗ್, ಮೇರಿ ಲೂಯಿಸ್ ಮೊದಲಾದವರ ವಿಚಾರದಲ್ಲೂ ಕೂಡ. ಎಂದಮೇಲೆ ಅವರನ್ನೆಲ್ಲ ಸ್ವಾಮೀಜಿ ಒಮ್ಮೆಗೇ ಏಕೆ ದೂರ ಮಾಡಿಬಿಡಲಿಲ್ಲ ಎಂಬ ಸಂದೇಹವೇಳಬಹುದು. ಈಗ ನಾವಿಲ್ಲಿ ಮತ್ತೊಂದು ಅಂಶವನ್ನೂ ಪರಿಗಣಿಸಬೇಕಾಗುತ್ತದೆ. ಏನೆಂದರೆ ಅವರಿಂದ ದೂರವಾದವರೂ ಕೂಡ ಒಂದು ಸಮಯದಲ್ಲಿ ಅವರ ನಿಷ್ಠಾವಂತ ಬೆಂಬಲಿಗರಾಗಿ ದುಡಿದು ಮಹತ್ತರವಾದ ಸೇವೆಯನ್ನು ಸಲ್ಲಿಸಿದರು ಎಂಬುದು. ಉದಾಹರಣೆಗೆ ಮಿಸ್ ಮುಲ್ಲರ್ ಉದಾರವಾಗಿ ನೀಡಿದ ೩೯,0ಂಂ ರೂಪಾಯಿಗಳಿಂದಲೇ ಅಲ್ಲವೆ ಬೇಲೂರು ಮಠದ ಆ ವಿಶಾಲ ನಿವೇಶನವನ್ನು ಕೊಂಡುಕೊಳ್ಳಲು ಸಾಧ್ಯವಾದುದು? ತನ್ಮೂಲಕ ಶ್ರೀರಾಮಕೃಷ್ಣ ಮಹಾಸಂಘಕ್ಕೆ ಆಧಾರವೊದಗಿ ದುದು? ಅದಕ್ಕೆ ಪ್ರತಿಯಾಗಿ ಸ್ವಾಮೀಜಿಯವರು ಆಕೆಗೆ ತಮ್ಮ ಶ್ರೇಷ್ಠತಮ ಆಧ್ಯಾತ್ಮಿಕತೆಯನ್ನೇ ಧಾರೆಯಾಗಿ ಎರೆದರೆಂಬುದು ಬೇರೆ ವಿಚಾರ. (ಮಿಸ್ ಮುಲ್ಲರಳಿಗೆ ಸ್ವಾಮೀಜಿಯವರ ಬಗ್ಗೆ ವೈಯಕ್ತಿಕವಾಗಿ ಅವರ ವಿಶ್ವಾಸಗಳಿನ್ನೂ ಇದ್ದುವೆಂಬುದು ಮುಂದೆ ಆಕೆ ಬರೆದ ಒಂದು ಪತ್ರ ದಿಂದ ವಿದಿತವಾಯಿತು).

ಆದರೆ ಸ್ವಾಮೀಜಿಯವರಿಗೆ ಹೀಗೆ ನಿರಾಶಾದಾಯಕವಾದ ಆಘಾತಗಳಾದರೂ ಅವರ ವ್ಯಕ್ತಿತ್ವದ ಉಜ್ವಲತೆ ಮಾತ್ರ ಕುಂದಲಿಲ್ಲ. ಉತ್ಸಾಹ ಕರಗಲಿಲ್ಲ. ಅವರ ದರ್ಶನಾರ್ಥಿಗಳಾಗಿ ಬಂದ ಪ್ರತಿಯೊಬ್ಬರೂ ಅವರಿಂದ ನಿರಂತರ ಹೊರಹೊಮ್ಮುತ್ತಿದ್ದ ಆತ್ಮಶಕ್ತಿಯ ಪ್ರಕಾಶವನ್ನು ಕಾಣದಿರುತ್ತಿರಲಿಲ್ಲ.

ಲಂಡನ್ನಿಗೆ ಆಗಮಿಸಿದ ಮೊದಲ ದಿನ ನಗರದಲ್ಲಿಯೇ ಇಳಿದುಕೊಂಡ ಸ್ವಾಮೀಜಿ ಮರುದಿನ ಹೊರವಲಯದಲ್ಲಿರುವ ವಿಂಬಲ್ಡನ್ನಿಗೆ ತೆರಳಿದರು. ಅಲ್ಲಿ ನಿವೇದಿತೆಯ ಸೋದರಿ ಮೇ ನೋಬೆಲ್ ಅವರಿಗಾಗಿ ವಸತಿಗೃಹವೊಂದನ್ನು ಏರ್ಪಡಿಸಿದ್ದಳು. ಅದೊಂದು ತುಂಬ ಪ್ರಶಾಂತ ವಾದ ಸ್ಥಳ. ಇಲ್ಲಿ ಅವರು ತುಂಬ ಶಾಂತಿಯಿಂದ ಕೆಲದಿನಗಳವರೆಗೆ ಇದ್ದರು. ಲಂಡನ್ನಿಗೆ ತಲುಪಿದಾಗ ಅವರು ಆರೋಗ್ಯದಿಂದಲೇ ಇದ್ದರೂ ಈಗ ಅದು ಮತ್ತೆ ಕೆಡಲಾರಂಭಿಸಿತು. ಆದ್ದರಿಂದ ಅವರು ಹೆಚ್ಚು ವಿಶ್ರಾಂತಿಯಲ್ಲೇ ಇರಬೇಕಾಯಿತು. ಅಲ್ಲದೆ ಅದು ಬೇಸಿಗೆಯ ರಜೆಯ ಸಮಯವಾದ್ದರಿಂದ ಸಾರ್ವಜನಿಕ ಕಾರ್ಯಕ್ರಮಗಳನ್ನು ಇಟ್ಟುಕೊಳ್ಳುವಂತೆಯೇ ಇರಲಿಲ್ಲ. ಹಿಂದೆ ಅವರನ್ನು ಮುತ್ತುತ್ತಿದ್ದ ಜನಸಮುದಾಯದ ಪೈಕಿ ಕೆಲವರು ಮಾತ್ರ ಹಿಂದಿನ ವಿಶ್ವಾಸ-ಶ್ರದ್ಧೆಗಳಿಂದ ಅವರಲ್ಲಿಗೆ ಬಂದು ಹೋಗುತ್ತಿದ್ದರು.

ಈ ದಿನಗಳಲ್ಲಿ ನಿವೇದಿತೆಯ ಕುಟುಂಬವರ್ಗದವರು ಸ್ವಾಮೀಜಿಯವರೊಂದಿಗೆ ನಿಕಟ ಸಂಪರ್ಕಕ್ಕೆ ಬರಲು ಸಾಧ್ಯವಾಯಿತು. ಇದರಿಂದ ಸ್ವಾಮೀಜಿಯವರ ಮೇಲಿನ ಅವರ ಭಕ್ತಿ ವಿಶ್ವಾಸ ಇನ್ನಷ್ಟು ಗಾಢವಾಯಿತು. ಅವರೆಲ್ಲ ಸ್ವಾಮೀಜಿಯವರಿಗೆ ಮಂಡಿಯೂರಿ ನಮಸ್ಕರಿಸುವು ದನ್ನು ಕಂಡಾಗಲಂತೂ ನಿವೇದಿತೆಗೆ ಆಶ್ಚರ್ಯಾನಂದ. ಸ್ವಾಮೀಜಿ ತಮ್ಮ ಉಜ್ವಲ ವಾಕ್​ಲಹರಿ ಯಿಂದ ಅವರೆಲ್ಲರನ್ನೂ ಮಂತ್ರಮುಗ್ಧರನ್ನಾಗಿಸಿದರು, ಅವರೆಲ್ಲರ ಹೃದಯವನ್ನು ಮಿಡಿದರು. ಶೀಘ್ರದಲ್ಲಿಯೇ ವಿವಾಹವಾಗಲಿದ್ದ ನಿವೇದಿತೆಯ ತಂಗಿ ಮೇ ನೋಬೆಲ್ಲಳಿಗೂ ಕೂಡ ತನ್ನ ಅಕ್ಕನಂತೆಯೇ ತಾನೂ ತನ್ನ ಜೀವನವನ್ನು ಸ್ವಾಮೀಜಿಯವರಿಗೆ ಅರ್ಪಿಸಿ ಅವರನ್ನು ಹಿಂಬಾಲಿ ಸುವ ಉತ್ಕಟೇಚ್ಛೆ ಉಂಟಾಯಿತು! ನಿವೇದಿತೆಯ ಹದಿವಯಸ್ಕ ಸೋದರ ರಿಚ್​ಮಂಡ್​ನಿಗೆ ಸ್ವಾಮೀಜಿ ‘ಕ್ರಿಸ್ತನಂತಹ ವ್ಯಕ್ತಿ’ಯಾಗಿ ತೋರಿದರು. ಮುಂದೆ ಅವನು ಅವರ ಬಗ್ಗೆ ಬರೆಯುತ್ತಾನೆ:

“ನನ್ನ ಅಕ್ಕ ಅವರ ಕರೆಗೆ ಓಗೊಟ್ಟು ಹೋದುದರಲ್ಲಿ ಅಚ್ಚರಿಯೇನೂ ಇಲ್ಲ. ಏಕೆಂದರೆ ನಾನೂ ಕೂಡ ಅವರನ್ನು ಕಂಡವನು, ಅವರ ಶಕ್ತಿಯನ್ನು ಅರಿತವನು. ಸ್ವಾಮೀಜಿಯವರನ್ನು ಕಂಡು ಅವರ ಮಾತುಗಳನ್ನು ಕೇಳಿದ ಯಾರೇ ಆದರೂ ಪ್ರಭಾವಿತರಾಗಲೇ ಬೇಕು. ಸ್ವಾಮೀಜಿ ಸತ್ಯವನ್ನೇ ನುಡಿಯುತ್ತಿದ್ದಾರೆಂಬುದು ಕೇಳುಗರಿಗೆ ಖಚಿತವಾಗುತ್ತಿತ್ತು. ಏಕೆಂದರೆ ಅವರೊಬ್ಬ ಪಂಡಿತನಂತಾಗಲಿ ಪಾದ್ರಿಯಂತಾಗಲಿ ಮಾತನಾಡುತ್ತಿರಲಿಲ್ಲ; ಅದು ಅಷ್ಟು ಅಧಿಕಾರಯುತ ವಾಗಿತ್ತು. ಅವರ ಮಾತುಗಳಲ್ಲಿ ಒಂದು ನಿಶ್ಚಿತತೆಯಿತ್ತು. ಅವರು ತಮ್ಮ ಮಾತುಗಳನ್ನು ಕೇಳಿದವರಲ್ಲೊಂದು ಭರವಸೆಯನ್ನು ಮೂಡಿಸುತ್ತಿದ್ದರು. ಧೈರ್ಯ ತುಂಬುತ್ತಿದ್ದರು. ನನಗನ್ನಿ ಸುತ್ತದೆ, ಇದನ್ನೇ ಅವರು ನನ್ನ ಸೋದರಿಗೂ ಮಾಡಿರಬೇಕು, ಎಂದು. ಆದ್ದರಿಂದಲೇ ಅವಳು ನಿರ್ಭಯವಾಗಿ ಅವರ ಕರೆಯನ್ನು ಮನ್ನಿಸಿದಳು. ಒಮ್ಮೆ ತನ್ನನ್ನು ಅವರಿಗೆ ಒಪ್ಪಿಸಿಕೊಂಡಮೇಲೆ ಅವಳೆಂದೂ ಅದಕ್ಕಾಗಿ ವಿಷಾದಿಸಬೇಕಾಗಿ ಬರಲಿಲ್ಲ.”

ನಿವೇದಿತೆಯ ತಾಯಿ ಮೇರಿ ನೋಬೆಲ್ ಸ್ವಾಮೀಜಿಯವರನ್ನು ತನ್ನ ಸ್ವಂತ ಮಗನೆಂದೇ ಭಾವಿಸಿ ಉಪಚರಿಸಿದಳು. ಅವರ ಬಗ್ಗೆ ಆಕೆಯ ಪ್ರೀತಿ-ಅಭಿಮಾನ ಎಷ್ಟರ ಮಟ್ಟಿಗಿತ್ತೆಂದರೆ, ಮಿಸ್ ಮುಲ್ಲರಳ ಅಪಪ್ರಚಾರದ ಮಾತುಗಳು ಅವಳ ಕಿವಿಗೆ ಬಿದ್ದಾಗ ತನಗೇ ಅಪಮಾನವಾಯಿ ತೆಂಬಂತೆ ಅವಳ ಕಣ್ಣುಗಳು ಹನಿಗೂಡಿದುವು.

ಈ ವೇಳೆಗೆ, ಇಂಗ್ಲೆಂಡಿನಲ್ಲಿ ಹೆಚ್ಚು ದಿನ ನಿಲ್ಲದೆ ಕೂಡಲೇ ಹೊರಡುವಂತೆ ಸ್ವಾಮೀಜಿ ಯವರಿಗೆ ಅಮೆರಿಕದ ಭಕ್ತರಿಂದ ಪತ್ರಗಳ ಮೇಲೆ ಪತ್ರಗಳು ಬರಲಾರಂಭಿಸಿದುವು. ಆದ್ದರಿಂದ ಅವರು ತಮ್ಮಿಬ್ಬರು ಅಮೆರಿಕನ್ ಶಿಷ್ಯೆಯರ ಹಾಗೂ ತುರೀಯಾನಂದರ ಜೊತೆಗೆ ಇಂಗ್ಲೆಂಡಿ ನಿಂದ ಹೊರಡಲು ಅನುವಾದರು. ತನ್ನ ಗೃಹಕೃತ್ಯದಲ್ಲಿ ಕೆಲಕಾಲ ನೆರವಾಗಿ, ಬಳಿಕ ಭಾರತದಲ್ಲಿನ ಶಾಲೆಗೆ ಸಂಬಂಧಿಸಿದಂತೆ ಸಾಧ್ಯವಾದಷ್ಟು ಹಣ ಸಂಗ್ರಹ ಮಾಡಲು ಶ್ರಮಿಸುವಂತೆ ಅವರು ನಿವೇದಿತೆಗೆ ನಿರ್ದೇಶಿಸಿದರು. ಆಗಸ್ಟ್ ೧೬ರಂದು ಲಂಡನ್ನಿನಿಂದ ಗ್ಲಾಸ್ಗೋಗೆ ಟ್ರೈನಿನಲ್ಲಿ ಹೋಗಿ, ಅಲ್ಲಿಂದ ಮರುದಿನ \eng{S. S.} ನ್ಯುಮಿಡಿಯನ್ ಎಂಬ ಹಡಗನ್ನೇರಿ ಅಮೆರಿಕದೆಡೆಗೆ ಸಾಗಿದರು.

ಅಟ್ಲಾಂಟಿಕ್ ಸಾಗರವನ್ನು ದಾಟುವ ತಮ್ಮ ಈ ಪ್ರಯಾಣದ ಬಗ್ಗೆ ಶ್ರೀಮತಿ ಮೇರಿ ಫಂಕೆ ಬರೆಯುತ್ತಾಳೆ:

“ಸಾಗರದ ಮೇಲೆ ನಾವು ಕಳೆದಂತಹ ಆ ಹತ್ತು ದಿನಗಳು ಎಂದೆಂದಿಗೂ ಮರೆಯಲಾರ ದಂಥವು. ಪ್ರತಿದಿನ ಬೆಳಿಗ್ಗೆ ಗೀತಾಪಾರಾಯಣ ಮತ್ತು ಗೀತಾಧ್ಯಯನದಲ್ಲಿ ತೊಡಗಿರುತ್ತಿದ್ದೆವು. ಅಲ್ಲದೆ ಸ್ವಾಮೀಜಿಯವರು ವೇದಮಂತ್ರಗಳನ್ನು ಅಥವಾ ಇತರ ಸಂಸ್ಕೃತ ಶ್ಲೋಕಗಳನ್ನು ಪಠಿಸುತ್ತಿದ್ದರು; ಕೆಲವೊಮ್ಮೆ (ನಿವೇದಿತೆಗಾಗಿ) ಸಂಸ್ಕೃತದ ಕಥೆಗಳನ್ನು ಅನುವಾದಿಸುತ್ತಿದ್ದರು. ಸಮುದ್ರ ಬಹಳ ಶಾಂತವಾಗಿತ್ತು. ರಾತ್ರಿಯ ವೇಳೆಯಲ್ಲಿ ಬೆಳದಿಂಗಳಂತೂ ಮೈಮರೆಸು ವಂತಿತ್ತು. ಆ ಬೆಳಕಿನಲ್ಲಿ ಸ್ವಾಮೀಜಿ ಹಡಗಿನ ಡೆಕ್ಕಿನ ಮೇಲೆ ಅತ್ತಿಂದಿತ್ತ ಓಡಾಡುತ್ತಿದ್ದರೆ ಅವರ ಮನೋಹರ-ಗಂಭೀರ ವ್ಯಕ್ತಿತ್ವವು ಇನ್ನಷ್ಟು ಶೋಭಾಯಮಾನವಾಗಿ ಕಂಡುಬರುತ್ತಿತ್ತು. ಆಗಾಗ ಅವರು ನಡೆದಾಟವನ್ನು ನಿಲ್ಲಿಸಿ ನಮ್ಮೊಡನೆ ಪ್ರಕೃತಿ ಸೌಂದರ್ಯದ ಬಗ್ಗೆ ಬಣ್ಣಿಸುತ್ತಿದ್ದರು. ‘ಈ ಮಾಯೆಯೇ ಇಷ್ಟು ಸುಂದರವಾಗಿದ್ದರೆ ಇದರ ಹಿಂದಿರುವ ಸತ್ಯ ಇನ್ನೆಷ್ಟು ಸುಂದರವಾಗಿರ ಬಹುದೆಂದು ಆಲೋಚಿಸು!’ ಎಂದು ಅವರು ಉದ್ಗರಿಸುತ್ತಿದ್ದರು.

“ಒಂದು ರಾತ್ರಿಯಂತೂ ಆಗಸದಲ್ಲಿ ಚಂದ್ರಮ ಸುಮನೋಹರ ಸುವರ್ಣವರ್ಣದಿಂದ ಕೂಡಿ ಪೂರ್ಣಕಾಂತಿಯಿಂದ ಬೆಳಗುತ್ತಿದ್ದ. ಆ ರಾತ್ರಿಯು ಮಂಕು ಕವಿಸುವಂತಹ ಮೋಹಕತೆ ಯಿಂದ ಕೂಡಿತ್ತು. ಈ ದಿವ್ಯ ಸುಂದರ ದೃಶ್ಯದ ಸೊಬಗನ್ನು ಹೀರುತ್ತ ಬಹಳ ಹೊತ್ತು ಮೌನವಾಗಿ ನಿಂತಿದ್ದ ಸ್ವಾಮೀಜಿ ಇದ್ದಕ್ಕಿದ್ದಂತೆ ನಮ್ಮತ್ತ ತಿರುಗಿ, ಆಗಸ-ಸಾಗರಗಳನ್ನು ತೋರಿಸುತ್ತ ಹೇಳಿದರು: ‘ಕವಿತೆಯ ಸಾರಸರ್ವಸ್ವವೇ ಅಲ್ಲಿರುವಾಗ ಕವನವನ್ನು ಹಾಡಬೇಕೇಕೆ?’

“ನಾವು ನ್ಯೂಯಾರ್ಕನ್ನು ಬಹಳ ಬೇಗ ತಲುಪಿಬಿಟ್ಟೆವು ಎನ್ನಿಸಿತು. ಗುರುದೇವನ ಸಾಮೀಪ್ಯ ದಲ್ಲಿ ಕಳೆದ ಆ ಪವಿತ್ರ ಹತ್ತುದಿನಗಳಿಗೆ ನಾವೆಷ್ಟು ಕೃತಜ್ಞರಾಗಿದ್ದರೂ ಸಾಲದು ಎಂಬ ಭಾವನೆ ನಮ್ಮಲ್ಲಿ ತುಂಬಿತ್ತು.”

ದಿವ್ಯ ಪುರುಷರ ಸಹವಾಸದ ಮಹಿಮೆ ಮಹತ್ವಗಳನ್ನು ಹಲವರು ಅರಿಯದೆ ಹೋಗಬಹುದು. ಆದರೆ ಅರಿತವರ ಆನಂದವನ್ನು ಅಳೆಯಬಲ್ಲವರಾರು!

