
\chapter{ಧೀರಪುತ್ರನಿಗೆ ಭಾರತದ ಸ್ವಾಗತ}

\noindent

೧೮೯೭ರ ಜನವರಿ ೧೫–ಸಿಂಹಳದ ಹಿಂದೂಸಮುದಾಯದ ಇತಿಹಾಸದಲ್ಲಿ ಸುವರ್ಣಾಕ್ಷರ ಗಳಿಂದ ಬರೆದಿಡಬೇಕಾದ ದಿನ. ಕೊಲಂಬೋದ ಭೂಮಿಯ ಮೇಲೆ ಪದಾರ್ಪಣ ಮಾಡಿದ ಸ್ವಾಮಿ ವಿವೇಕಾನಂದರನ್ನು ಅವರು ಸ್ವಾಗತಿಸಿದ ದಿನ.

ಆಗ ಸಂಧ್ಯಾಕಾಲವಾಗಿತ್ತು. ಗೈರಿಕ ವಸನಧಾರಿಯಾಗಿ ಶೋಭಿಸುತ್ತಿದ್ದ ಸ್ವಾಮೀಜಿ, ಸಿಂಹ ಗಾಂಭೀರ್ಯದಿಂದ ತಮ್ಮ ಸಂಗಡಿಗರೊಂದಿಗೆ ಹಡಗಿನಿಂದ ಹೊರಬಂದರು. ಉಗಿದೋಣಿಯ ಮೂಲಕ ಅವರನ್ನು ತೀರಕ್ಕೆ ಕರೆತರಲಾಯಿತು. ಅಲ್ಲಿ ಅವರನ್ನು ಸ್ವಾಗತಿಸಲು ಜನಸಾಗರವೇ ನೆರೆದಿತ್ತು. ಎಲ್ಲೆಲ್ಲೂ ಕರತಾಡನ ಹರ್ಷೋದ್ಗಾರಗಳ ಸಂಭ್ರಮ. ವಿಧಾನ ಪರಿಷತ್ತಿನ ಸದಸ್ಯರಾದ ಶ್ರೀ ಕುಮಾರಸ್ವಾಮಿ ಮತ್ತು ಅವರ ಸೋದರ ಮುಂದೆ ಬಂದು ಸ್ವಾಮೀಜಿಯವರಿಗೆ ಒಂದು ಸುಂದರವಾದ ಮಲ್ಲಿಗೆಯ ಹಾರವನ್ನು ಅರ್ಪಿಸಿ ಸ್ವಾಗತಿಸಿದರು. ಅಷ್ಟರಲ್ಲೇ ಜನರ ನೂಕು ನುಗ್ಗಲು ಶುರುವಾಯಿತು. ಭೋರ್ಗರೆಯುತ್ತ ಮುನ್ನುಗ್ಗಿದ ಆ ಜನಸಂದಣಿಯನ್ನು ತಡೆದು ನಿಲ್ಲಿಸಲು ಯಾರಿಂದಲೂ ಸಾಧ್ಯವಿರಲಿಲ್ಲ. ಸ್ವಾಮೀಜಿ ನಗರವನ್ನು ಪ್ರವೇಶಿಸುವ ಸ್ಥಳದಲ್ಲಿ ಅವರನ್ನು ಸ್ವಾಗತಿಸಲು ಎತ್ತರದ ಕಮಾನೊಂದನ್ನು ನಿರ್ಮಿಸಲಾಗಿತ್ತು. ತೆಂಗಿನ ಹೂಗಳಿಂದ ಒಂದು ಸುಂದರವಾದ ‘ಸುಸ್ವಾಗತ’ ಎಂಬ ಬರಹವನ್ನು ರಚಿಸಲಾಗಿತ್ತು. ಅಲಂಕರಿಸಿದ್ದ ಜೋಡುಕುದುರೆಗಾಡಿಯಲ್ಲಿ ಮೆರವಣಿಗೆಯ ಮೂಲಕ ಸ್ವಾಮೀಜಿ ಹಾಗೂ ಅವರ ಜೊತೆಗಾರ ರನ್ನು ವಿಜಯೋತ್ಸವದ ಚಪ್ಪರಕ್ಕೆ ಕರೆದೊಯ್ಯಲಾಯಿತು. ಲಭ್ಯವಿದ್ದ ಪ್ರತಿಯೊಂದು ಗಾಡಿ ಯನ್ನೂ ಏರಿ ಜನ ಅಲ್ಲಿಗೆ ಆಗಮಿಸಿದರು. ಜೊತೆಗೆ ಸಾವಿರಾರು ಜನ ಕಾಲ್ನಡಿಗೆಯಲ್ಲಿ ಬಂದು ಸೇರಿದರು. ವಿಜಯೋತ್ಸವದ ಚಪ್ಪರವನ್ನು ತಳಿರುತೋರಣಗಳಿಂದ, ತೆಂಗಿನ ಗರಿಗಳಿಂದ ವೈವಿಧ್ಯಮಯವಾಗಿ ಸಿಂಗರಿಸಲಾಗಿತ್ತು. ಸ್ವಾಮೀಜಿ ಗಾಡಿಯಿಂದಿಳಿಯುತ್ತಿದ್ದಂತೆ ಅವರನ್ನು ಶ್ವೇತಚ್ಛತ್ರ, ಬಾವುಟ ಮೊದಲಾದವುಗಳೊಂದಿಗೆ ಹಿಂದೂ ಸಂಪ್ರದಾಯಗಳಿಗನುಗುಣವಾಗಿ ಎದುರ್ಗೊಳ್ಳಲಾಯಿತು. ಸ್ವಾಮೀಜಿ ನಡೆದುಬರಲು ನೆಲದುದ್ದಕ್ಕೂ ಶ್ವೇತವಸ್ತ್ರವನ್ನು ಹಾಸ ಲಾಗಿತ್ತು. ಸಾಂಪ್ರದಾಯಿಕ ಸ್ವಾಗತವಾದ ಮೇಲೆ, ಅಲ್ಲಿಂದ ಮುಂದೆ ಅವರನ್ನು ಮತ್ತೊಂದು ಭಾರೀ ಚಪ್ಪರಕ್ಕೆ ಮೆರವಣಿಗೆಯಲ್ಲಿ ಕರೆದೊಯ್ದರು. ಅನೇಕರು ಪಂಜು ಹಿಡಿದು ನಡೆಯು ತ್ತಿದ್ದುದು ಮೆರವಣಿಗೆಗೊಂದು ವಿಶೇಷ ಆಕರ್ಷಣೆ ತಂದುಕೊಟ್ಟಿತ್ತು. ದಾರಿಯುದ್ದಕ್ಕೂ ಮಂಗಳ ವಾದ್ಯಗಳನ್ನು ಬಾಜಿಸಲಾಯಿತು. ಚಪ್ಪರದ ಹಿಂಬದಿಯಲ್ಲಿದ್ದ ಒಂದು ಹೊಚ್ಚಹೊಸ ಬಂಗಲೆ ಯನ್ನು ಸ್ವಾಮೀಜಿ ಮತ್ತು ಅವರ ಸಂಗಡಿಗರಿಗಾಗಿ ಸಜ್ಜುಗೊಳಿಸಲಾಗಿತ್ತು. ಮೆರವಣಿಗೆಯ ದಾರಿಯಲ್ಲಿ ಹಲವಾರು ಸುಂದರ ಕಮಾನುಗಳನ್ನು ರಚಿಸಲಾಗಿತ್ತು. ಅದೊಂದು ಅತ್ಯಂತ ರಮ್ಯವಾದ ದೃಶ್ಯ. ಕೊಲಂಬೋ ನಗರದ ಇತಿಹಾಸದಲ್ಲೇ ಆ ಬಗೆಯ ಸ್ವಾಗತ ಎಂದೂ ಯಾರಿಗೂ ದೊರಕಿರಲಿಲ್ಲ. ಜಯಘೋಷಗಳೊಂದಿಗೆ ಮೆರವಣಿಗೆ ನಿಧಾನವಾಗಿ ಸಾಗಿ ಸಮಾ ರಂಭದ ಸ್ಥಳವನ್ನು ಮುಟ್ಟಿತು. ಸ್ವಾಮೀಜಿಯವರೊಂದಿಗೆ ಅವರ ಐರೋಪ್ಯ ಶಿಷ್ಯರೂ ಸ್ವಾಮಿ ನಿರಂಜನಾನಂದರೂ ಇದ್ದರು. ಸ್ವಾಮೀಜಿ ಚಪ್ಪರವನ್ನು ಪ್ರವೇಶಿಸುತ್ತಿದ್ದಂತೆಯೇ ಅತ್ಯಂತ ಸುಂದರವಾದ ಕೃತಕ ಕಮಲವೊಂದು ಅರಳಿ, ಅದರೊಳಗಿನಿಂದ ಒಂದು ಪಕ್ಷಿ ಆಗಸಕ್ಕೆ ಹಾರಿಹೋಯಿತು. ಆದರೆ ಸ್ವಾಮೀಜಿಯವರನ್ನು ನೋಡುವ ಆತುರದಲ್ಲಿದ್ದ ಜನ ಈ ಅದ್ಭುತ ಅಲಂಕಾರಗಳನ್ನೆಲ್ಲ ಗಮನಿಸುವ ಸ್ಥಿತಿಯಲ್ಲಿರಲಿಲ್ಲ. ಸ್ವಾಮೀಜಿ ತಮ್ಮ ಶಿಷ್ಯರೊಂದಿಗೆ ವೇದಿಕೆಯ ಮೇಲೆ ಆಸೀನರಾಗುತ್ತಿದ್ದಂತೆಯೇ ಅವರ ಮೇಲೆ ಪುಷ್ಪವೃಷ್ಟಿಯಾಯಿತು. ಹರ್ಷೋ ದ್ಗಾರಗಳೆಲ್ಲ ನಿಂತು ಜನ ಶಾಂತರಾದ ಮೇಲೆ ಸಂಗೀತ ವಿದ್ವಾಂಸರೊಬ್ಬರು ಎಲ್ಲರ ಮನಮೆಚ್ಚು ವಂತೆ ಕೆಲಕಾಲ ಪಿಟೀಲು ನುಡಿಸಿದರು. ಬಳಿಕ ‘ತೇವಾರ’ಗಳೆಂದು ಕರೆಯಲ್ಪಡುವ ಸ್ತೋತ್ರ ಗಳನ್ನು ಹಾಡಲಾಯಿತು. ಅನಂತರ ಅಂದಿನ ಸಭಾಧ್ಯಕ್ಷರಾದ ಶ್ರೀ ಕುಮಾರಸ್ವಾಮಿಯವರು ಮುಂದೆ ಬಂದು ಸ್ವಾಮೀಜಿಯವರಿಗೆ ಪ್ರಣಾಮ ಸಲ್ಲಿಸಿ, ಹಿಂದೂಜನಗಳ ಪರವಾಗಿ ಅವರನ್ನು ಅಭಿನಂದಿಸಿ ಸ್ವಾಗತಿಸುವ ಭಾಷಣವನ್ನು ಓದಿದರು. ಆ ಭಾಷಣಕ್ಕೆ ಉತ್ತರಕೊಡಲು ಸ್ವಾಮೀಜಿ ಎದ್ದುನಿಂತಾಗ ಕಿವಿಗಡಚಿಕ್ಕುವಂತೆ ಕರತಾಡನದ ಸದ್ದು ಮೊಳಗಿತು. ಕೊಲಂಬೋದ ನಾಗರಿಕರು ತಮಗಿತ್ತ ಅಭೂತಪೂರ್ವ ಸ್ವಾಗತಕ್ಕೆ ಕೃತಜ್ಞತೆಗಳನ್ನರ್ಪಿಸುತ್ತ ಸ್ವಾಮೀಜಿ ಒಂದು ಪುಟ್ಟ ಭಾಷಣ ಮಾಡಿದರು.

ಸ್ವಾಗತ ಸಮಾರಂಭವಾದ ಮೇಲೆ ಅವರನ್ನು ಅವರಿಗಾಗಿ ಸಿದ್ಧಪಡಿಸಲಾಗಿದ್ದ ಬಂಗಲೆ ಯೊಳಗೆ ಗೌರವರಕ್ಷೆಯೊಡನೆ ಕರೆದೊಯ್ದು ಮತ್ತೊಮ್ಮೆ ಪುಷ್ಪಹಾರವನ್ನು ಸಮರ್ಪಿಸಿ ಗೌರವಿಸ ಲಾಯಿತು. ಆದರೆ ಇತ್ತ ಸಮಾರಂಭದಲ್ಲಿ ಭಾಗವಹಿಸಿದ್ದ ಸಾವಿರಾರು ಜನ ಇನ್ನೂ ಚದುರಿರ ಲಿಲ್ಲ. ಮತ್ತೊಮ್ಮೆ ಸ್ವಾಮೀಜಿಯವರ ಮುಖದರ್ಶನ ಮಾಡಲು ಅವರೆಲ್ಲ ಕಾತರರಾಗಿ ನಿಂತಿದ್ದರು. ಹೀಗೆ ಬಹು ಸಂಖ್ಯೆಯಲ್ಲಿ ಜನ ಕಾದು ನಿಂತಿದ್ದುದರಿಂದ ಸ್ವಾಮೀಜಿ ಹೊರಬಂದು, ಕೈಗಳನ್ನು ಮೇಲೆತ್ತಿ ಎಲ್ಲರಿಗೂ ಆಶೀರ್ವಾದ ಮಾಡಿದರು.

ಮರುದಿನ, ಎಂದರೆ ೧೬ನೇ ತಾರೀಕು ಶನಿವಾರ ಕೊಲಂಬೋದ ‘ಫ್ಲೋರಲ್ ಹಾಲ್​’ ಎಂಬಲ್ಲಿ ಸ್ವಾಮೀಜಿಯವರ ಸಾರ್ವಜನಿಕ ಭಾಷಣವನ್ನು ಏರ್ಪಡಿಸಲಾಗಿತ್ತು. ಸಹಸ್ರಾರು ನಾಗರಿಕರು ಕಿಕ್ಕಿರಿದಿದ್ದ ಆ ಸಭೆಯನ್ನುದ್ದೇಶಿಸಿ ಸ್ವಾಮೀಜಿ ಭಾವಪ್ರಚೋದಕವಾದ ಒಂದು ಸುಂದರ ಭಾಷಣ ಮಾಡಿದರು. ಭಾಷಣದ ವಿಷಯ ‘ಪುಣ್ಯಭೂಮಿ ಭಾರತ’. ಸ್ವಾಮೀಜಿ ಆಡಿದ ಆ ಮಾತುಗಳೆಲ್ಲ ಕೇವಲ ಶಬ್ದಗಳಲ್ಲ, ಶಕ್ತಿಯ ನುಡಿಕಿಡಿಗಳು! ಆ ಭಾಷಣದ ಕೆಲವಂಶಗಳು ಇವು: \footnote{*ಸ್ವಾಮೀಜಿಯವರು ತಮ್ಮ ಮಹಾಸಮಾಧಿಯ ದಿನವನ್ನು ಮುನ್ನುಡಿದಿದ್ದರೆಂಬ ಈ ವಿಷಯವು ನಮಗೆ ತಿಳಿದುಬಂದಿರುವುದು ಮೇಡಂ ಪಾಲ್ ವರ್ಡಿಯರ್ ಎಂಬವಳ ಟಿಪ್ಪಣಿಗಳಿಂದ. ಈಕೆ ಎಮ್ಮಾ ಕಾಲ್ವೆಯ ಆತ್ಮೀಯ ಸ್ನೇಹಿತೆ ಮತ್ತು ಸ್ವಾಮೀಜಿಯವರ ಭಕ್ತೆ; ಮೆಕ್​ಲಾಡಳಿಗೂ ಈಕೆ ನಿಕಟ ಪರಿಚಿತೆ. ಚರ್ಚಾಸ್ಪದವಾದ ಈ ವಿಷಯಗಳನ್ನು ಮೇಡಂ ಕಾಲ್ವೆ ತನ್ನ ಆತ್ಮಚರಿತ್ರೆಯಲ್ಲಿ ಉಲ್ಲೇಖಿಸಿಲ್ಲವಾದರೂ ಪಾಲ್ ವರ್ಡಿಯರ್​ಳಿಗೆ ಹೇಳಿದಳಂತೆ. ಕಾಲ್ವೆ ಹಾಗೂ ಮೆಕ್​ಲಾಡ್ ಇಬ್ಬರೂ ತನಗೆ ತಿಳಿಸಿದ ವಿಷಯಗಳನ್ನು ಪಾಲ್ ವರ್ಡಿಯರ್ ಬರೆದಿಟ್ಟುಕೊಳ್ಳುತ್ತಿದ್ದಳು. ಆದರೆ, ಒಂದು ಅಂಶವೇನೆಂದರೆ, ಈ ಘಟನೆಯ ಬಗ್ಗೆ ಮಿಸ್ ಮೆಕ್​ಲಾಡ್ ಎಲ್ಲೂ ಪ್ರಸ್ತಾಪಿಸಿಲ್ಲ, ಅಥವಾ ಪಾಲ್ ವರ್ಡಿಯರ್​ಳಿಗೂ ಹೇಳಿರಲಿಲ್ಲ.}

“ಪಶ್ಚಿಮ ರಾಷ್ಟ್ರಗಳಲ್ಲಿ ನನ್ನಿಂದ ಯಾವ ಯತ್ಕಿಂಚಿತ್ ಕಾರ್ಯ ನಡೆಯಿತೋ ಅದನ್ನು ನಾನು ಕೇವಲ ನನ್ನಲ್ಲಿರುವ ಶಕ್ತಿಯಿಂದ ಮಾಡಿದುದಲ್ಲ. ನನ್ನ ಪ್ರೀತಿಯ, ಪರಮ ಪವಿತ್ರ ಮಾತೃಭೂಮಿ ಯಿಂದ ನನಗೆ ದೊರಕಿದ ಉತ್ತೇಜನ, ಶುಭಾಶಯ, ಆಶೀರ್ವಾದಗಳೇ ಆ ನನ್ನ ಕಾರ್ಯಕ್ಕೆ ಆಧಾರ, ಬೆಂಬಲ. ಪಾಶ್ಚಾತ್ಯ ದೇಶಗಳಲ್ಲಿ ನನ್ನ ಮೂಲಕ ಸ್ವಲ್ಪ ಒಳ್ಳೆಯ ಕಾರ್ಯವಾಗಿರುವು ದೇನೋ ನಿಜವೆ. ಆದರೆ ಇದರಿಂದ ಸ್ವತಃ ನನಗೆ ತುಂಬ ಪ್ರಯೋಜನವಾಗಿದೆ. ಯಾವುದು ಹಿಂದೆ ಕೇವಲ ಸೈದ್ಧಾಂತಿಕವಾಗಿತ್ತೋ ಅದು ಈಗ ಪ್ರಮಾಣಸಿದ್ಧ ಸತ್ಯದಂತೆ ತೋರುತ್ತಿದೆ. ಈ ಸಭೆಯ ಘನ ಅಧ್ಯಕ್ಷರು ಈಗತಾನೆ ಹೇಳಿದಂತೆ, ಮತ್ತು ಸಕಲ ಹಿಂದೂಗಳೂ ಭಾವಿಸುವಂತೆ, ನಾನೂ ಸಹ ಭರತಖಂಡವು ಪುಣ್ಯಭೂಮಿ-ಕರ್ಮಭೂಮಿ ಎಂದು ಕೇವಲ ಭಾವಿಸಿದ್ದೆ. ಆದರೆ ಇಂದು ನಾನು ನಿಮ್ಮೆದುರಿನಲ್ಲಿ ಅದು ಸತ್ಯ ಎಂದು ಘಂಟಾಘೋಷವಾಗಿ ಸಾರುತ್ತೇನೆ. ಪ್ರಪಂಚದಲ್ಲಿ ಯಾವುದಾದರೊಂದು ದೇಶವು ಪುಣ್ಯಭೂಮಿ ಎಂದು ಕರೆಸಿಕೊಳ್ಳಲು ಅರ್ಹವಾಗಿದ್ದರೆ, ಜೀವಿ ಗಳು ತಮ್ಮ ಕರ್ಮವನ್ನು ಸವೆಸಲು ಕೊಟ್ಟಕೊನೆಯದಾಗಿ ಬರಬೇಕಾದ ಸ್ಥಳವೊಂದಿದ್ದರೆ, ಭಗವಂತನೆಡೆಗೆ ಸಾಗುತ್ತಿರುವ ಪ್ರತಿಯೊಂದು ಜೀವಿಯೂ ತನ್ನ ಕೊನೆಯ ಯಾತ್ರೆಯನ್ನು ಪೂರೈಸಲು ಬರಬೇಕಾದ ಕರ್ಮಭೂಮಿಯೊಂದಿದ್ದರೆ, ಮಾನವನು ಮಾಧುರ್ಯ-ಔದಾರ್ಯ- ಪವಿತ್ರತೆ-ಶಾಂತಿ–ಇವುಗಳಲ್ಲಿ, ಮತ್ತು ಎಲ್ಲಕ್ಕಿಂತ ಮಿಗಿಲಾಗಿ ಧ್ಯಾನದಲ್ಲಿ ಹಾಗೂ ಅಂತರ್ ಮುಖತೆಯಲ್ಲಿ, ಪರಾಕಾಷ್ಠೆಯನ್ನು ಮುಟ್ಟಿದ ರಾಷ್ಟ್ರವೊಂದಿದ್ದರೆ ಅದು ಈ ಭರತಖಂಡವೇ. ಅನಾದಿಕಾಲದಿಂದಲೂ ಧರ್ಮಸಂಸ್ಥಾಪನಾಚಾರ್ಯರು ಪವಿತ್ರವಾದ, ಎಂದೆಂದಿಗೂ ಬತ್ತದ ತಮ್ಮ ಆಧ್ಯಾತ್ಮಿಕ ಮಹಾಸತ್ಯದ ಪ್ರವಾಹದಿಂದ ಪೃಥ್ವಿಯನ್ನೆಲ್ಲ ತೋಯಿಸಿದುದು ಈ ರಾಷ್ಟ್ರ ದಿಂದಲೇ... ಜಡನಾಗರಿಕತೆಯ ಪ್ರಪಂಚಕ್ಕೆ ಆಧ್ಯಾತ್ಮಿಕ ಚೈತನ್ಯವನ್ನೆರೆಯುವ ಮಹಾ ಪ್ರವಾಹವೂ ಇಂದು ಈ ರಾಷ್ಟ್ರದಿಂದಲೇ ಚಿಮ್ಮಬೇಕಾಗಿದೆ. ಇಲ್ಲಿದೆ–ಬಾಳಿಗೆ ಹೊಸ ಬೆಳಕನ್ನು ಕೊಡುವ ಅಮೃತ ಪ್ರವಾಹ! ಇಲ್ಲಿದೆ, ಕೋಟ್ಯಂತರ ಜೀವಿಗಳ ಎದೆಯನ್ನು ದಹಿಸುತ್ತಿರುವ ಜಡವಾದದ ದಳ್ಳುರಿಯನ್ನು ಶಮನಗೊಳಿಸಬಲ್ಲ ಅಮೃತಪ್ರವಾಹ! ಸ್ನೇಹಿತರೇ, ನಂಬಿ ಇದನ್ನು, ಈ ಕಾರ್ಯ ಕೈಗೂಡುವುದು ಖಂಡಿತ...”

ಬಳಿಕ ಸ್ವಾಮೀಜಿ, ಹಿಂದೂ ಧರ್ಮ-ಸಂಸ್ಕೃತಿಗಳು ಇತರೆಲ್ಲ ನಾಗರಿಕತೆಗಳಿಗಿಂತಲೂ ಹೇಗೆ ಶ್ರೇಷ್ಠವಾದವು ಎಂಬುದನ್ನು ವಿವರಿಸಿದರು. ಹಿಂದೂ ಧರ್ಮದಲ್ಲಿ ಮೇಲ್ನೋಟಕ್ಕೆ ಕಾಣುವ ಭಿನ್ನತೆಗಳ ಹಾಗೂ ವೈವಿಧ್ಯಗಳ ಹಿನ್ನೆಲೆಯಲ್ಲಿ ಆಧಾರಭೂತವಾದ ಒಂದು ಏಕತೆಯಿದೆ ಎಂದು ಹೇಳಿ, ಈ ಭಿನ್ನತೆಗಳು ಇರಬೇಕಾದುದರ ಅಗತ್ಯವನ್ನು ವಿವರಿಸಿದರು. ಅಲ್ಲದೆ ಹಿಂದೂ ಧರ್ಮದ ಒಂದು ಶ್ರೇಷ್ಠ ಅಂಶವಾದ ಪರಧರ್ಮ ಸಹಿಷ್ಣುತೆ-ಪರಧರ್ಮ ಸ್ವೀಕಾರದ ಬಗ್ಗೆ ಪ್ರಸ್ತಾಪಿಸಿ ಅದು ಭಾರತದಿಂದ ಇಡೀ ಜಗತ್ತೇ ಕಲಿಯಬೇಕಾಗಿರುವ ಅತ್ಯಂತ ಪ್ರಮುಖವಾದ ಪಾಠ ಎಂದರು.

ಇದು, ಪಶ್ಚಿಮ ದೇಶಗಳ ವಿಜಯಯಾತ್ರೆಯನ್ನು ಮುಗಿಸಿ ಬಂದ ಸ್ವಾಮೀಜಿ ಭರತಖಂಡ ದಲ್ಲಿ ನೀಡಿದ ಪ್ರಥಮ ಉಪನ್ಯಾಸ.

ಸ್ವಾಮೀಜಿಯವರೊಂದಿಗೆ ಬಂದಿದ್ದ ಪಾಶ್ಚಾತ್ಯ ಶಿಷ್ಯರಾದ ಜೆ. ಜೆ. ಗುಡ್​ವಿನ್ ಹಾಗೂ ಸೇವಿಯರ್ ದಂಪತಿಗಳು ಅವರೊಂದಿಗೇ ಉಳಿದುಕೊಂಡರಲ್ಲದೆ ಎಲ್ಲ ಸಮಾರಂಭಗಳಲ್ಲೂ ಭಾಗವಹಿಸಿದರು. ಜನರು ಸ್ವಾಮೀಜಿಯವರನ್ನು ಸ್ವೀಕರಿಸಿ ಸ್ವಾಗತಿಸಿದ ಪರಿಯನ್ನು ಕಂಡು ಇವರಿ ಗಾದ ಆಶ್ಚರ್ಯಕ್ಕೆ ಪಾರವೇ ಇಲ್ಲ. ಆದರೆ ಜನರು ತೋರಿದ ಆ ಅಭಿಮಾನ-ಗೌರವ-ಭಕ್ತಿ ಗಳಿಗೆಲ್ಲಕ್ಕೂ ಸ್ವಾಮೀಜಿ ಅರ್ಹರಾಗಿದ್ದರು, ಮತ್ತು ಅದೆಲ್ಲ ಕೇವಲ ಆವೇಶದಿಂದ ಉಂಟಾದು ದಲ್ಲ ಎಂಬುದನ್ನು ಅವರು ಸ್ಪಷ್ಟವಾಗಿ ಕಾಣುತ್ತಿದ್ದರು.

ಭಾನುವಾರದಂದು ಕೊಲಂಬೋದ ನಾಗರಿಕರು ಸ್ವಾಮೀಜಿಯವರನ್ನು ನಗರದ ಪ್ರಸಿದ್ಧ ಶಿವದೇವಾಲಯಕ್ಕೆ ಕರೆದೊಯ್ದರು. ಜನರ ಪಾಲಿಗೆ ಈಗ ಸ್ವಾಮೀಜಿ ಕೇವಲ ಒಬ್ಬ ವ್ಯಕ್ತಿಯಾಗಿರ ಲಿಲ್ಲ, ಅಥವಾ ಒಬ್ಬ ಸಂತನೂ ಅಲ್ಲ; ಅವರು ಸಾಕ್ಷಾತ್ ಭಗವಂತನೇ ಎಂಬ ದೃಷ್ಟಿಯಿಂದ ಜನ ಪೂಜಿಸಲಾರಂಭಿಸಿದ್ದರು. ಸ್ವಾಮೀಜಿಯವರನ್ನು ಒಂದು ಸಾಲಂಕೃತ ಗಾಡಿಯಲ್ಲಿ ಕುಳ್ಳಿರಿಸ ಲಾಗಿತ್ತು. ಸಾವಿರಾರು ಜನ ಮೆರವಣಿಗೆಯಲ್ಲಿ ಭಾಗವಹಿಸಿದ್ದರು. ಅದು ಜನರ ಮೆರವಣಿಗೆ ಯಂತೆ ಕಾಣುತ್ತಿರಲಿಲ್ಲ; ಬದಲಾಗಿ ಅದೊಂದು ರಥೋತ್ಸವದಂತಿತ್ತು! ಶ್ರೀಮತಿ ಸಾರಾ ಬುಲ್​ಳಿಗೆ ಬರೆದ ಒಂದು ಪತ್ರದಲ್ಲಿ ಗುಡ್​ವಿನ್ ಆ ದೃಶ್ಯವನ್ನು ಬಣ್ಣಿಸುತ್ತ ಹೀಗೆ ಹೇಳುತ್ತಾನೆ: “ಭಾರತದಲ್ಲಿ ಸಂನ್ಯಾಸಿಯೆಂದರೆ ಶಿವನೊಂದಿಗೆ ತಾದಾತ್ಮ್ಯ ಹೊಂದಿದವನೆಂದು ಭಾವಿಸುತ್ತಾರೆ. ಆ ದೃಷ್ಟಿಯಿಂದ ಆತ ಶಿವನೇ ಆಗುತ್ತಾನೆ. ಆದ್ದರಿಂದ ಸ್ವಾಮೀಜಿಯವರನ್ನೂ ಜನರು ಸಾಕ್ಷಾತ್ ಭಗವಂತನೆಂಬ ದೃಷ್ಟಿಯಿಂದ ಕಂಡು ಪೂಜಿಸುತ್ತಿದ್ದಾರೆ. ಅವರ ಈ ಭಾವನೆಯು ನಿಸ್ಸಂಶಯ ವಾಗಿಯೂ ಸರಿಯಾದದ್ದು ಎಂದೇ ನನ್ನ ನಿಶ್ಚಿತ ಅಭಿಪ್ರಾಯ... ” ದೇವಸ್ಥಾನಕ್ಕೆ ಹೋಗುವ ದಾರಿಯಲ್ಲಿ ಸ್ವಾಮೀಜಿ ಜನರ ಉಪಚಾರವನ್ನು ಸ್ವೀಕರಿಸಲು ಸುಮಾರು ಐವತ್ತಕ್ಕೂ ಹೆಚ್ಚು ಮನೆಗಳ ಮುಂದೆ ನಿಲ್ಲಬೇಕಾಯಿತು. ಹಾಗೆ ನಿಂತಾಗಲೆಲ್ಲ ಜನ ಅವರಿಗೆ ಪುಷ್ಪಹಾರಗಳನ್ನು ಅರ್ಪಿಸಿ ಸುಗಂಧ ದ್ರವ್ಯಗಳನ್ನು ಚಿಮುಕಿಸಿ ಪ್ರಣಾಮ ಮಾಡುತ್ತಿದ್ದರು. ಇನ್ನು ಕೆಲವರು ಅವರ ಕೈಗೆ ಫಲಪುಷ್ಪಗಳನ್ನಿತ್ತರು. ಮೆರವಣಿಗೆ ದೇವಸ್ಥಾನವನ್ನು ತಲುಪುತ್ತಿದ್ದಂತೆ ಅವರನ್ನು ಮಂಗಳ ವಾದ್ಯಗಳೊಂದಿಗೆ ಸ್ವಾಗತಿಸಲಾಯಿತು. ಅನಂತರ ದೇವಸ್ಥಾನವನ್ನೊಮ್ಮೆ ಪ್ರದಕ್ಷಿಣೆ ಹಾಕಿ ಸ್ವಾಮೀಜಿಯವರನ್ನು ಒಳಗೆ ಕರೆದೊಯ್ಯಲಾಯಿತು. ಆಗ ಅಲ್ಲಿ ನೆರೆದಿದ್ದವರೆಲ್ಲ ಗಟ್ಟಿಯಾಗಿ “ಜೈ ಜೈ ಮಹಾದೇವ! ಜೈ ಜೈ ಮಹಾದೇವ!” ಎಂದು ಉದ್ಘೋಷಿಸುತ್ತ ಕರತಾಡನ ಮಾಡಿದರು. ಜನರ ಹರ್ಷೋದ್ಗಾರ ಎಲ್ಲೆಲ್ಲೂ ಪ್ರತಿಧ್ವನಿಸುತ್ತಿತ್ತು. ಬಳಿಕ ಸ್ವಾಮೀಜಿಯವರಿಗೆ ಮತ್ತಷ್ಟು ಪುಷ್ಪಮಾಲೆ-ಗಂಧಚಂದನ-ಸುಗಂಧದ್ರವ್ಯಗಳನ್ನು ಸಮರ್ಪಿಸಲಾಯಿತು.

ಸ್ವಾಮೀಜಿಯವರೊಂದಿಗೆ ಬಂದಿದ್ದ ಗುಡ್​ವಿನ್ ಹಾಗೂ ಸೇವಿಯರ್ ದಂಪತಿಗಳ ವಿಷಯ ದಲ್ಲಿ ಜನರಿಗೆ ಒಂದು ವಿಶೇಷ ಕುತೂಹಲ. ಜೊತೆಗೆ ಸ್ವಾಮೀಜಿ ಸುಶಿಕ್ಷಿತರಾದ ಆಂಗ್ಲ ಶಿಷ್ಯರನ್ನು ಸಂಪಾದಿಸಿಕೊಂಡು ಬಂದಿರುವ ಬಗ್ಗೆ ಹೆಮ್ಮೆ ಸಹ. ಗುಡ್​ವಿನ್ ಮತ್ತು ಸೇವಿಯರ್ ದಂಪತಿಗಳ ಮೇಲೂ ಜನರು ಪನ್ನೀರು ಚಿಮುಕಿಸಿ, ಅವರ ಕೈಗೆ ಗಂಧಚಂದನಗಳನ್ನು ಕೊಟ್ಟರು. ಅವರನ್ನು ಕುತೂಹಲದಿಂದ ಮುತ್ತಿಕೊಂಡು ಬಗೆಬಗೆಯ ಪ್ರಶ್ನೆಗಳನ್ನು ಕೇಳುತ್ತಿದ್ದರು. ಒಬ್ಬನಂತೂ ಗುಡ್​ವಿನ್ನನ ಬಳಿಗೆ ಬಂದು, “ಸ್ವಾಮೀಜಿಯವರೊಂದಿಗೆ ನೀವಿರುವ ಒಂದು ಛಾಯಾಚಿತ್ರವನ್ನು ತೆಗೆದುಕೊಂಡು, ಅವರ ಜೊತೆಯಲ್ಲಿ ನಿಮ್ಮನ್ನೂ ಪೂಜಿಸಬೇಕೆಂದಿದ್ದೇನೆ; ದಯವಿಟ್ಟು ಅವ ಕಾಶ ಮಾಡಿಕೊಡಬೇಕು” ಎಂದು ಕೇಳಿಕೊಂಡ! ಆಗಿನ ಕಾಲದಲ್ಲಿ ಭಾರತೀಯರಿಗೆಲ್ಲ ಬಿಳಿಚರ್ಮದವರೆಂದರೆ ಅದೇನೋ ಒಂದು ವಿಶೇಷ ಗೌರವ. ಬಿಳಿಚರ್ಮದವರು ತಮ್ಮೊಂದಿಗೆ ಒಂದು ಮಾತನಾಡಿದರೆ ತಮ್ಮ ಜನ್ಮವೇ ಸಾರ್ಥಕವಾಯಿತೆಂದು ಭಾವಿಸುತ್ತಿದ್ದ ಕಾಲ ಅದು. ಆದ್ದರಿಂದಲೇ ಆ ವ್ಯಕ್ತಿ ಸ್ವಾಮೀಜಿಯವರೊಂದಿಗೆ ಗುಡ್​ವಿನ್ನನ ಚಿತ್ರವನ್ನು ಪೂಜಿಸಲು ಮುಂದಾಗಿರಬಹುದು; ಆದರೆ ನಿಜಕ್ಕೂ ಗುಡ್​ವಿನ್ ಪೂಜಾರ್ಹನೇ ಸರಿ.

ಸ್ವಾಮೀಜಿ ಇಳಿದುಕೊಂಡಿದ್ದ ಬಂಗಲೆಗೆ ಜನ ನಿರಂತರವಾಗಿ ಮುತ್ತಿಕೊಂಡೇ ಇರುತ್ತಿದ್ದರು. ನಿಜಕ್ಕೂ ಅದೊಂದು ತೀರ್ಥಕ್ಷೇತ್ರವೇ ಆಗಿಬಿಟ್ಟಿತ್ತು. (ಆ ಬಂಗಲೆಗೆ ‘ವಿವೇಕಾನಂದ ಲಾಡ್ಜ್​’ ಎಂಬ ಹೆಸರಾಯಿತು.) ಈ ಸಂದರ್ಶಕರಲ್ಲಿ ಸರ್ಕಾರದ ಉನ್ನತ ಅಧಿಕಾರಿಗಳಿಂದ ಹಿಡಿದು ಸಮಾಜದ ಅತ್ಯಂತ ಕೆಳವರ್ಗದವರವರೆಗೆ ಎಲ್ಲ ಬಗೆಯ ಜನರಿದ್ದರು. ಧರ್ಮ-ಅಧ್ಯಾತ್ಮ ಪ್ರಧಾನವಾದ ಹಿಂದೂ ಸಮಾಜವು ಸಂನ್ಯಾಸಿಯೋರ್ವನಿಗೆ ಸಲ್ಲಿಸಿದ ಭಕ್ತಿ ಗೌರವಗಳನ್ನು ಕಂಡ ಪಾಶ್ಚಾತ್ಯ ಶಿಷ್ಯರು ವಿಸ್ಮಿತಮೂಕರಾಗಿದ್ದರು.

ಒಮ್ಮೆ, ತುಂಬ ಕಷ್ಟದಲ್ಲಿದ್ದಂತೆ ಕಾಣುತ್ತಿದ್ದ ಬಡಮಹಿಳೆಯೊಬ್ಬಳು ಸ್ವಾಮೀಜಿಯವರ ದರ್ಶನಕ್ಕಾಗಿ ಬಂದಳು. ಹಣ್ಣು-ಕಾಯಿಯ ಕಾಣಿಕೆಯನ್ನು ತರಲು ಆಕೆ ಮರೆತಿರಲಿಲ್ಲ. ಆಕೆಯ ಗಂಡನಿಗೆ ಸಂಸಾರದಲ್ಲಿ ವಿರಕ್ತಿ ಹುಟ್ಚಿದ್ದರಿಂದ, ನಿಶ್ಚಲ ಮನಸ್ಸಿನಿಂದ ಆಧ್ಯಾತ್ಮಿಕ ಸಾಧನೆಯನ್ನು ಕೈಗೊಳ್ಳಬೇಕೆಂದು ಪತ್ನಿಯನ್ನು ತ್ಯಜಿಸಿಬಿಟ್ಟಿದ್ದನಂತೆ. ಈಗ ಆ ಮಹಿಳೆ ತಾನೂ ಪತಿಯಂತೆಯೇ ವಿರಕ್ತ ಜೀವನ ನಡೆಸಲು ನಿರ್ಧರಿಸಿ, ಸ್ವಾಮೀಜಿಯವರ ಸಲಹೆ ಕೇಳಲು ಬಂದಿದ್ದಳು. ಅವಳ ಶಾಂತ-ಸ್ಥಿರ ಬುದ್ಧಿಯನ್ನು ಕಂಡು ಅಲ್ಲಿದ್ದವರೆಲ್ಲ ಬೆರಗಾದರು. ಅವಳು ಹೇಳಿದ್ದನ್ನೆಲ್ಲ ಕೇಳಿದ ಸ್ವಾಮೀಜಿ, “ನಿನ್ನ ಸದ್ಯದ ಪರಿಸ್ಥಿತಿಯಲ್ಲಿ, ಸಾಂಸಾರಿಕ ಜವಾಬ್ದಾರಿಯನ್ನು ಸರಿಯಾದ ರೀತಿ ಯಲ್ಲಿ ಪೂರೈಸುವುದೇ ನಿನಗಿರುವ ಅತ್ಯುತ್ತಮ ಮಾರ್ಗ” ಎಂದು ನಿರ್ದೇಶಿಸಿದರು. ಅಲ್ಲದೆ ಭಗವದ್ಗೀತೆಯನ್ನು ಅಧ್ಯಯನ ಮಾಡುವಂತೆ ಆಕೆಗೆ ಸಲಹೆ ನೀಡಿದರು. ಅದಕ್ಕೆ ಆ ಬಡ ಮಹಿಳೆ, “ನಾನು ಗೀತೆಯನ್ನೇನೋ ಓದಬಲ್ಲೆ ಸ್ವಾಮೀಜಿ; ಆದರೆ ನಾನು ಅದನ್ನು ಸರಿಯಾಗಿ ಅರ್ಥ ಮಾಡಿಕೊಂಡು, ಅದರಂತೆಯೇ ನಡೆದುಕೊಳ್ಳಲು ಸಾಧ್ಯವಾಗದೆ ಹೋದರೆ ಓದಿ ತಾನೆ ಏನು ಪ್ರಯೋಜನ?” ಎಂದುತ್ತರಿಸಿದಳು. ಧರ್ಮವಿರುವುದು ಆಚರಣೆಯಲ್ಲೇ ಹೊರತು ಮಾತಿನ ಲ್ಲಲ್ಲ ಅಥವಾ ಪಾರಾಯಣದಲ್ಲಲ್ಲ ಎಂಬ ಉನ್ನತ ಸತ್ಯವನ್ನು ಈ ಸರಳ-ಅಶಿಕ್ಷಿತ ಮಹಿಳೆಯ ಬಾಯಿಂದ ಕೇಳಿದವರೆಲ್ಲ, ಭಾರತದಲ್ಲಿ ಆಧ್ಯಾತ್ಮಿಕತೆಯು ಹೇಗೆ ಆಳವಾಗಿ ಬೇರೂರಿದೆ ಯೆಂಬುದನ್ನು ಭಾವಿಸಿ ಬೆರಗಾದರು.

ಸೋಮವಾರದಂದು ಬೆಳಿಗ್ಗೆ ಸ್ವಾಮೀಜಿ ಶ್ರೀ ಚಲ್ಲಯ್ಯ ಎಂಬುವರ ಆಹ್ವಾನವನ್ನು ಮನ್ನಿಸಿ ಅವರ ಮನೆಗೆ ಹೋದರು. ಅಷ್ಟರಲ್ಲಾಗಲೇ ಆ ಮನೆಯ ಸುತ್ತಮುತ್ತ ಸಾವಿರಾರು ಜನ ಸೇರಿ ಬಿಟ್ಟಿದ್ದರು. ಸ್ವಾಮೀಜಿ ಗಾಡಿಯಲ್ಲಿ ಬಂದಿಳಿಯುತ್ತಿದ್ದಂತೆಯೇ ಜನ ಸ್ಫೂರ್ತಿಭರಿತರಾಗಿ ಜಯ ಘೋಷ ಮಾಡಿ, ಅವರ ಮೇಲೆ ಹೂಮಳೆಗರೆದರು. ಮನೆಯೊಳಗೆ ಪ್ರವೇಶಿಸಿ ಸ್ವಾಮೀಜಿ ತಮಗಾಗಿ ಸಿದ್ಧವಾಗಿರಿಸಿದ್ದ ಆಸನವನ್ನಲಂಕರಿಸಿದರು. ಪವಿತ್ರ ಗಂಗಾಜಲವನ್ನು ಅವರ ಮೇಲೆ ಪ್ರೋಕ್ಷಿಸಲಾಯಿತು. ಗೋಡೆಯ ಮೇಲೆ ತೂಗು ಹಾಕಿದ್ದ ಶ್ರೀರಾಮಕೃಷ್ಣರ ಒಂದು ಭಾವಚಿತ್ರ ಸ್ವಾಮೀಜಿಯವರ ಕಣ್ಣಿಗೆ ಬಿತ್ತು. ತಕ್ಷಣ ಅವರು ಎದ್ದುನಿಂತು ಭಕ್ತಿಭಾವದಿಂದ ಗುರುದೇವನಿಗೆ ಪ್ರಣಾಮ ಸಲ್ಲಿಸಿದರು. ಈ ದೂರದ ಸಿಂಹಳದಲ್ಲಿ ಶ್ರೀರಾಮಕೃಷ್ಣರ ಭಾವಚಿತ್ರವನ್ನು ಕಂಡು ಅವರಿಗೆಷ್ಟು ಆನಂದವಾಗಿರಬಹುದೆಂಬುದನ್ನು ಊಹಿಸಿ ನೋಡಬೇಕು. ದಕ್ಷಿಣಭಾರತದಲ್ಲಿ ರಾಮಕೃಷ್ಣ ಮಠದ ಒಂದು ಶಾಖೆಯೂ ಇರದಿದ್ದ ಆ ಕಾಲದಲ್ಲೇ ಆ ಭಾವಚಿತ್ರವು ಸಿಲೋನಿನ ವರೆಗೂ ತಲುಪಿದ್ದುದು ದೊಡ್ಡ ಆಶ್ಚರ್ಯದ ಮಾತೇ ಸರಿ.

ಅದೇ ಸಂಜೆ ಸ್ವಾಮೀಜಿ ಕೊಲಂಬೋದ ಪುರಭವನದಲ್ಲಿ ಎರಡನೆಯ ಭಾಷಣ ಮಾಡಿದರು. ವಿಷಯ: ‘ಅದ್ವೈತ ವೇದಾಂತ’. ಸುದೃಢ ಶರೀರದ, ಆಧ್ಯಾತ್ಮಿಕ ಕಾಂತಿಯನ್ನು ಸೂಸುವಂತಹ ಮುಖಮಂಡಲದಿಂದ ಕೂಡಿದ, ಎತ್ತರದ ನಿಲುವಿನ ಕಾಷಾಯಧಾರಿ ಸಂನ್ಯಾಸಿಯೊಬ್ಬನು ವೇದಿಕೆಯ ಮೇಲೆ ನಿಂತು ನಿರರ್ಗಳ ಇಂಗ್ಲಿಷಿನಲ್ಲಿ ವೇದಾಂತದ ಬಗ್ಗೆ ಮಾತನಾಡುತ್ತಿದ್ದು ದೊಂದು ಅಪೂರ್ವ-ಭವ್ಯ ದೃಶ್ಯವಾಗಿತ್ತು. ಯಾರೂ ಹಿಂದೆಂದೂ ಕಂಡಿರದಿದ್ದ ನೋಟ ಅದು. ಅದನ್ನು ಕಣ್ಣಾರೆ ಕಂಡವರಿಗೆ, ಸ್ವಾಮೀಜಿ ಪಾಶ್ಚಾತ್ಯರ ಮೇಲೆ ಬೀರಿದ ಪ್ರಭಾವವೆಂತಹದೆಂಬ ಕಲ್ಪನೆಯಾಗುತ್ತಿತ್ತು. ವೇದಗಳ ಮೇಲೆ ಆಧಾರಿತವಾದ ಹಿಂದೂಧರ್ಮವು ಹೇಗೆ ವಿಶ್ವಧರ್ಮ ವಾಗಬಲ್ಲುದೆಂಬುದನ್ನು ಸ್ವಾಮೀಜಿ ವಿವರಿಸಿದರು. ಸಭಿಕರ ಪೈಕಿ ಹೆಚ್ಚಿನವರು ಪಾಶ್ಚಾತ್ಯ ಉಡುಗೆಯನ್ನು ಧರಿಸಿದ್ದುದನ್ನು ಗಮನಿಸಿದ ಅವರು, ಭಾಷಣದ ಸಂದರ್ಭದಲ್ಲೇ ಆ ಬಗ್ಗೆ ಛೀಮಾರಿ ಹಾಕಿದರು. ಪಾಶ್ಚಾತ್ಯ ಉಡುಗೆಯನ್ನು ಧರಿಸುವುದು ಭಾರತೀಯರಿಗೆ ಅನುಕೂಲ ಕರವಲ್ಲ; ಜೊತೆಗೆ ಈ ಅನುಕರಣೆಯೂ ಭಾರತೀಯರ ಗುಲಾಮಬುದ್ಧಿಯ ಪ್ರತೀಕ ಎಂದರು. ಯಾವುದೋ ಒಂದು ನಿರ್ದಿಷ್ಟ ಉಡುಗೆಯನ್ನೇ ಧರಿಸಬೇಕೆಂದು ಸ್ವಾಮೀಜಿ ಹೇಳುವವರಲ್ಲ. ಆದರೆ ಪಾಶ್ಚಾತ್ಯವಾದ ಪ್ರತಿಯೊಂದನ್ನೂ ಕುರುಡಾಗಿ ಅನುಕರಿಸುವ, ಭಾರತದ ವಿದ್ಯಾವಂತ ರೆನ್ನಿಸಿಕೊಂಡವರ ಮನೋಭಾವವನ್ನು ಅವರು ಕಟುವಾಗಿ ಟೀಕಿಸಿದರು. ಸ್ವತಃ ಸ್ವಾಮೀಜಿಯ ವರು, ಅಂತಹ ಕಲ್ಕತ್ತದಲ್ಲಿ ಕಾಲೇಜು ವಿದ್ಯಾರ್ಥಿಯಾಗಿದ್ದಾಗ ಬೇಕೆಂದೇ ಪಾಶ್ಚಾತ್ಯ ಸಮವಸ್ತ್ರ ವನ್ನು ತ್ಯಜಿಸಿ ಅಚ್ಚ ಭಾರತೀಯ ಉಡುಗೆಯನ್ನೇ ಧರಿಸುತ್ತಿದ್ದುದನ್ನು ಇಲ್ಲಿ ಸ್ಮರಿಸಬಹುದು.

ಕೊಲಂಬೋದಿಂದ ಸಮುದ್ರಮಾರ್ಗವಾಗಿ ಪಯಣಿಸಿ ನೇರವಾಗಿ ಮದ್ರಾಸು ತಲುಪ ಬೇಕೆಂಬುದು ಮೊದಲಿಗೆ ಸ್ವಾಮೀಜಿಯವರ ಆಲೋಚನೆಯಾಗಿತ್ತು. ಅದರೆ ಸಿಲೋನಿನ ಹಾಗೂ ದಕ್ಷಿಣ ಭಾರತದ ಹಲವಾರು ನಗರಗಳಿಂದ ಅವರಿಗೆ ತಂತಿಗಳ ಸುರಿಮಳೆಯಾಗುತ್ತಿತ್ತು. ಅವರು ಸುಮ್ಮನೆ ತಮ್ಮ ಊರುಗಳ ಮೂಲಕ ಹಾದು ಹೋದರೂ ಸಾಕು, ಕೇವಲ ಅವರ ದರ್ಶನಭಾಗ್ಯ ವನ್ನು ಕರುಣಿಸಿದರೂ ಸಾಕು, ಎಂದು ಎಲ್ಲೆಡೆಗಳಿಂದ ಜನ ಪ್ರಾರ್ಥಿಸಿಕೊಂಡಿದ್ದರು. ಹೀಗೆ ಅಸಂಖ್ಯಾತ ಜನರು ಬೇಡಿಕೊಂಡಾಗ ಸ್ವಾಮೀಜಿ ತಮ್ಮ ಕಾರ್ಯಕ್ರಮವನ್ನು ಬದಲಿಸಿಕೊಳ್ಳ ಬೇಕಾಯಿತು. ಈ ಸಂತೋಷ ಸಮಾರಂಭಗಳಲ್ಲೆಲ್ಲ ಭಾಗವಹಿಸುವ ಇಚ್ಛೆ ಅವರಿಗೆ ಸ್ವಲ್ಪವೂ ಇರಲಿಲ್ಲ. ಅವರು ಬಂಗಾಳಕ್ಕೆ ಹಿಂದಿರುಗಿ ಕೈಗೆತ್ತಿಕೊಳ್ಳಬೇಕಾದ ಕಾರ್ಯ ಪರ್ವತೋಪಮವಾ ಗಿತ್ತು. ಆಗಲೇ ಮೂರ್ನಾಲ್ಕು ವರ್ಷಗಳ ನಿರಂತರ ಶ್ರಮದಿಂದ ಜರ್ಜರಿತವಾಗಿದ್ದ ಅವರ ಶರೀರಕ್ಕೆ, ಸುದೀರ್ಘ ಸಮುದ್ರ ಪ್ರಯಾಣದಿಂದಾಗಿ ಮತ್ತಷ್ಟು ಪ್ರಯಾಸವಾಗಿತ್ತು. ಈಗ ಮತ್ತೆ ಸಿಲೋನ್ ಹಾಗೂ ದಕ್ಷಿಣ ಭಾರತಗಳಲ್ಲಿ ಪ್ರಯಾಣ ಮಾಡಬೇಕೆಂದರೆ ಅದೇನೂ ಹುಡುಗಾಟಿಕೆ ಯಲ್ಲ. ಆದರೂ ಅವರು ಜನಗಳ ಒತ್ತಾಯದ ಕೋರಿಕೆಯ ಮೇರೆಗೆ ತಮ್ಮ ಕಾರ್ಯಕ್ರಮವನ್ನು ಪುನರ್​ರೂಪಿಸಿಕೊಂಡರು. 

ಜನವರಿ ೧೯ರಂದು ಸ್ವಾಮೀಜಿ ಕೊಲಂಬೋದಿಂದ ಕ್ಯಾಂಡಿ ನಗರಕ್ಕೆ ವಿಶೇಷ ರೈಲುಬಂಡಿ ಯಲ್ಲಿ ಪ್ರಯಾಣ ಮಾಡಿದರು. ಕ್ಯಾಂಡಿಯ ನಿಲ್ದಾಣದಲ್ಲಿ ಅವರನ್ನು ಸ್ವಾಗತಿಸಲು ಭಾರೀ ಜನಸಮೂಹವೇ ಸೇರಿತ್ತು. ಅಲ್ಲದೆ ಬಾಜಾಬಜಂತ್ರಿಗಳೂ ದೇವಸ್ಥಾನದ ಧ್ವಜ-ಪತಾಕೆಗಳೂ ಸಿದ್ಧವಾಗಿದ್ದುವು. ಸ್ವಾಮೀಜಿಯವರನ್ನು ಅತ್ಯಂತ ಸಂಭ್ರಮದಿಂದ ಬರಮಾಡಿಕೊಂಡು, ಅವರಿ ಗಾಗಿ ಸಿದ್ಧಪಡಿಸಿದ್ದ ಬಂಗಲೆಗೆ ಮೆರವಣಿಗೆಯಲ್ಲಿ ಕರೆದೊಯ್ಯಲಾಯಿತು. ಜಯಘೋಷಗಳೆಲ್ಲ ನಿಂತಮೇಲೆ ಕ್ಯಾಂಡಿಯ ನಾಗರಿಕರ ಪರವಾಗಿ ಅವರಿಗೊಂದು ಬಿನ್ನವತ್ತಳೆಯನ್ನು ಸಮರ್ಪಿಸಲಾ ಯಿತು. ಸ್ವಾಮೀಜಿ ಅದಕ್ಕೆ ಸಂಕ್ಷೇಪವಾಗಿ ಉತ್ತರಿಸಿ, ತಮ್ಮ ಕೃತಜ್ಞತೆಯನ್ನು ಸಲ್ಲಿಸಿದರು. 

ಅಂದು ಅಲ್ಲಿನ ಸುತ್ತಮುತ್ತಲ ಕೆಲವು ಪ್ರೇಕ್ಷಣೀಯ ಸ್ಥಳಗಳನ್ನು ಸಂದರ್ಶಿಸಿದ ಸ್ವಾಮೀಜಿ, ಮರುದಿನ ಬೆಳಗ್ಗೆ ತಮ್ಮ ಸಂಗಡಿಗರೊಂದಿಗೆ ಪ್ರಯಾಣವನ್ನು ಮುಂದುವರಿಸಿದರು. ಪ್ರಯಾಣ ಕುದುರೆ ಸಾರೋಟಿನಲ್ಲಿ. ಈಗ ಅವರ ಗುರಿ, ೨ಂಂ ಮೈಲಿ ದೂರದಲ್ಲಿರುವ, ಸಿಲೋನಿನ ಉತ್ತರ ತುದಿಯ ಜಾಫ್ನಾ. ಇಷ್ಟು ದೂರವನ್ನು ಕುದುರೆಗಾಡಿಯಲ್ಲಿ–ಅದೂ ಆಗಿನ ಕಾಲದ ರಸ್ತೆಗಳಲ್ಲಿ –ಮಾಡಬೇಕೆಂದರೆ ಅದು ಹೇಗಿದ್ದೀತೆಂದು ಊಹಿಸಬಹುದು. ಆದರೆ ದಾರಿಯುದ್ದಕ್ಕೂ ಹಚ್ಚ ಹಸುರಿನ ತೋಟಗಳು, ವನಗಳು ಇದ್ದುದರಿಂದ ಪ್ರಯಾಣದ ಪರಿಶ್ರಮ ಎಷ್ಟೋ ಕಡಿಮೆ ಯಾಗಿತ್ತು. ಅವರ ಮೊದಲ ನಿಲ್ದಾಣವೆಂದರೆ ಎಪ್ಪತ್ತು ಮೈಲಿ ದೂರದ ಅನೂರಾಧಪುರ. ದಂಬೂಲ್ ಎಂಬಲ್ಲಿನವರೆಗೂ ಪ್ರಯಾಣ ಆಹ್ಲಾದಕರವಾಗಿಯೇ ಇತ್ತು. ದಾರಿಯ ಅಕ್ಕಪಕ್ಕ ದಲ್ಲಿದ್ದ ಏಲಕ್ಕಿ-ಮೆಣಸು-ತೆಂಗು-ಬಾಳೆ ಮೊದಲಾದ ತೋಟಗಳು, ವಿಧವಿಧದ ಹಣ್ಣಿನ ಮರಗಳು, ಬತ್ತದ ಗದ್ದೆಗಳು ಇವೆಲ್ಲ ಕಣ್ಣಿಗೆ ಹಬ್ಬವುಂಟುಮಾಡುತ್ತಿದ್ದುವು. ಅಲ್ಲದೆ ಬಗೆಬಗೆಯ ವರ್ಣ ಮಯ ಹಕ್ಕಿಗಳಿಂದ ಕೂಡಿದ ವನರಾಶಿಯನ್ನು ಕಂಡ ಪಾಶ್ಚಾತ್ಯ ಶಿಷ್ಯರು ಆನಂದತುಂದಿಲ ರಾದರು. ದಂಬೂಲ್​ನಲ್ಲಿ ಗಾಡಿಯಿಂದಿಳಿದು ಬೆಳಗಿನ ಉಪಾಹಾರವನ್ನು ಸೇವಿಸಿ, ಬಳಿಕ ಪ್ರಯಾಣವನ್ನು ಮುಂದುವರಿಸಿದರು. ಅಷ್ಟು ಹೊತ್ತಿಗೆ ಚೆನ್ನಾಗಿ ಬಿಸಿಲೇರಿತ್ತು. ಕುದುರೆಗಳೂ ಬಳಲಿದ್ದುವು. ಒಂದು ಕುದುರೆಯಂತೂ ತುಂಬ ತಂಟೆ ಮಾಡಿದ್ದರಿಂದ ಗಾಡಿಯನ್ನು ಪ್ರಯಾ ಣಿಕರೇ ತಳ್ಳಿಕೊಂಡು ಹೋಗಬೇಕಾಯಿತು. ಸ್ವಲ್ಪ ದೂರ ಹೀಗೆ ಹೋದಮೇಲೆ ಮತ್ತೆ ಅವರು ಗಾಡಿ ಹತ್ತಿದರು. ಆದರೆ ಇನ್ನೂ ಭಯಂಕರವಾದ ಕಷ್ಟ ಬಾಕಿಯಿತ್ತು. ಗಾಡಿಯವನು ಕುದುರೆ ಗಳಿಗೆ ಚಬುಕಿನಿಂದ ಬಾರಿಸಿದಾಗ ಅವು ಜೋರಾಗಿ ಓಡತೊಡಗಿದುವು. ಹೀಗೆ ಹೋಗುತ್ತಿದ್ದಾಗ ಆ ಗಾಡಿಯ ನಾಲ್ಕು ಚಕ್ರಗಳಲ್ಲಿ ಮುಂದಿನದೊಂದು ಇದ್ದಕ್ಕಿದ್ದಂತೆ ಅರ್ಧಕ್ಕೆ ಮುರಿದು ಕೊಂಡಿತು! ಅದು ಓಡುತ್ತಿದ್ದ ವೇಗಕ್ಕೆ ಗಾಡಿ ಉರುಳಿಕೊಳ್ಳಬಹುದಾಗಿತ್ತು. ಆದರೆ ಭಗವಂತನ ದಯೆಯಿಂದ ಚಕ್ರದ ಅರಗಳು ಭದ್ರವಾಗಿದ್ದುದರಿಂದ ಹಾಗೇನೂ ಆಗಲಿಲ್ಲ. ಆದರೆ ಈಗ ಅವರು ಜನವಸತಿಯ ಪ್ರದೇಶದಿಂದ ಬಹುದೂರದಲ್ಲಿ ಸಿಕ್ಕಿಹಾಕಿಕೊಂಡಿದ್ದರು. ಅಲ್ಲಿಗೆ ಅತಿ ಸಮೀಪದ ಮನೆಯಿದ್ದುದು ನಾಲ್ಕು ಮೈಲಿ ದೂರದಲ್ಲಿ! ಸರಿ, ಯಾವುದಾದರೊಂದು ಗಾಡಿ ಬರುವುದನ್ನೇ ನಿರೀಕ್ಷಿಸುತ್ತ ದಾರಿಯ ಪಕ್ಕದಲ್ಲಿ ಕಾದು ಕುಳಿತರು. ಮಟಮಟ ಮಧ್ಯಾಹ್ನ. ಆದರೆ ಕಾಯುವುದನ್ನು ಬಿಟ್ಟು ಬೇರೆ ದಾರಿಯಿರಲಿಲ್ಲ. ಅಂತೂ ಸುಮಾರು ಮೂರು ಗಂಟೆ ಕಾಲ ಕಾದಮೇಲೆ ಒಂದು ಎತ್ತಿನ ಗಾಡಿ ಬಂದಿತು. ಸಾಮಾನುಗಳನ್ನೆಲ್ಲ ಅದರಲ್ಲಿ ತುಂಬಿಸಿದರು. ಆದರೆ ಆ ಭಯಂಕರ ರಸ್ತೆಯ ಮೇಲೆ ಎತ್ತಿನ ಗಾಡಿಯಲ್ಲಿ ಪ್ರಯಾಣ ಮಾಡುವುದಕ್ಕಿಂತ ನಡೆಯುವುದೇ ಮೇಲೆಂದು ಎಲ್ಲರಿಗೂ ಅನ್ನಿಸಿತು. ಶ್ರೀಮತಿ ಸೇವಿಯರ್ ಮಾತ್ರ ಈಗಾಗಲೇ ಹಣ್ಣಾಗಿಬಿಟ್ಟಿದ್ದರಿಂದ ಅವರೊಬ್ಬರು ಗಾಡಿಯಲ್ಲಿ ಕುಳಿತರು. ಸ್ವಾಮೀಜಿ, ನಿರಂಜನಾನಂದರು, ಕ್ಯಾಪ್ಟನ್ ಸೇವಿಯರ್ ಮತ್ತು ಗುಡ್​ವಿನ್ ಪಾದಯಾತ್ರೆ ಹೊರಟರು. ಅಂತೂ ಸ್ವಾಮೀಜಿ ಯವರಿಗೆ ಮತ್ತೊಮ್ಮೆ ತಮ್ಮ ಪರಿವ್ರಾಜಕ ದಿನಗಳ ‘ಮಧುರ’ ಸ್ಮೃತಿಗಳು ಮರುಕಳಿಸಿದುವು. ಗುಡ್​ವಿನ್ನನಿಗೂ ಸೇವಿಯರ್ ದಂಪತಿಗಳಿಗೂ ಭಾರತದ ರುಚಿ ಸಿಗುವಂತಾಯಿತು. ಹೀಗೆ ನಾಲ್ಕೈದು ತಾಸು ‘ಪ್ರವಾಸ’ ಮಾಡಿ ಕೇಕಿರವ ಎಂಬಲ್ಲಿಗೆ ಬಂದರು. ರಾತ್ರಿ ಹತ್ತೂವರೆಗೆ ಅಲ್ಲಿ ಊಟ ಮಾಡಿದರು. ಬಹಳ ಕಷ್ಟಪಟ್ಟು ಇನ್ನೊಂದು ಎತ್ತಿನ ಗಾಡಿಯನ್ನು ಹೊಂದಿಸಿಕೊಂಡು, ಮತ್ತೆ ಮುಂದುವರಿದರು. ಈ ಪಾಶ್ಚಾತ್ಯ ಶಿಷ್ಯರಂತೂ ತಾವು ಹಿಂದೆಂದೂ ಅನುಭವಿಸಿರದಿದ್ದ ಕಷ್ಟಗಳನ್ನೆಲ್ಲ ಪಡಬೇಕಾಯಿತು. ಸೇವಿಯರ್ ದಂಪತಿಗಳು ಗಾಡಿಯಲ್ಲೇ ತೂಕಡಿಸಲು ಪ್ರಯತ್ನಿ ಸಿದರು. ಹೀಗೇ ಒಟ್ಟು ಇಪ್ಪತ್ತೆಂಟು ಮೈಲಿ ಪ್ರಯಾಣ ಸಾಗಿತು. ಈ ಧಡಕಾ-ಬಢಕ್ ಗಾಡಿಯ ಗೊಡವೆಯೇ ಬೇಡ ಎಂದು ಗುಡ್​ವಿನ್ ಸುಮಾರು ಹದಿನೆಂಟು ಮೈಲಿ ನಡೆದ! ಸ್ವಾಮಿಗಳಿ ಬ್ಬರಿಗೂ, ವಿಶ್ರಾಂತಿಯಿಲ್ಲವೆಂಬುದನ್ನು ಬಿಟ್ಟರೆ ಹೆಚ್ಚು ತೊಂದರೆಯಾಗಲಿಲ್ಲ. ಯುವಕ ಗುಡ್ ವಿನ್ನನೂ ಹೇಗೋ ಸಹಿಸಿಕೊಂಡ. ಆದರೆ ಸೇವಿಯರ್ ದಂಪತಿಗಳು ಮಾತ್ರ ಸಂಪೂರ್ಣ ಜರ್ಜರಿತರಾಗಿಬಿಟ್ಟರು. ಅಂತೂ ಬೆಳಗ್ಗೆ ಹತ್ತು ಗಂಟೆಗೆ ಬೌದ್ಧರ ಪವಿತ್ರ ನಗರವಾದ ಅನೂರಾಧಪುರಕ್ಕೆ ಕ್ಷೇಮವಾಗಿ ಬಂದು ಸೇರಿದರು. 

ಈ ನಗರದಲ್ಲಿ ಎರಡು ಸಾವಿರ ವರ್ಷಗಳ ಹಿಂದಿನ ಅರಳಿ ಮರವಿದೆ. ಈ ಮರವು ಬುದ್ಧಗಯೆಯ ಬೋಧಿವೃಕ್ಷದ ಕೊಂಬೆಯನ್ನು ನೆಟ್ಟು ಬೆಳೆಸಿದಂಥದು. ಸ್ವಾಮೀಜಿ ಹಾಗೂ ಅವರ ಸಂಗಡಿಗರು ಈ ಪವಿತ್ರ ವೃಕ್ಷವನ್ನು ಸಂದರ್ಶಿಸಿ ತಮ್ಮ ಪ್ರಣಾಮವನ್ನರ್ಪಿಸಿದರು. ಅಂದು ಮಧ್ಯಾಹ್ನ ಇದೇ ವೃಕ್ಷದ ಕೆಳಗೆ ಅನುರಾಧಪುರದ ನಾಗರಿಕರ ಪರವಾಗಿ ಸ್ವಾಮೀಜಿಯವರಿಗೆ ಒಂದು ಬಿನ್ನವತ್ತಳೆಯನ್ನು ಅರ್ಪಿಸಲಾಯಿತು. ಅಲ್ಲಿ ನೆರೆದಿದ್ದ ಸುಮಾರು ಎರಡು-ಮೂರು ಸಾವಿರ ಜನರನ್ನುದ್ದೇಶಿಸಿ ಸ್ವಾಮೀಜಿ ಇಂಗ್ಲಿಷಿನಲ್ಲಿ ಒಂದು ಪುಟ್ಟ ಭಾಷಣ ಮಾಡಿದರು. ವಿಷಯ: ‘ಭಗವಂತನ ಪೂಜೆ’. ಅವರ ಭಾಷಣವನ್ನು ದುಭಾಷಿಗಳು ತಮಿಳು ಮತ್ತು ಸಿಂಹಳೀ ಭಾಷೆಗಳಿಗೆ ಅನುವಾದಿಸುತ್ತಿದ್ದರು. ಅಂದಿನ ಭಾಷಣದಲ್ಲಿ ಸ್ವಾಮೀಜಿ, ಡಂಭಾಚಾರದ ಬಾಹ್ಯ ಪೂಜೆಗೆ ಬದಲಾಗಿ, ವೇದಗಳ ಅನುಷ್ಠಾನಾತ್ಮಕವಾದ ಬೋಧನೆಗಳನ್ನು ಜೀವನದಲ್ಲಿ ಅಳವಡಿಸಿ ಕೊಳ್ಳುವತ್ತ ಹೆಚ್ಚು ಗಮನ ಕೊಡಬೇಕು ಎಂದು ಜನರಿಗೆ ಕರೆ ನೀಡಿದರು. ಹೀಗೆ ಅವರು ಸ್ವಲ್ಪ ಹೊತ್ತು ಮಾತನಾಡುವಷ್ಟರಲ್ಲಿ ತೊಂದರೆ ಶುರುವಾಯಿತು. ಎಲ್ಲಿಂದಲೋ ಬೌದ್ಧರ ದೊಡ್ಡ ಗುಂಪೊಂದು ಅಲ್ಲಿಗೆ ಬಂದಿತು. ಬೌದ್ಧ ಭಿಕ್ಷುಗಳು, ಗೃಹಸ್ಥರು, ಹೆಂಗಸರು, ಮಕ್ಕಳು–ಎಲ್ಲರೂ ಸ್ವಾಮೀಜಿಯವರ ಸುತ್ತ ನಿಂತುಕೊಂಡು ಗಟ್ಟಿಯಾಗಿ ಡೋಲು, ಜಾಗಟೆ, ಶಂಖಗಳನ್ನು ಬಾರಿಸುತ್ತ ಅಸಾಧ್ಯ ಗಲಾಟೆ ಎಬ್ಬಿಸಿಬಿಟ್ಟರು. ಸ್ವಾಮೀಜಿ ತಮ್ಮ ಮಾತನ್ನು ಅರ್ಧಕ್ಕೇ ನಿಲ್ಲಿಸ ಬೇಕಾಯಿತು. ಇದರಿಂದಾಗಿ, ಭಾಷಣವನ್ನು ಆಸಕ್ತಿಯಿಂದ ಕೇಳುತ್ತಿದ್ದ ಸಾವಿರಾರು ಸಭಿಕರಿಗೆ ಆ ಬೌದ್ಧರ ಮೇಲೆ ರೇಗಿಹೋಯಿತು. ಮತ್ತಿನ್ನೇನು? ಹಿಂದೂ-ಬೌದ್ಧರ ನಡುವೆ ಒಂದು ಭಾರೀ ಮಾರಾಮಾರಿಯೇ ಶುರುವಾಗಬೇಕು; ಅಷ್ಟರಲ್ಲಿ ಸ್ವಾಮೀಜಿ, ಉದ್ರೇಕಗೊಳ್ಳದೆ ಶಾಂತರಾಗಿರ ಬೇಕೆಂದು ಸಭಿಕರನ್ನು ಕಳಕಳಿಯಿಂದ ಪ್ರಾರ್ಥಿಸಿಕೊಂಡದ್ದರಿಂದ ಅದು ಅಲ್ಲಿಗೇ ನಿಂತಿತು. 

ಆದರೆ ಬೌದ್ಧರ ಇಂತಹ ವರ್ತನೆಗೆ ಕಾರಣವೇನಿದ್ದಿರಬಹುದು? ವಿವೇಕಾನಂದರಂತಹ ಪ್ರಭಾವಶಾಲೀ ಹಿಂದುವೊಬ್ಬನು ತಮ್ಮ ಭಕ್ತರ ಶ್ರದ್ಧೆಯನ್ನು ಕದಲಿಸಿಬಿಡಬಹುದೆಂಬ ಭಯ ಆ ಬೌದ್ಧಮತೀಯರಲ್ಲಿ ಉಂಟಾಗಿರಬಹುದೆಂಬುದು ಒಂದಂಶ. ಆದರೆ ಅದಕ್ಕಿಂತ ಮುಖ್ಯ ವಾದುದೆಂದರೆ ಜನಾಂಗದ್ವೇಷ. ಸಿಲೋನಿನಲ್ಲಿ ಹಿಂದೂಗಳೆಂದರೆ ಮುಖ್ಯವಾಗಿ ತಮಿಳರು; ಬೌದ್ಧರೆಂದರೆ ಮೂಲತಃ ಸಿಲೋನಿನವರು. ಪರಕೀಯರಾದ ತಮಿಳರು ತಮ್ಮ ಸ್ಥಳಕ್ಕೆ ಬಂದು, ತಮ್ಮ ಮೇಲೆಯೇ ಮೇಲುಗೈ ಸಾಧಿಸುತ್ತಾರೆಂಬುದು ಅಲ್ಲಿನವರ ಸಿಟ್ಟು. (ಇಂದಿಗೂ ಶ್ರೀಲಂಕಾ ದಲ್ಲಿ ನಡೆಯುತ್ತಿರುವ ಭೀಕರ ಅಂತರ್ಯುದ್ಧಕ್ಕೆ ಇದೇ ಮೂಲ.) ಆದ್ದರಿಂದ, ಬೌದ್ಧರ ಸಂಖ್ಯೆ ಗಣನೀಯವಾಗಿರುವ ಈ ಸ್ಥಳಗಳಲ್ಲಿ ಸ್ವಾಮೀಜಿ, ಸರ್ವಧರ್ಮ ಸಮನ್ವಯದ ಭಾವವನ್ನು ಮನಮುಟ್ಟುವಂತೆ ಬೋಧಿಸಿದರು. ಯಾರನ್ನು ಶಿವನನ್ನಾಗಿ, ವಿಷ್ಣುವನ್ನಾಗಿ ಇಲ್ಲವೆ ಬುದ್ಧನನ್ನಾಗಿ ಪೂಜಿಸಲಾಗುತ್ತಿದೆಯೋ ಆ ದೇವರು ಒಬ್ಬನೇ ಎಂದು ಸಾರಿದರು. ಆದ್ದರಿಂದ ಬೇರೆಬೇರೆ ಧರ್ಮಗಳ ಅನುಯಾಯಿಗಳಲ್ಲಿ ಪರಸ್ಪರ ಸಹಿಷ್ಣುತೆ ಮಾತ್ರವಲ್ಲ, ಸಹಾನುಭೂತಿ ಉಂಟಾಗ ಬೇಕು ಎಂದು ಒತ್ತಿ ಹೇಳಿದರು. 

ಸ್ವಾಮೀಜಿಯವರು ಅನೂರಾಧಪುರದಿಂದ ೧೨ಂ ಮೈಲಿ ದೂರದ ಜಾಫ್ನಾದತ್ತ ತಮ್ಮ ಪ್ರಯಾಣವನ್ನು ಮುಂದುವರಿಸಿದರು. ಆ ರಸ್ತೆಯೋ ಅತ್ಯಂತ ಕೆಟ್ಟ ರಸ್ತೆ; ಕುದುರೆಗಳೋ ಆ ರಸ್ತೆಗೆ ತಕ್ಕಂಥವು. ಆದರೆ ಜಾಫ್ನಾದ ಜನರ ವಿಶ್ವಾಸದ ಒತ್ತಾಯಕ್ಕೆ ಮಣಿದು ಸ್ವಾಮೀಜಿ ಬರಲು ಒಪ್ಪಿಕೊಂಡಿದ್ದರು. ಪ್ರಯಾಣದ ದಾರಿ ಪಚ್ಚೆ ಹಸುರಿನ ವನಸಿರಿಯಿಂದ ಕೂಡಿದ್ದು ಕಣ್ಣಿಗೆ ತಂಪೆರೆಯುವಂತಿದ್ದುದರಿಂದ, ಪ್ರಯಾಣದ ಕಷ್ಟವನ್ನು ಸ್ವಲ್ಪ ಮಟ್ಟಿಗೆ ಮರೆಯಬಹುದಾಗಿತ್ತು. ಆದರೆ ಎರಡು ರಾತ್ರಿ ನಿದ್ರೆಗೆಡಬೇಕಾಯಿತು. ದಾರಿಯಲ್ಲಿನ ವವೂನಿಯ ಎಂಬ ಪಟ್ಟಣದ ಜನ ಸ್ವಾಮೀಜಿಯವರನ್ನು ಎದುರ್ಗೊಂಡು ಅವರಿಗೊಂದು ಅಭಿನಂದನಾ ಪತ್ರವನ್ನು ಸಲ್ಲಿಸಿದರು. ಅದಕ್ಕೆ ಅವರು ಚುಟುಕಾಗಿ ಉತ್ತರಿಸಿದರು. ಹೀಗೆ ದಾರಿಯಲ್ಲಿ ಸ್ವಲ್ಪ ವಿರಾಮ ಸಿಗುವಂತಾ ಯಿತು. ಇಲ್ಲಿಂದ ಹೊರಟು ಅವರು ಸಿಲೋನಿನ ಅರಣ್ಯಗಳ ಮೂಲಕ ಹಾದು ಎಲಿಫೆಂಟ್​ಪಾಸ್ ಎಂಬ ಸ್ಥಳವನ್ನು ತಲುಪಿದರು. ಇಲ್ಲಿ ಸಿಲೋನನ್ನೂ ಜಾಫ್ನಾ ದ್ವೀಪವನ್ನೂ ಸೇರಿಸುವ ದೊಡ್ಡ ಸೇತುವೆಯಿದೆ. ಜಾಫ್ನಾ ನಗರಕ್ಕೆ ಇಲ್ಲಿಂದ ಹನ್ನೆರಡು ಮೈಲಿ. ಸ್ವಾಮೀಜಿಯವರನ್ನು ಬರಮಾಡಿ ಕೊಳ್ಳಲು ಅಲ್ಲಿನ ಗಣ್ಯ ನಾಗರಿಕರು ಎಲಿಫೆಂಟ್​ಪಾಸ್​ನವರೆಗೂ ಬಂದಿದ್ದರು. ಇಲ್ಲಿಂದಲೇ ಜಾಫ್ನಾ ನಗರದವರೆಗೂ ಸುಮಾರು ಇಪ್ಪತ್ತು ಸಾರೋಟುಗಳ ಮೆರವಣಿಗೆ ಪ್ರಾರಂಭವಾಯಿತು. ಮೆರವಣಿಗೆ ನಗರವನ್ನು ತಲುಪಿದಾಗ ಸಾವಿರಾರು ಜನ ಅವರನ್ನು ಹರ್ಷೋದ್ಗಾರಗಳೊಂದಿಗೆ ಸ್ವಾಗತಿಸಿದರು. 

ಅಂದು ಸಂಜೆ ನಗರದ ಹಿಂದೂ ಕಾಲೇಜಿನಲ್ಲಿ ಭಾರೀ ಸಮಾರಂಭವೊಂದನ್ನು ಏರ್ಪಡಿಸ ಲಾಗಿತ್ತು. ನಗರದ ಸೌಂದರ್ಯವನ್ನಂತೂ ಕಂಡೇ ಆಸ್ವಾದಿಸಬೇಕು. ವಿಶ್ವವಿಜೇತನಾದ ವೀರ ಸಂನ್ಯಾಸಿಯೋರ್ವನನ್ನು ಬರಮಾಡಿಕೊಳ್ಳಲು ನಗರಕ್ಕೆ ನಗರವೇ ಸನ್ನದ್ಧವಾಗಿತ್ತು. ಅಲ್ಲದೆ ಸಿಲೋನಿನ ವಿವಿಧ ಭಾಗಗಳಿಂದ ಜನ ಇಲ್ಲಿಗೆ ಮುತ್ತಿದ್ದರು. ಪ್ರತಿಯೊಂದು ರಸ್ತೆಯನ್ನೂ ಬಾಳೆಕಂಬ-ತಳಿರುತೋರಣಗಳಿಂದ, ಬಾವುಟಗಳಿಂದ ಹಾಗೂ ಬಣ್ಣದ ಬಟ್ಟೆಗಳನ್ನು ಸುತ್ತಿದ ಕಂಬಗಳಿಂದ ಸಿಂಗರಿಸಲಾಗಿತ್ತು. ಅಲ್ಲದೆ ಮನೆ ಮನೆಗಳ ಮುಂದೆಯೂ ಪೂರ್ಣಕುಂಭ ಗಳನ್ನಿರಿಸಲಾಗಿತ್ತು ಮತ್ತು ನಂದಾದೀಪಗಳನ್ನು ಬೆಳಗಿಸಲಾಗಿತ್ತು. ಸ್ವಾಮೀಜಿ ಇಳಿದುಕೊಂಡ ಲ್ಲಿಂದ ಹಿಂದೂ ಕಾಲೇಜಿಗೆ ಎರಡು ಮೈಲಿ ದೂರ. ಅವರನ್ನು ಅತ್ಯಂತ ವಿಜೃಂಭಣೆಯ ಮೆರವಣಿಗೆಯಲ್ಲಿ ಕರೆದೊಯ್ಯುವ ಏರ್ಪಾಡಾಗಿತ್ತು. ಈ ದಾರಿಯಲ್ಲಿ ಉದ್ದಕ್ಕೂ ಜನ ತುಂಬಿ ಹೋಗಿದ್ದರು. ಅದೊಂದು ಮಾನವ ಶಿರಗಳ ಸಮುದ್ರದಂತಿತ್ತು. ಕನಿಷ್ಠ ಪಕ್ಷ ಹದಿನೈದು ಸಾವಿರ ಜನರಾದರೂ ಇದ್ದರೆಂಬುದು ಒಂದು ಅಂದಾಜು. ಅಷ್ಟು ಜನ ನೆರೆದಿದ್ದರೂ ಅಲ್ಲಿ ತುಂಬ ಶಿಸ್ತು-ಶಾಂತಿ ನೆಲಸಿತ್ತು. ರಾತ್ರಿ ಎಂಟೂವರೆ ಗಂಟೆಗೆ ಝಗಝಗಿಸುವ ಪಂಜಿನ ಮೆರವಣಿಗೆ ಪ್ರಾರಂಭವಾಯಿತು. ಜೊತೆಗೆ ವಾಲಗ-ಡೋಲು-ಕಹಳೆಗಳ ಮೊಳಗುವಿಕೆ ಊರಲ್ಲೆಲ್ಲ ಪ್ರತಿಧ್ವನಿ ಸುತ್ತಿತ್ತು. ಜನರ ಸಂಭ್ರಮಕ್ಕಂತೂ ಮೇರೆಯೇ ಇರಲಿಲ್ಲ! ಅದೊಂದು ಕಿನ್ನರ ಲೋಕವಾಗಿ ಬಿಟ್ಟಿತ್ತು. ಜಾಫ್ನಾ ನಗರ ಇಂಥದೊಂದು ದೃಶ್ಯವನ್ನು ಹಿಂದೆಂದೂ ಕಂಡಿರಲಿಲ್ಲ. ಹೀಗೆ ಅಲ್ಲಿನ ಜನ ಸ್ವಾಮೀಜಿಯವರಿಗೆ ‘ನ ಭೂತೋ ನ ಭವಿಷ್ಯತಿ’ ಎಂಬಂತಹ ಸತ್ಕಾರವನ್ನು ನೀಡಿದರು. 

ಮೆರವಣಿಗೆಯಲ್ಲಿ ಬಂದ ಸ್ವಾಮೀಜಿ, ದಾರಿಯಲ್ಲಿ ಶಿವನ ಹಾಗೂ ಕಾರ್ತಿಕೇಯನ ದೇವಸ್ಥಾನ ಗಳಿಗೆ ಹೋಗಿ ಪ್ರಣಾಮ ಸಲ್ಲಿಸಿದರು. ಆಗ ಅಲ್ಲಿನ ಅರ್ಚಕರು ಅವರನ್ನು ಸಕಲ ಗೌರವ ಗಳೊಂದಿಗೆ ಎದುರ್ಗೊಂಡು ಹೂಮಾಲೆಗಳನ್ನರ್ಪಿಸಿದರು. ಅಲ್ಲದೆ ದಾರಿಯಲ್ಲಿ ಇನ್ನೂ ಎಷ್ಟೋ ಜನ ಅವರಿಗೆ ಹಾರಗಳನ್ನು ಹಾಕಿ ಗೌರವಿಸಿದರು. ಈ ಎಲ್ಲ ಗೌರವವನ್ನೂ ಸ್ವಾಮೀಜಿ ಅತ್ಯಂತ ವಿನೀತ ಭಾವದಿಂದ ಸ್ವೀಕರಿಸಿದರು. ಈ ಸುಂದರ ಪುಷ್ಪಹಾರಗಳಿಂದ ಅವರ ತೇಜಸ್ಸು- ಸೌಂದರ್ಯ ಇಮ್ಮಡಿಯಾಯಿತು! ಮೆರವಣಿಗೆಯು ಸಮಾರಂಭದ ಚಪ್ಪರವನ್ನು ತಲುಪಿದಾಗ ರಾತ್ರಿ ಹತ್ತು ಗಂಟೆ. ಅಲ್ಲಿಗೆ ಆಗಮಿಸಿದ ಜನರಿಗೆ ಆ ಚಪ್ಪರ ಎಲ್ಲಿ ಸಾಕಾದೀತು? ಆಗಲೇ ಕಿಕ್ಕಿರಿದಿದ್ದ ಚಪ್ಪರದೊಳಗೆ ಹೊರಗಡೆಯಿಂದ ಮತ್ತಷ್ಟು ಜನ ಪ್ರವೇಶಿಸುವ ಪ್ರಯತ್ನ ಮಾಡುತ್ತಿ ದ್ದರು. ಅಲ್ಲಿ ನೆರೆದಿದ್ದವರಲ್ಲಿ ಹಿಂದೂಗಳಲ್ಲದೆ ಬೌದ್ಧರಿದ್ದರು, ಕ್ರೈಸ್ತರಿದ್ದರು, ಮುಸಲ್ಮಾನರೂ ಇದ್ದರು. ತಿರುವಾಂಕೂರಿನ ಮಾಜಿ ನ್ಯಾಯಾಧೀಶರಾದ ಚಲ್ಲಪ್ಪ ಪಿಳ್ಳೆಯವರು ಸ್ವಾಮೀಜಿಯ ವರನ್ನು ವೇದಿಕೆಯ ಮೇಲೆ ಕರೆದೊಯ್ದು ಹಾರ ಹಾಕಿದರು. ಬಳಿಕ ಬಿನ್ನವತ್ತಳೆಯನ್ನು ಓದ ಲಾಯಿತು. ಇದಕ್ಕೆ ಉತ್ತರವಾಗಿ ಸ್ವಾಮೀಜಿ ಅತ್ಯಂತ ನಿರರ್ಗಳವಾಗಿ ಸುಮಾರು ಒಂದು ಗಂಟೆ ಮಾತನಾಡಿದರು. ಅಂದಿನ ಕಾಲದಲ್ಲಿ ಧ್ವನಿವರ್ಧಕ ಯಂತ್ರದ ಸೌಲಭ್ಯವಿರಲಿಲ್ಲ. ಅದರೂ ಅಲ್ಲಿ ನೆರೆದಿದ್ದ ಹದಿನೈದು ಸಹಸ್ರ ಮಂದಿಯ ಸಭೆಯಲ್ಲಿ ಸ್ವಾಮೀಜಿ ಹೇಗೆ ಮಾತನಾಡಿದರೋ, ಆ ಜನ ಹೇಗೆ ಕೇಳಿಸಿಕೊಂಡರೋ! ಅಲ್ಲದೆ ಸ್ವಾಮೀಜಿ ಇನ್ನೂರು ಮೈಲಿಯಷ್ಟು ದೂರ ಅಂತಹ ಪ್ರಯಾಸಕರ ಪ್ರಯಾಣ ಮಾಡಿಬಂದು, ಆ ಜನಸಂದಣಿಯ ಗೌರವಾದರಗಳನ್ನು ಸ್ವೀಕರಿಸಲು ಹೇಗೆ ಸಾಧ್ಯವಾಯಿತೋ!

ಮರುದಿನ ಸಂಜೆ ಏಳು ಗಂಟೆಗೆ ಸ್ವಾಮೀಜಿ ಅಲ್ಲಿನ ಹಿಂದೂ ಕಾಲೇಜಿನಲ್ಲಿ ಒಂದು ಗಂಟೆ ನಲವತ್ತು ನಿಮಿಷಗಳ ಕಾಲ ಮಾತನಾಡಿದರು. ಜಾಫ್ನಾ ನಗರದ ಸಮಸ್ತ ವಿದ್ಯಾವಂತ ಗಣ್ಯ ವ್ಯಕ್ತಿಗಳಿಂದ ಕೂಡಿದ ನಾಲ್ಕು ಸಾವಿರ ಜನರ ಸಭೆಯಲ್ಲಿ ಅವರು ಮಾತನಾಡಿದ ವಿಷಯ ‘ವೇದಾಂತಧರ್ಮ’. ಅವರ ಈ ಭಾಷಣವು ಸಭಿಕರಲ್ಲಿ ವಿದ್ಯುತ್ಸಂಚಾರವನ್ನೇ ಮಾಡಿತು. ಬಳಿಕ ಕ್ಯಾಪ್ಟನ್ ಸೇವಿಯರ್​ರನ್ನು ಮಾತನಾಡುವಂತೆ ಕೇಳಿಕೊಳ್ಳಲಾಯಿತು. ಆಗ ಆವರು ತಾವು ಹಿಂದೂ ಧರ್ಮವನ್ನು ಏಕೆ ಸ್ವೀಕರಿಸಿದೆವು ಮತ್ತು ಸ್ವಾಮೀಜಿಯವರೊಂದಿಗೆ ಭಾರತಕ್ಕೆ ಬಂದದ್ದರ ಉದ್ದೇಶವೇನು ಎಂಬುದರ ಕುರಿತಾಗಿ ತಿಳಿಸಿದರು.

ಈ ಭಾಷಣದೊಂದಿಗೆ ಸ್ವಾಮೀಜಿಯವರ ಸಿಲೋನ್ ಸಂದರ್ಶನ ಸಮಾಪ್ತವಾಯಿತು. ಅವರು ಸಿಲೋನಿನಲ್ಲಿ ಭೇಟಿ ನೀಡಿದ ಪ್ರತಿಯೊಂದು ಸ್ಥಳದಲ್ಲೂ ಜನ ಅವರ ಬೋಧನೆಗಳಲ್ಲಿ ತುಂಬ ಆಸಕ್ತಿಯನ್ನು ವ್ಯಕ್ತಪಡಿಸಿದರು. ಅಲ್ಲದೆ ಧರ್ಮಪ್ರಸಾರಕ್ಕಾಗಿ ತಮ್ಮಲ್ಲಿಗೆ ಅವರ ಸೋದರ ಸಂನ್ಯಾಸಿಗಳನ್ನು ಕಳಿಸಿಕೊಡುವಂತೆ ಬೇಡಿಕೊಂಡರು. ಸಿಲೋನಿನ ಇನ್ನೂ ಹಲವಾರು ಪಟ್ಟಣಗಳಿಂದ ಅವರಿಗೆ ಕರೆಗಳು ಬಂದಿದ್ದುವು. ಆದರೆ ಅವರಿಗೆ ಇನ್ನು ಸಮಯವಿರಲಿಲ್ಲ. ಅಲ್ಲದೆ ಅವರ ಶರೀರ ಆಗಲೇ ಸಂಪೂರ್ಣ ಬಳಲಿ ಹೋಗಿತ್ತು. ಶ್ರೀಮತಿ ಸೇವಿಯರ್ ಹೇಳಿದಂತೆ, ‘ಸ್ವಾಮೀಜಿಯವರು ಸಿಲೋನಿನಲ್ಲಿ ಇನ್ನೂ ಹೆಚ್ಚುಕಾಲ ಉಳಿದುಕೊಂಡಿದ್ದರೆ ಆ ಜನರ ವಿಶ್ವಾಸದ ಆಧಿಕ್ಯದಿಂದ ಸತ್ತೇಹೋಗಿಬಿಡುತ್ತಿದ್ದರು.’

೧೮೯೭ರ ಜನವರಿ ೨೬; ಅಂದು ಸ್ವಾಮೀಜಿ ಜಾಫ್ನಾದಿಂದ ಹೊರಟು ಐವತ್ತು ಮೈಲಿಗಳ ಸಮುದ್ರ ಪ್ರಯಾಣ ಮಾಡಿ, ತಮ್ಮ ಪ್ರಾಣಪ್ರಿಯ ತಾಯ್ನಾಡಾದ ಪವಿತ್ರ ಭಾರತದ ಪಾಂಬನ್ನಿಗೆ ಬಂದು ಸೇರಿದರು. ರಾಮನಾಡಿನ ಮಹಾರಾಜ ಭಾಸ್ಕರ ಸೇತುಪತಿಯು ಅವರನ್ನು ರಾಮೇಶ್ವರಕ್ಕೆ ಆಹ್ವಾನಿಸಿದ್ದ. ಅಂತೆಯೇ ಸ್ವಾಮೀಜಿ ರಾಮೇಶ್ವರದತ್ತ ಹೊರಟುನಿಂತರು. ಅಷ್ಟರಲ್ಲಿ, ಅವರನ್ನು ಎದುರ್ಗೊಳ್ಳಲು ಸ್ವತಃ ಮಹಾರಾಜನೇ ಅಲ್ಲಿಗೆ ಬರಲಿದ್ದಾನೆಂಬ ವರ್ತಮಾನ ಬಂದಿತು. ಸ್ವಲ್ಪ ಹೊತ್ತಿನಲ್ಲೇ ರಾಜಾ ಭಾಸ್ಕರ ಸೇತುಪತಿ ಪರಿವಾರಸಮೇತನಾಗಿ ತನ್ನ ರಾಜನೌಕೆಯಲ್ಲಿ ಅಲ್ಲಿಗೆ ಆಗಮಿಸಿದ. ರಾಜನೂ ಅವನ ಸಂಗಡಿಗರೂ ಸ್ವಾಮೀಜಿಯವರಿಗೆ ದೀರ್ಘದಂಡಪ್ರಣಾಮ ಮಾಡಿದರು. ಸಂನ್ಯಾಸಿ-ಮಹಾರಾಜರ ನಡುವಿನ ಭೇಟಿಯ ದೃಶ್ಯ ಅತ್ಯಂತ ಹೃದಯಸ್ಪರ್ಶಿ ಯಾಗಿತ್ತು. ಆ ಸಂದರ್ಭದಲ್ಲಿ ಸ್ವಾಮೀಜಿ, ರಾಮನಾಡಿನ ಅರಸ ಹಿಂದೆ ತಮಗೆ ನೀಡಿದ ಸಹಾಯ ವನ್ನು ಕೃತಜ್ಞತೆಯಿಂದ ಸ್ಮರಿಸುತ್ತ, “ನಾನು ಪಾಶ್ಚಾತ್ಯ ರಾಷ್ಟ್ರಗಳಿಗೆ ಹೋಗಬೇಕೆಂಬ ಆಲೋಚನೆ ಯನ್ನು ಮೊತ್ತಮೊದಲಿಗೆ ನನ್ನಲ್ಲಿ ಉಂಟುಮಾಡಿದ ಕೆಲವರಲ್ಲಿ ಮಹಾರಾಜರೂ ಒಬ್ಬರು; ಅಲ್ಲದೆ, ಅಲ್ಲಿಗೆ ಹೋಗಲು ಪ್ರೋತ್ಸಾಹವನ್ನೂ ಧನಸಹಾಯವನ್ನು ನೀಡಿದವರು ಇವರು. ಆದ್ದ ರಿಂದ ನಾನು ನನ್ನ ಮಾತೃಭೂಮಿಗೆ ಕಾಲಿಟ್ಟಾಗ ಮೊತ್ತಮೊದಲನೆಯದಾಗಿ ಇವರನ್ನು ಭೇಟಿ ಯಾಗುತ್ತಿರುವುದು ಅತ್ಯಂತ ಸೂಕ್ತವಾಗಿದೆ” ಎಂದರು.

ದೋಣಿಯು ತೀರವನ್ನು ಮುಟ್ಟುತ್ತಿದ್ದಂತೆ ರಾಮನಾಡಿನ ಜನ ಕಡಲ ಅಬ್ಬರವನ್ನು ಮೀರಿಸು ವಂತಹ ಹರ್ಷೋದ್ಗಾರದೊಡನೆ ಸ್ವಾಮೀಜಿಯವರನ್ನು ಬರಮಾಡಿಕೊಂಡರು. ವರ್ಣಮಯ ವಾಗಿ ಅಲಂಕರಿಸಲಾಗಿದ್ದ ಚಪ್ಪರದಲ್ಲಿ ಸ್ವಾಗತ ಸಮಾರಂಭ; ಅಲ್ಲಿ ಅವರಿಗೆ ರಾಜ್ಯದ ಪರವಾಗಿ ಒಂದು ಅಭಿನಂದನಾ ಪತ್ರವನ್ನು ಸಮರ್ಪಿಸಲಾಯಿತು. ಇದರೊಂದಿಗೆ ಸ್ವತಃ ರಾಜಾ ಭಾಸ್ಕರ ಸೇತುಪತಿ ಕೆಲವು ಮಾತುಗಳನ್ನಾಡಿ ಅವರನ್ನು ಹೃತ್ಪೂರ್ವಕವಾಗಿ ಕೊಂಡಾಡಿದ. ಬಳಿಕ ಇವುಗಳಿಗೆ ಉತ್ತರರೂಪವಾಗಿ ಸ್ವಾಮೀಜಿಯವರು ಒಂದು ಪುಟ್ಟ ಭಾಷಣ ಮಾಡುತ್ತ ಹೇಳಿದರು: “ಭಾರತದ ಜೀವಾಳವಿರುವುದು ರಾಜಕಾರಣದಲ್ಲಲ್ಲ, ಸೈನ್ಯಶಕ್ತಿಯಲ್ಲಲ್ಲ, ವ್ಯಾಪಾರ-ವಾಣಿಜ್ಯ ದಲ್ಲೂ ಅಲ್ಲ, ಅಥವಾ ಯಂತ್ರಕೌಶಲದಲ್ಲೂ ಅಲ್ಲ. ಅದು ಇರುವುದು ಧರ್ಮದಲ್ಲಿ, ಕೇವಲ ಧರ್ಮದಲ್ಲಿ, ಈ ಧರ್ಮವನ್ನು ಭಾರತ ಮಾತ್ರವೇ ಜಗತ್ತಿಗೆ ನೀಡಲು ಸಾಧ್ಯ.”

ಸಭೆ ಮುಕ್ತಾಯಗೊಂಡಮೇಲೆ ಸ್ವಾಮೀಜಿ ಹಾಗೂ ಅವರ ಸಂಗಡಿಗರನ್ನು ಸಾರೋಟಿನಲ್ಲಿ ಮಹಾರಾಜನ ಅತಿಥಿಗೃಹಕ್ಕೆ ಕರೆದೊಯ್ಯಲಾಯಿತು. ಸ್ವತಃ ಮಹಾರಾಜ ಹಾಗೂ ಆಸ್ಥಾನಿಕರು ವಾಹನದೊಂದಿಗೆ ನಡೆದುಬಂದರು. ಇಷ್ಟಾದರೂ ರಾಜನಿಗೆ ತೃಪ್ತಿಯಿಲ್ಲ. ಸಾರೋಟಿನಿಂದ ಕುದುರೆಗಳನ್ನು ಬಿಡಿಸಿ ಸ್ವತಃ ತಾನೇ ಇತರ ಗಣ್ಯವ್ಯಕ್ತಿಗಳೊಂದಿಗೆ ಸಾರೋಟನ್ನು ಎಳೆದ! ಬಹುಶಃ ಧರ್ಮದ ಇತಿಹಾಸದಲ್ಲೇ ಇದೊಂದು ಅಭೂತಪೂರ್ವ ಘಟನೆ. ಸ್ವಾಮೀಜಿ ಪಾಂಬನ್ನಿನಲ್ಲಿ ಮೂರು ದಿನಗಳ ಕಾಲ ಉಳಿದುಕೊಳ್ಳಲು ನಿರ್ಧರಿಸಿದರು. ಅಲ್ಲಿನ ಜನರಿಗೆ ಇದರಿಂದಾದ ಸಂತಸ ಅಷ್ಟಿಷ್ಟಲ್ಲ. ಮರುದಿನ ಸ್ವಾಮೀಜಿ ರಾಮೇಶ್ವರ ದೇವಾಲಯವನ್ನು ದರ್ಶಿಸಲು ಹೊರಟರು. ಇಲ್ಲಿಗೆ ತಾವು ಹಿಂದೆ ಬಂದಿದ್ದಾಗಿನ ದಿನಗಳನ್ನು ನೆನಪಿಸಿಕೊಂಡು ಅವರು ಬಳಿಯಲ್ಲಿದ್ದವ ರೊಡನೆ ಹೇಳಿದರು. “ಐದು ವರ್ಷಗಳ ಹಿಂದೆ, ಭಾರತ ಯಾತ್ರೆಯ ಕಡೆಯ ಹೆಜ್ಜೆಯಾಗಿ ನಾನಿಲ್ಲಿಗೆ ಬಂದಿದ್ದೆ. ಕಾಲ್ನಡಿಗೆಯ ಪ್ರಯಾಣದಿಂದ ಪಾದಗಳಲ್ಲಿ ಬೊಕ್ಕೆಗಳು, ಮೈಯಲ್ಲೆಲ್ಲ ಆಯಾಸ, ಆಗ ಯಾರಿಗೂ ನನ್ನ ಪರಿಚಯವಿರಲಿಲ್ಲ; ನಾನೊಬ್ಬ ಅನಾಮಧೇಯ ಪರಿವ್ರಾಜಕ ನಷ್ಟೆ.” ಅಂದಿನ ಆ ಸ್ಥಿತಿಗೂ ಇಂದಿನ ಈ ಸ್ಥಿತಿಗೂ ಹೋಲಿಸಿ ನೋಡಿದಾಗ ಸ್ವಾಮೀಜಿಯವರ ಹೃದಯ ತುಂಬಿಬಂದಿತು.

ದೇವಸ್ಥಾನದಲ್ಲಿ ಸ್ವಾಮೀಜಿಯವರ ಸ್ವಾಗತಕ್ಕಾಗಿ ವಿಜೃಂಭಣೆಯ ವ್ಯವಸ್ಥೆಯಾಗಿತ್ತು. ಅವ ರನ್ನು ಹೊತ್ತ ವಾಹನ ಸಮೀಪಿಸುತ್ತಲೇ ಆನೆ ಒಂಟೆ ಕುದುರೆಗಳಿಂದ ಕೂಡಿದ ಭಾರೀ ಮೆರವಣಿಗೆ ಯೊಂದು ದೇವಸ್ಥಾನದ ಕಡೆಯಿಂದ ಬಂದು ಅವರನ್ನು ಸ್ವಾಗತಿಸಿತು. ದೇವಸ್ಥಾನದ ಲಾಂಛನ ಗಳಾದ ಛತ್ರಿ ಚಾಮರಗಳೂ ಕಹಳೆ ವಾದ್ಯಗಳೂ ಇದ್ದುವು. ಮಹಾತ್ಮನೊಬ್ಬನಿಗೆ ಮಾತ್ರ ದೊರೆಯಬಲ್ಲ ಸ್ವಾಗತ ಅದು. ಬಳಿಕ ಸ್ವಾಮೀಜಿ ದೇವಾಲಯವನ್ನು ಪ್ರವೇಶಿಸಿ ರಾಮೇಶ್ವರನ ದರ್ಶನ ಮಾಡಿದರು. ಅವರಿಗೂ ಅವರ ಸಹವರ್ತಿಗಳಿಗೂ ದೇವಸ್ಥಾನದ ರತ್ನಾಭರಣಗಳನ್ನು ತೋರಿಸಲಾಯಿತು. ಅನಂತರ ಅವರನ್ನು ಬೃಹತ್ತಾದ ಆ ದೇವಸ್ಥಾನದ ಸುತ್ತಲೂ ಕರೆದೊಯ್ದು ಅದರ ವಿವಿಧ ಭಾಗಗಳನ್ನೂ ವಾಸ್ತುಶಿಲ್ಪವನ್ನೂ ತೋರಿಸಲಾಯಿತು. ಅದರಲ್ಲೂ, ಸುಂದರವಾದ ಕೆತ್ತನೆಗಳುಳ್ಳ ಸಾವಿರ ಕಂಬಗಳು ಆ ದೇವಸ್ಥಾನದ ಒಂದು ವಿಶೇಷ ಆಕರ್ಷಣೆ. ಜಗತ್ತಿನಲ್ಲೇ ಅತಿ ದೊಡ್ಡದಾದ ಪ್ರಾಕಾರ ಅಲ್ಲಿಯದು. ಅಲ್ಲಿ ನೆರೆದಿದ್ದ ಜನರನ್ನುದ್ದೇಶಿಸಿ ಭಾಷಣ ಮಾಡು ವಂತೆ ಅವರನ್ನು ಕೇಳಿಕೊಳ್ಳಲಾಯಿತು.

ರಾಮೇಶ್ವರದ ಆ ಪ್ರಸಿದ್ಧ ದೇವಾಲಯದ ಪವಿತ್ರ ಸ್ಥಳದಲ್ಲಿ ನಿಂತು ಮಾತನಾಡುತ್ತ ಸ್ವಾಮೀಜಿ, ತೀರ್ಥಕ್ಷೇತ್ರಗಳ ಮಹತ್ವವನ್ನು ವಿವರಿಸಿದರು; ನಿಜವಾದ ಪೂಜೆಯ ಸ್ವರೂಪವನ್ನು ವಿಶ್ಲೇಷಿಸಿ ತಿಳಿಸಿದರು. ಬಳಿಕ ಸ್ಫೂರ್ತಿಭರಿತರಾಗಿ ನುಡಿದರು–ಶಿವನನ್ನು ಕೇವಲ ಮೂರ್ತಿಗಳಲ್ಲಿ ಮಾತ್ರ ಪೂಜಿಸಿದರೆ ಸಾಲದು; ಜೀವರಲ್ಲಿರುವ ಶಿವನನ್ನು, ಬಡವರಲ್ಲಿರುವ ಶಿವನನ್ನು, ದೀನ ದಲಿತರಲ್ಲಿರುವ ಶಿವನನ್ನು ಪೂಜಿಸಬೇಕು ಎಂದು. ಅವರ ಮಾತುಗಳನ್ನು ತಮಿಳಿಗೆ ಭಾಷಾಂತರಿಸಿ ಹೇಳಲಾಯಿತು. ಅವರ ಭಾಷಣದಿಂದ ರಾಮನಾಡಿನ ರಾಜ ಎಷ್ಟು ಪ್ರಭಾವಿತನಾದನೆಂದರೆ, ಮರುದಿನವೇ ಅವನು ಸಾವಿರಾರು ಮಂದಿ ಬಡವರಿಗೆ ಅನ್ನದಾನ-ವಸ್ತ್ರದಾನಗಳನ್ನು ಮಾಡಿದ. ‘ಒಬ್ಬ ರಾಜನ ಅಥವಾ ಶ್ರೀಮಂತನ ತಲೆ ಸರಿಮಾಡಿದರೆ ಅವನಿಂದ ಸಾವಿರಾರು ಜನ ಬದುಕಿ ಕೊಳ್ಳುತ್ತಾರೆ’ಎಂಬ ಸ್ವಾಮೀಜಿಯವರ ಮಾತಿಗೆ ಇದೊಂದು ಉದಾಹರಣೆ.

ಮಹಾರಾಜನು ಸ್ವಾಮೀಜಿಯವರನ್ನು ಪಾಂಬನ್ನಿನಿಂದ ತನ್ನ ರಾಜಧಾನಿಯಾದ ರಾಮನಾಡಿಗೆ ಕರೆತಂದ. ರಾಮನಾಡನ್ನು ತಲುಪಬೇಕಾದರೆ ದೊಡ್ಡ ಸರೋವರವೊಂದನ್ನು ದಾಟಿ ಬರಬೇಕು. ಸ್ವಾಮೀಜಿ ಹಾಗೂ ಅವರ ಸಂಗಡಿಗರು ರಾಜನೌಕೆಯಲ್ಲಿ ಈ ಸರೋವರವನ್ನು ಕ್ರಮಿಸಿ ತೀರವನ್ನು ಸಮೀಪಿಸುತ್ತಿದ್ದಂತೆ, ಅವರ ಆಗಮನ ಸೂಚಕವಾಗಿ ಫಿರಂಗಿ ಮೊಳಗಿಸಲಾಯಿತು. ಬಳಿಕ, ದೋಣಿ ದಡ ಸೇರುವಾಗಲೂ ಅವರನ್ನು ಮೆರವಣಿಗೆಯಲ್ಲಿ ನಗರಕ್ಕೆ ಕರೆದೊಯ್ಯು ವಾಗಲೂ ಆಕಾಶಬಾಣಗಳನ್ನು ಹಾರಿಸಲಾಯಿತು. ಅಲ್ಲಿ ಎಲ್ಲೆಲ್ಲೂ ಸಂಭ್ರಮ, ಎಲ್ಲೆಲ್ಲೂ ಸಡಗರ, ಎಲ್ಲೆಲ್ಲೂ ಆನಂದ! ಸ್ವಾಮೀಜಿ ರಾಜವಾಹನದಲ್ಲಿ ಕುಳಿತು ಬರುತ್ತಿದ್ದಂತೆ ದಾರಿಯ ಇಕ್ಕೆಲಗಳಲ್ಲಿ ಸಾವಿರಾರು ಜನ ನಿಂತು ಪಂಜುಗಳನ್ನು ಬೆಳಗಿದರು. ಆ ಉತ್ಸಾಹಪೂರ್ಣ ಮೆರವಣಿಗೆಗೆ ಕಳೆಕಟ್ಟುವಂತೆ ಒಂದೆಡೆ ಭಾರತೀಯ ಸಂಗೀತ ಮತ್ತೊಂದೆಡೆ ಪಾಶ್ಚಾತ್ಯ ಸಂಗೀತ ನಡೆಯುತ್ತಿತ್ತು. \eng{“See the Conquering Hero Comes”–}ಇದು ಇಂಗ್ಲಿಷ್ ಸಂಗೀತದ ಪಲ್ಲವಿಯಾಗಿತ್ತು. ಸ್ವತಃ ರಾಜನೇ ನಿಂತು ಮೆರವಣಿಗೆಯ ಮೇಲ್ವಿಚಾರಣೆ ನೋಡಿಕೊಂಡ. ಮೆರವಣಿಗೆಯು ನಗರದ ಮೂಲಕ ಅರ್ಧದಾರಿಗೆ ಬಂದಾಗ ಮಹಾರಾಜನು ಪಲ್ಲಕ್ಕಿಯೇರುವಂತೆ ಸ್ವಾಮೀಜಿಯವರನ್ನು ಕೇಳಿಕೊಂಡ. ಈಗ ಸ್ವಾಮೀಜಿ ಸರ್ವಾಲಂಕೃತವಾದ ಪಲ್ಲಕ್ಕಿಯನ್ನೇರಿ ‘ಶಂಕರ ವಿಲ್ಲಾ’ ಎಂಬ ಗೃಹಕ್ಕೆ ಬಂದು ತಲುಪಿದರು. ಅಲ್ಲಿ ಅವರಿಗೆ ಸ್ವಲ್ಪ ಹೊತ್ತು ವಿಶ್ರಾಂತಿ ಪಡೆಯಲು ಅವಕಾಶ ಕಲ್ಪಿಸಿಕೊಟ್ಟು, ಬಳಿಕ ಅಲ್ಲಿ ಅದಾಗಲೇ ನೆರೆದಿದ್ದ ಜನಸ್ತೋಮವನ್ನು ಉದ್ದೇಶಿಸಿ ಮಾತನಾಡುವಂತೆ ಅವರನ್ನು ವಿನಂತಿಸಿಕೊಳ್ಳಲಾಯಿತು. ಸ್ವಾಮೀಜಿ ವೇದಿಕೆಯನ್ನೇ ರುತ್ತಿದ್ದಂತೆಯೇ ಜನಸ್ತೋಮದಿಂದ ಜಯಘೋಷ ಮೊಳಗಲಾರಂಭವಾಯಿತು. ಜನರ ಉತ್ಸಾಹ ಆನಂದಗಳನ್ನು ಬಣ್ಣಿಸುವುದಕ್ಕೇ ಸಾಧ್ಯವಿಲ್ಲ. ಸ್ವಯಂ ರಾಜನೇ ಎದ್ದುನಿಂತು ಸ್ವಾಮೀಜಿಯವರನ್ನು ಸ್ವಾಗತಿಸುತ್ತ ಅವರನ್ನು–ಅವರ ಸಾಧನೆಗಳನ್ನು ಕೊಂಡಾಡಿದ. ರಾಜನ ಸೋದರನಾದ ದಿನಕರ ಸೇತುಪತಿ ರಾಜ್ಯದ ಪರವಾಗಿ ಬಿನ್ನವತ್ತಳೆಯನ್ನು ಓದಿ ಅದನ್ನು ಒಂದು ಅತ್ಯಂತ ಸುಂದರವಾಗಿ ಕೆತ್ತನೆ ಮಾಡಲ್ಪಟ್ಟ ಸುವರ್ಣ ಕರಂಡದಲ್ಲಿಟ್ಟು ಸ್ವಾಮೀಜಿಯವರಿಗೆ ಅರ್ಪಿಸಿದ. ಬಳಿಕ ಸ್ವಾಮೀಜಿ ಎದ್ದುನಿಂತು ತಮ್ಮ ಭಾಷಣವನ್ನಾರಂಭಿಸಿದರು. ಇದು ಭಾರತದ ಇತಿಹಾಸದಲ್ಲಿ ಸುವರ್ಣಾಕ್ಷರಗಳಿಂದ ಬರೆದಿಡಬೇಕಾದಂತಹ ಭಾಷಣ. ಈ ಭಾಷಣವನ್ನು ಸ್ವಾಮೀಜಿಯವರ ದೇವದುರ್ಲಭ ಕಂಠದಿಂದಲೇ ಕೇಳುವ ಭಾಗ್ಯ ನಮ್ಮ ಪಾಲಿಗಿಲ್ಲ. ಆದರೆ ಅದನ್ನು ಓದಿದರೂ ಕೂಡ ನಮ್ಮೊಳಗೆ ಸ್ಫೂರ್ತಿ-ಚೈತನ್ಯ ಸಂಚಾರವಾಗುವುದರಲ್ಲಿ ಸಂದೇಹ ವಿಲ್ಲ. ಸ್ವಾಮೀಜಿ ತಮ್ಮ ಭಾಷಣವನ್ನು ಹೀಗೆ ಪ್ರಾರಂಭಿಸಿದರು:

“ಸುದೀರ್ಘ ರಾತ್ರಿ ಕಡೆಗಿಂದು ಕೊನೆಗಾಣುತ್ತಿದೆ! ಬಹುಕಾಲದ ಶೋಕ ತಾಪಗಳು ಕಡೆಗಿಂದು ಮಾಯವಾಗುತ್ತಿವೆ! ಇದುವರೆಗೆ ಶವದಂತೆ ಬಿದ್ದಿದ್ದ ಶರೀರವಿಂದು ಸಚೇತನವಾಗುತ್ತಿದೆ! ಅದೋ ಕಿವಿಗೊಟ್ಟು ಕೇಳಿ, ವಾಣಿಯೊಂದು ಕೇಳಿ ಬರುತ್ತಿದೆ– ಕಾಲಗರ್ಭದಾಳದಿಂದ ಹೊಮ್ಮಿ, ಜ್ಞಾನ-ಪ್ರೇಮ-ಕರ್ಮಗಳೆಂಬ ಅನಂತ ಹಿಮಾಲಯ ಶಿಖರಗಳಿಂದ ಮರುದನಿಯಾಗಿ ಚಿಮ್ಮಿ, ಬರಬರುತ್ತ ಪ್ರಬಲವಾಗಿ, ಬಂದಂತೆಲ್ಲ ಅಪ್ರತಿಹತವಾಗಿ ಸುಸ್ಪಷ್ಟ ನಿನಾದವೊಂದು ಮೊಳಗು ತ್ತಿದೆ. ಅಗೋ ನೋಡಿ–ನಿದ್ರಾಪರವಶವಾಗಿದ್ದ ಭಾರತವಿಂದು ಎಚ್ಚರಗೊಳ್ಳುತ್ತಿದೆ! ಹಿಮಾ ಲಯದಿಂದ ಬೀಸಿ ಬರುವ ತಂಗಾಳಿಯಂತೆ, ನಿರ್ಜೀವವಾದಂತಿದ್ದ ಅಸ್ಥಿ ಮಾಂಸಗಳಿಗೆ ಜೀವ ದಾನ ಮಾಡುತ್ತಿದೆ; ಜಡನಿದ್ರೆಯನ್ನು ಹೊಡೆದೋಡಿಸುತ್ತಿದೆ. ಆದರಿದು ಕುರುಡರಿಗೆ ಕಾಣದು; ಕುಹಕಿಗಳಿಗೆ-ವಿಕೃತಬುದ್ಧಿಯವರಿಗೆ ಅರಿವಾಗದು. ಭಾರತಾಂಬೆ ಯುಗಯುಗಗಳ ನಿದ್ರೆಯಿಂದ ಮೇಲೇಳುತ್ತಿದ್ದಾಳೆ. ಇನ್ನಾಕೆ ನಿದ್ರಿಸುವುದಿಲ್ಲ. ಆಕೆಯನ್ನು ಯಾರೂ ತಡೆಯಬಲ್ಲವರಿಲ್ಲ. ಯಾವ ಬಾಹ್ಯಶಕ್ತಿಯೂ ಆಕೆಯನ್ನು ಕೆಳಗೊತ್ತಲಾರದು. ಏಕೆಂದರೆ, ಅಗೋ ನೋಡಿ!– ಮಹಾಕಾಳಿ ಮತ್ತೊಮ್ಮೆ ಎಚ್ಚತ್ತು ಮೈಕೊಡಹಿ ಉಸಿರೆಳೆದು ನಿಲ್ಲುತ್ತಿದ್ದಾಳೆ!... ”

ಬಳಿಕ ಸ್ವಾಮೀಜಿ ಅತ್ಯಂತ ವಿನೀತ ಭಾವದಿಂದ, ತಮಗೆ ರಾಮನಾಡಿನ ರಾಜನೂ ಪ್ರಜೆಗಳೂ ತೋರಿದ ಆದರಕ್ಕಾಗಿ ಹೃತ್ಪೂರ್ವಕ ಕೃತಜ್ಞತೆಗಳನ್ನು ಸಲ್ಲಿಸಿದರು. ಅಲ್ಲದೆ, ತಮ್ಮ ಸಾಧನೆಯು ಅತ್ಯಂತ ಗೌಣವಾದುದೆಂದೂ ಮತ್ತು ಅದರ ಕೀರ್ತಿಯೂ ಕೂಡ ತಮ್ಮನ್ನು ಅಮೆರಿಕೆಗೆ ಕಳಿಸಿಕೊಟ್ಟ ಮಹಾರಾಜರಿಗೂ ಮಹಾಜನತೆಗೂ ಸೇರಬೇಕಾದ್ದೆಂದೂ ಹೇಳಿದರು. ಧರ್ಮವೇ ಭಾರತದ ಜೀವಾಳ, ಮತ್ತು ಸಮಸ್ತ ಜಗತ್ತಿಗೆ ಭಾರತವು ನೀಡಬೇಕಾದ ಕಾಣಿಕೆಯೆಂದರೆ ತನ್ನ ಧಾರ್ಮಿಕ-ಆಧ್ಯಾತ್ಮಿಕ ಸಂಪತ್ತು ಎಂದು ಸ್ವಾಮೀಜಿ ಸಾರಿದರು. ನಾವು ಈ ಮಾತುಗಳ ಮಹತ್ವವನ್ನೂ ಅವುಗಳ ಪರಿಣಾಮವನ್ನೂ ಅರಿಯಬೇಕಾದರೆ, ಅಂದಿನ ಭಾರತದ ಚಿತ್ರವನ್ನು ಕಲ್ಪಿಸಿಕೊಳ್ಳಬೇಕಾಗುತ್ತದೆ. ಭಾರತೀಯರು ತಮ್ಮತನದಲ್ಲಿ, ತಮ್ಮ ಧರ್ಮದಲ್ಲಿ ಶ್ರದ್ಧೆಯನ್ನು ಸಂಪೂರ್ಣವಾಗಿ ಕಳೆದುಕೊಳ್ಳುತ್ತಿದ್ದ ಕಾಲ ಮತ್ತು ವಿದ್ಯಾವಂತರೆನ್ನಿಸಿಕೊಂಡವರು ಹಿಂದೂ ಶಾಸ್ತ್ರಗ್ರಂಥಗಳನ್ನು, ಆಚರಣೆಗಳನ್ನು, ಬಾಹ್ಯಪೂಜೆಗಳನ್ನು ನಿಂದಿಸಿ ನಿರ್ನಾಮಗೈಯಲು ಉದ್ಯುಕ್ತ ರಾಗಿದ್ದ ಕಾಲ ಅದು. ಅಂದು ಸ್ವಾಮೀಜಿ, ತ್ಯಾಗವೇ ಹಿಂದೂಧರ್ಮವು ಸಾರುವ ಅತ್ಯುನ್ನತ ಆದರ್ಶ ಎಂದು ಘೋಷಿಸಿ, ಯಾವನು ಇದನ್ನು ಒಪ್ಪಲು ಸಿದ್ಧನಿಲ್ಲವೋ ಅವನೊಬ್ಬ ಮಿಥ್ಯಾ ವಾದಿಯೆಂದು ನುಡಿದರು.

ಆದರೆ ಸ್ವಾಮೀಜಿ ಕೇವಲ ಒಬ್ಬ ಭಾವನಾಜೀವಿಯಲ್ಲ; ಕೇವಲ ಕನಸಿನಲ್ಲಿ ಗೋಪುರ ಕಟ್ಟುವವರಲ್ಲ. ಅವರು ಯಾವ ಅತ್ಯುನ್ನತ ಆದರ್ಶಗಳನ್ನು ಬೋಧಿಸಿದರೋ ಅವು ಅತ್ಯಂತ ಅನುಷ್ಠಾನಾತ್ಮಕವೆಂಬುದನ್ನೂ ತೋರಿಸಿಕೊಟ್ಟರು. ಹಿಂದೂಗಳೆಲ್ಲರ ಆದರ್ಶವು ತ್ಯಾಗ ಎಂದು ಸ್ವಾಮೀಜಿ ಸಾರಿದರಾದರೂ ಅದು ಎಷ್ಟರ ಮಟ್ಟಿಗೆ ಅನುಷ್ಠಾನಯೋಗ್ಯ ಮತ್ತು ಅದನ್ನು ಅನುಷ್ಠಾನಗೊಳಿಸುವುದು ಹೇಗೆ ಎಂಬುದನ್ನೂ ಅವರು ವಿಶದೀಕರಿಸದಿರಲಿಲ್ಲ. ಸ್ವಾಮೀಜಿ ಹೇಳಿದರು: “... ಇದೇ (ತ್ಯಾಗವೇ) ನಮ್ಮ ಆದರ್ಶ, ಇದೇ ನಮ್ಮ ಗುರಿ ಎಂದು ನಮಗೆ ತಿಳಿದಿದೆ. ಆದರೆ ಇದು ನಮಗೆ ಗೋಚರವಾಗುವುದು ಒಂದಷ್ಟು ಅನುಭವವಾದ ಮೇಲೆಯೇ. ನಾವೊಂದು ಮಗುವಿಗೆ ತ್ಯಾಗದ ಮಹತ್ವವನ್ನು ಮನಗಾಣಿಸಲಾರೆವು. ಏಕೆಂದರೆ ಮಕ್ಕಳು ಹುಟ್ಟಾ ಆಶಾವಾದಿಗಳು. ಮಗುವಿನ ಜೀವವಿರುವುದೇ ಅದರ ಇಂದ್ರಿಯಗಳಲ್ಲಿ. ಮಗುವಿನ ಪಾಲಿಗೆ ಜೀವನವೆಂದರೆ ಇಂದ್ರಿಯಸುಖ. ಆದ್ದರಿಂದ, ಇಂತಹ ಶಿಶುಪ್ರಾಯರಾದವರು ಎಲ್ಲೆಡೆಗಳಲ್ಲೂ ಇದ್ದೇ ಇರುತ್ತಾರೆ. ಇವರಿಗೆ ಒಂದಷ್ಟು ಸುಖದ ಅನುಭವವಾಗಬೇಕು. ಅನಂತರ ಅವರಿಗೆ ಅದರ ಪೊಳ್ಳುತನದ ಅರಿವಾಗುತ್ತದೆ; ಆಗ ವಿರಕ್ತಿ ತಾನಾಗಿಯೇ ಬರುತ್ತದೆ. ಅಲ್ಲದೆ, ಅಂಥವರಿಗಾಗಿ ನಮ್ಮ ಶಾಸ್ತ್ರಗ್ರಂಥಗಳಲ್ಲಿ ಸಾಕಷ್ಟು ಅವಕಾಶ ಕಲ್ಪಿಸಲಾಗಿದೆ. ಆದರೆ ದುರದೃಷ್ಟವಶಾತ್, ಬರಬರುತ್ತ ಏನಾಗಿದೆಯೆಂದರೆ ಸಂನ್ಯಾಸಿಗಳಿಗೆ ಮಾತ್ರ ಅನ್ವಯವಾಗುವ ನಿಯಮಗಳನ್ನು ಪ್ರತಿಯೊಬ್ಬನ ಮೇಲೂ ಹೇರುವ ಮನೋಭಾವ ಬೆಳೆದಿದೆ. ಇದೊಂದು ದೊಡ್ಡ ತಪ್ಪು. ಈ ತಪ್ಪು ಘಟಿಸಿರದಿದ್ದರೆ ಭರತಖಂಡದ ದುಃಖ ದಾರಿದ್ರ್ಯಗಳು ಎಷ್ಟೋ ಕಡಿಮೆಯಾಗುತ್ತಿದ್ದವು. ಒಬ್ಬ ನಿರ್ಗತಿಕನ ಜೀವನವೂ ಜಟಿಲವಾದ ನೈತಿಕ ಹಾಗೂ ಆಧ್ಯಾತ್ಮಿಕ ನಿಯಮಗಳಿಂದ ಬಿಗಿಯಲ್ಪಟ್ಟಿದೆ. ಅವುಗಳಿಂದೆಲ್ಲ ಅವನಿಗೆ ಏನೇನೂ ಪ್ರಯೋಜನವಿಲ್ಲ. ಛೆ! ಬಿಟ್ಟುಬಿಡಿ ಅವನನ್ನು! ಪಾಪ, ಆ ಬಡವ ಸ್ವಲ್ಪ ಸುಖ ಪಡಲಿ. ಆಮೇಲೆ ಅವನು ತಾನಾಗಿಯೇ ಎಚ್ಚರ ಗೊಳ್ಳುತ್ತಾನೆ; ತ್ಯಾಗಮನೋಭಾವ ಅದಾಗಿಯೇ ಉಂಟಾಗುತ್ತದೆ.”

ಭಾರತವು ಉಳಿದುಕೊಳ್ಳಬೇಕಾದರೆ ಅದು ತನ್ನ ಆಧ್ಯಾತ್ಮಿಕ ಸಂಪತ್ತನ್ನು ಕಾಯ್ದುಕೊಳ್ಳಲೇ ಬೇಕು ಎಂದು ಸ್ವಾಮೀಜಿ ಒತ್ತಿ ಹೇಳಿದರು. ತಮ್ಮ ಭಾಷಣವನ್ನು ಮುಕ್ತಾಯಗೊಳಿಸುತ್ತ ಅವರು ಹೀಗೆಂದರು: “ನಮ್ಮ ಜನಾಂಗದ ಶ್ರೇಯಸ್ಸಿಗೆ ಇರುವುದೊಂದೇ ಮಾರ್ಗ–ನಮ್ಮ ಪೂರ್ವಿಕರು ನಮಗಿತ್ತ ಅಧ್ಯಾತ್ಮದ ಅನರ್ಘ್ಯ ನಿಧಿಯನ್ನು ಸಂರಕ್ಷಿಸಿಕೊಳ್ಳುವುದು... ನಿಮಗೆ ಅಧ್ಯಾತ್ಮದಲ್ಲಿ ನಂಬಿಕೆಯಿರಲಿ ಇಲ್ಲದಿರಲಿ, ನಮ್ಮ ರಾಷ್ಟ್ರದ ಹಿತಕ್ಕಾಗಿ ಅದನ್ನು ಉಳಿಸಿಕೊಂಡು, ಅದರಂತೆ ನಡೆಯಬೇಕು. ಅನಂತರ ನೀವು ನಿಮ್ಮ ಮತ್ತೊಂದು ಕೈಯಿಂದ, ಇತರ ರಾಷ್ಟ್ರಗಳಿಂದ ಏನನ್ನಾದರೂ ಪಡೆದುಕೊಳ್ಳಿ. ಆದರೆ ಅವೆಲ್ಲವೂ ನಿಮ್ಮ ಜನಾಂಗದ ಪ್ರಧಾನ ಆದರ್ಶಕ್ಕೆ ಅಧೀನವಾಗಿರಬೇಕು. ತನ್ಮೂಲಕ ಅದ್ಭುತವಾದ, ಕೀರ್ತಿಶಾಲಿಯಾದ ಭವಿಷ್ಯದ ಭಾರತ ಉದಿಸು ತ್ತದೆ. ಅದು ಹಿಂದಿಗಿಂತ ಹೆಚ್ಚು ಭವ್ಯವಾಗಿರುತ್ತದೆ. ಮತ್ತು, ಅಂತಹ ಭಾರತವು ಉದಿಸುತ್ತಿದೆ ಎಂಬುದು ನನಗೆ ನಿಶ್ಚಯವಾಗಿದೆ... 

“ಸೋದರರೇ, ನಾವೆಲ್ಲರೂ ಶ್ರಮವಹಿಸಿ ದುಡಿಯೋಣ. ನಿದ್ರೆಗಿದು ಸಮಯವಲ್ಲ. ನಮ್ಮ ಇಂದಿನ ಕಾರ್ಯದ ಮೇಲೆ ಭಾವೀ ಭಾರತ ನಿಂತಿದೆ. ಭಾರತಾಂಬೆ ನಮಗಾಗಿ ಕಾಯುತ್ತಿದ್ದಾಳೆ. ಅವಳು ಕೇವಲ ನಿದ್ರಿಸುತ್ತಿದ್ದಾಳೆ ಅಷ್ಟೆ. ಏಳಿ, ಎಚ್ಚರಗೊಳ್ಳಿ; ನಮ್ಮ ಭಾರತಮಾತೆಯು ಹಿಂದೆಂದಿಗಿಂತ ವೈಭವಯುತಳಾಗಿ, ಬಲಿಷ್ಠಳಾಗಿ, ಶೋಭಾಯಮಾನವಾದ ತನ್ನ ಸನಾತನ ಸಿಂಹಾಸನದ ಮೇಲೆ ವಿರಾಜಿಸುವುದನ್ನು ನೋಡಿ! ಭಗವದ್ಭಾವನೆಯು ನಮ್ಮೀ ಭಾರತಭೂಮಿ ಯಲ್ಲಿ ಬೆಳೆದಿರುವಷ್ಟು ಸ್ಪಷ್ಟವಾಗಿ ಬೇರೆಲ್ಲೂ ಬೆಳೆದಿಲ್ಲ. ಏಕೆಂದರೆ ಭಗವಂತನ ಕುರಿತಾದ ಕಲ್ಪನೆಯೇ ಇಲ್ಲಿನಂತೆ ಬೇರೆಲ್ಲೂ ಇಲ್ಲ... ಪರಮ ದಯಾಳುವಾದ, ತಂದೆ-ತಾಯಿ-ಸಖನಾದ, ನಮ್ಮ ಆತ್ಮದ ಆತ್ಮನಾದ ಈಶ್ವರನ ಭಾವನೆಯನ್ನು ನೀವು ಭಾರತದಲ್ಲಿ ಮಾತ್ರವೇ ಕಾಣಬಲ್ಲಿರಿ. ಯಾರು ಶೈವರ ಶಿವನೊ, ವೈಷ್ಣವರ ವಿಷ್ಣುವೊ, ಮೀಮಾಂಸಕರ ಕರ್ಮವೊ, ಬೌದ್ಧರ ಬುದ್ಧನೊ, ಜೈನರ ಜಿನನೊ, ಕ್ರೈಸ್ತರ ಕ್ರಿಸ್ತನೊ, ಯಹೂದ್ಯರ ಜೆಹೋವನೊ, ಮಹಮ್ಮದೀಯರ ಅಲ್ಲಾನೊ, ವೇದಾಂತಿಗಳ ಪರಬ್ರಹ್ಮನೊ, ಯಾರು ಸರ್ವವ್ಯಾಪಿಯೋ ಮತ್ತು ಯಾರ ಮಹಿಮೆಯನ್ನು ಈ ದೇಶ ಮಾತ್ರವೇ ಸಂಪೂರ್ಣವಾಗಿ ಬಲ್ಲುದೊ ಆ ಭಗವಂತನು, ಗುರಿಮುಟ್ಟಲು ನಮಗೆ ಶಕ್ತಿ ನೀಡಲಿ, ಸಹಾಯ ನೀಡಲಿ, ನಮ್ಮನ್ನು ಹರಸಲಿ.”

ಸ್ವಾಮೀಜಿಯ ಮಾತುಗಳಿಂದ ಮಹಾರಾಜ ಭಾಸ್ಕರ ಸೇತುಪತಿ ಗಾಢವಾಗಿ ಪ್ರಭಾವಿತ ನಾಗಿದ್ದ. ಅಂದಿನ ಕಾರ್ಯಕ್ರಮದ ಮುಕ್ತಾಯದ ವೇಳೆಯಲ್ಲಿ ಅವನು, “ಸ್ವಾಮೀಜಿಯವರು ರಾಮನಾಡಿಗೆ ನೀಡಿದ ಭೇಟಿಯ ಸ್ಮರಣಾರ್ಥವಾಗಿ, ಮದ್ರಾಸು ರಾಜ್ಯದ ಬರಗಾಲ ಪರಿಹಾರ ನಿಧಿಗೆ ವಂತಿಗೆ ಸಂಗ್ರಹಿಸಿ ನೀಡಲಾಗುವುದು” ಎಂದು ಘೋಷಿಸಿದ.

ರಾಮನಾಡಿನ ಸತ್ಕಾರವನ್ನು ಸ್ವೀಕರಿಸಿದ ಸ್ವಾಮೀಜಿ, ತಮ್ಮ ಸಂಗಡಿಗರಾದ ಸೇವಿಯರ್ ದಂಪತಿಗಳು, ಗುಡ್​ವಿನ್ ಹಾಗೂ ಸ್ವಾಮಿ ನಿರಂಜನಾನಂದರೊಂದಿಗೆ ಉತ್ತರಾಭಿಮುಖವಾಗಿ ಪ್ರಯಾಣ ಮಾಡಿದರು. ಅವರು ಜನವರಿ ೩೧ರಂದು ಮಧ್ಯ ರಾತ್ರಿ ರಾಮನಾಡಿನಿಂದ ಹೊರಟು ಸಾರೋಟಿನಲ್ಲಿ ಪ್ರಯಾಣ ಮಾಡಿ, ಮರುದಿನ ಮುಂಜಾನೆ ೨೩ ಮೈಲಿ ದೂರದ ಪರಮಕುಡಿ ಯನ್ನು ಸೇರಿದರು. ಇಲ್ಲಿಯೂ ಸ್ವಾಮೀಜಿಯವರನ್ನು ಸ್ವಾಗತಿಸಿ ಅಭಿನಂದನೆಯನ್ನರ್ಪಿಸಲು ಸಾವಿರಾರು ಜನ ಉತ್ಸಾಹದಿಂದ ಕಾದಿದ್ದರು. ಅಭಿನಂದನಾ ಪತ್ರಕ್ಕೆ ಉತ್ತರವಾಗಿ ಸ್ವಾಮೀಜಿ ಮಾಡಿದ ಭಾಷಣ ತುಂಬ ಉತ್ಸಾಹಪೂರ್ಣವೂ ಶಕ್ತಿಭರಿತವೂ ಆಗಿತ್ತು. ಪಾಶ್ಚಾತ್ಯರ ಭೋಗ ವಾದದ ಬಗ್ಗೆ ಹಾಗೂ ಭಾರತದ ಆಧ್ಯಾತ್ಮಿಕತೆಯ ಬಗ್ಗೆ ಅವರು ಉಜ್ವಲವಾಗಿ ಮಾತನಾಡಿದರು. “ನಾಗರಿಕತೆಯಲ್ಲಿ ಭೋಗವಾದಕ್ಕೂ ಒಂದು ಸ್ಥಾನವಿದೆ; ಆದರೆ ಪಶ್ಚಿಮ ರಾಷ್ಟ್ರಗಳು ಧರ್ಮ ವನ್ನು ತಳಹದಿಯನ್ನಾಗಿ ಮಾಡಿಕೊಳ್ಳದೆ ಹೋದರೆ ಅವುಗಳ ಸಮಗ್ರ ನಾಗರಿಕತೆಯು ಕುಸಿದು ಬಿದ್ದು ಪುಡಿಪುಡಿಯಾಗುತ್ತದೆ!” ಎಂದು ಸ್ವಾಮೀಜಿ ಮುನ್ನುಡಿದರು. ಅವರು ಈ ಮಾತುಗಳ ನ್ನಾಡಿದ ಐವತ್ತು ವರ್ಷಗಳಲ್ಲೇ ಎರಡು ಭೀಕರ ಜಾಗತಿಕ ಯುದ್ಧಗಳು ನಡೆದುದನ್ನು, ವಿವಿಧ ಸಾಮಾಜಿಕ ಸಮಸ್ಯೆಗಳಿಂದ ಪಾಶ್ಚಾತ್ಯ ರಾಷ್ಟ್ರಗಳು ತತ್ತರಿಸುತ್ತಿರುವುದನ್ನು, ಆ ದೇಶಗಳು ಮತ್ತೊಮ್ಮೆ ಆಧ್ಯಾತ್ಮಿಕತೆಯತ್ತ ತಿರುಗುತ್ತಿರುವುದನ್ನು ನೋಡಿದಾಗ ಸ್ವಾಮೀಜಿಯವರ ಮಾತಿನ ತಥ್ಯ ಅರಿವಾಗುತ್ತದೆ.

ಪರಮಕುಡಿಯ ಸಮಾರಂಭವನ್ನು ಮುಗಿಸಿಕೊಂಡು ಹೊರಟ ಸ್ವಾಮೀಜಿ ಅಲ್ಲಿಗೆ ಹದಿನೈದು ಮೈಲಿ ದೂರದ ಮಾನಮಧುರೆಗೆ ಬಂದು ತಲುಪಿದರು. ಇಲ್ಲಿನ ಹಾಗೂ ಸಮೀಪದ ಶಿವಗಂಗೆಯ ಜನ ಅವರಿಗೆ ಒಂದಾದಮೇಲೊಂದು ತಂತಿ ಕಳಿಸುತ್ತ, ತಮ್ಮಲ್ಲಿ ಕೆಲ ಸಮಯವನ್ನಾದರೂ ಕಳೆಯಬೇಕೆಂದು ಬಲಾತ್ಕರಿಸಿದ್ದರು. ಆದ್ದರಿಂದ ಸ್ವಾಮೀಜಿ ಇಲ್ಲಿ ಮತ್ತೆ ನಿಲ್ಲಬೇಕಾಯಿತು. ಸಾವಿರಾರು ಜನ ಕಿವಿಗಡಚಿಕ್ಕುವಂತೆ ಜಯಘೋಷ ಮಾಡುತ್ತ, ಅವರನ್ನು ಸ್ವಾಗತಿಸಿದರು. ಇಲ್ಲಿ ಅವರಿಗೆ ಅರ್ಪಿಸಲಾದ ಬಿನ್ನವತ್ತಳೆಯಲ್ಲಿ, ಪಾಶ್ಚಾತ್ಯ ಭೋಗವಾದವು ಭಾರತೀಯರ ಧಾರ್ಮಿಕ ಶ್ರದ್ಧೆಯನ್ನೆಲ್ಲ ಸಡಿಲಗೊಳಿಸಿ ಧರ್ಮವನ್ನೇ ನಾಶಗೊಳಿಸುತ್ತಿದೆ ಎಂದು ಹೇಳಲಾಗಿತ್ತು. ತಮ್ಮ ಉತ್ತರದಲ್ಲಿ ಅವರು ಈ ಬಗ್ಗೆ ಪ್ರಸ್ತಾಪಿಸಿ, ಅದು ಹಾಗೇನಾದರೂ ಇದ್ದಲ್ಲಿ ಅದಕ್ಕೆ ಹೊಣೆ ಗಾರರು ಭಾರತೀಯರೇ ಹೊರತು ಇತರರಲ್ಲ ಎಂದರು. ಹಿಂದೂಗಳ ಅತಿ ಮಡಿವಂತಿಕೆಯನ್ನು, ‘ಅಡಿಗೆಮನೆಯ ಧರ್ಮ’ವನ್ನು ಸ್ವಾಮೀಜಿ ಕಟುವಾಗಿ ಟೀಕಿಸಿದರು. ಅರ್ಥರಹಿತ ಚರ್ಚೆಗಳನ್ನು, ಅಸಂಬದ್ಧ ಹೋರಾಟಗಳನ್ನು ನಿಲ್ಲಿಸಿ ಉಪಯುಕ್ತವಾದ ಕಾರ್ಯದಲ್ಲಿ ತೊಡಗುವಂತೆ ಕರೆನೀಡಿ ದರು. ಪುರಾತನ ಮಹರ್ಷಿಗಳು ಬಿಟ್ಟುಹೋಗಿರುವ ಭವ್ಯ ಪರಂಪರೆಯನ್ನು, ಅನಂತ ಜ್ಞಾನನಿಧಿ ಯನ್ನು ಉಳಿಸಿಕೊಳ್ಳಲು ಬೇಕಾದಂತಹ ಬುದ್ಧಿಶಕ್ತಿಯನ್ನು ಬೆಳೆಸಿಕೊಳ್ಳುವಂತೆ ಹೇಳಿದರು. “ಇಂದು ಇಡೀ ಜಗತ್ತಿಗೆ ಈ ನಿಧಿ ಬೇಕಾಗಿದೆ. ಈ ನಿಧಿಯನ್ನು ಹಂಚಿಕೊಡದಿದ್ದರೆ ಜಗತ್ತು ನಿರ್ನಾಮವಾಗುತ್ತದೆ. ಅದನ್ನು ಹೊರತನ್ನಿ, ಹಂಚಿ, ಪ್ರಚಾರ ಮಾಡಿ!” ಎಂದು ಕರೆ ನೀಡಿದ ಸ್ವಾಮೀಜಿ, ವ್ಯಾಸಮಹರ್ಷಿಗಳ ಮಾತನ್ನು ಉದ್ಧರಿಸಿದರು: ‘ದಾನವೇ ಈ ಕಲಿಯುಗದಲ್ಲಿ ಜನರ ಧರ್ಮ, ದಾನಗಳಲ್ಲಿ ಆಧ್ಯಾತ್ಮಿಕತೆಯ ದಾನ ಸರ್ವಶ್ರೇಷ್ಠವಾದುದು; ತರುವಾಯ ಅಕ್ಷರವಿದ್ಯಾ ದಾನ; ಆಮೇಲೆ ಜೀವದಾನ; ಮತ್ತು ಕಡೆಯದಾಗಿ ಅನ್ನದಾನ.’ ಭಾರತದಲ್ಲಿ ಅನ್ನದಾನ ಸಾಕಷ್ಟು ನಡೆದಿದೆ–ನಡೆಯುತ್ತಿದೆ ಎಂದ ಸ್ವಾಮೀಜಿಯವರು, ಲೌಕಿಕ ಹಾಗೂ ಪಾರಲೌಕಿಕ ವಿದ್ಯಾದಾನಕ್ಕೆ ಪ್ರಾಮುಖ್ಯ ನೀಡುವಂತೆ ಜನರನ್ನು ಒತ್ತಾಯಿಸಿದರು.

ಮಾನಮಧುರೆಯಿಂದ ಹೊರಟ ಸ್ವಾಮೀಜಿ ೨ಂ ಮೈಲಿ ದೂರ ಸಂಚರಿಸಿ ತಮ್ಮ ಸಂಗಡಿಗ ರೊಂದಿಗೆ ಪ್ರಸಿದ್ಧ ತೀರ್ಥಕ್ಷೇತ್ರವಾದ ಮಧುರೆಗೆ ಬಂದರು. ಹಲವಾರು ದೇವಾಲಯಗಳ ಬೀಡು, ಸುಪ್ರಸಿದ್ಧ ರಾಜರು ಆಳಿದ ನಾಡು, ಸಂಸ್ಕೃತ ಕಲಿಕೆಗೆ ಹೆಸರಾಗಿದ್ದ ಸ್ಥಳ ಈ ಮಧುರೆ. ಇದು ಆಗ ರಾಮನಾಡಿನ ಅರಸರ ಆಳ್ವಿಕೆಗೆ ಒಳಪಟ್ಟಿತ್ತು. ವಿವೇಕಾನಂದರನ್ನು ಮಧುರೆಯ ಮೀನಾಕ್ಷಿ ದೇವಾಲಯದ ವತಿಯಿಂದ ಸಕಲ ಗೌರವಗಳೊಂದಿಗೆ ಸ್ವಾಗತಿಸಲಾಯಿತು. ಬಳಿಕ ಮಹಾ ರಾಜರಿಗೆ ಸೇರಿದ ಬಂಗಲೆಯಲ್ಲಿ ಸ್ವಾಮೀಜಿ ಹಾಗೂ ಅವರ ಜೊತೆಗಾರರಿಗೆ ವಸತಿ ಕಲ್ಪಿಸ ಲಾಯಿತು. ಅಂದು ಸಂಜೆ ಭಾರೀ ಬಹಿರಂಗ ಸಭೆಯಲ್ಲಿ ನೀಡಲಾದ ಬಿನ್ನವತ್ತಳೆಗೆ ಉತ್ತರಿಸುತ್ತ ಸ್ವಾಮೀಜಿ, ಭಾರತದಲ್ಲಿ ನಡೆಯುತ್ತಿರುವ ಧಾರ್ಮಿಕ ಪುನರ್ಜಾಗೃತಿಯ ಬಗ್ಗೆ ಮಾತನಾಡಿದರು. ಅವರು ತುಂಬ ಬಳಲಿದ್ದರಾದರೂ ಸಾಕಷ್ಟು ದೀರ್ಘ ಭಾಷಣವನ್ನೇ ಮಾಡಿ, ಸನಾತನ ಧರ್ಮವು ಇಬ್ಬಗೆಯ ವಿಪತ್ತಿನಿಂದ ಪಾರಾಗಬೇಕಾಗಿದೆ ಎಂದರು. ಗೊಡ್ಡು ಸಂಪ್ರದಾಯಗಳನ್ನೇ ಧರ್ಮ ವೆಂದು ಸಾರುತ್ತ ಕೂಪಮಂಡೂಕಗಳಾಗಿರುವ ಮತಾಂಧರ ಹಿಡಿತ ಒಂದೆಡೆಯಾದರೆ, ಪಾಶ್ಚಾತ್ಯ ವಿಚಾರ-ಅವಿಚಾರ ಪ್ರೇರಿತರಾದ ಆಧುನಿಕ ಸುಧಾರಕರ ಹಿಡಿತ ಮತ್ತೊಂದೆಡೆ. ಇವೆರಡನ್ನೂ ತ್ಯಜಿಸಿ ಹಿಂದೂಧರ್ಮವು ಸಂಪ್ರದಾಯದಲ್ಲೂ ಆಧುನಿಕ ವಿಚಾರದಲ್ಲೂ ಇರುವ ಒಳ್ಳೆಯ ಅಂಶಗಳನ್ನು ಮಾತ್ರ ಉಳಿಸಿಕೊಂಡು ಮುಂಬರಿಯಬೇಕೆಂದು ಸ್ವಾಮೀಜಿ ಕರೆ ನೀಡಿದರು.

ಅಂದು ಮಧ್ಯಾಹ್ನ ಅವರು ನೂರಾರು ಜನರಿಗೆ ದರ್ಶನ ನೀಡಿದರು. ಅಲ್ಲದೆ ಕೆಲವು ವಿದ್ವಾಂಸರಿಗೆ ಸಂದರ್ಶನಕ್ಕೆ ಅವಕಾಶ ಕೊಟ್ಟರು. ಸಾಕಷ್ಟು ದೀರ್ಘಕಾಲ ನಡೆದ ಈ ಸಂದರ್ಶನದ ಅವಧಿಯಲ್ಲಿ ಅವರು ವೇದಗಳು, ವೇದಾಂತ ತತ್ತ್ವಶಾಸ್ತ್ರ, ಮಾಯೆ ಮತ್ತು ಮುಕ್ತಿ–ಇವುಗಳಿಗೆ ಸಂಬಂಧಿಸಿದ ಅನೇಕ ಜಟಿಲ ಪ್ರಶ್ನೆಗಳಿಗೆ ಉತ್ತರಿಸಿದರು. ಈ ಸಂದರ್ಶನವು ‘ಹಿಂದೂ’ ಹಾಗೂ ‘ಇಂಡಿಯನ್ ಮಿರರ್​’ ಪತ್ರಿಕೆಗಳಲ್ಲೂ ಪ್ರಕಟಗೊಂಡಿತು.

ಬಳಿಕ ಸ್ವಾಮೀಜಿ, ಮೀನಾಕ್ಷಿ ಅಮ್ಮನವರ ದರ್ಶನಕ್ಕೆ ಹೊರಟರು. ಅಲ್ಲಿ ಅವರನ್ನು ಅತ್ಯಂತ ಗೌರವಾದರಗಳಿಂದ ಬರಮಾಡಿಕೊಂಡು ದೇವಾಲಯವನ್ನೆಲ್ಲ ತೋರಿಸಲಾಯಿತು. ೧೬ನೇ ಶತ ಮಾನದಲ್ಲಿ ನಿರ್ಮಿತವಾದ ಈ ದೇವಾಲಯವು ದ್ರಾವಿಡ ವಾಸ್ತುಶಿಲ್ಪದ ಒಂದು ಶ್ರೇಷ್ಠ ಕಲಾಕೃತಿ. ಇದು ಅತ್ಯಂತ ಸುಂದರವಾಗಿ ಕಡೆಯಲ್ಪಟ್ಟ ಒಂಬತ್ತು ಗೋಪುರಗಳನ್ನೊಳಗೊಂಡಿದೆ. ಇವುಗಳ ಪೈಕಿ ಎಲ್ಲಕ್ಕಿಂತ ದೊಡ್ಡ ಗೋಪುರದ ಎತ್ತರ ೧೫೨ ಅಡಿ. ಈ ದೇವಾಲಯದ ಅನರ್ಘ್ಯ ರತ್ನಾಭರಣಗಳನ್ನೂ ಸ್ವಾಮೀಜಿಯವರಿಗೆ ತೋರಿಸಲಾಯಿತು.

ದಕ್ಷಿಣ ಭಾರತದ ಈ ಎಲ್ಲಾ ಸ್ಥಳಗಳಲ್ಲಿ–ಅದರಲ್ಲೂ ಮಧುರೆ, ರಾಮೇಶ್ವರಗಳಂತಹ ಪ್ರಸಿದ್ಧ ತೀರ್ಥಕ್ಷೇತ್ರಗಳಲ್ಲಿ–ಸ್ವಾಮೀಜಿಯವರಿಗೆ ಹೃತ್ಪೂರ್ವಕ ಸ್ವಾಗತ-ಆದರ-ಪೂಜ್ಯತೆ ದೊರೆತದ್ದು ಒಂದು ಅತ್ಯಂತ ವಿಶೇಷವಾದ ಸಂಗತಿ. ಇಲ್ಲಿನ ವಿದ್ಯಾವಂತರು, ಗಣ್ಯವ್ಯಕ್ತಿಗಳು, ಶ್ರೀಮಂತರು ಸೇರಿದಂತೆ ಸಮಸ್ತ ಜನಕೋಟಿಯೇ ಸ್ವಾಮಿ ವಿವೇಕಾನಂದರನ್ನು ಅವರು ತಮ್ಮವರೋ ಎಂಬಂತೆ ಪ್ರೀತ್ಯಾದರದಿಂದ ಎದುರ್ಗೊಂಡದ್ದೂ ಆಶ್ಚರ್ಯದ ವಿಷಯವೇ ಸರಿ. ಏಕೆಂದರೆ ಮೊದಲನೆಯದಾಗಿ ಸ್ವಾಮೀಜಿ ದಕ್ಷಿಣದವರಲ್ಲ. ಅಥವಾ ಯಾವುದಾದರೊಂದು ಸುಪ್ರಸಿದ್ಧವಾದ, ಸುಸ್ಥಾಪಿತವಾದ ಮಠದ ಸಂನ್ಯಾಸಿಯೂ ಅಲ್ಲ. ಕಡೆಗೆ ಜನ್ಮತಃ ಬ್ರಾಹ್ಮಣರೂ ಅಲ್ಲ! ಅಲ್ಲದೆ ಅವರ ಗುರುಗಳಾದ ಶ್ರೀರಾಮಕೃಷ್ಣರ ಹೆಸರಾಗಲಿ ಸಂದೇಶವಾಗಲಿ ಇಲ್ಲಿನ್ನೂ ಪ್ರಚಾರವಾಗಿರಲಿಲ್ಲ. ಇವೆಲ್ಲಕ್ಕಿಂತ ಹೆಚ್ಚಾಗಿ, ಸಂಪ್ರದಾಯ-ಮಡಿವಂತಿಕೆಗಳ ತವರೂರು ದಕ್ಷಿಣ ಭಾರತ. ವಿವೇಕಾನಂದರೋ, ಸಮುದ್ರವನ್ನು ದಾಟಿದುದೂ ಅಲ್ಲದೆ,‘ಮ್ಲೇಚ್ಛ’ರೊಂದಿಗೆ ಬೆರೆತು ಅವರ ಆಹಾರಾದಿಗಳನ್ನು ಸ್ವೀಕರಿಸಿದವರು; ಪರಮ ಪವಿತ್ರವಾದ ವೇದೋಪನಿಷತ್ತುಗಳನ್ನು ಪಾಶ್ಚಾತ್ಯರಿಗೆ ಬೋಧಿಸಿದವರು! ಸಂಪ್ರದಾಯಕ್ಕೆ ಇಂತಹ ‘ಅಪಚಾರ’ವನ್ನೆಸಗಿದ್ದೂ ಅಲ್ಲದೆ, ಭಾರತಕ್ಕೆ ಬಂದ ಮೇಲೆಯೂ ಅವರು ಅಂಧ ಸಂಪ್ರದಾಯಶರಣತೆಯನ್ನು, ಮಡಿವಂತಿಕೆಯನ್ನು ಬಹಿರಂಗವಾಗಿ ಖಂಡಿಸಿದರು; ಬಾಹ್ಯಾಚರಣೆಗಳನ್ನು ಕಡಿಮೆ ಮಾಡುವಂತೆ ಬೋಧಿಸಿದರು. ಇಂತಹ ಸ್ವಾಮಿ ವಿವೇಕಾನಂದರನ್ನು ಇಲ್ಲಿನ ಜನಕೋಟಿ ಮಾತ್ರವಲ್ಲದೆ ತೀರ್ಥಕ್ಷೇತ್ರಗಳ ಮೇಲ್ವಿ ಚಾರಕರೂ ಹೃತ್ಪೂರ್ವಕ ಗೌರವ-ಪೂಜ್ಯ ಭಾವದಿಂದ ಸ್ವೀಕರಿಸಿದ್ದು ಪರಮಾಶ್ಚರ್ಯದ ಸಂಗತಿ ಯಲ್ಲವೆ? ಅಲ್ಲದೆ ಅದು ಅತ್ಯಂತ ಹೆಮ್ಮೆಯ ವಿಷಯವೂ ಕೂಡ. ಹಾಗೆ ನೋಡಿದರೆ, ಸ್ವಯಂ ಶ್ರೀರಾಮಕೃಷ್ಣರು–ವಿವೇಕಾನಂದರ ಗುರು ಶ್ರೀರಾಮಕೃಷ್ಣ ಪರಮಹಂಸರು–ಯಾವ ದೇವಾ ಲಯದ ಅರ್ಚಕರಾಗಿದ್ದರೋ, ಯಾವ ದೇವಾಲಯದಲ್ಲಿ ಸಂಸ್ಥಾಪಿತಳಾದ ಕಾಳಿಕಾದೇವಿಯನ್ನು ಪೂಜಿಸಿ ಅರ್ಚಿಸಿ ಸಾಕ್ಷಾತ್ಕರಿಸಿಕೊಂಡರೋ, ಅಂತಹ ದಕ್ಷಿಣೇಶ್ವರದ ಕಾಳಿಯ ದೇವಸ್ಥಾನ ದೊಳಕ್ಕೇ ಸ್ವಾಮೀಜಿಯವರನ್ನು ಸೇರಿಸದೆ ಬಹಿಷ್ಕಾರ ಹಾಕುವ ಪ್ರಯತ್ನ ನಡೆಯುವುದನ್ನು ಮುಂದೆ ನೋಡಲಿದ್ದೇವೆ! ಒಟ್ಟಿನಲ್ಲಿ, ಸ್ವಾಮೀಜಿಯವರಿಗೆ ದಕ್ಷಿಣ ಭಾರತದಲ್ಲಿ ದೊರಕಿದ ಯಶಸ್ಸು, ಅವರ ಕಾರ್ಯೋದ್ದೇಶಗಳಿಗೆ ದೊರಕಿದ ಪ್ರಚಂಡ ವಿಜಯವಾಗಿತ್ತು, ಪ್ರೋತ್ಸಾಹ ವಾಗಿತ್ತು.

ಮಧುರೆಯ ಭೇಟಿಯನ್ನು ಸ್ವಾಮೀಜಿ ಒಂದೇ ದಿನದಲ್ಲಿ ಮುಕ್ತಾಯಗೊಳಿಸಿದರು. ಈಗ ಮೂರು ವಾರಗಳಿಂದಲೂ ಅವರು ನಿರಂತರವಾಗಿ ಪ್ರಯಾಣ ಮಾಡುತ್ತಿದ್ದರು. ಈ ದಿನಗಳಲ್ಲಿ ಅವರಿಗೆ ಹಗಲು-ಇರುಳುಗಳ ಮಧ್ಯೆ ವ್ಯತ್ಯಾಸವೇ ಇಲ್ಲದಂತಾಗಿತ್ತು. ನಿದ್ರೆ-ವಿಶ್ರಾಂತಿಯೆಂಬುದು ಸಂಪೂರ್ಣ ತಪ್ಪಿಹೋಗಿತ್ತು. ಪಾಶ್ಚಾತ್ಯದೇಶಗಳಲ್ಲಿ ಅವಿಶ್ರಾಂತವಾಗಿ ದುಡಿದು ಹಣ್ಣಾಗಿ ಹಿಂದಿರುಗಿದ್ದ ಸ್ವಾಮೀಜಿ ಮೊಟ್ಟಮೊದಲು ತಮ್ಮ ಶರೀರವನ್ನು ಶುಶ್ರೂಷೆ ಮಾಡಿಕೊಳ್ಳಬೇಕಾದ ಆವಶ್ಯಕತೆಯಿತ್ತು. ಆದರೆ ಭಾರತಕ್ಕೆ ಬರುತ್ತಿದ್ದಂತೆಯೇ ಅವರು ಮಾಡಿದ್ದು ತದ್ವಿರುದ್ಧವಾದು ದನ್ನು. ಸಮಯಾಸಮಯವೆನ್ನದೆ ಪ್ರಯಾಣ ಮಾಡುತ್ತ, ಸಾವಿರಾರು ಜನರನ್ನುದ್ದೇಶಿಸಿ ಒಂದಾದ ಮೇಲೊಂದು ಭಾಷಣ ಮಾಡುತ್ತ, ಇನ್ನೆಷ್ಟೋ ಸಾವಿರ ಜನರಿಗೆ ದರ್ಶನ-ಸಂದರ್ಶನ ನೀಡಿದರು. ಇದರಿಂದ ಮೊದಲೇ ದುರ್ಬಲವಾಗಿದ್ದ ಅವರ ಶರೀರದ ಮೇಲೆ ಶಾಶ್ವತವಾದ ದುಷ್ಪರಿಣಾಮ ಉಂಟಾಯಿತು. ಕಬ್ಬಿಣದ ಮಾಂಸಖಂಡಗಳು–ಉಕ್ಕಿನ ನರಗಳು ಕೂಡ ಈ ಒತ್ತಡವನ್ನು ತಡೆದುಕೊಳ್ಳಲಾರದೆ ಹೋದುವು! ಆದರೂ ಅವರ ಉತ್ಸಾಹ ಕಿಂಚಿತ್ತೂ ಕಡಿಮೆಯಾಗಲಿಲ್ಲ. ಕರ್ಮಯೋಗಿಗಳಾದ ಅವರು ತಮ್ಮ ದೇಹದ ಶ್ರಮವನ್ನು ಲೆಕ್ಕಿಸದೆ ಲೋಕಕಲ್ಯಾಣಾರ್ಥವಾಗಿ ತಮ್ಮನ್ನೇ ಶ್ರೀಗಂಧದಂತೆ ತೇದುಕೊಂಡರು.

ಮಧುರೆಯಲ್ಲಿ ಸ್ವಾಮಿ ಶಿವಾನಂದರು ಸ್ವಾಮೀಜಿಯವರನ್ನು ಕೂಡಿಕೊಂಡರು. ತಮ್ಮೆಲ್ಲ ಸಂಗಡಿಗರೊಂದಿಗೆ ಸ್ವಾಮೀಜಿ, ೧೫ಂ ಮೈಲಿ ದೂರದ ಕುಂಭಕೋಣಂಗೆ ಟ್ರೈನಿನಲ್ಲಿ ಹೊರ ಟರು. ಅವರ ಪ್ರಯಾಣದ ವಿವರಗಳು ತಂತಿಯ ಮೂಲಕ ಎಲ್ಲ ಊರುಗಳಿಗೂ ಮುಂಚಿತ ವಾಗಿಯೇ ತಿಳಿದುಬಿಡುತ್ತಿತ್ತು. ಯಾವ ಯಾವ ಊರುಗಳಲ್ಲಿ ಅವರ ಟ್ರೈನು ನಿಲ್ಲಲಿತ್ತೋ ಅಲ್ಲೆಲ್ಲ ಅವರನ್ನು ಎದುರ್ಗೊಂಡು ಬಿನ್ನವತ್ತಳೆಯನ್ನು ಅರ್ಪಿಸಲು ಜನರು ಸಿದ್ಧತೆಗಳನ್ನು ಮಾಡಿಕೊಂಡರು. ಟ್ರೈನು ಯಾವ ಸಮಯಕ್ಕೆ ಬಂದರೂ ಸರಿಯೇ, ಜನರಿಗೆ ಅವರ ಮುಖ ದರ್ಶನವಾದರೆ ಸಾಕು; ಅವರು ತಮ್ಮ ಅಭಿನಂದನೆಯನ್ನು ಸ್ವೀಕರಿಸಿ ಒಂದೆರಡು ಮಾತು ಗಳನ್ನಾಡಿದರಂತೂ ಪರಮ ಸಂತೋಷ! ಹೀಗೆ ಟ್ರೈನು ನಿಂತಲ್ಲೆಲ್ಲ ಜನ ನೂರುಗಟ್ಟಲೆ- ಸಾವಿರಗಟ್ಟಲೆ ಸಂಖ್ಯೆಯಲ್ಲಿ ಸೇರಿ ಹರ್ಷೋದ್ಗಾರಗಳಿಂದ ಸ್ವಾಗತಿಸಿದರು; ಹೂಮಾಲೆಗಳ ನ್ನರ್ಪಿಸಿ ಬಿನ್ನವತ್ತಳೆಗಳನ್ನು ಓದಿದರು. ಆಯಾ ಊರುಗಳಿಂದಲ್ಲದೆ, ದೂರದ ಊರು-ಹಳ್ಳಿ ಗಳಿಂದಲೂ ಜನ ಬಂದು, ತಮ್ಮ ಊರಿನ ಪರವಾಗಿ ಒಂದೊಂದು ಹಾರ ಬಿನ್ನವತ್ತಳೆ ಗಳನ್ನರ್ಪಿಸಿದರು. ಸಾಧ್ಯವಿದ್ದಲ್ಲೆಲ್ಲ ಸ್ವಾಮೀಜಿ ಸೂಕ್ತವಾದ ಕೆಲವು ಮಾತುಗಳನ್ನಾಡಿ ತಮ್ಮ ಕೃತಜ್ಞತೆಯನ್ನರ್ಪಿಸಿದರು. ಹೀಗೆ ಟ್ರೈನು ತಿರುಚನಾಪಳ್ಳಿ, ತಂಜಾವೂರುಗಳ ಮೂಲಕ ಹಾದು ಹೋಗಿ ಕುಂಭಕೋಣಂ ತಲುಪಿತು.

ತಮ್ಮ ಊರಿನಲ್ಲಿ ಸ್ವಾಮೀಜಿ ಒಂದು ದಿನವಾದರೂ ಉಳಿದುಕೊಳ್ಳಬೇಕೆಂಬುದು ತಿರುಚನಾ ಪಳ್ಳಿಯವರ ಆಶಯವಾಗಿತ್ತು. ಆದ್ದರಿಂದ ಅವರು, ಸ್ವಾಮೀಜಿ ಕುಂಭಕೋಣಂನಲ್ಲಿದ್ದಾಗ, ೭೫ಂ ಜನರ ಸಹಿಯಿದ್ದ ಮನವಿ ಪತ್ರವನ್ನು ಅಲ್ಲಿಗೇ ಕಳಿಸಿಕೊಟ್ಟರು. ಕುಂಭಕೋಣಂನಲ್ಲಿ ಸ್ವಾಮೀಜಿ ಮೂರು ದಿನ ಉಳಿದುಕೊಳ್ಳುವ ಏರ್ಪಾಡಾಗಿತ್ತು. ಅವುಗಳಲ್ಲಿ ಒಂದು ದಿನವನ್ನು ತಮಗೆ ಬಿಟ್ಟುಕೊಡಬೇಕೆಂಬುದು ತಿರುಚನಾಪಳ್ಳಿಯವರ ಒತ್ತಾಯ. ಆದರೆ, ಮುಂದೆ ಮದರಾಸಿ ನಲ್ಲಿ ತಮಗೆ ತೀವ್ರ ಚಟುವಟಿಕೆಯ ಕಾರ್ಯಕ್ರಮ ಬಾಕಿಯಿದೆ ಎಂದು ಊಹಿಸಿದ್ದ ಸ್ವಾಮೀಜಿ, ಹೆಚ್ಚು ವಿಶ್ರಾಂತಿ ತೆಗೆದುಕೊಳ್ಳುವುದಕ್ಕಾಗಿಯೇ ಇಲ್ಲಿ ಮೂರು ದಿನ ಉಳಿದುಕೊಳ್ಳುವ ಯೋಜನೆ ಹಾಕಿದ್ದರು. ಆದ್ದರಿಂದ, ಈಗ ತಾವು ಬರಲು ಸಾಧ್ಯವಿಲ್ಲದ್ದಕ್ಕಾಗಿ ವಿಷಾದ ವ್ಯಕ್ತಪಡಿಸಿದರು. ಆದರೆ, ಬರುವ ವರ್ಷ ತಾವು ದಕ್ಷಿಣಭಾರತದ ಪ್ರವಾಸ ಕೈಗೊಳ್ಳಲಿರುವುದಾಗಿ ತಿಳಿಸಿ, ಆಗ ಖಂಡಿತವಾಗಿಯೂ ಪ್ರತಿಯೊಂದು ಊರಿಗೂ ಮತ್ತೆ ಭೇಟಿ ನೀಡುವುದಾಗಿ ಆಶ್ವಾಸನೆಯಿತ್ತರು. ದುರದೃಷ್ಟವಶಾತ್, ಈ ಪ್ರವಾಸ ಎಂದೂ ಕೈಗೂಡಲಿಲ್ಲ.

ಕುಂಭಕೋಣಂ ಪಟ್ಟಣದಲ್ಲಿ ಸ್ವಾಮೀಜಿಯವರನ್ನು ಪ್ರಚಂಡ ಉತ್ಸಾಹದಿಂದ ಬರಮಾಡಿ ಕೊಳ್ಳಲಾಯಿತು. ಇಲ್ಲಿನ ಜನ ಹಿಂದಿನಿಂದಲೂ ಅವರ ಬಗ್ಗೆ ತುಂಬ ಆದರಾಭಿಮಾನ ವ್ಯಕ್ತಪಡಿಸಿದ್ದರು. ಶಿಕಾಗೋದ ಸಮ್ಮೇಳನದಲ್ಲಿ ಗಳಿಸಿದ ಪ್ರಚಂಡ ಯಶಸ್ಸಿಗಾಗಿ ಅವರನ್ನು ಅಭಿನಂದಿಸುವ ಠರಾವನ್ನು ಹೊರಡಿಸಿದ ಮೊದಲ ನಗರಗಳಲ್ಲಿ ಕುಂಭಕೋಣಂ ಕೂಡ ಒಂದು. ಈ ಊರಿನ ನಾಗರಿಕರ ಪರವಾಗಿ ಹಾಗೂ ವಿದ್ಯಾರ್ಥಿ ಸಮುದಾಯದ ಪರವಾಗಿ ಸ್ವಾಮೀಜಿಗೆ ಎರಡು ಬೇರೆ ಬೇರೆ ಬಿನ್ನವತ್ತಳೆಗಳನ್ನು ಸಮರ್ಪಿಸಲಾಯಿತು. ಇವುಗಳಿಗೆ ಉತ್ತರ ರೂಪವಾಗಿ ಸ್ವಾಮೀಜಿ ಮಾಡಿದ ಭಾಷಣವು, ಅವರು ಈ ಇಡೀ ಪ್ರವಾಸಕಾಲದಲ್ಲಿ ಮಾಡಿದ ಅತ್ಯಂತ ರೋಮಾಂಚಕರ ಭಾಷಣಗಳಲ್ಲೊಂದು. ಹಲವು ಗಂಟೆಗಳಷ್ಟು ದೀರ್ಘಕಾಲದ ಈ ಉಪನ್ಯಾಸ ದಲ್ಲಿ ಅವರು ಪ್ರಸ್ತಾಪಿಸಿದ ವಿಷಯಗಳು ಅನೇಕ. ಈ ಉಪನ್ಯಾಸದ ಶೀರ್ಷಿಕೆ “ವೇದಾಂತದ ಸಂದೇಶ”. ರಾಷ್ಟ್ರಕ್ಕೆ ವಿವೇಕಾನಂದರು ನೀಡಿದ ಕರೆಯ ಹಲವಾರು ಪ್ರಮುಖ ಅಂಶಗಳ ನ್ನೊಳಗೊಂಡ ಈ ಉಪನ್ಯಾಸವು ಭಾರತದ ಚರಿತ್ರೆಯಲ್ಲೇ ವಿಶೇಷವಾದುದು. ಈ ಸುದೀರ್ಘ ಉಪನ್ಯಾಸವನ್ನು ಸಂಕ್ಷೇಪವಾಗಿ ಹೇಳುವುದು ಕಷ್ಟ. ಆದರೆ ಅದರಲ್ಲಿನ ಕೆಲವು ಪ್ರಮುಖ ಅಂಶಗಳು ಹೀಗಿದ್ದುವು:

ಮೊದಲಿಗೆ ಸ್ವಾಮೀಜಿ, ‘ಧರ್ಮವು ಚಿನ್ನವನ್ನು ಕೊಡುವುದಿಲ್ಲ; ಆದ್ದರಿಂದ ಅದರಿಂದೇನೂ ಪ್ರಯೋಜನವಿಲ್ಲ’ ಎಂಬ ಆಪಾದನೆಯನ್ನು ಪ್ರಸ್ತಾವಿಸಿ, ಅದಕ್ಕೆ ಹೀಗೆ ಉತ್ತರಿಸಿದರು: “ನಿಜ, ಈ ಮೂರು ದಿನಗಳ ಲೌಕಿಕ ಪ್ರಪಂಚವೇ ಜೀವನದ ಸಾರಸರ್ವಸ್ವವಲ್ಲ ಎಂದು ಧರ್ಮವು ಸಾರುತ್ತದೆ. ಆದರೆ ಧರ್ಮದ ಹಿರಿಮೆಯನ್ನು ಸಾರುವ ಅತಿ ಮುಖ್ಯ ಅಂಶವೆಂದರೆ ಇದೇ! ಆದ್ದರಿಂದ ಧರ್ಮವೇ ನಿಜವಾದ, ಸತ್ವಯುತ ಸಿದ್ಧಾಂತ. ‘ಭಗವಂತನೇ ಸತ್ಯ, ಈ ಜಗತ್ತು ಮಿಥ್ಯೆ; ಸುವರ್ಣವೆಂಬುದೂ ಕೂಡ ಕೇವಲ ಮಣ್ಣುಹುಡಿ; ಈ ಜೀವನವೆಂಬುದು ಕೇವಲ ಹೊರೆ’ ಎಂದು ಧರ್ಮವು ಸಾರುವುದರಿಂದಲೇ ಅದು ನಿಜವಾದ ಸಿದ್ಧಾಂತ.” ಬಳಿಕ, ಅಂದಿನ ದಿನಗಳಲ್ಲಿ ಸಮಾಜ ಸುಧಾರಣೆಯ ಹುಯಿಲೆಬ್ಬಿಸುತ್ತಿದ್ದ ಅನೇಕ ಚಳವಳಿಗಳ ಬಗ್ಗೆ ಪ್ರಸ್ತಾಪಿಸಿದ ಸ್ವಾಮೀಜಿ ಯವರು ಹೇಳಿದರು, “ಪಾಶ್ಚಾತ್ಯ ರಾಷ್ಟ್ರಗಳಲ್ಲೂ ಒಂದು ಬಗೆಯ ಪ್ರಾಪಂಚಿಕ ವಿಮುಖತೆ ಕಂಡು ಬರುತ್ತಿದೆ. ಬಹು ಜನರು–ಸಾಧಾರಣವಾಗಿ ಸುಸಂಸ್ಕೃತ ಸ್ತ್ರೀಪುರುಷರೆಲ್ಲರೂ–ಈ ಸ್ಪರ್ಧೆ-ಹೋರಾಟ-ವ್ಯಾಪಾರಬುದ್ಧಿಗಳ ಅಮಾನುಷತೆಯಿಂದ ರೋಸಿಹೋಗಿದ್ದಾರೆ. ಈಗ ಅಲ್ಲಿನ ಕೆಲವು ಶ್ರೇಷ್ಠ ಚಿಂತಕರು ಭಾವಿಸುತ್ತಿದ್ದಾರೆ–ಕೇವಲ ಕೆಲವು ಸಾಮಾಜಿಕ ಮತ್ತು ರಾಜಕೀಯ ಬದಲಾವಣೆಗಳಿಂದ ಪಾಶ್ಚಾತ್ಯ ಸಂಸ್ಕೃತಿಯನ್ನು ಉಳಿಸಲಾಗದು; ಸಂಪೂರ್ಣ ಹೃದಯ ಪರಿ ವರ್ತನೆಯೊಂದೇ–ಎಂದರೆ ತನ್ನ ಧ್ಯೇಯಧೋರಣೆಗಳನ್ನೇ ಬದಲಾಯಿಸಿಕೊಳ್ಳುವುದರಿಂದ ಮಾತ್ರವೇ–ಅದು ವಿನಾಶ ಹೊಂದುವುದನ್ನು ತಪ್ಪಿಸಬಹುದು, ಎಂದು... ನಾನು ಕೇವಲ ಒಬ್ಬ ಸುಧಾರಕನಲ್ಲ; ಇಂದಿನ ಸುಧಾರಣಾ ಚಳವಳಿಗಳು ಹೆಚ್ಚಿನದೇನನ್ನೂ ಸಾಧಿಸಲಾರವು ಎಂಬುದು ನನ್ನ ಭಾವನೆ. ಏಕೆಂದರೆ, ಅವು ಪಾಶ್ಚಾತ್ಯ ರೀತಿನೀತಿಗಳ ಅಂಧಾನುಕರಣೆ ಮಾತ್ರವೇ ಆಗಿವೆ. ಪ್ರೀತಿ-ಸಹಾನುಭೂತಿಗಳಿಂದ ಮಾತ್ರವೇ ನಿಜವಾದ ಒಳಿತನ್ನು ಸಾಧಿಸಲು ಸಾಧ್ಯ.”

“ಪುರಾತನ ಆರ್ಯಾವರ್ತದಲ್ಲಿ, ಹಿಂದೂಧರ್ಮದ ಆದರ್ಶ ಮಾನವನೆಂದರೆ ಬ್ರಾಹ್ಮಣ. ಏಕೆಂದರೆ ಅವನಲ್ಲಿ ಪ್ರಾಪಂಚಿಕತೆಯು ಸಂಪೂರ್ಣವಾಗಿ ಲುಪ್ತವಾಗಿತ್ತು, ಮತ್ತು ಆತ ನಿಜವಾದ ಜ್ಞಾನಿಯಾಗಿದ್ದ... ಇಂದಿನ ದಿನದ ಈ ಜಾತಿಪದ್ಧತಿಯ ಸುಧಾರಣೋಪಾಯವೆಂದರೆ, ಈಗಾ ಗಲೇ ಮೇಲಿರುವವರನ್ನು ಕೆಳಗೆಳೆಯುವುದಲ್ಲ; ಬದಲಾಗಿ, ನಾವು ಪ್ರತಿಯೊಬ್ಬರೂ ವೇದಾಂತದ ಆದೇಶವನ್ನು ಅರಿತುಕೊಂಡು ಕಾರ್ಯಗತಗೊಳಿಸುವುದು, ನಿಜವಾದ ಆಧ್ಯಾತ್ಮಿಕತೆಯನ್ನು ಮೈ ಗೂಡಿಸಿಕೊಳ್ಳುವುದು ಮತ್ತು ತನ್ಮೂಲಕ ಆದರ್ಶ ಬ್ರಾಹ್ಮಣರಾಗುವುದು... ವೇದಾಂತವು ಮುಂದಿಡುವ ಈ ಆದರ್ಶವು ಭಾರತಕ್ಕೆ ಮಾತ್ರವಲ್ಲದೆ ಸಮಸ್ತ ಜಗತ್ತಿಗೇ ಅನ್ವಯಿಸುತ್ತದೆ.”

“ಓ ಹಿಂದುಗಳೇ, ಈ ನಮ್ಮ ಧರ್ಮನೌಕೆಯು ಯುಗಯುಗಾಂತರಗಳಿಂದಲೂ ಯಾನ ಮಾಡುತ್ತಿದೆ; ಸಾವಿರಾರು ಜನರನ್ನು ಭವಸಾಗರದಿಂದ ಪಾರುಮಾಡಿದೆ. ಇಂದು ಬಹುಶಃ ಅದು ಸ್ವಲ್ಪ ಸವೆದು ಹೋಗಿ, ಒಂದೆರಡು ತೂತು ಬಿದ್ದಿರಬಹುದು. ಹಾಗಿದ್ದರೆ ಆ ರಂಧ್ರವನ್ನು ಮುಚ್ಚಿ ಸರಿಪಡಿಸುವುದು ನನಗೂ ನಿಮಗೂ ಭೂಷಣವಾದುದು. ಮುಂದಿರುವ ಅಪಾಯವನ್ನು ನಮ್ಮ ದೇಶಬಾಂಧವರಿಗೆ ತಿಳಿಸೋಣ. ಅವರು ಜಾಗೃತರಾಗಿ ನಮ್ಮ ಈ ಕಾರ್ಯದಲ್ಲಿ ನೆರವಾಗಲಿ.”

“ವರ್ಣಧರ್ಮವು ಅನಿವಾರ್ಯವಾದುದು. ಆದರೆ ಡಾಲರ್ ವರ್ಣಭೇದಕ್ಕಿಂತ, ಎಂದರೆ ವ್ಯಕ್ತಿಯ ಆರ್ಥಿಕ ಸ್ಥಿತಿಗತಿಯ ಮೇಲೆ ಮಾಡುವ ಭೇದಭಾವಕ್ಕಿಂತ ಪರಿಶುದ್ಧತೆ-ಸಂಸ್ಕೃತಿ-ತ್ಯಾಗ ಗಳನ್ನವಲಂಬಿಸಿದ ವರ್ಣಭೇದವೇ ಲೇಸು ಎನ್ನುತ್ತೇನೆ ನಾನು. ಆದ್ದರಿಂದ ಖಂಡನೆಯ ಮಾತೆತ್ತಬೇಡಿ. ಬಾಯಿ ಮುಚ್ಚಿ; ಹೃದಯಗಳನ್ನು ತೆರೆಯಿರಿ. ಸಮಸ್ತ ಜವಾಬ್ದಾರಿಯೂ ನಿಮ್ಮದೇ ಎಂಬಂತೆ ಸಕಲರೂ ಈ ನಾಡಿನ ಮತ್ತು ಇಡೀ ಜಗತ್ತಿನ ಉನ್ನತಿಗಾಗಿ ಶ್ರಮಿಸಿರಿ. ವೇದಾಂತಜ್ಯೋತಿಯನ್ನು ಪ್ರತಿಯೊಂದು ಮನೆಗೂ ಒಯ್ದು, ಪ್ರತಿಯೊಂದು ಆತ್ಮದಲ್ಲಿಯೂ ಹುದುಗಿರುವ ದೈವತ್ವವನ್ನು ಮೇಲೆಬ್ಬಿಸಿ. ನಿಮ್ಮ ಯಶಸ್ಸು ಎಷ್ಟೇ ಅಲ್ಪ ಪ್ರಮಾಣದ್ದಾಗಿರಲಿ, ಈ ಉನ್ನತ ಆದರ್ಶಕ್ಕಾಗಿ ಬಾಳಿ-ದುಡಿದು-ಮಡಿದೆವೆಂಬ ತೃಪ್ತಿಯಾದರೂ ಇರುತ್ತದೆ. ಈ ಆದರ್ಶದ ಯಶಸ್ಸಿನಲ್ಲೇ ಮಾನವಕುಲದ ಮುಕ್ತಿಯಡಗಿದೆ.”

ಇದು ಭಾರತೀಯರಿಗೆ ಸ್ವಾಮಿ ವಿವೇಕಾನಂದರಿತ್ತ ಕರೆ.

ಸ್ವಾಮೀಜಿ ಕುಂಭಕೋಣಂನಲ್ಲಿದ್ದಾಗ ನಡೆದ ಒಂದು ಕುತೂಹಲಕರ ಘಟನೆ ಇತ್ತೀಚೆಗಷ್ಟೇ ಬೆಳಕಿಗೆ ಬಂದಿದೆ. ಮೊದಲನೆಯ ದಿನ ತಮ್ಮನ್ನು ಸ್ವಾಗತಿಸಲು ನೆರೆದಿದ್ದ ಜನಸಮೂಹದಲ್ಲಿ ಸ್ವಾಮೀಜಿಯವರು ಗೋವಿಂದ ಚೆಟ್ಟಿ ಎಂಬ ಒಬ್ಬ ಮನುಷ್ಯನನ್ನು ಗುರುತಿಸಿದರು. ಯಾರು ಈ ಗೋವಿಂದ ಚೆಟ್ಟಿ? ನಾಲ್ಕು ವರ್ಷಗಳ ಕೆಳಗೆ, ಎಂದರೆ ೧೮೯೩ರಲ್ಲಿ, ಮದ್ರಾಸಿಗೆ ಬಂದಿದ್ದಾಗ ಅವರಿಗೆ ಈತನನ್ನು ಭೇಟಿಯಾಗುವ ಸಂದರ್ಭ ಒದಗಿಬಂದಿತ್ತು. ಅವರು ಅಮೆರಿಕೆಗೆ ಹೋಗಲು ಸಿದ್ಧತೆಗಳನ್ನು ಮಾಡಿಕೊಳ್ಳುತ್ತಿದ್ದ ಸಂದರ್ಭದಲ್ಲಿ ಅವರಿಗೆ ತಮ್ಮ ತಾಯಿ ತೀರಿಹೋದಂತೆ ಕನಸು ಬಿದ್ದಿತ್ತು. ಸ್ವಾಮೀಜಿ ಆ ಬಗ್ಗೆ ತುಂಬ ವ್ಯಾಕುಲಗೊಂಡು ನಿಜಸಂಗತಿಯನ್ನು ತಿಳಿದುಕೊಳ್ಳ ಬೇಕೆಂದು ತೀವ್ರವಾಗಿ ಬಯಸಿದರು; ಅಲ್ಲದೆ ಕಲ್ಕತ್ತಕ್ಕೊಂದು ತಂತಿಯನ್ನೂ ಕಳಿಸಿದರು. ಆಗ ಅಳಸಿಂಗ ಪೆರುಮಾಳ್ ಹಾಗೂ ಮನ್ಮಥನಾಥ ಭಟ್ಟಾಚಾರ್ಯರು ಅವರನ್ನು ಗೋವಿಂದ ಚೆಟ್ಟಿ ಎಂಬ ಈ ಮನುಷ್ಯನ ಬಳಿಗೆ ಕರೆದೊಯ್ದಿದ್ದರು. ಪ್ರೇತಾರಾಧಕನಾದ ಈತ ಅನೇಕ ಪವಾಡಗಳನ್ನು ಮಾಡಬಲ್ಲವನಾಗಿದ್ದ. ಸ್ವಾಮೀಜಿಯವರ ತಾಯಿ ಬದುಕಿದ್ದಾರೆ, ಅವರ ಬಗ್ಗೆ ಏನೂ ಕಳವಳ ಪಡಬೇಕಾಗಿಲ್ಲ ಎಂದು ಈತ ಆಶ್ವಾಸನೆಯನ್ನು ಕೊಟ್ಟಿದ್ದ, ಮತ್ತು ಬಳಿಕ ಬಂದ ತಂತಿ ವರ್ತಮಾನದಿಂದ ಆ ಮಾತು ನಿಜವೆಂದು ದೃಢವಾಗಿತ್ತು. ಬಹುಶಃ ಆ ಮನುಷ್ಯನ ವಿಶಿಷ್ಟ ಬಣ್ಣ-ಆಕಾರಗಳಿಂದಾಗಿ ಸ್ವಾಮೀಜಿ ಆ ಗುಂಪಿನಲ್ಲೂ ಅವನನ್ನು ಗುರುತಿಸಲು ಸಮರ್ಥರಾದರು. ತಕ್ಷಣ ಅವನನ್ನು ಕೂಗಿ ಕರೆದು, ಆಮೇಲೆ ತಮ್ಮನ್ನು ಭೇಟಿಯಾಗುವಂತೆ ಹೇಳಿದರು. ಅಂತೆಯೇ ಗೋವಿಂದಚೆಟ್ಟಿ ಅನಂತರ ಬಂದು ಭೇಟಿಯಾದ. ಆಗ ಸ್ವಾಮೀಜಿ ಅವನನ್ನು ಕೇಳಿದರು, “ಅಯ್ಯಾ, ನಿನ್ನಲ್ಲಿ ಪವಾಡಶಕ್ತಿಯಿದೆ ಎಂಬುದು ನನಗೆ ಗೊತ್ತು. ನಿಜ, ಈ ಪವಾಡಶಕ್ತಿ ನಿನಗೆ ಹಣ-ಕೀರ್ತಿ ಎರಡನ್ನೂ ಗಳಿಸಿಕೊಟ್ಟಿದೆ. ಆದರೆ, ಈಗ ನೀನು ಹೇಳು, ಆ ಶಕ್ತಿಯನ್ನು ಪಡೆದುಕೊಳ್ಳುವುದಕ್ಕೋಸ್ಕರ ನೀನು ಎಷ್ಟೋ ಸಾಧನೆ ಮಾಡಿದ್ದೀಯಲ್ಲ, ಅದರಿಂದ ಆಧ್ಯಾತ್ಮಿಕ ವಾಗಿ ನಿನ್ನಲ್ಲಿ ಏನಾದರೂ ಸುಧಾರಣೆಯಾಗಿದೆಯೆ? ಆಧ್ಯಾತ್ಮಿಕ ದೃಷ್ಟಿಯಿಂದ ನೋಡಿದರೆ, ನೀನು ಎಲ್ಲಿ ಪ್ರಾರಂಭಿಸಿದೆಯೋ ಈಗಲೂ ಅಲ್ಲಿಯೇ ಇದ್ದೀಯೆ, ಅಲ್ಲವೆ?” ಗೋವಿಂದ ಚೆಟ್ಟಿ, “ಹೌದು ಸ್ವಾಮೀಜಿ, ನಿಮ್ಮ ಮಾತು ಸತ್ಯ. ಆಧ್ಯಾತ್ಮಿಕವಾಗಿ ನಾನು ಏನೇನೂ ಮುಂದುವರಿ ದಿಲ್ಲ” ಎಂದು ಒಪ್ಪಿಕೊಂಡ. ಆಗ ಸ್ವಾಮೀಜಿ ತುಂಬ ಆತ್ಮೀಯವಾಗಿ ಅವನೊಂದಿಗೆ ಮಾತನಾಡುತ್ತ ಹೇಳಿದರು, “ಹಾಗಿದ್ದ ಮೇಲೆ ಈ ಪವಾಡ ಶಕ್ತಿಯಿಂದ ನಿನಗಾದ ಲಾಭವೇನು ಹೇಳು? ನೀನು ಒಮ್ಮೆ ಭಗವದಾನಂದವನ್ನು ಸವಿದೆಯಾದರೆ, ಇಂಥವುಗಳೆಲ್ಲ ಕೆಲಸಕ್ಕೆ ಬಾರದ್ದು ಎಂದು ನೀನೇ ಕಂಡುಕೊಳ್ಳುವೆ.” ಹೀಗೆ ಹೇಳುತ್ತ ಸ್ವಾಮೀಜಿ ಅವನನ್ನು ಆಲಿಂಗಿಸಿಕೊಂಡರು. ಅದ್ಭುತ! ಅಂದಿನಿಂದ ಅವನ ಪವಾಡಶಕ್ತಿಯೆಲ್ಲ ಮಾಯವಾಯಿತು. ಬದಲಾಗಿ ಅವನಲ್ಲಿ ತೀವ್ರ ಭಗವದ್ ವ್ಯಾಕುಲತೆ ಉತ್ಪನ್ನವಾಯಿತು. ಕಡೆಗೆ ಅವನು ತನ್ನ ಹಳೆಯ ವೃತ್ತಿಯನ್ನು ಮಾತ್ರ ವಲ್ಲದೆ ಸಂಸಾರವನ್ನೂ ತ್ಯಜಿಸಿ ವಿರಾಗಿಯಾದ. ಹೀಗೆ ಸ್ವಾಮೀಜಿ ತಮ್ಮ ದಿವ್ಯ ಸ್ಪರ್ಶದಿಂದ ಆ ಮನುಷ್ಯನ ಜೀವನ ಪಥವನ್ನೇ ಬದಲಿಸಿದರು.

ಕುಂಭಕೋಣಂನಲ್ಲಿ ಸ್ವಾಮೀಜಿ ಮೂರು ದಿನಗಳನ್ನು ಕಳೆದು, ೧೯೫ ಮೈಲಿ ದೂರದ ಮದ್ರಾಸಿನತ್ತ ಹೊರಟರು. ಈಗ ಅವರೊಂದಿಗೆ ಗುರುಭಾಯಿಗಳಾದ ನಿರಂಜನಾನಂದರು ಮತ್ತು ಶಿವಾನಂದರು ಹಾಗೂ ಜೆ. ಜೆ. ಗುಡ್​ವಿನ್ ಇದ್ದರು. ಸೇವಿಯರ್ ದಂಪತಿಗಳು ಹಿಂದಿನ ದಿನವೇ ಮದ್ರಾಸಿಗೆ ಹೊರಟುಬಿಟ್ಟಿದ್ದರು–ಬಹುಶಃ ಸ್ವಲ್ಪವಾದರೂ ವಿಶ್ರಾಂತಿ ಪಡೆಯುವ ಉದ್ದೇಶದಿಂದ! ಮತ್ತೆ ದಾರಿಯುದ್ದಕ್ಕೂ ಉತ್ಸಾಹ-ಸಂಭ್ರಮದ ಸ್ವಾಗತ. ಟ್ರೈನು ನಿಂತಲ್ಲೆಲ್ಲ ಸ್ವಾಮೀಜಿ ಯವರ ದರ್ಶನ ಪಡೆಯುವ ಉದ್ದೇಶದಿಂದ ಜನ ನುಗ್ಗಿ ಬರುತ್ತಿದ್ದರು. ಕುಂಭಕೋಣಂನಿಂದ ಇಪ್ಪತ್ತು ಮೈಲಿ ದೂರದ ಮಾಯಾವರಂನಲ್ಲಿ ಟ್ರೈನು ಹತ್ತು ನಿಮಿಷಗಳ ಕಾಲ ನಿಂತಿತು. ನಿಲ್ದಾಣದಲ್ಲಿ ಜನ ತುಂಬಿಹೋಗಿದ್ದರು. ಜಿಲ್ಲಾ ಮುನ್ಸೀಫರಾದ ಶ್ರೀ ವೆಂಕಟರಾವ್ ಸಾಹೇಬರು ಹಾರ ಸಮರ್ಪಣೆ ಮಾಡಿದರು. ಸ್ವಾಗತ ಸಮಿತಿಯ ಮುಖ್ಯಸ್ಥರಾದ ಶ್ರೀ ನಟೇಶ್ ಅಯ್ಯರರು ಬಿನ್ನವತ್ತಳೆಯನ್ನೋದಿದರು. ಅದಕ್ಕುತ್ತರವಾಗಿ ಸ್ವಾಮೀಜಿಯವರು ಧನ್ಯವಾದಗಳನ್ನರ್ಪಿಸುವ ಕೆಲ ಮಾತುಗಳನ್ನಾಡಿದರು. ಬಳಿಕ ಟ್ರೈನು ಹೊರಟಿತು. ಜನ ಕೃತಕೃತ್ಯ ಭಾವದಿಂದ “ಜೈ ಸ್ವಾಮಿ ವಿವೇಕಾನಂದ ಮಹಾರಾಜ್ ಜೀ ಕೀ ಜೈ!” ಎಂದು ಜಯಘೋಷ ಮಾಡಿದರು.

ಮದ್ರಾಸಿಗೆ ೩೫ ಮೈಲಿ ದೂರದ ಚೆಂಗಲ್​ಪೇಟೆಯನ್ನು ಟ್ರೈನು ಬೆಳಿಗ್ಗೆ ಆರು ಗಂಟೆಗೆ ಮುಟ್ಟಿತು. ಆಗ \eng{‘The Madras Mail’} ಹಾಗೂ \eng{‘The Hindu’} ಪತ್ರಿಕೆಗಳ ಇಬ್ಬರು ಪ್ರತಿನಿಧಿ ಗಳು ಸ್ವಾಮೀಜಿಯವರ ಸಂದರ್ಶನಕ್ಕಾಗಿ ಅವರಿದ್ದ ಬೋಗಿಯನ್ನು ಹತ್ತಿದರು. ಅದೇ ಬೋಗಿಯಲ್ಲಿ ಪ್ರೊ ॥ ರಂಗಾಚಾರ್ಯರೂ ಪ್ರಯಾಣ ಮಾಡುತ್ತಿದ್ದರು. ಹಿಂದೆ ಸ್ವಾಮೀಜಿ ತಿರುವನಂತ ಪುರದಲ್ಲಿ ಸುಂದರರಾಮ ಅಯ್ಯರರ ಅತಿಥಿಯಾಗಿದ್ದಾಗ ರಂಗಾಚಾರ್ಯರನ್ನು ಸಂಧಿಸಿದ್ದುದನ್ನು ಇಲ್ಲಿ ಸ್ಮರಿಸಬಹುದು. ಪತ್ರಿಕಾ ಪ್ರತಿನಿಧಿಗಳ ಪರವಾಗಿ ಪ್ರೊ ॥ ರಂಗಾಚಾರ್ಯರು ಸ್ವಾಮೀಜಿ ಯವರನ್ನು ಹಲವಾರು ವಿಷಯಗಳ ಬಗ್ಗೆ ಆಳವಾಗಿ ಪ್ರಶ್ನಿಸಿದರು. ಅವುಗಳೆಲ್ಲದಕ್ಕೂ ಸ್ವಾಮೀಜಿ ಬಿಚ್ಚುಮನಸ್ಸಿನಿಂದ ಉತ್ತರಿಸಿದರು. ಇಬ್ಬರು ಪ್ರತಿನಿಧಿಗಳೂ ಈ ಸಂದರ್ಶನವನ್ನು ತಮ್ಮ ತಮ್ಮ ಪತ್ರಿಕೆಗಳಲ್ಲಿ ಪ್ರಕಟಿಸಿದರು. ಈ ಸಂದರ್ಶನವನ್ನು ವಿವೇಕಾನಂದರ “ಕೃತಿಶ್ರೇಣಿ”ಯಲ್ಲಿ ಕಾಣಬಹುದಾಗಿದೆ.

ಮದ್ರಾಸಿಗೆ ಇನ್ನು ಕೆಲವೇ ಮೈಲಿಗಳ ದೂರ; ಅಲ್ಲೊಂದು ಪುಟ್ಟ ನಿಲ್ದಾಣ. ಸ್ವಾಮೀಜಿ ಕುಳಿ ತಿದ್ದ ಟ್ರೈನು ಎಕ್ಸ್​ಪ್ರೆಸ್ ಗಾಡಿಯಾದ್ದರಿಂದ ಅಲ್ಲಿ ನಿಲುಗಡೆಯಿರಲಿಲ್ಲ. ಆದರೇನಂತೆ? ಜನ ರಂತೂ ನೆರೆದುಬಿಟ್ಟಿದ್ದರು! ಒಂದು ಸಲವಾದರೂ ವಿಶ್ವವಿಜೇತನಾದ ಜಗದ್ಗುರುವನ್ನು ಕಣ್ತುಂಬ ನೋಡಬೇಕೆಂದು ಅವರು ತೀರ್ಮಾನಿಸಿಬಿಟ್ಟಿದ್ದರು. ಸ್ವಾಮೀಜಿಯವರ ಟ್ರೈನನ್ನು ತಮ್ಮೂರಿನ ನಿಲ್ದಾಣದಲ್ಲಿಯೂ ನಿಲ್ಲಿಸಬೇಕೆಂದು ಸ್ಟೇಷನ್ ಮಾಸ್ಟರನ ಬಳಿ ಮನವಿ ಸಲ್ಲಿಸಿ ದರು. ಪರಿಪರಿಯಾಗಿ ಬೇಡಿಕೊಂಡರೂ ಆತ ನಿಯಮಗಳ ನೆಪವೊಡ್ಡಿ ನಿರಾಕರಿಸಿಬಿಟ್ಟ. ಅಷ್ಟು ಹೊತ್ತಿಗೆ ದೂರದಲ್ಲಿ ಟ್ರೈನು ಕಾಣಿಸಿಕೊಂಡಿತು. ಜನರಿಗೆ ಈಗ ಇದ್ದದ್ದು ಒಂದೇ ಉಪಾಯ– ನೂರಾರು ಜನ ಓಡಿ ಬಂದು ರೈಲು ಹಳಿಯ ಮೇಲೆ ಮಲಗಿಕೊಂಡುಬಿಟ್ಟರು! ಟ್ರೈನು ಹತ್ತಿರ ಬಂದೇ ಬಿಟ್ಟಿತು. ಸ್ಟೇಷನ್ ಮಾಸ್ಟರ್ ವಿಭ್ರಾಂತನಾದ. ಆದರೆ ಈ ದೃಶ್ಯ ಗಾರ್ಡಿನ ಕಣ್ಣಿಗೆ ಬಿತ್ತು. ತಕ್ಷಣ ಆತ ಟ್ರೈನು ನಿಲ್ಲಿಸಿದ. ಅಂತೂ ಗಾರ್ಡಿನ ಸಮಯಪ್ರಜ್ಞೆಯಿಂದ ಭೀಕರ ಅನಾಹುತ ತಪ್ಪಿತು. ಜನ ಸ್ವಾಮೀಜಿ ಕುಳಿತಿದ್ದ ಬೋಗಿಯನ್ನು ಮುತ್ತಿಕೊಂಡು ಜಯಘೋಷದ ಸುರಿಮಳೆ ಗರೆದರು. ತಕ್ಷಣ ಸ್ವಾಮೀಜಿ ಆಚೆ ಬಂದು ಜನರಿಗೆ ದರ್ಶನ ನೀಡಿದರು. ಜನರ ಆನಂದಕ್ಕೆ ಎಣೆಯೇ ಇಲ್ಲ. ಅವರ ಪ್ರೀತಿ ವಿಶ್ವಾಸವನ್ನು ಕಂಡು ಸ್ವಾಮೀಜಿಯವರ ಕಣ್ಣಂಚಿನಲ್ಲಿ ನೀರು ತುಂಬಿತು. ಎಲ್ಲರಿಗೂ ತಮ್ಮ ಹೃತ್ಪೂರ್ವಕ ಕೃತಜ್ಞತೆಯನ್ನರ್ಪಿಸಿದರು; ಕೈಯೆತ್ತಿ ಎಲ್ಲರನ್ನೂ ಹರಸಿದರು. ಬಳಿಕ ಟ್ರೈನು ಮುಂದೆ ಸಾಗಿತು.

