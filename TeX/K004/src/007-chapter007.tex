
\chapter{ನಿರ್ಲಕ್ಷ್ಯ! ನಿರ್ಲಕ್ಷ್ಯ!}

\noindent

ಹೀಗೆ ಇತ್ತ ಸ್ವಾಮೀಜಿ ಆಲ್ಮೋರದಲ್ಲಿ ವಿಶ್ರಾಂತಿ ಪಡೆಯುತ್ತ ಅಲ್ಲಿಂದಲೇ ಜಿಜ್ಞಾಸುಗಳ ಪ್ರಶ್ನೆ ಗಳಿಗೆ ಉತ್ತರಿಸುತ್ತ, ತಮ್ಮ ಸೋದರಸಂನ್ಯಾಸಿಗಳನ್ನೂ ಶಿಷ್ಯರನ್ನೂ ಹುರಿದುಂಬಿಸುತ್ತಿದ್ದಾಗ ಭಾರತದ ಇತರೆಡೆಗಳಲ್ಲಿ ಹಾಗೂ ಅಮೆರಿಕದಲ್ಲಿ ಅವರ ವಿರುದ್ಧ ಅಪಪ್ರಚಾರದ ಮತ್ತೊಂದು ಅಲೆ ಎದ್ದಿತ್ತು. ಇದಕ್ಕೆ ಕಾರಣಭೂತರಾದವರೆಂದರೆ ಸರ್ವಧರ್ಮ ಸಮ್ಮೇಳನಾಧ್ಯಕ್ಷರಾದ ಡಾ॥ ಹೆನ್ರಿ ಬರೋಸ್​ರವರು. ಹಿಂದೊಮ್ಮೆ ಇವರು ಸ್ವಾಮೀಜಿಯವರ ಮಹಿಮೆಯನ್ನು ಮನಗಂಡು ಅವರನ್ನು ಕೊಂಡಾಡಿದ್ದರಾದರೂ ಬಳಿಕ ಇವರು ಸ್ವಾಮೀಜಿ ಗಳಿಸಿದ ವಿಶ್ವವಿಖ್ಯಾತಿಯನ್ನು ಸಹಿಸ ಲಾರದೆ ಹೋದರು. ಸ್ವಾಮೀಜಿ ಭಾರತಕ್ಕೆ ಮರಳಿದ ಕೆಲದಿನಗಳಲ್ಲೇ ಡಾ ॥ ಬರೋಸರೂ ಕ್ರೈಸ್ತ ಧರ್ಮ ಪ್ರಚಾರಕ್ಕಾಗಿ ಮದ್ರಾಸಿಗೆ ಬಂದರು. ತಮಗೆ ಬಹಳಷ್ಟು ಸಹಾಯ ಮಾಡಿದ್ದ ಇವರಿಗೆ ಹೃತ್ಪೂರ್ವಕ ಸ್ವಾಗತ ನೀಡಬೇಕೆಂದು ಸ್ವಾಮೀಜಿ ತಮ್ಮ ಮದ್ರಾಸೀ ಶಿಷ್ಯರಿಗೆ ಸ್ಪಷ್ಟವಾಗಿ ತಿಳಿಸಿ ದ್ದರು. ಅದರಂತೆಯೇ ಅಲ್ಲಿನ ಜನ ವಿವೇಕಾನಂದರ ಮಾತಿಗೆ ಮನ್ನಣೆಕೊಟ್ಟು ಬರೋಸರಿಗೆ ಅದ್ಧೂರಿಯ ಸ್ವಾಗತವನ್ನೇ ನೀಡಿದರು. ಆದರೆ ಡಾ ॥ ಬರೋಸರು ಮಾತನಾಡಲು ತೊಡಗಿದಾಗ ಅವರ ಬಣ್ಣ ಹೊರಬಂದಿತು. ಇತರ ಧರ್ಮಪ್ರಚಾರಕರಂತೆಯೇ ಬರೋಸರೂ ಹಿಂದೂಧರ್ಮ ವನ್ನು ವಾಚಾಮಗೋಚರವಾಗಿ ನಿಂದಿಸತೊಡಗಿದಾಗ ಜನ ಸಿಟ್ಟಿಗೆದ್ದರು; ಅವರ ಮಾತುಗಳನ್ನು ಥೂತ್ಕರಿಸಿದರು. ಈಗ ಬರೋಸರ ಸಿಟ್ಟು ವಿವೇಕಾನಂದರ ಮೇಲೆ ತಿರುಗಿತು. ಭಾರತದಲ್ಲೆಲ್ಲ ಸಂಚರಿಸಿ ವಿವೇಕಾನಂದರ ವಿರುದ್ಧ ನಾನಾ ಬಗೆಯಾಗಿ ಅಪಪ್ರಚಾರ ಮಾಡಿದರು. ಅಲ್ಲದೆ ಅಮೆರಿಕಕ್ಕೆ ಹಿಂದಿರುಗಿ ಬೇರೊಂದು ರೀತಿಯಲ್ಲಿ ಅಪಪ್ರಚಾರ ಮಾಡಲಾರಂಭಿಸಿದರು. ಉದಾ ಹರಣೆಗೆ, ವಿವೇಕಾನಂದರು ಅಮೆರಿಕದ ಮಹಿಳೆಯರ ಬಗ್ಗೆ ಭಾರತದಲ್ಲೆಲ್ಲ ತುಚ್ಛವಾಗಿ ಮಾತನಾಡುತ್ತಿದ್ದಾರೆ ಎಂಬಿತ್ಯಾದಿಯಾಗಿ. ಈ ವಿಷಯಗಳೆಲ್ಲ ಸ್ವಾಮೀಜಿಯವರಿಗೆ ಪತ್ರಿಕಾ ವರದಿಗಳ ಹಾಗೂ ಅವರ ಆತ್ಮೀಯ ಶಿಷ್ಯರ ಪತ್ರಗಳ ಮುಖಾಂತರ ತಿಳಿದುಬಂತು. ಹಿಂದಿ ನಂತೆಯೇ ಸ್ವಾಮೀಜಿ ಇದನ್ನೆಲ್ಲ ನಿರ್ಲಕ್ಷಿಸಿದರು. ಆದರೆ ಈ ಅಪಪ್ರಚಾರಗಳು ಅಮೆರಿಕದಲ್ಲಿನ ಅವರ ಕಾರ್ಯೋದ್ದೇಶಕ್ಕೆ ಧಕ್ಕೆಯನ್ನುಂಟು ಮಾಡಬಹುದೆಂದು ಅಲ್ಲಿನ ಶಿಷ್ಯರು ಭಯ ಗೊಂಡರು. ಹೀಗೆ ತನ್ನ ಆತಂಕವನ್ನು ವ್ಯಕ್ತಪಡಿಸಿ ಮೇರಿ ಹೇಲ್ ಸ್ವಾಮೀಜಿಯವರಿಗೊಂದು ಪತ್ರ ಬರೆದರು. ಈ ವಿಷಯವನ್ನೆಲ್ಲ ಮತ್ತೆ ಮತ್ತೆ ಕೇಳಬೇಕಾಗಿ ಬಂದಾಗ ಅವರಿಗೂ ಬಿಸಿ ಯೇರಿತು. ಆಗ ಅವರು ಮೇರಿಗೊಂದು ಸುದೀರ್ಘ ಪತ್ರ ಬರೆದರು:

\noindent

ಪ್ರಿಯ ಸೋದರಿ,

ನಿನ್ನ ಪತ್ರದಲ್ಲಿ ಸೂಚಿತವಾದ ನಿರಾಶೆಯ ದನಿಗಾಗಿ ನನಗೆ ವಿಷಾದವಾಯಿತು. ಮತ್ತು ಅದರ ಕಾರಣ ನನಗೆ ಅರ್ಥವಾಗತ್ತದೆ. ನೀನು ಕೊಟ್ಟ ಎಚ್ಚರಿಕೆಗಾಗಿ ಧನ್ಯವಾದಗಳು. ನಿನ್ನ ಉದ್ದೇಶವನ್ನು ನಾನು ಸರಿಯಾಗಿ ಅರ್ಥಮಾಡಿಕೊಂಡಿದ್ದೇನೆ....

ಅಮೆರಿಕದ ಮಹಿಳೆಯರನ್ನು ಕುರಿತು ನನ್ನ ಹೇಳಿಕೆಗಳನ್ನು ಉಗ್ರವಾಗಿ ಖಂಡಿಸಿರುವ ಅಮೆರಿಕದ ವೃತ್ತಪತ್ರಿಕೆಗಳ ಬಹಳಷ್ಟು ಕತ್ತರಿಕೆಗಳು ನನಗೆ ತಲುಪಿವೆ. ನನ್ನನ್ನು ಜಾತಿಯಿಂದ ಹೊರಹಾಕಲಾಗಿದೆಯೆಂಬ ವಿಚಿತ್ರ ಸಮಾಚಾರವನ್ನು ಅವು ನನಗೆ ಒದಗಿಸಿವೆ!–ನನಗೊಂದು ಜಾತಿ ಇದ್ದಹಾಗೆ, ನಾನದನ್ನು ಕಳೆದುಕೊಂಡ ಹಾಗೆ! ಆದರೆ, ಸಂನ್ಯಾಸಿಯಾದ ನನಗೆ ಜಾತಿಯೆ!

ನನಗೆ ಯಾವ ಜಾತಿಯೂ ನಷ್ಟವಾಗಿಲ್ಲವಷ್ಟೇ ಅಲ್ಲ, ನನ್ನ ಪಾಶ್ಚಾತ್ಯ ಪ್ರವಾಸದಿಂದಾಗಿ ಬಾರತದಲ್ಲಿ ಸಮುದ್ರಯಾನದ ಬಗೆಗಿದ್ದ ವಿರೋಧವೂ ನುಚ್ಚು ನೂರಾಗಿದೆ. ನನ್ನನ್ನು ಜಾತಿ ಯಿಂದ ಹೊರಹಾಕಲೇಬೇಕಾದಲ್ಲಿ ಆಗ ನನ್ನೊಂದಿಗೆ ಭಾರತದ ಅರ್ಧಕ್ಕರ್ಧ ರಾಜರುಗಳನ್ನೂ ಮತ್ತು ಹೆಚ್ಚು ಕಡಿಮೆ ಭಾರತದ ಎಲ್ಲ ಸುಶಿಕ್ಷಿತ ಜನರನ್ನೂ ಹೊರಹಾಕಬೇಕಾಗುತ್ತದೆ... ಇತರರೆಲ್ಲ ಕೇವಲ ಮನುಷ್ಯರಾದರೆ, ಸಂನ್ಯಾಸಿಯು ನಾರಾಯಣಸ್ವರೂಪಿಯೆಂದು ಪರಿಗಣಿಸ ಲ್ಪಡುತ್ತಾನೆ. ಪ್ರಿಯ ಮೇರಿ, ನನ್ನ ಈ ಪಾದಗಳನ್ನು ರಾಜಾಧಿರಾಜರು ತೊಳೆದು ಪೂಜಿಸಿದ್ದಾರೆ. ನಾನು ರಸ್ತೆಯಲ್ಲಿ ಹೋಗುವಾಗ ಜನಸಂದಣಿಯನ್ನು ಹತೋಟಿಯಲ್ಲಿಡಲು ಪೋಲಿಸರು ಬೇಕಾಗಿದ್ದರು ಎಂದು ಹೇಳಿದರೆ ಸಾಕೆಂದು ತೋರುತ್ತದೆ. ನನಗೆ ಜಾತಿಯಿಂದ ಬಹಿಷ್ಕಾರ ಹಾಕುವುದೆಂದರೆ ಇದೇ ತಾನೆ!!! ಆದರೆ ಇದರಿಂದ ವಿಷನರಿಗಳ ಮುಖಕ್ಕೆ ತಣ್ಣೀರೆರಚಿದಂ ತಾಯಿತು. ಅಲ್ಲದೆ ಅವರೆಲ್ಲ ಯಾರಿಲ್ಲಿ?–ಲೆಕ್ಕಕ್ಕೆ ಬಾರದವರು. ಅವರು ಜೀವಂತವಾಗಿರುವು ದನ್ನೇ ಅರಿಯದೆ ನಾವಿಲ್ಲಿ ಸುಖವಾಗಿದ್ದೆವು. ಉಪನ್ಯಾಸವೊಂದರಲ್ಲಿ ನಾನು ಮಿಷನರಿಗಳ ಬಗ್ಗೆ ಮತ್ತು ಅವರ ಜಾತಿಯ ಉಗಮದ ಬಗ್ಗೆ ಏನನ್ನೋ ಹೇಳಿದ್ದೆ. ಆ ಸಂದರ್ಭದಲ್ಲಿ ನಾನು ಅಮೆರಿಕದ ‘ಚರ್ಚೀ’ ಮಹಿಳೆಯರ ಬಗ್ಗೆ, ಹಾಗೂ ಸುದ್ದಿಗಳನ್ನು ಹಬ್ಬಿಸುವಲ್ಲಿ ಅವರಿಗಿರುವ ಅದ್ಭುತ ಸಾಮರ್ಥ್ಯದ ಬಗ್ಗೆ ಪ್ರಸ್ತಾಪಿಸಬೇಕಾಯಿತು. ಇದರಿಂದ ನಾನು ಒಟ್ಟಾರೆಯಾಗಿ ಅಮೆರಿಕದ ಎಲ್ಲ ಮಹಿಳೆಯರನ್ನೂ ನಿಂದಿಸಿಬಿಟ್ಟೆನೋ ಎಂಬಂತೆ ಮಿಷನರಿಗಳು ಡಂಗುರ ಸಾರುತ್ತ ಅಲ್ಲಿನ ನನ್ನ ಕಾರ್ಯವನ್ನು ವಿಫಲಗೊಳಿಸಲು ನೋಡುತ್ತಿದ್ದಾರೆ. ಸ್ವತಃ ತಮ್ಮ (ಮಿಷನರಿಗಳ) ವಿರುದ್ಧವೇ ಆಡಿದಂತಹ ಯಾವುದೇ ಮಾತಿನಿಂದ ಅಮೆರಿಕನ್ನರಿಗೆ ನಿಜಕ್ಕೂ ಸಂತೋಷವೇ ಆಗುತ್ತದೆಯೆಂದು ಅವರಿಗೆ ಗೊತ್ತು! ಪ್ರಿಯ ಮೇರಿ, ಒಂದು ವೇಳೆ ನಾನು ಯಾಂಕಿಗಳ (ಅಮೆರಿಕನ್ನರ) ವಿರುದ್ಧ ಏನೇನು ಸಾಧ್ಯವೋ ಆ ಎಲ್ಲ ಟೀಕೆಗಳನ್ನೂ ಮಾಡಿದೆನೆಂದೇ ಇಟ್ಟುಕೊ–ಅದರಿಂದ, ಅಮೆರಿಕನ್ನರು ನಮ್ಮ ತಾಯಂದಿರ ಹಾಗೂ ಸೋದರಿಯರ ಮೇಲೆ ಆಡುವುದರ ಲಕ್ಷದಲ್ಲೊಂದು ಪಾಲನ್ನಾದರೂ ತೀರಿಸಿದಂತಾದೀತೇನು? ಯಾಂಕಿಗಳು ಟೀಕೆ ಗಳನ್ನು ತಾಳ್ಮೆಯಿಂದ ಸಹಿಸಿಕೊಳ್ಳುವುದನ್ನು ಕಲಿತುಕೊಂಡು, ಆಮೇಲೆ ಇತರರನ್ನು ಟೀಕಿಸಲಿ. ಇತರರನ್ನು ಟೀಕಿಸಲು ಸದಾ ಸಿದ್ಧರಾಗಿರುವವರು ಇತರರ ಅತ್ಯಲ್ಪ ಟೀಕೆಯನ್ನೂ ಸಹಿಸಲಾರರು ಎಂಬುದು ಎಲ್ಲರಿಗೂ ಗೊತ್ತಿರುವ ಮನಶ್ಶಾಸ್ತ್ರೀಯ ಸತ್ಯ. ಅಲ್ಲದೆ ನನಗೆ ಅವರ ಹಂಗಾದರೂ ಏನು? ನಿಮ್ಮ (ಮೇರಿಯ) ಕುಟುಂಬದವರು, ಶ್ರೀಮತಿ ಬುಲ್, ಲೆಗೆಟ್ ಕುಟುಂಬದವರು ಮತ್ತು ಇನ್ನು ಕೆಲವು ಉದಾರಿಗಳನ್ನು ಬಿಟ್ಟರೆ ನನ್ನ ವಿಷಯದಲ್ಲಿ ಇನ್ನಾರು ತಾನೆ ವಿಶ್ವಾಸ ತೋರಿದರು? ಅಮೆರಿಕನ್ನರು ಹೆಚ್ಚು ವಿಶಾಲಹೃದಯರಾಗಲು ಮತ್ತು ಹೆಚ್ಚು ಅಧ್ಯಾತ್ಮಶೀಲ ರಾಗಲು ಕಲಿಯುವಂತಾಗಲಿ ಎಂದು ನಾನು ಹೆಚ್ಚುಕಡಿಮೆ ನನ್ನ ಕೊನೆಯುಸಿರಿನವರೆಗೂ ದುಡಿದಿದ್ದೇನೆ... 

ಪ್ರಿಯ, ಪ್ರಿಯ ಮೇರಿ ನನಗಾಗಿ ನೀನು ಭಯ ಪಟ್ಟುಕೊಳ್ಳಬೇಡ. ಈ ಪ್ರಪಂಚ ದೊಡ್ಡದು, ಬಹಳ ದೊಡ್ಡದು. ‘ಯಾಂಕಿ’ಗಳು ಎಷ್ಟೇ ರೊಚ್ಚಿಗೆದ್ದರೂ ಈ ಪ್ರಪಂಚದಲ್ಲಿ ನನಗೊಂದಿಷ್ಟು ಜಾಗವಿರಲೇಬೇಕು. ಅದೇನೇ ಇರಲಿ, ನನಗಂತೂ ನನ್ನ ಕಾರ್ಯದಿಂದ ತುಂಬ ತೃಪ್ತಿಯಾಗಿದೆ. ನಾನೆಂದಿಗೂ ಯಾವುದನ್ನೂ ಲೆಕ್ಕಾಚಾರ ಹಾಕಿ ಮಾಡಲಿಲ್ಲ. ನಾನು ಎಲ್ಲವನ್ನೂ ಅವು ಬಂದಂತೆಯೇ ಸ್ವೀಕರಿಸಿದ್ದೇನೆ. ಆದರೆ ನನ್ನ ತಲೆಯಲ್ಲಿ ಕೊರೆಯುತ್ತಿದ್ದ ಒಂದೇ ಒಂದು ಆಲೋಚನೆಯೆಂದರೆ, ಭಾರತದ ಜನಸಮೂಹವನ್ನು ಮೇಲೆತ್ತಲು ಸಂಸ್ಥೆಯೊಂದನ್ನು ಪ್ರಾರಂಭಿ ಸುವುದು; ಆ ಕಾರ್ಯದಲ್ಲಿ ನಾನು ಸ್ವಲ್ಪ ಮಟ್ಟಿಗೆ ಯಶಸ್ವಿಯಾಗಿದ್ದೇನೆ. ನನ್ನ ಶಿಷ್ಯರು ಕ್ಷಾಮ- ರೋಗ-ಸಂಕಟಗಳ ನಡುವೆ ಕೆಲಸ ಮಾಡುವುದನ್ನು, ಕಾಲರಾ ಪೀಡಿತ ಹೊಲೆಯನ ಹಾಸಿಗೆಯ ಬದಿಯಲ್ಲಿ ಕುಳಿತು ಶುಶ್ರೂಷೆ ಮಾಡುವುದನ್ನು, ಹಸಿದ ಚಂಡಾಲನಿಗೆ ಉಣಬಡಿಸುವುದನ್ನು ನೋಡಿದ್ದರೆ ನಿನ್ನ ಹೃದಯವೂ ಸಂತಸಗೊಳ್ಳುತ್ತಿತ್ತು. ಭಗವಂತ ನಮ್ಮೆಲ್ಲರಿಗೂ ನೆರವಾಗು ತ್ತಾನೆ. ಭಗವಂತ, ನನ್ನೊಲವಿನ ಭಗವಂತ ನನ್ನೊಂದಿಗಿದ್ದಾನೆ... ಮಿಷನರಿಗಳು ಮಾತನಾಡಿ ಕೊಳ್ಳುವುದನ್ನು ಲಕ್ಷಿಸುತ್ತೇನೆಯೆ? ಅವರೆಲ್ಲ ಮಕ್ಕಳು! ಮಕ್ಕಳಿಗಿಂತ ಹೆಚ್ಚಿಗೆಯೇನೂ ಅರಿಯರು ಅವರು. ಏನು! ಪರಬ್ರಹ್ಮವನ್ನೇ ಸಾಕ್ಷಾತ್ಕರಿಸಿಕೊಂಡಿರುವ ನಾನು, ಪ್ರಾಪಂಚಿಕತೆಯ ಪೊಳ್ಳುತನವನ್ನು ಮನಗಂಡಿರುವ ನಾನು, ಈ ಬಾಲರ ಬಡಬಡಿಕೆಗೆ ಬೆದರಿ ಪಥಭ್ರಷ್ಟನಾಗುವುದೆ! ನನ್ನನ್ನು ನೋಡಿದರೆ ಹಾಗನ್ನಿಸುತ್ತದೆಯೇ?

ನನ್ನ ಬಗ್ಗೆ ನಾನು ಇಷ್ಟೊಂದನ್ನೆಲ್ಲ ಏಕೆ ಹೇಳಿದೆನೆಂದರೆ, ನನ್ನಿಂದ ನಿನಗದು ಬಾಕಿಯಿತ್ತು. ನನ್ನ ಕಾರ್ಯ ಈಗ ಮುಗಿದಿದೆಯೆಂದು ಭಾವಿಸುತ್ತೇನೆ–ನನ್ನ ಜೀವನದಲ್ಲಿ ಹೆಚ್ಚೆಂದರೆ ಇನ್ನು ಮೂರೋ ನಾಲ್ಕೋ ವರ್ಷಗಳು ಉಳಿದಿವೆ. ನನ್ನ ಮುಕ್ತಿಯ ಬಯಕೆಯೆಲ್ಲ ನನ್ನಿಂದ ದೂರ ವಾಗಿದೆ. ನಾನೆಂದಿಗೂ ಪ್ರಾಪಂಚಿಕ ಭೋಗಗಳನ್ನು ಬಯಸಲಿಲ್ಲ. ನಾನು ಸ್ಥಾಪಿಸಿದ ಈ ಯಂತ್ರವು (ಸಂಸ್ಥೆಯು) ಬಲವಾದ ಸ್ಥಿತಿಗೆ ಬರುವುದನ್ನು ಮಾತ್ರ ನಾನು ನೋಡಬೇಕು; ಬಳಿಕ ಮಾನವತೆಯ ಒಳಿತಿಗಾಗಿ–ಕನಿಷ್ಠ ಪಕ್ಷ ಭಾರತದಲ್ಲಾದರೂ–ಯಾವ ಶಕ್ತಿಯೂ ಹಿಂದಕ್ಕೆ ತಳ್ಳಲಾಗದಂತಹ ಯಂತ್ರವೊಂದನ್ನು ಕಾರ್ಯದಲ್ಲಿ ತೊಡಗಿಸಿದ್ದೇನೆಂದರಿತು ನಾನು ಕಣ್ಣು ಮುಚ್ಚುತ್ತೇನೆ, ಮುಂದೇನಾಗುತ್ತದೆ ಎಂಬುದನ್ನು ಲೆಕ್ಕಿಸದೆ. ಅಸ್ತಿತ್ವದಲ್ಲಿರುವ ಏಕೈಕ ಭಗವಂತ ನನ್ನು–ನಾನು ನಂಬುವ ಏಕೈಕ ಭಗವಂತನನ್ನು–ಎಲ್ಲ ಜೀವಿಗಳ ಸಮಷ್ಟಿ ಸ್ವರೂಪವನ್ನು ಪೂಜಿಸುವಂತಾಗಲು ನಾನು ಮತ್ತೆಮತ್ತೆ ಜನಿಸಿಬರುವಂತಾಗಲಿ! ಜನಿಸಿ ಸಹಸ್ರ ಸಹಸ್ರ ಕಷ್ಟ ಗಳನ್ನು ಅನುಭವಿಸುವಂತಾಗಲಿ! ಮತ್ತು ಎಲ್ಲಕ್ಕಿಂತ ಹೆಚ್ಚಾಗಿ ದುಷ್ಟರು ನನ್ನ ದೇವರು, ದರಿದ್ರರು ನನ್ನ ದೇವರು, ಎಲ್ಲ ಜನಾಂಗಗಳಲ್ಲೂ ಎಲ್ಲ ವರ್ಗಗಳಲ್ಲೂ ಕಡು ಬಡವರಾದವರು ನನ್ನ ದೇವರು–ಅವರೇ ನನ್ನ ಆರಾಧನೆಯ ವಸ್ತು...

ನನ್ನ ಕಾಲಾವಧಿ ಅಲ್ಪ, ನನ್ನ ಮನಸ್ಸಿನಲ್ಲಿರುವುದನ್ನೆಲ್ಲ ತೋಡಿಕೊಂಡು ನನ್ನೆದೆಯನ್ನು ಹಗುರ ಮಾಡಿಕೊಳ್ಳಬೇಕಾಗಿದೆ. ಅದರಿಂದ ಕೆಲವರಿಗೆ ಹಿತವಾಗಬಹುದು, ಕೆಲವರಿಗೆ ನೋವಾಗಬಹುದು, ನಾನದನ್ನು ಲೆಕ್ಕಿಸುವಂತಿಲ್ಲ. ಆದ್ದರಿಂದ ಪ್ರಿಯ ಮೇರಿ, ನನ್ನ ಬಾಯಿಂದ ಹೊರಬೀಳುವ ಮಾತುಗಳನ್ನು ಕೇಳಿ ದಿಗಿಲುಗೊಳ್ಳಬೇಡ. ಏಕೆಂದರೆ ನನ್ನ ಹಿಂದಿರುವ ಶಕ್ತಿ ವಿವೇಕಾನಂದನಲ್ಲ, ಸ್ವಯಂ ಭಗವಂತ. ಅವನಿಗೆಲ್ಲ ತಿಳಿದಿದೆ. ನಾನು ಜಗತ್ತನ್ನು ಖುಷಿಗೊಳಿಸಲು ಹೊರಟರೆ ಅದರಿಂದ ಜಗತ್ತಿಗೆ ಹಾನಿಯೇ ಹೆಚ್ಚು. ಬಹುಸಂಖ್ಯಾತರೆಂಬುವರ ಅಭಿಪ್ರಾಯ ತಪ್ಪಾಗಿರಲೇ ಬೇಕು; ಏಕೆಂದರೆ ಈ ಜಗತ್ತನ್ನು ಆಳುವವರು ಅವರೇ ಮತ್ತು ಜಗತ್ತಿನ ದುಃಸ್ಥಿತಿಗೆ ಕಾರಣರೂ ಅವರೇ. ಪ್ರತಿಯೊಂದು ಹೊಸ ಆಲೋಚನೆಯೂ ಕೂಡ ವಿರೋಧವನ್ನು ಎದುರಿಸಲೇಬೇಕು– ನಾಗರಿಕ ಸಮಾಜದಲ್ಲಿ ನಯವಾದ ತಿರಸ್ಕಾರಗಳನ್ನು ಎದುರಿಸಬೇಕಾಗಿ ಬಂದರೆ ಅಸಭ್ಯರ ಸಮಾಜದಲ್ಲಿ ಒರಟು ಬೈಗುಳಗಳನ್ನೂ, ಅಸಹ್ಯಕರ ಅಪಪ್ರಚಾರವನ್ನೂ ಎದುರಿಸಬೇಕಾಗುತ್ತದೆ.

ಆದರೆ ಈ ಕ್ಷುದ್ರಜಂತುಗಳೂ ಕೂಡ ನೆಟ್ಟಗೆ ನಿಲ್ಲಲೇಬೇಕು; ಬಾಲಬುದ್ಧಿಯವರೂ ಕೂಡ ಬೆಳಕನ್ನು ಕಾಣಲೇಬೇಕು... ನಮ್ಮ ದೇಶದ ಮೇಲೆ ವೈಭವದ ಅಲೆಗಳು ನೂರಾರು ಬಾರಿ ಹಾದು ಹೋಗಿವೆ. ಆದ್ದರಿಂದ ನಾವು ಭಾರತೀಯರು, ಯಾವ ಬಾಲಬುದ್ಧಿಯವರೂ ಇದುವರೆಗೆ ಅರಿತುಕೊಳ್ಳದಿರುವ ಪಾಠವನ್ನು ಕಲಿತಿದ್ದೇವೆ. ಅದೇ ಮಾಯೆ ಎಂಬುದು. ಈ ವಿಕೃತ ಜಗತ್ತೇ ಮಾಯೆ. ತ್ಯಾಗಮಾಡಿ ಸುಖಿಯಾಗು. ಕಾಮಕಾಂಚನಗಳ ಭಾವನೆಯನ್ನು ತೊರೆದುಬಿಡು. ಅದ ಕ್ಕಿಂತ ಬೇರೆ ಬಂಧನವಿಲ್ಲ. ವಿವಾಹ, ಕಾಮ ಮತ್ತು ಹಣ–ಇವು ಮಾತ್ರವೇ ಜೀವಂತ ಪಿಶಾಚಿ ಗಳು. ಪ್ರಾಪಂಚಿಕ ಪ್ರೇಮವೆಲ್ಲವೂ ಶರೀರಾಸಕ್ತಿಯಿಂದಲೇ ಪ್ರಾರಂಭ. ಕಾಮವಿಲ್ಲದಿದ್ದರೆ ಪರಿಗ್ರಹವೂ ಇಲ್ಲ. ಇವು ಕಳಚಿ ಬಿದ್ದಂತೆಲ್ಲ ಕಣ್ಣುಗಳು ಆಧ್ಯಾತ್ಮಿಕ ದರ್ಶನಕ್ಕೆ ತರೆದುಕೊಳ್ಳು ತ್ತವೆ. ಜೀವವು ತನ್ನ ಅನಂತ ಶಕ್ತಿಯನ್ನು ಮರಳಿ ಪಡೆದುಕೊಳ್ಳುತ್ತದೆ.

ಪ್ರೀತಿಪೂರ್ವಕವಾಗಿ

\begin{flushright}
ಎಂದೆಂದಿಗೂ ನಿನ್ನವ\\ವಿವೇಕಾನಂದ
\end{flushright}

ಕೆಲವು ಮತಾಂಧ ‘ಚರ್ಚೀ’ ಮಹಿಳೆಯರ ವಿರುದ್ಧ ಸ್ವಾಮೀಜಿ ಮಾಡಿದರೆನ್ನಲಾದ ಟೀಕೆಗಳು ಎಷ್ಟೇ ಕಟುವಾಗಿದ್ದಿರಲಿ, ಅಮೆರಿಕದ ಮಹಿಳೆಯರನ್ನೆಲ್ಲ ಅವರು ಒಟ್ಟಾರೆ ಟೀಕಿಸಿರಲು ಸಾಧ್ಯವೇ ಇರಲಿಲ್ಲ. ಬದಲಾಗಿ ಅವರು ಅಮೆರಿಕದ ಮಹಿಳೆಯರ ಬಗ್ಗೆ ಅತಿ ಹೆಚ್ಚಿನ ಆದರ-ಅಭಿಮಾನ- ಗೌರವಗಳನ್ನು ಹೊಂದಿದ್ದರು. ಅಮೆರಿಕದಿಂದ ಅವರು ಖೇತ್ರಿಯ ಮಹಾರಾಜನಿಗೆ ಬರೆದಿದ್ದ ಒಂದು ಪತ್ರದಿಂದ ಇದು ಅತ್ಯಂತ ಸ್ಪಷ್ಟವಾಗುತ್ತದೆ. ‘ಗೃಹಿಣೀ ಗೃಹಮುಚ್ಯತೇ’ ಎಂಬ ಮಾತನ್ನು ಉದ್ಧರಿಸುತ್ತ ಆ ಪತ್ರದಲ್ಲಿ ಹೇಳುತ್ತಾರೆ, “ಈ ದೃಷ್ಟಿಯಿಂದ ಅಮೆರಿಕದ ಮಹಿಳೆ ಯರು ಜಗತ್ತಿನ ಯಾವುದೇ ಜನಾಂಗದ ಮಹಿಳೆಯರೊಂದಿಗೆ ಸರಿದೂಗಬಲ್ಲರು” ಎಂದು. ಅಮೆರಿಕದ ಮಹಿಳೆಯರು ವಿಲಾಸಪ್ರಿಯರು, ಸ್ವೇಚ್ಛಾಚಾರಿಗಳು ಎಂಬ ಪ್ರಚಲಿತ ಭಾವನೆಯನ್ನು ಸ್ವಾಮೀಜಿ ಸಂಪೂರ್ಣ ತಳ್ಳಿಹಾಕುತ್ತಾರೆ. ಆ ಪತ್ರದಲ್ಲಿ ಅವರು ಉದ್ಗರಿಸುತ್ತಾರೆ, “ಅಮೆರಿಕದ ಮಹಿಳೆಯರೇ, ನಿಮ್ಮ ಪುಣವನ್ನು ತೀರಿಸಲು ನನಗೆ ನೂರು ಜನ್ಮಗಳೂ ಸಾಲಲಾರವು! ನಿಮಗೆ ಧನ್ಯವಾದಗಳನ್ನು ತಿಳಿಸಲು ನನ್ನಲ್ಲಿ ಶಬ್ದಗಳೇ ಇಲ್ಲ!” ದೂರದ ಭಾರತದಿಂದ ತಾವು ಅನಾಮಧೇಯರಾಗಿ ಕೈಯಲ್ಲಿ ಕಾಸಿಲ್ಲದೆ ಅಸಹಾಯಕರಾಗಿ ಬಂದಾಗ ಅಮೆರಿಕದ ಮಹಿಳೆಯರು ತಮ್ಮನ್ನು ಸ್ವಂತ ಮಗನಂತೆ, ಸೋದರನಂತೆ ಕಂಡು ವಿಶ್ವಾಸದಿಂದ ಉಪಚರಿಸಿದುದುನ್ನು ಈ ಪತ್ರದಲ್ಲಿ ಸ್ವಾಮೀಜಿ ಕೃತಜ್ಞತೆಯಿಂದ ಸ್ಮರಿಸುತ್ತಾರೆ.

ಅಲ್ಲದೆ ಸ್ವಾಮೀಜಿಯವರ ವಿರುದ್ಧದ ಈ ಅಪಪ್ರಚಾರದ ಸಮಯದಲ್ಲಿ ಶ್ರೀಮತಿ ಸಾರಾ ಬುಲ್ಲಳಿಗೆ ಸ್ವಾಮೀಜಿಯವರ ಸಂಗಾತಿಯಾದ ಗುಡ್​ವಿನ್ ಒಂದು ಪತ್ರ ಬರೆದ. ಅಮೆರಿಕದ ಮಹಿಳೆಯರ ವಿಷಯದಲ್ಲಿ ಸ್ವಾಮೀಜಿ ಹೊಂದಿದ್ದ ಗೌರವವನ್ನು ಈ ಪತ್ರ ಎತ್ತಿತೋರಿಸು ವಂತಿದೆ. ಈ ಪತ್ರ ಬರೆಯುವುದಕ್ಕೆ ಒಂದೆರಡು ವಾರಗಳ ಹಿಂದೆಯಷ್ಟೇ ಸ್ವಾಮೀಜಿ ಗುಡ್​ವಿನ್ನ ನಿಗೆ ಹೇಳಿದ್ದರಂತೆ, “ನಾನು ಮುಂದೇನಾದರೂ ಹೆಣ್ಣಾಗಿ ಜನ್ಮ ತಾಳಬೇಕಾಗಿ ಬಂದರೆ ಆಗ ನಾನು ಖಂಡಿತ ಅಮೆರಿಕದ ಮಹಿಳೆಯಾಗಿ ಜನ್ಮ ತಾಳುತ್ತೇನೆ” ಎಂದು. ಆದ್ದರಿಂದ ಸ್ವಾಮೀಜಿ ಅಮೆರಿಕದ ಮಹಿಳೆಯರ ವಿಷಯದಲ್ಲಿ ಹೀನಾಯವಾಗಿ ಮಾತನಾಡಿದರೆಂಬ ಪತ್ರಿಕಾ ವರದಿ ಗಳೆಲ್ಲವೂ ಅವರ ವ್ಯಕ್ತಿತ್ವಕ್ಕೆ ಮಸಿಬಳಿಯಲೆತ್ನಿಸುವ ಕುಟಿಲೋಪಾಯಗಳಷ್ಟೇ ಎಂದು ಗುಡ್ ವಿನ್ ದೃಢವಾಗಿ ಹೇಳುತ್ತಾನೆ.

ಸ್ವಾಮೀಜಿ ಸಿಕ್ಕಿಹಾಕಿಕೊಂಡ ಕಹಿ-ವಿವಾದಾಸ್ಪದ ಘಟನೆ ಇದೊಂದೇ ಅಲ್ಲ. ಭಾರತದ, ಮುಖ್ಯವಾಗಿ ಕಲ್ಕತ್ತ ಮದ್ರಾಸುಗಳ ಪತ್ರಿಕೆಗಳಲ್ಲಿ ಪ್ರಧಾನವಾಗಿ ಕಾಣಿಸಿಕೊಂಡ ಇನ್ನಿತರ ಸಂಗತಿಗಳಿವೆ. ಅವುಗಳನ್ನೀಗ ನೋಡೋಣ.

ಕಲ್ಕತ್ತದ ‘ವಂಗವಾಸಿ’ ಎಂಬುದು ಎಂದಿನಿಂದಲೂ ಸ್ವಾಮೀಜಿಯವರಿಗೆ ವಿರುದ್ಧವಾಗಿಯೇ ಕೆಲಸ ಮಾಡುತ್ತಿದ್ದ ಪತ್ರಿಕೆ. ಇದು ಅಂದಿನ ಅತ್ಯಂತ ಪ್ರಭಾವಶಾಲೀ ಪತ್ರಿಕೆಗಳಲ್ಲೊಂದು. ವಿವೇಕಾನಂದರು ಅಮೆರಿಕದಲ್ಲಿ ಪಡೆದ ಯಶಸ್ಸಿಗಾಗಿ ಕಲ್ಕತ್ತದಲ್ಲೊಂದು ಅಭಿನಂದನಾ ಸಮಾ ರಂಭ ನಡೆದಾಗ ಅದನ್ನು ತಪ್ಪಿಸಲು ಈ ಪತ್ರಿಕೆ ಶಕ್ತಿ ಮೀರಿ ಪ್ರಯತ್ನಿಸಿತ್ತು. ಬಳಿಕ ಸ್ವಾಮೀಜಿ ಅಮೆರಿಕದಿಂದ ಹಿಂದಿರುಗಿ ಬಂದಾಗ ಅವರ ವಿರುದ್ಧ ಸಾಧ್ಯವಾದ ಎಲ್ಲ ಬಗೆಯ ಅಪಪ್ರಚಾರ ವನ್ನೂ ಮಾಡಿತು. ಬಹುಶಃ ಇದು ಹಿಂದೆ ಅಮೆರಿಕದ ಪತ್ರಿಕೆಗಳಲ್ಲಿ ಕೆಲವೊಮ್ಮೆ ಅವರ ವಿರುದ್ಧ ನಡೆದಿದ್ದ ಅಪಪ್ರಚಾರಕ್ಕಿಂತಲೂ ಭಯಂಕರವಾಗಿತ್ತು. ಹೇಯವಾಗಿತ್ತು. ಏಕೆಂದರೆ ಇದು ಅವರ ಸ್ವದೇಶ. ಇಲ್ಲಿನ ಲಕ್ಷಾಂತರ ಜನ ಅವರನ್ನು ಗೌರವಿಸಿ ಪೂಜಿಸುತ್ತಿರುವಾಗ ಒಂದು ಪ್ರಭಾವ ಶಾಲೀ ಪತ್ರಿಕೆ ಹೀಗೆ ಅವಹೇಳನ ಮಾಡಿದರೆ ಅದರ ಆಘಾತ ಹೇಗಿರುತ್ತದೆಂಬುದನ್ನು ಊಹಿ ಸಿಯೇ ನೋಡಬೇಕು.

ಈ ದುರುದ್ದೇಶಪೂರಿತ ಪ್ರಚಾರಗಳ ಹಿಂದಿದ್ದ ವ್ಯಕ್ತಿ ಪಾಂಚ್​ಕೋರಿ ವಂದ್ಯೋಪಾಧ್ಯಾಯ (ಬ್ಯಾನರ್ಜಿ). ಈತ ‘ವಂಗವಾಸಿ’ಯ ಸಂಪಾದಕ. ಆ ಕಾಲದ ಅತ್ಯಂತ ಪ್ರತಿಭಾನ್ವಿತ, ಪ್ರಭಾವ ಶಾಲಿ ಪತ್ರಿಕೋದ್ಯಮಿ. ಈ ವಂಗವಾಸಿ ಪತ್ರಿಕೆ ಮೇಲ್ಜಾತಿಯವರ ಆಡಳಿತಕ್ಕೊಳಪಟ್ಟಿತ್ತು. ಇದು ಸ್ವಾಮೀಜಿಯವರ ವಿರುದ್ಧ ಕೆಂಡ ಕಾರಿದ್ದಕ್ಕೆ ಮುಖ್ಯ ಕಾರಣವೂ ಅದೇ–ಅವರು ಬ್ರಾಹ್ಮಣ ರಲ್ಲವೆಂಬುದು. (ಬ್ಯಾನರ್ಜಿ ಎಂಬುದೂ ಒಂದು ಬ್ರಾಹ್ಮಣ ಮನೆತನ.) ಪಾಂಚ್​ಕೋರಿಯು ವಿವೇಕಾನಂದರ ಬಾಲ್ಯಸ್ನೇಹಿತ. ತರುಣ ನರೇಂದ್ರನ ಪ್ರಚಂಡ ಬುದ್ಧಿಮತ್ತೆ, ಧೈರ್ಯ ಹಾಗೂ ಆಯಸ್ಕಾಂತೀಯ ವ್ಯಕ್ತಿತ್ವಕ್ಕೆ ಈತ ಸಂಪೂರ್ಣ ಮನಸೋತಿದ್ದ. ಆದರೆ ತಮ್ಮ ಮಧ್ಯದಲ್ಲೇ ಬೆಳೆದ ಒಬ್ಬ ‘ಹುಡುಗ’ ಜಗದ್ವಿಖ್ಯಾತನಾದಾಗ, ಕಲ್ಕತ್ತದ ಅನೇಕರಿಗೆ ಅದೊಂದು ನುಂಗಲಾರದ ತುತ್ತಾಯಿತು. ಅಂಥವರೆಲ್ಲರ ಪ್ರತಿನಿಧಿಯೇ ಪಾಂಚ್​ಕೋರಿ ಎನ್ನಬಹುದು. ಸ್ವಾಮೀಜಿ ಕಲ್ಕತ್ತಕ್ಕೆ ಮರಳಿದಾಗ, ಫೆಬ್ರುವರಿ ೨೮ರಂದು ನಡೆದ ಭಾರೀ ಸಮಾರಂಭದ ವಿಷಯವನ್ನು ಹಿಂದೆ ನೋಡಿ ದ್ದೇವೆ. ಅದರ ಹಿಂದಿನ ದಿನ ‘ವಂಗವಾಸಿ’ ವಿವೇಕಾನಂದರನ್ನು ಟೀಕಿಸಿ ಬರೆದ ಕುಹಕ ಪೂರಿತ ಸಂಪಾದಕೀಯದ ಒಂದಂಶ ಹೀಗಿತ್ತು:

“ವಿವೇಕಾನಂದರನ್ನು ಹಿಂದೂಧರ್ಮದ ಸಂರಕ್ಷಕರು, ಒಬ್ಬ ಸಂನ್ಯಾಸಿ, ಸ್ವಾಮಿ, ಯೋಗಿ, ಪರಮಹಂಸ–ಎಂಬಿತ್ಯಾದಿಯಾಗಿ ಬಣ್ಣಿಸುವುದನ್ನು ನಾವು ಬಲವಾಗಿ ವಿರೋಧಿಸುತ್ತೇವೆ. ಆದರೆ ಅವರನ್ನು ಈಗ ಬಾಬು ನರೇಂದ್ರನಾಥ ಎಂಬ ಅವರ ಹಿಂದಿನ ಹೆಸರಿನಿಂದ ಗುರುತಿಸುವುದಾದರೆ, ಅವರನ್ನು ನಾವು ಹಾರ್ದಿಕವಾಗಿ ಸ್ವಾಗತಿಸುತ್ತೇವೆ.”

ಈ ಸಂಪಾದಕೀಯದಿಂದ ಸಾರ್ವಜನಿಕ ಸಭೆಯನ್ನೇನೂ ನಿಲ್ಲಿಸಲು ಸಾಧ್ಯವಾಗಲಿಲ್ಲ. ಆದರೆ ಅದು ಮತ್ತಷ್ಟು ಜನ ಸ್ವಾಮೀಜಿಯವರಿಗೆ ವಿರುದ್ಧವಾಗಲು ಕಾರಣವಾಯಿತು. ಅಂದಿನ ಸಮಾ ರಂಭದ ಅಧ್ಯಕ್ಷತೆ ವಹಿಸಲು ಮೊದಲಿಗೆ ಒಪ್ಪಿದ್ದ ದರ್ಭಾಂಗದ ಮಹಾರಾಜ, ಅನಂತರ ತನಗೆ ಬೇರೊಂದು ಮುಖ್ಯ ಕಾರ್ಯವಿರುವುದಾಗಿ ಹೇಳಿ ತಪ್ಪಿಸಿಕೊಂಡುಬಿಟ್ಟ. ಆಗ ರಾಜಾ ವಿನಯಕೃಷ್ಣ ದೇವ್ ಬಹಾದ್ದೂರರನ್ನು ಅಧ್ಯಕ್ಷರನ್ನಾಗಿ ಗೊತ್ತುಪಡಿಸಲಾಯಿತು. ಅಂತೂ ಅಂದಿನ ಕಾರ್ಯ ಕ್ರಮ ಯಶಸ್ವಿಯಾಗಿ ನೆರವೇರಿತು.

ಆದರೆ ‘ವಂಗವಾಸಿ’ಯ ಆಕ್ರಮಣ ಮುಂದುವರಿಯಿತು. ಮಾರ್ಚ್ ೨೧ರಂದು ಸ್ವಾಮೀಜಿ ಯವರು ಖೇತ್ರಿಯ ಮಹಾರಾಜ ಅಜಿತ್​ಸಿಂಗನೊಂದಿಗೆ ದಕ್ಷಿಣೇಶ್ವರದ ಕಾಳೀ ದೇವಾಲಯಕ್ಕೆ ಭೇಟಿಯಿತ್ತಾಗ ಅಲ್ಲೊಂದು ‘ದುರ್ಘಟನೆ’ ನಡೆಯಿತು. ಆದರೆ ಆಗ ಅದು ಯಾರ ಗಮನಕ್ಕೂ ಬರಲಿಲ್ಲ. ಕಾಳೀ ದೇವಾಲಯದ ಸ್ಥಾಪಕಿಯಾದ ರಾಣೀ ರಾಸಮಣಿಯ ಮೊಮ್ಮಗನೂ ದೇವ ಸ್ಥಾನದ ಮೇಲ್ವಿಚಾರಕನೂ ಆದ ತ್ರೈಲೋಕ್ಯನಾಥ ವಿಶ್ವಾಸನೇ ಈ ಇಬ್ಬರು ಗಣ್ಯವ್ಯಕ್ತಿಗಳನ್ನು ಬರಮಾಡಿಕೊಳ್ಳುವ ವ್ಯವಸ್ಥೆಯಾಗಿತ್ತು. ಆದರೆ ಮಹಾರಾಜನೊಂದಿಗೆ ಸ್ವಾಮೀಜಿ ದಕ್ಷಿಣೇಶ್ವರಕ್ಕೆ ಆಗಮಿಸಿದಾಗ ತ್ರೈಲೋಕ್ಯನಾಥ ತನಗೆ ಬಹಳ ಅನಾರೋಗ್ಯವಾಗಿದೆಯೆಂದೂ ಆದ್ದರಿಂದ ತನ್ನನ್ನು ಕ್ಷಮಿಸಬೇಕೆಂದೂ ಹೇಳಿ, ದೇವಸ್ಥಾನದ ಖಜಾಂಚಿಯಾದ ಭೋಲಾನಾಥ ಮುಖರ್ಜಿಯನ್ನು ಕಳಿಸಿಕೊಟ್ಟ. ಭೋಲಾನಾಥ ತನ್ನ ಪಾಲಿನ ಕರ್ತವ್ಯವನ್ನು ಯಶಸ್ವಿಯಾಗಿ ನಿರ್ವಹಿಸಿ, ಎಲ್ಲರ ಮೆಚ್ಚಿಗೆ ಗಳಿಸಿದ. ತ್ರೈಲೋಕ್ಯನಾಥನ ಗೈರುಹಾಜರಿಯನ್ನು ಯಾರೂ ತಪ್ಪಾಗಿ ಭಾವಿಸಲಿಲ್ಲ.

ಆದರೆ ಮರುದಿನ ಈ ಘಟನೆಯ ಬಗ್ಗೆ ‘ವಂಗವಾಸಿ’ಯು ವರದಿಯೊಂದನ್ನು ಪ್ರಕಟಿಸಿತು. ಈ ವರದಿಯಲ್ಲಿ, ಉದ್ದೇಶಪೂರ್ವಕವಾಗಿಯೇ ತ್ರೈಲೋಕ್ಯಬಾಬು ವಿವೇಕಾನಂದರನ್ನು ಎದುರ್ ಗೊಳ್ಳಲು ಬರಲಿಲ್ಲವೆಂದು ವಿವರಿಸಲಾಗಿತ್ತು. ಇದು ಸುಳ್ಳೆಂದು ಕೆಲವು ಪ್ರತ್ಯಕ್ಷದರ್ಶಿಗಳು ಪ್ರತಿಭಟಿಸಿದರು. ಆಗ ಈ ಪತ್ರಿಕೆ, ತನ್ನ ವರದಿಯನ್ನು ಸಮರ್ಥಿಸಲು ತ್ರೈಲೋಕ್ಯನಿಂದಲೇ ಒಂದು ಹೇಳಿಕೆಯನ್ನು ಪಡೆದುಕೊಂಡು ಪ್ರಕಟಿಸಿತು! ತನ್ನ ಹೇಳಿಕೆಯಲ್ಲಿ ಆತ ಹೀಗೆ ತಿಳಿಸಿದ್ದ: “... ನಿಮ್ಮ ಪತ್ರಿಕೆಯ ವರದಿ ಸಂಪೂರ್ಣ ಸತ್ಯ... ಪಾಶ್ಚಾತ್ಯ ದೇಶಕ್ಕೆ ಹೋಗಿಯೂ ತನ್ನನ್ನು ಹಿಂದುವೆಂದು ಕರೆದುಕೊಳ್ಳುವ ಒಬ್ಬ ವ್ಯಕ್ತಿಯೊಂದಿಗೆ ಯಾವುದೇ ಸಂಬಂಧವನ್ನಿಟ್ಟುಕೊಳ್ಳ ಬಾರದೆಂದು ನಾನು ಭಾವಿಸಿದೆ...” ಇವೆಲ್ಲಕ್ಕಿಂತ ತುಚ್ಛವಾದ, ಅಪಮಾನಕರವಾದ ಮತ್ತೊಂದು ಕೃತ್ಯವೂ ನಡೆದಿದ್ದುದು ಆಗಲೇ ಪ್ರಕಟವಾಯಿತು–‘ಶೂದ್ರ ವಿವೇಕಾನಂದ’ರ ಪ್ರವೇಶದಿಂದಾಗಿ ‘ಅಪವಿತ್ರ’ಗೊಂಡಿದ್ದ ಕಾಳಿಯ ವಿಗ್ರಹವನ್ನು ಪುನಃ ಅಭಿಷೇಕ ಮಾಡಿ ತೊಳೆದು ಪರಿಶುದ್ಧ ಗೊಳಿಸಲಾಗಿತ್ತು!!

ಈ ಸುದ್ದಿಯನ್ನು ಕೇಳಿದ ಜನ ನಿಬ್ಬೆರಗಾದರು. ‘ರಾಣಿ ರಾಸಮಣಿಯ ವಂಶಸ್ಥನೊಬ್ಬ ಸ್ವಯಂ ಶ್ರೀರಾಮಕೃಷ್ಣರ ಮಹಾಶಿಷ್ಯನಿಗೆ ಹೀಗೆ ಅಪಮಾನ ಮಾಡಬಹುದೆ! ಏನಿದು ದುರಾ ಚಾರ!’ ಎಂದು ಮಾತನಾಡಿಕೊಂಡರು. ರಾಸಮಣಿಯ ವಂಶಸ್ಥರೇ ಆದ ಇತರರೂ ಈ ದುರ್ವರ್ತನೆಯನ್ನು ಖಂಡಿಸಿದರು. ಏನಾದರೇನಂತೆ? ಆಗಬಾರದ್ದು ಆಗಿಹೋಗಿತ್ತು. ಅಂತೂ ಸ್ವಾಮೀಜಿಯವರ ವಿರೋಧಿಗಳು ಒಂದು ದೊಡ್ಡ ವಿಜಯ ಸಾಧಿಸಿದಂತೆ ಕುಣಿದಾಡಿದರು. ‘ವಂಗವಾಸಿ’ ಮತ್ತಿತರ ಪತ್ರಿಕೆಗಳು ಈ ಸುದ್ದಿಯ ಬಗ್ಗೆ ಮತ್ತೆಮತ್ತೆ ಬರೆದುವು. ‘ವಂಗವಾಸಿ’ ಯಂತೂ, ಸಮುದ್ರವನ್ನು ದಾಟಿ ಮ್ಲೇಚ್ಛರೊಂದಿಗೆ ತಿಂದುಂಡು ಮೈಲಿಗೆಯಾಗಿದ್ದ ‘ಶೂದ್ರ ವಿವೇಕಾನಂದ’ರು, ತಮ್ಮನ್ನು ತಾವೇ ‘ಸ್ವಾಮಿ’ ಎಂದು ಕರೆದುಕೊಂಡು ದ್ರೋಹವೆಸಗಿದ್ದಾ ರೆಂದು ವಿಕಾರದನಿಯಿಂದ ಅರಚಿತು.

ಇದಕ್ಕೆಲ್ಲ ಸ್ವಾಮೀಜಿಯವರ ಪ್ರತಿಕ್ರಿಯೆಯೇನು? ಸಂಪೂರ್ಣ ನಿರ್ಲಕ್ಷ್ಯ! ಸಂಪೂರ್ಣ ನಿರ್ಲಿಪ್ತತೆ! ‘ವಂಗವಾಸಿ’ಯೇ ಮೊದಲಾದ ಪತ್ರಿಕೆಗಳು ಹಾಗೂ ಇತರ ಸಂಪ್ರದಾಯಸ್ಥರೆನ್ನಿಸಿ ಕೊಂಡವರು ಅಧಮಾಧಮರಂತೆ, ಮೃಗಗಳಂತೆ ವರ್ತಿಸಿದರೂ ಸ್ವಾಮೀಜಿ ಆ ಬಗ್ಗೆ ತೃಣಮಾತ್ರ ವಾದರೂ ಗಮನ ಕೊಡಲಿಲ್ಲ. ಕೆಲವೊಮ್ಮೆ ಆ ವಿಷಯದ ಪ್ರಸ್ತಾಪ ಬಂದಾಗ ತಮಾಷೆ ಮಾಡಿ ನಕ್ಕು ಸುಮ್ಮನಾದರು. ಅಷ್ಟೇ ಅಲ್ಲ, ತಮ್ಮ ಶಿಷ್ಯರಾಗಲಿ ಬೆಂಬಲಿಗರಾಗಲಿ ಈ ಪತ್ರಿಕೆಗಳೊಂದಿಗೆ ಪತ್ರವ್ಯವಹಾರ ನಡೆಸಲು, ಇಲ್ಲವೆ ಹೇಳಿಕೆ ಕೊಡಲು ನೋಡಿದರೆ ತಕ್ಷಣ ಅವರನ್ನು ಗದರಿ ಸುಮ್ಮನಾಗಿಸುತ್ತಿದ್ದರು. ‘ಶ್ರೀರಾಮಕೃಷ್ಣರು ತಮ್ಮನ್ನು ಸಮಗ್ರ ಭಾರತದ ಹಾಗೂ ವಿಶಾಲ ವಿಶ್ವದ ಹಿತಸಾಧನೆಯ ಸತ್ಪಥದಲ್ಲಿ ನಡೆಸಿದರೆ, ಇಲ್ಲವೆ?’–ಎಂಬುದನ್ನು ನಿಶ್ಚಯಿಸುವ ಹೊಣೆ ಇತಿಹಾಸಕ್ಕೆ ಸೇರಿದ್ದು ಅಥವಾ ಮುಂದಿನ ಪೀಳಿಗೆಗೆ ಸೇರಿದ್ದು ಎಂಬುದು ಸ್ವಾಮೀಜಿಯವರ ನಿಲುವು. ಅಂತೂ ಭಗವತ್ಸಂಕಲ್ಪದಂತೆ ಅವರ ಜೀವಿತಾವಧಿಯಲ್ಲೇ ಈ ವಿರೋಧವೆಲ್ಲ ಬಹು ಮಟ್ಟಿಗೆ ನಿರ್ನಾಮವಾಯಿತು, ಮತ್ತು ಅವರ ವಿರೋಧಿಗಳು ಹೇಳಹೆಸರಿಲ್ಲದಂತಾದರು.

ಆದರೆ ಪಾಂಚ್​ಕೋರಿ ವಂದ್ಯೋಪಾಧ್ಯಾಯನ ಕಥೆ ಮುಂದೇನಾಯಿತು? ಅದೊಂದು ತುಂಬ ಕುತೂಹಲಕರವಾದ ಘಟನೆಯೇ ಸರಿ. ಪಾಂಚ್​ಕೋರಿ, ಸ್ವಾಮೀಜಿಯವರ ಕಟ್ಟಾ ವಿರೋಧಿಯಾಗಿದ್ದರೂ ಅವರ ಆಕರ್ಷಣೆಯ ಶಕ್ತಿಯನ್ನು ಆತ ಬಹುಕಾಲ ವಿರೋಧಿಸಲಾರದವ ನಾದ. ಅವನಲ್ಲಿ ಅಡಗಿದ್ದ ಅತಂಸ್ಸತ್ವವೇ ಆತನನ್ನು ಸಂಪೂರ್ಣ ಅಧಃಪಾತಾಳಕ್ಕಿಳಿಯದಂತೆ ರಕ್ಷಿಸಿತು. ತಾನು ಅವರಿಗೆ ನೀಡಿದ ನಿರಂತರ ನಿಂದೆ-ಟೀಕೆಗಳಿಗೆ ಪ್ರತಿಯಾಗಿ ಅವರಿಂದ ಒಂದೇ ಒಂದು ಮಾತೂ ಹೊರಬರದಿದ್ದುದನ್ನು ಕಂಡಾಗ ಪಾಂಚ್​ಕೋರಿಯ ಕಣ್ಣು ತೆರೆಯಿತು. ಅವನ ವ್ಯಕ್ತಿತ್ವದಲ್ಲೇ ಒಂದು ಪರಿವರ್ತನೆಯಾಯಿತು. ತಾನು ಕಂಡ ಬಾಲ್ಯಸ್ನೇಹಿತ ನರೇಂದ್ರನಿಗಿಂತ ವಿಭಿನ್ನವಾದ ವ್ಯಕ್ತಿತ್ವವೊಂದನ್ನು ವಿವೇಕಾನಂದರಲ್ಲಿ ಕಂಡುಕೊಳ್ಳಲು ಆತ ಈಗ ಸಮರ್ಥನಾದ. ‘ವಂಗವಾಸಿ’ಯಂತಹ ಪತ್ರಿಕೆಯಲ್ಲಿ ಉಳಿದುಕೊಳ್ಳಲು ಇನ್ನು ಅವನಿಂದ ಸಾಧ್ಯವಾಗಲಿಲ್ಲ. ಅಲ್ಲಿಂದ ಹೊರಬಂದು ‘ರಂಗಾಲಯ’ ಎಂಬ ಮತ್ತೊಂದು ಪತ್ರಿಕೆಯ ಸಂಪಾದಕನಾದ. ಈಗ ಅವನಿಗೆ ಸ್ವತಂತ್ರವಾಗಿ ಮುಕ್ತಮನಸ್ಸಿನಿಂದ ವಿವೇಕಾನಂದರ ಮಹಿಮೆಯನ್ನು ಗುರುತಿಸಿ ಗ್ರಹಿ ಸಲು ಸಾಧ್ಯವಾಯಿತು. ೧೯ಂ೨ರಲ್ಲಿ ಸ್ವಾಮೀಜಿ ದೇಹತ್ಯಾಗ ಮಾಡಿದಾಗ ತನ್ನ ಹಿಂದಿನ ದೋಷಗಳಿಗೆಲ್ಲ ಪ್ರಾಯಶ್ಚಿತ್ತವಾಗಿಯೋ ಎಂಬಂತೆ ಪಾಂಚ್​ಕೋರಿ ತನ್ನ ಪತ್ರಿಕೆಯಲ್ಲಿ ಬರೆದ:

“ಬರುವ ಹಲವಾರು ಶತಮಾನಗಳಲ್ಲಿ ಬಂಗಾಳವು ವಿವೇಕಾನಂದರಂತಹ ಮತ್ತೊಂದು ರತ್ನ ವನ್ನು ಪ್ರಸವಿಸಲಾರದು. ಅವರ ಸುಂದರ ದೇವಸದೃಶ ಸುಧೃಢ ಶರೀರ, ಮಾಟವಾಗಿ ಕಡೆ ಯಲ್ಪಟ್ಟ ಮೈಕಟ್ಟು, ಆ ಕೋಕಿಲಕಂಠದಿಂದ ಹೊರಹೊಮ್ಮುತ್ತಿದ್ದ ಮೃದುಮಧುರ ಗಾನ, ಅವರ ಆತ್ಮಾಭಿಮಾನ, ಗಾಂಭೀರ್ಯ, ಜ್ಞಾನದೀಪ್ತಿ, ಎಲ್ಲಕ್ಕಿಂತ ಮಿಗಿಲಾಗಿ ಅವರ ಬೆರಗುಗೊಳಿಸುವ ಅಂತಶ್ಶಕ್ತಿ, ಆ ಸರಳತೆ, ಮತ್ತು ತಪೋಮಯ ಜೀವನದ ಬಗ್ಗೆ ಅವರಿಗಿದ್ದ ಒಲವು–ಇವೆಲ್ಲ ನನ್ನ ಕಣ್ಮುಂದೆ ಬಂದು ನಿಲ್ಲುತ್ತಿವೆ. ಈ ನೆನಪುಗಳೆಲ್ಲ ಒಂದಾದ ಮೇಲೊಂದು ಬಂದು ನುಗ್ಗಿ ನನ್ನ ಮನಸ್ಸನ್ನು ಮತ್ತಷ್ಟು ವಿದೀರ್ಣಗೊಳಿಸುತ್ತಿವೆ. ದೈವೀ ವ್ಯಕ್ತಿಯೊಬ್ಬ ಮಿಂಚಿನಂತೆ ಮಿನುಗಿ ನಮ್ಮಿಂದ ಮರೆಯಾದ.”

ಸ್ವಾಮೀಜಿಯವರ ಕಡು ವಿರೋಧಿಯಾಗಿದ್ದ ಪಾಂಚ್​ಕೋರಿಯಿಂದ ಬಂದ ಅಂತಃಕರಣ ಪೂರ್ವಕವಾದ ಪ್ರಶಂಸೆಯ ನುಡಿಗಳಿವು. ನಿಜಕ್ಕೂ ಇದೆಂಥ ಅದ್ಭುತ!

ಆದರೆ ಇತ್ತ ‘ವಂಗವಾಸಿ’ ಪತ್ರಿಕೆಯು ತನ್ನ ಸ್ವಭಾವವನ್ನು ಬಿಡದೆ ಸ್ವಾಮೀಜಿ ದೇಹತ್ಯಾಗ ಮಾಡಿದ ಮೇಲೂ ಅವರ ಬಗ್ಗೆ ಹೀನ ಕುಹಕದ ಟೀಕೆಯನ್ನೇ ಬರೆಯಿತು: “ಶ್ರೀರಾಮಕೃಷ್ಣರ ಬುದ್ಧಿವಂತ ಶಿಷ್ಯರಾದ ನರೇಂದ್ರನಾಥ ದತ್ತ–ಇವರನ್ನು ಈಚೆಗೆ ಕೆಲವರು, ಸ್ವಾಮಿ ವಿವೇಕಾ ನಂದ ಎಂದು ಕರೆಯುತ್ತಿದ್ದರು–ಇವರು ತಮ್ಮ ಅದ್ಭುತ ವಾಕ್​ಸಾಮರ್ಥ್ಯದಿಂದ ಜನರನ್ನು ಆಕರ್ಷಿಸಲು ಪ್ರಯತ್ನಿಸಿದರು. ತಮ್ಮ ಈ ವಾಗ್ಝರಿಯಿಂದ ಅವರು ಸ್ತ್ರೀಯರನ್ನೂ ಆಕರ್ಷಿಸಿ ತಮ್ಮ ಶಿಷ್ಯೆಯರನ್ನಾಗಿ ಮಾಡಿಕೊಂಡರು. ಇದು ಅವರ ಅದ್ಭುತ ಚಾಲಾಕಿತನಕ್ಕೆ ಒಳ್ಳೆಯ ನಿದರ್ಶನ... ”

ಈಗ ‘ವಂಗವಾಸಿ’ಯನ್ನು ತರಾಟೆಗೆ ತೆಗೆದುಕೊಳ್ಳುವ ಕೆಲಸ ಪಾಂಚ್​ಕೋರಿಯ ಪಾಲಿನ ದಾಯಿತು! ಅವನು ‘ವಂಗವಾಸಿ’ಯನ್ನು ಬಿಟ್ಟಮೇಲೆ ಅದರ ಆಡಳಿತವು ಬ್ರಾಹ್ಮಣೇತರರ ಕೈಗೆ ಬಂದಿತ್ತು. ಈ ಅಂಶವನ್ನು ಗಮನದಲ್ಲಿಟ್ಟುಕೊಂಡು ಪಾಂಚ್​ಕೋರಿ ತನ್ನ ‘ರಂಗಾಲಯ’ ಪತ್ರಿಕೆ ಯಲ್ಲಿ ‘ವಂಗವಾಸಿ’ಯ ನಿಲುವನ್ನು ಕಟುವಾಗಿ ಖಂಡಿಸಿ ಬರೆದ:

“ಹಿಂದೆ ‘ವಂಗವಾಸಿ’ಯು ವಿವೇಕಾನಂದರನ್ನು ಜರೆದಾಗ ಅದರ ಸಂಪಾದಕ ಮಂಡಲಿಯ ಲ್ಲಿದ್ದ ನಾವು ಬ್ರಾಹ್ಮಣರು. ಅಂದಿನ ಆ ಟೀಕೆಗಳಿಗೆ ನಾವೇ ಹೊಣೆಗಾರರು. ನಾವು ಬ್ರಾಹ್ಮಣ ರಾದುದರಿಂದ ಬ್ರಾಹ್ಮಣಿಕೆಯನ್ನು ಎತ್ತಿಹಿಡಿಯುವ ಆಸೆ ನಮ್ಮಲ್ಲಿದ್ದುದು ಸಹಜವೇ. ಆದರೆ ಈಗ ‘ವಂಗವಾಸಿ’ಯು ಪಕ್ಕಾ ಶೂದ್ರರ ಕೈಗೆ ಸೇರಿರುವಾಗ ಶೂದ್ರನ ವಿರುದ್ಧವಾದ ಹಳೆಯ ದ್ವೇಷವನ್ನೇ ಈಗಲೂ ಏಕೆ ಮುಂದುವರಿಸಬೇಕು?... ಹೇಗಿದ್ದರೂ ವಿವೇಕಾನಂದರು ದೇಹ ತ್ಯಾಗ ಮಾಡಿಯಾಯಿತು. ಸತ್ತವರ ಸುದ್ದಿಗೆ ಯಾರು ಹೋಗುತ್ತಾರೆ? ಈ ಬಂಗಾಳಿಯ ನಿಧನ ಕ್ಕಾಗಿ ಆಂಗ್ಲರ ಪತ್ರಿಕೆಯಾದ ‘ಇಂಗ್ಲಿಷ್​ಮನ್​’ ಕೂಡ ಶೋಕ ವ್ಯಕ್ತಪಡಿಸಬಹುದಂತೆ; ಹಾಗಿರು ವಾಗ, ಬಂಗಾಳದಲ್ಲಿ ಹುಟ್ಟಿದ ಈ ‘ವಂಗವಾಸಿ’ಗೆ ಅಮಾನುಷ ಕುಚೋದ್ಯ ಮಾಡುತ್ತ ಆ ಶೋಕವನ್ನು ಮರೆಯಲು ಹೇಗೆತಾನೆ ಸಾಧ್ಯವಾಯಿತು? ಥೂ! ‘ವಂಗವಾಸಿ’ಗೆ ಧಿಕ್ಕಾರವಿರಲಿ!”

ಸ್ವಾಮೀಜಿ ಕಣ್ಮರೆಯಾದ ನಂತರ ಪಾಂಚ್​ಕೋರಿ ತನ್ನ ಮನಸ್ಸಿನ ಹಳೆಯ ಕೊಳೆಯನ್ನು ಸಂಪೂರ್ಣವಾಗಿ ತೊಳೆದುಕೊಂಡು ಹೊಸ ಮನುಷ್ಯನಾಗಲು ಶ್ರಮಿಸಿದ. ಇತರರ ಇಚ್ಛೆಗೆ ಅನು ಸಾರವಾಗಿ ಬರೆದು ತನ್ನನ್ನು ಮಾರಿಕೊಳ್ಳುವ ಕೆಲಸ ಅವನಿಗೆ ಜುಗುಪ್ಸೆ ಹುಟ್ಟಿಸಿತ್ತು. ಇನ್ನಾದರೂ ರಾಜಕೀಯದ ಕೆಸರಿನಿಂದ ದೂರವಾಗುವ ಉದ್ದೇಶದಿಂದ ಅವನು ‘ಪ್ರವಾಹಿನಿ’ ಎಂಬ ಬೇರೊಂದು ಪತ್ರಿಕೆಯನ್ನು ಸೇರಿದ. ಇಲ್ಲಿ ಅವನ ಅಸಾಧಾರಣ ಪ್ರತಿಭೆಯು ಸಂಪೂರ್ಣವಾಗಿ ವ್ಯಕ್ತವಾಯಿತು. ಈ ಪತ್ರಿಕೆಯಲ್ಲಿ ಅವನು ಹಲವಾರು ಬಾರಿ ಶ್ರೀರಾಮಕೃಷ್ಣರ ಹಾಗೂ ವಿವೇಕಾ ನಂದರ ಬಗ್ಗೆ ಅತ್ಯಂತ ಪ್ರಭಾವಶಾಲಿಯಾಗಿ ಬರೆದ. ಅಲ್ಲದೆ ವಿವೇಕಾನಂದರ ಬಗ್ಗೆ ಈತ ಬರೆದ ಸ್ಮೃತಿ ಚಿತ್ರಣವು ಬಂಗಾಳೀ ಸಾಹಿತ್ಯದಲ್ಲೇ ಅದ್ವಿತೀಯವಾದುದು. ಅತ್ಯಂತ ಪ್ರಾಮಾಣಿಕ ಹೃದಯದಿಂದ ಬರೆದ ಈ ನೆನಪುಗಳಲ್ಲಿ ಪಾಂಚ್​ಕೋರಿ ತುಂಬ ಹೃದಯಸ್ಪರ್ಶಿಯಾದ ಭಾವನೆ ಗಳನ್ನು ವ್ಯಕ್ತಪಡಿಸಿದ್ದಾನೆ.

ಹೀಗೆ ಸ್ವಾಮೀಜಿಯವರನ್ನು ಕಟುವಾಗಿ ಟೀಕಿಸಿದ ವ್ಯಕ್ತಿಗಳಲ್ಲಿ ಕೊನೆಗೆ ಅವರ ಮಾಹಾತ್ಮ್ಯ ವನ್ನು ಮನಗಂಡು ಅವರಲ್ಲಿ ಶರಣಾದ ಪಾಂಚ್​ಕೋರಿ ವಂದ್ಯೋಪಾಧ್ಯಾಯನಂಥವರು ಹಲ ವರು. ಆದರೆ ಇಂತಹ ಹಲವಾರು ವ್ಯಕ್ತಿಗಳು, ಹಲವಾರು ಪತ್ರಿಕೆಗಳು ಏಕಕಾಲದಲ್ಲಿ ಅತ್ಯಂತ ನಿರ್ದಯವಾದ, ನೀಚಬುದ್ಧಿಪ್ರೇರಿತವಾದ ನಿಂದನೆಯಲ್ಲಿ ತೊಡಗಿದಾಗ, ಅದಕ್ಕೆ ಬಲಿಯಾದವನ ಸ್ಥಿತಿ ಹೇಗಿರಬಹುದು! ಮುಂದೆ ಕಾಲಕ್ರಮದಲ್ಲಿ ಏನೇನಾಯಿತೆಂಬುದು ಬೇರೆ ವಿಚಾರ. ಏಕೆಂ ದರೆ ಮುಂದೇನಾದೀತೆಂದು ಆಗ ಯಾರಿಗೆ ತಿಳಿದಿತ್ತು? ಆದರೆ ಇಂತಹ ಅಗ್ನಿಪರೀಕ್ಷೆಯ ಸಮಯ ದಲ್ಲೂ ಅತ್ಯಂತ ಸಮಾಧಾನಚಿತ್ತರಾಗಿದ್ದುದು ಸ್ವಾಮೀಜಿಯವರ ಧೀಮಂತಿಕೆಯನ್ನು ತೋರಿಸುತ್ತದೆ.

ಇರಲಿ, ಒಂದು ಮಹಾಕಾರ್ಯವನ್ನು ಹೊಸದಾಗಿ ಕೈಗೊಂಡಾಗ ಅದನ್ನು ವಿರೋಧಿಸುವವರು ಇದ್ದೇ ಇರುತ್ತಾರೆ. ಅಂತೆಯೇ ಅದರ ಮಹತ್ವವನ್ನು ಮನಗಂಡು ಅದಕ್ಕೆ ಪರವಾಗಿ ಕೆಲಸ ಮಾಡುವವರು ಇರುತ್ತಾರೆಂಬುದೂ ಅಷ್ಟೇ ನಿಜ. ಸ್ವಾಮೀಜಿಯವರ ವಿಷಯದಲ್ಲೂ ಇದು ಅನ್ವಯವಾಗುತ್ತದೆ. ಸ್ವಾಮಿ ಅಖಂಡಾನಂದರು ಮುರ್ಶಿದಾಬಾದಿನ ಬರಪೀಡಿತ ಪ್ರದೇಶಗಳಲ್ಲಿ ಕೆಲಸ ಮಾಡಿ ನೂರಾರು ಜನರಿಗೆ ನೆರವಾಗುತ್ತಿದ್ದ ವಿಷಯ ತಿಳಿದು ಸಂತಸಗೊಂಡ ಸ್ವಾಮೀಜಿ ಅವರಿಗೆ ಬರೆದ ಪತ್ರವನ್ನು ಹಿಂದೆ ನೋಡಿದ್ದೇವೆ. ಅಖಂಡಾನಂದರು ಆ ಪ್ರದೇಶಗಳಲ್ಲಿ ಇತರ ಯಾರದೇ ನೆರವಿಲ್ಲದೆ ಏಕಾಂಗಿಯಾಗಿ ದುಡಿಯುತ್ತಿದ್ದರು. ಇದನ್ನು ತಿಳಿದ ಸ್ವಾಮೀಜಿ ಅವರ ಕಾರ್ಯದಲ್ಲಿ ನೆರವಾಗಲು ನಿತ್ಯಾನಂದರನ್ನು ಹಾಗೂ ಸುರೇಶ್ವರಾನಂದರನ್ನು ಕಳಿಸಿಕೊಟ್ಟರು. ಅಲ್ಲದೆ ತಾವೂ ಈ ಬರಗಾಲಪರಿಹಾರದ ನಿಧಿಗಾಗಿ ಧನಸಂಗ್ರಹ ಮಾಡಲು ಪ್ರಾರಂಭಿಸಿದರು. ಕಲ್ಕತ್ತ ಮದ್ರಾಸು ವಾರಾಣಸಿಗಳಿಂದ ಮತ್ತು ಬೌದ್ಧರ ಮಹಾಬೋಧಿ ಸೊಸೈಟಿಯಿಂದ ಈ ಕಾರ್ಯಕ್ಕೆ ಸಹಾಯ ಒದಗಿತು. ಸ್ವಾಮಿ ಅಖಂಡಾನಂದರು ಪರಿಹಾರಕಾರ್ಯವನ್ನು ಎಷ್ಟು ಚೆನ್ನಾಗಿ ನೆರವೇರಿಸಿದರೆಂದರೆ ಸರ್ಕಾರೀ ಅಧಿಕಾರಿಯಾಗಿದ್ದ ಜಿಲ್ಲಾ ನ್ಯಾಯಾಧೀಶರು ತಮ್ಮ ವರದಿಯಲ್ಲಿ “ಸ್ವಾಮಿಗಳು ಕೆಲಸ ಮಾಡುತ್ತಿರುವ ಹಳ್ಳಿಗಳ ಜವಾಬ್ದಾರಿಯನ್ನೆಲ್ಲ ನಾನು ಸಂಪೂರ್ಣವಾಗಿ ಕಳಚಿಕೊಳ್ಳಲು ಸಾಧ್ಯವಾಗಿದೆ” ಎಂದು ತಿಳಿಸಿದರು.

ಸ್ವಾಮೀಜಿಯವರಿಗೆ ಸಂತೋಷವುಂಟುಮಾಡಿದ ಮತ್ತೊಂದು ಸಂಗತಿಯೆಂದರೆ, ಅತ್ತ ಕಲ್ಕತ್ತದಲ್ಲಿ ರಾಮಕೃಷ್ಣ ಮಿಷನ್ನಿನ ಕಾರ್ಯಚಟುವಟಿಕೆಗಳು ಯಶಸ್ವಿಯಾಗಿ ಮುಂದುವರಿಯುತ್ತಿ ದ್ದುದು. ಸ್ವಾಮಿ ಬ್ರಹ್ಮಾನಂದರ ಅಧ್ಯಕ್ಷತೆಯಲ್ಲಿ ಮಠದ ಸಂನ್ಯಾಸಿಗಳೂ ಪ್ರಮುಖ ಗೃಹಸ್ಥ ಭಕ್ತರೂ ಸಭೆ ಸೇರಿ ಅನೇಕ ಪ್ರಮುಖ ವಿಷಯಗಳ ಬಗ್ಗೆ ನಿರ್ಧಾರ ಕೈಗೊಂಡಿದ್ದರು. ಅಲ್ಲದೆ ದಕ್ಷಿಣದ ಮದ್ರಾಸಿನಲ್ಲಿ ಸ್ವಾಮಿ ರಾಮಕೃಷ್ಣಾನಂದರು ತಮ್ಮ ಸ್ವಭಾವಸಹಜವಾದ ಉತ್ಸಾಹ ದಿಂದ ಸ್ವಾಮೀಜಿ ತಮಗೊಪ್ಪಿಸಿದ್ದ ಕಾರ್ಯವನ್ನು ಸಮರ್ಥವಾಗಿ ನಿರ್ವಹಿಸುತ್ತಿದ್ದರು. ಅವರು ಅಲ್ಲಿ ಸಂತರ ಹಾಗೂ ಅವತಾರಪುರುಷರ ಜೀವನ-ಸಂದೇಶಗಳ ಮೇಲೆ ಉಪನ್ಯಾಸ ಮಾಲೆಯನ್ನು ನಡೆಸಿದ್ದರು; ವೇದಾಂತ ಉಪನಿಷತ್ತು ಗೀತೆಗಳ ಬಗ್ಗೆ ಮದ್ರಾಸಿನ ಹಲವಾರು ಸ್ಥಳಗಳಲ್ಲಿ ತರಗತಿಗಳನ್ನು ಪ್ರಾರಂಭಿಸಿದ್ದರು. ಹೀಗೆ ಅವರು ಕೆಲಕಾಲದಲ್ಲೇ ತಮ್ಮ ಅಂತಶ್ಶಕ್ತಿಯ ಬಲ ದಿಂದ ಹಾಗೂ ಕಾರ್ಯಕಲಾಪಗಳ ಸಮರ್ಥ ನಿರ್ವಹಣೆಯಿಂದ ಜನಮನವನ್ನು ಸೂರೆ ಗೊಂಡರು. ಇದೂ ಕೂಡ ಸ್ವಾಮೀಜಿಯವರ ಮನಸ್ಸಿಗೆ ಅತ್ಯಂತ ಆನಂದವನ್ನುಂಟುಮಾಡಿತು.

ಸ್ವಾಮೀಜಿಯವರ ಬೋಧನೆಗಳಿಂದ ಆಕರ್ಷಿತರಾಗಿದ್ದ ಸಿಲೋನಿನ ಹಿಂದೂಗಳು ತಮ್ಮಲ್ಲಿಗೆ ಯಾರಾದರೊಬ್ಬ ಸ್ವಾಮಿಗಳನ್ನು ಕಳಿಸಿಕೊಡಬೇಕೆಂದು ಮತ್ತೆ ಮತ್ತೆ ಕೇಳಿಕೊಳ್ಳುತ್ತಿದ್ದರು. ಅವರ ಬೇಡಿಕೆಯನ್ನು ಮನ್ನಿಸಿ ಸ್ವಾಮೀಜಿ, ಸ್ವಾಮಿ ಶಿವಾನಂದರ ಮೇಲೆ ಈ ಜವಾಬ್ದಾರಿಯನ್ನು ಹೊರಿಸಿ ಕಳಿಸಿಕೊಟ್ಟರು. ಅಲ್ಲದೆ ತಮ್ಮ ಸೇವೆಯಲ್ಲಿ ನಿರತನಾಗಿದ್ದ ಜೆ. ಜೆ. ಗುಡ್​ವಿನ್ನನನ್ನು ಮದ್ರಾಸಿಗೆ ಕಳಿಸಿಕೊಡುತ್ತ, “ನೀನು ಅಲ್ಲೊಂದು ಇಂಗ್ಲಿಷ್ ದಿನಪತ್ರಿಕೆಯನ್ನು ಹೊರಡಿಸಬೇಕು. ಜೊತೆಗೆ ‘ಬ್ರಹ್ಮವಾದಿನ್​’ಗಾಗಿ ಕೆಲಸ ಮಾಡುತ್ತಿರುವ ಅಳಸಿಂಗನಿಗೂ ನೆರವಾಗು. ‘ಬ್ರಹ್ಮವಾದಿನ್​’ನ ಪ್ರಸಾರ ಇನ್ನೂ ಅಧಿಕವಾಗುವಂತೆ ಎಲ್ಲರೂ ಶ್ರಮಿಸಿ” ಎಂದರು. ಅದರಂತೆಯೇ ಗುಡ್​ವಿನ್ ವಿಧೇಯನಾಗಿ ಹೊಸ ಕಾರ್ಯವನ್ನು ವಹಿಸಿಕೊಳ್ಳಲು ತಕ್ಷಣ ಮದ್ರಾಸಿಗೆ ಹೊರಟುಬಂದ. ಆದರೆ ದಿನಪತ್ರಿಕೆಯ ಯೋಜನೆ ಕೈಗೂಡದಿದ್ದುದರಿಂದ ‘ಬ್ರಹ್ಮವಾದಿನ್​’ನ ಕಾರ್ಯದಲ್ಲೇ ನಿರತನಾದ.

ಅಮೆರಿಕದಲ್ಲಿ ಸ್ವಾಮಿ ಶಾರದಾನಂದರ ಕಾರ್ಯ ತುಂಬ ಯಶಸ್ವಿಯಾಗಿ ಸಾಗಿತ್ತು. ಆದರೆ ಲಂಡನ್ನಿನಲ್ಲಿ ಸ್ವಾಮಿ ಅಭೇದಾನಂದರ ಬೋಧನಾಕಾರ್ಯ ಅಷ್ಟೇನೂ ಫಲಕಾರಿಯಾಗಲಿಲ್ಲ. ಆದ್ದರಿಂದ ಅವರು ಸ್ವಾಮೀಜಿಯವರ ಆದೇಶದಂತೆ ಲಂಡನ್ನಿನ ಕೆಲಸವನ್ನು ನಿಲ್ಲಿಸಿ, ಶಾರದಾ ನಂದರಿಗೆ ನೆರವಾಗಲು ನ್ಯೂಯಾರ್ಕಿಗೆ ತೆರಳಿದರು.

ಇತ್ತ ಸ್ವಾಮೀಜಿಯವರು ಆಲ್ಮೋರದಿಂದ ಹೊರಡುವ ಸಮಯ ಸನ್ನಿಹಿತವಾಯಿತು. ಈಗ ಅವರ ಆರೋಗ್ಯ ಮೊದಲಿಗಿಂತ ಸುಧಾರಿಸಿತ್ತು. ಸ್ವಾಮೀಜಿ ಆಲ್ಮೋರದಿಂದ ಹೊರಡುವ ಮೊದಲು ಅವರದ್ದೊಂದು ಉಪನ್ಯಾಸವನ್ನು ಏರ್ಪಡಿಸಬೇಕೆಂದು ಅಲ್ಲಿನ ನಾಗರಿಕರು ತುಂಬ ಉತ್ಸಾಹಿತರಾಗಿದ್ದರು. ಅಲ್ಲದೆ ಅಲ್ಲಿನ ಆಂಗ್ಲ ನಾಗರಿಕರೂ ಕೂಡ ಸ್ವಾಮೀಜಿಯವರ ಮಾತು ಗಳನ್ನು ಆಲಿಸಲು ಬಯಸಿದರು. ಭಾರತೀಯರಿಗಾಗಿ ಎರಡು ಉಪನ್ಯಾಸಗಳೂ ಆಂಗ್ಲ ಪ್ರಜೆಗಳಿ ಗಾಗಿ ಅಲ್ಲಿನ ‘ಇಂಗ್ಲಿಷ್​ಕ್ಲಬ್​’ನಲ್ಲಿ ಒಂದು ಉಪನ್ಯಾಸವೂ ಏರ್ಪಾಡಾದುವು.

‘ಇಂಗ್ಲಿಷ್ ಕ್ಲಬ್​’ನಲ್ಲಿ ಸುಶಿಕ್ಷಿತ ಸಭಿಕರನ್ನುದ್ದೇಶಿಸಿ ಸ್ವಾಮೀಜಿ ಮಾಡಿದ ಉಪನ್ಯಾಸದ ವಿಷಯ, ‘ವೇದಗಳ ಬೋಧನೆ ಮತ್ತು ಅದರ ಅನುಷ್ಠಾನ.’ ಪೂರ್ವಕಾಲದ ಅರೆನಾಗರಿಕ ಜನಗಳ ದೇವರಿಂದ ಹಿಡಿದು ವೇದಕಾಲದವರೆಗೆ ದೇವರ ಕಲ್ಪನೆಯು ಹೇಗೆ ವಿಕಾಸ ಹೊಂದಿದೆ ಎಂಬು ದನ್ನು ವಿವರಿಸಿದ ಸ್ವಾಮೀಜಿ, ಬಳಿಕ ವೇದಗಳ ಕಿರುಪರಿಚಯ ಮಾಡಿಕೊಟ್ಟರು. ಅನಂತರ ಆತ್ಮದ ವಿಚಾರ. ಪಾಶ್ಚಾತ್ಯರು ತಮ್ಮ ಧಾರ್ಮಿಕ ಸಮಸ್ಯೆಗಳಿಗೆ, ಆಧ್ಯಾತ್ಮಿಕ ರಹಸ್ಯಗಳಿಗೆ ಉತ್ತರವನ್ನು ಹುಡುಕುವುದು ಬಾಹ್ಯಜಗತ್ತಿನಲ್ಲಿ; ಆದರೆ ಭಾರತೀಯರು ಆ ಉತ್ತರವು ಬಾಹ್ಯಜಗತ್ತಿನಲ್ಲಿ ಸಿಗಲಾರದೆಂಬುದನ್ನು ಕಂಡುಕೊಂಡು ಅನಂತರ ಅದನ್ನು ತಮ್ಮೊಳಗೇ ಕಂಡುಕೊಳ್ಳಲು ಪ್ರಯತ್ನಿಸುತ್ತಾರೆ ಎಂದರು ಸ್ವಾಮೀಜಿ. ಬಳಿಕ ಜೀವಾತ್ಮ-ಪರಮಾತ್ಮರ ನಡುವಣ ಸಂಬಂಧ, ಕೊನೆಗೆ ಜೀವಾತ್ಮನು ಪರಮಾತ್ಮನೊಂದಿಗೆ ಹೊಂದುವ ಐಕ್ಯ–ಇವುಗಳನ್ನು ಬಣ್ಣಿಸುತ್ತ ಸ್ವಾಮೀಜಿ ಸಂಪೂರ್ಣವಾಗಿ ತನ್ಮಯರಾಗಿಬಿಟ್ಟರು. ಬಹುಶಃ ಭಾರತೀಯ ಶ್ರೋತೃಗಳಿಗೆ ಆ ಬಗೆಯ ಅನುಭವ ಅಪೂರ್ವವಾದದ್ದು. ಅತ್ಯಂತ ಸ್ಪೂರ್ತಿಯುತರಾಗಿ ಮಾತನಾಡುತ್ತಿದ್ದ ಸ್ವಾಮೀಜಿ ಪ್ರಖರ ತೇಜಸ್ಸಿನಿಂದ ಕಂಗೊಳಿಸಿದರು. ಇದಕ್ಕೆ ಸಾಕ್ಷಿ ಅಂದಿನ ಸಭೆಯಲ್ಲಿ ಉಪಸ್ಥಿತ ಳಿದ್ದ ಮಿಸ್ ಹೆನ್ರಿಟಾ ಮುಲ್ಲರ್. ಆಕೆ ಹೇಳುತ್ತಾರೆ, “ಸ್ವಾಮೀಜಿಯವರು ಅಂದು ಮಾತನಾಡುತ್ತಿ ದ್ದಾಗ ಕೆಲಹೊತ್ತಿನವರೆಗೆ, ಅವರು-ಅವರ ವಾಣಿ-ಅವರ ಸಭಿಕರು-ಮತ್ತು ಅವರೆಲ್ಲರನ್ನೂ ಆವರಿಸಿಕೊಂಡಿದ್ದ ದಿವ್ಯ ಸ್ಫೂರ್ತಿ–ಎಲ್ಲವೂ ಒಂದಾಗಿಬಿಟ್ಟಿತ್ತು! ಆಗ ಅಲ್ಲಿ ‘ನಾನು’ ‘ನೀನು’ ‘ಅದು’ ‘ಇದು’ ಎಂಬ ಪ್ರಭೇದದ ಪ್ರಜ್ಞೆಯೇ ಇರಲಿಲ್ಲ. ಅಲ್ಲಿದ್ದ ಪ್ರತಿಯೊಂದು ವಸ್ತುವೂ, ಪ್ರತಿಯೋರ್ವ ವ್ಯಕ್ತಿಯೂ ಆ ಮಹಾಚಾರ್ಯನ ವ್ಯಕ್ತಿತ್ವದಿಂದ ಹೊರಹೊಮ್ಮುತ್ತಿದ್ದ ಆಧ್ಯಾ ತ್ಮಿಕ ಪ್ರಭೆಯಲ್ಲಿ ಮುಳುಗಿ ಕರಗಿಯೇ ಹೋದಂತಿತ್ತು! ಹೀಗೆ ಅವರೆಲ್ಲರನ್ನೂ ಸ್ವಾಮೀಜಿ ಒಂದು ಅತ್ಯುನ್ನತ ಭಾವಾವಸ್ಥೆಯಲ್ಲಿ ಹಿಡಿದಿಟ್ಟಿದ್ದರು.”

ಸ್ವಾಮೀಜಿ ಆಲ್ಮೋರದಲ್ಲಿದ್ದ ಸಮಯದಲ್ಲೇ ಅವರಿಗೆ ಲಂಡನ್ನಿನ ಇ. ಟಿ. ಸ್ಟರ್ಡಿಯಿಂದ ಒಂದು ಪತ್ರ ಬಂದಿತು. ಮಿಸ್ ಮಾರ್ಗರೆಟ್ ನೋಬೆಲ್ಲಳು ಭಾರತಕ್ಕೆ ಬಂದು ಸ್ವಾಮೀಜಿಯವರ ಕಾರ್ಯದಲ್ಲಿ ನೆರವಾಗಲು ಇಚ್ಛಿಸಿದ್ದಾಳೆಂಬ ವಿಷಯವನ್ನು ಸ್ಟರ್ಡಿ ಬರೆದಿದ್ದ. ಇದನ್ನು ತಿಳಿದು ಸ್ವಾಮೀಜಿ ಮಾರ್ಗರೆಟ್ಟಳಿಗೊಂದು ಪತ್ರ ಬರೆದರು:

\noindent

ಪ್ರಿಯ ಮಿಸ್ ನೋಬೆಲ್,

ನಿನ್ನೆ ಸ್ಟರ್ಡಿಯಿಂದ ಒಂದು ಪತ್ರ ನನ್ನ ಕೈಸೇರಿತು. ನೀನು ಇಲ್ಲಿ ನಡೆಯುತ್ತಿರುವುದನ್ನು ಕಣ್ಣಾರೆ ಕಾಣುವ ಉದ್ದೇಶದಿಂದ ಭಾರತಕ್ಕೆ ಬರಲು ನಿರ್ಧರಿಸಿರುವೆಯೆಂಬ ವಿಷಯ ತಿಳಿಯಿತು. ಸ್ಟರ್ಡಿಯ ಆ ಪತ್ರಕ್ಕೆ ನಾನು ನಿನ್ನೆಯೇ ಉತ್ತರಿಸಿದೆ. ಆದರೆ ಇಲ್ಲಿನ (ಭಾರತದಲ್ಲಿನ) ನಿನ್ನ ಉದ್ದೇಶಿತ ಯೋಜನೆಗಳ ಬಗ್ಗೆ ಮಿಸ್ ಮುಲ್ಲರಳಿಂದ ಕೆಲವು ವಿಷಯಗಳು ತಿಳಿದುಬಂದುವು. ಆದ್ದರಿಂದ ನಾನೀಗ ಈ ಪತ್ರವನ್ನು ಬರೆಯಬೇಕಾಯಿತು. ಮತ್ತು ನಾನು ಹೇಳಬೇಕೆಂದಿರುವು ದನ್ನು ನೇರವಾಗಿ ತಿಳಿಸುವುದು ಒಳ್ಳೆಯದು.

ನಿನಗೊಂದು ಮಾತನ್ನು ಸ್ಪಷ್ಟವಾಗಿ ಹೇಳಬಯಸುತ್ತೇನೆ; ಏನೆಂದರೆ, ಭಾರತದಲ್ಲಿ ಆಗ ಬೇಕಾದ ಕಾರ್ಯದಲ್ಲಿ ನಿನಗೊಂದು ಮಹತ್ವದ ಪಾತ್ರವಿದೆ ಎಂಬುದು ನನಗೆ ದೃಢವಾಗಿದೆ. ಅದರಲ್ಲೂ ಮುಖ್ಯವಾಗಿ ಭಾರತೀಯ ಮಹಿಳೆಯರಿಗಾಗಿ ಕೆಲಸ ಮಾಡಲು ನಮಗಿಂದು ಬೇಕಾದುದು ಒಬ್ಬ ಪುರುಷನಲ್ಲ. ಸ್ತ್ರೀ–ಸಿಂಹಿಣಿಯಂತಹ ಒಬ್ಬ ಸ್ತ್ರೀ!

ಶ್ರೇಷ್ಠ ಸ್ತ್ರೀಯರನ್ನು ನಿರ್ಮಾಣ ಮಾಡುವಲ್ಲಿ ಭಾರತವಿನ್ನೂ ಸಮರ್ಥವಾಗಿಲ್ಲ; ಅಂತಹ ಸ್ತ್ರೀಯರನ್ನು ಅದು ಇತರ ರಾಷ್ಟ್ರಗಳಿಂದ ಎರವಲು ಪಡೆಯಬೇಕಾಗಿದೆ. ನಿನ್ನಲ್ಲಿರುವ ವಿದ್ಯೆ, ಪ್ರಾಮಾಣಿಕತೆ, ಪವಿತ್ರತೆ, ಅತಿಶಯ ಪ್ರೀತಿ, ಸಂಕಲ್ಪಶಕ್ತಿ ಮತ್ತು ಇವೆಲ್ಲಕ್ಕೂ ಮಿಗಿಲಾಗಿ ನಿನ್ನಲ್ಲಿ ಹರಿಯುತ್ತಿರುವ ವೀರ ಆಂಗ್ಲ ರಕ್ತ–ಇವುಗಳನ್ನೆಲ್ಲ ಗಮನಿಸಿದಾಗ ನಮ್ಮ ಆವಶ್ಯಕತೆಗೆ ತಕ್ಕಂತೆ ಮಾಡಿಸಿಟ್ಟಂತಹ ಮಹಿಳೆ ನೀನೇ ಸರಿ.

ಆದರೆ ಕಷ್ಟಗಳು ಹಲವಾರು. ಇಲ್ಲಿ ಭಾರತದಲ್ಲಿ ಇರುವ ಸಂಕಟಗಳು, ಮೂಢ ನಂಬಿಕೆಗಳು, ದಾಸ್ಯ–ಇವುಗಳ ಅರಿವಿವಿಲ್ಲ ನಿನಗೆ. ನೀನಿಲ್ಲಿಗೆ ಬಂದರೆ ಮಡಿವಂತಿಕೆಯೇ ಮೊದಲಾದ ವಿಚಿತ್ರ ಭಾವನೆಗಳ ಅರೆಬೆತ್ತಲೆ ಸ್ತ್ರೀ ಪುರುಷರ ನಡುವೆ ಇರಬೇಕಾಗುತ್ತದೆ. ಮತ್ತೆ ಅವರೆಲ್ಲ ಬಿಳಿಯರನ್ನು ಕಂಡರೆ ಹೆದರಿಕೆಯಿಂದಲೊ ದ್ವೇಷದಿಂದಲೊ ಅವರನ್ನು ದೂರವಿಡುತ್ತಾರೆ. ಆದ್ದರಿಂದ ನೀನು ಅವರುಗಳ ದ್ವೇಷದ ನಡುವೆ ಜೀವಿಸಬೇಕಾಗುತ್ತದೆ. ಅಲ್ಲದೆ ಇಲ್ಲಿರುವ ಬಿಳಿಯರು ನಿನ್ನನ್ನು ಒಂದು ವಿಚಿತ್ರ ಪ್ರಾಣಿಯೆಂಬಂತೆ ನೋಡುತ್ತಾರೆ. ಮತ್ತು ನಿನ್ನ ಪ್ರತಿಯೊಂದು ಚಲನವಲನ ವನ್ನೂ ಸಂಶಯದೃಷ್ಟಿಯಿಂದ ನೋಡುತ್ತಾರೆ.

ಇನ್ನು ಇಲ್ಲಿನ ಹವೆಯೋ ಭಯಂಕರ ಧಗೆ. ಇಲ್ಲಿನ ಹೆಚ್ಚಿನ ಸ್ಥಳಗಳಲ್ಲಿ ಚಳಿಗಾಲವೆಂದರೆ ಅದು ನಿಮ್ಮ ಸೆಖೆಗಾಲಕ್ಕೆ ಸಮ. ಇನ್ನು ದಕ್ಷಿಣ ಭಾರತವಂತೂ ಯಾವಾಗಲೂ ಧಗಧಗಿಸುತ್ತಲೇ ಇರುತ್ತದೆ. ನಗರಗಳ ಹೊರಗೆ ಹೋದರೆ ನಿನಗೆ ನಿನ್ನ ಪಾಶ್ಚಾತ್ಯ ರೀತಿಯ ಅನುಕೂಲತೆಗಳಾವುವೂ ಸಿಗಲಾರವು. ಇವುಗಳನ್ನೆಲ್ಲ ಗಮನಿಸಿಯೂ ನೀನು ಕಾರ್ಯರಂಗಕ್ಕೆ ಧುಮುಕುವ ಸಾಹಸ ಮಾಡುವೆಯಾದರೆ ನಿನಗೆ ಸ್ವಾಗತ, ನೂರು ಸಲ ಸ್ವಾಗತ... 

ಆದರೆ ಧುಮುಕುವ ಮುನ್ನ ಚೆನ್ನಾಗಿ ಪರ್ಯಾಲೋಚಿಸಿ ನೋಡು. ಒಂದು ವೇಳೆ, ನೀನು ಇಲ್ಲಿ ಮಾಡುವಷ್ಟು ಮಾಡಿಯೂ ಯಶಸ್ವಿಯಾಗದಿದ್ದರೆ, ಅಥವಾ ನಿನಗೇ ಸಾಕೆನಿಸಿದರೆ, ನನ್ನ ಕಡೆಯಿಂದ ನಿನಗೆ ಆಶ್ವಾಸನೆ ನೀಡುತ್ತೇನೆ–ನಾನು ನಿನಗೆ ನಿನ್ನ ಕೊನೆಯುಸಿರಿನವರೆಗೂ ಬೆಂಬಲವಾಗಿ ನಿಲ್ಲುತ್ತೇನೆ. ನೀನು ಭಾರತಕ್ಕಾಗಿ ಕೆಲಸ ಮಾಡು ಅಥವಾ ಬಿಡು, ನೀನು ವೇದಾಂತವನ್ನು ಅನುಸರಿಸು ಅಥವಾ ಬಿಡು, ನಾನಂತೂ ನಿನ್ನೊಂದಿಗೆ ಬೆಂಬಲವಾಗಿ ನಿಲ್ಲುತ್ತೇನೆ. ‘ಆನೆಯ ದಂತಗಳು ಹೊರಗೆ ಬರುತ್ತವೆಯೇ ಹೊರತು ಒಳಗೆಂದಿಗೂ ಹೋಗಲಾರವು’–ಇವು ಭಾಷೆಗೆ ತಪ್ಪದವನ ನುಡಿ. ನಾನಿದನ್ನು ನಿನಗೆ ಭರವಸೆ ನೀಡುತ್ತೇನೆ. ಆದರೆ ಇಲ್ಲಿ ನಿನಗೆ ಇನ್ನೊಂದು ಎಚ್ಚರಿಕೆಯನ್ನೂ ಕೊಡಬೇಕಾಗಿದೆ. ಏನೆಂದರೆ, ನೀನು ನಿನ್ನ ಕಾಲುಗಳ ಮೇಲೆ ನಿಲ್ಲಬೇಕೇ ಹೊರತು ಮಿಸ್ ಮುಲ್ಲರಳ ಅಥವಾ ಇನ್ನಾರ ರೆಕ್ಕೆಯ ಕೆಳಗೂ ಅಲ್ಲ...”

ಈ ಪತ್ರದಲ್ಲಿ ಸ್ವಾಮೀಜಿ ಮಾರ್ಗರೆಟ್ಟಳಿಗೆ ನೀಡುವ ಎಚ್ಚರಿಕೆಯನ್ನೂ ಭರವಸೆಯನ್ನೂ ಗಮನಿಸಬೇಕು. ಅವರ ಕಾರ್ಯದಲ್ಲಿ ಪಾಲ್ಗೊಳ್ಳಲು ಆಂಗ್ಲ ಯುವತಿಯೊಬ್ಬಳು ಉತ್ಸಾಹದಿಂದ ಮುಂಬರುತ್ತಿರುವಾಗ, ಅವಳನ್ನು ಒಮ್ಮೆಗೇ ಸ್ವೀಕರಿಸದೆ ಆಕೆ ಎದುರಿಸಬೇಕಾಗಬಹುದಾದ ಕಷ್ಟಗಳನ್ನು ಸ್ಪಷ್ಟವಾಗಿ ಚಿತ್ರಿಸಿ ಆಕೆ ಮತ್ತೆ ಮತ್ತೆ ಆಲೋಚಿಸುವಂತೆ ಮಾಡುತ್ತಾರೆ. ಆದರೆ ಅವಳು ಒಮ್ಮೆ ತಮ್ಮ ಕಾರ್ಯಕ್ಕಾಗಿ ಸರ್ವಸ್ವವನ್ನೂ ತ್ಯಾಗಮಾಡಿ ಬಂದಳಾದರೆ ಆಕೆಯ ಲೋಪ ದೋಷಗಳಾವುದನ್ನೂ ಗಣಿಸದೆ ಎಂದೆಂದಿಗೂ ತಾವು ಆಕೆಗೆ ಬೆಂಬಲವಾಗಿರುವ ಆಶ್ವಾಸನೆ ನೀಡುತ್ತಾರೆ. ಭಾರತಕ್ಕಾಗಿ ಬಲಿದಾನವಾಗಲು ಸಿದ್ಧರಿರುವ ವ್ಯಕ್ತಿಗಳಿಗಾಗಿ ಸ್ವಾಮೀಜಿ ಏನು ಮಾಡಲೂ ಸಿದ್ಧ.

ಹೀಗೆ ಮಾರ್ಗರೆಟ್ಟಳು ಭಾರತಕ್ಕೆ ಆಗಮಿಸುವ ಸಂಬಂಧವಾಗಿ ಆಕೆಯ ಹಾಗೂ ಸ್ವಾಮೀಜಿ ಯವರ ನಡುವೆ ಅನೇಕ ಮಹತ್ವಪೂರ್ಣ ಪತ್ರಗಳ ವಿನಿಮಯ ನಡೆಯಿತು. ಪ್ರಭುರಾಷ್ಟ್ರದವ ಳಾದ ಈ ಅಭಿಮಾನೀ ಆಂಗ್ಲ ಯುವತಿ ಸ್ವಾಮೀಜಿಯ ಆಂತರ್ಯವನ್ನು ಅರಿತಳು; ಜೊತೆಗೆ ತನ್ನ ಜೀವನೋದ್ದೇಶವನ್ನೂ ಅರಿತಳು. ಮತ್ತು ಸಂಪೂರ್ಣ ವಿಶ್ವಾಸದಿಂದ ಭಾರತದೆಡೆಗೆ ಧಾವಿಸಿದಳು.

