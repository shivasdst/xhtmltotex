
\chapter{“ಆತ್ಮನೋ ಮೋಕ್ಷಾರ್ಥಂ ಜಗದ್ಧಿತಾಯ ಚ”}

\noindent

ಸ್ವಾಮೀಜಿಯವರು ವೈದ್ಯನಾಥದಿಂದ ಕಲ್ಕತ್ತಕ್ಕೆ ಹಿಂದಿರುಗುತ್ತಿದ್ದಂತೆಯೇ ಸೋದರಿ ನಿವೇದಿತಾ ಅವರ ಶರೀರಸ್ಥಿತಿಯನ್ನು ನೋಡಿಕೊಂಡು ಹೋಗಲು ಬಂದಳು. ಆ ಸಂದರ್ಭದ ಬಗ್ಗೆ ಅವಳು ಮಿಸ್ ಮೆಕ್​ಲಾಡಳಿಗೆ ಬರೆದ ಪತ್ರವೊಂದು ತುಂಬ ಸ್ವಾರಸ್ಯಕರವಾಗಿದೆ:

“ನಿನ್ನೆ ರಾತ್ರಿತಾನೆ ಆಗಮಿಸಿದ ‘ರಾಜ’ರನ್ನು (ಸ್ವಾಮೀಜಿಯವರನ್ನು) ನೋಡಲು ನಾನು ಇಂದು ಬೆಳಿಗ್ಗೆ ಹೋಗಿದ್ದೆ. ಒಂದು ಗಂಟೆಯ ಕಾಲ ಅವರೊಂದಿಗಿದ್ದು ಇದೀಗ ಹಿಂದಿರುಗು ತ್ತಿದ್ದೇನೆ. ಸ್ವಾಮೀಜಿಯವರು, ತಾವು ವೈದ್ಯನಾಥದಲ್ಲಿ ಮೂರು ರಾತ್ರಿ ಉಸಿರಿಗಾಗಿ ಸೆಣಸಾಡ ಬೇಕಾಯಿತು ಎಂದು ನನಗೆ ಹೇಳಿದರು. ಆದರೂ ಕೂಡ ನೋಡಲು ಅವರು ದಿವ್ಯವಾಗಿ ಕಾಣು ತ್ತಿದ್ದರು. ಅವರು ಅಷ್ಟೊಂದು ಉನ್ನತ ಭಾವದಿಂದಿದ್ದುದನ್ನು ನಾನು ಹಿಂದೆಂದೂ ಕಂಡಿರಲಿಲ್ಲ.”

ಜನವರಿ ೨೬ರಂದು ಸ್ವಾಮೀಜಿಯವರು ಮೆಕ್​ಲಾಡಳಿಗೆ ತಾವೇ ಒಂದು ಪತ್ರವನ್ನು ಬರೆದರು –“ಹವಾ ಬದಲಾವಣೆಗಾಗಿ ನಾನು ವೈದ್ಯನಾಥಕ್ಕೆ ಬಂದರೂ ಅದರಿಂದ ಸ್ವಲ್ಪವೂ ಒಳ್ಳೆಯ ದಾಗಲಿಲ್ಲ. ಹಗಲೂ ರಾತ್ರಿ ಎಂಟು ದಿನ ನಾನು ಉಸಿರಿಗಾಗಿ ಒದ್ದಾಡಿದೆ. ಕಲ್ಕತ್ತಕ್ಕೆ ನನ್ನನ್ನು ಕರೆತರುವ ವೇಳೆಗೆ ನಾನು ಸತ್ತಂತೆಯೇ ಆಗಿಬಿಟ್ಟಿದ್ದೆ. ಇಲ್ಲಿ ನಾನು ಪ್ರಾಣ ಉಳಿಸಿಕೊಳ್ಳಲು ಹೋರಾಡುತ್ತಿದ್ದೇನೆ.

“ನಾನು ಮಾನಸಿಕವಾಗಿ, ಶಾರೀರಿಕವಾಗಿ ನನ್ನ ಜೀವನದುದ್ದಕ್ಕೂ ಕಷ್ಟಪಟ್ಟೆ. ಆದರೆ ಜಗ ನ್ಮಾತೆಯ ಕೃಪೆ ನನ್ನ ಮೇಲೆ ಅಪಾರವಾಗಿದೆ. ಅವಳಿಂದ ನನಗೆ ದೊರೆತ ಆನಂದ ನನ್ನ ಯೋಗ್ಯತೆಗೆ ಎಷ್ಟೋ ಪಾಲು ಮೀರಿದ್ದು. ನಾನು ನನ್ನ ಜಗನ್ಮಾತೆಯ ಕಾರ್ಯ ಯಶಸ್ವಿಯಾಗು ವಂತೆ ಹೋರಾಡುತ್ತೇನೆ. ಜಗನ್ಮಾತೆ ಯಾವಾಗಲೂ ನಾನು ಹೋರಾಡುತ್ತಿರುವುದನ್ನೇ ನೋಡ ಬಯಸುತ್ತಾಳೆ. ಮತ್ತು ನಾನು ಕೊನೆಯುಸಿರೆಳೆಯುವುದು ರಣರಂಗದಲ್ಲೇ.”

ಫೆಬ್ರುವರಿಯ ವೇಳೆಗೆ ಸ್ವಾಮೀಜಿಯವರ ಆರೋಗ್ಯ ಸುಧಾರಿಸತೊಡಗಿತು. ಆದ್ದರಿಂದ ಅವರು ಮತ್ತೊಮ್ಮೆ ತಮ್ಮ ಸಂನ್ಯಾಸಿ-ಬ್ರಹ್ಮಚಾರಿ ಶಿಷ್ಯರಿಗೆ ಶಿಕ್ಷಣ ನೀಡುವ ಕಾರ್ಯದಲ್ಲಿ ನಿರತರಾದರು. ಆದರೆ ಹೀಗೆ ಶಿಕ್ಷಣ ನೀಡುವ ಕಾರ್ಯವು ಪ್ರತ್ಯಕ್ಷವಾಗಿಯೋ ಪರೋಕ್ಷ ವಾಗಿಯೋ–ಅವರಿಂದ ನಿರಂತರವಾಗಿ ನಡೆಯುತ್ತಲೇ ಇತ್ತು ಎನ್ನಬೇಕು. ಅವರೊಂದಿಗೆ ಕಳೆದ ಪ್ರತಿಯೊಂದು ಕ್ಷಣವೂ ಇತರರಿಗೆ ತುಂಬ ಉಪಯುಕ್ತವಾದದ್ದು, ಬೋಧಪ್ರದವಾದದ್ದು. ಕೆಲವೊಮ್ಮೆ ಅವರು ತಮ್ಮ ಶಿಷ್ಯರಿಗೆ ತಮಗಾಗಿ ಅಡಿಗೆ ಮಾಡುವಂತೆ ಹೇಳುತ್ತಿದ್ದರು. ಆಶ್ರಮವಾಸಿಗಳು ಒಳ್ಳೆಯ ಅಡಿಗೆ ಮಾಡುವುದನ್ನೂ ಬಲ್ಲವರಾಗಿರಬೇಕೆಂದು ಅವರ ಅಭಿಮತ. ಇನ್ನು ಕೆಲವು ಸಲ ಅವರು ತಮ್ಮ ಶಿಷ್ಯರಿಗೆ ಯಾವುದಾದರೂ ಕೆಲಸವನ್ನು ಒಪ್ಪಿಸಿ, ಅದನ್ನು ಸಮಯಕ್ಕೆ ಸರಿಯಾಗಿ ಹಾಗೂ ಅತ್ಯಂತ ನಿಖರತೆಯಿಂದ ಮಾಡುವಂತೆ ಹೇಳುತ್ತಿದ್ದರು. ಮಾಡುವ ಕೆಲಸ ಯಾವುದೇ ಆದರೂ ಅತ್ಯಂತ ಉನ್ನತ ಮಟ್ಟದ್ದಾಗಿರಬೇಕು ಎಂಬುದು ಅವರ ಆಶಯ. ಸಾಧುಗಳೆಲ್ಲರೂ ಏಕಾಗ್ರತೆಯಿಂದ ಹಾಗೂ ಕೌಶಲದಿಂದ ಕೆಲಸ ಮಾಡುವುದನ್ನು ಕಲಿಯಬೇಕು ಎಂದು ಸ್ವಾಮೀಜಿ ಒತ್ತಿ ಹೇಳುತ್ತಿದ್ದರು. ಕೆಲಸವನ್ನು ಹೇಗೆ ಮಾಡಬೇಕು ಎಂಬ ವಿಷಯದಲ್ಲಿ ಅವರು ಪವಾಹಾರಿ ಬಾಬಾರ ಆದರ್ಶವನ್ನು ಮುಂದಿಡುತ್ತ ಹೇಳುತ್ತಾರೆ, ಜೀವನದ ಅತಿ ಸಾಧಾರಣ ಕೆಲಸ ಕಾರ್ಯಗಳಲ್ಲೂ ಏಕಾಗ್ರತೆ ಹಾಗೂ ಪರಿಪೂರ್ಣತೆಗಳು ಎದ್ದುಕಾಣು ವಂತಿರಬೇಕು ಎಂದು. ಅವರು ಆಗಾಗ ಹೇಳುತ್ತಿದ್ದರು, “ಯಾರಿಗೆ ಹುಕ್ಕಾದ ಚಿಲುಮೆಯನ್ನು ಸರಿಯಾದ ರೀತಿಯಲ್ಲಿ ತಂಬಾಕು ಹಾಕಿ ತುಂಬಿಸಲು ಬರುತ್ತದೆಯೋ ಅವರಿಗೆ ಸರಿಯಾದ ರೀತಿ ಯಲ್ಲಿ ಧ್ಯಾನ ಮಾಡಲೂ ಬರುತ್ತದೆ. ಮತ್ತು, ಯಾವನು ಸರಿಯಾದ ರೀತಿಯಲ್ಲಿ ಅಡಿಗೆ ಮಾಡಲು ಕಲಿತುಕೊಳ್ಳಲಾರನೋ ಅಂಥವನು ಒಬ್ಬ ಪರಿಪೂರ್ಣ ಸಂನ್ಯಾಸಿಯೂ ಆಗಲಾರ. ಅಡಿಗೆ ಮಾಡುವಾಗ ಶುದ್ಧ ಮನಸ್ಸು ಹಾಗೂ ಏಕಾಗ್ರತೆ ಇಲ್ಲದೆ ಹೋದರೆ ಅಡಿಗೆ ರುಚಿಯಾಗಿರುವುದಿಲ್ಲ.”

ಸ್ವಾಮೀಜಿಯವರು ಆಶ್ರಮವಾಸಿಗಳೆಲ್ಲರಿಗೂ ಉಪನ್ಯಾಸಾದಿಗಳನ್ನು ಮಾಡುವುದರಲ್ಲಿ ತರಬೇತಿ ಕೊಡುತ್ತಿದ್ದರು. ಪ್ರತಿಯೊಬ್ಬನೂ ಸ್ವಾಮೀಜಿ ಮತ್ತು ಇತರ ಸಂನ್ಯಾಸಿಗಳ ಹಾಗೂ ಗೃಹೀಭಕ್ತರ ಮುಂದೆ ಆಶುಭಾಷಣಗಳನ್ನು ಮಾಡಬೇಕಾಗಿತ್ತು. ಅವರಲ್ಲಿ ಕೆಲವರು ಸಭಾಕಂಪ ದಿಂದಾಗಿ ಭಾಷಣ ಮಾಡಲು ಹಿಂದೇಟು ಹಾಕಿದರೂ ಸ್ವಾಮೀಜಿ ಬಿಡುತ್ತಿರಲಿಲ್ಲ; ಅವರಲ್ಲಿ ಆತ್ಮವಿಶ್ವಾಸವನ್ನು ತುಂಬಿ ಮಾತನಾಡುವಂತೆ ಮಾಡುತ್ತಿದ್ದರು. ಸ್ವತಃ ಅವರ ಗುರುಭಾಯಿ ಗಳಲ್ಲೂ ಕೆಲವರು ಅವರ ಒತ್ತಾಯಕ್ಕೆ ಮಣಿದು ಮಾತನಾಡಬೇಕಾಗುತ್ತಿತ್ತು. ಆದರೆ ಸ್ವಾಮೀಜಿ ಯವರ ಪ್ರೋತ್ಸಾಹದಿಂದ ಧೈರ್ಯಗೊಂಡ ಶಿಷ್ಯರು, ಒಮ್ಮೆ ಬಿಸಿಯೇರಿದ ಮೇಲೆ, ಶಾಸ್ತ್ರಗಳಿಗೆ ಹಾಗೂ ಸಾಧನೆಗೆ ಸಂಬಂಧಿಸಿದಂತೆ ಹಲವಾರು ವಿಷಯಗಳ ಬಗ್ಗೆ ಸುಲಲಿತವಾಗಿ ಮಾತನಾಡು ತ್ತಿದ್ದರು. ಭಾಷಣ ಮುಗಿದಾಗ ಸ್ವಾಮೀಜಿ ಚಪ್ಪಾಳೆ ತಟ್ಟಿಯೋ ‘ಶಹಬಾಸ್​’ ಎನ್ನುತ್ತಲೋ ಅವರನ್ನು ಪ್ರೋತ್ಸಾಹಿಸುತ್ತಿದ್ದರು. ಸ್ವಾಮಿ ಶುದ್ಧಾನಂದರ ಬಗ್ಗೆ ಅವರು, “ಮುಂದೆ ಅವನೊಬ್ಬ ಶ್ರೇಷ್ಠ ವಾಗ್ಮಿಯಾಗುತ್ತಾನೆ” ಎಂದರು. ತಮ್ಮ ಶಿಷ್ಯರ ಪ್ರತಿಯೊಂದು ಸಣ್ಣಪುಟ್ಟ ಉತ್ತಮ ಅಂಶವನ್ನೂ ಶ್ಲಾಘಿಸಿ ಉತ್ತೇಜಿಸುವುದು ಅವರ ಸ್ವಭಾವವಾಗಿತ್ತು. ಅವರ ಈ ಒಂದು ಸದ್ಗುಣ ದೊಂದಿಗೆ ಜೊತೆಜೊತೆಯಾಗಿಯೇ, ತಮ್ಮೊಂದಿಗಿದ್ದ ಪ್ರತಿಯೊಬ್ಬರನ್ನೂ ಮಹಾನ್ ವ್ಯಕ್ತಿಗಳ ನ್ನಾಗಿಸುವ, ಯಾವುದೇ ಸಾಹಸಕಾರ್ಯದಲ್ಲಿ ತೊಡಗಬಲ್ಲಂತಹ ಧೈರ್ಯವಂತರನ್ನಾಗಿಸುವ ಒಂದು ಅದ್ಭುತ ಶಕ್ತಿಯೂ ಅವರಲ್ಲಿತ್ತು. ಶಿಷ್ಯರು ತಮ್ಮ ಕರ್ತವ್ಯಗಳನ್ನು ನೆರವೇರಿಸುವಲ್ಲಿ ಯಶಸ್ವಿಗಳಾಗಲಿ ಅಥವಾ ಆಗದಿರಲಿ, ಸ್ವಾಮಿಜಿಯವರ ಕಡೆಯಿಂದ ಮಾತ್ರ ಸದಾ ಅವರಿಗೆ ಸಿಗುತ್ತಿದ್ದುದು ಮನ್ನಣೆ ಪ್ರೋತ್ಸಾಹಗಳೇ. ಏಕೆಂದರೆ ಸ್ವಾಮೀಜಿಯವರು ತಮ್ಮ ಶಿಷ್ಯರನ್ನು ಅಳೆಯುತ್ತಿದ್ದುದು ಅವರವರ ಸಾಧನೆಗಳಿಂದಲ್ಲ. ಬದಲಾಗಿ ಅವರೆಲ್ಲ ತಮ್ಮ ಪಾಲಿನ ಕರ್ತವ್ಯ ವನ್ನು ಯಾವ ಭಾವದಿಂದ ಮಾಡುತ್ತಿದ್ದಾರೆ ಎಂಬುದರ ಆಧಾರದ ಮೇಲೆ; ಅವರು ಧೈರ್ಯ ಮಾಡಿ ತಮ್ಮ ಕೈಲಾದಷ್ಟನ್ನು ಸಾಧಿಸಿದರಲ್ಲ, ಅಷ್ಟು ಸಾಕು ಎಂದು. ಒಂದು ರೀತಿಯಲ್ಲಿ ಸ್ವಾಮೀಜಿ ಅವರನ್ನೆಲ್ಲ ಈಜಿನಲ್ಲಿ ಪರಿಣತರನ್ನಾಗಿ ಮಾಡಲು ಅವರನ್ನು ಆಳವಾದ ನೀರಿಗೆ ತಳ್ಳಿ ಬಿಡುತ್ತಿದ್ದರು–ಎಂದರೆ, ಅವರೆಲ್ಲ ಹೆಚ್ಚಿನ ಜವಾಬ್ದಾರಿಗಳನ್ನು ಹೊತ್ತು ಅವುಗಳನ್ನು ನಿರ್ವಹಿ ಸಲು ಬೇಕಾದ ಸಾಮರ್ಥ್ಯವನ್ನು ಉಂಟುಮಾಡುತ್ತಿದ್ದರು. ಸ್ವಾಮೀಜಿಯವರಿಗೆ ಮಾನವನ ಅಂತಶ್ಶಕ್ತಿಯಲ್ಲಿ ಅಪಾರ ವಿಶ್ವಾಸ. ತಮ್ಮ ಅನುಯಾಯಿಗಳನ್ನು ಅವರು ತಮ್ಮ ಅಪ್ರತಿಹತ ವಾದ, ಅಗ್ನಿಸದೃಶವಾದ ವಾಗ್ವೈಖರಿಯಿಂದ ಸ್ಫೂರ್ತಿಗೊಳಿಸುತ್ತಿದ್ದರು. ಅಲ್ಲದೆ ಅವರು ಹೇಳು ತ್ತಿದ್ದರು–“ನನ್ನಂತೆಯೇ ನೀವೂ ಕೂಡ ಇತರರನ್ನು ಸ್ಫೂರ್ತಿಗೊಳಿಸಲು ಸಮರ್ಥರಾಗಿದ್ದೀರಿ” ಎಂದು. ಅವರು ತಮ್ಮ ಶಿಷ್ಯರಲ್ಲಿ ಅಣುವಿನಷ್ಟು ಸದ್ಗುಣವನ್ನು ಕಂಡರೂ ಅದನ್ನೆ ಮಹಾ ಪರ್ವತವೆಂಬಂತೆ ಭಾವಿಸಿದರೆ, ಶಿಷ್ಯರ ಪರ್ವತದಷ್ಟು ದೊಡ್ಡ ದೋಷ ದೌರ್ಬಲ್ಯಗಳನ್ನೂ ಕೇವಲ ಅಣುಗಾತ್ರವೆಂಬಂತೆ ಪರಿಗಣಿಸುತ್ತಿದ್ದರು. ಈ ಮೂಲಕ ಅಲ್ಲಿ ಅವರು ಒಂದು ಅದ್ಭುತ ದಿವ್ಯ ಬಾಂಧವ್ಯವನ್ನು ಬೆಳೆಸಿದ್ದುದರಿಂದ ಅವರಾಡಿದ ಪ್ರತಿಯೊಂದು ಮಾತೂ, ಅವರ ಪ್ರತಿ ಯೊಂದು ಭಾವನೆಯೂ ಶಕ್ತಿಭರಿತವಾಗಿರುತ್ತಿದ್ದುವು; ಶಿಷ್ಯರ ದೃಷ್ಟಿಕೋನವನ್ನು ವಿಶಾಲ ಗೊಳಿಸುತ್ತಿದ್ದುವು. ಸ್ವಾಮೀಜಿ ಮಠದಲ್ಲಿದ್ದ ಆ ದಿನಗಳಲ್ಲಿ ನಿರ್ಮಾಣಗೊಂಡಿದ್ದ ವಾತಾವರಣ ಇಂಥದು.

ಮಠದ ಕಾರ್ಯಕಲಾಪಗಳಷ್ಟನ್ನೂ ಸ್ವಾಮೀಜಿ ತಾವೇ ನೋಡಿಕೊಳ್ಳದೆ, ತಮ್ಮ ಸೋದರ ಸಂನ್ಯಾಸಿಗಳಿಗೆ ಆ ಹೊಣೆಗಾರಿಕೆಯನ್ನು ಹಂಚಿದ್ದರು. ಇವುಗಳ ಮೇಲ್ವಿಚಾರಣೆಯನ್ನು ಸ್ವಾಮಿ ಶಾರದಾನಂದರು ಅತ್ಯಂತ ಸಮರ್ಪಕವಾಗಿ ನಿರ್ವಹಿಸುತ್ತಿದ್ದರು. ಅವರನ್ನು ಸ್ವಾಮೀಜಿ ಅಮೆರಿಕ ದಿಂದ ಕರೆಸಿಕೊಂಡದ್ದು ಈ ಉದ್ದೇಶಕ್ಕಾಗಿಯೇ. ಸ್ವಾಮಿ ಶಾರದಾನಂದರು ಅಮೆರಿಕದಲ್ಲಿನ ಕೆಲಸಕಾರ್ಯಗಳನ್ನು ಸಮರ್ಥವಾಗಿ ನಡೆಸಿಕೊಂಡು ಬರಬಲ್ಲರು, ಮತ್ತು ಅಲ್ಲಿ ಅವರ ಆವಶ್ಯಕತೆ ಬಹಳವಾಗಿಯೇ ಇದೆ ಎಂಬುದನ್ನು ಸ್ವಾಮೀಜಿ ಅರಿತಿದ್ದರು; ಆದರೆ ಮಠದ ಮುಖ್ಯ ಕೇಂದ್ರ ವನ್ನು ಸಮರ್ಪಕ ರೀತಿಯಲ್ಲಿ ಸಂಘಟಿಸುವ ಕಾರ್ಯವು ಅವರಿಗೆ ಹೆಚ್ಚಿನ ಮಹತ್ವದ್ದೂ ತ್ವರಿತದ್ದೂ ಆಗಿತ್ತು. ಅಲ್ಲದೆ ಪಾಶ್ಚಾತ್ಯ ಸಂಘಟನಾ ವಿಧಾನಗಳನ್ನೂ ಪಾಶ್ಚಾತ್ಯರ ದೃಷ್ಟಿಕೋನ ಗಳನ್ನೂ ಚೆನ್ನಾಗಿ ಅರಿತುಕೊಂಡವರೊಬ್ಬರಿಂದ ಕಿರಿಯ ಸಂನ್ಯಾಸಿ-ಬ್ರಹ್ಮಚಾರಿಗಳಿಗೆ ತರಬೇತಿ ಕೊಡಿಸುವುದು ಅವರ ಉದ್ದೇಶವಾಗಿತ್ತು. ಅಮೆರಿಕದಲ್ಲಿ ಸ್ವಾಮಿ ಅಭೇದಾನಂದರು ದಣಿವಿಲ್ಲದ ಉತ್ಸಾಹದಿಂದ, ಅತ್ಯಂತ ಯಶಸ್ವಿಯಾಗಿ ಅಲ್ಲಿನ ಕಾರ್ಯಗಳನ್ನು ನಿರ್ವಹಿಸುತ್ತಿದ್ದರು. ಆದ್ದ ರಿಂದ ಶಾರದನಂದರ ನಿರ್ಗಮನದಿಂದಾಗಿ ಅಲ್ಲಿನ ಕಾರ್ಯಕ್ಕೆ ಧಕ್ಕೆಯಾಗಲಾರದೆಂದು ಸ್ವಾಮೀಜಿ ಯವರಿಗೆ ತಿಳಿದಿತ್ತು. ಮಠಕ್ಕೆ ಮರಳಿದೊಡನೆಯೇ ಸ್ವಾಮಿ ಶಾರದಾನಂದರು ಅತ್ಯಂತ ಶ್ರದ್ಧೆ ಯಿಂದ ತಮ್ಮ ಈ ಹೊಸ ಕರ್ತವ್ಯದಲ್ಲಿ ನಿರತರಾಗಿದ್ದರು. ಅವರ ನೇತೃತ್ವದಲ್ಲಿ ಪ್ರತಿಯೊಂದು ಕಾರ್ಯಕ್ರಮವೂ ಗಡಿಯಾರದ ನಿಖರತೆಯಿಂದ ಮತ್ತು ಅತಿ ಹೆಚ್ಚಿನ ಉತ್ಸಾಹದಿಂದ ನಡೆಯ ತೊಡಗಿತು. ತುರೀಯಾನಂದರ ಸಹಕಾರದಿಂದ ಅವರು ಸಂಸ್ಕೃತಾಧ್ಯಯನ ಹಾಗೂ ಪಾಶ್ಚಾತ್ಯ- ಭಾರತೀಯ ತತ್ತ್ವಶಾಸ್ತ್ರಗಳ, ಧ್ಯಾನ-ಸಾಧನೆಗಳ ತರಗತಿಗಳನ್ನು ನಡೆಸುತ್ತಿದ್ದರು. ಇವೆಲ್ಲವೂ ಸ್ವಾಮೀಜಿಯವರ ಆದೇಶಗಳಿಗೆ ಅನುಸಾರವಾಗಿ ನಡೆದಿದ್ದುವು. ಪ್ರತಿಯೊಬ್ಬರಿಗೂ ಅವರವರ ಕ್ಷೇತ್ರಗಳಲ್ಲಿ ತೀರ್ಮಾನಗಳನ್ನು ತೆಗೆದುಕೊಳ್ಳುವ ಸ್ವಾತಂತ್ರ್ಯವನ್ನೂ ಜವಾಬ್ದಾರಿಯನ್ನೂ ಹೊರಿ ಸಿದ ಹೊರತು ಅವರಾರೂ ತಮ್ಮ ಕಾಲುಗಳ ಮೇಲೆ ನಿಲ್ಲಲಾರರು ಅಥವಾ ಹೃತ್ಪೂರ್ವಕವಾಗಿ ಕರ್ತವ್ಯವನ್ನು ನಿರ್ವಹಿಸಲಾರರು ಎಂಬುದು ಸ್ವಾಮೀಜಿಯವರ ನಿಲುವು. ಆಶ್ರಮವಾಸಿಗಳೆಲ್ಲ ಸಂಘಟಿತರಾಗಿ, ಪ್ರತಿ ತಿಂಗಳೂ ತಮ್ಮಲ್ಲೇ ಒಬ್ಬರನ್ನು ಮೇಲಧಿಕಾರಿಯನ್ನಾಗಿ ಚುನಾಯಿಸು ತ್ತಿದ್ದರು. ಮಠದ ಪ್ರತಿದಿನದ ಕಾರ್ಯಕಲಾಪಗಳು ಸರಿಯಾದ ರೀತಿಯಲ್ಲಿ ನಡೆಯುವಂತೆ ನೋಡಿಕೊಳ್ಳುವುದು ಈ ಮೇಲಧಿಕಾರಿಗಳ ಹೊಣೆ. ಇವರು ಪ್ರತಿಯೊಬ್ಬರಿಗೂ ಅವರವರ ಶಕ್ತಿ ಸಾಮರ್ಥ್ಯಗಳಿಗೆ ಅನುಸಾರವಾಗಿ ಕೆಲಸ ಕಾರ್ಯಗಳನ್ನು ಹಂಚುತ್ತಿದ್ದರು. ಪ್ರತಿಯೊಂದು ಕೆಲಸವೂ ಸಮರ್ಪಕವಾಗಿ, ಸಕಾಲದಲ್ಲಿ ನಡೆಯುವಂತೆ ನೋಡಿಕೊಳ್ಳುವುದು, ಪ್ರತಿಯೊಂದು ವಸ್ತುವೂ ಚೊಕ್ಕಟವಾಗಿ ಅದರದರ ಸ್ಥಾನದಲ್ಲಿ ಇರುವಂತೆ ನೋಡಿಕೊಳ್ಳುವುದು, ಅನಾರೋಗ್ಯ ಗೊಂಡವರ ಶುಶ್ರೂಷೆಗೆ ವ್ಯವಸ್ಥೆ ಮಾಡುವುದು–ಇವೆಲ್ಲ ಆ ಮೇಲಧಿಕಾರಿಗಳ ಹೊಣೆ. ಆಶ್ರಮವಾಸಿಗಳಲ್ಲಿ ಕೆಲಕೆಲವರು ಸರದಿಯ ಪ್ರಕಾರ ಇತರ ಕೆಲಸಗಳಿಂದ ಮುಕ್ತರಾಗಿ ಹೆಚ್ಚಿನ ಸಮಯವನ್ನು ಸಾಧನೆಗೆ ಬಳಸಿಕೊಳ್ಳಲೂ ಇವರು ಅನುವು ಮಾಡಿಕೊಡುತ್ತಿದ್ದರು. ಹೀಗೆ ಮಠದ ಕಾರ್ಯಕಲಾಪಗಳೆಲ್ಲ ಸುವ್ಯವಸ್ಥಿತವಾಗಿ ನಡೆಯುವಂತಾದುದು ಸ್ವಾಮೀಜಿಯವರಿಗೆ ತುಂಬ ತೃಪ್ತಿ-ಸಂತೋಷಗಳನ್ನುಂಟುಮಾಡಿತು.

ಈ ದಿನಗಳಲ್ಲಿ ಸ್ವಾಮೀಜಿಯವರು ವಿಶೇಷವಾಗಿ ಒಬ್ಬ ಸಂನ್ಯಾಸೀ ನಾಯಕನಂತೆ ವ್ಯಕ್ತವಾಗು ತ್ತಿದ್ದರು. ಅವರು ತಮ್ಮ ಶಿಷ್ಯರಿಗೆ ಸಂನ್ಯಾಸಜೀವನದ ಆದರ್ಶಗಳನ್ನು ಮತ್ತು ಅವುಗಳ ಆಚರಣಾವಿಧಾನವನ್ನು ಸತತವಾಗಿ ಬೋಧಿಸುತ್ತಿದ್ದರು. ಎಲ್ಲ ಸಾಧು ಬ್ರಹ್ಮಚಾರಿಗಳನ್ನು ಸೇರಿಸಿ, ಬ್ರಹ್ಮಚರ್ಯಜೀವನದ ಹಾಗೂ ಸಂನ್ಯಾಸಜೀವನದ ಜವಾಬ್ದಾರಿಗಳನ್ನು ನೆನಪಿಸಿಕೊಡುತ್ತಿದ್ದರು; ಆ ಜೀವನದ ಹಿರಿಮೆ ಗರಿಮೆಗಳನ್ನು, ಅದರ ಅಪಾರ ಸಾಮರ್ಥ್ಯ ಸಾಧ್ಯತೆಗಳನ್ನು ಬಣ್ಣಿಸು ತ್ತಿದ್ದರು. “ನಿಮ್ಮ ನರನಾಡಿಗಳಲ್ಲೆಲ್ಲ ಬ್ರಹ್ಮಚರ್ಯದ ತೇಜಸ್ಸು ಅಗ್ನಿಯಂತೆ ಪ್ರಜ್ವಲಿಸು ತ್ತಿರಬೇಕು” ಎಂದು ಅವರು ಹೇಳುತ್ತಿದ್ದರು. ‘ಆತ್ಮನೋ ಮೋಕ್ಷಾರ್ಥಂ ಜಗದ್ಧಿತಾಯ ಚ’ ಎಂಬುದೇ ಅವರೆಲ್ಲರ ಆದರ್ಶ ಎಂಬುದನ್ನು ನೆನಪಿಸುತ್ತಿದ್ದರು. ಸ್ವಾಮೀಜಿಯವರ ಪಾಲಿಗೆ, ವ್ಯಕ್ತಿಯ ಅಹಮಿಕೆಯು ನಿರಾಕಾರಬ್ರಹ್ಮದಲ್ಲಿ ಲೀನವಾಗುವವರೆಗೆ ತನ್ನೆಲ್ಲವನ್ನೂ ವಿಶ್ವಕಲ್ಯಾಣ ಕ್ಕಾಗಿ ತ್ಯಾಗ ಮಾಡುವುದೇ ಸಂನ್ಯಾಸ ಜೀವನದ ಅರ್ಥ. ಈ ಆದರ್ಶಗಳನ್ನು ಅವರು ಎಷ್ಟರ ಮಟ್ಟಿಗೆ ತಮ್ಮ ಜೀವನದಲ್ಲಿ ಅಳವಡಿಸಿಕೊಂಡು ಅವುಗಳ ಅನುಷ್ಠಾನ ಸಾಧ್ಯತೆಯನ್ನು ತೋರಿಸಿ ಕೊಟ್ಟರೆಂದರೆ, ಮುಂದೆ ಯಾರೂ ಅವುಗಳನ್ನು ಕೇವಲ ಕೆಲಸಕ್ಕೆ ಬಾರದ ಕಂತೆಗಳು ಎಂದು ತಳ್ಳಿಹಾಕಲು ಸಾಧ್ಯವಿರಲಿಲ್ಲ. ಒಮ್ಮೆ ಅವರು ಶ್ರದ್ಧೆಯ ಮಹತ್ವವನ್ನು ಈ ರೀತಿ ಬಣ್ಣಿಸಿದರು:

“ಈ ಇಡೀ ಜಗತ್ತಿನ ಇತಿಹಾಸವೆಂದರೆ ಆತ್ಮಶ್ರದ್ಧೆಯನ್ನು ಹೊಂದಿದ್ದ ಕೆಲವೇ ಮಂದಿಯ ಇತಿಹಾಸ. ಈ ಶ್ರದ್ಧೆ ನಮ್ಮೊಳಗಿನ ದೈವಿಕತೆಯನ್ನು ಪ್ರಕಟಿಸುತ್ತದೆ. ನೀವು ಏನನ್ನು ಬೇಕಾದರೂ ಸಾಧಿಸಬಲ್ಲಿರಿ. ನೀವು ನಿಮ್ಮೊಳಗಿನ ಅನಂತ ಶಕ್ತಿಯನ್ನು ವ್ಯಕ್ತಪಡಿಸಲು ಸಾಕಷ್ಟು ಪ್ರಯತ್ನ ಮಾಡದಿದ್ದಾಗ ಮಾತ್ರ ಸೋಲು ಕಾಣುವಿರಿ. ಯಾವಾಗ ಮನುಷ್ಯ ಆತ್ಮಶ್ರದ್ಧೆಯನ್ನು ಕಳೆದು ಕೊಳ್ಳುತ್ತಾನೋ ತಕ್ಷಣವೇ ಮೃತ್ಯು ಸಮೀಪಿಸುತ್ತದೆ.

“ಮೊದಲು ನಿಮ್ಮ ಮೇಲೆ ನಿಮಗೆ ನಂಬಿಕೆಯಿರಲಿ, ಅನಂತರ ದೇವರ ಮೇಲೆ. ಕೈಬೆರಳೆಣಿಕೆ ಯಷ್ಟು ಜನ ಶಕ್ತಿಶಾಲಿಗಳು ಈ ಜಗತ್ತನ್ನೇ ಅಲುಗಾಡಿಸಬಲ್ಲರು. ನಮಗೆ ಭಾವಿಸುವ ಹೃದಯ ಬೇಕಾಗಿದೆ. ಆಲೋಚಿಸುವ ಮೆದುಳು ಬೇಕಾಗಿದೆ. ಕೆಲಸ ಮಾಡುವ ಕೈಗಳು, ಬಲಿಷ್ಠವಾದ ಕೈಗಳು ಬೇಕಾಗಿವೆ. ಬುದ್ಧ ತನ್ನನ್ನು ತಾನು ಪ್ರಾಣಿಗಳಿಗಾಗಿ ಸಮರ್ಪಿಸಿಕೊಂಡ. ನಿಮ್ಮನ್ನು ನೀವು ಮನುಷ್ಯರಿಗಾಗಿಯಾದರೂ ಸಮರ್ಪಿಸಿಕೊಳ್ಳಿ; ಮತ್ತು ಅದಕ್ಕಾಗಿ ಕೆಲಸ ಮಾಡಲು ಯೋಗ್ಯವಾದ ಉಪಕರಣಗಳನ್ನಾಗಿಸಿಕೊಳ್ಳಿ. ಆದರೆ ನೆನಪಿಡಿ, ಕೆಲಸ ಮಾಡುವುದು ನೀವಲ್ಲ, ನಿಮ್ಮೊಳಗಿನ ದೇವರು. ಒಬ್ಬ ಮನುಷ್ಯನ ಒಳಗಡೆಯೇ ಸಮಸ್ತ ಬ್ರಹ್ಮಾಂಡವೂ ಅಡಗಿದೆ. ಒಂದು ಅಣುಕಣದ ಹಿಂದೆ ಸಮಸ್ತ ವಿಶ್ವದ ಶಕ್ತಿಯೂ ಅಡಗಿದೆ. ನಿಮ್ಮ ಹೃದಯಕ್ಕೂ ಬುದ್ಧಿಗೂ ಭಿನ್ನಾಭಿಪ್ರಾಯ ತಲೆದೋರಿದಾಗ ಹೃದಯದ ಮಾತನ್ನೇ ಕೇಳಿ.”

ಒಂದು ದಿನ ಶಿಷ್ಯರೆಲ್ಲ ಸ್ವಾಮೀಜಿಯವರೊಂದಿಗೆ ಕುಳಿತು ಸಂಭಾಷಿಸುತ್ತಿದ್ದಾಗ ಮಾತು ‘ಅಧಿಕಾರವಾದ’ದ ಕಡೆಗೆ ತಿರುಗಿತು. ಅಧಿಕಾರವಾದವೆಂದರೆ ಮೇಲ್ಜಾತಿಯವರಿಗೆ ವಿಶೇಷ ಹಕ್ಕು, ಸವಲತ್ತುಗಳನ್ನು ನೀಡುವ ಸಿದ್ಧಾಂತ. ಇದನ್ನು ಸ್ವಾಮೀಜಿ ತುಂಬ ಕಟುವಾದ ಶಬ್ದಗಳಿಂದ ಖಂಡಿಸಿದರು. ಈ ಪದ್ಧತಿಯಿಂದ ಉಂಟಾದ ಸಾಮಾಜಿಕ ದುರಾಚಾರಗಳ ಕುರಿತಾಗಿ ವಿವರಿಸಿ ದರು. ‘ಅತ್ಯುನ್ನತವಾದ ಪರಮ ಸತ್ಯಗಳನ್ನು ಭೇದಭಾವವಿಲ್ಲದೆ ಪ್ರತಿಯೊಬ್ಬರಿಗೂ ತಿಳಿಸಿಕೊಡ ಬೇಕು; ಇಂದಿನ ಕಾಲದ ಜನಜೀವನದಲ್ಲಿ ಬೇರುಬಿಟ್ಟಿರುವ ಪದ್ಧತಿಗಳನ್ನು ಲೆಕ್ಕಿಸದೆ ನನ್ನ ಅನು ಯಾಯಿಗಳು ಸತ್ಯವನ್ನೇ ಘೋಷಿಸುವ ಧೈರ್ಯಶಾಲಿಗಳಾಗಬೇಕು’ ಎಂದು ಸ್ವಾಮೀಜಿ ಒತ್ತಿ ಹೇಳಿದರು. ಬಳಿಕ ಅವರು ಗುಡುಗಿದರು, “ರಾಜಿಯ ಮಾತೇ ಬೇಡ! ತೇಪೆ ಹಾಕುವ ಕೆಲಸವೇ ಬೇಡ! ಶವವನ್ನು ಹೂಗಳಿಂದ ಮುಚ್ಚಿಡಲು ನೋಡಬೇಡಿ. ಅತ್ಯಂತ ತುಚ್ಛವಾದ ಹೇಡಿತನದಿಂ ದಾಗಿ ಮಾತ್ರ ಈ ಬಗೆಯ ಮುಚ್ಚಿಡುವ ಸ್ವಭಾವ ಉಂಟಾಗುತ್ತದೆ. ಧೈರ್ಯಶಾಲಿಗಳಾಗಿ. ಎಲ್ಲಕ್ಕಿಂತ ಹೆಚ್ಚಾಗಿ, ನನ್ನ ಪುತ್ರರು ಧೀರರಾಗಬೇಕು. ಯಾವ ಕಾರಣಕ್ಕಾಗಿಯೂ ರಾಜಿ ಮಾಡಿ ಕೊಳ್ಳಬೇಡಿ. ಅತ್ಯುನ್ನತ ಸತ್ಯಗಳನ್ನು ಘಂಟಾಘೋಷವಾಗಿ ಸಾರಿ. ನಿಮ್ಮ ಗೌರವಕ್ಕೆ ಧಕ್ಕೆ ಯುಂಟಾಗುತ್ತದೆಂದೋ ಅನಾವಶ್ಯಕವಾದ ಘರ್ಷಣೆಗೆ ಕಾರಣವಾಗಬಹುದೆಂದೋ ಹೆದರಬೇಡಿ. ಈ ರೀತಿ ಸತ್ಯವನ್ನು ಹೇಳಹೊರಟಾಗ ಎದುರಾಗುವ ಎಲ್ಲ ಆಮಿಷಗಳನ್ನೂ ನೀವು ಮೀರಿ ನಿಂತರೆ ನಿಮ್ಮಲ್ಲೊಂದು ಅಲೌಕಿಕ ಶಕ್ತಿಯುಂಟಾಗುತ್ತದೆ. ಆಗ ನೀವು ಯಾವುದನ್ನು ಅಸತ್ಯವೆಂದು ನಂಬಿದ್ದೀರೋ ಅಂತಹ ಮಾತುಗಳನ್ನು ನಿಮ್ಮೆದುರಿನಲ್ಲಿ ಆಡಲು ಜನ ಹಿಂಜರಿಯುತ್ತಾರೆ. ನೀವು ಎಡೆಬಿಡದೆ ಹದಿನಾಲ್ಕು ವರ್ಷಗಳವರೆಗೆ ಸತ್ಯವನ್ನೇ ನುಡಿಯಲು ಸಮರ್ಥರಾದರೆ ಜನ ನೀವು ಹೇಳುವುದನ್ನು ನಂಬುತ್ತಾರೆ. ತನ್ಮೂಲಕ ನೀವು ಜನಕೋಟಿಯ ಮೇಲೆ ಅತ್ಯುನ್ನತವಾದ ಕೃಪಾಶೀರ್ವಾದವನ್ನು ಹರಿಯಿಸುವಂತಾಗುವಿರಿ; ಅವರ ಶೃಂಖಲೆಗಳನ್ನು ಕಳಚಬಲ್ಲವರಾಗು ವಿರಿ; ಇಡೀ ರಾಷ್ಟ್ರವನ್ನೇ ಮೇಲೆತ್ತಬಲ್ಲವರಾಗುವಿರಿ.”

ಸ್ವಾಮೀಜಿ ಮತ್ತೆ ಮತ್ತೆ ಹೇಳುತ್ತಿದ್ದ ಒಂದು ಮಾತೆಂದರೆ ಮಹಾ ಸಂನ್ಯಾಸಿಯೊಬ್ಬ ಮಾತ್ರ ಮಹಾ ಕೆಲಸಗಾರನಾಗಬಲ್ಲ ಎಂಬುದು. “ಯಾರು ನಿಜವಾಗಿಯೂ ನಿರ್ಲಿಪ್ತರಾಗಿದ್ದಾರೋ ಅವರಿಂದ ಮಾತ್ರವೇ ಜಗತ್ತಿಗೆ ಅತಿ ಹೆಚ್ಚಿನ ಸೇವೆ ಸಲ್ಲುತ್ತದೆ. ಬುದ್ಧ ಅಥವಾ ಕ್ರಿಸ್ತನಿಗಿಂತ ಉತ್ತಮ ಕರ್ಮಯೋಗಿ ಎಂಬ ಹೆಗ್ಗಳಿಕೆ ಬೇರೆ ಯಾರ ಪಾಲಿಗೆ ಬಂದೀತು?” ಎಂದು ಅವರು ಉದ್ಗರಿಸುತ್ತಿದ್ದರು. ಅವರ ದೃಷ್ಟಿಯಲ್ಲಿ ಯಾವ ಕೆಲಸವೂ ಕೇವಲ ಲೌಕಿಕವಾದುದಲ್ಲ. ಎಲ್ಲ ಕೆಲಸವೂ ಪವಿತ್ರವಾದದ್ದು, ಭಗವಂತನ ಪೂಜೆಗೆ ಸಮನಾದದ್ದು ಎಂಬುದು ಅವರ ದೃಷ್ಟಿ.

ಸಂನ್ಯಾಸಿಗಳು ಯಾವ ಬಗೆಯ ಸೇವಾಕಾರ್ಯಗಳನ್ನು ಕೈಗೆತ್ತಿಕೊಳ್ಳಬಹುದು ಎಂಬುದರ ಬಗ್ಗೆ ಸ್ವಾಮೀಜಿ ವಿವರವಾಗಿ ಹೇಳುತ್ತಾರೆ. ದರಿದ್ರರಿಗೆ ಅನ್ನದಾನ ಮಾಡುವುದು, ಕ್ಷಾಮಪರಿಸ್ಥಿತಿ ಯಲ್ಲಿ ಪರಿಹಾರಕಾರ್ಯವನ್ನು ನಡೆಸುವುದು, ರೋಗಿಗಳ ಶುಶ್ರೂಷೆ ಮಾಡುವುದು, ಅಂಟು ರೋಗ ಪೀಡಿತವಾದ ಸ್ಥಳಗಳಲ್ಲಿ ನಿರ್ಮಲೀಕರಣದ ವ್ಯವಸ್ಥೆಯನ್ನು ಮಾಡಿಸುವುದು, ಅನಾಥಾ ಲಯಗಳನ್ನು ಮತ್ತು ಆಸ್ಪತ್ರೆಗಳನ್ನು ತೆರೆಯುವುದು, ಶಿಕ್ಷಣ ಹಾಗೂ ತರಬೇತಿಯ ಕೇಂದ್ರಗಳನ್ನು ಸ್ಥಾಪಿಸುವುದು–ಇವು ಅವುಗಳಲ್ಲಿ ಕೆಲವು. (ಇವೆಲ್ಲವೂ ಈಗಾಗಲೇ ರಾಮಕೃಷ್ಣ ಮಿಷನ್ನಿನಲ್ಲಿ ಕಾರ್ಯರೂಪಕ್ಕೆ ಬಂದಿವೆ.) ಮಠಗಳಲ್ಲಿ ಸಾಧುಗಳ ಆಧ್ಯಾತ್ಮಿಕವಾದ ಹಾಗೂ ಬುದ್ಧಿಪ್ರಧಾನ ವಾದ ಜೀವನವನ್ನು ನಡೆಸುವುದರೊಂದಿಗೆ ಸಂಗೀತ, ತೋಟಗಾರಿಕೆ ಮುಂತಾದವುಗಳ ಬಗ್ಗೆ ತಕ್ಕಮಟ್ಟಿನ ಜ್ಞಾನವನ್ನು ಪಡೆದುಕೊಂಡಿರಬೇಕೆಂದು ಅವರೆನ್ನುತ್ತಿದ್ದರು. ಕೆಲವೊಮ್ಮೆ ಅವರು ಶಿಷ್ಯರಿಗೆ ಸಮೂಹಗಾನವನ್ನು ಹೇಳಿಕೊಡುತ್ತಿದ್ದರು. ದೈಹಿಕ ವ್ಯಾಯಾಮದ ಆವಶ್ಯಕತೆಯನ್ನು ಎತ್ತಿಹಿಡಿಯುತ್ತ ಅವರು ಹೇಳುತ್ತಾರೆ, “ಧಾರ್ಮಿಕ ಸೈನ್ಯದಲ್ಲಿ ನನಗೆ ಗಣಿ ಕೆಲಸಗಾರರು, ಸುರಂಗ ತೋಡುವವರು (ಶ್ರಮಜೀವಿಗಳು) ಬೇಕಾಗಿದ್ದಾರೆ! ಆದ್ದರಿಂದ ನನ್ನ ಪುತ್ರರೇ, ನಿಮ್ಮ ಮಾಂಸಖಂಡಗಳನ್ನು ತರಬೇತುಗೊಳಿಸುವ ಕಾರ್ಯದಲ್ಲಿ ತೊಡಗಿ, ತಪಸ್ವಿಗಳಿಗಾದರೆ ಉಪ ವಾಸಾದಿ ದೇಹದಂಡನೆಯ ಕಾರ್ಯ. ಆದರೆ ನಮ್ಮಂತಹ ಕೆಲಸಗಾರರಿಗೆ ಬೇಕಾದದ್ದು ದೃಢವಾದ ಶರೀರಗಳು, ಕಬ್ಬಿಣದ ಮಾಂಸಖಂಡಗಳು, ಉಕ್ಕಿನ ನರಗಳು!”

ಸ್ವಾಮೀಜಿ ಆಗಾಗ ಹೇಳುತ್ತಲೇ ಇದ್ದ ಮತ್ತೊಂದು ಪ್ರಮುಖ ಅಂಶವೆಂದರೆ, ಅಖಂಡ ಬ್ರಹ್ಮಚರ್ಯದಿಂದೊಡಗೂಡಿದ ಸಂಪೂರ್ಣ ತ್ಯಾಗವೇ ಭಗವತ್ಸಾಕ್ಷಾತ್ಕಾರಕ್ಕೆ ಕೀಲಿಕೈ ಎಂಬುದು. ಸಂನ್ಯಾಸ ಜೀವನವೆಂದರೆ ತನ್ನೊಳಗಿನ ಪ್ರಕೃತಿಯೊಂದಿಗೆ ನಿರಂತರ ಹೋರಾಟ, ತನ್ನೊಳಗೆ ಇರುವ ಉಚ್ಚ ಹಾಗೂ ನೀಚ ಮನೋಭಾವಗಳ ನಡುವಿನ ಹೋರಾಟ;ಇದರಲ್ಲಿ ವಿಜಯಿಯಾಗಿ ಹೊರಬರಬೇಕಾದರೆ ತೀವ್ರ ತಪಸ್ಸು, ಆತ್ಮನಿಗ್ರಹ ಮತ್ತು ಏಕಾಗ್ರತೆಗಳನ್ನು ಅಭ್ಯಾಸ ಮಾಡಲೇ ಬೇಕು ಎನ್ನುತ್ತಿದ್ದರು, ಸ್ವಾಮೀಜಿ. ಯಾರಾದರೊಬ್ಬರು ಏಕಾಂತದಲ್ಲಿ ಧ್ಯಾನ ಜಪ ತಪಾದಿಗಳನ್ನು ಅಭ್ಯಾಸ ಮಾಡುವುದನ್ನು ಕಂಡರೆ ಅವರಿಗೆ ತುಂಬ ಸಂತೋಷ. ಒಮ್ಮೆ ಒಬ್ಬ ಶಿಷ್ಯ ಹೀಗೆಯೇ ಮಾತಿನ ಸಂದರ್ಭದಲ್ಲಿ ಯಾವುದೋ ಒಂದು ಪ್ರಾಪಂಚಿಕ ವಿಷಯವನ್ನು ಕುರಿತು ಪ್ರಶ್ನೆ ಕೇಳಿದ. ಇದಕ್ಕೆ ಉತ್ತರ ನೀಡುವ ಬದಲು ಸ್ವಾಮೀಜಿ ವ್ಯಗ್ರವಾಗಿ ಹೇಳಿದರು, “ಹೋಗು, ಕೆಲಕಾಲ ತಪಸ್ಸು ಮಾಡಿ ನಿನ್ನ ಮನಸ್ಸನ್ನು ಸ್ವಲ್ಪ ಶುದ್ಧಗೊಳಿಸಿಕೊ! ಆಮೇಲೆ ನೀನು ಇಂತಹ ವಿಕೃತ ಪ್ರಶ್ನೆಗಳನ್ನು ಕೇಳಲಾರೆ.”

ಸಾಧು-ಬ್ರಹ್ಮಚಾರಿಗಳು ತಮ್ಮ ಪ್ರಾರಂಭಿಕ ಹಂತಗಳಲ್ಲಿ ತಮ್ಮನ್ನು ತೀವ್ರವಾದ ಶಿಸ್ತಿಗೆ ಒಳಪಡಿಸಿಕೊಳ್ಳಬೇಕು ಮತ್ತು ಆಹಾರಾದಿಗಳ ವಿಚಾರದಲ್ಲಿನ ನಿಯಮಗಳನ್ನು ಕಟ್ಟುನಿಟ್ಟಾಗಿ ಪಾಲಿಸಬೇಕು ಎಂಬುದರ ಬಗ್ಗೆ ಸ್ವಾಮೀಜಿ ವಿಶೇಷ ಗಮನ ಕೊಡುತ್ತಿದ್ದರು. ಡಿಸೆಂಬರಿನಲ್ಲಿ ಅವರು ವೈದ್ಯನಾಥಕ್ಕೆ ಹೊರಡುವ ಹಿಂದಿನ ದಿನ ಆಶ್ರಮವಾಸಿಗಳನ್ನೆಲ್ಲ ಸೇರಿಸಿ ಒಂದು ದೀರ್ಘವಾದ ಸಭೆಯನ್ನು ಕರೆದು ಮಠದ ಕಿರಿಯ ಸದಸ್ಯರಿಗೆ ಆಹಾರಾದಿಗಳ ವಿಷಯದಲ್ಲಿ ಸ್ಪಷ್ಟ ಆದೇಶಗಳನ್ನು ಕೊಟ್ಟಿದ್ದರು–“ಆಹಾರದ ಮೇಲೆ ಹತೋಟಿಯನ್ನು ಸಾಧಿಸದೆ ಮನಸ್ಸಿನ ಹತೋಟಿ ಸಾಧ್ಯವಿಲ್ಲ. ಅತಿಯಾಗಿ ತಿನ್ನುವುದರಿಂದ ದೇಹ ಮನಸ್ಸುಗಳೆರಡೂ ಹಾಳಾಗುತ್ತವೆ. ಮತ್ತು ನೀವಿರುವ ಈಗಿನ ಸ್ಥಿತಿಯಲ್ಲಿ ಹಿಂದೂಗಳಲ್ಲದವರು ತಯಾರಿಸಿದ ಆಹಾರವನ್ನು ಮುಟ್ಟ ಬಾರದು... ನೀವು ನಿಮ್ಮ ಇಷ್ಟದೇವತೆಯ ಬಗೆಗಿನ ನಿಷ್ಠೆಯನ್ನು ಕದಲಿಸಬಾರದು; ಜೊತೆಗೇ ಕೂಪಮಂಡೂಕಗಳೂ ಮತಾಂಧರೂ ಆಗಬಾರದು. ಬ್ರಹ್ಮಚರ್ಯಜೀವನದಲ್ಲಿ ನೀವು ದೃಢ ವಾಗಿ ನಿಲ್ಲಬೇಕು. ಆದರೆ ಸಂನ್ಯಾಸಜೀವನದ ಈ ಉನ್ನತ ಆದರ್ಶಗಳಿಗೂ ಕಠಿಣ ಶಿಸ್ತಿಗೂ ಹೊಂದಿಕೊಂಡು ಬಾಳಲು ನಿಮ್ಮಿಂದ ಸಾಧ್ಯವಾಗುವುದಿಲ್ಲ ಎನ್ನಿಸಿದರೆ ಯಾವ ಕ್ಷಣದಲ್ಲಿ ಬೇಕಾ ದರೂ ಮನೆಗೆ ಹಿಂದಿರುಗಿ ಗೃಹಸ್ಥಜೀವನವನ್ನು ನಡೆಸಲು ನಿಮಗೆ ಸ್ವಾತಂತ್ರ್ಯವಿದೆ. ಇದು ಆಷಾಢಭೂತಿತನದ ಜೀವನವನ್ನು ನಡೆಸುತ್ತ ನಿಮಗೂ ಸಂಘಕ್ಕೂ ಅಗೌರವವನ್ನು ತರುವು ದಕ್ಕಿಂತ ಎಷ್ಟೋ ಯೋಗ್ಯವಾದದ್ದು. ನೀವು ಬೆಳಗಿನ ಜಾವದಲ್ಲಿ ಬೇಗನೆ ಎದ್ದು ಜಪಧ್ಯಾನ ಗಳನ್ನು ಮಾಡಿ, ನಿಯಮಬದ್ಧವಾಗಿ ನಿಮ್ಮ ಧಾರ್ಮಿಕ ಕರ್ತವ್ಯಗಳನ್ನು, ಆಧ್ಯಾತ್ಮಿಕ ಸಾಧನೆಗಳನ್ನು ಮಾಡಿಕೊಂಡು ಬರಬೇಕು, ತಪಸ್ಸಿನ ಕಡೆಗೆ ಹೆಚ್ಚಿನ ಗಮನ ಕೊಡಬೇಕು. ನೀವು ನಿಮ್ಮ ಆರೋಗ್ಯದ ಬಗ್ಗೆ ವಿಶೇಷ ಎಚ್ಚರಿಕೆ ವಹಿಸಬೇಕು. ಊಟಕ್ಕೆ ಹೊತ್ತಿಗೆ ಸರಿಯಾಗಿ ಬರಬೇಕು. ನಿಮ್ಮ ವೈಯಕ್ತಿಕ ಆವಶ್ಯಕತೆಗಳನ್ನು ಹೆಚ್ಚಿಸಿಕೊಂಡು ಹೋಗದಂತೆ ನೋಡಿಕೊಳ್ಳಬೇಕು. ನಿಮ್ಮ ಮಾತು ಕತೆಗಳು ಯಾವಾಗಲೂ ಆಧ್ಯಾತ್ಮಿಕ ವಿಚಾರಗಳ ಕುರಿತಾಗಿಯೇ ಇರಬೇಕು. ಕ್ರಿಶ್ಚಿಯನ್ ಮಠಗಳಂತೆ ನೀವು ಕೂಡ ತರಬೇತಿಯ ಅವಧಿಯಲ್ಲಿ ವೃತ್ತಪತ್ರಿಕೆಗಳನ್ನು ಓದುವಂತಿಲ್ಲ. ಮತ್ತು ಭಕ್ತಾದಿಗಳ ಜೊತೆಯಲ್ಲಿ ಸ್ವೇಚ್ಛೆಯಿಂದ ಬೆರೆಯುವಂತಿಲ್ಲ.”

ಕೆಲವೊಮ್ಮೆ ಸ್ವಾಮೀಜಿ ಜುಗುಪ್ಸೆಯ ದನಿಯಲ್ಲಿ ಹೇಳುತ್ತಿದ್ದರು, “ಈ ನಿಮ್ಮ ದೇಶದಲ್ಲಿ ನಾನೇನು ಕೆಲಸ ಮಾಡಲಿ, ಹೇಳಿ? ಇಲ್ಲಿ ಎಲ್ಲರೂ ಆಜ್ಞಾಪನೆ ಮಾಡುವವರೇ, ಅನುಸರಿಸಿ ನಡೆಯುವವರು ಒಬ್ಬರೂ ಇಲ್ಲ. ಮಹತ್ತರವಾದ ಕಾರ್ಯಗಳನ್ನು ಸಾಧಿಸಬೇಕಾದರೆ ನಾಯಕ ನಾದವನ ಆಣತಿಗಳಿಗೆ ಮರುಮಾತಿಲ್ಲದೆ ವಿಧೇಯರಾಗಿರಬೇಕು. ಈಗ ಒಂದು ವೇಳೆ, ನನ್ನ ಸೋದರ ಸಂನ್ಯಾಸಿಗಳು ನನಗೆ ನನ್ನ ಇನ್ನುಳಿದ ಆಯುಷ್ಯವನ್ನೆಲ್ಲ ಮಠದ ಚರಂಡಿಗಳನ್ನು ಶುಚಿಗೊಳಿಸುವುದರಲ್ಲೇ ಕಳೆಯಬೇಕು ಎಂದು ಹೇಳುವುದಾದರೆ, ನಿಶ್ಚಯವಾಗಿ ತಿಳಿದುಕೊಳ್ಳಿ, ನಾನದನ್ನು ಮರುಮಾತಿಲ್ಲದೆ ಮಾಡಿಕೊಂಡು ಬರುತ್ತೇನೆ. ಯಾರು ಸ್ವಲ್ಪವೂ ಗೊಣಗುಟ್ಟದೆ ವಿಧೇಯರಾಗಿರಬಲ್ಲರೋ ಅವರು ಮಾತ್ರವೇ ಮಹಾನಾಯಕರಾಗಬಲ್ಲರು. ಇಂತಹ ಮಹಾ ನಾಯಕನಿಂದ ಮಹತ್ತರವಾದ ಸೇವೆ ನಡೆಯಬಲ್ಲುದು.”

ಹೀಗೆ ಸ್ವಾಮೀಜಿ ತಮ್ಮ ಶಿಷ್ಯರನ್ನು ಉನ್ನತ ಜೀವನಕ್ಕೆ ಉದಾತ್ತ ಧ್ಯೇಯಕ್ಕೆ ಸಿದ್ಧಗೊಳಿಸು ತ್ತಿದ್ದರೂ ಸ್ವತಃ ಅವರ ಮನಸ್ಸಿನಲ್ಲಿಯೇ ಆಗಾಗ ಒಂದು ಬಗೆಯ ನಿರಾಶಾಭಾವ ಕವಿಯುತ್ತಿದ್ದು ದುಂಟು. ಎಲ್ಲಿ ತಮ್ಮ ಜೀವನೋದ್ದೇಶವು ಅಯಶಸ್ವಿಯಾಗಿಬಿಡುತ್ತದೋ ಎಂಬ ಸಂದೇಹ ಮೂಡುತ್ತಿದ್ದುದುಂಟು. ಏಕೆಂದರೆ ತಾವು ನಿರೀಕ್ಷಿಸಿದ್ದಂತಹ ಎರಡು ಸಾವಿರ ಉತ್ಸಾಹೀ ಯುವಕರು, ಸರ್ವಸಂಗ ಪರಿತ್ಯಾಗ ಮಾಡಿ ಸಂನ್ಯಾಸ ಸ್ವೀಕರಿಸಬಲ್ಲ ಯುವಕರು, ಇನ್ನೂ ತಮ್ಮ ಬಳಿಗೆ ಬರಲಿಲ್ಲವಲ್ಲ ಎಂಬ ನಿರಾಶೆ ಅವರನ್ನು ಆವರಿಸಿಕೊಳ್ಳುತ್ತಿತ್ತು. ತಮ್ಮ ಕಾರ್ಯಯೋಜನೆ ಗಳಿಗಾಗಿ ತಾವು ನಿರೀಕ್ಷಿಸುವ ಮೂವತ್ತು ಕೋಟಿ ರೂಪಾಯಿ (ಇಂದು ಅದರ ಬೆಲೆ ಮೂರು ಸಾವಿರ ಕೋಟಿ ರೂಪಾಯಿಯಾದೀತು) ಇನ್ನೂ ತಮಗೆ ಒದಗಿಬರಲಿಲ್ಲವಲ್ಲ ಎಂಬ ನಿರಾಶೆ ಅವರಲ್ಲುಂಟಾಗುತ್ತಿತ್ತು. ಏಕೆಂದರೆ ಈ ಎರಡು ವಸ್ತುಗಳು–ಎರಡು ಸಾವಿರ ಯುವಕರು ಮತ್ತು ಮೂವತ್ತುಕೋಟಿ ರೂಪಾಯಿ–ದೊರೆತರೆ, ಇವುಗಳ ಸಹಾಯದಿಂದ ಸಮಸ್ತ ಭಾರತ ವನ್ನು ಪುನರುಜ್ಜೀವನಗೊಳಿಸಿ, ಅದರ ಸಮಸ್ಯೆಗಳನ್ನೆಲ್ಲ ದೂರಗೊಳಿಸಿ, ಅದು ಸ್ವಾವಲಂಬಿ ಯಾಗುವಂತೆ ಮಾಡಬಹುದು ಎಂದು ಸ್ವಾಮೀಜಿ ನಂಬಿದ್ದರು. ಆದರೆ ಅವು ದೊರಕಲಿಲ್ಲವೆಂದ ಮಾತ್ರಕ್ಕೆ ಅವರು ಹತಾಶರಾಗಿ ಕೈಚೆಲ್ಲಿ ಕುಳಿತುಕೊಳ್ಳುವವರಲ್ಲ. ಆದ್ದರಿಂದ ಅವರು ಇನ್ನು ಕೆಲವೊಮ್ಮೆ ಹೇಳುತ್ತಿದ್ದರು, “ಹೇಗಾದರಾಗಲಿ, ನನ್ನ ಪಾಲಿನ ಕೆಲಸವನ್ನು ನಾನು ಸಮರ್ಪಕವಾಗಿ ಮುಗಿಸುತ್ತೇನೆ, ಮತ್ತು ಆ ಕೆಲಸವನ್ನು ಮುಂದೆಯೂ ನಿರಂತರವಾಗಿ ನಡೆಸಿಕೊಂಡು ಹೋಗಲು ಇತರರಲ್ಲಿ ಸ್ಫೂರ್ತಿಯನ್ನು ತುಂಬುತ್ತೇನೆ. ನನಗೆ ವಿಶ್ರಾಂತಿಯೇ ಬೇಡ! ಕರ್ಮ ಮಾಡುತ್ತಲೇ ನಾನು ಮರಣವನ್ನಪ್ಪುತ್ತೇನೆ. ಕರ್ಮ ಮಾಡುವುದು ನನಗಿಷ್ಟ! ಈ ಜೀವನವೊಂದು ಹೋರಾಟ; ಪ್ರತಿಯೊಬ್ಬನೂ ಸದಾ ಕಾರ್ಯಶೀಲನಾಗಿರಬೇಕು. ನಾನು ಕರ್ಮ ಮಾಡುತ್ತಲೇ ಬದುಕಿ ಕರ್ಮ ಮಾಡುತ್ತಲೇ ಸಾಯುವಂತಾಗಲಿ!”

ಒಂದಾನೊಂದು ಸಂಜೆ ಸ್ವಾಮೀಜಿ ಮಠದ ಆವರಣದಲ್ಲಿ ಅತ್ತಿಂದಿತ್ತ ನಡೆದಾಡುತ್ತಿದ್ದರು; ಅವರ ಮನಸ್ಸು ಪ್ರಚಂಡ ಭಾವದಲೆಗಳಿಂದ ಕೂಡಿರುವುದು ಗೋಚರವಾಗುತ್ತಿತ್ತು. ಹೀಗೆ ಓಡಾಡುತ್ತ ಅವರು ಇದ್ದಕ್ಕಿದ್ದಂತೆ ನಿಂತು ತಮ್ಮ ಸಂನ್ಯಾಸೀ ಶಿಷ್ಯರೊಬ್ಬರ ಹತ್ತಿರ ಹೇಳಿದರು: “ಮಗು, ಶ್ರೀರಾಮಕೃಷ್ಣರು ಜಗತ್ತಿಗಾಗಿ ತಮ್ಮ ಸಮಸ್ತ ಜೀವನವನ್ನೇ ಬಲಿದಾನವಾಗಿಸಿದರು. ಅಂತೆಯೇ ನಾನೂ ನನ್ನ ಜೀವನವನ್ನು ಅರ್ಪಿಸಿಬಿಡುತ್ತೇನೆ. ನೀನೂ, ನಿಮ್ಮಲ್ಲಿ ಪ್ರತಿಯೊಬ್ಬರೂ ಹಾಗೆಯೇ ಮಾಡಬೇಕು. ನಾವು ಮಾಡುತ್ತಿರುವ ಈ ಕೆಲಸಗಳೆಲ್ಲ ಕೇವಲ ಪ್ರಾರಂಭ ಮಾತ್ರ. ನಂಬು ನನ್ನನ್ನು–ಈ ರೀತಿ ನಮ್ಮ ರಕ್ತವನ್ನು ಹರಿಯಿಸುವುದರಿಂದ, ಮುಂದೆ ಸಮಸ್ತ ಜಗತ್ತನ್ನೇ ಅಲುಗಿಸಬಲ್ಲ, ದೇವಸೇವಕರಾದ ಮಹಾ ಪರಾಕ್ರಮಶಾಲಿಗಳು ಉದಿಸಲಿದ್ದಾರೆ!”

ತಮ್ಮ ತರುಣ ಶಿಷ್ಯರನ್ನು ಸ್ಫೂರ್ತಿಗೊಳಿಸುತ್ತ ಸ್ವಾಮೀಜಿಯವರು ಘೋಷಿಸಿದರು, “ಮರೆಯದಿರಿ, ‘ಆತ್ಮನೋ ಮೋಕ್ಷಾರ್ಥಂ ಜಗದ್ಧಿತಾಯ ಚ’–ಜಗದ ಹಿತಕ್ಕಾಗಿ ಸೇವೆ ಸಲ್ಲಿಸುತ್ತ ಭಗವತ್ಸಾಕ್ಷಾತ್ಕಾರ ಮಾಡಿಕೊಳ್ಳುವುದೇ ಸಂನ್ಯಾಸಿಗಳ ಮಹಾ ಆದರ್ಶ. ಹಿಡಿದುಕೊಳ್ಳಿ ಈ ಆದರ್ಶವನ್ನು. ಸಂನ್ಯಾಸಧರ್ಮವೇ ಅತ್ಯಂತ ಶೀಘ್ರ ಪಥ. ಸಂನ್ಯಾಸಿ ಮತ್ತು ಅವನ ದೇವರು–ಇವರಿಬ್ಬರ ನಡುವೆ ಯಾವ ವಿಗ್ರಹಗಳೂ ಇಲ್ಲ. ಸಂನ್ಯಾಸಿಯು ವೇದಗಳ ಶಿರಮೆಟ್ಟಿ ನಿಂತಿರುತ್ತಾನೆ ಎಂದು ವೇದಗಳು ಸಾರುತ್ತವೆ. ಏಕೆಂದರೆ ಅವನು ಮತಪಂಥಗಳಿಂದ, ಗುಡಿ ಮಸೀದಿ ಚರ್ಚುಗಳಿಂದ ಸಂಪೂರ್ಣ ಮುಕ್ತ. ಅವನು ಈ ಭೂಮಿಯ ಮೇಲಿನ ಸಾಕ್ಷಾತ್ ದೇವರು! ಇದನ್ನು ನೆನಪಿನಲ್ಲಿಡು, ಮತ್ತು ಓ ಧೀರ ಸಂನ್ಯಾಸಿ, ನಿನ್ನ ದಾರಿಯಲ್ಲಿ ನೀನು ಮುನ್ನಡೆಯುತ್ತಲೇ ಇರು; ಮಹಾತ್ಯಾಗದ ಧ್ವಜವನ್ನು, ಶಾಂತಿ ಮುಕ್ತಿ ಧನ್ಯತೆಗಳ ಬಾವುಟವನ್ನು ಹಿಡಿದು ಮುನ್ನಡೆಯುತ್ತಲೇ ಇರು.”

ವೈದ್ಯನಾಥದಲ್ಲಿದ್ದ ಅವಧಿಯಲ್ಲಿ ಆಸ್ತಮಾದಿಂದಾಗಿ ಸ್ವಾಮೀಜಿ ಬಹಳ ನರಳಿದರಾದರೂ ಅಲ್ಲಿ ವಿಶ್ರಾಂತಿ ಸಿಕ್ಕಿದ್ದರಿಂದ ಸ್ವಲ್ಪ ಒಳಿತೂ ಆಗಿತ್ತು. ಅವರೀಗ ನವೋತ್ಸಾಹದಿಂದ ಕೂಡಿ ದ್ದರು, ನವ ಯೋಜನೆಗಳನ್ನು ಹಾಕಿಕೊಂಡಿದ್ದರು. ಅವರು ಕಲ್ಕತ್ತಕ್ಕೆ ಮರಳಿದ ದಿನವೇ ಆಶ್ರಮವಾಸಿಗಳನ್ನೆಲ್ಲ ಸೇರಿಸಿ “ಇನ್ನೀಗ, ನೀವೆಲ್ಲರೂ ಬುದ್ಧನ ಅನುಯಾಯಿಗಳಂತೆ ಧರ್ಮ ಪ್ರಚಾರಕ್ಕಾಗಿ ಎಲ್ಲೆಡೆಗಳಿಗೂ ಹೋಗಬೇಕು; ಶ್ರೀರಾಮಕೃಷ್ಣರ ದಿವ್ಯ ಸಂದೇಶಗಳನ್ನು ಹರಡ ಬೇಕು” ಎಂದು ಹೇಳಿದರು. ಅದೇ ದಿನವೇ ತಮ್ಮ ಶಿಷ್ಯರಾದ ವಿರಜಾನಂದರು ಹಾಗೂ ಪ್ರಕಾಶಾನಂದರನ್ನು ಕರೆದು, ಒಡನೆಯೇ ಢಾಕಾಕ್ಕೆ ಹೋಗಿ ಆ ಕಾರ್ಯದಲ್ಲಿ ಪ್ರವೃತ್ತರಾಗುವಂತೆ ಆದೇಶ ನೀಡಿದರು. ಆಗ ವಿರಜಾನಂದರು ಅತ್ಯಂತ ವಿನಯದಿಂದ ನುಡಿದರು, “ಸ್ವಾಮೀಜಿ, ನಾನೇನು ಬೋಧಿಸಲಿ? ನನಗೇ ಏನೂ ಗೊತ್ತಿಲ್ಲ... ” “ಹಾಗಾದರೆ ಅದನ್ನೇ ಬೋಧಿಸು! ನಿಜಕ್ಕೂ ಅದೇ ಒಂದು ದೊಡ್ಡ ಸಂದೇಶ!” ಎಂದು ಉದ್ಗರಿಸಿದರು ಸ್ವಾಮೀಜಿ. ಆದರೆ ಇದು ವಿರಜಾನಂದರಿಗೆ ಸಮಾಧಾನವನ್ನುಂಟುಮಾಡಲಿಲ್ಲ. “ ಸ್ವಾಮೀಜಿ, ನಾನು ಇನ್ನಷ್ಟು ಆಧ್ಯಾತ್ಮಿಕ ಸಾಧನೆ ಮಾಡಿ ಮೊದಲು ಭಗವಂತನ ಸಾಕ್ಷಾತ್ಕಾರ ಮಾಡಿಕೊಳ್ಳುತ್ತೇನೆ. ಮೊದಲು ನಾನು ನನ್ನ ಮುಕ್ತಿಯನ್ನು ಗಳಿಸಿಕೊಳ್ಳುತ್ತೇನೆ. ದಯವಿಟ್ಟು ಅದಕ್ಕೆ ಅವಕಾಶ ಮಾಡಿಕೊಡಬೇಕು” ಎಂದರು. ಸ್ವಾಮೀಜಿ ಗುಡುಗಿದರು: “ನೀನು ನಿನ್ನ ಮುಕ್ತಿಯನ್ನೇ ನೋಡಿಕೊಳ್ಳುವಿಯಾದರೆ ನರಕಕ್ಕೆ ಬೀಳುವೆ, ಎಚ್ಚರಿಕೆ! ನೀನೇನಾದರೂ ಅತ್ಯುನ್ನತವಾದುದನ್ನು ಪಡೆಯಬೇಕಾದರೆ ಮೊದಲು ಇತರರ ಮುಕ್ತಿಗಾಗಿ ಪ್ರಯತ್ನಿಸು. ನಿನ್ನ ವೈಯಕ್ತಿಕ ಮುಕ್ತಿಯ ಆಸೆಯನ್ನು ನಾಶ ಮಾಡಿಬಿಡು. ಇದೇ ಅತಿ ದೊಡ್ಡ ಆಧ್ಯಾತ್ಮಿಕ ಸಾಧನೆ.” ಬಳಿಕ ಮತ್ತೆ ಅತ್ಯಂತ ಪ್ರೀತಿಯ ದನಿಯಲ್ಲಿ ಹೇಳಿದರು, “ನನ್ನ ಪ್ರಿಯ ಸುತರೆ, ಕಾರ್ಯನಿರತರಾಗಿ! ಮನಃಪೂರ್ವಕವಾಗಿ ಕರ್ಮಮಾಡಿ. ಮಾಡಬೇಕಾದ ದ್ದೆಂದರೆ ಇದು. ನೀವು ಮಾಡುವ ಕರ್ಮದ ಫಲದ ಕಡೆಗೆ ಗಮನ ಕೊಡಲೇಬೇಡಿ. ನೀವು ಇತರರ ಒಳಿತಿಗಾಗಿ ಸೇವಾನಿರತರಾಗಿರುವಾಗ ಒಂದು ವೇಳೆ ಸತ್ತು ನರಕಕ್ಕೆ ಹೋಗುವಂತಾದರೂ ಅದರಿಂದೇನಂತೆ! ಅದು ಸ್ವಂತ ಮುಕ್ತಿಗಾಗಿ ಶ್ರಮಿಸುತ್ತ ಸ್ವರ್ಗ ಗಳಿಸುವುದಕ್ಕಿಂತಲೂ ಎಷ್ಟೋ ಮೇಲು!” ಬಳಿಕ ಸ್ವಲ್ಪ ಹೊತ್ತಿನಮೇಲೆ ಸ್ವಾಮೀಜಿ ತಮ್ಮ ಆ ಇಬ್ಬರು ಶಿಷ್ಯರನ್ನೂ ಕರೆದು ಕೊಂಡು ಮಠದ ಪೂಜಾಗೃಹಕ್ಕೆ ಹೋದರು. ಅಲ್ಲಿ ಮೂವರೂ ಧ್ಯಾನಕ್ಕೆ ಕುಳಿತುಕೊಂಡರು. ಸ್ವಾಮೀಜಿ ಗಾಢ ಧ್ಯಾನದಲ್ಲಿ ಲೀನರಾದರು. ಸ್ವಲ್ಪ ಹೊತ್ತಿನ ಮೇಲೆ ಆ ಸ್ಥಿತಿಯಿಂದ ಹೊರ ಬಂದು ಗಂಭೀರ ದನಿಯಲ್ಲಿ ನುಡಿದರು, “ಈಗ ನಾನು ನಿಮ್ಮಲ್ಲಿ ಶಕ್ತಿಸಂಚಾರ ಮಾಡುತ್ತೇನೆ; ನನ್ನ ಶಕ್ತಿಯನ್ನು ನಿಮ್ಮಲ್ಲಿ ಹರಿಯಿಸುತ್ತೇನೆ. ಇನ್ನು ಸ್ವಯಂ ಭಗವಂತನೇ ನಿಮ್ಮ ಬೆನ್ನ ಹಿಂದಿರು ತ್ತಾನೆ.” ಆ ದಿನವೆಲ್ಲ ಸ್ವಾಮೀಜಿ ಆ ಇಬ್ಬರು ಶಿಷ್ಯರೊಂದಿಗೆ ಅತ್ಯಂತ ಆತ್ಮೀಯತೆಯಿಂದಿದ್ದು ಅವರಿಗೆ ಹಲವಾರು ಸಲಹೆಗಳನ್ನು ನೀಡಿದರು. ಅವರು ಮುಂದೆ ಹೇಗೆ ಆಧ್ಯಾತ್ಮಿಕ ತತ್ತ್ವಗಳನ್ನು ಬೋಧಿಸಬೇಕು, ಮಂತ್ರೋಪದೇಶವನ್ನು ಬಯಸಿ ಬಂದವರಿಗೆ ಅದನ್ನು ನೀಡುವ ಬಗೆ ಹೇಗೆ ಎಂಬುದನ್ನೆಲ್ಲ ತಿಳಿಸಿಕೊಟ್ಟರು. ಹೀಗೆ ಸ್ವಾಮೀಜಿಯವರಿಂದ ಆಶೀರ್ವಾದ ಪಡೆದುಕೊಂಡ, ಸ್ಫೂರ್ತಿ ಹೊಂದಿದ ವಿರಜಾನಂದರೂ ಪ್ರಕಾಶಾನಂದರೂ ಫೆಬ್ರುವರಿ ನಾಲ್ಕರಂದು ಢಾಕಾದ ಕಡೆಗೆ ನಡೆದರು. ಸ್ವಾಮೀಜಿಯವರು ಇವರಿಬ್ಬರನ್ನಲ್ಲದೆ, ತಮ್ಮ ಸೋದರ ಸಂನ್ಯಾಸಿಗಳಾದ ಸ್ವಾಮಿ ಶಾರದಾನಂದರಿಗೂ ಸ್ವಾಮಿ ತುರೀಯಾನಂದರಿಗೂ ಮಂತ್ರೋಪದೇಶ ನೀಡುವ ಅಧಿಕಾರ ಕೊಟ್ಟು ಗುಜರಾತ್ ಪ್ರದೇಶದಲ್ಲಿ ಧರ್ಮಪ್ರಸಾರ ಮಾಡಲು ಕಳಿಸಿಕೊಟ್ಟರು.

