
\chapter{“ಹುಟ್ಟಾ ಹೋರಾಟಗಾರ ನಾನು!”}

\noindent

ಡಾರ್ಜಿಲಿಂಗಿನಿಂದ ಹಿಂದಿರುಗಿದ ಮೇಲೆ ಸ್ವಾಮೀಜಿಯವರು ಕಲ್ಕತ್ತದಲ್ಲಿ ಸುಮಾರು ಒಂದು ವಾರ ಇದ್ದರು. ಈ ಅವಧಿಯಲ್ಲಿ ರಾಮಕೃಷ್ಣ ಸಂಘದ ಸಂಸ್ಥಾಪನೆಗೆ ಸಂಬಂಧಿಸಿದಂತೆ ಮುಖ್ಯ ಕಾರ್ಯವೊಂದು ನೆರವೇರಿದ್ದು ಅವರಿಗೆ ತುಂಬ ಸಮಾಧಾನ ತಂದಿತ್ತು. ಆದರೆ ಅವರೀಗ ಆ ಕಾರ್ಯವನ್ನು ಮುಂದುವರಿಸುವ ಸ್ಥಿತಿಯಲ್ಲಿರಲಿಲ್ಲ. ಅವರ ಶರೀರಕ್ಕೆ ಸಂಪೂರ್ಣ ವಿಶ್ರಾಂತಿ ಅತ್ಯಗತ್ಯವಾಗಿ ಬೇಕಾಗಿತ್ತು. ಡಾಕ್ಟರುಗಳು, ಸೋದರ ಸಂನ್ಯಾಸಿಗಳು ಅವರಿಗೆ ಸಾಧ್ಯವಾದಷ್ಟು ಬೇಗನೆ ಆಲ್ಮೋರಕ್ಕೆ ಹೋಗಿ ಮತ್ತೆ ಕೆಲಕಾಲ ವಿಶ್ರಾಂತಿ ತೆಗೆದುಕೊಳ್ಳುವಂತೆ ಕಳಕಳಿಯಿಂದ ಸೂಚಿಸಿದರು. ಅಲ್ಲಿನ ಹವೆ ತಂಪಾಗಿರುವುದಲ್ಲದೆ ಆರ್ದ್ರತೆ ಕಡಿಮೆಯಿರುವುದರಿಂದ ಆರೋಗ್ಯಕ್ಕೆ ಬಹಳ ಒಳ್ಳೆಯದು. ಅಂತೆಯೇ ಮೇ ೬ರಂದು ಸ್ವಾಮೀಜಿ ತಮ್ಮ ಕೆಲವು ಸೋದರ ಸಂನ್ಯಾಸಿಗಳು ಹಾಗೂ ಶಿಷ್ಯರೊಂದಿಗೆ ಆಲ್ಮೋರದತ್ತ ಪ್ರಯಾಣ ಬೆಳೆಸಿದರು.

ಈ ವೇಳೆಗಾಗಲೇ ಸ್ವಾಮಿ ಶಿವಾನಂದರು, ಸ್ವಾಮಿ ನಿರಂಜನಾನಂದರು, ಜೆ. ಜೆ. ಗುಡ್​ವಿನ್ ಮತ್ತು ಮಿಸ್ ಹೆನ್ರಿಟಾ ಮುಲ್ಲರ್ ಆಲ್ಮೋರವನ್ನು ತಲುಪಿದ್ದರು. ಸ್ವಾಮೀಜಿ ಲಂಡನ್ನಿನಿಂದ ಹೊರಟ ಕೆಲಕಾಲದಲ್ಲೇ ಭಾರತವನ್ನು ಸಂದರ್ಶಿಸುವ ಉದ್ದೇಶದಿಂದ ಮಿಸ್ ಮುಲ್ಲರಳೂ ಹೊರಟುಬಂದಿದ್ದಳು.

ಸ್ವಾಮೀಜಿ ತಮ್ಮಲ್ಲಿಗೆ ಆಗಮಿಸುವ ಸುದ್ದಿಯನ್ನು ತಿಳಿದು ಆಲ್ಮೋರದ ನಿವಾಸಿಗಳು ಅವರನ್ನು ಎದುರ್ಗೊಳ್ಳಲು ಸಂಭ್ರಮದ ಸಿದ್ಧತೆಗಳನ್ನು ಮಾಡಿದ್ದರು. ಮೇ ೯ ರಂದು ಸ್ವಾಮೀಜಿಯವರು ಆಲ್ಮೋರದ ಹತ್ತಿರದ ಊರಾದ ಲೊಡಿಯಾ ಎಂಬಲ್ಲಿಗೆ ಬಂದಾಗ, ಅಲ್ಲಿ ಅವರನ್ನು ಸ್ವಾಗತಿಸಲು ಬಹು ದೊಡ್ಡ ಜನಸಮುದಾಯವೇ ಸೇರಿತ್ತು. ಸುಂದರವಾಗಿ ಸಿಂಗರಿಸಿದ್ದ ಕುದುರೆಯ ಮೆಲೆ ಸ್ವಾಮೀಜಿಯವರನ್ನು ಕುಳ್ಳಿರಿಸಿ ಮೆರವಣಿಗೆಯಲ್ಲಿ ಟೌನಿಗೆ ಕರೆತರಲಾಯಿತು. ರಸ್ತೆಗಳಲ್ಲಿ ನೆರೆದಿದ್ದ ಸಾವಿರಾರು ಜನ ಜೈಕಾರ ಮಾಡುತ್ತಿದ್ದರು. ಸ್ತ್ರೀಯರು ಮಹಡಿಗಳ ಮೇಲಿಂದ ಸ್ವಾಮೀಜಿಯವರ ಮೇಲೆ ಮಂತ್ರಾಕ್ಷತೆ ಪುಷ್ಪಗಳ ಮಳೆಗರೆದರು. ಸುಮಾರು ಮೂರು ಸಾವಿರ ಜನ ಹಿಡಿಸುವಂತಹ ಚಪ್ಪರವೊಂದನ್ನು ಸನ್ಮಾನ ಸಮಾರಂಭಕ್ಕಾಗಿ ನಿರ್ಮಿಸಲಾಗಿತ್ತು. ಚಪ್ಪರದ ಸುತ್ತಮುತ್ತ ವಿಜಯಪತಾಕೆಗಳ ಹಾಗೂ ಪುಷ್ಪಗಳ ತೋರಣಗಳನ್ನು ಕಟ್ಟಿದ್ದರು. ಪ್ರತಿಯೊಂದು ಮನೆಯ ಮುಂದೆಯೂ ದೀಪಗಳು ಉರಿಯುತ್ತಿದ್ದು ಅದೊಂದು ದೀಪೋತ್ಸವ ದಂತೆ ಕಾಣುತ್ತಿತ್ತು! ಈ ದೀಪಗಳ ಬೆಳಕು ಟೌನಿನಲ್ಲೆಲ್ಲ ಪ್ರಕಾಶಿಸುತ್ತಿತ್ತು! ವಾದ್ಯಗಳು ಮೊಳಗಿದುವು, ಜಯಘೋಷಗಳು ಪ್ರತಿಧ್ವನಿಸಿದುವು. ಇವಿಷ್ಟು, ಶಬ್ದಗಳಿಂದ ಬಣ್ಣಿಸಬಹುದಾದ ಅಂಶಗಳು. ಆದರೆ ಅದನ್ನು ಕಣ್ಣಿಂದ ಕಂಡವರ ಅನುಭವ ಇದಕ್ಕಿಂತ ಎಷ್ಟೋ ಮಿಗಿಲಾದುದು!

ಆ ಚಪ್ಪರದಲ್ಲಿ ಮಾತ್ರವಲ್ಲದೆ ಹೊರಗೂ ಜನ ಕಿಕ್ಕಿರಿದಿದ್ದರು. ಸ್ವಾಮೀಜಿಯವರ ವಾಣಿ ಯನ್ನು ಆಲಿಸಲು ಜನರ ತವಕ ಎಷ್ಟು ಅಧಿಕವಾಗಿತ್ತೆಂದರೆ ಕಾರ್ಯಕ್ರಮದ ಸ್ವಾಗತ-ಪರಿ ಚಯಾದಿಗಳನ್ನೂ ಮೊಟಕುಗೊಳಿಸಬೇಕಾಯಿತು. ಪಂಡಿತ ಜ್ವಾಲಾದತ್ತ ಜೋಶಿಯವರು ಹಿಂದಿ ಯಲ್ಲಿ ಸ್ವಾಗತ ಸಮಿತಿಯ ಬಿನ್ನವತ್ತಳೆಯನ್ನು ಓದಿದರು. ಇಲ್ಲಿ ಸ್ವಾಮೀಜಿಯವರ ಆತಿಥೇ ಯರೂ ಅವರ ಹಳೆಯ ಪರಿಚಿತರೂ ಆದ ಲಾಲಾ ಬದರೀಸಾಹರ ಪರವಾಗಿ ಒಬ್ಬರು ಸ್ವಾಗತ ಕೋರಿದರು. ಬಳಿಕ ಮತ್ತೊಬ್ಬರು ಪಂಡಿತರು ಸಂಸ್ಕೃತದಲ್ಲೊಂದು ಬಿನ್ನವತ್ತಳೆಯನ್ನೋದಿ ದರು. ಈಗ ಸ್ವಾಮೀಜಿಯವರು ಎದ್ದುನಿಂತು ಸಂಕ್ಷೇಪವಾಗಿ, ಆದರೆ ಭಾವಪೂರ್ಣವಾಗಿ ಮಾತನಾಡಿದರು:

“ಈ ಪವಿತ್ರ ಹಿಮಾಲಯವು ಭಾರತೀಯ ಮನದ ಮೇಲೆ ಅತ್ಯುನ್ನತ ಆಧ್ಯಾತ್ಮಿಕ ಪ್ರಭಾವ ವನ್ನು ಬೀರಿದೆ. ನಾನು ಕೂಡ ಈ ಹಿಮಾಲಯದಲ್ಲೇ ನನ್ನ ಜೀವನ ಯಾಪನ ಮಾಡಬೇಕೆಂದು ಯುವಕನಾಗಿದ್ದಾಗಿನಿಂದಲೂ ಬಯಸಿದ್ದೆ. ನನ್ನ ಬಯಕೆಯಂತೆ ಹಾಗೆ ಇಲ್ಲಿ ವಾಸಿಸುವುದು ಸಾಧ್ಯವಾಗಲಾರದೆಂದು ನನಗೆ ಗೊತ್ತಿದೆ; ಆದರೂ ನನ್ನ ಕೊನೆಯ ದಿನಗಳನ್ನಾದರೂ ಈ ಹಿಮಾಲಯದಲ್ಲಿ ಕಳೆಯುವಂತಾಗಲಿ ಎಂಬುದೆ ನನ್ನ ಪ್ರಾರ್ಥನೆ. ನನ್ನ ಮನದಲ್ಲಿ ವರ್ಷ ಗಟ್ಟಲೆಯಿಂದ ಕುಣಿದಾಡುತ್ತಿದ್ದ ಹಲವಾರು ಕಾರ್ಯಯೋಜನೆಗಳೆಲ್ಲ ಈ ಭವ್ಯ ಹಿಮಾಲಯದ ದರ್ಶನ ಮಾತ್ರದಿಂದಲೇ ಅಡಗಿ ಶಾಂತವಾಗಿಬಿಟ್ಟವು! ಈಗ ನನ್ನ ಮನಸ್ಸು ಈ ಹಿಮಾಲಯವು ಪ್ರತಿನಿಧಿಸುವ ತ್ಯಾಗವೆಂಬ ಮಹಾದರ್ಶದೆಡೆಗೆ ಧಾವಿಸಿ ಹೋಗುತ್ತಿದೆ.” ಹಿಮಾಲಯದ ದರ್ಶನ ವಾಗುತ್ತಲೇ ಸ್ವಾಮೀಜಿಯವರ ಭಾವಪ್ರಪಂಚದಲ್ಲೊಂದು ಪ್ರಚಂಡ ಪರಿವರ್ತನೆಯಾಗುವು ದನ್ನು ಇಲ್ಲಿ ಗುರುತಿಸಬಹುದು.

ಸ್ವಾಮೀಜಿ ವಿಶ್ರಾಂತಿಗೆಂದು ಆಲ್ಮೋರಕ್ಕೆ ಬಂದವರಾದರೂ ಇಲ್ಲಿಯೂ ಅವರು ಹಲವಾರು ಕಾರ್ಯಗಳಲ್ಲಿ ಮಗ್ನರಾಗಿರಬೇಕಾಯಿತು. ಸಂದರ್ಶಕರೊಡನೆ ಧರ್ಮಸಂಬಂಧವಾಗಿ ಸಂಭಾಷಿ ಸುವುದರಲ್ಲಿಯೇ ದಿನವೆಲ್ಲ ಕಳೆದುಹೋಗುತ್ತಿತ್ತು. ಹೀಗಿದ್ದರೂ ಅಲ್ಲಿನ ಹವಾಗುಣದ ದೆಸೆ ಯಿಂದಾಗಿ ಅವರ ಆರೋಗ್ಯ ತಕ್ಕಮಟ್ಟಿಗೆ ಸುಧಾರಿಸಿತು.

ಸ್ವಾಮೀಜಿ ಆಲ್ಮೋರಕ್ಕೆ ಬಂದಂದಿನಿಂದ ಸೆಖೆ ಅಧಿಕವಾಗಿಬಿಟ್ಟಿತು. ಆದ್ದರಿಂದ ಅವರು ಅಲ್ಲಿಂದ ಇಪ್ಪತ್ತು ಮೈಲಿ ದೂರದ ದೇವಲ್​ಧಾರ್ ಎಂಬಲ್ಲಿಗೆ ಹೋಗಿ ಸುಮಾರು ಮೂರು ದಿನ ಇದ್ದರು. ಹಿಮಾಲಯವೇ ರಮ್ಯ, ಅದರಲ್ಲೂ ಈ ಸ್ಥಳ ಇನ್ನಷ್ಟು ರಮ್ಯ. ಮೂರು ದಿನಗಳ ಮೇಲೆ ಸ್ವಾಮೀಜಿ ತಮ್ಮ ಶಿಷ್ಯರೊಂದಿಗೆ ಪಿಂಡಾರಿ ಗ್ಲೇಷಿಯರ್​ಗೆ ಬಂದರು. ಇದು ಹಿಮಾ ಲಯದ ಇನ್ನಷ್ಟು ಒಳಪ್ರದೇಶ. ಇಲ್ಲಿಗೆ ಬಂದ ಮೇಲೆ ಅವರ ಆರೋಗ್ಯ ಬೇಗ ಸುಧಾರಿಸ ತೊಡಗಿತು.

ಕೆಲದಿನಗಳ ಮೇಲೆ ಸ್ವಾಮೀಜಿ ಆಲ್ಮೋರಕ್ಕೆ ಮರಳಿದರು. ಕಾರಣ, ಸಾಕಷ್ಟು ಮಳೆಯಾಗಿ ಹವಾಮಾನ ತಂಪಾಗಿತ್ತು. ಇಲ್ಲಿ ಅವರು ದೇಶ-ವಿದೇಶಗಳಲ್ಲಿನ ತಮ್ಮ ಶಿಷ್ಯರಿಂದ, ಇಲ್ಲವೆ ಇತರ ಪಂಡಿತರಿಂದ-ತತ್ವಜ್ಞಾನಿಗಳಿಂದ ಅನೇಕ ಪತ್ರಗಳನ್ನು ಪಡೆಯುತ್ತಿದ್ದರು. ಹೆಚ್ಚಿನವರು ತಾತ್ತ್ವಿಕ ವಿಚಾರಗಳ ಕುರಿತಾಗಿ ಹಲವಾರು ಸಂದೇಹಗಳನ್ನು ಮುಂದಿಟ್ಟು ಉತ್ತರವನ್ನು ನಿರೀಕ್ಷಿಸುತ್ತಿದ್ದರು. ಇವರುಗಳಿಗೆಲ್ಲ ಸ್ವಾಮೀಜಿ ತಕ್ಕ ಉತ್ತರಗಳನ್ನು ಬರೆಯುತ್ತಿದ್ದರು. ಈ ಉತ್ತರಗಳಲ್ಲಿ ಕೆಲವು, ಶಾಸ್ತ್ರಗ್ರಂಥಗಳ ಮೇಲೆ ಶ್ರೇಷ್ಠ ವ್ಯಾಖ್ಯಾನಗಳನ್ನೊಳಗೊಂಡಿವೆ. ಅಲ್ಲದೆ ತಮ್ಮ ಸೋದರ ಸಂನ್ಯಾಸಿಗಳೊಂದಿಗೆ, ಪ್ರಿಯ ‘ಸೋದರಿ’ ಮೇರಿ ಹೇಲ್​ಳೊಂದಿಗೆ ಹಾಗೂ ಮುಂದೆ ಸೋದರಿ ನಿವೇದಿತೆಯಾಗಲಿರುವ ಮಿಸ್ ಮಾರ್ಗರೆಟ್ಟಳೊಂದಿಗೆ ಪತ್ರವ್ಯವಹಾರವನ್ನು ನಡೆಸಿದ್ದರು. ಈ ನಡುವೆ ಮುರ್ಶಿದಾಬಾದಿನಲ್ಲಿ ಸ್ವಾಮಿ ಅಖಂಡಾನಂದರು ಬರಗಾಲ ಸಂತ್ರಸ್ತರ ಸೇವೆಯಲ್ಲಿ ತೊಡಗಿದ್ದರು. ಅವರ ಯಶಸ್ಸಿನ ಬಗ್ಗೆ ತಿಳಿದಾಗ ಸ್ವಾಮೀಜಿ, ಅವರಿಗೊಂದು ಆತ್ಮೀಯ ಪತ್ರ ಬರೆದರು:

\noindent

ಪ್ರಿಯ ಗಂಗಾಧರ,

ನಿನ್ನ ವಿಚಾರವಾಗಿ ವಿವರವಾದ ಸುದ್ದಿಗಳು ತಿಳಿದುಬರುತ್ತಿವೆ. ಇದರಿಂದ ನನಗೆ ತುಂಬ ಸಂತೋಷವಾಗುತ್ತಿದೆ. ಪ್ರಪಂಚವನ್ನು ಗೆಲ್ಲಬೇಕಾದರೆ ಅಂತಹ ಕೆಲಸವೇ ಆಗಬೇಕು, ಭಲೆ! ನಿನಗೆ ನನ್ನ ಲಕ್ಷ ಆಲಿಂಗನಗಳು, ಆಶೀವಾರ್ದಗಳು. ಕೆಲಸ, ಕೆಲಸ, ಕೆಲಸ–ಬೇರಾವುದನ್ನೂ ಲಕ್ಷಿಸುವವನಲ್ಲ ನಾನು. ಹಣದ ಬಗ್ಗೆ ಚಿಂತಿಸಬೇಡ. ಅದು ಆಕಾಶದಿಂದ ತಾನಾಗಿಯೇ ಕೆಳಗೆ ಉದುರುತ್ತದೆ. ಯಾರು ನಿನಗೆ ಧನಸಹಾಯ ನೀಡಲು ಮುಂದಾಗುತ್ತಾರೆಯೋ, ಅವರು ಅದನ್ನು ಬೇಕಾದರೆ ತಮ್ಮ ಹೆಸರಿನಲ್ಲೇ ವಿತರಣೆ ಮಾಡಲಿ. ಅದರಲ್ಲಿ ಯಾವ ತೊಂದರೆಯೂ ಇಲ್ಲ. ಯಾರ ಹೆಸರಾದರೇನಂತೆ! ಈ ಕೀರ್ತಿಯೆಲ್ಲ ಯಾರಿಗೆ ಬೇಕಾಗಿದೆ? ಅದನ್ನು ಮರೆತುಬಿಡು. ಹಸಿದವರಿಗೆ ಒಂದು ತುತ್ತು ಅನ್ನವನ್ನು ಒದಗಿಸುವ ಪ್ರಯತ್ನದಲ್ಲಿ ನಿನ್ನ ಹೆಸರು-ಗೌರವ ಹೋದರೆ ಹೋಗಲಿ–ಅಹೋ ಭಾಗ್ಯಮಹೋ ಭಾಗ್ಯಂ! ನಿನ್ನ ಜೀವನ ಪಾವನವಾಗುವುದು. ಪ್ರೀತಿ!ಪ್ರೀತಿ! ಕೊನೆಗೆ ಗೆಲ್ಲುವುದು ಪ್ರೀತಿಯೇ ಹೊರತು ಬುದ್ಧಿವಂತಿಕೆಯಲ್ಲ. ಈ ಪ್ರೀತಿಯ ಮುಂದೆ ಪುಸ್ತಕಗಳು, ಪಾಂಡಿತ್ಯ, ಯೋಗ, ಧ್ಯಾನ, ಸಾಕ್ಷಾತ್ಕಾರ–ಎಲ್ಲವೂ ಕೇವಲ ಮಣ್ಣು. ಈ ಬಗೆಯ ಪ್ರೀತಿಯು ಮಾನವನ ದೇಹಮಂದಿರದಲ್ಲಿರುವ ದೇವನ ಪೂಜೆಯೇ ಸರಿ. ಆದರೆ ಇದಿನ್ನೂ ಆರಂಭವಷ್ಟೆ. ನಾವು ಭಾರತದಲ್ಲೆಲ್ಲ ಅಲ್ಲ–ಇಡೀ ಪ್ರಪಂಚದಲ್ಲೆಲ್ಲ, ಈ ರೀತಿಯಾಗಿ (ಸೇವೆ ಇತ್ಯಾದಿಗಳ ಮೂಲಕ) ಹರಡದಿದ್ದರೆ ನಮ್ಮ ಭಗವಾನರ ಮಹಿಮೆಯೇನು ಬಂತು!

ನಮ್ಮ ಭಗವಾನರ ಪಾದಸ್ಪರ್ಶವು ಮಾನವನಿಗೆ ದೈವತ್ವವನ್ನು ಕೊಡುವುದೋ ಇಲ್ಲವೋ ಎಂಬುದನ್ನು ಜನ ನೋಡಲಿ! ಇದನ್ನೇ ಜೀವನ್ಮುಕ್ತಿ ಎನ್ನುವುದು. ಇದು ದೊರಕುವುದು ಅಹಂಕಾರ-ಸ್ವಾರ್ಥತೆಗಳ ಲವಲೇಶವೂ ಉಳಿಯದಂತಾದಾಗ.

ಒಳ್ಳೆಯದು ಮಾಡಿದೆ. ಕಾರ್ಯಕ್ಷೇತ್ರವನ್ನು ಕ್ರಮೇಣ ವಿಸ್ತರಿಸಲು ಪ್ರಯತ್ನಿಸು. ನಿನಗೆ ಸಾಧ್ಯವಾದರೆ ಕಲ್ಕತ್ತದಲ್ಲಿ ಮತ್ತೊಂದು ಸ್ವಯಂಸೇವಕರ ಗುಂಪಿನೊಂದಿಗೆ ವಂತಿಗೆ ಎತ್ತಲು ಪ್ರಯತ್ನಿಸು. ಅವರ ಪೈಕಿ ಒಂದಿಬ್ಬರನ್ನು ಅಲ್ಲಿ ಕೆಲಸ ಮಾಡಲು ಬಿಟ್ಟು ನೀನು ಮತ್ತೊಂದು ಸ್ಥಳದಲ್ಲಿ ಪ್ರಾರಂಭ ಮಾಡಬಹುದು. ಈ ರೀತಿಯಾಗಿ ವಿಸ್ತರಿಸುತ್ತ ಹೋಗು. ಮತ್ತು ಸದಾ ಅವರ ಕೆಲಸದ ಮೇಲೆ ನಿಗಾ ಇಟ್ಟಿರು. ಕೆಲಸ ಕ್ರಮೇಣ ಸ್ಥಿರವಾಗುವುದನ್ನು ನೀನೇ ನೋಡುವೆ. ಧರ್ಮ ಮತ್ತು ವಿದ್ಯೆಯ ಪ್ರಚಾರವೆಲ್ಲ ಕ್ರಮೇಣ ಸಾಧ್ಯವಾಗುತ್ತದೆ. ನೀನು ಹೀಗೆ ಮಾಡುತ್ತ ಹೋಗು, ಹೇ ಧೀರ, ನಿನ್ನನ್ನು ನಾನು ಭುಜದ ಮೇಲೆ ಕುಳ್ಳಿರಿಸಿಕೊಂಡು ಮೆರೆಸುತ್ತೇನೆ!

ನಾನಿನ್ನು ಶೀಘ್ರದಲ್ಲೇ ಬಯಲು ಸೀಮೆಯ ಕಡೆಗೆ ಹೊರಡಲಿದ್ದೇನೆ. ನಾನು ಹುಟ್ಟಾ ಹೋರಾಟಗಾರ, ರಣರಂಗದಲ್ಲೇ ಪ್ರಾಣ ಬಿಡುವವನು ನಾನು. ಜನಾನಾ ಹೆಂಗಸರಂತೆ ಇಲ್ಲಿ ಕುಳಿತಿರುವುದು ನನಗೆ ತಕ್ಕುದೆ?

\begin{flushright}
ಇತಿ, ನಿನ್ನ ಪ್ರೀತಿಯ ವಿವೇಕಾನಂದ
\end{flushright}

