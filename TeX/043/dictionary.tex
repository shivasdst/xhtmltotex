\sethyphenation{kannada}{
ಅ
ಅಂಕುಶವು
ಅಂಗಉಪಾಂಗ
ಅಂಗ-ಗಳಿಗೆ
ಅಂಗ-ಗಳು
ಅಂಗಡಿ-ಯನ್ನಿಟ್ಟು
ಅಂಗಭೂತರು
ಅಂಗ-ಲಗ್ನಾಂ
ಅಂಗಳ-ದಲ್ಲಿ
ಅಂಗಾಂಗ-ಗಳಿಗೆ
ಅಂಗಾವ-ಬದ್ಧಾಧಿಕ-ರಣ-ದಲ್ಲಿ
ಅಂಗಾಶ್ರಿತರಾದ
ಅಂಗಿರಾ
ಅಂಗೀರ
ಅಂಗೀರಸ
ಅಂಟಿ-ಕೊಳ್ಳು-ವುದಿಲ್ಲವೋ
ಅಂತಃಕ-ರಣ
ಅಂತಃಕ-ರಣ-ನೈರ್ಮಲ್ಯೇ
ಅಂತಃಕ-ರಣ-ಪೂರ್ವ-ಕ-ವಾಗಿ
ಅಂತಃಕ-ರಣ-ಯುಕ್ತ-ರಾ-ಗುತ್ತಾರೆ
ಅಂತಃಕ-ರಣ-ಯುಕ್ತ-ರಾದ
ಅಂತಃಕರ-ಣವು
ಅಂತಃಕ-ರಣ-ಶುದ್ಧಿಯೇ
ಅಂತಃಪುರ-ವನ್ನು
ಅಂತರಾಯೋ
ಅಂತರಿಕ್ಷ-ದಿಂದ
ಅಂತರ್ಯಾಮಿತ್ವ
ಅಂತರ್ಯಾಮಿ-ಯಾಗಿ
ಅಂತರ್ಯಾಮಿ-ಯಾದ
ಅಂತರ್ಯಾಮ್ಯ-ಧಿದೈ
ಅಂತಸ್ತಿಗೆ
ಅಂತಸ್ಥ-ನಾ-ಗಿ-ರುವ
ಅಂತಹ
ಅಂತಹ-ವನ
ಅಂತಹ-ವ-ನಿಗೆ
ಅಂತಹ-ವನು
ಅಂತಹ-ವನೇ
ಅಂತಹ-ವ-ರನ್ನು
ಅಂತಹ-ವ-ರಿಗೆ
ಅಂತಹ-ವರು
ಅಂತಹುದೇ
ಅಂತ್ಯ-ಕಾಲವು
ಅಂತ್ಯ-ದಲ್ಲಿ
ಅಂತ್ಯ-ಭಾಗ-ದಲ್ಲಿ
ಅಂಥ-ವನ
ಅಂದರೆ
ಅಂಶ-ಗಳನ್ನು
ಅಂಶಾಧಿಕರ-ಣದ
ಅಕರ್ಮ-ಣಶ್ಚ
ಅಕಸ್ಮಾದಂಗಿರಾ-ನಾಮ
ಅಕಾರ್ಯ
ಅಕಾಲ-ಮೃತ್ಯು-ವನ್ನು
ಅಕಾಲ-ಮೃತ್ಯು-ವಿಗೆ
ಅಕೃತ್ವಾ
ಅಕ್ಕರೆ-ಯುಳ್ಳ
ಅಕ್ಕ-ಸಾಲಿಗರು
ಅಕ್ಕಿ
ಅಕ್ಷ
ಅಕ್ಷತಂ
ಅಕ್ಷಯ
ಅಕ್ಷಯಾಃ
ಅಗ-ಸರು
ಅಗಸೆ
ಅಗಸ್ಯ-ಗೋತ್ರ-ದಲ್ಲಿ
ಅಗಾಧ
ಅಗೆದು
ಅಗ್ನಿ
ಅಗ್ನಿ-ಗಳಂತೆ
ಅಗ್ನಿ-ಚಿತ್
ಅಗ್ನಿ-ಯಲ್ಲಿ
ಅಗ್ನಿ-ಯಿಂದ
ಅಗ್ನಿಯು
ಅಗ್ನಿಯೂ
ಅಗ್ರ-ಭಾಗ-ದಲ್ಲಿದ್ದ
ಅಗ್ರಹಾರ-ದಲ್ಲಿ
ಅಘಮರ್ಷಣಸ್ಯಾನ-ವನ್ನು
ಅಚಿ-ರೇಣ
ಅಚ್ಯುತ
ಅಜಗರ-ದಿಂದ
ಅಜಾವಿಗೋಮಹಿಷ್ಯಶ್ಯ
ಅಜಾವಿಗೋಮಹಿಷ್ಯಾಣಾಂ
ಅಜ್ಞಶ್ಚಾಶ್ತ್ರದ್ಧಧಾನತ್ವ
ಅಜ್ಞಾ-ತಲಿಂಗಾ
ಅಜ್ಞಾತ್ವಾ
ಅಜ್ಞಾನ
ಅಜ್ಞಾ-ನದ
ಅಜ್ಞಾ-ನ-ದಿಂದ
ಅಜ್ಞಾ-ನ-ದಿಂದಾಗಲೀ
ಅಜ್ಞಾ-ನವು
ಅಜ್ಞಾ-ನ-ವೆಂಬ
ಅಜ್ಞಾ-ನಾಂಧಂ
ಅಜ್ಞಾ-ನಿ-ಗಳಾದ
ಅಜ್ಞಾ-ನಿ-ಗಳಿಗೆ
ಅಜ್ಞಾ-ನಿ-ಯಾಗಿದ್ದೆ
ಅಜ್ಞಾ-ನಿಯು
ಅಜ್ಞಾ-ನಿಯೂ
ಅಟಿ-ಮಾನಂ
ಅಟ್ಟಿಸಿ-ಕೊಂಡು
ಅಟ್ಟಿಸಿ-ಕೊಂಡು-ಹೋಗಿ
ಅಡಗಿ-ಕೊಂಡರು
ಅಡಗಿ-ಕೊಂಡಿದ್ದ
ಅಡಗಿ-ಕೊಂಡು
ಅಡಗಿ-ಕೊಂಡುವು
ಅಡ್ಡ
ಅಡ್ಡಿ
ಅಡ್ಡಿ-ಯಾಗುತ್ತದೆ
ಅತಂದ್ರಿತಃ
ಅತಃ
ಅತಸೀ-ಪತ್ರಸಂಕಾಶಂ
ಅತಸೀ-ಪುಷ್ಪ
ಅತಿ
ಅತಿ-ಕೀರ್ತಿಂ
ಅತಿಕ್ರಾಂತ್ವಾ
ಅತಿತ್ವರೆ-ಯಿಂದ
ಅತಿ-ಥಿ-ಗ-ಳಾಗಿ
ಅತಿ-ಥಿ-ಗಳಿಗೆ
ಅತಿ-ದೀ-ನ-ರಾದ
ಅತೀವ
ಅತುಲ
ಅತುಲಂ
ಅತೋ
ಅತೋದ್ಯಾಶುಚಿನಾ
ಅತೋಸ್ಮಾಭಿಸ್ತು
ಅತ್ತರು
ಅತ್ತಳು
ಅತ್ತಿಯ
ಅತ್ಯಂತ
ಅತ್ಯಜ್ಞೋ
ಅತ್ಯಧಿಕ
ಅತ್ಯಹಂಕಾರ-ಯುಕ್ತ
ಅತ್ಯುಗ್ರಂ
ಅತ್ಯುಗ್ರ-ಪಾಪಿನಂ
ಅತ್ಯುಗ್ರ-ವಾದ
ಅತ್ಯುತ್ತಮ-ವಾದ
ಅತ್ರಾಬ್ದೀನಿ
ಅತ್ರೈವೋದಾ-ಹರಂತೀಮ-ಮಿತಿ-ಹಾಸಂ
ಅಥ
ಅಥವ
ಅಥವಾ
ಅಥವಾ-ಽವ್ಯಕ್ತ-ತತ್ವಾತ್ಮಾ
ಅಥಾತಿ-ವಿಸ್ಮಿತೋ
ಅಥಾಭ್ರುವನ್
ಅದಕ್ಕಿಂತ
ಅದಕ್ಕಿಂತಲೂ
ಅದಕ್ಕೆ
ಅದನ್ನು
ಅದನ್ನೆಲ್ಲ
ಅದರ
ಅದ-ರಂತೆ
ಅದ-ರಲ್ಲಿ
ಅದ-ರಲ್ಲಿಯೂ
ಅದ-ರಿಂದ
ಅದ-ರಿಂದಲೇ
ಅದರಿ-ದವು
ಅದರು-ವು-ದನ್ನೂ
ಅದಾನ-ಮವ್ರತಂ
ಅದಾನೇ
ಅದು
ಅದೃಶ್ಯ-ನಾ-ಗಿ-ರುವಿ
ಅದೃಶ್ಯ-ನಾದನು
ಅದೆಲ್ಲ-ವನ್ನೂ
ಅದೇ
ಅದೈ-ವತ
ಅದೈ-ವತ-ನೆಂಬ
ಅದ್ಭುತ
ಅದ್ಭುತ-ಗಳನ್ನು
ಅದ್ಭುತ-ನಾದ
ಅದ್ಯ
ಅದ್ಯೈವ
ಅಧಃಪ-ತನ
ಅಧಃಸ್ಥಲಂ
ಅಧರ್ಮಂ
ಅಧರ್ಮ-ಕರ್ಮ-ಮಾಡು-ವುದು
ಅಧರ್ಮ-ದಿಂದ
ಅಧರ್ಮ-ನಿ-ರತಂ
ಅಧರ್ಮ-ಮಾರ್ಗ-ದಲ್ಲಿ
ಅಧರ್ಮ-ವನ್ನು
ಅಧರ್ಮ-ವೆಂತಲೂ
ಅಧರ್ಮಸ್ಥ-ಮಪಿ
ಅಧರ್ಮಾ-ಚ-ರಣೆ-ಯಲ್ಲಿ
ಅಧರ್ಮಾ-ದಿ-ದಾತಾರೋ
ಅಧರ್ಮೇ
ಅಧಿಕ
ಅಧಿಕ-ರಣ
ಅಧಿಕ-ರಣ-ದಲ್ಲಿ
ಅಧಿಕ-ವಾಗುತ್ತಿತ್ತು
ಅಧಿಕ-ವಾದ
ಅಧಿಕ-ವಾದ-ವು-ಗಳು
ಅಧಿಕ-ವೆಂದು
ಅಧಿ-ಕಾರ
ಅಧಿ-ಕಾರಂ
ಅಧಿ-ಕಾರ-ಕೊಟ್ಟು
ಅಧಿ-ಕಾರ-ವಿಲ್ಲ
ಅಧಿ-ಕಾರವೇ
ಅಧಿ-ಕಾರಸ್ಯ
ಅಧಿ-ಕಾರಿ
ಅಧಿ-ಕಾರಿ-ಗಳು
ಅಧಿ-ಕಾರಿ-ಯಾಗಿ
ಅಧಿ-ಕಾರಿಯು
ಅಧಿ-ಪತಿ-ಯಾದ
ಅಧೀತ್ಯ
ಅಧೀನ
ಅಧೀನ-ದಲ್ಲಿ
ಅಧೀನ-ದಲ್ಲಿಯೇ
ಅಧೀನ-ದಲ್ಲಿ-ರುವ
ಅಧೀನ-ರಾದ
ಅಧೀನರು
ಅಧೀನ-ವೆಂಬು-ದನ್ನು
ಅಧೀನವೇ
ಅಧೋ
ಅಧೋ-ಲೋಕ
ಅಧೋ-ಲೋಕ-ವೆಂಬುದು
ಅಧ್ಯ-ಯನ
ಅಧ್ಯಾತ್ಮಜ್ಞಾ-ನ-ರ-ಹಿತರು
ಅಧ್ಯಾಯ
ಅಧ್ಯಾ-ಯಕ್ಕೆ
ಅಧ್ಯಾ-ಯ-ಗಳಿಂದ
ಅಧ್ಯಾ-ಯದ
ಅಧ್ಯಾ-ಯ-ದಲ್ಲಿ
ಅಧ್ಯಾ-ಯವು
ಅಧ್ವಶ್ರಾಂತಂ
ಅಧ್ವಸ್ಥಿತೋ
ಅನಂತ
ಅನಂತಂ
ಅನಂತ-ಕೋಟಿ
ಅನಂತ-ಜನ್ಮಾನು-ಗತಂ
ಅನಂತ-ಪುಣ್ಯ
ಅನಂತ-ವಾಗಿದೆ
ಅನಘ
ಅನಘಸ್ಯ
ಅನನ್ಯ
ಅನನ್ಯ-ವಾದ
ಅನ-ಪತ್ರೋ
ಅನಯಾ
ಅನರ್ಹ-ನೆಂದು
ಅನವಶ್ಯಕ-ವಾದ
ಅನಸ್ತಿ
ಅನಾತ್ಮಜ್ಞ-ನಾದ
ಅನಾತ್ಮಜ್ಞಾ
ಅನಾತ್ಮಜ್ಞೇನ
ಅನಾಥ
ಅನಾಥಪ್ರೇತ
ಅನಾಥಪ್ರೇತ-ಸಂಸ್ಕಾರಂ
ಅನಾಥಪ್ರೇತ-ಸಂಸ್ಕಾರಾಕ್
ಅನಾದಿ
ಅನಾನಿತಾಃ
ಅನಿಂದ-ಯತ್
ಅನಿರ್ಮೂಲಂ
ಅನಿಷ್ಟ-ವಾದ
ಅನು-ಗುಣ
ಅನು-ಗುಣ-ವಾಗಿ
ಅನು-ಗುಣ-ವಾಗಿಯೇ
ಅನು-ಗುಣ-ವಾಗಿರು
ಅನು-ಗುಣ-ವಾದ
ಅನುಗೃಹೀತ-ನಾಗಿ-ರುತ್ತೇನೆ
ಅನುಗ್ರಹ
ಅನುಗ್ರಹಕ್ಕೆ
ಅನುಗ್ರಹದ
ಅನುಗ್ರಹ-ದಿಂದ
ಅನುಗ್ರಹ-ದಿಂದಲೇ
ಅನುಗ್ರಹ-ಪೂರ್ವ-ಕ-ವಾದ
ಅನುಗ್ರಹ-ಮಾಡ-ಬೇಕೆಂದು
ಅನುಗ್ರಹ-ವನ್ನು
ಅನುಗ್ರಹ-ವಾ-ಗಲು
ಅನುಗ್ರಹ-ವಿಲ್ಲದೇ
ಅನುಗ್ರಹವು
ಅನುಗ್ರಹವೂ
ಅನುಗ್ರಹಿಸು
ಅನುಗ್ರಹಿಸುತ್ತಾನೆ
ಅನುಗ್ರಹಿಸು-ವನು
ಅನುಚರರೊಂದಿಗೆ
ಅನುಜಗ್ಮುರ್ಭಕ್ಷ-ಯಿತುಂ
ಅನುಜ್ಞಾ
ಅನು-ತಾಪ
ಅನುಧ್ಯಾ-ಯನ್
ಅನುಭವಕ್ಕೆ
ಅನುಭವ-ಸಿದ್ದ
ಅನುಭವ-ಸಿದ್ಧ
ಅನುಭವಿಸ-ಬಹು-ದಾದ
ಅನುಭವಿಸಲಾರೆ
ಅನುಭವಿ-ಸ-ಲಿಲ್ಲ
ಅನುಭವಿಸಲೇ-ಬೇಕು
ಅನುಭವಿಸಿ
ಅನುಭವಿ-ಸಿದೆ
ಅನುಭವಿಸಿ-ದೆವು
ಅನುಭವಿಸಿಯೇ
ಅನುಭವಿ-ಸುತ್ತಾ
ಅನುಭವಿ-ಸುತ್ತಾನೆ
ಅನುಭವಿ-ಸುತ್ತಾರೆ
ಅನುಭವಿಸು-ವುದಿಲ್ಲ
ಅನುಭೂಯ
ಅನು-ರಕ್ತ-ನಾ-ಗಿ-ರುವುದು
ಅನು-ವಾದ
ಅನು-ವಾದ-ಕನ
ಅನು-ವಾದ-ಮಾಡಿ
ಅನು-ವಾದಿಸಿ
ಅನುಷ್ಠಾತಾ
ಅನುಷ್ಠಾನ
ಅನುಷ್ಠಾನ-ವತಾಂ
ಅನುಷ್ಠಾನ-ವಿಧಿಜ್ಞೇನ
ಅನುಷ್ಠಾನ-ವಿಧಿ-ಯನ್ನು
ಅನುಸಂಧಾನ
ಅನು-ಸರಿ-ಸಲು
ಅನು-ಸರಿಸಿ
ಅನು-ಸರಿ-ಸುತ್ತಾ
ಅನು-ಸರಿ-ಸುತ್ತಾರೆ
ಅನು-ಸರಿ-ಸುತ್ತಿ-ರುವಿರಿ
ಅನು-ಸರಿ-ಸು-ವ-ವನು
ಅನು-ಸರಿ-ಸುವ-ವನೂ
ಅನು-ಸರಿ-ಸುವ-ವರು
ಅನು-ಸಾರ-ವಾಗಿ
ಅನು-ಸಾರಿ-ಯಾದ
ಅನುಸ್ವರ-ಗಳೂ
ಅನೂರ್ಧ್ವ-ಲೋಕಗಾಃ
ಅನೇಕ
ಅನೇನ
ಅನ್ನ
ಅನ್ನಂ
ಅನ್ನ-ಗಳನ್ನೂ
ಅನ್ನ-ದಾನ
ಅನ್ನ-ದಾನಂ
ಅನ್ನ-ವಾನ್
ಅನ್ನವು
ಅನ್ನಾ
ಅನ್ನಾ-ದಿ-ಗಳನ್ನೂ
ಅನ್ನೋಪಾಯ-ವಿಲ್ಲ
ಅನ್ಯ
ಅನ್ಯಥಾ
ಅನ್ಯ-ದೇವ-ತೆ-ಗಳನ್ನು
ಅನ್ಯ-ದೇವ-ತೆ-ಗಳು
ಅನ್ಯ-ದೇವಸ್ಯ
ಅನ್ಯ-ಮನಸ್ಕ-ನಾಗಿ
ಅನ್ಯ-ರನ್ನು
ಅನ್ಯ-ರಲ್ಲ
ಅನ್ವೀಕ್ಷಿಕೀ
ಅಪ-ಕಾರ
ಅಪಜಯ
ಅಪಮತ್ಯು
ಅಪ-ಮೃತ್ಯು-ವನ್ನು
ಅಪ-ಮೃತ್ಯು-ವಿಗೆ
ಅಪ-ಮೃತ್ಯು-ವಿನಾ-ಶಾಯ
ಅಪ-ಮೃತ್ಯು-ವಿ-ನಿಂದ
ಅಪರ
ಅಪರಾ-ಧ-ಗಳನ್ನು
ಅಪರಾ-ಧ-ಗಳನ್ನೂ
ಅಪರಾ-ಧ-ವನ್ನು
ಅಪರಾ-ಧ-ವನ್ನೇ
ಅಪರಾ-ಧ-ವಲ್ಲ
ಅಪರಾ-ಧವೇ
ಅಪರಾ-ಧ-ಸಹಸ್ರಂ
ಅಪರೋಕ್ಷ
ಅಪರೋಕ್ಷಕ್ಕಿಂತ
ಅಪರೋಕ್ಷಜ್ಞಾ-ನಕ್ಕಾಗಿ
ಅಪರೋಕ್ಷಜ್ಞಾ-ನಕ್ಕೆ
ಅಪರೋಕ್ಷಜ್ಞಾ-ನ-ವನ್ನು
ಅಪರೋಕ್ಷಜ್ಞಾ-ನವು
ಅಪರೋಕ್ಷಜ್ಞಾ-ನ-ವೆಂಬ
ಅಪರೋಕ್ಷಜ್ಞಾ-ನಿ-ಗಳೇ
ಅಪರೋಕ್ಷ-ವನ್ನು
ಅಪರೋಕ್ಷಾ-ನಂತರ
ಅಪ-ವಾದ
ಅಪ-ವಾದ-ವನ್ನು
ಅಪಶಕು-ನ-ಗಳ
ಅಪಶಕುನ-ಗಳನ್ನು
ಅಪಶಕುನ-ಗಳನ್ನೂ
ಅಪಶಕು-ನ-ಗಳು
ಅಪ-ಹರಿ-ಸಲು
ಅಪ-ಹರಿ-ಸಿ-ಕೊಂಡು
ಅಪ-ಹರಿ-ಸಿ-ಕೊಂಡು-ಹೋ-ದನು
ಅಪ-ಹರಿ-ಸಿ-ದಂತೆ
ಅಪ-ಹರಿ-ಸಿದೆ
ಅಪ-ಹರಿ-ಸಿದೆವು
ಅಪ-ಹರಿ-ಸು-ವುದು
ಅಪಹಾರ-ಮಾಡು-ವುದು
ಅಪ-ಹಾಸ್ಯ
ಅಪಾತ್ರ-ರಿಗೆ
ಅಪಾನಃ
ಅಪಾನ-ನೆಂಬ
ಅಪಾರ
ಅಪಾರ-ವಾದ
ಅಪಿ
ಅಪುತ್ರಾಯ
ಅಪುಷ್ಪಂ
ಅಪೂ-ಜ-ಯಿತ್ವಾ
ಅಪೂಜಿತಾ
ಅಪೂಪ-ಗಳನ್ನು
ಅಪೇಕ್ಷಿಸದೆ
ಅಪೇಕ್ಷಿ-ಸುತ್ತಾ
ಅಪೇಕ್ಷಿ-ಸುವ
ಅಪ್ಪಣೆ
ಅಪ್ಪಣೆ-ಯಂತೆ
ಅಪ್ಪಣೆ-ಯನ್ನು
ಅಪ್ಪಣೆ-ಯಿಂದ
ಅಪ್ರಧಾನಂ
ಅಪ್ರಧಾನ-ವಾದುದೇ
ಅಪ್ರಾಮಾಣಿಕ-ವಾದ
ಅಪ್ಸರಾತ್ರೀ-ಯ-ರೊಡನೆ
ಅಭಕ್ಷ್ಯ
ಅಭದ್ರ
ಅಭದ್ರನ
ಅಭದ್ರ-ನೆಂಬ
ಅಭಾವೇ
ಅಭಿತಃ
ಅಭಿತೋ
ಅಭಿನಂದಿಸಿ
ಅಭಿ-ನಿಂದ್ಯ
ಅಭಿಪ್ರಾಯ
ಅಭಿಪ್ರಾಯ-ದಲ್ಲಿ
ಅಭಿಪ್ರಾಯ-ಪಡುತ್ತಾರೆ
ಅಭಿಪ್ರಾಯ-ಹಲ್ಲನ್ನು
ಅಭಿಮಾನ
ಅಭಿ-ಮಾನಿ
ಅಭಿ-ಮಾನಿ-ಗಳು
ಅಭಿ-ಮಾನಿ-ದೇವ-ತೆ-ಗಳಿದ್ದರೆ
ಅಭಿ-ಮಾನಿ-ದೇವ-ತೆ-ಗಳು
ಅಭಿ-ವೃದ್ಧಿ
ಅಭಿ-ವೃದ್ಧಿಗೆ
ಅಭಿ-ವೃದ್ಧಿ-ಪಡಿಸಿ-ಕೊಳ್ಳುವ-ವರು
ಅಭಿ-ವೃದ್ಧಿ-ಯನ್ನು
ಅಭಿ-ವೃದ್ಧಿ-ಯಾ-ಗಲು
ಅಭಿ-ವೃದ್ಧಿ-ಯಾಗಿ
ಅಭಿ-ವೃದ್ಧಿ-ಯಾಗುತ್ತದೆ
ಅಭಿ-ವೃದ್ಧಿ-ಯಾಗು-ವುದಿಲ್ಲ
ಅಭಿ-ವೃದ್ಧಿ-ಯಾಗುವುದು
ಅಭಿ-ವೃದ್ಧಿ-ಯಾ-ಯಿತು
ಅಭಿಷೇಕ
ಅಭೀಷ್ಟ-ಗಳನ್ನು
ಅಭೀಷ್ಟ-ಗಳೂ
ಅಭೂತ್
ಅಭೂದ್ಯುದ್ಧಂ
ಅಭ್ಯಂ
ಅಭ್ಯಂಜನ
ಅಭ್ಯಂಜನ-ವನ್ನು
ಅಭ್ಯಂಜನಸ್ನಾ-ನ-ಮಾಡುತ್ತಾ
ಅಭ್ಯಕ್ತಂ
ಅಭ್ಯಸಿಸಿ
ಅಭ್ಯಾಸ
ಅಭ್ಯಾಸಂ
ಅಭ್ಯಾಸ-ಮಾಡಿ
ಅಭ್ಯಾಸ-ವನ್ನಾಗಲೀ
ಅಭ್ಯಾ-ಸೇನ
ಅಭ್ಯುದಯ-ವನ್ನು
ಅಭ್ರ-ಸರೋ-ವರ-ದಲ್ಲಿ
ಅಮಂಗಳವು
ಅಮ-ವಾಸ್ಯೆಯು
ಅಮಾ
ಅಮಾಘ
ಅಮಾ-ಘನು
ಅಮಾ-ಘ-ನೆಂಬ
ಅಮಾ-ಘೇಽ
ಅಮಾ-ಘೋಯಂ
ಅಮಾ-ಮೇಕ-ದಶೀಂ
ಅಮಾ-ವಾಸ್ಯೆ
ಅಮಾ-ವಾಸ್ಯೆ-ಗಳಲ್ಲಿ
ಅಮಾ-ವಾಸ್ಯೆ-ಗಳೂ
ಅಮಾಸಂ
ಅಮುನಾ
ಅಮೃತ-ವನ್ನು
ಅಮೃತವು
ಅಮ್ಮ
ಅಮ್ಮತ
ಅಮ್ಮ-ತ-ಸೇ-ವನೆ-ಯಿಂದ
ಅಯಂ
ಅಯಮ್
ಅಯಾಚಿ-ತೇಭ್ಯಃ
ಅಯಾಜ್ಯ
ಅಯೋಧ್ಯಾ
ಅರಗು
ಅರಣ್ಯ
ಅರಣ್ಯಂ
ಅರಣ್ಯಕ್ಕೆ
ಅರಣ್ಯ-ದಲ್ಲಿ
ಅರಣ್ಯವು
ಅರಣ್ಯೇ
ಅರಣ್ಯೌದುಂಬರೇ
ಅರ-ಮನೆಗೆ
ಅರ-ಮನೆಯ
ಅರ-ಮನೆ-ಯಲ್ಲಿದ್ದು
ಅರಳಿದ
ಅರಳಿಸಿ
ಅರ-ವತ್ತಾರು
ಅರವತ್ತು
ಅರಸನ
ಅರಾಜಕಂ
ಅರಿತು
ಅರಿ-ಯದೆ
ಅರಿಯ-ಬೇಕು
ಅರಿಯೆ
ಅರಿಷಡ್ವರ್ಗ-ವನ್ನು
ಅರುಣೋದಯ
ಅರುಣೋದಯ-ದಲ್ಲಿ
ಅರುಣೋದಯ-ವೇ-ಲಾಯಾಂ
ಅರುಹಿ-ದನು
ಅರೆಯುಲ್ಪಟ್ಟು
ಅರ್ಘ
ಅರ್ಘ-ಗಳನ್ನು
ಅರ್ಘ್ಯ
ಅರ್ಘ್ಯ-ಕೊಡ-ಬೇಕು
ಅರ್ಚಕ-ನನ್ನು
ಅರ್ಚ-ಕನು
ಅರ್ಚ-ಕನೇ
ಅರ್ಚಕ-ರಲ್ಲಿ
ಅರ್ಚಿತಸ್ತುಲಸೀ-ಪತ್ರೈರ್ಮಾಧವೋ
ಅರ್ಚಿಸ-ಬೇಕು
ಅರ್ಚಿಸಿ
ಅರ್ಚಿ-ಸುವ
ಅರ್ಜಿತಂ
ಅರ್ಜುನ
ಅರ್ಜುನ-ನಿಗೆ
ಅರ್ಜುನನು
ಅರ್ಜುನನೇ
ಅರ್ಥ
ಅರ್ಥ-ಗಳನ್ನೂ
ಅರ್ಥ-ಗಳು
ಅರ್ಥ-ಗಳೂ
ಅರ್ಥ-ವಿ-ವರ-ಣೆ-ಯಲ್ಲಿ
ಅರ್ಥ-ಸ-ಹಿತ
ಅರ್ಥಾತುರಸ್ಯ
ಅರ್ಥಾಭಿಮಾನಕ್ಕೂ
ಅರ್ಪಣೆ
ಅರ್ಪಣೆ-ಮಾಡಿ-ದರೆ
ಅರ್ಪಣೆ-ಮಾಡುತ್ತಾರೆ
ಅರ್ಪಣೆಯ
ಅರ್ಪಿತ
ಅರ್ಪಿತ-ವಾದ
ಅರ್ಪಿಸದೆ
ಅರ್ಪಿಸದೇ
ಅರ್ಪಿಸ-ಬೇಕು
ಅರ್ಪಿ-ಸಿದ
ಅರ್ಪಿಸಿ-ದನು
ಅರ್ಪಿ-ಸಿ-ದರೆ
ಅರ್ಪಿಸಿ-ದಳು
ಅರ್ಪಿ-ಸಿದ-ವನ
ಅರ್ಪಿ-ಸಿದ್ದೇನೆ
ಅರ್ಪಿ-ಸಿ-ರುವ
ಅರ್ಪಿಸು
ಅರ್ಪಿಸುತ್ತಾರೆಯೋ
ಅರ್ಪಿಸುತ್ತಿದ್ದಿಲ್ಲ
ಅರ್ಬುದ
ಅರ್ಬು-ದಾನಿ
ಅರ್ಹನಲ್ಲ
ಅರ್ಹ-ನಾಗಿ-ರುತ್ತೀಯೆ
ಅಲಂಕರಿಸಿ
ಅಲಂಕರಿ-ಸುವಳೋ
ಅಲಂಕರೋತಿ
ಅಲಂಕಾರ
ಅಲಂಕಾರ-ಮಾಡುವ
ಅಲಕಾ-ಪು-ರಿಗೆ
ಅಲಕಾ-ಪುರಿ-ಯಲ್ಲಿ
ಅಲಕ್ಷಿಸಿದ್ದಕ್ಕೆ
ಅಲಸಾ
ಅಲಿಪ್ಯ
ಅಲೆ-ಗಳು
ಅಲೆ-ದೆವು
ಅಲೆಯುತ್ತಾ
ಅಲೆಯುತ್ತಿ-ರುವಾಗ
ಅಲ್ಪ
ಅಲ್ಪ-ಮಾತ್ರ-ಕೃತೋ
ಅಲ್ಪ-ಸೇವೆಯು
ಅಲ್ಪಾಯುಷಿ-ಯಾಗಿ
ಅಲ್ಪಾಯುಷಿ-ಯಾಗಿದ್ದಾನೆ
ಅಲ್ಲ
ಅಲ್ಲ-ಗಳೆದು
ಅಲ್ಲಲ್ಲಿ
ಅಲ್ಲಾಡಿ-ಸಿ-ದರೆ
ಅಲ್ಲಾಡಿಸುತ್ತದೆ
ಅಲ್ಲಾಡಿಸುತ್ತಿ-ರುವ
ಅಲ್ಲಿ
ಅಲ್ಲಿಂದ
ಅಲ್ಲಿಗೆ
ಅಲ್ಲಿಟ್ಟು
ಅಲ್ಲಿದ್ದ
ಅಲ್ಲಿಯ
ಅಲ್ಲಿಯೇ
ಅಳಂಬೀ
ಅಳ-ತೆಯು
ಅಳಿ-ಯನ
ಅಳಿಯ-ನನ್ನು
ಅಳಿಯ-ನಾಗಿದ್ದಾನೆ
ಅಳಿಯ-ನಿಗೂ
ಅಳಿ-ಯನು
ಅಳಿ-ಯನೂ
ಅಳಿಸಿ-ಬಿಡು-ವನು
ಅಳುತ್ತಿ-ರುವು-ದನ್ನು
ಅವಂತಿ
ಅವಕಾಶ-ಮಾಡಿ-ಕೊಳ್ಳುತ್ತಾನೆ
ಅವಕ್ತತ್ವ-ನಿ-ರೂಪಿಣ್ಯೌ
ಅವಜ್ಞಾತಾ
ಅವತಾರ
ಅವತಾ-ರ-ಗಳನ್ನು
ಅವತಾ-ರವು
ಅವದಾಮ
ಅವಧೀದ್ಬಲೀ
ಅವನ
ಅವ-ನನ್ನು
ಅವ-ನನ್ನೂ
ಅವ-ನಲ್ಲಿ
ಅವ-ನಿಂದ
ಅವ-ನಿಂದಲೇ
ಅವ-ನಿಗೆ
ಅವನು
ಅವನೂ
ಅವನೇ
ಅವ-ನೊಡನೆ
ಅವನ್ನೇ
ಅವಭೃತದ
ಅವಮಾನ
ಅವಮಾನ-ಗೊಳಿಸದೆ
ಅವಮಾನ-ಗೊಳಿ-ಸಿದ
ಅವಯವ-ಗಳನ್ನುಳ್ಳ
ಅವಯವ-ಗಳಲ್ಲಿ
ಅವಯವ-ಗಳು
ಅವರ
ಅವ-ರನ್ನು
ಅವ-ರಲ್ಲಿ
ಅವ-ರಲ್ಲಿಗೆ
ಅವ-ರಲ್ಲಿದ್ದ
ಅವ-ರಲ್ಲಿ-ರುವ
ಅವರ-ವರ
ಅವರ-ವ-ರಲ್ಲಿ
ಅವರ-ವ-ರಿಗೆ
ಅವ-ರಿಂದ
ಅವ-ರಿಗೂ
ಅವ-ರಿಗೆ
ಅವರಿ-ಗೋಸ್ಕರ
ಅವರಿಬ್ಬರ
ಅವರಿಬ್ಬ-ರನ್ನೂ
ಅವರಿಬ್ಬ-ರಿಗೂ
ಅವರಿಬ್ಬ-ರಿಗೆ
ಅವರಿಬ್ಬರು
ಅವರಿಬ್ಬರೂ
ಅವರು
ಅವರೆಲ್ಲರೂ
ಅವರೇ
ಅವಲಂಬಿ-ಸಿದ್ದೆ
ಅವ-ಲೋಕನ
ಅವ-ಲೋಕ್ಯ
ಅವಳ
ಅವಳನ್ನು
ಅವಳಲ್ಲಿ
ಅವಳಿಗೆ
ಅವಳಿಜವಳಿ
ಅವಳು
ಅವಳೂ
ಅವಶ್ಯ
ಅವಶ್ಯಂ
ಅವಶ್ಯಕ
ಅವಶ್ಯ-ಕತೆ
ಅವಶ್ಯ-ಕ-ವಾದ
ಅವಶ್ಯ-ಕ-ವೆಂದು
ಅವಶ್ಯ-ವಾಗಿ
ಅವಸರ-ವಾಗಿ
ಅವಸ್ಥೆ-ಗಳು
ಅವ-ಹತ್ಯ
ಅವಹೇಳನ
ಅವಾಂತರ
ಅವಿ-ದೂರೇಣ
ಅವಿದ್ಯಾ
ಅವಿದ್ವಾ-ನಯಮಿತ್ತುಕ್ತ್ವಾ
ಅವಿವೇಕಿ-ಗಳಿಗೆ
ಅವು
ಅವು-ಗಳ
ಅವು-ಗಳನ್ನು
ಅವು-ಗಳನ್ನೇ
ಅವು-ಗಳಲ್ಲಿ
ಅವು-ಗಳಿಂದ
ಅವು-ಗಳಿಗೆ
ಅವೆಲ್ಲ-ವನ್ನೂ
ಅವೇ
ಅವ್ಯಕ್ತ-ತತ್ತ್ವಕ್ಕೆ
ಅವ್ಯಕ್ತತ್ಯ-ದಿಂದ
ಅವ್ಯಕ್ತ್ಯಾದ್ಯಾಃ
ಅಶಕ್ತ-ನಾದ
ಅಶಕ್ತ-ನಾದ-ವನು
ಅಶಕ್ಯ
ಅಶನಾನಿ
ಅಶರೀರವಾಣಿ
ಅಶರೀರವಾಣಿಯು
ಅಶಿಮಿದಾ-ಭವಂತು
ಅಶಿವಂ
ಅಶುಚಿ-ಯಾದ
ಅಶೃಣ್ವನ್
ಅಶ್ರೋತಾ
ಅಶ್ವತ-ದಂತೆ
ಅಶ್ವತ್ಥ
ಅಶ್ವತ್ಥಃ
ಅಶ್ವತ್ಥ-ತುಲಸೀ-ಗಳು
ಅಶ್ವತ್ಥ-ದಲ್ಲಿ
ಅಶ್ವತ್ಥ-ಮರದ
ಅಶ್ವತ್ಥ-ವೃಕ್ಷ
ಅಶ್ವತ್ಥ-ವೃಕ್ಷ-ದಲ್ಲಿ
ಅಶ್ವತ್ಥೇ
ಅಶ್ವಮೇಧ
ಅಶ್ವಮೇಧ-ಸಹಸ್ರೇಷು
ಅಶ್ವಿನೀ-ದೇವ-ತೆ-ಗಳೂ
ಅಷ್ಟದಳ
ಅಷ್ಟಧೇತಿ
ಅಷ್ಟಮಾ
ಅಷ್ಟ-ಮಿ-ಯಂದು
ಅಷ್ಟಮೋಧ್ಯಾಯಃ
ಅಷ್ಟಾ-ದಶಾನಾಂ
ಅಷ್ಟು
ಅಷ್ಟೇ
ಅಷ್ಟೋ
ಅಸಂಖ್ಯ
ಅಸಂಖ್ಯಾತ-ವಾದ
ಅಸಂಶಯಂ
ಅಸಂಸ್ಕೃತಪ್ರಮೀ-ತಾನಾಂ
ಅಸತ್ಯ-ವಲ್ಲ
ಅಸತ್ಯವೋ
ಅಸಮರ್ಥ
ಅಸಾಧ್ಯ-ವಾ-ದುದು
ಅಸಾಧ್ಯವೋ
ಅಸಿಕ್ನ್ಯಾ
ಅಸೌ
ಅಸ್ಕಾನಂ
ಅಸ್ತೀತಿ
ಅಸ್ತುವಂ
ಅಸ್ಥಿ
ಅಸ್ಥಿ-ರ-ಹಿತ-ವಾದ
ಅಸ್ನಾತೋ
ಅಸ್ನಾತ್ವಾ
ಅಸ್ಮತ್
ಅಸ್ಮತ್ಸಹಾ-ಗ-ತಾನಾಂ
ಅಸ್ಮದ್ವೃತ್ತಾಂತ-ಮುತ್ತಮಮ್
ಅಸ್ಮಾಕಂ
ಅಸ್ಮಾತ್
ಅಸ್ಮಾನ್
ಅಸ್ಮಾಭಿಃ
ಅಸ್ಮಾಭಿರನುಗೈಃ
ಅಸ್ಮಾಭಿಸ್ತು
ಅಸ್ಮಿ
ಅಸ್ಮಿನ್
ಅಸ್ಮಿನ್ಮಾಗಾಹ್ವಯಾಂ
ಅಸ್ಮಿನ್ವನೇ
ಅಸ್ವರ್ಗಂ
ಅಸ್ವಾತಂತ್ರ್ಯವು
ಅಸ್ಸಾತ್ವಾ
ಅಹಂ
ಅಹಂಕಾರ
ಅಹಂಕಾರ-ತತ್ವಕ್ಕೂ
ಅಹಂಕಾರ-ತತ್ವಾಭಿ-ಮಾನಿಯು
ಅಹಂಕಾರ-ದಿಂದ
ಅಹಂಕಾರ-ನಿ-ದಾನೇನ
ಅಹಂಕಾರ-ಪಡುವ-ವರು
ಅಹಂಕಾರ-ಯುಕ್ತ-ನಾಗಿ
ಅಹಂಕಾರ-ಯುಕ್ತ-ನಾಗಿದ್ದೆ
ಅಹಂಕಾರ-ಯುಕ್ತ-ರಾದ
ಅಹಂಕಾರ-ರ-ಹಿತ-ನಾದ
ಅಹಂಕಾರ-ವಿರ-ಬಾರದು
ಅಹಂಕಾರ-ವಿ-ಹೀ-ನಸ್ಯ
ಅಹಮ್
ಅಹವಾತ್ಮಾ
ಅಹುತ್ವಾ
ಆ
ಆಂಧ್ರ
ಆಂಧ್ರ-ದೇಶದ
ಆಂಧ್ರ-ರಾಜ್ಯ-ದಲ್ಲಿ
ಆಕಳು-ಗಳೂ
ಆಕಾರ
ಆಕಾಶ
ಆಕಾಶದ
ಆಕಾಶ-ದಿಂದ
ಆಕಾಶ-ಮಾರ್ಗ-ದಲ್ಲಿ
ಆಕಾಶ-ವನ್ನು
ಆಕಾಶವು
ಆಕೆಯ
ಆಕ್ರಮಿಸಿ-ಕೊಳ್ಳಲ್ಪಟ್ಟಿತು
ಆಗ
ಆಗಂಧಂ
ಆಗಚ್ಛಂತೌ
ಆಗತಂ
ಆಗತಾನೆ
ಆಗತೋ
ಆಗದಿರಲಿ
ಆಗ-ಬಹು-ದಾದ
ಆಗ-ಬಾರದೆಂಬ
ಆಗಮ-ನ-ದಿಂದ
ಆಗಮುಂ
ಆಗಲಿ
ಆಗ-ಲಿಲ್ಲ
ಆಗಲೀ
ಆಗಲೂ
ಆಗಲೆಂದು
ಆಗಲೇ
ಆಗಸ್ತ್ಯ-ಗೋತ್ರೇ
ಆಗಾಗ್ಗೆ
ಆಗಾದ್ದೇವೇಂದ್ರಸದನೇ
ಆಗಿ
ಆಗಿದೆ
ಆಗಿದ್ದನು
ಆಗಿದ್ದರೂ
ಆಗಿನ
ಆಗಿ-ರುತ್ತಾರೆ
ಆಗಿ-ರು-ವರೋ
ಆಗಿ-ರುವಿ
ಆಗುತ್ತದೆ
ಆಗುತ್ತ-ದೆಯೋ
ಆಗುವ
ಆಗು-ವುದಿಲ್ಲ
ಆಗ್ನೇಯ
ಆಘ್ರಾಣಿಸಿ-ದರು
ಆಚ-ರಣೆ
ಆಚ-ರಣೆಗೆ
ಆಚ-ರಣೆ-ಯನ್ನು
ಆಚ-ರಣೆ-ಯಿಂದ
ಆಚರತಿ
ಆಚರಿ-ಸ-ತಕ್ಕ
ಆಚರಿ-ಸದೆ
ಆಚರಿ-ಸದೇ
ಆಚರಿ-ಸ-ಬಾರದು
ಆಚರಿ-ಸ-ಬೇಕಾದ
ಆಚರಿ-ಸ-ಬೇಕು
ಆಚರಿ-ಸಲಾ-ಯಿತು
ಆಚರಿ-ಸ-ಲಿಲ್ಲ
ಆಚರಿ-ಸಲು
ಆಚರಿ-ಸಲೇ
ಆಚರಿ-ಸಲ್ಪಟ್ಟ
ಆಚರಿ-ಸಲ್ಪಡ-ಲಿಲ್ಲ
ಆಚರಿಸಿ
ಆಚರಿ-ಸಿದ
ಆಚರಿ-ಸಿ-ದನು
ಆಚರಿ-ಸಿ-ದರೂ
ಆಚರಿ-ಸಿ-ದರೆ
ಆಚರಿ-ಸಿ-ದಲ್ಲಿ
ಆಚರಿ-ಸಿದೆ
ಆಚರಿ-ಸಿದ್ದ
ಆಚರಿ-ಸಿದ್ದರೂ
ಆಚರಿ-ಸಿ-ರುತ್ತಾರೆ
ಆಚರಿ-ಸುತ್ತಾ
ಆಚರಿ-ಸುತ್ತಾನೆ
ಆಚರಿ-ಸುತ್ತಾರೆ
ಆಚರಿ-ಸುತ್ತಾ-ರೆಯೋ
ಆಚರಿ-ಸುತ್ತಿದ್ದಾಗ
ಆಚರಿ-ಸುತ್ತಿದ್ದಿಲ್ಲ
ಆಚರಿ-ಸುತ್ತಿ-ರ-ಲಿಲ್ಲ
ಆಚರಿ-ಸುತ್ತೀಯೋ
ಆಚರಿ-ಸುವ
ಆಚರಿ-ಸುವನೋ
ಆಚರಿ-ಸುವರೋ
ಆಚರಿ-ಸುವ-ವನ
ಆಚರಿ-ಸುವ-ವನು
ಆಚರಿ-ಸುವ-ವರು
ಆಚರಿ-ಸುವ-ವರೂ
ಆಚರಿ-ಸುವಾಗ
ಆಚರಿ-ಸುವು-ದನ್ನು
ಆಚರಿ-ಸು-ವುದ-ರಿಂದ
ಆಚರಿ-ಸು-ವುದ-ರಿಂದಲೇ
ಆಚರಿ-ಸು-ವುದು
ಆಚಾರ
ಆಚಾರ-ಗಳನ್ನು
ಆಚಾರ-ವಿಚಾ-ರವ್ಯ-ವಹಾರ-ಗಳಲ್ಲಿಯೂ
ಆಚಾ-ರಾದ್ವ್ಯ
ಆಚಾರ್ಯಃ
ಆಚಾರ್ಯಾಣೀ
ಆಚ್ಛಾದಿಸಲ್ಪಟ್ಟ
ಆಜಗಾಮ
ಆಜಗಾಮು
ಆಜಗ್ಮತುಃ
ಆಜಗ್ಮುಃ
ಆಜಗ್ಮುರ್ದೇವ-ಸೇವಾರ್ಥಂ
ಆಜನ್ಮ
ಆಜನ್ಮ-ಸಾಂತ-ಪರ್ಯಂತಂ
ಆಜಾನಜ
ಆಜಾನಜಾಃ
ಆಜ್ಞಾ-ಧರೋ
ಆಜ್ಞಾ-ಧಾರಕ-ನಾಗಿ
ಆಜ್ಞಾ-ನು-ಸಾರ-ವಾಗಿ
ಆಜ್ಞಾ-ಪ-ಯತ್ತದಾ
ಆಜ್ಞಾ-ಪಿತಳಾದ
ಆಜ್ಞಾ-ಪಿಸಲ್ಪಟ್ಟ
ಆಜ್ಞೆ
ಆಜ್ಞೆ-ಗಳೇ
ಆಜ್ಞೆ-ಯಂತೆ
ಆಜ್ಞೆ-ಯನ್ನು
ಆಜ್ಞೆ-ಯಿಂದ
ಆಜ್ಞೆ-ಯಿ-ರು-ವುದ-ರಿಂದ
ಆಟ-ಗಳಲ್ಲಿ
ಆಟವಾ-ಡಲು
ಆಡದೆ
ಆಡಿಸುತ್ತ
ಆಡು
ಆಡುತ್ತಾ
ಆಡುತ್ತಿದ್ದರು
ಆಢಕಂ
ಆಢ್ರಾಕ್ಯೋ
ಆತ
ಆತಂಕ-ಗೊಂಡು
ಆತನ
ಆತ-ನನ್ನು
ಆತ-ನಿಗೆ
ಆತನು
ಆತಿಥ್ಯೇನಾ-ಗತೇ
ಆತ್ಮನಿಗಲ್ಲ
ಆತ್ಮ-ನಿಗೆ
ಆತ್ಮನೋ
ಆತ್ಮಪ್ರಶಂಸನಾ-ಕ-ನಾದ
ಆತ್ಮ-ವಾಚ್ಯ-ವನೂದ್ಯೈವಂ
ಆತ್ಮಾ
ಆತ್ಮಾ-ರಾಮಃ
ಆದ
ಆದ-ಕಾರಣ
ಆದತ್ತ
ಆದದೇ
ಆದರ
ಆದರ-ದಿಂದ
ಆದರೂ
ಆದರೆ
ಆದಾಗ್ಯೂ
ಆದಾಯ
ಆದಾವಂತೇ
ಆದಿತ್ಯಮವ-ಲೋಕ-ಯೇತ್
ಆದಿ-ಯಲ್ಲಿ
ಆದಿ-ಯಲ್ಲಿಯ
ಆದೀತು
ಆದುದ-ರಿಂದ
ಆದ್ಯನ್ನಂ
ಆಧಾರದ-ಮೇಲೆ
ಆಧಾ-ರ-ನಾದ
ಆನಂದ
ಆನಂದ-ದಿಂದ
ಆನಂದ-ಬಾಷ್ಪ-ಗಳು
ಆನಂದ-ವನ್ನುಂಟು-ಮಾಡುತ್ತಿದ್ದುವು
ಆನುಸ್ಮಿಕ
ಆನೆ
ಆನೆ-ಗಳಿಂದ
ಆನೆ-ಗಳಿಗೆ
ಆನೆ-ಯಿಂದ
ಆನೆಯು
ಆಪಃ
ಆಪಿ
ಆಪ್ತ-ರಿಗಾಗಲೀ
ಆಭ-ರಣಾದಿ-ಗಳು
ಆಭಾಸ
ಆಮಂತ್ರ್ಯ
ಆಮೇಲೆ
ಆಯಸ್ಸು
ಆಯಾ
ಆಯಾ-ಸ-ವನ್ನು
ಆಯಾ-ಸ-ವಿಲ್ಲದ
ಆಯಾ-ಸ-ಹೊಂದಿ
ಆಯಿತು
ಆಯುಃ
ಆಯುಧ-ಗಳನ್ನು
ಆಯುಧ-ಗಳಿಂದ
ಆಯುಧ-ಪಾಣಿ-ಗ-ಳಾಗಿ
ಆಯುಷ್ಯ-ವನ್ನು
ಆಯುಸ್ಸನ್ನು
ಆಯುಸ್ಸನ್ನೂ
ಆಯುಸ್ಸು
ಆರಂಭಿಸಿ
ಆರಂಭಿಸಿ-ದರು
ಆರತಿ
ಆರತಿ-ಯನ್ನು
ಆರ-ನಾಲಂ
ಆರನೆಯವ-ನಾದ
ಆರನೆ-ಯ-ವನು
ಆರನೇ
ಆರಭೇತ್
ಆರಾಧಿಸ-ಬೇಕು
ಆರಾಧ್ಯ
ಆರು
ಆರೋಪಿಸಿ
ಆರ್
ಆರ್ಜ-ಕಾನಾಂ
ಆರ್ಜ-ಯಿತ್ವಾ
ಆರ್ಯೈಶ್ಚ
ಆಲದ
ಆಲದ-ಮರದ
ಆಲದ-ಮರ-ದಿಂದ
ಆಲಸ್ಯ-ದಿಂದಾಗಲೀ
ಆಲಸ್ಯ-ರ-ಹಿತ-ನಾಗಿ
ಆಲಸ್ಯ-ವನ್ನು
ಆಲಿ-ಸುವ
ಆಲೋ-ಚನೆ-ಯನ್ನು
ಆಲೋಚಿಸಿ
ಆಳುತ್ತಿದ್ದನು
ಆವಯೋ-ರಿದಮತ್ಯನ್ನಂ
ಆವಯೋಸ್ತೇನ
ಆವರಿ-ಸಿ-ಕೊಳ್ಳಲ್ಪಟ್ಟವು
ಆವಾಂ
ಆವಾಭ್ಯಾಂ
ಆವಾಹನೆ-ಮಾಡಿದ
ಆವಾ-ಹಿತ
ಆವಾಹಿಸಿ
ಆವಾಹ್ಯ
ಆವಿರ್ಭವಿ-ಸಿದ
ಆವಿರ್ಭವಿಸಿ-ದನು
ಆವೃತ್ಯಧಿಕ-ರಣ-ದಲ್ಲಿ
ಆಶೀರ್ಭಿರಭಿನಂದ್ಯ
ಆಶೀರ್ವದಿಸ
ಆಶೀರ್ವಾದ-ಮಾಡಿ
ಆಶೆ-ಯಿಂದ
ಆಶ್ಚರ್ಯಕ-ರ-ವಾದ
ಆಶ್ಚರ್ಯಚಕಿ-ತ-ನಾದ
ಆಶ್ಚರ್ಯಚಕಿತ-ರಾಗಿ
ಆಶ್ಚರ್ಯ-ದಿಂದ
ಆಶ್ಚರ್ಯ-ಪಟ್ಟ
ಆಶ್ಚರ್ಯ-ವಾಗಿದೆ
ಆಶ್ರಮಕ್ಕೆ
ಆಶ್ರಮ-ದಲ್ಲಿ
ಆಶ್ರಮ-ದಲ್ಲಿಯೇ
ಆಶ್ರಮ-ದ-ವ-ರಿಗೂ
ಆಶ್ರಮ-ದ-ವರು
ಆಶ್ರಮ-ದ-ವರೂ
ಆಶ್ರಮ-ವಾಸಿ-ಯಾದ
ಆಶ್ರಯ
ಆಶ್ರಯ-ದಲ್ಲಿ
ಆಶ್ರಯ-ದಲ್ಲಿದ್ದ
ಆಶ್ರಯ-ದಲ್ಲಿ-ರುತ್ತವೆ
ಆಶ್ರಯ-ದಲ್ಲಿ-ರು-ವುದು
ಆಶ್ರಯಳು
ಆಶ್ರಯ-ವನ್ನು
ಆಶ್ರಯ-ವಾ-ಯಿತು
ಆಶ್ರಯಿಸಿ-ಕೊಂಡಿದೆ
ಆಶ್ರಯಿಸಿ-ಕೊಂಡು
ಆಶ್ರಯಿ-ಸು-ವಂತೆ
ಆಶ್ಲಿ
ಆಷಾಢ-ದಲ್ಲಿ
ಆಷಾಢೇ
ಆಸಕ್ತ-ನಾಗಿ
ಆಸಕ್ತ-ನಾಗಿದ್ದನು
ಆಸಕ್ತ-ನಾಗಿದ್ದೆ
ಆಸಕ್ತ-ನಾಗಿ-ರು-ವುದು
ಆಸಕ್ತ-ನಾದ
ಆಸಕ್ತ-ರಾಗಿ
ಆಸಕ್ತ-ರಾಗಿ-ರುತ್ತಾರೆಯೋ
ಆಸಕ್ತ-ರಾಗಿ-ರುವ-ವರು
ಆಸಕ್ತರಾ-ದ-ವರು
ಆಸಕ್ತರಾ-ದ-ವರು-ವಿಷ್ಣು
ಆಸಕ್ತರೂ
ಆಸಕ್ತ-ವಾಗು-ವುದಿಲ್ಲ
ಆಸಕ್ತಿ
ಆಸಕ್ತಿ-ಯನ್ನು
ಆಸ-ನದ-ಮೇಲೆ
ಆಸ-ಮಾಪ್ತಿಂ
ಆಸೀತ
ಆಸೀದಿ-ಯಾನ್
ಆಸೀನ್ನ
ಆಸೆ-ಯಾಗಿದೆ
ಆಸೆ-ಯಿಂದ
ಆಸ್ತಿಕ-ಜನರ
ಆಸ್ತಿಕ-ರಿಗೆ
ಆಸ್ತಿಕ್ಕ-ಬುದ್ದಿಯು
ಆಸ್ತಿಕ್ಯ
ಆಸ್ತಿಕ್ಯ-ಬುದ್ದಿ
ಆಸ್ತಿಕ್ಯ-ಬುದ್ದಿ-ಯುಳ್ಳ
ಆಸ್ತಿಪಾಸ್ತಿ-ಗಳನ್ನೆಲ್ಲ
ಆಸ್ತಿ-ಯನ್ನೂ
ಆಸ್ಥಿ
ಆಸ್ಪದ-ವಾಗಿ
ಆಹ
ಆಹರಾಮ
ಆಹವ-ನೀಯ-ಯೆಂಬ
ಆಹಾರ
ಆಹಾರಕ್ಕಾಗಿ
ಆಹಾರ-ರಾಗಲಿ
ಆಹಾರ-ವನ್ನು
ಆಹಾರ-ವನ್ನೇ
ಆಹಾರವು
ಆಹಾ-ರಾದಿ-ಗಳನ್ನು
ಆಹಾರಾರ್ಥಂ
ಆಹಾರಾರ್ಥ-ವಾಗಿ
ಆಹಾರೊ
ಆಹ್ನಿ
ಆಹ್ವಾನಿಸಲ್ಪಟ್ಟ
ಇಂತಹ
ಇಂದು
ಇಂದೋ
ಇಂದ್ರ
ಇಂದ್ರ-ಕಾಮರು
ಇಂದ್ರನ
ಇಂದ್ರ-ನಿಂದ
ಇಂದ್ರ-ನಿಗೆ
ಇಂದ್ರನು
ಇಂದ್ರನೇ
ಇಂದ್ರರು
ಇಂದ್ರಾದ್ಯಾ
ಇಂದ್ರಿಯ
ಇಂದ್ರಿಯ-ಗಳನ್ನು
ಇಂದ್ರಿಯ-ಗಳಾದರೋ
ಇಂದ್ರಿಯ-ಗಳು
ಇಂದ್ರಿಯ-ನಿಗ್ರಹ
ಇಂದ್ರಿಯ-ನಿಗ್ರಹ-ದಿಂದ
ಇಂದ್ರಿಯ-ನಿಗ್ರಹ-ವನ್ನು
ಇಂದ್ರಿಯ-ನಿಗ್ರಹ-ವುಳ್ಳ
ಇಂದ್ರಿಯಾಣಿ
ಇಂದ್ರೀಯ-ನಿಗ್ರಹ
ಇಂಧನಾನೀವ
ಇಂಪಾದ
ಇಕ್ಷುಯಂತ್ರೇ
ಇಚ್ಚಾ
ಇಚ್ಚಿಸದೇ
ಇಚ್ಚಿ-ಸುತ್ತಾನೆಯೋ
ಇಚ್ಚಿ-ಸುತ್ತಾರೆ
ಇಚ್ಚಿ-ಸುವ
ಇಚ್ಚಿ-ಸುವ-ವರು
ಇಚ್ಚಿಸು-ವುದಿಲ್ಲ
ಇಚ್ಛಾ
ಇಚ್ಛಾ-ನು-ಸಾರ-ವಾಗಿ
ಇಚ್ಛಿ-ಸಿ-ದರೆ
ಇಚ್ಛಿ-ಸುತ್ತಾರೆ
ಇಚ್ಛಿಸುತ್ತೇನೆ
ಇಚ್ಛಿ-ಸುವ
ಇಚ್ಛೆ
ಇಚ್ಛೆ-ಎಂಬ
ಇಚ್ಛೆಗೆ
ಇಚ್ಛೆ-ಯಾದಾಗ
ಇಚ್ಛೆ-ಯಿಂದ
ಇಚ್ಛೆ-ಯಿಂದಲೂ
ಇಚ್ಛೆ-ಯಿ-ದೆಯೋ
ಇಚ್ಛೆಯು
ಇಚ್ಛೆ-ಯುಳ್ಳ
ಇಚ್ಛೆ-ಯುಳ್ಳ-ವನು
ಇಚ್ಛೆ-ಯುಳ್ಳ-ವರು
ಇಚ್ಛೆಯೇ
ಇಟ್ಟನು
ಇಟ್ಟಿರು
ಇಟ್ಟಿರುತ್ತೇನೆ
ಇಟ್ಟಿ-ರುವ
ಇಟ್ಟು
ಇಟ್ಟು-ಕೊಂಡು
ಇಟ್ಟು-ಕೊಳ್ಳ-ಬೇಕು
ಇಟ್ಟು-ಕೊಳ್ಳುವುದು
ಇಡ-ಬೇಕು
ಇಡ-ಲಿಲ್ಲವೇ
ಇಡಲು
ಇಡಲ್ಪಟ್ಟಿದೆ
ಇಡುತ್ತಿದ್ದೇನೆ
ಇಡು-ವ-ವನು
ಇತ
ಇತರ
ಇತರರ
ಇತರ-ರನ್ನು
ಇತರ-ರಲ್ಲಿ
ಇತರ-ರಿಂದ
ಇತರ-ರಿಂದಲೂ
ಇತರ-ರಿಗೂ
ಇತರ-ರಿಗೆ
ಇತರರು
ಇತರರೂ
ಇತಿ
ಇತಿ-ವಾಚಾಂ
ಇತಿ-ಹಾಸ
ಇತಿ-ಹಾಸ-ಗಳನ್ನು
ಇತಿ-ಹಾಸ-ಪುರಾ-ಣ-ಗಳನ್ನೂ
ಇತಿ-ಹಾಸ-ಮಿಮಂ
ಇತಿ-ಹಾಸ-ವನ್ನು
ಇತಿ-ಹಾಸವು
ಇತೀರಿತಾಸ್ತೇ
ಇತೀವ
ಇತೋ
ಇತ್ತಂ
ಇತ್ತು
ಇತ್ಥಂ
ಇತ್ಯ
ಇತ್ಯಥ
ಇತ್ಯಪಿ
ಇತ್ಯ-ಭಿಧೀ-ಯತೇ
ಇತ್ಯಹಂಕಾರ-ಮೂಢೇನ
ಇತ್ಯ-ಹಮ್
ಇತ್ಯಾಕರ್ಣೋದಿತಂ
ಇತ್ಯಾಜ್ಞ
ಇತ್ಯಾಜ್ಞಪ್ತಾ
ಇತ್ಯಾದಿ
ಇತ್ಯಾದಿ-ಗಳನ್ನು
ಇತ್ಯಾದಿ-ತಥ್ಯಮಿತ್ಯೇವ
ಇತ್ಯಾದಿಷ್ಟಾ
ಇತ್ಯಾದಿ-ಹೇತುರ್ಭಿ-ಲೋಕೇ
ಇತ್ಯಾದೀನಿ
ಇತ್ಯಾದ್ಯಾ
ಇತ್ಯಾಶ್ಚಾಸ್ಯ
ಇತ್ಯಾಶ್ವಾ
ಇತ್ಯಾಶ್ವಾಸ್ಯ
ಇತ್ಯಾಸಂ
ಇತ್ಯುಕ್ತಃ
ಇತ್ಯುಕ್ತಾ
ಇತ್ಯುಕ್ತಾಸ್ತೇ
ಇತ್ಯುಕ್ತೇ
ಇತ್ಯುಕ್ತೇನ
ಇತ್ಯುಕ್ತೋ
ಇತ್ಯುಕ್ತ್ವಾ
ಇತ್ಯುಕ್ತ್ವಾತೇ
ಇತ್ಯೂಚಿ-ವಾನ್ಸ
ಇತ್ಯೂಚಿ-ವಾಸಂ
ಇತ್ಯೇ-ತತ್
ಇತ್ಯೇ-ತತ್ಸರ್ವ-ಮಾಖ್ಯಾತಂ
ಇತ್ಯೇವಂ
ಇದಂ
ಇದಕ್ಕೂ
ಇದನ್ನು
ಇದರ
ಇದ-ರಂತೆ
ಇದ-ರಲ್ಲಿ
ಇದ-ರಿಂದ
ಇದ-ರೊಡನೆ
ಇದಲ್ಲದೆ
ಇದಾನೀಂ
ಇದು
ಇದು-ವರೆಗೂ
ಇದೆ
ಇದೆಯೋ
ಇದೆಲ್ಲ
ಇದೇ
ಇದ್ದ
ಇದ್ದದ್ದು
ಇದ್ದನು
ಇದ್ದರು
ಇದ್ದರೆ
ಇದ್ದ-ವನು
ಇದ್ದಾಗ್ಯೂ
ಇದ್ದಾರೆ
ಇದ್ದಿಲ್ಲ
ಇದ್ದು
ಇದ್ದೆ
ಇದ್ದೆವು
ಇದ್ದೇವೆ
ಇನ್ನಾರೂ
ಇನ್ನು
ಇನ್ನೂ
ಇನ್ನೇ-ನನ್ನು
ಇನ್ನೊಂದಕ್ಕೆ
ಇನ್ನೊಂದು
ಇನ್ನೊಬ್ಬ
ಇನ್ನೊಬ್ಬರು
ಇನ್ನೊಮ್ಮೆ
ಇಪತ್ತ-ನಾಲ್ಕು
ಇಪ್ಪತೈದು
ಇಪ್ಪತ್ತು-ನಾಲ್ಕು
ಇಪ್ಪತ್ತೇಳು
ಇಪ್ಪತ್ತೊಂದು
ಇಬ್ಬರ
ಇಬ್ಬ-ರಿಗೂ
ಇಬ್ಬರು
ಇಬ್ಬರೂ
ಇಮಂ
ಇಮಾ
ಇಮಾಂ
ಇಮಾಃ
ಇಮೇ
ಇರದಿದ್ದ
ಇರ-ಬಾರದು
ಇರ-ಬೇಕು
ಇರ-ಬೇಕೆಂದು
ಇರಲಿ
ಇರ-ಲಿಲ್ಲ
ಇರು
ಇರುತ್ತದೆ
ಇರುತ್ತ-ದೆಯೋ
ಇರುತ್ತವೆ
ಇರುತ್ತವೆಯೋ
ಇರುತ್ತಾ
ಇರುತ್ತಾನೆ
ಇರುತ್ತಾ-ನೆ-ಯೆಂಬ
ಇರುತ್ತಾರೆ
ಇರುತ್ತಿದ್ದನು
ಇರುತ್ತಿದ್ದೆ
ಇರುವ
ಇರು-ವಂತೆ
ಇರು-ವನು
ಇರು-ವ-ವ-ನನ್ನು
ಇರು-ವ-ವನಿ-ಗಿಂತಲೂ
ಇರು-ವ-ವನು
ಇರು-ವ-ವರು
ಇರು-ವ-ವರು-ಚಲಿ-ಸುವ-ವ-ರಲ್ಲ
ಇರು-ವಾಗ
ಇರು-ವಾಗಲೇ
ಇರು-ವುದ-ರಿಂದ
ಇರು-ವು-ದಾದರೆ
ಇರು-ವುದಿಲ್ಲ
ಇರು-ವುದು
ಇರು-ವುದೇ
ಇರು-ವೆಯಂಥ
ಇಲಿಯ
ಇಲ್ಲ
ಇಲ್ಲದ
ಇಲ್ಲ-ದಂತಾ-ಯಿತು
ಇಲ್ಲ-ದ-ವ-ರಿಗೆ
ಇಲ್ಲ-ದಿದ್ದರೂ
ಇಲ್ಲ-ದಿ-ರುವ-ವರ
ಇಲ್ಲದೆ
ಇಲ್ಲದೇ
ಇಲ್ಲ-ವಾಗುತ್ತದೆ
ಇಲ್ಲ-ವಾದ
ಇಲ್ಲ-ವಾದರೆ
ಇಲ್ಲವೆ
ಇಲ್ಲ-ವೆಂತಲೂ
ಇಲ್ಲ-ವೆಂಬ
ಇಲ್ಲವೇ
ಇಲ್ಲವೋ
ಇಲ್ಲಿ
ಇಲ್ಲಿಂದ
ಇಲ್ಲಿಗೆ
ಇಲ್ಲಿ-ರುವ
ಇಳಿದು
ಇಳಿ-ಯುವ
ಇವ
ಇವ-ನನ್ನು
ಇವ-ನಿಗೆ
ಇವನು
ಇವನೂ
ಇವರ
ಇವ-ರಿಂದ
ಇವರಿ-ಗಾಗಿ
ಇವ-ರಿಗೆ
ಇವರು
ಇವರೆಲ್ಲರೂ
ಇವಳು
ಇವಾ
ಇವಾ-ಪರಾಃ
ಇವಾ-ಪರೌ
ಇವು
ಇವುಂ
ಇವು-ಗಳ
ಇವು-ಗಳನ್ನು
ಇವು-ಗಳಲ್ಲಿ
ಇವು-ಗಳಲ್ಲಿಲ್ಲ
ಇವು-ಗಳಿಂದ
ಇವು-ಗಳು
ಇವೆ
ಇವೆಲ್ಲಕ್ಕಿಂತ
ಇವೆಲ್ಲ-ವನ್ನು
ಇವೆಲ್ಲವೂ
ಇವೇ
ಇಷ್ಟ
ಇಷ್ಟಾರ್ಥ-ಗಳನ್ನು
ಇಷ್ಟಾರ್ಥ-ಗಳನ್ನೂ
ಇಷ್ಟಾರ್ಥ-ಗಳೂ
ಇಷ್ಟಾರ್ಥ-ಸಿದ್ದಿ-ಯನ್ನು
ಇಷ್ಟು
ಇಷ್ಟೂ
ಇಹ
ಇಹಪರ-ಗಳಲ್ಲಿ
ಈ
ಈಗ
ಈಗಲೂ
ಈಗಲೇ
ಈಗಿನ
ಈಗಿ-ರುವ
ಈಗಿಲ್ಲ
ಈಡಾ-ಗಿ-ರುವನೋ
ಈಡಾಗಿ-ರುವ-ವರು
ಈಡಾ-ದನು
ಈಡೇರಿ-ಸುವನು
ಈತ-ನಲ್ಲಿ
ಈತನು
ಈದೃಶಂ
ಈರೀತಿ
ಈರೀತಿ-ಯಿಂದ
ಈವರೆವಿಗೂ
ಈಶಾನ್ಯ
ಈಶಾನ್ಯ-ದಲ್ಲಿ
ಈಶ್ವರಃ
ಈಶ್ವರ-ನಿಂದ
ಈಶ್ವರನೇ
ಉಂಟಾಗದೆ
ಉಂಟಾಗಿದೆ
ಉಂಟಾಗು-ವುದಿಲ್ಲ
ಉಂಟಾಗು-ವುದೇ
ಉಂಟಾದ
ಉಂಟಾದೀತು
ಉಂಟಾ-ಯಿತು
ಉಂಟು-ಮಾಡಿತು
ಉಂಟು-ಮಾಡುವ
ಉಂಟು-ಹೆಚ್ಚಿನ
ಉಗ್ರ
ಉಗ್ರ-ವಾದ
ಉಚಿತ-ವಾದ
ಉಚ್ಚರಿ-ಸದೆ
ಉಚ್ಚರಿಸಿ
ಉಚ್ಚರಿ-ಸು-ವುದಿಲ್ಲವೋ
ಉಚ್ಚಾವಚೇಷು
ಉಚ್ಛಿಷ್ಟ-ಭೋ-ಜನ
ಉಚ್ಛಿಷ್ಟ-ಭೋ-ಜನಃ
ಉಚ್ಛಿಷ್ಟ-ಭೋ-ಜನನ
ಉಚ್ಛಿಷ್ಟ-ಭೋ-ಜನ-ನೆಂಬ
ಉಚ್ಛಿಷ್ಟ-ಭೋ-ಜನಶ್ಚಾದ್ಯೋ
ಉಚ್ಛ್ರಿತಃ
ಉಚ್ಯತೇ
ಉಜ್ಜಿ
ಉಜ್ಜೀವ-ಯಾಮಿ
ಉತ್ಕಟ
ಉತ್ತಮ
ಉತ್ತಮ-ನಾದ
ಉತ್ತಮ-ಯೋನಿ-ಗಳಲ್ಲಿ
ಉತ್ತಮ-ವಾದ
ಉತ್ತಮ-ವಾ-ದುದು
ಉತ್ತರ
ಉತ್ತರಕ್ರಿಯಾ
ಉತ್ತರಕ್ರಿಯಾದಿ
ಉತ್ತರಕ್ರಿಯಾ-ದಿ-ಗಳನ್ನು
ಉತ್ತರ-ದಿಕ್ಕಿಗೆ
ಉತ್ತರೋತ್ತರ-ರಾಗಿ
ಉತ್ತಸ್ಥೌ
ಉತ್ತಾನಸಾದ
ಉತ್ಥಾಯ
ಉತ್ಪನ್ನ-ನಾಗಿ
ಉತ್ಪನ್ನ-ನಾಗಿ-ರುತ್ತೇನೆ
ಉತ್ಪನ್ನ-ನಾಗಿ-ರುವ-ವ-ರಿಗೂ
ಉತ್ಪನ್ನ-ನಾ-ಗುತ್ತಾರೆ
ಉತ್ಪನ್ನ-ನಾದ
ಉತ್ಪನ್ನ-ನಾದೆ
ಉತ್ಪನ್ನರಾ-ಗಲು
ಉತ್ಪನ್ನ-ರಾಗಿ
ಉತ್ಪನ್ನರಾ-ದ-ವರು
ಉತ್ಪನ್ನ-ವಾಗುತ್ತಿತ್ತು
ಉತ್ಪಾದ-ಕನು
ಉತ್ಪಾದ-ಯತಿ
ಉತ್ಪಾದಿ-ಸು-ವಂತೆ
ಉತ್ಸರ್ಜನ
ಉತ್ಸಾಹ-ದಿಂದ
ಉದಗ್ದಿಗಸ್ಥಾ
ಉದನ್ವ
ಉದಯ-ಕಾಲದ
ಉದಯ-ನಾದ
ಉದಯ-ವಾ-ಗಲು
ಉದಯ-ವಾಗುವ
ಉದರ
ಉದ-ಹರಿ-ಸಿದ್ದೇನೆ
ಉದ-ಹರಿ-ಸಿ-ರುವರು
ಉದಾನ
ಉದಾನಶ್ಚ
ಉದಾಹ-ರಣೆಗೆ
ಉದಾಹೃತಃ
ಉದಾಹೃತ-ವಾಗಿ-ರುವ
ಉದ್ದ-ವಾದ
ಉದ್ದ-ವಿತ್ತು
ಉದ್ದಿಶ್ಯ
ಉದ್ದಿಶ್ಯ-ವೇ-ನೆಂದರೆ
ಉದ್ದೇಶ-ದಿಂದ
ಉದ್ದೇಶಿಸಿ
ಉದ್ಧರ
ಉದ್ಧರಿ-ಸಲು
ಉದ್ಧರಿಸಿ
ಉದ್ಧರಿಸಿರಿ
ಉದ್ಧರಿಸು
ಉದ್ಧರಿ-ಸುವ
ಉದ್ಭವ-ವಾದ
ಉದ್ಯ
ಉದ್ಯುಕ್ತ-ನಾದನು
ಉದ್ಯೋಗ-ವನ್ನು
ಉದ್ವತ
ಉನ್ಮತ್ತ-ನಾಗಿ
ಉನ್ಮತ್ತ-ರಾದ
ಉನ್ಮಾರ್ಗ-ವರ್ತಿನಃ
ಉಪಕ-ರಣ-ಗಳನ್ನು
ಉಪ-ಕಾರ
ಉಪ-ಕಾರ-ವನ್ನು
ಉಪ-ಕಾರ-ವಾಗ-ಲಿಕ್ಕೋಸ್ಕರ-ತಾನೇ
ಉಪ-ಕಾರ-ವಾಗುತ್ತ-ದೆಯೋ
ಉಪ-ಕಾರಾಯ
ಉಪಕ್ರಮಿಸಿತು
ಉಪಚಾರ
ಉಪಚಾರ-ಗಳನ್ನು
ಉಪಚಾರ-ಗಳಿಂದ
ಉಪ-ಜೀವ-ನ-ವನ್ನು
ಉಪ-ತಾಯಿ-ಯರೂ
ಉಪ-ದಿಶ್ಯ
ಉಪ-ದೇಶ
ಉಪ-ದೇಶ-ಕೊಡು-ವರೋ
ಉಪ-ದೇಶ-ಬಲ-ದಿಂದ
ಉಪ-ದೇಶ-ಮಾಡಲು
ಉಪ-ದೇಶ-ಮಾಡಿದೆ
ಉಪ-ದೇಶ-ಮಾಡುವ
ಉಪ-ದೇಶ-ವನ್ನು
ಉಪ-ದೇಶ-ವಾಕ್ಯ-ಗಳ
ಉಪ-ದೇಶ-ವಿಲ್ಲದ
ಉಪ-ದೇಶಾಮೃತ-ವನ್ನು
ಉಪ-ದೇಶಿಸಲ್ಪಟ್ಟ
ಉಪ-ದೇಶಿಸಿ
ಉಪ-ದೇಶಿಸಿತು
ಉಪನ-ಯನ-ವಾ-ಯಿತು
ಉಪ-ಪತಿ-ಯನ್ನಾಗಿ
ಉಪ-ಪತಿಯು
ಉಪ-ಪತಿರ್ಜಾ-ಮಾತಾ
ಉಪಮರ್ದ
ಉಪಯೋಗಿಸದೇ
ಉಪಯೋಗಿಸ-ಬಾರದು
ಉಪಯೋಗಿಸಿ
ಉಪಯೋಗಿ-ಸುತ್ತಾನೆಯೋ
ಉಪರಾ-ಗ-ಸಹಸ್ರಾಣಿ
ಉಪ-ವಾಸ
ಉಪವಿಶ್ಯ
ಉಪ-ವಿಷ್ಟಂ
ಉಪಸ್ಥಾತ್ಮಾ
ಉಪಸ್ಥಾಭಿ-ಮಾನಿಯು
ಉಪಾಂಗ-ಗಳೂ
ಉಪಾಧಿ
ಉಪಾಧಿಯು
ಉಪಾಯ
ಉಪಾಯ-ಕೈಶ್ಚ
ಉಪಾಯ-ವಿಲ್ಲ
ಉಪಾ-ಯವು
ಉಪಾ-ಯವೂ
ಉಪಾಯೈರ್ಬಹುರ್ಭಿಜ್ಞಾತ್ವಾ
ಉಪಾಸಕ-ನಾಗಿ
ಉಪಾಸನೆ
ಉಪಾಸನೆ-ಮಾಡಿ
ಉಪೋದಕೀ
ಉಪ್ತಂ
ಉಭಯಲಿಂಗಾಧಿಕ-ರಣ-ದಲ್ಲಿ
ಉಭಯೇ
ಉಭಯೋಃ
ಉಭೇ
ಉರುಜೇಭ್ಯಶ್ಚ
ಉರುಜೇಭ್ಯಸ್ತತೋ
ಉರುಜೇಭ್ಯೋಪಿ
ಉರು-ದೇಶ-ಶಿರೋ-ಜಾತೋ
ಉರು-ದೇಶೇ
ಉರ್ಮಿಲ
ಉರ್ಮಿಲನ
ಉರ್ಮಿಲ-ನಾಕ್ಯಾಂತೇ
ಉಲ್ಕಾಪಾತ-ಗಳಾದವು
ಉಲ್ಕಾಪಾತಾ
ಉಲ್ಲಂಘನೆ
ಉಳಿದ
ಉಳಿದಿರುತ್ತಿದ್ದಿಲ್ಲ
ಉಳ್ಳ-ವ-ರಾಗಿ
ಉಳ್ಳ-ವು-ಗಳು
ಉವಾಚ
ಉಷಃ
ಉಷಃಕಾಲ-ದಲ್ಲಿ
ಉಷಸಿ
ಊಚುಃ
ಊಚೇ
ಊಟ
ಊಟಕ್ಕೆ
ಊಟ-ಮಾಡಿ
ಊಟ-ಮಾಡಿ-ದರೆ
ಊಟ-ಮಾಡುತ್ತಾನೆ
ಊಟ-ಮಾಡುತ್ತಾ-ನೆಯೋ
ಊಟ-ಮಾಡುತ್ತಿದ್ದೆ
ಊಟ-ಮಾಡುತ್ತಿ-ರ-ಲಿಲ್ಲ
ಊಟ-ಮಾಡುತ್ತೀಯೋ
ಊಟ-ಮಾಡು-ವರೋ
ಊಟ-ವನ್ನೇ
ಊರ
ಊರಹಂದಿ-ಯಾಗಿ
ಊರ-ಹೊರ-ಗಿ-ರುವ
ಊರ-ಹೊರಗೆ
ಊರಿಗೆ
ಊರಿತ್ತು
ಊರಿನ
ಊರಿ-ನಲ್ಲಿ
ಊರಿನಲ್ಲಿದ್ದ
ಊರಿನಲ್ಲಿದ್ದೆ
ಊರಿನಲ್ಲಿಯೇ
ಊರಿನ-ವನೇ
ಊರಿ-ನಿಂದ
ಊರು-ಗಳಲ್ಲಿ
ಊರ್ಧ್ವ-ಕೇಶಂ
ಊರ್ಧ್ವ-ಕೇಶಾ
ಊರ್ಧ್ವಪುಂಡ್ರ
ಊರ್ಧ್ವಪುಂಡ್ರ-ವಿಲ್ಲ-ದ-ವನ
ಊರ್ಧ್ವಪುಂಡ್ರ-ವಿ-ಹೀ-ನಸ್ಯ
ಊರ್ಧ್ವ-ಲೋಕ
ಊರ್ಧ್ವ-ಲೋಕ-ಗಳಿಗೆ
ಊರ್ಧ್ವ-ಲೋಕ-ವೆಂದರೆ
ಊರ್ಮಿ-ಗಳು
ಊರ್ಮಿಳಾ
ಊರ್ಮಿಳೆಯು
ಊಹಿಸಿ
ಋಕ್ಷಗೊತ್ರಸಮುತ್ಪನ್ನೋ
ಋಣ
ಋಣಂ
ಋಣ-ಗಳನ್ನು
ಋಣ-ಗಳಿಂದಲೂ
ಋಣತೋ
ಋಣತ್ರಯ-ದಿಂದ
ಋಣತ್ರಯ-ವಿಮೋಚ-ಕಮ್
ಋಣತ್ರಯಾತ್
ಋಣ-ದಿಂದ
ಋಣ-ನಿಷ್ಕೃತಿಃ
ಋಣನಿಷ್ಕ್ರ-ತಿಮ್
ಋಣ-ವನ್ನು
ಋತು-ಕಾಲೇ
ಋತು-ಗಳಲ್ಲಿ
ಋತು-ಗಳಲ್ಲಿಯೂ
ಋತು-ವಿ-ನಲ್ಲಿ
ಋತೇ
ಋದ್ಧಿ
ಋಷಯಃ
ಋಷಯಸ್ತಂ
ಋಷಿ
ಋಷಿಃ
ಋಷಿ-ಗಳ
ಋಷಿ-ಗಳನ್ನು
ಋಷಿ-ಗಳಲ್ಲಿ
ಋಷಿ-ಗಳಿಂದ
ಋಷಿ-ಗಳಿಗೂ
ಋಷಿ-ಗಳಿಗೆ
ಋಷಿ-ಗಳಿದ್ದರು
ಋಷಿ-ಗಳು
ಋಷಿ-ಗಳೇ
ಋಷಿ-ಪುತ್ರ-ನನ್ನು
ಋಷಿಯ
ಋಷಿ-ಯಜ್ಞ
ಋಷಿಯು
ಋಷಿಯೇ
ಋಷೀಣಾಂ
ಋಷ್ಯಾದ್ಯವಜ್ಞಯಾ
ಎ
ಎಂಟನೇ
ಎಂಟು
ಎಂತಹ
ಎಂದನು
ಎಂದರು
ಎಂದರ್ಥ
ಎಂದಿಗಾ-ದರೂ
ಎಂದಿಗೂ
ಎಂದಿತು
ಎಂದು
ಎಂದೂ
ಎಂದೆಂದಿಗೂ
ಎಂಬ
ಎಂಬು-ದನ್ನು
ಎಂಬು-ದಾಗಿ
ಎಂಬು-ದಾಗಿತ್ತು
ಎಂಬುದು
ಎಂಬುದೇ
ಎಂಬು-ವನು
ಎಚ್ಚರಿಸಿ
ಎಡ
ಎಡಗೈಗೆ
ಎಡ-ಭಾಗದ
ಎಡಮುಂಗೈ
ಎಣಿಸ-ಬಲ್ಲರೋ
ಎಣ್ಣೆ
ಎತ್ತನ್ನು
ಎತ್ತರ-ವಾಗಿ
ಎತ್ತರ-ವಾದ
ಎತ್ತಿ
ಎತ್ತಿ-ಕೊಂಡು
ಎತ್ತು-ಕುದುರೆ-ಆನೆ-ಮೊದಲಾದ
ಎದು-ರಿಗೆ
ಎದುರಿ-ನಲ್ಲಿ
ಎದುರುನೋಡುತ್ತ
ಎದೆ
ಎದೆಗೆ
ಎದ್ದು
ಎನ್ನಲು
ಎನ್ನುತ್ತಿದ್ದರು
ಎಬ್ಬಿಸಿ
ಎಮ್ಮೆ
ಎಮ್ಮೆ-ಗಳನ್ನು
ಎರಗಿ-ದನು
ಎರಡನೆ-ಯದು
ಎರಡನೆ-ಯ-ವನು
ಎರಡನೇ
ಎರಡರಷ್ಟು
ಎರಡು
ಎರಡೂ
ಎರಡೇ
ಎಲೆ-ಗಳಿಂದ
ಎಲೆ-ಗಳು
ಎಲೆಗೆ
ಎಲೆ-ಯಲ್ಲಿ
ಎಲ್ಲ
ಎಲ್ಲರ
ಎಲ್ಲ-ರನ್ನೂ
ಎಲ್ಲ-ರಲ್ಲಿಯೂ
ಎಲ್ಲ-ರಿಂದ
ಎಲ್ಲ-ರಿಗೂ
ಎಲ್ಲರೂ
ಎಲ್ಲ-ವನ್ನೂ
ಎಲ್ಲವೂ
ಎಲ್ಲಾ
ಎಲ್ಲಿ
ಎಲ್ಲಿಗೋ
ಎಲ್ಲಿದ್ದಾರೆ
ಎಲ್ಲಿ-ಯಾದರೂ
ಎಲ್ಲಿಯೂ
ಎಲ್ಲೆಲ್ಲಿಯೂ
ಎಳೆಯುತ್ತವೆ
ಎಳ್ಳಿನ
ಎಳ್ಳು
ಎಷ್ಟು
ಎಷ್ಟೆಂದು
ಎಷ್ಟೇ
ಎಸೆ-ದಂತೆ
ಏ
ಏಕ
ಏಕ-ಕಾಲಕ್ಕೆ
ಏಕಾ
ಏಕಾಕೀ
ಏಕಾ-ದಶ
ಏಕಾ-ದಶೀ
ಏಕಾ-ದಶೀವ್ರತ
ಏಕಾ-ದಶೀವ್ರತ-ಜ-ಪುಣ್ಯ-ಮ-ನಂತರಾಧಸಃ
ಏಕಾ-ದಶೀವ್ರತ-ವನ್ನು
ಏಕಾ-ದಶೀವ್ರತಸ್ಥಾ
ಏಕಾ-ದಶೋಧ್ಯಾಯಃ
ಏಕಾ-ದಶ್ಯಾಂ
ಏಕೀ
ಏಕೋ
ಏಕೋದ್ದಿಷ್ಟ-ವೆಂಬ
ಏಕೋದ್ದಿಷ್ಟೇ
ಏತಕ್ಕಾಗಿ
ಏತಚ್ಚಾಸ್ತ್ರಪ್ರ-ವಕ್ತಾರಂ
ಏತ-ತತ್
ಏತತ್
ಏತತ್ಕು
ಏತತ್ಪುಣ್ಯ
ಏತತ್ಸರ್ವಂ
ಏತ-ದಶ್ವತ್ಥ-ಮೂಲೇ
ಏತದ್ದಾನಂ
ಏತದ್ದೇಶ-ಪತೇಃ
ಏತದ್ದ್ರುಮೇ
ಏತಸ್ಮಾತ್
ಏತಸ್ಮಿನ್ನಂತರೇ
ಏತಸ್ಮಿನ್ನೇವ
ಏತಸ್ಯಾಃ
ಏತೇ
ಏತೇಷಾಂ
ಏತೇ-ಷಾ-ಮೇವ
ಏತೇಷ್ವನ್ಯ-ತಮಂ
ಏನನ್ನು
ಏನನ್ನೂ
ಏನಾದರೂ
ಏನಿದೆ
ಏನು
ಏನೂ
ಏನೆಂದರೆ
ಏನೆಂದು
ಏನೇ
ಏನೇನು
ಏನೋ
ಏರಿ-ಗಳನ್ನು
ಏರೀ
ಏಳನೆ-ಯ-ದಾಗಿ
ಏಳನೆ-ಯ-ವನು
ಏಳನೇ
ಏಳು
ಏಳು-ವರ್ಷ
ಏವ
ಏವಂ
ಏವಮಾದೀನಿ
ಏವೇಜ್ಯತೇ
ಏಷ
ಏಷಾಂ
ಏಷ್ಟಾ
ಏಹೀತ್ಯುವಾಚೈನಂ
ಐದನೆಯ-ವನ
ಐದನೆ-ಯ-ವನು
ಐದ-ರಿಂದ
ಐದು
ಐವತ್ತು
ಐಶ್ವರ್ಯವಂತ-ನಾಗಿದ್ದೆ
ಐಶ್ವರ್ಯ-ವಂತ-ರಾದ
ಐಶ್ವರ್ಯ-ವನ್ನು
ಐಶ್ವರ್ಯ-ವನ್ನೂ
ಐಶ್ವರ್ಯವೀರ್ಯ-ಗಳಿಂದ
ಐಶ್ವರ್ಯಾದಿ-ಗಳು
ಐಹಿಕ
ಐಹಿಕ-ವಿಷ-ಯ-ಸುಖ-ಗಳಲ್ಲಿ
ಒಂಟಿ-ಯಾಗಿ
ಒಂದಂಶದಷ್ಟೂ
ಒಂದಂಶವೂ
ಒಂದಕ್ಕಿಂತ
ಒಂದಕ್ಕೊಂದು
ಒಂದನ್ನು
ಒಂದು
ಒಂದು-ದಿನ
ಒಂದೆಡೆ
ಒಂದೆರಡು
ಒಂದೆರಡು-ಬಾರಿ
ಒಂದೇ
ಒಂಬತ್ತನೇ
ಒಗ್ಗಟ್ಟು
ಒಟ್ಟಿಗೆ
ಒಟ್ಟಿ-ನಲ್ಲಿ
ಒಡೆಯ-ನಾದೆ
ಒಣಗಿದ
ಒಣಗಿದ್ದುವು
ಒಣಗಿ-ಸುತ್ತಾನೆಯೋ
ಒದ್ದೆ
ಒದ್ದೆ-ಯಾದ
ಒಪ್ಪಿಸಿ
ಒಬ್ಬ
ಒಬ್ಬ-ನಾದರೂ
ಒಬ್ಬನೇ
ಒಬ್ಬರ
ಒಬ್ಬ-ರನ್ನೊಬ್ಬರು
ಒಬ್ಬರು
ಒಬ್ಬೊಬ್ಬರೂ
ಒಯ್ಯಲ್ಪಡ-ಬೇಕು
ಒರಳಿ-ನಲ್ಲಿ
ಒಳ-ಗೊಂಡ
ಒಳ-ಪಡುತ್ತಾನೆ
ಒಳಪಡುವ
ಒಳ್ಳೆ
ಒಳ್ಳೆಯ
ಓಂ
ಓಡಲು
ಓಡಾಡುತ್ತಿ-ರ-ಲಿಲ್ಲ
ಓಡಾಡುವ
ಓಡಿ
ಓಡಿ-ದನು
ಓಡಿ-ದರು
ಓಡಿ-ದವು
ಓಡುತ್ತದೆ
ಓಡುತ್ತಿ-ರಲು
ಓದಿ
ಓದುತ್ತಿದ್ದ-ವ-ರಿಂದ
ಓಮಿತ್ಯೊಚೇ
ಔದಾರ್ಯ
ಔದುಂಬರ
ಔದುಂಬರೇ
ಔಷಧ
ಔಷಧಂ
ಔಷಧ-ಕೊಡುವ
ಔಷಧದ
ಔಷಧ-ವನ್ನು
ಔಷ-ಧವು
ಔಷಧಾದಿ-ಗಳಿಂದ
ಔಷಧಿ-ಗಳನ್ನು
ಔಷಧೈರ್ಮಂತ್ರ-ವಶ್ಯಾದ್ಯೈಃ
ಔಷಧ್ಯೋತ್ಕೃಷ್ಟ-ಫಲಿತಾ
ಕ
ಕಂ
ಕಂಚಿಚ್ಛತಶಾಖಿನಮುಚ್ಛ್ರಿ-ತಮ್
ಕಂಚಿಜ್ಜ್ಯೋತಿಷಾರ್ಣವ-ಸಂಜ್ಞ-ಕಮ್
ಕಂಚಿತ್ತೃಷಯಾ
ಕಂಚಿದ್ಯುವಾನಂ
ಕಂಚಿನ
ಕಂಚೀ
ಕಂಜನಾಭಾಯ
ಕಂಜನೇನ
ಕಂಟಕಪಾದಪೈಃ
ಕಂಟಕೌ
ಕಂಠ-ದಂತೆ
ಕಂಠ-ದಲ್ಲಿ
ಕಂಠಮಾ-ವಯೋಃ
ಕಂಠ-ಸೂತ್ರ
ಕಂಠ-ಸೂತ್ರಸ್ಯ
ಕಂಠೇ
ಕಂಠೋಷ್ಠ
ಕಂಡ
ಕಂಡನು
ಕಂಡರು
ಕಂಡಿತು
ಕಂಡಿದೆ
ಕಂಡಿಲ್ಲ
ಕಂಡು
ಕಂದ-ರ-ದಲ್ಲಿ
ಕಂದುಕಾಕ್ರೀಡಾವ್ಯಾ-ಜಾತ್
ಕಂಬಲ-ದಾತಾರೋ
ಕಂಬಳಿ
ಕಂಬಳಿ-ಯನ್ನು
ಕಃ
ಕಕ್ಷಪಾಲ-ವನ್ನು
ಕಕ್ಷೇ
ಕಚಾಕಚಿ
ಕಚ್ಚಲ್ಪಟ್ಟ
ಕಚ್ಚಿತ್ತೇ
ಕಚ್ಚು-ವುದ-ರಿಂದ
ಕಟ-ದೇಶ-ಶಿರಾ
ಕಟ-ಮೂರ್ಧಾಸ್ತಥಾ-ಪರಾಃ
ಕಟ್ಟಿ
ಕಟ್ಟಿ-ಕೊಂಡು
ಕಟ್ಟಿ-ದರು
ಕಠಿಣ-ವೆಂದು
ಕಡಲೆ
ಕಡಲೇ
ಕಡಿಮೆ
ಕಡಿಮೆ-ಯಾಗಿ
ಕಡೆ
ಕಡೆಗೆ
ಕಡೆ-ಯಲ್ಲಿ
ಕಡೆ-ಯಲ್ಲಿಯೂ
ಕಣ್ಣಿಗೆ
ಕಣ್ಣೀರು
ಕಣ್ಣು
ಕಣ್ಣು-ಗಳನ್ನು
ಕಣ್ಣು-ಗಳಲ್ಲಿ
ಕಣ್ಣು-ಗಳು
ಕಣ್ಮರೆ-ಯಾದ
ಕತಿಪಯೈಃ
ಕತಿಪಯೈರ್ದಿನೈಃ
ಕತೇನೋದ-ಕೇನೈವ
ಕತ್ತರಿಸಿ
ಕತ್ತರಿಸಿ-ದನು
ಕತ್ತರಿ-ಸು-ವುದ-ರಿಂದ
ಕತ್ತಲೆಯ
ಕತ್ತಲೆ-ಯನ್ನು
ಕತ್ತಲೆ-ಯಿಂದ
ಕತ್ತಲೆಯು
ಕತ್ತಿ-ಗಳನ್ನು
ಕತ್ತಿ-ಯಿಂದ
ಕತ್ತೆ
ಕತ್ರಿ-ಯರಿಗ
ಕತ್ರೃತ್ವ-ವರ್ಜಮ್
ಕಥಂ
ಕಥಂಚನ
ಕಥಮ-ಭೂತ್
ಕಥಮೇ-ತಚ್ಚ
ಕಥಮ್
ಕಥಾ
ಕಥಾಂ
ಕಥಾ-ನಿಮಿತ್ತಂ
ಕಥಾ-ಮಿ-ಮಾಮ್
ಕಥಾ-ಮೃತ-ವನ್ನು
ಕಥಾಮ್
ಕಥಾ-ಲಾ-ಪನೆ-ಯಲ್ಲಿ
ಕಥಾ-ಲಾಪೈಃ
ಕಥಾಶ್ರವಣ
ಕಥಾಶ್ರವಣ-ಯುಕ್ತ-ರಾದ
ಕಥಾಶ್ರವಣ-ವನ್ನು
ಕಥಾಶ್ರವಣವು
ಕಥಾ-ಸಕ್ತ
ಕಥಾ-ಸಕ್ತಾ
ಕಥಾಸ್ತಥಾ
ಕಥಿತಂ
ಕಥಿತಾ
ಕಥಿತೋ
ಕಥೆ
ಕಥೆ-ಗಳನ್ನು
ಕಥೆ-ಗಳೂ
ಕಥೆ-ಯನ್ನು
ಕಥೆಯು
ಕಥೆ-ಯೊಂದಿದೆ
ಕಥ್ಯತೇ
ಕದಂಬ
ಕದಂಬಂ
ಕದಂಬಗಾಃ
ಕದಂಬ-ವೃಕ್ಷ-ವೊಂದನ್ನು
ಕದಂಬೇ
ಕದಂಬೇಸ್ಮಿನ್
ಕದಡಿದ
ಕದಲೀಸ್ತಂಭೈರ್ವಿದಧಾತಿ
ಕದಾ-ಚನ
ಕದಾ-ಚಿತ್
ಕದಾ-ಚಿತ್ಸಾನುಗೋ
ಕದಾಚಿದಟವೀ-ಮಾಪ-ತುರ್ವಿಷ-ಯೇಚ್ಛಯಾ
ಕದಾಚಿದ-ಟಿವೀಂ
ಕದಾಚಿದಾ-ಗತೋ
ಕದಾಚಿದ್ಗುರವೇ
ಕದಿಯು-ವುದು
ಕದ್ದ
ಕದ್ದು-ಕೊಂಡು
ಕನೀ-ಯಾನ್
ಕನೈ-ಯನ್ನು
ಕನೈಯೂ
ಕನ್ನಡ
ಕನ್ನಡಕ್ಕೆ
ಕನ್ನಡ-ದಲ್ಲಿ
ಕನ್ನಡಿಯ
ಕನ್ನಡಿ-ಯನ್ನು
ಕನ್ನಡಿಯು
ಕನ್ನಡಿ-ಯೊಳ-ಗಿನ
ಕನ್ಯಾ
ಕನ್ಯಾ-ಮಾ-ದಾಯ
ಕಪಟಂ
ಕಪಟ-ವೇಷಧಾರಿಯು
ಕಪಿ-ಗಳೆಲ್ಲಿ
ಕಪಿತ್ಥ-ವತ್
ಕಪಿಲಾ
ಕಬಂಧಾಶ್ಚ
ಕಬ್ಬಿ-ಣದ
ಕಬ್ಬಿನಗಾಣ-ದಲ್ಲಿ
ಕಮಂಡಲುಃ
ಕಮಂಡಲು-ಜಲಂ
ಕಮಂಡಲು-ವನ್ನು
ಕಮಲ
ಕಮಲ-ಕಂದ
ಕಮಲ-ಗಳಿಂದ
ಕಮ-ಲದ
ಕಮಲ-ದಂತೆ
ಕಮಲ-ಪುಷ್ಪ
ಕಮಲ-ಪುಷ್ಪ-ಗಳಿಂದ
ಕಮಲಾ
ಕಮಲಾಸನನೇ
ಕಮಲಾಸನ-ಲೋಕಂ
ಕಮಲೇಕ್ಷ-ಣಮ್
ಕಮಲೈಃ
ಕಮಲೋತ್ಫುಲ್ಲಮಂಡಿ-ತಮ್
ಕಮಲೋದ್ಭವ
ಕಮಲೋದ್ಭ-ವಮ್
ಕಮ್
ಕರಗಿದ
ಕರಣ-ಶುದ್ಧಿಂ
ಕರ-ಣಾತ್
ಕರಯೊರ್ಭವೇತ್
ಕರಯೋಸ್ತ್ರಯಃ
ಕರಸಂಸ್ಪರ್ಶಂ
ಕರಾ
ಕರಾ-ಳ-ರೂಪದ
ಕರಿಷ್ಯಾಮಿ
ಕರಿಷ್ಯಾಮೀತೈವಂ
ಕರೀರ-ಜನ್ಮನೇ
ಕರುಣಾಂ
ಕರು-ಣಾಕರ
ಕರು-ಣಾ-ಬಲಾತ್
ಕರು-ಣಿಸಿ-ದರು
ಕರು-ಣೆ-ಯನ್ನು
ಕರು-ಣೆ-ಯಿಂದ
ಕರು-ಣೆಯಿದ್ದರೆ
ಕರು-ಣೆ-ಯೆಂಬ
ಕರು-ವಿ-ನಿಂದ
ಕರು-ಸ-ಹಿತ-ವಾದ
ಕರೆ-ತಂದನು
ಕರೆದು
ಕರೆದು-ಕೊಂಡು
ಕರೆದು-ಕೊಂಡು-ಹೋ-ದರು
ಕರೆದೊಯ್ಯಲು
ಕರೆ-ಯಲು
ಕರೆಯಲ್ಪಡುತ್ತಾನೆ
ಕರೆಯಲ್ಪಡುತ್ತಾರೆ
ಕರೆಯುತ್ತಿದ್ದೀಯಲ್ಲ
ಕರೇ
ಕರೋತಿ
ಕರೋತ್ಯದ್ಧಾ
ಕರೋತ್ಯಭ್ಯಂಜನಂ
ಕರೋಮಿ
ಕರೋಮ್ಯ-ಹಮ್
ಕರ್ಕಟೇ
ಕರ್ಕಶೋ
ಕರ್ಕಾಟಕ
ಕರ್ಣೌ
ಕರ್ತ
ಕರ್ತ-ನೆಂದು
ಕರ್ತ-ನೆಂಬ
ಕರ್ತರಿ
ಕರ್ತ-ವಾಗಿ
ಕರ್ತವ್ಯ
ಕರ್ತವ್ಯಂ
ಕರ್ತವ್ಯದ
ಕರ್ತವ್ಯ-ಮೇವ
ಕರ್ತವ್ಯಾ
ಕರ್ತವ್ಯೋ
ಕರ್ತಾ
ಕರ್ತಾ-ರಶ್ಚೋತ್ತರೋತ್ತ-ರಮ್
ಕರ್ತಾ-ಹ-ಮಿತಿ
ಕರ್ತುಂ
ಕರ್ತುಮರ್ಹಸಿ
ಕರ್ತುಮಿಚ್ಛತಿ
ಕರ್ತುಮಿಚ್ಛೋಃ
ಕರ್ತುವಿಹಾರ್ಹಸಿ
ಕರ್ತೃ-ಗಳು
ಕರ್ತೃ-ಗಳೇ
ಕರ್ತೃತಾ
ಕರ್ತೃತ್ಯವೇ
ಕರ್ತೃತ್ವ
ಕರ್ತೃತ್ವಂ
ಕರ್ತೃತ್ವವು
ಕರ್ತೃತ್ವಾಧಿಕ-ರಣ-ದಲ್ಲಿ
ಕರ್ತೃತ್ವಾಧಿಕ-ರಣ-ದಲ್ಲಿನ
ಕರ್ತೃ-ವಲ್ಲ
ಕರ್ನ-ಲೋಪೋ
ಕರ್ಪೂರ-ದಿಂದ
ಕರ್ಪೂರಸ್ಯಾರ್ತಿಕಂ
ಕರ್ಮ
ಕರ್ಮ-ಕರ-ಣಾತ್ಸಂಗತಿಃ
ಕರ್ಮ-ಕರ್ತು-ಮಿಹಾರ್ಹಸಿ
ಕರ್ಮಕ್ಕಿಂತ
ಕರ್ಮಕ್ಕೆ
ಕರ್ಮ-ಗಳ
ಕರ್ಮ-ಗಳನ್ನಾ-ಚರಿಸಿ
ಕರ್ಮ-ಗಳನ್ನು
ಕರ್ಮ-ಗಳನ್ನೂ
ಕರ್ಮ-ಗಳನ್ನೇ
ಕರ್ಮ-ಗಳಲ್ಲಿ
ಕರ್ಮ-ಗಳಿಂದ
ಕರ್ಮ-ಗಳಿಗೆ
ಕರ್ಮ-ಗಳು
ಕರ್ಮ-ಗಳೂ
ಕರ್ಮಗ್ರಂಥಿ-ನಿಕೃಂತ-ನಮ್
ಕರ್ಮಗ್ರಂಥಿ-ವಿಮೋಚ-ನಮ್
ಕರ್ಮ-ಚಿತಾಸ್ತತ್ಕರ್ಮ
ಕರ್ಮಜ
ಕರ್ಮ-ಜ-ದೇವ-ತೆ-ಗಳು
ಕರ್ಮಜಾಃ
ಕರ್ಮ-ಜಾ-ತಾನಾಂ
ಕರ್ಮ-ಜಾಶ್ಚ
ಕರ್ಮಣಃ
ಕರ್ಮಣಾ
ಕರ್ಮಣಾಂ
ಕರ್ಮ-ಣಾ-ನೇನ
ಕರ್ಮ-ಣಾ-ರಾಧ್ಯಮಾ-ನೇನ
ಕರ್ಮಣಿ
ಕರ್ಮ-ಣೈವ
ಕರ್ಮಣೋ
ಕರ್ಮ-ಣೋ-ಽಹ್ಯಪಿ
ಕರ್ಮಣ್ಯ
ಕರ್ಮಣ್ಯ-ಧಿ-ಕೃತಾ
ಕರ್ಮಣ್ಯೇವಾಧಿ-ಕಾ-ರಸ್ತೇ
ಕರ್ಮದ
ಕರ್ಮ-ದಲ್ಲಿ
ಕರ್ಮ-ದ-ಶೋದೃಶೀ
ಕರ್ಮ-ದಿಂದ
ಕರ್ಮ-ದಿಂದಲೂ
ಕರ್ಮ-ನಿರತಾ
ಕರ್ಮ-ಪಾಶ-ದಿಂದ
ಕರ್ಮ-ಪಾಶಾದ್ವಿ-ಮುಚ್ಯತೇ
ಕರ್ಮ-ಪಾಶೈರ್ವಿ-ಮುಚ್ಯತೇ
ಕರ್ಮ-ಫಲಂ
ಕರ್ಮ-ಫ-ಲದ
ಕರ್ಮ-ಫಲ-ಮಾಪ್ನೋತಿ
ಕರ್ಮ-ಫಲ-ವನ್ನು
ಕರ್ಮ-ಫಲವು
ಕರ್ಮ-ಫಲ-ಹೇತುರ್ಭೂರ್ಮಾ
ಕರ್ಮ-ಫ-ಲೇನ
ಕರ್ಮ-ಬಂಧ
ಕರ್ಮ-ಬಂಧಃ
ಕರ್ಮ-ಬಂಧ-ನ-ದಿಂದ
ಕರ್ಮ-ಬಂಧ-ನವು
ಕರ್ಮ-ಬಂಧ-ಮುಕ್ತಃ
ಕರ್ಮ-ಬಂಧಾನುಮೊಚ
ಕರ್ಮ-ಬಂಧಾನ್ಮುಮೋಚ
ಕರ್ಮ-ಬಂಧೋ
ಕರ್ಮಭಿಃ
ಕರ್ಮ-ಮಾಡ-ಬೇಡ
ಕರ್ಮ-ಮಾಡಲು
ಕರ್ಮ-ಮಾಡುತ್ತಿದ್ದೇನೆ
ಕರ್ಮ-ಮಾಡುವ
ಕರ್ಮ-ಮಾಡು-ವುದ-ರಲ್ಲಿಯೇ
ಕರ್ಮ-ಮಾರ್ಗ-ಗಳ
ಕರ್ಮ-ಮೂಲಾನಿ
ಕರ್ಮ-ರಾಶಿಯು
ಕರ್ಮ-ರಾಶಿಯೆಲ್ಲ
ಕರ್ಮ-ಲಭ್ಯಾಃ
ಕರ್ಮ-ಲೇಶ-ಮಾತ್ರಂ
ಕರ್ಮ-ಲೋಪಂ
ಕರ್ಮ-ಲೋಪೋ
ಕರ್ಮ-ವನ್ನು
ಕರ್ಮ-ವನ್ನೆಲ್ಲ
ಕರ್ಮ-ವನ್ನೇ
ಕರ್ಮ-ವಶ-ದಿಂದ
ಕರ್ಮ-ವಿಪಾ-ಕೇನ
ಕರ್ಮ-ವಿ-ಮುಖಾ
ಕರ್ಮ-ವಿ-ವರ-ಗಳನ್ನು
ಕರ್ಮವು
ಕರ್ಮ-ವೆಲ್ಲವೂ
ಕರ್ಮವೇ
ಕರ್ಮ-ಸಿದ್ದಿಯಾಗುತ್ತದೆ
ಕರ್ಮ-ಸಿದ್ಧಿ-ಮವಾಪ್ನೋತಿ
ಕರ್ಮ-ಹೀನರೂ
ಕರ್ಮಾಚ-ರಣೆ-ಯಲ್ಲಿ
ಕರ್ಮಾಣಿ
ಕರ್ಮಾಣ್ಯಥ
ಕರ್ಮಾಣ್ಯನ್ಯಾನ್ಯನೇಕಶಃ
ಕರ್ಮಾಣ್ಯಯಂ
ಕರ್ಮಾದಿ
ಕರ್ಮಾನುಗಂ
ಕರ್ಮಾನುಷ್ಟಾನವೇ
ಕರ್ಮಾನು-ಸಾರ-ವಾಗಿ
ಕರ್ಮಾರಂಭ-ದಲ್ಲಿಯ
ಕರ್ಮೇಂದ್ರಿಯ-ಗಳು
ಕರ್ಮೇಂದ್ರಿಯ-ಗಳುಈ
ಕರ್ವಾಣಿ
ಕಲತ್ರಾ-ಧೀನ-ಚೇತಸಾ
ಕಲಬೆರಕೆ-ಯಲ್ಲಿ
ಕಲವಿಂಕಾದೀನ್
ಕಲಶೋದಕ-ವನ್ನು
ಕಲಾ
ಕಲಾಂ
ಕಲಾಃ
ಕಲಾ-ಪ-ಗಳಲ್ಲೊಂದಾದ
ಕಲಾ-ವಿದ್ಯಾಸು
ಕಲಾ-ವೇಕಾ-ದಶೀ
ಕಲಿಂಗ
ಕಲಿಂಗ-ದೇಶಕ್ಕೆ
ಕಲಿಂಗ-ದೇಶ-ತನಯೇ
ಕಲಿಂಗ-ದೇಶದ
ಕಲಿಂಗ-ದೇಶೇ
ಕಲಿಂಗಾನಾಂ
ಕಲಿಂಗಾನಿ
ಕಲಿತರೆ
ಕಲಿತಿದ್ದೆ
ಕಲಿ-ಯದೇ
ಕಲಿಯ-ಲಿಲ್ಲ
ಕಲಿಯುಗ-ಗಳಲ್ಲಿ
ಕಲಿ-ಯುಗ-ದಲ್ಲಿ
ಕಲಿಸಿ
ಕಲಿ-ಸುವ-ವರು
ಕಲೆ-ಗಳ
ಕಲೆ-ಗಳನ್ನು
ಕಲೆ-ಗಳಿಗೆ
ಕಲೌ
ಕಲ್ಪ
ಕಲ್ಪ-ಕೋಟಿ-ಶತೈರಪಿ
ಕಲ್ಪ-ಗಳಲ್ಲಿ
ಕಲ್ಪ-ಗಳಲ್ಲಿಯೂ
ಕಲ್ಪ-ಗಳ-ವರೆಗೂ
ಕಲ್ಪ-ಗಳಾದರೂ
ಕಲ್ಪತೇ
ಕಲ್ಪ-ದ-ವರೆಗೂ
ಕಲ್ಪ-ಮಶ್ನುತೇ
ಕಲ್ಪಮ್
ಕಲ್ಪ-ಯಾ-ಮಾಸುಃ
ಕಲ್ಪ-ಯಿತ್ವಾ
ಕಲ್ಪ-ಸಾಹಸ್ರಂ
ಕಲ್ಯಾಣಮಿಚ್ಛಸಿ
ಕಲ್ಲಂಗಡಿ
ಕಲ್ಹಾ
ಕಲ್ಹಾರ
ಕಳಿಸಿ
ಕಳಿಸಿ-ದಳು
ಕಳೆದ
ಕಳೆದನು
ಕಳೆದರು
ಕಳೆದರೆ
ಕಳೆದಿದ್ದಾನೆ
ಕಳೆದಿದ್ದೇನೆಯೋ
ಕಳೆದು
ಕಳೆದು-ಕೊಂಡು
ಕಳೆದು-ಕೊಂಡುವು
ಕಳೆದು-ಕೊಳ್ಳ-ಬೇಕು
ಕಳೆದು-ಕೊಳ್ಳುತ್ತವೆಯೋ
ಕಳೆದು-ಕೊಳ್ಳುತ್ತಾನೆ
ಕಳೆ-ದುವು
ಕಳೆದೆ
ಕಳೆಯುತ್ತದೆ
ಕಳೆಯುತ್ತಾನೆಯೋ
ಕಳೆಯುತ್ತಿದ್ದೆ
ಕಳೆ-ಯುವ
ಕಳೆ-ಯುವ-ವನು
ಕಳ್ಳ-ತನ
ಕಳ್ಳ-ತನ-ದಿಂದ
ಕಳ್ಳ-ರನ್ನು
ಕಳ್ಳ-ರಿಂದ
ಕಳ್ಳರು
ಕವಚ-ದಂತೆ
ಕವರ್
ಕವಿಂಶತಿ-ಕುಲೈಃ
ಕಶ್ಚಿತ್
ಕಶ್ಚಿದ-ಸಮುದ್ರಗಾಃ
ಕಶ್ಚಿದ್ದ್ವಿ
ಕಶ್ಚಿದ್ವಿಪ್ರೋ-ಽಭೂನ್ಮೇಘ-ನಾಮಕಃ
ಕಶ್ಚೇದಿ-ರಾಡಯಮ್
ಕಷ್ಟ
ಕಷ್ಟ-ದಲ್ಲಿದ್ದೇನೆ
ಕಷ್ಟ-ದಿಂದ
ಕಷ್ಟವೇ
ಕಷ್ಟ-ಸಾಧ್ಯಾ
ಕಷ್ಟಾನ್ನಂ
ಕಷ್ಮಲ-ಗಳನ್ನು
ಕಸಿದು-ಕೊಂಡೆವು
ಕಸಿದು-ಕೊಳ್ಳುವ-ವರೇ
ಕಸ್ಮಾಚ್ಚರಸಿ
ಕಸ್ಮಾ-ದಯಂ
ಕಸ್ಮಾದಹೈತುಕೀ
ಕಸ್ಮಾದಿ-ದಾನೀಂ
ಕಸ್ಮಾದೇ-ತದ್ವಿಸ್ತಾರ್ಯ
ಕಸ್ಮಾದ್ದುಃ
ಕಸ್ಮಾದ್ವಾ
ಕಸ್ಮಿಂಶ್ಚಿತ್
ಕಸ್ಮಿಂಶ್ಚಿದ್ಗಿರಿ-ಕಂದರೇ
ಕಸ್ಮಿನ್
ಕಸ್ಯ
ಕಸ್ಯಾಪಿ
ಕಸ್ಯಾಸೌ
ಕಹೇತುಮ್
ಕಾ
ಕಾಂಕ್ಷಯಾ
ಕಾಂಕ್ಷಸೇಽನುಜಾಮ್
ಕಾಂಕ್ಷಿಣಾ
ಕಾಂಚೀ
ಕಾಂತಾರೇ
ಕಾಂತಿಮತೀ
ಕಾಂತಿಮತೀ-ಯೆಂಬುವಳನ್ನು
ಕಾಂಸ್ಯ-ಭುಗ್ಭವೇನ್ಮಾಘೇ
ಕಾಂಸ್ಯೇ
ಕಾಕಪಕ್ಷಾತ್ಮಜಾ
ಕಾಗೆಯಾಗುತ್ತಾನೆ
ಕಾಡಿಗೆ
ಕಾಡಿ-ನಲ್ಲಿ
ಕಾಡು
ಕಾಡು-ಜನ-ರಿಂದ
ಕಾಣದೆ
ಕಾಣಬಹದು
ಕಾಣ-ಬಹುದು
ಕಾಣಲಾರಂಭಿಸಿವೆ
ಕಾಣ-ಲಿಲ್ಲ
ಕಾಣಿಕೆ
ಕಾಣಿಸದಂತಾ-ಯಿತು
ಕಾಣಿಸದಿದ್ದ
ಕಾಣಿ-ಸ-ಲಿಲ್ಲ
ಕಾಣಿಸಿ-ಕೊಂಡವು
ಕಾಣಿ-ಸುತ್ತಿದ್ದ
ಕಾಣಿ-ಸುವ
ಕಾಣುತ್ತದೆ
ಕಾಣುತ್ತವೆ
ಕಾಣುತ್ತಾನೆ
ಕಾಣುತ್ತಾಳೆ
ಕಾಣುತ್ತಿದ್ದೆ
ಕಾಣುವ
ಕಾತಿಕ-ಮಾಸದ
ಕಾತುರ-ನಾಗಿದ್ದನು
ಕಾದಾ-ಡಲು
ಕಾದಿ-ದೆಯೋ
ಕಾದಿ-ರುವ
ಕಾನನ-ನೆಂಬ
ಕಾನಾಂ
ಕಾನಿ
ಕಾಪಿ-ಗೋತ್ರದ
ಕಾಪೇಯ-ಗೋತ್ರಜಃ
ಕಾಪೇಯ-ಗೋತ್ರ-ದಲ್ಲಿ
ಕಾಪೇಯೋ
ಕಾಮ
ಕಾಮಂ
ಕಾಮಃ
ಕಾಮಕ್ರೋಧಾದಿ
ಕಾಮ-ನಾಲೋಚ-ನದ್ವಯಮ್
ಕಾಮ-ಸರೋ-ವರೇ
ಕಾಮಾಃ
ಕಾಮಾದಿ
ಕಾಮಾದ್ಯ
ಕಾಮಾ-ನವಾಪ್ನು-ಯಾತ್
ಕಾಮಾನ್
ಕಾಮಾ-ಪೇಕ್ಷೆ
ಕಾಮೀ
ಕಾಮುಕಂ
ಕಾಮುಕ-ನಾಗಿದ್ದರೆ
ಕಾಮುಕಶ್ಚೇತ್ಸ-ಮಾಹ್ವಯ
ಕಾಮುಕಾಃ
ಕಾಮುಕಾನಭಿಕಾಂಕ್ಷತಿ
ಕಾಮುಕಾ-ರತಃ
ಕಾಮೇಂದ್ರಾ-ವರ್ಥ-ರೂಪಿಣೌ
ಕಾಮ್ಯ
ಕಾಮ್ಯಂ
ಕಾಮ್ಯ-ಮಿತಿ
ಕಾಮ್ಯೇನ
ಕಾಯಿ
ಕಾರ
ಕಾರಣ
ಕಾರಣಂ
ಕಾರ-ಣಕಂ
ಕಾರ-ಣ-ಗಳಿಂದ
ಕಾರ-ಣ-ಗಳಿಗೆ
ಕಾರ-ಣ-ದಿಂದ
ಕಾರ-ಣ-ದಿಂದಲೂ
ಕಾರ-ಣ-ದಿಂದಲೇ
ಕಾರ-ಣಮ್
ಕಾರ-ಣ-ವನ್ನು
ಕಾರ-ಣ-ವಮ್
ಕಾರ-ಣ-ವಾಗಿ
ಕಾರ-ಣ-ವಾಗಿ-ರು-ವುದ-ರಿಂದ
ಕಾರ-ಣ-ವಾಗುತ್ತದೆ
ಕಾರ-ಣ-ವಾಗುವ
ಕಾರ-ಣ-ವಾದ
ಕಾರ-ಣ-ವಿಲ್ಲ
ಕಾರ-ಣ-ವಿಲ್ಲದೆ
ಕಾರ-ಣ-ವಿಲ್ಲ-ವಷ್ಟೆ
ಕಾರ-ಣವೂ
ಕಾರ-ಣ-ವೆಂದು
ಕಾರ-ಣ-ವೇನು
ಕಾರ-ಣ-ವೇ-ನೆಂದರೆ
ಕಾರ-ಣವೋ
ಕಾರ-ಣಾಂತರ-ದಿಂದ
ಕಾರ-ಣಾಂತರಾತ್
ಕಾರ-ಣಾನಿ
ಕಾರ-ಯಂತ್ಯ
ಕಾರ-ಯತೇ
ಕಾರ-ಯತ್ಯದ್ಧಾ
ಕಾರ-ಯಾಮಿ
ಕಾರ-ಯಿತಾ
ಕಾರ-ಯಿತ್ವಾ
ಕಾರ-ಯೇತ್ಪೂಜಾಂ
ಕಾರ್ತವೀರ್ಯಾರ್ಜುನ-ನಿಗೂ
ಕಾರ್ತವೀರ್ಯಾರ್ಜುನ-ನೆಂಬ
ಕಾರ್ತವೀರ್ಯಾರ್ಜುನ-ನೊಡನೆ
ಕಾರ್ತವೀರ್ಯಾರ್ಜುನೋ
ಕಾರ್ತಿಕ
ಕಾರ್ತಿಕಃ
ಕಾರ್ತಿಕಸ್ನಾ-ನವು
ಕಾರ್ತಿಕ್ಯಾಂ
ಕಾರ್ಪಾಸಸಂಯು-ತಮ್
ಕಾರ್ಯ
ಕಾರ್ಯಂ
ಕಾರ್ಯ-ಗಳ
ಕಾರ್ಯ-ಗಳನ್ನು
ಕಾರ್ಯ-ಗಳನ್ನೂ
ಕಾರ್ಯ-ಗಳಲ್ಲಿ
ಕಾರ್ಯ-ಗಳಲ್ಲಿಯ
ಕಾರ್ಯ-ಗಳೂ
ಕಾರ್ಯ-ಗಳೆಲ್ಲ
ಕಾರ್ಯತೇ
ಕಾರ್ಯ-ದಲ್ಲಿ
ಕಾರ್ಯ-ದಲ್ಲಿಯೂ
ಕಾರ್ಯ-ವನ್ನು
ಕಾರ್ಯ-ವನ್ನೂ
ಕಾರ್ಯವು
ಕಾರ್ಯಾ
ಕಾರ್ಯಾ-ಕಾರ್ಯ-ಗಳು
ಕಾರ್ಯಾ-ಕಾರ್ಯವ್ಯವಸ್ಥಿತೌ
ಕಾರ್ಯಾಣ್ಯಪಿ
ಕಾಲ
ಕಾಲಂ
ಕಾಲಃ
ಕಾಲ-ಕಳೆದ
ಕಾಲ-ಕಾಲಕ್ಕೆ
ಕಾಲಕ್ಕೂ
ಕಾಲಕ್ಕೆ
ಕಾಲಕ್ರಮೇಣ
ಕಾಲ-ಗಳ
ಕಾಲ-ಗಳಲ್ಲಿ
ಕಾಲ-ಚೋದಿತಃ
ಕಾಲದ
ಕಾಲ-ದ-ನಂತರ
ಕಾಲ-ದಲ್ಲಿ
ಕಾಲ-ದಿಂದ
ಕಾಲದ್ರವ್ಯ-ಮಿತಿಂ
ಕಾಲ-ಧರ್ಮ-ಮುಪಾ-ಗತೌ
ಕಾಲ-ಮೃತ್ಯು-ವಿ-ನಿಂದ
ಕಾಲ-ಮೃತ್ಯು-ಹರಂ
ಕಾಲ-ಮೃತ್ಯೂ
ಕಾಲ-ವನ್ನು
ಕಾಲ-ವನ್ನೆಲ್ಲ
ಕಾಲವು
ಕಾಲಶ್ಚ
ಕಾಲಾಂತರೇ
ಕಾಲಾತ್
ಕಾಲಾ-ಯತೇ
ಕಾಲಾ-ಳು-ಗಳು-ಇವು-ಗಳನ್ನು
ಕಾಲಾವಕಾಶ-ಕೊಡು
ಕಾಲಿಂದೀ
ಕಾಲು-ಗಳನ್ನು
ಕಾಲುಗುರಿನ
ಕಾಲುವೆ
ಕಾಲುಷ್ಯಂ
ಕಾಲೇ
ಕಾಲೇನ
ಕಾಲೋ
ಕಾಳಿ-ನಷ್ಟು
ಕಾವೇರಿ
ಕಾವೇರಿ-ಯಲ್ಲಿ
ಕಾವೇರೀ
ಕಾವೇರೀಂ
ಕಾವೇರೀ-ಸಲಿಲೇ
ಕಾವೇರ್ಯಾಂ
ಕಾವ್ಯ
ಕಾವ್ಯ-ವೆಂದೂ
ಕಾಶಿ
ಕಾಶಿ-ಪಟ್ಟ-ಣ-ದಲ್ಲಿ
ಕಾಶಿ-ಯಲ್ಲಿ
ಕಾಶೀ
ಕಾಶೀಕ್ಷೇತ್ರ-ದಲ್ಲಿ
ಕಾಶ್ಚಿದುರು-ದೇಶ-ಶಿರಾಃ
ಕಾಶ್ಮೀರ-ದೇಶ-ದ-ವನು
ಕಾಶ್ಮೀರ-ದೇಶೀಯೋ
ಕಾಶ್ಯಪ
ಕಾಶ್ಯಪೋ
ಕಾಶ್ಯಾಂ
ಕಾಷ್ಠ
ಕಾಷ್ಠ-ದಿಂದ
ಕಿಂ
ಕಿಂಚನಾರೇ
ಕಿಂಚಿಜ್ಜಲಾರ್ಥಂ
ಕಿಂಚಿತ್
ಕಿಂಚಿತ್ಭರ್ತ್ಸಿತೌ
ಕಿಂಚಿತ್ಸಮ-ಫಲಂ
ಕಿಂಚಿದಂ
ಕಿಂಚಿದಚ್ಯುತಿದೇ
ಕಿಂಚಿ-ದನುಷ್ಠಿ-ತಮ್
ಕಿಂಚಿದಪಿ
ಕಿಂಚಿದ್ದತ್ವಾ
ಕಿಂಚಿನ್ನ
ಕಿಂಪುರುಷರು
ಕಿಂಪುರುಷಾ
ಕಿಂಶುಕಾ-ಶೋಕ-ಮಂಡಿತೇ
ಕಿತ್ತು-ಕೊಂಡೆವು
ಕಿನುತಾ-ಪರೇ
ಕಿನ್ನ
ಕಿನ್ನ-ರರು
ಕಿನ್ನ-ರಾಶ್ಚ
ಕಿಮ-ಕಾರ್ಯಂ
ಕಿಮತ್ರ
ಕಿಮಥ
ಕಿಮಸ್ತಿ
ಕಿಮು
ಕಿಯತ-ಕಾಲೇನ
ಕಿಯತಾ
ಕಿಯತಾ-ಕಾಲೇ
ಕಿಲ್ಬಿಷಾಂ
ಕಿವಿ-ಗಳನ್ನು
ಕಿವುಡನೋ
ಕೀನಾಶ-ಪಾಲಿ-ತಾಮ್
ಕೀರ್ತನ
ಕೀರ್ತ-ಯೇತ್
ಕೀರ್ತಿಂ
ಕೀರ್ತಿಮಿಚ್ಛಯಾ
ಕೀರ್ತಿ-ಯನ್ನು
ಕೀರ್ತಿ-ವಂತ-ನಾದ
ಕೀರ್ತಿ-ಶಾಲಿ-ಯಾದನು
ಕೀವು
ಕುಂಡ-ಗಳೂ
ಕುಂಡಿನೀ
ಕುಂಡಿನೇ
ಕುಂತಿಭೋಜ-ನೆಂಬ
ಕುಂತಿಭೋಜ-ಸುತಾತ್ರಯಃ
ಕುಂದರ-ದಲ್ಲಿ
ಕುಂಬಳಕಾ-ಯನ್ನು
ಕುಂಬಳ-ಕಾಯಿ
ಕುಂಬಳ-ಕಾಯಿ-ಯನ್ನು
ಕುಂಬಳ-ಕಾಯಿಯು
ಕುಂಬಾ-ರರು
ಕುಂಭ
ಕುಂಭೀಪಾಕ-ದಲ್ಲಿ
ಕುಂಭೀಪಾಕೇ
ಕುಂಭೋದ್ಭವ
ಕುಕ್ಕುಟ-ವೆಂಬ
ಕುಕ್ಕುಟ-ಸಂಜ್ಞಕೇ
ಕುಕ್ಕುಟಾನ್
ಕುಗತಿಃ
ಕುಚರೀ
ಕುಜ-ಪಂಚಮ್ಯಾಂ
ಕುಟೀರ-ಗಳಲ್ಲಿ
ಕುಟೀರೇಷು
ಕುಟುಂಬ
ಕುಟುಂಬದ
ಕುಟುಂಬ-ಭ-ರಣಾ-ಸಕ್ತೋ
ಕುಟುಂಬ-ಭ-ರಣೇ
ಕುಟುಂಬ-ಯುಕ್ತ-ನಾದ
ಕುಟುಂಬಿನೇ
ಕುಟ್ಟಿ
ಕುಡಿ-ದನು
ಕುಡಿದಿದ್ದು
ಕುಡಿ-ಯಲು
ಕುಡಿ-ಯುತ್ತಾರೆ
ಕುಡಿಯುತ್ತಿದ್ದನು
ಕುಡಿಯುತ್ತಿದ್ದೆ
ಕುಡ್ಮಲಂ
ಕುಡ್ಯಾದ್ಯಾರಂಭ-ಕಾರ್ಯೇಷು
ಕುತ
ಕುತರ್ಕೈಃ
ಕುತಸ್ತವ
ಕುತೂಹಲ-ಕಾರಿ-ಯಾದ
ಕುತೂಹಲ-ಜನ-ಕ-ವಾಗಿದೆ
ಕುತೂಹಲ-ದಿಂದ
ಕುತೂಹಲ-ವನ್ನು
ಕುತೂಹಲ-ವಾಗಿದೆ
ಕುತೋನ್ಯ-ಧರ್ಮೇ
ಕುತ್ತಿಗೆ-ಗಳನ್ನೂ
ಕುತ್ತಿಗೆ-ಯನ್ನು
ಕುತ್ರಾ-ಸೀತ್ಕೇನ
ಕುತ್ರಾ-ಸೀತ್ಸ
ಕುದುರೆ
ಕುಧೀಃ
ಕುಪಿತ-ರಾದರೆ
ಕುಪಿತೋ
ಕುಪ್ಪಯ್ಯ
ಕುಮಾರೌ
ಕುಯೋನಿರ್ನ
ಕುರಂಗಾಣಿ
ಕುರಿ
ಕುರಿತು
ಕುರು
ಕುರುಕ್ಷೇತ್ರ-ದಲ್ಲಿ
ಕುರುಕ್ಷೇತ್ರೇ
ಕುರುಡ
ಕುರು-ಡರು
ಕುರುತೇ
ಕುರು-ಬರು
ಕುರ್ಯಾತ್
ಕುರ್ಯಾತ್ಕರ್ಮಣಃ
ಕುರ್ಯಾತ್ಕರ್ಮ-ವಿಚಕ್ಷಣಃ
ಕುರ್ಯಾತ್ಕರ್ಮಾಣಿ
ಕುರ್ಯಾತ್ತತಃಪ-ರಮ್
ಕುರ್ಯಾತ್ಪ-ರಮಾತ್ಮನಃ
ಕುರ್ಯಾತ್ಪ್ರ-ಶಾಂತಾಯ
ಕುರ್ಯಾತ್ಸದಾ
ಕುರ್ಯಾತ್ಸಪ್ತಮ್ಯಾಂ
ಕುರ್ಯಾದಿತಿ
ಕುರ್ಯಾದ್ದೇವ-ತಾರ್ಚ-ನಮ್
ಕುರ್ಯಾದ್ಯದಿ
ಕುರ್ಯಾದ್ರಥ-ಸಪ್ತಮ್ಯಾಂ
ಕುರ್ಯಾನ್ನಿರ್ಮಾಲ್ಯಸ್ಯ
ಕುರ್ಯಾನ್ಮಂತ್ರಾದ-ತಂದ್ರಿತಃ
ಕುರ್ಯಾನ್ಮನೋ
ಕುರ್ಯಾನ್ಮಾಘ-ಮಾಸೇ
ಕುರ್ವಂಕರ್ಮ-ಫಲತ್ಯಾಗಂ
ಕುರ್ವಂತಿ
ಕುರ್ವಂತ್ಯೂರ್ಧ್ವ
ಕುರ್ವನ್
ಕುರ್ವನ್
ಕುರ್ವಾಣ-ಮಪಿ
ಕುಲ
ಕುಲಂ
ಕುಲ-ಕಾನಿ
ಕುಲ-ಕೊಟ್ಯುದ್ಧ
ಕುಲ-ಕೋಟಿ-ಸಮಾ-ಯುಕ್ತೋ
ಕುಲ-ಗಳನ್ನು
ಕುಲ-ಗಳಿಂದ
ಕುಲ-ಗಳಿಂದಲೂ
ಕುಲ-ಗಳು
ಕುಲ-ಗಳೂ
ಕುಲ-ಗುರು-ಗಳಾದ
ಕುಲ-ಗೆಡಿ-ಸಲು
ಕುಲದ
ಕುಲ-ದಲ್ಲಿ
ಕುಲ-ದ-ವ-ರನ್ನು
ಕುಲ-ದ-ವರೆಲ್ಲ
ಕುಲ-ದ-ವ-ರೊಡನೆ
ಕುಲ-ದಿಂದ
ಕುಲ-ದೇ-ವ-ರಾದ
ಕುಲ-ನಾಶೋ
ಕುಲ-ಪಾಂಸನಃ
ಕುಲಮ್
ಕುಲ-ವನ್ನು
ಕುಲವು
ಕುಲವೇ
ಕುಲ-ಶತೈರ್ಯುಕ್ತೋಸೌ
ಕುಲ-ಸೀಯಳಾದ
ಕುಲತ್ರಿ-ಯಾಮ್
ಕುಲಾಂಗನಾ
ಕುಲಾನಿ
ಕುಲಾ-ಲಹೈ-ಮೇಷು
ಕುಲೀನಶ್ಚ
ಕುಲೇ
ಕುಲೇಕ್ಷಣಾ
ಕುಲೇಶ್ವರಃ
ಕುಲೋತ್ತ-ರಣೈ
ಕುಲೋದ್ಧ
ಕುಲೋದ್ಭವಃ
ಕುಳಿತಿದ್ದರು
ಕುಳಿತು
ಕುಶ
ಕುಶಃ
ಕುಶಗ್ರಂಥಿ
ಕುಶಗ್ರಂಥಿ-ಯಿಂದ
ಕುಶಗ್ರಂಥಿ-ಸ-ಹಿತಃ
ಕುಶಗ್ರಂಥೇಶ್ಚ
ಕುಶಗ್ರಂಥೌ
ಕುಶ-ನೆಂದು
ಕುಶ-ಪಾಣಿಃ
ಕುಶ-ಭಾರ
ಕುಶಲಂ
ಕುಶ-ಲದ
ಕುಶ-ಲ-ನಾಗಿ
ಕುಶ-ಲನೇ
ಕುಶಾ-ದಾನ-ಪರಾತ್ಪು
ಕುಶಾದಿ-ಗಳ
ಕುಷ್ಠ
ಕುಷ್ಠ-ರೋಗಾದ್ವಿ-ಮುಚ್ಯತೇ
ಕುಸುಮೈರ್ಗಂಧ-ಲೇ-ಪನೈಃ
ಕುಸುಮೋತ್ಕರೇ
ಕೂಗಿ
ಕೂಗಿ-ದುವು
ಕೂಟ-ಬುದ್ಧಯಃ
ಕೂಟಾದ್ರಿ-ಕಂದರೇ
ಕೂಡಲೇ
ಕೂಡಿ
ಕೂಡಿ-ಕೊಂಡು
ಕೂಡಿದ
ಕೂಡಿದ್ದರೆ
ಕೂಡಿ-ರುತ್ತದೆ
ಕೂಡಿಸಿ
ಕೂಡಿ-ಸಿ-ಕೊಂಡು
ಕೂದಲಿ-ನಲ್ಲಿ
ಕೂದಲು
ಕೂದಲು-ಗಳನ್ನು
ಕೂಪ
ಕೂಪ-ಕುಲ್ಯ
ಕೂಪ-ದಲ್ಲಿ
ಕೂಪೇ
ಕೂಪೋದ-ಕಮಲಾಂಬುಂ
ಕೂರಂ
ಕೂರ-ರಾದ
ಕೂರ-ವಾದ
ಕೂಷ್ಮಾಂಡಂ
ಕೂಷ್ಮಾಂಡ-ಗಳನ್ನು
ಕೂಷ್ಮಾಂಡತಾಂ
ಕೂಷ್ಮಾಂಡ-ಮುತ್ತಮಮ್
ಕೂಷ್ಮಾಂಡಯೋನಿಜಃ
ಕೂಷ್ಮಾಂಡ-ವನ್ನು
ಕೂಷ್ಮಾಂಡ-ವೆಂಬ
ಕೂಷ್ಮಾಂಡಸ್ತು
ಕೂಷ್ಮಾಂಡಸ್ಯ
ಕೂಷ್ಮಾಂಡಾ
ಕೂಷ್ಮಾಂಡಾಖ್ಯಗಣೇ
ಕೂಷ್ಮಾಂಡಾನಾಂ
ಕೂಷ್ಮಾಂಡಾ-ನಾಮ-ದಾನೇನ
ಕೂಷ್ಮಾಂಡೋ
ಕೃಚಿತ್
ಕೃಣಂ
ಕೃತಂ
ಕೃತಃ
ಕೃತಕತ್ಯ-ರಾಗಿ
ಕೃತಕಿಲ್ಮಿಷಾತ್
ಕೃತ-ಕೃತ್ಯ-ನಾಗು-ವುದಿಲ್ಲ
ಕೃತ-ಕೃತ್ಯರು
ಕೃತ-ಕೃತ್ಯಾಃ
ಕೃತ-ಕೃತ್ಯೋ
ಕೃತಘ್ನ-ರಲ್ಲಿ
ಕೃತಘ್ನ-ರಾದ
ಕೃತಘ್ನಾಶ್ಚ
ಕೃತಘ್ನೇ
ಕೃತಘ್ನೇನ
ಕೃತಜ್ಞತೆ-ಗಳನ್ನು
ಕೃತತ್ವೌರ್ಧ್ವ-ದೇಹಿಕಂ
ಕೃತ-ಪಾಪಾನಿ
ಕೃತಪ್ರಯತ್ಯಾ-ಪೇಕ್ಷಸ್ತು
ಕೃತಪ್ರಶ್ನಾದ್ವಾಮ-ದೇವಾತ್
ಕೃತ-ಮಾಘಸ್ಯ
ಕೃತಮ್
ಕೃತ-ವಾನ್
ಕೃತವೀರ್ಯನು
ಕೃತ-ಸರ್ವ-ಕೃತ್ಯಾ
ಕೃತಸ್ನಾನಂ
ಕೃತಸ್ನಾ-ನ-ಫಲಂ
ಕೃತಾ
ಕೃತಾಂ
ಕೃತಾಃ
ಕೃತಾತ್ಮ-ನಾಮ್
ಕೃತಾರ್ಥ-ನಾಗುತ್ತಾನೆ
ಕೃತಾರ್ಥ-ನಾಗುವನು
ಕೃತಾರ್ಥ-ನಾದನು
ಕೃತಾರ್ಥರಾಗು-ವರು
ಕೃತಿನೊ
ಕೃತಿಯ-ನಿಗೆ
ಕೃತೇ
ಕೃತೇನ
ಕೃತೋ
ಕೃತೌ
ಕೃತ್ತಿಕಾನಕತ್ರದ
ಕೃತ್ತಿಕಾಯೋಗಸ್ತ-ದಾಸೀತ್ಪುಣ್ಯ-ವರ್ಧನಃ
ಕೃತ್ಯ
ಕೃತ್ಯಂ
ಕೃತ್ಯಮ್
ಕೃತ್ಯ-ವಿವೇಕ-ವಾನ್
ಕೃತ್ಯಾ
ಕೃತ್ಯಾ-ಢ-ಕಾನಿ
ಕೃತ್ಯೇಷು
ಕೃತ್ಯೈರ್ಬಹು-ವಿಧೈರಪಿ
ಕೃತ್ವಾ
ಕೃತ್ವಾ-ನಿಶಂ
ಕೃತ್ವಾಪ್ಯ
ಕೃತ್ವಾಯಂ
ಕೃತ್ವೋಷಸಿ
ಕೃದ್ಧಾ
ಕೃಪಣಂ
ಕೃಪಯಾ
ಕೃಪಾ
ಕೃಪಾಂ
ಕೃಪಾ-ಕರಃ
ಕೃಪಾ-ನಿಧೇ
ಕೃಪಾ-ಲವಃ
ಕೃಪಾ-ಲಸ್ತ್ವಂ
ಕೃಪಾ-ಲು-ಗಳೇ
ಕೃಪಾಲೋ
ಕೃಪಾ-ಳು-ಗಳು
ಕೃಪಾ-ವಿಷ್ಟೇನ
ಕೃಪಾ-ಸುಧೌಷಧಂ
ಕೃಪೆ-ಮಾಡಿ
ಕೃಪೆ-ಮಾಡಿರಿ
ಕೃಪೆ-ಯಿಂದ
ಕೃಮಿಃ
ಕೃಮಿ-ಯಾಗಿ
ಕೃಮಿಯಾಗುತ್ತಾನೆ
ಕೃಶ-ವಾದ
ಕೃಶಾನೋಃ
ಕೃಷಿ
ಕೃಷಿ-ಪರಃ
ಕೃಷ್ಣ
ಕೃಷ್ಣನ
ಕೃಷ್ಣ-ನಿಗೆ
ಕೃಷ್ಣನು
ಕೃಷ್ಣನೇ
ಕೃಷ್ಣಾಗಾರು-ಇಂತಹ
ಕೃಷ್ಣಾಗಾರುಸಮನ್ವಿ-ತಮ್
ಕೃಷ್ಣಾಮೃತ-ಮಹಾರ್ಣವ
ಕೆಂಪಗೆ
ಕೆಂಪು
ಕೆಂಪು-ಮೂಲಂಗಿ
ಕೆಟ್ಟ
ಕೆಡವುತ್ತಾನೆ
ಕೆಡವುತ್ತಾರೆ
ಕೆದರಿ-ಕೊಂಡ
ಕೆದರಿದ
ಕೆಮ್ಮಣ್ಣಿ-ನಿಂದ
ಕೆರಳಿ-ಸುವ
ಕೆರೆ
ಕೆರೆ-ಗಳ
ಕೆರೆ-ಗಳು
ಕೆರೆ-ಗಳೂ
ಕೆಲ-ವರು
ಕೆಲ-ವರೂ
ಕೆಲವು
ಕೆಲಸ
ಕೆಲಸ-ಮಾಡಲು
ಕೆಲಸ-ಮಾಡುತ್ತೇವೆ
ಕೆಲಸ-ವನ್ನೂ
ಕೆಲಸವೂ
ಕೆಳ-ಗಿನ
ಕೆಳಗೆ
ಕೆಸ-ರನ್ನು
ಕೇ
ಕೇಚಿಚ್ಚ
ಕೇಚಿತ್
ಕೇದಾರ
ಕೇದಾರಂ
ಕೇನ
ಕೇನಾಪಿ
ಕೇಯ-ಮಾನ್ವೀಕ್ಷಿಕೀ
ಕೇಳಲು
ಕೇಳಲ್ಪಟ್ಟಿಲ್ಲ
ಕೇಳಿ
ಕೇಳಿದ
ಕೇಳಿ-ದ-ನಂತರ
ಕೇಳಿ-ದನು
ಕೇಳಿ-ದರು
ಕೇಳಿ-ದರೆ
ಕೇಳಿ-ದ-ವನು
ಕೇಳಿ-ದೆವು
ಕೇಳು
ಕೇಳುತ್ತಿದ್ದರೂ
ಕೇಳು-ವ-ವನು
ಕೇಳು-ವುದ-ರಿಂದ
ಕೇಳು-ವುದು
ಕೇಳು-ಸಮಸ್ತ
ಕೇವಲ
ಕೇವಲಸ್ಥಲೇ
ಕೇಶ
ಕೇಶವಃ
ಕೇಶ-ವ-ನನ್ನು
ಕೇಶ-ವ-ನಿಗೆ
ಕೇಶ-ವನು
ಕೇಶ-ವಸ್ಯಾಗ್ರೇ
ಕೇಶ-ವಾಜ್ಞಯಾ
ಕೇಶ-ವಾ-ಲಯೇ
ಕೇಶವೋ
ಕೇಶ-ವೋ-ಪರಿ
ಕೇಽಪಿ
ಕೈ
ಕೈಕೊಂಡರೆ
ಕೈಗಳಿಗೆ
ಕೈಗೂಡು-ವುದಿಲ್ಲ
ಕೈಗೊಂಡನು
ಕೈಗೊಂಡು
ಕೈಗೊಳ್ಳ-ಬೇಕಾದ
ಕೈಟಭದ್ವಿಷಿ
ಕೈಯಲ್ಲಿ
ಕೈಲಾಸ-ದಲ್ಲಿ
ಕೈಲಾಸ-ದಿಂದ
ಕೈಲಾಸಶಿಖರೇ
ಕೈಲಾಸಾದಾ-ಗತೋ
ಕೈಶ್ಚಿತ್
ಕೊ
ಕೊಂಕಳಿ-ನಲ್ಲಿ
ಕೊಂಕುಳಲ್ಲಿ
ಕೊಂಡರು
ಕೊಂಡರೆ
ಕೊಂಡು
ಕೊಂದ
ಕೊಂದು
ಕೊಂಬೆ-ಗಳು
ಕೊಟ-ಯಜ್ಞ-ಫಲಂ
ಕೊಟಸ್ತು
ಕೊಟಿ-ರಾಣಿ
ಕೊಟಿ-ವಾರಂ
ಕೊಟಿ-ಸಹಸ್ರಂ
ಕೊಟಿ-ಸಹಸ್ರಾಣಿ
ಕೊಟೀಂದುಗ್ರಹಸಂಮತಃ
ಕೊಟ್ಟ
ಕೊಟ್ಟನು
ಕೊಟ್ಟರು
ಕೊಟ್ಟರೆ
ಕೊಟ್ಟ-ವರು
ಕೊಟ್ಟಿಗೆಯಲ್ಲಾಗಲೀ
ಕೊಟ್ಟಿಗೆ-ಯಲ್ಲಿ
ಕೊಟ್ಟಿಗೆ-ಯಲ್ಲಿನ
ಕೊಟ್ಟಿದ್ದ
ಕೊಟ್ಟಿರುತ್ತೇನೆ
ಕೊಟ್ಟು
ಕೊಟ್ಟೆ
ಕೊಡ
ಕೊಡ-ತಕ್ಕದ್ದು
ಕೊಡದೆ
ಕೊಡದೇ
ಕೊಡ-ಬೇಕು
ಕೊಡ-ಬೇಕೆಂದು
ಕೊಡ-ಲಾಗದಿದ್ದಲ್ಲಿ
ಕೊಡ-ಲಾರವು
ಕೊಡಲಿ
ಕೊಡ-ಲಿಲ್ಲ
ಕೊಡಲು
ಕೊಡಲೇ
ಕೊಡಲ್ಪಟ್ಟ
ಕೊಡಲ್ಪಡುತ್ತಿ-ರಲು
ಕೊಡಿರಿ
ಕೊಡಿಸಿ-ದನು
ಕೊಡು
ಕೊಡುತ್ತ
ಕೊಡುತ್ತವೆ
ಕೊಡುತ್ತಾನೆ
ಕೊಡುತ್ತಾನೆ-ಯಲ್ಲದೇ
ಕೊಡುತ್ತಾನೆಯೋ
ಕೊಡುತ್ತಾನೋ
ಕೊಡುತ್ತಾರೆಯೋ
ಕೊಡುತ್ತಿದ್ದೆ
ಕೊಡುತ್ತಿ-ರ-ಲಿಲ್ಲ
ಕೊಡುತ್ತಿ-ರುವ
ಕೊಡುತ್ತೀಯೋ
ಕೊಡುತ್ತೇನೆ
ಕೊಡುವ
ಕೊಡು-ವಂತೆ
ಕೊಡು-ವರು
ಕೊಡು-ವ-ವ-ರಲ್ಲ
ಕೊಡು-ವ-ವರು
ಕೊಡು-ವ-ವರೆವಿಗೂ
ಕೊಡು-ವಾಗ
ಕೊಡೆಂದು
ಕೊನೆ-ಗೊಂಡಿತು
ಕೊನೆ-ಯಲ್ಲಿ
ಕೊಪಕಷಾಯಾಕ್ಷೌ
ಕೊಲ್ಲ-ಬೇಕೆಂಬ
ಕೊಲ್ಲಲ್ಪಟ್ಟ
ಕೊಲ್ಲಲ್ಪಟ್ಟನು
ಕೊಲ್ಲಲ್ಪಟ್ಟು
ಕೊಳ್ಳ-ಬಾರದು
ಕೊಳ್ಳ-ಬೇಕು
ಕೊಳ್ಳಲು
ಕೊಳ್ಳುತ್ತಿದ್ದೆ
ಕೊಳ್ಳುವುದು
ಕೊಳ್ಳುವುದೂ
ಕೊವಿ-ದಮ್
ಕೋ
ಕೋಕನದೈಃ
ಕೋಕಿಲಕೂಜಿತೇ
ಕೋಕಿಲಾಕಲ್ಕಲಾ-ರಾವಮಿ-ಲದ್ವಂಶಸ್ವನೇ
ಕೋಕಿಲಾರಾವಕೋಮಲೇ
ಕೋಗಿಲೆ-ಗಳ
ಕೋಗಿಲೆ-ಗಳು
ಕೋಗಿಲೆಯ
ಕೋಟಿ
ಕೋಟಿ-ಕಲ್ಪ
ಕೋಟಿ-ಕುಲ-ಮುದ್ಧೃತ್ಯ
ಕೋಟಿ-ಕೋಟಿ-ಗುಣಂ
ಕೋಟಿ-ಗುಣ
ಕೋಟಿ-ಗುಣಂ
ಕೋಟಿ-ಗುಣಿತಂ
ಕೋಟಿ-ಗುರ್ವಂಗನಾ-ಸೇವಾ
ಕೋಟಿ-ಜನ್ಮ-ಗಳಲ್ಲಿ
ಕೋಟಿ-ಜನ್ಮಸು
ಕೋಟಿ-ಜನ್ಮಾರ್ಜಿತಂ
ಕೋಟಿ-ಪಾಪಾನಿ
ಕೋಟಿ-ಪುಣ್ಯಂ
ಕೋಟಿ-ಯಜ್ಞ-ಗಳಿಂದ
ಕೋಟಿ-ಯಷ್ಟು
ಕೋಟಿರಂ
ಕೋಟಿ-ವಾರಂ
ಕೋಟಿ-ಶತಮ-ಗಮ್ಯಾ-ನಾಮ-ನಂತ-ಕಮ್
ಕೋಟಿ-ಶತೈರ್ವಾಪಿ
ಕೋಟೆ-ಯಿಂದ
ಕೋಟ್ಯರ್ಧಕೊಟ್ಯ
ಕೋಪ
ಕೋಪ-ಗೊಂಡ
ಕೋಪ-ಗೊಂಡು
ಕೋಪ-ದಿಂದ
ಕೋಪನಃ
ಕೋಪ-ರುಕ್ಷಾಕ್ಷೋ
ಕೋಪಾ-ದಶ-ಪನ್ನೌ
ಕೋಪಿ
ಕೋಪಿತಃ
ಕೋಪಿಷ್ಠ-ನಾದ
ಕೋಪ್ಯ
ಕೋರಿ-ದಳು
ಕೋರೆ
ಕೋಲಿ-ನಿಂದ
ಕೋಳಿ
ಕೋವಾ-ಸೀತ್ಕೇ-ನಾಸೌ
ಕೋಸಲ
ಕೋಸಲೇ
ಕೋಸುಂಭಂ
ಕೋಸೌ
ಕೋಽದಾತ್
ಕೋಽಪಿ
ಕೋಽಸೌ
ಕೌಂಡಿನ್ಯ
ಕೌಂತೇಯ
ಕೌತುಕ
ಕೌತುಕಂ
ಕೌತುಕ-ವಿದ್ಯಾಸು
ಕೌತೂಹಲಂ
ಕೌತೂಹಲಾಯ
ಕೌರವೇ
ಕೌಲೀನಂ
ಕೌಶಲ್ಯ-ವನ್ನು
ಕೌಶಿಕ
ಕೌಶಿಕ-ಗೊತ್ರಜಃ
ಕೌಶಿಕ-ನಿಗೆ
ಕೌಶಿಕ-ನೆಂಬ
ಕೌಶಿಕ-ವೆಂಬ
ಕೌಶಿಕಾಖ್ಯಾಂ
ಕೌಶಿಕೀ
ಕೌಶಿಕೀ-ನಾಮ
ಕೌಶಿಕೋ
ಕೌಶಿಕ್ಯಾಂ
ಕೌಷಿ-ಕಾನಾಂ
ಕೌಷೀತಕ
ಕೌಷೀತಕ-ಕುಲೋದ್ಭವಃ
ಕ್ಕೆ
ಕ್ತಿರ್ನ
ಕ್ಯಾಪಿ
ಕ್ರಂದಂ
ಕ್ರತು-ಕೋಟಿ-ಶತಾಧಿಕೌ
ಕ್ರತುರ್ವಾ
ಕ್ರತು-ಶ-ತಾನಿ
ಕ್ರಮ
ಕ್ರಮಕ್ರಮ-ವಾಗಿ
ಕ್ರಮ-ದಲ್ಲಿ
ಕ್ರಮ-ದಿಂದ
ಕ್ರಮ-ವನ್ನು
ಕ್ರಮ-ವನ್ನೂ
ಕ್ರಮ-ವಾಗಿ
ಕ್ರಮವೂ
ಕ್ರಮಾತ್
ಕ್ರಮಾದ್ದಶಸು
ಕ್ರಮಾನ್ಮೋಕ್ಷಾಯ
ಕ್ರಮೇಣ
ಕ್ರಮೇ-ಣೈವ
ಕ್ರಯಕ್ಕಾಗಿ
ಕ್ರಯಕ್ಕೆ
ಕ್ರಯವಿಕ್ರಯದ್ರವ್ಯಾಣಾಂ
ಕ್ರಯಾಲ್ಲಬ್ಧ
ಕ್ರಿತಾಃ
ಕ್ರಿಯತೇ
ಕ್ರಿಯವಾ-ಣಸ್ಯ
ಕ್ರಿಯವಿಕ್ರಯಃ
ಕ್ರಿಯಾ
ಕ್ರಿಯಾಃ
ಕ್ರಿಯಾ-ದಲ್ಲಿ
ಕ್ರಿಯಾ-ದಿ-ಗಳನ್ನು
ಕ್ರಿಯಾ-ಫಲಮ್
ಕ್ರಿಯಾ-ಫಲಾನ್ಯರ್ಥ-ವಾ-ದಾನ್
ಕ್ರಿಯಾಯಾಂ
ಕ್ರಿಯಾ-ಯೋಗಂ
ಕ್ರಿಯಾ-ಲೋಪಗ-ತಾಶ್ಚ
ಕ್ರಿಯಾ-ವಾತ್ರಂ
ಕ್ರಿಯಾ-ಶೀಲತೆ
ಕ್ರೀಡಂತೋಃ
ಕ್ರೀಡಂತೌ
ಕ್ರೀಡಾ-ವನಂ
ಕ್ರೀಡಿಸ-ಬೇಕೆಂಬ
ಕ್ರೀಡಿ-ಸಲು
ಕ್ರೀಡಿಸಿ-ದನು
ಕ್ರೀಡಿ-ಸಿದೆವು
ಕ್ರೀತಂ
ಕ್ರೀತಾಃ
ಕ್ರೀತ್ವಾ
ಕ್ರೀಯತೇ
ಕ್ರುದ್ಧ
ಕ್ರುಧಾ
ಕ್ರೂರಃ
ಕ್ರೂರ-ತನ
ಕ್ರೂರ-ನಾದ
ಕ್ರೂರ-ರಾಗಿ
ಕ್ರೂರ-ರಾದ
ಕ್ರೂರ-ವಾದ
ಕ್ರೂರಸ್ತಥಾ
ಕ್ರೂರಾ
ಕ್ರೂರಾಸ್ತೇ
ಕ್ರೂರಿ
ಕ್ರೂರಿ-ಯಾಗಿ
ಕ್ರೂರಿ-ಯಾಗಿದ್ದೆ
ಕ್ರೂರೌ
ಕ್ರೋಧಾದಾರುರೋಹ
ಕ್ರೌರ್ಯ
ಕ್ಲಾಂತೋ
ಕ್ಲಿ
ಕ್ಲಿಷ್ಯಮಾ-ನಾನಾಂ
ಕ್ಲೀಯತೇ
ಕ್ವ
ಕ್ವಚಿತ್
ಕ್ವಚಿತ್ಕ್ವಾಪಿ
ಕ್ವಚಿದ್ದರ್ಮಮ-ಧರ್ಮ-ಮಿತಿ
ಕ್ವಚಿದ್ಭವೇತ್
ಕ್ವಥಿತ-ಧಾನ್ಯ-ವತ್
ಕ್ವಾಪಿ
ಕ್ವಾಪ್ಯ-ಟಿವೀ
ಕ್ವಿಂ
ಕ್ಷಗುಣೋ
ಕ್ಷಣ
ಕ್ಷಣಂ
ಕ್ಷಣ-ಮಾತ್ರ
ಕ್ಷಣ-ಮಾತ್ರಂ
ಕ್ಷಣ-ಮಾತ್ರ-ದಲ್ಲಿ
ಕ್ಷಣ-ಮುತಿ
ಕ್ಷಣ-ವಾದರೂ
ಕ್ಷಣಾತ್
ಕ್ಷಣಾ-ದ-ಭೂತ್
ಕ್ಷಣಾ-ದಯಃ
ಕ್ಷಣಾ-ದೇವ
ಕ್ಷತ್ರಿಯ
ಕ್ಷತ್ರಿ-ಯನ
ಕ್ಷತ್ರಿಯ-ನಾಗಿ
ಕ್ಷತ್ರಿಯ-ರಾಜ-ನಿದ್ದನು
ಕ್ಷತ್ರಿಯ-ರಿಗೆ
ಕ್ಷತ್ರಿ-ಯರು
ಕ್ಷತ್ರಿಯ-ವೈಶ್ಯೈಶ್ಚ
ಕ್ಷತ್ರಿ-ಯಸ್ಯ
ಕ್ಷತ್ರಿಯೇಭ್ಯೋ
ಕ್ಷತ್ರಿಯೋ
ಕ್ಷತ್ರೇಷು
ಕ್ಷಮಃ
ಕ್ಷಮತೇ
ಕ್ಷಮಸ್ವಾ-ಗಾಂಸಿ
ಕ್ಷಮಾಃ
ಕ್ಷಮಾಲಾಂ
ಕ್ಷಮಾಲೆ
ಕ್ಷಮಿಸಿ
ಕ್ಷಮಿ-ಸಿರಿ
ಕ್ಷಮಿಸು
ಕ್ಷಮಿಸುತ್ತಾನೆ
ಕ್ಷಮ್ಯತಾಂ
ಕ್ಷಾತ್
ಕ್ಷಿತಿಪಾ
ಕ್ಷಿಪ್ತ್ವಾ
ಕ್ಷಿಪ್ರಂ
ಕ್ಷೀಣ-ವಾಗುತ್ತದೆ
ಕ್ಷೀಣ-ವಾಗು-ವುದಿಲ್ಲ
ಕ್ಷೀಯಂತೇ
ಕ್ಷೀಯತೇ
ಕ್ಷೀರಂ
ಕ್ಷೀರ-ಸಮುದ್ರ-ದಲ್ಲಿ
ಕ್ಷೀರಾಬ್ಧಿ
ಕ್ಷೀರಾಬ್ಧಿ-ಶಾಯಿ-ಯಾದ
ಕ್ಷೀರಾಭಿಷೇಕ
ಕ್ಷುತ್ತೃಟ್
ಕ್ಷುದ್ರ
ಕ್ಷುಬ್ಧಂ
ಕ್ಷುಬ್ಧೋ
ಕ್ಷೇತ್ರಂ
ಕ್ಷೇತ್ರಕ್ಕೆ
ಕ್ಷೇತ್ರ-ಗಳು
ಕ್ಷೇತ್ರ-ದಲ್ಲಿ
ಕ್ಷೇತ್ರ-ವೆಂಬ
ಕ್ಷೇತ್ರಾಣಿ
ಕ್ಷೇತ್ರಿ-ಯದಾ-ಯಾದಾಃ
ಕ್ಷೇತ್ರೇ
ಕ್ಷೇಮಾಯ
ಕ್ಷೋಭಿ-ತಾತ್ಕಾಮ-ಕರ್ಮಭಿಃ
ಕ್ಷೋಭೆ-ಯನ್ನುಂಟು-ಮಾಡು-ವುದು
ಕ್ಷೌಮಂ
ಕ್ಷೌರ-ಕರ್ಮ
ಕ್ಷೌರ-ಕರ್ಮ-ವನ್ನು
ಖಂಡಯು-ತಾನಿ
ಖಗೌ
ಖಡೋ
ಖಡ್ಗತಃ
ಖಡ್ಗ-ಮಾ-ದಾಯ
ಖಡ್ಗ-ವನ್ನು
ಖಡ್ಗವು
ಖರ್ಚಾಗಿ
ಖಲು
ಖಶ-ತಾದಪಿ
ಖಿಲಸಂಶ್ರಯಃ
ಖುಷಿರು-ವಾಚ
ಖೇದ-ಗೊಂಡ
ಖ್ಯಾತಾ
ಗ
ಗಂಗಾ
ಗಂಗಾಂ
ಗಂಗಾ-ತಟಂ
ಗಂಗಾದಿ
ಗಂಗಾ-ದಿ-ತೀರ್ಥ-ಸೇವಾ
ಗಂಗಾ-ದಿ-ಸರ್ವ-ತೀರ್ಥಷು
ಗಂಗಾದ್ಯಾಃ
ಗಂಗಾ-ನದಿ-ಗಳೆಂದು
ಗಂಗಾ-ನದಿಯು
ಗಂಗಾ-ನದೀನಾಂ
ಗಂಗಾ-ಯಾತ್ರಾ-ಪರಾ-ಯಣಾಃ
ಗಂಗಾಸ್ನಾನ
ಗಂಗೆಯ
ಗಂಗೆ-ಯಂತೆ
ಗಂಗೆಯು
ಗಂಗೇ
ಗಂಜಿ-ಯನ್ನು
ಗಂಡು
ಗಂಡೂಷಂ
ಗಂಡೂಷಮ-ಥವಾ
ಗಂತುಮಭ್ರಾದ-ವತ-ರನ್
ಗಂತುಮಶಕ್ತಸ್ಯ
ಗಂತುಮುದ್ಯತಃ
ಗಂತುಮುದ್ಯ-ತಾವುದ್ಯತಾಯುಧೈಃ
ಗಂತುಮುದ್ಯುಕ್ತಾ
ಗಂತುಮುಪಾಕ್ರಮತ್
ಗಂಧ
ಗಂಧಕ್ಷಯ-ಕರಃ
ಗಂಧಕ್ಷಯೋ
ಗಂಧ-ದಲ್ಲಿ
ಗಂಧ-ದಿಂದಲೂ
ಗಂಧ-ಯುಕ್ತಂ
ಗಂಧರ್ವ
ಗಂಧರ್ವ-ನಾದ
ಗಂಧರ್ವನು
ಗಂಧರ್ವ-ಮಾಯಯಾ-ನೀಯ
ಗಂಧರ್ವರು
ಗಂಧರ್ವಸ್ಯ
ಗಂಧರ್ವೊ
ಗಂಧರ್ವೋ
ಗಂಧ-ವನ್ನು
ಗಂಧ-ವನ್ನೂ
ಗಂಧೈಃ
ಗಚ್ಚ
ಗಚ್ಛ
ಗಚ್ಛಂತಿ
ಗಚ್ಛಂತ್ಯ
ಗಚ್ಛಂಸ್ತುಂಬರುರ್ಯಾನಸಂಸ್ಥಿತಃ
ಗಚ್ಛಂಸ್ತೇನೈವ
ಗಚ್ಛತ
ಗಚ್ಛತಿ
ಗಚ್ಛಥ
ಗಚ್ಛನ್
ಗಚ್ಛನ್ನಾಶ್ರಮ-ಮಂಡಲಂ
ಗಚ್ಛನ್ಮಧ್ಯಾಹ್ನ-ವೇಲಾಯ-ಮ-ರಣ್ಯೇ
ಗಚ್ಛಾಮೋ
ಗಚ್ಛೇತಾಂ
ಗಚ್ಛೇತಿ
ಗಚ್ಛೇತ್
ಗಜ
ಗಜ-ಭುಕ್ತ
ಗಜವಕ್ತ್ರಶ್ಚ
ಗಜಾನ್
ಗಟ್ಟಿ-ಯಾಗಿ
ಗಣಕಶ್ಚಾಸ್ಮಿ
ಗಣಕ್ಕೆ
ಗಣದ
ಗಣ-ದಲ್ಲಿ
ಗಣ-ದಲ್ಲಿನ
ಗಣ-ದಲ್ಲಿಯೇ
ಗಣನಾ
ಗಣನೆಗೆ
ಗಣೇಗ್ರಣೀಃ
ಗಣ್ಯಂತೇ
ಗಣ್ಯತಾಂ
ಗತಂ
ಗತಃ
ಗತಕಲ್ಮಷಾಃ
ಗತಕ್ಷೇಶಭಯೋ
ಗತ-ದುಷ್ಟ
ಗತಮತೌ
ಗತಮಸ್ಮಾಭಿರ್ಲೀ-ಲಯಾ
ಗತಶ್ರಮೊ
ಗತಸ್ತೆ
ಗತಾ
ಗತಾಂಹಸೋ
ಗತಾಃ
ಗತಾ-ಯುರಭ-ವನ್ನೃಪ
ಗತಿ
ಗತಿಂ
ಗತಿಃ
ಗತಿ-ಗಿಂತ
ಗತಿ-ನುನ್ಯಥಾ
ಗತಿ-ಮವಾಪ್ನೋತಿ
ಗತಿ-ಮಸ್ಮಾಕಂ
ಗತಿ-ಮಾಪ್ನೋತಿ
ಗತಿಮ್
ಗತಿ-ಯನ್ನು
ಗತಿ-ಯಾಗುವ
ಗತಿಯು
ಗತಿಯೇ
ಗತಿ-ರದ್ಯ
ಗತಿ-ರುಕ್ತಾ
ಗತಿ-ವಿಹೀ-ನಾನಾಂ
ಗತಿಸ್ತಯೋರ್ಭಿನ್ನಾ
ಗತಿ-ಹೀನ-ರಾದ-ವ-ರಿಗೆ
ಗತೇ
ಗತೋ
ಗತೋ-ಽಸಾವಲಕಾ-ಪುರೀಮ್
ಗತೌ
ಗತ್ವಾ
ಗತ್ವಾ-ಭಾಷ್ಯಾ-ಗತಾ
ಗದ್ಗ-ದಕಂಠ-ದಿಂದ
ಗದ್ಗದ-ಭಾಷಣಃ
ಗದ್ಗದ-ವಾಗಿವೆ
ಗದ್ಯಂ
ಗನಿಷ್ಯತಃ
ಗಮನ
ಗಮನವೇ
ಗಮಿಷ್ಯಾಮಿ
ಗಮ್ಯತೇ
ಗಯ-ನೆಂಬ
ಗಯಸ್ಯ
ಗಯಾ
ಗಯಾಂ
ಗಯಾಶ್ರಾದ್ಧ
ಗಯಾಶ್ರಾದ್ಧಂ
ಗಯಾಶ್ರಾದ್ಧದ
ಗರದಾ
ಗರೀ-ಯಾನ್
ಗರುಡ
ಗರುಡಃ
ಗರುಡ-ನಂತೆ
ಗರುಡ-ನಿಗೆ
ಗರುಡನು
ಗರುಡ-ಭೀತಸ್ತು
ಗರುಡೋ
ಗರೇಣ
ಗರ್ಜಂತಿ
ಗರ್ಜತಿ
ಗರ್ಜಿಸುತ್ತದೆ
ಗರ್ದಭಾಂಶ್ಚ
ಗರ್ಭಿಣಿ
ಗರ್ಭೇ
ಗಲೇ
ಗಳನ್ನು
ಗಳನ್ನೂ
ಗಳಿಂದ
ಗಳಿಂದಲೂ
ಗಳಿಗೆ
ಗಳಿ-ಸಲು
ಗಳಿಸಿ
ಗಳಿ-ಸಿದ
ಗಳಿಸಿ-ದನು
ಗಳೆಲ್ಲ-ವನ್ನೂ
ಗವಿ
ಗವ್ಯೇನಾಸುರಘಾ-ತಿನಮ್
ಗಹನ
ಗಹನ-ವಾ-ದುದು
ಗಹನಾ
ಗಹ್ವರಸ್ಟೋ
ಗಾ
ಗಾಂ
ಗಾಂಧರ್ವ
ಗಾಂಧರ್ವೇಣ
ಗಾನ
ಗಾಯಕ
ಗಾಯಕಃ
ಗಾಯ-ಕರು
ಗಾಯತ್ರಿ
ಗಾಯತ್ರಿ-ಯಂತೆ
ಗಾಯತ್ರೀ
ಗಾಯತ್ರೀ-ಜಪ್ಯ
ಗಾರ್ಗ್ಯ-ಗೋತ್ರ-ದ-ವನು
ಗಾರ್ಗ್ಯ-ಗೋತ್ರೇ
ಗಾರ್ಹ-ಪತ್ಯ
ಗಾಳಿಯ
ಗಾಳಿ-ಯಂತೆ
ಗಾಳಿ-ಯನ್ನು
ಗಾವಶ್ಚಾಗ್ನಿಃ
ಗಾಶ್ಚ
ಗಿಡ-ಗಳ
ಗಿಡ-ಗಳಿಂದ
ಗಿಡ-ಗಳಿ-ಗಿಂತ
ಗಿಡ-ಗಳು
ಗಿಡ-ದಲ್ಲಿದ್ದ
ಗಿಡ-ಮರ-ಗಳು
ಗಿಡ-ಮರ-ಗಳೂ
ಗಿಡ-ಮೂಲಿಕೆ-ಗಳು
ಗಿಡ-ವಾಗಿ
ಗಿಡವೇ
ಗಿರಿಜಾ-ಕೃತೇ
ಗಿರಿಜಾ-ದೇವಿಯ-ರಿಗೆ
ಗಿರೀಂದ್ರಶೃಂಗೋದ್ಭು-ತ-ದೇವ-ಖಾತೇ
ಗಿರೌ
ಗೀತಾ
ಗೀತಾ-ಪಾಠ
ಗೀತಾ-ಪಾಠ-ಪರಾ-ಯಣಃ
ಗೀತಾ-ಪಾಠೇ
ಗೀತಾ-ಪಾಠೇಣ
ಗೀತಾ-ಪಾಠೋ
ಗೀತಾ-ಪಾರಾ-ಯಣ-ವನ್ನು
ಗೀತಾಯಾಃ
ಗೀತಾ-ಯಾಶ್ಚ
ಗೀತೆಯ
ಗೀತೆ-ಯನ್ನು
ಗೀಯತೇ
ಗುಂಪಿಗೆ
ಗುಗ್ಗು
ಗುಗ್ಗುಲಂ
ಗುಡಂ
ಗುಡಯುಕ್ತಾಶ್ಚ
ಗುಡಾ-ಕೇಶ
ಗುಡಾನ್ವಿ-ತಮ್
ಗುಡುಗು
ಗುಣ
ಗುಣ-ಗಳ
ಗುಣ-ಗಳನ್ನು
ಗುಣ-ಗಳಿಂದ
ಗುಣ-ತಾರ-ತಮ್ಯಕ್ಕೆ
ಗುಣತ್ರಯ
ಗುಣತ್ರಯೀ
ಗುಣ-ದಷ್ಟು
ಗುಣ-ನಾಮೈವಂ
ಗುಣ-ಪೂರ್ಣ-ನಾದ
ಗುಣ-ಭೇದ-ದಿಂದ
ಗುಣ-ವಂತ-ರಾದ
ಗುಣ-ವಂತೆ-ಯಾದ
ಗುಣ-ವತಿಯು
ಗುಣ-ವತೀ
ಗುಣ-ವಾಗು-ವಂತೆ
ಗುಣ-ಸಾಂದ್ರೇ
ಗುಣಾಃ
ಗುಣಾಢ್ಯಾಃ
ಗುಣಾನು-ಸಂಗಾಜ್ಜೀ-ವಾಶ್ಚ
ಗುಣಾಶ್ರ-ಯಾತ್
ಗುಣೇಷು
ಗುಣೋತ್ಕರ್ಷಣೆ-ಯಲ್ಲಿಯೇ
ಗುದದ್ವಾ-ರಕ್ಕೆ
ಗುದೇ
ಗುಪ್ತ-ದಾನ
ಗುಪ್ತ-ದಾನಂ
ಗುಪ್ತ-ದಾನ-ವನ್ನು
ಗುಪ್ತದಾ-ಯಿನೋ
ಗುರವೇ
ಗುರಿ-ಯಾಗಿಟ್ಟು-ಕೊಂಡು
ಗುರಿ-ಯಾಗಿದ್ದಾನೆಯೋ
ಗುರಿ-ಯಾದೆವು
ಗುರು
ಗುರುಂ
ಗುರುಃ
ಗುರು-ಕುಲೇ
ಗುರು-ಗಮನಾದಿ
ಗುರು-ಗಳ
ಗುರು-ಗಳನ್ನು
ಗುರು-ಗಳನ್ನೂ
ಗುರು-ಗಳನ್ನೇ
ಗುರು-ಗಳಲ್ಲಿ
ಗುರು-ಗಳಲ್ಲಿಯೂ
ಗುರು-ಗಳಾದ
ಗುರು-ಗಳಿಂದ
ಗುರು-ಗಳಿಂದಲೇ
ಗುರು-ಗಳಿಗೆ
ಗುರು-ಗಳು
ಗುರು-ಗಳೂ
ಗುರುಣಾ
ಗುರು-ಣೋದಿತಃ
ಗುರು-ತಲ್ಪಗ-ಪಾಪಂ
ಗುರು-ತಲ್ಪ-ಗಮ್
ಗುರು-ತು-ಗಳನ್ನೇ
ಗುರು-ದಕ್ಷಿ-ಣಾಮ್
ಗುರು-ದಕ್ಷಿಣೆ-ಯನ್ನು
ಗುರು-ದತ್ತಂ
ಗುರುದ್ರವ್ಯಾಪಹಾರಿಣ-ವಮ್
ಗುರು-ನಿಂದ-ನೆಗೆ
ಗುರು-ನಿಂದಾ
ಗುರು-ನಿಂದಾ-ಕೃತಾತ್
ಗುರು-ನಿಂದಾ-ಸಮಂ
ಗುರು-ನಿಂದೆಯ
ಗುರು-ನಿಂದೆ-ಯಿಂದ
ಗುರು-ಪತ್ನಿ
ಗುರು-ಪತ್ನಿ-ಗಮನ
ಗುರು-ಪುತ್ರನ
ಗುರು-ಪುತ್ರ-ನನ್ನು
ಗುರು-ಪುತ್ರನು
ಗುರು-ಪುತ್ರಸ್ತು
ಗುರು-ಪುತ್ರೋ-ಽಪಿ
ಗುರುಪ್ರಸಾದವೇ
ಗುರುಪ್ರಸಾದೋ
ಗುರು-ಭಿರ್ವಿಶಿಷ್ಟೈಃ
ಗುರು-ಭಿಶ್ಚೈವ
ಗುರು-ಮಗ್ರ-ಬುದ್ಧ್ಯಾ
ಗುರು-ಮಾ-ಸಾಧ್ಯ
ಗುರು-ಮೇ-ವಾಭಿ-ಗಚ್ಛೇತ್
ಗುರು-ರಾಜರ
ಗುರು-ರಾಜ-ರಿಂದ
ಗುರುರ್ಮಯಾ
ಗುರುರ್ವಾ
ಗುರು-ವನ್ನು
ಗುರು-ವಾರ-ದಿಂದ
ಗುರು-ವಿನ
ಗುರು-ಶುಶ್ರೂಷಣೇ
ಗುರು-ಸುತಾ
ಗುರು-ಸೇವಾ
ಗುರು-ಸೇವಾ-ದಿಂದ
ಗುರು-ಸೇವೆ-ಯೆಂಬುದು
ಗುರುತ್ರೀ-ಯರ
ಗುರೂನ್
ಗುರೂಪ-ದಿಷ್ಟ-ವಾದ
ಗುರೋಃ
ಗುರೋರ್ವಿಂದಾ
ಗುರೌ
ಗುರೌ-
ಗುರ್ವಂಗನಾ-ಕೋಟಿ
ಗುರ್ವನುಜ್ಞ
ಗುರ್ವನುಜ್ಞಯಾ
ಗುಲ್ಮಮದ್ವೇಷು
ಗುಲ್ಮ-ಮೂಲೇ
ಗುಹಾಶಯಃ
ಗುಹಾಶ-ಯಾತ್
ಗುಹೆಯ
ಗುಹೆ-ಯಲ್ಲಿ
ಗುಹೆ-ಯಿಂದ
ಗುಹ್ಯ
ಗುಹ್ಯಂ
ಗುಹ್ಯ-ಕ-ನಂತೆ
ಗುಹ್ಯ-ಕ-ನನ್ನು
ಗುಹ್ಯ-ಕ-ನಿಗೆ
ಗುಹ್ಯ-ಕನು
ಗುಹ್ಯ-ಕ-ನೆಂಬು-ವನು
ಗುಹ್ಯ-ಕನೇ
ಗುಹ್ಯಕೋ
ಗೂಢ
ಗೂಬೆ-ಗಳಿಗೆ
ಗೃಂಜನಂ
ಗೃಂಜ-ನಮ್
ಗೃತ್ಸಮದ
ಗೃತ್ಸಮದಂ
ಗೃತ್ಸಮದಃ
ಗೃತ್ಸಮದ-ರೆಂಬ
ಗೃತ್ಸ-ವದ
ಗೃಧ್ರಶಿ-ರಸಃ
ಗೃಧ್ರಶೀರ್ಷಾ
ಗೃರುಕೃಪೋದಯೇ
ಗೃಹಂ
ಗೃಹ-ಕೃತ್ಯದ
ಗೃಹ-ಪತಿಸ್ತೀರ್ಥ-ಯಾತ್ರಾರ್ಥಮಧುನಾ
ಗೃಹಮಾ-ಗತಃ
ಗೃಹ-ಮಿದಂ
ಗೃಹಮ್
ಗೃಹ-ವಮ್
ಗೃಹಸ್ಥ
ಗೃಹಸ್ಥರ
ಗೃಹಸ್ಥರು
ಗೃಹಸ್ಥಾನಾಂ
ಗೃಹಸ್ಥಾಶ್ರಮ-ವನ್ನು
ಗೃಹಸ್ಥೋ
ಗೃಹಾಣಾಂ
ಗೃಹಾಣಾರ್ಘ್ಯಂ
ಗೃಹಾರಾಮೇ
ಗೃಹಾಶ್ರಮ-ಮುಪೇಯಿ-ವಾನ್
ಗೃಹಿತಾ
ಗೃಹೀತಂ
ಗೃಹೀತ-ಮತುಲಂ
ಗೃಹೀತ್ಯಾ
ಗೃಹೀತ್ವಾ
ಗೃಹೀತ್ವಾಂತರಧೀಯತ
ಗೃಹೀತ್ವಾಪಿ
ಗೃಹೇ
ಗೃಹೇಷ್ಟಗ್ನಿ
ಗೃಹ್ಯತೇ
ಗೆಜೆಟೆಡ್
ಗೆದ್ದ-ವನೂ
ಗೆದ್ದು
ಗೆಲ್ಲಲು
ಗೇಹಸ್ಯ
ಗೇಹೇ
ಗೊತ್ತಿದೆ
ಗೊತ್ತಿಲ್ಲ
ಗೊತ್ತಿಲ್ಲ-ವೆಂದು
ಗೊಳಿ-ಸುವ
ಗೋ
ಗೋಕರ್ಣ
ಗೋಕರ್ಣೆ
ಗೋಘ್ನಂ
ಗೋಚರ್ಮ-ಮಾತ್ರಂ
ಗೋಡೆಯ
ಗೋತ್ರಜಃ
ಗೋತ್ರ-ದಲ್ಲಿ
ಗೋತ್ರ-ದವ-ನಾಗಿ
ಗೋತ್ರ-ದ-ವನು
ಗೋದಾಂ
ಗೋದಾನ
ಗೋದಾನ-ದೀತಟೇ
ಗೋದಾ-ನಸ್ಯ
ಗೋದಾಯಾಂ
ಗೋದಾ-ವರಿ
ಗೋದಾ-ವರಿ-ಯಲ್ಲಿ
ಗೋದಾ-ವರೀ
ಗೋದಾ-ವರೀಂ
ಗೋದಾ-ವರೀ-ನದಿಗೆ
ಗೋದಾ-ವರೀ-ಸಂಗೇ
ಗೋಪಾಲ-ಕರು
ಗೋಪಿಚಂದ-ನ-ದಿಂದ
ಗೋಪಿಚಂದನಲಿಪ್ತಾಂಗಾ
ಗೋಪೀಚಂದನ
ಗೋಪೀಚಂದನ-ದಿಂದ
ಗೋಪೀಥ
ಗೋಪೇಷು
ಗೋಪ್ಯಂ
ಗೋಮತೀ
ಗೋಮತೀಂ
ಗೋಮತೀ-ನದಿ-ಯಲ್ಲಿ
ಗೋಮತೀ-ಫಲಂ
ಗೋಮತ್ಯಾಂ
ಗೋಮ-ಯ-ದಿಂದ
ಗೋಮಯ-ವನ್ನು
ಗೋಮ-ಯೇನ
ಗೋಮ-ಯೇನೋಪಲಿಪ್ಯಾದ್ಯೋ
ಗೋಮ-ಯೇನೋಸ-ಲಿಪ್ಯ
ಗೋಮಾಂಸ
ಗೋಮಾಂಸ-ಭಕ್ಷ-ಣಮ್
ಗೋಮಾಯವೋ
ಗೋಮೇಧ
ಗೋಲಕವೇ
ಗೋವಧಾದಿ
ಗೋವನ್ನು
ಗೋವಿಂದ
ಗೋವಿಂದಾಚ್ಯುತ-ಮಾಧವ
ಗೋವಿ-ನಂತೆ
ಗೋವು
ಗೋವು-ಗಳಲ್ಲಿಯೂ
ಗೋಷು
ಗೋಷ್ಠೇ
ಗೋಷ್ಯಂ
ಗೋಷ್ಯತ್ವಾನ್ನೋದಿತಂ
ಗೋಸವಪ್ರದಾ
ಗೋಸಹಸ್ರ-ಫಲಂ
ಗೋಹತ್ಯ
ಗೋಹತ್ಯೆ
ಗೌಡ-ದೇಶ-ದಲ್ಲಿ
ಗೌಡ-ದೇಶೇ
ಗೌತಮರ
ಗೌತಮಸ್ಯ
ಗೌತಮಸ್ಯಾಘ-ನಾಶಿನಿ
ಗೌರವ
ಗೌರವ-ದಿಂದ
ಗೌರವ-ವೆಂಬ
ಗೌರವಿ-ಸ-ಲಿಲ್ಲ
ಗೌರವಿಸಲ್ಪಡುತ್ತಾಳೆ
ಗೌರ್ಯಥಾ
ಗ್ರಂಥ
ಗ್ರಂಥ-ಗಳ
ಗ್ರಂಥ-ಗಳನ್ನು
ಗ್ರಂಥ-ಗಳಲ್ಲಿ
ಗ್ರಂಥದ
ಗ್ರಂಥ-ದಲ್ಲಿ
ಗ್ರಂಥ-ದಿಂದ
ಗ್ರಂಥ-ವನ್ನು
ಗ್ರಂಥವು
ಗ್ರಂಥಾಂತರ-ಗಳಿಂದ
ಗ್ರಂಥಿ-ಗಳನ್ನು
ಗ್ರಸದ್ಭ್ಯಾಂ
ಗ್ರಹ-ಗತಿ-ಗಳು
ಗ್ರಹ-ಗಳೂ
ಗ್ರಹಣ
ಗ್ರಹಣ-ಗಳಷ್ಟು
ಗ್ರಹಣ-ಗಳಿ-ಗಿಂತ
ಗ್ರಹಣ-ಗಳಿ-ಗಿಂತಲೂ
ಗ್ರಹಣ-ಗಳೂ
ಗ್ರಹಪೀಡಾಂ
ಗ್ರಹಪೀಡಾದಿಭಿಃ
ಗ್ರಹಪೀಡೆ-ಗಳಿಂದ
ಗ್ರಹಪೀಡೆ-ಗಳೂ
ಗ್ರಹಾಣಾಂ
ಗ್ರಹಾಶ್ಚ
ಗ್ರಹಾಶ್ಚೈನಂ
ಗ್ರಹಿಸ-ಬೇಕೆಂದು
ಗ್ರಹಿ-ಸುವನೋ
ಗ್ರಹೇ
ಗ್ರಾಮಂ
ಗ್ರಾಮಕ್ಕೆ
ಗ್ರಾಮಜ್ಯೋತಿಷಃ
ಗ್ರಾಮದ
ಗ್ರಾಮದಲ್ಲಿದ್ದನು
ಗ್ರಾಮಸ್ಥಂ
ಗ್ರಾಮಸ್ಥರು
ಗ್ರಾಮಾಂತರಂ
ಗ್ರಾಮಾದ್ಬಹಿರ್ಜಲೇ
ಗ್ರಾಮಾಧಿ-ಪತೇಃ
ಗ್ರಾಮಿಣ್ಯ
ಗ್ರಾಮೀ
ಗ್ರಾಮೇ
ಗ್ರಾಮೋ
ಗ್ರಾಹ್ಯ-ಮಿತಿ
ಗ್ರೀಷ್ಮ
ಗ್ರೀಷ್ಮೇ
ಘಂಟಾ-ನಾದಂ
ಘಂಟಾ-ನಾದ-ವಿಲ್ಲದೆ
ಘಂಟಾ-ನಾದ-ವಿ-ಹೀನಂ
ಘಂಟೆ-ಯನ್ನು
ಘರ್ಮಃ
ಘಳಿಗೆ-ಯನ್ನು
ಘಾತ-ದಿಂದ
ಘೃತಂ
ಘೃತಕೋಶಾತಕೀ
ಘೃತ-ದಿಂದ
ಘೃತ-ಯುಕ್ತೇನ
ಘೃತಸಾಚಿ-ತಾನ್
ಘೃತೇನ
ಘೋರ
ಘೋರಂ
ಘೋರ-ವಾದ
ಘೋರಾಂ
ಘೋರಾಃ
ಘೋರಾನ್
ಘೋರೇ
ಘೋರೈಃ
ಘೋಷ-ಗಳೂ
ಘ್ರಾಣಸ್ವ-ರೂಪಾ-ವಶ್ಚಿನ್
ಘ್ರಾಣೇಂದ್ರಿ-ಯಕ್ಕೆ
ಚ
ಚಂಚಲ
ಚಂಚಲಂ
ಚಂಡಕ-ನೆಂಬ
ಚಂಡಕೋ
ಚಂಡಾಲ-ನಾಗಿ
ಚಂಡಾಲ-ರಾ-ಗುತ್ತಾರೆ
ಚಂಡಾಲ-ರಿಂದ
ಚಂದನೋನ್ಮಿಶ್ರಂ
ಚಂದ್ರ
ಚಂದ್ರ-ನಂತೆ
ಚಂದ್ರನು
ಚಂದ್ರನೂ
ಚಂದ್ರನೇ
ಚಂಪ-ಕೇಶ್ವರ-ನೆಂಬ
ಚಂಪಾ-ಪುರಿ-ಯಲ್ಲಿ
ಚಂಪಾ-ಪುರಿ-ಯಲ್ಲಿದ್ದ
ಚಂಪಾ-ಪುರೀಂ
ಚಂಪಾಯಾಂ
ಚಂಪಾ-ವತೀ
ಚಂಪಾ-ವತ್ಯಾಂ
ಚಕಾರ
ಚಕಾರಾಧ್ಯ
ಚಕಾರೈವಂ
ಚಕೃವಹೇ
ಚಕ್ರಂ
ಚಕ್ರತುರ್ಮಾಘ-ಮಾಸೇ
ಚಕ್ರತುರ್ಮುಹಾವ್ಯಾ-ಧಿಪೀಡಿ-ತಸ್ಯ
ಚಕ್ರ-ಪಾಣಿ-ಯಾದ
ಚಕ್ರ-ವರ್ತಿ-ಗಳು
ಚಕ್ರ-ವರ್ತಿ-ಗಳೂ
ಚಕ್ರ-ವರ್ತಿ-ಯಾಗಿ
ಚಕ್ರ-ವರ್ತಿ-ಯಾಗಿ-ರುತ್ತಾನೆ
ಚಕ್ರ-ವರ್ತಿಯು
ಚಕ್ರವೇ
ಚಕ್ರಾದೀನಿ
ಚಕ್ರಿಣಃ
ಚಕ್ಷು-ರಿಂದ್ರಿ-ಯಕ್ಕೆ
ಚಕ್ಷು-ರಿಂದ್ರಿಯ-ದೇ-ವತಾ
ಚಕ್ಷುಷೀ
ಚಣಕಾನ್
ಚತಸ್ರಸ್ತಿಥಯೊ
ಚತುರ್ಗುಣಮ್
ಚತುರ್ಥೇ
ಚತುರ್ಥೋ
ಚತುರ್ಥೋಧ್ಯಾಯಃ
ಚತುರ್ದಶ
ಚತುರ್ದಶ-ಭು-ವನಕ್ಕೂ
ಚತುರ್ದಶಸು
ಚತುರ್ದಶಸ್ವನಿರ್ವಚ್ಯಾಂ
ಚತುರ್ದಶಿ-ಯಲ್ಲಿ
ಚತುರ್ದಶೀ
ಚತುರ್ದಶೋಧ್ಯಾಯಃ
ಚತುರ್ದಶ್ಯಾಂ
ಚತುರ್ಬಾಹೂನ್
ಚತುರ್ಮುಖ
ಚತುರ್ಮುಖಬ್ರಹ್ಮ-ದೇವ-ರ-ವರೆವಿಗೂ
ಚತುರ್ಮುಖಬ್ರಹ್ಮ-ದೇವರೂ
ಚತುರ್ವಿಂಶತಿಪ್ರಸ್ಥಂ
ಚತುರ್ವಿಧ
ಚತ್ವಾರಿ
ಚತ್ವಾ-ರಿಂಶತ್ಸಮಾಯುಷ್ಯಮಾವಿಶಿಷ್ಟಮ-ಭೂತ್ತದಾ
ಚದರಿ
ಚಪಲಂ
ಚಪಲ-ಚಿತ್ತ-ನಾದ
ಚಪಲ-ದಿಂದ
ಚಪ್ಪರ-ದಲ್ಲಿ
ಚಮತ್ಕಾರ-ದಿಂದ
ಚರಂತಶ್ಚ
ಚರಂತಿ
ಚರಂತೀ
ಚರಂತೌ
ಚರನ್
ಚರಾಚರಂ
ಚರಾಚರಾತ್ಮಕ-ವಾದ
ಚರಿತಂ
ಚರಿತುಂ
ಚರಿ-ತೇನ
ಚರಿತ್ರೆ-ಗಳನ್ನು
ಚರಿತ್ರೆ-ಯನ್ನು
ಚರಿತ್ರೆಯೆಲ್ಲ-ವನ್ನೂ
ಚರೇತ್
ಚರ್ಚೆ-ಗಳನ್ನು
ಚರ್ತುರ್ಮಾಘೇಷು
ಚರ್ಮ
ಚರ್ಮ-ಕಾರ-ಕಮ್
ಚರ್ಮ-ಕೇಷು
ಚರ್ಮಣ್ವತೀ
ಚರ್ಮಣ್ವತೀ-ಜಲೇ
ಚರ್ಮದ
ಚರ್ಮ-ವನ್ನು
ಚಲಂತಿ
ಚಲ-ಜಿಹ್ವಂ
ಚಲತಿ
ಚಲ-ದೋಷ್ಠೋ
ಚಲನವಲ-ನಾದಿ
ಚಲನೇ
ಚಲಮ್
ಚಲಿ-ಸುತ್ತಿದ್ದೆ
ಚಳಿಗಾಲ-ದಲ್ಲಿ
ಚಳಿಗೆ
ಚಳಿ-ಯನ್ನು
ಚಳಿ-ಯಿಂದ
ಚಳಿವು
ಚವಳೇ-ಕಾಯಿ
ಚಾಂಜಸಾ
ಚಾಂಡಗೋಚರಾಃ
ಚಾಂಡಾಲಃ
ಚಾಂಡಾಲತಾಮೇತಿ
ಚಾಂಡಾಲ-ನಾಗಿ
ಚಾಂಡಾಲನು
ಚಾಂಡಾ-ಲಯೋ-ನಿಷು
ಚಾಂಡಾಲತ್ರೀ-ಯನ್ನೂ
ಚಾಂಡಾಲೋ
ಚಾಂತಃ
ಚಾಂದ್ರಾ-ಯಣ
ಚಾಂದ್ರಾ-ಯಣಂ
ಚಾಂದ್ರಾ-ಯಣ-ಶತಾಧಿ-ಕಮ್
ಚಾಂದ್ರೇ
ಚಾಂಧ್ರ-ದೇಶೇಷು
ಚಾಂಭಸಾ
ಚಾಂಭೋ
ಚಾಕ್ಷುಷಮನ್ವಂತರದ
ಚಾಕ್ಷುಷಮನ್ವಂತೇ
ಚಾಗಚ್ಛ
ಚಾಜ್ಞಾ
ಚಾಡಿ
ಚಾತುರ್ಮಾಸ
ಚಾತುರ್ಮಾಸ-ಗಳಲ್ಲಿ
ಚಾತುರ್ಮಾಸೇ
ಚಾತುರ್ಮಾಸ್ಯಾಂ
ಚಾದಾಯ
ಚಾದೈ-ವತೋ
ಚಾನುಗೈಃ
ಚಾನೈ
ಚಾನ್ನಮನ್ಯತ್ಪ್ರ-ದೀ-ಯತೇ
ಚಾಪರೇ
ಚಾಪಿ
ಚಾಪ್ನು-ಯಾತ್
ಚಾಪ್ನೋತಿ
ಚಾಪ್ಯ-ದತ್ವಾ
ಚಾಪ್ಯಹಂ
ಚಾಬ್ರವೀತ್
ಚಾಭಿಧಃ
ಚಾಭ್ಯರ್ಥಿತೋ
ಚಾಭ್ಯು-ದಿತೇ
ಚಾಮುಂಡೀ
ಚಾಮುಂಡೀ-ದೇವ-ತಾಭಕ್ತೋ
ಚಾಯಂ
ಚಾಲಾಪಾಃ
ಚಾಲೋಡಿತಾಃ
ಚಾಲ್ಪಾಯುರಸ್ಮಾನ್ಪಾಪಾನ್ಮೃ
ಚಾವ-ದಮ್
ಚಾವಯೋಃ
ಚಾವಯೋ-ರಾಶ್ರಮೋ-ಭ-ವತ್
ಚಾವಯೋಸ್ತಾತ
ಚಾಶನಿಪಾ-ತೇನ
ಚಾಶ್ರಮೇನ
ಚಾಷ್ಟಮ್ಯಾಂ
ಚಾಸೀತ್
ಚಾಸೀತ್ಕಾಲಃ
ಚಾಸೀನ್ನ
ಚಾಸ್ತಿ
ಚಾಸ್ಮಾಕಂ
ಚಾಸ್ಮಾಭಿಃ
ಚಾಸ್ಮಾಭಿರ್ಗರ್ಭಿಣ್ಯೋ
ಚಾಸ್ಮಿನ್
ಚಾಹ
ಚಾಹ-ರಾಮಿ
ಚಾಹಾರಂ
ಚಿಂತನಾ
ಚಿಂತನೆ
ಚಿಂತಯಾ-ಮಾಸ
ಚಿಂತಾ-ಕುಲಾ
ಚಿಂತಾ-ಕುಲಿತ-ಮಾನಸಃ
ಚಿಂತಾ-ಕುಲೇ
ಚಿಂತಾಕ್ರಾಂತ-ನಾಗಿ
ಚಿಂತಿಸಿ-ದರು
ಚಿಂತೆಯಾಗುತ್ತಿತ್ತು
ಚಿಂತೆ-ಯಿಂದ
ಚಿಂತೆ-ಯಿಂದಲೇ
ಚಿಂತೆ-ಯಿಲ್ಲ
ಚಿಕಿತ್ಸಾಮೌಷಧೀರ್ದಿವ್ಯೈಸ್ತಥಾ
ಚಿಕ್ಕ
ಚಿಕ್ಕಮ್ಮನ
ಚಿಕ್ಷೇಪಾಂತರ್ವಿಲೀನಾಂ
ಚಿಗುರಿದ್ದವು
ಚಿಗುರೆಲೆ-ಗಳಿಂದ
ಚಿಚ್ಛೇದ
ಚಿತ್ತಂ
ಚಿತ್ತೇನ
ಚಿತ್ರ
ಚಿತ್ರ-ಕೂಟ
ಚಿತ್ರ-ಕೇತು
ಚಿತ್ರ-ಗುಪ್ತನು
ಚಿತ್ರ-ಗುಪ್ತೋ
ಚಿತ್ರ-ರಥನ
ಚಿತ್ರ-ರಥ-ನಿಗೆ
ಚಿತ್ರ-ರಥನು
ಚಿತ್ರ-ರಥಸ್ಯಾಥ
ಚಿತ್ರ-ರಥಾಯ
ಚಿದ್ಧ್ವಾಂತ-ದಿವಾ-ಕರಾಯ
ಚಿದ್ರೂಪೀ
ಚಿರಂ
ಚಿರ-ಕಾಲ
ಚಿಹ್ನೆ-ಗಳನ್ನು
ಚಿಹ್ನೆ-ಗಳಿಂದಲೂ
ಚೀರ್ಣೆಷು
ಚುಂಬಿಸುತ್ತಿದ್ದುವು
ಚೂತಪಲ್ಲವೈರ್ವಿಷ್ಣೋಸ್ತೋ-ರಣಂ
ಚೆಂಡಾಟ-ವನ್ನು
ಚೆಟ್ಟ
ಚೆನ್ನಾಗಿ
ಚೆನ್ನಾಗೆ
ಚೇತನನ
ಚೇತನರ
ಚೇತನ-ರಲ್ಲಿಯೂ
ಚೇತನ-ರಿಗೂ
ಚೇತನರು
ಚೇತಶ್ಚ
ಚೇತಸಃ
ಚೇತಸಾ
ಚೇತಸಾಂ
ಚೇತಿ
ಚೇತಿ-ಹಾಸ-ಕಮ್
ಚೇತ್
ಚೇತ್ತದಾ
ಚೇದಂ
ಚೇದಸ್ತಿ
ಚೇದಿ
ಚೇದೀನಾಂ
ಚೇದ್ದಶಮ್ಯಾಂ
ಚೇದ್ವಿಧವಾ
ಚೇರತುಃ
ಚೈ
ಚೈಕಂ
ಚೈಕಾ
ಚೈಕಾ-ದಶೀವ್ರತಮ್
ಚೈತದಾತ್ಮ-ಕಮ್
ಚೈತಾನಿ
ಚೈತ್ರ-ರಥನ
ಚೈತ್ರ-ರಥಸ್ಯ
ಚೈವ
ಚೈವಾ-ಸೀತ್
ಚೋಕ್ತಾ
ಚೋಕ್ತಾನಿ
ಚೋಗ್ರಾಣಿ
ಚೋಚುರ್ನೇತಿ
ಚೋಚ್ಚೈಃ
ಚೋತ್ಥಾಯ
ಚೋತ್ಪನ್ನೋ
ಚೋದನಾ
ಚೋದಿತಂ
ಚೋದಿ-ತಮ್
ಚೋದಿತಾ
ಚೋದಿತಾಃ
ಚೋನ್ಮಾದೀ
ಚೋಪನೀತೋ
ಚೋಪ-ಪತಿಂ
ಚೋಪ-ವನೇ
ಚೋರಗಾಃ
ಚೋರಾಃ
ಚೋರಾನ್
ಚೋರೇಭ್ಯಶ್ಚ
ಚೋರೇಭ್ಯೋ
ಚೌರ್ಯ-ಧರ್ಮೌ
ಚೌರ್ಯ-ಮಂತ್ರೋಷಧಾದ್ಯಾಶ್ಚ
ಚ್ಛೇದ-ಸಂಜ್ಞ
ಛಂದಸಾಂ
ಛಂದಸ್ಸು
ಛಂದಸ್ಸು-ಗಳಲ್ಲಿ
ಛಂದಸ್ಸು-ಗಳು
ಛಂದಾಂಸಿ
ಛತ್ರಂ
ಛತ್ರೀ
ಛತ್ರೀ-ದಂಡ-ಗಳನ್ನು
ಛಾಯಾ
ಛಾಯಾ-ಗುಹ-ವೆಂಬ
ಛಾಯಾ-ಗುಹಾಹ್ವಯೇ
ಛಾಯಾಯಾಂ
ಛಾಯಾ-ಹೀನೇ
ಛಿತ್ವಾ
ಛಿತ್ವಾ-ವಯೋ-ಽಸ್ತ-ತಾದಾಯ
ಛಿನ್ನ-ರೂಪಸ್ತು
ಜಂತುಂ
ಜಂತುಃ
ಜಂತುರ್ನ
ಜಂತೂನಾಂ
ಜಂತೂ-ನಾಮಜ್ಞಾನಾಂ
ಜಂತೂ-ನಾಮ-ವಿವೇಕಿ-ನಾಮ್
ಜಂತೂನ್
ಜಂಬೂಕೋ
ಜಗತ್
ಜಗತ್ಕಾರ್ಯ-ಮತಂದ್ರಿತಃ
ಜಗತ್ಕ್ರಿಯಾಃ
ಜಗತ್ತಿಗೆ
ಜಗತ್ತಿನ
ಜಗತ್ತಿ-ನಲ್ಲಿ
ಜಗತ್ತು
ಜಗತ್ತೆಂಬ
ಜಗತ್ಪತೇ
ಜಗದಾಧಾರ
ಜಗದೀಶ-ನಾದ
ಜಗದೀಶ್ವರ-ನೆಂಬ
ಜಗದೇ-ತನ್ನಿರೀಶ್ವರಮ್
ಜಗದೊಡೆ-ಯನೇ
ಜಗದ್ಗುರು
ಜಗದ್ಬೀಜಂ
ಜಗದ್ಬೀಜಪ್ರಧಾ-ನಸ್ಯ
ಜಗಳ-ವಾಡಿ
ಜಗಾದ
ಜಗಾನ
ಜಗಾಮ
ಜಗಾಮು
ಜಗುಃ
ಜಗೃಹೇ
ಜಗ್ಮತುಃ
ಜಗ್ಮುಃ
ಜಗ್ರಾಹ
ಜಜ್ಞೇ
ಜಟಾಧಾರಿಯು
ಜಟಿಲಃ
ಜಟೆ-ಯನ್ನು
ಜಟೋದ್ಭವೇ
ಜಠರಾಗ್ನಯೇ
ಜಡ-ಜನ್ಮ-ಸಾಧಕಃ
ಜಡ-ದಂತೆ
ಜಡ-ಮತಿಂ
ಜಡ-ವಸ್ತು-ಗಳಂತೆ
ಜಡ-ವಾದ
ಜಡಾತ್ಮ-ನಾಮ್
ಜತುನಾ
ಜತೆ
ಜತೆ-ಯಲ್ಲಿ
ಜತೆ-ಯಲ್ಲಿಯೇ
ಜನ
ಜನಃ
ಜನ-ಕ-ರಾಜ
ಜನ-ಕಸ್ಯ
ಜನ-ಗಳಲ್ಲಿ
ಜನ-ಗಳಿಂದ
ಜನನ
ಜನ-ನ-ನನನ-ರೂಪ-ವಾದ
ಜನ-ನ-ಮ-ರಣ-ರೂಪ-ವಾದ
ಜನಪ್ರಭೃತಿ
ಜನ-ಯಂತಿ
ಜನರ
ಜನ-ರನ್ನು
ಜನ-ರನ್ನೂ
ಜನ-ರ-ಮೇಲೆ
ಜನ-ರಲ್ಲಿ
ಜನ-ರಿಂದ
ಜನ-ರಿಗೂ
ಜನ-ರಿಗೆ
ಜನ-ರಿ-ರುವ
ಜನ-ರಿಲ್ಲ-ದಂತೆ
ಜನರು
ಜನರೂ
ಜನ-ರೆಲ್ಲರೂ
ಜನ-ವರ್ಜಿತೇ
ಜನ-ಸಂಚಾರ-ವಿಲ್ಲದ
ಜನ-ಸಾ-ಮಾನ್ಯ-ರಿಂದ
ಜನಸ್ಯ
ಜನಾಂಸ್ತಾ-ನಿತಿ
ಜನಾಃ
ಜನಾಧ್ಯಕ್ಷಪ್ರಜಾ-ಧೀನಾ
ಜನಾನಾ-ಹೂಯ
ಜನಾನ್
ಜನಾರ್ದನ
ಜನಾರ್ದನಃ
ಜನಾಶ್ಚ
ಜನಿಃ
ಜನಿತ್ರೀ
ಜನಿಸಿ-ದರು
ಜನಿ-ಸಿದೆ
ಜನಿ-ಸಿದೆವು
ಜನಿಸ್ತಥಾ
ಜನೇ
ಜನೈಃ
ಜನೈ-ರೇವ
ಜನೈರ್ಭುವಿ
ಜನ್ಮ
ಜನ್ಮ-ಕೋಟಿ-ಶ-ತೇಷು
ಜನ್ಮ-ಕೋಟಿ-ಶತೈರಪಿ
ಜನ್ಮ-ಕೋಟಿ-ಸಹಸ್ರೇಷು
ಜನ್ಮಕ್ಕೆ
ಜನ್ಮ-ಗಳ
ಜನ್ಮ-ಗಳನ್ನು
ಜನ್ಮ-ಗಳಲ್ಲಿ
ಜನ್ಮ-ಗಳಲ್ಲಿಯೂ
ಜನ್ಮ-ಗಳಿಗೆ
ಜನ್ಮ-ಗಳು
ಜನ್ಮ-ಜಾ-ತಮ್
ಜನ್ಮದ
ಜನ್ಮ-ದಲ್ಲಿ
ಜನ್ಮ-ದಲ್ಲಿಯೂ
ಜನ್ಮ-ದಿಂದ
ಜನ್ಮ-ದಿವ-ಸ-ಗಳಲ್ಲಿ
ಜನ್ಮನಃ
ಜನ್ಮನಿ
ಜನ್ಮ-ನೈಕಾ-ದಶೇ
ಜನ್ಮ-ಭೂಮಿಂ
ಜನ್ಮ-ಮೃತ್ಯು
ಜನ್ಮ-ವನ್ನು
ಜನ್ಮ-ವಾರೇಷ್ವನ್ಯೇಷು
ಜನ್ಮವು
ಜನ್ಮ-ವೆತ್ತಲು
ಜನ್ಮ-ಶತೈರಪಿ
ಜನ್ಮಸು
ಜನ್ಮಾಂತರ-ಕೃತಾನಿ
ಜನ್ಮಾಂತರ-ಗಳಲ್ಲಿಯೂ
ಜನ್ಮಾಂತರ-ಗಳಿಗೆ
ಜನ್ಮಾಂತರ-ದಲ್ಲಿ
ಜನ್ಮಾಂತರೇಷು
ಜನ್ಮಾ-ಸಾದ್ಯ
ಜನ್ಯನ್ಯಪಿ
ಜಪ
ಜಪನ್ನದೀ-ಮಾಪ
ಜಪ-ವನ್ನಾಗಲೀ
ಜಪಿಸಿ
ಜಪಿಸು
ಜಪಿ-ಸುವ
ಜಪ್ತಂ
ಜಪ್ತ್ವಾ
ಜಮದಗ್ನಿ
ಜಮದಗ್ನಿ-ಕುಲೋತ್ಪನ್ನೋ
ಜಮದಗ್ನಿ-ಗೋತ್ರ-ದ-ವರು
ಜಯ
ಜಯಂತನೂ
ಜಯಂತಶ್ಚ
ಜಯಂತಿ
ಜಯಂತಿ-ಗಳನ್ನು
ಜಯಜ್ಞಾ-ದಿವ
ಜಯತಃ
ಜಯತಿ
ಜಯತೇ
ಜಯತ್ಯಮಿತಸದ್
ಜಯಶೀಲ-ನಾಗಿಯೂ
ಜಯಶೀಲೋ
ಜಯಾಜಯೌ
ಜಯಾಪಜಯ
ಜಯಿ-ಸಲು
ಜಯಿಸಿ
ಜಯಿ-ಸುತ್ತಾನೆ
ಜಯಿ-ಸುವ
ಜಯೇ
ಜರಠೋಸ್ಮ್ಯ
ಜರಠ್ಯೌ
ಜರಾ
ಜರಾ-ಕರ್ಕಶ-ದುಷ್ಟಾಂಗಃ
ಜರಾ-ಭಯಮ್
ಜರಾ-ಮೃತೀ
ಜರಾ-ಮೃತ್ಯೋಸ್ತು
ಜರುಗಿ-ದುವು
ಜಲ
ಜಲಂ
ಜಲ-ತತ್ವಕ್ಕೆ
ಜಲ-ತರ್ಪಣ-ವನ್ನು
ಜಲ-ದಲ್ಲಿ
ಜಲ-ದಾಯಿನಃ
ಜಲ-ದಿಂದ
ಜಲದ್ವಾ-ರೇಣ
ಜಲಧಾರಾ
ಜಲ-ಪಕ್ಷಿ-ಗಳಿಂದ
ಜಲ-ಪಕ್ಷಿ-ಭಿರಾವೃ-ತಮ್
ಜಲಪಾನ
ಜಲಮಯ-ವಾಗಿ-ರುವ
ಜಲ-ಮಾತ್ರಾ-ಶಯಂ
ಜಲ-ಮಾ-ದಾಯ
ಜಲಮಾಶ್ರಿತ್ಯ
ಜಲ-ವಲ್ಲ
ಜಲ-ಶಾಯಿನೇ
ಜಲಸ್ಪರ್ಶ
ಜಲಸ್ಪರ್ಶ-ದಿಂದ
ಜಲಸ್ಪರ್ಶ-ದಿಂದಲೂ
ಜಲಸ್ಪರ್ಶ-ಮಾ-ಚ-ರೇತ್
ಜಲಸ್ಪರ್ಶ-ಮಾತ್ರ-ದಿಂದ
ಜಲಸ್ಪರ್ಶ-ಮಾತ್ರ-ದಿಂದಲೇ
ಜಲಸ್ಪರ್ಶಾತ್
ಜಲಸ್ಪರ್ಶೋ
ಜಲಸ್ಯ
ಜಲಾತ್
ಜಲಾತ್ಮೋ
ಜಲಾವಾಪ್ಯೋ
ಜಲಾಶ-ಯಕ್ಕೆ
ಜಲಾಶಯ-ಗಳ
ಜಲಾಶಯ-ಗಳಲ್ಲಿ
ಜಲಾಶಯ-ಗಳಲ್ಲಿಯೇ
ಜಲಾಶಯ-ಗಳು
ಜಲಾಶಯ-ಗಳೆನಿ-ಸುತ್ತವೆ
ಜಲಾಶ-ಯದ
ಜಲಾಶಯ-ದಲ್ಲಿ
ಜಲಾಶಯಾಃ
ಜಲಾಶಯೇ
ಜಲೇ
ಜವರೋ
ಜವೇಗೋಧಿಯ
ಜಹ್ಯಾನ್ಮಾರೀ
ಜಹ್ರತುರ್ಧನಸಂಚಯಮ್
ಜಾಗರ
ಜಾಗರಂ
ಜಾಗ-ರಣೆ
ಜಾಗ-ರಾದ್ಯಾಸ್ತನ್ಮೂಲಂ
ಜಾಗರೂಕೋ
ಜಾಗೃತ-ನಾಗಿ-ರುತ್ತಾನೆ
ಜಾಗ್ರಾಹ
ಜಾತಂ
ಜಾತಕದ
ಜಾತಕ್ರುಧೌ
ಜಾತಮದ್ಭು-ತಮ್
ಜಾತಮ್
ಜಾತವಿಜ್ಞಪ್ತ್ಯೌ
ಜಾತಸಂಪದಃ
ಜಾತಾ
ಜಾತಾ-ಪೀಯೂಷ
ಜಾತಿ
ಜಾತಿ-ಗಳ
ಜಾತಿ-ಗಳನ್ನು
ಜಾತಿ-ಗಳು
ಜಾತಿಗೆ
ಜಾತಿಯ
ಜಾತಿ-ಯಲ್ಲಿ
ಜಾತಿ-ರಿತರಾ
ಜಾತಿಸ್ಮ-ರತ್ವಂ
ಜಾತು
ಜಾತು-ಚಿಕ್
ಜಾತು-ಚಿತ್
ಜಾತೇ
ಜಾತೋ
ಜಾತೋಸ್ಮಿ
ಜಾತೋಹಂ
ಜಾತೋ-ಽ-ಜಾತೋ
ಜಾತೋ-ಽಸ್ಮಿನ್
ಜಾತೋ-ಽಹಂ
ಜಾತೌ
ಜಾತ್ಯಾಂಧೋ
ಜಾನ-ತಾಪಿ
ಜಾನನ್ವೇದ-ವಿಧಿಂ
ಜಾನಾಸಿ
ಜಾನೀಯಾಚ್ಚ
ಜಾನು-ದೇಶ-ಶಿರಾಃ
ಜಾನು-ಶಿರ
ಜಾನು-ಶಿರ-ನೆಂಬ
ಜಾನೇ
ಜಾಬಾಲಿ
ಜಾಬಾಲೀರ್ನಾಮ-ತಶ್ಚಾ-ಸೀತ್
ಜಾಮಾ-ತರಂ
ಜಾಮಾತಾ
ಜಾಮಾತಾದ್ಯಾನಘ
ಜಾಮಾತುಃ
ಜಾಮಾತ್ರಾ
ಜಾಯತೇ
ಜಾಯತೇ-ಽ-ಹಮ್
ಜಾಯೇತ
ಜಾರ-ನಲ್ಲಿ
ಜಾರ-ನಿಂದ
ಜಾರ-ಪುರುಷನೂ
ಜಾರ-ಪುರುಷ-ರನ್ನು
ಜಾರಸ್ನೇಹಾದ್ಗೃಹಂ
ಜಾರಿಣಿ-ಯಾದ
ಜಾರೇಣ
ಜಾವಾತೃಮೃತಿ-ಕಾರ-ಣಮ್
ಜಾಸ್ತಿ
ಜಾಹ್ನವೀ
ಜಾಹ್ನವೀ-ಯಾತ್ರಾ
ಜಿಂಕೆಯ
ಜಿಗ-ವಿಷುಃ
ಜಿಗೀಷುಃ
ಜಿಗೀಷುರ್ವಿಷ-ಯಾನ್
ಜಿಘ್ರತಿ
ಜಿಘ್ರಾತಿ
ಜಿತಾಸ-ನಮ್
ಜಿತೇಂದ್ರಿಯಃ
ಜಿತೇಂದ್ರಿಯ-ರಾದ
ಜಿತೌ
ಜಿತ್ವಾ
ಜಿತ್ವಾ-ಯುರೇಷ್ಯತಿ
ಜಿನಾಂಬರಃ
ಜಿಹ್ನ
ಜಿಹ್ವ
ಜಿಹ್ವ-ನೆಂಬ
ಜಿಹ್ವಸ್ತಥಾ-ಪರಃ
ಜಿಹ್ವಾ
ಜಿಹ್ವಾ-ಚಾಪಲ-ದೊಷೇಣ
ಜಿಹ್ವಾಚ್ಛೇ-ದಾದ್ಗ
ಜಿಹ್ವಾತ್ಮಾ
ಜಿಹ್ವಾ-ಲಾಲ-ಸೇನ
ಜಿಹ್ವಾಸ್ತು
ಜಿಹ್ವೇಂದ್ರಿ-ಯಕ್ಕೆ
ಜಿಹ್ವೋಽಸ್ಮಿ
ಜೀವ
ಜೀವಂ
ಜೀವಃ
ಜೀವಕ್ರಿಯಾಸು
ಜೀವತೋ
ಜೀವತ್ಪತಿ-ಪುತ್ರಾಢ್ಯಾ
ಜೀವನ
ಜೀವನಂ
ಜೀವ-ನಕ್ಕಾಗಿ
ಜೀವ-ನದ
ಜೀವ-ನ-ಪರ್ಯಂತವೂ
ಜೀವ-ನಲ್ಲಿ
ಜೀವ-ನ-ವನ್ನು
ಜೀವ-ನವು
ಜೀವ-ನಿಂದ
ಜೀವ-ನಿಗೂ
ಜೀವ-ನಿಗೆ
ಜೀವನು
ಜೀವನೂ
ಜೀವ-ನೇಚ್ಛುಶ್ಚ
ಜೀವನ್ನೇಕ-ದಿನೇನ
ಜೀವನ್ನೇವ
ಜೀವನ್ನೈವ
ಜೀವನ್ಮುಕ್ತ-ನಾಗುವನು
ಜೀವರ
ಜೀವ-ರಲ್ಲಿ
ಜೀವ-ರಿಂದ
ಜೀವರು
ಜೀವರೂ
ಜೀವರೇ
ಜೀವಸ್ಯ
ಜೀವಸ್ವ-ರೂಪಕ್ಕೆ
ಜೀವ-ಹ-ನನಂ
ಜೀವಾ
ಜೀವಾಃ
ಜೀವಾ-ದಿವಿಣ್ವಂತಾಃ
ಜೀವಾಸ್ತೇ
ಜೀವಿ-ಕಾರ್ಥೇ
ಜೀವಿಸುತ್ತಿದ್ದರು
ಜೀವೈಃ
ಜೀವೋ
ಜೀವೋತ್ತಮ-ರಾದ
ಜುಗುಪ್ಪೆ-ಯನ್ನುಂಟು-ಮಾಡುವ
ಜುಗುಪ್ಸಿ-ತಮ್
ಜುಟ್ಟು-ಗಳನ್ನು
ಜೂಜಾಡು-ವುದು
ಜೂಜಿನ
ಜೃಂಭಚ್ಚಂಪಕಶೋಭಾಢ್ಯೇ
ಜೈ
ಜೈತ್ರ-ನೆಂಬ
ಜೈತ್ರೋ
ಜೈಮಿನೇರ್ಗೋತ್ರ-ಜಶ್ಚಾಹಂ
ಜೋಯಿಸ-ನಾಗಿದ್ದೆ
ಜ್ಞಾ
ಜ್ಞಾತಂ
ಜ್ಞಾತ-ಬುದ್ಧಿಃ
ಜ್ಞಾತ-ವಾನ್
ಜ್ಞಾತ್ವಾ
ಜ್ಞಾತ್ವಾಪಿ
ಜ್ಞಾನ
ಜ್ಞಾನಂ
ಜ್ಞಾನ-ಕರ್ಮಭಿಃ
ಜ್ಞಾನಕ್ಕೆ
ಜ್ಞಾನ-ದಿಂದ
ಜ್ಞಾನ-ದಿಂದಲೇ
ಜ್ಞಾನ-ದುರ್ಬಲ
ಜ್ಞಾನ-ದುರ್ಬಲ-ರಾದ
ಜ್ಞಾನ-ಪುರಸ್ಸರ-ವಾದ
ಜ್ಞಾನ-ಪೂರ್ವ-ಕ-ವಾದ
ಜ್ಞಾನ-ಭಕ್ತಿ-ವೈರಾಗ್ಯ-ನಿಧಿ-ಗಳಾ-ಗಿ-ರುವ
ಜ್ಞಾನ-ಭಕ್ತಿ-ವೈರಾಗ್ಯಾದಿ-ಗಳಿಂದ
ಜ್ಞಾನ-ಭಕ್ತ್ಯಾ-ದಿ-ವೈರಾಗ್ಯ-ಸದ್ದು-ಣಾಬ್ಧಿಂ
ಜ್ಞಾನ-ಮಾಪ್ನು-ಯಾತ್
ಜ್ಞಾನ-ಮಾರ್ಗ
ಜ್ಞಾನ-ಮುತ್ಪಾದ್ಯ
ಜ್ಞಾನ-ಯುಕ್ತ-ನಾಗಿ
ಜ್ಞಾನ-ಯುಕ್ತ-ನಾಗಿದ್ದರೂ
ಜ್ಞಾನ-ಯೋಗಂ
ಜ್ಞಾನ-ಯೋಗ-ಕರ್ಮ-ಯೋಗ-ಗಳನ್ನು
ಜ್ಞಾನ-ಯೋಗಾಭ್ಯಾಸ-ವನ್ನು
ಜ್ಞಾನ-ಯೋಗೊ
ಜ್ಞಾನ-ರ-ಹಿತ-ನಾಗಿ
ಜ್ಞಾನ-ರ-ಹಿತ-ನಾದ
ಜ್ಞಾನ-ರೂಪ-ನಾದ
ಜ್ಞಾನ-ವತೋ
ಜ್ಞಾನ-ವನ್ನು
ಜ್ಞಾನ-ವನ್ನೂ
ಜ್ಞಾನ-ವಾ-ಗಲು
ಜ್ಞಾನ-ವಾದರೋ
ಜ್ಞಾನ-ವಿಜ್ಞಾ-ನ-ಸಂಪದಾ
ಜ್ಞಾನ-ವಿದ್ಯಾಸು
ಜ್ಞಾನ-ವಿಲ್ಲ
ಜ್ಞಾನ-ವಿಲ್ಲದೆ
ಜ್ಞಾನ-ವಿಲ್ಲದೇ
ಜ್ಞಾನವು
ಜ್ಞಾನ-ವುಳ್ಳ-ವನು
ಜ್ಞಾನ-ವೆಂಬ
ಜ್ಞಾನ-ಸುಖ-ಶಕ್ತಿ
ಜ್ಞಾನಾಗ್ನಿಃ
ಜ್ಞಾನಾತ್
ಜ್ಞಾನಾತ್ಮ-ಕನು
ಜ್ಞಾನಾದ್ವಿಷ್ಣುಂ
ಜ್ಞಾನಾಪೇಕ್ಷಿಯ
ಜ್ಞಾನಾಭಿವೃದ್ಧಯೇ
ಜ್ಞಾನಾಭಿ-ವೃದ್ಧಿ
ಜ್ಞಾನಾಭಿ-ವೃದ್ಧಿಗೆ
ಜ್ಞಾನಾರ್ಥ-ವಾಗಿ
ಜ್ಞಾನಿ
ಜ್ಞಾನಿ-ಗಳ
ಜ್ಞಾನಿ-ಗಳಿಂದ
ಜ್ಞಾನಿ-ಗಳು
ಜ್ಞಾನಿ-ಗಳೂ
ಜ್ಞಾನಿಗೆ
ಜ್ಞಾನಿತ್ವ-ಮಾ-ಸಾದ್ಯ
ಜ್ಞಾನಿ-ಯಾಗಿ
ಜ್ಞಾನಿ-ಯಾದ
ಜ್ಞಾನಿ-ಯಾದ-ವನು
ಜ್ಞಾನಿಯು
ಜ್ಞಾನಿಯೂ
ಜ್ಞಾನೀ
ಜ್ಞಾನೇ
ಜ್ಞಾನೇಂದ್ರಿಯ
ಜ್ಞಾನೇಂದ್ರಿಯ-ಗಳು
ಜ್ಞಾನೇಚ್ಛುವು
ಜ್ಞಾನೇನ
ಜ್ಞಾನೋತ್ಪತ್ತಿ
ಜ್ಞಾನೋಪಾ-ಯ-ದಿಂದ
ಜ್ಞಾಪಿಸಿ-ಕೊಂಡು
ಜ್ಞೇಯೇ
ಜ್ಯೋತಿಷ
ಜ್ಯೋತಿಷ-ನಾಗಿದ್ದೆ
ಜ್ಯೋತಿಷಾರ್ಣವ
ಜ್ಯೋತಿಷಾರ್ಣವಃ
ಜ್ಯೋತಿಷಾರ್ಣ-ವನು
ಜ್ಯೋತಿಷಾರ್ಣವ-ನೆಂಬ
ಜ್ಯೋತಿಷ್ಮತೀ
ಜ್ಯೋತಿಷ್ಮತೀ-ಪತೇಃ
ಜ್ವಲನಾ-ಯತೇ
ಜ್ವಲಿತಃ
ಜ್ವಾಲಾಭಿರ್ವಕ್ರದಾರುಭಿಃ
ಜ್ವಾಲೆ-ಯಲ್ಲಿ
ಝೇಂಕಾರ
ಡಂಭಾಚಾರ
ಡಾಂಭಂ
ಡಾಂಭಿಕತ-ನ-ದಿಂದ
ಡಾಂಭಿಕ-ನಾಗಿದ್ದೆ
ಡಾಕಿಣೀ
ಡಾಕಿಣೀ-ಗಣಕ್ಕೆ
ಡಾಕಿನೀ
ಡಾಕಿನೀ-ಗಣಕ್ಕೆ
ಡಾಕಿನೀ-ಗಣ-ದಲ್ಲಿ
ಡಾಕಿನೀ-ಗಣಾ
ಡಾಕಿನೀ-ಗಣಾಃ
ಡಾಕಿನೀ-ಗುಂಪಿಗೆ
ಡಾಕಿನೀತ್ವಮವಾಪ್ತಾನಾಂ
ಡಾಕಿನೀ-ಪತಿಃ
ಡಾಕಿನೀ-ಭುಜ-ಶೀರ್ಷಕಃ
ಡಾಕಿನೀ-ಯೋನಿಮಾಶ್ರಿತಾಃ
ತ
ತಂ
ತಂಗಳನ್ನ-ವನ್ನೂ
ತಂಗಳು-ಗಂಜೀ
ತಂಗಿಯ
ತಂಗಿಯಂತೆ-ಯಾದೆ
ತಂಡುಲಾಮರಿಚಾ-ದಯಃ
ತಂತುಮಾಶ್ರೇಣ
ತಂತ್ರ-ಸಾರ-ವೆಂಬ
ತಂತ್ರ-ಸಾರೋಕ್ತ
ತಂದ-ಕೊಂಡು
ತಂದನು
ತಂದು
ತಂದು-ಲೀ-ಯಕ-ಶಾಕಾಶ್ಚ
ತಂದೆ
ತಂದೆ-ತಾಯಿ-ಯರ
ತಂದೆ-ತಾಯಿ-ಯ-ರಂತೆ
ತಂದೆ-ತಾಯಿ-ಯ-ರನ್ನು
ತಂದೆ-ತಾಯಿ-ಯ-ರನ್ನೂ
ತಂದೆ-ತಾಯಿ-ಯ-ರಲ್ಲಿ
ತಂದೆ-ತಾಯಿ-ಯ-ರಿಗೆ
ತಂದೆ-ತಾಯಿ-ಯರು
ತಂದೆಯ
ತಂದೆ-ಯ-ವರ
ತಂದೆ-ಯ-ವ-ರಿಂದ
ತಂದೆ-ಯ-ವರು
ತಂದೆ-ಯಾದ
ತಂದೆಯು
ತಂದೆಯೂ
ತಃ
ತಕ್ಕ
ತಕ್ಕಂತೆ
ತಗಡಿನ
ತಗುಹೋದ್ವಾ-ರಿಸ್ಥ
ತಚಃ
ತಚ್ಚ
ತಚ್ಚಿತ್ರ-ಕಾರ್ಯಂ
ತಚ್ಛ್ರುತೇಃ
ತಜ್ಞಾ-ತ-ಮತುಲಂ
ತಟಸ್ಥಾಃ
ತಟಾಕಂ
ತಟಾಕಸೇತುಚ್ಛೇ-ದಶ್ಚ
ತಟಾಕಾದಿಜಲೈಃ
ತಟ್ಟು-ವುದಿಲ್ಲ
ತಡ-ಮಾಡದೆ
ತಡ-ಮಾಡ-ಬೇಡ
ತಡ-ಮಾಡಿ-ರ-ಲಿಲ್ಲ
ತಡ-ವನ್ನೂ
ತಡ-ವಾ-ಗಲು
ತಡೆ-ದರು
ತಡೆ-ಬರು-ವುದಿಲ್ಲ-ವೆಂಬ
ತಡೆಯ-ಬೇಕು
ತತ
ತತಃ
ತತಃಪರಂ
ತತತ್ಕೃತಾ-ಹಾರಂ
ತತಶ್ಚಂಪಾ-ಪುರೀಂ
ತತಸ್ತು
ತತಸ್ತುಷ್ಟೋ
ತತಸ್ತೇನ
ತತೋ
ತತೋ-ನುಜ್ಞಾ
ತತೋ-ಭೂತ್
ತತೋ-ಭೂ-ದಶ-ರೀರ-ವಾಕ್
ತತೋರ್ಭಕಃ
ತತೋ-ಽಘಮರ್ಷಣಂ
ತತೋ-ಽಧುನಾ
ತತ್
ತತ್ಕರ್ಮಜಂ
ತತ್ಕಾರಣಂ
ತತ್ಕುರುಷ್ಟ
ತತ್ಕುರುಷ್ವ
ತತ್ಕೃತ್ವಾ
ತತ್ಕೃಪಯಾ
ತತ್ತತ್
ತತ್ತತ್ಕರ್ಮ-ಫಲಂ
ತತ್ತತ್ತತ್ವಾಶ್ಚ
ತತ್ತ್ರಿ-ವಿಧಂ
ತತ್ತ್ವಂ
ತತ್ತ್ವಕ್ಕೆ
ತತ್ತ್ವತಃ
ತತ್ತ್ವ-ವನ್ನು
ತತ್ತ್ವಾ-ಭಿ-ಮಾನಿನೋ
ತತ್ತ್ವಾಹ್ವಾಃ
ತತ್ಪರಃ
ತತ್ಪರಿ-ಣಾಮ-ವಾಗಿ
ತತ್ಪಾಪಂ
ತತ್ಪಿತಾ
ತತ್ಪುಣ್ಯಂ
ತತ್ಪುತ್ರಂ
ತತ್ಪುತ್ರೀ
ತತ್ಪುರುಷೇಷು
ತತ್ಫಲಂ
ತತ್ಫಲಮ್
ತತ್ಯಾ-ಭಿ-ಮಾನಿ-ದೇವ-ತೆ-ಗಳಲ್ಲಿ
ತತ್ರ
ತತ್ರಾಧೀತಾಃ
ತತ್ರಾಪಶ್ಯನ್ವಟಂ
ತತ್ರಾಪಿ
ತತ್ರಾಪ್ಯುಷಸಿ
ತತ್ರಾಪ್ರಾಮಾಣಿಕೀಂ
ತತ್ರಾಬ್ದಾನಿ
ತತ್ರಾ-ಭಯಂ
ತತ್ರಾರ್ಜಿತಂ
ತತ್ರಾಶ್ವತ್ಥ
ತತ್ರಾಶ್ವತ್ಥಂ
ತತ್ರಾಸ್ಮಾಭಿಃ
ತತ್ರೈವ
ತತ್ರೋದುಂಬರ-ಮಾ-ಸಾದ್ಯ
ತತ್ವ
ತತ್ವ-ಕೋವಿದ
ತತ್ವಕ್ಕೆ
ತತ್ವ-ಗಳ
ತತ್ವ-ಗಳನ್ನೂ
ತತ್ವ-ಗಳಲ್ಲಿ
ತತ್ವ-ಗಳಿಗೆ
ತತ್ವದ
ತತ್ವ-ವನ್ನು
ತತ್ವಾತ್ಮಕ-ಮಿದಂ
ತತ್ವಾತ್ಮಾ
ತತ್ವಾದ್ಯೈಃ
ತತ್ವಾಭಿ
ತತ್ವಾಭಿ-ಧಾನ್
ತತ್ವಾಭಿ-ಮಾನಿ
ತತ್ವಾಭಿ-ಮಾನಿ-ದೇವ-ತೆ-ಗಳ
ತತ್ವಾಭಿ-ಮಾನಿ-ದೇವ-ತೆ-ಗಳಲ್ಲಿ
ತತ್ವಾಭಿ-ಮಾನಿ-ದೇವ-ತೆ-ಗಳಿಗೆ
ತತ್ವಾಭಿ-ಮಾನಿ-ದೇವ-ತೆ-ಗಳೂ
ತತ್ವಾವ್ಹೇಷು
ತತ್ವಾಹ್ವಾನಾಂ
ತತ್ವೇಶಾಃ
ತತ್ಸಂಬಂಧಾ-ದಯಂ
ತತ್ಸಖೀಭ್ಯೋ
ತತ್ಸಜ್ಜನಾಶ್ರಯಾಃ
ತತ್ಸಾಧ್ಯಾಂಗಾನಿ
ತತ್ಸು-ಮತುಲಂ
ತಥಾ
ತಥಾ-ನಿ-ಕೇಷು
ತಥಾನ್ಯ-ದಿವ-ಸೇಷು
ತಥಾನ್ಯ-ದೇ-ವತಾ
ತಥಾನ್ಯ-ದೋಷೋಸ್ತಿ
ತಥಾ-ಪತ್ತಿಂ
ತಥಾಪಿ
ತಥಾ-ಮಂತ್ರ್ಯ
ತಥಾತ್ವಿತಿ
ತಥೈಕತಃ
ತಥೈವ
ತಥೈವಾನ್ನತ-ವಾ-ದಿನಮ್
ತಥೈವಾ-ಮುಖ್ಯ
ತಥೈವಾಯಾಜ್ಯ
ತಥೈವಾಸಿತದ್ವಾ-ದಶ್ಯಾಂ
ತಥೈವೋನ್ಮಾರ್ಗ-ವರ್ತಿ-ನಾಮ್
ತಥೋತ್ಕ್ರಾಂತಿವೈತ-ರಣೀ-ಗೋ-ದಾ-ನಾದೀತ್ಯನೇಕಶಃ
ತಥೌಷಧೀಃ
ತದಧಿ-ಗಮಾಧಿಕ-ರಣ-ದಲ್ಲಿ
ತದ-ಧೀನಸ್ತದಾಶ್ರಿಯಃ
ತದ-ಧೀನಾಂ
ತದನ್ನಂ
ತದರ್ಥಂ
ತದರ್ಥೇ
ತದಹೇ
ತದಾ
ತದಾ-ಗ-ತಾಮ್
ತದಾ-ಗಮನ-ವೇ-ಲಾಯಾಂ
ತದಾ-ಧೀನಾ
ತದಾನೀಂ
ತದಾಯಂ
ತದಾರ್ಜುನಶ್ಚ
ತದಾರ್ದ್ರ-ವಸ್ತ್ರಾಣಾಮಪ-ತನ್ನು
ತದಾರ್ದ್ರ-ವಸ್ತ್ರಾದ್ಗಲಿತೋದಬಿಂದುಭಿಃ
ತದಾ-ಲಯಮ್
ತದಾಶ್ನಾತಿ
ತದಾ-ಸೀನ್ಮಾಘ-ಮಾಸೋ-ಽಪಿ
ತದಾ-ಽಸಾವಶ-ಪನ್ಮೂಢ-ಮಹಂಕಾರೇಣ
ತದೀಯಂ
ತದೀಶತ್ವಂ
ತದುಸಾಂಗಾನಿ
ತದೆ
ತದೇ-ತತ್ಕಥಿತಂ
ತದೇ-ತದ್ವಕ್ತುಮರ್ಹಥ
ತದೇ-ತದ್ವ-ಚನಂ
ತದೇವ
ತದೈವ
ತದೈ-ವಾದ್ಭುತ-ರೂಪೋ-ಭೂತ್
ತದ್
ತದ್ಗೃಹಂ
ತದ್ದಾನಂ
ತದ್ದಿ
ತದ್ದೇವಾನ್ನಾವಜಾನೀ-ಯಾನ್ಮಭಕ್ತಸ್ತಾನ್
ತದ್ದೇಶೇ
ತದ್ದ್ರಷ್ಟುಂ
ತದ್ಧನಂ
ತದ್ಧರ್ಮ
ತದ್ಧರ್ಮ-ಸೂಕ್ಷ್ಮಂ
ತದ್ಬೀಜತಾಂ
ತದ್ರಸಮ್
ತದ್ರಸಾಪ್ಯಾ-ಯನಂ
ತದ್ರಸಾಹಾರಃ
ತದ್ರಾತ್ರೌ
ತದ್ವ-ಚನಂ
ತದ್ವತ್
ತದ್ವನೇ-ಽಪಿ
ತದ್ವಿಜ್ಞಾ-ನಾರ್ಥಂ
ತದ್ಗ್ರಾಮೇ
ತನಕ
ತನಗೆ
ತನಗೇ
ತನಯಂ
ತನಯಃ
ತನಯೊ
ತನಯೋ
ತನಯೌ
ತನಾದ-ನಂತರ
ತನು
ತನು-ಭೃತ್ಕರ್ಮ
ತನೂಂ
ತನೂದ್ಭವಃ
ತನ್ನ
ತನ್ನನ್ನು
ತನ್ಮಂತ್ರ-ತೇ-ಜಸಾ
ತನ್ಮಧ್ಯೇ
ತನ್ಮೂಲಂ
ತನ್ಮೂಲಕ
ತನ್ಮೂಲ-ರಸಂ
ತನ್ಮೃತ್ಯು
ತಪ
ತಪಃ
ತಪಸಾ
ತಪಸೇ
ತಪಸೋ
ತಪಸ್ತಪ್ತು
ತಪಸ್ವಿ-ಗಳಿಗೂ
ತಪಸ್ವಿಯೂ
ತಪಸ್ವಿಯೇ
ತಪಸ್ವೀ
ತಪಸ್ವೀನಾಂ
ತಪಸ್ಸನ್ನು
ತಪಸ್ಸನ್ನೂ
ತಪಸ್ಸಿಗೆ
ತಪಸ್ಸಿನ
ತಪಸ್ಸು
ತಪಸ್ಸೆಂದರೆ
ತಪಸ್ಸೇ
ತಪೋ
ತಪೋ-ಭಿಶ್ಚ
ತಪೋ-ಲೋಕಂ
ತಪೋ-ಲೋಕಕ್ಕೆ
ತಪ್ತಂ
ತಪ್ತ-ನಾಗಿ
ತಪ್ತಾಯಸೇ
ತಪ್ಪದೆ
ತಪ್ಪದೇ
ತಪ್ಪಿ
ತಪ್ಪಿ-ಗಾಗಿ
ತಪ್ಪಿದ್ದಲ್ಲ
ತಪ್ಪಿ-ಸಿ-ಕೊಂಡ-ಮೇಲೆ
ತಪ್ಪಿ-ಸಿ-ಕೊಂಡು
ತಪ್ಪುಣ್ಯಂ
ತಬ್ಬಿ-ಕೊಂಡನು
ತಬ್ಬಿ-ಕೊಂಡು
ತಮ
ತಮಗಾದ
ತಮ-ಗಿಂತ
ತಮಗೆ
ತಮಬ್ರವೀಶ್
ತಮಯಾ-ಚ-ಯನ್
ತಮರ್ಭಕಂ
ತಮಶ್ಚೈವ
ತಮಸಿ
ತಮಸ್ಯ
ತಮಸ್ಸು
ತಮಸ್ಸೆಂಬ
ತಮಾಲಹಸಿ-ತಾಂಬರೇ
ತಮೋ
ತಮ್
ತಮ್ಮ
ತಮ್ಮಂಥ-ವರ
ತಮ್ಮ-ತಮ್ಮಲ್ಲಿಯೇ
ತಮ್ಮ-ನಾದ
ತಮ್ಮನ್ನು
ತಮ್ಮಲ್ಲಿ
ತಮ್ಮಲ್ಲಿಯೇ
ತಮ್ಮಿಂದ
ತಯಾರಿಸಿ
ತಯಾರಿಸಿ-ಕೊಂಡ
ತಯಾರು-ಮಾಡುತ್ತಿದ್ದೆ
ತಯೊರ್ವಾ
ತಯೋಃ
ತಯೋರ್ಜೀವಪ್ರಧಾನಯೋಃ
ತಯೋರ್ಮುಕ್ತಿರ್ಭ-ವಿಷ್ಯತಿ
ತಯೋರ್ವಿಮಾನೇ
ತರ-ಕಾರಿ
ತರ-ಕಾರಿ-ಗಳನ್ನು
ತರ-ಕಾರಿ-ಗಳನ್ನೂ
ತರಲು
ತರಿತ್ವಾ
ತರುಣಿಯೆಲ್ಲಿ
ತರುಣೀ
ತರುತ್ತಿ-ರುವ
ತರುತ್ತೇನೆ
ತರ್ಕ-ಗಳಿಂದ
ತರ್ಕ-ವಿದ್ಯೆ-ಯನ್ನು
ತರ್ಜ-ನಮ್
ತರ್ಜ-ನಾದ್ಯೈಶ್ಚ
ತರ್ಪಣ
ತರ್ಪಣ-ವನ್ನು
ತರ್ಪ-ಯೇತ್
ತಲುಪಿ
ತಲೆ
ತಲೆ-ಗಳನ್ನು
ತಲೆ-ಗಳನ್ನುಳ್ಳ
ತಲೆ-ಗಳಿದ್ದವು
ತಲೆ-ತಗ್ಗಿಸಿ
ತಲೆಯ
ತಲೆ-ಯಂತೆ
ತಲೆ-ಯನ್ನು
ತಲೆ-ಯಲ್ಲಿ
ತಲೆ-ಯಲ್ಲಿನ
ತಲೆ-ಯಾ-ಯಿತು
ತಲೆ-ಯಿದ್ದದ್ದು
ತಲೆ-ಯಿ-ರುವ
ತಲೆ-ಯುಳ್ಳ
ತಲೆ-ಯುಳ್ಳ-ವ-ನಾದನು
ತಲೆ-ಯುಳ್ಳ-ವ-ರಾಗಿ
ತಲೆ-ಯೆತ್ತಿ-ಕೊಂಡು
ತಲೆಯೇ
ತವ
ತವಾಂಸೀವ
ತವಾದ್ಯ
ತವೋದಿತಾಃ
ತಸಮನ್ವಿ-ತಮ್
ತಸಸ್ಮಿನ್ವೈ
ತಸ್ಕಾಂ
ತಸ್ತಂ
ತಸ್ಥೌ
ತಸ್ಮಾ
ತಸ್ಮಾಚ್ಚ
ತಸ್ಮಾಚ್ಛಾಸ್ತ್ರಂ
ತಸ್ಮಾತ್
ತಸ್ಮಾತ್ಕರ್ಮ
ತಸ್ಮಾತ್ಕರ್ಮಣ್ಯ-ಧಿ-ಕೃತಾನ್
ತಸ್ಮಾತ್ಕೋಽನ್ವ-ಪರಃ
ತಸ್ಮಾತ್ತೇಽವಿದುಷೋ
ತಸ್ಮಾತ್ಪರಿ-ಚರೇನ್ನಿ-ಜಮ್
ತಸ್ಮಾತ್ಪಾಪಂ
ತಸ್ಮಾತ್ಪುನರದೃಶ್ಯೋಸಿ
ತಸ್ಮಾತ್ಸರ್ವ-ಮತಂ
ತಸ್ಮಾತ್ಸಹಾಯೌ
ತಸ್ಮಾತ್
ತಸ್ಮಾತ್-ಪಾಪಂ
ತಸ್ಮಾ-ದಗ್ನಿಂ
ತಸ್ಮಾ-ದನು-ಪಥಂ
ತಸ್ಮಾ-ದಪ್ಯಾ-ಯಯುಃ
ತಸ್ಮಾ-ದಯಂ
ತಸ್ಮಾ-ದಸ್ಯ
ತಸ್ಮಾ-ದಾ-ಗಚ್ಛ
ತಸ್ಮಾ-ದಾಯ-ಯತುರ್ವೃದ್ಧೌ
ತಸ್ಮಾ-ದೇತೌ
ತಸ್ಮಾ-ದೇನಂ
ತಸ್ಮಾದ್ದೋಷಾನ್ಮೃತಿಂ
ತಸ್ಮಾದ್ಧಿತಮಭೀಪ್ಸುನಾ
ತಸ್ಮಾದ್ಬಲ-ವತ್ತ-ರಮ್
ತಸ್ಮಾದ್ಯಥಾ-ಗುಣಂ
ತಸ್ಮಾದ್ವದ
ತಸ್ಮಾದ್ವಿಷ್ಣುಂ
ತಸ್ಮಾದ್ವಿಷ್ಣ್ವನು-ಗಾನ್
ತಸ್ಮಾದ್ವೈ
ತಸ್ಮಾನ್ಮಾಘಸ್ಯ
ತಸ್ಮಾಲ್ಪ
ತಸ್ಮಿಂಕಾಲೇ
ತಸ್ಮಿನ್
ತಸ್ಮಿನ್ನಸರ-ಪಕ್ಷೇ-ಽಪಿ
ತಸ್ಮಿನ್ನೈಕವುಪಿ
ತಸ್ಮೈ
ತಸ್ಯ
ತಸ್ಯಾ
ತಸ್ಯಾಂ
ತಸ್ಯಾಂಗಂ
ತಸ್ಯಾಃ
ತಸ್ಯಾನ್ಮುಕ್ತಿಃ
ತಸ್ಯಾ-ಪರಾ-ಧ-ಸಾಹಸ್ರಂ
ತಸ್ಯಾ-ಮಹಂ
ತಸ್ಯಾ-ಸೀತ್
ತಸ್ಯಾ-ಸೀತ್ತನಯೋ
ತಸ್ಯಾಸ್ತೀರೇ
ತಸ್ಯಾಹಂ
ತಸ್ಯಾ-ಹ-ಮ-ನುಗಃ
ತಸ್ಯಾ-ಹಮ-ಭ-ವತ್ಪುತ್ರೋ
ತಸ್ಯೇಯಂ
ತಸ್ಯೈತೇ
ತಸ್ಯೈಷ
ತಸ್ಯೋಪ-ಶಾಂತಯೇ
ತಾ
ತಾಂ
ತಾಂಬೂಲ
ತಾಂಬೂಲಂ
ತಾಂಬೂಲ-ಗಳನ್ನು
ತಾಃ
ತಾಕಾರಾ-ನುದೀಚೀಂ
ತಾಡಯಾ-ಮಾಸ
ತಾತ
ತಾತ್ಕಾಲಿಕಂ
ತಾತ್ಪರ್ಯಕಾಃ
ತಾತ್ಪರ್ಯ-ವನ್ನು
ತಾತ್ಪರ್ಯ-ವೇ-ನೆಂದರೆ
ತಾತ್ಸಾರ-ಮಾಡಿದೆ
ತಾದಿತ್ಯ
ತಾದೃಕ್
ತಾದೃಗಪಿ
ತಾದೃಗೇವಾನುಭೂ-ಯತೇ
ತಾದೃಗ್ಬಂಧ-ಮಹಾ-ಗುಲ್ಮ-ಮೂಲಂ
ತಾದೃಶೇ
ತಾದೃಶೇನ
ತಾದೃಶೋಹಂ
ತಾನದ್ಭು
ತಾನಧಃ
ತಾನಸ್ತಿ
ತಾನಾ-ಗತಾಂ
ತಾನಾ-ಮಂತ್ರ್ಯ
ತಾನಿ
ತಾನು
ತಾನುತ್ಥಾ
ತಾನು-ಸಾಶ್ಚ
ತಾನೂ
ತಾನೂ-ಚುರ್ಭಗ-ವದ್ದೂತಾ
ತಾನೇ
ತಾನೇ-ತಾನ್
ತಾನ್
ತಾನ್ಯ
ತಾನ್ಯತ್ರ
ತಾನ್ಯು
ತಾನ್ಯೇ-ತಾನಿ
ತಾಪಂ
ತಾಪಸ
ತಾಭ್ಯಾಂ
ತಾಮಭಿ-ಪದ್ಯತ
ತಾಮಸ-ಭಾ-ವನೆ
ತಾಮಸಭಾ-ವಾಶ್ಚ
ತಾಮ-ಸರು
ತಾಮ-ಸ-ರೆಂದು
ತಾಮಸಾ
ತಾಮಸಾಃ
ತಾಮಸೀ
ತಾಮಿ-ಯಾತ್
ತಾಮ್ರಪರ್ಣೀ
ತಾಮ್ರಪರ್ಣೀಂ
ತಾಮ್ರ-ವಿಲೋ-ಚನಾಃ
ತಾಯಂದಿರು
ತಾಯಿ
ತಾಯಿಗೆ
ತಾಯಿಯ
ತಾಯಿ-ಯನ್ನು
ತಾಯಿ-ಯನ್ನೇ
ತಾಯಿ-ಯ-ರನ್ನು
ತಾಯಿ-ಯರು
ತಾಯಿಯು
ತಾಯಿಯೂ
ತಾಯಿಯೇ
ತಾರ-ತಮ್ಯ
ತಾರ-ತಮ್ಯ-ವನ್ನು
ತಾರ-ಯಿತಾರ
ತಾಲ-ಪತ್ರ-ಕಮ್
ತಾಲವಃ
ತಾವತಾ
ತಾವತ್ತ್ವಯಾ
ತಾವದ್ಯಾ-ವನ್ಮೇ
ತಾವನ್ಮಾ
ತಾವಪ್ಯಾಜಗ್ಮ
ತಾವಾಗಿಯೇ
ತಾವು
ತಾವೂ
ತಾವೂ-ಚತುಃ
ತಾವೇ
ತಾಶ್ಚ
ತಾಸಾಂ
ತಾಹತ-ವಾಯುನಾ
ತಿಂ
ತಿಂಗಳ-ಕಾಲ
ತಿಂಗಳಾದ
ತಿಂಗಳಾದರೂ
ತಿಂಗಳಿ-ನಲ್ಲಿ
ತಿಂಗಳು
ತಿಂಗಳು-ಗಳಲ್ಲಿಯೇ
ತಿಂದ
ತಿಂದು
ತಿಕ್ಕಿ
ತಿಗ್ಮದ್ಯು
ತಿಗ್ಮದ್ಯುತಿ
ತಿಥಿಃ
ತಿಥಿ-ಗಳಲ್ಲಿಯೂ
ತಿಥಿ-ಗಳಿ-ಗಿಂತಲೂ
ತಿಥಿ-ಗಳು
ತಿಥಿಗೆ
ತಿಥಿಭ್ಯೋ
ತಿಥಿಷು
ತಿಥೀ-ನಾ-ಮಪಿ
ತಿನ್ನಲು
ತಿನ್ನುತ್ತಿದ್ದೆನೇ
ತಿನ್ನುವುದ-ರಲ್ಲಿ
ತಿನ್ನು-ವುದು
ತಿನ್ನೋ
ತಿಮವಾ-ಪನ್ನೋ
ತಿಮವಾಪ್ತೋ-ಽಹಂ
ತಿಮ್
ತಿರನುದಕಾಶ್ಚ
ತಿರಸ್ಕರಿ-ಸಿದೆ
ತಿರಸ್ಕಾರ
ತಿರಸ್ಕಾರ-ಭಾ-ವನೆ-ಯಿಂದ
ತಿರುಕೊಯಲೂರಿ-ನಲ್ಲಿ
ತಿರುಗಾಡಿ-ದರೂ
ತಿರುಗಾಡುತ್ತಾ
ತಿರುಗಾಡುತ್ತಿದ್ದರು
ತಿರುಗಿ
ತಿರುಗುತ್ತಿದ್ದೆ
ತಿರ್ಯಕ್ನರ-ಪಿಶಾಚಕಾಃ
ತಿಲಕಂ
ತಿಲಕ-ವನ್ನು
ತಿಲಗವ್ಯಾಢ್ಯಂ
ತಿಲಗವ್ಯೇನ
ತಿಲ-ಪಾತ್ರಂ
ತಿಲ-ಭಾರ
ತಿಲ-ಭಾರಃ
ತಿಳಿದ
ತಿಳಿ-ದರೂ
ತಿಳಿದ-ವರು
ತಿಳಿದಿದೆಯೆ
ತಿಳಿ-ದಿದ್ದರೂ
ತಿಳಿದಿದ್ದೆ
ತಿಳಿದಿ-ರುವ
ತಿಳಿದು
ತಿಳಿದು-ಕೊಂಡಿ-ರುವ
ತಿಳಿದು-ಕೊಂಡು
ತಿಳಿದು-ಕೊಳ್ಳ-ಬೇಕು
ತಿಳಿದು-ಕೊಳ್ಳ-ಬೇಕೆಂಬ
ತಿಳಿದು-ಕೊಳ್ಳಲು
ತಿಳಿದೂ
ತಿಳಿದೆ
ತಿಳಿ-ಯದು
ತಿಳಿ-ಯದೆ
ತಿಳಿ-ಯದೇ
ತಿಳಿಯ-ಬಹುದು
ತಿಳಿಯ-ಬಾರದು
ತಿಳಿಯ-ಬೇಕು
ತಿಳಿ-ಯರು
ತಿಳಿಯ-ಲಿಲ್ಲವೆ
ತಿಳಿಯ-ಲಿಲ್ಲವೇ
ತಿಳಿ-ಯಲು
ತಿಳಿಯುತ್ತಲೇ
ತಿಳಿ-ಯುತ್ತಾರೆ
ತಿಳಿಯು-ವುದಿಲ್ಲ
ತಿಳಿ-ಸ-ಲಿಲ್ಲ
ತಿಳಿಸಿ
ತಿಳಿಸಿ-ದರು
ತಿಳಿ-ಸಿರಿ
ತಿಳಿಸಿ-ರುತ್ತೇವೆ
ತಿಳಿಸುತ್ತದೆ
ತಿಳಿ-ಸುತ್ತವೆ
ತಿಷ್ಠ
ತಿಷ್ಠಂತೇ
ತಿಷ್ಠತಿ
ತಿಷ್ಠತ್ಯ-ಕರ್ಮ-ಕೃತ್
ತಿಷ್ಠತ್ಯಾ-ಭೂತಸಂಪ್ಲ-ವಮ್
ತಿಷ್ಠನ್
ತಿಷ್ಠಾಮಃ
ತಿಸ್ತಥಾ
ತಿಸ್ರ
ತಿಸ್ರಃ
ತಿಸ್ರಶ್ಚ
ತಿಸ್ರೋವಸ್ಥಾಃ
ತೀಕ್ಷ್ಣ
ತೀಕ್ಷ್ಮತುಂಡಾ
ತೀಕ್ಷ್ಮ-ವಾದ
ತೀತಾನಿ
ತೀರ-ದಲ್ಲಿ
ತೀರಿ-ತ-ನಮ್
ತೀರಿಸ-ಬೇಕು
ತೀರಿಸ-ಬೇಕೆಂಬು-ದನ್ನು
ತೀರಿಸಿ
ತೀರಿಸಿ-ದನು
ತೀರಿಸು
ತೀರಿಸು-ವ-ವರೆಗೆ
ತೀರೇ
ತೀರ್ತ್ವಾ
ತೀರ್ಥ-ಗಳ
ತೀರ್ಥ-ಗಳನ್ನು
ತೀರ್ಥ-ಗಳಲ್ಲಿ
ತೀರ್ಥ-ಗಳಿ-ರುತ್ತವೆಯೋ
ತೀರ್ಥ-ಗಳು
ತೀರ್ಥ-ಗಳೂ
ತೀರ್ಥಚರ್ಯಾದಿಭಿಸ್ತಥಾ
ತೀರ್ಥ-ಜಾ-ತಾನಿ
ತೀರ್ಥದ
ತೀರ್ಥ-ದಲ್ಲಿ
ತೀರ್ಥ-ದಿಂದ
ತೀರ್ಥ-ಪದೇ
ತೀರ್ಥ-ಭೂತಂ
ತೀರ್ಥ-ಯಾತ್ರಾ
ತೀರ್ಥ-ಯಾತ್ರಾ-ಪರ-ನಾಗಿ
ತೀರ್ಥ-ಯಾತ್ರಾ-ಪರಾ-ಯಣಃ
ತೀರ್ಥ-ಯಾತ್ರಾಪ್ರಸಂಗೇನ
ತೀರ್ಥ-ಯಾತ್ರಾ-ಫಲಪ್ರದಾ
ತೀರ್ಥ-ಯಾತ್ರಾರ್ಥೇ
ತೀರ್ಥ-ಯಾತ್ರೆ
ತೀರ್ಥ-ಯಾತ್ರೆ-ಗಳು
ತೀರ್ಥ-ಯಾತ್ರೆ-ಗಾಗಿ
ತೀರ್ಥ-ಯಾತ್ರೆ-ಗೆಂದು
ತೀರ್ಥ-ಯಾತ್ರೆಯ
ತೀರ್ಥ-ರೂಪರೂ
ತೀರ್ಥ-ವನ್ನು
ತೀರ್ಥಾನಾಂ
ತೀರ್ಥಾನಿ
ತೀರ್ಥಾಭಿ-ಮಾನಿ
ತೀವ್ರತೆ-ಯನ್ನು
ತು
ತುಂಗಾ-ನದಿ-ಯಲ್ಲಿ
ತುಂಗಾಯಾಂ
ತುಂದಿಲನು
ತುಂದಿಲಸ್ಯ
ತುಂಬ
ತುಂಬಾ
ತುಂಬಿ
ತುಂಬಿ-ಕೊಡುತ್ತಾರೆ
ತುಂಬಿತ್ತು
ತುಂಬಿದ
ತುಂಬಿದ್ದುವು
ತುಂಬಿ-ರುವ
ತುಂಬುರು-ನೆಂಬ
ತುಕಾ-ಮಾಸೀದ್ಯ
ತುಚ್ಛ-ವಾದ
ತುಟಿಕಂಠ-ಗಳು
ತುಟಿ-ಗಳನ್ನು
ತುದಂತೇ
ತುದಂತ್ಯೇನಂ
ತುದಿಯ
ತುದಿಲ
ತುದಿಲೋ
ತುಪ್ಪ
ತುಪ್ಪ-ದಿಂದ
ತುಪ್ಪ-ಹಾಲು
ತುಮರ್ಹಥ
ತುರಃ
ತುರ್ದೀನಂ
ತುರ್ಯಶ್ಚ
ತುಲಸಿಯ
ತುಲಸೀ
ತುಲಸೀ-ಕಾನನೇಂಽಬುಜೇ
ತುಲಸೀ-ಗಿಡ-ಗಳು
ತುಲಸೀ-ಪತ್ರೈಃ
ತುಲಸೀ-ಮಣಿ-ಮಾಲಾಂ
ತುಲಸೀ-ಮೃದಮಾ-ಲಿಪ್ಯ
ತುಲಸೀ-ವಿಪ್ರಾಸ್ತ್ರಿ-ವಿಧಾ
ತುಲಸೀ-ಹೀನಾಂ
ತುಲಸ್ಯಾ
ತುಲಾ
ತುಲಾ-ಮಾರೋ-ಪಿತಾಃ
ತುಲಾ-ಮಾಸ-ದಲ್ಲಿ
ತುಲಾ-ಮಾಸೇ
ತುಲಾ-ಸಂಸ್ಥೇ
ತುಲಾಸ್ನಾನಂ
ತುಲ್ಯ
ತುಲ್ಯಂ
ತುಳಸಿ-ಗಿಡ-ಗಳ
ತುಳಸಿ-ಯನ್ನು
ತುಳಸಿ-ಯಿಂದ
ತುಳಸಿಯಿದ
ತುಳಸೀ
ತುಳಸೀಂ
ತುಳಸೀ-ಕಾ-ನನಂ
ತುಳಸೀ-ದಳ-ಗಳನ್ನು
ತುಳಸೀ-ದಳ-ಗಳಿಂದ
ತುಳಸೀ-ದಳ-ವನ್ನು
ತುಳಸೀ-ಪತ್ರ-ಗಳನ್ನು
ತುಳಸೀ-ಮಣಿಯ
ತುಳಸೀ-ಯನ್ನೂ
ತುಳಸೀ-ರ-ಹಿತ-ವಾದ
ತುಳಸೀ-ವೃಕ್ಷ-ಗಳ
ತುಷಂ
ತುಷ್ಟಿ
ತುಷ್ಟುವುರ್ವಿ-ವಿಧೈಃ
ತುಷ್ಟೇ
ತೂಗಿ-ನೋ-ಡಲು
ತೂರ್ಣಂ
ತೂರ್ಣಮಗಮದ್ಗೇಹಂ
ತೂಷ್ಣೀಂ
ತೂಹಲಂ
ತೃಣಪರ್ಣೈಃ
ತೃಣ-ಮಪಿ
ತೃತೀಯ
ತೃತೀ-ಯಸ್ತು
ತೃತೀಯೇ
ತೃತೀಯೋ
ತೃತೀಯೋಯಂ
ತೃತೀಯೋ-ಽಧ್ಯಾಯಃ
ತೃಪ್ತ-ರಾದ
ತೃಪ್ತಿ
ತೃಪ್ತಿ-ಗಾಗಿ
ತೃಪ್ತಿ-ಗೋಸ್ಕರ
ತೃಪ್ತಿ-ಯನ್ನು
ತೃಪ್ತಿ-ಯಾಗ-ಲಿಲ್ಲ
ತೃಪ್ತಿ-ಯಾಗುತ್ತದೆ
ತೃಪ್ತಿರ್ನಾಸೀನ್ಮಮ-ವಾದ್ಯ
ತೃಪ್ತ್ಯೈ
ತೃಪ್ಯತಿ
ತೃಮಾದಿ-ವಕ್ತಾ
ತೃಷಯಾ
ತೆಂಗಿನ-ಕಾಯಿ
ತೆಗೆದು
ತೆಗೆ-ದು-ಕೊಂಡ
ತೆಗೆ-ದು-ಕೊಂಡು
ತೆಗೆ-ದು-ಬಿಟ್ಟರೆ
ತೆಗೆ-ಯಲು
ತೆರಳಲು
ತೆರಳಿ
ತೆರಳಿ-ದರು
ತೆರಳು
ತೇ
ತೇಜಃಪ್ರ-ಭಾವ-ದಿಂದ
ತೇಜಸಾ
ತೇಜಸ್ತತ್ತತ್ವಾ
ತೇಜಸ್ತತ್ವಕ್ಕೆ
ತೇಜಸ್ವಿ-ಯಾಗಿದ್ದ
ತೇಜಸ್ವಿ-ಯಾದ
ತೇಜಸ್ಸಿಗೆ
ತೇಜಸ್ಸಿ-ನಿಂದ
ತೇನ
ತೇನಾ-ಭದ್ರೇಣ
ತೇನಾರ್ಜಿತಂ
ತೇನಾಸ್ಮಾಭಿಸ್ತು
ತೇನಾಹಂ
ತೇನೇದಂ
ತೇನೇದಮಖಿಲಂ
ತೇನೇದೃಶೀ
ತೇನೇಷ್ಟಾಃ
ತೇನೈವ
ತೇಪಿ
ತೇಭ್ಯ
ತೇಭ್ಯಃ
ತೇಭ್ಯೋ
ತೇಷಾಂ
ತೇಷಾಮರ್ಥೆ
ತೇಷು
ತೇಷ್ವಯಂ
ತೇಷ್ವಾದೌ
ತೇಽಬ್ರುವನ್
ತೈಃ
ತೈರುದಿತಾ
ತೈರು-ದಿತೇ
ತೈಲಂ
ತೈಲ-ದಿಂದ
ತೈಲಭೋಗಂ
ತೈಲಶೃಂಗಾರ-ದಿಂದ
ತೈಲಸೇವ-ನಾತ್
ತೈಲಾಭ್ಯಂಗಂ
ತೈಲಾಭ್ಯಂಗ-ವನ್ನೂ
ತೈಲಾಭ್ಯಂಗೇ
ತೈಲಾಭ್ಯಂಜನ
ತೈಲಾಭ್ಯಂಜನ-ದೊಷೇಣ
ತೈಲೈಶ್ಚಂಪಕ-ಗಂಧಿಭಿಃ
ತೈಲೋಕ್ಯಾಂತರ್ಗ-ತಾನಿ
ತೊ
ತೊಂಡೆ-ಕಾಯಿ
ತೊಂದರೆ-ಗೀಡಾಗಿ
ತೊಂದರೆ-ಯನ್ನು
ತೊಂದರೆಯೂ
ತೊಗಟೆ-ಗಳು
ತೊಗರಿ
ತೊಗರಿ-ಗಳನ್ನು
ತೊಡಗ-ಬೇಕು
ತೊಡಗಿ-ದಳು
ತೊಡಗಿದ್ದ
ತೊಡಗು-ವುದು
ತೊಡೆ-ಯಲ್ಲಿ
ತೊನ್ನು
ತೊರೆದು
ತೊರೆದೆ
ತೊಳೆದು
ತೋ
ತೋಚ-ದಂತೆ
ತೋಟ-ಗಳಲ್ಲಿ
ತೋಯಂ
ತೋರಣ
ತೋರ-ತಕ್ಕ
ತೋರದೆ
ತೋರಿಸ-ಬೇಕು
ತೋರಿಸಿ-ದನು
ತೋರಿ-ಸಿದೆ
ತೋರಿ-ಸುತ್ತಾನೆ
ತೋರಿ-ಸುವ
ತೋರುತ್ತದೆ
ತೋರುವ
ತೋಳ
ತೋಳನ
ತೋಳ-ಹುಲಿ-ಗಳಂತೆ
ತೋಷಯಾ-ಮಾಸ
ತೋಸ್ಮ್ಯ-ಹಮ್
ತೋಹಂ
ತೋಽಗಾದ್ದ್ವಿಜಾನ್ವನೇ
ತೋಽಮುನಾ
ತೌ
ತ್ಕುಡ್ಯಪಾರ್ಶ್ವಮು-ಪಾಶ್ರಿತಃ
ತ್ತಿಃ
ತ್ತಿಯು
ತ್ಮಕೋ
ತ್ಯಕ್ತಾ
ತ್ಯಕ್ತೌ
ತ್ಯಕ್ತ್ವಾ
ತ್ಯಜತ್ಯ-ಮುಮ್
ತ್ಯಜನ್
ತ್ಯಜಾಮ್ಯ-ಹಮ್
ತ್ಯಜಿ-ತಾನಾಂ
ತ್ಯಜಿಸ-ಬೇಕೆಂದು
ತ್ಯಜಿಸಿ
ತ್ಯಜಿಸಿದ
ತ್ಯಜಿಸಿ-ದಳು
ತ್ಯಜಿಸಿದ್ದೆ
ತ್ಯಜಿಸುತ್ತೇನೆ
ತ್ಯಜಿಸು-ವುದಿಲ್ಲ
ತ್ಯಜೇತ್ತು
ತ್ಯಹಂ
ತ್ಯಾ
ತ್ಯಾಗ
ತ್ಯಾತ್ತೇ
ತ್ಯಾಧಿ-ಪಂಚ-ಕಮ್
ತ್ಯುದಿತಾ
ತ್ರಂಬ-ಕಸ್ಯ
ತ್ರಯ
ತ್ರಯಃ
ತ್ರಯತ್ರಿಂಶತ್ತಥಾ
ತ್ರಯೋ-ದಶಿ
ತ್ರಯೋ-ದಶಿ-ಯಲ್ಲಿ
ತ್ರಯೋ-ದಶೀ
ತ್ರಯೋ-ದಶೋಧ್ಯಾಯಃ
ತ್ರಯೋ-ದಶ್ಯಾಂ
ತ್ರಯೋ-ಽಪಿ
ತ್ರಾತಂ
ತ್ರಾತಾ
ತ್ರಾತುಮರ್ಹಸಿ
ತ್ರಿಕೂಟ
ತ್ರಿಕೂಟಾಗ್ರೇ
ತ್ರಿಗು-ಣಾತ್ಮಕ-ವಾದ
ತ್ರಿದಶೇಶ್ವರೀ
ತ್ರಿದಿನಂ
ತ್ರಿದಿ-ನಸ್ಯ
ತ್ರಿದಿನಾಂತೇ
ತ್ರಿದಿ-ನಾತ್ಮಕೇ
ತ್ರಿಪಥಗಾಮಿನಿ
ತ್ರಿಪಥಗಾಮಿನೀ
ತ್ರಿರಾತ್ರಂ
ತ್ರಿರ್ನಿರ್ಮೃಜ್ಯ
ತ್ರಿವಿಧಂ
ತ್ರಿವಿಧಾ
ತ್ರಿವಿಧಾನ್ಯಾಹುರ್ಗುಣ-ಭೇದ-ಫಲಾನ್ಯಪಿ
ತ್ರಿವೇಣಿ
ತ್ರಿವೇಣ್ಯಾಂ
ತ್ರಿಷು
ತ್ರಿಸಪ್ತ
ತ್ರೀಣಿ
ತ್ರೀಣ್ಯ
ತ್ರೀನೇ-ತಾನ್ನೇತು-ಕಾಮಾಸ್ತು
ತ್ರೇಣ
ತ್ರೇತಾ
ತ್ರೇತಾಗ್ನಯ
ತ್ರೇತಾದ್ಯೇಷು
ತ್ರೈಲೋಕ-ಪಾ-ವನೀ
ತ್ವಂ
ತ್ವಕ್
ತ್ವಕ್ಷಗೋಲ-ಕಮ್
ತ್ವಕ್-ಮಾಂಸ-ಚರ್ಮ-ಮಜ್ಞಾಶ್ಚ
ತ್ವಗಿಂದ್ರಿ-ಯಕ್ಕೆ
ತ್ವತ್
ತ್ವತ್ಸುತೊ
ತ್ವದರ್ಥಂ
ತ್ವದೀಯಂ
ತ್ವದ್ದಾ
ತ್ವಭಿ-ಶಸ್ತಾನಾಂ
ತ್ವಮಗ್ರಣೀರ್ಧರ್ಮ-ದೃಶಾಮೃಷೀನಾಂ
ತ್ವಮರ್ಭಕೋ
ತ್ವಮುದ್ಧರ್ತು-ಮಿಹಾರ್ಹಸಿ
ತ್ವಮೇವ
ತ್ವಯಮ್
ತ್ವಯಾ
ತ್ವಯಾ-ಽದ್ಯ
ತ್ವರಯಾ
ತ್ವರ-ಯಾನ್ವಿತಾಃ
ತ್ವಹ-ನದ್ಬಲ-ಮಂಜಸಾ
ತ್ವಾ
ತ್ವಾಂ
ತ್ವಾತ್
ತ್ವಾರ-ನಾಲಂ
ತ್ವಿಂದುಕ್ಷಯೇ
ತ್ವಿಂದ್ರೋ
ತ್ವಿತಿ
ತ್ವಿದಂ
ತ್ವಿದಮ್
ತ್ವೇತಿ
ತ್ವೈವ
ಥ
ಥನಂ
ಥರ್ಮಾಣಾಂ
ದಂ
ದಂಟು-ಗಳನ್ನು
ದಂಡ
ದಂಡ-ಗಳನ್ನು
ದಂಡ-ಧರೌ
ದಂಡ-ಪಾಶಾಸಿಪಾಣಯಃ
ದಂಡಪ್ರಣಾಮ
ದಂಡ-ಯಂತಿ
ದಂಡ-ಯತಿ
ದಂಡ-ವತ್ಪತಿತೋ
ದಂಡಾನಾಂ
ದಂಡಿತಾ
ದಂಡಿಸ-ಬೇಡಿರಿ
ದಂಡಿ-ಸಿದ
ದಂಡಿ-ಸಿದೆವು
ದಂಡಿ-ಸುತ್ತಾರೆ
ದಂಡಿ-ಸುತ್ತಿದ್ದೆ
ದಂಡಿ-ಸುತ್ತಿದ್ದೆವು
ದಂಡಿ-ಸು-ವಂತೆ
ದಂಡೇನಾಹಂ
ದಂಡೋಽಸ್ಮಾಭಿರ್ವಿಧೀ-ಯತೇ
ದಂಡ್ಯ
ದಂತ-ಕಾಷ್ಠಂ
ದಂತಧಾ-ವನ-ಮಾ-ಚ-ರೇತ್
ದಂಪತಿ-ಗಳನ್ನು
ದಂಪತಿ-ಗಳು
ದಂಪತೀ
ದಂಪತ್ಯೋ
ದಂಭಕಯಿತ್ಯ-ಹಮ್
ದಂಷ್ಟ್ರಾ-ಕರಾ-ಲ-ವದನಾಃ
ದಕ್ಷನೂ
ದಕ್ಷ-ಸಂಜ್ಞಕಃ
ದಕ್ಷಸ್ತತ್ತಸ್ಸರ್ವಕ್ರಿಯಾಸು
ದಕ್ಷಾ
ದಕ್ಷಿಣ
ದಕ್ಷಿಣಕ್ಕೆ
ದಕ್ಷಿಣ-ಗಾತ್ರಾಣಿ
ದಕ್ಷಿಣಾ
ದಕ್ಷಿಣಾಂ
ದಕ್ಷಿಣಾ-ದಿ-ಗಳಿಂದ
ದಕ್ಷಿಣಾ-ಮಾಶಾಂ
ದಕ್ಷಿಣೆ
ದಕ್ಷಿಣೆ-ಯನ್ನು
ದಕ್ಷಿಣೇ
ದಕ್ಷೋ
ದಗ್ಗ
ದಗ್ಧ
ದಗ್ಧ-ಪಾಣಿ
ದಗ್ಧ-ಪಾಣಿ-ರಹಂ
ದಗ್ಧ-ಪಾಣಿ-ರಿತಿಖ್ಯಾ
ದಗ್ಧಾನ್ಯ-ಸಂಶಯಃ
ದಗ್ಧೋ-ಽಯಂ
ದಟ್ಟ-ವಾದ
ದಡಕ್ಕೆ
ದಡ-ದಲ್ಲಿ
ದತ್ತ
ದತ್ತಂ
ದತ್ತ-ಮೌಲ್ಯಕಃ
ದತ್ತ-ವಾನುದಕಾಂಜಲೀನ್
ದತ್ತಾ
ದತ್ತಾ-ಧಿ-ಕಾರಸ್ಸು
ದತ್ವಾ
ದತ್ವಾನ್ನಂ
ದತ್ಸಾ
ದದರ್ಶ
ದದರ್ಶಾಂಬರಚುಂಬಿ-ತಮ್
ದದಾತಿ
ದದಾತೀಷ್ಟಾಂಸ್ತಥಾ
ದದಾಮಿ
ದದಾಸಿ
ದದೌ
ದದ್
ದದ್ಯಾತ್
ದದ್ಯಾತ್ಕಂಬಲಂ
ದದ್ಯಾತ್ತುಲಸೀ-ಗಂಧ-ಮಪಿ
ದದ್ಯಾತ್ತುಲಸೀ-ಪತ್ರಂ
ದದ್ಯಾತ್ಸಂಪೂಜ್ಯತಾಂ
ದದ್ಯಾದನ್ನಂ
ದದ್ಯಾದನ್ನಶ್ರಾದ್ಧಂ
ದದ್ಯಾದರ್ಘ್ಯಂ
ದದ್ಯಾ-ದೀಪಂ
ದದ್ಯಾದ್ಗುಡಂ
ದದ್ಯಾದ್ರಥ-ಸಪ್ತಮ್ಯಾಂ
ದದ್ಯಾದ್ವಿಜಾಯ
ದದ್ಯಾನ್ನಿದ್ರಾಂ
ದದ್ರುವುರ್ಭ-ಯತೋ
ದಧಿ
ದಧೀಚಿ-ಋಷಿ-ಗಳು
ದಧೀಚಿರ್ದರ್ಮೃತಿಂ
ದಧ್ನಾ
ದಧ್ನಾಂ
ದನ-ಗಳನ್ನು
ದನದ
ದನು-ವರ್ತತೇ
ದಬಿಂದವಃ
ದಯ-ದಿಂದ
ದಯ-ಪಾಲಿಸು
ದಯ-ಪಾಲಿಸುತ್ತಾನೆ
ದಯ-ಮಾಡಿರಿ
ದಯಾ
ದಯಾಂ
ದಯಾ-ಇವು-ಗಳಿಲ್ಲದೇ
ದಯಾನಿಧೇ
ದಯಾನ್ವಿ-ತೇನ
ದಯಾ-ಪರನೂ
ದಯಾ-ಪರಾ
ದಯಾ-ಮಾಪ
ದಯಾರ-ಹಿತ-ರಾಗಿ
ದಯಾ-ಲವಃ
ದಯಾಲುಃ
ದಯಾ-ಲು-ಗಳಾದ
ದಯಾ-ಲುಮ್
ದಯಾ-ಲೂನಾಂ
ದಯಾಲೋ
ದಯಾ-ಳು-ಗಳಾದ
ದಯಾವಂತರ
ದಯಾವಂತ-ರಿಗೂ
ದಯಾ-ಶಾಲಿ-ಗಳಾದ
ದಯಾ-ಶಾಲಿ-ಯಾದ
ದಯಾ-ಶಾಲಿಯೇ
ದಯಾ-ಸಾಗರನೇ
ದಯಾ-ಹೀನಾ
ದಯೆ
ದಯೆ-ಯಿಂದ
ದರಿದ್ರ
ದರಿದ್ರಃ
ದರಿದ್ರ-ನಾಗಿ
ದರಿದ್ರ-ರಾ-ದರು
ದರಿದ್ರೋ
ದರೋ-ಡೆ-ಗಳಲ್ಲಿ
ದರ್ಭ-ಪಾಣಿ-ಯಾಗಿ
ದರ್ಭೆ
ದರ್ಭೆ-ಗಳನ್ನು
ದರ್ಭೆಯ
ದರ್ಭೆ-ಯನ್ನು
ದರ್ಭೆ-ಯಲ್ಲಿ
ದರ್ಶನ
ದರ್ಶನಂ
ದರ್ಶನ-ಕೊಟ್ಟಿ-ರುವಿ
ದರ್ಶನ-ದಿಂದ
ದರ್ಶನ-ಭಾಗ್ಯ-ದಿಂದ
ದರ್ಶನ-ಮಾಡಿ
ದರ್ಶನ-ಮಾಡು-ವುದ-ರಿಂದ
ದರ್ಶ-ನಮ್
ದರ್ಶನ-ಲಾಭ-ದಿಂದ
ದರ್ಶನ-ವನ್ನು
ದರ್ಶ-ನವು
ದರ್ಶ-ನಾತ್
ದರ್ಶ-ನಾದನು
ದರ್ಶನಾ-ದೇವ
ದರ್ಶ-ಮಾತ್ರ-ದಿಂದಲೇ
ದರ್ಶ-ಯಂತಿ
ದರ್ಶಯಾ-ಮಾಸ
ದರ್ಶಶ್ರಾದ್ಧ
ದರ್ಶಶ್ರಾದ್ಧ-ದಲ್ಲಿ
ದರ್ಶೆಷು
ದಲ್ಲಿ
ದಳ-ವನ್ನು
ದವನು
ದಶ
ದಶ-ಪೂರ್ವೈಃ
ದಶ-ಪೂರ್ವೈರ್ದ-ಶಾಪರೈಃ
ದಶ-ಮಿ-ಯಲ್ಲಿ
ದಶಮೀ
ದಶ-ಮೋಧ್ಯಾಯಃ
ದಶಮ್ಯಾಂ
ದಶ-ವರ್ಷ-ಸಹಸ್ರಂ
ದಶ-ವರ್ಷ-ಸಹಸ್ರ-ಕಮ್
ದಶ-ವಾರಂ
ದಶಾಂಗ
ದಶಾಂಗಂ
ದಶಾಂಗ-ದಿಂದ
ದಶಾಕ್ಷಾಣಿ
ದಶಾಪರೈಃ
ದಶಾ-ವರಾಣಾಂ
ದಶೇಂದ್ರಿಯ-ಗಳು
ದಶೇಂದ್ರಿಯ-ಗಳೇ
ದಶೇಂದ್ರಿಯಾಣಿ
ದಶೈ-ಕಸ್ಮಿನ್
ದಶೋ
ದಸ್ತತ್ರ
ದಹತ್ಯಗ್ನಿರಿವೇಂಧ-ನಮ್
ದಹತ್ಯಾ
ದಹತ್ಯಾಶು
ದಹತ್ಯೇಷಾ
ದಾಂತಂ
ದಾಂತಸ್ತಪಸ್ವೀ
ದಾಂಭಿಕಃ
ದಾಟಿ
ದಾಟಿದ್ದಾರೆ
ದಾಟಿ-ಸಿ-ರುವರು
ದಾಟಿ-ಸುತ್ತಾನೆ
ದಾಟುವ-ನಲ್ಲದೆ
ದಾತವ್ಯಂ
ದಾತಾ
ದಾತಾರೌ
ದಾತಾ-ಸುಖಸ್ಯ
ದಾತುಂ
ದಾತು-ಕಾಮಾ
ದಾನ
ದಾನಂ
ದಾನ-ಕೊಟ್ಟರೆ
ದಾನ-ಗಳ
ದಾನ-ಗಳನ್ನು
ದಾನ-ಗಳನ್ನೂ
ದಾನ-ಗಳಿಂದ
ದಾನ-ದಲ್ಲಿ
ದಾನ-ದಿಂದ
ದಾನ-ಮಾಡ-ಬೇಕು
ದಾನ-ಮಾಡ-ಬೇಕೆಂದು
ದಾನ-ಮಾಡ-ಲಿಲ್ಲ
ದಾನ-ಮಾಡಿ
ದಾನ-ಮಾಡಿ-ದನು
ದಾನ-ಮಾಡಿ-ದರೆ
ದಾನ-ಮಾಡಿದೆ
ದಾನ-ಮಾಡುತ್ತಾನೋ
ದಾನ-ಮಾಡುತ್ತಾ-ರೆಯೋ
ದಾನ-ಮಾಡುತ್ತಿಯೋ
ದಾನ-ಮಾಡುವ
ದಾನ-ಮಾಡು-ವನೋ
ದಾನ-ಮುಕ್ತಿ-ವಿ-ವರ್ಜಿ-ತಮ್
ದಾನ-ಯಜ್ಞ
ದಾನ-ವನ್ನಾಗಿ
ದಾನ-ವನ್ನು
ದಾನ-ವ-ರಾಕ್ಷಸಾಃ
ದಾನ-ವರು
ದಾನ-ವಾಗಿ
ದಾನ-ವಿಲ್ಲ
ದಾನವು
ದಾನಸ್ವೀಸೀ-ಕಾರ
ದಾನಾದಿ-ಗಳನ್ನು
ದಾನಾದಿ-ಗಳು
ದಾನಾನಿ
ದಾನಾರ್ಥಮಸ್ಸಜದ್ವಿಭುಃ
ದಾನೇನ
ದಾಪಯ
ದಾಪಯಾ-ಮಾಸ
ದಾಪ-ಯಿತುಂ
ದಾಪಿತಂ
ದಾರಿ
ದಾರಿ-ಗಳಲ್ಲಿ
ದಾರಿ-ತಪ್ಪಿದ
ದಾರಿದ್ರ
ದಾರಿದ್ರ-ವನ್ನು
ದಾರಿದ್ರ್ಯಂ
ದಾರಿ-ಯನ್ನು
ದಾರಿ-ಯಲ್ಲಿ
ದಾರುಣ-ವಾದ
ದಾರುಣಾನ್
ದಾಸನೆಂದಾಗಲೀ
ದಾಸಿಯ
ದಾಸಿ-ಯನ್ನು
ದಾಸಿ-ಯರ
ದಾಸಿ-ಯಾಗಿಡು
ದಾಸಿ-ಯಿಂದ
ದಾಸಿ-ಯೆಂತಲೂ
ದಾಸೀ
ದಾಸೀತ್ವೇ
ದಾಸೀ-ಪುತ್ರಾಶ್ಚ
ದಾಸೀ-ರೂಪಂ
ದಾಸೋ-ಽಹಂ
ದಾಸ್ಯ
ದಿಂದಲೇ
ದಿಕ್ಕಿಗೆ
ದಿಕ್ಕಿ-ನಲ್ಲಿ
ದಿಕ್ಕಿ-ನಲ್ಲಿದ್ದ
ದಿಕ್ಕು-ಗಳ
ದಿಕ್ಕು-ಗಳಲ್ಲಿ
ದಿಕ್ಕು-ಗಳಲ್ಲಿಯೂ
ದಿಕ್ಕು-ಗಳೆಲ್ಲವೂ
ದಿಕ್ಕೇ
ದಿಕ್ಪಾಲ-ಕರು
ದಿಕ್ಷಾಲಕಾಸ್ತಥಾ
ದಿಗಂಬರ
ದಿಗಂಬರ-ನಾಗಿಯೂ
ದಿಗಂಬರ-ನಾದ
ದಿಗಂಬ-ರಮ್
ದಿಗಂಬರೋ
ದಿಗ್ದೇವ-ತೆ-ಗಳೂ
ದಿಗ್ವಿಜಯ
ದಿಗ್ವಿಜಯೇ
ದಿಗ್ವೇ-ವತಾಶ್ಚಂದ್ರ-ಮಾಶ್ಚ
ದಿತೇ
ದಿನ
ದಿನ-ಕರೇ
ದಿನಕ್ಕೆ
ದಿನ-ಗಳ
ದಿನ-ಗಳ-ಕಾಲ
ದಿನ-ಗಳನ್ನು
ದಿನ-ಗಳಲ್ಲಿ
ದಿನ-ಗಳು
ದಿನ-ಚರಿ
ದಿನತ್ರಯಂ
ದಿನತ್ರಯ-ಗಳಲ್ಲಿ
ದಿನತ್ರಯಮ್
ದಿನದ
ದಿನ-ದಲ್ಲಿ
ದಿನ-ದಿಂದ
ದಿನದ್ವಯ-ಕೃತಂ
ದಿನದ್ವಯ-ಫಲಂ
ದಿನದ್ವಯಸಮುದ್ಭ-ವಮ್
ದಿನ-ಮಾತ್ರೇಣ
ದಿನ-ವಾದರೂ
ದಿನವು
ದಿನವೂ
ದಿನವೇ
ದಿನಸ್ಯೈ-ಕಸ್ಯ
ದಿನಾನಿ
ದಿನೇ
ದಿನೇಷ್ವಪಿ
ದಿರ್ಘೋಷ್ಠ
ದಿಲೀ-ಪನೇ
ದಿಲೀಪಾದ್ಯಾ
ದಿವಂಗತ
ದಿವಸ
ದಿವಸಃ
ದಿವ-ಸ-ಗಳ
ದಿವ-ಸ-ಗಳನ್ನು
ದಿವ-ಸ-ಗಳಲ್ಲಿ
ದಿವ-ಸ-ಗಳಲ್ಲಿಯೂ
ದಿವ-ಸದ
ದಿವ-ಸ-ದಲ್ಲಿಯ
ದಿವ-ಸ-ವಾಗಿತ್ತು
ದಿವ-ಸ-ವಾದರೂ
ದಿವ-ಸವು
ದಿವಸೇ
ದಿವಾ-ಕರೇ
ದಿವಾಕ್ರೋ-ಶನ್
ದಿವಾನಿ-ಶಮ್
ದಿವಾರಾತ್ರಂ
ದಿವಿ
ದಿವ್ಯ
ದಿವ್ಯಂ
ದಿವ್ಯ-ಪುರುಷ-ರಿಂದ
ದಿವ್ಯ-ಪುರುಷರು
ದಿವ್ಯ-ಪುರುಷರೂ
ದಿವ್ಯ-ಪುರುಷಾ
ದಿವ್ಯೇ
ದಿಶ
ದಿಶಂ
ದಿಶ-ಮಾಪೇದೇ
ದಿಶಾಂ
ದಿಶೋ
ದಿಷ್ಟ್ಯಾಽಸ್ಮಾಕಂ
ದಿಸರ್ವ-ಧರ್ಮಾಣಾಂ
ದೀನ-ನಾಗಿ
ದೀನ-ನಾದ
ದೀನಸ್ತಥಾ
ದೀನೇ
ದೀಪ
ದೀಪಕ್ಕಾಗಿ
ದೀಪ-ಗಳನ್ನು
ದೀಪ-ದಾನ
ದೀಪ-ದಾನಂ
ದೀಪ-ದಾನದ
ದೀಪ-ದಾನೇ
ದೀಪವು
ದೀಪಾಯ
ದೀಪೋ
ದೀಯತೇ
ದೀರ್ಘಕಾಯ
ದೀರ್ಘಕಾಯೋ
ದೀರ್ಘಜಂಘ
ದೀರ್ಘಜಂಘೋ
ದೀರ್ಘ-ದೇಹಾ
ದೀರ್ಘಬಾಹು
ದೀರ್ಘಬಾಹುರ್ವಿ-ಲೋ-ಚನಃ
ದೀರ್ಘಾಯು
ದೀರ್ಘಾಯು-ವಾದ
ದೀರ್ಘಾಯುಷ್ಯಂ
ದೀರ್ಘೋಷ್ಠ
ದುಂಬಿ-ಗಳು
ದುಃಖ
ದುಃಖಂ
ದುಃಖಕ್ಕೆ
ದುಃಖ-ಗಳನ್ನು
ದುಃಖ-ದಿಂದ
ದುಃಖ-ನಾಶಕ್ಕೆ
ದುಃಖ-ಪಟ್ಟನು
ದುಃಖ-ಪಡಿ-ಸುವ
ದುಃಖ-ಪಡುವ-ವ-ರಂತೆ
ದುಃಖ-ಪೂರಿ-ತಮ್
ದುಃಖಪ್ರದ-ವಾದ
ದುಃಖ-ಭೂತ-ವಾದ
ದುಃಖ-ಮುಕ್ತಯೇ
ದುಃಖ-ಮೂಲಂ
ದುಃಖ-ಲೇಶಂ
ದುಃಖ-ಲೇಶವೂ
ದುಃಖ-ವನ್ನು
ದುಃಖ-ವಿ-ಮೋಕ್ಷ-ಣಮ್
ದುಃಖವು
ದುಃಖ-ವೆಂಬ
ದುಃಖಸ್ಯ
ದುಃಖಾತ್ತತ್ಯಜೇ
ದುಃಖಾ-ದಿ-ಗಳನ್ನು
ದುಃಖಾದ್ಬಾಷ್ಪಜಲಾ-ಕುಲಃ
ದುಃಖಾನಿ
ದುಃಖಾನ್ಮೃತಿಮವಾಪ
ದುಃಖಾಶ್ರು-ಗಳನ್ನು
ದುಃಖಿತಃ
ದುಃಖಿತ-ನಾಗಿ
ದುಃಖಿತ-ನಾಗಿ-ರುವು-ದನ್ನು
ದುಃಖಿ-ತಾಮ್
ದುಃಖಿತೇ
ದುಃಖಿಸಲೂ
ದುಃಖೀ
ದುಃಸಹಂ
ದುಗ್ಧ
ದುಗ್ಧ-ನದೀ-ಜಲೇ
ದುದ್ರುವತುಃ
ದುದ್ರುವ-ತುರ್ಭ-ಯಾತ್
ದುದ್ರುವುರ್ದ್ರು-ತಮ್
ದುದ್ರುವೇ
ದುರಹಂಕಾರ-ದಿಂದ
ದುರಾಗಮಾಃ
ದುರಾಚಾರ-ದಲ್ಲಿ
ದುರಾಚಾರಾ
ದುರಾ-ಚಾರಿ-ಗ-ಳಾಗಿ
ದುರಾ-ಚಾರಿಯೂ
ದುರಾ-ಚಾರೀ
ದುರಾತ್ಮನಾ
ದುರಾತ್ಮನಾಂ
ದುರಾತ್ಮನಾದ
ದುರಾತ್ಮನಾಮ್
ದುರಾತ್ಮರಾದ
ದುರಾತ್ಮ-ವಾನ್
ದುರಾತ್ಮಾಪಿ
ದುರಾಪಮನ್ಯೈಃ
ದುರ್ಗಂ
ದುರ್ಗಂಧ-ನಾಶಕ್ಕೋಸ್ಕರ
ದುರ್ಗಂಧವು
ದುರ್ಗತಿ
ದುರ್ಗತಿಃ
ದುರ್ಗತಿಯ
ದುರ್ಗ-ದಲ್ಲಿ
ದುರ್ಜನರ
ದುರ್ಜನ-ರಿಂದ
ದುರ್ಜ-ನಾನಾಂ
ದುರ್ಜಯಂ
ದುರ್ದಿನ
ದುರ್ದಿನಂ
ದುರ್ದಿನಮ್
ದುರ್ದಿನ-ವಲ್ಲ
ದುರ್ದೆಶೆಗೆ
ದುರ್ನಿಗ್ರಹಂ
ದುರ್ನಿಮಿತ್ತಂ
ದುರ್ನಿಮಿತ್ತಾನ್ಯಹಂ
ದುರ್ಬಲ-ನಾ-ಗಿ-ರುವ-ವನು
ದುರ್ಬಲೋ
ದುರ್ಬುದ್ದಿ-ಯಿಂದ
ದುರ್ಬುದ್ಧೇ
ದುರ್ಭಿಕ್ಷಂ
ದುರ್ಭಿಕ್ಷವು
ದುರ್ಮತಿಂ
ದುರ್ಮ-ರಣಕ್ಕೆ
ದುರ್ಮ-ರಣ-ವನ್ನು
ದುರ್ಯೊನಿಃ
ದುರ್ಯೊನಿಹೇತುನಾ
ದುರ್ಯೊನೇರ್ವದತಾಂ
ದುರ್ಯೋನಿ-ಯಲ್ಲಿ
ದುರ್ಲಭ
ದುರ್ಲಭರು
ದುರ್ಲಭ-ವಾದ
ದುರ್ಲಭಾ
ದುರ್ಲಭಾಸ್ತ್ವ
ದುರ್ವಾದಿ-ಗಳಿಗೆ
ದುರ್ವಿಚಾ-ರಣ-ತತ್ವ-ರಾನ್
ದುಷ್ಕರ-ಕಿಲ್ಬಿಷೈಃ
ದುಷ್ಕರ್ಮ
ದುಷ್ಕರ್ಮ-ಗಳಲ್ಲಿಯೂ
ದುಷ್ಕರ್ಮ-ಗಳು
ದುಷ್ಕರ್ಮದ
ದುಷ್ಕರ್ಮ-ಮೂಲಂ
ದುಷ್ಕರ್ಮಿ-ಗಳಿಂದ
ದುಷ್ಟ
ದುಷ್ಟಗ್ರಂಥ-ಗಳನ್ನು
ದುಷ್ಟ-ಚಿತ್ತೇ-ನಾತ್ಮಪ್ರಶಂಸಿನಾ
ದುಷ್ಟಧೀಃ
ದುಷ್ಟಪ್ರತಿಗ್ರ-ಹಮ್
ದುಷ್ಟ-ಮೃಗಾ-ಕುಲೇ
ದುಷ್ಟ-ವರ್ತನೆ-ಯಿಂದ
ದುಷ್ಯತಿ
ದುಸ್ರಾವ
ದುಹಿತಾ
ದುಹಿತುಃ
ದುಹಿತುಶ್ಚಾಪಿ
ದುಹಿತ್ರಾ
ದೂಡುತ್ತಾನೆ
ದೂತ
ದೂತ-ಕರ್ಮಾಣಂ
ದೂತರ
ದೂತ-ರನ್ನು
ದೂತ-ರಿಂದ
ದೂತ-ರಿಗೆ
ದೂತರು
ದೂತಾ
ದೂತಾಃ
ದೂತಾಸ್ತಂ
ದೂತಾಸ್ತಮಬ್ರುವನ್
ದೂತಿಕ
ದೂತಿ-ಕನು
ದೂತಿಕ-ನೆಂಬ
ದೂತಿಕೋ
ದೂತಿಗನು
ದೂತಿಗಾ
ದೂತಿಗೋ
ದೂತೀ
ದೂತೀ-ವಚಃ
ದೂತೈಸ್ತು
ದೂರ
ದೂರ-ದಲ್ಲಿ
ದೂರ-ದಲ್ಲಿ-ರುತ್ತವೆ
ದೂರ-ರಾಗಿದ್ದಾರೆ
ದೂರ-ವಿದ್ದ
ದೂರೇ
ದೂರೇಷು
ದೂರ್ವಾ-ಮೂಲ-ರಸಂ
ದೂರ್ವಾ-ಮೂಲಾನಿ
ದೂರ್ವಾ-ರಸಂ
ದೂರ್ವಾ-ರಸಃ
ದೂರ್ವಾವಲಾನ್ಯ
ದೂಷಣೆ
ದೂಷಿತಂ
ದೂಷಿ-ತಮ್
ದೂಷಿತಾಃ
ದೂಷಿತೋ
ದೂಷಿ-ಸಿದೆ
ದೂಷಿ-ಸುತ್ತಿದ್ದೆ
ದೃಅಷ್ಟ್ವಾಸ್ಮಿನ್
ದೃಢ-ವಾದ
ದೃಶ್ಯತೇ
ದೃಶ್ಯ-ಮಾನಂ
ದೃಶ್ಯ-ವನ್ನು
ದೃಷ್ಟಂ
ದೃಷ್ಟಃ
ದೃಷ್ಟ-ಮಾತ್ರಾಃ
ದೃಷ್ಟಾ
ದೃಷ್ಟಾಂತ-ಗಳೂ
ದೃಷ್ಟಾ-ತೀವ
ದೃಷ್ಟಿ
ದೃಷ್ಟಿಗೆ
ದೃಷ್ಟಿ-ಯಿಂದ
ದೃಷ್ಟಿರ್ಮಹಾತ್ಮನಾಮ್
ದೃಷ್ಟಿ-ಹೀನೌ
ದೃಷ್ಟೋ-ಽಯಂ
ದೃಷ್ಟ್ಯಾ
ದೃಷ್ಟ್ವಾ
ದೃಷ್ಟ್ವಾಚ
ದೃಷ್ಟ್ವಾ-ತೀವ
ದೃಷ್ಟ್ವಾ-ತೀವ-ನಿರ್ವಿಣ್ಣೋ
ದೃಷ್ಟ್ವಾ-ತೀವ-ನಿರ್ವೇದ-ಯಯೌ
ದೃಷ್ಟ್ವಾ-ಸರ್ವ-ಮಿದಂ
ದೃಷ್ಟ್ವೇಶಾನೀಂ
ದೃಷ್ಠಾ
ದೆಸೆ-ಯಿಂದ
ದೇಯ
ದೇಯಂ
ದೇಯ-ಮನರ್ಘ್ಯ-ವಸ್ತು
ದೇವ
ದೇವಂ
ದೇವ-ಖಾತಂ
ದೇವ-ಖಾ-ತಾತ್
ದೇವ-ಖಾ-ತಾನಿ
ದೇವ-ಗಣೈಃ
ದೇವ-ಗಾ-ಯಕಃ
ದೇವ-ಗುರುಂ
ದೇವ-ಗುರೋ
ದೇವತಾಂ
ದೇವ-ತಾಋಷೀನ್
ದೇವ-ತಾ-ಗಣಕ್ಕೆ
ದೇವ-ತಾನಾಂ
ದೇವ-ತಾಭಕ್ತಿಸ್ತಥಾ
ದೇವ-ತಾರ್ಚನೆ
ದೇವ-ತಾವಿಗ್ರಹ-ಗಳೂ
ದೇವ-ತಾ-ಸಂಬಂಧ-ವಾದ
ದೇವ-ತಾಸ್ತ್ವಿತರಾ
ದೇವ-ತೆ-ಗಳ
ದೇವ-ತೆ-ಗಳಂತೆ
ದೇವ-ತೆ-ಗಳನ್ನು
ದೇವ-ತೆ-ಗಳನ್ನೂ
ದೇವ-ತೆ-ಗಳಲ್ಲಿ
ದೇವ-ತೆ-ಗಳಿಂದ
ದೇವ-ತೆ-ಗಳಿಂದಲೂ
ದೇವ-ತೆ-ಗಳಿ-ಗಾಗಿ
ದೇವ-ತೆ-ಗಳಿಗೂ
ದೇವ-ತೆ-ಗಳಿಗೆ
ದೇವ-ತೆ-ಗಳು
ದೇವ-ತೆ-ಗಳೂ
ದೇವ-ತೆ-ಗಳೆಂದು
ದೇವ-ತೆ-ಗಳೆಂದೂ
ದೇವ-ತೆ-ಗಳೆಂದೆನಿ-ಸುವರು
ದೇವ-ತೆ-ಗಳೆಲ್ಲರೂ
ದೇವ-ತೆ-ಗಳೇ
ದೇವ-ತೆಗೆ
ದೇವ-ತೆಯ
ದೇವ-ತೆ-ಯಂತೆ
ದೇವ-ತೆ-ಯಾದ
ದೇವ-ತೆಯು
ದೇವ-ದಾನ-ವ-ಗಂಧರ್ವಾ
ದೇವ-ದೂತೈಸ್ತು
ದೇವದ್ವೇಷೋ
ದೇವನೇ
ದೇವ-ಪೂಜಾ-ಕಕ್ಷಪಾಲಿಃ
ದೇವ-ಭಕ್ತಾ
ದೇವ-ಯಜ್ಞ
ದೇವ-ಯಜ್ಞಃ
ದೇವ-ಯಾಜೀ
ದೇವ-ಯಾಜೀ-ತಿ-ನಾಮ್ನಾಹಂ
ದೇವರ
ದೇವ-ರಂತೆ
ದೇವ-ರ-ನಿಟ್ಟ
ದೇವ-ರನ್ನು
ದೇವ-ರ-ಪೂಜಾ
ದೇವ-ರಿಗೆ
ದೇವರೇ
ದೇವರ್ಷಯೊ
ದೇವರ್ಷಿ-ಗಳಲ್ಲಿ
ದೇವರ್ಷಿ-ಪಿತೃ-ತರ್ಪಣ-ವಮ್
ದೇವರ್ಷಿ-ಪಿತೄನುದ್ದಿಶ್ಯ
ದೇವರ್ಷೀಣಾಂ
ದೇವ-ಲಕಂ
ದೇವ-ಲೋಕಕ್ಕೆ
ದೇವ-ಲೋಕ-ದಲ್ಲಿ
ದೇವ-ಷಿ-ಗಳು
ದೇವಸ್ತದಾ
ದೇವಸ್ಥಾ-ನದ
ದೇವಸ್ಯ
ದೇವಸ್ವಂ
ದೇವಸ್ವಮ-ಪಾತ್ರೇ
ದೇವಸ್ವಹ-ರಣಂ
ದೇವಾ
ದೇವಾಃ
ದೇವಾದ್ಯೈರಪಿ
ದೇವಾ-ನನ್ಯಾನ್
ದೇವಾನಾಂ
ದೇವಾ-ನಾಮೃಣತಃ
ದೇವಾನ್
ದೇವಾನ್ಮುನೀ-ನಪಿ
ದೇವಾಭಂ
ದೇವಾಯ
ದೇವಾ-ಯಾರ್ಘಂ
ದೇವಾರ್ಥೇ
ದೇವಾ-ಲಯ
ದೇವಾ-ಲಯಂ
ದೇವಾ-ಲಯ-ಗಳಲ್ಲಿ-ರುವ
ದೇವಾ-ಲಯದ
ದೇವಾ-ಲಯ-ದಿಂದ
ದೇವಾ-ಲಯ-ವನ್ನು
ದೇವಾ-ಲಯ-ವಿತ್ತು
ದೇವಾ-ಲ-ಯವು
ದೇವಾ-ಲಯೇ
ದೇವಾಶ್ಚ
ದೇವಿಗೆ
ದೇವಿಯ-ರನ್ನು
ದೇವಿಯು
ದೇವೀ
ದೇವೇ
ದೇವೇಂದ್ರನ
ದೇವೇಂದ್ರನು
ದೇವೇಂದ್ರ-ಸಭಾಂ
ದೇವೇಂದ್ರಸ್ಯ
ದೇವೇ-ತರೋ
ದೇವೇಶಂ
ದೇವೇ-ಶ-ಮವ್ಯಯಮ್
ದೇವೇಶೇ
ದೇವೇಷು
ದೇವೋ
ದೇಶ
ದೇಶಂ
ದೇಶ-ಕಾಲಕ್ರಿಯಾಶ್ರ-ಯಾನ್
ದೇಶ-ಕಾಲಾದಿಕಾಂಸ್ತಥಾ
ದೇಶ-ಕಾಲಾನುಗಂ
ದೇಶ-ಕಾಲಾನು-ಸಾರ-ವಾಗಿ
ದೇಶಕ್ಕೆ
ದೇಶ-ಗಳಲ್ಲಿ
ದೇಶದ
ದೇಶ-ದಲ್ಲಿ
ದೇಶ-ವನ್ನು
ದೇಶೀ
ದೇಶೀ-ಯಾನು-ವಾಚ
ದೇಶೇ
ದೇಶೇಷು
ದೇಶೋ
ದೇಹ
ದೇಹಂ
ದೇಹಕ್ಕೆ
ದೇಹ-ಗಳನ್ನು
ದೇಹದ
ದೇಹ-ದಲ್ಲಿ
ದೇಹ-ದಲ್ಲಿಯೇ
ದೇಹ-ದಲ್ಲಿ-ರುವ
ದೇಹ-ದಿಂದ
ದೇಹ-ನಾಶಕ್ಕೂ
ದೇಹಭ್ರತ್
ದೇಹ-ವೆಂಬ
ದೇಹಾಂತರಾರಂಭ-ಕರ್ಮ
ದೇಹಾನಾಂ
ದೇಹಿ
ದೇಹಿನಾಂ
ದೇಹೇಂದ್ರಿಯ-ಗಳಿಗೆ
ದೇಹೇಂದ್ರಿಯ-ಮನಃಸಾಧ್ಯಂ
ದೈತ್ಯ
ದೈತ್ಯ-ರಿಗೆ
ದೈತ್ಯರು
ದೈತ್ಯಾನಾಂ
ದೈನ್ಯಾದ-ಥಾಬ್ರವೀತ್
ದೈವಕ್ಕಿಂತ
ದೈವತಃ
ದೈವ-ತಾನಿ
ದೈವ-ಮೇವ
ದೈವಯೋ-ಗ-ದಿಂದ
ದೈವಾಚ್ಚ
ದೈವಾತ್ಕಾಲೇ
ದೈವಾತ್ಸಂಗ-ತಿಮಾ-ಗತಾಃ
ದೈವಾ-ದೃಷ್ಟಿ-ಪಥಂ
ದೈವಾನ್ಮುದ್ಗಲ-ನಾಮಕಃ
ದೈವಿಧ್ಯೇನ
ದೈವೇಚ್ಛೆ-ಯಿಂದ
ದೈವೇಚ್ಛೆ-ಯಿಂದಲೇ
ದೈಶಿಕಮಾ-ಹೂಯ
ದೈಹಿಕ
ದೊಡ್ಡ
ದೊಡ್ಡ-ದಾಗಿ
ದೊಡ್ಡ-ದಾದ
ದೊರಕಿಸಿ
ದೊರಕಿಸಿ-ಕೊಡುತ್ತ-ವೆಯೋ
ದೊರಕಿ-ಸುತ್ತವೆ
ದೊರಕಿ-ಸುವುದೆಂದು
ದೊರೆತು
ದೊರೆಯಲಿದೆ
ದೊರೆಯ-ಲಿಲ್ಲ
ದೊರೆಯ-ಲಿಲ್ಲ-ವೆಂಬ
ದೊರೆ-ಯಲು
ದೊರೆಯಲೆಂಬ
ದೊರೆಯುತ್ತದೆ
ದೊರೆಯುತ್ತ-ದೆಯೋ
ದೊರೆಯುತ್ತವೆ
ದೊರೆಯುತ್ತಿದ್ದಿಲ್ಲ
ದೊರೆ-ಯುವ
ದೊರೆ-ಯುವುದಲ್ಲದೇ
ದೊರೆಯು-ವುದಿಲ್ಲ
ದೊರೆಯು-ವುದಿಲ್ಲ-ವಾದ-ಕಾರಣ
ದೊರೆಯು-ವುದು
ದೊರೆ-ಯುವುದೆಂದು
ದೊರೆ-ಯುವುದೋ
ದೋಷ
ದೋಷ-ಗಳನ್ನು
ದೋಷ-ಗಳಿರ-ಬಹುದು
ದೋಷ-ಗಳಿಲ್ಲ
ದೋಷ-ಗಳು
ದೋಷದ
ದೋಷ-ದಿಂದ
ದೋಷ-ವನ್ನು
ದೋಷವು
ದೋಷಾತ್
ದೋಷೇಣ
ದೌರ್ಭಾಗ್ಯವು
ದೌಹಿತ್ರ
ದೌಹಿತ್ರಃ
ದೌಹಿತ್ರಸ್ತಥಾ
ದ್ದು
ದ್ದೇವ-ಪದಾರೂಢಾಃ
ದ್ಯೂತ-ಪರಾ-ಯಣೌ
ದ್ರವ-ರೂಪೇಣ
ದ್ರವಿಡ-ದೇಶ-ದಲ್ಲಿ
ದ್ರವಿಣಂ
ದ್ರವಿಣಾ-ಭಾವಾತ್
ದ್ರವ್ಯ
ದ್ರವ್ಯದ
ದ್ರವ್ಯ-ಮತಿಕ್ರಮ್ಯೈವ
ದ್ರವ್ಯ-ವನ್ನು
ದ್ರಷ್ಟುಂ
ದ್ರಾವಿಡೇಷು
ದ್ರುತಂ
ದ್ರುಮೇ
ದ್ರೋಹದ
ದ್ವಯಂ
ದ್ವಯ-ಮೇವ
ದ್ವಯ-ವಮ್
ದ್ವಯೀ
ದ್ವಾ
ದ್ವಾತ್ರಿಂಶದಸರಾಥಾನಿ
ದ್ವಾದ-ಶನಾನ
ದ್ವಾದ-ಶಪುಂಡ್ರಾನ್
ದ್ವಾದ-ಶ-ವರ್ಣಿಷು
ದ್ವಾದ-ಶ-ವರ್ಷಂ
ದ್ವಾದ-ಶಾಖ್ಯೆ
ದ್ವಾದ-ಶಾತ್ಮಾ
ದ್ವಾದ-ಶಾಬ್ದಂ
ದ್ವಾದ-ಶಾರ್ಕಾ
ದ್ವಾದ-ಶಿ-ಗಳಲ್ಲಿ
ದ್ವಾದ-ಶಿಯ
ದ್ವಾದ-ಶಿ-ಯಲ್ಲಿ
ದ್ವಾದ-ಶಿಯು
ದ್ವಾದಶೀ
ದ್ವಾದ-ಶೀ-ಗಳಲ್ಲಿಯೂ
ದ್ವಾದ-ಶೀಗೆ
ದ್ವಾದ-ಶೀ-ನಾತಿಲಂಘಿನಃ
ದ್ವಾದ-ಶೀ-ಭೋ-ಜನಂ
ದ್ವಾದ-ಶೀಷು
ದ್ವಾದ-ಶೈ-ತಾನಿ
ದ್ವಾದ-ಶೋಧ್ಯಾಯಃ
ದ್ವಾದಶ್ಯಾಂ
ದ್ವಾದಶ್ಯಾಃ
ದ್ವಾಪರ
ದ್ವಾಪರ-ಯುಗದ
ದ್ವಾಪರಸ್ಯ
ದ್ವಾರಕಾ
ದ್ವಾರಕಾಂ
ದ್ವಾರ-ಪಾಲಕ-ನಾಗಿದ್ದನು
ದ್ವಾರ-ವತ್ಯಾ-ದಿಮೃತ್ಸ್ನಯಾ
ದ್ವಾರಸ್ಥಶಾಲಾವಲಭೇರಧೋ-ದೇಶೇ
ದ್ವಾರಾ
ದ್ವಾರಾ-ವತಿ-ಇವು-ಗಳಲ್ಲಿ
ದ್ವಾರಿ
ದ್ವಾವಪಶ್ಯದ್ಭಯಾನಕೌ
ದ್ವಿಗುಣಂ
ದ್ವಿಜ
ದ್ವಿಜಂ
ದ್ವಿಜಃ
ದ್ವಿಜ-ಕುಲೇ
ದ್ವಿಜ-ಜನ್ಮಸು
ದ್ವಿಜ-ನಿಗೆ
ದ್ವಿಜಮ್
ದ್ವಿಜರು
ದ್ವಿಜ-ವರ್ಯಾಯ
ದ್ವಿಜಶ್ಚೇದ್ದೃತಮಾ-ಗಚ್ಛ
ದ್ವಿಜಶ್ರೇಷ್ಠ
ದ್ವಿಜ-ಸತ್ತಮಃ
ದ್ವಿಜ-ಸತ್ತಮಾಃ
ದ್ವಿಜಾ
ದ್ವಿಜಾಃ
ದ್ವಿಜಾ-ತಯೇ
ದ್ವಿಜಾ-ಧಮಃ
ದ್ವಿಜಾ-ನತ್ರಾಗ-ತಾನ್
ದ್ವಿಜಾ-ನಾಕ್ಷಿಪ್ಯ
ದ್ವಿಜಾನ್
ದ್ವಿಜಾಯ
ದ್ವಿಜೇಂದ್ರಂ
ದ್ವಿಜೋ
ದ್ವಿಜೋತ್ತಮ
ದ್ವಿಜೋತ್ತಮಾಃ
ದ್ವಿತೀಯ
ದ್ವಿತೀಯೇ
ದ್ವಿತೀಯೋ
ದ್ವಿತೀಯೋಧ್ಯಾಯಃ
ದ್ವಿತೀಯೋ-ಭದ್ರ-ಸಂಜ್ಞ
ದ್ವಿತೀಯೋಯಂ
ದ್ವಿದಲ
ದ್ವಿದಲಾನಿ
ದ್ವಿವಾರಂ
ದ್ವಿಷಟ್ಸು
ದ್ವೇಷ
ದ್ವೇಷ-ವಿಲ್ಲ
ದ್ವೇಷಿಸಿ
ದ್ವೇಷಿ-ಸುತ್ತಿದ್ದೆ
ದ್ವೇಷಿ-ಸುವ-ವರು
ದ್ವೌ
ಧತಮಸ್ಯ
ಧನ
ಧನಂ
ಧನ-ಕಾಂಕ್ಷಯಾ
ಧನದ
ಧನಮತ್ಯಂತಂ
ಧನ-ಮಾ-ದಾಯ
ಧನಮ್
ಧನ-ಲೋ-ಭೇನ
ಧನ-ವನ್ನು
ಧನ-ವಸ್ತ್ರ-ದಾನೈ
ಧನಸ್ಯಾರ್ಜನ-ಚಿತ್ತೇನ
ಧನಸ್ಯಾರ್ಜನ-ತತ್ಪರಃ
ಧನಾಢ್ಯೈಶ್ಚ
ಧನಾಢ್ಯೋ
ಧನಾದಿ
ಧನಾಧಿಪಃ
ಧನಿಕರ
ಧನಿನಾಂ
ಧನಿನೊಸ್ಮಾಭಿರ್ಯುಗ-ಪತ್ಸಪ್ತ
ಧನೀ
ಧನುರ್ಬಾಣ-ಗಳನ್ನು
ಧನುರ್ಬಾಣಧರಾಂತಕಾಃ
ಧನ್ಯರು
ಧನ್ಯಾಃ
ಧರಾ-ದೇವಿಯ
ಧರಿಸ-ಬೇಕು
ಧರಿಸಿ
ಧರಿಸಿ-ರುವ
ಧರಿ-ಸುವ-ವರು
ಧರಿ-ಸು-ವುದ-ರಿಂದ
ಧರಿ-ಸು-ವುದು
ಧರ್ಮ
ಧರ್ಮಂ
ಧರ್ಮಃ
ಧರ್ಮಕ್ಕೆ
ಧರ್ಮ-ಗಳ
ಧರ್ಮ-ಗಳನ್ನು
ಧರ್ಮ-ಗಳನ್ನೂ
ಧರ್ಮ-ಗಳನ್ನೇ
ಧರ್ಮ-ಗಳಲ್ಲಿ
ಧರ್ಮ-ಗಳಲ್ಲಿಯ
ಧರ್ಮ-ಗಳಿ-ಗಿಂತಲೂ
ಧರ್ಮ-ಗಳಿಗೆ
ಧರ್ಮ-ಗಳು
ಧರ್ಮ-ಗಳೂ
ಧರ್ಮ-ಗಳೇನೋ
ಧರ್ಮ-ಚಿತ್ತಾ
ಧರ್ಮಜ್ಞಾನ್
ಧರ್ಮದ
ಧರ್ಮ-ದಲ್ಲಿ
ಧರ್ಮ-ದಲ್ಲಿಯೇ
ಧರ್ಮ-ಪತ್ನಿ
ಧರ್ಮ-ಪತ್ನ್ಯಾ
ಧರ್ಮ-ಪರಾಙ್ಮುಖೌ
ಧರ್ಮ-ಪ-ರಾಜ್
ಧರ್ಮ-ಪರಾ-ಯಣಾಃ
ಧರ್ಮಪ್ರ-ವಕ್ತಾ-ರಮನಂತ-ಬುದ್ಧಿಮ್
ಧರ್ಮ-ಫಲಂ
ಧರ್ಮ-ಬುದ್ಧಿ-ಯಿಂದ
ಧರ್ಮ-ಬೋಧನೆ-ಯನ್ನು
ಧರ್ಮ-ಭಟಾ
ಧರ್ಮ-ಮಪ್ಯಾಚರೇನ್ನ
ಧರ್ಮ-ಮಾರ್ಗ-ದಿಂದ
ಧರ್ಮ-ರತಿರ್ನೈವ
ಧರ್ಮ-ರಾಡಿ-ಮಮ್
ಧರ್ಮ-ವನ್ನರಿತ
ಧರ್ಮ-ವನ್ನು
ಧರ್ಮ-ವನ್ನೂ
ಧರ್ಮ-ವಾಕ್ಯೈಃ
ಧರ್ಮ-ವಾಗಿದೆ
ಧರ್ಮ-ವಾದೀ
ಧರ್ಮ-ವಿದಃ
ಧರ್ಮ-ವಿ-ಮುಖಾ
ಧರ್ಮ-ವಿಲ್ಲ
ಧರ್ಮ-ವಿ-ವರ್ಜಿತಃ
ಧರ್ಮವು
ಧರ್ಮ-ವೆಂತಲೂ
ಧರ್ಮ-ವೆಂದು
ಧರ್ಮವೇ
ಧರ್ಮ-ಶಾಸ್ತ್ರ
ಧರ್ಮ-ಶಾಸ್ತ್ರ-ವನ್ನು
ಧರ್ಮಶ್ರೇಷ್ಠರು
ಧರ್ಮ-ಸಂಜ್ಞ
ಧರ್ಮ-ಸಹಸ್ರಾಣಿ
ಧರ್ಮಸ್ಯ
ಧರ್ಮಾ
ಧರ್ಮಾಃ
ಧರ್ಮಾ-ಚ-ರಣೆ-ಯನ್ನು
ಧರ್ಮಾ-ಚ-ರಣೆಯು
ಧರ್ಮಾಣಾಂ
ಧರ್ಮಾ-ಣಾಮಾಶ್ರಮಾಣಾಂ
ಧರ್ಮಾದಿ
ಧರ್ಮಾ-ಧರ್ಮ-ಗಳು
ಧರ್ಮಾ-ಧಿಕಾ-ರಮಾ-ಪನ್ನಃ
ಧರ್ಮಾ-ಧಿ-ಕಾರಿ
ಧರ್ಮಾ-ನಪೇಕ್ಷ್ಯ
ಧರ್ಮಾನ್ಯ
ಧರ್ಮಾರ್ಥಂ
ಧರ್ಮಾರ್ಥಕಾ-ಮ-ಗಳು
ಧರ್ಮಾರ್ಥ-ಕಾಮಾ
ಧರ್ಮಾರ್ಥ-ವಾಗಿ
ಧರ್ಮಾಶ್ಚ
ಧರ್ಮಾಸ್ತು
ಧರ್ಮೇ
ಧರ್ಮೇಷು
ಧರ್ಮೋ
ಧರ್ಮೋನ
ಧಾತವೋ
ಧಾತು-ಗಳು
ಧಾತುನಾ
ಧಾತುವಿ-ಕಾರೈಶ್ಚ
ಧಾತೂ-ನಾಮಗ್ನಿಸಂಸ್ಕಾರೋ
ಧಾತ್ರಿ-ಫಲಾನಿ
ಧಾತ್ರೀಂ
ಧಾತ್ರೀಪಿಷ್ಟೇನ
ಧಾತ್ರೀ-ಫಲ-ಗೈರ್ದಿಪೈಃ
ಧಾತ್ರೀ-ಫಲ-ಮಾಲಾ-ಭಿರ್ಮಾಧವಂ
ಧಾತ್ರೀ-ಫಲ-ರಾಶಿಸ್ತು
ಧಾನಕಾತರಃ
ಧಾನ್ಯಂ
ಧಾನ್ಯ-ಗಳನ್ನೂ
ಧಾನ್ಯ-ಗಳು
ಧಾರಣ
ಧಾರಣ-ಇವು-ಗಳನ್ನು
ಧಾರಣೆ
ಧಾರ-ಯೇತ್ತತಃ
ಧಾರಾ-ಕಾರ-ವಾಗಿ
ಧಾರಾಘೃ-ತಾದೀನಾಂ
ಧಾರಾ-ಪೂರ್ವಂ
ಧಾರೆಯೆರದು
ಧಾರೆಯೆರೆದು
ಧಾವತೋ
ಧಾವತ್ಯೇವಂ
ಧಾವ-ಮಾನಂ
ಧಿಕ್ಕರಿಸಿ
ಧಿಕ್ಕರಿ-ಸು-ವುದು
ಧೀಃ
ಧೀಮತಾ
ಧೀಮತೇ
ಧೀಮತೋ
ಧೀಮಾನ್
ಧೀರಾಃ
ಧೂಪ
ಧೂಪಂ
ಧೂಪದ
ಧೂಪ-ದಾತಾರೋ
ಧೂಪ-ದಿಂದ
ಧೂಪ-ದೀಪ-ಗಳನ್ನು
ಧೂಪ-ವನ್ನು
ಧೂಪ-ಹೀನಾಂ
ಧೂಪೇನ
ಧೂಪೋ
ಧೂಮ್ರಾಶ್ಚ
ಧೃತ್ವಾ
ಧೃಷ್ಟ-ಕೇತು-ವಿನ
ಧೃಷ್ಟಕೇತೋಃ
ಧೈರ್ಯ-ದಿಂದ
ಧೌತ-ವಸ್ತ್ರಃ
ಧೌತಾ
ಧ್ಯಾ
ಧ್ಯಾತ್ವಾ
ಧ್ಯಾನ
ಧ್ಯಾನ-ಕರ್ಮಣಿ
ಧ್ಯಾನಕ್ಕೆ
ಧ್ಯಾನ-ಪರಾ-ಯಣಃ
ಧ್ಯಾನ-ಮಾಡಿ
ಧ್ಯಾನ-ಮಾಡುತ್ತಾ
ಧ್ಯಾನ-ಮಾತ್ರವೇ
ಧ್ಯಾನ-ಮೇವ
ಧ್ಯಾನಾ-ಸಕ್ತ-ರಾದ
ಧ್ಯಾನಿ-ಸ-ಬೇಕು
ಧ್ಯಾನಿ-ಸ-ಲಿಲ್ಲ
ಧ್ಯಾನಿ-ಸುತ್ತಲಿ-ರುವ
ಧ್ಯಾಯೇತ್
ಧ್ರುವಂ
ಧ್ರುವಮ್
ಧ್ವಂಸ
ಧ್ವನಾ
ಧ್ವನಿ
ಧ್ವನಿ-ಗೈಯುತ್ತಿದ್ದವು
ಧ್ವನಿ-ಬರುತ್ತಿತ್ತು
ಧ್ವನಿ-ಯಿಂದ
ಧ್ವನಿ-ಯುಳ್ಳ
ಧ್ವಾತ್ವಾ
ನ
ನಂ
ನಂತರ
ನಂದ-ವತೀ
ನಂದ-ವತ್ಯಾಂ
ನಂದಿಗ್ರಾಮ-ದಲ್ಲಿ
ನಂದಿಗ್ರಾಮೇ
ನಂದಿನೀ
ನಂಬಿ
ನಂಬಿಕೆ
ನಂಬಿ-ಕೆ-ಯಿಂದ
ನಃ
ನಕ್ಷತ್ರ-ಗಳನ್ನು
ನಕ್ಷತ್ರ-ದಿಂದ
ನಕ್ಷತ್ರಾಣಿ
ನಗರ-ಗಳಲ್ಲಿ
ನಗರದ
ನಗರ-ದಲ್ಲಿದ್ದು
ನಗರ-ದಲ್ಲಿದ್ದೆವು
ನಗರ-ವಿತ್ತು
ನಗರೀ-ಪಟ್ಟ-ಣಾದಿಷು
ನಗರೇ
ನಟರು
ನಟಾ-ದಯಃ
ನಡತೆ-ಯುಳ್ಳ
ನಡುಗುತ್ತಾ
ನಡುವೆ
ನಡೆದು
ನಡೆ-ಯಲು
ನಡೆಯಿ-ಸುತ್ತಾನೆ
ನಡೆಯುತ್ತದೆ
ನಡೆಯುತ್ತವೆ
ನಡೆಯುತ್ತಾನೆ
ನಡೆಯುತ್ತಿದ್ದ
ನಡೆಯುತ್ತಿ-ರುವಾಗ
ನಡೆಯುತ್ತೇನೆ
ನಡೆ-ಯುವ
ನಡೆಯು-ವಂತೆ
ನಡೆಸ-ಬೇಕು
ನಡೆಸ-ಬೇಕೆಂದು
ನದಿ
ನದಿ-ಗಳ
ನದಿ-ಗಳನ್ನು
ನದಿ-ಗಳಲ್ಲಿ
ನದಿ-ಗಳು
ನದಿ-ಗಳೂ
ನದಿಗೆ
ನದಿಯ
ನದಿ-ಯನ್ನು
ನದಿ-ಯಲ್ಲಿ
ನದಿ-ಯಿಂದ
ನದಿಯು
ನದಿಯೇ
ನದೀ
ನದೀಂ
ನದೀಃ
ನದೀ-ಜಲೇ
ನದೀ-ಜ-ಲೇನ
ನದೀ-ತಲೇ
ನದೀ-ತೀ-ರಕ್ಕೆ
ನದೀ-ತೀರ-ಜಲೇ
ನದೀ-ತೀರ-ದಲ್ಲಿ
ನದೀ-ತೀರ-ದಲ್ಲಿತ್ತು
ನದೀ-ತೀರೇ
ನದೀ-ಮನ್ಯಾಂ
ನದೀಮ್
ನದೀ-ವರ
ನದೀ-ವರೋ
ನದೀಸ್ತಾಃ
ನದ್ಯಃ
ನದ್ಯಭಿ-ಮಾನಿ
ನದ್ಯಾಂ
ನದ್ಯೋ
ನನಂ
ನನಗೂ
ನನಗೆ
ನನ-ಗೋಸ್ಕರ-ವಾಗಿ
ನನಾಮ
ನನ್ನ
ನನ್ನನ್ನಾದರೂ
ನನ್ನನ್ನು
ನನ್ನಲ್ಲಿ
ನನ್ನಲ್ಲಿದ್ದ
ನನ್ನ-ವನು
ನನ್ನಾಜ್ಞೆ-ಯಿಂದಲೇ
ನನ್ನಿಂದ
ನನ್ನಿಂದಲೇ
ನನ್ನೊಡನೆ
ನಭೋ
ನಮ
ನಮಗೆ
ನಮಗೆಲ್ಲ
ನಮಸ್ಕರಿಸಿ
ನಮಸ್ಕರಿಸಿ-ಕೊಳ್ಳಲ್ಪಟ್ಟ
ನಮಸ್ಕರಿಸಿ-ದವು
ನಮಸ್ಕರಿಸಿ-ದುವು
ನಮಸ್ಕರಿ-ಸಿದೆ
ನಮಸ್ಕರಿಸುತ್ತ
ನಮಸ್ಕರಿಸುತ್ತೇನೆ
ನಮಸ್ಕಾರ
ನಮಸ್ಕಾರ-ಗಳು
ನಮಸ್ಕಾರದ
ನಮಸ್ಕಾರ-ಮಾಡಿ-ದರು
ನಮಸ್ಕಾರ-ವನ್ನು
ನಮಸ್ಕೃತ್ಯ
ನಮಸ್ತೇ
ನಮಸ್ತೇಸ್ತು
ನಮಾಮ್ಯ-ಹಮ್
ನಮೋ
ನಮೋಸ್ತು
ನಮೋ-ಽಸ್ತು
ನಮ್ಮ
ನಮ್ಮನ್ನು
ನಮ್ಮನ್ನೂ
ನಮ್ಮಲ್ಲಿ
ನಮ್ಮಿಂದ
ನಮ್ಮಿಂದಲೂ
ನಮ್ಮಿಬ್ಬರ
ನಮ್ಮಿಬ್ಬ-ರಿಗೆ
ನಮ್ಮೆಲ್ಲರ
ನಮ್ಮೆಲ್ಲ-ರಿಗೂ
ನಮ್ರಧಿಯಾಂ
ನಯ
ನಯತಿ
ನಯಥ
ನಯನೇ
ನಯಾಮೈನಂ
ನಯೇದ್ಭೋಗಾದಾಲಸಾದ್ವಾ
ನಯೇದ್ಯದಿ
ನರಃ
ನರಕ
ನರಕಂ
ನರಕಕ್ಕೆ
ನರಕ-ಗಳಲ್ಲಿ
ನರಕ-ದಲ್ಲಿ
ನರಕ-ದಲ್ಲಿ-ರುತ್ತಾರೆ
ನರಕ-ದಿಂದ
ನರಕ-ಮಾಪ್ನೋತಿ
ನರಕ-ವನ್ನು
ನರಕ-ವಾಸವು
ನರಕಾಗ್ನಿನಾ
ನರಕಾದಿ
ನರಕಾದಿ-ಗಳನ್ನು
ನರಕಾನ್
ನರಕಾನ್ನ
ನರಕೇ
ನರ-ಕೇಷು
ನರನಾರಾ-ಯಣಾಶ್ರಮ-ದಲ್ಲಿ
ನರನಾರಾ-ಯಣಾಶ್ರಮೇ
ನರಶ್ಚಾಂಡಾಲತಾಂ
ನರಸಿಂಹ
ನರಸಿಂಹ-ಬೀಜಾಕ್ಷರ
ನರಸಿಂಹೋಽಖಿಲಾಜ್ಞಾ-ನ-ವತಧ್ವಾಂತ-ದಿವಾ-ಕರಃ
ನರಾ
ನರಾಃ
ನರಾ-ಕಾರಾ
ನರಾಣಾಂ
ನರಾ-ಧಮಃ
ನರಾ-ಧಮರೆಲ್ಲರೂ
ನರಾ-ಧಮಾಃ
ನರಾ-ಧಮೌ
ನರಾ-ನೇ-ತಾನ್
ನರಿ-ಗಳು
ನರಿ-ಯಾಗಿ
ನರೇಣ
ನರೈಃ
ನರೋ
ನರ್ತನ
ನರ್ತ-ವಿದ್ಯಾ-ರತಿಶ್ಚ
ನರ್ಮದಾ
ನಲವತ್ತು
ನಲ-ವತ್ತೈದು
ನಲಿನೀ
ನವಂತಿ
ನವನೀತಂ
ನವಮೀ
ನವಮೋಧ್ಯಾಯಃ
ನವಮ್ಯಾಂ
ನವ್ರ-ಮುಖೋ
ನಶಿಸಿ-ಹೋಗುತ್ತವೆ
ನಶ್ಯಂತಿ
ನಶ್ಯತಿ
ನಶ್ಯತೇ
ನಷ್ಟ
ನಷ್ಟಂ
ನಷ್ಟ-ವಾಗಿ
ನಷ್ಟ-ವಾಗುತ್ತದೆ-ಯೆಂಬ
ನಷ್ಟ-ವಾಗುತ್ತವೆ
ನಷ್ಟ-ವಾ-ಯಿತು
ನಾ
ನಾಕರೋನ್ಮಾಘ-ಮಾಸೇ
ನಾಕೃತಂ
ನಾಗ
ನಾಗ-ಇವು-ರು-ಗಳಿಗೆ
ನಾಗಾಃ
ನಾಗಾಹ್ವಯ-ವೆಂಬ
ನಾಗಾಹ್ವಯೇ
ನಾಗಿದ್ದಾನೆ
ನಾಗುತ್ತಾನೆ
ನಾಗುವನು
ನಾಗೇಭ್ಯಃ
ನಾಗ್ನ್ಯರ್ಕಸೋ-ಮಾನಿ-ಲ-ಬಂಧ-ದೇಹ
ನಾಘ್ಯಾ
ನಾಚಿಕೆ-ಯಿಂದ
ನಾತಿ-ದೂರೇ
ನಾತ್ಮಾರ್ಥ-ಮಿತಿ
ನಾತ್ಯಜಂ
ನಾತ್ರ
ನಾಥ
ನಾದ-ವನು
ನಾಧರ್ಮೇ
ನಾಧಿ-ಗಚ್ಛಂತಿ
ನಾಧೀತಂ
ನಾಧ್ಯಾ-ಪಿತಮಥಾಪಿ
ನಾನಾ
ನಾನಾಖ್ಯಾನ-ಪುರಃಸರಾಃ
ನಾನಾ-ಜನ-ಗಳಿಗೆ
ನಾನಾ-ಜನ-ಪದಾಶ್ರಯಮ್
ನಾನಾ-ದರೋ
ನಾನಾ-ಬಗೆಯ
ನಾನಾ-ಭಿಧಾಶ್ಚಾಪಿ
ನಾನಾ-ವಿಧಂ
ನಾನಾ-ವಿಧ-ದಿಂದ
ನಾನಾ-ವಿಧ-ವಾದ
ನಾನಾ-ವಿಧಾಂಕಾಮಾ-ನಾಪತುರ್ಮೋಕ್ಷ-ಮವ್ಯಯಮ್
ನಾನು
ನಾನುಷ್ಠಾತಾ
ನಾನುಷ್ಠಿ
ನಾನುಷ್ಠಿತಂ
ನಾನೂ
ನಾನೃತಂ
ನಾನೇ
ನಾನೊಬ್ಬನೇ
ನಾನ್ಯ
ನಾನ್ಯಥಾ
ನಾನ್ಯ-ಥೇದಂ
ನಾನ್ಯ-ಥೇಯಂ
ನಾನ್ಯ-ಭಾವಸಮಾಶ್ರಯಾಃ
ನಾನ್ಯೋ
ನಾನ್ಯೋ-ಽಸ್ತ್ಯು
ನಾಪಶ್ಯತ್
ನಾಪಿ
ನಾಪಿ-ತೇಷು
ನಾಪುತ್ರಾ
ನಾಪ್ನೋತಿ
ನಾಪ್ನೋತ್ಯ
ನಾಭೂತ್ಸಂತಾನ-ಮುತ್ತಮಮ್
ನಾಭೂದ್ವೈರಂ
ನಾಮ
ನಾಮಕಃ
ನಾಮ-ಗಳನ್ನು
ನಾಮ-ಗಳಿಂದ
ನಾಮ-ಗೋತ್ರ-ಗಳನ್ನು
ನಾಮ-ಗೋತ್ರಾಣಿ
ನಾಮ-ದೇವ
ನಾಮ-ದೇವರ
ನಾಮ-ಮುದ್ರೆ-ಗಳನ್ನು
ನಾಮ-ವನ್ನು
ನಾಮಾನಿ
ನಾಮ್ನಾ
ನಾಮ್ನಾಗ್ನಿ-ದೂತ
ನಾಮ್ನಾಹಂ
ನಾಮ್ಮಾ
ನಾಯಂ
ನಾಯಮರ್ಹತಿ
ನಾಯಾಂತಿ
ನಾಯಿ
ನಾಯಿ-ಯಾಗಿ
ನಾಯಿ-ಯೆಂದು
ನಾರಕಾಃ
ನಾರಕೀಂ
ನಾರದ
ನಾರದಂ
ನಾರದಃ
ನಾರದನೇ
ನಾರದರ
ನಾರದ-ರಿಗೆ
ನಾರ-ದರು
ನಾರದರೇ
ನಾರದಾದ್ಯಾಃ
ನಾರಾ-ಯಣಂ
ನಾರಾ-ಯಣನ
ನಾರಾ-ಯಣ-ನನ್ನು
ನಾರಾ-ಯಣನು
ನಾರಾ-ಯಣಪ್ರಸಾ-ದನ್ನು
ನಾರಾ-ಯಣಸ್ಸಾ
ನಾರೀ
ನಾರೀಭಿಃ
ನಾರೀ-ರಜೋಮಲಮ್
ನಾರ್ಚಿತಾ
ನಾರ್ಚಿತೌ
ನಾರ್ಜ-ಯಿತುಂ
ನಾರ್ಯಃ
ನಾರ್ಹಂತಿ
ನಾರ್ಹಸಿ
ನಾರ್ಹೇರ್ಹೇ
ನಾಲಂ
ನಾಲಿಗೆ
ನಾಲಿಗೆ-ಯನ್ನು
ನಾಲಿಗೆಯು
ನಾಲ್ಕನೆ
ನಾಲ್ಕನೆಯ
ನಾಲ್ಕನೆ-ಯ-ವನು
ನಾಲ್ಕನೇ
ನಾಲ್ಕರಷ್ಟು
ನಾಲ್ಕು
ನಾಳೆ
ನಾಳೆಯೋ
ನಾವಕಾಶೋ
ನಾವಿಬ್ಬರೂ
ನಾವು
ನಾವೂ
ನಾವೆಲ್ಲ
ನಾವೆಲ್ಲರೂ
ನಾವೇ
ನಾಶ
ನಾಶಕ್ಕಾಗಿ
ನಾಶಕ್ಕೂ
ನಾಶ-ಗೊಳಿ-ಸುವ
ನಾಶ-ಗೊಳಿ-ಸುವಲ್ಲಿ
ನಾಶ-ಗೊಳಿ-ಸುವ-ವ-ರಲ್ಲಿ
ನಾಶ-ಮಾಡಲು
ನಾಶ-ಮಾಡಿ-ದನು
ನಾಶ-ಮಾಡುತ್ತದೆ
ನಾಶ-ಮಾಡುತ್ತವೆ
ನಾಶ-ಮಾಡುತ್ತಾ-ನೆಯೋ
ನಾಶ-ಮಾಡುತ್ತೇವೆ-ಯೆಂದು
ನಾಶ-ಮಾಡುವ
ನಾಶ-ಮಾಡು-ವಂತೆ
ನಾಶ-ಮಾಡು-ವುದ-ರಲ್ಲಿ
ನಾಶ-ಮಾಡು-ವುದಿಲ್ಲ
ನಾಶ-ಮಾಡು-ವುದು
ನಾಶ-ಮಾ-ಯಾತಿ
ನಾಶ-ರ-ಹಿತ-ನಾದ
ನಾಶ-ರ-ಹಿತ-ವಾದ
ನಾಶ-ವಾ-ಗತಃ
ನಾಶ-ವಾಗದೇ
ನಾಶ-ವಾ-ಗಲು
ನಾಶ-ವಾಗುತ್ತದೆ
ನಾಶ-ವಾಗುತ್ತವೆ
ನಾಶ-ವಾಗು-ವಂತೆಯೂ
ನಾಶ-ವಾಗು-ವುದಲ್ಲದೇ
ನಾಶ-ವಾಗು-ವುದಿಲ್ಲ
ನಾಶ-ವಾ-ದುವು
ನಾಶ-ವಾ-ಯಿತು
ನಾಶ-ವೆಂಬುದಿಲ್ಲ
ನಾಶವೇ
ನಾಶ-ಹೊಂದಿದ
ನಾಶ-ಹೊಂದುತ್ತ-ದೆಯೋ
ನಾಶ-ಹೊಂದುತ್ತವೆ
ನಾಸನ್
ನಾಸೀತ್ಕೋ-ಽಪಿ
ನಾಸೀದ್ರಾಜ್ಯೇ
ನಾಸ್ತಿ
ನಾಸ್ಮಾಭಿಃ
ನಾಸ್ವಾದಿತೋ
ನಾಸ್
ನಾಹಮೇ-ತತ್ಸುಖಂ
ನಾಽಗಾತ್
ನಿಂತನು
ನಿಂತರು
ನಿಂತಿ-ರುತ್ತೇವೆ
ನಿಂತು
ನಿಂತು-ಕೊಂಡನು
ನಿಂತು-ಕೊಂಡಿದ್ದಳು
ನಿಂತು-ಕೊಂಡಿದ್ದಾನೆ
ನಿಂತು-ಕೊಂಡು
ನಿಂದಕ
ನಿಂದಕಂ
ನಿಂದಕಃ
ನಿಂದ-ಕನು
ನಿಂದ-ಕ-ನೆಂಬ
ನಿಂದಕೊ
ನಿಂದಕೋ
ನಿಂದತಿ
ನಿಂದನೆ
ನಿಂದ-ಯತ್ಯಯಂ
ನಿಂದ-ಯನ್
ನಿಂದಯಾ
ನಿಂದಾಂ
ನಿಂದಿತ-ಮತಿ
ನಿಂದಿತ-ರಾಗಿ
ನಿಂದಿತ-ವಸ್ತು-ಗಳನ್ನು
ನಿಂದಿತ-ವಾಗಿದ್ದರೂ
ನಿಂದಿತ-ವಾಗಿ-ರು-ವುದ-ರಿಂದ
ನಿಂದಿತ-ವಾದ
ನಿಂದಿತಾಃ
ನಿಂದಿ-ತಾನಾಂ
ನಿಂದಿತಾಶ್ರಯ-ನಮ್
ನಿಂದಿತೋ
ನಿಂದಿತೋಯಂ
ನಿಂದಿತೋಸ್ಮಾಭಿಃ
ನಿಂದಿಸಲ್ಪಟ್ಟನು
ನಿಂದಿಸಿ
ನಿಂದಿ-ಸಿದ
ನಿಂದಿಸಿ-ದನು
ನಿಂದಿ-ಸಿದೆ
ನಿಂದಿಸಿ-ದೆ-ನಾದ
ನಿಂದಿ-ಸುತ್ತಾನೆಯೋ
ನಿಂದಿ-ಸುತ್ತಿದ್ದ
ನಿಂದಿ-ಸುತ್ತಿದ್ದೆ
ನಿಂದಿ-ಸುವ-ವರು
ನಿಂದಿ-ಸು-ವುದು
ನಿಂದೆ
ನಿಂದೆಯ
ನಿಂದೆ-ಯನ್ನು
ನಿಂದ್ಯ
ನಿಃಪುನಂತಿ
ನಿಃಶಂಕ-ರೆಂಬ
ನಿಃಶಂಕಾಖ್ಯಸ್ಯ
ನಿಃಸಂಗಃ
ನಿಃಸಂಗ-ವಾದ
ನಿಃಸಂತಾನೌ
ನಿಃಸಂಶಯಂ
ನಿಕ್ಷಿಪ್ಯ
ನಿಕ್ಷಿಷ್ಯ
ನಿಗದಿ-ಯಾಗಿ-ದೆಯೋ
ನಿಗದ್ಯತೇ
ನಿಗೂ
ನಿಗೆ
ನಿಗ್ರಹ
ನಿಗ್ರಹಂ
ನಿಗ್ರಹ-ಣ-ದಂತೆ
ನಿಗ್ರಹ-ಮಾಡಿ-ಕೊಳ್ಳುವುದು
ನಿಗ್ರಹಿ-ಸಲು
ನಿಗ್ರಹಿಸಿ-ಕೊಂಡು
ನಿಜ-ವಾಗಿ
ನಿಜ-ವಾಗಿಯೂ
ನಿಜ-ಸಂಗತಿ
ನಿಜಾಂ
ನಿಜಾಂಶ-ವನ್ನು
ನಿಜಾನಾತಿ
ನಿತ್ಯ
ನಿತ್ಯಂ
ನಿತ್ಯ-ಕರ್ಮ-ಗಳನ್ನು
ನಿತ್ಯ-ಕರ್ಮ-ಗಳನ್ನೂ
ನಿತ್ಯ-ಕರ್ಮಾನುಷ್ಠಾನ-ದಲ್ಲಿ
ನಿತ್ಯ-ದಲ್ಲಿ
ನಿತ್ಯ-ದಲ್ಲಿಯ
ನಿತ್ಯ-ದಲ್ಲಿಯೂ
ನಿತ್ಯನೂ
ನಿತ್ಯನೋ
ನಿತ್ಯವೂ
ನಿತ್ಯಶಃ
ನಿತ್ಯಾಭ್ಯಂಗವ್ರತತ್ವೇನ
ನಿದಾನಂ
ನಿದ್ಯತೇ
ನಿದ್ರೆ-ಯನ್ನು
ನಿಧನಂ
ನಿಧಾನ-ವಾಗಿ
ನಿಧಾಯ
ನಿಧಾಸ್ಯತಿ
ನಿನಗೂ
ನಿನಗೆ
ನಿನಗ್ನಾನಿ
ನಿನಾ
ನಿನ್ನ
ನಿನ್ನಂತಹ
ನಿನ್ನಂತೆ
ನಿನ್ನನ್ನು
ನಿನ್ನಿಂದ
ನಿಪುಣಃ
ನಿಪುಣ-ನಾಗಿದ್ದೆ
ನಿಪುಣ-ನಾದ
ನಿಪುಣೋ
ನಿಪೇತುಶ್ಚ
ನಿಬಿಡ-ವಾಗಿತ್ತು
ನಿಮಗೆ
ನಿಮಜ್ಯ
ನಿಮಢಾತ್ಮಾ
ನಿಮಿತ್ತ-ದಿಂದ
ನಿಮಿಷವೇ
ನಿಮೂಢಾತ್ಮಾ
ನಿಮ್ಮ
ನಿಮ್ಮಂತಹ
ನಿಮ್ಮಂಥ-ವರು
ನಿಮ್ಮನ್ನೆಲ್ಲ
ನಿಮ್ಮಲ್ಲಿಯೇ
ನಿಮ್ಮಿಂದ
ನಿಯತಿ
ನಿಯತಿ-ರೇವ
ನಿಯತ್ಯರು
ನಿಯನಶ್ಚೇತಿ
ನಿಯಮ
ನಿಯಮಕ್ಕೆ
ನಿಯ-ಮ-ಗಳನ್ನು
ನಿಯಮ-ದಿಂದ
ನಿಯಮ-ನ-ಮಾಡಿ
ನಿಯಮ-ವಿದೆ
ನಿಯಮವೂ
ನಿಯಮಿಸಿ
ನಿಯಮಿ-ಸುವ
ನಿಯಮೇನ
ನಿಯಾಮ-ಕ-ನಾದ
ನಿಯುಕ್ತಶ್ಚ
ನಿಯುಕ್ತಾಃ
ನಿಯೋಜಿತಃ
ನಿಯೋಜಿಸಲ್ಪಟ್ಟೆ
ನಿರಂಕುಶಾಃ
ನಿರಂಗೌ
ನಿರಂತರ
ನಿರಂತರ-ವಾಗಿ
ನಿರಂತರ-ವಾಗಿ-ರುವ
ನಿರತ-ನಾ-ಗಿ-ರುವುದು
ನಿರ-ತ-ನಾದ
ನಿರ-ತ-ನಾದ-ವ-ರನ್ನು
ನಿರ-ತ-ನಾದೆ
ನಿರತ-ರಾಗಿದ್ದ
ನಿರತ-ರಾಗಿದ್ದೆವು
ನಿರತರಾ-ಗಿ-ರುವ
ನಿರತರಾ-ಗಿ-ರುವ-ವರು
ನಿರತ-ರಾದ
ನಿರತ-ರಾದ-ವರು
ನಿರತರೂ
ನಿರತಾ
ನಿರತಾಸ್ತೇ
ನಿರಯಾ
ನಿರ-ಶನಂ
ನಿರಾ-ಕುಲಃ
ನಿರಾ-ಕುಲಮ್
ನಿರಾ-ಕೃತಃ
ನಿರಾ-ಕೃತ್ಯ
ನಿರಾಮಯಂ
ನಿರಾಶೀರ್ನಿಮಮೋ
ನಿರಾಶ್ರಯ
ನಿರಾಶ್ರಯಃ
ನಿರಾಶ್ರ-ಯನು
ನಿರಾಶ್ರಯ-ನೆಂಬ
ನಿರಾಶ್ರಯೋದಿತಂ
ನಿರಾ-ಹಾರೋ
ನಿರೀಕ್ಷ
ನಿರೀಕ್ಷಿ-ಸುತ್ತಿದ್ದೆ
ನಿರುಕ್ತ
ನಿರೂಪಮಂ
ನಿರೂಪಿತ-ವಾಗಿದೆ
ನಿರೂಪಿತ-ವಾಗಿವೆ
ನಿರೂ-ಪಿಸಲ್ಪಟ್ಟ
ನಿರೂ-ಪಿಸಲ್ಪಟ್ಟಿದೆ
ನಿರೂಪಿ-ಸಿದ
ನಿರೂಪಿ-ಸಿರಿ
ನಿರೂಪಿ-ಸುತ್ತೇವೆ
ನಿರೋಧಂ
ನಿರೋಧ್ಯ
ನಿರ್ಗುಣೋ
ನಿರ್ಜನ-ವಾದ
ನಿರ್ಜನೇ
ನಿರ್ಜಲೇ
ನಿರ್ಜಿತಾಃ
ನಿರ್ಣಯಿಸಲ್ಪಟ್ಟಿದೆ
ನಿರ್ದಿಷ್ಟ-ವಾಗಿ-ರುವಿಕೆ
ನಿರ್ದೋಷಿಯೂ
ನಿರ್ಧೂತ-ಪಾಪಾ-ನಸ್ಮಾಕಂ
ನಿರ್ಮಲ-ವಾದ
ನಿರ್ಮಲಾಂತಃಕ-ರಣ-ರಾದ
ನಿರ್ಮಾಲ್ಯ
ನಿರ್ಮಿತಂ
ನಿರ್ಮಿತ-ವಾದ
ನಿರ್ಮಿತುಂ
ನಿರ್ಮಿ-ಸುತ್ತಾನೆಯೋ
ನಿರ್ಮಿ-ಸು-ವ-ವನು
ನಿರ್ಮೂಲ-ವಾಗುತ್ತದೆ
ನಿರ್ಮೂಲ-ವಾಗುತ್ತ-ದೆಯೋ
ನಿರ್ಲಕ್ಷಿಸಿ
ನಿರ್ವರ್ತಯತ್ವಾಶು
ನಿರ್ವಹಿಸ-ಬೇಕೆಂದು
ನಿರ್ವಹಿಸಲ್ಪಡುತ್ತ-ದೆಯೋ
ನಿರ್ವಹಿ-ಸುತ್ತಾ-ರೆಂಬ
ನಿರ್ವಿಘ್ನಂ
ನಿರ್ವಿಘ್ನ-ಮಾಪ್ನೋತಿ
ನಿರ್ವಿಘ್ನ-ವಾಗಿ
ನಿರ್ವಿಣ್ಣಃ
ನಿರ್ವೀರ್ಯ-ವಾಗುತ್ತದೆ
ನಿರ್ವೃತ್ಯ
ನಿರ್ವೇದ-ನಾಪೇದೇ
ನಿರ್ವೇದ-ಮಾಪನ್ನಾಃ
ನಿರ್ವ್ಯಲೀಕಂ
ನಿಲ್ಲ-ಲಿಲ್ಲ
ನಿಲ್ಲಿಸಿ
ನಿಲ್ಲುತ್ತವೆ
ನಿಲ್ಲು-ವುದಿಲ್ಲವೋ
ನಿವತ
ನಿವರ್ಜಿತಾಃ
ನಿವರ್ತಕಃ
ನಿವರ್ತ-ಕರೋ
ನಿವರ್ತತೇ
ನಿವರ್ತ-ಯಿತ್ವಾ
ನಿವರ್ತಿತಃ
ನಿವಾರ-ಣೆ-ಗಾಗಿ
ನಿವಾರ-ಣೆ-ಮಾಡುವ
ನಿವಾರ-ಣೆಯು
ನಿವಾರ-ಯಾ-ಮಾಸುರ್ಮುನೇ
ನಿವಿಷ್ಠಾ
ನಿವೃತ್ತ
ನಿವೃತ್ತಿ
ನಿವೃತ್ತಿಗೆ
ನಿವೃತ್ತಿ-ಮಾರ್ಗಸ್ಥೌ
ನಿವೃತ್ತಿ-ಯೆಂಬ
ನಿವೇದ-ಯತಿ
ನಿವೇದ-ಯೇತ್
ನಿವೇದಿತಃ
ನಿವೇದಿ-ತಮ್
ನಿವೇದಿಸಿ
ನಿವೇದ್ಯ
ನಿವೇಶ್ಯ
ನಿಶಂ
ನಿಶಮ್ಯೋಚ್ಛಿಷ್ಟ-ವಾಕ್ಯಂ
ನಿಶಾ-ಕರಾಯ
ನಿಶಾಚರ
ನಿಶಾಮಯ
ನಿಶಾಮ್
ನಿಶ್ಚಯ
ನಿಶ್ಚಯಮ್
ನಿಶ್ಚಯ-ವಾಗಿ
ನಿಶ್ಚಯ-ವಾಗಿಯೂ
ನಿಶ್ಚ-ಯಾದ-ಪರೋಕ್ಷಿಣಃ
ನಿಶ್ಚಯಿಸಿ
ನಿಶ್ಚಯಿಸಿ-ದನು
ನಿಶ್ಚಲನ
ನಿಶ್ಚಲಮದ್ವಿತೀಯಮ್
ನಿಶ್ಚಲಮ್
ನಿಶ್ಚಲ-ವಾದ
ನಿಶ್ಚಿತ
ನಿಶ್ಚಿತಃ
ನಿಶ್ಚಿ-ತಮ್
ನಿಷಣ್ಣಂ
ನಿಷಣ್ಣಾ
ನಿಷಣ್ಣಾನ್
ನಿಷಸಾದ
ನಿಷಾದ-ರಾಜನ
ನಿಷಾದಾಧಿ-ಪತೇಃ
ನಿಷಿದ್ದ-ಗಳನ್ನು
ನಿಷಿದ್ಧ
ನಿಷಿದ್ಧ-ವಾದ
ನಿಷೇಧ-ಗಳನ್ನು
ನಿಷೇಧ-ಮಾಡುತ್ತಿ-ರುವಿರಿ
ನಿಷೇಧಸಿ
ನಿಷೇವಣಂ
ನಿಷ್ಕಲ್ಮಷ
ನಿಷ್ಕಾಮ
ನಿಷ್ಕಾಮಂ
ನಿಷ್ಕಾಮ-ಕರ್ಮ-ಗಳ
ನಿಷ್ಕಾಮ-ಕರ್ಮ-ವನ್ನು
ನಿಷ್ಕಾಮತ್ವಂ
ನಿಷ್ಕಾಮ-ನಾಗಿದ್ದರೆ
ನಿಷ್ಕಾಮುಕಾ-ನಾಮಿಹ
ನಿಷ್ಕಾಮೇನ
ನಿಷ್ಕಾಮೋ
ನಿಷ್ಕಾಮ್ಯ
ನಿಷ್ಕಾಮ್ಯ-ಕರ್ಮ-ದಿಂದ
ನಿಷ್ಕಾಮ್ಯ-ಕರ್ಮವು
ನಿಷ್ಕಾರ-ಣ-ವಾಗಿ
ನಿಷ್ಕಾಸಿತೌ
ನಿಷ್ಕೃತಿಃ
ನಿಷ್ಕೃತಿ-ಕಾರ-ಣಮ್
ನಿಷ್ಕೃ-ತಿಮ್
ನಿಷ್ಕೃತಿರ್ನಾಸ್ಯ
ನಿಷ್ಕೃತಿರ್ಮಯಾ
ನಿಷ್ಟಾನ್ನ
ನಿಷ್ಟಾಪಂ
ನಿಷ್ಟು-ರ-ನಾದ
ನಿಷ್ಠ-ನಾದ
ನಿಷ್ಠುರ
ನಿಷ್ಠುರಂ
ನಿಷ್ಠುರನ
ನಿಷ್ಠುರ-ನೆಂಬ
ನಿಷ್ಠುರ-ಭಾಷಣಃ
ನಿಷ್ಠು-ರಮ್
ನಿಷ್ಠುರ-ವಾಕ್ಯಂ
ನಿಷ್ಠುರ-ವಾಕ್ಯ-ಗಳಿಂದ
ನಿಷ್ಠುರ-ವಾಗಿ
ನಿಷ್ಠುರೋ
ನಿಷ್ಠುರೋ-ಽಭ-ವಮ್
ನಿಷ್ಣ್ವರ್ಪಣಂ
ನಿಷ್ಪಲ
ನಿಷ್ಪಲ-ವಾಗುತ್ತವೆ
ನಿಷ್ಪಲ-ವಾಗು-ವುದಿಲ್ಲ
ನಿಷ್ಪಲವೋ
ನಿಷ್ಪಾವಾಣಿ
ನಿಷ್ಫಲವೇ
ನಿಷ್ಫಲಾ
ನಿಷ್ಫಲೋ
ನಿಸ್ಪೃಹಶ್ಚ
ನಿಸ್ಸಂದೇಹ-ವಾಗಿ
ನಿಸ್ಸಂಶಯ-ವಾಗಿ
ನಿಹತ್ಯ
ನೀಚ
ನೀಚನು
ನೀಚ-ಯೋನಿ-ಗಳಲ್ಲಿ
ನೀಚ-ಯೋನಿ-ಯಿಂದ
ನೀಚ-ವಾದ
ನೀಡ-ಲಿಲ್ಲ
ನೀಡಿ
ನೀಡಿ-ದನು
ನೀಡಿದೆ
ನೀಡು
ನೀಡುತ್ತಾನೆ
ನೀಡು-ವರೋ
ನೀತಃ
ನೀತಮಾಯುಷ್ಯ-ಶೇಷಿ-ತಮ್
ನೀತೊ
ನೀತ್ವಾ
ನೀತ್ವಾ-ಗಾದ್ದಿಶಂ
ನೀನಾದರೋ
ನೀನು
ನೀನೂ
ನೀನೆಲ್ಲಿ
ನೀಯ-ತುರ್ದಿವ-ಸಾನಿ
ನೀರನ್ನು
ನೀರಸ
ನೀರಸಃ
ನೀರಸ-ನಾಮಕಃ
ನೀರಸನು
ನೀರಸ-ನೆಂಬ
ನೀರಸೇ
ನೀರಸೋ
ನೀರಾಜನೇ
ನೀರಾ-ಜ-ಯತಿ
ನೀರಿಂದ
ನೀರಿನ
ನೀರಿ-ನಲ್ಲಿ
ನೀರಿ-ನಿಂದ
ನೀರಿ-ನಿಂದಲೇ
ನೀರಿಲ್ಲದ
ನೀರು
ನೀರು-ನೆರಳಿಲ್ಲದ
ನೀಲ
ನೀಲಂ
ನೀಲ-ಕೇತು
ನೀಲ-ಕೇತುಃ
ನೀಲಾಂಬರಸಮಪ್ರಖ್ಯಾ
ನೀಲಾಹ್ವಯೇ
ನೀಲಿ
ನೀಲಿ-ಬಟ್ಟೆಯಿದ್ದಂತೆ
ನೀಲಿ-ಬಣ್ಣ-ದಿಂದಿತ್ತು
ನೀಲೋತ್ಪಲ
ನೀಲೋತ್ಪಲೈಃ
ನೀವು
ನೀಹಾರಮಿವ
ನು
ನುಂಗುತ್ತಿದ್ದವು
ನುಗ್ಗೆ-ಕಾಯಿ
ನುಡಿ
ನುಡಿ-ಗಳನ್ನು
ನುಡಿದ
ನುಡಿ-ದನು
ನುಡಿ-ದರು
ನುಡಿ-ದಳು
ನುಡಿ-ದವು
ನುಡಿದು
ನುಡಿ-ದುವು
ನುಡಿ-ದೆನು
ನುಡಿ-ಯಲು
ನುಡಿ-ಯಿತು
ನುಡಿ-ಯುತ್ತ-ದೆಯೋ
ನುಡಿ-ಯುತ್ತಾರೆ
ನುಡಿ-ಯುತ್ತಿವೆ
ನುಡಿ-ಯುವ
ನುಲಭೋಜ
ನೂನಂ
ನೂರಕ್ಕೂ
ನೂರಾರು
ನೂರು
ನೂರು-ಕೊಟಿ
ನೂರು-ಜನ್ಮ-ಗಳಲ್ಲಿಯೂ
ನೃಣಾಮ್
ನೃಪ
ನೃಪಂ
ನೃಪಃ
ನೃಪತಿಂ
ನೃಪತೇ
ನೃಪತೇಃ
ನೃಪ-ಪಾತಿ-ತೇಷು
ನೃಯಜ್ಞಶ್ಚ
ನೃಸಿಂಹ-ಬೀಜಾಕ್ಷರದ
ನೃಸಿಂಹ-ಬೀಜಾಕ್ಷರ-ಮಂತ್ರ-ಮುಗ್ರಮ್
ನೃಹರೇರ್ಮನುಮ್
ನೄಣಾಂ
ನೆಂದು
ನೆಂಬ
ನೆಂಬು-ವನು
ನೆನ-ಪನ್ನೂ
ನೆನೆದ
ನೆಪಕ್ಕಾಗಿ
ನೆಪ-ದಿಂದ
ನೆರಳಿ-ರುವ
ನೆರಳು
ನೆರವೇರಿಸಿ
ನೆರವೇರಿ-ಸುತ್ತಾರೆ
ನೆರವೇ-ರುತ್ತವೆ
ನೆಲದ-ಮೇಲೆ
ನೆಲ-ವನ್ನು
ನೆಲ್ಲಿ-ಕಾಯಿ
ನೆಲ್ಲಿ-ಕಾಯಿಯ
ನೆಲ್ಲಿ-ಕಾಯಿ-ಯನ್ನು
ನೆಲ್ಲಿ-ಕಾಯಿ-ಯಿಂದ
ನೇ
ನೇಂದ್ರಿಯಾಣಿ
ನೇಂದ್ರೋ
ನೇಕ್ಷಿತುಂ
ನೇಚ್ಛಂತಿ
ನೇಚ್ಛಂತ್ಯನ್ಯತ್ಕದಾ-ಚನ
ನೇತುಮರ್ಹತಿ
ನೇತುಮುದ್ಯತಾಃ
ನೇತ್ರ-ಗಳನ್ನುಳ್ಳ
ನೇತ್ರ-ನಾದ
ನೇದಮನ್ಯಥಾ
ನೇಮಿ-ತ-ನಾದೆ
ನೇಮಿಸಲ್ಪಟ್ಟೆ
ನೇಮಿಸಿ-ದರು
ನೇಷ್ಟ-ಮೇವ
ನೈಕಮತ್ಯಂ
ನೈಮಿತ್ತಿ
ನೈಮಿತ್ತಿಕ
ನೈಮಿತ್ತಿಕಂ
ನೈರ್ಘೃಣ್ಯಾದಿ
ನೈವ
ನೈವಾಧಿರೋಹತಿ
ನೈವಾಸ್ತಿ
ನೈವೇದ್ಯ
ನೈವೇದ್ಯಂ
ನೈವೇದ್ಯಕ್ಕಾಗಿ
ನೈವೇದ್ಯ-ವನ್ನು
ನೈವೇದ್ಯವೇ
ನೈವೇದ್ಯಾದಿ
ನೈವೇದ್ಯಾದಿ-ಗಳು
ನೈವೇದ್ಯೈ
ನೈಸರ್ಗಿಕ
ನೈಸರ್ಗಿಕಸ್ವಭಾವಕ್ಕೆ
ನೊರೆಯು
ನೋ
ನೋಕ್ತ-ವಾನ್
ನೋಡ-ಬಾರದು
ನೋಡ-ಬೇಕು
ನೋಡಲು
ನೋಡಲ್ಪಟ್ಟಿಲ್ಲ
ನೋಡಿ
ನೋಡಿ-ಕೊಂಡರೆ
ನೋಡಿ-ದನು
ನೋಡಿ-ದರೆ
ನೋಡಿ-ದುದು
ನೋಡಿ-ರುತ್ತೇನೆ
ನೋಡುತ್ತಾನೆ
ನೋಡುತ್ತಿ-ರುವ-ವನು
ನೋಡುವ
ನೋತ್ತಸ್ಥೌ
ನೋಪ-ಪಾದಿ-ತಮ್
ನೋಪಲಭ್ಯಂ
ನೋಪಾರ್ಜಿತಾ
ನೋಲಭ್ಯಾಸ್ತತಸ್ತೌ
ನ್ನೊಬ್ಬರು
ನ್ಯಧಾಯ
ನ್ಯಪ-ತತ್ಸದ್ಯೋ
ನ್ಯಪ-ತದ್ಭುವಿ
ನ್ಯವರ್ತಯಮ್
ನ್ಯವಸತ್
ನ್ಯಸೇತ್ತತಃ
ನ್ಯಾನಿ
ನ್ಯಾಯ-ದಿಂದ
ನ್ಯೂ
ಪಂಕಮಿವ
ಪಂಕಶೋಷಣೇ
ಪಂಕ್ತಿಭೇದಂ
ಪಂಕ್ತಿ-ಯಲ್ಲಿ
ಪಂಚ
ಪಂಚ-ಗವ್ಯಂ
ಪಂಚ-ಗವ್ಯ-ದಿಂದ
ಪಂಚ-ಗವ್ಯ-ವನ್ನು
ಪಂಚ-ಗವ್ಯೇನ
ಪಂಚ-ಜನ
ಪಂಚ-ಜನಾಹ್ವಯೇ
ಪಂಚ-ತನ್ಮಾತ್ರಾ
ಪಂಚ-ತನ್ಮಾತ್ರಾ-ಗಳೂ
ಪಂಚತ್ವಮಭಿ-ಪದ್ಯಾಥ
ಪಂಚತ್ವಮೇತ್ಯ
ಪಂಚ-ದಶೋಧ್ಯಾಯಃ
ಪಂಚಪ್ರಾಣ-ಗಳೇ
ಪಂಚಪ್ರಾಣಾಃ
ಪಂಚಭಿಃ
ಪಂಚ-ಭೂತ-ಗಳು
ಪಂಚ-ಮಹಾ-ಭೂತ-ಗಳೂ
ಪಂಚ-ಮಿ-ಯಲ್ಲಿ
ಪಂಚಮೋ
ಪಂಚ-ಯಜ್ಞ-ಗಳನ್ನೂ
ಪಂಚ-ಯಜ್ಞಾಃ
ಪಂಚ-ಯಜ್ಞಾಶ್ವ
ಪಂಚ-ಯೋ-ಜನ-ದೂ-ರತಃ
ಪಂಚ-ರಂಗೈಃ
ಪಂಚ-ವಿಂಶತಿಧಾ
ಪಂಚ-ಶಿಖಂ
ಪಂಚಾಮೃತ-ದಿಂದ
ಪಂಚಾಮೃತೈರ್ಹರಿಮ್
ಪಂಚಾಶದ್ದ್ವಾ-ದಶಂ
ಪಂಚಾಸಾನೇ
ಪಂಡಿತ
ಪಂಡಿತ-ನಾದ
ಪಕ್ಕ-ದಲ್ಲಿ
ಪಕ್ಷಂ
ಪಕ್ಷಿ
ಪಕ್ಷಿ-ಗಳನ್ನು
ಪಕ್ಷಿ-ಗಳಲ್ಲಿ
ಪಕ್ಷಿ-ಗಳು
ಪಕ್ಷಿ-ಗಳೊಡನೆ
ಪಕ್ಷಿಣಾಂ
ಪಕ್ಷೀಂದ್ರೋ
ಪಕ್ಷೇ
ಪಚ್ಚತೇ
ಪಟ್ಟ-ಣಕ್ಕೆ
ಪಟ್ಟ-ಣದ
ಪಟ್ಟ-ಣ-ದಲ್ಲಿ
ಪಟ್ಟಿ-ಗಳನ್ನು
ಪಠಣ
ಪಠನ-ನಿರತಾಸ್ತೋತ್ರ-ಪಾಠಕಾಃ
ಪಠ-ನಾತ್ಸ
ಪಠಿ-ಸಿ-ದರೆ
ಪಠಿಸುತ್ತಾರೆಯೋ
ಪಠೇನ್ನಿತ್ಯಂ
ಪಡ-ಲಿಲ್ಲ
ಪಡಿಸಲಾಗಿದೆ
ಪಡಿ-ಸುತ್ತವೆ
ಪಡುತ್ತಿ-ರುವ
ಪಡುವ-ವನ
ಪಡು-ವುದಿಲ್ಲ
ಪಡೆದ
ಪಡೆ-ದಂತೆ
ಪಡೆ-ದನು
ಪಡೆ-ದರು
ಪಡೆ-ದರೂ
ಪಡೆ-ದರೆ
ಪಡೆ-ದಳು
ಪಡೆದ-ವನೂ
ಪಡೆದ-ವರು
ಪಡೆದಿ-ರುವ
ಪಡೆದಿ-ರುವ-ವರು
ಪಡೆದಿಲ್ಲ
ಪಡೆ-ದಿವೆ
ಪಡೆದಿ-ವೆಯೋ
ಪಡೆದು
ಪಡೆ-ಯದೇ
ಪಡೆಯ-ಬಹುದು
ಪಡೆಯ-ಬೇಕು
ಪಡೆ-ಯಲು
ಪಡೆ-ಯಿತು
ಪಡೆಯುತ್ತದೆ
ಪಡೆಯುತ್ತವೆ
ಪಡೆಯುತ್ತಾನೆ
ಪಡೆಯುತ್ತಾನೆಂಬುದ-ರಲ್ಲಿ
ಪಡೆ-ಯುತ್ತಾರೆ
ಪಡೆ-ಯುತ್ತಾರೆಂದು
ಪಡೆಯುತ್ತಿದ್ದೆ
ಪಡೆ-ಯುವನು
ಪಡೆ-ಯುವರು
ಪಡೆ-ಯುವಳು
ಪಡೆ-ಯುವ-ವರೇ
ಪಡೆ-ಯುವುದರ
ಪಡೆಯು-ವುದಿಲ್ಲ
ಪಡೆಯು-ವುದಿಲ್ಲವೋ
ಪಣಜೀ-ವಿನಾ
ಪತಂಗೇ
ಪತತ್ಯಾಶು
ಪತನಾಯ
ಪತಿ
ಪತಿಂ
ಪತಿ-ಗಳನ್ನು
ಪತಿಘ್ನೋ
ಪತಿ-ತಮಶ್ನಾತು
ಪತಿ-ತಾ-ನಪಿ
ಪತಿ-ತಾನಾಂ
ಪತಿಯ
ಪತಿ-ಯನ್ನು
ಪತಿ-ಯಾದ
ಪತಿಯು
ಪತಿ-ಯೊಡನೆ
ಪತಿವ್ರತಾ
ಪತಿವ್ರ-ತೆ-ಯಾದ
ಪತ್ತೀನ್
ಪತ್ನಿ
ಪತ್ನಿಯ
ಪತ್ನಿ-ಯನ್ನು
ಪತ್ನಿ-ಯರು
ಪತ್ನಿಯು
ಪತ್ನಿಯೂ
ಪತ್ನಿ-ಯೊಡನೆ
ಪತ್ನೀ
ಪತ್ಯಾ-ವ-ಗತೇ
ಪತ್ರ-ಭೋ-ಜನಂ
ಪತ್ರ-ಶತಂ
ಪತ್ರಾಣಿ
ಪತ್ರೇಷು
ಪಥಂ
ಪಥಿ
ಪಥೇ
ಪಥ್ಯದ
ಪಥ್ಯಮೇ-ವಾನ್ನಂ
ಪದಂ
ಪದಮೀಶ್ವರಸ್ಯ
ಪದವಿ-ಯನ್ನು
ಪದವೀಂ
ಪದವೂ
ಪದಾನ್ಯೇವ
ಪದಾರ್ಥ
ಪದಾರ್ಥ-ಗಳನ್ನು
ಪದಾರ್ಥ-ಗಳನ್ನೂ
ಪದಾರ್ಥ-ಗಳನ್ನೇ
ಪದಾರ್ಥ-ವನ್ನು
ಪದೇ
ಪದೇ-ಽಶ್ವಮೇಧಸ್ಯ
ಪದ್ಧ
ಪದ್ಧ-ತಿ-ಯನ್ನು
ಪದ್ಧತೌ
ಪದ್ಮ
ಪದ್ಮಕ
ಪದ್ಮ-ಕ-ನೆಂಬ
ಪದ್ಮಕೋ
ಪದ್ಮ-ನಾಭನ
ಪದ್ಮ-ನೆಂಬ
ಪದ್ಮ-ಪತ್ರಾಣಿ
ಪದ್ಮ-ಮಾಲಿಖ್ಯ
ಪದ್ಮಾ
ಪದ್ಮಾ-ಕರಂ
ಪದ್ಮಾ-ದೈರ್ನ
ಪದ್ಮೇ
ಪದ್ಮೇತಿ
ಪದ್ಮೈಶ್ಚ
ಪದ್ಯ-ಕಾಖ್ಯಸ್ಯ
ಪದ್ಯತ
ಪದ್ಯ-ವನ್ನು
ಪನಾ-ದಿನಃ
ಪಪಾತ
ಪಪೌ
ಪಪ್ರಚ್ಛ
ಪಪ್ರಚ್ಛಾಂಗಿ-ರಸಂ
ಪಪ್ರಚ್ಛುರ್ಮುನಿಪುಂಗ-ವಮ್
ಪಯತಿ
ಪಯತ್ಯಸೌ
ಪಯಸಾ
ಪಯಸಾಂ
ಪಯಸಾ-ಭಿಷಿಕ್ತಾ
ಪಯಸಾ-ಭಿಷೇಚೇ
ಪಯಸಾ-ಸಿನ್ವಮಾನಾಃ
ಪಯಸ್ತಥಾ
ಪಯೋಘೃ
ಪಯೋನಿಧಿಃ
ಪಯೋನಿಧೌ
ಪರಂ
ಪರಂಪರೆ-ಯಲ್ಲಿ
ಪರತ್ರ
ಪರ-ದತ್ತಂ
ಪರದಾರಾಭಿಕಾಂಕ್ಷಿಣೌ
ಪರದಾರಾಭಿ-ನಿ-ರತಂ
ಪರ-ದೀಪಪ್ರಬೋ-ಧ-ನಮ್
ಪರದೆಯ
ಪರ-ದೋಷಾಂಶ್ಚ
ಪರದ್ರವ್ಯ-ಗುರುದ್ರವ್ಯ-ಗಳನ್ನು
ಪರದ್ರವ್ಯಾಪಹಾರಿ-ಣಮ್
ಪರಪೀಡ-ನಮ್
ಪರಪೀಡನೆ-ಯಿಂದ
ಪರ-ಪುರುಷ-ನೊಡನೆ
ಪರಪೂರುಷ-ಸಂಗ-ಜಮ್
ಪರಬ್ರಹ್ಮನ
ಪರಬ್ರಹ್ಮ-ನನ್ನೇ
ಪರಬ್ರಹ್ಮ-ನಲ್ಲಿ
ಪರಬ್ರಹ್ಮನು
ಪರಮ
ಪರಮಂ
ಪರಮಃ
ಪರಮ-ಪವಿತ್ರ-ವಾದ
ಪರಮ-ಪೂಜ್ಯ-ರಾದ
ಪರಮಾತ್ಮ
ಪರಮಾತ್ಮನ
ಪರಮಾತ್ಮ-ನನ್ನು
ಪರಮಾತ್ಮ-ನಲ್ಲಿ
ಪರಮಾತ್ಮ-ನಿಗೆ
ಪರಮಾತ್ಮನು
ಪರಮಾತ್ಮನೇ
ಪರ-ಮಾನ್ನ
ಪರಮೇಶ್ವರಃ
ಪರಮೋ
ಪರಮೋಸ್ತಿ
ಪರಮೌಷಧಮ್
ಪರಮ್
ಪರಯಾ
ಪರಶುರಾ-ಮನು
ಪರ-ಸೇವೆ-ಯಿಂದ
ಪರಸ್ತ್ರೀ
ಪರತ್ರೀಯ
ಪರತ್ರೀ-ಯ-ರಲ್ಲಿ
ಪರತ್ರೀ-ಸಂಗಜಂ
ಪರತ್ರೀ-ಸಂಗ-ದೋಷೇಣ
ಪರತ್ರೀ-ಸಂಗ-ಮಾಡಿದ್ದೇ
ಪರಸ್ಪ-ರಮ್
ಪರಸ್ಪರವಧೇಚ್ಛವಃ
ಪರಸ್ಪರ-ವಾಗಿ
ಪರಸ್ಪರಾ-ಲಾಪಮು-ದಿತೇಂದ್ರಿಯಗಾತ್ರಕಾಃ
ಪರಸ್ವಸೂಚಕದ್ರವ್ಯಂ
ಪರಾ
ಪರಾಂ
ಪರಾಃ
ಪರಾಕ್ರಮಿ-ಯಾದ
ಪರಾ-ಗತಿಃ
ಪರಾತ್
ಪರಾತ್ತು
ಪರಾ-ಧೀನ
ಪರಾ-ಧೀನ-ತೆ-ಯನ್ನು
ಪರಾ-ನುಗಃ
ಪರಾ-ಭಕ್ತಿರ್ಯಥಾ
ಪರಾ-ಭವಂ
ಪರಾ-ಭವ-ವನ್ನು
ಪರಾ-ಮತಾ
ಪರಾ-ಯಣಃ
ಪರಾ-ಯಣ-ನಾದ
ಪರಾ-ಯ-ಣಮ್
ಪರಿ-ಕೀರ್ತಿ-ತಮ್
ಪರಿ-ಕೀರ್ತಿತಾಃ
ಪರಿಕ್ರಾಂತಃ
ಪರಿ-ಗಣಿ-ಸುವ
ಪರಿಗ್ರಹಃ
ಪರಿ-ಚ-ಯ-ದಿಂದ
ಪರಿ-ಚಯ-ವನ್ನು
ಪರಿ-ಣಾ-ಮ-ಗಳು
ಪರಿ-ಣಾಮದ
ಪರಿ-ಣಾಮವು
ಪರಿ-ಣಾಮಸ್ತಥಾ
ಪರಿ-ತಾಸ್ತೇ
ಪರಿತ್ಯಜಿಸಿ
ಪರಿತ್ಯಜ್ಯ
ಪರಿತ್ಯಾಗ
ಪರಿ-ಧಾನಾಯ
ಪರಿ-ಧಾಯ
ಪರಿಧಿ
ಪರಿಧಿಂ
ಪರಿಧಿಃ
ಪರಿ-ಧಿಗೆ
ಪರಿ-ಧಿ-ನಾಮಕಃ
ಪರಿ-ಧಿ-ನಾ-ಮಾಸೌ
ಪರಿ-ಧಿಯ
ಪರಿ-ಧಿ-ಯಿಂದಲೂ
ಪರಿ-ಧಿಯು
ಪರಿ-ಧಿ-ಯೆಂಬು-ವ-ನಂತೆ
ಪರಿ-ಧಿರ್ನಾಮ
ಪರಿ-ಧಿರ್ಮೋಹಾತ್
ಪರಿ-ಧಿರ್ಯಥಾ
ಪರಿ-ಧಿಸ್ತೀನ
ಪರಿಧೇಃ
ಪರಿ-ಧೇರ್ಮುಕ್ತಿಃ
ಪರಿ-ಧೇಸ್ತಸ್ಯ
ಪರಿ-ಪಾತಿ
ಪರಿ-ಪಾಲಿಸುತ್ತಾನೆ
ಪರಿ-ಪೀಡಿ-ತಮ್
ಪರಿ-ಪೂ-ಜ-ಯೇತ್
ಪರಿ-ಪೂಜಿತಾ
ಪರಿ-ಮಳ-ವಸ್ತು-ಗಳನ್ನು
ಪರಿಮಾರ್ಜ-ಯತಿ
ಪರಿ-ರಭ್ಯಾಗಾದ್ಗು
ಪರಿ-ವಾರ
ಪರಿ-ವಾರ-ತಯೈವ
ಪರಿ-ವಾರ-ದ-ವ-ರಾಗಿ
ಪರಿ-ವಾರ-ದವ-ರೆಂದು
ಪರಿ-ವಾರ-ದೇವ-ತೆ-ಗಳ
ಪರಿ-ವೆಯೇ
ಪರಿ-ಶುದ್ಧ-ಗೊಳಿ-ಸುತ್ತವೆ
ಪರಿ-ಶುದ್ಧ-ವಾಗಿ-ರಲು
ಪರಿ-ಶುದ್ಧ-ವಾದ
ಪರಿ-ಶೋಭಿ-ತಮ್
ಪರಿ-ಷಸ್ವ-ಜಾತೇ
ಪರಿ-ಸರ
ಪರಿಸ್ಥಿ-ತಿಗೆ
ಪರಿ-ಹರಿ-ಸಿ-ಕೊಂಡು
ಪರಿ-ಹರಿ-ಸಿ-ಕೊಳ್ಳಲು
ಪರಿ-ಹರಿ-ಸಿ-ಕೊಳ್ಳಿರಿ
ಪರಿ-ಹರಿ-ಸುವ
ಪರಿ-ಹರ್ತಾಸಿ
ಪರಿ-ಹಾರ-ಕ-ನಾದ
ಪರಿ-ಹಾರ-ವಾಗದೇ
ಪರಿ-ಹಾರ-ವಾಗಿ
ಪರಿ-ಹಾರ-ವಾಗುತ್ತದೆ
ಪರಿ-ಹಾರ-ವಾಗುತ್ತವೆ
ಪರಿ-ಹಾರ-ವಾಗುವ
ಪರಿ-ಹಾರ-ವಾದ
ಪರಿ-ಹಾರ-ವಾದ-ಮೇಲೆ
ಪರಿ-ಹೃತ್ಯ
ಪರೀಕ್ಷಕಃ
ಪರೀಕ್ಷಿ-ಸುವಿ
ಪರೀಕ್ಷಿಸು-ವುದೇ
ಪರೀತೋದ್ರಿಮಮುಂ
ಪರೀತೋಹ್ಯ-ಗಮಂ
ಪರೀಧಾಯ
ಪರೇ
ಪರೇಭ್ಯೋ
ಪರೇಷಾಂ
ಪರೈರಪಿ
ಪರೈ-ರಹಂ
ಪರೈರಾಕ್ರಾಂತ-ವಾಸ
ಪರೋ
ಪರೋಕ್ತಾಂ
ಪರೋ-ಪ-ಕಾರಃ
ಪರೋ-ಪ-ಕಾರ-ದಿಂದ
ಪರೋ-ಪಜೀ-ವಿಷ್ವಥ
ಪರೋ-ಽಸ್ತಿ
ಪರ್ಯಂಕದಾ
ಪರ್ಯಂತ
ಪರ್ಯಂತಂ
ಪರ್ಯ-ಕಲ್ಪ-ಯತ್
ಪರ್ವಣಿ
ಪರ್ವ-ತಕ್ಕೆ
ಪರ್ವತ-ಗಳಷ್ಟು
ಪರ್ವತ-ಗಳಿಗೆ
ಪರ್ವತದ
ಪರ್ವತ-ದಲ್ಲಿ
ಪರ್ವತ-ವನ್ನೇ-ರಿ-ದನು
ಪರ್ವ-ತಾನಾಂ
ಪರ್ವತೇ
ಪರ್ವ-ದಿವ-ಸ-ಗಳಲ್ಲಿಯೂ
ಪರ್ವ-ಷಷ್ಠೀಷು
ಪಲಾಶ-ನೀಲ-ಗಗನಭ್ರಾಜ-ಕುಲ-ತಾರಕೇ
ಪಲಾಶಾ-ಶನಸ್ತ್ವಂ
ಪಲ್ಲವಂ
ಪಲ್ಲವೇ
ಪವನಾ-ಯತೇ
ಪವನೋಸ್ಮಾಕಂ
ಪವಿತ್ರ
ಪವಿತ್ರ-ನನ್ನಾಗಿ
ಪವಿತ್ರ-ಮಿಹ
ಪವಿತ್ರ-ರನ್ನಾಗಿ
ಪವಿತ್ರ-ರಾ-ಗುತ್ತಾರೆ
ಪವಿತ್ರ-ವಾಗುತ್ತವೆ
ಪವಿತ್ರ-ವಾದ
ಪವಿತ್ರ-ವಾ-ಯಿತು
ಪಶು
ಪಶುಃ
ಪಶು-ಗಳಲ್ಲಿ
ಪಶು-ಗಳಿಂದಲೂ
ಪಶು-ಪಕ್ಷಿ-ಮೃಗಾ-ಶನಃ
ಪಶು-ಬಂಧ-ನಾತ್
ಪಶು-ವನ್ಮರ್ತ್ಯಃ
ಪಶು-ವಿಗೆ
ಪಶುವೇ
ಪಶೂನಾಂ
ಪಶ್ಚಾಚ್ಚ
ಪಶ್ಚಾತ್
ಪಶ್ಚಾತ್ಕರ್ಮ
ಪಶ್ಚಾತ್ತಾಪ
ಪಶ್ಚಾತ್ತಾಪ-ಗೊಂಡ
ಪಶ್ಚಾತ್ತಾಪೇನ
ಪಶ್ಚಾತ್ತಾಪೋ
ಪಶ್ಚಾತ್ತು
ಪಶ್ಚಾ-ದಾಗ-ತ-ಕಾಲೋ
ಪಶ್ಚಾದಾವಿರ-ಭೂತ್ಪಾಪೀ
ಪಶ್ಚಾದೇ-ತತ್ಕದಂಬಗಾಃ
ಪಶ್ಚಾದೇ-ತಾಂಶ್ಚ
ಪಶ್ಚಾದ್ಗುರುಪ್ರಭಾವೇನ
ಪಶ್ಚಾದ್ದಾಸೀ-ಸಂಗೇ
ಪಶ್ಚಾದ್ಧೇಮ-ಪುರೀ
ಪಶ್ಚಾದ್ಯೋ-ಗ-ಮವಾಪ್ನು-ಯಾತ್
ಪಶ್ಚಾದ್ರೌರವ-ಮಶ್ನುತೇ
ಪಶ್ಚಾದ್ವಿ-ನಷ್ಟೇ
ಪಶ್ಚಾನ್ನಿಃಸ್ವೌ
ಪಶ್ಚಾನ್ನಿಷ್ಕಾಮ-ಚೋದ-ನಾಮ
ಪಶ್ಚಿಮ-ದಿಕ್ಕಿಗೆ
ಪಶ್ಚಿಮ-ದಿಕ್ಕಿ-ನಲ್ಲಿ
ಪಶ್ಚಿಮೇ
ಪಶ್ಯಂತಿ
ಪಶ್ಯತಿ
ಪಶ್ಯೇತ
ಪಸಾತ-ಕ-ವರ್
ಪಸಿ
ಪಸ್ಯಸಿ
ಪಾಂಡಿತ್ಯ-ಪಡೆದು
ಪಾಂಡ್ಯ-ರಾಜನ
ಪಾಂಡ್ಯ-ರಾಜನು
ಪಾಂಡ್ಯ-ರಾಜಸ್ಯ
ಪಾಂಡ್ಯೇ-ಶನ-ಗರಂ
ಪಾಂಥಾ-ನಾಮುಪ-ಕಾರ-ಕಮ್
ಪಾಖಂಡ-ಮತ
ಪಾಖಂಡ-ಮತದ
ಪಾಖಂಡ-ಮತ-ವನ್ನು
ಪಾಖಂಡ-ಮಾರ್ಗೇಷು
ಪಾಖಂಡ-ವಾರ್ಗ-ವನ್ನು
ಪಾಖಂಡ-ವಾರ್ಗೇಣ
ಪಾಖಂಡ-ವುತ-ವನ್ನು
ಪಾಖಂಡಿನಾಂ
ಪಾಖಂಡೇ
ಪಾಚ-ಯಿತ್ವಾ
ಪಾಠ
ಪಾಠಪ್ರವ-ಚನ
ಪಾಠಪ್ರ-ವ-ಚನ-ಗಳನ್ನು
ಪಾಠಪ್ರವ-ಚನ-ಗಳಲ್ಲಿ
ಪಾಡ್ಯ
ಪಾಣಿ
ಪಾಣಿ-ಮೂರ್ಧಾ
ಪಾಣೆರ್ಮೇ
ಪಾಣೌ
ಪಾಣ್ಯಾ
ಪಾತಃಕಾಲ
ಪಾತಕ-ಗಳೇ
ಪಾತಕೇ
ಪಾತ-ಯಂತಿ
ಪಾತ-ಯತಿ
ಪಾತ-ಯತ್ಯದ್ದಾ
ಪಾತ-ಯೇತ್ಸಪ್ತ
ಪಾತಿ-ತಮ್
ಪಾತುಮಿಚ್ಛೋಸ್ತನಂ
ಪಾತ್ರಂ
ಪಾತ್ರ-ನಾಗಿದ್ದರೆ
ಪಾತ್ರ-ನಾಗಿದ್ದೆ
ಪಾತ್ರ-ರಾದ
ಪಾತ್ರರೂ
ಪಾತ್ರವು
ಪಾತ್ರಸ್ಥ-ನೈವ
ಪಾತ್ರೆ
ಪಾತ್ರೆ-ಗಳೇ
ಪಾತ್ರೆ-ಯನ್ನು
ಪಾತ್ರೆ-ಯಲ್ಲಿ
ಪಾತ್ರೆ-ಯಲ್ಲಿಟ್ಟು
ಪಾತ್ರೆ-ಯೊಳ-ಗಿನ
ಪಾತ್ರೆ-ಯೊಳಗೆ
ಪಾತ್ರೇ
ಪಾದ-ಕಮಲ-ಗಳಲ್ಲಿ
ಪಾದ-ಗಳ
ಪಾದ-ಗಳನ್ನು
ಪಾದ-ಗಳಲ್ಲಿ
ಪಾದ-ಗಳಿಗೆ
ಪಾದ-ದಲ್ಲಿ
ಪಾದ-ದಿಂದ
ಪಾದಪದ್ಮ-ಗಳನ್ನು
ಪಾದ-ಪದ್ಮ-ಗಳು
ಪಾದಪಶ್ವ
ಪಾದಪಾ
ಪಾದಪಾಶ್ಚ
ಪಾದಯೋರ್ಲೆಪಂ
ಪಾದಯೋಶ್ಚಕ್ರ-ಪಾಣಿನಃ
ಪಾದಯೋಸ್ತಿಸ್ರಃ
ಪಾದ-ವನ್ನು
ಪಾದಾತ್ಮಾ
ಪಾದಾಬ್ಜ-ಸಂಭೂತಾ
ಪಾದಾಬ್ಬ
ಪಾದುಕೆ-ಗಳನ್ನು
ಪಾದೌ
ಪಾದ್ಯಾದಿ-ಗಳಿಂದ
ಪಾನ-ಮಾಡ-ಲಾರನು
ಪಾನ-ಮಾಡಿ
ಪಾಪ
ಪಾಪಂ
ಪಾಪ-ಕರ್ಮ
ಪಾಪ-ಕರ್ಮ-ಗಳ
ಪಾಪ-ಕರ್ಮ-ಗಳಿಗೆ
ಪಾಪ-ಕರ್ಮಣಃ
ಪಾಪ-ಕರ್ಮ-ಣಾಮ್
ಪಾಪ-ಕರ್ಮದ
ಪಾಪ-ಕರ್ಮ-ದಿಂದ
ಪಾಪ-ಕಾರ-ಣಮ್
ಪಾಪ-ಕಾರಿ-ಣಾ-ವರ್
ಪಾಪ-ಕಾರ್ಯ-ಗಳ
ಪಾಪ-ಕೃತ್ಯ-ಗಳ
ಪಾಪ-ಕೋಟಿಭಿಃ
ಪಾಪಕ್ಕೆ
ಪಾಪ-ಗಳ
ಪಾಪ-ಗಳನ್ನು
ಪಾಪ-ಗಳನ್ನೂ
ಪಾಪ-ಗಳಿಂದ
ಪಾಪ-ಗಳಿಂದಲೂ
ಪಾಪ-ಗಳಿಗೆ
ಪಾಪ-ಗಳು
ಪಾಪ-ಗಳೂ
ಪಾಪ-ಗಳೆಲ್ಲ
ಪಾಪ-ಗಳೆಲ್ಲವೂ
ಪಾಪ-ಗೃಹೇ
ಪಾಪ-ಚಾರಿಣಃ
ಪಾಪ-ಚೇತಸಾ
ಪಾಪದ
ಪಾಪ-ದಿಂದ
ಪಾಪ-ದಿಂದಲೂ
ಪಾಪದ್ವಯಂ
ಪಾಪ-ನಾಶ-ಕರವು
ಪಾಪ-ನಾ-ಶನಃ
ಪಾಪ-ನಾಶ-ನಮ್
ಪಾಪ-ನಿರತೌ
ಪಾಪ-ನಿರ್ಮುಕ್ತೋ
ಪಾಪ-ಪರಿ-ಹಾರ-ಕ-ನಾದ
ಪಾಪ-ಬುದ್ದಿನಾ
ಪಾಪ-ಬುದ್ದಿ-ಯಿಂದ
ಪಾಪ-ಬುದ್ದಿರ್ನ
ಪಾಪ-ಭಾಜೋ
ಪಾಪ-ಮಾಪ್ನೋತಿ
ಪಾಪ-ಮುಕ್ತರನ್ನಾಗಿ
ಪಾಪ-ಮೇವಾರ್ಜಿತಂ
ಪಾಪ-ಯಂತ್ಯ
ಪಾಪ-ರ-ಹಿತ-ನಾದ
ಪಾಪ-ರ-ಹಿತ-ರಾದ
ಪಾಪ-ಲೇಶವೂ
ಪಾಪ-ಲೇಶೋ
ಪಾಪ-ಲೇಶೋಸ್ತಿ
ಪಾಪ-ವನ್ನು
ಪಾಪ-ವನ್ನೂ
ಪಾಪ-ವನ್ನೇ
ಪಾಪ-ವಿ-ಮುಕ್ತ-ರಾ-ಗುತ್ತಾರೆ
ಪಾಪ-ವಿಮೋ-ಚನೆ-ಯನ್ನೂ
ಪಾಪವು
ಪಾಪವೂ
ಪಾಪವೇ
ಪಾಪ-ಶತೈರ್ಯುಕ್ತಂ
ಪಾಪ-ಸಹಸ್ರ-ಮುಗ್ರಂ
ಪಾಪ-ಸಹಸ್ರಾಣಿ
ಪಾಪಸ್ಪರ್ಶ-ವಿಲ್ಲ
ಪಾಪ-ಹ-ರ-ನಾದ
ಪಾಪ-ಹಾನೀಂ
ಪಾಪಾ
ಪಾಪಾ-ಚ-ರಣೆ-ಯಿಂದ
ಪಾಪಾತ್ಮನಾದ
ಪಾಪಾತ್ಮನು
ಪಾಪಾತ್ಮರು
ಪಾಪಾತ್ಮಳು
ಪಾಪಾತ್ಮಾ
ಪಾಪಾದ್ವಯಂ
ಪಾಪಾನಾಂ
ಪಾಪಾನಿ
ಪಾಪಾನ್ಮುಚ್ಯತೇ
ಪಾಪಾನ್ಯ
ಪಾಪಾಯ
ಪಾಪಾ-ಸ-ಕ-ರಾಗಿ
ಪಾಪಿ-ಗಳ
ಪಾಪಿ-ಗಳನ್ನೂ
ಪಾಪಿ-ಗಳಾದ
ಪಾಪಿ-ಗಳಿಂದ
ಪಾಪಿ-ಗಳಿಬ್ಬರೂ
ಪಾಪಿ-ಜನರ
ಪಾಪಿನಂ
ಪಾಪಿನಃ
ಪಾಪಿ-ನಸ್ತಸ್ಯ
ಪಾಪಿನಾಂ
ಪಾಪಿನೋ
ಪಾಪಿಭಿಃ
ಪಾಪಿಯ
ಪಾಪಿ-ಯಾದ
ಪಾಪಿಯು
ಪಾಪಿಷ್ಟನು
ಪಾಪಿಷ್ಠ
ಪಾಪಿಷ್ಠ-ನಲ್ಲದ
ಪಾಪಿಷ್ಠ-ನಾದ
ಪಾಪಿಷ್ಠನು
ಪಾಪಿಷ್ಠ-ರಾದ
ಪಾಪಿಷ್ಠ-ರಾ-ದರೂ
ಪಾಪಿಷ್ಠ-ರಾದ-ವ-ರಿಗೆ
ಪಾಪಿಷ್ಠ-ರಿಗೆ
ಪಾಪೀ
ಪಾಪೇನ
ಪಾಪೇಭ್ಯಃ
ಪಾಪೈಃ
ಪಾಪ್ಮ-ನಾಮುನಾ
ಪಾಪ್ಮಾನೌ
ಪಾಯ-ಯಿತ್ವಾ
ಪಾಯಸ
ಪಾಯಸಂ
ಪಾಯಸ-ದಿಂದ
ಪಾಯಸಸ್ಕೋಪ-ಹಾ-ರತಃ
ಪಾಯೋ
ಪಾಯ್ವಾತ್ಮಕಸ್ತಥಾ
ಪಾರಂಗತ
ಪಾರಂಗತ-ನಾಗಿದ್ದೆ
ಪಾರಲೌಕಿ-ಕಮ್
ಪಾರಾಗುತ್ತಾನೆ
ಪಾರೇ
ಪಾರ್ವತಿಯು
ಪಾರ್ವತೀ
ಪಾರ್ವತೀ-ದೇ-ವಿಗೆ
ಪಾರ್ವತೀ-ದೇವಿ-ಯನ್ನು
ಪಾರ್ವತ್ಯೈ
ಪಾರ್ವತ್ಯೈ-ಶಂಕರೋ-ಽ-ವದತ್
ಪಾರ್ಷದಾಂಶ್ಚ
ಪಾರ್ಷದಾಃ
ಪಾರ್ಷ-ದಾನ್
ಪಾರ್ಷ್ಣಿ-ಮೂರ್ಧಾಶ್ಚ
ಪಾಲಿತೇ
ಪಾಲಿಸು
ಪಾಲಿಸುತ್ತಿದ್ದ
ಪಾವಧೀಃ
ಪಾವನಂ
ಪಾವ-ನಮ್
ಪಾವನ-ರನ್ನಾಗಿ
ಪಾವ-ನರಾ-ಗುತ್ತಾರೆ
ಪಾವಿತಂ
ಪಾಶ
ಪಾಶೈರಾ-ಬದ್ಧ್ಯ
ಪಾಸಿನೌ
ಪಾಹ-ಕಾರಿ-ಣಾಮ್
ಪಾಹೃತ್ಯ
ಪಿ
ಪಿಂಗಲಂ
ಪಿಂಡಾದಿ-ಗಳನ್ನು
ಪಿಣ್ಯಾಕಂ
ಪಿತರ
ಪಿತರಃ
ಪಿತರಶ್ಚಿರಾತ್
ಪಿತರಸ್ತಥಾ
ಪಿತರಸ್ತುಷ್ಟಾ
ಪಿತರಿ
ಪಿತರೋ
ಪಿತರೋ-ಽಖಿಲಾಃ
ಪಿತರೌ
ಪಿತರೌ-
ಪಿತಾ
ಪಿತಾನ್ದೇ-ವಾನ್
ಪಿತಾ-ಮಹನೆ
ಪಿತುಃ
ಪಿತುರ್ಭಾತಾ
ಪಿತುರ್ಮಮ
ಪಿತುರ್ಮಹದ್ಧ-ನಮ್
ಪಿತೃ
ಪಿತೃ-ಋ-ಣ-ದಿಂದ
ಪಿತೃ-ಗಣ-ದಲ್ಲಿದ್ದ
ಪಿತೃ-ಗಳ
ಪಿತೃ-ಗಳನ್ನು
ಪಿತೃ-ಗಳನ್ನೂ
ಪಿತೃ-ಗಳಿಂದ
ಪಿತೃ-ಗಳಿ-ಗಾಗಿ
ಪಿತೃ-ಗಳಿಗೆ
ಪಿತೃ-ಗಳಿ-ಗೋಸ್ಕರ
ಪಿತೃ-ಗಳು
ಪಿತೃ-ಗಳೂ
ಪಿತೃ-ಗಾಥಾಪಿ
ಪಿತೃಘ್ನಂ
ಪಿತೃ-ದೇ-ವತಾಃ
ಪಿತೃ-ಭಕ್ತಾಶ್ಚ
ಪಿತೃಭಿಃ
ಪಿತೃಭೇ
ಪಿತೃಭ್ಯಃ
ಪಿತೃ-ಮುದ್ದಿಶ್ಯ
ಪಿತೃ-ಯಜ್ಞ
ಪಿತೃ-ಯಜ್ಞಸ್ತಥೈವ
ಪಿತೄಣಾಂ
ಪಿತೄನಥ
ಪಿತೄನುದ್ದಿಶ್ಯ
ಪಿತೄನ್
ಪಿತೊರ್ವಧಶ್ಚಾಪಿ
ಪಿತ್ರರ್ಥೇ
ಪಿತ್ರಾ
ಪಿತ್ರಾರ್ಜಿತಂ
ಪಿತ್ರಾ-ಹಾರೋ
ಪಿತ್ರೇ
ಪಿಪಾಸಾ
ಪಿಪೀಲಿಕಾಂತಾ
ಪಿಪ್ಪಲೇ
ಪಿಪ್ಪಲೇ-ಶನ
ಪಿಪ್ಪಲೇ-ಶಪ್ರಸಾದೇನ
ಪಿಬಂತಿ
ಪಿಶಾಚ
ಪಿಶಾಚ-ಗಣಕ್ಕೆ
ಪಿಶಾಚ-ಗಣ-ದಲ್ಲಿ
ಪಿಶಾಚತಾಂ
ಪಿಶಾಚ-ದೇಹಃ
ಪಿಶಾಚ-ದೇಹವು
ಪಿಶಾಚಭ್ಯೇಶ್ಚ
ಪಿಶಾಚ-ಯಕ್ಷ-ನಾ-ಗಾನಾಂ
ಪಿಶಾಚಾ
ಪಿಶಾಚಾಃ
ಪಿಶಾಚಾನಾಂ
ಪಿಶಾಚಾ-ನಾಗ-ತಾನಿ-ಮಾನ್
ಪಿಶಾಚಾ-ವೃ-ತಯೇವ
ಪಿಶಾಚಾಸ್ತೇ
ಪಿಶಾಚಿ
ಪಿಶಾಚಿ-ಗಳ
ಪಿಶಾಚಿ-ಗಳನ್ನು
ಪಿಶಾಚಿ-ಗಳ-ಮೇಲೆ
ಪಿಶಾಚಿ-ಗಳಾಗಲಿ
ಪಿಶಾಚಿ-ಗ-ಳಾಗಿದ್ದೇವೆ
ಪಿಶಾಚಿ-ಗಳಿಂದ
ಪಿಶಾಚಿ-ಗಳಿಗೆ
ಪಿಶಾಚಿ-ಗಳಿರಾ
ಪಿಶಾಚಿ-ಗಳು
ಪಿಶಾಚಿ-ಗಳೂ
ಪಿಶಾಚಿಗೆ
ಪಿಶಾಚಿ-ಯಾಗಿ
ಪಿಶಾಚಿ-ಯಾಗಿದ್ದಾನೆ
ಪಿಶಾಚಿ-ಯಾಗಿದ್ದೇನೆ
ಪಿಶಾಚಿ-ಯಾಗುವನು
ಪಿಶಾಚಿ-ಯಾದನು
ಪಿಶಾಚಿ-ಯಾದೆ
ಪಿಶಾಚಿಯು
ಪಿಶಾಚಿಯೂ
ಪಿಶಾಚಿ-ಯೋನಿ-ಯಲ್ಲಿ
ಪಿಶಿತಾ-ಶನೌ
ಪಿಶುನೌ
ಪಿಷ್ಟಂ
ಪೀಠ-ದಲ್ಲಿ-ರುವ
ಪೀಠಸ್ಥಃ
ಪೀಠಾಧಿ-ಪತಿ-ಗ-ಳಾಗಿ
ಪೀಡಯಂತ್ಯೇನಂ
ಪೀಡಯಾ-ಮಾಸ
ಪೀಡಯಿತ್ವರ್ದ್ರ-ವಸ್ತ್ರ-ಕಮ್
ಪೀಡಿತಃ
ಪೀಡಿತ-ನಾಗಿಯೋ
ಪೀಡಿ-ತ-ನಾದ
ಪೀಡಿತ-ರಾದ
ಪೀಡಿತಸ್ತಿಷ್ಠ
ಪೀಡಿತಾಯ
ಪೀಡಿತೋ
ಪೀಡಿ-ತೋಮ್ಯ-ಹಮ್
ಪೀಡಿಸಿ-ಕೊಳ್ಳಲ್ಪಡುತ್ತಿದ್ದವು
ಪೀಡಿ-ಸುತ್ತವೆ
ಪೀಡೆ-ಯನ್ನುಂಟು-ಮಾಡಿತು
ಪೀತ-ಕೇತು
ಪೀತ-ಕೇತುಶ್ಚಿತ್ರ-ಕೇತು-ರಿತಿ
ಪೀತಾಂಬರಧಾರಿ-ಗಳಾದ
ಪೀತ್ವಾ
ಪುಂಡೇಷು
ಪುಂಡ್ರ-ಧಾ-ರಣಮ್
ಪುಂಸಃ
ಪುಂಸೋ
ಪುಂಸ್ತ್ವಂ
ಪುಣ್ಯ
ಪುಣ್ಯಂ
ಪುಣ್ಯ-ಕರ-ವಾದ
ಪುಣ್ಯ-ಕರ-ವಾ-ದುದು
ಪುಣ್ಯ-ಕರ್ಮ-ಗಳ
ಪುಣ್ಯ-ಕರ್ಮ-ಗಳು
ಪುಣ್ಯ-ಕರ್ಮ-ದಿಂದ
ಪುಣ್ಯ-ಕರ್ಮವೂ
ಪುಣ್ಯಕ್ಕಾಗಿ
ಪುಣ್ಯಕ್ಷೇತ್ರಕ್ಕೆ
ಪುಣ್ಯ-ಗಳನ್ನು
ಪುಣ್ಯ-ತಿಥಿ-ಗಳಲ್ಲಿ
ಪುಣ್ಯ-ದಲ್ಲಿ
ಪುಣ್ಯ-ದಿಂದ
ಪುಣ್ಯ-ದಿನತ್ರಯೇ
ಪುಣ್ಯ-ದಿನೇಷು
ಪುಣ್ಯ-ದಿವ-ಸವು
ಪುಣ್ಯ-ಪಾಪ-ಗಳೇ
ಪುಣ್ಯ-ಪಾಪೇ
ಪುಣ್ಯ-ಫಲಂ
ಪುಣ್ಯ-ಫಲ-ವನ್ನು
ಪುಣ್ಯ-ಮನಂತ-ಕಮ್
ಪುಣ್ಯ-ಮವಾಪ್ನೋತಿ
ಪುಣ್ಯ-ಮುತ್ತಮಮ್
ಪುಣ್ಯ-ರಾಶಿ-ಯನ್ನು
ಪುಣ್ಯ-ವಂತ-ರಾದ
ಪುಣ್ಯ-ವಂತರು
ಪುಣ್ಯ-ವನ್ನು
ಪುಣ್ಯ-ವನ್ನೂ
ಪುಣ್ಯ-ವನ್ನೆಲ್ಲ
ಪುಣ್ಯವು
ಪುಣ್ಯವೂ
ಪುಣ್ಯ-ಶೀಲಾಃ
ಪುಣ್ಯ-ಹೀನ-ರಾದ
ಪುಣ್ಯ-ಹೀನಾ
ಪುಣ್ಯಾ
ಪುಣ್ಯಾಃ
ಪುಣ್ಯಾನಿ
ಪುಣ್ಯಾಯ
ಪುಣ್ಯೇನ
ಪುಣ್ಯೋದಯಪಾದಪಶ್ಚ
ಪುತ್ರ
ಪುತ್ರಃ
ಪುತ್ರಕ
ಪುತ್ರತ್ವ-ಮಾಪ್ನು-ಯಾತ್
ಪುತ್ರ-ನನ್ನು
ಪುತ್ರ-ನಾಗಿ
ಪುತ್ರ-ನಾದ
ಪುತ್ರನೇ
ಪುತ್ರ-ಪೌತ್ರ-ಸಮಾವೃತೌ
ಪುತ್ರಯೋಃ
ಪುತ್ರ-ರಿದ್ದರೂ
ಪುತ್ರರು
ಪುತ್ರಸ್ಯ
ಪುತ್ರಾ
ಪುತ್ರಾಂತಿಕಮುಪಾ-ಯಯೌ
ಪುತ್ರಾಃ
ಪುತ್ರಾನ್
ಪುತ್ರಾ-ಭಾ-ವಾದಾ-ವಯೋಶ್ಚ
ಪುತ್ರಾಯ
ಪುತ್ರಾಯೈ
ಪುತ್ರಾಶ್ಚತ್ವಾ-ರಸ್ತೇ
ಪುತ್ರಾಸ್ತೇನ
ಪುತ್ರಿ-ಯರ
ಪುತ್ರಿ-ಯಾದ
ಪುತ್ರಿಯು
ಪುತ್ರೀತ್ವಮಾ-ಗತಾ
ಪುತ್ರೇ
ಪುತ್ರೇಣ
ಪುತ್ರೈಃ
ಪುತ್ರೋ
ಪುತ್ರೌ
ಪುನಂತಿ
ಪುನಂತ್ಯುರು-ಕಾಲೇನ
ಪುನಂತ್ಯೇತೇ
ಪುನಃ
ಪುನರಬ್ರುವನ್
ಪುನರಾ-ಗತಃ
ಪುನರುಜ್ಜೀವಿತಃ
ಪುನರುಜ್ಜೀವಿತೋ
ಪುನರ್ಜನ್ಮ
ಪುನರ್ಜನ್ಮ-ವಿರು-ವುದಿಲ್ಲ
ಪುನರ್ಜನ್ಮ-ವಿಲ್ಲ
ಪುನರ್ಜನ್ಮವೇ
ಪುನರ್ಜಾ-ಯತೇ
ಪುನರ್ಬಿಜತ್ವಮೇತಿ
ಪುನರ್ಭವೋ
ಪುನರ್ಯಾತ್ರಾಂ
ಪುನರ್ಯಾಮಿ
ಪುನಶ್ಚ
ಪುನಶ್ಚೋಕ್ತಂ
ಪುನಸ್ತದ್ವಿಮರ್ಶಿ-ತಮ್
ಪುನಾತಿ
ಪುನಾತೀತಿ
ಪುನಾತ್ಯೇವ
ಪುನೀತರನ್ನಾಗಿ
ಪುಮರ್ಥಾ
ಪುಮರ್ಥಾನಾಂ
ಪುಮಾನಿಚ್ಛತಿ
ಪುರಃ
ಪುರ-ಜನರೂ
ಪುರ-ವರ್ತಿನಃ
ಪುರಸ್ಕಾರ-ಮಾಡಿ-ದರೆ
ಪುರಸ್ಕೃತ್ಯ
ಪುರಸ್ಸರ-ವಾಗಿ
ಪುರಾ
ಪುರಾ-ಕಾಲೇ
ಪುರಾ-ಕೃತಂ
ಪುರಾಣಂ
ಪುರಾ-ಣ-ಗಳನ್ನು
ಪುರಾ-ಣ-ಗಳಲ್ಲಿ
ಪುರಾ-ಣ-ಪುರುಷನೂ
ಪುರಾ-ಣಾದಿ-ಗಳಲ್ಲಿ
ಪುರಾಣೇ
ಪುರಾ-ತನ
ಪುರಾ-ತ-ನಮ್
ಪುರಾ-ತನ-ವಮ್
ಪುರಾ-ತನ-ವಾದ
ಪುರಾಹಂ
ಪುರಾ-ಽಹಂ
ಪುರಿ
ಪುರೀ
ಪುರೀಂ
ಪುರೀ-ಮೇವ
ಪುರೀಮ್
ಪುರೀಷೇ
ಪುರುಷ
ಪುರುಷಂ
ಪುರುಷ-ನಾಗಲೀ
ಪುರುಷ-ನಿಗೆ
ಪುರುಷನು
ಪುರುಷನೂ
ಪುರುಷ-ನೊಬ್ಬನು
ಪುರುಷರ
ಪುರುಷ-ರನ್ನು
ಪುರುಷ-ರಾಗಲೀ
ಪುರುಷರು
ಪುರುಷರೇ
ಪುರುಷಶ್ಚೇತಿ
ಪುರುಷಸ್ತತ್ಮಾತ್ತದೈ-ವಾದ್ಭುತ-ದರ್ಶನಃ
ಪುರುಷಸ್ತಸ್ಮಾಛ್ರವಣಂ
ಪುರುಷಸ್ಯ
ಪುರುಷಾಃ
ಪುರುಷಾರ್ಥ-ಗಳ
ಪುರುಷಾರ್ಥ-ಗಳನ್ನೂ
ಪುರುಷಾರ್ಥ-ಗಳಿಗೆ
ಪುರುಷಾರ್ಥವೂ
ಪುರುಷಾರ್ಥೋಪಾಯ-ವನ್ನು
ಪುರುಷೈಃ
ಪುರುಷೈರ್ವಾಪಿ
ಪುರುಷೋ
ಪುರುಷೋತ್ತಮಃ
ಪುರೇ
ಪುರೋ
ಪುರೋ-ಗತಾಃ
ಪುರೋ-ಹಿತ-ನಾಗಿ
ಪುರೋ-ಹಿತ-ನಿಂದ
ಪುಲಿನಪರ್ಯಂಕೇ
ಪುಲ್ಕ
ಪುಷ್ಕರ
ಪುಷ್ಕರಂ
ಪುಷ್ಕರ-ಣಿ-ಗಳೂ
ಪುಷ್ಕರ-ಮಾ-ಸಾದ್ಯ
ಪುಷ್ಕರಿಣ್ಯೋ
ಪುಷ್ಕರೇ
ಪುಷ್ಕಲ-ದೇಶೇಷು
ಪುಷ್ಕಲ-ವೆಂಬ
ಪುಷ್ಕಲಾ
ಪುಷ್ಟಿ
ಪುಷ್ಟಿ-ಕರಿ-ಸುತ್ತವೆ
ಪುಷ್ಟಿಶ್ಚ
ಪುಷ್ಪ
ಪುಷ್ಪಂ
ಪುಷ್ಪಕ
ಪುಷ್ಪ-ಕ-ವಿಮಾನ-ವನ್ನೇರಿ
ಪುಷ್ಪ-ಗಳ
ಪುಷ್ಪ-ಗಳನ್ನು
ಪುಷ್ಪ-ಗಳಿಂದ
ಪುಷ್ಪ-ಗಳು
ಪುಷ್ಪ-ದಿಂದ
ಪುಷ್ಪ-ಮಂಡಪಮ್
ಪುಷ್ಪ-ಲಾ-ನಾಮ
ಪುಷ್ಪ-ವಿಲ್ಲದೇ
ಪುಷ್ಪಾದಿ-ಗಳಿಂದ
ಪುಷ್ಪಾರ್ಪಣೆ
ಪುಷ್ಪೇಣೈ-ಕೇನ
ಪುಷ್ಪೈಃ
ಪುಷ್ಪೈಶ್ಚ
ಪುಷ್ಪೌಘಂ
ಪುಷ್ಯ
ಪುಷ್ಯ-ಮಾಸೇ
ಪುಷ್ಯರೇ
ಪುಸ್ತಕಂ
ಪುಽರಾಹಂ
ಪೂಜ-ಕೇಷು
ಪೂಜನಂ
ಪೂಜ-ಯಂತಿ
ಪೂಜ-ಯತಿ
ಪೂಜ-ಯತೋ
ಪೂಜ-ಯದ್ಭಿರಯಂ
ಪೂಜ-ಯನ್
ಪೂಜ-ಯಿತ್ವಾ
ಪೂಜಯಿತ್ವಾನ್ಯ
ಪೂಜಯೇತ್ತಥಾ
ಪೂಜಯೇದ್ಯದಿ
ಪೂಜಯೇದ್ಯಸ್ತು
ಪೂಜಾ
ಪೂಜಾಂ
ಪೂಜಾಂತೇ
ಪೂಜಾ-ಕಾಲೇ
ಪೂಜಾತ್ಮಕ-ವೆಂದು
ಪೂಜಾ-ದಿ-ಗಳನ್ನು
ಪೂಜಾ-ನಂತರ-ದಲ್ಲಿ
ಪೂಜಾಯಾಃ
ಪೂಜಾಸ್ಥಳ-ವನ್ನು
ಪೂಜಿತಃ
ಪೂಜಿತ-ನಾಗುವನು
ಪೂಜಿತ-ನಾದನು
ಪೂಜಿತಶ್ಚ
ಪೂಜಿತಾಃ
ಪೂಜಿ-ತಾಶ್ಚ
ಪೂಜಿತೇ
ಪೂಜಿ-ತೇಷು
ಪೂಜಿತೋ
ಪೂಜಿತೋ-ಽಬ್ಧಿ
ಪೂಜಿಸದ
ಪೂಜಿಸದೇ
ಪೂಜಿಸ-ಬೇಕು
ಪೂಜಿ-ಸ-ಲಿಲ್ಲ
ಪೂಜಿ-ಸಲ್ಪಟ್ಟು
ಪೂಜಿಸಿ
ಪೂಜಿ-ಸಿದ
ಪೂಜಿಸಿ-ದಂತೆ
ಪೂಜಿಸಿ-ದರೂ
ಪೂಜಿಸಿ-ದರೆ
ಪೂಜಿಸಿ-ದ-ವ-ರನ್ನು
ಪೂಜಿ-ಸುತ್ತಾನೆಯೋ
ಪೂಜಿ-ಸುತ್ತಾರೆ
ಪೂಜಿಸುತ್ತಾರೆಯೋ
ಪೂಜಿಸುತ್ತಾರೋ
ಪೂಜಿ-ಸುತ್ತಿ-ರ-ಲಿಲ್ಲ
ಪೂಜಿ-ಸುವ
ಪೂಜಿ-ಸುವ-ವನು
ಪೂಜಿ-ಸು-ವುದ-ರಿಂದ
ಪೂಜಿಸು-ವುದಿಲ್ಲವೋ
ಪೂಜೆ
ಪೂಜೆ-ಗಾಗಿ
ಪೂಜೆ-ಗೊಂಡ
ಪೂಜೆ-ಮಾಡಿದ
ಪೂಜೆ-ಮಾಡಿ-ದರೆ
ಪೂಜೆಯ
ಪೂಜೆ-ಯನ್ನು
ಪೂಜೆ-ಯಲ್ಲಿ
ಪೂಜೆ-ಯಾಗಿ
ಪೂಜೆ-ಯಿಂದ
ಪೂಜ್ಯ
ಪೂಜ್ಯತೇ
ಪೂಜ್ಯಾ
ಪೂರ-ಯಂತಿ
ಪೂರಯತಾಶು
ಪೂರಿ
ಪೂರುಷಸೇವಿ-ತಮ್
ಪೂರುಷಾಃ
ಪೂರೈಸಿ
ಪೂರೈಸಿ-ದನು
ಪೂರ್ಣ-ಚಂದ್ರ-ನಂತೆ
ಪೂರ್ಣ-ನಾಗಿದ್ದೆ
ಪೂರ್ಣಪ್ರಜ್ಞ-ರೆಂಬ
ಪೂರ್ಣಪ್ರಜ್ಞಸ್ತೃತೀ-ಯಸ್ತು
ಪೂರ್ಣ-ಮನೋ-ರಥಃ
ಪೂರ್ಣ-ಯದ್ಭೂಮ್ರ-ತಾಮ್ರಾಕ್ಷಾ
ಪೂರ್ಣ-ರಾದ
ಪೂರ್ಣ-ವಾಗಿ
ಪೂರ್ಣ-ವಾಗುತ್ತವೆ
ಪೂರ್ಣ-ವಾದ
ಪೂರ್ಣಾ
ಪೂರ್ಣಾಯಾಂ
ಪೂರ್ಣಾಯುಃ
ಪೂರ್ಣೇಂದುಸಂಕಾಶ-ಮುಖಂ
ಪೂರ್ತಿ
ಪೂರ್ತಿ-ಯಾಗಿ
ಪೂರ್ವ
ಪೂರ್ವಂ
ಪೂರ್ವಕ
ಪೂರ್ವ-ಕರ್ಮಾನು-ರೂಪೇಣ
ಪೂರ್ವ-ಕರ್ಮಾನುರೊಧೇನ
ಪೂರ್ವ-ಕರ್ಮಾನುರೋಧೇನ
ಪೂರ್ವ-ಕರ್ಮಾರ್ಜಿ-ತಾನಿ
ಪೂರ್ವ-ಕ-ವಾಗಿ
ಪೂರ್ವ-ಜನ್ಮದ
ಪೂರ್ವ-ಜನ್ಮನಃ
ಪೂರ್ವ-ಜನ್ಮನಿ
ಪೂರ್ವ-ಜನ್ಮಸು
ಪೂರ್ವ-ದಲ್ಲಿ
ಪೂರ್ವ-ದಿಕ್ಕಿಗೆ
ಪೂರ್ವ-ದಿಕ್ಕಿ-ನಲ್ಲಿ
ಪೂರ್ವ-ದೇ-ಹಾಂಶ್ಚ
ಪೂರ್ವದ್ಯುಃ
ಪೂರ್ವ-ಪಾಪ-ಕರ್ಮ
ಪೂರ್ವ-ಪಾಪ-ವಶಾದಪಿ
ಪೂರ್ವ-ಪುಣ್ಯದ
ಪೂರ್ವ-ರೂಪಃ
ಪೂರ್ವ-ವೃತ್ತಾಂತ-ವನ್ನು
ಪೂರ್ವ-ವೃತ್ತಾಂತ-ವನ್ನೆಲ್ಲ
ಪೂರ್ವಸ್ಮಾತ್
ಪೂರ್ವಿ-ಕರು
ಪೃಕ್ತಾಮೃತಸಂಘಿ-ಸಂಗಿನೋ
ಪೃಚ್ಛಕಂ
ಪೃಚ್ಛತಿ
ಪೃಥಕ್
ಪೃಥಗೇವ
ಪೃಥಿವೀ
ಪೃಥಿವೀ-ತತ್ವಕ್ಕೆ
ಪೃಥಿವ್ಯಂತಾ
ಪೃಥಿವ್ಯಾತ್ಮಾ
ಪೃಥು
ಪೃಥೋಃ
ಪೃಷ್ಟಮತೋ
ಪೃಷ್ಟಸ್ತದಾ
ಪೃಷ್ಟಾಽಯುಷ್ಯಸ್ಯ
ಪೃಷ್ಟೋ
ಪೃಷ್ಠ
ಪೃಷ್ಠ-ಮನ್ವೀಯತುಃ
ಪೃಷ್ಠಾನುಯಾಯಿನಸ್ತೇಷಾಮಭ-ವನ್
ಪೈಂಗ್ಯ
ಪೈಂಗ್ಯ-ಗೋತ್ರ-ಸಮುದ್ಭವಃ
ಪೈಠೀನಸೀ
ಪೈಠೀನಸೀಕ್ಷೇತ್ರೇ
ಪೈತೃಕಾದೃ-ಣಾತ್
ಪೈತೃಷ್ವ
ಪೈಶಾಚೀಂ
ಪೈಶುನ್ಯನಾ-ದಿನಮ್
ಪೊದೆ
ಪೊದೆ-ಗಳ
ಪೊದೆ-ಗಳಲ್ಲಿ
ಪೊದೆ-ಗಳಲ್ಲಿಯೂ
ಪೋಚುಃ
ಪೋಷಿತಾಃ
ಪೋಷಿ-ಸುವ
ಪೋಸ್ಟ್
ಪೌತ್ರ
ಪೌರೋ-ಹಿತ್ಯೇ
ಪೌರ್ಣ-ಮಾಸ್ಯಾಂ
ಪೌರ್ಣಮ್ಯಾಂ
ಪೌರ್ಣಿಮಾ
ಪೌರ್ಣಿಮಾನಾಂ
ಪೌರ್ಣಿಮೆ
ಪೌರ್ಣಿಮೆ-ಗಳು
ಪೌರ್ಣಿಮೆ-ಯಲ್ಲಿ
ಪೌರ್ಣಿಮೆ-ಯಿಂದ
ಪೌರ್ಣಿಮೆ-ವರೆವಿಗೂ
ಪ್ತಾಪ್ತೋ
ಪ್ಯ
ಪ್ಯಾಕ್ಲಿಷ್ಯ
ಪ್ರಕಟನೆ
ಪ್ರಕಟ-ಪಡಿಸಿ-ದನು
ಪ್ರಕಾರ
ಪ್ರಕಾರ-ದಿಂದ
ಪ್ರಕಾರ-ವಾಗಿ-ರುತ್ತವೆ-ಯೆಂದು
ಪ್ರಕಾರ-ವುಳ್ಳದ್ದೆಂದು
ಪ್ರಕಾರ-ವೆಂದು
ಪ್ರಕಾರವೇ
ಪ್ರಕಾಶ
ಪ್ರಕಾಶಂತೇ
ಪ್ರಕಾಶ-ಕರು
ಪ್ರಕಾಶಕ್ಕೆ
ಪ್ರಕಾಶ-ದಿಂದ
ಪ್ರಕಾಶ-ಮಾನ-ನಾದ
ಪ್ರಕಾಶಶ್ಚ
ಪ್ರಕಾಶಿಸಿ-ದರು
ಪ್ರಕಾಶಿಸುತ್ತಿ-ರುವ
ಪ್ರಕೀರ್ತಿತಾಃ
ಪ್ರಕುರ್ವಂತಿ
ಪ್ರಕುರ್ವತೇ
ಪ್ರಕುರ್ವಾಣಂ
ಪ್ರಕುರ್ವೀತ
ಪ್ರಕೃತಿ
ಪ್ರಕೃತಿಂ
ಪ್ರಕೃತಿಃ
ಪ್ರಕೃತಿ-ಕಾರ್ಯ-ಗಳಾದ
ಪ್ರಕೃತಿ-ಜೀವಾನಾಂ
ಪ್ರಕೃತಿ-ಜೈರ್ಗುಣೈಃ
ಪ್ರಕೃತಿ-ಬದ್ಧ
ಪ್ರಕೃತಿಯ
ಪ್ರಕೃತಿ-ಯಿಂದ
ಪ್ರಕೃತಿಯು
ಪ್ರಕೃತಿಶ್ಚ
ಪ್ರಕೃತಿಸ್ಥಾ
ಪ್ರಕೃತ್ಯಾ
ಪ್ರಕೋಟ-ಮೂರ್ಧಾಶ್ಚ
ಪ್ರಕೋಷ್ಠ-ಮೂರ್ಧಾಯಂ
ಪ್ರಕ್ರಿಯೆ-ಗಳನ್ನು
ಪ್ರಕ್ಷಾ-ಲಯೇದ್ಯಸ್ತು
ಪ್ರಕ್ಷಾಲ್ಯ
ಪ್ರಗಾಥ
ಪ್ರಗಾಥನ
ಪ್ರಗಾಥ-ನಿಗೆ
ಪ್ರಗಾಥನು
ಪ್ರಗಾಥ-ಮನಿ-ಮಂತ್ರೈವ
ಪ್ರಗಾಥ-ಮುನಿಯ
ಪ್ರಗಾಥ-ಮುನಿಯು
ಪ್ರಗಾಥರೇ
ಪ್ರಗಾಥಸ್ತ
ಪ್ರಗಾಥಸ್ತ-ಪಸೇ
ಪ್ರಗಾಥಸ್ಯ
ಪ್ರಗಾಥೋ
ಪ್ರಗೃಹ್ಯ
ಪ್ರಚಾರಕ್ಕಾಗಿಯೇ
ಪ್ರಚಾರ-ದಲ್ಲಿ
ಪ್ರಚಾರ-ವಾಗುತ್ತದೆ
ಪ್ರಚೋದನೆ
ಪ್ರಚೋದಿಸುತ್ತದೆ
ಪ್ರಜ-ಹಾಸ
ಪ್ರಜಾಃ
ಪ್ರಜಾಪೀಡಾ-ಪರಾ-ಯಣೌ
ಪ್ರಜಾ-ಯತೇ
ಪ್ರಜೆ-ಗಳಿಗೆ
ಪ್ರಜ್ಞಾ
ಪ್ರಣಮ್ಯ
ಪ್ರಣ-ವ-ದಂತೆ
ಪ್ರಣ-ವಸ್ತು
ಪ್ರಣ-ವೇದ್ಯದಿ
ಪ್ರಣವೋ
ಪ್ರಣೇಮುಃ
ಪ್ರಣೇಮುಸ್ತಂ
ಪ್ರಣೇಮುಸ್ತೇ
ಪ್ರತಿ
ಪ್ರತಿಕ್ಷಣ-ದಲ್ಲಿಯೂ
ಪ್ರತಿ-ಗೃಹ್ಯ
ಪ್ರತಿಗ್ರಹೇಣ
ಪ್ರತಿಗ್ರಾಹ್ಯ
ಪ್ರತಿ-ಜೀವ-ಕೃತಾಃ
ಪ್ರತಿ-ಜೀವ-ವನ್ನೂ
ಪ್ರತಿಜ್ಞಾ
ಪ್ರತಿ-ದಿನ-ದಲ್ಲಿಯೂ
ಪ್ರತಿ-ದೇಹಿ
ಪ್ರತಿ-ನಿತ್ಯ
ಪ್ರತಿ-ನಿತ್ಯ-ದಲ್ಲಿಯೂ
ಪ್ರತಿ-ನಿತ್ಯವೂ
ಪ್ರತಿ-ಪತ್
ಪ್ರತಿ-ಪದ್ಯ
ಪ್ರತಿ-ಪದ್ಯತೇ
ಪ್ರತಿ-ಪಾದ-ಯತಿ
ಪ್ರತಿ-ಪಾದಿಸಿ
ಪ್ರತಿ-ಫಲ
ಪ್ರತಿ-ಬಂಧ-ಮುಗ್ರಮ್
ಪ್ರತಿ-ಬಂಧಾಶ್ಚ
ಪ್ರತಿ-ಬಿಂಬ
ಪ್ರತಿ-ಬಿಂಬ-ನಾಗಿ-ರಲು
ಪ್ರತಿ-ಬಿಂಬ-ನಾದ
ಪ್ರತಿ-ಬಿಂಬ-ನೆಂಬ
ಪ್ರತಿ-ಬಿಂಬ-ವನ್ನು
ಪ್ರತಿ-ಬಿಂಬವು
ಪ್ರತಿ-ಬಿಂಬವೂ
ಪ್ರತಿಮಾಂ
ಪ್ರತಿ-ಮೆ-ಗಳು
ಪ್ರತಿ-ಮೆ-ಯನ್ನು
ಪ್ರತಿ-ಯೊಂದು
ಪ್ರತಿ-ಯೊಬ್ಬ
ಪ್ರತಿ-ಯೊಬ್ಬ-ರಿಗೂ
ಪ್ರತಿ-ವರ್ಷಂ
ಪ್ರತಿ-ವರ್ಷವೂ
ಪ್ರತಿ-ವಾರ್ಷಿ-ಕಮ್
ಪ್ರತಿಶ್ರು-ತಮ್
ಪ್ರತಿಷ್ಠಾ
ಪ್ರತಿಷ್ಠಿತಾ
ಪ್ರತೀಕ್ಷತೇ
ಪ್ರತೀಚೀಂ
ಪ್ರತ್ಯ
ಪ್ರತ್ಯಕ್ಷಂ
ಪ್ರತ್ಯ-ಪಾದಿ
ಪ್ರತ್ಯ-ವಸ್ಥಾಯಿ
ಪ್ರತ್ಯಹಂ
ಪ್ರತ್ಯುತ್ಥಾ-ನಾದಿ-ಗಳನ್ನು
ಪ್ರತ್ಯುತ್ಥಾ-ಯಾರ್ಘ್ಯಪಾದಕೈಃ
ಪ್ರತ್ಯುತ್ಥಿತಂ
ಪ್ರತ್ಯುವಾಚ
ಪ್ರತ್ಯೂಚುಃ
ಪ್ರತ್ಯೂಚುಸ್ತೆ
ಪ್ರತ್ಯೇಕ
ಪ್ರತ್ಯೇಕಂ
ಪ್ರತ್ಯೇಕ-ವಾಗಿ
ಪ್ರಥತ್ಯ
ಪ್ರಥಮ
ಪ್ರಥಮೋ
ಪ್ರಥ-ಮೋಧ್ಯಾಯಃ
ಪ್ರದಕ್ಷಿಣಂ
ಪ್ರದಕ್ಷಿಣ-ಪದೇ-ಪದೇ
ಪ್ರದಕ್ಷಿಣಾಂ
ಪ್ರದಕ್ಷಿಣೆ
ಪ್ರದದಾತಿ
ಪ್ರದರ್ಶನ
ಪ್ರದರ್ಶನ-ಗಳೂ
ಪ್ರದರ್ಶಿ-ಸಲು
ಪ್ರದಾತಾರಂ
ಪ್ರದಾತಾರೊ
ಪ್ರದಾನಾಧಿಕ-ರಣ
ಪ್ರದಾನೇನ
ಪ್ರದೀಯ-ತಾಮ್
ಪ್ರದೇಶಕ್ಕೆ
ಪ್ರದೇಶ-ಗಳಲ್ಲಿ
ಪ್ರದೇಶ-ಗಳಲ್ಲಿಯೂ
ಪ್ರದೇಶ-ದಲ್ಲಿ
ಪ್ರದೇಶವು
ಪ್ರಧಾನ
ಪ್ರಧಾನಂ
ಪ್ರಧಾ-ನತಃ
ಪ್ರಧಾ-ನದ
ಪ್ರಧಾನ-ವಾದುದೇ
ಪ್ರಧಾ-ನವು
ಪ್ರಧಾನಾಂಗಂ
ಪ್ರಧಾನಾಖ್ಯಲಿಂಗ-ದೇಹೋ
ಪ್ರಧಾ-ನಾತ್
ಪ್ರಪದ್ಯಂತೇ
ಪ್ರಪದ್ಯೇ
ಪ್ರಪೂ-ಜ-ಯೇತ್
ಪ್ರಪೂಜಿತಃ
ಪ್ರಪೇದೇ
ಪ್ರಬಲ-ವಾಗಿದ್ದರೂ
ಪ್ರಬಲ-ವಾದ
ಪ್ರಬಲ-ವಾ-ದುದು
ಪ್ರಬೋಧ-ಯೇತ್
ಪ್ರಬೋಧಶ್ಚ
ಪ್ರಬೋಧಾಬ್ಧಿ
ಪ್ರಭವೋ
ಪ್ರಭಾ-ವತಃ
ಪ್ರಭಾವ-ದಿಂದ
ಪ್ರಭಾ-ವಾಚ್ಚ
ಪ್ರಭಾವೇನ
ಪ್ರಭಾಸ-ದಲ್ಲಿ
ಪ್ರಭಾಸೇ
ಪ್ರಭು-ಗಳ
ಪ್ರಭು-ಗಳು
ಪ್ರಭುಭ್ಯಸ್ತೇಭ್ಯಶ್ಚ
ಪ್ರಭು-ವಾಗಿದ್ದು
ಪ್ರಭು-ವಾಗಿ-ರುತ್ತಾರೆ
ಪ್ರಭು-ವಾದ
ಪ್ರಭುವೇ
ಪ್ರಭೇದ-ಗಳಿಂದ
ಪ್ರಭೋ
ಪ್ರಮದಾ-ಗಣೈಃ
ಪ್ರಮಾಣ
ಪ್ರಮಾಣಂ
ಪ್ರಮಾ-ಣ-ಗಳನ್ನು
ಪ್ರಮಾ-ಣ-ವಾಕ್ಯ-ಗಳು
ಪ್ರಮಾ-ಣ-ವಾದ
ಪ್ರಮಾ-ಣ-ವೆಂದು
ಪ್ರಮಾಥಿ
ಪ್ರಮಾ-ಥೀನಿ
ಪ್ರಮಾ-ದತಃ
ಪ್ರಮೇಯ
ಪ್ರಮೇ-ಯ-ಗಳನ್ನು
ಪ್ರಮೇ-ಯ-ಗಳಿಂದ
ಪ್ರಮೇ-ಯದ
ಪ್ರಮೇ-ಯ-ವನ್ನು
ಪ್ರಮೇ-ಯವು
ಪ್ರಯಚ್ಛಂತಿ
ಪ್ರಯಚ್ಛತಿ
ಪ್ರಯಚ್ಛತ್ಯೌಷಧಂ
ಪ್ರಯತೇತ
ಪ್ರಯತ್ನ
ಪ್ರಯತ್ನತಃ
ಪ್ರಯತ್ನ-ಪಟ್ಟರೂ
ಪ್ರಯತ್ನಿಸ-ಬೇಕು
ಪ್ರಯತ್ನಿಸುತ್ತಿ-ರುವ
ಪ್ರಯತ್ನೇನ
ಪ್ರಯಯೌ
ಪ್ರಯಾಗ
ಪ್ರಯಾಗಂ
ಪ್ರಯಾಗಕ್ಷೇತ್ರ
ಪ್ರಯಾಗ-ದರ್ಶನ-ಮಾತ್ರ-ದಿಂದಲೇ
ಪ್ರಯಾಗ-ದರ್ಶನಾನ್ಮು
ಪ್ರಯಾಗ-ದಲ್ಲಿ
ಪ್ರಯಾಗ-ಮುತ್ತಮಂ
ಪ್ರಯಾಗಸ್ಮ-ರಣಂ
ಪ್ರಯಾಗಸ್ಮ-ರಣ-ಮಾತ್ರ-ದಿಂದ
ಪ್ರಯಾಗೇ
ಪ್ರಯಾಣ
ಪ್ರಯಾಣ-ಮಾಡಿ
ಪ್ರಯಾಣಾಯೋದ್ಯತೇ
ಪ್ರಯಾಣಾಯೋಪಚಕ್ರಮೇ
ಪ್ರಯಾತಂ
ಪ್ರಯಾಸವೇ
ಪ್ರಯುಕ್ತ
ಪ್ರಯುಕ್ತಂ
ಪ್ರಯುಕ್ತ-ವಾಸ್ತೇ
ಪ್ರಯೋ-ಜನ-ಕಾರಿ-ಯಾಗಿ
ಪ್ರಯೋ-ಜನವೇ
ಪ್ರಯೋಜಿ-ತಮ್
ಪ್ರಳಯ-ಕಾಲ-ದ-ವರೆಗೂ
ಪ್ರವಕ್ತಾರಂ
ಪ್ರವಕ್ರೄಣಾಂ
ಪ್ರವಕ್ಷ್ಯಾಮಿ
ಪ್ರವಕ್ಷ್ಯಾಮೋ
ಪ್ರವಕ್ಷ್ಯಾಮ್ಯಜ್ಞಯಾ
ಪ್ರವ-ಚನ
ಪ್ರವ-ಚನ-ಗಳಲ್ಲಿ
ಪ್ರವ-ಚನ-ದಲ್ಲಿ
ಪ್ರವ-ಚನ-ಮಾಡಲ್ಪಡುತ್ತಿದ್ದಾಗ
ಪ್ರವತ
ಪ್ರವ-ತೇತಿ
ಪ್ರವತೋ
ಪ್ರವರೋ
ಪ್ರವರ್ತಂತೇ
ಪ್ರವರ್ತಕಾಃ
ಪ್ರವರ್ತತೇ
ಪ್ರವರ್ತಿ-ಸುತ್ತವೆ
ಪ್ರವರ್ತಿ-ಸು-ವಂತೆ
ಪ್ರವಾದ-ದಿಂದಲೂ
ಪ್ರವಾಲ-ಮುಕ್ತಾ-ಸ-ಹಿತಂ
ಪ್ರವಾಹ
ಪ್ರವಾಹಕ್ಕೆ
ಪ್ರವಿಷ್ವೋಽಂತಃಪುರಂ
ಪ್ರವೀಣನೂ
ಪ್ರವೃತರಾಗದ-ವ-ರಿಗೆ
ಪ್ರವೃತ್ತಯೇ
ಪ್ರವೃತ್ತಾ
ಪ್ರವೃತ್ತಿ
ಪ್ರವೃತ್ತಿಂ
ಪ್ರವೃತ್ತಿ-ಮಾರ್ಗ-ಕರ್ಮ
ಪ್ರವೃತ್ತಿ-ಯಾಗುತ್ತದೆ
ಪ್ರವೃತ್ತೇ
ಪ್ರವೃತ್ತೋಹಂ
ಪ್ರವೇಶ-ಮಾಡಿ-ದರು
ಪ್ರವೇಶಿಸಿ
ಪ್ರವೇಷ್ಟುಂ
ಪ್ರಶಂಸಂತಿ
ಪ್ರಶಂಸಿಸಿ
ಪ್ರಶಸ್ತ-ಮಿತಿ
ಪ್ರಶಸ್ತ-ವಾದ
ಪ್ರಶಸ್ಯತೇ
ಪ್ರಶ್ನಃ
ಪ್ರಶ್ನಿಸಲ್ಪಟ್ಟ
ಪ್ರಶ್ನಿಸಿ-ದನು
ಪ್ರಶ್ನೆ
ಪ್ರಶ್ನೆ-ಗಳಿಂದ
ಪ್ರಶ್ನೆ-ಗಳಿಗೆ
ಪ್ರಸಂಗಾ
ಪ್ರಸಂಗೇನ
ಪ್ರಸನ್ನ
ಪ್ರಸನ್ನಂ
ಪ್ರಸನ್ನೇನಾಂತರಾತ್ಮನಾ
ಪ್ರಸಭಂ
ಪ್ರಸಾದ
ಪ್ರಸಾದಂ
ಪ್ರಸಾ-ದವು
ಪ್ರಸಾದಾತ್
ಪ್ರಸಾ-ದಾನ್ಮಾಘಸ್ಯ
ಪ್ರಸಾದಿತಸ್ತೇನ
ಪ್ರಸಾದೊ
ಪ್ರಸಾದ್ಯ
ಪ್ರಸಾದ್ಯಾಥ
ಪ್ರಸಿದ್ದ-ವಾದ
ಪ್ರಸಿದ್ಧ-ನಾದ
ಪ್ರಸಿದ್ಧ-ವಾಗಿವೆ
ಪ್ರಸೀದ
ಪ್ರಸೀದತಿ
ಪ್ರಸೂಯತೇ
ಪ್ರಸ್ಥಾ
ಪ್ರಹ-ರಕ್ಕೆ
ಪ್ರಹರ್ತರಿ
ಪ್ರಹೃಷ್ಟೇ
ಪ್ರಹೃಷ್ಟೌ
ಪ್ರಾಂಜಲರ್ಯೋಽಭ-ವನ್
ಪ್ರಾಂತೀಯ
ಪ್ರಾಗಾತ್
ಪ್ರಾಗಾತ್ತತೋ
ಪ್ರಾಗಾಥಾಗನ-ಪರ್ಯಂತಂ
ಪ್ರಾಗಾಥೋಪ್ಯ-ಪಿಬದ್ರಸಮ್
ಪ್ರಾಗಾದ್ದೃಷ್ಟ್ವಾ-ವಸ್ಠಾಂ
ಪ್ರಾಚೀಂ
ಪ್ರಾಜ್ಞಃ
ಪ್ರಾಣ-ಗಳೇ
ಪ್ರಾಣನೂ
ಪ್ರಾಣಪರೀಪ್ಸು
ಪ್ರಾಣಪರೀಪ್ಸುಕಃ
ಪ್ರಾಣ-ರಕ್ಷಣೆ-ಗಾಗಿ
ಪ್ರಾಣಸ್ತ್ವಗಾತ್ಮಾ
ಪ್ರಾಣಾ
ಪ್ರಾಣಾಶ್ಛ-ದಾನಿ
ಪ್ರಾಣಿ
ಪ್ರಾಣಿ-ಗಳ
ಪ್ರಾಣಿ-ಗಳನ್ನು
ಪ್ರಾಣಿ-ಗಳಲ್ಲಿ
ಪ್ರಾಣಿ-ಗಳು
ಪ್ರಾಣಿಯೂ
ಪ್ರಾಣಿ-ಹಿಂಸಕ
ಪ್ರಾಣಿ-ಹಿಂಸಕಃ
ಪ್ರಾತಃ
ಪ್ರಾತಃಕಾಲ
ಪ್ರಾತಃಕಾಲದ
ಪ್ರಾತಃಕಾಲ-ದಲ್ಲಿ
ಪ್ರಾತಃಕಾಲ-ದಿಂದ
ಪ್ರಾತಃಕಾಲ-ವಾ-ಗಲು
ಪ್ರಾತಃಕಾಲೇ
ಪ್ರಾತಃಕಾಲೋ
ಪ್ರಾತಃಪ್ರಾತಃ
ಪ್ರಾತಃಸ್ಕಾನಂ
ಪ್ರಾತಃಸ್ನಾನ
ಪ್ರಾತಃಸ್ನಾನಂ
ಪ್ರಾತಃಸ್ನಾ-ನ-ದಲ್ಲಿ
ಪ್ರಾತಃಸ್ನಾ-ನ-ನುತಂದ್ರಿತಃ
ಪ್ರಾತಃಸ್ನಾ-ನ-ವನ್ನು
ಪ್ರಾತಃಸ್ನಾ-ನವು
ಪ್ರಾತಃಸ್ನಾ-ನಾತ್ಪು-ನೀಮಹೇ
ಪ್ರಾತಃಸ್ನಾನೇ
ಪ್ರಾತಃಸ್ನಾ-ನೇಷು
ಪ್ರಾತಃಸ್ಮಾನಂ
ಪ್ರಾತರಾರಭ್ಯ
ಪ್ರಾತರುತ್ಥಾಯ
ಪ್ರಾತರ್ಜಲಸ್ಪರ್ಶ-ನ-ಪುಣ್ಯ-ಲೇಶಾತ್
ಪ್ರಾತರ್ಜಲಸ್ಪರ್ಶ-ನ-ಮಾತ್ರ-ತೋಽಂಗ
ಪ್ರಾತರ್ಜಲಾಶ್ರಯೇ
ಪ್ರಾತರ್ನವಾಯೈ
ಪ್ರಾತರ್ನಾಡೀಚತುಷ್ಟಯೇ
ಪ್ರಾತರ್ಮಯಾ
ಪ್ರಾತರ್ಮಾಘೇ
ಪ್ರಾತರ್ಮೇ
ಪ್ರಾತರ್ಹೋಮಂ
ಪ್ರಾತಸ್ಮಾ-ನರತಾ
ಪ್ರಾತಿಲೋಭ್ಯೇನ
ಪ್ರಾದಾತ್ತಸ್ಯ
ಪ್ರಾದಾದ್ರಾಜ್ಞೋ
ಪ್ರಾದುರ್ಭಾವತ್ರಯಾನ್ವಿತಃ
ಪ್ರಾದ್ರವಂತಂ
ಪ್ರಾಧಾನ್ಯಂ
ಪ್ರಾಪಯಾ-ಮಾಸ
ಪ್ರಾಪ-ಯಿತ್ವಾ
ಪ್ರಾಪಿತಂ
ಪ್ರಾಪುಂ
ಪ್ರಾಪುಃ
ಪ್ರಾಪ್ತಂ
ಪ್ರಾಪ್ತ-ವಾ-ಗಲು
ಪ್ರಾಪ್ತ-ವಾಗಿ
ಪ್ರಾಪ್ತ-ವಾಗಿತ್ತು
ಪ್ರಾಪ್ತ-ವಾಗಿದೆ
ಪ್ರಾಪ್ತ-ವಾಗುತ್ತದೆ
ಪ್ರಾಪ್ತ-ವಾಗುತ್ತ-ದೆಯೋ
ಪ್ರಾಪ್ತ-ವಾಗುತ್ತವೆ
ಪ್ರಾಪ್ತ-ವಾಗು-ವಂತೆಯೂ
ಪ್ರಾಪ್ತ-ವಾಗು-ವುದಿಲ್ಲ
ಪ್ರಾಪ್ತ-ವಾ-ಯಿತು
ಪ್ರಾಪ್ತಾ
ಪ್ರಾಪ್ತಿಗೆ
ಪ್ರಾಪ್ತುಂ
ಪ್ರಾಪ್ತೇ
ಪ್ರಾಪ್ತೋ
ಪ್ರಾಪ್ತೌ
ಪ್ರಾಪ್ನುವಂತಿ
ಪ್ರಾಪ್ನೋತಿ
ಪ್ರಾಪ್ನೋತೀತ್ಯಬ್ರವೀದ್ಧರಿಃ
ಪ್ರಾಪ್ನೋತ್ಯ
ಪ್ರಾಪ್ಯ
ಪ್ರಾಯಃ
ಪ್ರಾಯಶ್ಚಿತ್ತಂ
ಪ್ರಾಯಶ್ಚಿತ್ತ-ಗಳೂ
ಪ್ರಾಯಶ್ಚಿತ್ತ-ರೂಪ-ವಾಗಿ
ಪ್ರಾಯಶ್ಚಿತ್ತ-ವಾಗಿ
ಪ್ರಾಯಶ್ಚಿತ್ತ-ವಿಲ್ಲದ
ಪ್ರಾಯಶ್ಚಿತ್ತ-ವಿ-ವರ್ಜಿತೇ
ಪ್ರಾಯಶ್ಚಿತ್ತವು
ಪ್ರಾಯಶ್ಚಿತ್ತ-ವೆಂದು
ಪ್ರಾಯಶ್ಚಿತ್ತಾನಿ
ಪ್ರಾಯಾತ್ಸನಂದೋ
ಪ್ರಾಯೋಯಂ
ಪ್ರಾರಂಭ
ಪ್ರಾರಂಭಿಸಿ
ಪ್ರಾರಂಭಿಸಿ-ದನು
ಪ್ರಾರಂಭಿಸಿ-ದರು
ಪ್ರಾರಂಭಿ-ಸಿದೆ
ಪ್ರಾರಬ್ದ-ವನ್ನು
ಪ್ರಾರಬ್ಧೇತ-ರ-ನಾದ
ಪ್ರಾರ್ಥನಾ-ಪೂರ್ವಕಂ
ಪ್ರಾರ್ಥನೆ
ಪ್ರಾರ್ಥನೆ-ಯನ್ನು
ಪ್ರಾರ್ಥಯಾ-ಮಾಸ
ಪ್ರಾರ್ಥಯಾ-ಮಾಸುರಂಜಸಾ
ಪ್ರಾರ್ಥಿತಃ
ಪ್ರಾರ್ಥಿತೋಪಿ
ಪ್ರಾರ್ಥಿಸ-ಬೇಕು
ಪ್ರಾರ್ಥಿಸಲ್ಪಟ್ಟ
ಪ್ರಾರ್ಥಿಸಿ
ಪ್ರಾರ್ಥಿಸಿ-ಕೊಳ್ಳಲ್ಪಟ್ಟ
ಪ್ರಾರ್ಥಿಸಿ-ದನು
ಪ್ರಾರ್ಥಿಸಿ-ದರು
ಪ್ರಾರ್ಥಿ-ಸಿದೆ
ಪ್ರಾರ್ಥಿಸುತ್ತೇನೆ
ಪ್ರಾರ್ಥಿ-ಸುವ-ವರೂ
ಪ್ರಾರ್ಥ್ಯ
ಪ್ರಾವಹತ್ವೈವ
ಪ್ರಾವೀಣ್ಯತೆ-ಯನ್ನು
ಪ್ರಾಶನ-ಮಾಡಿ-ದಂತಾಗುತ್ತದೆ
ಪ್ರಾಶನ-ಮಾಡುತ್ತಾ-ರೆಯೋ
ಪ್ರಾಶ-ನ-ವಿಲ್ಲ
ಪ್ರಾಶಸ್ತ್ಯಂ
ಪ್ರಾಹ
ಪ್ರಾಹಿಣೋದಾಶ್ರಮಾಯ
ಪ್ರಾಹುಃ
ಪ್ರಾಹುರಾಸ್ತಿಕೀಂ
ಪ್ರಾಹುರ್ಮನೀಷಿಣಃ
ಪ್ರಿಯ
ಪ್ರಿಯಃ
ಪ್ರಿಯ-ತಮೇ
ಪ್ರಿಯ-ಮಿತ್ಯು-ದಿತೇಪಿ
ಪ್ರಿಯಮ್
ಪ್ರಿಯ-ವಾಗಿ-ರುವ
ಪ್ರಿಯ-ವಾದ
ಪ್ರೀಣಂತಿ
ಪ್ರೀತಿ
ಪ್ರೀತಿಃ
ಪ್ರೀತಿ-ಕರಂಭೂಯಸ್ತತೋಪಿ
ಪ್ರೀತಿ-ಕರ-ವಾದ
ಪ್ರೀತಿ-ಗಾಗಿ
ಪ್ರೀತಿಗೆ
ಪ್ರೀತಿ-ಯನ್ನಿಟ್ಟು
ಪ್ರೀತಿ-ಯಿಲ್ಲ
ಪ್ರೀತ್ಯರ್ಥ-ವಾಗಿ
ಪ್ರೀತ್ಯರ್ಥ-ವಾಗಿಯೇ
ಪ್ರೀತ್ಯಾ
ಪ್ರೀತ್ಯಾ-ದರ-ಗಳಿಂದ
ಪ್ರೀತ್ಯೈ
ಪ್ರೇತ
ಪ್ರೇತ-ಗಳು
ಪ್ರೇತ-ಜನ್ಮವು
ಪ್ರೇತತ್ವಂ
ಪ್ರೇತ-ಯೋನಿ-ಯಲ್ಲಿದ್ದುವು
ಪ್ರೇತ-ಯೋನಿ-ಯಲ್ಲಿದ್ದೇವೆ
ಪ್ರೇತಾವಾಹಾರರ-ಹಿತೌ
ಪ್ರೇತಾವೌದುಂಬರಾಶ್ರಯೌ
ಪ್ರೇತಾವೌದುಂಬರೇ
ಪ್ರೇತ್ಯ
ಪ್ರೇಮಕಾತರಃ
ಪ್ರೇರಕಃ
ಪ್ರೇರಕ-ನಾಗಿ
ಪ್ರೇರಣಾನು-ಸಾರ-ವಾಗಿ
ಪ್ರೇರಣೆ-ಮಾಡು-ವ-ವನೂ
ಪ್ರೇರ-ಯಿತ್ವಾ
ಪ್ರೇರಿತಃ
ಪ್ರೇರಿ-ಸಲು
ಪ್ರೇರಿಸಿ
ಪ್ರೇಷಯಾ-ಮಾಸ
ಪ್ರೇಷಿತಾ
ಪ್ರೇಷಿತೇ
ಪ್ರೊಚುಶ್ಚ
ಪ್ರೋಕ್ತಂ
ಪ್ರೋಕ್ತಾ
ಪ್ರೋಕ್ತೋ
ಪ್ರೋಕ್ಷಣೆ
ಪ್ರೋಕ್ಷಣೆ-ಯ-ನಂತರ
ಪ್ರೋಕ್ಷ-ಯೇತ್
ಪ್ರೋಕ್ಷಿಸ-ಬೇಕು
ಪ್ರೋಕ್ಷಿಸಿ-ದನು
ಪ್ರೌಢಳಾ-ದಳು
ಪ್ರೌಢಾ
ಫಲ
ಫಲಂ
ಫಲಂತಿ
ಫಲ-ಕಾರಿ
ಫಲ-ಕಾರಿ-ಯಾಗುತ್ತದೆ
ಫಲ-ಕಾರಿ-ಯಾಗುತ್ತವೆ
ಫಲ-ಕಾರಿ-ಯಾ-ಯಿತು
ಫಲ-ಕೊ-ಡಲು
ಫಲ-ಕೊಡುತ್ತಾ-ರೆಂಬ
ಫಲಕ್ಕಿಂತ
ಫಲ-ಗಳ
ಫಲ-ಗಳನ್ನು
ಫಲ-ಗಳನ್ನೂ
ಫಲ-ಗಳಿಂದ
ಫಲ-ಗಳು
ಫಲ-ಗಳೂ
ಫಲ-ಗಳೇ
ಫಲತ್ಯಹೋ
ಫಲತ್ವೇವ
ಫಲದ
ಫಲದಂ
ಫಲದತ್ವಂ
ಫಲ-ದಲ್ಲಿ
ಫಲದಾ
ಫಲದಾ-ಯಕ
ಫಲದಾ-ಯಕ-ವೆಂದು
ಫಲ-ಪುಷ್ಪ-ಗಳಿಂದ
ಫಲಪ್ರದ-ವಮ್
ಫಲಪ್ರದೇ
ಫಲಪ್ರಾಪ್ತಿ
ಫಲ-ಭೋಕ್ತಾ
ಫಲಮ್
ಫಲಯುತಂ
ಫಲ-ರಾಶಿವಿ-ಮೋಕ್ಷ-ಣಮ್
ಫಲ-ವನ್ನಾದರೂ
ಫಲ-ವನ್ನು
ಫಲ-ವನ್ನೂ
ಫಲ-ವನ್ನೇ
ಫಲ-ವಾಗಿ
ಫಲ-ವಿಲ್ಲ
ಫಲ-ವಿಷ-ಯ-ದಲ್ಲಿ
ಫಲವು
ಫಲವೂ
ಫಲ-ವೆಂಬು-ದನ್ನು
ಫಲವೇ
ಫಲ-ಸಾಧನಜ್ಞಾ-ನೋತ್ಪತ್ತಿಗೆ
ಫಲ-ಸಾಧನಾ
ಫಲಾದಿ-ಕಮ್
ಫಲಾದಿ-ಗಳನ್ನು
ಫಲಾಧಿಕ್ಯಂ
ಫಲಾಪೇಕ್ಷಾರ-ಹಿತ-ನಾಗಿ
ಫಲಾಪೇಕ್ಷೆ-ಯಿಂದ
ಫಲಿತಂ
ಫಲೇ
ಫಲೇಚ್ಛಾರ-ಹಿತ-ನಾಗಿ
ಫಲೇಪ್ಸ
ಫಲೇಷು
ಫೇನಿಲಂ
ಫೇನಿಲೋ
ಬಂಗಾರದ
ಬಂಜೆಯು
ಬಂತು
ಬಂದ
ಬಂದನು
ಬಂದ-ಮೇಲೆ
ಬಂದರು
ಬಂದರೂ
ಬಂದರೆ
ಬಂದಳು
ಬಂದವು
ಬಂದಾಗ
ಬಂದಿತು
ಬಂದಿತ್ತು
ಬಂದಿದೆ
ಬಂದಿದ್ದ
ಬಂದಿದ್ದೆವು
ಬಂದಿ-ರುವ
ಬಂದಿ-ರುವ-ವನು
ಬಂದಿ-ರು-ವಾಗ
ಬಂದಿಲ್ಲ
ಬಂದಿವೆ
ಬಂದು
ಬಂದು-ಬಿಡು
ಬಂದುವು
ಬಂದೆ
ಬಂದೆವು
ಬಂಧ
ಬಂಧ-ಕ-ವಾದ
ಬಂಧ-ಗುಲ್ಮಂ
ಬಂಧ-ನಕ್ಕೆ
ಬಂಧ-ನ-ಗಳ
ಬಂಧ-ನ-ಗಳಿಗೆ
ಬಂಧ-ನ-ಗಳು
ಬಂಧ-ನ-ದಿಂದ
ಬಂಧ-ನ-ನಿ-ವೃತ್ತಿಯೂ
ಬಂಧ-ನ-ವನ್ನು
ಬಂಧ-ನ-ವನ್ನೂ
ಬಂಧ-ನ-ವಿರು-ವುದಿಲ್ಲ
ಬಂಧ-ನ-ವಿಲ್ಲ
ಬಂಧ-ನವು
ಬಂಧ-ನ-ವೆಂಬ
ಬಂಧ-ನಿ-ವೃತ್ತಿ-ಯೆಂಬ
ಬಂಧ-ಮೋ-ಚನಃ
ಬಂಧ-ಯಿತ್ವೇತಿ
ಬಂಧಸ್ಯೈ-ತಸ್ಯ
ಬಂಧಾದೇ-ತದ್ವಿಸ್ತಾರ್ಯ
ಬಂಧಾನಾಂ
ಬಂಧಿ-ತ-ನಾದ
ಬಂಧಿತ-ರಾ-ಗುತ್ತಾರೆ
ಬಂಧಿತ-ಳಾಗಿದ್ದಳು
ಬಂಧಿಸಲ್ಪಟ್ಟ
ಬಂಧಿಸಲ್ಪಡುತ್ತಾನೆ
ಬಂಧುಃ
ಬಂಧು-ಗಳನ್ನೂ
ಬಂಧು-ಜನರ
ಬಂಧುಭಿಃ
ಬಂಧುರ್ಭುವಿ
ಬಂಧು-ವಿಲ್ಲ
ಬಂಧುವೆಂದಾಗಲೀ
ಬಂಧೋ
ಬಕುಲ-ಪುಷ್ಪ-ಗಳು
ಬಗೆ
ಬಗೆ-ಪಾಪ
ಬಗೆಯ
ಬಗೆ-ಯನ್ನು
ಬಗೆ-ಯನ್ನೂ
ಬಗೆ-ಯಲ್ಲಿ
ಬಗೆ-ಯಾಗಿ-ರು-ವುದ-ರಿಂದ
ಬಗೆ-ಯಾದ
ಬಗೆ-ಯಿಂದ
ಬಗ್ಗೆ
ಬಗ್ಗೆಯೇ
ಬಟ್ಟೆ-ಯನ್ನು
ಬಡಿಸುತ್ತಿದ್ದಿಲ್ಲ
ಬಡ್ಡಿ
ಬಡ್ಡಿಯ
ಬಡ್ಡಿ-ಯನ್ನು
ಬಣ್ಣ
ಬಣ್ಣ-ಗಳಿಂದ
ಬಣ್ಣದ
ಬಣ್ಣ-ದಂತೆ
ಬಣ್ಣವು
ಬತ
ಬತ್ತಿ
ಬತ್ತಿ-ಗಳನ್ನು
ಬದನೆ-ಕಾಯಿ
ಬದನೇ
ಬದರಿಕಾಶ್ರಮಕ್ಕೆ
ಬದರಿಕಾಶ್ರಮಮ್
ಬದಲಾಗು-ವುದಿಲ್ಲ
ಬದಲಾ-ಯಿತು
ಬದುಕಿ-ದನು
ಬದುಕಿದ್ದನು
ಬದುಕಿ-ರುವ
ಬದುಕಿಸಿ-ದರು
ಬದುಕಿಸುತ್ತೇನೆ
ಬದ್ಧ-ರಾಗಿ-ರುತ್ತಾರೆ
ಬದ್ಧ-ವೃಕ್ಷಾಶ್ರಯಾ
ಬಧಿರೋ
ಬಧ್ಯತೇ
ಬಧ್ಯಾ
ಬನ್ನಿರಿ
ಬಭಕ್ಷ-ತುಸ್ತತೋ
ಬಭಾಷಿರೇ
ಬಭಾಷೇ
ಬಭೂವುರ್ಮುನಿ-ಪುತ್ರ
ಬಯ-ಸುತ್ತಾ
ಬಯ-ಸುತ್ತಿದ್ದ
ಬಯ-ಸು-ವ-ವನು
ಬಯ-ಸುವ-ವರು
ಬಯಸು-ವು-ದಾದರೆ
ಬಯ್ಯುತ್ತಿದ್ದನು
ಬರದಿರಲೆಂದು
ಬರ-ಬಾರದು
ಬರಲಿ
ಬರಲು
ಬರಲೇ
ಬರುತ್ತದೆ
ಬರುತ್ತದೆ-ಯೆಂದು
ಬರುತ್ತವೆ
ಬರುತ್ತಾನೆ
ಬರುತ್ತಾರೆಂದು
ಬರುತ್ತಿದ್ದ
ಬರುತ್ತಿದ್ದನು
ಬರುತ್ತಿದ್ದು-ದನ್ನು
ಬರುತ್ತಿದ್ದೆ
ಬರುವ
ಬರು-ವಂತೆ
ಬರು-ವ-ತನಕ
ಬರು-ವ-ರೆಂದು
ಬರು-ವುದಿಲ್ಲ
ಬರೆದು
ಬರೆಯ-ಬೇಕು
ಬಲ
ಬಲಃ
ಬಲ-ಭಾಗದ
ಬಲವಂತಿ
ಬಲ-ವತ್ತರ-ವಾ-ದುದು
ಬಲ-ವದ್ದೃಢಮ್
ಬಲ-ವಾ-ದುದು
ಬಲ-ವಾನ್
ಬಲಾಡ್ಯನು
ಬಲಾತ್
ಬಲಾತ್ಕಾರ-ದಿಂದ
ಬಲಾತ್ಕಾರ-ವಾಗಿ
ಬಲಾನ್ವಿತೌ
ಬಲಾಬಲವಿಚಿಂತ-ನಮ್
ಬಲಿಷ್ಟ-ನಾಗಿದ್ದೆ
ಬಲಿಷ್ಟ-ವಾಗಿ
ಬಲಿಷ್ಟ-ವಾದ
ಬಲಿಷ್ಠ
ಬಲಿಷ್ಠರು
ಬಲಿಷ್ಠ-ವಾಗಿದ್ದರೂ
ಬಲಿಷ್ಠ-ವಾ-ದುದು
ಬಲೀ
ಬಲೀ-ಯಸೀ
ಬಲೀ-ವರ್ದೊದ್ವಾ-ಹಿ-ತಾನಿ
ಬಲ್ಲ
ಬಲ್ಲ-ವನು
ಬಲ್ಲ-ವನೇ
ಬಲ್ಲ-ವರು
ಬಳಲಿ
ಬಳಲಿದ
ಬಳಲಿದ್ದ
ಬಳಲುತ್ತಿದ್ದ
ಬಳಲುವ
ಬಳಿ
ಬಳಿಕ
ಬಳಿಗೆ
ಬಳಿ-ಯಲ್ಲಿದ್ದು
ಬಹಳ
ಬಹಳ-ಕಾಲ
ಬಹಳ-ವಾಗಿ
ಬಹವಃ
ಬಹವೋ
ಬಹವೋಪಿ
ಬಹಿಃಸ್ಥಲೇ
ಬಹಿಃಸ್ನಾನಂ
ಬಹಿಃಸ್ನಾ-ನ-ರ-ತಸ್ಯ
ಬಹಿರಾಗತ್ಯ
ಬಹಿರ್ಗಚ್ಛೇದ್ಯದಾ
ಬಹಿರ್ಗಮನ-ಮಾತ್ರೇಣ
ಬಹಿರ್ಜನ-ಪದೇ
ಬಹಿರ್ಜಲಮ್
ಬಹಿರ್ಜಲೇ
ಬಹಿರ್ವಾಪ್ಯೋಪ್ಯ-ವಭೃಥ-ಫ-ಲದಾಃ
ಬಹಿಷ್ಕ-ತ-ನಾದ
ಬಹು
ಬಹು-ಕರ್ಮ-ವಿಪಾ-ಕೇನ
ಬಹು-ಕಾಲ
ಬಹು-ಗುಣಂ
ಬಹು-ದುಃಖ-ವನ್ನು
ಬಹು-ದುಃಖ-ಸಮಾ-ಕುಲೌ
ಬಹುಧಾ
ಬಹು-ನೋಕ್ತೇನ
ಬಹು-ಪರಿ-ಯಾಗಿ
ಬಹು-ಬೀಜೇನ
ಬಹು-ಮಾನಂ
ಬಹು-ರೂಪಕಾಃ
ಬಹು-ರೇವ
ಬಹುಳ
ಬಹು-ವಿದ್ಯಾನಾಂ
ಬಹು-ವಿಧಾ
ಬಹು-ವಿಧಾನ್
ಬಹು-ವಿಧೈಃ
ಬಹುಶಃ
ಬಹು-ಶಾಸ್ತ್ರ-ವಿಶಾರದಃ
ಬಹುಶೋ
ಬಹು-ಶೋ-ಽಥ
ಬಹು-ಶೋ-ಽಮುನಾ
ಬಹು-ಹಿಂಸಾ-ಯು-ತೇನ
ಬಹೂನಿ
ಬಹ್ಮ-ರಾಕ್ಷಸರ
ಬಾ
ಬಾಂಧವಃ
ಬಾಂಧ-ವಾಶ್ಚ
ಬಾಕಿಯ-ವರ
ಬಾಗಿನ-ವನ್ನು
ಬಾಗಿಲಿನ
ಬಾಗಿಲಿ-ನಲ್ಲಿ
ಬಾಗಿಲಿನಲ್ಲಿಯೇ
ಬಾಣ-ಗಳಿಂದ
ಬಾಣ-ವನ್ನು
ಬಾಧಿಸು-ವುದಿಲ್ಲ
ಬಾಯಾರಿಕೆ
ಬಾಯಾರಿಕೆ-ಯಿಂದ
ಬಾಯಿ
ಬಾಯಿಂದಲೂ
ಬಾಯಿ-ಮುಕ್ಕಳಿಸಿ
ಬಾರದ
ಬಾರ-ದಂತೆ
ಬಾರದೇ
ಬಾರಿ
ಬಾರಿಗೆ
ಬಾರಿ-ಬಾ-ರಿಗೂ
ಬಾರಿಯ
ಬಾರಿ-ಸು-ವ-ವನು
ಬಾಲಂ
ಬಾಲಃ
ಬಾಲಕ
ಬಾಲಕ-ನಾದ
ಬಾಲಘ್ನ್ಯಃ
ಬಾಲದುರ್ಗಾಂ
ಬಾಲದುರ್ಗೆ-ಯನ್ನು
ಬಾಲಾನಾಂ
ಬಾಲಾರುಣಸಮಪ್ರಖ್ಯೋ
ಬಾಲ್ಯ-ದಲ್ಲಿ
ಬಾಳೆ
ಬಾಳೆ-ಯಕಂಭ-ಗಳಿಂದ
ಬಾವಿಯ
ಬಾಷ್ಕಲ
ಬಾಷ್ಕಲ-ಸಂಜ್ಞಿತಃ
ಬಾಷ್ಪಾ
ಬಾಹೀ-ಗೆಂದು
ಬಾಹು-ಗಳನ್ನುಳ್ಳ
ಬಾಹು-ಗಳು
ಬಾಹುಜಃ
ಬಾಹುಜಾಯ
ಬಾಹ್ಮಣ-ಕುಲ-ದಲ್ಲಿ
ಬಾಹ್ಮಣನ
ಬಾಹ್ಮಣನೇ
ಬಾಹ್ಮಣ-ರಾಗಿ
ಬಾಹ್ಯಜ್ಞಾ-ನ-ವನ್ನೂ
ಬಾಹ್ಯಾಂತರ
ಬಾಹ್ಯಾಂತರ-ಶುದ್ದಿ
ಬಿಂಬ
ಬಿಂಬನ
ಬಿಂಬ-ನಾದ
ಬಿಂಬನು
ಬಿಂಬ-ನೆಂಬುದೂ
ಬಿಂಬ-ಬಿಲ್ವಾದಿ-ವರ್ಧಿತಾಃ
ಬಿಂಬವೂ
ಬಿಗಿದು
ಬಿಟ್ಟರೆ
ಬಿಟ್ಟಳು
ಬಿಟ್ಟಿ-ರುವ-ವರು
ಬಿಟ್ಟು
ಬಿಡ-ಬಾರದು
ಬಿಡ-ಬೇಕು
ಬಿಡ-ಲಿಲ್ಲ
ಬಿಡಲು
ಬಿಡಿ
ಬಿಡಿರಿ
ಬಿಡಿ-ಸಿರಿ
ಬಿಡಿ-ಸು-ವುದು
ಬಿಡು
ಬಿಡು-ಗಡೆ
ಬಿಡು-ಗಡೆ-ಯನ್ನು
ಬಿಡು-ಗಡೆ-ಯಾಗುತ್ತದೆ
ಬಿಡು-ಗಡೆ-ಯಾಗುವುದಲ್ಲದೇ
ಬಿಡು-ಗಡೆ-ಯಾಗುವುದೆಂದು
ಬಿಡು-ಗಡೆ-ಯಾ-ಯಿತು
ಬಿಡು-ಗಡೆಯು
ಬಿಡು-ಗಡೆಯೇ
ಬಿಡು-ವುದ-ರಿಂದ
ಬಿತ್ತಲು
ಬಿತ್ತಿ-ದರೆ
ಬಿದಿರಿನ
ಬಿದಿರು
ಬಿದಿರು-ಗಳ
ಬಿದ್ದ
ಬಿದ್ದನು
ಬಿದ್ದರು
ಬಿದ್ದರೆ
ಬಿದ್ದವು
ಬಿದ್ದಿ
ಬಿದ್ದಿತು
ಬಿದ್ದಿದ್ದ
ಬಿದ್ದಿ-ರುವ
ಬಿನ್ನಹ
ಬಿಭ್ಯತ
ಬಿಭ್ರಂತಿ
ಬಿಭ್ರತಃ
ಬಿಲ-ಗಳಿಗೆ
ಬಿಲ್ಕ
ಬಿಲ್ವಮದುಂಬರಂ
ಬಿಸಾನ್ಯಾದಾಯ
ಬಿಸಿ-ನೀರಿ-ನಲ್ಲಿ
ಬೀಜಂ
ಬೀಜ-ದಂತೆ
ಬೀಜ-ಯುಕ್ತ-ವಾದ
ಬೀಜ-ವನ್ನು
ಬೀಡಾದ
ಬೀಳದೇ
ಬೀಳಲು
ಬೀಳಿಸಲ್ಪಡುತ್ತಾನೆ
ಬೀಳಿಸಿ
ಬೀಳಿ-ಸುತ್ತಾನೆ
ಬೀಳುತ್ತಾನೆ
ಬೀಳುವ
ಬೀಳು-ವನು
ಬುಡ-ದಲ್ಲಿ
ಬುದ್ದಿ
ಬುದ್ದಿಂ
ಬುದ್ದಿ-ಯಿಂದ
ಬುದ್ದಿ-ಯಿಲ್ಲದ
ಬುದ್ದಿಯು
ಬುದ್ದಿ-ಯುಳ್ಳ
ಬುದ್ದಿ-ಯುಳ್ಳ-ವನು
ಬುದ್ದಿ-ಯುಳ್ಳ-ವರು
ಬುದ್ದಿರ್ಜಾ-ಯತೇ
ಬುದ್ದಿ-ವಂತನೂ
ಬುದ್ದಿ-ವಂತ-ರಾದ
ಬುದ್ಧಾಹಂ
ಬುದ್ಧಿ
ಬುದ್ಧಿಂ
ಬುದ್ಧಿಗೆ
ಬುದ್ಧಿ-ಯನ್ನೋಡಿ-ಸುತ್ತಿದ್ದೆ
ಬುದ್ಧಿ-ಯಿಂದ
ಬುದ್ಧಿಯು
ಬುದ್ಧಿ-ರಹಂಕಾರೋ
ಬುದ್ಧಿರ್ಮೆ
ಬುದ್ಧೇ
ಬುದ್ಧ್ಯಾ
ಬುದ್ಧ್ಯಾಸ್ಮಿನ್
ಬುಧಃ
ಬುಬೋಧ
ಬೂದು
ಬೂರಲ
ಬೂರುಗದ
ಬೃಂದಾವ-ನದ
ಬೃಹತಂತ್ರ
ಬೃಹತ್ತಂತ್ರ
ಬೃಹಸ್ಪತಿ
ಬೃಹಸ್ಪತಿಃ
ಬೆಂಕಿ
ಬೆಂಕಿಯ
ಬೆಂಕಿ-ಯಂತೆ
ಬೆಂಕಿ-ಯನ್ನು
ಬೆಂಕಿ-ಯಿಂದ
ಬೆಟ್ಟದ
ಬೆಟ್ಟದಲ್ಲಿದ್ದ
ಬೆಣ್ಣೆ
ಬೆಲೆಬಾಳುವ
ಬೆಲ್ಲ
ಬೆಲ್ಲ-ವನ್ನು
ಬೆಳ-ಕನ್ನು
ಬೆಳಸಿ-ಕೊಂಡು
ಬೆಳಿಗ್ಗೆ
ಬೆಳೆದಿದ್ದವು
ಬೆಳೆ-ಯಿತು
ಬೆಳೆಯುತ್ತಿದ್ದೆ
ಬೆಳ್ಳೊಳ್ಳಿ
ಬೇಕಾಗಿ
ಬೇಕಾಗುವಷ್ಟು
ಬೇಕಾದ
ಬೇಕಾದಷ್ಟು
ಬೇಕಿಲ್ಲ
ಬೇಕು
ಬೇಕೆಂಬ
ಬೇಗನೆ
ಬೇಟಗಾರನ
ಬೇಟೆಗಾರ-ನನ್ನು
ಬೇಟೆಗಾರ-ನಾಗಿ
ಬೇಟೆಗಾರ-ನಾಗಿದ್ದರೂ
ಬೇಟೆಗಾರ-ನಿಗೆ
ಬೇಟೆಗಾರನು
ಬೇಟೆಗಾರ-ರಿಂದ
ಬೇಟೆಯಾ-ಡಲು
ಬೇಟೆಯಾಡಿ
ಬೇಟೆಯಾಡುವ
ಬೇಡ-ನಿಗೆ
ಬೇಡನು
ಬೇಡ-ನೊಬ್ಬನು
ಬೇಡಿ
ಬೇಡಿ-ದನು
ಬೇಡುವ
ಬೇಡುವ-ವ-ರಿಗೆ
ಬೇಯಿಸಲ್ಪಡುತ್ತಾನೆ
ಬೇರಿನ
ಬೇರೆ
ಬೇರೆ-ಮಾಡು-ವುದು
ಬೇರೆ-ಯಾಗಿ
ಬೇರೆ-ಯಾದ
ಬೇರೆಲ್ಲಿಯೋ
ಬೇರೊಂದಿಲ್ಲ
ಬೇರೊಂದು
ಬೇಲದ
ಬೇಳೆ-ಗಳು
ಬೇವಿನ
ಬೇಸಿಗೆ-ಯಲ್ಲಿ
ಬೊದ್ಧವ್ಯಂ
ಬೊಧಿತಿಃ
ಬೋದ್ಧ
ಬೋದ್ಧವ್ಯಂ
ಬೋಧಕ
ಬೋಧಯಾ-ಮಾಸ
ಬೋಧಿತಂ
ಬೋಧಿತೋ-ಽಭ-ವಮ್
ಬೋಧಿತೋಽಹ್ಯ
ಬೋಧಿಸಿ-ದರೂ
ಬೋಧಿ-ಸುತ್ತಿದ್ದೆನೇ
ಬೋಧೇನ
ಬೋಳಿಸಿ-ದನು
ಬ್ರಹೋ-ವಾಚ
ಬ್ರಹ್ಮ
ಬ್ರಹ್ಮಂಸ್ತಸ್ಮಾದ್ವಿಸ್ತರತೋ
ಬ್ರಹ್ಮ-ಕಲ್ಪ
ಬ್ರಹ್ಮ-ಕಲ್ಪದ
ಬ್ರಹ್ಮಘ್ನಂ
ಬ್ರಹ್ಮ-ಚಾರಿ
ಬ್ರಹ್ಮ-ಚಾರಿ-ಗಳಿಗೂ
ಬ್ರಹ್ಮ-ಚಾರೀ
ಬ್ರಹ್ಮಣಾ
ಬ್ರಹ್ಮ-ಣಾ-ನೇನ
ಬ್ರಹ್ಮ-ತರ್ಕವ-ಚನ
ಬ್ರಹ್ಮ-ದಂಡೇನ
ಬ್ರಹ್ಮ-ದರ್ಶನ-ಮಿಚ್ಛುಭಿಃ
ಬ್ರಹ್ಮ-ದೇವ-ರಂತೆ
ಬ್ರಹ್ಮ-ದೇವ-ರಿಂದ
ಬ್ರಹ್ಮ-ದೇವರು
ಬ್ರಹ್ಮ-ದೇವರೇ
ಬ್ರಹ್ಮ-ನಿಷ್ಠಮ್
ಬ್ರಹ್ಮನೇ
ಬ್ರಹ್ಮನ್
ಬ್ರಹ್ಮ-ಪತ್ರೆ
ಬ್ರಹ್ಮ-ಭೂಯಾಯ
ಬ್ರಹ್ಮ-ಯಜ್ಞ
ಬ್ರಹ್ಮ-ಯಜ್ಞೋ
ಬ್ರಹ್ಮ-ರಾಕ್ಷಸ
ಬ್ರಹ್ಮ-ರಾಕ್ಷಸಃ
ಬ್ರಹ್ಮ-ರಾಕ್ಷಸ-ನನ್ನು
ಬ್ರಹ್ಮ-ರಾಕ್ಷಸ-ನಾಗಿ
ಬ್ರಹ್ಮ-ರಾಕ್ಷಸ-ನಾಗುವನು
ಬ್ರಹ್ಮ-ರಾಕ್ಷಸ-ನಾದ
ಬ್ರಹ್ಮ-ರಾಕ್ಷಸ-ನಾದನು
ಬ್ರಹ್ಮ-ರಾಕ್ಷಸ-ನಾದೆ
ಬ್ರಹ್ಮ-ರಾಕ್ಷಸ-ನಿಗೆ
ಬ್ರಹ್ಮ-ರಾಕ್ಷಸನು
ಬ್ರಹ್ಮ-ರಾಕ್ಷಸ-ನೆಂದೇ
ಬ್ರಹ್ಮ-ರಾಕ್ಷಸಮ್
ಬ್ರಹ್ಮ-ರಾಕ್ಷಸ-ರನ್ನು
ಬ್ರಹ್ಮ-ರಾಕ್ಷಸರು
ಬ್ರಹ್ಮ-ರಾಕ್ಷಸಾ
ಬ್ರಹ್ಮ-ರಾಕ್ಷಸಾಃ
ಬ್ರಹ್ಮರ್ಷೆರ್ವಹುಧಾ
ಬ್ರಹ್ಮ-ಲೋಕಂ
ಬ್ರಹ್ಮ-ಲೋಕಕ್ಕೆ
ಬ್ರಹ್ಮ-ವಚ್ಚಿ-ರಮ್
ಬ್ರಹ್ಮ-ವಾಚ
ಬ್ರಹ್ಮ-ವಾಯು-ಮಹಾತ್ಮಾನೌ
ಬ್ರಹ್ಮ-ಸಂಪನ್ಮಃ
ಬ್ರಹ್ಮ-ಸೂತ್ರ
ಬ್ರಹ್ಮ-ಸೂತ್ರ-ಭಾಷ್ಯದ
ಬ್ರಹ್ಮ-ಸೂತ್ರ-ಭಾಷ್ಯ-ದಲ್ಲಿ
ಬ್ರಹ್ಮ-ಸೂತ್ರ-ಭಾಷ್ಯ-ದಲ್ಲಿನ
ಬ್ರಹ್ಮಸ್ವಮಖಿಲಂ
ಬ್ರಹ್ಮಸ್ವಹ-ರಣಂ
ಬ್ರಹ್ಮ-ಹತ್ಯ
ಬ್ರಹ್ಮ-ಹತ್ಯವು
ಬ್ರಹ್ಮ-ಹತ್ಯಾ
ಬ್ರಹ್ಮ-ಹತ್ಯಾಂ
ಬ್ರಹ್ಮ-ಹತ್ಯಾದಿ
ಬ್ರಹ್ಮ-ಹತ್ಯಾ-ದಿಂದ
ಬ್ರಹ್ಮ-ಹತ್ಯಾ-ದಿ-ಕಾನಿ
ಬ್ರಹ್ಮ-ಹತ್ಯಾ-ಯು-ತಾನಿ
ಬ್ರಹ್ಮ-ಹತ್ಯೆ-ಗಿಂತ
ಬ್ರಹ್ಮ-ಹತ್ಯೆ-ಯಿಂದ
ಬ್ರಹ್ಮಾ
ಬ್ರಹ್ಮಾಂಡ-ದಲ್ಲಿ
ಬ್ರಹ್ಮಾದ್ಯಾ
ಬ್ರಹ್ಮಾದ್ಯಾಃ
ಬ್ರಹ್ಮಾಶ್ರಯಸ್ತಸ್ಯ
ಬ್ರಹ್ಮೇಶ್ವರ
ಬ್ರಹ್ಮೇಶ್ವರಂ
ಬ್ರಹ್ಮೇಶ್ವರ-ಲಿಂಗ-ವನ್ನು
ಬ್ರಹ್ಮೇಶ್ವರೇ
ಬ್ರಹ್ಮೋ-ವಾಚ
ಬ್ರಾಹ್ಮಣ
ಬ್ರಾಹ್ಮಣಃ
ಬ್ರಾಹ್ಮಣ-ಕುಲ-ದಲ್ಲಿ
ಬ್ರಾಹ್ಮಣನ
ಬ್ರಾಹ್ಮಣ-ನನ್ನು
ಬ್ರಾಹ್ಮಣ-ನಲ್ಲಿ
ಬ್ರಾಹ್ಮಣ-ನಾಗಿ
ಬ್ರಾಹ್ಮಣ-ನಾಗಿದ್ದನು
ಬ್ರಾಹ್ಮಣ-ನಾಗಿದ್ದರೆ
ಬ್ರಾಹ್ಮಣ-ನಾಗಿದ್ದೆ
ಬ್ರಾಹ್ಮಣ-ನಾದರೂ
ಬ್ರಾಹ್ಮಣ-ನಾ-ದ-ವನು
ಬ್ರಾಹ್ಮಣ-ನಿಗೆ
ಬ್ರಾಹ್ಮಣ-ನಿ-ಗೋಸ್ಕರ
ಬ್ರಾಹ್ಮಣ-ನಿದ್ದನು
ಬ್ರಾಹ್ಮಣನು
ಬ್ರಾಹ್ಮಣನೇ
ಬ್ರಾಹ್ಮಣ-ನೊಬ್ಬನು
ಬ್ರಾಹ್ಮಣರ
ಬ್ರಾಹ್ಮಣ-ರನ್ನು
ಬ್ರಾಹ್ಮಣ-ರಲ್ಲಿ
ಬ್ರಾಹ್ಮಣ-ರಾಗಿ
ಬ್ರಾಹ್ಮಣ-ರಿಂದ
ಬ್ರಾಹ್ಮಣ-ರಿಗೆ
ಬ್ರಾಹ್ಮಣರು
ಬ್ರಾಹ್ಮಣ-ರು-ಮೂರು
ಬ್ರಾಹ್ಮಣ-ವರ್ಣದ
ಬ್ರಾಹ್ಮಣಶ್ರೇಷ್ಠರೇ
ಬ್ರಾಹ್ಮಣ-ಸಾಹಸ್ರ-ಭೋ-ಜನಂ
ಬ್ರಾಹ್ಮಣತ್ರೀ-ಯನ್ನೂ
ಬ್ರಾಹ್ಮಣಾದುತ್ತಮಂ
ಬ್ರಾಹ್ಮ-ಣಾನಾಂ
ಬ್ರಾಹ್ಮಣಾನ್
ಬ್ರಾಹ್ಮಣಾಯ
ಬ್ರಾಹ್ಮಣಾ-ವಾಸೋ
ಬ್ರಾಹ್ಮಣೀಂ
ಬ್ರಾಹ್ಮಣೇಭ್ಯೋ
ಬ್ರಾಹ್ಮಣೈಃ
ಬ್ರಾಹ್ಮಣೋ
ಬ್ರಾಹ್ಮಣೋತ್ತಮಃ
ಬ್ರಾಹ್ಮಣೋಹಂ
ಬ್ರಾಹ್ಮಣೋ-ಽಹಂ
ಬ್ರುವಾಣಂ
ಬ್ರುವಾಣಾಂ
ಬ್ರುವೇ
ಬ್ರೂತ
ಬ್ರೂತೇ
ಬ್ರೂತೇತಿ
ಭ
ಭಂಗಾರದ
ಭಕ್ತನ
ಭಕ್ತ-ನಾಗಿದ್ದೆ
ಭಕ್ತ-ನಾದ
ಭಕ್ತ-ನಿಗೆ
ಭಕ್ತನು
ಭಕ್ತರ
ಭಕ್ತ-ರಲ್ಲಿ
ಭಕ್ತ-ರಲ್ಲಿಯೂ
ಭಕ್ತ-ರಾದ
ಭಕ್ತ-ರಿಗೆ
ಭಕ್ತರು
ಭಕ್ತ-ವತ್ಸಲಃ
ಭಕ್ತ-ವತ್ಸಲ-ನಾದ
ಭಕ್ತ-ವತ್ಸಲಮಚ್ಯು-ತಮ್
ಭಕ್ತಾ-ದಿ-ಗಳಿಂದ
ಭಕ್ತಿ
ಭಕ್ತಿ-ಮಹೈತುಕೀಮ್
ಭಕ್ತಿ-ಮಾಡಿ
ಭಕ್ತಿ-ಮಾಡು-ವ-ವರು
ಭಕ್ತಿಯ
ಭಕ್ತಿ-ಯಕ್ತ-ನಾಗಿ
ಭಕ್ತಿ-ಯನ್ನು
ಭಕ್ತಿ-ಯಿಂದ
ಭಕ್ತಿ-ಯಿಲ್ಲದೆ
ಭಕ್ತಿಯು
ಭಕ್ತಿ-ಯುಕ್ತರಾ-ದ-ವರು
ಭಕ್ತಿ-ಯುಳ್ಳ-ವರು
ಭಕ್ತಿ-ಲೇಶ-ವಿ-ವರ್ಜಿತಾಃ
ಭಕ್ತಿ-ವಿ-ವರ್ಧನಾನ್
ಭಕ್ತ್ಯಾ
ಭಕ್ಷ
ಭಕ್ಷಂ
ಭಕ್ಷ-ಕಾಮಿನೌ
ಭಕ್ಷಣ
ಭಕ್ಷಣಂ
ಭಕ್ಷಣೇ
ಭಕ್ಷತಿ
ಭಕ್ಷ-ಯಿತುಂ
ಭಕ್ಷ-ಯೇದ್ಯದಿ
ಭಕ್ಷಿತಂ
ಭಕ್ಷಿ-ತಮ್
ಭಕ್ಷಿ-ತಾನಿ
ಭಕ್ಷಿಸುತ್ತಿದ್ದನು
ಭಕ್ಷಿ-ಸುತ್ತಿದ್ದೆನು
ಭಕ್ಷಿ-ಸುತ್ತಿದ್ದೆ-ಯಾದ-ಕಾರಣ
ಭಕ್ಷ್ಯಾಣಿ
ಭಗವಂತನ
ಭಗವಂತ-ನಿಗೆ
ಭಗವಂತನು
ಭಗ-ವತಾ
ಭಗ-ವತ್ಕಾರ್ಯ-ಸಾಧಕಃ
ಭಗ-ವತ್ಪಾರ್ಶ್ವ-ವರ್ತಿಭಿಃ
ಭಗ-ವದ-ಪರೋಕ್ಷ
ಭಗ-ವದ್ಭಕ್ತನು
ಭಗ-ವದ್ಭಕ್ತನೂ
ಭಗ-ವದ್ಭಕ್ತ-ರಾದ
ಭಗ-ವನ್ನಿಷ್ಠ
ಭಗವಾಂಸ್ತುಷ್ಟೋ
ಭಗವಾ-ನಾದೌ
ಭಗ-ವಾನ್
ಭಗೋದಯೇ
ಭಗ್ನ-ವಾಗುತ್ತವೆ
ಭಜಂತಿ
ಭಜಂತ್ಯ
ಭಜತಾಂ
ಭಜಿ-ಸುವ-ವ-ರಿಗೆ
ಭಟೀಯ
ಭದ್ರ
ಭದ್ರಂ
ಭದ್ರಃ
ಭದ್ರನ
ಭದ್ರ-ನನ್ನು
ಭದ್ರ-ನಾದರೋ
ಭದ್ರ-ನಿಗೆ
ಭದ್ರನು
ಭದ್ರನೇ
ಭದ್ರ-ಮಭಿರೋಪ್ಯ
ಭದ್ರ-ಮಾನಯತ
ಭದ್ರರು
ಭದ್ರ-ವಚಃ
ಭದ್ರ-ವಾದ
ಭದ್ರ-ಸಂಜ್ಞ
ಭದ್ರ-ಸಂಜ್ಞಿತಃ
ಭದ್ರ-ಸಂಜ್ಞೆ
ಭದ್ರಾಣಿ
ಭದ್ರೇ
ಭದ್ರೇಣ
ಭದ್ರೋ
ಭದ್ರೋಪಿ
ಭನ್ಯಾಃ
ಭಯಂಕರ
ಭಯಂಕರ-ರಾದ
ಭಯಂಕರ-ರೂಪ-ವುಳ್ಳ
ಭಯಂಕರ-ವಾದ
ಭಯಂಕರಾಃ
ಭಯ-ಗಳು
ಭಯ-ಗೊಂಡ
ಭಯ-ಗೊಂಡು
ಭಯಗ್ರಸ್ತ-ನಾದ
ಭಯಗ್ರಸ್ತ-ವಾಗಿ
ಭಯದಾ
ಭಯ-ದಿಂದ
ಭಯ-ದುಃಖ-ರ-ಹಿತ-ನಾಗಿ
ಭಯ-ಪಟ್ಟ
ಭಯ-ಪಟ್ಟು
ಭಯಪಡಬೇಡ
ಭಯಪಡಲೂ
ಭಯರ-ಹಿತ-ನಾಗಿ
ಭಯ-ರ-ಹಿತ-ನಾಗಿಯೂ
ಭಯವಿಹ್ವಲನ್
ಭಯವು
ಭಯವೂ
ಭಯವೇ
ಭಯಾ-ಕುಲಃ
ಭಯಾ-ಕುಲಾ
ಭಯಾತ್
ಭಯಾತ್ಕರ್ತವ್ಯ
ಭಯಾದ-ಪಸ-ಸಾರಾಸೌ
ಭಯಾದಿ-ಗಳಿಂದ
ಭಯಾನ್ಮೃತಿಸಂತಾನ-ನಿಷ್ಕೃತಿಃ
ಭಯಾನ್ವಿತಾಃ
ಭಯೋಪಿ
ಭರಿತ-ವಾಗಿದ್ದುವು
ಭರ್ತರಿ
ಭರ್ತಾ
ಭರ್ತಾ-ರಮಾ-ಭಾಷ-ಯಿತುಂ
ಭರ್ತುಃ
ಭರ್ತೃಹನನಜಂ
ಭವ
ಭವಂತಶ್ಚ
ಭವಂತಸ್ತಾಮಸಾಃ
ಭವಂತಿ
ಭವಂತೋ
ಭವತಶ್ಚ
ಭವತಾ
ಭವತಾಂ
ಭವತಾ-ಮಿತಿ
ಭವತಿ
ಭವತಿ-ಗುರು-ಗಳು
ಭವತೇ-ಜೋಹತಾ
ಭವತ್ಯಲಮ್
ಭವತ್ಯಾಶು
ಭವತ್ಯೇವ
ಭವತ್ಯೇಷಃ
ಭವತ್ಯೇಷಾ
ಭವತ್ವಿತಿ
ಭವದಾ-ಗಮನಂ
ಭವದ್ಭಂಧಾನ್ಮೋಚಕಂ
ಭವದ್ಭಿ
ಭವದ್ಭಿರ್ನೇತುಮರ್ಹತಿ
ಭವದ್ಭಿರ್ನೇತುಮರ್ಹಾಸ್ತೇ
ಭವಾಜಗರ-ದಷ್ಟಾನಾಂದೀ-ನಾನಾ-ಮ-ಕೃತಾತ್ಮ-ನಾಮ್
ಭವಾದೃಶಾಃ
ಭವಾದೃಶಾನಾಂ
ಭವಾದೃಶೋ
ಭವಾನ್
ಭವಾನ್ಪೂರ್ವಂ
ಭವಾಪಹೋ
ಭವಾಮಹೇ
ಭವಾಮ್ಯ-ಹಮ್
ಭವಿತಾ
ಭವಿಷ್ಯತಸ್ತತೋ-ಮೋಕ್ಷಂ
ಭವಿಷ್ಯತಿ
ಭವೇಜ್ಜನೇ
ಭವೇತ್
ಭವೇತ್ಕ್ವಾಪೀತ್ಯುಕ್ತ್ವಾ
ಭವೇತ್ತಸ್ಯ
ಭವೇತ್ಪಾಪೀ
ಭವೇತ್ಪ್ರಚಾರೋ
ಭವೇದಿತಿ
ಭವೇದುತ್ತಮ-ಕೋಪತಃ
ಭವೇದ್
ಭವೇದ್ದೃ-ವಿರುದ್ಧಂ
ಭವೇದ್ಧ್ರು-ವಮ್
ಭವೇದ್ಭುವಿ
ಭವೇದ್ಯದಿ
ಭವೇದ್ವಾಯಸೋ
ಭವೇದ್ದ್ರು-ವಮ್
ಭವೇನ್ನರಃ
ಭವೇನ್ನೂನಂ
ಭವೇನ್ಮದೀಯೇ
ಭವೇನ್ಮೋಕ್ಷೋ
ಭಸ್ಮ-ವನ್ನು
ಭಸ್ಮ-ವಾಗುತ್ತವೆ
ಭಸ್ಮ-ವಾಗು-ವಂತೆ
ಭಸ್ಮ-ವಾ-ಯಿತು
ಭಸ್ಮ-ಸಾತ್ಕುರುತೇ
ಭಸ್ಮ-ಸಾದ್ಯ
ಭಸ್ಮ-ಸಾದ್ಯಾಂತಿ
ಭಸ್ಮಿಭೂತ-ವಾಗುತ್ತವೆ
ಭಾಗ
ಭಾಗ-ಗಳ
ಭಾಗ-ಗಳಿವೆ
ಭಾಗ-ದಲ್ಲಿ
ಭಾಗ-ವತ
ಭಾಗ-ವತಂ
ಭಾಗ-ವತಸ್ತನೂಭವಃ
ಭಾಗ-ವತಾ
ಭಾಗ-ವತೇ
ಭಾಗ-ವತೋತ್ತಮಮ್
ಭಾಗ-ವತೋತ್ತಮರು
ಭಾಗ-ವನ್ನು
ಭಾಗ-ವನ್ನೂ
ಭಾಗವು
ಭಾಗಿನೇಯೋ
ಭಾಗೀ-ರಥಿ
ಭಾಗೀ-ರಥೀ
ಭಾಗ್ಯತಃ
ಭಾಗ್ಯದಃ
ಭಾಗ್ಯ-ದಿಂದ
ಭಾಗ್ಯಮಥೋ
ಭಾಗ್ಯಮನಲ್ಪನಾಯು
ಭಾಗ್ಯ-ಮಹಂ
ಭಾಗ್ಯ-ವಂತನೂ
ಭಾಗ್ಯ-ವನ್ನು
ಭಾಗ್ಯವು
ಭಾಗ್ಯಾ
ಭಾನು-ರಥ-ಸಪ್ತಮ್ಯಾಂ
ಭಾನು-ವಾರ
ಭಾನು-ವಾರ-ಗಳಲ್ಲಿಯೂ
ಭಾನು-ವಾರದ
ಭಾನು-ವಾರ-ದಲ್ಲಿ
ಭಾನು-ವಾರ-ದಿಂದ
ಭಾನು-ವಾರೇ
ಭಾನು-ವಾರೇಣ
ಭಾನು-ವಾರೊ
ಭಾನು-ವಾಸರೇ
ಭಾನೌ
ಭಾಮಿತಂ
ಭಾರಂ
ಭಾರತ
ಭಾರ-ತ-ದಲ್ಲಿ
ಭಾರತೀ
ಭಾರ-ತೀ-ದೇವಿಯರೂ
ಭಾರತೇ
ಭಾರದ್ವಾಜ
ಭಾರದ್ವಾ-ಜ-ಕುಲೋತ್ಪನ್ನಃ
ಭಾರ್ಗವೋ
ಭಾರ್ಯಯಾ
ಭಾರ್ಯಾ
ಭಾರ್ಯಾಂ
ಭಾರ್ಯಾತ್ವೇ
ಭಾರ್ಯಾ-ಮಾ-ದಾಯ
ಭಾವ-ದಿಂದ
ಭಾವನೆ
ಭಾವನೆ-ಗಳಿಲ್ಲದೇ
ಭಾವನೆ-ಗಳಿವೆಯೋ
ಭಾವನೆ-ಯನ್ನು
ಭಾವನೆಯೇ
ಭಾವಮುಪೇಯುಷಃ
ಭಾವ-ಸಂಗ್ರಹ
ಭಾವಾಃ
ಭಾವಾದನ್ಯಾ-ನದ್ಭಿನ್ನ-ರೂಪಾತ್ಸಂಗಾತ್ಸತ್ವಂ
ಭಾವಿ
ಭಾವಿ-ಗಳೂ
ಭಾವಿ-ತಾತ್ಮ-ನಾಮ್
ಭಾವಿಸಿ
ಭಾವಿ-ಸುತ್ತೇನೆ
ಭಾವೋ
ಭಾಷ್ಟ್ರು
ಭಾಷ್ಯ
ಭಾಷ್ಯದ
ಭಾಷ್ಯ-ದಲ್ಲಿ
ಭಾಷ್ಯ-ದಲ್ಲಿಯೂ
ಭಾಸೌ
ಭಾಸ್ಕರಃ
ಭಾಸ್ವತಿ
ಭಿಃಸಟಂ
ಭಿಕ್ಷಕ್ಕಾಗಿ
ಭಿಕ್ಷಾ
ಭಿಕ್ಷಾ-ಕಾಂಕ್ಷೀ
ಭಿಕ್ಷಾಯೈ
ಭಿಕ್ಷಾ-ಶಿನಃ
ಭಿಕ್ಷುಕಾಃ
ಭಿಕ್ಷುರ್ಮ-ಹೋದಧಿಃ
ಭಿಕ್ಷೆ
ಭಿಕ್ಷೆ-ಗೋಸ್ಕರ
ಭಿಕ್ಷೆ-ಯನ್ನು
ಭಿತ್ವಾ
ಭಿನ್ನ
ಭಿನ್ನ-ನೆಂದೂ
ಭಿನ್ನ-ವಾದ
ಭೀತಾ
ಭೀತಾಶ್ಣೋರಾ
ಭೀತಿಭೀತಃ
ಭೀಮ
ಭೀಮ-ರಥೀ
ಭೀಮ-ರೂಪ
ಭುಂಕ್ತೇ
ಭುಂಕ್ತ್ಯೇವ
ಭುಂಕ್ಷ್ವ
ಭುಂಜಂತಂ
ಭುಂಜಿ-ಸಿದೆ
ಭುಂಜಿಸುತ್ತಿದ್ದಿಲ್ಲ
ಭುಂಜಿ-ಸುತ್ತಿದ್ದೆ
ಭುಂಜಿಸುತ್ತೀಯೋ
ಭುಕ್ತ
ಭುಕ್ತಂ
ಭುಕ್ತ-ಮತ್ಯಂತು
ಭುಕ್ತ-ಮನ್ನಂ
ಭುಕ್ತ-ಮನ್ವ-ಹಮ್
ಭುಕ್ತಿ
ಭುಕ್ತೋಪ-ವಿಷ್ಟೇ
ಭುಕ್ತ್ವಾ
ಭುಕ್ವಾ
ಭುಜ
ಭುಜ-ಗಳಲ್ಲಿಯೇ
ಭುಜ-ದಲ್ಲಿ
ಭುಜ-ಮೂರ್ಧಾ
ಭುಜ-ಶಿರಾ
ಭುಜಾಂಸಕೇ
ಭುಜಾ-ದಾಗ-ತಮೂರ್ಧಕಾಃ
ಭುಜ್ಯ
ಭುಜ್ಯತೇ
ಭುವನತ್ರಯೇ
ಭುವಿ
ಭುವಿ-ಲಂಬಿತಶಾಖಾಗ್ರಗಲತ್ಪುಷ್ಪಾಧಿ-ವಾಸಿನೇ
ಭೂಖನತೇ
ಭೂಜಯಾ-ಮಾಸ
ಭೂತಂ
ಭೂತ-ಪಂಚ-ಕಮ್
ಭೂತ-ಪೂರ್ವಾಣಿ
ಭೂತ-ಯಜ್ಞ
ಭೂತ-ಯಜ್ಞೋ
ಭೂತಾಕಾಶ-ದಲ್ಲಿ
ಭೂತಾನಿ
ಭೂತೋ
ಭೂತ್ವಾ
ಭೂದಾನ
ಭೂದಾನದ
ಭೂದಾನಾಪ್ತಂ
ಭೂಪ
ಭೂಪತಿಃ
ಭೂಪ-ತಿತಂ
ಭೂಪ-ಮುದ-ಜೀವ-ಯತ್
ಭೂಮಾ-ವನಾ-ವೃಷ್ಟಿರ್ಬಭೂವ
ಭೂಮಿ
ಭೂಮಿಂ
ಭೂಮಿ-ಚಾರೀ
ಭೂಮಿ-ದೇವಾಃ
ಭೂಮಿ-ಧಾನ್ಯ-ಧ-ನಾದಿ-ಕಮ್
ಭೂಮಿಪಃ
ಭೂಮಿಪಾ
ಭೂಮಿ-ಯನ್ನು
ಭೂಮಿ-ಯ-ಮೇಲೆ
ಭೂಮಿ-ಯಲ್ಲಿ
ಭೂಮೇರ್ದಾನಂ
ಭೂಮೌ
ಭೂಮ್ಯಾಂ
ಭೂಯಾತ್
ಭೂಯಾದ-ಥವಾ
ಭೂಯಾದ-ನಾಯಾ-ಸೇನ
ಭೂಯಾನ್
ಭೂಯಾನ್ಮೃತಿಸಂತಾನ-ನಿಷ್ಕೃತಿಃ
ಭೂಯೋ
ಭೂರಿ
ಭೂರಿ-ದಮ್
ಭೂರಿ-ಭಾಗೋಽನಿಮಿತ್ತ
ಭೂರಿ-ವಾರಿ
ಭೂರಿ-ವಿಭೂತಿ-ದಾಯ
ಭೂರಿಶಃ
ಭೂರಿಶೋ
ಭೂಲೋಕಕ್ಕೆ
ಭೂಲೋಕ-ದಲ್ಲಿ
ಭೂಲೋಕ-ದಲ್ಲಿ-ಯಾಗಲೀ
ಭೂಸಂಪರ್ಕ-ದಿಂದ
ಭೂಸು-ರರು
ಭೇದ-ಮಾಡು-ವುದು
ಭೇದಿಸಿ-ಕೊಂಡು
ಭೇನ
ಭೇಷಜೀಃ
ಭೈರವ
ಭೈರವಂ
ಭೈರವ-ವೆಂಬ
ಭೈರ-ವಾದಿ
ಭೈರ-ವಾದ್ಯೈಃ
ಭೈಷಿ
ಭೈಷೀ-ರಿತಿ
ಭೈಷೀಸ್ತಾ-ತೇತಿ
ಭೋ
ಭೋಕ್ತವ್ಯಂ
ಭೋಕ್ತಾ
ಭೋಕ್ತುಂ
ಭೋಕ್ತುಮಿಚ್ಛೋರ್ಯಥಾ
ಭೋಗ-ಗಳ
ಭೋಗ-ಗಳನ್ನೂ
ಭೋಗ-ವತೀ
ಭೋಗಾಃ
ಭೋಗಾನ್
ಭೋಗಾಶ್ಚ
ಭೋಗಿಸ-ಬೇಕು
ಭೋಗಿಸ-ಬೇಕೆಂಬ
ಭೋಗಿ-ಸುವ
ಭೋಗಿ-ಸುವ-ವನೂ
ಭೋಗೇ
ಭೋಗೇಚ್ಛೆ-ಯಿಂದಾಗಲೀ
ಭೋಜ-ದೇಶದ
ಭೋಜನ
ಭೋಜನಂ
ಭೋಜನದ
ಭೋಜನ-ದಿಂದ
ಭೋಜನ-ಮಾಡಿ-ಸ-ಬೇಕು
ಭೋಜನ-ಮಾಡಿಸಿ
ಭೋಜನ-ಮಾಡಿ-ಸಿ-ದನು
ಭೋಜನ-ಮಾಡಿ-ಸಿ-ದರೆ
ಭೋಜನ-ಮಾಡುತ್ತಾ-ರೆಯೋ
ಭೋಜನ-ಮಾಡುತ್ತಿದ್ದೆನು
ಭೋಜನಮ್
ಭೋಜನ-ವನ್ನು
ಭೋಜನ-ವರ್ಜಿ-ತಮ್
ಭೋಜನಾ-ನಂತರಂ
ಭೋಜನೇ
ಭೋಜ-ಯಂತಿ
ಭೋಜಯೇತ್ತತಃ
ಭೋಜಯೇದ್ಯ
ಭೋಜಯೇದ್ಯದಿ
ಭೋಜಯೇದ್ಯಸ್ತು
ಭೋಜಯೇದ್ವೇದ-ಪಾರ-ಗಮ್
ಭೋಜ-ವಂಶ-ದಲ್ಲಿ
ಭೋಜ-ವಂಶೇ
ಭೋಜೇ-ಶಮಗ-ಮಜ್ಜೇತುಂ
ಭೋಜ್ಯಂ
ಭೋಜ್ಯಾರ್ಹ
ಭೌಮಾಷ್ಟಮೀ-ದಿನೇ
ಭೌಮೇನ
ಭ್ರಮಣ
ಭ್ರಮತ್ಯಯಂ
ಭ್ರಮಮಾಣಸ್ತು
ಭ್ರಮಮಾಣಾ
ಭ್ರಮಮಾಣೇನ
ಭ್ರಮಿಸಿ
ಭ್ರಷ್ಟಂ
ಭ್ರಷ್ಟತೆ-ಯಿಂದ
ಭ್ರಾಂತಂ
ಭ್ರಾಂತಿ-ಕಲ್ಪಿತ
ಭ್ರಾಂತಿ-ಕಲ್ಪಿತಾಃ
ಭ್ರಾಂತ್ಯಾ
ಭ್ರಾತರಂ
ಭ್ರಾತರೌ
ಭ್ರಾಮ-ಯನ್ಸರ್ವ-ಭೂ-ತಾನಿ
ಭ್ರಾಮ್ಯಮಾ-ಣಾನಾಂ
ಭ್ರೂಣಹನಂ
ಮಂಗಳ
ಮಂಗಳ-ಕರ-ವಾದ
ಮಂಗಳ-ದೇ-ವತಾ
ಮಂಗಳ-ವನ್ನು
ಮಂಗಳ-ವಾರ
ಮಂಗಳ-ವಾರ-ದಲ್ಲಿ
ಮಂಗಳ-ವಾರ-ದಿಂದ
ಮಂಗಳಾರತಿ
ಮಂಗಳಾರತಿ-ಯನ್ನು
ಮಂಚದ-ಮೇಲೆ
ಮಂಚ-ವನ್ನು
ಮಂಜರಿ
ಮಂಜಸಾ
ಮಂಜುಳ-ವಾದ
ಮಂಡಪ-ವನ್ನು
ಮಂಡಲ-ದಲ್ಲಿ
ಮಂಡಲ-ವೆಂಬ
ಮಂಡಲ-ಸಂಜ್ಞಕೇ
ಮಂತ್ರ
ಮಂತ್ರ-ಗಳನ್ನು
ಮಂತ್ರ-ಗಳಲ್ಲಿ
ಮಂತ್ರ-ಗಳಿಂದ
ಮಂತ್ರತಃ
ಮಂತ್ರದ
ಮಂತ್ರ-ದಿಂದ
ಮಂತ್ರ-ಪಠನ
ಮಂತ್ರ-ಪಠನ-ದಿಂದ
ಮಂತ್ರ-ಮುಪಾಂಶು-ಕಮ್
ಮಂತ್ರ-ರಾಜ-ನಿಂದ
ಮಂತ್ರ-ರಾಜೇನ
ಮಂತ್ರ-ವನ್ನು
ಮಂತ್ರಾಣಾಂ
ಮಂತ್ರಾ-ಲಯ
ಮಂತ್ರಿ-ಗಳು
ಮಂತ್ರೇಣ
ಮಂತ್ರೇಣಾಸ್ಯ-ಫಲಂ
ಮಂತ್ರೈಃ
ಮಂದಾನಾಂ
ಮಂದಾಸ್ತೇಷಾಮಾದೌ
ಮಂದಿ
ಮಂದಿಯೂ
ಮಂದಿ-ರ-ದಲ್ಲಿ
ಮಂದಿ-ರ-ವನ್ನು
ಮಂದಿರೇ
ಮಕನೋ
ಮಕರ
ಮಕರಗೇ
ಮಕರ-ದಲ್ಲಿ-ರು-ವಾಗ
ಮಕರ-ಯುಕ್ತ-ವಾದ
ಮಕರ-ರಾಶಿ-ಯಲ್ಲಿ
ಮಕರ-ರಾಶಿ-ಯಲ್ಲಿ-ರು-ವಾಗ
ಮಕ-ರಸಂಕ್ರಮ-ಣವು
ಮಕರಸ್ಥ-ಭಾನೌ
ಮಕರಸ್ಥಿತೇ
ಮಕರಸ್ಥೇ
ಮಕರೇ
ಮಕ್ಕಳ
ಮಕ್ಕಳನ್ನು
ಮಕ್ಕಳನ್ನೂ
ಮಕ್ಕ-ಳಾಗಿದ್ದೆವು
ಮಕ್ಕ-ಳಾಗಿ-ರುತ್ತಾರೆಯೋ
ಮಕ್ಕಳಾದ
ಮಕ್ಕಳಿ-ಗಾಗಿ
ಮಕ್ಕಳಿಗೆ
ಮಕ್ಕಳಿ-ರುವ
ಮಕ್ಕಳಿಲ್ಲದ
ಮಕ್ಕಳಿಲ್ಲದ-ವರು
ಮಕ್ಕಳು
ಮಕ್ಕಳೇನೋ
ಮಕ್ಕಳೊಂದಿಗೆ
ಮಕ್ತೌ
ಮಖೈರ್ಧರ್ಮಾ
ಮಗ
ಮಗಧ
ಮಗಧೇ
ಮಗನ
ಮಗ-ನಂತೆ
ಮಗ-ನನ್ನು
ಮಗನ-ಮೇಲಿನ
ಮಗ-ನಾಗಿ
ಮಗ-ನಾಗಿದ್ದೆ
ಮಗ-ನಾಗಿ-ರುವ
ಮಗನಾದ
ಮಗ-ನಿಗೆ
ಮಗ-ನಿದ್ದನು
ಮಗನು
ಮಗನೂ
ಮಗಳ
ಮಗಳನ್ನು
ಮಗ-ಳಾಗಿ
ಮಗ-ಳಾಗಿದ್ದಳು
ಮಗ-ಳಾಗಿ-ರುವ-ವಳು
ಮಗಳಾದ
ಮಗಳಾದುದ-ರಿಂದ
ಮಗಳಿಗೆ
ಮಗಳು
ಮಗಳೂ
ಮಗು-ವನ್ನು
ಮಗು-ವಿಗೆ
ಮಗುವು
ಮಗುವೇ
ಮಗ್ನ-ರಾಗಿ
ಮಗ್ನರಾ-ಗಿ-ರುವ
ಮಗ್ನ-ರಾದ
ಮಘ
ಮಘಾ
ಮಘಾ-ನಕ್ಷತ್ರ-ದಿಂದ
ಮಘಾ-ಯುಕ್ತ
ಮಘಾ-ಯುಕ್ತಾ
ಮಘಾ-ಸಹಾಯೇ
ಮಚೋದ-ಯತ್
ಮಜ್ಜ
ಮಜ್ಜ-ನಾತ್
ಮಜ್ಞಧೀಃ
ಮಠದ
ಮಡಿ-ವಸ್ತ್ರ-ಧಾ-ರಣ
ಮಣ್ಣನ್ನು
ಮಣ್ಣುಕಲ್ಲುಲೋಹ-ಗಳಿಂದ
ಮತ
ಮತಂ
ಮತಂಗ
ಮತಂಗ-ಋಷಿ-ಗಳು
ಮತಂಗನೇ
ಮತಂಗಾನಾಂ
ಮತಂಗೇನ
ಮತಃ
ಮತ-ಗಳನ್ನೇ
ಮತ-ನಿಗೆ
ಮತಮ್
ಮತವು
ಮತಾಃ
ಮತಿ
ಮತಿಃ
ಮತಿ-ರಾಸ್ತಿಕಾ
ಮತಿರ್ಮಾಧವೇ
ಮತಿಶ್ಚೈವ
ಮತುಲಂ
ಮತ್ಕೃಪಾ-ಲವಃ
ಮತ್ತ
ಮತ್ತಃ
ಮತ್ತ-ಗಜಾದ್ವಯಮ್
ಮತ್ತತ್ವಾತ್
ಮತ್ತತ್ವಾನ್ನೋತ್ಥಿತೋ
ಮತ್ತಭ್ರಮರಝಂಕಾರ-ಜನಿತಾಟೋ-ಪನಿರ್ಭರೇ
ಮತ್ತ-ಮಾತಂಗಾತ್
ಮತ್ತು
ಮತ್ತೆ
ಮತ್ತೇನ
ಮತ್ತೈರ್ಗಜೈರ್ಮೃತಃ
ಮತ್ತೊಂದಕ್ಕೆ
ಮತ್ತೊಂದು
ಮತ್ತೊಬ್ಬ-ನನ್ನು
ಮತ್ತೋ
ಮತ್ತೌ
ಮತ್ಪದಂ
ಮತ್ಪರಃ
ಮತ್ಪರೀ-ವಾರ-ತೋ-ಽಖಿಲಾಃ
ಮತ್ಪ್ರಾಪ್ತೌ
ಮತ್ಯಾ-ರಿಗೂ
ಮತ್ವಾ-ಽಯಂ
ಮತ್ಸ್ಯ
ಮತ್ಸ್ಯ-ದೇಶ-ಪತೇಶ್ಚಾ
ಮತ್ಸ್ಯಾದಿಷು
ಮಥಾಬ್ರವೀತ್
ಮಥುರಾ
ಮಥೋದ್ಯತಃ
ಮದರಾಸಿನ
ಮದರಾಸು
ಮದರ್ಪ-ಣಮ್
ಮದಾಂಧ-ನಾದ
ಮದಾಂಧ-ರಾಗಿ
ಮದಾದ್ವಿ
ಮದಾನ್ವಿತಾಃ
ಮದಾ-ವಾ-ಸೇನ
ಮದಾಶ್ರಯಾಃ
ಮದಿರ-ವನ್ನು
ಮದಿರಾಪಾನ-ಮತ್ತ-ನಾದ
ಮದಿರಾ-ಮತ್ತೋ
ಮದಿ-ಸಿದ
ಮದೀ-ಯಸ್ಯ
ಮದೀಯಾಃ
ಮದು-ವೆಯಾದ
ಮದೋದ್ದತಃ
ಮದ್ಗೃ-ಹಮ್
ಮದ್ಗ್ರಾಮೀಣೋ
ಮದ್ಭಕ್ತೌ
ಮದ್ಯಪಾನ
ಮದ್ಯ-ವೆಂದು
ಮದ್ರಾಸಿನ
ಮದ್ರಾಸ್
ಮದ್ವಶೇ
ಮದ್ವೃತ್ತಾತಂ
ಮದ್-ಗೃಹಂ
ಮದ್-ಗೃಹೇ
ಮಧಿಕಂ
ಮಧ್ಯ
ಮಧ್ಯ-ದಲ್ಲಿ
ಮಧ್ಯ-ದಲ್ಲಿಯೂ
ಮಧ್ಯ-ಲೋಕ
ಮಧ್ಯ-ಲೋಕ-ವೆಂಬುದು
ಮಧ್ಯ-ಲೋಕಾಸ್ತು
ಮಧ್ಯಾಹ್ನ
ಮಧ್ಯಾಹ್ನ-ವಾ-ಯಿತು
ಮಧ್ಯಾಹ್ನ-ವೇ-ಲಾಯಾಂ
ಮಧ್ಯೇ
ಮಧ್ಯೋರ್ಧ್ವ-ಸಂಸ್ಥಾ-ನಾಸ್ತೆಧೋ-ಲೋಕಾಸ್ತು
ಮಧ್ವ
ಮಧ್ವಾಚಾರ್ಯರ
ಮನ
ಮನಃ
ಮನಃಸ್ವಾಸ್ಥ್ಯಂ
ಮನನ
ಮನನಂ
ಮನನ-ಶೀಲ-ನಾದ
ಮನ-ಮೋಹ-ಕ-ವಾದ
ಮನವಶ್ಚ
ಮನಸಿ-ನಲ್ಲಿ
ಮನಸೋ
ಮನಸೋ-ತಿಹರ್ಷ-ಣಮ್
ಮನಸ್ತತ್ವಂ
ಮನಸ್ತತ್ವಕ್ಕೂ
ಮನಸ್ತತ್ವಸ್ವ-ರೂಪೌ
ಮನಸ್ಸನ್ನಿಡ-ಬೇಕು
ಮನಸ್ಸನ್ನು
ಮನಸ್ಸಿಗೂ
ಮನಸ್ಸಿಟ್ಟು
ಮನಸ್ಸಿನ
ಮನಸ್ಸಿ-ನಲ್ಲಿ
ಮನಸ್ಸಿನಲ್ಲಿಯೂ
ಮನಸ್ಸಿ-ನಿಂದ
ಮನಸ್ಸಿ-ನಿಂದಲೂ
ಮನಸ್ಸು
ಮನಸ್ಸು-ಗಳಿಂದ
ಮನಸ್ಸುಳ್ಳ
ಮನಸ್ಸುಳ್ಳ-ವರೂ
ಮನೀಷಿಣಃ
ಮನೀಷಿಭಿಃ
ಮನು
ಮನು-ಗಂಧರ್ವಾ
ಮನು-ಗಳು
ಮನು-ಜಾತ್ಮ-ಹಿತಾ-ನಭಿಜ್ಞಾನ್
ಮನು-ಜಾಶ್ಚ
ಮನುಜೇ
ಮನು-ಜೇಶ್ವರ
ಮನುತ್ತಮಮ್
ಮನುವೇ
ಮನುಶ್ಚೈವ
ಮನುಷ್ಯ
ಮನುಷ್ಯನ
ಮನುಷ್ಯ-ನನ್ನು
ಮನುಷ್ಯನು
ಮನುಷ್ಯ-ಯಜ್ಞ
ಮನುಷ್ಯರ
ಮನುಷ್ಯ-ರಂತೆ
ಮನುಷ್ಯ-ರಿಗೆ
ಮನುಷ್ಯರು
ಮನುಷ್ಯಾಣಾಂ
ಮನುಷ್ಯೋತ್ತಮ-ರೆಂದೂ
ಮನೆ
ಮನೆ-ಗಳ
ಮನೆ-ಗಳಿಗೆ
ಮನೆಗೆ
ಮನೆಯ
ಮನೆ-ಯನ್ನು
ಮನೆ-ಯ-ಮೇಲೆ
ಮನೆ-ಯಲ್ಲಿ
ಮನೆ-ಯಲ್ಲಿದ್ದಾಗ
ಮನೆ-ಯಲ್ಲಿಯೇ
ಮನೆ-ಯಿಂದ
ಮನೆಯು
ಮನೆ-ಯೊಳ-ಗಿನ
ಮನೋ
ಮನೋ-ಬುದ್ಧಿ-ರಹಂಕೃತಿಃ
ಮನೋ-ಭಾವವೇ
ಮನೋ-ಭೀಷ್ಟ-ಗಳನ್ನೂ
ಮನೋ-ರಮೇ
ಮನೋ-ರಮ್ಯಂ
ಮನೋ-ಹ-ರಮ್
ಮನೋ-ಹರ-ವಾಗಿತ್ತು
ಮನೋ-ಹರ-ವಾದ
ಮನೋ-ಹ-ರಾನ್
ಮನ್ನೇ
ಮನ್ಯತೇ
ಮನ್ಯೇ
ಮನ್ವಂತರ-ಗಳಲ್ಲಿ
ಮನ್ವಂತರೇಷು
ಮಬ್ರವೀತ್
ಮಮ
ಮಮಾಜ್ಞಯಾ
ಮಮಾನುಗಾಃ
ಮಮಾ-ನುಜ್ಞಾಂ
ಮಮಾಯಂ
ಮಮಾಯಮ-ಪರಾಧೋ
ಮಮಾಯ-ಮಿತಿ
ಮಮಾರ್ಪಿ-ತಮ್
ಮಯ
ಮಯಾ
ಮಯಾನಿ-ಶಮ್
ಮಯಾಪಿ
ಮಯಾಸೌ
ಮಯಾ-ಽ-ರಣ್ಯಂ
ಮಯಾ-ಽಽಖ್ಯಾತಂ
ಮಯಿ
ಮಯೋದಿ-ತಮ್
ಮಯೋಪ್ತಾನಿ
ಮಯೋ-ಽಯಂ
ಮಯ್ಯವ
ಮರ
ಮರ-ಗಳ
ಮರ-ಗಳಲ್ಲಿ
ಮರ-ಗಳಿಗೆ
ಮರ-ಗಳು
ಮರ-ಗಿಡ-ಗಳು
ಮರಣ
ಮರಣಂ
ಮರಣ-ವಾ-ಗಲು
ಮರಣ-ವಾದರೂ
ಮರ-ಣವು
ಮರಣ-ಹೊಂದಿ-ದೆವು
ಮರಣ-ಹೊಂದು-ವನು
ಮರದ
ಮರ-ದಲ್ಲಿ
ಮರ-ದಲ್ಲಿಯೇ
ಮರ-ದಿಂದ
ಮರಳು
ಮರ-ವನ್ನು
ಮರ-ವಾಗಿ
ಮರ-ವಾಗಿದ್ದೆ
ಮರವೊಂದನ್ನು
ಮರಿಷ್ಯತಿ
ಮರಿಷ್ಯಾವಃ
ಮರುತ್ತಂತ್ರಾಃ
ಮರುದ್ವೃಧೇವಿತಸ್ತ-ಯಾರ್ಜೀಕೇಯೇ
ಮರೆತೆ
ಮರ್ಖ-ನಾದ
ಮರ್ಚಯೇದ್ದರಿಮ್
ಮರ್ಮ-ಗಳನ್ನರಿತು
ಮರ್ಮ-ಗಳನ್ನು
ಮರ್ಮ-ಗಳನ್ನೂ
ಮರ್ಯಾದೆ
ಮಲಕಪಿಷ್ಟೇನ
ಮಲಗಿದ್ದ
ಮಲಗಿದ್ದಾಗ
ಮಲಗುತ್ತಿದ್ದನು
ಮಲದ
ಮಲ-ದಲ್ಲಿ
ಮಲ-ದಲ್ಲಿಯ
ಮಲ-ಭೋ-ಜನ
ಮಲ-ಭೋ-ಜನಃ
ಮಲ-ಭೋ-ಜನನು
ಮಲ-ಭೋ-ಜನ-ವಾಕ್ಯಾಂತೇ-ಽದೈ-ವತೋ
ಮಲಮೂತ್ರ
ಮಲಮೂತ್ರಂ
ಮಲಮೂತ್ರ-ಗಳನ್ನು
ಮಲಮೂತ್ರ-ಗಳೇ
ಮಲಮೂತ್ರ-ವಿಸರ್ಜ-ನಮ್
ಮಲಮೂತ್ರೇ
ಮಲ-ವಿಸರ್ಜನೆ
ಮಲಾಪಹಾ
ಮಲಾಸನ
ಮಲ್ಲಿ-ಕಾಕುಸುಮೈಃ
ಮಲ್ಲಿಗೆ
ಮಳೆ
ಮಳೆ-ಯಲ್ಲಿ
ಮಳೆ-ಯಿಂದ
ಮಳೆಯು
ಮಳೆಯೇ
ಮವಾಪ್ಯಾಥ
ಮಶ್ನು
ಮಹ-ತತ್ತ್ವ
ಮಹತಾ
ಮಹತಾಂ
ಮಹತೀ
ಮಹತ್
ಮಹತ್ಕೌ-ತೂಹಲಂ
ಮಹತ್ತ-ತತ್ತ್ವ
ಮಹತ್ತತ್ವ
ಮಹತ್ತತ್ವಂ
ಮಹತ್ತತ್ವ-ವಿಧಾ-ಯಕೌ
ಮಹತ್ತತ್ವಾತ್ಮಕೋ
ಮಹತ್ತತ್ವಾಭಿ-ಮಾನಿಯು
ಮಹತ್ತರ-ವಾದ
ಮಹತ್ತಾಗಿದ್ದರೂ
ಮಹತ್ತಾದ
ಮಹತ್ಪುಣ್ಯಂ
ಮಹತ್ವವಿರ-ಬೇಕೆಂದು
ಮಹತ್ಸೇವಾ
ಮಹದಂಜಸಾ
ಮಹದಪಿ
ಮಹದಪ್ಯಲ್ಪದಂ
ಮಹಾ
ಮಹಾಂತಂ
ಮಹಾಂತ-ನಮ್
ಮಹಾಂತಸ್ತರಿಮತ್ರ
ಮಹಾಂತೋ-ಽಪಿ
ಮಹಾ-ಘೋರೇ
ಮಹಾ-ತಾತ್ಪರ್ಯ
ಮಹಾ-ತಾತ್ಪರ್ಯ-ಗೌರವಾಃ
ಮಹಾತ್ಮನ
ಮಹಾತ್ಮನಃ
ಮಹಾತ್ಮನಾ
ಮಹಾತ್ಮನಾದ
ಮಹಾತ್ಮನ್
ಮಹಾತ್ಮಭಿಃ
ಮಹಾತ್ಮರ
ಮಹಾತ್ಮ-ರನ್ನೂ
ಮಹಾತ್ಮರಾದ
ಮಹಾತ್ಮರಿಗೂ
ಮಹಾತ್ಮರು
ಮಹಾತ್ಮಾ
ಮಹಾತ್ಮಾನಂ
ಮಹಾತ್ಮಾನೋ
ಮಹಾತ್ಮಾನೌ
ಮಹಾಮ್ಯೇ-ಯಲ್ಲಿ
ಮಹಾಮ್ಯೈ-ಯಲ್ಲಿ
ಮಹಾ-ದೇವ
ಮಹಾದ್ಭು-ತಮ್
ಮಹಾದ್ರುಮೇ
ಮಹಾ-ನಪಿ
ಮಹಾ-ನಯಮ್
ಮಹಾ-ನಾಸೌ
ಮಹಾ-ನುಭಾ-ವ-ರಾದ
ಮಹಾ-ನು-ಭಾವಾಃ
ಮಹಾನ್
ಮಹಾ-ಪತ್ತೌ
ಮಹಾ-ಪಾತಕ-ಕೋಟಿಭಿಃ
ಮಹಾ-ಪಾತಕ-ಗಳನ್ನು
ಮಹಾ-ಪಾತಕ-ಗಳನ್ನೂ
ಮಹಾ-ಪಾತಕ-ನಾಶ-ನ-ವರ್
ಮಹಾ-ಪಾತಕಿ-ಗಳ
ಮಹಾ-ಪಾತಕಿನಾಂ
ಮಹಾ-ಪಾಪಂ
ಮಹಾ-ಪಾಪಿ
ಮಹಾ-ಪಾಪಿ-ಯಾದ
ಮಹಾ-ಪಾಪೀ
ಮಹಾ-ಪುಣ್ಯ-ದಾಯ-ಕ-ವೆಂದು
ಮಹಾ-ಪುಣ್ಯ-ಫ-ಲದಃ
ಮಹಾ-ಪುಣ್ಯೇ
ಮಹಾ-ಪುರ್ಯಾಂ
ಮಹಾಪ್ರಾಜ್ಞಾನ್
ಮಹಾ-ಬಲೌ
ಮಹಾ-ಬಾಹೋ
ಮಹಾ-ಭಾರ-ತ-ತಾತ್ಪರ್ಯ-ನಿರ್ಣಯ-ದಲ್ಲಿ
ಮಹಾ-ಮತಿಃ
ಮಹಾ-ಮನುಮ್
ಮಹಾ-ಮಾಹಾತ್ಮ್ಯೆ-ಯನ್ನು
ಮಹಾ-ಮುನಿಃ
ಮಹಾ-ಮುನಿ-ಗಳಾದ
ಮಹಾ-ಮುನೇ
ಮಹಾ-ಯಶಾಃ
ಮಹಾ-ಯುದ್ಧಂ
ಮಹಾ-ರುದ್ರಶ್ಚಂಪ-ಕೇಶ್ವ-ರಸಂಜ್ಞಿತಃ
ಮಹಾ-ರೋಗಪ್ರ-ಶಾಂತಯೇ
ಮಹಾ-ಲಕ್ಷ್ಮಿ
ಮಹಾ-ಲಕ್ಷ್ಮಿ-ಗೋಸ್ಕರ
ಮಹಾ-ಲಕ್ಷ್ಮಿ-ದೇವಿಯರು
ಮಹಾ-ಲಕ್ಷ್ಮೈ
ಮಹಾ-ಲಕ್ಷ್ಮ್ಯಾಃ
ಮಹಾ-ವಿದ್ಯಯಾ
ಮಹಾ-ವಿಷ್ಣುಂ
ಮಹಾ-ವಿಷ್ಣೋಃ
ಮಹಾ-ವಿಷ್ಣೋಸ್ತಥಾ
ಮಹಾ-ವಿಷ್ಣೌ
ಮಹಾ-ವೀರ್ಯಃ
ಮಹಾ-ವೃಕ್ಷಂ
ಮಹಾ-ವೃಷ್ಟಿ-ರಭೂದೇ-ಕಾರ್ಣವಂ
ಮಹಾವ್ಯಾಘ್ರಃ
ಮಹಾ-ಹಂಕಾರ-ದೂಷಿ-ತಮ್
ಮಹಿಮೆ-ಗಳನ್ನು
ಮಹಿಮೆ-ಯನ್ನು
ಮಹಿಮೆ-ಯಿಂದ
ಮಹಿಮೆಯು
ಮಹಿಮೋಪೇತ-ವಾದ
ಮಹಿಷಾದಿ-ಪುರಃಸ-ರಮ್
ಮಹೀ-ಪತೇ
ಮಹೀಭತಃ
ಮಹೀಭೃತಃ
ಮಹೀ-ಯತೇ
ಮಹೋತ್ಸವ-ವನ್ನು
ಮಹೋದರಾಃ
ಮಹೋದಿತಂ
ಮಹ್ಯಂ
ಮಹ್ಯ-ಮಾಹ
ಮಹ್ಯ-ಮುಪಶಿಕ್ಷ್ಯಾನುಗೃಹ್ಯ-ತಾಮ್
ಮಾ
ಮಾಂ
ಮಾಂಗಲ್ಯಕ್ಕೆ
ಮಾಂಬಲಂ
ಮಾಂಬಳಂ
ಮಾಂಸ
ಮಾಂಸ-ಗಳನ್ನು
ಮಾಂಸ-ಪೇಶಿನೌ
ಮಾಂಸ-ಭಕ್ಷಣ-ದಿಂದ
ಮಾಂಸ-ಭಕ್ಷ-ಣಮ್
ಮಾಂಸ-ರಕ್ತ-ವಿ-ವರ್ಜಿತಃ
ಮಾಂಸ-ರಕ್ತ-ಹೀನ-ನಾಗಿಯೂ
ಮಾಘ
ಮಾಘಂ
ಮಾಘ-ಕೃತಸ್ನಾ-ನಾತ್
ಮಾಘ-ದಲ್ಲಂತೂ
ಮಾಘ-ದಲ್ಲಿ
ಮಾಘ-ಧರ್ಮೋ
ಮಾಘ-ಧರ್ಮೋ-ಧಿಕೋಽ-ಭ-ವತ್
ಮಾಘ-ಮಾಸ
ಮಾಘ-ಮಾಸಂ
ಮಾಘ-ಮಾಸಕ್ಕಿಂತ
ಮಾಘ-ಮಾಸಕ್ಕೆ
ಮಾಘ-ಮಾಸ-ಗಳಲ್ಲಿ
ಮಾಘ-ಮಾಸದ
ಮಾಘ-ಮಾಸ-ದಲ್ಲಿ
ಮಾಘ-ಮಾಸ-ದಲ್ಲಿಯೇ
ಮಾಘ-ಮಾಸ-ದಶಮ್ಯಾಂ
ಮಾಘ-ಮಾಸಪ್ರಿಯಂ
ಮಾಘ-ಮಾಸ-ಮಾಹಾತ್ಮ್ಯೇ
ಮಾಘ-ಮಾಸ-ಮಾಹಾತ್ಮ್ಯೇ-ಯಲ್ಲಿ
ಮಾಘ-ಮಾಸ-ಮಾಹಾತ್ಮ್ಯೈ
ಮಾಘ-ಮಾಸ-ವನ್ನು
ಮಾಘ-ಮಾಸ-ವಾಗಿತ್ತು
ಮಾಘ-ಮಾಸ-ವಾಗಿದ್ದರೆ
ಮಾಘ-ಮಾಸ-ವಿಧಿಂ
ಮಾಘ-ಮಾಸವು
ಮಾಘ-ಮಾಸಶ್ಚೇತ್
ಮಾಘ-ಮಾಸಸ್ಯ
ಮಾಘ-ಮಾಸಾತ್
ಮಾಘ-ಮಾಸಾತ್ಪರನೋ
ಮಾಘ-ಮಾಸಾದ-ಪರಂ
ಮಾಘ-ಮಾಸಾದ-ಪರಃ
ಮಾಘ-ಮಾಸಿ
ಮಾಘ-ಮಾಸೇ
ಮಾಘ-ಮಾಸೇಪಿ
ಮಾಘ-ಮಾಸೋ
ಮಾಘ-ಮಾಸೋಯಂ
ಮಾಘ-ಮಾಸೋ-ಽಪಿ
ಮಾಘ-ಮಾಸ್ಕು
ಮಾಘ-ಮಾಹಾತ್ಮ್ಯ
ಮಾಘ-ಮಾಹಾತ್ಮ್ಯಂ
ಮಾಘ-ಮಾಹಾತ್ಮ್ಯೈ-ಯನ್ನು
ಮಾಘ-ಮಾಹಾತ್ಯೆ-ಯನ್ನು
ಮಾಘ-ವಲ್ಲ-ಭ-ವಮ್
ಮಾಘವು
ಮಾಘ-ಶುಕ್ಲ
ಮಾಘ-ಶುದ್ಧ
ಮಾಘಶ್ವಾಯಂ
ಮಾಘ-ಸಮೋ
ಮಾಘಸ್ಕಾ-ನಾತ್ಪರಮೋಸ್ತಿ
ಮಾಘಸ್ನಾನ
ಮಾಘಸ್ನಾನಂ
ಮಾಘಸ್ನಾ-ನಕ್ಕಿಂತ
ಮಾಘಸ್ನಾ-ನಕ್ಕೆ
ಮಾಘಸ್ನಾ-ನ-ಗಳನ್ನು
ಮಾಘಸ್ನಾ-ನದ
ಮಾಘಸ್ನಾ-ನ-ದಲ್ಲಿ
ಮಾಘಸ್ನಾ-ನ-ದಿಂದ
ಮಾಘಸ್ನಾ-ನ-ಪರಾ-ಯಣ
ಮಾಘಸ್ನಾ-ನ-ಪರಾ-ಯಣಃ
ಮಾಘಸ್ನಾ-ನ-ಪರಾ-ಯ-ಣಮ್
ಮಾಘಸ್ನಾ-ನ-ಪರಾ-ಯಣಾಃ
ಮಾಘಸ್ನಾ-ನಪ್ರಭಾವೇನ
ಮಾಘಸ್ನಾ-ನ-ಫಲಂ
ಮಾಘಸ್ನಾ-ನ-ಮಭೀಪ್ಸಿ-ತಮ್
ಮಾಘಸ್ನಾ-ನ-ಮಾಡುವ
ಮಾಘಸ್ನಾ-ನ-ಮಾತ್ರ-ದಿಂದ
ಮಾಘಸ್ನಾ-ನ-ರತಂ
ಮಾಘಸ್ನಾ-ನ-ವನ್ನು
ಮಾಘಸ್ನಾ-ನ-ವಿಧೌ
ಮಾಘಸ್ನಾ-ನ-ವಿರತಾ
ಮಾಘಸ್ನಾ-ನ-ವಿ-ವರ್ಜಿ-ತಮ್
ಮಾಘಸ್ನಾ-ನವು
ಮಾಘಸ್ನಾ-ನವ್ರತ-ವನ್ನು
ಮಾಘಸ್ನಾ-ನ-ಸದೃಶೋ
ಮಾಘಸ್ನಾ-ನಸ್ಯ
ಮಾಘಸ್ನಾ-ನಾತ್ತಯೋರ್ದೆಹಿ
ಮಾಘಸ್ನಾ-ನಾತ್ಪರಮೋಸ್ತಿ
ಮಾಘಸ್ನಾ-ನಾದ-ಪರಃ
ಮಾಘಸ್ನಾ-ನಾದ-ಸರಾ
ಮಾಘಸ್ನಾ-ನಾದ್ವಿ-ಮುಕ್ತಾಸ್ಯುರ್ನಾತ್ರ
ಮಾಘಸ್ನಾ-ನಾದ್ವಿ-ಮುಚ್ಯತೇ
ಮಾಘಸ್ನಾ-ನೇನ
ಮಾಘಸ್ಯ
ಮಾಘಸ್ಯೇಂದುಕ್ಷಯೇ
ಮಾಘಾನ್ನ
ಮಾಘೀ
ಮಾಘೇ
ಮಾಘೇ-ನಾಪ-ಹೃತಂ
ಮಾಘೋ
ಮಾಘೋ-ಽಯಂ
ಮಾಘೋ-ಽಯಮಿತ್ಯುಕ್ತ್ವಾ
ಮಾಘ್ಯಂ
ಮಾಘ್ಯಾಂ
ಮಾಡದ
ಮಾಡ-ದ-ವನು
ಮಾಡ-ದ-ವರು
ಮಾಡ-ದಿದ್ದ
ಮಾಡ-ದಿದ್ದರೆ
ಮಾಡದೆ
ಮಾಡದೇ
ಮಾಡ-ಬಲ್ಲರೋ
ಮಾಡ-ಬಹು-ದಂತೆ
ಮಾಡ-ಬಹು-ದಾದ
ಮಾಡ-ಬಾರದಂತಹ
ಮಾಡ-ಬಾರದು
ಮಾಡ-ಬೇಕಾದ
ಮಾಡ-ಬೇಕಾದರೂ
ಮಾಡ-ಬೇಕು
ಮಾಡ-ಬೇಕು-ಬತ್ತಿ-ಯನ್ನು
ಮಾಡ-ಬೇಕೆಂದರೆ
ಮಾಡ-ಬೇಕೆಂದು
ಮಾಡ-ಬೇಕೆಂಬ
ಮಾಡ-ಬೇಡ
ಮಾಡ-ಲಾಗಿದೆ
ಮಾಡ-ಲಾರದೆಂಬ
ಮಾಡ-ಲಾರನು
ಮಾಡ-ಲಾ-ರರು
ಮಾಡ-ಲಾರವು
ಮಾಡ-ಲಿಕ್ಕೆ
ಮಾಡ-ಲಿಲ್ಲ
ಮಾಡಲು
ಮಾಡಲೂ
ಮಾಡ-ಲೇ-ಬೇಕು
ಮಾಡಲ್ಪಟ್ಟ
ಮಾಡಲ್ಪಡುತ್ತದೆ
ಮಾಡಿ
ಮಾಡಿ-ಕೊಂಡ
ಮಾಡಿ-ಕೊಂಡರೆ
ಮಾಡಿ-ಕೊಂಡಳು
ಮಾಡಿ-ಕೊಂಡು
ಮಾಡಿ-ಕೊಳ್ಳ
ಮಾಡಿ-ಕೊಳ್ಳ-ಬೇಕು
ಮಾಡಿ-ಕೊಳ್ಳುವ
ಮಾಡಿ-ಕೊಳ್ಳುವರೋ
ಮಾಡಿ-ಕೊಳ್ಳುವ-ವ-ರಿಗೆ
ಮಾಡಿದ
ಮಾಡಿ-ದಂತಾಗುತ್ತದೆ
ಮಾಡಿ-ದ-ಕಾರಣ
ಮಾಡಿ-ದನು
ಮಾಡಿ-ದರು
ಮಾಡಿ-ದರೂ
ಮಾಡಿ-ದರೆ
ಮಾಡಿ-ದಳು
ಮಾಡಿ-ದ-ವನ
ಮಾಡಿ-ದ-ವ-ನಾದರೂ
ಮಾಡಿ-ದ-ವ-ರಿಗೂ
ಮಾಡಿ-ದ-ವ-ರಿಗೆ
ಮಾಡಿ-ದ-ವರು
ಮಾಡಿ-ದ-ವರೂ
ಮಾಡಿ-ದಷ್ಟೇ
ಮಾಡಿ-ದಾಗ
ಮಾಡಿದೆ
ಮಾಡಿ-ದೆವು
ಮಾಡಿದ್ದರೂ
ಮಾಡಿದ್ದಾನೆ
ಮಾಡಿದ್ದೆ
ಮಾಡಿಯೂ
ಮಾಡಿಯೇ
ಮಾಡಿರಿ
ಮಾಡಿ-ರುತ್ತಾನೆ
ಮಾಡಿ-ರುತ್ತಾರೆ
ಮಾಡಿ-ರುತ್ತೆ
ಮಾಡಿ-ರುತ್ತೇವೆ
ಮಾಡಿ-ರುವ
ಮಾಡಿ-ರುವಿ
ಮಾಡಿಲ್ಲ
ಮಾಡಿ-ಸ-ದಿದ್ದರೂ
ಮಾಡಿ-ಸ-ಬೇಕು
ಮಾಡಿ-ಸ-ಲಿಲ್ಲ
ಮಾಡಿ-ಸಲು
ಮಾಡಿಸಿ
ಮಾಡಿ-ಸಿ-ಕೊಳ್ಳಲ್ಪಟ್ಟ
ಮಾಡಿ-ಸಿ-ಕೊಳ್ಳಲ್ಪಡುತ್ತಾನೆ
ಮಾಡಿ-ಸಿ-ಕೊಳ್ಳುತ್ತಿದ್ದೆ
ಮಾಡಿ-ಸಿದ
ಮಾಡಿ-ಸಿ-ದನು
ಮಾಡಿ-ಸಿ-ದರೂ
ಮಾಡಿ-ಸಿ-ದರೆ
ಮಾಡಿ-ಸುತ್ತಾನೆ
ಮಾಡಿ-ಸುತ್ತಾನೆಯೋ
ಮಾಡಿ-ಸುತ್ತಾಳೆ
ಮಾಡಿ-ಸುತ್ತಿದ್ದೆ
ಮಾಡಿ-ಸುತ್ತಿ-ರ-ಲಿಲ್ಲ
ಮಾಡಿ-ಸುತ್ತಿ-ರುವಾಗ
ಮಾಡಿ-ಸುವ
ಮಾಡಿ-ಸು-ವಂತೆ
ಮಾಡಿ-ಸುವರೋ
ಮಾಡಿ-ಸುವ-ವನ
ಮಾಡಿ-ಸುವ-ವನು
ಮಾಡಿ-ಸುವ-ವನೂ
ಮಾಡಿ-ಸುವ-ವರು
ಮಾಡಿ-ಸು-ವುದ-ರಿಂದ
ಮಾಡಿ-ಸು-ವುದು
ಮಾಡು
ಮಾಡುತ್ತದೆ
ಮಾಡುತ್ತದೆ-ಯೆಂದು
ಮಾಡುತ್ತವೆ
ಮಾಡುತ್ತಾ
ಮಾಡುತ್ತಾನೆ
ಮಾಡುತ್ತಾ-ನೆಂದು
ಮಾಡುತ್ತಾ-ನೆಯೋ
ಮಾಡುತ್ತಾನೋ
ಮಾಡುತ್ತಾರೆ
ಮಾಡುತ್ತಾ-ರೆಂದು
ಮಾಡುತ್ತಾ-ರೆಯೋ
ಮಾಡುತ್ತಾರೋ
ಮಾಡುತ್ತಿದ್ದನು
ಮಾಡುತ್ತಿದ್ದಾಗ
ಮಾಡುತ್ತಿದ್ದಾರೆ
ಮಾಡುತ್ತಿದ್ದೆ
ಮಾಡುತ್ತಿದ್ದೆನು
ಮಾಡುತ್ತಿದ್ದೆನೇ
ಮಾಡುತ್ತಿದ್ದೇವೆ-ಯಾದುದ-ರಿಂದ
ಮಾಡುತ್ತಿ-ರ-ಲಿಲ್ಲ
ಮಾಡುತ್ತಿ-ರುವ
ಮಾಡುತ್ತಿ-ರುವಾಗ
ಮಾಡುತ್ತೀ-ಯೆಂದು
ಮಾಡುತ್ತೀಯೋ
ಮಾಡುತ್ತೇನೆ
ಮಾಡುತ್ತೇನೆಂಬ
ಮಾಡುತ್ತೇವೆಂದು
ಮಾಡುವ
ಮಾಡು-ವಂತೆ
ಮಾಡು-ವನೋ
ಮಾಡು-ವರು
ಮಾಡು-ವರೋ
ಮಾಡು-ವಲ್ಲಿ
ಮಾಡು-ವ-ವ-ನಿಗೆ
ಮಾಡು-ವ-ವನು
ಮಾಡು-ವ-ವನೂ
ಮಾಡು-ವ-ವರ
ಮಾಡು-ವ-ವ-ರಿಗೆ
ಮಾಡು-ವ-ವರು
ಮಾಡು-ವ-ವರೆಗೆ
ಮಾಡು-ವುದಕ್ಕಿಂತ
ಮಾಡು-ವುದ-ರಲ್ಲಿ
ಮಾಡು-ವುದ-ರಿಂದ
ಮಾಡು-ವುದಿಲ್ಲ
ಮಾಡು-ವುದಿಲ್ಲವೋ
ಮಾಡು-ವುದು
ಮಾಡು-ವುದೇ
ಮಾಡು-ವುದೊಂದೇ
ಮಾಡು-ವೆನು
ಮಾಣಸ್ಯ
ಮಾತನಾಡಿ-ಕೊಂಡರು
ಮಾತನಾಡಿತು
ಮಾತನಾಡಿದ
ಮಾತನಾಡಿ-ದನು
ಮಾತನಾಡಿ-ದರೆ
ಮಾತನಾಡಿದ-ವ-ರಿಗೆ
ಮಾತನಾಡಿ-ದುವು
ಮಾತನಾಡಿಸಿ
ಮಾತನಾಡುತ್ತಾನೆ
ಮಾತನಾಡುತ್ತಿದ್ದೆ
ಮಾತನ್ನು
ಮಾತನ್ನೇ
ಮಾತರಮಂಗ-ನಾಮ್
ಮಾತರಶ್ಚೋಪಮಾತರಃ
ಮಾತರೋ
ಮಾತರೋ-ಽಜ್ಞಾನಾಂ
ಮಾತಾ
ಮಾತಾ-ಪಿತರೌ
ಮಾತಾ-ಪಿತೃ-ಗಳ
ಮಾತಾ-ಪಿತ್ರೋಃ
ಮಾತಾ-ಪಿತ್ರೋರ್ದಿನೇ
ಮಾತಾ-ಪಿತ್ರೋರ್ವಿಘಾ-ತಾನ್ಯೋ
ಮಾತಿ-ನಲ್ಲಿ
ಮಾತು
ಮಾತುಃ
ಮಾತು-ಕೇಳುತ್ತಾ
ಮಾತು-ಗಳನ್ನಾಡಿ-ಕೊಂಡರು
ಮಾತು-ಗಳನ್ನು
ಮಾತು-ಗಳಲ್ಲಿ
ಮಾತು-ಗಳಿಂದ
ಮಾತು-ಗಳು
ಮಾತು-ರೇವಾ-ಸೀತ್
ಮಾತುರ್ದ್ರೋಹೇಣ
ಮಾತು-ಲಾಜ್ಞಯಾ
ಮಾತು-ಲಾಹೂತೈರಸ್ಮಾಭಿಃ
ಮಾತು-ಲಿಂಗಂ
ಮಾತು-ಲಿಂಗಾನಿ
ಮಾತುಲೋ
ಮಾತುಸ್ತನಂ
ಮಾತೃ
ಮಾತೃ-ಯೌ-ವನ-ಹಾರಿಣಃ
ಮಾತೃಷ್ಟಯಸ್ತಥಾ
ಮಾತೇ
ಮಾತೈಷಾ
ಮಾತ್ರ
ಮಾತ್ರಕ್ಕೆ
ಮಾತ್ರ-ದಿಂದ
ಮಾತ್ರ-ದಿಂದಲೇ
ಮಾತ್ರವೂ
ಮಾತ್ರವೇ
ಮಾತ್ರಾಃ
ಮಾತ್ರೆ-ಗಳೂ
ಮಾದಲ
ಮಾದಳ
ಮಾದಾಯ
ಮಾಧವ
ಮಾಧವಂ
ಮಾಧವಃ
ಮಾಧವನ
ಮಾಧವ-ನನ್ನು
ಮಾಧವ-ನಾಮ-ಧೇಯಂ
ಮಾಧವ-ನಿಗೆ
ಮಾಧವನು
ಮಾಧವ-ಪಾದ-ಪದ್ಮಮ್
ಮಾಧವ-ಪೂಜ-ನಮ್
ಮಾಧವ-ಪೂಜಾರ್ಥಂ
ಮಾಧವಪ್ರತಿಮಾಂ
ಮಾಧವ-ಮಭ್ಯರ್ಚ್ಯ
ಮಾಧವ-ಮವ್ಯಯಮ್
ಮಾಧವಮ್
ಮಾಧವ-ವಲ್ಲಭೇ
ಮಾಧವ-ಸಂಜ್ಞಿತಃ
ಮಾಧವ-ಸಂಜ್ಞಿತೇ
ಮಾಧವಸ್ಯ
ಮಾಧವಸ್ಯಾಗ್ರೇ
ಮಾಧವಸ್ಯಾರ್ಚ-ನೇನ
ಮಾಧವಾಯ
ಮಾಧವಾರ್ಚಾ-ವಿ-ವರ್ಜಿತಾಃ
ಮಾಧವೇ
ಮಾಧವೋ
ಮಾಧವೋ-ಽಹಂ
ಮಾನದ
ಮಾನವಃ
ಮಾನವಾ-ನಾಮ್
ಮಾನವೈರಪುರಸ್ಕೃತಾಃ
ಮಾನಸ
ಮಾನಸಃ
ಮಾನಸಮ್
ಮಾನಸಾ
ಮಾನಸಿಕ-ವೆಂದೂ
ಮಾನಸೇ
ಮಾನಾಪಮಾನ
ಮಾನಿತಾಃ
ಮಾನಿ-ತಾಸ್ತು
ಮಾನಿ-ದೇವ-ತೆ-ಗಳು
ಮಾನುಷ-ಸಂಜ್ಞಿತಾಃ
ಮಾನುಷೋತ್ತಮಾಃ
ಮಾಪ್ನೋತಿ
ಮಾಮಕಃ
ಮಾಮಬ್ರವೀದ್ವಪ್ರಾ
ಮಾಮಶರೀರ-ವಾಕ್
ಮಾಮ್
ಮಾಯಯಾ
ಮಾಯ-ವಾಗುತ್ತ-ದೆಯೋ
ಮಾಯಾ
ಮಾಯಾ-ಧವಮಾದಿಹೇತುಮ್
ಮಾಯಾ-ಭರ್ತುರ್ಮೂರ್ತಿ-ದಾನಂ
ಮಾಯೆ-ಯಿಂದ
ಮಾರನೆಯ-ದಿನದ
ಮಾರಾಟ-ದಿಂದ
ಮಾರಾಟ-ವಾಗಿ-ಬಿಡುತ್ತಿತ್ತು
ಮಾರಿ
ಮಾರಿತಾ
ಮಾರಿದ
ಮಾರಿ-ದನು
ಮಾರಿ-ದರೆ
ಮಾರಿದೆ
ಮಾರಿದ್ದ-ರಿಂದ
ಮಾರು-ವ-ವನು
ಮಾರು-ವುದು
ಮಾರ್ಕಂಡೇಯ
ಮಾರ್ಕಂಡೇಯಸ್ಯ
ಮಾರ್ಗಂ
ಮಾರ್ಗ-ದಲ್ಲಿ
ಮಾರ್ಗ-ವಿಲ್ಲ
ಮಾರ್ಗ-ವೆಂಬ
ಮಾರ್ಗೆ
ಮಾರ್ಗೇ
ಮಾರ್ಗೇಣ
ಮಾರ್ಜನಂ
ಮಾಲತೀ
ಮಾಲತೀ-ನಾಮ
ಮಾಲತೀಸ್ರಜಾ
ಮಾಲಾ
ಮಾಲಿನ್ಯವು
ಮಾಲೆ-ಯಿಂದ
ಮಾವಿನ
ಮಾವಿನ-ಗಿಡ-ಗಳು
ಮಾಸ
ಮಾಸಕ್ಕೆ
ಮಾಸ-ಗಳಲ್ಲಿ
ಮಾಸ-ಗಳಿ-ಗಿಂತ
ಮಾಸತ್ರಯಂ
ಮಾಸದ
ಮಾಸ-ದಲ್ಲಿ
ಮಾಸ-ಮೇಕಂ
ಮಾಸ-ಮೇತಂ
ಮಾಸ-ಮೇನಂ
ಮಾಸಾತ್
ಮಾಸಾನಾಂ
ಮಾಸಾ-ನಾಮುತ್ತಮೋತ್ತಮಃ
ಮಾಸಿ
ಮಾಸೇ
ಮಾಸೋ
ಮಾಸ್ಕು-ದಿತೇ
ಮಾಸ್ಟರ್
ಮಾಸ್ಮಿ-ಗತಿಸ್ತೇಷಾಂ
ಮಾಸ್ಯ
ಮಾಹಾತ್ಮ
ಮಾಹಾತ್ಮಜ್ಞಾ-ನ-ಪೂರ್ವ-ಕ-ವಾದ
ಮಾಹಾತ್ಮ-ಯಲ್ಲಿ
ಮಾಹಾತ್ಮ-ಯಿಂದ
ಮಾಹಾತ್ಮಯು
ಮಾಹಾತ್ಮಿ-ಯನ್ನು
ಮಾಹಾತ್ಮ್ಯ
ಮಾಹಾತ್ಮ್ಯಂ
ಮಾಹಾತ್ಮ್ಯ-ವನ್ನು
ಮಾಹಾತ್ಮ್ಯಾ-ದಾಯುಃಸಿದ್ಧಿಂ
ಮಾಹಾತ್ಮ್ಯೆ
ಮಾಹಾತ್ಮ್ಯೆಯ
ಮಾಹಾತ್ಮ್ಯೆ-ಯನ್ನು
ಮಾಹಾತ್ಮ್ಯೆ-ಯಲ್ಲಿ
ಮಾಹಾತ್ಮ್ಯೇ
ಮಾಹಾತ್ಮ್ಯೇ-ಯನ್ನು
ಮಾಹಾತ್ಮ್ಯೇ-ಯಲ್ಲಿ
ಮಾಹಾತ್ಮ್ಯೇಯು
ಮಾಹಾತ್ಮ್ಯೈ-ಯನ್ನು
ಮಾಹಾತ್ಮ್ಯೈ-ಯಲ್ಲಿ
ಮಾಹಾತ್ಯೆ-ಯನ್ನು
ಮಾಹಿಷ್ಮತೀ
ಮಾಹಿಷ್ಮತೀ-ಪತಿಃ
ಮಾಹಿಷ್ಮತೀ-ಪತೇಃ
ಮಿಕ್ಕಿ-ರುವ
ಮಿಕ್ಷುಕಾ
ಮಿತಂ
ಮಿತಿ-ಗೊಳಿಸಿ
ಮಿತ್ರ
ಮಿತ್ರಃ
ಮಿತ್ರಘ್ನಂ
ಮಿತ್ರನ
ಮಿತ್ರ-ನಿಲ್ಲ
ಮಿತ್ರನೂ
ಮಿತ್ರ-ರಿಗಾಗಲೀ
ಮಿಥ
ಮಿಥ್ಯಾ
ಮಿಥ್ಯಾ-ಭಿಶಂಸಿತೋಸ್ಮಾಭಿಃ
ಮಿದಾನೀಂ
ಮಿನಿ-ಗೋತ್ರ-ದಲ್ಲಿ
ಮಿಶ್ರಣ
ಮಿಶ್ರಬುದ್ಧಯಃ
ಮಿಶ್ರ-ವಾದ
ಮಿಶ್ರಸ್ವಭಾವ-ವನ್ನು
ಮಿಶ್ರಸ್ವ-ರೂಪಿಣಃ
ಮಿಶ್ರಿತ-ವಾದ
ಮಿಷ್ಟಂ
ಮಿಷ್ಟಮಶ್ನಾಮಿ
ಮಿಷ್ಟಾನ್ನೈರ್ಭೋ-ಜಯೇದ್ಯಸ್ತು
ಮೀನು
ಮೀಮಾಂಸ
ಮೀಮಾಂಸ-ಪೂರ್ವ
ಮೀಮಾಂಸಾ
ಮೀಲಿ-ತಾಶ್ಚ
ಮೀಸಲು
ಮುಂಚಂತು
ಮುಂಚಾಮಿ
ಮುಂಚೆ
ಮುಂಚೆಯೇ
ಮುಂಜಿ
ಮುಂತಾದ
ಮುಂತಾದು-ವನ್ನು
ಮುಂತಾ-ದುವು
ಮುಂತಾ-ದುವು-ಗಳನ್ನು
ಮುಂತಾ-ದುವು-ಗಳು
ಮುಂದಕ್ಕೆ
ಮುಂದಿಟ್ಟರು
ಮುಂದಿನ
ಮುಂದು-ಮಾಡಿ-ಕೊಂಡು
ಮುಂದು-ವರಿ-ಸಿ-ದರೆ
ಮುಂದು-ವರೆ-ಸುತ್ತಾನೆ
ಮುಂದೆ
ಮುಂದೆ-ಮಾಡಿ
ಮುಂದೆಯೂ
ಮುಂಭಾಗ-ದಲ್ಲಿ
ಮುಕುಂದ
ಮುಕುಂದೇತಿ
ಮುಕ್ಕಳಿ-ಸು-ವುದು
ಮುಕ್ತಃ
ಮುಕ್ತ-ನಾ-ಗಲು
ಮುಕ್ತ-ನಾಗಿ
ಮುಕ್ತ-ನಾಗುತ್ತಾನೆ
ಮುಕ್ತ-ನಾಗುವ
ಮುಕ್ತ-ನಾದ
ಮುಕ್ತ-ನಾದನು
ಮುಕ್ತ-ರಾಗಿ
ಮುಕ್ತ-ರಾ-ಗುತ್ತಾರೆ
ಮುಕ್ತರಾಗು-ವರು
ಮುಕ್ತ-ರಾ-ದರು
ಮುಕ್ತಾ
ಮುಕ್ತಾ-ಮಯೀ
ಮುಕ್ತಿ
ಮುಕ್ತಿಂ
ಮುಕ್ತಿಃ
ಮುಕ್ತಿ-ಗಳು
ಮುಕ್ತಿಗಾ
ಮುಕ್ತಿಗೆ
ಮುಕ್ತಿ-ಯನ್ನು
ಮುಕ್ತಿಯು
ಮುಕ್ತಿ-ಯೋಗ್ಯ-ರೆಂದೂ
ಮುಕ್ತಿ-ರಾ-ಸೀತ್
ಮುಕ್ತಿ-ರಿತ್ಯಥ
ಮುಕ್ತೇ
ಮುಕ್ತೋ
ಮುಕ್ತೋ-ಭೂದ್ದು
ಮುಕ್ತೌ
ಮುಖ
ಮುಖಂ
ಮುಖ-ಗಳನ್ನುಳ್ಳ
ಮುಖಮ್
ಮುಖ-ವನ್ನು
ಮುಖವು
ಮುಖ-ವುಳ್ಳ
ಮುಖ-ವುಳ್ಳ-ವ-ನಾಗಿ
ಮುಖಾದ್ಯಾ
ಮುಖಾನ್
ಮುಖಾಸ್ತದಾ
ಮುಖ್ಯ
ಮುಖ್ಯ-ವಾಗಿ
ಮುಖ್ಯಾದ್ವಿಷ್ಣುಂ
ಮುಖ್ಯಾಭಿ-ಮಾನಿ-ಯೆಂದೂ
ಮುಗಿದ-ಮೇಲೆ
ಮುಗಿ-ದರು
ಮುಗಿದಿಲ್ಲ
ಮುಗಿ-ಯಲು
ಮುಗಿ-ಯಿತು
ಮುಗಿ-ಯುವಷ್ಟ-ರಲ್ಲಿ
ಮುಗಿಸ-ಬೇಕು
ಮುಗಿ-ಸಲು
ಮುಗಿಸಿ
ಮುಚ್ಚಲ್ಪಟ್ಟ
ಮುಚ್ಯಂತೇ
ಮುಚ್ಯತೇ
ಮುಚ್ಯೇತ
ಮುಟ್ಟ-ಬಾರದು
ಮುಟ್ಟ-ಲಿಲ್ಲ
ಮುಟ್ಟಿ-ದುದು
ಮುಟ್ಟಿದ್ದುವು
ಮುತ್ತಮಮ್
ಮುತ್ತಿನ-ಹಾರ-ಗಳೆಲ್ಲಿ
ಮುತ್ತು-ಗದ
ಮುತ್ತು-ಗಳಿಂದ
ಮುದಾ
ಮುದಾಸ್ತೇ
ಮುದಿರಾಂ
ಮುದೃ-ಣಮ್
ಮುದ್ಗಲಃ
ಮುದ್ಗಲ-ಗೋತ್ರಜಃ
ಮುದ್ಗಲ-ಗೋತ್ರ-ದಲ್ಲಿ
ಮುದ್ಗಲನು
ಮುದ್ಗಲೋ
ಮುದ್ದಲ-ನೆಂಬ
ಮುದ್ದಾಸ್ಯೇ
ಮುದ್ದಿಶ್ಯ
ಮುದ್ದೆ
ಮುದ್ಧಲ-ನನ್ನು
ಮುದ್ಧಲನು
ಮುದ್ರೆ
ಮುನಯಃ
ಮುನಯೇ
ಮುನಿಂ
ಮುನಿಃ
ಮುನಿಕ್ರೀತಂ
ಮುನಿಗಣೇ
ಮುನಿ-ಗಣೈಃ
ಮುನಿ-ಗಳ
ಮುನಿ-ಗಳನ್ನು
ಮುನಿ-ಗಳನ್ನೂ
ಮುನಿ-ಗಳು
ಮುನಿನಾ
ಮುನಿನಾ-ಽತಿ-ದಯಾ-ಲುನಾ
ಮುನಿಪುಂಗವ
ಮುನಿ-ಪುತ್ರ
ಮುನಿ-ಪುತ್ರಕಃ
ಮುನಿ-ಪುತ್ರ-ನನ್ನು
ಮುನಿ-ಪುತ್ರ-ನಿಗೆ
ಮುನಿ-ಪುತ್ರನು
ಮುನಿ-ಪುತ್ರನೇ
ಮುನಿ-ಪುತ್ರ-ಮೂಚುಃ
ಮುನಿ-ಪುತ್ರೇಣ
ಮುನಿ-ಪುತ್ರೋ
ಮುನಿ-ಪುತ್ರೋಪಿ
ಮುನಿ-ಪುಷ್ಪೈಸ್ತು
ಮುನಿಭಿರ್ಭಾವಿ-ತಾತ್ಮಭಿಃ
ಮುನಿಮ್
ಮುನಿಯ
ಮುನಿ-ಯನ್ನು
ಮುನಿಯು
ಮುನಿಯೇ
ಮುನಿರ್ಗೃತ್ಸಮದಾಹ್ವಯಃ
ಮುನಿಶಾರ್ದೂಲ
ಮುನಿಶಾರ್ದೂಲಃ
ಮುನಿಶ್ಚಿಂತಾ-ಪರಾ-ಯಣಃ
ಮುನಿಶ್ರೇಷ್ಠ
ಮುನಿಶ್ರೇಷ್ಠನೆ
ಮುನಿಶ್ರೇಷ್ಠನೇ
ಮುನಿಶ್ರೇಷ್ಠ-ರನ್ನು
ಮುನಿಶ್ರೇಷ್ಠ-ರಾದ
ಮುನಿಶ್ರೇಷ್ಠರೇ
ಮುನಿಶ್ವರಾಃ
ಮುನಿ-ಸತ್ತಮಃ
ಮುನಿಸುತೇ
ಮುನೀಂದ್ರ
ಮುನೀಂದ್ರಂ
ಮುನೀಂದ್ರ-ಪುತ್ರೇಣ
ಮುನೀಂದ್ರಸ್ತೈಃ
ಮುನೀಂದ್ರೇಣ
ಮುನೀನಾಂ
ಮುನೀಮ್
ಮುನೀಶ್ವರ
ಮುನೀಶ್ವರಃ
ಮುನೇ
ಮುನೇಃ
ಮುನೇ-ಽದ್ಭು
ಮುನೌ
ಮುನ್ವಾದ್ಯಾಃ
ಮುಪ್ಪಿ-ನಿಂದ
ಮುಪ್ಪು
ಮುಮುಕ್ಷು-ಗಳಿಂದ
ಮುರಾರಿಃ
ಮುರ್ತ್ಯೋ
ಮುಲನ
ಮುಳುಗಿ
ಮುಳುಗಿ-ಸುವ
ಮುಳುಗುವ
ಮುಳ್ಳಿನ
ಮುಷ್ಟಾ-ಮುಷ್ಟಿ
ಮುಷ್ಟಿ
ಮುಷ್ಟಿ-ಯೆಂಬ
ಮುಷ್ಯಭಿಧೇ
ಮುಹುಃ
ಮುಹುರ್ಮುಹುಃ
ಮೂಕ
ಮೂಕಾಂಬಿಕೆ
ಮೂಕಾಂಬಿಕೇ
ಮೂಕೋ
ಮೂಢ
ಮೂಢಂ
ಮೂಢಃ
ಮೂಢಧೀಃ
ಮೂಢ-ಧೀರ-ಹಮ್
ಮೂಢ-ನನ್ನು
ಮೂಢನು
ಮೂಢನೇ
ಮೂಢರು
ಮೂಢಾಃ
ಮೂತ್ರೋತ್ಸರ್ಗ-ವಿಧಿತ್ವೇಷೋ
ಮೂನ್
ಮೂರನೆಯ
ಮೂರನೆ-ಯದು
ಮೂರನೆಯ-ವನು
ಮೂರನೇ
ಮೂರು
ಮೂರೂ-ವರೆ
ಮೂರ್ಖ
ಮೂರ್ಖ-ನಾಗಿ
ಮೂರ್ಖನು
ಮೂರ್ಖ-ರಿಗೆ
ಮೂರ್ಛ-ರೋ-ಗ-ದಿಂದ
ಮೂರ್ಛಾ-ಮಾಪ
ಮೂರ್ಛಾ-ಮಾಪುರ್ಮಹಾ-ದುಃಖಾ
ಮೂರ್ಛಿತಂ
ಮೂರ್ಛಿ-ತನಾ-ದು-ದನ್ನು
ಮೂರ್ಛೆ-ಯನ್ನು
ಮೂರ್ಧನ್ಯಾಘ್ರಾಯ
ಮೂರ್ಧ್ನಿ
ಮೂಲ
ಮೂಲಂ
ಮೂಲಂಗಿ
ಮೂಲಕ
ಮೂಲ-ಕಮ್
ಮೂಲ-ಗಳು
ಮೂಲಗ್ರಂಥ-ವನ್ನು
ಮೂಲ-ದಲ್ಲಿ
ಮೂಲ-ಭಾಗ
ಮೂಲ-ವಾದ
ಮೂಲ-ಸ-ಹಿತ
ಮೂವತ್ತ
ಮೂವತ್ತೆರಡು
ಮೂವ-ರನ್ನೂ
ಮೂವ-ರಿಗೆ
ಮೂವರೂ
ಮೂಷಕೋತ್ಖಾತಂ
ಮೃ
ಮೃಗ
ಮೃಗ-ಗಳನ್ನು
ಮೃಗ-ಗಳು
ಮೃಗಯಾಂ
ಮೃಗ-ಯಾ-ಮಾರ್ಗೇ
ಮೃಗ-ಯಾ-ಸಕ್ತ-ಚೇತಸಃ
ಮೃಗ-ಶಿರ
ಮೃಗ-ಸಿಂಹಾ-ಪರಾ-ಯಣಃ
ಮೃಗಾ
ಮೃಗಾನ್ನಾನಾ-ವಿಧಾನ್
ಮೃಗಾ-ಯು-ಧರ್ಮೇಣ
ಮೃಚ್ಛಿಲಾ-ಮಯಾಃ
ಮೃಡೋಹಂಕೃತಿ-ರೂಪ-ವಾನ್
ಮೃತಂ
ಮೃತ-ನಾದನು
ಮೃತ-ನಾದೆ
ಮೃತ-ನಿಗೆ
ಮೃತ-ಪಟ್ಟ-ವ-ರಿಗೆ
ಮೃತ-ಪಟ್ಟು
ಮೃತಪ್ರಾ-ಯ-ನಂತೆ
ಮೃತಮ್
ಮೃತ-ರಾಗಿ
ಮೃತರಾಗುತ್ತೇವೆ
ಮೃತ-ರಾದ
ಮೃತರಾ-ದಾಗ
ಮೃತಳಾ-ದಳು
ಮೃತಸ್ಯ
ಮೃತಸ್ಯಾಪಿ
ಮೃತಾ
ಮೃತಾಃ
ಮೃತಾನಾಂ
ಮೃತಾ-ಪತ್ಯಸ್ಯ
ಮೃತಾ-ಹಮುಪಿ
ಮೃತಿಂ
ಮೃತಿ-ಮಾಪನ್ನಾ
ಮೃತಿಮುಪಾ-ಗತಾಃ
ಮೃತಿ-ಯನ್ನು
ಮೃತಿರೌದುಂಬರೇ
ಮೃತಿರ್ಜಾತಾ
ಮೃತಿರ್ಮತ್ತ-ಮತಂಗ-ಜಾತ್
ಮೃತಿರ್ವಾ
ಮೃತೇ
ಮೃತೇನ
ಮೃತೋ
ಮೃತೋ-ಪ-ಮಾಮ್
ಮೃತೋ-ಮೃತಃ
ಮೃತೋ-ಽಸಾನಪ-ಮೃತ್ಯು
ಮೃತೌ
ಮೃತ್ತಿಕಾಃ
ಮೃತ್ತಿಕಾ-ದಿಂದ
ಮೃತ್ತಿ-ಕಾನಿಧಿಃ
ಮೃತ್ತಿ-ಕಾಲೇ-ಪನ
ಮೃತ್ತಿ-ಕಾಲೇ-ಪನ-ದಿಂದ
ಮೃತ್ತಿಕಾ-ಶೌಚ
ಮೃತ್ತಿಕಾಸ್ನಾ-ನ-ಮಾ-ಚ-ರೇತ್
ಮೃತ್ತಿಕೆ-ಯನ್ನು
ಮೃತ್ತಿಕೆ-ಯಿಂದ
ಮೃತ್ಯು
ಮೃತ್ಯುಂ
ಮೃತ್ಯುಂಜ-ಯೇನ್ನರಃ
ಮೃತ್ಯುಃ
ಮೃತ್ಯುರ್ವಿನಾ-ಶಿತಃ
ಮೃತ್ಯು-ವಿಗೆ
ಮೃತ್ಯುವು
ಮೃದಾ
ಮೃದಾ-ಲಿಪ್ಯ
ಮೃದ್ವಯಂ
ಮೃಷ್ಟಾನ-ವನ್ನು
ಮೃಷ್ಟಾನ್ನ
ಮೃಷ್ಟಾನ್ನ-ದಿಂದ
ಮೆಟ್ಟಲಿ-ನಂತೆ
ಮೆಣಸು-ಇವೇ
ಮೆರೆಯುತ್ತಾನೆ
ಮೆರೆ-ಯುತ್ತಾರೆ
ಮೇ
ಮೇಘ-ಗಳಿಂದ
ಮೇಘಚ್ಛನ್ನಂ
ಮೇಘನ
ಮೇಘನು
ಮೇಘ-ನೆಂಬ
ಮೇದಸ್ಸು
ಮೇದೋಸ್ಥಿ-ರುಧಿರಾಣಿ
ಮೇದ್ವ್ಯೇಷ್ಯೋಸ್ತಿ
ಮೇನೇ
ಮೇಯುತ್ತಿದ್ದ
ಮೇರು
ಮೇರುಣಾ
ಮೇರು-ಪರ್ವ-ತಮ್
ಮೇರು-ಮಂದರ
ಮೇರು-ಮಂದರ-ತುಲ್ಯಂ
ಮೇರೌ
ಮೇಲಕ್ಕೆ
ಮೇಲಿಂದ-ಮೇಲೆ
ಮೇಲಿನ
ಮೇಲಿ-ರುವ
ಮೇಲೆ
ಮೇಲೆಯೂ
ಮೇಷ
ಮೇಷ-ಮಾಸ-ದಲ್ಲಿ
ಮೇಷ-ಮಾಸೇ
ಮೇಷಸ್ಥೆ
ಮೇಷಸ್ನಾನಂ
ಮೇಽಧುನಾ
ಮೇಽ-ನಿಶಂ
ಮೇಽ-ವದತ್
ಮೈ
ಮೈಯೊಳ-ಗಿನ
ಮೊಗ್ಗು
ಮೊಚ-ಯಿತ್ವಾ
ಮೊದಲನೆಯ
ಮೊದಲನೆಯ-ವನ
ಮೊದ-ಲನೇ
ಮೊದಲಾದ
ಮೊದಲಾದ-ವರು
ಮೊದಲಾದ-ವರೂ
ಮೊದಲಾದ-ವು-ಗಳು
ಮೊದಲಾದು-ವನ್ನು
ಮೊದಲಾದುವು
ಮೊದಲಿನ
ಮೊದಲು
ಮೊದಲು-ಗೊಂಡು
ಮೊಮ್ಮಗನ
ಮೊಮ್ಮಗನು
ಮೊಳ-ಕಾಲಿ-ನಲ್ಲಿ
ಮೊಳಕೈ
ಮೊಳ-ಕೈ-ಯಲ್ಲಿ
ಮೊಸ-ರನ್ನು
ಮೊಸರಿ-ನಿಂದ
ಮೊಸರು
ಮೋಕ
ಮೋಕಕ್ಕೆ
ಮೋಕವು
ಮೋಕ್ಷ
ಮೋಕ್ಷಂ
ಮೋಕ್ಷಃ
ಮೋಕ್ಷಕ್ಕೆ
ಮೋಕ್ಷಕ್ಕೊಸ್ಕರ
ಮೋಕ್ಷ-ಪಾಯ-ವನ್ನು
ಮೋಕ್ಷ-ಬೀಜಂ
ಮೋಕ್ಷ-ಮವಾಪ್ನು-ಯಾತ್
ಮೋಕ್ಷ-ಮವಾಪ್ನೋತಿ
ಮೋಕ್ಷ-ಮಾಪುಸ್ತ್ರಯೋ-ಽಪಿ
ಮೋಕ್ಷ-ಮಾಪ್ನು-ಯಾತ್
ಮೋಕ್ಷ-ಮಾಪ್ಯ
ಮೋಕ್ಷ-ಮಾರ್ಗ-ಮುಪೇಯಿ-ವಾನ್
ಮೋಕ್ಷ-ಮಾರ್ಗ-ವನ್ನು
ಮೋಕ್ಷ-ವನ್ನು
ಮೋಕ್ಷ-ವನ್ನೂ
ಮೋಕ್ಷ-ವಾಗು-ವುದಿಲ್ಲ
ಮೋಕ್ಷ-ವಿಲ್ಲ
ಮೋಕ್ಷವು
ಮೋಕ್ಷ-ವೆಂಬ
ಮೋಕ್ಷ-ವೆಂಬುವು
ಮೋಕ್ಷಶ್ಚತೂ-ರಸಃ
ಮೋಕ್ಷ-ಸಾ-ಧ-ನಮ್
ಮೋಕ್ಷ-ಸಾಧನೆ-ಯಲ್ಲಿ
ಮೋಕ್ಷಾದಿ-ದಿವ್ಯ-ಪುರುಷಾರ್ಥ-ಕುಲಸ್ಯ
ಮೋಕ್ಷಾಧಿ-ಕಾರಿಯು
ಮೋಕ್ಷಾಯ
ಮೋಕ್ಷೇಚ್ಛುವು
ಮೋಕ್ಷೋ
ಮೋಚಕ-ವಾದ
ಮೋದತೇ
ಮೋದಿ-ತಾನ್
ಮೋಹ
ಮೌಂಜೀ-ವಿವಾಹಾದಿಷು
ಮೌದ್ಗಲಃ
ಮೌದ್ಗಲ್ಯ
ಮೌದ್ಗಲ್ಯ-ಗೋತ್ರಜಃ
ಮೌಲ್ಯ
ಮೌಲ್ಯ-ಕಮ್
ಮೌಲ್ಯ-ಮೇವ
ಮೌಲ್ಯಾತ್ಪುರೋಕ್ತೇಭ್ಯಃ
ಮೌಲ್ಯಾ-ದಾ-ನೀಯ
ಮೌಲ್ಯೈಶ್ಚ
ಯ
ಯಂತ್ರ-ದಲ್ಲಿ-ರುವ
ಯಂತ್ರ-ವನ್ನು
ಯಂತ್ರಾರೂಢಾನಿ
ಯಃ
ಯಕ್ಷ
ಯಕ್ಷ-ಗೋತ್ರ
ಯಕ್ಷರು
ಯಕ್ಷರೂ
ಯಕ್ಷಾಃ
ಯಕ್ಷಾದ್ಯಾ
ಯಚ್ಚ
ಯಚ್ಛಂತಿ
ಯಚ್ಛತಿ
ಯಚ್ಛ್ರದ್ಧಃ
ಯಚ್ಛ್ರು
ಯಜಂತೋಪಿ
ಯಜತಾಂ
ಯಜಮಾನರ
ಯಜಮಾನರು
ಯಜ್ಜು
ಯಜ್ಞ
ಯಜ್ಞಕ್ಕಾಗಿ
ಯಜ್ಞಕ್ಕೆ
ಯಜ್ಞಕ್ಕೋಸ್ಕರ
ಯಜ್ಞ-ಗಳ
ಯಜ್ಞ-ಗಳಂತೂ
ಯಜ್ಞ-ಗಳನ್ನು
ಯಜ್ಞ-ಗಳನ್ನೂ
ಯಜ್ಞ-ಗಳಲ್ಲಿಯೂ
ಯಜ್ಞ-ಗಳೂ
ಯಜ್ಞದ
ಯಜ್ಞ-ದೀಕ್ಷೆ-ಯಲ್ಲಿದ್ದ
ಯಜ್ಞ-ಫ-ಲದಾ
ಯಜ್ಞ-ಫಲವು
ಯಜ್ಞ-ಯಾಗಾದಿ-ಗಳಿಂದ
ಯಜ್ಞ-ವನ್ನಾ-ಚರಿ-ಸುವ-ವರು
ಯಜ್ಞ-ವಿಲ್ಲ
ಯಜ್ಞವೇ
ಯಜ್ಞಸ್ಯ
ಯಜ್ಞಾ
ಯಜ್ಞಾ-ದಿ-ಗಳನ್ನು
ಯಜ್ಞಾನ್
ಯಜ್ಞಾರ್ಥಂ
ಯಜ್ಞಾರ್ಥಿ
ಯಜ್ಞಾರ್ಥೇ
ಯಜ್ಞಾಶ್ಚ
ಯಜ್ಞೇಷು
ಯಜ್ಞೋ
ಯತಃ
ಯತತೋ
ಯತಶ್ಚ
ಯತಿ
ಯತಿಈ
ಯತಿ-ಗಳಿಗೂ
ಯತಿ-ರೇವ
ಯತೀನಾಂ
ಯತೀಶ್ವರಮ್
ಯತೇ-ತಾದೌ
ಯತೋ
ಯತ್
ಯತ್ಕ-ರೋಷಿ
ಯತ್ಕುರ್ವತೋ-ಪೈತಿ
ಯತ್ಕೃತಂ
ಯತ್ತ
ಯತ್ತ-ವಾಂಘ್ರಿಯು-ಗಾಂಭೋಜೇ
ಯತ್ನ
ಯತ್ನರ್ಮ
ಯತ್ಪಾದೇ
ಯತ್ಪಾಪಂ
ಯತ್ಪುಣ್ಯ
ಯತ್ಪುಣ್ಯಂ
ಯತ್ಪ್ರಮಾಣಂ
ಯತ್ಫಲಮ್
ಯತ್ರ
ಯತ್ರಾಸ್ತೇ
ಯಥಾ
ಯಥಾಂಕುಶಃ
ಯಥಾಕ್ರ-ಮಮ್
ಯಥಾಗ್ನಿಶಿಖಯಾ
ಯಥಾ-ದಾಯ
ಯಥಾ-ನು-ಪೂರ್ವ-ವಮ್
ಯಥಾಪ್ರಾಪ್ತಾಸು
ಯಥಾ-ಯೋಗ್ಯ-ವಾಗಿ
ಯಥಾ-ಯೋಗ್ಯ-ವಾದ
ಯಥಾರ್ಜುನಂ
ಯಥಾರ್ಥ
ಯಥಾರ್ಥಜ್ಞಾ-ನ-ವನ್ನು
ಯಥಾರ್ಥಜ್ಞಾ-ನ-ವಿಲ್ಲ
ಯಥಾರ್ಥಜ್ಞಾ-ನವು
ಯಥಾರ್ಥಜ್ಞಾ-ನವೂ
ಯಥಾರ್ಥಜ್ಞಾ-ನೋತ್ಪತ್ತಿ
ಯಥಾ-ವಿಧಿ
ಯಥಾ-ವಿಭವಮರ್ಚ-ಯೇತ್
ಯಥಾ-ಶಕ್ತಿ
ಯಥಾಷ್ಟಾ-ವಿಂಶತಿ
ಯಥಾಸ್ಥಾನಂ
ಯಥಾ-ಹೀನಾಂ
ಯಥಾ-ಽಶ್ವತ್ಥಃ
ಯಥೇಚ್ಛತಿ
ಯಥೇಚ್ಛಯಾ
ಯಥೇಚ್ಛ-ವಾಗಿ
ಯದಗ್ರೇ
ಯದನ್ಯತ್ತೇ
ಯದರ್ಪಣಂ
ಯದಶ್ನಾಸಿ
ಯದಾ
ಯದಾಶ್ನಾತಿ
ಯದಾಶ್ರಮೇ
ಯದಿ
ಯದುಕ್ತಂ
ಯದೈವ
ಯದೈವಂ
ಯದೋನ್ಮುಖಾ
ಯದೌರ್ಧ್ವ-ದೇಹಾ-ದಿಕ-ಕರ್ಮ-ಜಾ-ತಮ್
ಯದ್
ಯದ್ಗೃಹೇ
ಯದ್ಗೃಹೋ-ಪರಿ
ಯದ್ದತ್ತಂ
ಯದ್ದಾನಂ
ಯದ್ದಿನಂ
ಯದ್ದೇಯಂ
ಯದ್ಬಾಲ್ಯೇ
ಯದ್ಯದಾ-ಚರತಿ
ಯದ್ಯಪಿ
ಯದ್ಯಸ್ತಿ
ಯದ್ಯೇ-ಕೋ-ಽಪಿ
ಯದ್ವತ್
ಯದ್ವಾ
ಯನಂ
ಯನ್ನಃ
ಯನ್ನಖಾಗ್ರಾತ್
ಯನ್ನಿ-ವೇದಿ-ತ-ನಮ್
ಯನ್ನು
ಯನ್ಮಧ್ಯೇ
ಯನ್ಮೂಲೇ
ಯಮ
ಯಮಃ
ಯಮ-ಕಿಂಕರ-ರಾಗಿದ್ದರೆ
ಯಮ-ಕಿಂಕರಾಃ
ಯಮ-ಕಿಂಕ-ರಾನ್
ಯಮ-ದೂತರ
ಯಮ-ದೂತರು
ಯಮ-ದೂತಾ
ಯಮ-ದೂತಾಸ್ತೇ
ಯಮ-ದೇವ-ರಿಂದ
ಯಮ-ದೇವರು
ಯಮ-ದೇವರೂ
ಯಮ-ದೇವರೇ
ಯಮ-ಧರ್ಮ-ರಾಜನು
ಯಮ-ಧರ್ಮ-ರಾಜರು
ಯಮ-ನಂತೆ
ಯಮ-ಪಾಶತಃ
ಯಮ-ಪಾಶ-ದಿಂದ
ಯಮ-ಬಂಧ
ಯಮಭಟ-ರನ್ನು
ಯಮಭಟರು
ಯಮಭಟರೇ
ಯಮ-ಭಟಾ
ಯಮರು
ಯಮ-ಲೋಕಂ
ಯಮ-ಲೋಕಕ್ಕೆ
ಯಮ-ಲೋಕದ
ಯಮಶಾಸನೇ
ಯಮಸ್ತಂ
ಯಮಾಜ್ಞಯಾ
ಯಮಾ-ನುಗಃ
ಯಮಾ-ಲಯ-ನಮ್
ಯಮಾ-ಲಯಮ್
ಯಮುನಾ
ಯಮು-ನೆಯು
ಯಮುನೇ
ಯಮೇವೈಷ
ಯಮೋಕ್ತಾನ್ಯ-ಥಾಽಭ್ರುವನ್
ಯಮೌ
ಯಯಾಚಾತೇ
ಯಯಾಚೇ
ಯಯಾವ-ಮುಮ್
ಯಯುಃ
ಯಯುರ್ಯ-ವನಿಕೇ-ತ-ನಮ್
ಯಯುರ್ಲೋಕಂ
ಯಯೌ
ಯರ್ಥಾದಿಭ್ಯಃ
ಯಲ್ಲಿ
ಯವನ-ದೇಶ-ದಲ್ಲಿ
ಯವನ-ದೇಶೇಷು
ಯವನ-ರಾಜನ
ಯವನಾಧಿ-ಪತೇಃ
ಯವನು
ಯಶಸ್ತಸ್ಮಾತ್
ಯಶಸ್ಸು
ಯಶೋಽಮಲಮ್
ಯಸ್ತಿ-ಥಯೋ
ಯಸ್ತು
ಯಸ್ತುಲಸ್ಯಾಃ
ಯಸ್ಮಾಜ್ಜಗದಿದಂ
ಯಸ್ಮಾತ್
ಯಸ್ಮಾತ್ಫಲ-ಮವಾಪ್ನು-ಯಾತ್
ಯಸ್ಮಾತ್ಸುರಾಪೀತಿ
ಯಸ್ಮಾದ್ಭವ
ಯಸ್ಮಿಂಕಾಲೇ
ಯಸ್ಮಿನ್
ಯಸ್ಯ
ಯಸ್ಯಾವ-ತಿಷ್ಠತೇ
ಯಸ್ಯೇಂದ್ರಿಯಾಣಿ
ಯಾ
ಯಾಂ
ಯಾಂತಿ
ಯಾಃ
ಯಾಗ-ಗಳಿಂದ
ಯಾಗ-ಫಲ-ವನ್ನು
ಯಾಗ-ಮಾಡುತ್ತಿ-ರುವ-ವನು
ಯಾಗಲೀ
ಯಾಗಾದಿ
ಯಾಗಾದಿ-ಗಳನ್ನು
ಯಾಗಿ
ಯಾಚ-ಕಸ್ಯ
ಯಾಚ-ಯಿತ್ವಾ
ಯಾಚಿತರಾಗದೇ
ಯಾಚಿತ್
ಯಾಚಿ-ಸಲು
ಯಾಚಿಸಿ-ದನು
ಯಾಚಿಸಿ-ದರು
ಯಾಚೇ
ಯಾಜಕಂ
ಯಾಜಕಃ
ಯಾಜ್ಞಿ
ಯಾತಂ
ಯಾತಃ
ಯಾತಕ್ಕಾಗಿ
ಯಾತಕ್ಕೆ
ಯಾತನಾ
ಯಾತನಾಂ
ಯಾತ-ನಾಮು
ಯಾತ-ನಾಮ್
ಯಾತನಾ-ಲ-ಯಾನ್
ಯಾತನೆ-ಗಳನ್ನು
ಯಾತನೆ-ಯನ್ನು
ಯಾತಾ
ಯಾತಿ
ಯಾತುಧಾ-ನಾದ್ಯಾ
ಯಾತ್ರೆ-ಯಲ್ಲಿ
ಯಾದುದ-ರಿಂದ
ಯಾದೃಗೇವ
ಯಾದೃಶೇನ
ಯಾನಿ
ಯಾನುತ್ರಯಾ-ದೂರ್ಧ್ವಂ
ಯಾನ್ಯಹಂ
ಯಾಮ
ಯಾಮಂ
ಯಾಮ-ಮನ-ಯದ್ವಿಪ್ರೊ
ಯಾಮ-ಮೇವ
ಯಾಮು
ಯಾಯಾತ್ಸ
ಯಾರ
ಯಾರನ್ನು
ಯಾರಲ್ಲಿಯೂ
ಯಾರಾ-ದರೂ
ಯಾರಿಂದ
ಯಾರಿಂದ-ಲಾ-ದರೂ
ಯಾರಿಗೂ
ಯಾರಿಗೆ
ಯಾರಿದ್ದಾರೆ
ಯಾರು
ಯಾರೂ
ಯಾರೇ
ಯಾವ
ಯಾವಜ್ಜೀವಂ
ಯಾವತ್
ಯಾವತ್ಪಾಪ-ಭಯಂ
ಯಾವತ್
ಯಾವ-ದಾ-ಭೂತಸಂಪ್ಲ-ವಮ್
ಯಾವ-ದಾಮಲಶೋ-ಧ-ನಮ್
ಯಾವ-ದಿಂದ್ರಾಶ್ಚತುರ್ದಶ
ಯಾವನು
ಯಾವ-ರೀತಿ-ಯಿಂದಲೂ
ಯಾವವು
ಯಾವಾಗ
ಯಾವಾಗಲೂ
ಯಾವಾತ್ಪೂರ್ವಂ
ಯಾವು
ಯಾವು-ದನ್ನು
ಯಾವು-ದನ್ನೂ
ಯಾವು-ದಾದ-ರೊಂದು
ಯಾವುದು
ಯಾವುದೂ
ಯಾವು-ದೆಂಬು-ದನ್ನು
ಯಾವುವು
ಯಿಂದ
ಯು
ಯುಕ್ತ
ಯುಕ್ತ-ನಾಗಿ
ಯುಕ್ತ-ನಾಗಿಯೂ
ಯುಕ್ತ-ನಾದ
ಯುಕ್ತ-ರಾಗಿ
ಯುಕ್ತ-ಳಾಗಿ
ಯುಕ್ತ-ವಾಗಿದೆ
ಯುಕ್ತ-ವಾಗಿದ್ದರೆ
ಯುಕ್ತ-ವಾಗಿ-ರಲು
ಯುಕ್ತ-ವಾದ
ಯುಕ್ತ-ವೆಂದು
ಯುಕ್ತೇ
ಯುಗ-ದಲ್ಲಿ
ಯುಗ-ಪತ್
ಯುಗಸ್ಯಾಂತೇ
ಯುಗೇ
ಯುಗೇಷ್ಟೇವ
ಯುಗೇ-ಽಪಿ
ಯುಜ್ಯತೇ
ಯುದ್ಧ
ಯುದ್ಧಂ
ಯುದ್ಧಕ್ಕೆ
ಯುದ್ಧತ್ತಂ
ಯುದ್ಧ-ದಲ್ಲಿ
ಯುದ್ಧ-ಮಾಡಿ
ಯುದ್ಧ-ವಾ-ಯಿತು
ಯುಧಿ
ಯುಧ್ಯಸ್ವ
ಯುಯುಧೇ
ಯುರಭಿ-ವರ್ಧ-ಯಿತುಂ
ಯುವ-ಕನೊಂದಿಗೆ
ಯುವ-ಕ-ರಾದ
ಯುವಾನಂ
ಯುವಾನೌ
ಯುವಾ-ಸಮಗ್ರ್ಯಮ್
ಯುಷ್ಮತ್ತೀರ್ಥನಿಷೇವಣೇ
ಯುಷ್ಮತ್ಪುಣ್ಯ
ಯುಷ್ಮದ್ದರ್ಶನಂ
ಯೂಯಂ
ಯೆಂತಲೂ
ಯೇ
ಯೇನ
ಯೇನಾಹ
ಯೇನೋಪ-ಕಾರಃ
ಯೇನೌರಸಾಃ
ಯೇಯಂ
ಯೇಷಾಂ
ಯೊ
ಯೊಗ್ಯಾ
ಯೋ
ಯೋಗಃ
ಯೋಗತ್ವ-ಮಾಪ್ನು-ಯಾತ್
ಯೋಗವು
ಯೋಗ-ವೆಂದು
ಯೋಗಾದಿಷು
ಯೋಗಾದ್ಭಾವಂ
ಯೋಗಾಭ್ಯಾಸ-ದಿಂದ
ಯೋಗಿನಾಂ
ಯೋಗಿಯಾಗುತ್ತಾನೆ
ಯೋಗೀ
ಯೋಗೇನ
ಯೋಗೊ
ಯೋಗ್ಯ
ಯೋಗ್ಯ-ತಾನು-ಸಾರ
ಯೋಗ್ಯ-ತಾನು-ಸಾರ-ವಾಗಿ
ಯೋಗ್ಯ-ತೆಗೆ
ಯೋಗ್ಯ-ನಲ್ಲ-ವೆಂದು
ಯೋಗ್ಯ-ವಲ್ಲ
ಯೋಗ್ಯ-ವಾಗಿ
ಯೋಗ್ಯ-ವಾದ
ಯೋಚಿಸುತ್ತ
ಯೋಜನ
ಯೋಜನಾ-ಯಾಮಾಂ
ಯೋತಸೀ-ಪುಷ್ಪ-ಮಾಲ್ಯೇನ
ಯೋಧ್ಯಾ
ಯೋನಿ-ಗಳು
ಯೋನಿಮಾಶ್ರಿತ್ಯ
ಯೋನಿ-ಮಾ-ಸಾದ್ಯ
ಯೋನಿ-ಯಲ್ಲಿ
ಯೋನಿ-ಯಿಂದ
ಯೋನೌ
ಯೋಷಿತಾ-ಮಪಿ
ಯೋಷಿದ್ವಿನಿಮಯೋ
ಯೋಷಿನ್ಮಾಘತ್ರಯೋ-ದಶ್ಯಾಂ
ಯೋಸೌ
ಯೋಽಪೂಪಾನ್
ಯೌ
ಯೌವನ-ವನ್ನು
ರಂಗಕ್ಷೇತ್ರಂ
ರಂಗಕ್ಷೇತ್ರ-ವನ್ನು
ರಂಗರಾಡ್
ರಂಗವಲೀಭಿಃ
ರಂಗ-ವಲ್ಲಿ-ಯಿಂದ
ರಂಗೋಲೆ-ಯಿಂದ
ರಂಜ-ಯಂತಿ
ರಂತು-ಕಾಮೋ-ಽಪ್ಸರೋ-ಗಣೈಃ
ರಂತುಮುನಾ
ರಂತುಮುನಾ-ತಿ-ಕಾ-ಮಮ್
ರಂಭಾ-ಫಲಾನಿ
ರಕಿ-ಸುವಂತಹ
ರಕ್ತ
ರಕ್ತ-ಚಂದನ-ವಾಸಸಾ
ರಕ್ತದ
ರಕ್ತ-ಮಾಂಸ-ಗಳ
ರಕ್ತ-ಮಾಂಸ-ಗಳು
ರಕ್ತ-ಮಾಲ್ಯೈ
ರಕ್ತ-ಲೋ-ಚನಃ
ರಕ್ತ-ವನ್ನು
ರಕ್ತ-ವೃಷ್ಟಿಸ್ತದಾ
ರಕ್ತೈಃ
ರಕ್ಷ
ರಕ್ಷ-ಕರು
ರಕ್ಷಣೆ-ಗಾಗಿ
ರಕ್ಷ-ಮಮ್
ರಕ್ಷಿತಂ
ರಕ್ಷಿಸಲ್ಪಡುತ್ತದೆ
ರಕ್ಷಿಸಲ್ಪಡುತ್ತಿದೆಯೋ
ರಕ್ಷಿಸಲ್ಪಡುತ್ತಿದ್ದರು
ರಕ್ಷಿಸು
ರಕ್ಷಿಸು-ವುದ-ರಲ್ಲಿಯೇ
ರಘೋತ್ತಮರ
ರಚಿತ-ವಾದ
ರಜ
ರಜಸ್ತಮೋ-ಗಳಿಂದ
ರಜೇಷು
ರಜೋ-ಗುಣ-ದಿಂದ
ರಜೋಮಲ
ರಜೋಮಿಶ್ರ
ರಟಿಂತ್ಯಾಪೋ
ರಣತೋ
ರಣ-ರಂಗೇ
ರಣ-ಹೇತವಃ
ರತಃ
ರತಾಮ್
ರತಿಃ
ರತೋಸ್ಮ-ಹಮ್
ರಥ
ರಥಂ
ರಥಕ್ರಾಂತ
ರಥಕ್ರಾಂತ-ಸವ-ನಸ್ಯ
ರಥ-ಗಳು
ರಥ-ದಲ್ಲಿ
ರಥ-ಮಾಲಿಖ್ಯ
ರಥ-ವನ್ನು
ರಥ-ಸಪ್ತಮಿ
ರಥ-ಸಪ್ತ-ಮಿ-ಯಂದು
ರಥ-ಸಪ್ತ-ಮಿ-ಯಲ್ಲಿ
ರಥ-ಸಪ್ತಮೀ
ರಥ-ಸಪ್ತಮ್ಯಾಂ
ರಥಾನ್
ರಮಣೀಪ್ರಾಣೌ
ರಮಣೀಯ
ರಮಣೀಯ-ವಾದ
ರಮಮಾಣಾ
ರಮಾ
ರಮಾ-ದೇವಿಯರು
ರಮಾ-ಪತಿಯ
ರಮಾ-ಪತೇಃ
ರಮಾಬ್ರಹ್ಮಾ-ದಯಸ್ತತ್ರ
ರಮಾಬ್ರಹ್ಮಾ-ದಯೋ-ಽಖಿಲಾಃ
ರಮಾಬ್ರಹ್ಮಾದಿ
ರಮಾ-ವಲ್ಲ-ಭ-ನನ್ನು
ರಮಿ-ಸಲು
ರಮಿಸಿ
ರಮಿ-ಸುತ್ತಿದ್ದ
ರಮಿ-ಸುತ್ತಿದ್ದೆ
ರಮಿ-ಸು-ವುದು
ರಮ್ಯ-ವಾದ
ರಮ್ಯೇ
ರಲ್ಲಿ
ರವಿಃ
ರವಿಗ್ರಹ-ಶತಾಧಿಕೇ
ರವಿಣಾ
ರವಿರ್ಯಥಾ
ರವಿ-ವಾರೇ
ರವಿ-ವಾರೇಣ
ರವಿ-ವಾಸರೇ
ರವಿಸೂನುರುಗ್ರಃ
ರವೌ
ರವ್ಯ
ರಸ
ರಸಂ
ರಸಃ
ರಸ-ಗಳು
ರಸ-ದಲ್ಲಿ
ರಸ-ವನ್ನು
ರಸ-ವನ್ನೇ
ರಸವಿಕ್ರಯಿಕೋ-ಽಭ-ವಮ್
ರಸಾತ್ಮಾ
ರಸಿ-ಕತೆಯೇ
ರಸ್ಯಾಸೀನ್ಮಾ
ರಹರ್ನಿಶಂ
ರಹಸ್ಯ
ರಹಸ್ಯ-ಗಳು
ರಹಸ್ಯದ
ರಹಸ್ಯ-ದಲ್ಲಿ
ರಹಸ್ಯ-ದಲ್ಲಿಯ
ರಹಸ್ಯ-ವನ್ನರಿತು
ರಹಸ್ಯ-ವಾದ
ರಹಸ್ಯ-ವಾದುದ-ರಿಂದ
ರಹಿತ-ನಾಗಿಯೂ
ರಹಿತ-ನಾದ
ರಹಿ-ತನು
ರಹಿತನೂ
ರಹಿತ-ರಾದ
ರಹಿತ-ವಾಗಿ-ರುವುದೋ
ರಾಕಾಯಾಂ
ರಾಕ್ಷಸಃ
ರಾಕ್ಷಸ-ನನ್ನು
ರಾಕ್ಷಸನೇ
ರಾಕ್ಷಸರು
ರಾಕ್ಷಸರೇ
ರಾಕ್ಷಸಾ
ರಾಜ
ರಾಜದ್ವಾರಿ
ರಾಜನ
ರಾಜ-ನನ್ನು
ರಾಜ-ನಾಗಿದ್ದನು
ರಾಜ-ನಾದ
ರಾಜ-ನಿಂದ
ರಾಜ-ನಿಗೆ
ರಾಜ-ನಿಲ್ಲದೆ
ರಾಜನು
ರಾಜನೇ
ರಾಜ-ನೊಡನೆ
ರಾಜನ್
ರಾಜ-ಭತ್ಯ-ರನ್ನೂ
ರಾಜ-ಭತ್ಯ-ರಿಗೆ
ರಾಜ-ಭತ್ಯಾ
ರಾಜ-ಮಂದಿ-ರಮ್
ರಾಜ-ಮಂದಿ-ರ-ವಮ್
ರಾಜ-ರಾಗಿದ್ದ
ರಾಜ-ರಾದ
ರಾಜರು
ರಾಜ-ಸರು
ರಾಜ-ಸ-ರೆಂದು
ರಾಜ-ಸಸ್ವಭಾ-ವದ
ರಾಜಸಾಃ
ರಾಜಸೀ
ರಾಜ-ಸೂಯ
ರಾಜ-ಸೂಯ-ಅಶ್ವಮೇಧ
ರಾಜ-ಸೂಯಾಶ್ವಮೇಧಾನಾಂ
ರಾಜ-ಸೂಯಾಶ್ವಮೇಧೇಭ್ಯೋ
ರಾಜ-ಸೇವ-ಕರು
ರಾಜಾ
ರಾಜಾ-ಮಾತ್ಯ
ರಾಜೀವ-ಲೋ-ಚನಃ
ರಾಜೇಂದ್ರ
ರಾಜ್ಞಃ
ರಾಜ್ಞಾ
ರಾಜ್ಞೋ
ರಾಜ್ಯಂ
ರಾಜ್ಯ-ಗಳನ್ನೂ
ರಾಜ್ಯದ
ರಾಜ್ಯ-ದಲ್ಲಿ
ರಾಜ್ಯದಲ್ಲಿದ್ದ
ರಾಜ್ಯ-ಭಾರ
ರಾಜ್ಯ-ವನ್ನು
ರಾಜ್ಯವು
ರಾಜ್ಯಾಂಗೈರಿವ
ರಾಜ್ಯೇ
ರಾತ್ಮನಃ
ರಾತ್ರಿ
ರಾತ್ರಿಂ
ರಾತ್ರಿಯ
ರಾತ್ರಿ-ಯನ್ನು
ರಾತ್ರಿ-ಯಲ್ಲಿ
ರಾತ್ರಿ-ಹ-ಗಲು
ರಾತ್ರಿ-ಹಗಲೂ
ರಾತ್ರೌ
ರಾಮ-ಕಥಾ-ಸಕ್ತಾ
ರಾಮಾಚಾರ್ಯರ
ರಾಮಾಭಿರಾವೃತಃ
ರಾಮಾ-ಯಣ-ದಲ್ಲಿ
ರಾಮಾ-ಯಣೇ
ರಾಮೇಶ್ವ-ರಾದಿ
ರಾಮೋ
ರಾಶಿ-ಗಳಲ್ಲಿ
ರಾಶಿಗೆ
ರಾಶಿ-ಯನ್ನು
ರಾಶಿ-ಯಲ್ಲಿ
ರಾಷ್ಟಚ್ಛತೋ-ಮೃತಾಃ
ರಾಷ್ಟ್ರ
ರಾಹು-ವಿ-ನಿಂದ
ರಾಹೋರ್ಮುಕ್ತಮಿವೋಡುಪಮ್
ರಿ
ರಿಷಟ್ಕ-ವರ್ಗಮ್
ರೀತಿ
ರೀತಿ-ಯಲ್ಲಿ
ರೀತಿ-ಯಲ್ಲಿಯೂ
ರೀತಿ-ಯಾದ
ರೀತಿ-ಯಿಂದ
ರೀತಿ-ಯಿಂದಲೂ
ರು
ರುಂ
ರುಕ್ಕ-ರಥಾತ್ಮಜಃ
ರುಕ್ಮ-ರಥನ
ರುಚಿಕರ-ವಾದ
ರುಚಿ-ಗೋಸ್ಕರ
ರುಚಿ-ಯನ್ನು
ರುಚಿ-ಯಿಲ್ಲ
ರುಚಿ-ಯಿಲ್ಲದ
ರುಚ್ಯರ್ಥಮುದಿತಾ
ರುಜೋ
ರುಣೋದಯೇ
ರುದ್ರ
ರುದ್ರ-ದೇವರ
ರುದ್ರ-ದೇವರು
ರುದ್ರ-ದೇವರೂ
ರುದ್ರ-ದೇವರೇ
ರುದ್ರರು
ರುದ್ರಾ
ರುದ್ರಾ-ಯಾವೋ-ಚನಂಗಜ
ರುದ್ರೋಸಿ
ರುಧಿರೋದಯೇ
ರುರುದುಃ
ರುರುಧುಸ್ತೂರ್ಣಮಾಗತ್ಯ
ರುರೋದ
ರುರೋದೋಚ್ಚೈರಹರ್ನಿ-ಶ-ನಮ್
ರೂಪ
ರೂಪಕ್ಕೆ
ರೂಪ-ಗಳನ್ನುಳ್ಳ
ರೂಪ-ಗಳು
ರೂಪ-ದಿಂದ
ರೂಪ-ವಂತ-ರಾದ
ರೂಪ-ವನ್ನು
ರೂಪ-ವಾಗಿ
ರೂಪವು
ರೂಪ-ವುಳ್ಳ
ರೂಪಾಢ್ಯೌ
ರೂಪಾತ್ಮಾ
ರೂಪಿಣೀ
ರೆಂಬ
ರೆಂಬೆ-ಗಳು
ರೇಮಾತೇ
ರೇಮೇ
ರೇವಾ
ರೇವಾಯಾಂ
ರೈಃ
ರೋಗ-ಗಳಿಗೆ
ರೋಗ-ಗಳೂ
ರೋಗ-ದಿಂದ
ರೋಗ-ದಿಂದಿ-ರುವ
ರೋಗವು
ರೋಗಶಮನಾರ್ಥ-ವಾಗಿ
ರೋಗಾಃ
ರೋಗಾಣಾಂ
ರೋಚಿಷ್ಮತಿಗೂ
ರೋಚಿಷ್ಮ-ತಿಗೆ
ರೋಚಿಷ್ಮತಿಯ
ರೋಚಿಷ್ಮತಿಯು
ರೋಚಿಷ್ಮತೀ
ರೋದನ-ವನ್ನು
ರೋಮಸಂಖ್ಯಾ
ರೋಮಸಂಖ್ಯಾ-ಕ-ವಾದ
ರೋರಂತಿಕ-ಮೇವ
ರೋಷ-ತಾಮ್ರಾಕ್ಷೀ
ರೋಷಾದ್ವಿ
ರೋಷಿ
ರೋಷೇಣ
ರೋಸಿತಾ
ರೌದ್ರ
ರೌದ್ರೀಂ
ರೌದ್ರೇ
ರೌರವ
ರೌರವಂ
ರೌರವೇ
ರ್ಗೋಭೂಗಜಾದ್ಯೈರಪಿ
ರ್ಥಂ
ರ್ಧೂಪೈರ್ದೀಪೈರ್ಮನೋಹ್ರೈಃ
ರ್ಬ್ರಹ್ಮಾ
ರ್ಭುಕ್ತಿಂ
ರ್ಮನಸೋ
ರ್ಮಾಘೇ
ರ್ಮುಚ್ಯತೇ
ರ್ಯೋನೇರ್ದುಃಖ-ಕಾರ-ಣಾತ್
ರ್ಲೋಕೇ
ರ್ಹರೌ
ಲಕ್ಷ
ಲಕ್ಷಂ
ಲಕ್ಷಣ
ಲಕ್ಷಣಂ
ಲಕ್ಷಣ-ಗಳನ್ನು
ಲಕ್ಷಣ-ಸಂವೇತ್ತಾ
ಲಕ್ಷ-ದಷ್ಟು
ಲಕ್ಷಾಣಿ
ಲಕ್ಷ್ಮಿ
ಲಕ್ಷ್ಮೀ
ಲಕ್ಷ್ಮೀ-ರಮಣ-ನಾದ
ಲಕ್ಷ್ಮೀ-ರುದಾ-ಹೃತಾ
ಲಕ್ಷ್ಮೀಸ್ಥೈರ್ಯ-ಮವಾಪ್ನು-ಯಾತ್
ಲಕ್ಷ್ಯ
ಲಗ್ನ
ಲಗ್ನ-ಮಾಡಿ-ದನು
ಲಗ್ನ-ವಾ-ಯಿತು
ಲಗ್ರಾಹಾತ್
ಲಜ್ಜಿ-ತ-ನಾದ
ಲಬ್ಧ
ಲಬ್ಧಾ
ಲಬ್ಧ್ಬಾ
ಲಭತೇ
ಲಭಿ-ಸಿತು
ಲಭಿ-ಸಿದೆ
ಲಭಿ-ಸಿ-ರುವ
ಲಭಿಸುತ್ತದೆ
ಲಭಿಸುತ್ತ-ದೆಯೋ
ಲಭಿ-ಸುತ್ತವೆ
ಲಭಿ-ಸುತ್ತವೆ-ಯೆಂದು
ಲಭಿ-ಸು-ವುದು
ಲಭಿ-ಸುವುದೋ
ಲಭೇತ್
ಲಭೇಯ
ಲಭ್ಯಃ
ಲಭ್ಯತೇ
ಲಭ್ಯ-ವಾಗುತ್ತದೆ
ಲಭ್ಯ-ವಾಗುತ್ತವೆ
ಲಭ್ಯಾಃ
ಲಲಾಟೇ
ಲಶುನಂ
ಲಾಭ
ಲಾಭ-ಕರ-ವಾ-ದುದು
ಲಾಭಾ-ಲಾಭೌ
ಲಾಭೋನ
ಲಾಲಿಸಿ
ಲಾಲಿ-ಸಿದ
ಲಾಲಿಸು
ಲಿಂಗಂ
ಲಿಂಗಕ್ಕೆ
ಲಿಂಗ-ದೇಹವು
ಲಿಂಗಭಸ್ಯ-ಜಟಾಧಾರೀ
ಲಿಂಗಭೂಯಸ್ಕಾಧಿಕ-ರಣ
ಲಿಂಗೇ
ಲಿಖಿತಂ
ಲಿಖಿ-ತಾನಿ
ಲಿಖೇತ್
ಲಿನಾಸ್ತೇ
ಲಿಪ್ಯತಿ
ಲಿಪ್ಯತೇ
ಲೀನ-ವಾ-ಯಿತು
ಲೀನಾಂಶ್ಚಲಿ-ತಾನ್
ಲೀನಾಸ್ಯುಃ
ಲುಪ್ತ-ಪಿಂಡಾಃ
ಲುಪ್ತೇ
ಲುಬ್ದ
ಲುಬ್ದ-ಕೋಽಭೂನ್ಮಹಾಕ್ರೂರೋ
ಲುಬ್ಧ
ಲೂಖಲೇ
ಲೆಕ್ಕ
ಲೆಕ್ಕವೇ
ಲೇನ
ಲೇಪನ
ಲೇಪ-ವಾಗು-ವಂತೆ
ಲೇಪ-ವಾಗು-ವುದಿಲ್ಲ
ಲೇಪವು
ಲೇಪಿ-ತಮ್
ಲೇಪಿಸಿ-ಕೊಂಡನು
ಲೇಪಿಸಿ-ಕೊಂಡು
ಲೇಶ-ಮಾತ್ರ
ಲೇಶ-ಮಾತ್ರ-ಕಮ್
ಲೇಶ-ಮಾತ್ರವೂ
ಲೇಶವೂ
ಲೇಶೋಯಂ
ಲೊಕೇ
ಲೋಕಂ
ಲೋಕಃ
ಲೋಕ-ಕಂಟ-ಕ-ರಾಗಿ
ಲೋಕ-ಕರ್ಪಟಾಃ
ಲೋಕಕ್ಕೆ
ಲೋಕ-ಗರ್ಹಿತಾಃ
ಲೋಕ-ಗರ್ಹಿತೌ
ಲೋಕ-ಗಳನ್ನೂ
ಲೋಕ-ಗಳಲ್ಲಿ
ಲೋಕ-ಗಳಲ್ಲಿ-ರುವ
ಲೋಕ-ಗಳಿಗೆ
ಲೋಕ-ಗೃಹೀತಾರೌ
ಲೋಕದ
ಲೋಕ-ದಲ್ಲಿ
ಲೋಕ-ದಲ್ಲಿನ
ಲೋಕ-ಪತಯೋ
ಲೋಕ-ಪಾಲಾಶ್ಚ
ಲೋಕ-ಪಿತಾ
ಲೋಕ-ಪಿತಾ-ಮಹಃ
ಲೋಕ-ಪಿತಾ-ಮಹ-ರಾದ
ಲೋಕ-ಯಾತ್ರಾಪ್ರಸಂಗೇನ
ಲೋಕ-ವನ್ನು
ಲೋಕ-ವನ್ನೂ
ಲೋಕ-ವಲ್ಲ
ಲೋಕ-ವಿದ್ವಿಷ್ಟಂ
ಲೋಕ-ವಿಶ್ರುತಃ
ಲೋಕವು
ಲೋಕವೂ
ಲೋಕಸ್ತ
ಲೋಕಸ್ಯ
ಲೋಕಾ
ಲೋಕಾಂಶ್ಚ
ಲೋಕಾಃ
ಲೋಕಾ-ನುಗ್ರಹ-ಕಾರಕ
ಲೋಕಾನ್
ಲೋಕಾಸ್ತಗ್ರಾಮಸ್ಥಾ
ಲೋಕೇ
ಲೋಕೇಷು
ಲೋಕೇಸ್ಮಿನ್
ಲೋಕೊ
ಲೋಕೋ
ಲೋಕೋ-ಽಸ್ತಿ
ಲೋಪ-ಗಳನ್ನು
ಲೋಪ-ವಾಗಿದ್ದರೆ
ಲೋಪವೇ-ನಾದರೂ
ಲೋಪಾಮುದ್ರಪ್ರಸಾದೇನ
ಲೋಪಾ-ಮುದ್ರೆಯ
ಲೋಭ-ದಿಂದ
ಲೋಭಾತ್
ಲೋಭಾದ್ವಿಜೋತ್ತಮಾಃ
ಲೋಭೇನ
ಲೋಮಶ
ಲೋಮಶ-ರಿಂದ
ಲೋಮಶರು
ಲೋಮಶಸ್ತಂ
ಲೋಮಶೋ
ಲೋಹ-ಗಳು
ಲೌಲ್ಯಂ
ವ
ವಂಗ-ದೇಶಕ್ಕೆ
ವಂಗ-ದೇಶ-ದಲ್ಲಿ
ವಂಗ-ದೇಶ-ಪತೀ
ವಂಗ-ದೇಶೇ
ವಂಚ-ಯಿತ್ವಾ
ವಂಚಿಸಿ
ವಂದನೆ-ಗಳು
ವಂದೇ
ವಂದ್ಯಮಾನಶ್ಚ
ವಂಧ್ಯಾ
ವಂಶ-ಗಳನ್ನು
ವಂಶ-ದಲ್ಲಿ
ವಂಶದ-ವರ
ವಂಶೇ
ವಂಶ್ಯೈಃ
ವಃ
ವಕ್ತಾ
ವಕ್ತಾರಂ
ವಕ್ತಿ
ವಕ್ತುಂ
ವಕ್ತುಮರ್ಹಸಿ
ವಕ್ತೄನ್
ವಕ್ಷೇ
ವಕ್ಷ್ಯಾಮಃ
ವಕ್ಷ್ಯಾಮಿ
ವಚಃ
ವಚನಂ
ವಚನ-ಗಳನ್ನು
ವಚ-ನವೂ
ವಚಸಾ
ವಚೋ
ವಜ್ರ-ದಿಂದಲೂ
ವಜ್ರ-ವೃಕ್ಷೋ
ವಜ್ರಾಯು-ಧವು
ವಜ್ರೋ
ವಟ-ಜನ್ಮ-ಗತಂ
ವಟತಾಂ
ವಟಪಾದಪೇ
ವಟ-ಮೂಲೇ
ವಟ-ವನೇ
ವಟ-ವೃಕ್ಷದ
ವಟ-ವೃಕ್ಷ-ದಲ್ಲಿ
ವಟ-ವೃಕ್ಷ-ದಲ್ಲಿ-ರುತ್ತೇನೆ
ವಟ-ವೃಕ್ಷ-ದಲ್ಲಿ-ರುತ್ತೇವೆ
ವಟ-ವೃಕ್ಷ-ದ-ವರೆವಿಗೂ
ವಟ-ವೃಕ್ಷ-ವಾಗಿದ್ದ
ವಟಸ್ಯಾಗ್ರೇ
ವಟಾತ್ತ
ವಟೇ
ವಟೇ-ಽಸ್ಮಿನ್
ವಟೋ
ವಟೋಪಿ
ವಟೋಯಂ
ವಟೌ
ವಣಿಗ್ಭ್ಯೋ
ವತ್ಸಯುತಾಂ
ವತ್ಸರಾ
ವತ್ಸರಾಖ್ಯಸ್ಯ
ವದ
ವದಂತಿ
ವದಂತ್ಯನ್ಯೇ
ವದತ್ಯೇವಂ
ವದಾನಘ
ವದಾಮ್ಯ-ಹಮ್
ವದೇತ್
ವದೇತ್ಕರ್ಮ
ವದ್ಯತ್ವಂ
ವಧೆ-ಮಾಡಿ-ದ-ವರು
ವನಂ
ವನಕ್ಕೆ
ವನ-ದಲ್ಲಿ
ವನ-ದಲ್ಲಿದ್ದ
ವನ-ವಾಸಿನಾಂ
ವನವು
ವನಸ್ಥೋ
ವನಸ್ಥೌ
ವನಸ್ಪತಿ-ಗಳೂ
ವನೇ
ವನೇ-ಚ-ರಾಶ್ಚೌರ್ಯ-ಕರ್ಮ-ನಿರತಾಃ
ವನೇರ್ಬು-ದಮ್
ವನೇ-ಽಸ್ಮಿನ್
ವನ್ನು
ವನ್ನೂ
ವಯಂ
ವಯಮ್
ವಯಸಾ
ವಯಸ್ಥಾ
ವಯಸ್ಸಾ-ದಾಗ
ವಯಸ್ಸಿನ
ವಯಸ್ಸಿ-ನಲ್ಲಿ
ವಯೋ
ವರಂ
ವರ-ಯಿತ್ವಾ
ವರ-ವನ್ನು
ವರಾ
ವರಾ-ಹಾಂಶ್ಚ
ವರುಣ
ವರು-ಣನೂ
ವರುಣೋ
ವರೆವಿಗೂ
ವರೋ
ವರ್ಚಿಷ್ಮಾನಿತಿ
ವರ್ಚಿಷ್ಮಾನ್
ವರ್ಜ-ಯೇನ್ಮೃತ್ತಿಕಾಂ
ವರ್ಜಿತರೂ
ವರ್ಣ
ವರ್ಣ-ಧರ್ಮ-ವಿ-ವರ್ಜಿತಃ
ವರ್ಣ-ನೆ-ಮಾಡುವ
ವರ್ಣ-ಯಾಮಿ
ವರ್ಣ-ಯಿತುಂ
ವರ್ಣ-ಸಂಕರ-ಕಾರಕೌ
ವರ್ಣ-ಸಂಕರ-ಕಾರಿ-ಣಮ್
ವರ್ಣಾದ್ಗು
ವರ್ಣಾನಾಂ
ವರ್ಣಾಶ್ರಮ
ವರ್ಣಾಶ್ರಮಕ್ಕೆ
ವರ್ಣಾಶ್ರಮ-ಗಳಿಗೆ
ವರ್ಣಾಶ್ರಮ-ವಿ-ಹೀನೊಪಿ
ವರ್ಣಾಶ್ರಮೋಚಿತ-ವಾದ
ವರ್ಣಾಶ್ರಮೋಚಿತೇಷ್ಟೇವ
ವರ್ಣಿಸ-ಬಲ್ಲರು
ವರ್ಣಿಸಲಿ
ವರ್ಣಿ-ಸಲು
ವರ್ಣಿ-ಸಿದೆವು
ವರ್ತಕ-ರಿಗೆ
ವರ್ತತೇ
ವರ್ತತೇ-ಽದ್ಯಾತಿ
ವರ್ತನೆ-ಯಿಂದ
ವರ್ತಸೇ
ವರ್ತಿ
ವರ್ತಿ-ತೈಲಂ
ವರ್ತಿ-ಸಿ-ದರೆ
ವರ್ಧತೇ
ವರ್ಧತೇ-ಽನಿ-ಶಮ್
ವರ್ಧನೀ-ರಿಮಾ
ವರ್ಧಯಂತಃ
ವರ್ಧಿ-ತಾನಿ
ವರ್ಷ
ವರ್ಷ-ಗಳ
ವರ್ಷ-ಗಳ-ಕಾಲ
ವರ್ಷ-ಗಳನ್ನು
ವರ್ಷ-ಗಳಾದರೂ
ವರ್ಷ-ಗಳಿಂದಲೂ
ವರ್ಷ-ಗಳು
ವರ್ಷ-ಗಳೂ
ವರ್ಷದ
ವರ್ಷ-ದಲ್ಲಿ
ವರ್ಷ-ವಾಯುಷ್ಯಂ
ವರ್ಷವೂ
ವರ್ಷ-ಶತಂ
ವರ್ಷ-ಶತೈರಪಿ
ವರ್ಷಾಂತೇ
ವರ್ಷಾಣಿ
ವರ್ಷೇಷು
ವಲ್ಮೀಕಂ
ವಲ್ಲಭಾ
ವಲ್ಲೀಕೇ
ವವೃಧೆ
ವಶಂ
ವಶಂಕರೋ
ವಶ-ದಲ್ಲಿಟ್ಟು
ವಶ-ದಲ್ಲಿದ್ದ
ವಶ-ದಲ್ಲಿದ್ದಳು
ವಶ-ಪಡಿಸಿ-ಕೊಳ್ಳ-ಬಹುದೋ
ವಶ-ಮಾಡಿ
ವಶೀಕ-ರಣ-ವಿದ್ಯೆ
ವಶೇ
ವಶ್ಯೋಽದಾತ್ಕನ್ಯಕಾಂ
ವಸಂತಋತು-ವಿನ
ವಸಂತಋತು-ವಿ-ನಲ್ಲಿ
ವಸಂತ-ದಂತೆ
ವಸಂತಯಾಗದ
ವಸಂತಯಾಗ-ಫ-ಲದಾ
ವಸಂತೇ
ವಸಂತ್ಯಾಶ್ರಯ-ಕಾಂಕ್ಷಯಾ
ವಸತಿ
ವಸತ್ಯ
ವಸತ್ಯೇವ
ವಸವೊಪ್ರೌ
ವಸಾಮ
ವಸಾಮೋಽತ್ರ
ವಸಾಮ್ಯ-ಹಮ್
ವಸು-ಗಳು
ವಸು-ಮತಿ
ವಸು-ಮತಿಯು
ವಸು-ಮತಿರ್ಬಲೀ
ವಸುಷೇಣ
ವಸುಷೇಣ-ನೆಂಬ
ವಸೂಲಾಡುವುದ-ರಲ್ಲಿ
ವಸೇಚ್ಚಿ-ರಮ್
ವಸೇದ್ಯಸ್ತು
ವಸೇನ್ನಿತ್ಯಂ
ವಸ್ತು
ವಸ್ತು-ಗಳನ್ನು
ವಸ್ತು-ಗಳನ್ನೂ
ವಸ್ತು-ಗಳು
ವಸ್ತು-ಗಳೂ
ವಸ್ತು-ವನ್ನು
ವಸ್ತುವು
ವಸ್ತುವೇ
ವಸ್ತ್ರ
ವಸ್ತ್ರಂ
ವಸ್ತ್ರ-ಗಳನ್ನು
ವಸ್ತ್ರ-ಗಳಿಂದ
ವಸ್ತ್ರ-ಗಳು
ವಸ್ತ್ರ-ದಲ್ಲಿ
ವಸ್ತ್ರ-ದಿಂದ
ವಸ್ತ್ರ-ದಿಂದಲೂ
ವಸ್ತ್ರ-ಧರಿಸಿ
ವಸ್ತ್ರ-ಪರೀಧಾನಂ
ವಸ್ತ್ರ-ವನ್ನು
ವಸ್ತ್ರ-ವನ್ನುಟ್ಟು
ವಸ್ತ್ರೇಣ
ವಹಾರಾಚ್ಚ
ವಹ್ನಿಃ
ವಹ್ನಿಜ್ವಾಲಾಸಮಾ-ಕುಲಮ್
ವಹ್ನಿನಾ
ವಹ್ನಿ-ಪಾಲಿ-ತಾಮ್
ವಹ್ನಿಮಾ-ದಾತು-ಕಾಮಾನಾಂ
ವಹ್ನಿರಾಶೌ
ವಾ
ವಾಂಛಂತಿ
ವಾಂಛಾಂ
ವಾಕ್
ವಾಕ್ಪಟುತ್ವ
ವಾಕ್ಯ
ವಾಕ್ಯಂ
ವಾಕ್ಯ-ಗಳಿಂದ
ವಾಕ್ಯ-ಮಥಾಬ್ರುವನ್
ವಾಕ್ಯ-ಮಬ್ರವೀತ್
ವಾಕ್ಯ-ಮುದಾ-ಹ-ರಮ್
ವಾಕ್ಯ-ಮು-ವಾಚ
ವಾಕ್ಯ-ವಿರಾ-ಮಾಂತೇ
ವಾಕ್ಯಾಂತೇ
ವಾಗ-ಮೃತಂ
ವಾಗ-ಮೃತೆಃ
ವಾಗಿ
ವಾಗು-ರಿಕೈರ್ಬಹು-ಭಿಶ್ಚ
ವಾಚಸ್ಪತಿ-ಯೆಂಬು-ವನ
ವಾಚಸ್ಪತೇಸ್ತಥಾ
ವಾಚಾ
ವಾಚಾಲೋ
ವಾಚಿಕ
ವಾಚ್ಯ-ಮಾನಂ
ವಾಚ್ಯಮಾನಮಿಮಾಂ
ವಾಜಿ-ನಾಮ್ನಸ್ತನೂದ್ಭವಃ
ವಾಜಿ-ಯೆಂಬು-ವನ
ವಾಣ್ಯಾ
ವಾತಂ
ವಾತ-ರೋ-ಗ-ದಿಂದ
ವಾತ್ಸಲ್ಯ-ದಿಂದ
ವಾತ್ಸಲ್ಯ-ದಿಂದಲೂ
ವಾತ್ಸಾ-ಯನ
ವಾತ್ಸಾ-ಯನ-ಕುಲೋತ್ಪನ್ನೋ
ವಾಥ
ವಾದ
ವಾದ-ವಿ-ವಾದ-ಗಳೂ
ವಾದ-ವಿ-ವಾದ-ಗಳೆಲ್ಲಿ
ವಾದ-ಸ-ರಣಿ-ಯನ್ನು
ವಾದಿಷು
ವಾದೀ
ವಾದುದು
ವಾಧ್ರ
ವಾನಪ್ರಸ್ಥ
ವಾನಪ್ರಸ್ಥ-ರಿಗೂ
ವಾಪಸ್ಸು
ವಾಪಿ
ವಾಮ-ಗಾತ್ರಾಣಿ
ವಾಮ-ದೇವ
ವಾಮ-ದೇವಂ
ವಾಮ-ದೇವ-ಭಾಷಿತಂ
ವಾಮ-ದೇವ-ಮಥಾಬ್ರವೀತ್
ವಾಮ-ದೇವ-ಮುಖೇರಿ-ತ-ನಮ್
ವಾಮ-ದೇವರ
ವಾಮ-ದೇವ-ರಿಂದ
ವಾಮ-ದೇವ-ರಿಗೆ
ವಾಮ-ದೇವರು
ವಾಮ-ದೇ-ವಸ್ತು
ವಾಮ-ದೇವಸ್ಯ
ವಾಮ-ದೇವಾತ್ಸೋಯಮದ್ಭುತಪೂರುಷಃ
ವಾಮ-ದೇವೋ
ವಾಮ-ದೇವೋಪಿ
ವಾಮಪ್ರಕೋಷ್ಠೇ
ವಾಯು
ವಾಯುಂ
ವಾಯುಃ
ವಾಯು-ಜೀವೋತ್ತಮತ್ವ
ವಾಯು-ತತ್ವ-ದಲ್ಲಿ
ವಾಯು-ದೇವರ
ವಾಯು-ದೇವ-ರಿಂದ
ವಾಯು-ದೇವ-ರಿಗೆ
ವಾಯು-ದೇವರು
ವಾಯು-ದೇವರೂ
ವಾಯು-ಪುರಾ-ಣ-ದಲ್ಲಿ
ವಾಯು-ಪುರಾ-ಣಾಂತರ್ಗತ
ವಾಯು-ಪುರಾ-ಣಾಂತರ್ಗತ-ವಾದ
ವಾಯು-ಪುರಾಣೇ
ವಾಯುಪ್ರ-ಪೀಡಿತಃ
ವಾಯುರ್ವಾಯುಸ್ವ-ರೂಪ-ವಾನ್
ವಾಯು-ವಿಗೆ
ವಾಯುವು
ವಾಯುವೇ
ವಾಯುಶ್ಚ
ವಾಯೋರಿವ
ವಾರ-ಗಳಲ್ಲಿಯೂ
ವಾರ-ಯಿತುಂ
ವಾರಿಣಾ
ವಾರೇ
ವಾರ್ತೆ-ಯನ್ನು
ವಾರ್ಧುಷಿಕಃ
ವಾರ್ಷಿಕ-ಮಾಸ-ಕಮ್
ವಾಸ
ವಾಸಃ
ವಾಸನೆ
ವಾಸ-ನೆ-ನೋ-ಡಲು
ವಾಸ-ಮಾಡಲು
ವಾಸ-ಮಾಡುತ್ತಾರೆ
ವಾಸ-ಮಾಡುತ್ತಾಳೆ
ವಾಸ-ಮಾಡುತ್ತಿದ್ದನು
ವಾಸ-ಮಾಡುತ್ತಿದ್ದೇವೆ
ವಾಸ-ಮಾಡುತ್ತಿರುತ್ತೇ-ನೆಂಬ
ವಾಸ-ಮಾಡುವ
ವಾಸ-ಮಾಡು-ವುದ-ರಿಂದ
ವಾಸಸೀ
ವಾಸಸ್ಥಳ-ವಾಗಿ-ರುತ್ತದೆ
ವಾಸಾರ್ಥಂ
ವಾಸಿ
ವಾಸಿ-ಸುತ್ತದೆ
ವಾಸಿ-ಸುತ್ತಿದ್ದ
ವಾಸಿ-ಸುತ್ತಿದ್ದೇವೆ
ವಾಸಿ-ಸುವ
ವಾಸಿ-ಸುವನು
ವಾಸು-ದೇವ-ಕಥಾಶ್ರಯಃ
ವಾಸೋ
ವಾಸೋ-ದಾನಂ
ವಾಸೋಸ್ಯಾಂಧೇ
ವಾಸ್ಮತ್ಕುಲೇ
ವಾಽ-ಕರೋತ್
ವಾಽತ್ಮನಿ
ವಾಽರಸಿಕಹೃದ್ಯಾಸ್ತೇ
ವಾಽಸ್ಮತ್ಕುಲೇ
ವಿಂಗಡಣೆ
ವಿಂದತಿ
ವಿಕರ್ಮಣಃ
ವಿಕಲ್ಪೋ
ವಿಕಸಿತ
ವಿಕ್ರ-ಯಾತ್
ವಿಕ್ರೀತಂ
ವಿಗತಜ್ವರಃ
ವಿಗ್ರಹ
ವಿಗ್ರಹ-ವನ್ನು
ವಿಚಚಾರ
ವಿಚಾರ-ಗಳನ್ನೂ
ವಿಚಾ-ರಣಾ
ವಿಚಾರ-ದಲ್ಲಿ
ವಿಚಾರ-ಮಾಡದೇ
ವಿಚಾರ-ವನ್ನು
ವಿಚಾರ-ವನ್ನೂ
ವಿಚಾರವಾಡದೇ
ವಿಚಾರ-ವಿಲ್ಲ
ವಿಚಾರವು
ವಿಚಾರವೆಲ್ಲಿ
ವಿಚಾರ್ಯ
ವಿಚಾರ್ಯೈವಮ-ನಿಂದ್ಯ
ವಿಚಿತ್ರ
ವಿಚಿತ್ರ-ರಥ
ವಿಚಿತ್ರ-ರಥನ
ವಿಚಿತ್ರ-ರಥನು
ವಿಚಿತ್ರ-ರಥ-ನೆಂಬ
ವಿಚ್ಚತ್ತಿಯುಂಟಾಗುತ್ತದೆ
ವಿಚ್ಛತ್ತಿರ-ಹಿತ-ವಾದ
ವಿಚ್ಛಿ
ವಿಚ್ಛಿತ್ತಿರ-ಹಿತ-ವಾದ
ವಿಜನೇ
ವಿಜನೋಽ-ಭ-ವತ್
ವಿಜರಾ
ವಿಜಾನೀಥ
ವಿಜಾನೀ-ಯಾತ್
ವಿಜ್ಞಾ-ಪನೆ
ವಿಜ್ಞಾ-ಪಿತ-ರಾದ
ವಿಜ್ಞಾ-ಪಿತಾಸ್ತೈಸ್ತು
ವಿಜ್ಞಾ-ಪಿತೋ
ವಿಜ್ಞಾಪ್ಯ
ವಿಜ್ಞಾಯ
ವಿಟಿಪಾಃ
ವಿಟಿಪಾಸ್ತಥಾ
ವಿಡ್
ವಿಡ್-ವರಾಹೊ
ವಿತ್ತಂ
ವಿದಧಾತಿ
ವಿದಧಾತ್ರಿ-ಹರೇಃ
ವಿದರ್ಭ
ವಿದರ್ಭ-ದೇಶ-ವನ್ನು
ವಿದರ್ಭ-ನೆಂಬ
ವಿದರ್ಭಾನ್ಪರ್ಯಪಾಲ-ಯತ್
ವಿದರ್ಭೋ
ವಿದ-ಶನಃ
ವಿದುಃ
ವಿದುರ್ಬುಧಾಃ
ವಿದೇಹ-ರಾಜ-ನಿಗೆ
ವಿದೇಹ-ರಾಜನು
ವಿದೇಹಾ-ಧಿ-ಪತಿ-ಸೂರ್ಣಂ
ವಿದೋ
ವಿದ್ಧಿ
ವಿದ್ಧೀದಂ
ವಿದ್ಯತೇ
ವಿದ್ಯಾ
ವಿದ್ಯಾಃ
ವಿದ್ಯಾಢ್ಯೋ
ವಿದ್ಯಾ-ಧ-ರರು
ವಿದ್ಯಾ-ಧರಾಃ
ವಿದ್ಯಾ-ನಿಮಿತ್ತ-ದಿಂದ
ವಿದ್ಯಾನ್ಮಾತ್ರ
ವಿದ್ಯಾಭ್ಯಾಸಕ್ಕಾಗಿ
ವಿದ್ಯಾರ್ಥಿ-ಯಾಗಿ
ವಿದ್ಯಾರ್ಥೀ
ವಿದ್ಯೆ
ವಿದ್ಯೆ-ಗಳನ್ನೂ
ವಿದ್ಯೆ-ಗಳಲ್ಲಿ
ವಿದ್ಯೆಯ
ವಿದ್ಯೆ-ಯನ್ನು
ವಿದ್ಯೆ-ಯಿಂದ
ವಿದ್ಯೋಪ-ಜೀವಿತ್ವಂ
ವಿದ್ಯೋಪ-ಜೀ-ವಿನೀ
ವಿದ್ರವಂತಿ
ವಿದ್ವಾಂಸ-ನಾದ
ವಿದ್ವಾಂಸನು
ವಿದ್ವಾಂಸೋ
ವಿದ್ವಾನ್
ವಿಧ
ವಿಧದ
ವಿಧ-ದಿಂದಲೂ
ವಿಧ-ವಾಗಿವೆ
ವಿಧ-ವಾತ್ವಂ
ವಿಧ-ವಾದ
ವಿಧ-ವೆ-ಯಾಗಲೀ
ವಿಧ-ವೆ-ಯಾಗುವರು
ವಿಧಾನಂ
ವಿಧಾ-ನ-ದಿಂದ
ವಿಧಾ-ನ-ವನ್ನು
ವಿಧಾ-ನೇನ
ವಿಧಾಯ
ವಿಧಾ-ಯ-ಕ-ವಾಗಿ
ವಿಧಿ
ವಿಧಿಃ
ವಿಧಿನಾ
ವಿಧಿ-ನಿಷೇಧ-ಗಳನ್ನೂ
ವಿಧಿ-ಪೂರ್ವ-ಕ-ವಾಗಿ
ವಿಧಿ-ಸಿ-ರುವ
ವಿಧಿ-ಸುತ್ತದೆ
ವಿಧಿ-ಸುತ್ತವೆ
ವಿಧಿ-ಸುತ್ತೇವೆ
ವಿಧೀಯ-ತಾಮ್
ವಿಧೀ-ಯತೇ
ವಿಧೇ
ವಿಧೇಯ
ವಿಧೋಷಾ
ವಿಧೋಷಾ-ವಿತಿವಿಖ್ಯಾತೌ
ವಿಧ್ಯಾತ್ಮರ್ಮಾಣ್ಯಪಿ
ವಿಧ್ಯುಕ್ತ-ವಾದ
ವಿನ-ಯ-ದಿಂದ
ವಿನ-ಯಾತ್ತಂ
ವಿನಶ್ಯತಿ
ವಿನಹ
ವಿನಾ
ವಿನಾ-ಪರಾಧಂ
ವಿನಾ-ಯಕ
ವಿನಾ-ಽತ್ಯರ್ಥಪ್ರಸಾದಃ
ವಿನಿಂದಿತಾಃ
ವಿನಿಯೋಗದ
ವಿನಿಯೋಗ-ವಾಗದೇ
ವಿನಿ-ವರ್ತಿತುಂ
ವಿನಿಸ್ತೀರ್ಯ
ವಿನೀತಂ
ವಿನೀತಃ
ವಿಪತ್ತನ್ನು
ವಿಪತ್ತನ್ನೂ
ವಿಪತ್ತು
ವಿಪದಂ
ವಿಪಶ್ಚಿತಃ
ವಿಪಶ್ಚಿತೇ
ವಿಪಿನೇ
ವಿಪುಲ-ವಾದ
ವಿಪುಲಾನ್
ವಿಪ್ರ
ವಿಪ್ರಂ
ವಿಪ್ರಃ
ವಿಪ್ರ-ಕುಲೇ
ವಿಪ್ರ-ನನ್ನು
ವಿಪ್ರ-ನಿಗೆ
ವಿಪ್ರನು
ವಿಪ್ರ-ಭೋ-ಜನಮ್
ವಿಪ್ರ-ಮಥಾಬ್ರವೀತ್
ವಿಪ್ರ-ರನ್ನು
ವಿಪ್ರ-ವರ್ಯ
ವಿಪ್ರಾಣಾಂ
ವಿಪ್ರಾನ್
ವಿಪ್ರಾಯ
ವಿಪ್ರಾಸ್ತು
ವಿಪ್ರೇ
ವಿಪ್ರೇಂದ್ರ
ವಿಪ್ರೇಂದ್ರಂ
ವಿಪ್ರೇಷು
ವಿಪ್ರೋ
ವಿಫಲ-ವಾಗು-ವುದೇ
ವಿಭವ-ನಾಮ
ವಿಭುಃ
ವಿಭುಶ್ಚ
ವಿಭ್ರಮಸ್ಯ
ವಿಮನೀ-ಕೃತಃ
ವಿಮರ್ಶನೆ-ಯನ್ನಾಗಲೀ
ವಿಮರ್ಶಿಸಿ
ವಿಮರ್ಶಿಸಿ-ದರು
ವಿಮರ್ಶೆ
ವಿಮಾನ-ದಲ್ಲಿ
ವಿಮಾನಮಾರುಹ್ಯ
ವಿಮಾ-ನವು
ವಿಮಾ-ನೇನ
ವಿಮುಂಚಥ
ವಿಮುಕ್ತಃ
ವಿಮುಕ್ತ-ನಾಗುತ್ತಾನೆ
ವಿಮುಕ್ತ-ರಾದ
ವಿಮುಖ-ರಾದ
ವಿಮುಖಾ
ವಿಮುಚ್ಯಾಥ
ವಿಮುಚ್ಯೈವ
ವಿಮೂಢಾತ್ಮಾ
ವಿಮೋಕ್ಷ-ಣಸ್ಯ
ವಿಮೋ-ಚನೆ-ಗಾಗಿ
ವಿರರಾಮ
ವಿರರಾಮಾಥ
ವಿರಹಪ್ರ-ದಮ್
ವಿರಾಜ-ಮಾ-ನರಾ-ಗಿ-ರುವ
ವಿರಾ-ಮಾಂಸ-ಗಳನ್ನು
ವಿರಾಮೇ
ವಿರುದ್ದ
ವಿರುದ್ದಾನಾಂ
ವಿರುದ್ಧ
ವಿರುದ್ಧ-ವಾ-ದುವು-ಗಳು
ವಿರುದ್ಧಸ್ತದ್ವಿಪರ್ಯಯಃ
ವಿರೂಪಂ
ವಿರೋಚತೇ
ವಿರೋಧ
ವಿರೋಧ-ವುಳ್ಳ
ವಿರೋಧಾರ್ಥ-ವನ್ನು
ವಿಲಂಬಿ-ತಮ್
ವಿಲಂಬಿತುಮರ್ಹಸಿ
ವಿಲಯಂ
ವಿಲೋ-ಚನ
ವಿವದಂತಃ
ವಿವರ
ವಿವರ-ಗಳನ್ನು
ವಿವರ-ಗಳನ್ನೂ
ವಿವರ-ಗಳು
ವಿವರ-ಣೆ-ಗಳನ್ನು
ವಿವರ-ಣೆ-ಯನ್ನು
ವಿವರ-ಣೆ-ಯನ್ನೂ
ವಿವರ-ಣೆಯೂ
ವಿವರಿ-ಸಲು
ವಿವರಿಸಲ್ಪಟ್ಟವು
ವಿವರಿಸಲ್ಪಟ್ಟಿದೆ
ವಿವರಿಸಲ್ಪಟ್ಟಿವೆ
ವಿವರಿಸಿ
ವಿವರಿಸಿ-ದರು
ವಿವರಿ-ಸಿರಿ
ವಿವರಿ-ಸುವ
ವಿವರಿಸ್ಪಟ್ಟಿದೆ
ವಿವರ್ಜ-ಯೇತ್
ವಿವಸ್ತ್ರ-ನಾದ
ವಿವಸ್ತ್ರೋಽಂಗನಯಾ
ವಿವಾದ
ವಿವಾಹ
ವಿವಾಹಸ್ಯ
ವಿವಾಹೇನ
ವಿವಿಶುಶ್ಚ
ವಿವೃಣುತೇ
ವಿವೃದ್ಧಿಃ
ವಿವೇಕಕ್ಕೆ
ವಿವ್ಯಾಧ
ವಿಶಾಪ-ಮದದಾತ್ತದಾ
ವಿಶಾಪ-ವನ್ನು
ವಿಶಿಷ್ಯತೇ
ವಿಶೇಷ
ವಿಶೇಷ-ಗಳು
ವಿಶೇಷ-ದಿಂದ
ವಿಶೇಷ-ವಾಗಿ
ವಿಶೇಷ-ವಾಗಿದ್ದುವು
ವಿಶೇಷ-ವಾದ
ವಿಶೇಷಾಂಶ
ವಿಶೇಷಾಂಶ-ಪ-ರಮಾತ್ಮನ
ವಿಶೇಷಾಂಶಬ್ರಹ್ಮ-ಸೂತ್ರ-ಗಳಲ್ಲಿನ
ವಿಶೇಷಾಣಿ
ವಿಶೋ
ವಿಶ್ಚಂ
ವಿಶ್ನೋರ್ದಂಡಪ್ರಣಾವಸ್ಯ
ವಿಶ್ರಂಭಣಾದ್ವಯಮ್
ವಿಶ್ರಬ್ಧಾಶ್ಚ
ವಿಶ್ರಮಿಸಿ
ವಿಶ್ರಾಂತಿ-ಯನ್ನು
ವಿಶ್ರಾಮ
ವಿಶ್ರಾಮಯ
ವಿಶ್ರಾಮ-ಸಂಜ್ಜಿತಃ
ವಿಶ್ರುತಃ
ವಿಶ್ವಜಿತಂ
ವಿಶ್ವಜಿತು
ವಿಶ್ವಜಿತ್
ವಿಶ್ವಜಿದ್ಯಜ್ಞ
ವಿಶ್ವಸೇ
ವಿಶ್ವಾಸ-ವಿಟ್ಟು
ವಿಶ್ವೇಶ್ವರಸ್ಯ
ವಿಷ
ವಿಷ-ಕರ್ಮಪ್ರಯೋಗಿ-ನಮ್
ವಿಷ-ದಂತೆ
ವಿಷ-ಪಾ-ನ-ದಿಂದ
ವಿಷ-ಪೂರಿ-ತ-ವಾದ
ವಿಷಪ್ರಯೋಗ-ಮಾಡು-ವುದು
ವಿಷಯ
ವಿಷ-ಯ-ಗಳ
ವಿಷ-ಯ-ಗಳನ್ನು
ವಿಷ-ಯ-ಗಳು
ವಿಷ-ಯದ
ವಿಷ-ಯ-ದಲ್ಲಿ
ವಿಷ-ಯ-ದಲ್ಲಿ-ಯಂತೂ
ವಿಷ-ಯ-ವನ್ನು
ವಿಷ-ಯವು
ವಿಷ-ಯವೂ
ವಿಷ-ಯ-ಸುಖ-ಗಳ
ವಿಷ-ಯ-ಸುಖ-ಗಳ-ಕಡೆಗೆ
ವಿಷ-ಯ-ಸುಖ-ಗಳಲ್ಲಿ
ವಿಷ-ಯ-ಸುಖ-ಗಳಲ್ಲಿಯೇ
ವಿಷ-ಯ-ಸುಖದ
ವಿಷ-ಯಾಂತ-ರಮ್
ವಿಷ-ಯಾಂತರೇ
ವಿಷ-ಯಾನಲಮಿಚ್ಛತಃ
ವಿಷ-ಯಾನುಭವವು
ವಿಷ-ಯಾ-ಸಕ್ತ
ವಿಷ-ಯೇಷು
ವಿಷಾ-ಯತೇ
ವಿಷೀದಥ
ವಿಷ್ಟಾಕೃಮಿತ್ವ
ವಿಷ್ಟು
ವಿಷ್ಟು-ದೂತರು
ವಿಷ್ಠಾ
ವಿಷ್ಠಾಯಾಂ
ವಿಷ್ಣವೇ
ವಿಷ್ಣವೇರ್ಪಿತಾ
ವಿಷ್ಣು
ವಿಷ್ಣುಂ
ವಿಷ್ಣುಃ
ವಿಷ್ಣು-ದೂತೈಸ್ತಥಾಜ್ಞಪ್ರಾಸ್ತ್ರಯಸ್ತೆ
ವಿಷ್ಣುನಾ
ವಿಷ್ಣು-ಪಾದಾಬ್ಜ
ವಿಷ್ಣು-ಭಕ್ತಿಃ
ವಿಷ್ಣು-ಮತಂದ್ರಿತಃ
ವಿಷ್ಣು-ರಿತಿ
ವಿಷ್ಣುರ್ಜೀವಂ
ವಿಷ್ಣುರ್ದಾತಾ
ವಿಷ್ಣುರ್ದೇವೈಃ
ವಿಷ್ಣುರ್ಭ-ವತ್ತ್ಸು
ವಿಷ್ಣುರ್ವಾಯುಂ
ವಿಷ್ಣು-ಲೋಕಂ
ವಿಷ್ಣು-ಲೋಕ-ದಲ್ಲಿ
ವಿಷ್ಣು-ಲೋಕೇ
ವಿಷ್ಣು-ವನ್ನು
ವಿಷ್ಣು-ವನ್ನೂ
ವಿಷ್ಣು-ವನ್ನೇ
ವಿಷ್ಣು-ವಿಗೆ
ವಿಷ್ಣು-ವಿನ
ವಿಷ್ಣು-ವಿ-ನಲ್ಲಿ
ವಿಷ್ಣು-ವಿನಲ್ಲಿಯೂ
ವಿಷ್ಣು-ವಿನಲ್ಲಿಯೇ
ವಿಷ್ಣು-ವಿ-ನಿಂದ
ವಿಷ್ಣುವು
ವಿಷ್ಣುವೇ
ವಿಷ್ಣುಸ್ಸರ್ವತ್ರ
ವಿಷ್ಣೋಃ
ವಿಷ್ಣೋರ-ಧೀನಗಾಃ
ವಿಷ್ಣೋರರ್ಥೇ
ವಿಷ್ಣೋರರ್ಪಣಮೇಕೈಕಂ
ವಿಷ್ಣೋ-ರೇವ
ವಿಷ್ಣೋರ್ದಿನತ್ರಯೇ
ವಿಷ್ಣೋರ್ದೀಪದಾ
ವಿಷ್ಣೋರ್ದೇವಸ್ಯ
ವಿಷ್ಣೋರ್ಯಸ್ತು
ವಿಷ್ಣೋರ್ಲೋಕ-ಮಥಾಪ್ನುಯಃ
ವಿಷ್ಣೋರ್ಲೋಕೇ
ವಿಷ್ಣೋಶ್ಚ
ವಿಷ್ಣೌ
ವಿಷ್ಣ್ವರ್ಪಣಂ
ವಿಸರ್ಗ-ಗಳೂ
ವಿಸರ್ಜ-ನಮ್
ವಿಸರ್ಜನೆ
ವಿಸರ್ಜ-ಯೇತ್
ವಿಸರ್ಜಿಸ-ಬೇಕು
ವಿಸ್ತರತೋ
ವಿಸ್ತರಾತ್ಕ
ವಿಸ್ತಾ-ರ-ವಾಗಿ
ವಿಸ್ತಾ-ರ-ವಾಗಿಯೂ
ವಿಸ್ಮಯಂ
ವಿಸ್ಮ-ಯೋತ್ಫುಲ್ಲಲೋ-ಚನಾಃ
ವಿಸ್ಮಿತೊ
ವಿಸ್ಮಿತೋ
ವಿಸ್ಮೃತಂ
ವಿಸ್ಮೃ-ತಮ್
ವಿಹರಿ-ಸಲು
ವಿಹಾಯ
ವಿಹಾರಾರ್ಥ-ವಾಗಿ
ವಿಹಿತಪ್ರತಿ-ಷೇಧಾವೈ
ವಿಹಿತ-ವಾದ
ವಿಹಿತ-ವೆಂದು
ವಿಹ್ವಲಃ
ವೀಂದ್ರಃ
ವೀಚೀನಿಕ್ಷೇಪಸಂಭಿನ್ನೇ
ವೀಚೀವಿಕ್ಷೋಭಶೋಭಾಢ್ಯಂ
ವೀತಮಲಾಃ
ವೀತಿಹೋತ-ನೆಂಬ
ವೀತಿಹೋ-ತಸ್ಯ
ವೀರಬಾಹು
ವೀರಸ್ವರ್ಗ
ವೀರೌ
ವೀರ್ಯ-ವಾನ್
ವೀರ್ಯಶಾಲಿ
ವೀರ್ಯೈಶ್ವರ್ಯಮ-ದಾನ್ವಿತಃ
ವು
ವೃ
ವೃಂತಾಕಂ
ವೃಂತಾ-ಕಾನಿ
ವೃಂದಾ-ವನೇ-ಽಪಿ
ವೃಕವ್ಯಾಘ್ರ-ಸ-ಮಾನಸೌ
ವೃಕ-ಸಮಾನನಃ
ವೃಕಾನನ
ವೃಕಾನ್
ವೃಕ್ಷಂ
ವೃಕ್ಷಕ್ಕೆ
ವೃಕ್ಷ-ಗಳ
ವೃಕ್ಷ-ಗಳಲ್ಲಿ
ವೃಕ್ಷ-ಗಳಿ-ಗಿಂತ
ವೃಕ್ಷ-ತಲೇ
ವೃಕ್ಷದ
ವೃಕ್ಷ-ದಲ್ಲಿ
ವೃಕ್ಷ-ದಲ್ಲಿ-ರುವ
ವೃಕ್ಷ-ಮೂಲ-ಮು-ಪಾಶ್ರಿತಃ
ವೃಕ್ಷ-ಮೂಲೇ
ವೃಕ್ಷ-ಮೇ-ತನ್ನಿಶಾಮಯ
ವೃಕ್ಷ-ವನ್ನು
ವೃಕ್ಷವು
ವೃಕ್ಷವೇ
ವೃಕ್ಷಾಣಾಂ
ವೃಕ್ಷೇ
ವೃಕ್ಷೇಸ್ಮಿನ್
ವೃಕ್ಷೇ-ಽಸ್ಮಿನ್
ವೃಣುತೇ
ವೃತ್ತ
ವೃತ್ತಾಂಗಳನ್ನು
ವೃತ್ತಾಂತಂ
ವೃತ್ತಾಂತ-ಗಳನ್ನೂ
ವೃತ್ತಾಂತ-ಮಂಜಸಾ
ವೃತ್ತಾಂತ-ಮಚೋದ-ಯತ್
ವೃತ್ತಾಂತ-ಮಪಿ
ವೃತ್ತಾಂತ-ವನ್ನು
ವೃತ್ತಾಂತ-ವನ್ನೆಲ್ಲ
ವೃತ್ತಿಃ
ವೃತ್ತಿಗೆ
ವೃತ್ತಿಮಬೋಧ-ಯತ್
ವೃತ್ತಿ-ರಾ-ಸೀತ್
ವೃತ್ತಿರ್ದೈವೇನ
ವೃತ್ತಿರ್ಬಾಹ್ಮ-ಣಾನಾಂ
ವೃತ್ತ್ಯಾತು
ವೃತ್ರಂ
ವೃತ್ರ-ನೆಂಬ
ವೃತ್ರಹಾ
ವೃತ್ರಹೇತಿ
ವೃಥಾ
ವೃಥಾ-ಪ-ವಾದಚ್ಚೇತ್ತಾರಂ
ವೃಥಾಯಂ
ವೃಥಾ-ಽಹಂಕಾರ-ದೂಷಿತಾಃ
ವೃದ್ದ-ಗಂಗೇ
ವೃದ್ದ-ರಿಗೂ
ವೃದ್ದರು
ವೃದ್ದಾಃ
ವೃದ್ದಾನಾಂ
ವೃದ್ಧ-ಗಂಗಾ
ವೃದ್ಧ-ನಾದ
ವೃದ್ಧ-ರನ್ನೂ
ವೃದ್ಧ-ರಾದ
ವೃದ್ಧಿ
ವೃದ್ಧಿ-ಯಾಗುತ್ತದೆ
ವೃದ್ಧಿ-ಯಾಗುವಂತೆ
ವೃದ್ಧೋ
ವೃದ್ಧ್ಯಾ-ಭಿ-ಕಾಂಕ್ಷಯಾ
ವೃಷಂ
ವೃಷಭ
ವೃಷ-ಭಕ್ಷಿಪ್ತಂ
ವೃಷಭ-ಗಳನ್ನು
ವೃಷಭ-ಸ-ಹಿತ-ವಾದ
ವೃಷಭೋತ್ಸರ್ಗ-ದಿಂದ
ವೃಷಭೋದ್ವಾ-ಹಿತಾಸ್ತು
ವೃಷಾ-ದಶ್ವಾದ್ಗಜಾದ್ವಾಪಿ
ವೃಷೋತ್ಸರ್ಗ
ವೃಷೋತ್ಸರ್ಗಂ
ವೃಷೋತ್ಸರ್ಗಕ್ಕೆ
ವೃಷೋತ್ಸರ್ಗದ
ವೃಷೋತ್ಸರ್ಗ-ವನ್ನು
ವೃಷೋತ್ಸರ್ಗ-ಸಮಂ
ವೃಷೋತ್ಸರ್ಗಾತ್
ವೃಷ್ಟಿ
ವೃಷ್ಟಿ-ರಧ್ವ-ಗತೋ
ವೃಷ್ಟಿಸ್ತದಾ-ಭ-ವತ್
ವೃಷ್ಟ್ಯಾ
ವೆಂಬ
ವೆಃದಾಧ್ಯಾ-ಯನ
ವೆಸ್ಟ್
ವೇಣೀ
ವೇಣೀಈ
ವೇಣುಗುಲ್ಮಮಿವಾನಲಃ
ವೇಣುಗ್ಲ್ಮೇಷು
ವೇತ-ನಮ್
ವೇತ್ತಿ
ವೇತ್ಸಿ
ವೇದ
ವೇದ-ಕರ್ಮ-ಬಹಿಷ್ಕೃತಃ
ವೇದ-ಕೀರ್ತಿ-ಯೆಂಬು-ವರ
ವೇದ-ಕೀರ್ತಿಶ್ಚ
ವೇದ-ಗಳ
ವೇದ-ಗಳನ್ನು
ವೇದ-ಗಳಲ್ಲಿ
ವೇದ-ಗಳಿಗೆ
ವೇದ-ಗಳು
ವೇದ-ಗಳೂ
ವೇದ-ಚೋದಿ-ತಮ್
ವೇದ-ತತ್ವಾರ್ಥ-ಕೋವಿದಃ
ವೇದ-ಪರಾ-ಯಣ-ನಾದ
ವೇದ-ಪಾರಗಃ
ವೇದ-ಮಾರ್ಗ-ವನ್ನು
ವೇದ-ಮಾರ್ಗವಿ-ನಿಂದ-ಕಮ್
ವೇದ-ವನ್ನು
ವೇದ-ವಾದಃ
ವೇದ-ವಿಕ್ರಯ-ಮಾಡಿ
ವೇದ-ವಿಕ್ರಯೊ
ವೇದ-ವಿದಾಂ
ವೇದ-ವಿದಾಂತ್ವಿ-ದಮ್
ವೇದ-ವಿದ್ಯಾಶ್ಚ
ವೇದ-ವಿದ್ಯೆಗೆ
ವೇದ-ವಿದ್ಯೆ-ಯನ್ನು
ವೇದ-ವಿದ್ಯೆ-ಯನ್ನೆಲ್ಲ
ವೇದ-ವಿ-ನಿಂದಕಃ
ವೇದ-ವಿರೋಧ-ವಾದ
ವೇದ-ವಿಶ್ವಾಸ-ಹಾನಿರ್ಗೀತಾ-ಸಕ್ತೋ
ವೇದ-ವೇದಾಂಗ
ವೇದ-ವೇದಾಂಗ-ಗಳ
ವೇದ-ವೇದಾಂಗ-ಗಳನ್ನು
ವೇದ-ವೇದಾಂಗ-ಗಳಲ್ಲಿ
ವೇದ-ವೇದಾಂಗ-ತತ್ವಜ್ಞಃ
ವೇದ-ವೇದಾಂಗ-ಪಂಚ-ಕಮ್
ವೇದ-ವೇದಾಂಗ-ಪಾರಗಃ
ವೇದ-ವೇದಾರ್ಥ-ಗಳ
ವೇದ-ವೇದಾರ್ಥ-ತತ್ವಜ್ಞಃ
ವೇದವ್ಯಾಖ್ಯಾನ-ವನ್ನು
ವೇದ-ಶಾಸ್ತ್ರ-ಗಳಲ್ಲಿ
ವೇದ-ಶಾಸ್ತ್ರ-ನಿರತಾಸ್ತೇ
ವೇದ-ಸಚ್ಛಾತ್ರ-ಗಳ
ವೇದಸ್ಮೃತಿ
ವೇದಸ್ಮೃತಿ-ಪುರಾ-ಣೇಷು
ವೇದಾಂಗ-ಗಳು
ವೇದಾಂಗ-ಪಂಚ-ಕಮ್
ವೇದಾಃ
ವೇದಾದಿ-ಗಳನ್ನು
ವೇದಾಧ್ಯ-ಯನ
ವೇದಾಧ್ಯ-ಯನ-ದಲ್ಲಿ
ವೇದಾಧ್ಯ-ಯನ-ವನ್ನು
ವೇದಾಧ್ಯ-ಯನ-ಶೀಲೇಭ್ಯೋ
ವೇದಾನಾಂ
ವೇದಾನ್ಸಾಂಗಾಂಶ್ಚ
ವೇದಾಭಿ-ಮಾನಿ
ವೇದಾಭ್ಯಾಸೀ
ವೇದಾರ್ಥ-ಗಳನ್ನು
ವೇದಾಶ್ಚ
ವೇದಾಸ್ತಥಾ
ವೇದಿಕಾಯಾಂ
ವೇದಿಕೆಯ-ಮೇಲೆ
ವೇದೇ
ವೇದೈಃ
ವೇದೋಕ್ತ
ವೇದೋಕ್ತ-ವಾದ
ವೇಳೆ
ವೇಳೆ-ಗಳಲ್ಲಿ
ವೇಳೆ-ಯಲ್ಲಿ
ವೇಶ್ಯಾಸ್ವ-ಜನ್ಮನಿ
ವೇಶ್ಯಾ-ಽ-ಭೂತ್ಕರ್ಮ-ಶೇಷೇಣ
ವೇಶ್ಯೆ
ವೇಶ್ಯೆಯ
ವೇಶ್ಯೆ-ಯಾಗಿ
ವೇಶ್ಯೆ-ಯಾಗಿದ್ದ
ವೇಷಂ
ವೇಷ-ದಿಂದ
ವೇಷಧಾರಿಯು
ವೇಷ-ಯಿತ್ವಾ
ವೇಷೀ
ವೈ
ವೈಕುಂಠ
ವೈಕುಂಠದ
ವೈಕುಂಠ-ದಲ್ಲಿ
ವೈಕುಂಠ-ದಿವಸೇ
ವೈಕುಂಠ-ಲೋಕಕ್ಕೆ
ವೈಕುಂಠೇ
ವೈತ-ರಣೀ-ಗೋ-ದಾನವೇ
ವೈದರ್ಭೀ
ವೈದರ್ಭೋಪ್ಯ-ಭ-ವತ್ತ
ವೈದೇಹಾಯ
ವೈದ್ಯರ
ವೈಧವ್ಯ
ವೈಧವ್ಯಂ
ವೈಧವ್ಯ-ವನ್ನು
ವೈಧವ್ಯವು
ವೈರವೇ
ವೈರಾಗ್ಯ
ವೈರಾಗ್ಯಾತ್ಪ-ರಮಾತ್ಪಶ್ಚಾತ್ಸಂವಿದ್ಧಿ
ವೈರಾಗ್ಯೇಣ
ವೈಶಾಖ
ವೈಶಾಖ-ದಲ್ಲಿ
ವೈಶ್ಯ
ವೈಶ್ಯಃ
ವೈಶ್ಯ-ಪುತ್ರಕಾಃ
ವೈಶ್ಯ-ರಿಗೆ
ವೈಶ್ಯರು
ವೈಶ್ಯ-ವೃತ್ತಿ-ಯನ್ನು
ವೈಶ್ಯ-ವೃತ್ತಿ-ಯಿಂದ
ವೈಶ್ಯ-ವೃತ್ತೌ
ವೈಶ್ಯ-ಶೂದ್ರೇಷು
ವೈಶ್ಯಾನಾಂ
ವೈಶ್ಯೇಭ್ಯೋ
ವೈಶ್ಯೋಪ-ಜೀವಕಃ
ವೈಷಮ್ಮ-ದಿಂದ
ವೈಷಮ್ಯ
ವೈಷಮ್ಯಂ
ವೈಷಮ್ಯ-ನೈರ್ಘೃಣ್ಯಾಧಿಕ-ರಣ
ವೈಷ್ಣವ
ವೈಷ್ಣವ-ಧರ್ಮ-ದಿಂದ
ವೈಷ್ಣವಾನಾಂ
ವೈಷ್ಣವೀ
ವ್ಯಂ
ವ್ಯಗ್ರ-ಮನಸ್ಸುಳ್ಳ
ವ್ಯಚರತ್
ವ್ಯಚರದ್ಭುವಿ
ವ್ಯಜಹಾರ
ವ್ಯತಿಕ್ರಮೇತ್
ವ್ಯತೀ-ತಾನಿ
ವ್ಯತೀತ್ಯೇವ
ವ್ಯತೀಪಾತ-ಗಳು
ವ್ಯತೀಪಾತ-ದಿಂದ
ವ್ಯತೀಪಾತವಿ-ರು-ವಾಗ
ವ್ಯತೀಪಾತ-ಶ-ತಾನಿ
ವ್ಯತೀಪಾತೇ
ವ್ಯತೀಪಾ-ತೇನ
ವ್ಯಥ
ವ್ಯಪ-ದೇಶಾತ್
ವ್ಯಪೋಹತಿ
ವ್ಯಪೋ-ಹಿ-ತಮ್
ವ್ಯಭಿಚಾರ
ವ್ಯಯಂ
ವ್ಯಯೋ
ವ್ಯರ್ಥ
ವ್ಯರ್ಥ-ವಾಗಿ
ವ್ಯರ್ಥ-ವಾಗು-ವುದಿಲ್ಲ
ವ್ಯರ್ಥ-ವಾದ
ವ್ಯವಸಾಯ
ವ್ಯವಸ್ಥೆ-ಗಾಗಿ
ವ್ಯವಹಾರ-ಗಳಲ್ಲಿಯೇ
ವ್ಯವಹಾರಾಃ
ವ್ಯವಾಯ-ಕರ್ಮ-ನಿರತೌ
ವ್ಯಾ
ವ್ಯಾಕ-ರಣ
ವ್ಯಾಕುರ್ವಾಣಂ
ವ್ಯಾಕುರ್ವಾಣಾಂ
ವ್ಯಾಖ್ಯಾತಂ
ವ್ಯಾಖ್ಯಾ-ತಾಶ್ಚ
ವ್ಯಾಖ್ಯಾನ
ವ್ಯಾಖ್ಯಾ-ನದ
ವ್ಯಾಖ್ಯಾನೇಭ್ಯೋ
ವ್ಯಾಘ್ರ
ವ್ಯಾಘ್ರಂ
ವ್ಯಾಘ್ರಃ
ವ್ಯಾಘ್ರಗ್ರಸಿತಮಗ್ರ-ಜಮ್
ವ್ಯಾಘ್ರದ
ವ್ಯಾಘ್ರ-ನನ್ನು
ವ್ಯಾಘ್ರ-ನನ್ನೂ
ವ್ಯಾಘ್ರ-ನಿಗೂ
ವ್ಯಾಘ್ರ-ಮುಖ-ವನ್ನುಳ್ಳ
ವ್ಯಾಘ್ರ-ಸಮಾನನಃ
ವ್ಯಾಘ್ರಾನನ-ನೆಂಬ
ವ್ಯಾಘ್ರಾನನಸ್ತು
ವ್ಯಾಘ್ರೇಃ
ವ್ಯಾಘ್ರೋ
ವ್ಯಾಘ್ರೋಯಂ
ವ್ಯಾಜ-ದಿಂದ
ವ್ಯಾಜಮುತ್ಪಾತ-ಯಾಮ್ಯ
ವ್ಯಾಧಿಯು
ವ್ಯಾಧಿಶಮೋ
ವ್ಯಾಧೋ
ವ್ಯಾನ-ನಾಮಕಃ
ವ್ಯಾನ-ನೆಂಬ
ವ್ಯಾಪಾರ-ಗಳನ್ನೂ
ವ್ಯಾಪಾರ-ಗಳಿಗೆ
ವ್ಯಾಪಾರ-ಗಳೂ
ವ್ಯಾಪಾರ-ಗಳೇ
ವ್ಯಾಪಾರ-ದಲ್ಲಿ
ವ್ಯಾಪಾರ-ದಿಂದ
ವ್ಯಾಪಾ-ರಾದಿ-ಗಳನ್ನು
ವ್ಯಾಪ್ತಂ
ವ್ಯಾಪ್ತ-ನಾದ
ವ್ಯಾಪ್ತ-ನಾದರೂ
ವ್ಯಾಪ್ತನೂ
ವ್ಯಾಪ್ತ-ವಾಗಿದೆ
ವ್ಯಾಪ್ತೋ-ಽಪಿ
ವ್ಯಾಮೋ-ಹ-ದಿಂದ
ವ್ಯಾಲಾಧಿಷ್ಠಿ
ವ್ಯಾಲೋ
ವ್ಯಾಲೋದ್ರಿ-ಕಂದರೇ
ವ್ಯಾಲೋ-ಭೂದ್ಯೋ-ಜನಾ-ಯಾಮ-ಶರೀರೋಽನಿಲ-ಭೋ-ಜನಃ
ವ್ಯಾಸಸ-ಮಾಸಾಭ್ಯಾಂ
ವ್ಯಾಹರಂತ್ಯವಿಚಾ-ರತಃ
ವ್ಯೋಮಾತ್ಮಾ
ವ್ರಜಂತಿ
ವ್ರಜಂತ್ಯದ್ಧಾ
ವ್ರಜಖೇಟಕಪಲ್ಲೀಷು
ವ್ರಜೇ
ವ್ರಜೇತ್
ವ್ರಣ-ಗಳಿಂದ
ವ್ರಣ-ಶೋಣಿತಪೂ-ಯೇನ
ವ್ರತ
ವ್ರತಂ
ವ್ರತ-ಗಳ
ವ್ರತ-ಗಳನ್ನು
ವ್ರತ-ಗಳಲ್ಲಿಯೂ
ವ್ರತದ
ವ್ರತ-ದಲ್ಲಿ
ವ್ರತ-ನಿಷ್ಠ-ನಾದ
ವ್ರತ-ಮನುತ್ತಮಮ್
ವ್ರತಮ್
ವ್ರತ-ವನ್ನು
ವ್ರತ-ಸ-ಮಾಪ-ನಮ್
ವ್ರತಸ್ಥ-ರಾದ
ವ್ರತಸ್ಥೋ-ಽಹಂ
ವ್ರತಸ್ಯ
ವ್ರತಸ್ಯಾಸ್ಯ
ವ್ರತಾ-ನಾ-ಮಪಿ
ವ್ರತಾನಿ
ವ್ರತಿನಾಂ
ವ್ರತೀ
ವ್ರತೇ
ವ್ರಾತ್ಯರು
ವ್ರಾತ್ಯೇಷ್ಟಥ
ವ್ರೀಡಯಾ
ಶಂಕಾ-ಕುಲೇಕ್ಷಣಃ
ಶಂಕಾಸಮಾ-ಕುಲಃ
ಶಂಕು-ವರ್ಣ-ರೆಂಬು-ವರ
ಶಂಕು-ವರ್ಣ-ರೆಂಬು-ವರಿಂದ
ಶಂಕು-ವರ್ಣಸ್ಯ
ಶಂಖ
ಶಂಖಂ
ಶಂಖ-ಚಕ್ರಗ-ದಾದಿ-ಗಳನ್ನು
ಶಂಖ-ಚಕ್ರಗದಾಧರ
ಶಂಖ-ಚಕ್ರಗದಾಧರ-ನಮ್
ಶಂಖದ
ಶಂಖ-ದಲ್ಲಿ
ಶಂಖ-ದಿಂದ
ಶಂಖಭ್ರಮಣ-ವನ್ನು
ಶಂಖ-ಮಧ್ಯೇ
ಶಂಖ-ವನ್ನು
ಶಂಖ-ವರ್ತಿನಾ
ಶಂಖವು
ಶಂಖೈಃ
ಶಂಖೋ
ಶಂಖೋ-ದಕಂ
ಶಂಖೋ-ದಕ-ವನ್ನು
ಶಂಖೋ-ದ-ಕೇನ
ಶಂಬರಾರೇಃ
ಶಂಸಿತವ್ರತಂ
ಶಂಸಿತವ್ರತಾಃ
ಶಕ್ತಃ
ಶಕ್ತನಲ್ಲ
ಶಕ್ತ-ರಲ್ಲ
ಶಕ್ತರಿಲ್ಲ
ಶಕ್ತರು
ಶಕ್ತಾ
ಶಕ್ತಿ
ಶಕ್ತಿಃ
ಶಕ್ತಿ-ಗನು-ಸಾರ-ವಾಗಿ
ಶಕ್ತಿಗೆ
ಶಕ್ತಿ-ದೇವ-ತೆ-ಗಳು
ಶಕ್ತಿಭ್ಯೋ
ಶಕ್ತಿ-ಯುತ-ರಾಗಿಲ್ಲ
ಶಕ್ತಿರ್ದೂವಾ-ಮೂಲಾನಿ
ಶಕ್ತಿ-ವಿಮರ್ಶ-ನಮ್
ಶಕ್ತಿ-ಹೀನ-ನಾದರೆ
ಶಕ್ತೇನಾಕ್ಷಿ-ಪತ್
ಶಕ್ತೋ
ಶಕ್ತ್ಯಾ-ನು-ಸಾರ
ಶಕ್ಯ-ಮಿತಿ
ಶಕ್ಯಾ
ಶಕ್ರಾಯ
ಶಕ್ರೋ
ಶಚೀ-ಪತಿ-ಮನೋ-ಭವೌ
ಶಠಃ
ಶತಂ
ಶತ-ಕೋಟಿ
ಶತಕ್ರತು-ಫಲಂ
ಶತ-ಪತ್ರೈರ್ಮನೋ-ರಮೈಃ
ಶತ-ಪುರುಷ-ಗಳ-ವೆರಗೂ
ಶತಪೌರುಷೀ
ಶತಬಲಿ
ಶತಬಲಿರ್ನಾಮ
ಶತರ್ಚಿ
ಶತರ್ಚಿತೋ
ಶತರ್ಚಿನೋ
ಶತರ್ಚಿ-ರೆಂಬ
ಶತಾನಿ
ಶತೄನ್
ಶತ್ರು-ಗಳನ್ನು
ಶತ್ರು-ಗಳು-ಮಿತ್ರರು
ಶನಿ-ವಾರ
ಶನಿ-ವಾರೇಣ
ಶನೈಃ
ಶನೈರ್ನಿಯಮ-ಕಾರ-ಣಮ್
ಶನೈರ್ಮೋಕ್ಷಾಯ
ಶಪತಸ್ತತಸ್ತೇನ
ಶಪಲಾಶಾರ್ಥಂ
ಶಪಿಸಿ-ದನು
ಶಪಿಸಿ-ದರು
ಶಪ್ತ-ನಾದ
ಶಪ್ತೋ
ಶಫಲಾದ್ಯರ್ಥ
ಶಬ್ದಾತ್ಮಾ
ಶಬ್ದಾಭಿ-ಮಾನಿ
ಶಮನ-ವಾ-ಗಲು
ಶಯಾನಂ
ಶಯ್ಯಾಗೃಹ-ದಲ್ಲಿ
ಶಯ್ಯಾ-ಗೃಹೇ
ಶರಣಃ
ಶರಣು
ಶರಣು-ಹೊಂದ-ಬೇಕು
ಶರಣ್ಯಂ
ಶರಣ್ಯನು
ಶರೀರ-ಗಳಲ್ಲಿ
ಶರೀರದ
ಶರೀರವು
ಶರೀರ-ವೆಂಬ
ಶರೀರವೆಲ್ಲ
ಶರೀರಸ್ಥಂ
ಶರೀರಸ್ಥಾಃ
ಶರೀರಾಣ್ಯಪಿ
ಶರೀರಿಣಾಂ
ಶರೀರಿಸು
ಶರೈಸ್ತಯೋಃ
ಶರ್ಮಪಶ್ಯಾಮಿ
ಶರ್ವ
ಶರ್ವೋ
ಶಲ್ಯಂ
ಶವದ
ಶವಾಸೃಙ್
ಶಶಪುಸ್ತೇ
ಶಶಾಪ
ಶಶಾಪಾ-ತೀವ
ಶಶಿಗ್ರಹಶತಾಧಿಕಃ
ಶಶ್ವತ್
ಶಸ್ತಾ
ಶಸ್ತಾನಿ
ಶಸ್ತ್ರ-ದಿಂದ
ಶಸ್ತ್ರಪಾಣಯಃ
ಶಸ್ತ್ರ-ವಿದ್ಯಾಸು
ಶಸ್ತ್ರಾಜ್ಜಲಾದ್ವಿಷಾತ್ಸರ್ಪಾಚ್ಚಾಂಡಾಲಾತ್ಕುಲಿಶಾದಪಿ
ಶಸ್ತ್ರೈಃ
ಶಾಂಡಿಲ್ಯ
ಶಾಂಡಿಲ್ಯಂ
ಶಾಂಡಿಲ್ಯಃ
ಶಾಂಡಿಲ್ಯ-ಗೋತ್ರಜಃ
ಶಾಂಡಿಲ್ಯರ
ಶಾಂಡಿಲ್ಯ-ರನ್ನು
ಶಾಂಡಿಲ್ಯ-ರಿಗೆ
ಶಾಂಡಿಲ್ಯರು
ಶಾಂಡಿಲ್ಯ-ರೆಂಬ
ಶಾಂಡಿಲ್ಯಸ್ಯ
ಶಾಂಡಿಲ್ಯೋ
ಶಾಂತ
ಶಾಂತಂ
ಶಾಂತ-ಚಿತ್ತನೂ
ಶಾಂತ-ನಾಗಿದ್ದು
ಶಾಂತ-ನಾದ
ಶಾಂತ-ರಾದ
ಶಾಂತಿ
ಶಾಂತಿ-ಭ-ರಿತರಾ-ದರು
ಶಾಂತಿ-ಮಚಿ-ರೇಣಾಧಿ-ಗಚ್ಛತಿ
ಶಾಂತಿಶ್ಚ
ಶಾಂತೋ
ಶಾಂತೌ
ಶಾಕಂ
ಶಾಕ-ಗಳನ್ನೂ
ಶಾಕ-ವರ್ಜಿತಾಃ
ಶಾಕಿನಂ
ಶಾಕಿನೀಂ
ಶಾಕಿನೀ-ಗಣಕ್ಕೆ
ಶಾಕಿನೀ-ಗಣ-ದಲ್ಲಿ
ಶಾಕಿನೀತ್ವಮವಾಪ್ನು-ಯಾಮ್
ಶಾಕಿ-ನೀಯೋ-ನಿಗೆ
ಶಾಕಿನೀ-ಯೋನಿಮಾಶ್ರಿತಾಃ
ಶಾಕಿನೀ-ಯೋನಿಮಾಶ್ರಿತ್ಯ
ಶಾಖಾಮೃ-ಗಾಶ್ಚಲಾಃ
ಶಾಖೆ-ಗಳಿಂದ
ಶಾಖೆ-ಗಳಿದ್ದ
ಶಾಖೆ-ಗಳು
ಶಾಣನಿಶಾತವುಗ್ರಂ
ಶಾಣನಿಶಾ-ತಿತಂ
ಶಾಪ
ಶಾಪಂ
ಶಾಪ-ಕೊಟ್ಟರು
ಶಾಪ-ಕೊಟ್ಟಳು
ಶಾಪ-ಗಸ್ತ-ನಾದ
ಶಾಪದ
ಶಾಪ-ದಿಂದ
ಶಾಪ-ವನ್ನು
ಶಾಪವು
ಶಾಪೇ
ಶಾಬರೀ
ಶಾಯಿನಂ
ಶಾರ್ದೂಲಸ್ಯ
ಶಾಲಗ್ರಾಮ
ಶಾಲಗ್ರಾಮಕ್ಕೆ
ಶಾಲಗ್ರಾಮದ
ಶಾಲಗ್ರಾಮ-ಪೂಜಾಂ
ಶಾಲಗ್ರಾಮ-ವನ್ನು
ಶಾಲಗ್ರಾಮ-ವಿರುತ್ತ-ದೆಯೋ
ಶಾಲಗ್ರಾಮ-ಶಿಲಾ
ಶಾಲಗ್ರಾಮ-ಶಿಲಾಂ
ಶಾಲಗ್ರಾಮ-ಶಿಲಾ-ತೀರ್ಥ-ಕಲಾಂ
ಶಾಲಗ್ರಾಮ-ಶಿಲಾ-ಪೂಜಾಂ
ಶಾಲಗ್ರಾಮ-ಶಿಲಾ-ವಾಪಿ
ಶಾಲಗ್ರಾಮ-ಶಿಲಾ-ವಾರಿ
ಶಾಲಗ್ರಾಮ-ಶಿಲೋದ-ಕಮ್
ಶಾಲಗ್ರಾಮ-ಶಿಲೋದಕಾತ್
ಶಾಲಗ್ರಾಮ-ಶಿಲೋದಕೈಃ
ಶಾಲಗ್ರಾಮಾರ್ಪಿತಾ
ಶಾಲಗ್ರಾಮೋ-ದಕಂ
ಶಾಲ್ಮಲೀ
ಶಾಲ್ಮಲೀಂ
ಶಾಲ್ಮಲೀ-ತಟಾಮೀಯುಷಾಮ್
ಶಾಲ್ಮಲೀದ್ರು-ಮಮ್
ಶಾಲ್ಮಲೀದ್ರುಮೇ
ಶಾಲ್ಮಲೀ-ಫಲ-ಮಾ-ದಾಯ
ಶಾಲ್ಮಲೀ-ವೃಕ್ಷಕ್ಕೆ
ಶಾಲ್ಮಲ್ಯಾಂ
ಶಾಶ್ವತ-ವಾದ
ಶಾಸನ-ವನ್ನು
ಶಾಸೊಕ್ತ-ವಾದ
ಶಾಸ್ತಾರಃ
ಶಾಸ್ತೇಷು
ಶಾಸ್ತ್ರ
ಶಾಸ್ತ್ರಕ್ಕಾಗಿ
ಶಾಸ್ತ್ರ-ಗಳ
ಶಾಸ್ತ್ರ-ಗಳನ್ನು
ಶಾಸ್ತ್ರ-ಗಳಲ್ಲಿ
ಶಾಸ್ತ್ರ-ಗಳಲ್ಲಿಯೂ
ಶಾಸ್ತ್ರ-ಗಳಿಗೆ
ಶಾಸ್ತ್ರ-ಗಳು
ಶಾಸ್ತ್ರ-ದಲ್ಲಿ
ಶಾಸ್ತ್ರ-ವಿಧಾ-ನೋಕ್ತಂ
ಶಾಸ್ತ್ರ-ವಿಧಿ-ಮುತ್ಸೃಜ್ಯ
ಶಾಸ್ತ್ರ-ವಿ-ಹಿತಂ
ಶಾಸ್ತ್ರ-ವಿ-ಹಿತ-ವಾದ
ಶಾಸ್ತ್ರ-ವಿ-ಹಿತ-ವಾದವು
ಶಾಸ್ತ್ರವೇ
ಶಾಸ್ತ್ರಾಭ್ಯಾಸೋ
ಶಾಸ್ತ್ರೇಣ
ಶಾಸ್ತ್ರೋಕ್ತ
ಶಾಸ್ತ್ರೋಕ್ತ-ವಾದ
ಶಾಸ್ತ್ರೋದ್ವರ್ತ
ಶಾಸ್ತ್ರೋದ್ವರ್ತಃ
ಶಾಸ್ತ್ರೋದ್ವೃತ್ತಃ
ಶಾಸ್ತ್ರೋದ್ವೃತ್ತನು
ಶಾಸ್ತ್ರೋದ್ವೃತ್ತ-ನೆಂಬ
ಶಾಸ್ತ್ರೋದ್ವೃತ್ತೋ
ಶಾಸ್ರೋಕ್ತ-ವಾದ
ಶಿಂಶುಮಾರನ
ಶಿಂಶುಮಾರಸ್ಯ
ಶಿಕ್ಷಕಃ
ಶಿಕ್ಷತಿ
ಶಿಕ್ಷಾ
ಶಿಕ್ಷಿ-ತ-ನಾದ
ಶಿಕ್ಷಿತಾ
ಶಿಕ್ಷಿತಾಭ್ಯಾಂ
ಶಿಕ್ಷಿ-ಸುತ್ತಾರೆ
ಶಿಕ್ಷೆ
ಶಿಕ್ಷೆಗೆ
ಶಿಕ್ಷೆ-ಯನ್ನು
ಶಿಖಾ
ಶಿಖಾಂ
ಶಿಖಾ-ದಿಂದ
ಶಿಖಾ-ಬದ್ಧೋ
ಶಿಖಾ-ಸಂಸ್ಥಾ-ಮಾತ್ರ-ಭೂಮೌ
ಶಿಖಾಸು
ಶಿಖಾ-ಸುಧಾಮ್
ಶಿಖಾಸ್ಕಂಧೇ
ಶಿಖೆ-ಯನ್ನು
ಶಿಖೆ-ಯಲ್ಲಿ
ಶಿಖೆ-ಯಲ್ಲಿನ
ಶಿಖೆ-ಯಲ್ಲಿಯೇ
ಶಿಖೆ-ಯಿಂದ
ಶಿಖೋತ್ಸೃಷ್ಟ
ಶಿಖೋ-ದಕಂ
ಶಿಖೋದಕ-ವನ್ನು
ಶಿಗ್ರುಶಾಖಾನಿ
ಶಿಬಿ
ಶಿಬಿರ್ಮಾಂಸಾನಿ
ಶಿರಃ
ಶಿರಚ್ಛಿತ್ವಾ
ಶಿರಸಾ
ಶಿರಸೀ
ಶಿರಸ್ಸಿ-ನಲ್ಲಿ
ಶಿರಾ
ಶಿರೋ-ನಾಮೆ-ಯಲ್ಲಿ
ಶಿರೋ-ಹೀನೇ
ಶಿವತಮಾ
ಶಿವಾ
ಶಿವಾ-ದೇವೀ-ರಶಿಪದಾ-ಭವಂತು
ಶಿವಾನಿ
ಶಿವೋ
ಶಿವೋಕ್ತಂ
ಶಿಶುಂ
ಶಿಶುಭ್ಯೋ
ಶಿಶು-ಹತ್ಯ
ಶಿಶು-ಹತ್ಯೆ
ಶಿಶೂನಾಂ
ಶಿಶೂನ್
ಶಿಶೋಃ
ಶಿಶೋ-ರಾದೌ
ಶಿಶೋರ್ಮಾತುಶ್ಚ
ಶಿಷ್ಟರು
ಶಿಷ್ಟಾಃ
ಶಿಷ್ಟಾವಶಿಷ್ಟಾಂ
ಶಿಷ್ಟೌ
ಶಿಷ್ಯ
ಶಿಷ್ಯಂ
ಶಿಷ್ಯಃ
ಶಿಷ್ಯತಃ
ಶಿಷ್ಯತಾಂ
ಶಿಷ್ಯತೋ
ಶಿಷ್ಯತ್ವಾನ್ಮೇ
ಶಿಷ್ಯನ
ಶಿಷ್ಯ-ನನ್ನು
ಶಿಷ್ಯ-ನಾಗಿಟ್ಟು
ಶಿಷ್ಯ-ನಾಗಿದ್ದೆ
ಶಿಷ್ಯ-ನಾಗಿ-ರು-ವುದ-ರಿಂದ
ಶಿಷ್ಯ-ನಾದ
ಶಿಷ್ಯ-ನಿಂದ
ಶಿಷ್ಯ-ನಿಗೆ
ಶಿಷ್ಯ-ನಿದ್ದನು
ಶಿಷ್ಯನು
ಶಿಷ್ಯನೂ
ಶಿಷ್ಯ-ರನ್ನೂ
ಶಿಷ್ಯ-ರಿಂದ
ಶಿಷ್ಯ-ರಿಗೆ
ಶಿಷ್ಯರು
ಶಿಷ್ಯ-ವಿ-ಪತ್ತಿಂ
ಶಿಷ್ಯಸ್ಯ
ಶಿಷ್ಯಸ್ಯಾಸ್ಯಾ
ಶಿಷ್ಯಾನಧ್ಯಾ-ಪಯಾ-ಮಾಸ
ಶಿಷ್ಯಾನಧ್ಯಾ-ಪಿತಾ
ಶಿಷ್ಯಾಯ
ಶಿಷ್ಯೇಣ
ಶಿಷ್ಯೇಭ್ಯಃ
ಶಿಷ್ಯೇಭ್ಯೋ
ಶಿಷ್ಯೈಃ
ಶಿಷ್ಯೋ
ಶಿಷ್ಯೋ-ಽ-ಭೂತ್
ಶಿಷ್ಯೋ-ಽಹಂ
ಶಿಷ್ವ್ಯಃ
ಶೀಘ್ರಂ
ಶೀಘ್ರ-ದಲ್ಲಿಯೇ
ಶೀಘ್ರಮಾವಾಂ
ಶೀಘ್ರ-ವಾಗಿ
ಶೀಘ್ರವೇ
ಶೀತಾರ್ತಾಯ
ಶೀತೇ
ಶೀತೇನ
ಶುಕ-ಶಾಕಂ
ಶುಕ್ರ-ವಾರ
ಶುಕ್ರ-ವಾರೇ
ಶುಕ್ಲ
ಶುಕ್ಲಾ
ಶುಚಿಃ
ಶುಚಿತ್ವ
ಶುಚಿತ್ವಂ
ಶುಚಿ-ಯಾಗಿ-ರುತ್ತಿದ್ದೆ
ಶುಚಿರ್ಭೂತ-ಳಾಗಿ
ಶುತುದ್ರಿಸ್ತೋಮಂ
ಶುದ್ದ
ಶುದ್ದ-ನಾಗಿಯೂ
ಶುದ್ದ-ಭಾವಾಃ
ಶುದ್ದಿ
ಶುದ್ದಿ-ಯಾಗಿ
ಶುದ್ದಿಯು
ಶುದ್ದೇನ
ಶುದ್ಧ
ಶುದ್ಧಂ
ಶುದ್ಧ-ತಾಮ-ಸ-ರೆಂದು
ಶುದ್ಧದ್ವಾ-ದಶ್ಯಾಂ
ಶುದ್ಧ-ನಾದ
ಶುದ್ಧ-ರಾಜ-ಸರು
ಶುದ್ಧ-ರಾಜಸಾಃ
ಶುದ್ಧ-ವಾಗುತ್ತ-ದೆಯೋ
ಶುದ್ಧ-ವಾದ
ಶುದ್ಧ-ಸತ್ತ್ವಾ
ಶುದ್ಧ-ಸತ್ವಾಶ್ಚ
ಶುದ್ಧ-ಸಾತ್ವಿ-ಕರು
ಶುದ್ಧಾ
ಶುದ್ಧಾ-ಚಮನ
ಶುದ್ಧಾ-ಚಮನ-ಮಾಡಿ
ಶುದ್ಧಿ
ಶುದ್ಧಿ-ಗೊಳಿ-ಸುತ್ತಾರೆ
ಶುದ್ಧಿ-ಯಾಗಿ
ಶುದ್ಧಿ-ಯಾಗುತ್ತದೆ
ಶುದ್ಧೇ
ಶುದ್ಧೈಃ
ಶುನೋ
ಶುಭ
ಶುಭಂ
ಶುಭ-ಕ-ರ-ನಾದ
ಶುಭ-ಕರ-ವಾದ
ಶುಭ-ಕರ್ಮಣಿ
ಶುಭ-ಕಾರ್ಯಕ್ಕೆ
ಶುಭ-ದೇಹ-ನಾಶಕಾಃ
ಶುಭಪ್ರದ-ವಾದ
ಶುಭಮ್
ಶುಭ-ವಮ್
ಶುಭಾಃ
ಶುಭಾನಿ
ಶುಭಾನ್
ಶುಭಾವಹಂ
ಶುಭೇ
ಶುಭೌ
ಶುಭ್ರ-ಗೊಳಿ-ಸುವುದು
ಶುಭ್ರ-ವಾದ
ಶುಶೂಷಾತ್ರ
ಶುಶ್ರೂಷಾ
ಶುಶ್ರೂಷುಭ್ಯಶ್ಚ
ಶುಷ್ಕ
ಶುಷ್ರೂಷವೇ
ಶೂದ್ರ
ಶೂದ್ರಃ
ಶೂದ್ರನ
ಶೂದ್ರ-ನಾಗಿದ್ದರೆ
ಶೂದ್ರ-ನಾದ
ಶೂದ್ರ-ನೊಬ್ಬ-ನಿದ್ದನು
ಶೂದ್ರ-ಪಾಖಂಡ
ಶೂದ್ರ-ಪಾಖಂಡ-ವಾರ್ಗಣಃ
ಶೂದ್ರ-ಪುತ್ರಯೋಃ
ಶೂದ್ರ-ಪುತ್ರರು
ಶೂದ್ರರು
ಶೂದ್ರಸ್ಯ
ಶೂದ್ರಾದಿಜ್ಞಾ-ತಿಕ್ಷತ್ರಾಣಾಂ
ಶೂದ್ರೋ
ಶೂದ್ರೋ-ಥವಾ
ಶೂರ್ಪಯುಗ್ಮಂ
ಶೂಲಾದಿಕಂ
ಶೃಂಖಲಾ
ಶೃಂಖಲಾ-ನಾಮ
ಶೃಂಗಾರ-ವನ್ನೂ
ಶೃಗಾಲ
ಶೃಗಾಲಾಖ್ಯಂ
ಶೃಗಾಲೋ
ಶೃಣು
ಶೃಣು-ಯಾನ್ನಿತ್ಯಂ
ಶೃಣುಹ್ಯಾ-ಸುಷೋ
ಶೃಣೋತಿ
ಶೃಣ್ವತಾಂ
ಶೇಖರಿಸಲ್ಪಡುತ್ತಿತ್ತು
ಶೇಷ
ಶೇಷ-ದೇವರ
ಶೇಷ-ನಂತೆ
ಶೇಷಪ್ರತಿಮಾಂ
ಶೇಷ-ವೀಂದ್ರಮೃಡಾ
ಶೈತನು
ಶೈನ
ಶೈವಲ-ಗೋತ್ರ-ದಲ್ಲಿ
ಶೈವಲ-ಗೋತ್ರೋ-ಽಹಂ
ಶೋಕ
ಶೋಕಂ
ಶೋಕ-ತಪ್ತ-ನಾದ
ಶೋಕ-ಪ-ಡಲು
ಶೋಕ-ಪರಾ-ಯಣಃ
ಶೋಕ-ಮೋಹ-ಜರಾ-ಮೃತ್ಯುಃ
ಶೋಕಲಾ-ಲಸಮ್
ಶೋಕ-ವೆಂಬ
ಶೋಕಾನಲಸಂತಪ್ಪೋ
ಶೋಕಾಬ್ಧಿ
ಶೋಣ-ತಟಂ
ಶೋಣತಟ-ವೆಂಬ
ಶೋತುಂ
ಶೋತೃ-ಗಳಿಗೆ
ಶೋತ್ರು
ಶೋಭಿಸುತ್ತಿದ್ದುವು
ಶೌಚ
ಶೌಚಂ
ಶೌಚಃ
ಶೌಚ-ಕರ್ಮ
ಶೌಚ-ಕರ್ಮ-ವನ್ನು
ಶೌಚಕ್ಕೆ
ಶೌಚ-ವನ್ನು
ಶೌಚ-ವಿಧಿ
ಶೌಚಾಚಾರೌ
ಶೌರ್ಯಮ-ದಾನ್ಮಯಾ
ಶೌರ್ಯೌದಾರ್ಯಾದಿಸು-ಗುಣೈರ್ಮಹೇಂದ್ರ-ಸದೃಶಾ
ಶ್ಚೋರ್ಧ್ವಂ
ಶ್ಯಂತಿ
ಶ್ಯಾಂಡಿಲ್ಯ
ಶ್ಯಾಮಾ
ಶ್ಯೇನ
ಶ್ಯೇನ-ಪಕ್ಷಿ-ಗಳನ್ನು
ಶ್ಯೇನಾಭ್ಯಾಂ
ಶ್ಯೇನೌ
ಶ್ರದ್ದಧಾನಾಯ
ಶ್ರದ್ದಾ
ಶ್ರದ್ದೆ
ಶ್ರದ್ದೆ-ಯಿಂದ
ಶ್ರದ್ದೆಯು
ಶ್ರದ್ದೆಯೇ
ಶ್ರದ್ಧಧಾನಾಯ
ಶ್ರದ್ಧಾ
ಶ್ರದ್ಧಾ-ನು-ಸಾರ-ವಾಗಿ
ಶ್ರದ್ಧಾ-ರ-ಹಿತ-ನಾದ-ವನು
ಶ್ರದ್ಧಾ-ವಾನ್
ಶ್ರದ್ಧೆ
ಶ್ರದ್ಧೆ-ಯನ್ನಿಟ್ಟು
ಶ್ರದ್ಧೆ-ಯಿಂದ
ಶ್ರದ್ಧೆ-ಯುಳ್ಳ-ವನೋ
ಶ್ರದ್ಧೇಯಂ
ಶ್ರಮ
ಶ್ರವಣ
ಶ್ರವಣಂ
ಶ್ರವಣ-ಗಳಿಂದ
ಶ್ರವಣ-ದಲ್ಲಿ
ಶ್ರವಣ-ದಿಂದ
ಶ್ರವಣ-ನಾದ-ರಾತ್
ಶ್ರವಣ-ಮನ-ನಾದಿ-ಗಳು
ಶ್ರವಣ-ಮಾಡ
ಶ್ರವಣ-ಮಾಡದೇ
ಶ್ರವಣ-ಮಾಡಿ-ದರೆ
ಶ್ರವಣ-ಮಾಡಿ-ದ-ವರ
ಶ್ರವಣ-ಮಾಡಿ-ಸುವ
ಶ್ರವಣ-ಮಾಡಿ-ಸುವ-ವರೂ
ಶ್ರವಣ-ಮಾಡುತ್ತಾನೆ
ಶ್ರವಣ-ಮಾಡುತ್ತಿರ-ಬೇಕು
ಶ್ರವಣ-ಮಾಡುವ
ಶ್ರವಣ-ಮಾಡು-ವ-ವನು
ಶ್ರವಣ-ಮಾಡು-ವುದ-ರಿಂದ
ಶ್ರವಣ-ಮಾಡು-ವುದಿಲ್ಲವೋ
ಶ್ರವಣವು
ಶ್ರವಣಾತ್
ಶ್ರವಣಾತ್ಕೋನು
ಶ್ರವಣಾತ್ಪೂಜ-ನಾದಪಿ
ಶ್ರವಣೇ-ನೈವ
ಶ್ರಾಂತೋ
ಶ್ರಾದ್ದಂ
ಶ್ರಾದ್ದೇಷು
ಶ್ರಾದ್ಧ
ಶ್ರಾದ್ಧಂ
ಶ್ರಾದ್ಧ-ಗಳಲ್ಲಿ
ಶ್ರಾದ್ಧ-ಗಳಲ್ಲಿಯೂ
ಶ್ರಾದ್ಧದ
ಶ್ರಾದ್ಧ-ದಲ್ಲಿ
ಶ್ರಾದ್ಧ-ದಾನಾರ್ಹ-ಕಾಂಕ್ಷಯಾ
ಶ್ರಾದ್ಧ-ದಿನೇಷು
ಶ್ರಾದ್ಧ-ದಿವ-ಸ-ಗಳಲ್ಲಿಯೂ
ಶ್ರಾದ್ಧ-ಪಿಂಡ
ಶ್ರಾದ್ಧ-ವನ್ನು
ಶ್ರಾದ್ಧಾದಿ
ಶ್ರಾದ್ಧಾದಿ-ಕಾಂಕ್ಷಿಣಾ
ಶ್ರಾದ್ಧಾದಿ-ಗಳಲ್ಲಿ
ಶ್ರಾದ್ಧೇ
ಶ್ರೀ
ಶ್ರೀಕೃಷ್ಣ-ನಲ್ಲಿ
ಶ್ರೀಕೃಷ್ಣ-ನಲ್ಲಿಯೇ
ಶ್ರೀಕೃಷ್ಣನೇ
ಶ್ರೀಕೃಷ್ಣೇ
ಶ್ರೀಖಂಡ
ಶ್ರೀಖಂಡಂ
ಶ್ರೀಪಾದಂಗಳ-ವರ
ಶ್ರೀಪಾದಂಗಳ-ವ-ರಲ್ಲಿ
ಶ್ರೀಮದಾಚಾರೈರು
ಶ್ರೀಮದಾಚಾರ್ಯರ
ಶ್ರೀಮದಾಚಾರ್ಯ-ರಿಂದ
ಶ್ರೀಮದಾಚಾರ್ಯರು
ಶ್ರೀಮ-ದಾನಂದತೀರ್ಥ-ರಿಂದ
ಶ್ರೀಮದುತ್ತ-ರಾದಿ
ಶ್ರೀಮನ್ಮಧ್ವ-ಸಿದ್ಧಾಂತ
ಶ್ರೀಮನ್ಮಹಾ-ಭಾರ-ತ-ತಾತ್ಪರ್ಯ-ನಿರ್ಣಯ
ಶ್ರೀರಂಗ
ಶ್ರೀರಾಮನ
ಶ್ರೀವತ್ಸ
ಶ್ರೀವತ್ಸ-ಗೋತ್ರಜಃ
ಶ್ರೀವಾಯು-ಪುರಾ-ಣಾಂತರ್ಗತ
ಶ್ರೀವಾಯು-ಪುರಾಣೇ
ಶ್ರೀವಿಷ್ಣು
ಶ್ರೀವಿಷ್ಣುನ
ಶ್ರೀವಿಷ್ಣು-ವನ್ನು
ಶ್ರೀವಿಷ್ಣು-ವಿಗೆ
ಶ್ರೀವಿಷ್ಣು-ವಿನ
ಶ್ರೀವಿಷ್ಣು-ವಿ-ನಲ್ಲಿ
ಶ್ರೀವಿಷ್ಣುವು
ಶ್ರೀಹರಿ
ಶ್ರೀಹ-ರಿಗೆ
ಶ್ರೀಹರಿಯ
ಶ್ರೀಹರಿ-ಯಂತೆ
ಶ್ರೀಹರಿ-ಯನ್ನು
ಶ್ರೀಹರಿ-ಯನ್ನೇ
ಶ್ರೀಹರಿ-ಯಲ್ಲಿ
ಶ್ರೀಹರಿ-ಯಲ್ಲಿಯೂ
ಶ್ರೀಹರಿ-ಯಲ್ಲಿಯೇ
ಶ್ರೀಹರಿ-ಯಿಂದ
ಶ್ರೀಹರಿಯು
ಶ್ರೀಹರಿಯೇ
ಶ್ರೀಹರಿ-ಯೊಬ್ಬನೇ
ಶ್ರುತಂ
ಶ್ರುತಾ
ಶ್ರುತಿ
ಶ್ರುತಿಃ
ಶ್ರುತಿ-ಕೀರ್ತಿ-ಯೆಂಬ
ಶ್ರುತಿ-ಕೀರ್ತಿ-ರಿತಿ
ಶ್ರುತಿ-ಗಳಲ್ಲಿ
ಶ್ರುತಿ-ಚೋದಿ-ತಮ್
ಶ್ರುತಿ-ಚೋದಿತಾ
ಶ್ರುತಿ-ಚೋದಿತಾಮ್
ಶ್ರುತಿಯು
ಶ್ರುತಿ-ಸಮ್ಮತ-ವಾದ
ಶ್ರುತಿಸ್ಕೃತಿ-ಗಳಲ್ಲಿ
ಶ್ರುತಿಸ್ತಥಾ-ಚಾರಃ
ಶ್ರುತಿಸ್ಮೃ
ಶ್ರುತಿಸ್ಮೃ-ತಿ-ಪುರಾ-ಣ-ಗಳಲ್ಲಿ
ಶ್ರುತಿಸ್ಮೃ-ತಿ-ಪುರಾ-ಣ-ಗಳು
ಶ್ರುತಿಸ್ಮೃ-ತಿ-ಪುರಾ-ಣಾದ್ಯೈರುದಿತೋ
ಶ್ರುತಿಸ್ಮೃ-ತಿ-ಪುರಾ-ಣಾನಾಂ
ಶ್ರುತಿಸ್ಮೃ-ತಿ-ಪುರಾ-ಣಾನಿ
ಶ್ರುತಿಸ್ಮೃ-ತಿ-ಪುರಾ-ಣೋಕ್ತಂ
ಶ್ರುತಿಸ್ಮೃತೀ
ಶ್ರುತಿಸ್ಮೃತ್ಯಾ-ದಿ-ಶಾಸ್ತ್ರವೇ
ಶ್ರುತೀರಿತಾ
ಶ್ರುತ್ವಾ
ಶ್ರೂಯತಾಮವಧಾ-ನೇನ
ಶ್ರೂಯತೇ
ಶ್ರೇಯಸ್ಸನ್ನು
ಶ್ರೇಯಸ್ಸಿಗೂ
ಶ್ರೇಯೋ
ಶ್ರೇಷ್ಠ
ಶ್ರೇಷ್ಠಂ
ಶ್ರೇಷ್ಠ-ತಮಂ
ಶ್ರೇಷ್ಠತೆ
ಶ್ರೇಷ್ಠ-ತೆ-ಯನ್ನು
ಶ್ರೇಷ್ಠ-ತೆ-ಯಲ್ಲಿ
ಶ್ರೇಷ್ಠ-ನಾಗಿದ್ದೆ
ಶ್ರೇಷ್ಠ-ನಾದ
ಶ್ರೇಷ್ಠ-ನಾದ-ವನು
ಶ್ರೇಷ್ಠನೂ
ಶ್ರೇಷ್ಠ-ನೆಂದು
ಶ್ರೇಷ್ಠನೋ
ಶ್ರೇಷ್ಠ-ರಾದ
ಶ್ರೇಷ್ಠರು
ಶ್ರೇಷ್ಠರೇ
ಶ್ರೇಷ್ಠ-ವಾದ
ಶ್ರೇಷ್ಠ-ವಾದ-ವು-ಗಳು
ಶ್ರೇಷ್ಠ-ವಾ-ದುದು
ಶ್ರೇಷ್ಠ-ವೆಂದು
ಶ್ರೇಷ್ಠವೇ
ಶ್ರೇಷ್ಠಸ್ತತ್ತ
ಶ್ರೋತುಂ
ಶ್ರೋತುಮೇ-ತದ್ವಿಸ್ತರತೋ
ಶ್ರೋತೃತತ್ವಕ್ಕೂ
ಶ್ರೋತೄನ್
ಶ್ರೋತ್ರ-ತತ್ವಾಭಿ-ಮಾನಿನಃ
ಶ್ರೋತ್ರಿಯಂ
ಶ್ರೋತ್ರಿಯಾಃ
ಶ್ರೋತ್ರಿಯಾಯ
ಶ್ಲಾಘಿ-ಸಲ್ಪಟ್ಟು
ಶ್ಲೋಕ-ಗಳಲ್ಲಿ
ಶ್ಲೋಕ-ಗಳಲ್ಲಿಯೂ
ಶ್ಲೋಕ-ವನ್ನು
ಶ್ಲೋಕ-ವನ್ನೂ
ಶ್ವಪಚೀ-ಮಪಿ
ಶ್ವಾನ-ಮಾತ್ರಂ
ಶ್ವಾನಯೋನಿ-ಶತಂ
ಶ್ವಿತ್ರರೋಗಾದ್ವಿ-ಮುಚ್ಯತೇ
ಶ್ವೇತ
ಶ್ವೇತದ್ವೀಪ
ಶ್ವೇತದ್ವೀಪ-ದಲ್ಲಿ
ಶ್ವೇತದ್ವೀಪೇ
ಶ್ವೇತ-ನಾಮಾಭೂದಲಕಾದ್ವಾ-ರಪಾಲಕಃ
ಶ್ವೇತ-ನಾ-ಯಕ
ಶ್ವೇತ-ನಾ-ಯಕೇ
ಶ್ವೇತ-ನೆಂಬ
ಶ್ವೇತ-ವರಾಹ
ಶ್ವೇತ-ವರಾಹಂ
ಶ್ವೇತ-ಸರಸಿ
ಶ್ವೇತಸ್ತು
ಶ್ವೇತೊ
ಶ್ವೇತೋ
ಶ್ವೋ
ಷಟ್
ಷಡಂಗ
ಷಡಂಗಃ
ಷಡಪ್ಯಂಗಾನಿ
ಷಡಾತ್ಮಾ
ಷಡ್ಗುಣೈಶ್ವರ್ಯ-ನಾದ
ಷಡ್ಗುಣೈಶ್ವರ್ಯ-ಪೂರ್ಣ-ನಾದ
ಷಡ್ವಿಧಾಃ
ಷಡ್ವಿಧಾನಿ
ಷಷ್ಟಿ
ಷಷ್ಠಶ್ಚಾಪಿ
ಷಷ್ಠಾಂಶೋ
ಷಷ್ಠೀ
ಷಷ್ಠೇಽಹನಿ
ಷಷ್ಠೋ
ಷಷ್ಠೋಧ್ಯಾಯಃ
ಷಷ್ಠೋ-ಽಯಂ
ಷಾಡ್
ಷೋಡಶ-ವಾರ್ಷಿಕಃ
ಷೋಡಶಾದ್ಯಾನ್ನಂ
ಷೋಡಶೀಮ್
ಷೋಡಶೋಧ್ಯಾಯಃ
ಸ
ಸಂ
ಸಂಕಲ್ಪ
ಸಂಕಲ್ಪತೊ
ಸಂಕಲ್ಪ್ಯ
ಸಂಕೃತಿ
ಸಂಕೃತಿ-ಗೋತ್ರಜಃ
ಸಂಕೃತಿ-ರಿತ್ಯಪಿ
ಸಂಕ್ರಮ-ಣದ
ಸಂಕ್ರಮ-ಣ-ದಲ್ಲಿ
ಸಂಕ್ರಮ-ಣದ-ವರೆಗೂ
ಸಂಕ್ರಮ-ಣ-ದಿಂದ
ಸಂಕ್ರಮ-ಣ-ದಿವ-ಸ-ಗಳಲ್ಲಿ
ಸಂಕ್ರಮಾದಿತಃ
ಸಂಕ್ರಮೇಣ
ಸಂಕ್ರಮೇಷು
ಸಂಕ್ರಾಂತ-ಫ-ಲದಂ
ಸಂಕ್ರಾಂತಿ
ಸಂಕ್ರಾಂತಿ-ಶಶಿಗ್ರಹಾದಿ
ಸಂಕ್ಷಿಪ್ತ
ಸಂಕ್ಷಿಪ್ತ-ವಾಗಿ
ಸಂಕ್ಷೇಪ-ದಿಂದಲೂ
ಸಂಗ
ಸಂಗಃ
ಸಂಗಡ
ಸಂಗ-ಡಲೇ
ಸಂಗ-ಡಿಗ-ರಿಂದ
ಸಂಗ-ಡಿಗರೊಂದಿಗೆ
ಸಂಗತಿಂ
ಸಂಗತಿಃ
ಸಂಗ-ತಿರ್ಭವೇತ್
ಸಂಗ-ತಿರ್ಭಾವಿ-ತಾತ್ಮನಃ
ಸಂಗಮಃ
ಸಂಗ-ಮ-ಗಳಲ್ಲಿ
ಸಂಗ-ಮೋ-ಽಪಿ
ಸಂಗ-ವತಾಂ
ಸಂಗಸ್ತದೈವಾತ್ಮಾ
ಸಂಗೀತಗಾರ-ನಾಗಿದ್ದೆ
ಸಂಗೋಸ್ತ್ವ-ಕರ್ಮಣಿ
ಸಂಗ್ರಹಿ-ಸಲು
ಸಂಗ್ರಹಿ-ಸಿದೆ
ಸಂಚರನ್
ಸಂಚರಿಸಿ
ಸಂಚರಿ-ಸುತ್ತಲೇ
ಸಂಚರಿ-ಸುತ್ತಾ
ಸಂಚರಿ-ಸುತ್ತಿದ್ದನು
ಸಂಚಾರ-ಮಾಡು-ತಿ-ರುವಿ
ಸಂಜಾ-ಯತೇ
ಸಂಜೆ-ಯಲ್ಲಿ
ಸಂಜ್ಞತಃ
ಸಂತಃ
ಸಂತ-ತಮ್
ಸಂತತ-ವಾಗಿ
ಸಂತತಿ
ಸಂತತಿಃ
ಸಂತ-ತಿಗೆ
ಸಂತತಿಚ್ಛೇದೋ
ಸಂತತಿ-ಮಿಚ್ಛತಾ
ಸಂತತಿಯ
ಸಂತತಿ-ಯನ್ನು
ಸಂತತಿ-ಯಲ್ಲಿ
ಸಂತತಿ-ಯಾಗ-ಲಿಲ್ಲ
ಸಂತತಿ-ಯಿಂದ
ಸಂತತಿ-ಯಿಲ್ಲ-ದಿದ್ದರೆ
ಸಂತತಿಯು
ಸಂತತಿಯೂ
ಸಂತತಿರ್ವರ್ಧತಾಂ
ಸಂತತೇಃ
ಸಂತತೇರ್ಭವೇತ್
ಸಂತತೇರ್ಯೂಯಮಧಃಪ-ತನ-ವಾಪ್ಸ್ಯಥ
ಸಂತತೇರ್ಹಾನಿರಾ-ಕಲ್ಪಂ
ಸಂತತೇರ್ಹಾನಿರ್ನ
ಸಂತತೇರ್ಹಾನಿರ್ಭುಕ್ತಿಂ
ಸಂತತೇಸ್ತಸ್ಯ
ಸಂತತ್ಯಾ
ಸಂತಪ್ತಾ
ಸಂತರ್ಪಣೆ
ಸಂತರ್ಪ್ಯ
ಸಂತಾನ-ಹೀನ-ರಾದ
ಸಂತಾನ-ಹೀನ-ಳಾಗಲೀ
ಸಂತಾರಯಿಷ್ಯತಿ
ಸಂತಿ
ಸಂತುಷ್ಟಃ
ಸಂತುಷ್ಟ-ನಾಗಿದ್ದಾನೆ
ಸಂತುಷ್ಟ-ನಾದ
ಸಂತುಷ್ಟ-ರಾ-ದರು
ಸಂತೋಷ
ಸಂತೋಷ-ಗೊಂಡ
ಸಂತೋಷ-ಗೊಂಡು
ಸಂತೋಷ-ದಿಂದ
ಸಂತೋಷ-ದಿಂದ-ಕೂಡಿದ
ಸಂತೋಷ-ಪಡಿಸಿ
ಸಂತೋಷ-ಪಡಿಸಿ-ದನು
ಸಂತೋಷ-ಯತಿ
ಸಂತೋಷ-ವಾಗಿದೆ
ಸಂತೋಷ-ವಾ-ಯಿತು
ಸಂತೋಷವೇ
ಸಂತೋಷಾತ್ತಂ
ಸಂತೋಷೋ
ಸಂದದ್ಯಾತ್ಪುತ್ರಾನ್
ಸಂದರ್ಭ-ಗಳಲ್ಲಿ
ಸಂದರ್ಭ-ದಲ್ಲಿ
ಸಂದೇಹಃ
ಸಂದೇಹ-ವಿಲ್ಲ
ಸಂದೇಹ-ವಿಲ್ಲದೇ
ಸಂದೇಹವೇ
ಸಂದೇಹೋ
ಸಂದ್ರಷ್ಟುಮ-ಗಮನ್
ಸಂಧಿ-ತಾನಿ
ಸಂಧ್ಯಾ
ಸಂಧ್ಯಾಂ
ಸಂಧ್ಯಾ-ಕಾಲದ
ಸಂಧ್ಯಾಪಿ
ಸಂಧ್ಯಾ-ರುಣಾಂಬರಾನೇ-ತಾನ್
ಸಂಧ್ಯಾ-ವಂದ-ನಾದಿ
ಸಂಧ್ಯಾ-ವಂದನೆ
ಸಂಧ್ಯಾ-ವಂದನೆ-ಯನ್ನು
ಸಂನಿಧೌ
ಸಂನಿ-ಹಿತಾ
ಸಂನ್ಯಸ್ಯಾಧ್ಯಾತ್ಮ-ಚೇತಸಾ
ಸಂಪತ್ತನ್ನೂ
ಸಂಪದ್ಯುಕ್ತ-ರಾದ
ಸಂಪರ್ಕ
ಸಂಪರ್ಕತ್ಪಾಪಿನೋ
ಸಂಪರ್ಕ-ದಿಂದ
ಸಂಪರ್ಕ-ವಿಲ್ಲದೆ
ಸಂಪರ್ಕವೂ
ಸಂಪಾದನೆ
ಸಂಪಾದನೆ-ಗಾಗಿ
ಸಂಪಾದಿತಾ
ಸಂಪಾದಿ-ಸ-ಬೇಕೆಂಬ
ಸಂಪಾದಿ-ಸ-ಲಿಲ್ಲ
ಸಂಪಾದಿ-ಸಲು
ಸಂಪಾದಿಸಿ
ಸಂಪಾದಿ-ಸಿದ
ಸಂಪಾದಿ-ಸಿ-ದನು
ಸಂಪಾದಿ-ಸಿ-ದಳು
ಸಂಪಾದಿ-ಸಿದೆ
ಸಂಪಾದಿ-ಸಿ-ದೆವು
ಸಂಪಾದಿ-ಸಿದ್ದ
ಸಂಪಾದಿ-ಸುವ
ಸಂಪಾದಿ-ಸುವುದ-ರಲ್ಲಿ
ಸಂಪಿಗೆ
ಸಂಪಿಷ್ಟ್ವಾ
ಸಂಪೂಜ್ಯ
ಸಂಪೂಜ್ಯಾಢಕ-ದಾನಾನಿ
ಸಂಪೂರ್ಣ
ಸಂಪೂರ್ಣಂ
ಸಂಪೂರ್ಣ-ಫಲ-ಮಶ್ನುತೇ
ಸಂಪೂರ್ಣ-ಮುಪಾಯೈರ್ಬಹು-ಧಾರ್ಜಿ-ತಮ್
ಸಂಪ್ರಶ್ನೋ
ಸಂಪ್ರಾಪ್ತೇ
ಸಂಪ್ರಾರ್ಥ-ಯಚ್ಚಿಕಿತ್ಸಾರ್ಥಂ
ಸಂಬಂಧ
ಸಂಬಂಧದ
ಸಂಬಂಧ-ದಿಂದ
ಸಂಬಂಧ-ಪಟ್ಟ
ಸಂಬಂಧ-ವುಳ್ಳ
ಸಂಬಂಧಿ-ಸಿದ
ಸಂಬಳಕ್ಕಿಂತ
ಸಂಬೋಧ್ಯ
ಸಂಭವಿಸಲಿ
ಸಂಭವಿ-ಸಿತು
ಸಂಭವಿ-ಸಿ-ರುವು-ದಾಗಿ
ಸಂಭವಿಸು-ವುದಿಲ್ಲ
ಸಂಭಾರಾಂಸ್ತಾಮ್ರಪಾತ್ರಾಣಿ
ಸಂಭೂತಃ
ಸಂಭೂತಾ
ಸಂಭೂತೇ
ಸಂಭೂಯ
ಸಂಮ-ತಮ್
ಸಂಯತೇಂದ್ರಿಯಃ
ಸಂಯಮಾನಃ
ಸಂಯಮ್ಯ
ಸಂಯುತಾ
ಸಂಯೋಗೇ
ಸಂವಿಚ್ಚ
ಸಂವಿತ್
ಸಂವೃತೋ
ಸಂಶಯಃ
ಸಂಶಯಗ್ರಸ್ತ-ನಿಗೆ
ಸಂಶಯಮ್
ಸಂಶಯ-ಯುಕ್ತ-ನಾದ-ವನೂ
ಸಂಶಯ-ರ-ಹಿತ-ನಾದ
ಸಂಶಯ-ವಿಲ್ಲ
ಸಂಶಯ-ವಿಲ್ಲದೇ
ಸಂಶ-ಯಾತ್ಮನಃ
ಸಂಶಯಾತ್ಮಾ
ಸಂಶಿತವ್ರತಮ್
ಸಂಶ್ರ-ಯೇತ್
ಸಂಶ್ರಾವ್ಯ
ಸಂಸರತಿ
ಸಂಸರತೇ
ಸಂಸರ್ಗಾತ್
ಸಂಸಾರ
ಸಂಸಾ-ರಕ್ಕೆ
ಸಂಸಾರದ
ಸಂಸಾರ-ದಲ್ಲಿ
ಸಂಸಾರ-ದಲ್ಲಿ-ರುವ
ಸಂಸಾರ-ದಿಂದ
ಸಂಸಾರ-ದುಃಖಂ
ಸಂಸಾರ-ದುಃಖ-ವನ್ನು
ಸಂಸಾರ-ಬಂಧನ-ವನ್ನು
ಸಂಸಾರ-ವನ್ನು
ಸಂಸಾರ-ವರ್ತಿಭಿಃ
ಸಂಸಾರ-ವೆಂಬ
ಸಂಸಾರ-ಸಮುದ್ರ-ವನ್ನು
ಸಂಸಾರ-ಸರ್ಪ-ದಷ್ಟಾನಾಂ
ಸಂಸಾರ-ಸಾಗರ-ದಲ್ಲಿ
ಸಂಸಾರಾಖ್ಯಃ
ಸಂಸಾರಾಬ್ಧೌ
ಸಂಸಾರೇ
ಸಂಸಾರೋತ್ತಾರ-ಕಾರ-ಕಮ್
ಸಂಸ್ಕಾರ-ವನ್ನು
ಸಂಸ್ಕಾರ-ವಾಗದೇ
ಸಂಸ್ಕೃತಿಬುದ್ಧಾಸ್ತೇ
ಸಂಸ್ಕೃತಿಮೇತಿ
ಸಂಸ್ಥಾ
ಸಂಸ್ಥಾನ
ಸಂಸ್ಥಿ
ಸಂಸ್ಥೆ
ಸಂಸ್ನಾಪ್ಯ
ಸಂಸ್ಮ-ರೇತ್
ಸಂಸ್ಮ-ರೇದ್ಯದಿ
ಸಂಹರಿ-ಸಲು
ಸಂಹರಿ-ಸಲ್ಪಟ್ಟು
ಸಂಹರಿ-ಸಲ್ಪಟ್ಟೆವು
ಸಂಹರಿಸಿ
ಸಂಹರಿ-ಸಿತು
ಸಂಹರಿ-ಸಿದ
ಸಂಹರಿ-ಸಿ-ದನು
ಸಂಹರಿ-ಸಿ-ದರೂ
ಸಂಹರಿ-ಸಿ-ದೆವು
ಸಂಹರಿಸು
ಸಂಹರಿ-ಸು-ವುದು
ಸಂಹಾರ
ಸಂಹಾರ-ಇವೆಲ್ಲವೂ
ಸಃ
ಸಕಲ
ಸಕಲಂ
ಸಕಲ-ಕಾರ್ಯ-ಗಳನ್ನೂ
ಸಕಲ-ದೇವ-ತೆ-ಗಳೂ
ಸಕಲ-ಧರ್ಮ-ಗಳಲ್ಲಿಯೂ
ಸಕಲ-ಧರ್ಮೇಭ್ಯೋ
ಸಕಲರೂ
ಸಕಲಾ
ಸಕಲಾನ್
ಸಕಾಮಂ
ಸಕಾಮ-ಕರ್ಮ-ವನ್ನು
ಸಕಾಮ-ವಾಗಿ-ರ-ಬಹುದು
ಸಕಾಲ-ದಲ್ಲಿ
ಸಕಾಶಾಚ್ಛಾಧೀತಂ
ಸಕೃತ್
ಸಕೃದಭ್ಯರ್ಚ್ಯ
ಸಕೃದ್ವಾಪಿ
ಸಕ್ತಃ
ಸಖಾಯಂ
ಸಖಾಯಾ
ಸಖಾಸ್ತಿ
ಸಖಿ-ಯನ್ನು
ಸಖಿಯ-ರೊಡನೆ
ಸಖೀಂ
ಸಖೀ-ಜನ-ರಿಂದ
ಸಖೀ-ಜನರು
ಸಖೀ-ಜನಸ್ತಸ್ಯಾಃ
ಸಖ್ಯಂ
ಸಖ್ಯಾ
ಸಖ್ಯೇನ
ಸಗದ್ಗ
ಸಗೋಪ್ಯೋ
ಸಘೋಷಾಶ್ಚ
ಸಚ-ತಾಪ-ರುಷ್ಣ್ಯಾ
ಸಚೇನ್ಮಾಘ್ಯಾಂ
ಸಚ್ಚಾ
ಸಚ್ಚಾತ್ರಾಭ್ಯಾಸ
ಸಚ್ಛಾ
ಸಚ್ಛಾಸ್ತ್ರ
ಸಚ್ಛಾತ್ರ-ಗಳ
ಸಚ್ಛಾತ್ರ-ಗಳಲ್ಲಿ
ಸಜ್ಜನ
ಸಜ್ಜನರ
ಸಜ್ಜನ-ರನ್ನು
ಸಜ್ಜನ-ರಲ್ಲಿಯೂ
ಸಜ್ಜನ-ರಿಗೆ
ಸಜ್ಜನರು
ಸಜ್ಜನ-ಸ-ಮುದಾ-ಯಕ್ಕೆ
ಸಜ್ಞನ
ಸಜ್ಞಾನಂ
ಸಣ್ಣ
ಸತತಂ
ಸತಾಂ
ಸತಾಂಬೂಲಂ
ಸತಾಮೇತ್ಯ
ಸತಾಮ್
ಸತೀ
ಸತೀಮ್
ಸತುಷಂ
ಸತ್ಕಥಾ
ಸತ್ಕಥಾಶ್ರವಣ-ವಾದ
ಸತ್ಕಥಾಸು
ಸತ್ಕರಿ-ಸ-ಲಿಲ್ಲ
ಸತ್ಕರಿಸಲ್ಪಟ್ಟ
ಸತ್ಕರಿಸಿ
ಸತ್ಕರಿಸು
ಸತ್ಕರ್ಮ
ಸತ್ಕರ್ಮಕ್ಕೆ
ಸತ್ಕರ್ಮ-ಗಳ
ಸತ್ಕರ್ಮ-ಗಳನ್ನಾ-ಚರಿಸಿ
ಸತ್ಕರ್ಮ-ಗಳನ್ನು
ಸತ್ಕರ್ಮ-ಗಳನ್ನೂ
ಸತ್ಕರ್ಮ-ಗಳಲ್ಲಿ
ಸತ್ಕರ್ಮ-ಗಳಲ್ಲಿಯೂ
ಸತ್ಕರ್ಮ-ಗಳಿ-ಗಾಗಿ
ಸತ್ಕರ್ಮ-ಗಳಿಗೆ
ಸತ್ಕರ್ಮ-ಗಳು
ಸತ್ಕರ್ಮ-ಗಳೂ
ಸತ್ಕರ್ಮ-ಗಳೆಲ್ಲವೂ
ಸತ್ಕರ್ಮದ
ಸತ್ಕರ್ಮ-ದಿಂದ
ಸತ್ಕರ್ಮ-ಮಾಡಲು
ಸತ್ಕರ್ಮ-ಮಾಡುವ
ಸತ್ಕರ್ಮ-ವನ್ನು
ಸತ್ಕರ್ಮ-ವನ್ನೂ
ಸತ್ಕರ್ಮವು
ಸತ್ಕರ್ಮ-ಸಂಗತಿಃ
ಸತ್ಕರ್ಮಾಚ-ರಣೆ-ಯಲ್ಲಿಯೇ
ಸತ್ಕರ್ಮಾಚ-ರಣೆ-ಯಿಂದ
ಸತ್ಕರ್ಮಾನುಷ್ಠಾನ
ಸತ್ಕರ್ಮಾನುಷ್ಠಾನ-ವಿಲ್ಲದೇ
ಸತ್ಕಾರ
ಸತ್ಕಾರಾ-ದಿ-ಗಳನ್ನು
ಸತ್ಕೀರ್ತಿಮಂತಂ
ಸತ್ಕ್ರಿಯಾ
ಸತ್ಕ್ರಿಯಾಃ
ಸತ್ತನು
ಸತ್ತ-ವ-ರಿಗೆ
ಸತ್ತಾ
ಸತ್ತಾ-ರಂಭವಿ-ಹೀನಾ
ಸತ್ತಿಲ್ಲ-ವಾದುದ-ರಿಂದ
ಸತ್ತು
ಸತ್ತ್ವ
ಸತ್ತ್ವಾ-ದಯೋ
ಸತ್ಪಾತ್ರ-ರಿಗೆ
ಸತ್ಪುತ್ರ-ನಾದ
ಸತ್ಪುತ್ರ-ರನ್ನು
ಸತ್ಪುತ್ರ-ರನ್ನೂ
ಸತ್ಪುತ್ರರು
ಸತ್ಪುರುಷರು
ಸತ್ಯ
ಸತ್ಯಂ
ಸತ್ಯ-ನಾದೀ
ಸತ್ಯ-ಪರಾ-ಯಣಃ
ಸತ್ಯಪ್ರಮೋದತೀರ್ಥ
ಸತ್ಯಪ್ರಮೋದತೀರ್ಥಾಖ್ಯಂ
ಸತ್ಯ-ಭಾಷಣಃ
ಸತ್ಯ-ಭೂತ-ವಾದ
ಸತ್ಯರ್ಮ
ಸತ್ಯರ್ಮಾನುಷ್ಠಾನ
ಸತ್ಯ-ಲೋಕ-ದಲ್ಲಿ
ಸತ್ಯ-ಲೋಕವೇ
ಸತ್ಯ-ವಂತ-ನಾದ
ಸತ್ಯ-ವನ್ನೇ
ಸತ್ಯ-ವ-ರಿಗೆ
ಸತ್ಯ-ವಾಗಿ
ಸತ್ಯ-ವಾಗಿದ್ದರೆ
ಸತ್ಯ-ವೆಂದು
ಸತ್ಯವೋ
ಸತ್ಯ-ಶೌಚ-ದಯಾ-ಹೀನಾ
ಸತ್ಯಸ್ಯ
ಸತ್ಯಾ-ದಯೋ
ಸತ್ಯಾ-ನು-ರೂಪಾ
ಸತ್ಯಾ-ಭಿ-ಮಾನಿಭಿಃ
ಸತ್ರೀ
ಸತ್ವ
ಸತ್ವ-ರಜ-ತಮೋ-ಗುಣ-ಗಳಿಂದ
ಸತ್ವ-ರಮ್
ಸತ್ವಾದ್ಯಾ
ಸತ್ವೋ
ಸತ್ಸಂಗಃ
ಸತ್ಸಂಗ-ತಿರ್ಭವೇತ್
ಸತ್ಸಂಗ-ತಿಸ್ತಥಾ
ಸದಕ್ಷಿಣಂ
ಸದ-ಸನ್ನೈವ
ಸದಾ
ಸದಾಭ್ಯಾಸೀ
ಸದಾ-ಽಹಂಕಾರ-ದೂಷಿತಃ
ಸದುಪ-ದೇಶ-ವನ್ನು
ಸದೃಶ
ಸದೃಶಂ
ಸದೃಶ-ರಾಗಿದ್ದೆವು
ಸದೃಶ-ವಾದ
ಸದೃಶಾ-ಕಾರಂ
ಸದೃಶೋ
ಸದೈವಾತಿ-ಕೋ-ಪನಃ
ಸದ್ಗತಿಂ
ಸದ್ಗತಿಃ
ಸದ್ಗ-ತಿಮ್
ಸದ್ಗತಿ-ಯನ್ನು
ಸದ್ಗತಿ-ಯನ್ನೂ
ಸದ್ಗತಿ-ಯಾಗ-ಬೇಕೆಂಬ
ಸದ್ಗತಿ-ಯಾಗುತ್ತದೆ
ಸದ್ಗತಿ-ಯಾ-ಯಿತು
ಸದ್ಗತಿಯು
ಸದ್ಗುರು-ಗಳ
ಸದ್ಧತಿ
ಸದ್ಧತಿಯ
ಸದ್ಧತಿಯು
ಸದ್ಭಕ್ತ್ಯಾ
ಸದ್ಭ್ಯೋಽ-ನಿಶಂ
ಸದ್ಯಃ
ಸದ್ಯಸ್ತಾನ್ಪುರುಷಾನ್ಮುನೇ
ಸನಂದ-ರೆಂಬ
ಸನ-ಕಾಲೇ
ಸನಾ-ತನ
ಸನಾ-ತನಃ
ಸನಾಮಕೋ
ಸನಾ-ಮಾಸೌ
ಸನಾರೀ-ಕಾವಸ್ಮತ್
ಸನ್
ಸನ್ನಿಧಾನ
ಸನ್ನಿಧಾನಂ
ಸನ್ನಿಧಾನ-ದಿಂದ
ಸನ್ನಿಧಾನ-ವನ್ನು
ಸನ್ನಿಧಿ-ಯಲ್ಲಿ
ಸನ್ನಿಭೇ
ಸನ್ನಿ-ಹಿತಃ
ಸನ್ನಿ-ಹಿತ-ನಾ-ಗಿ-ರುವನು
ಸನ್ನಿ-ಹಿತ-ರಾಗಿ-ರು-ವುದ-ರಿಂದ
ಸನ್ನಿ-ಹಿತ-ರಾ-ಗಿ-ರುವು-ವೆಂದು
ಸನ್ನಿ-ಹಿತಸ್ತತ್ರ
ಸನ್ನಿ-ಹಿತಾ
ಸನ್ಮನಾಃ
ಸನ್ಮಾನಿಸಿ
ಸನ್ಮಾರ್ಗಕ್ಕೆ
ಸನ್ಯಸ್ಯ-ಮಾನಸ್ತು
ಸಪತ್ನಾನ್
ಸಪತ್ನೀಂ
ಸಪರ್ಯಾತು
ಸಪಾದಜಃ
ಸಪ್ತ
ಸಪ್ತ-ಕುಲ-ಗಳನ್ನು
ಸಪ್ತ-ಗಂಗಾ
ಸಪ್ತ-ಗಂಗಾಃ
ಸಪ್ತ-ಗಂಗಾಸು
ಸಪ್ತ-ಜನ್ಮಸು
ಸಪ್ತ-ನದೀ-ಸಂಗೇ
ಸಪ್ತಮಂ
ಸಪ್ತ-ಮಸ್ತು
ಸಪ್ತ-ಮಿ-ಯಂದು
ಸಪ್ತ-ಮಿ-ಯಲ್ಲಿ
ಸಪ್ತಮೀ
ಸಪ್ತಮೋ
ಸಪ್ತ-ಮೋಧ್ಯಾಯಃ
ಸಪ್ತ-ಮೋ-ಽದೈ-ವತೋ
ಸಪ್ತಮ್ಯಾ
ಸಪ್ತಮ್ಯಾಂ
ಸಪ್ತರ್ಷಯೋ
ಸಪ್ತ-ವಿಂಶತಿ
ಸಪ್ತ-ಶಾಖಾಸ-ವನ್ನಿ-ತ-ನಮ್
ಸಪ್ತಾದ್ಭುತಾಃ
ಸಪ್ತಾನಾಂ
ಸಪ್ತಾಪಿ
ಸಪ್ತೈತೇ
ಸಫಲಂ
ಸಫಲ-ವಾ-ಯಿತು
ಸಬಲಃ
ಸಭಾಂ
ಸಭಾಂಗ-ಣದ
ಸಭಾಂತರೇ
ಸಭಾಮ್
ಸಭಾಯಾಂ
ಸಭಾರ್ಯಃ
ಸಭೆಗೆ
ಸಭೆ-ಯಲ್ಲಿ
ಸಭೆ-ಯೊಳಗೆ
ಸಮಗ್ರ
ಸಮ-ಚಿತ್ತಾನಾಂ
ಸಮದರ್ಶಿನಃ
ಸಮದರ್ಶಿನೋ
ಸಮ-ನಾಗಿ
ಸಮ-ನಾದ
ಸಮನು-ಅಂದರೆ
ಸಮನೆ
ಸಮನ್ವಿತಃ
ಸಮ-ಭಾಗಿನಃ
ಸಮಭ್ಯರ್ಚ್ಯ
ಸಮ-ಮಾಹುರ್ಮನೀಷಿಣಃ
ಸಮಯ-ದಲ್ಲಿ
ಸಮಯವೇ
ಸಮ-ರಥೌ
ಸಮರ್ಥನಲ್ಲ
ಸಮರ್ಥ-ನಾಗಿದ್ದರೂ
ಸಮರ್ಥ-ನಾ-ಗಿ-ರುವಿ
ಸಮರ್ಥನು
ಸಮರ್ಥರಾಗು-ವುದಿಲ್ಲ
ಸಮರ್ಥರು
ಸಮರ್ಥ-ವಾದ
ಸಮರ್ಥ-ಶಾಲಿಯೇ
ಸಮರ್ಥೋ
ಸಮರ್ಪಣೆ
ಸಮರ್ಪ-ಯತಿ
ಸಮರ್ಪ-ಯೇತ್
ಸಮರ್ಪಿಣಃ
ಸಮರ್ಪಿ-ತಮ್
ಸಮರ್ಪಿಸದೇ
ಸಮರ್ಪಿಸ-ಬೇಕು
ಸಮರ್ಪಿಸಿ
ಸಮರ್ಪಿ-ಸಿದ
ಸಮರ್ಪಿಸಿ-ದರೆ
ಸಮರ್ಷಣೆ
ಸಮವಾಪ್ನೋತಿ
ಸಮವಿಭೂಷಣೌ
ಸಮ-ವೆಂದು
ಸಮಸ್ತ
ಸಮಸ್ತ-ಭೋಗ-ಗಳಿಂದ
ಸಮಸ್ತ-ರಿಂದಲೂ
ಸಮಸ್ತಶಃ
ಸಮಸ್ತಾಃ
ಸಮಸ್ಯ
ಸಮಸ್ಯ-ಶಾಸ್ತ್ರ-ಗಳನ್ನೂ
ಸಮಸ್ಯೆ
ಸಮಾಂಬರೌ
ಸಮಾಕರ್ಷಂತಿ
ಸಮಾ-ಕಾರೌ
ಸಮಾಖ್ಯಾತಂ
ಸಮಾಖ್ಯೆಯಂ
ಸಮಾ-ಗಚ್ಛಂತಿ
ಸಮಾ-ಗತಃ
ಸಮಾ-ಗತಾ
ಸಮಾ-ಗ-ತಾನ್
ಸಮಾ-ಗಮನ್
ಸಮಾಗಮ-ವಾದರೆ
ಸಮಾಗಮಾತ್
ಸಮಾ-ಚ-ರೇತ್
ಸಮಾಚಾರ-ಗಳನ್ನೂ
ಸಮಾಚಾರೌ
ಸಮಾಜ
ಸಮಾಜಕ್ಕೆ
ಸಮಾಜ-ದಲ್ಲಿ
ಸಮಾತ್ರಾಃ
ಸಮಾದಿ-ದೇಶಾತ್ಮಭುವೋದ್ಭವಾ-ನಾಮ್
ಸಮಾಧಾನಗೊಳಿಸಿ
ಸಮಾಧಾನ-ಪಡಿಸಿ
ಸಮಾಧಿಕ-ರ-ಹಿತನೂ
ಸಮಾನ
ಸಮಾನಂ
ಸಮಾನ-ಕುಲ-ಗೋತ್ರಾಯ
ಸಮಾನ-ನಾ-ದ-ವನು
ಸಮಾನ-ವಾಗುತ್ತದೆ
ಸಮಾನ-ವಾದ
ಸಮಾನೋ
ಸಮಾಪೇದೇ
ಸಮಾಪ್ತಯೇ
ಸಮಾಪ್ತಿ
ಸಮಾಪ್ತಿ-ಗೋಸ್ಕರ
ಸಮಾಪ್ತಿ-ಯಾ-ಯಿತು
ಸಮಾಪ್ಯಾಥ
ಸಮಾಭೂತಾ
ಸಮಾ-ಯಯುಃ
ಸಮಾರಭ್ಯ
ಸಮಾವಿಶೇತ್
ಸಮಾ-ವಿಷ್ಟೋ
ಸಮಾವೃತಾಃ
ಸಮಾಶ್ರಿತೌ
ಸಮಾಶ್ರಿತ್ಯ
ಸಮಾಸಃ
ಸಮಾಸತೇ
ಸಮಾ-ಸಾದ್ಯ
ಸಮಾ-ಹಿತಃ
ಸಮಿತ್
ಸಮಿತ್ಕು
ಸಮಿತ್ತು
ಸಮಿತ್ಪಾಣಿಃ
ಸಮಿತ್ರ-ನೆಂಬು-ವನ
ಸಮೀಕ್ಷತೇ
ಸಮೀಪ-ದಲ್ಲಿ-ಯಾಗಲೀ
ಸಮೀಪಿಸಿ-ದನು
ಸಮೀರಿ-ತಮ್
ಸಮೀರಿತಾ
ಸಮುಚ್ಚಾರ್ಯಾಕರೊತ್
ಸಮುತ್ಥಾಯ
ಸಮುತ್ಪನ್ನಃ
ಸಮು-ದಿನಂ
ಸಮುದ್ದಿಶ್ಯ
ಸಮುದ್ಧರ
ಸಮುದ್ಧೃತ್ಯ
ಸಮುದ್ಭವಃ
ಸಮುದ್ರ
ಸಮುದ್ರ-ಇವು-ಗಳನ್ನು
ಸಮುದ್ರಕ್ಕೆ
ಸಮುದ್ರ-ಗಳು
ಸಮುದ್ರಗಾಃ
ಸಮುದ್ರ-ದಲ್ಲಿ
ಸಮುದ್ರವು
ಸಮುಪಾರ್ಜಿ-ತಮ್
ಸಮೂಲ-ಕಮ್
ಸಮೂಲ-ಘಾ-ತಮ್
ಸಮೂಹ-ಗಳಲ್ಲಿ
ಸಮೂಹ-ದಲ್ಲಿದ್ದ
ಸಮೂಹ-ದಲ್ಲಿಯೂ
ಸಮ್ಮತ-ವಾದ
ಸಮ್ಯಕ್
ಸಮ್ಯಕ್ಚೀರ್ಣಾನಿ
ಸಮ್ಯಗಭ್ಯರ್ಚ್ಯ
ಸಮ್ಯಗಾಚಮ್ಯ
ಸಮ್ಯಗ್ಗೋಮ-ಯೇನೋಪ-ಲಿಪ್ಯ
ಸಮ್ಯಗ್ಮಾಮುಜ್ಜೀವಯ
ಸಯಾತಿ
ಸಯುಜಾ
ಸರಯೂ
ಸರಯೂ-ತೀರೇ
ಸರ-ವಾದಿ-ಗಳಿಗೆ
ಸರಸಿ
ಸರಸ್ವತಿ
ಸರಸ್ವತೀ
ಸರಿ
ಸರಿತಃ
ಸರಿ-ಯಲ್ಲ-ವೆಂದು
ಸರಿ-ಯಾಗಿ
ಸರಿ-ಯಾದ
ಸರಿ-ಯಾದ-ವ-ನಿಗೆ
ಸರಿ-ಯೆಂದು
ಸರಿ-ಸು-ವುದು
ಸರುಕ್ಮಂ
ಸರೋ
ಸರೋ-ವರ
ಸರೋ-ವರಂ
ಸರೋ-ವ-ರಕ್ಕೆ
ಸರೋ-ವರ-ಗಳ
ಸರೋ-ವರ-ಗಳೂ
ಸರೋ-ವರ-ದಲ್ಲಿ
ಸರೋ-ವ-ರಮ್
ಸರೋ-ವರವು
ಸರೋ-ವರೇ
ಸರ್ಪ
ಸರ್ಪ-ಗಳಲ್ಲಿ
ಸರ್ಪ-ಗಳಾಗಲಿ
ಸರ್ಪ-ದಿಂದ
ಸರ್ಪ-ನಾಗಿ
ಸರ್ಪ-ರೂಪ-ದಿಂದ
ಸರ್ಪ-ವಾಗು
ಸರ್ಪವು
ಸರ್ಪವೂ
ಸರ್ವ
ಸರ್ವಂ
ಸರ್ವಃ
ಸರ್ವ-ಕರ್ತಾ
ಸರ್ವ-ಕರ್ತೃ-ತಾಮ್
ಸರ್ವ-ಕರ್ತೃತ್ವ
ಸರ್ವ-ಕರ್ತೃತ್ವ-ವನ್ನೂ
ಸರ್ವ-ಕರ್ತೃತ್ವಾತ್
ಸರ್ವ-ಕರ್ಮ
ಸರ್ವ-ಕರ್ಮ-ಗಳನ್ನೂ
ಸರ್ವ-ಕರ್ಮ-ಣಾಮ್
ಸರ್ವ-ಕರ್ಮ-ಬಹಿಷ್ಕೃತಾಃ
ಸರ್ವ-ಕರ್ಮ-ವಿಮೋಚ-ನಮ್
ಸರ್ವ-ಕಾರ್ಯ-ಗಳನ್ನೂ
ಸರ್ವ-ಕಾಲ-ದಲ್ಲಿಯ
ಸರ್ವ-ಕಿಲ್ಬಿಷೈಃ
ಸರ್ವಗ್ರಹಾಧಾರ
ಸರ್ವತಃ
ಸರ್ವ-ತೀರ್ಥಷು
ಸರ್ವ-ತೀರ್ಥಾನಿ
ಸರ್ವತ್ರ
ಸರ್ವತ್ರಶ್ರೀ-ವಿಷ್ಣುವೇ
ಸರ್ವತ್ರಾದೌ
ಸರ್ವದಾ
ಸರ್ವ-ದಾ-ನೇಷು
ಸರ್ವ-ದೇ-ವತಾಃ
ಸರ್ವ-ದೇಶೇ
ಸರ್ವ-ದೇಶೇಷು
ಸರ್ವ-ಧರ್ಮ-ಪರಾಙ್ಮುಖಃ
ಸರ್ವ-ಧರ್ಮ-ಪರೋ-ಽಪಿ
ಸರ್ವ-ಧರ್ಮ-ಬಹಿಷ್ಕೃತಃ
ಸರ್ವ-ಧರ್ಮಾ-ಲಯೋ
ಸರ್ವ-ಪಾಪೇಭ್ಯೋ
ಸರ್ವ-ಪಾಪೈರ್ನ
ಸರ್ವಪ್ರಾಣಿ-ಗಳ
ಸರ್ವ-ಬಂಧ-ವಿನಾ-ಶ-ನಮ್
ಸರ್ವ-ಬಂಧ-ವಿನಿರ್ಮುಕ್ತೋ
ಸರ್ವ-ಭೂ-ತಾನಾಂ
ಸರ್ವ-ಭೂತಾಶಯಸ್ಥಿತಃ
ಸರ್ವ-ಭೂ-ತೇಷು
ಸರ್ವ-ಭೋಗಸಮಾ-ಯುಕ್ತಾ
ಸರ್ವ-ಮನುಷ್ಠಿ
ಸರ್ವ-ಮಾಸೇಭ್ಯಃ
ಸರ್ವ-ರಿಂದಲೂ
ಸರ್ವ-ರಿಗೂ
ಸರ್ವರ್ತು-ಫಲ-ಪುಷ್ಪೈಶ್ಚ
ಸರ್ವ-ಲಕ್ಷಣ-ದಿಂದ
ಸರ್ವ-ವನ್ನೂ
ಸರ್ವ-ವರ್ಣಾಶ್ರಮ-ಗಳೂ
ಸರ್ವ-ವರ್ಣಾಶ್ರಮೈಃ
ಸರ್ವ-ವಿದ್ಯಾ
ಸರ್ವ-ವೃಕ್ಷಾಣಾಂ
ಸರ್ವ-ವೇದಾಶ್ಚ
ಸರ್ವಶಃ
ಸರ್ವ-ಸಂಪತ್ಸಮಾವೃತಃ
ಸರ್ವ-ಸಜ್ಜನ-ಸಂಮ-ತಮ್
ಸರ್ವ-ಸಿದ್ಧಿ-ಮವಾಪ್ನು-ಯಾತ್
ಸರ್ವಸ್ಯ
ಸರ್ವಸ್ವಂ
ಸರ್ವಸ್ವ-ದಕ್ಷಿಣಾಂ
ಸರ್ವಾ
ಸರ್ವಾಂ
ಸರ್ವಾಃ
ಸರ್ವಾಣಿ
ಸರ್ವಾಣ್ಯಪಿ
ಸರ್ವಾ-ಧಿಕಮ-ಸಂಶಯಮ್
ಸರ್ವಾ-ನುಷ್ಠಾ-ನಕ್ಕೆ
ಸರ್ವಾನ್
ಸರ್ವಾನ್ವೇ-ದಾನ್ನಧೀತ್ಯಾಹಂ
ಸರ್ವಾಸ್ತಾಸು
ಸರ್ವೆ
ಸರ್ವೆಶ್ವರ-ನಾದ
ಸರ್ವೆ-ಽಪಿ
ಸರ್ವೇ
ಸರ್ವೇ-ವರ್ಣಾಶ್ರಮಾ
ಸರ್ವೇಷಾಂ
ಸರ್ವೇ-ಷಾ-ಮೇವ
ಸರ್ವೇಷು
ಸರ್ವೇಷ್ವೇ-ತೇಷು
ಸರ್ವೇ-ಽಪಿ
ಸರ್ವೈ
ಸರ್ವೊತ್ತಮತ್ವ
ಸರ್ವೊತ್ತಮ-ನಾದ
ಸರ್ವೋತ್ತಮತ್ವ
ಸರ್ವೋತ್ತಮತ್ವಜ್ಞಾ-ನ-ದಿಂದ
ಸರ್ವೋತ್ತಮತ್ವಜ್ಞಾ-ನ-ದಿಂದಲೂ
ಸರ್ವೋತ್ತಮ-ನಾದ
ಸರ್ವೋತ್ತಮ-ನೆಂದು
ಸರ್ವೋತ್ತಮ-ನೆಂಬ
ಸರ್ವೋಽಪ್ಯೇವಂ
ಸರ್ಷಪ-ಮಾತ್ರಂ
ಸರ್ಷಪ-ಮಾತ್ರ-ಕಮ್
ಸರ್ಷಪ-ಮಾತ್ರೋ-ಽಪಿ
ಸಲ
ಸಲಹೆ-ಯಂತೆ
ಸಲ್ಲ
ಸಲ್ಲ-ಬೇಕಾದ
ಸಲ್ಲ-ಬೇಕಾ-ದು-ದನ್ನು
ಸಲ್ಲಲಿ
ಸಲ್ಲಿ-ಸಿದ
ಸಲ್ಲಿಸುತ್ತೇನೆ
ಸವತ್ಸಾಂ
ಸವಾ-ಚ-ರೇತ್
ಸವಾಸವಾಃ
ಸವಿತುಃ
ಸವಿತ್ರಸ್ಯಾತ್ಮಜಃ
ಸವಿಸರ್ಗಾಶ್ಚ
ಸವೃದ್ಧಿ
ಸವೋ-ಽಹಂ
ಸಹ
ಸಹ-ಕಾರ-ವಿಲ್ಲದೇ
ಸಹ-ಕಾರ-ವಿ-ಹೀ-ನಸ್ಯ
ಸಹಜ-ವಾದ
ಸಹತೇ
ಸಹ-ವಾರ್ಷಿಕಃ
ಸಹ-ವಾಸ
ಸಹ-ವಾಸ-ಗಳು
ಸಹ-ವಾಸ-ದಲ್ಲಿ
ಸಹ-ವಾಸ-ದಲ್ಲಿದ್ದು
ಸಹ-ವಾಸ-ದಲ್ಲಿ-ರುತ್ತಾರೆಯೋ
ಸಹ-ವಾಸ-ದಿಂದ
ಸಹ-ವಾಸ-ದಿಂದಲೂ
ಸಹ-ವಾಸ-ದೋಷ-ದಿಂದ
ಸಹ-ವಾಸ-ವನ್ನು
ಸಹ-ವಾಸವು
ಸಹಸಾ
ಸಹಸ್ರ
ಸಹಸ್ರ-ಕೋಟಿ
ಸಹಸ್ರ-ಗೋ-ದಾನ-ಫಲ-ವನ್ನು
ಸಹಸ್ರ-ನಾಮ
ಸಹಸ್ರ-ನಾಮಸ್ತೋತ್ರ
ಸಹಸ್ರ-ಫ-ಲದೋ
ಸಹಸ್ರಶಃ
ಸಹಸ್ರಾಣಿ
ಸಹಸ್ರಾರು
ಸಹಸ್ರಾರ್ಜುನ-ನೆಂಬ
ಸಹಸ್ರೈರ್ವಾ
ಸಹಾ-ಗತಾ
ಸಹಾ-ನುಗಃ
ಸಹಾಯ
ಸಹಾಯಂ
ಸಹಾ-ಯಯೌ
ಸಹಾಯ-ವನ್ನು
ಸಹಾಯ-ವಿಲ್ಲದೆ
ಸಹಾಯ-ವಿಲ್ಲದೇ
ಸಹಾ-ಯವು
ಸಹಾಯೌ
ಸಹಾಸ್ಮಾಭಿರ್ಭೋಕ್ತುಂ
ಸಹಿತ
ಸಹಿತಂ
ಸಹಿತ-ನಾಗಿ
ಸಹಿತ-ನಾದ
ಸಹಿತ-ರಾಗಿ
ಸಹಿತ-ರಾದ
ಸಹಿತ-ವಾಗಿ
ಸಹಿತ-ವಾದ
ಸಹಿತಾ
ಸಹಿತಾಃ
ಸಹಿತೋ
ಸಹಿಸಿ
ಸಹಿ-ಸುತ್ತಾನೆ
ಸಹಿಸು-ವುದಿಲ್ಲ
ಸಹೋದ-ರ-ನನ್ನು
ಸಹೋದ-ರ-ನಾದ
ಸಹೋದ-ರರು
ಸಾ
ಸಾಂಗನೋ
ಸಾಂಜನಂ
ಸಾಂತ-ಪೂರ್ವಂ
ಸಾಂಸಾರಿಕ
ಸಾಕಂ
ಸಾಕಮಗಾದ್ದ್ರು-ಮಮ್
ಸಾಕಮಭ್ಯಾಗಾಚ್ಚೋರೇಭ್ಯಶ್ಚ
ಸಾಕಲ್ಯೇನ
ಸಾಕಷ್ಟು
ಸಾಕಾಗಿ
ಸಾಕೀ
ಸಾಕ್ಷಾತ್
ಸಾಕ್ಷಾತ್ತಾಗಿ
ಸಾಕ್ಷಾತ್ತ್ವಯಾ
ಸಾಕ್ಷಿಚೇ-ತನಃ
ಸಾಕ್ಷಿ-ಯಾಗಿದ್ದು
ಸಾಗರಗಾಮಿನಿ
ಸಾಗರಗಾಮಿನ್ಯಾಂ
ಸಾಗರ-ದಲ್ಲಿ
ಸಾಗರ-ವನ್ನು
ಸಾಗರಾಃ
ಸಾಗರೇ
ಸಾಗಿಸುತ್ತಿದ್ದನು
ಸಾಜಾತ್ಯಂ
ಸಾತ್ವಿಕ
ಸಾತ್ವಿಕ-ಭಾವ-ವನ್ನು
ಸಾತ್ವಿ-ಕರು
ಸಾತ್ವಿಕಸ್ವಭಾವ-ವುಳ್ಳ
ಸಾತ್ವಿಕಾ
ಸಾತ್ವಿಕಾಃ
ಸಾತ್ವಿಕಾ-ಹಾರವು
ಸಾತ್ವಿಕೀ
ಸಾಥ
ಸಾದ-ರಮ್
ಸಾಧ-ಕನು
ಸಾಧ-ಕರು
ಸಾಧನ
ಸಾಧನಃ
ಸಾಧನ-ಪೂರ್ತಿ-ಯನ್ನು
ಸಾಧನ-ವಾಗು-ವುದಿಲ್ಲ
ಸಾಧನ-ವಾದ
ಸಾಧನ-ವಿಲ್ಲ
ಸಾಧನೆ
ಸಾಧನೆ-ಯನ್ನು
ಸಾಧ-ಯಿತ್ವಾ
ಸಾಧ-ಯೇತ್ಸು-ಸಮಾ-ಹಿತಃ
ಸಾಧವಃ
ಸಾಧವೋ
ಸಾಧಿತಾ
ಸಾಧಿತೊ
ಸಾಧಿಸಿ
ಸಾಧಿ-ಸುತ್ತಿದ್ದೆ
ಸಾಧು
ಸಾಧು-ಗಳು
ಸಾಧು-ವಾದ
ಸಾಧು-ವೃತ್ತಾನಾಂ
ಸಾಧು-ಸಜ್ಜನರ
ಸಾಧು-ಸಜ್ಜನ-ರನ್ನು
ಸಾಧೂದಿತಂ
ಸಾಧೂನಾಂ
ಸಾಧ್ಯ
ಸಾಧ್ಯಂ
ಸಾಧ್ಯ-ವಾಗ-ದಿದ್ದರೂ
ಸಾಧ್ಯ-ವಾಗ-ದಿದ್ದರೆ
ಸಾಧ್ಯ-ವಿಲ್ಲ
ಸಾಧ್ಯವೇ
ಸಾಧ್ಯಾಂಗ-ಗಳೆಂದು
ಸಾಧ್ವಸಂ
ಸಾನಂದಬಾಷ್ಪೇ
ಸಾನುಗೋ
ಸಾನುಸ್ವರ-ಪ-ದಾನಿ
ಸಾಪರೋಕ್ಷಾ
ಸಾಮ
ಸಾಮಗ್ರಿ-ಗಳನ್ನು
ಸಾಮರ್ಥ್ಯ
ಸಾಮರ್ಥ್ಯ-ವನ್ನೂ
ಸಾಮರ್ಥ್ಯ-ವಿಲ್ಲ
ಸಾಮರ್ಥ್ಯ-ವಿಲ್ಲದ
ಸಾಮರ್ಥ್ಯ-ವಿಲ್ಲ-ವಾದ-ಕಾರಣ
ಸಾಮರ್ಥ್ಯ-ವುಳ್ಳ-ವನೂ
ಸಾಮ-ವೇದ-ದಂತೆ
ಸಾಮಾನ್ಯ
ಸಾಯಂ
ಸಾಯಂಕಾಲದ
ಸಾಯಂಕಾಲ-ದಲ್ಲಿ
ಸಾಯಂಕಾಲ-ವಾ-ಗಲು
ಸಾಯಂಕಾಲವೇ
ಸಾಯಮಗ್ನಿಂ
ಸಾಯ-ಮತಿಥಿಂ
ಸಾಯಾಹ್ನಿ
ಸಾಯಾಹ್ನೇ
ಸಾಯಾಹ್ನೇ-ಽಪಿ
ಸಾಯಿ-ಸಿದೆವು
ಸಾಯುಜ್ಯ
ಸಾಯುಜ್ಯಂ
ಸಾಯುಜ್ಯ-ಮಾಪ್ನು-ಯಾತ್
ಸಾಯುಜ್ಯ-ಮೋಕ್ಷ-ವನ್ನು
ಸಾಯು-ವುದೂ-ಇಂತಹ
ಸಾರ
ಸಾರಿ
ಸಾರಿಸಿ
ಸಾರಿ-ಸಿ-ದ-ವನು
ಸಾರುತ್ತವೆ
ಸಾರ್ಥಕ-ವೆಂದು
ಸಾರ್ಥಾನಿ
ಸಾರ್ಧಂ
ಸಾರ್ಭಕಃ
ಸಾರ್ವಭೌಮಃ
ಸಾರ್ವಭೌಮೋ
ಸಾಲ
ಸಾಲದು
ಸಾಲ-ವನ್ನು
ಸಾಲಿಗ್ರಾಮ-ಪೂಜೆ
ಸಾವಧಾ-ನ-ದಿಂದ
ಸಾವಿರ
ಸಾವಿರಾರು
ಸಾಷ್ಟಾಂಗ
ಸಾಷ್ಟಾಂಗಂ
ಸಾಸಿವೆ
ಸಾಸಿವೆ-ಕಾಳಿ-ನಷ್ಟು
ಸಾಹಸ
ಸಾಹಾಯ್ಯಂ
ಸಾಹಿ
ಸಾಽದಹನ್ನಿಶಿ
ಸಿಂಚ
ಸಿಂಚಯ
ಸಿಂಧು-ತೀರೇ
ಸಿಂಧೂ-ನದೀ
ಸಿಂಪಡಿಸು
ಸಿಂಹ-ದಂತೆ
ಸಿಂಹಪ್ರಾಯ-ರಾದ
ಸಿಂಹವು
ಸಿಂಹೋ
ಸಿಕ್ಕರೆ
ಸಿಕ್ತೋ
ಸಿಗ-ದ-ವರು
ಸಿಗದಿದ್ದರೆ
ಸಿಗದೆ
ಸಿಗದೇ-ಹೋಗಲಿ
ಸಿಗುತ್ತದೆ
ಸಿಗುತ್ತವೆ-ಯೆಂಬ
ಸಿಟ್ಟಿ-ನಿಂದ
ಸಿಟ್ಟಿ-ನಿಂದ-ಇತರ
ಸಿಡಿಲಿನ
ಸಿಡಿಲಿ-ನಿಂದ
ಸಿತ-ಸಪ್ತಮ್ಯಾಂ
ಸಿತೇ
ಸಿದ್ದ-ರಾಗಿ
ಸಿದ್ದಿಮವಾಪ್ನುಯುಃ
ಸಿದ್ದಿ-ಯನ್ನು
ಸಿದ್ಧ-ನಾಗಿದ್ದ
ಸಿದ್ಧ-ನಾಗುತ್ತಾನೆ
ಸಿದ್ಧಪಾದುಕಃ
ಸಿದ್ಧ-ರಾ-ದರು
ಸಿದ್ಧ-ವಾದಾಗ
ಸಿದ್ಧ-ವಾ-ದೆವು
ಸಿದ್ಧಿ
ಸಿದ್ಧಿ-ಮವಾಪ್ನೋತಿ
ಸಿಪ್ಪೆ-ಯನ್ನು
ಸೀತಾ
ಸೀಹತ್ಯ
ಸು
ಸುಂದರ-ಳಾಗಿದ್ದಳು
ಸುಂದರಳಾದ
ಸುಂದರ-ವಾದ
ಸುಂದರೌ
ಸುಕಚ್ಛ
ಸುಕಚ್ಛ-ಯಿತಿ-ನಾಮ್ನಾಹಂ
ಸುಕಥಾ-ಭಿಧಾಂ
ಸುಕರ್ಮ
ಸುಕರ್ಮ-ನಾ-ಮಾಹ-ಮ-ಭಕ್ಷ್ಯಾ
ಸುಕರ್ಮ-ಶೀಲೋಪಿ
ಸುಕೃತಂ
ಸುಕೃ-ತಮಸ್ಯ
ಸುಖ
ಸುಖಂ
ಸುಖ-ಕರ-ವಾದ
ಸುಖ-ಕಾರ-ಣಮ್
ಸುಖ-ಗಳನ್ನು
ಸುಖ-ಗಳನ್ನೂ
ಸುಖದ
ಸುಖ-ದಿಂದ
ಸುಖ-ದುಃಖ-ಗಳನ್ನು
ಸುಖ-ದುಃಖ-ಗಳು
ಸುಖ-ದುಃಖಾ-ದಿ-ಗಳು
ಸುಖ-ಪಡುತ್ತಾನೆ
ಸುಖ-ಮವಾಪ್ನೋತಿ
ಸುಖ-ಮವಾಪ್ಸ್ಯಸಿ
ಸುಖ-ಮಾಪೇದೇ
ಸುಖ-ಮಾಸಂ
ಸುಖ-ಮಿಚ್ಛಂತಿ
ಸುಖ-ಮೇವ
ಸುಖಮ್
ಸುಖ-ವನ್ನು
ಸುಖ-ವಾಗಿ
ಸುಖ-ವಾಗಿತ್ತು
ಸುಖ-ವಾಗಿದ್ದನು
ಸುಖ-ವಾಗಿದ್ದಾರೆ
ಸುಖ-ವಾಗಿ-ರ-ಲಿಲ್ಲ
ಸುಖ-ವಾಗಿರು
ಸುಖ-ವಾಗಿ-ರುತ್ತಾನೆ
ಸುಖ-ವಾಗಿ-ರು-ವರು
ಸುಖ-ವಾಗಿ-ರು-ವೆಯಾ
ಸುಖ-ವಾಗಿವೆ
ಸುಖವೂ
ಸುಖಾವಹೈಃ
ಸುಖಿ-ಗಳಾದೆವು
ಸುಖಿನೋ
ಸುಖಿ-ಯಾಗುವಿ
ಸುಖೀ
ಸುಖೀ-ಜನಃ
ಸುಖೋಪಾಸ್ಯಂ
ಸುಖೋಪಾಸ್ಯೋ
ಸುಗಂಧ
ಸುಗಂಧ-ಯುಕ್ತ-ವಾದ
ಸುಗಂಧಿ-ನಮ್
ಸುಗತಿರ್ಮೇದ್ಯ
ಸುಚರಿತ್ರನ
ಸುಚರಿತ್ರ-ಸುತೋ
ಸುಚಿರಂ
ಸುಚ್ಛಾ-ಯನ
ಸುಚ್ಛಾ-ಯಸ್ಯ
ಸುಜ-ನ-ವರ್ಜಿತೇ
ಸುಜೀವಿಯು
ಸುಜ್ಞಾನೀ
ಸುಟ್ಟವೆ
ಸುಟ್ಟು
ಸುಡಲ್ಪಡು-ವಂತೆ
ಸುತಂ
ಸುತಃ
ಸುತಪ
ಸುತ-ಪನು
ಸುತ-ಪನೇ
ಸುತಪಾ-ನಾಮ
ಸುತಪಾ-ಮುನಿಃ
ಸುತ-ಮವಾಪ್ನೋತಿ
ಸುತಸ್ನೇಹಾಜ್ಜಾತಕಂ
ಸುತಾ
ಸುತಾಃ
ಸುತೋ
ಸುತೌ
ಸುತ್ತುತ್ತಾನೆ
ಸುತ್ತುತ್ತಿ-ರುವ
ಸುತ್ತು-ವರಿಯಲ್ಪಟ್ಟ
ಸುದಾರುಣಾಃ
ಸುದುರ್ಗ್ರ-ಹಮ್
ಸುದುರ್ಲಭಾ
ಸುದುಷ್ಕ-ರ-ವಮ್
ಸುದ್ದಿ
ಸುದ್ಯುಮ್ನಯಿತ್ಯಾ-ಸೀತ್
ಸುದ್ಯುಮ್ಮ-ನೆಂಬ
ಸುಧರ್ಮಃ
ಸುಧರ್ಮನ
ಸುಧರ್ಮನು
ಸುಧರ್ಮ-ನೆಂಬ
ಸುಧರ್ಮಸ್ತನಯಾಂ
ಸುಧರ್ಮಸ್ಯ
ಸುಧರ್ಮಾ
ಸುಧಾ
ಸುಧಾಂ
ಸುಧಾದಿ
ಸುಧಾಮ್
ಸುಧೀಃ
ಸುಧೀರಪಿ
ಸುಧೆ-ಯನ್ನು
ಸುಧೆಯು
ಸುನಮ್ರಾ
ಸುಪರ್ಣಾ
ಸುಪುತ್ರಂ
ಸುಪುತ್ರ-ನನ್ನು
ಸುಪ್ತಮಭಾಷಮಾಣಂ
ಸುಪ್ರದೃಷ್ಟಮನಾ
ಸುಬರ್ಹಿಷಃ
ಸುಬರ್ಹಿಷ-ನೆಂಬ
ಸುಬಲಿ
ಸುಬಲೇಸ್ತನಯೋ
ಸುಬಾಹು
ಸುಬಾಹುರ್ವೀರಬಾಹುಶ್ಚ
ಸುಬ್ಬಣ್ಣಾಚಾರ್
ಸುಬ್ಬಣ್ಣಾಚಾರ್ಯ
ಸುಭಗಸ್ಯ
ಸುಭದ್ರ
ಸುಭದ್ರಂ
ಸುಭದ್ರಕಃ
ಸುಭದ್ರನ
ಸುಭದ್ರ-ನನ್ನು
ಸುಭದ್ರ-ನನ್ನೂ
ಸುಭದ್ರ-ನಿಗೆ
ಸುಭದ್ರನೂ
ಸುಭದ್ರರು
ಸುಭದ್ರಾ-ವರಜಂ
ಸುಭದ್ರೇಣ
ಸುಭದ್ರೋ
ಸುಭಾಗಿನಿ
ಸುಭಾಷಿತ
ಸುಭೋಜ್ಯಮೃಷಿಭಿಃ
ಸುಮಂಗಲಿಗೆ
ಸುಮತಿಯು
ಸುಮತಿರ್ದೃಷ್ಟ್ವಾ
ಸುಮನಾಃ
ಸುಮೇಧ-ನನ್ನು
ಸುಮೇಧನು
ಸುಮೇಧ-ನೆಂಬ
ಸುಮೇಧಾ-ನಾಮ
ಸುಮ್ಮ-ನಾ-ಗಲು
ಸುಮ್ಮ-ನಾದನು
ಸುಮ್ಮನಾ-ದವು
ಸುಮ್ಮ-ನಾ-ಯಿತು
ಸುಮ್ಮನೆ
ಸುಯಜ್ಞ
ಸುಯಜ್ಞಂ
ಸುಯಜ್ಞಃ
ಸುಯಜ್ಞ-ನನ್ನು
ಸುಯಜ್ಞ-ನಾದರೋ
ಸುಯಜ್ಞ-ನಿಗೆ
ಸುಯಜ್ಞನು
ಸುಯಜ್ಞ-ನೆಂಬ
ಸುಯಜ್ಞೋ
ಸುಯಜ್ಞೋಪಿ
ಸುಯಜ್ಞೋ-ಽಪಿ
ಸುಯೋಗ
ಸುರದ್ವಿಷಃ
ಸುರಾಃ
ಸುರಾನ್
ಸುರಾಪಂ
ಸುರಾಪಸ್ಯ
ಸುರಾಪಾನ
ಸುರಾಪಾನ-ನಿರತ-ನಾಗಿದ್ದೆ
ಸುರಾಪೀತಿ
ಸುರಾಪೀತ್ಯಭಿ-ನಿಂದಿತಃ
ಸುರಾಪೀತ್ಯೇವ
ಸುರಾಪೋ
ಸುರಿಯುತ್ತಿದೆ
ಸುರಿ-ಸುತ್ತಾ
ಸುರಿ-ಸುತ್ತಿದ್ದ
ಸುರುಚಿಂ
ಸುರೂಪಾ
ಸುರೂಪಿಣೀ
ಸುಲಭಂ
ಸುಲಭ-ವಾದ
ಸುಲಿಗೆ-ಮಾಡಲು
ಸುಳಿಯ-ಲಿಲ್ಲ
ಸುಳ್ಳಾಗಿದ್ದರೆ
ಸುಳ್ಳಾಗು-ವುದಿಲ್ಲ
ಸುಳ್ಳು
ಸುವಾಸ-ನೆಯು
ಸುಶರ್ಮನ
ಸುಶರ್ಮನು
ಸುಶರ್ಮಾ
ಸುಶಿಕ್ಷಿತಃ
ಸುಶಿಕ್ಷಿತಾಃ
ಸುಷುಪ್ತಿ
ಸುಸಮಂತ
ಸುಸಮಂತನು
ಸುಸಮಂತಾತ್ಮಜಃ
ಸುಸಮಾಗಮಃ
ಸುಸ್ನಾ-ಯಾತ್
ಸುಸ್ಮಿತಂ
ಸುಸ್ಮಿತ್
ಸುಹೃತ್
ಸುಹೃದಃ
ಸುಹೃ-ದಶ್ಚ
ಸೂಕ್ಷ-ವನ್ನು
ಸೂಕ್ಷವು
ಸೂಕ್ಷ್ಮ
ಸೂಕ್ಷ್ಮಂ
ಸೂಕ್ಷ್ಮ-ಗಳನ್ನು
ಸೂಚಿ-ಸಲು
ಸೂಚಿ-ಸುತ್ತವೆ
ಸೂಚಿ-ಸುವ
ಸೂತ
ಸೂತರ
ಸೂತರು
ಸೂತ್ರ-ದಲ್ಲಿ
ಸೂತ್ರ-ದಲ್ಲಿಯೂ
ಸೂತ್ರವೂ
ಸೂನುರುತ್ಸರ್ಜ-ಯಿತ್ವಾ
ಸೂಯ್ಯೋದಯ-ವಾಗುತ್ತಿ-ರಲು
ಸೂರ
ಸೂರ್ಯ
ಸೂರ್ಯಂ
ಸೂರ್ಯ-ಗತಂ
ಸೂರ್ಯ-ಚಂದ್ರ
ಸೂರ್ಯನ
ಸೂರ್ಯ-ನಂತಿ-ರುವ-ವನೇ
ಸೂರ್ಯ-ನಂತೆ
ಸೂರ್ಯ-ನನ್ನು
ಸೂರ್ಯ-ನನ್ನೂ
ಸೂರ್ಯ-ನಿ-ರುವ
ಸೂರ್ಯ-ನಿ-ರು-ವಾಗ
ಸೂರ್ಯನು
ಸೂರ್ಯನೂ
ಸೂರ್ಯನೇ
ಸೂರ್ಯ-ಮಂಡಲಗಂ
ಸೂರ್ಯ-ಮಂಡಲಗೋ
ಸೂರ್ಯ-ಮಂಡಲಸ್ಥಿ-ತ-ನಾದ
ಸೂರ್ಯರು
ಸೂರ್ಯಶ್ಚ
ಸೂರ್ಯ-ಸಂಕ್ರಮ-ಣವಿ-ರು-ವಾಗ
ಸೂರ್ಯಾಂತರ್ಗತ
ಸೂರ್ಯಾ-ದಯೋ-ಽಖಿಲಾಃ
ಸೂರ್ಯಾದಿ
ಸೂರ್ಯಾದಿಗ್ರಹ-ಗಳಾಗಲಿ
ಸೂರ್ಯೋ
ಸೂರ್ಯೋ-ದ-ಯಕ್ಕೆ
ಸೂರ್ಯೋ-ದಯ-ವಾ-ಗಲು
ಸೃಷ್ಟ-ವಾನ್
ಸೃಷ್ಟಿಸಲ್ಪಟ್ಟಿಲ್ಲ
ಸೃಷ್ಟಿಸಲ್ಪಡುತ್ತ-ದೆಯೋ
ಸೃಷ್ಟಿಸಿ
ಸೃಷ್ಟಿಸಿ-ದರು
ಸೃಷ್ಟ್ವಾ
ಸೆಳೆದು-ಕೊಂಡು-ಹೋಗುತ್ತವೆ
ಸೇತಿ-ಹಾಸ-ಪುರಾ-ಣಕಾನ್
ಸೇತುವೆಗೆ
ಸೇತೌ
ಸೇನ
ಸೇನಾಬಲಾನ್ವಿತಃ
ಸೇರಿ
ಸೇರಿದ
ಸೇರಿ-ದ-ವನು
ಸೇರಿ-ದೆವು
ಸೇರಿದ್ದರೆ
ಸೇರಿದ್ದಾಗ
ಸೇರಿದ್ದೇನೆ
ಸೇರಿ-ಬಿಡುತ್ತಾನೆ
ಸೇರಿ-ಸಿ-ಕೊಂಡು
ಸೇರುವ
ಸೇವಕ
ಸೇವಕ-ರಾಗಿ
ಸೇವಕಿಯ
ಸೇವಕಿಯು
ಸೇವಯಾ
ಸೇವ-ಯಾಂತೇ
ಸೇವಯೈವ
ಸೇವಾ
ಸೇವಾಂ
ಸೇವಾರ್ಥಂ
ಸೇವಿ-ತ-ನಾದ
ಸೇವಿತಾ
ಸೇವಿ-ಸ-ಲಿಲ್ಲ
ಸೇವಿಸಿ
ಸೇವಿ-ಸಿದ
ಸೇವಿಸುತ್ತಾರೆಯೋ
ಸೇವಿ-ಸುವ
ಸೇವಿ-ಸು-ವುದ-ರಿಂದ
ಸೇವೆ
ಸೇವೆ-ಗಾಗಿ
ಸೇವೆ-ಮಾಡಿದ
ಸೇವೆ-ಯನ್ನು
ಸೇವೆ-ಯಲ್ಲಿ
ಸೇವೆ-ಯಿಂದ
ಸೇವೆಯು
ಸೇವೆಯೇ
ಸೇವ್ಯರೇ
ಸೇವ್ಯಾಃ
ಸೇಷು
ಸೈ
ಸೈಚಾ-ಚಾರಿ-ಗಳಾದ
ಸೈನ್ಯ-ದಿಂದ
ಸೈನ್ಯ-ಬಲ-ದಿಂದ
ಸೈನ್ಯ-ವನ್ನು
ಸೈನ್ಯ-ವನ್ನೆಲ್ಲ
ಸೊಂಟ-ದಲ್ಲಿ
ಸೊಪ್ಪು-ಗಳು
ಸೋ
ಸೋದರತ್ತೆಯ
ಸೋದ-ರಮಾವ
ಸೋದ-ರಮಾ-ವನ
ಸೋದ-ರಮಾ-ವ-ನಿಂದ
ಸೋದರಳಿಯ
ಸೋಪಾಂಗಾಂಕಲಾ-ವಿದ್ಯಾಶ್ಚತುರ್ದಶ
ಸೋಪಾನಭೂತಾಃ
ಸೋಪಿ
ಸೋಬ್ರವೀದ್ರಾಕ್ಷಸಸ್ತದಾ
ಸೋಭ-ವತ್
ಸೋಮ-ವಾರ
ಸೋಮಾನ್ವಯೇ
ಸೋಲಿಸಿ
ಸೋಽಂತರಧೀಯತ
ಸೋಽಗಮದ್ವಿಷ-ಯಾಂತ-ರ-ವಮ್
ಸೋಽಪಿ
ಸೋಽಪ್ಯಸ್ನಾಭಿಃ
ಸೋಽಬ್ರವೀತ್
ಸೋಽ-ಭೂತ್
ಸೋಽವಾತ್ಸೀದ್ವರ್ಷ-ಸಾಹಸ್ರಂ
ಸೋಽಶ್ವಮೇಧ-ಫಲಂ
ಸೌ
ಸೌಖ್ಯ
ಸೌತಮರು
ಸೌತಾಮಣಿ
ಸೌತ್ರಾಮಣ್ಯಾಶ್ಚ
ಸೌದೆ-ಗಳು
ಸೌದೆಯು
ಸೌಮೇನ
ಸೌಮ್ಯಂ
ಸೌವೀರ
ಸೌವೀರ-ದೇಶೇಶೋ
ಸ್ಕಂಧೇ-ನೋದ್ವಹ್ಯ
ಸ್ಕಾಂದ-ಪುರಾಣ
ಸ್ಟೇನ
ಸ್ತನಂ
ಸ್ತನ-ದಲ್ಲಿ
ಸ್ತನ-ಮೂರ್ಧಜಃ
ಸ್ತನ-ಮೂರ್ಧಾಸ್ತಥಾ-ಪರಾಃ
ಸ್ತನ-ವನ್ನು
ಸ್ತನ್ಯಪಾನ
ಸ್ತನ್ಯಪಾನ-ಮಾಡಲು
ಸ್ತಾ
ಸ್ತು
ಸ್ತೂಯಮಾನೊ
ಸ್ತೇನ
ಸ್ತೇಯಿನಂ
ಸ್ತೋತ್ರ
ಸ್ತೋತ್ರಕ್ಕೆ
ಸ್ತೋತ್ರ-ಗಳ
ಸ್ತೋತ್ರ-ಗಳನ್ನು
ಸ್ತೋತ್ರ-ಪಾಠ-ಮೃತೇ
ಸ್ತೋತ್ರ-ಮಾಡಿ-ದರು
ಸ್ತೋತ್ರಾಣಾಂ
ಸ್ತೋತ್ರೈಃ
ಸ್ತ್ಯ್ತಕ್ತನಿಜಸ್ವಭಾ-ವಾನ್
ಸ್ತ್ರದೂಷ-ಣಮ್
ಸ್ತ್ರವನ್ನೂ
ಸ್ತ್ರವಾಕ್ಯಾರ್ಥ-ದಿಂದಈ
ಸ್ತ್ರೀ
ಸ್ತ್ರೀಗೆ
ಸ್ತ್ರೀಘ್ನಂ
ಸ್ತ್ರೀಜಿತಃ
ಸ್ತ್ರೀಣಾಂ
ಸ್ತ್ರೀತ್ವಂ
ಸ್ತ್ರೀಭೋಗ
ಸ್ತ್ರೀಯ
ಸ್ತ್ರೀಯಃ
ಸ್ತ್ರೀಯತೇ
ಸ್ತ್ರೀಯರ
ಸ್ತ್ರೀಯ-ರನ್ನು
ಸ್ತ್ರೀಯ-ರಾಗಲೀ
ಸ್ತ್ರೀಯ-ರಿಂದ
ಸ್ತ್ರೀಯ-ರಿಗೂ
ಸ್ತ್ರೀಯರು
ಸ್ತ್ರೀಯ-ರೊಡನೆ
ಸ್ತ್ರೀಯಾಗಲೀ
ಸ್ತ್ರೀಯಾ-ದರೆ
ಸ್ತ್ರೀಯು
ಸ್ತ್ರೀವಧಾದು
ಸ್ತ್ರೀವಧೆ
ಸ್ತ್ರೀವಶೈರ್ಜಿತಃ
ಸ್ತ್ರೀಶೂದ್ರಾಣಾಂ
ಸ್ತ್ರೀಸುಖ-ವನ್ನು
ಸ್ಥಲಂ
ಸ್ಥಳಕ್ಕೆ
ಸ್ಥಳ-ಗಳಲ್ಲಿ
ಸ್ಥಳ-ಗಳು
ಸ್ಥಳ-ದಲ್ಲಿ
ಸ್ಥಳ-ವನ್ನು
ಸ್ಥಾನಂ
ಸ್ಥಾನಕ್ಕೆ
ಸ್ಥಾನಯೋರುತ್ಕಟಾಪಿ
ಸ್ಥಾನಾನಿ
ಸ್ಥಾನೇ
ಸ್ಥಾಪಿತಂ
ಸ್ಥಾಪಿತಾ
ಸ್ಥಾಪಿ-ಸುತ್ತಾನೆ
ಸ್ಥಾವರ-ಜಂಗ-ನಾತ್ಮಕ-ವಾದ
ಸ್ಥಾವರ-ಜಂಗ-ಮಮ್
ಸ್ಥಾವರತ್ವೇನ
ಸ್ಥಾವ-ರರು
ಸ್ಥಾವರಾಣ್ಯಷ್ಟೌ
ಸ್ಥಿ
ಸ್ಥಿತಂ
ಸ್ಥಿತಃ
ಸ್ಥಿತಾ
ಸ್ಥಿತಿಯು
ಸ್ಥಿತಿರ್ಜಾತಾ
ಸ್ಥಿತೇ
ಸ್ಥಿತೇ-ಯಮೂರ್ಮಿಲಾ-ನಾಮ
ಸ್ಥಿತೋ
ಸ್ಥಿತೌ
ಸ್ಥಿತ್ವಾ
ಸ್ಥೀಯತಾಂ
ಸ್ಥೀಯತೇ
ಸ್ನಾ
ಸ್ನಾತಕ
ಸ್ನಾತಕವ್ರತ-ಮ-ಕಾರೋತ್
ಸ್ನಾತಕವ್ರತ-ವನ್ನು
ಸ್ನಾತಾ
ಸ್ನಾತೋ
ಸ್ನಾತ್ವಾ
ಸ್ನಾತ್ವಾ-ಽನಂತ-ಸರೋ-ವರೇ
ಸ್ನಾನ
ಸ್ನಾನಂ
ಸ್ನಾನ-ಕರ್ಮ
ಸ್ನಾನ-ಕಾಲ-ದಲ್ಲಿ
ಸ್ನಾನ-ಕಾಲೇ
ಸ್ನಾನ-ಕಾಲೇಷು
ಸ್ನಾನ-ಕೃನ್ನರಃ
ಸ್ನಾನಕ್ಕಾಗಿ
ಸ್ನಾನಕ್ಕಿಂತ
ಸ್ನಾನಕ್ಕೋಸ್ಕರ
ಸ್ನಾನಜಂ
ಸ್ನಾನದ
ಸ್ನಾನ-ದಾನತಃ
ಸ್ನಾನ-ದಾನ-ಫಲಾದಪಿ
ಸ್ನಾನ-ದಾ-ನಾದಿ
ಸ್ನಾನ-ದಾ-ನಾದಿಕಾಃ
ಸ್ನಾನ-ದಿಂದ
ಸ್ನಾನ-ಪುಣ್ಯ-ಫಲ-ಮುಂಬುಜ-ನಾಭನಶ್ಯಮ್
ಸ್ನಾನ-ಫಲ-ವೊಂದನ್ನೇ
ಸ್ನಾನ-ಮಂಜಸಾ
ಸ್ನಾನ-ಮ-ನಂತರಮ್
ಸ್ನಾನ-ಮನನ್ಯ
ಸ್ನಾನ-ಮನೂದಿತಾರ್ಕೇ
ಸ್ನಾನ-ಮಾ-ಚ-ರೇತ್
ಸ್ನಾನ-ಮಾಡದೇ
ಸ್ನಾನ-ಮಾಡ-ಬೇಕು
ಸ್ನಾನ-ಮಾಡ-ಬೇಡ-ವೆಂದು
ಸ್ನಾನ-ಮಾಡಲು
ಸ್ನಾನ-ಮಾಡಿ
ಸ್ನಾನ-ಮಾಡಿದ
ಸ್ನಾನ-ಮಾಡಿ-ದನು
ಸ್ನಾನ-ಮಾಡಿ-ದರೂ
ಸ್ನಾನ-ಮಾಡಿ-ದರೆ
ಸ್ನಾನ-ಮಾಡಿ-ರುತ್ತಾನೆ
ಸ್ನಾನ-ಮಾಡಿಸಿ
ಸ್ನಾನ-ಮಾಡು
ಸ್ನಾನ-ಮಾಡುತ್ತಾ-ನೆಯೋ
ಸ್ನಾನ-ಮಾಡುತ್ತೇನೆ
ಸ್ನಾನ-ಮಾಡುವ
ಸ್ನಾನ-ಮಾಡು-ವ-ವ-ನಿಗೆ
ಸ್ನಾನ-ಮಾಡು-ವ-ವನು
ಸ್ನಾನ-ಮಾಡು-ವಾಗಲೂ
ಸ್ನಾನ-ಮಾಡು-ವುದ-ರಿಂದ
ಸ್ನಾನ-ಮಾಡು-ವುದಿಲ್ಲವೋ
ಸ್ನಾನ-ಮಾಡು-ವುದು
ಸ್ನಾನ-ಮಾತ್ರತಃ
ಸ್ನಾನ-ಮಾತ್ರೇಣ
ಸ್ನಾನ-ಮೇಕ-ದಿನೇ
ಸ್ನಾನ-ವನ್ನು
ಸ್ನಾನ-ವನ್ನೇ
ಸ್ನಾನವು
ಸ್ನಾನಸ್ಯ
ಸ್ನಾನಾದಿ
ಸ್ನಾನಾದಿಕಂ
ಸ್ನಾನಾದಿ-ಗಳನ್ನು
ಸ್ನಾನಾ-ನಂತರ-ದಲ್ಲಿ
ಸ್ನಾನಾನ್ಮಾಘ್ಯಾತ್
ಸ್ನಾನಾನ್ಯೇ-ತಾನಿ
ಸ್ನಾನಾ-ಶಕ್ತೋ
ಸ್ನಾನೇ
ಸ್ನಾನೇನ
ಸ್ನಾನೋ
ಸ್ನಾಪ-ಯನ್
ಸ್ನಾಪಯೇದ್ಯದಿ
ಸ್ನಾಪಯೇದ್ಯಸ್ತು
ಸ್ನಾಪ-ಯೇನ್ಮಾಧವಂ
ಸ್ನಾಪಿತೇ
ಸ್ನಾಪಿತೋ
ಸ್ನಾಪ್ಯ
ಸ್ನಾಯಾದ್ಯೋ
ಸ್ನಾಯಾನ್ಮಾಘೇ
ಸ್ನೇಚ್ಚಾ-ಚಾರಿ-ಣಿಯು
ಸ್ನೇಹದ
ಸ್ನೇಹ-ಭಾವ-ದಿಂದ
ಸ್ನೇಹ-ವನ್ನು
ಸ್ಪರ್ಶ
ಸ್ಪರ್ಶಕ್ಕೆ
ಸ್ಪರ್ಶ-ತತ್ವಾತ್ಮಾ
ಸ್ಪರ್ಶನ
ಸ್ಪರ್ಶ-ಮಾಡಲು
ಸ್ಪರ್ಶ-ಮಾತ್ರೇಣ
ಸ್ಪರ್ಶಾಶೌ
ಸ್ಪರ್ಶಿ-ಸುತ್ತಾನೆ
ಸ್ಪಶಂತಿ
ಸ್ಪಷ್ಟ
ಸ್ಪಷ್ಟ-ಪಡಿಸಲಾಗಿದೆ
ಸ್ಪಷ್ಟ-ಪಡಿಸುತ್ತದೆ
ಸ್ಪುಟಮ್
ಸ್ಪೃಶಂತ್ಯೇನಂ
ಸ್ಪೃಶತೇ
ಸ್ಪೃಶತ್ವೇಷ
ಸ್ಪೃಷ್ಟಃ
ಸ್ಫೂರ್ತಿ
ಸ್ಫೂರ್ತಿಗೆ
ಸ್ಫೂರ್ತಿ-ರೇವ
ಸ್ಮ
ಸ್ಮತಿ
ಸ್ಮತಿ-ಬಂಧ-ಮೋ-ಚನ-ಕರೀ
ಸ್ಮತಿ-ವಾಕ್ಯ-ಗಳಿಂದ
ಸ್ಮರಣೆ
ಸ್ಮರಣೆ-ಮಾಡ-ದ-ವರು
ಸ್ಮರಣೆ-ಯನ್ನು
ಸ್ಮರ-ಣೆಯೂ
ಸ್ಮರಣೆ-ಯೆಂಬ
ಸ್ಮರತಿ
ಸ್ಮರಿಸ-ಬೇಕು
ಸ್ಮರಿಸಿ
ಸ್ಮರಿಸಿ-ದರೆ
ಸ್ಮರಿಸುತ್ತಾರೆಯೋ
ಸ್ಮರಿಸು-ವುದಿಲ್ಲವೋ
ಸ್ಮರ್ಯತೇ
ಸ್ಮಶಾನ
ಸ್ಮಶಾನ-ಸದೃಶಂ
ಸ್ಮಾದಥಾ-ಯಯುಃ
ಸ್ಮಾಭಿ-ರೇವ
ಸ್ಮಾಯಾ-ದುಷ್ಟ
ಸ್ಮೃತಃ
ಸ್ಮೃತಮ್
ಸ್ಮೃತಯಃ
ಸ್ಮೃತ-ಸಜ್ಜನೌ
ಸ್ಮೃತಾ
ಸ್ಮೃತಾಃ
ಸ್ಮೃತಿ
ಸ್ಮೃತಿ-ಗಳನ್ನು
ಸ್ಮೃತಿ-ಗಳು
ಸ್ಮೃತಿ-ಪುರಾ-ಣಕೈಃ
ಸ್ಮೃತಿ-ಯಲ್ಲಿ
ಸ್ಮೃತಿ-ವಾಕ್
ಸ್ಮೃತಿ-ವಾಕ್ಯ
ಸ್ಮೃತಿಸ್ತು
ಸ್ಮೃತ್ವಾ-ಶಿವೋಕ್ತಂ
ಸ್ಯ
ಸ್ಯಾಂತಿಕಂ
ಸ್ಯಾಚ್ಚ
ಸ್ಯಾತ್
ಸ್ಯಾತ್ಸತಾಂ
ಸ್ಯಾದಿತಿ
ಸ್ಯಾದ್ವೋಽನೃತಭಾಷಿ-ಣಾಮ್
ಸ್ಯಾನ-ಪರೋ
ಸ್ಯಾನ್ಮಾತ್ರ
ಸ್ಯುಃ
ಸ್ಯುರ್ನ
ಸ್ರವದ್ರಕ್ತ
ಸ್ರೇಯೋ
ಸ್ವಕಮ್
ಸ್ವಕರ್ಮಣಾ
ಸ್ವಕರ್ಮಸು
ಸ್ವಕಾಮ್
ಸ್ವಗೃಹಂ
ಸ್ವತಂತ್ರ
ಸ್ವತಂತ್ರನು
ಸ್ವತಂತ್ರನೂ
ಸ್ವತಂತ್ರ-ನೆಂದರೆ
ಸ್ವತಂತ್ರ-ವಾಗಿ
ಸ್ವತಂತ್ರೋಪಿ
ಸ್ವತಃ
ಸ್ವದತ್ತಂ
ಸ್ವದೇಶಂ
ಸ್ವಧರ್ಮಸ್ಯಾಪಿ
ಸ್ವಪುರೀಂ
ಸ್ವಪ್ನ
ಸ್ವಪ್ನ-ದಲ್ಲಿ
ಸ್ವಪ್ನೇ
ಸ್ವಭಾರ್ಯಾ-ವ-ಚನಂ
ಸ್ವಭಾವ
ಸ್ವಭಾವಕ್ಕೆ
ಸ್ವಭಾವಜಾ
ಸ್ವಭಾ-ವತಃ
ಸ್ವಭಾ-ವದ
ಸ್ವಭಾವ-ದ-ವನೇಸ್ವಭಾವವು
ಸ್ವಭಾವ-ದಿಂದ
ಸ್ವಭಾವ-ದಿಂದಲೇ
ಸ್ವಭಾವವು
ಸ್ವಭಾವ-ವುಳ್ಳ
ಸ್ವಭಾವ-ವುಳ್ಳ-ವರು
ಸ್ವಭಾವವೂ
ಸ್ವಭಾವ-ಸಿದ್ಧ-ವಾದುದೇ
ಸ್ವಭಾವಾ
ಸ್ವಭಾವಾಂಶ್ಚ
ಸ್ವಭಾವಾದ್ವಾ-ಽಸ್ವಭಾ-ವಾದ್ವಾ
ಸ್ವಭಾವೋ
ಸ್ವಮಾತೃ-ಗಮನಂ
ಸ್ವಮಾಶ್ರಮಮ್
ಸ್ವಯಂ
ಸ್ವಯಂಭುವಃ
ಸ್ವಯಂವರ-ದಲ್ಲಿ
ಸ್ವಯಂವರೇ
ಸ್ವಯಮೇಕೋ
ಸ್ವಯಮ್
ಸ್ವಯಾರ್ಜಿತ
ಸ್ವರಾ-ದಯಃ
ಸ್ವರೂಪ
ಸ್ವರೂ-ಪ-ದಿಂದ
ಸ್ವರೂ-ಪ-ದೇಹದ
ಸ್ವರೂ-ಪ-ದೇಹ-ದಲ್ಲಿ
ಸ್ವರೂ-ಪ-ವನ್ನು
ಸ್ವರೂ-ಪ-ವನ್ನೂ
ಸ್ವರೂ-ಪವೇ
ಸ್ವರ್ಗ
ಸ್ವರ್ಗಕ್ಕೆ
ಸ್ವರ್ಗತಃ
ಸ್ವರ್ಗ-ದಲ್ಲಿ-ಯಾಗಲೀ
ಸ್ವರ್ಗ-ದಲ್ಲಿ-ರುವ
ಸ್ವರ್ಗ-ಪುರೀಂ
ಸ್ವರ್ಗಪ್ರಾಪ್ತಿಗೆ
ಸ್ವರ್ಗ-ಲೋಕಕ್ಕೆ
ಸ್ವರ್ಗ-ಲೋಕ-ದಲ್ಲಿ
ಸ್ವರ್ಗ-ಲೋಕವು
ಸ್ವರ್ಗ-ಲೋಕೇ
ಸ್ವರ್ಗಸ್ಥ-ನಾದನು
ಸ್ವರ್ಗಸ್ಥ-ರಾದ
ಸ್ವರ್ಗಸ್ಥೈಃ
ಸ್ವರ್ಗಾದಿ
ಸ್ವರ್ಗಾದಿ-ಗಳನ್ನೂ
ಸ್ವರ್ಗಾದಿ-ಗಳು
ಸ್ವರ್ಗಾದಿ-ಗಳೂ
ಸ್ವರ್ಗಾದ್ಯಾ
ಸ್ವರ್ಗೇ
ಸ್ವರ್ಗೊ
ಸ್ವರ್ಗೋ
ಸ್ವರ್ಗೋ-ಽ-ಭೂದ್ದು
ಸ್ವರ್ಚಿತಾಂ
ಸ್ವರ್ಲೋಕಂ
ಸ್ವಲ್ಪ
ಸ್ವಲ್ಪ-ವಾದರೂ
ಸ್ವಲ್ಪವೂ
ಸ್ವವೃತ್ತಾಂತಮಸಾದ-ಯತ್
ಸ್ವವೃತ್ತಾಂತ-ವಚೋ-ದ-ಯಾತ್
ಸ್ವಶಕ್ತಿತಃ
ಸ್ವಸು-ತಸ್ಯಾ
ಸ್ವಸ್ತಿ
ಸ್ವಸ್ತಿಕೈಃ
ಸ್ವಸ್ವಜೇ
ಸ್ವಸ್ವ-ಯೋಗ್ಯ
ಸ್ವಸ್ವ-ವರ್ಣೋಕ್ತ
ಸ್ವಾಂತರಸ್ಥಂ
ಸ್ವಾಂಶ್ಚ
ಸ್ವಾತಂತ್ರ-ದಿಂದ
ಸ್ವಾತಂತ್ರಾಭಿಮಾನ-ವನ್ನು
ಸ್ವಾತಂತ್ರ್ಯಂ
ಸ್ವಾತಂತ್ರ್ಯದ
ಸ್ವಾತಂತ್ರ್ಯ-ವನ್ನೂ
ಸ್ವಾತಂತ್ರ್ಯ-ವಿಲ್ಲ-ವಾದ-ಕಾರಣ
ಸ್ವಾತಂತ್ರ್ಯಾತ್
ಸ್ವಾತಂತ್ರ್ಯಾತ್ಕದಾ-ಚನ
ಸ್ವಾತಂತ್ರ್ಯಾಭಿಮಾನ-ರ-ಹಿತ-ನಾದ
ಸ್ವಾತ್ವಾ
ಸ್ವಾತ್ವಾಂಗಾರ-ಚತುರ್ದಶ್ಯಾಂ
ಸ್ವಾದು
ಸ್ವಾಧೀನ-ದಲ್ಲಿ
ಸ್ವಾಧೀನ-ದಲ್ಲಿಟ್ಟು
ಸ್ವಾಧೀನ-ಪಡಿಸಿ-ಕೊಳ್ಳುವುದು
ಸ್ವಾಧೀನ-ವಾಗುತ್ತದೆ
ಸ್ವಾಧ್ವರ-ದೀಕ್ಷಿತಃ
ಸ್ವಾನ-ವಿಲ್ಲದೆ
ಸ್ವಾನು-ರೂಪಕ್ರಿಯತ್ವಾಚ್ಚ
ಸ್ವಾನು-ರೂಪ-ಗುಣೇರಿತಾಃ
ಸ್ವಾಪ-ಯಿತ್ವಾ
ಸ್ವಾಮಿ
ಸ್ವಾಮಿಕ್ಷೇತ್ರದ
ಸ್ವಾಮಿಕ್ಷೇತ್ರ-ಮನುತ್ತಮಮ್
ಸ್ವಾಮಿ-ದರ್ಶನ-ಪುಣ್ಯೇನ
ಸ್ವಾಮಿನೇ
ಸ್ವಾಮಿ-ಭಕ್ತ-ರಾದ
ಸ್ವಾಮಿ-ಭಕ್ತಾ
ಸ್ವಾಮಿ-ಯಾದ
ಸ್ವಾಮಿ-ಸೇವ-ನಮ್
ಸ್ವಾಮೀ
ಸ್ವಾಮೀಕ್ಷೇತ್ರ-ದಲ್ಲಿ
ಸ್ವಾಮೀ-ದರ್ಶನಂ
ಸ್ವಾಮ್
ಸ್ವಾರಾಧಕಾ-ನಾಮಿಷ್ಟಾನಿ
ಸ್ವಾರ್ಥ-ವಿ-ವರ್ಜಿತೌ
ಸ್ವಾರ್ಥಾನಭಿ-ಪದ್ಯತೇ
ಸ್ವಾಶ್ರಮಂ
ಸ್ವಾಶ್ರಮ-ಪದಮಂಗಿರಾ-ನಾಮ
ಸ್ವಾಶ್ರಮ-ಮಾ-ಪುರೋ-ಜಸಾ
ಸ್ವಾಶ್ರಮೇ
ಸ್ವೀಕರಿಸದೇ
ಸ್ವೀಕರಿಸ-ಬಾರದು
ಸ್ವೀಕರಿಸ-ಬೇಕು
ಸ್ವೀಕರಿ-ಸ-ಲಿಲ್ಲ
ಸ್ವೀಕರಿಸಿ
ಸ್ವೀಕರಿಸಿ-ದನು
ಸ್ವೀಕರಿಸಿ-ದರೂ
ಸ್ವೀಕರಿಸಿ-ದರೆ
ಸ್ವೀಕರಿ-ಸಿರಿ
ಸ್ವೀಕರಿ-ಸುತ್ತಾನೆಯೋ
ಸ್ವೀಕರಿ-ಸುತ್ತಿದ್ದೆ
ಸ್ವೀಕರಿ-ಸು-ವುದು
ಸ್ವೀಟ್
ಸ್ವೀಯ
ಸ್ವೀಯಾನ್ಯ
ಸ್ವೀಯೈರುಪಾರ್ಜಿ-ತಾಮ್
ಸ್ವೈರ-ಚಾರಿಣೀ
ಸ್ವೈರಿಣೀಕಾಂಕ್ಷಿಣೌ
ಸ್ವೈರಿಣ್ಯಃ
ಸ್ವೋತ್ತಮರ
ಹ
ಹಂತತ್ವಾಭಿ-ಮಾನಿನಃ
ಹಂತಿ
ಹಂತುಮುಪಾದ್ರ-ವತ್
ಹಗಲು
ಹಗ್ಗ-ದಿಂದ
ಹಚ್ಚ-ಲಿಲ್ಲ
ಹಚ್ಚಿ
ಹಚ್ಚಿ-ಕೊಂಡು
ಹಚ್ಚಿ-ರುವ
ಹಜಾಮರು
ಹಣ
ಹಣ-ಕೊಡು-ವುದ-ರಲ್ಲಿ
ಹಣಕ್ಕಾಗಿ
ಹಣಕ್ಕೆ
ಹಣದ
ಹಣ-ದಲ್ಲಿ
ಹಣ-ದಲ್ಲಿನ
ಹಣ-ದಿಂದ
ಹಣದೊಡನೆ
ಹಣವಂತ-ನಾಗಿದ್ದೆ
ಹಣವಂತ-ನಾಗಿಯೂ
ಹಣವಂತ-ರಿಗೆ
ಹಣ-ವನ್ನು
ಹಣ-ವನ್ನೂ
ಹಣ-ವನ್ನೆಲ್ಲ
ಹಣ-ವಿದ್ದರೆ
ಹಣವು
ಹಣವೂ
ಹಣವೆಲ್ಲ
ಹಣೆಯ-ಮೇಲೆ
ಹಣ್ಣಿ-ನಂತೆ
ಹಣ್ಣು
ಹಣ್ಣು-ಗಳಲ್ಲಿ
ಹಣ್ಣು-ಗಳು
ಹಣ್ಣೆಲೆ-ಗಳನ್ನು
ಹತತೇಜಸ್ಕಾ
ಹತ-ನಾಗಿ
ಹತಾಂಹಸಃ
ಹತಾಸ್ತಥಾ
ಹತೋಟಿ-ಯಲ್ಲಿ-ರು-ವುದ-ರಿಂದ
ಹತ್ತನೇ
ಹತ್ತಿ
ಹತ್ತಿರ
ಹತ್ತಿ-ರ-ದಲ್ಲಿಯೇ
ಹತ್ತು
ಹತ್ಯ
ಹತ್ಯಾಂ
ಹತ್ಯಾ-ಽಯುತಂ
ಹತ್ಯೆ
ಹತ್ವಾ
ಹತ್ವಾ-ದಾಯ
ಹದಿನಾರನೇ
ಹದಿನಾರ-ರಲ್ಲಿ
ಹದಿನಾರು
ಹದಿ-ನಾಲ್ಕನೇ
ಹದಿ-ನಾಲ್ಕು
ಹದಿನೆಂಟನೆಯ
ಹದಿನೆಂಟು
ಹದಿನೈದನೇ
ಹದಿ-ಮೂರನೇ
ಹದ್ದಿನ
ಹದ್ದು-ಗಳಂತೆ
ಹನಿ-ಗಳು
ಹನುಮದ್ರೂಪ
ಹನುಮನ್ಮಾಮ
ಹನ್ನೆರಡನೆಯ
ಹನ್ನೆರಡನೇ
ಹನ್ನೆರಡು
ಹನ್ನೊಂದನೆಯ
ಹನ್ನೊಂದನೇ
ಹನ್ನೊಂದು
ಹಮುರ್ಮಿಲೋ
ಹಮ್
ಹಯ-ದೇಶದ
ಹಯ-ದೇಶಾಧಿಪ-ನಾದ
ಹಯಾನ್
ಹರಂತಿ
ಹರಡಿತ್ತು
ಹರಯೇ
ಹರಿ
ಹರಿಂ
ಹರಿಃ
ಹರಿ-ಕಥಾ
ಹರಿ-ಕಥಾಶ್ರವಣ-ದಲ್ಲಿ
ಹರಿ-ಗೋಸ್ಕರ
ಹರಿಣಾ
ಹರಿ-ತ-ವಾದ
ಹರಿದ್ವಿಷೋ
ಹರಿ-ಪದ-ವನ್ನು
ಹರಿಪ್ರಿಯಃ
ಹರಿಪ್ರೀತಿ-ಯಾಗುತ್ತದೆ
ಹರಿ-ಭಕ್ತನು
ಹರಿ-ಭಕ್ತಾಃ
ಹರಿ-ಮೂರ್ತಯಃ
ಹರಿಮ್
ಹರಿಯ
ಹರಿ-ಯನ್ನು
ಹರಿ-ಯಲು
ಹರಿಯು
ಹರಿ-ಯುತ್ತಿತ್ತು
ಹರಿ-ಯುತ್ತಿದೆ
ಹರಿ-ಯುವ
ಹರಿ-ಯು-ವುದು
ಹರಿ-ರೀಶ್ವರಃ
ಹರಿರ್ಯತ್ನ-ವತಾಂ
ಹರಿ-ವಂಶೇಷು
ಹರಿ-ವಾಯು-ಗಳ
ಹರಿ-ಸಂಕೀರ್ತಿ-ಕಥಾ-ಮೃತ-ವಿ-ವರ್ಜಿ-ತಮ್
ಹರೇ
ಹರೇಃ
ಹರೇ-ಕೃಷ್ಣ
ಹರೇ-ರಾಜ್ಞೇ
ಹರೇ-ರಾಧನೋದ್ಯ
ಹರೇ-ರಿ-ಮಾಮ್
ಹರೇ-ರೇವ
ಹರೇರ್ಗೃ-ಹಮ್
ಹರೇರ್ಧ್ಯಾನೇ
ಹರೇರ್ನಿಂದಾ
ಹರೇರ್ನಿಷೇವಿತಂ
ಹರ್ನಿ-ಶಮ್
ಹರ್ಷೋತ್ಫುಲ್ಲ-ವಿ-ಲೋ-ಚನಃ
ಹಲಸಿನ
ಹಲ್ಲನ್ನು
ಹಲ್ಲು-ಗಳನ್ನು
ಹಲ್ಲು-ಗಳನ್ನುಳ್ಳ
ಹಲ್ಲು-ಗಳಿಲ್ಲದ
ಹಳ್ಳಿ-ಗಳಲ್ಲಿ
ಹವನ
ಹವನ-ಹೋ-ಮ-ಗಳನ್ನಾಗಲೀ
ಹವ್ಯ
ಹವ್ಯಂ
ಹಸನ್ಮುಖ-ದಿಂದ
ಹಸಿವು
ಹಸು
ಹಸು-ಗಳನ್ನು
ಹಸು-ವನ್ನು
ಹಸು-ವಿನ
ಹಸ್ತ-ಗಳಿಗೆ
ಹಸ್ತ-ಗಳು
ಹಸ್ತ-ದಲ್ಲಿ
ಹಸ್ತಲಾಘವ-ವನ್ನು
ಹಸ್ತಲಿಖಿತ
ಹಸ್ತಾತ್
ಹಸ್ತಿಗಿರೌ
ಹಸ್ತಿನಾಪುರ-ದಲ್ಲಿ
ಹಾಕಲ್ಪಟ್ಟ
ಹಾಕಿ-ಕೊಳ್ಳ-ಬೇಕು
ಹಾಕಿದೆ
ಹಾಕು-ವುದು
ಹಾಗಿದ್ದರೂ
ಹಾಗೂ
ಹಾಗೆ
ಹಾಗೆಯೇ
ಹಾಗೇ
ಹಾಡುತ್ತಿದ್ದುವು
ಹಾದಿಯಲ್ಲೇ
ಹಾನಿಂ
ಹಾನಿಃ
ಹಾನಿ-ದಾನಿ
ಹಾನಿ-ಯಾಗು-ವುದಿಲ್ಲ
ಹಾನಿಯುಂಟಾಗು-ವುದಿಲ್ಲ
ಹಾನಿಯೇ
ಹಾರ-ದಿಂದ
ಹಾರ-ವನ್ನು
ಹಾರಿ-ದವು
ಹಾರುತ್ತಿದ್ದಾಗ
ಹಾಲನ್ನು
ಹಾಲಿನ
ಹಾಲಿ-ನಿಂದ
ಹಾಲು-ಗಳನ್ನು
ಹಾಲು-ತುಪ್ಪ
ಹಾಲು-ಮೊ-ಸರು
ಹಾಲು-ಮೊ-ಸರು-ಎಳ್ಳು-ಗಳಿಂದ
ಹಾಲು-ಮೊ-ಸರು-ಗಳನ್ನು
ಹಾಳಾಗುತ್ತದೆ
ಹಿ
ಹಿಂಡಿ-ಯನ್ನೂ
ಹಿಂಡೀ
ಹಿಂತೆಗೆ-ದನು
ಹಿಂದಿನ
ಹಿಂದಿ-ನಿಂದಲೂ
ಹಿಂದಿರುಗಿ-ದನು
ಹಿಂದಿರುಗು-ವುದಿಲ್ಲ
ಹಿಂದೆ
ಹಿಂದೆಯೂ
ಹಿಂದೆಯೇ
ಹಿಂಭಾಗಕ್ಕೆ
ಹಿಂಸಾ
ಹಿಂಸೆ-ಗಳಿಗೆ
ಹಿಂಸೆಗೆ
ಹಿಂಸೆ-ಯನ್ನು
ಹಿಂಸೆ-ಯಿಂದ
ಹಿಟ್ಟು
ಹಿಡಿದಿದ್ದ
ಹಿಡಿದು
ಹಿಡಿದು-ಕೊಂಡು
ಹಿಡಿಯಲ್ಪಟ್ಟ
ಹಿಡಿ-ಯಿತು
ಹಿಡಿ-ಯುವ-ವರು
ಹಿತ
ಹಿತಂ
ಹಿತ-ಕರ-ವಾದ
ಹಿತ-ಕಾಮ್ಮ
ಹಿತಕ್ಕೋಸ್ಕರ-ವಾಗಿ
ಹಿತ-ಮಿಚ್ಛತಾ
ಹಿತಮ್
ಹಿತ-ವನ್ನು
ಹಿತ-ವನ್ನುಂಟು-ಮಾಡ-ಬೇಕೆಂಬ
ಹಿತ-ವಾಗ-ಬೇಕೆಂಬ
ಹಿತ-ವೇ-ನೆಂದು
ಹಿತಾ-ಹಿತ-ಗಳನ್ನು
ಹಿತಾ-ಹಿ-ತಮ್
ಹಿತೇಚ್ಛಯಾ
ಹಿತ್ವಾ
ಹಿನಸ್ತಿ
ಹಿಮಮ್
ಹಿಮವಂತಂ
ಹಿಮ-ವತ್
ಹಿಮ-ವತ್ಪರ್ವತದ
ಹಿಮ-ವತ್ಪೃಷ್ಠೇ
ಹಿಮ-ವನ್ನು
ಹಿಮಾನಾಂ
ಹಿಮು
ಹಿಯಾ
ಹಿರಣ್ಯಂ
ಹಿರಣ್ಯಶ್ರಾದ್ಧ
ಹಿರಿ-ಯರೂ
ಹೀಗಿದ್ದ
ಹೀಗಿದ್ದರೂ
ಹೀಗಿದ್ದುವು
ಹೀಗಿದ್ದೂ
ಹೀಗಿ-ರಲು
ಹೀಗಿ-ರು-ವಾಗ
ಹೀಗೆ
ಹೀಗೆಂದನು
ಹೀಗೆಂದು
ಹೀಗೆಂದೆನು
ಹೀಗೆಂಬ
ಹೀಗೆಂಬುದು
ಹೀಗೆನ್ನಲು
ಹೀಗೆಯೇ
ಹೀನ
ಹೀನಃ
ಹೀನ-ಜಾತೀನಾ-ಮೇಷ
ಹೀನ-ಜಾನಾಂ
ಹೀನ-ನಾಗಿ
ಹೀನ-ನಾಗಿದ್ದೆ
ಹೀನ-ನಾಮಕಃ
ಹೀನನು
ಹೀನನೂ
ಹೀನಾಃ
ಹೀನೇ
ಹೀನೋ
ಹೀರೇ-ಕಾಯಿ
ಹುಚ್ಚ-ನಾಗಿಯೋ
ಹುಟ್ಟಲಿ
ಹುಟ್ಟಲು
ಹುಟ್ಟಿ
ಹುಟ್ಟಿತು
ಹುಟ್ಟಿದ
ಹುಟ್ಟಿ-ದನು
ಹುಟ್ಟಿ-ದರು
ಹುಟ್ಟಿ-ದಾಗ
ಹುಟ್ಟಿದೆ
ಹುಟ್ಟಿದ್ದೆ
ಹುಟ್ಟಿದ್ದೇನೆ
ಹುಟ್ಟಿ-ರುತ್ತಾಳೆ
ಹುಟ್ಟಿ-ರುತ್ತೇನೆ
ಹುಟ್ಟಿ-ರುವಳೋ
ಹುಟ್ಟಿಸಿ
ಹುಟ್ಟು
ಹುಟ್ಟುತ್ತದೆ
ಹುಟ್ಟುತ್ತ-ದೆಯೋ
ಹುಟ್ಟುತ್ತಾನೆ
ಹುಟ್ಟು-ವಂತೆ
ಹುಟ್ಟು-ವನು
ಹುಟ್ಟು-ವುದಿಲ್ಲ
ಹುಟ್ಟು-ವುದಿಲ್ಲವೋ
ಹುಟ್ಟು-ವುದು
ಹುಟ್ಟು-ವುದೂ
ಹುಟ್ಟು-ವುದೇ
ಹುಡುಗರ
ಹುಡುಗ-ರಿಂದ
ಹುಣ್ಣಿಮೆ
ಹುಣ್ಣಿಮೆ-ಯಂದು
ಹುಣ್ಣಿಮೆ-ಯಲ್ಲಿ
ಹುಣ್ಣಿಮೆಯು
ಹುತಂ
ಹುತ್ತಕ್ಕೆ
ಹುತ್ವಾ
ಹುರಿದ
ಹುಲಿ-ಗಳ
ಹುಲಿ-ಯಂತೂ
ಹುಲಿ-ಯನ್ನು
ಹುಲಿ-ಯಿಂದ
ಹುಲಿಯು
ಹುಲಿ-ಹಸು
ಹುಲ್ಲನ್ನು
ಹುಲ್ಲಿನ
ಹುಲ್ಲಿನ-ರಸ-ವನ್ನು
ಹುಲ್ಲಿನಿಂದ
ಹುಲ್ಲು
ಹೂ
ಹೂಗಳಿಂದ
ಹೂಗಳುಳ್ಳ
ಹೂವಿನ
ಹೃತಂ
ಹೃತಮ್
ಹೃತಾ
ಹೃತಾಶ್ಣೋರೈಃ
ಹೃತೌ
ಹೃದಯ-ಗುಹೆಯಲ್ಲಿದ್ದು
ಹೃದಯ-ಗುಹೆಯಲ್ಲಿ-ರುತ್ತೇನೆ
ಹೃದಯಪ್ರ-ದೇಶ-ದಲ್ಲಿ
ಹೃದಿ
ಹೃದಿ-ಬಾಹೂದ-ರಾದಿಷು
ಹೃದ್ದೇಶೇ-ಽರ್ಜುನ
ಹೃಷೀ-ಕೇಶ
ಹೆ
ಹೆಗಲಿನ
ಹೆಚ್ಚಾದ
ಹೆಚ್ಚಿನ
ಹೆಚ್ಚಿಸಿ-ಕೊಡ-ಬೇಕೆಂದು
ಹೆಚ್ಚು
ಹೆಜ್ಜೆ
ಹೆಜ್ಜೆಗೂ
ಹೆಜ್ಜೆ-ಯಲ್ಲಿಯೂ
ಹೆದರಬೇಡ
ಹೆದರಿ
ಹೆದರಿದ
ಹೆಸ-ರಾಗಿದೆ
ಹೆಸರಾ-ಯಿತು
ಹೆಸರಿನ
ಹೆಸರಿ-ನಿಂದ
ಹೆಸರಿ-ನಿಂದಲೂ
ಹೆಸರು
ಹೆಸರು-ಗಳನ್ನು
ಹೆಸರು-ಗಳನ್ನೂ
ಹೆಸರು-ಗಳು
ಹೆಸರುಳ್ಳ-ವನು
ಹೆಸರೂ
ಹೇ
ಹೇಗಾ-ಯಿತು
ಹೇಗಿವೆ
ಹೇಗೆ
ಹೇತುತ್ವಾತ್
ಹೇತು-ವಾದ-ರತಂ
ಹೇತೌ
ಹೇಮ
ಹೇಮ-ನಾಮ್ನೀಂ
ಹೇಮ-ಪುರಿ
ಹೇಮ-ಪು-ರಿಗೆ
ಹೇಮ-ಪುರೀ
ಹೇಮ-ಪುರೀಂ
ಹೇಮ-ಪುರೀಕ್ಷೇತ್ರೇ
ಹೇಮ-ಪುರೀ-ಮಿ-ಮಾಮ್
ಹೇಮ-ಪುರೀ-ಮೇಭಿಃ
ಹೇಳತಕ್ಕದ್ದೇನಿದೆ
ಹೇಳದೆ
ಹೇಳದೇ
ಹೇಳ-ಬೇಕು
ಹೇಳಬೇಕೆ
ಹೇಳಲಾಗಿದೆ
ಹೇಳಲಾಗುತ್ತದೆ
ಹೇಳಲಿ
ಹೇಳಲು
ಹೇಳ-ಲೇನಿದೆ
ಹೇಳಲ್ಪಟ್ಟ
ಹೇಳಲ್ಪಟ್ಟಿತು
ಹೇಳಲ್ಪಟ್ಟಿದೆ
ಹೇಳಲ್ಪಟ್ಟಿವೆ
ಹೇಳಲ್ಪಟ್ಟಿ-ವೆಯೋ
ಹೇಳಲ್ಪಡದೇ
ಹೇಳಲ್ಪಡ-ಲಿಲ್ಲ
ಹೇಳಲ್ಪಡುತ್ತಾರೆ
ಹೇಳಿ
ಹೇಳಿ-ಕೊಂಡವು
ಹೇಳಿ-ಕೊಂಡು
ಹೇಳಿತು
ಹೇಳಿದ
ಹೇಳಿ-ದಂತೆ
ಹೇಳಿ-ದ-ನಂತರ
ಹೇಳಿ-ದನು
ಹೇಳಿ-ದ-ಮೇಲೆ
ಹೇಳಿ-ದರು
ಹೇಳಿ-ದಳು
ಹೇಳಿ-ದು-ದನ್ನು
ಹೇಳಿದೆ
ಹೇಳಿ-ದೆನು
ಹೇಳಿದ್ದನ್ನು
ಹೇಳಿದ್ದು
ಹೇಳಿದ್ದೇನೆ
ಹೇಳಿರಿ
ಹೇಳಿ-ರುತ್ತಾನೆ
ಹೇಳಿ-ರುತ್ತಾರೆ
ಹೇಳಿ-ರುತ್ತೇನೆ
ಹೇಳಿ-ರುವ
ಹೇಳಿ-ರೆಂದು
ಹೇಳಿ-ಸಿ-ಕೊಳ್ಳಲ್ಪಟ್ಟ
ಹೇಳು
ಹೇಳುತ್ತದೆ
ಹೇಳುತ್ತವೆ
ಹೇಳುತ್ತಾ
ಹೇಳುತ್ತಾರೆ
ಹೇಳುತ್ತಿದ್ದೆ
ಹೇಳುತ್ತೇನೆ
ಹೇಳುತ್ತೇವೆ
ಹೇಳುವ
ಹೇಳು-ವಂತಹುದಲ್ಲ
ಹೇಳು-ವರು
ಹೇಳು-ವುದ-ರಿಂದ
ಹೇಳು-ವುದು
ಹೇಳು-ವುದೂ
ಹೇಳು-ವುದೇ-ನಿದೆ
ಹೇಳು-ವುದೇನು
ಹೇಳು-ವೆನು
ಹೈ
ಹೈಟತೋ
ಹೈಭ-ವತ್
ಹೈವಶಃ
ಹೈಹಯ-ದೇಶ-ರಾಜಃ
ಹೈಹಯಾಧಿ
ಹೈಹಯಾಧಿ-ಪತೇಃ
ಹೈಹಯಾಧಿ-ಪತೇರ್ಮಮ
ಹೊಂದ-ಬಹುದು
ಹೊಂದ-ಬೇಕು
ಹೊಂದ-ಬೇಕೆಂಬ
ಹೊಂದಬೇಡ
ಹೊಂದಲಾರೆ
ಹೊಂದಲು
ಹೊಂದಿ
ಹೊಂದಿದ
ಹೊಂದಿ-ದನು
ಹೊಂದಿ-ದರು
ಹೊಂದಿ-ದರೂ
ಹೊಂದಿ-ದಳು
ಹೊಂದಿ-ದಳೋ
ಹೊಂದಿದೆ
ಹೊಂದಿದ್ದು
ಹೊಂದಿರಿ
ಹೊಂದಿ-ರುತ್ತಾನೆ
ಹೊಂದಿ-ರುತ್ತಾರೆಯೋ
ಹೊಂದಿ-ರುವ
ಹೊಂದಿ-ಸುತ್ತಾನೆ
ಹೊಂದುತ್ತದೆ
ಹೊಂದುತ್ತವೆ
ಹೊಂದುತ್ತವೆಯೋ
ಹೊಂದುತ್ತಾನೆ
ಹೊಂದುತ್ತಾರೆ
ಹೊಂದುತ್ತೇನೆ
ಹೊಂದು-ವನು
ಹೊಂದುವವು
ಹೊಂದುವಿರಿ
ಹೊಂದು-ವುದಿಲ್ಲ
ಹೊಗೆ-ಯಿಂದ
ಹೊಟ್ಟು
ಹೊಟ್ಟೆ
ಹೊಡೆತ-ದಿಂದ
ಹೊಡೆ-ದರು
ಹೊಡೆದು
ಹೊಡೆದೆ
ಹೊಡೆ-ದೆವು
ಹೊತ್ತಾದರೂ
ಹೊತ್ತಿಗೆ
ಹೊತ್ತಿ-ನಲ್ಲಿಯೇ
ಹೊತ್ತು
ಹೊದಿಕೆ-ಯಿಂದ
ಹೊರ
ಹೊರಕ್ಕೆ
ಹೊರ-ಗಿನ
ಹೊರ-ಗಿರ-ತಕ್ಕ
ಹೊರಗೆ
ಹೊರ-ಟನು
ಹೊರ-ಟರೆ
ಹೊರ-ಟಿ-ರುವ
ಹೊರಟು
ಹೊರ-ಟು-ಹೋಗಿರಿ
ಹೊರ-ಟು-ಹೋಗುತ್ತವೆ
ಹೊರ-ಟು-ಹೋ-ದನು
ಹೊರ-ಟು-ಹೋ-ದುವು
ಹೊರ-ಡಲು
ಹೊರ-ತಾಗಿ
ಹೊರತು
ಹೋಗ-ದಂತೆ
ಹೋಗಬಯಸಿ
ಹೋಗ-ಬೇಕು
ಹೋಗ-ಬೇಕೆಂಬ
ಹೋಗಲಿ
ಹೋಗಲು
ಹೋಗಲ್ಪ-ಡಲು
ಹೋಗಿ
ಹೋಗಿದ್ದು
ಹೋಗಿರಿ
ಹೋಗಿ-ರುತ್ತಾರೆ
ಹೋಗಿಲ್ಲ-ವೆಂಬುದು
ಹೋಗು
ಹೋಗುತ್ತದೆ
ಹೋಗುತ್ತ-ದೆಯೋ
ಹೋಗುತ್ತವೆ
ಹೋಗುತ್ತಾ
ಹೋಗುತ್ತಾನೆ
ಹೋಗುತ್ತಾ-ನೆಯೋ
ಹೋಗುತ್ತಾರೆ
ಹೋಗುತ್ತಿದ್ದ
ಹೋಗುತ್ತಿದ್ದರು
ಹೋಗುತ್ತಿ-ರ-ಲಿಲ್ಲ
ಹೋಗುತ್ತೇನೆ
ಹೋಗುವ
ಹೋಗು-ವಂತೆ
ಹೋಗು-ವನು
ಹೋಗು-ವರು
ಹೋಗು-ವ-ವನು
ಹೋಗು-ವ-ವರೆಗೂ
ಹೋಗು-ವಾಗ
ಹೋಗು-ವುದಿಲ್ಲ
ಹೋಗು-ವುದು
ಹೋಗು-ವುದೇ
ಹೋದ
ಹೋದ-ನಂತರ
ಹೋದನು
ಹೋದ-ನೆಂದು
ಹೋದರು
ಹೋದಾಗ
ಹೋದೆ
ಹೋದೆವು
ಹೋಮ
ಹೋಮ-ಗಳನ್ನು
ಹೋಮ-ಮಾಡಿ
ಹೋಮ-ಮಾಡುತ್ತಿಯೋ
ಹೋಮ-ಮಾಡುತ್ತೀಯೋ
ಹೋಮ-ವನ್ನು
ಹೋಮ-ಸಲ್ಪಡುತ್ತಿ-ರುವ
ಹೋಮಾದಿ
ಹೋಯಿತು
ಹೋಷಿ
ಹ್ಯ
ಹ್ಯಗಮತ್
ಹ್ಯಗಮತ್ತದಾ
ಹ್ಯಜ್ಞಾನಾಂ
ಹ್ಯನುಗೃಹೀತೋಽಸ್ಮಿ
ಹ್ಯನೇಕ-ಜನ್ಮಾರ್ಜಿತ-ಪಾಪ-ನಾಶಾತ್
ಹ್ಯಪಾಯಶ್ಚ
ಹ್ಯಪಿ
ಹ್ಯಭದ್ರೂ
ಹ್ಯಭವಂ
ಹ್ಯಮ್ಮ-ಯಾನಿ
ಹ್ಯರೀನ್
ಹ್ಯರ್ಥಾಃ
ಹ್ಯಲ್ಪಾಯುಷ್ಯಮವಾಪ
ಹ್ಯವಿ-ಚಾರಿ-ತ-ಕಾರಿ-ಣಾಮ್
ಹ್ಯಶಕ್ತಸ್ತು
ಹ್ಯಷ್ಟದಲಂ
ಹ್ಯಷ್ಟೌ
ಹ್ಯಸ್ಮಿನ್
ಹ್ಯಹಮ್
ಹ್ಯಾ
ಹ್ಯಾಗತಾ
ಹ್ಯಾತ್ಯಂತಿ-ಕನಿಷ್ಕೃತಿ
ಹ್ಯುತ್ತಮೋ
ಹ್ಯೇತೇ
ಹ್ಯೇನಂ
ಹ್ಯೇಷ
ಹ್ಯೇಷಾ
ಹ್ರಾಸವೃದ್ಧೀ-ಷಡಾತ್ಮಾಸ್ಯ
ಹ್ರಾಸವೇ
ಹರೌ
ೠತೂನಾಂ
}
