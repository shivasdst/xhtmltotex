

\begin{center}
\phantom{श्रीः}
\end{center}

\begin{center}
\phantom{।। श्री गुरुराजो विजयते~।।}
\end{center}

\phantom{\textbf{ಶ‍್ರೀವಾಯುಪುರಾಣಾಂತರ್ಗತ}}

\chapter{ ಶ‍್ರೀ ಮಾಘಮಾಸ ಮಾಹಾತ್ಮ್ಯ}

\begin{verse}
\textbf{ನರಸಿಂಹೋಽಖಿಲಾಜ್ಞಾನಮತಧ್ವಾಂತದಿವಾಕರಃ~। }\\\textbf{ಜಯತ್ಯಮಿತಸದ್ ಜ್ಞಾನಸುಖಶಕ್ತಿ ಪಯೋನಿಧಿಃ~।।}\\\textbf{ಆಚಾರ್ಯಃ ಪವನೋsಸ್ಮಾಕಂ ಆಚಾರ್ಯಾಣೀ ಚ ಭಾರತೀ~। }\\\textbf{ದೇವೋ ನಾರಾಯಣಸ್ಸಾ ಕ್ಷಾತ್ ದೇವೀ ಮಂಗಳದೇವತಾ~।।}
\end{verse}

\begin{flushleft}
\textbf{ನಾರದ ಉವಾಚ\enginline{-}}
\end{flushleft}

\begin{verse}
\textbf{ಸಂಸಾರೇ ಕ್ಲಿಷ್ಯಮಾನಾನಾಂ ಜಂತೂನಾಂ ಪಾಪಕಾರಿಣಾಮ್~। }\\\textbf{ಕರ್ಮಣಾ ಭ್ರಾಮ್ಯಮಾಣಾನಾಂ ಕಾ ಗತಿಃ ಕಮಲೋದ್ಭವ~।। ೧~।।}
\end{verse}

\noindent
ನಾರದರು ಬ್ರಹ್ಮ ದೇವರನ್ನು ಕುರಿತು ಪ್ರಶ್ನೆ ಮಾಡುತ್ತಾರೆ:\enginline{-}

ಈ ಸಂಸಾರದಲ್ಲಿ ಯಾವಾಗಲೂ ಕಷ್ಟ ಪಡುತ್ತಿರುವ, ಸದಾ ಪಾಪಕರ್ಮಗಳನ್ನು ಆಚರಿಸುತ್ತಾ ತತ್ಪರಿಣಾಮವಾಗಿ ಜನನಮನನರೂಪವಾದ ಸಂಸಾರದಲ್ಲಿ ಸುತ್ತುತ್ತಿರುವ ಚೇತನರ ಗತಿ ಏನು?

\begin{verse}
\textbf{ಸುಖಮಿಚ್ಛಂತಿ ತೇ ಮೂಢಾಃ ನೇಚ್ಛಂತಿ ಸುಖಕಾರಣಮ್~।}\\\textbf{ನೇಚ್ಛಂತಿ ದುಃಖಲೇಶಂ ವಾ ತತ್ ಹೇತೌ ಸತತಂ ಯತಃ~।। ೨~।।}
\end{verse}

ಇಂತಹ ಮೂಢರು ಸುಖವನ್ನು ಇಚ್ಚಿಸುತ್ತಾರೆ. ಆದರೆ ಅಂತಹ ಸುಖ ಬರುವ ಕಾರಣ\-ವನ್ನು ತಿಳಿಯಲು ಇಚ್ಛಿಸುವುದಿಲ್ಲ. ದುಃಖಲೇಶವೂ ಆಗಬಾರದೆಂಬ ಇಚ್ಛೆಯುಳ್ಳ ಅವರು ದುಃಖ ಉಂಟಾಗದೆ ಇರುವ ಕಾರಣವನ್ನು ತಿಳಿಯರು.

\begin{verse}
\textbf{ಅಲಸಾ ಧರ್ಮವಿಮುಖಾ ವಿಷಯಾಸಕ್ತ ಚೇತಸಃ~।}\\\textbf{ತೇಷಾಂ ಮುಕ್ತಿಃ ಕಥಂ ಬ್ರಹ್ಮನ್ ರತಿಃ ಧರ್ಮೇ ಕಥಂ ಭವೇತ್~।। ೩~।।}
\end{verse}

ಧರ್ಮಾಚರಣೆಯನ್ನು ಮಾಡದೆ ಐಹಿಕವಿಷಯಸುಖಗಳಲ್ಲಿ ಮಗ್ನರಾದ ಅಂತಹವರಿಗೆ ಸಂಸಾರದಿಂದ ಬಿಡುಗಡೆಯು ಹೇಗೆ ಆದೀತು? ಧರ್ಮದಲ್ಲಿ ಶ್ರದ್ಧೆಯು ಹೇಗೆ ಉಂಟಾದೀತು?

\begin{verse}
\textbf{ಕೃಪಾವಿಷ್ಟೇನ ಚಿತ್ತೇನ ನರಾನೇತಾನ್ ಸಮುದ್ಧರ~।}\\\textbf{ಧರ್ಮಾನಪೇಕ್ಷ್ಯ ವಚಸಾ ವಿಷ್ಣು ಭಕ್ತಿವಿವರ್ಧನಾನ್~।। ೪~।।}
\end{verse}

ಧರ್ಮವನ್ನು ಉಪದೇಶಮಾಡುವ ವಾಕ್ಯಗಳಿಂದ ವಿಷ್ಣುವಿನಲ್ಲಿ ಭಕ್ತಿಯು ಹುಟ್ಟುವಂತೆ ಮಾಡಿ ಇಂತಹ ಜನರನ್ನು ಕೃಪೆಯಿಂದ ಉದ್ಧರಿಸು.

\begin{verse}
\textbf{ವಹ್ನಿಮಾದಾತುಕಾಮಾನಾಂ ಬಾಲಾನಾಂ ಮಾತರೋ ಯಥಾ~।}\\\textbf{ನಿವರ್ತಯಿತ್ವಾ ಸುರುಚಿಂ ಜನಯಂತಿ ಫಲೇ ಯಥಾ~।। ೫~।।}
\end{verse}

ಆಟವಾಡಲು ಬೆಂಕಿಯನ್ನು ತರುತ್ತಿರುವ ಮಕ್ಕಳಿಗೆ ತಾಯಂದಿರು ಹಣ್ಣುಗಳಲ್ಲಿ ರುಚಿಯನ್ನು ಹುಟ್ಟಿಸಿ ಬೆಂಕಿಯಿಂದ ನಿವಾರಣೆಮಾಡುವ ಬಗೆಯಲ್ಲಿ

\begin{verse}
\textbf{ತಥಾ ಭಾಗವತಾ ಲೋಕೇ ಸಾಧವಃ ಸಮದರ್ಶಿನಃ~।}\\\textbf{ನಿವರ್ತಯಿತ್ವಾ ಬೋಧೇನ ವಿಷಯಾನಲಮಿಚ್ಛತಃ~।। ೬~।।}
\end{verse}

\begin{verse}
\textbf{ಜಂತೂನ್ ಧರ್ಮಫಲಂ ತೇಷಾಂ ದರ್ಶಯಂತಿ ಹಿತೇಚ್ಛಯಾ~।}\\\textbf{ಪಿತರೋ ಮಾತರೋಽಜ್ಞಾನಾಂ ಶಾಸ್ತಾರಃ ಸಾಧವೋ ನೃಣಾಮ್~।। ೭~।।}
\end{verse}

ಲೋಕದಲ್ಲಿ, ಜಯಾಪಜಯ, ಮಾನಾಪಮಾನ, ಸುಖದುಃಖಗಳನ್ನು ಸಮನಾಗಿ ಪರಿಗಣಿಸುವ ಭಗವದ್ಭಕ್ತರಾದ ಸಜ್ಜನರು ವಿಷಯಸುಖಗಳಲ್ಲಿಯೇ ಮಗ್ನರಾಗಿರುವ ಜನರಿಗೆ ಹಿತವನ್ನು ಉಂಟುಮಾಡುವ ಇಚ್ಛೆಯಿಂದ ಧರ್ಮವನ್ನೂ, ಧರ್ಮದ ಫಲವನ್ನೂ ಉಪದೇಶಿಸಿ, ತಂದೆ-ತಾಯಿಯರು ಮಾಡುವಂತೆ, ಶಾಸನವನ್ನು ಮಾಡುತ್ತಾರೆ.

\begin{verse}
\textbf{ತ್ವಮಾದಿವಕ್ತಾ ಧರ್ಮಾಣಾಂ ತ್ವಂ ಸರ್ವಂ ವೇತ್ಸಿ ತತ್ತ್ವತಃ~।}\\\textbf{ಕರ್ಮಣಾ ಕೇನ ಕಾಲುಷ್ಯಂ ಮನಸೋ ಗಚ್ಛತಿ ಧ್ರುವಮ್~।। ೮~।।}
\end{verse}

ನೀನು ಧರ್ಮಗಳನ್ನೂ, ತತ್ತ್ವಗಳನ್ನೂ ಬಲ್ಲವನು; ಉಪದೇಶಮಾಡಲು ಸಮರ್ಥನು. ಯಾವ ಕರ್ಮದಿಂದ ಮನಸ್ಸಿನ ಮಾಲಿನ್ಯವು ನಿಶ್ಚಯವಾಗಿಯೂ ಹೋಗುತ್ತದೆ?

\begin{verse}
\textbf{ಧರ್ಮಾಸ್ತು ಬಹವಃ ಸಂತಿ ಶ್ರುತಿಸ್ಮೃ ತ್ಯುದಿತಾ ಅಪಿ~।}\\\textbf{ತೇ ಧರ್ಮಾಃ ಕಷ್ಟಸಾಧ್ಯಾ ಹಿ ತೇಷ್ವಾದೌ ವಿಮುಖಾ ನರಾಃ~।। ೯~।।}
\end{verse}

ಶ್ರುತಿಸ್ಮೃತಿಗಳಲ್ಲಿ ಬಹಳ ಧರ್ಮಗಳೇನೋ ಹೇಳಲ್ಪಟ್ಟಿವೆ. ಅವುಗಳನ್ನು ಆಚರಿಸುವುದು ಕಷ್ಟ. ಈ ಕಾರಣದಿಂದ ಜನರು ಅವುಗಳಿಂದ ದೂರರಾಗಿದ್ದಾರೆ.

\begin{verse}
\textbf{ವದ ಧರ್ಮಂ ಸುಖೋಪಾಸ್ಯಂ ಸರ್ವಬಂಧವಿನಾಶನಮ್~।}\\\textbf{ಯತ್ಕುರ್ವತೋಪೈತಿ ಸಮಸ್ತ ಬಂಧೋ }\\\textbf{ಜಾಯೇತ ಶುದ್ಧಾ ಮತಿರಾಸ್ತಿಕಾ ಚ~।}\\\textbf{ಭಕ್ತಿ ರ್ಹರೌ ತತ್ಪುರುಷೇಷು ಸಖ್ಯಂ }\\\textbf{ತದೈವ ತೇಷಾಂ ಹಿತಕಾಮ್ಯ ಯಾ ವದ~।। ೧೦~।।}
\end{verse}

ಯಾವುದನ್ನು ಆಚರಿಸುವುದರಿಂದ ಸಂಸಾರದ ಎಲ್ಲ ಬಂಧನಗಳ ಬಿಡುಗಡೆಯು ಆಗುತ್ತದೆಯೋ, ಅಂತಃಕರಣವು ಶುದ್ಧವಾಗುತ್ತದೆಯೋ, ಪಾಪಹರನಾದ ವಿಷ್ಣುವಿನಲ್ಲಿಯೂ ಮತ್ತು ಅವನ ಭಕ್ತರಲ್ಲಿಯೂ ಭಕ್ತಿಯು ಹುಟ್ಟುತ್ತದೆಯೋ ಅಂತಹ ಧರ್ಮವನ್ನು ಜನರ ಹಿತಕ್ಕೋಸ್ಕರವಾಗಿ ನುಡಿ.

\begin{flushleft}
\textbf{ಸೂತ ಉವಾಚ\enginline{-}}
\end{flushleft}

\begin{verse}
\textbf{ಇತಿ ತದ್ವಚನಂ ಶ್ರುತ್ವಾ ಬ್ರಹ್ಮಾ ಲೋಕಪಿತಾಮಹಃ~।}\\\textbf{ಸುಪ್ರಹೃಷ್ಟಮನಾ ಭೂತ್ವಾ ನಾರದಂ ವಾಕ್ಯಮಬ್ರವೀತ್~।। ೧೧~।।}
\end{verse}

\noindent
ಸೂತರು ಹೇಳಿದರು:-

ನಾರದರ ಈ ಮಾತುಗಳನ್ನು ಕೇಳಿ ಸಂತೋಷಗೊಂಡ ಬ್ರಹ್ಮದೇವರು ಹೀಗೆ ಹೇಳಿದರು-

\begin{flushleft}
\textbf{ಶ‍್ರೀ ಬ್ರಹ್ಮೋವಾಚ\enginline{-}}
\end{flushleft}

\begin{verse}
\textbf{ತ್ವಮೇವ ನೂನಂ ಬತ ಭೂರಿಭಾಗೋಽ}\\\textbf{ನಿಮಿತ್ತ ಬಂಧುರ್ಭುವಿ ಮಾನವಾನಾಮ್~।}\\\textbf{ತ್ವಮಗ್ರಣೀರ್ಧರ್ಮದೃಶಾಮೃಷೀನಾಂ } \\\textbf{ದಯಾಪರಾ ಭಾಗವತಾ ಹಿ ಲೋಕೇ~।। ೧೨~।।}
\end{verse}

ನಾರದನೇ, ನೀನು ನಿಜವಾಗಿಯೂ ಭಾಗ್ಯವಂತನೂ, ಲೋಕದ ಸಜ್ಜನರಿಗೆ ಮಿತ್ರನೂ, ಧರ್ಮವನ್ನರಿತ ಋಷಿಗಳಲ್ಲಿ ಶ್ರೇಷ್ಠನೂ, ದಯಾಪರನೂ ಮತ್ತು ಸ್ವಯಂ ಭಗವದ್ಭಕ್ತನೂ ಆಗಿರುವಿ.

\begin{verse}
\textbf{ಕಿಂ ವರ್ಣಯಾಮಿ ಮಮ ಭಾಗ್ಯಮಹಂ ಮಹಾತ್ಮಾ}\\\textbf{ಭವಾದೃಶೋ ಭಾಗವತಸ್ತನೂಭವಃ~।}\\\textbf{ತ ಏವ ಧನ್ಯಾಃ ಖಲು ಪುಣ್ಯಶೀಲಾಃ}\\\textbf{ಏಷಾಂ ಸುತಾ ಭಾಗವತಾ ಹಿ ಲೋಕೇ~।। ೧೩~।।}
\end{verse}

ನಿನ್ನಂತಹ ಭಗವದ್ಭಕ್ತನು ನನ್ನ ಮಗನಾಗಿರುವ ನನ್ನ ಭಾಗ್ಯವನ್ನು ಹೇಗೆ ವರ್ಣಿಸಲಿ? ನಿನ್ನಂತಹ ಪುಣ್ಯವಂತರು ಯಾರಿಗೆ ಮಕ್ಕಳಾಗಿರುತ್ತಾರೆಯೋ ಅವರೇ ಲೋಕದಲ್ಲಿ ಧನ್ಯರು.

\begin{verse}
\textbf{ಹಿತೇಚ್ಛಯಾ ತು ಮಂದಾನಾಂ ಕೃತಃ ಪ್ರಶ್ನಃ ತ್ವಯಾ ಮುನೇ~।}\\\textbf{ಸಂತೋಷಯತಿ ಲೋಕಾಂಶ್ಚ ಸಂತೋಷಯತಿ ಮೇ ಮನಃ~।। ೧೪~।।}
\end{verse}

ಅಜ್ಞಾನಿಗಳಿಗೆ ಹಿತವನ್ನುಂಟುಮಾಡಬೇಕೆಂಬ ಕಾರಣದಿಂದ ನೀನು ಈ ಪ್ರಶ್ನೆಯನ್ನು ಮಾಡಿರುವಿ. ಇದರಿಂದ ಎಲ್ಲ ಜನರಿಗೂ ಮತ್ತು ನನ್ನ ಮನಸ್ಸಿಗೂ ಸಂತೋಷವಾಯಿತು.\footnote{\phantom{*} ಕಿಮತ್ರ ಚಿತ್ರಂ ಯತ್ಸಂತಃ ಪರಾನುಗ್ರಹತತ್ಪರಾಃ~।\\ ನ ಹಿ ಸ್ವದೇಹಶೈ ತ್ಯಾಯ ಜಾಯಂತೇ ಚಂದನದ್ರುಮಾಃ~।।

ಸಜ್ಜನರು ಪರರಮೇಲೆ ಅನುಗ್ರಹ ಮಾಡಲು ಕಾತುರರು, ಗಂಧದ ಮರವು ತನ್ನ ದೇಹವನ್ನು ತಂಪುಮಾಡಿಕೊಳ್ಳುವು\-ದಿಲ್ಲ.

ನಾರದರು ಲೋಕದ ಅಜ್ಞನರ ಪರವಾಗಿ ಬ್ರಹ್ಮದೇವರನ್ನು ಪ್ರಾರ್ಥಿಸಿದರು ಎಂಬುದು ಅವರ ಸ್ವಭಾವದ್ಯೋತಕ\-ವಾಗಿದೆ.}

\begin{verse}
\textbf{ಪುನಾತಿ ಶಾಸ್ತ್ರ ಸಂಪ್ರಶ್ನೋ ವಾಸುದೇವಕಥಾಶ್ರಯಃ~।}\\\textbf{ವಕ್ತಾರಂ ಪೃಚ್ಛಕಂ ಶ್ರೋತೄನ್ ಸದ್ಯಸ್ತಾನ್ಪುರುಷಾನ್ಮುನೇ~।। ೧೫~।।}
\end{verse}

ಪರಮಾತ್ಮನ ಮಾಹಾತ್ಮ್ಯೆಯಿಂದ ಸಹಿತವಾದ ಕಥಾಶ್ರವಣ ಮಾಡಬೇಕೆಂದು ಪ್ರಾರ್ಥಿಸು\-ವವರೂ, ಶ್ರವಣಮಾಡಿಸುವವರೂ, ಅದನ್ನು ಆಲಿಸುವ ಇತರರೂ ಕೂಡಲೇ ಪಾವನ\-ರಾಗುತ್ತಾರೆ.

\begin{verse}
\textbf{ಪ್ರಣಮ್ಯ ವಿಷ್ಣುಂ ಪುರುಷಂ} \\\textbf{ಪುರಾಣಂ ನಿರಾಮಯಂ ನಿಶ್ಚಲಮದ್ವಿತೀಯಮ್~।}\\\textbf{ವಕ್ಷ್ಯಾಮಿ ಗೋಪ್ಯಂ ಸುಲಭಂ ಶುಭಾವಹಂ}\\\textbf{ಧರ್ಮಂ ತ್ವಯಾ ಪೃಷ್ಟಮತೋ ಮಹಾತ್ಮನ್~।। ೧೬~।।}
\end{verse}

ಪುರಾಣಪುರುಷನೂ, ನಿಶ್ಚಲನೂ, ಸಮಾಧಿಕರಹಿತನೂ, ನಿರ್ದೋಷಿಯೂ, ಸರ್ವತ್ರ ವ್ಯಾಪ್ತನೂ ಆದ ವಿಷ್ಣುವನ್ನು ನಮಸ್ಕರಿಸಿ ನಿನ್ನಿಂದ ಪ್ರಶ್ನಿಸಲ್ಪಟ್ಟ ಆಚರಣೆಗೆ ಸುಲಭವಾದ, ಮಂಗಳಕರವಾದ, ರಹಸ್ಯವಾದ ಧರ್ಮವನ್ನು ಹೇಳುವೆನು.

\begin{verse}
\textbf{ಸರ್ವೇಷಾಮೇವ ಧರ್ಮಾಣಾಂ ಸ್ನಾನಂ ಮಾ ಘೇ ವಿದುರ್ಬುಧಾಃ~।}\\\textbf{ವಿನಾ ಸ್ನಾನಂ ಕೃತಂ ಕರ್ಮ ಗಜಭುಕ್ತ ಕಪಿತ್ಥವತ್~।। ೧೭~।।}
\end{verse}

ಎಲ್ಲ ಧರ್ಮಗಳಲ್ಲಿಯೂ ಮಾಘಮಾಸದಲ್ಲಿ ಸಕಾಲದಲ್ಲಿ ಸ್ನಾನಮಾಡುವುದು ಶ್ರೇಷ್ಠವೆಂದು ಜ್ಞಾನಿಗಳ ಮತವು. ಸ್ನಾನವಿಲ್ಲದೆ ಮಾಡಿದ ಕರ್ಮವೆಲ್ಲವೂ ಆನೆಯು ತಿಂದ ಬೇಲದ ಹಣ್ಣಿನಂತೆ ವ್ಯರ್ಥ.

\begin{verse}
\textbf{ತತ್ರಾಪ್ಯುಷಸಿ ಸುಸ್ನಾಯಾತ್ ವಿಷ್ಣು ಧರ್ಮಪರಾಯಣಾಃ~।}\\\textbf{ನ ತಸ್ಮಿನ್ ಪಾಪಲೇಶೋಸ್ತಿ ಕೃತಘ್ನೇ ಸುಕೃತಂ ಯಥಾ~।। ೧೮~।।}
\end{verse}

ಕೃತಘ್ನರಲ್ಲಿ ಪುಣ್ಯ ಲೇಶವೂ ಹೇಗೆ ನಿಲ್ಲುವುದಿಲ್ಲವೋ ಹಾಗೆ ಅರುಣೋದಯದಲ್ಲಿ ಮಾಘಸ್ನಾನ ಮಾಡುವ ವಿಷ್ಣು ಭಕ್ತರಲ್ಲಿ ಪಾಪಲೇಶವೂ ಇರುವುದಿಲ್ಲ.

\begin{verse}
\textbf{ಶೌಚಂ ಸ್ನಾನಂ ದ್ವಯಂ ಮಾತಾಪಿತರೌ ಸರ್ವಕರ್ಮಣಾಮ್~।}\\\textbf{ಪ್ರಾತಃಸ್ನಾನೇ ಹ್ಯಶಕ್ತಸ್ತು ಕುರ್ಯಾತ್ ಮಾಸತ್ರಯಂ ಬುಧಃ~।। ೧೯~।।}
\end{verse}

ಎಲ್ಲ ಕರ್ಮಾರಂಭದಲ್ಲಿಯ ಶೌಚ, ಸ್ನಾನ, ಇವು ತಂದೆತಾಯಿಯರಂತೆ ಅಂದರೆ ಬಿಡಲು ಸಾಧ್ಯವಿಲ್ಲ. ಅರುಣೋದಯದಲ್ಲಿ ಸರ್ವದಾ ಸ್ನಾನಮಾಡಲು ಅಶಕ್ತನಾದ ಜ್ಞಾನಿಯು ವರ್ಷದಲ್ಲಿ ಮೂರು ತಿಂಗಳಾದರೂ ಮಾಡಬೇಕು.

\begin{verse}
\textbf{ತುಲಾಸಂಸ್ಥೇ ದಿನಕರೇ ಮಕರಸ್ಥೇ ಚ ಭಾಸ್ವತಿ~।}\\\textbf{ಮೇಷಸ್ಥೇ ಚ ಸದಾ ಕುರ್ಯಾತ್ ಪ್ರಾತಃಸ್ನಾನಮತಂದ್ರಿತಃ~।। ೨೦~।।}
\end{verse}

ಮೇಷ, ತುಲಾ ಮತ್ತು ಮಕರ ಈ ರಾಶಿಗಳಲ್ಲಿ ಅರುಣೋದಯದಲ್ಲಿ ಸೂರ್ಯನಿರುವ ಮಾಸಗಳಲ್ಲಿ ಅಂದರೆ ವೈಶಾಖ, ಕಾರ್ತಿಕ ಮತ್ತು ಮಾಘಮಾಸಗಳಲ್ಲಿ ಆಲಸ್ಯವನ್ನು ಬಿಟ್ಟು ಪ್ರಾತಃಸ್ನಾನ ಮಾಡಬೇಕು.

\begin{verse}
\textbf{ನ ಕುರ್ಯಾತ್ ತ್ರಿಷು ಲೋಕೇಷು ಪ್ರಾತಃಸ್ನಾನಂ ನರಾಧಮಃ~।}\\\textbf{ರೌರವಂ ನರಕಂ ಯಾತಿ ಯಾವದಿಂದ್ರಾಶ್ಚತುರ್ದಶ~।। ೨೧~।।}
\end{verse}

ಹೀಗೆ ಪ್ರಾತಃಸ್ನಾನ ಮಾಡದೇ ಇರುವ ನೀಚನು ಹದಿನಾಲ್ಕು ಜನ ಇಂದ್ರರು ಇರುವ ತನಕ ರೌರವವೆಂಬ ನರಕದಲ್ಲಿ ಬೀಳುವನು.

\begin{verse}
\textbf{ಕಾರ್ತಿಕಃ ಸರ್ವಮಾಸೇಭ್ಯಃ ಸಹಸ್ರಫಲದೋ ವಿದುಃ~।}\\\textbf{ತಸ್ಮಾಲ್ಲ ಕ್ಷಗುಣೋ ಮಾಘ ಇತಿ ಪ್ರಾಹ ಜನಾರ್ದನಃ~।। ೨೨~।।}
\end{verse}

ಇತರ ಮಾಸಗಳಿಗಿಂತ ಕಾರ್ತಿಕಸ್ನಾನವು ಸಹಸ್ರ ಫಲ ಕೊಡತಕ್ಕದ್ದು ಎಂದು ಹೇಳುವರು. ಆದರೆ ಮಾಘಸ್ನಾನವು ಅದಕ್ಕಿಂತಲೂ ಲಕ್ಷದಷ್ಟು ಹೆಚ್ಚು ಫಲದಾಯಕವೆಂದು ಶ‍್ರೀಹರಿಯು ಹೇಳಿರುತ್ತಾನೆ.

\begin{verse}
\textbf{ನ ಮಾಘಸ್ನಾನಾತ್ಪರಮೋಸ್ತಿ ಧರ್ಮಃ~।}\\\textbf{ಸಂಸಾರದುಃಖಂ ವಿನಿವರ್ತಿತುಂ ಕಲೌ~।।}\\\textbf{ಯಥಾ ಕಲೌ ವಿಷ್ಣು ಕಥಾ ಪರಾಗತಿಃ~।}\\\textbf{ತಥೈವ ಸ್ನಾನಂ ಮಕರೇ ದಿವಾಕರೇ~।। ೨೩~।।}
\end{verse}

ಹೇಗೆ ಕಲಿಯುಗದಲ್ಲಿ ವಿಷ್ಣು ಮಾಹಾತ್ಮ್ಯೆಯ ಶ್ರವಣಮನನಾದಿಗಳು ಮೋಕ್ಷವನ್ನು ದೊರಕಿಸಿಕೊಡುತ್ತವೆಯೋ ಹಾಗೆ ಜನನಮರಣರೂಪವಾದ ಸಂಸಾರದಿಂದ ಬಿಡುಗಡೆಯನ್ನು ಮಾಡುವ ಧರ್ಮಗಳಲ್ಲಿ ಮಾಘಸ್ನಾನಕ್ಕಿಂತ (ಮಕರರಾಶಿಯಲ್ಲಿ ಅರುಣೋದಯದಲ್ಲಿ\break ಸೂರ್ಯನಿರುವಾಗ ಮಾಡುವ ಸ್ನಾನಕ್ಕಿಂತ) ಉತ್ತಮವಾದುದು ಕಲಿಯುಗದಲ್ಲಿ ಬೇರೊಂದಿಲ್ಲ.

\begin{verse}
\textbf{ಪಾಪಾನಿ ವಿಲಯಂ ಯಾಂತಿ ತಮಾಂಸೀವ ಭಗೋದಯೇ~।}\\\textbf{ಸಮಾಗಚ್ಛಂತಿ ಪುಣ್ಯಾನಿ ಯಥಾ ಗೃರುಕೃಪೋದಯೇ~।। ೨೪~।।}
\end{verse}

ಸೂರ್ಯೋದಯವಾಗಲು ಕತ್ತಲೆಯು ನಾಶವಾಗುವಂತೆಯೂ, ಮತ್ತು ಸದ್ಗುರುಗಳ ಅನುಗ್ರಹವಾಗಲು ಪುಣ್ಯವು ಪ್ರಾಪ್ತವಾಗುವಂತೆಯೂ, ಇಂತಹ ಮಾಘಸ್ನಾನದಿಂದ ಪಾಪಗಳು ಪರಿಹಾರವಾಗುತ್ತವೆ.

\begin{verse}
\textbf{ಜನ್ಮಕೋಟಿಸಹಸ್ರೇಷು ಯತ್ಪಾಪಂ ಸಮುಪಾರ್ಜಿತಮ್~।}\\\textbf{ತತ್ಪಾಪಂ ಸಕಲಂ ಘೋರಂ ಮಾಘಸ್ನಾನೇನ ನಶ್ಯತಿ~।। ೨೫~।।}
\end{verse}

ಸಹಸ್ರಕೋಟಿ ಜನ್ಮಗಳಲ್ಲಿ ಆಚರಿಸಿದ ಪಾಪಗಳು ಮಾಘಸ್ನಾನದಿಂದ ನಾಶವಾಗುತ್ತವೆ.

\begin{verse}
\textbf{ನ ಮಾಘಮಾಸಾತ್ಪರಮೋ ಹಿ ಲಾಭೋ}\\\textbf{ನ ಮಾಘಸ್ನಾನಾದಪರಾ ಪರಾ ಗತಿಃ~।}\\\textbf{ನ ಮಾಘಸ್ನಾನಾತ್ಪರಮೋsಸ್ತಿ ಧರ್ಮೋ}\\\textbf{ನ ಮಾಘಸ್ನಾನಾದಪರಃ ಪರಾಯಣಃ~।। ೨೬~।।}
\end{verse}

ಮಾಘಸ್ನಾನಕ್ಕಿಂತ ಲಾಭಕರವಾದುದು ಬೇರೊಂದಿಲ್ಲ, ಮೋಕ್ಷಕ್ಕೆ ಮಾಘ ಸ್ನಾನಕ್ಕಿಂತ ಬೇರೆ ಸಾಧನವಿಲ್ಲ, ಬೇರೆ ಧರ್ಮವಿಲ್ಲ, ಭಕ್ತಿಯ ಅಭಿವೃದ್ಧಿಗೆ ಅದಕ್ಕಿಂತ ಬೇರೆ ಯಾವ ಉಪಾಯವೂ ಇಲ್ಲ.

\begin{verse}
\textbf{ಅಕೃತಂ ಮಾಘಮಾಸೇ ತು ಪ್ರಾತಃಸ್ನಾನಂ ನರಾಧಮಃ~।}\\\textbf{ಪೈಶಾಚೀಂ ಯೋನಿಮಾಸಾದ್ಯ ತಿಷ್ಠತ್ಯಾಭೂತಸಂಪ್ಲವಮ್~।। ೨೭~।।}
\end{verse}

ಮಾಘಮಾಸದಲ್ಲಿ ಪ್ರಾತಃಕಾಲದಲ್ಲಿ ಸ್ನಾನವನ್ನು ಮಾಡದ ನೀಚನು ಪಿಶಾಚಿಯಾಗಿ ಜಲಾಶಯಗಳ ಬಳಿ ಇರುವನು.

\begin{verse}
\textbf{ನ ಪಾಪಲೇಶೋ ನ ಚ ಕರ್ಮಬಂಧಃ}\\\textbf{ಪುನರ್ಭವೋ ವಾ ಮನುಜೇ ಮುನೀಂದ್ರ~।}\\\textbf{ಯೇನಾಹ ಮಾಘೇ ಮಕರಸ್ಥಭಾನೌ} \\\textbf{ನದ್ಯಾಂ ಜಲೇ ಸ್ನಾನಮನೂದಿತಾರ್ಕೇ~।। ೨೮~।।}
\end{verse}

ಮುನಿಶ್ರೇಷ್ಠನೇ, ಮಾಘಮಾಸದಲ್ಲಿ ಸೂರ್ಯನು ಮಕರರಾಶಿಯಲ್ಲಿರುವಾಗ ಸೂರ್ಯೋದಯಕ್ಕೆ ಮೊದಲು ನದಿಯೇ ಮುಂತಾದ ಜಲಾಶಯಗಳಲ್ಲಿ ಸ್ನಾನ ಮಾಡುವವನಿಗೆ ಪಾಪಸ್ಪರ್ಶವಿಲ್ಲ, ಕರ್ಮದಿಂದ ಬಂಧನವಿಲ್ಲ, ಪುನರ್ಜನ್ಮವಿಲ್ಲ.

\begin{verse}
\textbf{ಗಂಗಾದ್ಯಾಃ ಸರಿತಃ ಸರ್ವಾ ದೇವಖಾತಾನಿ ಯಾನಿ ಚ~।}\\\textbf{ಪುಷ್ಕರಿಣ್ಯೋ ಜಲಾವಾಪ್ಯೋ ಯೇ ಚಾಸ್ಯೇ ಚ ಜಲಾಶಯಾಃ~।। ೨೯~।।}
\end{verse}

ಗಂಗಾದಿ ನದಿಗಳೂ, ದೇವಾಲಯಗಳಲ್ಲಿರುವ ಸರೋವರಗಳೂ, ಕೆರೆಗಳೂ ಮತ್ತು ಭಾವಿಗಳೂ ಜಲಾಶಯಗಳೆನಿಸುತ್ತವೆ.

\begin{verse}
\textbf{ಸನ್ನಿಧಾನಂ ಪ್ರಕುರ್ವೀತ ಜಲಮಾತ್ರಾಶಯಂ ಮುನೇ~।}\\\textbf{ಪ್ರಾತಃಕಾಲೇ ಮಾಘಮಾಸಿ ಮಕರಸ್ಥೇ ದಿವಾಕರೇ~।। ೩೦~।।}
\end{verse}

ನಾರದನೇ, ಮಾಘದಲ್ಲಿ ಸೂರ್ಯನು ಮಕರರಾಶಿಯಲ್ಲಿರುವಾಗ ಇಂತಹ ಯಾವುದಾದರೊಂದು ಜಲಾಶಯಕ್ಕೆ ಹೋಗಬೇಕು.

\begin{verse}
\textbf{ಗ್ರಾಮಾದ್ಬಹಿರ್ಜಲೇ ಸ್ನಾತ್ವಾ ಮುಚ್ಯೇತ ಸರ್ವಕಿಲ್ಬಿಷೈಃ~।। ೩೧~।।}
\end{verse}

ಊರಿನಿಂದ ಹೊರಗಿರತಕ್ಕ ಜಲಾಶಯಗಳಲ್ಲಿ ಸ್ನಾನಮಾಡುವುದರಿಂದ ಸಕಲ ಪಾಪಗಳಿಂದಲೂ ಮುಕ್ತನಾಗುತ್ತಾನೆ.

\begin{verse}
\textbf{ಬ್ರಹ್ಮಘ್ನಂ ವಾ ಪಿತೃಘ್ನಂ ವಾ ಗೋಘ್ನಂ ಭ್ರೂಣಹನಂ ತಥಾ~।}\\\textbf{ಸ್ತ್ರೀಘ್ನಂ ಸುರಾಪಂ ಮಿತ್ರಘ್ನಂ ಸ್ತೇಯಿನಂ ಗುರುತಲ್ಪಗಮ್~।। ೩೨~।।}
\end{verse}

\begin{verse}
\textbf{ಅಭಕ್ಷ್ಯ ಭಕ್ಷಣಂ ಕ್ರೂರಂ ಸದಾ ಪೈಶುನ್ಯವಾದಿನಮ್~।}\\\textbf{ಅಧರ್ಮನಿರತಂ ಡಾಂಭಂ ತಥೈವಾನೃತವಾದಿನಮ್~।। ೩೩~।।}
\end{verse}

\begin{verse}
\textbf{ಪರಸ್ವಸೂಚಕದ್ರವ್ಯಂ ಗುರುದ್ರವ್ಯಾಪಹಾರಿಣಮ್~।}\\\textbf{ನಿಂದಕಂ ಸಾಧುವೃತ್ತಾನಾಂ ತಥೈವೋನ್ಮಾರ್ಗವರ್ತಿನಾಮ್~।। ೩೪~।।}
\end{verse}

\begin{verse}
\textbf{ಹೇತುವಾದರತಂ ದ್ಯೂತಕರ್ಮಾಣಂ ನಿಂದಿತಾಶ್ರಯಮ್~।}\\\textbf{ಅಯಾಜ್ಯ ಯಾಜಕಂ ವಿಪ್ರಂ ತಥಾ ದುಷ್ಟಪ್ರತಿಗ್ರಹಮ್~।। ೩೫~।।}
\end{verse}

\begin{verse}
\textbf{ಪರದಾರಾಭಿನಿರತಂ ಪರದ್ರವ್ಯಾಪಹಾರಿಣಮ್~।}\\\textbf{ವೃಥಾಪವಾದಚ್ಚೇತ್ತಾರಂ ವರ್ಣಸಂಕರಕಾರಿಣಮ್~।। ೩೬~।।}
\end{verse}

\begin{verse}
\textbf{ಗೃಹೇಷ್ವಗ್ನಿ ಪ್ರದಾತಾರಂ ವಿಷಕರ್ಮಪ್ರಯೋಗಿನಮ್~।}\\\textbf{ಪಂಕ್ತಿಭೇದಂ ಪ್ರಕುರ್ವಾಣಂ ದಂಪತ್ಯೋ ವಿರಹಪ್ರದಮ್~।। ೩೭~।।}
\end{verse}

\begin{verse}
\textbf{ಏವಮಾದೀನಿ ಪಾಪಾನಿ ಕುರ್ವಾಣಮಪಿ ನಾರದ~।}\\\textbf{ಪ್ರಾತಃಸ್ನಾನಂ ಪುನಾತೀತಿ ಮಾಘೋ ಗರ್ಜತಿ ಗರ್ಜತಿ~।। ೩೮~।।}
\end{verse}

ಬ್ರಹ್ಮಹತ್ಯ, ತಂದೆತಾಯಿಯರ ಹತ್ಯ, ಗೋಹತ್ಯ, ಶಿಶುಹತ್ಯ, ಸ್ತ್ರೀಹತ್ಯ, ಸುರಾಪಾನ, ಮಿತ್ರನ ಹತ್ಯ, ಕಳ್ಳತನ, ಗುರುಪತ್ನೀಗಮನ, ನಿಷಿದ್ಧ ಪದಾರ್ಥ ತಿನ್ನುವುದು, ಕ್ರೂರತನ, ಚಾಡಿ ಹೇಳುವುದು, ಅಧರ್ಮಕರ್ಮಮಾಡುವುದು, ಡಂಭಾಚಾರ, ಸುಳ್ಳು ಹೇಳುವುದು, ಪರದ್ರವ್ಯ\enginline{-}ಗುರುದ್ರವ್ಯಗಳನ್ನು ಅಪಹರಿಸುವುದು, ಸಜ್ಜನರನ್ನು ನಿಂದಿಸುವುದು, ದುರಾಚಾರದಲ್ಲಿ ನಿರತನಾಗಿರುವುದು, ಜೂಜಾಡುವುದು, ದುರ್ಜನರ ಆಶ್ರಯದಲ್ಲಿರುವುದು, ಬಹಿಷ್ಕೃತನಾದ ಪುರೋಹಿತನಿಂದ ಕರ್ಮಗಳನ್ನು ಮಾಡಿಸುವುದು, ದುರ್ಜನರಿಂದ ನಿಂದಿತವಸ್ತುಗಳನ್ನು ದಾನವನ್ನಾಗಿ ಸ್ವೀಕರಿಸುವುದು, ಪರಸ್ತ್ರೀಯರಲ್ಲಿ ಅನುರಕ್ತನಾಗಿರುವುದು, ವ್ಯರ್ಥವಾದ ಅಪವಾದ ಪ್ರಚಾರದಲ್ಲಿ ತೊಡಗುವುದು, ಜಾತಿಗಳ ಕಲಬೆರಕೆಯಲ್ಲಿ ಆಸಕ್ತನಾಗಿರುವುದು, ಮನೆಗೆ ಬೆಂಕಿ ಹಾಕುವುದು, ಇತರರಿಗೆ ವಿಷಪ್ರಯೋಗಮಾಡುವುದು, ಭೋಜನ ಪಂಕ್ತಿಯಲ್ಲಿ ಭೇದಮಾಡುವುದು, ದಂಪತಿಗಳನ್ನು ಕಾರಣವಿಲ್ಲದೆ ಬೇರೆಮಾಡುವುದು ಇವೇ ಮೊದಲಾದ ಪಾಪಗಳನ್ನು ಮಾಡಿದ್ದರೂ ಸಹ, ನಾರದನೇ, ಮಾಘಮಾಸದ ಪ್ರಾತಃಸ್ನಾನವು ಪಾಪಮುಕ್ತರನ್ನಾಗಿ ಮಾಡುತ್ತದೆಯೆಂದು ಗರ್ಜಿಸುತ್ತದೆ.

\begin{verse}
\textbf{ಕಿಂಚಿದಭ್ಯುತಿತೇ ಭಾನೌ ರಟಂತ್ಯಾಪೋ ದಿವಾನಿಶಮ್~।। ೩೯~।।}
\end{verse}

ಸೂರ್ಯೋದಯಕ್ಕೆ ಮುಂಚೆ ಸ್ನಾನಮಾಡಲು ಮೇಲೆ ಹೇಳಿದ ಪಾಪಗಳನ್ನು ನಾಶಮಾಡುತ್ತೇವೆಯೆಂದು ಜಲಾಶಯಗಳು ಕೂಗಿ ಹೇಳುತ್ತವೆ.

\begin{verse}
\textbf{ಅತ್ಯುಗ್ರಪಾಪಿನಂ ವಾಪಿ ಪ್ರಾತಃಸ್ನಾನಾತ್ಪುನೀಮಹೇ~।}\\\textbf{ತಿಸ್ರಃ ಕೋಟ್ಯೋರ್ಧಕೋಟ್ಯ ಸ್ತು ತೀರ್ಥಾನಿ ಭುವನತ್ರಯೇ~।। ೪೦~।।}\\\textbf{ನಿರೀಕ್ಷ್ಯ, ಮಾಘಮಾಸಂ ತು ಪ್ರಾತಃ ಪ್ರಾತರ್ಜಲಾಶ್ರಯೇ~।}\\\textbf{ಅಸ್ನಾನಂ ಪಾಪಿನಂ ಮೂಢಂ ಶಾಪಂ ದತ್ವಾ ವ್ರಜಂತಿ ಚ~।। ೪೧~।।}
\end{verse}

ಮೂರು ಲೋಕಗಳಲ್ಲಿರುವ ಮೂರೂವರೆ ಕೋಟಿ ಪವಿತ್ರ ತೀರ್ಥಗಳು ಮಾಘ ಮಾಸದಲ್ಲಿ ಪ್ರಾತಃಕಾಲದಲ್ಲಿ ನದಿಯೇ ಮುಂತಾದ ಜಲಾಶಯಗಳಲ್ಲಿ ಇದ್ದು ಸ್ನಾನ ಮಾಡುವ ಉಗ್ರ ಪಾಪಿಗಳನ್ನೂ ಪುನೀತರನ್ನಾಗಿ ಮಾಡುತ್ತೇವೆಂದು ಸಾರಿ ಹೇಳುತ್ತವೆ. ಸ್ನಾನಮಾಡದೇ ಇರುವ ಮೂರ್ಖರಿಗೆ ಶಾಪವನ್ನು ಕೊಟ್ಟು ಆ ತೀರ್ಥಗಳು ಹೊರಟುಹೋಗುತ್ತವೆ.

\begin{verse}
\textbf{ಅಸ್ನಾತೋ ಮಾಘಮಾಸೇ ತು ಪ್ರಾತರ್ನಾಡೀಚತುಷ್ಟಯೇ~।}\\\textbf{ನರಶ್ಚಾಂಡಾಲತಾಂ ಯಾತಿ ನಾರೀ ಚೇದ್ವಿಧವಾ ಭವೇತ್~।। ೪೨~।।}
\end{verse}

ಮಾಘಮಾಸದಲ್ಲಿ ಪ್ರಾತಃಕಾಲದಲ್ಲಿ ಇನ್ನೂ ನಾಲ್ಕು ಘಳಿಗೆಯಷ್ಟು ರಾತ್ರಿ ಇರುವಾಗಲೇ ಯಾರು ಸ್ನಾನಮಾಡುವುದಿಲ್ಲವೋ ಅಂತಹವರು ಚಂಡಾಲರಾಗುತ್ತಾರೆ; ಅಂತಹ ಸ್ತ್ರೀಯರು ವಿಧವೆಯಾಗುವರು.

\begin{verse}
\textbf{ಸಮ್ಯಕ್‌ಚೀರ್ಣಾನಿ ಪುಣ್ಯಾನಿ ವಿಷ್ಣು ಧರ್ಮಪರಾಜ್ ಮುಖಾನ್~।}\\\textbf{ನಿಃಪುನಂತಿ ರಾಜೇಂದ್ರ ಸತ್ಯಂ ಸತ್ಯಂ ಮಯೋದಿತಮ್~।। ೪೩~।।}
\end{verse}

ವೈಷ್ಣವಧರ್ಮದಿಂದ ವಿಮುಖರಾದ ಜನರ ಪುಣ್ಯ ಕಾರ್ಯಗಳೆಲ್ಲ ಭಗ್ನವಾಗುತ್ತವೆ, ಮುಕ್ತಿಗೆ ಸಾಧನವಾಗುವುದಿಲ್ಲ. ನಾರದನೇ, ನನ್ನಿಂದ ಹೇಳಲ್ಪಟ್ಟ ಈ ಮಾತು ಸತ್ಯ.

\begin{verse}
\textbf{ಅನಂತಜನ್ಮಾನುಗತಂ ಮಹಾಂತಂ}\\\textbf{ದುಷ್ಕರ್ಮಮೂಲಂ ಪ್ರತಿಬಂಧಮುಗ್ರಮ್~।}\\\textbf{ಸ್ನಾನಂ ರವೌ ಚಾಭ್ಯುದಿತೇ ನರಾಣಾಂ }\\\textbf{ನಿರ್ವರ್ತಯತ್ವಾಶು ರವಿರ್ಯಥಾ ಹಿಮಮ್~।। ೪೪~।।}
\end{verse}

ಸೂರ್ಯನು ಉದಯವಾಗಲು ಹಿಮ (ಚಳಿ)ವು ಹೇಗೆ ಮಾಯವಾಗುತ್ತದೆಯೋ, ಹಾಗೆ ಪ್ರಾತಃಸ್ನಾನವು ಮಹತ್ತಾದ ಪಾಪವನ್ನೂ, ಬಂಧನವನ್ನೂ ಕಳೆಯುತ್ತದೆ.

\begin{verse}
\textbf{ಸತ್ವಸ್ಯ ಶುದ್ಧಿಸ್ತು ಅಚಿರೇಣ ಪುಂಸೋ}\\\textbf{ಭಕ್ತಿ ರ್ಹರೌ ತತ್ಪುರುಷೇಷು ಜಾಯತೇ~।}\\\textbf{ನಿರೂಪಮಂ ಭಾಗ್ಯಮನಲ್ಪಮಾಯು\enginline{-}}\\\textbf{ರ್ಮಾಘೇ ಬಹಿಃಸ್ನಾನರತಸ್ಯ ಜಾಯತೇ~।। ೪೫~।।}
\end{verse}

ಮಾಘಮಾಸದಲ್ಲಿ ಊರಹೊರಗಿರುವ ಜಲಾಶಯಗಳಲ್ಲಿ ಸ್ನಾನಮಾಡುವವನಿಗೆ ಅಂತಃ\-ಕರಣ ಶುದ್ಧಿಯು ಶೀಘ್ರವಾಗಿ ದೊರೆತು, ಶ‍್ರೀಹರಿಯಲ್ಲಿಯೂ, ಶ‍್ರೀಹರಿಯ ಭಕ್ತರಲ್ಲಿಯೂ ವಿಶೇಷವಾದ ಭಕ್ತಿಯು ಅಭಿವೃದ್ಧಿಯಾಗಿ, ಮಹತ್ತಾದ ಭಾಗ್ಯವು ಪ್ರಾಪ್ತವಾಗಿ ಆಯುಸ್ಸು ವೃದ್ಧಿಯಾಗುತ್ತದೆ.

\begin{verse}
\textbf{ಮಾಘಸ್ನಾನಸ್ಯ ಮಾಹಾತ್ಮ್ಯಂ ವಾಚ್ಯಮಾನಂ ದ್ವಿಜೋತ್ತಮ~।}\\\textbf{ನ ಶೃಣೋತಿ ವಿಮೂಢಾತ್ಮಾ ತಸ್ಮಾತ್ಕೋಽನ್ವಪರಃ ಪಶುಃ~।। ೪೬~।।}
\end{verse}

ಮಾಘಸ್ನಾನದ ಶ್ರೇಷ್ಠತೆಯನ್ನು ಕುರಿತು ಪ್ರವಚನ ನಡೆಯುತ್ತಿರುವಾಗ ಅದನ್ನು ಶ್ರವಣಮಾಡದೇ ಇರುವವನಿಗಿಂತಲೂ ಬೇರೆ ಪಶು ಇರುವುದೇ? ಅಂತಹವನೇ ಪಶು ಸಮಾನ.

\begin{verse}
\textbf{ಯಥಾ ಗಂಗಾನದೀನಾಂ ತು ದೇವಾನಾಂ ಚ ಯಥಾ ಹರಿಃ~।}\\\textbf{ವೃಕ್ಷಾಣಾಂ ಚ ಯಥಾಽಶ್ವತ್ಥಃ ಪಶೂನಾಂ ಗೌರ್ಯಥಾ ಮುನೇ~।। ೪೭~।। }
\end{verse}

\begin{verse}
\textbf{ತಥಾ ವೈ ಮಾಘಮಾಸೋsಯಂ ಮಾಸಾನಾಮುತ್ತಮೋತ್ತಮಃ~।}\\\textbf{ವೇದಾನಾಂ ಚ ಯಥಾ ಸಾಮ ಮಂತ್ರಾಣಾಂ ಪ್ರಣವೋ ಯಥಾ~।। ೪೮~।। }
\end{verse}

\begin{verse}
\textbf{ಗಾಯತ್ರೀ ಛಂದಸಾಂ ಯದ್ವತ್ ಪಕ್ಷಿಣಾಂ ಗರುಡೋ ಯಥಾ~।}\\\textbf{ವೈಷ್ಣವಾನಾಂ ಯಥಾ ರುದ್ರ ಋತೂನಾಂ ಮಾಧವೋ ಯಥಾ~।। ೪೯~।।}
\end{verse}

\begin{verse}
\textbf{ಮಾಸಾನಾಂ ಪ್ರವರೋ ಮಾಸೋ ಮಾಘಮಾಸ ಉದಾಹೃತಃ~।}\\\textbf{ಯಥಾಹೀನಾಂ ಶೇಷ ವರೋ ರೋಗಾಣಾಂ ಔಷಧಂ ಯಥಾ~।। ೫೦~।।}
\end{verse}

\begin{verse}
\textbf{ಪರ್ವತಾನಾಂ ಯಥಾ ವಜ್ರೋ ಮತಂಗಾನಾಂ ಯಥಾಂಕುಶಃ~।}\\\textbf{ಹಿಮಾನಾಂ ತು ಯಥಾ ವಹ್ನಿಃ ಕೌಷಿಕಾನಾಂ ಯಥಾ ರವಿಃ~।। ೫೧~।।}
\end{verse}

ನದಿಗಳಲ್ಲಿ ಗಂಗೆಯಂತೆ, ದೇವತೆಗಳಲ್ಲಿ ಶ‍್ರೀಹರಿಯಂತೆ, ವೃಕ್ಷಗಳಲ್ಲಿ ಅಶ್ವತ್ಥದಂತೆ, ಪಶುಗಳಲ್ಲಿ ಗೋವಿನಂತೆ, ಮಾಸಗಳಲ್ಲಿ ಮಾಘಮಾಸವು ತುಂಬ ಶ್ರೇಷ್ಠವಾದುದು. ವೇದಗಳಲ್ಲಿ ಸಾಮವೇದದಂತೆ, ಮಂತ್ರಗಳಲ್ಲಿ ಪ್ರಣವದಂತೆ (ಓಂ ಕಾರ) ಛಂದಸ್ಸುಗಳಲ್ಲಿ ಗಾಯತ್ರಿಯಂತೆ, ಪಕ್ಷಿಗಳಲ್ಲಿ ಗರುಡನಂತೆ, ವಿಷ್ಣು ಭಕ್ತರಲ್ಲಿ ರುದ್ರ ದೇವರಂತೆ (ಬ್ರಹ್ಮ, ವಾಯು, ಸರಸ್ವತಿ, ಭಾರತೀ ದೇವಿಯರನ್ನು ಬಿಟ್ಟು); ಋತುಗಳಲ್ಲಿ ವಸಂತದಂತೆ, ಸರ್ಪಗಳಲ್ಲಿ ಶೇಷನಂತೆ, ಮಾಸಗಳಲ್ಲಿ ಮಾಘವು ಶ್ರೇಷ್ಠವಾದುದು. ರೋಗಗಳಿಗೆ ಔಷಧವು, ಪರ್ವತಗಳಿಗೆ ವಜ್ರಾಯುಧವು, ಆನೆಗಳಿಗೆ ಅಂಕುಶವು, ಚಳಿಗೆ ಬೆಂಕಿ, ಗೂಬೆಗಳಿಗೆ ಸೂರ್ಯ.

\begin{verse}
\textbf{ಪಾಪಾನಾಂ ತು ಯಥಾ ಧರ್ಮಃ ದೈತ್ಯಾನಾಂ ಕೇಶವೋ ಯಥಾ~।}\\\textbf{ತಥಾ ಮನುಷ್ಯ ಬಂಧಾನಾಂ ಮಾಘಮಾಸೋ ನಿವರ್ತಕಃ~।। ೫೨~।।}
\end{verse}

ಪಾಪಗಳಿಗೆ ಧರ್ಮಾಚರಣೆಯು, ದೈತ್ಯರಿಗೆ ಕೇಶವನು ಹೇಗೆ ನಿವರ್ತಕರೋ ಹಾಗೆ ಮನುಷ್ಯರ ಸಂಸಾರ ಬಂಧನವನ್ನು ಬಿಡಿಸುವುದು ಮಾಘಮಾಸವು.

\begin{verse}
\textbf{ನ ಸೂರ್ಯಾದಿ ಗ್ರಹಾಶ್ಚೈನಂ ನ ಪಿಶಾಚಾ ನ ಚೋರಗಾಃ~।}\\\textbf{ಮಾಘಸ್ನಾನರತಂ ಜಂತುಂ ನ ಸ್ಪೃಶಂತಿ ಕದಾಚನ~।। ೫೩~।।}
\end{verse}

ಮಾಘಸ್ನಾನದಲ್ಲಿ ನಿರತನಾದವರನ್ನು ಸೂರ್ಯಾದಿಗ್ರಹಗಳಾಗಲಿ, ಪಿಶಾಚಿಗಳಾಗಲಿ, ಸರ್ಪಗಳಾಗಲಿ ಎಂದಿಗೂ ಏನನ್ನೂ ಮಾಡಲಾರವು.

\begin{verse}
\textbf{ಲಿಖಿತಾನಿ ಚ ಪಾಪಾನಿ ಬಹೂನಿ ಯಮಶಾಸನೇ~।}\\\textbf{ಪರಿಮಾರ್ಜಯತಿ ಕ್ಷಿಪ್ರಂ ಚಿತ್ರಗುಪ್ತೋ ಯಮಾಜ್ಞಯಾ~।}\\\textbf{ಇತಿ ಗರ್ಜಂತಿ ಪಾಪಾನಿ ಇತ್ಯಾದೀನಿ ಶತಾನಿ ಚ~।। ೫೪~।।}
\end{verse}

ಯಮ ದೇವರ ಆಜ್ಞೆಯಂತೆ ಶಿಕ್ಷೆಗೆ ಒಳಪಡುವ ಸಮಸ್ತ ಪಾಪಗಳನ್ನೂ ಚಿತ್ರಗುಪ್ತನು ಅಳಿಸಿಬಿಡುವನು ಹೀಗೆಂದು ಪಾಪಗಳು ಸಾರುತ್ತವೆ. (ಮಾಘ ಸ್ನಾನ ಮಾಡುವವರ ಪಾಪಗಳು ನಾಶವಾಗುತ್ತವೆ).

\begin{verse}
\textbf{ಕೃತ್ವೋಷಸಿ ಸ್ನಾನಮನನ್ಯ ಮಾನಸಾ}\\\textbf{ಮಾಘೇ ತು ಸಂಸ್ಥೇ ಮಕರೇ ದಿವಾಕರೇ~।}\\\textbf{ಭಕ್ತ್ಯಾ ಸಮಭ್ಯರ್ಚ್ಯ ತಥೈವ ಮಾಧವಂ} \\\textbf{ಮಾತುಸ್ತನಂ ನೈವ ಪುನಃ ಪಿಬಂತಿ~।। ೫೫~।।}
\end{verse}

ಮಾಘಮಾಸದಲ್ಲಿ ಮಕರರಾಶಿಯಲ್ಲಿ ಸೂರ್ಯನು ಇರುವಾಗ ಭಕ್ತಿಯಿಂದ ಸ್ನಾನಮಾಡಿ ಶ‍್ರೀಹರಿಯನ್ನು ಪೂಜಿಸಿದರೆ ಅಂತಹವನು ಇನ್ನೊಮ್ಮೆ ಮಾತೃ ಸ್ತನವನ್ನು ಪಾನಮಾಡಲಾರನು. (ಪುನರ್ಜನ್ಮವಿಲ್ಲ).

\begin{verse}
\textbf{ಸಂಸಾರಸರ್ಪದಷ್ಟಾನಾಂ ಜಂತೂನಾಮವಿವೇಕಿನಾಮ್~।}\\\textbf{ಮಾಘೇ ಉಷಸಿ ಸ್ನಾನಂ ಚ ಮಕರೇ ಪರಮೌಷಧಮ್~।। ೫೬~।।}
\end{verse}

ಸಂಸಾರವೆಂಬ ಸರ್ಪದಿಂದ ಬಂಧಿಸಲ್ಪಟ್ಟ ಅವಿವೇಕಿಗಳಿಗೆ ಮಕರಯುಕ್ತವಾದ ಮಾಘಮಾಸದಲ್ಲಿ ಪ್ರಾತಃಸ್ನಾನವು ಉತ್ತಮವಾದ ಔಷಧ.

\begin{verse}
\textbf{ಇಹ ಜನ್ಮನಿ ಯತ್ಪಾಪಂ ಯತ್ಪಾಪಂ ಪೂರ್ವಜನ್ಮಸು~।}\\\textbf{ತತ್ಪಾಪಂ ನಾಶಮಾಯಾತಿ ಮಾಘಸ್ನಾನೇನ ನಾರದ~।। ೫೭~।।}
\end{verse}

ನಾರದನೇ, ಈ ಜನ್ಮದ ಹಾಗೂ ಕಳೆದ ಜನ್ಮಗಳ ಪಾಪಗಳೆಲ್ಲವೂ ಮಾಘ ಸ್ನಾನದಿಂದ ನಾಶವಾಗುತ್ತವೆ.

\begin{verse}
\textbf{ಮೇರುಮಂದರತುಲ್ಯಂ ಚ ಯತ್ಪಾಪಂ ಪಾಪಿಭಿಃ ಕೃತಮ್~।}\\\textbf{ತತ್ಪಾಪಂ ನಾಶಮಾಯಾತಿ ಮಾಘಸ್ನಾನೇನ ನಾರದ~।। ೫೮~।।}
\end{verse}

ಪಾಪಿಗಳಿಂದ ಆಚರಿಸಲ್ಪಟ್ಟ ಪಾಪಗಳು ಮೇರು\enginline{-}ಮಂದರ ಪರ್ವತಗಳಷ್ಟು ಪ್ರಬಲವಾಗಿದ್ದರೂ ಮಾಘಸ್ನಾನದಿಂದ ನಾಶ ಹೊಂದುತ್ತವೆ.

\begin{verse}
\textbf{ಮಾಘಸ್ನಾನಂ ಕರಿಷ್ಯಾಮೀತ್ಯೇವಂ ಸಂಕಲ್ಪತೋ ಮುನೇ~।}\\\textbf{ಧ್ರುವಂ ನಶ್ಯಂತಿ ಪಾಪಾನಿ ಬಲವಂತಿ ಪ್ರಯತ್ನತಃ~।। ೫೯~।।}
\end{verse}

“ಮಾಘ ಸ್ನಾನವನ್ನು ಮಾಡುತ್ತೇನೆ” ಹೀಗೆಂದು ಧೈರ್ಯದಿಂದ ಯಾರು ಮನಸ್ಸಿನಲ್ಲಿ ಸಂಕಲ್ಪ ಮಾಡಿಕೊಳ್ಳುವರೋ ಅವರ ಪಾಪಗಳು, ಎಷ್ಟೇ ಬಲಿಷ್ಠವಾಗಿದ್ದರೂ, ನಶಿಸಿಹೋಗುತ್ತವೆ.

\begin{verse}
\textbf{ಬಹಿರ್ಗಮನಮಾತ್ರೇಣ ವಿದ್ರವಂತಿ ದಿಶೋ ದಶ~।}\\\textbf{ಬ್ರಾಹ್ಮಣಃ ಕ್ಷತ್ರಿಯೋ ವಾಪಿ ವೈಶ್ಯಃ ಶೂದ್ರೋsಥವಾ ಸತೀ~।। ೬೦~।। }\\\textbf{ಪ್ರಾತರ್ಮಾಘೇ ಬಹಿಃಸ್ನಾನಂ ಕುರ್ಯಾದಿತಿ ಸ್ಮೃತೀರಿತಮ್~।}\\\textbf{ಬ್ರಹ್ಮಚಾರೀ ಗೃಹಸ್ಥೋ ವಾ ವನಸ್ಥೋ ಯತಿರೇವ ಚ~।। ೬೧~।।} \\\textbf{ಪ್ರಾತಃಪ್ರಾತಃ ಸದಾ ಸ್ನಾಯಾನ್ಮಾಘೇ ಮಾಸ್ಯುದಿತೇ ರವೌ~।}
\end{verse}

ಮಾಘ ಸ್ನಾನಕ್ಕೋಸ್ಕರ ಮನೆಯಿಂದ ಹೊರಕ್ಕೆ ಹೋಗುವ ಮಾತ್ರದಿಂದಲೇ ಪಾಪಗಳು ಹತ್ತು ದಿಕ್ಕುಗಳಲ್ಲಿ ಚದರಿ ಹೋಗುತ್ತವೆ. ಬ್ರಾಹ್ಮಣ, ಕ್ಷತ್ರಿಯ, ವೈಶ್ಯ, ಶೂದ್ರ, ಪತಿವ್ರತಾ ಸ್ತ್ರೀ ಇವರೆಲ್ಲರೂ ಮಾಘ ಮಾಸದಲ್ಲಿ ಪ್ರಾತಃಸ್ನಾನ ಮಾಡಬೇಕೆಂದು ಸ್ಮೃತಿಯಲ್ಲಿ ಹೇಳಿದೆ. ಬ್ರಹ್ಮಚಾರಿ, ಗೃಹಸ್ಥ, ವಾನಪ್ರಸ್ಥ, ಯತಿ\enginline{-}ಈ ಆಶ್ರಮದವರೂ ನಿತ್ಯವೂ ಉಷಃಕಾಲದಲ್ಲಿ ಸ್ನಾನ ಮಾಡಬೇಕು. ಮಾಘದಲ್ಲಂತೂ ಸೂರ್ಯನು ಉದಯವಾಗುವ ಮೊದಲು ಅವಶ್ಯ ಮಾಡಬೇಕು.

\begin{verse}
\textbf{ಅಸ್ನಾತ್ವಾ ಚಾಪ್ಯದತ್ವಾ ಚ ಮಾಘಮಾಸೋ ಗತೋ ಯದಿ~।। ೬೨~।।}\\\textbf{ಶ್ವಾನಯೋನಿಶತಂ ಪ್ರಾಪ್ಯ ವಿಷ್ಠಾಯಾಂ ಜಾಯತೇ ಕೃಮಿಃ~।}\\\textbf{ಪ್ರಾತಃಸ್ನಾನಂ ಪ್ರಕುರ್ವೀತ ಮಾಘೇ ಮಾಸಿ ದ್ವಿಜೋತ್ತಮಾಃ~।।} \\\textbf{ಅಯಂ ಹಿ ಪರಮೋ ಧರ್ಮಃ ಸಂಸಾರೋತ್ತಾರಕಾರಕಮ್~।। ೬೩~।।}
\end{verse}

ಪ್ರಾತಃಸ್ನಾನವನ್ನು ಮಾಡದೆ, ದಾನಾದಿಗಳನ್ನು ಕೊಡದೆ ಮಾಘಮಾಸವನ್ನು ಕಳೆದರೆ ಅಂತಹವನು ನೂರು ಸಲ ನಾಯಿಯಾಗಿ ಹುಟ್ಟಿ ನಂತರ ಮಲದಲ್ಲಿ ಕೃಮಿಯಾಗುತ್ತಾನೆ. ಸೂತರು ಹೇಳುತ್ತಾರೆ:- ಬ್ರಾಹ್ಮಣ ಶ್ರೇಷ್ಠರೇ! ಮಾಘ ಮಾಸದಲ್ಲಿ ಪ್ರಾತಃಸ್ನಾನವನ್ನು ತಪ್ಪದೆ ಮಾಡಬೇಕು. ಸಂಸಾರದಿಂದ ಬಿಡುಗಡೆ ಹೊಂದಲು ಇದು ಶ್ರೇಷ್ಠವಾದ ಧರ್ಮ.

\begin{verse}
\textbf{ನ ಮಾಘಮಾಸಾತ್ ಪರಮೋ ಹಿ ಬಂಧುಃ}\\\textbf{ನ ಮಾಘಮಾಸಾತ್ ಪರಮಃ ಸಖಾಸ್ತಿ~।}\\\textbf{ನ ಮಾಘಮಾಸಾದಪರಂ ಚ ದಾನಂ} \\\textbf{ನ ಮಾಘಮಾಸಾದಪರಃ ಕ್ರತುರ್ವಾ~।। ೬೪~।। }
\end{verse}

ಮಾಘಮಾಸಕ್ಕಿಂತ ಬೇರೆಯಾದ ಬಂಧುವಿಲ್ಲ, ಮಿತ್ರನಿಲ್ಲ, ದಾನವಿಲ್ಲ ಮತ್ತು ಯಜ್ಞವಿಲ್ಲ.

\begin{center}
ಇತಿ ಶ‍್ರೀ ವಾಯುಪುರಾಣೇ ಮಾಘಮಾಸ ಮಾಹಾತ್ಮ್ಯೇ ಪ್ರಥಮೋಧ್ಯಾಯಃ
\end{center}

\begin{center}
ವಾಯುಪುರಾಣದಲ್ಲಿ ಮಾಘಮಾಸ ಮಾಹಾತ್ಮ್ಯೆಯಲ್ಲಿ \\ ಮೊದಲನೆಯ ಅಧ್ಯಾಯವು ಮುಗಿಯಿತು.
\end{center}

\newpage

\section*{ಅಧ್ಯಾಯ\enginline{-}೨}

\emptypage

\begin{flushleft}
\textbf{ನಾರದ ಉವಾಚ\enginline{-}}
\end{flushleft}

\begin{verse}
\textbf{ಸಾಧೂದಿತಂ ದೇವಗುರೋ ಮಹೋದಿತಂ}\\\textbf{ಶ್ರುತಂ ಸಗೋಪ್ಯೋ ಮನಸೋsತಿಹರ್ಷಣಮ್~।}\\\textbf{ಸಗದ್ಗ ದಂ ಮೇ ವಚನಂ ಚ ಜಾತಂ}\\\textbf{ಸಾನಂದಬಾಷ್ಪೇ ನಯನೇ ಪ್ರಹೃಷ್ಟೇ~।। ೧~।।}
\end{verse}

\begin{flushleft}
ನಾರದರು ಹೇಳಿದರು:-
\end{flushleft}

ದೇವತೆಗಳಿಗೆ ಗುರುಗಳಾದ ಬ್ರಹ್ಮದೇವರೇ, ನಿಮ್ಮಿಂದ ನಿರೂಪಿಸಲ್ಪಟ್ಟ ರಹಸ್ಯವಾದ ಸಾಧುವಾದ ಉಪದೇಶವನ್ನು ಕೇಳಿ ನನಗೆ ಬಹಳ ಸಂತೋಷವಾಗಿದೆ; ಮಾತುಗಳು ಗದ್ಗದವಾಗಿವೆ; ಕಣ್ಣುಗಳಲ್ಲಿ ಆನಂದಬಾಷ್ಪಗಳು ಬಂದಿವೆ.

\begin{verse}
\textbf{ಮಾಘಸ್ನಾನಂ ಕಥಂ ಕಾರ್ಯಂ ಮಾಸೇ ಮಾಧವವಲ್ಲಭೇ~।}\\\textbf{ಪುತ್ರಾಯ ವಾಥ ಶಿಷ್ಯಾಯ ವಿಸ್ತರಾತ್ಕ ಮಲಾಸನ~।। ೨~।।}
\end{verse}

ಕಮಲಾಸನನೇ, ಶ‍್ರೀಹರಿಗೆ ಪ್ರಿಯವಾಗಿರುವ ಮಾಘಮಾಸದಲ್ಲಿ ಸ್ನಾನ ಮಾಡುವ ಕ್ರಮ ಹೇಗೆ ಎಂಬ ವಿಷಯವನ್ನು ವಿಸ್ತಾರವಾಗಿ ನಿಮ್ಮ ಮಗನೂ, ಶಿಷ್ಯನೂ ಆದ ನನಗೆ ಹೇಳಿರಿ.

\begin{flushleft}
\textbf{ಬ್ರಹ್ಮೋವಾಚ \enginline{-}}
\end{flushleft}

\begin{verse}
\textbf{ಮಾಘಮಾಸವಿಧಿಂ ವಕ್ಷ್ಯೇ ಶೃಣು ನಾರದ ಸಾದರಮ್~।}\\\textbf{ಯಚ್ಛ್ರು ತ್ವಾ ನಾರದಂ ಭೂಯೋ ನ ಭೂಯೋ ಜಾಯತೇ ಭುವಿ~।। ೨~।।}
\end{verse}

\begin{flushleft}
ಬ್ರಹ್ಮದೇವರು ಹೇಳಿದರು:-
\end{flushleft}

ನಾರದನೇ, ಮಾಘಮಾಸದಲ್ಲಿ ಆಚರಿಸಬೇಕಾದ ನಿಯಮಗಳನ್ನು ಹೇಳುತ್ತೇನೆ. ಆದರದಿಂದ ಕೇಳು. ಆದರದಿಂದ ಕೇಳಿದವನು ಪುನಃ ಭೂಮಿಯಲ್ಲಿ ಹುಟ್ಟುವುದೇ ಇಲ್ಲ.

\begin{verse}
\textbf{ಪುಷ್ಯಮಾಸೇ ಸಿತೇ ಪಕ್ಷೇ ದಶಮ್ಯಾಂ ಆರಭೇತ್ ವ್ರತೀ~।}\\\textbf{ಮಾಘಮಾಸದಶಮ್ಯಾಂ ತು ಸಮಾಸಃ ಶುಭಕರ್ಮಣಿ~।। ೪~।।}
\end{verse}

ಈ ವ್ರತವನ್ನು ಆಚರಿಸುವವನು ಪುಷ್ಯ ಶುದ್ಧ ದಶಮಿಯಲ್ಲಿ ಪ್ರಾರಂಭಿಸಿ ಮಾಘ ಶುದ್ಧ ದಶಮಿಯಲ್ಲಿ ಆ ಶುಭ ಕಾರ್ಯವನ್ನು ಮುಗಿಸಬೇಕು.

\begin{verse}
\textbf{ದಶಮೀ ಚೇದ್ದಶಮ್ಯಾಂ ತು ಸಮಾಸಃ ಶುಭಕರ್ಮಣಿ~।}\\\textbf{ಪೌರ್ಣಿಮಾನಾಂ ಯಾಜ್ಞಿ ಕಾನಾಂ ಪಿತೄಣಾಂ ಸಂಕ್ರಮಾದಿತಃ~।। ೫~।।}
\end{verse}

ದಶಮಿಯಲ್ಲಿ ಪ್ರಾರಂಭ ಮಾಡಿದರೆ ಈ ಶುಭ ಕರ್ಮವನ್ನು ಮುಂದಿನ ದಶಮಿಯಲ್ಲಿ ಸಮಾಪ್ತಿ ಮಾಡಬೇಕು. ಯಜ್ಞವನ್ನಾಚರಿಸುವವರು ಪೌರ್ಣಿಮೆಯಿಂದ ಪೌರ್ಣಿಮೆವರೆವಿಗೂ, ಪಿತೃ ತೃಪ್ತಿಗೋಸ್ಕರ ಆಚರಿಸುವವರು ಮಕರ ಸಂಕ್ರಮಣದಲ್ಲಿ ಪ್ರಾರಂಭಿಸಿ ಕುಂಭ ಸಂಕ್ರಮಣದವರೆಗೂ ನಡೆಸಬೇಕು.

\begin{verse}
\textbf{ಏತೇಷ್ವನ್ಯತಮಂ ಪಕ್ಷಂ ಸಮಾರಭ್ಯ ವ್ರತಂ ಚರೇತ್~।}\\\textbf{ಮಾಘಸ್ನಾನಂ ಕರಿಷ್ಯಾಮಿ ಮಕರಸ್ಥೇ ದಿವಾಕರೇ~।। ೬~।।}\\\textbf{ಆಸಮಾಪ್ತಿಂ ಮಹಾದೇವ ನಿರ್ವಿಘ್ನಂ ಕುರು ಮಾಧವ~।}\\\textbf{ಇತಿ ಸಂಕಲ್ಪ್ಯ ಪೂರ್ವದ್ಯುಃ ಕುರ್ಯಾತ್ ಸ್ನಾನಮನಂತರಮ್~।। ೭~।।}
\end{verse}

ಮೇಲೆ ಹೇಳಿದ ಕ್ರಮದಲ್ಲಿ ಯಾವುದಾದರೊಂದು ಕ್ರಮವನ್ನು ಅನುಸರಿಸಿ ಹಿಂದಿನ ದಿನದಲ್ಲಿ ಈ ರೀತಿ ಸಂಕಲ್ಪ ಮಾಡಬೇಕು: "ಜಗದೊಡೆಯನೇ, ಸೂರ್ಯನು ಮಕರದಲ್ಲಿರುವಾಗ ನಾನು ಮಾಘ ಸ್ನಾನವನ್ನು ಮಾಡುವೆನು; ಈ ಕಾರ್ಯವು ನಿರ್ವಿಘ್ನವಾಗಿ ನಡೆಯುವಂತೆ ಅನುಗ್ರಹಿಸು.

\begin{verse}
\textbf{ಅರುಣೋದಯವೇಲಾಯಾಂ ಸಮುತ್ಥಾಯ ದ್ವಿಜೋತ್ತಮ~।}\\\textbf{ಗತ್ವಾ ಬಹಿಃಸ್ಥಲೇ ದೂರೇ ಮಲಮೂತ್ರೇ ವಿಸರ್ಜಯೇತ್~।। ೮~।।}
\end{verse}

ಅರುಣೋದಯ ಕಾಲದಲ್ಲಿ ಎದ್ದು ಮನೆಯಿಂದ ಹೊರಗೆ ದೂರದಲ್ಲಿ ಹೋಗಿ ಮಲಮೂತ್ರಗಳನ್ನು ವಿಸರ್ಜಿಸಬೇಕು.

\begin{verse}
\textbf{ನ ಗೋಷ್ಠೇ ನ ಶುಭೇ ದೇಶೇ ನ ವನೇ ಕೇವಲಸ್ಥಲೇ~।}\\\textbf{ನ ವಲ್ಲೀಕೇ ಚೋಪವನೇ ನ ಛಾಯಾಯಾಂ ನ ಪದ್ಧ ತೌ~।। ೯~।।}
\end{verse}

ದನದ ಕೊಟ್ಟಿಗೆಯಲ್ಲಿ, ಪವಿತ್ರವಾದ ಪ್ರದೇಶದಲ್ಲಿ, ಜಲಾಶಯಗಳಲ್ಲಿ, ಹೂ ಗಿಡಗಳ ತೋಟಗಳಲ್ಲಿ, ಮರಗಳ ನೆರಳಿರುವ ಸ್ಥಳಗಳಲ್ಲಿ ಮತ್ತು ಓಡಾಡುವ ದಾರಿಗಳಲ್ಲಿ ವಿಸರ್ಜನೆ ಮಾಡಬಾರದು.

\begin{verse}
\textbf{ನದೀತೀರೇ ನ ಕರ್ತವ್ಯಂ ಮಲಮೂತ್ರವಿಸರ್ಜನಮ್~।}\\\textbf{ಕುರ್ಯಾದ್ಯದಿ ವಿಮೂಢಾತ್ಮಾ ರೌರವಂ ಕಲ್ಪಮಶ್ನುತೇ~।। ೧೦~।।}
\end{verse}

ನದೀ ತೀರದಲ್ಲಿ ಮಲಮೂತ್ರ ವಿಸರ್ಜನೆ ಮಾಡಬಾರದು. ಹಾಗೆ ಮಾಡುವ ಮೂರ್ಖನು ಒಂದು ಕಲ್ಪದವರೆಗೂ ರೌರವ ನರಕದಲ್ಲಿ ಬೀಳುತ್ತಾನೆ.

\begin{verse}
\textbf{ತಸ್ಮಾತ್ ದೂರೇಷು ಕರ್ತವ್ಯಂ ಮಲಮೂತ್ರವಿಸರ್ಜನಮ್~।}\\\textbf{ತಚಃ ಶೌಚಂ ತು ಕರ್ತವ್ಯಂ ಪಾತ್ರಸ್ಥೇನೈವ ವಾರಿಣಾ~।। ೧೧~।।}
\end{verse}

ದೂರ ಪ್ರದೇಶಗಳಲ್ಲಿ ಮಾಡಿ ಪಾತ್ರೆಯೊಳಗೆ ನೀರನ್ನು ತೆಗೆದುಕೊಂಡು ಹೋಗಿದ್ದು ಆ ನೀರಿನಿಂದಲೇ ಶುದ್ಧಿ ಮಾಡಿಕೊಳ್ಳಬೇಕು.

\begin{verse}
\textbf{ನದೀಜಲೇನ ಯಃ ಕುರ್ಯಾತ್ ಶೌಚಕರ್ಮ ನರಾಧಮಃ~।}\\\textbf{ವಿಷ್ಠಾ ಕೂಪೇ ಪತತ್ಯಾಶು ಸ್ವರ್ಗಸ್ಥೈಃ ಪಿತೃಭಿಃ ಸಹ~।। ೧೨~।।}
\end{verse}

ನದಿಯಲ್ಲಿ ಶೌಚವನ್ನು ಮಾಡಿಕೊಳ್ಳುವ ನೀಚನು ಸ್ವರ್ಗದಲ್ಲಿರುವ ತನ್ನ ಪಿತೃಗಳ ಸಹಿತನಾಗಿ ಮಲದ ಕೂಪದಲ್ಲಿ ಬೀಳುವನು.

\begin{verse}
\textbf{ಕೂಪಕುಲ್ಯ ತಟಾಕಾದಿಜಲೈಃ ಶೌಚಂ ಕರೋತಿ ಯಃ~।}\\\textbf{ಕಲ್ಪ ಕೋಟಿಶತೈರ್ವಾಪಿ ನರಕಾನ್ನ ನಿವರ್ತತೇ~।। ೧೩~।।}
\end{verse}

ನದಿ, ಕಾಲುವೆ, ಕೆರೆ ಮುಂತಾದ ಜಲಾಶಯಗಳಲ್ಲಿಯೇ ಶೌಚಕರ್ಮವನ್ನು ಮಾಡುವವನು ಒಂದು ನೂರು ಕೋಟಿ ವರ್ಷಗಳಾದರೂ ನರಕದಿಂದ ಹಿಂದಿರುಗುವುದಿಲ್ಲ.

\begin{verse}
\textbf{ತಸ್ಮಾತ್ ಕಮಂಡಲುಜಲಂ ನ ತ್ಯಜೇತ್ತು ಕದಾಚನ~।। ೧೪~।।}
\end{verse}

ಆದುದರಿಂದ ಪಾತ್ರೆಯೊಳಗಿನ ನೀರಿನಿಂದಲೇ ಶೌಚಕರ್ಮವನ್ನು ಆಚರಿಸುವುದನ್ನು ಎಂದಿಗೂ ಬಿಡಬಾರದು.

\begin{verse}
\textbf{ಮೃದಾ ಶೌಚಂ ತು ಕರ್ತವ್ಯಂ ಯಥಾ ಗಂಧಕ್ಷಯೋ ಭವೇತ್~।}\\\textbf{ಮೂತ್ರೋತ್ಸರ್ಗವಿಧಿಸ್ತ್ವೇಷೋ ಗೃಹಸ್ಥಾನಾಂ ವಿಧಿಃ ಸ್ಮೃತಃ~।। ೧೫~।।}
\end{verse}

ದುರ್ಗಂಧವು ಪೂರ್ತಿಯಾಗಿ ಹೋಗುವವರೆಗೂ ಮೃತ್ತಿಕಾದಿಂದ ತೊಳೆದು ಕೊಳ್ಳಬೇಕು. ಗೃಹಸ್ಥರು ಈ ರೀತಿ ಮಾಡಬೇಕು.

\begin{verse}
\textbf{ಏತಸ್ಮಾತ್ ದ್ವಿಗುಣಂ ಪ್ರೋಕ್ತಂ ವ್ರತಿನಾಂ ವನವಾಸಿನಾಂ~।}\\\textbf{ಪೂರ್ವಸ್ಮಾತ್ ದ್ವಿಗುಣಂ ಪ್ರೋಕ್ತಂ ಯತೀನಾಂ ತು ಚತುರ್ಗುಣಮ್~।।}
\end{verse}

ಗೃಹಸ್ಥರ ಎರಡರಷ್ಟು ಬ್ರಹ್ಮಚಾರಿಗಳಿಗೂ, ಅವರ ಎರಡರಷ್ಟು ವಾನಪ್ರಸ್ಥರಿಗೂ, ಅವರ ನಾಲ್ಕರಷ್ಟು ಯತಿಗಳಿಗೂ ಹೇಳಲಾಗಿದೆ.

\begin{verse}
\textbf{ಗಂಧಕ್ಷಯಕರಃ ಶೌಚಃ ಸ್ತ್ರೀಶೂದ್ರಾಣಾಂ ವಿಧಿಃ ಸ್ಮೃತಃ~।}\\\textbf{ವಲ್ಮೀಕಂ ಮೂಷಕೋತ್ಖಾತಂ ಸೇತೌ ಮಾರ್ಗೇ ವ್ರಜೇ ಗೃಹೇ~।। ೧೭~।।}
\end{verse}

ಸ್ತ್ರೀ ಶೂದ್ರರು ಸಹ ದುರ್ಗಂಧನಾಶಕ್ಕೋಸ್ಕರ ಇದೇ ರೀತಿ ಶೌಚವನ್ನು ಆಚರಿಸಬೇಕು. ಹುತ್ತಕ್ಕೆ ಸಂಬಂಧಿಸಿದ, ಇಲಿಯ ಬಿಲಗಳಿಗೆ ಸೇರಿದ, ಸೇತುವೆಗೆ ಸೇರಿದ, ದಾರಿಯಲ್ಲಿ ಬಿದ್ದಿರುವ, ದನದ ಕೊಟ್ಟಿಗೆಯಲ್ಲಿನ ಮಣ್ಣನ್ನು ಶೌಚಕ್ಕೆ ಉಪಯೋಗಿಸಬಾರದು.

\begin{verse}
\textbf{ಅನ್ಯ ಶಿಷ್ಟಾವಶಿಷ್ಟಾಂ ಚ ವರ್ಜಯೇನ್ಮೃತ್ತಿಕಾಂ ಬುಧಃ~।}\\\textbf{ಏಕಾ ಲಿಂಗೇ ಕರೇ ತಿಸ್ರ ಉಭಯೋಃ ಮೃದ್ವಯಂ ಸ್ಮೃತಮ್~।। ೧೮~।।}
\end{verse}

ಇತರರು ಉಪಯೋಗಿಸಿ ಮಿಕ್ಕಿರುವ ಮೃತ್ತಿಕೆಯನ್ನು ಶೌಚಕ್ಕೆ ತೆಗೆದು ಕೊಳ್ಳಬಾರದು. ಲಿಂಗಕ್ಕೆ ಒಂದು ಸಲ, ಎಡಗೈಗೆ ಮೂರು ಸಲ, ಎರಡೂ ಕೈ ಗಳಿಗೆ ಮತ್ತೆ ಎರಡು ಸಲ ಮೃತ್ತಿಕಾ\-ಲೇಪನದಿಂದ ಶುದ್ಧ ಮಾಡಿಕೊಳ್ಳಬೇಕು.

\begin{verse}
\textbf{ಪಂಚಾಪಾನೇ ದಶೈಕಸ್ಮಿನ್ ಉಭಯೋಃ ಸಪ್ತ ಮೃತ್ತಿಕಾಃ~।}\\\textbf{ಲಿಂಗೇ ತಿಸ್ರೋ ಗುದೇ ಪಂಚ ಪ್ರತ್ಯೇಕಂ ಕರಯೋಸ್ತ್ರಯಃ~।। ೧೯~।।}
\end{verse}

ಮಲವಿಸರ್ಜನೆ ಮಾಡಿದಾಗ ಗುದದ್ವಾರಕ್ಕೆ ಐದು ಅಥವಾ ಹತ್ತು ಸಲ, ಕೈಗಳಿಗೆ ಏಳು ಸಲ, ಲಿಂಗಕ್ಕೆ ಮೂರು ಸಲ.

\begin{verse}
\textbf{ಉಭಯೋಃ ಪಾದಯೋಸ್ತಿಸ್ರಃ ಪುರೀಷೇ ಮೃತ್ತಿಕಾವಿಧಿಃ~।}\\\textbf{ಲಿಂಗೇ ಚೈಕಾ ಕರೇ ಚೈಕಾ ತಿಸ್ರಶ್ಚ ಕರಯೊರ್ಭವೇತ್~।। ೨೦~।।}
\end{verse}

ಎರಡೂ ಪಾದಗಳಿಗೆ ಮೂರು ಸಲ ಅಥವಾ ಲಿಂಗಕ್ಕೆ ಒಂದು ಸಲ, ಎಡಗೈಗೆ ಒಂದು ಸಲ, ಎರಡೂ ಕೈಗಳಿಗೆ ಪುನಃ ಮೂರು ಸಲ ಮೃತ್ತಿಕಾಶೌಚ ಮಾಡಿಕೊಳ್ಳ ಬೇಕು.

\begin{verse}
\textbf{ಪಾದೌ ಪ್ರಕ್ಷಾಲ್ಯ ವಿಪ್ರೇಂದ್ರ ದಂತಧಾವನಮಾಚರೇತ್~।}\\\textbf{ಪ್ರತಿಪತ್ ಪರ್ವಷಷ್ಠೀಷು ನವಮ್ಯಾಂ ದ್ವಾದಶೀ ದಿನೇ~।}\\\textbf{ದಂಡಾನಾಂ ಕಾಷ್ಠ ಸಂಯೋಗೇ ದಹತ್ಯಾ ಸಪ್ತಮಂ ಕುಲಮ್~।। ೨೨~।।}
\end{verse}

ಪಾಡ್ಯ, ಹುಣ್ಣಿಮೆ, ಅಮಾವಾಸ್ಯೆ, ಷಷ್ಠೀ, ನವಮೀ, ದ್ವಾದಶೀ ದಿನಗಳಲ್ಲಿ ಕಾಷ್ಠದಿಂದ ಹಲ್ಲನ್ನು ಉಜ್ಜಿ ಕೊಂಡರೆ ಏಳು ಕುಲಗಳು ಭಸ್ಮವಾಗುತ್ತವೆ.

\begin{verse}
\textbf{ತೃಣಪರ್ಣೈಃ ಸದಾ ಕುರ್ಯಾತ್ ಅಮಾಮೇಕದಶೀಂ ವಿನಾ~।। ೨೩~।।}
\end{verse}

ಅಮಾವಾಸ್ಯೆ, ಏಕಾದಶೀ ಬಿಟ್ಟು ಉಳಿದ ದಿನಗಳಲ್ಲಿ ಹುಲ್ಲು, ಬೇವಿನ ಅಥವಾ ಮಾವಿನ ಕಾಷ್ಠ ಇಲ್ಲವೆ ಎಲೆಗಳಿಂದ ಹಲ್ಲುಗಳನ್ನು ಶುದ್ಧ ಮಾಡಿಕೊಳ್ಳಬೇಕು.

\begin{verse}
\textbf{ಗಂಡೂಷಮಥವಾ ಕುರ್ಯಾತ್ ದಶವಾರಂ ದ್ವಿಜೋತ್ತಮ~।}\\\textbf{ನಾರಾಯಣಂ ದೇವಗುರುಂ ಮಾನಸೇ ತು ಸಮಾವಿಶೇತ್~।। ೨೪~।।}
\end{verse}

ಅಥವಾ ಹತ್ತು ಸಲ ಬಾಯಿಮುಕ್ಕಳಿಸಿ ಸರ್ವೋತ್ತಮನಾದ ನಾರಾಯಣನನ್ನು ಮನಸ್ಸಿನಲ್ಲಿ ಸ್ಮರಿಸಬೇಕು.

\begin{verse}
\textbf{ಸಮಸ್ತ ಜಗದಾಧಾರ ಶಂಖಚಕ್ರಗದಾಧರ~।}\\\textbf{ದೇವ ದೇಹಿ ಮಮಾನುಜ್ಞಾಂ ಯುಷ್ಮತ್ತೀರ್ಥನಿಷೇವಣೇ~।। ೨೫~।।}
\end{verse}

"ಚತುರ್ದಶಭುವನಕ್ಕೂ ಆಧಾರನಾದ ಶಂಖಚಕ್ರಗದಾದಿಗಳನ್ನು ಧರಿಸಿರುವ ದೇವನೇ, ನಿನ್ನ ಪಾದದಿಂದ ಉದ್ಭವವಾದ ತೀರ್ಥದಲ್ಲಿ ಸ್ನಾನಮಾಡಲು ಅಪ್ಪಣೆಯನ್ನು ದಯಪಾಲಿಸು” ಹೀಗೆಂದು ಪ್ರಾರ್ಥಿಸಬೇಕು.

\begin{verse}
\textbf{ಯನ್ನಖಾಗ್ರಾತ್ ಸಮಾಭೂತಾ ಗಂಗಾ ತ್ರೈಲೋಕ್ಯಪಾವನೀ~।।}\\\textbf{ಯತ್ಪಾದೇ ತೀರ್ಥಜಾತಾನಿ ಮಾಧವಂ ತಂ ನಮಾಮ್ಯಹಮ್~।। ೨೬~।।}
\end{verse}

“ಯಾರ ಕಾಲುಗುರಿನ ತುದಿಯ ಘಾತದಿಂದ ಮೂರು ಲೋಕವನ್ನೂ ಪವಿತ್ರ ಗೊಳಿಸುವ ಗಂಗೆಯು ಹುಟ್ಟಿರುವಳೋ, ಯಾರ ಪಾದವನ್ನು ಆಶ್ರಯಿಸಿಕೊಂಡು ಸಮಸ್ತ ಪವಿತ್ರ ತೀರ್ಥಗಳಿರುತ್ತವೆಯೋ ಅಂತಹ ಮಾಧವನನ್ನು ನಮಸ್ಕರಿಸುತ್ತೇನೆ” ಹೀಗೆ ಸ್ಮರಿಸಬೇಕು.

\begin{verse}
\textbf{ಯೋsಸೌ ತೀರ್ಥಪದೇ ವಿಷ್ಣುಃ ಚಿದ್ರೂಪೀ ಪರಮೇಶ್ವರಃ~।}\\\textbf{ಸ ಏವ ದ್ರವರೂಪೇಣ ವಸತ್ಯೇವ ಜಲಾಶಯೇ~।। ೨೭~।।}
\end{verse}

ಸಕಲ ತೀರ್ಥಗಳೂ ಯಾರ ಪಾದದಿಂದ ಆಶ್ರಯ ಪಡೆದಿವೆಯೋ ಅಂತಹ ಜ್ಞಾನರೂಪನಾದ ಸರ್ವೋತ್ತಮನಾದ ವಿಷ್ಣುವೇ ಜಲಾಶಯಗಳಲ್ಲಿ ನೀರಿನ ರೂಪದಿಂದ ಇರುತ್ತಾನೆ. (ವಿಷ್ಣುವೇ ಜಡವಾದ ಜಲವಲ್ಲ; ಜಲದಲ್ಲಿ ಅವನ ಸನ್ನಿಧಾನ ಇದೆ ಎಂದರ್ಥ).

\begin{verse}
\textbf{ಗಂಗಾದ್ಯಾಃ ಸರಿತಃ ಸರ್ವಾ ಯಸ್ಮಿನ್ ಕಸ್ಮಿನ್ ಜಲಾಶಯೇ~।}\\\textbf{ಸಂಸ್ಮರೇತ್ ಸ್ನಾನಕಾಲೇಷು ಸದ್ಯಃ ಸಂನಿಹಿತಾ ಸದಾ~।। ೨೮~।।}
\end{verse}

ಯಾವ ಜಲಾಶಯದಲ್ಲಿ ಸ್ನಾನಮಾಡಿದರೂ ಮಾಡುವ ಕಾಲದಲ್ಲಿ ಗಂಗಾದಿ ನದಿಗಳ ಸ್ಮರಣೆಯನ್ನು ಮಾಡಿ ಆ ತೀರ್ಥಾಭಿಮಾನಿ ದೇವತೆಗಳು ಸನ್ನಿಹಿತರಾಗಿರುವುವೆಂದು ತಿಳಿಯಬೇಕು:

\begin{verse}
\textbf{ನಂದಿನೀ ನಲಿನೀ ಸೀತಾ ಮಾಲತೀ ಚ ಮಲಾಪಹಾ~।}\\\textbf{ವಿಷ್ಣು ಪಾದಾಬ್ಜ ಸಂಭೂತಾ ಗಂಗಾ ತ್ರಿಪಥಗಾಮಿನೀ~।। ೨೯~।।}\\\textbf{`ಭಾಗೀರಥೀ ಭೋಗವತೀ ಜಾಹ್ನವೀ ತ್ರಿದಶೇಶ್ವರೀ~।}\\\textbf{ದ್ವಾದಶೈತಾನಿ ನಾಮಾನಿ ಯತ್ರ ಯತ್ರ ಜಲಾಶಯೇ~।। ೩೦~।।}\\\textbf{ಸ್ನಾನಕಾಲೇ ಪಠೇನ್ನಿತ್ಯಂ ತತ್ರ ಸನ್ನಿಹಿತಾ ತು ಸಾ~।}\\\textbf{ಯಾಃ ಪ್ರವತೇತಿ ಮಂತ್ರೇಣ ಇಮಂ ಮ ಇತಿ ಮಂತ್ರತಃ~।। ೩೧~।।}
\end{verse}

ನಂದಿನೀ, ನಲಿನೀ, ಸೀತಾ, ಮಾಲತೀ, ಮಲಾಪಹಾ, ವಿಷ್ಣು ಪಾದಾಬ್ಜಸಂಭೂತಾ, ಗಂಗಾ, ತ್ರಿಪಥಗಾಮಿನೀ, ಭಾಗೀರಥೀ, ಭೋಗವತೀ, ಜಾಹ್ನವೀ, ತ್ರಿದಶೇಶ್ವರೀ\enginline{-} ಈ ಹನ್ನೆರಡು ನಾಮಗಳಿಂದ ಸ್ನಾನಕಾಲದಲ್ಲಿ ಸ್ಮರಿಸಿದರೆ ಆ ಜಲಾಶಯದಲ್ಲಿ ಗಂಗೆಯ ಸನ್ನಿಧಾನ ಬರುತ್ತದೆ. ನಂತರ “ ಯಾಃ ಪ್ರವತ, ಇಮಂ ಮೇ ” ಈ ಮಂತ್ರಗಳಿಂದ ಅರ್ಘ್ಯ ಕೊಡಬೇಕು. (ಯಾಃ ಪ್ರವತೋ ನಿವತ ಉದ್ವತ ಉದನ್ವ ತಿರನುದಕಾಶ್ಚ ಯಾಃ ತಾ ಅಸ್ಮಭ್ಯಂ ಪಯಸಾಪಿನ್ವಮಾನಾಃ ಶಿವಾದೇವೀರಶಿಪದಾಭವಂತು ಸರ್ವಾ ನದ್ಯೋ ಅಶಿಮಿದಾಭವಂತು~।। ಇಮಂ ಮೇ ಗಂಗೇ ಯಮುನೇ ಸರಸ್ವತೀ ಶುತುದ್ರಿಸ್ತೋಮಂ ಸಚತಾಪರುಷ್ಣ್ಯಾ। ಅಸಿಕ್ನ್ಯಾ - ಮರುದ್ವೃಧೇವಿತಸ್ತಯಾರ್ಜೀಕೇಯೇ ಶೃಣುಹ್ಯಾಸುಷೋ ಮಯ~।।).

\begin{verse}
\textbf{ನದೀಸ್ತಾಃ ಪ್ರಾವಹತ್ವೈವ ದೇವಾಯಾರ್ಘಂ ನಿವೇದಯೇತ್~।}\\\textbf{ನಮಸ್ತೇಸ್ತು ಹೃಷೀಕೇಶ ಗೃಹಾಣಾರ್ಘ್ಯಂ ನಮೋಸ್ತು ತೇ~।। ೩೨~।।}\\\textbf{ನಮಸ್ತೇ ಕಂಜನಾಭಾಯ ನಮಸ್ತೇ ಜಲಶಾಯಿನೇ~।}\\\textbf{ಮಾಘೇ ಮಾಸಿ ಮಹಾದೇವ ವ್ರತಸ್ಥೋಽಹಂ ಜಗತ್ಪತೇ~।। ೩೩~।।}\\\textbf{ಗೃಹಾಣಾರ್ಘ್ಯಂ ಮಯಾ ದತ್ತಂ ಮಾಧವಾಯ ಪ್ರಸೀದ ಮೇ~।}\\\textbf{ವಿಷ್ಣುಪಾದಾಬ್ಜ ಸಂಭೂತೇ ಗಂಗೇ ತ್ರಿಪಥಗಾಮಿನಿ~।। ೩೪~।।}\\\textbf{ಗೃಹಾಣಾರ್ಘ್ಯಂ ಮಯಾ ದತ್ತಂ ಜಲೇ ಸನ್ನಿಹಿತಾ ಭವ~।}\\\textbf{ವೃದ್ದಗಂಗೇ ಮಹಾಪುಣ್ಯೇ ಗೌತಮಸ್ಯಾಘನಾಶಿನಿ~।}\\\textbf{ಗೋದಾವರಿ ಗೃಹಾಣಾರ್ಘ್ಯಂ ತ್ರ್ಯಂಬಕಸ್ಯ ಜಟೋದ್ಭವೇ~।। ೩೫~।।}\\\textbf{ಕಲಿಂಗದೇಶತನಯೇ ಪೃಥಕ್ ಸಾಗರಗಾಮಿನಿ~।}\\\textbf{ಗೃಹಾಣಾರ್ಘ್ಯಂ ಮಯಾ ದತ್ತಂ ಯಮುನೇ ಮೇ ಫಲಪ್ರದೇ~।। ೩೬~।।}
\end{verse}

ಈ ಮಂತ್ರಗಳಿಂದ ಪರಮಾತ್ಮ ನಿಗೂ, ನದ್ಯಭಿಮಾನಿ ದೇವತೆಗಳಿಗೂ (ಗಂಗಾ, ಯಮುನಾ) ಅರ್ಘ್ಯಗಳನ್ನು ಕೊಡಬೇಕು.

\begin{verse}
\textbf{ಅನ್ಯಾ ನದೀಃ ಸಮುದ್ದಿಶ್ಯ ದದ್ಯಾದರ್ಘ್ಯಂ ಯಥಾವಿಧಿ~।}\\\textbf{ಪೃಥಕ್ ಸಾಗರಗಾಮಿನ್ಯಾಂ ನದ್ಯಾಂ ಸ್ನಾನಪರೋ ನರಃ~।।}\\\textbf{ವಿನಾ ಗೋದಾಂ ವಿನಾ ಗಂಗಾಂ ನದೀಮನ್ಯಾಂ ನ ಕೀರ್ತಯೇತ್~।। ೩೭~।।}
\end{verse}

ಬೇರೆ ನದಿಗಳಲ್ಲಿ ಸ್ನಾನಮಾಡುವಾಗಲೂ ಹೀಗೆಯೇ ಅರ್ಘ್ಯಕೊಡಬೇಕು. ಸಮುದ್ರದಲ್ಲಿ ಸೇರುವ ಇತರ ನದಿಗಳಲ್ಲಿ ಸ್ನಾನಮಾಡುವವನು ಗಂಗಾ, ಗೋದಾವರೀ ನದಿಗಳನ್ನು ಹೊರತು ಇತರ ನದಿಗಳ ಸ್ಮರಣೆ ಮಾಡಬಾರದು.

\begin{verse}
\textbf{ತ್ರಿರ್ನಿರ್ಮೃಜ್ಯ ಚ ಪಶ್ಚಾತ್ತು ಮೃತ್ತಿಕಾಸ್ನಾನಮಾಚರೇತ್~।। ೩೮~।।} 
\end{verse}

ನೀರಿನಲ್ಲಿ ಮೂರು ಸಲ ಮುಳುಗಿ ನಂತರ ಮೃತ್ತಿಕಾಲೇಪನ ಮಾಡಿಕೊಂಡು ಸ್ನಾನಮಾಡಬೇಕು.

\begin{verse}
\textbf{ಯಥಾ ಮಲಕಪಿಷ್ಟೇನ ಗೋಮಯೇನ ಸಮಾಚರೇತ್~।}\\\textbf{ಪಿತೄನ್ ಗುರೂನ್ ಸಮುದ್ದಿಶ್ಯ ಪ್ರತ್ಯೇಕಂ ಸ್ನಾನಮಾಚರೇತ್~।। ೩೯~।।}
\end{verse}

ನೆಲ್ಲಿಕಾಯಿಯನ್ನು ಹಚ್ಚಿಕೊಂಡು, ಗೋಮಯವನ್ನು ಲೇಪಿಸಿಕೊಂಡು ಪ್ರತ್ಯೇಕವಾಗಿ ಸ್ನಾನಮಾಡಬೇಕು. ನಂತರ ಪಿತೃಗಳನ್ನೂ, ಗುರುಗಳನ್ನೂ ಉದ್ದೇಶಿಸಿ ಸ್ನಾನಮಾಡಬೇಕು,

\begin{verse}
\textbf{ತತೋಽಘಮರ್ಷಣಂ ಸ್ನಾನಂ ಕುರ್ಯಾನ್ಮಂತ್ರಾದತಂದ್ರಿತಃ~।}\\\textbf{ಶಾಲಗ್ರಾಮಶಿಲಾವಾರಿ ಶಿರಸಾ ಧಾರಯೇತ್ತತಃ~।। ೪೦~।।}
\end{verse}

ಮಂತ್ರಪಠನದಿಂದ ಅಘಮರ್ಷಣಸ್ನಾನವನ್ನು ಮಾಡಿ, ಆಲಸ್ಯರಹಿತನಾಗಿ ನಂತರ ಶಾಲಗ್ರಾಮದ ಮೇಲಿನ ನೀರನ್ನು ಶಿರಸ್ಸಿನಲ್ಲಿ ಹಾಕಿಕೊಳ್ಳಬೇಕು.

\begin{verse}
\textbf{ಜಲಾತ್ ಪಶ್ಚಾತ್ ವಿನಿಸ್ತೀರ್ಯ ತರ್ಪಯೇತ್ ದೇವತಾಋಷೀನ್~।}\\\textbf{ಪಿತೄನ್ ತತಃ ಶಿಖಾಸ್ಕಂಧೇ ದಕ್ಷಿಣೇ ತು ನ್ಯಸೇತ್ತತಃ~।। ೪೧~।। }
\end{verse}

ನೀರಿನಿಂದ ಹೊರಗೆ ಬಂದು ದೇವತೆಗಳು, ಋಷಿಗಳು, ಪಿತೃಗಳು ಇವರಿಗೆ ತರ್ಪಣವನ್ನು ಕೊಟ್ಟು ತನ್ನ ಶಿಖೆಯನ್ನು ಬಲ ಹೆಗಲಿನ ಮೇಲೆ ಹಾಕಿಕೊಳ್ಳಬೇಕು.

\begin{verse}
\textbf{ಶಿಖೋದಕಂ ಭೂಪತಿತಂ ಪಿಬಂತಿ ಪಿತರೋಽಖಿಲಾಃ~।। ೪೨~।।}
\end{verse}

ಶಿಖೆಯಿಂದ ನೆಲದಮೇಲೆ ಬೀಳುವ ನೀರನ್ನು ಸಮಸ್ತ ಪಿತೃಗಳೂ ಕುಡಿಯುತ್ತಾರೆ.

\begin{verse}
\textbf{ಪುರಾ ಗರುಡಭೀತಸ್ತು ಸುಧಾಂ ಶಕ್ರಃ ಶಿಖಾಸು ಸಃ~।। ೪೩~।।}
\end{verse}

ಪೂರ್ವದಲ್ಲಿ ಇಂದ್ರನು ಕಾರಣಾಂತರದಿಂದ ಗರುಡನಿಗೆ ಹೆದರಿ ಅಮೃತವನ್ನು ಬ್ರಾಹ್ಮಣರ ಶಿಖೆಯಲ್ಲಿ ಇಟ್ಟನು.

\begin{verse}
\textbf{ಚಿಕ್ಷೇಪಾಂತರ್ವಿಲೀನಾಂ ತಾಂ ಜ್ಞಾತ್ವಾ ವೀಂದ್ರಃ ಶಶಾಪ ಹ~।}\\\textbf{ಬ್ರಾಹ್ಮಣಾನಾಂ ಶಿಖಾಂ ಕ್ಷಿಪ್ತ್ವಾ ಸುಧಾ ಶಕ್ರಾಯ ನೋ ಭವೇತ್~।।೪೪~।।}
\end{verse}

\begin{verse}
\textbf{ಬ್ರಾಹ್ಮಣಾನಾಂ ಶಿಖೋತ್ಸೃಷ್ಟ ಜಲದ್ವಾರೇಣ ಸಾ ಸುಧಾ~।}\\\textbf{ಭೂಮೌ ಪತಿತಮಶ್ನಾತು ಪಿತರೋ ಮತ್ಕೃಪಾಲವಃ~।। ೪೫~।।}
\end{verse}

ಆ ಅಮೃತವು ಬ್ರಾಹ್ಮಣರ ಶಿಖೆಯಲ್ಲಿಯೇ ಲೀನವಾಯಿತು. ಅದು ತನಗೆ ದೊರೆಯಲಿಲ್ಲವೆಂಬ ಕಾರಣದಿಂದ ಗರುಡನು ಶಾಪ ಕೊಟ್ಟನು. ಏನೆಂದರೆ, ಬ್ರಾಹ್ಮಣರ ಶಿಖೆಯಲ್ಲಿನ ಸುಧೆಯು ಇಂದ್ರನಿಗೆ ಸಿಗದೇಹೋಗಲಿ; ಶಿಖೆಯಿಂದ ನೆಲದಮೇಲೆ ಬಿದ್ದ ನೀರಿನಲ್ಲಿ ಆ ಅಮೃತವು ಇದ್ದು ಅದು ಪಿತೃಗಳಿಗೆ ಸಲ್ಲಲಿ.

\begin{verse}
\textbf{ದ್ವಿಜಾನಾಕ್ಷಿಪ್ಯ ಪಕ್ಷೀಂದ್ರೋ ನ ಜಗ್ರಾಹ ಶಿಖಾಸುಧಾಮ್~।}\\\textbf{ದ್ವಿಜಾನ್ ಪಪ್ರಚ್ಛ ಗರುಡಃ ಕಿಂ ಶಕ್ತೇನಾಕ್ಷಿಪತ್ ಸುಧಾಮ್~।। ೪೬~।।}
\end{verse}

ಗರುಡನು ಬ್ರಾಹ್ಮಣರನ್ನು ನಿಂದಿಸಿದನು; ಆದರೆ ಅವರ ಶಿಖೆಯಲ್ಲಿನ ಸುಧೆಯನ್ನು ಸ್ವೀಕರಿಸಲಿಲ್ಲ. ಅವರನ್ನು ಕುರಿತು “ಇಂದ್ರನು ನಿಮ್ಮ ಶಿಖೆಯಲ್ಲಿ ಅಮೃತವನ್ನು ಇಡಲಿಲ್ಲವೇ?” ಎಂದು ಪ್ರಶ್ನಿಸಿದನು.

\begin{verse}
\textbf{ವಃ ಶಿಖಾಸು ಚ ತೇ ಚೋಚುರ್ನೇತಿ ಲೋಭಾದ್ವಿಜೋತ್ತಮಾಃ~।}\\\textbf{ತಸ್ಮಿನ್ನಪರಪಕ್ಷೇಽಪಿ ಪಿತರೋ ದ್ವಿಜಸತ್ತಮಾಃ~।। ೪೭~।।}\\\textbf{ಯೂಯಂ ತಸ್ಮಾತ್ ವಿಜಾನೀಥ ಸತ್ಯಂ ಬ್ರೂತೇತಿ ಸೋಽಬ್ರವೀತ್~।}\\\textbf{ಅಸ್ತೀತಿ ಪಿತರಃ ಪ್ರೋಚುಃ ತತಸ್ತುಷ್ಟೋ ವರಂ ದದೌ~।। ೪೮~।।}
\end{verse}

ಕೆಲವು ಬ್ರಾಹ್ಮಣರು “ಇಲ್ಲ”ವೆಂತಲೂ ಮತ್ತೆ ಪಿತೃಗಣದಲ್ಲಿದ್ದ ಕೆಲವರು ಬ್ರಾಹ್ಮಣರನ್ನು “ನಿಮಗೆ ನಿಜಸಂಗತಿ ಗೊತ್ತಿದೆ, ಹೇಳಿ” ಎನ್ನಲು ಅವರು “ನಮ್ಮ ಶಿಖೆಯಲ್ಲಿ ಅಮೃತವು ಇಡಲ್ಪಟ್ಟಿದೆ ” ಎಂದರು. ಇದರಿಂದ ಸಂತುಷ್ಟನಾದ ಗರುಡನು ವರವನ್ನು ಕೊಟ್ಟನು.

\begin{verse}
\textbf{ಶಿಖಾಸಂಸ್ಥಾಮಾತ್ರಭೂಮೌ ಪಿತ್ರಾಹಾರೋ ಭವತ್ವಿತಿ~।}\\\textbf{ಇತಿ ದತ್ವಾ ವರಂ ತೇಭ್ಯೋ ಗರುಡೋ ರಕ್ತಲೋಚನಃ~।। ೪೯~।।}\\\textbf{ವಿಪ್ರಾನ್ ಶಶಾಪ ರೋಷೇಣ ಜರಾಮೃತ್ಯೋಸ್ತು ನಿಷ್ಕೃತಿಃ~।}\\\textbf{ಜಾತಾಪೀಯೂಷ ಸಂಸರ್ಗಾತ್ ನ ಸ್ಯಾದ್ವೋಽನೃತಭಾಷಿಣಾಮ್~।। ೫೦~।।}\\\textbf{ನೈಕಮತ್ಯಂ ಭವೇತ್ಕ್ವಾಪೀತ್ಯುಕ್ತ್ವಾ ಸೋಽಂತರಧೀಯತ~।}\\\textbf{ತಸ್ಮಾತ್ ಶಿಖೋದಕಂ ದದ್ಯಾತ್ ಪಿತೃಭ್ಯಃ ತುಷ್ಟಿ ಕಾರಣಮ್~।। ೫೧~।।}
\end{verse}

ಏನೆಂದರೆ, ಶಿಖಾದಿಂದ ನೀರಿನ ರೂಪದಿಂದ ಬಿದ್ದ ಅಮೃತವು ಪಿತೃಗಳಿಗೆ ಆಹಾರ\-ವಾಗಲಿ. ಸಿಟ್ಟಿನಿಂದ\enginline{-}ಇತರ ಬ್ರಾಹ್ಮಣರಿಗೆ ಶಾಪವನ್ನು ಕೊಟ್ಟನು\enginline{-}ಏನೆಂದರೆ ಅಮೃತಸೇವನೆಯಿಂದ ಆಗಬಹುದಾದ ಜನ್ಮ\enginline{-}ಮೃತ್ಯು ಇವುಗಳ ನಿವಾರಣೆಯು ನಿಮಗೆ ಆಗದಿರಲಿ; ಬ್ರಾಹ್ಮಣರಲ್ಲಿ ಎಂದಿಗೂ ಒಗ್ಗಟ್ಟು ಇಲ್ಲದೇ ಹೋಗಲಿ, ಹೀಗೆಂದು ಹೇಳಿ ಗರುಡನು ಅದೃಶ್ಯನಾದನು. ಆದುದರಿಂದ ಪಿತೃಗಳ ತೃಪ್ತಿಗಾಗಿ ಶಿಖೋದಕವನ್ನು ಕೊಡಬೇಕು.

\begin{verse}
\textbf{ತ ಏವ ಪಿತರೋ ಲೋಕೇ ಭೂಮಿದೇವಾಃ ಪ್ರಕೀರ್ತಿತಾಃ~।}\\\textbf{ಪಶ್ಚಾತ್ ವಸ್ತ್ರಂ ಪರೀಧಾಯ ಪೀಡಯಿತ್ವಾರ್ದ್ರವಸ್ತ್ರಕಮ್~।। ೫೨~।।}
\end{verse}

ಈ ಕಾರಣದಿಂದ ಬ್ರಾಹ್ಮಣರಿಗೆ “ಭೂಸುರರು” ಎಂಬ ಹೆಸರು ಬಂದಿತು. ನಂತರ ಒದ್ದೆ ವಸ್ತ್ರವನ್ನು ತೆಗೆದು ಬೇರೆ ಒಣಗಿದ ಬಟ್ಟೆಯನ್ನು ಧರಿಸಬೇಕು.

\begin{verse}
\textbf{ವಾಮಪ್ರಕೋಷ್ಠೇ ನಿಕ್ಷಿಪ್ಯ ಸಮ್ಯಗಾಚಮ್ಯ ವಾರಿಣಾ~।}\\\textbf{ಭೂಮೌ ನಿಕ್ಷಿಪ್ಯ ಪಶ್ಚಾತ್ತು ಮಂತ್ರರಾಜೇನ ಪ್ರೋಕ್ಷಯೇತ್~।। ೫೩~।।}
\end{verse}

ಒದ್ದೆ ವಸ್ತ್ರವನ್ನು ಎಡಮುಂಗೈ ಮೇಲೆ ಇಟ್ಟುಕೊಂಡು ಶುದ್ಧಾಚಮನಮಾಡಿ ನಂತರ ಅದನ್ನು ನೆಲದಮೇಲೆ ಇಟ್ಟು ಮಂತ್ರರಾಜನಿಂದ (ಗಾಯತ್ರಿ ಮಂತ್ರದಿಂದ) ನೀರಿನಿಂದ ಪ್ರೋಕ್ಷಿಸಬೇಕು.

\begin{verse}
\textbf{ಧೃತ್ವಾ ದ್ವಾದಶಪುಂಡ್ರಾನ್ ಹಿ ಸಂಧ್ಯಾ ಕರ್ಮ ಸಮಾಚರೇತ್~।}\\\textbf{ಪ್ರಾತರ್ಹೋಮಂ ತಥಾ ಕೃತ್ವಾ ದೇವರ್ಷಿಪಿತೃತರ್ಪಣಮ್~।। ೫೪~।।}\\\textbf{ಮಾಘಮಾಸಪ್ರಿಯಂ ದೇವಂ ಶಂಖಚಕ್ರಗದಾಧರಮ್~।\\ ಪ್ರಸನ್ನಂ ಸುಸ್ಮಿತಂ ಶಾಂತಂ ಮಾಧವಂ ಪೂಜಯೇತ್ತಥಾ~।। ೫೫~।।}
\end{verse}

ಗೋಪೀಚಂದನದಿಂದ ದ್ವಾದಶನಾಮ ಮುದ್ರೆ ಧರಿಸಿ ಸಂಧ್ಯಾವಂದನೆ ಮಾಡಿ ನಂತರ ದೇವ, ಋಷಿ, ಪಿತೃಗಳಿಗೆ ತರ್ಪಣ ಕೊಟ್ಟು ಪ್ರಾತಃಕಾಲದ ಹೋಮವನ್ನು ಆಚರಿಸಬೇಕು. ಆಮೇಲೆ ಮಾಘಮಾಸಕ್ಕೆ ನಿಯಾಮಕನಾದ, ಶಾಂತನಾದ, ಹಸನ್ಮುಖದಿಂದ ಕೂಡಿದ ಮಾಧವನನ್ನು ಪೂಜಿಸಬೇಕು.

\begin{verse}
\textbf{ಗಂಧೈಃ ಪುಷ್ಪೈಶ್ಚ ನೈವೇದ್ಯೈ ರ್ಧೂಪೈರ್ದೀಪೈರ್ಮನೋಹರೈಃ।}\\\textbf{ಅಪುಷ್ಪಂ ಪೂಜಯೇದ್ಯಸ್ತು ಮಾಧವಂ ಮಾಘವಲ್ಲಭಮ್~।। ೫೬~।।}\\\textbf{ಕುಲನಾಶೋ ಭವೇತ್ತಸ್ಯ ಪುಣ್ಯಂ ತಸ್ಯ ವಿನಶ್ಯತಿ~।}
\end{verse}

ಗಂಧ, ಪುಷ್ಪ, ನೈವೇದ್ಯ, ಧೂಪ, ದೀಪ ಈ ರೀತಿ ಮನಮೋಹಕವಾದ ಉಪಚಾರಗಳಿಂದ ಮಾಸಕ್ಕೆ ನಿಯಾಮಕನಾದ ಮಾಧವನನ್ನು ಅರ್ಚಿಸಬೇಕು. ಪುಷ್ಪವಿಲ್ಲದೇ ಪೂಜೆಮಾಡಿದರೆ ಪುಣ್ಯವು ನಾಶವಾಗುವುದಲ್ಲದೇ ಕುಲವೇ ಹಾಳಾಗುತ್ತದೆ.

\begin{verse}
\textbf{ಪುಷ್ಪೇಣೈಕೇನ ಮಾಘೇ ತು ಮಾಧವಂ ಪೂಜಯೇದ್ಯದಿ~।}\\\textbf{ಕುಲಕೋಟಿಸಮಾಯುಕ್ತೋ ಮೋದತೇ ವಿಷ್ಣು ಮಂದಿರೇ~।। ೫೭~।।}
\end{verse}

ಮಾಘಮಾಸದಲ್ಲಿ ಒಂದೇ ಪುಷ್ಪದಿಂದ ಪೂಜಿಸಿದರೂ ಒಂದು ಕೋಟಿ ಕುಲದಿಂದ ಸಹಿತನಾಗಿ ವೈಕುಂಠದಲ್ಲಿ ಆನಂದ ಅನುಭವಿಸುತ್ತಾನೆ.

\begin{verse}
\textbf{ಮೌಲ್ಯಾದಾನೀಯ ಪುಷ್ಪೌಘಂ ಗಂಧಯುಕ್ತಂ ಮನೋಹರಮ್~।}\\\textbf{ಬ್ರಾಹ್ಮಣೈಃ ಕಾರಯೇತ್ಪೂಜಾಂ ಯಸ್ಮಾತ್ ಪುಷ್ಪಂ ಹರೇಃ ಪ್ರಿಯಮ್~।। ೫೮~।।}
\end{verse}

ಶ‍್ರೀಹರಿಗೆ ಪ್ರಿಯವಾದ ಸುಗಂಧಯುಕ್ತವಾದ ಪುಷ್ಪಗಳನ್ನು ಕ್ರಯಕ್ಕೆ ತೆಗೆದುಕೊಂಡು ಬ್ರಾಹ್ಮಣರಿಂದ ಪೂಜೆಯನ್ನು ಮಾಡಿಸಬೇಕು. (ತನಗೆ ಸ್ವತಃ ಮಾಡಲು ಸಾಧ್ಯವಾಗದಿದ್ದರೆ ಎಂದರ್ಥ):

\begin{verse}
\textbf{ಮಾಧವೇ ತುಲಸೀಪತ್ರೈಃ ಮಕರಸ್ಥೇ ದಿವಾಕರೇ~।}\\\textbf{ಸಕೃದಭ್ಯರ್ಚ್ಯ ದೇವೇಶಂ ನ ಪುನರ್ಜಾಯತೇ ಭುವಿ~।। ೫೯~।।}
\end{verse}

ಸೂರ್ಯನು ಮಕರರಾಶಿಯಲ್ಲಿ, ಮಾಘಮಾಸದಲ್ಲಿ, ಸರ್ವೊತ್ತಮನಾದ ಶ‍್ರೀಹರಿಯನ್ನು ತುಳಸೀದಳಗಳಿಂದ ಒಂದೇ ಸಲ ಪೂಜಿಸಿದರೂ ಅಂತಹವನಿಗೆ ಪುನರ್ಜನ್ಮವಿಲ್ಲ.

\begin{verse}
\textbf{ಯಃ ಪೂಜಾಂ ತುಲಸೀಹೀನಾಂ ಮಾಘೇ ಮಕರಗೇ ರವೌ~।}\\\textbf{ಕುರ್ಯಾದಿತಿ ವಿಮೂಢಾತ್ಮಾ ಹೃದಿ ಶಲ್ಯಂ ಮಮಾರ್ಪಿತಮ್~।। ೬೦~।।}
\end{verse}

ತುಳಸೀ ಇಲ್ಲದೆ ಯಾರು ಪೂಜಿಸುತ್ತಾರೋ ಅಂತಹ ಮೂಢರು ನನ್ನ ಎದೆಗೆ ಬಾಣವನ್ನು ಎಸೆದಂತೆ ಆಗುತ್ತದೆ.

\begin{verse}
\textbf{ಮಾಧವಂ ತುಲಸೀಹೀನಾಂ ಪೂಜಾಂ ಕರ್ತುಂ ದ್ವಿಜೋತ್ತಮ~।}\\\textbf{ಕುಲಾನಿ ಪಾತಯೇತ್ಸಪ್ತ ನರಕೇ ರುಧಿರೋದಯೇ~।। ೬೧~।।}
\end{verse}

ತುಳಸೀರಹಿತವಾದ ಪೂಜೆಯನ್ನು ಮಾಡಿದರೆ ಸಪ್ತಕುಲಗಳನ್ನು ನರಕದಲ್ಲಿ ಬೀಳಿಸುತ್ತಾನೆ.

\begin{verse}
\textbf{ಯೋ ದದ್ಯಾತ್ತುಲಸೀಪತ್ರಂ ಪಾದಯೋಶ್ಚಕ್ರಪಾಣಿನಃ~।}\\\textbf{ಕುಲಕೋಟಿಸಮಾಯುಕ್ತೋ ವಿಷ್ಣುಲೋಕೇ ಮಹೀಯತೇ~।। ೬೨~।। }
\end{verse}

ಚಕ್ರಪಾಣಿಯಾದ ಶ‍್ರೀಹರಿಯ ಪಾದಗಳಲ್ಲಿ ತುಳಸೀಪತ್ರಗಳನ್ನು ಅರ್ಪಿಸಿದರೆ ಒಂದು ಕೋಟಿ ಕುಲದಿಂದ ಯುಕ್ತನಾಗಿ ವೈಕುಂಠದಲ್ಲಿ ಮೆರೆಯುತ್ತಾನೆ.

\begin{verse}
\textbf{ಯೋ ದದ್ಯಾತ್ತುಲಸೀಗಂಧಮಪಿ ಸರ್ಷಪಮಾತ್ರಕಮ್~।}\\\textbf{ಕೋಟಿಜನ್ಮಾರ್ಜಿತಂ ಪಾಪಂ ತತ್ ಕ್ಷಣಾದೇವ ನಶ್ಯತಿ~।। ೬೩~।।}
\end{verse}

ಒಂದು ಸಾಸಿವೆಕಾಳಿನಷ್ಟು ಗಂಧವನ್ನೂ, ತುಳಸೀಯನ್ನೂ ಅರ್ಪಿಸಿದರೆ ಕೋಟಿಜನ್ಮಗಳಲ್ಲಿ ಮಾಡಿರುವ ಪಾಪಗಳು ಕೂಡಲೇ ನಾಶಹೊಂದುತ್ತವೆ.

\begin{verse}
\textbf{ಶ‍್ರೀಖಂಡಂ ಚಂದನೋನ್ಮಿಶ್ರಂ ಕೃಷ್ಣಾಗಾರುಸಮನ್ವಿತಮ್~।}\\\textbf{ವೈಕುಂಠೇ ಮೋದತೇ ನಿತ್ಯಂ ದಶಪೂರ್ವೈಃ ದಶಾಪರೈಃ~।। ೬೪~।।}
\end{verse}

ಶ‍್ರೀಖಂಡ, ಗಂಧ, ಕೃಷ್ಣಾಗಾರು\enginline{-}ಇಂತಹ ಪರಿಮಳವಸ್ತುಗಳನ್ನು ಅರ್ಪಿಸಿದರೆ ಅಂತಹವನು ತನ್ನ ಹಿಂದಿನ ಹತ್ತು ಕುಲಗಳಿಂದಲೂ ಮುಂದಿನ ಹತ್ತು ಕುಲಗಳಿಂದಲೂ ಸಹಿತನಾಗಿ ವೈಕುಂಠದಲ್ಲಿ ಶಾಶ್ವತವಾದ ಸುಖವನ್ನು ಪಡೆಯುತ್ತಾನೆ.

\begin{verse}
\textbf{ಯಸ್ತುಲಸ್ಯಾಃ ಪತ್ರಶತಂ ವಿಷ್ಣವೇ ಪರಮಾತ್ಮನೇ~।}\\\textbf{ಸಮರ್ಪಯತಿ ತತ್ಪುಣ್ಯಂ ವಕ್ತುಂ ಶಕ್ತೋ ನ ವಿದ್ಯತೇ~।। ೬೫~।।}
\end{verse}

ಸರ್ವೋತ್ತಮನಾದ ವಿಷ್ಣುವಿನಲ್ಲಿ ನೂರು ತುಳಸೀದಳಗಳನ್ನು ಅರ್ಪಿಸಿದವನ ಪುಣ್ಯವನ್ನು ವಿವರಿಸಲು ಶಕ್ತರು ಯಾರಿದ್ದಾರೆ?

\begin{verse}
\textbf{ಪಿತೄನ್ ಗುರೂನ್ ಸಮುದ್ದಿಶ್ಯ ತುಲಸೀ ವಿಷ್ಣವೇರ್ಪಿತಾ~।}\\\textbf{ಬ್ರಹ್ಮಹತ್ಯಾ ಕೋಟಿಶತಮಗಮ್ಯಾನಾಮನಂತಕಮ್~।। ೬೬~।।}\\\textbf{ಅಭಿತಃ ಕೋಟಿಪಾಪಾನಿ ದಹತ್ಯಗ್ನಿರಿವೇಂಧನಮ್~।। ೬೭~।।}
\end{verse}

ಪಿತೃಗಳನ್ನು, ಗುರುಗಳನ್ನು ಉದ್ದೇಶಿಸಿ ವಿಷ್ಣುವಿಗೆ ತುಳಸಿಯನ್ನು ಅರ್ಪಿಸಿದರೆ ಒಂದು ನೂರು ಕೋಟಿ ಬ್ರಹ್ಮಹತ್ಯಾದಿಂದ ಉಂಟಾದ ಪಾಪಗಳೂ, ಪರಸ್ತ್ರೀ ಗಮನವೇ ಮುಂತಾದ ಅಸಂಖ್ಯ ಪಾಪಗಳೂ ಅಗ್ನಿಯಿಂದ ಸೌದೆಯು ಸುಡಲ್ಪಡುವಂತೆ ನಾಶವಾಗುತ್ತವೆ.

\begin{verse}
\textbf{ಯಃ ಪೂಜಯತಿ ದೇವೇಶಂ ಕಮಲೈಃ ಕಮಲೇಕ್ಷಣಮ್~।}\\\textbf{ತಸ್ಯ ಪುಣ್ಯಫಲಂ ವಕ್ತುಂ ಕಃ ಶಕ್ತೋ ವರ್ತತೇ ಭುವಿ~।। ೬೮~।।}
\end{verse}

ಕಮಲದಂತೆ ನೇತ್ರಗಳನ್ನುಳ್ಳ ಶ‍್ರೀಹರಿಯನ್ನು ಕಮಲಪುಷ್ಪಗಳಿಂದ ಪೂಜಿಸಿದರೆ ಬರುವ ಪುಣ್ಯವನ್ನು ಹೇಳಲು ಜಗತ್ತಿನಲ್ಲಿ ಸಮರ್ಥರು ಯಾರಿದ್ದಾರೆ?

\begin{verse}
\textbf{ಕಮಲೈಃ ಪೂಜಿತೇ ವಿಷ್ಣೌ ಕಮಲಾ ವಸತಿ ಗೃಹೇ~।}\\\textbf{ಫಲಂತಿ ಕಾಮಾಃ ಸರ್ವೇಽಪಿ ನೈವ ದುಃಖಂ ಪ್ರಜಾಯತೇ~।। ೬೯~।।}
\end{verse}

ಅಂತಹವನ ಮನೆಯಲ್ಲಿ ಲಕ್ಷ್ಮೀ ದೇವಿಯು ವಾಸಮಾಡುತ್ತಾಳೆ, ಎಲ್ಲ ಅಭೀಷ್ಟಗಳೂ ನೆರವೇರುತ್ತವೆ, ದುಃಖವು ಉಂಟಾಗುವುದೇ ಇಲ್ಲ.

\begin{verse}
\textbf{ದ್ವಾತ್ರಿಂಶದಪರಾಧಾನಿ ಕ್ಷಮತೇ ಪುರುಷೋತ್ತಮಃ~।}\\\textbf{ಅರ್ಚಿತಸ್ತುಲಸೀಪತ್ರೈರ್ಮಾಧವೋ ಭಕ್ತವತ್ಸಲಃ~।। ೭೦~।।}
\end{verse}

ತುಳಸಿಯಿ೦ದ ಪೂಜೆಗೊಂಡ ಭಕ್ತವತ್ಸಲನಾದ ಶ‍್ರೀಹರಿಯು ಭಕ್ತನ ಮೂವತ್ತೆರಡು ಅಪರಾಧಗಳನ್ನು ಕ್ಷಮಿಸುತ್ತಾನೆ.

\begin{verse}
\textbf{ಅಪರಾಧಸಹಸ್ರಂ ವಾ ಸಹತೇ ನಾತ್ರ ಸಂಶಯಃ~।}\\\textbf{ಯಃ ಕುರ್ಯಾತ್ ತುಲಸೀಪತ್ರೈಃ ಪೂಜಾಂ ವಿಷ್ಣೋಃ ಮಹಾತ್ಮನಃ~।। ೭೧~।।}
\end{verse}

ತುಳಸಿಯಿಂದ ಪೂಜೆಮಾಡಿದ ಭಕ್ತನ ಸಾವಿರ ಅಪರಾಧಗಳನ್ನೂ ಸರ್ವೋತ್ತಮನಾದ ವಿಷ್ಣುವು ಸಹಿಸುತ್ತಾನೆ.

\begin{verse}
\textbf{ಜನ್ಮ ಕೊಟಿಸಹಸ್ರಂ ತು ದರಿದ್ರೋ ನ ಭವೇನ್ನರಃ~।}\\\textbf{ವಿಧವಾತ್ವಂ ನ ಚಾಪ್ನೋತಿ ನಾರೀ ಜನ್ಮಶತೈರಪಿ~।। ೭೨~।।}
\end{verse}

ಅಂತಹ ಭಕ್ತನು ಸಹಸ್ರಕೋಟಿ ಜನ್ಮಗಳಲ್ಲಿಯೂ ದರಿದ್ರನಾಗಿ ಹುಟ್ಟುವುದಿಲ್ಲ; ಸ್ತ್ರೀಯಾದರೆ ನೂರುಜನ್ಮಗಳಲ್ಲಿಯೂ ವೈಧವ್ಯವನ್ನು ಅನುಭವಿಸುವುದಿಲ್ಲ.

\begin{verse}
\textbf{ನಾರೀ ವಾ ಪುರುಷೋ ವಾಪಿ ಮಾಘೇ ಪುಷ್ಪಂ ಸುಗಂಧಿನಮ್~।}\\\textbf{ದತ್ವಾ ಚೈಕಂ ಮಹಾವಿಷ್ಣೋಃ ಬ್ರಹ್ಮಹತ್ಯಾಂ ವ್ಯಪೋಹತಿ~।। ೭೩~।।}
\end{verse}

ಮಾಘಮಾಸದಲ್ಲಿ ಸ್ತ್ರೀಯಾಗಲೀ ಪುರುಷನಾಗಲೀ ಸರ್ವೋತ್ತಮನಾದ ವಿಷ್ಣುವನ್ನು ಸುಗಂಧಯುಕ್ತವಾದ ಒಂದೇ ಪುಷ್ಪದಿಂದ ಪೂಜಿಸಿದರೂ ಬ್ರಹ್ಮಹತ್ಯಾ ದೋಷವು ತಟ್ಟುವುದಿಲ್ಲ.

\begin{verse}
\textbf{ಅಗಂಧಂ ಪೂಜಯೇದ್ಯಸ್ತು ಮಾಘೇ ಮಾಧವಮವ್ಯಯಮ್~।}\\\textbf{ನಾರಕೀಂ ಯಾತನಾಂ ಭುಂಕ್ತೇ ವಿಡ್‌ವರಾಹೋ ಭವೇದ್ಧ್ರುವಮ್~।। ೭೪~।।}
\end{verse}

ಗಂಧವನ್ನು ಅರ್ಪಿಸದೇ ಮಾಧವನನ್ನು ಪೂಜಿಸುವವನು ನರಕದಲ್ಲಿ ಯಾತನೆಯನ್ನು ಅನುಭವಿಸಿ ಊರಹಂದಿಯಾಗಿ ಹುಟ್ಟುವನು.

\begin{verse}
\textbf{ಧೂಪಹೀನಾಂ ಸಪರ್ಯಾತು ಯಃ ಕುರ್ಯಾತ್ಪರಮಾತ್ಮನಃ~।}\\\textbf{ವಿಷ್ಟಾಕೃಮಿತ್ವ ಮಾಪ್ನೋತಿ ಜನ್ಮಕೋಟಿಶತೈರಪಿ~।। ೭೫~।।}
\end{verse}

ಪರಮಾತ್ಮನಿಗೆ ಧೂಪವನ್ನು ಸಮರ್ಪಿಸದೇ ಪೂಜಿಸಿದರೆ ನೂರುಕೊಟಿ ಜನ್ಮಗಳಲ್ಲಿ ಮಲದಲ್ಲಿಯ ಕೃಮಿಯಾಗಿ ಹುಟ್ಟುತ್ತಾನೆ.

\begin{verse}
\textbf{ದಶಾಂಗಂ ಗುಗ್ಗುಲಂ ಧೂಪಂ ಯಃ ಸಮರ್ಪಯತಿ ವಿಷ್ಣವೇ~।}\\\textbf{ನ ತಸ್ಯ ಸಂತತೇರ್ಹಾನಿರ್ನ ಪಾಪೀ ಜಾಯತೇ ಕುಲೇ~।। ೭೬~।।}
\end{verse}

ದಶಾಂಗದಿಂದ ಸಹಿತವಾದ ಧೂಪದಿಂದ ವಿಷ್ಣುವನ್ನು ಪೂಜಿಸಿದರೆ ಅಂತಹವನಿಗೆ ಸಂತತಿಯ ಹಾನಿಯೇ ಆಗುವುದಿಲ್ಲ; ಅಂತಹವನ ಕುಲದಲ್ಲಿ ಪಾಪಿಷ್ಠನು ಹುಟ್ಟುವುದಿಲ್ಲ.

\begin{verse}
\textbf{ಸ್ವಯಂ ಮೋಕ್ಷಮವಾಪ್ನೋತಿ ನಾತ್ರ ಕಾರ್ಯಾ ವಿಚಾರಣಾ~।}\\\textbf{ಘಂಟಾನಾದಂ ತತಃ ಕುರ್ಯಾತ್ ಪೂಜಾಕಾಲೇ ಮಹಾತ್ಮನಃ~।। ೭೭~।।}\\\textbf{ಸ್ತೂಯಮಾನೋ ದೇವಗಣೈಃ ವಿಷ್ಣುಲೋಕೇ ಮಹೀಯತೇ~।।}
\end{verse}

ತಾನು ಸಹ ಮೋಕ್ಷವನ್ನು ಪಡೆಯುತ್ತಾನೆಂಬುದರಲ್ಲಿ ಸಂದೇಹವೇ ಇಲ್ಲ. ಪೂಜೆಯ ಕಾಲದಲ್ಲಿ ಘಂಟೆಯನ್ನು ಬಾರಿಸುವವನು ದೇವತೆಗಳಿಂದ ಶ್ಲಾಘಿಸಲ್ಪಟ್ಟು ವೈಕುಂಠಲೋಕಕ್ಕೆ ಹೋಗುವನು.

\begin{verse}
\textbf{ಘಂಟಾನಾದವಿಹೀನಂ ತು ಯಃ ಕುರ್ಯಾದ್ದೇವತಾರ್ಚನಮ್~।}\\\textbf{ಸ ಭುಕ್ತ್ವಾ ನರಕಾನ್ ಘೋರಾನ್ ಸ ಮೂಕೋ ಬಧಿರೋ ಭವೇತ್~।।}
\end{verse}

ಘಂಟಾನಾದವಿಲ್ಲದೆ ದೇವರ ಪೂಜೆಯನ್ನು ಮಾಡುವವನು ನರಕದಲ್ಲಿ ಕ್ರೂರವಾದ ದುಃಖಗಳನ್ನು ಅನುಭವಿಸಿ ಮುಂದೆ ಮೂಕನೋ, ಕಿವುಡನೋ ಆಗಿ ಹುಟ್ಟುತ್ತಾನೆ.

\begin{verse}
\textbf{ದೀಪದಾನಂ ತು ಯಃ ಕುರ್ಯಾತ್ ದೇವಾಯ ಪರಮಾತ್ಮನೇ~।}\\\textbf{ಅಜ್ಞಾನಾಂಧಂ ತಮೋ ಭಿತ್ವಾ ಸಾಕ್ಷಾತ್ ಪಶ್ಯತಿ ತಂ ಹರಿಮ್~।। ೭೯~।।}
\end{verse}

ದೇವರ ಪೂಜೆ ಕಾಲದಲ್ಲಿ ಪರಮಾತ್ಮನ ಮುಂದೆ ದೀಪಗಳನ್ನು ಹಚ್ಚಿ ಇಡುವವನು ಅಜ್ಞಾನ\-ವೆಂಬ ಕತ್ತಲೆಯನ್ನು ಭೇದಿಸಿಕೊಂಡು ಶ‍್ರೀಹರಿಯನ್ನು ಕಾಣುತ್ತಾನೆ.

\begin{verse}
\textbf{ಶ್ರೂಯತೇ ಪಿತೃಗಾಥಾಪಿ ದೀಪದಾನೇ ಮುನಿಶ್ವರಾಃ~।। ೮೦~।।}
\end{verse}

ಮುನಿಶ್ರೇಷ್ಠರೇ, ಈ ದೀಪದಾನದ ವಿಷಯದಲ್ಲಿ ಪಿತೃಗಳು ಹೇಳುವುದೂ ಕಂಡಿದೆ.

\begin{verse}
\textbf{ಯೋ ಕೋ ವಾಸ್ಮತ್ಕುಲೇ ಜಾತೋ ಮಾಘೇ ಮಾಸಿ ಹರಿಪ್ರಿಯಃ~।}\\\textbf{ದದ್ಯಾದ್ದೀಪಂ ಸ್ವಯಂ ವಾಪಿ ಸ ನಃ ಸಂತಾರಯಿಷ್ಯತಿ~।। ೮೧~।।}
\end{verse}

ನಮ್ಮ ವಂಶದಲ್ಲಿ ಹರಿಭಕ್ತನು ಯಾರಾದರೂ ಹುಟ್ಟಿ ಮಾಘಮಾಸದಲ್ಲಿ ದೀಪದಾನ ಮಾಡಿದರೆ ತಾನೂ ಸಂಸಾರವನ್ನು ದಾಟುವನಲ್ಲದೆ ನಮ್ಮನ್ನೂ ಸಹ ದಾಟಿಸುತ್ತಾನೆ.

\begin{verse}
\textbf{ದೀಪಾಯ ವರ್ತಿತೈಲಂ ವಾ ಪಾತ್ರಂ ಯೋ ವಾ ದದಾತಿ ಚ~।}\\\textbf{ನಾರೀ ವಾ ನ ಪುನಃ ಕರ್ಮಲೇಶಮಾತ್ರಂ ನ ವಿದ್ಯತೇ~।। ೮೨~।।}
\end{verse}

ದೀಪಕ್ಕಾಗಿ, ಎಣ್ಣೆ, ಬತ್ತಿ ಅಥವಾ ಪಾತ್ರೆಯನ್ನು ದಾನಮಾಡುವ ಸ್ತ್ರೀ ಯಾಗಲೀ, ಪುರುಷನಾಗಲೀ ಕರ್ಮಬಂಧನದಿಂದ ಮುಕ್ತರಾಗುವರು.

\begin{verse}
\textbf{ತೈಲಂ ಪಾತ್ರಂ ತಥಾ ವರ್ತಿ ನ ದದಾತಿ ಸ್ವಶಕ್ತಿತಃ~। }\\\textbf{ಸ ಜಾತ್ಯಾಂಧೋ ಭವೇನ್ನೂನಂ ಭುಕ್ತ್ವಾ}\\\textbf{ ಸರ್ವಾಂ ತು ಯಾತನಾಮ್~।। ೮೩~।।}
\end{verse}

ತನ್ನ ಶಕ್ತಿಗೆ ಅನುಗುಣವಾಗಿ ಎಣ್ಣೆ, ಪಾತ್ರೆ ಹಾಗೂ ಬತ್ತಿಗಳನ್ನು ದಾನ ಮಾಡದವನು ನರಕದಲ್ಲಿ ದುಃಖವನ್ನು ಅನುಭವಿಸಿ ಮುಂದೆ ಹುಟ್ಟು ಕುರುಡ ನಾಗುವನು.

\begin{verse}
\textbf{ಅಭಾವೇ ಸರ್ವತಃ ಕುರ್ಯಾತ್ ಪರದೀಪಪ್ರಬೋಧನಮ್~।}\\\textbf{ವಿದಧಾತ್ರಿಹರೇಃ ಪೂಜಾಂ ವಿನಾ ನೈವೇದ್ಯ ಮಜ್ಞಧೀಃ~।।}\\\textbf{ಕುಂಭೀಪಾಕೇ ಮಹಾಘೋರೇ ಪಚ್ಯತೇ ನರಕಾಗ್ನಿನಾ~।। ೮೪~।।}
\end{verse}

ಶಕ್ತಿಹೀನನಾದರೆ, ಇನ್ನೊಬ್ಬರು ಹಚ್ಚಿರುವ ದೀಪವು ಹೆಚ್ಚು ಬೆಳಕನ್ನು ಕೊಡುವಂತೆ ಮಾಡಬೇಕು\enginline{-}ಬತ್ತಿಯನ್ನು ಮುಂದಕ್ಕೆ ಸರಿಸುವುದು ಇತ್ಯಾದಿ. ನೈವೇದ್ಯ ಸಮರ್ಪಣೆ ಇಲ್ಲದೆ ದೇವತಾರ್ಚನೆ ಮಾಡುವ ಅಜ್ಞಾನಿಯು ನರಕದಲ್ಲಿ ಅಗ್ನಿಯಿಂದ ಕುಂಭೀಪಾಕದಲ್ಲಿ ಬೇಯಿಸಲ್ಪಡುತ್ತಾನೆ.

\begin{verse}
\textbf{ಅನ್ನಂ ಬಹುಗುಣಂ ಭಕ್ಷ್ಯಂ ಫಲಂ ನಾನಾವಿಧಂ ಶುಭಮ್~।}\\\textbf{ನಿವೇದಯತಿ ವಿಪ್ರೇಂದ್ರ ಬ್ರಹ್ಮಭೂಯಾಯ ಕಲ್ಪತೇ~।। ೮೫~।।}
\end{verse}

ರುಚಿಕರವಾದ ನಾನಾ ಬಗೆಯ ಭಕ್ಷ್ಯ, ಅನ್ನ, ಫಲ ಇವುಗಳನ್ನು ಸಮರ್ಷಣೆ ಮಾಡುವವನು ಅಪರೋಕ್ಷಜ್ಞಾನಕ್ಕೆ ಸಿದ್ಧನಾಗುತ್ತಾನೆ..

\begin{verse}
\textbf{ಲೋಕಯಾತ್ರಾಪ್ರಸಂಗೇನ ಯಃ ಕುರ್ಯಾದ್ದೇವತಾರ್ಚನಮ್~।}\\\textbf{ತಪ್ತಾಯಸೇ ಮಹಾಘೋರೇ ಯನ್ನಃ ಪಾತಯತಿ ಸ್ವಯಮ್~।। ೮೬~।।}
\end{verse}

ಹಣದ ಆಸೆಯಿಂದ ದೇವತಾರ್ಚನೆ ಮಾಡುವವನು ನರಕದಲ್ಲಿ ಕೆಂಪಗೆ ಕಾದಿರುವ\break ಕಬ್ಬಿಣದ ತಗಡಿನ ಮೇಲೆ ಬೀಳಿಸಲ್ಪಡುತ್ತಾನೆ.

\begin{verse}
\textbf{ಸ್ತೋತ್ರಪಾಠಮೃತೇ ಕುರ್ಯಾನ್ನಿರ್ಮಾಲ್ಯಸ್ಯ ವಿಸರ್ಜನಮ್~।}\\\textbf{ಸ ಚೋನ್ಮಾದೀ ಭವತ್ಯಾಶು ನಾತ್ರ ಕಾರ್ಯಾ ವಿಚಾರಣಾ~।। ೮೭~।।}
\end{verse}

ವೇದೋಕ್ತ ಮಂತ್ರಪಠನ ಇಲ್ಲದೇ ನಿರ್ಮಾಲ್ಯ ವಿಸರ್ಜನೆ ಮಾಡುವವನು ಹುಚ್ಚನಾಗಿಯೋ, ಮೂರ್ಛೆರೋಗದಿಂದ ಪೀಡಿತನಾಗಿಯೋ ಹುಟ್ಟುತ್ತಾನೆ.

\begin{verse}
\textbf{ನೀರಾಜಯತಿ ಯೋ ವಿಷ್ಣುಂ ಮಾಧವಂ ಕಮಲೋದ್ಭವಮ್~।}\\\textbf{ಕೋಟಿಗುರ್ವಂಗನಾಸೇವಾ ಬ್ರಹ್ಮಹತ್ಯಾಯುತಾನಿ ಚ~।।}\\\textbf{ಪಾಪಾನ್ಯ ನ್ಯಾನಿ ಚೋಗ್ರಾಣಿ ತಸ್ಯ ದಗ್ಧಾನ್ಯಸಂಶಯಃ~।। ೮೮~।।}
\end{verse}

ಪರಮಾತ್ಮನಿಗೆ ಮಂಗಳಾರತಿ ಮಾಡಿದರೆ ಒಂದು ಕೋಟಿ ಗುರುಸ್ತ್ರೀಯರ ಸಂಗ ಮಾಡಿದ ಪಾಪವೂ, ಅನೇಕ ಬ್ರಹ್ಮಹತ್ಯಾ ಮಾಡಿದ ಪಾಪವೂ ಮತ್ತು ಇತರ ಉಗ್ರವಾದ ಪಾಪಗಳೂ ಭಸ್ಮವಾಗುತ್ತವೆ.

\begin{verse}
\textbf{ಸುಖಂ ನೀರಾಜನೇ ವಿಷ್ಣೋಃ ಪಶ್ಯಂತಿ ಪುರುಷಾಃ ಸ್ತ್ರೀಯಃ~।}\\\textbf{ತೇ ಧನ್ಯಾಃ ಕೃತಿನೋ ಲೋಕೇ ತೇಷಾಂ ಪುಣ್ಯಮನಂತಕಮ್~।। ೮೯~।।}
\end{verse}

ಪರಮಾತ್ಮನಿಗೆ ಮಂಗಳಾರತಿಯನ್ನು ಮಾಡುತ್ತಿರುವಾಗ ದರ್ಶನ ಮಾಡುವ ಸ್ತ್ರೀ ಪುರುಷರೇ ಧನ್ಯರು, ಕೃತಕೃತ್ಯರು; ಅವರ ಪುಣ್ಯ ಅಗಾಧ.

\begin{verse}
\textbf{ಕರ್ಪೂರಸ್ಯಾರ್ತಿಕಂ ಭಕ್ತ್ಯಾ ವಿಷ್ಣೋರ್ಯಸ್ತು ನಿವೇದಯೇತ್~।}\\\textbf{ಬ್ರಹ್ಮಲೋಕಂ ಸಮಾಸಾದ್ಯ ಭುಕ್ತ್ವಾ ಭೋಗಾನ್ ಮನೋಹರಾನ್~।। ೯೦~।। }\\\textbf{ಕೋಟಿವಾರಂ ಸಾರ್ವಭೌಮಃ ಸರ್ವಸಂಪತ್ಸಮಾವೃತಃ~।}\\\textbf{ಭೂತ್ವಾ ಪಶ್ಚಾತ್ ದ್ವಿಜಕುಲೇ ಜನ್ಮಾಸಾದ್ಯ ಮಹಾಯಶಾಃ~।।} \\\textbf{ಜ್ಞಾನವಿದ್ಯಾಸು ನಿಪುಣೋ ವಿಷ್ಣೋಃ ಸಾಯುಜ್ಯಮಾಪ್ನುಯಾತ್~।। ೯೧~।।}
\end{verse}

ಭಕ್ತಿಯಕ್ತನಾಗಿ ಕರ್ಪೂರದಿಂದ ಆರತಿ ಮಾಡುವವನು ಸತ್ಯಲೋಕದಲ್ಲಿ ಅನೇಕ ಸುಖಗಳನ್ನು ಅನುಭವಿಸಿ, ಒಂದು ಕೋಟಿ ಸಲ ಚಕ್ರವರ್ತಿಯಾಗಿ ಸಮಸ್ತ ಸಂಪತ್ತನ್ನೂ ಪಡೆದು, ನಂತರ ಬ್ರಾಹ್ಮಣಕುಲದಲ್ಲಿ ಹುಟ್ಟಿ, ಜ್ಞಾನ, ವಿದ್ಯಾ ಇವುಗಳಲ್ಲಿ ಕುಶಲನಾಗಿ ಕಡೆಯಲ್ಲಿ ವಿಷ್ಣು ಸಾಯುಜ್ಯ ಮೋಕ್ಷವನ್ನು ಪಡೆಯುವನು.

\begin{verse}
\textbf{ಪೂಜಾಂತೇ ಮಾಧವಸ್ಯಾಗ್ರೇ ಸಾಷ್ಟಾಂಗಂ ಪ್ರಣವೇದ್ಯದಿ~।}\\\textbf{ಅಶ್ವಮೇಧಸಹಸ್ರೇಷು ಸಮ್ಯಕ್ ಚೀರ್ಣೆಷು ಯತ್ಫಲಮ್~।।}\\\textbf{ತತ್ಫಲಂ ಸಮವಾಪ್ನೋತಿ ನಾತ್ರ ಕಾರ್ಯಾ ವಿಚಾರಣಾ~।। ೯೨~।।}
\end{verse}

ಪೂಜಾನಂತರದಲ್ಲಿ ಪರಮಾತ್ಮನ ಮುಂಭಾಗದಲ್ಲಿ ಸಾಷ್ಟಾಂಗ ನಮಸ್ಕಾರವನ್ನು ಮಾಡುವವನು ಸಾವಿರ ಅಶ್ವಮೇಧ ಯಜ್ಞದ ಫಲವನ್ನು ಹೊಂದುತ್ತಾನೆ. ಇದರಲ್ಲಿ ಸಂದೇಹವಿಲ್ಲ.

\begin{verse}
\textbf{ಅಪಿ ಧರ್ಮಸಹಸ್ರಾಣಿ ತಥಾ ಕ್ರತುಶತಾನಿ ಚ~।}\\\textbf{ವಿಷ್ಣೋರ್ದಂಡಪ್ರಣಾಮಸ್ಯ ಕಲಾಂ ನಾರ್ಹಂತಿ ಷೋಡಶೀಮ್~।। ೯೩~।।}
\end{verse}

ಸಹಸ್ರಾರು ಧರ್ಮಗಳೂ ನೂರಾರು ಯಜ್ಞಗಳೂ ಸಹ ಶ‍್ರೀವಿಷ್ಣುವಿನ ಮುಂದೆ ಸಾಷ್ಟಾಂಗ ನಮಸ್ಕಾರದ ಫಲದಲ್ಲಿ ಹದಿನಾರರಲ್ಲಿ ಒಂದಂಶದಷ್ಟೂ ದೊರಕಿಸಿ ಕೊಡಲಾರವು.

\begin{verse}
\textbf{ಯಾನಿ ಕಾನಿ ಚ ಪಾಪಾನಿ ಜನ್ಮಾಂತರಕೃತಾನಿ ಚ~।}\\\textbf{ತಾನಿ ಸರ್ವಾಣಿ ನಶ್ಯಂತಿ ಪ್ರದಕ್ಷಿಣಪದೇಪದೇ~।। ೯೪~।। }
\end{verse}

ಶ‍್ರೀಹರಿಗೆ ಪ್ರದಕ್ಷಿಣೆ ಮಾಡಿದವನ ಅನೇಕ ಜನ್ಮಗಳ ಪಾಪಗಳು ಹೆಜ್ಜೆ ಹೆಜ್ಜೆಯಲ್ಲಿಯೂ ನಾಶವಾಗುತ್ತವೆ.

\begin{verse}
\textbf{ಯಾವತ್ಪಾಪಭಯಂ ನೄಣಾಂ ಜನ್ಮ ಮೃತ್ಯು ಜರಾಭಯಮ್~।}\\\textbf{ಯಾವತ್ ಪ್ರದಕ್ಷಿಣಂ ಕುರ್ಯಾನ್ಮನೋ ಯಸ್ಯ ನ ವಿದ್ಯತೇ~।। ೯೫~।।}
\end{verse}

ಪ್ರದಕ್ಷಿಣೆ ಮಾಡುವವರೆಗೆ ಪಾಪದ ಭಯವೂ, ಜನ್ಮ, ಮೃತ್ಯು, ಮುಪ್ಪು ಇವುಗಳ ಭಯವು ತಪ್ಪಿದ್ದಲ್ಲ. ಪ್ರದಕ್ಷಿಣೆ ಮಾಡಿದ ಕೂಡಲೇ ಆ ಭಯಗಳು ಹೋಗುತ್ತವೆ.

\begin{verse}
\textbf{ನಾರೀ ವಾ ಪುರುಷೋ ವಾಪಿ ಮಾರ್ಜನಂ ಕೇಶವಾಲಯೇ~।}\\\textbf{ಕೃತ್ವಾ ಗೋಚರ್ಮಮಾತ್ರಂ ವಾ ವಿಷ್ಣೋರ್ಲೋಕೇ ಮಹೀಯತೇ~।।೯೬।।}
\end{verse}

ದೇವರ ಮನೆಯಲ್ಲಿ ಸ್ವಲ್ಪ ಭಾಗ ನೆಲವನ್ನು ಗೋಮಯದಿಂದ ಸಾರಿಸಿ ಶುದ್ಧ ಮಾಡುವ ಸ್ತ್ರೀ ಪುರುಷರು ವೈಕುಂಠದಲ್ಲಿ ಮೆರೆಯುತ್ತಾರೆ.

\begin{verse}
\textbf{ಮಾಘೇ ಮಾಸಿ ಮಹಾವಿಷ್ಣೋಃ ಮಾಧವಸ್ಯ ಮಹಾತ್ಮನಃ~।}\\\textbf{ಗೋಮಯೇನೋಪಲಿಪ್ಯಾದ್ಯೋ ಗೋಸಹಸ್ರಫಲಂ ಲಭೇತ್~।। ೯೭~।।}
\end{verse}

ಮಾಘ ಮಾಸದಲ್ಲಿ ಲಕ್ಷ್ಮೀರಮಣನಾದ ವಿಷ್ಣುವಿನ ಮಂದಿರವನ್ನು ಗೋಮಯದಿಂದ ಸಾರಿಸಿದವನು ಸಹಸ್ರಗೋದಾನಫಲವನ್ನು ಹೊಂದುವನು,

\begin{verse}
\textbf{ಯಾ ನಾರೀ ರಂಗವಲ್ಲೀಭಿಃ ಪಂಚರಂಗೈಃ ಹರೇರ್ಗೃಹಮ್~।}\\\textbf{ಅಲಂಕರೋತಿ ಸದ್ಭಕ್ತ್ಯಾ ತಸ್ಯಾಃ ಪುಣ್ಯಫಲಂ ಶೃಣು~।। ೯೮~।।}
\end{verse}

ಯಾವ ಸ್ತ್ರೀಯು ದೇವರ ಮನೆಯನ್ನು ಐದು ಬಣ್ಣಗಳಿಂದ ಸಹಿತವಾದ ರಂಗವಲ್ಲಿಯಿಂದ ಅಲಂಕರಿಸುವಳೋ ಅಂತಹ ಸ್ತ್ರೀಯ ಪುಣ್ಯಫಲವನ್ನು ಕೇಳು.

\begin{verse}
\textbf{ಆಜನ್ಮಸಾಂತಪರ್ಯಂತಂ ವಿಧವಾತ್ವಂ ನ ಚಾಪ್ನುಯಾತ್~।}\\\textbf{ನಾಪುತ್ರಾ ಜಾಯತೇ ಕ್ವಾಪಿ ನಾಧರ್ಮೇ ಜಾಯತೇ ಮತಿಃ~।। ೯೯~।।}
\end{verse}

ಅಂತಹ ಸ್ತ್ರೀಯು ಎಂದಿಗೂ ವಿಧವೆಯಾಗಲೀ, ಸಂತಾನಹೀನಳಾಗಲೀ ಆಗುವುದಿಲ್ಲ; ಅವಳ ಮನಸ್ಸು ಅಧರ್ಮಾಚರಣೆಯಲ್ಲಿ ಎಂದಿಗೂ ಆಸಕ್ತವಾಗುವುದಿಲ್ಲ.

\begin{verse}
\textbf{ಮಾತಾಪಿತ್ರೋಃ ಕುಲಂ ಭರ್ತುಃ ಸಮುದ್ಧೃತ್ಯ ಸುಭಾಗಿನಿ~।}\\\textbf{ಸರ್ವಭೋಗಸಮಾಯುಕ್ತಾ ವಿಷ್ಣೋರ್ಲೋಕೇ ಮಹೀಯತೇ~।। ೧೦೦~।।}
\end{verse}

ತನ್ನ ಮಾತಾಪಿತೃಗಳ ಹಾಗೂ ಪತಿಯ ವಂಶಗಳನ್ನು ಉದ್ಧರಿಸಿ ತಾನು ಸಮಸ್ತಭೋಗಗಳಿಂದ ಯುಕ್ತಳಾಗಿ ವಿಷ್ಣು ಲೋಕದಲ್ಲಿ ಗೌರವಿಸಲ್ಪಡುತ್ತಾಳೆ.

\begin{verse}
\textbf{ಅಲಂಕರೋತಿ ಪೂಜಾಯಾಃ ಸ್ಥಲಂ ದೇವಸ್ಯ ಚಕ್ರಿಣಃ~।}\\\textbf{ಸದಾ ಧಾತುವಿಕಾರೈಶ್ಚ ಶಂಖೈಃ ಪದ್ಮೈಶ್ಚ ಸ್ವಸ್ತಿಕೈಃ~।। ೧೦೧~।।}\\\textbf{ತಸ್ಯಾಃ ಪುಣ್ಯಫಲಂ ವಕ್ತುಂ ನ ಶಕ್ತೋ ದಿವಿ ವಾ ಭುವಿ~।।}
\end{verse}

ಶ‍್ರೀಹರಿಯ ಪೂಜಾಸ್ಥಳವನ್ನು ನಾನಾವಿಧವಾದ ರೀತಿಯಿಂದಲೂ, ಶಂಖ ಪದ್ಮ ಮುಂತಾದ ಚಿಹ್ನೆಗಳಿಂದಲೂ ರಂಗವಲ್ಲಿಯಿಂದ ಅಲಂಕಾರಮಾಡುವ ಸ್ತ್ರೀಯ ಪುಣ್ಯವನ್ನು ಹೇಳಲು ಸ್ವರ್ಗದಲ್ಲಿಯಾಗಲೀ ಭೂಲೋಕದಲ್ಲಿಯಾಗಲೀ ಯಾರೂ ಶಕ್ತರಿಲ್ಲ.

\begin{verse}
\textbf{ಏವಂ ಸಂಪೂಜ್ಯ ದೇವೇಶಂ ಕಥಾಂ ಶ್ರುತ್ವಾ ಹರೇರಿಮಾಮ್~।। ೧೦೨~।।}\\\textbf{ಏವಂ ಸ್ನಾತ್ವಾ ಮಾಸಮೇಕಂ ಸಮ್ಯಗಭ್ಯರ್ಚ್ಯ ಮಾಧವಮ್~।}\\\textbf{ಇಮಾಃ ಕಥಾಸ್ತಥಾ ಶ್ರುತ್ವಾ ನೋ ಗರ್ಭೇ ವಸತಿ ಧ್ರುವಮ್~।। ೧೦೩~।।}
\end{verse}

ಈ ರೀತಿಯಲ್ಲಿ ಮಾಘಮಾಸ ಒಂದು ತಿಂಗಳಿನಲ್ಲಿ ಮೇಲೆ ಹೇಳಿದಂತೆ ಸ್ನಾನ ಮಾಡಿ, ಪರಮಾತ್ಮನನ್ನು ಪೂಜಿಸಿ ಭಕ್ತಿಯಿಂದ ಮಾಘಮಾಸ ಮಾಹಾತ್ಮ್ಯೆ ಕಥೆಯನ್ನು ಶ್ರವಣಮಾಡಿದರೆ ಪುನರ್ಜನ್ಮವೇ ಇರುವುದಿಲ್ಲ.

\begin{verse}
\textbf{ಸ್ನಾನಂ ಸಂಧ್ಯಾಂ ಸರ್ವಕರ್ಮ ಪರಿತ್ಯಜ್ಯ ಹರೇಃ ಕಥಾಮ್~।}\\\textbf{ಶೃಣೋತಿ ಬ್ರಹ್ಮಸಂಪನ್ನಃ ಕರ್ಮಪಾಶಾದ್ವಿಮುಚ್ಯತೇ~।। ೧೦೪~।।}
\end{verse}

ಸ್ನಾನ, ಸಂಧ್ಯಾವಂದನಾದಿ ನಿತ್ಯಕರ್ಮಗಳನ್ನು ಕಾಲ ಕ್ಕೆ ಸರಿಯಾಗಿ ಮಾಡಲಿಕ್ಕೆ ಸಾಧ್ಯವಾಗದಿದ್ದರೂ ಚಿಂತೆಯಿಲ್ಲ; ಪರಮಾತ್ಮನ ಮಹಿಮೆಯನ್ನು ಶ್ರವಣಮಾಡುವವನು ಶ‍್ರೀಹರಿಯಲ್ಲಿ ಕ್ರಮೇಣ ಭಕ್ತಿಯನ್ನು ಹೊಂದಿ ಕರ್ಮಪಾಶದಿಂದ ಬಿಡುಗಡೆ ಹೊಂದುತ್ತಾನೆ.

\begin{verse}
\textbf{ಖಡ್ಗೋ ಯಥಾ ಶಾಣನಿಶಾತಮುಗ್ರಂ}\\\textbf{ಘೋರೈಃ ಪ್ರಯುಕ್ತಂ ಖಲು ಹಂತಿ ಶತೄನ್~। }\\\textbf{ಏವಂ ಕಥಾ ಶಾಣನಿಶಾತಿತಂ ಮನೋ} \\\textbf{ಹಿನಸ್ತಿ ಕಾಮಾದ್ಯ ರಿಷಟ್ಕವರ್ಗಮ್~।। ೧೦೫~।।}
\end{verse}

ಹರಿತವಾದ ಖಡ್ಗವು ಶತ್ರುಗಳನ್ನು ನಾಶಮಾಡುವಂತೆ, ಶ‍್ರೀಹರಿಯ ಮಹಾ ಮಹಿಮೋಪೇತವಾದ ಕಥಾಶ್ರವಣವು ಕಾಮಾದಿ ಅರಿಷಡ್ವರ್ಗವನ್ನು ಧ್ವಂಸ ಮಾಡುತ್ತದೆ.

\begin{verse}
\textbf{ಕಥಾನಿಮಿತ್ತಂ ಯದಿ ಕರ್ಮಲೋಪೋ}\\\textbf{ಭವೇಜ್ಜನೇ ಭಾಗವತೇ ಹಿ ಲೋಕೇ~। }\\\textbf{ಸ ಕರ್ಮಲೋಪೋ ನ ಭವೇನ್ಮದೀಯೇ~।} \\\textbf{ಜನೇ ವಿಚಾರ್ಯೈವಮನಿಂದ್ಯ ಕರ್ಮ~।। ೧೦೬~।।}
\end{verse}

ಪರಮಾತ್ಮನ ಕಥಾಶ್ರವಣ ಕಾರಣದಿಂದ ನಿತ್ಯಕರ್ಮಾನುಷ್ಠಾನದಲ್ಲಿ ನನ್ನ ಭಕ್ತರಿಗೆ ಲೋಪವೇನಾದರೂ ಆದರೆ ಅವರಿಗೆ ಅಂತಹ ದೋಷವು ಇರುವುದಿಲ್ಲ.

\begin{verse}
\textbf{ಕೊಟಿಸ್ತು ತಿಸ್ರಃ ಕೃಪಯಾ ಮುನೀನಾಂ}\\\textbf{ಸಮಾದಿದೇಶಾತ್ಮಭುವೋದ್ಭವಾನಾಮ್~।}\\\textbf{ಯೂಯಂ ಮದೀಯಸ್ಯ ಜನಸ್ಯ ಕರ್ಮ} \\\textbf{ಕೃತ್ವಾನಿಶಂ ಪೂರಯತಾಶು ಕಲ್ಪಮ್~।। ೧೦೭~।।}
\end{verse}

ಹೀಗೆ ಕಥಾಶ್ರವಣಯುಕ್ತರಾದ ಭಕ್ತರ ಸತ್ಕರ್ಮಗಳನ್ನು ಕಾಲಕ್ಕೆ ಸರಿಯಾಗಿ ಮಾಡಲು ಪರಮಾತ್ಮನು ಕೃಪೆಯಿಂದ ಮೂರು ಕೋಟಿ ಋಷಿಗಳಿಗೆ ಆಜ್ಞೆ ಮಾಡಿರುತ್ತಾನೆ.

\begin{verse}
\textbf{ಇತ್ಯಾದಿಷ್ಟಾ ಭಗವತಾ ಋಷಯಃ ಶಂಸಿತವ್ರತಾಃ~।}\\\textbf{ಸದಾ ವಿಷ್ಣು ಕಥಾಸಕ್ತ ಚೇತಸಾಂ ಭಾವಿತಾತ್ಮನಾಮ್~।। ೧೦೮~।। }\\\textbf{ಕರ್ಮಲೋಪಂ ಪೂರಯಂತಿ ಕೃತ್ವಾ ಕೃತ್ವಾಪ್ಯ ಹರ್ನಿಶಮ್~।}
\end{verse}

ಹೀಗೆ ಆಜ್ಞಾಪಿಸಲ್ಪಟ್ಟ ಋಷಿಗಳು ನಿತ್ಯದಲ್ಲಿಯೂ ಹರಿಕಥಾಶ್ರವಣದಲ್ಲಿ ಆಸಕ್ತಿ ಹೊಂದಿರುವ ಭಕ್ತರ ಸಮಸ್ತ ಸತ್ಕರ್ಮಗಳನ್ನೂ ಆಚರಿಸಿ ಅವರ ಕರ್ಮ ಲೋಪಗಳನ್ನು ತುಂಬಿಕೊಡುತ್ತಾರೆ.

\begin{verse}
\textbf{ಸ್ನಾತ್ವಾ ವಿಷ್ಣು ಕಥಾಂ ಶ್ರುತ್ವಾ ತಥಾ ಸಂಪೂಜ್ಯ ಮಾಧವಮ್~।}\\\textbf{ಸರ್ವಬಂಧವಿನಿರ್ಮುಕ್ತೋ ವಿಷ್ಣೋಃ ಸಾಯುಜ್ಯಮಾಪ್ನುಯಾತ್~।। ೧೦೯~।।}
\end{verse}

ಮಾಘಸ್ನಾನವನ್ನು ಮಾಡಿ, ಶ‍್ರೀ ಹರಿ ಪೂಜೆಯನ್ನು ನೆರವೇರಿಸಿ, ಪರಮಾತ್ಮನ ಕಥೆಯನ್ನು ಕೇಳುವವನು ಎಲ್ಲ ಕರ್ಮಬಂಧನದಿಂದ ಬಿಡುಗಡೆ ಹೊಂದಿ ವಿಷ್ಣುವಿನ ಸಾಯುಜ್ಯಮೋಕ್ಷವನ್ನು ಪಡೆಯುವನು.

\begin{verse}
\textbf{ಆದಾವಂತೇ ತಥಾ ಮಧ್ಯೇ ಯಥಾಶಕ್ತಿ ಯಥಾವಿಧಿ~।}\\\textbf{ಮಾಧವಪ್ರತಿಮಾಂ ದದ್ಯಾತ್ ವ್ರತಸ್ಯಾಸ್ಯ ಸಮಾಪ್ತಯೇ~।। ೧೧೦~।।}
\end{verse}

ಈ ವ್ರತದ ಆದಿಯಲ್ಲಿಯೂ, ಮಧ್ಯದಲ್ಲಿಯೂ ಮತ್ತು ಕಡೆಯಲ್ಲಿಯೂ ವ್ರತದ ಸಮಾಪ್ತಿಗೋಸ್ಕರ ಮಾಧವನ ಪ್ರತಿಮೆಯನ್ನು ದಾನಮಾಡಬೇಕು.

\begin{verse}
\textbf{ಪಂಚಾಶದ್ದ್ವಾದಶಂ ವಾಪಿ ಬ್ರಾಹ್ಮಣಾನ್ ಭೋಜಯೇತ್ತತಃ~।। ೧೧೧~।।}
\end{verse}

ಐವತ್ತು ಅಥವಾ ಹನ್ನೆರಡು ಜನ ಬ್ರಾಹ್ಮಣರಿಗೆ ಭೋಜನಮಾಡಿಸಬೇಕು.

\begin{verse}
\textbf{ಏತಚ್ಚಾಸ್ತ್ರಪ್ರವಕ್ತಾರಂ ಯಥಾವಿಭವಮರ್ಚಯೇತ್~।}\\\textbf{ಅಪೂಜಯಿತ್ವಾ ವಕ್ತಾರಂ ಯಃ ಶೃಣೋತಿ ಕಥಾಮಿಮಾಮ್~।। ೧೧೨~।। }\\\textbf{ಇಕ್ಷುಯಂತ್ರೇ ಚಿರಂ ಸ್ಥಿತ್ವಾ ಮುನೇ ವ್ಯಾಲೋ ಹಿ ಜಾಯತೇ~।। ೧೧೩~।।}
\end{verse}

ಈ ಮಾಘಮಾಸ ಮಾಹಾತ್ಮ್ಯೆಯನ್ನು ಶ್ರವಣಮಾಡಿಸುವ ದ್ವಿಜನಿಗೆ ಯಥಾ ಯೋಗ್ಯವಾಗಿ ಪೂಜೆ ಮಾಡಬೇಕು. ಹಾಗೆ ಪೂಜಿಸದೇ ಕಥಾಶ್ರವಣವನ್ನು ಮಾಡಿದರೆ, ಅಂತಹವನು ನರಕದಲ್ಲಿ ಕಬ್ಬಿನಗಾಣದಲ್ಲಿ ಬಹುಕಾಲ ಅರೆಯುಲ್ಪಟ್ಟು ನಂತರ ಸರ್ಪನಾಗಿ ಹುಟ್ಟುತ್ತಾನೆ.

\begin{verse}
\textbf{ನ ಶೃಣೋತಿ ಕಥಾಂ ಯಸ್ತು ವಾಚ್ಯಮಾನಮಿಮಾಂ ದ್ವಿಜ~।}\\\textbf{ಅರಣ್ಯೇ ನಿರ್ಜಲೇ ದೇಶೇ ಭವತಿ ಬ್ರಹ್ಮರಾಕ್ಷಸಃ~।। ೧೧೪~।।}
\end{verse}

ಈ ಕಥೆಯು ಪ್ರವಚನಮಾಡಲ್ಪಡುತ್ತಿದ್ದಾಗ ಯಾರು ಶ್ರವಣಮಾಡುವುದಿಲ್ಲವೋ ಅಂತಹವನು ಅರಣ್ಯದಲ್ಲಿ ನೀರಿಲ್ಲದ ಸ್ಥಳದಲ್ಲಿ ಬ್ರಹ್ಮರಾಕ್ಷಸನಾಗಿ ಹುಟ್ಟುತ್ತಾನೆ.~।

\begin{verse}
\textbf{ಧರ್ಮಶಾಸ್ತ್ರ ಪ್ರವಕ್ತಾರಂ ಶಾಂತಂ ದಾಂತಂ ಜಿತಾಸನಮ್~।}\\\textbf{ಮಿಷ್ಟಾನ್ನೈರ್ಭೋಜಯೇದ್ಯಸ್ತು ಸರ್ವಪಾಪೈರ್ನ ಲಿಪ್ಯತೇ~।। ೧೧೫~।।}
\end{verse}

ಶಾಂತನಾಗಿದ್ದು, ಇಂದ್ರಿಯನಿಗ್ರಹ ಹೊಂದಿದ್ದು, ಧರ್ಮಶಾಸ್ತ್ರವನ್ನು ಉಪದೇಶ ಮಾಡುವ ಬ್ರಾಹ್ಮಣನನ್ನು ಮೃಷ್ಟಾನ್ನದಿಂದ ಭೋಜನಮಾಡಿಸಿದರೆ ಯಾವ ಪಾಪವೂ ಲೇಪವಾಗುವುದಿಲ್ಲ.

\begin{verse}
\textbf{ಅನೇನ ವಿಧಿನಾ ಯಸ್ತು ಮಾಘಸ್ನಾನಂ ಕರೋತಿ ಯಃ~।}\\\textbf{ತಸ್ಯ ಪುಣ್ಯಫಲಂ ವಕ್ತುಂ ವಿಷ್ಣು ರ್ಬ್ರಹ್ಮಾ ಶಿವೋ ಯಮಃ~।। }\\\textbf{ನೇಂದ್ರೋ ನ ಸೂರ್ಯೋ ವರುಣೋ ಮನ್ವಾದ್ಯಾಃ ಕಿಮುತಾಪರೇ~।।}
\end{verse}

ಮೇಲೆ ಹೇಳಿದ ರೀತಿಯಿಂದ ಮಾಘಸ್ನಾನವ್ರತವನ್ನು ಆಚರಿಸುವವನ ಪುಣ್ಯ ಫಲವನ್ನು ವಿವರಿಸಲು ವಿಷ್ಣು, ಬ್ರಹ್ಮ, ರುದ್ರ, ಯಮ, ಇಂದ್ರ, ಸೂರ್ಯ, ವರುಣ, ಮನುವೇ ಮೊದಲಾದವರೂ ಸಹ ಶಕ್ತರಲ್ಲ. ಬಾಕಿಯವರ ವಿಷಯ ಹೇಳತಕ್ಕದ್ದೇನಿದೆ?

\begin{verse}
\textbf{ಮಾಘಸ್ನಾನಂ ಸಮುದ್ದಿಶ್ಯ ಬಹಿರ್ಗಚ್ಛೇದ್ಯದಾ ನರಃ~।}\\\textbf{ಪದೇ ಪದೇಽಶ್ವಮೇಧಸ್ಯ ಫಲಂ ಪ್ರಾಪ್ನೋತ್ಯ ಸಂಶಯಃ~।। ೧೧೭~।।}
\end{verse}

ಮಾಘಸ್ನಾನ ಮಾಡಬೇಕೆಂಬ ಉದ್ದೇಶದಿಂದ ಜಲಾಶಯದ ಕಡೆಗೆ ಹೋಗುವವನು ಹೆಜ್ಜೆ ಹೆಜ್ಜೆಗೂ ಅಶ್ವಮೇಧ ಯಾಗಫಲವನ್ನು ನಿಸ್ಸಂಶಯವಾಗಿ ಹೊಂದುತ್ತಾನೆ.

\begin{verse}
\textbf{ಅತೀವ ಪ್ರತಿಬಂಧಾಶ್ಚ ನಶ್ಯಂತಿ ಸ್ನಾನಮಾತ್ರತಃ~।}\\\textbf{ಯಥಾ ಸುಧರ್ಮಸ್ಯ ಸುತಾ ಕರ್ಮಬಂಧಾನ್ಮುಮೋಚ ಹ~।। ೧೧೮~।।}
\end{verse}

ಹಿಂದೆ ಸುಧರ್ಮನ ಪುತ್ರಿಯು ಯಾವ ರೀತಿಯಲ್ಲಿ ಕರ್ಮಬಂಧನದಿಂದ ಬಿಡುಗಡೆಯನ್ನು ಹೊಂದಿದಳೋ ಹಾಗೆ ಮಾಘಸ್ನಾನಮಾತ್ರದಿಂದ ಸಮಸ್ಯ ಬಂಧನಗಳು ನಾಶವಾಗುತ್ತವೆ.

\begin{center}
ಇತಿ ಶ‍್ರೀ ವಾಯುಪುರಾಣೇ ಮಾಘಮಾಸ ಮಾಹಾತ್ಮ್ಯೇ, ದ್ವಿತೀಯೋsಧ್ಯಾಯಃ~।
\end{center}

\begin{center}
ಶ‍್ರೀ ವಾಯುಪುರಾಣಾಂತರ್ಗತ ಮಾಘಮಾಸ ಮಾಹಾತ್ಮ್ಯೆಯಲ್ಲಿ \\ ಎರಡನೇ ಅಧ್ಯಾಯವು ಸಮಾಪ್ತಿಯಾಯಿತು.
\end{center}

\newpage

\section*{ಅಧ್ಯಾಯ\enginline{-}೩}

\emptypage

\begin{flushleft}
\textbf{ನಾರದ ಉವಾಚ\enginline{-}}
\end{flushleft}

\begin{verse}
\textbf{ಕಥಂ ಧರ್ಮಸ್ಯ ಸುತಾ ಸಾ ಕರ್ಮಬಂಧಾನ್ಮುಮೋಚ ಹ~।}\\\textbf{ಸ ಕರ್ಮಬಂಧಃ ಕೋವಾಸೀತ್ಕೇನಾಸೌ ನಾಶಮಾಗತಃ~।। ೧~।।}
\end{verse}

\begin{flushleft}
ನಾರದರು ಹೇಳುತ್ತಾರೆ-
\end{flushleft}

ಪಿತಾಮಹನೆ, ಸುಧರ್ಮನ ಮಗಳು ಯಾವ ಕರ್ಮದಿಂದ ಬಂಧಿತಳಾಗಿದ್ದಳು, ಯಾವ ರೀತಿಯಿಂದ ಬಿಡುಗಡೆ ಹೊಂದಿದಳು?

\begin{verse}
\textbf{ಶ್ರೋತುಂ ಕೌತೂಹಲಂ ಬ್ರಹ್ಮಂಸ್ತಸ್ಮಾದ್ವಿಸ್ತರತೋ ವದ~।।}
\end{verse}

ಇದನ್ನು ಕೇಳಲು ಕುತೂಹಲವಾಗಿದೆ; ವಿಸ್ತಾರವಾಗಿ ನಿರೂಪಿಸಿರಿ.

\begin{flushleft}
\textbf{ಬ್ರಹ್ಮೋವಾಚ\enginline{-}}
\end{flushleft}

\begin{verse}
\textbf{ಪುರಾ ಗಯಸ್ಯ ದೌಹಿತ್ರಃ ಸುಧರ್ಮಾ ನಾಮ ಬಾಹುಜಃ~।}\\\textbf{ಕುಂಡಿನೇ ನಗರೇ ರಮ್ಯೇ ವಿದರ್ಭಾನ್ಪರ್ಯಪಾಲಯತ್~।। ೨~।। }\\\textbf{ತಸ್ಯ ಕಾಲೇನ ದುಹಿತಾ ನಾಮ್ನಾ ರೋಚಿಷ್ಮತೀ ಸತೀ~।। ೩~।।} \\\textbf{ವಿದರ್ಭೋ ನಾಮ ತನಯೋ ವೈದರ್ಭೋಪ್ಯಭವತ್ತ ತಃ~।} \\\textbf{ಸುಧರ್ಮಸ್ತನಯಾಂ ಪ್ರಾದಾದ್ರಾಜ್ಞೋ ಮಾಹಿಷ್ಮತೀಪತೇಃ} \\\textbf{ಷಷ್ಠೇಽಹನಿ ವಿವಾಹಸ್ಯ ಮೃತೋ ಮಾಹಿಷ್ಮತೀಪತಿಃ~।। ೪~।।}
\end{verse}

\begin{flushleft}
ಬ್ರಹ್ಮದೇವರು ಹೇಳಿದರು-
\end{flushleft}

ಹಿಂದೆ ಗಯನೆಂಬ ಕ್ಷತ್ರಿಯರಾಜನಿದ್ದನು. ಅವನಿಗೆ ಪರಾಕ್ರಮಿಯಾದ ಸುಧರ್ಮನೆಂಬ ಮಗನಿದ್ದನು. ಸುಧರ್ಮನು ಕುಂಡಿನೀ ನಗರದಲ್ಲಿದ್ದು ವಿದರ್ಭದೇಶವನ್ನು ನ್ಯಾಯದಿಂದ ಆಳುತ್ತಿದ್ದನು. ಕೆಲವು ಕಾಲದ ನಂತರ ಅವನಿಗೆ ರೋಚಿಷ್ಮತೀ ಎಂಬ ಕನ್ಯೆಯೂ ವಿದರ್ಭನೆಂಬ ಮಗನೂ ಜನಿಸಿದರು. ವಿದರ್ಭ ದೇಶದ ರಾಜನ ಮಗಳಾದುದರಿಂದ ರೋಚಿಷ್ಮತಿಗೆ ವೈದರ್ಭೀ ಎಂದು ಹೆಸರಾಯಿತು. ಸುಧರ್ಮನು ತನ್ನ ಮಗಳನ್ನು ಮಾಹಿಷ್ಮತೀ ನಗರದ ರಾಜನಿಗೆ ಕೊಟ್ಟು ಲಗ್ನಮಾಡಿದನು. ಆದರೆ ಮದುವೆಯಾದ ಆರನೇ ದಿನ ಆ ಮಾಹಿಷ್ಮತೀ ರಾಜನು ಮೃತನಾದನು.

\begin{verse}
\textbf{ಭರ್ತಾರಮಾಭಾಷಯಿತುಂ ಪ್ರವೃತ್ತಾ} \\\textbf{ರೋಚಿಷ್ಮತೀ ರಂತುಮನಾತಿಕಾಮಮ್~।}\\\textbf{ಶಯ್ಯಾಗೃಹೇ ಸುಪ್ತಮಭಾಷಮಾಣಂ }\\\textbf{ಜ್ಞಾತ್ವಾ ಮೃತಂ ಸಾಥ ರುರೋದ ಚೋಚ್ಚೈಃ~।। ೫~।।}
\end{verse}

ಶಯ್ಯಾಗೃಹದಲ್ಲಿ ಪತಿಯೊಡನೆ ಕ್ರೀಡಿಸಲು ರೋಚಿಷ್ಮತಿಯು ತೊಡಗಿದಳು. ಮಾತನ್ನೇ ಆಡದೆ ಮಲಗಿದ್ದ ಪತಿಯನ್ನು ಕಂಡು ಪತಿಯು ಸತ್ತು ಹೋದನೆಂದು ತಿಳಿದು ಗಟ್ಟಿಯಾಗಿ ಅತ್ತಳು.

\begin{verse}
\textbf{ಶ್ರುತ್ವಾ ಸಖೀಜನಸ್ತಸ್ಯಾಃ ಪ್ರವಿಷ್ವೋಽಂತಃಪುರಂ ಮೃತಮ್~।}\\\textbf{ಉಪಾಯೈರ್ಬಹುರ್ಭಿಜ್ಞಾತ್ವಾ ರುರೋದೋಚ್ಚೈರಹರ್ನಿಶಮ್~।। ೬~।।}\\\textbf{ಸಂಶ್ರಾವ್ಯ ತತ್ಪಿತಾ ಕ್ರಂದಂ ತತ್ಸಖೀಭ್ಯೋ ನಿವೇದಿತಮ್~।। ೭~।।}
\end{verse}

ಈ ರೋದನವನ್ನು ಕೇಳಿದ ಸಖೀಜನರು ಅಂತಃಪುರವನ್ನು ಪ್ರವೇಶಿಸಿ ಮೃತನಿಗೆ ನಾನಾ ಬಗೆಯ ಉಪಚಾರಗಳನ್ನು ಮಾಡಿದರು. ರೋಚಿಷ್ಮತಿಯು ಒಂದೇ ಸಮನೆ ಅಳುತ್ತಿರುವುದನ್ನು ಸಖೀಜನರಿಂದ ತಿಳಿದ ಅವಳ ತಂದೆಯಾದ ಸುಧರ್ಮನು,

\begin{verse}
\textbf{ಸ ಶೋಕಾನಲಸಂತಪ್ತೋ ಮೂರ್ಛಾಮಾಪ ಮೃತೋಪಮಾಮ್~।}
\end{verse}

ದುಃಖವೆಂಬ ಬೆಂಕಿಯಿಂದ ತಪ್ತನಾಗಿ ಮೃತಪ್ರಾಯನಂತೆ ಮೂರ್ಛೆಯನ್ನು ಹೊಂದಿದನು.

\begin{verse}
\textbf{ಮೂರ್ಛಿತಂ ತಂ ಪತಿಂ ದೃಷ್ಟ್ವಾ ಮಾತರಶ್ಚೋಪಮಾತರಃ~।}\\\textbf{ಮೂರ್ಛಾಮಾಪುರ್ಮಹಾದುಃಖಾ ರುರುದುಃ ಪುರವರ್ತಿನಃ~।। ೮~।।}
\end{verse}

ತನ್ನ ಪತಿಯು ಮೂರ್ಛಿತನಾದುದನ್ನು ಕಂಡು ರೋಚಿಷ್ಮತಿಯ ತಾಯಿಯೂ ಇತರ ಉಪತಾಯಿಯರೂ ಸಹ ಮೂರ್ಛೆಯನ್ನು ಹೊಂದಿದರು. ಇತರ ಪುರಜನರೂ ಸಹ ದುಃಖದಿಂದ ಅತ್ತರು.

\begin{verse}
\textbf{ಅಕಸ್ಮಾದಂಗಿರಾನಾಮ ಯಜ್ಞಾರ್ಥೇ ದ್ರವಿಣಂ ಪ್ರತಿ~।}\\\textbf{ಅಭ್ಯಾಸಂ ನೃಪತೇಃ ಪ್ರಾಗಾದ್ದೃಷ್ಟ್ವಾವಸ್ಥಾಂ ತದಾಗತಾಮ್~।। ೯~।। }\\\textbf{ದಯಾಲುಃ ಸ ಮುನಿಃ ಪ್ರಾಹ ನೃಪತಿಂ ಶೋಕಲಾಲಸಮ್~।।}
\end{verse}

ಯಜ್ಞಕ್ಕಾಗಿ ಹಣ ಸಂಗ್ರಹಿಸಲು ಅಂಗೀರ ಋಷಿಗಳು ಅಲ್ಲಿಗೆ ಬಂದು ರಾಜನು ದುಃಖಿತನಾಗಿರುವುದನ್ನು ಕಂಡರು. ದಯಾಳುಗಳಾದ ಆ ಋಷಿಗಳು ಶೋಕತಪ್ತನಾದ ರಾಜನಿಗೆ ಹೇಳಿದರು.

\begin{verse}
\textbf{ಧ್ಯಾತ್ವಾ ಚ ಸುಚಿರಂ ಕಾಲಂ ಜ್ಞಾತ್ವಾ ತನ್ಮೃತ್ಯು ಕಾರಣಮ್~।}\\\textbf{ಉಜ್ಜೀವಯಾಮಿ ನೃಪತೇ ಮೃತಂ ಜಾಮಾತರಂ ತವ~।। ೧೦~।।}
\end{verse}

ಸ್ವಲ್ಪ ಕಾಲ ಧ್ಯಾನಮಾಡಿ ಮೃತ್ಯುವಿಗೆ ಕಾರಣವನ್ನು ತಿಳಿದು “ರಾಜನೇ ನಿನ್ನ ಅಳಿಯನನ್ನು ಬದುಕಿಸುತ್ತೇನೆ.”

\begin{verse}
\textbf{ನ ಕಾಲೇ ಮೃತ್ಯು ರಸ್ಯಾಸೀನ್ಮಾ ಶೋಕಂ ಕರ್ತುಮರ್ಹಸಿ~।}\\\textbf{ಇತ್ಯಾಶ್ವಾಸ್ಯ ನೃಪಂ ಪ್ರಾಜ್ಞಃ ಪ್ರಾದಾತ್ತಸ್ಯ ಮೃತಸ್ಯ ಚ~।। ೧೧~।। }\\\textbf{ಮಾಘೇ ಮಾಸಿ ಕೃತಸ್ನಾನಫಲಂ ಚ ತ್ರಿದಿನಸ್ಯ ಚ~।।}
\end{verse}

ಈತನು ಕಾಲಮೃತ್ಯುವಿನಿಂದ ಸತ್ತಿಲ್ಲವಾದುದರಿಂದ ಶೋಕಪಡಲು ಕಾರಣವಿಲ್ಲ.\break ಹೀಗೆಂದು ಹೇಳಿ ಋಷಿಗಳು ತಾವು ಆಚರಿಸಿದ್ದ ಮಾಘ ಸ್ನಾನದ ಫಲದಲ್ಲಿ ಮೂರು ದಿನಗಳ ಸ್ನಾನದ ಪುಣ್ಯವನ್ನು ಮೃತನಿಗೆ ಕೊಟ್ಟರು.

\begin{verse}
\textbf{ದಿನಸ್ಯೈಕಸ್ಯ ಚ ಫಲಂ ಜಾಮಾತುಃ ಪರ್ಯಕಲ್ಪಯತ್~।}\\\textbf{ದಿನದ್ವಯಫಲಂ ತಸ್ಯ ಪಶ್ಚಾತ್ ಜಪ್ತ್ವಾ ಮಹಾಮನುಮ್~।। ೧೨~।। }\\\textbf{ಮುಂಚಾಮಿ ತ್ವೇತಿ ಋಷಯಸ್ತಂ ಭೂಪಮುದಜೀವಯತ್~।।}
\end{verse}

ಒಂದು ದಿವಸದ ಫಲವನ್ನು ಅಳಿಯನಿಗೂ, ಎರಡು ದಿವಸಗಳ ಸ್ನಾನದ ಫಲವನ್ನು ರೋಚಿಷ್ಮತಿಗೂ ಧಾರೆಯೆರದು ಶ್ರೇಷ್ಠವಾದ ಮಂತ್ರವನ್ನು ಜಪಿಸಿ ಅಳಿಯನನ್ನು ಬದುಕಿಸಿದರು.

\begin{verse}
\textbf{ತದಾ ಸುಧರ್ಮಃ ಸಂತುಷ್ಟಃ ಪಪ್ರಚ್ಛಾಂಗಿರಸಂ ಮುನೀಮ್~।। ೧೩~।।}\\\textbf{ಕೇನ ಕರ್ಮವಿಪಾಕೇನ ಜಾಮಾತಾ ಮೇ ಮೃತಿಂ ಗತಃ~।}
\end{verse}

ಸಂತೋಷಗೊಂಡ ಸುಧರ್ಮನು ಅಂಗೀರಸ ಮುನಿಯನ್ನು ಈ ರೀತಿ ಪ್ರಶ್ನಿಸಿದನು- ಯಾವ ಕರ್ಮ ವಿಶೇಷದಿಂದ ನನ್ನ ಅಳಿಯನು ಸತ್ತನು?

\begin{verse}
\textbf{ವೈಧವ್ಯಂ ದುಹಿತುಶ್ಚಾಪಿ ಕೇನ ಕರ್ಮಫಲೇನ ತು~।। ೧೪~।।}\\\textbf{ಪರಿಹರ್ತಾಸಿ ವೈಧವ್ಯಂ ದುಹಿತುಃ ಕೇನ ಕರ್ಮಣಾ~।}\\\textbf{ಪುನರುಜ್ಜೀವಿತಃ ಕೇನ ಜಾಮಾತಾದ್ಯಾನಘ ದ್ವಯಮ್~।। ೧೫~।।}\\\textbf{ಮಹತ್ಕೌತೂಹಲಂ ಶ್ರೋತುಮೇತದ್ವಿಸ್ತರತೋ ವದ~।। ೧೬~।।}
\end{verse}

ನನ್ನ ಮಗಳಿಗೆ ವೈಧವ್ಯವು ಯಾವ ಕರ್ಮದಿಂದ ಬಂದಿತು, ಯಾವ ಪುಣ್ಯ ಪ್ರಭಾವದಿಂದ ಆ ವೈಧವ್ಯವು ಹೋಯಿತು ಮತ್ತು ನನ್ನ ಅಳಿಯನು ಬದುಕಿದನು? ಈ ಕುತೂಹಲಕಾರಿಯಾದ ವಿಷಯವನ್ನು ಕೇಳಲು ಆಸೆಯಾಗಿದೆ, ವಿಸ್ತಾರವಾಗಿ ನಿರೂಪಿಸಿರಿ ಎಂದನು.

\newpage

\begin{flushleft}
\textbf{ಖುಷಿರುವಾಚ\enginline{-}}
\end{flushleft}

\begin{verse}
\textbf{ಗೌಡದೇಶೇ ಪುರಾ ಕಶ್ಚಿದ್ವಿಪ್ರೋಽಭೂನ್ಮೇಘನಾಮಕಃ~।}\\\textbf{ತಸ್ಯೇಯಂ ದುಹಿತಾ ಚಾಸೀತ್ ಸುರೂಪಾ ನಾಮ ರೂಪಿಣೀ~।। ೧೭~।।}
\end{verse}

ಹಿಂದೆ ಗೌಡದೇಶದಲ್ಲಿ ಮೇಘನೆಂಬ ಒಬ್ಬ ಬ್ರಾಹ್ಮಣನಿದ್ದನು. ಈಗ ನಿನ್ನ ಮಗಳಾಗಿರುವವಳು ಆಗ ಮೇಘನ ಮಗಳಾಗಿದ್ದಳು. ಆಗ ಅವಳ ಹೆಸರು ಸುರೂಪಾ ಎಂಬುದಾಗಿತ್ತು. ಅವಳೂ ಸಹ ಸುಂದರಳಾಗಿದ್ದಳು.

\begin{verse}
\textbf{ಸಮಾನಕುಲಗೋತ್ರಾಯ ವಶ್ಯೋಽದಾತ್ಕನ್ಯಕಾಂ ಸ್ವಕಾಮ್~।}\\\textbf{ಅಥ ಕಾಲೇನ ಕಿಯತಾ ವಯಸ್ಥಾ ಸಾ ಸುರೂಪಿಣೀ~।। ೧೮~।।}
\end{verse}

ಮೇಘನು ಅವಳನ್ನು ತನ್ನ ಅಂತಸ್ತಿಗೆ ಸರಿಯಾದವನಿಗೆ ಕೊಟ್ಟು ಲಗ್ನ ಮಾಡಿದನು. ಕಾಲಕ್ರಮೇಣ ಅವಳು ಪ್ರೌಢಳಾದಳು.

\begin{verse}
\textbf{ಭರ್ತರಿ ಪ್ರೇಷಿತೇ ಕಂಚಿದ್ಯುವಾನಂ ಚರ್ಮಕಾರಕಮ್~।}\\\textbf{ವರಯಿತ್ವಾ ಚೋಪಪತಿಂ ರಮಮಾಣಾ ಚಿರಂ ಸ್ಥಿತಾ~।। ೧೯~।।}
\end{verse}

ತನ್ನ ಪತಿಯನ್ನು ಎಲ್ಲಿಗೋ ಕಳಿಸಿ ತಾನು ಒಬ್ಬ ಚರ್ಮದ ಕೆಲಸ ಮಾಡುವ ಯುವಕ\-ನೊಂದಿಗೆ ರಮಿಸಿ ಅವನನ್ನು ತನ್ನ ಉಪಪತಿಯನ್ನಾಗಿ ಮಾಡಿಕೊಂಡಳು.

\begin{verse}
\textbf{ಅಥ ಕಾಲಾಂತರೇ ಪತ್ಯಾವಗತೇ ಸ್ವೈರಚಾರಿಣೀ~।}\\\textbf{ಜಾರಸ್ನೇಹಾದ್ಗೃಹಂ ಭರ್ತಾ ಸಹಿತಂ ಸಾಽದಹನ್ನಿಶಿ~।। ೨೦~।।}
\end{verse}

ಕೆಲವು ಕಾಲದ ನಂತರ ಪತಿಯು ತಿರುಗಿ ಬಂದಾಗ ಈ ಸ್ವೇಚ್ಛಾಚಾರಿಣಿಯು ತನ್ನ ಜಾರನಲ್ಲಿ ವಿಶೇಷ ಪ್ರೀತಿಯನ್ನಿಟ್ಟು ತನ್ನ ಪತಿಯು ಇದ್ದ ಮನೆಯನ್ನು ಪತಿ ಸಹಿತವಾಗಿ ರಾತ್ರಿಯಲ್ಲಿ ಸುಟ್ಟು ಬಿಟ್ಟಳು.

\begin{verse}
\textbf{ಜಾರೇಣ ಸಹಿತಾ ಪಾಪಾ ದೇಶೇಷು ವಿಚಚಾರ ಹ~।}\\\textbf{ತತೋ ಮೃ ತಿಂ ಗತಾ ಕಾಲೇ ನರಕೇ ಭುಕ್ತ ಯಾತನಾ~।। ೨೧~।। }\\\textbf{ವೇಶ್ಯಾಽಭೂತ್ಕರ್ಮಶೇಷೇಣ ಚತುರ್ದಶಸು ಜನ್ಮಸು~।} \\\textbf{ಅರ್ಜಿತಂ ಪುಣ್ಯ ಮತುಲಂ ಯಥಾ ವೇಶ್ಯಾಸ್ವಜನ್ಮನಿ~।। ೨೨~।।}
\end{verse}

ಜಾರನಿಂದ ಕೂಡಿ ಆ ಪಾಪಾತ್ಮಳು ಅನೇಕ ದೇಶಗಳಲ್ಲಿ ಸಂಚರಿಸಿ, ಒಂದು ದಿವಸ ಸತ್ತು, ನರಕದಲ್ಲಿ ಯಾತನೆಯನ್ನು ಅನುಭವಿಸಿ ಆಮೇಲೆ ಹದಿನಾಲ್ಕು ಜನ್ಮಗಳಲ್ಲಿ ವೇಶ್ಯೆಯಾಗಿ ಹುಟ್ಟಿ ಬಹಳ ಪುಣ್ಯವನ್ನು ಸಂಪಾದಿಸಿದಳು.

\begin{verse}
\textbf{ತೀರ್ಥಯಾತ್ರಾಪ್ರಸಂಗೇನ ಸುಮೇಧಾನಾಮ ವೈ ದ್ವಿಜಃ~।}\\\textbf{ರಂಗಕ್ಷೇತ್ರಂ ಜಿಗಮಿಷುಃ ಇಮಂ ದೇಶಂ ಸಮಾಗತಃ~।। ೨೩~।।}
\end{verse}

ಸುಮೇಧನೆಂಬ ಬ್ರಾಹ್ಮಣನು ತೀರ್ಥಯಾತ್ರಾಪರನಾಗಿ ಶ‍್ರೀರಂಗ ಕ್ಷೇತ್ರಕ್ಕೆ ಹೋಗಬಯಸಿ ದಾರಿಯಲ್ಲಿ ಈ ವೇಶ್ಯೆ ಇರುವ ಊರಿಗೆ ಬಂದನು.

\begin{verse}
\textbf{ಸ್ಥಿತೇಯಮೂರ್ಮಿಲಾನಾಮ ಹೈಹಯಾಧಿಪತೇಃ ಪುರಿ~।}\\\textbf{ದ್ವಾರಿ ಸ್ಥಿ ತಾ ಚ ಸಾಯಾಹ್ನಿ ಕಾಮುಕಾನಭಿಕಾಂಕ್ಷತಿ~।। ೨೪~।।}
\end{verse}

ಹೈಹಯಾಧಿ ಅರಸನ ಪಟ್ಟಣದಲ್ಲಿ ಊರ್ಮಿಳಾ ಎಂಬ ಹೆಸರಿನ ವೇಶ್ಯೆಯಾಗಿದ್ದ ಅವಳು ಸಂಜೆಯಲ್ಲಿ ಜಾರಪುರುಷರನ್ನು ಎದುರುನೋಡುತ್ತ ತನ್ನ ಮನೆ ಬಾಗಿಲಿನಲ್ಲಿ ನಿಂತುಕೊಂಡಿದ್ದಳು.

\begin{verse}
\textbf{ದಿನೇ ತಸ್ಮಿನ್ನೈಕಮಪಿ ಕಾಮುಕಂ ಪ್ರತ್ಯ ಪದ್ಯತ~।}\\\textbf{ತಸ್ಮಿನ್ ರಾತ್ರೌ ಮಹಾವೃಷ್ಟಿರಭೂದೇಕಾರ್ಣವಂ ಜಗತ್~।। ೨೫~।।}
\end{verse}

ಆದರೆ ಆ ದಿನ ಒಬ್ಬ ಜಾರಪುರುಷನೂ ಸುಳಿಯಲಿಲ್ಲ. ರಾತ್ರಿಯಲ್ಲಿ ಅಗಾಧ ಮಳೆ ಬಂದಿತು.

\begin{verse}
\textbf{ತದಾ ವೃಷ್ಟಿ ಪರಿಕ್ರಾಂತಃ ಸುಮೇಧಾನಾಮ ವೈ ದ್ವಿಜಃ~।}\\\textbf{ದ್ವಾರಸ್ಥಶಾಲಾವಲಭೇರಧೋದೇಶೇ ಸ್ಥಿತೋ ಮುನಿಃ~।। ೨೬~।।}
\end{verse}

ಮಳೆಯಲ್ಲಿ ನೆನೆದ ಸುಮೇಧನು ಆ ವೇಶ್ಯೆಯ ಮನೆಯ ಮುಂದಿನ ಚಪ್ಪರದಲ್ಲಿ ನಿಂತುಕೊಂಡನು.

\begin{verse}
\textbf{ಸ ಯಾಮಮನಯದ್ವಿಪ್ರೊ ಗೀತಾಪಾಠಪರಾಯಣಃ~।}\\\textbf{ಸಾ ಸಖೀಂ ಪ್ರೇಷಯಾಮಾಸ ಪ್ರಾಯೋsಯಂ ಬ್ರಾಹ್ಮಣೋ ಭವೇತ್~।।}
\end{verse}

ವಿಶ್ರಾಂತಿಯನ್ನು ತೆಗೆದುಕೊಂಡ ನಂತರ ಆ ಬ್ರಾಹ್ಮಣನು ಗೀತಾಪಾರಾಯಣವನ್ನು ಮಾಡಲು ಪ್ರಾರಂಭಿಸಿದನು. ಊರ್ಮಿಳೆಯು ಬಂದಿರುವವನು ಬಹುಶಃ ಬ್ರಾಹ್ಮಣನೇ ಇರಬೇಕೆಂದು ಊಹಿಸಿ ನಿಜಾಂಶವನ್ನು ತಿಳಿಯಲು ತನ್ನ ಸಖಿಯನ್ನು ಕಳಿಸಿದಳು.

\begin{verse}
\textbf{ದ್ವಿಜಶ್ಚೇದ್ದೃತಮಾಗಚ್ಛ ಕಾಮುಕಶ್ಚೇತ್ಸಮಾಹ್ವಯ~।}\\\textbf{ಇತ್ಯಾಜ್ಞ ಯಾ ತು ಸಾ ದೂತೀ ಗತ್ವಾಭಾಷ್ಯಾಗತಾ ಪುನಃ~।। ೨೮~।।}
\end{verse}

ಆತನು ಬ್ರಾಹ್ಮಣನಾಗಿದ್ದರೆ ಸುಮ್ಮನೆ ಬಂದುಬಿಡು, ಕಾಮುಕನಾಗಿದ್ದರೆ ಕರೆದುಕೊಂಡು ಬಾ-ಹೀಗೆಂದು ಆಜ್ಞಾಪಿತಳಾದ ಆ ಸೇವಕಿಯು ಬಂದು ಸುಮೇಧನನ್ನು ಮಾತನಾಡಿಸಿ ಪುನಃ ವಾಪಸ್ಸು ಮನೆಗೆ ಬಂದು ಹೀಗೆ ನುಡಿದಳು:

\begin{verse}
\textbf{ಅಧ್ವಸ್ಥಿತೋ ಗತಃ ಕಶ್ಚಿತ್ ತೀರ್ಥಯಾತ್ರಾಪರಾಯಣಃ~।}\\\textbf{ವೃಷ್ಟ್ಯಾ ತು ಪೀಡಿತಸ್ತಿಷ್ಠ ತ್ಕುಡ್ಯಪಾರ್ಶ್ವಮುಪಾಶ್ರಿತಃ~।। ೨೯~।।}
\end{verse}

ತೀರ್ಥಯಾತ್ರೆಗೆಂದು ಹೊರಟಿರುವ ಬ್ರಾಹ್ಮಣನೊಬ್ಬನು ಇಲ್ಲಿಗೆ ಬಂದು ಮಳೆಯಿಂದ ತೊಂದರೆಗೀಡಾಗಿ ಗೋಡೆಯ ಪಕ್ಕದಲ್ಲಿ ನಿಂತುಕೊಂಡಿದ್ದಾನೆ.

\begin{verse}
\textbf{ಇತಿ ದೂತೀವಚಃ ಶ್ರುತ್ವಾ ದಯಾಮಾಪ ಚ ದೈವತಃ~।}\\\textbf{ತದರ್ಥೇ ಸ್ವಗೃಹಂ ಸಮ್ಯಗ್ಗೋಮಯೇನೋಪಲಿಪ್ಯ ಚ~।। ೩೦~।। }
\end{verse}

\begin{verse}
\textbf{ತತ್ರ ನಿಕ್ಷಿಪ್ಯ ಸಂಭಾರಾಂಸ್ತಾಮ್ರಪಾತ್ರಾಣಿ ಸಾ ಸತೀ~।}\\\textbf{ದಾಸೀರೂಪಂ ಸ್ವಯಂ ಧೃತ್ವಾ ಗತ್ವಾ ವಿಪ್ರಮಥಾಬ್ರವೀತ್~।। ೩೧~।।}
\end{verse}

ಸೇವಕಿಯ ಮಾತನ್ನು ಕೇಳಿದ ಊರ್ಮಿಳೆಯು ದಯೆಯಿಂದ ಆ ಬ್ರಾಹ್ಮಣನಿಗೋಸ್ಕರ ತನ್ನ ಮನೆಯನ್ನು ಗೋಮಯದಿಂದ ಸಾರಿಸಿ, ತಾಮ್ರ ಪಾತ್ರೆಗಳೇ ಮೊದಲಾದ ಉಪಕರಣಗಳನ್ನು ಅಲ್ಲಿಟ್ಟು ತಾನೇ ದಾಸಿಯ ವೇಷದಿಂದ ಬ್ರಾಹ್ಮಣನ ಬಳಿ ಬಂದು ಹೇಳಿದಳು:

\begin{verse}
\textbf{ಬ್ರಾಹ್ಮಣಾನಾಂ ಗೃಹಮಿದಂ ಪಾಂಥಾನಾಮುಪಕಾರಕಮ್~।}\\\textbf{ಅಸೌ ಗೃಹಪತಿಸ್ತೀರ್ಥಯಾತ್ರಾರ್ಥಮಧುನಾ ಯಯೌ~।। ೩೨~।। }\\\textbf{ಗಚ್ಛನ್ ಮಾಮಬ್ರವೀದ್ವಿಪ್ರಾ ಯೇ ಗಚ್ಛಂತ್ಯ ಧ್ವನಾ ಗೃಹಮ್~।} \\\textbf{ತೇಷಾಂ ತ್ವಂ ದೇಹಿ ವಾಸಾರ್ಥಂ ಸ್ನಾನಂ ಭೋಜ್ಯಂ ಫಲಾದಿಕಮ್~।। ೩೩~।।}
\end{verse}

ಈ ಮನೆಯು ಬ್ರಾಹ್ಮಣರ ಸೇವೆಗಾಗಿ ಇರುವುದು; ಈ ದಾರಿಯಲ್ಲಿ ಪ್ರಯಾಣ ಮಾಡುವವರಿಗೆ ಉಪಕಾರ ಮಾಡಲು ಇರುತ್ತದೆ. ಈ ಮನೆಯ ಯಜಮಾನರು ಈಗ ತೀರ್ಥಯಾತ್ರೆಗಾಗಿ ಹೋಗಿರುತ್ತಾರೆ. ಹೋಗುವಾಗ ನನಗೆ ಈ ರೀತಿ ಹೇಳಿದರು: ಈ ಮಾರ್ಗದಲ್ಲಿ ಹೋಗುವ ಬ್ರಾಹ್ಮಣರು ಯಾರೇ ಈ ಮನೆಗೆ ಬರಲಿ ಅವರಿಗೆ ಸ್ನಾನ, ಒಳ್ಳೆಯ ಹಣ್ಣುಗಳು ಮುಂತಾದುವುಗಳನ್ನು ಕೊಟ್ಟು ಸತ್ಕರಿಸು.

\begin{verse}
\textbf{ತಸ್ಮಾದಾಗಚ್ಛ ಭದ್ರ ತ್ವಂ ಗೃಹೇ ವಿಶ್ರಾಮಯ ಸ್ವಯಮ್~।}\\\textbf{ವರ್ತತೇಽದ್ಯಾತಿ ಮಹತೀ ವೃಷ್ಟಿರಧ್ವಗತೋ ಭವಾನ್~।। ೩೪~।। }
\end{verse}

\begin{verse}
\textbf{ತಸ್ಮಾದಗ್ನಿಂ ಪರಿಸರ ಭುಂಕ್ಷ್ವ ಭೋಜ್ಯಂ ಫಲಾದಿಕಮ್~।}
\end{verse}

ಹೀಗೆ ಮನೆಯ ಯಜಮಾನರ ಆಜ್ಞೆಯಿರುವುದರಿಂದ ತಾವು ಈಗಲೇ ನಮ್ಮ ಮನೆಗೆ ಬನ್ನಿರಿ; ಆಯಾಸವನ್ನು ಪರಿಹರಿಸಿಕೊಳ್ಳಿರಿ; ಈಗಲೂ ಮಳೆಯು ಧಾರಾಕಾರವಾಗಿ ಸುರಿಯು\-ತ್ತಿದೆ; ಮನೆಯಲ್ಲಿ ಅಗ್ನಿಯಿಂದ ಚಳಿಯನ್ನು ಕಡಿಮೆ ಮಾಡಿಕೊಂಡು ಫಲಗಳೇ ಮೊದಲಾದುವನ್ನು ಸ್ವೀಕರಿಸಿರಿ.

\begin{verse}
\textbf{ಇತಿ ತಸ್ಯಾಃ ವಚಃ ಶ್ರುತ್ವಾ ವಿಪ್ರಃ ಶೀತೇನ ಪೀಡಿತಃ~।। ೩೫~।।}\\\textbf{ಓಮಿತ್ತ್ಯಚೇ ತತೋ ಗೇಹೇ ತದ್ರಾತ್ರೌ ನ್ಯವಸತ್ ಸುಖಮ್~। }\\\textbf{ಪರಿಧಾನಾಯ ಸಾ ಕ್ಷೌಮಂ ದದೌ ತಾತ್ಕಾಲಿಕಂ ಮುನೇ~।। ೩೬~।।}\\\textbf{ರಂಭಾಫಲಾನಿ ಚಣಕಾನ್ ದ್ವಿಜವರ್ಯಾಯ ಸಾ ದದೌ~। }\\\textbf{ಗೀತಾಪಾಠೇಣ ಸ್ತೋತ್ರಾಣಾಂ ಪಠನಾತ್ಸ ದ್ವಿಜೋ ನಿಶಾಮ್~।। ೩೭~।।} \\\textbf{ತಾಂ ನೀತ್ವಾಗಾದ್ದಿಶಂ ರೌದ್ರೀಂ ಯತ್ರಾಸ್ತೇ ರಂಗರಾಡ್ ವಿಭುಃ~। }\\\textbf{ತೇನ ಕರ್ಮವಿಪಾಕೇನ ತವ ಪುತ್ರೀತ್ವಮಾಗತಾ~।। ೩೮~।।}
\end{verse}

ಹಾಗೆಯೇ ಆಗಲೆಂದು ಚಳಿಯಿಂದ ಬಳಲಿದ್ದ ಆ ಬ್ರಾಹ್ಮಣನು ಆ ಸ್ತ್ರೀಯ ಮನೆಗೆ ಹೋಗಿ ಆ ರಾತ್ರಿಯನ್ನು ಅಲ್ಲಿ ಸುಖದಿಂದ ಕಳೆದನು. ಊರ್ಮಿಳೆಯು ಆ ವಿಪ್ರನಿಗೆ ಒಳ್ಳೆಯ ವಸ್ತ್ರಗಳನ್ನು ಕೊಟ್ಟು, ಬಾಳೆ ಹಣ್ಣು, ಕಡಲೆ ಇವುಗಳನ್ನು ತಿನ್ನಲು ಅರ್ಪಿಸಿದಳು. ಆ ಬ್ರಾಹ್ಮಣನು ಆ ರಾತ್ರಿಯನ್ನು ಗೀತಾ ಪಠಣ, ಇತರ ಸ್ತೋತ್ರಗಳ ಪಠಣ ಮಾಡುತ್ತಾ ಕಳೆದನು. ಪ್ರಾತಃಕಾಲದಲ್ಲಿ ರಂಗಕ್ಷೇತ್ರವನ್ನು ಕುರಿತು ಪ್ರಯಾಣ ಮಾಡಿದನು. (ಅಂಗಿರಾ ಋಷಿಯು ಹೇಳಿದರು) ಆ ಪುಣ್ಯ ಪ್ರಭಾವದಿಂದ ಇವಳು ನಿನ್ನ ಮಗಳಾಗಿ ಹುಟ್ಟಿರುತ್ತಾಳೆ.

\begin{verse}
\textbf{ಅಯಂ ತಸ್ಯಾ ಉಪಪತಿರ್ಜಾಮಾತಾ ತವ ಭೂಪತಿಃ~।।}\\\textbf{ಸ ಭುಕ್ತ್ವಾ ಯಾತನಾಂ ಘೋರಾಂ ಭುವಿ ಚಾಂಡಾಲಯೋನಿಷು~।। ೩೯~।।}
\end{verse}

\begin{verse}
\textbf{ಕ್ರಮಾದ್ದಶಸು ಚೋತ್ಪನ್ನೋ ಜನ್ಮನ್ಯೇಕಾದಶೇ ತತಃ~।}\\\textbf{ಲುಬ್ದಕೋಽಭೂನ್ಮಹಾಕ್ರೂರೋ ಮೃಗಸಿಂಹಾಪರಾಯಣಃ~।। ೪೦~।। }
\end{verse}

\begin{verse}
\textbf{ಕಾಂತಾರೇ ವಿಪಿನೇ ಘೋರೇ ಮೃಗಯಾಂ ವ್ಯಚರದ್ಭುವಿ~।}\\\textbf{ತತೋ ಮಧ್ಯಾಹ್ನವೇಲಾಯಾಂ ದೇಶೇ ಸುಜನವರ್ಜಿತೇ~।। ೪೧~।। }
\end{verse}

\begin{verse}
\textbf{ಅಧ್ವಶ್ರಾಂತಂ ದ್ವಿಜಂ ಕಂಚಿತ್ತೃಷಯಾ ಪರಿಪೀಡಿತಮ್~।}\\\textbf{ತಂ ದೃಷ್ಟ್ವಾಸ್ಮಿನ್ ದಯಾ ಜಜ್ಞೇ ಲುಬ್ಧ ಕಸ್ಯಾಪಿ ಭಾಗ್ಯತಃ~।। ೪೨~।।}
\end{verse}

ಅವಳ ಆಗಿನ ಉಪಪತಿಯು ಈಗ ನಿನ್ನ ಅಳಿಯನಾಗಿದ್ದಾನೆ. ಅವನು ನರಕದಲ್ಲಿ ಕ್ರೂರವಾದ ಯಾತನೆಗಳನ್ನು ಅನುಭವಿಸಿ ಭೂಮಿಯಲ್ಲಿ ಹತ್ತು ಜನ್ಮಗಳಲ್ಲಿ ಚಂಡಾಲನಾಗಿ ಹುಟ್ಟಿ ಹನ್ನೊಂದನೆಯ ಜನ್ಮದಲ್ಲಿ ಕ್ರೂರನಾದ, ಹಿಂಸಾ ಪರಾಯಣನಾದ ಬೇಟೆಗಾರನಾಗಿ ಹುಟ್ಟಿದನು. ಒಂದು ದಿನ ಬೇಟೆಯಾಡಲು ದಟ್ಟವಾದ ಕಾಡಿನಲ್ಲಿ ಅಲೆಯುತ್ತಿರುವಾಗ ಮಧ್ಯಾಹ್ನವಾಯಿತು; ಮನುಷ್ಯನ ಸಂಪರ್ಕವಿಲ್ಲದೆ ಇರುವ ಸ್ಥಳದಲ್ಲಿ ದಾರಿ ನಡೆದು ಆಯಾಸಹೊಂದಿ ಬಾಯಾರಿಕೆಯಿಂದ ಬಳಲಿದ ಒಬ್ಬ ಬ್ರಾಹ್ಮಣನನ್ನು ಕಂಡನು. ಬೇಟೆಗಾರನಾಗಿದ್ದರೂ ಸಹ ಅವನಿಗೆ ಬ್ರಾಹ್ಮಣನಲ್ಲಿ ದಯೆ ಹುಟ್ಟಿ ಹೀಗೆಂದನು:

\begin{verse}
\textbf{ಇತ ಏಹೀತ್ಯುವಾಚೈನಂ ಮತ್ತೋ ಮಾ ಭೈಷಿ ಮಾ ವ್ಯಥ~।}\\\textbf{ಭಯಾದಪಸಸಾರಾಸೌ ಕ್ಲಾಂತೋ ಗಚ್ಛನ್ ಶನೈಃ ಶನೈಃ~।। ೪೩~।। }
\end{verse}

ಬ್ರಾಹ್ಮಣನೇ ಹೀಗೆ ಬಾ, ನನ್ನನ್ನು ಕಂಡು ಭಯಪಡಬೇಡ; ಹೀಗೆನ್ನಲು ಆ ಬ್ರಾಹ್ಮಣನು ಭಯದಿಂದ ನಡುಗುತ್ತಾ ನಿಧಾನವಾಗಿ ನಡೆದು ಆ ಬೇಟಗಾರನ ಮುಂದೆ ನಿಂತುಕೊಂಡನು.

\begin{verse}
\textbf{ಅತಿಕ್ರಾಂತ್ವಾ ಗತಂ ಪ್ರಾಹ ವಟಮೂಲೇ ನಿವೇಶ್ಯ ಚ~।}\\\textbf{ಕಸ್ಮಾಚ್ಚರಸಿ ತ್ವಂ ಭದ್ರ ವನೇ ದುಷ್ಟಮೃಗಾಕುಲೇ~।। ೪೪~।। }
\end{verse}

ಬೇಟೆಗಾರನು ಆ ವಿಪ್ರನನ್ನು ಒಂದು ಆಲದ ಮರದ ಕೆಳಗೆ ಕೂಡಿಸಿ “ದುಷ್ಟ ಕಾಡುಜನರಿಂದ ಕೂಡಿದ ಈ ಅರಣ್ಯಕ್ಕೆ ಬಂದು ಯಾತಕ್ಕೆ ಸಂಚಾರಮಾಡುತ್ತಿರುವಿ?” ಎಂದನು.

\begin{verse}
\textbf{ಇತಿ ಬ್ರುವಾಣಂ ತಂ ಪ್ರಾಹ ತೃಷಯಾ ಪೀಡಿತೋsಸ್ಮ್ಯಹಮ್~।}\\\textbf{ನೋಪಲಭ್ಯಂ ಯದಿ ಜಲಂ ಪ್ರಾಣಾ ಗಚ್ಛಂತಿ ಮೇಽಧುನಾ~।। ೪೫~।। }
\end{verse}

\begin{verse}
\textbf{ನ ಚಾಂಭೋ ಭ್ರಮಮಾಣೇನ ದೃಷ್ಟಂ ಕ್ವಾಪ್ಯಟವೀ ಪಥೇ~।}\\\textbf{ಯದ್ಯಸ್ತಿ ನಯ ಮಾಂ ತತ್ರ ಮಯಿ ಚೇದಸ್ತಿ ತೇ ಕೃಪಾ~।। ೪೬~।।}
\end{verse}

ಹೀಗೆ ಕೇಳಿದ ಬೇಟೆಗಾರನನ್ನು ಕುರಿತು ಬ್ರಾಹ್ಮಣನು ಹೇಳಿದನು: ನನಗೆ ಈಗ ಕುಡಿಯಲು ನೀರು ಸಿಗದಿದ್ದರೆ ನನ್ನ ಪ್ರಾಣಗಳೇ ಹೋಗುತ್ತವೆ. ನಾನು ಎಷ್ಟು ತಿರುಗಾಡಿದರೂ ಈ ಕಾಡಿನಲ್ಲಿ ನೀರು ಕಾಣಲಿಲ್ಲ. ನನ್ನಲ್ಲಿ ನಿನಗೆ ಕರುಣೆಯಿದ್ದರೆ ನೀರು ಇರುವ ಸ್ಥಳಕ್ಕೆ ನನ್ನನ್ನು ಕರೆದುಕೊಂಡು ಹೋಗು.

\begin{verse}
\textbf{ಇತ್ಯೂಚಿವಾನ್ಸ ವಿಪ್ರೇಂದ್ರಂ ಸ್ಕಂಧೇನೋದ್ವಾಹ್ಯ ಸತ್ವರಮ್~।}\\\textbf{ಸರೋವರಂ ಮನೋರಮ್ಯಂ ಕಮಲೋತ್ಫುಲ್ಲಮಂಡಿತಮ್~।। ೪೭~।। }
\end{verse}

\begin{verse}
\textbf{ನೀಲೋತ್ಪಲೈಃ ಕೋಕನದೈಃ ಕಲ್ಹಾ ರೈಃ ಪರಿಶೋಭಿತಮ್~।}\\\textbf{ವೀಚೀವಿಕ್ಷೋಭಶೋಭಾಢ್ಯಂ ಜಲಪಕ್ಷಿಭಿರಾವೃತಮ್~।। ೪೮~।। }
\end{verse}

\begin{verse}
\textbf{ಪ್ರಾಪಯಾಮಾಸ ವಿಪ್ರೇಂದ್ರಂ ಕ್ಷುತ್ತೃಟ್ ಭ್ರಾಂತಂ ಸರೋವರಮ್~।}
\end{verse}

ಬ್ರಾಹ್ಮಣನ ಈ ಮಾತನ್ನು ಕೇಳಿ ಬೇಟೆಗಾರನು ಆತನನ್ನು ತನ್ನ ಹೆಗಲಿನ ಮೇಲೆ ಕೂಡಿಸಿಕೊಂಡು ಮನೋಹರವಾದ ಒಂದು ಸರೋವರಕ್ಕೆ ಕರೆತಂದನು. ಸರೋವರವು ನೀರಿನಿಂದ ತುಂಬಿತ್ತು; ನೀಲೋತ್ಪಲ, ಕಲ್ಹಾರ, ಕಮಲಪುಷ್ಪ, ಜಲಪಕ್ಷಿಗಳಿಂದ ನಿಬಿಡವಾಗಿತ್ತು. ಬ್ರಾಹ್ಮಣನಿಗೆ ಹಸಿವು, ಬಾಯಾರಿಕೆ ವಿಶೇಷವಾಗಿದ್ದುವು.

\begin{verse}
\textbf{ಸ ಪಪಾತ ಜಲೇ ತಸ್ಮಿನ್ ಭೂರಿವಾರಿ ಪಪೌ ಚ ಸಃ~।। ೪೯~।।} \\\textbf{ಬಿಸಾನ್ಯಾದಾಯ ವಿಪ್ರಾಯ ಲುಬ್ದ ಕೋಽದಾತ್ ಪ್ರಸನ್ನ ಧೀಃ~।}\\\textbf{ತತೋ ಗತಶ್ರಮೋ ವಿಪ್ರ ಆಶೀರ್ಭಿರಭಿನಂದ್ಯ ತಮ್~।। ೫೦~।। }\\\textbf{ಆಮಂತ್ರ್ಯ ಪ್ರಯಯೆೌ ಧೀಮಾನ್ ಸ್ವದೇಶಂ ದ್ವಿಜಸತ್ತಮಃ~।}
\end{verse}

ಬ್ರಾಹ್ಮಣನು ನೀರಿನಲ್ಲಿ ಇಳಿದು ಯಥೇಚ್ಛವಾಗಿ ಜಲಪಾನ ಮಾಡಿದನು. ಬೇಟೆಗಾರನು ವಿಪ್ರನಿಗೆ ಕಮಲದ ದಂಟುಗಳನ್ನು ತಿನ್ನಲು ತಂದು ಕೊಟ್ಟನು. ಶ್ರಮ ಪರಿಹಾರವಾದಮೇಲೆ ಬ್ರಾಹ್ಮಣನು ಆ ಬೇಟೆಗಾರನಿಗೆ ಆಶೀರ್ವಾದಮಾಡಿ, ಅಭಿನಂದಿಸಿ, ತನ್ನ ಸ್ಥಳಕ್ಕೆ ಹೊರಟು\-ಹೋದನು.

\begin{verse}
\textbf{ತೇನ ಪುಣ್ಯ ಪ್ರಭಾವೇನ ಲುಬ್ಧ ಕಶ್ಚೇದಿರಾಡಯಮ್~।। ೫೧~।।}
\end{verse}

ಆ ಪುಣ್ಯ ಪ್ರಭಾವದಿಂದ ಆ ಬೇಟೆಗಾರನು ಈಗ ಚೇದಿ ದೇಶಕ್ಕೆ ರಾಜನಾಗಿದ್ದಾನೆ.

\begin{verse}
\textbf{ಪರಸ್ತ್ರೀಸಂಗದೋಷೇಣ ಗತಾಯುರಭವನ್ನೃಪ~।}\\\textbf{ಭರ್ತೃಹನನಜಂ ಪಾಪಂ ಪರಪೂರುಷಸಂಗಜಮ್~।। ೫೨~।। }
\end{verse}

\begin{verse}
\textbf{ಇತೀವ ದುಹಿತುಃ ಪಾಪದ್ವಯಂ ತಸ್ಯೋಪಶಾಂತಯೇ~।}\\\textbf{ದಿನದ್ವಯಕೃತಂ ಸ್ನಾನಂ ಮಾಘೇ ಮಾಸಿ ಮಹೀಪತೇ~।। ೫೩~।। }
\end{verse}

\begin{verse}
\textbf{ದತ್ತಂ ಪುಣ್ಯಂ ತತೋ ಭೂಪ ತಸ್ಮಾತ್ಪಾಪಂ ವ್ಯಪೋಹಿತಮ್~।}\\\textbf{ಅತ್ಯುಗ್ರಂ ದುಃಖಮೂಲಂ ಯದ್ಬಾಲ್ಯೇ ವೈಧವ್ಯ ಕಾರಣಮ್~।। ೫೪~।।}
\end{verse}

ಆದರೆ ಪರಸ್ತ್ರೀಸಂಗದೋಷೇಣ ಅಲ್ಪಾಯುಷಿಯಾಗಿದ್ದಾನೆ. ನಿನ್ನ ಮಗಳು ತನ್ನ ಪತಿಯನ್ನು ಸಂಹರಿಸಿದ ಕಾರಣದಿಂದಲೂ, ಪರಪುರುಷನೊಡನೆ ಸಂಪರ್ಕ ಮಾಡಿದ ಪಾಪದಿಂದಲೂ ವೈಧವ್ಯವನ್ನು ಪಡೆದಳು. ಈ ಎರಡು ಪಾಪಗಳ ವಿಮೋಚನೆಗಾಗಿ ನಾನು ಎರಡು ದಿನಗಳ ಮಾಘ ಸ್ನಾನದ ಪುಣ್ಯಗಳನ್ನು ಕೊಟ್ಟೆ. ಇದರಿಂದ ಬಾಲ್ಯದಲ್ಲಿ ವೈಧವ್ಯ ಹೊಂದಿದ ಪಾಪವು ನಾಶವಾಯಿತು.

\begin{verse}
\textbf{ಪರಸ್ತ್ರೀಸಂಗಜಂ ಪಾಪಂ ಜಾಮಾತೃಮೃತಿಕಾರಣಮ್~।}\\\textbf{ಅಪಮೃತ್ಯುವಿನಾಶಾಯ ಸ್ನಾನಮೇಕದಿನೇ ಕೃತಮ್~। }\\\textbf{ತೇನ ಪುಣ್ಯ ಪ್ರಭಾವೇನ ಪುನರುಜ್ಜೀವಿತೋ ನೃಪ~।। ೫೫~।।}
\end{verse}

ನಿನ್ನ ಅಳಿಯನ ಅಕಾಲಮೃತ್ಯುವಿಗೆ ಪರಸ್ತ್ರೀಸಂಗಮಾಡಿದ್ದೇ ಕಾರಣ. ಅದರ ನಾಶಕ್ಕಾಗಿ ಮಾಘಸ್ನಾನದ ಒಂದು ದಿವಸದ ಫಲವನ್ನು ಕೊಟ್ಟಿರುತ್ತೇನೆ. ಅದರ ಪುಣ್ಯ ಪ್ರಭಾವದಿಂದ ನಿನ್ನ ಅಳಿಯನು ಪುನಃ ಬದುಕಿದನು.

\begin{verse}
\textbf{ಕಿಂ ವರ್ಣಯಾಮಿ ಮನುಜೇಶ್ವರ ಮಾಘಮಾಸೇ}\\\textbf{ವಿಶ್ವೇಶ್ವರಸ್ಯ ಕರುಣಾಂ ನೃಪಪಾತಿತೇಷು~। }\\\textbf{ಕಿಂ ವರ್ಣಯಾಮಿ ಮನುಜಾತ್ಮಹಿತಾನಭಿಜ್ಞಾನ್} \\\textbf{ಕಿಂ ಸ್ನಾನಪುಣ್ಯಫಲಮಂಬುಜನಾಭವಶ್ಯಮ್~।। ೫೬~।।}
\end{verse}

ರಾಜನೇ! ಮಾಘಮಾಸದಲ್ಲಿ ಸ್ನಾನಮಾಡುವ ಜನರ ಮೇಲೆ ಶ‍್ರೀ ಹರಿಯು ತೋರಿಸುವ ಕರುಣೆಯನ್ನು ಏನೆಂದು ವರ್ಣಿಸಲಿ! ತಮ್ಮ ಹಿತವೇನೆಂದು ತಿಳಿದುಕೊಂಡಿರುವ ಜನರನ್ನು ಹೇಗೆ ವರ್ಣಿಸಲಿ! ಶ‍್ರೀ ಪದ್ಮನಾಭನ ಪ್ರಾಪ್ತಿಗೆ ಕಾರಣವಾದ ಮಾಘ ಸ್ನಾನದ ಮಹಿಮೆಯನ್ನು ಎಷ್ಟೆಂದು ಹೇಳಲಿ!

\begin{verse}
\textbf{ನ ಕ್ಲಿ ಶ್ಯಂತಿ ನರಾ ಲೋಕೇ ಮಾಘಸ್ನಾನಪರಾಯಣಾಃ~।}\\\textbf{ತಸ್ಮಾತ್ ದುಹಿತ್ರಾ ಜಾಮಾತ್ರಾ ಕರ್ತವ್ಯಂ ಸ್ನಾನಮಂಜಸಾ~।। ೫೭~।। }
\end{verse}

\begin{verse}
\textbf{ತೇನ ಪುಣ್ಯ ಪ್ರಭಾವೇನ ದಶವರ್ಷಸಹಸ್ರಕಮ್~।}\\\textbf{ಭುಂಕ್ತ್ಯೇವ ಭೋಗಾನ್ ವಿಪುಲಾನ್ ಪುತ್ರಪೌತ್ರಸಮಾವೃತೌ~।। ೫೮~।। }
\end{verse}

\begin{verse}
\textbf{ಭವಿಷ್ಯತಸ್ತತೋಮೋಕ್ಷಂ ದಂಪತೀ ತೌ ಗಮಿಷ್ಯತಃ~।}\\\textbf{ಇತಿ ಸರ್ವಂ ಸಮಾಖ್ಯಾತಂ ಶ್ವೋ ಯಜ್ಞೋ ಭವಿತಾ ಮಮ~।। ೫೯~।।}
\end{verse}

ಮಾಘಸ್ನಾನ ನಿರತರಾದ ಯಾರೂ ಲೋಕದಲ್ಲಿ ಕಷ್ಟ ಪಡುವುದಿಲ್ಲ. ಆದುದರಿಂದ ನಿನ್ನ ಮಗಳೂ, ಅಳಿಯನೂ ಅವಶ್ಯವಾಗಿ ಮಾಘಸ್ನಾನ ಮಾಡಬೇಕು. ಅದರ ಪುಣ್ಯದಿಂದ ಇವರು ಈ ಲೋಕದಲ್ಲಿ ಹತ್ತು ಸಾವಿರ ವರ್ಷಗಳ ಕಾಲ ಪುತ್ರ ಪೌತ್ರ ಧನಾದಿ ಸುಖಗಳನ್ನು ಅನುಭವಿಸಿ ನಂತರ ಮೋಕ್ಷವನ್ನು ಪಡೆಯುತ್ತಾರೆ. ಹೀಗೆ ಸರ್ವವನ್ನೂ ನಿನಗೆ ಹೇಳಿದ್ದೇನೆ. ನಾಳೆ ನಮ್ಮಲ್ಲಿ ಯಜ್ಞ ಇರುತ್ತದೆ.

\begin{verse}
\textbf{ತದರ್ಥಂ ದಕ್ಷಿಣಾಂ ದೇಹಿ ಗಚ್ಛನ್ನಾಶ್ರಮಮಂಡಲಂ~।}\\\textbf{ಇತ್ಯುಕ್ತಃ ತೋಷಯಾಮಾಸ ತಂ ಮುನಿಂ ಶಂಸಿತವ್ರತಂ~।। ೬೦~।। }
\end{verse}

\begin{verse}
\textbf{ತತೋsನುಜ್ಞಾ ಮವಾಪ್ಯಾಥ ಸ್ವಧರ್ಮಸ್ಯಾಪಿ ನಾರದ~।}\\\textbf{ಜಗಾಮ ಸ್ವಾಶ್ರಮಪದಮಂಗಿರಾನಾಮ ವೈ ಋಷಿಃ~।। ೬೧~।। }
\end{verse}

\begin{verse}
\textbf{ಪ್ರಹೃಷ್ಟೌ ಸುಸ್ಮಿತೌ ಶಾಂತೌ ಜಾಮಾತಾ ದುಹಿತಾ ತಥಾ~।}\\\textbf{ಭಾರ್ಯಾಮಾದಾಯ ಜಾಮಾತಾ ಜಗಾಮ ಸ್ವಪುರೀಂ ಪ್ರತಿ~।। ೬೨~।।}
\end{verse}

ಯಜ್ಞಕ್ಕೋಸ್ಕರ ದಕ್ಷಿಣೆಯನ್ನು ಕೊಡು. ನಾನು ನನ್ನ ಆಶ್ರಮಕ್ಕೆ ಹೋಗುತ್ತೇನೆ. ರಾಜನು ಆ ಶ್ರೇಷ್ಠರಾದ ಮುನಿಗಳನ್ನು ದಕ್ಷಿಣಾದಿಗಳಿಂದ ಸಂತೋಷಪಡಿಸಿದನು. ನಾರದನೇ! ನಂತರ ಅಂಗೀರಸ ಋಷಿಗಳು ತಮ್ಮ ಆಶ್ರಮಕ್ಕೆ ತೆರಳಿದರು. ಅಳಿಯನೂ ಮಗಳೂ ಸಂತೋಷ ಮತ್ತು ಶಾಂತಿಭರಿತರಾದರು. ಅಳಿಯನು ತನ್ನ ಪತ್ನಿಯೊಡನೆ ತನ್ನ ಪಟ್ಟಣಕ್ಕೆ ಹಿಂದಿರುಗಿದನು.

\begin{verse}
\textbf{ಚಕ್ರತುರ್ಮಾಘಮಾಸೇ ತು ಪ್ರತಿವರ್ಷಂ ಚ ದಂಪತೀ~।}\\\textbf{ಸ್ನಾನಂ ಪೂಜಾಂ ಮಹಾವಿಷ್ಣೋಸ್ತಥಾ ಮಾಧವಸಂಜ್ಞಿತಃ~।। ೬೩~।। }
\end{verse}

\begin{verse}
\textbf{ದಶವರ್ಷಸಹಸ್ರಂ ತು ಏತತ್ಪುಣ್ಯ ಪ್ರಭಾವತಃ~।}\\\textbf{ಭುಕ್ವಾ ನಾನಾವಿಧಾನ್ಕಾಮಾನಾಪತುರ್ಮೋಕ್ಷಮವ್ಯಯಮ್~।। ೬೪~।।}
\end{verse}

ಅಲ್ಲಿಂದ ಮುಂದೆ ಆ ದಂಪತಿಗಳು ಪ್ರತಿವರ್ಷವೂ ತಪ್ಪದೇ ಮಾಘಸ್ನಾನವನ್ನು ಮಾಡಿ, ಮಾಧವನನ್ನು ಪೂಜಿಸಿ, ಹತ್ತು ಸಾವಿರ ವರ್ಷ ಈ ಪುಣ್ಯದಿಂದ ನಾನಾ ವಿಧವಾದ ಸುಖಗಳನ್ನು ಅನುಭವಿಸಿ ಕಡೆಯಲ್ಲಿ ನಾಶರಹಿತವಾದ ಮೋಕ್ಷವನ್ನು ಹೊಂದಿದರು.

\begin{center}
ಇತಿ ಶ‍್ರೀ ವಾಯುಪುರಾಣೇ ಮಾಘಮಾಸಮಾಹಾತ್ಮ್ಯೇ ತೃತೀಯೋಽಧ್ಯಾಯಃ
\end{center}

\begin{center}
ಶ‍್ರೀ ವಾಯುಪುರಾಣಾಂತರ್ಗತ ಮಾಘಮಾಸ ಮಾಹಾತ್ಮ್ಯೆಯಲ್ಲಿ \\ ಮೂರನೇ ಅಧ್ಯಾಯವು ಸಮಾಪ್ತಿಯಾಯಿತು.
\end{center}

\newpage

\section*{ಅಧ್ಯಾಯ\enginline{-}೪}

\emptypage

\begin{flushleft}
\textbf{ನಾರದ ಉವಾಚ\enginline{-}}
\end{flushleft}

\begin{verse}
\textbf{ತವ ವಾಗಮೃತಂ ಪೀತ್ವಾ ತೃಪ್ತಿರ್ನಾಸೀನ್ಮಮಮಾದ್ಯ ವೈ~।}\\\textbf{ನಿವಿ ಜಡಮತಿಂ ಲೋಕೇ ಶ್ರವಣಾತ್ಕೋನು ತೃಪ್ಯತಿ~।। ೧~।।}
\end{verse}

\begin{verse}
\textbf{ಭೂಯೋ ಮಾಘಸ್ಯ ಮಾಹಾತ್ಮ್ಯಂ ತಾತ ವಿಸ್ತರತೋ ವದ~।}
\end{verse}

\begin{flushleft}
ನಾರದರು ಹೇಳಿದರು-
\end{flushleft}

ತಮ್ಮ ಉಪದೇಶಾಮೃತವನ್ನು ಪಾನಮಾಡಿ ನನಗೆ ತೃಪ್ತಿಯಾಗಲಿಲ್ಲ. ಬುದ್ಧಿಯಿಲ್ಲದ ಜನರಿಗೆ ಹೊರತು ಇನ್ನು ಯಾರಿಗೆ ತಾನೇ ಭಗವಂತನ ಮಾಹಾತ್ಮ್ಯ ಶ್ರವಣದಿಂದ ತೃಪ್ತಿಯಾಗುತ್ತದೆ? ಬ್ರಹ್ಮದೇವರೇ, ಮಾಘಮಾಸದ ಮಾಹಾತ್ಮ್ಯವನ್ನು ಇನ್ನೂ ವಿಸ್ತಾರವಾಗಿ ಹೇಳಿರಿ.

\begin{flushleft}
\textbf{ಬ್ರಹ್ಮೋವಾಚ\enginline{-}}
\end{flushleft}

\begin{verse}
\textbf{ಶೃಣು ನಾರದ ವಕ್ಷ್ಯಾಮಿ ಮಾಘಮಾಹಾತ್ಮ್ಯ ಮುತ್ತಮಮ್~।। ೨~।। }
\end{verse}

\begin{verse}
\textbf{ಯದ್ ಜ್ಞಾ ತ್ವಾ ಜಾಯತೇ ಶುದ್ಧಿ ರ್ಮನಸೋ ಮಹದಂಜಸಾ~।}\\\textbf{ನಾರ್ಯಃ ಪತಿಘ್ನೋ ಬಾಲಘ್ನ್ಯಃ ಸ್ವೈರಿಣ್ಯಃ ಕೂಟಬುದ್ಧಯಃ~।। ೩~।। }
\end{verse}

\begin{verse}
\textbf{ಮಾತಾಪಿತ್ರೋರ್ವಿಘಾತಾನ್ಯೋ ಗರದಾ ಲೋಕಗರ್ಹಿತಾಃ~।}\\\textbf{ಪುರುಷಾಃ ಪಾಪಿನೋ ಘೋರಾಃ ಸರ್ವಕರ್ಮಬಹಿಷ್ಕೃತಾಃ~।। ೪~।। }
\end{verse}

\begin{verse}
\textbf{ಮಾಘಸ್ನಾನಾದ್ವಿಮುಕ್ತಾಸ್ಯುರ್ನಾತ್ರ ಕಾರ್ಯಾ ವಿಚಾರಣಾ~।}\\\textbf{ತತ್ಪಾಪಂ ನಾಸ್ತಿ ಲೋಕೇsಸ್ಮಿನ್ ಮಾಘೇನಾಪಹೃತಂ ಕಲೌ~।। ೫~।।}
\end{verse}

\begin{flushleft}
ಬ್ರಹ್ಮದೇವರು ಹೇಳಿದರು-
\end{flushleft}

ನಾರದನೇ! ಉತ್ತಮವಾದ ಮಾಘಮಾಸದ ಮಾಹಾತ್ಮ್ಯೆಯನ್ನು ಹೇಳುತ್ತೇನೆ ಕೇಳು. ಇದನ್ನು ಕೇಳುವುದರಿಂದ ಅಂತಃಕರಣ ಶುದ್ಧಿಯಾಗುತ್ತದೆ. ಪತಿಗಳನ್ನು ಕೊಂದ ಪತ್ನಿಯರು, ಶಿಶುಹತ್ಯೆ ಮಾಡಿದವರು, ಜಾರಿಣಿಯಾದ ಸ್ತ್ರೀಯರು, ಕೆಟ್ಟ ಬುದ್ಧಿಯುಳ್ಳವರು, ತಮ್ಮ ತಂದೆ ತಾಯಿಯರನ್ನು ವಧೆಮಾಡಿದವರು, ಇತರರಿಗೆ ವಿಷ ಕೊಟ್ಟವರು, ಇತರರನ್ನು ನಿಂದಿಸುವವರು, ಘೋರ ಪಾಪವನ್ನು ಆಚರಿಸಿದ ಪುರುಷರು, ಸಮಸ್ತ ಸತ್ಕರ್ಮಗಳನ್ನೂ ಪರಿತ್ಯಾಗ ಮಾಡಿದವರು, ಇವರೆಲ್ಲರೂ ಮಾಘಸ್ನಾನದಿಂದ ಪಾಪವಿಮುಕ್ತರಾಗುತ್ತಾರೆ. ಸಂಶಯವಿಲ್ಲ. ಮಾಘಸ್ನಾನದಿಂದ ಪರಿಹಾರವಾಗದೇ ಇರುವ ಪಾಪವೇ ಕಲಿಯುಗದಲ್ಲಿ ಯಾವುದೂ ಇಲ್ಲ.

\begin{verse}
\textbf{ಕೃತ್ವಾ ಪಾಪಸಹಸ್ರಾಣಿ ನ ಪಾಪೈಃ ಲಿಪ್ಯತೇ ನರಃ~।}\\\textbf{ಜಲೇ ಯಥಾ ನಿಮಗ್ನಾನಿ ಪದ್ಮಪತ್ರಾಣಿ ಚಾಂಭಸಾ~।। ೬~।। }
\end{verse}

\begin{verse}
\textbf{ಮಾಘಸ್ನಾನಂ ಸಮುದ್ದಿಶ್ಯ ಯೇ ಗಚ್ಛಂತಿ ಬಹಿರ್ಜಲಮ್~।}\\\textbf{ಪದೇ ಪದೇಽಶ್ವಮೇಧಸ್ಯ ಪ್ರಾಪ್ನುವಂತಿ ಫಲಂ ಧ್ರುವಮ್~।। ೭~।। }
\end{verse}

\begin{verse}
\textbf{ಮಾಘೇ ಮಾಸಿ ಮುನಿಶ್ರೇಷ್ಠ ಯೋ ಜಲಸ್ಪರ್ಶಮಾಚರೇತ್~।}\\\textbf{ಜಲಸ್ಯ ಸ್ಪರ್ಶಮಾತ್ರೇಣ ಮುಚ್ಯತೇ ಸರ್ವಕಿಲ್ಬಿಷೈಃ~।। ೮~।। }
\end{verse}

\begin{verse}
\textbf{ಯೊ ಗೃಹೇ ಮಾಘಮಾಸೇ ತು ಸ್ನಾನಕರ್ಮ ಸಮಾಚರೇತ್~।}\\\textbf{ಶ್ವಾನಮಾತ್ರಂ ಚ ತಂ ವಿದ್ಯಾನ್ನಾತ್ರ ಕಾರ್ಯಾ ವಿಚಾರಣಾ~।। ೯~।।}
\end{verse}

ಸಹಸ್ರಾರು ಪಾಪಗಳನ್ನು ಆಚರಿಸಿದ್ದರೂ ಸಹ ನೀರು ಕಮಲದ ಎಲೆಗೆ ಹೇಗೆ ಅಂಟಿಕೊಳ್ಳುವುದಿಲ್ಲವೋ ಹಾಗೆಯೇ ಆ ಪಾಪಗಳು ಮಾಘಸ್ನಾನದಿಂದ ಲೇಪವಾಗುವುದಿಲ್ಲ. ಮಾಘಸ್ನಾನ\-ಮಾಡುವ ಉದ್ದೇಶದಿಂದ ಯಾವನು ಊರ ಹೊರಗಿನ ಜಲಾಶಯಕ್ಕೆ ಹೋಗುತ್ತಾನೆಯೋ ಅಂತಹವನು ಹೆಜ್ಜೆ ಹೆಜ್ಜೆಗೂ ಅಶ್ವಮೇಧ ಯಜ್ಞದ ಫಲವನ್ನು ಪಡೆಯುತ್ತಾನೆ. ಇದು ನಿಶ್ಚಯ. ನಾರದನೇ! ಯಾವನು ಜಲಸ್ಪರ್ಶ ಮಾಡುತ್ತಾನೆಯೋ ಅಂತಹವನು ಸಕಲ ಪಾಪಗಳಿಂದ ವಿಮುಕ್ತನಾಗುತ್ತಾನೆ. ಮಾಘಮಾಸದಲ್ಲಿ ಯಾವನು ಮನೆಯಲ್ಲಿಯೇ ಸ್ನಾನಮಾಡುತ್ತಾನೆಯೋ ಅವನನ್ನು ನಾಯಿಯೆಂದು ತಿಳಿಯಬೇಕು, ವಿಚಾರವಿಲ್ಲ.

\begin{verse}
\textbf{ಸರ್ವಾಣ್ಯಪಿ ಚ ಪಾಪಾನಿ ತೈಲೋಕ್ಯಾಂತರ್ಗತಾನಿ ಚ~।}\\\textbf{ಗೇಹಸ್ಯ ಜಲಮಾಶ್ರಿತ್ಯ ವಸಂತ್ಯಾಶ್ರಯಕಾಂಕ್ಷಯಾ~।। ೧೦~।।}
\end{verse}

\begin{verse}
\textbf{ಮಾಘೇ ಮಾಸಿ ಸುಖಂ ತಿಷ್ಠನ್ ಯಃ ಸ್ನಾಯಾದುಷ್ಣ ವಾರಿಣಾ~।}\\\textbf{ವಹ್ನಿರಾಶೌ ಮಹಾಘೋರೇ ಪತತ್ಯಾಶು ನ ಸಂಶಯಃ~।। ೧೧~।।}
\end{verse}

ಮೂರು ಲೋಕಗಳಲ್ಲಿರುವ ಪಾಪಗಳೆಲ್ಲ ಮಾಘಮಾಸದಲ್ಲಿ ಮನೆಯೊಳಗಿನ ನೀರಿನ ಆಶ್ರಯದಲ್ಲಿರುತ್ತವೆ. ಯಾವನು ಸುಖವಾಗಿ ಬಿಸಿನೀರಿನಲ್ಲಿ ಮಾಘಮಾಸದಲ್ಲಿ ಸ್ನಾನ ಮಾಡುತ್ತಾನೆಯೋ ಅಂತಹವನು ಭಯಂಕರವಾದ ಬೆಂಕಿಯ ರಾಶಿಯಲ್ಲಿ ಬೀಳುತ್ತಾನೆ, ಸಂಶಯವಿಲ್ಲ.

\begin{verse}
\textbf{ಯೋ ಮಾಘೇ ಮಾಧವಂ ದೇವಂ ಪಯಸಾ ಸ್ನಾಪಯೇದ್ಯದಿ~।}\\\textbf{ಸ ಕೋಟಿಕುಲಮುದ್ಧೃತ್ಯ ವಿಷ್ಣುಲೋಕೇ ಮಹೀಯತೇ~।। ೧೨~।। }
\end{verse}

\begin{verse}
\textbf{ಘೃತೇನ ಸ್ನಾಪಯೇದ್ಯಸ್ತು ಗವ್ಯೇನಾಸುರಘಾತಿನಮ್~।}\\\textbf{ವೈಕುಂಠೇ ಮೋದತೇ ನಿತ್ಯಂ ದಶಪೂರ್ವೈರ್ದಶಾಪರೈಃ~।। ೧೩~।।}
\end{verse}

\begin{verse}
\textbf{ಪಂಚಗವ್ಯೇನ ದೇವೇಶಂ ಸ್ನಾಪಿತೇ ಕೈಟಭದ್ವಿಷಿ~।}\\\textbf{ಸ ಜೀವನ್ನೈವ ಮುಕ್ತಃ ಸ್ಯಾನ್ನಾತ್ರ ಕಾರ್ಯಾ ವಿಚಾರಣಾ~।। ೧೪~।।}
\end{verse}

ಯಾರು ಮಾಘಮಾಸದಲ್ಲಿ ಮಾಧವನಿಗೆ ಹಾಲಿನಿಂದ ಅಭಿಷೇಕ ಮಾಡುತ್ತಾನೆಯೋ ಅವನು ತನ್ನ ಕೋಟಿ ಕುಲಗಳನ್ನು ಉದ್ಧರಿಸಿ ತಾನು ವೈಕುಂಠ ಲೋಕದಲ್ಲಿ ಮೆರೆಯುತ್ತಾನೆ, ತುಪ್ಪ, ಮೊಸರು, ಇವುಗಳಿಂದ ಅಭಿಷೇಕ ಮಾಡುವವನು ತನ್ನ ಹಿಂದಿನ ಹತ್ತು ಮತ್ತು ಮುಂದಿನ ಹತ್ತು ಕುಲಗಳಿಂದ ವೈಕುಂಠದಲ್ಲಿ ಸುಖವಾಗಿರುತ್ತಾನೆ. ಸರ್ವೋತ್ತಮನಾದ ವಿಷ್ಣುವನ್ನು ಪಂಚಗವ್ಯದಿಂದ ಅಭಿಷೇಕ ಮಾಡಿದರೆ ಅವನು ಜೀವನ್ಮುಕ್ತನಾಗುವನು, ಇದರಲ್ಲಿ ವಿಚಾರವಿಲ್ಲ.

\begin{verse}
\textbf{ಶಾಲಗ್ರಾಮಶಿಲಾಪೂಜಾಂ ಮಾಘೇ ಮಾಸಿ ಕರೋತಿ ಯಃ~।}\\\textbf{ವಾಂಛಂತಿ ಕರಸಂಸ್ಪರ್ಶಂ ತೇಷಾಂ ದೇವಾಃ ಸವಾಸವಾಃ~।। ೧೫~।। }
\end{verse}

\begin{verse}
\textbf{ಶಾಲಗ್ರಾಮಶಿಲಾ ಯಸ್ಯ ಗೃಹೇ ತಿಷ್ಠತಿ ನಾರದ~।}\\\textbf{ಅಪಿ ಪಾಪಶತೈರ್ಯುಕ್ತಂ ಯಮಸ್ತಂ ನೇಕ್ಷಿತುಂ ಕ್ಷಮಃ~।। ೧೬~।।} 
\end{verse}

\begin{verse}
\textbf{ಯೇ ಮಾಘಸ್ನಾನವಿರತಾ ಮಾಧವಾರ್ಚಾವಿವರ್ಜಿತಾಃ~।}\\\textbf{ಯೇಷಾಂ ಧರ್ಮರತಿರ್ನೈವ ತೇಷಾಂ ಸ್ವಾಮೀ ಯಮಃ ಸ್ವಯಮ್~।। ೧೭~।।}
\end{verse}

ಯಾರು ಮಾಘಮಾಸದಲ್ಲಿ ಶಾಲಗ್ರಾಮ ಪೂಜೆಯನ್ನು ಮಾಡುತ್ತಾರೆಯೋ ಅಂತಹವರಿಗೆ ಇಂದ್ರನಿಂದ ಸಹಿತರಾದ ದೇವತೆಗಳು ಹಸ್ತಲಾಘವವನ್ನು ಕೊಡಲು ಇಚ್ಚಿಸುತ್ತಾರೆ. ಯಾರ ಮನೆಯಲ್ಲಿ ಶಾಲಗ್ರಾಮ ಇರುತ್ತದೆಯೋ, ಅಂತಹವರು ನೂರಾರು ಪಾಪಗಳನ್ನು ಮಾಡಿದ್ದರೂ ಸಹ, ಯಮದೇವರು ಅವರನ್ನು ನೋಡಲು ಸಮರ್ಥರಾಗುವುದಿಲ್ಲ. ಮಾಘ\-ಸ್ನಾನ ಮಾಡದೇ ಇರುವ ಜನರಿಗೆ, ಮಾಧವನನ್ನು ಪೂಜಿಸದ ಜನರಿಗೆ, ಧರ್ಮದಲ್ಲಿ ಆಸಕ್ತಿ ಇಲ್ಲದವರಿಗೆ, ಯಮದೇವರೇ ಪ್ರಭುವಾಗಿರುತ್ತಾರೆ.

\begin{verse}
\textbf{ರೋಗಾಃ ಪಿಶಾಚಾ ಯಕ್ಷಾದ್ಯಾ ಮಾಘಸ್ನಾನವಿವರ್ಜಿತಮ್~।}\\\textbf{ಗ್ರಹಾಶ್ಚ ಪೀಡಯಂತ್ಯೇನಂ ಯಮಃ ಪಶ್ಚಾಚ್ಚ ಶಿಕ್ಷತಿ~।। ೧೮~।।}
\end{verse}

ಮಾಘಸ್ನಾನ ಮಾಡದೇ ಇರುವವನನ್ನು ನಾನಾ ರೋಗಗಳೂ, ಪಿಶಾಚಿಗಳೂ, ಯಕ್ಷರೂ ಗ್ರಹಗಳೂ ಪೀಡಿಸುತ್ತವೆ; ನಂತರ ಯಮದೇವರೂ ಶಿಕ್ಷೆ ಕೊಡುವರು.

\begin{verse}
\textbf{ನಾರೀಭಿಃ ಪುರುಷೈರ್ವಾಪಿ ಕುಸುಮೈರ್ಗಂಧಲೇಪನೈಃ~।}\\\textbf{ಮಾಸಿ ಮಾಘೇ ಮಾಧವಂ ಚ ಪಾವಿತಂ ಸಕಲಂ ಕುಲಮ್~।। ೧೯~।। }
\end{verse}

\begin{verse}
\textbf{ಶಂಖೋದಕೇನ ದೇವೇಶಂ ಸ್ನಾಪಯೇನ್ಮಾಧವಂ ಯದಿ~।}\\\textbf{ಕಲ್ಪ ಕೋಟಿಸಹಸ್ರಾಣಿ ಮೋದತೇ ವಿಷ್ಣು ಮಂದಿರೇ~।। ೨೦~।।}
\end{verse}

ಸ್ತ್ರೀಯರಾಗಲೀ, ಪುರುಷರಾಗಲೀ ಮಾಘಮಾಸದಲ್ಲಿ ಮಾಧವನನ್ನು ಗಂಧ ಪುಷ್ಪಾದಿಗಳಿಂದ ಪೂಜಿಸಿದರೆ ಅವರ ಸಕಲ ಕುಲಗಳೂ ಪವಿತ್ರವಾಗುತ್ತವೆ. ಸರ್ವೊತ್ತಮನಾದ ವಿಷ್ಣುವನ್ನು ಶಂಖದಿಂದ ಅಭಿಷೇಕ ಮಾಡುವುದರಿಂದ ಸಹಸ್ರ ಕೋಟಿ ಕಲ್ಪಗಳಲ್ಲಿ ವೈಕುಂಠ ಲೋಕದಲ್ಲಿ ಸುಖವಾಗಿರುವರು.

\begin{verse}
\textbf{ಶಂಖಮಧ್ಯೇ ಸ್ಥಿತಂ ತೋಯಂ ಭ್ರಾಮಿತಂ ಕೇಶವೋಪರಿ~।}\\\textbf{ಅಂಗಲಗ್ನಾಂ ಮನುಷ್ಯಾಣಾಂ ಬ್ರಹ್ಮಹತ್ಯಾಂ ವ್ಯಪೋಹತಿ~।। ೨೧~।।}
\end{verse}

ಶಂಖದಲ್ಲಿ ನೀರನ್ನು ತುಂಬಿ ಕೇಶವನನ್ನು ಅದರಿಂದ ಭ್ರಮಣ ಮಾಡಿದರೆ ಆ ಮನುಷ್ಯ ದೇಹದಲ್ಲಿರುವ ಬ್ರಹ್ಮಹತ್ಯಾದಿ ಪಾಪಗಳು ನಾಶ ಹೊಂದುತ್ತವೆ

\begin{myquote}
\textbf{ವಿಶೇಷಾಂಶ:} ತಂತ್ರಸಾರೋಕ್ತ ರೀತಿಯಿಂದ ದೇವರ ಪೂಜೆಯಾಗಿ ಮಂಗಳಾರತಿ ಮಾಡಿದ ನಂತರ ಶಂಖಭ್ರಮಣವನ್ನು ಮಾಡಬೇಕು. ಶಂಖದಲ್ಲಿ ಕಲಶೋದಕವನ್ನು ತುಂಬಿ, ಮೇಲೆ ಒಂದು ತುಳಸೀದಳವನ್ನು ಇಟ್ಟು “ಓಂ ಇಮಾ ಆಪಃ ಶಿವತಮಾ ಇಮಾಃ ಸರ್ವಸ್ಯ ಭೇಷಜೀಃ~। ಇವಾ ರಾಷ್ಟ್ರ ಸ್ಯ ವರ್ಧನೀರಿಮಾ ರಾಷ್ಟ್ರಭೃತೋsಮೃತಾಃ” ಎಂಬ ಮಂತ್ರವನ್ನು ಹೇಳಿಕೊಂಡು ಪರಮಾತ್ಮನಿಗೆ ಮೂರು ಸಲ ಭ್ರಮಣ ಮಾಡಿ ಆ ನೀರನ್ನು ಬೇರೆ ಪಾತ್ರೆಯಲ್ಲಿ ಇಡಬೇಕು. ಅದನ್ನು ಪ್ರೋಕ್ಷಣೆ ಮಾಡಿಕೊಳ್ಳಬೇಕು. ಪ್ರಾಶನವಿಲ್ಲ.
\end{myquote}

\begin{verse}
\textbf{ಪೀಠಸ್ಥಃ ಪೂಜಿತಃ ಶಂಖೋ ಯತ್ರ ತಿಷ್ಠತಿ ನಾರದ~।}\\\textbf{ಪ್ರಯಾಗಕ್ಷೇತ್ರ ಸದೃಶಂ ತದ್ಗೃಹಂ ಪ್ರಾಹ ಕೇಶವಃ~।। ೨೨~।।}
\end{verse}

ನಾರದನೇ, ಪೂಜಿಸಲ್ಪಟ್ಟು, ಪೀಠದಲ್ಲಿರುವ ಶಂಖವು ಯಾರ ಮನೆಯಲ್ಲಿ ಇರುತ್ತದೆಯೋ ಆ ಮನೆಯು ಪ್ರಯಾಗ ಕ್ಷೇತ್ರಕ್ಕೆ ಸಮವೆಂದು ಸ್ವತಃ ಪರಮಾತ್ಮನೇ ಹೇಳಿರುತ್ತಾನೆ.

\begin{verse}
\textbf{ಯದ್ಗೃಹೇ ಕೇಶವಸ್ಯಾಗ್ರೇ ಪೂಜಿತೋಽಬ್ಧಿ ಸಮುದ್ಭವಃ~।}\\\textbf{ತೇನೇಷ್ಟಾಃ ಸಕಲಾ ಯಜ್ಞಾ ಸ್ತೇನ ಸರ್ವಮನುಷ್ಠಿ ತಮ್~।। ೨೩~।।}
\end{verse}

ಯಾವನು ತನ್ನ ಮನೆಯಲ್ಲಿ ಪರಮಾತ್ಮನ ಮುಂಭಾಗದಲ್ಲಿ ಶಂಖವನ್ನು ಪೂಜಿಸುತ್ತಾನೆಯೋ ಅಂತಹವನು ಸಮಸ್ತ ಯಜ್ಞಗಳನ್ನೂ ಮಾಡಿದಂತಾಗುತ್ತದೆ.

\begin{verse}
\textbf{ಯನ್ಮೂಲೇ ಸರ್ವವೇದಾಶ್ಚ ಯನ್ಮಧ್ಯೇ ಸರ್ವದೇವತಾಃ~।}\\\textbf{ಯದಗ್ರೇ ಸರ್ವತೀರ್ಥಾನಿ ತಸ್ಮಾತ್ ಶಂಖಂ ಪ್ರಪೂಜಯೇತ್~।। ೨೪~।। }
\end{verse}

\begin{verse}
\textbf{ಸಂಪ್ರಾಪ್ತೇ ಪಾತಕೇ ಘೋರೇ ಪ್ರಾಯಶ್ಚಿತ್ತವಿವರ್ಜಿತೇ~।}\\\textbf{ಮೂರ್ಧ್ನಿ ಶಂಖೋದಕಂ ಗ್ರಾಹ್ಯಮಿತಿ ಪ್ರಾಹುರ್ಮನೀಷಿಣಃ~।। ೨೫~।।}
\end{verse}

ಶಂಖದ ಮೂಲದಲ್ಲಿ ಸಮಸ್ತ ವೇದಾಭಿಮಾನಿ ದೇವತೆಗಳೂ, ಮಧ್ಯದಲ್ಲಿ ಸಮಸ್ತ ದೇವತೆಗಳೂ, ಮುಂಭಾಗದಲ್ಲಿ ಸಮಸ್ತ ತೀರ್ಥಾಭಿಮಾನಿ ದೇವತೆಗಳೂ ಸನ್ನಿಹಿತರಾಗಿರುವುದರಿಂದ ಶಂಖವನ್ನು ಅವಶ್ಯವಾಗಿ ಪೂಜಿಸಬೇಕು. ಪ್ರಾಯಶ್ಚಿತ್ತವಿಲ್ಲದ ಘೋರವಾದ ಪಾಪವನ್ನು ಮಾಡಿದರೆ ಶಂಖೋದಕವನ್ನು ಶಿರಸ್ಸಿನಲ್ಲಿ ಧಾರಣ ಮಾಡಬೇಕೆಂದು ಜ್ಞಾನಿಗಳು ಹೇಳುತ್ತಾರೆ.

\begin{verse}
\textbf{ಮಾ ಕಾಶೀ ಮಾ ಗಯಾ ಗಂಗಾ ಮಾ ಕಾಂಚೀ ಮಾ ಚ ಪುಷ್ಕರಂ~।}\\\textbf{ಮಾ ಕೇದಾರಂ ದ್ವಾರಕಾಂ ಚ ಮಾ ನದ್ಯೋ ಮಾ ಚ ಮಾನಸಮ್~।। ೨೬~।।} 
\end{verse}

\begin{verse}
\textbf{ತೇ ಸರ್ವೇ ಸೇವಿತಾ ಯತ್ರ ಶಾಲಗ್ರಾಮಶಿಲೋದಕಾತ್~।}\\\textbf{ಗಂಗಾದ್ಯಾಃ ಸರಿತಃ ಸರ್ವಾಃ ಸಾಗರಾಃ ಸಪ್ತ ನಾರದ~।। ೨೭~।।} \\\textbf{ಶಾಲಗ್ರಾಮಶಿಲಾತೀರ್ಥಕಲಾಂ ನಾರ್ಹಂತಿ ಷೋಡಶೀಮ್~।।}
\end{verse}

ಕಾಶೀ, ಗಯಾ, ಕಾಂಚೀ, ಪುಷ್ಕರ, ಕೇದಾರ, ದ್ವಾರಕಾ, ಮಾನಸ ಸರೋವರ ಇತರ ಎಲ್ಲ ಗಂಗಾದಿ ನದಿಗಳೂ ಸಪ್ತ ಸಮುದ್ರಗಳು ಇವೆಲ್ಲವೂ ಶಾಲಗ್ರಾಮ ತೀರ್ಥದಿಂದ ಲಭ್ಯವಾಗುತ್ತವೆ. ಶಾಲಗ್ರಾಮ ತೀರ್ಥದ ಹದಿನಾರರಲ್ಲಿ ಒಂದಂಶವೂ ಇವುಗಳಲ್ಲಿಲ್ಲ.

\begin{verse}
\textbf{ಮಾಘೇ ಮಾಸಿ ದ್ವಿಜಶ್ರೇಷ್ಠ ಶಾಲಗ್ರಾಮಶಿಲೋದಕೈಃ~।। ೨೮~।। }\\\textbf{ಶಿರಃ ಪ್ರಕ್ಷಾಲಯೇದ್ಯಸ್ತು ತಸ್ಯ ಪುಣ್ಯಫಲಂ ಶೃಣು~।}
\end{verse}

ನಾರದನೇ, ಮಾಘಮಾಸದಲ್ಲಿ ಶಾಲಗ್ರಾಮ ತೀರ್ಥದಿಂದ ತಲೆಯ ಮೇಲೆ ಪ್ರೋಕ್ಷಣೆ ಮಾಡಿಕೊಂಡರೆ ದೊರೆಯುವ ಪುಣ್ಯ ಫಲವನ್ನು ಕೇಳು.

\begin{verse}
\textbf{ಜನ್ಮಕೋಟಿಸಹಸ್ರೇಷು ಜನ್ಮಕೋಟಿಶತೇಷು ಚ~।। ೨೯~।।}\\\textbf{ಗಂಗಾದಿಸರ್ವತೀರ್ಥೇಷು ಸ್ನಾತೋ ಭವತಿ ನಾರದ~। }\\\textbf{ಆಜನ್ಮ ಕೃತಪಾಪಾನಿ ಜನ್ಮಾಂತರಕೃತಾನಿ ಚ~।। ೩೦~।।} \\\textbf{ಶಾಲಗ್ರಾಮೋದಕಂ ಮೂರ್ಧ್ನಿ ಭಸ್ಮಸಾದ್ಯಾಂತಿ ಬಿಭ್ರತಃ~।}
\end{verse}

ಸಹಸ್ರಕೋಟಿ ಶತಕೋಟಿ ಜನ್ಮಗಳಲ್ಲಿ ಗಂಗಾದಿ ಸಮಸ್ತ ತೀರ್ಥಗಳಲ್ಲಿ ಸ್ನಾನಮಾಡಿದ ಪುಣ್ಯ ಲಭ್ಯವಾಗುತ್ತದೆ. ಹಿಂದಿನ ಜನ್ಮಗಳಲ್ಲಿ ಹಾಗೂ ಈ ಜನ್ಮದಲ್ಲಿ ಮಾಡಿದ ಪಾಪಗಳು ಶಾಲಗ್ರಾಮ ತೀರ್ಥವನ್ನು ಶಿರಸ್ಸಿನಲ್ಲಿ ಧರಿಸುವುದರಿಂದ ಭಸ್ಮೀಭೂತವಾಗುತ್ತವೆ.

\begin{verse}
\textbf{ಭೋಜನಾನಂತರಂ ಯಸ್ತು ಶಾಲಗ್ರಾಮಶಿಲೋದಕಮ್~।। ೩೧~।।}\\\textbf{ಪೀತ್ವಾ ಪುಣ್ಯಮವಾಪ್ನೋತಿ ಚಾಂದ್ರಾಯಣಶತಾಧಿಕಮ್~। }\\\textbf{ಶಾಲಗ್ರಾಮಾರ್ಪಿತಾ ಯಸ್ತು ತುಳಸೀಂ ಭಕ್ಷಯೇದ್ಯದಿ~।। ೩೨~।।} \\\textbf{ತೇನ ಭುಕ್ತಂ ಪಂಚಗವ್ಯಂ ಕೋಟಿವಾರಂ ನ ಸಂಶಯಃ~।। ೩೩~।।}
\end{verse}

ಭೋಜನದ ನಂತರ ಯಾರು ಶಾಲಗ್ರಾಮ ತೀರ್ಥವನ್ನು ಪ್ರಾಶನಮಾಡುತ್ತಾರೆಯೋ ಅವರು ನೂರಕ್ಕೂ ಹೆಚ್ಚು ಚಾಂದ್ರಾಯಣ ವ್ರತವನ್ನು ಆಚರಣೆ ಮಾಡಿದ ಫಲವನ್ನು ಪಡೆಯುತ್ತಾರೆ. ಭೋಜನದ ನಂತರ ವಿಷ್ಣುವಿಗೆ ಅರ್ಪಿತವಾದ ತುಳಸೀ ದಳವನ್ನು ಭಕ್ಷಣ ಮಾಡಿದರೆ ಒಂದು ಕೋಟಿ ಬಾರಿ ಪಂಚಗವ್ಯವನ್ನು ಪ್ರಾಶನಮಾಡಿದಂತಾಗುತ್ತದೆ, ಸಂಶಯವಿಲ್ಲ.

\begin{verse}
\textbf{ಮಾಘಮಾಸೇ ತು ವಿಪ್ರೇಂದ್ರ ಯಃ ಕುರ್ಯಾತ್ ಪತ್ರಭೋಜನಂ~।}\\\textbf{ಸೋಽಶ್ವಮೇಧಫಲಂ ಯಾತಿ ವಿಷ್ಣುಲೋಕಂ ಚ ಗಚ್ಛತಿ~।। ೩೪~।। }
\end{verse}

\begin{verse}
\textbf{ಯಃ ಕುರ್ಯಾನ್ಮಾಘಮಾಸೇ ತು ಬ್ರಹ್ಮ ಪತ್ರೇಷು ಭೋಜನಮ್~।}\\\textbf{ಬ್ರಹ್ಮಲೋಕಂ ಸಮಾಸಾದ್ಯ ಮೋದತೇ ಬ್ರಹ್ಮವಚ್ಚಿರಮ್~।। ೩೫~।।}
\end{verse}

ನಾರದನೇ, ಯಾರು ಮಾಘಮಾಸದಲ್ಲಿ ಎಲೆಯಲ್ಲಿ ಭೋಜನಮಾಡುತ್ತಾರೆಯೋ ಅವರು ಅಶ್ವಮೇಧ ಯಜ್ಞದ ಫಲವನ್ನು ಹೊಂದಿ ವೈಕುಂಠ ಲೋಕಕ್ಕೆ ಹೋಗುತ್ತಾರೆ. ಯಾರು ಮಾಘಮಾಸದಲ್ಲಿ ಮುತ್ತುಗದ ಎಲೆಯಲ್ಲಿ ಭೋಜನ ಮಾಡುತ್ತಾರೆಯೋ ಅವರು ಬ್ರಹ್ಮಲೋಕಕ್ಕೆ ಹೋಗಿ ಬ್ರಹ್ಮದೇವರಂತೆ ಸುಖವನ್ನು ಅನುಭವಿಸುತ್ತಾರೆ.

\begin{verse}
\textbf{ಪ್ರಾತಃಸ್ನಾನಂ ಪೂಜನಂ ಮಾಧವಸ್ಯ}\\\textbf{ಗೀತಾಪಾಠೋ ವೈಷ್ಣವಾನಾಂ ಚ ಪೂಜಾ~। }\\\textbf{ಶಾಸ್ತ್ರಾಭ್ಯಾಸೋ ಭೋಜನಂ ಬ್ರಹ್ಮಪತ್ರೆ} \\\textbf{ಮಾಯಾಭರ್ತುರ್ಮೂರ್ತಿದಾನಂ ಚ ದದ್ಯಾತ್~।। ೩೬~।। }
\end{verse}

\begin{verse}
\textbf{ಸರ್ವೇಷಾಮೇವ ಧರ್ಮಾಣಾಮಾಶ್ರಮಾಣಾಂ ಚ ನಾರದ~।}\\\textbf{ಸ್ತ್ರೀಣಾಂ ಚ ಹೀನಜಾತೀನಾಮೇಷ ಧರ್ಮಃ ಸನಾತನಃ~।। ೩೭~।।}
\end{verse}

ಪ್ರಾತಃಕಾಲದಲ್ಲಿ ಸ್ನಾನ, ಮಾಧವನ ಪೂಜೆ, ಗೀತಾಪಾಠ, ವಿಷ್ಣು ಭಕ್ತರ ಸತ್ಕಾರ, ಸಚ್ಚಾಸ್ತ್ರಾಭ್ಯಾಸ, ಮುತ್ತುಗದ ಎಲೆಯಲ್ಲಿ ಭೋಜನ, ರಮಾಪತಿಯ ವಿಗ್ರಹ ದಾನವನ್ನು ಮಾಡುವುದು, ಎಲ್ಲ ಆಶ್ರಮದವರಿಗೂ, ಸ್ತ್ರೀಯರಿಗೂ, ಹೀನ ಜಾತಿಯಲ್ಲಿ ಉತ್ಪನ್ನನಾಗಿರುವವರಿಗೂ ಇದೇ ಸನಾತನ ಧರ್ಮವಾಗಿದೆ.

\begin{verse}
\textbf{ಪ್ರಾತಃಸ್ನಾನೇ ಮಾಧವಸ್ಯಾರ್ಚನೇನ~।}\\\textbf{ಯೇಷಾಂ ಬುದ್ಧಿರ್ಜಾಯತೇ ಸತ್ಕಥಾಸು~। }\\\textbf{ತೇಷಾಂ ಲೋಕಾಃ ಅಕ್ಷಯಾಃ ಕರ್ಮಲಭ್ಯಾಃ} \\\textbf{ಕುಲಂ ತೇಷಾಂ ಸ್ಥಾಪಿತಂ ವಿಷ್ಣುಲೋಕೇ~।। ೩೮~।।}
\end{verse}

ಮಾಘಮಾಸದಲ್ಲಿ ಪ್ರಾತಃಸ್ನಾನದಲ್ಲಿ, ಮಾಧವನ ಪೂಜೆಯಲ್ಲಿ, ಸತ್ಕಥಾ ಶ್ರವಣದಲ್ಲಿ ಯಾರಿಗೆ ಬುದ್ಧಿಯು ಹೋಗುತ್ತದೆಯೋ ಅವರಿಗೆ ಅವರ ಶ್ರದ್ಧಾನುಸಾರವಾಗಿ ಪುಣ್ಯ\break ದೊರೆಯುತ್ತದೆ ಮತ್ತು ಅವರ ಕುಲವು ವಿಷ್ಣು ಲೋಕದಲ್ಲಿ ವಾಸಿಸುತ್ತದೆ.

\begin{verse}
\textbf{ಯೇ ಪುಣ್ಯಹೀನಾ ಮನುಜಾಶ್ಚ ಪೂರ್ವಂ}\\\textbf{ನ ಶಿಕ್ಷಿತಾ ಯೇ ಗುರುಭಿರ್ವಿಶಿಷ್ಟೈಃ~। }\\\textbf{ತೇಷಾಂ ಮತಿರ್ಮಾಧವೇ ಮಾಘಮಾಸೇ} \\\textbf{ನ ಜಾಯತೇ ಕ್ವಾಪಿ ಕುತೋsನ್ಯಧರ್ಮೇ~।। ೩೯~।।}
\end{verse}

ಪುಣ್ಯಹೀನರಾದ, ಗುರುಗಳಿಂದ ಉಪದೇಶವಿಲ್ಲದ ಮನುಷ್ಯರಿಗೆ ಮಾಘಮಾಸ ವ್ರತದಲ್ಲಿ ಆಸಕ್ತಿ ಹುಟ್ಟುವುದೇ ಇಲ್ಲ; ಬೇರೆ ಇತರ ವೈಷ್ಣವ ಧರ್ಮಗಳಲ್ಲಿ ಹೇಗೆ ಪ್ರವೃತ್ತಿಯಾಗುತ್ತದೆ?

\begin{verse}
\textbf{ಯಃ ಕಾಂಸ್ಯಭುಗ್ಭವೇನ್ಮಾಘೇ ಸ ಜಂಬೂಕೋ ಭವೇದ್ ಧ್ರುವಮ್~।}\\\textbf{ತಸ್ಮಾತ್ ಕಾಂಸ್ಯೇ ನ ಭೋಕ್ತವ್ಯಂ ಮಾಘೇ ಮಾಸಿ ದ್ವಿಜೋತ್ತಮ~।। ೪೦~।।}
\end{verse}

ಮಾಘಮಾಸದಲ್ಲಿ ಕಂಚಿನ ಪಾತ್ರೆಯಲ್ಲಿ ಭೋಜನ ಮಾಡುವವನು ಮುಂದಿನ ಜನ್ಮದಲ್ಲಿ ನರಿಯಾಗಿ ಹುಟ್ಟುತ್ತಾನೆ. ಆದುದರಿಂದ, ನಾರದನೇ, ಮಾಘದಲ್ಲಿ ಕಂಚಿನ ಪಾತ್ರೆಯಲ್ಲಿ ಭೋಜನ ಮಾಡಬಾರದು.

\begin{verse}
\textbf{ತೈಲಾಭ್ಯಂಗಂ ತೈಲಭೋಗಂ ಮಾಘಮಾಸೇ ವಿವರ್ಜಯೇತ್~।}\\\textbf{ತೈಲಾಭ್ಯಂಗೇ ಕೃತೇ ವಿಪ್ರ ವಜ್ರವೃಕ್ಷೋ ಭವೇದ್‌ಧ್ರುವಮ್~।। ೪೦~।।}
\end{verse}

ಮಾಘಮಾಸದಲ್ಲಿ ತೈಲಾಭ್ಯಂಗವನ್ನೂ ತೈಲದಿಂದ ಕೇಶ ಶೃಂಗಾರವನ್ನೂ ಪ್ರಯತ್ನ ಪೂರ್ವಕವಾಗಿ ಬಿಡಬೇಕು. ಮಾಡಿದರೆ, ನಾರದ, ಅಂತಹವನು ಮುಂದಿನ ಜನ್ಮದಲ್ಲಿ ಕಡಲೇ ಗಿಡವಾಗಿ ಹುಟ್ಟುತ್ತಾನೆ.

\begin{verse}
\textbf{ಶೃಗಾಲೋ ಜಾಯತೇ ಭೂಮೌ ಮಾಘೇ ವೈ ತೈಲಸೇವನಾತ್~।। ೪೧~।।}
\end{verse}

ಮಾಘಮಾಸದಲ್ಲಿ ತೈಲಶೃಂಗಾರದಿಂದ ಮುಂದಿನ ಜನ್ಮದಲ್ಲಿ ನರಿಯಾಗಿ ಹುಟ್ಟುತ್ತಾನೆ.

\begin{verse}
\textbf{ಆರನಾಲಂ ಚ ಪಿಣ್ಯಾಕಂ ಯೋ ಮಾಘೇ ಮಾಸಿ ಭಕ್ಷತಿ~।। ೪೨~।।}
\end{verse}

\begin{verse}
\textbf{ಸುರಾಪಸ್ಯ ಚ ಲೋಕಂ ಚ ಸ ತು ಗಚ್ಛತಿ ನಿಶ್ಚಿತಮ್~।}\\\textbf{ಲಶುನಂ ಗೃಂಜನಂ ಛತ್ರಂ ವೃಂತಾಕಂ ಭಿಃಸಟಂ ತಥಾ~।। ೪೩~।। }
\end{verse}

\begin{verse}
\textbf{ಕೂಪೋದಕಮಲಾಂಬುಂ ಚ ಮಾಘೇ ಮಾಸಿ ವಿವರ್ಜಯೇತ್~।}\\\textbf{ಅತಸೀಪತ್ರಸಂಕಾಶಂ ಯೋ ಮಾಘೇ ಮಾಸಿ ಭಕ್ಷತಿ~।। ೪೪~।। }\\\textbf{ತಸ್ಯಾಪರಾಧಸಾಹಸ್ರಂ ಕ್ಷಮತೇ ವಿಷ್ಣು ರವ್ಯ ಯಃ~।}
\end{verse}

ಯಾವನು ಮಾಘ ಮಾಸದಲ್ಲಿ ತಂಗಳುಗಂಜೀ, ಎಳ್ಳಿನ ಹಿಂಡೀ ಇವುಗಳನ್ನು ಉಪಯೋಗಿಸುತ್ತಾನೆಯೋ ಅಂತಹವನು ಸುರಾಪಾನ ಮಾಡಿದವರು ಹೋಗುವ ಸ್ಥಳಕ್ಕೆ ನಿಶ್ಚಯವಾಗಿ ಹೋಗುತ್ತಾನೆ, ಬೆಳ್ಳೊಳ್ಳಿ, ಕೆಂಪು ಮೂಲಂಗಿ, ಬದನೇಕಾಯಿ, ಅಳಂಬೀ, ಕಮಲಕಂದ, ಬಾವಿಯ ನೀರು, ಕುಂಬಳಕಾಯಿ ಇವುಗಳನ್ನು ಮಾಘಮಾಸದಲ್ಲಿ ಉಪಯೋಗಿಸಬಾರದು. ಮಾಘದಲ್ಲಿ ಯಾವನು ಹಲಸಿನ ಎಲೆಯಲ್ಲಿ ಊಟಮಾಡುತ್ತಾನೆಯೋ ಅವನ ಸಹಸ್ರಾರು ಅಪರಾಧಗಳನ್ನೂ ನಾಶರಹಿತನಾದ ವಿಷ್ಣುವು ಕ್ಷಮಿಸುತ್ತಾನೆ.

\begin{verse}
\textbf{ಮಾಧವೇ ಮುನಿಪುಷ್ಪೈಸ್ತು ಯಃ ಪೂಜಯತಿ ಮಾಧವಮ್~।। ೪೫~।। }
\end{verse}

\begin{verse}
\textbf{ತೇನ ವರ್ಷಶತಂ ವಿಷ್ಣುಃ ಪೂಜಿತಶ್ಚ ನ ಸಂಶಯಃ~।}\\\textbf{ಯಃ ಪೂಜಯತಿ ದೇವೇಶಂ ಶತಪತ್ರೈರ್ಮನೋರಮೈಃ~।। ೪೬~।। }
\end{verse}

\begin{verse}
\textbf{ಪುಷ್ಪಂ ವಿಮಾನಮಾರುಹ್ಯ ಸ್ವರ್ಗಲೋಕೇ ಮಹೀಯತೇ~।}\\\textbf{ಯೋ ಧಾತ್ರೀಫಲಮಾಲಾಭಿರ್ಮಾಧವಂ ಪರಿಪೂಜಯೇತ್~।। ೪೭~।। }
\end{verse}

\begin{verse}
\textbf{ತಸ್ಯ ಪುತ್ರಾನ್ ಮಹಾಪ್ರಾಜ್ಞಾನ್ ಸ್ವಯಂ ಯಚ್ಛತಿ ಮಾಧವಃ~।}\\\textbf{ಯೋ ಧಾತ್ರೀಫಲಗೈರ್ದೀಪೈಃ ನೀರಾಜಯತಿ ಮಾಧವಮ್~।। ೪೮~।। }
\end{verse}

\begin{verse}
\textbf{ಮುಚ್ಯತೇ ನಾತ್ರ ಸಂದೇಹೋ ಮಹಾಪಾತಕಕೋಟಿಭಿಃ~।}
\end{verse}

ಮಾಘಮಾಸದಲ್ಲಿ ಯಾರು ಶ‍್ರೀ ಮಾಧವನನ್ನು ಅಗಸೆ ಹೂಗಳಿಂದ ಪೂಜಿಸುತ್ತಾನೆಯೋ ಅಂತಹವನು ನೂರು ವರ್ಷಗಳ ಪರಮಾತ್ಮನನ್ನು ಪೂಜಿಸಿದಂತೆ ಆಗುತ್ತದೆ. ಯಾವನು ಸರ್ವೊತ್ತಮನಾದ ವಿಷ್ಣುವನ್ನು ಮನೋಹರವಾದ ಕಮಲಗಳಿಂದ ಪೂಜಿಸುತ್ತಾನೆಯೋ ಅವನು ಪುಷ್ಪಕವಿಮಾನವನ್ನೇರಿ ಸ್ವರ್ಗಲೋಕದಲ್ಲಿ ಸುಖಪಡುತ್ತಾನೆ. ಯಾರು ನೆಲ್ಲಿಕಾಯಿಯಿಂದ ಮಾಡಿದ ಹಾರದಿಂದ ಪೂಜಿಸುತ್ತಾನೆಯೋ ಅವನಿಗೆ ಶ‍್ರೀ ಮಾಧವನು ಬಹಳ ಬುದ್ದಿವಂತರಾದ ಮಕ್ಕಳನ್ನು ನೀಡುತ್ತಾನೆ. ಯಾವನು ನೆಲ್ಲಿಕಾಯಿ ಸಹಿತ ಮಂಗಳಾರತಿ ಮಾಡುತ್ತಾನೆಯೋ ಅವನು ನೂರು ಕೋಟಿ ಪಾಪಗಳಿಂದ ವಿಮುಕ್ತನಾಗುತ್ತಾನೆ, ಸಂದೇಹವಿಲ್ಲ.

\begin{verse}
\textbf{ಅಲಿಪ್ಯ ಧಾತ್ರೀಪಿಷ್ಟೇನ ಯಃ ಸ್ನಾನಂ ಕರ್ತುಮಿಚ್ಛತಿ~।। ೪೯~।।} 
\end{verse}

\begin{verse}
\textbf{ತಾವನ್ಮಾ ತ್ರೇಣ ಮುಚ್ಯಂತೇ ನರಾ ದುಷ್ಕರಕಿಲ್ಬಿಷೈಃ~।}\\\textbf{ಯೋ ಧಾತ್ರೀಫಲರಾಶಿಸ್ತು ಬ್ರಾಹ್ಮಣಾಯ ಪ್ರಯಚ್ಛತಿ~।। ೫೦~।।}
\end{verse}

\begin{verse}
\textbf{ಸ ಪುತ್ರಾನ್ ಪಾಪಹಾನೀಂ ಚ ಲಭತೇ ನಾತ್ರ ಸಂಶಯಃ~।}\\\textbf{ಧಾತ್ರಿಫಲಾನಿ ಯೋ ಮಾಘೇ ಬ್ರಾಹ್ಮಣೇಭ್ಯೋ ವಿಸರ್ಜಯೇತ್~।।}\\\textbf{ನ ತಸ್ಯ ಸಂತತೇರ್ಹಾನಿರ್ಭುಕ್ತಿಂ ಮುಕ್ತಿಂ ಚ ವಿಂದತಿ~।। ೫೧~।।}
\end{verse}

ನೆಲ್ಲಿಕಾಯಿಯ ರಸ ಅಥವ ಸಿಪ್ಪೆಯನ್ನು ಲೇಪಿಸಿಕೊಂಡು ಸ್ನಾನಮಾಡಲು ಯಾವನು ಇಚ್ಚಿಸುತ್ತಾನೆಯೋ ಅಂತಹವನು ಆ ಇಚ್ಚಾ ಮಾತ್ರದಿಂದಲೇ ಸಕಲ ಪಾಪಗಳಿಂದ ಮುಕ್ತನಾಗುತ್ತಾನೆ. ಯಾರು ನೆಲ್ಲಿಕಾಯಿಯ ರಾಶಿಯನ್ನು ಬ್ರಾಹ್ಮಣನಿಗೆ ದಾನ ಮಾಡುತ್ತಾನೋ ಅವನು ಸತ್ಪುತ್ರರನ್ನು ಪಡೆದು ಪಾಪವಿಮೋಚನೆಯನ್ನೂ ಹೊಂದುತ್ತಾನೆ, ಸಂಶಯವಿಲ್ಲ. ಯಾರು ಬ್ರಾಹ್ಮಣರಿಗೆ ಮಾಘಮಾಸದಲ್ಲಿ ನೆಲ್ಲಿಕಾಯಿಯನ್ನು ದಾನಮಾಡುತ್ತಾನೋ ಅವನಿಗೆ ಎಂದಿಗೂ ಸಂತತಿ ಹಾನಿಯಾಗುವುದಿಲ್ಲ; ಭುಕ್ತಿ ಮುಕ್ತಿಗಳು ದೊರೆಯುತ್ತವೆ.

\begin{verse}
\textbf{ಯೋ ಮಾಘೇಮಾಸಿ ಸಪ್ತಮ್ಯಾಂ ಶುಕ್ಲಾ ಯಾಂ ರವಿವಾಸರೇ~।}\\\textbf{ಕೂಷ್ಮಾಂಡಂ ತಿಲಗವ್ಯಾಢ್ಯಂ ಬ್ರಹ್ಮಣಾ ನಿರ್ಮಿತಂ ಪುರಾ~।। ೫೨~।। }
\end{verse}

\begin{verse}
\textbf{ತಸ್ಮಾದಸ್ಯ ಪ್ರದಾನೇನ ಸಂತತಿರ್ವರ್ಧತಾಂ ಮಮ~।}\\\textbf{ಇತಿ ಮಂತ್ರೇಣಾಸ್ಯಫಲಂ ಸಂತತಿಃ ಶತಪೌರುಷೀ~।। ೫೩~।। }
\end{verse}

\begin{verse}
\textbf{ವರ್ಧತೇ ನಾತ್ರ ಸಂದೇಹೋ ವೈಶ್ಯಾನಾಂ ಯೋಷಿತಾಮಪಿ~।}\\\textbf{ಮಾಘೇ ತು ರಥಸಪ್ತಮ್ಯಾಂ ಕೂಷ್ಮಾಂಡಸ್ಯ ಪ್ರಧಾನತಃ~।। ೫೪~।। }
\end{verse}

\begin{verse}
\textbf{ವಿವೃದ್ಧಿಃ ಸಂತತೇಸ್ತಸ್ಯ ಸ ಕೂಷ್ಮಾಂಡೋ ಭವೇದ್ಭುವಿ~।। ೫೫~।।}
\end{verse}

ಮಾಘ ಶುದ್ಧ ಸಪ್ತಮೀ ಭಾನುವಾರದ ದಿನ ಯಾರು “ಕೂಷ್ಮಾಂಡಂ ತಿಲಗವ್ಯಾಢ್ಯಂ ಬ್ರಹ್ಮಣಾ ನಿರ್ಮಿತುಂ ಪುರಾ~। ತಸ್ಮಾದಸ್ಯ ಪ್ರದಾನೇನ ಸಂತತಿರ್ವರ್ಧತಾಂ ಮಮ~।।” ಈ ಮಂತ್ರದಿಂದ ಕುಂಬಳಕಾಯಿಯನ್ನು ದಾನ ಮಾಡುತ್ತಾರೆಯೋ ಅಂತಹವನು ಶತಪುರುಷಗಳವೆರಗೂ, ಉತ್ತಮ ಸಂತತಿಯನ್ನು ಪಡೆಯುವನು, ವೈಶ್ಯ ಸ್ತ್ರೀಯರ ಸಂತತಿಯೂ ಹಾಗೆಯೇ ಅಭಿವೃದ್ಧಿಯಾಗುವುದು, ಸಂಶಯವಿಲ್ಲ. ಮಾಘಮಾಸದಲ್ಲಿ ರಥಸಪ್ತಮಿ ದಿನ ಕುಂಬಳ\-ಕಾಯನ್ನು ದಾನ ಮಾಡುವುದರಿಂದ ಅವನ ಸಂತತಿಯು ಕುಂಬಳಕಾಯಿಯು ವೃದ್ಧಿಯಾಗುವಂತೆ ಅಭಿವೃದ್ಧಿಯನ್ನು ಹೊಂದುತ್ತದೆ.

\begin{verse}
\textbf{ಪುರಾ ಮೇರೌ ತಪಃ ಕುರ್ವನ್ ಬಭಾಷೇ ಪಿತೃದೇವತಾಃ~।}\\\textbf{ಸ್ಥಿ ತೋಹಂ ತಪಸೇ ಮಹ್ಯಂ ಶುಶ್ರೂಷಾತ್ರ ವಿಧೀಯತಾಮ್~।। ೫೬~।।}
\end{verse}

ಹಿಂದೆ ನಾನು ಮೇರು ಪರ್ವತದಲ್ಲಿ ತಪಸ್ಸನ್ನು ಆಚರಿಸುತ್ತಿದ್ದಾಗ ಪಿತೃ ದೇವತೆಗಳಿಗೆ ನನ್ನ ಸೇವೆ ಮಾಡಬೇಕೆಂದು ಹೇಳಿದೆ.

\begin{verse}
\textbf{ಇತ್ಯುಕ್ತೋ ಬ್ರಹ್ಮಣಾನೇನ ಪ್ರತ್ಯೂಚುಃ ಪಿತೃದೇವತಾಃ~।}\\\textbf{ವಯಂ ಮೋಕ್ಷಾಯ ತಪಸಾ ಹರೇರಾಧನೋದ್ಯ ತಾಃ~।। ೫೭~।।}
\end{verse}

\begin{verse}
\textbf{ಅತೋsಸ್ಮಾಭಿಸ್ತು ಶುಶ್ರೂಷಾ ನ ಶಕ್ಯಾ ಚರಿತುಂ ತವ~।}\\\textbf{ಇತ್ಯುಕ್ತೇ ಪಿತೃಭಿಃ ಸೋಪಿ ಶಶಾಪಾತೀವ ಕೋಪಿತಃ~।। ೫೮~।।}
\end{verse}

ಪಿತೃ ದೇವತೆಗಳು "ನಾವು ಮೋಕ್ಷಕ್ಕೊಸ್ಕರ ಶ‍್ರೀಹರಿಯನ್ನು ಕುರಿತು ತಪಸ್ಸನ್ನು ಮಾಡುತ್ತಿದ್ದೇವೆಯಾದುದರಿಂದ ನಿಮ್ಮ ಸೇವೆಯನ್ನು ಮಾಡಲು ಸಾಧ್ಯವಿಲ್ಲ" ಎಂದರು. ಬ್ರಹ್ಮದೇವರು ಕೋಪದಿಂದ ಪಿತೃ ದೇವತೆಗಳಿಗೆ ಶಾಪವನ್ನು ಕೊಟ್ಟರು:

\begin{verse}
\textbf{ಸಂತತ್ಯಾ ವಾ ಭವೇನ್ಮೋಕ್ಷೋ ನ ಸ್ವಾತಂತ್ರ್ಯಾತ್ಕದಾಚನ~।}\\\textbf{ಅಭಾವೇ ಸಂತತೇರ್ಯೂಯಮಧಃಪತನಮಾಪ್ಸ್ಯಥ~।। ೫೯~।। }
\end{verse}

\begin{verse}
\textbf{ದತ್ತೇ ಶಾಪೇ ಕಂಜಜೀನ ಸಾಧ್ವಸಂ ತಮಯಾಚಯನ್~।}\\\textbf{ವಂಶ್ಯೈಃ ಕೃತೇನ ದೋಷೇಣ ವಿಚ್ಛಿ ತ್ತಿಃ ಸಂತತೇರ್ಭವೇತ್~।। ೬೦~।। }
\end{verse}

\begin{verse}
\textbf{ದಯಾಲೋ ಗತಿಮಸ್ಮಾಕಂ ತ್ವಂ ತಸ್ಮಾದ್ವದ ಸತ್ವರಮ್~।}\\\textbf{ಏವಂ ವಿಜ್ಞಾಪಿತೋ ಬ್ರಹ್ಮಾ ಪ್ರತ್ಯುವಾಚ ಪಿತೄನಥ~।। ೬೧~।।}
\end{verse}

ನಿಮ್ಮ ಸ್ವಾತಂತ್ರ್ಯದಿಂದ ನಿಮಗೆ ಎಂದಿಗೂ ಮೋಕ್ಷವಾಗುವುದಿಲ್ಲ. ನಿಮ್ಮ ಸಂತತಿಯಿಂದ ನಿನಗೆ ಮೋಕ್ಷವು ಲಭಿಸುತ್ತದೆ. ಸಂತತಿಯಿಲ್ಲದಿದ್ದರೆ ನೀವು ಅಧಃಪತನ ಹೊಂದುವಿರಿ. ಹೀಗೆ ಬ್ರಹ್ಮದೇವರಿಂದ ಶಾಪವು ಕೊಡಲ್ಪಡುತ್ತಿರಲು, ಪಿತೃ ದೇವತೆಗಳು ಭಯದಿಂದ ಯಾಚಿಸಿದರು: ದಯಾಸಾಗರನೇ, ವಂಶದವರ ದೋಷದಿಂದ ಸಂತತಿಯಲ್ಲಿ ವಿಚ್ಛತ್ತಿಯುಂಟಾಗುತ್ತದೆ. ದಯಾಲು\-ಗಳಾದ ನೀವು ನಮಗೆ ಗತಿಯಾಗುವ ವಿಧಾನವನ್ನು ಶೀಘ್ರವಾಗಿ ಹೇಳಿರಿ. ಹೀಗೆ ವಿಜ್ಞಾಪಿತರಾದ ಬ್ರಹ್ಮದೇವರು ಪಿತೃಗಳನ್ನು ಕುರಿತು ಈ ರೀತಿ ಹೇಳಿದರು:

\begin{verse}
\textbf{ನಾನೃತಂ ಮೇ ವಚೋ ಭೂಯಾದಥವಾ ಗತಿಮನ್ಯಥಾ~।}\\\textbf{ಕೂಷ್ಮಾಂಡಂ ರಥಸಪ್ತಮ್ಯಾಂ ಯೋ ದದಾತಿ ದ್ವಿಜಾತಯೇ~।। ೬೨~।। }
\end{verse}

\begin{verse}
\textbf{ನ ತಸ್ಯ ಸಂತತಿಚ್ಛೇದೋ ಭವೇದಿತಿ ವರಂ ದದೌ~।}\\\textbf{ಕೂಷ್ಮಾಂಡಾನಾಮದಾನೇನ ವಿಚ್ಛಿ ತ್ತಿಃ ಸ್ಯಾಚ್ಚ ಸಂತತೇಃ~।। ೬೩~।। }
\end{verse}

\begin{verse}
\textbf{ಇತ್ಯಾಶ್ವಾ ಸ್ಯ ಪಿತಾನ್ದೇವಾನ್ ಬ್ರಹ್ಮಾ ಲೋಕಪಿತಾಮಹಃ~।}\\\textbf{ತೇಷಾಮರ್ಥೆ ಚ ಕೂಷ್ಮಾಂಡಂ ದಾನಾರ್ಥಮಸೃಜದ್ವಿಭುಃ~।। ೬೪~।।}
\end{verse}

ನನ್ನ ಮಾತು ಎಂದೆಂದಿಗೂ ಸುಳ್ಳಾಗುವುದಿಲ್ಲ. ರಥಸಪ್ತಮಿಯಲ್ಲಿ ಯಾರು ಕುಂಬಳ\-ಕಾಯನ್ನು ಬ್ರಾಹ್ಮಣನಿಗೆ ದಾನಮಾಡುತ್ತಾನೋ ಅವರಿಗೆ ಸಂತತಿಯಲ್ಲಿ ತಡೆಬರುವುದಿಲ್ಲವೆಂಬ ವರವನ್ನು ಕೊಟ್ಟರು. ಕುಂಬಳಕಾಯಿ ದಾನ ಮಾಡದಿದ್ದರೆ ಸಂತತಿಯಲ್ಲಿ ವಿಚ್ಛಿತ್ತಿಯು ಬರುತ್ತದೆ. ಲೋಕಪಿತಾಮಹರಾದ ಬ್ರಹ್ಮದೇವರು ಈ ರೀತಿ ಹೇಳಿ ಅವರಿಗೋಸ್ಕರ ಕುಂಬಳಕಾಯಿಯನ್ನು ಸೃಷ್ಟಿಸಿದರು.

\begin{verse}
\textbf{ಪ್ರವಾಲಮುಕ್ತಾಸಹಿತಂ ತಿಲಗವ್ಯೇನ ಲೇಪಿತಮ್~।}\\\textbf{ಸದಕ್ಷಿಣಂ ಸತಾಂಬೂಲಂ ದೇಯಂ ಸಂತತಿಮಿಚ್ಛತಾ~।। ೬೫~।।}
\end{verse}

ಸಂತತಿ ಇಚ್ಛೆಯುಳ್ಳವರು ಹಾಲು-ಮೊಸರು-ಎಳ್ಳುಗಳಿಂದ ಲೇಪನ ಮಾಡಿ, ಮುತ್ತುಗಳಿಂದ ಅಲಂಕರಿಸಿ ದಕ್ಷಿಣೆ ತಾಂಬೂಲ ಸಹಿತ ಬ್ರಾಹ್ಮಣನಿಗೆ ದಾನ ಕೊಡಬೇಕು.

\begin{verse}
\textbf{ಅದಾನೇ ರಥಸಪ್ತಮ್ಯಾಂ ಕೂಷ್ಮಾಂಡಸ್ಯ ದ್ವಿಜಾತಯೇ~।}\\\textbf{ಮಾತುಲಿಂಗಾನಿ ಸಂದದ್ಯಾತ್ಪುತ್ರಾನ್ ಭಾಗ್ಯಮಥೋ ಲಭೇತ್~।। ೬೬~।।} 
\end{verse}

\begin{verse}
\textbf{ಅದಾನಮವ್ರತಂ ಯಸ್ತು ಮಾಸಮೇತಂ ವ್ಯತಿಕ್ರಮೇತ್~।}\\\textbf{ಸ ಜೀವನ್ನೇವ ಚಾಂಡಾಲಃ ಪಶ್ಚಾದ್ರೌರವಮಶ್ನುತೇ~।। ೬೭~।। }
\end{verse}

\begin{verse}
\textbf{ಅಸ್ನಾತ್ವಾ ಚಾಪ್ಯದತ್ವಾ ಚ ಮಾಸಮೇನಂ ನಯೇದ್ಯದಿ~।}\\\textbf{ಸ್ವರ್ಗಸ್ಥೈಃ ಪಿತೃಭಿಃ ಸಾರ್ಧಂ ರೌರವಂ ಕಲ್ಪ ಮಶ್ನು ತೇ~।। ೬೮~।।}
\end{verse}

ರಥಸಪ್ತಮಿ ದಿನದಲ್ಲಿ ಬ್ರಾಹ್ಮಣನಿಗೆ ಕುಂಬಳಕಾಯಿಯನ್ನು ದಾನ ಕೊಡಲಾಗದಿದ್ದಲ್ಲಿ ಮಾದಳ ಫಲವನ್ನಾದರೂ ದಾನ ಕೊಟ್ಟರೆ ಸತ್ಪುತ್ರರನ್ನೂ, ಐಶ್ವರ್ಯವನ್ನೂ ಹೊಂದುತ್ತಾನೆ. ಮಾಘಮಾಸದಲ್ಲಿ ವಿಧಿಸಿರುವ ಕರ್ಮಗಳನ್ನು ಮಾಡದೇ, ದಾನಗಳನ್ನೂ ಕೊಡದೇ ಇರುವವನು ಚಾಂಡಾಲನಾಗಿ ಹುಟ್ಟುತ್ತಾನೆ ಮತ್ತು ಕಡೆಯಲ್ಲಿ ರೌರವವೆಂಬ ನರಕಕ್ಕೆ ಹೋಗುತ್ತಾನೆ. ಮಾಘಮಾಸದಲ್ಲಿ ಉಷಃ ಕಾಲದ ಸ್ನಾನವನ್ನು ಮಾಡದೇ, ದಾನ ಧರ್ಮಗಳನ್ನೂ ಮಾಡದೇ ವ್ಯರ್ಥವಾಗಿ ಕಾಲ ಕಳೆಯುವವನು ಸ್ವರ್ಗದಲ್ಲಿರುವ ತನ್ನ ಪಿತೃಗಳಿಂದ ಸಹಿತನಾಗಿ ರೌರವ ನರಕವನ್ನು ಹೊಂದುತ್ತಾನೆ.

\begin{verse}
\textbf{ಮಾಸಿ ಮಾಘೇ ಮಾಧವಾಯ ಯಃ ಕುರ್ಯಾತ್ ಪುಷ್ಪಮಂಡಪಮ್~।}\\\textbf{ಶ್ವೇತದ್ವೀಪೇ ವಸೇನ್ನಿತ್ಯಂ ದಶಪೂರ್ವೈಃ ದಶಾಪರೈಃ~।। ೬೯~।।}
\end{verse}

\begin{verse}
\textbf{ಯಃ ಚೂತಪಲ್ಲವೈರ್ವಿಷ್ಣೋಸ್ತೋರಣಂ ವಿದಧಾತಿ ಸಃ~।}\\\textbf{ತಪೋಲೋಕಂ ಕುಲಶತೈರ್ಯುಕ್ತೋಸೌ ಗಚ್ಛತಿ ಧ್ರುವಮ್~।। ೭೦~।।}
\end{verse}

ಮಾಘಮಾಸದಲ್ಲಿ ಯಾರು ಮಾಧವನ ಪೂಜೆಗಾಗಿ ಹೂವಿನ ಮಂಡಪವನ್ನು ನಿರ್ಮಿಸುತ್ತಾನೆಯೋ ಅವನು ಹಿಂದಿನ ಹತ್ತು ಮುಂದಿನ ಹತ್ತು ಕುಲದವರೊಡನೆ ನಿರಂತರ ಶ್ವೇತದ್ವೀಪದಲ್ಲಿ ವಾಸಿಸುವನು. ಮಾವಿನ ಚಿಗುರೆಲೆಗಳಿಂದ ಪರಮಾತ್ಮನಿಗೆ ತೋರಣ ನಿರ್ಮಿಸುವವನು ನೂರು ಕುಲಗಳಿಂದ ಯುಕ್ತನಾಗಿ ತಪೋಲೋಕಕ್ಕೆ ಹೋಗುವುದು ನಿಶ್ಚಯ.

\begin{verse}
\textbf{ಯಃ ಪೂಜಾಂ ಕದಲೀಸ್ತಂಭೈರ್ವಿದಧಾತಿ ರಮಾಪತೇಃ~।}\\\textbf{ಕಮಲಾಸನಲೋಕಂ ಚ ಪಿತೃಭಿಃ ಸಹ ಗಚ್ಛತಿ~।। ೭೦~।। }
\end{verse}

\begin{verse}
\textbf{ಕರೋತ್ಯಭ್ಯಂಜನಂ ವಿಷ್ಣೋಃ ತೈಲೈಶ್ಚಂಪಕಗಂಧಿಭಿಃ~।}\\\textbf{ತಸ್ಯ ಪುಣ್ಯಫಲಂ ವಕ್ತುಂ ನಾಲಂ ವರ್ಷಶತೈರಪಿ~।। ೭೨~।।}
\end{verse}

ಯಾರು ಬಾಳೆಯಕಂಭಗಳಿಂದ ಅಲಂಕರಿಸಿ ರಮಾವಲ್ಲಭನನ್ನು ಪೂಜಿಸುತ್ತಾನೆಯೋ ಅವನು ಪಿತೃಗಳಿಂದ ಸಹಿತನಾಗಿ ಬ್ರಹ್ಮಲೋಕಕ್ಕೆ ಹೋಗುತ್ತಾನೆ. ಸಂಪಿಗೆ ಹೂ ಮುಂತಾದ ಸುಗಂಧ ಮಿಶ್ರಿತವಾದ ತೈಲದಿಂದ ವಿಷ್ಣುವಿಗೆ ಅಭ್ಯಂಜನವನ್ನು ಮಾಡಿಸುವವನ ಪುಣ್ಯಫಲವನ್ನು ನೂರು ವರ್ಷಗಳ ಕಾಲ ಹೇಳಲು ಸಮರ್ಥನಲ್ಲ.

\begin{verse}
\textbf{ವಾಸೋದಾನಂ ತು ಯಃ ಕುರ್ಯಾತ್ಪ್ರಶಾಂತಾಯ ದ್ವಿಜಾತಯೇ~।}\\\textbf{ಸ ಚೈ ಕವಿಂಶತಿಕುಲೈಃ ಶ್ವೇತದ್ವೀಪೇ ವಸೇಚ್ಚಿರಮ್~।। ೭೩~।। }
\end{verse}

\begin{verse}
\textbf{ಯೋ ದದ್ಯಾತ್ಕಂಬಲಂ ಮಾಘೇ ಶೀತಾರ್ತಾಯ ದ್ವಿಜಾತಯೇ~।}\\\textbf{ಸಾರ್ವಭೌಮೋ ಭವತ್ಯೇಷಃ ಕೋಟಿಜನ್ಮಸು ನಾರದ~।। ೭೪~।।} 
\end{verse}

\begin{verse}
\textbf{ಅನ್ನದಾನಂ ತು ಯಃ ಕುರ್ಯಾತ್ ಶ್ರೋತ್ರಿಯಾಯ ಕುಟುಂಬಿನೇ~।}\\\textbf{ಅನ್ನವಾನ್ ಜಾಯತೇ ಕಾಮೀ ನಿಷ್ಕಾಮೋ ಮೋಕ್ಷಮಾಪ್ನುಯಾತ್~।। ೭೫~।।}
\end{verse}

ಶಾಂತ ಸ್ವಭಾವದ ಬ್ರಾಹ್ಮಣನಿಗೆ ಯಾರು ವಸ್ತ್ರ ದಾನಮಾಡುತ್ತಾರೆಯೋ, ಅವರು ಇಪ್ಪತ್ತೊಂದು ಕುಲದವರೊಡನೆ ಚಿರಕಾಲ ಶ್ವೇತದ್ವೀಪದಲ್ಲಿ ವಾಸಮಾಡುತ್ತಾರೆ. ಮಾಘಮಾಸದಲ್ಲಿ ಚಳಿಯಿಂದ ಪೀಡಿತನಾದ ಬ್ರಾಹ್ಮಣನಿಗೆ ಕಂಬಳಿಯನ್ನು ಕೊಡುತ್ತಾನೆಯೋ, ಅವನು ಕೋಟಿ ಜನ್ಮಗಳಲ್ಲಿ ಚಕ್ರವರ್ತಿಯಾಗಿರುತ್ತಾನೆ. ನಾರದ! ವೇದಾಧ್ಯಯನ ಮಾಡಿರುವ ಕುಟುಂಬಯುಕ್ತನಾದ ಬ್ರಾಹ್ಮಣನಿಗೆ ಯಾರು ಅನ್ನ ದಾನ ಮಾಡುತ್ತಾರೆಯೋ, ಅವರು ಐಹಿಕ ಫಲಾಪೇಕ್ಷೆಯಿಂದ ಇದ್ದರೆ ಐಹಿಕ ಸುಖವನ್ನು ಪಡೆಯುತ್ತಾರೆ; ನಿಷ್ಕಾಮನಾಗಿದ್ದರೆ ಮೋಕ್ಷವನ್ನು ಹೊಂದುತ್ತಾನೆ.

\begin{verse}
\textbf{ಮಿಷ್ಟಾನ್ನಂ ಭೋಜಯೇದ್ಯಸ್ತು ವ್ಯಾಕುರ್ವಾಣಂ ಹರೇಃ ಕಥಾಮ್~।}\\\textbf{ನ ತಸ್ಯ ಲೋಕಾಃ ಕ್ಷೀಯಂತೇ ಕಲ್ಪಕೋಟಿಶತೈರಪಿ~।। ೭೬~।।}
\end{verse}

ಬ್ರಾಹ್ಮಣನಿಗೆ ಮೃಷ್ಟಾನ್ನ ಭೋಜನಮಾಡಿಸಿ ಶ‍್ರೀಹರಿಯ ಮಾಹಾತ್ಮ್ಯವನ್ನು ಶ್ರವಣ ಮಾಡಿಸುತ್ತಾನೆಯೋ ಅವನ ಕುಟುಂಬದ ಜನರು ನೂರು ಕೋಟಿ ಕಲ್ಪಗಳಲ್ಲಿಯೂ ನಾಶವಾಗುವುದಿಲ್ಲ.

\begin{verse}
\textbf{ತಂ ಪೂಜಯಿತ್ವಾ ಧನವಸ್ತ್ರದಾನೈ\enginline{-}}\\\textbf{ರ್ಗೋಭೂಗಜಾದ್ಯೈರಪಿ ವಿಷ್ಣು ತೃಪ್ತ್ಯೈ~। }\\\textbf{ಭುಕ್ತ್ವಾ ಚ ಭೋಗಾನ್ ಸಕಲಾನ್ ಸಯಾತಿ} \\\textbf{ಪದಂ ಹರೇರೇವ ದುರಾಪಮನ್ಯೈಃ~।। ೭೭~।।}
\end{verse}

ಶ‍್ರೀಹರಿ ಪ್ರೀತಿಗಾಗಿ ಧನ, ವಸ್ತ್ರ, ಗೋ, ಭೂಮಿ, ಗಜ ಇತ್ಯಾದಿಗಳನ್ನು ಪೂಜಿಸಿ ದಾನ ಮಾಡಿದರೆ ಸಕಲ ಐಹಿಕ ಭೋಗಗಳನ್ನೂ ಅನುಭವಿಸಿ ಕೊನೆಯಲ್ಲಿ ದುರ್ಲಭವಾದ ಹರಿಪದವನ್ನು ಹೊಂದುವನು.

\begin{verse}
\textbf{ಅಪೂಜಯಿತ್ವಾ ಗುರುಮಗ್ರಬುದ್ಧ್ಯಾ}\\\textbf{ಧರ್ಮಪ್ರವಕ್ತಾರಮನಂತಬುದ್ಧಿಮ್~।}\\\textbf{ಛಿತ್ವಾ ಚ ಕರ್ಣೌ ರವಿಸೂನುರುಗ್ರಃ} \\\textbf{ತಂ ಪಾಪಯಂತ್ಯ ಧತಮಸ್ಯ ಪಾರೇ~।। ೭೮~।।}
\end{verse}

ಸದುಪದೇಶವನ್ನು ಮಾಡುವ ಗುರುಗಳನ್ನು ಯಾರು ಕೆಟ್ಟ ಬುದ್ಧಿಯಿಂದ ಯುಕ್ತರಾಗಿ ಪೂಜಿಸುವುದಿಲ್ಲವೋ ಅಂತಹವರನ್ನು ಯಮಧರ್ಮರಾಜನು ಕಿವಿಗಳನ್ನು ಕತ್ತರಿಸಿ ಅತಿ ಭಯಂಕರವಾದ ಕತ್ತಲೆಯ ನರಕದಲ್ಲಿ ದೂಡುತ್ತಾನೆ.

\begin{verse}
\textbf{ಯೋ ನಿಂದಯತ್ಯಯಂ ಮೂಢಃ ಪರೋಕ್ತಾಂ ತಾಂ ಶೃಣೋತಿ ವಾ~।}\\\textbf{ವಾಸೋಸ್ಯಾಂಧೇ ಚ ತಮಸಿ ನಿಷ್ಕೃತಿರ್ನಾಸ್ಯ ವರ್ತತೇ~।। ೭೯~।।}
\end{verse}

ಯಾವ ಮೂಢನು ಶ‍್ರೀಹರಿ ಮಾಹಾತ್ಮ್ಯೆಯನ್ನು ಓದುತ್ತಿದ್ದವರಿಂದ ಕೇಳಿ ನಿಂದಿಸುತ್ತಾನೆಯೋ ಅಂತಹವನು ಭಯಂಕರವಾದ ಕತ್ತಲೆಯಿಂದ ಕೂಡಿದ ನರಕದಲ್ಲಿ ಬೀಳುತ್ತಾನೆ. ಅಲ್ಲಿಂದ ಅವನಿಗೆ ಬಿಡುಗಡೆಯೇ ಇಲ್ಲ.

\begin{center}
ಇತಿ ಶ‍್ರೀ ವಾಯುಪುರಾಣೇ ಮಾಘಮಾಸಮಾಹಾತ್ಮ್ಯೇ ಚತುರ್ಥೋsಧ್ಯಾಯಃ
\end{center}

\begin{center}
 ಶ‍್ರೀ ವಾಯುಪುರಾಣಾಂತರ್ಗತ ಮಾಘಮಾಸ ಮಾಹಾತ್ಮ್ಯೆಯಲ್ಲಿ \\ ನಾಲ್ಕನೇ ಅಧ್ಯಾಯವು ಸಮಾಪ್ತಿಯಾಯಿತು.
\end{center}

\newpage

\section*{ಅಧ್ಯಾಯ\enginline{-}೫}

\emptypage

\begin{flushleft}
\textbf{ಬ್ರಹ್ಮೋವಾಚ\enginline{-}}
\end{flushleft}

\begin{verse}
\textbf{ಮಾಘಸ್ನಾನಸ್ಯ ಮಾಹಾತ್ಮ್ಯಂ ಕಥಿತಂ ವಿಷ್ಣುನಾ ಪುರಾ~। }\\\textbf{ಮಹಾಲಕ್ಷ್ಮ್ಯಾಃ ಸಿಂಧುತೀರೇ ಮಹ್ಯಮಾಹ ರಮಾ ಚ ಸಾ~।। ೧~।।}
\end{verse}

\begin{flushleft}
ಬ್ರಹ್ಮದೇವರು ಹೇಳಿದರು-
\end{flushleft}

ಪೂರ್ವದಲ್ಲಿ ಸಿಂಧೂನದೀ ತೀರದಲ್ಲಿ ವಿಷ್ಣುವಿನಿಂದ ಮಹಾಲಕ್ಷ್ಮಿಗೋಸ್ಕರ ಮಾಘಸ್ನಾನದ ಮಹಿಮೆಯು ಹೇಳಲ್ಪಟ್ಟಿತು. ರಮಾದೇವಿಯರು ಅದನ್ನು ನನಗೆ ಹೇಳಿದರು.

\begin{verse}
\textbf{ಬ್ರಹ್ಮಹತ್ಯಾ ಪೀಡಿತಾಯ ರುದ್ರಾಯಾವೋಚಮಂಗಜ~।}\\\textbf{ತತಃಪರಂ ನ ಕಸ್ಯಾಪಿ ಗೋಪ್ಯತ್ವಾನ್ನೋದಿತಂ ತ್ವಿದಮ್~।। ೨~।।} 
\end{verse}

\begin{verse}
\textbf{ಶುಷ್ರೂಷವೇ ಚ ಪುತ್ರಾಯ ಶ್ರದ್ದಧಾನಾಯ ಧೀಮತೇ~।}\\\textbf{ವಕ್ಷ್ಯಾಮಿ ಪರಮಂ ಗುಹ್ಯಂ ಮಾಘಮಾಹಾತ್ಮ್ಯ ಮುತ್ತಮಮ್~।। ೩~।।}
\end{verse}

ಬ್ರಹ್ಮಹತ್ಯೆಯಿಂದ ಪೀಡಿತರಾದ ರುದ್ರ ದೇವರಿಗೆ ನಂತರ ಹೇಳಲ್ಪಟ್ಟಿತು. ರಹಸ್ಯವಾದುದರಿಂದ ಅಲ್ಲಿಂದ ಮುಂದೆ ಮತ್ಯಾರಿಗೂ ಹೇಳಲ್ಪಡಲಿಲ್ಲ. ಶ್ರದ್ದೆಯಿಂದ ಶ್ರವಣಮಾಡುವ, ಜ್ಞಾನಿಯಾದ ಸತ್ಪುತ್ರನಾದ ನಿನಗೆ ರಹಸ್ಯವಾದ ಮಾಘ ಮಾಸದ ಮಾಹಾತ್ಮ್ಯೆಯನ್ನು ಹೇಳುತ್ತೇನೆ.

\begin{verse}
\textbf{ಮಾಘೇ ಮಾಸಿ ಸಕೃತ್ ಸ್ನಾತ್ವಾ ತಥಾ ಸಂಪೂಜ್ಯ ಮಾಧವಮ್~।}\\\textbf{ದತ್ವಾನ್ನಂ ದ್ವಿಜವರ್ಯಾಯ ಕೃತಕೃತ್ಯೋ ಭವೇನ್ನರಃ~।। ೪~।। }
\end{verse}

\begin{verse}
\textbf{ಯೋ ವದೇತ್ ಮಾಘಮಾಹಾತ್ಮ್ಯಂ ವಾಚ್ಯಮಾನಂ ಶೃಣೋತಿ ಯಃ~।}\\\textbf{ದುರ್ಯೊನಿಹೇತುನಾ ತೇನ ಮುಚ್ಯತೇ ಗುಹ್ಯಕೋ ಯಥಾ~।। ೫~।।}
\end{verse}

ಮಾಘಮಾಸದಲ್ಲಿ ನಿತ್ಯ ಸ್ನಾನಮಾಡಿ, ಮಾಧವನನ್ನು ಪೂಜಿಸಿ, ಶ್ರೇಷ್ಠನಾದ ಬ್ರಾಹ್ಮಣನಿಗೆ ಭೋಜನ ಮಾಡಿಸುವ ಮನುಷ್ಯನು ಕೃತಾರ್ಥನಾಗುತ್ತಾನೆ. ಯಾರು ಮಾಘ ಮಾಹಾತ್ಮ್ಯವನ್ನು ಪ್ರವಚನ ಮಾಡುತ್ತಾನೆಯೋ ಅಥವಾ ಶ್ರವಣ ಮಾಡುತ್ತಾನೆಯೋ ಅಂತಹವನು ಗುಹ್ಯಕನಂತೆ ದುಷ್ಟ ಯೋನಿಯಿಂದ ಬಿಡುಗಡೆ ಹೊಂದುತ್ತಾನೆ.

\begin{flushleft}
\textbf{ನಾರದ ಉವಾಚ\enginline{-}}
\end{flushleft}

\begin{verse}
\textbf{ಕೋ ನಾಮ ಗುಹ್ಯಕೋ ಬ್ರಹ್ಮನ್ ದುರ್ಯೊನಿಃ ಕೇನ ಕರ್ಮಣಾ~।}\\\textbf{ತಸ್ಮಿನ್ ಯೋನೌ ಕಥಮಭೂತ್ ಕುತ್ರಾಸೀತ್ಕೇನ ಸಂಗಮಃ~।। ೬~।।}
\end{verse}

\begin{flushleft}
\textbf{ನಾರದರು ಹೇಳಿದರು.}
\end{flushleft}

ಬ್ರಹ್ಮದೇವರೇ, ಗುಹ್ಯಕನೆಂಬುವನು ಯಾರು? ಯಾವ ಕರ್ಮದಿಂದ ದುರ್ಯೋನಿಯಲ್ಲಿ ಹುಟ್ಟಿದನು? ಯಾರ ಸಹವಾಸದಲ್ಲಿ ಇರುತ್ತಿದ್ದನು? ಹೇಗೆ ಮುಕ್ತನಾದನು?

\begin{flushleft}
\textbf{ಬ್ರಹ್ಮೋವಾಚ-}
\end{flushleft}

\begin{verse}
\textbf{ಗುಹ್ಯ ಕಃ ಶ್ವೇತನಾಮಾಭೂದಲಕಾದ್ವಾರಪಾಲಕಃ~।}\\\textbf{ಕಸ್ಮಿಂಶ್ಚಿತ್ ಹಿಮವತ್ಪೃಷ್ಠೇ ರಂತುಕಾಮೋಽಪ್ಸರೋಗಣೈಃ~।। ೭~।। }
\end{verse}

\begin{verse}
\textbf{ಸಂವೃತೋ ಮದಿರಾಂ ಪೀತ್ವಾ ಯಯೌ ಗಂಗಾತಟಂ ಶುಭಮ್~।}\\\textbf{ವನೇ ಮನೋರಮೇ ದಿವ್ಯೇ ಮತ್ತ ಕೋಕಿಲಕೂಜಿತೇ~।।}\\\textbf{ವೀಚೀವಿಕ್ಷೇಪಸಂಭಿನ್ನೇ ತಮಾಲಹಸಿತಾಂಬರೇ~।। ೮~।।}
\end{verse}

\begin{flushleft}
 ಬ್ರಹ್ಮದೇವರು ಹೇಳಿದರು-
\end{flushleft}

ಶ್ವೇತನೆಂಬ ಒಬ್ಬ ಗುಹ್ಯಕನು (ದೇವತಾಗಣಕ್ಕೆ ಸೇರಿದವನು) ಅಲಕಾಪುರಿಯಲ್ಲಿ ದ್ವಾರಪಾಲಕನಾಗಿದ್ದನು. ಒಂದು ದಿನ ಅವನು ಸ್ತ್ರೀಸುಖವನ್ನು ಪಡೆಯಲು ಹಿಮವತ್ಪರ್ವತದ ಹಿಂಭಾಗಕ್ಕೆ ಬಂದನು. ಮದಿರವನ್ನು ಪಾನಮಾಡಿ ಶುಭಕರವಾದ ಗಂಗಾ ನದಿಯ ದಡಕ್ಕೆ ಬಂದನು. ಅಲ್ಲಿದ್ದ ವನವು ಬಹಳ ಮನೋಹರವಾಗಿತ್ತು. ಸಂತೋಷದಿಂದಕೂಡಿದ ಕೋಗಿಲೆಗಳು ಹಾಡುತ್ತಿದ್ದುವು. ನದಿಯ ಅಲೆಗಳು ಆನಂದವನ್ನುಂಟುಮಾಡುತ್ತಿದ್ದುವು. ಅನೇಕ ವಿಧವಾದ ಮರಗಳು ಆಕಾಶವನ್ನು ಚುಂಬಿಸುತ್ತಿದ್ದುವು.

\begin{verse}
\textbf{ಭುವಿಲಂಬಿತಶಾಖಾಗ್ರಗಲತ್ಪುಷ್ಪಾಧಿವಾಸಿನೇ~।}\\\textbf{ರಮ್ಯೇ ಪುಲಿನಪರ್ಯಂಕೇ ರೇಮೇ ರಾಮಾಭಿರಾವೃತಃ~।। ೯~।। }
\end{verse}

\begin{verse}
\textbf{ನ ಕಿಂಚಿತ್ ಜ್ಞಾತವಾನ್ ಶ್ವೇತೊ ದೇಶಕಾಲಾದಿಕಾಂಸ್ತಥಾ~।}\\\textbf{ತದಾ ಚ ಲೋಮಶೋ ನಾಮ ಕೈಲಾಸಾದಾಗತೋ ಮುನಿಃ~।। ೧೦~।।}
\end{verse}

ವೃಕ್ಷಗಳ ಶಾಖೆಗಳು ನೆಲವನ್ನು ಮುಟ್ಟಿದ್ದುವು; ನೆಲದಮೇಲೆ ಬಿದ್ದಿದ್ದ ಪುಷ್ಪಗಳ ಸುವಾಸನೆಯು ಎಲ್ಲೆಲ್ಲಿಯೂ ಹರಡಿತ್ತು; ಅಂತಹ ರಮ್ಯವಾದ ಸ್ಥಳದಲ್ಲಿ ಆ ಗುಹ್ಯಕನು ಮಂಚದಮೇಲೆ ಅಪ್ಸರಾಸ್ತ್ರೀಯರೊಡನೆ ಕೂಡಿ ಕ್ರೀಡಿಸಿದನು. ಆಗ ಅವನಿಗೆ ದೇಶ ಕಾಲಗಳ ಪರಿವೆಯೇ ಇರಲಿಲ್ಲ. ಅದೇ ಕಾಲದಲ್ಲಿ ಲೋಮಶರೆಂಬ ಮುನಿಗಳು ಕೈಲಾಸದಿಂದ ಅಲ್ಲಿಗೆ ಬಂದರು.

\begin{verse}
\textbf{ತಂ ದೃಷ್ಟ್ವಾಮದಿರಾಮತ್ತೋ ವಿವಸ್ತ್ರೋಽಂಗನಯಾ ಸಹ~।}\\\textbf{ನೋತ್ತಸ್ಥೌ ಮುನಿಶಾರ್ದೂಲಃ ಪ್ರಜಹಾಸ ಸ ತಂ ಮುನಿಮ್~।। ೧೧~।।}
\end{verse}

ಮದಿರಾಪಾನಮತ್ತನಾದ, ವಿವಸ್ತ್ರನಾದ, ಸ್ತ್ರೀಯರೊಡನೆ ರಮಿಸುತ್ತಿದ್ದ ಗುಹ್ಯಕನು ಮುನಿಶ್ರೇಷ್ಠರನ್ನು ಕಂಡು ಎದ್ದು ನಿಲ್ಲಲಿಲ್ಲ; ಅವರನ್ನು ನೋಡಿ ಅಪಹಾಸ್ಯ ಮಾಡಿದನು.

\begin{verse}
\textbf{ತದಾ ಸ ಕೋಪರುಕ್ಷಾಕ್ಷೋ ಲೋಮಶಸ್ತಂ ಶಶಾಪ ಹ~।}\\\textbf{ಮತ್ತತ್ವಾನ್ನೋತ್ಥಿತೋ ಯಸ್ಮಾದ್ಭವ ವ್ಯಾಲೋದ್ರಿಕಂದರೇ~।। ೧೨~।।}
\end{verse}

ಲೋಮಶರು ಕೋಪದಿಂದ ಕಣ್ಣುಗಳನ್ನು ಕೆಂಪಗೆ ಮಾಡಿಕೊಂಡು “ಗುಹ್ಯಕನೇ, ನೀನು ಉನ್ಮತ್ತನಾಗಿ ಪ್ರತ್ಯುತ್ಥಾನಾದಿಗಳನ್ನು ಮಾಡದೇ ನನ್ನನ್ನು ಅಪಹಾಸ್ಯ ಮಾಡಿದಕಾರಣ ಪರ್ವತದ ಕಂದರದಲ್ಲಿ ಸರ್ಪವಾಗು” ಎಂದು ಶಪಿಸಿದರು.

\begin{verse}
\textbf{ಇತಿ ಶಪ್ತೋ ಮುನೀಂದ್ರೇಣ ಪ್ರಾರ್ಥಯಾಮಾಸ ಸಾದರಮ್~।}\\\textbf{ತ್ವಯಾ ಹ್ಯನುಗೃಹೀತೋಽಸ್ಮಿ ನ ಮನ್ಯೇ ದಂಡ ಇತ್ಯಹಮ್~।। ೧೩~।।}
\end{verse}

ಗುಹ್ಯಕನು ಲೋಮಶರಿಂದ ಶಪ್ತನಾದ ನಂತರ ವಿನಯದಿಂದ ಅವರನ್ನು ಪ್ರಾರ್ಥಿಸಿದನು: “ಋಷಿಗಳೇ ನಾನು ನಿಮ್ಮಿಂದ ಅನುಗೃಹೀತನಾಗಿರುತ್ತೇನೆ. ನನ್ನನ್ನು ದಂಡಿಸಬೇಡಿರಿ.

\begin{verse}
\textbf{ಉಪಕಾರಾಯ ಹಿ ಖಲು ಶಿಶೋರ್ಮಾತುಶ್ಚ ತರ್ಜನಮ್~।}\\\textbf{ಕೃಪಾಂ ಕುರು ಕೃಪಾಲಸ್ತ್ವಂ ದೀನೇ ಗತಮತೌ ಮಯಿ~।। ೧೪~।।}
\end{verse}

ತಾಯಿಯು ಮಕ್ಕಳಿಗೆ ಉಪಕಾರವಾಗಲಿಕ್ಕೋಸ್ಕರತಾನೇ ಧಿಕ್ಕರಿಸುವುದು; ಕೃಪಾಲುಗಳೇ, ದೀನನಾದ ನನ್ನಲ್ಲಿ ಅನುಗ್ರಹವನ್ನು ಮಾಡಿರಿ”.

\begin{verse}
\textbf{ಧಾತೂನಾಮಗ್ನಿಸಂಸ್ಕಾರೋ ಯಾವದಾಮಲಶೋಧನಮ್~।}\\\textbf{ಇತಿ ಪ್ರಸಾದಿತಸ್ತೇನ ವಿಶಾಪಮದದಾತ್ತದಾ~।। ೧೫~।।}
\end{verse}

ಲೋಹಗಳು ಬೆಂಕಿಯ ಸಂಪರ್ಕದಿಂದ ಹೇಗೆ ಕಷ್ಮಲಗಳನ್ನು ಕಳೆದುಕೊಳ್ಳುತ್ತವೆಯೋ ಅದರಂತೆ ನನ್ನನ್ನು ಉದ್ಧರಿಸಿರಿ.” ಹೀಗೆ ಪ್ರಾರ್ಥಿಸಿಕೊಳ್ಳಲ್ಪಟ್ಟ ಲೋಮಶ ಋಷಿಗಳು ಗುಹ್ಯಕನಿಗೆ ಶಾಪದ ತೀವ್ರತೆಯನ್ನು ಕಡಿಮೆ ಮಾಡಿದರು.

\begin{verse}
\textbf{ವಾಮದೇವಸ್ಯ ಶಿಷ್ಯೇಭ್ಯೋ ವ್ಯಾಕುರ್ವಾಣಾಂ ಇಮಾಂ ಕಥಾಮ್~।}\\\textbf{ಶ್ವೇತಸ್ತು ಬ್ರಹ್ಮದಂಡೇನ ಚಿತ್ರ ಕೂಟಾದ್ರಿಕಂದರೇ~।। ೧೬~।।}
\end{verse}

\begin{verse}
\textbf{ವ್ಯಾಲೋಭೂದ್ಯೋಜನಾಯಾಮಶರೀರೋಽನಿಲಭೋಜನಃ~।}\\\textbf{ಸೋಽವಾತ್ಸೀದ್ವರ್ಷಸಾಹಸ್ರಂ ಪಶುಪಕ್ಷಿಮೃಗಾಶನಃ~।। ೧೭~।।}
\end{verse}

ವಾಮದೇವ ಋಷಿಗಳ ಶಿಷ್ಯರಿಂದ ಸತ್ಕಥಾಶ್ರವಣವಾದ ನಂತರ ನಿನಗೆ ಮೊದಲಿನ ರೂಪ ಬರುತ್ತದೆಯೆಂದು ವಿಶಾಪವನ್ನು ಕೊಟ್ಟರು. ಬ್ರಾಹ್ಮಣರ ಶಾಪದಿಂದ ಶ್ವೇತನು ಚಿತ್ರಕೂಟ ಪರ್ವತದ ಕುಂದರದಲ್ಲಿ ಸರ್ಪರೂಪದಿಂದ ಬಿದ್ದನು. ಅದರ ಶರೀರವು ಒಂದು ಯೋಜನ ಉದ್ದವಿತ್ತು. ವಾಯುವೇ ಆಹಾರ. ಒಂದು ಸಹಸ್ರ ವರ್ಷ ಆ ರೀತಿಯಲ್ಲಿ ಬಿದ್ದಿದ್ದು ಅಲ್ಲಿಗೆ ಬರುತ್ತಿದ್ದ ಪಶು, ಪಕ್ಷಿ, ಮೃಗ ಗಳನ್ನು ಭಕ್ಷಿಸುತ್ತಿದ್ದನು.

\begin{verse}
\textbf{ಏತಸ್ಮಿನ್ನೇವ ಕಾಲೇ ತು ವಾಮದೇವೋ ಮಹಾಮುನಿಃ~।}\\\textbf{ಸ ತಂ ದ್ರಷ್ಟುಂ ಮುನಿಗಣೈಃ ಪರೀತೋದ್ರಿಮಮುಂ ಯಯೌ~।। ೧೮~।। }
\end{verse}

\begin{verse}
\textbf{ಸ ವ್ಯಾಲಾಧಿಷ್ಠಿ ತಗುಹೋದ್ವಾರಿಸ್ಥ ವಟಪಾದಪೇ~।}\\\textbf{ನೀತ್ವಾ ರಾತ್ರಿಂ ಬಹುವಿಧೈಃ ಕಥಾಲಾಪೈಃ ಸುಖಾವಹೈಃ~।। ೧೯~।।}
\end{verse}

ಈ ಕಾಲದಲ್ಲಿ ಮಹಾಮುನಿಗಳಾದ ವಾಮದೇವರು ಇತರ ಋಷಿಗಳಿಂದ ಕೂಡಿ ಗುಹ್ಯಕನನ್ನು ನೋಡಲು ಈ ಪರ್ವತಕ್ಕೆ ಬಂದರು. ಸರ್ಪವು ಇದ್ದ ಗುಹೆಯ ಬಾಗಿಲಿನ ಬಳಿ ವಟವೃಕ್ಷದ ಕೆಳಗೆ ಕುಳಿತು ಸುಖಕರವಾದ, ನಾನಾ ಬಗೆ ಕಥೆಗಳನ್ನು ಹೇಳುತ್ತಾ ಆ ರಾತ್ರಿಯನ್ನು ಕಳೆದರು.

\begin{verse}
\textbf{ತತೋ ಭಗೋದಯೇ ಜಾತೇ ಪ್ರಯಾಣಾಯೋದ್ಯತೇ ಮುನೌ~।। ೨೦~।।} 
\end{verse}

\begin{verse}
\textbf{ತದಾ ಶತರ್ಚಿತೋ ನಾಮ ಮುನಯಃ ಶಂಸಿತವ್ರತಾಃ~।}\\\textbf{ತ್ರಿಕೂಟಾಗ್ರೇ ದೇವಖಾತಾತ್ ಮಾಘಮಾಸ್ಯು ದಿತೇ ರವೌ~।। ೨೧~।। }
\end{verse}

\begin{verse}
\textbf{ಸ್ನಾತ್ವಾ ತೇ ಸ್ವಾಶ್ರಮಂ ಗಂತುಮಭ್ರಾದವತರನ್ ಕ್ಷಣಾತ್~।}\\\textbf{ಗತೇ ತದಾರ್ದ್ರವಸ್ತ್ರಾಣಾಮಪತನ್ನು ದಬಿಂದವಃ~।। ೨೨~।। }
\end{verse}

\begin{verse}
\textbf{ವಟಸ್ಯಾಗ್ರೇ ಕ್ಷಣಾದೇವ ವಟೋಪಿ ನ್ಯಪತದ್ಭುವಿ~।}\\\textbf{ಉತ್ತಸ್ಥೌ ಪುರುಷಸ್ತತ್ಮಾತ್ತದೈವಾದ್ಭುತದರ್ಶನಃ~।। ೨೩~।।}
\end{verse}

ಸೂರ್ಯೋದಯವಾಗುತ್ತಿರಲು ವಾಮದೇವ ಮುನಿಗಳು ಅಲ್ಲಿಂದ ಹೊರಡಲು ಸಿದ್ಧ\-ರಾದರು. ಆಗ ಶ್ರೇಷ್ಠರಾದ ಶತರ್ಚಿ ಎಂಬ ಮುನಿಗಳು ಮಾಘಮಾಸದಲ್ಲಿ ಸೂರ್ಯೋದಯಕ್ಕೆ ಮೊದಲು ತ್ರಿಕೂಟ ಪರ್ವತದ ಗುಹೆಯಿಂದ ಹೊರಟು ಪರ್ವತದ ಅಗ್ರಭಾಗದಲ್ಲಿದ್ದ ಜಲಾಶಯದಲ್ಲಿ ಸ್ನಾನಮಾಡಿ ತಮ್ಮ ಆಶ್ರಮಕ್ಕೆ ವಾಪಸ್ಸು ಹೋಗುವಾಗ ಅಂತರಿಕ್ಷದಿಂದ ಕೆಳಗೆ ಬಂದು ಪ್ರಯಾಣ ಮಾಡುವ ಕಾಲದಲ್ಲಿ ಅವರ ಮೈ ಮೇಲಿನ ಒದ್ದೆಯಾದ ವಸ್ತ್ರಗಳಿಂದ ನೀರಿನ ಹನಿಗಳು ಆ ವಟವೃಕ್ಷದ ಮೇಲೆ ಬಿದ್ದ ಕೂಡಲೇ ಆ ಆಲದಮರದಿಂದ ಅದ್ಭುತನಾದ ಪುರುಷನೊಬ್ಬನು ಆವಿರ್ಭವಿಸಿದನು.

\begin{verse}
\textbf{ತಂ ದೃಷ್ಟ್ವಾ ವಾಮದೇವಸ್ತು ವಿಸ್ಮಿತೋ ವಾಕ್ಯಮಬ್ರವೀತ್~।}\\\textbf{ಕೋಽಸೌ ಕಸ್ಮಾದಯಂ ಭಾವೋ ಮುಕ್ತಿಃ ಕಸ್ಮಾದಹೈತುಕೀ~।। ೨೪~।। }
\end{verse}

\begin{verse}
\textbf{ಚಿಂತಾಕುಲೇ ಮುನಿಗಣೇ ಗೌತಮಸ್ಯ ತನೂದ್ಭವಃ~।}\\\textbf{ಇತಿ ಪೃಷ್ಟೋ ವಾಮದೇವಾತ್ಸೋಯಮದ್ಭುತಪೂರುಷಃ~।। ೨೫~।।}
\end{verse}

\begin{verse}
\textbf{ಪ್ರಣಮ್ಯ ಶಿರಸಾ ತಂ ವೈ ವಾಮದೇವಮಥಾಬ್ರವೀತ್~।}
\end{verse}

ಈ ದೃಶ್ಯವನ್ನು ಕಂಡು ಆಶ್ಚರ್ಯಪಟ್ಟ ವಾಮದೇವ ಮುನಿಗಳು ಹೇಳಿದರು: ಆವಿರ್ಭವಿಸಿದ ಈ ಪುರುಷನು ಯಾರು? ಯಾತಕ್ಕಾಗಿ ಇವನಿಗೆ ಈ ಗತಿಯು ಬಂದಿತ್ತು? ಹೇಗೆ ಮುಕ್ತನಾದನು? ಮುನಿಗಳ ಸಮೂಹದಲ್ಲಿದ್ದ ಗೌತಮರ ಮಕ್ಕಳಾದ ವಾಮದೇವರ ಈ ಪ್ರಶ್ನೆಗಳಿಗೆ ಆ ಅದ್ಭುತ ಪುರುಷನು ನಮಸ್ಕರಿಸಿ ಈ ರೀತಿ ಹೇಳಿದನು.

\begin{verse}
\textbf{ಮಾರ್ಕಂಡೇಯಸ್ಯ ಶಿಷ್ಯೋಽಹಂ ಸುತಪಾನಾಮ ವೈ ಋಷಿಃ~।। ೨೬~।।} 
\end{verse}

\begin{verse}
\textbf{ಸಮಿತ್ಕು ಶಫಲಾದ್ಯರ್ಥಂ ಅರಣ್ಯಂ ಗುರುಣೋದಿತಃ~।}\\\textbf{ಆಗಮಂ ತತ್ರ ತತ್ಪುತ್ರೀ ನಾಮ್ನಾ ಗುಣವತೀ ಸತೀ~।। ೨೭~।।}
\end{verse}

ನಾನು ಮಾರ್ಕಂಡೇಯ ಋಷಿಗಳ ಶಿಷ್ಯನಾದ ಸುತಪ; ಒಂದು ದಿನ ಗುರುಗಳ ಆಜ್ಞೆಯಂತೆ ಸಮಿತ್ತು, ದರ್ಭೆ, ಫಲಾದಿಗಳನ್ನು ಸಂಗ್ರಹಿಸಲು ಅರಣ್ಯಕ್ಕೆ ಬಂದಾಗ ಗುರುಗಳ ಪುತ್ರಿಯಾದ ಗುಣವತಿಯು

\begin{verse}
\textbf{ಚರಂತೀ ಕಂದುಕಾಕ್ರೀಡಾವ್ಯಾಜಾತ್ ಸಖ್ಯಾ ಸಹಾಗತಾ~।}\\\textbf{ಗಾಂಧರ್ವೇಣ ವಿವಾಹೇನ ಭದ್ರೇ ರಂತುಮನಾ ಅಹಮ್~।। ೨೮~।।}
\end{verse}

ತನ್ನ ಸಖಿಯರೊಡನೆ ಚೆಂಡಾಟವನ್ನು ಆಡುತ್ತಾ ಅಲ್ಲಿಗೆ ಬಂದಳು; ಗಾಂಧರ್ವ ರೀತಿಯಿಂದ ಅವಳಲ್ಲಿ ಕ್ರೀಡಿಸಬೇಕೆಂಬ ಇಚ್ಛೆ ನನ್ನಲ್ಲಿ ಬಂತು.

\begin{verse}
\textbf{ಆಹ ಸಾ ರೋಷತಾಮ್ರಾಕ್ಷೀ ಕಿನ್ನ ಜಾನಾಸಿ ಮೂಢಧೀಃ~।}\\\textbf{ಶಿಷ್ಯತ್ವಾನ್ಮೇ ಪಿತುಭ್ರಾತಾ ಕಥಂ ವಾ ಕಾಂಕ್ಷಸೇಽನುಜಾಮ್~।। ೨೯~।। }
\end{verse}

\begin{verse}
\textbf{ಅತ್ಯಜ್ಞೋ ಭವ ದುರ್ಬುದ್ಧೇ ವಟೋ ಭೂಮ್ಯಾಂ ವನೇರ್ಬುದಮ್~।}\\\textbf{ಶಶಾಪ ಮಾಂ ಗುರುಸುತಾ ತೇನಾಹಂ ವಟತಾಂ ಗತಃ~।। ೩೦~।।}
\end{verse}

ಆಗ ಆ ಗುಣವತಿಯು ಕೋಪದಿಂದ ಕಣ್ಣುಗಳನ್ನು ಕೆಂಪಗೆ ಮಾಡಿಕೊಂಡು ಹೇಳಿದಳು: “ನೀನು ನನ್ನ ತಂದೆಯ ಶಿಷ್ಯನಾಗಿರುವುದರಿಂದ ನಾನು ನಿನಗೆ ತಂಗಿಯಂತೆಯಾದೆ. ತಂಗಿಯ ಜತೆ ಭೋಗಿಸಬೇಕೆಂಬ ಇಚ್ಛೆಯು ಹೇಗೆ ಹುಟ್ಟಿತು, ದುರ್ಬುದ್ದಿಯಿಂದ ಕೂಡಿದ ನಿನಗೆ ಇದು ತಿಳಿಯಲಿಲ್ಲವೇ? ಈಗಲೇ ನೀನು ದಟ್ಟವಾದ ಅರಣ್ಯದಲ್ಲಿ ಆಲದ ಮರವಾಗಿ ಹುಟ್ಟು” ಎಂದು ಶಾಪಕೊಟ್ಟಳು. ಆ ಶಾಪದ ದೆಸೆಯಿಂದ ನಾನು ಆಲದ ಮರವಾಗಿದ್ದೆ.

\begin{verse}
\textbf{ನ ಜಾನೇ ಕರ್ಮಣಾ ಕೇನ ವಟಜನ್ಮಗತಂ ಮಮ~।}\\\textbf{ಏತತ್ಕು ತೂಹಲಂ ಶ್ರೋತುಂ ಕಾರಣಂ ವಕ್ತುಮರ್ಹಸಿ~।। ೩೧~।।} 
\end{verse}

\begin{verse}
\textbf{ಗುರುತಲ್ಪಗಪಾಪಂ ಯತ್ ಕಲ್ಪಕೋಟಿಶತೈರಪಿ~।}\\\textbf{ನ ಕಸ್ಯ ಕ್ಷೀಯತೇ ಕಸ್ಮಾದಿದಾನೀಂ ಭಸ್ಮಸಾದ್ಯ ಯೌ~।। ೩೨~।। }
\end{verse}

\begin{verse}
\textbf{ತದ್ಧರ್ಮಸೂಕ್ಷ್ಮಂ ವದ ವಾಮದೇವ}\\\textbf{ಕರೋಮಿ ಭೂಯೋಪಿ ಚ ಋದ್ಧಿ ಕಾಮಃ~। }\\\textbf{ಕ್ಷೇಮಾಯ ಲೋಕಸ್ಯ ಭವೇತ್ಪ್ರಚಾರೋ} \\\textbf{ನಿಷ್ಕಾಮುಕಾನಾಮಿಹ ನಾಸ್ತಿ ಕೃತ್ಯಮ್~।। ೩೩~।।}
\end{verse}

ಯಾವ ಪುಣ್ಯ ಪ್ರಭಾವದಿಂದ ನನ್ನ ಆಲದಮರದ ರೂಪವು ಹೋಯಿತು ಎಂಬ ಕುತೂಹಲಕಾರಿಯಾದ ವೃತ್ತಾಂತವನ್ನು ಕೇಳಲು ಇಚ್ಛಿಸುತ್ತೇನೆ; ಹೇಳಲು ನೀವು ಸಮರ್ಥರು. ಗುರು ಪತ್ನಿ, ಗುರು ಪುತ್ರಿಯರ ಗಮನ ಪಾಪವು ನೂರು ಕೋಟಿ ಕಲ್ಪಗಳಾದರೂ ನಾಶವಾಗುವುದಿಲ್ಲ; ಹೀಗಿರುವಾಗ ನನ್ನ ಅಂತಹ ಪಾಪವು ಈಗ ಹೇಗೆ ಭಸ್ಮವಾಯಿತು? ವಾಮದೇವ ಋಷಿಗಳೇ, ಈ ಧರ್ಮ ಸೂಕ್ಷವನ್ನು ಹೇಳಿರಿ. ಸಾಧನಪೂರ್ತಿಯನ್ನು ಅಪೇಕ್ಷಿಸುವ ನಾನು ಪುನಃ ಅದರಂತೆ ನಡೆಯುತ್ತೇನೆ. ಲೋಕದ ಜನರ ಶ್ರೇಯಸ್ಸಿಗೂ ಈ ಧರ್ಮ ಸೂಕ್ಷ್ಮವು\break ಪ್ರಯೋಜನಕಾರಿಯಾಗಿ ಪ್ರಚಾರವಾಗುತ್ತದೆ. ಕಾಮಾಪೇಕ್ಷೆ ಇಲ್ಲದವರಿಗೆ ಇಲ್ಲಿ ಯಾವ ಕೆಲಸವೂ ಇಲ್ಲ.

\begin{verse}
\textbf{ಶ್ರುತ್ವಾ ತದ್ವಚನಂ ವಾಮದೇವೋ ಧ್ಯಾನಪರಾಯಣಃ~।}\\\textbf{ಜ್ಞಾತ್ವಾ ತತ್ಕಾರಣಂ ತಸ್ಯ ವಿಸ್ಮಿತೊ ವಾಕ್ಯ ಮಬ್ರವೀತ್~।। ೩೪~।।}
\end{verse}

ಇದನ್ನು ಕೇಳಿದ ವಾಮದೇವ ಋಷಿಗಳು ಸ್ವಲ್ಪ ಕಾಲ ಧ್ಯಾನಮಾಡಿ, ಆ ಕಾರಣವನ್ನು ತಿಳಿದು, ಆಶ್ಚರ್ಯದಿಂದ ಮುಕ್ತರಾಗಿ ಹೇಳಿದರು:

\begin{verse}
\textbf{ಶತರ್ಚಿನೋ ನಾನು ಪುರಾ ಭಗೋದಯೇ~।}\\\textbf{ಗಿರೀಂದ್ರಶೃಂಗೋದ್ಭುತದೇವಖಾತೇ~। }\\\textbf{ಮಾಘೇ ಚ ಮಾಸೇ ಕೃತಸರ್ವಕೃತ್ಯಾ} \\\textbf{ನಭೋ ಗತಾಃ ಸ್ವಾಶ್ರಮಮಾಪುರೋಜಸಾ~।। ೩೫~।। }
\end{verse}

\begin{verse}
\textbf{ತದಾರ್ದ್ರವಸ್ತ್ರಾದ್ಗಲಿತೋದಬಿಂದುಭಿಃ}\\\textbf{ಸಿಕ್ತೋ ವಟೋsಯಂ ತವ ಪೂರ್ವರೂಪಃ~। }\\\textbf{ಪ್ರಾತರ್ಜಲಸ್ಪರ್ಶನಪುಣ್ಯಲೇಶಾತ್} \\\textbf{ಬಂಧೋ ಗತಸ್ತೇ ಜಡಜನ್ಮಸಾಧಕಃ~।। ೩೬~।।}
\end{verse}

\begin{verse}
\textbf{ಹತ್ಯಾಽಯುತಂ ಪಾಪಸಹಸ್ರಮುಗ್ರಂ}\\\textbf{ಗುರ್ವಂಗನಾಕೋಟಿ ನಿಷೇವಣಂ ಚ~।}\\\textbf{ತಥಾನ್ಯದೋಷೋsಸ್ತಿ ಚ ಮಾಘಮಾಸೇ} \\\textbf{ಸ್ನಾನೇನ ನಶ್ಯಂತಿ ಸಮೂಲಘಾತಮ್~।। ೩೭~।।}
\end{verse}

ಸುತಪನೇ, ಶತರ್ಚಿರೆಂಬ ಋಷಿಗಳು ಮಾಘಮಾಸದಲ್ಲಿ ಬೆಟ್ಟದ ಮೇಲಿರುವ ಜಲಾಶಯದಲ್ಲಿ ಸೂರ್ಯೋದಯಕ್ಕೆ ಪೂರ್ವದಲ್ಲಿ ಸ್ನಾನಮಾಡಿ, ತಮ್ಮ ಸತ್ಕರ್ಮಗಳನ್ನು ತೀರಿಸಿ, ಆಕಾಶಮಾರ್ಗದಲ್ಲಿ ತಮ್ಮ ಆಶ್ರಮಕ್ಕೆ ಹೋಗುತ್ತಿದ್ದರು. ಅವರಲ್ಲಿದ್ದ ಒದ್ದೆಯಾದ ವಸ್ತ್ರದಿಂದ ನೀರಿನ ಹನಿಗಳು ವಟವೃಕ್ಷವಾಗಿದ್ದ ನಿನ್ನ ಮೇಲೆ ಬೀಳಲು ನಿನಗೆ ಮೊದಲಿನ ರೂಪವು ಬಂದಿತು. ಪ್ರಾತಃಕಾಲದಲ್ಲಿ ಜಲ ಸ್ಪರ್ಶನ ಮಾತ್ರದಿಂದ ಜಡದಂತೆ ಇದ್ದ ನಿನ್ನ ಬಂಧನವು ಹೋಯಿತು. ಅಸಂಖ್ಯಾತವಾದ ನಾನಾ ವಿಧವಾದ ಪಾಪಗಳೂ, ಗುರುಸ್ತ್ರೀಗಮನಾದಿ ಸಹಸ್ರಾರು ಪಾಪಗಳು ಮಾಘಸ್ನಾನದಿಂದ ಪೂರ್ಣವಾಗಿ ನಾಶವಾಗುತ್ತವೆ.

\begin{verse}
\textbf{ನ ಪಾಪಬುದ್ಧಿರ್ನ ಚ ಕರ್ಮಬಂಧೋ}\\\textbf{ನ ವಾ ಕುಯೋನಿರ್ನ ಪಿಶಾಚದೇಹಃ~। }\\\textbf{ನಾಗ್ನ್ಯರ್ಕಸೋಮಾನಿಲಬಂಧದೇಹ} \\\textbf{ಪ್ರಾತರ್ಜಲಸ್ಪರ್ಶನಮಾತ್ರತೋಽಂಗ~।। ೩೮~।।}
\end{verse}

ಮಾಘಸ್ನಾನದಿಂದ ಅಂತಃಕರಣ ಶುದ್ಧಿಯಾಗಿ ಪಾಪದ ಬುದ್ದಿಯು ಇರುವುದಿಲ್ಲ, ಕರ್ಮಬಂಧನವು ಇರುವುದಿಲ್ಲ, ಕೆಟ್ಟ ಯೋನಿಗಳು ಪ್ರಾಪ್ತವಾಗುವುದಿಲ್ಲ. ಪಿಶಾಚದೇಹವು ಬರುವುದಿಲ್ಲ, ಅಗ್ನಿ, ಸೂರ್ಯ, ಚಂದ್ರ, ವಾಯು ಇವರಿಂದ ಸಹ ಬಂಧನವು ಉಂಟಾಗುವುದಿಲ್ಲ. ಮಾಘದಲ್ಲಿ ಪಾತಃಕಾಲ ಜಲಸ್ಪರ್ಶಮಾತ್ರದಿಂದ ಇದೆಲ್ಲ ಲಭಿಸುತ್ತವೆ.

\begin{verse}
\textbf{ಜಿಹ್ವಾ ನ ವಕ್ತಿ ಯದಿ ಮಾಧವನಾಮಧೇಯಂ}\\\textbf{ಚೇತಶ್ಚ ನ ಸ್ಮರತಿ ಮಾಧವಪಾದಪದ್ಮಮ್~। }\\\textbf{ಪ್ರಾತರ್ನವಾಯೈ ನ ಚ ಮಾಘಮಾಸೇ} \\\textbf{ಸ್ನಾನಂ ಕೃತಂ ತಸ್ಯ ಹಿ ಕರ್ಮಬಂಧಃ~।। ೩೯~।।}
\end{verse}

ಯಾರ ನಾಲಿಗೆಯು ಮಾಧವನ ನಾಮವನ್ನು ಉಚ್ಚರಿಸುವುದಿಲ್ಲವೋ, ಯಾರ ಮನಸ್ಸು ಮಾಧವನ ಪಾದಪದ್ಮಗಳನ್ನು ಸ್ಮರಿಸುವುದಿಲ್ಲವೋ, ಮಾಘಮಾಸದಲ್ಲಿ ಪ್ರಾತಃಕಾಲದಲ್ಲಿ\break ಜಲಾಶಯದಲ್ಲಿ ಯಾರು ಸ್ನಾನ ಮಾಡುವುದಿಲ್ಲವೋ, ಅವನಿಗೆ ಕರ್ಮಬಂಧ (ಯಮಬಂಧ) ಇರುತ್ತದೆ.

\begin{verse}
\textbf{ಅಸ್ಮತ್ಸಹಾಗತಾನಾಂ ತೇ ಮುನೀನಾಂ ಚ ಸಮಾಗತಾನ್~।}\\\textbf{ಪ್ರಾತರ್ಮಾಘೇ ಜಲಸ್ಪರ್ಶೋ ಭಾಗ್ಯದಃ ಸುಸಮಾಗಮಃ~।। ೪೦~।।}
\end{verse}

ಸುತಪನೇ, ನನ್ನ ಸಂಗಡ ಬಂದಿರುವ ಮುನಿಗಳ ಸಹವಾಸದಿಂದಲೂ, ಮಾಘ ಮಾಸದಲ್ಲಿ ಪ್ರಾತಃಕಾಲದ ಜಲಸ್ಪರ್ಶದಿಂದಲೂ ನಿನಗೆ ಸೌಖ್ಯ ಉಂಟಾಗಿದೆ.

\begin{verse}
\textbf{ಸರ್ವದಾನೇಷು ಯತ್ಪುಣ್ಯಂ ಸರ್ವತೀರ್ಥೇಷು ಯತ್ಫಲಮ್~।}\\\textbf{ತತ್ಫಲಂ ಸಮವಾಪ್ನೋತಿ ಮಾಘಸ್ನಾನೇನ ನಾರದ~।। ೪೧~।। }
\end{verse}

\begin{verse}
\textbf{ತಾದೃಕ್ ಪಾಪಂ ಚ ನೈವಾಸ್ತಿ ಯಚ್ಚ ಮಾಘಾನ್ನ ಗಚ್ಛತಿ~।}\\\textbf{ತಸ್ಮಾತ್ ಮಾಘಕೃತಸ್ನಾನಾತ್ ಸುಖೀ ಭವ ಚಿರಂ ಮುನೇ~।। ೪೨~।।}
\end{verse}

ಸಮಸ್ತ ದಾನಗಳಿಂದ ಏನು ಪುಣ್ಯವು ಲಭಿಸುತ್ತದೆಯೋ, ಸಮಸ್ಯೆ ತೀರ್ಥಗಳಲ್ಲಿ ಸ್ನಾನ ಮಾಡುವುದರಿಂದ ಎಷ್ಟು ಫಲವು ದೊರೆಯುತ್ತದೆಯೋ, ಅಷ್ಟು ಫಲ ಕೇವಲ ಮಾಘಸ್ನಾನದಿಂದ ಪ್ರಾಪ್ತವಾಗುತ್ತದೆ, ನಾರದರೇ, ಮಾಘಸ್ನಾನದಿಂದ ನಾಶವಾಗದೇ ಇರುವ ಪಾಪವೇ ಇಲ್ಲ. ಆದುದರಿಂದ ಮಾಘಸ್ನಾನ ಮಾಡಿ ಚಿರಕಾಲ ಸುಖದಿಂದ ಇರು.

\begin{verse}
\textbf{ಇತಿ ಶ್ರುತ್ವಾ ವಾಮದೇವಭಾಷಿತಂ ಸುತಪಾಮುನಿಃ~।}\\\textbf{ಭೂಯೋ ಭೂಯೋ ನಮಸ್ಕೃತ್ಯ ತಥಾಮಂತ್ರ್ಯ ಯಯೌ ಮುನೇ~।। ೪೩~।।}
\end{verse}

ಹೀಗೆ ವಾಮದೇವ ಮುನಿಗಳ ಮಾತನ್ನು ಕೇಳಿದ ಸುತಪನು ವಾಮದೇವರಿಗೆ ಬಾರಿ ಬಾರಿ ನಮಸ್ಕರಿಸಿ, ಅವರ ಅಪ್ಪಣೆ ಪಡೆದು ಹೊರಟುಹೋದನು.

\begin{verse}
\textbf{ಶ್ರುತ್ವಾ ಮಾಘಸ್ಯ ಮಾಹಾತ್ಮ್ಯಂ ವಾಮದೇವಮುಖೇರಿತನಮ್~।}\\\textbf{ತದೈವ ವ್ಯಾಲೋ ಮುಕ್ತೋsಭೂದ್ದು ರ್ಯೋನೇರ್ದುಃಖಕಾರಣಾತ್~।।}
\end{verse}

ವಾಮದೇವರಿಂದ ಈ ರೀತಿ ಮಾಘ ಮಾಹಾತ್ಮ್ಯೆಯನ್ನು ಕೇಳಿದ ಸರ್ಪವೂ ತನ್ನ ದುಃಖಭೂತವಾದ ಯೋನಿಯಿಂದ ಮುಕ್ತಿ ಪಡೆಯಿತು.

\begin{verse}
\textbf{ತದೈವಾದ್ಭುತರೂಪೋಭೂತ್ ಶ್ವೇತೋ ನಾಮ ಸ ಗುಹ್ಯಕಃ~।}\\\textbf{ವಾಮದೇವಂ ನಮಸ್ಕೃತ್ಯ ಸ್ವವೃತ್ತಾಂತಮಚೋದಯಾತ್~।। ೪೫~।।}
\end{verse}

ಶ್ವೇತನೆಂಬ ಆ ಗುಹ್ಯಕನು ವಾಮದೇವರಿಗೆ ನಮಸ್ಕರಿಸಿ, ಅದ್ಭುತ ರೂಪವನ್ನು ಹೊಂದಿ, ತನ್ನ ವಿಚಾರವನ್ನು ಹೇಳಿದನು.

\begin{verse}
\textbf{ಕೃತಪ್ರಶ್ನಾದ್ವಾಮದೇವಾತ್ ಹರ್ಷೋತ್ಫುಲ್ಲವಿಲೋಚನಃ~।}\\\textbf{ಅನುಜ್ಞಾ ತೋ ವಿಮಾನೇನ ಗತೋಽಸಾವಲಕಾಪುರೀಮ್~।। ೪೬~।।}
\end{verse}

ವಾಮದೇವರ ಪ್ರಶ್ನೆಗಳಿಂದ ಸಂತೋಷಗೊಂಡು ಅವರ ಅಪ್ಪಣೆಯನ್ನು ಪಡೆದು ವಿಮಾನದಲ್ಲಿ ಅಲಕಾಪುರಿಗೆ ಪ್ರಯಾಣ ಮಾಡಿದನು.

\begin{verse}
\textbf{ವಾಮದೇವೋಪಿ ಮಾಘಸ್ಯ ಕಥಿತಂ ತವ ಪುತ್ರಕ~।}\\\textbf{ಅನಂತಂ ತಸ್ಯ ಮಾಹಾತ್ಮ್ಯಂ ಲೇಶೋsಯಂ ಕಥಿತೋ ಮಯಾ~।। ೪೭~।। }
\end{verse}

\begin{verse}
\textbf{ಸಾಕಲ್ಯೇನ ಮಯಾ ವಕ್ತುಂ ನಾಲಂ ವರ್ಷಶತೈರಪಿ~।}\\\textbf{ನಕ್ಷತ್ರಾಣಿ ಚ ಗಣ್ಯಂತೇ ಗಣ್ಯಂತೇ ಚ ಕ್ಷಣಾದಯಃ~।। ೪೮~।।} 
\end{verse}

\begin{verse}
\textbf{ನರೇಣ ಯೇನ ವಾ ಲೋಕೇ ತೇನೇದಂ ಗಣ್ಯತಾಂ ಮುನೇ~।}\\\textbf{ಯಥಾ ಮುರಾರಿಃ ಸರ್ವೇಷು ದೇವೇಷು ಪ್ರವರೋ ಮತಃ~।। ೪೯~।।}
\end{verse}

ನಾರದನೇ, ವಾಮದೇವ ಮುನಿಗಳು ನಿರೂಪಿಸಿದ ಮಾಘ ಮಾಸ ಮಾಹಾತ್ಮ್ಯೆಯು ಅನಂತವಾಗಿದೆ. ಅದರ ಸ್ವಲ್ಪ ಭಾಗವು ನನ್ನಿಂದ ಹೇಳಲ್ಪಟ್ಟಿತು. ವಿಸ್ತಾರವಾಗಿ ಹೇಳಲು ನೂರು ವರ್ಷಗಳೂ ಸಾಲದು. ಯಾರು ನಕ್ಷತ್ರಗಳನ್ನು ಎಣಿಸಬಲ್ಲರೋ, ನಿಮಿಷವೇ ಮೊದಲಾದ ಕಾಲದ ಅಂಶಗಳನ್ನು ಯಾರು ಲೆಕ್ಕ ಮಾಡಬಲ್ಲರೋ ಅಂತಹವರು ಮಾಘಮಾಹಾತ್ಮ್ಯೆಯನ್ನು ಪೂರ್ಣವಾಗಿ ವರ್ಣಿಸಬಲ್ಲರು. ಎಲ್ಲ ದೇವತೆಗಳಲ್ಲಿ ಹೇಗೆ ವಿಷ್ಣುವು ಶ್ರೇಷ್ಠನೋ,

\begin{verse}
\textbf{ತಥಾ ಸರ್ವೇಷು ಧರ್ಮೇಷು ಮಾಘಧರ್ಮೋ ವಿಶಿಷ್ಯತೇ~।}\\\textbf{ಸುಖೋಪಾಸ್ಯೋ ಮಹಾಪುಣ್ಯಫಲದಃ ಪಾಪನಾಶನಃ~।। ೫೦~।।}
\end{verse}

ಹಾಗೆಯೇ ಸಕಲಧರ್ಮಗಳಲ್ಲಿಯೂ ಮಾಘಮಾಸದ ಧರ್ಮವು ತುಂಬಾ ಶ್ರೇಷ್ಠ, ಸುಖವಾಗಿ ಮಾಡಬಹುದಂತೆ ಇರುವುದು, ಹೆಚ್ಚಾದ ಪುಣ್ಯಕರವಾದುದು, ಸಮಸ್ತ ಪಾಪನಾಶಕರವು.

\begin{verse}
\textbf{ತಸ್ಮಾತ್ ಮಾಘಸಮೋ ಧರ್ಮೋ ನ ಭೂತೋ ನ ಭವಿಷ್ಯತಿ~।}\\\textbf{ಇಂದ್ರಾದ್ಯಾ ಲೋಕಪಾಲಾಶ್ಚ ದಿಲೀಪಾದ್ಯಾ ಮಹೀಭೃತಃ~।। ೫೧~।।} 
\end{verse}

\begin{verse}
\textbf{ಮಾಘಸ್ನಾನಸ್ಯ ಮಾಹಾತ್ಮ್ಯಾದಾಯುಃಸಿದ್ಧಿಂ ಪರಾಂ ಪುರಾ~।}\\\textbf{ದೇವತಾನಾಂ ಋಷೀಣಾಂ ಚ ಮಾಘಸ್ನಾನಮಭೀಪ್ಸಿತಮ್~।। ೫೨~।।}
\end{verse}

ಆದುದರಿಂದ ಮಾಘಮಾಸಕ್ಕೆ ಸಮಾನವಾದ ಧರ್ಮವು ಹಿಂದೆ ಇರಲಿಲ್ಲ, ಮುಂದೆ ಇರುವುದಿಲ್ಲ. ಇಂದ್ರನೇ ಮೊದಲಾದ ದಿಕ್ಪಾಲಕರು, ದಿಲೀಪನೇ ಮೊದಲಾದ ಚಕ್ರವರ್ತಿಗಳೂ, ಮಾಘಸ್ನಾನದ ಮಹಿಮೆಯಿಂದ ಹಿಂದೆ ಆಯುಃ ಸಿದ್ದಿಯನ್ನು ಪಡೆದರು. ದೇವತೆಗಳಿಗೂ, ಋಷಿಗಳಿಗೂ ಮಾಘಸ್ನಾನವು ಬಹಳ ಪ್ರಿಯ.

\begin{verse}
\textbf{ಸಂಸಾರೇ ಕಿಷ್ಯಮಾಣಾನಾಂ ಮನುಷ್ಯಾಣಾಂ ತು ಕಾ ಕಥಾ~।}\\\textbf{ಅಹೋ ಮಾಧವನಾಮಾಽಸ್ತಿ ಮಾಘಸ್ನಾನಂ ಮಹಾಫಲಮ್~।। ೫೩~।।}
\end{verse}

ಹೀಗಿರುವಲ್ಲಿ ಸಂಸಾರದಲ್ಲಿ ಸಿಕ್ಕು ಕಷ್ಟಪಡುತ್ತಿರುವವರ ಪಾಡೇನು? ಮಹಾಫಲಪ್ರದವಾದ ಮಾಘಸ್ನಾನವೂ, ಮಾಧವನ ನಾಮೋಚ್ಛಾರಣೆಯೇ ಅವರಿಗೆ ತಾರಕ.

\begin{verse}
\textbf{ತಥಾಪಿ ಮನುಜಾ ಲೋಕೇ ಕ್ಲಿಷ್ಯಂತಿ ತ್ಯೇತದದ್ಭುತಮ್~।}\\\textbf{ಸ್ವಾಧೀನ ಏವ ದೇಹಸ್ತು ನ ಕಸ್ಯಾಪಿ ವಶೇ ಜಲಮ್~।}\\\textbf{ಮಾನೋಽಯಂ ಸ್ವಯಮಭ್ಯೇತಿ ತಥಾಪ್ಯೇತೇ ಜಡಾ ಇತಿ~।। ೫೪~।।}
\end{verse}

ಹೀಗಿದ್ದರೂ ಜನರು ಸಂಸಾರದಲ್ಲಿ ಬಹಳ ಕಷ್ಟಪಡುವುದು ಆಶ್ಚರ್ಯ! ದೇಹವು ತನ್ನ ಅಧೀನ, ಜಲಾಶಯಗಳಿಗೆ ಅಧಿಕಾರಿಗಳು ಯಾರೂ ಇಲ್ಲ. ಮಾಘಮಾಸವು ತಾನಾಗಿಯೇ ಬರುತ್ತದೆ. ಆದರೂ ಜನರು ಜಡದಂತೆ ಇರುತ್ತಾರೆ.

\begin{verse}
\textbf{ಯದಾ ನೃದೇಹೋಽಸ್ತಿ ಯದಾ ದ್ವಿಜತ್ವಂ}\\\textbf{ಯದಾ ಚ ಪಟ್ವೀಕರಣ ಪ್ರವೃತ್ತಿಃ~।}\\\textbf{ತದೈವ ನೈಜಂ ಪದಮೀಪ್ಸಿತವ್ಯಂ }\\\textbf{ಕದಾ ಮೃತಿಸ್ತ್ವೇತಿ ಜನೋ ನ ವೇಸ್ತಿ~।। ೫೫~।।}
\end{verse}

ಮನುಷ್ಯನಾಗಿ ಹುಟ್ಟಿದ್ದಾಗ, ಬ್ರಾಹ್ಮಣ ಕುಲದಲ್ಲಿ ಬಂದಾಗ, ಅವಯವಗಳು ಸಮರ್ಥ\-ವಾಗಿ ಇರುವಾಗಲೇ ತನ್ನ ಹಿತಕ್ಕಾಗಿ ಸತ್ಕರ್ಮಾನುಷ್ಠಾನ ಮಾಡಿ ಮೋಕ್ಷಕ್ಕಾಗಿ ಅಭಿಲಾಷೆ ಹೊಂದಬೇಕು; ಮೃತ್ಯುವು ಯಾವಾಗ ಬರುತ್ತದೆಯೆಂಬುದನ್ನು ಜನರು ತಿಳಿಯಲಾರರು.

\begin{verse}
\textbf{ವೃಥಾ ಜನ್ಮ ಮನುಷ್ಯಾಣಾಮನಾರಾಧಯತಾಂ ಹರಿಮ್~।}\\\textbf{ಮೃತ್ಯಸ್ಯ ಪುರುಷಸ್ಯೇವ ಸರ್ವಾಲಂಕಾರಮಂಡಲಂ~।। ೫೬~।।}
\end{verse}

ಶ‍್ರೀ ಹರಿಯ ಸೇವೆಯನ್ನು ಮಾಡದೇ ಇರುವ ಮನುಷ್ಯನ ಜನ್ಮವು ವ್ಯರ್ಥವೇ! ಮೃತ\-ದೇಹಕ್ಕೆ ಸಮಸ್ತ ಆಭರಣಗಳಿಂದ ಅಲಂಕಾರ ಮಾಡಿದಂತೆ ಅವನ ಜನ್ಮವು ವ್ಯರ್ಥ.

\begin{verse}
\textbf{ಸರ್ವೇ ವೇದಾಃ ಸರ್ವಧರ್ಮಾಶ್ಚ ಲೋಕೇ}\\\textbf{ಸರ್ವಾಣಿ ತೀರ್ಥಾನಿ ಚ ಯಾನಿ ಲೋಕೇ~।}\\\textbf{ದಾನಾನಿ ಯಜ್ಞಾಶ್ಚ ವೃಥೈವ ತೇಷಾಂ}\\\textbf{ಯೇಷಾಂ ನ ಚೇತ್ ಭಕ್ತಿರನಂತಬಾಹೌ~।। ೫೭~।।}
\end{verse}

\begin{verse}
\textbf{ಪ್ರಾಯಶ್ಚಿತ್ತಾನಿ ಜೀರ್ಣಾನಿ ಶ್ರುತಿಸ್ಮೃತ್ಯುದಿತಾನ್ಯಪಿ~।}\\\textbf{ಮಾಧವೇ ಭಕ್ತಿಹೀನಂ ತು ನ ಪುನಂತಿ ಕದಾಚನ~।। ೫೮~।।}
\end{verse}

ಸಮಸ್ತ ವೇದಾಧ್ಯಯನವೂ, ಸರ್ವ ತೀರ್ಥಗಳಲ್ಲಿ ಸ್ನಾನ, ಅನೇಕ ವಿಧವಾದ ದಾನಗಳು, ಯಜ್ಞಗಳು - ಇವೆಲ್ಲವೂ ಶ‍್ರೀ ಹರಿಯಲ್ಲಿ ಭಕ್ತಿರಹಿತವಾದಲ್ಲಿ ವ್ಯರ್ಥ. ಶೃತಿಸ್ಮೃತಿಗಳಲ್ಲಿ ಹೇಳಿರುವ ಪ್ರಾಯಶ್ಚಿತ್ತಾದಿ ಕರ್ಮಗಳೂ ಸಹ ಶ‍್ರೀ ಮಾಧವನಲ್ಲಿ ಭಕ್ತಿರಹಿತರಾದವರನ್ನು ಪವಿತ್ರರನ್ನಾಗಿ ಮಾಡುವುದಿಲ್ಲ.

\begin{verse}
\textbf{ಮಾಘಸ್ನಾನಂ ವಿಷ್ಣು ಭಕ್ತಿರ್ಗೀತಾಪಾಠಶ್ಚ ಸಂತತಮ್~।}\\\textbf{ಸಹಸ್ರನಾಮಪಠನಂ ಸಾಲ್ಪಸ್ಯ ತಪಸಃ ಫಲಮ್~।}\\\textbf{ನಿತ್ಯಂ ಸಹಸ್ರನಾಮ್ನೋತ್ಥ ಶ್ಲೋಕಾರ್ಧಂ ಶ್ಲೋಕಮೇವ ವಾ~।। ೫೯~।।}
\end{verse}

\begin{verse}
\textbf{ಪ್ರಾತರ್ಮಾಘೇ ಪಠಿತ್ವಾ ತ್ರು ವಿಷ್ಣೋಃ ಸಾಯುಜ್ಯಮಾಪ್ನುಯಾತ್~।}\\\textbf{ಶ್ಲೋಕಾರ್ಧ ಶ್ಲೋಕಪಾದಂ ವಾ ನಿತ್ಯಂ ಭಾಗವತೋದ್ಭವಮ್~।। ೬೦~।।}
\end{verse}

\begin{verse}
\textbf{ಪಠಿತ್ವಾ ಮೋಕ್ಷಮಾಪ್ನೋತಿ ನಾತ್ರಾ ಕಾರ್ಯಾ ವಿಚಾರಣಾ~।। ೬೧~।।}
\end{verse}

ನಿರಂತರವಾದ ವಿಷ್ಣುಭಕ್ತಿ, ಗೀತಾಪ್ರವಚನ-ಪಾಠ, ಮಾಘಸ್ನಾನ, ವಿಷ್ಣುಸಹಸ್ರನಾಮ ಪಾರಾಯಣ ಇವು ಮಹಾ ತಪಸ್ಸಿನ ಫಲಕ್ಕೆ ಸಮಾನ. ಪ್ರತಿನಿತ್ಯವೂ ವಿಷ್ಣುಸಹಸ್ರನಾಮದ ಒಂದು ಶ್ಲೋಕವನ್ನಾಗಲಿ, ಅರ್ಧ ಶ್ಲೋಕವನ್ನಾಗಲಿ ಉಚ್ಛರಿಸಿದರೆ ವೈಕುಂಠದಲ್ಲಿ ಸಾಯುಜ್ಯ ಮೋಕ್ಷವು ಲಭಿಸುತ್ತದೆ. ನಿತ್ಯದಲ್ಲಿ ಭಾಗವತ ಗ್ರಂಥದ ಅರ್ಧ ಶ್ಲೋಕವನ್ನಾಗಲಿ ಕಾಲು ಶ್ಲೋಕವನ್ನಾಗಲಿ ಉಚ್ಛರಿಸಿದರೆ ನಿಶ್ಚಯವಾಗಿಯೂ ಮೋಕ್ಷ ದೊರೆಯುತ್ತದೆ. ಇದರಲ್ಲಿ ವಿಚಾರವಿಲ್ಲ.

[ವಿಶೇಷಾಂಶಃ:

\begin{verse}
\textbf{(೧) ಸ್ಮರ್ತವ್ಯಃ ಸತತಂ ವಿಷ್ಣುಃ ದಿವ್ಮರ್ತವ್ಯೋ ನ ಜಾತುಚಿತ್~।}\\\textbf{ಸರ್ವೇ ವಿಧಿನಿಷೇಧಾಃ ಸ್ಯುಃ ಯೇತಯೋರವ ಕಿಂಕರಾಃ~।।}\\\vauthor{(ಸದಾಚಾರ ಸ್ಮೃತಿ)}
\end{verse}

ವಿಷ್ಣುವನ್ನು ಸದಾ ಸ್ಮರಿಸುತ್ತಿರಬೇಕು; ಸ್ವಲ್ಪ ಹೊತ್ತು ಸಹ ಅವನನ್ನು ಮರೆಯಬಾರದು. ಶಾಸ್ತ್ರೋಕ್ತವಾದ ವಿಧಿನಿಷೇಧಗಳ ಫಲಗಳು ಇದನ್ನೇ ಅವಲಂಬಿಸಿದೆ.

\begin{verse}
\textbf{(೨) ನ ಹ್ಯಪುಣ್ಯವತಾಂ ಲೋಕೇ ಮೂಢಾನಾಂ ಕುಟಿಲಾತ್ಮನಾಮ್~।}\\\textbf{ಭಕ್ತಿರ್ಭವತಿ ಗೋವಿಂದೇ ಸ್ಮರಣಂ ಕೀರ್ತನಂ ತಥಾ~।।}\\\vauthor{ (ಕೃಷ್ಣಾಮೃತ ಮಹಾರ್ಣವ)}
\end{verse}

ಲೋಕದಲ್ಲಿ ಪುಣ್ಯವಂತರಲ್ಲದವರಿಗೂ, ಮೂರ್ಖರಿಗೂ, ದುಷ್ಟಬುದ್ಧಿಯಿಂದ ಯುಕ್ತ\-ರಾದವರಿಗೂ ಶ‍್ರೀ ಗೋವಿಂದನಲ್ಲಿ ಭಕ್ತಿಯು ಹುಟ್ಟುವುದಿಲ್ಲ. ಅಂತಹವರು ಶ‍್ರೀಹರಿಯ ಸ್ಮರಣೆ, ನಾಮ ಸಂಕೀರ್ತನಾದಿಗಳನ್ನು ಮಾಡುವುದಿಲ್ಲ.]

\begin{center}
ಇತಿ ಶ‍್ರೀ ವಾಯುಪುರಾಣೇ ಮಾಘಮಾಸ ಮಾಹಾತ್ಮ್ಯೇ, ಪಂಚಮೋಧ್ಯಾಯಃ~।
\end{center}

\begin{center}
ಶ‍್ರೀ ವಾಯುಪುರಾಣಾಂತರ್ಗತ ಮಾಘಮಾಸ ಮಾಹಾತ್ಮ್ಯೆಯಲ್ಲಿ \\ ಐದನೇ ಅಧ್ಯಾಯವು ಸಮಾಪ್ತಿಯಾಯಿತು.
\end{center}

\newpage

\section*{ಅಧ್ಯಾಯ\enginline{-}೬}

\emptypage

\begin{flushleft}
\textbf{ನಾರದ ಉವಾಚ:\enginline{-}}
\end{flushleft}

\begin{verse}
\textbf{ಸ್ನಾನಸ್ಯ ಮಾಘಮಾಸೇsಪಿ ಶಸ್ತಾ ಯಸ್ತಿಥಯೋ ಮತಾಃ~।}\\\textbf{ಸ್ಥಾನಾನಿ ಕಾನಿ ಶಸ್ತಾನಿ ಕಾರ್ಯಾಣ್ಯಪಿ ವ್ರತಾನಿ ಚ~।। ೧~।।}
\end{verse}

\begin{verse}
\textbf{ವದ ವಿಸ್ತರತೋ ಬ್ರಹ್ಮನ್ ಯದನ್ಯತ್ತೇ ವಿರೋಚತೇ~।। ೨~।।}
\end{verse}

\begin{flushleft}
ನಾರದರು ಪ್ರಶ್ನೆ ಮಾಡುತ್ತಾರೆ\enginline{-}
\end{flushleft}

ಮಾಘಮಾಸದಲ್ಲಿ ಸ್ನಾನಮಾಡಲು ಪ್ರಶಸ್ತವಾದ ತಿಥಿಗಳು ಯಾವುವು? ಶ್ರೇಷ್ಠವಾದ ಸ್ಥಳಗಳು ಯಾವುವು? ಆಚರಣೆ ಮಾಡಬೇಕಾದ ಶ್ರೇಷ್ಠವಾದ ಕರ್ಮಗಳು ಯಾವುವು? ಮತ್ತು ನಿಮಗೆ ಯುಕ್ತವೆಂದು ತೋರತಕ್ಕ ಇತರ ವಿಚಾರಗಳನ್ನೂ, ಬ್ರಹ್ಮದೇವರೇ! ವಿಸ್ತಾರವಾಗಿ ನಿರೂಪಿಸಿರಿ.

\noindent
\textbf{ಬ್ರಹ್ಮವಾಚ:\enginline{-}}

\begin{verse}
\textbf{ಯಥಾ ಗಂಗಾ ಸರ್ವದೇಶೇಷು ಪೂಜ್ಯಾ}\\\textbf{ಯಥಾ ವಿಷ್ಣುಃ ಸರ್ವದೇಶೇ ವಿಭುಶ್ಚ~। }\\\textbf{ಗಂಗಾ ಖ್ಯಾತಾ ಸ್ಥಾನಯೋರುತ್ಕಟಾಪಿ} \\\textbf{ಕಾಶ್ಯಾಂ ತ್ರಿವೇಣ್ಯಾಂ ಖಲು ಪೂಜ್ಯತೇ ಸಾ~।। ೩~।।}
\end{verse}

\begin{verse}
\textbf{ವ್ಯಾಪ್ತೋಽಪಿ ಭಗವಾನ್ ವಿಷ್ಣುಃ ತುಲಸೀಕಾನನೇಂಽಬುಜೇ~।}\\\textbf{ಅಶ್ವತ್ಥೇ ಗವಿ ವಿಪ್ರೇಷು ಯಥಾ ಸನ್ನಿಹಿತಃ ಸದಾ~।। ೪~।।}
\end{verse}

\begin{verse}
\textbf{ತಥಾ ಮಾಘಸ್ಯ ತಿಥಿಷು ಪ್ರಾಶಸ್ತ್ಯಂ ಕಥ್ಯತೇ ಮಯಾ~।}\\\textbf{ತಸ್ಯಾಂ ಸ್ನಾನಾದಿಕಂ ಸರ್ವಂ ಕೋಟಿಕೋಟಿಗುಣಂ ಭವೇತ್~।। ೫~।।}
\end{verse}

\begin{flushleft}
ಬ್ರಹ್ಮದೇವರು ಹೇಳುತ್ತಾರೆ.-
\end{flushleft}

ಗಂಗಾ ನದಿಯು ಹರಿಯುವ ಎಲ್ಲ ಪ್ರದೇಶಗಳಲ್ಲಿಯೂ ಪೂಜ್ಯ; ಸರ್ವ ಪ್ರದೇಶಗಳಲ್ಲಿಯೂ ವಿಷ್ಣುವು ಸರ್ವೋತ್ತಮನೆಂದು ಪೂಜ್ಯ; ಆದರೂ ಗಂಗೆಯು ಕಾಶಿಯಲ್ಲಿ, ತ್ರಿವೇಣಿ ಸಂಗಮಗಳಲ್ಲಿ ವಿಶೇಷವಾಗಿ ಪೂಜೆಯನ್ನು ಪಡೆಯುತ್ತದೆ. ಸರ್ವತ್ರ ವ್ಯಾಪ್ತನಾದರೂ ವಿಷ್ಣುವು ಅಶ್ವತ್ಥವೃಕ್ಷ, ಹಸು, ಬ್ರಾಹ್ಮಣರು, ತುಳಸೀವೃಕ್ಷಗಳ ಸಮೂಹಗಳಲ್ಲಿ ವಿಶೇಷ ಸನ್ನಿಧಾನದಿಂದ ಸದಾ ಇರುವಂತೆ ಮಾಘದಲ್ಲಿ ಹೆಚ್ಚು ಶ್ರೇಷ್ಠವಾದವುಗಳು ಯಾವುದೆಂಬುದನ್ನು ಹೇಳುತ್ತೇನೆ. ಆ ದಿನಗಳಲ್ಲಿ ಸ್ನಾನಾದಿಗಳನ್ನು ಮಾಡಿದರೆ ಎಲ್ಲವೂ ಕೋಟಿಯಷ್ಟು ಅಧಿಕ ಫಲಕಾರಿಯಾಗುತ್ತವೆ.

\begin{flushleft}
\textbf{[ವಿಶೇಷಾಂಶ:}
\end{flushleft}

\begin{verse}
\textbf{(೧) ಅಶ್ವತ್ಥ ತುಲಸೀವಿಪ್ರಾಸ್ತ್ರಿವಿಧಾ ಹರಿಮೂರ್ತಯಃ~।}\\\textbf{ತತ್ರಾಶ್ವತ್ಥ ತುಲಸ್ಯಾ ಚ ಸ್ಥಾವರತ್ವೇನ ದುಃಖಿತೇ~।।}\\\textbf{ತಥಾಪಿ ಭಜತಾಂ ನೄಣಾಂ ಸಿದ್ಧಿ ದಸ್ತತ್ರ ಕೇಶವಃ~।।} \vauthor{(ವಿಷ್ಣು ರಹಸ್ಯ)}
\end{verse}

ಅಶ್ವತ್ಥ, ತುಲಸೀ ಮತ್ತು ಬ್ರಾಹ್ಮಣರು-ಮೂರು ವಿಧದ ಶ‍್ರೀಹರಿಯ ಪ್ರತಿಮೆಗಳು, ಅವರಲ್ಲಿ ಸದಾ ಸನ್ನಿಹಿತನಾಗಿರುವನು. ಅದರಲ್ಲಿ ಅಶ್ವತ್ಥತುಲಸೀಗಳು ಸ್ಥಾವರರು (ಒಂದೇ ಪ್ರದೇಶದಲ್ಲಿ ಇರುವವರು--ಚಲಿಸುವವರಲ್ಲ), ದುಃಖಕ್ಕೆ ಈಡಾಗಿರುವವರು; ಹಾಗಿದ್ದರೂ ತಮ್ಮನ್ನು ಭಜಿಸುವವರಿಗೆ ಅವರಲ್ಲಿರುವ ಕೇಶವನು ಫಲ ನೀಡುತ್ತಾನೆ.

\begin{verse}
\textbf{(೨) ಅಶ್ವತ್ಥಃ ಸರ್ವವೃಕ್ಷಾಣಾಂ ದೇವರ್ಷೀಣಾಂ ಚ ನಾರದಃ~।।} \vauthor{(ಗೀತಾ)}
\end{verse}

ಸಮಸ್ತ ವೃಕ್ಷಗಳಲ್ಲಿ ಅಶ್ವತ್ಥದಲ್ಲಿ ವಿಶೇಷ ಸನ್ನಿಧಾನವನ್ನು ಇಟ್ಟಿರುತ್ತೇನೆ. ಆದುದರಿಂದ ಇತರ ವೃಕ್ಷಗಳಿಗಿಂತ ಶ್ರೇಷ್ಠ, ದೇವರ್ಷಿಗಳಲ್ಲಿ ಹಾಗೆಯೇ ನಾರದರು ಶ್ರೇಷ್ಠರು.]

\begin{verse}
\textbf{ತಿಥೀನಾಮಪಿ ಸರ್ವೇಷಾಂ ವ್ರತಾನಾಮಪಿ ನಾರದ~।}\\\textbf{ದ್ವಾದಶೀ ಸರ್ವತಃ ಶಸ್ತಾ ತಿಥಿಭ್ಯೋ ವಿಷ್ಣು ವಲ್ಲಭಾ~।। ೬~।।}
\end{verse}

ಎಲ್ಲ ತಿಥಿಗಳಿಗಿಂತಲೂ, ಎಲ್ಲ ವ್ರತಗಳಲ್ಲಿಯೂ, ವಿಷ್ಣುವಿಗೆ ಪ್ರಿಯವಾದ ದ್ವಾದಶಿಯು ಶ್ರೇಷ್ಠ.

\begin{verse}
\textbf{ಉಪರಾಗಸಹಸ್ರಾಣಿ ವ್ಯತೀಪಾತಶತಾನಿ ಚ~।}\\\textbf{ಅಮಾ ಲಕ್ಷಂ ತು ದ್ವಾದಶ್ಯಾಃ ಕಲಾಂ ನಾರ್ಹಂತಿ ಷೋಡಶೀಮ್~।। ೭~।।}
\end{verse}

ಸಹಸ್ರಾರು ಗ್ರಹಣಗಳೂ, ನೂರು ವ್ಯತೀಪಾತಗಳು, ಲಕ್ಷ ಅಮಾವಾಸ್ಯೆಗಳೂ ಸಹ ದ್ವಾದಶಿಯ ಶ್ರೇಷ್ಠತೆಯಲ್ಲಿ ಹದಿನಾರನೇ ಒಂದು ಭಾಗವನ್ನೂ ಪಡೆದಿಲ್ಲ.

\begin{verse}
\textbf{ದ್ವಾದಶ್ಯಾಂ ಮಾಘಮಾಸಸ್ಯ ಪಯಸಾ ಶಂಖವರ್ತಿನಾ~।}\\\textbf{ಸ್ನಾಪ್ಯ ತಂ ಮಾಧವಂ ದೇವಂ ವಿಷ್ಣುಲೋಕೇ ಮಹೀಯತೇ~।। ೮~।।}
\end{verse}

ಮಾಘದಲ್ಲಿ ದ್ವಾದಶೀ ದಿವಸ ಪರಮಾತ್ಮನಿಗೆ ಶಂಖದಲ್ಲಿ ಕ್ಷೀರಾಭಿಷೇಕ ಮಾಡುವುದರಿಂದ ವಿಷ್ಣುವಿನ ಲೋಕದಲ್ಲಿ ಪೂಜಿತನಾಗುವನು.

\begin{verse}
\textbf{ಯೋ ಮಾಘಶುದ್ಧ ದ್ವಾದಶ್ಯಾಂ ಶುದ್ಧೈಃ ಪಂಚಾಮೃತೈರ್ಹರಿಮ್~।}\\\textbf{ಸ್ನಾಪಯನ್ ಯಾತಿ ಸಾಯುಜ್ಯಂ ನಾತ್ರ ಕಾರ್ಯಾ ವಿಚಾರಣಾ~।। ೯~।।}
\end{verse}

\begin{verse}
\textbf{ಸಕೃತ್ ಘೃತೇನ ಸಂಸ್ನಾಪ್ಯ ಮಾಘೇ ಮಾಧವಮವ್ಯಯಮ್~।}\\\textbf{ದ್ವಾದಶ್ಯಾಂ ಯಾತಿ ಪೂರ್ಣಾಯುಃ ಪರತ್ರ ಚ ಪರಾಂ ಗತಿಮ್~।। ೧೦~।।} 
\end{verse}

\begin{verse}
\textbf{ದ್ವಾದಶ್ಯಾಂ ಭೋಜಯೇದ್ಯಸ್ತು ಶುದ್ಧಂ ಭಾಗವತಂ ದ್ವಿಜಮ್~।}\\\textbf{ಕೃತಂ ಬ್ರಾಹ್ಮಣಸಾಹಸ್ರಭೋಜನಂ ನಾತ್ರ ಸಂಶಯಃ~।। ೧೧~।।}
\end{verse}

ಮಾಘ ಶುದ್ಧ ದ್ವಾದಶಿಯಲ್ಲಿ ಶುದ್ಧವಾದ ಪಂಚಾಮೃತದಿಂದ ಪರಮಾತ್ಮನಿಗೆ ಅಭಿಷೇಕ ಮಾಡುವವರು ಸಾಯುಜ್ಯ ಮೋಕ್ಷವನ್ನು ಹೊಂದುತ್ತಾರೆ, ಸಂದೇಹವಿಲ್ಲ. ಆ ದಿನ ಘೃತದಿಂದ ಅಭಿಷೇಕ ಮಾಡುವವನು ಪೂರ್ಣವಾದ ಆಯುಸ್ಸನ್ನೂ ಇಹಪರಗಳಲ್ಲಿ ಸದ್ಗತಿಯನ್ನೂ ಪಡೆಯುವರು. ಆ ದಿನ ಶುದ್ಧನಾದ, ವಿಷ್ಣು ಭಕ್ತನಾದ ಒಬ್ಬ ಬ್ರಾಹ್ಮಣನಿಗೆ ಭೋಜನ ಮಾಡಿಸಿದರೂ ಸಹಸ್ರ ಬ್ರಾಹ್ಮಣರಿಗೆ ಭೋಜನ ಮಾಡಿಸಿದ ಫಲ ದೊರೆಯುತ್ತದೆ; ಸಂದೇಹವಿಲ್ಲ.

\begin{verse}
\textbf{ದೇವಾನ್ ವಿಷ್ಣುಂ ಸಮುದ್ದಿಶ್ಯ ಬ್ರಾಹ್ಮಣಾನ್ ಭೋಜಯೇದ್ಯದಿ~।}\\\textbf{ಮಾಘಸ್ಯ ಶುದ್ಧದ್ವಾದಶ್ಯಾಂ ದೇವಾನಾಮೃಣತಃ ಶುಚಿಃ~।। ೧೨~।।} 
\end{verse}

\begin{verse}
\textbf{ತಥೈವಾಸಿತದ್ವಾದಶ್ಯಾಂ ಪಿತೄನುದ್ದಿಶ್ಯ ನಾರದ~।}\\\textbf{ಬ್ರಾಹ್ಮಣಾನ್ ಭೋಜಯೇದ್ಯ ಸ್ತುಸ ಮುಕ್ತಃ ಪೈತೃಕಾದೃಣಾತ್~।। ೧೩~।।}
\end{verse}

ದ್ವಾದಶಿಯಲ್ಲಿ ದೇವತೆಗಳನ್ನೂ, ವಿಷ್ಣುವನ್ನೂ ಉದ್ದೇಶಿಸಿ ಯಾರು ಬ್ರಾಹ್ಮಣರಿಗೆ ಭೋಜನ ಮಾಡಿಸುವರೋ ಅವರು ದೇವತೆಗಳ ಋಣದಿಂದ ಮುಕ್ತರಾಗುವರು. ಶುದ್ಧ ದ್ವಾದಶಿಯಲ್ಲಿ ಪಿತೃಗಳನ್ನು ಉದ್ದೇಶಿಸಿ ಬ್ರಾಹ್ಮಣ ಭೋಜನ ಮಾಡಿಸುವವರು ಪಿತೃಗಳ ಋಣದಿಂದ ಮುಕ್ತ\-ರಾಗುತ್ತಾರೆ.

\begin{verse}
\textbf{ಯೋ ಭಾನುವಾರೇ ಮಾಘಸ್ಯ ಭೋಜಯೇದ್ವೇದಪಾರಗಮ್~।}\\\textbf{ಋಷೀಣಾಂ ಋಣತೋ ಮುಕ್ತೋ ನಾತ್ರ ಕಾರ್ಯಾ ವಿಚಾರಣಾ~।। ೧೪~।।}
\end{verse}

ಮಾಘಮಾಸದಲ್ಲಿ ಭಾನುವಾರ ವೇದಪರಾಯಣನಾದ ಬ್ರಾಹ್ಮಣನಿಗೆ ಭೋಜನ ಮಾಡಿಸುವುದರಿಂದ ಋಷಿಗಳ ಋಣದಿಂದ ಮುಕ್ತನಾಗುತ್ತಾನೆ, ಇಲ್ಲಿ ಸಂಶಯವಿಲ್ಲ.

[\textbf{ವಿಶೇಷಾಂಶ:} ದೇವ ಋಣ, ಋಷಿ ಋಣ, ಪಿತೃ ಋಣಗಳನ್ನು ಪ್ರತಿಯೊಬ್ಬ ಬ್ರಾಹ್ಮಣ ಪುರುಷನೂ ತೀರಿಸಬೇಕು. ಯಾವ ಯಾವ ಋಣಗಳನ್ನು ಯಾವ ಯಾವ ಕ್ರಮದಿಂದ ತೀರಿಸಬೇಕೆಂಬುದನ್ನು ಗ್ರಂಥಾಂತರಗಳಿಂದ ತಿಳಿಯಬಹುದು.]

\begin{verse}
\textbf{ಯದ್ದಾನಂ ಕ್ರಿಯತೇ ಮಾಘೇ ದ್ವಾದಶ್ಯಾಂ ಚ ವಿಪಶ್ಚಿತೇ~।}\\\textbf{ತದ್ದಾನಂ ಕೋಟಿಗುಣಿತಂ ಭವತ್ಯೇವ ನ ಸಂಶಯಃ~।। ೧೫~।।}
\end{verse}

ಮಾಘ ದ್ವಾದಶಿಯಲ್ಲಿ ವಿದ್ವಾಂಸನಾದ ಬ್ರಾಹ್ಮಣನಿಗೆ ಕೊಟ್ಟ ದಾನವು ಕೋಟಿ ಗುಣದಷ್ಟು ಅಧಿಕ ಫಲಕಾರಿಯಾಗುತ್ತದೆ; ಸಂಶಯವಿಲ್ಲ.

\begin{verse}
\textbf{ದ್ವಾದಶೀ ರವಿವಾರೇಣ ಸಂಕ್ರಮೇಣ ಚ ಯುಜ್ಯತೇ~।}\\\textbf{ಸಾ ತಿಥಿಃ ಪುಷ್ಕಲಾನಾಮ ಕ್ರತುಕೋಟಿಶತಾಧಿಕೌ~।। ೧೬~।।} 
\end{verse}

\begin{verse}
\textbf{ದ್ವಾದಶೀ ಭಾನುವಾರೇಣ ವ್ಯತೀಪಾತೇನ ಸಂಯುತಾ~।}\\\textbf{ಸಾ ತಿಥಿಃ ಶೃಂಖಲಾನಾಮ ತೀರ್ಥಯಾತ್ರಾಫಲಪ್ರದಾ~।। ೧೭~।। }
\end{verse}

\begin{verse}
\textbf{ದ್ವಾದಶೀ ಶನಿವಾರೇಣ ವಿಷ್ಣು ಭೇನ ಚ ಸಂಯುತಾ~।}\\\textbf{ಸಾ ತಿಥಿಃ ವೈಷ್ಣವೀ ನಾಮ ಮೋದತೇ ದುರ್ಲಭಾ ಕಲೌ~।। ೧೮~।।}
\end{verse}

ದ್ವಾದಶೀ ಭಾನುವಾರ ಸಂಕ್ರಮಣದಿಂದ ಕೂಡಿದ್ದರೆ ಆ ತಿಥಿಗೆ 'ಪುಷ್ಕಲಾ? ಎಂದು ಹೆಸರು. ಅದು ಒಂದು ಕೋಟಿ ಒಂದು ನೂರು ಅಧಿಕ ಫಲಕಾರಿ. ಭಾನುವಾರ, ದ್ವಾದಶೀ, ವ್ಯತೀಪಾತದಿಂದ ಸೇರಿದ್ದರೆ ಅದಕ್ಕೆ 'ಶೃಂಖಲಾ' ಎಂದು ಹೆಸರು. ಆ ದಿನದ ಸತ್ಕರ್ಮಗಳು ತೀರ್ಥಯಾತ್ರೆಯ ಫಲವನ್ನು ದೊರಕಿಸುತ್ತವೆ. ದ್ವಾದಶೀ, ಶನಿವಾರ, ಮೃಗಶಿರ ನಕ್ಷತ್ರದಿಂದ ಯುಕ್ತವಾಗಿದ್ದರೆ ಅದಕ್ಕೆ 'ವೈಷ್ಣವೀ? ಎಂದು ಹೆಸರು. ಕಲಿಯುಗದಲ್ಲಿ ಈ ದಿನ ದೊರೆಯುವುದು ದುರ್ಲಭ.

\begin{verse}
\textbf{ಶಸ್ತ್ರಾಜ್ಜಲಾದ್ವಿಷಾತ್ಸರ್ಪಾಚ್ಚಾಂಡಾಲಾತ್ಕುಲಿಶಾದಪಿ~।}\\\textbf{ವೃಷಾದಶ್ವಾದ್ಗಜಾದ್ವಾಪಿ ಕೃಶಾನೋಃ ಪಶುಬಂಧನಾತ್~।। ೧೯~।। }
\end{verse}

\begin{verse}
\textbf{ಜಿಹ್ವಾಚ್ಛೇದಾದ್ಗ ಲಗ್ರಾಹಾತ್ ಕಸ್ಮಾದ್ದುಃ ಖಶತಾದಪಿ~।}\\\textbf{ಇತ್ಯಾದಿಹೇತುರ್ಭಿಲೋಕೇ ದುರ್ಮತಿಂ ಯೇ ಗತಾಃ ನರಾಃ~।। ೨೦~।। }
\end{verse}

\begin{verse}
\textbf{ತೇಷಾಂ ಗತಿವಿಹೀನಾನಾಂ ದ್ವಾದಶ್ಯಾಂ ವಿಪ್ರಭೋಜನಮ್~।}\\\textbf{ಮಾಘೇ ಮಾಸಿ ದ್ವಿಜಶ್ರೇಷ್ಠ ಗತಿರುಕ್ತಾ ಮನೀಷಿಭಿಃ~।। ೨೧~।।}
\end{verse}

ಶಸ್ತ್ರದಿಂದ, ನೀರಿನ ದೆಸೆಯಿಂದ, ವಿಷಪಾನದಿಂದ, ಸರ್ಪ ಕಚ್ಚುವುದರಿಂದ, ಚಂಡಾಲರಿಂದ, ವಜ್ರದಿಂದಲೂ, ಎತ್ತು-ಕುದುರೆ-ಆನೆ-ಮೊದಲಾದ ಪಶುಗಳಿಂದಲೂ, ನಾಲಿಗೆ ಕತ್ತರಿಸುವುದರಿಂದ, ಕುತ್ತಿಗೆಯನ್ನು ಕತ್ತರಿಸುವುದರಿಂದ ಹೀಗೆ ನೂರಾರು ಕಾರಣಗಳಿಂದ ಸತ್ತು ಪಿಂಡಾದಿಗಳನ್ನು ಇಡಲು ಕರ್ತೃಗಳು ಇಲ್ಲದಿರುವವರ ಉದ್ದೇಶದಿಂದ ಮಾಘದಲ್ಲಿ ದ್ವಾದಶೀ ದಿನದಲ್ಲಿ ಬ್ರಾಹ್ಮಣರಿಗೆ ಭೋಜನ ಮಾಡಿಸುವುದರಿಂದ ಅಂತಹ ಗತಿಹೀನರಾದವರಿಗೆ ಸದ್ಗತಿಯಾಗುತ್ತದೆ.

\begin{verse}
\textbf{ಯೇ ಚ ಲೋಕೇ ಲುಪ್ತಪಿಂಡಾಃ ಕ್ರಿಯಾಲೋಪಗತಾಶ್ಚ ಯೇ~।}\\\textbf{ಯೇಷಾಂ ವಂಶೇ ನ ಕಸ್ಯಾಪಿ ದ್ವಾದಶ್ಯಾಂ ಭೋಜನೇ ಕೃತೇ~।। ೨೨~।।} 
\end{verse}

\begin{verse}
\textbf{ಮಾಘೇ ಮಾಸಿಗತಿಸ್ತೇಷಾಂ ಭವತ್ಯೇವ ನ ಸಂಶಯಃ}\\\textbf{ಅಸಂಸ್ಕೃತಪ್ರಮೀತಾನಾಂ ತ್ಯಜಿತಾನಾಂ ಕುಲಸ್ತ್ರಿಯಾಮ್~।। ೨೩~।। }
\end{verse}

\begin{verse}
\textbf{ಪತಿತಾನಾಂ ಮಾಘಮಾಸೇ ದ್ವಾದಶೀಭೋಜನಂ ಗತಿಃ~।}\\\textbf{ನ ದ್ವಾದಶೀ ಸಮದಿನಂ ನ ಭೂತಂ ನ ಭವಿಷ್ಯತಿ~।। ೨೪~।।}
\end{verse}

ಲೋಕದಲ್ಲಿ ಶ್ರಾದ್ಧಪಿಂಡ ಸಿಗದವರು, ಸತ್ಯವರಿಗೆ ಕ್ರಿಯಾದಲ್ಲಿ ಲೋಪವಾಗಿದ್ದರೆ, ಅಂತಹವರಿಗೆ ಅವರ ವಂಶದಲ್ಲಿ ದ್ವಾದಶಿಯಲ್ಲಿ ಭೋಜನ ಮಾಡಿಸದಿದ್ದರೂ, ಮಾಘಮಾಸ ದ್ವಾದಶಿಯಲ್ಲಿ ಬ್ರಾಹ್ಮಣ ಭೋಜನ ಮಾಡಿಸುವುದರಿಂದ, ಅಂತಹ ಮೃತಪಟ್ಟವರಿಗೆ ಸದ್ಗತಿಯಾಗುತ್ತದೆ; ಸಂಶಯವಿಲ್ಲ. ಶಾಸ್ತ್ರೋಕ್ತ ರೀತಿಯಿಂದ ಸಂಸ್ಕಾರವಾಗದೇ ಸತ್ತವರಿಗೆ, ಕುಲಸ್ತ್ರೀಯಳಾದ ಪತ್ನಿಯನ್ನು ತ್ಯಾಗ ಮಾಡಿದವರಿಗೆ, ಕರ್ತವ್ಯ ಭ್ರಷ್ಟತೆಯಿಂದ ಪಾಪಿಷ್ಠರಾದವರಿಗೆ ಮಾಘ ದ್ವಾದಶೀ ದಿನ ಬ್ರಾಹ್ಮಣ ಸಂತರ್ಪಣೆ ಮಾಡಿಸುವುದರಿಂದ ಸದ್ಗತಿಯಾಗುತ್ತದೆ. ನಾರದನೇ, ದ್ವಾದಶೀಗೆ ಸಮನಾದ ದಿವಸವು ಈಗಲೂ ಇಲ್ಲ, ಹಿಂದೆಯೂ ಇರಲಿಲ್ಲ. ಮುಂದೆಯೂ ಇರುವುದಿಲ್ಲ.

\begin{verse}
\textbf{ಮಾಘಸ್ಯೇಂದುಕ್ಷಯೇ ಭಾನೌ ಸ್ನಾನದಾನಾದಿಕಾಃ ಕ್ರಿಯಾಃ~।}\\\textbf{ಪಿತೄನುದ್ದಿಶ್ಯ ಕುರ್ವಂತಿ ಕೃತಕೃತ್ಯಾಃ ಕಲೌ ಯುಗೇ~।। ೨೫~।।}
\end{verse}

ಮಾಘಮಾಸ ಭಾನುವಾರದಲ್ಲಿ ಚಂದ್ರನು ಮುಳುಗುವ ಸಮಯದಲ್ಲಿ ಪಿತೃಗಳನ್ನು ಉದ್ದೇಶಿಸಿ ಸ್ನಾನಾದಿ ಸತ್ಕರ್ಮಗಳನ್ನು ಆಚರಿಸಿದರೆ, ಕಲಿಯುಗದಲ್ಲಿ ಕೃತಾರ್ಥರಾಗುವರು.

\begin{verse}
\textbf{ಅಮಾ ಸೌಮೇನ ಭೌಮೇನ ರವಿಣಾ ಗುರುಣಾ ತಥಾ~।}\\\textbf{ಸ ಯೋಗಃ ಪದ್ಮಕೋ ನಾಮ ಶಶಿಗ್ರಹಶತಾಧಿಕಃ~।। ೨೬~।। }
\end{verse}

\begin{verse}
\textbf{ಸ ಯೋಗೊ ಮಾಘಮಾಸೇ ಚ ಕೊಟೀಂದುಗ್ರಹಸಂಮತಃ~।}\\\textbf{ಯೋ ಮಘಾಯುಕ್ತ ಪೂರ್ಣಾಯಾಂ ಗುರುಣಾ ಚ ಮುನೀಶ್ವರ~।। ೨೭~।। }
\end{verse}

\begin{verse}
\textbf{ಪ್ರತಿಮಾಂ ಶಿಂಶುಮಾರಸ್ಯ ಸ್ವರ್ಚಿತಾಂ ಯೋ ದದಾತಿ ಚ~।}\\\textbf{ಮೃತಾಪತ್ಯಸ್ಯ ದೋಷೇಣ ಮುಚ್ಯತೇ ನಾತ್ರ ಸಂಶಯಃ~।। ೨೮~।।}
\end{verse}

ಅಮವಾಸ್ಯೆಯು ಸೋಮವಾರ, ಮಂಗಳವಾರ, ಭಾನುವಾರ ಅಥವಾ ಗುರುವಾರದಿಂದ ಯುಕ್ತವಾಗಿದ್ದರೆ ಅದಕ್ಕೆ 'ಪದ್ಮಕ' ಯೋಗವೆಂದು ಹೆಸರು. ನೂರು ಚಂದ್ರ ಗ್ರಹಣಗಳಿಗಿಂತ ಶ್ರೇಷ್ಠ, ಅಂತಹ ಪದ್ಮಕ ಯೋಗವು ಮಾಘ ಮಾಸದಲ್ಲಿ ಬಂದರೆ, ಒಂದು ಕೋಟಿ ಚಂದ್ರ ಗ್ರಹಣಗಳಷ್ಟು ಶ್ರೇಷ್ಠ. ನಾರದನೇ, ಯಾರು ಮಘ ನಕ್ಷತ್ರದಿಂದ ಯುಕ್ತವಾದ ಹುಣ್ಣಿಮೆಯು ಗುರುವಾರದಿಂದ ಯುಕ್ತವಾಗಿರಲು ಶಿಂಶುಮಾರನ ವಿಗ್ರಹವನ್ನು (ಶ‍್ರೀ ವಿಷ್ಣುವಿನ ಇನ್ನೊಂದು ರೂಪ) ತಾನೇ ಪೂಜಿಸಿ ದಾನಮಾಡಿದರೆ ಶಿಶುಹತ್ಯ ದೋಷದಿಂದ ಮುಕ್ತನಾಗುತ್ತಾನೆ, ಸಂಶಯವಿಲ್ಲ.

\begin{verse}
\textbf{ಮನಃ ಸರ್ವಗ್ರಹಾಧಾರ ಲೋಕಾನುಗ್ರಹಕಾರಕ~।}\\\textbf{ತವ ದಾನೇನ ಮೇ ಭಯಾನ್ಮೃತಿಸಂತಾನನಿಷ್ಕೃತಿಃ~।। ೨೯~।। }
\end{verse}

\begin{verse}
\textbf{ಇತಿ ದದ್ಯಾತ್ ಪ್ರಯತ್ನೇನ ನಾನ್ಯಥೇಯಂ ಭವಿಷ್ಯತಿ~।}\\\textbf{ಪೂರ್ಣಾಯಾಂ ಮಾಘಮಾಸಸ್ಯ ನ ಸ್ನಾಯಾದ್ಯೋ ಬಹಿರ್ಜಲೇ~।। ೩೦~।। }
\end{verse}

\begin{verse}
\textbf{ಶೂದ್ರಃ ಪಿಶಾಚತಾಂ ಯಾತಿ ಬ್ರಾಹ್ಮಣೋ ಬ್ರಹ್ಮರಾಕ್ಷಸಃ~।}\\\textbf{ಅಮಾಸಂ ಯೋ ನಯೇದ್ಭೋಗಾದಾಲಸಾದ್ವಾ ಮುನೀಶ್ವರ~।। ೩೧~।।} 
\end{verse}

\begin{verse}
\textbf{ನಾಕರೋನ್ಮಾಘಮಾಸೇ ತು ಸ್ನಾನದಾನಾದಿಕಾಃ ಕ್ರಿಯಾಃ~।}\\\textbf{ಮಾಘ್ಯಂ ಸ್ನಾನಂ ಸ ಕೃತ್ವಾ ಚ ಸಂಪೂರ್ಣಫಲಮಶ್ನುತೇ~।। ೩೨~।।}
\end{verse}

ಶಿಂಶುಮಾರನ ಪ್ರತಿಮೆಯನ್ನು ಕೊಡುವಾಗ “ಮನಃ ಸರ್ವಗ್ರಹಾಧಾರ ಲೋಕಾನುಗ್ರಹಕಾರಕ~। ತವ ದಾನೇನ ಮೇ ಭೂಯಾನ್ಮೃತಿಸಂತಾನನಿಷ್ಕೃತಿಃ” ಎಂಬ ಮಂತ್ರವನ್ನು ಪಠಿಸಿದರೆ ಆ ದಾನವು ಎಂದಿಗೂ ವ್ಯರ್ಥವಾಗುವುದಿಲ್ಲ. ಮಾಘ ಶುದ್ಧ ಪೌರ್ಣಿಮೆ ದಿನದಲ್ಲಿ ಊರ ಹೊರಗಿನ ಜಲಾಶಯದಲ್ಲಿ ಯಾರು ಸ್ನಾನ ಮಾಡುವುದಿಲ್ಲವೋ, ಅವನು ಶೂದ್ರನಾಗಿದ್ದರೆ ಮುಂದಿನ ಜನ್ಮದಲ್ಲಿ ಪಿಶಾಚಿಯಾಗುವನು; ಬ್ರಾಹ್ಮಣನಾಗಿದ್ದರೆ ಬ್ರಹ್ಮರಾಕ್ಷಸನಾಗುವನು. ಭೋಗೇಚ್ಛೆಯಿಂದಾಗಲೀ, ಆಲಸ್ಯದಿಂದಾಗಲೀ ಯಾರು ಮಾಘಮಾಸದಲ್ಲಿ ಸ್ನಾನಾದಿಗಳನ್ನು ಮಾಡುವುದಿಲ್ಲವೋ ಅವನು ಮಾಘ ಶುದ್ಧ ಪೌರ್ಣಿಮೆಯಲ್ಲಿ ಸ್ನಾನ ಮಾಡುವುದರಿಂದ ಸಂಪೂರ್ಣ ಫಲ ಹೊಂದುತ್ತಾನೆ.

\begin{verse}
\textbf{ಯೋ ಮಾತುಲಿಂಗಂ ಪ್ರದದಾತಿ ಮಾಘ್ಯಾ-}\\\textbf{ಮುದ್ದಿಶ್ಯ ಮಾಯಾಧವಮಾದಿಹೇತುಮ್~।}\\\textbf{ಸತ್ಕೀರ್ತಿಮಂತಂ ತನಯಂ ಪ್ರಸೂಯತೇ} \\\textbf{ಯಥಾರ್ಜುನಂ ಹೈಹಯದೇಶರಾಜಃ~।। ೩೩~।।}
\end{verse}

ಯಾರು ಮಾಘದಲ್ಲಿ ಪೌರ್ಣಿಮೆ ದಿನ ಮಾಧವನ ಉದ್ದಿಶ್ಯ ಮಾಡಿ ಮಾದಲ ಫಲವನ್ನು ದಾನ ಮಾಡುವರೋ ಅವರು ಹಿಂದೆ ಹೈಹಯದೇಶದ ರಾಜನಾದ ಕೃತವೀರ್ಯನು ಸಹಸ್ರಾರ್ಜುನನೆಂಬ ಮಗನನ್ನು ಪಡೆದಂತೆ ಒಳ್ಳೆಯ ಕೀರ್ತಿವಂತನಾದ ಪುತ್ರನನ್ನು ಪಡೆಯುತ್ತಾರೆ.

\begin{verse}
\textbf{ಯಃ ಪೌರ್ಣಮಾಸ್ಯಾಂ ಪಿತೃಭೇ ಗುರೌ‌ ಚ} \\\textbf{ದದಾತಿ ಗಾಂ ವತ್ಸಯುತಾಂ ದ್ವಿಜಾಯ~।}\\\textbf{ಸ ರೋಮಸಂಖ್ಯಾ ಕುಲಕಾನಿ ಲೋಕೇ} \\\textbf{ವಿಷ್ಣೋಶ್ಚ ಸಂಸ್ಥಾ ಪ್ಯ ಸ್ವಯಂ ಮುದಾಸ್ತೇ~।। ೩೪~।।}
\end{verse}

ಹುಣ್ಣಿಮೆಯಲ್ಲಿ ಯಾರು ಪಿತೃಗಳನ್ನು, ಗುರುಗಳನ್ನು ಉದ್ದೇಶಿಸಿ ಬ್ರಾಹ್ಮಣರಿಗೆ ಕರುವಿ\-ನಿಂದ ಯುಕ್ತವಾದ ಹಸುವನ್ನು ದಾನ ಮಾಡುತ್ತಾರೋ ಅವರು ಆನಂದದಿಂದ ರೋಮ\-ಸಂಖ್ಯಾಕವಾದ, ತನ್ನ ಕುಲಗಳನ್ನು ವಿಷ್ಣು ಲೋಕದಲ್ಲಿ ಸ್ಥಾಪಿಸುತ್ತಾನೆ.

\begin{verse}
\textbf{ನೀಲಂ ವೃಷಂ ಪಿಂಗಲಂ ವಾಪಿ ಮಾಘ್ಯಾಂ} \\\textbf{ಪಿತೄನ್ ಸಮುದ್ದಿಶ್ಯ ಯುವಾಸಮಗ್ರ್ಯಮ್~।}\\\textbf{ಶೂಲಾದಿಕಂ ಸೂನುರುತ್ಸರ್ಜಯಿತ್ವಾ} \\\textbf{ಗಯಾಶ್ರಾದ್ಧಂ ಕೋಟಿವಾರಂ ಕೃತಂ ಸ್ಯಾತ್~।। ೩೫~।।}
\end{verse}

ಮಾಘಮಾಸದಲ್ಲಿ ಯಾರು ನೀಲಿ ಬಣ್ಣದ ಅಥವ ಬೂದು ಬಣ್ಣದ ಚಿಕ್ಕ ವಯಸ್ಸಿನ ಎತ್ತನ್ನು ಪಿತೃಗಳ ಉದ್ದೇಶದಿಂದ ಉತ್ಸರ್ಜನ ಮಾಡುತ್ತಾರೆಯೋ ಅವರು ಕೋಟಿ ಸಲ ಗಯಾಶ್ರಾದ್ಧ ಮಾಡಿದ ಫಲವನ್ನು ಪಡೆಯುತ್ತಾರೆ.

\begin{verse}
\textbf{ಸ್ನಾನಾನ್ಮಾಘ್ಯಾತ್ ವೃಷೋತ್ಸರ್ಗಾತ್ ಶ್ರವಣಾತ್ಪೂಜನಾದಪಿ~।}\\\textbf{ಜೀವತೋ ವಾಕ್ಯ ಕರಣಾತ್ ಪಂಚಭಿಃ ಪುತ್ರ ತಾಮಿಯಾತ್~।। ೩೬~।।}
\end{verse}

ಮಾಘ ಪೌರ್ಣಿಮೆಯಲ್ಲಿ ಸ್ನಾನದಿಂದ, ವೃಷಭೋತ್ಸರ್ಗದಿಂದ, ಹರಿಕಥಾ ಶ್ರವಣದಿಂದ, ವಿಷ್ಣುವಿನ ಪೂಜೆಯಿಂದ, ಸಚ್ಛಾಸ್ತ್ರವಾಕ್ಯಾರ್ಥದಿಂದ-ಈ ಐದರಿಂದ ಸತ್ಪುತ್ರರು ಉತ್ಪನ್ನ\-ನಾಗುತ್ತಾರೆ.

\begin{verse}
\textbf{ಅನಪತ್ಯೋ ವೃಷೋತ್ಸರ್ಗಂ ಕುರ್ಯಾದ್ಯದಿ ಯಥಾವಿಧಿ~।}\\\textbf{ದೀರ್ಘಾಯುಷ್ಯಂ ಸುಪುತ್ರಂ ಚ ಪ್ರಾಪ್ನೋತೀತ್ಯಬ್ರವೀದ್ಧರಿಃ~।। ೩೭~।।}
\end{verse}

ಮಕ್ಕಳಿಲ್ಲದವರು ಮಾಘದಲ್ಲಿ ವಿಧಿಪೂರ್ವಕವಾಗಿ ವೃಷೋತ್ಸರ್ಗ ಮಾಡಿದರೆ\break ದೀರ್ಘಾಯುವಾದ ಒಳ್ಳೆಯ ಮಗನನ್ನು ಪಡೆಯುತ್ತಾರೆಂದು ಶ‍್ರೀಹರಿಯು ಹೇಳಿರುತ್ತಾನೆ.

\begin{verse}
\textbf{ವೃಷೋತ್ಸರ್ಗಸಮಂ ಲೋಕೇ ಕರ್ಮಗ್ರಂಥಿನಿಕೃಂತನಮ್~।}\\\textbf{ನ ದೃಷ್ಟಂ ನ ಶ್ರುತಂ ವಾಪಿ ಋಣತ್ರಯವಿಮೋಚಕಮ್~।। ೩೮~।।}
\end{verse}

ಕರ್ಮಬಂಧನದಿಂದ ಹಾಗೂ ಋಣತ್ರಯದಿಂದ ಮುಕ್ತನಾಗಲು ಲೋಕದಲ್ಲಿ ವೃಷೋ\-ತ್ಸರ್ಗಕ್ಕೆ ಸಮನಾದ ಉಪಾಯವು ಬೇರೊಂದು ನೋಡಲ್ಪಟ್ಟಿಲ್ಲ, ಕೇಳಲ್ಪಟ್ಟಿಲ್ಲ.

\begin{verse}
\textbf{ಗಯಾಶ್ರಾದ್ಧಂ ವೃಷೋತ್ಸರ್ಗಂ ಸಮಮಾಹುರ್ಮನೀಷಿಣಃ~।}\\\textbf{ಸಚೇನ್ಮಾಘ್ಯಾಂ ಕೋಟಿಗುಣ ಇತಿ ಪ್ರಾಹ ಜನಾರ್ದನಃ~।। ೩೯~।।}
\end{verse}

ಗಯಾಶ್ರಾದ್ಧದ ಪುಣ್ಯವೂ, ವೃಷೋತ್ಸರ್ಗದ ಪುಣ್ಯವೂ ಸಮವೆಂದು ಜ್ಞಾನಿಗಳು ಹೇಳು\-ತ್ತಾರೆ. ಮಾಘಮಾಸದಲ್ಲಿ ವೃಷೋತ್ಸರ್ಗವನ್ನು ಮಾಡಿದರೆ ಕೋಟಿಯಷ್ಟು ಹೆಚ್ಚು ಫಲದಾಯಕವೆಂದು ಶ‍್ರೀಹರಿಯು ಹೇಳಿರುತ್ತಾನೆ.

\begin{verse}
\textbf{ಮಾಘ ಶುಕ್ಲ ತ್ರಯೋದಶ್ಯಾಂ ಫಲರಾಶಿವಿಮೋಕ್ಷಣಮ್~।}\\\textbf{ತಸ್ಯ ವೈ ಸಂತತೇರ್ಹಾನಿರಾಕಲ್ಪಂ ನೈವ ಜಾಯತೇ~।। ೪೦~।। }
\end{verse}

ಮಾಘ ಶುದ್ಧ ತ್ರಯೋದಶಿಯಲ್ಲಿ ಫಲಗಳ ರಾಶಿಯನ್ನು ದಾನ ಮಾಡಿದವರಿಗೆ ಒಂದು ಕಲ್ಪದವರೆಗೂ ಸಂತತಿಗೆ ಹಾನಿಯಾಗುವುದಿಲ್ಲ.

\begin{verse}
\textbf{ಯೋಷಿನ್ಮಾಘತ್ರಯೋದಶ್ಯಾಂ ಸರುಕ್ಮಂ ತಾಲಪತ್ರಕಮ್~।}\\\textbf{ಸಾಂಜನಂ ಸಫಲಂ ಶೂರ್ಪಯುಗ್ಮಂ ಕಾರ್ಪಾಸಸಂಯುತಮ್~।। ೪೧~।। }
\end{verse}

\begin{verse}
\textbf{ಯಾ ಜೀವತ್ಪತಿಪುತ್ರಾಢ್ಯಾ ದದ್ಯಾತ್ಸಂಪೂಜ್ಯತಾಂ ಸತೀಮ್~।}\\\textbf{ಏತಸ್ಯಾಃ ಕಂಠಸೂತ್ರಸ್ಯ ಹಾನಿಃ ಕ್ವಾಪಿ ನ ಜಾಯತೇ~।। ೪೨~।।}
\end{verse}

ಮಾಘ ಶುದ್ಧ ತ್ರಯೋದಶಿ ದಿವಸ ಯಾವ ಸ್ತ್ರೀಯು ಭಂಗಾರದ ಕಂಠಸೂತ್ರ, ಕಾಡಿಗೆ, ತೆಂಗಿನಕಾಯಿ, ಹಣ್ಣುಗಳು, ಹತ್ತಿ ಇವುಗಳಿಂದ ಸಹಿತವಾದ ಮರದ ಬಾಗಿನವನ್ನು ಮಕ್ಕಳಿರುವ ಸುಮಂಗಲಿಗೆ ಆದರದಿಂದ ದಾನ ಮಾಡುವಳೋ ಅಂತಹ ಸ್ತ್ರೀಯ ಮಾಂಗಲ್ಯಕ್ಕೆ ಎಂದಿಗೂ ಹಾನಿಯುಂಟಾಗುವುದಿಲ್ಲ.

\begin{verse}
\textbf{ಮಾಘೇ ಮಾಸಿ ಚತುರ್ದಶ್ಯಾಂ ಯೋಽಪೂಪಾನ್ ಘೃತಪಾಚಿತಾನ್~।}\\\textbf{ದದ್ಯಾದ್ವಿಜಾಯ ವೈ ಭಕ್ತ್ಯಾ ಜ್ಞಾನಂ ಮೋಕ್ಷಂ ಚ ವಿಂದತಿ~।। ೪೩~।।}
\end{verse}

ಮಾಘ ಶುದ್ಧ ಚತುರ್ದಶಿಯಲ್ಲಿ ಯಾರು ತುಪ್ಪದಿಂದ ಮಾಡಿದ ಅಪೂಪಗಳನ್ನು ಭಕ್ತಿಯಿಂದ ಬ್ರಾಹ್ಮಣರಿಗೆ ದಾನಮಾಡಿದರೆ ಅವರು ಜ್ಞಾನವನ್ನೂ ಮೋಕ್ಷವನ್ನೂ ಹೊಂದುತ್ತಾರೆ.

\begin{verse}
\textbf{ಮಾಘೇ ತು ರಥಸಪ್ತಮ್ಯಾಂ ನಾರೀ ಶ್ರೇಷ್ಠ ಫಲೇಪ್ಸ ಯಾ~।}\\\textbf{ಉತ್ಥಾಯ ಪಶ್ಚಿಮೇ ಯಾಮಂ ಸ್ನಾತಾ ಧೌತಾ ವರಾ ಸತೀ~।। ೪೪~।। }
\end{verse}

\begin{verse}
\textbf{ರಥಮಾಲಿಖ್ಯ ಸವಿತುಃ ಗೋಷ್ಠೇ ವೃಂದಾವನೇಽಪಿ ಚ~।}\\\textbf{ರಥಂ ಪ್ರೀತ್ಯಾ ಹ್ಯಷ್ಟದಲಂ ಪದ್ಮಮಾಲಿಖ್ಯ ಧಾತುನಾ~।। ೪೫~।। }
\end{verse}

\begin{verse}
\textbf{ಸೂರ್ಯಮಂಡಲಗಂ ವಿಷ್ಣುಂ ತತ್ರ ಪದ್ಮೇ ಪ್ರಪೂಜಯೇತ್~।}\\\textbf{ರಕ್ತೈಃ ಪುಷ್ಪೈ ರಕ್ತಮಾಲ್ಯೈ ರಕ್ತಚಂದನವಾಸಸಾ~।। ೪೬~।।}
\end{verse}

ಒಳ್ಳೆಯ ಫಲಗಳನ್ನು ಇಚ್ಚಿಸುವ ಸ್ತ್ರೀಯು ರಥಸಪ್ತಮಿ ದಿನದಲ್ಲಿ ಉಷಃಕಾಲದಲ್ಲಿ ಎದ್ದು ಸ್ನಾನಮಾಡಿ ಶುಚಿರ್ಭೂತಳಾಗಿ ಹಸುವಿನ ಕೊಟ್ಟಿಗೆಯಲ್ಲಾಗಲೀ ಅಥವ ತುಳಸೀ ಬೃಂದಾವನದ ಸಮೀಪದಲ್ಲಿಯಾಗಲೀ ಸೂರ್ಯನ ರಥವನ್ನು ರಂಗೋಲೆಯಿಂದ ಬರೆದು ಕೆಮ್ಮಣ್ಣಿನಿಂದ ಅಷ್ಟದಳ ಪದ್ಮವನ್ನು ಬರೆಯಬೇಕು. ನಂತರ ಆ ಮಂಡಲದಲ್ಲಿ ಸೂರ್ಯಾಂತರ್ಗತ ನಾರಾಯಣನನ್ನು ಕೆಂಪು ಹೂ ಗಳಿಂದಲೂ, ಗಂಧದಿಂದಲೂ ಕೆಂಪು ವಸ್ತ್ರದಿಂದಲೂ ಭಕ್ತಿಯಿಂದ ಪೂಜಿಸಬೇಕು.

\begin{verse}
\textbf{ಗುಗ್ಗು ಲೇನ ಚ ಧೂಪೇನ ಪಾಯಸಸ್ಯೋಪಹಾರತಃ~।}\\\textbf{ತಾಂಬೂಲಂ ಚ ತತೋ ದದ್ಯಾನ್ನಿದ್ರಾಂ ಕುರ್ಯಾತ್ತತಃಪರಮ್~।। ೪೭~।।}
\end{verse}

ಧೂಪದೀಪಗಳನ್ನು ಸಮರ್ಪಿಸಿ ಪಾಯಸ, ತಾಂಬೂಲಗಳನ್ನು ನಿವೇದಿಸಿ. ನಂತರ ನಿದ್ರೆಯನ್ನು ಮಾಡಬೇಕು.

\begin{verse}
\textbf{ಸೂರ್ಯಮಂಡಲಗೋ ವಿಷ್ಣುಃ ಸ್ವಪ್ನೇ ವಕ್ತಿ ಹಿತಾಹಿತಮ್‌~।}\\\textbf{ದದಾತೀಷ್ಟಾಂಸ್ತಥಾ ಕಾಮಾನ್ ನಾನ್ಯಥೇದಂ ಕ್ವಚಿದ್ಭವೇತ್~।। ೪೮~।। }
\end{verse}

\begin{verse}
\textbf{ಸ್ನಾತ್ವಾ ತು ರಥಸಪ್ತಮ್ಯಾಂ ದದ್ಯಾತ್ ಕೂಷ್ಮಾಂಡಮುತ್ತಮಮ್~।}\\\textbf{ನಾರೀ ಸುತಮವಾಪ್ನೋತಿ ನರೋ ಹತ್ಯಾಂ ವ್ಯಪೋಹತಿ~।। ೪೯~।। }
\end{verse}

\begin{verse}
\textbf{ದ್ವಿಜೇಂದ್ರಂ ರಥಸಪ್ತಮ್ಯಾಂ ಪಾಯಸಂ ಭೋಜಯೇದ್ಯದಿ~।}\\\textbf{ಗ್ರಹಪೀಡಾದಿಭಿಃ ಸರ್ವೈ ರ್ಮುಚ್ಯತೇ ನಾತ್ರ ಸಂಶಯಃ~।। ೫೦~।।}
\end{verse}

ಸೂರ್ಯಮಂಡಲಸ್ಥಿತನಾದ ಶ‍್ರೀಹರಿಯು ಸ್ವಪ್ನದಲ್ಲಿ ಅಂತಹ ಸ್ತ್ರೀಗೆ ಹಿತಾಹಿತಗಳನ್ನು ಹೇಳಿ ಆಕೆಯ ಸಕಲ ಇಷ್ಟಾರ್ಥಗಳನ್ನೂ ಈಡೇರಿಸುವನು, ಇದು ಅಸತ್ಯವಲ್ಲ. ರಥಸಪ್ತಮಿ ದಿವಸ ಸ್ನಾನಮಾಡಿ ಒಳ್ಳೆಯ ಕೂಷ್ಮಾಂಡವನ್ನು ಬ್ರಾಹ್ಮಣರಿಗೆ ದಾನಕೊಟ್ಟರೆ, ಸ್ತ್ರೀಯು ಸುಪುತ್ರನನ್ನು ಪಡೆಯುವಳು; ಪುರುಷನು ಬ್ರಹ್ಮಹತ್ಯಾ ದೋಷದಿಂದ ಮುಕ್ತನಾಗುತ್ತಾನೆ. ರಥಸಪ್ತಮೀ ದಿನದಲ್ಲಿ ಬ್ರಾಹ್ಮಣನಿಗೆ ಪರಮಾನ್ನ ಭೋಜನ ಮಾಡಿಸುವುದರಿಂದ ಸಕಲ ಗ್ರಹಪೀಡೆಗಳಿಂದ ಬಿಡುಗಡೆ ಹೊಂದುತ್ತಾನೆ; ಇಲ್ಲಿ ಸಂಶಯವಿಲ್ಲ.

\begin{verse}
\textbf{ಯೋ ದದ್ಯಾದ್ರಥಸಪ್ತಮ್ಯಾಂ ತಿಲಪಾತ್ರಂ ದ್ವಿಜಾತಯೇ~।}\\\textbf{ವಿಮುಕ್ತಃ ಸರ್ವಪಾಪೇಭ್ಯೋ ಗ್ರಹಪೀಡಾಂ ವ್ಯಪೋಹತಿ~।। ೫೧~।।}
\end{verse}

\begin{verse}
\textbf{ಯಃ ಕುರ್ಯಾದ್ರಥಸಪ್ತಮ್ಯಾಂ ಹಿರಣ್ಯಂ ಶ್ರಾದ್ಧ ಮಂಜಸಾ~।}\\\textbf{ಪ್ರೀಣಂತಿ ಪಿತರಃ ಸರ್ವೇ ನೇಚ್ಛಂತ್ಯನ್ಯತ್ಕದಾಚನ~।। ೫೨~।।} 
\end{verse}

\begin{verse}
\textbf{ಪಿತೃಮುದ್ದಿಶ್ಯ ಯೋ ದದ್ಯಾದನ್ನಶ್ರಾದ್ಧಂ ತಥಾ ದಿನೇ~।}\\\textbf{ತದನ್ನಂ ಮೇರುಣಾ ತುಲ್ಯಂ ತೇಷಾಂ ಭವತಿ ನಿಶ್ಚಿತಮ್~।। ೫೩~।।}
\end{verse}

ಯಾರು ರಥಸಪ್ತಮಿಯಲ್ಲಿ ಎಳ್ಳು ತುಂಬಿದ ಪಾತ್ರೆಯನ್ನು ಬ್ರಾಹ್ಮಣರಿಗೆ ದಾನ ಕೊಡುತ್ತಾ\-ರೆಯೋ ಅವನು ಸಕಲ ಪಾಪಗಳಿಂದ ಮುಕ್ತನಾಗುತ್ತಾನೆ. ಗ್ರಹಪೀಡೆಗಳೂ ನಿಲ್ಲುತ್ತವೆ. ಆ ದಿನ ಪಿತೃಗಳಿಗಾಗಿ ಭಕ್ತಿಯಿಂದ ಹಿರಣ್ಯಶ್ರಾದ್ಧ ಆಚರಿಸಿದರೆ, ಪಿತೃಗಳು ಸಂಪೂರ್ಣ ತೃಪ್ತಿಯನ್ನು ಹೊಂದಿ ಮುಂದೆ ಇನ್ನೂ ಏನನ್ನೂ ಇಚ್ಚಿಸುವುದಿಲ್ಲ. ಆ ದಿನ ಪಿತೃಗಳಿಗೋಸ್ಕರ ಅಗ್ನಿ ಶ್ರಾದ್ಧ ಮಾಡಿದರೆ ಆ ಅನ್ನವು ಮೇರು ಪರ್ವತದ ಸಮಾನವಾಗುತ್ತದೆ, ಇದು ನಿಶ್ಚಯ.

\begin{verse}
\textbf{ಯೋ ಮಾಘೇ ರಥಸಪ್ತಮ್ಯಾಂ ಭೂಮೇರ್ದಾನಂ ಕರೋತಿ ಚ~।}\\\textbf{ಸ ಕೋಟಿಕಲ್ಪ ಪರ್ಯಂತಂ ವಿಷ್ಣು ರ್ಲೋಕೇ ಮಹೀಯತೇ~।। ೫೪~।।}
\end{verse}

\begin{verse}
\textbf{ರಾಜಸೂಯಾಶ್ವಮೇಧೇಭ್ಯೋ ಪ್ರಾಪ್ತಂ ಲೊಕೇ ಹಿ ಕ್ಷೀಯತೇ~।}\\\textbf{ಕ್ಷೀಯತೇ ರಥಸಪ್ತಮ್ಯಾಂ ಭೂದಾನಾಪ್ತಂ ನ ವೈ ಕ್ವಚಿತ್~।। ೫೫~।। }
\end{verse}

\begin{verse}
\textbf{ಯೋ ಮಾಘೇ ಸಿತಸಪ್ತಮ್ಯಾಂ ದದ್ಯಾದನ್ನಂ ದ್ವಿಜಾತಯೇ~।}\\\textbf{ತದನ್ನಂ ಕೋಟಿಗುಣಿತಂ ಭವತ್ಯೇವ ನ ಸಂಶಯಃ~।। ೫೬~।।}
\end{verse}

ರಥಸಪ್ತಮಿಯಲ್ಲಿ ಭೂದಾನ ಮಾಡುವವನು ಕೋಟಿಕಲ್ಪ ಪರ್ಯಂತ ವಿಷ್ಣು ಲೋಕದಲ್ಲಿ ಸುಖದಿಂದ ಇರುವನು. ರಾಜಸೂಯ, ಅಶ್ವಮೇಧ ಯಾಗಗಳಿಂದ ದೊರೆಯುವ ಫಲವು ಕೆಲವು ಕಾಲದ ನಂತರ ಕ್ಷೀಣವಾಗುತ್ತದೆ. ಆದರೆ ರಥಸಪ್ತಮಿ ದಿನ ಮಾಡಿದ ಭೂದಾನದ ಫಲವೂ ಎಂದಿಗೂ ಕ್ಷೀಣವಾಗುವುದಿಲ್ಲ. ಮಾಘ ಶುದ್ಧ ಸಪ್ತಮಿಯಂದು ಬ್ರಾಹ್ಮಣರಿಗೆ ಅನ್ನದಾನ ಮಾಡಿದರೆ ಅದು ಕೋಟಿಯಷ್ಟು ಹೆಚ್ಚು ಫಲದಾಯಕ, ಸಂಶಯವಿಲ್ಲ.

\begin{verse}
\textbf{ಯೋsತಸೀಪುಷ್ಪಮಾಲ್ಯೇನ ಸಪ್ತಮ್ಯಾ ಮರ್ಚಯೇದ್ದರಿಮ್~।}\\\textbf{ಭಸ್ಮಸಾದ್ಯಾಂತಿ ಪಾಪಾನಿ ಇಂಧನಾನೀವ ವಹ್ನಿನಾ~।। ೫೭~।।}
\end{verse}

ರಥಸಪ್ತಮಿಯಲ್ಲಿ ಪರಮಾತ್ಮನನ್ನು ಅತಸೀಪುಷ್ಪ (ಜವೇಗೋಧಿಯ ಹೂ) ಮಾಲೆಯಿಂದ ಪೂಜಿಸಿದರೆ ಅವನ ಪಾಪಗಳು ಬೆಂಕಿಯಿಂದ ಸೌದೆಗಳು ಭಸ್ಮವಾಗುವಂತೆ ನಷ್ಟವಾಗುತ್ತವೆ.

\begin{verse}
\textbf{ಚತಸ್ರಸ್ತಿಥಯೊ ಯೊಗ್ಯಾ ದುರ್ಲಭಾಸ್ತ್ವ ಕೃತಾತ್ಮನಾಮ್~।}\\\textbf{ದಶಮೀ ದ್ವಾದಶೀ ಪೂರ್ಣಾ ತಥಾ ಶುದ್ಧಾ ತ್ರಯೋದಶೀ~।। ೫೮~।।}
\end{verse}

ಸಾಧನೆ ಮಾಡಿಕೊಳ್ಳುವವರಿಗೆ ದಶಮೀ, ದ್ವಾದಶೀ, ಶುದ್ಧ ತ್ರಯೋದಶೀ ಮತ್ತು ಪೌರ್ಣಿಮೆಗಳು ಬಹಳ ಶ್ರೇಷ್ಠ ಮತ್ತು ದುರ್ಲಭ.

\begin{verse}
\textbf{ಭುಕ್ತ್ವಾ ದ್ವಿವಾರಂ ಸಪ್ತಮ್ಯಾಂ ಸ ಭವೇದ್ವಾಯಸೋ ಧ್ರುವಮ್~।}\\\textbf{ಅಹುತ್ವಾ ಚಾಪ್ಯದತ್ವಾ ಚ ಯಸ್ಯ ಗಚ್ಛತಿ ಸಪ್ತಮೀ~।। ೫೯~।।}
\end{verse}

\begin{verse}
\textbf{ಸ ತು ಚಾಂಡಾಲತಾಮೇತಿ ಸಪ್ತಜನ್ಮಸು ನಿಶ್ಚಿತಮ್~।}\\\textbf{ಅಸ್ನಾತ್ವಾ ಚಾಪ್ಯದತ್ವಾ ಚ ಮಾಘಶ್ಚಾಯಂ ಗತೋ ಯದಿ~।। ೬೦~।। }\\\textbf{ಪಾಪಂ ನ ನಶ್ಯತೇ ಭೂಪ ಕಲ್ಪಕೋಟಿಶತೈರಪಿ~।}
\end{verse}

ರಥಸಪ್ತಮಿಯಂದು ಎರಡು ಸಲ ಭೋಜನ ಮಾಡುವ ಪುರುಷನು ನಿಶ್ಚಯವಾಗಿ ಮುಂದಿನ ಜನ್ಮದಲ್ಲಿ ಕಾಗೆಯಾಗುತ್ತಾನೆ. ಸಪ್ತಮೀ ದಿವಸ ಯಾರು ಹವನ, ದಾನಾದಿಗಳನ್ನು ಮಾಡದೇ ವ್ಯರ್ಥವಾಗಿ ಕಳೆಯುತ್ತಾನೆಯೋ ಅವನು ಮುಂದಿನ ಏಳು ಜನ್ಮಗಳಲ್ಲಿ ಚಾಂಡಾಲನಾಗಿ ಹುಟ್ಟುತ್ತಾನೆ; ಇದು ನಿಶ್ಚಯ. ಯಾರು ಸಕಾಲದಲ್ಲಿ ಸ್ನಾನ, ದಾನಗಳನ್ನು ಮಾಡದೇ ಮಾಘಮಾಸವನ್ನು ವ್ಯರ್ಥವಾಗಿ ಕಳೆಯುತ್ತಾನೆಯೋ ಅವನ ಪಾಪಗಳು ನೂರು ಕೋಟಿ ಕಲ್ಪಗಳವರೆಗೂ ನಾಶ ಹೊಂದುವುದಿಲ್ಲ.

\begin{verse}
\textbf{ಮಾಘೇ ತು ರಥಸಪ್ತಮ್ಯಾಂ ಗುಪ್ತದಾನಂ ಕರೋತಿ ಯಃ~।। ೬೧~।।} 
\end{verse}

\begin{verse}
\textbf{ತಸ್ಯ ವಿಷ್ಣುಃ ಸ್ವಯಂ ದಾತುಂ ಜಾಗರೂಕೋ ಭವತ್ಯಲಮ್~।}\\\textbf{ಏಷ್ಟಾ ವ್ಯಾ ಬಹವಃ ಪುತ್ರಾ ಯದ್ಯೇಕೋಽಪಿ ಗಯಾಂ ವ್ರಜೇತ್~।। ೬೨~।। }
\end{verse}

\begin{verse}
\textbf{ಮಾಘೇ ತು ರಥಸಪ್ತಮ್ಯಾಂ ಅನ್ನಂ ದದ್ಯಾತ್ ದ್ವಿಜಾತಯೇ~।}\\\textbf{ವಾಂಛಾಂ ಕುರ್ವಂತಿ ಪಿತರ ಏಕೋ ವಾಽಸ್ಮತ್ಕುಲೇ ತ್ವಿತಿ~।। ೬೩~।।}
\end{verse}

ರಥಸಪ್ತಮಿಯಂದು ಯಾರು ಗುಪ್ತದಾನವನ್ನು ನೀಡುವರೋ ಅವರಿಗೆ ಇಷ್ಟಾರ್ಥಗಳನ್ನು ತಾನೇ ಕೊಡಲು ವಿಷ್ಣುವು ಜಾಗೃತನಾಗಿರುತ್ತಾನೆ. ಬಹಳ ಜನ ಪುತ್ರರಿದ್ದರೂ ಯಾರಾದರೂ ಒಬ್ಬರು ತಮ್ಮ ವಂಶದಲ್ಲಿ ಬಂದು, ಗಯಾ ಕ್ಷೇತ್ರಕ್ಕೆ ಹೋಗಿ, ಅಲ್ಲಿ ರಥಸಪ್ತಮಿ ದಿನ ಬ್ರಾಹ್ಮಣರಿಗೆ ಅನ್ನದಾನ ಮಾಡಬೇಕು. ಇಂತಹ ಒಬ್ಬನಾದರೂ ನಮ್ಮ ವಂಶದಲ್ಲಿ ಆಗಾಗ್ಗೆ ಹುಟ್ಟಲಿ ಎಂದು ಪಿತೃಗಳು ಇಚ್ಛಿಸುತ್ತಾರೆ.

\begin{center}
ಇತಿ ಶ‍್ರೀ ವಾಯುಪುರಾಣೇ ಮಾಘಮಾಸಮಾಹಾತ್ಮ್ಯೇ ಷಷ್ಠೋsಧ್ಯಾಯಃ 
\end{center}

\begin{center}
ಶ‍್ರೀ ವಾಯುಪುರಾಣಾಂತರ್ಗತ ಮಾಘಮಾಸ ಮಾಹಾತ್ಮ್ಯೆಯಲ್ಲಿ \\ ಆರನೇ ಅಧ್ಯಾಯವು ಸಮಾಪ್ತಿಯಾಯಿತು.
\end{center}

\newpage

\section*{ಅಧ್ಯಾಯ\enginline{-}೭}

\emptypage

\begin{flushleft}
\textbf{ಬ್ರಹ್ಮೋವಾಚ\enginline{-} }
\end{flushleft}

\begin{verse}
\textbf{ಮಾಘಸ್ನಾನಂ ಪ್ರಯಾಗೇ ತು ಪ್ರಶಂಸಂತಿ ಮನೀಷಿಣಃ~।}\\\textbf{ತುಲಾಸ್ನಾನಂ ತು ಕಾವೇರ್ಯಾಂ ಪ್ರಶಸ್ತಮಿತಿ ಮೇ ಮತಿಃ~।। ೧~।। }
\end{verse}

\begin{verse}
\textbf{ಮೇಷಸ್ನಾನಂ ತು ರೇವಾಯಾಂ ತುಂಗಾಯಾಂ ಕರ್ಕಟೇ ತಥಾ~।}\\\textbf{ಸ್ನಾನಾನ್ಯೇತಾನಿ ಶಸ್ತಾನಿ ಪ್ರಾತಃಸ್ನಾನೇಷು ನಾರದ~।। ೨~।।}
\end{verse}

\begin{flushleft}
 ಬ್ರಹ್ಮದೇವರು ಹೇಳುತ್ತಾರೆ- 
\end{flushleft}

ನಾರದನೇ, ಪ್ರಯಾಗ ಕ್ಷೇತ್ರದಲ್ಲಿ ಮಾಘಸ್ನಾನವು ಶ್ರೇಷ್ಠವೆಂದು ಜ್ಞಾನಿಗಳು ಹೇಳುತ್ತಾರೆ. ನನ್ನ ಅಭಿಪ್ರಾಯದಲ್ಲಿ ತುಲಾಮಾಸದಲ್ಲಿ ಕಾವೇರಿ ನದಿ ಸ್ನಾನವು ಶ್ರೇಷ್ಠ. ಮೇಷಮಾಸದಲ್ಲಿ ರೇವಾ ನದಿಯಲ್ಲಿ ಪ್ರಾತಃಕಾಲದಲ್ಲಿ ಸ್ನಾನಮಾಡುವುದು ಹಾಗೂ ಕರ್ಕಾಟಕ ಮಾಸದಲ್ಲಿ ತುಂಗಾನದಿಯಲ್ಲಿ ಸ್ನಾನಮಾಡುವುದು ಶ್ರೇಷ್ಠ.

\begin{verse}
\textbf{ನದ್ಯಃ ಸಮುದ್ರಗಾಃ ಸರ್ವಾಸ್ತಾಸು ಸ್ನಾನಂ ಪ್ರಶಸ್ಯತೇ~।}\\\textbf{ತ್ರಿರಾತ್ರಂ ಯಜ್ಞಫಲದಾ ಯಾಃ ಕಶ್ಚಿದಸಮುದ್ರಗಾಃ~।। ೩~।। }
\end{verse}

ಸಮುದ್ರಕ್ಕೆ ಬೀಳುವ ನದಿಗಳಲ್ಲಿ ಸ್ನಾನವು ಶ್ರೇಷ್ಠವೇ. ಸಮುದ್ರದಲ್ಲಿ ಬೀಳದೇ ಇರುವ ನದಿಗಳಲ್ಲಿ ಮೂರು ದಿನಗಳ ಸ್ನಾನದಿಂದ ಯಜ್ಞಫಲವು ದೊರೆಯುತ್ತದೆ

\begin{verse}
\textbf{ಸ್ನಾನೇ ಫಲಂ ಪ್ರಯಚ್ಛಂತಿ ಫಲದಾ ಚ ಸರಸ್ವತೀ~।}\\\textbf{ವಸಂತಯಾಗಫಲದಾ ಕಾಲಿಂದೀ ಗೋಸವಪ್ರದಾ~।। ೪~।।} 
\end{verse}

\begin{verse}
\textbf{ಜಾಹ್ನವೀ ವತ್ಸರಾಖ್ಯಸ್ಯ ಯಜ್ಞಸ್ಯ ಫಲದಾ ಸ್ಮೃತಾ~।}\\\textbf{ಪದ್ಮಾಕರಂ ರಥಕ್ರಾಂತಸವನಸ್ಯ ಫಲಪ್ರದಮ್~।। ೫~।।} 
\end{verse}

\begin{verse}
\textbf{ತಥಾ ಸಂಕ್ರಾಂತಫಲದಂ ದೇವಖಾತಂ ಮುನೀಶ್ವರ~।}\\\textbf{ಸೌತ್ರಾಮಣ್ಯಾಶ್ಚ ಫಲದಂ ತಟಾಕಂ ಪಲ್ಲವಂ ತಥಾ~।। ೬~।।} 
\end{verse}

\begin{verse}
\textbf{ಬಹಿರ್ವಾಪ್ಯೋಪ್ಯವಭೃಥಫಲದಾಃ ಪರಿಕೀರ್ತಿತಾಃ~।}\\\textbf{ತತ್ತತ್ ಸ್ಥಾನೇ ಫಲಯುತಂ ಮಾಘೇ ಕೋಟಿಗುಣಂ ಭವೇತ್~।। ೭~।।}
\end{verse}

ಸರಸ್ವತೀ ನದಿಯ ಸ್ನಾನವು 'ವಸಂತಯಾಗ'ದ ಫಲವನ್ನೂ, ಯಮುನೆಯು 'ಗೋಮೇಧ' ಯಜ್ಞದ ಫಲವನ್ನೂ, ಗಂಗಾನದಿಯು 'ವತ್ಸರಾ' ಎಂಬ ಯಜ್ಞದ ಫಲವನ್ನೂ, ದೊಡ್ಡ ಸರೋವರಗಳ ಸ್ನಾನವು 'ರಥಕ್ರಾಂತ' ಯಜ್ಞದ ಫಲವನ್ನೂ, ಸಮುದ್ರವು ಸಂಕ್ರಮಣದ ಫಲವನ್ನೂ ಕೆರೆಗಳು, ಸಣ್ಣ ಸಣ್ಣ ಪುಷ್ಕರಣಿಗಳೂ 'ಸೌತ್ರಾಮಣಿ' ಎಂಬ ಯಜ್ಞದ ಫಲವನ್ನೂ, ಊರ ಹೊರಗಿನ ಇಳಿಯುವ ಭಾವಿ ಸ್ನಾನವು ಯಜ್ಞದ ಅವಭೃತದ ಫಲವನ್ನೂ ಕೊಡುತ್ತವೆ. ಆದರೆ ಅದೇ ಜಲಾಶಯಗಳು ಮಾಘಮಾಸದಲ್ಲಿ ಸ್ನಾನ ಮಾಡಿದರೆ ಕೋಟಿಯಷ್ಟು ಅಧಿಕ ಫಲವನ್ನು ಕೊಡುತ್ತವೆ.

\begin{verse}
\textbf{ಮಾಘೇ ಸಪ್ತನದೀಸಂಗೇ ಸ್ನಾನಂ ಯಃ ಕುರುತೇ ನರಃ~।}\\\textbf{ರಾಜಸೂಯಾಶ್ವಮೇಧಾನಾಂ ಫಲಂ ಪ್ರಾಪ್ನೋತ್ಯ ಸಂಶಯಮ್~।। ೮~।। }
\end{verse}

\begin{verse}
\textbf{ಯಃ ಕುರ್ಯಾತ್ ಸಪ್ತಗಂಗಾಸು ಯಥಾಪ್ರಾಪ್ತಾಸು ಮಾನವಃ~।}\\\textbf{ಮಾಘಸ್ನಾನಂ ಮುನಿಶ್ರೇಷ್ಠ ತೇಷಾಂ ಪುಣ್ಯಮನಂತಕಮ್~।। ೯~।। }
\end{verse}

\begin{verse}
\textbf{ಜಾಹ್ನವೀ ವೃದ್ಧಗಂಗಾ ಚ ಕಾಲಿಂದೀ ಚ ಸರಸ್ವತೀ~।}\\\textbf{ಕಾವೇರೀ ನರ್ಮದಾ ವೇಣೀ ಸಪ್ತಗಂಗಾಃ ಪ್ರಕೀರ್ತಿತಾಃ~।। ೧೦~।।}
\end{verse}

ಯಾರು ಮಾಘಮಾಸದಲ್ಲಿ 'ಸಪ್ತ ನದೀ ಸಂಗ' ದಲ್ಲಿ ಸ್ನಾನ ಮಾಡುವನೋ ಅವನು\break ರಾಜಸೂಯ-ಅಶ್ವಮೇಧ ಯಜ್ಞದ ಫಲವನ್ನು ಹೊಂದುತ್ತಾನೆ, ಸಂಶಯವಿಲ್ಲ. ನಾರದನೇ! ಸಪ್ತಗಂಗಾ ಎಂದು ಪ್ರಸಿದ್ದವಾದ ನದಿಗಳಲ್ಲಿ ಸ್ನಾನ ಮಾಡಿದರೆ ಅವರ ಪುಣ್ಯವು ಅನಂತ, ಗಂಗಾ, ಗೋದಾವರೀ, ಯಮುನಾ, ಸರಸ್ವತೀ, ಕಾವೇರೀ, ನರ್ಮದಾ, ವೇಣೀ-ಈ ನದಿಗಳು ಸಪ್ತ ಗಂಗಾನದಿಗಳೆಂದು ಪ್ರಸಿದ್ಧವಾಗಿವೆ.

\begin{verse}
\textbf{ಮಾಘೇ ಮಾಸಿ ಸಕೃತ್ ಸ್ನಾತ್ವಾ ಗೋಮತ್ಯಾಂ ಕೃಷ್ಣ ಸಂನಿಧೌ~।}\\\textbf{ಸಾಯುಜ್ಯ ಪದವೀಂ ಯಾತಿ ದುರಾತ್ಮಾಪಿ ಕೃತಾತ್ಮನಾಮ್~।। ೧೧~।। }
\end{verse}

\begin{verse}
\textbf{ಯೋ ಮಾಘೇ ಪ್ರಾತರುತ್ಥಾಯ ಗೋಮತೀಂ ಸಂಸ್ಮರೇದ್ಯದಿ~।}\\\textbf{ಸ ಗೋಮತೀಫಲಂ ಸದ್ಯಃ ಪ್ರಾಪ್ನೋತಿ ಕಿಮು ಸೇವಯಾ~।। ೧೨~।।} 
\end{verse}

\begin{verse}
\textbf{ಕ್ಷಣಮಾತ್ರಂ ವಸೇದ್ಯಸ್ತು ಗೋಮತ್ಯಾಂ ಕೃಷ್ಣ ಸಂನಿಧೌ~।}\\\textbf{ಮಾಘೇ ಮಾಸಿ ಮುನಿಶ್ರೇಷ್ಠ ಸ ಯೋಗೀ ಭವತಿ ಧ್ರುವಮ್~।। ೧೩~।।}
\end{verse}

ಯಾರು ಮಾಘಮಾಸದಲ್ಲಿ ಗೋಮತೀನದಿಯಲ್ಲಿ ಕೃಷ್ಣನ ಸನ್ನಿಧಿಯಲ್ಲಿ ಒಂದು ಬಾರಿ ಸ್ನಾನ ಮಾಡುತ್ತಾರೆಯೋ ಅವರು ಪಾಪಿಷ್ಠರಾದರೂ ಸಹ ಕೃತಕತ್ಯರಾಗಿ ಸಾಯುಜ್ಯ ಪದವಿಯನ್ನು ಹೊಂದುತ್ತಾರೆ. ಮಾಘಮಾಸದಲ್ಲಿ ಪ್ರಾತಃಕಾಲದಲ್ಲಿ ಯಾರು ಗೋಮತೀ ನದಿಯನ್ನು ಸ್ಮರಿಸುತ್ತಾರೆಯೋ ಅವರು ಆ ನದಿಯಲ್ಲಿ ಸ್ನಾನ ಮಾಡಿದಷ್ಟೇ ಫಲ ಪಡೆಯುತ್ತಾರೆ. ನಿಜವಾಗಿ ಸ್ನಾನ ಮಾಡಿದರೆ ಫಲ ಪಡೆಯುತ್ತಾರೆಂದು ಪ್ರತ್ಯೇಕ ಹೇಳಬೇಕೆ? ಮಾಘಮಾಸದಲ್ಲಿ ಗೋಮತೀ ತೀರದಲ್ಲಿ ಒಂದು ಕ್ಷಣಮಾತ್ರ ವಾಸಮಾಡುವುದರಿಂದ ನಿಶ್ಚಯವಾಗಿ ಯೋಗಿಯಾಗುತ್ತಾನೆ.

\begin{verse}
\textbf{ಕಾವೇರ್ಯಾಂ ಮಾಘಮಾಸೇ ತು ಮಕರಸ್ಥೇ ದಿವಾಕರೇ~।}\\\textbf{ತ್ರಿರಾತ್ರಂ ಸ್ನಾನಮಾತ್ರೇಣ ಮುಚ್ಯತೇ ಪಾಪಕೋಟಿಭಿಃ~।। ೧೪~।। }
\end{verse}

\begin{verse}
\textbf{ಕಾವೇರೀಸಲಿಲೇ ಸ್ನಾತ್ವಾ ಮಾಘೇ ಮಾಸಿ ದ್ವಿಜೋತ್ತಮ~।}\\\textbf{ಬ್ರಹ್ಮಹತ್ಯಾ ಸಹಸ್ರೈರ್ವಾ ಮುಚ್ಯತೇ ನಾತ್ರ ಸಂಶಯಃ~।। ೧೫~।।} 
\end{verse}

\begin{verse}
\textbf{ಮಾಘೇ ಪುಷ್ಕರಮಾಸಾದ್ಯ ಯಃ ಸ್ನಾನಂ ಕರ್ತುಮಿಚ್ಛತಿ~।}\\\textbf{ತಾವತಾ ಮುಚ್ಯತೇ ಪಾಪೈಃ ಕಿಮು ಸ್ನಾನೇ ಮುನೀಶ್ವರ~।। ೧೬~।।}
\end{verse}

ಮಾಘಮಾಸದಲ್ಲಿ ಸೂರ್ಯನು ಮಕರರಾಶಿಯಲ್ಲಿರುವಾಗ ಕಾವೇರಿ ನದಿಯಲ್ಲಿ ಮೂರು ದಿವಸ ಸ್ನಾನ ಮಾಡುವುದರಿಂದ ಒಂದು ಕೋಟಿ ಪಾಪಗಳಿಂದ ಮುಕ್ತನಾಗುತ್ತಾನೆ. ಮಾಘಮಾಸದಲ್ಲಿ ಕಾವೇರಿಯಲ್ಲಿ ಸ್ನಾನಮಾಡುವುದರಿಂದ ಸಹಸ್ರ ಬ್ರಹ್ಮಹತ್ಯ ದೋಷವು ನಾಶವಾಗುತ್ತದೆ. ಮಾಘಮಾಸದಲ್ಲಿ ಯಾರು ಪುಷ್ಕರ ಕ್ಷೇತ್ರಕ್ಕೆ ಹೋಗಿ ಅಲ್ಲಿ ಸ್ನಾನಮಾಡಲು ಇಚ್ಛಿಸುವ ಮಾತ್ರದಿಂದ ಸಕಲ ಪಾಪಗಳೂ ನಾಶ ಹೊಂದುತ್ತವೆ; ಇನ್ನು ಸ್ನಾನವನ್ನೇ ಮಾಡಿದರೆ ಹೇಳುವುದೇನು!

\begin{verse}
\textbf{ಪ್ರಭಾಸೇ ಸ್ನಾನಮಾತ್ರೇಣ ಮಾಘೇ ಮಕರಗೇ ರವೌ~।}\\\textbf{ಸ ವಿಶ್ವಜಿದ್ಯಜ್ಞ ಫಲಂ ದಿನಮಾತ್ರೇಣ ಗಚ್ಛತಿ~।। ೧೭~।। }
\end{verse}

\begin{verse}
\textbf{ಮಾಘೇ ತ್ವಿಂದುಕ್ಷಯೇ ಪ್ರೋಕ್ತೋ ಗೋಕರ್ಣೆ ಸಾಗರೇ ತು ಯಃ~।}\\\textbf{ಸ್ನಾನಂ ಶತಕ್ರತುಫಲಂ ಸರ್ವಕರ್ಮವಿಮೋಚನಮ್~।। ೧೮~।। }
\end{verse}

\begin{verse}
\textbf{ಕುರುಕ್ಷೇತ್ರೇ ತು ಯಃ ಕುರ್ಯಾತ್ಸಪ್ತಮ್ಯಾಂ ಭಾನುವಾಸರೇ~।}\\\textbf{ಸಂಪೂಜ್ಯ ಮಾಧವಂ ದೇವಂ ಕುಷ್ಠರೋಗಾದ್ವಿಮುಚ್ಯತೇ~।। ೧೯~।।} 
\end{verse}

\begin{verse}
\textbf{ಯೋ ಭಾನುರಥಸಪ್ತಮ್ಯಾಂ ಪುಷ್ಕರೇ ಮಕರೇ ರವೌ~।}\\\textbf{ಸ್ನಾತ್ವಾ ಸಂತರ್ಪ್ಯ ಚ ಪಿತೄನ್ ಮುಚ್ಯತೇ ಪೈತೃಕಾದೃಣಾತ್~।। ೨೦~।।}
\end{verse}

ಮಾಘಮಾಸದಲ್ಲಿ ಸೂರ್ಯನು ಮಕರರಾಶಿಯಲ್ಲಿರುವಾಗ ಪ್ರಭಾಸದಲ್ಲಿ ಒಂದು ದಿನ ಸ್ನಾನ ಮಾಡುವ ಮಾತ್ರದಿಂದ 'ವಿಶ್ವಜಿತ್' ಎಂಬ ಯಜ್ಞದ ಫಲವನ್ನು ಹೊಂದುತ್ತಾನೆ. ಮಾಘ ಬಹುಳ ಅಮಾವಾಸ್ಯೆ ದಿನ ಗೋಕರ್ಣ ಕ್ಷೇತ್ರದಲ್ಲಿ ಸಮುದ್ರ ಸ್ನಾನ ಮಾಡಿದವರಿಗೆ ಸಕಲ ಕರ್ಮಬಂಧನದಿಂದ ಬಿಡುಗಡೆಯಾಗುವುದಲ್ಲದೇ ಒಂದು ನೂರು ಯಜ್ಞದ ಫಲವು ಲಭಿಸುತ್ತದೆ. ಯಾರು ಭಾನುವಾರದಿಂದ ಕೂಡಿದ ಸಪ್ತಮಿಯಲ್ಲಿ ಕುರುಕ್ಷೇತ್ರದಲ್ಲಿ ಸ್ನಾನ ಮಾಡಿ ಶ‍್ರೀಹರಿಯನ್ನು ಪೂಜಿಸುತ್ತಾರೆಯೋ ಅವರು ಕುಷ್ಠ ರೋಗದಿಂದ ಮುಕ್ತಿ ಪಡೆಯುತ್ತಾರೆ. ಯಾರು ಭಾನುವಾರ ಯುಕ್ತವಾದ ರಥಸಪ್ತಮಿಯಂದು ಮಕರರಾಶಿಯಲ್ಲಿ ಸೂರ್ಯನಿರುವಾಗ ಪುಷ್ಕರ ಕ್ಷೇತ್ರದಲ್ಲಿ ಸ್ನಾನ ಮಾಡುತ್ತಾರೆಯೋ, ಮತ್ತು ಪಿತೃಗಳಿಗೋಸ್ಕರ ತರ್ಪಣ ಕೊಡುತ್ತಾರೆಯೋ ಅವರು ಪಿತೃಋಣದಿಂದ ಬಿಡುಗಡೆ ಹೊಂದುತ್ತಾರೆ.

\begin{verse}
\textbf{ಸ್ನಾತ್ವಾಂಗಾರಚತುರ್ದಶ್ಯಾಂ ಮಾಘೇ ಮಾಸಿ ಸರೋವರೇ~।}\\\textbf{ಶಾಲಗ್ರಾಮಶಿಲಾಂ ದತ್ವಾ ಮುಚ್ಯತೇ ಚ ಋಣತ್ರಯಾತ್~।। ೨೧~।। }
\end{verse}

\begin{verse}
\textbf{ಮಾಘೇ ಮಾಧವಮಭ್ಯರ್ಚ್ಯ ಸ್ನಾತ್ವಾ ಮಾಘೇ ಸರೋವರೇ~।}\\\textbf{ಸವತ್ಸಾಂ ಗಾಂ ತಥಾ ದತ್ವಾ ಗೋಸಹಸ್ರಫಲಂ ಭವೇತ್~।। ೨೨~।।} 
\end{verse}

\begin{verse}
\textbf{ಯೇ ವಾಭ್ರ ಸರಸಿ ಸ್ನಾತ್ವಾ ಮಾಘೇ ಭೌಮಾಷ್ಟಮೀದಿನೇ~।}\\\textbf{ನ ಭೂಮೌ ಜಾಯತೇ ಕ್ವಾಪಿ ವಿಭ್ರಮಸ್ಯ ಪ್ರಮಾದತಃ~।। ೨೩~।।}
\end{verse}

ಮಾಘಮಾಸ ಚತುರ್ದಶೀ ಮಂಗಳವಾರದಲ್ಲಿ ಸರೋವರದಲ್ಲಿ ಸ್ನಾನಮಾಡಿ ಶಾಲಗ್ರಾಮವನ್ನು ದಾನಮಾಡಿದರೆ ಮೂರು ವಿಧವಾದ ಋಣಗಳಿಂದಲೂ (ದೇವ ಋಣ, ಋಷಿ ಋಣ, ಪಿತೃ ಋಣ) ಮುಕ್ತನಾಗುತ್ತಾನೆ. ಮಾಘಮಾಸದಲ್ಲಿ ಸರೋವರದಲ್ಲಿ ಸ್ನಾನಮಾಡಿ ಶ‍್ರೀಹರಿಯನ್ನು ಪೂಜಿಸಿ, ಕರುಸಹಿತವಾದ ಗೋವನ್ನು ದಾನ ಕೊಟ್ಟರೆ ಸಹಸ್ರ ಗೋದಾನ ಮಾಡಿದ ಫಲವು ಬರುತ್ತದೆ. ಮಾಘಮಾಸ ಅಷ್ಟಮಾ ಮಂಗಳವಾರದಲ್ಲಿ ಅಭ್ರಸರೋವರದಲ್ಲಿ ಸ್ನಾನ ಮಾಡುವುದರಿಂದ ಪ್ರವಾದದಿಂದಲೂ ಪುನರ್ಜನ್ಮವಿರುವುದಿಲ್ಲ.

\begin{verse}
\textbf{ಮಾಘಶುಕ್ಲ ಚತುರ್ದಶ್ಯಾಂ ಪರ್ವತೇ ಶ್ವೇತನಾಯಕೇ~।}\\\textbf{ಸ್ನಾತ್ವಾ ಚ ಶ್ವೇತಸರಸಿ ಶ್ವೇತದ್ವೀಪೇ ವಸತ್ಯ ಸೌ~।। ೨೪~।। }
\end{verse}

\begin{verse}
\textbf{ಮಾಘೇ ಹಸ್ತಿಗಿ‌ರೌ ಯಸ್ತು ಸ್ನಾತ್ವಾಽನಂತಸರೋವರೇ~।}\\\textbf{ದತ್ವಾ ಚ ಶೇಷಪ್ರತಿಮಾಂ ಶ್ವಿತ್ರರೋಗಾದ್ವಿಮುಚ್ಯತೇ~।। ೨೫~।।}
\end{verse}

\begin{verse}
\textbf{ಮೂಕಾಂಬಿಕೇ ಗಿರೌ‌ ಯತ್ರ ಸ್ನಾತ್ವಾ ಕಾಮಸರೋವರೇ~।}\\\textbf{ಬಾಲದುರ್ಗಾಂ ತಥಾ ಪೂಜ್ಯ ಸರ್ವಸಿದ್ಧಿಮವಾಪ್ನುಯಾತ್~।। ೨೬~।।}
\end{verse}

ಮಾಘ ಶುದ್ಧ ಚತುರ್ದಶಿಯಲ್ಲಿ ಶ್ವೇತನಾಯಕ ಪರ್ವತದಲ್ಲಿ ಶ್ವೇತ ಸರೋವರದಲ್ಲಿ ಸ್ನಾನ ಮಾಡುವವನು ಶ್ವೇತದ್ವೀಪ (ವೈಕುಂಠದ ಒಂದು ಭಾಗ) ದಲ್ಲಿ ವಾಸ ಮಾಡುತ್ತಾನೆ. ಮಾಘದಲ್ಲಿ ಯಾರು ಅಯೋಧ್ಯಾ, ಮಥುರಾ, ಕಾಶೀ, ಕಂಚೀ, ಅವಂತಿ, ಪೂರಿ, ದ್ವಾರಾವತಿ-ಇವುಗಳಲ್ಲಿ ಯಾವುದಾದರೊಂದು ಸ್ಥಳದಲ್ಲಿ ಅಥವಾ ಅನಂತ ಸರೋವರದಲ್ಲಿ ಸ್ನಾನಮಾಡಿ ಶೇಷದೇವರ ಪ್ರತಿಮೆಯನ್ನು ದಾನ ಕೊಟ್ಟರೆ ಅವರು ತೊನ್ನು ರೋಗದಿಂದ ಬಿಡುಗಡೆ ಹೊಂದುತ್ತಾರೆ. ಮೂಕಾಂಬಿಕೆ ದುರ್ಗದಲ್ಲಿ ಕಾಮ ಸರೋವರದಲ್ಲಿ ಸ್ನಾನಮಾಡಿ ಬಾಲದುರ್ಗೆಯನ್ನು ಪೂಜಿಸಿದರೆ ಸಕಲ ಇಷ್ಟಾರ್ಥಗಳೂ ಪೂರ್ಣವಾಗುತ್ತವೆ.

\begin{verse}
\textbf{ಮಾಘೇ ಮಘಾಸಹಾಯೇ ತು ಸ್ನಾತ್ವಾ ದುಗ್ಧನದೀಜಲೇ~।}\\\textbf{ಕ್ಷೀರಾಬ್ಧಿ ಶಾಯಿನಂ ಪೂಜ್ಯ ತೇನ ಸಾಯುಜ್ಯಮಾಪ್ನುಯಾತ್~।। ೨೭~।।}
\end{verse}

\begin{verse}
\textbf{ಮಾಘೇ ಪೈಠೀನಸೀಕ್ಷೇತ್ರೇ ಗೋದಾಯಾಂ ಸ್ನಾನಕೃನ್ನರಃ~।}\\\textbf{ಪಿಪ್ಪಲೇಶಪ್ರಸಾದೇನ ಸುಜ್ಞಾನೀ ಜಾಯತೇ ಭುವಿ~।। ೨೮~।। }
\end{verse}

\begin{verse}
\textbf{ಬ್ರಹ್ಮೇಶ್ವರೇ ತಥಾ ರೌದ್ರೇ ಕ್ಷೇತ್ರೇ ಗೋದಾನದೀತಟೇ~।}\\\textbf{ಸ್ನಾತ್ವಾ ಬ್ರಹ್ಮೇಶ್ವರಂ ಲಿಂಗಂ ದೃಷ್ಟ್ವಾ ಮೋಕ್ಷಮವಾಪ್ನುಯಾತ್~।।೨೯।।} 
\end{verse}

\begin{verse}
\textbf{ರಾಕಾಯಾಂ ಮಾಘಮಾಸೇ ತು ಶ‍್ರೀಕೃಷ್ಣೇ ಸಾಗರೇ ದ್ವಿಜ~।}\\\textbf{ಸ್ನಾತ್ವಾ ಶ್ವೇತವರಾಹಂ ತು ದೃಷ್ಟ್ವಾ ಮೋಕ್ಷಮವಾಪ್ನುಯಾತ್~।। ೩೦~।।}
\end{verse}

ಮಘಾ ನಕ್ಷತ್ರದಿಂದ ಯುಕ್ತವಾದ ಮಾಘಮಾಸದ ಹುಣ್ಣಿಮೆಯಲ್ಲಿ ದುಗ್ಧ ನದಿಯಲ್ಲಿ ಸ್ನಾನಮಾಡಿ ಕ್ಷೀರಾಬ್ಧಿಶಾಯಿಯಾದ ಶ‍್ರೀಹರಿಯನ್ನು ಪೂಜಿಸುವುದರಿಂದ ಸಾಯುಜ್ಯ ಮೋಕ್ಷವು ಲಭಿಸುತ್ತದೆ. ಮಾಘದಲ್ಲಿ ಪೈಠೀನಸೀ ಕ್ಷೇತ್ರದಲ್ಲಿ ಗೋದಾವರಿ ನದಿಯಲ್ಲಿ ಸ್ನಾನಮಾಡುವವನು ಪಿಪ್ಪಲೇಶನ ಅನುಗ್ರಹದಿಂದ ಒಳ್ಳೆಯ ಜ್ಞಾನಿಯಾಗಿ ಹುಟ್ಟುತ್ತಾನೆ. ಬ್ರಹ್ಮೇಶ್ವರ ಹಾಗೂ ರೌದ್ರ ಕ್ಷೇತ್ರದಲ್ಲಿ ಗೋದಾವರಿಯಲ್ಲಿ ಸ್ನಾನಮಾಡಿ ಬ್ರಹ್ಮೇಶ್ವರಲಿಂಗವನ್ನು ದರ್ಶನಮಾಡುವುದರಿಂದ ಮೋಕ್ಷವು ದೊರೆಯುವುದು. ಮಾಘ ಶುದ್ಧ ಹುಣ್ಣಿಮೆಯಂದು ಕೃಷ್ಣ ಸಾಗರದಲ್ಲಿ ಸ್ನಾನಮಾಡಿ ಶ್ವೇತವರಾಹ ದೇವರನ್ನು ದರ್ಶನಮಾಡುವುದರಿಂದ ಮೋಕ್ಷವು ಲಭಿಸುವುದು.

\begin{verse}
\textbf{ಸ್ನಾನಾಶಕ್ತೋ ಮಾಘಮಾಸಿ ಪರಿಧಾಯ ಚ ವಾಸಸೀ~।}\\\textbf{ಪ್ರಯಾಗಸ್ಮರಣಂ ಕೃತ್ವಾ ಮಾಘಸ್ನಾನಫಲಂ ಲಭೇತ್~।। ೩೧~।। }
\end{verse}

\begin{verse}
\textbf{ಪ್ರಯಾಗದರ್ಶನಾನ್ಮು ಕ್ತಿರ್ನ ಜಾನೇ ಸ್ನಾನಜಂ ಫಲಮ್~।}\\\textbf{ತತ್ರಾಪಿ ಮಾಘಮಾಸಶ್ಚೇತ್ ಕೋ ವಾ ವರ್ಣಯಿತುಂ ಕ್ಷಮಃ~।। ೩೨~।। }
\end{verse}

\begin{verse}
\textbf{ಪ್ರಯಾಗೇ ಮಾಧವಂ ದೃಷ್ಟ್ವಾ ಮಾಘೇ ಸ್ನಾತ್ವಾ ಪಿತೄನಥ~।}\\\textbf{ಉದ್ದಿಶ್ಯ ಕೃತ್ವಾ ಶ್ರಾದ್ಧಂ ಚ ಕೃತಕೃತ್ಯೋ ಭವೇನ್ನರಃ~।। ೩೩~।।}
\end{verse}

ಮಾಘಮಾಸದಲ್ಲಿ ಸ್ನಾನಮಾಡಲು ಅಶಕ್ತನಾದವನು ಪ್ರಾತಃಕಾಲದಲ್ಲಿ ಶುಭ್ರವಾದ ವಸ್ತ್ರವನ್ನುಟ್ಟು, ಪ್ರಯಾಗಸ್ಮರಣಮಾತ್ರದಿಂದ ಮಾಘಸ್ನಾನದ ಫಲವನ್ನು ಪಡೆಯುವನು. ಪ್ರಯಾಗದರ್ಶನಮಾತ್ರದಿಂದಲೇ ಮುಕ್ತಿಯು ಲಭಿಸುವುದು; ಅಲ್ಲಿ ಸ್ನಾನಮಾಡಿದರೆ ಎಷ್ಟು ಫಲವೆಂಬುದನ್ನು ನಾನು ಅರಿಯೆ; ಅದರಲ್ಲಿಯೂ ಆಗ ಮಾಘಮಾಸವಾಗಿದ್ದರೆ ಅದರ ಫಲವನ್ನು ವರ್ಣಿಸಲು ಯಾರು ತಾನೇ ಸಮರ್ಥರು! ಮಾಘಮಾಸದಲ್ಲಿ ಪ್ರಯಾಗದಲ್ಲಿ ಸ್ನಾನಮಾಡಿ ಮಾಧವನ ದರ್ಶನಮಾಡಿ ಪಿತೃಗಳಿಗೋಸ್ಕರ ಶ್ರಾದ್ಧವನ್ನು ಆಚರಿಸಿದಲ್ಲಿ ಕೃತಾರ್ಥನಾಗುವನು.

\begin{verse}
\textbf{ವ್ಯತೀಪಾತೇ ರವೌ ಮಾಘೇ ಬ್ರಾಹ್ಮಣಾನ್ ಭೋಜಯೇದ್ಯದಿ~।}\\\textbf{ಪಾಯಸಂ ಘೃತಯುಕ್ತೇನ ವಂಧ್ಯಾ ಪುತ್ರತ್ವಮಾಪ್ನುಯಾತ್~।। ೩೪~।।}
\end{verse}

ಮಾಘಮಾಸದಲ್ಲಿ ಭಾನುವಾರ ವ್ಯತೀಪಾತವಿರುವಾಗ ತುಪ್ಪ\enginline{-}ಪಾಯಸದಿಂದ ಯುಕ್ತವಾದ ಭೋಜನವನ್ನು ಬ್ರಾಹ್ಮಣರಿಗೆ ಮಾಡಿಸುವುದರಿಂದ ಬಂಜೆಯು ಮಗನನ್ನು ಪಡೆಯುವಳು.

\begin{verse}
\textbf{ಮಾಘಮಾಸೇ ಭಾನುವಾರೇ ರವಿಗ್ರಹಶತಾಧಿಕೇ~।}\\\textbf{ಮಾಘೇ ಮಕರಗೇ ಭಾನೌ ಭಾನುವಾರೊ ವಿಶಿಷ್ಯತೇ~।। ೩೫~।।}
\end{verse}

ಮಾಘಮಾಸ ಭಾನುವಾರ ಮಕರಸಂಕ್ರಮಣವು ಬಂದರೆ ಅಂತಹ ಪರ್ವ ಕಾಲವು ನೂರು ಗ್ರಹಣಗಳಿಗಿಂತಲೂ ಶ್ರೇಷ್ಠ.

\begin{verse}
\textbf{ಭಾನೌ ಸೂರ್ಯಗತಂ ವಿಷ್ಣುಂ ಪೂಜ್ಯ ಕಾಮಾನವಾಪ್ನುಯಾತ್~।}\\\textbf{ರವಿವಾರೇ ಕೃತಸ್ನಾನಂ ರಾತ್ರೌ ಭೋಜನವರ್ಜಿತಮ್~।। ೩೬~।।} 
\end{verse}

\begin{verse}
\textbf{ಪಾಯಸಂ ಭೋಜನಂ ದತ್ವಾ ಬ್ರಹ್ಮಹತ್ಯಾ ವ್ಯಪೋಹತಿ~।}\\\textbf{ಯಃ ಶುಕ್ರವಾರೇ ಚಾಷ್ಟಮ್ಯಾಂ ಸ್ನಾತ್ವಾ ಮಾಧವಪೂಜನಮ್~।। ೩೭~।। }
\end{verse}

\begin{verse}
\textbf{ಮಲ್ಲಿಕಾಕುಸುಮೈಃ ಕೃತ್ವಾ ಲಕ್ಷ್ಮೀಸ್ಥೈರ್ಯಮವಾಪ್ನುಯಾತ್~।}\\\textbf{ಯೊ ಮಾಘೇ ಕುಜಪಂಚಮ್ಯಾಂ ಮಾಧವಂ ಮಾಲತೀಸ್ರಜಾ~।। ೩೮~।।} 
\end{verse}

\begin{verse}
\textbf{ಸಂಪೂಜ್ಯಾಢಕದಾನಾನಿ ಕೃತ್ವಾ ಮೃತ್ಯುಂಜಯೇನ್ನರಃ~।}\\\textbf{ಅತ್ರೈವೋದಾಹರಂತೀಮಮಿತಿಹಾಸಂ ಪುರಾತನಮ್~।। ೩೯~।।}
\end{verse}

ಭಾನುವಾರದಲ್ಲಿ ಸೂರ್ಯಸಂಕ್ರಮಣವಿರುವಾಗ ಶ‍್ರೀ ವಿಷ್ಣುವನ್ನು ಪೂಜಿಸುವುದರಿಂದ ಸಕಲ ಮನೋಭೀಷ್ಟಗಳನ್ನೂ ಹೊಂದುವನು; ಭಾನುವಾರ ಮಾಘ ಸ್ನಾನಮಾಡಿ ರಾತ್ರಿ ಭೋಜನ ಮಾಡದೇ ಬ್ರಾಹ್ಮಣರಿಗೆ ಪಾಯಸ ಸಹಿತ ಭೋಜನ ಮಾಡಿಸಿದರೆ ಬ್ರಹ್ಮಹತ್ಯಾ ದೋಷವು ಪರಿಹಾರವಾಗುತ್ತದೆ. ಯಾವನು ಶುಕ್ರವಾರ ಅಷ್ಟಮಿಯಂದು ಸ್ನಾನಮಾಡಿ ಶ‍್ರೀಹರಿಯನ್ನು ಮಲ್ಲಿಗೆ ಹೂಗಳಿಂದ ಪೂಜಿಸುತ್ತಾನೆಯೋ ಅವನು ನಿರಂತರವಾಗಿರುವ ಐಶ್ವರ್ಯವನ್ನು ಪಡೆಯುತ್ತಾನೆ. ಯಾವನು ಮಂಗಳವಾರ ಸಹಿತವಾದ ಪಂಚಮಿಯಲ್ಲಿ ಮಾಘಮಾಸದಲ್ಲಿ ಸ್ನಾನ ಮಾಡಿ ಮಾಧವನನ್ನು ಮಾಲತೀ ಪುಷ್ಪಗಳಿಂದ ಅರ್ಚಿಸಿ ಗೋಪೀಚಂದನ, ಬೇಳೆಗಳು ಮುಂತಾದ ವಸ್ತುವನ್ನು ದಾನಮಾಡುವನೋ ಅವನು ಅಪಮೃತ್ಯುವಿನಿಂದ ಪಾರಾಗುತ್ತಾನೆ. ಈ ವಿಷಯದಲ್ಲಿ ಉದಾಹರಣೆಗೆ ಪುರಾತನ ಕಥೆಯೊಂದಿದೆ.

\begin{verse}
\textbf{ಕಾಲಮೃತ್ಯುಹರಂ ಸೌಮ್ಯಂ ಮಹಾಪಾತಕನಾಶನಮ್~।}\\\textbf{ಪುರಾ ಸೌವೀರದೇಶೇಶೋ ನಾಮ್ನಾ ವಸುಮತಿರ್ಬಲೀ~।। ೪೦~।। }
\end{verse}

\begin{verse}
\textbf{ತಸ್ಯಾಸೀತ್ತನಯೋ ನಾಮ ವಿಚಿತ್ರರಥ ಇತ್ಯಥ~।}\\\textbf{ಶಸ್ತ್ರವಿದ್ಯಾಸು ನಿಪುಣಃ ಕ್ರುದ್ಧ ಸೇನಾಬಲಾನ್ವಿತಃ~।। ೪೧~।।}
\end{verse}

\begin{verse}
\textbf{ಬಾಲಾರುಣಸಮಪ್ರಖ್ಯೋ ದ್ವಿತೀಯ ಇವ ಭಾಸ್ಕರಃ~।}\\\textbf{ತಪ್ಪಿ ತಾ ತು ದಿಶೋ ಜಿತ್ವಾ ಕೃತ್ವಾ ವಿಶ್ವಜಿತಂ ಮುಖಮ್~।। ೪೨~।।}
\end{verse}

ಯಮದೇವರಿಂದ ಬರುವ ಅಕಾಲಮೃತ್ಯುವನ್ನು ಪರಿಹರಿಸುವ, ಶುಭಕರವಾದ, ಸಕಲ ಮಹಾಪಾತಕಗಳನ್ನೂ ನಾಶಮಾಡುವ ಕಥೆ. ಪೂರ್ವದಲ್ಲಿ ವಸುಮತಿ ಎಂಬ ಬಲಾಢ್ಯನು ಸೌವೀರ ದೇಶಕ್ಕೆ ರಾಜನಾಗಿದ್ದನು. ಅವನಿಗೆ ಸರ್ವ ಶಾಸ್ತ್ರಗಳಲ್ಲಿಯೂ ನಿಪುಣನಾದ, ಒಳ್ಳೆಯ ಸೈನ್ಯಬಲದಿಂದ ಕೂಡಿದ ವಿಚಿತ್ರರಥನೆಂಬ ಮಗನಿದ್ದನು. ಆಗತಾನೆ ಉದಯನಾದ ಸೂರ್ಯನಂತೆ ತೇಜಸ್ವಿಯಾದ, ಎರಡನೇ ಸೂರ್ಯನಂತೆ ಪ್ರಕಾಶಮಾನನಾದ ಈ ವಿಚಿತ್ರರಥನ ತಂದೆಯಾದ ವಸುಮತಿಯು ಎಲ್ಲ ದಿಕ್ಕುಗಳ ರಾಜ್ಯಗಳನ್ನೂ ಗೆದ್ದು 'ವಿಶ್ವಜಿತು' ಎಂಬ ಯಜ್ಞ ಮಾಡಿದನು.

\begin{verse}
\textbf{ಸರ್ವಸ್ವದಕ್ಷಿಣಾಂ ದತ್ವಾ ಸೋಽಭೂತ್ ಪೂರ್ಣಮನೋರಥಃ~।}\\\textbf{ಏತಸ್ಮಿನ್ನೇವ ಕಾಲೇ ತು ಮುನಿರ್ಗೃತ್ಸಮದಾಹ್ವಯಃ~।। ೪೩~।। }
\end{verse}

\begin{verse}
\textbf{ಆಜಗಾಮ ಸ ಭಿಕ್ಷಾಯೈ ಯಜ್ಞ ಸ್ಯಾಧ್ವರದೀಕ್ಷಿತಃ~।}\\\textbf{ಮುನಿಂ ಚ ಸುಮತಿರ್ದೃಷ್ಟ್ವಾ ಪ್ರತ್ಯುತ್ಥಾಯಾರ್ಘ್ಯಪಾದಕೈಃ~।। ೪೪~।। }
\end{verse}

\begin{verse}
\textbf{ಪಪ್ರಚ್ಛ ಕುಶಲಂ ಪೂಜ್ಯ ಸ್ವಸ್ತಿ ತೇ ತಪಸೋ ಮುನೇ~।}\\\textbf{ಅದ್ಯ ಮೇ ಸಫಲಂ ಜನ್ಮ ಪಾವಿತಂ ಸಕಲಂ ಕುಲಮ್~।। ೪೫~।।}
\end{verse}

ಯಜ್ಞ ಕಾಲದಲ್ಲಿ ಸರ್ವರಿಗೂ ದಕ್ಷಿಣೆ ಕೊಟ್ಟು ರಾಜನು ಕೃತಾರ್ಥನಾದನು. ಅದೇ ಕಾಲದಲ್ಲಿ ಗೃತ್ಸಮದರೆಂಬ ಋಷಿಗಳು ಭಿಕ್ಷೆಗೋಸ್ಕರ ಅಲ್ಲಿಗೆ ಬಂದರು. ಯಜ್ಞದೀಕ್ಷೆಯಲ್ಲಿದ್ದ ಸುಮತಿಯು (ವಸುಮತಿಯು) ಎದ್ದು, ಋಷಿಗಳಿಗೆ ಅರ್ಘ್ಯ, ಪಾದ್ಯಾದಿಗಳಿಂದ ಸತ್ಕರಿಸಿ ಅವರ ಕುಶಲದ ಬಗ್ಗೆ ಪ್ರಶ್ನೆ ಮಾಡಿದನು. ಈ ದಿನ ತಮ್ಮ ಆಗಮನದಿಂದ ನನ್ನ ಜನ್ಮ ಸಫಲವಾಯಿತು; ನನ್ನ ಕುಲವು ಪವಿತ್ರವಾಯಿತು.

\begin{verse}
\textbf{ಯತ್ತವಾಂಘ್ರಿಯುಗಾಂಭೋಜೇ ದೈವಾದೃಷ್ಟಿಪಥಂ ಗತೇ~।}\\\textbf{ಅದ್ಯ ಮೇ ಪಿತರಸ್ತುಷ್ಟಾ ಸುಕೃತಂ ಫಲಿತಂ ಮಮ~।। ೪೬~।। }
\end{verse}

\begin{verse}
\textbf{ಭವದಾಗಮನಂ ಮಹ್ಯಮುಪಶಿಕ್ಷ್ಯಾನುಗೃಹ್ಯತಾಮ್~।}\\\textbf{ಇತಿ ತಸ್ಯ ವಚಃ ಸೃಷ್ಟ್ವಾ ಪ್ರಾಹಿಣೋದಾಶ್ರಮಾಯ ಚ~।। ೪೭~।।}
\end{verse}

ದೈವಯೋಗದಿಂದ ನಿಮ್ಮ ಪಾದಪದ್ಮಗಳು ನನ್ನ ದೃಷ್ಟಿಗೆ ಬಿದ್ದವು. ಈ ದಿನ ನನ್ನ ಪಿತೃಗಳು ಸಂತುಷ್ಟರಾದರು. ನನ್ನ ಹಿಂದಿನ ಪುಣ್ಯ ಫಲಕಾರಿಯಾಯಿತು. ನನ್ನ ಮೇಲೆ ಕೃಪೆಮಾಡಿ ಉಪದೇಶವನ್ನು ಮಾಡಿ ಎಂದು ಪ್ರಾರ್ಥಿಸಿದನು. ಅರಸನ ಈ ಮಾತುಗಳನ್ನು ಕೇಳಿ ಋಷಿಗಳು ಅರಸನ ಆಶ್ರಮಕ್ಕೆ ಹೋದರು,

\begin{verse}
\textbf{ಮುನೀಂದ್ರಂ ಭೋಜಯಾಮಾಸ ಶಾಂತಂ ಗೃತ್ಸಮದಂ ಮುದಾ~।}\\\textbf{ಭುಕ್ತೋಪವಿಷ್ಟೇ ತುಷ್ಟೇ ಚ ಪುನಃ ಪಪ್ರಚ್ಛ ತಂ ಮುನಿಮ್~।। ೪೮~।।}
\end{verse}

ರಾಜನು ಶಾಂತರಾದ ಗೃತ್ಸಮದ ಋಷಿಗಳಿಗೆ ಭೋಜನಮಾಡಿಸಿದನು. ಭೋಜನದಿಂದ ತೃಪ್ತರಾದ ಋಷಿಗಳನ್ನು ಪ್ರಶ್ನೆ ಮಾಡಿದನು:

\begin{verse}
\textbf{ಸುತಸ್ನೇಹಾಜ್ಜಾತಕಂ ತು ಪೃಷ್ಟ್ವಾಽಯುಷ್ಯಸ್ಯ ಕಾಂಕ್ಷಯಾ~।}\\\textbf{ಇತಿ ಪೃಷ್ಟಸ್ತದಾ ರಾಜ್ಞಾ ಕಿಂಚಿತ್ ಧ್ಯಾತ್ವಾ ಜಗಾದ ಹ~।। ೪೯~।।}
\end{verse}

ಮಗನಮೇಲಿನ ವಾತ್ಸಲ್ಯದಿಂದಲೂ, ಅವನ ಆಯುಷ್ಯವನ್ನು ತಿಳಿದುಕೊಳ್ಳಬೇಕೆಂಬ ಇಚ್ಛೆಯಿಂದಲೂ ಮಗನ ಜಾತಕದ ಬಗ್ಗೆ ಪ್ರಶ್ನಿಸಲ್ಪಟ್ಟ ಋಷಿಗಳು ಸ್ವಲ್ಪ ಹೊತ್ತು ಧ್ಯಾನಮಾಡಿ ಹೇಳಿದರು:

\begin{verse}
\textbf{ಅಯಂ ಪುತ್ರಃ ಪುರಾಕಾಲೇ ಸುಶರ್ಮಾ ನಾಮ ವೈ ದ್ವಿಜಃ~।}\\\textbf{ಸುಸಮಂತಾತ್ಮಜಃ ಪ್ರಾಜ್ಞಃ ಸ್ತ್ರೀಜಿತಃ ಸ್ತ್ರೀವಶೈರ್ಜಿತಃ~।। ೫೦~।।}
\end{verse}

ರಾಜನೇ, ಈಗ ನಿನ್ನ ಮಗನಾಗಿರುವ ವಿಚಿತ್ರರಥನು ಪೂರ್ವದಲ್ಲಿ 'ಸುಶರ್ಮಾ' ಎಂಬ ಹೆಸರಿನ ಬ್ರಾಹ್ಮಣನಾಗಿದ್ದನು; ಬುದ್ದಿವಂತನೂ, ಸ್ತ್ರೀಯರನ್ನು ಗೆದ್ದವನೂ ಆಗಿದ್ದನು. ಅವನ ತಂದೆಯ ಹೆಸರು ಸುಸಮಂತ.

\begin{verse}
\textbf{ಪಿತಾ ಸನ್ಯಸ್ಯಮಾನಸ್ತು ಭಾರ್ಯಾಂ ಪುತ್ರೇ ನಿಧಾಯ ಚ~।}\\\textbf{ತಾಮ್ರಪರ್ಣೀಂ ನದೀಂ ಪ್ರಾಪ್ಯ ಮೋಕ್ಷಮಾರ್ಗಮುಪೇಯಿವಾನ್~।। ೫೧~।।}
\end{verse}

ಸುಸಮಂತನು ತನ್ನ ಪತ್ನಿಯನ್ನು ಮಗನ ವಶದಲ್ಲಿಟ್ಟು ತಾನು ತಾಮ್ರಪರ್ಣೀ ನದೀತೀರಕ್ಕೆ ಹೋಗಿ ಮೋಕ್ಷಮಾರ್ಗವನ್ನು ಪಡೆದನು.

\begin{verse}
\textbf{ತತಃ ಸ್ವಭಾರ್ಯಾವಚನಂ ಶ್ರುತ್ವಾ ಮಾತುಃ ಕರೋತಿ ಚ~।}\\\textbf{ನಿರೋಧಂ ತರ್ಜನಾದ್ಯೈಶ್ಚ ಕೃತ್ವಾ ನಿಂದತಿ ದುಃಖಿತಾಮ್~।। ೫೨~।। }
\end{verse}

\begin{verse}
\textbf{ನೈವ ಪೃಚ್ಛತಿ ಚಾಹಾರಂ ದುಃಖಾನ್ಮೃತಿಮವಾಪ ಸಾ~।}\\\textbf{ದೋಷಾತ್ ಷಷ್ಟಿ ಸಮಾವಿಷ್ಟೋ ಹ್ಯಲ್ಪಾಯುಷ್ಯಮವಾಪ ಚ~।। ೫೩~।।}
\end{verse}

ಸುಶರ್ಮನು ತನ್ನ ಪತ್ನಿಯ ಮಾತುಕೇಳುತ್ತಾ ತನ್ನ ತಾಯಿಗೆ ವಿರೋಧ ಮಾಡಿ ಕೆಟ್ಟ ಮಾತುಗಳಿಂದ ಬಯ್ಯುತ್ತಿದ್ದನು. ಅವಳಿಗೆ ಕಾಲಕಾಲಕ್ಕೆ ಆಹಾರ ಕೊಡುತ್ತಿರಲಿಲ್ಲ. ಈ ದುಃಖವನ್ನು ಸಹಿಸಿ ಸಾಕಾಗಿ ಅವನ ತಾಯಿಯು ಮೃತಳಾದಳು. ಈ ದೋಷದ ದೆಸೆಯಿಂದ ಸುಶರ್ಮನ ಆಯಸ್ಸು ಕಡಿಮೆಯಾಗಿ ಅರವತ್ತು ವರ್ಷ ಮಾತ್ರ ಬದುಕಿದ್ದನು.

\begin{verse}
\textbf{ಚತ್ವಾರಿಂಶತ್ಸಮಾಯುಷ್ಯಮಾವಿಶಿಷ್ಟಮಭೂತ್ತದಾ~।}\\\textbf{ತಥಾ ಪಿ ಮಾತುರ್ದ್ರೋಹೇಣ ಮೃತೋಽಸಾವಪಮೃತ್ಯು ನಾ~।। ೫೪~।।}
\end{verse}

ಅವನಿಗೆ ಇನ್ನೂ ನಲವತ್ತು ವರ್ಷಗಳ ಆಯಸ್ಸು ಇದ್ದಾಗ್ಯೂ ತಾಯಿಗೆ ಮಾಡಿದ ದ್ರೋಹದ ಕಾರಣದಿಂದ ಅಪಮೃತ್ಯುವಿಗೆ ಈಡಾದನು.

\begin{verse}
\textbf{ಇದಾನೀಂ ತ್ವತ್ಸುತೊ ಭೂತ್ವಾ ನೀತಮಾಯುಷ್ಯಶೇಷಿತಮ್~।}\\\textbf{ಇತೋ ಯಾಮತ್ರಯಾದೂರ್ಧ್ವಂ ಮರಿಷ್ಯತಿ ನ ಸಂಶಯಃ~।। ೫೫~।।}
\end{verse}

ಈಗ ಅವನು ನಿನ್ನ ಮಗನಾಗಿ ಹುಟ್ಟಿ, ಉಳಿದ ನಲವತ್ತು ವರ್ಷ ಆಯುಸ್ಸನ್ನು ಕಳೆದಿದ್ದಾನೆ. ಇನ್ನು ಮೂರು ಯಾಮು ಮುಗಿಯುವಷ್ಟರಲ್ಲಿ ಸಂಶಯವಿಲ್ಲದೇ ಮರಣಹೊಂದುವನು.

\begin{verse}
\textbf{ಇತಿ ತದ್ವಚನಂ ಶ್ರುತ್ವಾ ನೃಪಃ ಶೋಕಪರಾಯಣಃ~।}\\\textbf{ಪಾದೌ ಪ್ರಗೃಹ್ಯ ಬ್ರಹ್ಮರ್ಷೇರ್ವಹುಧಾ ಪ್ರಾರ್ಥ್ಯ ದುಃಖಿತಃ~।। ೫೬~।। }
\end{verse}

\begin{verse}
\textbf{ಯಯಾಚೇ ಸ್ವಸುತಸ್ಯಾಯುರಭಿವರ್ಧಯಿತುಂ ಪುನಃ~।}\\\textbf{ಇತಿ ಚಾಭ್ಯರ್ಥಿತೋ ರಾಜ್ಞಾ ಮುನಿಶ್ಚಿಂತಾಪರಾಯಣಃ~।। ೫೭~।।}
\end{verse}

ಋಷಿಗಳ ಈ ಮಾತನ್ನು ಕೇಳಿದ ರಾಜನು ಅತಿ ದುಃಖಿತನಾಗಿ ಅವರ ಪಾದಗಳನ್ನು ಹಿಡಿದುಕೊಂಡು ಬಹಳವಾಗಿ ಪ್ರಾರ್ಥಿಸಿದನು. ತನ್ನ ಮಗನ ಆಯುಸ್ಸನ್ನು ಹೆಚ್ಚಿಸಿಕೊಡಬೇಕೆಂದು ಬೇಡಿದನು. ಹೀಗೆ ಪ್ರಾರ್ಥಿಸಲ್ಪಟ್ಟ ಗೃತ್ಸಮದ ಮುನಿಗಳು ಚಿಂತಿಸಿದರು.

\begin{verse}
\textbf{ಸ್ಮೃತ್ವಾಶಿವೋಕ್ತಂ ಪಾರ್ವತ್ಯೈ ತಂ ಜಗಾದ ಕೃಪಾಕರಃ~।}\\\textbf{ಶೃಣು ತತ್ತ್ವಂ ಪ್ರವಕ್ಷ್ಯಾಮಿ ಶಿವೋಕ್ತಂ ಗಿರಿಜಾಕೃತೇ~।। ೫೮~।।}
\end{verse}

ಪಾರ್ವತೀದೇವಿಗೆ ರುದ್ರದೇವರು ಮಾಡಿದ ಉಪದೇಶವನ್ನು ಜ್ಞಾಪಿಸಿಕೊಂಡು ಋಷಿಗಳು ರಾಜನಿಗೆ ಹೇಳಿದರು: ರುದ್ರದೇವರು ಗಿರಿಜಾದೇವಿಯರಿಗೆ ಹೇಳಿದ ತತ್ತ್ವವನ್ನು ನಿನಗೆ ಹೇಳುತ್ತೇನೆ, ಕೇಳು.

\begin{verse}
\textbf{ಯೋ ಮಾಘೇ ಕುಜಪಂಚಮ್ಯಾಂ ಸ್ನಾತ್ವಾ ಮಾಧವಪೂಜನಮ್~।}\\\textbf{ಕೃತ್ಯಾಢಕಾನಿ ಯೊ ದದ್ಯಾದ್ಗುಡಂ ಖಂಡಯುತಾನಿ ಚ~।। ೫೯~।।} 
\end{verse}

\begin{verse}
\textbf{ಬಲೀವರ್ದೋದ್ವಾಹಿತಾನಿ ಮೃತ್ಯುಂ ಜಯತಿ ನಿಶ್ಚಯಮ್~।}\\\textbf{ರಾಜನ್ ಭೋ ತವ ಪುತ್ರಸ್ಯ ಯದಿ ಕಲ್ಯಾಣಮಿಚ್ಛಸಿ~।। ೬೦~।। }
\end{verse}

\begin{verse}
\textbf{ಏತದ್ದಾನಂ ತದಾ ದೇಹಿ ಮೃತ್ಯುಂ ಜಿತ್ವಾಯುರೇಷ್ಯತಿ~।}\\\textbf{ಮಾಘೇ ತು ಕುಜಪಂಚಮ್ಯಾಂ ಸುತಂ ಸ್ನಾಪ್ಯ ನದೀತಲೇ~।। ೬೧~।। }
\end{verse}

\begin{verse}
\textbf{ದಾನಂ ದಾಪಯ ಶೀಘ್ರಂ ತ್ವಂ ಮಾ ವಿಲಂಬಿತುಮರ್ಹಸಿ~।}
\end{verse}

ಯಾವನು ಮಂಗಳವಾರದಿಂದ ಕೂಡಿದ ಮಾಘಮಾಸದ ಪಂಚಮಿಯಲ್ಲಿ ಪ್ರಾತಃಕಾಲ ಸ್ನಾನಮಾಡಿ, ಶ‍್ರೀಹರಿಯನ್ನು ಅರ್ಚಿಸಿ, ವೃಷಭಸಹಿತವಾದ ಬೆಲ್ಲ, ತೊಗರಿಗಳನ್ನು ದಾನ ಕೊಡುತ್ತಾನೋ ಅವನು ನಿಶ್ಚಯವಾಗಿ ಅಪಮೃತ್ಯುವನ್ನು ಜಯಿಸುತ್ತಾನೆ. ರಾಜನೇ, ನೀನು ನಿನ್ನ ಮಗನ ಹಿತವನ್ನು ಬಯಸುವುದಾದರೆ ಈ ದಾನವನ್ನು ಕೊಡು; ನಿನ್ನ ಮಗನು ಅಪಮೃತ್ಯುವನ್ನು ಜಯಿಸಿ ದೀರ್ಘಾಯು ವಾಗುತ್ತಾನೆ. ಮಾಘಮಾಸದಲ್ಲಿ ಮಂಗಳವಾರದಿಂದ ಯುಕ್ತವಾದ ಪಂಚಮಿಯಲ್ಲಿ ನದಿಯಲ್ಲಿ ನಿನ್ನ ಮಗನು ಸ್ನಾನಮಾಡಿ ಈ ದಾನಗಳನ್ನು ಕೊಡಲಿ; ಶೀಘ್ರವಾಗಿ ಈ ಕೆಲಸ ಮಾಡು. ತಡಮಾಡಬೇಡ.

\begin{verse}
\textbf{ಇತಿ ತದ್ವಚನಂ ಶ್ರುತ್ವಾ ರಾಜಾ ಚರ್ಮಣ್ವತೀಜಲೇ~।। ೬೨~।।} 
\end{verse}

\begin{verse}
\textbf{ಸ್ನಾಪಯಿತ್ವಾ ವಿಧಾನೇನ ಪೂಜಯಿತ್ವಾ ಚ ಮಾಧವಮ್~।}\\\textbf{ಆಢ್ರಾಕ್ಯೋ ಗುಡಯುಕ್ತಾಶ್ಚ ವೃಷಭೋದ್ವಾಹಿತಾಸ್ತು ತಾಃ~।। ೬೩~।। }
\end{verse}

\begin{verse}
\textbf{ದಾಪಯಾಮಾಸ ವಿಪ್ರಾಯ ತೇನ ಮೃತ್ಯುರ್ವಿನಾಶಿತಃ~।}\\\textbf{ಷಟ್ ಷಷ್ಟಿ ವರ್ಷಮಾಷ್ಯಂ ವವೃಧೇ ಸ್ನಾನದಾನತಃ~।। ೬೪~।।}
\end{verse}

ಈ ಮಾತುಗಳನ್ನು ಕೇಳಿದ ರಾಜನು ಕೂಡಲೇ ತನ್ನ ಮಗನಿಗೆ ಚರ್ಮಣ್ವತೀ ನದಿಯಲ್ಲಿ ಸ್ನಾನಮಾಡಿಸಿ ವಿಧಿಪೂರ್ವಕವಾಗಿ ಶ‍್ರೀಹರಿಯನ್ನು ಪೂಜಿಸಿ ವೃಷಭ ಸಹಿತವಾದ ಬೆಲ್ಲ, ತೊಗರಿಗಳನ್ನು ಬ್ರಾಹ್ಮಣನಿಗೆ ದಾನ ಕೊಡಿಸಿದನು. ಅದರಿಂದ ಅವನ ಮಗನ ಅಪಮತ್ಯು ಪರಿಹಾರವಾಗಿ ಅರವತ್ತಾರು ವರ್ಷ ಆಯುಸ್ಸು ಅಭಿವೃದ್ಧಿಯಾಯಿತು.

\begin{verse}
\textbf{ತತೋ ಗೃತ್ಸಮದಃ ಪ್ರಾಗಾತ್ ಸಂಯಮಾನಃ ಸ್ವಮಾಶ್ರಮಮ್~।}\\\textbf{ತಸ್ಮಾದ್ವೈ ಕುಜಪಂಚಮ್ಯಾಂ ಮಾಘೇ ಸ್ನಾತ್ವಾ ಗುಡಾನ್ವಿತಮ್~।। ೬೫~।। }
\end{verse}

\begin{verse}
\textbf{ಆಢಕಂ ವೃಷಭಕ್ಷಿಪ್ತಂ ದಾತವ್ಯಂ ಹಿತಮಿಚ್ಛತಾ~।}\\\textbf{ನಿಷ್ಕಾಮಂ ವಾ ಸಕಾಮಂ ವಾ ಮಾಘಂ ನಿರ್ವೃತ್ಯ ಮಾನವಃ~।। ೬೬~।।} 
\end{verse}

\begin{verse}
\textbf{ಮಾಸಾತ್ ಪರಾತ್ ಕೋಟಿಪುಣ್ಯಂ ಫಲತ್ವೇವ ನ ಸಂಶಯಃ~।}\\\textbf{ಅವಶ್ಯಂ ಕಾಮುಕಾಃ ಕೃತ್ವಾ ಕಾಮಂ ವ್ರತಮನುತ್ತಮಮ್~।। ೬೭~।।} 
\end{verse}

\begin{verse}
\textbf{ಮಾಘೇ ಮಾಸಿ ಮುನಿಶ್ರೇಷ್ಠ ಫಲಂ ಪ್ರಾಪ್ನೋತಿ ನಿಶ್ಚಿತಮ್~।। ೬೮~।।}
\end{verse}

ನಂತರ ಜಿತೇಂದ್ರಿಯರಾದ ಗೃತ್ಸಮದ ಋಷಿಗಳು ತಮ್ಮ ಆಶ್ರಮಕ್ಕೆ ತೆರಳಿದರು. ನಾರದನೇ, ಮಂಗಳವಾರದಿಂದ ಕೂಡಿದ ಮಾಘಮಾಸದ ಪಂಚಮಿಯಲ್ಲಿ ಸ್ನಾನಮಾಡಿ ಬೆಲ್ಲ, ತೊಗರಿ, ವೃಷಭಗಳನ್ನು ಹಿತವನ್ನು ಬಯಸುವವರು ದಾನ ಕೊಡಬೇಕು. ದಾನವು ನಿಷ್ಕಾಮ ಅಥವ ಸಕಾಮವಾಗಿರಬಹುದು. ಬೇರೆ ಮಾಸಗಳಲ್ಲಿ ಇವುಗಳನ್ನು ದಾನಮಾಡಿದರೆ ಬರುವ ಫಲಕ್ಕಿಂತ ಮಾಘಮಾಸದಲ್ಲಿ ಕೋಟಿಯಷ್ಟು ಹೆಚ್ಚು ಪುಣ್ಯ ದೊರೆಯುತ್ತದೆ. ಐಹಿಕ ಇಷ್ಟಾರ್ಥಸಿದ್ದಿಯನ್ನು ಇಚ್ಚಿಸುವವರು ಉತ್ತಮವಾದ ಈ ವ್ರತವನ್ನು ಮಾಘಮಾಸದಲ್ಲಿ ಮಾಡಿ ಸಕಲ ಫಲಗಳನ್ನೂ ನಿಶ್ಚಯವಾಗಿ ಪಡೆಯಬಹುದು.

\begin{center}
ಇತಿ ಶ‍್ರೀ ವಾಯುಪುರಾಣೇ ಮಾಘಮಾಸಮಾಹಾತ್ಮ್ಯೇ ಸಪ್ತಮೋsಧ್ಯಾಯಃ 
\end{center}

\begin{center}
ಶ‍್ರೀ ವಾಯುಪುರಾಣಾಂತರ್ಗತ ಮಾಘಮಾಸ ಮಾಹಾತ್ಮ್ಯೆಯಲ್ಲಿ \\ ಏಳನೇ ಅಧ್ಯಾಯವು ಸಮಾಪ್ತಿಯಾಯಿತು.
\end{center}

\newpage

\section*{ಅಧ್ಯಾಯ\enginline{-}೮}

\emptypage

\begin{flushleft}
\textbf{ಬ್ರಹ್ಮೋವಾಚ\enginline{-} }
\end{flushleft}

\begin{verse}
\textbf{ವಕ್ಷ್ಯಾಮಿ ಭೂಯೋ ಮಾಘಸ್ಯ ಮಾಹಾತ್ಮ್ಯಂ ಪಾಪನಾಶನಮ್~।}\\\textbf{ಅನುಷ್ಠಾನವತಾಂ ನೄಣಾಂ ಕರ್ಮಗ್ರಂಥಿವಿಮೋಚನಮ್~।। ೧~।। }
\end{verse}

\begin{verse}
\textbf{ಜನ್ಮಪ್ರಭೃತಿ ಯೋ ಮರ್ತ್ಯೋ ದುರಾಚಾರೀ ನಿರಾಶ್ರಯಃ~।}\\\textbf{ವರ್ಣಾಶ್ರಮವಿಹೀನೋಪಿ ಮಾಘಸ್ನಾನಾದ್ವಿಮುಚ್ಯತೇ~।। ೨~।।}
\end{verse}

\begin{verse}
\textbf{ಅತ್ರೈವೋದಾಹರಂತೀಮಮಿತಿಹಾಸಂ ಪುರಾತನಮ್~।}\\\textbf{ಶ್ರೂಯತಾಮವಧಾನೇನ ಶೃಣ್ವತಾಂ ಪಾಪನಾಶನಮ್~।। ೩~।। }
\end{verse}

\begin{flushleft}
ಬ್ರಹ್ಮದೇವರು ಹೇಳುತ್ತಾರೆ\enginline{-}
\end{flushleft}

ಅನುಷ್ಠಾನ ಮಾಡುವ ಜನರ ಕರ್ಮರಾಶಿಯೆಲ್ಲ ಪರಿಹಾರವಾಗುವ ಮಾಘ ಮಾಸದ ಮಾಹಾತ್ಮ್ಯೆಯನ್ನು ಇನ್ನೂ ಹೇಳುತ್ತೇನೆ. ಯಾವನು ಹುಟ್ಟಿದ ದಿನದಿಂದ ದುರಾಚಾರಿಯೂ, ಆಶ್ರಯ ರಹಿತನೂ, ವರ್ಣಾಶ್ರಮ ಕರ್ಮಗಳನ್ನು ಪರಿತ್ಯಾಗ ಮಾಡಿದವನಾದರೂ ಸಹ ಮಾಘಸ್ನಾನದಿಂದ ಪಾಪಗಳನ್ನು ಕಳೆದುಕೊಳ್ಳುತ್ತಾನೆ. ಈ ಸಂದರ್ಭದಲ್ಲಿ ಪುರಾತನವಾದ ಇತಿಹಾಸ ಒಂದನ್ನು ಹೇಳುತ್ತೇನೆ. ಸಾವಧಾನದಿಂದ ಅದನ್ನು ಶ್ರವಣಮಾಡಿದವರ ಪಾಪಗಳೆಲ್ಲವೂ ನಾಶವಾಗುತ್ತವೆ.

\begin{verse}
\textbf{ಪುರಾ ಶತಬಲಿರ್ನಾಮ ಶೂದ್ರೋ ಧರ್ಮವಿವರ್ಜಿತಃ~।}\\\textbf{ಲೋಭಾತ್ ಕೃಷಿಪರಃ ಕ್ರೂರಸ್ತಥಾ ವಾರ್ಧುಷಿಕಃ ಶಠಃ~।। ೪~।। }
\end{verse}

\begin{verse}
\textbf{ತೇನಾರ್ಜಿತಂ ಧನಂ ಭೂರಿ ದಾನಮುಕ್ತಿವಿವರ್ಜಿತಮ್~।}\\\textbf{ಜಾತೌ ತೌ ತನಯೌ ತಸ್ಯ ಸುಭದ್ರೋ ಭದ್ರಸಂಜ್ಞಿ ಕಃ~।। ೫~।।}
\end{verse}

ಪೂರ್ವದಲ್ಲಿ ಶತಬಲಿ ಎಂಬ ಶೂದ್ರನೊಬ್ಬನಿದ್ದನು. ಅವನು ಧರ್ಮವನ್ನು ಪರಿತ್ಯಜಿಸಿ, ಲೋಭದಿಂದ ಕೃಷಿ ಉದ್ಯೋಗವನ್ನು ಕೈಗೊಂಡು ಕ್ರೂರಿಯಾಗಿ ಮೂರ್ಖನಾಗಿ ಬಡ್ಡಿ ಹಣದಿಂದ ಜೀವನ ಸಾಗಿಸುತ್ತಿದ್ದನು. ದಾನ ಧರ್ಮಗಳಿಗೆ ವಿನಿಯೋಗವಾಗದೇ ಇದ್ದ ಅವನು ಸಂಪಾದಿಸಿದ ಹಣ ದೊಡ್ಡದಾಗಿ ಬೆಳೆಯಿತು. ಅವನಿಗೆ ಸುಭದ್ರ, ಭದ್ರ ಎಂಬ ಇಬ್ಬರು ಮಕ್ಕಳು ಹುಟ್ಟಿದರು.

\begin{verse}
\textbf{ತತ್ಪಿತಾ ತು ಮೃತಿಂ ಪ್ರಾಪ್ತೋ ನರಕಂ ಪ್ರತ್ಯ ಪದ್ಯತ~।}\\\textbf{ತತೋ ಯುವಾನೌ ರೂಪಾಢ್ಯೌ ಮತ್ತೌ ಧರ್ಮಪರಾಙ್ಮುಖೌ~।। ೬~।। }
\end{verse}

\begin{verse}
\textbf{ವ್ಯವಾಯಕರ್ಮನಿರತೌ ಸದಾ ದ್ಯೂತಪರಾಯಣೌ~।}\\\textbf{ಸ್ವೈರಿಣೀಕಾಂಕ್ಷಿಣೌ ನಿತ್ಯಂ ಚರಂತೌ ಲೋಕಗರ್ಹಿತೌ~।। ೭~।।}
\end{verse}

ಅವರ ತಂದೆ ಶತಬಲಿ ಸತ್ತು ನರಕಕ್ಕೆ ಹೋದನು. ಯುವಕರಾದ ಸುಭದ್ರ, ಭದ್ರರು ಅಹಂಕಾರದಿಂದ ಯುಕ್ತರಾಗಿ ಧರ್ಮವನ್ನು ಅಲ್ಲಗಳೆದು, ವ್ಯಭಿಚಾರ ಮತ್ತು ಜೂಜಿನ ಆಟಗಳಲ್ಲಿ ಆಸಕ್ತರಾಗಿ, ಸೈಚಾಚಾರಿಗಳಾದ ಸ್ತ್ರೀಯರ ಸಹವಾಸವನ್ನು ಬಯಸುತ್ತಾ, ಜನಗಳಿಂದ ನಿಂದಿತರಾಗಿ ತಿರುಗಾಡುತ್ತಿದ್ದರು.

\begin{verse}
\textbf{ವ್ರಜಖೇಟಕಪಲ್ಲೀಷು ನಗರೀಪಟ್ಟಣಾದಿಷು~।}\\\textbf{ಪ್ರಾತಿಲೋಭ್ಯೇನ ಪಿಶುನೌ ವರ್ಣಸಂಕರಕಾರಕೌ~।। ೮~।। }
\end{verse}

\begin{verse}
\textbf{ಏವಂ ತು ಕಿಯತಾಕಾಲೇ ತದ್ಧನಂ ನಿಧನಂ ಗತೇ~।}\\\textbf{ಪಿತ್ರಾರ್ಜಿತಂ ತತಃ ಪಶ್ಚಾನ್ನಿಃಸ್ವೌ ಜಾತೌ ನರಾಧಮೌ~।। ೯~।।}
\end{verse}

ದನಗಳನ್ನು ಪೋಷಿಸುವ ಜನರಿರುವ ಹಳ್ಳಿಗಳಲ್ಲಿ, ವ್ಯವಸಾಯ ಮಾಡುವ ಜನರಿರುವ ಊರುಗಳಲ್ಲಿ, ಸಣ್ಣ ಮತ್ತು ದೊಡ್ಡ ನಗರಗಳಲ್ಲಿ ತಿರುಗಾಡುತ್ತಾ, ನಿಂದ್ಯ ಕರ್ಮವನ್ನು ಆಚರಿಸುತ್ತಾ, ಜಾತಿಗಳನ್ನು ಕುಲಗೆಡಿಸಲು ಆರಂಭಿಸಿದರು. ಹೀಗಿರಲು ಅವರ ತಂದೆ ಸಂಪಾದಿಸಿದ್ದ ಹಣವೆಲ್ಲ ಖರ್ಚಾಗಿ ಇವರು ದರಿದ್ರರಾದರು.

\begin{verse}
\textbf{ಔಷಧೈರ್ಮಂತ್ರವಶ್ಯಾದ್ಯೈಃ ಪರದಾರಾಭಿಕಾಂಕ್ಷಿಣೌ~।}\\\textbf{ಚೌರ್ಯಧರ್ಮೌ ವನಸ್ಥೌ ತೌ ಜಹ್ರತುರ್ಧನಸಂಚಯಮ್~।। ೧೦~।।}
\end{verse}

ಔಷಧ, ಮಂತ್ರ, ವಶೀಕರಣವಿದ್ಯೆ ಇವುಗಳಿಂದ ಪರಸ್ತ್ರೀಯ ಸಂಬಂಧ ಹೊಂದಲು ಬಯಸುತ್ತಿದ್ದ ಇವರು ಕಾಡಿನಲ್ಲಿ ಕಳ್ಳತನದಿಂದ ಹಣ ಸಂಪಾದನೆ ಮಾಡಲು ಪ್ರಾರಂಭಿಸಿದರು,

\begin{verse}
\textbf{ತಾಭ್ಯಾಂ ಭೀತಾ ನರಾಃ ಕೇಽಪಿ ನ ಚರಂತಿ ಹಿ ಪದ್ಧತೌ~।}\\\textbf{ತತೋ ಮೃಗಾಯುಧರ್ಮೇಣ ಚೇರತುಃ ಪಿಶಿತಾಶನೌ~।। ೧೧~।।}
\end{verse}

ಇವರಿಂದ ಭಯಗೊಂಡ ಜನರು ಆ ದಾರಿಯಲ್ಲಿ ಓಡಾಡುತ್ತಿರಲಿಲ್ಲ. ಸುಭದ್ರ, ಭದ್ರರು ಬೇಟೆಯಾಡಿ ಮಾಂಸಭಕ್ಷಣದಿಂದ ಜೀವಿಸುತ್ತಿದ್ದರು.

\begin{verse}
\textbf{ತದ್ವನೇಽಪಿ ಮೃಗಾ ನಾಸನ್ ತಾಭ್ಯಾಂ ಮೃತಿಮುಪಾಗತಾಃ~।}\\\textbf{ಸಂಪರ್ಕತ್ಪಾಪಿನೋ ಭೂಮಾವನಾವೃಷ್ಟಿರ್ಬಭೂವ ಹ~।। ೧೨~।।}
\end{verse}

ಸ್ವಲ್ಪ ಕಾಲದಲ್ಲಿ ವನದಲ್ಲಿದ್ದ ಮೃಗಗಳು ಸತ್ತು, ಉಳಿದ ಮೃಗಗಳು ಬೇರೆ ಕಡೆ ಹೊರಟುಹೋದುವು. ಈ ಪಾಪಿಗಳ ಭೂಸಂಪರ್ಕದಿಂದ ಮಳೆಯೇ ಇಲ್ಲದಂತಾಯಿತು.

\begin{verse}
\textbf{ದುರ್ಭಿಕ್ಷಂ ಅತುಲಂ ಜಾತಂ ಸ ದೇಶೋ ವಿಜನೋಽಭವತ್~।}\\\textbf{ಬಭಕ್ಷತುಸ್ತತೋ ಗಾಶ್ಚ ಮಹಿಷಾದಿಪುರಃಸರಮ್~।। ೧೩~।।}
\end{verse}

ಆ ದೇಶದಲ್ಲಿ ದುರ್ಭಿಕ್ಷವು ಉಂಟಾಯಿತು. ಜನರಿಲ್ಲದಂತೆ ಆಯಿತು. ಭದ್ರ, ಸುಭದ್ರರು ಹಸು, ಎಮ್ಮೆ ಮುಂತಾದ ಪ್ರಾಣಿಗಳನ್ನು ಕೊಂದು ಜೀವಿಸುತ್ತಿದ್ದರು.

\begin{verse}
\textbf{ಶುನೋ ವೃಕಾನ್ ಗರ್ದಭಾಂಶ್ಚ ವಿಡ್ ವರಾಹಾಂಶ್ಚ ಕುಕ್ಕುಟಾನ್~।}\\\textbf{ತದ್ದೇಶೇ ತೇsಪಿ ನೋಲಭ್ಯಾಸ್ತತಸ್ತೌ ವಿಷಯಾಂತರಮ್~।। ೧೪~।। }
\end{verse}

\begin{verse}
\textbf{ಜಗ್ಮತುಃ ಪಾಪನಿರತೌ ಕ್ರೂರೌ ಸ್ವಾರ್ಥವಿವರ್ಜಿತೌ~।}\\\textbf{ತತಃ ಕಲಿಂಗದೇಶೇ ತು ಪ್ರಾಪ್ಯ ಚಂಪಾಪುರೀಂ ವಟೌ~।। ೧೫~।।}
\end{verse}

ನಾಯಿ, ತೋಳ, ಕತ್ತೆ, ಕೋಳಿ ಮೊದಲಾದ ಯಾವ ಪ್ರಾಣಿಯೂ ಸಿಗದೆ ಆ ಸಹೋದರರು ಪಾಪಾಸಕ್ತರಾಗಿ, ಕ್ರೂರರಾಗಿ, ಕಲಿಂಗದೇಶಕ್ಕೆ ಬಂದು ಅಲ್ಲಿ ಒಂದು ಆಲದ ಮರದ ಆಶ್ರಯದಲ್ಲಿ ನಿಂತರು.

\begin{verse}
\textbf{ಚೌರ್ಯಮಂತ್ರೋಷಧಾದ್ಯಾಶ್ಚ ನೀಯತುರ್ದಿವಸಾನಿ ಚ~।}\\\textbf{ಚಂಪಾಯಾಂ ಬ್ರಾಹ್ಮಣಃ ಕಶ್ಚಿತ್ ವೃದ್ಧೋ ವಾಯುಪ್ರಪೀಡಿತಃ~।। ೧೬~।।}
\end{verse}

\begin{verse}
\textbf{ಸಂಪ್ರಾರ್ಥಯಚ್ಚಿಕಿತ್ಸಾರ್ಥಂ ಜೀವನೇಚ್ಛುಶ್ಚ ತಂ ತದಾ~।}\\\textbf{ತೌ ಚಕ್ರತುರ್ಮಹಾವ್ಯಾಧಿಪೀಡಿತಸ್ಯ ಮಹಾತ್ಮನಃ~।। ೧೭~।।}
\end{verse}

\begin{verse}
\textbf{ಚಿಕಿತ್ಸಾಮೌಷಧೀರ್ದಿವ್ಯೈಸ್ತಥಾ ವ್ಯಾಧಿಶಮೋ ಭವೇತ್~।}\\\textbf{ತತೋ ವಿಪ್ರಂ ಯಯಾಚಾತೇ ವದ್ಯತ್ವಂ ಚ ಪ್ರತಿಶ್ರುತಮ್~।। ೧೮~।।}
\end{verse}

ಕಳ್ಳತನ, ಮಂತ್ರ, ಔಷಧಾದಿಗಳಿಂದ ದಿನಗಳನ್ನು ಕಳೆದರು. ಚಂಪಾಪುರಿಯಲ್ಲಿದ್ದ, ವಾತರೋಗದಿಂದ ಬಳಲುತ್ತಿದ್ದ ಒಬ್ಬ ಬ್ರಾಹ್ಮಣನು ಅವರಲ್ಲಿಗೆ ಬಂದು ತನ್ನ ರೋಗಶಮನಾರ್ಥವಾಗಿ ಸಹಾಯವನ್ನು ಬೇಡಿದನು. ವ್ಯಾಧಿಯು ಶಮನವಾಗಲು ಬೇಕಾದ ಔಷಧಿಗಳನ್ನು ಅವರು ಉಪಯೋಗಿಸಿ ಆ ಮಹಾತ್ಮನ ರೋಗವು ಗುಣವಾಗುವಂತೆ ಮಾಡಿದರು. ಬಳಿಕ ಭದ್ರ, ಸುಭದ್ರರು ಆ ಬ್ರಾಹ್ಮಣನನ್ನು ಯಾಚಿಸಿದರು.

\begin{verse}
\textbf{ಸ ವಿಪ್ರೋ ದ್ರವಿಣಾಭಾವಾತ್ ಸುಭದ್ರಾವರಜಂ ತದಾ~।}\\\textbf{ಯಾಚಯಿತ್ವಾ ದಾಪಯಿತುಂ ದೌಹಿತ್ರ ಸ್ಯಾಂತಿಕಂ ಯಯೌ~।। ೧೯~।। }
\end{verse}

\begin{verse}
\textbf{ಗ್ರಾಮೋ ನದೀವರೋ ನಾಮ ಪಂಚಯೋಜನದೂರತಃ~।}\\\textbf{ಚಂಪಾಯಾಂ ಬ್ರಾಹ್ಮಣಾವಾಸೋ ಯತ್ರ ಚಂಪಾವತೀ ನದೀ~।। ೨೦~।।}
\end{verse}

ತನ್ನ ಬಳಿ ಹಣ ಇರದಿದ್ದ ಕಾರಣದಿಂದ ಆ ಬ್ರಾಹ್ಮಣನು ಸುಭದ್ರನ ತಮ್ಮನಾದ ಭದ್ರನನ್ನು ಪ್ರಾರ್ಥಿಸಿ ಅವನ ಜತೆಯಲ್ಲಿ ತನ್ನ ಮೊಮ್ಮಗನ ಬಳಿ ಹೋದನು. ಆ ಮೊಮ್ಮಗನು ಐದು ಯೋಜನ ದೂರವಿದ್ದ ನದೀವರ ಎಂಬ ಗ್ರಾಮದಲ್ಲಿದ್ದನು. ಅಲ್ಲಿ ಚಂಪಾವತೀ ಎಂಬ ನದಿ ಹರಿಯುತ್ತಿತ್ತು. ಬ್ರಾಹ್ಮಣರು ವಾಸಿಸುವ ಅಗ್ರಹಾರದಲ್ಲಿ ಆ ಮೊಮ್ಮಗನು ಇದ್ದನು.

\begin{verse}
\textbf{ತಸ್ಯಾಸ್ತೀರೇ ಮಹಾರುದ್ರಶ್ಚಂಪಕೇಶ್ವರಸಂಜ್ಞಿತಃ~।}\\\textbf{ನಿತ್ಯಂ ಸನ್ನಿಹಿತಸ್ತತ್ರ ಕಲಿಂಗಾನಾಂ ಕುಲೇಶ್ವರಃ~।। ೨೧~।। }
\end{verse}

\begin{verse}
\textbf{ದಿನಾನಿ ತ್ರೀಣಿ ಹ್ಯಗಮತ್ ತದ್‌ಗ್ರಾಮೇ ಭದ್ರಸಂಜ್ಞಿತಃ~।}\\\textbf{ಬ್ರಾಹ್ಮಣೋ ದ್ರವಿಣಂ ದಾತುಂ ದೈವಾತ್ಕಾಲೇ ಬಲೀಯಸೀ~।। ೨೨~।।}
\end{verse}

ಕಲಿಂಗದೇಶದ ಕುಲದೇವರಾದ ಚಂಪಕೇಶ್ವರನೆಂಬ ರುದ್ರದೇವರ ದೇವಾಲಯವು ಆ ನದೀತೀರದಲ್ಲಿತ್ತು. ಭದ್ರನ ಅಪ್ಪಣೆ ಪಡೆದು ಅವನೊಡನೆ ಹಣ ತರಲು ಬಂದು ಮೂರು ದಿನಗಳು ಕಳೆದುವು. ದೈವಕ್ಕಿಂತ ಕಾಲವು ಬಲವಾದುದು.

\begin{verse}
\textbf{ತದಾಸೀನ್ಮಾಘಮಾಸೋಽಪಿ ಸರ್ವಧರ್ಮಾಲಯೋ ಮಹಾನ್~।}\\\textbf{ಬ್ರಾಹ್ಮಣಾನಾಂ ಪ್ರಸಂಗೇನ ಜ್ಞಾತಬುದ್ಧಿಃ ಸಪಾದಜಃ~।। ೨೩~।।}
\end{verse}

ಧರ್ಮಗಳ ಬೀಡಾದ ಮಾಘಮಾಸವಾಗಿತ್ತು. ಬ್ರಾಹ್ಮಣರ ಸಹವಾಸದಿಂದ ಆ ಶೂದ್ರನಾದ ಭದ್ರನಿಗೆ ಬುದ್ದಿ ಬದಲಾಯಿತು.

\begin{verse}
\textbf{ಚಂಪಾವತ್ಯಾಂ ಚ ತ್ರಿದಿನಂ ಮಾಘಸ್ನಾನಂ ಚಕಾರ ಸಃ~।}\\\textbf{ತತಶ್ಚಂಪಾಪುರೀಂ ಪ್ರಾಪ್ಯ ಸುಭದ್ರೇಣ ಸಮನ್ವಿತಃ~।। ೨೪~।।}
\end{verse}

ಮೂರು ದಿನಗಳಕಾಲ ಭದ್ರನು ಚಂಪಾವತೀ ನದಿಯಲ್ಲಿ ಸ್ನಾನಮಾಡಿದನು. ನಂತರ ಸುಭದ್ರನನ್ನೂ ಕರೆದುಕೊಂಡು ಚಂಪಾಪುರಿಯಲ್ಲಿ ಇಬ್ಬರೂ ವಾಸ ಮಾಡಿದರು.

\begin{verse}
\textbf{ಜನ್ಮಭೂಮಿಂ ಪ್ರತಿ ಪುನಃ ಪ್ರಯಾಣಾಯೋಪಚಕ್ರಮೇ~।}\\\textbf{ಗಚ್ಛನ್ಮಧ್ಯಾಹ್ನವೇಲಾಯಮರಣ್ಯೇ ಜನವರ್ಜಿತೇ~।। ೨೫~।। }
\end{verse}

\begin{verse}
\textbf{ಶ್ರಾಂತೋ ಭದ್ರಃ ಸುಭದ್ರಂ ತು ಕಸ್ಮಿಂಶ್ಚಿದ್ಗಿರಿಕಂದರೇ~।}\\\textbf{ಉಪವಿಶ್ಯ ಕ್ಷಣಂ ಕಿಂಚಿಜ್ಜಲಾರ್ಥಂ ಚ ಸರೋ ಯಯೌ~।। ೨೬~।।}
\end{verse}

ತಮ್ಮ ಊರಿಗೆ ಹೊರಟು ಭದ್ರ, ಸುಭದ್ರರು ಮಧ್ಯಾಹ್ನ ಕಾಲದ ಹೊತ್ತಿಗೆ ಜನಸಂಚಾರವಿಲ್ಲದ ಅರಣ್ಯಕ್ಕೆ ಬಂದರು. ಅಲ್ಲಿ ಸ್ವಲ್ಪ ಹೊತ್ತು ಕುಳಿತು ವಿಶ್ರಮಿಸಿ ಕೊಂಡರು. ಭದ್ರನು ಸುಭದ್ರನನ್ನು ಅಲ್ಲಿಯೇ ಕೂಡಿಸಿ ಬೆಟ್ಟದ ಗುಹೆಯಿಂದ ಬರುತ್ತಿದ್ದ ನೀರಿನ ಸರೋವರಕ್ಕೆ ನೀರನ್ನು ತರಲು ಹೋದನು.

\begin{verse}
\textbf{ಗಹ್ವರಸ್ಥೋ ಮಹಾವ್ಯಾಘ್ರಃ ಸುಭದ್ರಂ ಕಾಲಚೋದಿತಃ~।}\\\textbf{ಕಂಠೇ ಜಗ್ರಾಹ ಭದ್ರೋsಪಿ ಜಲಮಾದಾಯ ಸತ್ವರಮ್~।। ೨೭~।।}
\end{verse}

ಬೆಟ್ಟದಲ್ಲಿದ್ದ ಒಂದು ಹುಲಿಯು ಸುಭದ್ರನ ಕುತ್ತಿಗೆಯನ್ನು ಹಿಡಿಯಿತು. ಸುಭದ್ರನಿಗೆ ಅಂತ್ಯಕಾಲವು ಪ್ರಾಪ್ತವಾಗಿತ್ತು. ಆ ಹೊತ್ತಿಗೆ ಭದ್ರನು ನೀರನ್ನು ತಂದನು.

\begin{verse}
\textbf{ಆಗತೋ ಭ್ರಾತರಂ ದೃಷ್ಟ್ವಾ ವ್ಯಾಘ್ರಗ್ರಸಿತಮಗ್ರಜಮ್~।}\\\textbf{ವಿಹ್ವಲಃ ಖಡ್ಗಮಾದಾಯ ವ್ಯಾಘ್ರಂ ಹಂತುಮುಪಾದ್ರವತ್~।। ೨೮~।।}
\end{verse}

ಹುಲಿಯಿಂದ ಹಿಡಿಯಲ್ಪಟ್ಟ ತನ್ನ ಸಹೋದರನನ್ನು ನೋಡಿ ಭದ್ರನು ಸಿಟ್ಟಿನಿಂದ ವ್ಯಾಘ್ರನನ್ನು ಸಂಹರಿಸಲು ಖಡ್ಗವನ್ನು ತೆಗೆದುಕೊಂಡು ಮೈ ಮೇಲೆ ಎರಗಿದನು.

\begin{verse}
\textbf{ತಂ ದೃಷ್ಟ್ವಾ ಕುಪಿತೋ ವ್ಯಾಘ್ರಃ ತಂ ವಿಹಾಯ ಯಯಾವಮುಮ್~।}\\\textbf{ತಂ ಜಗ್ರಾಹ ಗಲೇ ತದ್ವತ್ ಸೋಽಪಿ ವಿವ್ಯಾಧ ಖಡ್ಗತಃ~।। ೨೯~।।}
\end{verse}

ಭದ್ರನನ್ನು ನೋಡಿ ಹುಲಿಯು ಕೋಪಗೊಂಡು ಸುಭದ್ರನ ಕುತ್ತಿಗೆಯನ್ನು ಬಿಟ್ಟು ಭದ್ರನ ಮೇಲೆ ಬಿದ್ದಿತು. ಭದ್ರನು ಕತ್ತಿಯಿಂದ ಹುಲಿಯನ್ನು ಸಂಹರಿಸಿದನು.

\begin{verse}
\textbf{ತ್ರಯೋಽಪಿ ಮರಣಂ ಜಗ್ಮುಃ ವ್ಯಾಘ್ರೋ ಭದ್ರಃ ಸುಭದ್ರಕಃ~।}\\\textbf{ತ್ರೀನೇತಾನ್ನೇತುಕಾಮಾಸ್ತು ಯಮಲೋಕಂ ಯಮಾನುಗಾಃ~।। ೩೦~।।} 
\end{verse}

\begin{verse}
\textbf{ಆಜಗ್ಮುಃ ಸಹಸಾ ಕ್ರೂರಾ ದಂಡಪಾಶಾಸಿಪಾಣಯಃ~।}\\\textbf{ಪಾಶೈರಾಬದ್ಧ್ಯ ತಾನೇತಾನ್ ಯಯುರ್ಯಮನಿಕೇತನಮ್~।। ೩೧~।।}
\end{verse}

ವ್ಯಾಘ್ರ, ಭದ್ರ, ಸುಭದ್ರ ಮೂರು ಜನರೂ ಮರಣ ಹೊಂದಿದರು. ಕ್ರೂರರಾದ ಯಮಭಟರು ದಂಡ, ಪಾಶ, ಕತ್ತಿಗಳನ್ನು ಕೈಯಲ್ಲಿ ಹಿಡಿದುಕೊಂಡು ಹಗ್ಗದಿಂದ ಈ ಮೂವರನ್ನೂ ಬಿಗಿದು ಯಮಲೋಕಕ್ಕೆ ಕರೆದುಕೊಂಡುಹೋದರು.

\begin{verse}
\textbf{ಆಜ್ಞಾಪಯತ್ತದಾ ವಿಷ್ಣುಃ ಪಾರ್ಷದಾಂಶ್ಚ ತುರಃ ಶುಭಾನ್~।}\\\textbf{ಭದ್ರಮಾನಯತ ಕ್ಷಿಪ್ರಂ ಮಾಘಸ್ನಾನಪರಾಯಣಮ್~।। ೩೨~।।} 
\end{verse}

\begin{verse}
\textbf{ಇತ್ಯಾಜ್ಞಪ್ತಾ ಭಗವತಾ ವಿಷ್ಣು ದೂತಾಃ ಸಮಾಗಮನ್~।}\\\textbf{ಸಂಧ್ಯಾರುಣಾಂಬರಾನೇತಾನ್ ಚತುರ್ಬಾಹೂನ್ ಮನೋಹರಾನ್~।। ೩೩~।। }
\end{verse}

\begin{verse}
\textbf{ದದ್ರುವುರ್ಭಯತೋ ವಿಪ್ರ ತಾನ್ ದೃಷ್ಟಾ ವಿಷ್ಣು ಪಾರ್ಷದಾನ್~।}\\\textbf{ತಾನೂಚುರ್ಭಗವದ್ದೂತಾ ಧರ್ಮಜ್ಞಾನ್ ಯಮಕಿಂಕರಾನ್~।। ೩೪~।।}
\end{verse}

ಆಗ ಶ‍್ರೀಹರಿಯು ಮಾಘಸ್ನಾನ ಮಾಡಿದ ಭದ್ರನನ್ನು ಶೀಘ್ರವೇ ಕರೆದುಕೊಂಡು ಬರಲು ತನ್ನ ದೂತರಿಗೆ ಆಜ್ಞೆ ಮಾಡಿದನು. ಸಂಧ್ಯಾಕಾಲದ ಸೂರ್ಯನ ಬಣ್ಣದಂತೆ ಇದ್ದ ಪೀತಾಂಬರಧಾರಿಗಳಾದ, ನೋಡಲು ಒಳ್ಳೆ ರೂಪವಂತರಾದ, ನಾಲ್ಕು ಬಾಹುಗಳನ್ನುಳ್ಳ ವಿಷ್ಣು ದೂತರು ಯಮದೂತರ ಬಳಿಗೆ ಹೋದರು. ನಾರದನೇ, ಆ ವಿಷ್ಣು ದೂತರನ್ನು ಕಂಡು ಯಮಭಟರು ಓಡಿದರು. ಆಗ ಧರ್ಮವನ್ನು ತಿಳಿದ ಆ ಯಮಭಟರನ್ನು ಕುರಿತು ವಿಷ್ಣು ದೂತರು ಹೇಳಿದರು:

\begin{verse}
\textbf{ಭದ್ರಂ ಮುಂಚಂತು ನಿಷ್ಪಾಪಂ ನಾಯಮರ್ಹತಿ ಯಾತನಾಮ್~।}\\\textbf{ಇತ್ಯುಕ್ತಾ ಯಮದೂತಾಸ್ತೇ ಭಗವತ್ಪಾರ್ಶ್ವವರ್ತಿಭಿಃ~।। ೩೫~।।}
\end{verse}

ಯಮಭಟರೇ, ಪಾಪಿಷ್ಠನಲ್ಲದ ಈ ಭದ್ರನನ್ನು ಬಿಟ್ಟು ಬಿಡಿ. ಅವನು ಶಿಕ್ಷೆಗೆ ಅರ್ಹನಲ್ಲ. ಈ ರೀತಿ ಭಗವಂತನ ದೂತರಿಂದ ಹೇಳಿಸಿಕೊಳ್ಳಲ್ಪಟ್ಟ ಯಮದೂತರು,

\begin{verse}
\textbf{ಪ್ರತ್ಯೂಚುಸ್ತೇ ಯಮಭಟಾ ವಿಸ್ಮಯೋತ್ಫುಲ್ಲಲೋಚನಾಃ~।}\\\textbf{ನಾಯಂ ಪಾಪೀ ದುರಾಚಾರೀ ಭವದ್ಭಿರ್ನೇತುಮರ್ಹತಿ~।। ೩೬~।। }
\end{verse}

\begin{verse}
\textbf{ಭವದ್ಭಿರ್ನೇತುಮರ್ಹಾಸ್ತೇ ಯೇ ಚ ಧರ್ಮಪರಾಯಣಾಃ~।}\\\textbf{ಅಪಿ ಸರ್ಷಪಮಾತ್ರೋಽಪಿ ಧರ್ಮೋ ನಾನುಷ್ಠಿ ತೋಽಮುನಾ~।। ೩೭~।।}
\end{verse}

ಆಶ್ಚರ್ಯಚಕಿತರಾಗಿ ಕಣ್ಣುಗಳನ್ನು ಅರಳಿಸಿ ಹೇಳಿದರು: ಈ ಪಾಪಿಷ್ಠನು ತಮ್ಮಿಂದ ಕರೆದುಕೊಂಡು ಹೋಗಲ್ಪಡಲು ಅರ್ಹನಲ್ಲ. ಧರ್ಮ ನಿರತರಾದ ಜನರು ನಿಮ್ಮಿಂದ ಒಯ್ಯಲ್ಪಡಬೇಕು. ಈ ಭದ್ರನಾದರೋ ಒಂದು ಸಾಸಿವೆಕಾಳಿನಷ್ಟು ಧರ್ಮವನ್ನೂ ಮಾಡಿಲ್ಲ.

\begin{verse}
\textbf{ಇತಿ ತೈರುದಿತಾ ವಿಷ್ಣು ಪಾರ್ಷದಾಃ ಪುನರಬ್ರುವನ್~।}\\\textbf{ಯದಿ ಯೂಯಂ ಧರ್ಮಭಟಾ ಬ್ರೂತ ಧರ್ಮಾನ್ಯ ಮೋದಿತಾನ್~।। ೩೮~।।}
\end{verse}

ಹೀಗೆಂದು ಹೇಳಿದ ಯಮಭಟರನ್ನು ಕುರಿತು ವಿಷ್ಣು ದೂತರು ಪುನಃ ಹೇಳಿದರು: ನೀವು ನಿಜವಾಗಿ ಯಮಕಿಂಕರರಾಗಿದ್ದರೆ ಯಮದೇವರು ಹೇಳಿರುವ ಧರ್ಮ ಸೂಕ್ಷ್ಮಗಳನ್ನು ವಿವರಿಸಿರಿ.

\begin{verse}
\textbf{ಇತ್ಯುಕ್ತಾ ವಿಷ್ಣು ದೂತೈಸ್ತು ತೇ ಯಮೋಕ್ತಾನ್ಯಥಾಽಭ್ರುವನ್~।}\\\textbf{ಶ್ರುತಿಸ್ಮೃತಿಪುರಾಣಾದ್ಯೈರುದಿತೋ ಧರ್ಮ ಉಚ್ಛ್ರಿತಃ~।। ೩೯~।।} 
\end{verse}

\begin{verse}
\textbf{ಶ್ರುತಿಸ್ಮೃತಿಪುರಾಣಾನಾಂ ವಿರುದ್ಧಸ್ತದ್ವಿಪರ್ಯಯಃ~।}\\\textbf{ಯೇ ನಿತ್ಯ ಕರ್ಮನಿರತಾ ಯೇ ಚ ಶ್ರಾದ್ಧ ಕರಾ ನರಾಃ~।। ೪೦~।।}
\end{verse}

ವಿಷ್ಟು ದೂತರಿಂದ ಹೀಗೆ ಪ್ರಶ್ನಿಸಲ್ಪಟ್ಟ ಯಮದೂತರು ಯಮದೇವರಿಂದ ಉಪದೇಶಿಸಲ್ಪಟ್ಟ ಧರ್ಮ (ಪುಣ್ಯಕರ್ಮಗಳ ವಿವರ) ಗಳನ್ನು ಹೀಗೆ ಹೇಳಿದರು: ಶ್ರುತಿ ಸ್ಮೃತಿ ಪುರಾಣಾದಿಗಳಲ್ಲಿ ಧರ್ಮವೆಂದು ಏನು ಹೇಳಲ್ಪಟ್ಟಿವೆಯೋ ಅವೇ ಶ್ರೇಷ್ಠವಾದ ಪುಣ್ಯಕರ್ಮಗಳು. ಅವುಗಳಲ್ಲಿ ಹೇಳಲ್ಪಟ್ಟ ಕರ್ಮಗಳಿಗೆ ವಿರುದ್ಧವಾದುವುಗಳು ಪಾಪ ಕರ್ಮಗಳು. ಯಾರು ಸ್ವಸ್ವಯೋಗ್ಯ ವರ್ಣಾಶ್ರಮಗಳಿಗೆ ವಿಹಿತವಾದ ಕರ್ಮಗಳನ್ನು ಆಚರಿಸುತ್ತಾ, ಪಿತೃ ಕಾರ್ಯಗಳನ್ನು ಶ್ರದ್ಧೆಯಿಂದ ಮಾಡುತ್ತಾರೆಯೋ,

\begin{verse}
\textbf{ಪ್ರಾತಃಸ್ಮಾನರತಾ ಯೇ ಚ ತೇ ನ ಯಾಂತಿ ಯಮಾಲಯಮ್~।}\\\textbf{ಯೇಷಾಂ ಗೃಹೇ ಚ ತುಲಸೀ ರೋಪಿತಾ ವರ್ಧತೇಽನಿಶಮ್~।। ೪೧~।। }
\end{verse}

\begin{verse}
\textbf{ಶಾಲಗ್ರಾಮಶಿಲಾವಾಪಿ ನ ತೇ ಯಾಂತಿ ಯಮಾಲಯಮ್~।}\\\textbf{ಯೇ ಚ ವಿಷ್ಣು ಕಥಾಸಕ್ತಾ ಯೇ ಚ ತತ್ಸಜ್ಜನಾಶ್ರಯಾಃ~।। ೪೨~।।}
\end{verse}

\begin{verse}
\textbf{ಗಾಯತ್ರೀಜಪ್ಯ ನಿರತಾಸ್ತೇ ನ ಯಾಂತಿ ಯಮಾಲಯಮ್~।}\\\textbf{ಸಹಸ್ರನಾಮ ಪಠನನಿರತಾಸ್ತೋತ್ರಪಾಠಕಾಃ~।। ೪೩~।।}
\end{verse}

ಯಾರು ಪ್ರಾತಃಕಾಲದಲ್ಲಿ ಸ್ನಾನವನ್ನು ಮಾಡುತ್ತಾರೋ, ಯಾರ ಮನೆಯ ಅಂಗಳದಲ್ಲಿ ತುಲಸೀಗಿಡಗಳು ನಿತ್ಯದಲ್ಲಿಯೂ ಅಭಿವೃದ್ಧಿ ಹೊಂದುತ್ತವೆಯೋ, ಯಾರ ಮನೆಯಲ್ಲಿ ಶಾಲಗ್ರಾಮವಿರುತ್ತದೆಯೋ, ಯಾರು ಶ‍್ರೀ ವಿಷ್ಣುವಿನ ಕಥೆಗಳನ್ನು ಶ್ರವಣ ಮಾಡುವುದರಲ್ಲಿ ಆಸಕ್ತರಾಗಿರುತ್ತಾರೆಯೋ, ಯಾರು ಸಜ್ಜನರ ಸಹವಾಸದಲ್ಲಿರುತ್ತಾರೆಯೋ, ಯಾರು ನಿತ್ಯದಲ್ಲಿಯೂ ಗಾಯತ್ರೀ ಜಪ ಮಾಡುವುದರಲ್ಲಿ ಆಸಕ್ತಿಯನ್ನು ಹೊಂದಿರುತ್ತಾರೆಯೋ, ವಿಷ್ಣು ಸಹಸ್ರನಾಮಸ್ತೋತ್ರ ಮತ್ತು ಇತರ ವಿಷ್ಣು ಸ್ತೋತ್ರಗಳನ್ನು ಪಠಿಸುತ್ತಾರೆಯೋ ಅವರು ಯಮಲೋಕಕ್ಕೆ ಹೋಗುವುದಿಲ್ಲ.

\textbf{[ವಿಶೇಷಾಂಶ:} ತುಲಸಿಯ ಶ್ರೇಷ್ಠತೆಯನ್ನು ಸ್ಮೃತಿಗಳು ಹೀಗೆ ತಿಳಿಸುತ್ತವೆ:

\begin{verse}
\textbf{ತುಳಸೀಕಾನನಂ ಚೈವ ಗೃಹೇ ಯಸ್ಯಾವತಿಷ್ಠತೇ~।}\\\textbf{ತದ್ ಗೃಹಂ ತೀರ್ಥಭೂತಂ ಹಿ ನಾಯಾಂತಿ ಯಮಕಿಂಕರಾಃ~।।}
\end{verse}

ಯಾರ ಮನೆಯ ಅಂಗಳದಲ್ಲಿ ತುಳಸಿಗಿಡಗಳ ಕಾಡು ಇರುತ್ತದೆಯೋ ಅಂತಹ ಪ್ರದೇಶವು ಎಲ್ಲ ತೀರ್ಥಾಭಿಮಾನಿ ದೇವತೆಗಳ ವಾಸಸ್ಥಳವಾಗಿರುತ್ತದೆ. ಆ ಮನೆಗೆ ಯಮದೂತರು ಬರುವುದಿಲ್ಲ. (ಮೂರು ತುಳಸೀ ಗಿಡಗಳಿಗಿಂತ ಜಾಸ್ತಿ ಇದ್ದರೆ ಕಾಡು ಎಂದರ್ಥ.)]

\begin{verse}
\textbf{ಗೀತಾಪಾಠೇ ತು ನಿರತಾಸ್ತೇ ನ ಯಾಂತಿ ಯಮಾಲಯಮ್~।}\\\textbf{ಏಕಾದಶೀವ್ರತಸ್ಥಾ ಯೇ ದ್ವಾದಶೀನಾತಿಲಂಘಿನಃ~।। ೪೪~।।} 
\end{verse}

\begin{verse}
\textbf{ದರ್ಶಶ್ರಾದ್ಧ ಪರಾ ಯೇ ಚ ತೇ ನ ಯಾಂತಿ ಯಮಾಲಯನಮ್~।}\\\textbf{ಯೇ ವಿಷ್ಣೋರ್ದೀಪದಾ ಲೋಕೇ ಯೇ ಚ ಪುಷ್ಪ ಸಮರ್ಪಿಣಃ~।। ೪೫~।। }
\end{verse}

ಗೀತಾ ಪಾಠ - ಪ್ರವಚನದಲ್ಲಿ ಆಸಕ್ತರಾದವರು, ಏಕಾದಶೀ ವ್ರತವನ್ನು ಅನುಸರಿಸುವವರು, ದ್ವಾದಶೀ ದಿನದ ವ್ರತಗಳನ್ನು ಉಲ್ಲಂಘನೆ ಮಾಡದವರು, ದರ್ಶಶ್ರಾದ್ಧದಲ್ಲಿ ಆಸಕ್ತ\break ರಾದವರು,ವಿಷ್ಣು ಪ್ರೀತ್ಯರ್ಥವಾಗಿ ದೀಪದಾನ, ಪುಷ್ಪಾರ್ಪಣೆ ಮಾಡುವವರು ಯಮಲೋಕಕ್ಕೆ ಹೋಗುವುದಿಲ್ಲ.

\begin{verse}
\textbf{ಗುಗ್ಗುಲಂ ಧೂಪದಾತಾರೋ ನ ತೇ ಯಾಂತಿ ಯಮಾಲಯಮ್~।}\\\textbf{ಯೇ ನವಂತಿ ಸಕೃದ್ವಾಪಿ ವಿಷ್ಣುಂ ನಾರಾಯಣಂ ಹರಿಮ್~।। ೪೬~।। }
\end{verse}

\begin{verse}
\textbf{ಯೇ ವೇದಶಾಸ್ತ್ರನಿರತಾಸ್ತೇ ನ ಯಾಂತಿ ಯಮಾಲಯಮ್~।}\\\textbf{ಯೇಷಾಂ ಜಿಹ್ವಾ ವದತ್ಯೇವಂ ಗೋವಿಂದಾಚ್ಯುತಮಾಧವ~।। ೪೭~।।}
\end{verse}

\begin{verse}
\textbf{ಹರೇಕೃಷ್ಣ ಮುಕುಂದೇತಿ ನ ತೇ ಯಾಂತಿ ಯಮಾಲಯಮ್~।}\\\textbf{ಸ್ವಯಂ ಧರ್ಮಂ ಚರಂತಶ್ಚ ಪರೇಷಾಂ ಚ ಪ್ರವರ್ತಕಾಃ~।। ೪೮~।।}
\end{verse}

ಶ‍್ರೀಹರಿಗೆ ದಶಾಂಗ ಸಹಿತ ಧೂಪದ ಆರತಿಯನ್ನು ಮಾಡುವವರು, ಸರ್ವತ್ರ ವ್ಯಾಪ್ತನಾದ, ಪಾಪ ಪರಿಹಾರಕನಾದ, ಗುಣಪೂರ್ಣನಾದ ನಾರಾಯಣನನ್ನು ಕೇವಲ ಒಂದು ಬಾರಿ ನಮಸ್ಕಾರ ಮಾಡಿದವರೂ ಸಹ, ವೇದ-ಸಚ್ಛಾಸ್ತ್ರಗಳ ಪಾಠ-ಪ್ರವಚನಗಳಲ್ಲಿ ನಿರತರಾಗಿರುವವರು, ಯಾರ ನಾಲಿಗೆಯು ಗೋವಿಂದ, ಅಚ್ಯುತ, ಮಾಧವ, ಹರೇ, ಕೃಷ್ಣ, ಮುಕುಂದ ಎಂಬುದಾಗಿ ನುಡಿಯುತ್ತದೆಯೋ ಅಂತಹ ಜನರು, ಯಾರು ತಾವೂ ಸಹ ಸತ್ಕರ್ಮಗಳನ್ನು ಅನುಸರಿಸಿ ಇತರರಿಗೂ ಅವುಗಳನ್ನು ಉಪದೇಶ ಮಾಡುತ್ತಾರೆಯೋ ಅವರು ಯಮಲೋಕಕ್ಕೆ ಹೋಗುವುದಿಲ್ಲ.

\begin{verse}
\textbf{ಪ್ರದಕ್ಷಿಣಾಂ ವಾ ಕುರ್ವಂತಿ ನ ತೇ ಯಾಂತಿ ಯಮಾಲಯಮ್~।}\\\textbf{ಯೇಷಾಂ ಗೃಹೇ ಭಾಗವತಂ ಲಿಖಿತಂ ಪುಸ್ತಕಂ ಯದಿ~।। ೪೯~।। }
\end{verse}

\begin{verse}
\textbf{ಗೋದಾನಸ್ಯ ಪ್ರದಾತಾರೊ ಗಂಗಾಯಾತ್ರಾಪರಾಯಣಾಃ~।}\\\textbf{ಗೋಪಿಚಂದನಲಿಪ್ತಾಂಗಾ ನ ತೇ ಯಾಂತಿ ಯಮಾಲಯಮ್~।। ೫೦~।। }
\end{verse}

\begin{verse}
\textbf{ಗೀತಾಯಾಃ ಪುಸ್ತಕಂ ವಾಪಿ ನ ತೇ ಯಾಂತಿ ಯಮಾಲಯಮ್~।}
\end{verse}

ಶ‍್ರೀ ವಿಷ್ಣುವಿಗೆ ಪ್ರದಕ್ಷಿಣೆ ಮಾಡುವವರು, ಯಾರ ಮನೆಯಲ್ಲಿ ಹಸ್ತಲಿಖಿತ ಭಾಗವತ ಗ್ರಂಥವು ಇರುತ್ತದೆಯೋ ಅವರು, ಗೋದಾನ ಮಾಡಿದವರು, ಗಂಗಾ ಸ್ನಾನಕ್ಕಾಗಿ ಯಾತ್ರೆಯಲ್ಲಿ ನಿರತರಾದವರು, ಗೋಪಿಚಂದನದಿಂದ ನಾಮ-ಮುದ್ರೆಗಳನ್ನು ಧರಿಸುವವರು, ಯಾರ ಮನೆಯಲ್ಲಿ ಗೀತಾ ಗ್ರಂಥವು ಇರುತ್ತದೆಯೋ ಅವರು ಯಮಲೋಕಕ್ಕೆ ಹೋಗುವುದಿಲ್ಲ.

\begin{verse}
\textbf{ಯೇ ಸಾಯಮತಿಥಿಂ ಪ್ರಾಪ್ತಂ ಭೋಜಯಂತಿ ಸ್ವಶಕ್ತಿತಃ~।। ೫೧~।।} 
\end{verse}

\begin{verse}
\textbf{ಯೇ ಮಾತೃ ಪಿತೃಭಕ್ತಾಶ್ಚ ನ ತೇ ಯಾಂತಿ ಯಮಾಲಯಮ್~।}\\\textbf{ಯೇ ವಾ ಮಾಧವಪೂಜಾರ್ಥಂ ಗೋಮಯೇನೋಪಲಿಪ್ಯ ಚ~।। ೫೨~।। }
\end{verse}

\begin{verse}
\textbf{ರಂಜಯಂತಿ ಚ ಪದ್ಮಾದೈರ್ನ ತೇ ಯಾಂತಿ ಯಮಾಲಯಮ್~।}\\\textbf{ಯೇ ವಾ ಶಿಶುಭ್ಯೋ ಯಚ್ಛಂತಿ ಪಯೋಘೃ ತಸಮನ್ವಿತಮ್~।। ೫೩~।।}
\end{verse}

ಯಾರು ಸಂಧ್ಯಾ ಕಾಲದಲ್ಲಿ ಬಂದ ಅತಿಥಿಗಳಿಗೆ ಶಕ್ತ್ಯಾನುಸಾರ ಭೋಜನ ಮಾಡಿಸುವರೋ, ತಂದೆ-ತಾಯಿಯರಲ್ಲಿ ಭಕ್ತಿಯುಕ್ತರಾದವರು, ಶ‍್ರೀಹರಿಯ ಪೂಜಾ ನಡೆಯುವ ಸ್ಥಳವನ್ನು ಗೋಮಯದಿಂದ ಸಾರಿಸಿ, ಬಣ್ಣ ಬಣ್ಣದ ರಂಗೋಲೆಯಿಂದ ಚಿತ್ರ ಬರೆದು ಅಲಂಕಾರ ಮಾಡುವವರು, ಮಕ್ಕಳಿಗೆ ಹಾಲುತುಪ್ಪ ಇವುಗಳನ್ನು ಕೊಡುವವರು ಯಮಲೋಕಕ್ಕೆ ಹೋಗುವುದಿಲ್ಲ.

\begin{verse}
\textbf{ಯೇ ವಾ ಪರ್ಯಂಕದಾ ಲೋಕೇ ನ ತೇ ಯಾಂತಿ ಯಮಾಲಯಮ್~।}\\\textbf{ತುಲಸೀಮೃದಮಾಲಿಪ್ಯ ಯೇ ವಾ ಸ್ನಾನಂ ಪ್ರಕುರ್ವತೇ~।। ೫೪~।। }
\end{verse}

\begin{verse}
\textbf{ಲಲಾಟೇ ತಿಲಕಂ ವಾಪಿ ನ ತೇ ಯಾಂತಿ ಯಮಾಲಯಮ್~।}\\\textbf{ಚಾತುರ್ಮಾಸೇ ವ್ರತೇ ಪ್ರಾಪ್ತೇ ಆಷಾಢೇ ಶಾಕವರ್ಜಿತಾಃ~।। ೫೫~।।} 
\end{verse}

\begin{verse}
\textbf{ದಧಿ ತ್ಯಕ್ತ್ವಾ ದ್ವಿತೀಯೇ ತು ತೃತೀಯೇ ತು ಪಯಸ್ತಥಾ~।}\\\textbf{ಚತುರ್ಥೇ ಬಹುಬೀಜೇನ ದ್ವಿದಲಾನಿ ವಿವರ್ಜಿತಾಃ~।। ೫೬~।। }
\end{verse}

\begin{verse}
\textbf{ಜಯಂತಿ ನಿರತಾ ಯೇ ಚ ನ ತೇ ಯಾಂತಿ ಯಮಾಲಯಮ್~।}
\end{verse}

ಲೋಕದಲ್ಲಿ ಮಂಚವನ್ನು ದಾನ ಮಾಡುವವರು, ತುಳಸೀ ಮೃತ್ತಿಕೆಯನ್ನು ಮೈಯಲ್ಲಿ ಲೇಪಿಸಿಕೊಂಡು ಸ್ನಾನ ಮಾಡುವವರು, ತುಳಸೀ ಮೃತ್ತಿಕೆಯಿಂದ ಹಣೆಯಮೇಲೆ ತಿಲಕವನ್ನು ಧಾರಣೆ ಮಾಡುವವರು, ಚಾತುರ್ಮಾಸ ಬಂದಾಗ ಮೊದಲನೇ ತಿಂಗಳಾದ ಆಷಾಢದಲ್ಲಿ ತರಕಾರಿಗಳನ್ನು ಪರಿತ್ಯಾಗ ಮಾಡುವವರು, ಎರಡನೇ ತಿಂಗಳಿನಲ್ಲಿ ಮೊಸರನ್ನು ಬಿಟ್ಟಿರುವವರು, ಮೂರನೆಯ ತಿಂಗಳಿನಲ್ಲಿ ಹಾಲನ್ನು ಉಪಯೋಗಿಸದೇ ಇರುವವರು, ನಾಲ್ಕನೆಯ ತಿಂಗಳಿನಲ್ಲಿ ಬೀಜಯುಕ್ತವಾದ ಶಾಕಗಳನ್ನೂ ಹಾಗೂ ದ್ವಿದಲ ಧಾನ್ಯಗಳನ್ನೂ ಪರಿತ್ಯಾಗ ಮಾಡಿದವರು, ನರಸಿಂಹ ಜಯಂತಿ, ಕೃಷ್ಣ ಜಯಂತಿ ಮುಂತಾದ ಪರಮಾತ್ಮನ ಅವತಾರ ಜಯಂತಿಗಳನ್ನು ಭಕ್ತಿ ಪುರಸ್ಸರವಾಗಿ ಆಚರಿಸುವವರೂ ಯಮಲೋಕಕ್ಕೆ ಹೋಗುವುದಿಲ್ಲ.

\begin{verse}
\textbf{ಏಕಾದಶ್ಯಾಂ ಪ್ರಕುರ್ವಂತಿ ಜಾಗರಂ ಕೇಶವಾಲಯೇ~।। ೫೭~।। }
\end{verse}

\begin{verse}
\textbf{ಯೇ ಚ ರಾಮಕಥಾಸಕ್ತಾ ನ ತೇ ಯಾಂತಿ ಯಮಾಲಯಮ್~।}\\\textbf{ಯೇ ಗುಪ್ತದಾಯಿನೋ ಲೋಕೇ ಗ್ರೀಷ್ಮೇ ಚ ಜಲದಾಯಿನಃ~।। ೫೮~।। }
\end{verse}

\begin{verse}
\textbf{ಶೀತೇ ಕಂಬಲದಾತಾರೋ ನ ತೇ ಯಾಂತಿ ಯಮಾಲಯಮ್~।}\\\textbf{ತುಲಸೀಮಣಿಮಾಲಾಂ ತು ಯೇ ಬಿಭ್ರಂತಿ ಗಲೇ ಮುದಾ~।। ೫೯~।। }
\end{verse}

\begin{verse}
\textbf{ಪದ್ಮಾ ಕ್ಷಮಾಲಾಂ ಧಾತ್ರೀಂ ವಾ ನ ತೇ ಯಾಂತಿ ಯಮಾಲಯಮ್~।}\\\textbf{ಇತ್ಯಾದ್ಯಾ ಬಹವೋ ಧರ್ಮಾ ಅಸ್ಮಾಭಿಸ್ತು ತವೋದಿತಾಃ~।। ೬೦~।।}
\end{verse}

ಏಕಾದಶೀ ರಾತ್ರಿಯಲ್ಲಿ ವಿಷ್ಣು ಮಂದಿರದಲ್ಲಿ ಜಾಗರಣೆ ಮಾಡುವವರು, ಶ‍್ರೀರಾಮನ ಕಥಾಲಾಪನೆಯಲ್ಲಿ ಆಸಕ್ತರಾಗಿರುವವರು, ಗುಪ್ತದಾನ ಮಾಡುವವರು, ಬೇಸಿಗೆಯಲ್ಲಿ ನೀರನ್ನು ದಾನ ಮಾಡುವವರು, ಚಳಿಗಾಲದಲ್ಲಿ ಸತ್ಪಾತ್ರರಿಗೆ ಕಂಬಳಿ ದಾನ ಮಾಡುವವರು, ತುಳಸೀಮಣಿಯ ಹಾರವನ್ನು ಭಕ್ತಿಯಿಂದ ಕಂಠದಲ್ಲಿ ಧಾರಣೆ ಮಾಡುವವರು, ಪದ್ಮಾ ಕ್ಷಮಾಲೆ ಹಾಗೂ ನೆಲ್ಲಿಕಾಯಿಯ ಹಾರವನ್ನು ಕಂಠದಲ್ಲಿ ಧಾರಣೆ ಮಾಡುವವರು ಯಮಲೋಕಕ್ಕೆ ಹೋಗುವುದಿಲ್ಲ. ಈ ರೀತಿಯಲ್ಲಿ ಅನೇಕ ಬಗೆಯ ಧರ್ಮಗಳು ನಮ್ಮಿಂದ ವಿವರಿಸಲ್ಪಟ್ಟವು.

\begin{verse}
\textbf{ನ ತೇಷ್ವಯಂ ಕಿಂಚಿದಪಿ ಕೃತವಾನ್ ಭದ್ರಸಂಜ್ಞ ಕಃ~।}\\\textbf{ಕಥಂ ಚಾಯಂ ಭವದ್ಭಿ ಸ್ತು ನೇತುಮರ್ಹತಿ ಪಾವಧೀಃ~।। ೬೧~।। }
\end{verse}

\begin{verse}
\textbf{ನಯಾಮೈನಂ ವಯಂ ಮೂಢಂ ಭವತ್ತೇಜೋಹತಾ ವಯಮ್~।}\\\textbf{ಇತಿ ತೇಷಾಂ ವಚಃ ಶ್ರುತ್ವಾ ವಿಷ್ಣು ದೂತಾ ಅಥಾಭ್ರುವನ್~।। ೬೨~।।}
\end{verse}

ಇಂತಹ ಧರ್ಮಗಳಲ್ಲಿ ಯಾವುದನ್ನೂ ಈ ಭದ್ರನು ಆಚರಿಸಲಿಲ್ಲ. ಹೀಗಿರುವಾಗ ಅವನು ನಿಮ್ಮಿಂದ ಕರೆದುಕೊಂಡು ಹೋಗಲ್ಪಡಲು ಅರ್ಹನಲ್ಲ; ಅವನು ಮಹಾ ಪಾಪಿಷ್ಟನು. ನಿಮ್ಮ ತೇಜಃಪ್ರಭಾವದಿಂದ (ದೆಸೆಯಿಂದ) ಈ ಮೂಢನನ್ನು ನಾವೂ ಸಹ ಕರೆದೊಯ್ಯಲು ಶಕ್ತಿಯುತರಾಗಿಲ್ಲ. ಹೀಗೆಂಬ ಯಮದೂತರ ಮಾತುಗಳನ್ನು ಕೇಳಿದ ವಿಷ್ಣು ದೂತರು ಹೇಳಿದರು:

\begin{verse}
\textbf{ವಿಸ್ಮಯಂ ವಾಥ ಕಿಮಥ ನೋಕ್ತವಾನ್ ಧರ್ಮರಾಡಿಮಮ್~।}\\\textbf{ಅಯಂ ಭದ್ರಃ ಕೃತಃ ಸ್ನಾನೋ ಮಾಘೇ ಮಾಸಿ ದಿನತ್ರಯಮ್~।। ೬೩~।। }
\end{verse}

\begin{verse}
\textbf{ಮಾಘಸ್ನಾನಂ ತು ಪಾಪಾನಾಂ ಪ್ರಾಯಶ್ಚಿತ್ತಂ ಪರಂ ಸ್ಮೃತಮ್~।}\\\textbf{ನ ಚಾಸ್ಮಿನ್ ಪಾಪಲೇಶೋsಸ್ತಿ ನಾಯಮರ್ಹತಿ ಯಾತನಾಮ್~।। ೬೪~।। }
\end{verse}

\begin{verse}
\textbf{ತಸ್ಮಾದೇನಂ ವಿಮುಚ್ಯೈವ ಯೂಯಂ ಗಚ್ಛತ ಸತ್ವರಮ್~।}\\\textbf{ಇತಿ ತೈರುದಿತೇ ಭೀತಾ ಯಮದೂತಾ ವಿಹಾಯ ತಮ್~।। ೬೫~।।}
\end{verse}

ಯಮಧರ್ಮರಾಜರು ನಿಮಗೆ ಈ ಧರ್ಮವನ್ನು ಹೇಳದೆ ಇರುವುದು ಆಶ್ಚರ್ಯವಾಗಿದೆ. ಈ ಭದ್ರನು ಮಾಘಮಾಸದಲ್ಲಿ ಮೂರು ದಿನಗಳ ಕಾಲ ಸಕಾಲದಲ್ಲಿ ಸ್ನಾನಮಾಡಿರುತ್ತಾನೆ. ಮಾಘಸ್ನಾನವು ಪಾಪಗಳಿಗೆ ಪ್ರಾಯಶ್ಚಿತ್ತವೆಂದು ಹೇಳಲ್ಪಟ್ಟಿದೆ. ಭದ್ರನಿಗೆ ಈಗ ಯಾವ ಪಾಪದ ಸಂಪರ್ಕವೂ ಇಲ್ಲ. ಯಮಲೋಕದ ಹಿಂಸೆಗಳಿಗೆ ಅವನು ಅರ್ಹನಲ್ಲ. ಆದುದರಿಂದ ಅವನನ್ನು ಬಿಟ್ಟು ನೀವು ಹೊರಟುಹೋಗಿರಿ, ವಿಷ್ಣು ದೂತರು ಹೀಗೆ ಹೇಳಿದುದನ್ನು ಕೇಳಿದ ಯಮಭಟರು ಭದ್ರನನ್ನು ಅಲ್ಲಿಯೇ ಬಿಟ್ಟು,

\begin{verse}
\textbf{ಭದ್ರಂ ಸುಭದ್ರಂ ವ್ಯಾಘ್ರಂ ಚ ದ್ವೌ ಶಿಷ್ಟೌ ನೇತುಮುದ್ಯತಾಃ~।}\\\textbf{ತೇ ವಿಷ್ಣು ದೂತಾಸ್ತಂ ಭದ್ರಮಭಿರೋಪ್ಯ ಚ ಪುಷ್ಕರೇ~।। ೬೬~।। }
\end{verse}

\begin{verse}
\textbf{ಆದಾಯ ಗಂತುಮುದ್ಯುಕ್ತಾ ಭದ್ರೋ ವೈ ವಾಕ್ಯ ಮಬ್ರವೀತ್~।}\\\textbf{ಭ್ರಾತರೌ ತುಲ್ಯ ಪಾಪ್ಮಾನೌ ವ್ಯಾಘ್ರೋsಯಂ ಪ್ರಾಣಿಹಿಂಸಕಃ~।। ೬೭~।।} 
\end{verse}

\begin{verse}
\textbf{ತನ್ಮಧ್ಯೇ ಸುಗತಿರ್ಮೇದ್ಯ ತಯೊರ್ವಾ ದುರ್ಗತಿ ಕಥಮ್~।}\\\textbf{ವೈಷಮ್ಯಂ ಭವತಾಂ ವಾಪಿ ತದೇತದ್ವಕ್ತುಮರ್ಹಥ~।। ೬೮~।।}
\end{verse}

ಸುಭದ್ರನನ್ನೂ, ವ್ಯಾಘ್ರನನ್ನೂ ಕರೆದುಕೊಂಡು ಹೋಗಲು ಸಿದ್ಧರಾದರು. ವಿಷ್ಣುದೂತರು ಭದ್ರನನ್ನು ಪುಷ್ಪಕ ವಿಮಾನದಲ್ಲಿ ಕೂಡಿಸಿಕೊಂಡು ಹೊರಡಲು ಸಿದ್ಧರಾದರು. ಆಗ ಭದ್ರನು ಹೀಗೆ ಪ್ರಶ್ನೆ ಮಾಡಿದನು: “ನಾನೂ ನನ್ನ ಸಹೋದರನಾದ ಸುಭದ್ರನೂ ಒಂದೇ ಸಮನಾಗಿ ಪಾಪ ಮಾಡಿರುತ್ತೇವೆ. ಹುಲಿಯಂತೂ ಪ್ರಾಣಿಹಿಂಸಕ. ಹೀಗಿದ್ದೂ ನನಗೆ ಮಾತ್ರ ಸದ್ಗತಿಯಾ, ಅವರಿಬ್ಬರಿಗೆ ದುರ್ಗತಿಯೂ ಪ್ರಾಪ್ತವಾಗಲು ಕಾರಣವೇನು? ಅಥವಾ ನಿಮ್ಮಲ್ಲಿಯೇ ವೈಷಮ್ಯದಿಂದ ಕೂಡಿದ ಬೇರೆ ಬೇರೆ ಭಾವನೆಗಳಿವೆಯೋ? ಇದನ್ನು ನನಗೆ ವಿವರಿಸಿರಿ.

\begin{verse}
\textbf{ನಾಹಮೇತತ್ಸುಖಂ ಮನ್ನೇ ಸಹಾಯೌ ತೌ ವಿಹಾಯ ಚ~।}\\\textbf{ತಸ್ಮಾತ್ಸಹಾಯೌ ನಯಥ ಮಾಂ ವಿಮುಚ್ಯಾಥ ಗಚ್ಛಥ~।। ೬೯~।।}
\end{verse}

ಅವರಿಬ್ಬರನ್ನೂ ಬಿಟ್ಟು ನಾನೊಬ್ಬನೇ ಸುಖವನ್ನು ಅನುಭವಿಸಲಾರೆ. ಅವರಿಬ್ಬರನ್ನೂ ನನ್ನೊಡನೆ ಕರೆದುಕೊಂಡು ಬನ್ನಿರಿ; ಅಥವಾ ನನ್ನನ್ನಾದರೂ ಬಿಟ್ಟು ಬಿಡಿರಿ.”

\begin{verse}
\textbf{ಇತ್ಯುಕ್ತಾಸ್ತೇ ವಿಷ್ಣು ದೂತಾ ಭದ್ರಂ ವಾಕ್ಯಮಥಾಬ್ರುವನ್~।}\\\textbf{ಶೃಣು ಭದ್ರ ಪ್ರವಕ್ಷ್ಯಾಮೋ ಲೋಕಾ ವೈ ತ್ರಿವಿಧಾ ಮತಾಃ~।। ೭೦~।। }
\end{verse}

\begin{verse}
\textbf{ಅಧೋ ಮಧ್ಯೋರ್ಧ್ವಸಂಸ್ಥಾನಾಸ್ತೇಧೋಲೋಕಾಸ್ತು ನಾರಕಾಃ~।}\\\textbf{ಮಧ್ಯಲೋಕಾಸ್ತು ಸ್ವರ್ಗಾದ್ಯಾ ಶ್ಚೋರ್ಧ್ವಂ ಸತ್ಯಾದಯೋ ಮತಾಃ~।। ೭೧~।।}
\end{verse}

ಭದ್ರನು ಹೀಗೆನ್ನಲು ವಿಷ್ಣು ದೂತರು ಹೇಳಿದರು: “ಹೇಳುತ್ತೇವೆ, ಕೇಳು; ಲೋಕದಲ್ಲಿ ಅಧೋಲೋಕ, ಮಧ್ಯಲೋಕ, ಊರ್ಧ್ವಲೋಕ ಎಂಬುದಾಗಿ ಮೂರು ಭಾಗಗಳಿವೆ. ಅಧೋಲೋಕವೆಂಬುದು ನರಕ, ಮಧ್ಯಲೋಕವೆಂಬುದು ಸ್ವರ್ಗಾದಿಗಳು, ಊರ್ಧ್ವಲೋಕವೆಂದರೆ ಸತ್ಯಲೋಕವೇ ಮೊದಲಾದುವು.

\begin{verse}
\textbf{ಸರ್ವೇ ಲೋಕಾಃ ಕರ್ಮಚಿತಾಸ್ತತ್ಕರ್ಮ ತ್ರಿವಿಧಂ ಮತಮ್~।}\\\textbf{ಪಾಪಂ ಚ ಕಾಮ್ಯಂ ಯತ್ಕರ್ಮ ನಿಷ್ಕಾಮೋ ಧರ್ಮ ಏವ ಚ~।। ೭೨~।।} 
\end{verse}

\begin{verse}
\textbf{ಪಾಪೇನ ನಿರಯಾ ಲೋಕಾಃ ಕಾಮ್ಯೇನ ಚ ಸ್ವರಾದಯಃ~।}\\\textbf{ನಿಷ್ಕಾಮೇನ ತು ಸತ್ವಾದ್ಯಾ ಇತಿ ವೇದವಿದಾಂ ಮತಮ್~।। ೭೩~।।}
\end{verse}

ಸಮಸ್ತ ಜೀವರೂ ಕರ್ಮದಿಂದ ಬದ್ಧರಾಗಿರುತ್ತಾರೆ. ಆ ಕರ್ಮಗಳು ಮೂರು ಬಗೆ-ಪಾಪ, ಕಾಮ್ಯ ಕರ್ಮ, ನಿಷ್ಕಾಮ್ಯ ಕರ್ಮ, ನಿಷ್ಕಾಮ್ಯಕರ್ಮವು ಶ್ರೇಷ್ಠ. ಪಾಪಾಚರಣೆಯಿಂದ ಜನರು ನರಕಕ್ಕೆ ಹೋಗುತ್ತಾರೆ. ಕಾಮ್ಯ ಕರ್ಮಗಳಲ್ಲಿ ನಿರತರಾದವರು ಸ್ವರ್ಗಾದಿ ಲೋಕಗಳಿಗೆ ಹೋಗುತ್ತಾರೆ. ನಿಷ್ಕಾಮ್ಯಕರ್ಮದಿಂದ ಸತ್ಯಲೋಕವೇ ಮೊದಲಾದ ಉತ್ತಮ ಲೋಕಗಳಿಗೆ ಹೋಗುತ್ತಾರೆ; ಹೀಗೆಂಬುದು ವೇದಗಳ ಸಾರ.

\begin{verse}
\textbf{ತ್ವಯಾ ತತ್ರಾರ್ಜಿತಂ ಪುಣ್ಯಂ ಬ್ರಾಹ್ಮಣಾನಾಂ ಸಮಾಗಮಾತ್~।}\\\textbf{ದಿನತ್ರಯಂ ಮಾಘಮಾಸೇ ಸ್ನಾನಂ ತೇನೇದೃಶೀ ಗತಿಃ~।। ೭೪~।।}
\end{verse}

ನೀನು ಬ್ರಾಹ್ಮಣರ ಸಹವಾಸದಿಂದ ಮೂರು ದಿನಗಳಲ್ಲಿ ಮಾಘಸ್ನಾನವನ್ನು ಮಾಡಿರುತ್ತಿ. ಅದರಿಂದ ಬಂದಿರುವ ಪುಣ್ಯದಿಂದ ನಿನಗೆ ಇಂತಹ ಸದ್ಧತಿಯು ಲಭಿಸಿದೆ.

\begin{verse}
\textbf{ಪಾಪಮೇವಾರ್ಜಿತಂ ತಾಭ್ಯಾಂ ನ ಪುಣ್ಯಂ ಲೇಶಮಾತ್ರಕಮ್~।}\\\textbf{ತತೋ ಗತಿಸ್ತಯೋರ್ಭಿನ್ನಾ ತವ ಯೇsಯಂ ಸಮಾಗತಾ~।। ೭೫~।। }
\end{verse}

\begin{verse}
\textbf{ಅತೋ ಭದ್ರ ಸುಖಂ ಗಚ್ಛ ತೌ ಗಚ್ಛೇತಾಂ ಚ ಪಾಪಿನೌ~।}\\\textbf{ಇತ್ಯುಕ್ತೋ ದೇವದೂತೈಸ್ತು ಭದ್ರೋ ದೈನ್ಯಾದಥಾಬ್ರವೀತ್~।। ೭೬~।।}
\end{verse}

ಅವರಿಬ್ಬರೂ ಪಾಪವನ್ನೇ ಆಚರಿಸಿರುತ್ತಾರೆ. ಲೇಶಮಾತ್ರ ಪುಣ್ಯವನ್ನೂ ಮಾಡಿಲ್ಲ. ಆದುದರಿಂದ ನಿನಗೆ ಅವರಿಗೆ ಲಭಿಸಿರುವ ಗತಿಗಿಂತ ಭಿನ್ನವಾದ ಗತಿ ಲಭಿಸಿದೆ. ಭದ್ರನೇ ನೀನು ಸುಖದ ಲೋಕಕ್ಕೆ ತೆರಳು, ಆ ಪಾಪಿಗಳಿಬ್ಬರೂ ಬೇರೆ ಹೋಗಲಿ, ಹೀಗೆ ವಿಷ್ಣು ದೂತರು ಹೇಳಿದಮೇಲೆ ಭದ್ರನು ಅವರನ್ನು ಕುರಿತು ಹೀಗೆ ಮಾತನಾಡಿದನು:

\begin{verse}
\textbf{ಸಾಧೂನಾಂ ಸಮಚಿತ್ತಾನಾಂ ಯೇ ಜನಾಶ್ಚ ಪುರೋಗತಾಃ~।}\\\textbf{ನ ತೇಷಾಂ ದುರ್ಗತಿಃ ಕ್ವಾಪಿ ತಸ್ಮಾದೇತೌ ವಿಮುಂಚಥ~।। ೭೭~।।} 
\end{verse}

\begin{verse}
\textbf{ಇತ್ಯುಕ್ತಾಸ್ತೇ ತು ಭದ್ರೇಣ ವಿಷ್ಣು ದೂತಾಸ್ತಮಬ್ರುವನ್~।}\\\textbf{ದಯಾ ಚೇದಸ್ತಿ ತೇ ಪುಣ್ಯಂ ದಿನದ್ವಯಸಮುದ್ಭವಮ್~।। ೭೮~।। }
\end{verse}

\begin{verse}
\textbf{ಮಾಘಸ್ನಾನಾತ್ತಯೋರ್ದೇಹಿ ತಯೋರ್ಮುಕ್ತಿರ್ಭವಿಷ್ಯತಿ~।}
\end{verse}

ಸುಖ-ದುಃಖಗಳು, ಐಶ್ವರ್ಯಾದಿಗಳು, ಶತ್ರುಗಳು-ಮಿತ್ರರು, ದಾರಿದ್ರ್ಯ ಮುಂತಾದುವುಗಳು ಬಂದಾಗ ಅಥವ ಹೋದಾಗ ಒಂದೇ ವಿಧವಾದ ಮನಸ್ಸುಳ್ಳ ಸಾಧು ಜನರ ಸಮಾಗಮವಾದರೆ ಅವರಿಗೆ ಎಂದಿಗೂ ದುರ್ಗತಿ ಬರಬಾರದು; ಆದುದರಿಂದ ಸುಭದ್ರನನ್ನೂ, ವ್ಯಾಘ್ರನನ್ನೂ ಕೃಪೆಯಿಂದ ನೋಡಿ ಬಿಡಿಸಿರಿ ಎಂದನು. ಭದ್ರನ ಈ ಮಾತನ್ನು ಕೇಳಿ ಎಷ್ಟು ದೂತರು ಹೇಳಿದರು. “ನಿನಗೆ ಅವರಿಬ್ಬರ ಮೇಲೆ ಅನುತಾಪ ಇರುವುದಾದರೆ, ನಿನ್ನ ಮಾಘ\break ಸ್ನಾನದ ಪುಣ್ಯದಲ್ಲಿ ಎರಡು ದಿನಗಳ ಪುಣ್ಯವನ್ನು ಅವರಿಬ್ಬರಿಗೂ ಕೊಡು; ಆಗ ಅವರಿಬ್ಬರು ಬಂಧನದಿಂದ ಮುಕ್ತರಾಗುತ್ತಾರೆ.

\begin{verse}
\textbf{ಇತಿ ತೇಷಾಂ ವಚಃ ಶ್ರುತ್ವಾ ಭದ್ರಃ ಪುಣ್ಯಂ ದದೌ ತಯೋಃ~।। ೭೯~।।} 
\end{verse}

\begin{verse}
\textbf{ತೇನ ಪುಣ್ಯ ಪ್ರಭಾವೇನ ತೌ ಮಕ್ತೌ ಯಮಪಾಶತಃ~।}\\\textbf{ತಯೋರ್ವಿಮಾನೇ ಸಂಪ್ರಾಪ್ತೇ ಉದ್ಯ ತಾದಿತ್ಯ ಸನ್ನಿಭೇ~।। ೮೦~।। }
\end{verse}

\begin{verse}
\textbf{ದಯಾಲೂನಾಂ ಸ್ವಭಾವೋ ಹಿ ಯಾ ಪ್ರಹರ್ತರಿ ಸತ್ಕ್ರಿಯಾ~।}\\\textbf{ಶಾರ್ದೂಲಸ್ಯ ತತೋ ಭದ್ರೋ ದದೌ ಪುಣ್ಯ ಮನುತ್ತಮಮ್~।। ೮೧~।।}
\end{verse}

ವಿಷ್ಣು ದೂತರ ಈ ಮಾತನ್ನು ಕೇಳಿ ಭದ್ರನು ಅವರಿಗೆ ಮಾಘಸ್ನಾನದ ಪುಣ್ಯವನ್ನು ಅರ್ಪಿಸಿದನು. ಆ ಪ್ರಭಾವದಿಂದ ಅವರಿಬ್ಬರೂ ಯಮಪಾಶದಿಂದ ಮುಕ್ತರಾದರು. ಉದಯಕಾಲದ ಸೂರ್ಯನ ತೇಜಸ್ಸಿಗೆ ಸಮಾನವಾದ ಪ್ರಕಾಶದಿಂದ ಕೂಡಿದ ವಿಮಾನವು ಪ್ರಾಪ್ತವಾಯಿತು. ಅಪಕಾರ ಮಾಡಿದವರಿಗೂ ಸಹ ಉಪಕಾರ ಮಾಡುವುದೇ ದಯಾವಂತರ ಸ್ವಭಾವ. ಅದರಂತೆ ಭದ್ರನು ವ್ಯಾಘ್ರನಿಗೂ ಸಹ ಪುಣ್ಯವನ್ನು ದಾನಮಾಡಿದನು.

\begin{verse}
\textbf{ವಿಷ್ಣುದೂತೈಸ್ತಥಾಜ್ಞಪ್ತಾಸ್ತ್ರಯಸ್ತೇ ಕಿಲ್ಬಿಷಾಂ ಯಯುಃ~।}\\\textbf{ಮಾಘಸ್ನಾನಪ್ರಭಾವೇನ ತ್ರೇತಾಗ್ನಯ ಇವಾಪರಾಃ~।। ೮೨~।।}
\end{verse}

ವಿಷ್ಣು ದೂತರಿಂದ ಆಜ್ಞೆ ಪಡೆದ ಆ ಮೂರು ಮಂದಿಯೂ ಮಾಘಸ್ನಾನದ ಪುಣ್ಯ ಪ್ರಭಾವದಿಂದ ಪಾಪಗಳಿಂದ ಮುಕ್ತರಾಗಿ ದಕ್ಷಿಣ, ಗಾರ್ಹಪತ್ಯ, ಆಹವನೀಯಯೆಂಬ ಮೂರು ಅಗ್ನಿಗಳಂತೆ ಪ್ರಕಾಶಿಸಿದರು.

\begin{verse}
\textbf{ಯೂಯಂ ಗಚ್ಛತ ಸ್ವರ್ಲೋಕಂ ನ ಯಾಮ ಇತಿ ತೇಽಬ್ರುವನ್~।}\\\textbf{ಸ್ವರ್ಗೋ ನ ಹ್ಯುತ್ತಮೋ ಲೋಕೋ ನ ಗಚ್ಛಾಮೋ ವಯಂ ತತಃ~।। ೮೩~।।}
\end{verse}

ವಿಷ್ಣು ದೂತರು ಅವರಿಗೆ “ನೀವು ಸ್ವರ್ಗಲೋಕಕ್ಕೆ ಹೋಗಿರಿ” ಎಂಬುದಾಗಿ ಹೇಳಲು ಅವರು ಸ್ವರ್ಗಲೋಕವು ಉತ್ತಮವಾದ ಲೋಕವಲ್ಲ; ಆದುದರಿಂದ ನಾವು ಅಲ್ಲಿಗೆ ಹೋಗುವುದಿಲ್ಲ” ಎಂದರು.

\begin{verse}
\textbf{ಯುಷ್ಮತ್ಪುಣ್ಯ ಪ್ರಭಾವಾಚ್ಚ ಭೂತ್ವಾ ಚ ದ್ವಿಜಜನ್ಮಸು~।}\\\textbf{ತತ್ರ ಮಾಘೇ ಪುನಃ ಕೃತ್ವಾ ಸ್ನಾನದಾನಾದಿಕಾಃ ಕ್ರಿಯಾಃ~।। ೮೪~।।} 
\end{verse}

\begin{verse}
\textbf{ತೇನ ಪುಣ್ಯ ಪ್ರಭಾವೇನ ವಿಷ್ಣೋರ್ಲೋಕಮಥಾಪ್ನುಯಃ~।}\\\textbf{ಇತಿ ವಿಜ್ಞಾಪಿತಾಸ್ತೈಸ್ತು ದತ್ವಾ ವಿಪ್ರೇಷು ಜನ್ಮ ಚ~।। ೮೫~।। }
\end{verse}

\begin{verse}
\textbf{ಧರ್ಮೇ ಗತಿಂ ತಥಾ ಪುಣ್ಯಂ ವಿಷ್ಣು ಭಕ್ತಿಮಹೈತುಕೀಮ್~।}\\\textbf{ವಿಷ್ಣು ದೂತಾ ಯಯುರ್ಲೋಕಂ ವಿಷ್ಣೋರೇವ ಮಹಾತ್ಮನಃ~।। ೮೬~।।}
\end{verse}

“ಈಗಿರುವ ಪುಣ್ಯ ಪ್ರಭಾವದಿಂದ ನಾವು ಮುಂದಿನ ಜನ್ಮದಲ್ಲಿ ಬಾಹ್ಮಣರಾಗಿ ಹುಟ್ಟಿ ಮಾಘಮಾಸದಲ್ಲಿ ಸ್ನಾನದಾನಾದಿ ಸತ್ಕರ್ಮಗಳನ್ನಾಚರಿಸಿ ಅದರ ಪ್ರಭಾವದಿಂದ ನಾವು ವಿಷ್ಣು ಲೋಕವನ್ನು ಹೊಂದುವೆವು” ಎಂದರು. ಹೀಗೆ ಪ್ರಾರ್ಥನೆ ಮಾಡಿದ ಆ ಮೂವರಿಗೆ ವಿಷ್ಣು ದೂತರು ಅದರಂತೆ ಕರುಣಿಸಿದರು. ಆ ಮೂವರೂ ಬ್ರಾಹ್ಮಣ ಕುಲದಲ್ಲಿ ಉತ್ಪನ್ನರಾಗಿ ಧರ್ಮಬುದ್ಧಿಯಿಂದ ಕರ್ಮಗಳನ್ನಾಚರಿಸಿ ವಿಷ್ಣುವಿನಲ್ಲಿ ಮಾಹಾತ್ಮಜ್ಞಾನಪೂರ್ವಕವಾದ ಶ್ರೇಷ್ಠವಾದ ಭಕ್ತಿಯನ್ನು ಸಂಪಾದಿಸಿ ವಿಷ್ಣು ಲೋಕಕ್ಕೆ ಹೋಗುವಂತೆ ಅನುಗ್ರಹ ಮಾಡಿದರು. ವಿಷ್ಣು ದೂತರು ತಮ್ಮ ಲೋಕಕ್ಕೆ ತೆರಳಿದರು.

\begin{verse}
\textbf{ತೇಷಾಂ ಪ್ರಸಾದಾನ್ಮಾಘಸ್ಯ ಸ್ನಾನದಾನಫಲಾದಪಿ~।}\\\textbf{ಚರ್ತುರ್ಮಾಘೇಷು ತೇ ಸ್ನಾನಂ ಭುವಿ ವೈ ದ್ವಿಜಜನ್ಮಸು~।। ೮೭~।। }
\end{verse}

\begin{verse}
\textbf{ತತೋ ಜ್ಞಾನಿತ್ವಮಾಸಾದ್ಯ ಮೋಕ್ಷಮಾಪುಸ್ತ್ರಯೋಽಪಿ ಚ~।}\\\textbf{ಇತಿಹಾಸಮಿಮಂ ಪುಣ್ಯಂ ಪುತ್ರ ತ್ಯಾತ್ತೇ ಸಮೀರಿತಮ್~।। ೮೮~।। }
\end{verse}

\begin{verse}
\textbf{ನ ಕಸ್ಯಾಪಿ ಸಮಾಖ್ಯೇಯಂ ಗೋಪ್ಯಂ ಭಾಗವತೋತ್ತಮಮ್~।}\\\textbf{ಯ ಇದಂ ಶೃಣುಯಾನ್ನಿತ್ಯಂ ಮಾಘಸ್ನಾನಪರಾಯಣಃ~।। ೮೯~।। }
\end{verse}

\begin{verse}
\textbf{ಸ ಕರ್ಮಬಂಧಮುಕ್ತಃ ಸನ್ ವಿಷ್ಣು ಲೋಕೇ ಮಹೀಯತೇ~।। ೯೦~।।}
\end{verse}

ವಿಷ್ಣು ದೂತರ ಅನುಗ್ರಹದಿಂದ ಭದ್ರ, ಸುಭದ್ರ, ವ್ಯಾಘ್ರ ಈ ಮೂವರೂ ಬ್ರಾಹ್ಮಣಕುಲದಲ್ಲಿ ಹುಟ್ಟಿ, ನಾಲ್ಕು ಮಾಘಸ್ನಾನಗಳನ್ನು ಮಾಡಿ, ಒಳ್ಳೆಯ ಯಥಾರ್ಥಜ್ಞಾನವನ್ನು ಪಡೆದು ಮೋಕ್ಷವನ್ನು ಹೊಂದಿದರು. ನಾರದನೇ, ನೀನು ನನ್ನ ಮಗನಾದ ಪ್ರಯುಕ್ತ ಈ ಇತಿಹಾಸವನ್ನು ಹೇಳಿದೆ. ಈ ಇತಿಹಾಸವು ರಹಸ್ಯ, ಅತ್ಯಂತ ಶ್ರೇಷ್ಠ, ಎಲ್ಲರಿಗೂ ಹೇಳುವಂತಹುದಲ್ಲ. ಮಾಘಸ್ನಾನಪರಾಯಣನಾದವನು ಇದನ್ನು ನಿತ್ಯದಲ್ಲಿ ಶ್ರವಣ ಮಾಡುವುದರಿಂದ ಸಕಲ ಕರ್ಮಬಂಧನದಿಂದ ಬಿಡುಗಡೆ ಹೊಂದಿ ವಿಷ್ಣುಲೋಕದಲ್ಲಿ ಸಂತೋಷದಿಂದ ಇರುವನು.

\begin{center}
ಇತಿ ಶ‍್ರೀ ವಾಯುಪುರಾಣೇ ಮಾಘಮಾಸಮಾಹಾತ್ಮ್ಯೇ ಅಷ್ಟಮೋsಧ್ಯಾಯಃ
\end{center}

\begin{center}
 ಶ‍್ರೀ ವಾಯುಪುರಾಣಾಂತರ್ಗತ ಮಾಘಮಾಸ ಮಾಹಾತ್ಮ್ಯೆಯಲ್ಲಿ \\ ಎಂಟನೇ ಅಧ್ಯಾಯವು ಸಮಾಪ್ತಿಯಾಯಿತು.
\end{center}

\newpage

\section*{ಅಧ್ಯಾಯ\enginline{-}೯}

\emptypage

\begin{flushleft}
\textbf{ಬ್ರಹ್ಮೋವಾಚ:\enginline{-} }
\end{flushleft}

\begin{verse}
\textbf{ಸ್ನಾತ್ವಾ ಮಾಘೇ ಮಘಾಯುಕ್ತ ಪೌರ್ಣಮ್ಯಾಂ ವಿಷ್ಣು ತತ್ಪರಃ~।}\\\textbf{ಗುರುನಿಂದಾಕೃತಾತ್ ಪಾಪಾನ್ಮುಚ್ಯತೇ ಪರಿಧಿರ್ಯಥಾ-।। ೧~।।}
\end{verse}

\begin{flushleft}
ಬ್ರಹ್ಮದೇವರು ಹೇಳುತ್ತಾರೆ -
\end{flushleft}

ಮಾಘ ಮಾಸದಲ್ಲಿ ಮಘ ನಕ್ಷತ್ರದಿಂದ ಯುಕ್ತವಾದ ಪೌರ್ಣಿಮೆಯಲ್ಲಿ ವಿಷ್ಣು ಭಕ್ತನು ಸ್ನಾನಮಾಡಿದರೆ ಹಿಂದೆ ಪರಿಧಿಯೆಂಬುವನಂತೆ ಗುರುನಿಂದೆಯಿಂದ ಬಂದ ಪಾಪದಿಂದ ಮುಕ್ತಿ ಪಡೆಯುತ್ತಾನೆ. 

\noindent
ನಾರದ ಉವಾಚ:-

\begin{verse}
\textbf{ಕೋ ವಾ ಪರಿಧಿನಾಮಾಸೌ ಕಸ್ಯಾಸೌ ಶಿಷ್ಯತಾಂ ಗತಃ~।}\\\textbf{ಅನಿಂದಯತ್ ಗುರುಃ ಕಸ್ಮಾದೇತದ್ವಿಸ್ತಾರ್ಯ ಮೇ ವದ~।। ೨~।। }
\end{verse}

\begin{flushleft}
ನಾರದರು ಹೇಳುತ್ತಾರೆ\enginline{-}
\end{flushleft}

ಬ್ರಹ್ಮ ದೇವರೇ, ಪರಿಧಿ ಎಂಬುವನು ಯಾರು, ಯಾರ ಶಿಷ್ಯ, ಗುರುವನ್ನು ಯಾವ\break ಕಾರಣದಿಂದ ನಿಂದಿಸಿದ ಎಂಬ ವೃತ್ತಾಂತವನ್ನು ವಿಸ್ತಾರವಾಗಿ ಹೇಳಿರಿ.

\begin{flushleft}
\textbf{ಬ್ರಹ್ಮೋವಾಚ:\enginline{-} }
\end{flushleft}

\begin{verse}
\textbf{ಕೋಸಲೇ ಸರಯೂತೀರೇ ಪ್ರಗಾಥೋ ನಾಮ ವೈ ಋಷಿಃ~।}\\\textbf{ಶಾಂತೋ ದಾಂತಸ್ತಪಸ್ವೀ ಚ ವೇದವೇದಾಂಗಪಾರಗಃ~।। ೩~।।}
\end{verse}

\begin{flushleft}
ಬ್ರಹ್ಮದೇವರು ಹೇಳುತ್ತಾರೆ\enginline{-}
\end{flushleft}

ಕೋಸಲ ದೇಶದಲ್ಲಿ ಸರಯೂ ನದೀ ತೀರದಲ್ಲಿ ಶಾಂತಚಿತ್ತನೂ, ಇಂದ್ರಿಯ ನಿಗ್ರಹ ಪಡೆದವನೂ, ತಪಸ್ವಿಯೂ ವೇದವೇದಾಂಗಗಳಲ್ಲಿ ಪ್ರವೀಣನೂ ಆದ ಪ್ರಗಾಥ ಎಂಬ ಋಷಿಯು ಇದ್ದನು.

\begin{verse}
\textbf{ಯದಾಶ್ರಮೇ ವಿರುದ್ಧಾನಾಂ ನಾಭೂದ್ವೈರಂ ಮಹಾತ್ಮನಃ~।}\\\textbf{ಸರ್ವರ್ತುಫಲಪುಷ್ಪೈಶ್ಚ ಸಹಿತಾಃ ಪಾದಪಾ ಅಪಿ~।। ೪~।। }
\end{verse}

\begin{verse}
\textbf{ಔಷಧ್ಯೋತ್ಕೃಷ್ಟಫಲಿತಾ ಭವಂತಿ ತಪಸೋ ಬಲಾತ್~।}\\\textbf{ತಸ್ಯಾಸೀತ್ ಪರಿಧಿರ್ನಾಮ ಶಿಷ್ಯಃ ಕೌಶಿಕಗೊತ್ರಜಃ~।। ೫~।।}
\end{verse}

ಆ ಆಶ್ರಮದಲ್ಲಿ ವಿರೋಧವುಳ್ಳ ಪ್ರಾಣಿಗಳಲ್ಲಿ ವೈರವೇ ಇರಲಿಲ್ಲ (ಹುಲಿ-ಹಸು ಇತ್ಯಾದಿ). ಎಲ್ಲ ಗಿಡಮರಗಳೂ ಸರ್ವ ಋತುಗಳಲ್ಲಿಯೂ ಫಲಪುಷ್ಪಗಳಿಂದ ಭರಿತವಾಗಿದ್ದುವು. ಔಷಧದ ಗಿಡಮೂಲಿಕೆಗಳು ಋಷಿಯ ತಪಸ್ಸಿನ ಪ್ರಭಾವದಿಂದ ಬಹಳ ಚೆನ್ನಾಗಿ ಬೆಳೆದಿದ್ದವು. ಇಂತಹ ಆಶ್ರಮವಾಸಿಯಾದ ಪ್ರಗಾಥನಿಗೆ ಕೌಶಿಕ ಗೋತ್ರದಲ್ಲಿ ಉತ್ಪನ್ನನಾದ ಪರಿಧಿ ಎಂಬ ಶಿಷ್ಯನಿದ್ದನು.

\begin{verse}
\textbf{ವೇದಾನ್ಸಾಂಗಾಂಶ್ಚ ಸೋಪಾಂಗಾನ್ಕಲಾವಿದ್ಯಾಶ್ಚತುರ್ದಶ~।}\\\textbf{ಅಧೀತ್ಯ ಸ್ನಾತಕವ್ರತಮಕರೋತ್ ಗುರ್ವನುಜ್ಞ ಯಾ~।। ೬~।।}
\end{verse}

ಅಂಗ-ಉಪಾಂಗ ಸಹಿತ ವೇದಾಧ್ಯಯನ ಮುಗಿಸಿ, ಹದಿನಾಲ್ಕು ಕಲೆಗಳನ್ನು ಅಭ್ಯಸಿಸಿ, ಗುರುಗಳ ಅಪ್ಪಣೆಯಿಂದ ಸ್ನಾತಕವ್ರತವನ್ನು ಕೈಗೊಂಡನು. (ಸ್ನಾತಕ ವ್ರತ: ಇನ್ನೂ ಹೆಚ್ಚು ವಿದ್ಯಾಭ್ಯಾಸಕ್ಕಾಗಿ ಪ್ರಯತ್ನ)

\begin{verse}
\textbf{ಜಗಾಮು ತೀರ್ಥಯಾತ್ರಾರ್ಥೇ ಛತ್ರೀ ದಂಡ್ಯ ಜಿನಾಂಬರಃ~।}\\\textbf{ಸರ್ವಾ ವಿದ್ಯಾಃ ಸಮಾಪ್ಯಾಥ ಪ್ರಗಾಥಸ್ತ ಪಸಿ ಸ್ಥಿತಃ~।। ೭~।।}
\end{verse}

ಪರಿಧಿಯು ಛತ್ರೀ-ದಂಡಗಳನ್ನು ಹಿಡಿದುಕೊಂಡು, ಜಿಂಕೆಯ ಚರ್ಮವನ್ನು ಧರಿಸಿ ತೀರ್ಥಯಾತ್ರೆಗಾಗಿ ಹೊರಟನು. ಶಿಷ್ಯನಿಗೆ ಸಕಲ ವಿದ್ಯೆಗಳನ್ನೂ ಕಲಿಸಿ ಪ್ರಗಾಥನು ತಪಸ್ಸಿಗೆ ನಿಂತನು.

\begin{verse}
\textbf{ದೂರ್ವಾಮೂಲಾನಿ ಚಾದಾಯ ವೇಷಯಿತ್ವಾ ಚ ತದ್ರಸಮ್~।}\\\textbf{ಆಹಾರಾರ್ಥಂ ಕಲ್ಪಯಿತ್ವಾ ಸ ಚಕಾರ ತಪೋ ಮಹತ್~।। ೮~।।}
\end{verse}

ಹುಲ್ಲನ್ನು ತಂದು ಅದರ ರಸ ತೆಗೆದು, ಆ ರಸವನ್ನೇ ತನ್ನ ಆಹಾರಕ್ಕಾಗಿ ಉಪಯೋಗಿಸಿ ಪ್ರಗಾಥನು ಉಗ್ರವಾದ ತಪಸ್ಸನ್ನು ಆಚರಿಸಿದನು.

\begin{verse}
\textbf{ವರ್ಷಾಂತೇ ಪರಿರಭ್ಯಾಗಾದ್ಗು ರೋರಂತಿಕಮೇವ ಸಃ~।}\\\textbf{ತದಾಗಮನವೇಲಾಯಾಂ ಗುರುಃ ಸಾಯಾಹ್ನಿ ಸತ್ಕ್ರಿಯಾಃ~।। ೯~।। }
\end{verse}

\begin{verse}
\textbf{ಕೃತ್ವಾ ದೂರ್ವಾಮೂಲರಸಂ ಫೇನಿಲಂ ಸ ಪಪೌ ಮುನಿಃ~।}\\\textbf{ತಂ ದೃಷ್ಟ್ವಾ ಪರಿಧಿರ್ಮೋಹಾತ್ ಸುರಾಪೀತ್ಯೇವ ನಿಶ್ಚಿತಃ~।। ೧೦~।।}
\end{verse}

ಒಂದು ವರ್ಷದ ಕೊನೆಯಲ್ಲಿ ತೀರ್ಥಯಾತ್ರೆ ಮುಗಿಸಿ ಪರಿಧಿಯು ತನ್ನ ಗುರುಗಳಾದ ಪ್ರಗಾಥನ ಬಳಿಗೆ ಬಂದನು. ಆಗ ಪ್ರಗಾಥನು ಸಾಯಂಕಾಲದ ಕರ್ಮವನ್ನೆಲ್ಲ ಮುಗಿಸಿ, ಹುಲ್ಲಿನ ಬೇರಿನ ರಸವನ್ನು ಕುಡಿದನು. ಇದನ್ನು ಕಂಡು ಪರಿಧಿಯು “ಗುರುಗಳು ಸುರಾಪಾನ ಮಾಡುತ್ತಿದ್ದಾರೆ” ಎಂದು ನಿಶ್ಚಯಿಸಿದನು.

\begin{verse}
\textbf{ನ ನನಾಮ ಗುರೋಃ ಪಾದೌ ನ ವೃತ್ತಾಂತಮಚೋದಯತ್~।}\\\textbf{ಜಹ್ಯಾನ್ನಾರೀ ಪಥಿ ಭ್ರಷ್ಟಂ ಪತಿಂ ವರ್ಣಾದ್ಗು ರುಂ ಸ್ವಕಮ್~।। ೧೧~।। }
\end{verse}

\begin{verse}
\textbf{ಇತಿ ಧರ್ಮವಿದಃ ಪ್ರಾಹುಃ ತಸ್ಮಾದೇನಂ ತ್ಯಜಾಮ್ಯಹಮ್~।}\\\textbf{ಪ್ರಗಾಥಮನಿಮಂತ್ರೈವ ಸುರಾಪೀತಿ ಚ ನಿಂದಯನ್~।। ೧೨~।।}
\end{verse}

ಗುರುಗಳ ಪಾದದಲ್ಲಿ ನಮಸ್ಕಾರ ಮಾಡಲಿಲ್ಲ. ತನ್ನ ವಿಚಾರವನ್ನೂ ತಿಳಿಸಲಿಲ್ಲ. ದಾರಿತಪ್ಪಿದ ಪತಿಯನ್ನು ಸ್ತ್ರೀಯರು, ವರ್ಣಾಶ್ರಮೋಚಿತವಾದ ಕರ್ಮವನ್ನು ಮಾಡದೇ ಇರುವ ಗುರುವನ್ನು ಶಿಷ್ಯನು ತ್ಯಜಿಸಬೇಕೆಂದು ಧರ್ಮವನ್ನು ತಿಳಿದವರು ಹೇಳುವುದರಿಂದ “ನಾನು ಈ ಗುರುಗಳನ್ನು ತ್ಯಜಿಸುತ್ತೇನೆ' ಎಂಬುದಾಗಿ ನಿಶ್ಚಯಿಸಿ ಪ್ರಗಾಥನಿಗೆ ಏನನ್ನೂ ಹೇಳದೆ ಅವನು ಸುರಾಪಾನ ಮಾಡುತ್ತಾನೆಂದು ನಿಂದಿಸಿದನು.

\begin{verse}
\textbf{ಸ್ವದೇಶಂ ಪ್ರಯಯೌ ಪಾಪೀ ಗೃಹಾಶ್ರಮಮುಪೇಯಿವಾನ್~।}\\\textbf{ತತಃ ಪ್ರಗಾಥೋ ದೇವೇಂದ್ರಸಭಾಂ ಪ್ರತಿ ಯಯೌ ಮುನಿಃ~।। ೧೩~।।}
\end{verse}

ಪಾಪಿಯಾದ ಪರಿಧಿಯು ತನ್ನ ಊರಿಗೆ ಹೋಗಿ ಗೃಹಸ್ಥಾಶ್ರಮವನ್ನು ಸ್ವೀಕರಿಸಿದನು. ಒಂದುದಿನ ಪ್ರಗಾಥಮುನಿಯು ದೇವೇಂದ್ರನ ಸಭೆಗೆ ಹೋದನು.

\begin{verse}
\textbf{ದೇವಾ ನಿವಾರಯಾಮಾಸುರ್ಮುನೇ ನಾರ್ಹಸಿ ತಾಂ ಸಭಾಮ್~।}\\\textbf{ಪ್ರವೇಷ್ಟುಂ ನಿಂದಿತೋ ಯಸ್ಮಾತ್ಸುರಾಪೀತಿ ಚ ಶಿಷ್ಯತಃ~।। ೧೪~।।}
\end{verse}

ನಿನ್ನ ಶಿಷ್ಯನಿಂದ ಸುರಾಪಾನ ಮಾಡುತ್ತೀಯೆಂದು ನೀನು ನಿಂದಿತವಾಗಿರುವುದರಿಂದ ಈ ಸಭೆಯೊಳಗೆ ಬರಲು ನೀನು ಯೋಗ್ಯನಲ್ಲವೆಂದು ದೇವತೆಗಳು ತಡೆದರು.

\begin{verse}
\textbf{ಸ್ವಭಾವಾದ್ವಾಽಸ್ವಭಾವಾದ್ವಾ ನಿಂದಿತೋ ಯೇ ಜನೈರ್ಭುವಿ~।}\\\textbf{ತತೋ ಭೂಮೌ ಕೃತಾಂ ನಿಂದಾಂ ಪರಿಹೃತ್ಯ ಮಹಾಮುನಿಃ~।। ೧೫~।। }
\end{verse}

\begin{verse}
\textbf{ತತೋ ಭೂಮಿಂ ಪುನಃ ಪ್ರಾಪ್ಯ ತಪಸ್ತಪ್ತು ಮಥೋದ್ಯತಃ~।}\\\textbf{ಜನಾನಾಹೂಯ ದೇಶೀಯಾನುವಾಚ ಜರಠೋಸ್ಮ್ಯ ಹಮ್~।। ೧೬~।।} 
\end{verse}

\begin{verse}
\textbf{ನ ಮೇ ಭೂಖನನೇ ಶಕ್ತಿರ್ದೂವಾಮೂಲಾನಿ ಹೇ ಜನಾಃ~।}\\\textbf{ಆದತ್ತ ತದ್ರಸಾಹಾರಃ ತದ್ರಸಾಪ್ಯಾಯನಂ ವ್ರತಮ್~।। ೧೭~।।}
\end{verse}

ಸತ್ಯವೋ ಅಸತ್ಯವೋ ತನಗೆ ಭೂಲೋಕದಲ್ಲಿ ನಿಂದನೆ ಬಂದಿದೆ. ಈ ನಿಂದೆಯನ್ನು ಪರಿಹರಿಸಿಕೊಳ್ಳಲು ಮುನಿಯು ಪುನಃ ಭೂಲೋಕಕ್ಕೆ ಬಂದು ತಪಸ್ಸು ಮಾಡಲು ಉದ್ಯುಕ್ತನಾದನು. ಆ ಪ್ರಾಂತೀಯ ಜನರನ್ನು ಕರೆದು ಹೇಳಿದನು. “ನನಗೆ ಶಕ್ತಿ ಇಲ್ಲವಾದ ಕಾರಣದಿಂದ ಭೂಮಿಯನ್ನು ಅಗೆದು ಹುಲ್ಲಿನಿಂದ ರಸ ತೆಗೆಯಲು ಸಾಧ್ಯವಿಲ್ಲ. ಆದುದರಿಂದ ಹುಲ್ಲಿನ ರಸವನ್ನು ನನಗೆ ತಂದು ಕೊಡಿರಿ. ಅದೇ ನನ್ನ ಆಹಾರ. ಅದೇ ನನ್ನ ನಿಯಮ.

\begin{verse}
\textbf{ತಸ್ಮಾಚ್ಚ ದೂರ್ವಾಮೂಲಾನಿ ಮುನಯೇ ಜಠರಾಗ್ನಯೇ~।}\\\textbf{ಆದಾಯ ಪೀತ್ವಾ ತನ್ಮೂಲರಸಂ ದಾಸ್ಯಥ ಪ್ರತ್ಯ ಹಮ್~।। ೧೮~।। }
\end{verse}

“ಈ ನಿಯಮಕ್ಕೆ ಅನುಗುಣವಾಗಿ ಪ್ರತಿನಿತ್ಯ ನನಗೆ ಹುಲ್ಲಿನ ರಸ ಕೊಡಿರಿ”.

\begin{verse}
\textbf{ತಪಸ್ವೀನಾಂ ಚ ವೃದ್ಧಾನಾಂ ಸೇವಾಂ ಕುರ್ವಂತಿ ಯೇ ನರಾಃ~।}\\\textbf{ತೇಷಾಂ ದಾತಾ ಸ್ವಯಂ ವಿಷ್ಣು ರ್ಭುಕ್ತಿಂ ಮುಕ್ತಿಂ ನ ಸಂಶಯಃ~।। ೧೯~।।}
\end{verse}

ತಪಸ್ವಿಗಳಿಗೂ, ವೃದ್ದರಿಗೂ ಸೇವೆ ಮಾಡುವ ಜನರಿಗೆ ಸಾಕ್ಷಾತ್ ವಿಷ್ಣುವು ಐಹಿಕ ಸುಖಗಳನ್ನೂ ಅಂತ್ಯದಲ್ಲಿ ಮೋಕ್ಷವನ್ನೂ ಅನುಗ್ರಹಿಸುವನು. ಇದರಲ್ಲಿ ಸಂಶಯವಿಲ್ಲ.

\begin{verse}
\textbf{ವಿಶ್ರಬ್ಧಾಶ್ಚ ತತೋ ಲೋಕಾಸ್ತದ್‌ಗ್ರಾಮಸ್ಥಾ ದಯಾಲವಃ~।}\\\textbf{ತದ್ ಗ್ರಾಮಸ್ಥಂ ದ್ವಿಜಂ ಕಂಚಿಜ್ಜ್ಯೋತಿಷಾರ್ಣವಸಂಜ್ಞಕಮ್~।। ೨೦~।। }
\end{verse}

\begin{verse}
\textbf{ಸೇವಾರ್ಥಂ ಕಲ್ಪಯಾಮಾಸುಃ ಪ್ರಗಾಥಸ್ಯ ಮಹಾತ್ಮನಃ~।}\\\textbf{ಪ್ರಗಾಥಸ್ತಪಸೇ ಪ್ರಾಗಾತ್ತತೋ ವೈ ಜ್ಯೋತಿಷಾರ್ಣವ~।। ೨೧~।।}
\end{verse}

ಅಂತಃಕರಣಯುಕ್ತರಾದ ಆ ಗ್ರಾಮಸ್ಥರು ಪ್ರಗಾಥಮುನಿಯ ಸೇವೆಗಾಗಿ ಜ್ಯೋತಿಷಾರ್ಣವನೆಂಬ ಬ್ರಾಹ್ಮಣನನ್ನು ನೇಮಿಸಿದರು. ಪ್ರಗಾಥನು ತಪಸ್ಸಿಗೆ ಹೋದನಂತರ ನಿತ್ಯವೂ ಜ್ಯೋತಿಷಾರ್ಣವನು

\begin{verse}
\textbf{ದೂರ್ವಾಮೂಲಾನ್ಯ ಪಾಹೃತ್ಯ ಸಾಯಾಹ್ನೇ ತಾನ್ಯು ಲೂಖಲೇ~।}\\\textbf{ಅವಹತ್ಯ ಚ ಸಂಪಿಷ್ಟ್ವಾ ರಸಂ ಪಾತ್ರೇ ನಿವೇಶ್ಯ ಚ~।। ೨೨~।।}
\end{verse}

\begin{verse}
\textbf{ಪ್ರಾಗಾಥಾಗಮಪರ್ಯಂತಂ ತದ್ಗೃಹೇ ಸ್ಥಾ ಪಯತ್ಯಸೌ~।}\\\textbf{ಸ ಚ ದೂರ್ವಾರಸಃ ಪಾತ್ರೇ ಫೇನಿಲೋ ಜಾಯತೇ ಕ್ಷಣಾತ್~।। ೨೩~।।}
\end{verse}

ಹುಲ್ಲನ್ನು ತಂದು, ಒರಳಿನಲ್ಲಿ ಕುಟ್ಟಿ ರಸವನ್ನು ಒಂದು ಪಾತ್ರೆಯಲ್ಲಿಟ್ಟು ಪ್ರಗಾಥಮುನಿಯು ಬರುವತನಕ ಆಶ್ರಮದಲ್ಲಿಯೇ ಇರುತ್ತಿದ್ದನು. ಪಾತ್ರೆಯಲ್ಲಿ ಸ್ವಲ್ಪ ಹೊತ್ತಿನಲ್ಲಿಯೇ ನೊರೆಯು ಉತ್ಪನ್ನವಾಗುತ್ತಿತ್ತು.

\begin{verse}
\textbf{ಸಾಯಮಗ್ನಿಂ ತತೋ ಹುತ್ವಾ ಪ್ರಾಗಾಥೋಪ್ಯಪಿಬದ್ರಸಮ್~।}\\\textbf{ತಂ ದೃಷ್ಟ್ವಾತೀವನಿರ್ವೇದಮಾಯಯೌ ಜ್ಯೋತಿಷಾರ್ಣವಃ~।। ೨೪~।। }
\end{verse}

\begin{verse}
\textbf{ವೃಥಾಯಂ ನಿಂದಿತೋsಸ್ಮಾಭಿಃ ಸುರಾಪೀತಿ ವಚೋ ಬಲಾತ್~।}\\\textbf{ಪರಿಧೇಸ್ತಸ್ಯ ಶಿಷ್ಯಸ್ಯ ಜನಾಂಸ್ತಾನಿತಿ ಚಾಬ್ರವೀತ್~।। ೨೫~।।}
\end{verse}

ಪ್ರಗಾಥಮುನಿಯು ಸಾಯಂಕಾಲದಲ್ಲಿ ಅಗ್ನಿಯಲ್ಲಿ ಹೋಮಮಾಡಿ ಆ ಹುಲ್ಲಿನರಸವನ್ನು ಕುಡಿಯುತ್ತಿದ್ದನು. ಜ್ಯೋತಿಷಾರ್ಣವನು ಇದನ್ನು ಕಂಡು ದುಃಖಪಟ್ಟನು. ನಮ್ಮಿಂದಲೂ ಮತ್ತು ಈ ಮುನಿಯ ಶಿಷ್ಯನಾದ ಪರಿಧಿಯಿಂದಲೂ ವ್ಯರ್ಥವಾಗಿ ಪ್ರಗಾಥನು ಮದ್ಯಪಾನ ಮಾಡುತ್ತಾನೆ” ಎಂದು ನಿಂದಿಸಲ್ಪಟ್ಟನು ಎಂಬುದಾಗಿ ಹೇಳಿದನು.

\begin{verse}
\textbf{ಪಶ್ಚಾತ್ತಾಪೇನ ಸಂತಪ್ತಾ ಮಿಥ ಏವಂ ಬಭಾಷಿರೇ~।}\\\textbf{ಮಿಥ್ಯಾಭಿಶಂಸಿತೋಸ್ಮಾಭಿಃ ನಿಂದಿತೋಯಂ ಮಹಾಮುನಿಃ~।। ೨೬~।।} 
\end{verse}

\begin{verse}
\textbf{ಧರ್ಮಾರ್ಥಕಾಮಾ ನಶ್ಯಂತಿ ಹ್ಯವಿಚಾರಿತಕಾರಿಣಾಮ್~।}\\\textbf{ಯೇ ಲೋಕೇ ಪರದೋಷಾಂಶ್ಚ ವ್ಯಾಹರಂತ್ಯವಿಚಾರತಃ~।। ೨೭~।।}
\end{verse}

ಪಶ್ಚಾತ್ತಾಪಗೊಂಡ ಗ್ರಾಮಸ್ಥರು ತಮ್ಮ ತಮ್ಮಲ್ಲಿಯೇ ಮಾತನಾಡಿಕೊಂಡರು. ಈ ಮಹಾತ್ಮನಾದ ಮುನಿಯು ನಮ್ಮಿಂದ ಕಾರಣವಿಲ್ಲದೆ ನಿಂದಿಸಲ್ಪಟ್ಟನು. ಲೋಕದಲ್ಲಿ ಸರಿಯಾಗಿ ವಿಚಾರಮಾಡದೇ ಇತರರ ಕರ್ಮಗಳಲ್ಲಿ ದೋಷವನ್ನು ಹೇಳುವ ಜನರ ಧರ್ಮಾರ್ಥಕಾಮಗಳು ನಾಶವಾಗುತ್ತವೆ.

\begin{verse}
\textbf{ಮಿಥ್ಯಾ ಚೇತ್ ದ್ವಿಗುಣಂ ಪಾಪಂ ಸತ್ಯಂ ಚೇತ್ ಸಮಭಾಗಿನಃ~।}\\\textbf{ಕೊ ಲೋಕಃ ಪರಿಧೇಸ್ತಸ್ಯ ಶಿಷ್ಯಸ್ಯಾಸ್ಯಾ ಪನಾದಿನಃ~।। ೨೮~।। }
\end{verse}

\begin{verse}
\textbf{ಪರಿಧೇಃ ಪಾಪಿನಸ್ತಸ್ಯ ವಚೋ ವಿಶ್ರಂಭಣಾದ್ವಯಮ್~।}\\\textbf{ಅಭಿನಿಂದ್ಯ ಮಹಾತ್ಮಾನಂ ಪಾಪಭಾಜೋ ಭವಾಮಹೇ~।। ೨೯~।।}
\end{verse}

ನಾವು ಹೇಳುವ ದೋಷಗಳು ಸುಳ್ಳಾಗಿದ್ದರೆ ನಮಗೆ ಎರಡರಷ್ಟು ಪಾಪ. ಸತ್ಯವಾಗಿದ್ದರೆ ಎಷ್ಟು ನಿಗದಿಯಾಗಿದೆಯೋ ಅಷ್ಟೇ ಪಾಪ, ಈ ಮುನಿಯ ಶಿಷ್ಯನಾದ ಪರಿಧಿಗೆ ಪಾಪದ ಅನುಭವಕ್ಕೆ ಯಾವ ಲೋಕವು ಕಾದಿದೆಯೋ? ಪರಿಧಿಯ ಮಾತುಗಳಲ್ಲಿ ವಿಶ್ವಾಸವಿಟ್ಟು ನಾವೆಲ್ಲ ಈ ಮುನಿಯನ್ನು ನಿಂದಿಸಿ ಪಾಪಕ್ಕೆ ಗುರಿಯಾದೆವು.

\begin{verse}
\textbf{ಇತಿ ನಿರ್ವೇದಮಾಪನ್ನಾಃ ಪ್ರಾರ್ಥಯಾಮಾಸುರಂಜಸಾ~।}\\\textbf{ಪ್ರಗಾಥ ತೇ ನಮೋ ಭದ್ರ ಕ್ಷಮಸ್ವಾಗಾಂಸಿ ಪಾಪಿನಃ~।। ೩೦~।।}\\\textbf{ಅಸ್ಮಾಕಂ ತ್ವಭಿಶಸ್ತಾನಾಂ ದಯಾಂ ಕುರು ದಯಾನಿಧೇ~।।}
\end{verse}

ಮನಸಿನಲ್ಲಿ ಖೇದಗೊಂಡ ಗ್ರಾಮಸ್ಥರು ಆ ಮುನಿಯನ್ನು ಈ ರೀತಿ ಪ್ರಾರ್ಥಿಸಿದರು. “ದಯಾಳುಗಳಾದ ಪ್ರಗಾಥರೇ, ನಮ್ಮ ಪಾಪಗಳನ್ನು ಕ್ಷಮಿಸಿರಿ, ನಮ್ಮ ಮೇಲೆ ದಯಮಾಡಿರಿ.

\begin{verse}
\textbf{ಇತಿ ತೇಷಾಂ ವಚಃ ಶ್ರುತ್ವಾ ಪ್ರಗಾಥೋ ವಾಕ್ಯಮಬ್ರವೀತ್~।। ೩೧~।।}
\end{verse}

ಅವರ ಮಾತುಗಳನ್ನು ಲಾಲಿಸಿದ ಪ್ರಗಾಥನು ಈ ರೀತಿ ನುಡಿದನು.

\begin{verse}
\textbf{ಸುಖಂ ದುಃಖಂ ತಥಾ ನಿಂದಾಂ ಬಹುಮಾನಂ ಜಯಾಜಯೌ~।}\\\textbf{ಶೋಕಂ ದುಃಖಂ ಚ ಹಾನಿಂ ಚ ಲಾಭಾಲಾಭೌ ಜರಾಮೃತೀ~।। ೩೨~।।}
\end{verse}

\begin{verse}
\textbf{ಪೂರ್ವಕರ್ಮಾನುರೂಪೇಣ ದೈವಮೇವ ದದಾತಿ ಹಿ~।}\\\textbf{ಯಸ್ಮಿನ್ಕಾಲೇ ಯದಾ ದೇಶೇ ವಯಸಾ ಯಾದೃಶೇನ ಚ~।। ೩೩~।।}
\end{verse}

\begin{verse}
\textbf{ಯಾದೃಗೇವ ಕೃತಂ ಕರ್ಮ ಜೀವೈಃ ಸಂಸಾರವರ್ತಿಭಿಃ~।}\\\textbf{ತಸ್ಮಿನ್ಕಾಲೇ ಚ ದೇಶೇ ಚ ವಯಸಾ ತಾದೃಶೇನ ಚ~।। ೩೪~।।}
\end{verse}

ಸುಖ, ದುಃಖ, ಜನರಿಂದ ನಿಂದೆ, ಜನರಿಂದ ಆದರ, ಜಯ, ಅಪಜಯ, ನಷ್ಟ, ಲಾಭ, ಮುಪ್ಪು, ಮೃತ್ಯು, ಇವೆಲ್ಲವೂ ಪ್ರತಿಯೊಬ್ಬರಿಗೂ ಅವರವರ ಹಿಂದಿನ ಕರ್ಮಾನುಸಾರವಾಗಿ ದೈವೇಚ್ಛೆಯಿಂದಲೇ ಪ್ರಾಪ್ತವಾಗುತ್ತವೆ. ಯಾವ ಕಾಲದಲ್ಲಿ ಯಾವ ದೇಶದಲ್ಲಿ, ಯಾವ ವಯಸ್ಸಿನಲ್ಲಿ, ಸಂಸಾರದಲ್ಲಿರುವ ಜನರು ಎಂತಹ ಕರ್ಮವನ್ನು ಆಚರಿಸುತ್ತಾರೆಯೋ ಅದೇ ದೇಶದಲ್ಲಿ, ಅಂತಹುದೇ ಕಾಲದಲ್ಲಿ, ಅದೇ ವಯಸ್ಸಿನಲ್ಲಿ, ಕರ್ಮಫಲವು ಲಭಿಸುತ್ತದೆ.

\begin{verse}
\textbf{ಅವಶ್ಯಂ ವೈ ಜನೈರೇವ ತಾದೃಗೇವಾನುಭೂಯತೇ~।}\\\textbf{ಅಸ್ವರ್ಗಂ ಲೋಕವಿದ್ವಿಷ್ಟಂ ಧರ್ಮಮಪ್ಯಾಚರೇನ್ನ ತು~।। ೩೫~।। }
\end{verse}

\begin{verse}
\textbf{ಕೃತ್ವಾ ನರಕಮಾಪ್ನೋತಿ ತಸ್ಮಾತ್ಸರ್ವಮತಂ ಚರೇತ್~।}\\\textbf{ಅತೀವ ನಿಂದಿತಮತಿ ತತ್ಕೃತ್ವಾ ನೈವ ದುಷ್ಯತಿ~।। ೩೬~।।}
\end{verse}

ಅಂತಹ ಕರ್ಮಫಲವನ್ನು ಜನರು ಅನುಭವಿಸಲೇಬೇಕು. ಸ್ವರ್ಗಪ್ರಾಪ್ತಿಗೆ ಬಂಧಕವಾದ, ನಿಂದಿತವಾದ ಕರ್ಮಗಳನ್ನು, ಜನರು ಆಚರಿಸಬಾರದು, ಆಚರಿಸಿದರೆ ನರಕವಾಸವು ಪ್ರಾಪ್ತವಾಗುತ್ತದೆ. ಆದುದರಿಂದ ಸರ್ವರಿಗೂ ಸರಿಯೆಂದು ತೋರುವ (ಶಾಸ್ತ್ರೋಕ್ತವಾದ) ಧರ್ಮಗಳನ್ನೇ ಆಚರಿಸಬೇಕು. ಶಾಸ್ತ್ರೋಕ್ತವಾದ ಕರ್ಮವು ಜನಸಾಮಾನ್ಯರಿಂದ ನಿಂದಿತವಾಗಿದ್ದರೂ, ಅದನ್ನು ಆಚರಿಸುವುದರಿಂದ ಪಾಪವು ಉಂಟಾಗುವುದಿಲ್ಲ.

\begin{verse}
\textbf{ಸತಾಂ ಯೇನೋಪಕಾರಃ ಸ್ಯಾತ್ ಬಹುಹಿಂಸಾಯುತೇನ ಚ~।}\\\textbf{ದಧೀಚಿರ್ದರ್ಮೃತಿಂ ಪ್ರಾಪ್ಯ ದೇವಾರ್ಥೇ ಸದ್ಗತಿಂ ಯಯೌ~।। ೩೭~।। }
\end{verse}

\begin{verse}
\textbf{ಶಿಬಿರ್ಮಾಂಸಾನಿ ಭಕ್ಷ್ಯಾಣಿ ದತ್ವಾ ಕೀರ್ತಿಂ ಚ ಸದ್ಗತಿಮ್~।}\\\textbf{ಹತ್ವಾ ವೃತ್ರಂ ದ್ವಿಜಂ ತ್ವಿಂದ್ರೋ ವೃತ್ರಹೇತಿ ಚ ವಿಶ್ರುತಃ~।। ೩೮~।। }
\end{verse}

\begin{verse}
\textbf{ಪಿತ್ರರ್ಥೇ ಭಾರ್ಗವೋ ರಾಮೋ ಹತ್ವಾ ಮಾತರಮಂಗನಾಮ್~।}\\\textbf{ಅತಿಕೀರ್ತಿಂ ಸಮಾಪೇದೇ ರಾಮೋ ರಾಜೀವಲೋಚನಃ~।। ೩೯~।।}
\end{verse}

ಯಾವ ಕರ್ಮದಿಂದ ಸಜ್ಜನರಿಗೆ ಉಪಕಾರವಾಗುತ್ತದೆಯೋ ಅಂತಹ ಕರ್ಮವು ಬಹು ಹಿಂಸೆಯಿಂದ ಮಾಡಬೇಕಾದರೂ ಸಹ ಪಾಪ ಬರುವುದಿಲ್ಲ. ದಧೀಚಿಋಷಿಗಳು ದೇವತೆಗಳಿಗಾಗಿ ದುರ್ಮರಣವನ್ನು ಹೊಂದಿದರೂ ಅವರಿಗೆ ಸದ್ಗತಿಯಾಯಿತು. ಶಿಬಿ ಚಕ್ರವರ್ತಿಯು ತನ್ನ ಮೈಯೊಳಗಿನ ಮಾಂಸಗಳನ್ನು ದಾನಮಾಡಿ ಸದ್ಗತಿಯನ್ನು ಪಡೆದನು. ದೇವೇಂದ್ರನು ವೃತ್ರನೆಂಬ ಬ್ರಾಹ್ಮಣನನ್ನು ಸಂಹಾರ ಮಾಡಿದರೂ "ವೃತ್ರಹಾ” ಎಂಬ ಕೀರ್ತಿಶಾಲಿಯಾದನು. ಕಮಲ ನೇತ್ರನಾದ ಪರಶುರಾಮನು ತಂದೆಯ ಆಜ್ಞಾನುಸಾರವಾಗಿ ತನ್ನ ತಾಯಿಯನ್ನೇ ಸಂಹರಿಸಿದರೂ ಅಪಾರವಾದ ಕೀರ್ತಿಯನ್ನು ಪಡೆದನು.

\begin{verse}
\textbf{ಪುರಾ ಕಶ್ಚಿದ್ದ್ವಿ ಜವರೋ ಸತ್ಯನಾದೀ ಜಿತೇಂದ್ರಿಯಃ~।}\\\textbf{ತೀರ್ಥಯಾತ್ರಾಪ್ರಸಂಗೇನ ವಿಪಿನೇ ವ್ಯಚರತ್ ಕ್ವಚಿತ್~।। ೪೦~।। }
\end{verse}

\begin{verse}
\textbf{ದಿಗಂಬರೋ ನಿರಾಹಾರೋ ಮಾಂಸರಕ್ತವಿವರ್ಜಿತಃ~।}\\\textbf{ಪಲ್ಲವೇ ವೇಣುಗುಲ್ಮೇಷು ನಿಷಸಾದ ಮುನೀಶ್ವರಃ~।। ೪೧~।।}
\end{verse}

ಹಿಂದೆ ಸತ್ಯವನ್ನೇ ನುಡಿಯುವ ಇಂದ್ರಿಯನಿಗ್ರಹದಿಂದ ಯುಕ್ತನಾದ ಒಬ್ಬ ಬ್ರಾಹ್ಮಣನು ತೀರ್ಥಯಾತ್ರಾ ಕಾರಣದಿಂದ ಕಾಡಿನಲ್ಲಿ ಸಂಚರಿಸುತ್ತಿದ್ದನು. ಅವನು ದಿಗಂಬರನಾಗಿಯೂ, ಆಹಾರ ರಹಿತನಾಗಿಯೂ, ಮಾಂಸರಕ್ತಹೀನನಾಗಿಯೂ ಇದ್ದು ಅರಣ್ಯದಲ್ಲಿ ಪೊದೆಗಳಲ್ಲಿಯೂ, ಬಿದಿರುಗಳ ಸಮೂಹದಲ್ಲಿಯೂ ಮಲಗುತ್ತಿದ್ದನು.

\begin{verse}
\textbf{ತದಾನೀಂ ಲುಬ್ಧ ಕಃ ಕಶ್ಚಿತ್ ಪೃಷ್ಠ ತೋಽಗಾದ್ದ್ವಿಜಾನ್ವನೇ~।}\\\textbf{ತೇ ಗುಲ್ಮಮೂಲೇ ಲೀನಾಸ್ಯುಃ ನಾಪಶ್ಯತ್ ಲುಬ್ಧ ಕೋಪ್ಯ ಮೂನ್~।। ೪೨~।।}\\\textbf{ನಿಷಣ್ಣಂ ಪಲ್ಲವೇ ದೃಷ್ಟಾ ವ್ಯಾಧೋ ವಾಕ್ಯಮುವಾಚ ತಮ್~।}
\end{verse}

ಆ ಕಾಲದಲ್ಲಿ ಬೇಡನೊಬ್ಬನು ಬ್ರಾಹ್ಮಣರನ್ನು ಸುಲಿಗೆಮಾಡಲು ಅವರ ಹಿಂದೆ ಬರುತ್ತಿದ್ದನು. ಬ್ರಾಹ್ಮಣರು ಭಯಗೊಂಡು ಪೊದೆಗಳಲ್ಲಿ ಅಡಗಿಕೊಂಡರು. ಬೇಡನಿಗೆ ಅವರು ಕಾಣಿಸಲಿಲ್ಲ. ದಿಗಂಬರ ಬ್ರಾಹ್ಮಣನನ್ನು ನೋಡಿ ಬೇಡನು ಹೇಳಿದನು:

\begin{verse}
\textbf{ಕಿಂ ನು ಜಾನಾಸಿ ತ್ವಂ ವಿಪ್ರ ದ್ವಿಜಾನತ್ರಾಗತಾನ್ ಕ್ವಚಿತ್~।। ೪೩~।।}\\\textbf{ವದ ತಾನಸ್ತಿ ಮೇ ಕೃತ್ಯಂ ತತಃ ಸತ್ಯಪರಾಯಣಃ~। }\\\textbf{ತೇ ದ್ವಿಜಾ ಗುಲ್ಮಮಧ್ಯೇಷು ನಿಷಣ್ಣಾ ಇತಿ ಚಾಬ್ರವೀತ್~।। ೪೪~।।}
\end{verse}

“ಬ್ರಾಹ್ಮಣನೇ, ಈಗ ಕೆಲವು ಬ್ರಾಹ್ಮಣರು ಈ ಪ್ರದೇಶದಲ್ಲಿ ಬಂದರು, ಅವರು ಎಲ್ಲಿದ್ದಾರೆ ಎಂಬುದು ನಿನಗೆ ತಿಳಿದಿದೆಯೆ? ಅವರ ಬಳಿ ನನಗೆ ಸ್ವಲ್ಪ ಕೆಲಸ ಇದೆ?” ಹೀಗೆನ್ನಲು ಆ ದಿಗಂಬರ ಸತ್ಯವಂತನಾದ ವಿಪ್ರನು ``ಆ ಬ್ರಾಹ್ಮಣರು ಪೊದೆಗಳ ಮಧ್ಯದಲ್ಲಿ ಅಡಗಿಕೊಂಡು ಇದ್ದಾರೆ” ಎಂದನು.

\begin{verse}
\textbf{ಅಥ ತಂ ಲುಬ್ದ ಕೊ ದೃಷ್ಟ್ವಾ ಹತ್ವಾದಾಯ ಧನಂ ಯಯೌ~।}\\\textbf{ಸಾ ಬ್ರಹ್ಮಹತ್ಯಾ ವಿಪ್ರೇಂದ್ರಂ ಪೀಡಯಾಮಾಸ ಚಾಂಜಸಾ~।। ೪೫~।।}
\end{verse}

ನಂತರ ಬೇಡನು ಆ ಅಡಗಿಕೊಂಡಿದ್ದ ವಿಪ್ರರನ್ನು ಕೊಂದು ಅವರ ಧನವನ್ನು ಅಪಹರಿಸಿಕೊಂಡುಹೋದನು. ಆ ಬ್ರಹ್ಮಹತ್ಯವು ದಿಗಂಬರ ಬ್ರಾಹ್ಮಣನಿಗೆ ಪೀಡೆಯನ್ನುಂಟುಮಾಡಿತು.

\begin{verse}
\textbf{ದೇಶಕಾಲಾನುಗಂ ಶುದ್ಧಂ ಕೌಲೀನಂ ವೇದಚೋದಿತಮ್~।}\\\textbf{ಕುರ್ವನ್ ಸುಖಮವಾಪ್ನೋತಿ ಸರ್ವಸಜ್ಜನಸಂಮತಮ್~।। ೪೬~।। }
\end{verse}

\begin{verse}
\textbf{ಭವೇದ್ದೃವಿರುದ್ಧಂ ತು ಪೀತ್ವಾ ದೂರ್ವಾರಸಂ ಮಯಾ~।}\\\textbf{ಮಮಾಯಮಪರಾಧೋ ಹಿ ನ ಕಸ್ಯಾಪಿ ಹತಾಂಹಸಃ~।। ೪೭~।।}
\end{verse}

ದೇಶಕಾಲಾನುಸಾರವಾಗಿ, ಶುದ್ದನಾಗಿಯೂ, ಕುಲದ ಪದ್ಧತಿಯನ್ನು ಅನುಸರಿಸುವವನೂ, ವೇದಶಾಸ್ತ್ರಗಳಲ್ಲಿ ಹೇಳಿರುವ ಕರ್ಮಗಳನ್ನು ಸಜ್ಜನಸಮುದಾಯಕ್ಕೆ ಸರಿಯೆಂದು ತೋರುವ ರೀತಿಯಲ್ಲಿ ಯಾವನು ಆಚರಿಸುವನೋ ಅಂತಹವನು ಇಹಪರಗಳಲ್ಲಿ ಸುಖವನ್ನು ಹೊಂದುವನು. ನಾನು ಹುಲ್ಲಿನ ರಸವನ್ನು ಕುಡಿದಿದ್ದು ಇತರರಲ್ಲಿ ವಿರುದ್ಧ ಭಾವನೆಯನ್ನು ಉಂಟುಮಾಡಿತು. ಇದು ನನ್ನ ಅಪರಾಧವೇ ವಿನಹ ಪಾಪರಹಿತರಾದ ಜನರ ಅಪರಾಧವಲ್ಲ.

\begin{verse}
\textbf{ನ ಪಾಪಲೇಶೋ ಭವತಾಮಿತಿ ತೇಷಾಂ ವರಂ ದದೌ~।}\\\textbf{ತತೋ ಭೂಮೌ ಕೃತಾಂ ನಿಂದಾಂ ಪರಿಹೃತ್ಯ ಮಹಾಮುನಿಃ~।। ೪೮~।।}\\\textbf{ದೇವೇಂದ್ರಸ್ಯ ಸಭಾಂ ಪ್ರಾಪ್ತೋ ದೇವಾದ್ಯೈರಪಿ ಪೂಜಿತಃ~।}
\end{verse}

ನಿಮಗೆ ಪಾಪಲೇಶವೂ ಸಹ ಬರದಿರಲೆಂದು ವರವನ್ನು ಕೊಟ್ಟು, ತನಗೆ ಬಂದಿದ್ದ ಅಪವಾದವನ್ನು ಪರಿಹರಿಸಿಕೊಂಡು ದೇವೇಂದ್ರನ ಸಭೆಗೆ ಹೋಗಿ ಅಲ್ಲಿ ದೇವತೆಗಳಿಂದಲೂ ಪೂಜಿತನಾದನು.

\begin{verse}
\textbf{ಪರಿಧಿಸ್ತೀನ ದೋಷೇಣ ಭುಕ್ತ್ವಾ ಮನ್ವಂತರೇಷು ಚ~।। ೪೯~।।}
\end{verse}

\begin{verse}
\textbf{ಚತುರ್ದಶಸ್ವನಿರ್ವಚ್ಯಾಂ ಯಾತನಾಂ ಕೌರವೇ ಯತಃ~।}\\\textbf{ಅರಣ್ಯೇ ನಿರ್ಜಲೇ ದೇಶೇ ಯುಕ್ತೇ ಕಂಟಕಪಾದಪೈಃ~।। ೫೦~।।}
\end{verse}

\begin{verse}
\textbf{ಛಾಯಾಹೀನೇ ಮಹಾಘೋರೇ ಸೋsಭವತ್ ಬ್ರಹ್ಮರಾಕ್ಷಸಃ~।}\\\textbf{ತತ್ರಾಬ್ದಾನಿ ವ್ಯತೀತಾನಿ ಅರ್ಬುದಾನಿ ತ್ರಿಸಪ್ತ ಚ~।। ೫೧~।।}
\end{verse}

ಪರಿಧಿಯು ಗುರುನಿಂದೆಯ ದೋಷದಿಂದ ಹದಿನಾಲ್ಕು ಮನ್ವಂತರಗಳಲ್ಲಿ ರೌರವ ನರಕದಲ್ಲಿ ಬಹಳ ಹಿಂಸೆಯನ್ನು ಅನುಭವಿಸಿ ನಂತರ ಭೂಲೋಕದಲ್ಲಿ ನೀರು-ನೆರಳಿಲ್ಲದ ಭಯಂಕರವಾದ ಅರಣ್ಯದಲ್ಲಿ ಬ್ರಹ್ಮರಾಕ್ಷಸನಾದನು. ಅರಣ್ಯವು ಮುಳ್ಳಿನ ಗಿಡಗಳಿಂದ ತುಂಬಿತ್ತು. ಈ ರೀತಿ ಅರಣ್ಯದಲ್ಲಿ ಇಪ್ಪತ್ತೊಂದು ಅರ್ಬುದ ವರ್ಷಗಳನ್ನು ಕಳೆದನು.

\begin{verse}
\textbf{ಆಜಗಾಮ ವನಂ ತಚ್ಚ ದೈವಾನ್ಮುದ್ಗಲನಾಮಕಃ~।}\\\textbf{ಯಜ್ಞಾರ್ಥಂ ಪಾಂಡ್ಯರಾಜಸ್ಯ ತಸ್ಮಿನ್ ಮಾರ್ಗೆ ದದರ್ಶ ತಮ್~।। ೫೨~।। }
\end{verse}

\begin{verse}
\textbf{ಆಹಾರಾರ್ಥಂ ಪ್ರಾದ್ರವಂತಂ ತೇಜಸಾ ಚ ನಿವರ್ತಿತಃ~।}\\\textbf{ಊರ್ಧ್ವಕೇಶಂ ವಿರೂಪಂ ಚ ಚಲಜಿಹ್ವಂ ದಿಗಂಬರಮ್~।। ೫೩~।।}
\end{verse}

\begin{verse}
\textbf{ತಂ ದೃಷ್ಟ್ವಾ ಮೌದ್ಗಲಃ ಪ್ರಾಹ ಕೋಸೌ ಸತ್ಯಂ ವದಾನಘ~।}\\\textbf{ಇತಿ ತದ್ವಚನಂ ಶ್ರುತ್ವಾ ಸೋಬ್ರವೀದ್ರಾಕ್ಷಸಸ್ತದಾ~।। ೫೪~।।}
\end{verse}

ಹೀಗಿರಲು ಒಂದು ದಿನ ದೈವಯೋಗದಿಂದ ಮುದ್ಗಲನೆಂಬ ಬ್ರಾಹ್ಮಣನು ಪಾಂಡ್ಯರಾಜನು ಮಾಡುವ ಯಜ್ಞಕ್ಕೆ ಹೋಗುತ್ತಾ ಆ ಮಾರ್ಗದಲ್ಲಿ ಬಂದನು. ಬ್ರಹ್ಮರಾಕ್ಷಸನು ಮುದ್ಗಲನನ್ನು ನೋಡಿ ಆಹಾರಾರ್ಥವಾಗಿ ಅವನನ್ನು ತಿನ್ನಲು ಹತ್ತಿರ ಬಂದು ಮುದ್ಗನ ತೇಜಸ್ಸಿನಿಂದ ಹಿಂತೆಗೆದನು. ಕೂದಲುಗಳನ್ನು ಮೇಲಕ್ಕೆ ಕೆದರಿಕೊಂಡ, ಕೆಟ್ಟ ರೂಪವುಳ್ಳ, ನಾಲಿಗೆಯನ್ನು ಅಲ್ಲಾಡಿಸುತ್ತಿರುವ, ದಿಗಂಬರನಾದ ಆ ಬ್ರಹ್ಮರಾಕ್ಷಸನನ್ನು ನೋಡಿ ಮುದ್ಗಲನು ಕೇಳಿದನು ನೀನು ಯಾರು? ಸತ್ಯವಾಗಿ ಹೇಳು” ಎನ್ನಲು ಬ್ರಹ್ಮರಾಕ್ಷಸನು ಹೇಳಿದನು:

\begin{verse}
\textbf{ಅಹಂ ಪುರಾ ಪ್ರಗಾಥಸ್ಯ ಶಿಷ್ವ್ಯಃ ಪರಿಧಿನಾಮಕಃ~।}\\\textbf{ಅಜ್ಞಾತ್ವಾ ತತ್ಕೃತಾಹಾರಂ ಸುರಾಪೀತ್ಯಭಿನಿಂದಿತಃ~।। ೫೫~।। }
\end{verse}

\begin{verse}
\textbf{ಗುರುರ್ಮಯಾ ಕೃತಘ್ನೇನ ಪ್ರಗಾಥೋ ಲೋಕವಿಶ್ರುತಃ~।}\\\textbf{ತೇನ ಕರ್ಮವಿಪಾಕೇನ ಜಾತೋಹಂ ಬ್ರಹ್ಮರಾಕ್ಷಸಃ~।। ೫೬~।।}
\end{verse}

ನಾನು ಹಿಂದೆ ಪ್ರಗಾಥಮುನಿಯ ಶಿಷ್ಯನಾಗಿದ್ದೆ. ಹೆಸರು ಪರಿಧಿ. ನನ್ನ ಗುರುಗಳು ಆಹಾರಕ್ಕಾಗಿ ತಯಾರಿಸಿಕೊಂಡ ಪದಾರ್ಥವನ್ನು ಮದ್ಯವೆಂದು ಅಜ್ಞಾನದಿಂದ ತಿಳಿದು ಗುರುಗಳು ಮದ್ಯಪಾನ ಮಾಡುತ್ತಾರೆಂದು ದೂಷಿಸಿದೆ. ಆ ಗುರು ನಿಂದೆಯ ದೋಷದಿಂದ ಈಗ ಬ್ರಹ್ಮರಾಕ್ಷಸನಾಗಿ ಹುಟ್ಟಿದ್ದೇನೆ.

\begin{verse}
\textbf{ಅತ್ರಾಬ್ದಾನಿ ವ್ಯತೀತಾನಿ ನ ಜಾನೇ ನಿರ್ಜನೇ ವನೇ~।}\\\textbf{ವದ ಮುಕ್ತಿಂ ಕಥಂ ಭೂಯಾದನಾಯಾಸೇನ ಕರ್ಮಣಾ~।। ೫೭~।।} 
\end{verse}

\begin{verse}
\textbf{ಛಾಯಾ ವಿಷಾಯತೇ ಮಹ್ಯಂ ಜಲಂ ಮೇ ಜ್ವಲನಾಯತೇ~।}\\\textbf{ಕಾಲಾಯತೇ ಪಾದಪಶ್ವ ಬುದ್ಧಿರ್ಮೆ ಪವನಾಯತೇ~।। ೫೮~।। }
\end{verse}

\begin{verse}
\textbf{ಆಹಾರೊ ಮಲಮೂತ್ರಂ ಮೇ ಶವಾಸೃಙ್ ಮಾಂಸಭಕ್ಷಣಮ್~।}\\\textbf{ವ್ರಣಶೋಣಿತಪೂಯೇನ ತಥಾ ನಾರೀರಜೋಮಲಮ್~।। ೫೯~।। }
\end{verse}

\begin{verse}
\textbf{ಅತೋದ್ಯಾಶುಚಿನಾ ಕರ್ತುಂ ನ ಶಕ್ಯಾ ನಿಷ್ಕೃತಿರ್ಮಯಾ~।}\\\textbf{ದಯಾಂ ಕುರು ದಯಾ ಹೀನೇ ಸಾಧವೋ ಹಿ ಕೃಪಾಲವಃ~।। ೬೦~।।}
\end{verse}

ಈ ನಿರ್ಜನವಾದ ಕಾಡಿನಲ್ಲಿ ಎಷ್ಟು ವರ್ಷಗಳನ್ನು ಕಳೆದಿದ್ದೇನೆಯೋ ತಿಳಿಯದು. ಹೆಚ್ಚು ಆಯಾಸವಿಲ್ಲದ ಯಾವ ಕರ್ಮದಿಂದ ನನಗೆ ಈ ಜನ್ಮದಿಂದ ಬಿಡುಗಡೆಯಾಗುವುದೆಂದು ತಿಳಿಸಿರಿ. ನನಗೆ ನೆರಳು ವಿಷದಂತೆ ಕಾಣುತ್ತದೆ. ನೀರು ಬೆಂಕಿಯಂತೆ ತೋರುತ್ತದೆ. ಮರಗಿಡಗಳು ಮೃತ್ಯು ದೇವತೆಯಂತೆ ಕಾಣುತ್ತವೆ. ನನ್ನ ಬುದ್ಧಿಯು ಗಾಳಿಯಂತೆ ಇರುತ್ತದೆ. ನನ್ನ ಆಹಾರವು ಮಲಮೂತ್ರಗಳೇ. ಶವದ ರಕ್ತ-ಮಾಂಸಗಳು, ದೇಹದಲ್ಲಿರುವ ವ್ರಣಗಳಿಂದ ಬರುವ ಕೀವು ಮೊದಲಾದವುಗಳು, ಸ್ತ್ರೀಯರ ರಜೋಮಲ ಇವುಗಳು ನನಗೆ ಪ್ರಿಯವಾದ ವಸ್ತುಗಳು, ಇಂತಹ ಅಶುಚಿಯಾದ ನಾನು ನನ್ನ ಈ ಜನ್ಮ ನಿವೃತ್ತಿಗೆ ಮಾಡಬೇಕಾದ ಕರ್ಮವನ್ನು ಆಚರಿಸಲು ಅಸಮರ್ಥ, ದಯಾಳುಗಳಾದ ನೀವು ನನ್ನ ಮೇಲೆ ಕೃಪೆಮಾಡಿರಿ. ಸಾಧುಗಳು ಸ್ವಭಾವದಿಂದಲೇ ಕೃಪಾಳುಗಳು.

\begin{verse}
\textbf{ಅಥಾತಿವಿಸ್ಮಿತೋ ದೃಷ್ಟ್ವಾ ಮುದ್ಗಲೋ ಬ್ರಹ್ಮರಾಕ್ಷಸಮ್~।}\\\textbf{ಜಗಾದ ಪರಯಾ ವಾಣ್ಯಾ ಮಾ ಭೈಷೀರಿತಿ ತಂ ಸುಧೀಃ~।। ೬೧~।।} 
\end{verse}

\begin{verse}
\textbf{ಇತ್ಯ ತೊ ನಾತಿದೂರೇ ತು ಕಾವೇರೀ ನಾನು ವೈ ನದೀ~।}\\\textbf{ತಸ್ಯಾಮಹಂ ಕರಿಷ್ಯಾಮಿ ಮಾಘೇ ಮಾಸ್ಯರುಣೋದಯೇ~।। ೬೨~।। }
\end{verse}

\begin{verse}
\textbf{ಪ್ರಾತಃಸ್ಮಾನಂ ಮಾಘಮಾಸೇ ಪ್ರೀತ್ಯೈ ಮಾಧವಸಂಜ್ಞಿತೇ~।}\\\textbf{ತ್ವಂ ಚಾಗಚ್ಛ ಮಯಾ ಸಾರ್ಧಂ ಕರಿಷ್ಯಾಮಿ ಚ ನಿಷ್ಕೃತಿಮ್~।। ೬೩~।।}
\end{verse}

ಆಶ್ಚರ್ಯಚಕಿತನಾದ ಮುದ್ಗಲನು ಬ್ರಹ್ಮರಾಕ್ಷಸನಿಗೆ ಅಂತಃಕರಣಪೂರ್ವಕವಾಗಿ ಹೇಳಿದನು: “ರಾಕ್ಷಸನೇ, ಹೆದರಬೇಡ, ಇಲ್ಲಿಗೆ ಹತ್ತಿರದಲ್ಲಿಯೇ ಕಾವೇರಿ ನದಿ ಹರಿಯುತ್ತಿದೆ. ನಾನು ಆ ನದಿಯಲ್ಲಿ ಮಾಘಮಾಸದಲ್ಲಿ ಪ್ರಾತಃಕಾಲ ಮಾಧವನ ಪ್ರೀತ್ಯರ್ಥವಾಗಿ ಸ್ನಾನಮಾಡುತ್ತೇನೆ. ನೀನು ನನ್ನ ಸಂಗಡ ಅಲ್ಲಿಗೆ ಬಾ, ಈ ಜನ್ಮದಿಂದ ಬಿಡುಗಡೆ ಮಾಡುತ್ತೇನೆ.

\begin{verse}
\textbf{ಆವಾಹ್ಯ ತ್ವಾಂ ಕುಶಗ್ರಂಥೌ ತೇನ ಸಾರ್ಧಂ ನಿಶಾಚರ~।}\\\textbf{ಯತಃ ಸಾಕ್ಷಾತ್ತ್ವಯಾ ಕರ್ತುಂ ನ ಶಕ್ಯಮಿತಿ ಚಾಬ್ರವೀತ್~।। ೬೪~।।}
\end{verse}

ನಿನಗೆ ಸ್ವತಃ ಸ್ನಾನಮಾಡಲು ಸಾಮರ್ಥ್ಯವಿಲ್ಲವಾದಕಾರಣ, ನಿನ್ನನ್ನು ದರ್ಭೆಯಲ್ಲಿ ಆವಾಹಿಸಿ ಆ ದರ್ಭೆಯ ಸಹಿತ ನಾನು ಸ್ನಾನಮಾಡುತ್ತೇನೆ.

\begin{verse}
\textbf{ಇತ್ಯಾಶ್ವಾಸ್ಯ ತತಸ್ತೇನ ಸಾಕಂ ಕೀನಾಶಪಾಲಿತಾಮ್~।}\\\textbf{ದಿಶಂ ಯಯೌ ಮುನೀಂದ್ರಸ್ತೈಃ ಕ್ರಮಾತ್ ಕತಿಪಯೈರ್ದಿನೈಃ~।। ೬೫~।। }
\end{verse}

\begin{verse}
\textbf{ಕಾವೇರೀಂ ಪ್ರಾಪ್ಯ ಮಾಘೇ ತು ಪ್ರಾತಃಸ್ಮಾನಂ ವಿಧಾಯ ಚ~।}\\\textbf{ಆವಾಹಿತ ಕುಶಗ್ರಂಥಿಸಹಿತಃ ಸತ್ಯಭಾಷಣಃ~।। ೬೬~।। }
\end{verse}

\begin{verse}
\textbf{ದೇವರ್ಷಿಪಿತೄನುದ್ದಿಶ್ಯ ದತ್ತವಾನುದಕಾಂಜಲೀನ್~।}\\\textbf{ಮಾಸಮೇಕಂ ಚಕಾರೈವಂ ಕುಶಗ್ರಂಥಿ ಸಮನ್ವಿತಃ~।। ೬೭~।।}
\end{verse}

ಹೀಗೆ ರಾಕ್ಷಸನನ್ನು ಸಮಾಧಾನಪಡಿಸಿ ಮುದ್ಗಲನು ಅವನೊಡನೆ ದಕ್ಷಿಣ ದಿಕ್ಕಿಗೆ ಪ್ರಯಾಣಮಾಡಿ ಕೆಲವು ದಿನಗಳ ನಂತರ ಕಾವೇರಿ ನದಿಯನ್ನು ತಲುಪಿ ಅದರಲ್ಲಿ ಮಾಘಮಾಸದಲ್ಲಿ ಪ್ರಾತಃಸ್ನಾನವನ್ನು ಆವಾಹನೆಮಾಡಿದ ದರ್ಭೆಯ ಸಹಿತ ಸ್ನಾನಮಾಡಿದನು. ಸ್ನಾನಾನಂತರದಲ್ಲಿ ನಿತ್ಯವೂ ಆ ಸತ್ಯವನ್ನೇ ನುಡಿಯುವ ಮುದ್ಗಲನು ದೇವತೆಗಳು, ಋಷಿಗಳು, ಪಿತೃಗಳ ತೃಪ್ತಿಗೋಸ್ಕರ ಜಲತರ್ಪಣವನ್ನು ಕೊಟ್ಟನು. ಕುಶಗ್ರಂಥಿಯಿಂದ ಸಹಿತನಾಗಿ ಹೀಗೆ ಒಂದು ತಿಂಗಳಕಾಲ (ಮಾಘ ಮಾಸ ಪೂರ್ತಿ) ಮಾಡಿದನು.

\begin{verse}
\textbf{ತತೋsಭೂತ್ ಪರಿಧೇರ್ಮುಕ್ತಿಃ ಕುಶಗ್ರಂಥೇಶ್ಚ ಮಜ್ಜನಾತ್~।}\\\textbf{ಕಿಂ ಸಾಕ್ಷಾತ್ ಕೃತಮಾಘಸ್ಯ ಮುಕ್ತಿರಿತ್ಯಥ ಕೋ ವದೇತ್~।। ೬೮~।।} 
\end{verse}

\begin{verse}
\textbf{ತಸ್ಮಾನ್ಮಾಘಸ್ಯ ಮಾಹಾತ್ಮ್ಯಂ ಕೋ ವದೇತ್ ಭುವನತ್ರಯೇ~।}\\\textbf{ಗುರುನಿಂದಾಸಮಂ ಪಾಪಂ ನ ಕ್ವಾಪಿ ಶ್ರೂಯತೇ ಕ್ವಚಿತ್~।। ೬೯~।।}
\end{verse}

ಇದರಿಂದ ಪರಿಧಿಗೆ ಬ್ರಹ್ಮರಾಕ್ಷಸ ಜನ್ಮದಿಂದ ಬಿಡುಗಡೆಯಾಯಿತು. ಕುಶ ಗ್ರಂಥಿಗಳನ್ನು ಕಾವೇರಿಯಲ್ಲಿ ಮುಳುಗಿಸುವ ಮಾತ್ರದಿಂದಲೇ ಮುಕ್ತಿ ಸಿಕ್ಕರೆ, ಸಾಕ್ಷಾತ್ತಾಗಿ ಸ್ನಾನಮಾಡುವುದರಿಂದ ಮುಕ್ತಿ ದೊರೆಯುವುದೆಂದು ಹೇಳಲೇನಿದೆ! ಮಾಘಸ್ನಾನದ ಮಾಹಾತ್ಮ್ಯೆಯನ್ನು ವರ್ಣನೆ\-ಮಾಡುವ ಸಮರ್ಥನು ಮೂರು ಲೋಕಗಳಲ್ಲಿ ಯಾರಿದ್ದಾರೆ? ಗುರುನಿಂದನೆಗೆ ಸಮಾನವಾದ ಪಾಪವು ಎಲ್ಲಿಯೂ ಕೇಳಲ್ಪಟ್ಟಿಲ್ಲ.

\begin{verse}
\textbf{ಗುರೋರ್ನಿಂದಾ ಹರೇರ್ನಿಂದಾ ತಥಾ ಸಚ್ಛಾ ಸ್ತ್ರದೂಷಣಮ್~।}\\\textbf{ದೇವಸ್ವಹರಣಂ ಚೈವ ಬ್ರಹ್ಮಸ್ವಹರಣಂ ತಥಾ~।। ೭೦~।।} 
\end{verse}

\begin{verse}
\textbf{ತಟಾಕಸೇತುಚ್ಛೇದಶ್ಚ ಬ್ರಹ್ಮಹತ್ಯಾದಿಕಾನಿ ಷಟ್~।}\\\textbf{ತಸ್ಮಾತ್‌ಪಾಪಂ ತಾದೃಗಪಿ ಮಾಘಸ್ನಾನಂ ವ್ಯಪೋಹತಿ~।। ೭೧~।।}
\end{verse}

ಗುರುನಿಂದಾ, ಶ‍್ರೀಹರಿಯ ದೂಷಣೆ, ಸಚ್ಛಾಸ್ತ್ರಗಳ ದೂಷಣೆ, ದೇವರ ಹಣವನ್ನು ಅಪಹಾರಮಾಡುವುದು, ಬ್ರಾಹ್ಮಣರ ದ್ರವ್ಯವನ್ನು ಕದಿಯುವುದು, ಕೆರೆಗಳ ಏರಿಗಳನ್ನು ನಾಶಮಾಡುವುದು ಈ ಆರು ದುಷ್ಕರ್ಮಗಳು, ಬ್ರಹ್ಮಹತ್ಯೆಗಿಂತ ಅಧಿಕವಾದವುಗಳು. ಇಂತಹ ಪ್ರಬಲವಾದ ಪಾಪಗಳನ್ನೂ ಸಹ ಮಾಘಸ್ನಾನವು ನಾಶಮಾಡುತ್ತದೆ.

\begin{verse}
\textbf{ಇತಿ ಸಂಬೋಧ್ಯ ಪರಿಧಿಂ ತೇನ ಸಮ್ಯಕ್ ಪ್ರಪೂಜಿತಃ~।}\\\textbf{ಪಾಂಡ್ಯೇಶನಗರಂ ಪ್ರಾಪ್ಯ ಸ ಶಿಷ್ಯೈಃ ಸಹ ಮುದ್ಗಲಃ~।। ೭೨~।।}\\\textbf{ಪರಿಧಿಃ ಪಾಪನಿರ್ಮುಕ್ತೋ ಯಯೌ ಬದರಿಕಾಶ್ರಮಮ್~।।}
\end{verse}

ಹೀಗೆ ಪರಿಧಿಗೆ ಹೇಳಿ, ಅವನಿಂದ ಸತ್ಕರಿಸಲ್ಪಟ್ಟ ಮುದ್ಗಲನು ತನ್ನ ಶಿಷ್ಯರಿಂದ ಸಹಿತನಾಗಿ ಪಾಂಡ್ಯರಾಜನ ಪಟ್ಟಣಕ್ಕೆ ಹೋದನು. ಪಾಪರಹಿತನಾದ ಪರಿಧಿಯು ಬದರಿಕಾಶ್ರಮಕ್ಕೆ ಹೋದನು.

\begin{verse}
\textbf{ಯದುಕ್ತಂ ಹರಿಣಾ ಪೂರ್ವಂ ಮಹಾಲಕ್ಷ್ಮೈ ಪಯೋನಿಧೌ~।। ೭೩~।।}\\\textbf{ತದೇತತ್ಕಥಿತಂ ಪುತ್ರ ಮಾಘಮಾಹಾತ್ಮ್ಯ ಮುತ್ತಮಮ್~। }
\end{verse}

\begin{verse}
\textbf{ಮಯಾ ಚಾಲೋಡಿತಾಃ ಸರ್ವೇ ಧರ್ಮಾಃ ಶಾಸ್ತ್ರೇಷು ಚೋದಿತಾಃ~।। ೭೪~।।}\\\textbf{ನ ಮಾಘಸ್ನಾನಸದೃಶೋ ಧರ್ಮೋ ದೃಷ್ಟಃ ಕದಾಚನ~।}
\end{verse}

ಶ‍್ರೀಹರಿಯು ಹಿಂದೆ ಕ್ಷೀರಸಮುದ್ರದಲ್ಲಿ ಮಹಾಲಕ್ಷ್ಮೀದೇವಿಗೆ ಹೇಳಿದ ಉತ್ತಮವಾದ ಮಾಘಮಾಹಾತ್ಮ್ಯೆಯನ್ನು, ನಾರದನೇ, ನಿನಗೆ ನಾನು ಹೇಳಿರುತ್ತೇನೆ. ನಾನು ಸಮಸ್ತಶಾಸ್ತ್ರ\break ಗಳನ್ನೂ, ಧರ್ಮಗಳನ್ನೂ ಚೆನ್ನಾಗಿ ವಿಮರ್ಶಿಸಿ ನೋಡಿರುತ್ತೇನೆ. ಮಾಘಸ್ನಾನಕ್ಕೆ ಸಮಾನವಾದ ಸತ್ಕರ್ಮ ಇನ್ನೊಂದು ಕಂಡಿಲ್ಲ.

\begin{verse}
\textbf{ಸರ್ವೇ ಧರ್ಮಾಶ್ಚ ಯಜ್ಞಾಶ್ಚ ದಾನಾನಿ ಚ ವ್ರತಾನಿ ಚ~।। ೭೫~।।} 
\end{verse}

\begin{verse}
\textbf{ತುಲಾಮಾರೋಪಿತಾಃ ಪೂರ್ವಂ ಮಾಘಸ್ನಾನಂ ತಥೈಕತಃ~।}\\\textbf{ತದಾ ಸಕಲಧರ್ಮೇಭ್ಯೋ ಮಾಘಧರ್ಮೋಧಿಕೋಽಭವತ್~।। ೭೬~।।}
\end{verse}

ಎಲ್ಲಾ ಧರ್ಮಗಳ, ಯಜ್ಞಗಳ, ದಾನಗಳ, ವ್ರತಗಳ ಫಲಗಳನ್ನೂ ಮಾಘ ಸ್ನಾನಫಲವೊಂದನ್ನೇ ತೂಗಿನೋಡಲು ಸಕಲ ಧರ್ಮಗಳಿಗಿಂತಲೂ ಮಾಘಮಾಸದ ಧರ್ಮವೇ ಅಧಿಕವೆಂದು ಕಂಡಿತು.

\begin{verse}
\textbf{ಬಹುಕರ್ಮವಿಪಾಕೇನ ಜಾತಾ ಬಹುವಿಧಾ ಗತಿಃ~।}\\\textbf{ಮಾಘೇ ಮಾಸಿ ಜಲಸ್ಪರ್ಶಾತ್ ಮುಚ್ಯತೇ ಕೃತಕಿಲ್ಮಿಷಾತ್~।। ೭೭~।।}
\end{verse}

ಅನೇಕ ಜನ್ಮಗಳಲ್ಲಿ ಮಾಡಿದ ಪಾಪಗಳಿಂದ ಜನರು ನಾನಾವಿಧವಾದ ದುಃಖವನ್ನು ಅನುಭವಿಸುತ್ತಾರೆ. ಆದರೆ ಮಾಘಮಾಸದಲ್ಲಿ ಜಲಸ್ಪರ್ಶಮಾತ್ರದಿಂದಲೇ ಸಕಲ ಪಾಪಗಳಿಂದ ಬಿಡುಗಡೆ ಹೊಂದುತ್ತಾರೆ.

\begin{verse}
\textbf{ಯಥಾಷ್ಟಾವಿಂಶತಿ ಪುರಾ ಪಿಶಾಚಾ ಬಹುರೂಪಕಾಃ~।}\\\textbf{ಸಮಿತ್ ಕುಶಾದಾನಪರಾತ್ಸು ಜಯಜ್ಞಾದಿವ ವೈ ಮುನೇಃ~।। ೭೮~।।}
\end{verse}

ಅನೇಕ ವಿಚಿತ್ರ ರೂಪಗಳನ್ನುಳ್ಳ ಇಪ್ಪತ್ತೊಂದು ಪಿಶಾಚಿಗಳು, ಸಮಿತ್ತು ಕುಶಾದಿಗಳ ದಾನದಲ್ಲಿ ನಿರತನಾದ ಸುಯಜ್ಞನೆಂಬ ಬಾಹ್ಮಣನ ಸಹವಾಸದಿಂದ, ತನ್ನ ಪಾಪಗಳನ್ನು ಕಳೆದುಕೊಂಡುವು. ಅದರಂತೆ ಮಾಘಮಾಸದಲ್ಲಿ ಕೇವಲ ಜಲಸ್ಪರ್ಶದಿಂದ ಸಕಲ ಪಾಪಗಳೂ ನಾಶಹೊಂದುತ್ತವೆ.

\begin{center}
ಇತಿ ಶ‍್ರೀ ವಾಯುಪುರಾಣೇ ಮಾಘಮಾಸಮಾಹಾತ್ಮ್ಯೇ ನವಮೋsಧ್ಯಾಯಃ 
\end{center}

\begin{center}
ಶ‍್ರೀ ವಾಯುಪುರಾಣಾಂತರ್ಗತ ಮಾಘಮಾಸ ಮಾಹಾತ್ಮ್ಯೆಯಲ್ಲಿ \\ ಒಂಬತ್ತನೇ ಅಧ್ಯಾಯವು ಸಮಾಪ್ತಿಯಾಯಿತು.
\end{center}

\newpage

\section*{ಅಧ್ಯಾಯ\enginline{-}೧೦}

\emptypage

\begin{flushleft}
\textbf{ನಾರದ ಉವಾಚ:\enginline{-} }
\end{flushleft}

\begin{verse}
\textbf{ಸುಯಜ್ಞಃ ಕಸ್ಯ ಶಿಷ್ಯೋಽಭೂತ್ ಕುತ್ರಾಸೀತ್ಸ ಮುನೀಶ್ವರಃ~।}\\\textbf{ಪಿಶಾಚಾನಾಂ ಕಥಂ ತೇಷಾಂ ಸಂಗತಿರ್ಭಾವಿತಾತ್ಮನಃ~।। ೧~।। }
\end{verse}

\begin{verse}
\textbf{ಕಿಂ ವಾಽಕರೋತ್ ಸುಯಜ್ಞೋಽಪಿ ತೇಷಾಂ ನಿಷ್ಕೃತಿಕಾರಣಮ್~।}\\\textbf{ಶಿಷ್ಯಾಯ ಶ್ರದ್ಧಧಾನಾಯ ಶ್ರೋತುಂ ಕೌತೂಹಲಾಯ ಚ~।। ೨~।।}
\end{verse}

\begin{verse}
\textbf{ಪುತ್ರಾಯೈ ತಚ್ಚಿತ್ರಕಾರ್ಯಂ ಭೂಯೋ ವಿಸ್ತರತೋ ವದ~।}
\end{verse}

\begin{flushleft}
ನಾರದರು ಪ್ರಶ್ನೆ ಮಾಡುತ್ತಾರೆ-
\end{flushleft}

ಸುಯಜ್ಞ ನೆಂಬುವನು ಯಾರು? ಅವನು ಯಾರ ಶಿಷ್ಯ? ಅವನು ಎಲ್ಲಿ ವಾಸಮಾಡುತ್ತಿದ್ದನು? ಪಿಶಾಚಿಗಳ ಸಂಗಡ ಅವನ ಸಹವಾಸವು ಹೇಗಾಯಿತು? ಅದರ ನಿವಾರಣೆಗಾಗಿ ಏನು ಮಾಡಿದನು? ಇವೆಲ್ಲವನ್ನು ನಿಮ್ಮ ಪುತ್ರನಾದ, ಶಿಷ್ಯನಾದ, ಶ್ರದ್ದೆಯಿಂದ ಕೂಡಿದ, ಕೇಳಲು ಕುತೂಹಲದಿಂದ ಕೂಡಿದ ನನಗೆ ವಿಸ್ತಾರವಾಗಿ ಹೇಳಿರಿ.

\begin{flushleft}
\textbf{ಬ್ರಹ್ಮೋವಾಚ:\enginline{-}}
\end{flushleft}

\begin{verse}
\textbf{ಪುರಾ ಚಾಕ್ಷುಷಮನ್ವಂತೇ ದ್ವಾದಶಾಖ್ಯೆ ಯುಗೇಽಪಿ ಚ~।। ೩~।।}\\\textbf{ಜಾಬಾಲೀರ್ನಾಮತಶ್ಚಾಸೀತ್ ನರನಾರಾಯಣಾಶ್ರಮೇ~।}\\\textbf{ಸುಯಜ್ಞೋ ನಾಮ ತಸ್ಯಾಸೀತ್ತನಯೋ ವೇದಪಾರಗಃ~।। ೪~।।}
\end{verse}

\begin{flushleft}
ಬ್ರಹ್ಮದೇವರು ನುಡಿಯುತ್ತಾರೆ-
\end{flushleft}

ಹಿಂದೆ ಚಾಕ್ಷುಷಮನ್ವಂತರದ ಹನ್ನೆರಡನೆಯ ಯುಗದಲ್ಲಿ ನರನಾರಾಯಣಾಶ್ರಮದಲ್ಲಿ ಜಾಬಾಲಿ ಎಂಬ ಋಷಿಗಳಿದ್ದರು. ಅವರ ಪುತ್ರನೇ ಸುಯಜ್ಞ ವೇದ ಪಾರಂಗತ.

\begin{verse}
\textbf{ಸ ಚೋಪನೀತೋ ವಿದ್ಯಾರ್ಥೀ ಸುಮನಾಃ ಸಹವಾರ್ಷಿಕಃ~।}\\\textbf{ಶಾಂಡಿಲ್ಯ ಗುರುಮಾಸಾದ್ಯ ಚಕಾರಾಧ್ಯ ಯನಂ ತತಃ~।। ೫~।। }
\end{verse}

\begin{verse}
\textbf{ಏವಂ ಸ ದಶ ವರ್ಷಾಣಿ ಗುರುಶುಶ್ರೂಷಣೇ ರತಃ~।}\\\textbf{ಸಮಿತ್ಕು ಶಪಲಾಶಾರ್ಥಂ ಕದಾಚಿದಟವೀಂ ಯಯೌ~।। ೬~।।}
\end{verse}

ಒಳ್ಳೆಯ ನಡತೆಯುಳ್ಳ ಸುಯಜ್ಞನಿಗೆ ಏಳುವರ್ಷ ವಯಸ್ಸಾದಾಗ ಉಪನಯನ\-ವಾಯಿತು. ಶಾಂಡಿಲ್ಯರೆಂಬ ಗುರುಗಳಲ್ಲಿ ವಿದ್ಯಾರ್ಥಿಯಾಗಿ ವೇದಾಧ್ಯಯನವನ್ನು ಮಾಡಿದನು. ಹತ್ತು ವರ್ಷಗಳಕಾಲ ಗುರುಗಳ ಸೇವೆಯಲ್ಲಿ ಆಸಕ್ತನಾಗಿದ್ದನು. ಒಂದು ದಿನ ಸುಯಜ್ಞನು ಸಮಿತ್ತು, ದರ್ಭೆಗಳನ್ನು ತರಲು ಅರಣ್ಯಕ್ಕೆ ಹೋದನು.

\begin{verse}
\textbf{ತತ್ರಾಶ್ವತ್ಥಂ ಮಹಾವೃಕ್ಷಂ ಸಪ್ತಶಾಖಾಸಮನ್ವಿತಮ್~।}\\\textbf{ನಿರ್ಜಲೇ ವಿಜನೇ ದೇಶೇ ದದರ್ಶಾಂಬರಚುಂಬಿತಮ್~।। ೭~।।}
\end{verse}

ನಿರ್ಜನವಾದ, ನೀರಿಲ್ಲದ ಒಂದು ಸ್ಥಳದಲ್ಲಿ ಏಳು ಶಾಖೆಗಳಿದ್ದ, ಬಹು ಎತ್ತರವಾದ ಅಶ್ವತ್ಥವೃಕ್ಷ ಒಂದನ್ನು ನೋಡಿದನು.

\begin{verse}
\textbf{ತತಃ ಸಪ್ತಾದ್ಭುತಾಃ ಕ್ರೂರಾಸ್ತೇ ದೃಷ್ಟಾ ಬ್ರಹ್ಮರಾಕ್ಷಸಾಃ~।}\\\textbf{ದೃಷ್ಟ್ವಾ ಭಕ್ಷಯಿತುಂ ಬಾಲಂ ಆಜಗ್ಮುಃ ತ್ವರಯಾನ್ವಿತಾಃ~।। ೮~।।}
\end{verse}

ಅಲ್ಲಿ ಕ್ರೂರರಾದ, ಭಯಂಕರರಾದ ಏಳು ಜನ ಬ್ರಹ್ಮರಾಕ್ಷಸರು ಸುಯಜ್ಞನನ್ನು ತಿನ್ನಲು ಅತಿ ತ್ವರೆಯಿಂದ ಬಂದರು.

\begin{verse}
\textbf{ನೀಲಾಂಬರಸಮಪ್ರಖ್ಯಾ ಭಯದಾ ಲೋಕಕರ್ಪಟಾಃ~।}\\\textbf{ದಂಷ್ಟ್ರಾಕರಾಲವದನಾಃ ಶುಷ್ಕ ಕಂಠೋಷ್ಠ ತಾಲವಃ~।। ೯~।।} 
\end{verse}

\begin{verse}
\textbf{ಘೂರ್ಣಯದ್ಭೂಮ್ರತಾಮ್ರಾಕ್ಷಾ ಊರ್ಧ್ವಕೇಶಾ ಮಹೋದರಾಃ~।}\\\textbf{ತಾನ್ ದೃಷ್ಟ್ವಾ ಭೀತಿಭೀತಃ ಸನ್ ಸ ಪ್ರಾಚೀಂ ದುದ್ರುವೇ ಭಯಾತ್~।।}
\end{verse}

ಆ ಬಹ್ಮರಾಕ್ಷಸರ ಮೈ ಬಣ್ಣವು ನೀಲಿಬಟ್ಟೆಯಿದ್ದಂತೆ ಇತ್ತು. ಉದ್ದವಾದ ಕೋರೆ ಹಲ್ಲುಗಳನ್ನುಳ್ಳ ಬಾಯಿ, ಭಯಂಕರ ರೂಪ, ತುಟಿಕಂಠಗಳು ಒಣಗಿದ್ದುವು, ಕೆಂಪು ಕಣ್ಣುಗಳು, ಕೆದರಿದ ಕೂದಲು, ದೊಡ್ಡ ಹೊಟ್ಟೆ ಹೀಗಿದ್ದ ಬ್ರಹ್ಮರಾಕ್ಷಸರನ್ನು ನೋಡಿ ಬಾಲಕನಾದ ಸುಯಜ್ಞನು ಭಯದಿಂದ ಪೂರ್ವದಿಕ್ಕಿಗೆ ಓಡಿದನು.

\begin{verse}
\textbf{ಅನುಜಗ್ಮುರ್ಭಕ್ಷಯಿತುಂ ತಂ ಚ ತೇ ಭಯವಿಹ್ವಲಮ್~।}\\\textbf{ರುರುಧುಸ್ತೂರ್ಣಮಾಗತ್ಯ ಪ್ರಾಚೀಂ ಮಾರ್ಗಂ ತತೋsರ್ಭಕಃ~।। ೧೧~।। }
\end{verse}

\begin{verse}
\textbf{ದುದ್ರಾವ ದಕ್ಷಿಣಾಮಾಶಾಂ ತೇಭ್ಯಃ ಪ್ರಾಣಪರೀಪ್ಸುಕಃ~।}\\\textbf{ತತ್ರಾಪಶ್ಯನ್ವಟಂ ಕಂಚಿಚ್ಛತಶಾಖಿನಮುಚ್ಛ್ರಿತಮ್~।। ೧೨~।। }
\end{verse}

\begin{verse}
\textbf{ಅಟಿಮಾನಂ ತತೋ ದೃಷ್ಟ್ವಾ ವಟಾತ್ತ ಸ್ಮಾದಥಾಯಯುಃ~।}\\\textbf{ಪಿಶಾಚಾಃ ಸಪ್ತ ಕೂಷ್ಮಾಂಡಾ ಭುಜಾದಾಗತಮೂರ್ಧಕಾಃ~।। ೧೩~।।}
\end{verse}

ಭಯಗ್ರಸ್ತನಾದ ಸುಯಜ್ಞನನ್ನು ಅಟ್ಟಿಸಿಕೊಂಡುಹೋಗಿ, ಆ ಬ್ರಹ್ಮ ರಾಕ್ಷಸರು ಪೂರ್ವದಿಕ್ಕಿನಲ್ಲಿ ದಾರಿಯನ್ನು ಅಡ್ಡ ಕಟ್ಟಿದರು. ಅಲ್ಲಿಂದ ಸುಯಜ್ಞನು ಪ್ರಾಣರಕ್ಷಣೆಗಾಗಿ ದಕ್ಷಿಣಕ್ಕೆ ಓಡಿದನು. ಅಲ್ಲಿ ನೂರಾರು ಶಾಖೆಗಳಿಂದ ಕೂಡಿದ ದೊಡ್ಡ ಆಲದ ಮರವೊಂದನ್ನು ಕಂಡನು. ಆ ಆಲದಮರದಿಂದ ಕೂಷ್ಮಾಂಡವೆಂಬ ಏಳು ಪಿಶಾಚಿಗಳು ಬಂದವು. ಅವುಗಳಿಗೆ ಭುಜಗಳಲ್ಲಿಯೇ ತಲೆಗಳಿದ್ದವು.

\begin{verse}
\textbf{ಸ್ರವದ್ರಕ್ತ ಕಬಂಧಾಶ್ಚ ತಾನ್ ದೃಷ್ಟಾತೀವ ವಿಹ್ವಲಃ~।}\\\textbf{ಪ್ರತೀಚೀಂ ದಿಶಮಾಪೇದೇ ದ್ರುತಂ ಪ್ರಾಣಪರೀಪ್ಸು ಕಃ~।। ೧೪~।।}
\end{verse}

ರಕ್ತವನ್ನು ಸುರಿಸುತ್ತಿದ್ದ ಆ ಕೂಷ್ಮಾಂಡಗಳನ್ನು ಕಂಡು ಭಯಪಟ್ಟ ಸುಯಜ್ಞನು ಪಶ್ಚಿಮದಿಕ್ಕಿಗೆ ಓಡಿದನು.

\begin{verse}
\textbf{ಶಾಲ್ಮಲೀಂ ಯೋಜನಾಯಾಮಾಂ ದೃಷ್ಟ್ವಾ ತಾಮಭಿಪದ್ಯತ~।}\\\textbf{ತಸ್ಮಾದಪ್ಯಾಯಯುಃ ಸಪ್ತ ಪಿಶಾಚಾ ಡಾಕಿನೀಗಣಾಃ~।। ೧೫~।। }
\end{verse}

\begin{verse}
\textbf{ಜಾನುದೇಶಶಿರಾಃ ಕಾಶ್ಚಿದೂರುದೇಶಶಿರಾಃ ಪರಾಃ~।}\\\textbf{ತಥೈವ ಭುಜಮೂರ್ಧಾ ಚ ಕಟಿಮೂರ್ಧಾಸ್ತಥಾಪರಾಃ~।। ೧೬~।।} 
\end{verse}

\begin{verse}
\textbf{ತಥಾ ಪ್ರಕೋಷ್ಠಮೂರ್ಧಾಶ್ಚ ಸ್ತನಮೂರ್ಧಾಸ್ತಥಾಪರಾಃ~।}\\\textbf{ಪಾರ್ಷ್ಣಿಮೂರ್ಧಾಶ್ಚ ಸಪ್ತೈತೇ ಡಾಕಿನೀಯೋನಿಮಾಶ್ರಿತಾಃ~।। ೧೭~।।}
\end{verse}

ಆ ಪಶ್ಚಿಮದಿಕ್ಕಿನಲ್ಲಿ ಒಂದು ಯೋಜನ ದೂರದಲ್ಲಿ ಒಂದು ದೊಡ್ಡ ಬೂರಲ ಮರವನ್ನು ಸಮೀಪಿಸಿದನು. ಆ ಮರದಿಂದ ಡಾಕಿನೀಗಣಕ್ಕೆ ಸೇರಿದ ಏಳು ಪಿಶಾಚಿಗಳು ಬಂದವು. ಮೊಳಕಾಲಿನಲ್ಲಿ ತಲೆಯಿದ್ದದ್ದು ಒಂದು, ತೊಡೆಯಲ್ಲಿ ತಲೆಯಿದ್ದದ್ದು ಒಂದು, ಭುಜದಲ್ಲಿ ತಲೆ ಮತ್ತೊಂದಕ್ಕೆ, ಸೊಂಟದಲ್ಲಿ ತಲೆ ಇನ್ನೊಂದಕ್ಕೆ, ಮೊಳಕೈಯಲ್ಲಿ ತಲೆ ಇನ್ನೊಂದಕ್ಕೆ, ಸ್ತನದಲ್ಲಿ ತಲೆ ಇನ್ನೊಂದಕ್ಕೆ, ಪೃಷ್ಟ ಭಾಗದಲ್ಲಿ ತಲೆ ಇನ್ನೊಂದು ಪಿಶಾಚಿಗೆ ಹೀಗೆ ಏಳು ಕರಾಳರೂಪದ ಪಿಶಾಚಿಗಳು ಬಂದವು.

\begin{verse}
\textbf{ದೃಷ್ಟ್ವಾ ತಾನದ್ಭು ತಾಕಾರಾನುದೀಚೀಂ ದುದ್ರುವೇ ಭಯಾತ್~।}\\\textbf{ತತ್ರಾಭಯಂ ಮಹಾವೃಕ್ಷಂ ದೃಷ್ಟ್ವಾ ಚ ಪ್ರತಿಪದ್ಯ ಚ~।। ೧೮~।।} 
\end{verse}

\begin{verse}
\textbf{ತಸ್ಮಾದಾಯಯತುರ್ವೃದ್ಧೌ ವೃಕವ್ಯಾಘ್ರಸಮಾನಸೌ~।}\\\textbf{ಆಗಚ್ಛಂತೌ ದಂಡಧರೌ ದೃಷ್ಟ್ವೇಶಾನೀಂ ದಿಶಾಂ ಯಯೌ~।। ೧೯~।।}
\end{verse}

ಡಾಕಿನೀ ಪಿಶಾಚಿಗಳಿಗೆ ಭಯಪಟ್ಟು ಸುಯಜ್ಞನು ಉತ್ತರದಿಕ್ಕಿಗೆ ಓಡಿದನು. ಅಲ್ಲಿ ಅತಿ ದೊಡ್ಡದಾದ ಮರದ ಬಳಿಗೆ ಬಂದನು. ಆ ಮರದಿಂದ ಎರಡು ವೃದ್ಧರಾದ, ತೋಳ-\-ಹುಲಿಗಳಂತೆ ಮುಖಗಳನ್ನುಳ್ಳ, ಕೈಯಲ್ಲಿ ದಂಡಗಳನ್ನು ಹಿಡಿದಿದ್ದ ಪಿಶಾಚಿಗಳು ಬರುತ್ತಿದ್ದುದನ್ನು ನೋಡಿ ಸುಯಜ್ಞನು ಈಶಾನ್ಯ ದಿಕ್ಕಿಗೆ ಓಡಿದನು.

\begin{verse}
\textbf{ದೃಷ್ಟ್ವಾ ಕದಂಬಂ ತತ್ರಾಪಿ ತನ್ಮೂಲಂ ಪ್ರತ್ಯ ಪದ್ಯತ~।}\\\textbf{ತಸ್ಮಾಚ್ಚ ತ್ರಯ ಆಜಗ್ಮುಃ ಗೃಧ್ರಶೀರ್ಷಾ ಭಯಂಕರಾಃ~।। ೨೦~।। }
\end{verse}

\begin{verse}
\textbf{ನರಾಕಾರಾ ದೀರ್ಘದೇಹಾ ಶಾಕಿನೀಯೋನಿಮಾಶ್ರಿತಾಃ~।}\\\textbf{ತಮರ್ಭಕಂ ಭಕ್ಷಯಿತುಂ ತೀಷ್ಣತುಂಡಾ ಭಯಂಕರಾಃ~।। ೨೧~।।}
\end{verse}

ಈಶಾನ್ಯದಲ್ಲಿ ಕದಂಬವೃಕ್ಷವೊಂದನ್ನು ನೋಡಿ ಅದರ ಬಳಿ ಬಂದನು. ಆ ಮರದಿಂದ ಶಾಕಿನೀಯೋನಿಗೆ ಸೇರಿದ, ಮನುಷ್ಯರಂತೆ ದೇಹ, ಹದ್ದುಗಳಂತೆ ತಲೆಗಳನ್ನುಳ್ಳ, ಎತ್ತರವಾಗಿ ಭಯಂಕರರೂಪವುಳ್ಳ ಪಿಶಾಚಿಗಳು ಸುಯಜ್ಞನನ್ನು ತಿನ್ನಲು ಬಂದವು.

\begin{verse}
\textbf{ತಾನ್ ದೃಷ್ಟ್ವಾತೀವ ನಿರ್ವಿಣ್ಣಃ ಪ್ರಪೇದೇ ವಹ್ನಿಪಾಲಿತಾಮ್~।}\\\textbf{ತತ್ರೋದುಂಬರಮಾಸಾದ್ಯ ದ್ವಾವಪಶ್ಯದ್ಭಯಾನಕೌ~।। ೨೨~।। }
\end{verse}

\begin{verse}
\textbf{ಪ್ರೇತಾವಾಹಾರರಹಿತೌ ನಿರಂಗೌ ಮಾಂಸಪೇಶಿನೌ~।}\\\textbf{ತಾವಪ್ಯಾಜಗ್ಮ ತುರ್ದೀನಂ ಸ್ಪರ್ಶಾಶೌ ಭಕ್ಷಕಾಮಿನೌ~।। ೨೩~।।}
\end{verse}

ಅವರನ್ನು ನೋಡಿ ಸುಯಜ್ಞ ನು ಬಹಳ ಭಯಪಟ್ಟು ಆಗ್ನೇಯ ದಿಕ್ಕಿನಲ್ಲಿ ಓಡಿದನು. ಅಲ್ಲಿ ಒಂದು ಔದುಂಬರ ವೃಕ್ಷದ ಬಳಿಗೆ ಬಂದಾಗ ಎರಡು ಭಯಂಕರವಾದ ಪಿಶಾಚಿಗಳನ್ನು ನೋಡಿದನು. ಅವು ಪ್ರೇತಯೋನಿಯಲ್ಲಿದ್ದುವು, ಶರೀರವೆಲ್ಲ ರಕ್ತ-ಮಾಂಸಗಳ ಮುದ್ದೆ, ಆಹಾರವನ್ನು ಅಪೇಕ್ಷಿಸುತ್ತಾ ಸುಯಜ್ಞನನ್ನು ತಿನ್ನಲು ಬಂದವು.

\begin{verse}
\textbf{ತೌ ದೃಷ್ಟ್ವಾತೀವನಿರ್ವಿಣ್ಣೋ ದುದ್ರುವೇ ಚ ದಶೋ ದಿಶ~।}\\\textbf{ಧಾವತ್ಯೇವಂ ಮುನಿಸುತೇ ಪಿಶಾಚಾಸ್ತೇ ಪರಸ್ಪರಮ್~।। ೨೪~।। }
\end{verse}

\begin{verse}
\textbf{ಅಹಂ ಪುರೋ ಗಮಿಷ್ಯಾಮಿ ಮಾ ಸ್ಪೃಶತ್ವೇಷ ಮಾಮಕಃ~।}\\\textbf{ಮಮಾಯಂ ಚ ಮಮಾಯಂ ಚ ಮಮಾಯಮಿತಿ ಚಾಪರೇ~।। ೨೫~।।}
\end{verse}

ಮುನಿಪುತ್ರನು ಭಯಗ್ರಸ್ತವಾಗಿ ಹತ್ತು ದಿಕ್ಕುಗಳಲ್ಲಿಯೂ ಓಡುತ್ತಿರಲು ಆ ಪಿಶಾಚಿ ತಮ್ಮತಮ್ಮಲ್ಲಿಯೇ ಹೇಳಿಕೊಂಡವು: ನಾನು ಮೊದಲು ತಿನ್ನಲು ಹೋಗುತ್ತೇನೆ. ಇವನನ್ನು ಮತ್ತೆ ಯಾರೂ ಮುಟ್ಟಬಾರದು. ಇನ್ನೊಂದು ಪಿಶಾಚಿಯು ಇವನು ನನ್ನವನು ಎಂದಿತು.

\begin{verse}
\textbf{ಮಯಾ ತು ಪೂರ್ವಂ ದೃಷ್ಟೋಽಯಂ ಮಯಾ ಸ್ಪೃಷ್ಟಃ ಪುರಾ ತ್ವಯಮ್~।}\\\textbf{ಇತಿ ತೇಷಾಂ ಅಭೂದ್ಯುದ್ಧಂ ಮುಷ್ಟಾಮುಷ್ಟಿ ಕಚಾಕಚಿ~।। ೨೬~।।}
\end{verse}

\begin{verse}
\textbf{ನ ಕೇಚಿಚ್ಚ ಸ್ಪೃಶಂತ್ಯೇನಂ ಪರಸ್ಪರವಧೇಚ್ಛವಃ~।}\\\textbf{ತತೋ ನಿರ್ವೇದಮಾಪೇದೇ ಮುನಿಪುತ್ರೋ ಭಯಾಕುಲಃ~।। ೨೭~।।}
\end{verse}

ನಾನೇ ಇವನನ್ನು ಮೊದಲು ನೋಡಿದುದು, ನಾನೇ ಇವನನ್ನು ಮೊದಲು ಮುಟ್ಟಿದುದು, ಇನ್ನಾರೂ ಇವನನ್ನು ಮುಟ್ಟಬಾರದು ಎಂಬುದಾಗಿ ಜುಟ್ಟುಗಳನ್ನು ಹಿಡಿದುಕೊಂಡು ಪರಸ್ಪರವಾಗಿ ಯುದ್ಧವಾಯಿತು. ಮುಷ್ಟಿಗಳಿಂದ ಒಬ್ಬರನ್ನೊಬ್ಬರು ಹೊಡೆದರು. ಈ ರೀತಿ ಯುದ್ಧದಲ್ಲಿ ತೊಡಗಿದ್ದ ಆ ಪಿಶಾಚಿಗಳು ಒಬ್ಬರನ್ನೊಬ್ಬರು ಕೊಲ್ಲಬೇಕೆಂಬ ಇಚ್ಛೆಯಿಂದ ಸುಯಜ್ಞನನ್ನು ಯಾರೂ ಮುಟ್ಟಲಿಲ್ಲ. ಮುನಿಪುತ್ರನು ಬಹಳ ಭಯದಿಂದ ಇದ್ದನು.

\begin{verse}
\textbf{ಅನನ್ಯ ಶರಣಃ ಕ್ಷುಬ್ಧೋ ಭಯಾತ್ಕರ್ತವ್ಯ ಮೂಢಧೀಃ~।}\\\textbf{ಶರಣ್ಯಂ ಚಿಂತಯಾಮಾಸ ದುಃಖಾದ್ಬಾಷ್ಪಜಲಾಕುಲಃ~।। ೨೮~।।}
\end{verse}

ಅತ್ಯಂತ ಭಯಗೊಂಡ ಸುಯಜ್ಞನು ಏನು ಮಾಡಲೂ ತೋರದೆ ಶರಣು ಹೋಗಲು ಅನ್ಯರನ್ನು ಕಾಣದೆ ದುಃಖದಿಂದ ಕಣ್ಣೀರು ಸುರಿಸುತ್ತಾ ಸಾಕ್ಷಾತ್ ಶ‍್ರೀ ವಿಷ್ಣುವನ್ನು ಸ್ಮರಿಸಿ ಶರಣು ಹೊಂದಿದನು.

\begin{verse}
\textbf{ದಿಶೋ ನ ಜಾನೇ ನ ಲಭೇಯ ಶರ್ಮ\enginline{-}} \\\textbf{ಪಶ್ಯಾಮಿ ಕಂ ವಾ ಪುರುಷಂ ದಯಾಲುಮ್~।}\\\textbf{ನಾನ್ಯೋ ಹಿ ವಿಷ್ಣೋಃ ಪರಮೋಸ್ತಿ ದಾತಾ\enginline{-}} \\\textbf{ಸುಖಸ್ಯ ದುಃಖಸ್ಯ ವಿಮೋಕ್ಷಣಸ್ಯ~।। ೨೯~।। }
\end{verse}

\begin{verse}
\textbf{ತಸ್ಮಾದ್ವಿಷ್ಣುಂ ಶರಣ್ಯಂ ತಂ ಭಕ್ತವತ್ಸಲಮಚ್ಯುತಮ್~।}\\\textbf{ಯಸ್ಮಾಜ್ಜಗದಿದಂ ಜಾತಂ ರಕ್ಷಿತಂ ಭಕ್ಷಿತಂ ಯತಃ~।। ೩೦~।।} 
\end{verse}

\begin{verse}
\textbf{ತೇನೇದಮಖಿಲಂ ವ್ಯಾಪ್ತಂ ವಿಶ್ಚಂ ಸ್ಥಾವರಜಂಗಮಮ್~।}\\\textbf{ಕರ್ತಾ ಕಾರಯಿತಾ ವಿಷ್ಣುಃ ಪ್ರೇರಕಃ ಸಾಕ್ಷಿಚೇತನಃ~।। ೩೧~।।}
\end{verse}

“ಪ್ರಭುವೇ, ನನಗೆ ದಿಕ್ಕೇ ತೋಚದಂತೆ ಆಗಿದೆ; ನಿನ್ನನ್ನು ಹೊರತು ದಯಾಶಾಲಿಯಾದ ಮತ್ತೊಬ್ಬನನ್ನು ನಾನು ಹೊಂದಲಾರೆ; ಸುಖವನ್ನು ನೀಡಿ, ದುಃಖವನ್ನು ನಾಶಗೊಳಿಸುವವರಲ್ಲಿ ವಿಷ್ಣುವನ್ನು ಬಿಟ್ಟರೆ ಮತ್ತೆ ಯಾರೂ ಇಲ್ಲ; ಯಾರಿಂದ ಈ ಸತ್ಯಭೂತವಾದ ಜಗತ್ತು ಸೃಷ್ಟಿಸಲ್ಪಡುತ್ತದೆಯೋ, ಯಾರಿಂದ ರಕ್ಷಿಸಲ್ಪಡುತ್ತಿದೆಯೋ, ಕೊನೆಯಲ್ಲಿ ಯಾರಿಂದ ನಾಶಹೊಂದುತ್ತದೆಯೋ, ಅಂತಹ ಭಕ್ತವತ್ಸಲನಾದ, ನಾಶರಹಿತನಾದ ವಿಷ್ಣುವೇ ಶರಣ್ಯನು. ಸ್ಥಾವರಜಂಗಮಾತ್ಮಕವಾದ ಈ ಜಗತ್ತು ಆ ವಿಷ್ಣುವಿನಿಂದ ವ್ಯಾಪ್ತವಾಗಿದೆ. ಸರ್ವರಿಂದಲೂ ಸರ್ವಕಾರ್ಯಗಳನ್ನೂ ಮಾಡಿಸುವವನು ವಿಷ್ಣುವೇ, ಅನ್ಯರಲ್ಲ. ಸಮಸ್ತ ವ್ಯಾಪಾರಗಳನ್ನೂ ಅವನೇ ಮಾಡುವವನು; ಎಲ್ಲರ ಹೃದಯಗುಹೆಯಲ್ಲಿದ್ದು, ಪ್ರೇರಕನಾಗಿ ಎಲ್ಲವನ್ನೂ ನೋಡುತ್ತಿರುವವನು ಆ ವಿಷ್ಣುವೇ.

\begin{verse}
\textbf{ಶರಣ್ಯಂ ತಂ ಮಹಾವಿಷ್ಣುಂ ಪ್ರಪದ್ಯೇ ದುಃಖಮುಕ್ತಯೇ~।}\\\textbf{ಏವಂ ವಿಚಾರ್ಯಮಾಣಸ್ಯ ತತೋsಭೂದಶರೀರವಾಕ್~।। ೩೨~।।}
\end{verse}

ನನಗೆ ಬಂದಿರುವ ದುಃಖನಾಶಕ್ಕೆ ಅಂತಹ ವಿಷ್ಣುವನ್ನೇ ಶರಣು ಹೊಂದುತ್ತೇನೆ'' ಎಂಬುದಾಗಿ ಆಲೋಚನೆಯನ್ನು ಮಾಡುತ್ತಿದ್ದಾಗ ಅಶರೀರವಾಣಿ ನುಡಿಯಿತು.

\begin{verse}
\textbf{ಮುನಿಪುತ್ರ ನದೀಂ ಗಚ್ಚ ಜಪ ಮಂತ್ರಮುಪಾಂಶುಕಮ್~।}\\\textbf{ತನ್ಮಂತ್ರತೇಜಸಾ ದೂರೇ ತಿತ್ಯೇತೇ ಸುದಾರುಣಾಃ~।। ೩೩~।। }
\end{verse}

\begin{verse}
\textbf{ತೇಜಸಾ ಹತತೇಜಸ್ಕಾ ವಿವದಂತಃ ಪರಸ್ಪರಮ್~।}\\\textbf{ಏತಸ್ಮಿನ್ನಂತರೇ ತ್ವಂ ಚ ನಿಮಜ್ಯ ಚ ನದೀಜಲೇ~।। ೩೪~।।} 
\end{verse}

\begin{verse}
\textbf{ಉತ್ಥಾಯ ಪಯಸಾ ಸಿಂಚ ಪಿಶಾಚಾನಾಗತಾನಿಮಾನ್~।}\\\textbf{ತೇ ಮುಕ್ತಾ ಸ್ಯುರ್ನ ಸಂದೇಹಃ ತತಃ ಸುಖಮವಾಪ್ಸ್ಯಸಿ~।। ೩೫~।।}
\end{verse}

“ಮುನಿಪುತ್ರನೇ, ನದಿಗೆ ಹೋಗಿ ಅತಿ ರಹಸ್ಯವಾದ ಗುರೂಪದಿಷ್ಟವಾದ ಮಂತ್ರವನ್ನು ಜಪಿಸು. ಆ ಮಂತ್ರದ ಪ್ರಭಾವದಿಂದ ಈ ಪಿಶಾಚಿಗಳು ನಿನ್ನ ಬಳಿಗೆ ಬಾರದೇ ದೂರದಲ್ಲಿರುತ್ತವೆ. ಪಿಶಾಚಿಗಳು ತಮ್ಮತಮ್ಮಲ್ಲಿಯೇ ವಾದ ವಿವಾದ ಮಾಡುತ್ತಾ ಇರುತ್ತವೆ. ಆ ಕಾಲದಲ್ಲಿ ನೀನು ನದಿಯಲ್ಲಿ ಸ್ನಾನಮಾಡು. ನಂತರ ನದಿಯಿಂದ ಹೊರಗೆ ಬಂದು ಪಿಶಾಚಿಗಳಮೇಲೆ ನದಿ ನೀರನ್ನು ಸಿಂಪಡಿಸು. ಸಂದೇಹವಿಲ್ಲದೇ ಆ ಪಿಶಾಚಿಗಳು ತಮ್ಮ ಜನ್ಮದಿಂದ ಮುಕ್ತಿಯನ್ನು ಪಡೆಯುತ್ತವೆ; ನೀನೂ ಸಹ ಸುಖಿಯಾಗುವಿ.

\begin{verse}
\textbf{ಆಗತೋ ವಿಪ್ರ ಮಾಘೋಽಯಮಿತ್ಯುಕ್ತ್ವಾ ವಿರರಾಮ ಸಾ~।}
\end{verse}

ಈಗ ಮಾಘಮಾಸವು ಪ್ರಾಪ್ತವಾಗಿದೆ” ಹೀಗೆಂದು ಹೇಳಿ ಆ ಅಶರೀರವಾಣಿ ಸುಮ್ಮನಾಯಿತು.

\begin{verse}
\textbf{ತತಃ ಪಿಶಾಚಾವೃತಯೇವ ವಿಪ್ರೋ}\\\textbf{ನೃಸಿಂಹಬೀಜಾಕ್ಷರಮಂತ್ರಮುಗ್ರಮ್~।। ೩೬~।।} 
\end{verse}

\begin{verse}
\textbf{ಜಪನ್ನದೀಮಾಪ ಚ ಕೌಶಿಕಾಖ್ಯಾಂ}\\\textbf{ನಿಮಜ್ಯ ತಸ್ಯಾಂ ಪಯಸಾಭಿಷೇಚೇ~।। ೩೭~।।}
\end{verse}

ನಂತರ ಪಿಶಾಚಿಗಳಿಂದ ಸುತ್ತುವರಿಯಲ್ಪಟ್ಟ ಸುಯಜ್ಞನು ಕೌಶಿಕವೆಂಬ ನದಿಯಲ್ಲಿ ಸ್ನಾನಮಾಡಿ, ಅತ್ಯುಗ್ರವಾದ ನರಸಿಂಹಬೀಜಾಕ್ಷರ ಮಂತ್ರವನ್ನು ಜಪಿಸಿ ನದಿಯ ನೀರನ್ನು ಆ ಪಿಶಾಚಿಗಳಮೇಲೆ ಪ್ರೋಕ್ಷಿಸಿದನು.

\textbf{[ವಿಶೇಷಾಂಶ:-}ನೃಸಿಂಹಬೀಜಾಕ್ಷರದ ವಿವರಣೆಯೂ, ಜಪಿಸುವ ಕ್ರಮವೂ ಶ‍್ರೀಮದಾನಂದ\-ತೀರ್ಥರಿಂದ ರಚಿತವಾದ “ತಂತ್ರಸಾರ”ವೆಂಬ ಗ್ರಂಥದ ನಾಲ್ಕನೆಯ ಅಧ್ಯಾಯದಲ್ಲಿ ನಿರೂಪಿತವಾಗಿವೆ.]

\begin{verse}
\textbf{ತೇ ವೈ ತಟಸ್ಥಾಃ ಪಯಸಾಭಿಷಿಕ್ತಾ} \\\textbf{ಮುನೀಂದ್ರಪುತ್ರೇಣ ದಯಾನ್ವಿತೇನ~।}\\\textbf{ತೇ ಶುದ್ದಭಾವಾಃ ಶಿರಸಾ ಪ್ರಣೇಮುಃ} \\\textbf{ಪೂರ್ಣೇಂದುಸಂಕಾಶಮುಖಂ ಮಹಾಂತಮ್~।। ೩೮~।।}
\end{verse}

ಪ್ರೋಕ್ಷಣೆಯನಂತರ ಆ ಪಿಶಾಚಿಗಳು ಶುದ್ದ ಭಾವದಿಂದ ಪೂರ್ಣಚಂದ್ರನಂತೆ ಮುಖವುಳ್ಳ, ಮಹಾತ್ಮನಾದ ಆ ಸುಯಜ್ಞನನ್ನು ಸಾಷ್ಟಾಂಗ ನಮಸ್ಕರಿಸಿದವು.

\begin{verse}
\textbf{ತಾನಾಗತಾಂ ಸ್ತ್ಯ್ತಕ್ತನಿಜಸ್ವಭಾವಾನ್} \\\textbf{ಊಚೇ ಗತಕ್ಲೇಶಭಯೋ ಮಹಾತ್ಮಾ~।}\\\textbf{ಕೇ ಕೇ ಭವಂತೋ ಭವತಾಂ ಚ ಕರ್ಮ-} \\\textbf{ದಶೋದೃಶೀ ಯೇನ ಸಮಾಗತಾ ಚ~।। ೩೯~।। }
\end{verse}

\begin{verse}
\textbf{ಬ್ರೂತೇ ಹ ನೂನಂ ಪೃಥಗೇವ ಮಹ್ಯಂ} \\\textbf{ಕರ್ಮಾನುಗಂ ಜನ್ಮ ಯತಶ್ಚ ಜಾತಮ್~।}\\\textbf{ಇತೀರಿತಾಸ್ತೇ ಮುನಿಪುತ್ರಮೂಚುಃ} \\\textbf{ಪುರಾಕೃತಂ ಕರ್ಮ ಯಥಾನುಪೂರ್ವಮ್~।। ೪೦~।।}
\end{verse}

ಭಯಾದಿಗಳಿಂದ ಮುಕ್ತನಾದ ಮಹಾತ್ಮನಾದ ಆ ಮುನಿಪುತ್ರನು ಪಿಶಾಚಿಗಳನ್ನು ಕುರಿತು “ನಿಮಗೆ ಯಾವ ಯಾವ ಕರ್ಮದ ಫಲವಾಗಿ ಇಂತಹ ನೀಚ ಜನ್ಮಗಳು ಬಂದುವು?” ಎಂದು ಕೇಳಿದನು. ಒಬ್ಬೊಬ್ಬರೂ ಪ್ರತ್ಯೇಕವಾಗಿ ತನ್ನ ದುರ್ದೆಶೆಗೆ ಕಾರಣವಾದ ಕರ್ಮಗಳನ್ನು ಹೇಳಿರೆಂದು ಮುನಿಪುತ್ರನು ಸೂಚಿಸಲು ಒಬ್ಬೊಬ್ಬರೂ ತಮ್ಮ ಪೂರ್ವ ಚರಿತ್ರೆಯನ್ನು ವಿವರಿಸಿದರು.

\begin{flushleft}
\textbf{ಬ್ರಹ್ಮರಾಕ್ಷಸಾ ಊಚುಃ\enginline{-}}
\end{flushleft}

\begin{verse}
\textbf{ಶೃಣು ನಾಮಾನಿ ವಕ್ಷ್ಯಾಮಃ ಸಪ್ತಾನಾಂ ಪಾಪಕರ್ಮಣಾಮ್~।}\\\textbf{ದಗ್ಧಪಾಣಿರಹಂ ಪೂರ್ವಂ ದಗ್ಧ ಜಿಹ್ವಸ್ತಥಾಪರಃ~।। ೪೧~।। }
\end{verse}

\begin{verse}
\textbf{ಅಮಾಘೋsಯಂ ತೃತೀಯಸ್ತು ಚತುರ್ಥೋ ದೂತಿಕೋ ಮತಃ~।}\\\textbf{ನಿಂದಕಃ ಪಂಚಮೋ ಹ್ಯೇಷ ಷಷ್ಠೋ ವೈ ಮಲಭೋಜನಃ~।। ೪೨~।।}
\end{verse}

\begin{verse}
\textbf{ಸಪ್ತಮೋಽದೈವತೋ ನಾಮ ಸಪ್ತಾನಾಂ ಚರಿತಂ ಶೃಣು~।}
\end{verse}

\noindent
 ಬ್ರಹ್ಮರಾಕ್ಷಸರು ಹೇಳಿದರು:-

ಮೊದಲು ನಮ್ಮ ಏಳು ಜನರ ಹೆಸರುಗಳನ್ನು ಹೇಳುತ್ತೇನೆ ಕೇಳು. ನಾನು ದಗ್ಧ ಪಾಣಿ, ಎರಡನೆಯವನು ದಗ್ಧ ಜಿಹ್ವ, ಮೂರನೆಯವನು ಅಮಾಘ, ನಾಲ್ಕನೆ ಯವನು ದೂತಿಕ, ಐದನೆಯವನು ನಿಂದಕ, ಆರನೆಯವನು ಮಲಭೋಜನ, ಏಳನೆಯವನು ಅದೈವತ. ಈ ಏಳು ಜನರ ಚರಿತ್ರೆಯನ್ನು ಲಾಲಿಸು.

\begin{verse}
\textbf{ಮಾಧವೋಽಹಂ ಪುರಾ ಕಾಶ್ಯಾಂ ಬ್ರಾಹ್ಮಣೋ ಗ್ರಾಮಜ್ಯೋತಿಷಃ~।। ೪೩~।।}
\end{verse}

ನಾನು ಹಿಂದೆ ಕಾಶಿಯಲ್ಲಿ ಬಾಹ್ಮಣಕುಲದಲ್ಲಿ ಉತ್ಪನ್ನನಾಗಿ ಜ್ಯೋತಿಷನಾಗಿದ್ದೆ.

\begin{verse}
\textbf{ವೇದವೇದಾಂಗತತ್ವಜ್ಞಃ ಸರ್ವಧರ್ಮಪರಾಙ್ಮುಖಃ~।}\\\textbf{ಪರೈರಹಂ ಕಾರಯಿತ್ವಾ ಶ್ರಾದ್ಧ ಕರ್ಮಾಣಿ ಭೂರಿಶಃ~।। ೪೪~।। }\\\textbf{ಪ್ರತ್ಯಹಂ ತತ್ರ ಭೋಕ್ತಾ ಚ ನ ಗೃಹೇ ಕ್ವಾಪಿ ಭುಜ್ಯತೇ~।}
\end{verse}

ವೇದವೇದಾಂಗಗಳ ಜ್ಞಾನಯುಕ್ತನಾಗಿದ್ದರೂ ಸಕಲ ಸತ್ಕರ್ಮಗಳನ್ನೂ ತ್ಯಜಿಸಿದ್ದೆ, ಇತರರಿಂದ ಶ್ರಾದ್ಧಾದಿ ಪಿತೃ ಕಾರ್ಯಗಳನ್ನು ಮಾಡಿಸಿ ಅವರ ಮನೆಯಲ್ಲಿಯೇ ನಿತ್ಯವೂ ಊಟಮಾಡುತ್ತಿದ್ದೆ. ನನ್ನ ಮನೆಯಲ್ಲಿ ನಾನು ಎಂದೂ ಊಟ ಮಾಡಲಿಲ್ಲ.

(ವೇದಾಂಗಗಳು: ವ್ಯಾಕರಣ, ನಿರುಕ್ತ, ಛಂದಸ್ಸು, ಶಿಕ್ಷಾ, ಕಲ್ಪ, ಜ್ಯೋತಿಷ).

\begin{verse}
\textbf{ನದೀತೀರಜಲೇ ಶಾಲಗ್ರಾಮಪೂಜಾಂ ಕರೋಮ್ಯಹಮ್~।। ೪೫~।।} 
\end{verse}

\begin{verse}
\textbf{ಧನಸ್ಯಾರ್ಜನಚಿತ್ತೇನ ನ ಪುಣ್ಯಾ ರ್ಥಂ ಕೃತಂ ಮಯಾ~।}\\\textbf{ಲೋಭೇನ ಧೂಪೋ ದೀಪೋ ವಾ ದೇವಾಲಯ ನ ನಿವೇದಿತಃ~।। ೪೬~।।}
\end{verse}

ನದೀತೀರದಲ್ಲಿ ನದಿಯ ಜಲದಿಂದ ಶಾಲಗ್ರಾಮಕ್ಕೆ ಅಭಿಷೇಕ, ಪೂಜೆ ಮಾಡುತ್ತಿದ್ದೆ. ದ್ರವ್ಯ ಸಂಪಾದನೆಗಾಗಿ ಮಾಡುತ್ತಿದ್ದೆನೇ ಹೊರತು ಪುಣ್ಯಕ್ಕಾಗಿ ಅಲ್ಲ. ಲೋಭದಿಂದ ದೇವರಿಗೆ ಧೂಪ, ದೀಪ, ನೈವೇದ್ಯವನ್ನು ಅರ್ಪಿಸುತ್ತಿದ್ದಿಲ್ಲ.

\begin{verse}
\textbf{ವೇದಿಕಾಯಾಂ ಗ್ರಹೇ ಕ್ವಾಪಿ ಪೂಜಿತೋ ನೈವ ಕೇಶವಃ~।}\\\textbf{ಪೂಜಿತೇಷು ಗೃಹೇ ವಿಷ್ಣೌ ನೈವೇದ್ಯಾದಿ ಭವೇದ್ಯದಿ~।। ೪೭~।। }
\end{verse}

\begin{verse}
\textbf{ಸೈ ಕತೇನೋದಕೇನೈವ ಪೂಜಯಿತ್ವಾನ್ಯ ಮಾನಸಃ~।}\\\textbf{ವಸ್ತ್ರೇಣ ಬಂಧಯಿತ್ವೇತಿ ಕಕ್ಷೇ ನಿಕ್ಷಿಪ್ಯ ದಾಂಭಿಕಃ~।। ೪೮~।। }
\end{verse}

\begin{verse}
\textbf{ಪುನರ್ಯಾತ್ರಾಂ ಪ್ರವೃತ್ತೋsಹಂ ಶ್ರಾದ್ಧದಾನಾರ್ಹಕಾಂಕ್ಷಯಾ~।}\\\textbf{ಯಾನ್ಯಹಂ ತತ್ರ ತತ್ರೈವ ನ ಜಪ್ತಂ ನ ಹುತಂ ಮಯಾ~।। ೪೯~।।}
\end{verse}

ಮನೆಯಲ್ಲಿ ವೇದಿಕೆಯಮೇಲೆ ಶ‍್ರೀಹರಿಯನ್ನು ಪೂಜಿಸುತ್ತಿರಲಿಲ್ಲ. ಒಂದು ವೇಳೆ ಮನೆಯಲ್ಲಿ ಪೂಜಿಸಿದರೂ, ನೈವೇದ್ಯಾದಿಗಳು ಅವಶ್ಯಕವೆಂದು ತಿಳಿದಿದ್ದರೂ ಅರ್ಪಿಸುತ್ತಿದ್ದಿಲ್ಲ. ನದಿಯ ನೀರು, ಮರಳು ಇವುಗಳಿಂದ ಅನ್ಯಮನಸ್ಕನಾಗಿ ಪೂಜಿಸಿ ವಸ್ತ್ರದಲ್ಲಿ ದೇವರನ್ನು ಕಟ್ಟಿ ಕೊಂಕಳಿನಲ್ಲಿ ಇಟ್ಟುಕೊಂಡು ಡಾಂಭಿಕತನದಿಂದ ತಿರುಗುತ್ತಿದ್ದೆ. ಬೇರೆ ಬೇರೆ ಪ್ರದೇಶಗಳಲ್ಲಿ ಶ್ರಾದ್ಧದ ದಾನಾದಿಗಳು ಸಿಗುತ್ತವೆಯೆಂಬ ಆಸೆಯಿಂದ ಸುಮ್ಮನೆ ಚಲಿಸುತ್ತಿದ್ದೆ. ಹೋದ ಸ್ಥಳಗಳಲ್ಲಿ ಜಪವನ್ನಾಗಲೀ, ಹವನ-ಹೋಮಗಳನ್ನಾಗಲೀ ಮಾಡುತ್ತಿರಲಿಲ್ಲ.

\begin{verse}
\textbf{ಮಾತಾಪಿತ್ರೋರ್ದಿನೇ ಪ್ರಾಪ್ತೇ ವ್ಯಾಜಮುತ್ಪಾತಯಾಮ್ಯ ಹಮ್~।}\\\textbf{ಗ್ರಾಮಾಂತರಂ ವ್ಯತೀತ್ಯೈವ ಪುನರ್ಯಾಮಿ ಚ ವೈ ಗೃಹಮ್~।। ೫೦~।।} 
\end{verse}

\begin{verse}
\textbf{ನ ವ್ಯಾಖ್ಯಾತಂ ಮಯಾ ಕ್ವಾಪಿ ನಾಧ್ಯಾಪಿತಮಥಾಪಿ ವಾ~।}\\\textbf{ಶ್ರಾದ್ಧಾದಿಕಾಂಕ್ಷಿಣಾ ಕ್ವಾಪಿ ನ ಶ್ರುತಾ ತು ಹರೇಃ ಕಥಾ~।। ೫೧~।। }
\end{verse}

\begin{verse}
\textbf{ಯಾಮಾತ್ಪೂರ್ವಂ ತು ಬುದ್ಧಾಹಂ ರಾತ್ರೌ ತು ಯಾಮಮೇವ ಚ~।}\\\textbf{ಪ್ರತೀಕ್ಷತೇ ಮಯಾ ಕಾಲೋ ಮೃತಾಹಮಪಿ ಕಾಂಕ್ಷಿಣಾ~।। ೫೨~।।}
\end{verse}

ತಂದೆ-ತಾಯಿಯರ ಶ್ರಾದ್ಧದ ದಿವಸ ಬಂದರೆ ಮನೆಯಲ್ಲಿ ಏನಾದರೂ ಒಂದು ನೆಪದಿಂದ ಜಗಳವಾಡಿ ಬೇರೊಂದು ಗ್ರಾಮಕ್ಕೆ ಹೋಗಿ ಆ ಶ್ರಾದ್ಧದ ದಿವಸ ಕಳೆದ ಮೇಲೆ ಪುನಃ ಮನೆಗೆ ಬರುತ್ತಿದ್ದೆ. ಯಾವ ಸಚ್ಛಾ ಸ್ತ್ರವನ್ನೂ ನಾನು ಪಾಠಪ್ರವಚನ ಮಾಡಲಿಲ್ಲ; ಯಾರಿಗೂ ವೇದಾಧ್ಯಯನ ಮಾಡಿಸಲಿಲ್ಲ. ಶ್ರಾದ್ಧಾದಿಗಳಲ್ಲಿ ದೊರೆಯುವ ಹಣದ ಆಸೆಯಿಂದ ರಾತ್ರಿ-ಹಗಲೂ ಅದರ ಬಗ್ಗೆಯೇ ಯೋಚಿಸುತ್ತ ಶ‍್ರೀಹರಿಯ ಕಥಾಶ್ರವಣವನ್ನು ಮಾಡಲಿಲ್ಲ. ಪ್ರಾತಃಕಾಲವಾಗಲು ಒಂದು ಪ್ರಹರಕ್ಕೆ ಮುಂಚೆಯೇ ಎದ್ದು, ರಾತ್ರಿ ಒಂದು ಯಾಮ ಕಳೆಯುವ ತನಕ ಶ್ರಾದ್ಧ ಮಾಡಿಸಲು ಯಾರಾದರೂ ಕರೆಯಲು ಬರುತ್ತಾರೆಂದು ನಿರೀಕ್ಷಿಸುತ್ತಿದ್ದೆ.

\begin{verse}
\textbf{ತಿಲಭಾರಃ ಕುಶಭಾರ ಇತಿ ಲೋಕಾ ವದಂತಿ ಮಾಮ್~।}\\\textbf{ಕುಶಪಾಣಿಃ ಶಿಖಾಬದ್ಧೋ ಧೌತವಸ್ತ್ರಃ ಸದಾ ಶುಚಿಃ~।। ೫೩~।।}
\end{verse}

ಜನರು ನನ್ನನ್ನು “ತಿಲಭಾರ”, “ಕುಶಭಾರ” ಎನ್ನುತ್ತಿದ್ದರು. ನಾನು ಯಾವಾಗಲೂ ಶುಭ್ರವಾದ ವಸ್ತ್ರಧರಿಸಿ, ಶಿಖೆಯನ್ನು ಕಟ್ಟಿಕೊಂಡು, ಕೈಯಲ್ಲಿ ದರ್ಭೆಯನ್ನು ಹಿಡಿದುಕೊಂಡು ಶುಚಿಯಾಗಿರುತ್ತಿದ್ದೆ.

\begin{verse}
\textbf{ದೇವಪೂಜಾಕಕ್ಷಪಾಲಿಃ ಕಕ್ಷೇ ಪಾಣೌ ಕಮಂಡಲುಃ~।}\\\textbf{ಚಲದೋಷ್ಠೋ ವೃಥಾ ಧರ್ಮವಾದೀ ದಂಭಕಯಿತ್ಯಹಮ್~।। ೫೪~।।}
\end{verse}

ದೇವರನಿಟ್ಟ ಕಕ್ಷಪಾಲವನ್ನು ಕೊಂಕುಳಲ್ಲಿ ಇಟ್ಟುಕೊಂಡು, ಕೈಯಲ್ಲಿ ಕಮಂಡಲುವನ್ನು ಹಿಡಿದು, ತುಟಿಗಳನ್ನು ಸುಮ್ಮನೆ ಆಡಿಸುತ್ತ ಮಂತ್ರಗಳನ್ನು ಉಚ್ಚರಿಸದೆ, ಧರ್ಮದ ವಿಷಯದಲ್ಲಿ ಅನವಶ್ಯಕವಾದ ಚರ್ಚೆಗಳನ್ನು ಮಾಡುತ್ತಾ ಡಾಂಭಿಕನಾಗಿದ್ದೆ.

\begin{verse}
\textbf{ವಂಚಯಿತ್ವಾ ಜನಾನ್ ಸರ್ವಾನ್ ಆರ್ಜಯಿತ್ವಾ ಧನಂ ಬಹು~।}\\\textbf{ನ ಭುಕ್ತಂ ನ ಚ ದತ್ತಂ ಚ ಸುಹೃದಃ ಪ್ರೇಷಿತಾ ಜನಾಃ~।। ೫೫~।।}
\end{verse}

ಎಲ್ಲ ಜನರನ್ನೂ ವಂಚಿಸಿ ಬಹಳ ಹಣವನ್ನು ಸಂಗ್ರಹಿಸಿದೆ. ನಾನು ಆ ಹಣವನ್ನು ಅನುಭವಿಸಲಿಲ್ಲ. ಮಿತ್ರರಿಗಾಗಲೀ, ಆಪ್ತರಿಗಾಗಲೀ ದಾನ ಮಾಡಲಿಲ್ಲ.

\begin{verse}
\textbf{ಪಶ್ಚಾದಾಗತಕಾಲೋ ಹಿ ಜಾತೋsಹಂ ಬ್ರಹ್ಮರಾಕ್ಷಸಃ~।}\\\textbf{ಪ್ರತಿಗ್ರಹೇಣ ದಗ್ಧೋಽಯಂ ಪಾಣಿರ್ಮೇ ತೇನ ಕರ್ಮಣಾ~।। ೫೬~।।} 
\end{verse}

\begin{verse}
\textbf{ದಗ್ಧಪಾಣಿರಿತಿಖ್ಯಾ ತೋ ನಾಮ್ನಾಹಂ ಬ್ರಹ್ಮರಾಕ್ಷಸಃ~।}\\\textbf{ಲಕ್ಷಾಣಿ ತ್ರೀಣ್ಯ ತೀತಾನಿ ಮಮ ವರ್ಷಾಣಿ ಸಂತತಮ್~।। ೫೭~।। }
\end{verse}

\begin{verse}
\textbf{ತವಾದ್ಯ ದರ್ಶನಾದೇವ ಜಾತಂ ದುಃಖವಿಮೋಕ್ಷಣಮ್~।}\\\textbf{ಇತಿ ಸರ್ವಂ ಮಯಾಽಽಖ್ಯಾತಂ ಮದ್ವೃತ್ತಾತಂ ಜುಗುಪ್ಸಿತಮ್~।। ೫೮~।।}
\end{verse}

ಈ ರೀತಿಯಲ್ಲಿ ಕಾಲಕಳೆದ ನಾನು, ಮೃತನಾದನಂತರ ಬ್ರಹ್ಮರಾಕ್ಷಸನಾದೆ. ದಾನಸ್ವೀಕಾರ ಕರ್ಮದಿಂದ ನನ್ನ ಹಸ್ತಗಳು ಸುಟ್ಟಿವೆ. ನನಗೆ "ದಗ್ಧಪಾಣಿ ಬ್ರಹ್ಮರಾಕ್ಷಸ'ನೆಂದೇ ಹೆಸರು. ಈ ಜನ್ಮವನ್ನು ಮೂರು ಲಕ್ಷ ವರ್ಷಗಳ ಕಾಲ ಅನುಭವಿಸಿದೆ. ಈ ದಿನ ನಿನ್ನ ದರ್ಶನದಿಂದ ನನ್ನ ದುಃಖವು ಕೊನೆಗೊಂಡಿತು. ಕೇಳಿದರೆ ಜುಗುಪ್ಪೆಯನ್ನುಂಟುಮಾಡುವ ನನ್ನ ಪೂರ್ವವೃತ್ತಾಂತವನ್ನೆಲ್ಲ ಹೇಳಿರುತ್ತೇನೆ.

\begin{flushleft}
\textbf{ದಗ್ಧ ಜಿಹ್ವ ಉವಾಚ\enginline{-}}
\end{flushleft}

\begin{verse}
\textbf{ಅಹಂ ದ್ವಿಜಃ ಪುರಾ ನಂದಿಗ್ರಾಮೇ ಮುದ್ಗಲಗೋತ್ರಜಃ~।}\\\textbf{ವೇದವೇದಾರ್ಥತತ್ವಜ್ಞಃ ಕೃತ್ಯಾ ಕೃತ್ಯವಿವೇಕವಾನ್~।। ೫೯~।।}
\end{verse}

\begin{flushleft}
ದಗ್ಧ ಜಿಹ್ವನೆಂಬ ಬ್ರಹ್ಮರಾಕ್ಷಸನು ಹೇಳಿದನು:-
\end{flushleft}

ನಾನು ಹಿಂದಿನ ಜನ್ಮದಲ್ಲಿ ನಂದಿಗ್ರಾಮದಲ್ಲಿ ಮುದ್ಗಲಗೋತ್ರದಲ್ಲಿ ಉತ್ಪನ್ನನಾದ ಬ್ರಾಹ್ಮಣನಾಗಿದ್ದೆ. ವೇದವೇದಾರ್ಥಗಳ ತತ್ವಗಳನ್ನೂ ವಿಧಿನಿಷೇಧಗಳನ್ನೂ ಚೆನ್ನಾಗೆ ತಿಳಿದಿದ್ದೆ.

\begin{verse}
\textbf{ನಾಮ್ನಾ ಸುಕರ್ಮನಾಮಾಹಮಭಕ್ಷ್ಯಾ ಭಕ್ಷಣೇ ರತಃ~।}\\\textbf{ನ ತಪ್ತಂ ನ ಹುತಂ ಜಪ್ತಂ ನ ಚ ಕಿಂಚಿದನುಷ್ಠಿತಮ್~।। ೬೦~।।}
\end{verse}

ಆಗ ನನ್ನ ಹೆಸರು ಸುಕರ್ಮ, ಜಪ, ತಪಸ್ಸು, ಹೋಮ ಇವುಗಳನ್ನು ಎಂದೂ ಮಾಡಲಿಲ್ಲ. ಯಾವ ಸತ್ಕರ್ಮವನ್ನೂ ಅನುಷ್ಠಾನ ಮಾಡಲಿಲ್ಲ. ನಿಷಿದ್ಧ ವಸ್ತುಗಳನ್ನು ತಿನ್ನುವುದರಲ್ಲಿ ತುಂಬ ಆಸಕ್ತನಾಗಿದ್ದೆ.

\begin{verse}
\textbf{ಗೃಹಾರಾಮೇ ಮಯೋಪ್ತಾನಿ ಶಿಗ್ರುಶಾಖಾನಿ ಮೂಲಕಮ್~।}\\\textbf{ವೃಂತಾಕಾನಿ ಕಲಿಂಗಾನಿ ನಿಷ್ಪಾವಾಣಿ ಚ ಗೃಂಜನಮ್~।। ೬೧~।।} 
\end{verse}

\begin{verse}
\textbf{ಸಂಧಿತಾನಿ ಕುರಂಗಾಣಿ ಬಿಂಬಬಿಲ್ವಾದಿವರ್ಧಿತಾಃ~।}\\\textbf{ಉಪೋದಕೀ ಚ ಕೋಸುಂಭಂ ಘೃತಕೋಶಾತಕೀ ತಥಾ~।। ೬೨~।।}
\end{verse}

\begin{verse}
\textbf{ಶಾಬರೀ ಕುಚರೀ ಚೈವ ಶುಕಶಾಕಂ ತಥೈವ ಚ~।}\\\textbf{ತಂದುಲೀಯಕಶಾಕಾಶ್ಚ ವರ್ಧಿತಾನಿ ಗೃಹೇ ಮಯಾ~।। ೬೩~।।}
\end{verse}

ನನ್ನ ಮನೆಯ ಅಂಗಳದಲ್ಲಿ ನುಗ್ಗೆಕಾಯಿ, ಮೂಲಂಗಿ, ಬದನೆಕಾಯಿ, ಕಲ್ಲಂಗಡಿ ಹಣ್ಣು, ಚವಳೇಕಾಯಿ, ಕೆಂಪುಮೂಲಂಗಿ, ತೊಂಡೆಕಾಯಿ, ನಾನಾ ವಿಧವಾದ ಸೊಪ್ಪುಗಳು, ಹೀರೇಕಾಯಿ, ಇವೇ ಮೊದಲಾದ ನಾನಾವಿಧವಾದ ತರಕಾರಿಗಳನ್ನು ಬೆಳೆಯುತ್ತಿದ್ದೆ.

\begin{verse}
\textbf{ಯದ್ಗೃಹೋಪರಿ ವೈ ಗಚ್ಛಂಸ್ತುಂಬರುರ್ಯಾನಸಂಸ್ಥಿತಃ~।}\\\textbf{ಸಾಂಗನೋ ನ್ಯಪತತ್ಸದ್ಯೋ ಯದ್ವಾ ತಾಹತವಾಯುನಾ~।। ೬೪~।।}
\end{verse}

ಒಂದು ದಿನ ತುಂಬುರುನೆಂಬ ಗಂಧರ್ವನು ವಿಮಾನದಲ್ಲಿ ನನ್ನ ಮನೆಯ ಮೇಲೆ ತನ್ನ ಸ್ತ್ರೀಯರಿಂದ ಸಹಿತನಾಗಿ ಹಾರುತ್ತಿದ್ದಾಗ ಗಾಳಿಯ ಹೊಡೆತದಿಂದ ವಿಮಾನವು ನನ್ನ ಮನೆಯಮೇಲೆ ಬಿದ್ದಿತು.

\begin{verse}
\textbf{ಭೂಮಿಚಾರೀ ಸ ಗಂಧರ್ವೋ ಗತ್ವಾ ಗೋದಾವರೀಂ ನದೀಮ್~।}\\\textbf{ಮಾಘೇ ಸ್ನಾತ್ವಾ ಮಾಸಮೇಕಂ ಪತಂಗೇ ಮಕರಸ್ಥಿತೇ~।। ೬೫~।। }\\\textbf{ತೇನ ಲಬ್ಧ ಪ್ರಭಾವೇನ ಯಯೌ ಲೋಕಂ ಸ್ವಯಂಭುವಃ~।}
\end{verse}

ನೆಲದಮೇಲೆ ಸಂಚರಿಸುತ್ತಲೇ ಆ ಗಂಧರ್ವನು ಗೋದಾವರೀನದಿಗೆ ಹೋಗಿ ಸೂರ್ಯನು ಮಕರರಾಶಿಯಲ್ಲಿರುವಾಗ ಮಾಘಮಾಸದಲ್ಲಿ ಒಂದು ತಿಂಗಳಕಾಲ ಸ್ನಾನಮಾಡಿ ಆ ಪುಣ್ಯ ಪ್ರಭಾವದಿಂದ ದೇವಲೋಕಕ್ಕೆ ಹೋದನು.

\begin{verse}
\textbf{ಚಾತುರ್ಮಾಸ್ಯಾಂ ದ್ವಾದಶೀಷು ತಥಾ ಪುಣ್ಯದಿನೇಷು ಚ~।। ೬೬~।।}\\\textbf{ಸಂಕ್ರಮೇಷು ಚ ಸರ್ವೇಷು ತಥಾನ್ಯದಿವಸೇಷು ಚ~। }\\\textbf{ಭಕ್ಷಿತಾನಿ ಮಯಾ ತಾನಿ ಪಿಣ್ಯಾಕಂ ಚೈವ ಭಕ್ಷಿತಮ್~।। ೬೭~।।}
\end{verse}

ನಾನಾದರೋ ಯಾವ ಸತ್ಕರ್ಮವನ್ನೂ ಆಚರಿಸದೆ ಚಾತುರ್ಮಾಸಗಳಲ್ಲಿ, ದ್ವಾದಶಿಗಳಲ್ಲಿ, ಇತರ ಪುಣ್ಯತಿಥಿಗಳಲ್ಲಿ, ಸಂಕ್ರಮಣದಿವಸಗಳಲ್ಲಿ ಹಾಗೂ ಇತರ ಸಮಸ್ತ ದಿವಸಗಳಲ್ಲಿಯೂ ನಿಷಿದ್ಧ ತರಕಾರಿಗಳನ್ನೂ ಅತಿ ನಿಂದಿತವಾದ ಹಿಂಡಿಯನ್ನೂ ಸಹ ಭಕ್ಷಿಸುತ್ತಿದ್ದೆನು.

\begin{verse}
\textbf{ಪಾಚಯಿತ್ವಾ ತ್ವಾರನಾಲಂ ಭುಕ್ತಮನ್ನಂ ಮಯಾನಿಶಮ್~।}\\\textbf{ಆದ್ಯನ್ನಂ ಷೋಡಶಾದ್ಯಾನ್ನಂ ಪ್ರಾರ್ಥನಾಪೂರ್ವಕಂ ತಥಾ~।। ೬೮~।। }\\\textbf{ಬಹುಧಾ ಭುಕ್ತಮತ್ಯಂತು ಬಿಲ್ವಮದುಂಬರಂ ತಥಾ~।}\\\textbf{ಪಿಶಾಚಭ್ಯೇಶ್ಚ ನಾಗೇಭ್ಯಃ ಶಕ್ತಿಭ್ಯೋ ಯನ್ನಿವೇದಿತಮ್~।। ೬೯~।।} \\\textbf{ತಥಾನ್ಯದೇವತಾ ಭುಕ್ತಂ ಮಯಾ ಭುಕ್ತಂ ಸಹಸ್ರಶಃ~।।}
\end{verse}

ನಾನು ನಿತ್ಯವೂ ಗಂಜಿಯನ್ನು ತಯಾರಿಸಿ ಕುಡಿಯುತ್ತಿದ್ದೆ. ತಂಗಳನ್ನವನ್ನೂ, ಇತರ ನಾನಾವಿಧವಾದ ಅನ್ನಗಳನ್ನೂ ಇತರರಿಂದ ಬೇಡಿ ಪಡೆದು ಭುಂಜಿಸುತ್ತಿದ್ದೆ. ಇದಲ್ಲದೆ, ಬಿಲ್ವ, ಔದುಂಬರ, ಪಿಶಾಚಿಗಳು, ನಾಗ, ಶಕ್ತಿದೇವತೆಗಳು ಇವರಿಗೆ ನೈವೇದ್ಯ ಮಾಡಿದ ಪದಾರ್ಥಗಳನ್ನೂ, ಶ‍್ರೀಹರಿಗೆ ಅರ್ಪಿಸದೆ ಇತರ ದೇವತೆಗಳಿಗೆ ಅರ್ಪಿಸಿದ ಅನ್ನಾದಿಗಳನ್ನೂ ಸಾವಿರಾರು ಸಲ ಭುಂಜಿಸಿದೆ.

[ಶ‍್ರೀಹರಿಯ ನೈವೇದ್ಯವೇ ಭೋಜ್ಯಾರ್ಹ. ಇತರ ದೇವತೆಗಳಿಗೆ ಅರ್ಪಿಸಿದ ವಸ್ತುಗಳನ್ನು ಸ್ವೀಕರಿಸಬಾರದು. ಹಾಗೆ ಸ್ವೀಕರಿಸಿದರೆ ಪ್ರಾಯಶ್ಚಿತ್ತರೂಪವಾಗಿ ಚಾಂದ್ರಾಯಣ ವ್ರತವನ್ನು ಆಚರಿಸಬೇಕು.

\begin{verse}
\textbf{ಪಾವನಂ ವಿಷ್ಣು ನೈವೇದ್ಯಂ ಸುಭೋಜ್ಯಮೃಷಿಭಿಃ ಸ್ಮೃತಮ್~।}\\\textbf{ಅನ್ಯದೇವಸ್ಯ ನೈವೇದ್ಯಂ ಭುಕ್ತ್ವಾ ಚಾಂದ್ರಾಯಣಂ ಚರೇತ್~।।}
\end{verse}

\vauthor{\textbf{(ಶ‍್ರೀ ಕೃಷ್ಣಾಮೃತಮಹಾರ್ಣವ)]}}

\begin{verse}
\textbf{ಭಾನುವಾರೇ ಪರ್ವಣಿ ಚ ತಥಾ ಶ್ರಾದ್ಧದಿನೇಷು ಚ~।। ೭೦~।।} 
\end{verse}

\begin{verse}
\textbf{ರಾತ್ರೌ ಭುಕ್ತಂ ಮಯಾ ಜಿಹ್ವಾಲಾಲಸೇನ ದುರಾತ್ಮನಾ~।}\\\textbf{ಆರ್ಯೈಶ್ಚ ಗುರುಭಿಶ್ಚೈವ ಬಹುಧಾ ಬೋಧಿತೋಽಹ್ಯ ಹಮ್~।। ೭೧~।। }
\end{verse}

\begin{verse}
\textbf{ಜಿಹ್ವಾಚಾಪಲದೋಷೇಣ ನಾತ್ಯಜಂ ನ ಶುಭಂ ಕೃತಮ್~।}\\\textbf{ದಗ್ಧ ಜಿಹ್ವೋಽಸ್ಮಿ ತೇನೈವ ನಾಮ್ನಾಹಂ ಬ್ರಹ್ಮರಾಕ್ಷಸಃ~।। ೭೨~।।}
\end{verse}

ಭಾನುವಾರಗಳಲ್ಲಿಯೂ, ಪರ್ವದಿವಸಗಳಲ್ಲಿಯೂ, ಶ್ರಾದ್ಧದಿವಸಗಳಲ್ಲಿಯೂ ಸಹ ರಾತ್ರಿ ವೇಳೆಯಲ್ಲಿ ನಾಲಿಗೆ ಚಪಲದಿಂದ ದುರಾತ್ಮನಾದ ನಾನು ಭೋಜನ ಮಾಡುತ್ತಿದ್ದೆ. ಹಿರಿಯರೂ, ಗುರುಗಳೂ ಈ ವಿಚಾರದಲ್ಲಿ ನನಗೆ ಬೇಕಾದಷ್ಟು ಬೋಧಿಸಿದರೂ ಸಹ ನಾಲಿಗೆ ಚಪಲದಿಂದ ನಾನು ಬಿಡಲಿಲ್ಲ, ಸತ್ಕರ್ಮವನ್ನು ಆಚರಿಸಲೇ ಇಲ್ಲ. ಈ ಕಾರಣದಿಂದಲೇ ಬ್ರಹ್ಮರಾಕ್ಷಸನಾದ ನನಗೆ "ದಗ್ಧ ಜಿಹ್ವ” ನೆಂದು ಹೆಸರಾಗಿದೆ.

\begin{verse}
\textbf{ಇತ್ಯುಕ್ತ್ವಾ ದಗ್ಧ ಜಿಹ್ವಾಸ್ತು ಶಿರಸಾ ತಂ ಪ್ರಣಮ್ಯ ಚ~।}\\\textbf{ವಾಚಾ ಹಿತಂ ಪ್ರಸಾದ್ಯಾಥ ವಿರರಾಮ ಸ ರಾಕ್ಷಸಃ~।। ೭೩~।।}
\end{verse}

ಹೀಗೆಂದು ನುಡಿದು ದಗ್ಧ ಜಿಹ್ವನೆಂಬ ಬ್ರಹ್ಮರಾಕ್ಷಸನು ಮುನಿಪುತ್ರನಿಗೆ ಸಾಷ್ಟಾಂಗ ನಮಸ್ಕರಿಸಿ, ತನ್ನ ಮೇಲೆ ಅನುಗ್ರಹಮಾಡಬೇಕೆಂದು ಪ್ರಾರ್ಥಿಸಿ ಸುಮ್ಮನಾದನು.

\begin{center}
ಇತಿ ಶ‍್ರೀ ವಾಯುಪುರಾಣೇ ಮಾಘಮಾಸಮಾಹಾತ್ಮ್ಯೇ ದಶಮೋsಧ್ಯಾಯಃ
\end{center}

\begin{center}
ಶ‍್ರೀ ವಾಯುಪುರಾಣಾಂತರ್ಗತ ಮಾಘಮಾಸ ಮಾಹಾತ್ಮ್ಯೆಯಲ್ಲಿ \\ ಹತ್ತನೇ ಅಧ್ಯಾಯವು ಸಮಾಪ್ತಿಯಾಯಿತು.
\end{center}

\newpage

\section*{ಅಧ್ಯಾಯ\enginline{-}೧೧}

\emptypage

\begin{flushleft}
\textbf{ಅಮಾಘ ಉವಾಚ:\enginline{-} }
\end{flushleft}

\begin{verse}
\textbf{ಅಹಂ ಪೂರ್ವಂ ದ್ರಾವಿಡೇಷು ನಾಮ್ನಾ ಶರ್ವೋ ದ್ವಿಜಾಧಮಃ~।}\\\textbf{ಜಮದಗ್ನಿಕುಲೋತ್ಪನ್ನೋ ವೇದವೇದಾಂಗಪಾರಗಃ~।। ೧~।।} 
\end{verse}

\begin{verse}
\textbf{ಲಿಂಗಭಸ್ಮಜಟಾಧಾರೀ ಶೂದ್ರಪಾಖಂಡಮಾರ್ಗಣಃ~।}\\\textbf{ಬ್ರಾಹ್ಮಣಾದುತ್ತಮಂ ಮನ್ಯೇ ವರ್ಣಧರ್ಮವಿವರ್ಜಿತಃ~।। ೨~।।}
\end{verse}

\begin{flushleft}
ಅಮಾಘನು ಹೇಳಿದನು:-
\end{flushleft}

ನಾನು ಹಿಂದೆ ದ್ರವಿಡದೇಶದಲ್ಲಿ ಜಮದಗ್ನಿ ಕುಲದಲ್ಲಿ ಹುಟ್ಟಿ ವೇದವೇದಾಂಗ ಪಾರಂಗತನಾಗಿದ್ದೆ. ಆಗ ನನ್ನ ಹೆಸರು ಶರ್ವ, ಬ್ರಾಹ್ಮಣರಲ್ಲಿ ಅತ್ಯಂತ ಹೀನನಾಗಿದ್ದೆ. ಆದರೂ ತಲೆಯಲ್ಲಿ ಜಟೆಯನ್ನು ಬೆಳಸಿಕೊಂಡು, ಭಸ್ಮವನ್ನು ಅಂಗಾಂಗಗಳಿಗೆ ಲೇಪಿಸಿಕೊಂಡು, ಬ್ರಾಹ್ಮಣವರ್ಣದ ಧರ್ಮಗಳನ್ನು ಪರಿತ್ಯಜಿಸಿ ಶೂದ್ರಪಾಖಂಡ ಮತಗಳನ್ನೇ ಅನುಸರಿಸುತ್ತಾ ಬ್ರಾಹ್ಮಣರಲ್ಲಿ ನಾನೇ ಶ್ರೇಷ್ಠನೆಂದು ತಿಳಿದಿದ್ದೆ.

\begin{verse}
\textbf{ತೀಕ್ಷ್ಣ ಕೌತುಕವಿದ್ಯಾಸು ವಾಚಾಲೋ ಗಾಯಕಃ ಶಠಃ~।}\\\textbf{ವಶಂಕರೋ ಭೈರವಾದ್ಯೈಃ ಮಂತ್ರೈಃ ವೇದವಿನಿಂದಕಃ~।। ೩~।।}
\end{verse}

ತೀಕ್ಷ್ಮವಾದ ಮತ್ತು ಕುತೂಹಲವನ್ನು ಕೆರಳಿಸುವ ವಿದ್ಯೆಗಳಲ್ಲಿ ನಿಪುಣನಾಗಿದ್ದೆ. ಒಳ್ಳೆಯ ವಾಕ್ಪಟುತ್ವ ಇತ್ತು. ಸಂಗೀತಗಾರನಾಗಿದ್ದೆ. ಆದರೂ ಮೂರ್ಖ, ಭೈರವಾದಿ ವಿದ್ಯೆ ಮತ್ತು ಕೆಟ್ಟ ಮಂತ್ರಗಳಿಂದ ಜನರನ್ನು ವಶಮಾಡಿ ಕೊಳ್ಳುತ್ತಿದ್ದೆ. ವೇದಗಳನ್ನು ನಿಂದಿಸುತ್ತಿದ್ದೆ.

\begin{verse}
\textbf{ಕದಾಚಿದಾಗತೋ ವೇಷೀ ಭಿಕ್ಷಾಕಾಂಕ್ಷೀ ಚ ಮದ್ಗೃಹಮ್~।}\\\textbf{ಕೌತುಕಂ ದರ್ಶಯಾಮಾಸ ಯೇನ ಲೋಕೋ ವಶಂ ಭವೇತ್~।। ೪~।।}
\end{verse}

ಒಂದು ದಿವಸ ಒಬ್ಬ ವೇಷಧಾರಿಯು ನನ್ನ ಮನೆಗೆ ಭಿಕ್ಷಕ್ಕಾಗಿ ಬಂದ. ಯಾವ ವಿದ್ಯೆಯ ಚಮತ್ಕಾರದಿಂದ ಜನರನ್ನು ವಶಪಡಿಸಿಕೊಳ್ಳಬಹುದೋ ಅಂತಹ ಕೌತುಕ ವಿದ್ಯೆಯನ್ನು ನನ್ನ ಮುಂದೆ ಪ್ರದರ್ಶನ ಮಾಡಿದನು.

\begin{verse}
\textbf{ಭಿಕ್ಷಾ ದತ್ತಾ ಮಯಾ ತಸ್ಮೈ ಗೃಹೀತ್ವಾಂತರಧೀಯತ~।}\\\textbf{ತತೋ ಮಹತ್ತ್ವ ಬುದ್ಧ್ಯಾಸ್ಮಿನ್ ಪ್ರಣಾವಸ್ತು ಮಯಾ ಕೃತಃ~।। ೫~।।}
\end{verse}

ನಾನು ಆತನಿಗೆ ಭಿಕ್ಷೆಯನ್ನು ನೀಡಿದೆ. ತೆಗೆದುಕೊಂಡು ಕಣ್ಮರೆಯಾದ. ಈತನಲ್ಲಿ ಏನೋ ಮಹತ್ವವಿರಬೇಕೆಂದು ತಿಳಿದು ನಮಸ್ಕರಿಸಿದೆ.

\begin{verse}
\textbf{ಅಸ್ತುವಂ ಕಪಟಂ ವೇಷಂ ದಂಡವತ್ಪತಿತೋ ಭುವಿ~।}\\\textbf{ತ್ವಂ ರುದ್ರೋsಸಿ ನ ಸಂದೇಹೋ ಭಾಗ್ಯಾ ದೃಷ್ಟಿ ಪಥಂ ಗತಃ~।। ೬~।।}
\end{verse}

ಆ ಕಪಟವೇಷಧಾರಿಯು ಸ್ತೋತ್ರಕ್ಕೆ ಅನರ್ಹನೆಂದು ತಿಳಿದೂ ಸಹ ನಾನು ಆತನಿಗೆ ಭೂಮಿಯಮೇಲೆ ದಂಡಪ್ರಣಾಮ ಮಾಡಿದೆ. “ನೀನು ಸಾಕ್ಷಾತ್ ರುದ್ರದೇವರೇ ಸರಿ. ನನ್ನ ಭಾಗ್ಯದಿಂದ ದರ್ಶನಕೊಟ್ಟಿರುವಿ.

\begin{verse}
\textbf{ತಸ್ಮಾತ್ಪುನರದೃಶ್ಯೋಸಿ ಕೃಪಾಂ ಕುರು ಕೃಪಾನಿಧೇ~।}\\\textbf{ತವ ವಾಗಮೃತೆಃ ಸಮ್ಯಗ್ಮಾಮುಜ್ಜೀವಯ ತಾಪಸ~।। ೭~।।} 
\end{verse}

\begin{verse}
\textbf{ನಿಷ್ಫಲಾ ದೃಶ್ಯತೇ ಕ್ವಾಪಿ ಜಾತಾ ದೃಷ್ಟಿರ್ಮಹಾತ್ಮನಾಮ್~।}\\\textbf{ತ್ವದರ್ಥಂ ಜೀವನಂ ಚೇದಂ ರಕ್ಷ ವಾ ಭುಂಕ್ಷ್ವ ವಾ ಪ್ರಭೋ~।। ೮~।।}
\end{verse}

ಈಗ ಅದೃಶ್ಯನಾಗಿರುವಿ. ನನ್ನ ಮೇಲೆ ಅನುಗ್ರಹ ಮಾಡು, ಹೇ ಕೃಪಾನಿಧೇ! ನಿನ್ನ ಅಮ್ಮತ ಸದೃಶವಾದ ಮಾತುಗಳಿಂದ, ತಪಸ್ವಿಯೇ ನನಗೆ ಜೀವನದ ಅಭ್ಯುದಯವನ್ನು ನೀಡು\enginline{-} ಮಹಾತ್ಮರ ದರ್ಶನವು ಎಂದಿಗಾದರೂ ವಿಫಲವಾಗುವುದೇ? ಆದುದರಿಂದ ಕೃಪೆಮಾಡಿ ನನ್ನನ್ನು ರಕ್ಷಿಸು ಇಲ್ಲವೆ ಸಂಹರಿಸು” ಎಂದು ಪ್ರಾರ್ಥಿಸಿದೆ.

\begin{verse}
\textbf{ಪಶ್ಚಾದಾವಿರಭೂತ್ಪಾಪೀ ಜಟಿಲಃ ಸಿದ್ಧಪಾದುಕಃ~।}\\\textbf{ಕೃತ್ವಾ ತು ಪಾದಯೋರ್ಲೇಪಂ ತರಿತ್ವಾ ಮೇರುಪರ್ವತಮ್~।। ೯~।।}
\end{verse}

ನಂತರ ಆ ಪಾಪಿಯಾದ ಜಟಾಧಾರಿಯು ಪಾದುಕೆಗಳನ್ನು ಧರಿಸಿ ಎದುರಿಗೆ ನಿಂತುಕೊಂಡನು. ಪಾದಗಳಿಗೆ ಏನನ್ನೋ ಲೇಪಿಸಿಕೊಂಡನು. ನಂತರ ಮೇರು ಪರ್ವತವನ್ನೇರಿದನು.

\begin{verse}
\textbf{ಜಗಾಮ ತತ್ರ ಸರ್ವಾಣಿ ವಿಶೇಷಾಣಿ ಚ ತಾನಿ ಚ~।}\\\textbf{ದರ್ಶಯಾಮಾಸ ವಿಪ್ರಾಯ ಸೂರ್ಯಂ ದೇವಾನ್ಮುನೀನಪಿ~।। ೧೦~।। }
\end{verse}

\begin{verse}
\textbf{ಹಿಮವಂತಂ ತತೋ ಗತ್ವಾ ಮದ್‌ಗೃಹಂ ಪುನರಾಗತಃ~।}\\\textbf{ಭುಕ್ತ್ವಾ ತು ಮದ್‌ಗೃಹೇ ಪಶ್ಚಾತ್ ಸ್ವಯಂ ಗಂತುಮುಪಾಕ್ರಮತ್~।। ೧೧~।।}
\end{verse}

ಅಲ್ಲಿಗೆ ನನಗೆ ಅನೇಕ ಆಶ್ಚರ್ಯಕರವಾದ ವಸ್ತುಗಳನ್ನೂ ತೋರಿಸಿದನು. ಸೂರ್ಯನನ್ನೂ, ದೇವತೆಗಳನ್ನೂ, ಮುನಿಗಳನ್ನೂ ತೋರಿಸಿದನು. ಅಲ್ಲಿಂದ ಹಿಮವತ್ ಪರ್ವತಕ್ಕೆ ತೆರಳಿ ಪುನಃ ನನ್ನ ಮನೆಗೆ ಬಂದನು. ನನ್ನ ಮನೆಯಲ್ಲಿ ಚೆನ್ನಾಗಿ ಊಟಮಾಡಿ ಮುಂದೆ ಹೊರಡಲು ಉದ್ಯುಕ್ತನಾದನು.

\begin{verse}
\textbf{ಪಾದೌ ಗೃಹೀತ್ವಾ ತಂ ಪಾಪಂಗೃಹೇ ಪ್ರಸ್ಥಾ ತುಮರ್ಹಥ~।}\\\textbf{ಪ್ರಾರ್ಥಿತೋsಪಿ ಮಯಾ ಸೋಪಿ ತಸ್ಥೌ ವಾರ್ಷಿಕಮಾಸಕಮ್~।। ೧೨~।।}
\end{verse}

ಆಗ ನಾನು ಆ ಪಾಪಿಯ ಪಾದಗಳನ್ನು ಹಿಡಿದು ಕೆಲವು ಕಾಲ ನನ್ನ ಮನೆಯಲ್ಲಿಯೇ ವಾಸಮಾಡಲು ಪ್ರಾರ್ಥಿಸಿದೆ. ನನ್ನ ಪ್ರಾರ್ಥನೆಯನ್ನು ಲಾಲಿಸಿ ಅವನು ಒಂದು ವರ್ಷದ ಮೇಲೆ ಒಂದು ತಿಂಗಳು ಕಾಲ ಇದ್ದನು.

\begin{verse}
\textbf{ಸ್ಥಿ ತ್ವಾ ಪಾಖಂಡವಾರ್ಗೇಣ ಮಾಂ ಬುಬೋಧ ಪುನಶ್ಚ ಸಃ~।}\\\textbf{ತೇನಾಹಂ ಬೋಧಿತಃ ಶೀಘ್ರಂ ಪಾಖಂಡೇ ಬೋಧಿತೋಽಭವಮ್~।। ೧೩~।।}
\end{verse}

ನನ್ನ ಮನೆಯಲ್ಲಿದ್ದಾಗ ನನಗೆ ಪಾಖಂಡಮತವನ್ನು ಪುನಃ ಉಪದೇಶ ಮಾಡಿದನು. ಅವನಿಂದ ಉಪದೇಶ ಪಡೆದು ನಾನು ಶೀಘ್ರವಾಗಿ ಪಾಖಂಡವಾರ್ಗವನ್ನು ಅನುಸರಿಸಲು ಪ್ರಾರಂಭಿಸಿದೆ. (ಪಾಖಂಡಮತ: ಪರಮ ಪ್ರಮಾಣವಾದ ವೇದಗಳನ್ನು ಅಲ್ಲಗಳೆದು ವೇದವಿರೋಧವಾದ ಕರ್ಮಗಳನ್ನು ಆಚರಿಸುವ ಮತ)~।

\begin{verse}
\textbf{ಮಾಂ ವಿಧಾಯ ತತಃ ಶಿಷ್ಯಂ ಸೋಽಗಮದ್ವಿಷಯಾಂತರಮ್~।}\\\textbf{ತಸ್ಮಾದನುಪಥಂ ಮಾಘಮಾಸೋಽಪಿ ಹ್ಯಗಮತ್ತದಾ~।। ೧೪~।।}
\end{verse}

ನನ್ನನ್ನು ತನ್ನ ಶಿಷ್ಯನಾಗಿಟ್ಟು ಅವನು ಬೇರೆಲ್ಲಿಯೋ ಹೊರಟುಹೋದನು. ಆಗ ಮಾಘಮಾಸವು ಪ್ರಾಪ್ತವಾಯಿತು.

\begin{verse}
\textbf{ಭಾರ್ಯಾ ಮೇ ಸ್ನಾ ತುಕಾಮಾಸೀದ್ಯ ಯಾಚೇ ಮಾಂ ಪುನಃ ಪುನಃ~।}\\\textbf{ನಾಪಿ ಚಾಜ್ಞಾ ಮಯಾ ದತ್ತಾ ವೇದಮಾರ್ಗವಿನಿಂದಕಮ್~।। ೧೫~।।}
\end{verse}

ವೇದಮಾರ್ಗವನ್ನು ನಿಂದಿಸುತ್ತಿದ್ದ ನನ್ನನ್ನು ನನ್ನ ಪತ್ನಿಯು ತಾನು ಮಾಘ ಸ್ನಾನಮಾಡಲು ಅಪ್ಪಣೆಯನ್ನು ಕೋರಿದಳು. ನಾನು ಆಜ್ಞೆಯನ್ನು ನೀಡಲಿಲ್ಲ.

\begin{verse}
\textbf{ಸಾಹಿ ಮಾಂ ಬೋಧಯಾಮಾಸ ಧರ್ಮವಾಕ್ಯೈಃ ಪುನಃ ಪುನಃ~।}\\\textbf{ವೇದಾಭ್ಯಾಸೀ ಕುಲೀನಶ್ಚ ಜಮದಗ್ನಿ ಕುಲೋದ್ಭವಃ~।। ೧೬~।।} 
\end{verse}

\begin{verse}
\textbf{ಕಥಂ ಪಾಖಂಡಮಾರ್ಗೇಷು ವರ್ತಸೇ ಜ್ಞಾನದುರ್ಬಲ~।}\\\textbf{ಮಾಂ ನಿಷೇಧಸಿ ಕಸ್ಮಾದ್ವಾ ಮಾಘಸ್ನಾನವಿಧೌ ರತಾಮ್~।। ೧೭~।।}
\end{verse}

ನನ್ನ ಪತ್ನಿ ಯು ಅನೇಕ ಸ್ಮೃತಿವಾಕ್ಯಗಳಿಂದ ಧರ್ಮಬೋಧನೆಯನ್ನು ಮಾಡಿದಳು. “ನೀವು ಒಳ್ಳೆಯ ಕುಲದಲ್ಲಿ ಉತ್ಪನ್ನರಾದವರು; ಜಮದಗ್ನಿಗೋತ್ರದವರು; ವೇದಾಧ್ಯಯನ ಮಾಡಿದವರು. ಜ್ಞಾನದುರ್ಬಲರಾದ ನೀವು ಈಗ ಪಾಖಂಡವುತವನ್ನು ಹೇಗೆ ಅನುಸರಿಸುತ್ತಿರುವಿರಿ? ಮಾಘಸ್ನಾನದಲ್ಲಿ ಅಕ್ಕರೆಯುಳ್ಳ ನನ್ನನ್ನು ಏತಕ್ಕಾಗಿ ಸ್ನಾನಮಾಡಬೇಡವೆಂದು ನಿಷೇಧಮಾಡುತ್ತಿರುವಿರಿ?”

\begin{verse}
\textbf{ಇತಿ ಬ್ರುವಾಣಾಂ ತಾಂ ಭಾರ್ಯಾಂ ದಂಡೇನಾಹಂ ನ್ಯವರ್ತಯಮ್~।}\\\textbf{ಸಾ ದುಃಖಾತ್ತತ್ಯಜೇ ಮಾಘಸ್ನಾನಂ ಬಾಷ್ಪಾ ಕುಲೇಕ್ಷಣಾ~।। ೧೮~।।}
\end{verse}

ಹೀಗೆ ನುಡಿದ ಪತ್ನಿಯನ್ನು ಕೋಲಿನಿಂದ ಹೊಡೆದೆ. ಕಣ್ಣೀರು ಸುರಿಸುತ್ತಾ ಅವಳು ಮಾಘಸ್ನಾನಮಾಡುವ ಕರ್ಮವನ್ನು ತ್ಯಜಿಸಿದಳು.

\begin{verse}
\textbf{ನ ಕೃತಂ ತು ಮಯಾ ಸ್ನಾನಂ ದೂಷಿತಂ ತು ಪುನಃ ಪುನಃ~।}\\\textbf{ತೇನ ದೋಷೇಣ ಚಾಲ್ಪಾಯುರಸ್ಮಾನ್ಪಾಪಾನ್ಮೃ ತೋsಸ್ಮ್ಯಹಮ್~।। ೧೯~।।}
\end{verse}

ನಾನು ಒಂದು ದಿನವೂ ಮಾಘಸ್ನಾನ ಮಾಡಲಿಲ್ಲ. ಮಾಘಸ್ನಾನವನ್ನು ಪದೇ ಪದೇ ನಿಂದಿಸುತ್ತಿದ್ದೆ, ಈ ತಪ್ಪಿಗಾಗಿ ನಾನು ಅಲ್ಪಾಯುಷಿಯಾಗಿ ಬೇಗನೆ ಸತ್ತು ಹೋದೆ.

\begin{verse}
\textbf{ಜಾತೋಽಹಂ ಗುಣನಾಮೈವಂ ಪಿಪ್ಪಲೇ ಬ್ರಹ್ಮರಾಕ್ಷಸಃ~।}\\\textbf{ಅಮಾಘೇಽ ನೈವ ಮುಕ್ತೇ ತು ದೂತಿಗೋ ವಾಕ್ಯಮಬ್ರವೀತ್~।। ೨೦~।।}
\end{verse}

ನಾನು ಕರ್ಮಾನುಸಾರವಾಗಿ ಈ ಅಶ್ವತ್ಥವೃಕ್ಷದಲ್ಲಿ "ಅಮಾಘ"ನೆಂಬ ಹೆಸರಿನಿಂದ ಬ್ರಹ್ಮರಾಕ್ಷಸನಾದೆ. ಅಮಾಘನು ಈ ರೀತಿ ನುಡಿಯಲು ದೂತಿಗನು ಹೇಳಿದನು.

\textbf{[ವಿಶೇಷಾಂಶ:\enginline{-}}ಪತಿಯು ಅಧರ್ಮಮಾರ್ಗದಲ್ಲಿ ವರ್ತಿಸಿದರೆ ಆತನನ್ನು ಎಚ್ಚರಿಸಿ,\break ಸನ್ಮಾರ್ಗಕ್ಕೆ ಬರುವಂತೆ ಹೇಳುವುದು ಪತಿವ್ರತಾ ಸ್ತ್ರೀಯರ ಕರ್ತವ್ಯ- ಅಧರ್ಮಸ್ಥಮಪಿ ಹ್ಯೇನಂ ಸಾಂತಪೂರ್ವಂ ಪ್ರಬೋಧಯೇತ್- ವಿಷ್ಣು ರಹಸ್ಯ.]

\begin{flushleft}
\textbf{ದೂತಿಗ ಉವಾಚ\enginline{-}}
\end{flushleft}

\begin{verse}
\textbf{ಅಹಂ ಕಾಶ್ಮೀರದೇಶೀಯೋ ಬ್ರಾಹ್ಮಣೋ ವೇದಪಾರಗಃ~।}\\\textbf{ಧನೀ ಶೈವಲಗೋತ್ರೋಽಹಂ ನಾಮ್ನಾ ಬಾಷ್ಕಲಸಂಜ್ಞಿತಃ~।। ೨೧~।। }
\end{verse}

\begin{flushleft}
ದೂತಿಗನು ಹೇಳಿದನು:-
\end{flushleft}

ನಾನು ಹಿಂದೆ ಕಾಶ್ಮೀರದೇಶದವನು, ಬ್ರಾಹ್ಮಣ, ವೇದಾಧ್ಯಯನ ಮಾಡಿದ್ದೆ. ಶೈವಲಗೋತ್ರದಲ್ಲಿ ಉತ್ಪನ್ನನಾಗಿ ಹಣವಂತನಾಗಿದ್ದೆ. ಆಗ ನನ್ನ ಹೆಸರು ಬಾಷ್ಕಲ.

\begin{verse}
\textbf{ಮೃತೇ ಪಿತರಿ ಮೇ ಚಾಸೀತ್ಕಾಲಃ ಷೋಡಶವಾರ್ಷಿಕಃ~।}\\\textbf{ಪಿತುಃ ಸಕಾಶಾಚ್ಛಾಧೀತಂ ವೇದವೇದಾಂಗಪಂಚಕಮ್~।। ೨೨~।। }
\end{verse}

\begin{verse}
\textbf{ವಿಸ್ಮೃತಂ ಚ ಮಯಾ ಸರ್ವಂ ನ ಪುನಸ್ತದ್ವಿಮರ್ಶಿತಮ್~।}\\\textbf{ಜ್ಞಾನೇ ಲುಪ್ತೇ ತತಃ ಪಶ್ಚಾದ್ದಾಸೀಸಂಗೇ ರತೋsಸ್ಮ್ಯಹಮ್~।। ೨೩~।।}
\end{verse}

ನನ್ನ ತಂದೆಯು ಮೃತರಾದಾಗ ನನಗೆ ಹದಿನಾರು ವರ್ಷ, ತಂದೆಯವರಿಂದ ಉಪದೇಶಿಸಲ್ಪಟ್ಟ ವೇದವೇದಾಂಗಗಳನ್ನು ಮರೆತೆ. ಪುನಃ ಆ ಗ್ರಂಥಗಳ ವಿಮರ್ಶನೆಯನ್ನಾಗಲೀ, ಅಭ್ಯಾಸವನ್ನಾಗಲೀ ಮಾಡಲಿಲ್ಲ. ನನ್ನ ಜ್ಞಾನವು ನಾಶಹೊಂದಿದ ನಂತರ ಸದಾ ದಾಸಿಯರ ಜತೆ ರಮಿಸುತ್ತಿದ್ದೆ.

\begin{verse}
\textbf{ತ್ಯಕ್ತಾ ಗುಣವತೀ ಭಾರ್ಯಾ ಕಾಕಪಕ್ಷಾತ್ಮಜಾ ಮಯಾ~।}\\\textbf{ಹವ್ಯಂ ನಾನುಷ್ಠಿತಂ ಕಿಂಚಿನ್ನ ಶ್ರಾದ್ಧಾದಿ ಮಯಾ ಕೃತಮ್~।। ೨೪~।।}
\end{verse}

ಗುಣವಂತೆಯಾದ, ಕೋಗಿಲೆಯ ಕಂಠದಂತೆ ಇಂಪಾದ ಧ್ವನಿಯುಳ್ಳ ನನ್ನ ಪತ್ನಿಯನ್ನು ತೊರೆದೆ. ಹವ್ಯ, ಶ್ರಾದ್ಧಾದಿ ಕರ್ಮಗಳನ್ನು ಆಚರಿಸಲಿಲ್ಲ.

\begin{verse}
\textbf{ಭಾರ್ಯಾತ್ವೇ ಸ್ಥಾಪಿತಾ ದಾಸೀ ದಾಸೀತ್ವೇ ಚ ಕುಲಾಂಗನಾ~।}\\\textbf{ದಾಸೀಪುತ್ರಾಶ್ಚ ಮೇ ಪುತ್ರಾಶ್ಚತ್ವಾರಸ್ತೇ ಮಮಾನುಗಾಃ~।। ೨೫~।।}
\end{verse}

ದಾಸಿಯನ್ನು ಪತ್ನಿ ಯೆಂತಲೂ, ಪತ್ನಿಯನ್ನು ದಾಸಿಯೆಂತಲೂ ಕಾಣುತ್ತಿದ್ದೆ. ದಾಸಿಯಿಂದ ನನಗೆ ನಾಲ್ಕು ಜನ ಪುತ್ರರು ಹುಟ್ಟಿದರು. ಆ ಪುತ್ರರು ನನ್ನ ಹಾದಿಯಲ್ಲೇ ಬರುವರೆಂದು ತಿಳಿದೆ.

\begin{verse}
\textbf{ಏವಂ ಕಾಲೋ ಮಯಾ ನೀತಃ ಸರ್ವೋಽಪ್ಯೇವಂ ದುರಾತ್ಮನಾ~।}\\\textbf{ತತೋ ಮೃ ತಿಮವಾಪ್ತೋಽಹಂ ಸರ್ವಧರ್ಮಬಹಿಷ್ಕೃತಃ~।। ೨೬~।।}
\end{verse}

ದುರಾತ್ಮನಾದ ನಾನು ಹೀಗೆ ಕಾಲವನ್ನು ಕಳೆದೆ. ಸಕಲ ಸತ್ಕರ್ಮಗಳನ್ನೂ ತ್ಯಜಿಸಿದ ನಾನು ಮೃತನಾದೆ.

\begin{verse}
\textbf{ಕರ್ಮಣಾ ತೇನ ಜಾತೋಽಹಂ ದೂತಿಕೋ ಬ್ರಹ್ಮರಾಕ್ಷಸಃ~।}\\\textbf{ಇತ್ಯುಕ್ತ್ವಾ ವಿರರಾಮಾಥ ನಿಂದಕೊ ವಾಕ್ಯಮಬ್ರವೀತ್~।। ೨೭~।।}
\end{verse}

ಅಂತಹ ದುಷ್ಕರ್ಮದ ಫಲವಾಗಿ “ದೂತಿಕ'ನೆಂಬ ಹೆಸರಿನಿಂದ ಬ್ರಹ್ಮರಾಕ್ಷಸನಾಗಿ ಹುಟ್ಟಿದೆ. ಹೀಗೆಂದು ಹೇಳಿ ದೂತಿಕನು ಸುಮ್ಮನಾದನು. ನಂತರ `ನಿಂದಕನು' ನುಡಿದನು.

\begin{flushleft}
\textbf{ನಿಂದಕ ಊವಾಚ:\enginline{-}}
\end{flushleft}

\begin{verse}
\textbf{ಶೃಣು ವಿಪ್ರ ಪ್ರವಕ್ಷ್ಯಾಮಿ ಮಮ ವೃತ್ತಾಂತಮಂಜಸಾ~।}\\\textbf{ಪುರಾಹಂ ಮಗಧೇ ದೇಶೇ ಗ್ರಾಮೇ ಪಂಚಜನಾಹ್ವಯೇ~।। ೨೮~।।}
\end{verse}

ಬ್ರಾಹ್ಮಣನೇ ನನ್ನ ವೃತ್ತಾಂತವನ್ನು ಪಾಲಿಸು. ನಾನು ಹಿಂದೆ ಮಗಧ ದೇಶದಲ್ಲಿ 'ಪಂಚಜನ' ವೆಂಬ ಊರಿನಲ್ಲಿದ್ದೆ.

\begin{verse}
\textbf{ಬ್ರಾಹ್ಮಣಃ ಕಾಶ್ಯಪೋ ನಾಮ ಪೈಂಗ್ಯಗೋತ್ರಸಮುದ್ಭವಃ~।}\\\textbf{ಶಂಕುವರ್ಣಸ್ಯ ಶಿಷ್ಯೋಽಹಂ ತತ್ರಾಧೀತಾಃ ಸಮಸ್ತಶಃ~।। ೨೯~।।}
\end{verse}

ನನ್ನ ಹೆಸರು ಕಾಶ್ಯಪ, ಪೈಂಗ್ಯ ಗೋತ್ರದವನು, ಬ್ರಾಹ್ಮಣ, ಶಂಕುವರ್ಣರೆಂಬುವರ ಶಿಷ್ಯ. ಸಮಸ್ತ ವಿದ್ಯೆಗಳನ್ನೂ ಅವರಲ್ಲಿ ಅಧ್ಯಯನ ಮಾಡಿದ್ದೆ.

\begin{verse}
\textbf{ವೇದಾಃ ಷಡಂಗಃ ಸ್ಮೃತಯಃ ಪುರಾಣಂ ಚೇತಿಹಾಸಕಮ್~।}\\\textbf{ಅನ್ವೀಕ್ಷಿಕೀ ತಥಾ ವಿದ್ಯಾ ಮೀಮಾಂಸಾ ದ್ವಯಮೇವ ಚ~।। ೩೦~।।}
\end{verse}

\begin{verse}
\textbf{ಗೀತಾಯಾಶ್ಚ ಕಲಾಃ ಸರ್ವಾ ನಾಧೀತಂ ವಿದ್ಯತೇ ಮಯಾ~।}
\end{verse}

ಷಡಂಗ ಸಹಿತ ವೇದಗಳನ್ನು, ಪುರಾಣಗಳನ್ನು, ಸ್ಮೃತಿಗಳನ್ನು, ಇತಿಹಾಸಗಳನ್ನು, ತರ್ಕವಿದ್ಯೆಯನ್ನು, ಎರಡು ವಿರಾಮಾಂಸಗಳನ್ನು, ಅರ್ಥಸಹಿತ ಸಮಗ್ರ ಗೀತೆಯನ್ನು, ಸಕಲ ಕಲೆಗಳನ್ನು ಶಂಕುವರ್ಣರೆಂಬುವರಿಂದ ಕಲಿತಿದ್ದೆ. (ಮೀಮಾಂಸ-ಪೂರ್ವ ಮೀಮಾಂಸ; ಉತ್ತರ ಮೀಮಾಂಸ) ನಾನು ಕಲಿಯದೇ ಇದ್ದದ್ದು ಯಾವುದೂ ಇರಲಿಲ್ಲ.

\begin{verse}
\textbf{ದ್ವಾಪರಸ್ಯ ಯುಗಸ್ಯಾಂತೇ ರಾಜ್ಞೋ ಜ್ಯೋತಿಷ್ಮತೀಪತೇಃ~।। ೩೧~।।}
\end{verse}

\begin{verse}
\textbf{ಧರ್ಮಾಧಿಕಾರಮಾಪನ್ನಃ ಪೌರೋಹಿತ್ಯೇ ನಿಯೋಜಿತಃ~।}\\\textbf{ಲೋಪಾಮುದ್ರಪ್ರಸಾದೇನ ಜಯಶೀಲೋ ನಿರಾಕುಲಃ~।। ೩೨~।।}
\end{verse}

ದ್ವಾಪರಯುಗದ ಅಂತ್ಯಭಾಗದಲ್ಲಿ ಜ್ಯೋತಿಷ್ಮತೀ ಪಟ್ಟಣದ ರಾಜನಿಂದ ನಾನು ಧರ್ಮಾಧಿ\-ಕಾರಿ ಪದವಿಯನ್ನು ಪಡೆದು ಪುರೋಹಿತನಾಗಿ ನಿಯೋಜಿಸಲ್ಪಟ್ಟೆ. ಲೋಪಾಮುದ್ರೆಯ ಅನುಗ್ರಹದಿಂದ ಸಕಲ ಕಾರ್ಯಗಳಲ್ಲಿಯೂ ಜಯಶೀಲನಾಗಿಯೂ, ಭಯರಹಿತನಾಗಿಯೂ ಇದ್ದೆ.

\begin{verse}
\textbf{ರಾಜದ್ವಾರಿ ಮಯಾ ಸರ್ವೇ ವಿದ್ವಾಂಸೋ ನಿರ್ಜಿತಾಃ ಕ್ಷಣಾತ್~।}\\\textbf{ಅವಿದ್ವಾನಯಮಿತ್ಯುಕ್ತ್ವಾ ಪ್ರತ್ಯಪಾದಿ ಸುಧೀರಪಿ~।। ೩೩~।। }
\end{verse}

\begin{verse}
\textbf{ನ ದಾಪಿತಂ ಯಾಚಕಸ್ಯ ಕ್ವಾಪಿ ಪುಣ್ಯ ದಿನೇಷ್ವಪಿ~।}\\\textbf{ಮದೀಯಾಃ ಪೋಷಿತಾಃ ಸರ್ವೇ ತೇನ ರಾಜ್ಞಾ ಮಹಾತ್ಮನಾ~।। ೩೪~।।}
\end{verse}

ಅರಮನೆಯ ಸಭಾಂಗಣದ ಬಾಗಿಲಿನಲ್ಲಿಯೇ ಎಂತಹ ವಿದ್ವಾಂಸನು ಬಂದರೂ ಅವನನ್ನು ಸೋಲಿಸಿ ಇವನಿಗೆ ಏನೂ ಗೊತ್ತಿಲ್ಲವೆಂದು ಹೇಳುತ್ತಿದ್ದೆ, ಎಂತಹ ಪುಣ್ಯದಿವಸವು ಬಂದರೂ ರಾಜನಿಂದ ಯಾರಿಗೂ ಏನನ್ನೂ ದಾನ ಮಾಡಿಸುತ್ತಿರಲಿಲ್ಲ. ಆ ಮಹಾತ್ಮನಾದ ರಾಜನಿಂದ ನನಗೆ ಸಂಬಂಧಪಟ್ಟ ಜನರೆಲ್ಲರೂ ಚೆನ್ನಾಗಿ ರಕ್ಷಿಸಲ್ಪಡುತ್ತಿದ್ದರು.

\begin{verse}
\textbf{ಧನಂ ಗೃಹೀತ್ವಾ ಕೇನಾಪಿ ಅಧರ್ಮೇ ಧರ್ಮ ಇತ್ಯಪಿ~।}\\\textbf{ಸಾಧಯಿತ್ವಾ ಕ್ವಚಿದ್ದರ್ಮಮಧರ್ಮಮಿತಿ ಚಾವದಮ್~।। ೩೫~।।}
\end{verse}

ನಾನು ಯಾರಿಂದಲಾದರೂ ಅಧರ್ಮದಿಂದ ದ್ರವ್ಯವನ್ನು ಪಡೆದರೆ ಧರ್ಮವೆಂತಲೂ, ಇತರರು ಧರ್ಮಮಾರ್ಗದಿಂದ ಪಡೆದರೂ ಅಧರ್ಮವೆಂತಲೂ ಸಾಧಿಸುತ್ತಿದ್ದೆ.

\begin{verse}
\textbf{ಆಚಾರಾದ್ವ್ಯ ವಹಾರಾಚ್ಚ ಮಯಾ ಸರ್ವೇಽಪಿ ನಿಂದಿತಾಃ~।}\\\textbf{ನ ಕೋಽಪಿ ಸದೃಶೋ ಲೋಕೇ ನ ಭೂತೋ ನ ಚ ವರ್ತತೇ~।। ೩೬~।।}
\end{verse}

\begin{verse}
\textbf{ಇತ್ಯಹಂಕಾರಮೂಢೇನ ಮಹಾಂತೋಽಪಿ ವಿನಿಂದಿತಾಃ~।}
\end{verse}

ಆಚಾರ-ವಿಚಾರ-ವ್ಯವಹಾರಗಳಲ್ಲಿಯೂ ನಾನು ಎಲ್ಲರನ್ನೂ ದೂಷಿಸುತ್ತಿದ್ದೆ. ನನ್ನ ಸಮಾನನಾದವನು ಲೋಕದಲ್ಲಿ ಹಿಂದೆ ಇದ್ದಿಲ್ಲ, ಈಗಿಲ್ಲ ಎಂಬ ಅಹಂಕಾರದಿಂದ ಮಹಾತ್ಮರನ್ನೂ ಸಹ ನಿಂದಿಸುತ್ತಿದ್ದೆ.

\begin{verse}
\textbf{ಕದಾಚಿತ್ ಗುರುಪುತ್ರೋಽಪಿ ಯಜ್ಞಾರ್ಥಿ ರಾಜಮಂದಿರಮ್~।। ೩೭~।।} 
\end{verse}

\begin{verse}
\textbf{ಆಗತೋ ಮಮ ಸಖ್ಯೇನ ಸ ಮಯಾಪಿ ನಿರಾಕೃತಃ~।}\\\textbf{ದೂಷಿತೋ ಬಹುವಿದ್ಯಾನಾಂ ಪ್ರಸಂಗೇನ ದುರಾತ್ಮನಾಮ್~।। ೩೮~।। }
\end{verse}

\begin{verse}
\textbf{ಸ ತೂರ್ಣಮಗಮದ್ಗೇಹಂ ಮಯಾಸೌ ವಿಮನೀಕೃತಃ~।}\\\textbf{ತತಃ ಮೃತಿಂ ಗತಃ ಪಾಪೀ ಕರ್ಮಣಾನೇನ ದುಷ್ಟಧೀಃ~।। ೩೯~।।}
\end{verse}

ಒಂದು ದಿನ ನನ್ನ ಗುರುಪುತ್ರನು ಯಜ್ಞಕ್ಕಾಗಿ ದ್ರವ್ಯವನ್ನು ಯಾಚಿಸಲು ನನ್ನ ಸ್ನೇಹದ ವ್ಯಾಜದಿಂದ ಅರಮನೆಗೆ ಬಂದನು. ಅವನನ್ನೂ ಸಹ ನಾನು ತಿರಸ್ಕರಿಸಿದೆ. ಆದರೂ ಸಹ ಅವನು ವಿದ್ಯಾನಿಮಿತ್ತದಿಂದ ನನ್ನ ಮನೆಗೆ ಬಂದನು. ಆಗಲೂ ಸಹ ನಾನು ಅವನನ್ನು ತಾತ್ಸಾರಮಾಡಿದೆ. ಇಂತಹ ಪಾಪಾತ್ಮನಾದ ನಾನು ಕರ್ಮವಶದಿಂದ ಮೃತಿಯನ್ನು ಹೊಂದಿದೆ.

\begin{verse}
\textbf{ನಾಮ್ನಾ ಚ ನಿಂದಕೋ ನಾಮ ಜಾತೋಽಹಂ ಬ್ರಹ್ಮರಾಕ್ಷಸಃ~।}\\\textbf{ತಸ್ಯ ವಾಕ್ಯವಿರಾಮಾಂತೇ ಷಷ್ಠೋ ವಾಕ್ಯ ಮಥಾಬ್ರವೀತ್~।। ೪೦।।}
\end{verse}

ನಿಂದಕನೆಂಬ ಹೆಸರಿನ ಬ್ರಹ್ಮರಾಕ್ಷಸನಾದೆ. ಹೀಗೆಂದು ಹೇಳಿದನಂತರ ಆರನೆಯವನಾದ ಮಲಭೋಜನನು ಮಾತನಾಡಿದನು.

\begin{flushleft}
\textbf{ಮಲಭೋಜ ಉವಾಚ\enginline{-}}
\end{flushleft}

\begin{verse}
\textbf{ದೇವಯಾಜೀತಿನಾಮ್ನಾಹಂ ವಂಗದೇಶೇ ಧನಾಧಿಪಃ~।}\\\textbf{ಆಗಸ್ತ್ಯಗೋತ್ರೇ ಚೋತ್ಪನ್ನೋ ಗ್ರಾಮೇ ಮಂಡಲಸಂಜ್ಞಕೇ~।। ೪೧~।।}
\end{verse}

\begin{flushleft}
ಮಲಭೋಜನನು ನುಡಿದನು-
\end{flushleft}

ನಾನು ಹಿಂದೆ ವಂಗದೇಶದಲ್ಲಿ ಅಗಸ್ಯಗೋತ್ರದಲ್ಲಿ ಉತ್ಪನ್ನನಾಗಿ ಮಂಡಲವೆಂಬ ಊರಿ\-ನಲ್ಲಿದ್ದೆ. ನನ್ನ ಹೆಸರು ದೇವಯಾಜೀ, ಐಶ್ವರ್ಯವಂತನಾಗಿದ್ದೆ.

\begin{verse}
\textbf{ದೃಷ್ಟಿಹೀನೌ ಮಹಾತ್ಮಾನೌ ಜರಠ್ಯೌ ಪಿತರೌ ಮಮ~।}\\\textbf{ನಾರ್ಚಿತೌ ಪಿತರೌ‌ ತೌ ತು ಕಲತ್ರಾಧೀನಚೇತಸಾ~।। ೪೨~।।}
\end{verse}

ಮಹಾತ್ಮರಾದ ನನ್ನ ತಂದೆ-ತಾಯಿಯರು ಕುರುಡರು, ವೃದ್ದರು. ಆದಾಗ್ಯೂ ಪತ್ನಿಯ ವಶದಲ್ಲಿದ್ದ ನಾನು ಅವರನ್ನು ಒಂದು ದಿನವೂ ಸತ್ಕರಿಸಲಿಲ್ಲ, ಗೌರವಿಸಲಿಲ್ಲ.

\begin{verse}
\textbf{ಕಷ್ಟಾನ್ನಂ ದೀಯತೇ ತಾಭ್ಯಾಂ ಮಿಷ್ಟಮಶ್ನಾಮಿ ಸಾರ್ಭಕಃ~।}\\\textbf{ಅದ್ಯ ಶ್ವೋ ವಾ ಮರಿಷ್ಯಾವಃ ಸ್ವಾದು ಕಿಂಚಿತ್ ಪ್ರದೀಯತಾಮ್~।। ೪೩~।।}
\end{verse}

ನಾನು ನನ್ನ ತಂದೆತಾಯಿಯರಿಗೆ ತುಂಬ ತುಚ್ಛವಾದ, ರುಚಿಯಿಲ್ಲದ ಆಹಾರವನ್ನು ಕೊಡುತ್ತಿದ್ದೆ. ನಾನು ಮಾತ್ರ ನನ್ನ ಮಕ್ಕಳೊಂದಿಗೆ ಮೃಷ್ಟಾನವನ್ನು ಊಟಮಾಡುತ್ತಿದ್ದೆ. ನನ್ನ ತಂದೆ-ತಾಯಿಯರು ಕೇಳಿದರು. ಪುತ್ರನೇ, ನಾವು ಇಂದೋ ನಾಳೆಯೋ, ಮೃತರಾಗುತ್ತೇವೆ; ರುಚಿಕರವಾದ ಆಹಾರವನ್ನು ಸ್ವಲ್ಪವಾದರೂ ಕೊಡು.”

\begin{verse}
\textbf{ಆವಯೋರಿದಮತ್ಯನ್ನಂ ಪ್ರಿಯಮಿತ್ಯುದಿತೇsಪಿ ಚ~।}\\\textbf{ನ ದತ್ತಂ ಚ ನ ಮಯಾ ಕಿಂಚಿತ್ಭರ್ತ್ಸಿತೌ ಚ ಪುನಃ ಪುನಃ~।। ೪೪~।।}
\end{verse}

'ಈಗ ಕೊಡುತ್ತಿರುವ ಅನ್ನವು ಸ್ವಲ್ಪವೂ ರುಚಿಯಿಲ್ಲ' ಹೀಗೆಂದು ಅವರು ಪದೇ ಪದೇ ಕೇಳುತ್ತಿದ್ದರೂ ನಾನು ಅವರಿಗೆ ಸರಿಯಾದ ಆಹಾರವನ್ನು ಕೊಡಲಿಲ್ಲ. ಪುನಃ ಪುನಃ ಅವರನ್ನು ದಂಡಿಸುತ್ತಿದ್ದೆ.

\begin{verse}
\textbf{ಆತಿಥ್ಯೇನಾಗತೇ ವಿಪ್ರೇ ಚಾನ್ನಮನ್ಯತ್ಪ್ರದೀಯತೇ~।}\\\textbf{ಮಯಾ ಚ ಭುಜ್ಯತೇ ಮಿಷ್ಟಂ ನಾರ್ಚಿತಾ ಮಿಕ್ಷುಕಾ ಮಯಾ~।। ೪೫~।।}
\end{verse}

ಬ್ರಾಹ್ಮಣರು ಅತಿಥಿಗಳಾಗಿ ನನ್ನ ಮನೆಗೆ ಬಂದರೆ ಅವರಿಗೆ ಬೇರೆ ಆಹಾರ ಕೊಟ್ಟು ನಾನು ಮಾತ್ರ ಮೃಷ್ಟಾನ್ನ ಭೋಜನ ಮಾಡುತ್ತಿದ್ದೆ. ಅವರಿಗೆ ಉಪಚಾರ, ಸತ್ಕಾರಾದಿಗಳನ್ನು ಮಾಡು\break ತ್ತಿರಲಿಲ್ಲ.

\begin{verse}
\textbf{ತೇನ ಕರ್ಮವಿಪಾಕೇನ ಮಹತಾ ಮುನಿಪುಂಗವ~।}\\\textbf{ನಾಮ್ನಾ ಚ ಕರ್ಮಣಾ ಚಾಪಿ ಜಾತೋಽಹಂ ಮಲಭೋಜನಃ~।। ೪೬~।।}
\end{verse}

ಅಂತಹ ಮಹತ್ತರವಾದ ಕರ್ಮದ ಕಾರಣದಿಂದ, ಮುನಿಶ್ರೇಷ್ಠನೇ, ನಾನು ಹೆಸರಿನಿಂದಲೂ, ಕರ್ಮದಿಂದಲೂ, ಮಲಭೋಜನ ಬ್ರಹ್ಮರಾಕ್ಷಸನಾಗಿ ಉತ್ಪನ್ನನಾಗಿರುತ್ತೇನೆ.

\begin{verse}
\textbf{ಮಲಭೋಜನವಾಕ್ಯಾಂತೇಽದೈವತೋ ವಾಕ್ಯಮಬ್ರವೀತ್~।}
\end{verse}

ಮಲಭೋಜನನು ಹೀಗೆ ನುಡಿದು ಸುಮ್ಮನಾಗಲು ಅದೈವತನೆಂಬ ಬ್ರಹ್ಮರಾಕ್ಷಸನು ನುಡಿದನು.

\newpage

\begin{flushleft}
\textbf{ಅದೈವತ ಉವಾಚ\enginline{-}}
\end{flushleft}

\begin{verse}
\textbf{ಶೃಣು ಬ್ರಹ್ಮನ್ ಪ್ರವಕ್ಷ್ಯಾಮಿ ನಿದಾನಂ ಮಮ ಜನ್ಮನಃ~।। ೪೭~।।}
\end{verse}

ಬ್ರಾಹ್ಮಣನೇ, ನನ್ನ ಈಗಿನ ಜನ್ಮಕ್ಕೆ ಕಾರಣವನ್ನು ಹೇಳುತ್ತೇನೆ, ಲಾಲಿಸು.

\begin{verse}
\textbf{ಪುರಾಹಂ ಗಣಕಶ್ಚಾಸ್ಮಿ ಕಾಶ್ಯಾಂ ಕಾಪೇಯಗೋತ್ರಜಃ~।}\\\textbf{ನಾಮ್ನಾಗ್ನಿದೂತ ಇತ್ಯಾಸಂ ವೇದವೇದಾಂಗಪಾರಗಃ~।। ೪೮~।।}
\end{verse}

ಹಿಂದೆ ನಾನು ಕಾಶಿಪಟ್ಟಣದಲ್ಲಿ ಕಾಪೇಯಗೋತ್ರದಲ್ಲಿ ಹುಟ್ಟಿದ್ದೆ. ನನ್ನ ಹೆಸರು ಅಗ್ನಿ ದೂತ, ಜೋಯಿಸನಾಗಿದ್ದೆ. ವೇದವೇದಾಂಗಗಳಲ್ಲಿ ಪಾರಂಗತನಾಗಿದ್ದೆ.

\begin{verse}
\textbf{ರಾಜ್ಞಾ ದತ್ತಾಧಿಕಾರಸ್ತು ತತೋ ದೇವಾಲಯೇ ಮಯಾ~।}\\\textbf{ಹೃತಂ ದೇವಸ್ಯ ಸರ್ವಸ್ವಂ ಧಾನ್ಯಂ ತೈಲಂ ಘೃತಂ ತಥಾ~।। ೪೯~।। }
\end{verse}

\begin{verse}
\textbf{ಪಿಷ್ಟಂ ಶಾಕಂ ಗುಡಂ ಚೈವ ತಂಡುಲಾಮರಿಚಾದಯಃ~।}\\\textbf{ಕುಡ್ಯಾದ್ಯಾರಂಭಕಾರ್ಯೇಷು ಜೀವಿಕಾರ್ಥೇ ಮಯಾ ಹೃತಮ್~।। ೫೦~।।}
\end{verse}

ರಾಜನಿಂದ ನಾನು ದೇವಸ್ಥಾನದ ಅಧಿಕಾರಿಯಾಗಿ ನೇಮಿತನಾದೆ, ದೇವರಿಗೆ ಸಂಬಂಧಪಟ್ಟ ಸಮಸ್ತ ವಸ್ತುಗಳನ್ನೂ ನಾನು ಅಪಹರಿಸಿದೆ. ದೇವರಿಗೆ ಸಲ್ಲ ಬೇಕಾದ ಅಕ್ಕಿ ಮೊದಲಾದ ಧಾನ್ಯಗಳು, ಎಣ್ಣೆ, ತುಪ್ಪ, ಹಿಟ್ಟು, ತರಕಾರಿ, ಬೆಲ್ಲ, ಮೆಣಸು-ಇವೇ ಮೊದಲಾದ ಪದಾರ್ಥಗಳನ್ನು ನನ್ನ ಜೀವನಕ್ಕಾಗಿ ಅಪಹರಿಸಿದೆ.

\begin{verse}
\textbf{ಗೃಹೀತಮತುಲಂ ದ್ರವ್ಯಮತಿಕ್ರಮ್ಯೈವ ವೇತನಮ್~।}\\\textbf{ರಾಜದ್ವಾರಿ ಪ್ರವಕ್ರೄಣಾಂ ದೇವಸ್ವಂ ದೀಯತೇ ಮಯಾ~।। ೫೧~।। }
\end{verse}

\begin{verse}
\textbf{ಅಪಿ ಸರ್ಷಪಮಾತ್ರಂ ವಾ ಧರ್ಮಾರ್ಥಂ ನೋಪಪಾದಿತಮ್~।}\\\textbf{ವ್ಯಯಂ ಕೃತಂ ತು ದೇವಸ್ವಮಪಾತ್ರೇ ಕೀರ್ತಿಮಿಚ್ಛಯಾ~।। ೫೨~।।}
\end{verse}

ಈ ರೀತಿಯಾದ ದುಷ್ಟವರ್ತನೆಯಿಂದ ನನ್ನ ಸಂಬಳಕ್ಕಿಂತ ಬಹಳ ಅಧಿಕವಾದ ದ್ರವ್ಯವನ್ನು ಅಪಹರಿಸಿದೆ. ರಾಜನ ಸಭೆಯಲ್ಲಿ ನನ್ನನ್ನು ಪ್ರಶಂಸಿಸಿ ಮಾತನಾಡಿದವರಿಗೆ ದೇವಾಲಯದ ಹಣವನ್ನು ಕೊಡುತ್ತಿದ್ದೆ. ಧರ್ಮಾರ್ಥವಾಗಿ ಅತ್ಯಂತ ಸ್ವಲ್ಪವನ್ನೂ ಸಹ ನಾನು ದಾನಮಾಡಲಿಲ್ಲ. ಆದರೆ ಕೀರ್ತಿಯನ್ನು ಗಳಿಸಲು ಅಪಾತ್ರರಿಗೆ ಕೊಡುತ್ತಿದ್ದೆ.

\begin{verse}
\textbf{ನಾರ್ಹೇರ್ಹೇ ಚ ವ್ಯಯಂ ಕರ್ತುಂ ವ್ಯಯೋ ಮೇ ಬಹುರೇವ ತು~।}\\\textbf{ನ ಶಿಷ್ಟಾಃ ಪೂಜಿತಾಃ ಕ್ವಾಪಿ ಪೂಜಿತಾಶ್ಚ ನಟಾದಯಃ~।। }\\\textbf{ಪಶ್ಚಾತ್ತಾಪೋ ನ ಚೈವಾಸೀತ್ ಸಂತೋಷೋ ವರ್ಧತೇ ಮಮ~।। ೫೩~।।}
\end{verse}

ಅಪಾತ್ರರಿಗೆ ಹಣಕೊಡುವುದರಲ್ಲಿ ನಾನು ಬಹು ಉತ್ಸಾಹದಿಂದ ಇರುತ್ತಿದ್ದೆ. ಶಿಷ್ಟರು, ಸಾಧು-\-ಸಜ್ಜನರನ್ನು ನಾನು ಎಂದೂ ಸತ್ಕರಿಸಲಿಲ್ಲ. ನಟರು, ಗಾಯಕರು ಇವರಿಗಾಗಿ ಬೇಕಾದಷ್ಟು ದಾನಮಾಡಿದೆ. ಈ ನನ್ನ ವರ್ತನೆಯಿಂದ ನನಗೆ ಪಶ್ಚಾತ್ತಾಪ ಆಗಲಿಲ್ಲ. ಸಂತೋಷವೇ ಅಧಿಕವಾಗುತ್ತಿತ್ತು.

\begin{verse}
\textbf{ದೇವದ್ವೇಷೋ ವೇದವಿಶ್ವಾಸಹಾನಿರ್ಗೀತಾಸಕ್ತೋ ನರ್ತವಿದ್ಯಾರತಿಶ್ಚ~।}\\\textbf{ಭೋಗೇ ಲೌಲ್ಯಂ ನಿಂದಯಾ ಕರ್ಮಣಾಂ ಚ ಸ್ಥಿತಿರ್ಜಾತಾ} \\\textbf{ಪಾಪಿನೋ ಮೇಽನಿಶಂ ಚ~।। ೫೪~।।}
\end{verse}

ಆಗ ನಾನು ದೇವರನ್ನು ದ್ವೇಷಿಸುತ್ತಿದ್ದೆ, ವೇದಾದಿಗಳನ್ನು ನಿಂದಿಸುತ್ತಿದ್ದೆ. ಗಾನ, ನರ್ತನ, ಸ್ತ್ರೀಭೋಗ, ಇತರರನ್ನು ನಿಂದಿಸುವುದು ಇವುಗಳಲ್ಲಿ ಆಸಕ್ತನಾಗಿ ಪಾಪಿಜನರ ಸ್ನೇಹವನ್ನು ಮಾಡುತ್ತಿದ್ದೆ.

\begin{verse}
\textbf{ನ ಧ್ಯಾ ತಂ ಪದಮೀಶ್ವರಸ್ಯ ಮಹತೀ ಸಂಧ್ಯಾಪಿ ನಾನುಷ್ಠಿ ತಾ~।}\\\textbf{ಸಂತಃ ಕ್ವಾಪಿ ನ ಸೇವಿತಾ ಮಹದಪಿ ಸ್ಥಾನಂ ನ ಯಾತಂ ಮಯಾ~।। }\\\textbf{ನೋ ದೇವಾಃ ಪರಿಪೂಜಿತಾ ಅಪಿ ಮಖೈರ್ಧರ್ಮಾ ಹಿ ನೋಪಾರ್ಜಿತಾ~।}\\\textbf{ಬುದ್ಧ್ಯಾ ವಾ ಸ್ಮೃತಿಬಂಧಮೋಚನಕರೀ ವಿದ್ಯಾ ಮಯಾ ಸಾಧಿತಾ~।। ೫೫~।।}
\end{verse}

ಜಗದೀಶನಾದ ಶ‍್ರೀಹರಿಯನ್ನು ಧ್ಯಾನಿಸಲಿಲ್ಲ; ಸಂಧ್ಯಾವಂದನಾದಿ ಸತ್ಕರ್ಮಗಳನ್ನು ಆಚರಿಸಲಿಲ್ಲ; ಸಾಧು-ಸಜ್ಜನರನ್ನು ಸತ್ಕರಿಸಲಿಲ್ಲ, ಮಹಾತ್ಮರು ಇದ್ದ ಸ್ಥಳಕ್ಕೆ ಹೋಗುತ್ತಿರಲಿಲ್ಲ; ದೇವತೆಗಳನ್ನು ಪೂಜಿಸಲಿಲ್ಲ; ಯಜ್ಞಯಾಗಾದಿಗಳಿಂದ ಪುಣ್ಯವನ್ನು ಸಂಪಾದಿಸಲಿಲ್ಲ; ಸಂಸಾರ ಬಂಧನದಿಂದ ಮುಕ್ತನಾಗುವ ಸಾಧನೆಯನ್ನು ವಿವರಿಸುವ ವಿದ್ಯೆಯನ್ನು ಕಲಿಯಲಿಲ್ಲ.

\begin{verse}
\textbf{ತೇನ ಚಾದೈವತೋ ನಾಮ ಜಾತೋಽಹಂ ಬ್ರಹ್ಮರಾಕ್ಷಸಃ~।। ೫೬~।।}
\end{verse}

ಇಂತಹ ದುಷ್ಕರ್ಮಿಗಳಿಂದ ನಾನು `ಅದೈವತ' ವೆಂಬ ಬ್ರಹ್ಮರಾಕ್ಷಸನಾಗಿ ಹುಟ್ಟಿದೆ.

\begin{verse}
\textbf{ಭಿನ್ನ ದೇಶಂ ಮೃತಾನಾಂ ಚ ಸಂಗತಿಂ ನೋ ವದಾಮ್ಯಹಮ್~।}\\\textbf{ಪುರಾ ಸಪ್ತಾಪಿ ಚ ವಯಂ ನಿಷಾದಾಧಿಪತೇಃ ಸುತಾಃ~।। ೫೭~।।}
\end{verse}

ಬೇರೆ ಬೇರೆ ಪ್ರದೇಶಗಳಲ್ಲಿ ಮೃತರಾದ ನಮ್ಮ ಏಳು ಜನರ ವೃತ್ತಾಂತವನ್ನು ಹೇಳುತ್ತೇನೆ. ಪೂರ್ವದಲ್ಲಿ ನಾವು ಏಳು ಜನರೂ ನಿಷಾದರಾಜನ ಮಕ್ಕಳಾಗಿದ್ದೆವು.

\begin{verse}
\textbf{ಸುಬಾಹುರ್ವೀರಬಾಹುಶ್ಚ ದೀರ್ಘಬಾಹುರ್ವಿಲೋಚನಃ~।}\\\textbf{ದೀರ್ಘಕಾಯೋ ದೀರ್ಘಜಂಘೋ ದಿರ್ಘೋಷ್ಠ ಇತಿ ಸಪ್ತ ತೇ~।। ೫೮~।।}
\end{verse}

ನಮ್ಮ ಏಳು ಜನರ ಹೆಸರುಗಳು ಹೀಗಿದ್ದುವು: ಸುಬಾಹು, ವೀರಬಾಹು, ದೀರ್ಘಬಾಹು, ವಿಲೋಚನ, ದೀರ್ಘಕಾಯ, ದೀರ್ಘಜಂಘ ಮತ್ತು ದೀರ್ಘೋಷ್ಠ.

\begin{verse}
\textbf{ವನೇಚರಾಶ್ಚೌರ್ಯಕರ್ಮನಿರತಾಃ ಪಾಪಚಾರಿಣಃ~।। ೫೯~।।}\\\textbf{ಅಸ್ಮಿನ್ವನೇ ಚರಂತಶ್ಚ ಧನುರ್ಬಾಣಧರಾಂತಕಾಃ~। }\\\textbf{ಏತದಶ್ವತ್ಥಮೂಲೇ ತು ಕದಾಚಿತ್ ವೈಶ್ಯಪುತ್ರಕಾಃ~।। ೬೦~।।} \\\textbf{ಮಾರಿತ್ಮಾ ಧನಿನೊಸ್ಮಾಭಿರ್ಯುಗಪತ್ಸಪ್ತ ತೇ ವನೇ~।}
\end{verse}

ಪಾಪಿಗಳಾದ ನಾವು ಕಾಡಿನಲ್ಲಿ ಕಳ್ಳತನ, ದರೋಡೆಗಳಲ್ಲಿ ನಿರತರಾಗಿದ್ದೆವು. ಧನುರ್ಬಾಣಗಳನ್ನು ಹಿಡಿದು ಸಂಚರಿಸುತ್ತಾ ಒಂದು ಸಲ ಈ ಅಶ್ವತ್ಥ ಮರದ ಬಳಿ ಬಂದೆವು. ಇಲ್ಲಿ ಏಳು ಜನ ವೈಶ್ಯರು ಇದ್ದರು. ಅವರಲ್ಲಿದ್ದ ಹಣವನ್ನು ಅಪಹರಿಸಲು ನಾವು ಆ ಏಳು ಜನರನ್ನೂ ಒಂದೇ ಬಾರಿಗೆ ಒಟ್ಟಿಗೆ ಸಂಹರಿಸಿದೆವು.

\begin{verse}
\textbf{ತತಃ ಕಿಯತಕಾಲೇನ ಸ್ವಾಮಿಕ್ಷೇತ್ರಮನುತ್ತಮಮ್~।। ೬೧~।।}\\\textbf{ಪ್ರಯಾತಂ ಮಾತುಲಾಹೂತೈರಸ್ಮಾಭಿಃ ಶುಭಕರ್ಮಣಿ~। }\\\textbf{ಕಾರ್ತಿಕ್ಯಾಂ ಕೃತ್ತಿಕಾಯೋಗಸ್ತದಾಸೀತ್ಪುಣ್ಯವರ್ಧನಃ~।। ೬೨~।।}
\end{verse}

ನಮ್ಮ ಸೋದರಮಾವನಿಂದ ಒಂದು ಶುಭಕಾರ್ಯಕ್ಕೆ ಆಹ್ವಾನಿಸಲ್ಪಟ್ಟ ನಾವು ಅತ್ಯುತ್ತಮವಾದ ಸ್ವಾಮಿ ಕ್ಷೇತ್ರವೆಂಬ ಪುಣ್ಯಕ್ಷೇತ್ರಕ್ಕೆ ಹೋದೆವು. ಆಗ ಪುಣ್ಯವನ್ನು ಅಭಿವೃದ್ಧಿ ಮಾಡುವ ಕಾತಿ೯ಕಮಾಸದ ಕೃತ್ತಿಕಾನಕ್ಷತ್ರದ ದಿವಸವಾಗಿತ್ತು.

\begin{verse}
\textbf{ತತ್ರಾಸ್ಮಾಭಿಃ ಕೃತಂ ಸ್ವಾಮೀದರ್ಶನಂ ಮಾತುಲಾಜ್ಞಯಾ~।}\\\textbf{ತೇನಾಸ್ಮಾಭಿಸ್ತು ಪುಣ್ಯೇನ ಪ್ರಾಪ್ತಾ ವಿಪ್ರಕುಲೇ ಜನಿಃ~।। ೬೩~।।}
\end{verse}

ನನ್ನ ಸೋದರಮಾವನ ಅಪ್ಪಣೆಯಂತೆ ಆ ಸ್ವಾಮೀಕ್ಷೇತ್ರದಲ್ಲಿ ದೇವರ ದರ್ಶನವನ್ನು ಮಾಡಿದೆವು. ಆ ಪುಣ್ಯ ಪ್ರಭಾವದಿಂದ ನಾವು ಮುಂದಿನ ಜನ್ಮದಲ್ಲಿ ಬ್ರಾಹ್ಮಣರಾಗಿ ಜನಿಸಿದೆವು.

\begin{verse}
\textbf{ಪೂರ್ವಕರ್ಮಾನುರೋಧೇನ ನಾಸ್ಮಾಭಿಃ ಸುಕೃತಂ ಕೃತಮ್~।}\\\textbf{ತದ್ದೇಶೇ ತಾದೃಶೇ ಕಾಲೇ ತಸ್ಮಿನ್ವೈ ಪಿಪ್ಪಲೇ ದ್ರುಮೇ~।। ೬೪~।। }
\end{verse}

\begin{verse}
\textbf{ತೀರ್ಥಯಾತ್ರಾಪ್ರಸಂಗೇನ ಮೀಲಿತಾಶ್ಚ ಸ್ವಕರ್ಮಣಾ~।}\\\textbf{ವಯಂ ತು ಮೃತಿಮಾಪನ್ನಾ ವಸಾಮೋಽತ್ರ ತತೋಽಧುನಾ~।। ೬೫~।।}
\end{verse}

ಹಿಂದಿನ ಜನ್ಮಗಳ ಕರ್ಮ ನಿಮಿತ್ತದಿಂದ ನಮ್ಮಿಂದ ಯಾವ ವಿಧವಾದ ಪುಣ್ಯಕರ್ಮವೂ ಆಚರಿಸಲ್ಪಡಲಿಲ್ಲ. ಆ ಪ್ರದೇಶದಲ್ಲಿ ಆ ಕಾಲದಲ್ಲಿ ನಮ್ಮ ಕರ್ಮಾನುಸಾರವಾಗಿ ಈ ಅಶ್ವತ್ಥಮರದ ಬಳಿ ತೀರ್ಥಯಾತ್ರಾ ನೆಪದಿಂದ ಒಂದೆಡೆ ಸೇರಿದ ಸ್ಥಳದಲ್ಲಿ ಮೃತಿಯನ್ನು ಹೊಂದಿ ಈವರೆವಿಗೂ ಇಲ್ಲಿ ವಾಸಿಸುತ್ತಿದ್ದೇವೆ.

\begin{verse}
\textbf{ಪೂರ್ವಕರ್ಮಾನುರೋಧೇನ ಸುಹೃದಶ್ಚ ಪರಸ್ಪರಮ್~।}\\\textbf{ಸ್ವಾಮಿದರ್ಶನಪುಣ್ಯೇನ ಫಲಿತಂ ತವ ದರ್ಶನಮ್~।। ೬೬~।।}
\end{verse}

ಹಿಂದಿನ ಕರ್ಮಾನುಸಾರವಾಗಿ ನಾವೆಲ್ಲ ಪರಸ್ಪರವಾಗಿ ಸ್ನೇಹಭಾವದಿಂದ ಇದ್ದೇವೆ. ಸ್ವಾಮಿಕ್ಷೇತ್ರದ ದರ್ಶನಭಾಗ್ಯದಿಂದ ಇಂದು ನನಗೆ ನಿನ್ನ ದರ್ಶನವು ಲಭಿಸಿತು.

\begin{verse}
\textbf{ಜಾತಿಸ್ಮರತ್ವಂ ಚೇದೀನಾಂ ಜಾಯತೇ ಪೂರ್ವಜನ್ಮನಃ~।}\\\textbf{ಅದ್ಯ ನೋ ಭಗವಾಂಸ್ತುಷ್ಟೋ ಗತಿರದ್ಯ ಭವಿಷ್ಯತಿ~।। ೬೭~।।}
\end{verse}

ಪೂರ್ವಜನ್ಮದ ಸ್ಮರಣೆಯೂ ಬಂದಿತು. ಭಗವಂತನು ಅನುಗ್ರಹ ಮಾಡಿದ್ದಾನೆ, ಸಂತುಷ್ಟನಾಗಿದ್ದಾನೆ. ಈಗ ನನಗೆ ಸದ್ಗತಿಯು ಲಭಿಸುವುದು.

\begin{verse}
\textbf{ನ ಕ್ವಾಪಿ ಚ ವೃಥಾ ಯಾತಿ ಮಹತಾಂ ದರ್ಶನಂ ಶುಭಮ್~।}\\\textbf{ಇತ್ಯೇತತ್ಸರ್ವಮಾಖ್ಯಾತಂ ಚೋದಿತಂ ಪಾಪಕಾರಣಮ್~।। ೬೮~।।} 
\end{verse}

\begin{verse}
\textbf{ಅಸ್ಮಾಕಂ ಮುನಿಶಾರ್ದೂಲ ರಕ್ಷ ತ್ವಂ ಕರುಣಾಬಲಾತ್~।}\\\textbf{ಇತ್ಯುಕ್ತ್ವಾತೇ ಪ್ರಣೇಮುಸ್ತಂ ಸರ್ವೇ ಪ್ರಾಂಜಲರ್ಯೋಽಭವನ್~।। ೬೯~।।}
\end{verse}

ಮಹಾತ್ಮರ ದರ್ಶನವು ಎಂದಿಗೂ ನಿಷ್ಪಲವಾಗುವುದಿಲ್ಲ. ನಮ್ಮೆಲ್ಲರ ವಿಚಾರವನ್ನೂ ಪಾಪಕರ್ಮಗಳ ವಿವರಣೆಯನ್ನೂ ತಿಳಿಸಿರುತ್ತೇವೆ. ಮುನಿಶ್ರೇಷ್ಠನೆ, ಕರುಣೆಯಿಂದ ನಮ್ಮನ್ನು ರಕ್ಷಿಸು. ಹೀಗೆ ನುಡಿದು ಅವರೆಲ್ಲರೂ ನಮಸ್ಕರಿಸಿ ಕೈ ಮುಗಿದರು.

\begin{center}
ಇತಿ ಶ‍್ರೀ ವಾಯುಪುರಾಣೇ ಮಾಘಮಾಸಮಾಹಾತ್ಮ್ಯೇ ಏಕಾದಶೋಧ್ಯಾಯಃ 
\end{center}

\begin{center}
ಶ‍್ರೀ ವಾಯುಪುರಾಣಾಂತರ್ಗತ ಮಾಘಮಾಸ ಮಹಾತ್ಮ್ಯೆಯಲ್ಲಿ \\ ಹನ್ನೊಂದನೇ ಅಧ್ಯಾಯವು ಸಮಾಪ್ತಿಯಾಯಿತು.
\end{center}

\newpage

\section*{ಅಧ್ಯಾಯ\enginline{-}೧೨}

\emptypage

\begin{flushleft}
\textbf{ಕೂಷ್ಮಾಂಡಾ ಊಚುಃ\enginline{-}}
\end{flushleft}

\begin{verse}
\textbf{ಶೃಣು ನಾಮಾನಿ ಸಾರ್ಥಾನಿ ನ ಕರ್ಮಾಣಿ ಶುಭಾನಿ ಚ~।}\\\textbf{ಕಾರಣಾನಿ ಚ ದುರ್ಯೊನೇರ್ವದತಾಂ ಪಾಪಕಾರಿಣಾಮ್~।। ೧~।। }
\end{verse}

\begin{flushleft}
ಕೂಷ್ಮಾಂಡವೆಂಬ ಪಿಶಾಚಿಗಳು ನುಡಿದವು-
\end{flushleft}

ನಮ್ಮೆಲ್ಲರ ಹೆಸರುಗಳನ್ನೂ, ಅವುಗಳ ಅರ್ಥಗಳನ್ನೂ ಈ ಜನ್ಮದಲ್ಲಿ ಉತ್ಪನ್ನರಾಗಲು ನಾವು ಆಚರಿಸಿದ ಪಾಪಕಾರ್ಯಗಳ ವಿವರಗಳನ್ನೂ ಹೇಳುತ್ತೇವೆ, ಲಾಲಿಸು.

\begin{verse}
\textbf{ಉಚ್ಛಿಷ್ಟಭೋಜನಶ್ಚಾದ್ಯೋ ದ್ವಿತೀಯೋsಭದ್ರಸಂಜ್ಞ ಕಃ~।}\\\textbf{ತೃತೀಯೋ ನೀರಸೋ ನಾಮ ಚತುರ್ಥೋ ಹೀನನಾಮಕಃ~।। ೨~।। }
\end{verse}

\begin{verse}
\textbf{ಪಂಚಮೋ ನಿಷ್ಠುರೋ ನಾಮ ಷಷ್ಠಶ್ಚಾಪಿ ನಿರಾಶ್ರಯಃ~।}\\\textbf{ಶಾಸ್ತ್ರೋದ್ವರ್ತಃ ಸಪ್ತಮಸ್ತು ಕರ್ಮಾಣ್ಯಥ ನಿಶಾಮಯ~।। ೩~।।}
\end{verse}

ಮೊದಲನೆಯವನ ಹೆಸರು ಉಚ್ಛಿಷ್ಟಭೋಜನ, ಎರಡನೆಯವನು ಅಭದ್ರ, ಮೂರನೆಯವನು ನೀರಸ, ನಾಲ್ಕನೆಯವನು ಹೀನ, ಐದನೆಯವನ ಹೆಸರು ನಿಷ್ಠುರ, ಆರನೆಯವನು ನಿರಾಶ್ರಯ, ಏಳನೆಯವನು ಶಾಸ್ತ್ರೋದ್ವರ್ತ, ನಾವು ಮಾಡಿದ ಕರ್ಮವಿವರಗಳನ್ನು ಕೇಳು,

\noindent
\textbf{ಉಚ್ಛಿಷ್ಟಭೋಜನ ಉವಾಚ\enginline{-}}

\begin{verse}
\textbf{ಬ್ರಾಹ್ಮಣೋsಹಂ ಪುರಾ ಚಾಸೀತ್ ಕ್ಷೇತ್ರೇ ಛಾಯಾಗುಹಾಹ್ವಯೇ~।}\\\textbf{ವೀತಿಹೋತಸ್ಯ ತನಯಃ ಕಲಿಂಗ ಇತಿ ನಾಮಕಃ~।। ೪~।। }
\end{verse}

\begin{flushleft}
ಉಚ್ಛಿಷ್ಟಭೋಜನನೆಂಬ ಪಿಶಾಚಿಯು ನುಡಿಯಿತು-
\end{flushleft}

ನಾನು ಹಿಂದಿನ ಜನ್ಮದಲ್ಲಿ ಛಾಯಾಗುಹವೆಂಬ ಊರಿನಲ್ಲಿ ವೀತಿಹೋತನೆಂಬ ಬ್ರಾಹ್ಮಣನ ಮಗನಾಗಿದ್ದೆ. ನನ್ನ ಹೆಸರು ಕಲಿಂಗ.

\begin{verse}
\textbf{ಚಾಮುಂಡೀದೇವತಾಭಕ್ತೋ ವೇದಕರ್ಮಬಹಿಷ್ಕೃತಃ~।}\\\textbf{ಪಿಶಾಚಯಕ್ಷನಾಗಾನಾಂ ಗರೀಯಾನ್ ಪೂಜಕೇಷು ಚ~।। ೫~।। }
\end{verse}

\begin{verse}
\textbf{ಪಿಶಾಚಯಕ್ಷನಾಗಾನಾಂ ನೈವೇದ್ಯಂ ಭುಕ್ತಮನ್ವಹಮ್~।}\\\textbf{ಹರೇರ್ನಿಷೇವಿತಂ ಕ್ವಾಪಿ ಭಕ್ಷಿತಂ ನ ಮಯಾ ದ್ವಿಜ~।। ೬~।।}
\end{verse}

ಚಾಮುಂಡೀ ಎಂಬ ದೇವತೆಗೆ ಭಕ್ತನಾಗಿದ್ದೆ. ವೇದೋಕ್ತವಾದ ಸತ್ಕರ್ಮಗಳನ್ನು ಆಚರಿಸುತ್ತಿರಲಿಲ್ಲ. ಪಿಶಾಚಿಗಳು, ಯಕ್ಷರು, ನಾಗ ಇವುಗಳ ಅರ್ಚಕರಲ್ಲಿ ಶ್ರೇಷ್ಠನಾಗಿದ್ದೆ. ಪಿಶಾಚಿ, ಯಕ್ಷ, ನಾಗ-ಇವುರುಗಳಿಗೆ ಅವರ ಭಕ್ತರು ನೈವೇದ್ಯಕ್ಕಾಗಿ ಸಮರ್ಪಿಸಿದ ಪದಾರ್ಥಗಳನ್ನು ನೈವೇದ್ಯ ಮಾಡಿ ಅದನ್ನೆಲ್ಲ ತಿನ್ನುತ್ತಿದ್ದೆನೇ ವಿನಹ ಶ‍್ರೀಹರಿಯನ್ನು ಯಾವ ರೀತಿಯಲ್ಲಿಯೂ ಸೇವಿಸಲಿಲ್ಲ. ಶ‍್ರೀಹರಿಯ ನೈವೇದ್ಯವನ್ನು ಊಟಮಾಡುತ್ತಿರಲಿಲ್ಲ.

\begin{verse}
\textbf{ಪಿಶಾಚಯಕ್ಷನಾಗಾನಾಂ ಹೀನಜಾನಾಂ ಮುನೀಶ್ವರ~।}\\\textbf{ನಿಂದಿತಾನಾಂ ಚ ದಾಸೋಽಹಂ ತತ ಉಚ್ಛಿಷ್ಟಭೋಜನಃ~।। ೭~।।}
\end{verse}

ಹೀನ ಮತ್ತು ನೀಚ ಜಾತಿಯ ಜನರು ದೇವತೆಗಳೆಂದು ಅರ್ಚಿಸುವ ಪಿಶಾಚ, ಯಕ್ಷ ಮತ್ತು ನಾಗ ಈ ದೇವತೆಗಳ ಉಪಾಸಕನಾಗಿ ಅವರ ನೈವೇದ್ಯವನ್ನು ಭಕ್ಷಿಸುತ್ತಿದ್ದೆಯಾದಕಾರಣ ನನಗೆ ಉಚ್ಛಿಷ್ಟಭೋಜನನೆಂಬ ಹೆಸರು ಬಂದಿತು.

\begin{verse}
\textbf{ತೇನ ಕರ್ಮವಿಪಾಕೇನ ಜಾತೋsಸ್ಮಿ ವಟಪಾದಪೇ~।}\\\textbf{ಕೂಷ್ಮಾಂಡಾಖ್ಯಗಣೇ ವಿಪ್ರ ಶಿರೋಹೀನೇ ಭುಜಾಂಸಕೇ~।। ೮~।।}
\end{verse}

ಅಂತಹ ಪಾಪಕರ್ಮದ ದೆಸೆಯಿಂದ ಈ ವಟವೃಕ್ಷದಲ್ಲಿರುತ್ತೇನೆ. ನನಗೆ ಭುಜ ಮತ್ತು ಕಂಠದಲ್ಲಿ ತಲೆಯೇ ಇಲ್ಲ. ಕೂಷ್ಮಾಂಡವೆಂಬ ಪಿಶಾಚಗಣಕ್ಕೆ ಸೇರಿದ್ದೇನೆ.

\begin{verse}
\textbf{ನಿಶಮ್ಯೋಚ್ಛಿಷ್ಟವಾಕ್ಯಂ ತು ಹ್ಯಭದ್ರೋ ವಾಕ್ಯ ಮಬ್ರವೀತ್~। }
\end{verse}

ಉಚ್ಛಿಷ್ಟಭೋಜನನ ಮಾತುಗಳನ್ನು ಕೇಳಿ ಅಭದ್ರನೆಂಬ ಪಿಶಾಚಿಯು ನುಡಿದನು:-

\begin{verse}
\textbf{ಪುರಾ ಭೀಮರಥೀ ತೀರೇ ಕ್ಷೇತ್ರೇ ನೀಲಾಹ್ವಯೇ ಶುಭೇ~।। ೯~।।} 
\end{verse}

\begin{verse}
\textbf{ಜೈಮಿನೇರ್ಗೋತ್ರಜಶ್ಚಾಹಂ ನಾಮ್ನಾ ಜೈತ್ರೋ ಮಹಾತ್ಮನಃ~।}\\\textbf{ನಿಃಶಂಕಾಖ್ಯಸ್ಯ ತನಯೋ ವೇದತತ್ವಾರ್ಥಕೋವಿದಃ~।। ೧೦~।।}
\end{verse}

ಹಿಂದೆ ಭೀಮರಥೀ ನದಿಯ ದಡದಲ್ಲಿ ನೀಲ ಎಂಬ ಶುಭಪ್ರದವಾದ ಊರಿತ್ತು. ಅಲ್ಲಿ ಜೈಮಿನಿ\-ಗೋತ್ರದಲ್ಲಿ ಜೈತ್ರನೆಂಬ ಹೆಸರಿನಿಂದ ಹುಟ್ಟಿದ್ದೆ. ವೇದ, ವೇದಗಳ ಅರ್ಥವಿವರಣೆಯಲ್ಲಿ ನಿಪುಣನಾಗಿದ್ದೆ. ನನ್ನ ತಂದೆಯವರು ನಿಃಶಂಕರೆಂಬ ಬ್ರಾಹ್ಮಣರು.

\begin{verse}
\textbf{ಪಾಖಂಡಿನಾಂ ಚ ಸಂಸರ್ಗಾತ್ ಪಠಿತಾಸ್ತೇ ದುರಾಗಮಾಃ~।}\\\textbf{ವ್ಯಾಖ್ಯಾತಾಶ್ಚ ಮಯಾ ತೇನ ಬಹುಶೋಽಥ ನಿರಂಕುಶಾಃ~।। ೧೧~।।}
\end{verse}

ಪಾಖಂಡಮತದ ಜನರ ಸಹವಾಸದೋಷದಿಂದ ಅವರ ದುಷ್ಟಗ್ರಂಥಗಳನ್ನು ಅಭ್ಯಾಸಮಾಡಿ, ಅದರಲ್ಲಿ ಪಾಂಡಿತ್ಯಪಡೆದು ಆ ಗ್ರಂಥಗಳನ್ನು ಅನೇಕ ಬಾರಿ ಭಯರಹಿತನಾಗಿ ಬಹುಪರಿಯಾಗಿ ವ್ಯಾಖ್ಯಾನ ಮಾಡುತ್ತಿದ್ದೆ.

\begin{verse}
\textbf{ಕರೋಮಿ ಭೈರವಂ ಚಕ್ರಂ ಕಾರಯಾಮಿ ಪರೈರಪಿ~।}\\\textbf{ಅಸ್ಮಿನ್ ಪ್ರವೃತ್ತೇ ವಿಪ್ರೇಂದ್ರ ಸಾಜಾತ್ಯಂ ಪ್ರತಿಪದ್ಯತೇ~।। ೧೨~।।}
\end{verse}

ಭೈರವ ಯಂತ್ರವನ್ನು ತಯಾರುಮಾಡುತ್ತಿದ್ದೆ, ಇತರರಿಂದಲೂ ಮಾಡಿಸುತ್ತಿದ್ದೆ. ಈ ಭೈರವ ವಿದ್ಯಾ ಕಲಿತರೆ "ಭೈರವ'ವೆಂಬ ಗಣದಲ್ಲಿಯೇ ಸೇರಿಬಿಡುತ್ತಾನೆ.

\begin{verse}
\textbf{ಅಷ್ಟಾದಶಾನಾಂ ವರ್ಷಾನಾಂ ಯಾವತ್‌ ವ್ರತಸಮಾಪನಮ್~।}\\\textbf{ಯೋಷಿದ್ವಿನಿಮಯೋ ಯತ್ರ ನ ಫಲಂ ಪಾಪಕರ್ಮಣಃ~।। ೧೩~।।}
\end{verse}

ಹದಿನೆಂಟು ವರ್ಷಗಳ ಕಾಲ "ಭೈರವ" ವ್ರತ ಮಾಡಿ ಸಮಾಪ್ತಿ ಮಾಡಿದ ನಂತರ ಎಂತಹ ನೀಚ ಕೃತ್ಯ ಮಾಡಿದರೂ ಪಾಪವೇ ಇಲ್ಲವೆಂಬ ನಂಬಿಕೆ ಬರುತ್ತದೆ. ಸ್ತ್ರೀಯರ ವಿಷಯದಲ್ಲಿಯಂತೂ ಯಾವ ನಿಯಮವೂ ಇರುವುದಿಲ್ಲ.

\begin{verse}
\textbf{ಚಾಂಡಾಲೋ ಬ್ರಾಹ್ಮಣೀಂ ಗಚ್ಛೇತ್ ಶ್ವಪಚೀಮಪಿ ವೈ ದ್ವಿಜಃ~।}\\\textbf{ಸ್ತ್ರೀತ್ವಂ ಪುಂಸ್ತ್ವಂ ದ್ವಯೀ ಜಾತಿರಿತರಾ ಭ್ರಾಂತಿಕಲ್ಪಿತಾಃ~।। ೧೪~।।}
\end{verse}

ಚಾಂಡಾಲನು ಬ್ರಾಹ್ಮಣಸ್ತ್ರೀಯನ್ನೂ, ಬ್ರಾಹ್ಮಣನು ಚಾಂಡಾಲಸ್ತ್ರೀಯನ್ನೂ ಭೋಗಿಸಬೇಕು. “ಸ್ತ್ರೀ” ಮತ್ತು “ಪುರುಷ” ಎರಡೇ ಜಾತಿ, ಉಳಿದ ಜಾತಿಗಳು ಭ್ರಾಂತಿಕಲ್ಪಿತ.

\begin{verse}
\textbf{ಇಮೇ ಭೋಗಾಶ್ಚ ಬುದ್ಧೇ ಸ್ಯುಃ ಭ್ರಾಂತ್ಯಾ ವಾಽತ್ಮನಿ ವೈ ಮತಿಃ~।}\\\textbf{ಕರ್ಮಣಾ ಜಾಯತೇ ಜಂತುರ್ನ ಪರೋಽಸ್ತಿ ಹಿ ಕರ್ಮಣಃ~।। ೧೫~।।}
\end{verse}

ಈ ಜನ್ಮದಲ್ಲಿ ಬರುವ ಸುಖ-ದುಃಖಾದಿಗಳು ಆತ್ಮನಿಗಲ್ಲ. ಆದರೆ ಆತ್ಮನಿಗೆ ಎಂದು ಬುದ್ಧಿಗೆ ತೋರುತ್ತದೆ. ಜಗದೀಶ್ವರನೆಂಬ ವಸ್ತುವೇ ಇಲ್ಲ. ಕರ್ಮಗಳಿಂದ ತಾವಾಗಿಯೇ ಪುನರ್ಜನ್ಮ ಪಡೆಯುತ್ತಾನೆ. ಕರ್ಮವೇ ಅತ್ಯಂತ ಶ್ರೇಷ್ಠ, ಕರ್ಮಕ್ಕಿಂತ ಉತ್ತಮವಾದುದು ಯಾವುದೂ ಇಲ್ಲ.

\begin{verse}
\textbf{ಅಪ್ರಧಾನಂ ತು ಪ್ರಾಧಾನ್ಯಂ ಕರ್ತಾ ಕರ್ಮ ನಿಗದ್ಯತೇ~।}\\\textbf{ಚರಾಚರಂ ದೃಶ್ಯಮಾನಂ ಜಗದೇತನ್ನಿರೀಶ್ವರಮ್~।। ೧೬~।।}
\end{verse}

ಸರ್ವಕರ್ತಾ ಎಂಬ ಮಾತೇ ಇಲ್ಲ. ಸಕಲ ಕಾರ್ಯಗಳೂ ಕರ್ಮವಶದಿಂದ ನಡೆಯುತ್ತವೆ. ಅಪ್ರಧಾನವಾದುದೇ ಕರ್ತಾ, ಪ್ರಧಾನವಾದುದೇ ಕರ್ಮ. ನಮ್ಮ ಕಣ್ಣಿಗೆ ಕಾಣಿಸುವ ಚರಾಚರಾತ್ಮಕವಾದ ಜಗತ್ತು ಈಶ್ವರನಿಂದ ಸೃಷ್ಟಿಸಲ್ಪಟ್ಟಿಲ್ಲ. ಜಗತ್ತಿಗೆ ಈಶ್ವರನೇ ಇಲ್ಲ.

\begin{verse}
\textbf{ವೈರಾಗ್ಯಾತ್ಪರಮಾತ್ಪಶ್ಚಾತ್ಸಂವಿದ್ಧಿ ಚ್ಛೇದಸಂಜ್ಞ ಕಃ~।}\\\textbf{ಆತ್ಮನೋ ಜಾಯತೇ ಮೋಕ್ಷೋ ವಿಕಲ್ಪೋ ಯತ್ರ ನಶ್ಯತಿ~।। ೧೭~।।}
\end{verse}

ಆತ್ಮನಿಗೆ ವೈರಾಗ್ಯ ಹುಟ್ಟಲು ವಿಚ್ಛತ್ತಿರಹಿತವಾದ ಮೋಕ್ಷ ದೊರೆಯುತ್ತದೆ. (ನಡುವೆ ಅಡ್ಡಿ ಬಾರದ \enginline{uninterrupted} ಮೋಕ್ಷ) ಈ ರೀತಿ ಮೋಕ್ಷ ದೊರೆಯಲು ಕಾಣಿಸುತ್ತಿದ್ದ ಎಲ್ಲ ವಸ್ತುಗಳೂ ನಾಶಹೊಂದುತ್ತವೆ.

\begin{verse}
\textbf{ಇತ್ಯಾದಿತಥ್ಯಮಿತ್ಯೇವ ಜ್ಞಾತಂ ವೈ ಬೋಧಿತಂ ಮಯಾ~।}\\\textbf{ಕುತರ್ಕೈಃ ವೇದವಿದ್ಯಾಶ್ಚ ಮುಖಾದ್ಯಾ ಅಪಿ ದೂಷಿತಾಃ~।। ೧೮~।।}
\end{verse}

ಇಂತಹ ಪ್ರಕ್ರಿಯೆಗಳನ್ನು ಸತ್ಯವೆಂದು ನಂಬಿ ಇತರರಿಗೂ ಉಪದೇಶಮಾಡಿದೆ. ಕೆಟ್ಟ ತರ್ಕಗಳಿಂದ ವೇದವಿದ್ಯೆಗೆ ವಿರೋಧಾರ್ಥವನ್ನು ಹೇಳುತ್ತಾ ಯಜ್ಞ - ಯಾಗಾದಿ ಕರ್ಮಗಳನ್ನು ದೂಷಿಸಿದೆ.

\begin{verse}
\textbf{ಮದಾವಾಸೇನ ತತ್ ಕ್ಷೇತ್ರಂ ನೀಲ ಇತ್ಯಭಿಧೀಯತೇ~।}\\\textbf{ತತೋ ಮೃ ತಿಮವಾಪನ್ನೋ ನರಕಾನ್ ಪ್ರತಿಪದ್ಯ ವೈ~।। ೧೯~।। }
\end{verse}

\begin{verse}
\textbf{ತೇನಾಭದ್ರೇಣ ಜಾತೋಹಂ ಕೂಷ್ಮಾಂಡಾನಾಂ ಗಣೇಗ್ರಣೀಃ~।}\\\textbf{ಇತ್ಯಭದ್ರವಚಃ ಶ್ರುತ್ವಾ ವ್ಯಾಜಹಾರ ಚ ನೀರಸಃ~।। ೨೦~।।}
\end{verse}

ನಾನು ವಾಸಿಸುತ್ತಿದ್ದ ಸ್ಥಳಕ್ಕೆ ನೀಲ ಎಂಬ ಹೆಸರು. ನಂತರ ನಾನು ಮೃತಿಯನ್ನು ಹೊಂದಿ ಅನೇಕ ನರಕಗಳಲ್ಲಿ ನಾನಾ ದುಃಖಗಳನ್ನು ಅನುಭವಿಸಿ ಅಭದ್ರನೆಂಬ ಹೆಸರಿನಿಂದ ಬ್ರಹ್ಮರಾಕ್ಷಸನಾಗಿ ಉತ್ಪನ್ನನಾದೆ. ಕೂಷ್ಮಾಂಡವೆಂಬ ಗುಂಪಿಗೆ ಒಡೆಯನಾದೆ. ಹೀಗೆಂಬ ಅಭದ್ರನ ನುಡಿಗಳನ್ನು ಕೇಳಿದನಂತರ `ನೀರಸ'ನೆಂಬ ಪಿಶಾಚಿಯು ಮಾತನಾಡಿದನು

\begin{verse}
\textbf{ಹ್ರಿಯಾ ನವ್ರಮುಖೋ ದೀನಸ್ತಥಾ ಗದ್ಗದಭಾಷಣಃ~।}
\end{verse}

ನಾಚಿಕೆಯಿಂದ ತಲೆತಗ್ಗಿಸಿ ದೀನನಾಗಿ ಗದ್ಗದಕಂಠದಿಂದ ನುಡಿದನು.

\begin{verse}
\textbf{ಪುರಾ ಯವನದೇಶೇಷು ಮಾಲತೀನಾಮ ವೈ ಪುರೀ~।। ೨೧~।। }
\end{verse}

\begin{verse}
\textbf{ತಸ್ಯಾಂ ಸುದ್ಯುಮ್ನಯಿತ್ಯಾಸೀತ್ ಕಾಪೇಯೋ ಬ್ರಾಹ್ಮಣೋತ್ತಮಃ~।}\\\textbf{ತಸ್ಯಾಹಮಭವತ್ಪುತ್ರೋ ನಾಮ್ನಾ ತಿಗ್ಮದ್ಯುತಿಸ್ತಥಾ~।। ೨೨~।।}
\end{verse}

ಪೂರ್ವದಲ್ಲಿ ಯವನದೇಶದಲ್ಲಿ ಮಾಲತೀ ಎಂಬ ನಗರವಿತ್ತು. ಅಲ್ಲಿ ಕಾಪಿಗೋತ್ರದ\break ಸುದ್ಯುಮ್ನನೆಂಬ ಉತ್ತಮನಾದ ಬ್ರಾಹ್ಮಣನಿದ್ದನು. ನಾನು ಆತನ ಮಗ. ಹೆಸರು ತಿಗ್ಮದ್ಯುತಿ.

\begin{verse}
\textbf{ಗತೇ ತು ಪಿತರಿ ಸ್ವರ್ಗೇ ರಸವಿಕ್ರಯಿಕೋಽಭವಮ್~।}\\\textbf{ಅಜಾವಿಗೋಮಹಿಷ್ಯಶ್ಚ ಮಯಾ ಸಂಪಾದಿತಾ ದ್ವಿಜ~।। ೨೩~।।}
\end{verse}

ತಂದೆಯು ಸ್ವರ್ಗಸ್ಥರಾದ ಮೇಲೆ ಹಾಲು-ಮೊಸರು ವ್ಯಾಪಾರದಲ್ಲಿ ನಿರತನಾದೆ. ಆಡು, ಕುರಿ, ಹಸು, ಎಮ್ಮೆಗಳನ್ನು ಸಂಪಾದಿಸಿದೆ.

\begin{verse}
\textbf{ತಾಸಾಂ ದಧಿ ಘೃತಂ ಕ್ಷೀರಂ ವಿಕ್ರೀತಂ ಭೂರಿಶೋ ಮಯಾ~।}\\\textbf{ಘೃತೇನ ಪಯಸಾ ದಧ್ನಾ ಸ್ನಾಪಿತೋ ನೈವ ಕೇಶವಃ~।। ೨೪~।।}
\end{verse}

ಈ ಪ್ರಾಣಿಗಳ ಮೊಸರು, ತುಪ್ಪ, ಹಾಲುಗಳನ್ನು ಬೇಕಾದಷ್ಟು ಮಾರಿದೆ. ಆದರೆ ತುಪ್ಪದಿಂದ, ಹಾಲಿನಿಂದ, ಮೊಸರಿನಿಂದ ಕೇಶವನಿಗೆ ಎಂದಿಗೂ ಅಭಿಷೇಕ ಮಾಡಲಿಲ್ಲ.

\begin{verse}
\textbf{ನ ದೀಪೋ ಜ್ವಲಿತಃ ಕ್ವಾಪಿ ಹರಯೇ ಪರಮಾತ್ಮನೇ~।}\\\textbf{ನವನೀತಂ ದಧಿ ಕ್ಷೀರಂ ಶಿಶೂನಾಂ ನೋಪಪಾದಿತಮ್~।। ೨೫~।।}
\end{verse}

ಶ‍್ರೀ ಹರಿಗೋಸ್ಕರ ಒಂದು ದಿವಸವಾದರೂ ದೀಪ ಹಚ್ಚಲಿಲ್ಲ. ಮಕ್ಕಳಿಗಾಗಿ ಬೆಣ್ಣೆ, ಹಾಲು-ಮೊಸರುಗಳನ್ನು ಕೊಡಲೇ ಇಲ್ಲ.

\begin{verse}
\textbf{ಚತುರ್ವಿಂಶತಿಪ್ರಸ್ಥಂ ಹಿ ಘೃತಂ ಹಿ ಜಾಯತೇಽನ್ವಹಮ್~।}\\\textbf{ದಧ್ನಾಂ ಚ ಪಯಸಾಂ ಚೈವ ಪ್ರಮಾಣಂ ನೈವ ವಿದ್ಯತೇ~।। ೨೬~।।}
\end{verse}

ಒಂದು ದಿನಕ್ಕೆ ಇಪತ್ತನಾಲ್ಕು ಕೊಡ ತುಪ್ಪ ಶೇಖರಿಸಲ್ಪಡುತ್ತಿತ್ತು. ಮೊಸರು ಹಾಲಿನ ಅಳತೆಯು ತಿಳಿಯುತ್ತಲೇ ಇರಲಿಲ್ಲ.

\begin{verse}
\textbf{ಅಪಿ ಸರ್ಷಪಮಾತ್ರಂ ವಾ ಸಾಯಾಹ್ನೇ ನೈವ ವಿದ್ಯತೇ~।}\\\textbf{ವಿನಾ ಶಿಶೂನ್‌ ಗೃಹೇ ಕೈಶ್ಚಿತ್ ಕ್ವಾಪಿ ನಾಸ್ವಾದಿತೋ ರಸಃ~।। ೨೭~।।}
\end{verse}

ಸಾಯಂಕಾಲದ ಹೊತ್ತಿಗೆ ಒಂದು ಸಾಸಿವೆ ಕಾಳಿನಷ್ಟು ಉಳಿದಿರುತ್ತಿದ್ದಿಲ್ಲ. ಎಲ್ಲವೂ ಮಾರಾಟವಾಗಿಬಿಡುತ್ತಿತ್ತು. ಮನೆಯಲ್ಲಿ ಸಣ್ಣ ಮಕ್ಕಳನ್ನು ಹೊರತು ಮತ್ತೆ ಯಾರೂ ಈ ವಸ್ತುಗಳನ್ನು ಭುಂಜಿಸುತ್ತಿದ್ದಿಲ್ಲ.

\begin{verse}
\textbf{ತಥಾ ಧಾರಾಘೃತಾದೀನಾಂ ಶ್ರಾದ್ಧಂ ನ ಕ್ರಿಯತೇ ಮಯಾ~।}\\\textbf{ಅಜಾವಿಗೋಮಹಿಷ್ಯಾಣಾಂ ಕ್ರೀಯತೇ ಕ್ರಯವಿಕ್ರಯಃ~।। ೨೮~।। }
\end{verse}

\begin{verse}
\textbf{ಕ್ರಯವಿಕ್ರಯದ್ರವ್ಯಾಣಾಂ ಗಣನಾ ನೈವ ವಿದ್ಯತೇ~।}\\\textbf{ಏವಂ ಕಾಲೋ ಮಯಾ ನೀತೊ ನ ಧ್ಯಾತೋ ಹರಿರೀಶ್ವರಃ~।। ೨೯~।।}
\end{verse}

ನನ್ನ ತಂದೆ-ತಾಯಿಯರ ಶ್ರಾದ್ಧದಲ್ಲಿ ತುಪ್ಪ-ಹಾಲು ಮೊದಲಾದ ರುಚಿಕರವಾದ\break ಪದಾರ್ಥಗಳನ್ನೇ ಬ್ರಾಹ್ಮಣರಿಗೆ ಬಡಿಸುತ್ತಿದ್ದಿಲ್ಲ. ಆಡು, ಕುರಿ, ಹಸು, ಎಮ್ಮೆ ಇವುಗಳನ್ನು ಕೊಳ್ಳುವುದು, ಮಾರುವುದು ಇದೇ ನನ್ನ ದಿನಚರಿ. ಈ ಮಾರಾಟದಿಂದ ಬಂದ ಹಣಕ್ಕೆ ಲೆಕ್ಕವೇ ಇರಲಿಲ್ಲ. ಹೀಗೆ ಕಾಲವನ್ನು ಕಳೆದ ನಾನು ಒಂದು ದಿನವಾದರೂ ಆ ಪ್ರಭುವಾದ, ಪಾಪಪರಿಹಾರಕನಾದ ಶ‍್ರೀ ಹರಿಯನ್ನು ಧ್ಯಾನ ಮಾಡಲಿಲ್ಲ.

\begin{verse}
\textbf{ಕರ್ಮಣಾ ತೇನ ಜಾತೋಽಹಂ ನಾಮ್ನಾ ನೀರಸನಾಮಕಃ~।}\\\textbf{ತೂಷ್ಣೀಂ ಸ್ಥಿತೇ ನೀರಸೇ ತು ಹೀನೋ ವೃತ್ತಿಮಬೋಧಯತ್~।। ೩೦~।।}
\end{verse}

ಇಂತಹ ದುಷ್ಕರ್ಮದ ಫಲವಾಗಿ “ನೀರಸ'ನೆಂಬ ಪಿಶಾಚಿಯಾಗಿ ಹುಟ್ಟಿರುತ್ತೇನೆ. ಹೀಗೆಂದು ಹೇಳಿ ನೀರಸನು ಸುಮ್ಮನಾಗಲು “ಹೀನ” ನೆಂಬ ಪಿಶಾಚಿಯೂ ನುಡಿಯಲು ಉಪಕ್ರಮಿಸಿತು.

\begin{verse}
\textbf{ಪುರಾ ಪುಷ್ಕಲದೇಶೇಷು ಗ್ರಾಮೇ ಮುಷ್ಟ್ಯಭಿಧೇ ದ್ವಿಜಃ~।}\\\textbf{ಜಾತೋಽಹಂ ಚ ಪೃಥೋಃ ಪುತ್ರೋ ನಾಮ್ನಾ ವಿಶ್ರಾಮಸಂಜ್ಞಿತಃ~।। ೩೧~।।}
\end{verse}

ಹಿಂದೆ ನಾನು ಪುಷ್ಕಲವೆಂಬ ದೇಶದ ಮುಷ್ಟಿಯೆಂಬ ಊರಿನಲ್ಲಿ ಬ್ರಾಹ್ಮಣನಾಗಿ ಹುಟ್ಟಿದ್ದೆ. ನನ್ನ ತಂದೆಯ ಹೆಸರು ಪೃಥು, ನನ್ನ ಹೆಸರು ವಿಶ್ರಾಮ.

\begin{verse}
\textbf{ವಾತ್ಸಾಯನಕುಲೋತ್ಪನ್ನೋ ಗ್ರಾಮಣ್ಯ ಖಿಲಸಂಶ್ರಯಃ~।}\\\textbf{ಪೌರೋಹಿತ್ಯೇ ನಿಯುಕ್ತಶ್ಚ ತಥೈವಾಯಾಜ್ಯ ಯಾಜಕಃ~।। ೩೨~।।}
\end{verse}

ನಾನು ವಾತ್ಸಾಯನ ಗೋತ್ರದವನಾಗಿ ಗ್ರಾಮದ ಜನರ ಆಶ್ರಯ ಪಡೆದು ಗ್ರಾಮದ ಪುರೋಹಿತನಾಗಿ ನೇಮಿಸಲ್ಪಟ್ಟೆ; ಯಜ್ಞ ಯಾಗಾದಿಗಳನ್ನು ಮಾಡಿಸುತ್ತಿದ್ದೆ.

\begin{verse}
\textbf{ವಿಪ್ರೇಷು ಕ್ಷತ್ರೇಷು ತಥೈವ ವೈಶ್ಯ\enginline{-}}\\\textbf{ಶೂದ್ರೇಷು ಗೋಪೇಷು ತಥಾವಿಕೇಷು~।}\\\textbf{ಕುಲಾಲಹೈಮೇಷು ಚ ಚರ್ಮಕೇಷು ಚ} \\\textbf{ಪರೋಪಜೀವಿಷ್ವಥ ಪುಲ್ಕ ಸೇಷು~।। ೩೩~।।}
\end{verse}

\begin{verse}
\textbf{ರಜೇಷು ವ್ರಾತ್ಯೇಷ್ವಥ ನಾಪಿತೇಷು}\\\textbf{ತಥೈವ ಮತ್ಸ್ಯಾದಿಷು ವೃತ್ತಿರಾಸೀತ್~।}\\\textbf{ಮೌಂಜೀವಿವಾಹಾದಿಷು ಯಚ್ಚ ಕರ್ಮ}\\\textbf{ಯದೌರ್ಧ್ವದೇಹಾದಿಕಕರ್ಮಜಾತಮ್~।। ೩೪~।।}
\end{verse}

ಬ್ರಾಹ್ಮಣರು ಕ್ಷತ್ರಿಯರು, ವೈಶ್ಯರು, ಶೂದ್ರರು, ಗೋಪಾಲಕರು, ಕುರುಬರು, ಕುಂಬಾರರು, ಅಕ್ಕಸಾಲಿಗರು, ಚರ್ಮದ ಕೆಲಸ ಮಾಡುವವರು, ಪರಸೇವೆಯಿಂದ ಜೀವನ ಮಾಡುವವರು, ಅಗಸರು, ವ್ರಾತ್ಯರು, ಹಜಾಮರು, ಮೀನು ಹಿಡಿಯುವವರು, ಈ ಜಾತಿಗಳ ಜನಗಳಲ್ಲಿ ಮುಂಜಿ, ವಿವಾಹ, ಉತ್ತರಕ್ರಿಯಾದಿ ಕರ್ಮಗಳನ್ನು ಮಾಡಿಸುತ್ತಿದ್ದೆ.

\begin{verse}
\textbf{ತಥೈವ ಸಂಕ್ರಾಂತಿಶಶಿಗ್ರಹಾದಿ ಹೋಮಾದಿ ತೇಷಾಂ ಚ ತಥಾ ವಿಧೀಯತೇ~।}\\\textbf{ಶಾಂತಿಶ್ಚ ಪುಷ್ಟಿಶ್ಚ ಭವೇದಿತಿ ಸ್ಮ ಚಿಂತಾಕುಲಾ ಬುದ್ದಿ ರಹರ್ನಿಶಂ ಚ~।। ೩೫~।।}
\end{verse}

ಇಂತಹ ಜನರಲ್ಲಿ ಸಂಕ್ರಾಂತಿ, ಸೂರ್ಯ-ಚಂದ್ರ ಗ್ರಹಣ ಕಾಲಗಳಲ್ಲಿ ನಾನಾ ವಿಧವಾದ ಹೋಮಗಳನ್ನು ಮಾಡಿಸುತ್ತಿದ್ದೆ. ಇನ್ನೇನನ್ನು ಮಾಡಿದರೆ ನನಗೆ ಶಾಂತಿ, ಪುಷ್ಟಿ ದೊರೆಯುವುದೋ ಆ ಬಗ್ಗೆ ಯಾವಾಗಲೂ ಚಿಂತಾಕ್ರಾಂತನಾಗಿ ಹಗಲು ರಾತ್ರಿ ಬುದ್ಧಿಯನ್ನೋಡಿಸುತ್ತಿದ್ದೆ.

\begin{verse}
\textbf{ಶ್ರಾದ್ದೇಷು ದರ್ಶೆಷು ಚ ಜನ್ಮವಾರೇಷ್ವನ್ಯೇಷು ಯೋಗಾದಿಷು ತೈಃ ಪರಿಗ್ರಹಃ~।}\\\textbf{ತಥೋತ್ಕ್ರಾಂತಿವೈತರಣೀಗೋದಾನಾದೀತ್ಯನೇಕಶಃ~।। ೩೬~।।}
\end{verse}

ಶ್ರಾದ್ಧಗಳಲ್ಲಿ, ಅಮಾವಾಸ್ಯೆಗಳಲ್ಲಿ, ಜನ್ಮದಿವಸಗಳಲ್ಲಿ, ಇತರ ಪುಣ್ಯಕರವಾದ ದಿವಸಗಳಲ್ಲಿ ಆ ಜನರಿಂದ ದಾನಗಳನ್ನು ಸ್ವೀಕರಿಸುತ್ತಿದ್ದೆ. ಅಪರ ಕರ್ಮಗಳಲ್ಲಿ ವೈತರಣೀಗೋದಾನವೇ ಮೊದಲಾದ ದಾನಗಳನ್ನು ಪ್ರಾಯಃ ನಾನೇ ಪಡೆಯುತ್ತಿದ್ದೆ.

\begin{verse}
\textbf{ಪ್ರತಿಗ್ರಾಹ್ಯ ಪ್ರತಿಗ್ರಾಹ್ಯ ಸುಖಮಾಸಂ ನಿರಾಕುಲಮ್~।}\\\textbf{ಏಕೋದ್ದಿಷ್ಟೇ ತಥಾ ಶ್ರಾದ್ಧೇ ಭೋಜನಂ ತು ಮಯಾ ಕೃತಮ್~।। ೩೭~।।}
\end{verse}

ಹೀಗೆ ದಾನದಿಂದ ಬಂದ ವಸ್ತುಗಳನ್ನೂ, ಹಣವನ್ನೂ, ಸ್ವೀಕರಿಸಿ ಸ್ವೀಕರಿಸಿ ನನ್ನ\break ಜೀವನವು ಸುಖವಾಗಿತ್ತು, ಯಾವ ತೊಂದರೆಯೂ ಇರಲಿಲ್ಲ. ಉತ್ತರಕ್ರಿಯಾ ಕಲಾಪಗಳಲ್ಲೊಂದಾದ “ಏಕೋದ್ದಿಷ್ಟ”ವೆಂಬ ಕಾರ್ಯದಲ್ಲಿಯೂ, ಶ್ರಾದ್ಧಗಳಲ್ಲಿಯೂ ಭೋಜನಮಾಡುತ್ತಿದ್ದೆನು.

\begin{verse}
\textbf{ಶೂದ್ರಾದಿಜ್ಞಾತಿಕ್ಷತ್ರಾಣಾಂ ಗೃಹಾಣಾಂ ಚೈವ ಚಿಂತನಾ~।}\\\textbf{ಪ್ರಾತರ್ಮೇ ಹೈಟತೋ ಸಾಯಂ ಸಾಯಾಹ್ನೇಽಪಿ ನ ಭುಜ್ಯತೇ~।। ೩೮~।।}
\end{verse}

ಶೂದ್ರರು, ಕ್ಷತ್ರಿಯರು ಮುಂತಾದ ಜನರ ಮನೆಗಳಿಗೆ ಹೋಗುತ್ತಿದ್ದ ನನಗೆ ಯಾರ ಮನೆಗೆ ಹೋದೆ, ಯಾರ ಮನೆಗೆ ಇನ್ನೂ ಹೋಗಿಲ್ಲವೆಂಬುದು ತಿಳಿಯದೆ ಚಿಂತೆಯಾಗುತ್ತಿತ್ತು. ಬೆಳಿಗ್ಗೆ ಮನೆ ಬಿಟ್ಟು ಹೊರಟರೆ ಅಲೆಯುತ್ತಾ ಸಾಯಂಕಾಲವೇ ಮತ್ತೆ ಮನೆಗೆ ಬರುತ್ತಿದ್ದೆ. ಚಿಂತೆಯಿಂದ ಬಳಲಿ ರಾತ್ರಿ ಊಟವನ್ನೇ ಮಾಡುತ್ತಿರಲಿಲ್ಲ.

\begin{verse}
\textbf{ರಾತ್ರೌ ಗ್ರಹಾಣಾಂ ಸರ್ವೇಷಾಂ ಬಲಾಬಲವಿಚಿಂತನಮ್~।}\\\textbf{ಕ್ರಿಯಾಮಾತ್ರಂ ಸದಾ ಸಂಧ್ಯಾ ಸಾಯಂ ಪ್ರಾತರ್ಮಯಾ ಕೃತಾ~।। ೩೯~।।}
\end{verse}

ನಾನು ಹೋಗುತ್ತಿದ್ದ ಮನೆಗಳ ಜನರ ಗ್ರಹಗತಿಗಳು ಹೇಗಿವೆ ಎಂಬುದೇ ನನ್ನ ರಾತ್ರಿಯ ಚಿಂತನೆ. ಪ್ರಾತಃಕಾಲದ ಮತ್ತು ಸಾಯಂಕಾಲದ ಸಂಧ್ಯಾವಂದನೆಯನ್ನು ಕೇವಲ ನೆಪಕ್ಕಾಗಿ ಮಾಡುತ್ತಿದ್ದೆ.

\begin{verse}
\textbf{ನಾವಕಾಶೋ ಹರೇರ್ಧ್ಯಾನೇ ಚಿಂತಾಕುಲಿತಮಾನಸಃ~।}\\\textbf{ತೇನ ಕರ್ಮವಿಪಾಕೇನ ಹೀನಃ ಕೂಷ್ಮಾಂಡಯೋನಿಜಃ~।। ೪೦~।। }
\end{verse}

\begin{verse}
\textbf{ಜಾತೋಽಹಂ ಮುನಿಶಾರ್ದೂಲ ವೃಕ್ಷೇsಸ್ಮಿನ್ ವಟಪಾದಪೇ~।}\\\textbf{ಇತ್ಯಾಕರ್ಣೋದಿತಂ ವಾಕ್ಯಂ ನಿಷ್ಠುರೋ ವಾಕ್ಯ ಮಬ್ರವೀತ್~।। ೪೧~।।}
\end{verse}

ಹೀಗೆ ಚಿಂತೆಯಿಂದಲೇ ವ್ಯಗ್ರಮನಸ್ಸುಳ್ಳ ನನಗೆ ಶ‍್ರೀಹರಿಯ ಧ್ಯಾನಕ್ಕೆ ಸಮಯವೇ ದೊರೆಯುತ್ತಿದ್ದಿಲ್ಲ. ಈ ದುಷ್ಕರ್ಮ ಕಾರಣದಿಂದ ಹೀನನಾಗಿ ಕೂಷ್ಮಾಂಡವೆಂಬ ಪಿಶಾಚ ಯೋನಿಯಲ್ಲಿ ಹುಟ್ಟಿರುತ್ತೇನೆ. ಈ ವಟವೃಕ್ಷದಲ್ಲಿ ನನ್ನ ವಾಸ. ಹೀಗೆಂದು “ಹೀನ”ನು ನುಡಿಯಲು “ನಿಷ್ಠುರ”ನೆಂಬ ಪಿಶಾಚಿಯು ಮಾತನಾಡಿತು.

\begin{verse}
\textbf{ಪುರಾಽಹಂ ಚಂಡಕೋ ವಿಪ್ರಃ ಕೌಷೀತಕಕುಲೋದ್ಭವಃ~।}\\\textbf{ಸುಭಗಸ್ಯ ತಥಾ ಪುತ್ರೋ ಧನಾಢ್ಯೋ ವೇದಪಾರಗಃ~।। ೪೨~।। }
\end{verse}

\begin{verse}
\textbf{ಶುದ್ಧ ಗೋದಾವರೀಸಂಗೇ ಜಾತೋ ನಿಷ್ಠುರಭಾಷಣಃ~।}\\\textbf{ವೈಶ್ಯವೃತ್ತೌ ಪ್ರವೃತ್ತೋsಹಂ ಸದೈವಾತಿಕೋಪನಃ~।। ೪೩~।।}
\end{verse}

ಹಿಂದೆ ನಾನು ಚಂಡಕನೆಂಬ ಬ್ರಾಹ್ಮಣನಾಗಿ ಕೌಷೀತಕ ವಂಶದಲ್ಲಿ ಹುಟ್ಟಿದ್ದೆ. ಹಣವಂತನಾಗಿಯೂ, ವೇದವನ್ನು ಅಧ್ಯಯನ ಮಾಡಿಯೂ ಇದ್ದೆ. ನನ್ನ ತಂದೆಯ ಹೆಸರು “ಸುಭಗ ”. ಪರಮಪವಿತ್ರವಾದ ಗೋದಾವರೀ ತೀರದಲ್ಲಿ ವಾಸ. ಬ್ರಾಹ್ಮಣನಾದರೂ ವೈಶ್ಯವೃತ್ತಿಯನ್ನು ಅವಲಂಬಿಸಿದ್ದೆ. ಅತ್ಯಂತ ಕೋಪಿ, ಮತ್ತು ನಿಷ್ಠುರವಾಗಿ ಮಾತನಾಡುತ್ತಿದ್ದೆ.

\begin{verse}
\textbf{ಅರ್ಥಾತುರಸ್ಯ ಮೇ ಚಾಸೀನ್ನ ಗುರುರ್ವಾ ನ ವಾ ಸುಹೃತ್~।}\\\textbf{ಸದಾ ವಾರ್ಧುಷಿಕಃ ಕ್ರೂರಃ ಕೃತ್ಯಾ ಕೃತ್ಯೇಷು ಮೂಢಧೀಃ~।। ೪೪~।।}
\end{verse}

ಹಣ ಸಂಪಾದಿಸಬೇಕೆಂಬ ಉತ್ಕಟ ಇಚ್ಛೆಯಿಂದ ಇದ್ದ ನನಗೆ ಗುರು, ಮಿತ್ರ ಎಂಬ ಭಾವನೆಯೇ ಇರಲಿಲ್ಲ. ಬಡ್ಡಿಯನ್ನು ವಸೂಲ್ಮಾಡುವುದರಲ್ಲಿ ತುಂಬ ಕ್ರೂರಿಯಾಗಿದ್ದೆ. ಮಾಡಬಹುದಾದ ಮತ್ತು ಮಾಡಬಾರದಂತಹ ಕಾರ್ಯಗಳ ಬಗ್ಗೆ ಅಜ್ಞಾನಿಯಾಗಿದ್ದೆ.

\begin{verse}
\textbf{ಕದಾಚಿದ್ಗುರವೇ ದತ್ತಂ ಋಣಂ ವೃದ್ಧ್ಯಭಿಕಾಂಕ್ಷಯಾ~।}\\\textbf{ತತ್ ಗೃಹೀತ್ವಾ ಗುರುಃ ಪ್ರಾಯಾತ್ಸನಂದೋ ನಾಮ ವೈ ದ್ವಿಜಃ~।। ೪೫~।।}
\end{verse}

ಒಂದು ದಿನ ನನ್ನ ಕುಲಗುರುಗಳಾದ ಸನಂದರೆಂಬ ಬ್ರಾಹ್ಮಣರು ನನ್ನ ಬಳಿ ಸಾಲ ಕೇಳಿದರು. ಬಡ್ಡಿಯ ಆಸೆಯಿಂದ ಅವರಿಗೆ ಸಾಲ ಕೊಟ್ಟೆ.

\begin{verse}
\textbf{ಪ್ರಯಾಗಂ ಭಾರ್ಯಯಾ ಸಾರ್ಧಂ ಪುತ್ರೇ ಭಾರಂ ನಿಧಾಯ ಚ~।}\\\textbf{ಪ್ರಯಾಗಮುತ್ತಮಂ ಕ್ಷೇತ್ರಂ ಗತ್ವಾ ಕತಿಪಯೈರ್ದಿನೈಃ~।। ೪೬~।। }\\\textbf{ಪಂಚತ್ವಮಭಿಪದ್ಯಾಥ ಮೋಕ್ಷಮಾರ್ಗಮುಪೇಯಿವಾನ್~।}
\end{verse}

ಗೃಹಕೃತ್ಯದ ವಿಚಾರವನ್ನು ತಮ್ಮ ಮಗನಿಗೆ ಒಪ್ಪಿಸಿ ಗುರುಗಳು ತಮ್ಮ ಪತ್ನಿ ಸಹಿತರಾಗಿ ಉತ್ತಮವಾದ ಪ್ರಯಾಗ ಕ್ಷೇತ್ರಕ್ಕೆ ಹೋಗಿ, ಅಲ್ಲಿ ಕೆಲವು ದಿವಸಗಳನ್ನು ಕಳೆದು, ಮೃತಿಯನ್ನು ಹೊಂದಿ, ಮೋಕ್ಷಕ್ಕೆ ಪ್ರಯಾಣ ಮಾಡಿದರು.

\begin{verse}
\textbf{ಭಾರ್ಯಾ ಚ ಬಂಧುಭಿಃ ಸಾರ್ಧಂ ಪುತ್ರಾಂತಿಕಮುಪಾಯಯೌ~।। ೪೭~।।}
\end{verse}

ಗುರುಗಳ ಪತ್ನಿ ಯು ಇತರ ಬಂಧುಜನರ ಸಹಿತ ತಮ್ಮ ಮಗನ ಬಳಿಗೆ ಬಂದರು.

\begin{verse}
\textbf{ತಸ್ಯ ಶ್ರುತ್ವಾ ತಥಾಪತ್ತಿಂ ತತ್ಪುತ್ರಂ ಜಗೃಹೇ ಕುಧೀಃ~।}\\\textbf{ಪಿತಾ ತೇ ಸ್ವರ್ಗತಃ ಪಾಪೀ ಹ್ಯ ಕೃತ್ವಾ ಋಣನಿಷ್ಕ್ರತಿಮ್~।। ೪೮~।। }
\end{verse}

\begin{verse}
\textbf{ಋಣಂ ಸವೃದ್ಧಿ ಕಂ ದೇಹಿ ಮಾ ವಿಲಂಬಿತುಮರ್ಹಸಿ~।}\\\textbf{ಕೃಣಂ ನ ಸ್ಥೀಯತೇ ಪುತ್ರ ನೋ ಗೃಹೀತ್ವಾ ನ ಗಮ್ಯತೇ~।। ೪೯~।।}
\end{verse}

ಈ ವಾರ್ತೆಯನ್ನು ಕೇಳಿದ ನಾನು ಗುರುಪುತ್ರನನ್ನು ಹಿಡಿದು ನಿಲ್ಲಿಸಿ ನಿಷ್ಠುರವಾಗಿ ಹೀಗೆ ನುಡಿದೆನು: "ಪಾಪಿಯಾದ ನಿನ್ನ ತಂದೆಯು ನನ್ನಿಂದ ಪಡೆದ ಸಾಲವನ್ನು ತಿರುಗಿ ಕೊಡದೆ ಸ್ವರ್ಗಸ್ಥನಾದನು. ತಡಮಾಡದೆ ಸಾಲವನ್ನು, ಬಡ್ಡಿ ಸಹಿತ ಕೊಟ್ಟು ಬಿಡು. ಹಣವನ್ನು ನಿನ್ನಿಂದ ಪಡೆಯದೇ ನಾನು ಇಲ್ಲಿಂದ ಹೋಗುವುದೇ ಇಲ್ಲ. ಒಂದು ಕ್ಷಣ ತಡವನ್ನೂ ಮಾಡಬೇಡ.”

\begin{verse}
\textbf{ಇತ್ಯುಕ್ತೇನ ತು ಶಿಷ್ಯೇಣ ಗುರುಪುತ್ರಸ್ತು ಮೇಽವದತ್~।}\\\textbf{ ಕೃತತ್ವೌರ್ಧ್ವದೇಹಿಕಂ ಕರ್ಮ ಯದ್ದೇಯಂ ತಚ್ಚ ದೀಯತೇ~।। ೫೦~।। }
\end{verse}

\begin{verse}
\textbf{ಕ್ಷಮ್ಯತಾಂ ಚ ಮಹಾಪತ್ತೌ ತ್ವದ್ದಾ ಸಃ ಖಲು ಬಾಂಧವಃ~।}\\\textbf{ಇತ್ಯುಕ್ತೇ ತು ಪುನಶ್ಚೋಕ್ತಂ ಮಯಾ ಪಾಪೇನ ಚೇತಸಾ~।। ೫೧~।।}
\end{verse}

ನಾನು ಹೀಗೆ ಹೇಳಿದ ಬಳಿಕ ಗುರುಪುತ್ರನು ಈ ರೀತಿ ನುಡಿದನು: “ನಮ್ಮ ತಂದೆಯವರ ಉತ್ತರಕ್ರಿಯಾದಿ ಕರ್ಮಗಳನ್ನು ಪೂರೈಸಿ ನಿನಗೆ ಸಲ್ಲಬೇಕಾದುದನ್ನು ಕೊಡುತ್ತೇನೆ. ಈಗ ಬಹಳ ಕಷ್ಟದಲ್ಲಿದ್ದೇನೆ ಕ್ಷಮಿಸು. ನನ್ನನ್ನು ನಿನ್ನ ದಾಸನೆಂದಾಗಲೀ ಅಥವಾ ಬಂಧುವೆಂದಾಗಲೀ ತಿಳಿದು ಕಾಲಾವಕಾಶಕೊಡು.” ಹೀಗೆಂಬ ಗುರುಪುತ್ರನ ಮಾತುಗಳನ್ನು ಕೇಳಿ ಪಾಪಿಷ್ಠನಾದ ನಾನು ಹೇಳಿದೆನು:

\begin{verse}
\textbf{ತ್ವಮರ್ಭಕೋ ನಾರ್ಜಯಿತುಂ ಸಮರ್ಥೋ}\\\textbf{ನ ವಾ ಗೃಹೇ ಚಾಸ್ತಿ ಪಿತುರ್ಮಹದ್ಧನಮ್~।}\\\textbf{ನ ವಿಶ್ವಸೇ ತ್ವಾಂ ಚಪಲಂ ಯುವಾನಂ} \\\textbf{ತಾವತ್ತ್ವಯಾ ದೇಯಮನರ್ಘ್ಯವಸ್ತು~।। ೫೨~।।}
\end{verse}

“ನೀನಾದರೋ ಇನ್ನೂ ಬಾಲಕ, ಸಂಪಾದಿಸುವ ಸಾಮರ್ಥ್ಯವಿಲ್ಲ, ಮನೆಯಲ್ಲಿ ನಿನ್ನ ತಂದೆ ಸಂಪಾದಿಸಿದ ಹಣವೂ ಇಲ್ಲ, ಚಪಲಚಿತ್ತನಾದ ನಿನ್ನ ಮಾತಿನಲ್ಲಿ ನನಗೆ ನಂಬಿಕೆ ಇಲ್ಲ. ನೀನು ನನ್ನ ಹಣವನ್ನು ಕೊಡುವವರೆವಿಗೂ ಏನಾದರೂ ಒಂದು ಬೆಲೆಬಾಳುವ ವಸ್ತುವನ್ನು ನನ್ನಲ್ಲಿ ಇಟ್ಟಿರು.”

\begin{verse}
\textbf{ಇತ್ಯೂಚಿವಾಸಂ ಕೃಪಣಂ ಮಹಾತ್ಮಾ} \\\textbf{ಪ್ರಯುಕ್ತವಾಸ್ತೇ ತ್ವರಯಾ ಕಿಮಸ್ತಿ}\\\textbf{ಅಹಂ ಚ ಭಾರ್ಯಾ ಮಮ ಚೈವ ಮಾತಾ} \\\textbf{ತ್ವದ್ದಾ ಸತಾಮೇತ್ಯ ವಸಾಮ ಗೇಹೇ~।। ೫೩~।।}
\end{verse}

ಹೀಗೆ ನಾನು ಹೇಳಿದ ಬಳಿಕ ಗುರುಪುತ್ರನು ನುಡಿದನು. (ಇಷ್ಟು ಅವಸರವಾಗಿ ಕೇಳಿದರೆ ನನ್ನ ಬಳಿ ಏನಿದೆ? ಆದುದರಿಂದ ನಾನೂ, ನನ್ನ ಪತ್ನಿಯೂ, ಮತ್ತು ನನ್ನ ತಾಯಿಯೂ ನಿನ್ನ ಮನೆಯಲ್ಲಿ ಸೇವಕರಾಗಿ ಕೆಲಸಮಾಡುತ್ತೇವೆ.”

\begin{verse}
\textbf{ಇತ್ಯುಕ್ತೇ ಮುನಿಪುತ್ರೇಣ ಅಹಂ ವಾಕ್ಯಮುದಾಹರಮ್~।}\\\textbf{ತ್ವಂ ಸಭಾರ್ಯಃ ಸುಖಂ ತಿಷ್ಠ ಚರನ್ ಕ್ವಾಪಿ ಚ ಮುದೃಣಮ್~।। ೫೪~।। }\\\textbf{ಪ್ರತಿದೇಹಿ ತು ಮುದ್ದಾಸ್ಯೇ ಮಾತೈಷಾ ತು ವಿಧೀಯತಾಮ್~।।}
\end{verse}

ಗುರುಪುತ್ರನ ಮಾತನ್ನು ಕೇಳಿ ನಾನು ಹೀಗೆಂದೆನು “ನೀನು ನಿನ್ನ ಪತ್ನಿ ಯಿಂದ ಸಹಿತನಾಗಿ ಸುಖವಾಗಿರು. ಎಲ್ಲಿಯಾದರೂ ತಿರುಗಿ ನನ್ನ ಸಾಲವನ್ನು ತೀರಿಸು. ಅಲ್ಲಿಯ ತನಕ ನಿನ್ನ ತಾಯಿಯನ್ನು ನಮ್ಮ ಮನೆಯಲ್ಲಿ ದಾಸಿಯಾಗಿಡು.”

\begin{verse}
\textbf{ತದೇತದ್ವಚನಂ ಶ್ರುತ್ವಾ ವಿಸ್ಮಿತೋ ಮುನಿಪುತ್ರಕಃ~।। ೫೫~।।} 
\end{verse}

\begin{verse}
\textbf{ಮಾತೈಷಾ ಚಾವಯೋಸ್ತಾತ ಶಿಷ್ಯಂ ಪುತ್ರೇಣ ಪಶ್ಯತಿ~।}\\\textbf{ಕಿಂ ನ ಜಾನಾಸಿ ತ್ವಂ ಮೂಢ ಭವತ್ಯೇಷಾ ಗೃಹೇ ತವ~।। ೫೬~।। }
\end{verse}

\begin{verse}
\textbf{ಭಾರ್ಯಾ ಮೇ ಸ್ಥೀಯತಾಂ ತಾವದ್ಯಾವನ್ಮೇ ಋಣನಿಷ್ಕೃತಿಃ~।।}
\end{verse}

ಈ ಮಾತನ್ನು ಕೇಳಿದ ಗುರುಪುತ್ರನು ಆಶ್ಚರ್ಯದಿಂದ ಹೇಳಿದನು: “ಗುರುಪತ್ನಿಯು ಶಿಷ್ಯನನ್ನು ಮಗನಂತೆ ಕಾಣುತ್ತಾಳೆ. ಮೂರ್ಖನಾದ ನಿನಗೆ ಇಷ್ಟೂ ಸಹ ತಿಳಿಯಲಿಲ್ಲವೆ? ತಾಯಿಯನ್ನು ದಾಸೀ ವೃತ್ತಿಗೆ ಕರೆಯುತ್ತಿದ್ದೀಯಲ್ಲ! ಇದು ಯೋಗ್ಯವಲ್ಲ. ನಿನ್ನ ಋಣವನ್ನು ತೀರಿಸುವವರೆಗೆ ನನ್ನ ಪತ್ನಿಯು ನಿನ್ನ ಮನೆಯಲ್ಲಿ ಇರಲಿ.”

\begin{verse}
\textbf{ಇತ್ಯುಕ್ತೇ ಮುನಿನಾ ತೇನ ತಥಾಸ್ತ್ವಿತಿ ಮಯೋದಿತಮ್~।। ೫೭~।।} 
\end{verse}

ಗುರುಪುತ್ರನು ಹೀಗೆ ಹೇಳಲು ನಾನು ಹಾಗೆಯೇ ಮಾಡು” ಎಂದೆನು.

\begin{verse}
\textbf{ಆಸ್ಥಿ ತಾ ಮದ್ವಶೇ ಪುಣ್ಯಾ ತ್ರಿದಿನಂ ಚ ಪತಿವ್ರತಾ~।}\\\textbf{ಆದದೇ ಮುನಿಪುತ್ರೋsಪಿ ನ ಗತ್ವಾ ರಾಜಮಂದಿರಮ್~।। ೫೮~।। }
\end{verse}

\begin{verse}
\textbf{ವ್ರತಂ ನಿರಶನಂ ಕೃತ್ವಾ ತ್ರಿದಿನಾಂತೇ ಋಣಂ ದದೌ~।}\\\textbf{ಮೊಚಯಿತ್ವಾ ಸಪತ್ನೀಂ ಚ ಜನಕಸ್ಯ ಚಕಾರ ಸಃ~।। ೫೯~।।}
\end{verse}

ಪತಿವ್ರತೆಯಾದ ಆ ಸ್ತ್ರೀಯು ಮೂರು ದಿನಗಳ ಕಾಲ ನನ್ನ ವಶದಲ್ಲಿದ್ದಳು. ಗುರುಪುತ್ರನು ಮೂರು ದಿವಸ ಉಪವಾಸ ವ್ರತ ಕೈಗೊಂಡು, ರಾಜನ ಅರಮನೆಯಲ್ಲಿದ್ದು, ನನಗೆ ಸಲ್ಲಬೇಕಾದ ಸಾಲವನ್ನು ತೀರಿಸಿದನು. ನಂತರ ಅವನ ಪತ್ನಿಯು ಬಿಡುಗಡೆ ಹೊಂದಿದಳು. ತನ್ನ ತಂದೆಯ ಉತ್ತರ ಕ್ರಿಯಾದಿಗಳನ್ನು ಆಮೇಲೆ ಪೂರೈಸಿದನು.

\begin{verse}
\textbf{ಭಕ್ತ್ಯಾ ಚ ಸ ಮಹಾನಾಸೌ ಸಕಲಂ ಪಾರಲೌಕಿಕಮ್~।}\\\textbf{ತೇನ ಕರ್ಮವಿಪಾಕೇನ ಕೂಷ್ಮಾಂಡೋ ನಿಷ್ಠುರೋಽಭವಮ್~।। ೬೦~।।}
\end{verse}

ಪರಮ ಭಕ್ತಿಯಿಂದ ಉತ್ತರ ಕ್ರಿಯಾದಿಗಳನ್ನು ಗುರುಪುತ್ರನು ಮಾಡಿದನು. ಅಂತಹ ಗುರುಪುತ್ರನನ್ನು ನಿಷ್ಠುರವಾಕ್ಯಗಳಿಂದ ನಿಂದಿಸಿದೆನಾದ ಕಾರಣ ನಾನು ಕೂಷ್ಮಾಂಡವೆಂಬ ಪಿಶಾಚಿಯೋನಿಯಲ್ಲಿ ನಿಷ್ಠುರನೆಂಬ ಹೆಸರಿನಿಂದ ಪಿಶಾಚಿಯಾದೆ.

\begin{verse}
\textbf{ಏವಂ ನಿಷ್ಠುರವಾಕ್ಯಂ ತು ಶ್ರುತ್ವಾ ಚಾಹ ನಿರಾಶ್ರಯಃ~।। ೬೧~।}
\end{verse}

ಹೀಗೆಂಬ ನಿಷ್ಠುರನ ಮಾತನ್ನು ಕೇಳಿ “ನಿರಾಶ್ರಯ”ನೆಂಬ ಪಿಶಾಚಿಯು ನುಡಿಯಿತು.

\begin{verse}
\textbf{ಸಮಿತ್ರಸ್ಯಾತ್ಮಜಃ ಪೂರ್ವಂ ನಾಮ್ನಾ ಗೋಪೀಥ ಸಂಜ್ಞಿತಃ~।}\\\textbf{ಗಾರ್ಗ್ಯಗೋತ್ರೇ ಸಮುತ್ಪನ್ನಃ ಪುರೇ ನಾಗಾಹ್ವಯೇ ದ್ವಿಜ~।। ೬೨~।।}
\end{verse}

ಹಿಂದೆ ನಾನು ಗೋಪೀಥ ಎಂಬ ಹೆಸರುಳ್ಳವನು. ಸಮಿತ್ರನೆಂಬುವನ ಮಗ ಗಾರ್ಗ್ಯಗೋತ್ರದವನು. ಹಸ್ತಿನಾಪುರದಲ್ಲಿ ವಾಸ..

\begin{verse}
\textbf{ಪಿತ್ರಾ ಮಯಾ ಚ ಸಂಪೂರ್ಣಮುಪಾಯೈರ್ಬಹುಧಾರ್ಜಿತಮ್~।}\\\textbf{ಧನಂ ಪಶ್ಚಾದ್ವಿನಷ್ಟೇ ಚ ಚಕ್ಷುಷೀ ಚ ಪಿತುರ್ಮಮ~।। ೬೩~।।}
\end{verse}

\begin{verse}
\textbf{ತತಸ್ತು ಧನಲೋಭೇನ ಮಹತಾ ಮೂಢಧೀರಹಮ್~।}\\\textbf{ನಿಷ್ಕಾಸಿತೌ ಚ ಪಿತರೌ‌ ಬಾಂಧವಾಶ್ಚ ಮದಾಶ್ರಯಾಃ~।। ೬೪~।।}
\end{verse}

ನಾನೂ ನನ್ನ ತಂದೆಯೂ ಸೇರಿ ಬಹಳ ಹಣ ಸಂಪಾದಿಸಿದೆವು. ಸ್ವಲ್ಪ ಕಾಲದ ನಂತರ ನನ್ನ ತಂದೆಯ ಕಣ್ಣು ಕಾಣಿಸದಂತಾಯಿತು. ಹಣದ ವ್ಯಾಮೋಹದಿಂದ ಅತ್ಯಂತ ಕೆಟ್ಟ ಬುದ್ದಿಯುಳ್ಳ ನಾನು ನನ್ನ ತಂದೆ-ತಾಯಿಯರನ್ನೂ ಹಾಗೂ ನನ್ನ ಆಶ್ರಯದಲ್ಲಿದ್ದ ಇತರ ಬಂಧುಗಳನ್ನೂ ಮನೆಯಿಂದ ಹೊರಗೆ ಹಾಕಿದೆ.

\begin{verse}
\textbf{ನೋ ವೃದ್ದಾಃ ಪೂಜಿತಾಃ ಕ್ವಾಪಿ ನ ಚ ಭಿಕ್ಷಾಶಿನಃ ಪರೇ~।}\\\textbf{ಪಿತರೌ ತು ಮಯಾ ತ್ಯಕ್ತೌ ಮೃತೌ ಸ್ವರ್ಗಪುರೀಂ ಗತೌ~।। ೬೫~।।}
\end{verse}

ಯಾವ ವೃದ್ಧರನ್ನೂ ಯಾವರೀತಿಯಿಂದಲೂ ಸತ್ಕರಿಸಲಿಲ್ಲ. ಭಿಕ್ಷೆ ಬೇಡುವವರಿಗೆ ಒಂದು ದಿನವೂ ಭಿಕ್ಷೆ ನೀಡಲಿಲ್ಲ. ನನ್ನಿಂದ ಹೊರಗೆ ಹಾಕಲ್ಪಟ್ಟ ತಂದೆ ತಾಯಿಯರು ಮೃತರಾಗಿ ಸ್ವರ್ಗಕ್ಕೆ ಹೋದರು.

\begin{verse}
\textbf{ತಂತುಮಾತ್ರೇಣ ಸಕಲಂ ಕೃತಂ ವೈ ಪಾರಲೌಕಿಕಮ್~।}\\\textbf{ನ ಕೃತಾಃ ಪಂಚಯಜ್ಞಾಶ್ಚ ನಾಧೀತಂ ನೇಷ್ಟಮೇವ ಚ~।। ೬೬~।।}
\end{verse}

ನಮ್ಮ ತಂದೆ-ತಾಯಿಯರ ಉತ್ತರಕ್ರಿಯಾದಿಗಳನ್ನು ಹಣದ ಲೋಭದಿಂದ ಸುಮ್ಮನೆ ಶಾಸ್ತ್ರಕ್ಕಾಗಿ ಮಾಡಿದೆ. ಒಂದು ದಿನವೂ ಪಂಚಯಜ್ಞಗಳನ್ನೂ ಮಾಡಲಿಲ್ಲ. ವೇದಾಧ್ಯಯನ, ಯಜ್ಞಗಳಂತೂ ಇಲ್ಲವೇ ಇಲ್ಲ.

\begin{flushleft}
\textbf{(ವಿಶೇಷಾಂಶ:\enginline{-}}
\end{flushleft}

\begin{verse}
\textbf{ಬ್ರಹ್ಮಯಜ್ಞೋ ದೇವಯಜ್ಞಃ ಪಿತೃಯಜ್ಞಸ್ತಥೈವ ಚ~।}\\\textbf{ಭೂತಯಜ್ಞೋ ನೃಯಜ್ಞಶ್ಚ ಪಂಚಯಜ್ಞಾಃ ಪ್ರಕೀರ್ತಿತಾಃ~।।}
\end{verse}

ಬ್ರಹ್ಮಯಜ್ಞ (ಋಷಿಯಜ್ಞ), ದೇವಯಜ್ಞ, ಪಿತೃಯಜ್ಞ, ಭೂತಯಜ್ಞ, ಮನುಷ್ಯಯಜ್ಞ ಹೀಗೆ ಐದು ಯಜ್ಞಗಳನ್ನು ಬ್ರಾಹ್ಮಣನಾದವನು ಪ್ರತಿದಿನದಲ್ಲಿಯೂ ಆಚರಿಸಬೇಕು. ಇವುಗಳ ವಿವರಣೆಗಳನ್ನು ಆಹ್ನಿಕ ಮಂಜರಿ ಮುಂತಾದ ಗ್ರಂಥಗಳಲ್ಲಿ ಕಾಣಬಹದು).

\begin{verse}
\textbf{ತೇನ ಕರ್ಮವಿಪಾಕೇನ ಕೂಷ್ಮಾಂಡಸ್ತು ನಿರಾಶ್ರಯಃ~।}\\\textbf{ನಿರಾಶ್ರಯೋದಿತಂ ಶ್ರುತ್ವಾ ಶಾಸ್ತ್ರೋದ್ವೃತ್ತಃ ತಮಬ್ರವೀತ್~।। ೬೭~।।}
\end{verse}

ಈ ಪಾಪಕೃತ್ಯಗಳ ಫಲವಾಗಿ ನಾನು ಕೂಷ್ಮಾಂಡವೆಂಬ ಪಿಶಾಚಗಣದಲ್ಲಿ `ನಿರಾಶ್ರಯ'ನೆಂಬ ಹೆಸರಿನಿಂದ ಹುಟ್ಟಿರುತ್ತೇನೆ. ನಿರಾಶ್ರಯನು ಹೇಳಿದ್ದನ್ನು ಕೇಳಿದ ನಂತರ ಶಾಸ್ತ್ರೋದ್ವೃತ್ತನು ನುಡಿದನು:

\begin{verse}
\textbf{ಪುಽರಾಹಂ ಚಾಂಧ್ರದೇಶೇಷು ಗ್ರಾಮೇ ಕುಕ್ಕುಟಸಂಜ್ಞಕೇ~।}\\\textbf{ಸುಕಚ್ಛಯಿತಿನಾಮ್ನಾಹಂ ಸುಚ್ಛಾಯಸ್ಯ ಸುತೋ ಮಹಾನ್~।। ೬೮~।।}
\end{verse}

ನಾನು ಹಿಂದೆ ಆಂಧ್ರದೇಶದ ಕುಕ್ಕುಟವೆಂಬ ಊರಿನಲ್ಲಿದ್ದೆ. ನನ್ನ ಹೆಸರು ಸುಕಚ್ಛ, ಸುಚ್ಛಾಯನ ಮಗ.

\begin{verse}
\textbf{ಸರ್ವಾನ್ವೇದಾನ್ನಧೀತ್ಯಾಹಂ ಸೇತಿಹಾಸಪುರಾಣಕಾನ್~।}\\\textbf{ನ ಕೃತಂ ಶಾಸ್ತ್ರವಿಹಿತಂ ಜಾನತಾಪಿ ಮಹಾಮುನೇ~।। ೬೯~।।}
\end{verse}

ಮುನಿಶ್ರೇಷ್ಠನೇ, ಎಲ್ಲ ವೇದಗಳನ್ನು, ಇತಿಹಾಸ-ಪುರಾಣಗಳನ್ನೂ ಅಧ್ಯಯನ ಮಾಡಿದ್ದೆ. ಶಾಸ್ತ್ರವಿಹಿತವಾದ ಆಚಾರಗಳನ್ನು ತಿಳಿದಿದ್ದರೂ ಆಚರಿಸುತ್ತಿದ್ದಿಲ್ಲ.

\begin{verse}
\textbf{ಧರ್ಮಸ್ಯ ವಕ್ತಾ ಸಂಪೂರ್ಣಂ ನಾನುಷ್ಠಾತಾ ಕದಾಚನ~।}\\\textbf{ತೇನ ಕರ್ಮವಿಪಾಕೇನ ಶಾಸ್ತ್ರೋದ್ವೃತ್ತೋ ಭವಾಮ್ಯಹಮ್~।। ೭೦~।।}
\end{verse}

ಇತರರಿಗೆ ಧರ್ಮವನ್ನು ಬೋಧಿಸುತ್ತಿದ್ದೆನೇ ಹೊರತು ನಾನು ಒಂದು ಬಾರಿಯ ಅನುಷ್ಠಾನ ಮಾಡಲಿಲ್ಲ. ಈ ಪಾಪಕರ್ಮದ ದೆಸೆಯಿಂದ ನಾನು ಶಾಸ್ತ್ರೋದ್ವೃತ್ತನೆಂಬ ಪಿಶಾಚಿಯಾದೆ.

\begin{verse}
\textbf{ಏತೇಷಾಮೇವ ಸಪ್ತಾನಾಂ ಜಾತಾ ದೈವಾಚ್ಚ ಸಂಗತಿಃ~।}\\\textbf{ವಟೇಽಸ್ಮಿನ್ ವಿಪಿನೇ ಘೋರೇ ಮೃತಾ ಮತ್ತಗಜಾದ್ವಯಮ್~।। ೭೧~।।}
\end{verse}

ನಾವು ಏಳು ಜನರೂ ದೈವಯೋಗದಿಂದ ಒಂದೆಡೆ ಸೇರಿದೆವು. ಮದಿಸಿದ ಆನೆಯಿಂದ ನಾವು ಏಳು ಜನರೂ ಕೊಲ್ಲಲ್ಪಟ್ಟು ಈ ಘೋರವಾದ ಕಾಡಿನಲ್ಲಿ ಈ ವಟವೃಕ್ಷದಲ್ಲಿ ವಾಸಮಾಡುತ್ತಿದ್ದೇವೆ.

\begin{verse}
\textbf{ವಟೇ ಕೂಷ್ಮಾಂಡತಾಂ ಯಾತಃ ಸಪ್ತ ವೈ ನಿಧನಂ ಗತಾಃ~।}
\end{verse}

ನಾವೆಲ್ಲರೂ ಕೂಷ್ಮಾಂಡವೆಂಬ ಪಿಶಾಚಗಣಕ್ಕೆ ಸೇರಿ ಈ ವಟವೃಕ್ಷದಲ್ಲಿರುತ್ತೇವೆ. ನಮ್ಮೆಲ್ಲರಿಗೂ ಮೃತ್ಯುವು ಏಕಕಾಲಕ್ಕೆ ಬಂತು.

\begin{verse}
\textbf{ಪುರಾ ಸೋಮಾನ್ವಯೇ ಜಾತಾ ಅಸ್ಮಿನ್ನಾಗಾಹ್ವಯಾಂ ಪುರೀಮ್~।। ೭೨~।।} 
\end{verse}

\begin{verse}
\textbf{ಸುಬರ್ಹಿಷಃ ಕ್ಷತ್ರಿಯಸ್ಯ ಸುತಾಃ ಸಪ್ತಾಪಿ ಮಾನದ~।}\\\textbf{ಬಹಿರ್ಜನಪದೇ ಗಾಶ್ಚ ಹೃತಾಶ್ಚೋರೈಃ ಕದಾಚನ~।। ೭೩~।।}
\end{verse}

ನಾವೆಲ್ಲರೂ ಹಿಂದೆ ನಾಗಾಹ್ವಯವೆಂಬ ಊರಿನಲ್ಲಿದ್ದ ಸುಬರ್ಹಿಷನೆಂಬ ಕೃತಿಯನಿಗೆ\break ಮಕ್ಕಳಾಗಿದ್ದೆವು. ಒಂದು ಸಲ ಊರಹೊರಗೆ ಮೇಯುತ್ತಿದ್ದ ಹಸುಗಳನ್ನು ಕಳ್ಳರು ಕದ್ದುಕೊಂಡು ಹೋದರು.

\begin{verse}
\textbf{ಪೃಷ್ಠಾನುಯಾಯಿನಸ್ತೇಷಾಮಭವನ್ ಶಸ್ತ್ರಪಾಣಯಃ~।}\\\textbf{ಭೀತಾಶ್ಚೋರಾ ವಟೇ ಯಸ್ಮಿನ್ ನಿಷಣ್ಣಾನ್ ಧ್ಯಾನಕರ್ಮಣಿ~।। ೭೪~।।}
\end{verse}

ನಾವು ಆಯುಧಗಳನ್ನು ಹಿಡಿದುಕೊಂಡು ಕಳ್ಳರನ್ನು ಅಟ್ಟಿಸಿಕೊಂಡು ಹೋದೆವು. ಆ ಕಳ್ಳರು ಹೆದರಿ ಈ ವಟವೃಕ್ಷದವರೆವಿಗೂ ಬಂದು ಅಲ್ಲಿಂದ ಬೇರೆ ಬೇರೆಯಾಗಿ ತಪ್ಪಿಸಿಕೊಂಡು ಓಡಿದರು. ವಟವೃಕ್ಷದ ಕೆಳಗೆ ಧ್ಯಾನಾಸಕ್ತರಾದ ಬ್ರಾಹ್ಮಣರು ಕುಳಿತಿದ್ದರು.

\begin{verse}
\textbf{ಅಜ್ಞಾತಲಿಂಗಾ ಅಭಿತೋ ಲೀನಾಸ್ತೇ ದುದ್ರುವುರ್ದ್ರುತಮ್~।}\\\textbf{ತಾಡಯಾಮಾಸ ತಾನ್ ವಿಪ್ರಾನ್ ಶಸ್ತ್ರೈಃ ಉನ್ಮಾರ್ಗವರ್ತಿನಃ~।। ೭೫~।।}
\end{verse}

ಕಳ್ಳರು ತಪ್ಪಿಸಿಕೊಂಡಮೇಲೆ ಧ್ಯಾನಾಸಕ್ತರಾದ ಬ್ರಾಹ್ಮಣರನ್ನು ಉನ್ಮತ್ತರಾದ ನಾವು ಆಯುಧಗಳಿಂದ ಹೊಡೆದೆವು.

\begin{verse}
\textbf{ಶಶಪುಸ್ತೇ ಮಹಾತ್ಮಾನೋ ದುರ್ವಿಚಾರಣತತ್ಪರಾನ್~।}\\\textbf{ಮತ್ತತ್ವಾತ್ ಭವತಾಂ ಭೂಯಾತ್ ಮೃತಿರ್ಮತ್ತಮತಂಗಜಾತ್~।। ೭೬~।।}
\end{verse}

ವಿಚಾರವಾಡದೇ ಬ್ರಾಹ್ಮಣರನ್ನು ದಂಡಿಸಿದ ನನಗೆ ಬ್ರಾಹ್ಮಣರು ಶಾಪಕೊಟ್ಟರು: `ದುರ\-ಹಂಕಾರದಿಂದ ನಿಮಗೆ ಮದಿಸಿದ ಆನೆಯಿಂದ ಮರಣ ಸಂಭವಿಸಲಿ.'

\begin{verse}
\textbf{ಅಸ್ಮಿ ವಟವನೇ ಹ್ಯಸ್ಮಿನ್ ಸಮಕಾಲೇ ಮೃತಿಂ ಯಯುಃ~।}\\\textbf{ಅಸ್ಮಾತ್ ಪಾಪಾದ್ವಯಂ ಸರ್ವೇ ದೈವಾತ್ಸಂಗತಿಮಾಗತಾಃ~।। ೭೭~।। }\\\textbf{ಯುಗಪತ್ ಮತ್ತಮಾತಂಗಾತ್ ಮೃತಾಃ ಸರ್ವೇ ವಟೇ ವಯಮ್~।। ೭೮~।।}
\end{verse}

ಈ ಶಾಪದ ಫಲವಾಗಿ ಈ ಅರಣ್ಯದಲ್ಲಿ ನಾವೆಲ್ಲರೂ ಒಂದೇ ಕಾಲದಲ್ಲಿ ಸೇರಿದ್ದಾಗ ಈ ವಟವೃಕ್ಷದ ಬುಡದಲ್ಲಿ ಮದಿಸಿದ ಆನೆಯಿಂದ ಸಂಹರಿಸಲ್ಪಟ್ಟೆವು.

\begin{verse}
\textbf{ಇತ್ಯೇತತ್ ಕಥಿತಂ ಸರ್ವಂ ಅಸ್ಮಾನ್ ಉದ್ಧರ ಪಂಡಿತ~।}\\\textbf{ದಯಾಲೂನಾಂ ಚ ಮಹತಾಂ ನ ವಯೋ ಹಿ ಸಮೀಕ್ಷತೇ~।। ೭೯~।।}
\end{verse}

ಪಂಡಿತನಾದ ಮುನಿಪುತ್ರನೇ, ನಮ್ಮ ಚರಿತ್ರೆಯೆಲ್ಲವನ್ನೂ ಇದುವರೆಗೂ ವರ್ಣಿಸಿದೆವು. ನಮ್ಮನ್ನು ಉದ್ಧರಿಸು. ಮಹಾತ್ಮರಿಗೂ, ದಯಾವಂತರಿಗೂ ವಯಸ್ಸಿನ ಗಣನೆಗೆ ಕಾರಣವಿಲ್ಲವಷ್ಟೆ!

\begin{center}
ಇತಿ ಶ‍್ರೀ ವಾಯುಪುರಾಣೇ ಮಾಘಮಾಸಮಾಹಾತ್ಮ್ಯೇ ದ್ವಾದಶೋsಧ್ಯಾಯಃ
\end{center}

\begin{center}
 ಶ‍್ರೀ ವಾಯುಪುರಾಣಾಂತರ್ಗತ ಮಾಘಮಾಸ ಮಾಹಾತ್ಮ್ಯೆಯಲ್ಲಿ \\ ಹನ್ನೆರಡನೇ ಅಧ್ಯಾಯವು ಸಮಾಪ್ತಿಯಾಯಿತು.
\end{center}

\newpage

\section*{ಅಧ್ಯಾಯ\enginline{-}೧೩}

\emptypage

\begin{flushleft}
\textbf{ಡಾಕಿನೀಗಣಾ ಊಚುಃ\enginline{-}}
\end{flushleft}

ಡಾಕಿನೀಗುಂಪಿಗೆ ಸೇರಿದ ಪಿಶಾಚಿಗಳು ನುಡಿದುವು:-

\begin{verse}
\textbf{ವೃತ್ತಾಂತಮಪಿ ಚಾಸ್ಮಾಕಂ ಶೃಣು ವಿಪ್ರ ಸಮಾಹಿತಃ~।}\\\textbf{ಡಾಕಿನೀತ್ವಮವಾಪ್ತಾನಾಂ ಶಾಲ್ಮಲೀತಟಾಮೀಯುಷಾಮ್~।। ೧~।।}
\end{verse}

ಬ್ರಾಹ್ಮಣನೇ, ಡಾಕಿನೀಗಣಕ್ಕೆ ಸೇರಿ ಈ ಶಾಲ್ಮಲೀ ವೃಕ್ಷದಲ್ಲಿರುವ ನಮ್ಮ ಪೂರ್ವವೃತ್ತಾಂತವನ್ನು ನಿರೂಪಿಸುತ್ತೇವೆ. ಮನಸ್ಸಿಟ್ಟು ಲಾಲಿಸು.

\begin{flushleft}
ಜಾನುಶಿರ ಉವಾಚ-
\end{flushleft}

\begin{verse}
\textbf{ಬ್ರಾಹ್ಮಣೋಽಹಂ ಪುರಾ ಚೌಂಧ್ರೇ ದೇಶೇ ಕೌಂಡಿನ್ಯ ಗೋತ್ರಜಃ~।}\\\textbf{ಸುಬಲೇಸ್ತನಯೋ ನಾಮ್ನಾ ವರ್ಚಿಷ್ಮಾನಿತಿ ವಿಶ್ರುತಃ~।। ೨~।।}
\end{verse}

ಜಾನುಶಿರನೆಂಬ ಪಿಶಾಚಿಯು ಹೇಳಿತು:-

ನಾನು ಹಿಂದೆ ಆಂಧ್ರರಾಜ್ಯದಲ್ಲಿ ಕೌಂಡಿನ್ಯ ಗೋತ್ರದಲ್ಲಿ ಉತ್ಪನ್ನನಾದ ಪ್ರಸಿದ್ಧನಾದ ಸುಬಲಿ ಎಂಬ ಬ್ರಾಹ್ಮಣನ ಮಗನಾಗಿದ್ದೆ. ನನ್ನ ಹೆಸರು ವರ್ಚಿಷ್ಮಾನ್.

\begin{verse}
\textbf{ದಾರಿದ್ರ್ಯಂ ಕರ್ಮಭಿಃ ಪೂರ್ವಂ ಪರೀತೋಹ್ಯಗಮಂ ಬಲೀ~।}\\\textbf{ಕುಟುಂಬಭರಣೇ ಸಕ್ತಃ ಕೃತ್ಯೈರ್ಬಹುವಿಧೈರಪಿ~।। ೩~।।}
\end{verse}

ಹಿಂದಿನ ಕರ್ಮದ ಫಲವಾಗಿ ದಾರಿದ್ರವನ್ನು ಅನುಭವಿಸುತ್ತಾ ಕುಟುಂಬ ರಕ್ಷಣೆಗಾಗಿ ನಾನಾವಿಧವಾದ ಕಾರ್ಯಗಳನ್ನು ಮಾಡುತ್ತಿದ್ದೆನು.

\begin{verse}
\textbf{ಪಶ್ಚಾದ್ಧೇಮಪುರೀ ಪ್ರಾಪ್ತಾ ಮಯಾ ಹ್ಯಾ ಪಣಜೀವಿನಾ~।}\\\textbf{ಮಯಾ ದ್ವಾದಶವರ್ಷಂ ತು ಕೃತಂ ಚೈಕಾದಶೀವ್ರತಮ್~।। ೪~।।}
\end{verse}

ಆಂಧ್ರ ದೇಶವನ್ನು ಬಿಟ್ಟು ಹೇಮಪುರಿ ಎಂಬ ಊರಿಗೆ ಬಂದು ಅಂಗಡಿಯನ್ನಿಟ್ಟು, ವ್ಯಾಪಾರದಿಂದ ಜೀವನ ಮಾಡಿದನು. ಹನ್ನೆರಡು ವರ್ಷಗಳ ಕಾಲ ಏಕಾದಶೀ ವ್ರತವನ್ನು ಆಚರಿಸಿದೆ.

\begin{verse}
\textbf{ದಾನಯಜ್ಞ ತಪೋಭಿಶ್ಚ ತೀರ್ಥಚರ್ಯಾದಿಭಿಸ್ತಥಾ~।}\\\textbf{ಯತ್ಪುಣ್ಯ ಮಧಿಕಂ ಪ್ರೋಕ್ತಂ ವೇದೈಃ ಸ್ಮೃತಿಪುರಾಣಕೈಃ~।। ೫~।।}
\end{verse}

\begin{verse}
\textbf{ಅಕೃತ್ವಾ ಪಾಪಮಾಪ್ನೋತಿ ಸರ್ವಾಧಿಕಮಸಂಶಯಮ್~।}\\\textbf{ವರಂ ಸ್ವಮಾತೃಗಮನಂ ವರಂ ಗೋಮಾಂಸಭಕ್ಷಣಮ್~।। ೬~।।}
\end{verse}

\begin{verse}
\textbf{ಕಾಶ್ಯಾಂ ಪಿತೊರ್ವಧಶ್ಚಾಪಿ ತಥಾ ಹ್ಯ ತ್ಯಾಧಿಪಂಚಕಮ್~।}\\\textbf{ಗೋವಧಾದಿ ಮಹಾಪಾಪಂ ಸ್ತ್ರೀವಧಾದ್ಯು ಪಪಾತಕಮ್~।। ೭~।। }
\end{verse}

\begin{verse}
\textbf{ಅನಸ್ಥಿ ಜೀವಹನನಂ ತೇಭ್ಯಃ ಶ್ರೇಷ್ಠತಮಂ ಮತಮ್~।}\\\textbf{ಕುಗತಿಃ ಪಾಪಿನಾಂ ಲೋಕೇ ಕಲಾವೇಕಾದಶೀ ವ್ರತಮ್~।। ೮~।।}
\end{verse}

ವೇದ, ಸ್ಮತಿ, ಪುರಾಣಗಳಲ್ಲಿ ದಾನ, ಯಜ್ಞ, ತಪಸ್ಸು, ತೀರ್ಥಯಾತ್ರೆಗಳು ಮಹಾಪುಣ್ಯದಾಯಕವೆಂದು ಹೇಳಲ್ಪಟ್ಟಿದೆ. ಯಾವ ಸತ್ಕರ್ಮವನ್ನು ಮಾಡದಿದ್ದರೆ ನಿಸ್ಸಂದೇಹವಾಗಿ ಅತ್ಯಧಿಕ ಪಾಪವು ಪ್ರಾಪ್ತವಾಗುತ್ತದೆಯೋ ಅದೇ ಏಕಾದಶೀ ವ್ರತ. ತನ್ನ ತಾಯಿಯ ಸಂಗಡ ರಮಿಸುವುದು, ಗೋಮಾಂಸ ತಿನ್ನುವುದು, ಕಾಶೀಕ್ಷೇತ್ರದಲ್ಲಿ ತನ್ನ ತಂದೆ-ತಾಯಿಯರನ್ನು ಸಂಹರಿಸುವುದು, ಹಾಗೂ ಪಂಚ ಮಹಾಪಾತಕಗಳನ್ನು ಆಚರಿಸುವುದು, ಗೋಹತ್ಯೆ, ಸ್ತ್ರೀವಧೆ, ಅಸ್ಥಿರಹಿತವಾದ ಪ್ರಾಣಿಗಳ ಸಂಹಾರ-ಇವೆಲ್ಲವೂ ಪಾತಕಗಳೇ ಸರಿ. ಆದರೆ ಏಕಾದಶೀ ವ್ರತವನ್ನು ಪರಿತ್ಯಾಗ ಮಾಡುವುದಕ್ಕಿಂತ ಉತ್ತಮ. ಕಲಿಯುಗದಲ್ಲಿ ಏಕಾದಶೀ ವ್ರತವನ್ನು ಆಚರಿಸದೇ ಇರುವ ಜನರಿಗೆ ಅತ್ಯಂತ ನೀಚವಾದ ಗತಿಯು ಪ್ರಾಪ್ತವಾಗುತ್ತದೆ.

\begin{verse}
\textbf{ಏಕಾದಶೀವ್ರತಜಪುಣ್ಯಮನಂತರಾಧಸಃ} \\\textbf{ಪ್ರೀತ್ಯೈ ಕೃತಂ ಪಿತೃ ಕುಲೋತ್ತರಣೈ ಕಹೇತುಮ್~।}\\\textbf{ಕ್ರೀತಂ ಮಯಾ ಸುಕೃತಮಸ್ಯ ವ್ರತಸ್ಯ ಕಾರಣಂ} \\\textbf{ಮೋಕ್ಷಾದಿದಿವ್ಯಪುರುಷಾರ್ಥಕುಲಸ್ಯ ಪಾವನಮ್~।। ೯~।।}
\end{verse}

ಪಿತೃ ಕುಲಗಳನ್ನು ಉದ್ಧರಿಸುವ, ಧರ್ಮ, ಅರ್ಥ ಕಾಮ, ಮೋಕ್ಷವೆಂಬ ಪುರುಷಾರ್ಥಗಳಿಗೆ ಕಾರಣವಾದ, ಅತ್ಯಂತ ಪವಿತ್ರವಾದ ಏಕಾದಶೀವ್ರತವನ್ನು ಹನ್ನೆರಡು ವರ್ಷಗಳು ಆಚರಿಸಿ ಪುಣ್ಯವನ್ನು ಗಳಿಸಿ ಭಗವಂತನ ಪ್ರೀತಿಗೆ ಪಾತ್ರನಾಗಿದ್ದೆ.

\begin{verse}
\textbf{ದತ್ತಂ ವಣಿಗ್ಭ್ಯೋ ಧನಕಾಂಕ್ಷಯಾ ಮಯಾ} \\\textbf{ದ್ವಿಷಟ್ಸು ವರ್ಷೇಷು ಕೃತಂ ಚ ಭೂರಿಶಃ~।}\\\textbf{ತೇನೈವ ಪಾಪೇನ ಧನಂ ಚ ನಷ್ಟಂ} \\\textbf{ಕುಲಂ ಚ ಸರ್ವಂ ನರಕೇಷು ಪಾತಿತಮ್~।। ೧೦~।।}
\end{verse}

ಅಂತಹ ಉತ್ತಮ ಪುಣ್ಯವನ್ನು ಧನದ ಆಶೆಯಿಂದ ವರ್ತಕರಿಗೆ ಮಾರಿದೆ. ಆ ಪಾಪದ ಫಲವಾಗಿ ನನ್ನಲ್ಲಿದ್ದ ಹಣವೆಲ್ಲ ನಷ್ಟವಾಯಿತು; ನನ್ನ ಕುಲದವರೆಲ್ಲ ನರಕದಲ್ಲಿ ಬಿದ್ದರು.

\begin{verse}
\textbf{ಅರ್ಜಿತಂ ತು ಮಯಾ ಭೂರಿ ಧನಂ ಪಾಪೇನ ಕರ್ಮಣಾ~।}\\\textbf{ಗೃಹಿತಾ ಶಾಲ್ಮಲೀಂ ಪ್ರಾಪ್ತಂ ಜ್ಞಾತ್ವಾ ಚೋರಾಃ ಸಮಾಯಯುಃ~।। ೧೧~।।}
\end{verse}

\begin{verse}
\textbf{ನಿಹತ್ಯ ಮಾಂ ದ್ರುಮೇ ಕಸ್ಮಿನ್ ಧನಮಾದಾಯ ತೇ ಯಯುಃ~।}\\\textbf{ಡಾಕಿನೀ ಹ್ಯಭವಂ ನಾಮ್ನಾ ಜಾತೋಽಸ್ಮಿನ್ ಶಾಲ್ಮಲೀದ್ರುಮೇ~।। ೧೨~।।}
\end{verse}

ಹೀಗೆ ಪಾಪಕರ್ಮದಿಂದ ಗಳಿಸಿದ ಅಪಾರ ಹಣವನ್ನು ತೆಗೆದುಕೊಂಡು ಈ ಶಾಲ್ಮಲೀ ವೃಕ್ಷದ ಬಳಿ ಬಂದೆ. ಕಳ್ಳರು ನನ್ನನ್ನು ಸಂಹರಿಸಿ ಹಣವನ್ನು ತೆಗೆದುಕೊಂಡು ಹೋದರು. ಈ ಶಾಲ್ಮಲೀ ಮರದಲ್ಲಿ ಡಾಕಿನೀ ಗಣಕ್ಕೆ ಸೇರಿದ ಪಿಶಾಚಿಯಾದೆ.

\begin{verse}
\textbf{ಅಧಃಸ್ಥಲಂ ಪಿತೃ ಕುಲಂ ಪ್ರಾಪಿತಂ ಪುಣ್ಯ ವಿಕ್ರಯಾತ್~।}\\\textbf{ಅನಂತಪುಣ್ಯ ಮೌಲ್ಯೈಶ್ಚ ಊರುಜೇಭ್ಯಶ್ಚ~।। ೧೩~।।}
\end{verse}

ಈ ಪುಣ್ಯವನ್ನು ಕ್ರಯಕ್ಕಾಗಿ ವೈಶ್ಯರಿಗೆ ಮಾರಿದ ಕಾರಣ ನನ್ನ ಪಿತೃ ಕುಲವು ನರಕದಲ್ಲಿ ಬಿದ್ದಿತು.

\begin{verse}
\textbf{ದ್ವಿತೀಯೋsಯಂ ಮಮ ಗ್ರಾಮೀ ಊರುದೇಶೇ ಶಿರಾ ಅಯಮ್~।}\\\textbf{ಮಾತುಲೋ ಮಮ ಪಾಪಾತ್ಮಾ ಕ್ರೂರಃ ಶಾಂಡಿಲ್ಯಗೋತ್ರಜಃ~।। ೧೪~।।}
\end{verse}

ಮೊಳಕಾಲಿನಲ್ಲಿ (ತೊಡೆಯಲ್ಲಿ) ತಲೆಯುಳ್ಳ ಈ ಎರಡನೆಯವನು ನನ್ನ ಊರಿನವನೇ, ನನ್ನ ಸೋದರಮಾವ, ಶ್ಯಾಂಡಿಲ್ಯ ಗೋತ್ರದವನು, ಮಹಾಪಾಪಿ ಮತ್ತು ಕ್ರೂರಿ.

\begin{verse}
\textbf{ಧೀಮತೋ ವೇದಕೀರ್ತಿಶ್ಚ ತನಯೋ ಧರ್ಮಸಂಜ್ಞ ಕಃ~।}\\\textbf{ಹೇಮನಾಮ್ನೀಂ ಪುರೀಮೇವ ಪ್ರಾಪ್ತೋ ವೈಶ್ಯೋಪಜೀವಕಃ~।। ೧೫~।।}
\end{verse}

ಇವನು ಒಳ್ಳೆ ಜ್ಞಾನಿ, ವೇದಕೀರ್ತಿಯೆಂಬುವರ ಮಗ, ಧರ್ಮ ಎಂದು ಹೆಸರು. ನನ್ನ ಜತೆಯಲ್ಲಿಯೇ ಹೇಮಪುರಿ ಗ್ರಾಮಕ್ಕೆ ಬಂದು ವೈಶ್ಯವೃತ್ತಿಯಿಂದ ಜೀವನ ಮಾಡುತ್ತಿದ್ದನು.

\begin{verse}
\textbf{ದ್ವಾದಶಾಬ್ದಂ ಸ್ಥಿತಂ ತೇನ ಮಯಾ ಸಾರ್ಧಂ ದ್ವಿಜೋತ್ತಮ~।}\\\textbf{ಕುಟುಂಬಭರಣಾಸಕ್ತೋ ಧನಸ್ಯಾರ್ಜನತತ್ಪರಃ~।। ೧೬~।।}
\end{verse}

ಸಂಸಾರವನ್ನು ರಕ್ಷಿಸುವುದರಲ್ಲಿಯೇ ವಿಶೇಷ ಆಸಕ್ತನಾದ ಇವನು ಹಣವನ್ನು ಸಂಪಾದಿಸುವುದರಲ್ಲಿ ಕಾಲವನ್ನೆಲ್ಲ ಕಳೆದು ನನ್ನ ಸಂಗಡ ಹನ್ನೆರಡು ವರ್ಷಗಳ ಕಾಲ ಇದ್ದನು.

\begin{verse}
\textbf{ಅಮುನಾ ಜಾಹ್ನವೀಯಾತ್ರಾ ದಶ ವಾರಂ ಕೃತಾ ಆಪಿ~।}\\\textbf{ಕ್ರೀತಂ ತಸ್ಯಾ ಮಹತ್ಪುಣ್ಯಂ ಊರುಜೇಭ್ಯಸ್ತತೋ ಹ್ಯಹಮ್~।। ೧೭~।।}
\end{verse}

ಇವನು ಹತ್ತು ಬಾರಿ ಗಂಗಾಸ್ನಾನ ಮಾಡಿ ಅದರ ಪುಣ್ಯವನ್ನೆಲ್ಲ ದ್ರವ್ಯದ ಆಸೆಯಿಂದ ವೈಶ್ಯರಿಗೆ ಮಾರಿದನು. ನಂತರ ತೊಡೆಯಲ್ಲಿ ತಲೆಯಾಯಿತು.

\begin{verse}
\textbf{ಉರುದೇಶಶಿರೋಜಾತೋ ಡಾಕಿನೀ ಶಾಲ್ಮಲೀದ್ರುಮೇ~।}\\\textbf{ಮಯ್ಯೆವ ಸಾಕಮಭ್ಯಾಗಾಚ್ಚೋರೇಭ್ಯಶ್ಚ ಮೃತಿಂ ಗತಃ~।। ೧೮~।।}
\end{verse}

ಇವನೂ ಸಹ ನನ್ನ ಸಂಗಡಲೇ ಕಳ್ಳರಿಂದ ಕೊಲ್ಲಲ್ಪಟ್ಟ, ಈ ಶಾಲ್ಮಲೀ ಮರದಲ್ಲಿ ಡಾಕಿನೀ ಗಣಕ್ಕೆ ಸೇರಿದ ಪಿಶಾಚಿಯಾಗಿದ್ದಾನೆ.

\begin{verse}
\textbf{ತೃತೀಯೋsಯಂ ಭುಜಶಿರಾ ಪುರಾ ಪದ್ಮೇತಿ ಚಾಭಿಧಃ~।}\\\textbf{ಋಕ್ಷಗೋತ್ರ ಸಮುತ್ಪನ್ನೋ ವಾಜಿನಾಮ್ನಸ್ತನೂದ್ಭವಃ~।। ೧೯~।।}
\end{verse}

\begin{verse}
\textbf{ಭಾಗಿನೇಯೋ ಮಯಾ ಸಾರ್ಧಂ ಯಯೌ ಹೇಮಪುರೀಂ ದ್ವಿಜ~।}\\\textbf{ಶಿಷ್ಯಾನಧ್ಯಾಪಯಾಮಾಸ ದ್ವಾದಶಾಬ್ದಂ ಅತಂದ್ರಿತಃ~।। ೨೦~।।}
\end{verse}

ಭುಜದಲ್ಲಿ ತಲೆಯಿರುವ ಈ ಮೂರನೆಯವನು ಪೂರ್ವದಲ್ಲಿ ಪದ್ಮನೆಂಬ ಬ್ರಾಹ್ಮಣನು. ವಾಜಿಯೆಂಬುವನ ಮಗ, ನನಗೆ ಸೋದರಳಿಯ, ಋಕ್ಷಗೋತ್ರದವನು, ಅವನೂ ಸಹ ನನ್ನ ಸಂಗಡ ಹೇಮಪುರೀ ಗ್ರಾಮಕ್ಕೆ ಬಂದು ಹನ್ನೆರಡು ವರ್ಷಗಳ ಕಾಲ ಶಿಷ್ಯರಿಗೆ ವೇದಾಧ್ಯಯನವನ್ನು ಮಾಡಿಸಿದನು.

\begin{verse}
\textbf{ಕ್ರೀತಂ ತತ್ಪುಣ್ಯಮತುಲಂ, ಗೃಹೀತ್ಯಾ ತುಲ್ಯ ಮೌಲ್ಯ ಕಮ್~।}\\\textbf{ಧನಂ ಹೇಮಪುರೀಕ್ಷೇತ್ರೇ ಶಿಷ್ಯೇಭ್ಯೋ ಗುರುದಕ್ಷಿಣಾಮ್~।। ೨೧~।।} 
\end{verse}

\begin{verse}
\textbf{ಗೃಹೀತ್ವಾಪಿ ಮಯಾ ಸಾರ್ಧಂ ವೈ ಪ್ರಾಪ್ತೋ ಶಾಲ್ಮಲೀದ್ರುಮಮ್~।}\\\textbf{ಚೋರೇಭ್ಯಶ್ಚ ಮೃತಿಂ ಪ್ತಾಪ್ತೋ ಡಾಕಿನೀಭುಜಶೀರ್ಷಕಃ~।। ೨೨~।।}
\end{verse}

ಶಿಷ್ಯರಿಂದ ಗುರುದಕ್ಷಿಣೆಯನ್ನು ಸ್ವೀಕರಿಸಿದರೂ ಸಹ ವೇದಾಧ್ಯಾಯನ ಮಾಡಿಸಿದ ಅತುಲ ಪುಣ್ಯವನ್ನು ಹೇಮಪುರೀ ನಗರದ ಜನರಿಗೆ ಮಾರಿ ವಿಶೇಷ ಹಣ ಸಂಪಾದಿಸಿದನು. ಆ ಹಣವನ್ನೆಲ್ಲ ಸೇರಿಸಿಕೊಂಡು ಈ ಶಾಲ್ಮಲೀವೃಕ್ಷಕ್ಕೆ ನನ್ನ ಬಳಿಗೆ ಬಂದನು. ಕಳ್ಳರಿಂದ ಸಂಹರಿಸಲ್ಪಟ್ಟು ಡಾಕಿನೀ ಗಣಕ್ಕೆ ಸೇರಿದ ಪಿಶಾಚಿಯಾಗಿ ಭುಜದಲ್ಲಿ ತಲೆಯನ್ನು ಹೊಂದಿರುತ್ತಾನೆ.

\begin{verse}
\textbf{ಬಾಹುಜಾಯ ಯತಃ ಕ್ರೀತಂ ತಸ್ಮಾತ್ ಭುಜಶಿರಾ ಅಯಮ್~।}\\\textbf{ವಿದ್ಯೋಪಜೀವಿನೀ ವೃತ್ತಿರ್ಬಾಹ್ಮಣಾನಾಂ ಶ್ರುತೀರಿತಾ~।। ೨೩~।।}
\end{verse}

ಕ್ಷತ್ರಿಯರಿಗೆ ಮಾರಿದ್ದರಿಂದ ಭುಜದಲ್ಲಿ ತಲೆಯಾಯಿತು. ಬ್ರಾಹ್ಮಣರು ತಮ್ಮ ವಿದ್ಯೆಯಿಂದ ಉಪಜೀವನವನ್ನು ನಡೆಸಬೇಕೆಂದು ಶ್ರುತಿ ಹೇಳುತ್ತದೆ.

\begin{verse}
\textbf{ಅಯಾಚಿತೇಭ್ಯಃ ಶಿಷ್ಯೇಭ್ಯಃ ಶುಶ್ರೂಷುಭ್ಯಶ್ಚ ಮಾನದ~।}\\\textbf{ದಕ್ಷಿಣಾ ಗುರವೇ ದತ್ತಾ ಸಾ ವೃತ್ತಿಃ ಶ್ರುತಿಚೋದಿತಾ~।। ೨೪~।।}
\end{verse}

ಶಿಷ್ಯರು ಗುರುಗಳಿಂದ ಯಾಚಿತರಾಗದೇ ಕೇವಲ ಭಕ್ತಿಯಿಂದ ಸೇವಾ ರೂಪವಾಗಿ ಏನು ದಕ್ಷಿಣೆಯನ್ನು ಗುರುಗಳಿಗೆ ಅರ್ಪಿಸುತ್ತಾರೆಯೋ ಅದೇ “ಗುರು ದಕ್ಷಿಣೆ”, ಹೀಗೆಂದು ಶ್ರುತಿ ಹೇಳುತ್ತದೆ.

\begin{verse}
\textbf{ಬ್ರಹ್ಮ ಕ್ಷತ್ರಿಯವೈಶ್ಯೈಶ್ಚ ಧನಾಢ್ಯೈಶ್ಚ ಮಹಾತ್ಮಭಿಃ~।}\\\textbf{ವೇದಾಧ್ಯಯನಶೀಲೇಭ್ಯೋ ವ್ಯಾಖ್ಯಾನೇಭ್ಯೋ ಮುನೀಶ್ವರ~।। ೨೫~।।} 
\end{verse}

\begin{verse}
\textbf{ಷಷ್ಠಾಂಶೋ ದೇಯ ಇತ್ಯೇವಂ ಚೋದನಾ ಶ್ರುತಿಚೋದಿತಾ~।}\\\textbf{ಆರ್ಜಕಾನಾಂ ಚ ಷಷ್ಠಾಂಶೋ ವಿಪ್ರಾಣಾಂ ಸ್ವೀಯ ಉಚ್ಯತೇ~।। ೨೬~।।}
\end{verse}

ಐಶ್ವರ್ಯವಂತರಾದ ಬ್ರಾಹ್ಮಣ, ಕ್ಷತ್ರಿಯ, ವೈಶ್ಯ ಇವರು ವೇದಾಧ್ಯಯನದಲ್ಲಿ ನಿರತ\-ರಾಗಿರುವ ಅಥವಾ ವೇದವ್ಯಾಖ್ಯಾನವನ್ನು ಮಾಡುವ ಬ್ರಾಹ್ಮಣರಿಗೆ ತಮ್ಮ ಹಣದಲ್ಲಿನ ಆರನೇ ಒಂದು ಭಾಗವನ್ನು ದಾನಮಾಡಬೇಕೆಂದು ಶ್ರುತಿ ತಿಳಿಸುತ್ತದೆ. ಇದೇ ಜನರು ತಾವು ಸಂಪಾದಿಸುವ ಹಣದಲ್ಲಿ ಆರನೇ ಒಂದು ಭಾಗವನ್ನು ಬ್ರಾಹ್ಮಣರಿಗೆ ದಾನವಾಗಿ ಕೊಡಬೇಕೆಂದು ಶ್ರುತಿ ವಿಧಿಸುತ್ತದೆ.

\begin{verse}
\textbf{ಷಷ್ಠಾಂಶೋ ಬ್ರಾಹ್ಮಣಾನಾಂ ಚ ಕ್ರಿಯಾಯಾಂ ಧನಿನಾಂ ಭವೇತ್~।}\\\textbf{ಇತಿ ವಿದ್ಯೋಪಜೀವಿತ್ವಂ ವಿದ್ಧೀದಂ ಶ್ರುತಿಚೋದಿತಮ್~।। ೨೭~।।}
\end{verse}

ಧನಿಕರ ಹಣದ ಆರನೇ ಒಂದು ಭಾಗವು ಬ್ರಾಹ್ಮಣರ ಸತ್ಕರ್ಮಗಳಿಗಾಗಿ ಮೀಸಲು. ಬ್ರಾಹ್ಮಣರು ಈ ರೀತಿಯಿಂದ ತಮ್ಮ ಜೀವನವನ್ನು ನಿರ್ವಹಿಸಬೇಕೆಂದು ಶ್ರುತಿ ಹೇಳುತ್ತದೆ.

\begin{verse}
\textbf{ಕಾಲದ್ರವ್ಯಮಿತಿಂ ಕೃತ್ವಾ ಯೋsಧ್ಯಾ ಪಯತಿ ಮೂಢಧೀಃ~।}\\\textbf{ಶಿಷ್ಯಾನಧ್ಯಾಪಿತಾ ವೇದಾಸ್ತಥಾ ವೇದಾಂಗಪಂಚಕಮ್~।। ೨೮~।।}
\end{verse}

\begin{verse}
\textbf{ಸ ವೇದವಿಕ್ರಯೋ ನಾಮ ತೇನ ಪಾತಯತಿ ಧ್ರುವಮ್~।}\\\textbf{ಶತಂ ಕುಲಾನಿ ನರಕೇ ವಿಷ್ಠಾ ಕೂಪೇ ಚ ರೌರವೇ~।। ೨೯~।।}
\end{verse}

ಕಾಲ, ಹಣ, ಇವುಗಳನ್ನು ಮಿತಿಗೊಳಿಸಿ ಶಿಷ್ಯರಿಗೆ ವೇದಾಧ್ಯಯನ ಮಾಡಿಸುವ ಮೂರ್ಖ ಬ್ರಾಹ್ಮಣನು 'ವೇದವಿದ್ಯೆಯನ್ನು ಮಾರುವವನು' ಎಂಬುದಾಗಿ ಕರೆಯಲ್ಪಡುತ್ತಾನೆ. ಇಂತಹ ಪಾಪದಿಂದ ತನ್ನ ನೂರು ಕುಲದವರನ್ನು, 'ವಿಷ್ಠಾ ಕೂಪ' ಮತ್ತು 'ರೌರವ' ಎಂಬ ನರಕಗಳಲ್ಲಿ ಕೆಡವುತ್ತಾನೆ.

\begin{verse}
\textbf{ಬ್ರಹ್ಮಕಲ್ಪ ಸಹಸ್ರಾಣಿ ಸ್ವಯಂ ಭುಕ್ತ್ವಾ ಚ ಯಾತನಾಮ್~।}\\\textbf{ಸ ಡಾಕಿನೀ ಭವೇತ್ಪಾಪೀ ಮತಂ ವೇದವಿದಾಂತ್ವಿದಮ್~।। ೩೦~।।}
\end{verse}

ತಾನೂ ಸಹ ಒಂದು ಸಾವಿರ ಬ್ರಹ್ಮಕಲ್ಪದ ಪರ್ಯಂತ ನರಕ ಯಾತನೆಯನ್ನು ಅನುಭವಿಸಿ ನಂತರ ಪಾಪಿಯಾದ ಅವನು ಡಾಕಿನೀಗಣದಲ್ಲಿ ಪಿಶಾಚಿಯಾಗಿ ಹುಟ್ಟುತ್ತಾನೆ. ವೇದಗಳನ್ನು ಬಲ್ಲವರು ಈ ರೀತಿ ಅಭಿಪ್ರಾಯಪಡುತ್ತಾರೆ.

\begin{verse}
\textbf{ತಸ್ಮಾದಯಂ ಮಹಾಪಾಪೀ ಕ್ರೀತ್ವಾ ವೈ ಬಾಹುಜಾಯ ಚ~।}\\\textbf{ಗೃಹೀತ್ವಾ ಧನಮತ್ಯಂತಂ ಮಯಾ ಸಾಕಮಗಾದ್ದ್ರುಮಮ್~।। ೩೧~।। }
\end{verse}

\begin{verse}
\textbf{ಚೋರೇಭ್ಯಶ್ಚ ಮೃತಿಂ ಪ್ರಾಪ್ತೋ ಪಾಪೀ ತೇನೈವ ಕರ್ಮಣಾ~।}\\\textbf{ಅಯಂ ಜಾತೋ ಭುಜಶಿರಾ ಡಾಕಿನೀ ಶಾಲ್ಮಲೀದ್ರುಮೇ~।। ೩೨~।।}
\end{verse}

ಮಹಾಪಾಪಿಯಾದ ಇವನು ಕ್ಷತ್ರಿಯರಿಗೆ ವೇದವಿಕ್ರಯಮಾಡಿ ವಿಪುಲವಾದ ಹಣದೊಡನೆ ನನ್ನ ಸಹಿತ ಈ ವೃಕ್ಷಕ್ಕೆ ಬಂದನು. ಕಳ್ಳರಿಂದ ಕೊಲ್ಲಲ್ಪಟ್ಟ ಇವನು ತನ್ನ ಪಾಪದ ಫಲವಾಗಿ ಡಾಕಿನೀ ಗಣಕ್ಕೆ ಸೇರಿದ ಪಿಶಾಚಿಯಾಗಿ ಭುಜಶಿರಾ ಎಂಬ ಹೆಸರಿನಿಂದ ಈ ಶಾಲ್ಮಲೀ ವೃಕ್ಷದಲ್ಲಿ ಇರುತ್ತಾನೆ.

\begin{verse}
\textbf{ತುರ್ಯಶ್ಚ ಪೈತೃಷ್ವ ಸ್ರೇಯೋ ಮದ್ಗ್ರಾಮೀಣೋ ದುರಾತ್ಮವಾನ್~।}\\\textbf{ಭಾರದ್ವಾಜಕುಲೋತ್ಪನ್ನಃ ಸುಚರಿತ್ರಸುತೋ ಬಲಃ~।। ೩೩~।।}
\end{verse}

ನಾಲ್ಕನೆಯವನು ನನ್ನ ಸೋದರತ್ತೆಯ ಮಗ ಸುಚರಿತ್ರನ ಮಗ, ಭಾರದ್ವಾಜ ಗೋತ್ರದವನು, ಬಲಿಷ್ಠ, ಕೆಟ್ಟ ಬುದ್ಧಿಯುಳ್ಳವನು. ನನ್ನ ಊರಿನಲ್ಲಿಯೇ ಇದ್ದವನು.

\begin{verse}
\textbf{ನಾಮ್ನಾ ದರಿದ್ರಃ ಪಾಪಾತ್ಮಾ ಬಹುಶಾಸ್ತ್ರವಿಶಾರದಃ~।}\\\textbf{ಮಯಾ ಹೇಮಪುರೀಂ ಪ್ರಾಪ್ತೋ ವೈಶ್ಯೇಭ್ಯೋ ದತ್ತಮೌಲ್ಯಕಃ~।। ೩೪~।।} 
\end{verse}

\begin{verse}
\textbf{ತೇಭ್ಯ ಏವ ಪ್ರತಿಗೃಹ್ಯ ಹೈಭವತ್ ಕುಲಪಾಂಸನಃ~।}\\\textbf{ಯಜ್ಞಾನ್ ಬಹುವಿಧಾನ್ ಕ್ರೀತ್ವಾ ಪ್ರತಿಜ್ಞಾ ತಂ ಚ ಮೌಲ್ಯಕಮ್~।।೩೫।।}
\end{verse}

ಅವನ ಹೆಸರು ದರಿದ್ರ, ಪಾಪಿಷ್ಠ, ಅನೇಕ ಶಾಸ್ತ್ರಗಳನ್ನು ಬಲ್ಲವನು. ನನ್ನ ಜತೆಯಲ್ಲಿ ಹೇಮಪುರೀ ಗ್ರಾಮಕ್ಕೆ ಬಂದು ತನ್ನ ವೇದವಿದ್ಯೆಯನ್ನೆಲ್ಲ ವೈಶ್ಯರಿಗೆ ಮಾರಿ ಬಹಳ ಹಣ ಸಂಪಾದಿಸಿದನು. ಅನೇಕ ಯಜ್ಞಗಳನ್ನು ಆಚರಿಸಿ ಅವುಗಳ ಫಲಗಳನ್ನು ವೈಶ್ಯರಿಗೆ ಮಾರಿ ದ್ರವ್ಯ ಗಳಿಸಿದನು.

\begin{verse}
\textbf{ಪ್ರತಿಗೃಹ್ಯ ಮುನಿಕ್ರೀತಂ ತತ್ಪುಣ್ಯಂ ಬಹುಶೋಽಮುನಾ~।}\\\textbf{ಊರುಜೇಭ್ಯೋsಪಿ ಯುದ್ಧತ್ತಂ ಪೂರ್ವಪಾಪವಶಾದಪಿ~।। ೩೬~।।}
\end{verse}

\begin{verse}
\textbf{ಯತೋ ಲಬ್ಧಾ ಯಶಸ್ತಸ್ಮಾತ್ ಕಟಿದೇಶಶಿರಾ ಅಯಮ್~।}\\\textbf{ಶಾಲ್ಮಲೀಂ ಪ್ರಾಪ್ಯ ಚಾಸ್ಮಾಭಿಃ ಚೋರೇಭ್ಯಶ್ಚ ಮೃತಿಂ ಗತಃ~।। ೩೭~।।}
\end{verse}

ಪೂರ್ವಪಾಪಕರ್ಮ ನಿಮಿತ್ತದಿಂದ ವೈಶ್ಯರಿಗೆ ವೇದವಿಕ್ರಯಮಾಡಿ ಹಣವನ್ನು ಸಂಪಾದಿಸಿದ ಇವನು ಸೊಂಟದಲ್ಲಿ ತಲೆಯುಳ್ಳವನಾದನು. ನಮ್ಮ ಸಂಗಡ ಶಾಲ್ಮಲೀ ವೃಕ್ಷಕ್ಕೆ ಬಂದಾಗ ಕಳ್ಳರಿಂದ ಕೊಲ್ಲಲ್ಪಟ್ಟನು.

\begin{verse}
\textbf{ಪಂಚಮೋ ಮಾತೃಷ್ವಸ್ತ್ರೀಯಸ್ತಥಾ ಸಂಕೃತಿಗೋತ್ರಜಃ~।}\\\textbf{ಅನಘಸ್ಯ ತಥಾ ಪುತ್ರೋ ನಾಮ್ನಾ ಸಂಕೃತಿರಿತ್ಯಪಿ~।। ೩೮~।।}
\end{verse}

ಐದನೆಯವನು ನನ್ನ ಚಿಕ್ಕಮ್ಮನ ಮಗ, ಸಂಕೃತಿ ಗೋತ್ರದವನು, ತಂದೆಯ ಹೆಸರು ಅನಘ, ಅವನ ಹೆಸರು ಸಂಕೃತಿ.

\begin{verse}
\textbf{ಪ್ರಾಪ್ಯ ಹೇಮಪುರೀಮೇಭಿಃ ಸಾಹಾಯ್ಯಂ ಧಾನಕಾತರಃ~।}\\\textbf{ತುಲಾಮಾಸೇ ಮೇಷಮಾಸೇ ಮಾಘಮಾಸೇ ದ್ವಿಜಾಧಮಃ~।। ೩೯~।। }
\end{verse}

\begin{verse}
\textbf{ಪ್ರಾತಃಸ್ನಾನಂ ತಥಾ ಸಮ್ಯಕ್ ಕೃತ್ವಾಯಂ ಪ್ರತಿವಾರ್ಷಿಕಮ್~।}\\\textbf{ತಜ್ಜಾತಮತುಲಂ ಪುಣ್ಯಂ ಕುಲಕೋಟ್ಯುದ್ಧ ರಕ್ಷಮಮ್~।। ೪೦~।। }
\end{verse}

\begin{verse}
\textbf{ಕ್ರೀತ್ವಾ ಪ್ರಭುಭ್ಯಸ್ತೇಭ್ಯಶ್ಚ ಗೃಹೀತಂ ಭೂರಿ ವೈ ಧನಮ್~।}\\\textbf{ಧಾರಾಪೂರ್ವಂ ಯತೋ ಹಸ್ತಾತ್ ಗೃಹೀತಂ ಪಾಪ್ಮನಾಮುನಾ~।। ೪೧~।।}
\end{verse}

ಇವನು ಹೇಮಪುರೀ ಊರಿಗೆ ಇತರರಿಂದ ಸಹಿತನಾಗಿ ಬಂದನು. ಹಣವನ್ನು ಸಂಪಾದಿಸಲು ಕಾತುರನಾಗಿದ್ದನು. ಪ್ರತಿ ವರ್ಷವೂ ತುಲಾಮಾಸದಲ್ಲಿ, (ಕಾರ್ತಿಕ ಮಾಸದಲ್ಲಿ) ಮೇಷಮಾಸದಲ್ಲಿ (ವೈಶಾಖದಲ್ಲಿ), ಮಾಘಮಾಸದಲ್ಲಿ ಪ್ರಾತಃ ಸ್ನಾನವನ್ನು ವಿಧಿಪೂರ್ವಕವಾಗಿ ಮಾಡುತ್ತಿದ್ದನು. ಒಂದು ಕೋಟಿ ಕುಲವನ್ನು, ರಕಿಸುವಂತಹ ಆ ಪುಣ್ಯರಾಶಿಯನ್ನು ಹಣವಂತರಿಗೆ ದ್ರವ್ಯದ ಆಶೆಯಿಂದ ಜಲಧಾರಾ ಪೂರ್ವಕವಾಗಿ ಮಾರಿ ತುಂಬ ಹಣವನ್ನು ಸಂಪಾದಿಸಿದನು.

\begin{verse}
\textbf{ತತಃ ಪ್ರಕೋಷ್ಠಮೂರ್ಧಾಯಂ ಡಾಕಿನೀ ಶಾಲ್ಮಲೀದ್ರುಮೇ~।}\\\textbf{ಅಸ್ಮಾಭಿಃ ಪ್ರೇತ್ಯ ತನ್ಮೂಲಂ ಚೋರೇಭ್ಯೋ ನಿಧನಂ ಗತಃ~।। ೪೨~।। }
\end{verse}

ಶಾಲ್ಮಲೀ ವೃಕ್ಷದಲ್ಲಿ ನಮ್ಮ ಸಂಗಡ ಬಂದ ಇವನು ಕಳ್ಳರಿಂದ ಹತನಾಗಿ ಮೊಳಕೈ ಕೆಳಗೆ ತಲೆಯನ್ನು ಹೊಂದಿ ಡಾಕಿನೀ ಗಣದಲ್ಲಿ ಪಿಶಾಚಿಯಾದನು.

\begin{verse}
\textbf{ಷಷ್ಠೋಽಯಂ ಮಮ ದೌಹಿತ್ರಸ್ತಥಾ ಮೌದ್ಗಲ್ಯಗೋತ್ರಜಃ~।}\\\textbf{ಸನಾಮಕೋ ಮಹಾಪಾಪೀ ಪ್ರಾಪ್ತೋ ಹೇಮಪುರೀಮಿಮಾಮ್~।।೪೩~।। }
\end{verse}

ಆರನೆಯವನು ನನ್ನ ಮಗಳ ಮಗ, ಮೌದ್ಗಲ್ಯ ಗೋತ್ರದವನು ಹೆಸರೂ ಸಹ ಮೌದ್ಗಲ್ಯ, ಮಹಾಪಾಪಿ. ಇದೇ ಹೇಮಪುರಿಗೆ ಬಂದನು.

\begin{verse}
\textbf{ಅಪುತ್ರಾಯ ತ್ರಯಃ ಪುತ್ರಾಃ ಕ್ರೀತಾಃ ಪಾಪೇನ ಚೇತಸಾ~।}\\\textbf{ಗುಣಾಢ್ಯಾಃ ಶ್ರೋತ್ರಿಯಾಃ ಪುಣ್ಯಾಃ ಕುಲೋದ್ಧ ರಣಹೇತವಃ~।। ೪೪~।।} 
\end{verse}

\begin{verse}
\textbf{ಕ್ರಿತಾಃ ಯೇನೌರಸಾಃ ಪುತ್ರಾಸ್ತೇನ ವೈ ಸ್ತನಮೂರ್ಧಜಃ~।}\\\textbf{ಕ್ರಯಾಲ್ಲಬ್ಧ ಧನಂ ಭೂರಿ ಗೃಹೀತ್ವಾ ಸ್ಮಾಭಿರೇವ ಚ~।। ೪೫~।। }
\end{verse}

\begin{verse}
\textbf{ಚೋರೇಭ್ಯೋ ಮರಣಂ ಪ್ರಾಪ್ಯ ಶಾಲ್ಮಲ್ಯಾಂ ಡಾಕಿನೀಪತಿಃ~।}
\end{verse}

ಈ ಪಾಪಾತ್ಮನು ಹಣದ ಆಸೆಯಿಂದ ಗುಣವಂತರಾದ, ವೇದಾಧ್ಯಯನ ಮಾಡಿದ ಪುಣ್ಯವಂತರಾದ, ಕುಲವನ್ನು ಉದ್ಧರಿಸುವ ತನ್ನ ಮೂರು ಜನ ಗಂಡು ಮಕ್ಕಳನ್ನು ಮಾರಿ, ಹಣವನ್ನು ತೆಗೆದುಕೊಂಡು ನಮ್ಮ ಜತೆಯಲ್ಲಿ ಈ ಶಾಲ್ಮಲೀ ವೃಕ್ಷದಲ್ಲಿ ಬಂದಾಗ ಕಳ್ಳರಿಂದ ಹತನಾಗಿ, ಸ್ತನದಲ್ಲಿ ತಲೆಯನ್ನು ಹೊಂದಿ, ಡಾಕಿನೀ ಗಣಕ್ಕೆ ಅಧಿಪತಿಯಾದ ಪಿಶಾಚಿಯಾದನು.

\begin{verse}
\textbf{ಸಖಾಯಂ ಸಪ್ತಮೋ ನಾಮ್ನಾ ಕುಶಃ ಶ‍್ರೀವತ್ಸಗೋತ್ರಜಃ~।। ೪೬~।। }
\end{verse}

\begin{verse}
\textbf{ವಾಚಸ್ಪತೇಸ್ತಥಾ ಪುತ್ರಃ ಪ್ರಾಪ್ತೋ ಹೇಮಪುರೀಮಿಮಾಮ್~।}\\\textbf{ಅನಾಥಪ್ರೇತಸಂಸ್ಕಾರಂ ಪರೇಭ್ಯೋ ಮೋಕ್ಷಮಾಪ್ಯ ಚ~।। ೪೭~।। }
\end{verse}

ಏಳನೆಯವನು ನನ್ನ ಮಿತ್ರ ಕುಶನೆಂದು ಹೆಸರು, ಶ‍್ರೀವತ್ಸ ಗೋತ್ರದವನು, ವಾಚಸ್ಪತಿಯೆಂಬುವನ ಮಗ. ಈ ಹೇಮಪುರಿಗೆ ಬಂದನು. ಅನಾಥ ಪ್ರೇತ ಸಂಸ್ಕಾರವನ್ನು ಮಾಡುತ್ತಿದ್ದನು.

\begin{verse}
\textbf{ಕೃತಂ ದ್ವಾದಶವರ್ಷೇಷು ಪ್ರತ್ಯಕ್ಷಂ ಪಾಪಚೇತಸಾ~।}\\\textbf{ಅನಾಥಪ್ರೇತಸಂಸ್ಕಾರಾತ್ ಕೋಟಿಯಜ್ಞಫಲಂ ಲಭೇತ್~।। ೪೮~।।}
\end{verse}

ಹನ್ನೆರಡು ವರ್ಷಗಳ ಕಾಲ ಅನಾಥಪ್ರೇತ ಸಂಸ್ಕಾರವನ್ನು ಈ ಪಾಪಿಯು ಆಚರಿಸಿ ಕೋಟಿಯಜ್ಞಗಳಿಂದ ಬರುವ ಪುಣ್ಯವನ್ನು ಸಂಪಾದಿಸಿದನು.

\begin{verse}
\textbf{ಇತಿ ಜ್ಞಾತ್ವಾಪಿ ತಪ್ಪುಣ್ಯಂ ಮೂರ್ಧ್ನಿ ಪಾದೌ ನ್ಯಧಾಯ ಚ~।}\\\textbf{ಕ್ಷತ್ರಿಯೇಭ್ಯೋ ದದೌ ಪುಣ್ಯಂ ಗೃಹೀತ್ವಾ ಮೌಲ್ಯಮೇವ ಚ~।। ೪೯~।। }
\end{verse}

ಈ ವಿಷಯ ತಿಳಿದಿದ್ದರೂ ಆ ಪುಣ್ಯ ರಾಶಿಯನ್ನು ಕ್ಷತ್ರಿಯರಿಗೆ ಮಾರಿ ಬಹಳ ಹಣವನ್ನು ಸಂಪಾದಿಸಿದನು.

\begin{verse}
\textbf{ತಸ್ಮಾದಯಂ ಪಾಣಿಮೂರ್ಧಾ ಸೋಽಪ್ಯಸ್ನಾಭಿಃ ಸಹಾಯಯೌ~।}\\\textbf{ಪಂಚತ್ವಮೇತ್ಯ ಚೋರೇಭ್ಯೋ ಡಾಕಿಣೀ ಶಾಲ್ಮಲೀದ್ರುಮೇ~।। ೫೦~।। }
\end{verse}

ನಮ್ಮ ಜತೆಯಲ್ಲಿಯೇ ಈ ಶಾಲ್ಮಲೀವೃಕ್ಷಕ್ಕೆ ಬಂದ ಇವನು ಕಳ್ಳರಿಂದ ಹತನಾಗಿ ಇದೇ ಮರದಲ್ಲಿಯೇ ಡಾಕಿಣೀ ಗಣದ ಪಿಶಾಚಿಯಾಗಿದ್ದಾನೆ. ಹಸ್ತದಲ್ಲಿ ತಲೆ ಇದೆ.

\begin{verse}
\textbf{ದತ್ವಾ ಮೌಲ್ಯಾತ್ಪುರೋಕ್ತೇಭ್ಯಃ ಪಾಪೇಭ್ಯಃ ಪುಣ್ಯಮುತ್ತಮಮ್~।}\\\textbf{ಪೂರ್ವಸ್ಮಾತ್ ಕೋಟಿಗುಣಿತಂ ಭವತ್ಯೇವ ನ ಸಂಶಯಃ~।। ೫೧~।।}
\end{verse}

ಪಾಪಿಷ್ಠರಾದ ಜನರಿಗೆ ಉತ್ತಮವಾದ ಪುಣ್ಯವನ್ನು ಹಣಕ್ಕಾಗಿ ಮಾರಿದರೆ ಅನಂತಕೋಟಿ ಪಾಪವು ಲಭಿಸುತ್ತದೆ, ಸಂಶಯವಿಲ್ಲ.

\begin{verse}
\textbf{ಮೌಲ್ಯ ಮಾದಾಯ ಯದ್ದತ್ತಂ ಪುಣ್ಯಂ ಪಾಪೇನ ಚೇತಸಾ~।}\\\textbf{ಅಸ್ಮಾಕಂ ಗತಿಮಾಪ್ನೋತಿ ನಾತ್ರ ಕಾರ್ಯಾ ವಿಚಾರಣಾ~।। ೫೨~।। }
\end{verse}

ಪಾಪಾತ್ಮರು ಹಣದ ಆಸೆಯಿಂದ ತಾವು ಗಳಿಸಿದ ಉತ್ತಮ ಪುಣ್ಯವನ್ನು ಮಾರಿದರೆ ನಮಗೆಲ್ಲ ಬಂದಿರುವ ಗತಿಯೇ ಬರುತ್ತದೆ, ಸಂಶಯವಿಲ್ಲ.

\begin{verse}
\textbf{ಪರೋಪಕಾರಃ ಪುಣ್ಯಾಯ ಪಾಪಾಯ ಪರಪೀಡನಮ್~।}\\\textbf{ಪುರಾ ಕೈಲಾಸಶಿಖರೇ ಪಾರ್ವತ್ಯೈಶಂಕರೋಽವದತ್~।। ೫೩~।। }
\end{verse}

ಪರೋಪಕಾರದಿಂದ ಪುಣ್ಯವೂ, ಪರಪೀಡನೆಯಿಂದ ಪಾಪವೂ ಲಭಿಸುತ್ತವೆಯೆಂದು ಹಿಂದೆ ಕೈಲಾಸದಲ್ಲಿ ಪಾರ್ವತೀದೇವಿಯನ್ನು ಕುರಿತು ರುದ್ರದೇವರು ಹೇಳಿದರು.

\begin{verse}
\textbf{ದುಃಸಹಂ ಕಿಂ ನ ಸಾಧೂನಾಂ ಕಿಮಕಾರ್ಯಂ ದುರಾತ್ಮನಾಮ್~।}\\\textbf{ಏತತ್ಸರ್ವಂ ಸಮಾಖ್ಯಾತಂ ಅಸ್ಮದ್ವೃತ್ತಾಂತಮುತ್ತಮಮ್~।। ೫೪~।। }
\end{verse}

\begin{verse}
\textbf{ತ್ವಯಾಽದ್ಯ ಸುಖಿನೋ ಜಾತಾ ಯತಃ ತಸ್ಮಾತ್ ಸಮುದ್ಧರ~।}\\\textbf{ಇತಿ ವಿಜ್ಞಾಪ್ಯ ತೇ ತೂಷ್ಣೀಂ ಬಭೂವುರ್ಮುನಿಪುತ್ರ ಕಮ್~।। ೫೫~।।}
\end{verse}

ಸಜ್ಜನರಿಗೆ ಅಸಾಧ್ಯವಾದುದು ಯಾವುದೂ ಇಲ್ಲ. ಪಾಪಿಷ್ಠರಿಗೆ ಸತ್ಕರ್ಮ ಯಾವುದು, ದುಷ್ಕರ್ಮ ಯಾವುದು ಎಂಬ ಜ್ಞಾನವಿಲ್ಲ. ನಮ್ಮ ಹಿಂದಿನ ಜನ್ಮದ ವೃತ್ತಾಂತವನ್ನೆಲ್ಲ ನಿನಗೆ ತಿಳಿಸಿರುತ್ತೇವೆ. ನಿನ್ನ ದರ್ಶನಲಾಭದಿಂದ ನಾವು ಸುಖಿಗಳಾದೆವು; ನಮ್ಮನ್ನು ಉದ್ಧರಿಸು. ಈ ರೀತಿ ಹೇಳಿ ಆ ಡಾಕಿಣೀಗಣಕ್ಕೆ ಸೇರಿದ ಪಿಶಾಚಿಗಳು ಸುಮ್ಮನಾದವು.

\begin{center}
ಇತಿ ಶ‍್ರೀ ವಾಯುಪುರಾಣೇ ಮಾಘಮಾಸಮಾಹಾತ್ಮ್ಯೇ ತ್ರಯೋದಶೋsಧ್ಯಾಯಃ 
\end{center}

\begin{center}
ಶ‍್ರೀ ವಾಯುಪುರಾಣಾಂತರ್ಗತ ಮಾಘಮಾಸ ಮಹಾತ್ಮ್ಯೆಯಲ್ಲಿ\\ ಹದಿಮೂರನೇ ಅಧ್ಯಾಯವು ಸಮಾಪ್ತಿಯಾಯಿತು.
\end{center}

\newpage

\section*{ಅಧ್ಯಾಯ\enginline{-}೧೪}

\emptypage

\begin{flushleft}
\textbf{ಉದಗ್ದಿಗಸ್ಥಾ ಊಚುಃ\enginline{-}}
\end{flushleft}

ಉತ್ತರ ದಿಕ್ಕಿನಲ್ಲಿದ್ದ ಶಾಕಿನೀಗಣಕ್ಕೆ ಸೇರಿದ ಪಿಶಾಚಿಗಳು ಮಾತನಾಡಿದುವು.

\begin{flushleft}
\textbf{ವೃಕಾನನ ಉವಾಚ\enginline{-}}
\end{flushleft}

ಮೊದಲು ವೃ ಕಾನನನೆಂಬ ಪಿಶಾಚಿಯು ಹೇಳಿತು:-

\begin{verse}
\textbf{ಪುರಾಽಹಂ ತುದಿಲೋ ನಾಮ ಗಂಧರ್ವೋ ದೇವಗಾಯಕಃ~।}\\\textbf{ಆಗಾದ್ದೇವೇಂದ್ರಸದನೇ ಕದಾಚಿತ್ಸಾನುಗೋ ಮುನೇ~।। ೧~।। }
\end{verse}

ಮುನಿಯೇ! ಹಿಂದೆ ನಾನು ತುದಿಲ ಎಂಬ ಹೆಸರಿನ ಗಂಧರ್ವ, ದೇವಲೋಕದಲ್ಲಿ ಗಾಯಕ. ಒಂದು ಬಾರಿ ದೇವೇಂದ್ರನ ಸಭೆಗೆ ಎಲ್ಲರಿಂದ ಕೂಡಿಕೊಂಡು ಹೋದೆ.

\begin{verse}
\textbf{ಶಂಬರಾರೇಃ ಸಭಾಯಾಂ ತೇ ನಿವಿಷ್ಟಾ ವಿಜರಾ ಇಮೇ~।}\\\textbf{ವಸವೋಪ್ಟೌ ದ್ವಾದಶಾರ್ಕಾ ಅಷ್ಟೌ ದಿಕ್ಪಾಲಕಾಸ್ತಥಾ~।। ೨~।। }
\end{verse}

ದೇವೇಂದ್ರನ ಸಭೆಯಲ್ಲಿ ಜರಾ (ಮುಪ್ಪು) ರಹಿತರಾದ ಎಂಟು ಜನ ವಸುಗಳು, ಹನ್ನೆರಡು ಜನ ಸೂರ್ಯರು, ಎಂಟು ದಿಕ್ಪಾಲಕರು ಇದ್ದರು.

\begin{verse}
\textbf{ಏಕಾದಶ ತಥಾ ರುದ್ರಾ ಮನವಶ್ಚ ಚತುರ್ದಶ~।}\\\textbf{ದೇವರ್ಷಯೋ ನಾರದಾದ್ಯಾಃ ತಥಾ ಸಪ್ತರ್ಷಯೋ ಮುನೇ~।। ೩~।। }
\end{verse}

\begin{verse}
\textbf{ಯಕ್ಷಾಃ ಕಿಂಪುರುಷಾ ನಾಗಾಃ ಕರ್ಮಜಾಃ ಪಿತರಸ್ತಥಾ~।}\\\textbf{ವಿದ್ಯಾಧರಾಃ ಕಿನ್ನರಾಶ್ಚ ಭೂಮಿಪಾ ಬಹವೋsಪಿ ಚ~।। ೪~।। }
\end{verse}

ಹನ್ನೊಂದು ಜನ ರುದ್ರರು, ಹದಿನಾಲ್ಕು ಮನುಗಳು, ಕರ್ಮಜದೇವತೆಗಳು, ಪಿತೃ ದೇವತೆಗಳು, ನಾರದರೇ ಮೊದಲಾದ ದೇವಷಿ೯ಗಳು, ಸಪ್ತ ಋಷಿಗಳು, ಯಕ್ಷರು, ಕಿಂಪುರುಷರು, ವಿದ್ಯಾಧರರು, ಕಿನ್ನರರು, ಚಕ್ರವರ್ತಿಗಳು ಹೀಗೆ ಬಹಳ ಜನರು ಇದ್ದರು.

\begin{verse}
\textbf{ಪ್ರಸಂಗಾ ಬಹವೋ ಜಾತಾ ನಾನಾಖ್ಯಾನಪುರಃಸರಾಃ~।}\\\textbf{ತದಾ ಪ್ರಸಂಗಾ ಬಹವಃ ಕಲಾವಿದ್ಯಾಸು ವೈ ಕೃತಾಃ~।। ೫~।।} 
\end{verse}

ಆ ಕಾಲದಲ್ಲಿ ಸಭೆಯಲ್ಲಿ ನಾನಾ ವಿಷಯದ ಬಗ್ಗೆ ವಾದ-ವಿವಾದಗಳೂ, ನಾನಾ ಕಲೆಗಳ ಪ್ರದರ್ಶನಗಳೂ ಜರುಗಿದುವು.

\begin{verse}
\textbf{ಪ್ರತ್ಯವಸ್ಥಾಯಿ ಮುನಿನಾ ಮತಂಗೇನ ಮಹಾತ್ಮನಾ~।}\\\textbf{ಸ ನಿಂದಿತೋ ಮಯಾ ದುಷ್ಟಚಿತ್ತೇನಾತ್ಮಪ್ರಶಂಸಿನಾ~।। ೬~।। }
\end{verse}

ಮಹಾತ್ಮರಾದ ಮತಂಗಋಷಿಗಳು ತಮ್ಮ ವಾದಸರಣಿಯನ್ನು ಮುಂದಿಟ್ಟರು. ಆತ್ಮಪ್ರಶಂಸ\-ನಾಕನಾದ ನಾನು ಕೆಟ್ಟ ಬುದ್ಧಿಯಿಂದ ಅವರನ್ನು ನಿಂದಿಸಿದೆ.

\begin{verse}
\textbf{ಕ್ವ ಪಲಾಶಾಶನಸ್ತ್ವಂ ಚ ಕ್ವ ಚಾಲಾಪಾಃ ಕಲಾಃ ಶುಭಾಃ~।}\\\textbf{ಕ್ವ ಶ್ಯಾಮಾ ತರುಣೀ ಪ್ರೌಢಾ ವೃದ್ಧೋ ವಿದಶನಃ ಕ್ವಚಿತ್~।। ೭~।। }
\end{verse}

ಮತಂಗನೇ, ಹಣ್ಣೆಲೆಗಳನ್ನು ತಿಂದು ಬದುಕಿರುವ ನೀನೆಲ್ಲಿ? ರಮಣೀಯ ವಾದ ಕಲೆಗಳಿಗೆ ಸಂಬಂಧಪಟ್ಟ ವಾದ-ವಿವಾದಗಳೆಲ್ಲಿ? ಸುಂದರಳಾದ ತರುಣಿಯೆಲ್ಲಿ? ಹಲ್ಲುಗಳಿಲ್ಲದ ವೃದ್ಧನಾದ ನೀನೆಲ್ಲಿ?

\begin{verse}
\textbf{ಕ್ವ ವಾಽರಸಿಕಹೃದ್ಯಾಸ್ತೇ ಕೇಯಮಾನ್ವೀಕ್ಷಿಕೀ ಕಲಾ~।}\\\textbf{ಕ್ವ ವಾ ಮುಕ್ತಾಮಯೀ ಮಾಲಾ ಕ್ವ ವಾ ಶಾಖಾಮೃಗಾಶ್ಚಲಾಃ~।। ೮~।। }
\end{verse}

ರಸಿಕತೆಯೇ ಇಲ್ಲದ ನೀನೆಲ್ಲಿ, ಕಲೆಗಳ ವಿಚಾರವೆಲ್ಲಿ? ಮುತ್ತಿನಹಾರಗಳೆಲ್ಲಿ? ಮರಗಳಲ್ಲಿ ವಾಸಿಸುವ ಕಪಿಗಳೆಲ್ಲಿ?~।

\begin{verse}
\textbf{ಜರಾಕರ್ಕಶದುಷ್ಟಾಂಗಃ ಸ್ವಯಮೇಕೋ ಗುಹಾಶಯಃ~।}\\\textbf{ಸುಶಿಕ್ಷಿತಃ ಸದಾಭ್ಯಾಸೀ ಸದಾ ಲಕ್ಷ್ಯ ಪರೀಕ್ಷಕಃ~।। ೯~।। }
\end{verse}

\begin{verse}
\textbf{ಲಕ್ಷ್ಯ ಲಕ್ಷಣಸಂವೇತ್ತಾ ತಾದೃಶೋsಹಂ ಸಭಾಂತರೇ~।}\\\textbf{ಅನಾತ್ಮಜ್ಞೇನ ಮುನಿನಾ ಪ್ರತ್ಯವಸ್ಥಾಯಿ ಕೋಪನಃ~।। ೧೦~।।}
\end{verse}

ಮುಪ್ಪಿನಿಂದ ಕೃಶವಾದ ಅವಯವಗಳನ್ನುಳ್ಳ, ಗುಹೆಯಲ್ಲಿ ಒಂಟಿಯಾಗಿ ವಾಸಮಾಡುವ ನೀನು, ಚೆನ್ನಾಗಿ ಶಿಕ್ಷಿತನಾದ, ನಿತ್ಯವೂ ಕಲೆಗಳನ್ನು ಅಭ್ಯಾಸ ಮಾಡುತ್ತಿರುವ, ಸರ್ವಲಕ್ಷಣದಿಂದ ಯುಕ್ತನಾದ ನನ್ನನ್ನು, ಪರೀಕ್ಷಿಸುವುದೇ? ಲಕ್ಷ್ಯ ಯಾವುದು, ಲಕ್ಷಣ ಯಾವುದು ಎಂಬುದನ್ನು ಚೆನ್ನಾಗಿ ಬಲ್ಲ ನನ್ನನ್ನು ಈ ಸಭೆಯಲ್ಲಿ ಅನಾತ್ಮಜ್ಞನಾದ ನೀನು ಏನೆಂದು ಪರೀಕ್ಷಿಸುವಿ?

\begin{verse}
\textbf{ತದಾಽಸಾವಶಪನ್ಮೂಢಮಹಂಕಾರೇಣ ದೂಷಿತಮ್~।}\\\textbf{ಅತ್ಯಹಂಕಾರಯುಕ್ತ ತ್ವಾತ್ ತ್ವಂ ಭೂಮೌ ಕ್ಷತ್ರಿಯೋ ಭವ~।। ೧೧~।।} 
\end{verse}

\begin{verse}
\textbf{ಇತಿ ಶಪ್ತತಸ್ತತಸ್ತೇನ ಮುನಿನಾಽತಿದಯಾಲುನಾ~।}\\\textbf{ಭೋಜವಂಶೇ ತ್ವಹಂ ನಾಮ್ನಾ ಶ್ರುತಿಕೀರ್ತಿರಿತಿ ಕ್ಷಣಾತ್~।। ೧೨~।।} 
\end{verse}

ಅತ್ಯಂತ ಅಹಂಕಾರದಿಂದ ಮಾತನಾಡಿದ ನನಗೆ ದಯಾಳುಗಳಾದ ಮತಂಗ ಮುನಿಗಳು ಶಾಪಕೊಟ್ಟರು- "ಮೂಢನೇ, ಅಹಂಕಾರದಿಂದ ಮಾತನಾಡಿದ ನೀನು ಭೂಮಿಯಲ್ಲಿ ಕ್ಷತ್ರಿಯನಾಗಿ ಹುಟ್ಟು”, ಹೀಗೆ ಶಾಪಗಸ್ತನಾದ ನಾನು ಭೋಜವಂಶದಲ್ಲಿ ಶ್ರುತಿಕೀರ್ತಿಯೆಂಬ ಹೆಸರಿನಿಂದ ಜನಿಸಿದೆ.

\begin{verse}
\textbf{ಬಲೀ ಸುರಾಪೋ ವಿದ್ಯಾಢ್ಯೋ ಜಾತೋ ರುಕ್ಮರಥಾತ್ಮಜಃ~।}\\\textbf{ಬ್ರಾಹ್ಮಣಾನ್ ನಿಷ್ಠುರಂ ವಾದೀ ಸದಾಽಹಂಕಾರದೂಷಿತಃ~।। ೧೩~।। }
\end{verse}

\begin{verse}
\textbf{ತಾದೃಶೇನ ಮಯಾ ಕ್ಷೌರಕರ್ಮ ವಿಷ್ಣೋರ್ದಿನತ್ರಯೇ~।}\\\textbf{ಕೃತಂ ನಿಷಿದ್ಧ ದಿವಸೇ ವಾರೇ ಋಷ್ಯಾದ್ಯವಜ್ಞಯಾ~।। ೧೪~।।}
\end{verse}

ಬಲಿಷ್ಟನಾಗಿದ್ದೆ, ಸುರಾಪಾನನಿರತನಾಗಿದ್ದೆ, ವಿದ್ಯಾ ಪೂರ್ಣನಾಗಿದ್ದೆ, ಹಾಗೂ ರುಕ್ಮರಥನ ಪುತ್ರನಾಗಿ ಅಹಂಕಾರಯುಕ್ತನಾಗಿ ಬ್ರಾಹ್ಮಣರನ್ನು ನಿಷ್ಠುರವಾಗಿ ಅವಹೇಳನ ಮಾಡುತ್ತಿದ್ದೆ. ಅಂತಹ ನಾನು ಋಷಿ ಮುಂತಾದ ಸಜ್ಜನರನ್ನು ನಿರ್ಲಕ್ಷಿಸಿ, ಶ‍್ರೀಹರಿಯ ಮೂರು ದಿವಸಗಳಲ್ಲಿ, ಅಂದರೆ ದಶಮೀ, ಏಕಾದಶೀ, ದ್ವಾದಶೀಗಳಲ್ಲಿಯೂ, ನಿಷಿದ್ಧವಾದ ತಿಥಿಗಳಲ್ಲಿಯೂ, ನಿಷಿದ್ಧವಾದ ವಾರಗಳಲ್ಲಿಯೂ ಕ್ಷೌರಕರ್ಮ ಮಾಡಿಸಿಕೊಳ್ಳುತ್ತಿದ್ದೆ.

\begin{verse}
\textbf{ತೇನ ಕರ್ಮವಿಪಾಕೇನ ಜಾತೋ ವೃಕಸಮಾನನಃ~।}\\\textbf{ಅನಯಾ ಶಾಕಿನಂ ಪ್ರಾಪ್ಯ ಶಾಕಿನೀತ್ವಮವಾಪ್ನುಯಾಮ್~।। ೧೫~।।} 
\end{verse}

ಈ ಪಾಪಗಳ ಫಲವಾಗಿ ತೋಳನ ಮುಖವನ್ನು ಹೊಂದಿ ಈ ಶಾಕಿನೀಗಣದಲ್ಲಿ ಪಿಶಾಚಿಯಾಗಿ ಹುಟ್ಟಿರುತ್ತೇನೆ.

\begin{verse}
\textbf{ಪುರಾ ದಿಗ್ವಿಜಯೇ ಕಾಲೇ ಕಾರ್ತವೀರ್ಯಾರ್ಜುನೋ ಬಲೀ~।}\\\textbf{ಭೋಜೇಶಮಗಮಜ್ಜೇತುಂ ಮಹಾಹಂಕಾರದೂಷಿತಮ್~।। ೧೬~।।} 
\end{verse}

\begin{verse}
\textbf{ಪ್ರತ್ಯುತ್ಥಿತಂ ಶೌರ್ಯಮದಾನ್ಮಯಾ ಮತ್ತೇನ ಧೀಮತಾ~।}\\\textbf{ಅಭೂತ್ ಘೋರಂ ಮಹಾಯುದ್ಧಂ ಹೈಹಯಾಧಿಪತೇರ್ಮಮ~।। ೧೭~।।} 
\end{verse}

\begin{verse}
\textbf{ತದಾರ್ಜುನಶ್ಚ ಚಿಚ್ಛೇದ ಪಾದೌ ಮೇ ಯುಧಿ ನಿಷ್ಠುರಮ್~।}\\\textbf{ಏತತ್ ವೃಕ್ಷತಲೇ ಮಹ್ಯಂ ವೃತ್ತಿರ್ದೈವೇನ ಚೋದಿತಾ~।। ೧೮~।। }
\end{verse}

ಹಿಂದೆ ಕಾರ್ತವೀರ್ಯಾರ್ಜುನನೆಂಬ ರಾಜನು ದಿಗ್ವಿಜಯ ಮಾಡುವ ಕಾಲದಲ್ಲಿ\break ಅಹಂಕಾರದಿಂದ ಕೂಡಿದ ಭೋಜದೇಶದ ರಾಜನೊಡನೆ ಯುದ್ಧಕ್ಕೆ ಬಂದನು. ಈ ವಿಷಯವನ್ನು ತಿಳಿದ ಮದಾಂಧನಾದ ನಾನು ಕಾರ್ತವೀರ್ಯಾರ್ಜುನನೊಡನೆ ಕಾದಾಡಲು ಹೋದೆ. ನನಗೂ ಹೈ ಹಯದೇಶಾಧಿಪನಾದ ಕಾರ್ತವೀರ್ಯಾರ್ಜುನನಿಗೂ ಘೋರವಾದ ಯುದ್ಧ\-ವಾಯಿತು. ಆಗ ಆತನು ಮಹಾ ನಿಷ್ಠುರನಾದ ನನ್ನ ಕಾಲುಗಳನ್ನು ಕತ್ತರಿಸಿದನು. ದೈವೇಚ್ಛೆಯಿಂದ ನನ್ನ ಮರಣವು ಈ ವೃಕ್ಷದ ಕೆಳಗೆ ಸಂಭವಿಸಿತು.

\begin{verse}
\textbf{ರಣರಂಗೇ ಮೃತಸ್ಯಾಪಿ ನ ಸ್ವರ್ಗೋಽಭೂದ್ದು ರಾತ್ಮನಃ~।}
\end{verse}

ಯುದ್ಧ ಭೂಮಿಯಲ್ಲಿ ಮರಣವಾದರೂ ನನಗೆ ವೀರಸ್ವರ್ಗ ದೊರೆಯಲಿಲ್ಲ.

\begin{verse}
\textbf{ವಿಷ್ಣು ಪ್ರಿಯತಮೇ ಶುದ್ಧೇ ವ್ರತೇ ವಾ ತ್ರಿದಿನಾತ್ಮಕೇ~।। ೧೯~।।} 
\end{verse}

\begin{verse}
\textbf{ನಿರಾಕೃತ್ಯ ದ್ವಿಜಾನ್ ವಕ್ತೃನ್ ಕ್ಷೌರಕರ್ಮ ಕೃತಂ ಮಯಾ~।}\\\textbf{ಇತಿವಾಚಾಂ ವಿರಾಮೇ ತು ತುಂದಿಲಸ್ಯ ಮಹಾತ್ಮನಃ~।। ೨೦~।। }
\end{verse}

ಕಾರಣವೇನೆಂದರೆ, “ಪವಿತ್ರವಾದ ವಿಷ್ಣು ದಿನತ್ರಯಗಳಲ್ಲಿ ಬ್ರಾಹ್ಮಣರ ಉಪದೇಶವನ್ನು ಧಿಕ್ಕರಿಸಿ ಕ್ಷೌರಕರ್ಮವನ್ನು ಮಾಡಿದೆ.” ಹೀಗೆಂದು ನುಡಿದು ತುಂದಿಲನು ಸುಮ್ಮನಾದನು.

\begin{verse}
\textbf{ವ್ಯಾಘ್ರಾನನಸ್ತು ಮುನಯೇ ಸ್ವವೃತ್ತಾಂತಮಸಾದಯತ್~।}\\\textbf{ಮತ್ಸ್ಯದೇಶಪತೇಶ್ಚಾ ಹಮುರ್ಮಿಲೋ ನಾಮ ವೀರ್ಯವಾನ್~।। ೨೧~।। }
\end{verse}

\begin{verse}
\textbf{ಧೃಷ್ಟಕೇತೋಃ ಸುತಃ ಪಾಪೀ ವೀರ್ಯೈಶ್ವರ್ಯಮದಾನ್ವಿತಃ~।}\\\textbf{ನಿತ್ಯಾಭ್ಯಂಗವ್ರತತ್ವೇನ ದಿವಸಃ ಸಾಧಿತೋ ಮಯಾ~।। ೨೨~।। }
\end{verse}

ಋಷಿಪುತ್ರನನ್ನು ಕುರಿತು ವ್ಯಾಘ್ರಾನನನೆಂಬ ಪಿಶಾಚಿಯು ನುಡಿಯಿತು. ನಾನು ಹಿಂದೆ ಮತ್ಸ್ಯ ದೇಶದ ರಾಜನಾದ ಧೃಷ್ಟಕೇತುವಿನ ಮಗ, ನನ್ನ ಹೆಸರು ಉರ್ಮಿಲ, ಒಳ್ಳೆ ವೀರ್ಯಶಾಲಿ, ಐಶ್ವರ್ಯ-ವೀರ್ಯಗಳಿಂದ ಅಹಂಕಾರಯುಕ್ತನಾಗಿದ್ದೆ. ಪಾಪಿಯಾದ ನಾನು ನಿತ್ಯವೂ ಅಭ್ಯಂಜನಸ್ನಾನಮಾಡುತ್ತಾ ಕಾಲ ಕಳೆಯುತ್ತಿದ್ದೆ.

\begin{verse}
\textbf{ಕೃತಂ ವೈಕುಂಠದಿವಸೇ ಅಭ್ಯಕ್ತಂ ಪಾಪಬುದ್ಧಿನಾ~।}\\\textbf{ತೇನ ಪಾಪೇನ ಮಹತಾ ಜಾತೋ ವ್ಯಾಘ್ರಸಮಾನನಃ~।। ೨೩~।। }
\end{verse}

ಏಕಾದಶೀ ದಿವಸದಲ್ಲಿಯ ಪಾಪಬುದ್ದಿಯಿಂದ ನಾನು ಅಭ್ಯಂಜನ ಮಾಡಿ ಕೊಳ್ಳುತ್ತಿದ್ದೆ. ಆ ಪಾಪದ ಫಲವಾಗಿ ವ್ಯಾಘ್ರಮುಖವನ್ನುಳ್ಳ ಪಿಶಾಚಿಯಾಗಿದ್ದೇನೆ.

\begin{verse}
\textbf{ಪ್ರವಕ್ಷ್ಯಾಮ್ಯಾಜ್ಞಯಾ ವೃಕ್ಷಮೂಲೇ ವಸತಿ ಕಾರಣವಮ್~।}\\\textbf{ಏತದ್ದೇಶಪತೇಃ ಕನ್ಯಾ ನಾಮ್ನಾ ಚೈತ್ರರಥಸ್ಯ ಚ~।। ೨೪~।।} 
\end{verse}

\begin{verse}
\textbf{ದತ್ತಾ ಕಾಂತಿಮತೀ ನಾಮ ವೈದೇಹಾಯ ಸ್ವಯಂವರೇ~।}\\\textbf{ಮಯಾ ಹೃತಾ ತು ಸಾ ಕನ್ಯಾ ತಸ್ಯ ರಾಜ್ಞಃ ಸಭಾಂತರೇ~।। ೨೫~।। }
\end{verse}

ಯಾವ ಕಾರಣದಿಂದ ಈ ಮರದಲ್ಲಿ ವಾಸಮಾಡುತ್ತಿರುತ್ತೇನೆಂಬ ವಿಷಯವನ್ನು ಹೇಳು\-ತ್ತೇನೆ. ಈ ದೇಶದ ರಾಜನಾದ ಚೈತ್ರರಥನ ಮಗಳಾದ ಕಾಂತಿಮತೀಯೆಂಬುವಳನ್ನು ವಿದೇಹರಾಜನಿಗೆ ಸ್ವಯಂವರದಲ್ಲಿ ಕೊಟ್ಟು ಲಗ್ನವಾಯಿತು. ಆದರೆ ನಾನು ಬಲಾತ್ಕಾರದಿಂದ ಆ ಕನ್ಯೆಯನ್ನು ಅಪಹರಿಸಿಕೊಂಡು ಹೋದೆ.

\begin{verse}
\textbf{ವಿದೇಹಾಧಿಪತಿಸ್ತೂರ್ಣಂ ಸಬಲಃ ಸಾನುಗೋ ಬಲೀ~।}\\\textbf{ಯುಯುಧೇ ಚ ಮಹಾವೀರ್ಯಃ ತ್ವಹನದ್ಬಲಮಂಜಸಾ~।। ೨೬~।।}
\end{verse}

\begin{verse}
\textbf{ಹಯಾನ್ ರಥಾನ್ ಗಜಾನ್ ಪತ್ತೀನ್ ನಿರೋಧ್ಯ ಚ ಮದೋದ್ದತಃ~।}\\\textbf{ಗೃಹೀತ್ವಾ ಮಾಂ ತಥಾ ಬದ್ವಾ ಕೃತ್ವಾ ಪಂಚಶಿಖಂ ಶಿರಃ~।। ೨೭~।।} 
\end{verse}

ಕೋಪಿಷ್ಠನಾದ ವಿದೇಹರಾಜನು ತನ್ನ ಸೈನ್ಯದಿಂದ ಸಹಿತನಾಗಿ ಬಂದು ನನ್ನೊಡನೆ ಯುದ್ಧ ಮಾಡಿ ನನ್ನ ಸೈನ್ಯವನ್ನೆಲ್ಲ ನಾಶಮಾಡಿದನು. ಕುದುರೆ, ರಥ, ಆನೆ, ಕಾಲಾಳುಗಳು-ಇವುಗಳನ್ನು ಒಳಗೊಂಡ ನನ್ನ ಸೈನ್ಯವನ್ನು ನಿರ್ಲಕ್ಷಿಸಿ, ನನ್ನನ್ನು ಹಿಡಿದು ಕಟ್ಟಿ ತಲೆಯಲ್ಲಿನ ಕೂದಲಿನಲ್ಲಿ ಐದು ಪಟ್ಟಿಗಳನ್ನು ಬೋಳಿಸಿದನು.

\begin{verse}
\textbf{ವಿಹಾಯ ಮಾಂ ಯಯೌ ಕನ್ಯಾಮಾದಾಯ ಸ್ವಪುರೀಂ ನೃಪಃ~।}\\\textbf{ವ್ರೀಡಯಾ ಮೇ ಪುರೀಂ ಪ್ರಾಪ್ತುಂ ಮನ ಆಸೀನ್ನ ಜಾತುಚಿತ್~।। ೨೮~।।} 
\end{verse}

ನನ್ನನ್ನು ಬಿಟ್ಟು ತನ್ನ ಮಗಳನ್ನು ಕರೆದುಕೊಂಡು ತನ್ನ ಊರಿಗೆ ಹೋದನು. ತುಂಬ ಲಜ್ಜಿತ\-ನಾದ ನಾನು ನನ್ನ ಊರಿಗೆ ಹೋಗಲು ಇಷ್ಟ ಪಡಲಿಲ್ಲ.

\begin{verse}
\textbf{ಏಕಾಕೀ ಭ್ರಮಮಾಣಸ್ತು ವೃಕ್ಷಮೂಲಮುಪಾಶ್ರಿತಃ~।}\\\textbf{ರಾತ್ರೌ ವೃಕ್ಷೇ ಶಯಾನಂ ಚ ಸಿಂಹೋ ಮಾಂ ಅವಧೀದ್ಬಲೀ~।। ೨೯~।। }
\end{verse}

\begin{verse}
\textbf{ತೈಲಾಭ್ಯಂಜನದೋಷೇಣ ವಿಷ್ಣೋಃ ಪುಣ್ಯದಿನತ್ರಯೇ~।}\\\textbf{ಶಾಕಿನೀಯೋನಿಮಾಶ್ರಿತ್ಯ ಜಾತೋ ವ್ಯಾಘ್ರಸಮಾನನಃ~।। ೩೦~।।} 
\end{verse}

ಒಬ್ಬನೇ ಅಲ್ಲಲ್ಲಿ ಅಲೆಯುತ್ತಾ ಈ ವೃಕ್ಷದ ಬುಡದಲ್ಲಿ ರಾತ್ರಿ ಮಲಗಿದ್ದಾಗ ಸಿಂಹವು ಬಂದು ನನ್ನನ್ನು ಸಂಹರಿಸಿತು. ಶ‍್ರೀಹರಿಯ ಮೂರು ದಿನಗಳಲ್ಲಿ ತೈಲಾಭ್ಯಂಜನ ಮಾಡಿಕೊಂಡ ಪಾಪದ ಫಲವಾಗಿ ವ್ಯಾಘ್ರದ ಮುಖವುಳ್ಳವನಾಗಿ ಶಾಕಿನೀಗಣದಲ್ಲಿ ಪಿಶಾಚಿಯಾದೆ.

\begin{verse}
\textbf{ಇತಿ ಉರ್ಮಿಲವಾಕ್ಯಾಂತೆ ತ್ರಯ ಊಚುಃ ಕದಂಬಗಾಃ~।}\\\textbf{ವಯಂ ಕ್ಷತ್ರಿಯದಾಯಾದಾಃ ಕುಂತಿಭೋಜಸುತಾಸ್ತ್ರಯಃ~।। ೩೧~।। }
\end{verse}

\begin{verse}
\textbf{ನೀಲಕೇತುಃ ಪೀತಕೇತುಶ್ಚಿತ್ರಕೇತುರಿತಿ ತ್ರಯಃ~।}\\\textbf{ಶೌರ್ಯೌದಾರ್ಯಾದಿಸುಗುಣೈರ್ಮಹೇಂದ್ರಸದೃಶಾ ವಯಮ್~।। ೩೨~।। }
\end{verse}

ಹೀಗೆ ಉರ್ಮಿಲನ ಮಾತು ಮುಗಿಯಲು ಅದೇ ಗಣದಲ್ಲಿನ ಇನ್ನು ಮೂರು ಪಿಶಾಚಿಗಳು ನುಡಿದುವು. ನೀಲಕೇತು, ಪೀತಕೇತು, ಚಿತ್ರಕೇತು ಎಂಬ ಹೆಸರಿನ ನಾವು ಕುಂತಿಭೋಜನೆಂಬ ಕ್ಷತ್ರಿಯನ ಪುತ್ರರು, ಸಾಹಸ, ಔದಾರ್ಯ ಇವುಗಳಲ್ಲಿ ಇಂದ್ರನ ಸದೃಶರಾಗಿದ್ದೆವು.

\begin{verse}
\textbf{ಕದಾಚಿದಟಿವೀಂ ಪ್ರಾಪ್ತಾ ಮೃಗಯಾಸಕ್ತಚೇತಸಃ~।}\\\textbf{ಭ್ರಮಮಾಣಾ ವಾಗುರಿಕೈರ್ಬಹುಭಿಶ್ಚ ಸಮಾವೃತಾಃ~।। ೩೩~।।} 
\end{verse}

ಒಂದು ದಿವಸ ಬೇಟೆಯಾಡುವ ಇಚ್ಛೆಯಿಂದ ನಾವು ಬೇಟೆಗಾರರಿಂದ ಸಹಿತರಾಗಿ ಈ ಕಾಡಿನಲ್ಲಿ (ಅಲೆದೆವು).

\begin{verse}
\textbf{ಮೃಗಾ ನಾನಾಭಿಧಾಶ್ಚಾಪಿ ಜ್ವಾಲಾಭಿರ್ವಕ್ರದಾರುಭಿಃ~।}\\\textbf{ರಣತೋ ಧಾವತೋ ಲೀನಾಂಶ್ಚಲಿತಾನ್ ಪತಿತಾನಪಿ~।। ೩೪~।। }
\end{verse}

ನಾನಾವಿಧವಾದ ಮೃಗಗಳನ್ನು ಬಾಣಗಳಿಂದ ಹೊಡೆದು ಸಾಯಿಸಿದೆವು. ಕೆಲವು ಮೃಗಗಳು ಹೆದರಿ ಓಡಿದವು, ಕೆಲವು ಅಲ್ಲಿಯೇ ಅಡಗಿಕೊಂಡುವು. ಮತ್ತೆ ಕೆಲವು ಅಲ್ಲಲ್ಲಿ ಬಿದ್ದವು.

\begin{verse}
\textbf{ಮೃಗಾನ್ನಾನಾವಿಧಾನ್ ಹತ್ವಾ ಗೃಹೀತ್ವಾ ತಾನ್ ಮದಾನ್ವಿತಾಃ~।}\\\textbf{ಪ್ರಾಪ್ತಂ ಶೋಣತಟಂ ಭೋಕ್ತುಂ ಅಸ್ಮಾಭಿರನುಗೈಃ ಸಹ~।। ೩೫~।।} 
\end{verse}

ಅಹಂಕಾರಯುಕ್ತರಾದ ನಾವು ನಾನಾವಿಧವಾದ ಮೃಗಗಳನ್ನು ಕೊಂದು, ಅವೆಲ್ಲವನ್ನೂ ತೆಗೆದುಕೊಂಡು, ಶೋಣತಟವೆಂಬ ಸ್ಥಳಕ್ಕೆ ಬಂದು ನಮ್ಮ ಸಂಗಡಿಗರೊಂದಿಗೆ ತಿನ್ನಲು ಸಿದ್ಧ\-ವಾದೆವು.

\begin{verse}
\textbf{ತತ್ರ ದೇವಾಲಯಂ ದಿವ್ಯಂ ವಿಷ್ಣೋರ್ದೇವಸ್ಯ ಚಕ್ರಿಣಃ~।}\\\textbf{ತದ್ದ್ರಷ್ಟುಂ ಗತಮಸ್ಮಾಭಿರ್ಲೀಲಯಾ ಚಾನುಗೈಃ ಸಹ~।। ೩೬~।। }
\end{verse}

ಅಲ್ಲಿ ಚಕ್ರಪಾಣಿಯಾದ ಶ‍್ರೀವಿಷ್ಣುವಿನ ದೇವಾಲಯವಿತ್ತು. ಅದನ್ನು ನೋಡಲು ನಾವೆಲ್ಲ ನಮ್ಮ ಅನುಚರರೊಂದಿಗೆ ಅಲ್ಲಿಗೆ ಹೋದೆವು.

\begin{verse}
\textbf{ತದಾ ವಿಪ್ರಾಸ್ತು ಬಹವೋ ಋಷಯಃ ಶಂಸಿತವ್ರತಾಃ~।}\\\textbf{ಆಜಗ್ಮುರ್ದೇವಸೇವಾರ್ಥಂ ವಿವಿಶುಶ್ಚ ತದಾಲಯಮ್~।। ೩೭~।। }
\end{verse}

ಬಹಳ ಮಂದಿ ಬ್ರಾಹ್ಮಣರು, ವ್ರತಸ್ಥರಾದ ಋಷಿಗಳು ದೇವರ ಸೇವೆಗಾಗಿ ದೇವಾಲಯವನ್ನು ಪ್ರವೇಶಮಾಡಿದರು.

\begin{verse}
\textbf{ಮಾನಿತಾಸ್ತು ಪುರಾ ತೇ ವೈ ಧರ್ಮಚಿತ್ತಾ ದಯಾಲವಃ~।}\\\textbf{ತದಾ ದೈಶಿಕಮಾಹೂಯ ಕ್ರುಧಾ ದೇವಲಕಂ ವಯಮ್~।। ೩೮~।। }
\end{verse}

\begin{verse}
\textbf{ಅವದಾಮ ತದಾ ರೋಷಾದ್ವಿ ಸ್ಮೃತಿಸ್ತು ಕುತಸ್ತವ~।}\\\textbf{ನ ಜ್ಞಾತಂ ಕ್ವಿಂ ತ್ವಯಾ ಲೋಕಪತಯೋ ಹ್ಯಾಗತಾ ಇತಿ~।। ೩೯~।। }
\end{verse}

ದೇವಸ್ಥಾನದ ಅರ್ಚಕನು ಧರ್ಮದಲ್ಲಿಯೇ ನಿರತರಾದ ದಯಾಶಾಲಿಗಳಾದ ಆ\break ಬ್ರಾಹ್ಮಣರು ಮತ್ತು ಋಷಿಗಳಿಗೆ ಮೊದಲು ಮರ್ಯಾದೆ ಮಾಡಿದನು. ಆಗ ನಮಗೆ ಬಹಳ ಕೋಪ ಬಂದಿತು. ಕೂಡಲೇ ಅರ್ಚಕನನ್ನು ಕರೆದು ಹೀಗೆ ದಂಡಿಸಿದೆವು. ``ಅರ್ಚಕನೇ ನಿನಗೆ ಸರಿಯಾದ ಸ್ಮೃತಿ ಇದೆಯೋ ಇಲ್ಲವೋ? ರಾಜರಾದ ನಾವು ಇಲ್ಲಿಗೆ ಬಂದಿರುವಾಗ ನಮ್ಮನ್ನು ಅಲಕ್ಷಿಸಿದ್ದಕ್ಕೆ ಕಾರಣವೇನು?”

\begin{verse}
\textbf{ಭಿಕ್ಷುಕಾಃ ಮಾನಿತಾಃ ಪೂರ್ವಂ ತ್ವಯಾ ವಯಂ ಅಮಾನಿತಾಃ~।}\\\textbf{ತಸ್ಮಾತ್ತೇಽವಿದುಷೋ ಲೋಕೇ ದಂಡೋಽಸ್ಮಾಭಿರ್ವಿಧೀಯತೇ~।। ೪೦~।। }
\end{verse}

ಭಿಕ್ಷೆ ಬೇಡುವ ಜನರಿಗೆ ಮೊದಲು ಸತ್ಕರಿಸಿ ನನಗೆ ಅವಮಾನ ಮಾಡಿರುವಿ. ಜ್ಞಾನರಹಿತನಾದ ನಿನಗೆ ಯೋಗ್ಯವಾದ ಶಿಕ್ಷೆಯನ್ನು ವಿಧಿಸುತ್ತೇವೆ.

\begin{verse}
\textbf{ಅನಾತ್ಮಜ್ಞಾ ಕೃತಘ್ನಾಶ್ಚ ವೃಥಾಽಹಂಕಾರದೂಷಿತಾಃ~।}\\\textbf{ರಾಜಾ ಲೋಕಪಿತಾ ತ್ರಾತಾ ದಾತಾ ಮಾತಾ ಚ ಶಿಕ್ಷಕಃ~।। ೪೧~।। }
\end{verse}

ಈ ಬ್ರಾಹ್ಮಣರು ಅಧ್ಯಾತ್ಮಜ್ಞಾನರಹಿತರು, ಉಪಕಾರವನ್ನು ಸ್ಮರಣೆಮಾಡದವರು, ತಮ್ಮಲ್ಲಿ ಯಾವ ಗುಣ ಇಲ್ಲದಿದ್ದರೂ ವ್ಯರ್ಥವಾಗಿ ಅಹಂಕಾರಪಡುವವರು; ಆದರೆ ರಾಜರು ಲೋಕದಲ್ಲಿನ ಜನರಿಗೆ ತಂದೆ-ತಾಯಿಯರಂತೆ ರಕ್ಷಕರು, ಆಹಾರಾದಿಗಳನ್ನು ಕೊಡುವವರು, ವಿದ್ಯೆಯನ್ನು ಕಲಿಸುವವರು.

\begin{verse}
\textbf{ಅಸ್ಮಾಭಿಃ ಸುಖಿನೋ ಹ್ಯೇತೇ ಕಥಮೇತಚ್ಚ ವಿಸ್ಮೃತಮ್~।}\\\textbf{ಇತ್ಯುಕ್ತ್ವಾ ಬಹಿರಾಗತ್ಯ ಕೃದ್ಧಾ ತಾಮ್ರವಿಲೋಚನಾಃ~।। ೪೨~।। }
\end{verse}

ಈ ಬ್ರಾಹ್ಮಣರು ರಾಜರಾದ ನಮ್ಮ ದಯದಿಂದ ಸುಖವಾಗಿದ್ದಾರೆ. ಈ ವಿಚಾರವನ್ನು ನೀನು ಹೇಗೆ ಮರೆತೆ?” ಹೀಗೆಂದು ಕೋಪದಿಂದ ಕಣ್ಣುಗಳನ್ನು ಕೆಂಪಗೆ ಮಾಡಿಕೊಂಡು ದೇವಾಲಯದಿಂದ ಹೊರಕ್ಕೆ ಬಂದೆವು.

\begin{verse}
\textbf{ಆಹರಾಮ ಚ ದೇವಸ್ವಂ ಬ್ರಹ್ಮಸ್ವಮಖಿಲಂ ತತಃ~।}\\\textbf{ಸ್ವದತ್ತಂ ಪರದತ್ತಂ ಚ ಭೂಮಿಧಾನ್ಯಧನಾದಿಕಮ್~।। ೪೩~।।} 
\end{verse}

ಆ ದೇವಾಲಯದ ವ್ಯವಸ್ಥೆಗಾಗಿ ಪೂರ್ವಿಕರು ಕೊಟ್ಟಿದ್ದ ಎಲ್ಲ ಭೂಮಿ, ದ್ರವ್ಯ ಮತ್ತು ಇತರ ಆಸ್ತಿಪಾಸ್ತಿಗಳನ್ನೆಲ್ಲ ಕಿತ್ತುಕೊಂಡೆವು. ಇತರ ಭಕ್ತಾದಿಗಳಿಂದ ಕೊಡಲ್ಪಟ್ಟ ಆಸ್ತಿಯನ್ನೂ ಮತ್ತು ನಾವೇ ಹಿಂದೆ ಕೊಟ್ಟಿದ್ದ ಆಸ್ತಿಯನ್ನೂ ಕಸಿದುಕೊಂಡೆವು.

\begin{verse}
\textbf{ತಸ್ಮಾದ್ದೋಷಾನ್ಮೃತಿಂ ಪ್ರಾಪ್ತಾ ಅನುಭೂಯ ಚ ಯಾತನಾಮ್~।}\\\textbf{ಬಹುಧಾ ಕಲ್ಪಸಾಹಸ್ರಂ ಪಶ್ಚಾದೇತತ್ಕದಂಬಗಾಃ~।। ೪೪~।। }
\end{verse}

\begin{verse}
\textbf{ಶಾಕಿನೀಂ ಯೋನಿಮಾಶ್ರಿತ್ಯ ಗೃಧ್ರಶೀರ್ಷಾ ಭಯಾನ್ವಿತಾಃ~।}\\\textbf{ತಿಷ್ಠಾಮಃ ಸುಚಿರಂ ಕಾಲಂ ವಹ್ನಿಜ್ವಾಲಾಸಮಾಕುಲಮ್~।। ೪೫~।। }
\end{verse}

ಆ ದೋಷ ನಿಮಿತ್ತದಿಂದ ಮೃತಿಯನ್ನು ಹೊಂದಿದ ಮೇಲೆ ಅನೇಕ ಸಹಸ್ರ ಕಲ್ಪಗಳಲ್ಲಿ ನರಕದಲ್ಲಿ ಯಾತನೆಯನ್ನು ಅನುಭವಿಸಿ ನಂತರ ಶಾಕಿನೀಗಣದಲ್ಲಿ ಹದ್ದಿನ ತಲೆಯಂತೆ ತಲೆಯುಳ್ಳವರಾಗಿ ಭಯಂಕರರಾದ ಪಿಶಾಚಿಗಳಾಗಿದ್ದೇವೆ. ಬೆಂಕಿಯ ಜ್ವಾಲೆಯಲ್ಲಿ ದುಃಖಪಡುವವರಂತೆ ನಾವು ನಾನಾ ದುಃಖಗಳನ್ನು ಅನುಭವಿಸುತ್ತಾ ಬಹಳ ಕಾಲದಿಂದ ನಿಂತಿರುತ್ತೇವೆ.

\begin{verse}
\textbf{ಪೂರ್ವಜನ್ಮನಿ ಚ ವಯಂ ಯವನಾಧಿಪತೇಃ ಸುತಾಃ~।}\\\textbf{ಅಸ್ಮಿನ್ ಕದಂಬೇ ಚಾಸ್ಮಾಭಿರ್ಗರ್ಭಿಣ್ಯೋ ಗಾ ಹತಾಸ್ತಥಾ~।। ೪೬~।। }
\end{verse}

\begin{verse}
\textbf{ಅಸ್ಮಿನ್ ಜನ್ಮನ್ಯಪಿ ತತಃ ಕದಂಬೇsಸ್ಮಿನ್ ಮಹಾದ್ರುಮೇ~।}\\\textbf{ವಸಂತೇ ಋತುಕಾಲೇ ಚ ವನೇಽಸ್ಮಿನ್ ಪ್ರಮದಾಗಣೈಃ~।। ೪೭~।।} 
\end{verse}

ಹಿಂದಿನ ಜನ್ಮದಲ್ಲಿ ನಾವು ಯವನರಾಜನ ಮಕ್ಕಳಾಗಿದ್ದೆವು. ಈ ಕದಂಬ ವೃಕ್ಷದ ಕೆಳಗೆ ಗರ್ಭಿಣಿ ಸ್ತ್ರೀಯರೊಡನೆ ಕ್ರೀಡಿಸಿದೆವು. ಹಸುವನ್ನು ಸಂಹರಿಸಿದೆವು. ಈ ಜನ್ಮದಲ್ಲಿಯೂ ನಾವೆಲ್ಲರೂ ವಸಂತಋತುವಿನಲ್ಲಿ ಸ್ತ್ರೀಯರೊಡನೆ ರಮಿಸಲು ಕದಂಬ ವೃಕ್ಷಕ್ಕೆ ಬಂದಿದ್ದೆವು.

\begin{verse}
\textbf{ಉಪವಿಷ್ಟಂ ಸಹಾಸ್ಮಾಭಿರ್ಭೋಕ್ತುಂ ವೃಷ್ಟಿಸ್ತದಾಭವತ್~।}\\\textbf{ತದಾ ಚಾಶನಿಪಾತೇನ ಮೃತಿರ್ಜಾತಾ ದುರಾತ್ಮನಾಮ್~।। ೪೮~।।} 
\end{verse}

ಭೋಗಿಸುವ ಕಾಲದಲ್ಲಿ ಗುಡುಗು, ಸಿಡಿಲಿನಿಂದ ಸಹಿತವಾದ ದೊಡ್ಡ ಮಳೆ ಬಂತು. ಸಿಡಿಲಿನ ಹೊಡೆತದಿಂದ ದುರಾತ್ಮರಾದ ನಾವು ಮರಣಹೊಂದಿದೆವು.

\begin{verse}
\textbf{ತಸ್ಮಾತ್ ಏತದ್ದ್ರುಮೇ ವಾಸೋ ಭವಿತಾ ಕರ್ಕಶೋ ಮಹಾನ್~।}\\\textbf{ವಾಕ್ಯಾಂತೇ ಗೃಧ್ರಶಿರಸಃ ಪ್ರೇತಾವೌದುಂಬರಾಶ್ರಯೌ~।। ೪೯~।।} 
\end{verse}

ಈ ಕಾರಣದಿಂದ ಈ ವೃಕ್ಷದಲ್ಲಿ ಕಷ್ಟದಿಂದ ವಾಸಿಸುವ ದೌರ್ಭಾಗ್ಯವು ಪ್ರಾಪ್ತವಾಯಿತು. ಹೀಗೆಂದು ಹದ್ದಿನ ತಲೆಯಂತೆ ತಲೆಯುಳ್ಳ ಪಿಶಾಚಿಯು ತನ್ನ ಚರಿತ್ರೆಯನ್ನು ಮುಗಿಸಲು ಅತ್ತಿಯ ಗಿಡದಲ್ಲಿದ್ದ ಎರಡು ಪ್ರೇತಗಳು ನುಡಿದುವು:

\begin{verse}
\textbf{ಪುರಾ ಬ್ರಹ್ಮನ್ ಸುತೌ ವೀರೌ ವಸುಷೇಣ ಮಹೀಭೃತಃ~।}\\\textbf{ಸಮಾಕಾರೌ ಸನಾಮಾನ್ ಯಮೌ ಸಮವಿಭೂಷಣೌ~।। ೫೦~।।} 
\end{verse}

\begin{verse}
\textbf{ಸಮಾಂಬರೌ ಸಮರಥೌ ಸನಾಮಾನೌ ಬಲಾನ್ವಿತೌ~।}\\\textbf{ವಿಧೋಷಾವಿತಿವಿಖ್ಯಾತೌ ಕಾಲಮೃತ್ಯೂ ಇವಾಪರೌ‌~।। ೫೧~।। }
\end{verse}

ಬಾಹ್ಮಣನೇ! ನಾವಿಬ್ಬರೂ ಹಿಂದೆ ವಸುಷೇಣನೆಂಬ ಅರಸನ ಅವಳಿಜವಳಿ ಮಕ್ಕಳು. ಇಬ್ಬರಿಗೂ ಒಂದೇ ವಿಧವಾದ ರೂಪ, ಆಕಾರ, ಆಚಾರ ಆಭರಣಾದಿಗಳು, ವಸ್ತ್ರಗಳು, ರಥಗಳು, ಇಬ್ಬರಿಗೂ ವಿಧೋಷಾ ಎಂಬುದಾಗಿ ಒಂದೇ ಹೆಸರು, ಬಲಿಷ್ಠರು. ಇನ್ನೊಬ್ಬ ಯಮನಂತೆ ಇದ್ದೆವು.

\begin{verse}
\textbf{ವಂಗದೇಶಪತೀ ಆವಾಂ ಪ್ರಜಾಪೀಡಾಪರಾಯಣೌ~।}\\\textbf{ನಂದವತ್ಯಾಂ ಮಹಾಪುರ್ಯಾಂ ಸ್ಥಿತೌ ಲೋಕಸ್ಯ ಕಂಟಕೌ~।। ೫೨~।।} 
\end{verse}

ವಂಗದೇಶಕ್ಕೆ ರಾಜರಾಗಿದ್ದ ನಾವು ಪ್ರಜೆಗಳಿಗೆ ತೊಂದರೆಯನ್ನು ಕೊಡುತ್ತ, ಲೋಕಕಂಟಕರಾಗಿ ನಂದವತೀ ಎಂಬ ದೊಡ್ಡ ನಗರದಲ್ಲಿದ್ದೆವು.

\begin{verse}
\textbf{ಆವಾಭ್ಯಾಂ ಪಾಲಿತೇ ರಾಜ್ಯೇ ನಾಸೀತ್ಕೋಽಪಿ ಸುಖೀಜನಃ~।}\\\textbf{ವಿನಾಪರಾಧಂ ಸರ್ವೇಪಿ ದಂಡಿತಾ ವಿಷಯಾಂತರೇ~।। ೫೩~।। }
\end{verse}

ನಾವು ಪಾಲಿಸುತ್ತಿದ್ದ ರಾಜ್ಯದಲ್ಲಿ ಯಾರೂ ಸುಖವಾಗಿರಲಿಲ್ಲ. ಅಪರಾಧವನ್ನೇ ಮಾಡದಿದ್ದ ಜನರಮೇಲೆ ಸುಳ್ಳು ಅಪರಾಧವನ್ನು ಆರೋಪಿಸಿ ದಂಡಿಸುತ್ತಿದ್ದೆವು.

\begin{verse}
\textbf{ಶ್ವೋ ಭೋಕ್ತುಂ ವಿತ್ತಂ ಕಸ್ಯಾಪಿ ನಾಸೀದ್ರಾಜ್ಯೇ ಮುನೀಶ್ವರ~।}\\\textbf{ಆವಾಂ ಕದಾಚಿದಟವೀಮಾಪತುರ್ವಿಷಯೇಚ್ಛಯಾ~।। ೫೪~।। }
\end{verse}

ನಮ್ಮ ರಾಜ್ಯದಲ್ಲಿದ್ದ ಜನರಿಗೆ ಮಾರನೆಯದಿನದ ಊಟಕ್ಕೆ ಬೇಕಾಗುವಷ್ಟು ಹಣವು ಇರಲಿಲ್ಲ. ನಾವಿಬ್ಬರೂ ಒಂದು ದಿವಸ ವಿಹಾರಾರ್ಥವಾಗಿ ಅರಣ್ಯಕ್ಕೆ ಹೋದೆವು.

\begin{verse}
\textbf{ವಸಂತೇ ಕೋಕಿಲಾರಾವಕೋಮಲೇ ಕುಸುಮೋತ್ಕರೇ~।}\\\textbf{ಜೃಂಭಚ್ಚಂಪಕಶೋಭಾಢ್ಯೇ ಕಿಂಶುಕಾಶೋಕಮಂಡಿತೇ~।। ೫೫~।। }
\end{verse}

ವನದಲ್ಲಿ ವಸಂತಋತುವಿನ ಕಾರಣದಿಂದ ಕೋಗಿಲೆಗಳ ಮಂಜುಳವಾದ ಧ್ವನಿ, ಅರಳಿದ ಹೂಗಳುಳ್ಳ ಗಿಡ-ಮರಗಳು, ರಮಣೀಯವಾದ ವಿಕಸಿತ ಸಂಪಿಗೆ ಪುಷ್ಪಗಳು, ಮುತ್ತುಗದ ಮತ್ತು ಮಾವಿನಗಿಡಗಳು ಶೋಭಿಸುತ್ತಿದ್ದುವು.

\begin{verse}
\textbf{ಪಲಾಶನೀಲಗಗನಭ್ರಾಜದ್ಬಕುಲತಾರಕೇ~।}\\\textbf{ಮತ್ತಭ್ರಮರಝಂಕಾರಜನಿತಾಟೋಪನಿರ್ಭರೇ~।। ೫೬~।।} 
\end{verse}

ಮರ-ಗಿಡಗಳು ಚಿಗುರಿದ್ದವು, ಆಕಾಶವು ನೀಲಿಬಣ್ಣದಿಂದಿತ್ತು, ಮದಿಸಿದ ದುಂಬಿಗಳು ಝೇಂಕಾರ ಧ್ವನಿಗೈಯುತ್ತಿದ್ದವು, ಬಕುಲಪುಷ್ಪಗಳು ತುಂಬಿದ್ದುವು.

\begin{verse}
\textbf{ಕೋಕಿಲಾಕಲ್ಕಲಾರಾವಮಿಲದ್ವಂಶಸ್ವನೇ ಶುಭೇ~।}\\\textbf{ರೇಮಾತೇ ರಮಣೀಪ್ರಾಣೌ ಕುಟೀರೇಷು ಯಥೇಚ್ಛಯಾ~।। ೫೭~।। }
\end{verse}

ಇಂಪಾದ ಕೋಗಿಲೆಗಳ ಧ್ವನಿ ಬಂದಾಗ, ಬಿದಿರು ಗಿಡಗಳು ಒಂದಕ್ಕೊಂದು ತಿಕ್ಕಿ ಅದರಿಂದ ಇನ್ನೊಂದು ಬಗೆಯಾದ ಧ್ವನಿಬರುತ್ತಿತ್ತು. ಇಂತಹ ಹಿತಕರವಾದ ಸಂದರ್ಭದಲ್ಲಿ ವನದಲ್ಲಿದ್ದ ಕುಟೀರಗಳಲ್ಲಿ ನಾವು ಸ್ತ್ರೀಯರೊಡನೆ ಕ್ರೀಡಿಸಿದೆವು.

\begin{verse}
\textbf{ತತಃ ಸಾಯಂ ಪುರೀಂ ಗಂತುಮುದ್ಯತಾವುದ್ಯತಾಯುಧೈಃ~।}\\\textbf{ಗ್ರಾಮಂ ಪ್ರಾಪ್ತೌ ಶೃಗಾಲಾಖ್ಯಂ ನಾನಾಜನಪದಾಶ್ರಯಮ್~।। ೫೮~।। }
\end{verse}

ಸಾಯಂಕಾಲವಾಗಲು ನಮ್ಮ ಊರಿಗೆ ವಾಪಸ್ಸು ಹೋಗಲು ಸಿದ್ದರಾಗಿ, ಆಯುಧಪಾಣಿಗಳಾಗಿ ಹೊರಟು ದಾರಿಯಲ್ಲಿ ನಾನಾಜನಗಳಿಗೆ ಆಶ್ರಯವನ್ನು ಕೊಟ್ಟಿದ್ದ “ಶೃಗಾಲ” ಎಂಬ ಊರಿಗೆ ಬಂದೆವು.

\begin{verse}
\textbf{ಪದ್ಯಕಾಖ್ಯಸ್ಯ ಶೂದ್ರಸ್ಯ ತತ್ ಗ್ರಾಮಾಧಿಪತೇಃ ಸುತೌ~।}\\\textbf{ಕ್ರೀಡಂತೌ ಮೃಗಯಾಮಾರ್ಗೇ ಶ್ಯೇನಾಭ್ಯಾಂ ತೌ ಮಹಾಬಲೌ~।। ೫೯~।।}
\end{verse}

ಅಲ್ಲಿ ಆ ಊರಿನ ಅಧಿಪತಿಯಾದ ಪದ್ಮಕನೆಂಬ ಶೂದ್ರನ ಇಬ್ಬರು ಪುತ್ರರು ಬೇಟೆಯಾಡುವ ನೆಪದಿಂದ ಎರಡು ಶ್ಯೇನ ಪಕ್ಷಿಗಳೊಡನೆ ಆಡುತ್ತಿದ್ದರು.

\begin{verse}
\textbf{ಗ್ರಸದ್ಭ್ಯಾಂ ಕಲವಿಂಕಾದೀನ್ ಶಿಕ್ಷಿತಾಭ್ಯಾಂ ಮುಹುರ್ಮುಹುಃ~।}\\\textbf{ಆವಾಭ್ಯಾಂ ಯಾಚಿತೌ ಶ್ಯೇನೌ ಕ್ರೀಡಂತೋಃ ಶೂದ್ರಪುತ್ರಯೋಃ~।। ೬೦~।।}
\end{verse}

ಆ ಶ್ಯೇನ ಪಕ್ಷಿಗಳು ಮೇಲಿಂದಮೇಲೆ ಈ ಹುಡುಗರಿಂದ ಪೀಡಿಸಿಕೊಳ್ಳಲ್ಪಡುತ್ತಿದ್ದವು. ಪಕ್ಷಿಗಳು ಇತರ ಕ್ಷುದ್ರ ಪ್ರಾಣಿಗಳನ್ನು ನುಂಗುತ್ತಿದ್ದವು. ಇದನ್ನು ಕಂಡ ನಾವಿಬ್ಬರೂ ಅ ಶೂದ್ರ ಹುಡುಗರ ಬಳಿ ಹೋಗಿ ಆ ಶ್ಯೇನ ಪಕ್ಷಿಗಳನ್ನು ಕೊಡೆಂದು ಕೇಳಿದೆವು.

\begin{verse}
\textbf{ಲೋಭಾತ್ ದುದ್ರುವತುಃ ಶೂದ್ರ ಕುಮಾರೌ ಸುಂದರೌ ಶುಭೌ~।}\\\textbf{ಆವಾಂ ಕೋಪಕಷಾಯಾಕ್ಷೌ ಪೃಷ್ಠಮನ್ವೀಯತುಃ ಕ್ಷಣಾತ್~।। ೬೧~।।} 
\end{verse}

\begin{verse}
\textbf{ಶರೈಸ್ತಯೋಃ ಶಿರಚ್ಛಿತ್ವಾ ತಾಭ್ಯಾಂ ಶ್ಯೇನೌ ಹೃತೌ ತದಾ~।}\\\textbf{ಆವಯೋಸ್ತೇನ ದೋಷೇಣ ನಾಭೂತ್ಸಂತಾನಮುತ್ತಮಮ್~।। ೬೨~।।}
\end{verse}

ಸುಂದರವಾದ ಆ ಶೂದ್ರಪುತ್ರರು ಅಲ್ಲಿಂದ ಓಡಲು ಪ್ರಾರಂಭಿಸಿದರು. ನಾವೂ ಸಹ ಕೋಪದಿಂದ ಕಣ್ಣುಗಳನ್ನು ಕೆಂಪಗೆ ಮಾಡಿಕೊಂಡು ಅವರ ಹಿಂದೆಯೇ ಓಡಿ ಬಂದು ಕ್ಷಣಮಾತ್ರದಲ್ಲಿ ಬಾಣಗಳಿಂದ ಆ ಇಬ್ಬರ ಹುಡುಗರ ತಲೆಗಳನ್ನು ಕತ್ತರಿಸಿ ಶ್ಯೇನಪಕ್ಷಿಗಳನ್ನು ಅಪಹರಿಸಿದೆವು. ಈ ದೋಷದಿಂದ ನಮಗೆ ಸಂತತಿಯಾಗಲಿಲ್ಲ.

\begin{verse}
\textbf{ನಿಃಸಂತಾನೌ ತತಃ ಕಾಲಾತ್ ಕಾಲಧರ್ಮಮುಪಾಗತೌ~।}\\\textbf{ಪ್ರೇತಾವೌದುಂಬರೇ ವೃಕ್ಷೇ ಬಹುದುಃಖಸಮಾಕುಲೌ~।। ೬೩~।।}
\end{verse}

ಸಂತಾನಹೀನರಾದ ನಾವು ಕೆಲವು ಕಾಲದನಂತರ ಮೃತಪಟ್ಟು ಈ ಔದುಂಬರ ವೃಕ್ಷದಲ್ಲಿ ಬಹುದುಃಖವನ್ನು ಅನುಭವಿಸುತ್ತಾ ಪ್ರೇತಯೋನಿಯಲ್ಲಿದ್ದೇವೆ.

\begin{verse}
\textbf{ಮೃತಿರೌ‌ದುಂಬರೇ ವೃಕ್ಷೇ ಚಾವಯೋಃ ಕಾರಣಾಂತರಾತ್~।}\\\textbf{ಪುರಾ ತು ಕೌಶಿಕೋ ನಾಮ ಯಜ್ಞಾರ್ಥಂ ಗೃಹಮಾಗತಃ~।। ೬೪~।।}
\end{verse}

\begin{verse}
\textbf{ಯಯಾಚೇ ದ್ರವಿಣಂ ಸೋಽಪಿ ಆವಾಭ್ಯಾಂ ಚ ನಿರಾಕೃತಃ~।}\\\textbf{ಆವಾಂ ಲೋಕಗೃಹೀತಾರೌ ನ ದಾತಾರೌ ಕದಾಚನ~।। ೬೫~।।}
\end{verse}

ಈ ಔದುಂಬರ ವೃಕ್ಷದಲ್ಲಿ ಮರಣವಾಗಲು ಕಾರಣವೇನೆಂದರೆ: ಹಿಂದೆ ಕೌಶಿಕನೆಂಬ ಬ್ರಾಹ್ಮಣನು ನಮ್ಮ ಮನೆಗೆ ಬಂದು ಯಜ್ಞಕ್ಕೋಸ್ಕರ ಹಣವನ್ನು ಯಾಚಿಸಿದನು. ನಾವು ಕೊಡಲಿಲ್ಲ. ನಾವು ಲೋಕದ ಜನರಿಂದ ಹಣವನ್ನು ಕಸಿದುಕೊಳ್ಳುವವರೇ ಹೊರತು ನಾವು ಯಾರಿಗೂ ಏನನ್ನೂ ಕೊಡುವವರಲ್ಲ.

\begin{verse}
\textbf{ತ್ವದೀಯಂ ಚೇದಸ್ತಿ ಧನಂ ದತ್ವಾ ಗಚ್ಛೇತಿ ವೈ ಮುನಿಃ~।}\\\textbf{ಸ ಕೋಪಾದಶಪನ್ನೌ ಚ ಮದಾದ್ವಿ ಸ್ಮೃತಸಜ್ಜನೌ~।। ೬೬~।।} 
\end{verse}

\begin{verse}
\textbf{ಅರಣ್ಯೌದುಂಬರೇ ವೃಕ್ಷೇ ಮೃತಿರ್ವಾ ಸ್ಯಾದಿತಿ ಸ್ಫುಟಮ್~।}\\\textbf{ಪುತ್ರೌ ಚಿತ್ರರಥಸ್ಯಾಥ ಗಂಧರ್ವಸ್ಯ ಮಹಾತ್ಮನಃ~।। ೬೭~।।} 
\end{verse}

\begin{verse}
\textbf{ಆಜಗ್ಮತುಃ ಸನಾರೀಕಾವಸ್ಮತ್ ಕ್ರೀಡಾವನಂ ಶುಭಮ್~।}\\\textbf{ತತ್ ಶ್ರುತ್ವಾ ತು ತದಾ ಶೀಘ್ರಮಾವಾಂ ಜಾತಕ್ರುಧೌ ತತಃ~।। ೬೮~।।}
\end{verse}

ನಿನ್ನ ಬಳಿ ಏನಾದರೂ ಹಣವಿದ್ದರೆ ಅದನ್ನು ಕೊಟ್ಟು ಹೋಗು ಎಂಬುದಾಗಿ ನಾವು ಆ ಕೌಶಿಕನಿಗೆ ಹೇಳಲು ಕೋಪಗೊಂಡ ಆ ಬ್ರಾಹ್ಮಣನು ಮದಾಂಧರಾಗಿ ಸಜ್ಜನರಿಗೆ ತಿರಸ್ಕಾರ ಮಾಡಿದ ನೀವು ಅರಣ್ಯದಲ್ಲಿ ಔದುಂಬರ ವೃಕ್ಷದಲ್ಲಿ ಮೃತಿಯನ್ನು ಹೊಂದಿರಿ ಎಂದು ಶಪಿಸಿದನು. ನಂತರ ಗಂಧರ್ವನಾದ ಚಿತ್ರರಥನ ಇಬ್ಬರು ಮಕ್ಕಳು ಸ್ತ್ರೀಯರೊಡನೆ ವಿಹರಿಸಲು ಈ ಅರಣ್ಯಕ್ಕೆ ಬಂದರು. ಈ ವಿಷಯವನ್ನು ಕೇಳಿ ನಮಗೆ ಬೇಗನೆ ಕೋಪ ಬಂದಿತು.

\begin{verse}
\textbf{ತಾಭ್ಯಾಂ ಯುದ್ಧಂ ಚಕೃವಹೇ ಜಿತೌ ದುದ್ರುವತುರ್ಭಯಾತ್~।}\\\textbf{ತಾವೂಚತುಃ ಸುತೌ ಪಿತ್ರೇ ದುಃಖಂ ಚಿತ್ರರಥಾಯ ಚ~।। ೬೯~।।} 
\end{verse}

\begin{verse}
\textbf{ಶ್ರುತ್ವಾ ಪರಾಭವಂ ಕ್ರೋಧಾದಾರುರೋಹ ರಥಂ ಶನೈಃ~।}\\\textbf{ಗಂಧರ್ವಮಾಯಯಾನೀಯ ವೃಕ್ಷೇಽಸ್ಮಿನ್ ಕಂಠಮಾವಯೋಃ~।। ೭೦~।।}
\end{verse}

ನಾವು ಅವರ ಸಂಗಡ ಯುದ್ಧಮಾಡಿ ಜಯಿಸಲು ಅವರು ಭಯದಿಂದ ಓಡಿ ಹೋಗಿ ತಮ್ಮ ತಂದೆಯಾದ ಚಿತ್ರರಥನಿಗೆ ತಮಗಾದ ಪರಾಭವವನ್ನು ತಿಳಿಸಿದರು. ಇದನ್ನು ಕೇಳಿದ ಚಿತ್ರರಥನು ರಥದಲ್ಲಿ ಕುಳಿತು, ಗಂಧರ್ವ ಮಾಯೆಯಿಂದ ಈ ವೃಕ್ಷದಲ್ಲಿ ನಮ್ಮಿಬ್ಬರ ಕುತ್ತಿಗೆಗಳನ್ನೂ

\begin{verse}
\textbf{ಛಿತ್ವಾವಯೋಽಸ್ತತಾದಾಯ ಶಿರಸೀ ಪುತ್ರಯೋಃ ಪುರಃ~।}\\\textbf{ನಿಧಾಯ ಸುಖಮಾಪೇದೇ ನಿರ್ವ್ಯಲೀಕಂ ಮಹಾಮತಿಃ~।}\\\textbf{ತಸ್ಮಾತ್‌ ಔದುಂಬರೇ ವೃಕ್ಷೇ ಚಾವಯೋರಾಶ್ರಮೋsಭವತ್~।। ೭೧~।।}
\end{verse}

ಕತ್ತರಿಸಿ, ಪುನಃ ತನ್ನ ಪಟ್ಟಣಕ್ಕೆ ಹೋಗಿ ನಮ್ಮ ತಲೆಗಳನ್ನು ತನ್ನ ಮಕ್ಕಳ ಮುಂದೆ ಇಟ್ಟು ಸುಖವಾಗಿದ್ದನು. ಈ ಕಾರಣದಿಂದ ನಮಗೆ ಔದುಂಬರ ವೃಕ್ಷವೇ ಆಶ್ರಯವಾಯಿತು.

\begin{verse}
\textbf{ಅರಾಜಕಂ ತತೋ ರಾಜ್ಯಂ ಪರೈರಾಕ್ರಾಂತಮಾಸ ಚ~।}\\\textbf{ಪುತ್ರಾಭಾವಾದಾವಯೋಶ್ಚ ಪ್ರೇತತ್ವಂ ದುಃಖಪೂರಿತಮ್~।। ೭೨~।।}\\\textbf{ನ ನಿವೃತ್ತ ಮಿದಾನೀಂ ಚ ಕರ್ತವ್ಯಾ ಭವತಾ ದಯಾ~।। ೭೩~।।}
\end{verse}

ರಾಜನಿಲ್ಲದೆ ರಾಜ್ಯವು ಇತರರಿಂದ ಆಕ್ರಮಿಸಿಕೊಳ್ಳಲ್ಪಟ್ಟಿತು. ಮಕ್ಕಳಿಲ್ಲದ ನಮ್ಮಿಬ್ಬರಿಗೆ ಪ್ರೇತಜನ್ಮವು ಬಂದಿತು. ಬಹಳ ದುಃಖವನ್ನು ಅನುಭವಿಸಿದೆವು. ಅದು ಇನ್ನೂ ಮುಗಿದಿಲ್ಲ. ನೀನು ನಮ್ಮಿಬ್ಬರ ಮೇಲೆ ಅನುಗ್ರಹ ಮಾಡು.

\begin{center}
ಇತಿ ಶ‍್ರೀ ವಾಯುಪುರಾಣೇ ಮಾಘಮಾಸಮಾಹಾತ್ಮ್ಯೇ ಚತುರ್ದಶೋsಧ್ಯಾಯಃ 
\end{center}

\begin{center}
ಶ‍್ರೀ ವಾಯುಪುರಾಣಾಂತರ್ಗತ ಮಾಘಮಾಸ ಮಹಾತ್ಮ್ಯೆಯಲ್ಲಿ \\ ಹದಿನಾಲ್ಕನೇ ಅಧ್ಯಾಯವು ಸಮಾಪ್ತಿಯಾಯಿತು.
\end{center}

\newpage

\section*{ಅಧ್ಯಾಯ\enginline{-}೧೫}

\emptypage

\begin{flushleft}
\textbf{ಬ್ರಹ್ಮೋವಾಚ\enginline{-}}
\end{flushleft}

\begin{verse}
\textbf{ತತಃ ಸಂಭೂಯ ತೇ ಸರ್ವೇ ಪಿಶಾಚಾ ಗತಕಲ್ಮಷಾಃ~।}\\\textbf{ಆತ್ಮವಾಚ್ಯಮನೂದ್ಯೈವಂ ಪ್ರಣಮ್ಯ ಶಿರಸಾ ಮುನಿಮ್~।। ೧~।।}
\end{verse}

\begin{flushleft}
ಬ್ರಹ್ಮದೇವರು ನುಡಿದರು:
\end{flushleft}

ಎಲ್ಲ ಪಿಶಾಚಿಗಳೂ ತಮ್ಮ ತಮ್ಮ ಪಾಪಗಳನ್ನು ಕಳೆದುಕೊಂಡು, ತಮ್ಮ ತಮ್ಮ ಹಿಂದಿನ ಚರಿತ್ರೆಗಳನ್ನು ಮುನಿಪುತ್ರನಿಗೆ ತಿಳಿಸಿ ಎಲ್ಲರೂ ಅವನಿಗೆ ನಮಸ್ಕರಿಸಿದುವು.

\begin{verse}
\textbf{ತುಷ್ಟುವುರ್ವಿವಿಧೈಃ ಸ್ತೋತ್ರೈಃ ಸಂತೋಷಾತ್ತಂ ಮುಹುರ್ಮುಹುಃ~।}\\\textbf{ನಮಸ್ತೇ ಮುನಿಶಾರ್ದೂಲ ನಮಸ್ತೇ ಕರುಣಾಕರ~।। ೨~।। }
\end{verse}

\begin{verse}
\textbf{ಮಹಾಪಾತಕಿನಾಂ ಪಂಕಶೋಷಣೇ ತತ್ವಕೋವಿದ~।}\\\textbf{ಅಸ್ಮತ್ ಪ್ರಬೋಧಾಬ್ಧಿ ನಿಶಾಕರಾಯ} \\\textbf{ನಮೋಽಸ್ತು ಶೋಕಾಬ್ಧಿ ಕರೀರಜನ್ಮನೇ~।। ೩~।।}
\end{verse}

ಸಂತೋಷದಿಂದ ತುಂಬಿದ ಅವರು ಆ ಮುನಿಪುತ್ರನನ್ನು ಬಾರಿಬಾರಿಗೂ ನಾನಾವಿಧದಿಂದ ಸ್ತೋತ್ರಮಾಡಿದರು: ಮುನಿಶ್ರೇಷ್ಠನೇ, ದಯಾಶಾಲಿಯೇ, ತತ್ವವನ್ನು ಚೆನ್ನಾಗಿ ಬಲ್ಲವನೇ, ಮಹಾಪಾತಕಿಗಳ ಪಾಪಗಳನ್ನು ನಾಶಮಾಡುವುದರಲ್ಲಿ ಕುಶಲನೇ, ನಮ್ಮ ಹಿಂದಿನ ಜನ್ಮಗಳ ಸ್ಮರಣೆಯೆಂಬ ಸಮುದ್ರಕ್ಕೆ ಚಂದ್ರನೇ, ನನ್ನ ಶೋಕವೆಂಬ ಸಮುದ್ರಕ್ಕೆ ಕುಂಭೋದ್ಭವ ಋಷಿಯೇ, ನಿನಗೆ ಅನೇಕ ನಮಸ್ಕಾರಗಳು.

\begin{verse}
\textbf{ನಮೋಸ್ತು ಚಿದ್ಧ್ವಾಂತದಿವಾಕರಾಯ} \\\textbf{ನಮೋಸ್ತು ತೇ ಭೂರಿವಿಭೂತಿದಾಯ}\\\textbf{ಭವಾಜಗರದಷ್ಟಾನ್ಯಾಂದೀನಾನಾಮಕೃತಾತ್ಮನಾಮ್~।। ೪~।।} 
\end{verse}

\begin{verse}
\textbf{ಕೃಪಾಸುಧೌಷಧಂ ದೇಹಿ ಕ್ವಪಾಲೋ ತ್ರಾತುಮರ್ಹಸಿ~।}\\\textbf{ಪುರಾ ಮಹಾಂತಸ್ತರಿಮತ್ರ ಯಾತಾ}\\\textbf{ನಿಧಾಯ ಧೀರಾಃ ಸುಕಥಾಭಿಧಾಂ ಚ~।। ೫~।।}
\end{verse}

ದಟ್ಟವಾದ ಕತ್ತಲೆಯನ್ನು ನಾಶಗೊಳಿಸುವಲ್ಲಿ ಸೂರ್ಯನಂತಿರುವವನೇ, ನಾನಾವಿಧವಾದ ಅನುಗ್ರಹ ಮಾಡುವ ಸಮರ್ಥಶಾಲಿಯೇ, ಸಂಸಾರವೆಂಬ ಅಜಗರದಿಂದ ಕಚ್ಚಲ್ಪಟ್ಟ, ಅತಿ\-ದೀನರಾದ, ಕೃತಘ್ನರಾದ ನಮ್ಮೆಲ್ಲರಿಗೂ ನಿನ್ನ ಕರುಣೆಯೆಂಬ ಅಮೃತವನ್ನು ನೀಡಿ ರಕ್ಷಿಸು. ಹಿಂದಿನಿಂದಲೂ ಬಹಳ ಜನರು ಪರಮಾತ್ಮನ ಕಥಾಮೃತವನ್ನು ಪಾನಮಾಡಿ ಸಂಸಾರವೆಂಬ ಸಾಗರವನ್ನು ದಾಟಿದ್ದಾರೆ.

\begin{verse}
\textbf{ಏವಂ ಪುರಾ ಭಾಗವತಾ ಭವಾದೃಶಾಃ} \\\textbf{ತೀರ್ತ್ವಾ ಸ್ವಯಂ ತಾರಯಿತಾರ ಏತೇ~।}\\\textbf{ಭವಾದೃಶಾನಾಂ ಖಲು ಸೇವಯೈವ} \\\textbf{ಲಭ್ಯಾಃ ಪುಮರ್ಥಾ ಇತಿ ವೇದವಾದಃ~।। ೬~।।}
\end{verse}

ಹೀಗೆ ಹಿಂದೆ ನಿನ್ನಂತೆ ಇರುವ ಭಾಗವತೋತ್ತಮರು ಸಂಸಾರಸಮುದ್ರವನ್ನು ಸ್ವಯಂ ದಾಟಿ ಇತರರನ್ನು ಅದರಿಂದ ದಾಟಿಸಿರುವರು. ನಿನ್ನಂತೆ ಇರುವ ಸಜ್ಜನರ ಸೇವೆಯು ಚತುರ್ವಿಧ ಪುರುಷಾರ್ಥಗಳನ್ನೂ ದೊರಕಿಸುವುದೆಂದು ವೇದಗಳು ನುಡಿಯುತ್ತಿವೆ.

\begin{verse}
\textbf{ಭವಾಪಹೋ ಮಾಧವ ಏಷ ವಿಷ್ಣು\enginline{-}} \\\textbf{ರ್ಭವತ್ಸು ಯಜ್ಞೇಷು ಚ ಗೋಷು ಯಸ್ಮಾತ್~।}\\\textbf{ಭವಾದೃಶಾನಾಂ ಖಲು ಸಂಗಮೋಽಪಿ} \\\textbf{ಹ್ಯನೇಕಜನ್ಮಾರ್ಜಿತಪಾಪನಾಶಾತ್~।। ೭~।।}
\end{verse}

ಸಂಸಾರದುಃಖವನ್ನು ನಾಶಗೊಳಿಸುವ ಲಕ್ಷ್ಮಿ ಪತಿಯಾದ ವಿಷ್ಣುವು ಯಜ್ಞಗಳಲ್ಲಿಯೂ, ಗೋವುಗಳಲ್ಲಿಯೂ ಮತ್ತು ನಿಮ್ಮಂತಹ ಸಜ್ಜನರಲ್ಲಿಯೂ ವಿಶೇಷವಾದ ಸನ್ನಿಧಾನ ಇಟ್ಟಿರುವ ಕಾರಣದಿಂದ ತಮ್ಮಂಥವರ ದರ್ಶನ ಸಹವಾಸಗಳು ಅನೇಕ ಜನ್ಮಗಳ ಪಾಪಗಳನ್ನು ನಾಶಮಾಡುತ್ತವೆ.

\begin{verse}
\textbf{ಸಮಸ್ತ ಪುಣ್ಯೋದಯಪಾದಪಶ್ಚ} \\\textbf{ಫಲತ್ಯಹೋ ನಾನ್ಯ ಉಪಾಯಕೈಶ್ಚ~।}\\\textbf{ಮಹಾನುಭಾವಾಃ ಸಮದರ್ಶಿನೋ ಯೇ} \\\textbf{ಸುದುರ್ಲಭಾ ಭಾಗವತಾ ಹಿ ಲೋಕೇ~।। ೮~।।}
\end{verse}

ಅನೇಕ ಪುಣ್ಯಕರ್ಮಗಳ ಫಲವಾಗಿ ಮರಗಳು ಫಲಗಳಿಂದ ಪೂರ್ಣವಾಗುತ್ತವೆ, ಬೇರೆ ಕಾರಣಗಳಿಂದ ಅಲ್ಲ. ಯಾರಲ್ಲಿಯೂ ದ್ವೇಷ ಕ್ರೌರ್ಯ ಮುಂತಾದ ಭಾವನೆಗಳಿಲ್ಲದೇ ಎಲ್ಲರನ್ನೂ ಕರುಣೆಯಿಂದ ಕಾಣುವ ಭಗವದ್ಭಕ್ತರಾದ ಮಹಾನುಭಾವರಾದ ನಿಮ್ಮಂಥವರು ಲೋಕದಲ್ಲಿ ದುರ್ಲಭರು.

\begin{verse}
\textbf{ದುರಾತ್ಮನಾಂ ಸಂಗವತಾಂ ಹಿ ಸಂಗಃ} \\\textbf{ತಮಸ್ಯ ಪಾರೇ ಪತನಾಯ ಸೋ ಭವೇತ್~।}\\\textbf{ಗರೇಣ ಪೃಕ್ತಾಮೃತಸಂಘಿಸಂಗಿನೋ} \\\textbf{ಭೋಗಾಃ ಸಮಸ್ತಾಃ ಶುಭದೇಹನಾಶಕಾಃ~।। ೯~।।}
\end{verse}

ವಿಷಪೂರಿತವಾದ ಆಹಾರವು ಈ ಪವಿತ್ರವಾದ ದೇಹನಾಶಕ್ಕೂ ದೇಹದಿಂದ ಅನುಭವಿಸಬಹುದಾದ ಭೋಗಗಳ ನಾಶಕ್ಕೂ ಕಾರಣವಾಗುವ ರೀತಿಯಲ್ಲಿ ದುರ್ಜನರ ಸಹವಾಸವು ತಮಸ್ಸೆಂಬ ನರಕದಲ್ಲಿ ಬೀಳಲು ಸಾಧನ.

\begin{verse}
\textbf{ಅಂತಃಕರಣನೈರ್ಮಲ್ಯೇ ನಿದಾನಂ ಸತ್ಕ್ರಿಯಾ ಯಥಾ~।}\\\textbf{ಪುಮರ್ಥಾನಾಂ ನಿದಾನಂ ವಾ ಮಹತ್ಸೇವಾ ಸಮೀರಿತಾ~।। ೧೦~।।}
\end{verse}

ಅಂತಃಕರಣವು ಪರಿಶುದ್ಧವಾಗಿರಲು ವೇದೋಕ್ತವಾದ ಸತ್ಕರ್ಮಗಳು ಹೇಗೆ ಕಾರಣವೋ, ಅದರಂತೆ ಧರ್ಮಾದಿ ಪುರುಷಾರ್ಥಗಳ ಪ್ರಾಪ್ತಿಗೆ ಮಹಾತ್ಮರ ಸೇವೆಯೇ ಕಾರಣವೆಂದು\break ಸಚ್ಛಾಸ್ತ್ರಗಳಲ್ಲಿ ನಿರೂಪಿತವಾಗಿದೆ.

\begin{verse}
\textbf{ಕ್ಷೇತ್ರಾಣಿ ತೀರ್ಥಾನಿ ಚ ದೈವತಾನಿ} \\\textbf{ಪುನಂತಿ ಶಶ್ವತ್ ಖಲು ಸೇವಯಾಂತೇ~।}\\\textbf{ಗತಾಂಹಸೋ ವೀತಮಲಾಃ ಸುಶಿಕ್ಷಿತಾಃ} \\\textbf{ಪುನಂತಿ ಲೋಕಾನ್ ಖಲು ದರ್ಶನಾದನು~।। ೧೧~।।}
\end{verse}

ಕಾಶಿ, ರಾಮೇಶ್ವರಾದಿ ಪುಣ್ಯ ಕ್ಷೇತ್ರಗಳು, ದೇವತಾಸಂಬಂಧವಾದ ಗಂಗಾದಿ ತೀರ್ಥಗಳೂ ಬಹಳಕಾಲ ಸೇವೆಮಾಡಿದ ನಂತರ ಅಂತಃಕರಣಶುದ್ಧಿಯೇ ಮುಂತಾದುವನ್ನು ಕೊಟ್ಟು ನಮ್ಮನ್ನು ಪವಿತ್ರರನ್ನಾಗಿ ಮಾಡುತ್ತವೆ. ಆದರೆ ದೋಷ ವರ್ಜಿತರೂ, ಜ್ಞಾನಿಗಳೂ, ನಿಷ್ಕಲ್ಮಷ ಮನಸ್ಸುಳ್ಳವರೂ ಆದ ಸತ್ಪುರುಷರು ತಮ್ಮ ದೃಷ್ಟಿಯಿಂದ ಅವಲೋಕನ ಮಾತ್ರದಿಂದ ಜನರನ್ನು ಪಾವನರನ್ನಾಗಿ ಮಾಡುವರು.

\begin{verse}
\textbf{[ವಿಶೇಷಾಂಶ: ಅಗ್ನಿಚಿತ್ ಕಪಿಲಾ ಸತ್ರೀ ರಾಜಾ ಭಿಕ್ಷುರ್ಮಹೋದಧಿಃ~।}\\\textbf{ದೃಷ್ಟಮಾತ್ರಾಃ ಪುನಂತ್ಯೇತೇ ತಸ್ಮಾತ್ ಪಶ್ಯೇತ ನಿತ್ಯಶಃ~।।} (ಸುಭಾಷಿತ)
\end{verse}

ಹೋಮಸಲ್ಪಡುತ್ತಿರುವ ಅಗ್ನಿ, ಗೋವು, ಯಾಗಮಾಡುತ್ತಿರುವವನು, ರಾಜ, ಯತಿ, ಸಮುದ್ರ-ಇವುಗಳನ್ನು ನೋಡುವ ಮಾತ್ರದಿಂದಲೇ ಜನರು ಪವಿತ್ರರಾಗುತ್ತಾರೆ; ಆದುದರಿಂದ ನಿತ್ಯವೂ ಇವರ ದರ್ಶನ ಮಾಡಬೇಕು.

\begin{verse}
\textbf{ನ ಹ್ಯಮ್ಮಯಾನಿ ತೀರ್ಥಾನಿ ನ ದೇವಾ ಮೃಚ್ಛಿಲಾಮಯಾಃ~।}\\\textbf{ತೇ ಪುನಂತ್ಯುರುಕಾಲೇನ ದರ್ಶನಾದೇವ ಸಾಧವಃ~।।} (ಭಾಗವತ)
\end{verse}

ಜಲಮಯವಾಗಿರುವ ಪವಿತ್ರ ತೀರ್ಥಗಳು, ಮಣ್ಣು-ಕಲ್ಲು-ಲೋಹಗಳಿಂದ ನಿರ್ಮಿತವಾದ ದೇವತಾವಿಗ್ರಹಗಳೂ ಬಹಳಕಾಲ ಸೇವಿಸಿದ ನಂತರ ನಮ್ಮನ್ನು ಪರಿಶುದ್ಧಗೊಳಿಸುತ್ತವೆ. ಆದರೆ ಸತ್ಪುರುಷರು ತಮ್ಮ ದರ್ಶಮಾತ್ರದಿಂದಲೇ ನಮ್ಮನ್ನು ಶುದ್ಧಿಗೊಳಿಸುತ್ತಾರೆ].

\begin{verse}
\textbf{ತೀರ್ಥಾನಾಂ ಸೇವಯಾ ಕರ್ಮಕರಣಾತ್ಸಂಗತಿಃ ಸತಾಮ್~।}\\\textbf{ಸತ್ಸಂಗಃ ತತ್ ಕ್ಷಣಾದೇವ ಪುನಾತ್ಯೇವ ನ ಸಂಶಯಃ~।। ೧೨~।।}
\end{verse}

ಗಂಗಾದಿ ತೀರ್ಥಗಳನ್ನು ಬಹುಕಾಲ ಸೇವಿಸುವುದರಿಂದ ಸತ್ಕರ್ಮಮಾಡಲು ಬುದ್ಧಿ ಹುಟ್ಟುತ್ತದೆ. ಸತ್ಕರ್ಮಾಚರಣೆಯಿಂದ ಸಜ್ಜನರ ಸಹವಾಸವು ಲಭಿಸುತ್ತದೆ. ಸಜ್ಜನರ ಸಹವಾಸವು ಮನುಷ್ಯನನ್ನು ಬಹು ಶೀಘ್ರವಾಗಿ ಪವಿತ್ರನನ್ನಾಗಿ ಮಾಡುತ್ತದೆ. ಈ ವಿಷಯದಲ್ಲಿ ಸಂಶಯವಿಲ್ಲ.

\begin{verse}
\textbf{[ವಿಶೇಷಾಂಶ:}\\\textbf{ಗಂಗಾದಿತೀರ್ಥಸೇವಾ ಚ ಫಲಂ ದಾತುಂ ಯದೋನ್ಮುಖಾ~।}\\\textbf{ಜನ್ಮಾಂತರೇಷು ಜಾಯತೇ ತದಾ ಸತ್ಸಂಗತಿರ್ಭವೇತ್~।।}\\\textbf{ಯದೈವ ಸ್ಯಾತ್ಸತಾಂ ಸಂಗಸ್ತದೈವಾತ್ಮಾ ಪ್ರಸೀದತಿ~।।} \vauthor{\enginline{-}ವಿಷ್ಣು ರಹಸ್ಯ} 
\end{verse}

ಗಂಗಾದಿ ತೀರ್ಥಗಳ ಸೇವೆಯು ಜನ್ಮಾಂತರದಲ್ಲಿ ಫಲಕೊಡಲು ಸಿದ್ಧವಾದಾಗ, ಸಾಧು ಸಜ್ಜನರ ಸಹವಾಸವು ಲಭಿಸುತ್ತದೆ. ಯಾವಾಗ ಸಾಧು-ಸಜ್ಜನರ ಸಹವಾಸವು ಲಭಿಸುವುದೋ, ಆಗಲೇ (ಆಗ ಮಾತ್ರವೇ) ಶ‍್ರೀಹರಿಯು ಅನುಗ್ರಹ ಮಾಡುತ್ತಾನೆ].

\begin{verse}
\textbf{ನಿರ್ಧೂತಪಾಪಾನಸ್ಮಾಕಂ ದರ್ಶನಾತ್ ಜಾತಸಂಪದಃ~।}\\\textbf{ಗತದುಷ್ಟ ಸ್ವಭಾವಾಂಶ್ಚ ತ್ವಮುದ್ಧರ್ತುಮಿಹಾರ್ಹಸಿ~।। ೧೩~।।}
\end{verse}

ನಿನ್ನ ದರ್ಶನಲಾಭದಿಂದ ನಮ್ಮ ಸಕಲ ಪಾಪಗಳೂ ನಾಶವಾದುವು. ನಮ್ಮ ನೀಚ ಸ್ವಭಾವವೂ ಸಹ ನಾಶವಾಯಿತು. ಇಂತಹ ಸಂಪದ್ಯುಕ್ತರಾದ ನಮ್ಮನ್ನು ಉದ್ಧರಿಸಲು ಸಮರ್ಥನಾಗಿರುವಿ.

\begin{verse}
\textbf{ಇತಿ ತೈಃ ಪ್ರಾರ್ಥಿತಃ ಸಮ್ಯಕ್ ಪ್ರತ್ಯುವಾಚ ಮುನೀಶ್ವರಃ~।}
\end{verse}

ಹೀಗೆ ಅವರಿಂದ ಸ್ತೋತ್ರ ಮಾಡಿಸಿಕೊಳ್ಳಲ್ಪಟ್ಟ ಮುನಿಪುತ್ರನು ಉತ್ತರ ನೀಡಿದನು.

\begin{verse}
\textbf{ಮಾ ಬಿಭ್ಯತ ಭವಂತಶ್ಚ ಮಾ ವಿಷೀದಥ ವಾ ಕ್ವಚಿತ್~।। ೧೪~।।} 
\end{verse}

\begin{verse}
\textbf{ಭವತಾಂ ಭವಿತಾ ವಿಷ್ಣೋಃ ಪ್ರಸಾದೋ ಬಂಧಮೋಚನಃ~।}\\\textbf{ಅವಿದೂರೇಣ ಕಾಲೇನ ತ್ರಸ್ತಂ ಮಾಮಶರೀರವಾಕ್~।। ೧೫~।।}
\end{verse}

ಪಿಶಾಚಿಗಳಿರಾ, ಇನ್ನು ನೀವು ಭಯಪಡಲೂ, ದುಃಖಿಸಲೂ ಕಾರಣವಿಲ್ಲ. ಬಂಧನದಿಂದ ಬಿಡುಗಡೆ ಮಾಡಲು ಸಮರ್ಥವಾದ ಶ‍್ರೀವಿಷ್ಣುವಿನ ಅನುಗ್ರಹವು ಶೀಘ್ರದಲ್ಲಿಯೇ ನಿಮಗೆ ದೊರೆಯಲಿದೆ. ನಿಮ್ಮನ್ನೆಲ್ಲ ನೋಡಿ ಹೆದರಿದ ನನಗೆ ಅಶರೀರವಾಣಿಯು ಹೀಗೆ ಹೇಳಿತು.

\begin{verse}
\textbf{ಆಹ ಮಾ ಭೈಷೀಸ್ತಾತೇತಿ ಸಂಬೋಧ್ಯ ನೃಹರೇರ್ಮನುಮ್~।}\\\textbf{ಉಪದಿಶ್ಯ ಭವದ್ಬಂಧಾನ್ಮೋಚಕಂ ಫಲದಂ ಶುಭಮ್~।। ೧೬~।।}
\end{verse}

ಮುನಿಪುತ್ರನೇ, ಪಿಶಾಚಿಗಳನ್ನು ಕಂಡು ಭಯಪಡಬೇಡ ಎಂದು ಹೇಳಿ ಬಂಧ ಮೋಚಕವಾದ, ಶುಭಕರವಾದ, ಸಕಲ ಇಷ್ಟಾರ್ಥಗಳನ್ನೂ ಕೊಡುವ ನರಸಿಂಹ ಮಂತ್ರವನ್ನು ಅಶರೀರ\-ವಾಣಿಯು ಉಪದೇಶಿಸಿತು.

\begin{verse}
\textbf{ಮಾಘೋಽಯಂ ವರ್ತತೇ ಮಾಸೋ ಮಘಾಯುಕ್ತಾ ಚ ಪೌರ್ಣಿಮಾ~।}\\\textbf{ನದೀ ಚ ಕೌಶಿಕೀನಾಮ ಪ್ರಾತಃಕಾಲೋ ಮಹಾನಯಮ್~।। ೧೭~।।}
\end{verse}

ಈಗ ಮಾಘಮಾಸ, ಈ ದಿನ ಮಘಾನಕ್ಷತ್ರದಿಂದ ಯುಕ್ತವಾದ ಪೌರ್ಣಿಮೆ, ಪಕ್ಕದಲ್ಲಿ ಹರಿಯುವುದು ಕೌಶಿಕೀ ಎಂಬ ನದಿ, ಈಗ ರಮಣೀಯವಾದ ಪ್ರಾತಃಕಾಲ.

\begin{verse}
\textbf{ತತ್ರ ಸ್ನಾತ್ವಾ ಭವಾನ್ಪೂರ್ವಂ ಪಶ್ಚಾದೇತಾಂಶ್ಚ ಸಿಂಚಯ~।}\\\textbf{ತೇನ ತೇ ಶುದ್ಧ ಭಾವಾಃ ಸ್ಯುಃ ಭವತಶ್ಚ ಯಶೋಽಮಲಮ್~।। ೧೮~।।}
\end{verse}

ಆದುದರಿಂದ ಮೊದಲು ನೀನು ನದಿಯಲ್ಲಿ ಸ್ನಾನಮಾಡು, ನಂತರ ನದೀ ನೀರನ್ನು ಪಿಶಾಚಿಗಳ ಮೇಲೆ ಸಿಂಪಡಿಸು. ಅದರಿಂದ ಎಲ್ಲರೂ ಪರಿಶುದ್ಧವಾದ ಅಂತಃಕರಣಯುಕ್ತ\-ರಾಗುತ್ತಾರೆ. ಹೀಗೆ ಮಾಡುವುದರಿಂದ ನಿನಗೂ ಯಶಸ್ಸು ಲಭಿಸುತ್ತದೆ.

\begin{verse}
\textbf{ಅದ್ಯೈವ ಭಗವಾನ್ ವಿಷ್ಣುಃ ಪ್ರಾಯಃ ಶ್ರೇಯೋ ವಿಧಾಸ್ಯತಿ~।।}
\end{verse}

ಬಹುಶಃ ಈ ದಿನವೇ ಆ ಷಡ್ಗುಣೈಶ್ವರ್ಯಪೂರ್ಣನಾದ ವಿಷ್ಣುವು ಎಲ್ಲರಿಗೂ ಶ್ರೇಯಸ್ಸನ್ನು ಅನುಗ್ರಹಿಸುತ್ತಾನೆ \enginline{-} ಹೀಗೆಂದು ಅಶರೀರವಾಣಿಯು ನುಡಿಯಿತು.

\begin{flushleft}
\textbf{ಬ್ರಹ್ಮೋವಾಚ\enginline{-}}
\end{flushleft}

\begin{verse}
\textbf{ಇತ್ಯಾಶ್ವಾಸ್ಯ ತತಃ ತಸ್ಮಾತ್ ಕೌಶಿಕ್ಯಾಂ ಸ ಮುನೀಶ್ವರಃ~।। ೧೯~।।} 
\end{verse}

\begin{verse}
\textbf{ಪ್ರತ್ಯೇಕಂ ನಾಮಗೋತ್ರಾಣಿ ಮುನಿಃ ಶುದ್ಧೇನ ಚೇತಸಾ~।}\\\textbf{ಸಮುಚ್ಚಾರ್ಯಾಕರೋತ್ ಸ್ನಾನಂ ಮೃದಾಲಿಪ್ಯ ಮಹಾಮತಿಃ~।। ೨೦~।।}
\end{verse}

ಬ್ರಹ್ಮದೇವರು (ನಾರದರಿಗೆ) ನುಡಿದರು:

ಈ ರೀತಿಯಿಂದ ಮುನಿಪುತ್ರನು ಎಲ್ಲರನ್ನೂ ಸಮಾಧಾನಗೊಳಿಸಿ, ತಾನು ಪ್ರತ್ಯೇಕ ಪ್ರತ್ಯೇಕವಾಗಿ ಪಿಶಾಚಿಗಳ ನಾಮ-ಗೋತ್ರಗಳನ್ನು ಉಚ್ಚರಿಸಿ ಪರಿಶುದ್ಧವಾದ ಮನಸ್ಸಿನಿಂದ ಕೌಶಿಕೀ ನದಿಯಲ್ಲಿ ಮೃತ್ತಿಕಾಲೇಪನ ಮಾಡಿಕೊಂಡು ಸ್ನಾನ ಮಾಡಿದನು.

\begin{verse}
\textbf{ದದೌ ತತ್ಕರ್ಮಜಂ ಪುಣ್ಯಂ ತಸ್ಮಾನ್ಮುಕ್ತಿಃ ಕ್ಷಣಾದಭೂತ್~।}\\\textbf{ಶರೀರಾಣ್ಯಪಿ ತೇ ಸರ್ವೇ ಪೂರ್ವಕರ್ಮಾರ್ಜಿತಾನಿ ಚ~।। ೨೧~।। }
\end{verse}

\begin{verse}
\textbf{ವಿಹಾಯ ಪೂರ್ವದೇಹಾಂಶ್ಚ ಪ್ರಾಪುಃ ಶುದ್ದೇನ ಕರ್ಮಣಾ~।}\\\textbf{ತತಃ ಪರಸ್ಪರಾಲಾಪಮುದಿತೇಂದ್ರಿಯಗಾತ್ರಕಾಃ~।। ೨೨~।। }
\end{verse}

\begin{verse}
\textbf{ಸಾಷ್ಟಾಂಗಂ ಚ ಪ್ರಣೇಮುಸ್ತೇ ಮುನಿಂ ತಂ ಸಂಶಿತವ್ರತಮ್~।}\\\textbf{ತಾನುತ್ಥಾ ಪ್ಯ ಸುಯಜ್ಞೋsಪಿ ಸ್ವಸ್ವಜೇ ಪ್ರೇಮಕಾತರಃ~।। ೨೩~।।}
\end{verse}

ಆ ಸ್ನಾನದ ಫಲವನ್ನು ಆ ಪಿಶಾಚಿಗಳಿಗೆ ಧಾರೆಯೆರೆದು ಕೊಟ್ಟನು. ಇದರ ಫಲವಾಗಿ ಆ ಪಿಶಾಚಿಗಳಿಗೆ ಕೂಡಲೇ ಆ ಜನ್ಮದಿಂದ ಬಿಡುಗಡೆಯಾಯಿತು. ಎಲ್ಲ ಪಾಪಗಳಿಂದಲೂ ವಿಮುಕ್ತರಾದ ಅವರು ತಮ್ಮ ತಮ್ಮ ಹಿಂದಿನ ದೇಹಗಳನ್ನು ಪಡೆದರು. ನಂತರ ಅವರೆಲ್ಲರೂ ತಮ್ಮ ತಮ್ಮಲ್ಲಿಯೇ ಸಂತೋಷದಿಂದ ಮಾತುಗಳನ್ನಾಡಿಕೊಂಡರು. ವ್ರತನಿಷ್ಠನಾದ ಮುನಿಪುತ್ರನಿಗೆ ಸಾಷ್ಟಾಂಗ ನಮಸ್ಕಾರಮಾಡಿದರು. ಸುಯಜ್ಞನಾದರೋ ಅವರನ್ನು ಎಬ್ಬಿಸಿ ಪ್ರೀತ್ಯಾದರಗಳಿಂದ ತಬ್ಬಿಕೊಂಡನು.

\begin{verse}
\textbf{ಏತಸ್ಮಿನ್ನಂತರೇ ಚಾಸೀತ್ ದುರ್ನಿಮಿತ್ತಂ ಮಹಾದ್ಭುತಮ್~।}\\\textbf{ಉಲ್ಕಾಪಾತಾ ನಿಪೇತುಶ್ಚ ದಿಶೋ ಧೂಮ್ರಾಶ್ಚ ಸರ್ವಶಃ~।। ೨೪~।। }
\end{verse}

\begin{verse}
\textbf{ಗೋಮಾಯವೋ ದಿವಾಕ್ರೋಶನ್ ಕುರ್ವಂತ್ಯೂರ್ಧ್ವ ಮುಖಾಸ್ತದಾ~।}\\\textbf{ರಕ್ತವೃಷ್ಟಿಸ್ತದಾ ಚಾಸೀತ್ ದಿವಾರಾತ್ರಂ ಗುಹಾಶಯಾತ್~।। ೨೫~।।}
\end{verse}

ಈ ಕಾಲದಲ್ಲಿ ಅನೇಕ ಅಪಶಕುನಗಳು ಕಾಣಿಸಿಕೊಂಡವು. ಉಲ್ಕಾಪಾತಗಳಾದವು. ದಿಕ್ಕುಗಳೆಲ್ಲವೂ ಹೊಗೆಯಿಂದ ಆವರಿಸಿಕೊಳ್ಳಲ್ಪಟ್ಟವು. ನರಿಗಳು ಆಕಾಶದ ಕಡೆಗೆ ತಲೆಯೆತ್ತಿಕೊಂಡು ಕೆಟ್ಟ ಧ್ವನಿಯಿಂದ ಕೂಗಿದುವು. ರಾತ್ರಿ-ಹಗಲು ಆಕಾಶದಿಂದ ರಕ್ತದ ಹನಿಗಳು ಬಿದ್ದವು.

\begin{verse}
\textbf{ಚಲಂತಿ ವಾಮಗಾತ್ರಾಣಿ ಶಾಂಡಿಲ್ಯಸ್ಯ ಮಹಾತ್ಮನಃ~।}\\\textbf{ತಥಾ ದಕ್ಷಿಣಗಾತ್ರಾಣಿ ಧರ್ಮಪತ್ನ್ಯಾ ಮಹಾತ್ಮನಃ~।। ೨೬~।।}
\end{verse}

ಮಹಾತ್ಮರಾದ ಶಾಂಡಿಲ್ಯರ ದೇಹದ ಎಡಭಾಗದ ಅವಯವಗಳು ಅದರಿದವು. ಹಾಗೆಯೇ ಅವರ ಧರ್ಮಪತ್ನಿ ಯ ಶರೀರದ ಬಲಭಾಗದ ಅವಯವಗಳು ಹಾರಿದವು.

\begin{verse}
\textbf{ತದಾ ಭಯಾಕುಲಾ ಪತ್ನೀ ಸ್ವಾಶ್ರಮೇ ಜಾತಮದ್ಭುತಮ್~।}\\\textbf{ದೃಷ್ಟ್ವಾಸರ್ವಮಿದಂ ಪ್ರಾಹ ಪತಿಂ ಸಂಧ್ಯಾ ಪರಾಯಣಮ್~।। ೨೭~।।}
\end{verse}

ತಮ್ಮ ಆಶ್ರಮದಲ್ಲಿ ನಡೆಯುತ್ತಿದ್ದ ಅದ್ಭುತಗಳನ್ನು ನೋಡಿ ಭಯಪಟ್ಟ ಶಾಂಡಿಲ್ಯರ ಪತ್ನಿಯು ಸಂಧ್ಯಾವಂದನಾದಿ ಕರ್ಮಗಳಲ್ಲಿ ನಿರತರಾಗಿದ್ದ ತಮ್ಮ ಪತಿಯನ್ನು ಕುರಿತು ಹೇಳಿದರು:

\begin{verse}
\textbf{ಮುನೇಽದ್ಭು ತಾನ್ಯತ್ರ ಭವಂತಿ ಚಾಶ್ರಮೇ}\\\textbf{ನ ಭೂತಪೂರ್ವಾಣಿ ಚ ಹಾನಿದಾನಿ~।}\\\textbf{ಪುತ್ರಾ ಇಮೇ ಮೇ ಸುಖಮೇವ ಚಾಂತಃ} \\\textbf{ಸಮಾಸತೇ ತತ್ಕೃಪಯಾ ನಿಯುಕ್ತಾಃ~।। ೨೮~।।}
\end{verse}

ಹಿಂದೆ ಎಂದೂ ಕಾಣಿಸದಿದ್ದ ಭಯಂಕರವಾದ ಅಪಶಕುನಗಳು ನಮ್ಮ ಆಶ್ರಮದಲ್ಲಿ ಕಾಣಲಾರಂಭಿಸಿವೆ. ಇವು ಏನೋ ವಿಪತ್ತನ್ನು ಸೂಚಿಸುತ್ತವೆ. ತಮ್ಮ ಅನುಗ್ರಹಕ್ಕೆ ಪಾತ್ರರಾದ ಈ ಮಕ್ಕಳೇನೋ ಸುಖದಿಂದ ಇರುತ್ತಾರೆ.

\begin{verse}
\textbf{ಶಿವಾ ಇಮೇ ಪಾದಪಾಶ್ಚ ಗಾವಶ್ಚಾಗ್ನಿಃ ತಥೌಷಧೀಃ~।}\\\textbf{ಶಿಷ್ಯೋ ವನಂ ಗತೋ ನಾಽಗಾತ್ ಕಾಲೋ ಭೂಯಾನ್ ಗತೋ ಬತ~।।}
\end{verse}

ಇಲ್ಲಿರುವ ಗಿಡ-ಮರಗಳೂ, ಆಕಳುಗಳೂ, ಅಗ್ನಿ ಕುಂಡಗಳೂ, ವನಸ್ಪತಿಗಳೂ ಸುಖವಾಗಿವೆ. ಆದರೆ ವನಕ್ಕೆ ಹೋದ ಸುಯಜ್ಞನು ಎಷ್ಟು ಹೊತ್ತಾದರೂ ಇನ್ನೂ ಬಂದಿಲ್ಲ.

\begin{verse}
\textbf{ಕಿಂ ವ್ಯಾಘ್ರೇಃ ಪೀಡಿತೋ ಬಾಲಃ ಕಿಂ ವಾ ಮತ್ತೈರ್ಗಜೈರ್ಮೃತಃ~।}\\\textbf{ಅಶಿವಂ ವಾ ಭವೇತ್ತಸ್ಯ ನಾನ್ಯಥಾ ತತ್ ವೃಥಾ ಭವೇತ್~।। ೩೦~।।}
\end{verse}

ಸುಯಜ್ಞನು ಹುಲಿಗಳ ಹಿಂಸೆಗೆ ಗುರಿಯಾಗಿದ್ದಾನೆಯೋ, ಇಲ್ಲವೇ ಮದಿಸಿದ ಆನೆಗಳಿಂದ ದುರ್ಮರಣಕ್ಕೆ ಈಡಾಗಿರುವನೋ ಗೊತ್ತಿಲ್ಲ. ಏನೇ ಆಗಲೀ ಈ ಅಪಶಕುನಗಳು ವ್ಯರ್ಥವಾಗುವುದಿಲ್ಲ. ಏನೋ ಅಮಂಗಳವು ಆಗುತ್ತದೆ.

\begin{verse}
\textbf{ಇತಿ ತಸ್ಯಾ ವಚಃ ಶ್ರುತ್ವಾ ಶಾಂಡಿಲ್ಯೋ ಮುನಿಸತ್ತಮಃ~।}\\\textbf{ಸ ದೃಷ್ಟ್ವಾ ತಾನ್ಯ ಭದ್ರಾಣಿ ಸ್ವೀಯಾನ್ಯ ಪ್ಯ ಶಿವಾನಿ ಚ~।। ೩೧~।।}
\end{verse}

ಹೀಗೆಂಬ ತಮ್ಮ ಪತ್ನಿಯ ಮಾತನ್ನು ಕೇಳಿದ ಮುನಿಶ್ರೇಷ್ಠರಾದ ಶಾಂಡಿಲ್ಯರು ಆ ಅಪಶಕುನಗಳನ್ನೂ, ತಮಗೇ ಆದ ಎಡ ಅವಯವಗಳು ಅದರುವುದನ್ನೂ ವಿಮರ್ಶಿಸಿದರು.

\begin{verse}
\textbf{ಶಿಷ್ಯಸ್ಯ ವಿಪದಂ ಮೇನೇ ಸಂದ್ರಷ್ಟುಮಗಮನ್ ಮುನಿಃ~।}\\\textbf{ಪುತ್ರೈಃ ಕತಿಪಯೈಃ ಶಿಷ್ಯೈಃ ತತಃ ಶಂಕಾಕುಲೇಕ್ಷಣಃ~।। ೩೨~।।} 
\end{verse}

\begin{verse}
\textbf{ದೃಷ್ಟ್ವಾ ದೃಷ್ಟ್ವಾಚ ಶಿಷ್ಯಸ್ಯ ಪದಾನ್ಯೇವ ಚ ಪದ್ಧತೌ~।}\\\textbf{ಗಚ್ಛಂಸ್ತೇನೈವ ಮಾರ್ಗೇಣ ದಿವ್ಯ ಪೂರುಷಸೇವಿತಮ್~।। ೩೩~।।} 
\end{verse}

\begin{verse}
\textbf{ಶಿಷ್ಯಂ ದದರ್ಶ ದೇವಾಭಂ ರಾಹೋರ್ಮುಕ್ತಮಿವೋಡುಪಮ್~।}\\\textbf{ಆಶ್ಲಿ ಪ್ಯಾಶ್ಲಿಶ್ಯ ತಂ ಶಿಷ್ಯಂ ಮೂರ್ಧನ್ಯಾಘ್ರಾಯ ತಂ ಮುಹುಃ~।। ೩೪~।।}
\end{verse}

ಶಿಷ್ಯನಿಗೆ ಏನೋ ವಿಪತ್ತು ಸಂಭವಿಸಿರುವುದಾಗಿ ತಿಳಿದ ಶಾಂಡಿಲ್ಯರು, ತಮ್ಮ ಮಕ್ಕಳನ್ನೂ, ಕೆಲವು ಜನ ಶಿಷ್ಯರನ್ನೂ ಕರೆದುಕೊಂಡು ದುಃಖಾಶ್ರುಗಳನ್ನು ಹೊಂದಿ, ಶಿಷ್ಯನ ಪಾದಗಳ ಗುರುತುಗಳನ್ನೇ ಅನುಸರಿಸುತ್ತಾ ಅರಣ್ಯದಲ್ಲಿ ಸಂಚರಿಸಿ ದಿವ್ಯಪುರುಷರಿಂದ ಸೇವಿತನಾದ ಸುಯಜ್ಞನನ್ನು ಕಂಡರು. ಅವನು ದೇವತೆಗಳಂತೆ ತೇಜಸ್ವಿಯಾಗಿದ್ದ. ರಾಹುವಿನಿಂದ ಬಿಡುಗಡೆ ಮಾಡಲ್ಪಟ್ಟ ಚಂದ್ರನಂತೆ ಇದ್ದ ಸುಯಜ್ಞನನ್ನು ಬಾರಿಬಾರಿಗೂ ತಬ್ಬಿಕೊಂಡು ವಾತ್ಸಲ್ಯದಿಂದ ಅವನ ತಲೆಯನ್ನು ಆಘ್ರಾಣಿಸಿದರು.

\begin{verse}
\textbf{ವಂದ್ಯಮಾನಶ್ಚ ತೈಃ ಸಾರ್ಧಂ ಶಾಂಡಿಲ್ಯೋ ವಾಕ್ಯ ಮಬ್ರವೀತ್~।}\\\textbf{ಕುತ ಆಸೀದಿಯಾನ್ ಕಾಲಃ ಪೂರ್ವಂ ನೈವ ವಿಲಂಬಿತಮ್~।। ೩೫~।।}
\end{verse}

ಸುಯಜ್ಞ ಮತ್ತು ಇತರರಿಂದ ನಮಸ್ಕರಿಸಿಕೊಳ್ಳಲ್ಪಟ್ಟ ಶಾಂಡಿಲ್ಯರು ತಮ್ಮ ಶಿಷ್ಯನನ್ನು ಕೇಳಿದರು: “ನೀನು ಎಂದೂ ಆಶ್ರಮಕ್ಕೆ ಬರಲು ಇಷ್ಟು ತಡಮಾಡಿರಲಿಲ್ಲ. ಇಂದು ಇಷ್ಟು ತಡ\-ವಾಗಲು ಕಾರಣವೇನು?

\begin{verse}
\textbf{ದುರ್ನಿಮಿತ್ತಾನ್ಯಹಂ ದೃಷ್ಟ್ವಾ ಅಪಿ ಶಂಕಾಸಮಾಕುಲಃ~।}\\\textbf{ಆಗತಂ ತು ಮಯಾಽರಣ್ಯಂ ಕಚ್ಚಿತ್ತೇ ಸ್ವಸ್ತಿ ಮೇ ವದ~।। ೩೬~।।}
\end{verse}

ಅನೇಕ ಅಪಶಕುನಗಳನ್ನು ನೋಡಿ ನಿನ್ನ ಬಗ್ಗೆ ತುಂಬ ಆತಂಕಗೊಂಡು ನಿನ್ನನ್ನು ನೋಡಲು ಅರಣ್ಯಕ್ಕೆ ಬಂದೆ; ನೀನು ಸುಖವಾಗಿರುವೆಯಾ?

\begin{verse}
\textbf{ಶ್ರುತ್ವಾ ತದ್ವಚನಂ ಪೂರ್ವಂ ಶಿಷ್ಯೋ ವೃತ್ತ ಮಚೋದಯತ್~।}\\\textbf{ತೇsಪಿ ಪ್ರೊಚುಶ್ಚ ವೃತ್ತಾಂತಂ ಪ್ರಣಮ್ಯ ಚ ಪುನಃ ಪುನಃ~।। ೩೭~।।}
\end{verse}

ಈ ಮಾತನ್ನು ಕೇಳಿದ ಸುಯಜ್ಞನು ತಮ್ಮ ಗುರುಗಳಿಗೆ ಎಲ್ಲ ಸಮಾಚಾರಗಳನ್ನೂ ಅರುಹಿದನು. ದಿವ್ಯಪುರುಷರೂ ಸಹ ತಮ್ಮ ತಮ್ಮ ವೃತ್ತಾಂಗಳನ್ನು ಶಾಂಡಿಲ್ಯರಿಗೆ ಬಾರಿಬಾರಿಗೂ ನಮಸ್ಕರಿಸುತ್ತ ಹೇಳಿದರು.

\begin{verse}
\textbf{ಶ್ರುತ್ವಾ ಶಿಷ್ಯವಿಪತ್ತಿಂ ಚ ಶ್ರುತ್ವಾ ತೇಷಾಂ ಚ ಸಂಸ್ಥಿ ತಿಮ್~।}\\\textbf{ದುರ್ನಿಮಿತ್ತಂ ತತಃ ಶ್ರುತ್ವಾ ವಿಸ್ಮಿತೋ ವಾಕ್ಯ ಮಬ್ರವೀತ್~।। ೩೮~।।}
\end{verse}

ಶಿಷ್ಯನಿಗೆ ಬಂದಿದ್ದ ವಿಪತ್ತನ್ನೂ, ಅದು ಪರಿಹಾರವಾದ ಬಗೆಯನ್ನೂ, ದಿವ್ಯ ಪುರುಷರ ಹಿಂದಿನ ಜನ್ಮಗಳ ವೃತ್ತಾಂತಗಳನ್ನೂ ಕೇಳಿ, ತನ್ನ ಆಶ್ರಮದಲ್ಲಿ ಆದ ಅಪಶಕುನಗಳ ನೆನಪನ್ನೂ ತಂದಕೊಂಡು ಶಾಂಡಿಲ್ಯರು ನುಡಿದರು:

\begin{verse}
\textbf{ತಾನಾಮಂತ್ರ್ಯ, ತತೋ ಶಿಷ್ಯಃ ಸಹಿತೋ ಗಂತುಮುದ್ಯತಃ~।}\\\textbf{ಸುಯಜ್ಞಂ ತೇ ಪುರಸ್ಕೃತ್ಯ ಸುನಮ್ರಾ ದಿವ್ಯ ಪೂರುಷಾಃ~।। ೩೯~।।}\\\textbf{ಶಾಂಡಿಲ್ಯಂ ತಂ ಮಹಾತ್ಮಾನಂ ಪಪ್ರಚ್ಛುರ್ಮುನಿಪುಂಗವಮ್~।।}
\end{verse}

ಆಶ್ರಮಕ್ಕೆ ತೆರಳಲು ಸಿದ್ಧನಾಗಿದ್ದ ಶಿಷ್ಯನಾದ ಸುಯಜ್ಞನನ್ನು ಸನ್ಮಾನಿಸಿ ಅತ್ಯಂತ ನಮ್ರ ಭಾವದಿಂದ ಆ ದಿವ್ಯ ಪುರುಷರು ಮಹಾತ್ಮರಾದ ಮುನಿಶ್ರೇಷ್ಠರಾದ ಶಾಂಡಿಲ್ಯರನ್ನು ಕುರಿತು ಕೇಳಿದರು:

\begin{flushleft}
\textbf{ದಿವ್ಯಪುರುಷಾ ಊಚುಃ\enginline{-}}
\end{flushleft}

\begin{verse}
\textbf{ದಿಷ್ಟ್ಯಾಽಸ್ಮಾಕಂ ಶಿಷ್ಯತೋ ಮುಕ್ತಿರಾಸೀತ್}\\\textbf{ಜಾತಂ ಯುಷ್ಮದ್ದರ್ಶನಂ ವಿಪ್ರವರ್ಯ~।। ೪೦~।।}
\end{verse}

\begin{flushleft}
ದಿವ್ಯಪುರುಷರು ನುಡಿದರು:
\end{flushleft}

ಬ್ರಾಹ್ಮಣಶ್ರೇಷ್ಠರೇ, ನಿಮ್ಮ ಶಿಷ್ಯನ ಅನುಗ್ರಹದಿಂದ ನಮಗೆ ನೀಚಯೋನಿಯಿಂದ ಬಿಡುಗಡೆಯಾಯಿತು. ನಮ್ಮ ಪೂರ್ವಪುಣ್ಯದ ಫಲವಾಗಿ ತಮ್ಮ ದರ್ಶನವು ಲಭಿಸಿತು.

\begin{verse}
\textbf{ಜ್ಞಾನಂ ಯಥಾ ನ ಸ್ಪೃಶತೇ ಚ ಮಾಯಯಾ}\\\textbf{ಭೂಯೋ ನಾಲಂ ಲಭ್ಯತೇ ಜನ್ಮಜಾತಮ್~। }\\\textbf{ಕೇನ ವಾ ಜಾಯತೇ ಜಂತುಃ ಕೇನ ವಾ ಬಧ್ಯತೇ ಪುನಃ~।। ೪೧~।। }
\end{verse}

\begin{verse}
\textbf{ಕೇನ ವಾ ಮುಚ್ಯತೇ ಬಂಧಾದೇತದ್ವಿಸ್ತಾರ್ಯ ನೋ ವದ~।}\\\textbf{ಇತಿ ನಮ್ರಧಿಯಾಂ ತೇಷಾಂ ವಚಃ ಶ್ರುತ್ವಾ ಮಹಾಮತಿಃ~।। ೪೨~।।}\\\textbf{ತಾನ್ ಪ್ರತ್ಯುವಾಚ ಶಾಂಡಿಲ್ಯಃ ಪ್ರಸನ್ನೇನಾಂತರಾತ್ಮನಾ~।।}
\end{verse}

ಅಜ್ಞಾನವು ಯಾವ ಕ್ರಮವನ್ನು ಕೈಕೊಂಡರೆ ಜ್ಞಾನವನ್ನು ನಾಶಮಾಡುವುದಿಲ್ಲ? ಅನೇಕ ಚೇತನರು ಯಾರಿಂದ ಜನ್ಮವನ್ನು ಪಡೆಯುತ್ತಾರೆ? ಜನ್ಮ ಪಡೆದವರು ಯಾವ ಪ್ರಕಾರದಿಂದ ಸಂಸಾರದಲ್ಲಿ ಬಂಧಿತರಾಗುತ್ತಾರೆ? ಏನು ಮಾಡಿದರೆ ಪುನಃ ಜನ್ಮವು ಬಾರದಂತೆ ಇರುತ್ತದೆ? ಯಾರಿಂದ ಜೀವರು ಬಂಧನದಿಂದ ಬಿಡುಗಡೆಯನ್ನು ಪಡೆಯುತ್ತಾರೆ \enginline{-}ಈ ವಿಷಯಗಳನ್ನು ನಮಗೆ ತಿಳಿಸಿರಿ. ಹೀಗೆಂದು ಕೇಳಿದ ದಿವ್ಯ ಪುರುಷರನ್ನು ಕುರಿತು ಪ್ರಸನ್ನ ಮನಸ್ಸಿನಿಂದ ಶಾಂಡಿಲ್ಯರು ಹೇಳಿದರು:

\begin{verse}
\textbf{ಬದ್ಧವೃಕ್ಷಾಶ್ರಯಾ ಹ್ಯೇಷಾ ಪ್ರಕೃತಿಶ್ಚ ಗುಣತ್ರಯೀ~।। ೪೩~।।}
\end{verse}

ಸಂಸಾರ ಬಂಧನವೆಂಬ ಈ ಜಗತ್ತೆಂಬ ವೃಕ್ಷವನ್ನು ತ್ರಿಗುಣಾತ್ಮಕವಾದ ಪ್ರಕೃತಿಯು ಆಶ್ರಯಿಸಿಕೊಂಡಿದೆ.

\begin{verse}
\textbf{ಮೂಲಂ ಮಾತ್ರಾಃ ಶಿಖಾ ತಸ್ಯ ಪುಣ್ಯಪಾಪೇ ಫಲೇ ತಥಾ~।}\\\textbf{ಪರಿಣಾಮಸ್ತಥಾ ಸತ್ತಾ ಹ್ಯಪಾಯಶ್ಚ ಜನಿಸ್ತಥಾ~।। ೪೪~।।}
\end{verse}

ಈ ಜಗತ್ತೆಂಬ ವೃಕ್ಷಕ್ಕೆ ಇಂದ್ರಿಯ ವಿಷಯಗಳು ಮೂಲಭಾಗ, ಪುಣ್ಯಪಾಪಗಳೇ ಫಲಗಳು, ಸ್ಥಿತಿಯು ಪರಿಣಾಮವು, ನಾಶವೇ ಜನನ.

\begin{verse}
\textbf{ಹ್ರಾಸವೃದ್ಧೀಷಡಾತ್ಮಾಸ್ಯ ಮೇದೋsಸ್ಥಿರುಧಿರಾಣಿ ಚ~।}\\\textbf{ತ್ವಕ್‌ಮಾಂಸಚರ್ಮಮಜ್ಜಾಶ್ಚ ಧಾತವೋ ವಿಟಿಪಾಸ್ತಥಾ~।। ೪೫~।।}
\end{verse}

ವೃದ್ಧಿ, ಹ್ರಾಸವೇ ಮುಂತಾದ ಆರು ಪರಿಣಾಮಗಳು ಈ ವೃಕ್ಷಕ್ಕೆ ಮಧ್ಯ ಭಾಗ, ಮೇದಸ್ಸು, ಅಸ್ಥಿ, ರಕ್ತ, ಚರ್ಮ, ಮಾಂಸ, ತ್ವಕ್, ಮಜ್ಜ ಎಂಬ ಏಳು ಧಾತುಗಳು ಈ ವೃಕ್ಷಕ್ಕೆ ಕೊಂಬೆಗಳು.

\begin{verse}
\textbf{ದೇವದಾನವಗಂಧರ್ವಾ ತಿರ್ಯಕ್‌ನರಪಿಶಾಚಕಾಃ~।}\\\textbf{ರಾಕ್ಷಸಾ ಸ್ಥಾವರಾಣ್ಯಷ್ಟೌ ಕೋಟಿರಂ ತ್ವಕ್ಷಗೋಲಕಮ್~।। ೪೬~।।}
\end{verse}

ದೇವತೆಗಳು, ದಾನವರು, ಗಂಧರ್ವರು, ಪಶುವೇ ಮೊದಲಾದ ಮೂಕ ಪ್ರಾಣಿಗಳು, ಮನುಷ್ಯರು, ಪಿಶಾಚಿಗಳು, ರಾಕ್ಷಸರು, ಸ್ಥಾವರರು (ಮರ, ಪೊದೆ ಮುಂತಾದುವು) ಎಂಬ ಎಂಟು ಈ ವೃಕ್ಷಕ್ಕೆ ತೊಗಟೆಗಳು, ಗೋಲಕವೇ ಅಕ್ಷ,

\begin{verse}
\textbf{ದಶೇಂದ್ರಿಯಾಣಿ ಪತ್ರಾಣಿ ಕುಡ್ಮಲಂ ಭೂತಪಂಚಕಮ್~।}\\\textbf{ಪ್ರವೃತ್ತಿ ನಿವೃತ್ತಿಮಾರ್ಗಸ್ಥೌ ಖಗೌ ನಿತ್ಯಂ ಸಮಾಶ್ರಿತೌ~।। ೪೭~।।}
\end{verse}

ಹತ್ತು ಇಂದ್ರಿಯಗಳು (ಜ್ಞಾನೇಂದ್ರಿಯ, ಕರ್ಮೇಂದ್ರಿಯಗಳು) ಈ ವೃಕ್ಷಕ್ಕೆ ಎಲೆಗಳು; ಪಂಚಭೂತಗಳು ಮೊಗ್ಗು; ಪ್ರವೃತ್ತಿಮಾರ್ಗಕರ್ಮ, ನಿವೃತ್ತಿ ಮಾರ್ಗವೆಂಬ ಎರಡು ಪಕ್ಷಿಗಳು ಈ ವೃಕ್ಷದಲ್ಲಿ ನಿತ್ಯವೂ ಆಶ್ರಯ ಪಡೆದಿವೆ.

\begin{verse}
\textbf{ಏತತ್ ಕಾರಣಕಂ ದೇಹಂ ವೃಕ್ಷಮೇತನ್ನಿಶಾಮಯ~।}\\\textbf{ಬ್ರಹ್ಮಾಶ್ರಯಸ್ತಸ್ಯ ಫಲಂ ಪ್ರವೃತ್ತಿಂ ವೈ ನಿವೃತ್ತಿ ಕಮ್~।। ೪೮~।।}
\end{verse}

ಎಲ್ಲ ಕಾರಣಗಳಿಗೆ ಮೂಲವಾದ ಈ ದೇಹವೆಂಬ ವೃಕ್ಷದ ಉದ್ದಿಶ್ಯವೇನೆಂದರೆ, ಈ ದೇಹಕ್ಕೆ ಬ್ರಹ್ಮನೇ ಆಶ್ರಯ, ಪ್ರವೃತ್ತಿ ಮತ್ತು ನಿವೃತ್ತಿಯೆಂಬ ಎರಡು ಫಲಗಳು.

\begin{verse}
\textbf{ತಿಸ್ರೋವಸ್ಥಾಃ ಜಾಗರಾದ್ಯಾಸ್ತನ್ಮೂಲಂ ಪರಿಕೀರ್ತಿತಮ್~।}\\\textbf{ತಮಶ್ಚೈವ ಪ್ರಕಾಶಶ್ಚ ಸ್ವರ್ಗೊ ಮೋಕ್ಷಶ್ಚತೂರಸಃ~।। ೪೯~।।}
\end{verse}

ಜಾಗರ, ಸುಷುಪ್ತಿ, ಸ್ವಪ್ನ, ಎಂಬ ಮೂರು ಅವಸ್ಥೆಗಳು; ತಮಸ್ಸು, ಪ್ರಕಾಶ, ಸ್ವರ್ಗ, ಮೋಕ್ಷವೆಂಬುವು ನಾಲ್ಕು ರಸಗಳು.

\begin{verse}
\textbf{ತಥೈವ ಕರ್ಮಮೂಲಾನಿ ಪಂಚಪ್ರಾಣಾಃ ಪ್ರಕೀರ್ತಿತಾಃ~।}\\\textbf{ಶೋಕಮೋಹಜರಾಮೃತ್ಯುಃ ಪಿಪಾಸಾ ಅಶನಾನಿ ಚ~।। ೫೦~।।}
\end{verse}

\begin{verse}
\textbf{ಏತತತ್ ಷಡಾತ್ಮಾ ಛಂದಾಂಸಿ ಧಾತವೋ ಭೂತಪಂಚಕಮ್~।}\\\textbf{ಮನೋ ಬುದ್ಧಿರಹಂಕಾರೋ ಹ್ಯಷ್ಟೌ ಚ ವಿಟಿಪಾಃ ಸ್ಮೃತಾಃ~।। ೫೧~।।}
\end{verse}

ಪಂಚಪ್ರಾಣಗಳೇ ಕರ್ಮಗಳಿಗೆ ಮೂಲಗಳು. ಶೋಕ, ಮೋಹ, ಮುಪ್ಪು, ಮರಣ, ಬಾಯಾರಿಕೆ, ಹಸಿವು ಈ ಆರು ಛಂದಸ್ಸುಗಳು (ಊರ್ಮಿಗಳು), ಧಾತುಗಳು, ಪಂಚಭೂತಗಳು, ಮನಸ್ಸು, ಬುದ್ಧಿ, ಅಹಂಕಾರ ಎಂಬ ಎಂಟು ರೆಂಬೆಗಳು.

\begin{verse}
\textbf{ವ್ಯವಹಾರಾಃ ಕೋಟಿರಾಣಿ ದಶ ಪ್ರಾಣಾಶ್ಛದಾನಿ ಚ~।}\\\textbf{ಅವಿದ್ಯಾ ಕಾಮ್ಯ ಕರ್ಮಾದಿ ಬಂಧಸ್ಯೈತಸ್ಯ ಕಾರಣಮ್~।। ೫೨~।।}
\end{verse}

ದೇಹದಿಂದ ನಡೆಯುವ ವ್ಯಾಪಾರಗಳೇ ತೊಗಟೆಗಳು, ದಶೇಂದ್ರಿಯಗಳೇ ಎಲೆಗಳು, ಇಂತಹ ದೇಹವೆಂಬ ವೃಕ್ಷವು ಅವಿದ್ಯಾ, ಕಾಮ, ಕರ್ಮ ಮೊದಲಾದ ಬಂಧನಗಳಿಗೆ ಕಾರಣ.

\begin{verse}
\textbf{ಜಗದ್ಬೀಜಪ್ರಧಾನಸ್ಯ ತ್ರಯಃ ಸತ್ತ್ವಾದಯೋ ಗುಣಾಃ~।}\\\textbf{ಚಿದ್ರೂಪೀ ನಿರ್ಗುಣೋ ಜೀವಃ ತಯೋರ್ಜೀವಪ್ರಧಾನಯೋಃ~।। ೫೩~।। }
\end{verse}

\begin{verse}
\textbf{ಅಹಂಕಾರನಿದಾನೇನ ಕರ್ಮಣಾ ಸಂಗತಿರ್ಭವೇತ್~।}\\\textbf{ಪ್ರಕೃತ್ಯಾ ಛಿನ್ನರೂಪಸ್ತು ಜಾತೋಽಜಾತೋ ಮೃತೋsಮೃತಃ~।। ೫೪~।। }
\end{verse}

\begin{verse}
\textbf{ಭ್ರಮತ್ಯಯಂ ಮಹಾಘೋರೇ ಸಂಸಾರಾಬ್ಧೌ ಪುನಃ ಪುನಃ~।}\\\textbf{ಸತುಷಂ ತು ಯಥಾ ಬೀಜಂ ಪುನರ್ಬಿಜತ್ವಮೇತಿ ಚ~।। ೫೫~।।}
\end{verse}

ಜಗತ್ತಿಗೆ ಕಾರಣವಾದ ಪ್ರಕೃತಿಯು ಮೂರು ಗುಣಗಳಿಂದ (ಸತ್ವ, ರಜ, ತಮ) ಯುಕ್ತವಾಗಿದೆ. ಜೀವನು ಸ್ವರೂಪದಿಂದ ಚಿದ್ರೂಪೀ (ಜ್ಞಾನಾತ್ಮಕನು), ಸ್ವರೂಪದಿಂದ ಸತ್ವರಜತಮೋಗುಣಗಳಿಂದ ರಹಿತನು. ಹೀಗೆ ಗುಣತ್ರಯ ಯುಕ್ತವಾದ ಪ್ರಕೃತಿ ಮತ್ತು ಜೀವರ ಸಂಬಂಧದಿಂದ ಅಹಂಕಾರದಿಂದ ಮಾಡಿದ ಕರ್ಮವೇ (ನಾನು ಸ್ವತಂತ್ರನು, ನಾನೇ ಕರ್ಮಮಾಡುತ್ತಿದ್ದೇನೆ, ಇದರಿಂದ ಫಲ ಬರಲೇ ಬೇಕೆಂಬ ಇಚ್ಛೆ-ಎಂಬ ಮನೋಭಾವವೇ ಅಹಂಕಾರ) ಜನ್ಮಾಂತರಗಳಿಗೆ ಕಾರಣ. ಪ್ರಕೃತಿಯಿಂದ ಆಚ್ಛಾದಿಸಲ್ಪಟ್ಟ ಅಜ್ಞಾನವೆಂಬ ಪರದೆಯ ದೆಸೆಯಿಂದ ಜೀವನು ಅನೇಕ ಬಾರಿ ಹುಟ್ಟುವುದೂ, ಅನೇಕ ಬಾರಿ ಸಾಯು\-ವುದೂ-ಇಂತಹ ಪರಿಸ್ಥಿತಿಗೆ ಒಳಪಡುತ್ತಾನೆ. ಈ ಸಂಸಾರವೆಂಬ ಘೋರವಾದ ಸಮುದ್ರದಲ್ಲಿ ಈ ಜೀವನು ಸುತ್ತುತ್ತಾನೆ. ಹೊಟ್ಟು, ಮುಂತಾದ ಮೇಲಿನ ಹೊದಿಕೆಯಿಂದ ಸಹಿತವಾದ ಬೀಜವನ್ನು ಬಿತ್ತಿದರೆ ಅದು ಗಿಡವಾಗಿ ಮತ್ತೆ ಬೀಜವನ್ನು ಉತ್ಪಾದಿಸುವಂತೆ ಜೀವನು ಮತ್ತೆ ಮತ್ತೆ ಜನ್ಮವೆತ್ತಲು ತಾನೇ ಅವಕಾಶಮಾಡಿಕೊಳ್ಳುತ್ತಾನೆ.

\begin{verse}
\textbf{ತಥಾ ಪ್ರಕೃತಿಬದ್ಧ ಸ್ತು ಪುನಃ ಸಂಸೃತಿಮೇತಿ ಚ~।}
\end{verse}

ಪ್ರಕೃತಿಯಿಂದ ಬಂಧಿತನಾದ ಜೀವನು ಪುನಃ ಪುನಃ ಸಂಸಾರಕ್ಕೆ ಬರುತ್ತಾನೆ.

\begin{verse}
\textbf{ಯಥಾ ದಗ್ಧ ತುಷಂ ಬೀಜಂ ಉಪ್ತಂ ನೈವಾಧಿರೋಹತಿ~।। ೫೬~।।}\\\textbf{ಏವಂ ದಗ್ಧ ಪ್ರಧಾನಾಖ್ಯಲಿಂಗದೇಹೋ ನ ಜಾಯತೇ~।}
\end{verse}

ಹೇಗೆ ಹುರಿದ ಬೀಜವನ್ನು ಬಿತ್ತಲು ಗಿಡವೇ ಹುಟ್ಟುವುದಿಲ್ಲವೋ ಹಾಗೆಯೇ ಪ್ರಕೃತಿಯಿಂದ ಮಾಡಲ್ಪಟ್ಟ ಲಿಂಗದೇಹವು (ಸ್ವರೂಪದೇಹದ ಮೇಲೆ ಕವಚದಂತೆ ಇರುವ ಲಿಂಗದೇಹವು) ನಾಶವಾಗಲು ಜೀವನಿಗೆ ಸಂಸಾರದಲ್ಲಿ ಪುನರ್ಜನ್ಮವೇ ಇರುವುದಿಲ್ಲ.

\begin{verse}
\textbf{ಭೂತಾನಿ ಪಂಚತನ್ಮಾತ್ರಾ ಮನೋಬುದ್ಧಿರಹಂಕೃತಿಃ~।। ೫೭~।।}\\\textbf{ಮಹತ್ತತ್ವಂ ದಶಾಕ್ಷಾಣಿ ಪ್ರಧಾನಂ ಚೈತದಾತ್ಮಕಮ್~।। }\\\textbf{ಸೂಕ್ಷ್ಮಂ ನಿತ್ಯಂ ಜಗದ್ಬೀಜಂ ಮಿತಂ ತ್ರಾತಂ ಚ ವಿಷ್ಣುನಾ~।। ೫೮~।।}
\end{verse}

ಪಂಚಮಹಾಭೂತಗಳೂ, ಪಂಚತನ್ಮಾತ್ರಾಗಳೂ (ರೂಪ, ಸ್ಪರ್ಶ, ಗಂಧ, ಶೋತ್ರು, ರಸ) ಮನಸ್ಸು, ಬುದ್ಧಿ, ಅಹಂಕಾರ, ಮಹತ್ತತ್ವ, ದಶೇಂದ್ರಿಯಗಳು (ಐದು ಜ್ಞಾನೇಂದ್ರಿಯಗಳು, ಐದು ಕರ್ಮೇಂದ್ರಿಯಗಳು)-ಈ ಇಪ್ಪತ್ತುನಾಲ್ಕು ತತ್ವಗಳಿಗೆ “ಪ್ರಧಾನ” ಎಂದು ಹೆಸರು. ಈ ಪ್ರಧಾನವು ಜಗತ್ತಿಗೆ ಕಾರಣ, ನಿತ್ಯ, ಸೂಕ್ಷ್ಮ ರೂಪದಿಂದ ಇರುತ್ತದೆ, ಶ‍್ರೀವಿಷ್ಣುವಿನ ಅಧೀನದಲ್ಲಿಯೇ ಇರುತ್ತಾ ಅವನಿಂದಲೇ ರಕ್ಷಿಸಲ್ಪಡುತ್ತದೆ.

\begin{verse}
\textbf{ಅಷ್ಟಧೇತಿ ವದಂತ್ಯನ್ಯೇ ಸಪ್ತವಿಂಶತಿ ಚಾಪರೇ~।}\\\textbf{ಪಂಚವಿಂಶತಿಧಾ ಕೇಚಿತ್ ತ್ರಯಸ್ತ್ರಿಂಶತ್ತಥಾ ಜಗುಃ~।। ೫೯~।।}
\end{verse}

ಈ “ಪ್ರಧಾನ”ವು ಎಂಟು ಪ್ರಕಾರವುಳ್ಳದ್ದೆಂದು ಕೆಲವರೂ, ಇಪ್ಪತ್ತೇಳು ಪ್ರಕಾರವೆಂದು ಕೆಲವರೂ, ಇಪ್ಪತೈದು ಪ್ರಕಾರವೆಂದು ಕೆಲವರೂ, ಮೂವತ್ತ ಮೂರು ಪ್ರಕಾರವೆಂದು ಮತ್ತೆ ಕೆಲವರೂ ಹೇಳುತ್ತಾರೆ.

\begin{verse}
\textbf{ತೇ ತು ವ್ಯಾಸಸಮಾಸಾಭ್ಯಾಂ ತತ್ವಾತ್ಮಕಮಿದಂ ಸ್ಮೃತಮ್~।}
\end{verse}

ಸಂಕ್ಷೇಪದಿಂದಲೂ, ವಿಸ್ತಾರವಾಗಿಯೂ ಈ ತತ್ವಗಳ ವಿಂಗಡಣೆ ಈ ರೀತಿ ಇರುತ್ತದೆ.

\begin{verse}
\textbf{ಯಥಾಗ್ನಿಶಿಖಯಾ ಕ್ಷುಬ್ಧಂ ಜತುನಾ ಹೇಮ ಲಿಪ್ಯತೇ~।। ೬೦~।।}\\\textbf{ತಥಾ ಮಹಾವಿದ್ಯಯಾ ಚ ಕ್ಷೋಭಿತಾತ್ಕಾಮಕರ್ಮಭಿಃ~।}\\\textbf{ಪ್ರಧಾನಾತ್ ಬಧ್ಯತೇ ಜೀವಃ ತಸ್ಮಾತ್ ದುಃಖಾನಿ ವಿಂದತಿ~।। ೬೧~।।}
\end{verse}

ಬೆಂಕಿಯಿಂದ ಕರಗಿದ ಅರಗು ಬಂಗಾರದ ಮೇಲೆ ಲೇಪವಾಗುವಂತೆ, ಅಜ್ಞಾನ ಮತ್ತು ಕಾಮ್ಯ ಕರ್ಮಗಳಿಂದ ಕದಡಿದ ಪ್ರಧಾನದ ದೆಸೆಯಿಂದ ಜೀವನು ಸಂಸಾರದಲ್ಲಿ ಬಂಧಿಸಲ್ಪಡು\-ತ್ತಾನೆ. ಅದರಿಂದ ದುಃಖಾದಿಗಳನ್ನು ಪಡೆಯುತ್ತಾನೆ.

\begin{verse}
\textbf{ತತ್ಸಂಬಂಧಾದಯಂ ಜೀವಃ ಕರ್ಮಾಣ್ಯನ್ಯಾನ್ಯನೇಕಶಃ~।}\\\textbf{ದೇಹಾಂತರಾರಂಭಕರ್ಮ ಕರೋತ್ಯದ್ಧಾ ಪುನಃ ಪುನಃ~।। ೬೨~।।}
\end{verse}

ಈ ಅಜ್ಞಾನದ ಕಾರಣವಾಗಿ ಜೀವನು ಪ್ರತಿಕ್ಷಣದಲ್ಲಿಯೂ ಅಸಂಖ್ಯಾತವಾದ ಕರ್ಮಗಳನ್ನು ಮಾಡುತ್ತಾನೆ; ಇದರಿಂದ ಮುಂದೆ ಅನೇಕ ಜನ್ಮಗಳಿಗೆ ಆಸ್ಪದವಾಗಿ ಜನ್ಮಾಂತರಗಳಲ್ಲಿಯೂ ಸಹ ಹೀಗೆ ಕರ್ಮಗಳನ್ನು ಮುಂದುವರೆಸುತ್ತಾನೆ.

\begin{flushleft}
\textbf{[ವಿಶೇಷಾಂಶ:}
\end{flushleft}

\begin{verse}
\textbf{ನ ಹಿ ಕಶ್ಚಿತ್ ಕ್ಷಣಮತಿ ಜಾತು ತಿಷ್ಠತ್ಯಕರ್ಮಕೃತ್~।}\\\textbf{ಕಾರ್ಯತೇ ಹ್ಯವಶಃ ಕರ್ಮ ಸರ್ವಃ ಪ್ರಕೃತಿಜೈರ್ಗುಣೈಃ~।।} \vauthor{-ಗೀತಾ} 
\end{verse}

ಯಾವ ಜೀವನೂ ಕರ್ಮಗಳನ್ನೇ ಮಾಡದೆ ಎಂದಿಗೂ ಒಂದು ಕ್ಷಣವಾದರೂ ಇರುವುದಿಲ್ಲ. ವಿಷ್ಣುವಿನ ಅಧೀನದಲ್ಲಿರುವ ಸಮಸ್ತ ಜೀವರಿಂದ ಪ್ರಕೃತಿಕಾರ್ಯಗಳಾದ ಸತ್ತ್ವ ರಜಸ್ತಮೋಗಳಿಂದ ಕರ್ಮವು ಮಾಡಲ್ಪಡುತ್ತದೆ.]

\begin{verse}
\textbf{ಮುನಿಃ ಸುಕರ್ಮಶೀಲೋsಪಿ ಜೀವನ್ನೇಕದಿನೇನ ವೈ~।}\\\textbf{ದಶಾವರಾಣಾಂ ದೇಹಾನಾಂ ಕಾರಣಂ ಪ್ರತಿಪದ್ಯತೇ~।। ೬೩~।।}
\end{verse}

ಮನನಶೀಲನಾದ ಸತ್ಕರ್ಮಾಚರಣೆಯಲ್ಲಿಯೇ ನಿಷ್ಠನಾದ ಸುಜೀವಿಯು ಸಹ ಒಂದು ದಿನದಲ್ಲಿ ಮುಂದೆ ತನಗೆ ಹತ್ತು ಜನ್ಮಗಳು ಬರಲು ಸಾಕಷ್ಟು ಕರ್ಮವನ್ನು ಆಚರಿಸುತ್ತಾನೆ.

\begin{verse}
\textbf{ತಾದೃಗ್ಬಂಧಮಹಾಗುಲ್ಮಮೂಲಂ ಸತ್ಕರ್ಮಸಂಗತಿಃ~।}\\\textbf{ಅನಿರ್ಮೂಲಂ ದಹತ್ಯೇಷಾ ವೇಣುಗುಲ್ಮಮಿವಾನಲಃ~।। ೬೪~।।}
\end{verse}

ಬಂಧನಕ್ಕೆ ಕಾರಣವಾದ, ಅಜ್ಞಾನದಿಂದ ನಡೆಯುವ ಕರ್ಮರಾಶಿಯು, ಬಿದಿರಿನ ಮರಗಳಿಗೆ ಬೆಂಕಿ ಬಿದ್ದರೆ ಹೇಗೆ ಅದು ನಿರ್ಮೂಲವಾಗುತ್ತದೆಯೋ ಹಾಗೆ, ಸಜ್ಜನರ ಸಹವಾಸದಿಂದ ನಿರ್ಮೂಲವಾಗುತ್ತದೆ.

\begin{verse}
\textbf{ಕರ್ಮಣಃ ಕರ್ಮಜಾತಾನಾಂ ನ ಹ್ಯಾತ್ಯಂತಿಕನಿಷ್ಕೃತಿಃ~।}\\\textbf{ಪ್ರಾಯಶ್ಚಿತ್ತಾನಿ ಚೋಕ್ತಾನಿ ಮುನಿಭಿರ್ಭಾವಿತಾತ್ಮಭಿಃ~।। ೬೫~।।}
\end{verse}

ಪಾಪಕರ್ಮಗಳ ರಾಶಿಗೆ ಸಂಪೂರ್ಣ ನಾಶವೆಂಬುದಿಲ್ಲ. ನಿರ್ಮಲಾಂತಃಕರಣರಾದ ಜ್ಞಾನಿಗಳಿಂದ ಅಂತಹ ಪಾಪಕರ್ಮಗಳಿಗೆ ಪ್ರಾಯಶ್ಚಿತ್ತವು ಹೇಳಲ್ಪಟ್ಟಿದೆ.

[\textbf{ವಿಶೇಷಾಂಶ:} ಪರಬ್ರಹ್ಮನ ಅಪರೋಕ್ಷ ಜ್ಞಾನವಾಗಲು, ಅಪರೋಕ್ಷಕ್ಕಿಂತ ಮೊದಲು ಆಚರಿಸಿದ ಹಾಗೂ ಅಪರೋಕ್ಷಾನಂತರ ಮಾಡಬಹುದಾದ ಪಾಪಗಳು ನಾಶವಾಗುತ್ತವೆ ಮತ್ತು ಲೇಪವಾಗುವುದಿಲ್ಲ. ಪ್ರಾರಬ್ದವನ್ನು ಮಾತ್ರ ಅನುಭವಿಸಿಯೇ ಕಳೆದುಕೊಳ್ಳಬೇಕು. ಕೆಲವು ವಿಶೇಷ ಸಂದರ್ಭಗಳಲ್ಲಿ ಇದಕ್ಕೂ ಉಪಮರ್ದ ಉಂಟು-ಹೆಚ್ಚಿನ ವಿವರಣೆಯನ್ನು ಬ್ರಹ್ಮಸೂತ್ರಭಾಷ್ಯದಲ್ಲಿನ “ತದಧಿಗಮಾಧಿಕರಣ'ದಲ್ಲಿ ಕಾಣಬಹುದು.]

\begin{verse}
\textbf{ತಾನಿ ಲೋಕೇ ಚ ಜಂತೂನಾಮಜ್ಞಾನಾಂ ಚ ಜಡಾತ್ಮನಾಮ್~।}\\\textbf{ಜ್ಞಾನಯೋಗೋ ದಹತ್ಯಾಶು ಬಂಧಗುಲ್ಮಂ ಸಮೂಲಕಮ್~।। ೬೬~।।}
\end{verse}

ಪ್ರಾಯಶ್ಚಿತ್ತಗಳೂ, ಯಥಾರ್ಥಜ್ಞಾನವೂ ಜಡವಸ್ತುಗಳಂತೆ ಅಜ್ಞಾನಿಗಳಾದ ಜನರ ಬಂಧನವನ್ನು ಮೂಲಸಹಿತ ನಾಶಮಾಡುತ್ತವೆ.

\textbf{[ವಿಶೇಷಾಂಶ:}

\begin{verse}
\textbf{ಜ್ಞಾನಾಗ್ನಿಃ ಸರ್ವ ಕರ್ಮಾಣಿ ಭಸ್ಮಸಾತ್ಕುರುತೇ ತಥಾ~।}\\\textbf{ನ ಹಿ ಜ್ಞಾನೇನ ಸದೃಶಂ ಪವಿತ್ರಮಿಹ ವಿದ್ಯತೇ~।।} \vauthor{-ಗೀತಾ}
\end{verse}

ಜ್ಞಾನವೆಂಬ ಅಗ್ನಿಯು ಪ್ರಾರಬ್ಧೇತರವಾದ ಅನಿಷ್ಟವಾದ ಸಕಲ ಕರ್ಮಗಳನ್ನೂ ನಾಶಮಾಡುತ್ತದೆ. ಜ್ಞಾನಕ್ಕೆ ಸದೃಶವಾದ ಬೇರೊಂದು ಶುದ್ಧವಾದ ವಸ್ತುವು ಇಲ್ಲ]

\begin{verse}
\textbf{ಗ್ರೀಷ್ಮೇ ಘರ್ಮಃ ಪಂಕಮಿವ ನೀಹಾರಮಿವ ಭಾಸ್ಕರಃ~।}\\\textbf{ತತ್ ಜ್ಞಾನಂ ಕರ್ಮಣಾ ಸಾಧ್ಯಂ ಕರ್ಮ ಸಮ್ಯಕ್ ಪ್ರಯೋಜಿತಮ್~।। }
\end{verse}

\begin{verse}
\textbf{ಉತ್ಪಾದಯತಿ ತತ್ ಜ್ಞಾನಂ ತಸ್ಮಾತ್ಕರ್ಮ ಸಮಾಚರೇತ್~।}\\\textbf{ವಿನಾ ಕರ್ಮ ಚ ತತ್ ಜ್ಞಾನೀ ನ ಸ್ವಾರ್ಥಾನಭಿಪದ್ಯತೇ~।। ೬೮~।।}
\end{verse}

ಗ್ರೀಷ್ಮ ಋತುವಿನಲ್ಲಿ ಸೂರ್ಯನು ಹೇಗೆ ಕೆಸರನ್ನು ಒಣಗಿಸುತ್ತಾನೆಯೋ ಹಿಮವನ್ನು, ಹೇಗೆ ನಾಶಮಾಡುತ್ತಾನೆಯೋ ಅದರಂತೆ ಸಂಸಾರಬಂಧನವನ್ನು ನಾಶಮಾಡಲು ಜ್ಞಾನವು ಅವಶ್ಯಕ. ಆ ಜ್ಞಾನವಾದರೋ ಸತ್ಕರ್ಮಾಚರಣೆಯಿಂದ ಅಭಿವೃದ್ಧಿಯಾಗುತ್ತದೆ. ಆದುದರಿಂದ ಸತ್ಕರ್ಮವನ್ನು ಮೊದಲು ಆಚರಿಸಬೇಕು. ಸತ್ಕರ್ಮಾನುಷ್ಠಾನ ವಿನಹ ಜ್ಞಾನವು ಅಭಿವೃದ್ಧಿಯಾಗುವುದಿಲ್ಲ, ಪುರುಷಾರ್ಥವೂ ಕೈಗೂಡುವುದಿಲ್ಲ.

\begin{verse}
\textbf{ಅಹಂಕಾರವಿಹೀನಸ್ಯ ಪುರುಷಸ್ಯ ಫಲಂ ಯಥಾ~।}\\\textbf{ವಿನಾ ಕರ್ಮಾಣ್ಯಯಂ ಜೀವಃ ಫಲಂ ನಾಪ್ನೋತಿ ಜಾತುಚಿತ್~।। ೬೯~।।}
\end{verse}

ಅಂತಹ ಯಥಾರ್ಥ ಜ್ಞಾನಾಭಿವೃದ್ಧಿಗೆ ಅಹಂಕಾರವಿರಬಾರದು (ನಾನೇ ಕರ್ಮವನ್ನು ಮಾಡುತ್ತೇನೆಂಬ ಸ್ವಾತಂತ್ರ್ಯದ ಅಭಿಮಾನ). ಇಂತಹ ಅಹಂಕಾರ ರಹಿತನಾದ ಪುರುಷನಿಗೆ ಮೇಲೆ ಹೇಳಿದ ಬಂಧನನಿವೃತ್ತಿಯೂ ಆಗುತ್ತದೆ. ಸತ್ಕರ್ಮದ ಹೊರತಾಗಿ ಜ್ಞಾನಾಭಿವೃದ್ಧಿಗೆ ಅನ್ಯ ಮಾರ್ಗವಿಲ್ಲ.

\begin{verse}
\textbf{ಸಹಕಾರವಿಹೀನಸ್ಯ ಪುರುಷಸ್ಯ ಫಲಂ ಯಥಾ~।}\\\textbf{ಕರ್ಮಣೈವ ವಿನಾ ಜ್ಞಾನಾತ್ ಫಲಂ ನಾಪ್ನೋತಿ ಜಾತುಚಿತ್~।। ೭೦~।।}
\end{verse}

ಇತರರ ಸಹಕಾರವಿಲ್ಲದೇ ಪುರುಷನು ತನ್ನ ಕರ್ತವ್ಯದ ಫಲವನ್ನು ಹೇಗೆ ಪಡೆಯುವುದಿಲ್ಲವೋ, ಹಾಗೆಯೇ ಕರ್ಮದ ಸಹಕಾರವಿಲ್ಲದೇ ಯಥಾರ್ಥಜ್ಞಾನವಿಲ್ಲ, ಜ್ಞಾನವಿಲ್ಲದೆ ಬಂಧನಿವೃತ್ತಿಯೆಂಬ ಫಲವಿಲ್ಲ.

\begin{verse}
\textbf{ಶಾಲ್ಮಲೀಫಲಮಾದಾಯ ಭೋಕ್ತುಮಿಚ್ಛೋರ್ಯಥಾ ಕ್ರಿಯಾ~।}\\\textbf{ಉಭೇ ವಿಧೇ ಚ ತೇ ಜ್ಞೇಯೇ ತಸ್ಮಾದ್ಧಿತಮಭೀಪ್ಸುನಾ~।। ೭೦~।। }
\end{verse}

\begin{verse}
\textbf{ಅನುಷ್ಠಾನವಿಧಿಜ್ಞೇನ ಯೋಗಿನಾಂ ಚರಿತೇನ ಚ~।}\\\textbf{ಕರ್ಮಣಾ ಕ್ಷೀಯತೇ ಮೃತ್ಯುಃ ಸಂಸಾರಾಖ್ಯಃ ಸಹಾನುಗಃ~।। ೭೨~।।}
\end{verse}

ಬೂರುಗದ ಮರದ ಫಲವನ್ನು ತಂದು ಅದನ್ನು ತಿನ್ನಲು ಪ್ರಯತ್ನ ಪಡುವವನ ಕಾರ್ಯ ಹೇಗೆ ನಿಷ್ಫಲವೋ, ಅದರಂತೆ ಸತ್ಕರ್ಮಾನುಷ್ಠಾನವಿಲ್ಲದೇ ಇರುವ ಜ್ಞಾನಾಪೇಕ್ಷಿಯ ಪ್ರಯತ್ನ ನಿಷ್ಫಲ. ಆದುದರಿಂದ, ತನಗೆ ಹಿತವಾಗಬೇಕೆಂಬ ಇಚ್ಛೆಯುಳ್ಳವನು ಅನುಷ್ಠಾನವಿಧಿಯನ್ನು ಬಲ್ಲ ಜ್ಞಾನಿಗಳ ಆಚರಣೆಯನ್ನು ನೋಡಿ ತಾನೂ ಅದರಂತೆ ಆಚರಿಸಬೇಕು. ಈ ಸತ್ಕರ್ಮದಿಂದ ಜ್ಞಾನಾಭಿವೃದ್ಧಿ ಯಾಗಿ ಅದರ ಫಲವಾಗಿ ಸಂಸಾರವೆಂಬ ಮೃತ್ಯುವು ತನ್ನ ಸಂಗಡಿಗರಿಂದ ಸಹಿತವಾಗಿ ನಾಶವಾಗುತ್ತದೆ.

\textbf{[ವಿಶೇಷಾಂಶ:}

\begin{verse}
\textbf{ಯದ್ಯದಾಚರತಿ ಶ್ರೇಷ್ಠಸ್ತತ್ತ ದೇವೇತರೋ ಜನಃ~।}\\\textbf{ಸ ಯತ್ಪ್ರಮಾಣಂ ಕುರುತೇ ಲೋಕಸ್ತ ದನುವರ್ತತೇ~।।} \vauthor{-ಗೀತಾ}
\end{verse}

ಶ್ರೇಷ್ಠನಾದವನು ಯಾವ ಯಾವ ಸತ್ಕರ್ಮಗಳನ್ನು ಆಚರಿಸುವನೋ ಇತರ ಜನರೂ ಸಹ ಅವುಗಳನ್ನೇ ಆಚರಿಸುತ್ತಾರೆ. ಆತನು ಯಾವ ಗ್ರಂಥವನ್ನು ಪ್ರಮಾಣವೆಂದು ಗ್ರಹಿಸುವನೋ ಇತರ ಜನರೂ ಸಹ ಅವನ್ನೇ ಪ್ರಮಾಣವೆಂದು ಅನುಸರಿಸುತ್ತಾರೆ].

\begin{verse}
\textbf{ಯೋಗೀ ಮೋಕ್ಷಮವಾಪ್ನೋತಿ ಜ್ಞಾನವಿಜ್ಞಾನಸಂಪದಾ~।}\\\textbf{ಕರ್ಮ ಕುರ್ಯಾತ್ಸದಾ ವಿದ್ವಾನ್ ಕರ್ತೃತ್ವಂ ತ್ವಂ ವಿಹಾಯ ಚ~।। ೭೩~।।}
\end{verse}

ಮೋಕ್ಷಪಾಯವನ್ನು ತಿಳಿದಿರುವ ಅಧಿಕಾರಿಯು ಬಾಹ್ಯಜ್ಞಾನವನ್ನೂ ಮತ್ತು ಅಪರೋಕ್ಷ\-ಜ್ಞಾನವೆಂಬ ಸಂಪತ್ತನ್ನೂ ಪಡೆದು ಮೋಕ್ಷವನ್ನು ಹೊಂದುತ್ತಾನೆ. ಆದಕಾರಣ ಮೋಕ್ಷಾಧಿಕಾರಿಯು ತಾನೇ ಕರ್ತನೆಂಬ ಸ್ವಾತಂತ್ರಾಭಿಮಾನವನ್ನು ತೊರೆದು ಸತ್ಕರ್ಮಗಳನ್ನು ಆಚರಿಸಬೇಕು.

\begin{verse}
\textbf{ಕರ್ತಾ ಕಾರಯಿತಾ ವಿಷ್ಣುಃ ಪ್ರೇರಕಃ ಸಾಕ್ಷಿಚೇತನಃ~।}\\\textbf{ಕರ್ಮಣಾಂ ಫಲಭೋಕ್ತಾ ಚ ಸ್ಥಿತಃ ಕರ್ಮಣ್ಯ ಕರ್ಮಣಿ~।। ೭೪~।।}
\end{verse}

ಸಮಸ್ತ ಚೇತನರಲ್ಲಿಯೂ ಸಾಕ್ಷಿಯಾಗಿದ್ದು, ಸಮಸ್ತರಿಂದಲೂ ಸರ್ವಕರ್ಮಗಳನ್ನೂ ಮಾಡುವವನೂ, ಮಾಡಿಸುವವನೂ, ಕರ್ಮಮಾಡಲು ಪ್ರೇರಣೆಮಾಡುವವನೂ, ಸತ್ಕರ್ಮಗಳಲ್ಲಿಯೂ, ದುಷ್ಕರ್ಮಗಳಲ್ಲಿಯೂ ತಾನೇ ನಿಂತು, ಅವುಗಳ ಫಲಗಳನ್ನು ಭೋಗಿಸುವವನೂ ವಿಷ್ಣುವೇ.

\begin{flushleft}
\textbf{[ವಿಶೇಷಾಂಶ:} 
\end{flushleft}

\begin{verse}
\textbf{ಈಶ್ವರಃ ಸರ್ವಭೂತಾನಾಂ ಹೃದ್ದೇಶೇಽರ್ಜುನ ತಿಷ್ಠತಿ~।}\\\textbf{ಭ್ರಾಮಯನ್ಸರ್ವಭೂತಾನಿ ಯಂತ್ರಾರೂಢಾನಿ ಮಾಯಯಾ~।।} \vauthor{-ಗೀತಾ}
\end{verse}

ಸರ್ವರಿಗೂ ನಿಯಮಿಸುವ ಸ್ವಭಾವವುಳ್ಳ ನಾರಾಯಣನು ಶರೀರವೆಂಬ ಯಂತ್ರದಲ್ಲಿರುವ ಸಮಸ್ತ ಪ್ರಾಣಿಗಳನ್ನು, ತನ್ನ ಇಚ್ಛೆಯಿಂದ ನಾನಾ ಕರ್ಮಗಳಲ್ಲಿ ಪ್ರವರ್ತಿಸುವಂತೆ ಮಾಡುತ್ತಾ ಸರ್ವಪ್ರಾಣಿಗಳ ಹೃದಯಪ್ರದೇಶದಲ್ಲಿ ಇರುತ್ತಾನೆ.

\begin{verse}
\textbf{ಏವಂ ಜ್ಞಾನೀ ಸದಾ ಕುರ್ವನ್ ನ ಸ ಲಿಪ್ಯತಿ ಕರ್ಮಣಾ~।}\\\textbf{ಯಥಾದಾಯ ಶಿಶುಂ ಮಾತಾ ಪಾಯಯಿತ್ವಾ ಸ್ತನಂ ಸ್ವಯಮ್~।।೭೫~।। }
\end{verse}

\begin{verse}
\textbf{ಪರಿಪಾತಿ ತಥಾ ವಿಷ್ಣುರ್ದಾತಾ ಕಾರಯಿತಾ ಸ್ವಯಮ್~।}\\\textbf{ಕರ್ತಾಹಮಿತಿ ಮತ್ವಾಽಯಂ ಜೀವಃ ಸಂಸರತಿ ಧ್ರುವಮ್~।।}
\end{verse}

ಹೀಗೆ ಸ್ವಾತಂತ್ರ್ಯಾಭಿಮಾನರಹಿತನಾದ ಜ್ಞಾನಿಗೆ ಕರ್ಮಗಳ ಲೇಪವು ಸಂಭವಿಸುವುದಿಲ್ಲ. ತಾಯಿಯು ಮಗುವನ್ನು ತಾನೇ ಎತ್ತಿಕೊಂಡು ಸ್ತನ್ಯಪಾನ ಮಾಡಿಸುವಂತೆ ಸರ್ವೆಶ್ವರನಾದ ವಿಷ್ಣುವು ಎಲ್ಲ ಚೇತನರಿಗೂ ಅವರವರ ಕರ್ಮಗಳಿಗೆ ಅನುಸಾರವಾಗಿ ಫಲವನ್ನು ಕೊಡುತ್ತಾನೆ. ಪ್ರತಿ ಜೀವನಲ್ಲಿ ನಿಂತು ಅವರವರಿಗೆ ಯೋಗ್ಯವಾದ ಕರ್ಮಗಳನ್ನು ಮಾಡಿಸುತ್ತಾನೆ. ಇದನ್ನು ತಿಳಿಯದೆ ಜೀವನು ತಾನೇ ಸ್ವತಂತ್ರ ಕರ್ತನೆಂದು ಭ್ರಮಿಸಿ ಈ ಸಂಸಾರದಲ್ಲಿ ಸುತ್ತುತ್ತಾನೆ.

\begin{verse}
\textbf{ಶಾಂತಂ ದಾಂತಂ ವಿನೀತಂ ಚ ಸರ್ವವಿದ್ಯಾ ಸು ಕೋವಿದಮ್~।। ೭೬~।।} 
\end{verse}

\begin{verse}
\textbf{ನಿಃಸಂಶಯಂ ಗುರುಂ ಪ್ರಾಪ್ಯ ಶ್ರುತ್ವಾ ಶಾಸ್ತ್ರೇಣ ಚೋದಿತಮ್~।}\\\textbf{ವಿಧಾನಂ ಕರ್ಮಣೋ ಜ್ಞಾತ್ವಾ ತತಃ ಸಮ್ಯಕ್ ಸಮಾಚರೇತ್~।। ೭೭~।।}
\end{verse}

ಈ ಸಂಸಾರದಿಂದ ಬಿಡುಗಡೆಯನ್ನು ಹೊಂದಲು ಏನು ಮಾಡಬೇಕೆಂದರೆ- ಭಗವನ್ನಿಷ್ಠ ಬುದ್ದಿಯುಳ್ಳ, ಇಂದ್ರೀಯನಿಗ್ರಹ ಹೊಂದಿರುವ, ಅಹಂಕಾರರಹಿತನಾದ, ಎಲ್ಲ ಶಾಸ್ತ್ರಗಳಲ್ಲಿಯೂ ಪ್ರಾವೀಣ್ಯತೆಯನ್ನು ಪಡೆದಿರುವ, ಸಂಶಯರಹಿತನಾದ, ಗುರುವಿನ ಬಳಿ ತೆರಳಿ, ಅವರಿಂದ ಶಾಸ್ರೋಕ್ತವಾದ ಕರ್ಮಗಳ ವಿವರಣೆಯನ್ನೂ, ಕರ್ಮಗಳನ್ನು ಆಚರಿಸುವ ಬಗೆಯನ್ನೂ ತಿಳಿದುಕೊಂಡು ಸತ್ಕರ್ಮಗಳನ್ನು ಆಚರಿಸಬೇಕು.

\begin{flushleft}
\textbf{[ವಿಶೇಷಾಂಶ:} 
\end{flushleft}

\begin{verse}
\textbf{ತಸ್ಮಾಚ್ಛಾಸ್ತ್ರಂ ಪ್ರಮಾಣಂ ತೇ ಕಾರ್ಯಾಕಾರ್ಯವ್ಯವಸ್ಥಿತೌ~।}\\\textbf{ಜ್ಞಾತ್ವಾ ಶಾಸ್ತ್ರವಿಧಾನೋಕ್ತಂ ಕರ್ಮ ಕರ್ತುವಿಹಾರ್ಹಸಿ~।।} \vauthor{-ಗೀತಾ}
\end{verse}

ಶಾಸ್ತ್ರದಲ್ಲಿ ಹೇಳಿದ ಪ್ರಕಾರ ಆಚರಿಸದೆ ತನ್ನ ಇಚ್ಛಾನುಸಾರವಾಗಿ ಕರ್ಮವನ್ನು ಆಚರಿಸುವುದರಿಂದ ಫಲ ದೊರೆಯುವುದಿಲ್ಲವಾದಕಾರಣ, ಕಾರ್ಯ ಯಾವುದು, ಅಕಾರ್ಯ ಯಾವುದು ಎಂಬ ವಿಚಾರದಲ್ಲಿ ಸಾತ್ವಿಕಸ್ವಭಾವವುಳ್ಳ ನಿನಗೆ (ಅರ್ಜುನನಿಗೆ) ಶ್ರುತಿಸ್ಮೃತ್ಯಾದಿಶಾಸ್ತ್ರವೇ ಪ್ರಮಾಣ; ಆದುದರಿಂದ ಶಾಸ್ತ್ರಗಳಲ್ಲಿ ವಿಧಾಯಕವಾಗಿ ಹೇಳಲ್ಪಟ್ಟ ಕರ್ಮವನ್ನು ತಿಳಿದುಕೊಂಡು ಅದರಂತೆ ಆಚರಿಸಲು ಅರ್ಹನಾಗಿರುತ್ತೀಯೆ].

\begin{verse}
\textbf{ಕುರ್ವನ್ಕರ್ಮಫಲತ್ಯಾಗಂ ಕುರ್ವನ್ ವಿಷ್ಣ್ವರ್ಪಣಂ ತಥಾ~।}\\\textbf{ತತಃ ಕರಣಶುದ್ಧಿಂ ಚ ತದ್ ದ್ವಾರಾ ಜ್ಞಾನಮಾಪ್ನುಯಾತ್~।। ೭೮~।।}
\end{verse}

ಸತ್ಕರ್ಮಗಳನ್ನು ಆಚರಿಸುವಾಗ ಕರ್ಮಗಳ ಫಲಗಳನ್ನು ಇಚ್ಚಿಸದೇ, ಸಕಲ ಕರ್ಮಗಳನ್ನೂ ವಿಷ್ಣುವಿನಲ್ಲಿ ಅರ್ಪಿಸಬೇಕು. ಹೀಗೆ ಮಾಡುವುದರಿಂದ ಅಂತಃಕರಣ ಶುದ್ದಿಯಾಗಿ ಅದರ ಫಲವಾಗಿ ಪರಮಾತ್ಮನ ಸಂಬಂಧದ ಯಥಾರ್ಥಜ್ಞಾನವು ಯಥಾಯೋಗ್ಯವಾಗಿ ಲಭಿಸುತ್ತದೆ.

\begin{flushleft}
\textbf{[ವಿಶೇಷಾಂಶ:} 
\end{flushleft}

\begin{verse}
\textbf{ಮಯಿ ಸರ್ವಾಣಿ ಕರ್ಮಾಣಿ ಸಂನ್ಯಸ್ಯಾಧ್ಯಾತ್ಮಚೇತಸಾ~।}\\\textbf{ನಿರಾಶೀರ್ನಿಮಮೋ ಭೂತ್ವಾ ಯುಧ್ಯಸ್ವ ವಿಗತಜ್ವರಃ~।।} \vauthor{-ಗೀತಾ}
\end{verse}

ಶ್ರೇಷ್ಠನಾದ ಪರಮಾತ್ಮನಲ್ಲಿ ಮನಸ್ಸನ್ನು ಇಟ್ಟು, ಫಲೇಚ್ಛಾರಹಿತನಾಗಿ, ಕರ್ತೃತ್ವ ಅಭಿಮಾನ ಪರಿತ್ಯಜಿಸಿ, ಎಲ್ಲ ಸತ್ಕರ್ಮಗಳನ್ನೂ, ನನ್ನಲ್ಲಿ (ಶ‍್ರೀಕೃಷ್ಣನಲ್ಲಿ) ಅರ್ಪಣೆ ಮಾಡಿ, ವಿಷ್ಣು ಪೂಜಾತ್ಮಕವೆಂದು ಭಾವಿಸಿ, ಭಯ-ದುಃಖರಹಿತನಾಗಿ ಯುದ್ಧ ಮಾಡು].

\begin{verse}
\textbf{ಕರ್ಮಾಣಿ ತ್ರಿವಿಧಾನ್ಯಾಹುರ್ಗುಣಭೇದಫಲಾನ್ಯಪಿ~।}\\\textbf{ಗುಣಾನುಸಂಗಾಜ್ಜೀವಾಶ್ಚ ಷಡ್ವಿಧಾಃ ಪರಿಕೀರ್ತಿತಾಃ~।। ೭೯~।।}
\end{verse}

ಕರ್ಮಗಳು ಮೂರು ವಿಧ, ಹಾಗೂ ಅವುಗಳ ಫಲಗಳೂ ಸಹ ಗುಣಭೇದದಿಂದ ಮೂರು ವಿಧ. ಗುಣಗಳ ಸಂಬಂಧದಿಂದ ಜೀವರು ಆರು ಬಗೆಯಿಂದ ಇರುತ್ತಾರೆ.

\begin{verse}
\textbf{ಸಾತ್ವಿಕಾಃ ಶುದ್ಧಸತ್ವಾಶ್ಚ ರಾಜಸಾಃ ಶುದ್ಧರಾಜಸಾಃ~।}\\\textbf{ತಾಮಸಾಶ್ಚ ತಥಾ ಶುದ್ಧ ತಾಮಸಾ ಲಕ್ಷಣಂ ಬ್ರುವೇ~।। ೮೦~।।}
\end{verse}

ಸಾತ್ವಿಕರು, ಶುದ್ಧಸಾತ್ವಿಕರು, ರಾಜಸರು, ಶುದ್ಧರಾಜಸರು, ತಾಮಸರು ಮತ್ತು ಶುದ್ಧತಾಮಸರೆಂದು ಜೀವರಲ್ಲಿ ಆರು ಬಗೆ; ಅವರ ಲಕ್ಷಣಗಳನ್ನು ಹೇಳುತ್ತೇನೆ.

\begin{verse}
\textbf{ಭಾವಾದನ್ಯಾದ್ಭಿನ್ನರೂಪಾತ್ಸಂಗಾತ್ಸತ್ವಂ ಸಮಾವಿಶೇತ್~।}\\\textbf{ಯೋ ನಿಜಾಂ ಪ್ರಕೃತಿಂ ಯಾಯಾತ್ಸ ಸತ್ವೋ ಜೀವ ಉಚ್ಯತೇ~।। ೮೧~।।}
\end{verse}

ತಮಗಿಂತ ಭಗವಂತನು ನಿತ್ಯನೂ ಅತ್ಯಂತ ಭಿನ್ನನೆಂದೂ ತಿಳಿದು, ಶಾಸ್ತ್ರೋಕ್ತವಾದ ಸತ್ಕರ್ಮಗಳಲ್ಲಿ ತಮ್ಮ ಸಹಜವಾದ ಶ್ರದ್ಧೆಯನ್ನಿಟ್ಟು ಆಚರಿಸುವವರು ಸಾತ್ವಿಕರು.

\begin{verse}
\textbf{ಕ್ಷಿತಿಪಾ ಮನುಗಂಧರ್ವಾ ದೇವಾಶ್ಚ ಪಿತರಶ್ಚಿರಾತ್~।}\\\textbf{ಆಜಾನಜಾಃ ಕರ್ಮಜಾಶ್ಚ ದೇವಾ ಯೇ ಸಾತ್ವಿಕಾ ಮತಾಃ~।। ೮೨~।।}
\end{verse}

ಚಕ್ರವರ್ತಿಗಳು (ಜನಕರಾಜ, ಉತ್ತಾನಪಾದ ಮೊದಲಾದವರು), ಮನುಗಳು, ಗಂಧ\-ರ್ವರು, ಪಿತೃಗಳು, ಆಜಾನಜ ದೇವತೆಗಳು, ಕರ್ಮಜ ದೇವತೆಗಳು ಸಾತ್ವಿಕರು.

\begin{verse}
\textbf{ಸಾಪರೋಕ್ಷಾ ಹ‌ರೌ ಯೇ ತು ನಾನ್ಯಭಾವಸಮಾಶ್ರಯಾಃ~।}\\\textbf{ಯೋಗಾದ್ಭಾವಂ ವರ್ಧಯಂತಃ ಶುದ್ಧಸತ್ತ್ವಾ ಇಮೇ ಮತಾಃ~।। ೮೩~।।}
\end{verse}

ಶ‍್ರೀಹರಿಯ ಅಪರೋಕ್ಷವನ್ನು ಪಡೆದಿರುವವರು, ಶ‍್ರೀಹರಿಯಲ್ಲಿಯೇ ಅನನ್ಯವಾದ ಭಕ್ತಿಯುಳ್ಳವರು, ಯೋಗಾಭ್ಯಾಸದಿಂದ ಶ‍್ರೀಹರಿಯಲ್ಲಿ ಭಕ್ತಿಯನ್ನು ಅಭಿವೃದ್ಧಿಪಡಿಸಿಕೊಳ್ಳುವವರು ಶುದ್ಧ ಸಾತ್ವಿಕರು.

\begin{verse}
\textbf{ತತ್ತ್ವಾಭಿಮಾನಿನೋ ದೇವಾ ಬ್ರಹ್ಮಾದ್ಯಾಃ ಪರಿಕೀರ್ತಿತಾಃ~।}
\end{verse}

ಚತುರ್ಮುಖ ಬ್ರಹ್ಮದೇವರೇ ಮೊದಲಾದ ದೇವತೆಗಳು ತತ್ವಾಭಿಮಾನಿ ದೇವತೆಗಳೆಂದೆನಿಸುವರು.

\begin{verse}
\textbf{ರಜೋಮಿಶ್ರ ಸ್ವಭಾವಾ ಯೇ ಭಜಂತ್ಯ ಜ್ಞಾನಕರ್ಮಭಿಃ~।। ೮೪~।।}\\\textbf{ಪಶ್ಚಾದ್ಗುರುಪ್ರಭಾವೇನ ಶುದ್ಧ ಭಾವಮುಪೇಯುಷಃ~। }\\\textbf{ಜೀವಾಸ್ತೇ ರಾಜಸಾಃ ಪ್ರೋಕ್ತಾ ಮುಕ್ತಿಗಾ ಮಾನುಷೋತ್ತಮಾಃ~।। ೮೫~।।}
\end{verse}

ರಜೋಗುಣದಿಂದ ಮಿಶ್ರವಾದ ಸಾತ್ವಿಕ ಸ್ವಭಾವವುಳ್ಳವರು ಮೊದಲು ಅಜ್ಞಾನದಿಂದ ಯುಕ್ತರಾಗಿ ಅದಕ್ಕೆ ಅನುಸಾರವಾಗಿ ಕರ್ಮಗಳನ್ನು ಆಚರಿಸಿ ನಂತರ ಯೋಗ್ಯ ಗುರುಗಳ ಉಪದೇಶಬಲದಿಂದ ಶುದ್ಧವಾದ ಸಾತ್ವಿಕಭಾವವನ್ನು ಪಡೆಯುತ್ತಾರೆ. ಇಂತಹ ಜೀವರೇ ಮನುಷ್ಯೋತ್ತಮರೆಂದೂ ಮುಕ್ತಿಯೋಗ್ಯರೆಂದೂ ಹೇಳಲ್ಪಡುತ್ತಾರೆ.

\begin{verse}
\textbf{ಕ್ವಚಿತ್ ತಾಮಸಭಾವಾಶ್ಚ ಕ್ವಚಿತ್ ಮಿಶ್ರಸ್ವರೂಪಿಣಃ~।}\\\textbf{ಸತ್ತ್ವಾರಂಭವಿಹೀನಾ ಯೇ ತೇ ಸ್ಮೃತಾಃ ಶುದ್ಧರಾಜಸಾಃ~।। ೮೬~।।}
\end{verse}

ಕೆಲವು ವೇಳೆಗಳಲ್ಲಿ ಕೇವಲ ತಾಮಸಭಾವನೆ, ಮತ್ತೆ ಕೆಲವು ವೇಳೆಗಳಲ್ಲಿ ರಾಜಸಸ್ವಭಾವದ ಮಿಶ್ರಣ ಉಳ್ಳವರಾಗಿ ಸಾತ್ವಿಕ ಕರ್ಮಹೀನರೂ ಆಗಿರುವರೋ ಅವರು ಶುದ್ಧ ರಾಜಸರೆಂದು ಕರೆಯಲ್ಪಡುತ್ತಾರೆ.

\begin{verse}
\textbf{ಸತ್ಯಶೌಚದಯಾಹೀನಾ ಸದಾ ಯೇ ಮಿಶ್ರಬುದ್ಧಯಃ~।}\\\textbf{ಉಚ್ಚಾವಚೇಷು ಲೋಕೇಷು ಭವಂತಸ್ತಾಮಸಾಃ ಸ್ಮೃತಾಃ~।। ೮೭~।।}
\end{verse}

ಸತ್ಯ, ಬಾಹ್ಯಾಂತರಶುದ್ದಿ, ದಯಾ-ಇವುಗಳಿಲ್ಲದೇ ಸದಾ ಮಿಶ್ರಸ್ವಭಾವವನ್ನು ಹೊಂದಿ, ಕೆಲವು ವೇಳೆ ಉತ್ತಮಯೋನಿಗಳಲ್ಲಿ, ಕೆಲವು ವೇಳೆ ನೀಚಯೋನಿಗಳಲ್ಲಿ ಜನ್ಮವನ್ನು ಪಡೆಯುವವರೇ ತಾಮಸರೆಂದು ಹೇಳಲ್ಪಡುತ್ತಾರೆ.

\begin{verse}
\textbf{ಇತ್ತಂ ಸಂಸೃತಿಬುದ್ಧಾಸ್ತೇ ಜೀವಾ ಮಾನುಷಸಂಜ್ಞಿತಾಃ~।}\\\textbf{ದುರಾಚಾರಾ ದಯಾಹೀನಾ ಭಕ್ತಿಲೇಶವಿವರ್ಜಿತಾಃ~।। ೮೮~।।} 
\end{verse}

\begin{verse}
\textbf{ಅನೂರ್ಧ್ವಲೋಕಗಾಃ ಕ್ವಾಪಿ ತೇ ಜೀವಾಃ ಶುದ್ಧ ತಾಮಸಾಃ~।}\\\textbf{ತೇ ಜೀವಾ ಯಾತುಧಾನಾದ್ಯಾ ದೈತ್ಯ ದಾನವರಾಕ್ಷಸಾಃ~।। ೮೯~।।}
\end{verse}

ಮನುಷ್ಯ ಯೋನಿಯಲ್ಲಿ ಹುಟ್ಟಿದಾಗ ಕೇವಲ ಸಾಂಸಾರಿಕ ವ್ಯವಹಾರಗಳಲ್ಲಿಯೇ ಮಗ್ನರಾಗಿ, ದುರಾಚಾರಿಗಳಾಗಿ, ದಯಾರಹಿತರಾಗಿ ಪರಮಾತ್ಮನಲ್ಲಿ ಲೇಶಮಾತ್ರವೂ ಭಕ್ತಿಯಿಲ್ಲದೆ, ಊರ್ಧ್ವಲೋಕಗಳಿಗೆ ಹೋಗಬೇಕೆಂಬ ಇಚ್ಛೆಯೇ ಇಲ್ಲದ ಜೀವರು ಶುದ್ಧ ತಾಮಸರು. ಅವರು ದೈತ್ಯರು, ರಾಕ್ಷಸರೇ ಮೊದಲಾದವರು.

\begin{verse}
\textbf{ಏವಂ ಜೀವಸ್ಯ ಷಾಡ್ ವಿಧ್ಯಾತ್ಕರ್ಮಾಣ್ಯಪಿ ಗುಣಾಶ್ರಯಾತ್~।}\\\textbf{ದ್ವೈವಿಧ್ಯೇನ ಚ ಪ್ರತ್ಯೇಕಂ ಷಡ್ವಿಧಾನಿ ವಿದುರ್ಬುಧಾಃ~।। ೯೦~।।}
\end{verse}

ಹೀಗೆ ಜೀವರು ಆರು ಬಗೆಯಾಗಿರುವುದರಿಂದ ಅವರು ಆಚರಿಸುವ ಕರ್ಮಗಳೂ ಸಹ ಆರು ಪ್ರಕಾರವಾಗಿರುತ್ತವೆಯೆಂದು ಜ್ಞಾನಿಗಳು ತಿಳಿಯುತ್ತಾರೆ.

\begin{verse}
\textbf{ಪ್ರಕೃತಿಸ್ಥಾ ಇಮೇ ಜೀವಾಃ ಸ್ವಾನುರೂಪಗುಣೇರಿತಾಃ~।}\\\textbf{ಸ್ವಾನುರೂಪಕ್ರಿಯತ್ವಾಚ್ಚ ಪ್ರಪದ್ಯಂತೇ ನಿಜಾಂ ಗತಿಮ್~।। ೯೧~।।}
\end{verse}

ಪ್ರಕೃತಿಯ ಸಂಬಂಧವುಳ್ಳ ಈ ಜೀವರು ತಮ್ಮ ತಮ್ಮ ನೈಸರ್ಗಿಕಸ್ವಭಾವಕ್ಕೆ ಅನುಸಾರಿಯಾದ ಕರ್ಮಗಳನ್ನು ಆಚರಿಸಿ ತಮಗೆ ಯೋಗ್ಯವಾದ ಗತಿಯನ್ನು ಹೊಂದುತ್ತಾರೆ.

\begin{flushleft}
\textbf{[ವಿಶೇಷಾಂಶ:} 
\end{flushleft}

\begin{verse}
\textbf{ತ್ರಿವಿಧಾ ಭವತಿ ಶ್ರದ್ಧಾ ದೇಹಿನಾಂ ಸಾ ಸ್ವಭಾವಜಾ~।}\\\textbf{ಸಾತ್ವಿಕೀ ರಾಜಸೀ ಚೈವ ತಾಮಸೀ ಚೇತಿ ತಾಂ ಶೃಣು~।।}
\end{verse}

\begin{verse}
\textbf{ಸತ್ವಾನುರೂಪಾ ಸರ್ವಸ್ಯ ಶ್ರದ್ದಾ ಭವತಿ ಭಾರತ~।}\\\textbf{ಶ್ರದ್ದಾ ಮಯೋಽಯಂ ಪುರುಷೋ ಯೋ ಯಚ್ಛ್ರದ್ಧಃ ಸ ಏವ ಸಃ~।।} \vauthor{-ಗೀತಾ}
\end{verse}

ಜೀವರ ಆಸ್ತಿಕ್ಯ ಬುದ್ದಿಯು ಅವರ ಸ್ವಭಾವಸಿದ್ಧವಾದುದೇ. ಆ ಶ್ರದ್ದೆಯು ಸಾತ್ವಿಕೀ, ರಾಜಸೀ, ತಾಮಸೀ ಎಂದು ಮೂರು ವಿಧವಾಗಿವೆ. ಅದನ್ನು ಕೇಳು \enginline{-} ಸಮಸ್ತ ಜೀವರ ಆಸ್ತಿಕ್ಯಬುದ್ದಿಯು ಆಯಾ ಜೀವಸ್ವರೂಪಕ್ಕೆ ಅನುಗುಣವಾಗಿರುತ್ತದೆ. ಯಾವ ಜೀವನು ಯಾವ ವಿಧವಾದ ಶ್ರದ್ಧೆಯುಳ್ಳವನೋ ಅವನು ಅಂತಹ ಸ್ವಭಾವದವನೇ\enginline{-}ಸ್ವಭಾವವು ಬದಲಾಗುವುದಿಲ್ಲ.

ಜೀವರ ಸ್ವಭಾವವು ನೈಸರ್ಗಿಕ; ಎಂದಿಗೂ ಬದಲಾಗುವುದಿಲ್ಲ. ಈ ವಿವರಗಳು ಬ್ರಹ್ಮಸೂತ್ರಭಾಷ್ಯದ ಉಭಯಲಿಂಗಾಧಿಕರಣದಲ್ಲಿ ನಿರೂಪಿತವಾಗಿವೆ. 'ಓಂ ಅಪಿ ಸ್ಮರ್ಯತೇ ಓಂ' ಎಂಬ ಅದೇ ಅಧಿಕರಣ ಸೂತ್ರವೂ, ಅದರ ಕೆಳಗೆ ಶ‍್ರೀಮದಾಚಾರ್ಯರಿಂದ ಉದಾಹೃತವಾಗಿರುವ ಸ್ಕಾಂದಪುರಾಣ ವಚನವೂ ಈ ವಿಷಯವನ್ನು ಸ್ಪಷ್ಟ ಪಡಿಸುತ್ತವೆ.]

\begin{verse}
\textbf{ತತ್ರ ಯಃ ಸದ್ಗತಿಂ ಪ್ರಾಪ್ತುಂ ಪುಮಾನಿಚ್ಛತಿ ಚೇತ್ತದಾ~।}\\\textbf{ವರ್ಣಾಶ್ರಮೋಚಿತೇಷ್ವೇವ ಪ್ರಯತೇತ ಸ್ವಕರ್ಮಸು~।। ೯೨~।।}
\end{verse}

ಸದ್ಗತಿಯಾಗಬೇಕೆಂಬ ಇಚ್ಛೆಯುಳ್ಳ ಪುರುಷನು ತನ್ನ ವರ್ಣಾಶ್ರಮಗಳಿಗೆ ಉಚಿತವಾದ ಶಾಸ್ತ್ರೋಕ್ತವಾದ ಕರ್ಮವನ್ನು ಆಚರಿಸಲು ಪ್ರಯತ್ನಿಸಬೇಕು.

\begin{verse}
\textbf{ಕರ್ತಾ ಕಾರಯಿತಾ ವಿಷ್ಣುರಿತಿ ಯದ್ಯಪಿ ಸಂಮತಮ್~।}\\\textbf{ಸ್ವಾತಂತ್ರ್ಯಾತ್ ಸರ್ವಕರ್ತೃತ್ವಾತ್ ಹರಿರ್ಯತ್ನವತಾಂ ನೃಣಾಮ್~।।೯೩।। }
\end{verse}

\begin{verse}
\textbf{ಕಾರಯಿತ್ವಾ ಪ್ರೇರಯಿತ್ವಾ ಕೃತ್ವಾ ಕರ್ಮ ಚ ತತ್ಫಲಮ್~।}\\\textbf{ದದಾತಿ ಸದೃಶಾಕಾರಂ ನಾನ್ಯಥಾ ತು ಕದಾಚನ~।। ೯೪~।।}
\end{verse}

ಸಮಸ್ತ ಕರ್ಮಗಳನ್ನೂ ಮಾಡುವವನೂ, ಮಾಡಿಸುವವನೂ ವಿಷ್ಣುವೇ ಎಂಬುದು ಶಾಸ್ತ್ರ ಸಮ್ಮತವಾದ ವಿಷಯ. ಜೀವರಲ್ಲಿ ನಿಂತು ಪ್ರೇರಿಸಿ, ಕಾರ್ಯವನ್ನು ಜೀವರಿಂದ ಮಾಡಿಸಿ ಕರ್ಮಕ್ಕೆ ಅನುಗುಣವಾದ ಫಲವನ್ನು ಕೊಡುತ್ತಾನೆಯಲ್ಲದೇ ಅನ್ಯಥಾ ಎಂದಿಗೂ ಮಾಡುವುದಿಲ್ಲ.

\textbf{[ವಿಶೇಷಾಂಶ\enginline{-}} ಶ‍್ರೀಹರಿಗೆ ವೈಷಮ್ಯ ನೈರ್ಘೃಣ್ಯಾದಿ ದೋಷಗಳಿಲ್ಲ. ಜೀವರಿಂದ ಮಾಡಲ್ಪಟ್ಟ ಕರ್ಮಗಳನ್ನು ಅನುಸರಿಸಿ ಅದಕ್ಕೆ ತಕ್ಕ ಫಲಗಳನ್ನು ಕೊಡುತ್ತಾನೆ. ಈ ಪ್ರಮೇಯವು ವೈಷಮ್ಯನೈರ್ಘೃಣ್ಯಾಧಿಕರಣ (ಬ್ರಹ್ಮಸೂತ್ರಭಾಷ್ಯದಲ್ಲಿ) ವಿಸ್ತಾರವಾಗಿ ನಿರೂಪಿಸಲ್ಪಟ್ಟಿದೆ.\break “ಪುಣ್ಯೇನ ಪುಣ್ಯಂ ಲೋಕಂ ನಯತಿ ಪಾಪೇನ ಪಾಪಂ" ಇತಿ ಹಿ ಶ್ರುತಿಃ ಎಂಬುದಾಗಿ ಶ‍್ರೀಮದಾಚಾರ್ಯರು ಭಾಷ್ಯದಲ್ಲಿ ಹೇಳಿರುತ್ತಾರೆ\enginline{-} ಅಂದರೆ ಜೀವನು ಮಾಡಿದ ಪುಣ್ಯಕರ್ಮದಿಂದ ಸ್ವರ್ಗಾದಿಗಳನ್ನೂ ಜೀವನು ಮಾಡಿದ ಪಾಪಕರ್ಮದಿಂದ ನರಕಾದಿ ಲೋಕಗಳನ್ನೂ ಹೊಂದಿಸುತ್ತಾನೆ.

\begin{verse}
\textbf{ಸಮೋಽಹಂ ಸರ್ವಭೂತೇಷು ನ ಮೇದ್ವ್ಯೇಷ್ಯೋsಸ್ತಿ ನ ಪ್ರಿಯಃ~।}\\\textbf{ಯೇ ಭಜಂತಿ ತು ಮಾಂ ಭಕ್ತ್ಯಾ ಮಯಿ ತೇ ತೇಷು ಚಾಪ್ಯಹಂ~।।} \vauthor{-ಗೀತಾ}
\end{verse}

ನಾನು ಸಮಸ್ತ ಚೇತನರಲ್ಲಿಯೂ ಸಮನು-ಅಂದರೆ ನನಗೆ ನಿಷ್ಕಾರಣವಾಗಿ ಯಾರಲ್ಲಿಯೂ ದ್ವೇಷವಿಲ್ಲ; ನಿಷ್ಕಾರಣವಾಗಿ ಯಾರ ಮೇಲೆಯೂ ಪ್ರೀತಿಯಿಲ್ಲ. ಯಾರು ನನ್ನನ್ನು ಭಕ್ತಿಯಿಂದ ಸೇವಿಸುತ್ತಾರೆಯೋ ಅವರಿಗೆ ನನ್ನ ಮೇಲೆ ಪ್ರೀತಿ ಇರುತ್ತದೆ, ನಾನೂ ಸಹ ಅವರಲ್ಲಿ ಅನುಗ್ರಹ ಮಾಡುತ್ತೇನೆ.

\begin{verse}
\textbf{ಕರ್ತುಮಿಚ್ಛೋಃ ಸ್ವಯಂ ವಿಷ್ಣುಃ ಕರ್ತಾ ಕಾರಯಿತಾ ಸದಾ~।}\\\textbf{ಸ್ವಯಂ ಗಂತುಮಶಕ್ತಸ್ಯ ಪಾತುಮಿಚ್ಛೋಸ್ತನಂ ಶಿಶೋಃ~।। ೯೫~।।} 
\end{verse}

\begin{verse}
\textbf{ಕರ್ತೃತ್ವಂ ಮಾತುರೇವಾಸೀತ್ ತದಧೀನಸ್ತದಾಶ್ರಿಯಃ~।}\\\textbf{ತತ್ ಕರ್ತೃತಾ ತದಾಧೀನಾ ತಥಾ ಜೀವಸ್ಯ ಕರ್ತೃತಾ~।। ೯೬~।।}
\end{verse}

ಯಾವ ಜೀವನಿಗೆ ಸತ್ಕರ್ಮಮಾಡುವ ಇಚ್ಛೆಯಿದೆಯೋ ಅವನಲ್ಲಿ ಶ‍್ರೀಹರಿಯು ತಾನೇ ಕರ್ತೃವಾಗಿ ನಿಂತು ಅವನಿಂದ ಸತ್ಕರ್ಮವನ್ನು ಮಾಡಿಸುತ್ತಾನೆ. ಸ್ತನ್ಯಪಾನಮಾಡಲು ಮಗುವಿಗೆ ಇಚ್ಛೆಯಾದಾಗ ಮಗುವೇ ತಾಯಿಯ ಬಳಿ ಹೋಗಲು ಸಾಮರ್ಥ್ಯವಿಲ್ಲದ ಕಾರಣ ತಾಯಿಯೇ ಮಗುವಿಗೆ ಸ್ತನ್ಯಪಾನ ಮಾಡಿಸುತ್ತಾಳೆ. ಮಗುವು ತಾಯಿಯ ಅಧೀನ; ಸ್ವತಂತ್ರವಾಗಿ ಯಾವ ಕೆಲಸ ಮಾಡಲೂ ಅಸಮರ್ಥ, ತಾಯಿಯೇ ಆಶ್ರಯಳು. ಅದರಂತೆ ಜೀವನ ಕರ್ತೃತ್ವವು ಪರಮಾತ್ಮನ ಅಧೀನವೇ,

ಬ್ರಹ್ಮಸೂತ್ರಭಾಷ್ಯದ "ಕರ್ತೃತ್ವಾಧಿಕರಣದಲ್ಲಿ" ಜೀವನ ಪರಾಧೀನ ಕರ್ತೃತ್ವವು ನಿರೂಪಿತವಾಗಿದೆ. ಜೀವನಿಗೆ ಕರ್ತೃತ್ಯವೇ ಇಲ್ಲವಾದರೆ ವಿಧಿ- ನಿಷೇಧಗಳನ್ನು ಸೂಚಿಸುವ ಶಾಸ್ತ್ರಗಳಿಗೆ ಪ್ರಯೋಜನವೇ ಇಲ್ಲವಾಗುತ್ತದೆ. “ಓಂ ಪರಾತ್ತು ತಚ್ಛ್ರುತೇಃ ಓಂ” ಎಂಬ ಸೂತ್ರದಲ್ಲಿ ಜೀವನ ಕರ್ತೃತ್ವವು ಪರಮಾತ್ಮನ ಅಧೀನವೇ ಎಂದು ಸ್ಪಷ್ಟಪಡಿಸಲಾಗಿದೆ.

\begin{verse}
\textbf{ಯಥಾ ಪ್ರಕೃತಿಜೀವಾನಾಂ ಇಚ್ಛಾ ಸಂಜಾಯತೇ ಸದಾ~।}\\\textbf{ಯಥೇಚ್ಛತಿ ತಥಾ ವಿಷ್ಣುರ್ಜೀವಂ ಕಾರಯತೇ ಧ್ರುವಮ್~।। ೯೭~।।}
\end{verse}

ಜೀವನ ನೈಸರ್ಗಿಕ ಸ್ವಭಾವಕ್ಕೆ ಅನುಗುಣವಾಗಿಯೇ ಕರ್ಮಗಳಲ್ಲಿ ಇಚ್ಛೆ ಹುಟ್ಟುತ್ತದೆ. ಕರ್ಮಮಾಡಲು ಸ್ವಾತಂತ್ರ್ಯವಿಲ್ಲವಾದಕಾರಣ ಪರಮಾತ್ಮನು ಜೀವನಿಂದ ಆ ಇಚ್ಛೆಗೆ ಅನುಸಾರವಾಗಿ ಕರ್ಮವನ್ನು ಮಾಡಿಸುತ್ತಾನೆ. (ಈ ಪ್ರಮೇಯದ ವಿವರಣೆಗಳನ್ನು "ಕರ್ತೃತ್ವಾಧಿಕರಣ”ದಲ್ಲಿನ ಓಂ ಕೃತಪ್ರಯತ್ನಾಪೇಕ್ಷಸ್ತು ವಿಹಿತಪ್ರತಿಷೇಧಾವೈ ಯರ್ಥಾದಿಭ್ಯಃ ಓಂ” ಎಂಬ ಸೂತ್ರದಲ್ಲಿಯೂ ಮತ್ತು ಅದರ ಕೆಳಗಿನ ಭಾಷ್ಯದಲ್ಲಿಯೂ ಕಾಣಬಹುದು)

\begin{verse}
\textbf{ಪಿಪೀಲಿಕಾಂತಾ ಬ್ರಹ್ಮಾದ್ಯಾ ಜೀವಾ ವ ಚಾಂಡಗೋಚರಾಃ~।}\\\textbf{ತೇಷು ಕೋಽಪಿ ಕ್ವಚಿತ್ಕ್ವಾಪಿ ನ ಕರ್ತುಂ ಕ್ಷಮತೇ ನರಃ~।। ೯೮~।।}
\end{verse}

ಈ ಬ್ರಹ್ಮಾಂಡದಲ್ಲಿ ಸಣ್ಣ ಇರುವೆಯಂಥ ಪ್ರಾಣಿ ಮೊದಲುಗೊಂಡು ಚತುರ್ಮುಖಬ್ರಹ್ಮದೇವರವರೆವಿಗೂ ಇರುವ ಯಾವ ಜೀವನೂ, ಯಾವ ಕಾಲಕ್ಕೂ, ಎಷ್ಟು ಮಾತ್ರವೂ ಸ್ವತಂತ್ರವಾಗಿ ಕರ್ಮಮಾಡಲು ಶಕ್ತನಲ್ಲ.

\begin{verse}
\textbf{ಯದಾಶ್ನಾತಿ ತದಾಶ್ನಾತಿ ಯದಾ ವಕ್ತಿ ತದೈವ ಸಃ~।}\\\textbf{ಯದಾ ಜಿಘ್ರತಿ ಜಿಘ್ರಾತಿ ತದಾ ಜೀವಃ ಪರಾನುಗಃ~।। ೯೯~।।} 
\end{verse}

\begin{verse}
\textbf{ಯದಾ ಗಚ್ಛತಿ ವಾ ದೇವಸ್ತದಾ ಗಚ್ಛತಿ ದೇಹಭೃತ್~।}\\\textbf{ಶೃಣೋತಿ ಚ ಯದಾ ಬ್ರಹ್ಮಾ ತದಾ ಜೀವಃ ಶೃಣೋತಿ ಚ~।। ೧೦೦~।।} 
\end{verse}

\begin{verse}
\textbf{ಯದೈವಂ ಸ್ಪೃಶತೇ ಕಿಂಚಿತ್ ತದಾಯಂ ಸ್ಪೃಶತೇ ಜನಃ~।}\\\textbf{ಕಿಮತ್ರ ಬಹುನೋಕ್ತೇನ ನಾಕೃತಂ ಕ್ರಿಯತೇ ಜನೈಃ~।। ೧೦೧~।।}
\end{verse}

ಸರ್ವೆಶ್ವರನಾದ (ಬಿಂಬನಾದ) ಶ‍್ರೀಹರಿಯು ಊಟಮಾಡಿದರೆ ಪ್ರತಿಬಿಂಬನಾದ ಜೀವನು ಊಟಮಾಡುತ್ತಾನೆ; ಬಿಂಬನು ಮಾತನಾಡಿದರೆ ಪ್ರತಿಬಿಂಬನಾದ ಜೀವನು ಮಾತ\-ನಾಡುತ್ತಾನೆ. ಬಿಂಬನಾದ ಶ‍್ರೀಹರಿಯು ವಾಸನೆನೋಡಲು ಪ್ರತಿ ಬಿಂಬನಾದ ಜೀವನು ವಾಸನೆ ನೋಡುತ್ತಾನೆ. ಬಿಂಬನಾದ ಶ‍್ರೀಹರಿಯು ನಡೆಯಲು ಪ್ರತಿಬಿಂಬನಾದ ಜೀವನು ನಡೆಯುತ್ತಾನೆ. ಶ‍್ರೀಹರಿಯು ಶ್ರವಣ ಮಾಡಿದಾಗ ಜೀವನು ಶ್ರವಣಮಾಡುತ್ತಾನೆ. ಶ‍್ರೀಹರಿಯು ಸ್ಪರ್ಶಮಾಡಲು ಜೀವನು ಸ್ಪರ್ಶಿಸುತ್ತಾನೆ. ಇನ್ನು ಹೆಚ್ಚು ಹೇಳುವುದೇನಿದೆ? ಬಿಂಬನಾದ ಪರಬ್ರಹ್ಮನು ಮಾಡದೇ ಇರುವ ಯಾವ ಕೆಲಸವನ್ನೂ ಪ್ರತಿಬಿಂಬನಾದ ಜೀವನು ಮಾಡಲಾರನು.

\textbf{[ವಿಶೇಷಾಂಶ:} ಜೀವನ ಅಸ್ವಾತಂತ್ರ್ಯವು ಅನುಭವಸಿದ್ಧ. ಸ್ವತಂತ್ರನೆಂದರೆ ಶ‍್ರೀಹರಿ\-ಯೊಬ್ಬನೇ. ಶ‍್ರೀಹರಿಯೇ ಪ್ರತಿ ಜೀವನಿಗೂ ಬಿಂಬ. ಪ್ರತಿ ಜೀವನೂ ಪರಮಾತ್ಮನ ಪ್ರತಿಬಿಂಬ. ಬಿಂಬ ಮಾಡದೇ ಇರುವ ಯಾವ ಕಾರ್ಯವನ್ನೂ ಪ್ರತಿಬಿಂಬವು ಮಾಡಲಾರದೆಂಬ ವಿಷಯವೂ ಅನುಭವಸಿದ್ದ. ಕನ್ನಡಿಯ ಎದುರಿನಲ್ಲಿ ನಾವು ನಿಂತುಕೊಂಡು ನೋಡಿಕೊಂಡರೆ ನಮ್ಮ ಪ್ರತಿಬಿಂಬವು ಕಾಣುತ್ತದೆ. ನಾವು ಕೈ ಕಾಲುಗಳನ್ನು ಅಲ್ಲಾಡಿಸಿದರೆ ಕನ್ನಡಿಯೊಳಗಿನ ಪ್ರತಿಬಿಂಬವೂ ಅಲ್ಲಾಡಿಸುತ್ತದೆ. ಹೀಗೆ ಪ್ರತಿಬಿಂಬವು ಬಿಂಬನ ಅಧೀನ. ಕನ್ನಡಿಯು ಪ್ರತಿಬಿಂಬವನ್ನು ಪ್ರದರ್ಶಿಸಲು ಒಂದು ಉಪಾಧಿ. ಕನ್ನಡಿಯನ್ನು ತೆಗೆದುಬಿಟ್ಟರೆ ಪ್ರತಿಬಿಂಬವು ಇರುವುದೇ ಇಲ್ಲ. ಇದರಂತೆ ಜೀವನು ಪ್ರತಿಬಿಂಬನಾಗಿರಲು ಜೀವನ ಸ್ವರೂಪವೇ ಉಪಾಧಿ. ಬಿಂಬನಾದ ಶ‍್ರೀಹರಿಯು ಸರ್ವಕಾಲದಲ್ಲಿಯ, ನಿತ್ಯದಲ್ಲಿಯೂ, ಜೀವನ ಸ್ವರೂಪ ದೇಹದಲ್ಲಿಯೇ ಇರುವುದರಿಂದ ಉಪಾಧಿಯು ನಷ್ಟವಾಗಿ ಪ್ರತಿ ಬಿಂಬವೂ ನಷ್ಟವಾಗುತ್ತದೆಯೆಂಬ ಭಯವೇ ಇಲ್ಲ. ಪರಮಾತ್ಮನು ಹೇಗೆ ಅನಾದಿ ನಿತ್ಯನೋ ಹಾಗೆಯೇ ಜೀವನೂ ಸಹ ಸ್ವರೂಪ\-ದಿಂದ ಅನಾದಿ ನಿತ್ಯ, ಪರಮಾತ್ಮನು ಜೀವನ ಸ್ವರೂಪದೇಹದಲ್ಲಿ ಸದಾ ಇರುತ್ತಾನೆಯೆಂಬ ಪ್ರಮೇಯವನ್ನು

\begin{verse}
\textbf{ದ್ವಾ ಸುಪರ್ಣಾ ಸಯುಜಾ ಸಖಾಯಾ ಸಮಾನಂ ವೃಕ್ಷಂ ಪರಿಷಸ್ವಜಾತೇ~।}
\end{verse}

\noindent
ಎಂಬ ಶ್ರುತಿಯು ಸ್ಪಷ್ಟಪಡಿಸುತ್ತದೆ.

ಶ‍್ರೀಹರಿಯು ಜೀವನ ಬಿಂಬನೆಂಬುದೂ ಜೀವನು ಶ‍್ರೀಹರಿಯ ಪ್ರತಿಬಿಂಬನೆಂಬ ಪ್ರಮೇಯವು ಬ್ರಹ್ಮಸೂತ್ರಭಾಷ್ಯದ ಅಂಶಾಧಿಕರಣದ "ಓಂ ಆಭಾಸ ಏವ ಚ ಓಂ” ಎಂಬ ಸೂತ್ರದಲ್ಲಿಯೂ ಅದರ ಭಾಷ್ಯದಲ್ಲಿಯೂ ವಿವರಿಸಲ್ಪಟ್ಟಿದೆ.]

\begin{verse}
\textbf{ಏವಂ ಕರ್ತರಿ ದೇವೇಶೇ ಕರ್ತಾಹಮಿತಿ ಮನ್ಯತೇ~।}\\\textbf{ತೇನ ಸಂಸರತೇ ಜೀವೋ ದುಃಖೀ ಜನ್ಮಸು ಜನ್ಮಸು~।। ೧೦೨~।।}
\end{verse}

ಈ ರೀತಿ ಸರ್ವೆಶ್ವರನಾದ ಶ‍್ರೀಹರಿಯು ಪ್ರತಿಜೀವವನ್ನೂ ನಿಯಮನಮಾಡಿ ಕರ್ಮಗಳನ್ನು ಮಾಡಿಸುತ್ತಿರುವಾಗ ಜೀವನು ತಾನೇ ಸ್ವತಂತ್ರ ಕರ್ತ ಎಂಬುದಾಗಿ ತಿಳಿದು ಆ ಪಾಪದ ಫಲವಾಗಿ ಸಂಸಾರದಲ್ಲಿ ಅನೇಕ ಜನ್ಮಗಳನ್ನು ಎತ್ತಿ ದುಃಖವನ್ನು ಅನುಭವಿಸುತ್ತಾನೆ.

\begin{verse}
\textbf{ಏವಂ ವಿದ್ವಾನ್ ವಿನೀತಃ ಸನ್ ನಿಸ್ಪೃಹಶ್ಚ ಫಲೇಷು ಚ~।}\\\textbf{ಅನುಧ್ಯಾಯನ್ ಸದಾ ವಿಷ್ಣೋಃ ಸ್ವಾತಂತ್ರ್ಯಂ ಸರ್ವಕರ್ತೃತಾಮ್~।। }
\end{verse}

\begin{verse}
\textbf{ಫಲದತ್ವಂ ತದೀಶತ್ವಂ ಕುರ್ಯಾತ್ಕರ್ಮವಿಚಕ್ಷಣಃ~।}\\\textbf{ಅನುಷ್ಠಾತಾ ತಪಸ್ವೀ ಚ ಸರ್ವಧರ್ಮಪರೋಽಪಿ ವಾ~।। ೧೦೪~।।}
\end{verse}

ಜ್ಞಾನಿಯಾದವನು ತನ್ನ ಪರಾಧೀನತೆಯನ್ನು ತಿಳಿದು, ತಾನು ಆಚರಿಸತಕ್ಕ ಕರ್ಮಗಳ ಫಲವನ್ನು ಅಪೇಕ್ಷಿಸದೆ, ಶ‍್ರೀಹರಿಯ ಸ್ವಾತಂತ್ರ್ಯವನ್ನೂ ಸರ್ವಕರ್ತೃತ್ವವನ್ನೂ ನಿತ್ಯವೂ ಧ್ಯಾನಮಾಡುತ್ತಾ, ಕರ್ಮದ ಫಲವು ಭಗವಂತನ ಅಧೀನವೆಂಬುದನ್ನು ತಿಳಿದು ಶಾಸ್ತ್ರೋಕ್ತವಾದ ಸತ್ಕರ್ಮಗಳನ್ನೂ, ತಪಸ್ಸನ್ನೂ ಆಚರಿಸಬೇಕು,

\begin{flushleft}
\textbf{[ವಿಶೇಷಾಂಶ:}
\end{flushleft}

\begin{verse}
\textbf{ಕರ್ಮಣ್ಯೇವಾಧಿಕಾರಸ್ತೇ ಮಾ ಫಲೇಷು ಕದಾಚನ~।}\\\textbf{ಮಾ ಕರ್ಮಫಲಹೇತುರ್ಭೂರ್ಮಾ ತೇ ಸಂಗೋsಸ್ತ್ವಕರ್ಮಣಿ~।।} \vauthor{-ಗೀತಾ}
\end{verse}

ನಿನಗೆ ಕರ್ಮಮಾಡುವುದರಲ್ಲಿಯೇ ಅಧಿಕಾರ; ಫಲವಿಷಯದಲ್ಲಿ ಎಂದಿಗೂ ಅಧಿಕಾರವಿಲ್ಲ; ಆದುದರಿಂದ ನೀನು ಫಲವನ್ನೇ ಗುರಿಯಾಗಿಟ್ಟುಕೊಂಡು ಕರ್ಮಮಾಡಬೇಡ. ಹಾಗೆಯೇ ಕರ್ಮವನ್ನೇ ಮಾಡದೆ ಇರುವ ವಿಷಯದಲ್ಲಿ ಆಸಕ್ತಿಯನ್ನು ಹೊಂದಬೇಡ\enginline{-}ಅಂದರೆ ವಿಧ್ಯುಕ್ತವಾದ ಕರ್ಮವನ್ನು ಮಾಡಲೇಬೇಕು.

ಮೇಲೆ ಹೇಳಿದ ತಪಸ್ಸೆಂದರೆ ಕಾಡಿಗೆ ಹೋಗಿ, ಆಸನದಮೇಲೆ ಕುಳಿತು, ಮಾಡುವ ಧ್ಯಾನಮಾತ್ರವೇ ಅಲ್ಲ. ದೇಹ, ಇಂದ್ರಿಯ, ಮನಸ್ಸುಗಳಿಂದ ನಾವು ಮಾಡಬೇಕಾದ ಶಾಸ್ತ್ರೋಕ್ತವಾದ ಸತ್ಕರ್ಮಗಳೆಲ್ಲವೂ (ಸ್ನಾನ, ಸಂಧ್ಯಾವಂದನೆ, ಸಾಲಿಗ್ರಾಮಪೂಜೆ, ಏಕಾದಶೀವ್ರತ ಇತ್ಯಾದಿ) ತಪಸ್ಸೇ, ಇದು ನಿತ್ಯ, ನೈಮಿತ್ತಿಕ, ಕಾಮ್ಯವೆಂದು ದೈಹಿಕ, ವಾಚಿಕ, ಮಾನಸಿಕವೆಂದೂ ಪ್ರಭೇದಗಳಿಂದ ಕೂಡಿರುತ್ತದೆ. ಇದರ ವಿವರಣೆಗಳನ್ನು "ವಿಷ್ಣು ರಹಸ್ಯ"ದಲ್ಲಿಯ ಗೀತೆಯ ೧೭ನೇ ಅಧ್ಯಾಯದ ೧೪,೧೫,೧೬ನೇ ಶ್ಲೋಕಗಳಲ್ಲಿಯೂ ಕಾಣಬಹುದು.

\begin{verse}
\textbf{ದೇಹೇಂದ್ರಿಯಮನಃಸಾಧ್ಯಂ ಸತ್ಕರ್ಮ ತಪ ಉಚ್ಯತೇ~।}\\\textbf{ನಿತ್ಯಂ ನೈಮಿತ್ತಿ ಕಂ ಕಾಮ್ಯಮಿತಿ ತತ್ತ್ರಿವಿಧಂ ಮತಮ್~।।} \vauthor{-ವಿಷ್ಣು ರಹಸ್ಯ}
\end{verse}

\begin{verse}
\textbf{ವಿನಾ ಯದರ್ಪಣಂ ಸಮ್ಯಕ್ ಫಲಂ ನಾಪ್ನೋತ್ಯ ಸಂಶಯಮ್~।}\\\textbf{ದೇಹಿನಾಂ ಯಾಃ ಪ್ರವರ್ತಂತೇ ಪ್ರಾತರಾರಭ್ಯ ಯಾಃ ಕ್ರಿಯಾಃ~।। ೧೦೫~।।}\\\textbf{ತಾಶ್ಚ ವಿಷ್ಣ್ವರ್ಪಣಂ ಕುರ್ವನ್‌ ಕರ್ಮಪಾಶೈರ್ವಿಮುಚ್ಯತೇ~।}
\end{verse}

ಸತ್ಕರ್ಮಗಳನ್ನು ಶ‍್ರೀಹರಿಗೆ ಸಮರ್ಪಣೆ ಮಾಡದಿದ್ದರೆ ಕರ್ಮದ ಫಲವು ದೊರೆಯುವುದಿಲ್ಲ. ಮನುಷ್ಯರು ಪ್ರಾತಃಕಾಲದಿಂದ ಆರಂಭಿಸಿ ಏನೇನು ಸತ್ಕರ್ಮಗಳನ್ನು ಆಚರಿಸುವರೋ ಅವೆಲ್ಲವನ್ನೂ ಶ‍್ರೀವಿಷ್ಣುವಿನಲ್ಲಿ ಅರ್ಪಿಸಬೇಕು. ಹಾಗೆ ಮಾಡುವುದರಿಂದ ಸಮಸ್ತ ಕರ್ಮಬಂಧನದಿಂದ ಬಿಡುಗಡೆಯಾಗುತ್ತದೆ.

\begin{flushleft}
\textbf{[ವಿಶೇಷಾಂಶ:}
\end{flushleft}

\begin{verse}
\textbf{ನಿತ್ಯಂ ನೈಮಿತ್ತಿಕಂ ಕರ್ಮ ಕೃತ್ವಾ ವಿಷ್ಣೌ ಸಮರ್ಪಯೇತ್~।}\\\textbf{ನೈವ ತದ್ಬೀಜತಾಂ ಯಾನಿ ಭಾಷ್ಟ್ರು ಕ್ವಥಿತಧಾನ್ಯವತ್~।।} \vauthor{-ವಿಷ್ಣು ರಹಸ್ಯ}
\end{verse}

ನಿತ್ಯಕರ್ಮಗಳನ್ನೂ (ಸಂಧ್ಯಾವಂದನೆ ಇತ್ಯಾದಿ) ನೈಮಿತ್ತಿಕ (ಶ್ರಾದ್ಧ ಇತ್ಯಾದಿ) ಕರ್ಮಗಳನ್ನೂ ವಿಷ್ಣುವಿನಲ್ಲಿ ಅರ್ಪಿಸಬೇಕು. ಹೀಗೆ ಮಾಡಿದರೆ ಕರ್ಮವು, ಹುರಿದ ಬೀಜದಂತೆ, ನಿರ್ವೀರ್ಯವಾಗುತ್ತದೆ, ಅಂದರೆ ಆ ಕರ್ಮದಿಂದ ಮತ್ತೊಂದು ಕರ್ಮವು ಹುಟ್ಟುವುದು ಮುಂತಾದ ಬಂಧನವಿರುವುದಿಲ್ಲ.

\begin{verse}
\textbf{ಯತ್ಕರೋಷಿ ಯದಶ್ನಾಸಿ ಯಜ್ಜು ಹೋಷಿ ದದಾಸಿ ಯತ್~।}\\\textbf{ಯತ್ತ ಪಸ್ಯಸಿ ಕೌಂತೇಯ ತತ್ಕುರುಷ್ವ ಮದರ್ಪಣಮ್~।।} \vauthor{-ಗೀತಾ}
\end{verse}

ಅರ್ಜುನನೇ, ಯಾವ ಸತ್ಕರ್ಮವನ್ನು ಮಾಡುತ್ತೀಯೋ, ಯಾವುದನ್ನು ಊಟಮಾಡು\-ತ್ತೀಯೋ, ಏನು ಹೋಮಮಾಡುತ್ತೀಯೋ, ಏನು ದಾನ ಕೊಡುತ್ತೀಯೋ, ಏನು ತಪಸ್ಸನ್ನು ಆಚರಿಸುತ್ತೀಯೋ, ಅದೆಲ್ಲವನ್ನೂ ನನಗೆ (ಕೃಷ್ಣನಿಗೆ) ಅರ್ಪಿಸು.

ಅರ್ಪಣೆಯ ಅನುಸಂಧಾನ ಕ್ರಮವನ್ನು ಶ‍್ರೀ ಶ‍್ರೀ ಮಂತ್ರಾಲಯ ಪ್ರಭುಗಳ “ಪ್ರಾತಃ ಸಂಕಲ್ಪ ಗದ್ಯಂ” ಎಂಬ ಗ್ರಂಥದಲ್ಲಿ ಕಾಣಬಹುದು.]

\begin{verse}
\textbf{ಸೇವಾರ್ಥಂ ಭಗವಾನ್ ವಿಷ್ಣುಃ ದೇಶಕಾಲಕ್ರಿಯಾಶ್ರಯಾನ್~।। ೧೦೬~।।} 
\end{verse}

\begin{verse}
\textbf{ಸೃಷ್ಟವಾನ್ ಭಗವಾನಾದೌ ತಸ್ಮಾತ್ಪರಿಚರೇನ್ನಿಜಮ್~।}\\\textbf{ಇತಿ ಸಂಕಲ್ಪ್ಯ ಚೋತ್ಥಾಯ ಸರ್ವಕರ್ಮ ಸಮಾಚರೇತ್~।। ೧೦೭~।।}
\end{verse}

ಷಡ್ಗುಣೈಶ್ವರ್ಯನಾದ ಶ‍್ರೀವಿಷ್ಣುವು ಆಸ್ತಿಕರಿಗೆ ತನ್ನ ಸೇವೆಯು ದೊರೆಯಲೆಂಬ ಇಚ್ಛೆಯಿಂದ ಶ್ರುತಿಸಮ್ಮತವಾದ ಸತ್ಕರ್ಮಗಳನ್ನು ಸೃಷ್ಟಿಸಿ, ಅವುಗಳನ್ನು ಆಚರಿಸುವ ದೇಶ, ಕಾಲ, ಅಧಿಕಾರಿ ಮುಂತಾದ ವಿವರಗಳನ್ನು ಶಾಸ್ತ್ರಗಳ ಮೂಲಕ ಪ್ರಕಟಪಡಿಸಿದನು. ಅಧಿಕಾರಿಗಳು ತಮ್ಮ ವರ್ಣಾಶ್ರಮಗಳಿಗೆ ಉಚಿತವಾದ ಸತ್ಕರ್ಮಗಳನ್ನು ಶಾಸ್ತ್ರೋಕ್ತ ವಿಧಾನದಿಂದ ಆಚರಿಸಿ ಶ‍್ರೀವಿಷ್ಣುವಿನಲ್ಲಿ ಸಮರ್ಪಿಸಬೇಕು.

\begin{verse}
\textbf{ವಿಷ್ಣೋರರ್ಥೇ ಕರಿಷ್ಯಾಮಿ ಚಾಹರಾಮಿ ದದಾಮಿ ಚ~।}\\\textbf{ನಾತ್ಮಾರ್ಥಮಿತಿ ವೈ ಕ್ವಾಪಿ ವದೇತ್ಕರ್ಮ ಕರೋತಿ ಯಃ~।। ೧೦೮~।।}
\end{verse}

“ನಾನು ಮಾಡುವ ಸಕಲ ಸತ್ಕರ್ಮಗಳನ್ನೂ ಶ‍್ರೀವಿಷ್ಣು ಪ್ರೀತ್ಯರ್ಥವಾಗಿಯೇ ಮಾಡುತ್ತೇನೆ, ಕರ್ಮಕ್ಕೆ ಬೇಕಾದ ಸಾಮಗ್ರಿಗಳನ್ನು ಅವನ ಪ್ರೀತ್ಯರ್ಥವಾಗಿಯೇ ತರುತ್ತೇನೆ, ನನಗೋಸ್ಕರವಾಗಿ ಅಲ್ಲ" ಎಂಬುದಾಗಿ ಮನಸ್ಸಿನಲ್ಲಿಯೂ ಅನುಸಂಧಾನ ಮಾಡಿಕೊಳ್ಳಬೇಕು, ಹಾಗೂ ಬಾಯಿಂದಲೂ ಹೇಳಬೇಕು.

\begin{verse}
\textbf{ಶ್ರುತಿಸ್ಮೃತೀ ಹರೇರಾಜ್ಞೇ ತಸ್ಯಾಹಮನುಗಃ ಸದಾ~।}\\\textbf{ತಸ್ಮಾತ್ ಆಜ್ಞಾಧರೋ ಭೂತ್ವಾ ಯಾವಜ್ಜೀವಂ ವಸಾಮ್ಯಹಮ್~।। ೧೦೯~।।}
\end{verse}

“ಶ್ರುತಿ ಸ್ಮೃತಿಗಳು ಶ‍್ರೀಹರಿಯ ಆಜ್ಞೆಗಳೇ ಸರಿ; ನಾನಾದರೋ ಆ ಶ್ರುತಿ ಸ್ಮೃತಿಗಳನ್ನು ಸದಾ ಅನುಸರಿಸುವವನು. ಆದುದರಿಂದ ವಿಷ್ಣುವಿನ ಆಜ್ಞಾಧಾರಕನಾಗಿ ನನ್ನ ಜೀವನಪರ್ಯಂತವೂ ನಡೆಯುತ್ತೇನೆ.”

\begin{verse}
\textbf{ಇತ್ಥಂ ತ್ಯಜನ್ ಸದಾ ಕುರ್ವನ್ ಧ್ಯಾಯೇತ್ ವಿಷ್ಣುಮತಂದ್ರಿತಃ~।}\\\textbf{ಕರ್ಮಸಿದ್ಧಿಮವಾಪ್ನೋತಿ ನಾನ್ಯಥಾ ತು ಕದಾಚನ~।। ೧೧೦~।।}
\end{verse}

ಈ ರೀತಿಯಿಂದ ಸಂಕಲ್ಪ ಮಾಡಿಕೊಂಡು ನಿತ್ಯದಲ್ಲಿಯೂ ಆಲಸ್ಯರಹಿತನಾಗಿ ಶ‍್ರೀ ವಿಷ್ಣುವನ್ನು ಧ್ಯಾನಮಾಡುತ್ತಾ ಸತ್ಕರ್ಮಗಳನ್ನು ಆಚರಿಸಿ ಭಗವಂತನಿಗೆ ಅರ್ಪಣೆಮಾಡಿದರೆ ಮಾತ್ರ ಕರ್ಮಸಿದ್ದಿಯಾಗುತ್ತದೆ; ಅನ್ಯಥಾ ಎಂದಿಗೂ ಇಲ್ಲ.

\begin{center}
ಇತಿ ಶ‍್ರೀ ವಾಯುಪುರಾಣೇ ಮಾಘಮಾಸಮಾಹಾತ್ಮ್ಯೇ ಪಂಚದಶೋsಧ್ಯಾಯಃ 
\end{center}

\begin{center}
ಶ‍್ರೀ ವಾಯುಪುರಾಣಾಂತರ್ಗತ ಮಾಘಮಾಸ ಮಹಾತ್ಮ್ಯೆಯಲ್ಲಿ \\ ಹದಿನೈದನೇ ಅಧ್ಯಾಯವು ಸಮಾಪ್ತಿಯಾಯಿತು.
\end{center}

\newpage

\section*{ಅಧ್ಯಾಯ\enginline{-}೧೬}

\emptypage

\begin{flushleft}
\textbf{ಶಾಂಡಿಲ್ಯ ಉವಾಚ \enginline{-}}
\end{flushleft}

\begin{verse}
\textbf{ಸ್ವತಂತ್ರೋsಪಿ ಸ್ವಯಂ ವಿಷ್ಣುರ್ದೇವೈಃ ಸತ್ಯಾಭಿಮಾನಿಭಿಃ~।}\\\textbf{ಕರೋತಿ ತನುಭೃತ್ಕರ್ಮ ರಾಜ್ಯಾಂಗೈರಿವ ಭೂಮಿಪಃ~।। ೧~।।}
\end{verse}

\begin{flushleft}
ಶಾಂಡಿಲ್ಯರು ಹೇಳಿದರು:
\end{flushleft}

ವಿಷ್ಣುವು ತಾನೇ ಸ್ವತಂತ್ರನೂ, ಸಾಮರ್ಥ್ಯವುಳ್ಳವನೂ ಆಗಿದ್ದರೂ, ರಾಜನು ಹೇಗೆ ಅನೇಕ ಜನರನ್ನು ನಿಯಮಿಸಿ ಅವರಿಂದ ಕೂಡಿ ರಾಜ್ಯಭಾರ ಮಾಡುತ್ತಾನೆಯೋ, ಹಾಗೇ ತತ್ವಾಭಿಮಾನಿ ದೇವತೆಗಳ ಮೂಲಕ ಜಗತ್ತೆಂಬ ರಾಜ್ಯವನ್ನು ವಿಷ್ಣುವು ಪರಿಪಾಲಿಸುತ್ತಾನೆ.

\begin{verse}
\textbf{ಅವಕ್ತತ್ವನಿರೂಪಿಣ್ಯೌ ಭಾರತೀ ಚ ಸರಸ್ವತೀ~।}\\\textbf{ಬ್ರಹ್ಮವಾಯುಮಹಾತ್ಮಾನೌ ಮಹತ್ತತ್ವವಿಧಾಯಕೌ~।। ೨~।।}
\end{verse}

ಅವ್ಯಕ್ತತತ್ತ್ವಕ್ಕೆ ಸರಸ್ವತೀ ಮತ್ತು ಭಾರತೀದೇವಿಯರೂ ಅಭಿಮಾನಿಗಳು. ಮಹಾತ್ಮರಾದ ಚತುರ್ಮುಖಬ್ರಹ್ಮದೇವರೂ, ವಾಯುದೇವರೂ ಮಹತ್ತತತ್ತ್ವ ಕ್ಕೆ ಅಭಿಮಾನಿಗಳು.

\begin{verse}
\textbf{ಶೇಷವೀಂದ್ರಮೃಡಾ ಹ್ಯೇತೇ ಹೂ ಹಂತತ್ವಾಭಿಮಾನಿನಃ~।}\\\textbf{ಮನಸ್ತತ್ವಸ್ವರೂಪೌ ತು ಶಚೀಪತಿಮನೋಭವೌ~।। ೩~।।}
\end{verse}

ಶೇಷ, ಗರುಡ, ರುದ್ರರು ಅಹಂಕಾರತತ್ವಕ್ಕೂ, ಇಂದ್ರ, ಯಮರು ಮನಸ್ತತ್ವಕ್ಕೂ ಅಭಿಮಾನಿದೇವತೆಗಳು.

\begin{verse}
\textbf{ಅಥವಾಽವ್ಯಕ್ತತತ್ವಾತ್ಮಾ ದೇವೀ ಲಕ್ಷ್ಮೀರುದಾಹೃತಾ~।}\\\textbf{ಮಹತ್ತತ್ವಾತ್ಮಕೋ ಬ್ರಹ್ಮಾ ಮೃಡೋsಹಂಕೃತಿರೂಪವಾನ್~।। ೪~।।}
\end{verse}

ಅವ್ಯಕ್ತತತ್ತ್ವಕ್ಕೆ ಮಹಾಲಕ್ಷ್ಮಿದೇವಿಯರು ಅಭಿಮಾನಿ; ಮಹತ್ತತ್ವಾಭಿಮಾನಿಯು ಚತು\-ರ್ಮುಖಬ್ರಹ್ಮದೇವರೂ, ಅಹಂಕಾರತತ್ವಾಭಿಮಾನಿಯು ರುದ್ರದೇವರೂ ಆಗಿರುತ್ತಾರೆ.

(ಒಂದೇ ತತ್ತ್ವಕ್ಕೆ ಒಂದಕ್ಕಿಂತ ಹೆಚ್ಚು ಅಭಿಮಾನಿದೇವತೆಗಳಿದ್ದರೆ, ಶ್ರೇಷ್ಠರಾದ ದೇವತೆಯು ಮುಖ್ಯಾಭಿಮಾನಿಯೆಂದೂ, ಇತರ ದೇವತೆಗಳು ಅವಾಂತರ ಅಭಿಮಾನಿ ದೇವತೆಗಳೆಂದೂ ತಿಳಿಯಬೇಕು).

\begin{verse}
\textbf{ಮನಸ್ತತ್ವಂ ಪಾರ್ವತೀ ಚ ಕಾಮೇಂದ್ರಾವರ್ಥರೂಪಿಣೌ~।}\\\textbf{ದಿಗ್ವೇವತಾಶ್ಚಂದ್ರಮಾಶ್ಚ ಶ್ರೋತ್ರತತ್ವಾಭಿಮಾನಿನಃ~।। ೫~।।}
\end{verse}

ಪಾರ್ವತಿಯು ಮನಸ್ತತ್ವಕ್ಕೂ, ಇಂದ್ರಕಾಮರು ಅರ್ಥಾಭಿಮಾನಕ್ಕೂ, ದಿಗ್ದೇವತೆಗಳೂ ಚಂದ್ರನೂ ಶ್ರೋತೃತತ್ವಕ್ಕೂ ಅಭಿಮಾನಿದೇವತೆಗಳು.

\begin{verse}
\textbf{ಪ್ರಾಣಸ್ತ್ವಗಾತ್ಮಾ ಸೂರ್ಯಶ್ಚ ಚಕ್ಷುರಿಂದ್ರಿಯದೇವತಾ~।}\\\textbf{ಜಿಹ್ವಾತ್ಮಾ ವರುಣೋ ಘ್ರಾಣಸ್ವರೂಪಾವಶ್ವಿನೌ ತಥಾ~।। ೬~।।}
\end{verse}

ತ್ವಗಿಂದ್ರಿಯಕ್ಕೆ ಪ್ರಾಣನೂ, ಚಕ್ಷುರಿಂದ್ರಿಯಕ್ಕೆ ಸೂರ್ಯನೂ, ಜಿಹ್ವೇಂದ್ರಿಯಕ್ಕೆ \break ವರುಣನೂ, ಘ್ರಾಣೇಂದ್ರಿಯಕ್ಕೆ ಅಶ್ವಿನೀದೇವತೆಗಳೂ ಅಭಿಮಾನಿಗಳು.

\begin{verse}
\textbf{ವಾಕ್ ತತ್ವಾತ್ಮಾ ತಥಾ ವಹ್ನಿಃ ಪಾಣ್ಯಾ ತ್ಮ ದಕ್ಷಸಂಜ್ಞಕಃ~।}\\\textbf{ಪಾದಾತ್ಮಾ ಚ ಜಯಂತಶ್ಚ ಮಿತ್ರಃ ಪಾಯ್ವಾತ್ಮಕಸ್ತಥಾ~।। ೭~।।}
\end{verse}

ವಾಕ್ ತತ್ವಕ್ಕೆ ಅಗ್ನಿಯೂ, ಪಾಣಿ (ಹಸ್ತ)ಗಳಿಗೆ ದಕ್ಷನೂ, ಪಾದಗಳಿಗೆ ಜಯಂತನೂ, ವಾಯುವಿಗೆ ಮಿತ್ರನೂ ಅಭಿಮಾನಿಗಳು.

\begin{verse}
\textbf{ಉಪಸ್ಥಾತ್ಮಾ ಮನುಶ್ಚೈವ ಶಬ್ದಾತ್ಮಾ ಚ ಬೃಹಸ್ಪತಿಃ~।}\\\textbf{ಅಪಾನಃ ಸ್ಪರ್ಶತತ್ವಾತ್ಮಾ ರೂಪಾತ್ಮಾ ವ್ಯಾನನಾಮಕಃ~।। ೮~।।}
\end{verse}

ಉಪಸ್ಥಾಭಿಮಾನಿಯು ಮನು, ಶಬ್ದಾಭಿಮಾನಿ ಬೃಹಸ್ಪತಿ, ಸ್ಪರ್ಶಕ್ಕೆ ಅಪಾನನೆಂಬ ವಾಯು, ರೂಪಕ್ಕೆ ವ್ಯಾನನೆಂಬ ವಾಯು ಅಭಿಮಾನಿಗಳು.

\begin{verse}
\textbf{ರಸಾತ್ಮಾ ಚ ಉದಾನಶ್ಚ ಸಮಾನೋ ಗಂಧ ಉಚ್ಯತೇ~।}\\\textbf{ವ್ಯೋಮಾತ್ಮಾ ಗಜವಕ್ತ್ರಶ್ಚ ವಾಯುರ್ವಾಯುಸ್ವರೂಪವಾನ್~।। ೯~।।}
\end{verse}

ರಸದಲ್ಲಿ ಉದಾನ, ಗಂಧದಲ್ಲಿ ಸಮಾನ, ಆಕಾಶ (ಭೂತಾಕಾಶ)ದಲ್ಲಿ ವಿನಾಯಕ, ವಾಯುತತ್ವದಲ್ಲಿ ವಾಯುವು ಅಭಿಮಾನಿದೇವತೆಗಳು.

\begin{verse}
\textbf{ತೇಜಸ್ತತ್ತತ್ವಾ ತ್ಮಕೋ ವಹ್ನಿಃ ಜಲಾತ್ಮೋ ವರುಣೋ ಮಹಾನ್~।}\\\textbf{ಪೃಥಿವ್ಯಾತ್ಮಾ ಚ ಪೃಥಿವೀ ದೇವಾಃ ವಿಷ್ಣೋರಧೀನಗಾಃ~।। ೧೦~।।}
\end{verse}

ತೇಜಸ್ತತ್ವಕ್ಕೆ ಅಗ್ನಿಯೂ, ಜಲತತ್ವಕ್ಕೆ ವರುಣನೂ, ಪೃಥಿವೀತತ್ವಕ್ಕೆ ಧರಾದೇವಿಯ ಅಭಿಮಾನಿಗಳು, ಎಲ್ಲ ತತ್ವಾಭಿಮಾನಿದೇವತೆಗಳೂ ಶ‍್ರೀವಿಷ್ಣುವಿಗೆ ಅಧೀನರು.

\begin{verse}
\textbf{ಶಕ್ತಿಃ ಪ್ರತಿಷ್ಠಾ ಸಂವಿಚ್ಚ ಪ್ರಕೃತಿಃ ಸ್ಫೂರ್ತಿರೇವ ಚ~।}\\\textbf{ಕಲಾ ವಿದ್ಯಾ ಮತಿಶ್ಚೈವ ಮಾಯಾ ನಿಯತಿರೇವ ಚ~।। ೧೧~।। }
\end{verse}

\begin{verse}
\textbf{ಕಾಲಶ್ಚ ಪುರುಷಶ್ಚೇತಿ ದ್ವಾದಶಾತ್ಮಾ ಹರಿಃ ಸ್ಮೃತಃ~।}\\\textbf{ಅವ್ಯಕ್ತ್ಯಾದ್ಯಾಃ ಪೃಥಿವ್ಯಂತಾ ಯೇ ತತ್ತ್ವಾಹ್ವಾಃ ಸುರಾಃ ಸ್ಮೃತಾಃ~।। ೧೨~।।}
\end{verse}

\begin{verse}
\textbf{ಶರೀರಿಣಾಂ ಶರೀರಸ್ಥಾಃ ತತ್ತತ್ತತ್ವಾಶ್ಚ ನಿತ್ಯಶಃ~।}\\\textbf{ವಿಷ್ಣುನಾ ಪ್ರೇರಿತಃ ಸರ್ವೆ ಕಾರಯಂತ್ಯ ನಿಶಂ ಕ್ರಿಯಾಃ~।। ೧೩~।।}
\end{verse}

ಶಕ್ತಿ, ಪ್ರತಿಷ್ಠಾ, ಸಂವಿತ್, ಪ್ರಕೃತಿ, ಸ್ಫೂರ್ತಿ, ಕಲಾ, ವಿದ್ಯಾ, ಮತಿ, ಮಾಯಾ, ನಿಯತಿ, ಕಾಲ ಮತ್ತು ಪುರುಷ ಎಂಬ ಹನ್ನೆರಡು ಪರಮಾತ್ಮನ ರೂಪಗಳು. ಅವ್ಯಕ್ತ ತತ್ವದಿಂದ ಪೃಥಿವೀ ತತ್ವದ ವರೆವಿಗೂ ಮೇಲೆ ಹೇಳಿದ ತತ್ವಾಭಿ ಮಾನಿದೇವತೆಗಳು ಸಕಲ ಪ್ರಾಣಿಗಳ ಶರೀರಗಳಲ್ಲಿ ತಮ್ಮ ತಮ್ಮ ತತ್ವಗಳಲ್ಲಿ ನಿಂತು ಶ‍್ರೀ ಹರಿಯ ಪ್ರೇರಣಾನುಸಾರವಾಗಿ ತಮ್ಮ ತಮ್ಮ ಕಾರ್ಯಗಳನ್ನು ನಿತ್ಯದಲ್ಲಿಯೂ ಮಾಡುತ್ತಾರೆ.

\begin{verse}
\textbf{ತೇಷು ಸ್ಥಿತ್ವಾ ಸ್ವಯಂ ವಿಷ್ಣುಃ ತತ್ವಾವ್ಹೇಷು ಶರೀರಿಸು~।}\\\textbf{ತತ್ವಾದ್ಯೈಃ ಕಾರಯತ್ಯದ್ಧಾ ಪೃಥಕ್ ಶಕ್ತಾ ನ ತೇ ಯತಃ~।। ೧೪~।।}
\end{verse}

ಈ ತತ್ವಾಭಿಮಾನಿದೇವತೆಗಳಲ್ಲಿ ಸ್ವಯಂ ವಿಷ್ಣುವು ತಾನೇ ನಿಂತು ಅವರವರ ಕಾರ್ಯಗಳನ್ನು ಮಾಡಿಸುತ್ತಾನೆ; ತತ್ವಾಭಿಮಾನಿದೇವತೆಗಳಿಗೆ ತಾವೇ ಸ್ವತಂತ್ರವಾಗಿ ಕೆಲಸಮಾಡಲು ಸಾಮರ್ಥ್ಯವಿಲ್ಲ.

\begin{verse}
\textbf{ವಿಷ್ಣುರ್ವಾಯುಂ ಪುರಸ್ಕೃತ್ಯ ಸರ್ವೇಷ್ವೇತೇಷು ಸಂಚರನ್~।}\\\textbf{ಕರೋತಿ ಕಾರಯತ್ಯದ್ಧಾ ಜಗತ್ಕಾರ್ಯಮತಂದ್ರಿತಃ~।। ೧೫~।।}
\end{verse}

ಶ‍್ರೀಹರಿಯು ಸ್ವಯಂ ಸಮರ್ಥನಾಗಿದ್ದರೂ ಜೀವೋತ್ತಮರಾದ ವಾಯು ದೇವರನ್ನು ಮುಂದೆಮಾಡಿ ಎಲ್ಲರಲ್ಲಿಯೂ ನಿಂತು ಜಗತ್ತಿನ ಸಕಲಕಾರ್ಯಗಳನ್ನೂ ಆಲಸ್ಯರಹಿತನಾಗಿ ಮಾಡುತ್ತಾನೆ, ಮಾಡಿಸುತ್ತಾನೆ.

\textbf{ವಿಶೇಷಾಂಶ\enginline{-}}ಪರಮಾತ್ಮನ ಕಾರ್ಯಗಳಲ್ಲಿ ವಾಯುದೇವರು ಪ್ರಥಮ ಅಂಗಭೂತರು. ಈ ವಿಚಾರವು ಶ‍್ರೀ ಮಹಾಭಾರತತಾತ್ಪರ್ಯನಿರ್ಣಯದಲ್ಲಿ ಹೀಗೆ ಸ್ಪಷ್ಟ ಪಡಿಸಲಾಗಿದೆ:

\begin{verse}
\textbf{ತಸ್ಯಾಂಗಂ ಪ್ರಥಮೋ ವಾಯುಃ ಪ್ರಾದುರ್ಭಾವತ್ರಯಾನ್ವಿತಃ~।}\\\textbf{ಪ್ರಥಮೋ ಹನುಮನ್ನಾಮ ದ್ವಿತೀಯೋ ಭೀಮ ಏವ ಚ~। }\\\textbf{ಪೂರ್ಣಪ್ರಜ್ಞಸ್ತೃತೀಯಸ್ತು ಭಗವತ್ಕಾರ್ಯಸಾಧಕಃ~।।} \\\textbf{ತ್ರೇತಾದ್ಯೇಷು ಯುಗೇಷ್ವೇವ ಸಂಭೂತಃ ಕೇಶವಾಜ್ಞಯಾ~।।}
\end{verse}

ಶ‍್ರೀಹರಿಗೆ ವಾಯುದೇವರು ಪ್ರಥಮ ಅಂಗಭೂತರು. ವಾಯುದೇವರು ತ್ರೇತಾ, ದ್ವಾಪರ, ಕಲಿಯುಗಗಳಲ್ಲಿ ಪರಮಾತ್ಮನ ಆಜ್ಞೆಯಿಂದ ಮೂರು ಅವತಾರಗಳನ್ನು ಪಡೆದು ಭಗವಂತನ ಕಾರ್ಯದಲ್ಲಿ ಪ್ರಥಮ ಸಾಧಕರು. ಮೊದಲನೆಯ ಅವತಾರವು ಹನುಮದ್ರೂಪ, ಎರಡನೆಯದು ಭೀಮರೂಪ, ಮೂರನೆಯದು ಪೂರ್ಣಪ್ರಜ್ಞರೆಂಬ ರೂಪ,

ಮುಖ್ಯ ವಾಯುದೇವರ ಶ್ರೇಷ್ಠತೆ, ಅವರ ಸಾಮರ್ಥ್ಯ ವಿಶೇಷಗಳು ಬ್ರಹ್ಮಸೂತ್ರ ಭಾಷ್ಯದ "ಅಣ್ವಧಿಕರಣ”ದಲ್ಲಿ ವಿವರಿಸಲ್ಪಟ್ಟಿವೆ. ಹೀಗಿದ್ದರೂ ಅವರು ಶ‍್ರೀಹರಿಗೆ ಅಧೀನರು, ಶ‍್ರೀಹರಿಯಿಂದ ನಿಯಮ್ಯರು. “ವಿಷ್ಣು ರಹಸ್ಯ"ದ ಹದಿನೆಂಟನೆಯ ಅಧ್ಯಾಯದಲ್ಲಿ ವಾಯುದೇವರು ಜಗತ್ತಿನಲ್ಲಿ, ಜೀವರಲ್ಲಿ ನಡೆಯುವ ಸಕಲ ವ್ಯಾಪಾರಾದಿಗಳನ್ನು ಹೇಗೆ ನಿರ್ವಹಿಸುತ್ತಾರೆಂಬ ವಿಚಾರವು ವಿವರಿಸ್ಪಟ್ಟಿದೆ.

\begin{verse}
\textbf{ರಾಜಾಮಾತ್ಯ ಜನಾಧ್ಯಕ್ಷಪ್ರಜಾಧೀನಾ ಜಗತ್ಕ್ರಿಯಾಃ~।}\\\textbf{ತಥಾ ವಿಷ್ಣು ಮರುತ್ತಂತ್ರಾಃ ಪ್ರತಿಜೀವಕೃತಾಃ ಕ್ರಿಯಾಃ~।। ೧೬~।।}
\end{verse}

ಒಂದು ರಾಜ್ಯದ ಕಾರ್ಯವು ಹೇಗೆ ರಾಜ, ಅವನ ಅಧೀನರಾದ ಮಂತ್ರಿಗಳು ಮತ್ತು ಇತರ ಮುಖ್ಯ ಜನರಿಂದ ನಿರ್ವಹಿಸಲ್ಪಡುತ್ತದೆಯೋ, ಹಾಗೆಯೇ ಈ ಜಗತ್ತಿನಲ್ಲಿ ಪ್ರತಿಯೊಬ್ಬ ಚೇತನನ ಸಕಲ ಕರ್ಮಗಳೂ ವಿಷ್ಣು ಮತ್ತು ವಾಯುದೇವರಿಂದ ನಡೆಯುತ್ತವೆ.

\begin{verse}
\textbf{ಏತೇ ಜೀವಾದಿವಿಷ್ಣ್ವಂತಾಃ ಕರ್ತಾರಶ್ಚೋತ್ತರೋತ್ತರಮ್~।। ೧೭~।।}
\end{verse}

ಈ ಜೀವರಿಂದ ವಿಷ್ಣುವಿನ ಪರ್ಯಂತ ಸಕಲರೂ ಉತ್ತರೋತ್ತರರಾಗಿ ಕರ್ತೃಗಳೇ.

\begin{verse}
\textbf{ಜೀವಂ ವಿನಾ ತು ತತ್ವೇಶಾಃ ದಕ್ಷಾ ಜೀವಕ್ರಿಯಾಸು ಚ~।}\\\textbf{ತಥಾ ತತ್ವಾಭಿಧಾನ್ ಹಿತ್ವಾ ದಕ್ಷೋ ವಾಯುಶ್ಚ ತತ್ ಕೃತೌ~।। ೧೮~।।}
\end{verse}

ಜೀವರಲ್ಲಿ ನಡೆಯುವ ಕಾರ್ಯಗಳಲ್ಲಿ ಜೀವನ ಸಹಾಯವಿಲ್ಲದೆ ತತ್ವಾಭಿಮಾನಿ ದೇವತೆಗಳೇ ಸಮರ್ಥರು. ಆದರೆ ಅವರಿಗೆ ವಾಯುದೇವರ ಸಹಾಯವು ಅವಶ್ಯಕ. ವಾಯುದೇವರಿಗೆ ತತ್ವಾಭಿಮಾನಿದೇವತೆಗಳ ಸಹಾಯ ಬೇಕಿಲ್ಲ.

\begin{verse}
\textbf{ವಿನಾ ವಾಯುಂ ಸ್ವಯಂ ದಕ್ಷಸ್ತತ್ತಸ್ಸರ್ವಕ್ರಿಯಾಸು ಚ~।}\\\textbf{ವಿಷ್ಣುಂ ವಿನಾ ತಥಾ ಸರ್ವೇ ನಾಪಿ ತೇ ಚಲನೇ ಕ್ಷಮಾಃ~।। ೧೯~।।}
\end{verse}

ಶ‍್ರೀಹರಿಯು ವಾಯುದೇವರ ಸಹಾಯವಿಲ್ಲದೇ ತಾನೇ ಸಮಸ್ತ ಕಾರ್ಯಗಳನ್ನೂ ನಡೆಯಿಸುತ್ತಾನೆ. ಆದರೆ ವಿಷ್ಣುವಿನ ಸಹಾಯವಿಲ್ಲದೇ ಯಾವ ದೇವತೆಯ ಚಲನವಲನಾದಿ ಕರ್ಮಗಳನ್ನೂ ಮಾಡಲಾರನು.

\textbf{[ವಿಶೇಷಾಂಶ:} “ತೇನ ವಿನಾ ತೃಣಮಪಿ ನ ಚಲತಿ”, “ನ ಋತೇ ತ್ವತ್ ಕ್ರಿಯತೇ ಕಿಂಚನಾರೇ'' ಮುಂತಾದ ಪ್ರಮಾಣವಾಕ್ಯಗಳು ಶ‍್ರೀಹರಿಯ ಸಹಾಯವಿಲ್ಲದೆ ಯಾರೂ ಯಾವ ಕೆಲಸವನ್ನೂ ಎಂದಿಗೂ ಮಾಡಲಾರರು ಎಂಬ ಪ್ರಮೇಯವನ್ನು ಪುಷ್ಟಿಕರಿಸುತ್ತವೆ).

\begin{verse}
\textbf{ಅಹಂ ಸರ್ವಸ್ಯ ಪ್ರಭವೋ ಮತ್ತಃ ಸರ್ವಂ ಪ್ರವರ್ತತೇ~।।} \vauthor{-ಗೀತಾ}
\end{verse}

ನಾನು ಸಮಸ್ತ ಜಗತ್ತಿಗೆ ಉತ್ಪಾದಕನು. ನನ್ನಿಂದಲೇ ಸಮಸ್ತ ವ್ಯಾಪಾರಗಳೂ ನಡೆಯುತ್ತವೆ.]

\begin{verse}
\textbf{ಅಧಿಕಾರಸ್ಯ ಹೇತುತ್ವಾತ್ ತೇಷಾಂ ಮುಕ್ತೌ ತಥಾ ಹರಿಃ~।}\\\textbf{ಅಧಿಕಾರಂ ಪ್ರಾಪಯಿತ್ವಾ ತತೋ ಮೋಕ್ಷಂ ದದಾತಿ ಚ~।। ೨೦~।।}
\end{verse}

ಈ ತತ್ವಾಭಿಮಾನಿದೇವತೆಗಳ ಮೋಕ್ಷಕ್ಕೆ ಅಧಿಕಾರವೇ ಕಾರಣವಾಗಿರುವುದರಿಂದ ಶ‍್ರೀಹರಿಯು ಅವರಿಗೆ ಅವರ ಯೋಗ್ಯತಾನುಸಾರ ಅಧಿಕಾರಕೊಟ್ಟು ನಂತರ ಮೋಕ್ಷವನ್ನು ದಯಪಾಲಿಸುತ್ತಾನೆ.

\begin{verse}
\textbf{ತತ್ವಾಹ್ವಾನಾಂ ಚ ದೇವಾನಾಂ ಆತ್ಮಾರಾಮಃ ಸ್ವಯಂ ಹರಿಃ~।। ೨೧~।।}
\end{verse}

ತತ್ವಾಭಿಮಾನಿದೇವತೆಗಳಲ್ಲಿ ಸಾಕ್ಷಾತ್ ಶ‍್ರೀಹರಿಯೇ ಅಂತರ್ಯಾಮಿಯಾಗಿ ಇರುತ್ತಾನೆ.

\textbf{[ವಿಶೇಷಾಂಶ\enginline{-}}ಬ್ರಹ್ಮಸೂತ್ರಗಳಲ್ಲಿನ “ಓಂ ಅಂತರ್ಯಾಮ್ಯಧಿದೈ ವಾದಿಷು ತದ್ಧರ್ಮ ವ್ಯಪದೇಶಾತ್ ಓಂ” ಎಂಬ ಸೂತ್ರದಲ್ಲಿ ಈ ವಿಷಯವು ನಿರ್ಣಯಿಸಲ್ಪಟ್ಟಿದೆ).

\begin{verse}
\textbf{ಅಹಮಾತ್ಮಾ ಗುಡಾಕೇಶ ಸರ್ವಭೂತಾಶಯಸ್ಥಿತಃ~।।} \vauthor{-ಗೀತಾ}
\end{verse}

ನಾನು ಪ್ರಭುವಾಗಿದ್ದು, ಆಯಾ ಜೀವರ ವ್ಯಾಪಾರಗಳಿಗೆ ಪ್ರೇರಕನಾಗಿ, ಸಮಸ್ತ ಪ್ರಾಣಿಗಳ ಹೃದಯಗುಹೆಯಲ್ಲಿರುತ್ತೇನೆ. ]

\begin{verse}
\textbf{ಏವಂ ಜ್ಞಾತ್ವಾ ಚ ಕರ್ತೃತ್ವಂ ಜಾನೀಯಾಚ್ಚ ಕ್ರಿಯಾಫಲಮ್~।}\\\textbf{ತತ್ತತ್ಕರ್ಮಫಲಂ ಜ್ಞಾತ್ವಾ ಶ್ರದ್ಧಾವಾನ್ ಜಯತೇ ಯತಃ~।। ೨೨~।।}
\end{verse}

ಹೀಗೆ ಶ‍್ರೀಹರಿಯ ಅಂತರ್ಯಾಮಿತ್ವ, ಸರ್ವಕರ್ತೃತ್ವ, ಕರ್ಮಗಳಿಂದ ದೊರೆಯುವ ಫಲಗಳನ್ನು ತಿಳಿದುಕೊಂಡು ಕರ್ಮವನ್ನು ಆಚರಿಸಿದರೆ ಕೃತಾರ್ಥನಾಗುತ್ತಾನೆ.

\begin{verse}
\textbf{ಶ್ರದ್ಧಾ ಜನಿತ್ರೀ ಧರ್ಮಾಣಾಂ ಶ್ರದ್ಧಾ ಮಾತಾ ಪರಾಮತಾ~।}\\\textbf{ಶ್ರದ್ಧಾ ಹೀನಾಃ ಕ್ರಿಯಾಃ ಸರ್ವಾಃ ನ ಫಲಂತಿ ಕದಾಚನ~।। ೨೩~।।}
\end{verse}

ಶ್ರದ್ದೆಯೇ (ಆಸ್ತಿಕ್ಯಬುದ್ದಿ, ಕ್ರಿಯಾಶೀಲತೆ) ಸತ್ಕರ್ಮ ಮಾಡಲು ಪ್ರಚೋದಿಸುತ್ತದೆ. ಆದುದರಿಂದ ಸತ್ಕರ್ಮಕ್ಕೆ ಶ್ರದ್ದೆಯೇ ತಾಯಿ. ಶ್ರದ್ಧಾರಹಿತನಾದವನು ಮಾಡುವ ಸಕಲ ಸತ್ಕರ್ಮಗಳೂ ನಿಷ್ಪಲವಾಗುತ್ತವೆ.

\begin{flushleft}
\textbf{[ವಿಶೇಷಾಂಶ:}
\end{flushleft}

\begin{verse}
\textbf{ಶ್ರದ್ಧಾವಾನ್ ಲಭತೇ ಜ್ಞಾನಂ ಮತ್ಪರಃ ಸಂಯತೇಂದ್ರಿಯಃ~।}\\\textbf{ಜ್ಞಾನಂ ಲಬ್ಧ್ವಾಪರಾಂ ಶಾಂತಿಮಚಿರೇಣಾಧಿಗಚ್ಛತಿ~।।} \vauthor{-ಗೀತಾ}
\end{verse}

ಇಂದ್ರಿಯನಿಗ್ರಹವುಳ್ಳ, ಶ್ರದ್ದೆ (ಆಸ್ತಿಕ್ಯಬುದ್ದಿ)ಯುಳ್ಳ ಶ‍್ರೀಕೃಷ್ಣನೇ ಸರ್ವೋತ್ತಮನೆಂಬ ದೃಢವಾದ ಜ್ಞಾನವುಳ್ಳವನು ಅಪರೋಕ್ಷಜ್ಞಾನವನ್ನು ಹೊಂದಿ ನಂತರ ಶೀಘ್ರವಾಗಿ ಮೋಕ್ಷವನ್ನು ಹೊಂದುತ್ತಾನೆ.

\begin{verse}
\textbf{ಅಜ್ಞಶ್ಚಾಶ್ತ್ರದ್ಧಧಾನತ್ಚ ಸಂಶಯಾತ್ಮಾ ವಿನಶ್ಯತಿ~।}\\\textbf{ನಾಯಂ ಲೋಕೋಽಸ್ತಿ ನ ಪರೋ ನ ಸುಖಂ ಸಂಶಯಾತ್ಮನಃ~।।} \vauthor{-ಗೀತಾ}
\end{verse}

ಶ್ರದ್ಧಾ ಹೀನನೂ, ಅಜ್ಞಾನಿಯೂ, ವಿರುದ್ದ ಜ್ಞಾನಿಯೂ, ಸಂಶಯಯುಕ್ತನಾದವನೂ ನರಕಾದಿಗಳನ್ನು ಹೊಂದುತ್ತಾನೆ. ಸಂಶಯಗ್ರಸ್ತನಿಗೆ ಈ ಮನುಷ್ಯ ಲೋಕವೂ ಇಲ್ಲ, ಸ್ವರ್ಗಾದಿಗಳೂ ಇಲ್ಲ, ಸುಖವೂ ಇಲ್ಲ.]

\begin{verse}
\textbf{ತಾಂ ಪ್ರಾಹುರಾಸ್ತಿಕೀಂ ಬುದ್ದಿಂ ವಿದ್ಯಾ ಕರ್ಮಫಲಂ ಯತಃ~।}\\\textbf{ತತ್ರಾಪ್ರಾಮಾಣಿಕೀಂ ಬುದ್ಧಿಂ ನ ಕುರ್ಯಾತ್ಕರ್ಮಣಃ ಫಲೇ~।। ೨೪~।।}
\end{verse}

ಶ್ರದ್ಧೆ ಮತ್ತು ಕರ್ಮದ ಜ್ಞಾನ, ಕರ್ಮಫಲದ ವಿವೇಕಕ್ಕೆ "ಆಸ್ತಿಕ್ಯ ಬುದ್ದಿ" ಎಂದು ಹೆಸರು. ಕರ್ಮಗಳಿಂದ ಬರುವ ಫಲದ ವಿಷಯದಲ್ಲಿ ಅಪ್ರಾಮಾಣಿಕವಾದ ಬುದ್ಧಿ ಇರಬಾರದು.

\begin{verse}
\textbf{ಕ್ರಿಯಾಫಲಾನ್ಯರ್ಥವಾದಾನ್ ಯೇ ವದಂತಿ ನರಾಧಮಾಃ~।}\\\textbf{ತೇ ಸರ್ವೆ ನರಕಂ ಯಾಂತಿ ಯಾವದಾಭೂತಸಂಪ್ಲವಮ್~।। ೨೫~।।}
\end{verse}

ಕರ್ಮಫಲದ ವಿನಿಯೋಗದ ವಿಚಾರದಲ್ಲಿ ನಾನಾವಿಧವಾದ ಚರ್ಚೆಗಳನ್ನು ಮಾಡುವ ನರಾಧಮರೆಲ್ಲರೂ ಪ್ರಳಯಕಾಲದವರೆಗೂ ನರಕದಲ್ಲಿರುತ್ತಾರೆ.

\begin{verse}
\textbf{ಮಹದಪ್ಯಲ್ಪದಂ ಕಿಂಚಿತ್ ಕಿಂಚಿದಲ್ಪಂ‌ ಚ ಭೂರಿದಮ್~।}\\\textbf{ಕಿಂಚಿತ್ಸಮಫಲಂ ಕರ್ಮ ಗಹನಾ ಕರ್ಮಣೋ ಗತಿಃ~।। ೨೬~।।}
\end{verse}

ಕೆಲವು ವೇಳೆ ಕರ್ಮವು ಮಹತ್ತಾಗಿದ್ದರೂ ಅಲ್ಪ ಫಲವೇ ದೊರೆಯುತ್ತದೆ. ಕೆಲವು ವೇಳೆ ಅಲ್ಪ ಕರ್ಮಕ್ಕೆ ಮಹತ್ತಾದ ಫಲವು ಲಭಿಸುತ್ತದೆ. ಮತ್ತೆ ಕೆಲವು ವೇಳೆ ಕರ್ಮಕ್ಕೆ ತಕ್ಕ ಸಮನಾದ ಪ್ರತಿಫಲ ಸಿಗುತ್ತದೆ; ಕರ್ಮದ ಜ್ಞಾನವು ಬಹಳ ಗಹನವಾದುದು.

\textbf{[ವಿಶೇಷಾಂಶ:}

\begin{verse}
\textbf{ಕರ್ಮಣೋಽಹ್ಯಪಿ ಬೋದ್ಧ ವ್ಯಂ ಬೋದ್ಧವ್ಯಂ ಚ ವಿಕರ್ಮಣಃ~।}\\\textbf{ಅಕರ್ಮಣಶ್ಚ ಬೋದ್ಧವೂಂ ಯಂ ಗಹನಾ ಕರ್ಮಣೋ ಗತಿಃ~।।}\vauthor{-ಗೀತಾ}
\end{verse}

ಕರ್ಮಮಾಡುವ ಕ್ರಮವನ್ನೂ ಅರಿಯಬೇಕು, ವಿರುದ್ಧ ಕರ್ಮದ ಬಗೆಯನ್ನೂ, ತಿಳಿದುಕೊಳ್ಳಬೇಕು. ಕರ್ಮವನ್ನೇ ಮಾಡದಿದ್ದರೆ ಆಗುವ ಪರಿಣಾಮದ ಬಗೆಯನ್ನೂ ತಿಳಿದುಕೊಳ್ಳಬೇಕು. ಒಟ್ಟಿನಲ್ಲಿ ಕರ್ಮದ ಜ್ಞಾನವು ಗಹನ (ಗೂಢ).

\begin{verse}
\textbf{ಅಲ್ಪಮಾತ್ರಕೃತೋ ಧರ್ಮೋ ಭವೇತ್ ಜ್ಞಾನವತೋ ಮಹಾನ್~।}\\\textbf{ಮಹಾನಪಿ ಕೃತೋ ಧರ್ಮೋ ಹ್ಯಜ್ಞಾನಾಂ ನಿಷ್ಫಲೋ ಭವೇತ್~।।}\vauthor{-ಭಾರತ}
\end{verse}

ಜ್ಞಾನಯುಕ್ತನಾಗಿ ಅಲ್ಪ ಧರ್ಮಗಳನ್ನು ಮಾಡಿದರೂ ಮಹತ್ತಾದ ಫಲ, ಜ್ಞಾನರಹಿತನಾಗಿ ಮಹತ್ತಾದ ಕರ್ಮವನ್ನು ಮಾಡಿದರೂ ಅದು ನಿಷ್ಫಲವೇ.]

\begin{verse}
\textbf{ಶ್ರುತಿಸ್ಮೃತಿಪುರಾಣೋಕ್ತಂ ಶ್ರದ್ಧೇಯಂ ಹಿತಮಿಚ್ಛತಾ~।}\\\textbf{ಕರ್ಮಣಾರಾಧ್ಯಮಾನೇನ ವಿಷ್ಣುಭಕ್ತಿಃ ಸದಾ ಭವೇತ್~।। ೨೭~।।}
\end{verse}

ತನಗೆ ಹಿತವಾಗಬೇಕೆಂಬ ಇಚ್ಛೆಯುಳ್ಳವನು ಶ್ರುತಿ, ಸ್ಮೃತಿ, ಪುರಾಣಗಳಲ್ಲಿ ಹೇಳಲ್ಪಟ್ಟ ಸತ್ಕರ್ಮಗಳನ್ನು ಶ್ರದ್ಧೆಯಿಂದ ಆಚರಿಸಬೇಕು. ಇದರಿಂದ ವಿಷ್ಣು ಭಕ್ತಿಯು ಹುಟ್ಟುತ್ತದೆ.

\textbf{[ವಿಶೇಷಾಂಶ:} ಶಾಸ್ತ್ರಗಳಲ್ಲಿ ವಿಹಿತವೆಂದು ಹೇಳಿರುವ ಕರ್ಮಗಳನ್ನು ಶ್ರದ್ಧೆಯಿಂದ ಆಚರಿಸ\-ಬೇಕು, ಶಾಸ್ತ್ರದಲ್ಲಿ ಹೇಳದೇ ಇರುವ ಸತ್ಕರ್ಮಗಳ ಆಚರಣೆಯಿಂದ ಫಲವಿಲ್ಲ.

\begin{verse}
\textbf{ಯಃ ಶಾಸ್ತ್ರವಿಧಿಮುತ್ಸೃಜ್ಯ ವರ್ತತೇ ಕಾಮಕಾರತಃ~।}\\\textbf{ನ ಸ ಸಿದ್ಧಿಮವಾಪ್ನೋತಿ ನ ಸುಖಂ ನ ಪರಾಂ ಗತಿಮ್~।।} \vauthor{-ಗೀತಾ}
\end{verse}

ಶಾಸ್ತ್ರದಲ್ಲಿ ಹೇಳಿದ ವಿಧಾನವನ್ನು ಬಿಟ್ಟು ತನ್ನ ಇಚ್ಛಾನುಸಾರವಾಗಿ ಸತ್ಕರ್ಮಗಳನ್ನು ಆಚರಿಸು\-ವವನು ಪುರುಷಾರ್ಥೋಪಾಯವನ್ನು ಹೊಂದುವುದಿಲ್ಲ. ಐಹಿಕ ಸುಖವನ್ನು ಪಡೆಯುವುದಿಲ್ಲ. ಮೋಕ್ಷವನ್ನೂ ಪಡೆಯುವುದಿಲ್ಲ.

\begin{verse}
\textbf{ತಸ್ಮಾಚ್ಛಾಸ್ತ್ರಂ ಪ್ರಮಾಣಂ ತೇ ಕಾರ್ಯಾಕಾರ್ಯವ್ಯವಸ್ಥಿತೌ~।}\\\textbf{ಜ್ಞಾತ್ವಾ ಶಾಸ್ತ್ರವಿಧಾನೋಕ್ತಂ ಕರ್ಮಕರ್ತುಮಿಹಾರ್ಹಸಿ~।।} \vauthor{-ಗೀತಾ}
\end{verse}

ಹೇ ಅರ್ಜುನ, ಸತ್ಕರ್ಮಗಳು ಶಾಸ್ತ್ರವಿಹಿತವಾದವು ಯಾವವು, ಶಾಸ್ತ್ರದಲ್ಲಿ ಹೇಳದೇ ಇರುವ ಕರ್ಮಗಳು ಯಾವವು ಎಂಬುದನ್ನು ಅರಿತು ಶಾಸ್ತ್ರದಲ್ಲಿ ಹೇಳಿದ ಪ್ರಕಾರವೇ ನೀನು ಕರ್ಮಗಳನ್ನು ಆಚರಿಸಬೇಕು. ಕರ್ಮಾಚರಣೆಯಲ್ಲಿ ನಿನಗೆ ಶಾಸ್ತ್ರವೇ ಪ್ರಮಾಣ.]

\begin{verse}
\textbf{ಷಡಪ್ಯಂಗಾನಿ ಚೈತಾನಿ ಕ್ರಿಯಮಾಣಸ್ಯ ಕರ್ಮಣಃ~।}\\\textbf{ಶ್ರುತಿಸ್ಮೃತಿಪುರಾಣಾನಿ ಶ್ರದ್ಧಾ ಸತ್ಸಂಗತಿಸ್ತಥಾ~।। ೨೮~।। }
\end{verse}

\begin{verse}
\textbf{ಆರಾಧ್ಯ ದೇವತಾಭಕ್ತಿಸ್ತಥಾ ಕತ್ರೃತ್ವವರ್ಜಮ್~।}\\\textbf{ನಿಯಮಶ್ಚೇತಿ ಚತ್ವಾರಿ ತತ್ಸಾಧ್ಯಾಂಗಾನಿ ತಾನಿ ತು~।। ೨೯~।।}
\end{verse}

ಆಚರಿಸುವ ಸತ್ಕರ್ಮಗಳಿಗೆ ಆರು ಅಂಗಗಳು: ಶ್ರುತಿ-ಸ್ಮೃತಿ-ಪುರಾಣಗಳಲ್ಲಿ ನಿರ್ದಿಷ್ಟವಾಗಿರುವಿಕೆ, ಮಾಡುವಲ್ಲಿ ಶ್ರದ್ಧೆ, ಸಜ್ಜನರ ಸಹವಾಸ ಮತ್ತು ಸಹಾಯ, ಆರಾಧ್ಯ ದೇವತೆಯಾದ ಸರ್ವೋತ್ತಮನಾದ ವಿಷ್ಣುವಿನಲ್ಲಿ ದೃಢವಾದ ಭಕ್ತಿ, ನಾನೇ ಸ್ವತಂತ್ರ ಕರ್ತೃವಲ್ಲ ಎಂಬ ಭಾವನೆ ಮತ್ತು ನಿಯಮ. ಈ ಆರು ಮುಖ್ಯ ಅಂಗಗಳಿಗೆ ಸಾಧ್ಯಾಂಗಗಳೆಂದು ನಾಲ್ಕು ಇರುತ್ತವೆ.

\begin{verse}
\textbf{ಕಾಲೇ ಕಾಲೇ ಪ್ರಬೋಧಶ್ಚ ಯಥಾ ಶಕ್ತಿವಿಮರ್ಶನಮ್~।}\\\textbf{ಮನಃಸ್ವಾಸ್ಥ್ಯಂ ಶುಚಿತ್ವಂ ಚ ಪ್ರಧಾನಾಂಗಂ ತ್ವಿದಂ ಮತಮ್~।। ೩೦~।।}
\end{verse}

ಕಾಲಕಾಲಕ್ಕೆ ಉಪದೇಶ, ಶಕ್ತಿಗನುಸಾರವಾಗಿ ಶ‍್ರೀಹರಿಯ ಮಾಹಾತ್ಮ್ಯೆಯನ್ನು ಹೇಳುವುದು, ಕೇಳುವುದು ಮತ್ತು ವಿಮರ್ಶೆ, ನಿರ್ಮಲವಾದ ಮನಸ್ಸು, ಬಾಹ್ಯಾಂತರ ಶುಚಿತ್ವ.

\begin{verse}
\textbf{ವಿಷ್ಣೋರರ್ಪಣಮೇಕೈಕಂ ಸರ್ವೇಷಾಂ ಕರ್ಮಣಾಂ ಮತಮ್~।}\\\textbf{ನಿಷ್ಕಾಮತ್ವಂ ಹರೇಃ ಪ್ರೀತಿಃ ಕಾಮನಾಲೋಚನದ್ವಯಮ್~।। ೩೧~।।}
\end{verse}

ಸಕಲ ಸತ್ಕರ್ಮಗಳನ್ನೂ ಪ್ರತ್ಯೇಕ ಪ್ರತ್ಯೇಕವಾಗಿ ಅರ್ಪಣೆ ಮಾಡುವುದು, ಫಲಾಪೇಕ್ಷಾರಹಿತನಾಗಿ ಕರ್ಮವನ್ನು ಮಾಡುವುದು, ಹೇಗೆ ಆಚರಿಸುವುದರಿಂದ ಹರಿಪ್ರೀತಿಯಾಗುತ್ತದೆ ಎಂಬುದನ್ನು ಆಲೋಚಿಸಿ ಅದರಂತೆ ಆಚರಿಸಬೇಕು.

\begin{flushleft}
\textbf{[ವಿಶೇಷಾಂಶ:}
\end{flushleft}

\begin{verse}
\textbf{ಯತ್ಕ ರೋಷಿ ಯದಶ್ನಾಸಿ ಯಜ್ಜು ಹೋಷಿ ದದಾಸಿ ಯತ್~।}\\\textbf{ಯತ್ತ ಪಸ್ಯಸಿ ಕೌಂತೇಯ ತತ್ಕುರುಷ್ವ ಮದರ್ಪಣಮ್~।।} \vauthor{-ಗೀತಾ}
\end{verse}

ಯಾವ ಸತ್ಕರ್ಮವನ್ನು ಆಚರಿಸುತ್ತೀಯೋ, ಏನನ್ನು ಭುಂಜಿಸುತ್ತೀಯೋ, ಏನು ಹೋಮಮಾಡುತ್ತಿಯೋ, ಏನು ದಾನಮಾಡುತ್ತಿಯೋ, ಯಾವ ತಪಸ್ಸನ್ನು ಆಚರಿಸುತ್ತೀಯೋ ಅವೆಲ್ಲವನ್ನೂ ನನಗೆ (ಶ‍್ರೀಕೃಷ್ಣನಲ್ಲಿ) ಅರ್ಪಿಸು.]

\begin{verse}
\textbf{ತಥೈವ ತದುಪಾಂಗಾನಿ ತಾನ್ಯೇತಾನಿ ವಿದೋ ವಿದುಃ~।}\\\textbf{ಗಂಡೂಷಂ ದಂತಕಾಷ್ಠಂ ಚ ಶೌಚಾಚಾರೌ ತಥೈವ ಚ~।। ೩೨~।।}
\end{verse}

ಸತ್ಕರ್ಮಾನುಷ್ಠಾನಕ್ಕೆ ಉಪಾಂಗಗಳೂ ಸಹ ಇವೆ; ಹೀಗೆಂದು ಜ್ಞಾನಿಗಳ ಅಭಿಪ್ರಾಯ-\-ಹಲ್ಲನ್ನು ಚೆನ್ನಾಗಿ ಶುಭ್ರಗೊಳಿಸುವುದು, ಶುದ್ಧವಾದ ನೀರಿಂದ ಬಾಯಿ ಮುಕ್ಕಳಿಸುವುದು, ಶೌಚವಿಧಿ (ಮಲಮೂತ್ರ ವಿಸರ್ಜನೆ].

\begin{verse}
\textbf{ಸ್ನಾನಂ ವಸ್ತ್ರಪರೀಧಾನಂ ಸಪ್ತಮಂ ಪುಂಡ್ರಧಾರಣಮ್~।}\\\textbf{ಏವಂ ಕರ್ಮ ವಿಜಾನೀಯಾತ್ ಪಶ್ಚಾತ್ಕರ್ಮ ಸಮಾಚರೇತ್~।। ೩೩~।।}
\end{verse}

ವಿಧಿಪೂರ್ವಕವಾಗಿ ಸ್ನಾನ, ಶುದ್ಧಾಚಮನ, ಮಡಿವಸ್ತ್ರಧಾರಣ, ಮತ್ತು ಏಳನೆಯದಾಗಿ ಊರ್ಧ್ವಪುಂಡ್ರ ಧಾರಣ-ಇವುಗಳನ್ನು ಮಾಡಿಯೇ ನಿತ್ಯ ಕರ್ಮದಲ್ಲಿ (ಸಂಧ್ಯಾವಂದನೆ, ದೇವರ\-ಪೂಜಾ ಇತ್ಯಾದಿ) ತೊಡಗಬೇಕು.

\begin{flushleft}
\textbf{[ವಿಶೇಷಾಂಶ:}
\end{flushleft}

\begin{verse}
\textbf{ಊರ್ಧ್ವಪುಂಡ್ರವಿಹೀನಸ್ಯ ಶ್ಮಶಾನಸದೃಶಂ ಮುಖಮ್~।}\\\textbf{ಅವಲೋಕ್ಯ ಮುಖಂ ತೇಷಾಂ ಆದಿತ್ಯಮವಲೋಕಯೇತ್~।।\vauthor{(ಸ್ಮೃತಿವಾಕ್)}}
\end{verse}

ಊರ್ಧ್ವಪುಂಡ್ರವಿಲ್ಲದವನ ಮುಖವು ಸ್ಮಶಾನ ಸದೃಶ. ಅಂಥವನ ಮುಖ ನೋಡಿದರೆ ಆ ಪಾಪದ ಪ್ರಾಯಶ್ಚಿತ್ತವಾಗಿ ಸೂರ್ಯನನ್ನು ನೋಡಬೇಕು.

\begin{verse}
\textbf{ವಿಷ್ಟು ಪ್ರೀತಿಕರಂಭೂಯಸ್ತತೋsಪಿ ಪ್ರತ್ಯಹಂ ಲಿಖೇತ್~।}\\\textbf{ಸ್ವಸ್ವವರ್ಣೋಕ್ತ ಪುಂಡ್ರೇಷು ಹೃದಿಬಾಹೂದರಾದಿಷು~।। }\\\textbf{ಚಕ್ರಾದೀನಿ ಯಥಾಸ್ಥಾನಂ ದ್ವಾರವತ್ಯಾದಿಮೃತ್ಸ್ನಯಾ~।।} \vauthor{\enginline{-}ವಿಷ್ಣು ರಹಸ್ಯ}
\end{verse}

ವಿಷ್ಣುವಿನ ಚಿಹ್ನೆಗಳನ್ನು ದೇಹದಲ್ಲಿ ಧರಿಸುವುದು ವಿಷ್ಣು ಪ್ರೀತಿಕರವಾದ ಕರ್ಮ. ಅವರವರ ಜಾತಿಗೆ ಅನುಸಾರವಾಗಿ ಪ್ರತಿನಿತ್ಯವೂ ಗೋಪಿಚಂದನದಿಂದ ನಾಮಗಳನ್ನು ಧರಿಸಿ ಚಕ್ರವೇ ಮುಂತಾದ ಚಿಹ್ನೆಗಳನ್ನು ಎದೆ, ಬಾಹುಗಳು, ಉದರ ಮುಂತಾದ ಅವಯವಗಳಲ್ಲಿ ತಪ್ಪದೇ ಧರಿಸಬೇಕು.]

\begin{verse}
\textbf{ಕರ್ಮಣಾನೇನ ಸಜ್ಞಾನಂ ಜಾಯತೇ ಮೋಕ್ಷಸಾಧನಮ್~।}\\\textbf{ಏತತ್ ಜ್ಞಾನಂ ವಿನಾ ಕರ್ಮ ಕೃತಂ ವಾ ಬಹುಶೋ ನರೈಃ~।। ೩೪~।। }
\end{verse}

\begin{verse}
\textbf{ನ ತೇನ ವರ್ಧತೇ ಕ್ವಾಪಿ ನೋ ಕನೀಯಾನ್ ಭವೇತ್ ಕ್ವಚಿತ್~।}\\\textbf{ವೇದಸ್ಮೃತಿಪುರಾಣೇಷು ಯಾ ಚೋಕ್ತಾ ಫಲಸಾಧನಾ~।। ೩೫~।।}
\end{verse}

ಈ ರೀತಿಯಿಂದ ಸತ್ಕರ್ಮವನ್ನು ಆಚರಿಸುವುದರಿಂದ ಮೋಕ್ಷಕ್ಕೆ ಸಾಧನವಾದ ಯಥಾರ್ಥ\-ಜ್ಞಾನವು ಹುಟ್ಟುತ್ತದೆ. ಕರ್ಮದ ಈ ಮರ್ಮಗಳನ್ನೂ ಮತ್ತು ಅದರ ಜ್ಞಾನೋತ್ಪತ್ತಿ ಸಾಮರ್ಥ್ಯವನ್ನೂ ಅರಿಯದೆ ಜನರು ಎಷ್ಟು ಸತ್ಕರ್ಮಗಳನ್ನು ಆಚರಿಸಿದರೂ ಅದರಿಂದ ಅವನಿಗೆ ಪುಣ್ಯವೂ ಇಲ್ಲ, ಪಾಪವೂ ಇಲ್ಲ. ವೇದ-ಸ್ಮೃತಿ ಪುರಾಣಗಳಲ್ಲಿ ಹೇಳಿದ ಕರ್ಮಾನುಷ್ಟಾನವೇ ಫಲ\-ಸಾಧನ-ಜ್ಞಾನೋತ್ಪತ್ತಿಗೆ ಕಾರಣ.

\begin{verse}
\textbf{ಯೇ ಕರ್ಮವಿಮುಖಾ ಮಂದಾಸ್ತೇಷಾಮಾದೌ ಪ್ರವೃತ್ತಯೇ~।}\\\textbf{ರುಚ್ಯರ್ಥಮುದಿತಾ ಸಾ ಚ ನಿಃಸಂಗಃ ಸರ್ವತಃ ಸುಖೀ~।। ೩೬~।।}
\end{verse}

ಜ್ಞಾನಕ್ಕೆ ಕಾರಣವಾದ ಇಂತಹ ಕರ್ಮದಲ್ಲಿ ಆಲಸ್ಯದಿಂದಾಗಲೀ, ಅಜ್ಞಾನದಿಂದಾಗಲೀ ಪ್ರವೃತರಾಗದವರಿಗೆ ಕರ್ಮವನ್ನು ಆಚರಿಸಲು ಪ್ರೇರಿಸಲು ರುಚಿಗೋಸ್ಕರ ನಿಃಸಂಗವಾದ, ಎಲ್ಲ ಐಹಿಕ ಆಮುಷ್ಮಿಕ ಸುಖವನ್ನು ಮೊದಲು ಕೊಡುವ ಕರ್ಮದ ಬಗೆಯನ್ನು ಪುರಾಣಾದಿಗಳಲ್ಲಿ ಹೇಳಲಾಗಿದೆ.

\begin{verse}
\textbf{ಯಥಾ ಮಾತಾ ಶಿಶೋರಾದೌ ಮಹಾರೋಗಪ್ರಶಾಂತಯೇ~।}\\\textbf{ಔಷಧಂ ದಾತುಕಾಮಾ ಸಾ ಗುಡಂ ದತ್ವಾ ತತಃ ಪರಮ್~।। ೩೭~।। } 
\end{verse}

\begin{verse}
\textbf{ಪ್ರಯಚ್ಛತ್ಯೌಷಧಂ ಪಶ್ಚಾತ್ ತಥೈವ ಶ್ರುತಿಚೋದಿತಾಮ್~।}\\\textbf{ಶ್ರುತಿಸ್ಮೃತಿಪುರಾಣಾನಿ ಪಶ್ಚಾನ್ನಿಷ್ಕಾಮಚೋದನಾಮ್~।। ೩೮~।।}
\end{verse}

ರೋಗದಿಂದಿರುವ ಮಗುವಿಗೆ ಔಷಧಕೊಡುವ ಮೊದಲು ಸ್ವಲ್ಪ ಬೆಲ್ಲವನ್ನು ಕೊಟ್ಟು ನಂತರ ಔಷಧವನ್ನು ಕೊಡುವಂತೆ, ಸಂಸಾರವೆಂಬ ರೋಗದಿಂದ ಬಿಡುಗಡೆ ಹೊಂದಬೇಕೆಂಬ ಇಚ್ಛೆಯುಳ್ಳ ಸಜ್ಜನರಿಗೆ ಶ್ರುತಿ-ಸ್ಮೃತಿ-ಪುರಾಣಗಳು ಮೊದಲು ಐಹಿಕ ಸುಖವನ್ನು ಕೊಡುವ ಸಕಾಮಕರ್ಮವನ್ನು ಉಪದೇಶಿಸಿ, ನಂತರ ವೈರಾಗ್ಯ ಬಂದಮೇಲೆ ನಿಷ್ಕಾಮಕರ್ಮವನ್ನು ವಿಧಿಸುತ್ತವೆ.

\begin{verse}
\textbf{ತಥಾ ವೇದಾಶ್ಚ ಸರ್ವತ್ರ ಕ್ರಿಯಾ ತಾತ್ಪರ್ಯಕಾಃ ಪುರಾ~।}\\\textbf{ತಥೈವಾಮುಖ್ಯ ವೃತ್ತ್ಯಾತು ದೇವತಾಸ್ತ್ವಿತರಾ ಅಪಿ~।। }\\\textbf{ಸರ್ವೇ ವೇದಾಃ ಸಘೋಷಾಶ್ಚ ಸರ್ವೇವರ್ಣಾಶ್ರಮಾ ಅಪಿ~।। ೩೯~।। }
\end{verse}

\begin{verse}
\textbf{ಸಮಾತ್ರಾಃ ಸವಿಸರ್ಗಾಶ್ಚ ಸಾನುಸ್ವರಪದಾನಿ ಚ~।}\\\textbf{ಗುಣಸಾಂದ್ರೇ ಮಹಾವಿಷ್ಣೌ ಮಹಾತಾತ್ಪರ್ಯಗೌರವಾಃ~।। ೪೦~।।} 
\end{verse}

\begin{verse}
\textbf{ಪ್ರತಿಪಾದಯತಿ ಹ್ಯೇಷಾ ಮುಖ್ಯಾದ್ವಿಷ್ಣುಂ ತತಃ ಪರಮ್~।}\\\textbf{ತದಧೀನಾಂ ದೇವತಾಂ ಚ ಪಶ್ಚಾತ್ಕರ್ಮ ಯಥಾಕ್ರಮಮ್~।। ೪೧~।।}
\end{verse}

ವೇದಗಳಿಗೆ ಸರ್ವತ್ರ ಮುಖ್ಯ ತಾತ್ಪರ್ಯವೇನೆಂದರೆ ನಿಷ್ಕಾಮಕರ್ಮಗಳ ಪ್ರಚೋದನೆ, ತನ್ಮೂಲಕ ಯಥಾರ್ಥಜ್ಞಾನೋತ್ಪತ್ತಿ, ನಂತರ ಅಪರೋಕ್ಷಜ್ಞಾನಕ್ಕೆ ದಾರಿ; ಸಮಸ್ತ ವೇದಗಳೂ, ವೇದಗಳ ಘೋಷಗಳೂ, ಸರ್ವವರ್ಣಾಶ್ರಮಗಳೂ, ವೇದಗಳ ಮಾತ್ರೆಗಳೂ, ವಿಸರ್ಗಗಳೂ, ಅನುಸ್ವರಗಳೂ, ಪ್ರತಿಯೊಂದು ಪದವೂ ಶ‍್ರೀವಿಷ್ಣುವಿನ ಗುಣೋತ್ಕರ್ಷಣೆಯಲ್ಲಿಯೇ ಮಹಾತಾತ್ಪರ್ಯ ಉಳ್ಳವುಗಳು. ಹೀಗೆ ವೇದಗಳು ಮುಖ್ಯವಾಗಿ ವಿಷ್ಣುವಿನ ಗುಣಗಳನ್ನು ಪ್ರತಿಪಾದಿಸಿ, ನಂತರ ವಿಷ್ಣುವಿನ ಅಧೀನರಾದ ಇತರ ದೇವತೆಗಳ ಸ್ವರೂಪವನ್ನೂ, ಆಮೇಲೆ ಕರ್ಮಗಳನ್ನು ಸಾರುತ್ತವೆ.

\begin{flushleft}
\textbf{[ವಿಶೇಷಾಂಶ: }
\end{flushleft}

\begin{verse}
\textbf{ವೇದೇ ರಾಮಾಯಣೇ ಚೈವ ಪುರಾಣೇ ಭಾರತೇ ತಥಾ~।}\\\textbf{ಆದಾವಂತೇ ಚ ಮಧ್ಯೇ ಚ ವಿಷ್ಣುಸ್ಸರ್ವತ್ರ ಗೀಯತೇ~।।} \vauthor{ಇತಿ ಹರಿವಂಶೇಷು}
\end{verse}

ಸಕಲ ವೇದಗಳಲ್ಲಿ, ರಾಮಾಯಣದಲ್ಲಿ, ಎಲ್ಲ ಪುರಾಣಗಳಲ್ಲಿ, ಹಾಗೂ ಮಹಾಭಾರತ\-ದಲ್ಲಿ ಆದಿಯಲ್ಲಿ, ಮಧ್ಯದಲ್ಲಿ, ಅಂತ್ಯದಲ್ಲಿ-ಸರ್ವತ್ರ-ಶ‍್ರೀವಿಷ್ಣುವೇ ಸ್ತೋತ್ರ ಮಾಡಿಸಿ\break ಕೊಳ್ಳಲ್ಪಡುತ್ತಾನೆ].

\begin{verse}
\textbf{ಇತಿ ಜಾನನ್ವೇದವಿಧಿಂ ಯ ಆಚರತಿ ಸನ್ಮನಾಃ~।}\\\textbf{ಸ ಕರ್ಮಫಲಮಾಪ್ನೋತಿ ನಾನ್ಯಥಾ ತು ಕಥಂಚನ~।। ೪೨~।।}
\end{verse}

ಇಂತಹ ವಿಷ್ಣು ಸರ್ವೋತ್ತಮತ್ವ ಜ್ಞಾನದಿಂದ ಯುಕ್ತನಾಗಿಯೂ, ನಿರ್ಮಲವಾದ ಮನಸ್ಸಿನಿಂದಲೂ ವಿಧಿಪೂರ್ವಕವಾಗಿ ಕರ್ಮಗಳನ್ನು ಆಚರಿಸಿದರೆ ಆಗ ಫಲ ದೊರೆಯುತ್ತದೆ. ಅನ್ಯಥಾ ಫಲವಿಲ್ಲ. 

\newpage

\begin{flushleft}
\textbf{[ವಿಶೇಷಾಂಶ:}
\end{flushleft}

\begin{verse}
\textbf{ಸತ್ಕರ್ಮ ವಿಷ್ಣು ಮುದ್ದಿಶ್ಯ ಕೃತಂ ತಸ್ಮಿನ್ ಸಮರ್ಪಿತಮ್~।}\\\textbf{ಫಲಾಧಿಕ್ಯಂ ಭವೇತ್ತಸ್ಯ ಮೋಕ್ಷಬೀಜಂ ಚ ಜಾಯತೇ~।।} \vauthor{-ವಿಷ್ಣು ರಹಸ್ಯ}
\end{verse}

ವಿಷ್ಣುವನ್ನೇ ಉದ್ದೇಶಿಸಿ ಮಾಡಲ್ಪಟ್ಟ ಸತ್ಕರ್ಮವು ವಿಷ್ಣುವಿನಲ್ಲಿಯೇ ಭಕ್ತಿ ಪೂರ್ವಕ ಸಮರ್ಪಿಸಿದರೆ, ಆ ಕರ್ಮಕ್ಕೆ ಅಧಿಕ ಫಲವು ದೊರೆಯುವುದಲ್ಲದೇ, ಅದು ಮೋಕ್ಷಕ್ಕೆ\break ಕಾರಣವೂ ಆಗುತ್ತದೆ.]

\begin{verse}
\textbf{ಜ್ಞಾ ತ್ವೈವ ಕುರ್ಯಾತ್ಕರ್ಮಾಣಿ ಯಸ್ಮಾತ್ಫಲಮವಾಪ್ನುಯಾತ್~।}\\\textbf{ತಸ್ಮಾತ್ ಗುರುಕುಲೇ ವಾಸಃ ಸರ್ವತ್ರಾದೌ ವಿಧೀಯತೇ~।। ೪೩~।।}
\end{verse}

ಕರ್ಮದ ಮರ್ಮಗಳನ್ನರಿತು ಆಚರಿಸುವುದರಿಂದ ಫಲ ದೊರೆಯುತ್ತದೆಯಾದುದರಿಂದ ಆ ಮರ್ಮಗಳನ್ನು ತಿಳಿದುಕೊಳ್ಳಲು ಗುರುಗಳ ಬಳಿಯಲ್ಲಿದ್ದು ಅವರನ್ನು ಸೇವಿಸಿ ಅವರಿಂದ ತಿಳಿದುಕೊಳ್ಳಬೇಕೆಂಬ ನಿಯಮವಿದೆ.

\begin{verse}
\textbf{ಯಜ್ಞಾ ದಿಸರ್ವಧರ್ಮಾಣಾಂ ಗುರುಸೇವಾ ಪರಾ ಸ್ಮೃತಾ~।}\\\textbf{ತತೋ ಗುರುಂ ಸಮಾಸಾದ್ಯ ವಿನಯಾತ್ತಂ ಪ್ರಸಾದ್ಯ ಚ~।। ೪೪~।। }
\end{verse}

\begin{verse}
\textbf{ಜ್ಞಾನಯೋಗಂ ಕ್ರಿಯಾಯೋಗಂ ಜ್ಞಾತ್ವಾ ಗುರ್ವನುಜ್ಞಯಾ~।}\\\textbf{ಕ್ರಿಯಾಯೋಗಂ ಜ್ಞಾನಯೋಗಂ ಸಾಧಯೇತ್ಸುಸಮಾಹಿತಃ~।।}\\\textbf{ಜ್ಞಾನಾದ್ವಿಷ್ಣುಂ ಪ್ರಸಾದ್ಯಾಥ ಪ್ರಸಾದಾತ್ ಮೋಕ್ಷಮಾಪ್ನುಯಾತ್~।। ೪೫~।।}
\end{verse}

ಯಜ್ಞವೇ ಮುಂತಾದ ಪುಣ್ಯ ಕರ್ಮಗಳಲ್ಲಿ ಗುರುಸೇವೆಯೆಂಬುದು ಬಹಳ ಶ್ರೇಷ್ಠವೆಂದು ಹೇಳಲಾಗುತ್ತದೆ. ಆದುದರಿಂದ ಯೋಗ್ಯ ಗುರುವಿನ ಬಳಿ ಹೋಗಿ ವಿನಯದಿಂದ, ಸೇವೆಯಿಂದ ಅವರನ್ನು ಸಂತೋಷಪಡಿಸಿ, ಅವರಿಂದ ಜ್ಞಾನಮಾರ್ಗ, ಕರ್ಮಮಾರ್ಗಗಳ ವಿವರಗಳನ್ನು ಅರಿಯಬೇಕು. ನಂತರ ಅವರ ಆಜ್ಞೆಯನ್ನು ಪಡೆದು ಜ್ಞಾನಯೋಗ-ಕರ್ಮಯೋಗಗಳನ್ನು ಯಥಾಯೋಗ್ಯವಾಗಿ ಸಾಧಿಸಿ ಅದರಿಂದ ಶ‍್ರೀವಿಷ್ಣುನ ಅನುಗ್ರಹವನ್ನು ಪಡೆಯಬೇಕು. ಅನುಗ್ರಹದ ಫಲವಾಗಿ ಮೋಕ್ಷವನ್ನು ಹೊಂದಬೇಕು.

\textbf{[ವಿಶೇಷಾಂಶ: }ಮೋಕ್ಷಸಾಧನೆಯಲ್ಲಿ ಗುರುಗಳ ಪಾತ್ರವು ಬಹಳ ಶ್ರೇಷ್ಠ ಎಂದು ಶಾಸ್ತ್ರಗಳು ಹೇಳುತ್ತವೆ.

\begin{verse}
\textbf{ತದ್ವಿಜ್ಞಾನಾರ್ಥಂ ಸ ಗುರುಮೇವಾಭಿಗಚ್ಛೇತ್~।}\\\textbf{ಸಮಿತ್ಪಾಣಿಃ ಶ್ರೋತ್ರಿಯಂ ಬ್ರಹ್ಮನಿಷ್ಠಮ್~।।} \vauthor{ -ಭಾಷ್ಯ}
\end{verse}

ಪರಬ್ರಹ್ಮನ ವಿಶೇಷ ಜ್ಞಾನಾರ್ಥವಾಗಿ ಜ್ಞಾನೇಚ್ಛುವು ದರ್ಭಪಾಣಿಯಾಗಿ ವೇದಾರ್ಥಗಳನ್ನು ಚೆನ್ನಾಗಿ ಬಲ್ಲ, ಸದಾ ಪರಬ್ರಹ್ಮನನ್ನೇ ಧ್ಯಾನಿಸುತ್ತಲಿರುವ ಗುರುಗಳನ್ನೇ ಶರಣುಹೊಂದಬೇಕು.

\begin{verse}
\textbf{ಯಸ್ಯ ದೇವೇ ಪರಾಭಕ್ತಿರ್ಯಥಾ ದೇವೇ ತಥಾ ಗುರೌ~।}\\\textbf{ತಸ್ಯೈತೇ ಕಥಿತಾ ಹ್ಯರ್ಥಾಃ ಪ್ರಕಾಶಂತೇ ಮಹಾತ್ಮನಃ~।।} \vauthor{ -ಭಾಷ್ಯ}
\end{verse}

ಪರಬ್ರಹ್ಮನಲ್ಲಿ ಹೇಗೆ ಶ್ರೇಷ್ಠವಾದ, ವಿಚ್ಛಿತ್ತಿರಹಿತವಾದ, ಭಕ್ತಿ ಇರುತ್ತದೆಯೋ ಹಾಗೆಯೇ ಗುರುಗಳಲ್ಲಿಯೂ ಯಥಾಯೋಗ್ಯವಾದ ಭಕ್ತಿ ಇರಬೇಕು. ಆಗ ಗುರುಗಳಿಂದ ಹೇಳಲ್ಪಟ್ಟ ಉಪದೇಶವಾಕ್ಯಗಳ ಅರ್ಥಗಳು ಚೆನ್ನಾಗಿ ಪ್ರಕಾಶಕ್ಕೆ ಬರುತ್ತವೆ. ಇನ್ನೂ ಹೆಚ್ಚಿನ ಅನುಗ್ರಹಕ್ಕೆ ಪಾತ್ರನಾಗಿದ್ದರೆ ಗುರುಗಳಿಂದ ಹೇಳಲ್ಪಡದೇ ಇರುವ ಅರ್ಥಗಳೂ ಸಹ ಸ್ಫೂರ್ತಿಗೆ ಬರುತ್ತವೆ.

ಗುರುಗಳಿಂದಲೇ ಉಪದೇಶ ಪಡೆಯುವುದರ ಅವಶ್ಯಕತೆ, ಗುರುಗಳ ಅನುಗ್ರಹದ ಶ್ರೇಷ್ಠತೆ, ಗುರುಸೇವಾದಿಂದ ಬರುವ ಫಲ ಮುಂತಾದ ರಹಸ್ಯಗಳು ಬ್ರಹ್ಮಸೂತ್ರಭಾಷ್ಯದ “ಪ್ರದಾನಾಧಿಕರಣ” ಹಾಗೂ “ಲಿಂಗಭೂಯಸ್ತ್ವಾಧಿಕರಣ?” ಇವುಗಳಲ್ಲಿ ವಿವರಿಸಲ್ಪಟ್ಟಿವೆ. "ಯಥಾ ಗುರುದತ್ತಂ ತಥೈವ ಭವತಿ" -ಗುರುಗಳು ಹೇಗೆ ಉಪದೇಶಕೊಡುವರೋ ಅದರಂತೆ ನಡೆಯುತ್ತದೆ. “ಗುರುಪ್ರಸಾದೋ ಬಲವಾನ್ ನ ತಸ್ಮಾದ್ಬಲವತ್ತರಮ್” -ಗುರುಪ್ರಸಾದವೇ ಬಲವತ್ತರವಾದುದು, ಅದಕ್ಕಿಂತ ಪ್ರಬಲವಾದುದು ಯಾವುದೂ ಇಲ್ಲ.

ಹೀಗೆ ಹೇಳಿದ ಮಾತ್ರಕ್ಕೆ ಗುರುಗಳ ಅನುಗ್ರಹದಿಂದಲೇ ಆಗಲಿ, ಸ್ವಯಾರ್ಜಿತ ಜ್ಞಾನದಿಂದಲೇ ಆಗಲಿ ಮೋಕ್ಷ ದೊರೆಯುವುದೆಂದು ತಿಳಿಯಬಾರದು. ಇವೆಲ್ಲಕ್ಕಿಂತ ಪ್ರಬಲ\-ವಾದುದು ಶ‍್ರೀಹರಿಯ ಪ್ರಸಾದ.

\begin{verse}
\textbf{ಯಮೇವೈಷ ವೃಣುತೇ ತೇನ ಲಭ್ಯಃ}\\\textbf{ತಸ್ಯೈಷ ಆತ್ಮಾ ವಿವೃಣುತೇ ತನೂಂ ಸ್ವಾಮ್~।।} \vauthor{ -ಭಾಷ್ಯ}
\end{verse}

ತನ್ನ ಇಚ್ಛೆಯಿಂದ ಶ‍್ರೀಹರಿಯು ಯಾರನ್ನು ಸ್ವೀಕರಿಸುತ್ತಾನೆಯೋ ಅವನಿಗೆ ಮಾತ್ರ ತನ್ನ ಸ್ವರೂಪವನ್ನು ಯಥಾಯೋಗ್ಯವಾಗಿ ತೋರಿಸುತ್ತಾನೆ.

\begin{verse}
\textbf{ಯತೋ ನಾರಾಯಣಪ್ರಸಾದಮೃತೇ ನ ಮೋಕ್ಷಃ}\\\textbf{ನ ಚ ಜ್ಞಾನಂ ವಿನಾಽತ್ಯರ್ಥಪ್ರಸಾದಃ~।।} \vauthor{ -ಭಾಷ್ಯ}
\end{verse}

ಶ‍್ರೀ ನಾರಾಯಣನ ಅನುಗ್ರಹವಿಲ್ಲದೇ ಮೋಕ್ಷವಿಲ್ಲ. ಭಗವದಪರೋಕ್ಷ ಜ್ಞಾನವಿಲ್ಲದೇ ಮೋಕ್ಷಕ್ಕೆ ಬೇಕಾದ ಅನುಗ್ರಹವು ಇಲ್ಲ. ಆದುದರಿಂದ ಪರಬ್ರಹ್ಮನ ವಿಷಯದ ವಿಚಾರವನ್ನು ಮಾಡಬೇಕು].

\begin{verse}
\textbf{ಭುಂಜಂತಂ ಪಥ್ಯಮೇವಾನ್ನಂ ನ ತುದಂತೇ ರುಜೋ ಯಥಾ~।}\\\textbf{ನೇಂದ್ರಿಯಾಣಿ ತುದಂತ್ಯೇನಂ ಶನೈರ್ನಿಯಮಕಾರಿಣಮ್~।। ೪೬~।।}
\end{verse}

ರೋಗದಿಂದ ಬಳಲುವ ಜನರು ವೈದ್ಯರ ಸಲಹೆಯಂತೆ ಪಥ್ಯದ ಊಟಮಾಡುವರೋ ಹಾಗೆಯೇ ಇಂದ್ರಿಯನಿಗ್ರಹವನ್ನು ಅಪೇಕ್ಷಿಸುವ ಮನುಷ್ಯನು ನಿಷಿದ್ದಗಳನ್ನು ತ್ಯಜಿಸಿ ಸಾತ್ವಿಕ ಆಹಾರವನ್ನೇ ಸ್ವೀಕರಿಸಬೇಕು. ಸಾತ್ವಿಕಾಹಾರವು ಇಂದ್ರಿಯಗಳನ್ನು ಬಾಧಿಸುವುದಿಲ್ಲ.

\begin{verse}
\textbf{ಇಂದ್ರಿಯಾಣಿ ಪ್ರವರ್ತಂತೇ ವಿಷಯೇಷು ಸ್ವಭಾವತಃ~।}\\\textbf{ತೇಷಾಂ ಜಯೇ ಯತೇತಾದೌ ಪಶ್ಚಾದ್ಯೋಗಮವಾಪ್ನುಯಾತ್~।। ೪೭~।।}
\end{verse}

ಇಂದ್ರಿಯಗಳು ತಮ್ಮ ಸ್ವಭಾವದಿಂದಲೇ ವಿಷಯಸುಖಗಳಲ್ಲಿ ಪ್ರವರ್ತಿಸುತ್ತವೆ. ಆದುದರಿಂದ ಅವುಗಳನ್ನು ಜಯಿಸಲು ಮೊದಲು ಪ್ರಯತ್ನಿಸಬೇಕು. ನಂತರ ಜ್ಞಾನಯೋಗಾಭ್ಯಾಸವನ್ನು ಹೊಂದಬಹುದು.

\begin{flushleft}
\textbf{[ವಿಶೇಷಾಂಶ:}
\end{flushleft}

\begin{verse}
\textbf{ಯತತೋ ಹ್ಯಪಿ ಕೌಂತೇಯ ಪುರುಷಸ್ಯ ವಿಪಶ್ಚಿತಃ~।}\\\textbf{ಇಂದ್ರಿಯಾಣಿ ಪ್ರಮಾಥೀನಿ ಹರಂತಿ ಪ್ರಸಭಂ ಮನಃ~।।} \vauthor{ -ಗೀತಾ}
\end{verse}

ಐಹಿಕ ವಿಷಯಾನುಭವವು ಸರಿಯಲ್ಲವೆಂದು ತಿಳಿದಿರುವ ಜ್ಞಾನಿಗಳೂ ಇಂದ್ರಿಯಗಳನ್ನು ವಿಷಯಸುಖದ ಕಡೆಗೆ ಹೋಗದಂತೆ ಪ್ರಯತ್ನಪಟ್ಟರೂ ಸಹ, ದುಃಖಪಡಿಸುವ ಇಂದ್ರಿಯಗಳು ಬಲಾತ್ಕಾರವಾಗಿ ಮನಸ್ಸನ್ನು ವಿಷಯಗಳ ಕಡೆಗೆ ಸೆಳೆದುಕೊಂಡುಹೋಗುತ್ತವೆ.

\begin{verse}
\textbf{ತಾನಿ ಸರ್ವಾಣಿ ಸಂಯಮ್ಯ ಯುಕ್ತ ಆಸೀತ ಮತ್ಪರಃ~।}\\\textbf{ವಶೇ ಹಿ ಯಸ್ಯೇಂದ್ರಿಯಾಣಿ ತಸ್ಯ ಪ್ರಜ್ಞಾ ಪ್ರತಿಷ್ಠಿತಾ~।।}\vauthor{ -ಗೀತಾ}
\end{verse}

ಇಂದ್ರಿಯಗಳನ್ನು ನಿಗ್ರಹಿಸಿಕೊಂಡು ಕೃಷ್ಣನೇ ಸರ್ವೋತ್ತಮನೆಂಬ ಜ್ಞಾನ ಯುಕ್ತನಾಗಿ ಶ‍್ರೀಕೃಷ್ಣನಲ್ಲಿಯೇ ಮನಸ್ಸನ್ನಿಡಬೇಕು. ಯಾರ ಅಧೀನದಲ್ಲಿ ಇಂದ್ರಿಯಗಳು ಇರುತ್ತವೆಯೋ ಅವನಿಗೆ ಅಪರೋಕ್ಷಜ್ಞಾನವು ನಿಶ್ಚಿತ.]

\begin{verse}
\textbf{ಸಹಾಯಂ ಮನ ಏತೇಷಾಂ ದುರ್ಜನಾನಾಂ ಸ್ವಭಾವತಃ~।}\\\textbf{ತತ್ರಾಪಿ ದುರ್ಜಯಂ ಚಿತ್ತಂ ವಾಯೋರಿವ ಸುದುರ್ಗ್ರಹಮ್~।। ೪೮~।।}
\end{verse}

ಇಂದ್ರಿಯಗಳು ಸ್ವಭಾವದಿಂದ ಮನಸ್ಸಿನ ಹತೋಟಿಯಲ್ಲಿರುವುದರಿಂದ ಮನಸ್ಸನ್ನು ಸ್ವಾಧೀನದಲ್ಲಿ ಇಟ್ಟುಕೊಳ್ಳಬೇಕು. ಗಾಳಿಯನ್ನು ಹಿಡಿದು ಇಟ್ಟುಕೊಳ್ಳುವುದು ಹೇಗೆ ಅಸಾಧ್ಯವೋ ಹಾಗೆಯೇ ಮನಸ್ಸನ್ನು ಸ್ವಾಧೀನದಲ್ಲಿಟ್ಟು ಕೊಳ್ಳುವುದೂ ಸಹ ಪ್ರಯಾಸವೇ.

\begin{flushleft}
\textbf{[ವಿಶೇಷಾಂಶ:}
\end{flushleft}

\begin{verse}
\textbf{ಚಂಚಲಂ ಹಿ ಮನಃ ಕೃಷ್ಣ ಪ್ರಮಾಥಿ ಬಲವದ್ದೃಢಮ್~।}\\\textbf{ತಸ್ಯಾಹಂ ನಿಗ್ರಹಂ ಮನ್ಯೇ ವಾಯೋರಿವ ಸುದುಷ್ಕರಮ್~।।} \vauthor{ -ಗೀತಾ}
\end{verse}

ಅರ್ಜುನನು ಕೃಷ್ಣನಿಗೆ ಮನಸ್ಸು ಚಂಚಲ; ದೇಹೇಂದ್ರಿಯಗಳಿಗೆ ಕ್ಷೋಭೆಯನ್ನುಂಟುಮಾಡುವುದು, ಬಲಿಷ್ಠವಾದುದು. ಸಾಧು ವಿಷಯಗಳ ಕಡೆಗೆ ಹೋಗುವಂತೆ ಮಾಡಲು ಅಶಕ್ಯ; ಮನಸ್ಸಿನ ನಿಗ್ರಹಮಾಡಿಕೊಳ್ಳುವುದು ಗಾಳಿಯ ನಿಗ್ರಹಣದಂತೆ ಕಠಿಣವೆಂದು ನನ್ನ ಅಭಿಪ್ರಾಯ.

\begin{flushleft}
ಕೃಷ್ಣನು ಅರ್ಜುನನಿಗೆ ಹೇಳಿದ್ದು:
\end{flushleft}

\begin{verse}
\textbf{ಅಸಂಶಯಂ ಮಹಾಬಾಹೋ ಮನೋ ದುರ್ನಿಗ್ರಹಂ ಚಲಮ್~।}\\\textbf{ಅಭ್ಯಾಸೇನ ತು ಕೌಂತೇಯ ವೈರಾಗ್ಯೇಣ ಚ ಗೃಹ್ಯತೇ~।।} \vauthor{ -ಗೀತಾ}
\end{verse}

ಮನಸ್ಸನ್ನು ಸ್ವಾಧೀನಪಡಿಸಿಕೊಳ್ಳುವುದು ಬಹಳ ಕಷ್ಟವೇ, ಸಂಶಯವಿಲ್ಲ. ಆದರೂ ಕ್ರಮಕ್ರಮವಾಗಿ ಅಭ್ಯಾಸಮಾಡಿ, ವಿಷಯಗಳ ಕಡೆಗೆ ಸ್ನೇಹವನ್ನು ಬಿಡುವುದರಿಂದ ಮನಸ್ಸು ಸ್ವಾಧೀನವಾಗುತ್ತದೆ.]

\begin{verse}
\textbf{ಸತತಂ ಶ್ರವಣೇನೈವ ಮನೋ ವೈ ಜಯತಿ ಧ್ರುವಮ್~।}\\\textbf{ನಾನ್ಯೋಽಸ್ತ್ಯು ಪಾಯೋ ಮನಸೋ ವಿನಾ ಶ್ರವಣಮಾದರಾತ್~।। ೪೯~।।}
\end{verse}

ಪ್ರತಿನಿತ್ಯದಲ್ಲಿಯೂ ಶ‍್ರೀಹರಿಯ ಮಾಹಾತ್ಮ್ಯೆಯನ್ನು ಆದರದಿಂದ ಶ್ರವಣ ಮಾಡುವುದೊಂದೇ ಮನಸ್ಸನ್ನು ಜಯಿಸಲು ಉಪಾಯ. ಇದನ್ನು ಬಿಟ್ಟರೆ ಅನ್ಯೋಪಾಯವಿಲ್ಲ.

\textbf{[ವಿಶೇಷಾಂಶ\enginline{-}} ಯಜ್ಞ ಯಾಗಾದಿಗಳನ್ನು ಕೇವಲ ಒಂದು ಬಾರಿ ಆಚರಿಸುವುದರಿಂದಲೇ ಅವುಗಳ ಫಲಪ್ರಾಪ್ತಿ ಕಂಡಿದೆ. ಅದರಂತೆ ಅಪರೋಕ್ಷಜ್ಞಾನವನ್ನು ಹೊಂದಲು ಶ‍್ರೀಹರಿಯ ಮಾಹಾತ್ಮ್ಯೆಯನ್ನು ಒಂದೆರಡುಬಾರಿ ಮಾತ್ರ ಶ್ರವಣ ಮಾಡಿದರೆ ಸಾಲದು. ಪ್ರತಿ ನಿತ್ಯವೂ ಶ‍್ರೀಹರಿಯ ಮಾಹಾತ್ಮ್ಯೆಯನ್ನು ಶ್ರವಣ ಮಾಡಿದರೆ ಮಾತ್ರವೇ ಶ‍್ರೀಹರಿಯಲ್ಲಿ ನಿಶ್ಚಲವಾದ ಮಾಹಾತ್ಮ್ಯೆ ಜ್ಞಾನಪೂರ್ವಕವಾದ ಭಕ್ತಿಯು ಅಭಿವೃದ್ಧಿಯಾಗಲು ಸಾಧ್ಯವೇ ವಿನಹ ಕೇವಲ ಒಂದೆರಡು ಬಾರಿ ಶ್ರವಣ ಮಾಡುವುದರಿಂದ ಕೃತಕೃತ್ಯನಾಗುವುದಿಲ್ಲ. ಈ ವಿಚಾರವು ಬ್ರಹ್ಮಸೂತ್ರ ಭಾಷ್ಯದ “ಆಮೃತ್ತ್ಯಧಿಕರಣ”ದಲ್ಲಿ ವಿವರಿಸಲ್ಪಟ್ಟಿದೆ.

\begin{verse}
\textbf{ನಿತ್ಯಶಃ ಶ್ರವಣಂ ಚೈವ ಮನನಂ ಧ್ಯಾನಮೇವ ಚ~।}\\\textbf{ಕರ್ತವ್ಯಮೇವ ಪುರುಷೈಃ ಬ್ರಹ್ಮದರ್ಶನಮಿಚ್ಛುಭಿಃ~।।} \vauthor{ - ಬೃಹತಂತ್ರ}
\end{verse}

ಅಪರೋಕ್ಷಜ್ಞಾನಕ್ಕಾಗಿ ಪ್ರಯತ್ನಿಸುತ್ತಿರುವ ಪುರುಷನು ಪ್ರತಿನಿತ್ಯವೂ ಶ‍್ರೀಹರಿಯ ಮಾಹಾತ್ಮ್ಯೆಯನ್ನು ಶ್ರವಣ, ಮನನ ಮಾಡಬೇಕು-ಹಾಗೂ ನಿತ್ಯವೂ ಶ‍್ರೀಹರಿಯನ್ನು ಧ್ಯಾನಿಸಬೇಕು.

\newpage

\begin{verse}
\textbf{ಅಶ್ರೋತಾ ಪಶುವನ್ಮರ್ತ್ಯಃ ಸದಸನ್ನೈವ ವೇತ್ತಿ ಸಃ~।।} 
\end{verse}

\begin{verse}
\textbf{ವೃಥಾ ಜನ್ಮ ಭವೇತ್ತಸ್ಯ ಮಾತೃಯೌವನಹಾರಿಣಃ~।}\\\textbf{ಅಧರ್ಮಂ ನ ವಿಜಾನಾತಿ ಜ್ಞಾತ್ವಾಪಿ ನ ತ್ಯಜತ್ಯಮುಮ್~।। }
\end{verse}

\begin{verse}
\textbf{ಅಶೃಣ್ವನ್ ಪುರುಷಸ್ತಸ್ಮಾಚ್ಛ್ರವಣಂ ಸರ್ವದಾ ಚರೇತ್~।}\\\textbf{ಈದೃಶಂ ಶ್ರವಣಂ ಶ್ರೇಷ್ಠಂ ಸರ್ವೇಷಾಂ ಪರಮಂ ಹಿತಮ್~।। }
\end{verse}

\begin{verse}
\textbf{ಅತಃ ಸದ್ಭ್ಯೋಽನಿಶಂ ಕುರ್ಯಾತ್ ಶ್ರವಣಂ ತೇನ ಸದ್ಗತಿಃ~।}\\\textbf{ತದ್ದಿ ನಂ ದುರ್ದಿನಂ ವಿದ್ಧಿ ಮೇಘಚ್ಛನ್ನಂ ನ ದುರ್ದಿನಮ್~।। }\\\textbf{ಯದ್ದಿನಂ ಹರಿಸಂಕೀರ್ತಿಕಥಾಮೃತವಿವರ್ಜಿತಮ್~।।} \vauthor{ - ವಿಷ್ಣು ರಹಸ್ಯ}
\end{verse}

ಭಗವಂತನ ಮಹಿಮೆಗಳನ್ನು ಶ್ರವಣಮಾಡದೇ ಇರುವವನು ಪಶುವಿಗೆ ಸಮಾನ. ಅವನಿಗೆ ಕಾರ್ಯಾಕಾರ್ಯಗಳು, ಧರ್ಮಾಧರ್ಮಗಳು ತಿಳಿಯುವುದಿಲ್ಲ. ಅವನ ಜನ್ಮವು ವ್ಯರ್ಥ, ತನ್ನ ತಾಯಿಯ ಯೌವನವನ್ನು ಅಪಹರಿಸಿದಂತೆ ಆಗುತ್ತದೆ. ಅಂತಹವನು ಅಧರ್ಮವನ್ನು ತಿಳಿದರೂ ತ್ಯಜಿಸುವುದಿಲ್ಲ. ಆದುದರಿಂದ ಶ್ರೇಷ್ಠವಾದ ಶ‍್ರೀಹರಿಯ ಮಾಹಾತ್ಮ್ಯ ಶ್ರವಣಮಾಡುತ್ತಿರಬೇಕು. ಶ್ರವಣವು ಸರ್ವರಿಗೂ ಹಿತ, ಸಾಧುಸಜ್ಜನರ ಸಹವಾಸದಲ್ಲಿದ್ದು ನಿತ್ಯವೂ ಶ್ರವಣಮಾಡ ಬೇಕು. ಅದರಿಂದಲೇ ಸದ್ಗತಿ. ಯಾವ ದಿನ ಶ‍್ರೀಹರಿಯ ಕೀರ್ತನ ಶ್ರವಣಗಳಿಂದ ರಹಿತವಾಗಿರುವುದೋ ಆ ದಿನವೇ ದುರ್ದಿನ. ಮೇಘಗಳಿಂದ ಮುಚ್ಚಲ್ಪಟ್ಟ ದಿನವು ದುರ್ದಿನವಲ್ಲ.]

\begin{verse}
\textbf{ಇಂದ್ರಿಯಾಣಿ ಪುರಸ್ಕೃತ್ಯ ಧಾವಮಾನಂ ಗುಣೇಷು ತತ್~।}\\\textbf{ಶ್ರವಣಾತ್ ಜಾತವಿಜ್ಞಪ್ತ್ಯೌ ಮನೋ ವಾರಯಿತುಂ ಕ್ಷಮಃ~।। ೫೦~।।}
\end{verse}

ಮನಸ್ಸು ಇಂದ್ರಿಯಗಳನ್ನು ಮುಂದುಮಾಡಿಕೊಂಡು ವಿಷಯಸುಖಗಳ ಕಡೆಗೆ ಓಡುತ್ತದೆ. ಶ‍್ರೀಹರಿಯ ಮಹಾಮಾಹಾತ್ಮ್ಯೆಯನ್ನು ನಿರಂತರ ಶ್ರವಣಮಾಡುವುದರಿಂದ ಮನಸ್ಸನ್ನು ಸ್ವಾಧೀನದಲ್ಲಿ ಇಟ್ಟು ಕೊಳ್ಳಲು ಸಾಧ್ಯ.

\begin{verse}
\textbf{ಷಟ್ ಸಪತ್ನಾನ್ ಜಿಗೀಷುಃ ಸನ್ ಗೃಹಂ ಪೂರ್ವಂ ಚ ಸಂಶ್ರಯೇತ್~।}\\\textbf{ತಥಾ ದುರ್ಗಂ ಸಮಾಶ್ರಿತ್ಯ ದುರ್ಬಲೋ ಜಯತಿ ಹ್ಯರೀನ್~।। ೫೧~।।}
\end{verse}

ದುರ್ಬಲನಾಗಿರುವವನು ಶತ್ರುಗಳನ್ನು ಜಯಿಸುವ ಮೊದಲು ಕೋಟೆಯಿಂದ ಭದ್ರವಾದ ಸ್ಥಳವನ್ನು ಆಶ್ರಯಿಸುವಂತೆ, ಕಾಮಕ್ರೋಧಾದಿ ಅರಿಷಡ್ವರ್ಗವನ್ನು ಗೆಲ್ಲಲು ಮೊದಲು ಒಂದು ಬಲಿಷ್ಟವಾದ ಮನೆಯಲ್ಲಿ ಆಶ್ರಯ ಪಡೆಯಬೇಕು.

\begin{verse}
\textbf{ಕ್ರಮಾತ್ ಜಿಗೀಷುರ್ವಿಷಯಾನ್ ಶನೈರ್ಮೋಕ್ಷಾಯ ಕಲ್ಪತೇ~।}\\\textbf{ತೂರ್ಣಂ ವೈ ಜಯತಃ ಪುಂಸಃ ಸಮಾಕರ್ಷಂತಿ ವೈ ಬಲಾತ್~।। ೫೨~।।} 
\end{verse}

\begin{verse}
\textbf{ತಾನಿ ಜಿತ್ವಾ ಕ್ರಮೇಣೈವ ಯೋಗೀ ಯೋಗತ್ವಮಾಪ್ನುಯಾತ್~।।}
\end{verse}

ನಂತರ ಕ್ರಮವಾಗಿ ಇಂದ್ರಿಯಗಳನ್ನು ಗೆದ್ದು ಅವು ವಿಷಯಸುಖಗಳ ಕಡೆಗೆ ಹೋಗದಂತೆ ತಡೆಯಬೇಕು. ಇಂದ್ರಿಯಗಳಾದರೋ ಬಲಿಷ್ಟವಾಗಿ ನಿಗ್ರಹಿಸಲು ಕಷ್ಟವೇ. ಬಲಾತ್ಕಾರವಾಗಿ ಆ ವಿಷಯಸುಖಗಳಕಡೆಗೆ ಎಳೆಯುತ್ತವೆ. ಆದರೂ ಮೋಕ್ಷೇಚ್ಛುವು ಅವುಗಳನ್ನು ಕ್ರಮವಾಗಿ ಜಯಿಸಿ ಜ್ಞಾನೋಪಾಯದಿಂದ ಸಾಧನೆಯನ್ನು ಮುಂದುವರಿಸಿದರೆ ಫಲ ದೊರೆಯುತ್ತದೆ.

\begin{verse}
\textbf{ಕರ್ಮಣ್ಯಧಿಕೃತಾ ಯೇ ಚ ದೇವಾಃ ಸೂರ್ಯಾದಯೋಽಖಿಲಾಃ~।}\\\textbf{ವಿನಾ ವಿಷ್ಣುಂ ಸ್ವಾಂತರಸ್ಥಂ ಸ್ವಾಂಶ್ಚ ಪೂಜಯತೋ ಜನಾನ್~। }\\\textbf{ತಾನಧಃ ಪಾತಯತ್ಯದ್ದಾ ಸಂಸಾರಾಬ್ಧೌ ಪುನಃ ಪುನಃ~।। ೫೪~।।}
\end{verse}

ಸೂರ್ಯನೇ ಮುಂತಾದ ದೇವತೆಗಳಲ್ಲಿ ಅಂತಸ್ಥನಾಗಿರುವ ಶ‍್ರೀವಿಷ್ಣುವನ್ನು ತಿಳಿಯದೇ ಕೇವಲ ಆಯಾ ದೇವತೆಗಳೇ ನನಗೆ ಫಲಕೊಡುತ್ತಾರೆಂಬ ನಂಬಿಕೆಯಿಂದ ಆ ದೇವತೆ\-ಗಳನ್ನು ಮಾತ್ರವೇ ನಾನಾ ವಿಧವಾದ ಕರ್ಮಗಳಿಂದ ಪೂಜಿಸಿದರೆ, ಆ ದೇವತೆಗಳು ಅಂತಹ ಪೂಜಾದಿಗಳನ್ನು ಸ್ವೀಕರಿಸದೇ ಪೂಜಿಸಿದ ಜನರನ್ನು ಸಂಸಾರಸಾಗರದಲ್ಲಿ ಪುನಃ ಪುನಃ ಕೆಡವುತ್ತಾರೆ.

\begin{verse}
\textbf{ದೇವಾನನ್ಯಾನ್ ಪೂಜಯದ್ಭಿರಯಂ ಯದ್ಯಪಿ ಪೂಜಿತಃ~।}\\\textbf{ದೇವಾ ವಿಷ್ಣ್ವರ್ಪಣಂ ಕೃತ್ವಾ ಪೂಜಾಂ ಸ್ವೀಯೈರುಪಾರ್ಜಿತಾಮ್~।। ೫೫~।।}\\\textbf{ಕಿಂಚಿದ್ದತ್ವಾ ದಂಡಯತಿ ಸಂಸಾರಾಬ್ಧೌ ಪುನಃ ಪುನಃ~।}
\end{verse}

ಅನ್ಯದೇವತೆಗಳನ್ನು ಪೂಜಿಸುವ ಜನರು ವಿಷ್ಣುವನ್ನೂ ಸಹ ಸಾಮಾನ್ಯ ಜ್ಞಾನದಿಂದ ಪೂಜಿಸಿದರೆ ಅನ್ಯದೇವತೆಗಳು ತಮಗೆ ಸಲ್ಲಿಸಿದ ಪೂಜಾದಿಗಳನ್ನು ವಿಷ್ಣುವಿಗೆ ಅರ್ಪಣೆಮಾಡುತ್ತಾರೆ. ತಮ್ಮನ್ನು ಪೂಜಿಸಿದ ಜನರಿಗೆ ಸ್ವಲ್ಪ ಫಲವನ್ನು ಕೊಟ್ಟು ಪುನಃ ಪುನಃ ಸಂಸಾರದಲ್ಲಿ ಬೀಳಿಸಿ ಶಿಕ್ಷಿಸುತ್ತಾರೆ.

\begin{verse}
\textbf{ಸ್ವಾಮಿಭಕ್ತಾ ಯಥಾ ಚೋರಾನ್ ತದೀಯಂ ಪ್ರತಿಗೃಹ್ಯ ಚ~।। ೫೬~।।}\\\textbf{ನಿವೇದ್ಯ ಸ್ವಾಮಿನೇ ಸರ್ವಂ ದಂಡಯಂತಿ ಚ ಯೇ ಭುವಿ~।}
\end{verse}

ಸ್ವಾಮಿಭಕ್ತರಾದ ರಾಜಸೇವಕರು ಕಳ್ಳರನ್ನು ಹಿಡಿದು ಅವರು ಕದ್ದ ವಸ್ತುಗಳೆಲ್ಲವನ್ನೂ ರಾಜನಿಗೆ ಸಮರ್ಪಿಸಿ ಕಳ್ಳರನ್ನು ದಂಡಿಸುವಂತೆ ಅನ್ಯದೇವತೆಗಳು ತಮಗೆ ಸಮರ್ಪಿಸಿದ ಪೂಜೆಯನ್ನು ಶ‍್ರೀಹರಿಗೆ ಸಮರ್ಪಿಸಿ ಪೂಜಿಸಿದವರನ್ನು ದಂಡಿಸುತ್ತಾರೆ.

\begin{verse}
\textbf{ತಸ್ಮಾತ್ಕರ್ಮಣ್ಯಧಿಕೃತಾನ್ ಪೂಜಯಂತಿ ಸುರಾನ್ ದ್ವಿಜಾಃ~।। ೫೭~।।}\\\textbf{ಪೂಜಯಂತಿ ಶರೀರಸ್ಥಂ ಹರಿಂ ಸಿದ್ದಿಮವಾಪ್ನುಯುಃ~। }\\\textbf{ದೇವಾನ್ ಹಿತ್ವಾ ಯಜಂತೋsಪಿ ವಿಷ್ಣುಂ ದೇವೇಶಮವ್ಯಯಮ್~।। ೫೮~।।}
\end{verse}

ಆದುದರಿಂದ ಇತರ ದೇವತೆಗಳನ್ನು ಪೂಜಿಸುವ ದ್ವಿಜರು ಆಯಾ ದೇವತೆಗಳ ಅಂತ\-ರ್ಯಾಮಿಯಾದ ಶ‍್ರೀಹರಿಯನ್ನೇ ಪೂಜಿಸಿ ತಮ್ಮ ಇಷ್ಟಾರ್ಥಗಳನ್ನು ಪಡೆಯುತ್ತಾರೆ. ಯಜ್ಞಾದಿಗಳನ್ನು ಆಚರಿಸುವ ಜ್ಞಾನಿಗಳೂ ಇತರ ದೇವತೆಗಳನ್ನು ಉದ್ದೇಶಿಸಿ ಯಜ್ಞ ಮಾಡಿದರೂ ಸರ್ವೊತ್ತಮನಾದ ನಾಶರಹಿತನಾದ ವಿಷ್ಣುವನ್ನೇ ಪೂಜಿಸುತ್ತಾರೆ. "

\begin{verse}
\textbf{ನಾಧಿಗಚ್ಛಂತಿ ತೇ ವಿಷ್ಣೋಃ ಪ್ರಸಾದಂ ಮೋಕ್ಷಸಾಧನಮ್~।}\\\textbf{ಅಪೂಜಿತಾ ಅಧರ್ಮಾದಿದಾತಾರೋ ಯಜತಾಂ ಸುರಾಃ~।। ೫೯~।। }
\end{verse}

\begin{verse}
\textbf{ಯಥಾ ಪ್ರಜಾಃ ಪಾತಯಂತಿ ಮಾನವೈರಪುರಸ್ಕೃತಾಃ~।}\\\textbf{ಅವಜ್ಞಾತಾ ರಾಜಭೃತ್ತಾ ನೇಚ್ಛಂತಿ ಸ್ವಾಮಿಸೇವನಮ್~।। ೬೦~।।}
\end{verse}

ಹಾಗೆ, ವಿಷ್ಣುವನ್ನು ಪೂಜಿಸಿದರೂ ಮೋಕ್ಷಕ್ಕೆ ಅವಶ್ಯಕವಾದ ವಿಷ್ಣುವಿನ ಪ್ರಸಾದವು ದೊರೆಯುವುದಿಲ್ಲ. ಇತರ ದೇವತೆಗಳನ್ನು ಪೂಜಿಸದೇ ಕೇವಲ ವಿಷ್ಣುವನ್ನೇ ಸರ್ವೋತ್ತಮತ್ವ\-ಜ್ಞಾನದಿಂದ ಪೂಜಿಸಿದರೆ ಅಂತಹ ಜನರು ಧರ್ಮಶ್ರೇಷ್ಠರು. ರಾಜನನ್ನು ಅವಮಾನಗೊಳಿಸಿದ ಜನರನ್ನು ರಾಜಸೇವಕರು ಶಿಕ್ಷಿಸುತ್ತಾರೆ. ಅಜ್ಞಾನದಿಂದ ರಾಜನನ್ನು ಪುರಸ್ಕಾರಮಾಡಿದರೆ ರಾಜಸೇವಕರು ಸಹಿಸುವುದಿಲ್ಲ. ತಮ್ಮ ಸ್ವಾಮಿಯಾದ ರಾಜನನ್ನು ಸರಿಯಾದ ಕ್ರಮದಿಂದ, ಗೌರವದಿಂದ, ಪೂಜಿಸಿದರೆ ಮಾತ್ರವೇ ರಾಜಭೃತ್ಯರಿಗೆ ತೃಪ್ತಿ. ಆದರೆ ರಾಜಭೃತ್ಯರನ್ನೂ ಅವಮಾನಗೊಳಿಸದೆ ಯಥಾಯೋಗ್ಯವಾಗಿ ಅವರಿಗೂ ಮರ್ಯಾದೆ ತೋರಿಸಬೇಕು.

\begin{verse}
\textbf{ಹರಿದ್ವಿಷೋ ದೇವಭಕ್ತಾ ಹರಿಭಕ್ತಾಃ ಸುರದ್ವಿಷಃ~।}\\\textbf{ಉಭಯೇ ತೇ ವ್ರಜಂತ್ಯದ್ಧಾ ದಾರುಣಾನ್ ಯಾತನಾಲಯಾನ್~।। ೬೦~।।}
\end{verse}

ಶ‍್ರೀಹರಿಯನ್ನು ದ್ವೇಷಿಸಿ ಇತರ ದೇವತೆಗಳಲ್ಲಿ ಭಕ್ತಿಮಾಡುವವರು, ಶ‍್ರೀಹರಿಯಲ್ಲಿ ಭಕ್ತಿಮಾಡಿ ಇತರ ದೇವತೆಗಳನ್ನು ದ್ವೇಷಿಸುವವರು-ಈ ಎರಡು ವಿಧವಾದ ಜನರೂ ದಾರುಣವಾದ ದುಃಖಪ್ರದವಾದ ಪ್ರದೇಶಕ್ಕೆ (ನರಕಕ್ಕೆ) ಹೋಗುವರು.

(ಸರ್ವೊತ್ತಮತ್ವ ಜ್ಞಾನದಿಂದ ಮಾಹಾತ್ಮ್ಯ ಜ್ಞಾನಪುರಸ್ಸರವಾದ ಭಕ್ತಿಯಿಂದ ಶ‍್ರೀವಿಷ್ಣುವನ್ನು ಆರಾಧಿಸಬೇಕು. ಇತರ ದೇವತೆಗಳನ್ನು ಶ‍್ರೀವಿಷ್ಣುವಿನ ಪರಿವಾರದವರೆಂದು ತಿಳಿದು ಯಥಾಯೋಗ್ಯವಾಗಿ ಪೂಜಿಸಬೇಕು.)

\textbf{[ವಿಶೇಷಾಂಶ:} ಶ‍್ರೀವಿಷ್ಣುವನ್ನು ಸರ್ವೋತ್ತಮತ್ವಜ್ಞಾನದಿಂದಲೂ ಇತರ ಅಂಗಾಶ್ರಿತರಾದ ದೇವತೆಗಳನ್ನು ವೇದೋಕ್ತವಾದ ಅವರವರ ಗುಣಗಳನ್ನು ಉಪಾಸನೆಮಾಡಿ ಪೂಜಿಸಬೇಕು. ಈ ವಿಷಯವು ಬ್ರಹ್ಮಸೂತ್ರಭಾಷ್ಯದ "ಅಂಗಾವಬದ್ಧಾಧಿಕರಣ”ದಲ್ಲಿ ಸ್ಪಷ್ಟಪಡಿಸಲಾಗಿದೆ. ದೇವತೆ\-ಗಳ ತಾರತಮ್ಯವನ್ನು ಅರಿತು ಅದರಂತೆ ಉಪಾಸನೆ ಮಾಡಬೇಕು. ಅವರವರಲ್ಲಿ ಸ್ವೋತ್ತಮರ ಗುಣಗಳನ್ನು ಉಪಾಸನೆ ಮಾಡಬಾರದು. ಶ‍್ರೀಮದಾಚಾರ್ಯರು ಈ ವಿಷಯ\-ದಲ್ಲಿ ಬ್ರಹ್ಮತರ್ಕವಚನ, ಬೃಹತ್ತಂತ್ರ ವಚನಗಳನ್ನು ಉದಹರಿಸಿರುವರು.

\begin{verse}
\textbf{ಸರ್ವವರ್ಣಾಶ್ರಮೈಃ ವಿಷ್ಣುಃ ಏಕ ಏವೇಜ್ಯತೇ ಸದಾ~।}\\\textbf{ರಮಾಬ್ರಹ್ಮಾದಯಸ್ತತ್ರ ಪರಿವಾರತಯೈವ ತು~।।} \vauthor{ - ಸ್ಮೃತಿವಾಕ್ಯ}
\end{verse}

ಎಲ್ಲ ವರ್ಣ, ಎಲ್ಲ ಆಶ್ರಮದವರು ಸರ್ವೊತ್ತಮನಾದ ವಿಷ್ಣುವನ್ನು ನಿತ್ಯದಲ್ಲಿಯೂ ಯಜ್ಞಯಾಗಾದಿಗಳಿಂದ ಪೂಜಿಸಬೇಕು. ಆದರೆ ರಮಾ, ಬ್ರಹ್ಮ, ವಾಯು, ರುದ್ರ ಮುಂತಾದ ದೇವತೆ\-ಗಳನ್ನು ಶ‍್ರೀಹರಿಯ ಪರಿವಾರ ದೇವತೆಗಳೆಂದು ತಿಳಿದು ಯಥಾಯೋಗ್ಯವಾಗಿ ಅವರವರ ತಾರತಮ್ಯ ಪ್ರಕಾರ ಪೂಜಿಸಬೇಕು.

\begin{flushleft}
ವಿಷ್ಣು ರಹಸ್ಯದಲ್ಲಿ ಈ ರೀತಿ ಹೇಳಿದೆ: 
\end{flushleft}

\begin{verse}
\textbf{ತಸ್ಮಾ ದ್ದೇವಪದಾರೂಢಾಃ ನಿಶ್ಚಯಾದಪರೋಕ್ಷಿಣಃ~।}\\\textbf{ಸ್ವಾರಾಧಕಾನಾಮಿಷ್ಟಾನಿ ಪೂರಯಂತಿ ಮಮಾಜ್ಞಯಾ~।। }
\end{verse}

\begin{verse}
\textbf{ತದ್ದೇವಾನ್ನಾವಜಾನೀಯಾನ್ಮಭಕ್ತಸ್ತಾನ್ ಕಥಂಚನ~।}\\\textbf{ಸೋಪಾನಭೂತಾಃ ಮತ್ಪ್ರಾಪ್ತೌ ಮತ್ಪರೀವಾರತೋಽಖಿಲಾಃ~।।} 
\end{verse}

\begin{verse}
\textbf{ಅಂತರಾಯೋ ಹಿ ಮದ್ಭಕ್ತೌ ಭವೇದುತ್ತಮಕೋಪತಃ~।}\\\textbf{ತಸ್ಮಾದ್ಯಥಾಗುಣಂ ಸೇವ್ಯಾಃ ರಮಾಬ್ರಹ್ಮಾದಯೋಽಖಿಲಾಃ~।। }\\\textbf{ತೇನ ನಿರ್ವಿಘ್ನಮಾಪ್ನೋತಿ ಮತ್ಪದಂ ನೇದಮನ್ಯಥಾ~।।}
\end{verse}

ದೇವತೆಗಳೆಲ್ಲರೂ ಭಗವಂತನ ಅಪರೋಕ್ಷಜ್ಞಾನಿಗಳೇ. ಅವರು ನನ್ನನ್ನು ಸೇವಿಸುವ ಎಲ್ಲ ಭಕ್ತರ ಅಭಿಷ್ಠಗಳನ್ನು ನನ್ನಾಜ್ಞೆಯಿಂದಲೇ ನೆರವೇರಿಸುತ್ತಾರೆ. ನನ್ನ ಭಕ್ತರಾದ ದೇವತೆ\-ಗಳನ್ನೂ ಯಾವ ವಿಧದಿಂದಲೂ ತಿರಸ್ಕಾರಭಾವನೆಯಿಂದ ನೋಡಬಾರದು. ಇವರೆಲ್ಲರೂ ನನ್ನ ಪರಿವಾರದವರಾಗಿ ಭಕ್ತರು ನನ್ನ ಸ್ಥಾನಕ್ಕೆ ಬರಲು ಮೆಟ್ಟಲಿನಂತೆ ಇದ್ದಾರೆ. ಸ್ವೋತ್ತಮರು ಕುಪಿತರಾದರೆ ಭಕ್ತನ ಭಕ್ತಿ ಪ್ರವಾಹಕ್ಕೆ ಅಡ್ಡಿಯಾಗುತ್ತದೆ. ಭಕ್ತನಿಗೆ ನನ್ನಲ್ಲಿ ಸಂತತವಾಗಿ ಭಕ್ತಿ ಪ್ರವಾಹ ಹರಿಯಲು ನನ್ನ ಪರಿವಾರದೇವತೆಗಳ ಅನುಗ್ರಹವೂ ಅವಶ್ಯಕ. ಆದುದರಿಂದ ರಮಾಬ್ರಹ್ಮಾದಿ ಸಕಲದೇವತೆಗಳೂ ಅವರ ಗುಣತಾರತಮ್ಯಕ್ಕೆ ಅನುಸಾರವಾಗಿ ಮುಮುಕ್ಷುಗಳಿಂದ ಸೇವ್ಯರೇ. ಆ ಎಲ್ಲರ ಅನುಗ್ರಹಪೂರ್ವಕವಾದ ನನ್ನ ಪ್ರಸಾದದಿಂದಲೇ ನಿರ್ವಿಘ್ನವಾಗಿ ಮೋಕ್ಷವು ಲಭಿಸುತ್ತದೆ. ಇದು ಬಿಟ್ಟರೆ ಮೋಕ್ಷಕ್ಕೆ ಬೇರೆ ಉಪಾಯವಿಲ್ಲ.]

\begin{verse}
\textbf{ಯಥಾ ಶ್ರುತಿಸ್ತಥಾಚಾರಃ ಕರ್ತವ್ಯೋ ಹಿತಮಿಚ್ಛತಾ~।}\\\textbf{ತಸ್ಮಾದ್ವಿಷ್ಣ್ವನುಗಾನ್ ದೇವಾನ್ ವಿಷ್ಣುಂ ಚೈವ ಯಥಾಕ್ರಮಮ್~।।}
\end{verse}

ಹಿತವನ್ನು ಬಯಸುವವನು ಶ್ರುತಿಗಳಲ್ಲಿ ಹೇಳಿರುವ ಕ್ರಮದಿಂದ ವಿಷ್ಣುವನ್ನೂ, ವಿಷ್ಣುವಿನ ಭಕ್ತರಾದ ಇತರ ದೇವತೆಗಳನ್ನೂ ತಾರತಮ್ಯ ಜ್ಞಾನದಿಂದ ಯುಕ್ತನಾಗಿ ಯಥಾಯೋಗ್ಯವಾಗಿ ಪೂಜಿಸಬೇಕು.

\begin{verse}
\textbf{ಪೂಜಯನ್ ಕರ್ಮಣಾ ಸ್ವೇನ ನಿಯಮೇನ ಚ ವೈ ನರಃ~।}\\\textbf{ನಿಜಾಂ ಗತಿಮವಾಪ್ನೋತಿ ನಾತ್ರ ಕಾರ್ಯಾ ವಿಚಾರಣಾ~।। ೬೩~।।}
\end{verse}

ಯಾರು ತನ್ನ ವರ್ಣಾಶ್ರಮಕ್ಕೆ ಉಚಿತವಾದ ನಿಯಮದಿಂದ ಈರೀತಿ ವಿಷ್ಣುವನ್ನೂ ಇತರ ದೇವತೆಗಳನ್ನೂ ಪೂಜಿಸಿ, ತನ್ನ ಕರ್ಮವನ್ನು ಆಚರಿಸುವನೋ, ಅಂತಹವನು ಮೋಕ್ಷಕ್ಕೆ ಸಾಧನವಾದ ಜ್ಞಾನವನ್ನು ಪಡೆಯುತ್ತಾನೆ, ಇದರಲ್ಲಿ ಸಂಶಯವಿಲ್ಲ,

\begin{verse}
\textbf{ಏವಂ ವಿಜ್ಞಾಯ ಯೋಗೇನ ಯತ್ಕೃತಂ ಕರ್ಮ ವಿದ್ಯತೇ~।}\\\textbf{ತದೇವ ಜ್ಞಾನಮುತ್ಪಾದ್ಯ ಕ್ರಮಾನ್ಮೋಕ್ಷಾಯ ಕಲ್ಪತೇ~।। ೬೪~।।}
\end{verse}

ಈ ರಹಸ್ಯವನ್ನರಿತು ಜ್ಞಾನೋಪಾಯದಿಂದ ಸತ್ಕರ್ಮವನ್ನು ಆಚರಿಸಿ ವಿಷ್ಣುವಿನಲ್ಲಿ ಸಮರ್ಪಣೆ ಮಾಡುವುದರಿಂದ ಒಳ್ಳೆಯ ಯಥಾರ್ಥಜ್ಞಾನವು ಹುಟ್ಟಿ ಮೋಕ್ಷಕ್ಕೆ ಕಾರಣವಾಗುತ್ತದೆ.

\begin{center}
ಇತಿ ಶ‍್ರೀ ವಾಯುಪುರಾಣೇ ಮಾಘಮಾಸಮಾಹಾತ್ಮ್ಯೇ ಷೋಡಶೋsಧ್ಯಾಯಃ
\end{center}

\begin{center}
ಶ‍್ರೀ ವಾಯುಪುರಾಣಾಂತರ್ಗತ ಮಾಘಮಾಸಮಾಹಾತ್ಮ್ಯೆಯಲ್ಲಿ \\ ಹದಿನಾರನೇ ಅಧ್ಯಾಯವು ಸಮಾಪ್ತಿಯಾಯಿತು.
\end{center}

\delimiter

