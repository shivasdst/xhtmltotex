
\chapter*{ಅನುವಾದಕನ ಬಿನ್ನಹ}

 ಶ‍್ರೀ ವಾಯುಪುರಾಣಾಂತರ್ಗತವಾದ “ ಮಾಘಮಾಸ ಮಾಹಾತ್ಮ”ಯು ಅನೇಕ ಪ್ರಮೇಯಗಳಿಂದ ತುಂಬಿರುವ ಗ್ರಂಥ, ವಿಷ್ಣು ಸರ್ವೋತ್ತಮತ್ವ, ವಾಯುಜೀವೋತ್ತಮತ್ವ, ಆನಂದ ತಾರತಮ್ಯ, ಮುಂತಾದ ಅನೇಕ ಪ್ರಮೇಯಗಳನ್ನು ಈ ಗ್ರಂಥದಲ್ಲಿ ಕಾಣಬಹುದು. ಇದರೊಡನೆ ಮಾಘಮಾಸದಲ್ಲಿ ಸಾಧಕನು ಕೈಗೊಳ್ಳಬೇಕಾದ ಸತ್ಕರ್ಮಗಳು ಯಾವವು, ಅವುಗಳನ್ನು ಯಾವ ರೀತಿಯಲ್ಲಿ ಆಚರಿಸಬೇಕು, ಅವುಗಳಿಂದ ಬರುವ ಫಲ ಮುಂತಾದ ವಿಷಯಗಳು ಈ ಗ್ರಂಥದಲ್ಲಿ ನಿರೂಪಿತವಾಗಿವೆ. ದೃಷ್ಟಾಂತಗಳೂ, ಅನೇಕ ಕಥೆಗಳೂ ಸೇರಿ ಈ ಗ್ರಂಥವು ಕುತೂಹಲಜನಕವಾಗಿದೆ.

1980 ರಲ್ಲಿ ಮದರಾಸಿನ “ಶ‍್ರೀ ಮಧ್ವ ಸಮಾಜ”ದಲ್ಲಿ ಈ ಗ್ರಂಥದ ಪ್ರವಚನ ಮಾಡಿದೆ. ಮಂಗಳ ಮಹೋತ್ಸವವನ್ನು ಮಾಘಮಾಸದಲ್ಲಿಯೇ ತಿರುಕೊಯಲೂರಿನಲ್ಲಿ ಶ‍್ರೀ ಶ‍್ರೀ ರಘೋತ್ತಮರ ದಿವ್ಯ ಸನ್ನಿಧಿಯಲ್ಲಿ ಆಚರಿಸಲಾಯಿತು.

ಇಂತಹ ತತ್ವ ಬೋಧಕ ಗ್ರಂಥವನ್ನು ಕನ್ನಡಕ್ಕೆ ನನ್ನ ಯೋಗ್ಯತೆಗೆ ಅನುಗುಣ ವಾಗಿ ಅನುವಾದಿಸಿ ಆಸ್ತಿಕಜನರ ಮುಂದೆ ಇಡುತ್ತಿದ್ದೇನೆ. ಇದರಲ್ಲಿ ಅನೇಕ ದೋಷಗಳಿರಬಹುದು. ಸಜ್ಜನರು ಅಂತಹ ದೋಷಗಳನ್ನು ಕ್ಷಮಿಸಿ, ಒಳ್ಳೆಯ ಅಂಶಗಳನ್ನು ಗ್ರಹಿಸಬೇಕೆಂದು ಪ್ರಾರ್ಥಿಸುತ್ತೇನೆ. ಪ್ರಮೇಯ ಭಾಗಗಳ ವಿವರಣೆಯನ್ನು “ವಿಶೇಷಾಂಶ” ಎಂಬ ಶಿರೋನಾಮೆಯಲ್ಲಿ ಮಾಡಿ ಅನೇಕ ಪ್ರಮಾಣಗಳನ್ನು ನನ್ನ ಯೋಗ್ಯತೆಗೆ ತಕ್ಕಂತೆ ಉದಹರಿಸಿದ್ದೇನೆ.

ಈ ಗ್ರಂಥವನ್ನು ಜ್ಞಾನ-ಭಕ್ತಿ-ವೈರಾಗ್ಯಾದಿಗಳಿಂದ ಪೂರ್ಣರಾದ, ಸುಧಾದಿ ಗ್ರಂಥಗಳ ಪಾಠಪ್ರವಚನಗಳನ್ನು ನಲವತ್ತೈದು ವರ್ಷಗಳಿಂದಲೂ ನಿರಂತರವಾಗಿ ಮಾಡುತ್ತಿರುವ, ಸರವಾದಿಗಳಿಗೆ ಸಿಂಹದಂತೆ ಇರುವ, ಪರಮಪೂಜ್ಯರಾದ ಶ‍್ರೀ ಶ‍್ರೀ ಸತ್ಯಪ್ರಮೋದತೀರ್ಥ ಶ‍್ರೀಪಾದಂಗಳವರಲ್ಲಿ ಭಕ್ತಿಯಿಂದ ಅರ್ಪಿಸಿದ್ದೇನೆ.

ಈ ಗ್ರಂಥವನ್ನು ಪ್ರಕಟನೆ ಮಾಡುತ್ತಿರುವ ಶ‍್ರೀ ಮಧ್ವ ಸಮಾಜಕ್ಕೆ ನನ್ನ ಕೃತಜ್ಞತೆಗಳನ್ನು ಸಲ್ಲಿಸುತ್ತೇನೆ.

ಈ ಗ್ರಂಥದ ಎರಡನೇ ಭಾಗವು ಇನ್ನು ಕೆಲವು ತಿಂಗಳುಗಳಲ್ಲಿಯೇ ಹೊರ ಬರುತ್ತದೆ.

\begin{flushright}
ಸಜ್ಜನ ಸೇವಕ \\\textbf{ ಏಕೀ ಸುಬ್ಬಣ್ಣಾಚಾರ್ಯ}
\end{flushright}

\begin{flushleft}
ಮದರಾಸು-33 \\ 10-9-1993
\end{flushleft}

\begin{center}
श्रीः
\end{center}

ಜಗದ್ಗುರು ಶ‍್ರೀ ಮಧ್ವಾಚಾರ್ಯರ ಮೂಲ ಸಂಸ್ಥಾನ ಶ‍್ರೀಮದುತ್ತರಾದಿ ಮಠದ ಪರಂಪರೆಯಲ್ಲಿ ಈಗ ಪೀಠಾಧಿಪತಿಗಳಾಗಿ ವಿರಾಜಮಾನರಾಗಿರುವ, ದುರ್ವಾದಿಗಳಿಗೆ ಸಿಂಹಪ್ರಾಯರಾದ, ಜ್ಞಾನ-ಭಕ್ತಿ-ವೈರಾಗ್ಯನಿಧಿಗಳಾಗಿರುವ, ನಿತ್ಯದಲ್ಲಿಯ ಸುಧಾದಿ ಸಚ್ಛಾಸ್ತ್ರ ಪಾಠ, ಪ್ರವಚನಗಳಲ್ಲಿ ನಿರತರಾಗಿರುವ

\begin{center}
\textbf{ಶ‍್ರೀ ೧೦೦೮ ಶ‍್ರೀ ಸತ್ಯಪ್ರಮೋದತೀರ್ಥ ಶ‍್ರೀಪಾದಂಗಳವರ} \\ ದಿವ್ಯ ಸನ್ನಿಧಿಯಲ್ಲಿ \\ ಅರ್ಪಿತ 
\end{center}

\begin{verse}
ಜ್ಞಾನಭಕ್ತ್ಯಾದಿವೈರಾಗ್ಯಸದ್ದುಣಾಬ್ಧಿಂ ಯತೀಶ್ವರಮ್ |\\ ಶ‍್ರೀ ಸತ್ಯಪ್ರಮೋದತೀರ್ಥಾಖ್ಯಂ ವಂದೇ ಜ್ಞಾನಾಭಿವೃದ್ಧಯೇ ||
\end{verse}

\chapter*{PUBLISHER'S NOTE}

The Madhwa Samaj at Mambalam started over twenty-six years ago with the object of study, dissemination of knowledge of, propagation of Dwaita Siddhānta and tenets and precepts of Madhva cult among the general public and particularly among Mādhvas, has in its own humble way continued and continues to serve the ardent devotees of Mādhva faith. The Samaj has been conducting classes for study of works of Sri Madhva and his followers like Sri Jayathirtha, Sri Vādirāja, Sri Raghavendraswamy and others including Haridasas and has covered to some extent works on Prasthānatrayas, Harikathāmộtasāra etc. The Samaj also conducts Ārādhana Mahotsavas, celebrates Madhwa Navami, Dhātrihavana and such other religious functions. Inviting and paying respects to Peethadhipathis, Swamijis and invoking their blessings is another activity of the Samaj. Its main activity is conducting regular classes for the study of Madhwa Philosophy and it is a matter of some satisfaction that by the grace of Sri Hari and Sri Vāyu this has been going on uninterruptedly for the last twenty-six years i.e., since the inception of the Samaj. Special works like "Vishnurahasya', 'Sattattvaratnamāla' have also been covered in the studies.

Book publication on topics and works giving an elucidation of the deeper meanings therein, in simple Kannada, has also been one of the activities of the Samaj. Nārāyana Sabdārtha', 'Sriman Mahābhāratha Tātparya Nirņaya Bhāvasangraha', 'Sri Raghavendra Stotra' (Kannada meanings in Kannada script and Tamil script) Sri Rama Chāritra Manjari' are some of the publications by the Samaj.

The present book is titled 'Māghamāsa Māhātme' (in Kannada) by Sri A. R. Subbannachar-the patron of the Samaj and mainstay of all the activities of the Samaj (intellectual, spiritual, religious, study of dissemination of sāstric knowledge, publication etc.)

Māghamāsa Māhātme' is from Vāyu Purāņa-one of the Purāṇas elucidating Sāndilya Tattvas and Harimahātme and Hari Sarvottamatva. It is hoped that this publication will give the readers the knowledge of the greatness of Lord Vishnu and his Glory and Majesty and inculcate the spirit of deep devotion in Him.

This publication is rendered possible due to the munificent grant of financial assistance by 'Tirumala Tirupathi Devastanams' and we offer our deep-felt thanks to them.

Our thanks are due to Sri D. S. Krishnachar of Prabha Printing House, Bangalore-4, in getting this printed neatly unmindful of the difficulties.

Our Special thanks to Sri A. R. Subbannachar, the author of this work, for giving us the opportunity of publishing the work and helping us fulfil one of the objectives of our Samaj.

\begin{center}
\textbf{"Sri Krishnarpanamastu”}
\end{center}

\begin{flushright}
\textbf{R. Srinivasa Rao \textit{President, Madhva Samaj}}
\end{flushright}

\begin{flushleft}
13, Kuppaiah Street \\ West Mambalam \\ Madras-600033
\end{flushleft}

