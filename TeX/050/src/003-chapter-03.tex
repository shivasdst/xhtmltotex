
\chapter{Hindu-Buddhist Framework: Detonator of Western Indology}\label{chapter3}

\setcounter{endnote}{0}

\Authorline{Ravi Joshi\endnote{ In ***}}

\lhead[{\small\thepage}\quad\small Ravi Joshi]{}

\section*{Abstract}

Western Indology is prone to looking at India from its own lenses, in spite of voluminous and well-structured material available on India's own self-perception over the ages. This is especially true when they deal with India's "Religions". Having developed a sophisticated globally applicable framework with the category of "Religion" as fulcrum, they rarely pause to question the appropriateness of their criteria, preferring to impose and extend an inadequate framework, adding untenable riders and extra assumptions, just to continue using "Religion" as a lynch-pin of their analysis of societies. This framework is so hegemonic that its lack of coherence for vast non-Abrahamic societies has become all but invisible.

This framework and its derived methodology have been profitably utilized by Indologists in contrasting the categorization of Hinduism with that of Buddhism. Though there is much contestation on what the term "Religion" means when used for non-Abrahamic societies, its use is still dominant.

\newpage

Prof. Sheldon Pollock, eminent Sanskritist and Indologist, seems to have taken the existing tropes of "Protestant egalitarian secular Buddhism" vs. "Theistic Hierarchical Hinduism" to new depths. This paper specifically investigates the validity of his assertions on "spiritualistic Evangelistic Buddhism" vs. "Ritualistic escapist Hinduism", and what follows from them in his claims on the ills of Indian society being products of Hinduism. In this paper, key aspects of the terminology are unpacked, and the internal \textit{emic} Dharmic categories are introduced and used to refute the artificial "Hinduism vs. Buddhism" divide upon which this scholarship rests. Pollock's relevant works are cited. Full use is made of existing Western academic grade scholarship on Hindu-Buddhist philosophy, as well as \textit{emic}sources such as \textit{śāstra-}s, \textit{darśana-}s, \textit{sutra-}s etc. that explicate the Vedic-Hindu and \textit{Bauddha} Dharma categories in an integral framework. What may appear as "deep foundational difference" via an \textit{etic} reading is seen as integral and organic systems spawned from the same meta framework. Some comments are also made regarding the distortion of timelines of India's history - required in order to make Pollock's claims of Hindu "reaction" to Buddhist "innovation" tenable.


\section*{Introduction}

“Religion” has been a category used for describing certain aspects of a society no matter where that society was located in time and space. Though it looks like a compelling and obvious choice, on deeper inspection, this category loses coherence in its ability to describe the lived-in reality of most societies outside of those dominated by the Abrahamic constructs of Judaism, Christianity and Islam. As a category native to the West, it went global following the last few centuries of Western dominance via colonization of an overwhelming majority of societies of the world (Staal 1996: Ch28; Balagangadhara 1994).

This situation is especially true of India, where after British colonization for two centuries, the word ‘religion’ has almost ‘gone native’, especially for the English educated elite and followers. India is no isolated case though, as (Staal 1996) shows. European colonialism has remade at least the thought world of the rest of the globe – even non-colonized Japan - in this respect by universalizing the category ‘Religion’ to explain all activity that does not fit into the easily comprehensible economic, sociopolitical, and other empirically available domains. Needless to say, this is a manifestation of the European Enlightenment based Secular/Religious divide wrought onto European society, which has a valid, if peculiarly European, genesis. Axiomatic imposition of this divide on Indian social reality is highly questionable, and has often been contested.

This brings us to the key issues for discussion here. It has become conventional wisdom to assume that in India there have existed, and still exist, many religions. Hinduism and Buddhism are cited as two of the big religions of India, and are also considered to be two of the five major global religions, the other three being the Abrahamic ‘trinity’ of Judaism, Christianity and Islam. The possibility of categorizing based on the available meta framework of India’s Dharmic systems has been studiously ignored until recently. In his book \textit{Being Different} (Malhotra 2011: 41 refs 38, 39) he refers also to works of McKim Marriot and A. K. Ramanujan who did this earlier, but in a much more limited and less focused sense. While it is understandable why Western scholars and followers would be unwilling to let go of a framework in which they have been so heavily invested, it just needs to be pointed out that India as a civilization did quite well without the “Religion” construct until the advent of large scale Abrahamic influences starting with the Islamic conquests. Buddhism was then known merely as “Bauddha \textit{saṃpradāya}” one among many Dharmic \textit{saṃpradāya}-s (i.e. paths / sects) of India. The meta-category of Dharma – not to be wrongly equated with “Religion” - had and still has deep roots, with subcategories enough to encompass virtually all indigenous systems.

When looked at as two separate entities i.e. religions, scholars find it easy to show up contrasts, and elaborate on their differences, with impressive confidence in the validity of their categorization. Here, in summary form, is the Western academic consensus on these two religions. Hinduism is considered ahistorical, theistic, talks about ‘being’, and of ultimate reality as permanent; whereas Buddhism is considered historical, atheistic, talks about ‘becoming’, and says that ultimate reality is ‘nothingness’. This much itself seems to show a radical difference in orientation, perhaps unbridgeable. Not only this, we are told, it seems but obvious that there are strong parallels between how in the West Protestant Christianity grew out of and attempted to reform Catholicism – away from outmoded ritual and priestly hierarchy, and similarly in India how Buddhism grew out of, and attempted to reform, Hinduism – away from outmoded ritual and priestly hierarchy. (Smith 1991: 92-94 “The Rebel Saint”; Upadhyay 1970:pp)

Another interesting factor is the historical evolutionary timelines given to Hinduism and Buddhism. Current historicizing trends are built on shaky foundations of Western impositions of linear historical sensibilities on the timelines of Indian history. The emic insider mode of recording India’s past via \textit{itihāsa-}s and \textit{purāṇa-}s is considered too alien to be of value, and mostly used selectively as fragmentary pieces of data where it helps to buttress the pre-derived timelines of the Indologists. Inconvenient data is thrown out or ignored until it becomes too compelling, for example the current excavations that are pushing back the timelines of the Indus Valley civilization complex by a few more thousand years. More assorted genetic and other challenges are now forthcoming to the fiat currency of the Aryan Invasion Theory, and its weaker variants.

What directly concerns us here though, are the controversial moves to completely invert the traditional timelines so as to establish Buddhism as the ‘source’ of all the innovations in Hindu religion and philosophy. Specifically, if the \textit{Rāmāyaṇa} (traditionally from the \textit{Tretāyuga}) can be shown as being after the \textit{Mahābhārata} (traditionally from the\textit{Dvāparayuga} that follows the \textit{Tretā}) then it becomes easy to locate the Buddha as prior to both \textit{Rāmāyaṇa} and the \textit{Bhagavad Gītā}. A few more moves and the field is set for “a reactionary Hinduism” – led by its wily Brahmin elite - that consistently seems to subsume ‘foreign’ ideas like Buddhist thought and appropriate them.

The key point to remember in all this is that these timelines - at their axiomatic base - still have not unshackled themselves from Colonial Indology of Max Mueller’s time. His and other colonial Indologists’ wide ranging and determinedly post-Biblical-flood estimates compressed a few thousand years of pre Christian era Indian history into a few hundred years, full of highly linearized interpolations. (Arya 2016:pp) for example shows the deliberate confusion engendered by Colonial Indologists regarding the date of the beginning of the \textit{Śaka} era, which pervasively influences most dates regarding Hindu history upto the time of the Buddha and later. Even if the Indology timeline is the best one can do with available data, it is a far cry from being acceptably precise for one to start simply overturning the traditional view, and establishing a new priority of one event over another. All the speculative reading is passed off as research, but the question remains: to what end?

More historicizing often attempts to show that Buddhism was hounded out of India by ‘rival’ Hinduism’s zealous advocates, although this is quite untenable due to the self-attested history of the 10th century onward Islamic invaders – Bakhtiar Khilji onwards, without exception - which unabashedly celebrates their accomplishment in this regard. Their own historical record shows their relentless destruction of ‘idolatrous temples’ that, unbeknownst to them, also meant wholesale extinction of Buddhist \textit{vihāra}-s (havens of “\textit{but parastī”} i.e. idol-worship, “\textit{but”} also being a corruption of the word \textit{Buddha}) with their ‘clean shaven Brahmins’. Indologists of late seem to have a hard time acknowledging that this was what actually led to the vanishing of institutionally dependent Buddhism from India.

As Western Indology, under the lead given by prestigious scholars like Sheldon Pollock, proceeds further into its research program on India, there is more that is being built upon this current edifice of India’s Religions. Now the next layer of academic knowledge has begun to whittle away at the perceptions of the positives of Hinduism. It is asserted that it is not really the inherently well-grounded and dynamic nature of Hinduism, but its appropriations from and reactions to Buddhism that have given the world today’s sophisticated philosophies and practices like \textit{Vedānta}, \textit{Yoga} etc. Per this theorizing, “the wily elite” of the times of the Buddha and after, incorporated basically ‘foreign’ ideas into their own inferior and backward looking mystifying Vedic philosophies since Buddhism introduced “Axial” (Pollock 2004, Bellah 2012) ideas into Indian civilization. (Malhotra 2014) has also comprehensively shown that this same mode of theorizing is in vogue to establish and then deride the academically created entity “neo-Hinduism” –which incidentally delegitimizes modern Hinduism - and its supposed founders, the ‘Nationalist elite’ of colonial era India. Here the supposed source of their borrowing is the European Enlightenment that the British colonizers carried into India, and the Indian elite were interested in appropriating. So it appears Hinduism – neo or otherwise - is something to be reliably invoked when a straw man is required to show how it is the wily elites of India who illegitimately borrow from others.

\newpage

After the positives of Hinduism are thus analyzed away, it is the turn of the perceived negatives to be brought in. This would include the blaming of current political problems of India on some inherent Hindu essence, or some variant of the persistent ‘wily Brahmin keeping power to himself’. (Upadhyay 1979) is a good example of this, having completely internalized the Indological framework in his sweeping historicization of Brahmins from Vedic times to the twentieth century. To top it off, we now have Pollock’s students, Indologists like Audrey Truschke, supported by her students, using this genre of selective textual ‘research’ to even rehabilitate proven Islamic bigoted Mughal emperors like Aurangzeb into some benign presence beneficial to Sanskrit, irrespective of how cruelly he oppressed its native practitioners, and regardless of how gleefully his official historians –and the mute evidence of archaeology - recorded fact after fact emphasizing the same.


\section*{Western Frameworks: Buddhism vs. Hinduism}

All current academic frameworks are Western defaults, entirely dependent on how the West has historically constructed knowledge of India. One key item to remember (Elst 2013, Weber 2001) is that the West initially got exposed to Buddhism not from its original cradle in India, but via looking at its exported varieties developed in other Asian civilizations such as China, Japan and South East Asia. Hence Western knowledge frameworks are almost axiomatic in their belief of Buddhism - associated mainly with other ‘Asian’/‘Oriental’ countries, being distinct from Hinduism - associated exclusively with India. Also Buddhism, since it has been ‘triangulated’ – i.e. abstracted and cross checked - via various independent cultural sources, appears in scholars’ eyes to be a pre-existing coherent and stable category, as opposed to Hinduism which is ‘constructed’ and chimerical in their eyes. This Western study of Buddhism has been generally under the charge of the Buddhologist, who generally has a lot more pan Asian linguistic and cross cultural competence (Staal 1996: 409).

It is also an interesting fact of history that typically China is considered via frameworks emphasizing ‘rationality’ since studying it was the fashion during the European Enlightenment era, whereas India is studied via frameworks emphasizing ‘religion’ since studying it began to be fashionable during the European Romanticism era. These two trajectories have important bearing on where we have ended up today, especially in explaining why Indology is still trapped in the ‘religion’ framework.

In contrast to “stable” Buddhism, the fate of Hinduism – as an academic construct in the charge of the Indologist - has kept changing with the changing perception of India in European, and now Westernized academic eyes globally. As India moved over time from being an exotic ‘mother’ civilization (\textit{via} Voltaire and Romantics), to becoming a colony of defeated kingdoms and a broken down intellectual structure (via Colonial Indology), to finally ending up as a post-colonial ‘developing third world’ country (via Modernization theory, Area Studies, Postcolonial studies etc., all dependent on Indology for their primary base framework even when reacting against it); the valuation given to Hinduism also kept going up and down with it (Breckenridge 1993: Selected summarization in my previous SI-1 paper).

The well founded fact that the backbone of the Indian cultural mainstream is – even now – the continuation of the primarily Vedic, even pre-Vedic civilization cannot be easily wished away; but this shows up mostly via negative critical analysis of today’s politics and problems. The current problems of Indian society are still blamed on Vedic values and traditions – i.e. Hinduism - that refuse to die away in the face of modernity. (Deshpande 1999) shows the history of this in a detailed survey of the history of Westernization, Modernization, Development paradigms in use for Indian sociology/anthropology.

Indologists in turn make indiscriminate use of whatever suits their preset theses from these social science fields. The way in which Indologists totally refuse to do any inter-cultural comparisons to justify this implicit but persistent verdict on Hinduism and India is certainly highly unscientific. No serious comparisons are shown with, for example Christianity or Islam’s relation to the violence prevalent in societies dominated by them. Moreover, all the negativity about Hinduism is juxtaposed against either Western best-case scenarios, or egalitarian intellectually sophisticated Buddhism, which is shown as somehow grandly aloof from all this social context, other than vainly trying to ‘civilize Axially from inside’ and then walking sadly away with tears in its eyes, as it were.


\section*{How It Works In Indology: The Hindu\hfill \break Buddhist ‘Divide’}

The Western construct of ‘Religion” plays a key role in establishing a divide between two entities, namely the religions of Hinduism and Buddhism. Some of the key axiomatic, hence unquestioned, assumptions are as follows. Firstly, it is assumed that there were historically two separate, self-standing Religions of Hinduism and Buddhism, at least after the life of Siddhartha Gautama, the Buddha. Secondly, it is assumed that Gautama started and established a protest movement against the prevalent Vedic religion that is now known as Hinduism. Thirdly, it is assumed that much of the post \textit{Ṛgvedic} writings - especially the \textit{Āraṇyaka-}s and \textit{Upaniṣad-}s that deal with \textit{jñāna}, i.e. knowledge mostly abstracted from Vedic rituals - either (a) did not exist at Buddha’s time, or (b) if they did, did not contain knowledge framed coherently as knowledge. This knowledge-as-knowledge is assumed to be a reactive construction based actually on Buddhist knowledge. (Staal 1996: 115: for details on \textit{Mīmāṁsā}, its chronology) shows a different, more coherent way of reading this, and a way more in consonance with the traditional outlook.

In order for this characterization to stick, there also are unresolved issues that have to be forcibly interpreted one way, i.e. to only favor the hypothesis. A lot of Hindu (non-Buddhist) literature has to be postdated to comply, eg. The \textit{Bhagavad Gītā}, the \textit{Mahābhārata} etc. (Bronkhorst 2003)


\section*{Across Disciplines: A Philosophical View}

If one provisionally puts aside the mode of historicizing, and pays attention to what scholars of philosophy have been saying about the contents of the two thought systems of Hinduism and Buddhism, there is much to be learnt. It is quite easy to see that the commonalities are such that any claim of “axial breakthrough” by Buddhism - in terms of abandoning or radically overturning or altering an existing system in India - ring quite false (Staal 1996: 406ff; Ch28 with detailed reasoning, especially section 28B).

The Buddha’s own statements as saved for posterity via the \textit{Tipiṭaka} are quite clear in this regard. Most emphatically, he never advocated any radical social revolution, i.e. any repudiation of the existing \textit{jāti-varṇa} (‘caste’) system. He insistently said he was only reiterating the ancient knowledge. Also already in place in India was the \textit{Śramaṇa} tradition (\textit{Jaina}, \textit{Ājīvika} and many others) in its many variations which made – and still makes - a person’s world-renunciation a non-unique social event, even a commonplace event. This is what India has known for ages as the distinction between \textit{pravṛtti-mārga} (worldly householder’s way) and the \textit{nivṛtti}-\textit{mārga} (renunciant’s way).

Without taking away any of the merit of the Buddha’s message, there was and is nothing unique in the basic contours of his social experience, found too commonly in many other less publicized traditions across the length and breadth of India. The one possible uniqueness could be the pan Asian spread of his message and the impact it had and continues to have, on other civilizations. Even here, Buddhism-as-Buddhism is not advocating anything as socially radical or revolutionary the way Judaic prophets, or later Christian institutions were doing. It is the receiving culture that adopts and adapts Buddhism to its own social formations, without traumatizing itself by abandoning any of its own pre-existing foundations. (For Buddhism in China/Japan, see Baird 1971; also, Introduction in Puligandla 1994)


\section*{Why Are Both Religions At All, and How Different Really?}

Let us see if it is possible to talk about things without recourse to the idea of ‘religion’ and associated key terminology like theism, founder, scripture, and the like. (Staal 1996:401; Frazer:64-65 Chin Kung on why Buddhism is not a religion) If one can still talk coherently about these two systems, this should alert one to the fact that this ‘religion’ based terminology - grafted fairly recently, let us remember - is not only superfluous, but also misrepresents the entities we are talking about. When one looks at the preserved debates amongst various schools – obviously including Buddhism - in India, one sees just that. They are all based on the standards of \textit{pramāṇa śāstra} (methods of proof). \textit{Pratyakṣa} (available to cognition/senses), \textit{anumāna} (inference), \textit{upamāna} (analogy), and other auxiliaries obviously are \textit{pramāṇa}-s grounded in the logical and empirical. Even with regards to \textit{śabdapramāṇa} (the ‘word’), there is interpretation of the Veda (\textit{śabda}) only for \textit{āstika} schools, whereas since Buddhists will not accept Veda as \textit{pramāṇa}, the debates are carried out without recourse to it. Hence the Buddha’s silence on – and not denial of – theism was hardly the kind of ‘religion’ vs. ‘atheism’ issue that we see in contentious modern Western style debates.

It is not difficult to show that Buddhism in essence is an abstracted, culturally decontextualized path entirely based on and compatible with a huge subset of Hindu thought and practice. Its career inside and outside India has ample evidence showing this (Malhotra 2011, Bronkhorst 2003, Staal 1996). Neither India – the donor – nor the Asian countries that were the receivers were looking at ‘religion’ – especially in the sense of doctrines or truth claims, rather the focus seems to have been to learn and adopt ideas, mores and methods that enhanced their own civilizations. Hence we find that much of the Vedic practices that made it into places as far away as Japan, now considered to have happened under the umbrella of Buddhism (Staal 1996:403-405). This would hardly have been possible if they thought they were dealing with two separable and conflicting entities called “Buddhism” and “Hinduism”.

The existing Hindu cosmology during the Buddha’s time was – and still is - entirely compatible to his message. There is a strong correlation between the Hindu framework of the \textit{pañca-koṣa}-s and the Buddhist five \textit{Skandha}-s, when it comes to modeling the macro and micro cosmos, and the living human body, at increasing levels of subtlety. The analogy between the prevalent Sāṅkhya \textit{/} Vaiśeṣika \textit{tattva}-s and the Buddha’s framework is strong enough to show that he did not have to propose any radically new cosmology, or even do much more than extend existing terminology to put across his teachings. (Dalai Lama 2005:52-53 on \textit{Vaibhāṣika} framework in works of Dharmasri, among others). One might concede that there was an increasing effort to reflect upon this cosmology in terms of developing a structured thinking, and that these would later become the \textit{darśana}-s, of which Buddhism is considered a member, as part of the \textit{avaidika/nāstika} subset.

This makes Buddhism not an originator, but only a part of a trend towards increasing reflexivity – with greater emphasis on structured thinking and conceptual communication in the culture, historically speaking. It is hard to see how Buddhist thought can become the unique originator of some culturally disruptive Axiality when it started out quite within existing frameworks of thought. Ever since the Buddha’s time, Mīmāṁsā / Nyāya / Buddhist debates have a long history showing the topics coalescing around key preexistent themes and issues, and show only an intensification of focus on specific issues post Buddhism, but not any disruption in topics of debate (Vidyabhushana 1988, Puligandla 1994, Phillips 1995)

The genius of Buddhist thought was and is its disciplined self-limitation to actionable aspects of human psychology to solve the existential problem of human existence, i.e. \textit{dukkha}, without getting too entangled in distracting arguments over underlying metaphysics, theology etc (Baird 1971 and others: Sutta about healing from arrow without worrying about the ‘caste’ of the shooter) But this is a far cry from claiming that Buddhism caused some radical shift from a ‘ritualistic’ thinking to a ‘spiritual’ thinking, and that it ‘broke up existing social hierarchies’ by engendering conversions from Hinduism (Weber 2001: on Buddhism having predominantly elite converts).

Buddhist thought, throughout its history, has only sharpened the focus of existing culture towards a more pragmatic psychological emphasis on attaining personal enlightenment, using and building upon tools already at hand before his time. The tools were always geared towards a soteriology that was for a unitive vision of the cosmos, and for personal and social efforts to attain the same via harmony within the person and the society. Staal also clearly shows that Vedic rituals, with high discipline and rigor, but also with their utter lack of interest in this-worldly rewards (the reward-based justifications and explanations came much much later than the rituals themselves proper); can be considered to this day as a continuously existing living prototype of the \textit{niṣkāma karma} or ‘work without expecting results’ that was later spelled out in the \textit{Upaniṣad}-s and the \textit{Bhagavad Gītā}, and also was fleshed out in a uniquely refashioned way in Buddhism. (Staal 1996: 121-122).


\section*{Current State Of Indology Research}

Since Indology derives its authority from its claim to objective knowledge using the latest cutting edge technologies from the social-sciences, we have to examine these tools. Objectivity in analysis is one goal considered worthy in any contemporary endeavor to obtain knowledge. In any social science, this would be great, provided there is a reliable way of removing the inherent subjectivity when one deals with human beings and their social motivations. There is no known reliable way to assess how truly objective an analysis is, since there is no repeatable experiment that can be performed to verify it, and no really unbiased observer to collect ‘clean’ data. The best we can do here is to lay out the current axioms guiding Indology. This will at least clarify how and why the Indologist methodically and repeatably arrives at the conclusions he does. It is only by accepting certain axioms as true can Indological claims seem logical. These axioms need detailed scrutiny.

We have already talked about at length in regard to, and problematized, the first one viz. Religion. We have shown that there is much incoherence in analyzing Hinduism and Buddhism as two separate members of this category viz. Religion. In fact Staal is on record as saying and showing that this category – even in its more universal extended Durkheimian formation – is inapplicable for anything outside the three Abrahamic Monotheisms (Staal 1996: 401, 406, 415).

The next axiom is the pervasively used idea of any society primarily consisting of two well defined, fairly static camps, viz. an elite (presumably free of the ‘taint’ of meritocracy) and the remainder (an exploited proletariat and peasant class), both with diametrically opposed social interests. While admittedly this is a good first cut in analyzing an industrialized capitalist producer/consumer society as consisting of the Owner vs. the Worker; it is an oversimplified binary, and too reductionist for purposes of cultural analysis, especially of pre-Westernized/ pre-capitalist-industrialized, but still sophisticated and complex societies. But this thinking is entrenched in academic social science, with its cultural studies etc., much of which uses this Marxist inspired binary axiom as an unproblematized starting point. In any case, Indian society is implied to be an unregenerate example of elite vs. ‘the people’, in that the “Hindu Caste System” with its supposed static social hierarchy is always brought in to explain (away) whatever latest Indological conjectures there are regarding the causes of India’s social problems.

There are a series of attempts to explain away the empirical fact that over its vast history, Indian society has shown remarkable continuity. It has been exceptionally stable and resilient even in the face of dynamic long duration disruptive events - axial or otherwise - as shown by (Malhotra 2011 on Integral Unity).\endnote{ India itself cannot be viewed only as a bundle of the old and the new, accidentally and uncomfortably pieced together, an artificial construct without a natural unity. Nor is she just a repository of quaint, fashionable accessories to Western lifestyles; nor a junior partner in a global capitalist world. India is its own distinct and unified civilization with a proven ability to manage profound differences, engage creatively with various cultures, religions and philosophies, and peacefully integrate many diverse streams of humanity. These values are based on ideas about divinity, the cosmos and humanity that stand in contrast to the fundamental assumptions of Western civilization.}

Never allowed in the Indological analysis is the capacity for constant adjustment and resulting social dynamism exhibited by the civilization/society, in the face of both internal and external origin events. Evidence attesting internal dynamism and material progress is being rediscovered as more and more historical data comes out debunking the Orientalist/Marx-inspired stereotypical construct of the timeless static India of the sleepy villages. The response against major externally imposed events are well known - as evidenced by India’s survival after centuries of Islamic imperial violence as well as Western colonialism that followed almost immediately.


\section*{Pollock as Indologist: A Scholar, a Doctor, or a Prosecutor?}

This brings us to the attitude with which Indologists conduct the practice of Indology, both in general, and in Pollock’s case in particular.

A scholar is supposed to dispassionately study facts and data that then would lead to a hypothesis that shows causal patterns and explains events, i.e. create a valid model from objectively researching available data. On the other hand, a doctor is supposed to diagnose and ‘solve’ the ‘case’, where the solution is to restore the patient’s health. A doctor is different from a medical researcher (i.e. a scholar) who may study the same phenomena, since he – the doctor, but not the scholar - has the responsibility to make life and death decisions affecting the patient’s future, but in a positive way. A medical researcher – typically not responsible for the patient’s health - cannot advocate solutions, let alone make direct life and death decisions. Amongst attorneys, a public prosecutor is also supposed to use all the facts/data at his command, and ‘solve the case’. But in his case, success typically means a ‘negative’ outcome of a ‘guilty’ verdict for the accused. In actual practice, the prosecutor - though supposed to be fair and objective - is under huge pressure and incentivized to produce guilty verdicts. So he conducts his case with a specific outcome in mind, it is not open ended inquiry the way a scholarly inquiry is supposed to work.

If the subject/case is a society/culture/civilization, each of these above approaches necessarily lead to different results, different ‘negative’ or ‘positive’ outcomes. The prosecutorial approach that Pollock arguably adopts is pre-ordained to the tenets of his Political Philology. The ‘verdict’ already built into its presumptions necessarily leads to a ‘negative’ outcome for the culture/civilization under scrutiny, making this an overwhelmingly ‘prosecutorial’ venture. There may be pretensions of being a ‘doctor’ with the civilization as a ‘patient’ to be ‘cured’, but then the ‘doctor’ would have to be held up to a much higher standard of accountability, along with the Hippocratic oath: “First do no harm”. Wallerstein asks for “reopening entirely the epistemological question” instead of being caught in the binary of Orientalism vs. its reverse (2006:47). When research holds on to its Orientalist roots, but still morphs into and becomes muscular activism – as Pollock’s work avowedly does; however scholarly its claims may be, the bar should be much higher. The question is, do Indologists like Pollock hold themselves up to such a high standard?

Here it might be instructive to see the prosecutorial approach that is in the DNA of Indology. Right from its inception during colonial times, Germany has been at the forefront of Indology both as a center of major activity and as a source for the overwhelming majority of authoritative scholars, including the very important Sanskritists. Adluri and Bagchee (2017) show, with comprehensive first-hand references and footnotes, that the clearly discernible motive for German Indology was to demonize and then displace the traditional Brahmin led scholarship framework of Sanskrit texts and practice.

The authors elaborately show how the explicitly Enlightenment-critical and implicitly Protestant supersessionist bias in their \textit{Wissenschaft} framework meant that the Indian tradition was \textit{a-priori} held guilty of being responsible for India’s ‘decadence’ and other gross social ills and injustices, which the said \textit{Wissenschaft} would correct. They clearly show that German Indologists qualify as a ‘caste’ of neo-Brahmins by Max Weber’s own influential sociological criteria, whereby their whole \textit{Wissenschaft} methodology was predicated upon usurping the Brahmins’ role for the sake of ‘progress’ and ‘science’. Not only that, they also clearly show that current Indology – including the Pollockian variety - is not too far methodologically from its original roots, and so has the same issues of bias and usurpation of \textit{adhikāra}. This notwithstanding loud cries of “Deep Orientalism” (Breckenridge 1993) where Pollock tries to make out Brahmins as the “original Orientalists” by another typical masterly sleight-of-hand, projecting historical eras backwards in time and inverting the aggressor and victim relationship.


\section*{History of Axiality and Its Relevance for Indology}

The quasi hypothesis of the “Axial Age” in its strong form, or the “Axial Breakthroughs” in its weak form – is being generally revived in academic global studies involving Religion, and is specifically being deployed as a major tool in the analytic toolbox of Indologists. In a simplified form it can be understood as the claim – that there was a global breakthrough/disruption in human history, across most major world civilizations (hence the name ‘Axial civilizations’). This breakthrough is supposed to be based on an increased reflexivity (tendency to look in as if from outside, i.e. as a detached observer) with the historical progress of human thought. This led to people writing down objective descriptions of their own existing societal structures and their cosmologies, and following it up by discussion and critique, and eventual problematization of the same.

Inevitably, owing to its genesis in the Western thought world, the theory depends heavily on the Greco-Semitic cases as a paradigmatic starting point, and attempts to extend this to other major civilizations as cases of a similar process. Buddhism, with its pan Asian spread and deep history across cultures and civilizations, is invoked as also “Axial”, especially to give the putative theory a global ring beyond what would in essence be just a Mediterranean phenomenon, which used to suffice as a stand-in for “humanity” in the Eurocentric past.

This is not finding uncritical acceptance: “But we must beware of fitting them into a master teleological narrative composed from what is itself a parochial and selective point of view.” (Bellah 2012:333).

In its strong form, the “Axial Age” idea depends on the profusion of written philosophy by Greek philosophers, and the writing down of Judaic Prophetic revelations, both of which happened around the time of 500 BCE plus or minus 200 odd years. India and China are pulled into this framework in ways that are still under contention (Bellah 2012). In its weak form the hypothesis merely says there were different breakthroughs in different civilizations, and opens up the timeline to allow for more cases from these and other civilizations to appear part of a relevant dataset, thus lending the hypothesis more validity.

The extension of Axiality for India involves the appearance of Buddhism (and Jainism, \textit{Ajīvika}-s, etc. to a lesser extent) around the time of 5th century BCE. Here’s where we start seeing the key anchor for justification of the idea of radical difference between Hinduism and Buddhism. Since Axiality involves disruption to existing societal patterns, Buddhism must have provided this disruption – a logical deduction from these premises based on extant Indological scholarship.

This however is refutable. Buddhism, even if a breakthrough, by no means is proven to be a civilization level disruption. Nor is the breakthrough unique even inside India. In fact, the move from \textit{Saṃhitā-}s\textit{ to Brāhmaṇa-}s\textit{ to} \textit{Āraṇyaka-}s to \textit{Upaniṣad-}s is a textually mapped journey for Vedic thought, very similar to the Buddha inspired one, or to the Mahavira inspired one for Jaina thought. The simplest explanation would be of them all crossing evolution thresholds and undergoing cross fertilization around this timeframe. This is the simplest data-backed explanation of Indian conditions over time, and would work for an objective observer. This argument is also given strongly at a global level at various places (See Bellah 2014). There have been and still are many scholars who still hold this view. In spite of being heavily invested in Buddhist scholarship, Obeyesekere (2012) does not quite endorse the idea of Buddhism being a huge disruption in India. “Obeyesekere, with respect to early Buddhism, takes almost an opposite approach. He suggests that our modern notion of the theoretic, what he calls “conceptualism,” though found in Axial India, is inadequate as an exclusive way of understanding what was happening. He emphasizes the presence of visionary experience and aphoristic thinking as moving beyond purely rational thought, though with universalizing consequences.” (Bellah 2014:5).

It might be pertinent to add that if these Western scholars invested in some deeper study of the entirety of the Indian tradition, they would also easily discover the sequence: that the “vision” / “\textit{darśana}” always comes first – whether Vedic, Buddhist or otherwise, even and including current gurus – and only later is it broken out into a philosophical framework. Dissemination happens orally next, and only later is transmission stabilized with help of writing, if at all. Obeyesekere acknowledges this clearly for the Buddha (all emphases mine) (2012:131-133, and elsewhere):

\begin{myquote}
“\textbf{Visionary Knowledge}
\end{myquote}

\begin{myquote}
The kind of \underline{visionary knowledge} that I have discussed thus far entails, I think, the abdication of the Cartesian \textit{cogito}, at least when knowledge appears before the “eye” of the seer, irrespective of the religious tradition involved. Thus Julian’s characterization of her visions as “showings.”$^{8}$ The Buddha’s showings during the first and second watches of the night occur when discursive thought is in abeyance, as is clearly recognized in early Buddhist texts. This means that the thinking-I is suspended during trance, dreaming, and psychotic fantasies and also in fleeting moments when pictures as well as thoughts of a non discursive nature float into our ken. One must not assume that cerebral activity is suspended during this state. I am inclined to postulate the idea of passive cerebration as against the active I-dependent cerebral activity involved in our rational discursive thinking processes.
\end{myquote}

\begin{myquote}
(…)
\end{myquote}

\begin{myquote}
Given the preceding discussion, the conventional view of Buddhism as an exclusively rational religion has to be seriously reconsidered. It was the theosophist-cum-rationalist Colonel H. S. Olcott who asserted that “Buddhism was, in a word, a philosophy, and not a creed,” and this credo has become the standard view of native intellectuals in contemporary Buddhist societies.$^{13}$ Yet contrary to modern Buddhist intellectuals, the Buddhist ratio is radically different from both the Greek and the European Enlightenments.$^{14}$ \underline{The European Enlightenment with its reification of}\break \underline{rationality ignored or condemned visionary experiences; not so the}\break \underline{Greek, it seems to me}.
\end{myquote}

\begin{myquote}
Plato employed reason for discovering true knowledge, but neither he nor Socrates condemned or ignored such things as the work of visionaries and prophets and personally believed in the oracle at Delphi. By contrast, the Buddha condemned all sorts of popular “superstitions” as base or beastly arts in a famed discourse known as the \textit{Brahmajāla Sutta} (“Tenet of Brahma”) but never visions and knowledge emerging through meditative trance (\textit{jhāna}).$^{15}$ During the first and second watches the Buddha sees his own life histories that then are extended to include those of human beings in general, their births and rebirths in various realms of existence. Through the “pictorialization” of births and rebirths, the Buddha can grasp the doctrine of \textit{karma} and rebirth, can see it operating. During the third watch he discovered the existential foundation of Buddhism, these being the Four Noble Truths of Buddhism: \textit{dukkha}, suffering, the unsatisfactory nature of existence owing to the fact of impermanency; \textit{samudaya}, how \textit{dukkha} arises owing to \textit{tanhā}, thirst, attachment, greed, desire, or craving; \textit{nirodha}, cessation of craving that might ultimately lead to \textit{nirvana}; and \textit{magga}, or the path that can help us realize \textit{nirvana}, also known as the “noble eightfold path” including right understanding, right thought, right speech, right action, right livelihood, right effort, right mindfulness, and right concentration. Right concentration is \textit{samādhi} or the meditative disciplines leading to complexly graded states of trance (\textit{jhāna, dhyāna}) that permitted the Buddha to intuit the very truths mentioned above.
\end{myquote}

\begin{myquote}
We do not know how the Four Noble Truths appeared to the sage in the dawn watch. An early text, however, gives us a clue regarding \textbf{the manner in which intuitively derived knowledge is given rational reworking}. It says that when an Awakened Being has arisen in the world, there is a great light and radiance (associated with direct visionary knowledge), and \underline{then “there is the explaining, teaching, proclaiming, establishing,}\break \underline{disclosing, analyzing, and elucidating} of the Four Noble Truths.”$^{16}$ The Buddha adds: “This, \textit{bhikkhus} [monks], is the middle way awakened by the \textit{Tathagata} [Buddha], which gives rise to vision, which gives rise to knowledge, which leads to peace, to direct knowledge, to enlightenment, to \textit{Nibbana} [\textit{nirvana}].”$^{17}$” 

~\hfill (Obeyesekere 2012:131-133)
\end{myquote}

Writing is held up as a major innovative feature of Axiality, but many scholars, for example Staal, also recognize that a civilization can be a ‘high’ civilization in spite of – or even because of – its text transmission being oral for the most part, with writing being only a later ‘back-up’ feature (Staal 1996:37, 142-143, 385). Indologists, especially of the Pollockian mode, still have a hard time dealing with – or explaining reasonably - the precision/accuracy and integrality of the oral transmission that the Vedic corpus maintains to this day. They would rather stick to elevating and anointing dead written texts as ‘normative’ and speculate on motives of the writers/readers, than engage with the living tradition.

Lack of interest in going deep enough would naturally leave one looking selectively to show Axial disruptions even when the civilization under study has repeatedly shown its enduring continuity, even while allowing a multiplicity of systems to thrive.

Arnason (2005) and Bellah (2014) show that there is much justification, as well as contestation of the validity of this Axiality hypothesis among academics. It is one among many alternate ways of looking at the world’s historical trajectory, which could also be independent of any idea of Axiality. There is nothing universally valid about it, in fact it is certainly a product of Western thinking, which tends to be Universalistic. For example Wallerstein says, as part of his “World Systems” hypothesis\endnote{ \textbf{World Systems Analysis} (Wallerstein 2004):

World Systems analysis is a methodology a few decades old, pioneered by Immanuel Wallerstein, and taken up and developed by many mainstream scholars. This is an attempt to systematically, and falsifiably (hence scientifically), set up a framework to analyze social events in a unidisciplinary way. The reason why it makes an appealing contrast to patently unscientific hypotheses like Axiality is that these have a built in history of Western Universalistic axioms that are not possibly to explicitly falsify( being based on authoritativeness, but can be seen increasingly to be less and less applicable to societies that do not share the peculiarities of Western European history.

Being unidisciplinary is to eschew the historical problems, involving the way disciplines in current social science have evolved from a Eurocentric base. It puts categories of time and space in a frame with a beginning and end, i.e. understandable limits. Time is based on the ‘\textit{longue duree}’, lasting from the beginning to the end of the particular ‘world system’. The geographical space is the space which functions as a ‘world’ with its own multiple nation-states with their own production processes and interstate trade and other interactions, e.g. the Mediterranean world.

Wallerstein and others have established that the European world since the industrial revolution can be seen as a prototypical world-system, a system that, moreover, has over the centuries expanded to encompass most of the globe today, and a system that is in its end stages now.

While this system needs much more development and broadening to be really applicable to our topic at hand, i.e. ancient Indian history, its value lies in the fact that we can see it as a clear mirror that can show up the problems and plain incapability of the current systems of social science to show a satisfactory model of ancient India. Needless to say, one key category that India and other non-Western civilizations can add to the world systems is the \textit{ādhyātmika} aspect; since the current European Enlightenment based systems – even world-systems – only have an ill-suited and narrowly relevant category of “Religion” to cover a vast aspect of human personal and social life.}

\begin{myquote}
“So we may start with the paradoxical argument that there is nothing so ethnocentric, so particularist, as the claim of universalism. Still, the strange thing about the modern world-system—what is uniquely true of it—is that such doubt [regarding universalist arguments]is theoretically legitimate. I say theoretically because, in practice, the powerful in the modern world-system tend to show the claws of orthodox suppression whenever doubt goes to the point of undermining efficaciously some of the critical premises of the system.” 

~\hfill (Wallerstein 2006:39-40)
\end{myquote}

By advancing the Axiality argument to show Buddhist ‘disruption’, Pollock is implicitly doing just that, using a mostly Western particularist reading to claim universal applicability, in this case to Indian conditions. (Pollock 2004)’s influence is such that others are now quoting him on the Buddhist Axiality hypothesis without critical comment, as Wittrock is in (Bellah 2012:115-116).

The Axiality hypothesis is just not compelling enough to be accepted willy-nilly for India, as Pollock, Shulman, etc. have been doing, per their contributions to the theorizing in (Arnason 2005). In fact, from a truly reflexive objective viewpoint, it seems quite obvious that the hypothesis is yet another attempt to construct a new grand narrative of world history, sophisticated on the surface, but one which has not abandoned the deep structures of previous Western attempts to universalize its own experience and knowledge as world knowledge as shown in (Wallerstein 2006) quoted above. Critics have indeed noted this, and hence the theory has not gained currency outside of select circles. But that has not stopped Pollock and his acolytes from using it, since it is so convenient to explain, or rather explain away their ideas/conjectures as if based on solid historical foundations (Kennedy 1996, Wallerstein 2006).

\vspace{-.3cm}

\section*{Pollock’s Buddhism and Its Axiality}

Pollock’s take on Buddhism draws upon existing Indological scholarship, but it draws upon only a specific strand – as is only to be expected. As explained earlier (Staal 1996), the Indologist has a different background and training which is highly India-specific, as against a Buddhologist who is familiar with Buddhism’s interaction with many diverse host cultures such as Tibet, China, Japan, Sri Lanka, Thailand etc. What Pollock does is to extend a very sketchy understanding of Buddhism’s Indian origin and history, and use the premise of Axiality’s global ‘disruptive’ pervasiveness. This is in order to claim that though there is not much ground evidence for Axiality being existent in ancient India in the realm of Power (no Empire level Axial disruptions), there are indeed strong grounds for axiality in the realm of Culture. So he says: “Accordingly, alternative explanations of imperial practices need to be elaborated, along with alternative models of the relationship of culture and power beyond those familiar from Western history and the EuroAmerican social theory that this produced” (Pollock 2004:400).

In other words, though nobody has so far shown Buddhism as leading to any large scale political disruption (realm of Power), Pollock’s model seeks to show that it did lead to a huge cultural disruption, and thus reemphasize its qualification as an Axial disruption.

Readers may remember that much of Pollock’s huge body of analysis is about the relationship between Culture and Power, using his study of India’s Sanskrit texts as source. His study of texts is exclusive, needless to say, and focused to a degree that does not allow non-textual evidence to mar his conclusions. And his conclusions are aligned inevitably with his ideological goals of promoting his Political Philology.

Here’s where it is indeed very important to also do an in-depth reading of other scholars of India, both Western and non-Western. This helps one to not get caught in skewed readings that are more polemical than factually objective. Frits Staal is a very pertinent case as a counterpoint to Pollock, especially since he has gone to the trouble of moving way past the textual-only analysis and done extensive field work on Vedic rituals for decades. In his book, Staal (1996) draws upon multiple textual traditions including Western Buddhology and Indology, along with the tradition’s own view of what it is doing. For very clearly argued reasons, Staal thinks (a) that the category of “Religion” misrepresents both Buddhism and Hinduism, and also (b) that Vedic ritual is not originally or primarily meant to have instrumental meaning in the mundane world. It is ‘science’, a ritual for the sake of ritual, an Orthopraxy, not Orthodoxy.

The implication of (b) above is a body blow to Pollock’s theorization that pre-Buddhist Vedic culture was all about mystifying the people for political ends, i.e. ‘in the realm of power’ as Pollock would put it. The implication of (a) basically lends strong support to what is contended in this current paper, that the Pollock thesis on Hinduism vs. Buddhism – with Buddhism producing major Civilizational disruption – even makes sense only in the categorization scheme of both being Religions and separate and self-standing, not otherwise.

In his 2004 paper, “Axiality and Empire”, Pollock is saying, in effect, that though the advent of Buddhism caused a lot of cultural change in India; it is indeed quite remarkable that this effect did not seem to show as spilling over into the domain of politics. He sees this as a flaw in the Axiality model, and brings in his model to show how cultural disruption meant political disruption in the Indian context. Needless to say, this depends on his characterization of the \textit{Veda} as nothing more than a hegemonic political instrument in the hands of the elite Brahmin-Kshatriya combine.

He again asserts that his analysis will not address any ‘transcendental’ issues as he finds the whole idea of transcendence problematic, that it “illegitimately privileges religion”. This is consistent with his theorizing that religion – even and perhaps especially the Hindu variety - is basically a cover for politics.

As he correctly notes, building of ‘translocal’ empires is a key feature denoting axiality in a civilization, examples being Achaemenid, Roman, etc. His effort is to show that though India did not have an explicitly political translocal empire, its elites – primarily Brahmins – made clever use of the \textit{Veda}-Sanskrit combine to gain political hegemony over a huge landmass and population, i.e. an empire in all but name.

The rest of his paper is about detailing firstly why he considers Buddhism Axial to India specifically, and then to contrast this with a detailed analysis of how Indian empire formations were consistently different from the other axial empires of Persia and Rome.

\vspace{.1cm}

To show the axiality of Buddhism, he can lean on Jaspers himself, one of the founding fathers of the current incarnations of Axial theory. Schwartz follows Jaspers in this, but Eisenstadt had initially called late Vedic thought as ‘wholly axial’ and Buddhism as a secondary breakthrough. Pollock disagrees “according to the typology offered above”, i.e. his re-theorizing. He also quickly dismisses Heesterman’s similar, nuanced view with: “(He had argued) that it was the “gap” between Vedic revelation and ritual routinization, where rational order replaced “unsettling... revelatory vision,” that constituted India’s “axial turning point,” a conception again too vague to be of much use” (Pollock 2004:401). Accepting that the Vedic world itself took care of ‘updating’ itself Axially, would of course render Buddhist Axiality superfluous. He then finds “not further elucidated” Kulke’s view that stresses the social aspects of Buddhism, i.e. \textit{Sangha} etc., as being more pertinent to Buddhism’s axiality than the Buddhists precepts in themselves.

\vspace{.1cm}

Here he gets to his key points on the subject of Buddhism vs. Hinduism, with early \textit{Mīmāṁsā} standing in as representing Hinduism. Saying that: “No adequately detailed and textually sensitive account is available of what the critique enunciated by the early Buddhists meant within the larger intellectual history of South Asia”, he moves on to lay out its outline. Readers can note the axiomatic assertion of Buddhism being a ‘critique’. The core of his position is as quoted here (all emphases mine):

\vspace{.1cm}

\begin{myquote}
“While there can be hardly any doubt that the principal thrust of the Buddhist critique was directed toward actually-existing elements of the thought-world of early Brahmanism, it also seems likely that at least some of the most salient articulations of this world, what we now tend to think of as its foundational principles, may have first been conceptualized as a defensive, even anti-axial, reaction to Buddhism”
\end{myquote}

\vspace{.1cm}

\begin{myquote}
…
\end{myquote}

\vspace{.1cm}

\begin{myquote}
“It is self-evident that no one would elaborate propositions of the sort we find \textit{Mīmāṁsā} to have elaborated, such as the thesis of the authorlessness of the Veda, unless the authority of the Veda and its putative authors had first been seriously challenged.” 

~\hfill Pollock (2004:402)
\end{myquote}

There are a number of ‘self-evident’ axiomatic assertions here, beginning with the certainty of Buddhism being a critique of Hinduism at large. It is this point we can tackle here, since everything else is supposed to logically follow from this and the resulting “transvaluation”, for the next millennia or more.

Again, refocusing on categories of analysis, our first question is whether any of above assertions make sense outside the ‘religions’ criteria, i.e. what if this criteria were not employed? Do we see any serious assertions/hypothesis/‘proven theory’ in philosophical discussions – either Indian or Western – that Buddhism did effect any far reaching ‘transvaluation’? The philosophical categories and related discussions carried on in India for centuries are nothing if not about ‘valuation’. Here we have extensive material (Puligandla 1994, Phillips 1995, Potter 1990) that shows that while there were intense disputes about language, meaning, about soteriology (how to achieve ‘Salvation’, i.e. \textit{Nirvāṇa/Mokṣa}), there was never any major ‘shift’ in the categories / cosmology used in the discussion. Yes, there was addition to the categories, creation of a few sub-categories, and a deepening of the definitions; but that cannot qualify as a sweeping change, as the assertion of ‘transvaluation’ seems to imply. Staal shows this in great detail. “… the Buddha did not preach in a vacuum. His teachings were not only preceded by those of Jainism, but they formed part of the general intellectual ferment that characterizes the seventh and sixth century B.C.E. context in India. In that context it should be simple to determine what doctrinal innovations the Buddha offered. In fact, they are surprisingly difficult to find and formulate” (Staal 1996:406-410). Staal continues to survey putative differences and shows there are no significant ones.

Pollock pulls in philosophically tertiary (but politically sensitive present day) issues, for example on ‘sacrifice’ (presumably animal killing, implied though not explicitly stated) via \textit{kūṭa dantasutta} which is not a major \textit{sutta} (Baird 1971, Grimes 1996: 367 charts on all Indian systems, showing classification of the main \textit{Tipīṭaka}-s). This is good drama and tabloid style ‘crime report’, but hardly scholarly evidence.

\newpage

He then calls ‘nonviolence coupled with noncoercion’ (the story of the Buddha dissuading a Brahmin from live animal sacrifice to symbolic sacrifice) as a ‘major ethical inversion’ (Pollock 2004: 402-403). But (Staal 1996:408) emphatically shows the commonality of outgrowing animal sacrifice in both \textit{Upaniṣad}-s and \textit{Jaina} works too, without dependence on the Buddha.

Pollock’s claim is that “Buddhist meditation” is somehow a precursor of all meditation systems, including the Hindu/Yogic \textit{dhyāna}/medita\-tion. Here he utterly ignores the fact that the meditations – like \textit{Ānāpānasati} / \textit{Ānāpānasmṛti} - that Buddha learnt from ‘pre-Buddhist’ Udraka Rāmaputra and Ālāra Kālāma still form the 3rd and 4th \textit{Jhāna}-s prior to the final awakening/enlightenment (Elst 2013).

So Pollock’s implication that Vedic life was all about ‘sacrifice’ (implying animal killing) shows a lack of in depth explanation / understanding of what a \textit{Yajña} really meant then, or means even today. Staal shows \textit{yajña} as much more than ‘sacrifice’ usually used in an Abrahamic sense (1996:69). Pollock’s mischaracterization of Vedic ritual is thus way off the mark, but very convenient for his overall thesis.

The other point that Pollock’s analysis focuses on is the idea of “normative inversion”. Here he attempts to show that Buddhist thought basically took existing words from the Vedic universe and inverted their meanings. Quoting:

\begin{myquote}
“Consider the name chosen for the Buddha’s teaching, \textit{dharma} (Pali \textit{dhamma}), or even more combatively, \textit{saddharma}, the real or true \textit{dharma} (already in the oldest parts of the Pali canon). An ancient, even primary, meaning of \textit{dharma}, the key word of Vedic ritualism, is sacrifice — it is to sacrifice that the \textit{Mīmāṁsāsūtra} is referring when it opens with the words “Now, then, the inquiry into \textit{dharma}.” Early Buddhism thus sought to annex and redefine the term that expressed what Buddhism most fundamentally rejected. (Even \textit{dharma}’s somewhat later sense of “duty” as an expression of one’s essential nature is turned upside down in the antiessentialist Buddhist appropriation.) Similarly transgressive redefinitions pertain to \textit{ārya}, recoded from its old meaning, “noble,” a member of the “twice-born” social order, to “adherent” of the Buddhist spiritual order. 

~\hfill Pollock (2006:51-52), slightly reworked from Pollock (2004:403)
\end{myquote}

\begin{myquote}

~\hfill (\textit{italics as in the original})
\end{myquote}

He wishes to make it appear as if the Buddha really was thinking not like an enlightened master, but like a political strategist of modern times, “combatively” relabeling names like \textit{Dharma} and \textit{Ārya} and related concepts, to steal ‘membership’ and legitimacy from the competitor Hinduism. In this whole approach, what is clear is that both Vedic and Buddhist thought are reduced to ideologies that are competing for market share in society, and nothing more.

This again underscores the key point made by Malhotra (2015) that there is a major and consistent effort in Pollock’s works to deny the existence of \textit{ādhyātmika/pāramārthika} components in Dharmic thought, be it Vedic or Buddhist. Obeyesekere (2012) is also pretty clear on this point, as quoted earlier here.

Here it is also glaringly obvious for anyone who knows their history that the reason why Chinese and other Asian scholars flocked to India was not because they were looking for newer \textit{vyāvahārika}/sociopolitical ideologies, but because they saw a chance to learn knowledge and wisdom including, but quite often beyond the mundane \textit{vyāvahārika} realm. “The diffusion was not just of Buddhism, but included the exportation of Indian philosophy, logic, science, medicine, astronomy, grammar, and Sanskrit legends, lore and literature” (Staal 1996:402). Also “We should first of all take into account what the Chinese were looking for. They were not waiting for a "religion."” (Staal 1996:404-405). Here Staal then details the items of interest to the Chinese, quoting Strickmann.

In fact, Staal goes so far as to say that the more appropriate word would be “Indianization”, since Hinduism also was a huge export –especially to South East Asia – and it is difficult to extricate India’s Hindu and Buddhist exports from one another into neat separable ‘religious’ packages (Staal 1996:403-404 giving the sequence and items in detail). (Staal 1996:192-193, 228, 261) also shows that they also absorbed huge amount of Vedic mantras, which – contrary to what Pollock’s spin may lead one to believe – the \textit{Mahāyāna / Vajrayāna} Buddhists of India/Tibet had also respectfully preserved and were actively using – as mantras - in their practices.

Another very important point that is repeatedly de-emphasized in the Pollock type of Indological analyses is that - with their text based fixations - they virtually ignore the fact that India even during Buddha’s times was a sophisticated civilization, despite not being a ‘literary’ one, i.e. one where writing was the key means of preservation and propagation of culture. (Staal 1996) shows in great depth how the science of ritual - Vedic in particular - can be seen to be the key structure upon which language – Sanskrit in particular – has been built, with phonetics and syntax in place structurally - ritually, even before the search for semantic meaning of utterances led to the full blown development of language as we know it (Staal 1996:138-139).

The key point for us here is that Buddhist thought merely used and expanded on the already existing sophisticated base, and had no reason for, or interest in, any Axial disruption of the existing knowledge base, either \textit{ādhyātmika}, \textit{ādhidaivika} or\textit{ ādhibhautika}.

\vspace{-.3cm}

\section*{Conclusions}

This paper tries to show that the thesis of radical difference between Hinduism and Buddhism is built on very shaky premises, and is basically untenable. It brings back the question to whether it is suitable to use the ‘Religions’ construct to talk sensibly about Buddhism in opposition to Hinduism. It shows that when one ventures beyond purely textual analysis - that too based on some questionable \textit{a priori} axioms as Indologists are wont to do - the idea of the Buddha’s teachings being civilizationally disruptive in India - in the Axial sense - also does not hold much water.

Further this paper surveys the domains of philosophy, both Western and Indian, as well as anthropological fieldwork by scholars like Staal, to show how thoroughly is misrepresented the Indological interpretation of Vedic ritual - a foundational aspect of Hinduism. Anyone in touch with the practicing tradition will resonate with Staal’s claim that Hinduism, particularly the Vedic ritual, is more about “Orthopraxy” than about “Orthodoxy” as Indologists are wont to treat it.

For someone studying Hinduism and Buddhism, the key difference essentially is that between (a) studying written texts and then projecting one’s \textit{a priori} axioms onto the civilization that produced it; vs. (b) taking the trouble to do an open minded study of the same civilization, by opening the field beyond texts, to the actual practices of the culture; a culture, moreover, that deserves to be studied as the living culture that it is, and not in the form of “the wonder that was”, i.e. a voiceless museum specimen represented only via dead texts.

Scholars like Staal (who with his deep engagement with the living culture followed up by his opus (Staal 1996)) have shown it i.e. (b) above can be done – to the point of restoring to Vedic ritual the status of science, whereas the Indological tradition exemplified by Pollock, unfortunately has not risen up to the challenge.


\section*{Acknowledgments}

The inspiration behind this paper is the series of thoroughly researched books of Rajiv Malhotra, beginning with \textit{Invading The Sacred} (2007), then \textit{Breaking India} (2011), followed by the seminal \textit{Being Different} (2011) that laid out the basis for an alternate framework for categorizing Indian civilizational knowledge systems, then \textit{Indra’s Net} (2014), culminating in the \textit{Battle For Sanskrit} (2016) which established a standard from which to begin a \textit{pūrvapakṣa} of Western Indology. It is hoped this paper is a small but meaningful effort in strengthening that argument.


\section*{Bibliography}

\begin{thebibliography}{99}
\itemsep=1pt
\bibitem{chap3-key01} Adluri, Vishwa., and Bagchee, Joydeep (2017). “Jews and Hindus In Indology”. \url{https://www.academia.edu/30937643/Jews_and_Hindus_in_Indology}. Accessed on 01 April, 2017.

 \bibitem{chap3-key02} Arnason, Johann P., Eisenstadt, S.N, and Wittrock, Bjorn (Ed.s) (2005). \textit{Axial Civilizations and World History.} Boston: Brill Leiden.

 \bibitem{chap3-key03} Arya, Vedveer. (2016). “The Epoch of Śaka Era: A Critical Study”. \textit{India Facts.} \textless \url{http://indiafacts.org/epoch-saka-era-critical-study/}\textgreater . Accessed on 01 April, 2017.

 \bibitem{chap3-key04} Baird, Robert D., and Bloom, Alfred. (1971). \textit{Indian and Far Eastern Religious Traditions.} New York: Harper \& Row.

 \bibitem{chap3-key05} Balagangadhara, S. N. (1994). \textit{“The Heathen in his Blindness…”: Asia, the West and the Dynamic of Religion}. Amsterdam: E J Brill.

 \bibitem{chap3-key06} Bellah, R. N., and Joas, H. (2012). \textit{The Axial Age and Its Consequences}. Harvard University Press.

 \bibitem{chap3-key07} Breckenridge, Carol A. and van der Veer, Peter. (Ed.s) (1993). \textit{Orientalism and the Postcolonial Predicament: Perspectives On South Asia, New Cultural Studies}. Philadelphia: University of Pennsylvania Press.

 \bibitem{chap3-key08} Bronkhorst, Johannes. (2003). “Hinduism and Buddhism”; In Buswell Jr. (2003). pp 327--330.

 \bibitem{chap3-key09} Buswell Jr., Robert E. (Ed.) (2003). \textit{Gale Encyclopedia of Buddhism.} New York: Macmillan Reference USA.

 \bibitem{chap3-key10} Dalai Lama, The. (2005). \textit{The Universe in an Atom, The convergence of Science and spirituality}. New York: Broadway.

 \bibitem{chap3-key11} Das, Veena. (1999). \textit{Oxford Companion to Sociology and Social Anthropology.} New Delhi: Oxford University Press.

 \bibitem{chap3-key12} Deshpande, Satish. (1999). “Mapping a Distinctive Modernity: ‘Modernization’ as a Theme in Indian Sociology”. In Das (1999).

 \bibitem{chap3-key13} Elst, Koenraad. (2013). “When did the Buddha break away from Hinduism?” \url{http://koenraadelst.blogspot.com/2013/08/when-did-buddha-break-away-from-hinduism.html?m=1}. Accessed on 02 April, 2017.

 \bibitem{chap3-key14} Frazer, Charles A. (Ed.). (1999). \textit{World History: Original and Secondary Source Readings, Vol 1.} San Diego: Greenhaven Press.

 \bibitem{chap3-key15} Grimes, John. (1996). \textit{A Concise Dictionary of Indian Philosophy}. Albany: SUNY Press.

 \bibitem{chap3-key16} Kennedy, Dane. (1996). “Imperial History and Post-Colonial Theory”. \textit{The Journal of Imperial and Commonwealth History}, Vol.24, No.3, September 1996, pp.345--363.

 \bibitem{chap3-key17} \url{http://users.clas.ufl.edu/harlandj/courses/5934/empire_sp13/ken.pdf#page=1&zoom=auto,371,849}. Accessed on 02 April, 2017.

 \bibitem{chap3-key18} Malhotra, Rajiv. (2011). \textit{Being Different: An Indian Challenge to Western Universalism.} New Delhi: Harper Collins.

 \bibitem{chap3-key19} — (2014).\textit{ Indra's Net: Defending Hinduism's Philosophical Unity.} New Delhi: Harper Collins.

 \bibitem{chap3-key20} — (2015). \textit{The Battle For Sanskrit.} New Delhi: Harper Collins.

 \bibitem{chap3-key21} Obeyesekere, Gananath. (2012). “The Buddha’s Meditative Trance: Visionary Knowledge, Aphoristic Thinking, and Axial Age Rationality”. In Bellah and Joas (2012). pp 126--145.

 \bibitem{chap3-key22} Phillips, Stephen H. (1995). \textit{Classical Indian Metaphysics}. Chicago: Open Court.

 \bibitem{chap3-key23} Pollock, Sheldon. (2005). “Axialism and Empire”. In Arnason \textit{et al.} (2005). pp 397--450.

 \bibitem{chap3-key24} — (2006). \textit{The Language of the Gods in the World of Men; Sanskrit, Culture, and Power in Premodern India}. Berkeley: University Of California Press.

 \bibitem{chap3-key25} Potter, K. H. (Ed.). (1990). \textit{The Encyclopedia of Indian Philosophies, Volume 5: The Philosophy of the Grammarians} (Vol. V). Princeton: Princeton University Press.

 \bibitem{chap3-key26} Puligandla, Ramakrishna. (1994). \textit{Fundamentals of Indian Philosophy}. New Delhi: D K Printworld.

 \bibitem{chap3-key27} Raju, P. T. (1962). \textit{Introduction to Comparative Philosophy}. Nebraska: University Of Nebraska Press.

 \bibitem{chap3-key28} Smith, Huston. (1991). \textit{The World’s Religions}. San Francisco: Harper Collins.

 \bibitem{chap3-key29} Saraswati, Niranjanananda Swami. (2008). \textit{Samkhya Darshan: Yogic Perspective on Theories of Realism}. Munger: Yoga Publications Trust.

 \bibitem{chap3-key30} Saraswati, Sarvanand Swami. (1994). “Parabrahma Aur Śūnyatā: Māyā Se Riktatā”. \textit{Hindū Tathā Bauddha Dharm: Samān Ādhār}. New Delhi: Vishwa Bauddh Sanskritik Pratisthan.

 \bibitem{chap3-key31} Staal, Frits. (1996). \textit{Ritual and Mantras: Rules Without Meaning}. New Delhi: Motilal Banarsidass.

 \bibitem{chap3-key32} Upadhyay, Govind Prasad. (1979). \textit{Brahmanas in Ancient India}. New Delhi: Munshiram Manoharlal.

 \bibitem{chap3-key33} Vidyabhusana, Satis Chandra. (1988). \textit{A History of Indian Logic, Ancient, Mediaeval and Modern Schools}. New Delhi: Motilal Banarsidass.

 \bibitem{chap3-key34} Wallerstein, Immanuel M. (2006). \textit{European Universalism: the Rhetoric of Power}. New York: The New Press.

 \bibitem{chap3-key35} Wallerstein, Immanuel M. (2004). \textit{World-systems Analysis: An Introduction}. Durham: Duke University Press.

 \bibitem{chap3-key36} Weber, Edmund. (2001). “Buddhism: An Atheistic and Anti-Caste Religion?”. \textit{Journal of Religious Culture / Journal für Religionskultur No. 50.}

 \end{thebibliography}

\theendnotes

