
\chapter{Origin and Dispersal of Iron in India}

The period around the closing centuries of the 2nd millennium BCE witnessed a changing configurations due to the emergence and adaptation of newer technologies. The technologies provided incentive and the infrastructure for material prosperity and over all growth. The innovative changes set the pace for a relatively vigorous socio-economic and techno-cultural change. One of the important technological innovations that took place during the period spanning over the 2nd-1st millennium BCE was the advent and development of iron metallurgy in the Indian subcontinent. In the large and diverse eco-zones of Indian subcontinent, iron technology made appearance in divergent contexts and in different ways. The exact time and circumstances of the introduction and adaptation of iron in has been investigated at different levels. In regions closer to the raw material experiments could have started easily if need for metal was felt. In other areas, the circumstances differed. Therefore, a different approach to understanding of advent and adaptation of emerging technologies has to be adopted. It is desirable to address the issue of beginning of iron in India in different eco-zones which were not easily accessible due to geographical barriers. We will attempt to do so in course of discussions that ensue. Each zone may present us with divergent chronological framework and adaptation pattern. New evidences are being brought forth by excavations being carried out every year. Radiometric characterizations from excavated sites are pushing back the antiquity of iron. In view of fresh light being thrown by issue of iron technology, the related issues have to be addressed afresh while taking cognizance of the earlier viewpoints expressed by historians/ archaeo-metallurgists from time to time. Broadly speaking, there are two main viewpoints on origin of iron in the Indian subcontinent, that is, 1- diffusion of iron from outside; 2- indigenous origin of iron. We propose to examine both the viewpoints in some detail here. First we take a look at the diffusionistic theory of advent of iron in India. Three basic points were raised in this regard:

(1) Metallurgists strongly believed that iron technology is too complex to be learnt independently1 (Forbes 1950). It had to be learnt and perfected under the guidance of artisans who were well adept in the process of iron working. (2) It was strongly believed that Aryans had entered India with horses and superior weapons (iron?) overpowered the indigenous inhabitants and occupied the land of seven rivers \textit{(saptasaindhav desh).} There is reference to \textit{ayas} in the Rigveda, the earliest Aryan treatise.

(3) Iron was introduced by the Greeks or Bactrians as they were present in north-west India. This view was proposed on discovery of iron from sites like Taxila in the early parts of the 20th century the British archaeologists who were at the helm of affairs at the time. Since the Bactrians and Greeks had ruled the North-western part of the subcontinent during 5th-4th centuries BCE and iron was found in those strata, it reinforced the assumption that iron was brought in by them (Gordon 1958: 50-78). Similarly, Sir Mortimer Wheeler (1958) recovered iron artifacts from excavations of megalithic sites like Brahmagiri, Maski etc. Interestingly, as per the norm and the colonial temper, both the archaeologists had ruled out the possibility of use of iron in 

the subcontinent prior to 600-500 BCE. These views have to be examined closely in light of recent researches.

The other set of scholars worked on literary evidences on iron. Scholars like Neogi (1914) and M.N. Banerji (1929) who had gone into the rich literary accounts of India chose to closely examine the occurrence of ayas which meant iron in Sanskrit. Reference to \textit{ayas} in Early texts, including the Rigveda led to assumption that antiquity of iron coincides with early Vedic period. It has been argued that the Rigvedic Aryans were well conversant with iron(\textit{ayas}) technology wherein one comes across tools and implements made with \textit{ayas.} The question is what did Vedic \textit{ayas} stand for? The term \textit{ayas}, it seems had different connotations. More in-depth researches show that the term \textit{ayas} is a generic term used for metal. As we will see below, intense debate followed on the etymology of word ayas and whether it stood for iron in the Early Vedic texts.

Lallanji Gopal (1961) synthesized the existing literary data and came to the conclusion that the word \textit{ayas }in the \textit{Rigveda}, stood for \textit{metal }in general, not specifically for iron. According to him iron was introduced in India during the Later Vedic times. This interpretation of the word \textit{ayas }at the earliest stage is significant indeed to the issue of introduction of iron in India. A detailed analysis of the prevalent views on the advent of iron in India is called for here. In recent years (as discussed in chapter II), scholars have taken a second look at the diffusionistic viewpoint of introduction of iron due to the fact that iron was more probably a by-product of copper or lead metallurgy. However, we would first take up the theories of diffusion of iron technology followed by the alternative viewpoints.

The theory of diffusion took roots because it was believed by a set of scholars, especially Indologists that the Aryans migrated from a common homeland. It was argued that the Aryans dispersed and settled in different parts of the world in course of their move; this included the Indian subcontinent. This has given rise to theory of diffusion of people and / or ideas. This includes advent of iron technology in India. It is another matter, however that the idea of Rigvedic Aryans being outsiders has now been contested by scholars on the basis of researches being carried out in the field of genetics, especially DNA of population from the concerned regions. However, following the earlier theories propagated regarding the Aryan immigration, we would first take a look at the diffusionistic viewpoint of origin of iron in India.

\textbf{1. THE THEORY OF DIFFUSION OF TECHNOLOGY}

The philological evidence suggesting similarities between Indo-European languages and Sanskrit, the common features noticeable in the Iranian sacred text the \textit{Avesta} and the \textit{Rigveda}, the oldest Ayran text and the inscriptional evidence like the Boghaz Keui (dated 1365 BCE, a treaty between Hittite king Shubbiluliuma and the Mitanni King Mattiuaza, referring to four Vedic gods) have given rise to a theory of common homeland of authors of these cultures. The linguistic affinity had gained cultural attributes in due course of time. In the specific context of introduction of iron technology in India, the theory of Aryan immigration from the west, though debated has come in handy. It is believed that iron, which was known to the people of Asia Minor in the 2nd millennium BCE, entered the Indian sub-continent with the immigrating Aryans. As examined in detail above (also see Tripathi 2001: 57-78), there are indisputable literary references to the knowledge and use of 

iron in the pre-1200 BCE period in Mitanni, Hittite and Egyptian records. The king of Mitanni, Tushratta sent the Egyptian Pharaoh, a dagger with iron blade with lapis lazuli studded gold handle (using 14 `shekels' of gold) (quoted by Maddin 1982: 16).

Another reference belonging to the early 13th century BCE narrates the contents of a letter of a Hittite king Hattusilius III to the king of Assyria, mentioned above, ``As for the good iron about which you wrote to me, there is no good iron in my store house in Kizzuwatna. The iron (ore?) is (of) too low (a grade) for smelting. I have given orders and they are (now) smelting good iron (ores?). But up till now they have not finished, I shall send (it) to you. Meanwhile I am sending to you a blade of iron for a dagger.''(Maddin op cit, 16-17). These passages are sufficient to prove that iron was a precious, prized and naturally also a scarce commodity at this stage. The metal however was being smelted in small amounts sporadically in some parts of the ancient world. The Hittites kept the knowledge of iron working a closely guarded secret, confining the art to the Anatolian Plateau till about 1200 BCE. Their monopoly was broken by disruption of the Hittite Empire by the Thraco-Phrygian invaders who forced the former to migrate to the peripheries of the Assyrian Empire. Around this very time the use of iron objects multiplies and the world witnessed large-scale migrations by warrior tribes with superior weapons, horses and horse-drawn chariots.

In close succession to this incident (\textit{c.} 1000 BCE), Ghirshman (1954: 73) observed two perceptible phenomena in the Iranian plateau: The invasion of the Indo-Europeans and the increased use of iron in Iran. In this region, iron was first noticed in the necropolis of Sialk, Cemetery-A along with a new grey ceramic. Iron, however was restricted to just a few pieces as 

part of royal costume being exclusively ornamental in nature. It assumed a utilitarian role only in the succeeding phase in Cemetery-B. The number of objects increased considerably from a couple of objects in previous period to a noticeably large number as well as a diversified typology reflected in a variety of objects like swords, daggers, shields, javelins, arrowheads, horse bits, head and chest ornaments of horses along with utensils and ornaments like anklets. This has been interpreted as an incursion of new cultural elements in Iran introduced from farther west or north.

In western Iran, Young (1976) defined a three-fold stratification of emergence of iron, calling them Iron I, Iron II and Iron III. Iron makes its earliest appearance by way of stray occurrences primarily as bi-metallic objects in the pre-1000 BCE period, classified as Iron I. Its presence is restricted to its use by a selected few of the society. A grey pottery is associated with this cultural phase impinging on the local Bronze Age cultures. Young held that there were widespread migrations in the western part of the Iranian Plateau in this period. There were new pottery traditions and also change in the pattern of use of metals. Iron was introduced here at this stage by itinerant metal smiths or through some kind of trade. Young cites the parallel of a known situation of influx of gypsies in certain parts of the world.

Interestingly, there is a gap between Iron I and Iron II according to Young. Iron I has been dated between \textit{c.} 1300/1250-1100 BCE. Iron II, according to Young is datable roughly to about 1000-800 BCE. Geographically, India is close to the Iranian borders. It is likely, that there were intermittent folk movements into India through this region. The inherent similarity between Avesta and Rigveda corroborates familiarity 

between the two neighbouring people. This has given rise to the assumption that iron was brought in India by these incoming tribes (the Aryans?) through Iran. (It is noteworthy that there is a near absence of iron in Iron I level dated between 1300-1100; it means that they were not the people who could introduce iron in India). Recently, D.P. Agrawal (1998, 2002) has underlined the – linguistic as well as geographical and migratory pattern of movement of people into the Central Himalayas.Agrawal hasrelated this to the beginningof iron in this part. The presence of cist burials with use of iron in Kumaon region is significant in this respect. The presence of the Caucasoid racial elements in the population of Central Himalayan region is said to be a noteworthy feature in this regard.

Similar evidence of burials of skeletal remains with Caucasoid features has also been brought to light in China recently. The richness of iron ore and presence of iron working tribes from time immemorial give credence to the hypothesis of diffusion of people in this region. Iron has been shown to be present in about 1000-800 BCE in Kumaon – Garhwal region. There exists a possibility of migrations and technology transfer through the immigrating folk at least in these Himalayan zones (Agrawal and Tripathi 1995, Agrawal and Kharakwal 1998). Future research may throw more light on it. With this, we may go back to the issue of iron in \textit{Rigveda}. To evaluate the hypothesis of introduction of iron in India with incoming Aryans through the Iranian plateau, we need to examine in depth- a) the evidence of iron in \textit{Rigveda}; b) the cultural material including iron objects in Indo-Iranian borders for similarities that followed to substantiate the said intrusions by the Aryans in the region.

\textbf{1. I. Iron in \textit{Rigveda}}

The composers of \textit{Rigveda }call themselves Arya-the noble one, a fine or superior people. It is well known and needs no elaborations here that there was some kinship bond between them and the people of the \textit{Avesta} and that the two separated at some point of time. There is no definite evidence to suggest that the Aryans who lived in India in the `\textit{sapta saindhav desha' } (the land of seven rivers) came from outside nor they had prior knowledge of iron technology and that they brought iron weapons with them. We do come across the word `\textit{ayas'} which stands for iron in later Vedic times. But we are faced with several problems here, such as: what was the original connotation of the word `\textit{ayas'}? Did the Rgvedic `\textit{ayas'} stand for iron? It is relevant for the present discussion. Scholars have examined it time and again and various interpretations are available to us today. M.N. Banerji (1927, 1929, 1932), N.R. Banerjee (1965), Gopal (1961), Roy (1984), Tripathi (1994, 1997) etc. have taken a close look and have offered interpretations of the word \textit{ayas}. M.N. Banerji, N.R. Banerjee and S.N. Roy have come to the conclusion that the Rgvedic `\textit{ayas'} stands for iron; consequent to this they conclude that iron in India is an outcome of India's relations with the Aryan tribes coming in India from the west. On the other hand, Gopal and the present author argue that in the initial stage the term `\textit{ayas'} stood for metal in general. The author also feels that adjectives \textit{Krishna } or \textit{Shyama} and \textit{Lohita }were prefixed to \textit{ayas} at a relatively later stage in \textit{Yajurveda samhita} with the beginning of iron to distinguish the 'black metal' (\textit{Krishna ayas} or \textit{Shyama ayas }= iron) from the red metal (`\textit{Lohita}\textit{ayas} `= copper). As stated above, it is more likely that the word `\textit{ayas'} was a generic term for metal interchangeably used to denote iron or copper till much later date as is clear from literary sources. If so, one wonders whether the Early Aryans, even if they came from outside had knowledge of iron.

A quick look at the precise context of occurrence of the word \textit{ayas }in Rigveda may be useful here. There are ten such references in Rigveda. In the verse I.CLXIII.9 there is a description of horse of god Indra whose colour is like \textit{ayas}, which shines, in sharp contrast to the golden rays of the sun. In Rg. X 99.8 the word is used in the sense of physical strength \textit{ayopasti} – strong powerful hands like one made of sharp `\textit{ayas'}. The other two references are to (X 79.6) \textit{asi} or axe with sharp cutting edge and (I, 162, 20) a sword or dagger having an excellent edge. Rigveda X 112.2 and X 72.2 refer to the artisan \textit{(Karmar)} working with `\textit{ayas'} using bellows. M.N. Banerjee thinks that it is only the ironworker who uses bellows and therefore these verses specifically refer to ironworkers. Rigveda VI 3.4 describes molten metal (`\textit{dravi'}-liquified). The next reference X 31.3 uses the word `\textit{damdhamant}'or `\textit{samdamanti'} interpreted as `inspiring' or `leading to' by Sayana, the commentator. Griffith translates it in the sense of welding. M.N. Banerjee thinks that it indicates joining two pieces of metal after heating them.

If we take a close look at these contexts most of them appear to be more suitable for copper than iron. The earliest reference is only indicative of the colour. It may be interpreted both ways – red or black. Similarly, hands or sharp nails could be strong like copper-bronze or iron both. We have come across axes or razor blades of copper-bronze in early cultures right from early Harappan and Chalcolithic cultures. On the other hand, swords or daggers of iron appear much later though axes have been found.

It is not correct to assume that metallic iron, especially the wrought iron of the earliest phases, was stronger and better suited for weapons and other sharp edged tools. We have noted before (chapter II) that good bronze, 

which has undergone through the process of annealing, has the strength of 1,20,000 P.S.I. (per square inch) while a wrought iron (with no carbon) has the strength of only 40,000 P.S.I. We know that the early iron found in India was wrought iron having no carbon in it. Thus technologically also the strong, sharp weapons of Rigvedic people were more likely to be that of copper–bronze than of iron, as indicated in similes and metaphors in the literature. This gets further corroboration a reference in the Rigveda (VI 3.4), which clearly narrates about to liquefaction of metal ayas. We know for definite that it was copper or bronze, which was liquefied. It reached a molten state at the time of smelting as well as casting of objects. Iron, on the other hand was produced in semi-solid state in the form of a bloom which had to be hammered and treated repeatedly to extricate the trapped residue in the form of slag and consolidate the iron into a homogenised metallic ingot. Interestingly, all the above references except for two are from the late sections of Rigveda. Thus technologically speaking, the passages making a mention of `\textit{ayas' }are more likely to be suggesting copper-bronze than iron. If iron was not known to the Rigvedic Aryans, it would be difficult to sustain the argument that the knowledge of iron was acquired from outside from the brethren who occupied the distant lands outside the `s\textit{apta sandhava desha'.} It will be in fitness of things to look for evidence of iron on the borders of India to see if chronologically there is a case of early occurrence of iron so as to influence transfer of technology to India through that route.

\textbf{1.II. ARCHAEOLOGICAL EVIDENCE OF IRON ON THE BORDERLANDS}

Archaeologically, the area adjacent to Iranian borderlands, modern Baluchistan (extending over Indo-Iranian plateau) has yielded a large number of cairn burials. Stein (1929) has reported as many as 5100 cairns. Many of these cairns have yielded iron objects along with copper-bronze objects and other cultural material along with pottery. Gordon (1950) suggested Iranian connections of Sialk Cementery B and the cairn burials of Baluchistan on the basis of similarities in pottery, burials and the metal objects. However, Lamberg-Karlovsky and Humphries (1968) disapprove of the `Sialk B connections' or Indo-European movements to east' towards the cairn burials of Baluchistan because of lack of `convincing parallels'. The ecology also plays a role in isolating this area as the ``natural barriers of mountain desert in Baluchistan and southeast Iran have isolated the inhabitants from the domination of any neighbouring power in the 20th century AD. Thus, it seems likely that the occupants of Baluchistan, separated from both east and west, always maintained a relatively independent existence'' (Lamberg-Karlovsky and Humphries, 1968). The distinctive painted pottery types could not readily be related to the Iranian Plateau or to the painted pottery tradition further to the east. Talking of the possible areas exerting their influence on the Dashtiari and interior Baluchistan, one must look-first to the Persian Gulf trading areas as an outside source of contact. Secondly, there is a connection among the cultures of the northwest India area. The Iranian plateau is an un-distinguished third''.

A close comparison of chronology, typology and pottery traditions of Baluchi cairns and that of North India tends to lend weight to the contention ofLamberg Karlovsky and Humphries (quoted above). The burden of archaeological evidence does not favour the thesis of diffusion of iron into 

India from the neighbouring West Asian and Central Asian countries. Firstly, a closer examination of tool typology in Iranian and Afghan sites and those in Sindh and Baluchistan area display little common features with iron objects of mainland India. Secondly, the cultural material corroborates the typological study, i.e. the two areas appear to be culturally distinct. Thirdly, the chronological considerations go against any notion ofdiffusion. On Iran-Afghan sites as well as Indian North-west, iron emerges more or less simultaneously \textit{viz}. around 1100-1000 BCE. In recent years, the Swat-Gandhar region has yielded C14 dates going back to circa 1200 BCE.\\ It may, however be noted that recent 14C dates from the middle Ganga Plain sites are much earlier than this.

\item 
III. \textbf{IRON IN CHINA}


Having examined the circumstances and processes of the advent of iron in the neighbouring countries on the western sides of India, it may be proper now to look in the opposite direction. On the eastern and north-eastern neighbourhood of India is situated China. There is a large mountainous zone of the Himalayas connecting the two countries. There is definite evidence of the existence of interaction between India and China, right from the Neolithic times. There are clear indications of cultural exchange in the sites like Burzahom and Gufkral in the Kashmir Neolithic culture. Presence of jade beads, ring stones and axes with holes at both ends, burial of a dog are some definite proofs substantiating the contacts between these two regions. At the site of Gufkral iron appears for the first time in the post-Neolithic, megalithic phase. The earliest iron there comes from the burials dated by the TL dates to 1800-1700 BCE. In the very next phase (without any break) iron shows up in the megalithic burial culture at Gufkral 

that has been dated by the excavator to 1400 BCE. Recently 14C dates take back the antiquity of iron bearing levels to 1900 BCE (uncelebrated date). Such early date of iron bearing deposit is significant indeed. It may be worth examining the evidence of the earliest occurrence of iron in China with a view to examine the genesis of iron technology in this region.

China has a unique position in the Old World civilizations, being the only place where cast iron comes to be used at an early age. Cast iron was hardly used in Mesopotamia, Anatolia or in the Aegean or central Europe; it is apparently an invention of the ancient Chinese metal craftsmen. Though metallurgists like Cyril Stanley Smith do not quite support the contention of independent origin of metallurgy including that of China, "this by no means denies that there is something uniquely Chinese in the technique and the beauty of the superb bronzes of Shang and Chou China, but this seems to lie in the making of moulds and the decoration on them, not necessarily in the origins of metallurgy itself" (Smith 1973: 23).

One agrees with Smith that the Chinese metalworkers were ingenious in moulding beautiful bronze artefacts; there is every chance that using the same ingenuity they could master iron casting. As opposed to Smith's contention, Renfrew has (1986: 145) argued, ``It has effectively been demonstrated that metallurgy in the new world had separate origins and developments from those in the Old World. It is even likely that metallurgy in China developed independently from that in West Asia. It has been argued on several occasions that there is a good case for the independent origin of metallurgy in South-East Europe. The same is true of Iberia". Thus origin of metallurgy is no longer taken to be an outcome of diffusion of people or 

ideas as was earlier believed to be. This is true in the case of China, as argued by Tylecote (1992: 75),

``... China was to be the one country where iron technology was to take, a different course; towards cast iron rather than wrought iron, and it is therefore possible that ferrous metallurgy evolved independently from the highly sophisticated non-ferrous techniques of the region. The same sort of accident that produced recognizable ductile iron at the bottom of a copper smelting furnace could, under certain conditions, have produced cast iron and the Chinese would have realized that they had found a bronze substitute".

This view of Tylecote has been corroborated by the tool typology even at the early levels of occurrence of iron in China. Jueming also supported Tylecote saying that iron in China is an outcome of bronze metallurgy around 700-600 BCE. At the earliest stage balls were found at Liuhe and cauldron at Changsha in around 600 BCE. Iron soon replaced copper-bronze and even stone. A rich variety of commodities like ploughshare, spade, sickle, axe, adze, chisels, arrows, cloth-hooks etc. were discovered. Even moulds used in manufacture of iron objects were found at sites like Xinlung. The technology developed in due course of time for producing excellent quality malleable iron. To quote Hua Jueming,

``..we can conclude without any exaggeration that without the wide application of the toughening, it would have been impossible for cast iron to be so widely used during the Warring States. It was precisely this wonderful invention that played a historical role in the advancing of early Chinese Iron Age. This technique was perfected during the Han dynasty at the foundry at Tienshangou, Gongxian. On analysis the structure represents typical spherical graphite. (It may be noted that this technology was invented in Europe by J. H. Morrogh, as late as 1947)".

For the discovery of cast iron in China Jueming credits the highly developed bronze industry with efficient furnace design (Muozi advises, 

"each stone should be equipped with four bellows"), mechanical ventilators and selection of good refractory material. The incidence of cast iron could be possible in the furnaces generating high temperature due to strong draughts. Grey iron was obtained in the western Han and also in a foundry at Zhengzhou in eastern Han. These furnaces were 5 to 6 meters high having 50 cubic meter capacities. In certain areas blast systems were driven by water – the 'Shuipai' or hydraulic expeller.

"Thus, cast iron is the foundation of Chinese iron metallurgy. Its early invention and wide application created a specific style of its own with the invention of puddling iron in western Han and soon afterwards the uniquely Chinese guan steel, China firmly established its specific iron and steel system".

The 50 tonne lion statue is the product of this technique – a manifestation of an extraordinary skill! Even if there was an early occurrence of iron in China, it did not pass the know-how to India as the two regions had quite distinct metallurgical traits.

This brief discussion of the advent of Chinese iron in antiquity makes it clear that (1) Chinese iron was in all probability an independent invention of the bronze and copper smiths; (2) though both wrought iron and cast iron were in use at the early levels, it is the latter which gained popularity in China; (3) they were the only ancient civilization of the world to cast colossal statues like a 50 ton lion figure; (4) and last but not least, both in time and technique, there is little similarity in the iron production between India and China; (5) Iron had a longer antiquity in India going back to 1800-1700 BCE while it started in the opening centuries of the first millennium BCE in China. India never adopted the cast iron technique of iron making. Therefore, it is not likely that introduction of iron in India was in any way 

influenced by China or its neighbourhood. On the contrary, there exists a strong case of diffusion of know how from India to China.

The following points that emerge from the foregoing discussion are significant:

\item 
In the Iranian plateau at the earliest stage iron was very scarce and was confined to graves only. Importantly, there was also a gap between this stage and the next stage. An uninterrupted use of iron starts only around 1100-1000 BCE.

\item 
In the neighbouring regions of the Indian borders, none of the areas appear to be in a position to pass on knowledge of iron metallurgy to India. Chronologically or typologically these regions are distinct and disparate.

\item 
The Rigvedic society does not seem to possess iron technology therefore their interaction with Avestan or other brethren is ruled out as source of iron technology.

\item 
It is well established now that iron metallurgy was a by-product of copper or lead working.

\item 
If iron did not reach India through the western source, we may now explore the possibility of an indigenous origin of iron in India. As ^14C dates from recent excavations push back the antiquity of iron in India, we need to take a fresh look at the alternative viewpoint of beginning of iron in India.


\textbf{2. INDIGENOUS ORIGIN OF IRON IN INDIA }

In recent years interesting evidence of early use of iron warrants fresh thinking on the antiquity of iron in India. The balance seems to tilt in favour of an older and indigenous origin of iron. It is therefore necessary to examine afresh the evidence available to us from the latest archeological 

researches going on in India. The maps (Fig. 1, 2 and 4) show the important Iron Age cultures and different cultural zones in which early iron sites have been discovered. In addition to the archaeological sites, number of ethnological evidences have been brought to light time to time (Fig. 3).

\textbf{2.I. Iron in the Chalcolithic Milieu}

In majority of cases iron in India makes its earliest appearance in a Chalcolithic setting. However, the incidence of iron at Ahar, (district Udaipur, Rajasthan) a Chalcolithic site, almost revolutionized the theories of advent of iron in India in early seventies. M.D.N. Sahi (1979) pointed out the anomalies in the report of Ahar (Sankalia \textit{et al}. 1969). According to the original report of Ahar iron was said to have started with NBP (Period II) around 600-300 BCE in this part of Rajasthan. Period I having three subdivisions – Ia, Ib and Ic, which belonged to the Chalcolithic cultural phase. It was however noticed by Sahi that iron pieces were present in the tool repertoire of the earlier period itself. According to Sahi's observation two iron objects were reported to have come from period IB and 10 from IC. Period IC at Ahar has been dated by 14C between 1270-1550 BCE. An analogous situation may be seen at Eran in district Saugar, M.P.as will be shown below.

Although, no categorical refutation has come from the excavators of Ahar, it is stated, however that the trenches were laid down on a sloping hill at Ahar. Under such a situation there is always a possibility that material from upper layers got re-deposited at a lower level. It may be worth mentioning here that the site of Ahar lies almost within an ore rich belt of chalcopyrite. There is also evidence of extensive metal working by way of slag heaps and furnace remains all over the site. An in-depth examination of 

these remains will go a long way in understanding of metallurgy as and when and how it developed there.

At Eran in Saugar district of Madhya Pradesh also there is evidence of occurrence of early iron. However, later excavations have exposed a mixing up of material on the basis of which K.K. Tripathi (1995) suggested a gap between Chalcolithic and iron yielding layers. He has reiterated the 6th –7th century BCE date for the latter. A more definite evidence of iron from the Chalcolithic level has come forth from sites in Bengal, Bihar and UP.

As reported earlier some of the copper objects at Chalcolithic sites contain as high as 32.9\% iron in an axe at Rewari, 20.6\% and 25.8\% in another axe at Bhiwani and 7\% at Ahar.This has raised the question of simultaneous reduction of copper and iron during smelting, and that the iron smelting technology has emerged from the (in) experience of the metalworkers of the Chalcolithic period. The feasibility of co-reduction of CuO and Fe_2O_3 or Cu and Fe from chalcopyrite has been examined by scholars in recent years.

These studies explain the optimum conditions for the production of copper and possibly also iron from mixed oxide or sulphide ores. For actual production higher temperatures have to be used. The output of metal during smelting is determined by rate kinetics, which depends upon several other factors, as shown in the graph, below:

The presence of iron in ancient copper objects is important to the issue of recognition and beginning of use of iron. Several ancient civilizations, it has been reported that copper/bronze objects contained certain percentage of iron in the matrix of the metal. In India, as mentioned earlier such objects 

have been reported from several Harappan and Chalcolithic sites, (Tripathi 2001: 67-71).

More recently, a good amount of research has been done to explore the possibilities of simultaneous reduction of iron with copper under certain conditions. Wertime (1980: 16) commented,

"---the earliest smelting must not be seen as a conscious selection of a pure ore but as the experimental employment of an ore mix which could alternatively yield lead, copper or iron or a mixture of all three".

However, if the temperature in the furnace reaches a higher level under 'extreme reducing atmosphere, small bits of relatively pure iron would have been produced', concluded Moorey (1994: 279) on the basis of researches done by Cooke and Aschenprenner (1975, quoted by Moorey, op.cit). It is said to be a consequence of the use of heterogeneous sulphide ore or use of iron as flux in copper smelting. Moorey (op.cit, p. 280) also states, "In certain circumstances small nodules of iron could have been separated from the contents of copper smelting furnace".

The study of many ancient period artifacts of copper from several Chalcolithic sites like Ahar prove beyond doubt the skill of the metal smelters in producing copper from chalcopyrite. The possibility that they incidentally produced magnetic copper cannot be ruled out (see Tripathi 1986: 78, 2001: 67-70, Table 1-3). This also indicates that the Chalcolithic metal smelters had the knowledge of the use of iron ore as flux for smelting of copper.

The crucial question is whether these copper smelters were aware of the presence of iron in copper and also that they were producing Cu-Fe alloy i.e., magnetic copper instead of pure copper. We have not studied the recent 

iron artifacts from the sites of the Ganga Plain to be able to comment whether the copper smelters of this region were aware of the presence of copper in iron ore. Such knowledge could lead to experimentation and understanding of iron smelting technology in due course of time. Alternatively, it is also possible that the metal smiths of the Gangetic plains used limonite or hematite ore, which was more easily available in the Vindhyan ranges. It is quite likely that due to scarcity of copper ore in this region, the artisans experimented with alternative ores and accidentally produced iron. It may be noted that copper objects are very rare in this region in the so-called Chalcolithic period. Bone and stone have been used profusely for tool making with a lower proficiency in dealing with copper working; there is a higher possibility that the artisans picked up iron ores and tried to work with them. In this process of trial and error, they could have accidentally produced iron.

\delimiter

