
\chapter*{Acknowledgements to the Second Edition}

I am grateful to the Infinity Foundation for initiating a project on the History of Science and Technology in India and providing us the opportunity to be a part of this ongoing project. When complete, I hope it will fill in the lacunae in our knowledge about the glorious past that India has had in the field of Science and Technology. First and foremost, I would like to express my appreciation to Shri Rajiv Malhotra and Prof. D.P. Agrawal for initiating a multidimensional study on history of Indian Science and Technology, a subject which has not received due attention of scholars, so far.

I am indeed thankful to Shri Rajiv Malhotra and Prof. D.P. Agrawal for associating me with this ambitious project and inviting me to contribute a volume on the subject of Iron Technology. It is known now that Indians excelled especially in the field of iron and zinc metallurgy. Zinc distillation is India's contribution to the world. Likewise, Indian iron and steel had created a niche for itself in the ancient world. However, a detailed history of iron technology in India has not been available to scholarly world; hence this book. During preparation of this volume, valuable suggestions were offered to me in course of the brain storming sessions that were taking place to discuss different aspects of science and technology when the project was being muted. I most humbly acknowledge the valuable suggestions offered by scholars during the initial sessions. The enriching suggestions gave direction to conceptualize the work to be undertaken.

History of technology is a subject that requires multi-disciplinary approach. The present work on Archaeo-metallurgy of iron has to draw from researches carried out in the fields of archaeology, history and metallurgy. It may not be possible to individually name each one of them. I take this opportunity to acknowledge and express my indebtedness to all the scholars and researchers whose works I have used here as resource material.

Archaeology is an ever growing discipline. New evidences are unearthed year after year adding to the existing mass of data. Unless writings on the subject are constantly revised and updated, works easily become obsolete. A few weeks back when Dr. O.C. Handa, the dynamic `Chief Editor' of these volumes who has taken over from Prof. D.P. Agrawal contacted me to say that the Infinity Foundation volumes are being reprinted, I was overjoyed. This was primarily because it gives me a chance to include the illustrations which were, somehow not included in the volume printed by Rupa. In the intervening period of nearly a decade (2008 to 2019), new excavations have been conducted shedding new light on iron technology in the past. Publishing a volume after so many years and not incorporating the new evidences may not be appropriate.

In this connection, I will like to make a special mention of an opportunity which came my way further pursue my interest in the field of archaeo-metallurgy. I was fortunate to receive an award to work at Rutherford Appleton Laboratory, Oxford facilitating in-depth analytical work on heritage of Indian iron metallurgy. For this opportunity I wish to express my sincere thanks to the Jawaharlal Nehru Centre for Advance Scientific Research (JNCASR), Bangalore and STFC Rutherford Appleton Laboratory. In this research project undertaken in November, 2017 at ISIS, RAL, my co-investigators, Evelyn Godfrey and Prabhakar Upadhyay were actively involved. Together, we could analyse about twenty two iron samples from sites excavated by us in the Department of Ancient India History, Culture \& Archaeology, B.H.U. Varanasi. We utilized INES and CHRONUS facilities available at RAL. The results of the investigations were revealing in many ways.. I have been able to incorporate some of the results of the study in the present edition of the volume. I am thankful to both my co-investigators and the scientists who helped us at RAL. I am also thankful to the Archaeological Survey of India for granting me permission to take the iron samples abroad for conducting analytical study.

My colleagues and students helped me in several ways in preparation and revision of the manuscript for which I am indebted to them. Without such unconditional help of friends, colleagues and students, it would not have been possible to complete the work. Last but not the least; I would like to admit that the work could not have been completed without the loving support of Ajit who inspires me to contribute to the field of knowledge to my utmost capability.

\begin{flushright}
\textbf{Vibha Tripathi}\\ September, 2019
\end{flushright}

