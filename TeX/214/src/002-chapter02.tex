
\chapter{Iron in Bronze Age Cultures of Old World}\label{chapter1}

Iron made its first appearance in the Old World during the Bronze Age. It appears to be an incidental discovery made inadvertently by the early metal workers. At the same time, we may presume it to be an outcome of skilfull interplay of circumstances and innovative spirit inherent in humankind. Developments in the field of pyro-technology may be attributed to control and manipulate fire as a source of energy. With experience, it became possible for the metal workers to generate higher temperature in furnaces used for metal extraction. More efficacious furnaces and longer experience of working with different kinds of ores eventually led to the production of metallic iron, by stroke of chance at first and with conscious effort, subsequently. It may be worth examining the process of advent of iron in ancient world civilizations to trace the emergence ofiron technology in India.

Some important issues that may be raised in the context of developments in the field of metallurgy, especially in iron metallurgy are: When and where was iron recognised for the first time as a metal? Did the meteoritic iron found at early stages in certain ancient civilizations lead to recognition of iron as a distinct metal? Did it lead to subsequent experiments with available minerals to produce metallic iron? Did the metallurgy so learnt, spread to different parts of the world from such centres? Or alternatively, was it a consequence of efforts made by artisans in different parts of the ancient world? What was the plausible picture of advent of iron in the Indian subcontinent? These are some of the issues that we propose to briefly examine here in the early context of occurrence of iron in the ancient world civilizations.

Iron first makes its appearance in the Bronze Age cultures of Mesopotamia. Whether the earliest iron was smelted or was of meteoritic origin has a bearing on the present discussion. We come across expressions like ‘iron from heaven’ in several texts of the ancient world. This appears to be referring to meteoritic iron. Earlier, only high nickel content was considered to be a deciding factor for identification of meteorites. But recent researchers do not consider this to be a sufficient criterion; in addition to high nickel, it should also have Widmannstatten structure. “The presence of nickel - which will anyway be reduced in the process of smelting nickel-rich terrestrial iron ores - in an iron artefact can only imply a meteoric origin if the original Widmannstatten structure is also present” (Moorey 1994: 279). In view of this, several so called ancient iron objects datable to as early as 5000-4000 BCE need to be re-examined. The meteoritic iron has nickel content; it shows larger grains and also Widmannstatten pattern. The most characteristic feature of such iron however, is its large grain size.

The study of ancient iron by Piaskowski (1988) has thrown a welcome light on the problem of identification of type of iron in ancient collections. His study has demonstrated that iron objects earlier believed to be of meteoritic origin could simply be smelted iron. He cites the case of Chalybean iron from northern Anatolia. It was a high nickel iron therefore it was said to be meteoritic in origin. Instead, his analysis shows it to be produced with iron-bearing river sand. It was more beautiful and did not rust (perhaps due to the nickel content). More such specimens need to be analysed to ascertain the nature and type of iron used much earlier on in ancient civilizations.

For a better understanding of emergence of iron, the circumstances in whichthe earliest iron appeared need to be examined. The earliest literary reference to smelted iron comes from the Hittite civilization (1700 to 1200 BCE).

\begin{myquote}
“…the dolerite they brought from the earth. The black iron of heaven they brought from heaven. Copper (and) bronze they brought from Mt. Taggata in Alasia (Cyprus)…”. The reference belongs to a ritual for building a house and clearly mentions both iron and black iron – one from earth, the other from heaven, which is also black because of the magnetic coating produced by heat (Bjorkman 1973). We need to examine as to how, when and under what circumstances the earliest iron was recognized and produced.
\end{myquote}

Wertime (1980: 13) in this context comments, “…that the working of iron was an inevitable technical byproduct of copper and lead smelting anticipated by the pyrotechnologic uses of iron…”. Supporting hiscontention, Moorey (1994: 279) states “… it is increasingly thought likely that a significant proportion of the iron current in the Near East, at least before the Late Bronze Age, might have been produced in the process of smelting copper, or possibly lead”. The basis for such an assertion by Wertime is a discovery of an actual smelting furnace by him and Cyril Stanley Smith in 1962. Smith’s comment quoted by Wertime (1980: 14) in this context is significant indeed. “...since lead smelting preceded iron by many centuries, the smelting of iron could accidentally have been discovered in a lead smelting furnace and later developed for its own sake". This is the most convincing hypothesis regarding the origins of iron yet advanced. It has also been reiterated by Percy (1861: 423, 431, in Wertime 1980: 14).

\begin{myquote}
“… It is also possible that iron may have originated in copper smelting, for copper smelting furnaces also produce bear (iron) under certain conditions”. This was also suggested by Percy (1864: 873) who believed that iron was much easier to produce than bronze and that iron was an unpredictable metal.
\end{myquote}

Wertime, Pleiner and Smith conducted experiments to produce iron under similar conditions. ‘Iron thus smelted would have a great deal of copper in it and it would be quite difficult to forge’, according to Cyril Smith (quoted by Wertime 1963 as personal communication).

\begin{myquote}
“Deliberate fluxing was at the heart of metallurgy as well as of all pyrotechnology” (Wertime 1980: 14). 'Hematite (Fe$\_2$O$\_3$) was the most common flux used in copper and lead smelting. Gossans must have been commonly used because of their natural association with ores and because they contained some of the coloured minerals initially sought; when added gave a better result. Ultimately this developed into use of limonite alone…' (Charles 1980: 165). Charles further elaborated this point, “The use of iron oxide as a flux is particularly important since it constituted a new type of development - incorporation of a constituent for an effect rather than for a product. It demonstrates a growing understanding of the behaviour of materials in fire and how their behaviour might be controlled. More than this, however, it can clearly indicate the origin of the production of metallic iron....Under certain specific conditions viz. excess of iron either in ore or added to the charge as fluxing agent, or if the furnace received high draft from below or if excessive charcoal was added, iron could be inadvertently produced” either in the form of ‘bear’ as at Timna or as magnetic copper. Though this could not be of much metallurgical value, rich iron bits could be broken or cut and used for fashioning small iron objects.”
\end{myquote}

Copper with high content of iron, when melted for alloying with tin in crucibles, especially with gossan as flux, may also yield small bits of iron. It could easily appear in the crucible, which could be separated as tiny bits. Interestingly, the earliest reported objects of iron in antiquity are generally tiny pieces used for special effect in bimetallic objects, which might have been procured under some such accidental situations. Such observations have been made on the basis of laboratory observation by archaeometallurgists like Wertime (1980), Charles (1980) and Tylecote (1962), Maddin (1977). Presumably it is due to this fact that in the beginning we come across only tiny bits of iron used to embellish copper/bronze objects. It could also indicate that iron was precious – far more expensive than gold. We may refer here to iron in pre-3000 BCE in Egypt, Mesopotamia or even Anatolia in the early levels. Because there was no real control over the process, the artisans could occasionally get a few pieces from the charge. This is clearly borne out by the Table 1.1 given below.

Similar conditions must have prevailed in the Bronze Age Mesopotamian and Iranian furnaces. Shaffer (1984) noted similar evidence in certain Harappan period furnaces at Afghanistan. The bronze smiths there utilized bits of iron for ornamentation of copper bronze objects. Heavy chunks of iron or iron oxide-rich slag were found littered all over the working area. This must have been discarded as a waste product of copper smelting by the Harappan coppersmiths who generally used chalcopyrite ore.

However, with growing understanding of alloys during the Copper-Bronze Age, possibilities of recognising iron as a separate metal familiar to smiths grew manifold. Nevertheless, even after the initial discovery of metallic iron, the metal smiths must have faced several difficulties:

\begin{enumerate}
\item Unlike copper and bronze, iron could not be melted at available temperature. Therefore, a different way had to be devised to extract and forge iron.

\item Mass production of copper/bronze objects through casting was easy without much effort. Iron objects, on the contrary, could be forged only singly involving greater labour and time.

\item Iron objects forged using the above method would be much inferior in strength compared to bronze. Experiments must have been naturally conducted to improve the quality of iron in successive ages.

\item The technique of cold hammering practiced to harden bronze could have been only partially successful with iron. To enhance the proficiency of iron, it had to be forged at a higher temperature that necessitated a longer contact with charcoal. This must have been learned through experiments over a long period of time.

\end{enumerate}

It must have indeed taken a long time to develop effective and efficient methods to produce iron. Some of the techniques adopted by the early ironworkers as prevalent among pre-industrial societies can be traced back to the earlier methods used by the copper workers; for example, the practice of arranging ore and charcoal in alternate layers for smelting. In many of the tribal societies roasting of ores broken into small pieces is also a common practice. Since the melting temperature of iron (15350C) was difficult to achieve in the existing copper smelting furnaces, the metal was extracted by repeated hammering of the bloom. The similarity between the primitive copper smelting and the early iron working furnaces led Tylecote to believe that \textit{'invention of iron may be attributed to primitive copper smelters than to the more sophisticated ones' }(Tylecote 1962: 211; italics by author to emphasize the point). Thus heating and hammering repeatedly paved the way for a more useful iron as useful metal. Otherwise bronze, which was being used and produced by smiths of copper/bronze age, was superior in strength as well as appearance than iron of the earliest levels. To quote Maddin (1977: 128):

\begin{myquote}
“the early metal workers produced iron from ores, mostly hematite and magnetite, by smelting process much like the one used to produce copper. There was, however, an important difference, pure iron does not melt at temperature below 1537 0C and the highest temperature that could generally be reached in a small primitive smelter appears to have been about 1200 0C. Smelting iron ore at that temperature yields not a puddle of metal but a spongy mass mixed with iron oxide and iron silicate. These non-metallic substances, which collectively represent slag, arise from the combustion of ferrous oxide and silica gangue in the reduction process.
\end{myquote}

\begin{myquote}
"The commonest of the non-metallic substances is fayalite, which remains viscous at temperature down to 1,1770C. The metal worker, therefore withdrew a mass of spongy iron from the furnace, reheated it in a forge and quite literally squeezed the fayalite out of it by hammering. The hammering at the same time turned the porous iron ‘bloom’ into a continuous network of iron grains interspersed with a few stringers of slag that had not been eliminated. The bloom was the blacksmith’s raw material: the iron articles were made by heating the bloom further.
\end{myquote}

\begin{myquote}
“What the blacksmith had to work with was a poor substitute for bronze. Bloomery iron is a soft metal; its tensile strength is about 40,000 per square inch, only slightly more than the strength of pure copper (about 32,000 P.S.I.) work-hardening, that is, continuous hammering, will bring the strength of iron up to almost 100,000 P.S.I. A bronze containing 11 percent tin, however, has tensile strength after casting of some 60,000 P.S.I. and the strength after cold working is as much as 1,20,000 P.S.I. Bronze was clearly a better material than bloomery iron for the manufacture of weapons and pots”.
\end{myquote}

He further asserted, “Bronze had other major advantage over iron. Since it melted at a temperature that the early metal workers could attain, it was suited to casting”, (Maddin 1977: 129). Therefore, large-scale production was easy by recasting. Recycling of metal was also easy. “Bronze could easily be produced into many complex shapes through casting in moulds. Bronze had one further advantage. It corrodes slowly, and the characteristic green patina is considered decorative. Iron corrodes rapidly, and in the process often suffers considerable damage” (Maddin 1977: 129).

One has then to ask, why was iron adopted at all? One probable reason could be the need to fulfil the general scarcity of metal in day-to-day life by an inexpensive metal. In several parts of the world tin became scarce affecting manufacturing of bronze. The search for an alternative must have become a necessity. Not every metallurgist agrees with Maddin's views quoted above. Prakash, for example feels that copper/bronze tools do not achieve a hardness of 250 Brinell and their edges are also brittle, (personal communication). Cast bronze objects are also brittle and do not get hardness comparable to iron. He thinks that even pure iron on cold working achieves a hardness of 250-300 Brinell. But iron is much more abundant than copper and thus accessible for production of metal objects. Man experimented with it at an early age.

The contention that scarcity of tin was a cause of adoption of iron has been contested by Moorey (1994: 286). In certain parts of the world like Anatolia, parts of Iran, Afghanistan etc. scarcity of tin was not a problem. In his considered opinion “the vital factors may have been more political than technological in the strict sense”. He refers to production of iron in Anatolia and Syria where iron ore was plentifully available. It may be added here that in India bronze was never plentiful. Harappan civilization is the only exception to this fact. In most ancient cultures, iron replaced bone and stone than copper/bronze for tools and implements.

\textbf{Table 1.1: Iron objects in antiquity at diverse stages of development in different countries (after Waldbaum, 1980)}

\begin{longtable}{|l|l|l|l|l|l|}
\hline
Country & Period & Meteoritic & Smelted & Not Analysed & Total \\
\hline
Iran & Pre-3000 BC\\ 3000-2000 BC\\ 2000-1600 BC\\ 1600-1200 BC & 3\\ x\\ x & x\\ x\\ x & x\\ x\\ x & 3 balls\\ Nil\\ x \\
\hline
Mesopotamia & Pre-3000 BC\\ 3000-2000 BC\\ 2000-1600 BC\\ 1600-1200 BC & x\\ 2 fragment tool (?)\\ x\\ - & x\\ 2dagger blades (fragment)\\ x\\ - & 5\\ x\\ not analysed & Tool (?)\\ 9\\ Nil\\ 2 \\
\hline
Egypt & Pre-3000 BC\\ 3000-2000 BC\\ 2000-1600 BC\\ 1600-1200 BC & 10\\ 4\\ 1\\ 20 & 1 bead\\ 1 amulet\\ -\\ 16(headrest, chisel & -\\ 3(tool rusted)\\ 1(spearhead)\\ - & 9\\ -\\ 4\\ - \\
\hline
Anatolia & Pre-3000 BC\\ 3000-2000 BC\\ 2000-1600 BC\\ 1600-1200 BC & -\\ 9\\ 4\\ 20 & -3(pin, plaque, macehead)\\ -\\ - & -\\ 1 dagger\\ -\\ - & -\\ 5\\ 4\\ 20 \\
\hline
Syria-Palestine & Starts in the last\\ lag of the classification\\ i.e. 1600-1200 BC & 10 & 1 (axe head) & - & 9 \\
\hline
Cyprus & 2000-1600 BC\\ 1600-1200 BC & 1\\ 3 & -\\ - & 1\\ - & -\\ 3 \\
\hline
Greece & 1600-1200 BC & 13 & - & - & 13 \\
\hline
Aegean Islands & - & 2 & - & - & 2 \\
\hline
Crete & - & 4 & - & - & 4 \\
\hline
\end{longtable}

The accompanying table of iron objects in the pre-1200 BCE context shows that the pieces belonging to pre-3000 BCE period that were subjected to analytical examination turned out to be of meteoritic origin. The meteors could be chiselled like stone in desirable shapes. The 5000-4000 BCE Mesopotamian specimen, a 4.30 cm long unidentifiable object from a grave in Samarra (North Iraq), is by far the earliest iron reported anywhere in the world. It is now known to be of meteoritic origin. Similarly, the three polishers from Iran (4600-4100 BCE) are also meteoritic. In the subsequent centuries (3000-2000 BCE) stray specimens of smelted iron start showing up in the Hittite, Mitanni, Kassite and Assiriyan lands.

Most of the iron objects of 3000-2000 BCE come from graves, palaces, treasure hoards or some ritualistic context. Even the weapons are ceremonial perhaps both in context and function. These are generally found along with gold and other precious materials. The Mesopotamian sites yielding such objects do not show evidence of local smelting. According to Moorey (1994: 287) these objects would have to be imports from Anatolia. During early second millennium BCE there is near absence of iron in Mesopotamia as well as in other near eastern sites. An 18th century palace at Acemhöyük (Mesopotamia) yielded an ivory box adorned with studs of gold, iron and lapis lazuli. In late Bronze Age Mesopotamia, a spherical bead of iron and a bronze dagger with iron inlays set into its hilt (Nuzi, stratum II dated 1400-1350 BCE) and a mace head from Aqar Quf also inlaid with iron clearly demonstrate the precious and prized nature of iron. Assyrian graves at Assur yielded a finger ring so did a grave at Mari. An iron toggle-pin was reported from a 13th century BCE context at tell Zubeidi (for details see Moorey 1994).

A Hittite text dated to 16th century BCE states that some centuries earlier the ruler of Purushkhanda (?) gave king Anittas of a neighbouring state a 'throne of iron and a sceptre (?) of iron as gift' (Neu 1974, quoted by Moorey 1994: 288). Perhaps it must have been a wooden throne decorated with iron as no such object has been found in excavations. One may thus visualize that objects of smelted iron came to be fashioned, though in a restricted number, from 1600 BCE onwards. Between 2000 BCE and 1600 BCE, we come across references to ornaments or exclusive ornamental iron objects. Iron at this stage was used to inlay beautiful bronze objects to enhance their value. Special mention may be made of a bronze head of a pin with a decorative inlay found in Anatolia. In Egypt (Buhen, Nubia), a spearhead with a leaf shaped blade came from a grave dated to 1991-1786 BCE. Whether it was a piece of smelted iron has not been ascertained (Waldbaum 1980: 75-76). Iron at this stage was undoubtedly scarce and thus a precious commodity – more expensive than gold.

After undertaking an overview of the evidence about theoccurrence of iron, Waldbaum remarked,

\begin{myquote}
"---the combined textual and material evidence in the early 2nd millennium BCE attests that iron is still used as a precious and rare material, associated with the royalty or the temples, or buried with other valuables in tombs. For the first time there emerges a picture of iron as a commodity or as an item of trade. Here too, whether the terms used refer to meteoritic or smelted iron or both, it is in the market as a luxury good, more costly even than gold, much sought after, and not readily available." With the combined testimony of archaeology and literature, one may deduce that till the middle of the 2nd millennium BCE iron remained a precious and exclusive material associated with temples and places used for ritualistic purposes.
\end{myquote}

In the next phase of 'Late Bronze Age Cultures’ dated between \textit{circa} 1600-1200 BCE,there is evidence of slight change in the pattern of use of iron in the Old World. To quote Waldbaum again, (1980: 76)

\begin{myquote}
"Not only do iron objects increase in number during these centuries but they are also distributed over a wider geographical area. This area stretches from Mesopotamia to mainland Greece and includes most of the regions of the Levant (Syria, Palestine, Anatolia, Egypt) as well as the islands of the Eastern Mediterranean (Cyprus, Rhodes, Lesbos, Crete)."
\end{myquote}

The objects she mentions in this context are, 'a dagger with a copper blade and hilt made of two iron plates fastened to the blade with an iron rivet.' A small spherical bead came from Nuzi dated to the 15th century BCE. But none of these have been analysed, so one cannot say with certainty whether they are smelted or meteoritic iron objects.

Ras Shamra in Ugarit yields a battle-axe with 'cast-on copper socket decorated with gold'. Rings (from a tomb) and arrowheads came from Alalakh level II (1350-1273 BCE), a spatula (1270-1185 BCE), two arrowheads and a handle are reported from Tell es-Zuweyid, (1400-1230/1170 BCE). Only the first object from Ugarit has been analysed and is said to be meteoritic in nature having 3.25 percent nickel and 0.41 percent carbon – a mild steel object. Anatolia yielded a good number of iron objects ranging from nails and needles to arrowheads and daggers. The other objects reported are plaque, bracelet and a long conical socketed handle with remains of a wooden shaft from Alca Hüyük (strata IV –II dated to 1800-1200 BCE). Boghazkui yielded a chisel and a fragment from 1450-1350 BCE level; a lugged axe-head from 1300-1200 BCE and another such piece from 1350-1300 BCE phase and spearheads from 1300-1200 BCE level.

A similar picture emergesfrom other Old World countries such as Cyprus, Egypt, Greece, Crete etc. There are a few iron objects like nails, pieces or fragments of some bigger, objects ornaments like bracelets or rings, pins etc. and utility objects like arrowheads, chisels or daggers. But many of these are found to be meteoritic in origin. Most of the iron objects have not been analysed, therefore, it is not possible to say whether these objects were of smelted iron or were meteoritic in origin. The status of metallurgical expertise acquired by the ironworkers during this period of use of iron is difficult to assess. But it may be stated that even at this stage iron continued to be far from common and remained a precious commodity - available only to the selected few in the society.

The iron objects mentioned above continue to be mostly found in royal tombs or from palace complexes, sanctuaries and temples as in the previous stage. Thus there is little doubt that they were valuable and were associated with the privileged class. They remained more of a status symbol than anything else due to their high cost.

Although, analysis of most of the middle Bronze Age artefacts has not been done but from the data available to us, it appears a good number of iron objects belonging to pre-1200 BCE were meteoritic in origin. It is confirmed that these pieces were cut off from the meteorites to fashion some small objects. Waldbaum (1980: 79) comments, "they show that reliance on smelted metal as a source was not absolute". The excavations of Late Bronze Age sites in the large region mentioned above have not yielded any evidence of iron smelting so far; specimens analysed are rarely found to be smelted pieces. It needs further archaeological investigation and more analysis to confirm that actual iron smeltingtook place during this period.

The correspondence between the Hittite King Hattusilis III (1250 BCE) and Assyrian king Shalmaneser I in connection with the supply of iron has been interpreted to be an evidence of iron smelting. The response of the former to the request for iron (objects?) probably indicates that the particular time was not a suitable season for producing iron. The royal store at Kizzuwatna did not contain 'good' iron. In this context, it may be assumed that perhaps to keep the Assyrian king in good humour, a single iron dagger blade with bronze handle was sent as a gift. This clearly suggests that even if iron was being smelted, the metallurgy was not developed enough to produce iron as and when required and not certainly in sufficient quantity. That a single iron blade was considered worth presenting to a king sufficiently proves the precious nature of the commodity. Also, that it could not be smelted at convenience. Iron objects, up to 1200 BCE it would appear, continued to be scarce and were used more for ceremonial or symbolic items than for objects of utility.

Interestingly, it is around this very time that a fight ensued between Mitannis and Hittites (1380-1346 BCE). According to Maxwell Hyslop (1974), the war was fought over the iron ore rich territory. The war ended in the victory of Hittites giving them full control of the region and the important routes leading to different parts of the world. In due course, the name of Hittites comes to be closely associated with iron. The real use of iron (read smelted iron) in the Old World starts after 1200 BCE. But even at this stage, as shown above, the quantity of iron artifacts continued to be fairly small.

It may be noted that there is no evidence of iron smelting at the above-mentioned settlement sites. It is more reasonable to assume that actual smelting work was likely to have been done near the ore deposits closer to the mining zones. In the pre-1200 BCE maximum number of iron artefacts have been reported from Egypt followed by Anatolia (Waldbaum 1978: 22), "Anatolia and Egypt, with thirty three and thirty-eight pieces respectively, have the greatest number of early iron objects in the near east; the other areas log by a considerable degree...". During this period Egypt had started local manufacture of objects. In her opinion this was due to the exclusive tool typology found there. She states that the forms are unquestionably indigenous'. "This increases the probability that manufacture of the objects themselves, if not the extraction of ores, was carried out locally and not all in one centre" (Waldbaum, op.cit, p. 23). This fact has also been suggested by Tylecote (1980: 209) in connection with the highlands of the Near East that contained iron ore. Such regions are also rich in forests, the main source of fuel required in large quantities in the smelting process. Our own observations in the central Gangetic plains corroborate this.The situation in the eastern Vindhyas is discussed in greater detail in chapter VII. There is extensive evidence of pre-industrial iron smelting in this iron-rich forested region. The antiquity of iron smelting in that part goes back to the first half of the second millennium BCE. This kind of evidence is frequent in several such mining zones. This explains the near absence of smelting evidence at most of the habitation sites. Be that as it may, it is undisputed fact that iron took a long time to be adopted as a metal of day to day use. It took centuries to be adopted by the Bronze Age societies.

A delayed adoption of iron may presumably be attributed to the complexity of metallurgical process involved in production of iron, at least in the beginning. Significantly enough, experts doubt even about superiority of iron over copper bronze at least at the early stage of advent of iron. It may be attributed to the nature of metallurgy as well. It is well known, as discussed earlier that in comparison to wrought iron which was initially produced, copper bronze is supposed to have definite advantage. Firstly, copper is easier to smelt that too at much lower temperature in simpler furnaces and can be produced on larger scale in moulds if required. It is also a myth that a (good quality) copper-bronze is inferior to iron in strength. Any bronze even with ten per cent tin alloying has superior strength than wrought iron having no carbon content in the matrix of metal. Talking about change over from bronze to iron in the Chinese context, Taylor and Shell (1988: 205-221) remarked,

\begin{myquote}
“Two misunderstandings regarding the change over from bronze to iron plague archaeological literature... they are that iron is better than bronze and that the difficulty in smelting iron lies in the high temperature required. In fact it is only steel that is consistently stronger than bronze... Furthermore, a usable bloom, or molten iron, if it contains a large amount of carbon or other impurities can be produced at temperatures close to or even below those needed to melt copper or gold... the difficulty with iron lay not in obtaining high temperatures but in developing the new techniques necessary to hot forge the bloom...”
\end{myquote}

Thus iron required a new kind of technological configuration to be evolved that necessitated a long process of experimentation through trial and error. The nature and quality of early iron objects remain uncertain and scarce. This resulted into a slow production of iron that too on a considerable scale for a long duration of time. If the nature and type of iron objects at early Iron Age is a guide, we find that there is a very gradual change in pattern of utilization of metal objects. Copper-bronze remains a favoured medium for fashioning not only toiletries and ornaments but even war and hunting tools. Was it a compulsion or a choice made by the Copper-bronze using societies which actually first explored the possibilities of use of metallic iron?

To answer this, we need to examine the circumstances of emergence of iron in the Indian context. Metallic iron, in view of R.F. Tylecote (1962: 211) was discovered by some copper using communities which were not well adept in copper-bronze metallurgy. This has been further argued that for a long time it is the same metal worker who produced copper and later experimented with an alternative material, which is iron. This fact has been corroborated by other studies also. For instance, according to Maxwell Hyslop (1974), ‘black smithy and bronze smithy were not yet strictly separated, a fact which is also suggested by analysis of certain objects’. An iron sword from Luristan (7th century BCE) and an 8th century implement from Khorsabad show that the technology employed in their manufacturing were more apt for bronze than iron. In the opinion of Waldbaum (1980: 90-91) this resulted into a lack of “complete control over the special technique for producing high quality iron.” This must be cause of ‘rather a halting adaption of iron and the continued production of bronze utilitarian implements”.

It is evident from the foregoing that even in the period immediately after advent of iron, it remained precious and therefore was used more as a status symbol than utilitarian metal. Metallurgical expertise of iron making was mostly restricted to fashioning beautiful ceremonial or ornamental objects from meteorites. Iron smelting evidences are few and far between before 1200 BCE. Further work in the ore rich areas could reveal something worthwhile concerningiron smelting. That would confirm the beginning of iron smelting in the true sense. It is only in the period between 1200-900 BCE that iron starts being used in the real sense. Waldbaum, (1980: 70)reports 848 iron object in this period. It is around 10th century BCE that a true iron age came into being as it was during this time that iron "ceased to be considered precious and was finally accepted as the predominant material for making tools and weapons" (Waldbaum, op.cit, p. 82). She goes on to argue that at this time one comes across a sizeable number of iron weapons and other artefacts. There are references in texts of taking unworked iron as tribute or booty. Iron implements thus came to be used in West Asia for the first time in 11th-10th century BCE. Their number increases, gradually as evident from archaeological excavations of Palestinian sites like Tell Jemmeh yielding axe, adze, hoe, ploughshare, sickle, chisel etc. There are certain Assyrian expressions, which confirm the commonality of iron. The Greek poet Hesiod (late 8th century BCE) refers to his people as a "race of iron" to symbolize the decline of values and morals from the golden age of the past. One may infer that it is around this time that iron was brought down to an ordinary metal status and becomes a useful metal within the reach of common man.

On the basis of the above discussions, it may be concluded that:

\begin{enumerate}
\item The earliest iron pieces are mostly meteoritic in origin. However, a proper analysis of the early iron objects is necessary to confirm this.

\item Although the date of man made iron in view of some scholars goes back to \underline{ca }3000 BCE but they seem to be incidental production. Regular production of iron came much later. It took long to adequately master the metallurgical skill to produce iron as and when required. It is borne out by the communication between Hittite and Assyrian rulers (mentioned above).

\item Iron in the true sense comes to be used in most parts of the Old World, including the Western Asiatic countries only around 1200-900 BCE. It is only around this time that it became possible to produce iron in a planned way and as and when required.

\end{enumerate}

If we compare the situation in West Asia, Mediterranean, Iran, China etc. on the one hand and that of the Indian subcontinent regarding the use of metallic iron on the other, we fail to find a direct relationship. Theredoesnotseem to be any direct link by way of intrusion, interaction or even inspiration of ideas in 2nd millennium BCE that too in the heartland of India.Most of the iron objects, which came from West Asia or Iran, in the earliest period,appear to be meteoritic in nature. Between the 3rd millennium BCE and 1200 BCE with the recognition of the true potentiality of iron, there appears to be a constant effort in the Old World civilizations to enhance the production of iron as its demand increased at least among the nobility of that age. The data from recent archaeological investigations in India reveal cases of much earlier incidence of iron that is around 1600-1500 BCE or even earlier in the heartland of India; and almost at this very time in the peninsular India. The site of Gufkral in Kashmir valley has yielded 14C dates ranging from 1900 to 1400 BCE from iron-bearing Megalithic levels. Comparable dates have been reported from several megalithic burials in southern India. These dates of beginning of iron going back to first half of second millennium BCE cannot be wished away. One has to closely evaluate their significance while dealing with introduction of iron in India. These early iron objects are not meteoritic as in the other Old World civilizations as discussed above. It is noteworthy that iron at many early iron sites is also accompanied by evidence of iron working. Furnace remains, like, slag, charcoal, ash etc. are found to substantiate the evidence of local smelting of iron at many of these sites.

If we confine our inquiry to the Indian context against the background of the Old World civilizations, we find that in India, bronze was scarce throughout the Chalcolithic period and even afterwards in the post-Chalcolithic time. The Harappans were the only exception to it. The number of bronze objects of post-Harappan cultures may be counted on one’s fingertips. The known deposits of tin in the country are in Chhota Nagpur Plateau such as at Giridih, Gaya, Hazaribagh, Ranchi, in Jharkhand and adjacent areas of Bengal and in Haryana. Tin is also found in the hills of Kumaon-Garhwal and in Tusham hills in Haryana. It is available in the neighbouring parts of India, such as at Myanmar (Burma) and Malaysia. This scarcity of bronze in early cultures suggests non-utilization or little utilization of tin from the above sources at that stage of cultural development. Even otherwise, the theory of scarcity of tin and search for an alternative to bronze has not gained universal acceptance (Moorey 1994). On the contrary, iron was profusely and easily available in almost every region. This may be partially responsible for its larger use once the metallurgy was mastered.

In India, scholars have rarely examined circumstances of occurrence of the earliest iron and the experimental stages leading to its actual smelting. The Harappans were masters in metallurgy. They could produce copper-bronze objects in large numbers. Highly complex techniques like \textit{ciré perdué} were known to them. The famous statue of the dancing girl is an excellent example of the lost wax technique. But at a number of Harappan sites large chunks of iron – perhaps salamanders –were found. Several Harappan objects contained a fair amount of iron in the matrix of copper (Tripathi 2001: 67-70). It shows that the Harappan copper smiths used chalcopyrite ore for copper. They must have recognised iron as an undesirable mix in copper, which they must have tried to get rid off. Generally they successfully eliminated iron from the matrix of the copper but in a number of objects, as pointed above, possibly it could not be done. May be iron was also recognised as a fluxing agent by the Harappan copper workers.

The Chalcolithic cultures, unlike the riverine culture of the Indus Valley, mostly occupying the remoter parts of the subcontinent stayed closer to hilly tracts. They were generally partly contemporaneous to the Harappans. The Neolithic-Chalcolithic community had exploited different minerals – stones of different variety for different purposes. The Chalcolithic folk had exploited different metals like, copper, tin, arsenic, lead and at times gold. It is they who must have chosen to experiment with different types of minerals. As Tylecote contended (above) that iron could have been inadvertently produced by such metal smiths who were less proficient in their craft. We know that iron appears in the chalcolithic set up of Ahar in Rajasthan, Eran in Madhya Pradesh and at many sites in Bengal, Bihar and Uttar Pradesh in Chalcolithic context. These metal smiths must have had the basic understanding of minerals they were using. It is possible that they stumbled upon iron accidentally; and later they started experimenting with iron ore and even producing them \textit{albeit} on a limited scale.

An intensive analytical study of the metals of the Chalcolithic culture has to be undertaken before we can state with certainty as to when iron was first recognized in India and when it was used. However, at the present state of our knowledge, we may say that in India the advent of iron can be attributed to the chalcolithic metalworkers in the second millennium BCE. With such early dates of iron coming from recent excavations, the logical deduction would now be to look at the possibilities of movement of iron technology from India to outside world. With this, we may now examine the status of iron in India in greater detail.

