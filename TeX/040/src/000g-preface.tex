\thispagestyle{empty}


\begin{center}
\Huge\textbf{PREFACE}
\end{center}

\vskip 20pt

India is a great country with a glorious history. Ancient India was at the forefront of science. Our ancient ayurvedic treatise \textit{Charaka\break Samhitha} described the four cardinal principles to prevent lifestyle\break related diseases such as diabetes as far back as 2300 years ago, namely \textit{“achar, ahar, vyayam and yoga”}. However, today India and the entire world are at the center of a growing diabetes epidemic.

Diabetes is a consequence of our own making. We have deviated from the lifestyle prescribed in our ancient scriptures, which advocate a positive mindset, active lifestyle and moderation in life. Nowadays, our lives are not in harmony with the laws of nature. We are either overeating or eating the wrong food. We are sedentary either by choice or compulsions of a big city life. We are struggling to cope with the stresses of modern life.

Diabetes is a disease that spares no organ in the body. Diabetics are three times more likely to have high blood pressure, 4 times more likely to die from a heart attack, and 4 times more likely to have a stroke. Diabetes is particularly hard on the Indian kidneys causing\break kidney failure at a 10–40 times higher rate than Caucasians. On a more sensitive note, diabetic men are 3 times more likely to develop erectile dysfunction, and they do so 10 years earlier than non-diabetics.

Today, diabetes is no longer a disease of just the rich and old. Type 2 diabetes, which is usually diagnosed at the age of 40–50 years, is\break now being reported in kids as young as 9! In fact, Indian kids have an\break 11-fold higher risk of developing diabetes than Caucasian children.\break Youth account for 70\% of India's population. Our future is crippled if the status quo prevails.

Diabetes is not a disease limited by a country's borders, and has\break followed Indians to every part of the globe. Indians are the fastest growing minority in the USA, and are the second largest ethnic group in Britain after Caucasians. They have double the rate of diabetes\break compared to Whites.

Thus, whether we like it or not, diabetes is at everyone's doorstep today. Look around you, and you will see a friend, family member, neighbor or co-worker suffering from diabetes.

Unfortunately, this major public crisis is not getting the attention it warrants. In our effort to fight this epidemic, we have meticulously combed through the entire scientific literature to identify research work that specifically addresses diabetes and its related problems\break affecting the Indian population. We have drawn on the 20 years of professional experience in treating patients with diabetes, the\break diabetes research experience gained at the All India Institute of\break Medical Sciences, New Delhi, and research conducted at the Sreehari Diabetes and Research Center, Mysuru, and the diabetes education\break acquired at the University of Newcastle, Australia, in authoring this book. The contents of this book is spread over 28 chapters, divided into 3 parts: (The fundamentals of diabetes, the 10 major complications of diabetes, and The solutions to the problems.

We have highlighted simple, but very effective measures to\break counter diabetes. For example, eating a diet that contains complex\break carbohydrates (whole grains, vegetables, and fruits) instead of simple sugars (Coke, Pepsi, sweets, maida, white rice, etc) causes less stress on the pancreas. Consuming plant-based protein (dals, peas, nuts, seeds, etc) leads to less hunger, and consequently, better blood sugar\break control. Healthy fats (mono and poly unsaturated fats) in moderation are both essential and beneficial for health. On the contrary, the\break so-called "fat free" foods are likely to lead to poor diabetes control. The spices we routinely use in our kitchen (lndian spices - jeera, methi,\break sarson, kali mirch, etc) tend to lower our blood sugar and cholesterol levels. A healthy diet, combined with regular physical activity can\break reduce the risk of diabetes by 71\%. For every kilogram of excess weight that you burn, the risk of diabetes falls by 16\%. Yoga reduces blood sugar levels within hours, and leads to sustained benefit. This book is packed with many such facts and figures, which will bring diabetes prevention and control within the grasp of every reader.

It is our sincere hope and prayer that this book serves uncover this silently burgeoning epidemic of diabetes, and provide the readers\break useful tips to combat it.

\begin{flushright}
\textbf{Dr. V. Lakshminarayan \& Dr. Sooraj Tejaswi}
\end{flushright}


\newpage
~\phantom{a}
\thispagestyle{empty}
\newpage
