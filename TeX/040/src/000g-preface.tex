
\chapter*{PREFACE}

India is a great country with a glorious history. Ancient India was at the forefront of science. Our ancient ayurvedic treatise Charaka Samhitha described the four cardinal principles to prevent lifestyle related diseases such as diabetes as far back as 2300 years ago, namely \textit{“achar, ahar, vyayam and yoga”}. However, today not only India, but the whole world unintentionally finds itself bang in the center of the diabetes epidemic.

It is a \textit{consequence} of our own making, and left unchecked will turn into the greatest man–made health disaster of our times. We have deviated from the lifestyle prescribed in our ancient scriptures, which advocate a positive mindset, active lifestyle and moderation in life. Nowadays, our lives are not in harmony with the laws of nature. We are either overeating or eating the wrong food. We are sedentary either by choice or compulsions of a big city life. In short, we are struggling to cope with the stresses of modern of life.

Diabetes is a disease that spares no organ in the body. Diabetics are three times more likely to have high blood pressure, 4 times more likely to die from a heart attack, and 4 times more likely to have a stroke. Diabetes is particularly hard on the Indian kidneys causing kidney failure at a 10–40 times higher rate than Caucasians. On a more sensitive note, diabetic men are 3 times more likely to develop erectile dysfunction, and they do so 10 years earlier than non–diabetics.

Today, diabetes is no longer a disease of just the rich and old. Type 2 diabetes, which is usually diagnosed at the age of 40–50 years, is now being reported in kids as young as 9! In fact, Indian kids develop diabetes 11 times higher rate than Caucasian children in Britain. Youth account for 70\% of India’s population. If status quo prevails, a large chunk of them will go on to suffer from diabetes, and hence cripple country’s future.

Diabetes is not a disease limited by a country’s borders, and has followed Indians to every part of the globe. Indians are the fastest growing minority in the USA, and are the second largest ethnic group in Britain after Caucasians. They have double the rate of diabetes compared to Whites.

Thus, whether we like it or not, diabetes is at everyone’s doorstep today. Look around you, and you will see a friend, family member, neighbor or co–worker suffering from diabetes.

Diabetes is undoubtedly the biggest health challenge facing India, and Indians in any part of the world. Unfortunately, this major public crisis is not getting the attention it warrants. In our effort to fight this epidemic, we have meticulously combed through the entire scientific literature to identify research work that specifically addresses diabetes and its related problems affecting the Indian population. We have drawn on the 20–years of professional experience in treating patients with diabetes, the diabetes research experience gained at the All India Institute of Medical Sciences, New Delhi, and research conducted at the Sreehari Diabetes and Research Center, Mysuru, and the diabetes education acquired at the University of Newcastle, Australia, in authoring this book. The contents of this book is spread over 28 chapters, divided into 3 sections; the fundamentals of diabetes, the 10 major complications of diabetes, and the solutions to the diabetes menace.

We have highlighted simple, but very effective measures to counter diabetes. For example, eating a diet that contains complex carbohydrates (whole grains, vegetables, and fruits) instead of simple sugars (Coke, Pepsi, sweets, maida, polished rice, etc) causes less stress on the pancreas. Eating a plant–based protein diet (dals, peas, nuts, seeds, etc) leads to less hunger, and consequently, better blood sugar control. Healthy fats (mono and poly unsaturated fats) in moderation are both essential and beneficial for health. On the contrary, the so–called “fat free” foods are likely to lead to poor diabetes control! The spices we routinely use in our kitchen (Indian spices – jeera, methi, sarson, kali mirch, etc) tend to lower our blood sugar and cholesterol levels. A healthy diet, combined with regular physical activity can reduce the risk of diabetes by 71\%! For every kilogram of excess weight that you burn, the risk of diabetes falls by 16\%. Yoga reduces blood sugar levels within hours, and leads to sustained benefits by as early as 3 months. This book is packed with many such facts and figures, which will bring diabetes prevention and control within the grasp of every reader. The original Kannada version of this book \textit{“Madhumeha – Dashavyaadhigala Moola”} (Diabetes – The Root Cause of Ten Diseases), published by the Kuvempu Bhashabharathi Pradhikara, \textit{Government of Karnataka} in 2011, was chosen as the best work in medical literature for the year 2011, and was awarded the prestigious \textit{“Shreshtha Vaidya Sahitya Kruti Award”} by “Shankara Prathishtana”. Subsequently, Sapna Book House published the revised and enlarged version in Kannada, \textit{“Madhumeha– Bharatada Agochara Shatru”}, which has been in the list of top 10 books sold for more than four years now.

It is our sincere hope and prayer that this book will wake the world population from this dangerously sweet slumber, and will empower us to deal with our indulgences, be it the \textit{samosa} that we savor, the \textit{gulab jamun} that we crave, cakes and ice creams, or the endless hours spent enslaved to the TV, computer, laptop, ipad or smart phone. Armed with awareness and intent we can beat this invisible enemy!

\begin{flushright}
\textbf{Dr. V. Lakshminarayan \& Dr. Sooraj Tejaswi}
\end{flushright}

