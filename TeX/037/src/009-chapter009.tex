
\chapter{ಸಮಾರೋಪ}

ಮಂಡ್ಯ ಜಿಲ್ಲೆಯ ಭೂ ಪ್ರದೇಶವು, ಕಾವೇರಿ ಮತ್ತು ಹೇಮಾವತಿ ನೀರಿನಿಂದ ಯಾವರೀತಿ ಸಮೃದ್ಧಿಯಾಗಿದೆಯೋ, ಹೊಲಗದ್ದೆ, ತೆಂಗಿನ ತೋಟಗಳಿಂದ ಯಾವರೀತಿ ಸಂಪದ್​ಭರಿತವಾಗಿದೆಯೋ, ಅದೇ ರೀತಿ, ಶಾಸನಗಳು, ದೇವಾಲಯಗಳು ಮತ್ತು ಜನಪದ ಸಂಸ್ಕೃತಿಯಿಂದಲೂ ಕೂಡಾ ಸಮೃದ್ಧವಾಗಿದೆ. ಮಂಡ್ಯ ಜಿಲ್ಲೆಯಲ್ಲಿ ಭತ್ತ ಕಬ್ಬು ಬಿಟ್ಟರೆ ಏನೂ ಇಲ್ಲ, ಶಾಸನಗಳು, ದೇವಾಲಯಗಳೂ ಕಡಿಮೆ ಎಂದು ತಿಳಿದವರೇ ಹೇಳುವುದನ್ನು ಕೇಳಿದ್ದೇನೆ. ಪ್ರವಾಸಿಗರಿಗೆ ತಿಳಿದಿರುವುದು ಹೊಸಹೊಳಲು ದೇವಾಲಯ ಒಂದೇ. ಇತ್ತೀಚೆಗೆ ಕಲ್ಲಹಳ್ಳಿಯ ಭೂವರಾಹನಾಥ ದೇವಾಲಯಕ್ಕೆ, ಕನ್ನಂಬಾಡಿಯ ಗೋಪಾಲ ಕೃಷ್ಣ ದೇವಾಲಯಕ್ಕೆ ಅಪಾರ ಸಂಖ್ಯೆಯಲ್ಲಿ ಜನರು ಭೇಟಿ ನೀಡುತ್ತಿದ್ದಾರೆ. ಕೃಷ್ಣರಾಜಸಾಗರ ಅಂದರೆ ಕನ್ನಂಬಾಡಿ ಕಟ್ಟೆಯ ನೀರು ಕೃಷ್ಣರಾಜಪೇಟೆ ತಾಲ್ಲೂಕಿಗೇ ಒತ್ತಿಕೊಂಡು ನಿಂತಿದೆ. ಹೇಮಾವತಿ ನದಿಯು ಸೀದಾ ಕೃಷ್ಣರಾಜಸಾಗರಕ್ಕೇ ಸೇರುತ್ತದೆ. ಈ ನೋಟ ಬಹಳ ರಮ್ಯವಾಗಿದೆ. ಮಳವಳ್ಳಿ ತಾಲ್ಲೂಕಿನಲ್ಲಿ ಕಾವೇರಿ ನದಿಯ ಪಕ್ಕದಲ್ಲೇ ಇರುವ ಭೀಮನಕಂಡಿ ಬೆಟ್ಟ,ದ ಸಾಲು, ಹೇಮಗಿರಿ ಬೆಟ್ಟದ ಪಕ್ಕದಲ್ಲೇ ಹರಿಯುವ ಹೇಮಾವತಿ ನದಿ, ತೊಣ್ಣೂರಿನ ಕೆರೆ ದೇವಾಲಯಗಳು, ಬೆಳಕವಾಡಿಯ ಬಳಿ ಹರಿಯುವ ಕಾವೇರಿ ನದಿ, ಮಾರೆಹಳ್ಳಿ ದೇವಾಲಯ ಇವೆಲ್ಲಾ ಪ್ರಕೃತಿ ರಮ್ಯ ತಾಣಗಳಾಗಿವೆ. ಆದರೆ ಇನ್ನೂ ಒಳಹೊಕ್ಕು ನೋಡಿದಾಗ ಜಿಲ್ಲೆಯಲ್ಲಿ ನೂರಾರು ಪ್ರಮುಖ ಶಾಸನಗಳು, ದೇವಾಲಯಗಳು, ವೀರರಗುಡಿಗಳು, ಸ್ಮಾರಕ ಶಿಲೆಗಳು, ಬೆಳಕಿಗೆ ಬಾರದೆ ನಿಂತಿವೆ. 

ಈ ಕೃತಿಯು ನನ್ನ ಪಿಎಚ್​.ಡಿ. ಪ್ರಬಂಧದ ಸಂಪೂರ್ಣ ಪರಿಷ್ಕೃತ ರೂಪವಾಗಿದೆ. ಸಂಕ್ಷೇಪಿಸುವುದು, ಅಧ್ಯಾಯಗಳನ್ನು ಕೈಬಿಡುವುದು, ಹೊಸದಾಗಿ ವಿಷಯಗಳನ್ನು ಸೇರಿಸುವುದು, ಇವುಗಳಿಂದ ಈ ಕೃತಿಯು ನನ್ನ ಪಿಎಚ್​.ಡಿ. ಪ್ರಬಂಧಕ್ಕಿಂತ ಸಂಪೂರ್ಣವಾಗಿ ಭಿನ್ನವಾಗಿದೆ. ಮುಖ್ಯವಾಗಿ ರಾಜಕೀಯ ವ್ಯವಸ್ಥೆಯನ್ನು ಪೂರ್ತಿಯಾಗಿ ಸಂಕ್ಷೇಪಗೊಳಿಸಿ ಅಥವಾ ಕೈಬಿಟ್ಟು, ಕೇವಲ ಸಾಂಸ್ಕೃತಿಕ ಅಧ್ಯಯನದ ಅಧ್ಯಾಯಗಳನ್ನು ಅಳವಡಿಸುವಂತೆ, ನನ್ನ ಗುರುಗಳು ಮಾರ್ಗದರ್ಶನ ಮಾಡಿ ಸೂಚನೆ ನೀಡಿದ್ದರು. ಕೆಲವು ವಿಷಯಗಳಿಗೆ ಸಂಬಂಧಿಸಿದಂತೆ ಕೇವಲ ಪಟ್ಟಿಯನ್ನು ನೀಡುವಂತೆ ಹೇಳಿದ್ದರು. ಆದರೆ ಯಾವ ರಾಜ, ಯಾವಾಗ ಆಳ್ವಿಕೆ ನಡೆಸಿದ, ನಮ್ಮ ಜಿಲ್ಲೆಯಲ್ಲಿ ಅವನು ಏನು ಮಾಡಿದ್ದಾನೆ, ಅವನ ಸಾಧನೆ ಏನು, ಎಂಬ ರಾಜಕೀಯ ವಿಚಾರವು, ಸಾಮಾನ್ಯ ಜನರಿಗೆ ಅತ್ಯಂತ ಆಸಕ್ತಿದಾಯಕವಾಗಿರುವುದರಿಂದ, ಆ ಅಧ್ಯಾಯವನ್ನು ಜಿಲ್ಲೆಯ ಶಾಸನಗಳಿಗೆ ಸೀಮಿತಗೊಳಿಸಿ ಸಂಕ್ಷಿಪ್ತಗೊಳಿಸಲು ಪ್ರಯತ್ನಿಸಿದೆ. ಅಷ್ಟಾದರೂ ಅದು ಹೆಚ್ಚಿನ ಪುಟವನ್ನು ಆಕ್ರಮಿಸಿತು. ಆ ನಂತರ ಕೆಲವು ವಿಷಯಗಳನ್ನು ಪಟ್ಟಿ ಮಾಡಲು ಹೇಳಿದ್ದರು. ಪಟ್ಟಿ ಮಾಡುವಾಗ, ಒಂದು ವಿಷಯಕ್ಕೆ ಎರಡೆರಡು ಸಾಲು ಎಂದು ನಮೂದಿಸುತ್ತಾ ಬಂದು, ಅದೂ ಹೆಚ್ಚು ಪುಟವನ್ನು ತೆಗೆದುಕೊಂಡಿತು. 

ನನ್ನ ಮೂಲ ಪ್ರಬಂಧದಲ್ಲಿದ್ದ, ಅನೇಕ ಅಧ್ಯಾಯಗಳನ್ನು ಮುಖ್ಯವಾಗಿ ಕೃಷಿ, ಸಸ್ಯ ಸಂಪತ್ತು, ಸ್ಥಳನಾಮಗಳು, ಕಾಲ ನಿರೂಪಣೆ, ಶಾಸನಗಳಲ್ಲಿ ಬರುವ ಪ್ರಾರ್ಥನಾ ಶ್ಲೋಕಗಳು, ಶಾಪಾಶಯಗಳು, ಇತ್ಯಾದಿ ಅಧ್ಯಾಯಗಳನ್ನು ವಿಸ್ತಾರದ ಭಯದಿಂದ ಕೈಬಿಟ್ಟಿದ್ದೇನೆ. ಉಳಿದ ಎಲ್ಲ ಅಧ್ಯಾಯಗಳನ್ನು ಸಾಧ್ಯವಾದ ಮಟ್ಟಿಗೆ ಸಂಕ್ಷೇಪಿಸಿದ್ದೇನೆ. ಅನುಬಂಧಗಳನ್ನಂತೂ ಪೂರ್ತಿಯಾಗಿ ಬಿಟ್ಟಿದ್ದೇನೆ. ಇಷ್ಟಾದರೂ ಈ ಕೃತಿಯು ಇಷ್ಟೊಂದು ಬೆಳೆಯಿತು. ಕೇವಲ ಶಾಸನಗಳಲ್ಲಿ ಪರಿಣತರಾದವರನ್ನು, ವಿದ್ವಾಂಸರನ್ನು ಗಮನದಲ್ಲಿಟ್ಟುಕೊಂಡು, ಬರಹವನ್ನು ಇನ್ನೂ ಸ್ವಲ್ಪಮಟ್ಟಿಗೆ ಸಂಕ್ಷೇಪಿಸಬಹುದಾಗಿತ್ತು. ವಿಷಯಗಳನ್ನು ತೀರಾ ಸಂಕ್ಷಿಪ್ತಗೊಳಿಸಿದರೆ, ಕೊನೆಗೆ ಅದು ಸಾಮಾನ್ಯ ಓದುಗರಿಗೆ ಅರ್ಥವೇ ಆಗದೇ ಹೋಗಬಹುದು. ಇಂದು ಜನಸಾಮಾನ್ಯರನ್ನು ಇತಿಹಾಸ, ಸಂಸ್ಕೃತಿ ಹಾಗೂ ಶಾಸನ ಕ್ಷೇತ್ರದತ್ತ ಸೆಳೆಯುವ ಕೆಲಸ ಆಗಬೇಕು. ಅವರು ತಮ್ಮ ತಮ್ಮ ಊರಿನಲ್ಲಿರುವ ಶಾಸನಗಳು, ಸ್ಮಾರಕ ಶಿಲೆಗಳು, ದೇವಾಲಯಗಳ ಬಗ್ಗೆ, ಈ ಮೂಲಕ ಪರಿಚಯ ಮಾಡಿಕೊಂಡು, ಸ್ವಲ್ಪ ಮಟ್ಟಿಗೆ ಆಸಕ್ತಿಯನ್ನು ಬೆಳೆಸಿಕೊಂಡು, ಇಂತಹ ಅಪರೂಪದ ಸ್ಮಾರಕಗಳನ್ನು, ನೋಡಲು, ಕಾಪಾಡಲು ಮುತುವರ್ಜಿ ವಹಿಸುವಂತಾಗಬೇಕು. ಈ ಗ್ರಂಥದ ಮೇಲೆ ಕಣ್ಣಾಡಿಸಿದ ಜನಸಾಮಾನ್ಯ ಓದುಗರಿಗೆ ಅಷ್ಟಾದರೂ ಒಂದು ಪ್ರೇರೇಪಣೆ ಆಗಬಹುದೇನೋ ಎಂದು ಸ್ವಲ್ಪ ವಿವರವಾಗಿ ನೀಡಿದ್ದೇನೆ. 

ಸಾಧ್ಯವಾದ ಮಟ್ಟಿಗೆ ಕ್ಷೇತ್ರಕಾರ್ಯ ಮಾಡಿ, ಮಾಹಿತಿಗಳನ್ನು ಸಂಗ್ರಹಿಸಿದ್ದೇನೆ. ಸುಮಾರು ೨೦೦ ಹಳ್ಳಿಗಳಿಗೆ ಭೇಟಿ ನೀಡಿದ್ದೇನೆ. ಶಾಸನ, ದೇವಾಲಯ ಇರುವ ಸ್ಥಳವನ್ನು ನೋಡಿದರೆ, ಅದನ್ನು ನೋಡದೇ ಕೇವಲ ಶಾಸನಗಳನ್ನು ಓದಿ ವಿವರಿಸುವುದಕ್ಕಿಂತ, ಬೇರೆಯ ರೀತಿಯಲ್ಲೇ ವಿವರಿಸಬಹುದು ಎಂಬ ಸಂಗತಿ ನನ್ನ ಅರಿವಿಗೆ ಬಂದಿದೆ. ಅನೇಕ ಹೊಸ ವಿಷಯಗಳನ್ನು ಹೇಳಿದ್ದೇನೆ. ಅದರಲ್ಲೂ ಮುಖ್ಯವಾಗಿ ಜಿಲ್ಲೆಯ ಶಾಸನೋಕ್ತ ಮಹಾಪ್ರಧಾನ ದಂಡನಾಯಕರುಗಳು, ಆಡಳಿತಾಧಿಕಾರಿಗಳು, ವರ್ತಕರು, ತೆರಿಗೆಗಳು, ಶಿಲ್ಪಿಗಳು, ಶಾಸನ ಕವಿಗಳು, ನೀರಾವರಿ ವ್ಯವಸ್ಥೆ, ವೈಷ್ಣವಧರ್ಮದ ರಾಮಾನುಜಾಚಾರ್ಯರ ನಂತರದ ಯತಿಗಳು, ವೀರಶೈವಧರ್ಮ, ಇವುಗಳ ವಿಷಯದಲ್ಲಿ, ಶಾಸನಗಳ ಅಧ್ಯಯನದ ಮೂಲಕ ಅನೇಕ ಹೊಸ ವಿಚಾರಗಳು ಬೆಳಕಿಗೆ ಬಂದಿವೆ. ಮೈಸೂರು ಒಡೆಯರ ಕಾಲದ ಕೆಲವು ಹೊಸ ವಿಷಯಗಳಿವೆ. ಜಿಲ್ಲೆಯ ಶಾಸನಗಳಿಗೆ ಸೀಮಿತರಾಗಿರುವ, ಅಕ್ಕಪಕ್ದ ಜಿಲ್ಲೆಗಳ ಶಾಸನಗಳಲ್ಲೂ ಉಕ್ತರಾಗಿರುವ, ನಮ್ಮ ಜಿಲ್ಲೆಯಲ್ಲೇ ಹುಟ್ಟಿಬೆಳೆದಿರಬಹುದಾದ, ಅನೇಕ ಮಹಾಪ್ರಧಾನ ದಂಡನಾಯಕರುಗಳಂತ ಸಾಧನೆಗಳನ್ನು ನೋಡಿದಾಗ, ಇಂತಹ ಚಾರಿತ್ರಿಕ ವ್ಯಕ್ತಿಗಳು ನಮ್ಮ ಜಿಲ್ಲೆಯಲ್ಲಿ ಬಾಳಿ ಬದುಕಿದರು ಎಂಬುದು ನಮಗೆ ಹೆಮ್ಮೆಯ ವಿಷಯ. ದೇವಾಲಯಗಳು ಮತ್ತು ಶಾಸನಗಳ ಇಂದಿನ ಪರಿಸ್ಥಿತಿಯನ್ನು ನೋಡಿದರೆ ಬಹಳ ಬೇಸರವಾಗುತ್ತದೆ. ಇವುಗಳ ಸಂರಕ್ಷಣೆ ಇಂದಿನ ಅಗತ್ಯವಾಗಿದೆ. ನಾನು ಒಂದು ಬಾರಿ ನೋಡಿಕೊಂಡು ಬಂದಿದ್ದ, ಮೂರು ಪ್ರಾಚೀನ ದೇವಾಲಯಗಳನ್ನು, ಇನ್ನೊಮ್ಮೆ ಹೋಗಿ ನೋಡುವ ಹೊತ್ತಿಗೆ ಉರುಳಿಸಿ, ಹೊಸದಾಗಿ ಸಿಮೆಂಟ್​ ದೇವಾಲಯಗಳನ್ನು ನಿರ್ಮಿಸಲಾಗಿತ್ತು. ಅಲ್ಲಿದ್ದ ಶಾಸನಗಳೂ ಕೂಡಾ ಈಗ ಸಿಗುತ್ತಿಲ್ಲ. 

ನಾನು ಶಾಸನ ಕ್ಷೇತ್ರದ ಪರಿಣತನಲ್ಲ. ಕೇವಲ ಆಸಕ್ತನು. ಇದೊಂದು ನನ್ನ ಮೊದಲ ಪ್ರಯತ್ನ. ವಿದ್ವನ್​ ಮಂಡಲಿಯವರು, ನನ್ನ ಈ ಪ್ರಯತ್ನವನ್ನು ಆದರದಿಂದ, ಸಹಾನುಭೂತಿಯಿಂದ ಸ್ವೀಕರಿಸಬೇಕು. ಇದರ ಓರೆಕೋರೆಗಳನ್ನು, ತಪ್ಪು ತವಡಿಗಳನ್ನು, ಎತ್ತಿ ತೋರಿಸಬೇಕು. ಅದನ್ನು ತಿದ್ದಿಕೊಳ್ಳುತ್ತೇನೆ. ಈ ಬಗ್ಗೆ ನನ್ನ ಅಧ್ಯಯನ, ಲೇಖನವನ್ನು ಮುಂದುವರಿಸುತ್ತೇನೆ. ಈ ಗ್ರಂಥದ ಒಂದೊಂದು ಅಧ್ಯಾಯದ ಮೇಲೂ ಒಂದೊಂದು ದೊಡ್ಡ ಪ್ರಬಂಧಗಳನ್ನು ಬರೆಯುವಷ್ಟು ಸರಕಿದೆ. ಮಂಡ್ಯ ಜಿಲ್ಲೆಯಲ್ಲಿ ವೈಷ್ಣವಧರ್ಮ, ಪಂಚಿಕೇಶ್ವರ, ಜೈನ ಯತಿಪರಂಪರೆ, ಇಂತಹ ಅನೇಕ ವಿಷಯಗಳ ಬಗ್ಗೆ ದೊಡ್ಡ ದೊಡ್ಡ ಪ್ರಬಂಧಗಳನ್ನು ಪ್ರತ್ಯೇಕವಾಗಿ ಬರೆಯಬಹುದೆಂದು ನನ್ನ ಅನಿಸಿಕೆ. ನಾನು ಛಾಯಾಗ್ರಾಹಕನಲ್ಲ. ಸುಮಾರು ೧೦೦೦ ಛಾಯಾಚಿತ್ರಗಳನ್ನು ನನಗೆ ಬಂದಹಾಗೆ ತೆಗೆದಿದ್ದೇನೆ. ಅವುಗಳಲ್ಲಿ ಸುಮಾರು ೨೦೦ಕ್ಕೂ ಹೆಚ್ಚು ಛಾಯಾಚಿತ್ರಗಳನ್ನು ಈ ಕೃತಿಯಲ್ಲಿ ಅಳವಡಿಸಿದ್ದೇನೆ. ಛಾಯಾಚಿತ್ರಗಳಿಂದ ಈ ಕೃತಿಯ ಅಂದ ಹೆಚ್ಚಿದೆ ಹಾಗೂ ಗಮನ ಸೆಳೆಯುತ್ತದೆ ಎಂದು ನನ್ನ ನಂಬಿಕೆ. 

