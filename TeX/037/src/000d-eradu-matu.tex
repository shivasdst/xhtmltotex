
\chapter*{ಎರಡು ಮಾತು}

ನಾನು ಮೂಲತಃ ಮಂಡ್ಯ ಜಿಲ್ಲೆ, ಕೃಷ್ಣರಾಜಪೇಟೆ ತಾಲ್ಲೂಕಿನ, ಸಂತೇಬಾಚಹಳ್ಳಿಯವನು. ಬಾಲ್ಯದಲ್ಲಿ ಸುಮಾರು 14-15\break ವರ್ಷ ಈ ಹಳ್ಳಿಯಲ್ಲೇ ಕಳೆದು ವಿದ್ಯಾಭ್ಯಾಸ ಮಾಡಿದೆ. ಕಾಲೇಜು ವಿದ್ಯಾಭ್ಯಾಸಕ್ಕಾಗಿ ಮಂಡ್ಯಕ್ಕೆ ಬಂದೆ. ಆದರೆ ನನ್ನ\break ವಿದ್ಯಾಭ್ಯಾಸವು, ನಾನಾ ಕಾರಣಗಳಿಂದ, ಎಲ್ಲಿಂದ ಎಲ್ಲಿಗೋ ಹೋಗಿ, ಕೊನೆಗೆ ಕನ್ನಡ ಮತ್ತು ಆಂಗ್ಲ ಶೀಘ್ರಲಿಪಿ ಮತ್ತು\break ಬೆರಳಚ್ಚು ಪರೀಕ್ಷೆಗಳಲ್ಲಿ ಉನ್ನತದರ್ಜೆಯಲ್ಲಿ ತೇರ್ಗಡೆಯಾಗಿ, ಸರ್ಕಾರಿ ಸೇವೆಗೆ ಸೇರಿದೆ. ತಾಯಿ ನುಡಿಯಾದ ಕನ್ನಡ ಭಾಷೆಯ ಬಗ್ಗೆ ನನಗೆ ಮೊದಲಿನಿಂದಲೂ ಒಂದು ರೀತಿಯ ಪ್ರೇಮ, ಅಭಿಮಾನ. ವಿವಿಧ ಪರೀಕ್ಷೆಗಳಲ್ಲಿ ಕನ್ನಡ ವಿಷಯದಲ್ಲಿಯೇ ಹೆಚ್ಚು ಅಂಕ ಗಳಿಸುತ್ತಿದ್ದೆ. ಓದುವ ಹವ್ಯಾಸವೂ ಬಹಳವಾಗಿತ್ತು. 

ದೂರದ ಕಲಬುರ್ಗಿಯಲ್ಲಿ, 1976ರಲ್ಲಿ ಸರ್ಕಾರಿ ಸೇವೆಗೆ ಸೇರಿದೆ. ವಿದ್ಯಾಭ್ಯಾಸವನ್ನು ಮುಂದುವರಿಸುವ ದೃಷ್ಟಿಯಿಂದ, ಕಲಬುರ್ಗಿಯ ಕಾಲೇಜಿನಲ್ಲಿ, ಅಲ್ಲಿದ್ದ ಕನ್ನಡದ ಕಟ್ಟಾಳು, ಹೈದರಾಬಾದ್​ ಕರ್ನಾಟಕದ ಪ್ರಸಿದ್ಧ ಸಾಹಿತಿ, ಲೇಖಕ ಪ್ರೊ. ವಸಂತ ಕುಷ್ಟಗಿ ಅವರ ನೆರವಿನಿಂದ, ವಿದ್ಯಾಭ್ಯಾಸ ಮುಂದುವರಿಸಿ, ಕನ್ನಡವನ್ನು ಮುಖ್ಯ ವಿಷಯವನ್ನಾಗಿ, ಇತಿಹಾಸ ಮತ್ತು ಸಮಾಜಶಾಸ್ತ್ರವನ್ನು ಸಹ ವಿಷಯಗಳನ್ನಾಗಿ ಆರಿಸಿಕೊಂಡು ಪದವಿ ಪರೀಕ್ಷೆಯಲ್ಲಿ ತೇರ್ಗಡೆಯಾದೆ. ವೇತನ ಬರುತ್ತಿದ್ದುದರಿಂದ, ಸಾಹಿತ್ಯ ಚರಿತ್ರೆ, ಇತಿಹಾಸ ಮತ್ತು ಸಂಸ್ಕೃತಿಗೆ ಸಂಬಂಧಿಸಿದ ಪುಸ್ತಕಗಳನ್ನು ಖರೀದಿಸಿ ಓದುತ್ತಿದ್ದೆ. ನಾನು ಆಗಲೇ ಸರ್ಕಾರಿ ಸೇವೆಗೆ ಸೇರಿದ್ದರಿಂದ, ಮುಂದೆ ಓದಿ ಪದವಿ ಗಳಿಸುವುದರಿಂದ ನನಗೆ ಸರ್ಕಾರಿ ಸೇವೆಯಲ್ಲಿ ಏನೂ\break ಲಾಭವಿರಲಿಲ್ಲ. ನನ್ನ ಆದರೂ ನನ್ನ ಅಧ್ಯಯನಕ್ಕೆ ಏತಕ್ಕೆ ಒಂದು ರೂಪು ಕೊಡಬಾರದು ಎನಿಸಿತು. ಮೈಸೂರು\break ವಿಶ್ವವಿದ್ಯಾನಿಲಯದ ಅಂಚೆ ಮತ್ತು ತೆರಪಿನ ಶಿಕ್ಷಣ ಸಂಸ್ಥೆಯಿಂದ ಶಾಸನಶಾಸ್ತ್ರ, ಸಾಂಸ್ಕೃತಿಕ ಅಧ್ಯಯನ ವಿಷಯಗಳನ್ನು ಪ್ರಧಾನ ವಿಷಯಗಳನ್ನಾಗಿ ಆರಿಸಿಕೊಂಡು ಕನ್ನಡ ಎಂ.ಎ. ಪದವಿಯಲ್ಲಿ ಉನ್ನತ ದ್ವಿತೀಯ ದರ್ಜೆಯಲ್ಲಿ ತೇರ್ಗಡೆಯಾದೆ. ಕನ್ನಡ ಎಂ.ಎ. ಸಂಪರ್ಕ ಶಿಬಿರದಲ್ಲಿ ಡಾ. ಡಿ.ಟಿ. ಬಸವರಾಜ್​ ಅವರು ನಮಗೆ ಹದಿನೈದು ದಿನಗಳ ಕಾಲವೂ, ಕನ್ನಡ ಸಾಹಿತ್ಯದ ಹಿನ್ನೆಲೆ, ಇತ್ಯಾದಿ ಒಂದೆರಡು ವಿಷಯಗಳ ಬಗ್ಗೆ ಬೋಧಿಸುತ್ತಿದ್ದರು. ವಿಷಯದ ಆಚೀಚೆ ಹೋಗದೆ, ಹದಿನೈದು ದಿನದಲ್ಲಿ ನಮಗೆ\break ಗೊತ್ತುಪಡಿಸಿದ್ದ ವಿಷಯವನ್ನು ಪೂರ್ಣವಾಗಿ, ಕ್ರಮಬದ್ಧವಾಗಿ ಬೋಧಿಸುತ್ತಿದ್ದ ಅವರ ಬೋಧನೆ ನನಗೆ ಹಿಡಿಸಿತು. ಶಾಸನ ಶಾಸ್ತ್ರ, ಸಾಂಸ್ಕೃತಿಕ ಚರಿತ್ರೆಯಲ್ಲಿ ಕನ್ನಡ ಎಂ.ಎ. ಮಾಡುತ್ತಿದ್ದಾಗ, ಪರಾಮರ್ಶನ ಗ್ರಂಥವಾಗಿ, ಡಾ. ಎಂ. ಚಿದಾನಂದಮೂರ್ತಿಯವರ ‘ಕನ್ನಡ ಶಾಸನಗಳ ಸಾಂಸ್ಕೃತಿಕ ಅಧ್ಯಯನ’ ಕೃತಿಯನ್ನು ಆಮೂಲಾಗ್ರವಾಗಿ ಎರಡು ಮೂರು ಸಲ ಓದಿದೆ. ಶಾಸನ ಶಾಸ್ತ್ರ\-ವೆಂದರೆ ಕಬ್ಬಿಣದ ಕಡಲೆ ಎಂಬ ಮನೋಭಾವನೆ ಇದ್ದ ನನಗೆ, ಶಾಸನ ಶಾಸ್ತ್ರವನ್ನು ಇಷ್ಟೊಂದು ಸರಳ\-ವಾಗಿ, ಲಲಿತ ಶೈಲಿಯಲ್ಲಿ, ಸರಳವಾಗಿ ಅರ್ಥವಾಗುವ ರೀತಿಯಲ್ಲಿ ಬರೆಯಬಹುದೇ ಎಂದು ಆಶ್ಚರ್ಯವಾಯಿತು. ಇಂದಿಗೂ ನನ್ನ ಬಳಿ ಆ ಕೃತಿಯ 1979ನೇ ಇಸವಿಯ ಮುದ್ರಣದ ಪ್ರತಿ ಇದೆ. ಆಗಾಗ್ಗೆ ತಿರುವಿ ಹಾಕುತ್ತಿರುತ್ತೇನೆ. ಕನ್ನಡ ಶಾಸನ ಕ್ಷೇತ್ರದ ಸಾಂಸ್ಕೃತಿಕ ಅಧ್ಯಯನದ\break ಹೆಬ್ಬಾಗಿಲನ್ನು ತೆರೆದವರು, ಡಾ. ಚಿದಾನಂದ ಮೂರ್ತಿಗಳು, ನನಗೆ ನನ್ನಂತಹ ಅನೇಕರಿಗೆ ಅವರು ಮಾನಸಗುರುಗಳು. ಅವರ ಎಲ್ಲ ಕೃತಿಗಳನ್ನೂ ನಾನು ಖರೀದಿಸಿ ಇಟ್ಟುಕೊಂಡಿದ್ದೇನೆ. ಓದಿದ್ದೇನೆ.

ಹಾಗೇ ದಿನ ತಳ್ಳುತ್ತಿದ್ದಾಗ, ಮೈಸೂರಿನ ಕರ್ನಾಟಕ ರಾಜ್ಯ ಮುಕ್ತ ವಿಶ್ವವಿದ್ಯಾನಿಲಯದಿಂದ, ದೂರ ಶಿಕ್ಷಣದ ಮೂಲಕ ಎಂ.ಫಿಲ್​ ಪದವಿಯನ್ನು ಆರಂಭಿಸುತ್ತಿರುವ ವಿಷಯ ತಿಳಿಯಿತು. ಅರ್ಜಿ ಸಲ್ಲಿಸಿದೆ. ಅಲ್ಲಿ ಕನ್ನಡ ವಿಭಾಗದ ರೀಡರ್​ ಆಗಿದ್ದ ಡಾ. ಡಿ.ಟಿ. ಬಸವರಾಜ್​ ಅವರನ್ನು ನಮಗೆ ಮಾರ್ಗದರ್ಶಕರನ್ನಾಗಿ ಗೊತ್ತುಪಡಿಸಲಾಯಿತು. ಡಾ. ಡಿ.ಟಿ. ಬಸವರಾಜ್​(ಡಾ.ಡಿಟಿಬಿ) ಅವರ ವ್ಯಕ್ತಿತ್ವ ನನ್ನ ಮೇಲೆ ಪರಿಣಾಮ ಬೀರಿತು. ಅವರು ಅಧ್ಯಯನಶೀಲರು, ಸಜ್ಜನರು, ವಿನಯಶೀಲರು, ವಿದ್ಯಾರ್ಥಿಗಳಿಗೆ ಸೂಕ್ತವಾದ ರೀತಿಯಲ್ಲಿ ಮಾರ್ಗದರ್ಶನ ಮಾಡಿ ಪ್ರೋತ್ಸಾಹ ನೀಡುತ್ತಿದ್ದರು. ಒಂದು ದಿನವೂ ಬೇಸರಿಸಿದವರಲ್ಲ. ವಿಶ್ವವಿದ್ಯಾನಿಲಯದಲ್ಲಿ ನಮಗೆ ಸಂಬಂಧಿಸಿದ ಕಚೇರಿ ಕೆಲಸಗಳಿಗೆ ಅವರೇ ಖುದ್ದಾಗಿ ಬಂದು, ಸಂಬಂಧಿಸಿದ ಶಾಖೆಯವರ ಬಳಿ ಹೇಳಿ ಮಾಡಿಸಿಕೊಡುತ್ತಿದ್ದರು. ಅವರ ಮಾರ್ಗದರ್ಶನದಲ್ಲಿ ಎಂ.ಫಿಲ್​ ಪದವಿಯಲ್ಲಿ ಉನ್ನತ ದರ್ಜೆಯಲ್ಲಿ ತೇರ್ಗಡೆಯಾದೆ. ಡಾ. ಡಿಟಿಬಿ ಅವರು, ನಾನೂ ಸೇರಿದಂತೆ, ಅವರ ಬಳಿ ಎಂ.ಫಿಲ್​ ಮಾಡುತ್ತಿದ್ದ ವಿದ್ಯಾರ್ಥಿಗಳಿಗೆ (ಇವರಲ್ಲಿ ಅನೇಕರು ಕಾಲೇಜು ಉಪನ್ಯಾಸಕರುಗಳು, ಪ್ರೌಢಶಾಲೆ ಉಪಾಧ್ಯಾಯರುಗಳು ಆಗಿದ್ದರು) ಅಧ್ಯಯನವನ್ನು ನಿಲ್ಲಿಸದೆ,\break ಪಿಎಚ್​.ಡಿ. ಮಾಡಬೇಕೆಂದು ಹೇಳುತ್ತಿದ್ದರು.

ಅಷ್ಟು ಹೊತ್ತಿಗೆ, ಕರ್ನಾಟಕ ರಾಜ್ಯ ಮುಕ್ತ ವಿಶ್ವವಿದ್ಯಾನಿಲಯದಿಂದ ಪಿಎಚ್​.ಡಿ. ಪದವಿ ಅಧ್ಯಯನಕ್ಕೆ ಅರ್ಜಿಗಳನ್ನು ಆಹ್ವಾನಿಸಲಾಯಿತು. ಡಾ. ಡಿಟಿಬಿ. ಅವರು ನನಗೆ ದೂರವಾಣಿ ಮಾಡಿ ಪಿಎಚ್​.ಡಿ.ಗೆ ಅರ್ಜಿ ಸಲ್ಲಿಸಬೇಕೆಂದು ಹೇಳಿದರು. ಈ ವೇಳೆಗೆ ಪ್ರೊಫೆಸರ್​ ಆಗಿದ್ದ ಡಾ. ಡಿ.ಟಿ.ಬಿ. ಅವರ ಹತ್ತಿರ ಹೋಗಿ ಅರ್ಜಿ ಸಲ್ಲಿಸುವ ವಿಷಯ ತಿಳಿಸಿ, ‘ಮಂಡ್ಯ ಜಿಲ್ಲೆಯ ಶಾಸನಗಳು- ಒಂದು ಅಧ್ಯಯನ’ ಎಂಬ ವಿಷಯದಲ್ಲಿ ಪಿಎಚ್​.ಡಿ. ಮಾಡಲು ಉದ್ದೇಶಿಸಿದ್ದು, ತಾವೇ ನನಗೆ ಮಾರ್ಗದರ್ಶಕರಾಗ ಬೇಕೆಂದು ವಿನಂತಿಸಿಕೊಂಡೆ. ತಕ್ಷಣ ಅವರು ತಮ್ಮ ಒಪ್ಪಿಗೆ ಪತ್ರವನ್ನು ಕೈಯಲ್ಲೇ ಬರೆದು ನೀಡಿದರು. ವಿಶ್ವವಿದ್ಯಾನಿಲಯದ ಇಂದಿನ ಪರಿಸರದಲ್ಲಿ ಇಂತಹ ಘಟನೆಗಳು ಅಪರೂಪ. ಅವರ ಶಿಷ್ಯ ವಾತ್ಸಲ್ಯವನ್ನು ಎಷ್ಟೆಂದು ಹೇಳಿದರೂ ತೀರದು. ನಮ್ಮ ಊರು, ನನ್ನ ತಾಲ್ಲೂಕು, ಜಿಲ್ಲೆಯ ಬಗ್ಗೆ ನನಗೆ ಅಪಾರ ಅಭಿಮಾನ. ಆದುದರಿಂದ ನನ್ನ ಜಿಲ್ಲೆಯ ಶಾಸನಗಳ ಬಗ್ಗೆಯೇ ಪಿಎಚ್​.ಡಿ. ಮಾಡಲು ನಿರ್ಧರಿಸಿದೆ. ಕರ್ನಾಟಕ ಸರ್ಕಾರದ, ಪ್ರಾಚ್ಯ ವಸ್ತು ನಿರ್ದೇಶನಾಲಯವು 2008 ರಲ್ಲಿ ಹೊರತಂದ, ಮಂಡ್ಯ ಜಿಲ್ಲೆಯ ಇತಿಹಾಸ ಮತ್ತು ಪುರಾತತ್ವದ ಬಗ್ಗೆ ಮಂಡ್ಯದಲ್ಲಿ, ದಿನಾಂಕ 11 ಮತ್ತು 12 ಆಗಸ್ಟ್​ 2007 ರಂದು, ಏರ್ಪಡಿಸಲಾಗಿದ್ದ ವಿಚಾರ ಸಂಕಿರಣದಲ್ಲಿ ಮಂಡಿಸಲಾದ ಪ್ರಬಂಧಗಳ ಸಂಕಲನ ‘ಮಂಡ್ಯ ಜಿಲ್ಲೆಯ ಇತಿಹಾಸ ಮತ್ತು ಪುರಾತತ್ವ’ ಗ್ರಂಥದಲ್ಲಿ ಪ್ರಕಟವಾಗಿರುವ ಅಧ್ಯಕ್ಷ ಭಾಷಣದಲ್ಲಿ ಡಾ. ಎಮ್.ಎಸ್​. ಕೃಷ್ಣಮೂರ್ತಿಯವರು “ಇಷ್ಟು ಪ್ರವರ್ಧಮಾನವಾಗಿ ಸಂಪದ್ಭರಿತವಾಗಿ ನೆಲೆಸಿದ್ದ ನಾಡಿನ ಇತಿಹಾಸವನ್ನು, ಸಂಸ್ಕೃತಿಯನ್ನು ಪ್ರತ್ಯೇಕವಾಗಿ ಆಳವಾಗಿ ಅಧ್ಯಯನ ಮಾಡಲು ಯಾರೂ ಪ್ರಯತ್ನಿಸದಿರುವುದು ಅಚ್ಚರಿಯ ಸಂಗತಿಯೇ ಸರಿ...... ಆದರೆ ಮಂಡ್ಯ ಜಿಲ್ಲೆಯಲ್ಲಿ ಹೊಯ್ಸಳ ಸಂಸ್ಕೃತಿಯ ವಿವಿಧಮುಖಗಳನ್ನು ಪರಿಚಯ ಮಾಡಿಕೊಡುವ ಗ್ರಂಥ ಇನ್ನೂ ಪ್ರಕಟವಾಗಬೇಕಾಗಿದೆ” ಎಂದು ಹೇಳಿರುವುದು ನನ್ನ ಅಧ್ಯಯನಕ್ಕೆ ಇಂಬು ಕೊಟ್ಟಿತು. 

ಪಿಎಚ್​.ಡಿ. ಗೆ ರಿಜಿಸ್ಟರ್​ ಮಾಡಿಸಿದ ಒಂದು ವರ್ಷದ ಮೇಲೆ, ನಾವು ಕಲೋಕ್ವಿಯಂಗೆ (ವಿದ್ವತ್​ ಸಭೆಗೆ) ಹಾಜರಾಗಬೇಕೆಂದು, ಅಲ್ಲಿ ತಜ್ಞರು ನೀಡುವ ವರದಿಯ ಮೇಲೆ ಪಿಎಚ್​.ಡಿ. ಮುಂದುವರಿಸಲು ಅವಕಾಶ ನೀಡುವುದಾಗಿಯೂ ವಿಶ್ವವಿದ್ಯಾನಿಲಯದಿಂದ ಪತ್ರ ಬಂದಿತು. ಕಲೋಕ್ವಿಯಂಗೆ ಹಾಜರಾದೆ. ಅಲ್ಲಿ ಪ್ರೊ. ಮಲ್ಲೇಪುರಂ ವೆಂಕಟೇಶ್​ ಅವರು ಹಾಗೂ ಇನ್ನೊಂದಿಬ್ಬರು ತಜ್ಞರು ಇದ್ದರು. ನಾನು ನನ್ನ ವಿಷಯದ ಬಗ್ಗೆ ವಿವರಿಸಿದೆ. ಮಲ್ಲೇಪುರಂ ವೆಂಟೇಶ್​ ಅವರು ‘ಮಂಡ್ಯ ಜಿಲೆಯ ಶಾಸನಗಳು-ಒಂದು ಅಧ್ಯಯನ’ ಬಹಳ ವಿಸ್ತಾರವಾದ ವಿಷಯವಾಗುತ್ತದೆಂದು, ಅದರಲ್ಲಿ ಯಾವುದಾದರೂ ಒಂದು ವಿಶೇಷ ಭಾಗವನ್ನು ಆರಿಸಿಕೊಂಡು ಪಿಎಚ್​.ಡಿ. ಮಾಡಬೇಕೆಂದು ಹೇಳಿದರು. ಆದರೂ ನಾನು ಜಿಲ್ಲೆಯ ಶಾಸನಗಳ ಬಗ್ಗೆ ಸಮಗ್ರವಾಗಿ ಅಧ್ಯಯನ ಮಾಡಲು ಉದ್ದೇಶಿಸಿದ್ದೇನೆ ಎಂದು ಹೇಳಿದೆ. ಆ ರೀತಿ ಮಾಡುವುದಾದರೆ ನಮ್ಮದೇನೂ ಅಭ್ಯಂತರವಿಲ್ಲ ಎಂದು ವಿದ್ವತ್​ ಸಭೆಗೆ ನಿಗದಿಪಡಿಸಿಕೊಂಡಿದ್ದ 10 ಅಂಕಗಳಿಗೆ 8 ಅಂಕಗಳನ್ನು ನೀಡಿ ನನಗೆ ಅನುಮತಿ ನೀಡಿದರು. ಹೀಗೆ ಒಂದು ಒಂದೂವರೆ ವರ್ಷ ಕಳೆದು ಹೋಯಿತು. ನಂತರ ಅಧ್ಯಯನವನ್ನು ಪ್ರಾರಂಭಿಸಿ, ಬಿಡುವಿಲ್ಲದ ಕಚೇರಿ ಕೆಲಸಗಳ ನಡುವೆಯೇ ಅಧ್ಯಯನ ಮುಂದುವರಿಸಿದೆ. ಪ್ರತಿ ಆರು ತಿಂಗಳಿಗೊಮ್ಮೆ ಹೋಗಿ ಪ್ರಗತಿ ವರದಿ ಸಲ್ಲಿಸಿ, ನನ್ನ ಮಾರ್ಗದರ್ಶಕರಾದ ಪ್ರೊ. ಡಿ.ಟಿ. ಬಸವರಾಜ್​ ಅವರು ನನಗೆ ಹಾಗೂ ನನ್ನ ಜೊತೆ ಅವರ ಬಳಿ ಪಿಎಚ್​.ಡಿ. ಅಧ್ಯಯನ ಮಾಡುತ್ತಿದ್ದ ಇನ್ನೊಂದಿಬ್ಬರಿಗೆ, ಅವರ ಮುಂದೆಯೇ ಕೂರಿಸಿಕೊಂಡು, ಬರವಣಿಗೆ ಯಾವರೀತಿ ಮಾಡಬೇಕು, ಅಧ್ಯಾಯಗಳನ್ನು ಯಾವರೀತಿ ವಿಂಗಡಿಸಿಕೊಳ್ಳಬೇಕು, ಯಾವ ಪುಸ್ತಕಗಳನ್ನು ಪರಾಮರ್ಶೆ ಮಾಡಬೇಕು ಎಂಬುದನ್ನು ವಿವರವಾಗಿ ತಿಳಿಸುತ್ತಿದ್ದರು. ನಮ್ಮ ಬರವಣಿಗೆಯನ್ನು ಪರಿಶೀಲಿಸುತ್ತಿದ್ದರು. 

\newpage

ನಾನು ಆರಿಸಿಕೊಂಡ ವಿಷಯವು ಬಹಳ ವಿಸ್ತಾರವಾಗಿತ್ತು. ಅಧ್ಯಯನದ ಕಾಲದಲ್ಲಿ ನಮ್ಮ ಜಿಲ್ಲೆಯ ಪ್ರಮುಖ ಶಾಸನಗಳು ಮತ್ತು ಸ್ಮಾರಕಗಳಿರುವ ಸುಮಾರು 200 ಹಳ್ಳಿಗಳಿಗೆ ಭೇಟಿ ನೀಡಿದೆ. ನೂರಾರು ಛಾಯಾಚಿತ್ರಗಳನ್ನು ತೆಗೆದೆ. ನಾನು ತೊಳಸಿ, ಹರಿಹರಪುರ, ಸುಬ್ಬರಾಯನ ಕೊಪ್ಪಲಿಗೆ ಇನ್ನೊಂದು ಬಾರಿ ಹೋಗುವ ಹೊತ್ತಿಗೆ ಅಲ್ಲಿದ್ದ ದೇವಾಲಯಗಳನ್ನು ಕೆಡವಿ ಹಾಕಿ ಹೊಸ ದೇವಾಲಯ ನಿರ್ಮಿಸುತ್ತಿರುವುದು ಕಂಡು ಬಂದಿತು. ಜಿಲ್ಲೆಯ ಶಾಸನ ಸಂಪುಟಗಳ ಅಧ್ಯಯನ ಮಾಡುತ್ತಾ, ವಿಷಯಕ್ಕೆ ತಕ್ಕಂತೆ ಸಾವಿರಾರು ಚೀಟಿಗಳನ್ನು ಬರೆದುಕೊಂಡು, ಅದನ್ನೆಲ್ಲಾ ಕ್ರಮಬದ್ಧವಾಗಿ ಜೋಡಿಸಿ, ಮೊದಲು ವಿವರವಾಗಿ ಬರೆದುಕೊಂಡು, ನಂತರ ಅಗತ್ಯವಾದುದನ್ನು ಉಳಿಸಿಕೊಂಡು, ಸಂಕ್ಷೇಪವಾಗಿ ಮಾರ್ಪಡಿಸಿ ಬರೆಯುವ ಹೊತ್ತಿಗೆ ಸಾಕು ಬೇಕಾಗಿ ಹೋಗುತ್ತಿತ್ತು. ಹಾಗೂ ಹೀಗೂ, ಡಿ.ಟಿ.ಪಿ. ಮಾಡಿದ ಸುಮಾರು 1200 ಪುಟಗಳಷ್ಟು ಬರವಣಿಗೆ ಮಾಡಿಕೊಂಡೆ. ಈ ಕಾರ್ಯಕ್ಕೆ ಸುಮಾರು ಐದು ವರ್ಷ ಹಿಡಿಯಿತು. ಎಲ್ಲಾ ಆಗಿ ಕರಡು ಪ್ರತಿ ಸಿದ್ಧವಾದ ಮೇಲೆ ಸುಮಾರು ಒಂದು ತಿಂಗಳ ಕಾಲ, ನನ್ನ ಮಾರ್ಗದರ್ಶಕರಾದ ಪ್ರೊ. ಡಿ.ಟಿ.ಬಿ. ಅವರ ಮುಂದೆ ಕುಳಿತು, ಅವರು ಸೂಚಿಸಿದಂತೆ ಅದನ್ನು ಸರಿಪಡಿಸಿ, ಸಂಕ್ಷೇಪಿಸುವ ಕೆಲಸ ಮಾಡುತ್ತಾ ಬಂದೆ. 

ನನಗೆ ಶಾಸನ ಶಾಸ್ತ್ರ ತರಗತಿಯಲ್ಲಿ ಅಧ್ಯಾಪಕರಾಗಿದ್ದ ಡಾ. ಕೆ.ಆರ್​. ಗಣೇಶ್​ ಅವರು ಶಾಸನಗಳ ಅಧ್ಯಯನದಲ್ಲಿ, ಹಳಗನ್ನಡ ಸಾಹಿತ್ಯದಲ್ಲಿ ಪ್ರಕಾಂಡ ಪಂಡಿತರು. ನಾನು ಪಿಎಚ್​.ಡಿ. ಮಾಡುತ್ತಿರುವ ವಿಷಯ ಅವರಿಗೆ ತಿಳಿಸಿ, ಒಂದೊಂದು ಅಧ್ಯಾಯಗಳನ್ನು ಬರೆದ ನಂತರ ತಂದು ಒಪ್ಪಿಸುವುದಾಗಿಯೂ, ಅದನ್ನು ಪರಿಶೀಲಿಸಿ, ಸೂಕ್ತ ಸೂಚನೆಗಳನ್ನು ನೀಡುವಂತೆ ವಿನಂತಿ ಮಾಡಿಕೊಂಡೆ. ಅದಕ್ಕೆ ಅವರು ಸಂತೋಷದಿಂದ ಒಪ್ಪಿದರು. ಅವರಿಗೆ ನಾನು ಅಧ್ಯಾಯವಾರು ಬರೆದ ಬರವಣಿಗೆಯನ್ನು ತೆಗೆದುಕೊಂಡು ಹೋಗಿ ಕೊಡುತ್ತಿದ್ದೆ. ಹದಿನೈದು ಇಪ್ಪತ್ತು ದಿನದಲ್ಲಿ ಅವರು ಅದನ್ನು ನೋಡಿ, ಸೂಚನೆಗಳನ್ನು ಗುರುತು ಹಾಕಿ ಕೊಡುತ್ತಿದ್ದರು. ಕೊನೆಗೆ ಹೆಚ್ಚಿನ ವಿಷಯಗಳನ್ನು ಪ್ರಬಂಧದ ಹೊರಗಿಟ್ಟು, ಅನುಬಂಧ ರೂಪದಲ್ಲಿ ಸೇರಿಸುವಂತೆ ಹೇಳಿದರು. ನಾನು ಈ ಕ್ಷೇತ್ರದವನಲ್ಲವಾದುದರಿಂದ ನನಗೆ ಅಳುಕು. ಪ್ರತಿ ಬಾರಿಯೂ ಡಾ. ಕೆ.ಆರ್​. ಗಣೇಶ್​ ಅವರನ್ನು ಸರ್​, ಇದು ಪಿಎಚ್​.ಡಿ. ಪ್ರಬಂಧಕ್ಕೆ ಅರ್ಹವಾಗಿದೆಯೇ, ಇದರಲ್ಲಿ ಏನಾದರೂ ಸತ್ವ ಇದೆಯೇ ಎಂದು ಕೇಳುತ್ತಿದ್ದೆ. ಅವರು ಇದೆ. ಮಾಡಿ ಎಂದು ಹೇಳುತ್ತಿದ್ದರು.

ಆದರೂ ಪ್ರಬಂಧ ಬಹಳ ದೊಡ್ಡದಾಯಿತು. ಮೂಲ ಪ್ರಬಂಧವು ಸುಮಾರು 800 ಪುಟಗಳಷ್ಟಾಯಿತು. ಅನುಬಂಧಗಳು ಸುಮಾರು 140 ಪುಟಗಳಾಯಿತು. ನನ್ನ ಮಾರ್ಗದರ್ಶಕರಾದ ಪ್ರೊ. ಡಿ.ಟಿ.ಬಿ. ಅವರಿಗೆ ಬಹಳ ದೊಡ್ಡದಾಯಿತು ಸರ್​ ಎಂದೆ. ಆಗಲಿ, ವಿಷಯವನ್ನು ವಿವರವಾಗಿ ತಿಳಿಸಬೇಕು. 200-300 ಪುಟದಲ್ಲಿ ಜಿಲ್ಲೆಯ ಶಾಸನಗಳ ಬಗ್ಗೆ ಏನೂ ಹೇಳಲು ಸಾಧ್ಯವಿಲ್ಲ. ವಿವರವಾಗಿಯೇ ಇರಲಿ, ನಾನು ಸಬ್​ಮಿಟ್​ ಮಾಡಿಸುತ್ತೇನೆ. ಎಂದು ಧೈರ್ಯ ನೀಡಿ ಹುರಿದುಂಬಿಸಿದರು. ಎರಡು ಸಂಪುಟಗಳಲ್ಲಿ ಬೈಂಡ್​ ಮಾಡಿಸಿದೆ. ಸಬ್​ಮಿಟ್​ ಮಾಡಿದ ಒಂದು ವರ್ಷದ ನಂತರ, ವೈವಾ ನಡೆಯಿತು. ಈಗ ಮೈಸೂರು ವಿಶ್ವವಿದ್ಯಾನಿಲಯದ ಕನ್ನಡ ಅಧ್ಯಯನ ಸಂಸ್ಥೆಯಲ್ಲಿ ಪ್ರೊಫೆಸರ್​ ಆಗಿರುವ ಶಾಸನ ಶಾಸ್ತ್ರ ಪರಿಣತರಾದ, ಡಾ. ಎಂ. ಜಿ. ಮಂಜುನಾಥ್​ ಅವರು ವೈವಾಗೆ ಬಂದಿದ್ದರು. ಅವರು ನನ್ನ ಪ್ರಬಂಧವನ್ನು ಮೌಲ್ಯಮಾಪನ ಮಾಡಿದ್ದಾರೆಂದು ತಿಳಿದು ನನಗೆ ಬಹಳ ಧೈರ್ಯ ಬಂತು, ಸಂತೋಷವಾಯಿತು. ಕಾರಣ ಶಾಸನ ಶಾಸ್ತ್ರ ತರಗತಿಯಲ್ಲಿ ನಾನು ಅವರ ಕೃತಿಗಳನ್ನು, ಸಂಶೋಧನಾ ಲೇಖನಗಳನ್ನು ಬಹಳವಾಗಿ ಓದಿದ್ದೆ. ಹೀಗೆ ಆರು ವರ್ಷಗಳ ಅಧ್ಯಯನದ ನಂತರ, ನನಗೆ ಕರ್ನಾಟಕ ರಾಜ್ಯ ಮುಕ್ತ ವಿಶ್ವವಿದ್ಯಾನಿಲಯದಿಂದ ಪಿಎಚ್​.ಡಿ. ಪದವಿ ಪ್ರದಾನವಾಯಿತು. 

ಶಾಸನ ಶಾಸ್ತ್ರದ ಬಗ್ಗೆ ನನಗೆ ಆಸಕ್ತಿ ಮೂಡಲು, ನಾನು ಶಾಸನಗಳ ವಿಷಯದಲ್ಲಿ ಪಿಎಚ್​.ಡಿ. ಅಧ್ಯಯನ ಮಾಡಲು, ಕನ್ನಡ ಸಾಹಿತ್ಯ ಪರಿಷತ್ತಿನ, ಶಾಸನ ಶಾಸ್ತ್ರ ತರಗತಿಗಳಲ್ಲಿ ನಮಗೆ ಬೋಧನೆ ಮಾಡಿದ, ಈ ಕ್ಷೇತ್ರದ ಮೇರು ಪ್ರತಿಭೆಗಳಾದ, ಡಾ. ಕೆ.ಆರ್​.ಗಣೇಶ್​, ಡಾ. ದೇವರಕೊಂಡಾ ರೆಡ್ಡಿ, ಡಾ.ಪಿ.ವಿ. ಕೃಷ್ಣಮೂರ್ತಿ, ಪ್ರೊ. ಲಕ್ಷ್ಮಣ ತೆಲಗಾವಿ, ಡಾ.ಎಸ್​.ಎಲ್​. ಶ‍್ರೀನಿವಾಸಮೂರ್ತಿ, ಡಾ. ಎಚ್​.ಎಸ್​. ಗೋಪಾಲರಾವ್​, ಹಾಗೂ ಪರಿಷತ್ತಿನ ಶಾಸನ ಶಾಸ್ತ್ರ ವಿಭಾಗದ ಮುಖ್ಯಸ್ಥರಾಗಿದ್ದ, ಡಾ.ನಾಗರಾಜ್​ ಅವರು ಕಾರಣ ಕರ್ತರು. ಅವರುಗಳು ಈ ವಿಷಯದ ಬಗ್ಗೆ ಆಮೂಲಾಗ್ರವಾಗಿ ನಮಗೆ ಅಭಿರುಚಿ ಹುಟ್ಟುವಂತೆ ಬೋಧನೆ ಮಾಡಿದರು. ನಮಗೆ ಪ್ರೋತ್ಸಾಹ ನೀಡಿದರು. ಈ ಗುರುಪರಂಪರೆಗೆ ನನ್ನ ಅನಂತ ನಮನಗಳು. 

\newpage

ಪಿಎಚ್​.ಡಿ ಮಾಡಿ ಆಯಿತು. ಪಿಎಚ್​.ಡಿ. ಪ್ರಬಂಧವನ್ನು ಆಗಾಗ್ಗೆ ಮತ್ತೆ ಮತ್ತೆ ಓದುತ್ತಿದ್ದ ನನಗೆ, ಇಷ್ಟೊಂದು ಬರೆಯಬೇಕಾಗಿತ್ತೆ, ಬರೆದುದು ಸರಿಯೇ, ಎಂದು ಒಂದೊಂದು ಸಲ ಅನಿಸುತ್ತದೆ. ಆದರೆ ನಮ್ಮ ಮಂಡ್ಯ ಜಿಲ್ಲೆಯ ಶಾಸನಗಳ ಬಗ್ಗೆ ಸುಮಾರಾಗಿ ಎಲ್ಲ ವಿಷಯಗಳೂ ಒಂದೆಡೆ ಶಾಸ್ತ್ರಬದ್ಧವಾಗಿ ದೊರೆಯುತ್ತವೆ ಎಂಬ ಸಮಾಧಾನ ಒಂದು ಕಡೆ ಇದೆ. ಮಂಡ್ಯ ಜಿಲ್ಲೆಯ ಬಗ್ಗೆ, ಜಿಲ್ಲೆಯ ಶಾಸನಗಳ ಬಗ್ಗೆ ಇದುವರೆಗೆ ಬಂದಿರುವ ಅನೇಕ ಗ್ರಂಥಗಳನ್ನು, ಕೃತಿಗಳನ್ನು ಅಧ್ಯಯನ ಮಾಡಿ, ಈ ಪ್ರಬಂಧದಲ್ಲಿ ಉಲ್ಲೇಖಿಸಿದ್ದೇನೆ, ಎಂಬ ತೃಪ್ತಿ ಇನ್ನೊಂದು ಕಡೆ ಇದೆ. 

ನಾನು ಅಧ್ಯಯನ ಹಾಗೂ ಬೋಧನಾ ಕ್ಷೇತ್ರದವನಲ್ಲ. ನಾನು ಸರ್ಕಾರಿ ಕಚೇರಿಯಲ್ಲಿ ಕೆಲಸ ಮಾಡಿಕೊಂಡಿದ್ದವನು. ಅಲ್ಲಿ ಸಾಹಿತ್ಯದ, ಅಧ್ಯಯನದ ಸೆಲೆ ಬತ್ತಿ ಹೋಗಿದೆ ಎನಿಸುತ್ತದೆ. ಎಲ್ಲೋ ಡಾ. ಎಚ್​.ಎಸ್​. ಗೋಪಾಲರಾಯರಂತಹ ಕೆಲವರು ಈ ಶಾಸನ ಕ್ಷೇತ್ರ ಹಾಗೂ ಸಾಹಿತ್ಯ ಅಧ್ಯಯನದ ಬಗ್ಗೆ, ಕೃತಿ ರಚನೆಯ ಬಗ್ಗೆ ಆಸಕ್ತರು, ಸಾವಿರಕ್ಕೆ ಒಬ್ಬರು ಸಿಗುವುದು ಕಷ್ಟ. ನನ್ನ ಓದಿನಿಂದ, ನನ್ನ ಪದವಿಯಿಂದ ಅಲ್ಲಿ ಯಾವುದೇ ಉಪಯೋಗವಿಲ್ಲ. ನನ್ನ ಸ್ವಸಂತೋಷಕ್ಕೆ ನಾನು ಓದುತ್ತಿದ್ದೇನೆ ಬರೆಯುತ್ತಿದ್ದೇನೆ. ಇಡೀ ಕರ್ನಾಟಕ ವಿಧಾನ ಮಂಡಲ ಕಾರ್ಯಾಲಯದ ಇತಿಹಾಸದಲ್ಲಿ ಮೊದಲಬಾರಿಗೆ ಪಿಎಚ್​.ಡಿ. ಪಡೆದವನು ನಾನು. ಈ ವಿಷಯ ತಿಳಿದ ಕರ್ನಾಟಕ ವಿಧಾನ ಪರಿಷತ್ತಿನ ಸದಸ್ಯರಾಗಿದ್ದ ಶ‍್ರೀ ಜಿ. ಮಧುಸೂದನ್​ ಅವರು, ನನ್ನನ್ನು ಕರೆದು ಅಭಿನಂದಿಸಿದರು. ಆಗ ವಿಧಾನ ಪರಿಷತ್ತಿನ ಅಧಿವೇಶನ ನಡೆಯುತ್ತಿತು. ಅವರು ಕರ್ನಾಟಕ ವಿಧಾನ ಪರಿಷತ್ತಿನ ಸನ್ಮಾನ್ಯ ಸಭಾಪತಿಯವರಾದ ಶ‍್ರೀ ಡಿ.ಎಚ್​. ಶಂಕರಮೂರ್ತಿಯವರಿಗೆ ವಿಷಯ ತಿಳಿಸಿ, ನಮ್ಮ ಕಾರ್ಯಾಲಯದ ಒಬ್ಬ ನೌಕರ ಕನ್ನಡ ಶಾಸನ ಕ್ಷೇತ್ರದಲ್ಲಿ ಪಿಎಚ್​.ಡಿ. ಮಾಡಿದ್ದಾನೆ, ಅವನನ್ನು ಅಭಿನಂದಿಸಬೇಕು ಎಂದು ಹೇಳಿದರು. ಗೌರವಾನ್ವಿತ ಸಭಾಪತಿಯವರಾಗಿದ್ದ ಡಿ.ಎಚ್​. ಶಂಕರ ಮೂರ್ತಿಯವರೂ, ಸಾಹಿತ್ಯ ಸಂಸ್ಕೃತಿಯಲ್ಲಿ ಆಸಕ್ತರು. ಅವರು ನನ್ನನ್ನು ಅಭಿನಂದಿಸಿ ಒಂದು ಅಭಿನಂದನಾ ನಿರ್ಣಯವನ್ನು ವಿಧಾನ ಪರಿಷತ್ತಿನ ಅಧಿವೇಶನದಲ್ಲಿ ಮಂಡಿಸುವಂತೆ ಶ‍್ರೀ ಜಿ. ಮಧುಸೂದನ್​ಗೆ ಸೂಚಿಸಿದರು. ಹೀಗೆ ನನ್ನ ಬಗ್ಗೆ ಒಂದು ಅಭಿನಂದನಾ ನಿರ್ಣಯವು ವಿಧಾನ ಪರಿಷತ್ತಿನಲ್ಲಿ ಮಂಡಿಸಲ್ಪಟ್ಟಿತು. ನನ್ನಂತಹ ಸಾಧಾರಣ ನೌಕರನ ಒಂದು ಸಣ್ಣ ಅಧ್ಯಯನವನ್ನು ಗಮನಿಸಿ, ಪ್ರೋತ್ಸಾಹಿಸಿದ ಹಿರಿಯರ ಮನೆಯ, ಹಿರಿಯರಾದ ವಿಧಾನ ಪರಿಷತ್ತಿನ ಗೌರವಾನ್ವಿತ ಸಭಾಪತಿಯವರಾಗಿದ್ದ ಶ‍್ರೀ ಡಿ.ಎಚ್​. ಶಂಕರಮೂರ್ತಿಯವರಿಗೆ, ವಿಧಾನ ಪರಿಷತ್ತಿನ ಸನ್ಮಾನ್ಯ ಸದಸ್ಯರಾಗಿದ್ದ ಶ‍್ರೀ ಜಿ. ಮಧುಸೂದನ್​ ಅವರಿಗೆ ನಾನು ವಿನಯಪೂರ್ವಕವಾಗಿ ನಮಿಸಿ, ಅವರಿಗೆ ಆಭಾರಿಯಾಗಿದ್ದೇನೆ. ಈ ಹಿಂದೆ ಸಭಾಪತಿಯವರಾಗಿದ್ದ ಗುಲಬರ್ಗಾ ಜಿಲ್ಲೆಯ ಅಳಂದ್​ನವರಾದ ದಿವಂಗತ ಸನ್ಮಾನ್ಯ ಶ‍್ರೀ ಡಿ.ಬಿ. ಕಲ್ಮಣಕರ್​ ಅವರು ನನ್ನ ಬರವಣಿಗೆಗೆ ಪ್ರೋತ್ಸಾಹಿಸಿದರು. ನಾನು ಮೊದಲನೆಯದಾಗಿ ಬರೆದ ‘ದೇಶಪ್ರೇಮಿ ಧೀರಸೇನಾನಿ’ ಎಂಬ ಸುಭಾಶ್​ ಚಂದ್ರ ಬೋಸ್​ರವರ ಸಂಕ್ಷಿಪ್ತ ಜೀವನ ಚರಿತ್ರೆಯು, ಅಗಡಿ ಆನಂದವನ ಪ್ರಕಟಣೆಯಾಗಿ ಪ್ರಕಟಗೊಳ್ಳುವಂತೆ ಮಾಡಿದರು. ಹೈದರಾಬಾದ್​ ಕರ್ನಾಟಕದ ಹಿರಿಯ ಧೀಮಂತ ಸ್ವಾತಂತ್ರ್ಯ ಹೋರಾಟಗಾರರಾಗಿದ್ದ, ಶ‍್ರೀ ದಿಗಂಬರರಾವ್​ ಕಲ್ಮಣಕರ್​ ಅವರ ನೆನಪು ನನ್ನಲ್ಲಿ ಅಮರ.

ಹೀಗೆ ಬರೆದು ಪಿಎಚ್​.ಡಿ. ಪ್ರಬಂಧವನ್ನು ಇಟ್ಟುಕೊಂಡು ಕುಳಿತಿದ್ದೆ. ಪ್ರಕಟಣೆ ಸಾಧ್ಯವಿಲ್ಲದ ಮಾತಾಗಿತ್ತು. ಹಿರಿಯ ವಿದ್ವಾಂಸರಾದ, ಶಾಸನ ಕ್ಷೇತ್ರದ ಅಧ್ವರ್ಯುಗಳಾದ ಡಾ. ಎಮ್. ಚಿದಾನಂದ ಮೂರ್ತಿಯವರು ನನ್ನಂತಹ ಕಿರಿಯ ಅಲ್ಪಮತಿಯ ಪ್ರಬಂಧವನ್ನು ನೋಡಿ, ಪರವಾಗಿಲ್ಲ ಎಂದು ಬೆನ್ನು ತಟ್ಟಿದರು, ಪ್ರಕಟಿಸಬೇಕು, ಸುಮ್ಮನೆ ಇಟ್ಟುಕೊಂಡು ಕುಳಿತಿದ್ದರೆ ಏನೂ ಪ್ರಯೋಜವಿಲ್ಲ ಎಂದಿದ್ದರು. 

ಕಾಮಧೇನು ಪ್ರಕಾಶನದವರ ಪುಸ್ತಕದ ಅಂಗಡಿ ಶಾಸಕರ ಭವನದಲ್ಲಿ ಪ್ರಾರಂಭವಾಯಿತು. ನಾನು ಆಗಾಗ್ಗೆ ಹೋಗಿ ಅಲ್ಲಿ ಪುಸ್ತಕಗಳನ್ನು ಖರೀದಿಸುತ್ತಿದೆ. ಕನ್ನಡ ಸಾಹಿತ್ಯ ಲೋಕದ ಘಟಾನುಘಟಿಗಳು ಅಲ್ಲಿ ಬಂದು ಕುಳಿತುಕೊಂಡು, ಶಾಮಸುಂದರ ರಾಯರೊಡನೆ ಮಾತನಾಡುತ್ತಿದ್ದರು. ಕನ್ನಡ ಸಾಹಿತ್ಯ ಲೋಕದ ಹಿರಿಯ ತಲೆಗಳಾದ ಅ.ರಾ.ಮಿತ್ರ, ಚಿ.ಶ‍್ರೀನಿವಾಸರಾಜು, ಬೋಳುವಾರು ಮಹಮ್ಮದ್​, ಪ್ರೊ. ಲೋಹಿತಾಶ್ವ, ನ್ಯಾ. ಕೊ.ಚನ್ನಬಸಪ್ಪ ಡಾ. ಸಿದ್ಧಲಿಂಗಯ್ಯ, ಜಿ. ಕೃಷ್ಣಪ್ಪ, ಆರಾಸೆ, ಮೊದಲಾದವರನ್ನು ನಾನು ಅಲ್ಲೇ ಕಂಡದ್ದು. ಇಷ್ಟೊಂದು ಜನ ಸಾಹಿತಿಗಳು ಇವರನ್ನು ಬಂದು ಭೇಟಿಯಾಗುತ್ತಾರೆಂದರೆ, ಶಾಮಸುಂದರ ರಾಯರು ಅಂತಿಂಥವರಲ್ಲ, ಬಹಳ ಸತ್ವಶಾಲಿಗಳಿರಬೇಕು ಎನಿಸಿತು. ಶಾಮಸುಂದರರಾಯರು ಬರೆದಿದ್ದ ‘ಹಿಮಾಲಯದ ಮಹಾತ್ಮರ ಸನ್ನಿಧಿಯಲ್ಲಿ’ ಕೃತಿಯನ್ನು ಅನೇಕ ಸಾರಿ ಓದಿದ್ದೇನೆ. ಇದರಿಂದ ಅವರ ಮೇಲೆ ನನಗೆ ಪೂಜ್ಯತೆಯ ಭಾವನೆ ಬಂದಿದೆ. ಹತ್ತಾರು ಪ್ರತಿಗಳನ್ನು ಖರೀದಿಸಿ ಅಮೆರಿಕೆಯಲ್ಲಿರುವ ಹಾಗೂ ಇಲ್ಲಿರುವ ನಮ್ಮ ಬಂಧುಬಾಂಧವರಿಗೆ ಮಿತ್ರರಿಗೆ ಉಡುಗೊರೆಯಾಗಿ ನೀಡಿದ್ದೇನೆ. ಶಾಮಸುಂದರ ರಾಯರು ಕಾಲೇಜು ಶಿಕ್ಷಣ ಇಲಾಖೆಯಲ್ಲಿ ರೀಡರ್​ ಆಗಿ ಸೇವೆ ಸಲ್ಲಿಸಿ ನಿವೃತ್ತರಾಗಿದ್ದಾರೆಂದಾಗ ನನಗೆ ಅವರ ಬಗ್ಗೆ ಗೌರವ ಮೂಡಿತು. ಹೀಗಿದ್ದಾಗ ಒಂದು ದಿನ ಕಾಮಧೇನು ಪ್ರಕಾಶನದ ಶ‍್ರೀ ಡಿ.ಕೆ. ಶಾಮಸುಂದರರಾವ್​ ಅವರು ನಿಮ್ಮ ಪ್ರಬಂಧ ಏನು ಮಾಡುತ್ತೀರಿ, ಯಾರಿಗಾದರೂ ಪ್ರಕಟಣೆಗೆ ಕೊಟ್ಟಿದ್ದೀರಾ, ನಮ್ಮ ಪ್ರಕಾಶನದಿಂದ ಪ್ರಕಟಣೆ ಮಾಡುತ್ತೇನೆ ಕೊಡಿ ಎಂದಾಗ, ನನಗೆ ನಂಬಲು ಆಗಲೇ ಇಲ್ಲ. ಕಾಮಧೇನು ಪ್ರಕಾಶನದ ಕೃತಿಗಳನ್ನು ಕೊಳ್ಳುವ ಪರಿಪಾಠ ನನಗೆ ಹಿಮಾಲಯದ ಮಹಾತ್ಮರ ಸನ್ನಿಧಿಯಲ್ಲಿ ಕೃತಿಯಿಂದ ಆರಂಭವಾಯಿತು. ಕಾಮಧೇನು ಪ್ರಕಟಣೆಗಳು ಬಹಳ ಮೌಲಿಕವು ಸುಂದರವೂ ಆಗಿವೆ. ಕರ್ನಾಟಕದ ಇತಿಹಾಸ ಸಂಸ್ಕೃತಿಗೆ ಸಂಬಂಧಿಸಿದ ಹತ್ತಾರು ಪ್ರಕಟಣೆಗಳನ್ನು ಅವರು ಹೊರತಂದಿದ್ದಾರೆ. ಅದರಲ್ಲಿ ಇತಿಹಾಸ ತಜ್ಞರಾದ ಡಾ. ಎಸ್​. ಶ‍್ರೀಕಂಠಶಾಸ್ತ್ರೀ ಅವರ ಪ್ರಕಟಣೆಗಳು ಮಹತ್ತರವಾದವು. ಸಾಹಿತ್ಯಕ್ಕೆ ಸಂಬಂಧಿಸಿದಂತೆ, ಪಂಪಭಾರತ, ಕುಮಾರವ್ಯಾಸ ಭಾರತ, ಜೈಮಿನಿ ಭಾರತ, ನಾಗಚಂದ್ರನ ರಾಮಾಯಣ, ಮೊದಲಾದವುಗಳನ್ನು ಛಂದಸ್ಸು, ವ್ಯಾಕರಣ ಕೃತಿಗಳನ್ನೂ ಅವರು ಹೊರತಂದಿದ್ದಾರೆ. ಈ ಕಾಲದಲ್ಲಿ ಇವುಗಳನ್ನು ಕೊಂಡು ಓದುವವರು ಕಡಿಮೆ. ಆದರೂ ಶಾಮಸುಂದರ ರಾಯರು ಕನ್ನಡ ಸಾಹಿತ್ಯದ ಮೇಲಿನ, ನಾಡು ನುಡಿಯ ಮೇಲಿನ ಪ್ರೀತಿಯಿಂದ ಇವುಗಳನ್ನು ಹೊರ ತಂದಿದ್ದಾರೆ. ಅವರು ಪ್ರಕಟಿಸಿರುವ ಧಾರ್ಮಿಕ ಕೃತಿಗಳು, ಧಾರ್ಮಿಕ ಪದಕೋಶಗಳು, ಕನ್ನಡ ಭಾಷೆಯ ಸಂದರ್ಭದಲ್ಲಿ ಬಹಳ ಮಹತ್ವವನ್ನು ಪಡೆದಿವೆ. ಅವರು ಪ್ರಕಟಿಸಿರುವ ಕೃತಿಗಳಿಗೆ ಪುಸ್ತಕ ಪ್ರಾಧಿಕಾರದ ಬಹುಮಾನಗಳೂ ಬಂದಿವೆ. ಶಾಮಸುಂದರರಾಯರ ಪತ್ನಿ ಶ‍್ರೀಮತಿ ಮುದ್ದಮ್ಮನವರು ನಮ್ಮ ಕಚೇರಿಯಲ್ಲೇ ಅಧಿಕಾರಿಯಾಗಿ ಡೆಪ್ಯುಟೇಷನ್​ ಮೇಲೆ ಸೇವೆ ಸಲ್ಲಿಸುತ್ತಿದ್ದರು. ಅವರ ಪರಿಚಯವೂ ಆಯಿತು. ಅವರೂ ಕೂಡಾ ಸಹೃದಯರು, ಸಾಹಿತ್ಯ ಪ್ರಿಯರು. 

ಅಂತಹವರು ನನ್ನಂತಹ ಒಬ್ಬ ಅಜ್ಞಾತನ, ಈ ಕಾಲದಲ್ಲಿ ಯಾವುದೇ ಬೇಡಿಕೆ ಇಲ್ಲದ, ಸಂಶೋಧನಾ ಪ್ರಬಂಧವನ್ನು ಪ್ರಕಟಿಸುತ್ತೇವೆಂದು ಹೇಳಿದಾಗ ನನಗೆ ಆಶ್ಚರ್ಯವಾಯಿತು. ಅವರು ಹೇಳಿ ಸುಮಾರು ಒಂದು, ಒಂದೂವರೆ ವರ್ಷದ ಮೇಲಾಯಿತು. ನಾನು ನನ್ನ ಪ್ರಬಂಧವನ್ನು ಸಂಕ್ಷಿಪ್ತಗೊಳಿಸಿ, ಪರಿಷ್ಕರಿಸಲು, ಮತ್ತೆ ನನ್ನ ಗುರುಗಳಾದ ಡಾ.ಕೆ.ಆರ್​. ಗಣೇಶ್​ ಅವರಲ್ಲಿಗೆ ಹೋಗಿ, ಪ್ರಕಟಣೆಗೆ ಯಾವರೀತಿ ಸಿದ್ಧಪಡಿಸಬೇಕೆಂದು ಕೇಳಿದೆ. ಅವರು ನನ್ನ ಪ್ರಬಂಧವನ್ನು ಇನ್ನೊಮ್ಮೆ ನೋಡಿ, ಯಾವುದನ್ನು ಪರಿಷ್ಕರಿಸಬೇಕು, ಯಾವುದನ್ನು ಕೈಬಿಡಬಹುದು ಎಂದು ವಿವರವಾಗಿ ತಿಳಿಸಿ ಗುರುತು ಹಾಕಿಕೊಟ್ಟರು. ಈ ಕೆಲಸವನ್ನು ಮಾಡಲು ಸುಮಾರು ಆರು ತಿಂಗಳ ಕಾಲ ಆಯಿತು. 

ಈಗ ನನ್ನ ಪ್ರಬಂಧದ ಪೂರ್ಣ ಪರಿಷ್ಕೃತ ರೂಪವು ಪ್ರಕಟವಾಗುತ್ತಿದೆ. ನನ್ನ ವಿದ್ಯಾಭ್ಯಾಸ ಯಾವ ಯಾವ ಕಡೆಗೋ ಹೋಗಿ, ಕೊನೆಗೆ ನನ್ನ ಜಿಲ್ಲೆಯ ಬಗ್ಗೆ ಪಿಎಚ್​.ಡಿ. ಮಾಡುವುದರೊಂದಿಗೆ, ಈಗ ನನ್ನ ಪಿಎಚ್​.ಡಿ. ಪ್ರಬಂಧದ ಪರಿಷ್ಕೃತ ರೂಪ ಪ್ರಕಟವಾಗುವುದರೊಂದಿಗೆ ಒಂದು ಹಂತಕ್ಕೆ ಬಂದು ನಿಂತಿದೆ. 

ಕರಡಚ್ಚನ್ನು ತಿದ್ದುವಲ್ಲಿ ನನ್ನ ಆತ್ಮೀಯ ಮಿತ್ರರಾದ ಡಾ. ಕೆ.ವಿ. ಅನಂತಪದ್ಮನಾಭ ಇವರು ಪೂರ್ಣವಾಗಿ ಸಹಕರಿಸಿದ್ದಾರೆ. ನನ್ನ ಅಧ್ಯಯನದಲ್ಲಿ, ಬರವಣಿಗೆಯಲ್ಲಿ ಯಾವತ್ತೂ, ಮನೆಯ ಕಡೆ ನಾನು ಮಾಡಬೇಕಾಗಿದ್ದ ಕೆಲಸಗಳೆಲ್ಲವನ್ನೂ ನೋಡಿಕೊಂಡು, ನನ್ನ ಪತ್ನಿ ರಾಜೇಶ್ವರಿ, ನನ್ನ ಮಕ್ಕಳು ಚಂದ್ರಮೌಳಿ, ಯಶಸ್ವಿನಿ, ನನ್ನ ಸೊಸೆ ಸುಮನಾ ಪೂರ್ತಿ ಸಹಕಾರ ನೀಡಿದ್ದಾರೆ. ಅವರಿಗೆ ನಾನು ಆಭಾರಿಯಾಗಿದ್ದೇನೆ. 

ನಾನು ಶಿಕ್ಷಣ ಕ್ಷೇತ್ರದವನಲ್ಲ. ಎಲ್ಲಿಂದಲೋ ಬಂದು ಆಕಸ್ಮಿಕವಾಗಿ, ಶಾಸನ ಅಧ್ಯಯನದ ಕ್ಷೇತ್ರಕ್ಕೆ ಸೇರಿದವನು. ನನ್ನ ಬರವಣಿಗೆಯಲ್ಲಿ, ವಿಷಯ ಮಂಡನೆಯಲ್ಲಿ ಅನೇಕ ಲೋಪದೊಷಗಳಿರಬಹುದು. ಆದರೂ ನನಗೆ ಶಾಸನ ಮತ್ತು ಸಂಸ್ಕೃತಿಯ ಬಗ್ಗೆ ಇರುವ ಪ್ರೀತಿಯಿಂದ ಅಭಿಮಾನದಿಂದ ಹೇಗೋ ಈ ಒಂದು ಬರವಣಿಗೆಯನ್ನು ಮಾಡಿ, ಶಾಮಸುಂದರ ರಾಯರ ಕೃಪೆಯಿಂದ ಪ್ರಕಟಿಸಿ ಜನತೆಯ ಮುಂದಿಡುತ್ತಿದ್ದೇನೆ. ಈ ಕ್ಷೇತ್ರದ ಸಂಶೋಧಕರು, ವಿದ್ವಾಂಸರು, ಆಸಕ್ತರು, ಸಾಮಾನ್ಯ ಜನರು, ಅದರಲ್ಲೂ, ನನ್ನ ಜಿಲ್ಲೆಯ ಜನರು, ನನ್ನ ಈ ಕೃತಿಯನ್ನು ಪ್ರೋತ್ಸಾಹಿಸಬೇಕೆಂದು ಪ್ರಾರ್ಥಿಸುತ್ತೇನೆ. ನಮ್ಮ ಮಾತೃಭಾಷೆಗೆ, ನಮ್ಮ ರಾಜ್ಯದ, ಜಿಲ್ಲೆಯ, ಸಾಂಸ್ಕೃತಿಕ ಕ್ಷೇತ್ರಕ್ಕೆ ಕಿಂಚಿತ್​ ಸೇವೆ ಸಲ್ಲಿಸಿದ ಧನ್ಯತೆ ನನ್ನಲ್ಲಿದೆ.

ಇನ್ನು ಇದರ ಮುದ್ರಣ ಕಾರ್ಯ ಆದಷ್ಟು ಬೇಗ ಆಗಿ ಪುಸ್ತಕ ಹೊರಬಂದರೆ ಒಳ್ಳೆಯದು ಎಂದು\break ಶಾಮಸುಂದರರಾಯರು ಹೇಳಿದರು. ಅದರಂತೆ ನನಗೆ ಸುಮಾರು 25 ವರ್ಷಗಳಿಂದ ಪರಿಚಯ ಇರುವ ಪರಿಮಳ ಮುದ್ರಣಾಲಯದ ಒಡೆಯರಾದ ಶ‍್ರೀ ಕೆ.ಸಿ. ಪ್ರಭಾಕರ್​, ನಿರ್ವಾಹಕರಾದ ಶ‍್ರೀ ಶ‍್ರೀಕಾಂತ್​ ಅವರು ತಮ್ಮ, ಬಿಡು\break ವಿಲ್ಲದ ಅಗಾಧ ಕೆಲಸದ ನಡುವೆಯೂ, ನನ್ನ ಒತ್ತಡವನ್ನು ತಡೆದುಕೊಂಡು ಈ ಗ್ರಂಥವನ್ನು ಸುಂದರವಾಗಿ ಮುದ್ರಣ ಮಾಡಿಸಿಕೊಟ್ಟಿದ್ದಾರೆ.

ನನ್ನ ಪಿಎಚ್​.ಡಿ. ಪ್ರಬಂಧವನ್ನು ವಿಶ್ವವಿದ್ಯಾನಿಲಯಕ್ಕೆ ಸಲ್ಲಿಸುವಾಗ, ಈಗ ಆರು ವರ್ಷಗಳ ಹಿಂದೆ, ಅತ್ಯಂತ ಸುಂದರವಾಗಿ ಡಿ.ಟಿ.ಪಿ. ಮಾಡಿಸಿ, ಛಾಯಾಚಿತ್ರಗಳನ್ನು ಅಳವಡಿಸಿ, ಬೈಂಡ್​ ಮಾಡಿಸಿಕೊಟ್ಟವರು, ಶ‍್ರೀರಂಗಪಟ್ಟಣದ, ಶ‍್ರೀರಂಗ ಡಿಜಿಟಲ್​ ಟೆಕ್ನಾಲಜೀಸ್​ನ ಮಾರ್ಗದರ್ಶಕರಾದ ಪ್ರೊ. ಸಿ.ಎಸ್​. ಯೋಗಾನಂದ್​ ಮತ್ತು ವ್ಯವಸ್ಥಾಪಕರಾದ ಶ‍್ರೀ ಅರ್ಜುನ್​ ಕಶ್ಯಪ್ ಅವರು. ಆಗ ಅವರೇ ಮುಂದೆ ನಿಂತು ಎಲ್ಲ ಕಾರ್ಯಗಳನ್ನು ಸುಸೂತ್ರವಾಗಿ ಆಗುಮಾಡಿಸಿಕೊಟ್ಟರು. ಈಗಲೂ ಸಹ, ಈ ಗ್ರಂಥವನ್ನು ಅಲ್ಪ ಸಮಯದಲ್ಲಿ, ಕ್ಷಿಪ್ರವಾಗಿ ಸಿದ್ಧಪಡಿಸಿಕೊಡುವುದರಲ್ಲಿ ಅವರುಗಳ ಪಾತ್ರ ಅಪಾರವಾಗಿದೆ. ಕೇಳಿದ ತಕ್ಷಣ ಆಯಿತು ತೆಗೆದುಕೊಂಡು ಬನ್ನಿ ಮಾಡಿಕೊಡೋಣ ಎಂದರು. ಇಂತಹ ವ್ಯಕ್ತಿಗಳು ಅಪರೂಪ. ಅವರ ಮುಂದಿದ್ದ ಅಪಾರವಾದ ಕೆಲಸದ ನಡುವೆಯೂ, ಸಮಯಾವಕಾಶ ಕಲ್ಪಿಸಿಕೊಂಡು ಇದನ್ನು ಮಾಡಿಕೊಟ್ಟಿದ್ದಾರೆ. ಅವರಿಗೆ ನಾನು ಚಿರಋಣಿಯಾಗಿದ್ದೇನೆ.

\begin{flushright}
\textbf{ಡಾ.ಎಸ್​. ನಂಜುಂಡಸ್ವಾಮಿ}
\end{flushright}

\noindent
ಬೆಂಗಳೂರು \\ ಡಿಸೆಂಬರ್​ 2018

