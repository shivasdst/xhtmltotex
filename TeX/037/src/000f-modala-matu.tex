
\chapter*{ಮೊದಲ ಮಾತು}

ಮಂಡ್ಯ ಜಿಲ್ಲೆಯ, ಕೃಷ್ಣರಾಜಪೇಟೆ ತಾಲ್ಲೂಕಿನ, ಸಂತೇಬಾಚಹಳ್ಳಿ ನಮ್ಮ ಊರು. ಮಂಡ್ಯ-ಹಾಸನ ಜಿಲ್ಲೆಯ ಗಡಿ ಗ್ರಾಮವಾದ ನಮ್ಮ ಊರಿನ ಸುತ್ತಲೂ ಸುಮಾರು 5 ಮೈಲಿಯಿಂದ 15 ಮೈಲಿ ಫಾಸಲೆಯೊಳಗೆ, ನೂರಾರು ಶಾಸನಗಳು, ಸುಂದರ ದೇವಾಲಯಗಳು, ಬಸದಿಗಳು ಮತ್ತು ಸ್ಮಾರಕ ಶಿಲೆಗಳನ್ನು ಹೊಂದಿರುವ, ಐತಿಹಾಸಿಕವಾಗಿ ಮತ್ತು ಸಾಂಸ್ಕೃತಿಕವಾಗಿ ಬಹಳ ಮಹತ್ವವನ್ನು ಪಡೆದಿರುವ ಶ್ರವಣಬೆಳಗೊಳ, ಕಂಬದಹಳ್ಳಿ, ಮೇಲುಕೋಟೆ, ಅಗ್ರಹಾರ ಬಾಚಹಳ್ಳಿ, ಕಿಕ್ಕೇರಿ, ಹೊಸಹೊಳಲು, ನಾಗಮಂಗಲ, ಸಿಂದಘಟ್ಟ ಮೊದಲಾದ ಊರುಗಳಿವೆ. ಈ ಊರುಗಳು ಮೊದಲಿನಿಂದಲೂ ನನ್ನನ್ನು ಸೆಳೆಯುತ್ತಿದ್ದವು. ನಮ್ಮ ಊರಿನಲ್ಲಿಯೂ ಹೊಯ್ಸಳರ ಕಾಲದ ಮಹಲಿಂಗೇಶ್ವರ, ವಿಜಯನಗರ ಕಾಲದ ವೀರನಾರಾಯಣ, ವೀರಭದ್ರೇಶ್ವರ ದೇವಾಲಯಗಳೂ, ಶಾಸನಗಳೂ ಇವೆ. ನಮ್ಮ ಊರಿನ ಬಳಿ ಇರುವ ಬೆಣ್ಣೆ ಸಿದ್ಧನ ಗುಡ್ಡದಲ್ಲಿ ಕಲ್ಲಿನ ಸಮಾಧಿಗಳಿವೆ. ಹೀಗಾಗಿ ಮೊದಲಿನಿಂದಲೂ, ದೇವಾಲಯಗಳು, ಶಾಸನಗಳ ಬಗ್ಗೆ ನನಗೆ ಆಸಕ್ತಿ ಬೆಳೆದಿತ್ತು. ಆದರೆ ಮೊದಲು ಆ ದಿಕ್ಕಿನಲ್ಲಿ ನನ್ನ ಓದು ಸಾಗಲಿಲ್ಲ. ನಮ್ಮ ಮಂಡ್ಯ ಜಿಲ್ಲೆ ಎಂದರೆ ನನಗೆ ಅಭಿಮಾನ. ಜಿಲ್ಲೆಯ ಎಲ್ಲ ತಾಲ್ಲೂಕುಗಳಲ್ಲೂ ತಿರುಗಾಡಿದ್ದೇನೆ.

ಕನ್ನಡ, ಇಂಗ್ಲಿಷ್​, ಬೆರಳಚ್ಚು ಮತ್ತು ಶೀಘ್ರಲಿಪಿ ಪರೀಕ್ಷೆಗಳಲ್ಲಿ ತೇರ್ಗಡೆಯಾಗಿ, ಜೀವನೋಪಾಯಕ್ಕಾಗಿ ಸರ್ಕಾರಿ ಸೇವೆಗೆ ಸೇರಿದೆ. ನಂತರ ಸೇವೆಯಲ್ಲಿದ್ದಾಗಲೇ ಕನ್ನಡ ಮೇಜರ್​, ಇತಿಹಾಸ ಮತ್ತು ಸಮಾಜ ಶಾಸ್ತ್ರವನ್ನು ಮೈನರ್​ ವಿಷಯವಾಗಿ ತೆಗೆದುಕೊಂಡು ಬಿ.ಎ. ಪದವಿ ಪಡೆದೆ. ಮೈಸೂರು ವಿ.ವಿ. ಅಂಚೆ ಮತ್ತು ತೆರಪಿನ ಶಿಕ್ಷಣ ಸಂಸ್ಥೆಯಿಂದ ಶಾಸನಶಾಸ್ತ್ರ ಮತ್ತು ಕರ್ನಾಟಕದ ಸಾಂಸ್ಕೃತಿಕ ಚರಿತ್ರೆಯನ್ನು ಮುಖ್ಯ ವಿಷಯವನ್ನಾಗಿ ತೆಗೆದುಕೊಂಡು ಎಂ.ಎ. ಮಾಡಿದೆ. ಕನ್ನಡ ಸಾಹಿತ್ಯ ಪರಿಷತ್ತಿನ ಮೂಲಕ ಹಂಪಿ ಕನ್ನಡ ವಿಶ್ವವಿದ್ಯಾನಿಲಯವು ನಡೆಸುತ್ತಿದ್ದ ಶಾಸನ ಶಾಸ್ತ್ರ ಡಿಪ್ಲೊಮಾ ಪರೀಕ್ಷೆಯಲ್ಲಿ ತೇರ್ಗಡೆಯಾದೆ. ಅಂಚೆ ಮತ್ತು ತೆರಪಿನ ಶಿಕ್ಷಣ ಸಂಸ್ಥೆಯಲ್ಲಿ ಎಂ.ಎ. ಅಧ್ಯಯನ ಮಾಡುವಾಗ, ನನಗೆ ಪರಿಚಯವಾದವರು ನಮಗೆ ಪಾಠ ಮಾಡುತ್ತಿದ್ದ ಡಾ. ಡಿ.ಟಿ. ಬಸವರಾಜ್​ ಅವರು. ನಂತರ ಅವರು ಕರ್ನಾಟಕ ರಾಜ್ಯ ಮುಕ್ತ ವಿಶ್ವವಿದ್ಯಾನಿಲಯಕ್ಕೆ ರೀಡರ್​ ಆಗಿ ಬಂದರು. ಮುಕ್ತ ವಿ.ವಿ. ಯಿಂದ ಎಂ.ಫಿಲ್​. ಪರೀಕ್ಷೆಗೆ ಅರ್ಜಿಗಳನ್ನು ಆಹ್ವಾನಿಸಲಾಗಿತ್ತು. ಅರ್ಜಿ ಹಾಕಿದೆ. ವ್ಯಾಸಂಗಕ್ಕೆ ಅವಕಾಶ ಸಿಕ್ಕಿತು, ಡಾ.ಡಿ.ಟಿ. ಬಸವರಾಜ್​ ಅವರೇ ನಮಗೆ ಮಾರ್ಗದರ್ಶಕರಾಗಿ ದೊರೆತರು. ಗಂಭೀರ ಚಿತ್ತರೂ, ಪ್ರಾಮಾಣಿಕರು, ಶಿಷ್ಯ ವತ್ಸಲರೂ ಆದ ಡಾ. ಡಿ.ಟಿ.ಬಿ. ಅವರಂತಹ ಗುರುಗಳು ವಿರಳ. ಅವರ ಮಾರ್ಗದರ್ಶನಲ್ಲಿ ಶಾಸನಗಳ ವಿಷಯದಲ್ಲಿಯೇ ಎಂ.ಫಿಲ್​. ಪರೀಕ್ಷೆ ಮುಗಿಸಿದೆ. ಆ ನಂತರ ಡಾ. ಡಿಟಿಬಿ ಯವರು ನನಗೆ ಸ್ವಲ್ಪ ಆತ್ಮೀಯರಾದರು. ಕೆಲವು ತಿಂಗಳ ನಂತರ ದೂರವಾಣಿ ಮಾಡಿ, ಮುಕ್ತ ವಿ.ವಿ.ಯಲ್ಲಿ ಪಿಎಚ್​.ಡಿ. ಅಧ್ಯಯನಕ್ಕೆ ಅವಕಾಶ ಕಲ್ಪಿಸಲಾಗಿದೆ, ಅರ್ಜಿ ಹಾಕಿ ಎಂದರು, ನಾನು ಅರ್ಜಿ ಹಾಕಲು ಸಿದ್ಧತೆ ಮಾಡಿಕೊಂಡು ಮೈಸೂರಿಗೆ ಹೋಗಿ, ಸರ್​ ತಾವೇ ನನಗೆ ಗೈಡ್​ ಮಾಡಬೇಕೆಂದೆ. ತಕ್ಷಣ ತಮ್ಮ ಲೆಟರ್​ಹೆಡ್​ನಲ್ಲಿ ಕೈಬರಹ\-ದಲ್ಲೇ, ನಾನು ಮಾರ್ಗದರ್ಶಕನಾಗಿರಲು ಒಪ್ಪಿದ್ದೇನೆ ಎಂದು ಬರೆದುಕೊಟ್ಟರು. ಡಾ. ಡಿಟಿಬಿ ಅವರು ಖ್ಯಾತ ವಿದ್ವಾಂಸರಾದ ಡಾ. ಎಲ್​. ಬಸವರಾಜು ಮತ್ತು ಡಾ. ಎನ್​.ಎಸ್​. ತಾರಾನಾಥ್​ ಅವರ ಆತ್ಮೀಯ ಶಿಷ್ಯರು. ಹಮ್ಮು ಬಿಮ್ಮಿಲ್ಲದ, ಸ್ನೇಹಪರರಾದ, ಕನ್ನಡ ಸಾಹಿತ್ಯ ಮತ್ತು ಸಂಸ್ಕೃತಿಯ ಬಗ್ಗೆ ಅಪಾರ ತಿಳಿವಳಿಕೆಯುಳ್ಳ ಅವರು ನನಗೆ ಮಾರ್ಗದರ್ಶಕರಾದುದು ನನ್ನ ಅದೃಷ್ಟ.

ಮಂಡ್ಯ ಜಿಲ್ಲೆಯ ಶಾಸನಗಳು-ಒಂದು ಅಧ್ಯಯನ ಎಂಬ ವಿಷಯದ ಮೇಲೆ ಪಿಎಚ್​.ಡಿ. ಅಧ್ಯಯನಕ್ಕೆ ರಿಜಿಸ್ಟರ್​ ಮಾಡಿಸಿ, ಸಿದ್ಧತೆ ಮಾಡಿಕೊಳ್ಳುತ್ತಿದ್ದಾಗ, ಒಂದು ವರ್ಷ ಕಳೆದ ನಂತರ ಮುಕ್ತ ವಿಶ್ವವಿದ್ಯಾನಿಲಯದಿಂದ ಒಂದು ಪತ್ರ ಬಂದಿತು. ನೀವು ಕಲೋಕ್ವಿಯಂ ಅಂದರೆ ವಿದ್ವತ್​ ಸಭೆಗೆ ಹಾಜರಾಗಬೇಕು, ನಿಮ್ಮ ವಿಷಯಕ್ಕೆ ಒಪ್ಪಿಗೆ ಸಿಕ್ಕಿದರೆ ಪಿಎಚ್​.ಡಿ. ಅಧ್ಯಯನವನ್ನು ಮುಂದುವರಿಸಬಹುದು ಎಂದು ಆ ಪತ್ರ ಸೂಚಿಸಿತ್ತು. ಕಲೋಕ್ವಿಯಂಗೆ ಹಾಜರಾದೆ, ಅಲ್ಲಿ ಸಂಸ್ಕೃತಿ ವಿ.ವಿ.ಯ ನಿವೃತ್ತ ಕುಲ\-ಪತಿಗಳಾದ ಶ‍್ರೀ ಮಲ್ಲೇಪುರಂ ವೆಂಕಟೇಶ್​ ಅವರು ಇನ್ನೂ ಒಂದಿಬ್ಬರು ವಿದ್ವಾಂಸರು ವಿದ್ವತ್​ ಸಭೆಯಲ್ಲಿದ್ದರು. ಅವರ ಮುಂದೆ ವಿಷಯ ಮಂಡಿಸಿದೆ. ಮಲ್ಲೇಪುರಂ ವೆಂಕಟೇಶ್​ ಅವರು ನಿಮ್ಮ ವಿಷಯ ಬಹಳ ದೊಡ್ಡದಾಯಿತು, ಜಿಲ್ಲೆಯ ಶಾಸನಗಳ ಯಾವುದಾದರೂ ಒಂದು ವಿಷಯವನ್ನು ಅಧ್ಯಯನ ಮಾಡಿ ಎಂದರು. ಆದರೆ ನಾನು ಮೊದಲೇ ಮನಸ್ಸು ಮಾಡಿದ್ದಂತೆ, ಇಲ್ಲ ಪೂರ್ತಿಯಾಗಿ ಅಧ್ಯಯನ ಮಾಡುತ್ತೇನೆ, ಜಿಲ್ಲೆಯ ಶಾಸನಗಳಲ್ಲಿ ಪ್ರಸ್ತಾಪಿಸಲ್ಪಟ್ಟಿರುವ ವಿಷಯಗಳೆಲ್ಲಾ ಒಂದೇ ಕಡೆ ಇದ್ದರೆ, ಅದು ಮ್ಯಾಕ್ರೋ ಸ್ಟಡಿ ಎನಿಸಿದರೂ, ಚೆನ್ನಾಗಿರುತ್ತದೆ, ಅದಕ್ಕಾಗಿ ಈ ಅಧ್ಯಯನ ಕೈಗೊಳ್ಳುತ್ತೇನೆ ಎಂದೆ. ಅದಕ್ಕೆ ಅವರು ಆಯಿತು, ನೀವು ಅದೇ ವಿಷಯದ ಮೇಲೆ ಪಿಎಚ್​.ಡಿ. ಮಾಡುವುದಾದರೆ ನಮ್ಮ ಅಭ್ಯಂತರವಿಲ್ಲ ಎಂದರು. ಕಲೋಕ್ವಿಯಂನಿಂದ ನನಗೆ ಅನುಮತಿ ದೊರೆಯಿತು. ಇದರಲ್ಲಿ ಒಂದು ವರ್ಷ ಹೋಯಿತು, ಮುಂದೆ ಐದು ವರ್ಷಗಳ ಕಾಲ ಸತತವಾಗಿ ಕುಳಿತು ಅಧ್ಯಯನ ಮಾಡಿದೆ, ಪ್ರಮುಖ ಶಾಸನಗಳು, ಸ್ಮಾರಕಗಳು ಇರುವ ಸುಮಾರು 200ಕ್ಕೂ ಹೆಚ್ಚು ಹಳ್ಳಿಳಿಗೆ ಭೇಟಿ ನೀಡಿ ಕ್ಷೇತ್ರ ಕಾರ್ಯ ಮಾಡಿದೆ. ಜಿಲ್ಲೆಯ ಬಗ್ಗೆ ಪ್ರಕಟವಾದ ಸುಮಾರು 50-60 ಕೃತಿಗಳೂ ಸೇರಿದಂತೆ, ಒಟ್ಟು ಸುಮಾರು 300 ಕ್ಕೂ ಹೆಚ್ಚು ಕೃತಿಗಳನ್ನು ಅಧ್ಯಯನ ಮಾಡಿದೆ. ಸಾವಿರಾರು ಚೀಟಿಗಳಲ್ಲಿ (ಕಾರ್ಡ್ಸ್ ಅಥವಾ ಸ್ಲಿಪ್ಸ್​) ವಿಷಯ ಗುರುತು ಹಾಕಿಕೊಂಡೆ. ವಿಷಯವಾರು, ಕಾಲವಾರು ವಿಂಗಡಿಸಿ\-ಕೊಂಡೆ. ಕೊನೆಗೆ ಇದು ಒಂದು ಅಗಾದವಾದ ಕಾರ್ಯ ಎನಿಸಿತು. ಪ್ರತಿ ಅಧ್ಯಾಯದ ಬರವಣಿಗೆ ಮುಗಿದ ತಕ್ಷಣ ಮಾರ್ಗದರ್ಶಕರಿಗೆ ತೋರಿಸಿ, ಆರು ತಿಂಗಳಿಗೆ ಒಂದು ಸಲ ಅಧ್ಯಯನದ ಪ್ರಗತಿಯ ವರದಿಯನ್ನು ವಿಶ್ವವಿದ್ಯಾನಿಲಯಕ್ಕೆ ಸಲ್ಲಿಸಿ ಬರುತ್ತಿದೆ. ನನ್ನ ಮಾರ್ಗದರ್ಶಕರಾದ ಡಾ. ಡಿಟಿಬಿ ಅವರು ಪ್ರತಿ ಅಧ್ಯಾಯವನ್ನು ಪರಿಶೀಲಿಸಿ, ಸೂಕ್ತ ಸೂಚನೆಗಳನ್ನು ನೀಡಿ, ಬರವಣಿಗೆಯ ಬಗ್ಗೆ ಮಾರ್ಗದರ್ಶನ ಮಾಡುತ್ತಿದ್ದರು. ಸಾಧ್ಯವಾದಷ್ಟು ವಿವರವಾಗಿ ಬರೆಯಿರಿ, ಕೊನೆಯಲ್ಲಿ ಅಗತ್ಯವಾದುದನ್ನು ಉಳಿಸಿಕೊಂಡು,\break ಸಂಕ್ಷೇಪಿಸಬಹುದು. ಹೆಚ್ಚಿನ ವಿಷಯಗಳನ್ನು ಅನುಬಂಧಲ್ಲಿ ಇಡಬಹುದು. ಆದರೆ ಬರೆಯುವಾಗಲೇ ಕಡಿಮೆ ಬರೆದರೆ, ಮತ್ತೊಮ್ಮೆ ವಿಸ್ತರಿಸಲು ಸಾಧ್ಯವಿಲ್ಲ ಎಂದು ಹೇಳುತ್ತಿದ್ದರು. ಅದರ ಜೊತೆಗೆ ಶಾಸನ ಶಾಸ್ತ್ರ ತರಗತಿಗಳಲ್ಲಿ ನನ್ನ ಗುರುಗಳಾಗಿದ್ದ ಪ್ರಖ್ಯಾತ ಶಾಸನ ಶಾಸ್ತ್ರ ವಿದ್ವಾಂಸರು, ಪ್ರಾಚಾರ್ಯರೂ ಆದ ಡಾ. ಕೆ.ಆರ್​. ಗಣೇಶ್​ ಅವರಿಗೂ ನಾನು ಬರೆದ ಅಧ್ಯಾಯಗಳನ್ನು ಒಪ್ಪಿಸುತ್ತಿದ್ದೆ. ಅವರು ಅವುಗಳನ್ನು ಆಮೂಲಾಗ್ರವಾಗಿ ಪರಿಶೀಲಿಸಿ, ವಿಷಯ ಮಂಡನೆ ಬಗ್ಗೆ, ಬರವಣಿಗೆ ಬಗ್ಗೆ, ಪರಾಮರ್ಶೆಯ ಬಗ್ಗೆ ಸೂಕ್ತ ಸಲಹೆ ಸೂಚನೆಗಳನ್ನು ನೀಡುತಿದ್ದರು.

ಮಂಡ್ಯ ಜಿಲ್ಲೆಯ ಇತಿಹಾಸ ಮತ್ತು ಪುರಾತತ್ವದ ಬಗ್ಗೆ, ಮಂಡ್ಯದಲ್ಲಿ ದಿನಾಂಕ 11 ಮತ್ತು 12 ಆಗಸ್ಟ್​ 2007 ರಂದು ಏರ್ಪಡಿಸಲಾಗಿದ್ದ ವಿಚಾರ ಸಂಕಿರಣದಲ್ಲಿ ಹಾಗೂ ಅದರಲ್ಲಿ ಮಂಡಿಸಲಾದ ಪ್ರಬಂಧಗಳ ಸಂಕಲನ “ಮಂಡ್ಯ ಜಿಲ್ಲೆಯ ಇತಿಹಾಸ ಮತ್ತು ಪುರಾತತ್ವ” ಎಂಬ ಗ್ರಂಥದಲ್ಲಿ ಪ್ರಕಟವಾಗಿರುವಂತೆ, ತಮ್ಮ ಅಧ್ಯಕ್ಷ ಭಾಷಣದಲ್ಲಿ ಖ್ಯಾತ ಪ್ರಾಚೀನ ಇತಿಹಾಸ ಮತ್ತು ಪುರಾತತ್ವ ವಿದ್ವಾಂಸರಾದ ಡಾ. ಎಂ.ಎಸ್​. ಕೃಷ್ಣಮೂರ್ತಿಯವರು, “ಇಷ್ಟು ಪ್ರವರ್ಧಮಾನವಾಗಿ, ಸಂಪದ್ಭರಿತವಾಗಿ ನೆಲೆಸಿದ್ದ ನಾಡಿನ ಇತಿಹಾಸವನ್ನೂ, ಸಂಸ್ಕೃತಿಯನ್ನೂ ಪ್ರತ್ಯೇಕವಾಗಿ ಆಳವಾಗಿ ಅಧ್ಯಯನ ಮಾಡಲು ಯಾರೂ ಪ್ರಯತ್ನಿಸದಿರುವುದು ಅಚ್ಚರಿಯ ಸಂಗತಿಯೇ ಸರಿ...... “ಮಂಡ್ಯ ಜಿಲ್ಲೆಯಲ್ಲಿ ಹೊಯ್ಸಳ ಸಂಸ್ಕೃತಿಯ ವಿವಿಧ ಮುಖ\-ಗಳನ್ನು ಪರಿಚಯ ಮಾಡಿಕೊಡುವ ಗ್ರಂಥ ಇನ್ನೂ ಪ್ರಕಟವಾಗಬೇಕಾಗಿದೆ, ಇದೇ ಅವಲೋಕನ, ಅಭಿಪ್ರಾಯ ವಿಜಯನಗರ ಮತ್ತು ವಿಜಯನಗರೋತ್ತರ ಕಾಲದ ಸಂಶೋಧನೆಗಳಿಗೂ ಅನ್ವಯಿಸುತ್ತದೆ”.... \textbf{“ಈ ಎಲ್ಲಾ ಬದಲಾವಣೆ, ಬೆಳವಣಿಗೆಗಳ ಒಂದು ಜ್ಞಾನ, ಸಮಗ್ರ ಅಧ್ಯಯನಗಳಿಂದ ಬರಬೇಕೇ ಹೊರತು, ಆಳವಾದರೂ ಛಿದ್ರ-ಛಿದ್ರ ಅಧ್ಯಯನಗಳಿಂದ ಖಂಡಿತಾ ಅಲ್ಲ”} ಎಂದು ತಮ್ಮ ಅಭಿಪ್ರಾಯವನ್ನು ವ್ಯಕ್ತಪಡಿಸಿದ್ದಾರೆ. ಈ ಅಭಿಪ್ರಾಯ ನನ್ನ ಪಿಎಚ್​.ಡಿ.ಅಧ್ಯಯನಕ್ಕೆ ಹಾಗೂ ಈಗ ನನ್ನ ಈ ಕೃತಿ ರಚನೆಗೆ ಇಂಬು ನೀಡಿದೆ. ಈ ನಿಟ್ಟಿನಲ್ಲಿ ನನ್ನ ಅಧ್ಯಯನ ಮೊದಲ ಮೆಟ್ಟಿಲು ಎಂದು ಹೇಳಬಹುದು.

ಬಿಡುವಿಲ್ಲದ ಕಚೇರಿ ಕೆಲಸಗಳು, ಪ್ರವಾಸ, ಸಾಂಸಾರಿಕ ವಿಷಯಗಳು ಇವುಗಳ ನಡುವೆ, ಐದು ವರ್ಷಗಳ ಕಾಲ ಸತತವಾಗಿ ಬಿಡುವಿನ ಸಮಯದಲ್ಲಿ, ಕೆಲಸ ಮಾಡಿ, ಈ ಪ್ರಬಂಧವನ್ನು ಸಿದ್ಧಪಡಿಸಿದೆ. ಕೊನೆಗೆ ಬರವಣಿಗೆ ಮುಕ್ತಾಯ\-ವಾದಾಗ ಅದು ಸುಮಾರು 1200-1300 ಪುಟಗಳಿಗೆ ಬಂದು ನಿಂತಿತು. ನನ್ನ ಮಾರ್ಗದರ್ಶಕರಾದ ಡಾ. ಡಿ.ಟಿ.ಬಿ. ಹಾಗೂ ನನ್ನ ಗುರುಗಳಾದ ಡಾ.ಕೆ.ಆರ್​.ಗಣೇಶ್​ ಅವರ ಸೂಕ್ತ ಸಲಹೆ ಸೂಚನೆಗಳ ಮೇರೆಗೆ ಅದನ್ನು ಮತ್ತೆ ಬರೆದು, ಸುಮಾರು 850 ಪುಟ\-ಗಳಿಗೆ(ಅಡಿಟಿಪ್ಪಣಿಗಳೂ ಸೇರಿ) ಸಂಕ್ಷೇಪಿಸಿದೆ. ಪ್ರಬಂಧ ಬಹಳ ವಿಸ್ತಾರವಾಯಿತು. ಇದನ್ನು ವಿಶ್ವವಿದ್ಯಾನಿಲಯಕ್ಕೆ ಸಲ್ಲಿಸಲು ಅನುಮತಿ ದೊರೆಯುತ್ತದೋ ಇಲ್ಲವೋ ಎಂಬ ಅಳುಕಿತ್ತು. ಮಾರ್ಗದರ್ಶಕರಾದ ಡಾ. ಡಿ.ಟಿ.ಬಿ. ಅವರ ಬಳಿ ವಿಷಯ ಪ್ರಸ್ತಾಪಿಸಿದೆ. ಅದಕ್ಕೆ ಅವರು ಒಂದು ಜಿಲ್ಲೆಯ ಶಾಸನಗಳನ್ನು ನಾನಾ ಮುಖಗಳಿಂದ ಅಧ್ಯಯನ ಮಾಡಿ, ಅದನ್ನು ಕೇವಲ 300-400 ಪುಟಗಳಲ್ಲಿ ನಿರೂಪಿಸಲು ಸಾಧ್ಯವಾಗುವುದಿಲ್ಲ. ನಾನು ಸಬ್​ಮಿಟ್​ ಮಾಡಿಸುತ್ತೇನೆ ಸಿದ್ಧಮಾಡಿ ಎಂದರು ಹೇಳಿ ವಿಶ್ವವಿದ್ಯಾನಿಲಯಕ್ಕೆ ಸಬ್​ಮಿಟ್​ ಮಾಡಿಸಿದರು. ಮೌಲ್ಯಮಾಪನವಾಗಿ ಒಂದು ವರ್ಷದ ನಂತರ ನನ್ನೊಬ್ಬನಿಗೇ ವೈವಾ ನಡೆಯಿತು. ನನ್ನ ವೈವಾಕ್ಕೆ ನನ್ನ ಮಾರ್ಗದರ್ಶಕರಾದ ಡಾ. ಡಿಟಿಬಿ. ಅವರ ಜೊತೆಗೆ, ಮೈಸೂರು ವಿ.ವಿ. ಕನ್ನಡ ಅಧ್ಯಯನ ಸಂಸ್ಥೆಯ ಪ್ರೊಫೆಸರ್​, ಶಾಸನ ಶಾಸ್ತ್ರ ಕ್ಷೇತ್ರದ ಯುವ ವಿದ್ವಾಂಸ, ಡಾ. ಎಂ.ಜಿ. ಮಂಜುನಾಥ್​ ಅವರು ಬಂದಿದ್ದರು. ಆಗಲೇ ನನಗೆ ಅವರು ನನ್ನ ಪ್ರಬಂಧವನ್ನು ಮೌಲ್ಯಮಾಪನ ಮಾಡಿದ್ದಾರೆ ಎಂದು ತಿಳಿಯಿತು. ಅವರು ಹಂಪಿ ವಿ.ವಿ.ಯವರು ಶ್ರವಣಬೆಳಗೊಳದಲ್ಲಿ ನಡೆಸಿದ ಶಾಸನಶಾಸ್ತ್ರ ಅಧ್ಯಯನದ ಮೂರು ದಿನಗಳ ಕಮ್ಮಟದಲ್ಲಿ ನಮಗೆ ಪ್ರಾಯೋಗಿಕವಾಗಿ ಹಾಗೂ ತರಗತಿಗಳಲ್ಲಿ ಶಾಸನಗಳ ಬಗ್ಗೆ ಉಪನ್ಯಾಸ ನೀಡಿದ್ದರು. ಅವರ ಸಂಶೋಧನಾ ಕೃತಿಗಳು, ಲೇಖನಗಳನ್ನೆಲ್ಲಾ ನಾನು ಓದಿದ್ದೇನೆ. ನನ್ನ ಪ್ರಬಂಧದ ಬಗ್ಗೆ ಅವರೂ ಮೆಚ್ಚುಗೆಯ ಮಾತುಗಳನ್ನು ಆಡಿದರು. ನನ್ನ ಈ ಕೃತಿ ಅಚ್ಚಿಗೆ ಬಂದಮೇಲೆ ಅವರು ತಮ್ಮ ಅಭಿಪ್ರಾಯವನ್ನು ಸಂತೋಷದಿಂದ ಬರೆದುಕೊಟ್ಟರು. ಅವರಿಗೆ ನಾನು ಅತ್ಯಂತ ಕೃತಜ್ಞನಾಗಿದ್ದೇನೆ. ಒಂದು ವರ್ಷದ ನಂತರ ಅಂದರೆ 2013 ರಲ್ಲಿ ನನ್ನ ಪ್ರಬಂಧಕ್ಕೆ ಪಿಎಚ್​.ಡಿ. ಪದವಿ ದೊರೆಯಿತು.

ಕನ್ನಡ ಎಂ.ಎ., ಶಾಸನಶಾಸ್ತ್ರ, ಎಂ.ಫಿಲ್​. ತರಗತಿಗಳಲ್ಲಿ ನಾನು ಹೆಚ್ಚಾಗಿ ಓದುತ್ತಿದ್ದುದು, ಕನ್ನಡ ಶಾಸನಗಳ ಸಾಂಸ್ಕೃತಿಕ ಅಧ್ಯಯನ ಲೋಕದ ಹೆಬ್ಬಾಗಿಲನ್ನು ತೆರೆದ, ಪ್ರಖ್ಯಾತ ಸಂಶೋಧಕರು, ಚಿಂತಕರು, ಕರ್ನಾಟಕದ ಇತಿಹಾಸ ಮತ್ತು ಸಂಸ್ಕೃತಿಯ ವಿದ್ವಾಂಸರು, ಕನ್ನಡ ನಾಡು ನುಡಿಯ ಅನನ್ಯ ಪ್ರೇಮಿಗಳೂ ಆದ ಡಾ. ಎಂ. ಚಿದಾನಂದ ಮೂರ್ತಿಯವರ ಸಂಶೋಧನಾ ಕೃತಿಗಳನ್ನು. ಅವರನ್ನು ಅನೇಕ ಸಭೆ ಸಮಾರಂಭದಲ್ಲಿ ನೋಡಿದ್ದೆ. ಅವರ ಭಾಷಣಗಳನ್ನು, ಉಪನ್ಯಾಸಗಳನ್ನು ಕೇಳಿದ್ದೆ. ನನಗೆ, ನನ್ನಂತಹ ಸಾವಿರಾರು ಶಾಸನಶಾಸ್ತ್ರ ಅಧ್ಯಯನ ಆಸಕ್ತರಿಗೆ ಅವರು ಮಾನಸ ಗುರುಗಳು. ನನ್ನ ಪ್ರಬಂಧವನ್ನು ಅವರಿಗೆ ತೋರಿಸಿ, ಅವರ ಅಭಿಪ್ರಾಯವನ್ನು ಪಡೆದುಕೊಳ್ಳಬೇಕೆಂದು ಕಾತರನಾಗಿದ್ದೆ. ನನ್ನ ಮಿತ್ರರ ಮೂಲಕ ಅವರ ಭೇಟಿ\-ಯಾದೆ. ಅವರು ಮೊದಲು ಇದನ್ನೆಲ್ಲಾ ಈಗ ನೋಡಲು ನನ್ನಿಂದ ಸಾಧ್ಯವಿಲ್ಲ ಎಂದರು. ಕೊನೆಗೆ ಪರಿವಿಡಿಯನ್ನು ನೋಡಿ, ಸಾಧ್ಯ\-ವಾದರೆ ನೋಡುತ್ತೇನೆ, ಇಟ್ಟು ಹೋಗು ಎಂದರು. ಒಂದೆರಡು ತಿಂಗಳ ನಂತರ ನನ್ನ ಮಿತ್ರರು ದೂರವಾಣಿ ಮಾಡಿ ಡಾ. ಎಂ. ಚಿದಾನಂದಮೂರ್ತಿಯವರು ನಿಮ್ಮ ಪ್ರಬಂಧವನ್ನು ನೋಡಿ, ತಮ್ಮ ಅಭಿಪ್ರಾಯವನ್ನು ಬರೆದಿಟ್ಟಿದ್ದಾರೆ, ಹೋಗಿ ತೆಗೆದುಕೊಂಡು ಬನ್ನಿ ಎಂದರು. ತಮ್ಮ ಲೆಟರ್​ಹೆಡ್​ನಲ್ಲಿ ಕೈಬರಹದಲ್ಲೇ ಅವರ ಅಭಿಪ್ರಾಯವನ್ನು ಬರೆದಿಟ್ಟಿದ್ದರು. ಪರವಾಗಿಲ್ಲ, ಆದರೆ ಪ್ರಬಂಧ ಪ್ರಕಟವಾಗಬೇಕು, ಪ್ರಕಟವಾಗದಿದ್ದರೆ ಏನೂ ಪ್ರಯೋಜನವಿಲ್ಲದಂತಾಗುತ್ತದೆ ಎಂದು ಪ್ರೋತ್ಸಾಹದ ಮಾತುಗಳನ್ನು ಆಡಿದರು. ಬೆನ್ನುತಟ್ಟಿದರು. ಅವರು ನನ್ನ ಪ್ರಬಂಧಕ್ಕೆ ಬರೆದ ಅಭಿಪ್ರಾಯವನ್ನು ಈ ಕೃತಿಯಲ್ಲಿ ಅಚ್ಚು ಹಾಕಿಸಿದ್ದೇನೆ.

ಪ್ರಬಂಧವೇನೋ ಸಿದ್ಧವಾಯಿತು. ಆದರೆ ಪ್ರಕಟಣೆ ಸಾಧ್ಯವಿಲ್ಲದ ಮಾತು. 800 ಪುಟಗಳ ಪ್ರಬಂಧ (ಎ4)\break ಅಚ್ಚಾದರೆ ಅದು 1200 ಪುಟಗಳಾಗುತ್ತದೆ. ಇಂತಹ ಕೃತಿಗಳನ್ನು ಯಾರೂ ಪ್ರಕಟಿಸಲು ಮುಂದೆ ಬರುವುದಿಲ್ಲ. ಹೆಚ್ಚೆಂದರೆ\break 250 ರಿಂದ 300 ಪುಟಗಳಾದರೆ ಪ್ರಕಟಿಸಲು ಪ್ರಯತ್ನಿಸ ಬಹುದು. ಹೀಗಾಗಿ ನಾಲ್ಕು ವರ್ಷ ಸುಮ್ಮನಿದ್ದೆ. ಕಾಮಧೇನು ಪ್ರಕಾಶನದ ಶ‍್ರೀ ಶಾಮಸುಂದರ ರಾವ್​ ಅವರು ನನಗೆ ಪರಿಚಿತರು. ಸರ್ಕಾರಿ ಪದವಿ ಕಾಲೇಜಿನಲ್ಲಿ ಕನ್ನಡ ರೀಡರ್​ ಆಗಿದ್ದ ಅವರು ಸರಳರು, ಸಜ್ಜನರು. ಶಾಸಕರ ಭವನದಲ್ಲಿ ಪುಸ್ತಕದ ಅಂಗಡಿಯನ್ನು ತೆರೆದಾಗಿನಿಂದ ಅವರು ಪರಿಚಿತರಾದರು. ಸಾಧ್ಯವಾದಾಗಲೆಲ್ಲಾ ಅವರ ಪುಸ್ತಕಾಲಯಕ್ಕೆ ಹೋಗುತ್ತಿದ್ದೆ. ಅನೇಕ ಪುಸ್ತಕಗಳನ್ನು ಖರೀದಿಸುತ್ತಿದ್ದೆ. ಅವರು ಅನೇಕ ಮೌಲ್ಯಯುತ, ಸತ್ವಯುತ ಪ್ರಾಚೀನ, ಅರ್ವಾಚೀನ ಕೃತಿಗಳನ್ನು ತಮ್ಮ ಪ್ರಕಾಶನದ ಮೂಲಕ ಪ್ರಕಟಿಸಿ ಪ್ರಖ್ಯಾತರಾಗಿದ್ದಾರೆ. ಲಾಭನಷ್ಟದ\break ಲೆಕ್ಕಾಚಾರವಿಲ್ಲದೆ ಕನ್ನಡ ಕೃತಿಗಳು ಪ್ರಕಟವಾಗಬೇಕು, ಓದುಗರ ಕೈಸೇರಬೇಕು ಎಂಬ ಒಂದೇ ಉದ್ದೇಶದಿಂದ ಅವರು ಪುಸ್ತಕ ಪ್ರಕಟಣೆಯನ್ನು ಮಾಡುತ್ತಿದ್ದಾರೆ. ಒಂದು ವರ್ಷದ ಹಿಂದೆ ಇರಬಹುದು. ಒಂದು ದಿನ ಅವರು ನಿಮ್ಮ ಪ್ರಬಂಧ ಪ್ರಕಟಿಸು\-ತ್ತೀರಾ, ಯಾರಿಗಾದರೂ ಕೊಟ್ಟಿದ್ದೀರಾ, ಎಂದು ಕೇಳಿದರು. ಇಲ್ಲ ಅದನ್ನು ಯಾರು ಪ್ರಕಟಿಸುತ್ತಾರೆ, ಬಹಳ ದೊಡ್ಡದು ಎಂದೆ. ನಮ್ಮ ಕಡೆಯವರ ಒಂದು ಪ್ರಕಾಶನ ಸಂಸ್ಥೆ ಇದೆ. ಅದರ ಮೂಲಕ ಪ್ರಕಟಿಸೋಣ, ಸಂಕ್ಷೇಪಿಸಿ ಸಿದ್ಧಪಡಿಸಿಕೊಡಿ, ಪುಟಗಳ ಲೆಕ್ಕಾಚಾರ ಬೇಡ, ವಿಷಯ ಮುಖ್ಯ ಒಂದು ನೂರು ಪುಟ ಹೆಚ್ಚಾದರೂ ಪರವಾಗಿಲ್ಲ ಎಂದರು. ಅವರ ಮಾತನ್ನು ಕೇಳಿ ನಾನು ಆಶ್ಚರ್ಯಚಕಿತನಾದೆ. ಪ್ರಖ್ಯಾತ ಲೇಖಕರು, ವಿದ್ವಾಂಸರನ್ನು ಹೊರತುಪಡಿಸಿ, ನನ್ನಂತಹ ಅಜ್ಞಾತ ಲೇಖಕರ, ಅದರಲ್ಲೂ ಬೇಡಿಕೆ ಇಲ್ಲದ ಸಂಶೋಧನಾ ಕೃತಿಯನ್ನು ಪ್ರಕಟಿಸಲು ಕೇಳಿದರೆ, ಮುಂದೆ ಬರದಿರುವ, ಬಂದರೂ ನಾನಾ ರೀತಿಯ ಕಂಡೀಷನ್​ಗಳನ್ನು ಹಾಕುವ ಪ್ರಕಾಶಕರು ಇರುವಾಗ, ಇವರು ಇಷ್ಟೊಂದು ಸುಲಲಿತವಾಗಿ ಈ ರೀತಿ ಹೇಳಿದ್ದನ್ನು ನಂಬಲು ಸಾಧ್ಯವಾಗಲೇ ಇಲ್ಲ. ಮೂಕನಾಗಿ ಆಯಿತು ಎಂದೆ. ಪ್ರಬಂಧವನ್ನು ಈಗಿನ ರೂಪಕ್ಕೆ ಸುಮಾರು 500 ಪುಟಗಳಿಗೆ ಸಂಕ್ಷೇಪಿಸ ಬೇಕೆಂದುಕೊಂಡೆ. ನಮ್ಮ ಗುರುಗಳಾದ ಡಾ. ಕೆ.ಆರ್​.ಗಣೇಶ್​ ಅವರ ಬಳಿ ಹೋದೆ. ಅವರು ನನ್ನ ಪ್ರಬಂಧವನ್ನು ಮತ್ತೊಮ್ಮೆ ಆಮೂಲಾಗ್ರವಾಗಿ ನೋಡಿ, ಎಲ್ಲೆಲ್ಲಿ ಸಂಕ್ಷೇಪಿಸಬೇಕು, ಯಾವುದನ್ನು ಬಿಡಬೇಕು, ಯಾವುದನ್ನು ಹಾಗೇ ಇಡಬೇಕು, ಎಂಬುದರ ಬಗ್ಗೆ ವಿವರವಾಗಿ ಗುರುತು ಹಾಕಿಕೊಟ್ಟರು. ಆ ಪ್ರಕಾರ ಪ್ರಬಂಧವನ್ನು ಹೊಸದಾಗಿ ಪ್ರಕಟಣೆಗೆ ಸಿದ್ಧಪಡಿಸತೊಡಗಿದೆ. ಅನೇಕ ಅಧ್ಯಾಯಗಳನ್ನು ಸಂಕ್ಷೇಪಿಸಿದೆ, ಕೆಲವು ಅಧ್ಯಾಯಗಳನ್ನು, ಅನುಬಂಧಗಳನ್ನು ಪೂರ್ಣವಾಗಿ ಕೈಬಿಟ್ಟೆ, ಸಂಶೋಧನೆಯ ನಂತರದಲ್ಲಿ ಅಧ್ಯಯನ ಹಾಗೂ ಕ್ಷೇತ್ರಕಾರ್ಯದ ಮೂಲಕ ನನ್ನ ಗಮನಕ್ಕೆ ಬಂದ ಅನೇಕ ಹೊಸ ವಿಷಯಗಳನ್ನು, ಛಾಯಾಚಿತ್ರಗಳನ್ನು ಸೇರಿಸಿದೆ, ಹೀಗೆ ನನ್ನ ಪ್ರಬಂಧವು ಹೊಸ ರೂಪು ತಳೆಯಿತು. ಮೂಲ ಪ್ರಬಂಧಕ್ಕಿಂತ ಈ ನನ್ನ ಕೃತಿಯು ವಿಭಿನ್ನವಾಗಿಯೇ ನಿಲ್ಲುತ್ತದೆ.

ವಿದ್ವಾಂಸರ ಜೊತೆಗೆ ಜನಸಾಮಾನ್ಯರನ್ನು ದೃಷ್ಟಿಯಲ್ಲಿಟ್ಟುಕೊಂಡು ಪ್ರಬಂಧವನ್ನು ಹೊಸದಾಗಿ ಕೃತಿಯ ರೂಪದಲ್ಲಿ ಸಿದ್ಧಪಡಿಸಿದೆ. ಅಡಿಟಿಪ್ಪಟಣಿಗಳನ್ನು ಉಳಿಸಿಕೊಳ್ಳಬೇಕೋ, ಬಿಡಬೇಕೋ ಎಂಬ ಜಿಜ್ಞಾಸೆ ಶುರುವಾಯಿತು. ಕೊನೆಗೆ ಸಂಶೋಧನಾತ್ಮಕ ಕೃತಿಯಾಗಿದ್ದರಿಂದ ಅವುಗಳನ್ನು ಹಾಗೇ ಉಳಿಸಿಕೊಂಡು, ಮೂಲ ಪ್ರಬಂಧದಲ್ಲಿ ಆಯಾ ಪುಟದಲ್ಲೇ ಇದ್ದ ಅಡಿಟಿಪ್ಪಣಿಗಳನ್ನು ಆಯಾ ಅಧ್ಯಾಯಗಳ ಕೊನೆಗೆ ತಂದು ಜೋಡಿಸಿದೆ. ನನ್ನ ಗುರುಗಳಾದ ಡಾ. ಕೆ.ಆರ್​.ಗಣೇಶ್​ ಅವರು ಕೈಬಿಡಬಹುದು ಎಂದು ಹೇಳಿದ್ದ ಕೆಲವು ಅಧ್ಯಾಯಗಳನ್ನು ಸಂಕ್ಷೇಪಿಸಿ ಸೇರಿಸಿದೆ. ಅದರಿಂದ ಕೃತಿ ಮತ್ತೆ ದೊಡ್ಡದಾಗಿ ಈಗಿನ ಸ್ವರೂಪ ಪಡೆಯಿತು. ಕೊನೆಗೆ ಸುಮಾರು 650 ಪುಟಗಳ ಈ ಕೃತಿ ಸಿದ್ಧವಾಯಿತು. ಅದರಲ್ಲಿ ಸುಮಾರು 3500ಕ್ಕೂ ಹೆಚ್ಚು ಅಡಿ ಟಿಪ್ಪಣಿಗಳು, 130 ಪುಟಗಳನ್ನು ಆಕ್ರಮಿಸಿದೆ. ಶೇಕಡಾ 80ಕ್ಕೂ ಹೆಚ್ಚು ಅಡಿಟಿಪ್ಪಣಿಗಳು ಮೂಲ ಶಾಸನಗಳದ್ದಾಗಿವೆ. ಜಿಲ್ಲೆಯ ಒಳನಾಡಿನಲ್ಲಿರುವ ನೂರಾರು ಹಳ್ಳಿಗಳಿಗೆ ಹೋಗಿ ತೆಗೆದಿರುವ ಅಪರೂಪದ ಸ್ಮಾರಕಗಳ ಸುಮಾರು 200ಕ್ಕೂ ಛಾಯಾಚಿತ್ರಗಳನ್ನು ಈ ಕೃತಿಯಲ್ಲಿ ಅಳವಡಿಸಿದ್ದೇನೆ. ಇವುಗಳನ್ನು ಬಿಟ್ಟರೆ ನ್ ಅನ್ನ ಮೂಲ ಕೃತಿ ಸುಮಾರು 500 ಪುಟಗಳಾಗುತ್ತದೆ. ನಮಗೆ ಎಂ.ಫಿಲ್​. ತರಗತಿಯಲ್ಲಿ ಸಂಶೋಧನೆಯ ಬಗ್ಗೆ ಉಪನ್ಯಾಸ ನೀಡುತ್ತಿದ್ದ, ಹಿರಿಯ ಸಂಶೋಧಕರಾದ ಡಾ.ಎಸ್​. ಗುರುರಾಜಾಚಾರ್​ ಅವರು, ಸಂಶೋಧನಾ ಪ್ರಬಂಧದಲ್ಲಿ, ನೀವು ಹೇಳುವ ಪ್ರತಿಯೊಂದಕ್ಕೂ ಆಧಾರ ಇರಬೇಕು, ಅದರ ಮೇಲೆ ತಕ್ಕಮಟ್ಟಿಗೆ ಊಹೆಯನ್ನು ಮಾಡಬಹುದು, ಆಧಾರ ಸೂಚಿ ಅಡಿಟಿಪ್ಪಣಿಗಳ ಸಂಖ್ಯೆ ಹೆಚ್ಚಿದ್ದಷ್ಟೂ ನಿಮ್ಮ ಪ್ರಬಂಧಕ್ಕೆ ಖಚಿತತೆ ಬರುತ್ತದೆ ಎಂದು ಹೇಳಿದ್ದರು. ಈ ಹಿಂದೆ ಸಾವಿರಾರು ಅಡಿಟಿಪ್ಪಣಿಗಳೊಂದಿಗೆ ಪ್ರಕಟವಾಗಿರುವ ಇಂತಹ ಅನೇಕ ಪ್ರಮುಖ ದೊಡ್ಡ ಸಂಶೋಧನಾ ಪ್ರಬಂಧಗಳನ್ನು ಗಮನಿಸಬಹುದಾಗಿದೆ.

ಜನಸಾಮಾನ್ಯರನ್ನು ದೃಷ್ಟಿಯಲ್ಲಿಟ್ಟುಕೊಂಡು ಬರೆದಿರುವುದರಿಂದ ಕೆಲವೆಡೆ ನಿರೂಪಣೆ ಹೆಚ್ಚಾಯಿತು ಎಂದೆನಿಸು\-ತ್ತದೆ. ಕೆಲವು ಕಡೆ ಪುನರಾವರ್ತನೆಯಾಗಿರುವಂತೆ ತೋರುತ್ತದೆ. ಶಾಸನಗಳನ್ನು ಆಧರಿಸಿ ಕೃತಿ ರಚನೆ ಮಾಡುವಾಗ ಕೆಲವು ಕಡೆ ಈ ರೀತಿ ವಿಷಯಗಳ ಪುನರಾವರ್ತನೆ ಆಗೇ ಆಗುತ್ತದೆ. ಇದರಿಂದ ಆಗುವ ಪ್ರಯೋಜನ ಎಂದರೆ ಪ್ರತಿ ಅಧ್ಯಾಯಗಳನ್ನು ಓದುವಾಗ ಅವು ಪ್ರತ್ಯೇಕವಾಗಿಯೇ ನಿಲ್ಲುತ್ತವೆ. ಇಂತಹ ಕೃತಿಗಳ ಅಧ್ಯಯನದಿಂದ ಜನಸಾಮಾನ್ಯರಿಗೆ ಶಾಸನಗಳು ಹಾಗೂ\break ಸ್ಮಾರಕಗಳ ಬಗ್ಗೆ, ತಮ್ಮ ಊರಿನ ಬಗ್ಗೆ ಒಂದು ರೀತಿಯ ಅಭಿಮಾನ ಮೂಡಬೇಕು, ಅದರಿಂದ ಶಾಸನಗಳು ಮತ್ತು ಸ್ಮಾರಕಗಳ ರಕ್ಷಣೆಗೆ ದಾರಿಯಾಗುತ್ತದೆ. ತಮ್ಮ ಊರು ಇಷ್ಟೊಂದು ಪ್ರಾಚೀನವಾದುದು, ಐತಿಹಾಸಿಕವಾಗಿ ಇಷ್ಟೊಂದು ಮುಖ್ಯವಾದುದು ಎಂಬ ಅಭಿಮಾನ ಮೂಡುತ್ತದೆ. ಇಂತಹ ಒಂದು ಮನೋಭಾವನೆ ಇಂದು ನಮ್ಮ ಯುವಕರಲ್ಲಿ, ಜನಸಾಮಾನ್ಯರಲ್ಲಿ ಮೂಡು\-ತ್ತಿದೆ. ನನ್ನ ಈ ಒಂದು ಕೃತಿಯಿಂದ ಅದಕ್ಕೆ ಇಂಬು ದೊರೆತರೆ ಅದೇ ನನಗೆ ತೃಪ್ತಿ. ಅಲ್ಲಲ್ಲಿ ಶಾಸನೋಕ್ತ ಪದ್ಯಗಳನ್ನು, ಶಾಸನದ ಪದಗಳನ್ನು, ವಾಕ್ಯಗಳನ್ನು ಹಾಗೆಯೇ ಉಳಿಸಿಕೊಂಡು ಉಲ್ಲೇಖಿಸಿದ್ದೇನೆ. ಇದರಿಂದ ನಿರೂಪಣೆಗೆ ಖಚಿತತೆ ಬಂದಿದೆ. ಕನ್ನಡ ಸಾಹಿತ್ಯ ಪರಿಷತ್ತಿನಲ್ಲಿ, ಶಾಸನ ಶಾಸ್ತ್ರ ತರಗತಿಯಲ್ಲಿ ಅತ್ಯುತ್ತಮ ರೀತಿಯಲ್ಲಿ ಬೋಧನೆ ಮಾಡಿ, ಈ ಕ್ಷೇತ್ರದ ಕಡೆಗೆ ನಮ್ಮನ್ನು ಸಂಪೂರ್ಣವಾಗಿ ಆಕರ್ಷಿಸಿದವರು, ಶಾಸನಗಳ ಸಂಶೋಧನೆಯ ಮೂಲಪಾಠಗಳನ್ನು ಕಲಿಸಿದವರು, ಕರ್ನಾಟಕ ಶಾಸನ ಅಧ್ಯಯನ ಕ್ಷೇತ್ರದಲ್ಲಿ ಅತ್ಯಂತ ಪ್ರಸಿದ್ಧರಾದ ಡಾ.ಕೆ.ಆರ್​.ಗಣೇಶ್​, ಡಾ.ದೇವರ ಕೊಂಡಾರೆಡ್ಡಿ, ಡಾ. ಪಿ.ವಿ. ಕೃಷ್ಣಮೂರ್ತಿ, ಡಾ.ಹೆಚ್​.ಎಸ್​. ಗೋಪಾಲರಾವ್​, ಡಾ.ಎಲ್​.ಎಸ್​. ಶ‍್ರೀನಿವಾಸಮೂರ್ತಿ, ಪ್ರೊ. ಲಕ್ಷ್ಮಣತೆಲಗಾವಿ ಈ ಗುರುಗಳ ಸಮೂಹ. ಇವರಿಗೆ ನಾನು ಅತ್ಯಂತ ವಿನೀತನಾಗಿ ನನ್ನ ವಂದನೆಗಳನ್ನು ಸಲ್ಲಿಸುತ್ತೇನೆ. ನಮ್ಮ ಊರಿನವರು, ನನ್ನ ಸಹಪಾಠಿಗಳೂ ಮಿತ್ರರೂ ಆದ, ನಿವೃತ್ತ ಇತಿಹಾಸ ಪ್ರಾಧ್ಯಾಪಕ ಡಾ. ಎಸ್​.ಎನ್​. ಶಿವರುದ್ರಸ್ವಾಮಿಯರೂ, ಈ ಕ್ಷೇತ್ರದಲ್ಲಿ ನಾನು ಅಧ್ಯಯನ ಮಾಡಿ ಬರವಣಿಗೆ ಮಾಡಲು ಕಾರಣಕರ್ತರು. ಕರ್ನಾಟಕದ ಪ್ರಖ್ಯಾತ ಗಮಕಿಗಳು, ಹಿರಿಯ ಅಧಿಕಾರಿಗಳೂ ಆದ ಡಾ. ಎ.ವಿ. ಪ್ರಸನ್ನ ಅವರು ನನಗೆ ಪರಿಚಿತರು. ಅವರಿಗೂ ನನ್ನ ಪಿಎಚ್​.ಡಿ. ಅಧ್ಯಯನದ ಬಗ್ಗೆ ಹೇಳಿದ್ದೆ. ನನ್ನ ಪ್ರಬಂಧವನ್ನು ಅವರ ಪರಾಮರ್ಶೆಗೆ ನೀಡಿದ್ದೆ. ಅವರೂ ಕೂಡಾ ಪ್ರೋತ್ಸಾಹದಾಯಕ ಮಾತುಗಳನ್ನು ಆಡಿ, ತಮ್ಮ ಅಭಿಪ್ರಾಯವನ್ನು ನೀಡಿದ್ದಾರೆ. ಅವರಿಗೂ ನಾನು ಆಭಾರಿಯಾಗಿದ್ದೇನೆ. ನಮ್ಮ ಗುರುಗಳಾದ ಡಾ.ಕೆ.ಆರ್​.ಗಣೇಶ್​ ಅವರು ಕೇಳಿದ ತಕ್ಷಣ ತಮ್ಮ ಮೆಚ್ಚುಗೆಯ ನುಡಿಯನ್ನು ಬರೆದು\-ಕೊಟ್ಟರು. ನನ್ನ ಸಹಪಾಠಿಗಳು, ಬಾಲ್ಯ ಸ್ನೇಹಿತರು, ಇತಿಹಾಸ ವಿದ್ವಾಂಸರೂ ಆದ ಡಾ.ಎಸ್​.ಎನ್​.ಶಿವರುದ್ರ ಸ್ವಾಮಿಯವರು ನನ್ನ ಅಧ್ಯಯನವನ್ನು ಪ್ರೋತ್ಸಾಹಿಸಿ ಮೆಚ್ಚುಗೆಯ ಮಾತುಗಳನ್ನು ಆಡಿದ್ದಾರೆ. ಅವರೆಲ್ಲರಿಗೂ ನಮ್ರತಾಪೂರ್ವಕ ವಂದನೆಗಳು.

ನನ್ನ ಈ ಒಂದು ಸಂಶೋಧನಾತ್ಮಕ ಕೃತಿಯನ್ನು ಬಹಳ ರಿಸ್ಕ್​ ತೆಗೆದುಕೊಂಡು ಪ್ರಕಟಿಸುತ್ತಿರುವ ಸುಮೇರು ಸಾಹಿತ್ಯದ ಶ‍್ರೀಮತಿ ಸುಮಿತ್ರಾ ದರ್ಶನ್​ ಅವರಿಗೆ ನಾನು ಅತ್ಯಂತ ಕೃತಜ್ಞನಾಗಿದ್ದೇನೆ. ಸುಮೇರು ಸಾಹಿತ್ಯದ ಮೂಲಕ ಅವರು ಇಂತಹ ಅನೇಕ ಮೌಲ್ಯಯುತವಾದ ಮೇರು ಕೃತಿಗಳನ್ನು ಪ್ರಕಟಿಸಿದ್ದಾರೆ. ಅವರಿಗೆ ನನ್ನನ್ನು ಪರಿಚಯಿಸಿ, ನನ್ನ ಕೃತಿಯು ಪ್ರಕಟಗೊಳ್ಳಲು ಅನುವು ಮಾಡಿಕೊಟ್ಟ ಶ‍್ರೀ ಶಾಮಸುಂದರ ರಾಯರಿಗೆ, ಶ‍್ರೀಮತಿ ಮುದ್ದಮ್ಮ ಶಾಮಸುಂದರರಾಯರಿಗೆ ನಾನು ಅತ್ಯಂತ ಆಭಾರಿಯಾಗಿದ್ದೇನೆ. ಅವರು ಈ ಕೃತಿಯನ್ನು ಸಿದ್ಧಪಡಿಸಿಕೊಡಬೇಕೆಂದು ಹೇಳಿದ ಕೊನೆಯ ಗಡುವುಗಿಂತ ಆರು ತಿಂಗಳ ನಂತರ ನಾನು ಈ ಕೃತಿಯನ್ನು ನೀಡಿದರೂ, ಬೇಸರ ಪಟ್ಟುಕೊಳ್ಳದೆ ಪ್ರಕಟಣೆ ಮಾಡಿದ್ದಾರೆ.

ನನ್ನ ಪಿಎಚ್​.ಡಿ. ಸಂಶೋಧನಾ ಪ್ರಬಂಧವು, ಸುಮಾರು 3000ಕ್ಕೂ ಹೆಚ್ಚು ಅಡಿಟಿಪ್ಪಣಿಗಳು, ಅನುಬಂಧಗಳು ಹಾಗೂ ಛಾಯಾಚಿತ್ರಗಳೂ ಸೇರಿದಂತೆ ಸುಮಾರು 1100 ಪುಟಗಳಾಗಿತ್ತು. ಈ ಪ್ರಬಂಧದ ಟೈಪ್​ಸೆಟ್ಟಿಂಗ್​(ಡಿಟಿಪಿ) ಕಾರ್ಯವೇ ಕ್ಲಿಷ್ಟಕರವಾದುದು. ನನ್ನ ಪಿಎಚ್​.ಡಿ. ಪ್ರಬಂಧವನ್ನು ವಿಶೇಷವಾದ ರೀತಿಯ ಯೂನಿಕೋಡ್​ ಫಾಂಟ್​ನಲ್ಲಿ ಬಹಳ ಶ್ರಮ\-ವಹಿಸಿ, ಬಹಳ ಸುಂದರವಾಗಿ ಟೈಪ್​ಸೆಟ್​ ಮಾಡಿಸಿ, ಎರಡು ಸಂಪುಟಗಳಲ್ಲಿ ಬೈಂಡಿಂಗ್​ ಮಾಡಿಸಿಕೊಟ್ಟವರು, ಶ‍್ರೀರಂಗಪಟ್ಟಣದ ಶ‍್ರೀರಂಗ ಡಿಜಿಟಲ್ಸ್​ನ ಮಾರ್ಗದರ್ಶಕರಾದ ಪ್ರೊ. ಡಾ. ಸಿ.ಎಸ್​.ಯೋಗಾನಂದ್​ ಹಾಗೂ ವ್ಯವಸ್ಥಾಪಕರಾದ ಶ‍್ರೀ ಅರ್ಜುನ್​ ಅವರು. ಈಗಲೂ ಅವರು ಬದಲಾದ, ಕ್ಲಿಷ್ಟಕರವಾದ ಈ ಕೃತಿಯನ್ನು, ತಮ್ಮ ಬಿಡುವಿಲ್ಲದ ಕಾರ್ಯಭಾರದ ನಡುವೆಯೂ, ಪ್ರೀತಿಯಿಂದ ಬಹಳ ಸುಂದರವಾಗಿ ಟೈಪ್​ಸೆಟ್ಟಿಂಗ್​ ಮಾಡಿಸಿಕೊಟ್ಟಿದ್ದಾರೆ. ಅವರಿಗೆ ನಾನು ಎಷ್ಟು ಆಭಾರಿ\-ಯಾಗಿದ್ದರೂ ಸಾಲದು. ಅವರ ಸಂಸ್ಥೆಯಲ್ಲಿ ಈ ಕಾರ್ಯಭಾರವನ್ನು ಅವರ ನಿರ್ದೇಶನದಂತೆ ಸುಂದರವಾಗಿ ಮಾಡಿಕೊಟ್ಟ, ಶ‍್ರೀ ಶಿವಶಂಕರ್​, ಶ‍್ರೀ ರಾಘವೇಂದ್ರ ಮತ್ತು ಅವರ ಸಹೋದ್ಯೋಗಿಗಳಿಗೆ ನಾನು ಚಿರಋಣಿಯಾಗಿದ್ದೇನೆ. ಈ ಕೃತಿಗೆ ಸುಂದರವಾದ ಮುಖಪುಟ ರಚನೆ ಮಾಡಿ, ಛಾಯಾ ಚಿತ್ರಗಳನ್ನು ಜೋಡಿಸಿ ಲೇಔಟ್​ ಮಾಡಿಕೊಟ್ಟ, ಮಂಜುಶ‍್ರೀ ಎಂಟರ್​\-ಪ್ರೈಸಸ್​ನ ಬಿ.ವಿ. ಗೋಪಾಲಕೃಷ್ಣ ಅವರಿಗೆ ನನ್ನ ವಂದನೆಗಳು. ಇಂತಹ ಕೃತಿಯನ್ನು ಸುಂದರವಾಗಿ ಮುದ್ರಣ ಮಾಡುವುದೂ ತಾಂತ್ರಿಕವಾಗಿ ಒಂದು ತಾಂತ್ರಿಕ ಮಹತ್ವದ. ಸುಮಾರು 25-30 ವರ್ಷಗಳಿಂದ ನನಗೆ ಪರಿಚಿತರಾದ, ಸಜ್ಜನರಾದ, ಹೆಸರಾಂತ ಪರಿಮಳ ಮುದ್ರಣಾಲಯದ ಮಾಲೀಕರಾದ ಶ‍್ರೀ ಕೆ.ಸಿ. ಪ್ರಭಾಕರ್​ ಅವರು ಅಲ್ಪ ಅವಧಿಯಲ್ಲಿ ಬಹಳ ಸುಂದರವಾಗಿ ಈ ಕೃತಿಯನ್ನು ಮುದ್ರಿಸಿ ಕೊಟ್ಟಿದ್ದಾರೆ. ಅವರಿಗೆ ಅವರ ಮುದ್ರಣಾಯಲದ ಶ‍್ರೀಕಾಂತ್​ ಅವರಿಗೆ ಹಾಗೂ ಅವರ ಸಹೋದ್ಯೋಗಿ\-ಗಳಿಗೆ ನಾನು ಆಭಾರಿಯಾಗಿದ್ದೇನೆ.

ಮನೆಯ ಕಡೆ ಎಲ್ಲಾ ಜವಾಬ್ದಾರಿಯನ್ನೂ ಹೊತ್ತುಕೊಂಡು, ಯಾವಾಗಲೂ ನನ್ನ ಓದುಬರಹಕ್ಕೆ ಒತ್ತಾಸೆಯಾಗಿರುವ ನನ್ನ ಪತ್ನಿ ಶ‍್ರೀಮತಿ ರಾಜೇಶ್ವರಿ, ನನ್ನ ಮಕ್ಕಳು, ಚಂದ್ರಮೌಳಿ, ಯಶಸ್ವಿನಿ, ನನ್ನ ಸೊಸೆ ಸುಮನಾ ಇವರಿಗೆ ನನ್ನ ಹೃತ್ಪೂರ್ವಕ ಪ್ರೀತಿಭಾವನೆಯನ್ನು ವ್ಯಕ್ತಪಡಿಸುತ್ತೇನೆ. ನನ್ನನ್ನು ಹಳ್ಳಿಗಳಿಗೆ ಬೈಕ್​ ಮೇಲೆ ಕರೆದುಕೊಂಡು ಹೋಗಿದ್ದ ನನ್ನ ತಮ್ಮನ ಮಕ್ಕಳಾದ ರೋಹಿತ್​ ಶ‍್ರೀಪತಿ, ಶ‍್ರೀಕಾಂತ್​ ಸುಬ್ರಹ್ಮಣ್ಯ, ಹಾಗೂ ಹಿರಿಯರಾದ ಬೆಳ್ಳೂರಿನ ಶ‍್ರೀ ಲಕ್ಷ್ಮೀನಾರಾಯಣ ಅವರಿಗೆ ನನ್ನ ವಂದನೆಗಳು.

ಸುಮಾರು 38 ವರ್ಷಗಳ ಕಾಲ ಸರ್ಕಾರಿ ನೌಕರನಾಗಿದ್ದು, ಶಾಸನಗಳ ಅಧ್ಯಯನ ಮತ್ತು ಸಂಶೋಧನಾ ಕ್ಷೇತ್ರಕ್ಕೆ, ಕೃತಿ ರಚನೆಗೆ ತೀರಾ ಹೊಸಬನಾದ ನಾನು, ನಮ್ಮ ಗುರುಗಳ ಬೋಧನೆ ಮತ್ತು ಮಾರ್ಗದರ್ಶನದಿಂದ, ಈ ವಿಷಯದಲ್ಲಿ ಅಧ್ಯಯನ ಮಾಡಿ, ನಮ್ಮ ಮಂಡ್ಯ ಜಿಲ್ಲೆಯ ಮೇಲಿನ ಅಭಿಮಾನದಿಂದ, ಒಂದು ಕೃತಿಯನ್ನು ರಚಿಸಿದ್ದೇನೆ. ಈ ಕ್ಷೇತ್ರದಲ್ಲಿ\break ವಿದ್ವಾಂಸರಾಗಿರುವ ಅನೇಕ ಮಹನೀಯರಿಗೆ, ಈ ಕೃತಿಯಲ್ಲಿ ಅನೇಕ ಲೋಪದೋಷಗಳು ಕಾಣಬಹುದು. ದಯಮಾಡಿ ಅದನ್ನು ಮನ್ನಿಸಿ ಈ ಕೃತಿಯನ್ನು ಪರಾಂಬರಿಸಿ ಮಾರ್ಗದರ್ಶನ ಮಾಡಬೇಕೆಂದು ವಿನಂತಿಸುತ್ತೇನೆ. ಸಾಹಿತ್ಯ, ಇತಿಹಾಸ ಮತ್ತು ಸಂಸ್ಕೃತಿ ಪ್ರಿಯರಾದ ಕನ್ನಡ ನಾಡಿನ ಜನರು, ಮಂಡ್ಯ ಜಿಲ್ಲೆಯ ಜನರು, ಇತಿಹಾಸ ಮತ್ತು ಸಂಸ್ಕೃತಿಯ ಕ್ಷೇತ್ರದ ವಿದ್ವಾಂಸರು, ವಿದ್ಯಾರ್ಥಿಗಳು, ಈ ನನ್ನ ಕೃತಿಯನ್ನು ಆದರದಿಂದ ಬರಮಾಡಿಕೊಳ್ಳುವರೆಂಬ ವಿಶ್ವಾಸವನ್ನು ವ್ಯಕ್ತಪಡಿಸುತ್ತೇನೆ.

\begin{flushright}
\textbf{ಸಂತೇಬಾಚಹಳ್ಳಿ ಡಾ. ಎಸ್​. ನಂಜುಂಡಸ್ವಾಮಿ.}
\end{flushright}

