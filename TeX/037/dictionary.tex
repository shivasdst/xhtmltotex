\sethyphenation{kannada}{
ಅಂಕ-ಕಾರ
ಅಂಕ-ಕಾರ-ನಾಗಿದ್ದ-ನೆಂದಯ
ಅಂಕ-ಕಾರ-ನೆಂದು
ಅಂಕ-ಕಾರ-ಸೇನಾ-ಧಿ-ಪತಿ
ಅಂಕ-ಗಾವುಂಡ
ಅಂಕನ-ಹಳ್ಳಿ
ಅಂಕನ-ಹಳ್ಳಿ-ಯನ್ನು
ಅಂಕ-ನಾಥ-ಪುರದ
ಅಂಕಿ-ಗಳು
ಅಂಕುಶ-ರಾಯ
ಅಂಕುಶೇಂದ್ರನ
ಅಂಕುಶೇಂದ್ರನಿರ-ಬಹುದು
ಅಂಕೆಯ
ಅಂಕೆಯ-ದಂಣ್ನಾಯ-ಕರ
ಅಂಕೆಯ-ನಾಯ-ಕನು
ಅಂಗ-ಭೋಗ
ಅಂಗರಕ್ಕ
ಅಂಗರಕ್ಕ-ನೆಂದು
ಅಂಗ-ವಾಗಿದ್ದ
ಅಂಘರಕ-ರಿಗೆ
ಅಂಡಲೆಯು-ತ್ತಿದ್ದ
ಅಂಣ
ಅಂಣ-ನೆಂದೆನಿಸಿ
ಅಂಣ್ನಯ್ಯನು
ಅಂತಃಕಲಹದ
ಅಂತಃಪುರಾಧ್ಯಕ್ಷ
ಅಂತರ
ಅಂತ-ರಂಗದ
ಅಂತರ-ವಳ್ಳಿ
ಅಂತರ-ವಳ್ಳಿಯ
ಅಂತರ-ವಳ್ಳಿ-ಯಲ್ಲಿ
ಅಂತರಿಕ
ಅಂತರ್ಗತ-ವಾಗಿತ್ತು
ಅಂತರ್ಗತ-ವಾಗಿದ್ದಂತೆ
ಅಂತರ್ಗತ-ವಾಗಿದ್ದವು
ಅಂತರ್ಗತ-ವಾಗಿದ್ದ-ವೆಂದು
ಅಂತಸ್ತು-ಗಳನ್ನು
ಅಂತಹ
ಅಂತಹವ-ರಲ್ಲಿ
ಅಂತೆಂಬರ-ಗಂಡ
ಅಂತ್ಯ-ಕಾಲ-ದಲ್ಲಿ
ಅಂತ್ಯ-ದಲ್ಲಿ
ಅಂದ-ಮೇಲೆ
ಅಂದರೆ
ಅಂದರೆ-ರಲ್ಲಿ
ಅಂದಿನ
ಅಂದಿ-ನಿಂದ
ಅಂದೇ
ಅಂಬಲಿ
ಅಂಬಾರಿಯ
ಅಂಶ
ಅಂಶ-ಗಳ
ಅಂಶ-ಗಳನ್ನು
ಅಂಶ-ಗಳಾವುವೂ
ಅಂಶ-ಗಳು
ಅಂಶ-ವಾಗಿದೆ
ಅಂಶ-ವಾಗು-ವು-ದಿಲ್ಲ
ಅಂಶವು
ಅಉಬಳ-ದೇವ
ಅಕಜಾ-ಪುರವು
ಅಕರ-ವಾಗಿ
ಅಕಳಂಕನ
ಅಕ-ಸಾಲೆ-ಗಳು
ಅಕಾಡೆಮಿ-ಯಿಂದ
ಅಕಾರಾದಿ-ಯಾಗಿ
ಅಕಾಲ-ವರ್ಷ
ಅಕಾಲ-ವರ್ಷನು
ಅಕಾಲ-ವರ್ಷ-ನು-ಇಮ್ಮಡಿ
ಅಕ್ಕ
ಅಕ್ಕ-ಜಾ-ಪುರದ
ಅಕ್ಕನ
ಅಕ್ಕ-ಪಕ್ಕದ
ಅಕ್ಕ-ಪಕ್ಕ-ದಲ್ಲಿ
ಅಕ್ಕ-ಪಕ್ಕ-ದಲ್ಲಿದ್ದ
ಅಕ್ಕಪ್ಕ-ದಲ್ಲಿ
ಅಕ್ಕಮಾ
ಅಕ್ಕ-ರ-ಸಾಕ್ಷಿ
ಅಕ್ಕ-ಸಾಲಿ-ಗರು
ಅಕ್ಕಿಯೆಬ್ಬಾಳು
ಅಕ್ಕಿ-ಹೆಬ್ಬಾ-ಳಿಗೆ
ಅಕ್ಕಿ-ಹೆಬ್ಬಾಳು
ಅಕ್ಟೋಬರ್
ಅಕ್ಷತೃತೀಯ-ದಂದು
ಅಕ್ಷರ-ವನ್ನು
ಅಖಂಡ
ಅಖಂಡ-ಬಾ-ಗಿಲಿನ
ಅಖಂಡಿತ
ಅಖಿಲ-ಭಾರತ
ಅಖಿಳ-ಗುಣ-ಧಾರೆ
ಅಗಣ್ಯ-ಪುಣ್ಯವೇ
ಅಗತಿ-ಯಪ್ಪ
ಅಗ-ತಿಯಪ್ಪನ
ಅಗತಿ-ಯಪ್ಪ-ನಿಗೆ
ಅಗತ್ತಿ
ಅಗತ್ಯ
ಅಗತ್ಯ-ಗ-ಳಿಗೆ
ಅಗತ್ಯ-ವಾಗಿದೆ
ಅಗತ್ಯ-ವಿಲ್ಲ-ವೆಂದು
ಅಗಮ್ಯ-ವಾಗಿತ್ತೆಂದು
ಅಗರ-ದಿಂದ
ಅಗರ-ದುರ್ಗ-ಅಗ್ರಹಾರ
ಅಗ-ಸರ-ಹಳ್ಳಿ
ಅಗಸ್ತ್ಯೇಶ್ವರ
ಅಗ್ರಅ-ಹಾರ
ಅಗ್ರಗಣ್ಯ-ನೆನಿಸಿ
ಅಗ್ರಹಾರ
ಅಗ್ರಹಾರಕ್ಕೆ
ಅಗ್ರಹಾರ-ಗಳ
ಅಗ್ರಹಾರ-ಗಳನ್ನಾಗಿ
ಅಗ್ರಹಾರ-ಗಳನ್ನು
ಅಗ್ರಹಾರ-ಗಳಲ್ಲಿ
ಅಗ್ರಹಾರ-ಗಳು
ಅಗ್ರಹಾರದ
ಅಗ್ರಹಾರ-ದ-ಮಜ್ಜಿಗೆ-ಪುರದ
ಅಗ್ರಹಾರ-ದಲ್ಲಿ
ಅಗ್ರಹಾರ-ದಲ್ಲಿದ್ದ
ಅಗ್ರಹಾರ-ದ-ವ-ರೆಂದು
ಅಗ್ರಹಾರ-ದಿಂದ
ಅಗ್ರಹಾರದ್ವ-ಯವನ್ನಾಗಿ
ಅಗ್ರಹಾರ-ನ-ವನ್ನು
ಅಗ್ರಹಾರ-ಬಾಚ-ಹಳ್ಳಿ
ಅಗ್ರಹಾರ-ಬಾಚ-ಹಳ್ಳಿಯ
ಅಗ್ರಹಾರ-ಬಾಚ-ಹಳ್ಳಿ-ಯನ್ನು
ಅಗ್ರಹಾರ-ಬೆಳಗಲಿ
ಅಗ್ರಹಾರ-ಬೆಳಗಲಿ-ಯಲ್ಲಿ
ಅಗ್ರಹಾರ-ಬೆಳಗುಲಿ
ಅಗ್ರಹಾರ-ಬೆಳುಗಲಿ-ಯಲ್ಲಿ
ಅಗ್ರಹಾರವ
ಅಗ್ರಹಾರ-ವನ್ನಾಗಿ
ಅಗ್ರಹಾರ-ವನ್ನಾಗಿಯೂ
ಅಗ್ರಹಾರ-ವನ್ನು
ಅಗ್ರಹಾರ-ವಾಗಿ
ಅಗ್ರಹಾರ-ವಾ-ಗಿತ್ತು
ಅಗ್ರಹಾರ-ವಾಗಿದ್ದು
ಅಗ್ರಹಾರ-ವಾಗಿಯೂ
ಅಗ್ರಹಾರ-ವಾದ
ಅಗ್ರಹಾರವು
ಅಚ್ಚುತ-ರಾಯ
ಅಚ್ಚೊತ್ತಿದ
ಅಚ್ಯತ-ರಾಯ-ವೀರಣ್ಣ
ಅಚ್ಯತೇಂದ್ರ
ಅಚ್ಯುತ
ಅಚ್ಯುತ-ದೇವ
ಅಚ್ಯುತ-ದೇವನ
ಅಚ್ಯುತ-ದೇವ-ನನ್ನು
ಅಚ್ಯುತ-ದೇವನು
ಅಚ್ಯುತ-ದೇವ-ಮಹಾ-ರಾಯನು
ಅಚ್ಯು-ತನ
ಅಚ್ಯುತ-ಪುರ-ವೆಂಬ
ಅಚ್ಯುತ-ಮಹಾ-ರಾಯ-ರಿಗೆ
ಅಚ್ಯುತ-ರಾಯ
ಅಚ್ಯುತ-ರಾಯನ
ಅಚ್ಯುತ-ರಾಯ-ನನ್ನು
ಅಚ್ಯುತ-ರಾಯ-ನಿಗೆ
ಅಚ್ಯುತ-ರಾಯನು
ಅಚ್ಯುತ-ರಾಯ-ರಿಗೆ
ಅಚ್ಯುತ-ಸಮುದ್ರ
ಅಚ್ಯುತಾಖ್ಯೋ
ಅಚ್ಯುತಿ-ಮಯ್ಯ
ಅಚ್ಯುತೇಂದ್ರ
ಅಚ್ಯುತೇಂದ್ರನು
ಅಜಿತ-ಮುನಿ
ಅಜಿತ-ಸೇನ-ರೆಂದು
ಅಜಿತ-ಸೇನಾ-ಚಾರ್ಯರ
ಅಜಿತಾ-ದೇವಿ
ಅಜ್ಜ
ಅಜ್ಜ-ನ-ಹಳ್ಳಿ
ಅಜ್ಜ-ನಾಯ-ಕ-ನ-ಹಳ್ಳಿ
ಅಜ್ಜ-ನಿಗೆ
ಅಜ್ಜ-ವೂ-ರನ್ನೇ
ಅಜ್ಜ-ವೂರು
ಅಜ್ಜಿ
ಅಜ್ಞಾತ
ಅಜ್ಞೆ
ಅಜ್ಞೆಯ
ಅಟ್ಟಾಡಿಸಿ-ಕೊಂಡು
ಅಟ್ಟಿ-ದನು
ಅಟ್ಟಿದ-ನೆಂದು
ಅಟ್ಟಿ-ದ-ನೆಂದೂ
ಅಟ್ಟಿಸಿ-ಕೊಂಡು
ಅಟ್ಟುಣ್ಣಲೀ-ಯದೆ
ಅಠವಣ
ಅಠವಣೆ
ಅಠವ-ಣೆಯ
ಅಡಕೆ
ಅಡಕೆ-ಮ-ರದ
ಅಡಕೆ-ಯತೊಟ
ಅಡಕೆ-ಯ-ತೋಟ
ಅಡಗಿ-ದರೂ
ಅಡಗಿದ್ದನ್ನು
ಅಡಗಿ-ಸಲು
ಅಡಗಿಸಿ
ಅಡಗಿ-ಸಿದ
ಅಡಗಿಸಿ-ದ-ನೆಂದೂ
ಅಡಗಿಸು-ವು-ದಕ್ಕಾಗಿ
ಅಡಪ
ಅಡಳಿತ-ದಲ್ಲಿ
ಅಡಿ
ಅಡಿ-ಕೆ-ಮ-ರದ
ಅಡಿ-ಗೈ-ಮಾನ್
ಅಡೆತಡೆ
ಅಡೆತಡೆ-ಗಳನ್ನು
ಅಡೆತಡೆ-ಗಳೂ
ಅಡ್ಡ-ಲಾಗಿ
ಅಡ್ಡ-ಹೆ-ಸರು
ಅಡ್ಡಾಯಿದದ
ಅಡ್ಡಾಯುಧಅಢಾಯುಧ
ಅಡ್ಡಾಯ್ದದ
ಅಣತಿ
ಅಣಿ-ಲೇಶ್ವರ
ಅಣುವ-ಸಮುದ್ರದ
ಅಣುವ-ಸಮುದ್ರ-ದಲ್ಲಿ-ಇಂದಿನ
ಅಣೆ-ಕಟ್ಟನ್ನು
ಅಣೆ-ಕಟ್ಟೆ-ಯನ್ನು
ಅಣ್ಣ
ಅಣ್ಣಂದಿರು
ಅಣ್ಣಂದಿರೂ
ಅಣ್ಣ-ತಮ್ಮಂದಿರ
ಅಣ್ಣ-ತಮ್ಮಂದಿ-ರಂತೆ
ಅಣ್ಣ-ತಮ್ಮಂದಿ-ರಾಗಿದ್ದು
ಅಣ್ಣ-ತಮ್ಮಂದಿ-ರೊಳಗೆ
ಅಣ್ಣನ
ಅಣ್ಣ-ನಂಕುರ
ಅಣ್ಣ-ನನ್ನು
ಅಣ್ಣ-ನ-ವರ
ಅಣ್ಣ-ನಾದ
ಅಣ್ಣ-ನಿ-ಗಿಂತ
ಅಣ್ಣನು
ಅಣ್ಣನೂ
ಅಣ್ಣ-ನೆಂದು
ಅಣ್ಣ-ನೆಂಬುದೂ
ಅಣ್ಣಪ್ಪ
ಅಣ್ಣ-ಬೂಚಣ-ನಿಗೆ
ಅಣ್ಣಯ್ಯನ-ವರ
ಅಣ್ಣಾ-ಮಲೆ
ಅಣ್ನಯ್ಯ
ಅಣ್ನಾ-ಮಲೆ
ಅತಿ-ಕುಪ್ಪೆ
ಅತಿರಾತ್ರಿಗೆ
ಅತಿವಿಷಮ
ಅತಿಶ-ಯ-ವಾಗಿ
ಅತಿಶಯ-ವಾದ
ಅತಿ-ಸಣ್ಣ
ಅತೀ
ಅತೀ-ತ-ನಾ-ದನು
ಅತು-ವಾಸು-ವಿನ
ಅತ್ತಿ-ಕುಪ್ಪೆ
ಅತ್ತಿ-ಕುಪ್ಪೆ-ಗಳನ್ನು
ಅತ್ತಿ-ಕುಪ್ಪೆ-ಯೆಂಬ
ಅತ್ತಿ-ಗುಪ್ಪೆ
ಅತ್ತಿ-ಗುಪ್ಪೆ-ಕೃಷ್ಣ-ರಾಜ-ಪೇಟೆ
ಅತ್ತಿ-ಗುಪ್ಪೆಗೆ
ಅತ್ತಿಗೆ
ಅತ್ತಿ-ಗೊಂಡ-ನ-ಹಳ್ಳಿ
ಅತ್ತಿತಾಳಾ-ದತ್ತ-ಲ-ಪುರ-ವನ್ನು
ಅತ್ತಿಮಬ್ಬೆ-ಯನ್ನು
ಅತ್ಯಂತ
ಅತ್ಯಮ-ನಾಯಕ
ಅಥವ
ಅಥವಾ
ಅದಕೆ
ಅದಕ್ಕಾಗಿ
ಅದಕ್ಕೂ
ಅದಕ್ಕೆ
ಅದನ್ನ
ಅದನ್ನು
ಅದನ್ನೂ
ಅದನ್ನೇ
ಅದರ
ಅದ-ರಂತೆ
ಅದ-ರಂತೆಯೇ
ಅದ-ರಲ್ಲಿ
ಅದ-ರಲ್ಲೂ
ಅದ-ರಿಂದ
ಅದ-ರಿಂದಲೇ
ಅದ-ರಿಂದಾಗಿ
ಅದ-ರಿಂದಾಗಿಯೇ
ಅದ-ರೊಡನೆ
ಅದರೊಳಗೆ
ಅದಲ-ಗೆರೆ
ಅದಲ್ಲ
ಅದಾ-ಗುತ್ತಲೂ
ಅದಾದ-ನಂತರ
ಅದಿ-ರದ
ಅದು
ಅದೂ
ಅದೇ
ಅದೇ-ರೀತಿ
ಅದೇವುದೋ
ಅದೊಂದು
ಅದ್ದಿಯಾಪಳ್ತಿಯ
ಅದ್ದಿ-ಹಳ್ಳಿ
ಅದ್ಯ-ತನ
ಅದ್ಯಾಪಿ
ಅದ್ವಿತೀಯ-ವಾದುದೆಂದರೆ
ಅಧಿಕಬಳ
ಅಧಿ-ಕಾರ
ಅಧಿ-ಕಾರಕ್ಕಾಗಿ
ಅಧಿ-ಕಾರಕ್ಕೆ
ಅಧಿ-ಕಾರ-ಗಳನ್ನು
ಅಧಿ-ಕಾರದ
ಅಧಿ-ಕಾರ-ದಲ್ಲಿ
ಅಧಿ-ಕಾರ-ದಲ್ಲಿದ್ದನು
ಅಧಿ-ಕಾರ-ದಲ್ಲಿದ್ದು
ಅಧಿ-ಕಾರ-ದಲ್ಲಿ-ರುವಾಗ
ಅಧಿ-ಕಾರ-ದಿಂದ
ಅಧಿ-ಕಾರ-ಪದದ
ಅಧಿ-ಕಾರ-ವನ್ನು
ಅಧಿ-ಕಾರ-ವರ್ಗ-ಗಳ
ಅಧಿ-ಕಾರ-ವರ್ಗದ-ವರು
ಅಧಿ-ಕಾರ-ವಿ-ರ-ಲಿಲ್ಲ
ಅಧಿ-ಕಾರವು
ಅಧಿ-ಕಾರವೂ
ಅಧಿ-ಕಾರ-ವೆಲ್ಲಾ
ಅಧಿ-ಕಾರಿ
ಅಧಿ-ಕಾರಿ-ಗಳ
ಅಧಿ-ಕಾರಿ-ಗಳನ್ನು
ಅಧಿ-ಕಾರಿ-ಗಳಾಗಿ
ಅಧಿ-ಕಾರಿ-ಗಳಾಗಿದ್ದರು
ಅಧಿ-ಕಾರಿ-ಗಳಾಗಿದ್ದ-ರೆಂದು
ಅಧಿ-ಕಾರಿ-ಗಳಾಗಿದ್ದ-ರೆಂಬುದು
ಅಧಿ-ಕಾರಿ-ಗಳಾಗಿದ್ದಿರ-ಬಹುದು
ಅಧಿ-ಕಾರಿ-ಗಳಾಗಿದ್ದು
ಅಧಿ-ಕಾರಿ-ಗಳಾಗಿ-ರ-ಲಿಲ್ಲ
ಅಧಿ-ಕಾರಿ-ಗ-ಳಾದ
ಅಧಿ-ಕಾರಿ-ಗಳಿಂದ
ಅಧಿ-ಕಾರಿ-ಗಳಿಗಿ-ರುವು-ದ-ರಿಂದ
ಅಧಿ-ಕಾರಿ-ಗ-ಳಿಗೆ
ಅಧಿ-ಕಾರಿ-ಗಳಿದ್ದರು
ಅಧಿ-ಕಾರಿ-ಗಳಿರ-ಬಹುದು
ಅಧಿ-ಕಾರಿ-ಗಳಿ-ರುತ್ತಿದ್ದರು
ಅಧಿ-ಕಾರಿ-ಗಳು
ಅಧಿ-ಕಾರಿ-ಗ-ಳೆಂದು
ಅಧಿ-ಕಾರಿಗೇ
ಅಧಿ-ಕಾರಿಯ
ಅಧಿ-ಕಾರಿ-ಯನ್ನು
ಅಧಿ-ಕಾರಿ-ಯನ್ನೂ
ಅಧಿ-ಕಾರಿ-ಯಾಗಿ
ಅಧಿ-ಕಾರಿ-ಯಾಗಿದ್ದ
ಅಧಿ-ಕಾರಿ-ಯಾಗಿದ್ದಂತೆ
ಅಧಿ-ಕಾರಿ-ಯಾಗಿದ್ದನು
ಅಧಿ-ಕಾರಿ-ಯಾಗಿದ್ದ-ನೆಂದು
ಅಧಿ-ಕಾರಿ-ಯಾಗಿದ್ದ-ನೆಂಬ
ಅಧಿ-ಕಾರಿ-ಯಾಗಿದ್ದರೂ
ಅಧಿ-ಕಾರಿ-ಯಾಗಿದ್ದಿರ-ಬಹುದು
ಅಧಿ-ಕಾರಿ-ಯಾಗಿದ್ದಿರ-ಬಹು-ದೆಂದು
ಅಧಿ-ಕಾರಿ-ಯಾಗಿದ್ದು
ಅಧಿ-ಕಾರಿ-ಯಾಗಿ-ರ-ಬಹುದು
ಅಧಿ-ಕಾರಿ-ಯಾಗಿ-ರ-ಬೇಕು
ಅಧಿ-ಕಾರಿ-ಯಾಗಿ-ರುತ್ತಾನೆ
ಅಧಿ-ಕಾರಿ-ಯಾದ
ಅಧಿ-ಕಾರಿಯು
ಅಧಿ-ಕಾರಿಯೂ
ಅಧಿ-ಕಾರಿಯೇ
ಅಧಿ-ಕಾರಿ-ಯೊಬ್ಬನ
ಅಧಿ-ಕಾರಿಯೋ
ಅಧಿಕೃತ-ಗೊಳಿಸಿ-ರ-ಬಹು-ದೆಂದು
ಅಧಿಕೃತಗೊಳಿ-ಸುತ್ತವೆ
ಅಧಿ-ದೈವ-ವಾಗಿ
ಅಧಿನ-ನಾಗಿದ್ದ
ಅಧಿನ-ರಾಗಿ
ಅಧಿಪ-ತಿ-ಗಳಾಗಿದ್ದ
ಅಧಿಪ-ತಿ-ಗಳಾಗಿದ್ದರು
ಅಧಿಪ-ತಿ-ಗಳಾಗಿದ್ದ-ರೆಂದು
ಅಧಿ-ಪತಿ-ಯನ್ನಾಗಿ
ಅಧಿ-ಪತಿ-ಯಾಗಿದ್ದು
ಅಧಿ-ಪತಿ-ಯಾದ-ನೆಂಬ
ಅಧಿ-ರಾಜ
ಅಧಿ-ರಾಜತ್ವದ
ಅಧಿ-ರಾಜತ್ವ-ವನ್ನು
ಅಧಿ-ರಾಜರು
ಅಧಿವಸಿತ-ನಾಗಿದ್ದ-ನೆಂದು
ಅಧೀನ
ಅಧೀನ-ತೆ-ಯನ್ನು
ಅಧೀನ-ದಲ್ಲಿ
ಅಧೀನ-ದಲ್ಲಿತ್ತು
ಅಧೀನ-ದಲ್ಲಿದ್ದ
ಅಧೀನ-ನಾಗಿ
ಅಧೀನ-ರಾಗಿ
ಅಧೀನ-ರಾಗಿದ್ದ-ರೆಂದು
ಅಧೀನ-ವಾಗಿತ್ತೆಂದು
ಅಧ್ಯಕ್ಷತೆ-ಯಲ್ಲಿ
ಅಧ್ಯ-ಯನ
ಅಧ್ಯ-ಯನಕ್ಕೆ
ಅಧ್ಯ-ಯನ-ಗ-ಳನ್ನೂ
ಅಧ್ಯ-ಯನ-ಗಳು
ಅಧ್ಯಯ-ನದ
ಅಧ್ಯ-ಯನ-ದಲ್ಲಿ
ಅಧ್ಯ-ಯನ-ದಿಂದ
ಅಧ್ಯ-ಯನ-ವನ್ನು
ಅಧ್ಯ-ಯನ-ವಾಗಲೀ
ಅಧ್ಯಾಯ
ಅಧ್ಯಾಯ-ದಲ್ಲಿ
ಅನಂತ
ಅನಂತ-ಪುರ
ಅನಂತರ
ಅನಂತಾ-ಚಾರ್ಯರ
ಅನಂತಾ-ಚಾರ್ಯರು
ಅನಂತೋಜಿ
ಅನತಿ
ಅನತಿ-ದೂ-ರದಲ್ಲಿ-ರುವ
ಅನ-ವರತ
ಅನಾದಿ
ಅನಾದಿ-ಕಾಲ-ದಿಂದ
ಅನಾಯ-ಕತ್ವ
ಅನಾಹುತ-ವನ್ನು
ಅನಿರುದ್ಧ-ನಂತಹ
ಅನು-ಕರಿಸಿ
ಅನು-ಕರಿ-ಸಿ-ದ-ನೆಂದು
ಅನು-ಕೂಲ
ಅನು-ಕೂಲಕ್ಕಾಗಿ
ಅನು-ಕೂಲಕ್ಕೋಸ್ಕರ
ಅನು-ಕೂಲ-ತೆಯ
ಅನು-ಗುಣ-ವಾಗಿ
ಅನುಜ
ಅನುಜ-ನಾದ
ಅನುನ-ಯದಿಂ
ಅನುಪಮ-ತೇಜಂ
ಅನುಪಮ-ದಾನಿ
ಅನುಪಮ-ವಾದ
ಅನು-ಬಂಧ-ದಲ್ಲಿ
ಅನುಭವ
ಅನುಭವ-ವನ್ನು
ಅನುಭವಿಸ-ಬೇಕಾಗಿ
ಅನು-ಮತ-ದಿಂದ
ಅನು-ಮತಿ
ಅನುಮ-ತಿಯ
ಅನು-ಮತಿ-ಯನ್ನು
ಅನು-ಮತಿ-ಯಿಂದ
ಅನುಮ-ತಿಯೂ
ಅನು-ಮಾನ
ಅನುಯಾಯಿ-ಗಳಾಗಿದ್ದು
ಅನುಯಾಯಿ-ಗಳೂ
ಅನುಯಾಯಿ-ಯಾಗಿದ್ದನು
ಅನುಯಾಯಿ-ಯಾಗಿದ್ದ-ನೆಂದು
ಅನುಯಾಯಿ-ಯಾದ
ಅನು-ಲಕ್ಷಿಸಿ
ಅನು-ವಂಶಿಕ-ವಾದ
ಅನು-ವಂಶೀಯ
ಅನು-ವರ-ದೊಳ್
ಅನು-ವಾದ-ವಿದ್ದಂತಿದೆ
ಅನು-ಸರಿಸಿ
ಅನು-ಸರಿ-ಸಿ-ದಂತೆ
ಅನು-ಸರಿ-ಸುತ್ತಾ
ಅನು-ಸರಿ-ಸುತ್ತಿದ್ದರೂ
ಅನು-ಸರಿ-ಸುತ್ತಿದ್ದ-ರೆಂಬುದು
ಅನೇಕ
ಅನೇಕ-ಕಡೆ
ಅನೇಕ-ರಿಗೆ
ಅನೇ-ಕರು
ಅನೇಕ-ವೇಳೆ
ಅನ್ನ
ಅನ್ನ-ಛತ್ರ
ಅನ್ನ-ದಾನಕ್ಕೆಂದು
ಅನ್ನ-ದಾನ-ಪಲ್ಲಿ-ಯಲ್ಲಿ
ಅನ್ನ-ದಾನ-ಪಳ್ಳಿ-ಯನ್ನು
ಅನ್ನ-ದಾನ-ಪಳ್ಳಿ-ಯನ್ನು-ಇಂದಿನ
ಅನ್ನ-ದಾನ-ವಿನೋದ
ಅನ್ನ-ಸತ್ರ-ವನ್ನು
ಅನ್ಯ
ಅನ್ಯಗ್ರಂಥ-ಗ-ಳಿಗೆ
ಅನ್ಯ-ರೂಪ-ಗಳು
ಅನ್ಯೋನ್ಯತೆ-ಯಿಂದ
ಅನ್ವಯ
ಅನ್ವ-ಯಕ್ಕೆ
ಅನ್ವಯ-ವನ್ನು
ಅನ್ವಯಾಗತ
ಅನ್ವಯಾಗತ-ವಾಗಿ
ಅನ್ವಯಾಗದ
ಅನ್ವ-ಯಾವ-ತಾರವೆಂತೆಂದಡೆ
ಅನ್ವಯಿ-ಸುತ್ತ-ದೆಂದು
ಅಪತಿಯ
ಅಪಭ್ರಂಶ
ಅಪ-ಮಾನ-ಗಳು
ಅಪರಿಮಿತ-ದಾನ-ಸಾರವೃಷ್ಟಿ
ಅಪರೂ-ಪಕ್ಕೆ
ಅಪ-ರೂಪದ
ಅಪ-ವಾದಾತ್ಮ-ಕ-ವಾಗಿ
ಅಪ-ಹರಣ
ಅಪಾಯ
ಅಪಾರ
ಅಪಾರಪ್ರ-ಮಾ-ಣದ
ಅಪಾರ-ವಾಗಿ
ಅಪಾರ-ವಾದ
ಅಪಾರ-ಸೈನ್ಯ
ಅಪೂರ್ಬ್ಬಾಯ
ಅಪ್ಪ
ಅಪ್ಪಣೆ
ಅಪ್ಪ-ಣೆಯ
ಅಪ್ಪ-ಣೆ-ಯಂತೆ
ಅಪ್ಪ-ಣೆ-ಯನ್ನು
ಅಪ್ಪಣ್ಣ
ಅಪ್ಪಣ್ಣ-ನಾಯ-ಕನು
ಅಪ್ಪಣ್ಣ-ಭೂ-ಪತಿಯು
ಅಪ್ಪ-ನಿಂದ
ಅಪ್ಪ-ಳಕ್ಕನ-ಹಳ್ಳಿ
ಅಪ್ಪಾಜಿ
ಅಪ್ಪಿದ-ನೆಂದು
ಅಪ್ಪೆ-ನಾಯಕ
ಅಪ್ಪೆಯ
ಅಪ್ರತಿ-ಕ-ಮಲ್ಲ
ಅಪ್ರತಿಮ
ಅಪ್ರತಿ-ಮ-ತೇಜಂ
ಅಪ್ರತಿ-ಮ-ನಾಗಿದ್ದು
ಅಪ್ರತಿ-ಮ-ವೀರ
ಅಪ್ರತಿ-ಮ-ವೀರ-ನರ-ಪತಿ
ಅಪ್ರಮೇಯ
ಅಪ್ರಮೇ-ಯನ
ಅಪ್ರಮೇ-ಯನು
ಅಫ್ತಬ್ಖಾನ್
ಅಬಲ-ವಾಡಿ
ಅಬಲ-ವಾಡಿ-ಯಲ್ಲಿ
ಅಬ-ಸಮುದ್ರ-ಅಹೋ-ಬಲ-ಸಮುದ್ರ
ಅಬ್ದುಲ್
ಅಬ್ಬಗಂಜೂರು
ಅಬ್ಬ-ರಾಜ-ಗಳ
ಅಬ್ಬ-ರಾಜನ
ಅಬ್ಬ-ರಾಜು-ಗಳ
ಅಬ್ಬಾಸ್ಗಾರ್ಡನ್
ಅಬ್ಬೂರು
ಅಭಯಾರಣ್ಯ-ವಾಗಿದ್ದು
ಅಭಿನನ್ನೆಂದು
ಅಭಿನವ
ಅಭಿನವ-ಕುಲ-ಶೇಖರ-ರಾದ
ಅಭಿನವ-ಮ-ದನಾ-ವತಾರ
ಅಭಿನ್ನ-ರಾಗಿದ್ದು
ಅಭಿನ್ನ-ರಿದ್ದು
ಅಭಿನ್ನ-ರಿರ-ಬಹುದು
ಅಭಿನ್ನರು
ಅಭಿನ್ನ-ರೆಂದು
ಅಭಿಪ್ರಾಯ
ಅಭಿಪ್ರಾಯ-ಗಳನ್ನು
ಅಭಿಪ್ರಾಯ-ಪಟಿದ್ದಾರೆ
ಅಭಿಪ್ರಾಯ-ಪಟ್ಟಿದ್ದಾರೆ
ಅಭಿಪ್ರಾಯ-ಪಟ್ಟಿ-ರು-ವುದು
ಅಭಿಪ್ರಾಯ-ಪಡ-ಲಾಗಿದೆ
ಅಭಿಪ್ರಾಯ-ಪಡುತ್ತಾರೆ
ಅಭಿಪ್ರಾಯ-ವನ್ನು
ಅಭಿಪ್ರಾಯ-ವಾಗಿದೆ
ಅಭಿಪ್ರಾ-ಯವು
ಅಭಿಪ್ರಾಯವೂ
ಅಭಿ-ಮತ
ಅಭಿಮನ್ಯುವು
ಅಭಿ-ಮಾ-ನ-ದಿಂದ
ಅಭಿ-ಮಾನಿ-ಯಾಗಿದ್ದ-ನೆಂದು
ಅಭಿಯೋಗ
ಅಭಿ-ವೃದ್ಧಿ
ಅಭಿ-ವೃದ್ಧಿಗೆ
ಅಭಿ-ವೃದ್ಧಿ-ಯಾಗ-ಬೇಕೆಂದು
ಅಭಿವ್ಯಕ್ತಿಗೆ
ಅಭಿಷಿಕ್ತ
ಅಭಿಷೇ-ಕಕ್ಕೆ
ಅಭ್ಯುದಯ-ದಲ್ಲಿ
ಅಭ್ಯುದಯಾರ್ಥ
ಅಮರ
ಅಮರಂಬೋದು
ಅಮರ-ನಾಯ-ಕ-ತನಕ್ಕೆ
ಅಮರ-ನಾಯ-ಕ-ತನಕ್ಕೆ-ಸೇರಿತ್ತು
ಅಮರ-ನಾಯ-ಕ-ನಾಗಿ-ರುತ್ತಾನೆ
ಅಮರ-ನಾಯ-ಕರ
ಅಮರ-ನಾಯ-ಕ-ರಿಗೆ
ಅಮರ-ನಾಯ-ಕರು
ಅಮರ-ಪಡೆಯ
ಅಮರ-ಮಹಲೆ
ಅಮರ-ಮಾಗಣಿ
ಅಮರ-ಮಾಗಣಿಗೆ
ಅಮರ-ಮಾಗಣಿ-ಯಾಗಿ
ಅಮರ-ಮಾಗಣೆಗೆ
ಅಮರ-ಮಾಗಣೆ-ಯಾಗಿ
ಅಮರೇಂದ್ರ
ಅಮಲ್ದಾರ್
ಅಮಾತ್ಯ
ಅಮಾತ್ಯ-ನಾಗಿದ್ದ
ಅಮಾತ್ಯ-ಪದ
ಅಮಾತ್ಯರು
ಅಮಿಲ್
ಅಮೀಲ
ಅಮೀಲ್ದಾರ-ನನ್ನು
ಅಮುಕ್ತ
ಅಮುದ-ಸಮುದ್ರ
ಅಮುಲ್ದಾರ್
ಅಮೃತ-ನಾಥ-ಪುರ-ವಾದ
ಅಮೃತಪಡಿ
ಅಮೃತಪ-ಡಿಗೆ
ಅಮೃ-ತಾಂಬಾ
ಅಮೃತಾ-ಪುರ
ಅಮೃತಿ
ಅಮೃತೂರಿನ
ಅಮೃತೂರು
ಅಮೃತೇಶ್ವರ
ಅಮೋಘ-ವರ್ಷ
ಅಮೋಘ-ವರ್ಷನ
ಅಮೋಘ-ವರ್ಷನು
ಅಮ್ಮನ-ಪುರದ
ಅಮ್ಮ-ನ-ವರ
ಅಮ್ಮ-ನ-ವರ-ಗುಡಿ
ಅಮ್ಮ-ನ-ವ-ರಿಗೆ
ಅಮ್ಮ-ನ-ವರು
ಅಮ್ಮನ-ವ-ರೆಂಬ
ಅಮ್ಮಮ್ಮ
ಅಮ್ರಿತಪ-ಡಿಗೆ
ಅಯಪ
ಅಯಿರಮೆ-ನಾಯಕ
ಅಯ್ಕಣಂ
ಅಯ್ಕ-ಣದ
ಅಯ್ಕ-ಣನ
ಅಯ್ಕ-ಣನನ್ನು
ಅಯ್ಕ-ಣನಿಗೆ
ಅಯ್ಕ-ಣನು
ಅಯ್ಕ-ಣನೆಂಬ
ಅಯ್ಯ
ಅಯ್ಯಂಗಾರ್
ಅಯ್ಯ-ಗೊಂಡ-ನ-ಪಲ್ಲಿ
ಅಯ್ಯಣ
ಅಯ್ಯ-ದೇವ
ಅಯ್ಯನ
ಅಯ್ಯ-ನ-ವರ
ಅಯ್ಯ-ನ-ವ-ರಿಗೆ
ಅಯ್ಯ-ನ-ವ-ರಿಗೆ-ಕೃಷ್ಣ-ದೇವ-ರಾಯ
ಅಯ್ಯ-ನ-ವರು
ಅಯ್ಯನು
ಅಯ್ಯಪ್ಪನ-ಹಳ್ಳಿ
ಅಯ್ಯ-ರ-ವೀರ
ಅಯ್ಯಾವೊಳೆ
ಅಯ್ಯಾವೊಳೆಯ
ಅರ-ಕನ-ಕೆರೆ
ಅರಕಲಗೂಡು
ಅರ-ಕೆರೆ
ಅರ-ಕೆರೆಯ
ಅರ-ಕೆರೆ-ಯನ್ನು
ಅರಕೆಲ್ಲ
ಅರಕೇಸಿ
ಅರಕೇಸಿಯ
ಅರಕೇಸಿ-ಯ-ರಅ
ಅರಕೇಸಿಯು
ಅರಣ್ಯ
ಅರಣ್ಯ-ವನ್ನು
ಅರನ-ಕೆರೆಯ
ಅರಬ್ಬೀ
ಅರಮನೆ
ಅರಮ-ನೆಯ
ಅರ-ಮ-ನೆ-ಯಲ್ಲಿ
ಅರ-ಮ-ನೆಯಲ್ಲಿದ್ದ
ಅರ-ಮ-ನೆಯಲ್ಲಿಯೂ
ಅರಮನೆ-ಯಾನ್ತ
ಅರಮನೆ-ಯಿಂದ
ಅರಲು-ಕುಪ್ಪೆ
ಅರಳಿಮರ-ಗ-ಳಿಗೆ
ಅರ-ವತ್ತೊಕ್ಕಲಿನ
ಅರವ-ಮ-ನೆ-ಯಲ್ಲಿ
ಅರವೀಟಿ
ಅರವೀಟಿ-ರಂಗ-ರಾಜ
ಅರ-ವೀಡು
ಅರಸ
ಅರಸಂಕಸೂನೆಗಾರ
ಅರಸನ
ಅರಸ-ನ-ಕೆರೆ
ಅರಸ-ನ-ಕೆರೆಯ
ಅರಸ-ನನ್ನಾಗಿ
ಅರಸ-ನನ್ನು
ಅರಸ-ನಲ್ಲಿ
ಅರಸ-ನಾಗಿ
ಅರಸ-ನಾಗಿದ್ದ
ಅರಸ-ನಾಗಿದ್ದಲ್ಲದೆ
ಅರಸ-ನಾಗಿದ್ದು
ಅರಸ-ನಾಗಿ-ರ-ಬಹುದು
ಅರಸ-ನಾದ
ಅರಸನು
ಅರಸ-ನೆಂದು
ಅರಸನೇ
ಅರಸನೋ
ಅರಸರ
ಅರಸ-ರನ್ನು
ಅರಸ-ರಲ್ಲಿ
ಅರಸ-ರಲ್ಲಿಯೇ
ಅರಸ-ರಿಗೂ
ಅರಸ-ರಿಗೆ
ಅರಸರು
ಅರಸ-ರು-ಗಳು
ಅರಸ-ರು-ಮಣಲೇರ
ಅರಸರೂ
ಅರಸ-ರೆಂದು
ಅರಸ-ರೆಲ್ಲರೂ
ಅರಸರ್
ಅರಸರ್ಮ್ಮಹಾ-ಸಾಮನ್ತಾಧಿಪತೀ
ಅರಸಾದಿತ್ಯ
ಅರಸಿ
ಅರಸಿ-ಕೆರೆ
ಅರಸಿ-ಕೆರೆಯ
ಅರಸಿ-ಕೆರೆ-ಯಲ್ಲಿ
ಅರಸಿ-ದಂತಾ-ದುದು
ಅರಸಿನ
ಅರಸಿ-ನ-ಕೆರೆ
ಅರಸಿ-ನ-ಕೆರೆಯ
ಅರಸಿ-ಯ-ಕೆರೆ
ಅರಸಿ-ಯ-ಕೆರೆಯ
ಅರಸೀ-ಕೆರೆ
ಅರಸು
ಅರಸು-ಗಂಡ
ಅರಸು-ಗಳ
ಅರಸು-ಗಳು
ಅರಸು-ಮಕ್ಕಳು
ಅರಸೊತ್ತಿ-ಗೆಯ
ಅರಸೊತ್ತಿಗೆ-ಯನ್ನು
ಅರಾಜ-ಕತೆ-ಯನ್ನು
ಅರಿ-ಕನ-ಕಟ್ಟ
ಅರಿ-ಕನ-ಕಟ್ಟ-ವನ್ನು
ಅರಿ-ಕುಂಟೆ
ಅರಿಕುಠಾರ-ಪುರ-ದ-ವನು
ಅರಿಕುಶಕುಠಾರ
ಅರಿಕೆ
ಅರಿಗೋಧೂಮಘ-ರಟ್ಟ
ಅರಿ-ನೃಪರ
ಅರಿ-ಬಿರುದರ
ಅರಿ-ಬಿರುದ-ರ-ದಂಡ-ನಾಥ
ಅರಿಯಪ್ಪನು
ಅರಿ-ರಾಯ-ದಟ್ಟ
ಅರಿ-ರಾಯ-ವಿಭಾಡ
ಅರಿ-ರೂಪ-ಸಿಂಗ
ಅರಿ-ವರ್ಮ
ಅರುಮುಳಿ
ಅರುಮುಳಿ-ದೇವ
ಅರುಮುಳಿ-ದೇವನ
ಅರುಮುಳಿ-ದೇವನು
ಅರುಮೋಳಿ-ದೇವ
ಅರುಳ್ನಾದ-ನಿಗೆ
ಅರುವ-ನ-ಹಳ್ಳಿ
ಅರುವ-ನ-ಹಳ್ಳಿಯ
ಅರುವ-ನ-ಹಳ್ಳಿಯೇ
ಅರು-ಹನ-ಹಳಿ್ಳ
ಅರು-ಹನ-ಹಳ್ಳಿ
ಅರು-ಹನ-ಹಳ್ಳಿಯ
ಅರು-ಹನ-ಹಳ್ಳಿ-ಯನ್ನು
ಅರು-ಹನ-ಹಳ್ಳಿ-ಯಲ್ಲಿರುವ
ಅರು-ಹನ-ಹಳ್ಳಿ-ಯ-ವ-ರಿಗೂ
ಅರುಹಳಿ-ಅರು-ಹನ-ಹಳ್ಳಿ
ಅರೆ
ಅರೆ-ಕೊಠಾ-ರದ
ಅರೆ-ಕೊಠಾರ-ದಲ್ಲಿ
ಅರೆ-ತಿಪ್ಪೂರಿನ
ಅರೆ-ತಿಪ್ಪೂರು
ಅರೆ-ತಿಪ್ಪೂರೇ
ಅರೆ-ಬಂಡೆ-ಗಳು
ಅರೆ-ಬೊಪ್ಪ-ನ-ಹಳ್ಳಿ
ಅರೆ-ಯ-ಹಳ್ಳಿ
ಅರೇಬಿಕ್
ಅರ್ಕ-ಗುಪ್ತಿ-ಪುರ-ವೆಂದು
ಅರ್ಕಾವತಿ
ಅರ್ಕೇಶ್ವರ
ಅರ್ಕೇಶ್ವ-ರನ
ಅರ್ಕೇಶ್ವರಸ್ವಾಮಿ
ಅರ್ಕ್ಕ-ರದುರ್ಕ್ಕೆ-ಯಿಂದ
ಅರ್ಕ್ಕೇಶ್ವರ
ಅರ್ಚಕ-ರಂಗಸ್ವಾಮಿ-ಯ-ವರು
ಅರ್ಚ-ಕರು
ಅರ್ಚನಾ-ವೃತ್ತಿಗೆ
ಅರ್ಚನಾ-ವೃತ್ತಿ-ಯಾಗಿ
ಅರ್ಚನೆಗೆ
ಅರ್ಜುನ-ಹಳ್ಳಿ
ಅರ್ತಿ-ಗಳಿಗೂ
ಅರ್ಥ
ಅರ್ಥ-ಗಳೂ
ಅರ್ಥದ
ಅರ್ಥ-ದಲ್ಲಿ
ಅರ್ಥ-ಪೂರ್ಣ-ವಾಗಿ
ಅರ್ಥ-ವನ್ನು
ಅರ್ಥ-ವಾಗುತ್ತದೆ
ಅರ್ಥ-ವಾಗು-ವು-ದಿಲ್ಲ
ಅರ್ಥ-ವಿದೆ
ಅರ್ಥ-ವಿದೆಯೇ
ಅರ್ಥ-ವಿವೇಚ-ನೆ-ಯನ್ನು
ಅರ್ಥ-ವೆಂದು
ಅರ್ಥೈಸ-ಬಹುದು
ಅರ್ಥೈಸ-ಬಹುದೇ
ಅರ್ಥೈಸಿ
ಅರ್ಥೈಸಿದ್ದಾರೆ
ಅರ್ಥೈಸಿದ್ದಾರೆಂದು
ಅರ್ಧ
ಅರ್ಧಕ್ಕೇ
ಅರ್ಧ-ಭಾಗ
ಅರ್ಧಾಂಗ
ಅರ್ಪಿ-ಸ-ಲಾ-ಯಿತೆಂದು
ಅರ್ಪಿ-ಸಲು
ಅರ್ಪಿ-ಸುವ
ಅರ್ಪೊಳೆಯ
ಅರ್ಯಾಬಿಕ್
ಅರ್ವಾಚೀನ
ಅರ್ಹಗೇಹ-ಗಳನ್ನು
ಅರ್ಹ-ತೆಯೂ
ಅರ್ಹ-ನಲ್ಲ-ವೆಂದು
ಅರ್ಹ-ಪೂಜೆಗೆ
ಅರ್ಹ-ರನ್ನು
ಅರ್ಹ-ವಾಗಿವೆ
ಅರ್ಹ-ವಾದ
ಅಲಂಕರಿ-ಸಿದ
ಅಲಂಕರಿ-ಸಿದ್ದ
ಅಲಂಕರಿ-ಸಿದ್ದನು
ಅಲಂಕರಿ-ಸಿದ್ದರು
ಅಲಂಕರಿ-ಸಿ-ರು-ವು-ದನ್ನು
ಅಲಂಕರಿ-ಸುತ್ತಿದ್ದರು
ಅಲಂಘ್ಯ
ಅಲ-ಸಿಂಗ-ರಾರ್ಯಸ್ಯ
ಅಲಾ
ಅಲಿ
ಅಲಿ-ಖಾನ್
ಅಲಿಯ
ಅಲಿ-ಯ-ವರು
ಅಲಿಯು
ಅಲಿ-ಹೈದ-ರನು
ಅಲೀ
ಅಲ್
ಅಲ್ಪ-ಕಾಲ
ಅಲ್ಪ-ಕಾಲ-ದ-ವ-ರೆಗೆ
ಅಲ್ಲ
ಅಲ್ಲದೆ
ಅಲ್ಲಪ್ಪ
ಅಲ್ಲಪ್ಪ-ದಂಡ-ನಾಯ-ಕನು
ಅಲ್ಲಪ್ಪ-ನ-ಹಳ್ಳಿ
ಅಲ್ಲಲ್ಲಿ
ಅಲ್ಲ-ವೆಂದು
ಅಲ್ಲಾಂಬಾ
ಅಲ್ಲಾಳ
ಅಲ್ಲಾಳ-ದೇವ
ಅಲ್ಲಾಳ-ನಾಥ
ಅಲ್ಲಾಳ-ನಾಥ-ದೇವ-ರಿಗೆ
ಅಲ್ಲಾಳ-ಪೆರುಮಾಳ
ಅಲ್ಲಾಳ-ಪೆರುಮಾಳ-ದೇವ-ರಿಗೆ-ವರ-ದ-ರಾಜ
ಅಲ್ಲಾಳ-ಪೆರುಮಾಳೆ
ಅಲ್ಲಾಳ-ಸಮುದ್ರ-ವೆಂಬ
ಅಲ್ಲಿ
ಅಲ್ಲಿಂದ
ಅಲ್ಲಿಂದಲೇ
ಅಲ್ಲಿಗೆ
ಅಲ್ಲಿ-ಡ-ಲಾ-ಯಿತೆಂದು
ಅಲ್ಲಿದ್ದ
ಅಲ್ಲಿದ್ದ-ವರು
ಅಲ್ಲಿದ್ದು
ಅಲ್ಲಿನ
ಅಲ್ಲಿಯ
ಅಲ್ಲಿ-ಯ-ವ-ರೆಗೆ
ಅಲ್ಲಿಯೇ
ಅಲ್ಲಿ-ರುವ
ಅಲ್ಲೂ
ಅಲ್ಲೇ
ಅಲ್ಲೋಲ-ಕಲ್ಲೋಲ
ಅಳಗಿಯ
ಅಳಗುವಂಣನು
ಅಳ-ತೆಯ
ಅಳವಡಿಸಿ-ಕೊಂಡರೂ
ಅಳವಡಿಸಿ-ಕೊಂಡು
ಅಳಹ
ಅಳಹಿಯ
ಅಳಿಯ
ಅಳಿಯಂದಿ-ರಾದ
ಅಳಿಯ-ನಾಗಿದ್ದು
ಅಳಿಯ-ನಾಗಿ-ರ-ಬಹುದು
ಅಳಿಯ-ನಾದ
ಅಳಿಯನೂ
ಅಳಿಯ-ನೆಂದು
ಅಳಿಯನೇ
ಅಳಿಯ-ರಾಮ-ರಾಯ
ಅಳಿಯ-ರಾಮ-ರಾಯನ
ಅಳಿಯ-ರಾಮ-ರಾಯನೂ
ಅಳಿ-ಸಂದ್ರ
ಅಳಿ-ಸಂದ್ರ-ಶಾ-ಸನ-ದಲ್ಲಿ
ಅಳಿಸಿ
ಅಳಿಸಿ-ಹೋಗಿ
ಅಳಿಸಿ-ಹೋಗಿದೆ
ಅಳಿಸಿ-ಹೋಗಿವೆ
ಅಳೀ-ಸಂದ್ರ
ಅಳೀ-ಸಂದ್ರದ
ಅವಕಾಶ-ವಾ-ಯಿತೆಂದು
ಅವಕಾಶ-ವಿದೆ
ಅವಧಿಗೆ
ಅವಧಿಯ
ಅವಧಿ-ಯಲ್ಲಿ
ಅವನ
ಅವ-ನನ್ನು
ಅವ-ನಾದ
ಅವ-ನಿಂದ
ಅವ-ನಿಗೂ
ಅವ-ನಿಗೆ
ಅವನಿಪನೆನಗಿತ್ತಪ-ನೆಂದ-ವರಿ-ವರ-ವೊಲುಳಿದ
ಅವನು
ಅವನೇ
ಅವನೊಬ್ಬ
ಅವರ
ಅವ-ರದ್ದೇ
ಅವ-ರನ್ನು
ಅವ-ರಲ್ಲಿ
ಅವರ-ವರ
ಅವ-ರಿಂದ
ಅವ-ರಿಂದಲೇ
ಅವ-ರಿ-ಗಾಗಿ
ಅವ-ರಿ-ಗಿಂತ
ಅವ-ರಿಗೆ
ಅವರಿಬ್ಬ-ರಿಗೂ
ಅವರಿಬ್ಬರೂ
ಅವರಿ-ವ-ರಂತೆ
ಅವರೀರ್ವ-ರಲ್ಲಿ
ಅವರು
ಅವರು-ಹೇಳಿದ್ದಾರೆ
ಅವರೂ
ಅವ-ರೆಗೆ-ರೆಯ
ಅವರೆಲ್ಲರ
ಅವರೇ
ಅವ-ರೊಡನೆ
ಅವಲಂಬಿಸಿತ್ತು
ಅವಲಂಬಿಸಿದ್ದಿತು
ಅವ-ಲೋಕನ
ಅವ-ಲೋಕಿ-ಸಿದಾಗ
ಅವಳ
ಅವಶೇಷ-ಗಳಿಂದ
ಅವಶೇಷ-ಗಳಿವೆ
ಅವಶೇಷ-ಗಳು
ಅವಸಾ-ನದ
ಅವಾರ್ಯ-ವೀರ್ಯ-ನಾದ
ಅವು
ಅವು-ಗಳ
ಅವು-ಗಳನ್ನು
ಅವು-ಗಳಲ್ಲಿ
ಅವು-ಗ-ಳಿಗೆ
ಅವು-ಗಳು
ಅವು-ಬಳ-ರಾಜಯ್ಯ-ದೇವ
ಅವೆಲ್ಲಾ
ಅವ್ವೆ-ಯರ-ಕೆರೆ
ಅವ್ವೆ-ಯರಾಣೆ
ಅವ್ವೇರ-ಹಳ್ಳಿ
ಅಶರೀರವಾಣಿ-ಯಾಯಿತು
ಅಶೇಷ
ಅಶೇಷ-ರಾಜ್ಯ-ಭಾರ
ಅಶ್ವ-ಪತಿ
ಅಶ್ವ-ಸೇನೆಯು
ಅಶ್ವಾರೋಹಿ
ಅಷ್ಟಗ್ರಾಮ
ಅಷ್ಟಗ್ರಾಮ-ಗಳ
ಅಷ್ಟಗ್ರಾಮದ
ಅಷ್ಟದಿಕ್ಕು-ರಾಯ
ಅಷ್ಟದಿಕ್ಪಾಲ-ಕರ
ಅಷ್ಟ-ವಿಧಾರ್ಚನೆಗೆ
ಅಷ್ಟಾದಶಪ್ರಧಾನ-ರಲ್ಲಿ
ಅಷ್ಟೇ
ಅಸಮ
ಅಸರಿಸ್ವಯಂಭು
ಅಸರು
ಅಸವಯ್ಯನುಂ
ಅಸಿ-ವರ-ದಲಿ
ಅಸುನೀಗಿ-ದಂತೆ
ಅಸೂಯೆ
ಅಸೆಂಬ್ಲಿ-ಎಂದೂ
ಅಸೋಫ-ನಿದ್ದನು
ಅಸೋಫಿ-ಗಳಾಗಿ
ಅಸೋಫಿಗೆ
ಅಸೋಫ್
ಅಸ್ಕರ್
ಅಸ್ತ-ಮಾನ-ವಾ-ದಾಗ
ಅಸ್ತಿತ್ವ
ಅಸ್ತಿತ್ವಕ್ಕೆ
ಅಸ್ತಿತ್ವ-ದಲ್ಲಿತ್ತು
ಅಸ್ತಿತ್ವ-ದಲ್ಲಿದ್ದ
ಅಸ್ತಿತ್ವ-ದಲ್ಲಿದ್ದವು
ಅಸ್ತಿತ್ವ-ದಲ್ಲಿದ್ದು
ಅಸ್ತಿತ್ವ-ದಲ್ಲಿರುವ
ಅಸ್ತಿತ್ವ-ದಲ್ಲಿವೆ
ಅಸ್ತಿತ್ವ-ವನ್ನು
ಅಸ್ತಿ-ಯನ್ನು
ಅಸ್ಥಿರ-ತೆ-ಯನ್ನು
ಅಸ್ಪಷ್ಟ-ವಾಗಿವೆ
ಅಸ್ಯ
ಅಹುಬಳ
ಅಹುಬಳ-ದೇವ-ರಾಜಯ್ಯ-ದೇವ
ಅಹುಬಳ-ರಾಜಯ್ಯ-ನಿಗೆ
ಅಹೊ-ಬಲ-ದೇವ-ರಾಜಯ್ಯನು
ಅಹೋ-ಬಲ
ಅಹೋ-ಬಲ-ದೇವ
ಅಹೋ-ಬಲ-ದೇವ-ಗಳ
ಅಹೋ-ಬಲ-ದೇವನ
ಅಹೋ-ಬಲ-ದೇವ-ರಾ-ಜಯ್ಯ
ಅಹೋ-ಬಲಯ್ಯನ
ಅಹೋ-ಬಲ-ವಾಡಿ
ಅಹೋಬಳ
ಅಹೋಬಳ-ಪುರ-ವೆಂಬ
ಅಹೋಬಳ-ರಾಜನ
ಆ
ಆಂಗೀ-ರಸ
ಆಂಗ್ಲ
ಆಂಗ್ಲ-ಭಾಷೆ-ಯಲ್ಲಿ
ಆಂಗ್ಲ-ಭಾಷೆ-ಯಲ್ಲೂ
ಆಂಗ್ಲೋ
ಆಂಜನೇಯ
ಆಂಜನೇ-ಯನ
ಆಂಡಾನ್
ಆಂಧ್ರ
ಆಂಧ್ರ-ನಾಡಿ-ನಲ್ಲಿ
ಆಂಧ್ರಪ್ರ-ದೇಶದ
ಆಂಧ್ರ-ರಾಜ-ಮದಗಜ-ಗ-ಳಿಗೆ
ಆಕರ-ಗಳು
ಆಕೆಯ
ಆಕ್ರಮಣ
ಆಕ್ರಮ-ಣಕ್ಕೆ
ಆಕ್ರಮ-ಣ-ಗಳನ್ನು
ಆಕ್ರಮ-ಣದ
ಆಕ್ರಮ-ಣ-ದಲ್ಲಿ
ಆಕ್ರಮ-ಣ-ನಡೆ-ಸಲು
ಆಕ್ರಮ-ಣ-ವನ್ನು
ಆಕ್ರಮ-ಣ-ವಾಗಿ-ರ-ಬಹುದು
ಆಕ್ರಮಿತ-ವಾಗಿ
ಆಕ್ರಮಿಸಿ
ಆಕ್ರಮಿಸಿ-ಕೊಂಡನು
ಆಕ್ರಮಿಸಿ-ಕೊಂಡರು
ಆಕ್ರಮಿಸಿ-ಕೊಂಡು
ಆಕ್ರಮಿ-ಸಿದ
ಆಕ್ರಮಿಸಿ-ದನು
ಆಕ್ರಮಿಸಿ-ದ-ಮೇಲೂ
ಆಕ್ರಮಿ-ಸಿದ್ದು
ಆಗ
ಆಗಂತುಕ
ಆಗ-ತಾನೆ
ಆಗ-ಬೇಕೆಂದು
ಆಗವ-ಹಾಳ
ಆಗಸ್ಟ್
ಆಗಸ್ಟ್ನಿಂದ
ಆಗಾಗ್ಗೆ
ಆಗಿ
ಆಗಿತ್ತು
ಆಗಿತ್ತೆಂದು
ಆಗಿದೆ
ಆಗಿದ್ದ
ಆಗಿದ್ದನು
ಆಗಿದ್ದ-ನೆಂದು
ಆಗಿದ್ದ-ರಿಂದ
ಆಗಿದ್ದರು
ಆಗಿದ್ದರೂ
ಆಗಿದ್ದ-ರೆಂದು
ಆಗಿದ್ದಲ್ಲಿ
ಆಗಿದ್ದಾಗ
ಆಗಿದ್ದಾನೆ
ಆಗಿದ್ದಾನೆಂದು
ಆಗಿದ್ದಿರ-ಬಹುದು
ಆಗಿದ್ದು
ಆಗಿನ
ಆಗಿನ್ನೂ
ಆಗಿ-ರಬಹದು
ಆಗಿ-ರ-ಬಹುದು
ಆಗಿ-ರ-ಬಹು-ದೆಂದು
ಆಗಿ-ರ-ಲಿಲ್ಲ
ಆಗಿ-ರುತ್ತಾನೆ
ಆಗಿ-ರುತ್ತಾ-ನೆಂದು
ಆಗಿ-ರುತ್ತಿದ್ದರು
ಆಗಿ-ರುವ
ಆಗಿ-ರು-ವಂತೆ
ಆಗಿ-ರು-ವು-ದನ್ನು
ಆಗಿ-ರು-ವುದು
ಆಗಿವೆ
ಆಗಿ-ಹೋದ-ರೆಂದೂ
ಆಗುತ್ತದ
ಆಗುತ್ತದೆ
ಆಗು-ವು-ದಿಲ್ಲ
ಆಗ್ನೇಯ
ಆಗ್ನೇಯ-ದಲ್ಲಿ
ಆಚನ-ಹಳ್ಳಿ
ಆಚಮಂಗೆ
ಆಚಮಆಚಮ್ಮ
ಆಚ-ಮನು
ಆಚ-ರಾಜ
ಆಚರಿಸಿ
ಆಚರಿ-ಸಿದ
ಆಚರಿಸಿ-ರ-ಬಹುದು
ಆಚಾಂಬಿಕೆ
ಆಚಾಂಬಿಕೆಗೆ
ಆಚಾಂಬಿ-ಕೆಯ
ಆಚಾರ್ಯ
ಆಚಾರ್ಯರ
ಆಚಿಕಬ್ಬೆ
ಆಚಿಯಕ್ಕನು
ಆಚೆ
ಆಚೆಗೆ
ಆಚೆಯೇ
ಆಜ್ಞಾಪಿ-ಸಿದ-ನೆಂದಿದೆ
ಆಜ್ಞಾಪಿ-ಸಿದ-ನೆಂದು
ಆಜ್ಞಾಪಿಸಿ-ದ-ವನು
ಆಜ್ಞಾಪಿ-ಸುವ
ಆಜ್ಞೆ
ಆಜ್ಞೆಯ
ಆಜ್ಞೆ-ಯಂತೆ
ಆಠವ-ಣೆಯ
ಆಡಳಿಗಾ-ರನನ್ನಾಗಿ-ರಾಜ್ಯ-ಪಾಲ
ಆಡಳಿತ
ಆಡಳಿ-ತಕ್ಕೆ
ಆಡಳಿತ-ಗಾರ-ರನ್ನು
ಆಡಳಿತ-ಗಾ-ರರು
ಆಡಳಿ-ತದ
ಆಡಳಿತ-ದಲ್ಲಿ
ಆಡಳಿತ-ದಲ್ಲಿದ್ದ
ಆಡಳಿತ-ದಲ್ಲಿದ್ದ-ರೆಂದು
ಆಡಳಿತ-ದಲ್ಲೂ
ಆಡಳಿತ-ನ-ವನ್ನು
ಆಡಳಿತ-ವನ್ನು
ಆಡಳಿತ-ವನ್ನೂ
ಆಡಳಿತ-ವರ್ಷ-ದಲ್ಲಿ
ಆಡಳಿತ-ವಿ-ಭಾಗ-ವಾಗಿತ್ತೆಂದು
ಆಡಳಿತ-ವೆಲ್ಲವೂ
ಆಡಳಿತವ್ಯವ-ಹಾರ-ಗಳು
ಆಡಳಿತ-ಸೂತ್ರ-ಗ-ಳನ್ನೂ
ಆಡಳಿತಾಧಿ-ಕಾರಿ
ಆಡಳಿತಾಧಿ-ಕಾರಿ-ಗಳ
ಆಡಳಿತಾಧಿ-ಕಾರಿ-ಗಳೊಡನೆ
ಆಡಳಿತಾವಧಿ-ಯಲ್ಲಿ
ಆಡು
ಆಡುಂಬೊಲ-ವಾದ
ಆಣೆ
ಆತ
ಆತಕೂರು
ಆತಕೂರು-ರಲ್ಲಿ
ಆತನ
ಆತ-ನನ್ನು
ಆತನಿಗಿ-ರ-ಲಿಲ್ಲ
ಆತ-ನಿಗೆ
ಆತನು
ಆತನೇ
ಆತಿಶ್
ಆತಿಶ್ಖಾನ್
ಆತೂರು
ಆತ್ಕೂರು-ಆತಕೂರು
ಆತ್ಮ-ಭಕ್ತಿ-ಯಿಂದ
ಆತ್ಮಹತ್ಯೆ
ಆತ್ಮಾಗ್ರಜ
ಆತ್ಮಾರ್ಪಣೆ
ಆತ್ರೇಯ
ಆತ್ರೇಯ-ಗೋತ್ರದ
ಆತ್ರೇಯಸ
ಆಥವಾ
ಆದ
ಆದ-ಕಾರಣ
ಆದರು
ಆದರೂ
ಆದರೆ
ಆದಾಯ-ದಲ್ಲಿ
ಆದಾಯ-ವನ್ನು
ಆದಾಯ-ವಿ-ರುವ
ಆದಾ-ಯವುಳ್ಳ
ಆದಿಗುಂಜ-ನರ-ಸಿಂಹ-ದೇವ-ರಿಗೆ
ಆದಿಗುಂಜೆಯ
ಆದಿ-ಚುಂಚನ-ಗಿರಿ
ಆದಿ-ಚುಂಚನ-ಗಿರಿ-ಒಂದು
ಆದಿ-ಚುಂಚನ-ಗಿರಿಯ
ಆದಿ-ದೇವನ
ಆದಿ-ಪುರಾಣ-ದಲ್ಲಿ
ಆದಿಯಮ
ಆದಿಯಮ-ನನ್ನು
ಆದಿಯ-ಮನು
ಆದಿಯಮ-ನೋಡಿದೋಟ
ಆದಿಲ್ನು
ಆದಿ-ವರಾಹ-ನಿಗೂ
ಆದಿ-ವಾರ
ಆದಿ-ಸಿಂಗೆಯ
ಆದಿ-ಸಿಂಗೆಯ-ದಣ್ಣಾಯ-ಕರುಮ
ಆದುದ-ರಿಂದ
ಆದುದ-ರಿಂದಲೇ
ಆದು-ರಿಂದ
ಆದೇಶ
ಆದೇಶದ
ಆದೇಶ-ದಂತೆ
ಆದೇಶ-ದಿಂದ
ಆದೇಶ-ವಾಗು-ವುದು
ಆದ್ಯತೆ
ಆಧರಿಸಿ
ಆಧಾರ
ಆಧಾರ-ಗಳನ್ನು
ಆಧಾರ-ಗಳಲ್ಲಿ
ಆಧಾರ-ಗಳಿಂದ
ಆಧಾರ-ಗಳಿಲ್ಲ
ಆಧಾರ-ಗಳು
ಆಧಾರ-ಗಳೂ
ಆಧಾ-ರದ
ಆಧಾರ-ದಿಂದ
ಆಧಾರ-ವಾಗಿ
ಆಧಾರ-ವಾಗಿಟ್ಟು-ಕೊಂಡು
ಆಧಾರವೂ
ಆಧಿಕ್ಯ-ವನ್ನು
ಆಧಿಪತ್ಯ-ದಲ್ಲಿ
ಆಧಿಪತ್ಯ-ವನ್ನು
ಆಧುನಿಕ
ಆನಂತರ
ಆನಂದ
ಆನಂದಾನ್ಪುಳ್ಳೆ
ಆನಂದೂರಿ-ನಲ್ಲಿ
ಆನತರಾಗು-ವಂತೆ
ಆನೆ
ಆನೆ-ಕೆರೆ
ಆನೆ-ಗನಕೆರಿ-ಆನೆ-ಕೆರೆ
ಆನೆ-ಗಳ
ಆನೆ-ಗಳನ್ನು
ಆನೆ-ಗಳು
ಆನೆ-ಗೊಂದಿ-ಯಲ್ಲಿ
ಆನೆ-ಬ-ಸದಿಗೆ
ಆನೆ-ಬ-ಸದಿಯ
ಆನೆ-ಬ-ಸದಿಯು
ಆನೆಯ
ಆನೆಯಂ
ಆನೆಯನು
ಆನೆ-ಯನ್ನು
ಆನೆಯು
ಆನೆ-ಯೊಡನೆ
ಆನೆ-ವಾಳ
ಆನೆ-ಸಲಗ
ಆನೆ-ಸಾ-ಸಲು
ಆನೆ-ಹಾಳು
ಆಪಸ್ತಂಭ
ಆಪಸ್ತಂಭ-ಸೂತ್ರದ
ಆಪ್ತ
ಆಪ್ತರ
ಆಪ್ತ-ಸಹಾಯ-ಕರ
ಆಪ್ತೇಷ್ಟ-ರಲ್ಲಿ
ಆಪ್ತೇಷ್ಟರು
ಆಭರಣ
ಆಭರ-ಣ-ಗಳನ್ನು
ಆಮೇಲೆ
ಆಯಆಯತ
ಆಯ-ಕಟ್ಟಿನ
ಆಯ-ಗಳನ್ನು
ಆಯ-ಗಾ-ರರು
ಆಯ-ತದ
ಆಯ-ವನ್ನು
ಆಯಸ್ಸು
ಆಯಾ
ಆಯಿತು
ಆಯಿತೆಂದು
ಆಯಿದು
ಆಯಿದು-ಮೊತ್ತದ
ಆಯುಧ
ಆಯುಧ-ಗಳನ್ನು
ಆಯುಧ-ಗಳು
ಆಯುರಾರೋಗ್ಯ
ಆಯ್ಕೆ
ಆಯ್ಕೆ-ಮಾಡಿ-ಕೊಳ್ಳುತ್ತಿದ್ದ-ರೆಂದು
ಆಯ್ಕೆ-ಯಾಗುತ್ತಿದ್ದರು
ಆಯ್ಕೆ-ಯಾ-ದ-ವರು
ಆಯ್ಕೆ-ಯಾದ-ವ-ರೆಂದು
ಆಯ್ದು-ಕೊಂಡಿದೆ
ಆರಂಭ
ಆರಂಭ-ಗೊಳ್ಳುತ್ತದೆ
ಆರಂಭದ
ಆರಂಭ-ದಲ್ಲಿ
ಆರಂಭಲ್ಲನು
ಆರಂಭಲ್ಲ-ವನು
ಆರಂಭ-ವಾಗಿದೆ
ಆರಂಭ-ವಾಗಿ-ರುವುದು
ಆರಂಭ-ವಾಗುತ್ತದೆ
ಆರಂಭ-ವಾ-ದುದು
ಆರಂಭ-ವಾಯಿತು
ಆರಂಭ-ವಾ-ಯಿತೆಂದು
ಆರಂಭ-ವಾಯಿ-ತೆಂದೂ
ಆರಂಭ-ವಾಯಿತೆನ್ನ-ಬಹುದು
ಆರಂಭಿ-ಸಿದನು
ಆರಂಭಿ-ಸಿದರು
ಆರಂಭಿಸಿದ್ದನ್ನು
ಆರಣಿ
ಆರಣಿಯ
ಆರಣಿಯು
ಆರಣಿ-ಸಯಸ್ಥಳದ
ಆರ-ನೆಯ
ಆರನೇ
ಆರಾಧಿಸಿ
ಆರಾಧಿ-ಸುತ್ತಿದ್ದ
ಆರಾಧ್ಯ
ಆರಾಧ್ಯ-ದೈವ
ಆರಿದ-ವಾಳಿ-ಕೆಯ
ಆರಿಸಿ-ಕೊಂಡನು
ಆರಿಸಿ-ಕೊಳ್ಳುತ್ತಿದ್ದರು
ಆರು
ಆರುಗ್ರಾಮ-ಗಳನ್ನು
ಆರು-ತಲೆ-ಮಾರು-ಗಳ
ಆರೆಂಟು
ಆರೋಗಣೆ-ಯನ್ನು
ಆರೋಪಿಸ-ಲಾಗಿ-ರು-ವು-ದನ್ನು
ಆರೋಪಿಸಿಲ್ಲ
ಆರೋಪಿಸು-ವುದು
ಆರ್ಎಸ್ಪಂಚಮುಖಿ
ಆರ್ಕಾಟಿನ
ಆರ್ಕಾಟಿನ-ವ-ನಾದ
ಆರ್ಕಿಯಾಲಾಜಿಕಲ್
ಆರ್ಕಿಯೋಲಜಿಕಲ್
ಆರ್ಥರ್
ಆರ್ಥಿಕ
ಆರ್ಯ-ಮಂಡು-ನದ
ಆರ್ಶೇಷ-ಶಾಸ್ತ್ರಿ-ಯ-ವರ
ಆಲಂಬಾಡಿ
ಆಲ-ಗಾವುಂಡ
ಆಲತಿ
ಆಲತ್ತೂ-ರನಿಱಿದು
ಆಲತ್ತೂರಿನ
ಆಲದ-ಹಳ್ಳಿ
ಆಲದ-ಹಳ್ಳಿಯ
ಆಲಪ್ಪ
ಆಲು-ಗೋಡನ್ನು
ಆಲು-ಗೋಡು
ಆಲು-ಗೋಡು-ರಾಜ್ಯ
ಆಲೂರಿನ
ಆಲೂರಿನ-ವ-ರಿಗೂ
ಆಲೂರು
ಆಲೇನ-ಹಳ್ಳಿ
ಆಲ್ಗೋಡು
ಆಳತ್ತಿದ್ದ-ನೆಂದು
ಆಳದೇ
ಆಳಲು
ಆಳ-ವಾಗಿ
ಆಳ-ವಾದ
ಆಳಿ-ಕೊಂಡು
ಆಳಿದ
ಆಳಿ-ದನು
ಆಳಿದ-ನೆಂದು
ಆಳಿದ-ನೆಂದೂ
ಆಳಿ-ದರು
ಆಳಿದ-ರೆಂದೂ
ಆಳು-ಗೋಡೀ
ಆಳು-ಗೋಡು
ಆಳು-ತಿದ್ದನು
ಆಳು-ತಿದ್ದರು
ಆಳು-ತಿದ್ದಾಗ
ಆಳುತ್ತಾ
ಆಳುತ್ತಿದ್ದ
ಆಳುತ್ತಿದ್ದಂತೆ
ಆಳುತ್ತಿದ್ದನು
ಆಳುತ್ತಿದ್ದ-ನೆಂದಿದೆ
ಆಳುತ್ತಿದ್ದ-ನೆಂದು
ಆಳುತ್ತಿದ್ದ-ನೆಂದೂ
ಆಳುತ್ತಿದ್ದ-ನೆಂಬುದು
ಆಳುತ್ತಿದ್ದರು
ಆಳುತ್ತಿದ್ದರೂ
ಆಳುತ್ತಿದ್ದ-ರೆಂದು
ಆಳುತ್ತಿದ್ದ-ರೆಂದೂ
ಆಳುತ್ತಿದ್ದ-ರೆಂಬ
ಆಳುತ್ತಿದ್ದಳು
ಆಳುತ್ತಿದ್ದ-ಳೆಂದು
ಆಳುತ್ತಿದ್ದ-ವನು
ಆಳುತ್ತಿದ್ದ-ವ-ರಿಗೆ
ಆಳುತ್ತಿದ್ದ-ವ-ರೆಂದರೆ
ಆಳುತ್ತಿದ್ದಾಗ
ಆಳುತ್ತಿದ್ದಿರ-ಬಹುದು
ಆಳುತ್ತಿದ್ದು
ಆಳುತ್ತಿದ್ದು-ದನ್ನು
ಆಳುತ್ತಿದ್ದು-ದ-ರಿಂದ
ಆಳುತ್ತಿದ್ದುದು
ಆಳುತ್ತಿ-ರಲು
ಆಳುತ್ತಿರುತ್ತಾನೆ
ಆಳುವ
ಆಳುವ-ಖೇಡ
ಆಳ್ತನ-ವನ್ನು
ಆಳ್ದನ
ಆಳ್ವಕೆ
ಆಳ್ವ-ಖೇಡ
ಆಳ್ವಾರ್
ಆಳ್ವಿಕೆ
ಆಳ್ವಿಕೆಗೆ
ಆಳ್ವಿಕೆಯ
ಆಳ್ವಿಕೆ-ಯನ್ನು
ಆಳ್ವಿಕೆ-ಯನ್ನೇ
ಆಳ್ವಿಕೆ-ಯಲ್ಲಿ
ಆಳ್ವಿಕೆ-ಯಲ್ಲೂ
ಆಳ್ವಿಕೆಯೇ
ಆವರ-ಣ-ದಲ್ಲಿಯೇ
ಆವರ-ಣ-ದಲ್ಲಿರುವ
ಆವೃತ-ವಾದ
ಆಶಾದಾಯಕ
ಆಶ್ಚರ್ಯ-ಕರ
ಆಶ್ಚರ್ಯ-ಕರ-ವಾಗಿದೆ
ಆಶ್ರಯ
ಆಶ್ರಯ-ದಲ್ಲಿ
ಆಶ್ರಯ-ವರ್ತಿ-ಯಾಗಿದ್ದು-ಕೊಂಡು
ಆಶ್ರಯಿ-ಸಿದ-ನೆಂದು
ಆಶ್ರಯಿ-ಸಿದ್ದು
ಆಶ್ರಿ-ತ-ಜನ-ಕಲ್ಪ-ವೃಕ್ಷ
ಆಶ್ರಿ-ತ-ನಾಗಿದ್ದು
ಆಶ್ವಲಾ-ಯನ
ಆಶ್ವಲಾ-ಯನ-ಸೂತ್ರದ
ಆಶ್ವೀಜ
ಆಸಂದಿ
ಆಸಂದಿ-ನಾಡ
ಆಸಂಧಿ-ನಾಡ
ಆಸಂನ್ನ
ಆಸನ್ನ
ಆಸೆ
ಆಸೆ-ಮಾಡುವ
ಆಸೇತು-ಮೇರು-ಪರ್ಯಂತಂ
ಆಸ್ತಿ
ಆಸ್ತಿ-ಯನ್ನು
ಆಸ್ತಿ-ಹಂಚಿಕೆ
ಆಸ್ಥಾನ
ಆಸ್ಥಾನ-ಕವಿ-ಯಾಗಿದ್ದ
ಆಸ್ಥಾನಕ್ಕೆ
ಆಸ್ಥಾನ-ಜಗಜೆಟಿ
ಆಸ್ಥಾನದ
ಆಸ್ಥಾನ-ದಲ್ಲಿ
ಆಸ್ಥಾನ-ದಲ್ಲಿದ್ದ
ಆಸ್ಥಾನ-ದಲ್ಲಿದ್ದು
ಆಸ್ಥಾನ-ವನ್ನು
ಆಸ್ಥಾಯಿಕಾ
ಆಸ್ಪದ
ಆಹಾರ
ಆಹಾರ-ದಾನಕ್ಕಾಗಿ
ಆಹಾರ-ದಾನಕ್ಕೆ
ಆಹಾರ-ಮಂಡ-ಲ-ಭುಕ್ತಿ-ವಿಷಯ-ದೇಶ
ಆಹಾರಾಭಯ
ಆಹಾರಾಭ-ಯನುಂ
ಆಹಾರಾಭಯ-ಭೈಷಜ್ಯ-ಶಾಸ್ತ್ರ-ವಿನೋದನುಂ
ಆಹ್ವಾನಿ-ಸುತ್ತಿದ್ದನು
ಇ
ಇಂಗಲ-ಗುಪ್ಪೆ
ಇಂಗಲ-ಗುಪ್ಪೆಯ
ಇಂಗ್ಲಿಷ್
ಇಂತಹ
ಇಂತಿ-ವರ
ಇಂತೀ
ಇಂಥ
ಇಂದಿಗೂ
ಇಂದಿನ
ಇಂದು
ಇಂದು-ಕೊಟ್ಟು
ಇಂದ್ರ-ನಂತೆ
ಇಂದ್ರ-ನಾಗಿದ್ದಾ-ನೆಂದು
ಇಂದ್ರ-ನಿಗೆ
ಇಂದ್ರನು
ಇಂದ್ರ-ರಾಜನ
ಇಂದ್ರ-ವರ್ಮ-ನೆಂಬ
ಇಂಮಡಿದೇವ
ಇಕ್ಕಿ-ದಂಥಾ
ಇಕ್ಕಿ-ಸಿದೆವು
ಇಕ್ಕೇರಿಯ
ಇಗ್ಗ-ಲೂರು
ಇಚ್ಚಿ-ಸದೇ
ಇಜ್ಜಲ-ಘಟ್ಟ-ವೆಂಬ
ಇಜ್ಜಲನ್ನು
ಇಟಗಿ
ಇಟ್ಟನು
ಇಟ್ಟ-ರೆಂದೂ
ಇಟ್ಟಾಡಿ
ಇಟ್ಟಿಗೆ-ಯಲ್ಲಿ
ಇಟ್ಟಿದ್ದನು
ಇಟ್ಟಿದ್ದಾ-ನೆಂದು
ಇಟ್ಟಿರ-ಬೇಕಾಗುತ್ತಿತ್ತು
ಇಟ್ಟು-ಕೊಂಡರು
ಇಟ್ಟು-ಕೊಂಡರೂ
ಇಟ್ಟು-ಕೊಂಡರೆ
ಇಟ್ಟು-ಕೊಂಡ-ರೆಂದು
ಇಟ್ಟು-ಕೊಂಡಿದ್ದ-ನೆಂದೂ
ಇಟ್ಟು-ಕೊಂಡಿದ್ದರು
ಇಟ್ಟು-ಕೊಂಡಿದ್ದ-ರೆಂದು
ಇಟ್ಟು-ಕೊಂಡು
ಇಟ್ಟು-ಕೊಳ್ಳ-ಬಹುದು
ಇಟ್ಟು-ಕೊಳ್ಳುತ್ತಿದ್ದರು
ಇಟ್ಟು-ಕೊಳ್ಳುತ್ತಿದ್ದ-ರೆಂದು
ಇಡಗೂರು
ಇಡ-ಲಾಗುತ್ತಿತ್ತು
ಇಡೀ
ಇಡುಗೂರ
ಇಡುಗೂರು
ಇಡುತುರೈ-ನಾಟ್ಟು
ಇಡುತ್ತಾನೆ
ಇಡುತ್ತಿದ್ದುದು
ಇಡುದುರೈ
ಇಡುವ
ಇಡೆಯ-ನಾಡು
ಇಡೈತುರೈ-ನಾಡನ್ನು
ಇಡೈತುರೈ-ನಾಡು-ಕಾವೇರಿ
ಇಡೈಮು-ನೂರು-ಇಡೈಕುನ್ದ-ನಾಡು
ಇತರ
ಇತರ-ರನ್ನು
ಇತರೆ
ಇತರೆ-ಉಳಿದ-ವರು
ಇತಿ
ಇತಿ-ಹಾಸ
ಇತಿ-ಹಾಸ-ಕಾರರ
ಇತಿ-ಹಾಸ-ಕಾರರು
ಇತಿ-ಹಾಸಕ್ಕಿಂತ
ಇತಿ-ಹಾಸಕ್ಕೆ
ಇತಿ-ಹಾಸ-ಗಳನ್ನು
ಇತಿ-ಹಾಸದ
ಇತಿ-ಹಾಸ-ದಲ್ಲಿ
ಇತಿ-ಹಾಸ-ದಿಂದ
ಇತಿ-ಹಾಸಪ್ರ-ಸಿದ್ಧ
ಇತಿ-ಹಾಸ-ವನ್ನು
ಇತಿ-ಹಾಸ-ವಿದ್ವಾಂಸರು
ಇತಿ-ಹಾಸವು
ಇತ್ತ
ಇತ್ತಣ
ಇತ್ತೀಚಿನ-ವರೆಗೂ
ಇತ್ತೀಚೆಗೆ
ಇತ್ತು
ಇತ್ತೆಂದು
ಇತ್ತೆಂಬುದು
ಇತ್ತೇ
ಇತ್ಯಾದಿ
ಇತ್ಯಾದಿ-ಯಾಗಿ
ಇದಕ್ಕಾಗಿ
ಇದಕ್ಕೂ
ಇದಕ್ಕೆ
ಇದನ್ನು
ಇದನ್ನೂ
ಇದನ್ನೇ
ಇದರ
ಇದ-ರಲ್ಲಿ
ಇದ-ರಲ್ಲಿದೆ
ಇದ-ರಲ್ಲಿದ್ದು
ಇದ-ರಿಂದ
ಇದ-ರಿಂದಾಗಿ
ಇದಾಗಿದೆ
ಇದಾಗಿವೆ
ಇದಾದ
ಇದು
ಇದು-ಳೆ-ಯನ್ನು
ಇದು-ವರೆ-ಗಿನ
ಇದು-ವ-ರೆಗೆ
ಇದೂ
ಇದೆ
ಇದೆಯೇ
ಇದೇ
ಇದೊಂದು
ಇದೊಂದೇ
ಇದ್ದ
ಇದ್ದಂತಹ
ಇದ್ದಂತೆ
ಇದ್ದಂತೆಯೂ
ಇದ್ದಕ್ಕಿದ್ದ-ಹಾಗೆ
ಇದ್ದ-ನಂತೆ
ಇದ್ದನು
ಇದ್ದ-ನೆಂದು
ಇದ್ದ-ನೆಂದೂ
ಇದ್ದ-ನೆನ್ನು-ವು-ದರ
ಇದ್ದರು
ಇದ್ದರೂ
ಇದ್ದರೆ
ಇದ್ದ-ರೆಂದು
ಇದ್ದ-ರೆಂದೂ
ಇದ್ದಳು
ಇದ್ದವು
ಇದ್ದ-ವೆಂದು
ಇದ್ದ-ಹಾಗೆ
ಇದ್ದಾಗ
ಇದ್ದಾರೆ
ಇದ್ದಿತು
ಇದ್ದಿತೆಂದು
ಇದ್ದಿ-ತೆಂದೂ
ಇದ್ದಿತೆಂಬ
ಇದ್ದಿತೆಂಬುದು
ಇದ್ದಿತೇ
ಇದ್ದಿರ-ಬಹು-ದಾದ
ಇದ್ದಿರ-ಬಹುದು
ಇದ್ದಿರ-ಬೇಕು
ಇದ್ದು
ಇದ್ದು-ಕೊಂಡು
ಇದ್ದು-ದನ್ನು
ಇದ್ದು-ದ-ರಿಂದ
ಇದ್ದುದು
ಇದ್ದುದೇ
ಇನನ
ಇನಾಮಾಗಿ
ಇನಿಗೆ
ಇನ್ದರ
ಇನ್ನಿಬ್ಬರು
ಇನ್ನು
ಇನ್ನೂ
ಇನ್ನೂ-ರ-ಹನ್ನೊಂದ
ಇನ್ನೂರು
ಇನ್ನೂ-ರೆಂಬತ್ತು
ಇನ್ನೊಂದು
ಇನ್ನೊಬ್ಬ
ಇನ್ನೊಬ್ಬ-ರಿಗೆ
ಇನ್ನೊಬ್ಬಳು
ಇಪ್ಪತ್ತ-ನಾಲ್ಕು
ಇಪ್ಪತ್ತು
ಇಪ್ಪತ್ತೈದು
ಇಪ್ಪತ್ತೊಂದ-ನೆಯ
ಇಬ್ಬರ
ಇಬ್ಬ-ರನ್ನೂ
ಇಬ್ಬ-ರಿಗೂ
ಇಬ್ಬ-ರಿಗೆ
ಇಬ್ಬರು
ಇಬ್ಬರೂ
ಇಭಾಟು
ಇಮ್ಮಡಿ
ಇಮ್ಮಡಿದೇವ-ನ-ದೇ-ರಾಯನ
ಇಮ್ಮಡಿದೇವ-ರಾಯ-ನನ್ನು
ಇಮ್ಮಡಿದೇವ-ರಾಯ-ನೆಂಬ
ಇಮ್ಮಡಿ-ಬಲ್ಲಾಳನ
ಇಮ್ಮಡಿ-ಬಲ್ಲಾಳ-ನಿಂದ
ಇಮ್ಮಡಿ-ಬೀರ
ಇಮ್ಮಡಿ-ಬೂತುಗನು
ಇಮ್ಮಡಿ-ಯಾಯಿತು
ಇಮ್ಮಡಿ-ರಾಯ
ಇಮ್ಮಡಿ-ರಾವುತ್ತ-ರಾಯ
ಇರಣ್ಡು-ಕರೈ
ಇರ-ಬಹುದು
ಇರ-ಬಹು-ದೆಂದು
ಇರ-ಬೇಕಾಗಿತ್ತೆಂಬು-ದನ್ನು
ಇರ-ಲಿಲ್ಲ
ಇರ-ಲಿಲ್ಲ-ವೆಂದು
ಇರ-ಲಿಲ್ಲ-ವೆಂದೂ
ಇರ-ಲಿಲ್ಲ-ವೆಂಬುದು
ಇರಲು
ಇರಲೂ-ಬಹುದು
ಇರಲೇ
ಇರಾಮನ್
ಇರಿದಂ
ಇರಿ-ದನು
ಇರಿ-ದನು-ಯುದ್ಧ-ಮಾಡಿದನು
ಇರಿದ-ನೆಂದು
ಇರಿದು
ಇರಿವಬೆಡಂಗ
ಇರಿ-ಸಲಾ-ಗಿತ್ತು
ಇರಿಸಿ-ಕೊಂಡಿದ್ದ-ನೆಂದು
ಇರಿ-ಸಿದ್ದ-ನೆಂದು
ಇರಿ-ಸಿದ್ದರು
ಇರುಂಗೋ-ಳನ
ಇರುಂಗೋ-ಳನ-ಕೋಟೆ
ಇರುಂಗೋ-ಳನೂ
ಇರು-ಗಂಗಣ್ಣ
ಇರುಗಪ್ಪನು
ಇರು-ತಿದ್ದ-ನೆಂದು
ಇರುತ್ತಿತತ್ತೆಂದು
ಇರುತ್ತಿತ್ತು
ಇರುತ್ತಿತ್ತೆಂದು
ಇರುತ್ತಿದ್ದ
ಇರುತ್ತಿದ್ದನು
ಇರುತ್ತಿದ್ದ-ನೆಂದು
ಇರುತ್ತಿದ್ದರು
ಇರುತ್ತಿದ್ದ-ರೆಂದು
ಇರುತ್ತಿದ್ದುದು
ಇರುಮುಡಿ-ಚೋಳ
ಇರುವ
ಇರು-ವಂತೆ
ಇರು-ವು-ದನ್ನು
ಇರುವು-ದನ್ನೂ
ಇರುವು-ದ-ರಿಂದ
ಇರು-ವು-ದಿಲ್ಲ
ಇರು-ವುದು
ಇರ್ರಾಜೇಂದ್ರ
ಇಲಾಖೆ
ಇಲಾಖೆ-ಗಳ
ಇಲಾಖೆ-ಗಳನ್ನು
ಇಲಾಖೆ-ಗ-ಳಿಗೆ
ಇಲಾಖೆಗೆ
ಇಲಾಖೆಯ
ಇಲಾಖೆ-ಯನ್ನು
ಇಲಾಖೆಯು
ಇಲ್ಲ
ಇಲ್ಲ-ದಿ-ರಲು
ಇಲ್ಲ-ದಿ-ರುವ
ಇಲ್ಲ-ದಿ-ರುವಂಶ
ಇಲ್ಲ-ದಿ-ರು-ವುದು
ಇಲ್ಲ-ದಿಲ್ಲ
ಇಲ್ಲದೇ
ಇಲ್ಲವೇ
ಇಲ್ಲಾದೆರೆ
ಇಲ್ಲಿ
ಇಲ್ಲಿಂದ
ಇಲ್ಲಿಗೆ
ಇಲ್ಲಿದ್ದ
ಇಲ್ಲಿದ್ದ-ನೆಂದು
ಇಲ್ಲಿನ
ಇಲ್ಲಿಯೂ
ಇಲ್ಲಿಯೇ
ಇಲ್ಲಿ-ರುವ
ಇಲ್ಲಿಲ್ಲ
ಇಲ್ಲೇ
ಇಲ್ಲೊಂದು
ಇಳಿ-ಯಿತೆಂದು
ಇಳೆಯ
ಇಳೈ-ಯಾಳ್ವಾನ್
ಇವನ
ಇವ-ನದೇ
ಇವನ-ನಿಗೆ
ಇವ-ನನ್ನು
ಇವ-ನಿಂದಲೇ
ಇವ-ನಿಗೂ
ಇವ-ನಿಗೆ
ಇವ-ನಿಗೇ
ಇವನಿ-ರ-ಬೇಕೆಂದು
ಇವನು
ಇವನು-ಮರಿಯಾನೆ
ಇವನೂ
ಇವನೇ
ಇವರ
ಇವರ-ಗಳು
ಇವ-ರನ್ನು
ಇವ-ರನ್ನೂ
ಇವರನ್ನೆಲಾ
ಇವ-ರನ್ನೇ
ಇವ-ರಲ್ಲಿ
ಇವ-ರಲ್ಲೂ
ಇವ-ರಲ್ಲೇ
ಇವರ-ವನ್ನು
ಇವರಾರೂ
ಇವ-ರಿಂದ
ಇವರಿ-ಗಿಲ್ಲ-ದಿ-ರು-ವುದು
ಇವ-ರಿಗೂ
ಇವ-ರಿಗೆ
ಇವ-ರಿಗೆಲ್ಲಾ
ಇವರಿಗೇ
ಇವರಿಬ್ಬರ
ಇವರಿಬ್ಬ-ರಲ್ಲಿ
ಇವರಿಬ್ಬ-ರಿಗೆ
ಇವರಿಬ್ಬರು
ಇವರಿಬ್ಬರೂ
ಇವರು
ಇವರು-ಗಳ
ಇವರು-ಗಳನ್ನು
ಇವರು-ಗ-ಳಿಗೆ
ಇವರು-ಗಳು
ಇವರು-ಗಳೂ
ಇವರೂ
ಇವ-ರೆಲ್ಲರೂ
ಇವರೆಲ್ಲಾ
ಇವರೇ
ಇವ-ರೊಡ-ಗೂಡಿ
ಇವರೊಳಗಾದ
ಇವಳ
ಇವಳು
ಇವು
ಇವು-ಗಳ
ಇವು-ಗ-ಳನೂ
ಇವು-ಗಳನ್ನು
ಇವು-ಗಳಲ್ಲಿ
ಇವು-ಗಳಿಗೂ
ಇವು-ಗ-ಳಿಗೆ
ಇವು-ಗಳು
ಇವು-ಗಳೆಲ್ಲ-ವನ್ನೂ
ಇವು-ಗಳೆಲ್ಲಾ
ಇವು-ಗಳೇ
ಇವೆ
ಇವೆ-ರಡನ್ನೂ
ಇವೆ-ರಡು
ಇವೆ-ರಡೂ
ಇವೆಲ್ಲ-ವನ್ನೂ
ಇವೆಲ್ಲವೂ
ಇವೆಲ್ಲಾ
ಇವೇ
ಇಶಾಮುದ್ರ
ಇಷ್ಟಲ್ಲದೆ
ಇಷ್ಟೂ
ಇಷ್ಟೊಂದು
ಇಸವಿ
ಇಸವಿಯ
ಇಸ್ಲಾಂನ
ಈ
ಈಕೆ
ಈಕೆಯು
ಈಗ
ಈಗಲೂ
ಈಗಾಗಲೇ
ಈಗಿನ
ಈಚೆಗೆ
ಈಡೇ-ರಿದು-ದ-ರಿಂದ
ಈತ
ಈತನ
ಈತ-ನನ್ನು
ಈತ-ನಿಗೆ
ಈತನು
ಈತನೂ
ಈತನೇ
ಈತನೇ-ನಾನ-ದರೂ
ಈರೀತಿ
ಈರೋಡು
ಈವ-ರೆಗೆ
ಈಶಾನ್ಯ
ಈಶಾನ್ಯದ
ಈಶಾನ್ಯ-ದಲ್ಲಿ
ಈಶ್ವರ
ಈಶ್ವರ-ದೇವ
ಈಶ್ವರ-ನನ್ನು
ಈಶ್ವರ-ಭಕ್ತ-ನಾ-ದರೂ
ಈಶ್ವರಯ್ಯಈ-ಸರಯ್ಯ
ಈಶ್ವರಯ್ಯನ
ಈಶ್ವರಯ್ಯನೂ
ಈಶ್ವರಾಂಕ
ಈಸರ-ಗಂಡನ
ಈಸರ-ಗಂಡ-ನಿಗೆ
ಈಸರ-ಗಂಡನು
ಈಸರಯ್ಯ-ಈಶ್ವರಯ್ಯ
ಈಸರಯ್ಯನ
ಈಸರಯ್ಯ-ನೆಂಬ
ಈಸರಯ್ಯನೇ-ಈಶ್ವರಯ್ಯ
ಉಂಟಾಗಲು
ಉಂಟಾಗಿದೆ
ಉಂಟಾಗಿದ್ದ
ಉಂಟಾ-ಗಿದ್ದು
ಉಂಟಾಗುತ್ತದೆ
ಉಂಟಾದ
ಉಂಟಾಯಿತು
ಉಂಟಾ-ಯಿತೆಂದು
ಉಂಟು
ಉಂಟು-ಮಾಡುತ್ತವೆ
ಉಂಡಿಗನ-ಹಾಳು
ಉಂಡಿಗೆ-ಯಾಗಿ
ಉಂಬಳಿ
ಉಂಬಳಿ-ಗ-ಳನ್ನೂ
ಉಂಬಳಿ-ಯಾಗಿ
ಉಂಮರ-ಹಳ್ಳಿ
ಉಂಮರ-ಹಳ್ಳಿ-ಉಮ್ಮಡ-ಹಳ್ಳಿ
ಉಕ್ತ-ನಾಗಿದ್ದು
ಉಕ್ತ-ನಾಗಿ-ರುವ
ಉಕ್ತನಾದ
ಉಕ್ತ-ರಾಗಿ-ರುವ
ಉಕ್ತ-ವಾಗಿದೆ
ಉಕ್ತ-ವಾಗಿ-ರುವ
ಉಕ್ತ-ವಾದ
ಉಗತ್ತಿ
ಉಗಮದ
ಉಗಮ-ವಾಗಿ
ಉಚ್ಚಂಗಿ
ಉಚ್ಚಂಗಿ-ಗಳನ್ನು
ಉಚ್ಚರಿ-ಸುತ್ತಿ-ದರು
ಉಚ್ಛಂಗಿ
ಉಚ್ಛಂಗಿಯೋ
ಉಚ್ಛರಿ-ಸುತ್ತಾರೆ
ಉಚ್ಛಾರಣಾ
ಉಡಿಯಗಾಲ
ಉಡುಗೊರೆ-ಗಳನ್ನು
ಉಡುಪು
ಉಡುವ
ಉಡುವಂಕ-ನಾಡ
ಉಡೈಯಾರ್
ಉತ್ತಮ
ಉತ್ತಮ-ಚೋಳ
ಉತ್ತರ
ಉತ್ತರ-ಕರ್ನಾಟ-ಕ-ದ-ವರೋ
ಉತ್ತರ-ಕರ್ನಾಟದ
ಉತ್ತರಕ್ಕಿ-ರುವ
ಉತ್ತರಕ್ಕೂ
ಉತ್ತರಕ್ಕೆ
ಉತ್ತ-ರದ
ಉತ್ತರ-ದಲ್ಲಿ
ಉತ್ತರ-ದಿಂದ
ಉತ್ತರ-ಭಾಗ-ಗಳನ್ನು
ಉತ್ತರ-ಭಾಗ-ಗಳು
ಉತ್ತರ-ಭಾಗ-ದಲ್ಲಿ
ಉತ್ತರ-ಭಾಗ-ದಲ್ಲಿಯೂ
ಉತ್ತರ-ಭಾಗ-ವನ್ನು
ಉತ್ತರಾಧಿ-ಕಾರಕ್ಕೆ
ಉತ್ತರಾಧಿ-ಕಾರತ್ವವು
ಉತ್ತರಾಧಿ-ಕಾರಿ
ಉತ್ತರಾಧಿ-ಕಾರಿಯ
ಉತ್ತರಾಧಿ-ಕಾರಿ-ಯಾದ
ಉತ್ತ-ರಾರ್ಧ-ದಲ್ಲಿ
ಉತ್ತ-ರಾರ್ಧ-ದಲ್ಲಿಯೇ
ಉತ್ತುಂಗ
ಉತ್ತುಮ
ಉತ್ಪನ್ನದ
ಉತ್ಪ್ರೇಕ್ಷೆ-ಯಿಂದ
ಉತ್ಸವ
ಉತ್ಸವ-ಮಂಟಪ
ಉತ್ಸಾಹ-ದಿಂದ
ಉತ್ಸಾಹಿಯೂ
ಉದಯ-ಗಿರಿ
ಉದಯ-ಗಿರಿಯ
ಉದಯ-ಗಿರಿ-ಯಲ್ಲಿ
ಉದಯ-ಮಯ್ಯ
ಉದಯ-ವಾ-ಯಿತೆಂದು
ಉದ-ಯಾದಿತ್ಯ
ಉದ-ಯಾದಿತ್ಯರು
ಉದಾರ
ಉದಾರ-ವಾಗಿ
ಉದಾರ-ವಾರಿ-ನಿಧಿ
ಉದಾರಿಯೂ
ಉದಾ-ಹಣೆ-ಗಳಿವೆ
ಉದಾ-ಹರ-ಣೆ-ಗಳೂ
ಉದಾ-ಹರ-ಣೆಗೆ
ಉದಾ-ಹರ-ಣೆ-ಯನ್ನು
ಉದಾ-ಹರ-ಣೆ-ಯನ್ನೂ
ಉದಾ-ಹರ-ಣೆಯೂ
ಉದಿತೋದಿತ
ಉದಿ-ಸಿದ
ಉದ್ಗರಿಸು-ವಂತೆ
ಉದ್ಗಾರವೆತ್ತಿದೆ
ಉದ್ಘಾಟನೆಗೆ
ಉದ್ದಂಡ
ಉದ್ದಕ್ಕೂ
ಉದ್ದೇಶ
ಉದ್ದೇಶ-ದಿಂದ
ಉದ್ದೇಶವೂ
ಉದ್ದೇಶಿ-ಸಿದ್ದ
ಉದ್ಭವ
ಉದ್ಭವ-ನರ-ಸಿಂಹ-ಪುರ-ವೆಂಬ
ಉದ್ಭವ-ವಿಶ್ವ-ನಾಥ-ಪುರ-ಬಾಳಗಂಚಿ
ಉದ್ಭವ-ಸರ್ವಜ್ಞ
ಉದ್ಯಮ-ಗ-ಳಿಗೆ
ಉದ್ಯೋಗ-ಮಲ್ಲ-ನೆನಿಸಿ-ದನು
ಉಧಾರಣ
ಉನ್ನತ
ಉನ್ನತ-ದರ್ಜೆಯ
ಉಪ
ಉಪ-ಕಾರದ
ಉಪಕ್ರಮಿಸಿರ-ಬಹು-ದೆಂದು
ಉಪಗ್ರಾಮ
ಉಪಗ್ರಾಮ-ಗಳ
ಉಪಗ್ರಾಮ-ಗಳನ್ನು
ಉಪಗ್ರಾಮ-ಗಳಾಗಿ
ಉಪಗ್ರಾಮ-ಗಳಾಗಿದ್ದವು
ಉಪಗ್ರಾಮ-ಗಳಾಗಿದ್ದ-ವೆಂದು
ಉಪಗ್ರಾಮ-ಗ-ಳಾದ
ಉಪಗ್ರಾಮ-ವನ್ನು
ಉಪಟಳ
ಉಪ-ತಾಲ್ಲೂ-ಕನ್ನು
ಉಪದೇಶ
ಉಪದ್ರವ-ದಿಂದ
ಉಪ-ನದಿ-ಗಳಲ್ಲಿ
ಉಪ-ನದಿ-ಗ-ಳಾದ
ಉಪನ-ಯನ
ಉಪ-ನಾಡು-ಗಳಾಗಿದ್ದ-ವೆಂದು
ಉಪಪಂಗಡ-ವಿದ್ದು
ಉಪ-ಭೋಗಕ್ಕಾಗಿ
ಉಪಯೋಗಕ್ಕಾಗಿ
ಉಪಯೋಗ-ವಾಗಿದೆ
ಉಪಯೋಗಿಸಿ-ಕೊಂಡು
ಉಪಯೋಗಿ-ಸಿದ್ದಾರೆ
ಉಪಯೋಗಿ-ಸುವ
ಉಪಯೋಗಿ-ಸುವ-ವರು
ಉಪ-ವಿ-ಭಾಗಕ್ಕೆ
ಉಪ-ವಿ-ಭಾಗ-ಗಳನ್ನಾಗಿ
ಉಪ-ವಿ-ಭಾಗ-ಗಳಿದ್ದವು
ಉಪ-ವಿ-ಭಾಗ-ಗಳಿದ್ದು
ಉಪ-ವಿ-ಭಾಗ-ದಲ್ಲಿ
ಉಪವಿ-ಭಾಗ-ವಿದ್ದು
ಉಪಶಾಂತ-ವಾಯಿತು
ಉಪಸ್ಥಿತಿ
ಉಬ್ಬು
ಉಬ್ಬು-ಶಿಲ್ಪ-ಗಳಿವೆ
ಉಬ್ಬು-ಶಿಲ್ಪ-ವಿದೆ
ಉಭಯ
ಉಭಯತ್ರರೂ
ಉಭಯ-ದೇಸಿ
ಉಭಯ-ನಾಚ್ಚಿಯಾ-ರರು-ಗ-ಳಿಗೆ
ಉಭಯ-ಬಲ-ಸುಭಟ
ಉಭಯ-ರಾಯ
ಉಭಯ-ವೇದಾಂತಾಚಾರ್ಯ
ಉಭಯಾನ್ವಯ
ಉಮೆ-ಯಕ್ಕ-ನೆಂದೂ
ಉಮ್ಮತೂರ
ಉಮ್ಮತ್ತೂರ
ಉಮ್ಮತ್ತೂ-ರನ್ನು
ಉಮ್ಮತ್ತೂ-ರಿಗೆ
ಉಮ್ಮತ್ತೂರಿನ
ಉಮ್ಮತ್ತೂರಿ-ನಿಂದ
ಉಮ್ಮತ್ತೂರಿ-ನಿಂದಲೂ
ಉಮ್ಮತ್ತೂರು
ಉಮ್ಮತ್ತೂರು-ಗಳನ್ನು
ಉಯ್ಯ-ಕೊಂಡ-ಪಿಳ್ಳೆ
ಉರ-ದಿದಿರಾನ್ತ
ಉರವಣೆ-ಯನ್ನು
ಉರವ-ಣೆ-ಯಿಂದ
ಉರಿ-ಯುವ
ಉರೋದ್ಗು-ಕರತೆ
ಉರ್ಕಣೆ
ಉರ್ದುಶಾಸ-ದಿಂದ
ಉಲ-ಗಾ-ಮುಂಡನ
ಉಲ್
ಉಲ್ಲೆಖಿ-ಸಿದ
ಉಲ್ಲೆಖಿ-ಸುತ್ತದೆ
ಉಲ್ಲೇಖ
ಉಲ್ಲೇಖ-ಗಳನ್ನು
ಉಲ್ಲೇಖ-ಗಳಲ್ಲಿ
ಉಲ್ಲೇಖ-ಗಳಿಂದ
ಉಲ್ಲೇಖ-ಗಳು
ಉಲ್ಲೇಖ-ಗೊಳ್ಳುವ
ಉಲ್ಲೇಖದೆ
ಉಲ್ಲೇಖ-ವನ್ನು
ಉಲ್ಲೇಖ-ವಾಗಿದೆ
ಉಲ್ಲೇಖ-ವಾಗಿ-ರುತ್ತಾರೆ
ಉಲ್ಲೇಖ-ವಾಗಿ-ರು-ವು-ದನ್ನು
ಉಲ್ಲೇಖ-ವಾಗಿವೆ
ಉಲ್ಲೇಖ-ವಿದೆ
ಉಲ್ಲೇಖ-ವಿದೆಯೇ
ಉಲ್ಲೇಖ-ವಿದ್ದು
ಉಲ್ಲೇಖ-ವಿರ-ಬಹು-ದೆಂದು
ಉಲ್ಲೇಖ-ವಿ-ರುವ
ಉಲ್ಲೇಖ-ವಿಲ್ಲ
ಉಲ್ಲೇಖ-ವಿಲ್ಲದ
ಉಲ್ಲೇಖವು
ಉಲ್ಲೇಖವೂ
ಉಲ್ಲೇಖವೇ
ಉಲ್ಲೇಖಿತ
ಉಲ್ಲೇಖಿತ-ನಾಗಿದ್ದಾನೆ
ಉಲ್ಲೇಖಿತ-ನಾಗಿ-ರುವ
ಉಲ್ಲೇಖಿತ-ನಾದ
ಉಲ್ಲೇಖಿತ-ವಾಗಿದೆ
ಉಲ್ಲೇಖಿತ-ವಾದ
ಉಲ್ಲೇಖಿ-ಸದೇ
ಉಲ್ಲೇಖಿಸ-ಬಹುದು
ಉಲ್ಲೇಖಿ-ಸ-ಲಾಗಿದೆ
ಉಲ್ಲೇಖಿಸ-ಲಾ-ಗಿದ್ದು
ಉಲ್ಲೇಖಿಸಿ
ಉಲ್ಲೇಖಿಸಿದ
ಉಲ್ಲೇಖಿಸಿದೆ
ಉಲ್ಲೇಖಿಸಿದ್ದರೂ
ಉಲ್ಲೇಖಿಸಿದ್ದಾರೆ
ಉಲ್ಲೇಖಿಸಿದ್ದು
ಉಲ್ಲೇಖಿಸಿ-ರ-ಬಹು-ದೆಂದು
ಉಲ್ಲೇಖಿಸಿ-ರುವ
ಉಲ್ಲೇಖಿಸಿಲ್ಲ
ಉಲ್ಲೇಖಿಸಿವೆ
ಉಲ್ಲೇಖಿ-ಸುತ್ತದೆ
ಉಲ್ಲೇಖಿ-ಸುತ್ತವೆ
ಉಲ್ಲೇಖಿ-ಸುತ್ತಾ
ಉಲ್ಲೇಖಿ-ಸುತ್ತಾರೆ
ಉಲ್ಲೇಖಿ-ಸುವ
ಉಳಿದ
ಉಳಿ-ದಂತೆ
ಉಳಿ-ದನು
ಉಳಿದ-ವರ
ಉಳಿದ-ವ-ರಲ್ಲಿ
ಉಳಿ-ದವು
ಉಳಿ-ದಿತ್ತು
ಉಳಿ-ದಿದೆ
ಉಳಿ-ದಿ-ರುವ
ಉಳಿದಿವೆ
ಉಳಿ-ದು-ದನ್ನು
ಉಳಿದೆಲ್ಲವೂ
ಉಳಿಯು-ವಂತೆ
ಉಳಿ-ಸಲು
ಉಳಿಸಿ-ಕೊಂಡನು
ಉಳಿಸಿ-ಕೊಂಡ-ನೆಂದು
ಉಳಿಸಿ-ಕೊಂಡಿತ್ತು
ಉಳಿಸಿ-ಕೊಂಡಿವೆ
ಉಳಿಸಿ-ಕೊಂಡು
ಉಳಿಸಿ-ಕೊಳ್ಳ-ಲಾಯಿತು
ಉಳುಮೆ
ಉಳುವ
ಉಳು-ವರ್ತಿ
ಉಳುವ-ವ-ರಿಗೆ
ಊಂಚ-ಹಳ್ಳಿ
ಊರ
ಊರ-ಗಾವುಂಡ-ನಿ-ರುತ್ತಿದ್ದನು
ಊರ-ಗಾವುಂಡನೇ
ಊರ-ಗಾವುಂಡ-ರನ್ನು
ಊರನ್ನಾಗಿ
ಊರನ್ನು
ಊರನ್ನೇ
ಊರಳಿ-ವಿನ
ಊರಳಿ-ವಿ-ನಲ್ಲಿ
ಊರಳಿ-ವಿಲ್ಲಿ
ಊರಳಿವು
ಊರವ-ರಿಂದ
ಊರ-ಸೇನ-ಬೋವ
ಊರಾ-ಗಿತ್ತು
ಊರಾಗಿದೆ
ಊರಾಗಿದ್ದ-ರಿಂದ
ಊರಾಗಿದ್ದು
ಊರಾಗಿ-ರ-ಬಹುದು
ಊರಾದ
ಊರಿಗೆ
ಊರಿದೆ
ಊರಿದ್ದು
ಊರಿನ
ಊರಿ-ನಲ್ಲಿ
ಊರಿನಲ್ಲಿ-ರುವ
ಊರಿನ-ವ-ನಾದ
ಊರಿನ-ವರು
ಊರು
ಊರು-ಗಳ
ಊರು-ಗಳನ್ನು
ಊರು-ಗಳಲ್ಲಿ
ಊರು-ಗಳಾಗಿವೆ
ಊರು-ಗ-ಳಾದ
ಊರು-ಗ-ಳಿಗೆ
ಊರು-ಗಳಿವೆ
ಊರು-ಗಳು
ಊರು-ಗಳೂ
ಊರು-ಗಳೆಲ್ಲಾ
ಊರೂ
ಊರೆ
ಊರೇ
ಊರೊಡ-ಯರೆಂದ
ಊರೊಡೆ-ಯರು
ಊರೊಳಗಿನ
ಊಳಿಗಕ್ಕೆ
ಊಳಿಗದ
ಊಳಿಗದ-ವರು
ಊಳಿಗ-ವನ್ನು
ಊಹಿಸ-ಬಹುದ
ಊಹಿಸ-ಬಹುದು
ಊಹಿಸ-ಬಹು-ದು-ಜಿನಗೃಹಮಂ
ಊಹಿಸಬುದು
ಊಹಿ-ಸ-ಲಾಗಿದೆ
ಊಹಿ-ಸಲು
ಊಹಿಸಹುದು
ಊಹಿಸಿ
ಊಹಿಸಿದ್ದಾರೆ
ಊಹಿಸು-ವುದು
ಊಹೆ
ಊಹೆ-ಗಿಂತ
ಊಹೆ-ಯನ್ನು
ಊಹೆ-ಯಾಗಿದೆ
ಋಕ್ಶಾಖೆಗೆ
ಋಣಕ್ಕೆ
ಋಷಿ-ಗಳ
ಋಷಿ-ಗಳು
ಋಷಿ-ಗಳೇ
ಋಷಿ-ಯರ
ಋಷಿಯು
ಎ
ಎಂ
ಎಂಎಂ
ಎಂಎಚ್
ಎಂಎಚ್ನಾಗ-ರಾಜ-ರಾವ್
ಎಂಟು
ಎಂದ
ಎಂದ-ಮೇಲೆ
ಎಂದರೆ
ಎಂದ-ರೇನು
ಎಂದಾಗ
ಎಂದಾಗಿ-ರ-ಬಹುದು
ಎಂದಾಗಿ-ರುವ
ಎಂದಾಗುತ್ತದೆ
ಎಂದಿದೆ
ಎಂದಿದ್ದರೂ
ಎಂದಿದ್ದು
ಎಂದಿರ-ಬಹುದು
ಎಂದಿರ-ಬೇಕು
ಎಂದು
ಎಂದೂ
ಎಂದೆಂದಿಗೂ
ಎಂದೆನಿಸಿ-ಕೊಳ್ಳುವುದ-ರಲ್ಲಿಯೇ
ಎಂದೆಲ್ಲಾ
ಎಂದೇ
ಎಂಬ
ಎಂಬಂತೆ
ಎಂಬಲ್ಲಿ
ಎಂಬವು
ಎಂಬಾತ-ನಿಗೆ
ಎಂಬಿವು
ಎಂಬು
ಎಂಬು-ದಕ್ಕೆ
ಎಂಬು-ದನ್ನು
ಎಂಬು-ದರ
ಎಂಬು-ದ-ರಿಂದ
ಎಂಬು-ದಾಗಿ
ಎಂಬು-ದಾಗಿಯೂ
ಎಂಬು-ದಾಗಿಯೇ
ಎಂಬುದು
ಎಂಬುದೂ
ಎಂಬುದೇ
ಎಂಬು-ವ-ನನ್ನು
ಎಂಬು-ವರು
ಎಂಬು-ವ-ವನ
ಎಂಬು-ವ-ವ-ನನ್ನು
ಎಂಬು-ವವ-ನಿಗೆ
ಎಂಬು-ವ-ವನು
ಎಂಬು-ವ-ವ-ರನ್ನು
ಎಂಬು-ವ-ವರು
ಎಂಬು-ವ-ವಳ
ಎಂಬು-ವ-ವಳು
ಎಂಬೆರು-ಮಾನರು
ಎಂಭತ್ತು-ವರ-ಹವ
ಎಂಮ
ಎಂವಿ
ಎಂವಿ-ಕೃಷ್ಣ-ರಾವ್
ಎಕಾಎಕುವಾಘನಮಾರಿತ್ಯಾ-ರಾಯಾಂಬಾಮುಲಾ
ಎಕ್ಕವ್ವೆ-ಯರ
ಎಕ್ಕೋಟಿ
ಎಡಗಯ್ಯ
ಎಡ-ಗೆಯ್ಯ
ಎಡಗೈ
ಎಡಗೈ-ಎಡ-ಭಾಗ
ಎಡಗೈಯ
ಎಡಗೈಯ್ಯ
ಎಡ-ತಲೆಯ
ಎಡ-ತಲೆ-ಯ-ಹೆಡ-ತಲೆ
ಎಡ-ತೊರೆ-ಮಠದ
ಎಡ-ದರೆ
ಎಡ-ದರೆ-ಸಾಯಿರ
ಎಡ-ದೊರೆ
ಎಡ-ದೊರೆ-ನಾ-ಡಿಗೆ
ಎಡ-ದೊರೆ-ನಾಡು
ಎಡ-ಬಲ-ಗಳಲ್ಲಿ
ಎಡ-ಬಲ-ದಲ್ಲಿರುವ
ಎಡವಂಕ-ದಲ್ಲಿ
ಎಡವದ
ಎಡ-ವಾರಯ
ಎಡೂರು
ಎಣಿಸು-ಮಗ
ಎಣ್ಣೆ-ನಾಡ
ಎತ್ತ-ರದ
ಎತ್ತ-ರದಲ್ಲಿದ್ದು
ಎತ್ತರ-ವಿದೆ
ಎತ್ತಿ-ಕಟ್ಟಿದ
ಎತ್ತಿ-ಕೊಂಡು
ಎತ್ತಿಸಿ-ದ-ನೆಂದು
ಎತ್ತಿಸಿ-ದ-ರೆಂದು
ಎತ್ತುತ್ತಿದ್ದ
ಎದು-ರಾಗಿ
ಎದು-ರಾದ
ಎದು-ರಿಗೆ
ಎದುರಿಸ
ಎದುರಿಸ-ಬೇಕಾಗಿ
ಎದುರಿಸ-ಬೇಕಾಯಿತು
ಎದುರಿ-ಸಲು
ಎದುರಿಸಿ
ಎದುರಿಸಿ-ದಾಗ
ಎದುರಿ-ಸುತ್ತಾನೆ
ಎದುರಿ-ಸುತ್ತಿದ್ದಾಗ
ಎದುರಿ-ಸುವಂತ
ಎದುರಿಸು-ವಂತೆ
ಎದ್ದನು
ಎದ್ದರು
ಎದ್ದಾಗ
ಎದ್ದಿದ್ದ
ಎದ್ದು
ಎದ್ದು-ಹೋಗಲು
ಎನ್
ಎನ್ನ-ಬಹುದು
ಎನ್ನಲು
ಎನ್ನುತ್ತಾರೆ
ಎನ್ನುತ್ತಿದ್ದರು
ಎನ್ನುವ
ಎನ್ನು-ವ-ವನ
ಎನ್ನು-ವ-ವರು
ಎನ್ನು-ವಷ್ಟ-ರಲ್ಲಿ
ಎನ್ನು-ವು-ದನ್ನು
ಎನ್ನು-ವುದು
ಎಪಿಗ್ರಾಪಿಯಾ
ಎಪಿಗ್ರಾಫಿಯಾ
ಎಪ್ಪತ್ತಕ್ಕೆ
ಎಪ್ಪತ್ತೆ-ರಡು
ಎಪ್ಪತ್ತೆ-ರೆಡು
ಎಬ್ಬೆ-ಬಸ-ವನು
ಎಬ್ಬೆಯ
ಎಮ್ಎಆರ್
ಎಮ್ಜಿ
ಎಮ್ಮದೂರ
ಎಮ್ಮಳ್ದಕ್ಕೆ
ಎಮ್ಮೆಯ
ಎಮ್ಮೆಯ-ಕೇತ-ನ-ಹಟ್ಟಿ-ಯನ್ನು
ಎರಗನ-ಹಳ್ಳಿ-ಯನ್ನು
ಎರಗಿತು
ಎರಡನೆ
ಎರಡ-ನೆಯ
ಎರಡನೆ-ಯ-ವನು
ಎರಡನೇ
ಎರಡನ್ನೂ
ಎರಡರ
ಎರಡ-ರಲ್ಲೂ
ಎರಡರು-ನೂರು
ಎರಡು
ಎರಡು-ಕಟ್ಟೆ
ಎರಡು-ಬಾರಿ
ಎರಡು-ಮೂರು
ಎರಡೂ
ಎರದಿಂಮ-ರಾ-ಜಯ್ಯ
ಎರೆ-ಗಂಗ
ಎರೆ-ಗಂಗನ
ಎರೆ-ಗಂಗ-ನದು
ಎರೆ-ಗಂಗ-ನಿಗೆ
ಎರೆ-ಗಂಗನು
ಎರೆ-ಗಂಗ-ನೆಂದು
ಎರೆದು
ಎರೆಯಂಗ
ಎರೆಯಂಗ-ದೇವ
ಎರೆಯಂಗ-ದೇವನ
ಎರೆಯಂಗನ
ಎರೆಯಂಗ-ನ-ಕಾಲ-ದಿಂದ
ಎರೆಯಂಗ-ನಿಗೆ
ಎರೆಯಂಗನು
ಎರೆಯಂಗ-ನೆಂಬ
ಎರೆಯಂಗನೇ
ಎರೆ-ಯಣ್ಣ
ಎರೆ-ಯಣ್ಣನ
ಎರೆ-ಯಪ್ಪ
ಎರೆಯಪ್ಪನ
ಎರೆ-ಯಪ್ಪ-ನನ್ನು
ಎರೆಯಪ್ಪನು
ಎರೆ-ಯಪ್ಪ-ನೆಂಬ
ಎರೆ-ಯಪ್ಪ-ನೊಂದಿಗೆ
ಎರೆ-ಯಪ್ಪ-ನೊಡನೆ
ಎರೆ-ಯಪ್ಪ-ರಸ
ಎರೆ-ಯಪ್ಪ-ರಸ-ನನ್ನು
ಎರೆ-ಯಪ್ಪ-ರಸನು
ಎರೆ-ಯಪ್ಪ-ರಸ-ನೆಂಬ
ಎರೆ-ಯಪ್ಪ-ರ-ಸರು
ಎರೆಯ-ಮಂಗಲದ
ಎರೆ-ಯಮ್ಮನ
ಎರೆಯಮ್ಮ-ನೆಂಬು-ವ-ವನು
ಎಱೆಯಂಗ
ಎಲಿಗಾರ
ಎಲಿಗಾ-ರರ
ಎಲೆ
ಎಲೆ-ಕೊಪ್ಪದ
ಎಲೆ-ಚಾ-ಕನ-ಹಳ್ಳಿಯ
ಎಲೆಯ
ಎಲೆ-ಯ-ಗುಳಿ-ಯನ್ನು
ಎಲ್ಲ
ಎಲ್ಲ-ರನ್ನೂ
ಎಲ್ಲರೂ
ಎಲ್ಲ-ವನ್ನೂ
ಎಲ್ಲವೂ
ಎಲ್ಲಾ
ಎಲ್ಲಾ-ದರೂ
ಎಲ್ಲಿಂದ
ಎಲ್ಲಿತ್ತು
ಎಲ್ಲಿದೆ
ಎಲ್ಲಿಯೂ
ಎಲ್ಲೂ
ಎಲ್ಲೆ-ಗಳನ್ನು
ಎಲ್ಲೆ-ಗಳಾಗಿದ್ದು
ಎಲ್ಲೆ-ಯನ್ನು
ಎಲ್ಲೆ-ಯಲ್ಲಿ
ಎಳಂದೂರು
ಎಳಗ
ಎಳಗ-ನೆಂಬ
ಎಳ-ವಾರೆ
ಎಳೆಯ
ಎಳೆಯ-ನೆಂದು
ಎವಿ
ಎಷ್ಟೊಂದು
ಎಷ್ಟೋ
ಎಷ್ಟೋ-ವೇಳೆ
ಎಸಗೂರು
ಎಸು-ವ-ರಾದಿತ್ಯ
ಎಸ್ಕೆ
ಎಸ್ಕೆ-ಮೋ-ಹನ್ರ-ವರ
ಎಸ್ವು-ರಾದಿತ್ಯನುಂ
ಎಸ್ಶಿವಣ್ಣ
ಏಕ-ಭೋಗ
ಏಕಾಂಗ-ವೀರ
ಏಕಾಂಗಿ-ಯಾದನು
ಏಕಾದಶಪಲ್ಲೀ
ಏಕೀ-ಕರಣ
ಏಕೆಂದರೆ
ಏಕೈಕ
ಏಕೋಜಿಯು
ಏಗಗವುಂಡನ
ಏಚಣ್ಣ
ಏಚಣ್ಣ-ದಂಡ-ನಾಯಕ
ಏಚಣ್ಣನ
ಏಚಣ್ಣನು
ಏಚ-ದಂಡಾಧೀಶ
ಏಚ-ನೆಂಬ
ಏಚಬ್ಬೆ
ಏಚಲ-ದೇವಿ
ಏಚಲ-ದೇವಿ-ಯರ
ಏಚಲ-ದೇವಿರು
ಏಚವ್ವೆ
ಏಚಿಕಬ್ಬೆ
ಏಚಿ-ದಂಡಾಧಿಪ-ನನ್ನು
ಏಚಿ-ಮಯ್ಯ
ಏಚಿ-ಮಯ್ಯ-ದಂಡ-ನಾಯಕ
ಏಚಿ-ರಾಜ
ಏಚಿ-ರಾಜ-ದಂಡಾಧೀಶನ
ಏಚಿ-ರಾಜನ
ಏಚಿ-ರಾಜ-ನನ್ನು
ಏಚಿ-ರಾಜ-ನಿಗೆ
ಏಚಿ-ರಾಜನು
ಏಚಿ-ರಾಜನೂ
ಏಚಿ-ರಾಜ-ಹಿರಿಯ
ಏಟೂರುನಿ-ವಾಸಿ
ಏತಕ್ಕೆ
ಏನಾ-ದರೂ
ಏನಿದ್ದರೂ
ಏನು
ಏನು-ಬೇಕಾದರೂ
ಏಪ್ರಿಲ್
ಏರಲು
ಏರಿದ
ಏರಿ-ದನು
ಏರಿದ-ನೆಂದು
ಏರಿದ-ರೆಂದು
ಏರಿದ್ದನು
ಏರಿದ್ದ-ರೆಂದು
ಏರಿದ್ದಾನೆ
ಏರಿರ-ಬಹುದು
ಏರಿರ-ಬಹು-ದೆಂದು
ಏರಿ-ರುವ
ಏರುತ್ತಿದ್ದ-ರೆಂದು
ಏರ್ಪಡಿ-ಸಲು
ಏರ್ಪಡಿಸಿ
ಏರ್ಪಡಿಸಿ-ದಂತೆ
ಏರ್ಪಡಿಸಿ-ದನು
ಏರ್ಪಡಿಸಿ-ರ-ಬಹುದು
ಏರ್ಪಡಿ-ಸುವ
ಏರ್ಪಡಿ-ಸು-ವಲ್ಲಿ
ಏರ್ಪಡಿ-ಸುವುದು
ಏರ್ಪಾಡು-ಮಾಡುತ್ತಾನೆ
ಏಳನೆ
ಏಳನೇ
ಏಳನೇ-ಬಾರಿಗೆ
ಏಳರಲಕ್ಕ-ಏಳೂ-ವರೆ-ಲಕ್ಷ
ಏಳು
ಏಳುತ್ತದೆ
ಏಳು-ದಿನ-ಗಳಲ್ಲಿ
ಏಳು-ನೂರು
ಏಳು-ನೂರುಮ್
ಏಳು-ಪುರ
ಏಳು-ಪುರದ
ಏಳು-ಬತ್ತೆಟ್ಟು
ಏಳು-ಬೀಳು-ಗಳು
ಏಳು-ಬೀಳು-ಗಳೇ
ಏಳು-ಮಲೆ
ಏಳು-ಹ-ಣದ
ಏಳೂ-ವರೆ-ಲಕ್ಷ
ಏಳೂ-ವರೆ-ಲಕ್ಷ-ವನ್ನು
ಏಳೆಂಟು
ಏವೊ-ಗಳ್ವೆನುನ್ನ-ತಿಯಂ
ಏೞ-ನೆಯ
ಐಂದ್ರ-ಪರ್ವಕ್ಕೆ
ಐಕ್ಯ-ವಾ-ದಂತೆ
ಐತಪಾರ್ಯನ
ಐತಿ-ಹಾಸಿಕ
ಐತಿ-ಹಾಸಿಕ-ವಾಗಿ
ಐತಿಹ್ಯ
ಐತಿಹ್ಯದ
ಐತಿಹ್ಯ-ದಂತೆ
ಐತಿಹ್ಯ-ದಿಂದಲೂ
ಐದ-ನೆಯ
ಐದನೇ
ಐದು
ಐದು-ಜನ
ಐದು-ಜನ-ರಿಗೆ
ಐದು-ಹಳ್ಳಿ-ಗಳನ್ನು
ಐನೂರು
ಐನ್
ಐನ್ಉಲ್ಮುಲ್ಕ್
ಐಮಂಗಳ
ಐವತ್ತು
ಐವತ್ತೊಕ್ಕಲು
ಐಶ್ವರ್ಯ
ಐಶ್ವರ್ಯ-ವನ್ನು
ಒಂಟೆ
ಒಂದಕ್ಕಿಂತ
ಒಂದಕ್ಕೆ
ಒಂದ-ನೆಯ
ಒಂದ-ನೆಯ-ಬಲ್ಲಾಳನ
ಒಂದನೇ
ಒಂದರ
ಒಂದಾಗಿ
ಒಂದಾ-ಗಿತ್ತು
ಒಂದಾಗಿದೆ
ಒಂದಾಗಿ-ರುವು-ದ-ರಿಂದ
ಒಂದಾದ
ಒಂದು
ಒಂದು-ಕಡೆ
ಒಂದು-ಸಲಗೆ
ಒಂದು-ಸಾವಿರ
ಒಂದು-ಸಾವಿರ-ಹೊನ್ನನ್ನು
ಒಂದೂ
ಒಂದೂ-ವರೆ
ಒಂದೆ-ರಡು
ಒಂದೇ
ಒಂದೊಂದು
ಒಂದೋ
ಒಂಬತ್ತ-ನೆಯ
ಒಂಬತ್ತು
ಒಕ್ಕಣೆ
ಒಕ್ಕಣೆ-ಗಿಂತ
ಒಕ್ಕಣೆ-ಯನ್ನೇ
ಒಕ್ಕ-ಣೆಯು
ಒಕ್ಕ-ಣೆಯೂ
ಒಕ್ಕಲಾದ
ಒಕ್ಕಲಿಕ್ಕಿದ-ನೆಂದು
ಒಕ್ಕಲಿಗ
ಒಕ್ಕಲು
ಒಕ್ಕಲು-ಗಳಿಂದ
ಒಕ್ಕಲು-ಗಳು
ಒಕ್ಕಲು-ಗೂಡಿದ್ದ-ರೆಂದು
ಒಕ್ಕಲ್ಗೆ
ಒಕ್ಕುವ
ಒಕ್ಕೂಟ-ವನ್ನು
ಒಗೊಂಡಿತ್ತು
ಒಟ್ಟಾಗಿ
ಒಟ್ಟಾಗಿಯೇ
ಒಟ್ಟಾರೆ
ಒಟ್ಟಿಗೆ
ಒಟ್ಟಿ-ನಲ್ಲಿ
ಒಟ್ಟು
ಒಡಂಬಟ್ಟು
ಒಡ-ಗೂಡಿ
ಒಡ-ಗೆರೆ
ಒಡ-ಗೆರೆ-ಮಲ್ಲ-ಇದೂ
ಒಡ-ಗೆರೆ-ಮಲ್ಲನುಂ
ಒಡ-ನೆಯೇ
ಒಡ-ಹುಟ್ಟಿದ
ಒಡೆ-ತನ
ಒಡೆ-ತನಕ್ಕೆ
ಒಡೆ-ತನ-ವನ್ನು
ಒಡೆ-ತನವೂ
ಒಡೆದು
ಒಡೆಯ
ಒಡೆ-ಯನ
ಒಡೆಯ-ನಂಬಿ-ಯಾದ
ಒಡೆಯ-ನನ್ನು
ಒಡೆಯ-ನನ್ನೇ
ಒಡೆಯ-ನಾಗಿದ್ದ-ನೆಂದು
ಒಡೆಯ-ನಾದ
ಒಡೆಯ-ನಿಗೂ
ಒಡೆಯ-ನಿಗೆ
ಒಡೆ-ಯನು
ಒಡೆಯ-ನು-ದೊಡ್ಡ-ದೇವ-ರಾಜ
ಒಡೆಯ-ನೂ-ಮಹಾಪ್ರಧಾನಿ
ಒಡೆಯ-ನೆಂದರೆ
ಒಡೆಯ-ನೆಂದು
ಒಡೆಯ-ನೆಂಬ
ಒಡೆ-ಯನೇ
ಒಡೆಯ-ನೊಬ್ಬ-ನಿಗೆ
ಒಡೆ-ಯರ
ಒಡೆಯ-ರ-ಕಟ್ಟೆ
ಒಡೆಯ-ರನ್ನು
ಒಡೆಯ-ರಾಗಿ
ಒಡೆಯ-ರಾಗಿದ್ದರು
ಒಡೆಯ-ರಾಗಿದ್ದ-ರೆಂದು
ಒಡೆಯ-ರಾದ
ಒಡೆಯ-ರಿಂದ
ಒಡೆಯ-ರಿಗೆ
ಒಡೆ-ಯರು
ಒಡೆ-ಯರ್
ಒಡೆಯಾರ
ಒಡೆಯುನು
ಒಡೆಯುರ
ಒಡೆರ-ಯರು
ಒಡ್ಡ-ಗಲ್ಲು-ರಂಗಸ್ವಾಮಿ-ಬೆಟ್ಟ
ಒಡ್ಡು
ಒತ್ತರಿಸಿ-ದ-ವನು
ಒತ್ತಿ
ಒತ್ತೆ
ಒತ್ತೆ-ಇಟ್ಟಿದ್ದ
ಒತ್ತೆ-ಯಾಗಿ
ಒತ್ತೆ-ಯಾಗಿ-ರಿಸಿ-ಕೊಂಡಿದ್ದ-ನೆಂದೂ
ಒದಗಿ-ದಾಗ
ಒದಗಿ-ಸಿದರೆ
ಒದಗಿ-ಸುತ್ತದೆ
ಒದಗಿ-ಸುತ್ತವೆ
ಒದಗಿ-ಸುತ್ತಿದ್ದ-ವರೇ
ಒದಗಿಸು-ವು-ದನ್ನು
ಒದೆ-ಯೂರು
ಒಪ್ಪ
ಒಪ್ಪಂದ
ಒಪ್ಪಂದ-ಅ-ವನ್ನು
ಒಪ್ಪಂದಕ್ಕೆ
ಒಪ್ಪಂದದ
ಒಪ್ಪಂದ-ವನ್ನು
ಒಪ್ಪಂದ-ವಾಗಿ
ಒಪ್ಪ-ತಕ್ಕದ್ದೇ
ಒಪ್ಪದ
ಒಪ್ಪ-ಬಹುದು
ಒಪ್ಪಿ
ಒಪ್ಪಿ-ಕೊಳ್ಳದೆ
ಒಪ್ಪಿ-ಕೊಳ್ಳ-ಬೇಕಾಯಿತು
ಒಪ್ಪಿಗೆ
ಒಪ್ಪಿ-ಗೆಗೆ
ಒಪ್ಪಿ-ಗೆ-ಯನ್ನು
ಒಪ್ಪಿ-ಸ-ಲಾ-ಯಿತೆಂದು
ಒಪ್ಪಿಸಿ
ಒಪ್ಪಿ-ಸಿದ
ಒಪ್ಪಿ-ಸಿ-ದ-ನಂತೆ
ಒಪ್ಪಿ-ಸಿ-ದ-ನೆಂದು
ಒಪ್ಪಿ-ಸಿ-ದಾಗ
ಒಪ್ಪುತ್ತಾರೆ
ಒಬ್ಬ
ಒಬ್ಬ-ನಾಗಿದ್ದನು
ಒಬ್ಬ-ನಾಗಿದ್ದು
ಒಬ್ಬ-ನಾಗಿ-ರ-ಬಹುದು
ಒಬ್ಬ-ನಾಗಿ-ರುವ
ಒಬ್ಬ-ನಾದ
ಒಬ್ಬನೇ
ಒಬ್ಬರ
ಒಬ್ಬ-ರಾಗಿದ್ದರು
ಒಬ್ಬ-ರಾಗಿದ್ದ-ರೆಂದು
ಒಬ್ಬ-ರಾಜ-ನಾಗಿ
ಒಬ್ಬ-ರಾದರು
ಒಬ್ಬ-ರಿಂದ
ಒಬ್ಬ-ರಿ-ಗಿಂತ
ಒಬ್ಬ-ರಿಗೊಬ್ಬ-ರಿಗೆ
ಒಬ್ಬ-ರೆಂದು
ಒಬ್ಬರೇ
ಒಬ್ಬೊಬ್ಬ
ಒಮ-ಲೂರು
ಒಮ್ಮತ-ವಿಲ್ಲ
ಒಮ್ಮೆಲೇ
ಒಯ್ದರು
ಒರಟೂರು
ಒರಿಸ್ಸಾ
ಒರಿಸ್ಸಾದ
ಒಲವು
ಒಳ
ಒಳ-ಕೇರಿಯ
ಒಳ-ಕೇರಿ-ಯಲ್ಲಿ
ಒಳಕ್ಕೆ
ಒಳಗಣ
ಒಳಗಾದ
ಒಳಗಾ-ದರು
ಒಳಗಿತ್ತೆಂದು
ಒಳಗೆ
ಒಳ-ಗೊಂಡ
ಒಳ-ಗೊಂಡಿತ್ತು
ಒಳ-ಗೊಂಡಿತ್ತೆಂದು
ಒಳ-ಗೊಂಡಿವೆ
ಒಳ-ನಾಡಾಗಿ-ರ-ಬಹುದು
ಒಳ-ಪಟ್ಟ
ಒಳ-ಪಟ್ಟರೂ
ಒಳ-ಪಟ್ಟಿತ್ತು
ಒಳ-ಪಟ್ಟಿತ್ತೆಂದು
ಒಳ-ಪಟ್ಟಿದ್ದಂತೆ
ಒಳ-ಪಡಿ-ಸ-ಲಾಗಿದೆ
ಒಳ-ಪಡಿಸಿ
ಒಳ-ಪಡಿಸಿ-ಕೊಳ್ಳಲು
ಒಳ-ಪಡಿಸಿ-ದನು
ಒಳ-ಪಡು-ವು-ದಕ್ಕೆ
ಒಳಪ್ರಾ-ಕಾರದ
ಒಳ-ಭಾಗ-ದಲ್ಲಿದ್ದು
ಒಳಮುಟ್ಟನ-ಹಳ್ಳಿ-ಗಳನ್ನು
ಒಳಲು
ಒಳ-ವಾರು
ಒಳಹೊಕ್ಕ-ನೆಂದು
ಒಳಹೊಕ್ಕವ-ರಲ್ಲಿ
ಒಳಹೊಕ್ಕು
ಒಳ್ಳೆಯ
ಓಕದ-ಕಲ್ಲು
ಓಡಾಡುತ್ತಿದ್ದ-ನಷ್ಟೆ
ಓಡಿ
ಓಡಿ-ಸಲು
ಓಡಿಸಿ
ಓಡಿ-ಸಿ-ಕೊಂಡು
ಓಡಿ-ಸಿ-ದಂತೆ
ಓಡಿ-ಸಿ-ದ-ನೆಂದೂ
ಓಡಿ-ಸಿ-ದುದು
ಓಡಿ-ಹೋಗಿದ್ದು
ಓಡಿ-ಹೋದದ್ದು
ಓಡಿ-ಹೋದ-ನಂತೆ
ಓಡಿ-ಹೋದನು
ಓಡಿ-ಹೋಯಿತು
ಓದಿ
ಓದಿ-ಕೊಂಡರೆ
ಓಬಾಂಬಿಕೆ
ಓಬಾಂಬಿಕೆ-ಯಿಂದ
ಓಬಾಂಬೆ-ಯರ
ಓರಂಗಲ್
ಓರಪಣ-ಪುರ
ಓಲಗ-ಸಾಲೆ-ಯನ್ನು
ಓಲೆ
ಓಷಧಿಪತ್ಯುಪ-ಮಾಯಿ-ತ-ಗಂಡಸ್ತೋಷಣ-ರೂಪ-ಜಿತಾಸಮಕಾಂಡಃ
ಔದಾರ್ಯಕ್ಕೆ
ಕಂಗು
ಕಂಚಲ-ದೇವಿ
ಕಂಚಿ
ಕಂಚಿ-ಗ-ಹಳ್ಳಿಯ
ಕಂಚಿ-ಗುರಿಯಪ್ಪನ-ಮೋಡಿದ
ಕಂಚಿಗೆ
ಕಂಚಿ-ಗೊಂಡ
ಕಂಚಿ-ನ-ಕೆರೆ
ಕಂಚಿ-ಪಟ್ಟಣ-ದತ್ತ
ಕಂಚಿ-ಮಠದ
ಕಂಚಿಯ
ಕಂಚಿ-ಯತ್ತ
ಕಂಚಿ-ಯನ್ನೇ
ಕಂಚಿ-ಯಿಂದ
ಕಂಚುಗ-ಹಳ್ಳಿ-ಗಳು
ಕಂಟಕ-ಗಳನ್ನು
ಕಂಟಿ-ಮಯ್ಯ
ಕಂಟಿ-ಮಯ್ಯಂ
ಕಂಟಿ-ಮಯ್ಯನ
ಕಂಟಿ-ಮಯ್ಯನು
ಕಂಟಿ-ಮಯ್ಯನೂ
ಕಂಠೀರವ
ಕಂಠೀರವ-ಗು-ಳಿಗೆ
ಕಂಠೀರವ-ನ-ರಸ-ನೃಪಾಂಬೋಧಿ
ಕಂಠೀರವ-ನ-ರಸ-ರಾಜ
ಕಂಠೀರವ-ನ-ರಸ-ರಾಜನು
ಕಂಠೀರವನು
ಕಂಠೀರವ-ನೆನಿಸಿ
ಕಂಠೀರವಾ-ಕೃತಿಃ
ಕಂಠೀರವಾಕ್ರುತಿಃ
ಕಂಠೀ-ರಾಯ-ವರ-ಹ-ವನ್ನು
ಕಂಠೀ-ವರ
ಕಂಡ
ಕಂಡನು
ಕಂಡ-ರಿಸಲ್ಪಟ್ಟಿದೆ
ಕಂಡು
ಕಂಡುಂದಿದ್ದು
ಕಂಡುಗ
ಕಂಡು-ಬಂದಿದ್ದು
ಕಂಡು-ಬಂದಿ-ರುವ
ಕಂಡು-ಬರುತ್ತದೆ
ಕಂಡು-ಬರುತ್ತವೆ
ಕಂಡು-ಬ-ರುತ್ತಾರೆ
ಕಂಡು-ಬರುತ್ತಿದೆ
ಕಂಡು-ಬ-ರುವ
ಕಂಡು-ಬ-ರು-ವು-ದಿಲ್ಲ
ಕಂಡು-ಹಿಡಿದು
ಕಂಡೆವು
ಕಂಣಂಬಾಡಿಯ
ಕಂಣನೂರ
ಕಂಣ್ನಂಬಿ-ನಾ-ತನುಂ
ಕಂಣ್ನಯ
ಕಂಣ್ನಯ-ನಾಯ-ಕನು
ಕಂದ
ಕಂದಂ
ಕಂದ-ಕುದ್ದಾಳ
ಕಂದಾಚಾರ
ಕಂದಾಚಾ-ರದ
ಕಂದಾಡಿ
ಕಂದಾಯ
ಕಂದಾ-ಯಕ್ಕೆ
ಕಂದಾ-ಯದ
ಕಂದಾಯ-ವನ್ನು
ಕಂಧರ
ಕಂನಂಬಾಡಿ
ಕಂನ-ಗೊಂಡೇಶ್ವರ
ಕಂನಡಿ
ಕಂನಾರ-ದೇವ
ಕಂನಾರ-ದೇವ-ನೆಂದು
ಕಂನೆಯ-ನಾಯ-ಕನು
ಕಂಪಂಣ-ಗಳು
ಕಂಪಣ
ಕಂಪ-ಣಕ್ಕೆ
ಕಂಪಣ-ಗಳೆಂಬ
ಕಂಪಣದ
ಕಂಪಣನ
ಕಂಪಣ್ಣ
ಕಂಪಣ್ಣನು
ಕಂಪನಿ-ಯ-ವರು
ಕಂಪನು
ಕಂಪ-ಮಂತ್ರಿ-ಯಿದ್ದ-ನೆಂದು
ಕಂಪ-ರಾಜನ
ಕಂಪ-ರಾಜನು
ಕಂಪಿಲ-ದೇವ-ನೊಡನೆ
ಕಂಪಿಲನ
ಕಂಪಿಲನು
ಕಂಪಿಲ-ನೊಡನೆ
ಕಂಪಿಲಿ-ದೇವನು
ಕಂಪಿ-ಲಿಯ
ಕಂಪೆಲನ
ಕಂಬದ
ಕಂಬ-ದ-ಹಳ್ಳಿ
ಕಂಬ-ದ-ಹಳ್ಳಿಯ
ಕಂಬ-ದ-ಹಳ್ಳಿ-ಯಲ್ಲಿ
ಕಂಬ-ದಿಂದ
ಕಂಬನ
ಕಂಬಯ್ಯ
ಕಂಬಯ್ಯ-ನನ್ನು
ಕಂಬಯ್ಯನು
ಕಂಬ-ರಾಜನು
ಕಂಬ-ವನ್ನು
ಕಂಬೇಶ್ವರ
ಕಂಭದ
ಕಂಮಗಾರ
ಕಂಸರ
ಕಇ-ವಾರ
ಕಇ-ವಾರಕ
ಕಇ-ವಾರ-ಜಗದ್ಧಳ
ಕಕುದ್ಗಿರಿ
ಕಕ್ಕಗ
ಕಗ್ಗಲೀ-ಪುರ
ಕಗ್ಗಲೀ-ಹಳ್ಳಿ
ಕಚ
ಕಚೇರಿ
ಕಚ್ಚ-ವರದ
ಕಚ್ಚಾಣ-ಗದ್ಯಾಣ-ವನ್ನು
ಕಚ್ಚೆಗ
ಕಟ
ಕಟಕ-ಕತ್ತಿ
ಕಟಕ-ತೋಟಿ-ಕಾರ
ಕಟ-ಕದ
ಕಟಕ-ದೊಡನೆ
ಕಟಕ-ರಕ್ಷಣೆ
ಕಟಕವು
ಕಟವಪ್ರ
ಕಟವಪ್ರ-ಗಿರಿ
ಕಟವಪ್ರ-ಶೈಲ
ಕಟಿ-ಸಿದ-ನೆಂದು
ಕಟೆಯೇರಿ
ಕಟ್ಟ-ಡದೊಳಕ್ಕೆ
ಕಟ್ಟ-ಲಾಗಿದೆ
ಕಟ್ಟ-ಲಾಯಿತು
ಕಟ್ಟಲು
ಕಟ್ಟ-ಳೆಯ
ಕಟ್ಟಿ
ಕಟ್ಟಿ-ಕೊಂಡು
ಕಟ್ಟಿ-ಕೊಟ-ಬಹುದು
ಕಟ್ಟಿ-ಕೊಟ್ಟರು
ಕಟ್ಟಿ-ಕೊಟ್ಟಿದ್ದಾರೆ
ಕಟ್ಟಿ-ಕೊಡಅ-ಬಹುದು
ಕಟ್ಟಿ-ಕೊಡಬಹದ್ದು
ಕಟ್ಟಿ-ಕೊಡ-ಬಹುದು
ಕಟ್ಟಿ-ಕೊಡ-ಲಾಗಿದೆ
ಕಟ್ಟಿ-ದನು
ಕಟ್ಟಿ-ದ-ನೆಂದು
ಕಟ್ಟಿ-ದ-ರಂದು
ಕಟ್ಟಿ-ದಲಗಿನಂತಿದ್ದ
ಕಟ್ಟಿ-ರುವ
ಕಟ್ಟಿಸಿ
ಕಟ್ಟಿ-ಸಿ-ಕೊಡುತ್ತಾನೆ
ಕಟ್ಟಿ-ಸಿದ
ಕಟ್ಟಿ-ಸಿ-ದನು
ಕಟ್ಟಿ-ಸಿ-ದ-ನೆಂದು
ಕಟ್ಟಿ-ಸಿ-ದ-ನೆಂದೂ
ಕಟ್ಟಿ-ಸಿ-ದರು
ಕಟ್ಟಿ-ಸಿ-ದಳು
ಕಟ್ಟಿ-ಸಿ-ದಾಗ
ಕಟ್ಟಿ-ಸಿ-ದೆವು
ಕಟ್ಟಿ-ಸಿದ್ದ
ಕಟ್ಟಿ-ಸಿದ್ದನು
ಕಟ್ಟಿ-ಸಿದ್ದ-ನೆಂದು
ಕಟ್ಟಿ-ಸಿದ್ದಾನೆ
ಕಟ್ಟಿ-ಸಿ-ರ-ಬಹುದು
ಕಟ್ಟಿ-ಸಿ-ರುವ
ಕಟ್ಟಿ-ಸುತ್ತಾನೆ
ಕಟ್ಟಿ-ಸುವ
ಕಟ್ಟು
ಕಟ್ಟು-ಕಾಲುವೆ-ಗಳು
ಕಟ್ಟು-ಕಾಲುವೆ-ಯೊಳಗಾದ
ಕಟ್ಟು-ಕಾಲುವೆ-ಯೊಳಗಿನ
ಕಟ್ಟು-ಕಾಲುವೆ-ಯೊಳಗೆ
ಕಟ್ಟು-ಪಾಡು-ಗಳನ್ನು
ಕಟ್ಟು-ವು-ದಕ್ಕಾಗಿ
ಕಟ್ಟು-ವು-ದಕ್ಕೆ
ಕಟ್ಟೆ-ಕಾಲುವೆ-ಗಳನ್ನು
ಕಟ್ಟೆ-ಕೇತ-ನ-ಹಳ್ಳಿ
ಕಟ್ಟೆಯ
ಕಟ್ಟೆ-ಯನ್ನು
ಕಟ್ಟೆ-ಯಲ್ಲಿ
ಕಟ್ಟೇನ-ಹಳ್ಳಿ
ಕಟ್ಟೇರಿ
ಕಟ್ಟೇರಿನ
ಕಟ್ಟೇ-ಸೋಮ-ನ-ಹಳ್ಳಿ
ಕಠಾರಿ
ಕಠಾರಿ-ರಾಯ
ಕಠಾರಿ-ರಾಯ-ರಾದ
ಕಠಾರಿ-ಸಾಳುವ
ಕಡತ-ವನ್ನು
ಕಡಬದ
ಕಡಲ
ಕಡಲ-ವಾಗಿಲ
ಕಡಲ-ವಾಗಿಲು
ಕಡಲು-ವಾಗಿಲು
ಕಡವು
ಕಡವು-ಅಪ್ಪು
ಕಡ-ವೂರ
ಕಡಾರಂನ್ನು
ಕಡಿ-ತಕ್ಕೇರಿತು
ಕಡಿದು
ಕಡಿದು-ಕೊಂಡ-ನೆಂದು
ಕಡಿದು-ಕೊಂಡು
ಕಡಿಮೆ
ಕಡಿಮೆ-ಯಾಗುತ್ತಾ
ಕಡು-ಕಲಿ
ಕಡುಪಿಂ
ಕಡೂರು
ಕಡೆ
ಕಡೆ-ಗಣಿಸಿ
ಕಡೆ-ಗಳಲ್ಲಿ
ಕಡೆ-ಗಳಲ್ಲಿ-ರುವ
ಕಡೆಗೆ
ಕಡೆಯ
ಕಡೆ-ಯಲ್ಲಿ
ಕಡೆ-ಯ-ವರು
ಕಡೆ-ಯಿಂದ
ಕಡ್ವಪ್ಪು
ಕಣ
ಕಣ-ಗಳ
ಕಣಿವೆ
ಕಣಿ-ವೆ-ಗಳಿವೆ
ಕಣಿ-ವೆಯ
ಕಣಿ-ವೆ-ಯಲ್ಲಿ
ಕಣಿ-ವೆ-ಯಲ್ಲಿದ್ದು
ಕಣಿ-ವೆಯು
ಕಣ್ಡೆವೆನೆ
ಕಣ್ಣಂಬಾಡಿ
ಕಣ್ಣ-ನೆಂಬು-ವ-ವನು
ಕಣ್ಣನ್ನಿಟ್ಟೇ
ಕಣ್ಣವಂಗಲ-ವನು
ಕಣ್ಣಾನೂರ
ಕಣ್ಣಾನೂ-ರನ್ನೂ
ಕಣ್ಣಾನೂರಿ-ನಲ್ಲಿ
ಕಣ್ಣಾನೂರಿ-ನಿಂದಲೇ
ಕಣ್ಣಾನೂರೇ
ಕಣ್ಣಾರೆ-ಕಂಡು
ಕಣ್ಣೆಗಾಲ-ದಲ್ಲಿ
ಕಣ್ಣೇಗಾಲ-ದಲ್ಲಿ
ಕಣ್ನಂಬಾಡಿ
ಕಣ್ನಂಬಾಡಿಯ
ಕಣ್ನಂಬಿ-ನೊಡೆಯ
ಕಣ್ವ
ಕಣ್ವೇಶ್ವರ
ಕಣ್ವೇಶ್ವರಕ್ಕೆ
ಕತ್ತರಿಗಟ್ಟ
ಕತ್ತರಿ-ಗಟ್ಟದ
ಕತ್ತರಿಗಟ್ಟ-ದಲ್ಲಿ
ಕತ್ತರಿ-ಘಟ್ಟ
ಕತ್ತರಿ-ಘಟ್ಟದ
ಕತ್ತರಿ-ಘಟ್ಟ-ದಲ್ಲಿ
ಕತ್ತರಿಸಿ
ಕತ್ತಿ
ಕಥನಮನ್ತೆಂದಡೆ
ಕಥೆ
ಕಥೆ-ಗಳನ್ನು
ಕಥೆಯ
ಕಥೆ-ಯನ್ನು
ಕಥೆ-ಯನ್ನು-ಹೇಳುವ
ಕಥೆ-ಯಲ್ಲಿ
ಕಥೆ-ಯಲ್ಲಿಯೂ
ಕಥೆಯು
ಕದಂಬ
ಕದಂಬರ
ಕದಂಬರು
ಕದಂಬ-ರೆಂದರೆ
ಕದಂಬೆ-ಹಳ್ಳಿ
ಕದಂವ
ಕದ-ನ-ಗಳಲ್ಲಿ
ಕದ-ನ-ಗಳು
ಕದ-ನತ್ರಿಣೇತ್ರನುಂ
ಕದ-ನ-ದಲ್ಲಿ
ಕದ-ನದೊಳಾಂತು
ಕದ-ನ-ದೊಳ್
ಕದ-ನಪ್ರಚಂಡ
ಕದ-ನ-ವಾಗಿ
ಕದ-ನ-ವಾಗಿ-ರ-ಬಹುದು
ಕದ-ನವು
ಕದ-ನೈಕ
ಕದ-ನೈಕ-ಸೂದ್ರಕ
ಕದ-ಪ-ನಾಯಕ
ಕದ-ಬಳ್ಳಿ
ಕದ-ಬಳ್ಳಿಯು
ಕದ-ಬ-ಹಳ್ಳಿ-ಯನ್ನು
ಕದಬೆ
ಕದ-ಬೆ-ಹಳ್ಳಿಯ
ಕದ-ರಪ್ಪ
ಕದರೂರು
ಕದ-ರೆಯ-ನಾಯಕ
ಕದ-ರೆಯ-ನಾಯ-ಕನು
ಕದ-ಲ-ಗೆರೆ
ಕದ-ಲ-ಗೆರೆ-ಯಿಂದ
ಕದ-ಳ-ಗೆರೆಯ
ಕದವಿ
ಕದವೆ-ಹಳ್ಳಿ
ಕದ್ದ-ಳಗೆರ-ಇಂದಿನ
ಕದ್ದ-ಳ-ಗೆರೆ
ಕದ್ದ-ಳ-ಗೆರೆಯ
ಕದ್ದ-ಳ-ಗೆರೆ-ಯು-ಇಂದಿನ
ಕನ-ಕ-ಕರ್ಪ್ಪೂರಧಾರಾಪ್ರವಾಹ
ಕನ-ಕ-ಗಿರಿ
ಕನ-ಕ-ಗಿರಿಯು
ಕನ-ಕ-ಚಾಮರ
ಕನ-ಕ-ಛತ್ರಂಗಳಂ
ಕನ-ಕ-ದಂಡಿಗೆ
ಕನ-ಕನ-ಘಟ್ಟ
ಕನ-ಕ-ಪುರ
ಕನ-ಕ-ಸೇನ
ಕನ-ಗನ-ಮರಡಿ
ಕನೂರು
ಕನೂರ್
ಕನ್ನಂಬಾಡಿ
ಕನ್ನಂಬಾಡಿಯ
ಕನ್ನಂಬಾಡಿ-ಯನ್ನು
ಕನ್ನಂಬಾಡಿ-ಯಲ್ಲಿ
ಕನ್ನಂಬಾಡಿ-ಯಲ್ಲಿದ್ದ
ಕನ್ನಂಬಾಡಿ-ಯ-ವ-ರೆಗೆ
ಕನ್ನಂಬಾಡಿಯು
ಕನ್ನಡ
ಕನ್ನಡದ
ಕನ್ನಡ-ದಲ್ಲಿ
ಕನ್ನಡ-ನಾಡನ್ನು
ಕನ್ನಡ-ನಾ-ಡಿನ
ಕನ್ನಡ-ನಾಡು
ಕನ್ನಡ-ವನ್ನು
ಕನ್ನಡಿಗ
ಕನ್ನಡಿಗ-ನಾಗಿದ್ದ
ಕನ್ನಡಿ-ಗರ
ಕನ್ನಡಿಗ-ರಾದ
ಕನ್ನ-ಯನ-ಪಳ್ಳಿ-ಯಲ್ಲಿ
ಕನ್ನಯ್ಯನ
ಕನ್ನರ
ಕನ್ನರ-ದೇವ
ಕನ್ನರ-ದೇವನು
ಕನ್ನ-ರನು
ಕನ್ನಲ್ಲಿ
ಕನ್ನ-ಸತ್ತಿ
ಕನ್ನಿಕೇಶ್ವರ
ಕನ್ನೆ-ಗೆರೆ-ಮಲ್ಲ
ಕನ್ನೆ-ಗೆರೆ-ಯನ್ನು
ಕನ್ನೆ-ಗೆರೆ-ಯಾಗಿ
ಕನ್ನೆಯನ
ಕನ್ನೆವ-ಸದಿ-ಯನ್ನು
ಕನ್ಯ-ಕೆಯ-ರನೊಂದೆ
ಕನ್ಯಾ-ಕುಮಾರಿ
ಕನ್ಯಾ-ದಾನ
ಕನ್ಯಾ-ದಾನ-ಗಳನ್ನು
ಕಪಿಲಾ
ಕಪಿಲಾ-ನದಿ-ಗಳ
ಕಪಿ-ಲೆಯ
ಕಪಿಳೆಯ
ಕಪ್ಪ-ಕಾಣಿ-ಕೆ-ಗಳನ್ನು
ಕಪ್ಪ-ವನ್ನು
ಕಬರಸ್ಥಾನಕ್ಕಾಗಿ
ಕಬಾಹು
ಕಬಾಹು-ನಾಡಾಳುವ
ಕಬ್ಬಪ್ಪು
ಕಬ್ಬಪ್ಪು-ನಾಡ
ಕಬ್ಬರೆ
ಕಬ್ಬಹ-ಲಿನ
ಕಬ್ಬಹಲು
ಕಬ್ಬಹು
ಕಬ್ಬಹು-ನಾಡಾಳುವ
ಕಬ್ಬಹು-ನಾ-ಡಿನ
ಕಬ್ಬಾರೆ
ಕಬ್ಬಾಳ
ಕಬ್ಬಾಳು-ದುರ್ಗ
ಕಬ್ಬಾಳು-ದುರ್ಗವು
ಕಬ್ಬಾಹು
ಕಬ್ಬಾಹು-ಕಬ್ಬಹು
ಕಬ್ಬಾಹು-ನಾಡನ್ನು
ಕಬ್ಬಾಹು-ನಾಡಾಳುವ
ಕಬ್ಬಾಹು-ನಾ-ಡಿನ
ಕಬ್ಬಾಹು-ನಾಡು
ಕಬ್ಬಾಹು-ನಾಡೇ
ಕಬ್ಬಾಹು-ಸಾಸಿರದ
ಕಬ್ಬಿಣಯುಗ
ಕಬ್ಬಿನ-ಹಳ್ಳಿ
ಕಬ್ಬು-ನಾಡ
ಕಬ್ಬು-ನಾಡಾಗಿ-ರ-ಬಹುದು
ಕಬ್ಬು-ನಾ-ಡಿನ
ಕಬ್ಬುಹು-ನಾಡ
ಕಮನೀಯಾ
ಕಮಲ-ನಾದ
ಕಮಲ-ವನ
ಕಮಳ-ರಾಜ
ಕಮೀ-ಷನರ್ಗಳ
ಕಮ್ಮಗಾರ
ಕಮ್ಮಟದ
ಕಮ್ಮಾರ
ಕಮ್ಮೆ-ಕುಲಕ್ಕೆ
ಕಯಿಸೆ-ರೆಯ-ನಿಕ್ಕಿದ
ಕಯ್ಯಲು
ಕರಂಡ
ಕರ-ಗುಂದ
ಕರಗ್ರಾಮ-ಗಳು
ಕರ-ಜಿತಸುರಭೂಜಃ
ಕರ-ಡ-ಹಳ್ಳಿ
ಕರ-ಡಿ-ಕೊಪ್ಪಲು
ಕರ-ಡಿಯ-ಹಳ್ಳಿ
ಕರಡು
ಕರಣ
ಕರ-ಣ-ಕರು
ಕರ-ಣದ
ಕರ-ಣ-ನಾಗಿದ್ದ-ನೆಂದು
ಕರ-ಣ-ನಾದ
ಕರ-ಣ-ನೆಂದರೆ
ಕರ-ಣ-ರಾಗಿದ್ದು
ಕರ-ಣರು
ಕರ-ಣರೂ
ಕರಣಿ
ಕರ-ಣಿಕ
ಕರ-ಣಿ-ಕ-ರಾದ
ಕರ-ಣಿ-ಕರು
ಕರ-ಣಿ-ಕ-ರೆಂದೇ
ಕರ-ಣಿ-ಕ-ಸೇನ-ಬೋವ-ಕುಲ-ಕರಣಿ
ಕರ-ಣಿ-ಕ-ಸೇನ-ಬೋವ-ಕುಳ-ಕರಣಿ
ಕರ-ದಾಖಿಲ-ಭೂ-ಪಾಲಃ
ಕರ-ಬಸಾಣಿ
ಕರ-ಮೆಸೆದಂ
ಕರ-ಮೆ-ಸೆಯೆ
ಕರಾ-ರುವಕ್ಕಾಗಿ
ಕರಾ-ವಳಿ
ಕರಿ
ಕರಿ-ಅಯ್ಕ-ಣನ
ಕರಿ-ಎಮ್ಮಾಉರ
ಕರಿ-ಕಲ್ಲು-ಮಂಟಿ
ಕರಿ-ಘಟ್ಟ
ಕರಿ-ತುರಕಗ-ಪಟ್ಟ-ಸಾಹಣಿ
ಕರಿ-ತುರಗ
ಕರಿ-ಪ-ಗವುಡ
ಕರಿಯ
ಕರಿ-ಯ-ಅಯ್ಕ-ಣನೆಂದು
ಕರಿ-ಯ-ನೆಚ್ಚಡೆ
ಕರಿ-ಯಯ್ಕಣ-ನೆಂದು
ಕರಿಲ
ಕರೀ-ಘಟ್ಟದ
ಕರೀ-ಜೀರ-ಹಳ್ಳಿ
ಕರು-ಣಿ-ಸುತ್ತಾನೆ
ಕರು-ಣೈಕ
ಕರೆಕಂಠ-ಜೀ-ಯನು
ಕರೆ-ದರೆ
ಕರೆ-ದಿದೆ
ಕರೆ-ದಿದ್ದಾನೆ
ಕರೆದಿದ್ದಾರೆ
ಕರೆ-ದಿದ್ದು
ಕರೆದಿರ-ಬಹುದು
ಕರೆದಿರ-ಬಹು-ದುಪ್ರಿಯ-ಸುತ
ಕರೆದಿರು-ಬಹುದು
ಕರೆ-ದಿ-ರುವ
ಕರೆ-ದಿ-ರು-ವು-ದನ್ನು
ಕರೆ-ದಿ-ರುವುದ-ರಿಂದ
ಕರೆದಿ-ರು-ವು-ದಿಲ್ಲ
ಕರೆ-ದಿಲ್ಲ
ಕರೆದಿವೆ
ಕರೆದಿವೆ-ಮೇಲ್ಕಂಡ
ಕರೆದೀವ-ದಾನಿವುಂ
ಕರೆದು
ಕರೆದು-ಕೊಂಡ
ಕರೆದು-ಕೊಂಡರೇ
ಕರೆದು-ಕೊಂಡಿದ್ದಾನೆ
ಕರೆದು-ಕೊಂಡಿದ್ದಾರೆ
ಕರೆದು-ಕೊಂಡಿ-ರುವುದ-ರಲ್ಲಿ
ಕರೆದು-ಕೊಂಡಿ-ರುವುದ-ರಿಂದ
ಕರೆದೇ
ಕರೆಯ
ಕರೆಯ-ತೊಡಗಿ-ದರು
ಕರೆಯತ್ತಿದ್ದ-ರೆಂದು
ಕರೆಯ-ಬಹುದು
ಕರೆಯ-ಲಾಗತ್ತಿತ್ತು
ಕರೆಯ-ಲಾಗಿ
ಕರೆಯ-ಲಾಗಿದೆ
ಕರೆಯ-ಲಾಗಿ-ದೆೆ
ಕರೆಯ-ಲಾ-ಗಿದ್ದು
ಕರೆಯ-ಲಾಗುತ್ತದೆ
ಕರೆಯ-ಲಾಗುತ್ತಿತು
ಕರೆಯ-ಲಾಗುತ್ತಿ-ತೆಂದೂ
ಕರೆಯ-ಲಾಗುತ್ತಿತ್ತು
ಕರೆಯ-ಲಾಗುತ್ತಿತ್ತೆಂದು
ಕರೆಯ-ಲಾಗುತ್ತಿತ್ತೆಂದೂ
ಕರೆಯ-ಲಾಯಿತು
ಕರೆಯ-ಲಾ-ಯಿತೆಂದು
ಕರೆಯಲ್ಪಡುತ್ತಿದ್ದ
ಕರೆಯಲ್ಪಡುವ
ಕರೆಯಿಸಿ-ಕೊಂಡು
ಕರೆಯುತ್ತಾರೆ
ಕರೆ-ಯುತ್ತಿದ್ದ-ನೆಂದು
ಕರೆಯುತ್ತಿದ್ದರು
ಕರೆಯುತ್ತಿದ್ದ-ರೆಂದು
ಕರೆಯು-ವು-ದರ
ಕರೆಯು-ವುದು
ಕರೆಸಿ
ಕರ್ಕ-ನನ್ನು
ಕರ್ಣನು
ಕರ್ಣವೃತ್ತಾಂತದ
ಕರ್ಣಾಟ
ಕರ್ಣಾಟಕ
ಕರ್ಣಾಟ-ಕದ
ಕರ್ಣಾಟ-ಕರ್ಣಾಟ-ಕ-ಕರ್ನಾಟ-ಕರ್ನಾಟಕ
ಕರ್ಣಾಟೇಶ್ವರ-ರಾಯ
ಕರ್ಣ್ನಾಟಧ-ರಾಮ-ರೋತ್ತಂಸಂ
ಕರ್ತ-ನಾದ
ಕರ್ತ-ನಾದ-ಅಧಿ-ಕಾರಿ
ಕರ್ತ-ರಾದ
ಕರ್ತವ್ಯ
ಕರ್ತವ್ಯ-ಗಳನ್ನು
ಕರ್ತವ್ಯ-ವಾ-ಗಿತ್ತು
ಕರ್ತೃ
ಕರ್ನಲ್
ಕರ್ನಾಟ
ಕರ್ನಾಟಕ
ಕರ್ನಾಟ-ಕದ
ಕರ್ನಾಟ-ಕ-ದಲ್ಲಿ
ಕರ್ನಾಟ-ಕ-ದಲ್ಲಿದ್ದ
ಕರ್ನಾಟ-ಕರ್ಣ್ನಾಟಕ
ಕರ್ನಾಟ-ಕ-ವನ್ನು
ಕರ್ನಾಟಕೇ
ಕರ್ನಾಟ-ಕೇಂದು-ವಾಗಿದ್ದ-ನೆಂದು
ಕರ್ನಾಟ-ಲಕ್ಷ್ಮೀ
ಕರ್ನಾಟಿಕಾ
ಕರ್ನಾಟಿಕಾದ
ಕರ್ನಾಟಿಕ್
ಕರ್ನಾಟೇಶ್ವರ-ರಾಯ
ಕರ್ನ್ನರುಂಮಪ
ಕರ್ನ್ನಾಟ
ಕರ್ನ್ನಾಟಕ
ಕರ್ಪೂ-ರದಾರ-ತಿಯ
ಕರ್ಬಪ್ಪು-ಕಳ್ಬಪ್ಪು
ಕರ್ಮಗ-ರಾಜ
ಕರ್ಮಟೇಶ್ವರ
ಕರ್ಮವಿಪಾಕ
ಕರ್ಮಾಚ್ಯುತೇಂದ್ರಃ
ಕರ್ಮ್ಮ-ಗರಾಚ
ಕರ್ಮ್ಮ-ಗರಾಚನು
ಕರ್ಮ್ಮಠೇಶ್ವರ
ಕರ್ಮ್ಮನ
ಕರ್ಮ್ಮವಿಪಾಕ
ಕಱಿಗಟ್ಟಿದ
ಕಲ-ಕುಣಿ
ಕಲಚುರಿ
ಕಲಬುರ್ಗಿ-ಯ-ವರ
ಕಲಬುರ್ಗಿ-ಯ-ವರು
ಕಲ-ವೂರ
ಕಲಶ
ಕಲಸ್ತ-ವಾಡಿ
ಕಲಹದ
ಕಲಹ-ದಲ್ಲಿ
ಕಲಹ-ಳಿಯನು
ಕಲಹ-ವಾಗಿ
ಕಲಿ
ಕಲಿ-ಕಣಿ
ಕಲಿ-ಕಣಿ-ನಾಡ
ಕಲಿ-ಕಾಲ-ಧರ್ಮ್ಮ-ರಾಜ
ಕಲಿ-ಕಾಳೇಸ್ಮಿನ್ಗಂಗ-ಮಂಡಲ
ಕಲಿ-ಗಳಂಕುಸ
ಕಲಿ-ಗಳನ್ನು
ಕಲಿ-ತ-ನದ
ಕಲಿತು
ಕಲಿ-ದೇವ
ಕಲಿ-ದೇವನ
ಕಲಿ-ದೇವ-ನ-ಹಳ್ಳಿ
ಕಲಿ-ದೇವರ
ಕಲಿ-ದೇವ-ರಿಗೆ
ಕಲಿ-ನಾಥ-ಪುರ
ಕಲಿ-ನೊಳಂಬಾದಿ
ಕಲಿ-ನೊಳಂಬಾದಿ-ರಾಜ
ಕಲಿ-ನೊಳಂಬಾದಿ-ರಾಜನ
ಕಲಿ-ನೊಳಂಬಾದಿ-ರಾಜ-ನಾಗಿದ್ದು
ಕಲಿ-ನೊಳಂಬಾದಿ-ರಾಜನು
ಕಲಿ-ನೊಳಂಬಾದಿ-ರಾಜ-ನೆಂದು
ಕಲಿ-ನೊಳಂಬಾಧಿ-ರಾಜನು
ಕಲಿಯ
ಕಲಿ-ಯಂಣನ
ಕಲಿ-ಯಣ್ಣ
ಕಲಿ-ಯಣ್ಣನು
ಕಲಿ-ಯ-ರ-ಗಂಡ
ಕಲಿ-ಯುಗ-ಭೀಮ-ನೆಂದು
ಕಲಿ-ಯುಗ-ಭೀಮಾರ್ಹಗೇಹಾದಿ
ಕಲಿ-ಯುಗಮಾತ್ತಂಡನುಂ
ಕಲಿ-ಯೂರಿನ
ಕಲಿ-ಯೂರು
ಕಲಿ-ರತ್ನ-ಪಾಲನ
ಕಲಿ-ರತ್ನ-ಪಾಲ-ನನ್ನು
ಕಲಿ-ರತ್ನ-ಪಾಲನು
ಕಲಿ-ರತ್ನ-ಪಾ-ಳನ
ಕಲಿ-ವರ್ಷ-ವನ್ನು
ಕಲಿ-ಹೃದುವ
ಕಲು-ಕಣಿ
ಕಲು-ಕಣಿ-ಕುಣಿ
ಕಲು-ಕಣಿ-ನಾಡ
ಕಲು-ಕಣಿ-ನಾ-ಡಿಗೆ
ಕಲು-ಕಣಿ-ನಾ-ಡಿನ
ಕಲು-ಕಣಿ-ನಾಡು
ಕಲು-ಕಣಿಯ
ಕಲು-ಕಣಿ-ಯೆಪ್ಪತ್ತಕ್ಕೆ
ಕಲು-ಕರೆ
ಕಲು-ಕುಣಿ
ಕಲೆಗಾರ-ರಿಗೆ-ಇ-ವರು
ಕಲ್ಕಣಿ
ಕಲ್ಕಣಿ-ನಾಡ
ಕಲ್ಕಣಿ-ನಾಡು
ಕಲ್ಕರೆ-ಕಲ್ಕುಣಿ
ಕಲ್ಕಱೆ
ಕಲ್ಕುಣಿ
ಕಲ್ಕುಣಿ-ಕಾಲು-ಕಣಿ
ಕಲ್ಕುಣಿಯ
ಕಲ್ಕುಣಿ-ಯಲ್ಲಿ
ಕಲ್ನಾಟ್ಟಾಗಿ
ಕಲ್ನಾಡಾಗಿ
ಕಲ್ಪನೆ
ಕಲ್ಪನೆ-ಯಾಗಿದೆ
ಕಲ್ಪಯತ್
ಕಲ್ಪಿಸು-ವು-ದಿಲ್ಲ
ಕಲ್ಯ
ಕಲ್ಯ-ದಲ್ಲಿ
ಕಲ್ಯಾಣ
ಕಲ್ಯಾಣ-ಚಾಲುಕ್ಯ
ಕಲ್ಯಾಣ-ಚಾಲುಕ್ಯರ
ಕಲ್ಯಾ-ಣದ
ಕಲ್ಯಾಣಿ
ಕಲ್ಯಾಣಿಯ
ಕಲ್ಲ
ಕಲ್ಲ-ಕೆರೆಯ
ಕಲ್ಲ-ಗುಂಡಿಯ-ಹಳ್ಳಿ-ಯನು
ಕಲ್ಲ-ದೇಗುಲ-ವನ್ನು
ಕಲ್ಲ-ನೆಟ್ಟು
ಕಲ್ಲನ್ನು
ಕಲ್ಲ-ಬಾ-ಗಿಲ
ಕಲ್ಲಬ್ಬ-ರಸಿಯ
ಕಲ್ಲಯ್ಯ-ಕಾಳಯ್ಯ
ಕಲ್ಲ-ಹಳ್ಳಿ
ಕಲ್ಲ-ಹಳ್ಳಿಯ
ಕಲ್ಲ-ಹಳ್ಳಿ-ಯನ್ನು
ಕಲ್ಲ-ಹಳ್ಳಿಯೇ
ಕಲ್ಲ-ಹಳ್ಳಿ-ಶಾ-ಸನ-ದಲ್ಲಿ
ಕಲ್ಲಾಗಿ-ರಬಹದು
ಕಲ್ಲಿಂದ
ಕಲ್ಲಿ-ದೇವನ
ಕಲ್ಲಿನ
ಕಲ್ಲಿನಕ್ರಮ
ಕಲ್ಲಿನ-ಬೆಟ್ಟ
ಕಲ್ಲಿನಲ್ಲಿ
ಕಲ್ಲು
ಕಲ್ಲು-ಗಳಿಂದ
ಕಲ್ಲು-ಗುಂಡಿ-ಯನ್ನು
ಕಲ್ಲು-ಗುಡ್ಡೆ
ಕಲ್ಲು-ಬಂಡೆ-ಗಳಿಂದ
ಕಲ್ಲು-ಬಂಡೆ-ಗಳು
ಕಲ್ಲು-ಮಂಟಿ
ಕಲ್ಲು-ಮಂಟಿ-ಗಳಿಂದ
ಕಲ್ಲು-ಮರಡಿ-ಯೊಳಾಡುವುದೇ
ಕಲ್ಲು-ಮಸೀತಿಗೆ
ಕಲ್ಲು-ಮಸೀದಿಗೆ
ಕಲ್ಲೆಯ-ನಾಯಕ
ಕಳಚುರಿ
ಕಳಚುರ್ಯರ
ಕಳ-ನಾಗಿ
ಕಳನಿ
ಕಳಭ್ರ
ಕಳ-ಲದ
ಕಳಲೆ
ಕಳಲೆಯ
ಕಳಶನೂ
ಕಳಸ
ಕಳ-ಸತ್ತು
ಕಳಸ-ದಂತಿದ್ದನು
ಕಳಸ-ನಿರ್ವಾಣ-ಗೆಯ್ಸಿ
ಕಳಸ್ತ-ವಾಡಿ
ಕಳಾಭ್ಯಸ್ತ-ರಿಗೆ
ಕಳಿಂಗ
ಕಳಿಯೂರ
ಕಳುಹಿ-ಸಲಾ-ಗಿತ್ತು
ಕಳುಹಿಸಿ
ಕಳುಹಿ-ಸಿದ
ಕಳುಹಿಸಿ-ದನು
ಕಳುಹಿಸಿ-ದ-ನೆಂದು
ಕಳುಹಿ-ಸಿದ್ದ
ಕಳುಹಿ-ಸುತ್ತಾನೆ
ಕಳುಹಿ-ಸುತ್ತಾರೆ
ಕಳೆದ
ಕಳೆದಿರ-ಬಹುದು
ಕಳೆದು-ಕೊಂಡಿರ-ಬಹುದು
ಕಳ್
ಕಳ್ಬಪ್ಪ
ಕಳ್ಬಪ್ಪಿನಾ
ಕಳ್ಬಪ್ಪು
ಕಳ್ಳನ-ಕೆರೆ
ಕಳ್ಳರು
ಕಳ್ಳರ್ವಾಡಿ
ಕಳ್ಳರ್ವಾಡಿಯು
ಕಳ್ವಪ್ಪಿನಾ
ಕಳ್ವಪ್ಪು
ಕಳ್ವಪ್ಪು-ನಾಡು
ಕವ
ಕವಡಯ್ಯ
ಕವನ-ಗಳಿದ್ದು
ಕವಿ
ಕವಿ-ಗಳಿಗೂ
ಕವಿ-ಗಳು
ಕವಿಗೆ
ಕವಿ-ಚರಿತೆ-ಯನ್ನು
ಕವಿ-ದಿದ್ದ
ಕವಿ-ಬುಧಾರ್ತಿಂ
ಕವಿ-ಯಲು
ಕವಿಯೂ
ಕವುಂಗಿನ
ಕವುಂಗು
ಕಷ್ಟ
ಕಷ್ಟ-ವಾಗುತ್ತದೆ
ಕಸಬ
ಕಸಬಾ
ಕಸಲಗೆರೆ
ಕಸಲಗೆ-ರೆಗೆ
ಕಸಲಗೆ-ರೆಯ
ಕಸಲಗೆ-ರೆಯೇ
ಕಸವಯ್ಯ
ಕಸಿದನು
ಕಸಿದು-ಕೊಂಡು
ಕಹಿನ
ಕಾಂಚಿಕಾಂಚನ
ಕಾಂಚಿ-ಗೊಂಡ
ಕಾಂಚಿ-ಪುರ-ವನ್ನು
ಕಾಂಚೀ-ಪುರದ
ಕಾಂತಯ್ಯನ-ವರ
ಕಾಂತಾ
ಕಾಂತಿ-ರಾಜಿಷ್ಣು
ಕಾಂತೈಯ್ಯನ-ವರ
ಕಾಂದು
ಕಾಂಭೋಜಭೋಜ-ಕಾಲಿಂಗ-ಕರ-ಹಾತಾದಿಪಾರ್ಥಿವೈಃ
ಕಾಕಡೆ
ಕಾಕತೀಯ
ಕಾಕನ-ಹಳ್ಳಿ
ಕಾಗದ-ಪತ್ರ
ಕಾಗೆ
ಕಾಗೆ-ಗಳು
ಕಾಚಿ-ದೇವ
ಕಾಚಿ-ದೇವನ
ಕಾಚಿ-ದೇವನು
ಕಾಚಿ-ನಾಯಕ
ಕಾಚಿಯ
ಕಾಚೀ-ದೇವ
ಕಾಚೀ-ದೇವನ
ಕಾಚೀ-ದೇವ-ನನ್ನು
ಕಾಚೀ-ದೇವನು
ಕಾಚೀ-ದೇವ-ನೆಂದು
ಕಾಚೆಯ
ಕಾಟ
ಕಾಡಂಕಾಖ್ಯ-ಪುರ-ವನ್ನು
ಕಾಡಕ್ಕಿ-ಯನ್ನು
ಕಾಡನ್ನು
ಕಾಡಯ-ನಾಯ-ಕನು
ಕಾಡವ-ರಾಯ
ಕಾಡಾ-ನೆಯು
ಕಾಡಿಗೆ
ಕಾಡಿನ
ಕಾಡಿನಲ್ಲಿ
ಕಾಡಿನಲ್ಲಿದ್ದ
ಕಾಡಿ-ಯೂ-ರನ್ನು
ಕಾಡಿ-ಯೂರಿನತ್ತ
ಕಾಡು
ಕಾಡು-ಕೊತ್ತನ-ಹಳ್ಳಿ
ಕಾಡು-ಕೊತ್ತನ-ಹಳ್ಳಿಯ
ಕಾಡು-ಕೊತ್ತ-ಹಳ್ಳಿ
ಕಾಡು-ಗಳಲ್ಲಿ
ಕಾಡು-ಗಳಾಗಿ
ಕಾಡು-ಗಳಿದ್ದು
ಕಾಡುಪ್ರಾಣಿ-ಗಳಿವೆ
ಕಾಡು-ಮೆಣಸಿಗೆ-ಕಾಡು-ಮೆಣಸ
ಕಾಡು-ವಿಟ್ಟಿಯ
ಕಾಡು-ವಿಟ್ಟಿಯನ್ನು
ಕಾಡು-ವಿಟ್ಟಿಯು
ಕಾಡು-ಹಂದಿ
ಕಾಡೆಮ್ಮೆ
ಕಾಡೇ
ಕಾಣ-ಬಹುದು
ಕಾಣ-ಲಿಲ್ಲ-ವಲ್ಲ
ಕಾಣಿ-ಕೆ-ಗಳನ್ನು
ಕಾಣಿಕೆಯ
ಕಾಣಿಕೆ-ಯನ್ನು
ಕಾಣಿಸ-ಕೊಂಡಾಗ
ಕಾಣಿಸ-ಕೊಂಡಿದೆ
ಕಾಣಿಸ-ಕೊಳ್ಳುತ್ತದೆ
ಕಾಣಿಸ-ಕೊಳ್ಳುತ್ತವೆ
ಕಾಣಿಸಿ
ಕಾಣಿಸಿ-ಕೊಂಡವು
ಕಾಣಿಸಿ-ಕೊಂಡಿತು
ಕಾಣಿಸಿ-ಕೊಂಡಿದೆ
ಕಾಣಿಸಿ-ಕೊಂಡಿದ್ದು
ಕಾಣಿಸಿ-ಕೊಂಡಿಲ್ಲ
ಕಾಣಿಸಿ-ಕೊಳ್ಳುತ್ತದೆ
ಕಾಣಿಸಿ-ಕೊಳ್ಳುತ್ತದೆಂದು
ಕಾಣಿಸಿ-ಕೊಳ್ಳುತ್ತಾರೆ
ಕಾಣಿಸಿ-ಕೊಳ್ಳುವ
ಕಾಣಿಸಿ-ಕೊಳ್ಳುವಾತ
ಕಾಣಿಸಿ-ಕೊಳ್ಳುವು-ದಿಲ್ಲ
ಕಾಣಿಸಿ-ಕೊಳ್ಳುವುದು
ಕಾಣಿಸು-ವು-ದಿಲ್ಲ-ವೆಂದು
ಕಾಣುತ್ತದೆ
ಕಾಣುತ್ತವೆ
ಕಾಣುತ್ತಿದ್ದರು
ಕಾಣುತ್ತಿದ್ದವು
ಕಾಣೆಮೆ
ಕಾದಾಡಿ
ಕಾದಾಡಿದ
ಕಾದಾಡಿದ್ದಾರೆ
ಕಾದಿ
ಕಾದಿ-ಕೊಂದು
ಕಾದಿ-ದ-ರೆಂದು
ಕಾದಿ-ಬಿದ್ದಾಗ
ಕಾದುವಲಿ
ಕಾದುವು-ದೆಂದು
ಕಾನನ
ಕಾನನಂ
ಕಾನೀನ-ನೆನಿಸಿ
ಕಾನ್ಸ್ಟಾಂಟಿನೋಪಲ್ಗೆ
ಕಾಪಾಡಿದ-ವನು
ಕಾಫು-ರನು
ಕಾಬೈ-ಯನು
ಕಾಮ
ಕಾಮ-ಕೋಟಿ-ದೇವಿ
ಕಾಮ-ಗವುಡನ
ಕಾಮ-ಗೆರೆ
ಕಾಮ-ಗೆರೆಯ
ಕಾಮ-ತಮ್ಮ
ಕಾಮತ್
ಕಾಮತ್ರ-ವರು
ಕಾಮ-ದೇವನ
ಕಾಮ-ನ-ಹಳ್ಳಿ
ಕಾಮಪ್ಪ-ನಾಯ-ಕ-ನೆಂಬ
ಕಾಮಯ್ಯ
ಕಾಮಯ್ಯನು
ಕಾಮ-ಲ-ದೇವಿ
ಕಾಮಿಕಬ್ಬೆ
ಕಾಮಿನೀನಾಂ
ಕಾಮಿ-ಯಕ್ಕ
ಕಾಮೆ-ನಾಯ-ಕ-ನ-ಹಳ್ಳಿ
ಕಾಮೆಯ
ಕಾಮೆಯ-ದಂಡ-ನಾಯ-ಕನು
ಕಾಮೆಯ-ದಣ್ನಾಯ-ಕರ
ಕಾಮೆಯ-ನಾಯಕ
ಕಾಮೆಯ-ನಾಯ-ಕನ
ಕಾಮೆಯ-ನಾಯ-ಕ-ನ-ಹಳ್ಳಿ
ಕಾಮೆಯ-ನಾಯ-ಕನು
ಕಾಮೆಯ-ನಾಯ-ಕರ
ಕಾಯಕ-ದಲ್ಲಿದ್ದರು
ಕಾಯುತ್ತಾ
ಕಾಯುತ್ತಿದ್ದ
ಕಾರ
ಕಾರ-ಕೂ-ನನೂ
ಕಾರ-ಕೂನರು
ಕಾರಕ್ಕೆ
ಕಾರ-ಗನ-ಹಳ್ಳಿ
ಕಾರಣ
ಕಾರ-ಣಕ್ಕಾಗಿ
ಕಾರ-ಣಕ್ಕಾಗಿಯೆ
ಕಾರ-ಣ-ದಿಂದ
ಕಾರ-ಣ-ದಿಂದಾಗಿಯೇ
ಕಾರ-ಣ-ನಾ-ದ-ನೆಂದೂ
ಕಾರ-ಣ-ರಾದ
ಕಾರ-ಣ-ವಾಗಿತ್ತೆಂಬುದು
ಕಾರ-ಣ-ವಾಗಿ-ರ-ಬಹುದು
ಕಾರ-ಣ-ವಾದಂತಿದೆ
ಕಾರ-ಣ-ವಿರ-ಬಹುದು
ಕಾರ-ಬಯಲಿನ
ಕಾರ-ಬ-ಯಲು
ಕಾರ-ಯಿತ್ವಾಜಜಾಖ್ಯಕಾಂ
ಕಾರ-ಸ-ವಾಡಿ
ಕಾರಾಗೃಹದಲ್ಲಿಟ್ಟು
ಕಾರಿ-ಕುಡಿ
ಕಾರಿ-ಮಂಗಲ-ನಾಡ
ಕಾರು
ಕಾರು-ಕನ-ಕೊಳ್ಳ
ಕಾರು-ಗ-ಹಳ್ಳಿಯ
ಕಾರುಣ್ಯಂಗೆಯ್ದು
ಕಾರುಣ್ಯ-ದಿಂದ
ಕಾರೈಕುಡಿ
ಕಾರೈಕುಡಿಯ
ಕಾರ್ತೀಕ
ಕಾರ್ಮನ
ಕಾರ್ಯ-ಕರ್ತ
ಕಾರ್ಯ-ಕರ್ತನ
ಕಾರ್ಯ-ಕರ್ತ-ನಾದ
ಕಾರ್ಯ-ಕರ್ತನೂ
ಕಾರ್ಯಕೆ
ಕಾರ್ಯ-ಕೆ-ಕರ್ತ
ಕಾರ್ಯ-ಕೆ-ಕರ್ತ-ನಾದ
ಕಾರ್ಯ-ಕೆ-ಕರ್ತ-ರನ್ನು
ಕಾರ್ಯ-ಕೆ-ಕರ್ತ-ರಾದ
ಕಾರ್ಯ-ಕೆ-ಕರ್ತರು
ಕಾರ್ಯ-ಕೆ-ಕರ್ತ-ಹೆ-ಸರು
ಕಾರ್ಯಕ್ಕಾಗಿ
ಕಾರ್ಯಕ್ಕೆ
ಕಾರ್ಯಕ್ಷೇತ್ರ-ವನ್ನಾಗಿ
ಕಾರ್ಯ-ಗಳ
ಕಾರ್ಯ-ಗಳನ್ನು
ಕಾರ್ಯ-ಗಳಲ್ಲಿ
ಕಾರ್ಯ-ಗಳು
ಕಾರ್ಯದ
ಕಾರ್ಯ-ದಲ್ಲಿ
ಕಾರ್ಯ-ನಿಮಿತ್ತ
ಕಾರ್ಯ-ನಿರ್ವ-ಹಣೆಗೆ
ಕಾರ್ಯ-ನಿರ್ವ-ಹಣೆಯ
ಕಾರ್ಯ-ನಿರ್ವಹಿ-ಸುತ್ತಿದ್ದ-ರೆಂದು
ಕಾರ್ಯ-ಮಠದ
ಕಾರ್ಯ-ವ-ನಿರ್ವಹಿ-ಸುತ್ತಿದ್ದ-ರೆಂದು
ಕಾರ್ಯ-ವನ್ನು
ಕಾರ್ಯಸ್ಥಾನ-ವನ್ನಾಗಿ
ಕಾರ್ಯಾವ-ಧಿಯು
ಕಾಲ
ಕಾಲ-ಆಂಗ್ಲ-ಭಾಷೆಯ
ಕಾಲಕ್ಕಾಗಲೇ
ಕಾಲಕ್ಕಿಂತ
ಕಾಲಕ್ಕೂ
ಕಾಲಕ್ಕೆ
ಕಾಲಕ್ಕೇ
ಕಾಲ-ಗಳಲ್ಲಿ
ಕಾಲಜ್ಞಾನ
ಕಾಲಜ್ಞಾನ-ವನ್ನು
ಕಾಲದ
ಕಾಲ-ದಂದು
ಕಾಲ-ದಲಿ
ಕಾಲ-ದಲ್ಲಿ
ಕಾಲ-ದಲ್ಲಿದ್ದ
ಕಾಲ-ದಲ್ಲಿದ್ದನು
ಕಾಲ-ದಲ್ಲಿದ್ದ-ವ-ನೆಂದು
ಕಾಲ-ದಲ್ಲಿಯೂ
ಕಾಲ-ದಲ್ಲಿಯೇ
ಕಾಲ-ದಲ್ಲೂ
ಕಾಲ-ದಲ್ಲೇ
ಕಾಲ-ದ-ವನಿರ-ಬಹುದು
ಕಾಲ-ದ-ವ-ನೆಂದು
ಕಾಲ-ದ-ವರೆಗೂ
ಕಾಲ-ದ-ವ-ರೆಗೆ
ಕಾಲ-ದ-ಸುಪ್ರ-ಸಿದ್ಧ
ಕಾಲ-ದಿಂದ
ಕಾಲ-ದಿಂದಲೂ
ಕಾಲ-ದಿಂದಲೇ
ಕಾಲದ್ದಾ-ಗಿದ್ದು
ಕಾಲದ್ದಿರ-ಬಹು-ದೆಂದು
ಕಾಲದ್ಲಲಿ
ಕಾಲ-ನಿ-ರೂಪ-ಣೆ-ಯಲ್ಲಿ
ಕಾಲನ್ನು
ಕಾಲಮ್ರಿತ್ತು
ಕಾಲ-ರಾಜ
ಕಾಲ-ವನ್ನು
ಕಾಲ-ವಾಗಿದೆ
ಕಾಲ-ವಾದ
ಕಾಲ-ವಾದ-ನೆಂದು
ಕಾಲವು
ಕಾಲವೂ
ಕಾಲವೆ
ಕಾಲವೇ
ಕಾಲಸ
ಕಾಲಾಳು-ಗಳ
ಕಾಲಾವಧಿಯ
ಕಾಲು-ಕಣಿಯ
ಕಾಲು-ಕುಣಿಯ
ಕಾಲು-ಕುಣಿ-ಯಲ್ಲಿ
ಕಾಲುವಲಿ-ಗಳು
ಕಾಲು-ವಳಿ
ಕಾಲು-ವಳಿ-ಯಾಗಿ
ಕಾಲು-ವಳ್ಳಿ
ಕಾಲು-ವಳ್ಳಿಗ
ಕಾಲು-ವಳ್ಳಿ-ಗಳ
ಕಾಲು-ವಳ್ಳಿ-ಗಳನ್ನು
ಕಾಲು-ವಳ್ಳಿ-ಗ-ಳಾದ
ಕಾಲು-ವಳ್ಳಿಯ
ಕಾಲು-ವಳ್ಳಿ-ಯಾದ
ಕಾಲುವೆ
ಕಾಲುವೆ-ಗ-ಳಿಗೆ
ಕಾಲುವೆಯ
ಕಾಲುವೆ-ಯನ್ನು
ಕಾಲುವೆ-ಯೊಳಗೆ
ಕಾಲೈಯ-ನಾಯಕ
ಕಾಲ್ಗಾ-ಹಿನ
ಕಾಲ್ದಳದ
ಕಾಲ್ವೆ
ಕಾಳಗ-ದಲಿ
ಕಾಳಗ-ದಲ್ಲಿ
ಕಾಳಗ-ವಾಗಿ-ರಬಹದು
ಕಾಳನ
ಕಾಳನ್ರಿಪಾ-ಳನ
ಕಾಳ-ಬೋವ-ನ-ಹಳ್ಳಿ-ಯನ್ನು
ಕಾಳಯ್ಯ
ಕಾಳಯ್ಯಂ
ಕಾಳಯ್ಯನು
ಕಾಳ-ರಾಜ-ನನ್ನು
ಕಾಳಲ-ದೇವಿ
ಕಾಳಲ-ದೇವಿ-ಯರ
ಕಾಳ-ಲೇಶ್ವರ
ಕಾಳಾಂಚಿಯ
ಕಾಳಾಂತಕ-ನಲುತೆ
ಕಾಳಿ
ಕಾಳಿಂಗನ-ಹಳ್ಳಿ
ಕಾಳಿಂಗ-ರಾಮ-ನ-ಹಳ್ಳಿಯ
ಕಾಳಿಯಂ
ಕಾಳಿ-ಯಕ್ಕ
ಕಾಳಿ-ಯೆಂಬ
ಕಾಳುಪಳ್ಳಿ-ಗಳನ್ನು
ಕಾಳುಪಳ್ಳಿ-ಗಳು
ಕಾಳೆಯ
ಕಾಳೆಯ-ನಾಯಕ
ಕಾಳೆಯ-ನಾಯ-ಕನ
ಕಾಳೆಯ-ನಾಯ-ಕನು
ಕಾವ
ಕಾವಣ್ಣ
ಕಾವಣ್ಣ-ನೆಂಬು-ವ-ವನೂ
ಕಾವ-ನ-ಹಳ್ಳಿ
ಕಾವ-ನ-ಹಳ್ಳಿ-ಯನ್ನು
ಕಾವಪ್ಪ
ಕಾವ-ರಾಜ
ಕಾವ-ಲುಗಾರ
ಕಾವಿಧಾರಿ-ಯಾಗಿ
ಕಾವೇಟಿ-ರಂಗ
ಕಾವೇರಿ
ಕಾವೇರಿಗೆ
ಕಾವೇರಿ-ನದಿ
ಕಾವೇರಿ-ನದಿಗೆ
ಕಾವೇರಿಯ
ಕಾವೇರಿ-ಯಿಂದ
ಕಾವೇರೀ
ಕಾವ್ಯದ
ಕಾವ್ಯ-ದಿಂದ
ಕಾವ್ಯ-ವನ್ನು
ಕಾಶಿ-ಯಲ್ಲಿ
ಕಾಶೀ-ರಾವ್ಗೆ
ಕಾಶ್ಯಪ
ಕಾಶ್ಯಪ-ಗೋತ್ರದ
ಕಿಕ್ಕೇರಿ
ಕಿಕ್ಕೇರಿಕ್ಕೆ
ಕಿಕ್ಕೇರಿಗೆ
ಕಿಕ್ಕೇರಿನ್ನು
ಕಿಕ್ಕೇರಿ-ಪುರ-ದಲ್ಲಿ
ಕಿಕ್ಕೇರಿಯ
ಕಿಕ್ಕೇರಿ-ಯನ್ನು
ಕಿಕ್ಕೇರಿ-ಯ-ಪುರ-ದಲ್ಲಿ
ಕಿಟ್ಟೆಲ್ರ-ವರು
ಕಿಡ-ದಂತೆ
ಕಿಡಿ-ಸಿದ
ಕಿತ್ತನ-ಕೆರೆ
ಕಿತ್ತನ-ಕೆರೆ-ಯನ್ನು
ಕಿತ್ತಪ್ಪ
ಕಿತ್ತಪ್ಪ-ದಂಡ-ನಾಯ-ಕನು
ಕಿತ್ತಿಪ್ಪ
ಕಿತ್ತು-ಕೊಂಡನು
ಕಿತ್ತು-ಕೊಂಡು
ಕಿತ್ತೂ-ರನ್ನು
ಕಿಮೀ
ಕಿರಂಗೂ-ರಿಗೆ
ಕಿರಂಗೂರಿನ
ಕಿರಂಗೂರೇ
ಕಿರಗತೂರ
ಕಿರಗಸೂರು
ಕಿರಾತ-ಕ-ರನ್ನು
ಕಿರಿಯ
ಕಿರಿಯದು
ಕಿರಿಯ-ನಾದುದ-ರಿಂದ
ಕಿರಿಯಯ್ಯ
ಕಿರಿಯ-ವಯಸ್ಸಿ-ನಲ್ಲೇ
ಕಿರು-ಕಾವ-ಲನ್ನು-ಸೇನೆ
ಕಿರು-ಕಾವಲು
ಕಿರುಗಣಬ್ಬೆ
ಕಿರುಗ-ವರ
ಕಿರುಗ-ವರೆ
ಕಿರುಗ-ವರೆಅ
ಕಿರುಗಾವಲನ್ನು
ಕಿರುಗಾವ-ಲಾಗಿದೆ
ಕಿರುಗಾವ-ಲಿನ
ಕಿರುಗಾವಲಿರ-ಬಹುದು
ಕಿರುಗಾವಲು
ಕಿರು-ನ-ಗರ
ಕಿರು-ನ-ಗರವು
ಕಿರು-ನ-ಗರವೇ
ಕಿರು-ಭಾಗ
ಕಿರುವೆಳ್ನ-ಗರ
ಕಿರುವೆಳ್ನ-ಗರದ
ಕಿರುವೆಳ್ನ-ಗರ-ವನ್ನು
ಕಿರುವೆಳ್ನ-ಗರವು
ಕಿರುವೆಳ್ನ-ನ-ಗರ
ಕಿಳಲೆ
ಕಿಳಲೆ-ನಾಡ
ಕಿಳಲೆ-ನಾಡನ್ನು
ಕಿಳಲೆ-ನಾಡು
ಕಿಳಲೆ-ಸಹಸ್ರದ
ಕಿಳಲೈ
ಕಿಳಲೈ-ನಾಟ್ಟ
ಕಿಳು-ವನ-ಹಳ್ಳಿ
ಕಿಳು-ವನ-ಹಳ್ಳಿ-ಕೆರೆ
ಕಿವುಡನೂ
ಕಿಸು-ಕಾಡು
ಕಿಸುಕಾಡೆಪ್ಪತ್ತು
ಕೀರ್ತಿ
ಕೀರ್ತಿ-ಅ-ರಸರ
ಕೀರ್ತಿ-ತ-ನಾಗಿದ್ದಾನೆ
ಕೀರ್ತಿ-ತನೂಭವ
ಕೀರ್ತಿ-ದೇವ
ಕೀರ್ತಿ-ದೇವನ
ಕೀರ್ತಿ-ದೇವ-ನಿಗೆ
ಕೀರ್ತಿ-ದೇವನು
ಕೀರ್ತಿ-ನಾ-ರಾಯ-ಣ-ದೇವರ
ಕೀರ್ತಿ-ನಾ-ರಾಯ-ಣ-ದೇ-ವ-ರಿಗೆ
ಕೀರ್ತಿ-ನಾ-ರಾಯ-ಣ-ರಾಯ
ಕೀರ್ತಿ-ಮಾನ್
ಕೀರ್ತಿ-ಯ-ರಸ
ಕೀರ್ತಿ-ಯ-ರಸನ
ಕೀರ್ತಿ-ಯ-ರಸ-ನ-ನನ್ನು
ಕೀರ್ತಿ-ಯ-ರಸ-ನಿಗೆ
ಕೀರ್ತಿ-ಯ-ರಸರ
ಕೀರ್ತಿಯು
ಕೀರ್ತಿ-ರಾಜು-ವಿಗೆ
ಕೀರ್ತಿರ್ಹರತಿ
ಕೀರ್ತಿ-ವಂತ-ನಾಗಿದ್ದ-ನೆಂದು
ಕೀರ್ತಿ-ವಂತ-ನೆಂದೂ
ಕೀರ್ತಿ-ವಿಲಾಸಿ-ಯಾಗಿ
ಕೀರ್ತಿ-ಸಮುದ್ರ
ಕೀರ್ತೌ
ಕೀರ್ತ್ತಿಗ
ಕೀರ್ತ್ತಿ-ದೇವ
ಕೀರ್ತ್ತಿ-ದೇವಂಗಳು
ಕೀರ್ತ್ತಿ-ರಾಜನ
ಕೀರ್ತ್ಯಂಗನ-ವಲ್ಲಭ
ಕೀರ್ತ್ಯಾವ-ತಾರವೆನ್ತೆಂದಡೆ
ಕೀರ್ತ್ರಯಾಂ
ಕೀಲಾರ
ಕೀಳಿನಿ
ಕೀಳ್
ಕುಂಚಗನ-ಹಳ್ಳಿ
ಕುಂಚದ-ಹಳ್ಳಿ
ಕುಂಚನ-ಹಳ್ಳಿ-ಗಳ
ಕುಂಚನ-ಹಳ್ಳಿ-ಯನ್ನು
ಕುಂಚಿ-ಕೊಂಡ
ಕುಂಚಿಗನ-ಹಳ್ಳಿ
ಕುಂಚಿಗನ-ಹಳ್ಳಿ-ಬೇಚಿರಾಕ್
ಕುಂಚಿ-ಗರ
ಕುಂಚಿಯ
ಕುಂಜರ
ಕುಂಟ
ಕುಂತಲ
ಕುಂತಲ-ಗಳ
ಕುಂತಲ-ಗಳು
ಕುಂತಲವು
ಕುಂತಲ-ವೆಂದೂ
ಕುಂತಲೇಂದ್ರ
ಕುಂತಿ
ಕುಂತಿ-ಬೆಟ್ಟ
ಕುಂತಿ-ಬೆಟ್ಟದ
ಕುಂತಿ-ಬೆಟ್ಟ-ದಲ್ಲಿ
ಕುಂತೂ-ರನ್ನು
ಕುಂತೂರು-ಮಠದ
ಕುಂದ-ಘಟ್ಟ
ಕುಂದಣ
ಕುಂದನ-ಹಳ್ಳಿ
ಕುಂದನ್ನಾ-ಡಿನ
ಕುಂದನ್ನಾಡು
ಕುಂದನ್ನಾಡು-ಗ-ಳಾದ್ದಿರ-ಬಹು-ದೆಂದು
ಕುಂದ-ಸತ್ತಿ
ಕುಂದಾಚಿ
ಕುಂದಾಚಿಯು
ಕುಂದಾಚ್ಚಿ
ಕುಂದಾಚ್ಚಿ-ಯರ
ಕುಂದೂ-ರನ್ನು
ಕುಂದೂರು
ಕುಂದೇಂದುಮಂದಾಕಿನೀವಿಶದಯಶಂ
ಕುಂನೆಯ-ನಾಯಕ
ಕುಂನ್ದ
ಕುಂಪೆ-ನಾಡಾಳುವ-ವ-ನನ್ನು
ಕುಂಪೆ-ನಾಡು
ಕುಂಬೇನ-ಹಳ್ಳಿ-ಯಲ್ಲಿ
ಕುಂಬೇನ-ಹಳ್ಳಿಯು
ಕುಂಭಸ್ಥಳಕ್ಕೆ
ಕುಂಭಸ್ಥಳ-ವನ್ನು
ಕುಂಭಸ್ಥಳವು
ಕುಂಮಟ
ಕುಕ-ನೂರು
ಕುಕ್ಕ-ನೂರು
ಕುಟುಂಬ-ದಲ್ಲಿ
ಕುಟುಂಬದ-ವರೂ
ಕುಟುಂಬ-ನಾಮ-ವನ್ನು
ಕುಟ್ಟಾಡಿ
ಕುಡಗಬಾಳ
ಕುಡುಗುಕೋಲಾಹಲ
ಕುಡುಗು-ನಾಡ
ಕುಡುಗು-ನಾಡನ್ನು
ಕುಡುಗು-ನಾಡು-ಗಳನ್ನು
ಕುಡುಗು-ನಾಡೊಳಗಣ
ಕುಡುಗುಬಾಳು
ಕುಡೆ
ಕುಣರಪಾಕಂ
ಕುಣಿಂಗಲ
ಕುಣಿಂಗಲ-ವೂರ
ಕುಣಿಂಗಲ-ವೂರ-ಇಂದಿನ
ಕುಣಿಂಗಲ-ವೂರಿನ
ಕುಣಿಂಗಲ-ವೂರಿನಲ್ಲಿ
ಕುಣಿಂಗಲಾಚಾರ
ಕುಣಿ-ಗಲ
ಕುಣಿ-ಗಲು
ಕುಣಿ-ಗಲ್
ಕುಣಿ-ಗಲ್ನಲ್ಲಿ
ಕುಣಿ-ಗಲ್ನಾ-ಡಿನ
ಕುಣಿ-ಗಲ್ಲು
ಕುತನೀಯೆ
ಕುತುಬ್ಷಾ
ಕುತೂಹಲ-ಕರ
ಕುತೂಹಲ-ಭರಿತ
ಕುತ್ತಾಲ
ಕುತ್ತುವ
ಕುತ್ತೂರು
ಕುದಿಹೆರುಕುದೇರು
ಕುದುರೆ-ಗಳನ್ನು
ಕುದುರೆಗುಂಡಿ-ಯನ್ನು
ಕುದು-ರೆಯ
ಕುದ್ದಾಳ
ಕುದ್ಧಾಳ
ಕುನ್ದ-ಗಾಮುಣ್ಡರು
ಕುನ್ದನ್ನಾಡನ್ನು
ಕುನ್ದನ್ನಾಡು
ಕುನ್ದ-ಸತ್ತಿ
ಕುನ್ದ-ಸತ್ತಿ-ಅ-ರಸನು
ಕುನ್ದು-ನಾಟ್ಟು
ಕುನ್ದು-ನಾಡಾಳ್ವ
ಕುನ್ದೂರು
ಕುನ್ನಂಪಾಕಂ
ಕುನ್ನಪಾಕಂ
ಕುನ್ನಲ-ಬೊಪ್ಪ
ಕುನ್ನಿಯ
ಕುಪಂಣ
ಕುಪ್ಪಣ್ಣ
ಕುಪ್ಪಾಲ್ಮಾ-ದ-ವರುಂ
ಕುಪ್ಪೆ-ಮಂಚನ-ಹಳ್ಳಿ
ಕುಪ್ಪೆ-ಮದ್ದೂರನ್ನು
ಕುಪ್ಪೆ-ಯನ್ನು
ಕುಬೇರ-ಪುರ
ಕುಮಾರ
ಕುಮಾರ-ಗೋವಿಯಂಣ್ನನ
ಕುಮಾರ-ನಾದ
ಕುಮಾರ-ನೆಂದು
ಕುಮಾರರ
ಕುಮಾರ-ರ-ಗಂಡ
ಕುಮಾರ-ರಾದ
ಕುಮಾರ-ರಾಮನ
ಕುಮಾರರು
ಕುಮಾರ-ವೃತ್ತಿ-ಯಿಂದ
ಕುಮಾರಸ್ವಾಮಿ-ಯ-ವರ
ಕುಮಾರ-ಹೆಗ್ಗಡೆ-ದೇವ
ಕುಮ್ಮಟ-ದುರ್ಗದ
ಕುರ-ವಂಕ-ನಾಡ
ಕುರಿತಂತೆ
ಕುರಿತು
ಕುರುಕಿ-ಮಾಳೆಯರ
ಕುರುಕ್ಕಿ
ಕುರುಚಲು
ಕುರು-ಣೆಯ-ನ-ಹಳ್ಳಿ-ಯನ್ನು
ಕುರು-ಣೆಯ-ನ-ಹಳ್ಳಿಯು
ಕುರುಣೇನ-ಹಳ್ಳಿ
ಕುರು-ನಂದನರ
ಕುರುಬರಕಾಳೇನ-ಹಳ್ಳಿ
ಕುರು-ಭೂಮಿ-ಯಲ್ಲಿ
ಕುರುವಂಕ
ಕುರುವಂಕದ
ಕುರುವಂಕ-ನಾಡ
ಕುರುವಂಕ-ನಾ-ಡಿಗೆ
ಕುರುವಂಕ-ನಾ-ಡಿನ
ಕುರುವಂಕ-ನಾಡು
ಕುರುವಂದ
ಕುರುವಂದ-ಕುಲ
ಕುರುವಂದ-ಕುಲ-ಕ-ಮಲ-ಮಾರ್ತಾಂಡ
ಕುರುವಂದ-ಕುಲೈಕಭೂಷಣ-ನೆನಿಸಿದ
ಕುರುವ-ಮಕ
ಕುರುವೈ
ಕುರುಹು-ಗಳಿವೆ
ಕುರುಹು-ಗಳು
ಕುರುಹೆಂದು
ಕುರ್ತಕೋಟಿ
ಕುರ್ರಬೂಲಾಡು
ಕುರ್ವಂಕ-ನಾ-ಡಿನ
ಕುರ್ವಂಕಸ್ಥಳದ
ಕುರ್ವ್ವಂಕ-ನಾಡ
ಕುಱಡುವ
ಕುಱುವಂದೇಶ್ವರ
ಕುಲ
ಕುಲಕ
ಕುಲ-ಕ-ಮಲ
ಕುಲ-ಕರ-ಣಿ-ಗಳು
ಕುಲ-ಕರ್ಣಿ
ಕುಲಕೆ
ಕುಲಕ್ಕೆ
ಕುಲಕ್ಷತ್ರಿಯ
ಕುಲ-ಗಳು
ಕುಲ-ಗಾಣಾ
ಕುಲ-ತಿಲಕ
ಕುಲ-ತಿಲಕ-ನಾಗಿದ್ದಾನೆ
ಕುಲ-ತಿಲಕ-ರಾದ
ಕುಲದ
ಕುಲ-ದಲ್ಲಿ
ಕುಲ-ದ-ವ-ನಾಗಿದ್ದು
ಕುಲ-ದ-ವರು
ಕುಲ-ದೀಪಕ-ನಾದ
ಕುಲ-ದೀಪನುಮೆನಿಪ
ಕುಲ-ದೇವ-ರಾದ
ಕುಲ-ಧ-ವಳ
ಕುಲ-ವನ-ಹಳ್ಳ-ದಲ್ಲಿ
ಕುಲ-ವೆಂಬ
ಕುಲ-ಶೇಖರ-ನೆಂಬ
ಕುಲಾಂತಕ
ಕುಲಾನ್ವ-ಯದ
ಕುಲಾನ್ವಯ-ರಾದ
ಕುಲಾನ್ವ-ಯರುಂ
ಕುಲುಮೆ-ಯಲ್ಲಿ
ಕುಲೋತ್ತುಂಗ
ಕುಲೋತ್ತುಂಗ-ಚೋಳನ
ಕುಲೋತ್ತುಂಗ-ನಿಗೆ
ಕುಲೋದ್ಧರಣ
ಕುಲೋದ್ಭ-ವನೂ
ಕುಳ
ಕುಳ-ಕರಣಿ
ಕುಳ-ಕರ-ಣಿ-ಕ-ರನ್ನು
ಕುಳ-ಕರ-ಣಿ-ಗಳ
ಕುಳ-ಕರ್ಣಿ
ಕುಳ-ತಿಳಕ
ಕುಳತ್ತೂರು
ಕುಳವ
ಕುಳ-ವ-ಕಟ್ಟಿಸಿ
ಕುಳ-ಸುಂಕ-ವನ್ನು
ಕುಳಾಂತಕ
ಕುಳಿ
ಕುಳಿ-ಗಳಿಂದ
ಕುಳಿತ
ಕುಳಿ-ತಿದ್ದ
ಕುಳಿ-ತಿ-ರುತ್ತಿದ್ದರು
ಕುಳಿ-ತಿ-ರುವ
ಕುಳಿ-ಯಿಂದ
ಕುಳ್ಳಿರ್ದ್ದು
ಕುವರ
ಕುವಳಾಲ-ಪುರ-ವರೇಶ್ವರ
ಕುವೆಂಪು
ಕುಶಲಮೆಂದಿ-ರದೋಡಿದನೊಂದೆ
ಕುಶಲಮೆಂದೋಡಿದನೊಂದೆ
ಕುಶೇಶೆ-ಯನ
ಕುಹುಯೋಗ-ವನ್ನು
ಕೂಂಡಿ
ಕೂಂಡಿ-ನಾಡ-ಕು-ಹುಂಡಿ
ಕೂಂಬಡಿ
ಕೂಗಿ-ದನು
ಕೂಚಿ-ತಂದೆ
ಕೂಟ
ಕೂಟಕ್ಕೆ
ಕೂಟ-ದೊಳು
ಕೂಟ-ವನ್ನು
ಕೂಡಲುಕುಪ್ಟೆ
ಕೂಡಲು-ಕುಪ್ಪೆ
ಕೂಡಲೂರ
ಕೂಡ-ಲೂರು
ಕೂಡಲೇ
ಕೂಡಾ
ಕೂಡಿತಪ್ಪು-ನಾಯ-ಕ-ರ-ಗಂಡ
ಕೂಡಿ-ತಪ್ಪುವ
ಕೂಡಿದ
ಕೂಡಿದ್ದ
ಕೂಡಿ-ಸಂದರು
ಕೂಡುವ-ನಾಯ-ಕರ
ಕೂಡೆ
ಕೂಡೆ-ಸೋಮೆಯ
ಕೂಡ್ಲಿ
ಕೂಡ್ಲು-ಕುಪ್ಪೆ
ಕೂತ್ತ-ಗಾವುಂಡ
ಕೂತ್ತಾಂಡಿ
ಕೂತ್ತಾನ್
ಕೂಮ್ಬಡಿ
ಕೂರಯ
ಕೂರಯ-ನಾಯ-ಕನ
ಕೂರಯ-ನಾಯ-ಕನು
ಕೂರಲ-ಗನ್ನು
ಕೂರಿಗಿ-ಹಳ್ಳಿಯ
ಕೂರಿಸಲ್ಪಟ್ಟನು
ಕೂರಿ-ಸಿದನು
ಕೂರೆಯ-ನಾಯಕ
ಕೂರೆಯ-ನಾಯ-ಕನ
ಕೂರೆಯ-ನಾಯ-ಕರು
ಕೂರ್ಗಲ್ಲನ್ನು
ಕೂಲಿಗ್ಗೆರೆ
ಕೂಲಿಗ್ಗೆರೆ-ಕೂಳ-ಗೆರೆ
ಕೂಲಿಗ್ಗೆರೆಯು
ಕೂಳಣ-ವಾಗಿ
ಕೂಸಂ
ಕೂಸಅಪರ್ಣ
ಕೂಸು-ಗ-ಳೆಂದು
ಕೃತಕ
ಕೃತಜ್ಞಂ
ಕೃತಜ್ಞ-ನಾದ
ಕೃತಯುಗ-ದಲ್ಲಿ
ಕೃತ-ವತಿ
ಕೃತಿ
ಕೃತಿ-ಗಳ
ಕೃತಿ-ಗಳನ್ನು
ಕೃತಿ-ಗಳಲ್ಲಿ
ಕೃತಿ-ಗಳಿಂದ
ಕೃತಿ-ಗಳು
ಕೃತಿಯ
ಕೃತಿ-ಯನ್ನು
ಕೃತಿ-ಯಲ್ಲಿ
ಕೃತಿ-ಯಾಗಿದೆ
ಕೃಪೆ-ಯನ್ನು
ಕೃಷಿ-ಪದ್ಧತಿ
ಕೃಷ್ಣ
ಕೃಷ್ಣ-ಕಂಧರ
ಕೃಷ್ಣ-ಕಂಧರ-ನನ್ನು
ಕೃಷ್ಣ-ಕಂಧರ-ನುಮಂ
ಕೃಷ್ಣ-ದೇವ
ಕೃಷ್ಣ-ದೇವ-ರಾಯ
ಕೃಷ್ಣ-ದೇವ-ರಾಯನ
ಕೃಷ್ಣ-ದೇವ-ರಾಯ-ನನ್ನು
ಕೃಷ್ಣ-ದೇವ-ರಾಯ-ನಿಂದ
ಕೃಷ್ಣ-ದೇವ-ರಾಯ-ನಿಗೆ
ಕೃಷ್ಣ-ದೇವ-ರಾಯನು
ಕೃಷ್ಣ-ದೇವ-ರಾಯ-ನೆಂದು
ಕೃಷ್ಣ-ದೇವ-ರಾಯನೇ
ಕೃಷ್ಣ-ದೇವ-ರಾಯ-ಪಟ್ಟಣಕ್ಕೆ
ಕೃಷ್ಣ-ದೇವಾಲಯದ
ಕೃಷ್ಣ-ದೇ-ವೊಡೆಯರ
ಕೃಷ್ಣನ
ಕೃಷ್ಣ-ನನ್ನೇ
ಕೃಷ್ಣ-ನಾಗುತ್ತಾನೆ
ಕೃಷ್ಣ-ನಿ-ಗಿಂತಲೂ
ಕೃಷ್ಣ-ನಿಗೆ
ಕೃಷ್ಣನು
ಕೃಷ್ಣನೂ
ಕೃಷ್ಣನೇ
ಕೃಷ್ಣಪ್ಪ
ಕೃಷ್ಣಪ್ಪ-ನ-ವರ
ಕೃಷ್ಣಪ್ಪ-ನಾಯ-ಕನ
ಕೃಷ್ಣಪ್ಪ-ನಾಯ-ಕ-ನಿಗೆ
ಕೃಷ್ಣಪ್ಪ-ನಾಯ-ಕನು
ಕೃಷ್ಣಪ್ಪ-ನಾಯ-ಕನೂ
ಕೃಷ್ಣಪ್ಪ-ನಾಯ-ಕ-ರಿಗೆ
ಕೃಷ್ಣ-ಮಹಾ-ಧಿ-ರಾಜನ
ಕೃಷ್ಣ-ಮೂರ್ತಿ
ಕೃಷ್ಣಯ್ಯ
ಕೃಷ್ಣ-ರಾಜ
ಕೃಷ್ಣ-ರಾಜ-ಒಡೆ-ಯನು
ಕೃಷ್ಣ-ರಾಜನ
ಕೃಷ್ಣ-ರಾಜ-ನ-ಗರ
ಕೃಷ್ಣ-ರಾಜ-ನನ್ನು
ಕೃಷ್ಣ-ರಾಜನು
ಕೃಷ್ಣ-ರಾಜ-ಪೇಟೆ
ಕೃಷ್ಣ-ರಾಜ-ಪೇಟೆಯ
ಕೃಷ್ಣ-ರಾಜರು
ಕೃಷ್ಣ-ರಾಜ-ವಡರೈಯ-ನ-ವರು
ಕೃಷ್ಣ-ರಾಜ-ವಡೆ-ಯರೈಯ್ಯಾ-ನ-ವರು
ಕೃಷ್ಣ-ರಾಜ-ಸಾ-ಗರ
ಕೃಷ್ಣ-ರಾಜ-ಸಾ-ಗರದ
ಕೃಷ್ಣ-ರಾಜ-ಸಾ-ಗರ-ದೊಳಗೆ
ಕೃಷ್ಣ-ರಾಜು
ಕೃಷ್ಣ-ರಾಜೊಡೆಯರ
ಕೃಷ್ಣ-ರಾಜೊಡೆಯರು
ಕೃಷ್ಣ-ರಾಯ
ಕೃಷ್ಣ-ರಾಯ-ನಾಯಕ
ಕೃಷ್ಣ-ರಾಯ-ನಾಯ-ಕನು
ಕೃಷ್ಣ-ರಾಯನು
ಕೃಷ್ಣ-ರಾಯ-ಪುರ-ಗಳೆಂಬ
ಕೃಷ್ಣ-ರಾಯ-ಪುರ-ವನ್ನಾಗಿ
ಕೃಷ್ಣ-ರಾಯ-ಪುರ-ವೆಂಬ
ಕೃಷ್ಣ-ರಾಯ-ಮಹಾ-ರಾಯನ
ಕೃಷ್ಣ-ರಾಯ-ರ-ಕೆರೆಯ
ಕೃಷ್ಣ-ರಾಯ-ಸಮುದ್ರ-ವೆಂದು
ಕೃಷ್ಣ-ರಾಯೇ
ಕೃಷ್ಣ-ರಾವ್
ಕೃಷ್ಣ-ವರ್ಮ-ಮಹಾ-ಧಿ-ರಾಜನು
ಕೃಷ್ಣ-ವಿಲಾ-ಸದ
ಕೃಷ್ಣ-ವೇಣಿ-ತೀರ-ದಲ್ಲಿ
ಕೃಷ್ಣಾ-ನದಿ
ಕೆಂಗಲ್ಕೊಪ್ಪ-ಲಿನ
ಕೆಂಚಪ-ನಾಯಕ
ಕೆಂಚಪ-ನಾಯ-ಕರು
ಕೆಂದನ-ಹಾಳು
ಕೆಂದನ-ಹಾಳು-ಕೆನ್ನಾಳು
ಕೆಂಪ
ಕೆಂಪ-ದೇವಯ್ಯ-ರಸ-ನಿಗೆ
ಕೆಂಪ-ನಂಜಮ್ಮಣ್ಣಿಗೆ
ಕೆಂಪ-ನಂಜೇ-ದೇವ-ರಿಗೆ
ಕೆಂಪ-ಬಯಿರ-ರಸ
ಕೆಂಪು-ನಾಯಕ
ಕೆಂಪೇ-ಗೌಡ-ನ-ಕೊಪ್ಪಲು
ಕೆಂಬಾ-ಳಿಗೆ
ಕೆಂಬಾವಿ
ಕೆಂಬಾವಿಯ
ಕೆಂಬೊಳಲನ್ನು
ಕೆಂಬೊಳ-ಲಿಗೆ
ಕೆಂಬೊಳಲು
ಕೆಅನಂತ-ರಾಮು
ಕೆಆರ್ನ-ಗರ
ಕೆಎಸ್ಶಿವಣ್ಣ
ಕೆಡಿಸಿ
ಕೆತ್ತಿಸಿದ
ಕೆನರಾ
ಕೆನ್ನ
ಕೆಬೆಟ್ಟ-ಹಳ್ಳಿ
ಕೆಬ್ಬೆ-ಹಳ್ಳಿ
ಕೆಯ್ದಾರ್
ಕೆರಗೋ-ಡಿಗೆ
ಕೆರ-ಗೋಡು
ಕೆರೆ
ಕೆರೆ-ಏರಿಯ
ಕೆರೆ-ಕಟ್ಟೆ-ಗಳ
ಕೆರೆ-ಕಟ್ಟೆ-ಗಳನ್ನು
ಕೆರೆ-ಕೋಡಿ-ಕೆರ-ಗೋಡು
ಕೆರೆ-ಗಳನ್ನು
ಕೆರೆ-ಗ-ಳನ್ನೂ
ಕೆರೆಗೆ
ಕೆರೆ-ಗೊಡಗೆ-ಯಾಗಿ
ಕೆರೆ-ಗೋಡ
ಕೆರೆ-ಗೋಡಿ-ನಾಡ
ಕೆರೆ-ಗೋಡಿ-ನಾ-ಡಿನ
ಕೆರೆ-ಗೋಡು
ಕೆರೆಯ
ಕೆರೆ-ಯ-ಕೆಳಗೆ
ಕೆರೆ-ಯನ್ನು
ಕೆರೆ-ಯನ್ನೂ
ಕೆರೆ-ಯಲ್ಲಿ
ಕೆರೆ-ಯಾಗಿ-ರ-ಬಹುದು
ಕೆರೆಯೂ
ಕೆರೆ-ಹಳ್ಳಿ
ಕೆಲ-ಗೆರೆಯ
ಕೆಲ-ಬಲ-ಗಳಲ್ಲಿದ್ದ
ಕೆಲ-ವರ
ಕೆಲ-ವರಂತೂ
ಕೆಲ-ವ-ರನ್ನು
ಕೆಲ-ವ-ರಿಗೆ
ಕೆಲ-ವರು
ಕೆಲವು
ಕೆಲವು-ಕಾಲ
ಕೆಲವು-ಭಾಗ
ಕೆಲವೆಡೆ
ಕೆಲವೇ
ಕೆಲವೊಂದ-ರಲ್ಲಿ
ಕೆಲ-ವೊಂದು
ಕೆಲವೊಮ್ಮೆ
ಕೆಲಸ
ಕೆಲಸಕ್ಕಾಗಿ
ಕೆಲಸ-ಗಳನ್ನು
ಕೆಲಸ-ವನ್ನು
ಕೆಲಸ-ವಾಗಿ-ರುವುದು
ಕೆಲ್ಲಂಗೆರೆ
ಕೆಲ್ಲಂಗೆರೆಯ
ಕೆಲ್ಲಂಗೆರೆ-ಯನು
ಕೆಲ್ಲಂಗೆರೆ-ಯನ್ನು
ಕೆಲ್ಲ-ಬ-ಸದಿಯ
ಕೆಲ್ಲ-ವತ್ತಿ
ಕೆಳ-ಕಂಡ
ಕೆಳ-ಕಂಡಂತೆ
ಕೆಳಗಣ
ಕೆಳಗಿನ
ಕೆಳಗಿನಂತಿದೆ
ಕೆಳಗಿನಂತಿವೆ
ಕೆಳಗಿ-ನಂತೆ
ಕೆಳಗಿ-ರುವ
ಕೆಳಗಿಳಿಸಿ
ಕೆಳಗಿಳಿಸಿ-ದರು
ಕೆಳಗಿಳಿಸಿ-ದು-ದಕ್ಕಾಗಿ
ಕೆಳಗೆ
ಕೆಳಗೆರೆ
ಕೆಳ-ತಿರು-ಪತಿಯ
ಕೆಳ-ದರ್ಜೆ
ಕೆಳದಿ
ಕೆಳದಿಯ
ಕೆಳದಿ-ರಾಜರ
ಕೆಳ-ಭಾಗ-ದಲ್ಲಿಯೇ
ಕೆಳಲಿ
ಕೆಳಲಿ-ನಾಡ
ಕೆಳಲೆ
ಕೆಳಲೆ-ಕಿಳಲೆ
ಕೆಳಲೆ-ನಾಡ
ಕೆಳಲೆ-ನಾಡನ್ನು
ಕೆಳಲೆ-ನಾ-ಡಿನ
ಕೆಳಲೆ-ನಾ-ಡಿನಲ್ಲಿದ್ದವು
ಕೆಳಲೆ-ನಾಡು
ಕೆಳ-ಲೆಯ
ಕೆಳಲೆ-ಯ-ನಾಡ
ಕೆಳ-ವಾಡಿ
ಕೆಳಸ್ಥರ-ದಲ್ಲಿ
ಕೆಳ-ಹಂತದ
ಕೆಳ-ಹಂತ-ದಲ್ಲಿ
ಕೆಳೆಯಬ್ಬ-ರಸಿ
ಕೆಳೆಯಬ್ಬ-ರಸಿಯು
ಕೆಸ-ವಿನ-ಕಟ್ಟೆ
ಕೆಸ್ತೂರು
ಕೇಂದ್ರ
ಕೇಂದ್ರ-ಗಳಾಗಿದ್ದವು
ಕೇಂದ್ರ-ಗಳಾಗಿ-ರು-ವು-ದನ್ನು
ಕೇಂದ್ರ-ಗಳು
ಕೇಂದ್ರ-ಗಳೂ
ಕೇಂದ್ರ-ವನ್ನಾಗಿ
ಕೇಂದ್ರ-ವನ್ನಾಗಿ-ರಿಸಿ-ಕೊಂಡು
ಕೇಂದ್ರ-ವನ್ನಾಗಿ-ಸಿ-ಕೊಂಡು
ಕೇಂದ್ರ-ವನ್ನು
ಕೇಂದ್ರ-ವಾಗಿ
ಕೇಂದ್ರ-ವಾ-ಗಿತ್ತು
ಕೇಂದ್ರ-ವಾಗಿದೆ
ಕೇಂದ್ರ-ವಾಗಿದ್ದ
ಕೇಂದ್ರ-ವಾಗಿದ್ದ-ರಿಂದ
ಕೇಂದ್ರ-ವಾಗಿದ್ದು
ಕೇಂದ್ರ-ವಾಗಿ-ಸಿ-ಕೊಂಡಿದ್ದ
ಕೇಂದ್ರ-ವಾದ
ಕೇಂದ್ರ-ವಾಯಿತು
ಕೇಂದ್ರಸ್ಥಳ-ಗಳಲ್ಲಿ
ಕೇಂದ್ರಸ್ಥಳ-ವನ್ನಾಗಿ
ಕೇಂದ್ರಸ್ಥಳ-ವಾಗಿ
ಕೇಂದ್ರೀಯ
ಕೇಂದ್ರೀ-ಯವೇ
ಕೇತ
ಕೇತ-ಗಉಡ
ಕೇತ-ಚಮೂಪತಿ
ಕೇತಣ
ಕೇತ-ಣ-ವಾ-ಹಿನೀ
ಕೇತಣ್ಣ
ಕೇತ-ನ-ಹಟ್ಟಿ
ಕೇತ-ನ-ಹಳ್ಳಿ
ಕೇತ-ನ-ಹಳ್ಳಿ-ಇಂದಿನ
ಕೇತ-ನ-ಹಳ್ಳಿ-ಯನ್ನು
ಕೇತನು
ಕೇತಪ್ಪ
ಕೇತ-ಮ-ಗೆರೆ
ಕೇತ-ಮಲ್ಲ
ಕೇತಯ್ಯ
ಕೇತಯ್ಯ-ದಂಡ-ನಾಯಕ
ಕೇತಯ್ಯನೂ
ಕೇತ-ಲ-ದೇವಿ
ಕೇತ-ಲೇಶ್ವರ
ಕೇತವ್ವೆ
ಕೇತಿ
ಕೇತಿ-ಗಾವುಂಡ
ಕೇತಿ-ಸೆಟ್ಟಿ
ಕೇತೆ-ಮಾದೆಯ-ನಾಯಕ
ಕೇತೆಯ
ಕೇತೆಯ-ಕೇತ-ಚಮೂಪತಿ
ಕೇತೆಯ-ದಂಡ-ನಾಯ-ಕನು
ಕೇತೆಯ-ನಾಯಕ
ಕೇತ್ರ
ಕೇರಳವಡ್ಡಿಯ
ಕೇರಳಾಧಿ-ಪತಿ-ಯಾಗಿರ್ದೆ
ಕೇರಳಾ-ಪುರ-ವೆಂಬ
ಕೇರಳೇ
ಕೇರಳೇ-ನಾ-ಡಿನ
ಕೇರ-ಹಳ್ಳಿಯ
ಕೇರಾಳ-ನಾಯಕ
ಕೇರಾಳ-ನಾಯ-ಕನು
ಕೇರಾಳ-ನಾಯ-ಕ-ನೆಂದಿದೆ
ಕೇರಾಳ-ಪುರವು
ಕೇಳಲು
ಕೇಳಿ
ಕೇಳಿದ
ಕೇಳಿ-ದ-ನೆಂದು
ಕೇಳಿ-ಪಡೆ-ದನು
ಕೇಳಿ-ಬ-ರು-ವು-ದಿಲ್ಲ-ವೆಂದೂ
ಕೇಳುತ್ತಿದ್ದ-ನೆಂದು
ಕೇಳ್ದಿ-ದಿ-ರುವಂದು
ಕೇವಲ
ಕೇವಲಿ-ಗಳ
ಕೇಶವ
ಕೇಶವ-ದೇವರ
ಕೇಶವ-ದೇವ-ರಿಗೆ
ಕೇಶವ-ದೇವರು
ಕೇಶವ-ದೇವಾಲಯದ
ಕೇಶವ-ನಾಥ
ಕೇಶವಾ-ಪುರ
ಕೇಶಾಲಂಕಾರ-ಗ-ಳನ್ನೂ
ಕೇಶಿ-ಯಣ್ಣ
ಕೇಶಿ-ಯಣ್ಣನು
ಕೇಸವಯ್ಯ-ನಿಗೆ
ಕೇಸಿಗ
ಕೇಸಿ-ಮಯ್ಯ
ಕೇಸಿ-ಯಣ್ಣನು
ಕೈಂಕರ್ಯಕ್ಕೆ
ಕೈಂಕರ್ಯ-ಗಳನ್ನು
ಕೈಂಕರ್ಯ-ಗ-ಳಿಗೆ
ಕೈಕಾಲು
ಕೈಕೆಳಗಿನ
ಕೈಕೆಳಗೆ
ಕೈಕೊಂಡ-ರಾರ್
ಕೈಕೊಂಡ-ರಾರ್ಚ್ಚೋಳನಂ
ಕೈಕೊಂಡು
ಕೈಕೊಳೆ
ಕೈಗಿತ್ತ
ಕೈಗೆ
ಕೈಗೊಂಡ-ನ-ಪಲ್ಲಿ-ಯನ್ನು
ಕೈಗೊಂಡ-ನ-ಪಲ್ಲಿಯು
ಕೈಗೊಂಡು
ಕೈಗೊಳ್ಳುವ
ಕೈಗೋನ-ಹಳ್ಳಿ
ಕೈತಪ್ಪಿ-ಹೋಗಿದ್ದವು
ಕೈದಾಳದ
ಕೈದೀ-ವಿಗೆಗೆ
ಕೈಫಿ-ಯತ್ತು
ಕೈಫಿ-ಯತ್ತು-ಗಳನ್ನು
ಕೈಫಿ-ಯತ್ತು-ಗಳಲ್ಲಿ
ಕೈಫಿ-ಯತ್ತು-ಗಳು
ಕೈಬಿಟ್ಟರು
ಕೈಬಿಟ್ಟು-ಹೋಗಿದ್ದ
ಕೈಬಿಟ್ಟು-ಹೋದವು
ಕೈಯ
ಕೈಯಲಿ
ಕೈಯಲು
ಕೈಯಲ್ಲಿ
ಕೈಯಲ್ಲಿತ್ತು
ಕೈಯಲ್ಲಿಯೇ
ಕೈಯಲ್ಲೇ
ಕೈಯಿಂದ
ಕೈಯ್ಯಲ್ಲಿ
ಕೈಲಾಸ-ನಾಥ
ಕೈಲಾಸಪ್ರಾಪ್ತ-ರಾಗುತ್ತಾರೆ
ಕೈಲಾಸಸ್ಥಾನ-ದಲ್ಲಿ
ಕೈಲಿ
ಕೈವಲ್ಯೇಶ್ವರ
ಕೈವಶ-ವಾಗ-ದಾಯಿತು
ಕೈವಾರ-ಕ-ರನಿರೋಧಕ
ಕೈವಾರ-ನಿಸಂಕ-ಮಲ್ಲ
ಕೈಸಾರ್ವ್ವಿನಂ
ಕೈಸೆರೆ-ಯಾಗಿದ್ದ
ಕೈಸೇರಿ
ಕೈಹಾಕಿ
ಕೊಂಕಣ
ಕೊಂಗಣಿ
ಕೊಂಗಣಿ-ವರ್ಮ
ಕೊಂಗ-ನಾಡನ್ನಾಳುತ್ತಿದ್ದಾಗ
ಕೊಂಗಮಾರಿ
ಕೊಂಗ-ಯರ್
ಕೊಂಗರ
ಕೊಂಗರ-ದಿಶಾ-ಪಟ್ಟ
ಕೊಂಗರ-ನಡಗಿಸಿ
ಕೊಂಗ-ರನ್ನು
ಕೊಂಗರಿಳಂಚಿಂಗರು
ಕೊಂಗಲ್ನಾ-ಡಿನ
ಕೊಂಗಲ್ನಾಡೊಳಗಣ
ಕೊಂಗಳ್ನಾ-ಡಿಗೆ
ಕೊಂಗಳ್ನಾಡು
ಕೊಂಗಳ್ನಾಡೆಂದು
ಕೊಂಗ-ಸೇನೆ-ಯನ್ನು
ಕೊಂಗಾ-ಳೇಶ್ವರ
ಕೊಂಗಾಳ್ನಾಡ
ಕೊಂಗಾಳ್ನಾಡಿ-ನಲ್ಲಿ
ಕೊಂಗಾಳ್ನಾಡಿ-ನಲ್ಲಿದ್ದ
ಕೊಂಗಾಳ್ನಾಡು
ಕೊಂಗಾಳ್ವ-ದೇವನು
ಕೊಂಗಾಳ್ವರ
ಕೊಂಗಾಳ್ವ-ರನ್ನು
ಕೊಂಗಾಳ್ವ-ರಾಜ-ಕುಮಾರಿ
ಕೊಂಗಾಳ್ವ-ರಿಗಿದ್ದ
ಕೊಂಗಾಳ್ವರು
ಕೊಂಗಾಳ್ವ-ರು-ಚೆಂಗಾಳ್ವರ
ಕೊಂಗು
ಕೊಂಗುಣಿ
ಕೊಂಗು-ಣಿ-ಮುತ್ತ-ರಸ
ಕೊಂಗು-ಣಿ-ವರ್ಮ
ಕೊಂಗು-ದೇಶ
ಕೊಂಗು-ದೇಶೈಕ
ಕೊಂಗು-ನಾಡನ್ನು
ಕೊಂಗು-ನಾ-ಡಿನ
ಕೊಂಗು-ನಾ-ಡಿನಿಂದ
ಕೊಂಗು-ನಾಡು
ಕೊಂಗು-ರಿಳಅಂಚಿಂಗ-ರುಮ್
ಕೊಂಚನಿರಾಶ-ನಾದ
ಕೊಂಚ-ಭಾಗ
ಕೊಂಡ
ಕೊಂಡ-ನಸಮ
ಕೊಂಡ-ನಿಂತು
ಕೊಂಡಯ್ಯ-ದೇವ
ಕೊಂಡ-ರಾಜಯ್ಯ-ದೇವ
ಕೊಂಡ-ಹಾಗೆ
ಕೊಂಡಾನ್
ಕೊಂಡು
ಕೊಂತದ
ಕೊಂತಿ-ದೇವಿ-ಗಧಿಕಂ
ಕೊಂತಿಯ
ಕೊಂದ
ಕೊಂದ-ನಲ್ಲದೆ
ಕೊಂದನು
ಕೊಂದ-ನೆಂದು
ಕೊಂದರೆ
ಕೊಂದಿಕ್ಕಿ
ಕೊಂದಿಕ್ಕಿ-ದನೊಕ್ಕಿ-ಲಿಕ್ಕಿ
ಕೊಂದಿ-ರುವ
ಕೊಂದು
ಕೊಂದು-ದಕ್ಕಾಗಿ
ಕೊಂದು-ದಕ್ಕಾಗಿಯೇ
ಕೊಂದು-ದಕ್ಕೆ
ಕೊಂದು-ಹಾಕಿದ
ಕೊಂದು-ಹಾಕಿ-ದಂತೆ
ಕೊಂಬಾಳೆ-ಯಲ್ಲಿ
ಕೊಂಬುದುಮಾ-ತನ
ಕೊಂಮೆ-ಯರ
ಕೊಟ
ಕೊಟಟ್ಟ
ಕೊಟ್ಟ
ಕೊಟ್ಟಂತಹ
ಕೊಟ್ಟಂತೆ
ಕೊಟ್ಟನು
ಕೊಟ್ಟ-ನೆಂದು
ಕೊಟ್ಟರ
ಕೊಟ್ಟ-ರದ
ಕೊಟ್ಟ-ರ-ವೆಗ್ಗಡೆ
ಕೊಟ್ಟ-ರೆಂದು
ಕೊಟ್ಟ-ಲಿಗೆ-ಎಂದು
ಕೊಟ್ಟಾಗ
ಕೊಟ್ಟಿದೆ
ಕೊಟ್ಟಿದ್ದ
ಕೊಟ್ಟಿದ್ದ-ನೆಂದು
ಕೊಟ್ಟಿದ್ದ-ರಿಂದ
ಕೊಟ್ಟಿದ್ದಾನೆ
ಕೊಟ್ಟಿದ್ದಾರೆ
ಕೊಟ್ಟಿದ್ದೇ
ಕೊಟ್ಟಿರ-ಬಹು-ದೆಂದು
ಕೊಟ್ಟಿ-ರುವ
ಕೊಟ್ಟು
ಕೊಠಾರ
ಕೊಡ-ಗನ್ನು
ಕೊಡಗ-ಹಳ್ಳಿ
ಕೊಡಗ-ಹಳ್ಳಿಯ
ಕೊಡಗಿಯ
ಕೊಡಗು
ಕೊಡಗೆ-ಹಳ್ಳಿ
ಕೊಡದೇ
ಕೊಡ-ಬಹುದು
ಕೊಡ-ಬೇಕು
ಕೊಡಲಾಗದ
ಕೊಡಲು
ಕೊಡಿ-ಸಿದ-ನೆಂದು
ಕೊಡಿಸಿ-ದ-ರೆಂದು
ಕೊಡುಗೆ
ಕೊಡುಗೆ-ಗಳು
ಕೊಡು-ಗೆಗೆ
ಕೊಡುಗೆ-ಯನ್ನು
ಕೊಡುಗೆ-ಯಲ್ಲಿ
ಕೊಡುಗೆ-ಯಾಗಿ
ಕೊಡುಗೆ-ಹಳ್ಳಿ
ಕೊಡುತ್ತ-ದೆಂದು
ಕೊಡುತ್ತಾನೆ
ಕೊಡುತ್ತಾರೆ
ಕೊಡುತ್ತಿದ್ದರು
ಕೊಡುವ
ಕೊಡುವುದ-ರಲ್ಲಿ
ಕೊಡೆ
ಕೊಡೆ-ಹಾಳ
ಕೊಣ-ನೂರು
ಕೊಣೆ-ಹಳ್ಳಿ
ಕೊಣ್ಡನಾ
ಕೊತ್ತತ್ತಿ
ಕೊತ್ತತ್ತಿಯ
ಕೊತ್ತತ್ತಿ-ಯನ್ನು
ಕೊತ್ತಲ-ವಾಡಿ
ಕೊತ್ತಲು-ಗಳಿವೆ
ಕೊತ್ತಾಗಾಲ
ಕೊತ್ತಿ-ವರ-ದ-ಹಳ್ಳಿ-ಯನ್ನು
ಕೊನಯ
ಕೊನೆಗಾಲ-ದಲ್ಲಿ
ಕೊನೆಗೆ
ಕೊನೆ-ಗೊಳಿಸಿ-ದನು
ಕೊನೆಯ
ಕೊನೆಯ-ಬಾರಿಗೆ
ಕೊನೆ-ಯಲ್ಲಿ
ಕೊನೆಯ-ವರ್ಷದ
ಕೊನೆ-ಯಾದ
ಕೊನೆಯು-ಸಿರೆಳೆಯುತ್ತಾನೆ
ಕೊನೆರಿ
ಕೊನ್ತದ
ಕೊನ್ದಡಿಯು
ಕೊನ್ದು
ಕೊಪಣಾದಿ-ತೀರ್ಥ
ಕೊಪ್ಪ
ಕೊಪ್ಪದ
ಕೊಪ್ಪ-ಲಿನ
ಕೊಪ್ಪ-ಲಿನಲ್ಲಿಯೂ
ಕೊಪ್ಪ-ಲಿನ-ವರು
ಕೊಪ್ಪಲು
ಕೊಪ್ಪಳ
ಕೊಪ್ಪ-ಳ-ದಲ್ಲಿ
ಕೊಮಾರ
ಕೊಮಾರ-ತಿ-ಯನ್ನು
ಕೊಮಾ-ರರು
ಕೊಮಾರ-ಸೇನ
ಕೊಮ್ಮಣ್ಣ
ಕೊಮ್ಮಣ್ಣನು
ಕೊಮ್ಮ-ರಾಜ
ಕೊಮ್ಮ-ರಾಜಂ
ಕೊಮ್ಮ-ರಾಜ-ನಿರ-ಬಹುದು
ಕೊಮ್ಮೇಶ್ವರ
ಕೊಯಳ-ರಸನು
ಕೊಯಿಲೋ
ಕೊರತೆ
ಕೊರಳಹಾ-ರದ
ಕೊಲು-ವಲ್ಲಿ
ಕೊಲೆ
ಕೊಲ್ಲಲು
ಕೊಲ್ಲಿ-ಪಲ್ಲವ
ಕೊಲ್ಲಿ-ಪಲ್ಲವ-ನೊಳಂಬ-ನೆಂಬ
ಕೊಲ್ಲಿಪೊಲ್ಲವ
ಕೊಲ್ಲಿಯ-ರಸ
ಕೊಲ್ಲಿಯ-ರಸನು
ಕೊಲ್ಲಿ-ಸಿದ-ನೆಂದು
ಕೊಲ್ಲುತ್ತಿ-ರುವ
ಕೊಲ್ಲುವ
ಕೊಳಗ
ಕೊಳ-ಗಳು
ಕೊಳ-ತೂರು-ಇಂದಿನ
ಕೊಳ-ವನ್ನು
ಕೊಳಾ-ಲದ
ಕೊಳುಗುಂದದ
ಕೊಳು-ವಲ್ಲಿ
ಕೊಳೆ-ಗೊಳು
ಕೊಳೆತು
ಕೊಳ್ಳಿ
ಕೊಳ್ಳಿ-ಅಯ್ಯ
ಕೊಳ್ಳಿ-ಅಯ್ಯನ
ಕೊಳ್ಳಿ-ಪಾಕೆ
ಕೊಳ್ಳಿ-ಯಮ್ಮನ
ಕೊಳ್ಳಿ-ಯಮ್ಮೆಯಂಗಳ
ಕೊಳ್ಳಿ-ಯಮ್ಮೆಯ್ಯ
ಕೊಳ್ಳಿ-ಯಮ್ಮೆಯ್ಯನ
ಕೊಳ್ಳೆಗಾಲ
ಕೊಳ್ಳೇಗಾಲ
ಕೊಳ್ಳೇಗಾ-ಲದ
ಕೊವಳೆವೆಟ್ಟು
ಕೋಗಳಿ
ಕೋಗಳಿ-ನಾಡು
ಕೋಗಿಲಲಿ
ಕೋಟಗಾ-ರರ
ಕೋಟೆ
ಕೋಟೆ-ಕುರ
ಕೋಟೆ-ಕೊತ್ತಲು-ಗಳ
ಕೋಟೆ-ಕೊತ್ತಲು-ಗಳನ್ನು
ಕೋಟೆ-ಗಳನ್ನು
ಕೋಟೆಗೆ
ಕೋಟೆ-ಬೆಟ್ಟ
ಕೋಟೆಯ
ಕೋಟೆ-ಯನ್ನು
ಕೋಟೆ-ಯ-ಬಯಲ
ಕೋಟೆ-ಯಲ್ಲಿ
ಕೋಟೆ-ಯೆಂಬ
ಕೋಟೆ-ಯೊಳಗಿ-ರುವ
ಕೋಟ್ಟ
ಕೋಡಾಲ
ಕೋಡಾ-ಲದ
ಕೋಡಾಲ-ವನ್ನು
ಕೋಡಾಲವು
ಕೋಡಿ
ಕೋಡಿ-ನ-ಕೊಪ್ಪ
ಕೋಡಿ-ನ-ಕೊಪ್ಪ-ಕೋಡಿ-ಹಳ್ಳಿ
ಕೋಡಿ-ಪುರ
ಕೋಣನ-ಕಲ್ಲು
ಕೋಣೆಯ
ಕೋತನ-ಪುರ
ಕೋದಂಡ-ರಾಮ
ಕೋದಂಡ-ರಾಮಸ್ವಾಮಿಯ
ಕೋನಾ-ಪುರ
ಕೋನೇಟಿ
ಕೋನೇರಿನ್ಮೈ
ಕೋನೇರಿಮ್ಮೈ
ಕೋಮಟಿ-ಗಳ
ಕೋಮನ-ಹಳ್ಳಿ
ಕೋರವಂಗಲ
ಕೋರವಂಗ-ಲದ
ಕೋರಿ-ಕೆಯ
ಕೋರಿ-ದನು
ಕೋರೆಗಾಲ
ಕೋರೆಗಾ-ಲದ
ಕೋಲಾರ
ಕೋಳಾಲ
ಕೋಳಾಲ-ಪುರದ
ಕೋಳಾಲ-ಪುರ-ಪರಮೇಶ್ವರ
ಕೋಳಾಹಳ
ಕೋಳಿಗಾಲ
ಕೋಳೋಗಾಲ
ಕೋಳೋಗಾಲ-ವನ್ನು
ಕೋಶವು
ಕೋಶಸ್ಯ
ಕೋಶಾಧಿ-ಕಾರಿ
ಕೌಂಡಿಣ್ಯ
ಕೌಂಡಿನ್ಯ
ಕೌಂಡಿಲ್ಯ
ಕೌಡಲಿ
ಕೌಡಲಿ-ಇಂದಿನ
ಕೌಡ್ಲೆ
ಕೌಡ್ಲೆಯು
ಕೌಣ್ಡಿಲ್ಯ
ಕೌಲನ್ನು
ಕೌಶಿಕ
ಕೌಶಿಕ-ಕುಳಾಂಬರ
ಕೌಶಿಕ-ಗೋತ್ರ-ಅ-ಪವಿತ್ರನೂ
ಕೌಶಿಕ-ಗೋತ್ರದ
ಕೌಸಲ್ಯ
ಕೌಸಲ್ಯಾ
ಕ್ಕೂ
ಕ್ಕೆ
ಕ್ಕೋವಿದಂ
ಕ್ಯಾತನ-ಹಳ್ಳಿ
ಕ್ಯಾತುಂಗೆರೆ
ಕ್ರಮ-ಬದ್ಧ-ವಾಗಿ
ಕ್ರಮ-ವಾಗಿ
ಕ್ರಮೇಣ
ಕ್ರಯದ
ಕ್ರಯ-ದಾನ-ವಾಗಿ
ಕ್ರಯ-ಪತ್ರ-ವನ್ನು
ಕ್ರಯ-ವಾಗಿ
ಕ್ರಯ-ವಾಗಿ-ಕೊಂಡು-ಅ-ದನ್ನು
ಕ್ರಯ-ಶಾ-ಸನದ
ಕ್ರಯ-ಶಾ-ಸನ-ವನ್ನು
ಕ್ರಿ
ಕ್ರಿಪೂ-ದಲ್ಲಿಯೇ
ಕ್ರಿರ
ಕ್ರಿಶ
ಕ್ರಿಶಕ್ಕೆ
ಕ್ರಿಶ-ಗಂಗ-ರಾಷ್ಟ್ರ-ಕೂಟರ
ಕ್ರಿಶ-ನೆಯ
ಕ್ರಿಶನೇ
ಕ್ರಿಶರ
ಕ್ರಿಶ-ರದ್ದೇ
ಕ್ರಿಶ-ರ-ನಂತರ
ಕ್ರಿಶ-ರ-ರಲ್ಲಿ
ಕ್ರಿಶ-ರಲ್ಲಿ
ಕ್ರಿಶ-ರಲ್ಲಿಯೂ
ಕ್ರಿಶ-ರಲ್ಲೇ
ಕ್ರಿಶ-ರ-ವರೆ-ಗಿನ
ಕ್ರಿಶ-ರ-ವ-ರೆಗೆ
ಕ್ರಿಶ-ರಿಂದ
ಕ್ರಿಶ-ರಿಂದಲೇ
ಕ್ರಿಶ-ಸು-ಮಾರು
ಕ್ರಿಶ-ಹಿಜರಿ
ಕ್ಲಪ್ತವಿಷ್ಣ್ವೀಶಪೂಜಃ
ಕ್ವ್ಮಾಪತೇಃ
ಕ್ಷತಮಱೆವಾತಂ
ಕ್ಷತ್ರಿಯ-ರಾದ
ಕ್ಷತ್ರಿಯರು
ಕ್ಷತ್ರಿಲಾಡರಿ
ಕ್ಷಮಾಧೀಶ
ಕ್ಷಮಾಧೀಶತಾ
ಕ್ಷಮಿಸಿ
ಕ್ಷಾಮಢಾಮರ-ಗಳು
ಕ್ಷಾಮಿಸ್ಫು-ರನ್
ಕ್ಷಿತಿ-ನಾಥ
ಕ್ಷಿತಿ-ಪಾಲ-ಕನು
ಕ್ಷಿತಿ-ಪಾಲ-ಮೌಳಿರ್ವ-ದಾನ್ಯ-ಮೂರ್ತಿಃ
ಕ್ಷಿತೀಂದ್ರ
ಕ್ಷಿತೀಂದ್ರನ
ಕ್ಷಿತೀಂದ್ರನು
ಕ್ಷಿತೀಂದ್ರ-ನೆಂಬ
ಕ್ಷಿತೀಶ್ವ-ರನು
ಕ್ಷೇತ್ರ-ಕಾರ್ಯದ
ಕ್ಷೇತ್ರ-ಗ-ಳಾದ
ಕ್ಷೇತ್ರ-ದಲ್ಲಿ
ಕ್ಷೇತ್ರ-ವಾದ
ಕ್ಷೇತ್ರಾಯ
ಕ್ಷೋಣೀವಧೂಭೂಷಣೇ
ಕ್ಷೋಭೆ-ಗಳನ್ನು
ಕ್ಷೋಭೆ-ಯನ್ನು
ಕ್ಷ್ಮಾಯಾಂರಾಜ್ಯ
ಕೞಅ್ಬಹು-ಕೞ್ಬಾಹು-ಕಬ್ಬಾಹು
ಕೞ್ಬಪ್ಪು
ಖಂಡಸ್ಫುಟಿತ
ಖಂಡಿಸಿ
ಖಂಡುಗ
ಖಂಡೆಯ-ರಾಯ
ಖಂತಿ-ಕಾರ
ಖಗ-ರಾಜ-ನಿನೇ-ಕದೊಡನಿಂ
ಖಚಿತ
ಖಚಿತ-ಗುಂಪೇ
ಖಚಿತ-ಪಡಿ-ಸುತ್ತದೆ
ಖಚಿತ-ಪಡಿ-ಸುತ್ತವೆ
ಖಚಿತ-ಪಡಿ-ಸು-ವಲ್ಲಿ
ಖಚಿತ-ಪಡುತ್ತದೆ
ಖಚಿತ-ವಾಗಿ
ಖಚಿತ-ವಾಗುತ್ತದೆ
ಖಚಿತ-ವಿಲ್ಲ
ಖಜಾನೆಗೆ
ಖಜಾ-ನೆಯ
ಖಡಿಲೆ-ಗೊಂಡು
ಖದೀರ್
ಖರ
ಖರದೂಷಣ-ರನ್ನು
ಖರೀದಿ
ಖರೀದಿಸಿ
ಖರೀದಿ-ಸುತ್ತಾರೆ
ಖರ್ಚುವೆಚ್ಚ-ಗಳ
ಖಲೀಫ-ನಾದ
ಖಳತ್ರಿಣೇತ್ರ
ಖಳ್ಗದಿಂ
ಖಾತಿ-ಧರಿತಿತೆ
ಖಾದ್ರಿ
ಖಾಯಿಲೆ
ಖಾಲಿ
ಖಾಸ
ಖಾಸಗಿ
ಖಾಸಾ
ಖಾಸಾ-ಬೊಕ್ಕ-ಸದ
ಖಿಡಿಜ್ಚಿಣ್ಡಿಜ್ಡಿ
ಖಿಲ-ವಾಗಿದ್ದ
ಖಿಲ-ವಾಗಿ-ರಲು
ಖುದಾದಾದ್
ಖೈಬ-ರದ
ಖೊಟ್ಟಿ-ಗನು
ಖ್ಯಾತ
ಖ್ಯಾತ-ನಾಗಿದ್ದನು
ಖ್ಯಾತ-ನಾಗಿದ್ದು
ಖ್ಯಾತ-ನಾದ-ನೆಂದು
ಖ್ಯಾತಸ್ಯಾ-ನಸ್ಯ
ಖ್ಯಾತಿಯ
ಖ್ಯಾತೆಯಾಗ
ಖ್ಯಾತೆ-ಯಾದ
ಖ್ವಾಜಾ
ಗಂಗ
ಗಂಗ-ಕುಳ-ಚಂದ್ರಂ
ಗಂಗ-ಗಾ-ಮುಂಡ
ಗಂಗ-ಚಮೂಪಂ
ಗಂಗ-ಡಿ-ಕಾರ
ಗಂಗಣ್ಣ
ಗಂಗ-ದಂಡಾಧೀಶ
ಗಂಗ-ದಂಡಾಧೀಶನ
ಗಂಗ-ದಂಡಾಧೀಶ-ನನ್ನು
ಗಂಗ-ದಂಡಾಧೀಶನು
ಗಂಗ-ದಂಡೇಶ
ಗಂಗ-ದೇಶಾಧಿಪ
ಗಂಗ-ದೊರೆಯ
ಗಂಗ-ನನ್ನು
ಗಂಗ-ನ-ಹಳ್ಳಿ
ಗಂಗ-ನಾ-ರಾಯಣ
ಗಂಗ-ನೃಪನು
ಗಂಗ-ನೊಳಂಬರ
ಗಂಗ-ಪಯ್ಯನ
ಗಂಗ-ಪಲ್ಲವರ
ಗಂಗ-ಪೆರ್ಮಾನಡಿ
ಗಂಗ-ಪೆರ್ಮಾನ-ಡಿ-ಯನ್ನು
ಗಂಗ-ಪೆರ್ಮಾನ-ಡಿಯು
ಗಂಗ-ಪೆರ್ಮ್ಮಾನ-ಡಿಯು
ಗಂಗಪ್ಪಯ್ಯ
ಗಂಗಪ್ರವಾ-ಹೋದಾರ
ಗಂಗ-ಮಂಡಲ
ಗಂಗ-ಮಂಡ-ಲ-ಗ-ಳಿಗೆ
ಗಂಗ-ಮಂಡ-ಲ-ವೆಂದು
ಗಂಗ-ಮಂಡ-ಲಾಧಿಪತ್ಯ-ವನ್ನು
ಗಂಗ-ಮಂಡಳ
ಗಂಗ-ಮಂಡ-ಳ-ವನ್ನು
ಗಂಗ-ಮಂಡ-ಳೇಶ್ವರ
ಗಂಗರ
ಗಂಗ-ರಅ
ಗಂಗ-ರ-ಕಾಲ
ಗಂಗ-ರ-ಕಾಲದ
ಗಂಗ-ರ-ಕಾಲ-ದಲ್ಲಿ
ಗಂಗ-ರ-ಮೇ-ಲಿನ
ಗಂಗ-ರ-ಸ-ರಾಗಿ
ಗಂಗ-ರ-ಸರು
ಗಂಗ-ರಾಜ
ಗಂಗ-ರಾಜಂ
ಗಂಗ-ರಾಜ-ಕುಮಾರ
ಗಂಗ-ರಾಜನ
ಗಂಗ-ರಾಜ-ನ-ಕಾಲ-ದಲ್ಲಿಯೇ
ಗಂಗ-ರಾಜ-ನನ್ನು
ಗಂಗ-ರಾಜ-ನನ್ನೂ
ಗಂಗ-ರಾಜ-ನಾದ
ಗಂಗ-ರಾಜ-ನಿಗೆ
ಗಂಗ-ರಾಜ-ನಿರ-ಬಹುದು
ಗಂಗ-ರಾಜ-ನಿರ-ಬಹು-ದೆಂಬ
ಗಂಗ-ರಾಜನು
ಗಂಗ-ರಾಜ-ನೆಂಬುದು
ಗಂಗ-ರಾಜ-ನೊಡನೆ
ಗಂಗ-ರಾಜನ್ನು
ಗಂಗ-ರಾಜ್ಯ
ಗಂಗ-ರಾಜ್ಯಕ್ಕೆ
ಗಂಗ-ರಾಜ್ಯ-ವನ್ನು
ಗಂಗ-ರಾಷ್ಟ್ರ-ಕೂಟರ
ಗಂಗ-ರಿಂದ
ಗಂಗ-ರಿಗೂ
ಗಂಗ-ರಿಗೆ
ಗಂಗರು
ಗಂಗ-ರೊಂದಿಗೆ
ಗಂಗ-ರೊಡನೆ
ಗಂಗ-ವಂಶದ
ಗಂಗ-ವಂಶ-ದ-ರೆೆಂದೂ
ಗಂಗ-ವಂಶ-ದ-ವ-ನೆಂದು
ಗಂಗ-ವಾಡಿ
ಗಂಗ-ವಾಡಿಗೆ
ಗಂಗ-ವಾಡಿ-ತೊಂಭತ್ತ-ರು-ಸಾಸಿರದ
ಗಂಗ-ವಾಡಿಯ
ಗಂಗ-ವಾಡಿ-ಯನ್ನು
ಗಂಗ-ವಾಡಿ-ಯಲ್ಲಿ
ಗಂಗ-ವಾಡಿ-ಯಲ್ಲೇ
ಗಂಗ-ವಾಡಿ-ಯಿಂದ
ಗಂಗ-ವಾಡಿಯು
ಗಂಗ-ವಾಡಿ-ಯೊಳಕ್ಕೆ
ಗಂಗ-ಸಂದ್ರ
ಗಂಗ-ಸಮುದ್ರ
ಗಂಗ-ಸೇನಾ-ಪತಿಯ
ಗಂಗಾ-ದೇವಿ
ಗಂಗಾ-ಧರ
ಗಂಗಾ-ಧರ-ನೆಂಬ
ಗಂಗಾ-ಧರ-ಪುರ-ವೆಂದು
ಗಂಗಾ-ಧರಯ್ಯನು
ಗಂಗಾ-ಧರೇಶ್ವರ
ಗಂಗಾ-ಧರೇಶ್ವ-ರನ
ಗಂಗಾ-ಧರೇಶ್ವರಸ್ವಾಮಿಯ
ಗಂಗಾ-ನದಿ
ಗಂಗಾ-ನದಿಯ
ಗಂಗಾನ್ವಯ
ಗಂಗಾ-ವನಿ-ರಟ್ಟ
ಗಂಗಾ-ವನಿ-ರಟ್ಟ-ವಾಡಿ
ಗಂಗಾ-ಸಮುದ್ರ
ಗಂಗಿಗವುಂಡನ
ಗಂಗೇ-ಗೌಡ
ಗಂಗೇಶ್ವರ
ಗಂಗೈ-ಕೊಂಡ
ಗಂಗೈ-ಕೊಂಡ-ಚೋಳ-ಪುರಕ್ಕೆ
ಗಂಜಾಂ
ಗಂಜಾಮ್
ಗಂಜಾಮ್ನಲ್ಲಿರುವ
ಗಂಟು
ಗಂಡ
ಗಂಡಃ
ಗಂಡ-ಗೂಳಿ
ಗಂಡ-ನಾದ
ಗಂಡ-ನಾ-ರಾಯಣ
ಗಂಡ-ನಾ-ರಾಯ-ಣ-ಸೆಟ್ಟಿ
ಗಂಡ-ನಾ-ರಾಯ-ಣ-ಸೆಟ್ಟಿಗೆ
ಗಂಡ-ನಾ-ರಾಯ-ಣ-ಸೆಟ್ಟಿಯ
ಗಂಡ-ನಾ-ರಾಯ-ಣ-ಸೆಟ್ಟಿ-ಯರು
ಗಂಡನು
ಗಂಡ-ನೆನಿಸಿದ
ಗಂಡ-ಪೆಂಡಾರ
ಗಂಡ-ಪೆಂಡಾರ-ಗೊಂಡ-ನೆಂದು
ಗಂಡ-ಪೆಂಡಾರ-ವನ್ನು
ಗಂಡ-ಬೇರುಂಡ
ಗಂಡ-ಭೇರುಂಡ
ಗಂಡರ
ಗಂಡ-ರ-ಗಂಡ-ಮುಂಡ
ಗಂಡ-ರ-ನಾಂಪೆ-ವೆಂದು
ಗಂಡ-ರಾಗಿದ್ದ-ರೆಂದು
ಗಂಡ-ರಾಜ
ಗಂಡರುಂ
ಗಂಡ-ವಿಮುಕ್ತ
ಗಂಡಾಂತರ-ದಿಂದ
ಗಂಡಾನೆ
ಗಂಡುಗಲಿ
ಗಂಡು-ಮಕ್ಕಳೂ
ಗಂಧ-ಗೋಡಿ
ಗಂಧನ-ಹಳ್ಳಿ
ಗಂಧ-ವಾರಣ
ಗಂಧ-ವಾರ-ಣ-ನೆಂದು
ಗಂಭೀ-ರಪ್ಪ
ಗಂಭೀರವೂ
ಗಉಡು
ಗಉಡು-ಕುಲ-ತಿಲಕರುಂ
ಗಉಡು-ಗಳು
ಗಜಖೇಟಕ
ಗಜಗಾರ
ಗಜಗಾರ-ಕುಪ್ಪೆ
ಗಜಗಾರ-ಕುಪ್ಪೆಯೇ
ಗಜಗಾರ-ಗುಪ್ಪೆ
ಗಜಗಾರ-ರೆಂದು
ಗಜ-ಪತಿ
ಗಜ-ಪತಿ-ಗಳು
ಗಜ-ಪತಿಯ
ಗಜ-ಪತಿಯು
ಗಜ-ಬೇಂಟೆ-ಕಾರ
ಗಜ-ರಾಜ-ಗಿರಿ
ಗಜವು
ಗಜ-ಸಿಂಹ
ಗಜ-ಸೇನೆ-ಯೊಂದಿಗೆ
ಗಜಾಂಕುಶ
ಗಜಾಖೇಟಕ-ಮತ್ಯುಗ್ರಂ
ಗಜಾರಣ್ಯ
ಗಜೇಂದ್ರ-ಮಂಟಪ-ವನ್ನು
ಗಜೌಘ
ಗಜೌಘ-ಗಂಡ-ಭೇರುಂಡೋ
ಗಟ್ಟಿಗೊಳ್ಳ-ಬೇಕಾಗಿದೆ
ಗಟ್ಟಿ-ಯಾದ
ಗಟ್ಟೇಶ್ವರ
ಗಡ
ಗಡದ
ಗಡದ-ಗಡ್ಡದ
ಗಡಿ
ಗಡಿಗೆ
ಗಡಿದ
ಗಡಿದ-ತಿರು-ಮಲಾ-ಪುರ
ಗಡಿ-ಮೂಡಲು
ಗಡಿಯ
ಗಡಿ-ಯಲ್ಲಿ
ಗಡಿ-ಯಲ್ಲಿ-ರುವ
ಗಡಿ-ಯಲ್ಲೇ
ಗಡಿ-ಯಾಗಿದ್ದಿರ-ಬಹುದು
ಗಡಿ-ಯಿಂದ
ಗಡಿಯು
ಗಡ್ಡದ
ಗಡ್ಡದ-ದಾಡಿಯ
ಗಡ್ಡದ-ದಾಡಿಯ-ಸೋಮೆಯ-ದಂಡ-ನಾಯಕ
ಗಣ-ಕರು
ಗಣ-ಪತಿ
ಗಣ-ಹಳ್ಳಿ
ಗಣ-ಹಳ್ಳಿ-ಯನ್ನು
ಗಣ್ಡಪೆಣ್ಡಾರ
ಗಣ್ದಪೆಣ್ಡಾರ
ಗಣ್ಯ-ದೊರೆ
ಗತಿಸಿ-ದ-ನೆಂದು
ಗತಿಸಿ-ರ-ಬಹುದು
ಗತಿಸಿ-ರ-ಬಹು-ದೆಂಬುದು
ಗದಗ
ಗದಿ-ರದೆ
ಗದ್ದೆ
ಗದ್ದೆ-ಗಳನ್ನು
ಗದ್ದೆ-ಗ-ಳಿಗೆ
ಗದ್ದೆ-ಬೆದ್ದಲು
ಗದ್ದೆ-ಬೆದ್ದ-ಲು-ಗಳನ್ನು
ಗದ್ದೆ-ಯನ್ನು
ಗದ್ದೆ-ಯನ್ನೂ
ಗದ್ಯಾಣ
ಗದ್ಯಾಣ-ಗಳನ್ನು
ಗನ್ನು
ಗಮನ-ವನ್ನು
ಗಮನಾರ್ಹ
ಗಮನಿಸ-ತಕ್ಕದ್ದಾಗಿದೆ
ಗಮನಿಸ-ತಕ್ಕದ್ದಾಗಿವೆ
ಗಮನಿಸ-ಬಹುದು
ಗಮನಿಸಬೇಕಾಗುತ್ತದೆ
ಗಮನಿಸ-ಬೇಕಾದುದು
ಗಮನಿ-ಸ-ಲಾಗಿದೆ
ಗಮನಿಸಿ
ಗಮನಿಸಿ-ಬಹುದು
ಗರುಜೆ
ಗರುಡ
ಗರು-ಡ-ಗಂಬದ
ಗರು-ಡ-ನಂತೆ
ಗರು-ಡ-ನ-ನಪ್ಪಿ
ಗರು-ಡ-ನನ್ನು
ಗರು-ಡ-ನ-ಹಳ್ಳಿ-ಯನ್ನು
ಗರು-ಡ-ನಾದ
ಗರು-ಡ-ರಾಗಿ
ಗರು-ಡ-ರಾಗಿದ್ದ-ವರು
ಗರು-ಡರು
ಗರು-ಡ-ಲೆಂಕ-ರಾಗಿ
ಗರು-ಡ-ಶಾ-ಸನದ
ಗರ್ಬ್ಭ-ಸರ್ಬ್ಬಸ್ವಾಪಹಾರ
ಗರ್ವೋದ್ಧ-ತನಾದ
ಗಳ-ಹನ
ಗಳಿಗೆ
ಗಳಿಸಿ-ಕೊಟ್ಟಿದ್ದಕ್ಕಾಗಿಯೇ
ಗಳಿ-ಸಿದ
ಗಳಿ-ಸಿದಂತೆ
ಗಳಿ-ಸಿದ-ವ-ನಾಗಿ-ರ-ಬೇಕು
ಗಳಿ-ಸಿದ್ದನು
ಗಳು
ಗವರೆ
ಗವರೇಶ್ವರ
ಗವುಡ
ಗವುಡ-ನಾಗಿದ್ದ
ಗವುಡ-ರನ್ನು
ಗವುಡ-ರಿಗೆ
ಗವುಡರು
ಗವುಡ-ರು-ಗಳು
ಗವುಡಿ-ಕೆಗೆ
ಗವುಡಿ-ಕೆ-ಯನ್ನು
ಗವುಡಿ-ಗೆರೆ
ಗವುಡು
ಗವುಡು-ಗಳ
ಗವುಡು-ಗ-ಳಾದ
ಗವುಡು-ಗ-ಳಾದ-ಗಾವುಂಡರು
ಗವುಡು-ಗ-ಳಿಗೆ
ಗವುಡು-ಗಳು
ಗವುಡು-ಗಳೆಂದರೆ
ಗವುಡುಪ್ರಜೆ-ಗಳ
ಗವುಡುಪ್ರಜೆ-ಗಳನ್ನು
ಗವುಡುಪ್ರಜೆ-ಗಳು
ಗಾಂಗೇಯ
ಗಾಂಚ-ನೂ-ರನ್ನು
ಗಾಂನ್ಧ-ವರಾನೆ
ಗಾಣದ
ಗಾಣದೆರೆ-ಗಳನ್ನು
ಗಾಣ-ವನ್ನು
ಗಾಣಿಗನ-ಪುರ
ಗಾತ್ರ
ಗಾತ್ರಂ
ಗಾತ್ರಃ
ಗಾಮಬ್ಬೆ-ಯನ್ನು
ಗಾಮ-ವನ್ನು
ಗಾಮುಂಡ
ಗಾಮುಂಡನ
ಗಾಮುಂಡರ
ಗಾಮುಂಡ-ರನ್ನು
ಗಾಮುಂಡ-ರಾಗಿ-ರುತ್ತಿದ್ದ-ರೆಂದು
ಗಾಮುಂಡ-ರಿಗೆಲ್ಲ
ಗಾಮುಂಡರು
ಗಾಮುಂಡಸ್ವಾಮಿ
ಗಾಮುಂಡಸ್ವಾಮಿ-ಗಳ
ಗಾಮುಂಡಿಯ
ಗಾಮುಂಡಿಯ-ರೆಂದು
ಗಾಮುಣ್ಡ
ಗಾಮುಣ್ಡರು
ಗಾಮುಣ್ಡಸ್ವಾಮಿ-ಗಳ
ಗಾಮುಣ್ಡಸ್ವಾಮಿಯು
ಗಾಯಿ-ಗೋಪಾಳ
ಗಾಯಿಗೋ-ವಳ
ಗಾಳಿಯಂದೊಡಿಂ
ಗಾವುಂಡ
ಗಾವುಂಡ-ತನಕ್ಕೆ
ಗಾವುಂಡನ
ಗಾವುಂಡ-ನದು
ಗಾವುಂಡ-ನನ್ನೇ
ಗಾವುಂಡ-ನಿ-ರುತ್ತಿದ್ದನು
ಗಾವುಂಡನು
ಗಾವುಂಡರ
ಗಾವುಂಡ-ರನ್ನು
ಗಾವುಂಡ-ರನ್ನೇ
ಗಾವುಂಡ-ರಲ್ಲಿ
ಗಾವುಂಡ-ರಾದ
ಗಾವುಂಡ-ರಿಂದ
ಗಾವುಂಡ-ರಿ-ಗಿಂತ
ಗಾವುಂಡ-ರಿಗೆ
ಗಾವುಂಡ-ರಿರುತ್ತಿದ್ದರೂ
ಗಾವುಂಡರು
ಗಾವುಂಡ-ರು-ಗಳ
ಗಾವುಂಡ-ರು-ಗಳು
ಗಾವುಂಡರೇ
ಗಾವುಂಡು-ಗಳ
ಗಾವುಂಡು-ಗಳು
ಗಾವುಡ
ಗಾವುಡಂರು
ಗಾವುಡ-ನಿಗೆ
ಗಾವುಡನು
ಗಿಜಿ-ಹಳ್ಳಿ
ಗಿಡದ
ಗಿರಿ-ದುರ್ಗ-ಮಲ್ಲ
ಗಿರಿ-ಯಣ್ಣ-ನಾಯಕ
ಗಿರಿ-ಯಣ್ಣ-ನಾಯ-ಕರ
ಗಿರಿ-ಯ-ಲಲ್ಲದೆ
ಗಿರಿಶ್ರೇಣಿ-ಗಳ
ಗಿರಿಶ್ರೇಣಿ-ಯಿದೆ
ಗೀತೆ-ಗಳಲ್ಲಿ
ಗು
ಗುಂಡಲು-ಪೇಟೆ
ಗುಂಡೇನ-ಹಳ್ಳಿ
ಗುಂಡ್ಲು-ಪೇಟೆ
ಗುಂಡ್ಲು-ಪೇಟೆ-ಯಲ್ಲಿ
ಗುಂಪನ್ನು
ಗುಂಪಿಗೆ
ಗುಂಪಿತ್ತು
ಗುಂಪಿನ
ಗುಂಪು
ಗುಂಬಜ್
ಗುಂಬಜ್ನಲ್ಲಿ
ಗುಂಬಜ್ನಲ್ಲಿ-ರುವ
ಗುಂಬದ್ಇ-ಅಲಾ
ಗುಂಮಂಣನು
ಗುಜರಾಥಿ
ಗುಜ್ಜಯ-ನಾಯ್ಕನ
ಗುಜ್ಜರ-ರೊಡನೆ
ಗುಜ್ಜಲೆ
ಗುಜ್ಜೆಯ
ಗುಡಿ-ಗಳು
ಗುಡಿಯ
ಗುಡಿ-ಯನ್ನು
ಗುಡಿ-ಯಲ್ಲಿ
ಗುಡಿ-ಯಲ್ಲಿ-ರುವ
ಗುಡಿ-ಯಾಗಿ
ಗುಡಿ-ಯಾಗಿದೆ
ಗುಡೇನ-ಹಳ್ಳಿ
ಗುಡ್ಡ
ಗುಡ್ಡಂ
ಗುಡ್ಡ-ಗಳಿ-ರುವ
ಗುಡ್ಡ-ಗಳು
ಗುಡ್ಡದ
ಗುಡ್ಡಿ-ಯಾಗಿದ್ದಳು
ಗುಡ್ಡು-ಗ-ಳಾದ
ಗುಡ್ಡೆ-ಹಳ್ಳಿ
ಗುಣ
ಗುಣ-ಗಣ-ದಿನಾ-ತನೆಣೆ-ಯಪ್ಪಂನಂ
ಗುಣ-ಗ-ನೆ-ಯಿಂದ
ಗುಣ-ಗಾನ
ಗುಣ-ಗಾನ-ವನ್ನು
ಗುಣ-ದಭಿ
ಗುಣ-ದಿನಾ-ದ-ನದಾವಂ
ಗುಣ-ಸಂಪಂನ್ನ
ಗುಣ-ಸಂಪನ್ನ
ಗುಣ-ಸಂಪನ್ನನುಂ
ಗುಣ-ಸಂಪನ್ನ-ರಪ್ಪ
ಗುಣೋದಗ್ರ
ಗುತ್ತಲ
ಗುತ್ತಲನ್ನು
ಗುತ್ತಲನ್ನೂ
ಗುತ್ತ-ಲಲ್ಲಿ
ಗುತ್ತ-ಲಿನ
ಗುತ್ತಲು
ಗುತ್ತಿ-ಗೆಗೆ
ಗುತ್ತಿಯ
ಗುತ್ತಿಯ-ಗಂಗ
ಗುತ್ತಿಯ-ಗಂಗ-ನೆಂದು
ಗುದ್ಲಿ-ಕಲ್ಲು-ಮಂಠಿ
ಗುಬ್ಬಿಯ
ಗುಬ್ಬಿ-ಹಳ್ಳಿ
ಗುಮ್ಮಟ-ದೇವನು
ಗುಮ್ಮಣ್ಣ
ಗುಮ್ಮನ-ವೃತ್ತಿ
ಗುಮ್ಮನ-ವೃತ್ತಿಯ
ಗುಮ್ಮನ-ಹಳ್ಳಿ
ಗುಮ್ಮನ-ಹಳ್ಳಿ-ಯನ್ನು
ಗುಮ್ಮಳಾ-ಪುರದ
ಗುರಿ-ಕಾರ
ಗುರಿ-ಯನ್ನಾಗಿ-ಸಿ-ಕೊಂಡು
ಗುರು
ಗುರು-ಕವಿಪ್ರಾಜ್ಞೈಃವೃತೇ
ಗುರು-ಗಳ
ಗುರು-ಗಳಾಗಿದ್ದು
ಗುರು-ಗ-ಳಾದ
ಗುರು-ಗ-ಳಾದರು
ಗುರು-ಗ-ಳಿಗೆ
ಗುರು-ಗಳಿರ-ಬಹುದು
ಗುರು-ಗಳು
ಗುರು-ಗಳೊಡನೆ
ಗುರು-ತಿದೆ
ಗುರು-ತಿಸ-ಬಹು-ದಾಗಿದೆ
ಗುರು-ತಿಸ-ಬಹುದು
ಗುರು-ತಿಸ-ಬಹು-ದೆಂದು
ಗುರು-ತಿಸಲಾಗ-ದೆಂದು
ಗುರು-ತಿ-ಸ-ಲಾಗಿದೆ
ಗುರು-ತಿ-ಸಲು
ಗುರು-ತಿಸಲ್ಲ
ಗುರು-ತಿಸಿ
ಗುರು-ತಿ-ಸಿದೆ
ಗುರು-ತಿಸಿದ್ದಾರೆ
ಗುರು-ತಿಸಿ-ರುವ
ಗುರು-ತಿಸು-ವುದು
ಗುರು-ಪರಂಪರೆ-ಯನ್ನು
ಗುರು-ಪೀಠದ
ಗುರು-ಲಿಂಗಜಂಗಮ
ಗುರು-ವಿಗೆ
ಗುರ್ಜ್ಜರ
ಗುಲಬರ್ಗಾ
ಗುಲಾಮ್
ಗುಲ್ಲಯ್ಯನು
ಗುಳಿಯ
ಗುಹೆ
ಗೂಬೆ-ಕಲ್ಲು-ಮಂಠಿ
ಗೂರ್ಜರ
ಗೂರ್ಜ-ರರು
ಗೂಳಿ-ಗೌಡ
ಗೂಳೂ-ರನ್ನು
ಗೂಳೂರು
ಗೂಳೂರೇ
ಗೃಹೋಪ-ಕರ-ಣ-ಗಳು
ಗೆ
ಗೆಜಗಾರ-ಕುಪ್ಪೆ-ಯಾಗಿರ
ಗೆಜ್ಜಗಾರ-ಗುಪ್ಪೆಯ
ಗೆಜ್ಜಗಾರ-ಗುಪ್ಪೆ-ಯಲ್ಲಿಯೂ
ಗೆದೆಗಾಂತು-ಕಂಬಳ
ಗೆದ್ದ
ಗೆದ್ದನು
ಗೆದ್ದ-ನೆಂದು
ಗೆದ್ದಾಗ
ಗೆದ್ದಿರ-ಬಹು-ದೆಂದು
ಗೆದ್ದು
ಗೆದ್ದು-ಕೊಂಡ-ನೆಂದು
ಗೆದ್ದು-ಕೊಂಡರು
ಗೆದ್ದು-ಕೊಟ್ಟ
ಗೆದ್ದು-ಕೊಟ್ಟದ್ದಕ್ಕಾಗಿ
ಗೆದ್ದು-ಕೊಟ್ಟನು
ಗೆದ್ದು-ಕೊಟ್ಟ-ನೆಂದು
ಗೆದ್ದುದು
ಗೆಯಾದ-ವಗಂ
ಗೆಯ್ಸಿದಂ
ಗೆಲವು
ಗೆಲಿದು
ಗೆಲುವು
ಗೆಲ್ದಡೆ
ಗೆಲ್ಲಲು
ಗೆಲ್ಲುತ್ತಿದ್ದ-ನಂತೆ
ಗೇಟ್ನ
ಗೇಣಾಂಕ-ಚಕ್ರೇಶ್ವರ
ಗೇರ-ಹಳ್ಳಿ
ಗೇಹದ
ಗೇಹ-ವನ್ನು
ಗೈಯಲು
ಗೊಂದಲ-ವನ್ನು
ಗೊಂದಲ-ವಾಗು-ವುದು
ಗೊಂದಲ-ವಿದೆ
ಗೊಂಮಟೇಶ್ವರ
ಗೊಡಗೆ
ಗೊತ್ತಾಗುತ್ತದೆ
ಗೊತ್ತಾಗು-ವು-ದಿಲ್ಲ
ಗೊತ್ತಿ-ರುವ
ಗೊತ್ತಿಲ್ಲ
ಗೊಮ್ಮಟಜಿನಸ್ತುತಿ
ಗೊಮ್ಮಟದೆವ-ನಿಗೆ
ಗೊಮ್ಮಟ-ದೇವರ
ಗೊಮ್ಮಟ-ದೇವರು
ಗೊಮ್ಮಟನ-ನನ್ನು
ಗೊಮ್ಮಟನಲ್ತೆ
ಗೊಮ್ಮಟ-ನಿಗೆ
ಗೊಮ್ಮಟೇಶ್ವರ
ಗೊಮ್ಮಟೇಶ್ವ-ರನ
ಗೊಮ್ಮಟೇಶ್ವರ-ನಿಗೆ
ಗೊರ-ಊರು
ಗೊರವಂಕ
ಗೊರೂರು
ಗೊಲ್ಲರಚೆಟ್ಟನ-ಹಳ್ಳಿ
ಗೊಲ್ಲರಚೆಟ್ಟನ-ಹಳ್ಳಿ-ಗಳ
ಗೊಲ್ಲರಚೆಟ್ಟನ-ಹಳ್ಳಿಯ
ಗೊಲ್ಲರ-ಹೊಸ-ಹಳ್ಳಿ
ಗೋಂವಿ-ದರ-ನೆಂದೂ
ಗೋಗ್ರಹಣ-ವಾಗಿ-ರ-ಬಹುದು
ಗೋಜಲು-ಗಳಿವೆ
ಗೋಡೆಯ
ಗೋಣಿ-ಸೋಮ-ನ-ಹಳ್ಳಿ
ಗೋತ್ರಕ್ಕೆ
ಗೋತ್ರ-ಚಿಂತಾಮಣಿ
ಗೋತ್ರದ
ಗೋತ್ರ-ದ-ವನು
ಗೋತ್ರ-ದ-ವರು
ಗೋತ್ರ-ಪವಿತ್ರಂ
ಗೋತ್ರ-ಪವಿತ್ರನುಂ
ಗೋತ್ರ-ವನ್ನು
ಗೋತ್ರ-ಸೂತ್ರ-ಗಳನ್ನು
ಗೋತ್ರೋದಯಃ
ಗೋದಾನ
ಗೋಪಣ್ಣ
ಗೋಪಯ್ಯ-ನನ್ನು
ಗೋಪಾಲ
ಗೋಪಾಲ-ಕೃಷ್ಣ
ಗೋಪಾಲ-ಕೃಷ್ಣ-ದೇವಾಲಯ-ವನ್ನು
ಗೋಪಾಲ-ಕೃಷ್ಣ-ದೇವಾಲಯ-ವಾಗಿದೆ
ಗೋಪಾಲ-ರಾಜ
ಗೋಪಾಲ-ರಾವ್
ಗೋಪಾಲಸ್ವಾಮಿ
ಗೋಪಾಲ್
ಗೋಪಾಳ
ಗೋಪಾಳ-ದೇವ
ಗೋಪಾಳ-ದೇವನು
ಗೋಪಾಳ-ದೇವರ
ಗೋಪಾಳ-ದೇವರು
ಗೋಪಾಳ-ರಾಜನ
ಗೋಪಿ-ನಾಥ-ದೇವರ
ಗೋಪಿ-ನಾಥ-ದೇವ-ರನ್ನು
ಗೋಪಿ-ನಾಥ-ದೇವ-ರಿಗೆ
ಗೋಪಿಯ
ಗೋಪಿಯ-ನಾಯಕ
ಗೋಪಿಯ-ನಾಯ-ಕನ
ಗೋಪೀ-ನಾಥ-ದೇವ-ರಿಗೆ
ಗೋಪುರ-ಗ-ಳಿಗೆ
ಗೋಪುರ-ವತಿ
ಗೋಬ್ರಾಹ್ಮಣಪ್ರಿಯ
ಗೋಬ್ರಾಹ್ಮಣ-ಹಯಧೂಳಿಧೂ-ಸರ
ಗೋಮಹಿಷಿ-ಗಳ
ಗೋಮಹಿಷಿ-ಗಳನ್ನು
ಗೋಯರ
ಗೋಯ-ರನು
ಗೋಯಿಂದರ
ಗೋಯಿಂದ-ರನ
ಗೋಯಿಗ
ಗೋಲೂರು
ಗೋಲ್ಕೊಂಡ-ದ-ವರ
ಗೋಳ-ಗವುಡನು
ಗೋವರ್ಧನ-ಗಿರಿ-ಯಲ್ಲಿ
ಗೋವಿಂದ
ಗೋವಿಂದನ
ಗೋವಿಂದ-ನನ್ನು
ಗೋವಿಂದ-ನ-ಹಳ್ಳಿ
ಗೋವಿಂದ-ನ-ಹಳ್ಳಿಯ
ಗೋವಿಂದನು
ಗೋವಿಂದ-ಮಯ್ಯ
ಗೋವಿಂದ-ಮಯ್ಯನ
ಗೋವಿಂದ-ಮಯ್ಯ-ನೆಂಬ
ಗೋವಿಂದಯ್ಯ
ಗೋವಿಂದಯ್ಯ-ನ-ವ-ರಿಗೆ
ಗೋವಿಂದಯ್ಯನು
ಗೋವಿಂದಯ್ಯಾಖ್ಯ
ಗೋವಿಂದರ
ಗೋವಿಂದ-ರ-ದೇವ
ಗೋವಿಂದ-ರ-ದೇವನ
ಗೋವಿಂದ-ರ-ದೇವ-ನಿಗೂ
ಗೋವಿಂದ-ರ-ದೇವ-ನಿಗೆ
ಗೋವಿಂದ-ರ-ದೇವನು
ಗೋವಿಂದ-ರನ
ಗೋವಿಂದ-ರ-ನಾಗಿ-ರ-ಬಹುದು
ಗೋವಿಂದ-ರ-ನಿರ-ಬಹುದು
ಗೋವಿಂದ-ರನು
ಗೋವಿಂದ-ರನೂ
ಗೋವಿಂದ-ರನೇ
ಗೋವಿಂದ-ರ-ರಕ್ಕಸ-ಗಂಗ
ಗೋವಿಂದ-ರ-ರಲ್ಲಿ
ಗೋವಿಂದ-ರಸ
ಗೋವಿಂದ-ರ-ಸನು
ಗೋವಿಂದ-ರ-ಸ-ನು-ಗೋವಿಂದ-ಮಯ್ಯ
ಗೋವಿಂದ-ರಾಜ
ಗೋವಿಂದ-ರಾಜ-ಗುರು-ಗಳ
ಗೋವಿಂದ-ರಾಜ-ಗುರು-ವಿಗೆ
ಗೋವಿಂದ-ರಾಜ-ಗುರು-ವಿನ
ಗೋವಿಂದ-ರಾಜ-ನನ್ನು
ಗೋವಿಂದ-ರಾಜಯ್ಯನ-ವರ
ಗೋವಿಂದ-ರಾಜರ
ಗೋವಿಂದ-ರಾಜ-ರಿಗೆ
ಗೋವಿಂದ-ವಾಡಿ-ಯನ್ನು
ಗೋವಿ-ದೇವಿ-ಯರು
ಗೋವಿ-ನಾಯಕ
ಗೋವು-ಗಳನ್ನು
ಗೌಂಡ
ಗೌಡ
ಗೌಡ-ಗೆರೆ
ಗೌಡ-ರನ್ನು
ಗೌಡ-ರಿಗೆ
ಗೌಡರು
ಗೌಡರೇ
ಗೌಡಿಕೆ
ಗೌಡಿಕೆಯ
ಗೌಡಿಕೆ-ಯನ್ನು
ಗೌಡಿಕೆಯು
ಗೌಡು-ಗಳ
ಗೌಡು-ಗಳು
ಗೌಡು-ಗೆರೆಯ
ಗೌಡುಪ್ರಜೆ-ಗಳು
ಗೌಣ-ವಾಗಿವೆ
ಗೌತಮ
ಗೌತಮ-ಗೋತ್ರದ
ಗೌತಮಿ
ಗೌತಮೇಶ್ವರ
ಗೌರವ
ಗೌರವಕ್ಕೆ
ಗೌರವ-ದಿಂದ
ಗೌರವ-ಪೂರ್ವ-ಕ-ವಾಗಿ
ಗೌರವ-ವನ್ನು
ಗೌರವವು
ಗೌರವ-ಸೂಚಕ
ಗೌರವ-ಸೂಚಕ-ವಾದ
ಗೌರವ-ಸೂಚಿ
ಗೌರಿ
ಗೌರಿ-ಯ-ಹಳ್ಳಿ
ಗೌರ್ನರ್
ಗ್ರಂಥ
ಗ್ರಂಥ-ಗಳ
ಗ್ರಂಥ-ಗಳನ್ನು
ಗ್ರಂಥ-ಗಳಿಂದ
ಗ್ರಂಥ-ಗಳು
ಗ್ರಂಥ-ಲಿಪಿ
ಗ್ರಂಥ-ಲಿಪಿ-ಕನ್ನಡ
ಗ್ರಹಗ-ತಿ-ಗಳ
ಗ್ರಾಮ
ಗ್ರಾಮಂ
ಗ್ರಾಮಕ್ಕೆ
ಗ್ರಾಮಕ್ಕೆಸ್ಥಳ
ಗ್ರಾಮ-ಗಳ
ಗ್ರಾಮ-ಗಳನ್ನು
ಗ್ರಾಮ-ಗ-ಳನ್ನೇ
ಗ್ರಾಮ-ಗಳಲ್ಲಿ
ಗ್ರಾಮ-ಗಳಾಗಿದ್ದವು
ಗ್ರಾಮ-ಗಳಾಗಿದ್ದ-ವೆಂದು
ಗ್ರಾಮ-ಗಳಾಗಿವೆ
ಗ್ರಾಮ-ಗಳಿಂದ
ಗ್ರಾಮ-ಗ-ಳಿಗೆ
ಗ್ರಾಮ-ಗಳಿದ್ದವು
ಗ್ರಾಮ-ಗಳು
ಗ್ರಾಮ-ಗಳುಳ್ಳ
ಗ್ರಾಮ-ಗಳೂ
ಗ್ರಾಮ-ಗ-ಳೆಂದು
ಗ್ರಾಮ-ಗಳೆಲ್ಲ
ಗ್ರಾಮ-ಗಳೆಲ್ಲವೂ
ಗ್ರಾಮ-ಗಳೇ
ಗ್ರಾಮ-ಗಾ-ಮುಂಡ
ಗ್ರಾಮ-ಗೊಡಗೆ-ಯನ್ನು
ಗ್ರಾಮ-ಗೊಡುಗೆ-ಯಾಗಿ
ಗ್ರಾಮದ
ಗ್ರಾಮ-ದಲ್ಲಿ
ಗ್ರಾಮ-ದಲ್ಲಿದ್ದ
ಗ್ರಾಮ-ದ-ಸೇನ-ಬೋವ
ಗ್ರಾಮ-ದೇವ-ತಾ-ಪುರದ
ಗ್ರಾಮ-ದೇವ-ತೆಯು
ಗ್ರಾಮ-ನಾ-ಮ-ಗಳನ್ನು
ಗ್ರಾಮ-ಪಂಚಕ-ವೆಂದು
ಗ್ರಾಮ-ಮಟ್ಟ-ದಲ್ಲಿ
ಗ್ರಾಮ-ವನು
ಗ್ರಾಮ-ವನ್ನು
ಗ್ರಾಮ-ವನ್ನು-ಧನಗೂರು
ಗ್ರಾಮ-ವಾಗಿತು
ಗ್ರಾಮ-ವಾ-ಗಿತ್ತು
ಗ್ರಾಮ-ವಾಗಿತ್ತೆಂದು
ಗ್ರಾಮ-ವಾಗಿದೆ
ಗ್ರಾಮ-ವಾಗಿದ್ದ
ಗ್ರಾಮ-ವಾಗಿ-ರ-ಬಹುದು
ಗ್ರಾಮ-ವಿತ್ತೆಂದು
ಗ್ರಾಮವು
ಗ್ರಾಮ-ವೆಂದು
ಗ್ರಾಮವೇ
ಗ್ರಾಮ-ಸಂಖ್ಯಾ-ಧಾರಿತ
ಗ್ರಾಮ-ಸಭೆ-ಗಳು
ಗ್ರಾಮ-ಸಭೆಗೆ
ಗ್ರಾಮ-ಸಭೆಯ
ಗ್ರಾಮಾಧಿ
ಗ್ರಾಹ್ಯರಾಹುಃ
ಘಟಕ
ಘಟಕ-ಗಳ
ಘಟಕ-ಗಳಾಗಿದ್ದ
ಘಟಕ-ಗಳಾಗಿದ್ದವು
ಘಟಕ-ಗ-ಳಾದ
ಘಟಕ-ಗಳಿದ್ದವು
ಘಟಕ-ಗಳು
ಘಟಕ-ವಾಗಿ
ಘಟನೆ
ಘಟನೆ-ಗಳನ್ನು
ಘಟನೆ-ಗಳು
ಘಟನೆ-ಯನ್ನು
ಘಟ-ನೆಯು
ಘಟನೆ-ಯೊಂದನ್ನು
ಘಟ-ವೆಂಬ
ಘಟಿಸಿ-ರ-ಬೇಕೆಂದು
ಘಟೆಯಂ
ಘಟ್ಟ
ಘಟ್ಟವು
ಘಣ್ಟಮ್ಮ
ಘಣ್ಟಮ್ಮನು
ಘನ-ಗಿರಿ
ಘನ-ಗಿರಿಗೆ
ಘನಘೋರ
ಘನತೆವೆತ್ತ
ಘನವೃತ್ತ
ಘಮ್ಮ-ನಾಯಕ
ಘರ್ಜಿ-ಸುತ್ತಾ
ಘರ್ಷ-ಣೆ-ಗಳನ್ನು
ಘರ್ಷಣೆ-ಗಳು
ಘರ್ಷ-ಣೆಯ
ಘರ್ಷಣೆ-ಯನ್ನು
ಘರ್ಷ-ಣೆ-ಯಲ್ಲಿ
ಘೃತ-ಪರ್ವ-ತ-ದಾನ-ವನ್ನು
ಘೇಣಾಂಕ
ಘೋಷಿ-ಸ-ಲಾಯಿತು
ಘೋಷಿಸಿ
ಘೋಷಿಸಿ-ಕೊಂಡಿದ್ದನು
ಚ
ಚಂಗ
ಚಂಗ-ಭೂಪನ
ಚಂಗ-ವಾಡಿ
ಚಂಗ-ವಾಡಿ-ಯನ್ನು
ಚಂಗ-ವಾಡಿ-ಯಲ್ಲಿ
ಚಂಗಾಳ್ವನಂ
ಚಂಗಿ-ಕುಳ
ಚಂಗಿ-ಕುಳ-ಕಮಳ
ಚಂಗಿ-ಕುಳ-ಕಮಳ-ಮಾರ್ತ್ರಂಡನತುಳ
ಚಂದಪ್ಪ-ವೊಡೆಯನ
ಚಂದಯ್ಯ
ಚಂದಯ್ಯನು
ಚಂದಲ-ದೇವಿ-ಯನ್ನು
ಚಂದಲ-ದೇವಿ-ಯರು
ಚಂದ-ಹಳ್ಳಿಯ
ಚಂದ-ಹಳ್ಳಿ-ಯನ್ನು
ಚಂದಿಗ
ಚಂದಿಗಾಲು
ಚಂದಿ-ಯಕ್ಕರ
ಚಂದ್ರ
ಚಂದ್ರ-ಗಿರಿ-ಗಳನ್ನು
ಚಂದ್ರ-ಗುಪ್ತ
ಚಂದ್ರ-ನಂತಹ
ಚಂದ್ರ-ನಂದಿಯ
ಚಂದ್ರ-ನನ್ನು
ಚಂದ್ರ-ಮಾಶ್ಚಂದ್ರ-ಕೀರ್ತಿ-ಮಾನ್
ಚಂದ್ರ-ಮೌಳಿ
ಚಂದ್ರ-ಮೌಳಿ-ಯಣ್ಣ
ಚಂದ್ರ-ಮೌಳಿ-ಯಣ್ಣನ
ಚಂದ್ರ-ಮೌಳಿಯು
ಚಂದ್ರ-ಮೌಳೀಶ್ವರ
ಚಂದ್ರ-ಮೌ-ಳೇಶ್ವರ
ಚಂದ್ರ-ರೂಪೋ
ಚಂದ್ರ-ವ-ನದ
ಚಂದ್ರ-ಶೇಖರ
ಚಂದ್ರೊಬ್ಬಲಬ್ಬೆ-ಯನ್ನು
ಚಂನ-ರಾಜ
ಚಂಪೂ-ಕಾವ್ಯವನ್ನಾಗಿ
ಚಂಬಲ್ಲೀ-ಪುರ
ಚಂಬಿನ
ಚಉಗಾವೆಯ
ಚಕಿತ
ಚಕ್ಕೆರೆ
ಚಕ್ತ-ವರ್ತಿ
ಚಕ್ರ
ಚಕ್ರ-ಕೊಳ
ಚಕ್ರ-ಗೊಟ್ಟ
ಚಕ್ರ-ವರ್ತಿ
ಚಕ್ರ-ವರ್ತಿ-ಗಳ
ಚಕ್ರ-ವರ್ತಿ-ಗ-ಳೆಂದು
ಚಕ್ರ-ವರ್ತಿಗೆ
ಚಕ್ರ-ವರ್ತಿಯ
ಚಕ್ರ-ವರ್ತಿ-ಯಾಗಿದ್ದ
ಚಕ್ರ-ವರ್ತಿ-ಯಾದ
ಚಕ್ರ-ವರ್ತಿ-ಯಿಂದ
ಚಕ್ರ-ವರ್ತಿಯು
ಚಕ್ರ-ವರ್ತಿಯೂ
ಚಕ್ರಾಧಿ-ಪತಿ-ಯಾದ
ಚಕ್ರೇಶನ
ಚಕ್ರೇಶ್ವರ
ಚಟಯ-ನಾಯ-ಕನ
ಚಟುವಟಿಕೆ-ಗಳ
ಚಟುವಟಿ-ಕೆ-ಗಳನ್ನು
ಚಟ್ಟಂಗೆರೆ
ಚಟ್ಟಣ-ಕೆರೆ-ಚಟ್ಟಂಗೆರೆ
ಚಟ್ಟಣ-ಕೆರೆ-ಚಟ್ಟಮೆರೆ
ಚಟ್ಟ-ದೇವ
ಚಟ್ಟ-ಪಯ್ಯ
ಚಟ್ಟಮ-ಗೆರೆ
ಚಟ್ಟಮ-ಗೆರೆಯ
ಚಟ್ಟಯ
ಚಟ್ಟಯ್ಯನ-ಹಳ್ಳಿ
ಚಟ್ಟಲ-ದೇವಿ
ಚಟ್ಟೇನ-ಹಳ್ಳಿ-ಯನ್ನು
ಚಟ್ಟೊಡೆಯ
ಚಟ್ಟೊಡೆ-ಯನು
ಚತು-ರಂಗ-ಬಲ-ವನ್ನು
ಚತುರಙ್ಗ
ಚತುರಳಾ-ಗಿದ್ದು
ಚತುರ್ತ್ಥ-ವಂಶರು
ಚತುರ್ತ್ಥ-ವಂಶ-ರೊಳು
ಚತುರ್ಥ
ಚತುರ್ಥ-ಕುಲ-ದ-ವರು
ಚತುರ್ಥ-ಗೋತ್ರ
ಚತುರ್ಥ-ಗೋತ್ರದ
ಚತುರ್ದ್ಧಶ-ವಿದ್ಯಾಸ್ಥಾನಾಧಿಗಮ-ವಿ-ಮಲ-ಮತಿಃ
ಚತುರ್ವೇದಿ
ಚತುರ್ವೇದಿ-ಮಂಗಲದ
ಚತುರ್ವೇದಿ-ಮಂಗಲ-ವೆಂಬ
ಚತುರ್ವ್ವಿಧಾನೂನ-ದಾನ-ವಿನೋದಂ
ಚತುಷಷ್ಟಿಕಳಾಕಳಿತ
ಚತುಸಮಯ-ಸಮುದ್ಧರಣಂ
ಚತು-ಸಮುದ್ರಾಧಿ-ಪತಿ
ಚತುಸ್ಸಮುದ್ರಾಧಿ-ಪತಿ
ಚತುಸ್ಸೀಮೆಗೆ
ಚತುಸ್ಸೀಮೆ-ಯನ್ನು
ಚತುಸ್ಸೀಮೆ-ಯಲಿ
ಚತುಸ್ಸೀಮೆ-ಯಾಗಿ
ಚತುಸ್ಸೀಮೆ-ಯೊಳಗಾದ
ಚತುಸ್ಸೀಮೆ-ಯೊಳಗೆ
ಚತ್ರ-ರಿಗೆ-ಬಹುಶಃ
ಚತ್ರಾಧಿ-ಕಾರಿ
ಚತ್ರಾಧಿ-ಕಾರಿ-ಯಾಗಿದ್ದನು
ಚನ್ದಯ್ಯನು
ಚನ್ನ-ಕೇಶವ
ಚನ್ನ-ಕೇಶವ-ದೇವಾಲಯ-ವನ್ನು
ಚನ್ನ-ಕೇಶವ-ಪುರ-ವೆಂಬ
ಚನ್ನ-ನಂಜ-ರಾಜ-ನಿಗೇ
ಚನ್ನನಂಜ-ರಾಜ-ನೆಂದು
ಚನ್ನ-ಪಟ್ಟಣ
ಚನ್ನ-ಪಟ್ಟಣದ
ಚನ್ನ-ಪಟ್ಟಣ-ರಾಜ್ಯದ
ಚನ್ನ-ಪಟ್ಟಣ-ರಾಜ್ಯ-ವನ್ನು
ಚನ್ನ-ಪಟ್ಟಣ-ವನ್ನು
ಚನ್ನ-ಪಟ್ಟಣಸ್ಥಳಕ್ಕೆ
ಚನ್ನಪ್ಪನ-ದೊಡ್ಡಿ-ಯಲ್ಲಿದೆ
ಚನ್ನಪ್ಪನು
ಚನ್ನಯ್ಯನು
ಚನ್ನ-ರಾಜ
ಚನ್ನ-ರಾಯ-ಪಟ್ಟಣ
ಚಪೂತ
ಚಮದ್ರ-ಮೌಳಿಯ
ಚಮೂ-ಧರ
ಚಮೂಪ
ಚಮೂಪತಿ
ಚಮೂಪ-ತಿಯ
ಚಮೂಪನ
ಚಮೂಪನು
ಚಮೂಪನೋ
ಚಮೂಪ-ರೆಂದೂ
ಚಮ್ಮಾವು-ಗೆಯ
ಚರಣ-ಗ-ಳಿಗೆ
ಚರಣಾಕ್ಯನೆ-ನಲು
ಚರಣಾರವಿಂದ
ಚರಪಿಗೆ
ಚರ-ಮಗೀ-ತೆಯಂತಿದೆ
ಚರಿತಂ
ಚರಿತಃ
ಚರಿತೆ-ಯಲ್ಲಿ
ಚರಿತ್ರೆಯ
ಚರಿತ್ರೆ-ಯನ್ನು
ಚರಿತ್ರೆ-ಯಲ್ಲಿ
ಚರಿತ್ರೆ-ಯಿಂದ
ಚರುಪಿಗೆ
ಚರ್ಪನ್ನು
ಚಲಕ-ದೇವ
ಚಲಕೆ-ಬಲು-ಗಂಡ
ಚಲದ
ಚಲದುತ್ತ-ರಂಗ
ಚಲನವಲನ-ಗಳನ್ನು
ಚಲನೆ
ಚಲಾಯಿ-ಸಲು
ಚಲುಕ್ಯ
ಚಲುಕ್ಯರ
ಚಲುವವ್ವೆ-ಯರ
ಚವರಂ
ಚವರ-ಬಂಬಾಳು
ಚವುಗಾವು-ಗಳ
ಚವುಗಾವು-ಗಳು
ಚವುಗಾವೆ
ಚವುಡಪ್ಪ
ಚವುಡಪ್ಪನ
ಚವುಡಯ್ಯ
ಚವುಡಯ್ಯ-ನ-ಹಳ್ಳಿ
ಚವುಡಿ-ಗೌಡ
ಚವುಡೆ-ಗೊಂಡ-ನಿಗೆ
ಚವುಡೆ-ಗೌಡ-ನಿಗೆ
ಚವುಡೋ-ಜನ
ಚಾಕಲೆ
ಚಾಕ-ಲೆಯ
ಚಾಕಳ-ಹಳ್ಳಿಯ
ಚಾಕೇನ-ಹಳ್ಳಿಯ
ಚಾಗಿ-ಪೆರ್ಮಾನ-ಡಿ-ಗಳ
ಚಾಣಕ್ಯನೆನಿಪ
ಚಾತುರ್ವರ್ಣ-ದಲ್ಲಿ
ಚಾತುರ್ವ್ವೈದ್ಯ
ಚಾಮ
ಚಾಮಂಡ-ಹಳ್ಳಿ
ಚಾಮ-ಡ-ಹಳ್ಳಿ
ಚಾಮಣ್ಣನು
ಚಾಮ-ದೇವ-ನಿರ-ಬಹು-ದೆಂದು
ಚಾಮ-ನೃಪ-ನಿಗೆ
ಚಾಮ-ನೃಪ-ನು-ಬೋಳು-ಚಾಮ-ರಾಜ
ಚಾಮಪ್ಪನು
ಚಾಮಯ್ಯ
ಚಾಮರ
ಚಾಮ-ರಸ
ಚಾಮ-ರ-ಸ-ಗೌಡನು
ಚಾಮ-ರ-ಸನು
ಚಾಮ-ರ-ಸ-ವೊಡೆಯನು
ಚಾಮ-ರ-ಸ-ವೊಡೆಯರ
ಚಾಮ-ರ-ಸ-ವೊಡೆಯ-ರಿಗೆ
ಚಾಮ-ರ-ಸ-ವೊಡೆಯರು
ಚಾಮ-ರ-ಸೊಡೆಯ-ರ-ವರ
ಚಾಮ-ರ-ಸೊಡೆ-ಯರೈಯ-ನ-ವರ
ಚಾಮ-ರಾಜ
ಚಾಮ-ರಾಜನ
ಚಾಮ-ರಾಜ-ನ-ಗರ
ಚಾಮ-ರಾಜ-ನ-ಗ-ರದ
ಚಾಮ-ರಾಜ-ನ-ಗರ-ದಲ್ಲಿ
ಚಾಮ-ರಾಜ-ನನ್ನು-ಹತ್ತ-ನೆಯ
ಚಾಮ-ರಾಜ-ನಿಗೆ
ಚಾಮ-ರಾಜನು
ಚಾಮ-ರಾಜ-ನೆಂದು
ಚಾಮ-ರಾಜ-ನೆಂಬ
ಚಾಮ-ರಾಜನೇ
ಚಾಮ-ರಾಜ-ವೊಡೇರ
ಚಾಮ-ರಾಜೇಂದ್ರ
ಚಾಮ-ರಾಜೊಡೆಯರ
ಚಾಮ-ರಾಜೊಡೆಯ-ರಿಗೆ
ಚಾಮ-ರಾಜೊಡೆಯರು
ಚಾಮ-ಲ-ದೇವಿ
ಚಾಮ-ಲಾ-ಪುರ
ಚಾಮ-ಲಾ-ಪುರದ
ಚಾಮ-ಲಾ-ಪುರ-ವನ್ನು
ಚಾಮ-ಲೆಯು
ಚಾಮವ್ವೆ
ಚಾಮಾಂಬಿಕೆಗೆ
ಚಾಮುಂಡನು
ಚಾಮುಂಡ-ರಾಯ
ಚಾಮುಂಡ-ರಾಯಂ
ಚಾಮುಂಡ-ರಾಯನ
ಚಾಮುಂಡ-ರಾಯ-ನಿಗೆ
ಚಾಮುಂಡ-ರಾಯ-ನಿರ-ಬಹುದು
ಚಾಮುಂಡ-ರಾಯನು
ಚಾಮುಂಡ-ರಾಯ-ನೆಂಬ
ಚಾಮುಂಡ-ರಾಯ-ನೆಂಬು-ವ-ವನು
ಚಾಮುಣ್ಡಯ್ಯನೂ
ಚಾಮುಣ್ಡರಿಬ್ಬರೂ
ಚಾಮೇಂದ್ರ-ನಿಗೆ
ಚಾರ-ಗೌಂಡ
ಚಾರಿತ್ರ
ಚಾರಿತ್ರನುಂ
ಚಾರಿತ್ರ-ಲಕ್ಷ್ಮೀ-ಕರ್ಣ್ನಪೂರಂ
ಚಾರಿತ್ರಿಕ
ಚಾರು-ಪೊನ್ನೇರ
ಚಾಲುಕ್ಯ
ಚಾಲುಕ್ಯರ
ಚಾಲುಕ್ಯ-ರನ್ನು
ಚಾಲುಕ್ಯ-ರಿಗೆ
ಚಾಳಿಸಿ
ಚಾಳುಕ್ಯ
ಚಾಳುಕ್ಯ-ವಿಕ್ರಮ-ಕಾಲದ
ಚಾವ-ಗೌಂಡ
ಚಾವಡಿ
ಚಾವಡಿ-ಗ-ಳನ್ನಾಗಿ
ಚಾವಡಿಯ
ಚಾವಡಿ-ಯಲ್ಲಿ
ಚಾವಡಿ-ಯಾಗಿ-ರ-ಬಹುದು
ಚಾವಡಿಯು
ಚಾವಯ್ಯ
ಚಾವಲ-ದೇವಿ
ಚಾವಾಟ
ಚಾವುಂಡ
ಚಾವುಂಡ-ರಾಯ-ಚಾ-ಮುಂಡ-ರಾಯ
ಚಾವುಂಡ-ರಾಯನ
ಚಾವುಂಡವ್ವೆ-ಯರ
ಚಾವುಣ್ಡ
ಚಾವುಣ್ಡನು
ಚಾವುಣ್ಡ-ನೆಂಬ
ಚಿಂಣ್ನಂ
ಚಿಂತಿಸಿ-ದ-ನೆಂದು
ಚಿಂಮತೂರು
ಚಿಂಮತ್ತೂರು
ಚಿಕ-ಕೇತಯ
ಚಿಕ-ಗವುಡ
ಚಿಕ-ದೇವ-ರಾಜ
ಚಿಕ-ದೇವ-ರಾಜನ
ಚಿಕ-ದೇವ-ರಾಜ-ನ-ವ-ರೆಗೆ
ಚಿಕ-ದೇವ-ರಾಜನು
ಚಿಕ-ದೇವ-ರಾಜ-ವಿಜಯ-ದಲ್ಲಿ
ಚಿಕ-ದೇವ-ರಾಯ-ನೃಪತೀ
ಚಿಕ-ದೇವ-ರಾಯ-ವಂಶಾ-ವಳಿ-ಯಲ್ಲೂ
ಚಿಕ-ರಾಜನ
ಚಿಕವಂಗಲವಕ್ಕೆ
ಚಿಕವಡೆಯ
ಚಿಕವಡೆ-ಯರು
ಚಿಕ-ವೀರ-ಗವುಡ
ಚಿಕ-ಸಿದ್ಧಯ್ಯ-ಗವುಡ
ಚಿಕ್ಕ
ಚಿಕ್ಕ-ಅಬ್ಬಾ-ಗಿಲಿನ
ಚಿಕ್ಕ-ಅ-ರಸಿ-ನ-ಕೆರೆ
ಚಿಕ್ಕ-ಅಲ್ಲಪ್ಪ
ಚಿಕ್ಕ-ಅಲ್ಲಪ್ಪ-ನಾಯಕ
ಚಿಕ್ಕ-ಅಲ್ಲಪ್ಪ-ನಾಯ-ಕನು
ಚಿಕ್ಕ-ಅಲ್ಲಪ್ಪ-ನಾಯ-ಕರ
ಚಿಕ್ಕ-ಕಂನೆಯ
ಚಿಕ್ಕ-ಕಂನೆಯ-ನ-ಹಳ್ಳಿ-ಯನ್ನು
ಚಿಕ್ಕ-ಕಂಪಣ್ಣನು
ಚಿಕ್ಕ-ಕಳಲೆ
ಚಿಕ್ಕ-ಕೇತಣ್ಣ
ಚಿಕ್ಕ-ಕೇತಯ
ಚಿಕ್ಕ-ಕೇತ-ಯನು
ಚಿಕ್ಕ-ಕೇತಯ್ಯ
ಚಿಕ್ಕ-ಕೇತಯ್ಯ-ದಂಡ-ನಾಯಕ
ಚಿಕ್ಕ-ಕೇತಯ್ಯನು
ಚಿಕ್ಕ-ಕೇತಯ್ಯನೇ
ಚಿಕ್ಕ-ಕೇ-ತೆಯ
ಚಿಕ್ಕ-ಕೇ-ತೆಯನು
ಚಿಕ್ಕ-ಕೇ-ತೆಯ್ಯ-ನಾಯ-ಕನು
ಚಿಕ್ಕ-ಗಂಗ-ವಾಡಿ
ಚಿಕ್ಕ-ಗಂಗ-ವಾಡಿಯ
ಚಿಕ್ಕಗ್ರಾಮ-ಗ-ಳಿಗೆ
ಚಿಕ್ಕ-ಜಟಕ
ಚಿಕ್ಕ-ಜಟ್ಟಿಗ-ಹಳ್ಳಿ-ಇಂದಿನ
ಚಿಕ್ಕ-ದೇವ-ರಾಜ
ಚಿಕ್ಕ-ದೇವ-ರಾಜನ
ಚಿಕ್ಕ-ದೇವ-ರಾಜನು
ಚಿಕ್ಕ-ದೇವ-ರಾಜರ
ಚಿಕ್ಕ-ದೇವ-ರಾಜೇಂದ್ರ
ಚಿಕ್ಕ-ದೇವ-ರಾಯ-ನಿ-ಗಿಂತ
ಚಿಕ್ಕ-ದೇವೇಂದ್ರನು
ಚಿಕ್ಕ-ನ-ಹಳ್ಳಿ
ಚಿಕ್ಕ-ನಾಯ-ಕ-ನ-ಪುರ-ದಿಂದ
ಚಿಕ್ಕ-ನಾಯ-ಕ-ನ-ಹಳ್ಳಿ
ಚಿಕ್ಕ-ನಾಯ-ಕರು
ಚಿಕ್ಕಪ್ಪ
ಚಿಕ್ಕಪ್ಪನ
ಚಿಕ್ಕಪ್ಪ-ನ-ಹಳ್ಳಿ
ಚಿಕ್ಕಪ್ಪ-ನಿಂದ
ಚಿಕ್ಕಪ್ರ-ದೇಶದ
ಚಿಕ್ಕ-ಬಯಿಚಪ್ಪ
ಚಿಕ್ಕ-ಬಳ್ಳಿ
ಚಿಕ್ಕ-ಬಾ-ಗಿಲು
ಚಿಕ್ಕ-ಬೆಟ್ಟಕ್ಕೆ
ಚಿಕ್ಕ-ಬೆಟ್ಟದ
ಚಿಕ್ಕ-ಬೆಟ್ಟ-ದಲ್ಲಿ
ಚಿಕ್ಕ-ಬೆಳೂರ
ಚಿಕ್ಕ-ಬೆಳೂರು
ಚಿಕ್ಕಬ್ಬೆ-ಹಳ್ಳಿ
ಚಿಕ್ಕ-ಮಂಟೆ-ಯಕ್ಕೆ
ಚಿಕ್ಕ-ಮಂಟೆಯ-ಚಿಕ್ಕ-ಮಂಡ್ಯ
ಚಿಕ್ಕ-ಮಂಠೆ-ಯ-ಚಿಕ್ಕ-ಮಂಡ್ಯ
ಚಿಕ್ಕ-ಮಂಠೆ-ಯವೇ
ಚಿಕ್ಕ-ಮಂಡ್ಯ
ಚಿಕ್ಕ-ಮ-ಗಳೂರು
ಚಿಕ್ಕ-ಮರಲಿ
ಚಿಕ್ಕ-ಮಲ್ಲಯ್ಯ-ನಾಯ-ಕನ
ಚಿಕ್ಕ-ಮಲ್ಲೆಯ-ನಾಯ-ಕನ
ಚಿಕ್ಕ-ಮಲ್ಲೆಯ-ನಾಯ-ಕನು
ಚಿಕ್ಕ-ಮಲ್ಲೆಯ-ನಾಯ-ಕ-ನೆಂಬ
ಚಿಕ್ಕ-ಮಳಲಿ
ಚಿಕ್ಕ-ಮಾಯಿ
ಚಿಕ್ಕ-ಯಗಟಿ
ಚಿಕ್ಕ-ರಸ
ಚಿಕ್ಕ-ರಾಜ
ಚಿಕ್ಕ-ರಾಮ-ರಾಜನು
ಚಿಕ್ಕ-ರಾಯ
ಚಿಕ್ಕ-ರಾಯನ
ಚಿಕ್ಕ-ರಾಯ-ನಿಗೆ
ಚಿಕ್ಕ-ರಾಯನು
ಚಿಕ್ಕ-ರಾಯನೂ
ಚಿಕ್ಕ-ರಾಯ-ಪಟ್ಟಣ
ಚಿಕ್ಕ-ರಾಯ-ಪಟ್ಟಣ-ವನಾಳುವ
ಚಿಕ್ಕ-ರಾಯ-ಪಟ್ಟಣ-ವನ್ನಾಳುತ್ತಿದ್ದ-ನೆಂದು
ಚಿಕ್ಕ-ರಾಯ-ಪಟ್ಟವೇ
ಚಿಕ್ಕ-ರಾಯ-ಪುರ-ವೆಂಬ
ಚಿಕ್ಕ-ರಾಯಪ್ಪ-ನ-ವ-ರಿಗೆ
ಚಿಕ್ಕ-ರಾಯರು
ಚಿಕ್ಕ-ರಾಯ-ಸಾ-ಗರ
ಚಿಕ್ಕ-ಲಿಂಗನ-ಕೊಪ್ಪಲು
ಚಿಕ್ಕ-ವಡ್ಡರ-ಗುಡಿ
ಚಿಕ್ಕ-ವನ-ಹಳ್ಳಿ
ಚಿಕ್ಕ-ವ-ನಾಗಿ-ರುವಾಗಲೇ
ಚಿಕ್ಕ-ವೊಡೆಯ
ಚಿಕ್ಕ-ವೊಡೆಯ-ನೆಂಬ
ಚಿಕ್ಕ-ವೋಡೆ
ಚಿಕ್ಕ-ಸಾದಿಪ್ಪ
ಚಿಕ್ಕ-ಸಾದಿಪ್ಪ-ನಿಗೆ
ಚಿಕ್ಕ-ಸಾದಿಪ್ಪನು
ಚಿಕ್ಕ-ಸಾದಿ-ಯಪ್ಪ-ನಿಗೆ
ಚಿಕ್ಕ-ಸಾದಿಯಪ್ಪನೂ
ಚಿಕ್ಕ-ಹಡೆ-ವಳ್ಳ
ಚಿಕ್ಕ-ಹರಿ-ಯಲೆ
ಚಿಕ್ಕ-ಹೊಸ-ಹಳ್ಳಿ
ಚಿಕ್ಕಾಡೆ
ಚಿಕ್ಕೆಯ-ನಾಯಕ
ಚಿಕ್ಕೆ-ಹಳ್ಳಿ-ಮಂಡ್ಯ
ಚಿಕ್ಕೊಡೆಯ-ನಿಗೆ
ಚಿಕ್ಕೊಡೆ-ಯರ
ಚಿಕ್ಕೊಲೆ
ಚಿಗುಡ-ಹಳ್ಳಿ
ಚಿಗುಡ-ಹಳ್ಳಿ-ತಿಗಡ-ಹಳ್ಳಿ
ಚಿಗುಲಿ-ಹಳ್ಳಿ
ಚಿಟ್ಟನ-ಹಳ್ಳಿ
ಚಿಟ್ಟನ-ಹಳ್ಳಿಯ
ಚಿಣ್ಣ
ಚಿಣ್ಣ-ನನ್ನು
ಚಿಣ್ಣನು
ಚಿಣ್ಣಯ್ಯ
ಚಿಣ್ಣಯ್ಯನು
ಚಿಣ್ನಂ
ಚಿಣ್ನಯ
ಚಿಣ್ನಯ-ನಾಡ
ಚಿತ್ತಮಂ
ಚಿತ್ತ-ವಲ್ಲಭೆ-ಯಾದ
ಚಿತ್ತೈಸಿ
ಚಿತ್ರ-ಕಾರರೂ
ಚಿತ್ರಣ
ಚಿತ್ರಣ-ವನ್ನು
ಚಿತ್ರ-ದುರ್ಗ
ಚಿತ್ರ-ದುರ್ಗ-ದಲ್ಲಿ
ಚಿತ್ರ-ದುರ್ಗವು
ಚಿತ್ರಭಾನು
ಚಿತ್ರಮ-ಕೊಂಡ-ನಾಯ-ಕನ
ಚಿತ್ರ-ವನ್ನು
ಚಿತ್ರ-ವಿದೆ
ಚಿದಾನಂದ
ಚಿದಾನಂದ-ಮೂರ್ತಿ-ಯ-ವರ
ಚಿದಾನಂದ-ಮೂರ್ತಿ-ಯ-ವರು
ಚಿನ-ಕುರಳಿ
ಚಿನ-ಕುರ-ಳಿಗೆ
ಚಿನ-ಕುರ-ಳಿ-ಬೆಟ್ಟ
ಚಿನ-ಕುರ-ಳಿಯ
ಚಿನ-ಕುರ-ಳಿ-ಯಲ್ಲಿ
ಚಿನ್ನ-ದೇವ-ಚೋಡ-ಮಹಾ-ಅ-ರಸನು
ಚಿನ್ನಮ
ಚಿನ್ನಾ-ದೇವಿ-ಪುರ
ಚಿನ್ನಾ-ದೇವಿಯ
ಚಿಮತೂರ-ಕಲ್ಲ
ಚಿಮತೂರಬೆ-ಮತೂರ
ಚಿಮತೂರಿನ
ಚಿಮತ್ತೂರ-ಕಲ್ಲು
ಚಿಮ್ಮತ್ತನ-ಕಲ್ಲು
ಚಿಮ್ಮತ್ತನೂರಿನ
ಚಿಮ್ಮತ್ತೂರ-ಕಲ್ಲ
ಚಿಮ್ಮತ್ತೂರು
ಚಿರತೆ
ಚಿರತೆ-ಗಳು
ಚಿರತೆ-ಗಳೂ
ಚಿರಸ್ಥಾಯಿ-ಯಾಗಿ
ಚಿಲ-ಕುರ್ಲಿ-ಚಿನ-ಕುರಳಿ
ಚಿಹ್ನೆ-ಯಾಗಿ
ಚಿಹ್ನೆ-ಯಾದ
ಚೀಣ್ಯ
ಚುಂಚನ
ಚುಂಚನ-ಕೋಟೆ
ಚುಂಚನ-ಕೋಟೆಯ
ಚುಂಚನ-ಗಿರಿ
ಚುಂಚನ-ಗಿರಿಯ
ಚುಂಚನ-ಭಯಿರವ
ಚುಂಚನ-ಹಳ್ಳಿ
ಚುಂಚನ-ಹಳ್ಳಿ-ಯನ್ನು
ಚೂಡಾಮಣಿ
ಚೂರ್ಣೀ-ಕರಿ-ಸಿದ
ಚೆಂಗ-ವಾಡಿಯ
ಚೆಂಗ-ವಾಡಿ-ಯನ್ನು
ಚೆಂಗ-ವಾಡಿ-ಯಲ್ಲಿ
ಚೆಂಗಾಳ್ವನಂ
ಚೆಂಗಾಳ್ವ-ನಾಗಿ-ರ-ಬಹು-ದೆಂದು
ಚೆಂಗಾಳ್ವನು
ಚೆಂಗಾಳ್ವರ
ಚೆಂಗಾಳ್ವರನ್ನು
ಚೆಂಗಾಳ್ವರು
ಚೆಂಗಿರಿ-ಯನ್ನು
ಚೆಂಡಾಡಿದ-ನೆಂದು
ಚೆಂದಾ-ಪುರ-ವೆಂಬ
ಚೆಂಬಿನ
ಚೆಂಬೊಂಗಳಂ
ಚೆಟ್ಟ-ಹಳ್ಳಿ
ಚೆಟ್ಟಿಯ-ರನ್ನು
ಚೆನ್ನ-ಕೇಶವ
ಚೆನ್ನ-ಕೇಶವ-ದೇವರ
ಚೆನ್ನ-ಕೇಶವ-ದೇವಾಲಯದ
ಚೆನ್ನ-ಕೇಶವ-ನಿಗೆ
ಚೆನ್ನ-ಕೇಶವ-ಪದಾಂಭೋಜ-ಕಮಳಿನೀಕಳಹಂಸಭಿನವಪ್ರಹ-ರಾಜ
ಚೆನ್ನ-ಕೇಶವ-ಪುರ-ವೆಂಬ
ಚೆನ್ನಕ್ಕ
ಚೆನ್ನ-ದೀಕ್ಷಿತ-ನಿಗೆ
ಚೆನ್ನ-ದೇವ-ಚೋಡ-ಮಹಾ-ಅ-ರಸನು
ಚೆನ್ನ-ದೇವ-ಚೋಡ-ಮಹಾ-ಅ-ರಸು
ಚೆನ್ನ-ನಂಜ-ರಾಜ
ಚೆನ್ನಯ್ಯ-ನೆಂಬು-ವ-ವನ
ಚೆನ್ನ-ರಸ
ಚೆನ್ನ-ರಾಜಯ್ಯನು
ಚೆನ್ನ-ರಾಯ-ಪಟ್ಟಣ
ಚೆನ್ನ-ರಾಯ್ಯನ-ವರಿಗೆ
ಚೆನ್ನಾ-ದೇವಿ-ಪುರ-ವೆಂಬ
ಚೆನ್ನಿ-ಸೆಟ್ಟಿ-ಯರ
ಚೆಲು-ಪಿಳ್ಳೆ-ದೇವರ
ಚೆಲುವ-ದೇವಾಂಬುಧಿ
ಚೆಲುವ-ನಾ-ರಾಯಣ
ಚೆಲುವ-ನಾ-ರಾಯ-ಣನ
ಚೆಲು-ವನಾ-ರಾಯಸ್ವಾಮಿ
ಚೆಲುವ-ಪಿಳ್ಳೆ
ಚೆಲುವ-ಪಿಳ್ಳೆ-ದೇವ-ರಿಗೆ
ಚೆಲುವ-ಪಿಳ್ಳೆ-ರಾಯರ
ಚೆಲುವ-ಪಿಳ್ಳೆ-ರಾಯ-ರಿಗೆ
ಚೆಲುವವ್ವೆ-ಯರ
ಚೆಲುವಾಂಬಾ
ಚೆಲ್ಲ-ಬಹು-ದಾಗಿದೆ
ಚೆಲ್ಲುತ್ತದೆ
ಚೆಲ್ವಡ-ರಾಯ-ನೆಂಬ
ಚೆಲ್ವಾಜ-ಮಾಂಬ
ಚೆಲ್ವಾಜ-ಮಾಂಬೆಯ
ಚೇರ
ಚೇರ-ಮನ-ಹಳ್ಳಿ-ಯನ್ನು
ಚೈತ್ಯಾಲಯ-ವನ್ನು
ಚೈತ್ಯಾಲಯ-ವೆಂದೂ
ಚೊಕ್ಕ-ಜಿನಾಲ-ಯಕ್ಕೆ
ಚೊಕ್ಕಣ್ಣನ
ಚೊಕ್ಕ-ನಾಥನ
ಚೊಟ್ಟನ-ಹಳ್ಳಿ
ಚೊತ್ತರಳಿ
ಚೊಳ-ನಿಗೆ
ಚೋಕಲ
ಚೋಕಲ-ದೇವಿ
ಚೋಕಲ-ದೇ-ವಿಗೆ
ಚೋಕವ್ವೆ
ಚೋಳ
ಚೋಳ-ಕೊಟ್ಟ
ಚೋಳ-ಗಂಗ
ಚೋಳ-ಗವುಂಡನು
ಚೋಳ-ಗವುಡ
ಚೋಳ-ಗೌಂಡನು
ಚೋಳ-ತುರು-ನಾಡನ್ನು
ಚೋಳ-ದೇಶ-ದಲ್ಲಿ
ಚೋಳನ
ಚೋಳ-ನ-ಕೋಟೆ
ಚೋಳ-ನನ್ನು
ಚೋಳ-ನ-ರಾಜಾದಿತ್ಯ
ಚೋಳ-ನಾಡನ್ನಾಗಿ
ಚೋಳ-ನಾಡನ್ನು
ಚೋಳ-ನಿಗೆ
ಚೋಳನು
ಚೋಳನೆ
ಚೋಳ-ನೆಂಬ
ಚೋಳ-ಪರಾಂತಕ
ಚೋಳ-ಪರಾಂತಕನ
ಚೋಳ-ಪುರದ
ಚೋಳಪ್ಪಯ್ಯನ
ಚೋಳ-ಬಲ-ವನ್ನು
ಚೋಳ-ಭೂಮಿಯ
ಚೋಳ-ಮಂಡ-ಲದ
ಚೋಳ-ಮಹಾ-ಅ-ರಸ
ಚೋಳ-ಮಹಾ-ಅ-ರಸು-ಗಳ
ಚೋಳರ
ಚೋಳ-ರ-ಕಾಲದ
ಚೋಳ-ರನ್ನು
ಚೋಳ-ರಾಜ-ನನ್ನು
ಚೋಳ-ರಾಜನೇ
ಚೋಳ-ರಾಜೇಂದ್ರನು
ಚೋಳ-ರಾಜ್ಯ
ಚೋಳ-ರಾಜ್ಯಕ್ಕೆ
ಚೋಳ-ರಾಜ್ಯದ
ಚೋಳ-ರಾಜ್ಯ-ದಲ್ಲಿ
ಚೋಳ-ರಾಯ
ಚೋಳ-ರಾಯಸ್ಥಾಪನಾಚಾರ್ಯ್ಯ
ಚೋಳ-ರಿಂದ
ಚೋಳ-ರಿಗೂ
ಚೋಳರು
ಚೋಳ-ರೊಡನೆ
ಚೋಳ-ಲಾ-ಳಾದಿ-ಗಳು
ಚೋಳ-ಸಿಂಹಾಸ-ನದ
ಚೋಳ-ಸಿಂಹಾಸ-ನ-ದಲ್ಲಿ
ಚೋಳ-ಸೇನೆ-ಯನ್ನು
ಚೋಳಿ
ಚೋಳಿ-ಕ-ರನ್ನು
ಚೋಳೆ-ಯನ-ಹಳ್ಳಿ
ಚೋಳೋರ್ವ್ವಿಯಂ
ಚೌಕಟ್ಟನ್ನು
ಚೌಗಾವೆಯ
ಚೌಡಪ್ಪನ
ಚೌಡ-ಹಳ್ಳಿ-ಯಲ್ಲಿರುವ
ಚೌತ
ಚೌತನ್ನು
ಚೌತ-ವನ್ನು
ಚ್ಚರಿ-ಸುವ-ರೆಲ್ಲರುಂ
ಛತ್ರ
ಛತ್ರ-ಗಳ
ಛತ್ರ-ಗಳನ್ನು
ಛತ್ರ-ಛಾಯೆಯಿಂ
ಛತ್ರ-ವನ್ನು
ಛಾಯಾಚಿತ್ರ-ಗಳನ್ನು
ಜಂಗಮರು
ಜಂಗಮೊಡೆ-ಯರ
ಜಂಗುಳಿ
ಜಂಗುಳಿ-ಯ-ವರು
ಜಂಟಿ
ಜಂಟಿ-ಯಾಗಿ
ಜಕ-ಗೌಡನ
ಜಕ್ಕಣಬ್ಬೆ
ಜಕ್ಕಣಬ್ಬೆಯ
ಜಕ್ಕಣಬ್ಬೆ-ಯರ
ಜಕ್ಕಣ್ಣ-ನಾಯ-ಕ-ನಿಗೆ
ಜಕ್ಕನ-ಹಳ್ಳಿ
ಜಕ್ಕಯ್ಯ
ಜಕ್ಕಲ-ದೇವಿ
ಜಕ್ಕಲೆಗೆ
ಜಕ್ಕಿ-ಕಟ್ಟೆ
ಜಕ್ಕಿಮವ್ವೆಯು
ಜಗ
ಜಗ-ಕಾರಿಗ
ಜಗತೀ-ಸಾಮ್ರಾಜ್ಯ-ದೀಕ್ಷಾಂ
ಜಗದಾಳಮೊ-ನೆ-ಯೊಳು
ಜಗದು-ಗೋಪಾಳ
ಜಗದೇಕ
ಜಗದೇಕ-ಮಲ್ಲ
ಜಗದೇಕ-ಮಲ್ಲನು
ಜಗದೇಕ-ವೀ-ರನು
ಜಗದೇಕ-ವೊಡೆಯನ
ಜಗ-ದೇವ-ಕಅ-ರಾಯ
ಜಗ-ದೇವನ
ಜಗ-ದೇವ-ರಾಯ
ಜಗ-ದೇವ-ರಾಯನ
ಜಗ-ದೇವ-ರಾಯ-ನಿಗೆ
ಜಗ-ದೇವ-ರಾಯನು
ಜಗ-ದೇವ-ರಾಯ-ನೆಂಬ
ಜಗನ್ನಾಥ-ವಿಜಯ-ವೆಂಬ
ಜಗಳ-ದಲ್ಲಿ
ಜಗಳವು
ಜಗಳೂರು
ಜಗ್ಗ-ಗವುಡನ
ಜಟಕ
ಜಟಾ-ವರ್ಮ-ಸುಂದರ-ಪಾಂಡ್ಯನ
ಜಟ್ಟಿಗ-ವನ್ನು
ಜಡೆಯ
ಜಡೆ-ಯದ
ಜತ್ತ
ಜನ
ಜನಕ
ಜನ-ಕನ
ಜನ-ಗನ್ನು-ತನಾ
ಜನ-ಗಳಿಗೋಸ್ಕರ
ಜನ-ಜನಿತ-ವಾದ
ಜನ-ಜೀ-ವನ
ಜನ-ಜೀ-ವನವು
ಜನ-ತಾಧಾ-ರನು-ದಾ-ರನನ್ಯ
ಜನ-ತಾಪ್ರಿಯೇಣ
ಜನ-ತೆ-ಯನ್ನು
ಜನನ
ಜನ-ನದ
ಜನ-ನ-ವಾಗಿದೆ
ಜನನಿ
ಜನ-ನಿಯ
ಜನ-ಪದ
ಜನಪ್ರಿಯ
ಜನರ
ಜನ-ರಲ್
ಜನ-ರಲ್ಲಿ
ಜನ-ರಿಂದ
ಜನ-ರಿಗೆ
ಜನರು
ಜನರೂ
ಜನ-ವರಿ
ಜನ-ಸಂಖ್ಯೆಯು
ಜನ-ಹಳ್ಳಿಯ
ಜನಾಂಗ-ದಲ್ಲಿ
ಜನಾಂಗ-ದ-ವರು
ಜನಾರ್ದನ
ಜನಾರ್ದನ-ದೇವರ
ಜನಿಸಿ
ಜನಿ-ಸಿದ
ಜನಿಸಿ-ದನು
ಜನಿಸಿ-ದ-ನೆಂದು
ಜನಿಸಿ-ಬಂದು
ಜನಿ-ಸುತ್ತೇನೆ
ಜನೋಪ-ಕಾರಿ
ಜನ್ನನ
ಜನ್ಮಕ್ಷೇತ್ರ-ವನ್ನಾಗಿ
ಜನ್ಮತಪಃ
ಜನ್ಮೋತ್ಸವ-ದಂದು
ಜಮರಂಣನು
ಜಮರಣ್ಣನು
ಜಮೀನಿನ
ಜಯ
ಜಯ-ಜೀಯ-ವರ್ದ್ಧನ-ಕರಂ
ಜಯತಿ
ಜಯತ್ಯಸೌ
ಜಯದ
ಜಯದುತ್ತ-ರಂಗ
ಜಯ-ಪತ್ರದ
ಜಯಮ್ಮನ
ಜಯರೇಖೆ-ಯನ್ನು
ಜಯ-ಲಕ್ಷ್ಮಿ-ಗಿತ್ತು
ಜಯ-ಲಕ್ಷ್ಮಿಗೆ
ಜಯ-ವನ್ನು
ಜಯ-ಸಿಂಹ-ನನ್ನು
ಜಯಸ್ಥಂಭವ-ನೆತ್ತಿಸಿ
ಜಯಾಂಗ-ನಾ-ವಲ್ಲಭಂ
ಜಯಿಸಿ
ಜಯಿಸಿ-ಕೊಳ್ಳಿ
ಜಯಿ-ಸಿದ
ಜಯಿಸಿ-ದ-ನಂತೆ
ಜಯಿಸಿ-ದ-ನೆಂದು
ಜಯಿ-ಸಿದ್ದ
ಜರಯ್ಯಂ
ಜಲ
ಜಲಕ್ರೀಡೆ-ಯಾಡುತ್ತಿದ್ದ-ನೆಂದು
ಜಲ-ದುರ್ಗ
ಜಲಧಾಮ
ಜಲಾಶ-ಯದ
ಜಲಾಶ-ಯವು
ಜಲ್ಮೋತ್ಸವ
ಜಲ್ಲ
ಜಳಧಿ-ಯೊಳ್
ಜವನಿಕೆ
ಜವನಿಕೆ-ನಾ-ರಾಯಣ
ಜವನಿಕೆ-ಯೊಡಲಿರ್ವ್ವ-ಲದ
ಜವನೊಡ-ನಾದಡಂ
ಜವಾದಿ
ಜವಾದಿ-ಕೋಳಾಹಳ
ಜವಾಬ್ದಾರಿ-ಯನ್ನು
ಜವಾಬ್ದಾರಿ-ಯುತ-ವಾದ
ಜಸ-ವತರ
ಜಸಹಿತ-ದೇವ
ಜಹಾಂಗೀರ್
ಜಾಗ
ಜಾಗ-ದಲ್ಲಿ
ಜಾಗ-ನ-ಕೆರೆ
ಜಾಗ-ವನ್ನು
ಜಾಗ-ವಿದ್ದು
ಜಾಗವು
ಜಾತಿ
ಜಾತಿಯ
ಜಾತ್ಯಶ್ವದಿಂ
ಜಾತ್ರೆ
ಜಾನ-ಪದ
ಜಾನ-ಪದೀಯ
ಜಾರಿಗೆ
ಜಾರಿಯಲ್ಲಿತ್ತು
ಜಾಸ್ತಿ
ಜಾಸ್ತಿ-ಯಾಗಿ
ಜಾಸ್ತಿ-ಯಾಗಿದ್ದ
ಜಾಸ್ತಿ-ಯಾದಾಗ
ಜಿಂಕೆ
ಜಿಂಕೆ-ಯಂತೆ
ಜಿಂಜಿಯ
ಜಿಆರ್ಕುಪ್ಪುಸ್ವಾಮಿ
ಜಿಎಸ್ದೀಕ್ಷಿತ್
ಜಿಎಸ್ದೀಕ್ಷಿತ್ರ-ವರು
ಜಿತ-ಪಾರ್ಶ್ವಂ
ಜಿನಗನ್ಧೋದಕ
ಜಿನ-ಚಂದ್ರ-ಪಂಡಿ-ತನಿಗೆ
ಜಿನ-ದೇವನ
ಜಿನ-ದೇವ-ನ-ನಜಿತ-ಸೇನ-ಮುನಿ-ಪ-ವರ
ಜಿನದೊಣೆಲಕ್ಕದೊ-ಣೆಯ
ಜಿನ-ಧರ್ಮಾಗ್ರಣಿ
ಜಿನ-ನಾಥ-ಪುರ
ಜಿನ-ನಾಥ-ಪುರ-ದಲ್ಲಿ
ಜಿನ-ನಾಥ-ಪುರ-ವನ್ನು
ಜಿನ-ಪಾದ-ಪಂಕಜ
ಜಿನ-ಪಾರ್ಶ್ವ-ದೇವರ
ಜಿನ-ಪಾರ್ಶ್ವ-ನಾಥ
ಜಿನ-ಪೂಜೆಗೆ
ಜಿನಭಕ್ತ
ಜಿನಮುಖ-ಚಂದ್ರ-ವಾಕ್ಚಂದ್ರಿಕಾಚಕೋರಂ
ಜಿನ-ಮುನಿ
ಜಿನ-ಮುನಿ-ಯಲ್ಲಿ
ಜಿನ-ರಾಜ-ರಾಜತ್ಪೂಜಾ-ಪುರಂದರಂ
ಜಿನ-ಶಾ-ಸನ-ರಕ್ಷಾಮಣಿ
ಜಿನಸಮಯ
ಜಿನಾರ್ಚ-ನಲುಬ್ಧಂ
ಜಿನಾರ್ಚನೆಗೆ
ಜಿನಾಲಯ
ಜಿನಾಲ-ಯಕ್ಕೆ
ಜಿನಾಲ-ಯದ
ಜಿನಾಲಯ-ವನ್ನು
ಜಿನಾಲಯ-ವನ್ನೂ
ಜಿನಾಲಯ-ವೆಂದು
ಜಿಫ್ರಿ
ಜಿಲೆ-ಯಲ್ಲಿದ್ದು
ಜಿಲ್ಲಾ
ಜಿಲ್ಲಾ-ವಾರು
ಜಿಲ್ಲೆ
ಜಿಲ್ಲೆ-ಗಳ
ಜಿಲ್ಲೆ-ಗಳನ್ನು
ಜಿಲ್ಲೆ-ಗಳಲ್ಲಿ
ಜಿಲ್ಲೆ-ಗಳಲ್ಲಿ-ರುವ
ಜಿಲ್ಲೆ-ಗಳಾಗಿ
ಜಿಲ್ಲೆ-ಗ-ಳಿಗೆ
ಜಿಲ್ಲೆ-ಗಳಿದ್ದವು
ಜಿಲ್ಲೆಗೆ
ಜಿಲ್ಲೆಯ
ಜಿಲ್ಲೆ-ಯನ್ನು
ಜಿಲ್ಲೆ-ಯಲ್ಲಿ
ಜಿಲ್ಲೆ-ಯಲ್ಲಿಯೂ
ಜಿಲ್ಲೆ-ಯಲ್ಲಿಯೇ
ಜಿಲ್ಲೆ-ಯಲ್ಲಿ-ರುವ
ಜಿಲ್ಲೆ-ಯಲ್ಲಿವೆ
ಜಿಲ್ಲೆ-ಯಲ್ಲೂ
ಜಿಲ್ಲೆ-ಯಾಗಿದೆ
ಜಿಲ್ಲೆ-ಯಿಂದ
ಜಿಲ್ಲೆಯು
ಜೀಗುಂಡಿ-ಪಟ್ಟಣ
ಜೀಗುಂಡಿ-ಪಟ್ಟಣ-ದಲ್ಲೂ
ಜೀಯ
ಜೀಯ-ನಿಗೆ
ಜೀಯ-ನೆಂಬ
ಜೀಯಾತು
ಜೀಯಾದಾಚ್ಚಂದ್ರ-ತಾರಕಂ
ಜೀರ-ಹಳ್ಳಿ
ಜೀರ-ಹಳ್ಳಿ-ಗಳು-ಅಂಬಲ-ಜೀರ-ಹಳ್ಳಿ
ಜೀರಿಗೆ-ಯೊಕ್ಕಲಿಕ್ಕಿ
ಜೀರ್ಣ-ವಾಗಿದೆ
ಜೀರ್ಣ-ವಾಗಿದ್ದಾಗ
ಜೀರ್ಣ-ವಾಗಿ-ರಲು
ಜೀರ್ಣ-ವಾಗಿವೆ
ಜೀರ್ಣ-ವಾದ
ಜೀರ್ಣೋ
ಜೀರ್ಣೋದ್ಧಾರ
ಜೀರ್ಣೋದ್ಧಾರಕ
ಜೀರ್ಣೋದ್ಧಾರಕ್ಕೆ
ಜೀರ್ಣೋದ್ಧಾರ-ವನ್ನು
ಜೀವಂತ-ವಾಗಿ
ಜೀವಿತ
ಜೀವಿತ-ವನ್ನು
ಜೀವಿತ-ವಾಗಿ
ಜೀವಿತವೂ
ಜೀವಿತಾವಧಿಯ
ಜೀವಿ-ಸಿದ್ದನು
ಜೀವಿ-ಸಿದ್ದ-ನೆಂದು
ಜೀವಿಸಿ-ರುವ-ವ-ರೆಗೆ
ಜುಂಜಾ-ಪುರ
ಜುಮ್ಮಾ-ಮಸೀದಿ
ಜುಲೈ
ಜೂನ್
ಜೂನ್ಜುಲೈ
ಜೆಂನಿಗೆ
ಜೆಎಂನಾಗಯ್ಯ-ನ-ವರು
ಜೆಟ್ಟಿಗದ
ಜೆಟ್ಟಿಗ-ದಲ್ಲಿ
ಜೆಡಲ
ಜೇನು-ಗುಡ್ಡ
ಜೈತಾಜಿ-ಕಾಟ್ಕರ್
ಜೈತಾಜಿಯ-ರನ್ನು
ಜೈತುಗಿಯ
ಜೈದೇವ
ಜೈನ
ಜೈನ-ಕೇಂದ್ರ
ಜೈನ-ತೀರ್ಥ-ಗ-ಳಿಗೆ
ಜೈನ-ಧರ್ಮಕ್ಕೆ
ಜೈನ-ಧರ್ಮದ
ಜೈನ-ಧರ್ಮ-ನಿರ್ಮಳಾಂಬರ
ಜೈನ-ಧರ್ಮ-ವನ್ನು
ಜೈನ-ಪುರಾಣ-ಗಳಲ್ಲಿ
ಜೈನ-ಬ-ಸದಿ-ಗಳನ್ನು
ಜೈನ-ಬ-ಸದಿ-ಗ-ಳಿಗೆ
ಜೈನ-ಬ-ಸದಿಗೆ
ಜೈನ-ಬ-ಸದಿಯು
ಜೈನ-ಬಸ್ತಿಯ
ಜೈನ-ಮತಾವಲಂಬಿ-ಯಾಗಿದ್ದಳು
ಜೈನ-ಮುನಿ
ಜೈನ-ಮುನಿ-ಗಳು
ಜೈನ-ಯತಿ-ಗ-ಳಿಗೆ
ಜೈನರ
ಜೈನ-ವೈಷ್ಣವ
ಜೊತಗೆ
ಜೊತೆ
ಜೊತೆ-ಗಿದ್ದು
ಜೊತೆ-ಗೂಡಿ
ಜೊತೆಗೆ
ಜೊತೆ-ಗೆ-ಹೋಗಿ
ಜೊತೆಗೇ
ಜೊತೆ-ಯಲ್ಲಿ
ಜೊತೆ-ಯಲ್ಲಿದ್ದ
ಜೊತೆ-ಯಲ್ಲಿಯೇ
ಜೊತೆ-ಯಾಗಿ
ಜೊರೆಗ
ಜೊಸೆಫ್
ಜೋಗುಂಡಯ್ಯ
ಜೋಗುಣ್ಡಯ್ಯ-ನಿಗೆ
ಜೋಡು
ಜೋಸೆಫ್
ಜ್ಞಾಪ-ಕಾರ್ಥ-ವಾಗಿ
ಜ್ವರದ
ಝರಾ-ಧರಾ
ಝುಮ್ರಾ
ಟಂಕ-ಸಾಲೆ
ಟಂಕ-ಸಾಲೆಗೆ
ಟನಿಟುರ
ಟಪ್ಪು-ಸುಲ್ತಾನನು
ಟಿಪು
ಟಿಪು-ವಿನ
ಟಿಪ್ಪಣಿ-ಗಳನ್ನು
ಟಿಪ್ಪ-ವಿನ
ಟಿಪ್ಪು
ಟಿಪ್ಪು-ವನ್ನು
ಟಿಪ್ಪು-ವಿಗ
ಟಿಪ್ಪು-ವಿನ
ಟಿಪ್ಪುವು
ಟಿಪ್ಪು-ಸುಲ್ತಾನನು
ಟಿಪ್ಪು-ಸುಲ್ತಾನ್
ಟಿಪ್ಪೂ
ಟಿಪ್ಪೂ-ವಿನ
ಟಿಪ್ಪೂ-ಸುಲ್ತಾನನ
ಟಿಪ್ಪೂ-ಸುಲ್ತಾನನು
ಟಿಪ್ಪೂ-ಸುಲ್ತಾನ್
ಟೇಕಲ್
ಠಾಣೆಯ
ಡಂಕನ್
ಡಂಕನ್ಡೆರೆಟ್
ಡಂಕನ್ಡೆರೆಟ್ರ-ವರು
ಡಾ
ಡಾಅಲ-ನರ-ಸಿಂಹನ್
ಡಾಎಂ
ಡಾಎಂಎಂ
ಡಾಎಂಚಿ-ದಾನಂದ-ಮೂರ್ತಿ-ಯ-ವರು
ಡಾಎಂಬಿ-ಪದ್ಮ
ಡಾಎಚ್ಎಸ್
ಡಾಎವಿ-ನರ-ಸಿಂಹ-ಮೂರ್ತಿ-ಯ-ವರು
ಡಾಎ-ಸತ್ಯ-ನಾ-ರಾಯಣ
ಡಾಎಸ್
ಡಾಎಸ್ಎನ್
ಡಾಎಸ್ಗುರು-ರಾಜಾ-ಚಾರ್ಯರ
ಡಾಎಸ್ನಾಗ-ರಾಜು
ಡಾಎಸ್ರಂಗ-ರಾಜು
ಡಾಕರಸ
ಡಾಕರ-ಸ-ನನ್ನು
ಡಾಕರ-ಸ-ನೆಂದು
ಡಾದೇವ-ರ-ಕೊಂಡಾ-ರೆಡ್ಡಿ
ಡಾದೇವ-ರ-ಕೊಂಡಾ-ರೆಡ್ಡಿ-ಯ-ವರು
ಡಾಬಿಆರ್
ಡಾಬಿಶೇಕ್ಅಲಿ
ಡಾರಾಧಾ-ಪಟೇಲ್
ಡಾಸಾಲೆ-ತೂರ್
ಡಾಸೂರ್ಯ-ನಾಥ-ಕಾ-ಮತ್
ಡಿಂಕ
ಡಿಡಗದ
ಡಿವಿ-ಜನ್
ಡಿವಿ-ಜನ್ಗೆ
ಡಿವಿ-ಜನ್ನಲ್ಲಿ
ಡಿಸೆಂಬರ್
ಡೆಂಕಣಿ-ಕೋಟೆ
ಡೆಕ್ಕನ್
ಡೆರೆಟ್
ಡೆರೆಟ್ರ-ವರು
ಡೊಣೆ
ಡೊಣೆ-ಗಳು
ಡೊಣೆ-ಯಿದ್ದು
ಢಣಾಯ-ಕನ
ತ
ತಂಗಡಗಿ
ತಂಗಿ
ತಂಗಿದ್ದ
ತಂಗಿದ್ದು
ತಂಗಿಯ
ತಂಗಿ-ಯನ್ನೇ-ನಾ-ದರೂ
ತಂಗಿ-ರ-ಬಹು-ದೆಂದು
ತಂಜಾವೂ-ರನ್ನು
ತಂಜಾ-ವೂರು
ತಂಡವ-ತಂದು
ತಂಡಸೇ-ಹಳ್ಳಿ
ತಂತಕ-ನಂತೆ-ಸಂಗ-ರದೊಳೋವದೆ
ತಂತ್ರ-ಗಳ
ತಂತ್ರ-ವೆಗ್ಗಡೆ
ತಂತ್ರ-ವೆಗ್ಗಡೆ-ತನ-ವನ್ನು
ತಂತ್ರ-ವೆಗ್ಗಡೆಯು
ತಂತ್ರಾಧಿಷ್ಟಾಯಕ
ತಂತ್ರಾಧಿಷ್ಠಾಯಕ
ತಂದ
ತಂದಂತೆ
ತಂದಿದ್ದ
ತಂದು
ತಂದೆ
ತಂದೆ-ಗಳ
ತಂದೆಗೆ
ತಂದೆ-ತಾಯಿ-ಗ-ಳೆಂದು
ತಂದೆಯ
ತಂದೆ-ಯ-ಕಾಲ-ದಲ್ಲೇ
ತಂದೆ-ಯ-ಗಂಧ-ವಾರಣ
ತಂದೆ-ಯನ್ನು
ತಂದೆ-ಯಾಗಿದ್ದು
ತಂದೆ-ಯಾದ
ತಂದೆ-ಯೊಲ್
ತಂನ
ತಂನ-ನುಜಾತರ್ಬ್ಬೋಕಣಂ
ತಂನಾಮ್ನಾ-ಯದ
ತಂಮ
ತಂಮಂಗೆ
ತಂಮ-ಡಿಗಟ್ಟೆಯ
ತಂಮಯ
ತಂಮೆಯ-ದೇವ
ತಕ್ಕ
ತಕ್ಕಂತೆ
ತಕ್ಕೋಲಂ
ತಕ್ಕೋಲಂನಲ್ಲಿ
ತಕ್ಕೋಲ-ದಲ್ಲಿ
ತಕ್ಕೋ-ಲದೊಲ್ಕಾದಿ
ತಕ್ಕೋ-ಲಮ್
ತಕ್ಷಣ
ತಕ್ಷಣ-ದಲ್ಲಿ
ತಕ್ಷಣ-ದಲ್ಲಿಯೇ
ತಗಚೆ-ಗೆರೆ-ಗಳನ್ನು
ತಗಡೂರಿ-ನಲ್ಲಿ
ತಗಡೂರು
ತಗರೆ
ತಗ್ಗ-ಲೂರು
ತಗ್ಗಿನ
ತಗ್ಗಿಳೂರು
ತಗ್ಗುಪ್ರ-ದೇಶ-ದಲ್ಲಿದ್ದು-ದ-ರಿಂದ
ತಜ್ಜ-ವನಿಕೆ-ಗೊಂಡ-ಗಂಡ
ತಜ್ಞ-ರಾದ
ತಜ್ಞರು
ತಟದ
ತಟ್ಟೇ-ಹಳ್ಳಿ
ತಡರೆ-ಬಲ್ಗಂಡನುಂ
ತಡಿಮಾಲಿಂಗಿ
ತಡಿ-ಯಲ್ಲಿದ್ದ
ತಡೆಗಟ್ಟಲು
ತಡೆದಿರ-ಬಹು-ದೆಂದು
ತಡೆದು
ತಡೆಯಲಾಗದೆ
ತಡೆ-ಯುವ
ತತೇಜೋ-ನಿಳಯಂ
ತತ್ಪುತ್ರಃ
ತತ್ರಯಿಕ
ತತ್ವ
ತತ್ವೈ-ಕನಿಷ್ಠುರ
ತತ್ಸಾಹಸಾಭ್ಯುದಯಂ
ತದನು
ತದನು-ಜನ್ಮಾ
ತನಂಗಾಡಿ
ತನಂಗಾಡಿ-ಯಾದ
ತನಕ
ತನಗೆ
ತನದ-ಕಯ್ಯ
ತನ-ದಟ್ಟಿಬಡಿವಂ
ತನಯ
ತನಯಃ
ತನ-ಯರ
ತನುಜರು
ತನೂಭವ
ತನೆ-ಯರು
ತನ್ನ
ತನ್ನನ್ನು
ತನ್ನೂ-ರಾದ
ತನ್ನೊಂದೆ
ತಪಃಫಲ
ತಪಸಿಯ
ತಪಸಿಯ-ತಪಸಿ-ಹಳ್ಳಿ
ತಪಸೀ-ರಾಯನ
ತಪಸ್ಸು
ತಪುವ-ರಾಯರ
ತಪೋಧನ-ನಿಗೆ
ತಪ್ಪದೆ
ತಪ್ಪದೇ
ತಪ್ಪಾಗಿ
ತಪ್ಪಾಗಿ-ರುವ
ತಪ್ಪಾಗು-ವುದು
ತಪ್ಪಿದ್ದು
ತಪ್ಪುವ
ತಪ್ಪುವ-ನಾಯ-ಕರ
ತಪ್ಪುವ-ರಾಯರ
ತಪ್ಪೆ-ತಪ್ಪುವಂ
ತಮಗೆ
ತಮಿಳು
ತಮಿಳುಗ್ರಂಥ-ಲಿಪಿಯ
ತಮಿಳು-ನಾಡನ್ನು
ತಮಿಳು-ನಾ-ಡಿನ
ತಮಿಳು-ನಾಡಿ-ನಲ್ಲಿ
ತಮಿಳು-ನಾ-ಡಿನಲ್ಲಿದ್ದನು
ತಮಿಳು-ನಾಡಿ-ನಲ್ಲೇ
ತಮಿಳು-ನಾಡು
ತಮಿಳು-ರೂಪ
ತಮಿಳು-ಶಾ-ಸನ-ದಲ್ಲಿ
ತಮ್ಮ
ತಮ್ಮಂದಿ-ರಿರ-ಬಹುದು
ತಮ್ಮಂದಿರು
ತಮ್ಮಂದಿ-ರೆಂದು
ತಮ್ಮಂದಿರೋ
ತಮ್ಮ-ಡಿ-ಹಳ್ಳಿ
ತಮ್ಮ-ತೀರ್ವ್ವರ್ಗ್ಗೆ
ತಮ್ಮನ
ತಮ್ಮ-ನಂತೆ
ತಮ್ಮ-ನನ್ನೋ
ತಮ್ಮ-ನಾದ
ತಮ್ಮ-ನಿದ್ದನು
ತಮ್ಮ-ನಿದ್ದ-ನೆಂದು
ತಮ್ಮ-ನಿರ-ಬಹುದು
ತಮ್ಮ-ನೀರ್ವ್ವರ್ಗೆ
ತಮ್ಮನೂ
ತಮ್ಮ-ನೆಂದು
ತಮ್ಮ-ನೆಂದೂ
ತಮ್ಮನ್ನು
ತಮ್ಮ-ಯಣ್ಣ
ತಮ್ಮವ್ವೆ
ತಮ್ಮೊಳಗೆ
ತಮ್ಮೋಜಿ
ತಯಾರಿಸಿ
ತಯಾರಿ-ಸುತ್ತಿದ್ದ-ರೆಂದು
ತಯಾರು
ತರ-ಲಾಗುತ್ತಿತ್ತು
ತರಿದಿಕ್ಕಿ-ದನು
ತರಿದು-ಹಾಕಿದ
ತರಿಸಿ
ತರಿಸಿ-ಕೊಡುತ್ತಾರೆ
ತರುಣ-ದಲ್ಲಿಯೇ
ತರುವ
ತರುವಾಯ
ತರೈ
ತರ್ದ-ವಾಡಿ
ತಲಕಾಡ
ತಲ-ಕಾಡನ್ನು
ತಲಕಾಡಾದ
ತಲ-ಕಾ-ಡಿಗೆ
ತಲ-ಕಾ-ಡಿನ
ತಲ-ಕಾ-ಡಿನಲ್ಲಿ
ತಲ-ಕಾ-ಡಿನಲ್ಲಿದ್ದ
ತಲ-ಕಾ-ಡಿನಲ್ಲಿದ್ದ-ನೆಂದು
ತಲ-ಕಾ-ಡಿನಲ್ಲಿದ್ದಾಗ
ತಲ-ಕಾ-ಡಿನ-ವರೆಗೂ
ತಲ-ಕಾ-ಡಿನಿಂದ
ತಲ-ಕಾಡು
ತಲ-ಕಾಡು-ಗೊಂಡ
ತಲಗ-ವಾಡಿ
ತಲೆ
ತಲೆ-ಕಾ-ಡಿನ
ತಲೆ-ಗ-ಳಿಯಿಸಿ-ಕೊಂಡು
ತಲೆಗೆ
ತಲೆ-ಗೊಂಡ-ನಿ-ರದೆ
ತಲೆ-ದೋರಿತೆಂದು
ತಲೆ-ನೆ-ರೆಯಲ
ತಲೆ-ಬಾಗಿ-ರ-ಲಿಲ್ಲ-ವೆಂದು
ತಲೆ-ಮಾ-ರಿಗೆ
ತಲೆ-ಮಾರಿನ
ತಲೆ-ಮಾರಿನ-ವರು
ತಲೆ-ಮಾರು
ತಲೆ-ಮಾರು-ಗಳ
ತಲೆಯ
ತಲೆ-ಯನ್ನು
ತಲೆ-ಯ-ಮಾಳೆಯ
ತಳಕಾಡ
ತಳಕಾಡಂ
ತಳಕಾಡ-ಅಧಿ-ಕಾರಿ
ತಳಕಾಡ-ನಾಡ
ತಳಕಾಡ-ನಾಡನ್ನು
ತಳಕಾಡ-ನಾಡಪ್ರಭು
ತಳಕಾಡಪ್ರಭು
ತಳ-ಕಾ-ಡಿನ
ತಳ-ಕಾಡು
ತಳ-ಕಾಡು-ಗೊಂಡ-ನೆಂದು
ತಳ-ಕಾಡು-ನಾಡು
ತಳ-ಪಾದಿಯ
ತಳ-ವನ-ಪುರ
ತಳ-ವನ-ಪುರ-ತಲ-ಕಾಡು
ತಳ-ವನ-ಪುರ-ತಲ-ಕಾಡು-ವಿಜಯಸ್ಕಾಂದ-ವಾರ-ದಲ್ಲಿದ್ದಾಗ-ರಾಜ-ಧಾನಿ
ತಳ-ವನ-ಪುರವೇ
ತಳ-ವಾರ
ತಳ-ವಾರ-ರಿಗೆ
ತಳ-ವಾರ-ರಿದ್ದರು
ತಳ-ವಾರಿಕೆ
ತಳ-ವಾರಿಕೆ-ಗಳಿಂದ
ತಳ-ವಾರಿಕೆಗೆ
ತಳ-ವಾರಿ-ಕೆಯ
ತಳ-ವಾರಿಕೆ-ಯಿಂದ
ತಳ-ವೃತ್ತಿ-ಯನ್ನು
ತಳವ್ರಿತ್ತಿಯಂ
ತಳಾರ
ತಳಾ-ರರು
ತಳಾರಿ
ತಳಾರಿ-ಕೆಯು
ತಳಿಗ್ರಾಮದ
ತಳಿ-ಯಣ್ಣ
ತಳಿ-ಯೂ-ರನ್ನು
ತಳೆಕಾಡಂ
ತಳೆಕಾಡ-ಸೀಮೆಯ
ತಳೆ-ಕಾಡು-ಪಟ್ಟಣ
ತಳೆದ-ನೆಂದು
ತಳೆ-ದಿದ್ದರು
ತಳೆದು
ತಳ್ತಿಯದ
ತಳ್ತಿ-ರಿದು
ತಳ್ತಿರಿವೆಡೆಗೊರ್ವ್ವ-ರಪ್ಪೊಡಮಿದಿರ್ಚುವ
ತಳ್ತಿರ್ದ್ದ
ತಳ್ಳಿಯದ
ತಳ್ಳಿಹಾಕ
ತವಿ-ಸಿದನು
ತಸ್ಯ
ತಸ್ಯಾ
ತಸ್ಯಾತ್ಮಜೋ
ತಾ
ತಾಂ
ತಾಂಜಂ
ತಾಂಡ-ವೇಶ್ವರ
ತಾಂಬೂಲ
ತಾಂಬೂಲ-ಸೇವೆಯ
ತಾಗಲಾರಂಭಿಸಿತು
ತಾಗಿ
ತಾಗಿತ್ತು
ತಾಣ
ತಾಣ-ಪನ-ಹಳ್ಳಿ
ತಾತ
ತಾತಾಚಾರ್ಯ-ನಿಗೆ
ತಾತಾ-ಚಾರ್ಯರಿಗೆ
ತಾನಿತ್ತು
ತಾನು
ತಾನೂ
ತಾನೆ
ತಾನೇ
ತಾಮೊನೆಬೆನ್ನಬಾ-ರನೆ
ತಾಮ್ರ
ತಾಮ್ರ-ಪಟ-ಗಳಲ್ಲಿ
ತಾಮ್ರ-ಪಟ-ಗಳಿಂದ
ತಾಮ್ರ-ಪಟ-ಗಳು
ತಾಮ್ರ-ಪಟ-ದಲ್ಲಿ
ತಾಮ್ರ-ಪಟ-ದಲ್ಲಿದೆ
ತಾಮ್ರ-ಶಾಸ-ಗಳು
ತಾಮ್ರ-ಶಾ-ಸನ
ತಾಮ್ರ-ಶಾ-ಸನ-ಗಳ
ತಾಮ್ರ-ಶಾ-ಸನ-ಗಳನ್ನು
ತಾಮ್ರ-ಶಾ-ಸನ-ಗಳಲ್ಲಿ
ತಾಮ್ರ-ಶಾ-ಸನ-ಗ-ಳಲ್ಲೂ
ತಾಮ್ರ-ಶಾ-ಸನ-ಗಳು
ತಾಮ್ರ-ಶಾ-ಸನದ
ತಾಮ್ರ-ಶಾ-ಸನ-ದ-ಅಲ್ಲಿ
ತಾಮ್ರ-ಶಾ-ಸನ-ದಲ್ಲಿ
ತಾಮ್ರ-ಶಾ-ಸನ-ದಲ್ಲಿದೆ
ತಾಮ್ರ-ಶಾ-ಸನ-ದಲ್ಲೂ
ತಾಮ್ರ-ಶಾ-ಸನ-ದಿಂದ
ತಾಮ್ರ-ಶಾ-ಸನ-ವನ್ನು
ತಾಮ್ರ-ಶಾ-ಸನ-ವಾಗಿದ್ದು
ತಾಮ್ರ-ಶಾ-ಸನ-ವಿದ್ದು
ತಾಮ್ರ-ಶಾ-ಸನವು
ತಾಮ್ರ-ಶಾ-ಸನವೇ
ತಾಮ್ರ-ಶಾ-ಸನ್
ತಾಯ
ತಾಯಣ್ಣನು
ತಾಯ-ಲೂರಿನ
ತಾಯ-ಲೂರಿನಲ್ಲಿರುವ
ತಾಯ-ಲೂರು
ತಾಯಿ
ತಾಯಿ-ಗಳ
ತಾಯಿ-ಗಳೇ
ತಾಯಿ-ಮಂಚಿಯಕ್ಕನ
ತಾಯಿಯ
ತಾಯಿ-ಸಾಕು-ತಾಯಿ
ತಾಯೂರು
ತಾಯ್ಮುದ್ದ-ರಸಿ
ತಾರೀಖನ್ನು
ತಾರೀಖಿನ
ತಾರೀಖಿನಂದು
ತಾರೀಖಿನಂದೇ
ತಾರೀಖು
ತಾರೀಖೆಂದು
ತಾಲೂಕು-ಗಳು
ತಾಲ್ಲೂ-ಕನ್ನಾಗಿ
ತಾಲ್ಲೂ-ಕನ್ನು
ತಾಲ್ಲೂ-ಕನ್ನೂ
ತಾಲ್ಲೂಕಾಗಿದೆ
ತಾಲ್ಲೂಕಾ-ಗಿದ್ದು
ತಾಲ್ಲೂಕಿ
ತಾಲ್ಲೂಕಿಗೂ
ತಾಲ್ಲೂಕಿಗೆ
ತಾಲ್ಲೂಕಿನ
ತಾಲ್ಲೂಕಿ-ನಅ
ತಾಲ್ಲೂಕಿ-ನಲ್ಲಿ
ತಾಲ್ಲೂಕಿ-ನಲ್ಲಿದೆ
ತಾಲ್ಲೂಕಿ-ನಲ್ಲಿದ್ದರೆ
ತಾಲ್ಲೂಕಿ-ನಲ್ಲಿದ್ದ-ವೆಂದು
ತಾಲ್ಲೂಕಿ-ನಲ್ಲಿ-ರುವ
ತಾಲ್ಲೂಕು
ತಾಲ್ಲೂಕು-ಗಳ
ತಾಲ್ಲೂಕು-ಗಳನ್ನು
ತಾಲ್ಲೂಕು-ಗಳನ್ನೊಳ-ಗೊಂಡ
ತಾಲ್ಲೂಕು-ಗಳಲ್ಲಿ
ತಾಲ್ಲೂಕು-ಗಳಲ್ಲಿದ್ದ
ತಾಲ್ಲೂಕು-ಗಳ-ವೆ-ರೆಗೆ
ತಾಲ್ಲೂಕು-ಗಳಾಗಿ
ತಾಲ್ಲೂಕು-ಗ-ಳಾದ
ತಾಲ್ಲೂಕು-ಗಳಿದ್ದವು
ತಾಲ್ಲೂಕು-ಗಳು
ತಾಲ್ಲೂಕು-ಗಳೂ
ತಾಲ್ಲೂಕು-ಗಳೆಂಬ
ತಾಲ್ಲೂಕು-ವಾರು
ತಾಲ್ಲೂಕ್ನಲ್ಲಿ
ತಾಳಗುಂದ
ತಾಳ-ತಿಟ್ಟು
ತಾಳ್ದಿದಂ
ತಾವರೆ-ಕೆರೆಯ
ತಾವಾಗಿಯೇ
ತಾವು
ತಾವೂ
ತಾವೇ
ತಿಂಗಳ
ತಿಂಗಳಲ್ಲಿ
ತಿಂಗಳ-ವ-ರೆಗೆ
ತಿಂಗಳಿ-ನಲ್ಲಿ
ತಿಂಗಳು
ತಿಂಮಂಣ್ನ
ತಿಂಮಣ
ತಿಂಮಪ-ನಾಯಕ
ತಿಂಮ-ರಾಜ-ಗಳ
ತಿಕ್ಕಮ
ತಿಕ್ಕಮ-ನನ್ನು
ತಿಕ್ಕಾಟ
ತಿಗಡ-ಹಳ್ಳಿ
ತಿಗಳ
ತಿಗ-ಳನು
ತಿಗಳರು
ತಿಗು-ಳನ-ಕೆರೆ
ತಿಟ್ಟು
ತಿನ-ರಸಿ-ಪುರ
ತಿನ-ರಸೀ-ಪುರ
ತಿನಿ-ರಾಯ
ತಿನಿ-ರಾಯಾಂಬಾಮುಲ
ತಿಪಟೂರು
ತಿಪ್ಪಣ್ಣ
ತಿಪ್ಪಣ್ಣ-ನಾಯ-ಕ-ನೆಂಬು-ನನೂ
ತಿಪ್ಪಣ್ಣ-ನಾಯ-ಕ-ರಿಗೆ
ತಿಪ್ಪಯ್ಯನ
ತಿಪ್ಪಯ್ಯನು
ತಿಪ್ಪ-ರ-ಸರು
ತಿಪ್ಪ-ರಸಾರ್ಯನ-ಪುತ್ರ
ತಿಪ್ಪಳಿ-ನಾಯ-ಕನು
ತಿಪ್ಪವ್ವೆ
ತಿಪ್ಪವ್ವೆಗೂ
ತಿಪ್ಪಾಜಿ
ತಿಪ್ಪಾಜಿಯ
ತಿಪ್ಪೂರ
ತಿಪ್ಪೂ-ರಿಗೆ
ತಿಪ್ಪೂರು
ತಿಪ್ಪೂರೇ
ತಿಪ್ಪೆ-ಯೂರಿನ
ತಿಪ್ಪೆ-ಯೂರು
ತಿಪ್ಪೆರು-ವಳ್ಳಿಯ
ತಿಪ್ಪೆರೂ-ರನ್ನು
ತಿಪ್ಪೆರೂರಿನ
ತಿಪ್ಪೆರೂರು
ತಿಬ್ಬನ-ಹಳ್ಳಿ
ತಿಬ್ಬನ-ಹಳ್ಳಿಗೆ
ತಿಬ್ಬನ-ಹಳ್ಳಿಯ
ತಿಬ್ಬನ-ಹಳ್ಳಿ-ಯನ್ನು
ತಿಬ್ಬ-ಸೆಟ್ಟಿಯು
ತಿಮ್ಮ
ತಿಮ್ಮ-ಅ-ರಸಯ್ಯ
ತಿಮ್ಮ-ಕವಿಯ
ತಿಮ್ಮ-ಜಗ-ದೇವ-ರಾಯನ
ತಿಮ್ಮಣ್ಣ
ತಿಮ್ಮಣ್ಣ-ದಂಡ-ನಾಯ-ಕನ
ತಿಮ್ಮಣ್ಣ-ನಾಯಕ
ತಿಮ್ಮಣ್ಣನು
ತಿಮ್ಮಣ್ಣನ್ನು
ತಿಮ್ಮನ
ತಿಮ್ಮ-ನಾಯ-ಕನ
ತಿಮ್ಮ-ನಾಯ-ಕನು
ತಿಮ್ಮನು
ತಿಮ್ಮಪ್ಪ
ತಿಮ್ಮಪ್ಪ-ನಾಯಕ
ತಿಮ್ಮಪ್ಪಯ್ಯ
ತಿಮ್ಮಯ್ಯ-ದೇವ
ತಿಮ್ಮಯ್ಯನ
ತಿಮ್ಮ-ರಸನು
ತಿಮ್ಮ-ರಸಯ್ಯನು
ತಿಮ್ಮ-ರ-ಸರು
ತಿಮ್ಮ-ರಾಜ
ತಿಮ್ಮ-ರಾಜನ
ತಿಮ್ಮ-ರಾಜ-ನಿಗೆ
ತಿಮ್ಮ-ರಾಜನು
ತಿಮ್ಮ-ರಾ-ಜಯ್ಯ
ತಿಮ್ಮ-ರಾಜು
ತಿಮ್ಮ-ಸಮುದ್ರ
ತಿಮ್ಮಾಂಬೆ-ಯರ
ತಿರಸ್ಕರಿ-ಸಿದ್ದಾರೆ
ತಿರಿಮಣ್ಣ
ತಿರುಕುಡಿ
ತಿರುಗಾಡುತ್ತಾ
ತಿರುಗಿ
ತಿರುಗಿ-ಬಿದ್ದ
ತಿರುಗಿ-ಬಿದ್ದನು
ತಿರುಗಿ-ಬಿದ್ದ-ರೆಂದೂ
ತಿರುಗಿ-ಬಿದ್ದಿ-ರುವ
ತಿರುಗಿ-ಬಿದ್ದು
ತಿರುಚಿರಾಪಳ್ಳಿ
ತಿರುಣ-ನಾಯಕ
ತಿರು-ನಂದಾದೀ-ಪಕ್ಕೆ
ತಿರು-ನಾ-ರಾಯಣ
ತಿರು-ನಾ-ರಾಯ-ಣ-ದೇ-ವ-ರಿಗೆ
ತಿರು-ನಾ-ರಾಯ-ಣ-ಪುರ
ತಿರು-ನಾ-ರಾಯ-ಣ-ಪುರಕ್ಕೆ
ತಿರುನಾ-ಳಿಗೆ
ತಿರು-ಪತಿ
ತಿರು-ಪತಿ-ಯಲ್ಲಿ
ತಿರುಪ್ರತಿಷ್ಠೆ-ಯನ್ನು
ತಿರುಮಂಜನಕ್ಕೆ
ತಿರುಮಕೂಡಲು
ತಿರುಮಕೂಡಲು-ನ-ರಸೀ-ಪುರ
ತಿರುಮಕೂಡು
ತಿರು-ಮಲ
ತಿರು-ಮಲ-ಗಿರಿಯ
ತಿರು-ಮಲ-ದೇವ
ತಿರು-ಮಲ-ದೇವರ
ತಿರು-ಮಲ-ದೇವ-ರಿಗೆ
ತಿರು-ಮಲನ
ತಿರು-ಮಲ-ನನ್ನು
ತಿರು-ಮಲ-ನಾಥ
ತಿರು-ಮಲ-ನಾಥನ
ತಿರು-ಮಲ-ನಾಥನು
ತಿರು-ಮಲ-ನಾಯ-ಕನ
ತಿರು-ಮಲ-ನಿಂದ
ತಿರು-ಮಲನು
ತಿರು-ಮಲ-ನೆಂಬ
ತಿರು-ಮಲನೇ
ತಿರುಮ-ಲಮ್ಮ
ತಿರು-ಮಲ-ಯಾರ್ಯೋವ್ಯತಾನೀತ್ತಾಂಬ್ರ
ತಿರು-ಮಲ-ರಾಜ
ತಿರು-ಮಲ-ರಾಜನ
ತಿರು-ಮಲ-ರಾಜ-ನನ್ನು
ತಿರು-ಮಲ-ರಾಜ-ನಾಯ-ಕಗೆ
ತಿರು-ಮಲ-ರಾಜ-ನಿಗೂ
ತಿರು-ಮಲ-ರಾಜನು
ತಿರು-ಮಲ-ರಾಜ-ನೆಂದು
ತಿರು-ಮಲ-ರಾಜ-ಯ-ದೇವ
ತಿರು-ಮಲ-ರಾ-ಜಯ್ಯ
ತಿರು-ಮಲ-ರಾಜಯ್ಯ-ದೇವ
ತಿರು-ಮಲ-ರಾಜಯ್ಯನ
ತಿರು-ಮಲ-ರಾಜಯ್ಯ-ನ-ವರು
ತಿರು-ಮಲ-ರಾಜಯ್ಯನು
ತಿರು-ಮಲ-ರಾಜಯ್ಯ-ನೆಂಬ
ತಿರು-ಮಲ-ರಾಜಯ್ಯನೇ
ತಿರು-ಮಲ-ರಾಜರು
ತಿರು-ಮಲ-ರಾಜು
ತಿರು-ಮಲ-ರಾಯ
ತಿರು-ಮಲ-ರಾಯರ
ತಿರು-ಮಲಾ-ಚಾರ್ಯರು
ತಿರು-ಮಲಾರ್ಯ
ತಿರು-ಮಲಾರ್ಯ-ನಿಂದ
ತಿರು-ಮಲಾರ್ಯನು
ತಿರು-ಮಲೆ
ತಿರು-ಮಲೆ-ಯಾರ್ಯರ
ತಿರು-ಮಲೈಯ್ಯಂಗಾ-ರರ
ತಿರುಮಾಲೆ
ತಿರುಳು-ನಾಡು
ತಿರುವಣ್ಣಾ-ಮಲೆಗೂ
ತಿರುವಣ್ಣಾ-ಮಲೆ-ಯನ್ನೇ
ತಿರುವಣ್ಣಾ-ಮಲೆ-ಯಲ್ಲಿದ್ದು-ಕೊಂಡು
ತಿರುವ-ನಂತ-ಪುರದ
ತಿರುವ-ರಂಗ-ದಾಸನು
ತಿರುವ-ರುಂಗ-ದಾಸನ
ತಿರುವವ್ವೆ
ತಿರುವಿಂದಳೂರ
ತಿರುವಿಡಿಯಾಟಕ್ಕೆ
ತಿರುವಿಡಿಯಾಟದ
ತಿರುವಿಡಿಯಾಟ್ಟಕ್ಕೆ
ತಿರು-ವೆಂಕಟ-ನಾಯಕ
ತಿರು-ವೆಂಕಟಾದ್ರಿ
ತಿರುವೆಂಗಟಯ್ಯನ
ತಿರುವೆಂಗಳ-ನಾಥ
ತಿರುವೆಂಗಳ-ನಾಥನ
ತಿರುವೇಂಕಟ-ನಾಯ-ಕನು
ತಿರೆ
ತಿಲಕ-ನೆನಿಸಿದ
ತಿಲಕ-ರಂತೆ
ತಿಲಕರು
ತಿಲಿ-ಕೂತ್ತಾಂಡಿ
ತಿಲೆ
ತಿಲ್ಲೆಕೂತ್ತ
ತಿಳಂಕಂಗೀ
ತಿಳಕ
ತಿಳದು-ಬರುತ್ತದೆ
ತಿಳಿದ
ತಿಳಿದ-ಬರುತ್ತದೆ
ತಿಳಿ-ದಿದೆ
ತಿಳಿ-ದಿದೆ-ಯಾಗಿ
ತಿಳಿ-ದಿ-ರುವ
ತಿಳಿದು
ತಿಳಿದು-ಕೊಂಡಿದ್ದರು
ತಿಳಿದು-ಕೊಳ್ಳ-ಬಹುದು
ತಿಳಿದು-ಕೊಳ್ಳುವುದು
ತಿಳಿದು-ಬರುತ್ತದೆ
ತಿಳಿದು-ಬರುತ್ತದೆ-ಲಾಳ-ನ-ಕೆರೆಯ
ತಿಳಿದು-ಬರುತ್ತದೆೆ
ತಿಳಿದು-ಬರುತ್ತವೆ
ತಿಳಿದು-ಬರುತ್ತೆ
ತಿಳಿದು-ಬ-ರುವ
ತಿಳಿದು-ಬ-ರುವುದ-ರಿಂದ
ತಿಳಿದು-ಬ-ರು-ವು-ದಿಲ್ಲ
ತಿಳಿಯದ
ತಿಳಿಯದು
ತಿಳಿಯ-ಬಹುದು
ತಿಳಿಯ-ಬಾ-ರದು
ತಿಳಿ-ಯುತ್ತದೆ
ತಿಳಿಯು-ವುದು
ತಿಳಿವ-ಳಿಕೆ
ತಿಳಿ-ಸಿದ
ತಿಳಿ-ಸಿದಂತೆ
ತಿಳಿ-ಸಿದನು
ತಿಳಿ-ಸಿದ-ನೆಂದೂ
ತಿಳಿ-ಸುತ್ತದೆ
ತಿಳಿ-ಸುತ್ತ-ದೆಂದು
ತಿಳಿ-ಸುತ್ತವೆ
ತಿಳಿ-ಸುತ್ತಿದೆ
ತಿಳಿ-ಸುವ
ತಿಳಿಸು-ವಂತೆ
ತಿಳಿ-ಸು-ವಲ್ಲಿ
ತೀತಳಮರಿ-ವನ್ತು
ತೀರ
ತೀರದ
ತೀರ-ದಲ್ಲಿ
ತೀರ-ದಲ್ಲಿತ್ತು
ತೀರ-ದಲ್ಲಿ-ರುವ
ತೀರ-ದ-ವರೆಗೂ
ತೀರಪ್ರ-ದೇಶ
ತೀರಾ
ತೀರಿ-ಕೊಂಡನು
ತೀರಿ-ಕೊಂಡ-ನೆಂದು
ತೀರಿ-ಕೊಂಡಾಗ
ತೀರಿ-ಕೊಂಡಿದ್ದನು
ತೀರಿ-ಕೊಂಡಿರ-ಬೇಕು
ತೀರಿ-ಕೊಳ್ಳಲು
ತೀರಿಸಿ-ಕೊಳ್ಳುವ
ತೀರ್ಥಂಕ-ರರ
ತೀರ್ಥಕ್ಕೆ
ತೀರ್ಥದ
ತೀರ್ಥ-ದಲ್ಲಿ
ತೀರ್ಥ-ಯಾತ್ರಾದಿ-ಗಳಲ್ಲಿ
ತೀರ್ಥ-ವನ್ನು
ತೀರ್ಥ-ವಾಗಿತ್ತೆಂದು
ತೀರ್ಥವು
ತೀರ್ಥ-ವು-ಇಂದಿನ
ತೀರ್ಥ-ವೆಂದು
ತೀರ್ಮಾನಿಸ-ಬಹುದು
ತೀವ್ರ
ತು
ತುಂಗ-ಭದ್ರಾ
ತುಂಗ-ಭದ್ರಾ-ತೀರ
ತುಂಗ-ಭದ್ರಾ-ತೀರದ
ತುಂಗ-ಭದ್ರಾ-ತೀರ-ದಲ್ಲಿ
ತುಂಗ-ಭದ್ರಾ-ತೀರಲ್ಲಿದ್ದಾಗ
ತುಂಗ-ಭದ್ರಾ-ತೀರ್ಥ-ದಲ್ಲಿ
ತುಂಗ-ಭದ್ರಾ-ನದಿಗೆ
ತುಂಗ-ಭದ್ರೆಯ
ತುಂಗ-ಭದ್ರೆ-ಯನ್ನು
ತುಂಡನುಂನತಂ
ತುಂಡು
ತುಂಬ-ದೇವ-ನ-ಹಳ್ಳಿಯ
ತುಂಬಲ
ತುಂಬಿ-ದನು
ತುಂಬಿರ-ಬಹುದು
ತುಂಬಿ-ಹರಿ-ಯುತ್ತಿದ್ದ
ತುಕಡಿ-ಯನ್ನು
ತುಕಡಿ-ಯಲ್ಲಿ
ತುಗಲಕ್
ತುತ್ತಾಗಿ
ತುಪದ
ತುಮಕೂರಿನ
ತುಮಕೂರು
ತುರಂಗಮಂ
ತುರಂಗ-ಮಂಗಳಂ
ತುರಗ
ತುರಗ-ಕ-ಳನಿ-ರಿದು
ತುರಗ-ಕಳ-ವನಿ-ರಿದು
ತುರಗ-ಗಳನ್ನು
ತುರಗ-ಗ-ಳನ್ನೆಲ್ಲಾ
ತುರಲೋಭತು
ತುರು-ಕರ
ತುರು-ಕರು
ತುರುಕಳಗನಿ-ರಿದು
ತುರು-ಕಳ-ವನಿ-ರಿದು
ತುರುಗಮಂಗಳಂ
ತುರು-ಗಳ
ತುರು-ಗಳನ್ನು
ತುರುಗೋಳಿನ
ತುರುಗೋಳಿ-ನಲ್ಲಿ
ತುರುಗೋಳು
ತುರುಗೋಳು-ಗಳ
ತುರುಪರಿ-ವಿ-ನಲ್ಲಿ-ತುರುಗೊಳ್
ತುರುವ-ನಿಕ್ಕಿಸಿ
ತುರು-ವನ್ನು
ತುರುವೆ-ಕೆರೆ
ತುರುಷ್ಕ-ತುರಗಾರೂಢ
ತುರುಷ್ಕ-ಮುಸ್ಲಿಂ
ತುರುಷ್ಕರ
ತುರುಷ್ಕ-ರಾಜ
ತುರ್ಕಿಯ
ತುಲಗಣ್ಡ
ತುಲಾ-ಪುರ-ಷಾದಿ
ತುಲಾ-ಪುರುಷಾದಿ
ತುಲುವೇಂದ್ರ-ನಾದ
ತುಳಿದಂ
ತುಳುವ
ತುಳುವ-ನರ-ಸಿಂಹ
ತುಳುವಲ
ತುಳುವ-ಲ-ದೇವಿ
ತುಳುವ-ಲೇಶ್ವರ
ತುಳುವ-ವಂಶದ
ತುವ್ವ-ಲೇಶ್ವರ
ತುಷ್ಟಾಶೇಷದ್ವಿ-ಜನ್ಮನಃ
ತೂಬನ್ನಿಡಿಸಿ
ತೂಬನ್ನು
ತೂಬಿನ-ಕೆರೆ
ತೂರ್ಯ
ತೃಪ್ತಿ
ತೆಂಕಣ
ತೆಂಕಣ-ಭಾಗ
ತೆಂಕಣ-ರಾಯ
ತೆಂಕ-ಭಾಗ-ದಲ್ಲಿದ್ದ
ತೆಂಕಲಂಕದ
ತೆಂಕಲು
ತೆಂಕಳಣ
ತೆಂಗಿನ-ಕಟ್ಟ
ತೆಂಗಿನ-ಕಟ್ಟ-ಇಂದಿನ
ತೆಂಗಿನ-ಕಟ್ಟದ
ತೆಂಗಿನ-ಕಟ್ಟ-ದಲ್ಲಿ
ತೆಂಗಿನ-ಕಟ್ಟ-ವನು
ತೆಂಗಿನ-ಘಟ್ಟ
ತೆಂಗಿನ-ಘಟ್ಟ-ವನ್ನು
ತೆಂಗು
ತೆಂಪಾಗೈ
ತೆಗಡರ-ಹಳ್ಳಿ
ತೆಗೆ-ದಿಡುತ್ತಾರೆ
ತೆಗೆ-ದಿರಿಸು-ವುದು
ತೆಗೆದು
ತೆಗೆ-ದು-ಕೊಂಡನು
ತೆಗೆ-ದು-ಕೊಂಡ-ರೆಂದು
ತೆಗೆ-ದು-ಕೊಂಡಿದ್ದನು
ತೆಗೆ-ದು-ಕೊಂಡಿದ್ದ-ನೆಂದೂ
ತೆಗೆ-ದು-ಕೊಂಡು
ತೆಗೆ-ದು-ಕೊಳ್ಳುತ್ತಿದ್ದ-ರೆಂಬುದು
ತೆತ್ತಿಗ-ನೆನೆ-ವುದು
ತೆನದಂಕಾನ್ವಯ
ತೆನದಂಕಾನ್ವ-ಯದ
ತೆಪ್ಪ-ಕೊಳದ
ತೆಪ್ಪ-ಕೊಳ-ವನ್ನು
ತೆಪ್ಪಣ್ಣತೇಪಣ್ಣ-ದೇವಣ್ಣ
ತೆಪ್ಪತಿ-ರುನಾಳು
ತೆಪ್ಪದ
ತೆಪ್ಪದ-ನಾಗಣ್ಣನು
ತೆರಕಣಾಂಬಿ
ತೆರಕಣಾಂಬಿಯ
ತೆರಕಣಾಂಬಿ-ಯಲ್ಲಿ
ತೆರಕಣಾಂಬಿ-ಯಲ್ಲಿದ್ದ
ತೆರಕಣಾಂಬಿ-ಯಿಂದ
ತೆರಕಣಾಂಬಿ-ಸೀಮೆಯ
ತೆರಕಣಾಂಬೆ
ತೆರಕಣಾಂಬೆಯ
ತೆರಕಣಾಂಬೆ-ಯನ್ನು
ತೆರಣೆನ-ಹಳ್ಳಿ
ತೆರದಿಂದವೆ
ತೆರ-ನಾಗಿದ್ದರೂ
ತೆರ-ನಾದ
ತೆರಲು
ತೆರಳಿ
ತೆರಾಯಾಂಬಾಮುಲ
ತೆರಿಗೆ
ತೆರಿಗೆ-ಗಳ
ತೆರಿಗೆ-ಗಳನ್ನು
ತೆರಿಗೆ-ಗಳನ್ನು-ದತ್ತಿ-ಗಳನ್ನು
ತೆರಿಗೆ-ಗಳಲ್ಲಿ
ತೆರಿಗೆಯ
ತೆರಿಗೆ-ಯನ್ನು
ತೆರೆ-ಯನ್ನು
ತೆಲುಂಗನ
ತೆಲುಂಗ-ರಾಯಸ್ಥಾಪನಾಚಾರ್ಯ
ತೆಲುಗು
ತೆಲುಗು-ಚೋಡ
ತೆಲುಗು-ಚೋಡರ
ತೆಲುಗು-ಚೋಳ-ರನ್ನು
ತೆಲುಗು-ಭಾಷೆ
ತೆಲುಗು-ಭಾಷೆಯ
ತೆಲುಗು-ಮೂಲದ
ತೆಲುಗು-ಮೂಲ-ವೆಂದು
ತೆಲುಗು-ಸೀಮೆ-ಯ-ವರು
ತೆಲ್ಲಿಗ
ತೆಳರ-ಕುಲ-ತಿಲಕ
ತೆಳರ-ಕುಲದ
ತೆಳ್ಳರ
ತೆಳ್ಳರ-ಕುಲದ
ತೇಕಲ್ಲು
ತೇಗಿನ-ಹಳ್ಳಿ
ತೇಜೋ-ಮೂರ್ತಿ
ತೇದಿ
ತೇದಿ-ಗಳನ್ನು
ತೇದಿ-ಗಳಿವೆ
ತೇದಿ-ಗಳು
ತೇದಿಯ
ತೇದಿ-ಯನ್ನು
ತೇದಿ-ಯಲ್ಲಿ
ತೇದಿ-ಯಿಲ್ಲದ
ತೇದಿಯು
ತೇದಿ-ಯುಕ್ತ
ತೇದಿ-ಯುಳ್ಳ
ತೇದಿ-ರಹಿತ
ತೇದಿ-ರಹಿತ-ವಾಗಿದೆ
ತೇದಿ-ರಹಿತ-ವಾದ
ತೇರಣ್ಯ
ತೈರೂರ
ತೈಲನ
ತೈಲನು
ತೈಲಪನು
ತೈಲೂರು
ತೊಂಡನೂರ
ತೊಂಡನೂ-ರಿಗೂ
ತೊಂಡನೂರಿನ
ತೊಂಡನೂರಿನಲ್ಲಿ
ತೊಂಡನೂರಿನಲ್ಲಿದ್ದಾಗ
ತೊಂಡ-ನೂರು
ತೊಂಡ-ಮಂಡ-ಲದ
ತೊಂಡೇ-ಹಳ್ಳಿ
ತೊಂಡೈ-ಮಂಡ-ಲಮ್
ತೊಂದರೆ
ತೊಂದರೆ-ಯಾದಾಗ
ತೊಂಬತ್ತರು-ಸಾವಿರ
ತೊಂಬತ್ತರು-ಸಾಸಿರ
ತೊಂಬತ್ತರು-ಸಾಸಿರದ
ತೊಂಬತ್ತರು-ಸಾಸಿರಮಂ
ತೊಂಬತ್ತರು-ಸಾಸಿರ-ಮನು
ತೊಂಬತ್ತರು-ಸಾಸಿರ-ಮನೇಕ
ತೊಂಬತ್ತಾರು
ತೊಂಬತ್ತಾರು-ಸಾವಿರ
ತೊಂಬತ್ತಾರು-ಸಾವಿರದ
ತೊಂಬತ್ತಾರು-ಸಾವಿರನ್ನು
ತೊಂಬತ್ತಾರು-ಸಾವಿರ-ವನ್ನು
ತೊಗರ-ವಾಡಿ
ತೊಟ್ಟಿಲು
ತೊಡಗಿ-ಕೊಂಡಿದ್ದನು
ತೊಡಗಿದ್ದನು
ತೊಡಗಿದ್ದು-ದನ್ನು
ತೊಡಗಿ-ಸಲು
ತೊಡಗಿಸಿ-ಕೊಂಡರು
ತೊಡಗುತ್ತಿದ್ದ
ತೊಡರ್ದರ-ಡೊಂಕಿಯುಂ
ತೊಡರ್ದ್ದ-ರಂಕುಸ
ತೊಣಚಿ
ತೊಣ್ಣೂರಿನ
ತೊಣ್ಣೂರಿ-ನಲ್ಲಿ
ತೊಣ್ಣೂರಿನಲ್ಲಿ-ರುವ
ತೊಣ್ಣೂರಿ-ನಿಂದ
ತೊಣ್ಣೂರು
ತೊಣ್ಣೂರು-ಗಳಲ್ಲಿ
ತೊರೆ
ತೊರೆ-ಕಾಡನ-ಹಳ್ಳಿ
ತೊರೆ-ಗಳಾಗಿವೆ
ತೊರೆ-ಗಳು
ತೊರೆದು
ತೊರೆ-ನಾಡು
ತೊರೆ-ಬೊಮ್ಮನ-ಹಳ್ಳಿ
ತೊರೆ-ಮಗ್ಗ
ತೊರೆಯ
ತೊಱಗ-ಲೆಯ
ತೊಲ-ಗಂಡ
ತೊಲಗದ
ತೊಳಂಚೆಯ
ತೊಳಂಚೆ-ಯಲ್ಲಿ
ತೊಳಲ್ದು
ತೊಳಸಿ
ತೊಳಸಿಯ
ತೊಳೆದು
ತೋಟ
ತೋಟ-ಗಳನ್ನು
ತೋಟದ
ತೋಟ-ದಲ್ಲಿದ್ದು
ತೋಟ-ನ-ವನ್ನು
ತೋಟ-ವನ್ನು
ತೋಟ-ವೃತ್ತಿ
ತೋಟಸ್ಥಳ-ಗ-ಳನು
ತೋಟಿ
ತೋಟಿ-ಗರು
ತೋಡಿ-ಸಿದ್ದ-ನೆಂದೂ
ತೋಡಿ-ಸುತ್ತಾರೆ
ತೋರಣ-ವನ್ನು
ತೋರಿ-ನಾಡ
ತೋರಿ-ನಾಡು
ತೋರಿ-ಸಲೆಂದು
ತೋರಿ-ಸಿದರೆ
ತೋರಿ-ಸಿದ್ದಾರೆ
ತೋರಿ-ಸುತ್ತದೆ
ತೋರಿ-ಸುತ್ತ-ವೇನೋ
ತೋರಿ-ಸುತ್ತಿದೆ
ತೋರಿ-ಸುವುದೇ
ತೋರುತ್ತದೆ
ತೋರುತ್ತದೆ-ಎಂದು
ತೋರುತ್ತಿ-ರುವಂತಿದ್ದರೆ
ತೋಳ
ತೋಳ-ಬಿಂಕಮಂ
ತೌಳಿಯಮ್ಮ
ತ್ತುತ್ತಿರೆ
ತ್ತೊಮ್ಭತ್ತಱು
ತ್ಯಂತ-ವಾಗಿ
ತ್ಯಾಗದ
ತ್ಯಾಗದ-ಕೊಡುಗೆ-ಯಾಗಿ
ತ್ಯಾಗನ-ಹಳ್ಳಿ
ತ್ಯಾಗ-ವಾಗಿ
ತ್ರಿಕೂಟ
ತ್ರಿಕೂಟಬ
ತ್ರಿಕೂಟ-ರತ್ನತ್ರಯ
ತ್ರಿಣೇತ್ರ
ತ್ರಿಣೇತ್ರನಂ
ತ್ರಿಣೇತ್ರ-ನೆಂದು
ತ್ರಿಭುವ-ಚಕ್ರ-ವರ್ತಿ
ತ್ರಿಭು-ವನ
ತ್ರಿಭು-ವನ-ಕಠಾರಿ-ರಾಯನೂ
ತ್ರಿಭು-ವನ-ಚಕ್ರ-ವರ್ತಿ
ತ್ರಿಭು-ವನ-ತೀರ್ಥದ
ತ್ರಿಭು-ವನ-ಮಲ್ಲ
ತ್ರಿಭು-ವನೀ-ರಾಯ
ತ್ರಿಮೂರ್ತಿ-ಗಳ
ತ್ರಿಯಂಬಕೇಶ್ವರ
ತ್ರುಟಿತ
ತ್ರುಟಿತ-ಭಾಗ
ತ್ರುಟಿತ-ವಾಗಿತ್ತೆಂದು
ತ್ರುಟಿತ-ವಾಗಿದೆ
ತ್ರುಟಿತ-ವಾಗಿದ್ದು
ತ್ರುಟಿತ-ವಾಗಿ-ರುವ
ತ್ರುಟಿತ-ವಾಗಿ-ರುವುದ-ರಿಂದ
ತ್ರುಟಿತ-ವಾದ
ತ್ರುಟಿದ
ತ್ರೈಲೋಕ್ಯರಂಜನ
ತ್ರೈವಿದ್ಯ-ದೇವರ
ತ್ವಂ
ಥಾಣ
ದ
ದಂಗೆ
ದಂಗೆಯ
ದಂಗೆ-ಯನ್ನು
ದಂಡ
ದಂಡಂಗಳು
ದಂಡಗಿ
ದಂಡ-ಡ-ನಾಯ-ಕನ
ದಂಡ-ದಧಿಷ್ಠಾಯಕ
ದಂಡ-ದಧಿಷ್ಠಾಯಕರು
ದಂಡ-ನಾ-ತಾಂಬ-ರಾರ್ಕ್ಕಂ
ದಂಡ-ನಾಥ
ದಂಡ-ನಾಥನ
ದಂಡ-ನಾಥ-ನನ್ನು
ದಂಡ-ನಾಥ-ನಾಗಿ
ದಂಡ-ನಾಥನು
ದಂಡ-ನಾಥಾಧಿಪ
ದಂಡ-ನಾಥೋ
ದಂಡ-ನಾಯಕ
ದಂಡ-ನಾಯ-ಕಂಗೆ
ದಂಡ-ನಾಯ-ಕತ್ವ-ದಲ್ಲಿ
ದಂಡ-ನಾಯ-ಕನ
ದಂಡ-ನಾಯ-ಕ-ನದ್ದೇ
ದಂಡ-ನಾಯ-ಕ-ನನೂ
ದಂಡ-ನಾಯ-ಕ-ನನ್ನಾಗಿ
ದಂಡ-ನಾಯ-ಕ-ನನ್ನು
ದಂಡ-ನಾಯ-ಕ-ನಾಗಿ
ದಂಡ-ನಾಯ-ಕ-ನಾಗಿದ್ದ
ದಂಡ-ನಾಯ-ಕ-ನಾಗಿದ್ದಂತೆ
ದಂಡ-ನಾಯ-ಕ-ನಾಗಿದ್ದನು
ದಂಡ-ನಾಯ-ಕ-ನಾಗಿದ್ದ-ನೆಂದು
ದಂಡ-ನಾಯ-ಕ-ನಾಗಿದ್ದ-ನೆಂಬುದು
ದಂಡ-ನಾಯ-ಕ-ನಾಗಿದ್ದು
ದಂಡ-ನಾಯ-ಕ-ನಾಗಿ-ರ-ಬಹು-ದಾದ
ದಂಡ-ನಾಯ-ಕ-ನಾಗಿ-ರಲು
ದಂಡ-ನಾಯ-ಕ-ನಾಗಿ-ರುತ್ತಿದ್ದನು
ದಂಡ-ನಾಯ-ಕ-ನಾದ
ದಂಡ-ನಾಯ-ಕ-ನಿಕ್ಕ-ಯಣ್ಣನು
ದಂಡ-ನಾಯ-ಕ-ನಿ-ಗಿಂತ
ದಂಡ-ನಾಯ-ಕ-ನಿಗಿದ್ದ
ದಂಡ-ನಾಯ-ಕ-ನಿಗೂ
ದಂಡ-ನಾಯ-ಕ-ನಿಗೆ
ದಂಡ-ನಾಯ-ಕನು
ದಂಡ-ನಾಯ-ಕನೂ
ದಂಡ-ನಾಯ-ಕ-ನೂ-ಸೋಮ-ದಂಡ-ನಾಯಕ
ದಂಡ-ನಾಯ-ಕ-ನೆಂದು
ದಂಡ-ನಾಯ-ಕ-ನೆಂಬ
ದಂಡ-ನಾಯ-ಕ-ನೆನಿಸಿ-ದನು
ದಂಡ-ನಾಯ-ಕನೇ
ದಂಡ-ನಾಯ-ಕರ
ದಂಡ-ನಾಯ-ಕ-ರನ್ನು
ದಂಡ-ನಾಯ-ಕ-ರಲ್ಲಿ
ದಂಡ-ನಾಯ-ಕ-ರಾಗಲೀ
ದಂಡ-ನಾಯ-ಕ-ರಾಗಿ
ದಂಡ-ನಾಯ-ಕ-ರಾಗಿದ್ದ
ದಂಡ-ನಾಯ-ಕ-ರಾಗಿದ್ದರು
ದಂಡ-ನಾಯ-ಕ-ರಾಗಿದ್ದ-ರೆಂದು
ದಂಡ-ನಾಯ-ಕ-ರಾಗಿದ್ದ-ವರು
ದಂಡ-ನಾಯ-ಕ-ರಾಗಿದ್ದ-ವರೇ
ದಂಡ-ನಾಯ-ಕ-ರಾದ
ದಂಡ-ನಾಯ-ಕ-ರಿಂದ
ದಂಡ-ನಾಯ-ಕ-ರಿ-ಗಿಂತ
ದಂಡ-ನಾಯ-ಕ-ರಿಗೆ
ದಂಡ-ನಾಯ-ಕರು
ದಂಡ-ನಾಯ-ಕ-ರು-ಗಳ
ದಂಡ-ನಾಯ-ಕ-ರು-ಗಳನ್ನು
ದಂಡ-ನಾಯ-ಕ-ರು-ಗಳಾಗಿ
ದಂಡ-ನಾಯ-ಕ-ರು-ಗಳಾಗಿದ್ದ
ದಂಡ-ನಾಯ-ಕ-ರು-ಗಳಿ-ಗಿಂತ
ದಂಡ-ನಾಯ-ಕ-ರು-ಗಳು
ದಂಡ-ನಾಯ-ಕ-ರು-ಗಳೂ
ದಂಡ-ನಾಯ-ಕ-ರು-ದಂಡಾಧೀಶರು
ದಂಡ-ನಾಯ-ಕ-ರು-ಮಂತ್ರಿ-ಗಳು
ದಂಡ-ನಾಯ-ಕ-ರು-ಮಹಾಪ್ರಧಾನ
ದಂಡ-ನಾಯ-ಕರೂ
ದಂಡ-ನಾಯ-ಕ-ರೆಂದು
ದಂಡ-ನಾಯ-ಕ-ರೆಂಬ
ದಂಡ-ನಾಯ-ಕರೇ
ದಂಡ-ನಾಯ-ಕ-ವೀರಯ್ಯ
ದಂಡ-ನಾಯ-ಕಸು
ದಂಡ-ನಾಯ-ಕ-ಸು-ರಿಗೆ
ದಂಡ-ನಾಯ-ಕಿತಿ
ದಂಡ-ನಾಯ-ಕಿತ್ತಿ
ದಂಡ-ನಾಯ-ಕಿತ್ತಿಗೆ
ದಂಡ-ನಾಯ-ಕಿತ್ತಿಯ
ದಂಡ-ನಾಯ-ಕಿತ್ತಿ-ಯರ
ದಂಡ-ನಾಯ-ಕಿತ್ತಿಯು
ದಂಡ-ನಾಯನ
ದಂಡ-ನಾಯು-ಕನ
ದಂಡನ್ನು
ದಂಡ-ಯಾತ್ರೆ
ದಂಡ-ಯಾತ್ರೆ-ಗಳ
ದಂಡ-ಯಾತ್ರೆ-ಗಳಲ್ಲಿ
ದಂಡ-ಯಾತ್ರೆ-ಗಳಿಂದ
ದಂಡ-ಯಾತ್ರೆ-ಗಳಿಗೂ
ದಂಡ-ಯಾತ್ರೆಯ
ದಂಡ-ಯಾತ್ರೆ-ಯನ್ನು
ದಂಡ-ಯಾತ್ರೆ-ಯಲ್ಲಿ
ದಂಡ-ವನ್ನು
ದಂಡಾಧಿಪನದ್ದಲ್ಲ-ವೆಂದು
ದಂಡಾಧಿಪರೊಳತಿಶಯಂ
ದಂಡಾಧೀಶ
ದಂಡಾಧೀಶ-ದಾ-ವಾನ-ಲನೂ
ದಂಡಾಧೀಶನ
ದಂಡಾಧೀಶ-ನಾಗಿದ್ದು
ದಂಡಾಧೀಶನು
ದಂಡಾಧೀಶ-ನೆಂದು
ದಂಡಾಧೀಶರ
ದಂಡಿಗೆ
ದಂಡಿಗೆತ್ತಿ
ದಂಡಿನ-ಹಳ್ಳಿ
ದಂಡಿ-ನೊಡನೆ
ದಂಡು
ದಂಡು-ಅಹೋ-ಬಲ-ದೇವನ
ದಂಡೆ-ಗಳಿಗೂ
ದಂಡೆತ್ತಿ
ದಂಡೆತ್ತಿ-ಬಂದನು
ದಂಡೆತ್ತಿ-ಬಂದು
ದಂಡೆತ್ತಿ-ಹೋಗಿ
ದಂಡೆತ್ತಿ-ಹೋದ
ದಂಡೆತ್ತಿ-ಹೋದರು
ದಂಡೆಯ
ದಂಡೆಯ-ಗುಂಟ
ದಂಡೆ-ಯಲ್ಲಿ
ದಂಡೆಯಲ್ಲಿ-ರುವ
ದಂಡೇಶ
ದಂಡೇಶನು
ದಂಡೇಶನೇ
ದಂಣಾಯಕ
ದಂಣಾಯಕರ
ದಂಣಾಯಕ-ರಿಗೆ
ದಂಣಾಯಕರು
ದಂಣ್ನಾಯಕ
ದಂಣ್ನಾಯ-ಕ-ನನ್ನು
ದಂಣ್ನಾಯ-ಕನೂ
ದಂಣ್ನಾಯ-ಕರ
ದಂಣ್ನಾಯ-ಕರು
ದಂಪತಿ-ಗ-ಳಿಗೆ
ದಕ್ಷ
ದಕ್ಷ-ತೆ-ಗಳಿಂದ
ದಕ್ಷಿಣ
ದಕ್ಷಿಣಕ್ಕಿ-ರುವ
ದಕ್ಷಿಣಕ್ಕೂ
ದಕ್ಷಿ-ಣಕ್ಕೆ
ದಕ್ಷಿಣ-ಚಕ್ರ-ವರ್ತಿ
ದಕ್ಷಿ-ಣದ
ದಕ್ಷಿಣ-ದ-ಕಡೆಗೆ
ದಕ್ಷಿಣ-ದಲ್ಲಿ
ದಕ್ಷಿಣ-ಭಾಗ-ದಲ್ಲಿದ್ದ
ದಕ್ಷಿಣ-ಭಾಗ-ದಲ್ಲಿಯೂ
ದಕ್ಷಿಣ-ಭಾಗ-ದಲ್ಲಿ-ರುವ
ದಕ್ಷಿಣ-ಭಾರ-ತ-ದಲ್ಲೆಲ್ಲಾ
ದಕ್ಷಿಣ-ಭಾರ-ತ-ವನ್ನು
ದಕ್ಷಿಣ-ಭುಜಾ-ದಂಡ-ನೆನಿಸಿದ್ದ
ದಕ್ಷಿಣಾಪಥದ
ದಕ್ಷಿಣಾ-ಮೂರ್ತಿ
ದಗಂಡ-ಪೆಂಡಾರ
ದಟ್ಟ-ವಾದ
ದಡಗ
ದಡಗದ
ದಡಗ-ದಡಿಗ-ನ-ಕೆರೆ
ದಡಗ-ಳಲ್ಲೂ
ದಡ-ದಲ್ಲಿ
ದಡದ-ಹಳ್ಳಿ
ದಡಿಗ
ದಡಿಗ-ದೀಡಿಗನ
ದಡಿಗ-ನ-ಕೆರೆ
ದಡಿಗ-ನ-ಕೆರೆಗೆ
ದಡಿಗ-ನ-ಕೆರೆಯ
ದಡಿಗ-ನ-ಕೆರೆ-ಯ-ಇಂದಿನ
ದಡಿಗ-ವಾಡಿ
ದಡಿಗೇಶ್ವರ
ದಡಿ-ಘಟ್ಟ
ದಣ್ಡೆ
ದಣ್ಣಾಯ-ಕನ-ಪುರ
ದಣ್ಣಾಯ-ಕನು
ದಣ್ನಾಯ-ಕನೂ
ದಣ್ನಾಯ-ಕಿತಿ
ದತ್ತಂ
ದತ್ತಕ
ದತ್ತ-ಕ-ಪಡೆ-ದಳು
ದತ್ತ-ಕ-ಪಡೆ-ದ-ಳೆಂದು
ದತ್ತ-ಯಾಗಿ
ದತ್ತಿ
ದತ್ತಿ-ಕೊಟ್ಟಿ-ರುವುದು
ದತ್ತಿ-ಗಳ
ದತ್ತಿ-ಗಳನ್ನು
ದತ್ತಿ-ಗ-ಳನ್ನೂ
ದತ್ತಿ-ಗಳಾಗಿವೆ
ದತ್ತಿಗೆ
ದತ್ತಿ-ನೀಡ-ಲಾಗಿದೆ
ದತ್ತಿ-ನೀಡುತ್ತಾನೆ
ದತ್ತಿ-ಪಡೆದು
ದತ್ತಿ-ಬಿಟ್ಟ
ದತ್ತಿ-ಬಿಟ್ಟನು
ದತ್ತಿ-ಬಿಟ್ಟ-ನೆಂದು
ದತ್ತಿ-ಬಿಟ್ಟ-ನೆಂದೂ
ದತ್ತಿ-ಬಿಟ್ಟರು
ದತ್ತಿ-ಬಿಟ್ಟ-ರೆಂದು
ದತ್ತಿ-ಬಿಟ್ಟಾಗ
ದತ್ತಿ-ಬಿಟ್ಟಿದ್ದಾರೆ
ದತ್ತಿ-ಬಿಟ್ಟಿ-ನೆಂದು
ದತ್ತಿ-ಬಿಟ್ಟಿ-ರುವ
ದತ್ತಿ-ಬಿಡ-ಲಾಗಿದೆ
ದತ್ತಿ-ಬಿಡ-ಲಾಯಿತು
ದತ್ತಿ-ಬಿಡುತ್ತಾನೆ
ದತ್ತಿ-ಬಿಡುತ್ತಾರೆ
ದತ್ತಿ-ಬಿಡುತ್ತಾಳೆ
ದತ್ತಿ-ಬಿಡುವ
ದತ್ತಿ-ಬಿಡು-ವುದು
ದತ್ತಿಯ
ದತ್ತಿ-ಯನ್ನು
ದತ್ತಿ-ಯಾಗಿ
ದತ್ತಿ-ಯಾಗಿ-ಬಿಟ್ಟನು
ದತ್ತಿ-ಯಾಗಿ-ಬಿಟ್ಟ-ನೆಂದು
ದತ್ತಿಯು
ದತ್ತಿ-ಶಾ-ಸನ
ದತ್ತಿ-ಶಾ-ಸನ-ಗ-ಳೆಂದು
ದತ್ತಿ-ಹಾಕಿ-ಕೊಟ್ಟ-ನೆಂದು
ದತ್ತಿ-ಹಾಕಿ-ಕೊಟ್ಟಿರುತ್ತಾನೆ
ದತ್ತಿ-ಹಾಕಿ-ಕೊಡಲಾ-ಗಿತ್ತು
ದತ್ತಿ-ಹಾಕಿ-ಕೊಡುತ್ತಾನೆ
ದತ್ತಿ-ಹಾಕಿ-ಕೊಡುತ್ತಾರೆ
ದತ್ತಿ-ಹಾಕಿ-ಕೊಡುತ್ತಾಳೆ
ದತ್ತು
ದತ್ತು-ಪುತ್ರನೇ
ದದತಾ
ದನ-ಗೂರು
ದನು-ಗೂರು
ದಬಗ
ದಬ-ಗಾವುಡ
ದಬ್ಬಾ-ಳಿಕೆ-ಯನ್ನು
ದಮ್ಮಿ-ಸೆಟ್ಟಿ-ಯರ
ದಯಂಣ
ದಯಂಣ-ದೇವಣ್ಣ
ದಯಣ್ಣ
ದಯ-ಪಾಲಿಸಿ
ದಯ-ಪಾಲಿ-ಸಿದ್ದ
ದಯಾಂಬುಧಿ-ಸೋಮ
ದಯೆ-ಗೆಯ್ಯೆನ್ದು
ದಯೆಯ
ದಯೆ-ಯನ್ನು
ದರ-ಗಿರಿ-ತುಂಗ-ನಿಂದು-ಕುಮುದೋಜ್ವಳ-ಕೀರ್ತ್ತಿ
ದರಸ-ಗುಪ್ಪೆ-ಯಾಗಿ-ರುವ
ದರಸಿ-ಕುಪ್ಪೆ
ದರಿದ್ರರ
ದರಿ-ಯಾದೌಲತ್ನಲ್ಲಿ
ದರ್ಗಾಕ್ಕೆ
ದರ್ಗಾ-ದಲ್ಲಿ
ದರ್ಪ್ಪದ-ಳನ
ದರ್ವೇಷ-ನಿಗೆ
ದರ್ಶನ
ದರ್ಶನದ
ದರ್ಶನಾರ್ಥ-ವಾಗಿ
ದಳದ-ಳ-ವಾಗಿ
ದಳ-ಪತಿ
ದಳ-ಪತಿ-ಗಳ
ದಳ-ಪತಿ-ಗಳಲ್ಲಿ
ದಳ-ಪತಿ-ಗಳಲ್ಲೊಬ್ಬನು
ದಳ-ಪತಿ-ಗ-ಳಾದ
ದಳ-ಪತಿ-ಗಳು
ದಳ-ಭಾರ-ಸಹಿತ
ದಳವಾಯಿ
ದಳವಾಯಿ-ಗಳ
ದಳವಾಯಿ-ಗಳಿಂದ
ದಳವಾಯಿ-ಗಳು
ದಳವಿಟ್ಟಿರಿ-ದಲ್ಲಿ
ದಳೇ
ದವಳ-ವಾಯಿ
ದವೆ
ದಶಮೀ
ದಶವಂದ-ವನ್ನು
ದಶೇಕಾದಶ-ವರ್ಷ-ದಲ್ಲಿ
ದಾಖಲಿ-ಸ-ಲಾಗಿದೆ
ದಾಖಲಿ-ಸಿದೆ
ದಾಖಲಿಸಿವೆ
ದಾಖಲು-ಪತ್ರ
ದಾಖಲೆ
ದಾಖಲೆ-ಗಳ
ದಾಖಲೆ-ಗಳನ್ನು
ದಾಖಲೆ-ಗಳಲ್ಲಿ
ದಾಖಲೆ-ಯಾಗಿದೆ
ದಾಖಲೆ-ಯಾಗಿದ್ದು
ದಾಟಿ
ದಾಡಿಯ
ದಾಡಿಯ-ಗಡ್ಡದ
ದಾಡಿಯ-ಸೋಮೆಯ
ದಾದಾಜಿ
ದಾದಿ
ದಾದೋಜಿ
ದಾನ
ದಾನಕ್ಕೆ
ದಾನ-ಗಳನ್ನು
ದಾನ-ಗ-ಳನ್ನೂ
ದಾನ-ಗುಣ
ದಾನದ
ದಾನ-ದಂನ-ಪುರದ
ದಾನ-ದತ್ತಿ-ಗಳನ್ನು
ದಾನ-ದುನ್ನತಿ-ಯಿಂದ
ದಾನ-ದೊಳು
ದಾನ-ಧರ್ಮ-ಗಳನ್ನು
ದಾನ-ಧರ್ಮ್ಮದ
ದಾನ-ಬಿಟ್ಟಿದ್ದಾನೆ
ದಾನ-ಮದ್ಭುತಂ
ದಾನ-ವನ್ನು
ದಾನ-ವಾಗಿ
ದಾನ-ಶಾ-ಸಣ
ದಾನಶ್ರೇಯಾಂಸಂ
ದಾನಸ್ಯ
ದಾಮ
ದಾಮಣ್ಣ
ದಾಮಣ್ಣನು
ದಾಮಣ್ಣ-ನೆಂದು
ದಾಮನೂ
ದಾಮ-ನೆಂಬ
ದಾಮ-ನೆಯ್ದನೆ
ದಾಮ-ಪಯ್ಯ-ನನ್ನು
ದಾಮ-ಪಯ್ಯ-ನೆಂಬು-ವ-ನನ್ನು
ದಾಮ-ರಲೈಯಪೇಂದ್ರ
ದಾಮಾದ್
ದಾಮೋ-ದರ
ದಾಮೋದ-ರನ
ದಾಮೋದ-ರನು
ದಾಯಾದ
ದಾಯಾದಿ
ದಾಯಾದಿ-ಗಳು
ದಾಯಾದಿ-ಯಾಗಿ-ರ-ಬಹುದು
ದಾಯಾದ್ಯಕ್ಕೆ
ದಾಯಿಗ-ಬೇಂಟೆ-ಕಾರ
ದಾಯ್ಗರು
ದಾರಿ
ದಾರಿ-ಯಲ್ಲಿಯೇ
ದಾರುಸ್ಸಲ್ತನತ್
ದಾಳಿ
ದಾಳಿಂಅ
ದಾಳಿ-ಗಳಲ್ಲಿ
ದಾಳಿ-ಮಾಡಿ
ದಾಳಿಯ
ದಾಳಿ-ಯನ್ನು
ದಾಳಿ-ಯಲ್ಲಿ
ದಾವಹವಿ
ದಾವಾನಳ
ದಾಶ-ರಾಜ-ನಲ್ಲಿಗೆ
ದಾಸ-ನ-ದೊಡ್ಡಿ
ದಾಸ-ನ-ಪುರ
ದಾಸ-ಪ-ನಾಯ-ಕರ
ದಾಸಾನು-ದಾಸ-ನೆಂದು
ದಾಸೋಹಕ್ಕೆ
ದಾಸ್ತಾನು
ದಿಂಡಕ
ದಿಂಡಿಕ
ದಿಂಡಿಕಗ
ದಿಂಡಿಗ
ದಿಂಡಿಗನ
ದಿಂಡಿಗ-ನ-ಕೆರೆಯ
ದಿಂಡಿಗ-ನಾಡಿ-ಯರು
ದಿಂಡಿ-ಗನು
ದಿಂಡಿಗ-ಮಹಾಪ್ರಭುವೇ
ದಿಂಡಿಗ-ರಾಜ
ದಿಂಡಿಗ-ರಾಜನು
ದಿಂಡಿ-ಗರು
ದಿಂಡಿಗಲ್ಲಿನ
ದಿಂಮ-ರಾಜಯ್ಯನು
ದಿಕ್ಕಿಗೆ
ದಿಕ್ಕಿನಲ್ಲಿ
ದಿಕ್ಕಿನಿಂದ
ದಿಕ್ಕೆಟ್ಟನು
ದಿಗ್ವಿಜಯ
ದಿಗ್ವಿಜಯ-ಗಳನ್ನು
ದಿಗ್ವಿಜಯದ
ದಿಗ್ವಿಜಯಾರ್ಥ-ವಾಗಿ
ದಿಡಗ
ದಿಡಗದ
ದಿಡಗ-ವಾಗಿದೆ
ದಿಡಗವು
ದಿಡುಗ-ವನ್ನು
ದಿಣ್ಡಿಗ
ದಿಣ್ಡಿಗ-ಕೂಡ-ಲೂರು
ದಿಣ್ಡಿಗೋ
ದಿಣ್ಣೆಯ
ದಿನ
ದಿನಕ್ಕೆ
ದಿನ-ಗಳ
ದಿನ-ದಲಿ
ದಿನ-ದಿ-ನದ
ದಿನ-ಸಿ-ಗಳ
ದಿನಾ
ದಿನಾಂಕ
ದಿನಾಂಕದ
ದಿನ್
ದಿವಂಗ-ತನಾ-ದಾಗ
ದಿವಾ-ಕರ-ನೆನಿಸಿದ
ದಿವಾನರ
ದಿವಾನ್
ದಿವಿಜಲಲ-ನೆ-ಯರು
ದಿವ್ಯ
ದಿವ್ಯ-ದೇಶ-ವಾದ
ದಿವ್ಯ-ಮುನಿ-ವರ-ನೆಂದು
ದಿವ್ಯ-ವಾ-ಹನ
ದಿವ್ಯವ್ರತ
ದಿವ್ಯಶ್ರೀ
ದಿವ್ಯಶ್ರೀ-ಪಾದ-ಪದ್ಮದ
ದಿಶಾ-ಪಟ್ಟನುಂ
ದಿಸೆ-ಯಲ್ಲಿ
ದೀಕ್ಷಿತ
ದೀಕ್ಷಿತ್
ದೀಕ್ಷಿತ್ರ-ವರು
ದೀಕ್ಷೆ
ದೀಪಮಾಲೆ
ದೀಪಮಾಲೆ-ಕಂಬ
ದೀಪಮಾ-ಲೆಯ
ದೀಪಾಂಕುರನಂ
ದೀಪಿತ
ದೀರ್ಘ
ದೀರ್ಘ-ಕಾಲ
ದೀರ್ಘ-ಕಾಲ-ದಿಂದ
ದೀರ್ಘ-ವಾಗಿ
ದೀರ್ಘ-ವಾದ
ದೀವಿಗೆಗೆ
ದುಂಡು
ದುಂಡು-ವಿನ
ದುಂಡುವು
ದುಇಪಹ-ರರಾಉತು
ದುಗ್ಗಮಾ-ರನ
ದುಗ್ಗಮಾ-ರನು
ದುಗ್ಗಯ್ಯಂ
ದುಗ್ಗಲೆ
ದುಗ್ಗವೆ
ದುಗ್ಗವ್ವೆ-ಯರಿಗೆ
ದುದ್ದ
ದುದ್ದದ
ದುದ್ದ-ಮಲ್ಲ-ದೇವನ
ದುಬಿ-ಗಾವುಂಡಿ
ದುಮ್ಮೆ-ತನಕ
ದುಮ್ಮೆಯ
ದುಮ್ಮೆಯ-ನಾಯಕ
ದುಮ್ಮೆಯ-ನಾಯ-ಕನ
ದುಮ್ಮೆಯ-ನಾಯ-ಕನು
ದುಮ್ಮೆಯ-ನಾಯ-ಕರು
ದುಮ್ಮೆ-ವ-ರೆಗೆ
ದುರಂಧರ-ನೆಂದು
ದುರಂಧರೋ
ದುರಸ್ತೆ
ದುರಾದೃಷ್ಟವಶಾತ್
ದುರಿ-ತದೂರಂ
ದುರ್ಗ-ಗಳನ್ನು
ದುರ್ಗದ
ದುರ್ಗ-ದೊಳಗೆ
ದುರ್ಗ-ಮನು-ರವ-ಣೆಯಿಂ
ದುರ್ಗ-ವನಾಳುವಲ್ಲಿ
ದುರ್ಗ-ವನ್ನು
ದುರ್ಗಾಧಿ-ಪತಿ
ದುರ್ಗಾಧಿ-ಪತಿ-ಗಳು
ದುರ್ಜನ
ದುರ್ಬಲ
ದುರ್ಬಲ-ವಾಗಿದ್ದ
ದುರ್ಬಲ-ವಾಯಿತು
ದುರ್ಯೋಧನನು
ದುಷ್ಟಜ-ನದುರ್ಲಭ
ದುಷ್ಟ-ನಿರ್ಮೂಲ-ನೆ-ಗಾಗಿ
ದುಷ್ಟ-ಶಾರ್ದೂಲ-ಮರ್ದನಃ
ದುಸ್ಸಾಧ್ಯ-ವಾದ
ದೂರದ
ದೂರ-ದಲ್ಲಿ
ದೂರದಲ್ಲಿ-ರುವ
ದೃಢಂ
ದೃಢ-ಪಡಿಸಿ
ದೃಢಪಡಿ-ಸುತ್ತದೆ
ದೃಢಪಡಿ-ಸುತ್ತವೆ
ದೃಷ್ಟಿಕೋನ-ಗಳಿಂದ
ದೃಷ್ಟಿ-ಯಿಂದ
ದೆತ್ತಿದ
ದೆಲೆ-ಗೌಡ
ದೆವ-ರಾಜೊಡೆಯರ
ದೆಸೆ
ದೇಕವೆ-ದಂಡ-ನಾಯ-ಕಿತಿಯ
ದೇಕವ್ವೆ
ದೇಕೆಯ-ನಾಯಕ
ದೇಕೆಯ-ನಾಯ-ಕನ
ದೇಕೆಯ-ನಾಯ-ಕನು
ದೇಕೆಯ-ನಾಯ-ಕರ
ದೇಕೆಯ-ನಾಯ-ಕರು
ದೇಗುಲಗೌಣ್ಡಿ
ದೇಗುಲಗೌಣ್ಡಿ-ಯನ್ನು
ದೇಪಂಣೊಡೆ-ಯರ
ದೇಪಣ್ಣ
ದೇಪಯ-ನಾಮ-ಧೇಯೋ
ದೇಪಯಸ್ತು
ದೇಪಯ್ಯ
ದೇಪಯ್ಯ-ನನ್ನು
ದೇಪಯ್ಯನು
ದೇಮಲ-ದೇವಿ-ದೇವ-ಲ-ದೇವಿ
ದೇಮಲ-ದೇವಿಯು
ದೇಮಲಾ-ದೇವಿಯ
ದೇಮಲಾ-ದೇವಿ-ಯನ್ನು
ದೇಮಲಾ-ಪುರ-ವೆಂಬ
ದೇಮ-ಸಮುದ್ರ
ದೇಮಾಂಬಿ-ಕೆಯರ
ದೇಮಿಕಬ್ಬೆ
ದೇಮಿಕಬ್ಬೆ-ಯರು
ದೇಯಾತು
ದೇವ
ದೇವಕಿ
ದೇವ-ಕೀರ್ತಿ-ಪಂಡಿತರ
ದೇವ-ಕುಮಾರ
ದೇವಕ್ಷೇತ್ರ-ವಾದ
ದೇವ-ಚಂದ್ರನ
ದೇವ-ಚಂದ್ರನು
ದೇವಣ
ದೇವಣ್ಣನು
ದೇವತಾ
ದೇವ-ತಾ-ಗೃಹ-ಮಲ್ಲಿ-ಕಾರ್ಜುನ
ದೇವ-ತಾಗ್ರಾಮಂ
ದೇವ-ತಾ-ಪೂಜೆಗೆ
ದೇವ-ತಾ-ಮಂದಿರ
ದೇವ-ತೆ-ಗಳನ್ನು
ದೇವ-ತೆಯ
ದೇವ-ತೆ-ಯಾಗಿ-ರ-ಬಹುದು
ದೇವ-ದಂಡ-ನಾಯ-ಕ-ನಿಗೆ
ದೇವ-ದಾನ-ವನ್ನು
ದೇವದ್ವಿಜ-ಬಂಧುಮಿತ್ರ-ವರ್ಗ್ಗಾಣಾಂ
ದೇವನ
ದೇವನಂ
ದೇವ-ನ-ಣುಗಿನರ್ಕ್ಕರಿನ
ದೇವನಿಂ
ದೇವ-ನಿಗೆ
ದೇವನು
ದೇವನೂ
ದೇವ-ನೂ-ರನ್ನು
ದೇವ-ಪರ್ವ
ದೇವ-ಪುರಿ
ದೇವ-ಪೂಜೆ
ದೇವಪ್ಪ
ದೇವಪ್ಪ-ನಾಯಕ
ದೇವಪ್ಪ-ನಾಯ-ಕನ
ದೇವಪ್ಪ-ನಾಯ-ಕನು
ದೇವಪ್ಪ-ನಾಯ-ಕರು
ದೇವಪ್ಪನು
ದೇವಬ್ರಾಹ್ಮಣ
ದೇವ-ಭಟ್ಟ-ರಿಗೆ
ದೇವ-ಭು-ವನೇ
ದೇವ-ಮಾಂಬ
ದೇವ-ಮಾನ್ಯ-ವನ್ನು
ದೇವಯ್ಯನು
ದೇವರ
ದೇವ-ರ-ಕೊಂಡಾ
ದೇವ-ರ-ಕೊಂಡಾ-ರೆಡ್ಡಿ-ಯ-ವರ
ದೇವ-ರ-ಕೊಂಡಾ-ರೆಡ್ಡಿ-ಯ-ವರು
ದೇವ-ರ-ದರ್ಶನ
ದೇವ-ರನ್ನು
ದೇವ-ರಸ
ದೇವ-ರ-ಸ-ಗವುಡ
ದೇವ-ರ-ಸನ
ದೇವ-ರ-ಸ-ನನ್ನು
ದೇವ-ರ-ಸನು
ದೇವ-ರ-ಸರ
ದೇವ-ರ-ಸ-ರಿಗೆ
ದೇವ-ರ-ಸರು
ದೇವ-ರ-ಹಳ್ಳಿ
ದೇವ-ರ-ಹಳ್ಳಿ-ಗಳು
ದೇವ-ರ-ಹಳ್ಳಿಯ
ದೇವ-ರ-ಹಳ್ಳಿ-ಯನ್ನು
ದೇವ-ರಾಜ
ದೇವ-ರಾಜ-ದೇವ-ರಾಜ
ದೇವ-ರಾಜನ
ದೇವ-ರಾಜ-ನನ್ನು
ದೇವ-ರಾಜ-ನಿಗೆ
ದೇವ-ರಾಜನು
ದೇವ-ರಾಜ-ಪುರ
ದೇವ-ರಾಜ-ಪುರ-ವೆಂಬ
ದೇವ-ರಾಜ-ಭೂ-ಪಾಲನು
ದೇವ-ರಾಜ-ಮಹೀ-ಪಾಲ-ಕರು
ದೇವ-ರಾಜ-ಮಹೀ-ಪಾಲರು
ದೇವ-ರಾ-ಜಯ್ಯ
ದೇವ-ರಾಜಯ್ಯ-ದೇವನ
ದೇವ-ರಾಜಯ್ಯನ
ದೇವ-ರಾಜರ
ದೇವ-ರಾಜರು
ದೇವ-ರಾಜ-ವೊಡೆಯನು
ದೇವ-ರಾಜೇಂದ್ರ
ದೇವ-ರಾಜೇಂದ್ರ-ನಿಗೆ
ದೇವ-ರಾಜೊಡೆಯನೂ
ದೇವ-ರಾಜೊಡೆಯರ
ದೇವ-ರಾಜೊಡೆಯರು
ದೇವ-ರಾಜ್ಯಂ
ದೇವ-ರಾಯ
ದೇವ-ರಾಯನ
ದೇವ-ರಾಯ-ನನ್ನು
ದೇವ-ರಾಯ-ನಿಂದ
ದೇವ-ರಾಯ-ನಿಗೆ
ದೇವ-ರಾಯನು
ದೇವ-ರಾಯ-ಪಟ್ಟಣ
ದೇವ-ರಾಯಪ್ರೌಢ-ದೇವ-ರಾಯ
ದೇವ-ರಾಯ-ಮಹಾ-ರಾಯರ
ದೇವ-ರಿಗೆ
ದೇವರು
ದೇವ-ರು-ಗಳ
ದೇವ-ರು-ಗಳನ್ನು
ದೇವ-ರು-ಗ-ಳಿಗೆ
ದೇವ-ರೆಂದೇ
ದೇವ-ರೆಂಬ
ದೇವರೇ
ದೇವರ್
ದೇವರ್ವಲ್ಲ-ವನ್
ದೇವಲ
ದೇವ-ಲ-ಪುರ-ವಾಗಿ
ದೇವ-ಲ-ಮಹಾ-ಸಮುದ್ರ
ದೇವ-ಲಾ-ಪುರ
ದೇವ-ಲಾ-ಪುರದ
ದೇವ-ಲಾ-ಪುರ-ವನ್ನು
ದೇವ-ಲಾ-ಪುರವು
ದೇವ-ಲಾ-ಪುರವೂ
ದೇವ-ಲಾ-ಪುರ-ವೆಂಬ
ದೇವ-ಲಾ-ಪುರವೇ
ದೇವ-ವೃಂದ
ದೇವ-ಸತ್ತಿ
ದೇವಸ್ಥಾನ-ಗ-ಳಿಗೆ
ದೇವಾಜ-ಮಾಂಬ
ದೇವಾಜಮ್ಮ
ದೇವಾಜಮ್ಮಣ್ಣಿಯು
ದೇವಾ-ಪುರ
ದೇವಾರಾಧ್ಯರು
ದೇವಾಲ-ಪುರದ
ದೇವಾಲಯ
ದೇವಾಲಯಕ್ಕೆ
ದೇವಾಲಯ-ಗಳ
ದೇವಾಲಯ-ಗಳನ್ನು
ದೇವಾಲಯ-ಗಳಲ್ಲಿ
ದೇವಾಲಯ-ಗಳಾಗಿವೆ
ದೇವಾಲಯ-ಗಳಿಗೂ
ದೇವಾಲಯ-ಗ-ಳಿಗೆ
ದೇವಾಲಯ-ಗಳು
ದೇವಾಲಯ-ಗಳು-ಒಂದು
ದೇವಾಲಯ-ಗಳೂ
ದೇವಾಲಯದ
ದೇವಾಲಯ-ದಲ್ಲಿ
ದೇವಾಲಯ-ದಲ್ಲಿ-ರುವ
ದೇವಾಲಯನ್ನು
ದೇವಾಲಯನ್ನೂ
ದೇವಾಲಯ-ವನ್ನು
ದೇವಾಲಯ-ವನ್ನೂ
ದೇವಾಲಯ-ವಾಗಿದೆ
ದೇವಾಲಯ-ವಾದ
ದೇವಾಲಯ-ವಿದೆ
ದೇವಾಲಯ-ವಿದ್ದು
ದೇವಾಲಯವು
ದೇವಾಲಯವೂ
ದೇವಾಲಯ-ವೆಂದರೆ
ದೇವಾಲಯವೇ
ದೇವಾಲವೂ
ದೇವಿ
ದೇವಿಯ
ದೇವಿ-ಯನ್ನು
ದೇವಿ-ಯ-ರಿಗೆ
ದೇವೀರಮ್ಮಣ್ಣಿ
ದೇವೀರಮ್ಮನ
ದೇವೇಂದ್ರ
ದೇಶ
ದೇಶ-ಕಾವ-ಲುಗಾರ-ರಿದ್ದರು
ದೇಶ-ಗಳನ್ನು
ದೇಶದ
ದೇಶ-ದಲ್ಲಿ
ದೇಶ-ದಲ್ಲಿ-ರುವ
ದೇಶ-ರಾಜ್ಯ-ನಾಡು-ಮಂಡಲ
ದೇಶ-ವಾಗಿತ್ತೆಂದು
ದೇಶಶ್ರೀ
ದೇಶಸ್ಥಂ
ದೇಶಸ್ಯ
ದೇಶಾಂತರ
ದೇಶಾಂತ್ರಿ
ದೇಶಾಖ್ಯೇ
ದೇಶೇ
ದೇಸಾಯಿ
ದೇಸಾಯಿ-ಯ-ವರು
ದೇಸಿಯ
ದೇಸಿ-ಯಪ್ಪನ
ದೇಸಿ-ಯರು
ದೇಹ-ವನ್ನು
ದೈವ
ದೈವ-ಕೃಪೆ-ಯಿಂದ
ದೈವ-ದತ್ತ-ವಾಗಿ
ದೈವ-ದತ್ತ-ವಾದ
ದೈವ-ಭಕ್ತ-ರಾಗಿದ್ದರು
ದೊಡ್ಡ
ದೊಡ್ಡ-ಅ-ರಸಿ-ನ-ಕೆರೆ
ದೊಡ್ಡ-ಉಳು-ವರ್ತಿ
ದೊಡ್ಡ-ಕೃಷ್ಣ-ರಾಯರು
ದೊಡ್ಡ-ಗದ್ದ-ವಳ್ಳಿ
ದೊಡ್ಡ-ಗದ್ದ-ವಳ್ಳಿಯ
ದೊಡ್ಡ-ಗರು-ಡ-ನ-ಹಳ್ಳಿ
ದೊಡ್ಡ-ಗಾಡಿಗನ-ಹಳ್ಳಿ
ದೊಡ್ಡ-ಜಟಕ
ದೊಡ್ಡ-ಜಟಕಾ
ದೊಡ್ಡ-ದಾ-ಗಿದ್ದು
ದೊಡ್ಡದು
ದೊಡ್ಡ-ದೇವಯ್ಯನ
ದೊಡ್ಡ-ದೇವಯ್ಯನ-ವರು
ದೊಡ್ಡ-ದೇವ-ರಾಜ
ದೊಡ್ಡ-ದೇವ-ರಾಜನ
ದೊಡ್ಡ-ದೇವ-ರಾಜನು
ದೊಡ್ಡ-ದೇವ-ರಾಜನೂ
ದೊಡ್ಡ-ದೇವ-ರಾಜ-ರಿಗೆ
ದೊಡ್ಡ-ದೇವ-ರಾಯರು
ದೊಡ್ಡ-ದೊಂದು
ದೊಡ್ಡದ್ಯಾಮ-ಗೌಡ-ನಿಗೆ
ದೊಡ್ಡಪ್ಪ
ದೊಡ್ಡಪ್ಪಂದಿರು
ದೊಡ್ಡಪ್ಪ-ನೊಡನೆ
ದೊಡ್ಡ-ಬೆಟ್ಟದ
ದೊಡ್ಡ-ಮಸೀದಿ
ದೊಡ್ಡಮ್ಮ
ದೊಡ್ಡ-ಯಗಟಿ
ದೊಡ್ಡಯ್ಯ
ದೊಡ್ಡಯ್ಯ-ನ-ಹಳ್ಳಿ
ದೊಡ್ಡಯ್ಯನು
ದೊಡ್ಡ-ರಸಿ-ನ-ಕೆರೆ
ದೊಡ್ಡ-ವಡ್ಡಅರ-ಗುಡಿ
ದೊಡ್ಡ-ಹುಂಡಿ
ದೊಡ್ಡಾದಣ್ಣನ
ದೊಡ್ಡಿ
ದೊಡ್ಡಿ-ಘಟ್ಟ
ದೊಡ್ಡಿಯೇ
ದೊಡ್ಡೈಯ-ನ-ವರ
ದೊರಕಿದೆ
ದೊರಕಿದ್ದು
ದೊರಕಿ-ರುವ
ದೊರಕಿಲ್ಲ
ದೊರಕಿವೆ
ದೊರಕುತ್ತವೆ
ದೊರಕುವ
ದೊರಭಕ್ಕೆರೆ
ದೊರೆ
ದೊರೆ-ಗಳ
ದೊರೆ-ಗಳಲ್ಲಿ
ದೊರೆ-ಗಳಾಗಿ-ರ-ಬಹುದು
ದೊರೆ-ಗಳು
ದೊರೆತ
ದೊರೆ-ತ-ನದಿ-ರ-ವನ್ನು
ದೊರೆ-ತವು
ದೊರೆ-ತಿದ್ದು
ದೊರೆ-ತಿ-ರುವ
ದೊರೆ-ತಿವೆ
ದೊರೆತು
ದೊರೆ-ಯ-ದಿ-ರು-ವುದು
ದೊರೆ-ಯದೇ
ದೊರೆ-ಯವ
ದೊರೆ-ಯಿತೆಂದು
ದೊರೆಯು
ದೊರೆ-ಯುತ್ತದೆ
ದೊರೆ-ಯುತ್ತವೆ
ದೊರೆ-ಯುತ್ತಿತ್ತು
ದೊರೆ-ಯುವ
ದೊರೆ-ಯು-ವು-ದಿಲ್ಲ
ದೊರೆ-ವರು
ದೋರ
ದೋರನು
ದೋರ-ಸಮುದ್ರ
ದೋರ-ಸಮುದ್ರಕ್ಕೆ
ದೋರ-ಸಮುದ್ರದ
ದೋರ-ಸಮುದ್ರ-ದ-ದಲ್ಲಿದ್ದನು
ದೋರ-ಸಮುದ್ರ-ದಲು
ದೋರ-ಸಮುದ್ರ-ದಲ್ಲಿ
ದೋರ-ಸಮುದ್ರ-ದಲ್ಲೇ
ದೋರ-ಸಮುದ್ರ-ದಿಂದ
ದೋರ-ಸಮುದ್ರ-ವನ್ನು
ದೋರ-ಸಮುದ್ರ-ಹಳೆಯ-ಬೀಡು
ದೋರ-ಸಮುದ್ರಾಖ್ಯಾಂ
ದೋರ-ಸಮುದ್ರಿಂದ
ದೋಸ್ತಂಭ-ದೊಳು
ದೌರ್ಬಲ್ಯ-ಗಳ
ದ್ಧರಣಂ
ದ್ಯಾವಣ್ಣನು
ದ್ಯಾವಣ್ಣಹೆಮ್ಮಾಡಿ-ಯಣ್ಣ
ದ್ಯಾವ-ರ-ಹಳ್ಳಿ
ದ್ಯಾವ-ರ-ಹಳ್ಳಿ-ಯಲ್ಲೂ
ದ್ರಮಿಳ
ದ್ರಮಿಳ-ಸಂಘದ
ದ್ರಾವಿಡಾನ್ವ-ಯದ
ದ್ರುವನು
ದ್ರೋಹಘ-ರಟ್ಟ
ದ್ರೋಹಘ-ರಟ್ಟಂ
ದ್ರೋಹಘ-ರಟ್ಟ-ನೆಂಬ
ದ್ವಾಪರ-ಯುಗದ
ದ್ವಾರ-ಕೆ-ಯಿಂದ
ದ್ವಾರ-ದಲ್ಲಿರುವ
ದ್ವಾರ-ಪಕ್ಷದ
ದ್ವಾರ-ಸಮುದ್ರ-ದಿಂದ
ದ್ವಾರಾ-ವತಿ
ದ್ವಾರಾ-ವತಿ-ನಾಥನ
ದ್ವಾವೇತಾವಥ
ದ್ವಿಗುಣ-ಮತ್ರಿ-ಗುಣಂಚತುರ್ಗಣಂ
ದ್ವಿಗುಣೀಕೃತ
ದ್ವಿಜ-ವಂಶ-ತಿಲಕನೂ
ದ್ವಿತೀ-ಯವಿಭವಂ
ದ್ವೀಪ-ವನ್ನು
ದ್ವೀಪ-ವಿದ್ದು
ದ್ವೇಷ-ವನ್ನು
ಧಕ್ಕೆ-ಯಾದ
ಧನ
ಧನಂಜಯ-ಪುರ-ವನ್ನಾಗಿ
ಧನಂಜಯ-ರಾಯ
ಧನಂಜಯ-ರಾಯ-ವೊಡೆಯನು
ಧನಗೂ-ರನ್ನು
ಧನಗೂ-ರಿಗೆ
ಧನಗೂರಿನ
ಧನಗೂರು
ಧನಗೂರುಸ್ಥಳದ
ಧನವೆಲ್ಲ
ಧನ-ಸಹಾಯ-ವನ್ನು
ಧನುಗೂ-ರನ್ನು
ಧನುಗೂರು
ಧನುರು
ಧನುರ್ಮಾಸ
ಧನುರ್ವಿದ್ಯಾ-ಪರಿಣತರುಂ
ಧರ-ಣಿ-ದೇವ-ತಾ-ರುದ್ರನುಂ
ಧರ-ಣೀತೇಃ
ಧರ-ಣೀ-ದೇವ
ಧರ-ಣೀ-ವರಾಹ
ಧರಾಮ-ರೋತ್ತಂಸ-ನಾಗಿದ್ದ-ನೆಂದು
ಧರಾ-ರಾಜ್ಯ-ವಾಳುತ್ತಿದ್ದಾಗ
ಧರಿ-ಸದೇ
ಧರಿಸಿ
ಧರಿಸಿದ
ಧರಿಸಿ-ದಂತೆ
ಧರಿಸಿ-ದನು
ಧರಿಸಿ-ದ-ನೆಂದು
ಧರಿ-ಸಿದ್ದ
ಧರಿಸಿದ್ದನು
ಧರಿಸಿದ್ದ-ನೆಂದು
ಧರಿಸಿದ್ದನ್ನು
ಧರಿಸಿದ್ದ-ರಿಂದ
ಧರಿಸಿದ್ದ-ರೆಂದೂ
ಧರಿಸಿದ್ದಾನೆ
ಧರಿಸಿ-ರುವ
ಧರಿ-ಸುತ್ತಿದ್ದ
ಧರಿ-ಸುವ
ಧರೆ
ಧರೆ-ಕೂರ್ತ್ತುಕೀರ್ತ್ತಿಕುಂ
ಧರೆ-ಗೆಲ್ಲಂ
ಧರೆ-ತನ್ನಂ
ಧರೆ-ಯನ್ನು
ಧರೆ-ಯೊಳ್
ಧರ್ಮ
ಧರ್ಮಂಗಳನ್ನು
ಧರ್ಮ-ಕಾರ್ಯ-ಗಳನ್ನು
ಧರ್ಮ-ಕಾರ್ಯ-ಗ-ಳಿಗೆ
ಧರ್ಮ-ಕಾರ್ಯ-ಗಳು
ಧರ್ಮಕ್ಕೆ
ಧರ್ಮ-ಗಳನ್ನು
ಧರ್ಮದ
ಧರ್ಮ-ದಿಂದ
ಧರ್ಮ-ದೊಳು
ಧರ್ಮ-ಪತ್ನಿ
ಧರ್ಮ-ಪ-ರಾಯ-ಣ-ರಾದ
ಧರ್ಮಪ್ರಸಂಗದ
ಧರ್ಮ-ಬುದ್ಧಿ
ಧರ್ಮ-ಬೊಜ್ಜ-ವಿಷ್ಣು-ವರ್ಧನ
ಧರ್ಮ-ಭೂಮಿ-ಯಾಗಿದ್ದ
ಧರ್ಮ-ಮಹಾ-ಧಿ-ರಾಜ
ಧರ್ಮ-ಮಹಾ-ಧಿ-ರಾಜ-ರೆಂದು
ಧರ್ಮ-ಮಹಾ-ರಾಜಾಧಿ-ರಾಜ
ಧರ್ಮ-ಮಹಾ-ರಾಜಾಧಿ-ರಾಜನು
ಧರ್ಮ-ವನ್ನು
ಧರ್ಮ-ವನ್ನೇ
ಧರ್ಮ-ವಾಗ-ಬೇಕೆಂದು
ಧರ್ಮ-ವಾಗ-ಲೆಂದು
ಧರ್ಮ-ವಾಗಿ
ಧರ್ಮವು
ಧರ್ಮವೇ
ಧರ್ಮ-ಶಾಸ್ತ್ರಕ್ಕನು-ಗುಣ-ವಾಗಿ
ಧರ್ಮಶ್ಚ
ಧರ್ಮ-ಸತ್ರ-ಗಳನ್ನು
ಧರ್ಮಾಗ್ರಹಾರ-ವಾಗಿ
ಧರ್ಮಾನುಯಾಯಿ-ಗ-ಳಾದ
ಧರ್ಮಾ-ಪುರ
ಧರ್ಮಾ-ಪುರ-ವನ್ನು
ಧರ್ಮಾ-ಪುರ-ವೆಂಬ
ಧರ್ಮೇಶ್ವರ-ದೇವರೇ
ಧರ್ಮ್ಮಪ್ರತಿ-ಪಾಳಕ-ರು-ಮಪ್ಪ
ಧರ್ಮ್ಮಶೀಲಾಕ್ಕಮಾ-ಗರ್ಭಶುಕ್ತಿಮುಕ್ತಾಫಲಾತ್ಮನಃ
ಧವಳಂಕ-ಭೀಮ
ಧವಳಾಂಕ-ಭೀಮ
ಧಾನ್ಯದ
ಧಾನ್ಯ-ವನ್ನು
ಧಾರಾಧತ್ತ-ವಾಗಿ
ಧಾರಾ-ನ-ಗರ-ವನ್ನು
ಧಾರಾ-ಪುರ-ಗಳನ್ನು
ಧಾರಾ-ಪುರ-ದಿಂದ
ಧಾರಾ-ಪೂರ್ವಕ
ಧಾರಾ-ಪೂರ್ವ-ಕ-ವಾಗಿ
ಧಾರಿಣಿಯು
ಧಾರೆ
ಧಾರೆಯಂ
ಧಾರೆ-ಯನಾತ್ಮ
ಧಾರೆ-ಯ-ನೆರೆದು
ಧಾರೆ-ಯನೆರೆಸಿ
ಧಾರೆ-ಯೆರೆದು
ಧಾರೆ-ಯೆರೆದು-ಕೊಡುತ್ತಾರೆ
ಧಾರೆ-ಯೆರೆಸಿ
ಧಾರ್ಮಿಕ
ಧಾರ್ಮಿಸ್ಥಳ-ವಾದ
ಧಾವಿ-ಸಿದ
ಧೀರ
ಧುರಂಧರಂಮ-ಮಾತ್ಯ
ಧುರದ
ಧುರೀಣ
ಧುರೀಣಸ್ಯ
ಧುರ್ಮ್ಮಣ್ಣ
ಧೂಳೀಪಟ-ಮಾಡಿ
ಧೃತ-ಸತ್ಯ-ವಾಕ್ಯಂ
ಧೈರ್ಯಸುರ-ಗಾತ್ರ
ಧ್ರುವ
ಧ್ರುವ-ಉಂಡಿಗೆಯ
ಧ್ರುವನ
ಧ್ರುವನು
ಧ್ವಂಸ-ಮಾಡಿದನು
ಧ್ವಜ-ವನ್ನು
ಧ್ವಜಸ್ಥಂಭ-ವನ್ನು
ಧ್ವಜಿನೀ-ಪತಿ
ಧ್ವಜಿನೀ-ಪತಿ-ಯಾದ
ಧ್ವನಿಯೂ
ನಂಗಲಿ
ನಂಗ-ಲಿಯ
ನಂಗಲಿ-ಯನ್ನು
ನಂಜನ-ಗೂಡಿನ
ನಂಜನ-ಗೂಡು
ನಂಜಯ
ನಂಜಯ-ದೇವ-ರಿಗೆ
ನಂಜ-ರಾಜ
ನಂಜ-ರಾಜನ
ನಂಜ-ರಾಜನು
ನಂಜ-ರಾಜ-ನೆಂದೇ
ನಂಜ-ರಾಜನೇ
ನಂಜ-ರಾ-ಜಯ್ಯ
ನಂಜ-ರಾಜಯ್ಯನ
ನಂಜ-ರಾಜಯ್ಯನು
ನಂಜ-ರಾಜ-ಸಮುದ್ರ-ವೆಂಬ
ನಂಜ-ರಾಜೈಯ್ಯನ-ವರ
ನಂಜ-ರಾಜೊಡೆಯನ
ನಂಜ-ರಾಯ
ನಂಜ-ರಾಯಜ
ನಂಜ-ರಾಯನು
ನಂಜ-ರಾಯ-ಪಟ್ಟಣದ
ನಂಜ-ರಾಯೊಡೆ-ಯನ
ನಂಜ-ರಾಯ್ಯ-ನನ್ನು
ನಂಜ-ವೊಡೇರಿಗೆ
ನಂಜಿನ
ನಂಜೀ-ನಾಥ-ನಿಗೆ
ನಂಜುಂಡಸ್ವಾಮಿ
ನಂಜುಂಡೇಶ್ವರ
ನಂಜುಡೇಶ್ವರ
ನಂಜೇ-ಗವುಡ
ನಂಜೇ-ಹೆಬ್ಬಾ-ರುವ-ನಿಗೆ
ನಂಟ-ರಂಗ-ರ-ಕನುಂ
ನಂತರ
ನಂತರದ
ನಂತರ-ದಲ್ಲಿ
ನಂತರ-ವಷ್ಟೇ
ನಂತರವೂ
ನಂತರವೇ
ನಂದ-ಗಿರಿ
ನಂದ-ಗಿರಿ-ನಾಥ
ನಂದನ
ನಂದಾದೀಪ
ನಂದಾದೀ-ಪಕ್ಕೆ
ನಂದಾದೀ-ವಿಗೆ
ನಂದಾದೀ-ವಿಗೆಗೆ
ನಂದಾದೀ-ವಿಗೆಯು
ನಂದಿ
ನಂದಿನ್ತ-ವರಿ-ವರ-ಳವೆ
ನಂದಿ-ಯಾ-ಲದ
ನಂದಿ-ವರ್ಮನ
ನಂದಿ-ವರ್ಮ-ನಿಂದ
ನಂದೀಧ್ವಜ-ವನ್ನು
ನಂದ್ಯಾಲ
ನಂದ್ಯಾ-ಲದ
ನಂನಿಯ-ಮೇರು
ನಂನಿಯ-ಮೇರು-ವನ್ನು
ನಂಬ-ಲಾಗಿದೆ
ನಂಬಿ
ನಂಬಿಕೆ
ನಂಬಿ-ನಾಯ-ಕ-ನ-ಹಳ್ಳಿ
ನಂಬಿ-ನಾಯ-ಕ-ನ-ಹಳ್ಳಿಯ
ನಂಬಿ-ಪಿಳ್ಳೆ
ನಂಬು-ಗೆಯ
ನಕರ-ಗಳ
ನಖರ
ನಖರ-ಗಳೆಲ್ಲರೂ
ನಖರೇಶ್ವರ
ನಗರ
ನಗರಕ್ಕೆ
ನಗರದ
ನಗರ-ದಲ್ಲಿ
ನಗರ-ವಿ-ರುವ
ನಗರುರ
ನಗರೂರು
ನಗರೇರ
ನಗರೇಶ್ವರ-ದೇವ-ರಿಗೆ
ನಗುಲನ
ನಗುಲನ-ಹಳ್ಳಿ
ನಗು-ವನ-ಹಳ್ಳಿ
ನಟಿ
ನಟಿ-ವೆಂಕಟಪ್ಪ-ನಾಯಕ
ನಟ್ಟ-ಕಲ್ಲು
ನಡದ
ನಡ-ಸಿದಂ
ನಡು-ನಾಡಿ-ನಲ್ಲಿ
ನಡುವಣ
ನಡು-ವಿನ
ನಡುವೆ
ನಡುವೆಯೂ
ನಡೆದ
ನಡೆ-ದಲ್ಲಿ
ನಡೆದವು
ನಡೆದಾಗ
ನಡೆದಿದೆ
ನಡೆದಿರ-ಬಹು-ದಾದ
ನಡೆದಿರ-ಬಹುದು
ನಡೆದಿರ-ಬಹು-ದೆಂದು
ನಡೆದಿ-ರ-ಬೇಕೆಂದು
ನಡೆ-ದಿ-ರುವ
ನಡೆ-ದಿ-ರು-ವು-ದನ್ನು
ನಡೆ-ದಿ-ರು-ವುದು
ನಡೆದು
ನಡೆದು-ದರ
ನಡೆ-ದುದು
ನಡೆದು-ಬಂದು
ನಡೆದು-ಹೋಗ-ಬೇಕಾಯಿತು
ನಡೆ-ಯದೇ
ನಡೆಯ-ಬೇಕಾಗಿದೆ
ನಡೆ-ಯಲಿ-ಎಂದು
ನಡೆಯಿತು
ನಡೆ-ಯಿತೆಂದು
ನಡೆಯು
ನಡೆಯುತ
ನಡೆಯುತ್ತ-ದೆಂದು
ನಡೆಯುತ್ತಲೇ
ನಡೆಯುತ್ತಿತ್ತು
ನಡೆಯುತ್ತಿದ್ದ
ನಡೆಯುತ್ತಿದ್ದು-ದನ್ನು
ನಡೆಯುತ್ತಿ-ರುವಾಗಲೇ
ನಡೆ-ಯುವ
ನಡೆಯು-ವಂತೆ
ನಡೆವಂದು
ನಡೆ-ವಲ್ಲಿ
ನಡೆ-ವಲ್ಲಿ-ಗೆ-ಪುರ-ದಲ್ಲಿದ್ದ
ನಡೆಸ-ತೊಡಗಿ-ದರು
ನಡೆಸತೊಡಗಿ-ದಳು
ನಡೆಸ-ಬೇಕು
ನಡೆ-ಸ-ಲಾಗಿದೆ
ನಡೆಸ-ಲಿಲ್ಲ
ನಡೆ-ಸಲು
ನಡೆಸಿ
ನಡೆಸಿ-ಕೊಂಡು
ನಡೆಸಿ-ಕೊಡ-ಬೇಕೆಂದು
ನಡೆಸಿದ
ನಡೆಸಿ-ದನು
ನಡೆಸಿ-ದ-ನೆಂದು
ನಡೆಸಿ-ದ-ನೆಂದೂ
ನಡೆಸಿ-ದರು
ನಡೆಸಿ-ದರೂ
ನಡೆಸಿ-ದ-ರೆಂದು
ನಡೆಸಿದ್ದಾರೆಂದು
ನಡೆ-ಸಿದ್ದು
ನಡೆಸಿ-ರುವ
ನಡೆಸಿ-ರುವು
ನಡೆಸಿ-ರು-ವುದು
ನಡೆ-ಸುತ್ತ
ನಡೆ-ಸುತ್ತಾ
ನಡೆ-ಸುತ್ತಿದ-ರೆಂಬುದು
ನಡೆ-ಸುತ್ತಿದ್ದ
ನಡೆ-ಸುತ್ತಿದ್ದಂತೆ
ನಡೆ-ಸುತ್ತಿದ್ದನು
ನಡೆ-ಸುತ್ತಿದ್ದ-ನೆಂದು
ನಡೆ-ಸುತ್ತಿದ್ದ-ನೆಂದೂ
ನಡೆ-ಸುತ್ತಿದ್ದ-ನೆಂಬುದು
ನಡೆ-ಸುತ್ತಿದ್ದರು
ನಡೆ-ಸುತ್ತಿದ್ದರೂ
ನಡೆ-ಸುತ್ತಿದ್ದ-ರೆಂದು
ನಡೆ-ಸುತ್ತಿದ್ದ-ರೆಂಬುದು
ನಡೆ-ಸುತ್ತಿದ್ದರೇ
ನಡೆ-ಸುತ್ತಿದ್ದಾಗ
ನಡೆ-ಸುತ್ತಿದ್ದಿರ-ಬಹುದು
ನಡೆ-ಸುತ್ತಿದ್ದು
ನಡೆ-ಸುತ್ತಿದ್ದು-ದನ್ನು
ನಡೆ-ಸುತ್ತಿದ್ದು-ದ-ರಿಂದ
ನಡೆ-ಸುತ್ತಿರುತ್ತಾನೆ
ನಡೆ-ಸುತ್ತೇ-ವೆಂದು
ನಡೆ-ಸುವ
ನಡೆಸು-ವಂತಹ
ನಡೆಸು-ವುದು
ನದಿ
ನದಿ-ಗಳ
ನದಿ-ಗಳನ್ನು
ನದಿ-ಗಳಾಗಿವೆ
ನದಿ-ಗಳು
ನದಿಗೆ
ನದಿಯ
ನದಿ-ಯನ್ನು
ನದಿ-ಯನ್ನೂ
ನದಿ-ಯ-ಮಡು
ನದಿ-ಯಲ್ಲಿ
ನದಿ-ಯ-ವರೆಗೂ
ನದಿ-ಯಾಚೆ
ನದಿಯು
ನನ್ದನ-ನೊಲವಿಂ
ನನ್ನವ್ವೆ
ನನ್ನಿ
ನನ್ನಿ-ಕಂದರ್ಪ-ನೆಂಬು-ವ-ವನು
ನನ್ನಿಗ
ನನ್ನಿ-ನೊಳಂಬನು
ನನ್ನಿ-ಮಳಲೂರಂ
ನನ್ನಿಯ
ನನ್ನಿ-ಯ-ಗಂಗ
ನನ್ನಿ-ಯ-ಮೇರು
ನನ್ನಿ-ಯ-ಸೇ-ಕರ
ನಮಗೆ
ನಮಸ್ಕಾರ-ವನ್ನು
ನಮೂ-ದಾಗಿ-ರುವ
ನಮೂ-ದಾಗಿ-ರು-ವಂತೆ
ನಮೂದಿ-ಸ-ಲಾಗಿದೆ
ನಮೂದಿ-ಸಿದ್ದು
ನಮೂದಿಸಿರ-ಬಹುದು
ನಮೂದಿಸಿ-ರು-ವು-ದಿಲ್ಲ
ನಮೂದಿಸಿಲ್ಲ
ನಮ್ಮ
ನಮ್ಮಾಳ್ವಾರ್
ನಯ-ಕೀರ್ತಿ
ನಯ-ಣದ
ನಯ-ಧೀರ
ನಯ-ಧೀರ-ರೊಡ-ಗೂಡಿ
ನಯೋಂನತೇಃ
ನರಅ-ಸೀ-ಪುರ
ನರಗ
ನರಗಲು
ನರಗುಂದ
ನರಣ-ನಾ-ರಾಯಣ
ನರ-ಪತಿ
ನರ-ಪತಿ-ಬೆನ್ನೊಳಿರ್ದೊನಿದಿರಾಂತುದು
ನರಭಕ್ಷಕ
ನರಮನ-ಕಟ್ಟೆ
ನರಸ
ನರಸಂಣ-ನಾಯ್ಕರು
ನರ-ಸಣ್ಣ
ನರ-ಸಣ್ಣ-ನಾಯ-ಕರ
ನರಸನ
ನರಸ-ನಾಯಕ
ನರಸ-ನಾಯ-ಕನ
ನರಸ-ನಾಯ-ಕ-ನಿಗೆ
ನರಸ-ನಾಯ-ಕನು
ನರಸ-ನಿಗೆ
ನರಸನು
ನರಸಯ್ಯ
ನರಸಯ್ಯನು
ನರಸ-ರಾಜ
ನರಸ-ರಾಜನ
ನರಸ-ರಾಜನು
ನರಸ-ರಾಜ-ನೆಂದೇ
ನರಸ-ರಾಜರ
ನರಸ-ರಾಜೊಡೆಯ
ನರಸ-ರಾಜೊಡೆಯರ
ನರ-ಸಾಕ್ಷಿ-ಯಾಗಿ-ರುತ್ತಾರೆ
ನರಸಾ-ವನಿ-ಪಾಲಜ
ನರ-ಸಿಂಗ
ನರ-ಸಿಂಗಣ್ಣ-ಗ-ಳಿಗೆ
ನರ-ಸಿಂಗ-ದೇವ
ನರ-ಸಿಂಗ-ದೇವನು
ನರ-ಸಿಂಗ-ದೇವರು
ನರ-ಸಿಂಗನ
ನರ-ಸಿಂಗ-ನಾಯಕ
ನರ-ಸಿಂಗ-ನಾಯಕಂ
ನರ-ಸಿಂಗ-ನಾಯ-ಕನ
ನರ-ಸಿಂಗ-ನಾಯ-ಕನು
ನರ-ಸಿಂಗಯ್ಯ-ದೇವ
ನರ-ಸಿಂಗಯ್ಯನ
ನರ-ಸಿಂಗ-ರಾಜ-ವೊಡೆಯರ
ನರ-ಸಿಂಗ-ರಾಯ
ನರ-ಸಿಂಗ-ವರ್ಮ
ನರ-ಸಿಂಗ-ವರ್ಮನು
ನರ-ಸಿಂಹ
ನರ-ಸಿಂಹ-ದೇವರ
ನರ-ಸಿಂಹ-ದೇವ-ರಿಗೆ
ನರ-ಸಿಂಹನ
ನರ-ಸಿಂಹ-ನಂತೆಯೇ
ನರ-ಸಿಂಹ-ನ-ಕಾಲ-ದಲ್ಲಿ
ನರ-ಸಿಂಹ-ನ-ಕಾಲ-ದ-ವ-ರೆಗೆ
ನರ-ಸಿಂಹ-ನನ್ನು
ನರ-ಸಿಂಹ-ನನ್ನೂ
ನರ-ಸಿಂಹ-ನ-ರಪಾಳಂ
ನರ-ಸಿಂಹ-ನ-ರಸು-ಗೆಯ್ಯುತ್ತಿರ್ದ್ದಂ
ನರ-ಸಿಂಹ-ನಲ್ಲಿ
ನರ-ಸಿಂಹ-ನಿಗೂ
ನರ-ಸಿಂಹ-ನಿಗೆ
ನರ-ಸಿಂಹನು
ನರ-ಸಿಂಹನೂ
ನರ-ಸಿಂಹ-ಮಹಾ-ರಾಯರು
ನರ-ಸಿಂಹ-ರನ್ನು
ನರ-ಸಿಂಹ-ರಾಯನ
ನರ-ಸಿಂಹ-ರಾಯರು
ನರ-ಸಿಂಹ-ವರ್ಮನ
ನರ-ಸಿಂಹ-ವರ್ಮನೂ
ನರ-ಸಿಂಹ-ವರ್ಮ್ಮ-ನೋಡಿದ
ನರ-ಸಿಂಹಸ್ವಾಮಿ
ನರ-ಸಿಂಹಸ್ವಾಮಿಗೆ
ನರ-ಸಿಂಹಸ್ವಾಮಿಯ
ನರ-ಸಿಂಹಸ್ವಾಮಿ-ಯ-ವರ
ನರ-ಸಿಂಹಾ-ಚಾರ್ಯರು
ನರ-ಸಿಂಹೋರ್ವ್ವೀಶನ
ನರಸೀ-ಪುರ
ನರಸೀ-ಪುರದ
ನರಸೇಂದ್ರ-ನೆಂಬ
ನರಿ
ನರಿ-ಹಳ್ಳಿ
ನರಿ-ಹಳ್ಳಿ-ಯಲ್ಲೂ
ನರೆಗನ-ಹಳ್ಳಿ
ನಲ-ಗೌಡ
ನಲ-ವತ್ತು
ನಲ್ಲ-ತಂಬಿಯು
ನಳನ-ಹು-ಷಾದಿ-ಗಳಂತೆ
ನಳ-ಮಾರುಡು
ನಳ-ಸಂವತ್ಸರ
ನಳ-ಸಂವತ್ಸರದ
ನಳ-ಸಂವತ್ಸರವು
ನಳಿನ-ಕೆರೆ-ಯನ್ನು
ನವಗ್ರಹ-ಗ-ಳೆಂದು
ನವ-ದಂಡ-ನಾಯ-ಕ-ರು-ಗ-ಳೆಂದು
ನವದಣ್ಣಾಯಕ-ರೆಂಬ
ನವ-ನಿಧಿ-ಕುಲ-ಪರ್ವ-ತದ
ನವ-ರಂಗದ
ನವ-ರತ್ನ-ಕಿರೀಟ-ವನ್ನು
ನವಲೆ-ನಾಡು
ನವ-ಶಿಲಾ-ಯುಗದ
ನವಾಬ
ನವಾಬ್
ನವಿಲು
ನವೀ-ಕರಿ-ಸಿದನು
ನವೆಂಬರ್
ನವ್ಯ
ನವ್ವಾಬ್
ನಾಕಣ
ನಾಕ-ನಿಪ್ಪ
ನಾಕಯ್ಯನು
ನಾಕ-ಹಳ್ಳಿ
ನಾಗ
ನಾಗಂಣ
ನಾಗಂಣ-ವೊಡೆಯ
ನಾಗಂಣ-ವೊಡೆಯನು
ನಾಗಂಣ್ಣ
ನಾಗಣ್ಣ
ನಾಗಣ್ಣನ
ನಾಗಣ್ಣ-ನ-ವರು
ನಾಗಣ್ಣನು
ನಾಗಣ್ಣನ್ನನು
ನಾಗ-ದೇವ
ನಾಗ-ದೇವನ
ನಾಗ-ದೇವನು
ನಾಗ-ದೇವ-ಭಟ್ಟ-ರಿಗೆ
ನಾಗನ
ನಾಗ-ನ-ಹಳ್ಳಿ
ನಾಗ-ನಾಯ-ಕನ
ನಾಗ-ನಿಂದ
ನಾಗ-ನಿಂದಲೇ
ನಾಗನು
ನಾಗಪ್ಪ
ನಾಗಪ್ಪನ
ನಾಗಪ್ಪ-ನಾಗ-ರಸ
ನಾಗ-ಭಟ್ಟ-ನಿಗೆ
ನಾಗ-ಭಟ್ಟನು
ನಾಗ-ಮಂಗಲ
ನಾಗ-ಮಂಗಲಕೆ
ನಾಗ-ಮಂಗಲಕ್ಕೆ
ನಾಗ-ಮಂಗಲಕ್ಕೆ-ರಾಜ್ಯಕ್ಕೆ
ನಾಗ-ಮಂಗಲದ
ನಾಗ-ಮಂಗಲ-ದಲ್ಲಿ
ನಾಗ-ಮಂಗಲ-ರಾಜ್ಯ
ನಾಗ-ಮಂಗಲ-ರಾಜ್ಯದ
ನಾಗ-ಮಂಗಲ-ವನ್ನು
ನಾಗ-ಮಂಗಲವು
ನಾಗ-ಮಂಗಲಶ್ರವಣ-ಬೆಳಗೊಳ
ನಾಗ-ಮಂಗಲಸ್ಥಳದ
ನಾಗ-ಮಯ್ಯ
ನಾಗ-ಮಯ್ಯ-ನಿಗೆ
ನಾಗ-ಮಯ್ಯನು
ನಾಗ-ಮರ್ವ
ನಾಗ-ಮಾರ್ಜಿತಮದಾತ್ತಿಂಮಕ್ಷಿತೀಂದ್ರಾತ್ಮಜಃ
ನಾಗಯ್ಯ
ನಾಗಯ್ಯ-ಗಳ
ನಾಗಯ್ಯನ
ನಾಗಯ್ಯ-ನನ್ನು
ನಾಗಯ್ಯ-ನ-ವರ
ನಾಗಯ್ಯ-ನ-ವರು
ನಾಗಯ್ಯ-ನ-ವರೂ
ನಾಗಯ್ಯ-ನಿಗೆ
ನಾಗಯ್ಯನು
ನಾಗಯ್ಯನೂ
ನಾಗಯ್ಯ-ನೆಂಬು-ವ-ವನು
ನಾಗಯ್ಯನೇ
ನಾಗ-ರಖಂಡ
ನಾಗ-ರಖಂಡ-ನ-ವನ್ನು
ನಾಗ-ರ-ಘಟ್ಟದ
ನಾಗ-ರದ-ಮೊಲೆ-ಗೋಡನ್ನು
ನಾಗ-ರಸ
ನಾಗ-ರಸನ
ನಾಗ-ರಸನು
ನಾಗ-ರ-ಸರ
ನಾಗ-ರ-ಸರು
ನಾಗ-ರಸ-ರೆಂದು
ನಾಗ-ರ-ಹಾಳ
ನಾಗ-ರಾಜ-ರಾವ್
ನಾಗ-ರಾಸಿ
ನಾಗ-ರಿ-ಲಿಪಿ
ನಾಗ-ಲ-ದೇವಿ
ನಾಗ-ಲ-ದೇವಿ-ಯಿಂದ
ನಾಗ-ಲಾಂಬಿಕಾ
ನಾಗ-ಲಾಂಬಿ-ಕೆಯ
ನಾಗ-ಲಾಂಬಿ-ಕೆಯರು
ನಾಗ-ಲಾ-ದೇವಿ
ನಾಗ-ಲಾ-ದೇವಿ-ಯಿಂದ
ನಾಗ-ಲಾ-ಪುರ-ವೆಂದು
ನಾಗ-ಲಾ-ಪುರ-ವೆಂಬ
ನಾಗಲೆ
ನಾಗ-ವರ್ಮ
ನಾಗ-ವರ್ಮ್ಮಯ್ಯ
ನಾಗ-ವಲ್ಲಿ
ನಾಗ-ಶಕ್ತಿ
ನಾಗಾಂಬಾ
ನಾಗಾ-ಭಟ್ಟ-ರಿಗೆ
ನಾಗಿದ್ದ-ನೆಂದು
ನಾಗಿ-ಯಕ್ಕನು
ನಾಗಿ-ಯಣ್ಣ-ನೆಂಬು-ವ-ವನು
ನಾಗುತ್ತಾನೆ
ನಾಗೆಯ
ನಾಗೆಯ-ನಾಯಕ
ನಾಗೆಯ-ನಾಯ-ಕನ
ನಾಗೆಯ-ನಾಯ-ಕರು
ನಾಗೇಣ
ನಾಗೇಶ್ವರ
ನಾಗೊಡೆಯ-ನಿಗೆ
ನಾಗೊಡೆ-ಯನು
ನಾಗೋ-ಹಳ್ಳಿ
ನಾಚಿಯಾರಮ್ಮ-ನಿಗೆ
ನಾಚ್ಚಾರಮ್ಮ-ನ-ವರ
ನಾಚ್ಚಿಯಾರ್ಗೆ
ನಾಟನ-ಹಳ್ಳಿ
ನಾಟ್ಟು
ನಾಟ್ಟೋರ್ಗಳ್
ನಾಡ
ನಾಡ-ಅನ್ನು
ನಾಡ-ಗವುಡ
ನಾಡ-ಗವುಡ-ಗಳ
ನಾಡ-ಗವುಡರು
ನಾಡ-ಗಾವುಂಡ-ನನ್ನು
ನಾಡ-ಗಾವುಂಡರು
ನಾಡ-ಗೌಡ-ರೆಂದು
ನಾಡ-ಗೌಡಿಕೆ
ನಾಡನ್ನು
ನಾಡನ್ನೂ
ನಾಡಪ್ರಭು
ನಾಡಪ್ರಭು-ಗಳು
ನಾಡ-ಬೋ-ಯನ-ಹಳ್ಳಿ
ನಾಡ-ಬೋ-ಯನ-ಹಳ್ಳಿ-ನಾಡ-ಬೋವ-ನ-ಹಳ್ಳಿ
ನಾಡ-ಬೋವ-ನ-ಹಳ್ಳಿ
ನಾಡ-ಮಂಡ-ಳಿಕ
ನಾಡ-ಮಾಣಿ-ಕ-ದೊಡಲೂ-ರನ್ನು
ನಾಡ-ಮಾಣಿ-ಕ-ದೊಡ-ಲೂರಿನಲ್ಲಿ
ನಾಡ-ಮಾಣಿ-ಕ-ದೊಡ-ಲೂರಿನಲ್ಲಿದ್ದ
ನಾಡ-ಮಾಣಿ-ಕ-ದೊಡ-ಲೂರು
ನಾಡ-ರ-ಸರಾದ
ನಾಡ-ವಂಡಳೀ-ಕರು
ನಾಡ-ಸುಂಕ-ವನ್ನು
ನಾಡಾಗಿತ್ತು
ನಾಡಾಗಿತ್ತೆಂದು
ನಾಡಾಗಿದ್ದ
ನಾಡಾಗಿದ್ದು
ನಾಡಾಗಿ-ರ-ಬಹುದು
ನಾಡಾಗಿ-ರುವ
ನಾಡಾದುದೆಲ್ಲವ-ಮನೇಕಚ್ಛತ್ರಂ
ನಾಡಾಳುತ್ತಿದ್ದ-ನೆಂದು
ನಾಡಾಳುವ
ನಾಡಾಳ್ವ
ನಾಡಾಳ್ವಂ
ನಾಡಾಳ್ವನು
ನಾಡಾಳ್ವರು
ನಾಡಿ-ಗರು
ನಾಡಿ-ಗವುಡ-ನ-ವರ
ನಾಡಿ-ಗವುಡ-ನ-ವರು
ನಾಡಿ-ಗವುಡರ
ನಾಡಿಗೂ
ನಾಡಿಗೆ
ನಾಡಿಗೇ
ನಾಡಿತ್ತೆಂದು
ನಾಡಿನ
ನಾಡಿ-ನಲ್ಲಿ
ನಾಡಿನಲ್ಲಿತ್ತು
ನಾಡಿನಲ್ಲಿತ್ತೆಂದು
ನಾಡಿನಲ್ಲಿದ್ದ
ನಾಡಿನಲ್ಲಿದ್ದವು
ನಾಡಿನಲ್ಲಿಯೇ
ನಾಡಿನ-ವರ
ನಾಡಿನಿಂದ
ನಾಡಿನೊಳಗಿ-ರುವ
ನಾಡಿನೊಳಗೇ
ನಾಡಿ-ಯರು
ನಾಡು
ನಾಡು-ಕಬ್ಬಪ್ಪು
ನಾಡು-ಕಲ್ಕಣಿ-ನಾಡು-ಕಲಿ-ಕಣಿ-ನಾಡು
ನಾಡು-ಗಳ
ನಾಡು-ಗಳನ್ನು
ನಾಡು-ಗಳಾಗಿ
ನಾಡು-ಗ-ಳಿಗೆ
ನಾಡು-ಗಳು
ನಾಡು-ಗ-ಳೆಂದು
ನಾಡು-ಗಳೆಂಬ
ನಾಡು-ನೀರ್ಗುಂದ
ನಾಡು-ಬಡಗುಂದ
ನಾಡು-ಬಡಗು-ನಾಡು-ವಡ-ಗೆರೆ
ನಾಡು-ವಟ್ಟ-ವಾಗಿ
ನಾಡೆಂದರೆ
ನಾಡೆಂದು
ನಾಡೆ-ಹಳ್ಳಿ-ಗಳನ್ನು
ನಾಡೇ
ನಾಡೊಳಗಣ
ನಾಡೊಳಗಿನ
ನಾಣ್ಯ-ಗಳನ್ನು
ನಾತನ-ಸುತ-ರಗಣಿತ
ನಾಥ
ನಾಥ-ನನ್ನು
ನಾನಲ
ನಾನಲ-ಕೆರೆ
ನಾನಲ-ಕೆರೆಯ
ನಾನಲ-ಕೆರೆ-ಯನ್ನು
ನಾನಲ-ಕೆರೆ-ಯಲ್ಲಿ
ನಾನಲ-ಕೆರೆ-ಯು-ಇಂದಿನ
ನಾನಲ-ಕೆರೆ-ಲಾಳ-ನ-ಕೆರೆ-ಯನ್ನು
ನಾನಲ-ಕೆಱೆಯ
ನಾನಾ
ನಾನಾ-ದೇಸಿ
ನಾನಾ-ದೇಸಿ-ಯಿಂದ
ನಾನಾ-ವರ್ನ
ನಾನು
ನಾಮ
ನಾಮ-ಕರಣ
ನಾಮ-ಕರ-ಣ-ಮಾಡಿ
ನಾಮಗ್ರಾಮ
ನಾಮದ
ನಾಮಾಯಂ
ನಾಮಾವಳಿ
ನಾಮಾವ-ಳಿ-ಸಮಾಲಂಕೃತರುಂ
ನಾಮಾವ-ಶೇಷ-ಗೊಳಿಸಿ-ದ-ನೆಂದು
ನಾಮೆ
ನಾಮ್ನಾ
ನಾಯ
ನಾಯಂಕರ
ನಾಯಕ
ನಾಯಕಂ
ನಾಯ-ಕ-ಅ-ರಸ
ನಾಯ-ಕ-ತನ
ನಾಯ-ಕ-ತನಕೆ
ನಾಯ-ಕ-ತನಕ್ಕೆ
ನಾಯ-ಕ-ತ-ನದ
ನಾಯ-ಕ-ತನ-ದಿಂದ
ನಾಯ-ಕ-ತನಮಂ
ನಾಯ-ಕ-ತನ-ವನ್ನು
ನಾಯ-ಕ-ತನ-ವೆಂಬ
ನಾಯ-ಕ-ತ-ವನ್ನು
ನಾಯ-ಕತ್ವ
ನಾಯ-ಕತ್ವಕ್ಕೆ
ನಾಯ-ಕ-ದೇವ
ನಾಯ-ಕ-ದೇವ-ಪಿಳ್ಳೆ
ನಾಯ-ಕನ
ನಾಯ-ಕ-ನನ್ನಾಗಿ
ನಾಯ-ಕ-ನನ್ನು
ನಾಯ-ಕ-ನ-ಹಳ್ಳಿ
ನಾಯ-ಕ-ನಾಗಿ
ನಾಯ-ಕ-ನಾಗಿದ್ದ
ನಾಯ-ಕ-ನಾಗಿ-ರ-ಬಹುದು
ನಾಯ-ಕ-ನಾದ
ನಾಯ-ಕ-ನಿಗೆ
ನಾಯ-ಕ-ನಿರ-ಬಹುದು
ನಾಯ-ಕನು
ನಾಯ-ಕನೂ
ನಾಯ-ಕ-ನೆಂಬ
ನಾಯ-ಕನೇ
ನಾಯ-ಕ-ಮಕ್ಕಳು
ನಾಯ-ಕರ
ನಾಯ-ಕ-ರ-ಗಂಡ
ನಾಯ-ಕ-ರನ್ನು
ನಾಯ-ಕ-ರಲ್ಲಿ
ನಾಯ-ಕ-ರಾಗಿದ್ದರು
ನಾಯ-ಕ-ರಿಗೂ
ನಾಯ-ಕ-ರಿಗೆ
ನಾಯ-ಕ-ರಿದ್ದ-ರೆಂದು
ನಾಯ-ಕ-ರಿದ್ದು
ನಾಯ-ಕರು
ನಾಯ-ಕ-ರು-ಗ-ಳಿಗೆ
ನಾಯ-ಕ-ರು-ಗಳು
ನಾಯ-ಕರೂ
ನಾಯ-ಕ-ಹೆಗ್ಗಡೆ
ನಾಯ-ಕಿತ್ತಿ
ನಾಯ-ಕಿತ್ತಿಗೆ
ನಾಯ-ಕಿತ್ತಿಯ
ನಾಯ-ಕಿತ್ತಿ-ಯರ
ನಾಯ-ಕಿತ್ತಿ-ಯರು
ನಾಯ-ಕಿತ್ತಿಯು
ನಾಯಿಂದ-ರಿಗೆ
ನಾಯಿಂದ-ರು-ಗ-ಳಿಗೆ
ನಾಯಿ-ಯನ್ನು
ನಾಯಿ-ಯನ್ನೂ
ನಾಯಿ-ಯನ್ನೇ
ನಾಯ್ಕ
ನಾಯ್ಕ-ರಸನು
ನಾಯ್ಕ-ರ-ಸರನ
ನಾರಣ-ದೇವಿ
ನಾರಣ-ವೆಗ್ಗಡೆ
ನಾರಣ-ವೆಗ್ಗಡೆ-ಯಾಗಿ-ರು-ವಂತೆ
ನಾರಣ-ವೆಗ್ಗಡೆಯು
ನಾರಣ-ವೆಗ್ಗಡೆಯೇ
ನಾರಣ-ವೆರ್ಗಡೆ
ನಾರಣವೆರ್ಗ್ಗಡೆ
ನಾರಣವೆರ್ಗ್ಗಡೆ-ಯಯು
ನಾರಣಾಂಕವಿದು
ನಾರಪ್ಪ-ರಾಜ
ನಾರಪ್ಪ-ರಾಜಯ್ಯನ
ನಾರಪ್ಪ-ರಾಜಯ್ಯನು
ನಾರಯ-ದೇವ
ನಾರಯ-ದೇವನು
ನಾರಯ್ಯ-ದೇವ
ನಾರಯ್ಯ-ದೇವನ
ನಾರ-ಸಿಂಗ
ನಾರ-ಸಿಂಗ-ಚತುರ್ವೇದಿ
ನಾರ-ಸಿಂಗ-ದೇವನ
ನಾರ-ಸಿಂಗ-ದೇವನು
ನಾರ-ಸಿಂಗ-ದೇವರು
ನಾರ-ಸಿಂಗನು
ನಾರ-ಸಿಂಗಯ್ಯ-ದೇವನ
ನಾರ-ಸಿಂಘ
ನಾರ-ಸಿಂಘ-ದೇವನ
ನಾರ-ಸಿಂಘ-ದೇವನು
ನಾರ-ಸಿಂಘ-ಯ-ದೇವ
ನಾರ-ಸಿಂಹ
ನಾರ-ಸಿಂಹ-ಚತುರ್ವೇದಿ-ಮಂಗಲದ
ನಾರ-ಸಿಂಹ-ದೇವ
ನಾರ-ಸಿಂಹ-ದೇವನ
ನಾರ-ಸಿಂಹ-ದೇವ-ರಸ
ನಾರ-ಸಿಂಹ-ದೇವ-ರ-ಸರು
ನಾರ-ಸಿಂಹ-ದೇವ-ರಿಗೆ
ನಾರ-ಸಿಂಹ-ದೇವರು
ನಾರ-ಸಿಂಹನ
ನಾರ-ಸಿಂಹ-ನನ್ನು
ನಾರ-ಸಿಂಹ-ನಿಗೂ
ನಾರ-ಸಿಂಹ-ನಿಗೆ
ನಾರ-ಸಿಂಹನು
ನಾರ-ಸಿಂಹ-ರಸ
ನಾರ-ಸಿಂಹ-ರಾಯನ
ನಾರ-ಸಿಂಹಸ್ವಾಮಿಗೆ
ನಾರಾಯಣ
ನಾರಾಯ-ಣ-ಗಿರಿ
ನಾರಾಯ-ಣ-ಗಿರಿ-ದುರ್ಗ
ನಾರಾಯ-ಣ-ಗಿರಿ-ದುರ್ಗದ
ನಾರಾಯ-ಣದ
ನಾರಾಯ-ಣ-ದೇವರ
ನಾರಾಯ-ಣ-ದೇ-ವ-ರಿಗೆ
ನಾರಾಯ-ಣ-ದೇವರು
ನಾರಾಯ-ಣ-ದೇವಾಲಯ-ದಲ್ಲಿ-ರುವ
ನಾರಾಯ-ಣನ
ನಾರಾಯ-ಣ-ನಿಗೆ
ನಾರಾಯ-ಣ-ನೆಂದೂ
ನಾರಾಯ-ಣ-ಪಾದ-ಪಙ್ಕಜಯುಗೀ-ವನ್ಯಸ್ತವಿಪ್ಪಗ್ಭರಃ
ನಾರಾಯ-ಣ-ಪುರ-ವನ್ನು
ನಾರಾಯ-ಣ-ರಾಯ-ರೆಂಬುವ-ವರ
ನಾರಾಯ-ಣ-ಶೈಲಕ್ಕೆ
ನಾರಾಯ-ಣಸ್ವಾಮಿ
ನಾರಾಯ-ಣಾಂಬಿ-ಕೆಯರ
ನಾರಿ
ನಾರಿ-ಯಪ್ಪ
ನಾರಿ-ವಾಳ-ವನು
ನಾರ್ತ್ಬ್ಯಾಂಕ್
ನಾಲಯ್ಯ-ನಿಗೆ
ನಾಲೂರಿನ
ನಾಲೆ-ಯಿಂದ
ನಾಲ್ಕ-ನೆಯ
ನಾಲ್ಕ-ನೆಯ-ವನು
ನಾಲ್ಕನೇ
ನಾಲ್ಕ-ರಲ್ಲಿ
ನಾಲ್ಕು
ನಾಲ್ಕು-ಜನ
ನಾಲ್ಕೂ
ನಾಲ್ದೆಸೆಗೆ
ನಾಲ್ಮಡಿ
ನಾಲ್ವತ್ತು
ನಾಲ್ವರು
ನಾಲ್ವರೂ
ನಾಳನ-ಕೆರೆ
ನಾಳೆಯಿಲ್ಲೆಂದ
ನಾಳ್ಗಾವುಂಡರು
ನಾವಿದ್ದೇವೆ
ನಾವು
ನಾವೇ
ನಾಶ-ಪಡಿಸಿ-ತೆಂದು
ನಾಶ-ಪಡಿಸಿ-ದ-ನೆಂದು
ನಾಶ-ಮಾಡಿ
ನಾಶ-ವಾಗಿ
ನಾಸಿರ್ಜಂಗ್
ನಾಸಿರ್ಜಂಗ್ನ
ನಾೞ್ಪ್ರಭು
ನಿಂತರು
ನಿಂತರು-ಎಂದು
ನಿಂತಿದ್ದಾರೆ
ನಿಂತಿರ-ಬಹುದು
ನಿಂತಿ-ರು-ವುದು
ನಿಂತು
ನಿಂದ-ರಿನ್ರಿಪಾಳರ
ನಿಂನಂತಾರೊಳವೊಕ್ಕು
ನಿಕ್ಕಿ-ಯಣ್ಣ
ನಿಕ್ಕಿ-ಯ-ರಸ
ನಿಕ್ಕಿ-ಯ-ರಸನ
ನಿಕ್ಕಿ-ರಸ-ನಿಕ್ಕರಸ
ನಿಕ್ಕೀಶ್ವರ
ನಿಕ್ಕೇಶ್ವರ
ನಿಖರ-ವಾಗಿ
ನಿಖಿಳ-ಲಕ್ಷ್ಮೀ
ನಿಖಿಳಾಂ
ನಿಗದಿ
ನಿಗದಿ-ಪಡಿ-ಸ-ಲಾಗಿದೆ
ನಿಗದಿ-ಪಡಿ-ಸಲು
ನಿಗದಿ-ಪಡಿಸಿ
ನಿಗದಿ-ಪಡಿಸಿದ್ದನ್ನು
ನಿಗದಿ-ಪಡಿಸಿದ್ದಾರೆ
ನಿಗದಿ-ಪಡಿ-ಸುವ
ನಿಗೆ
ನಿಗ್ರಹಿ-ಸಿದನು
ನಿಜ
ನಿಜ-ಕಳತ್ರ
ನಿಜಕ್ಕೂ
ನಿಜಪ್ರತಾಪ-ದಿಂದ
ನಿಜಪ್ರತಾಪಾದಧಿಗತ್ಯ
ನಿಜಪ್ರಧಾನ
ನಿಜ-ರಾಜ-ಧಾನಿ
ನಿಜ-ರಾಧಾ-ನಿಮಧಿವ-ಸನ್
ನಿಜ-ರಾಮ
ನಿಜ-ವಿಜಯ
ನಿಜಸ್ವಾಮಿ
ನಿಜಾಂ
ನಿಜಾಂಶಂ
ನಿಜಾಮರು
ನಿಜಾಮ-ರೊಂದಿಗೆ
ನಿಡದವೋಲು
ನಿಡುಗಲ್ಲಿನ
ನಿಡುವುಟೆ-ಯನ್ನು
ನಿಡುವುಟೆಯು
ನಿಡು-ವೊಳಲಾಗಿರ-ಬಹುದು
ನಿತ್ತರಿಸಲಾ-ರದೆ
ನಿತ್ಯ
ನಿತ್ಯಂ
ನಿತ್ಯೋತ್ಸಾಹ-ಗ-ಳಿಗೆ
ನಿದರ್ಶನ-ವನ್ನು
ನಿಧನ-ನಾಗಿದ್ದ-ನೆಂದು
ನಿಧನಾ-ನಂತರ
ನಿಧಾನಃ
ನಿಧಾನ-ವಾಗಿ
ನಿನಗೆಂದು
ನಿನಗೇನು
ನಿನ್ನ
ನಿಪ್ಪ
ನಿಬಂಧ
ನಿಬಂಧಿ-ಯಾಗಿ
ನಿಮಿತ್ತ
ನಿಮ್ಮಡಿ-ಯನ್ನು
ನಿಯ-ಗಾ-ಮುಂಡನ
ನಿಯತ-ಕಾಲಿಕ-ಗಳಲ್ಲಿ
ನಿಯಮ-ದಂತೆ
ನಿಯುಕ್ತ-ರಾದರೆ
ನಿಯುಕ್ತ-ಳಾದಳು
ನಿಯೋಗ-ಗಳ
ನಿಯೋಗ-ದಿಂದ
ನಿಯೋಗದುರಂಧರ
ನಿಯೋಗನ-ವನು
ನಿಯೋಗಾಧಿ-ಪತಿ
ನಿಯೋಗಾಧಿ-ಪತಿ-ಗಳ
ನಿಯೋಗಾಧಿ-ಪತಿ-ಗಳು
ನಿಯೋಗಾಧಿ-ಪತಿ-ಗಳೂ
ನಿಯೋಗಿ
ನಿರಂತರ
ನಿರಂತರಂ
ನಿರಂತರ-ವೆನ್ನಲು
ನಿರ-ತನಾ-ದ-ನೆಂದು
ನಿರ-ನಾಗಿದ್ದ-ನೆಂದು
ನಿರವದ್ಯ
ನಿರಾ-ಕುಳ-ದಿಂದ
ನಿರಾತಂಕ-ವಾಗಿ
ನಿರಿ-ಸಿದರ್ಸ್ಸೋ-ಮಾನ್ವಯೋರ್ವ್ವೀಶ್ವ-ರರ್
ನಿರೀಕ್ಷಿ-ಸುತ್ತಿದ್ದ
ನಿರೀಕ್ಷೆ
ನಿರುಪಾಧಿಕ
ನಿರುಪಾಯ-ನಾಗಿ
ನಿರೂಪ
ನಿರೂಪ-ಗಳನ್ನು
ನಿರೂಪ-ಣೆ-ಗಳನ್ನು
ನಿರೂಪ-ದಂತೆ
ನಿರೂಪ-ದಲಿ
ನಿರೂಪ-ದಿಂದ
ನಿರೂಪ-ವನ್ನು
ನಿರೂಪಿತ
ನಿರೂಪಿತ-ವಾಗಿದ್ದು
ನಿರೂಪಿ-ಸ-ಲಾಗಿದೆ
ನಿರೂಪಿಸ-ಲಾ-ಗಿದ್ದು
ನಿರೂಪಿಸಿ
ನಿರೂಪಿಸಿದ್ದಾರೆ
ನಿರೂಪಿ-ಸುವ
ನಿರ್ಗ್ಗುಂದ
ನಿರ್ಜಿತ್ಯ
ನಿರ್ಣಯ
ನಿರ್ದಿಷ್ಟಪ್ರ-ಮಾ-ಣದ
ನಿರ್ದೇಶಿ-ಸಲು
ನಿರ್ಧರಿತ-ವಾಗುತ್ತಿದ್ದವು
ನಿರ್ಧರಿಸಿ-ದನು
ನಿರ್ಧರಿ-ಸುತ್ತಿರ-ಬಹುದು
ನಿರ್ಮಾಣ
ನಿರ್ಮಾ-ಣಕ್ಕೆ
ನಿರ್ಮಾ-ಣದ
ನಿರ್ಮಾಣ-ವನ್ನು
ನಿರ್ಮಾಣ-ವಾಗಿ
ನಿರ್ಮಾಣ-ವಾಗಿ-ರುವ
ನಿರ್ಮಾಣ-ವಾಯಿತು
ನಿರ್ಮಾಣವೂ
ನಿರ್ಮಿತ-ವಾಗಿದ್ದ
ನಿರ್ಮಿತ-ವಾಗಿ-ರ-ಬಹುದು
ನಿರ್ಮಿತ-ವಾದ
ನಿರ್ಮಿತ-ವಾ-ದವು
ನಿರ್ಮಿ-ಸ-ಲಾಗಿದೆ
ನಿರ್ಮಿ-ಸಲು
ನಿರ್ಮಿಸಿ
ನಿರ್ಮಿಸಿ-ಕೊಳ್ಳುವಂತಿ-ರ-ಲಿಲ್ಲ-ವೆಂದು
ನಿರ್ಮಿ-ಸಿದ
ನಿರ್ಮಿಸಿ-ದ-ನಷ್ಟೆ
ನಿರ್ಮಿಸಿ-ದನು
ನಿರ್ಮಿಸಿ-ದ-ನೆಂದು
ನಿರ್ಮಿಸಿ-ದರು
ನಿರ್ಮಿಸಿ-ದ-ರೆಂದು
ನಿರ್ಮಿಸಿ-ದಳು
ನಿರ್ಮಿಸಿ-ದಾಗ
ನಿರ್ಮಿಸಿದ್ದಾನೆ
ನಿರ್ಮಿಸಿ-ರ-ಬಹುದು
ನಿರ್ಮಿಸಿ-ರ-ಬಹು-ದೆಂದು
ನಿರ್ಮಿಸಿ-ರುವ
ನಿರ್ಮಿಸಿ-ರುವು-ದ-ರಿಂದ
ನಿರ್ಮಿ-ಸುತ್ತಾನೆ
ನಿರ್ಮಿ-ಸುವುದ-ರಲ್ಲಿ
ನಿರ್ಮೂಲ
ನಿರ್ಮೂಲನ
ನಿರ್ಮೂಲನೆ
ನಿರ್ಮೂ-ಳನ
ನಿರ್ಮ್ಮಡಿ-ಯನ್ನು
ನಿರ್ವ-ಹಣೆ
ನಿರ್ವ-ಹಣೆಗೆ
ನಿರ್ವ-ಹಣೆ-ಚಾರಿತ್ರಿಕ
ನಿರ್ವ-ಹಣೆಯ
ನಿರ್ವಹಿ-ಸುತ್ತಿದ್ದರು
ನಿರ್ವಹಿ-ಸುತ್ತಿದ್ದ-ರೆಂದು
ನಿರ್ವಹಿ-ಸುತ್ತಿದ್ದು-ದನ್ನು
ನಿರ್ವಹಿ-ಸುತ್ತಿದ್ದುದು
ನಿರ್ವಹಿ-ಸುವ
ನಿರ್ವಹಿ-ಸುವ-ವರು
ನಿರ್ವಹಿ-ಸುವ-ವರೂ
ನಿಲವಂದ-ದೇವ-ರಿಗೆ
ನಿಲಿ-ಸಿದಂ
ನಿಲಿ-ಸಿಳೆಯಂ
ನಿಲ್ಲಿಸ-ದರು
ನಿಲ್ಲಿಸಿ
ನಿಲ್ಲಿ-ಸಿದ
ನಿಲ್ಲಿಸಿ-ದನು
ನಿಲ್ಲಿಸಿ-ದ-ನೆಂಬುದು
ನಿಲ್ಲಿಸಿ-ದ-ರೆಂದು
ನಿಲ್ಲಿ-ಸುತ್ತಾನೆ
ನಿಲ್ಲಿ-ಸುತ್ತಾರೆ
ನಿಲ್ಲಿ-ಸುತ್ತಾಳೆ
ನಿಲ್ಲುವ
ನಿವನ
ನಿವಾರ-ಣೆ-ಗಾಗಿ
ನಿವಾರಿಸಿ-ಕೊಂಡು
ನಿವಾರಿ-ಸಿದನು
ನಿವಾಸಾಶ್ರಾಯಾಂ
ನಿವಾ-ಸಿ-ಗಳ
ನಿವಾಸಿ-ಗ-ಳಾದ
ನಿವೃತ್ತ-ನಾಗ-ಲಿಚ್ಚಿಸಿ
ನಿವೃತ್ತಿ-ಗೊಳಿಸಿ
ನಿವೇದ್ಯ
ನಿಶಂಕ-ಮಲ್ಲು
ನಿಶಿದಿಗೆ-ಯನ್ನು
ನಿಶಿಧಿಯಂ
ನಿಶ್ಶಂಕ
ನಿಶ್ಶಂಕಪ್ರತಾಪ
ನಿಷ್ಕಂಟಕಂ
ನಿಷ್ಕಂಟಕ-ವನ್ನಾಗಿ
ನಿಷ್ಕಂಟಕ-ವಾದ
ನಿಷ್ಕಾಮೇಶ್ವರ
ನಿಷ್ಕ್ರಿ-ಯತೆ-ಯನ್ನು
ನಿಷ್ಟ
ನಿಷ್ಠ-ನಾಗಿ
ನಿಷ್ಠ-ನಾಗಿದ್ದರೂ
ನಿಷ್ಠ-ರಾಗಿ
ನಿಷ್ಠ-ರಾಗಿದ್ದ-ರೆಂಬು-ದನ್ನು
ನಿಷ್ಠ-ರಾಗಿದ್ದ-ರೆಂಬುದು
ನಿಷ್ಠ-ಸಾಮಂತರು
ನಿಷ್ಠಾವಂತ-ರಾಗಿದ್ದ-ವರು
ನಿಷ್ಠೆ-ಯಿಂದ
ನಿಷ್ಪತ್ತಿ
ನಿಷ್ಪತ್ತಿಯ
ನಿಷ್ಪತ್ತಿ-ಯನ್ನು
ನಿಷ್ಪನ್ನಗೊಳಿ-ಸಲು
ನಿಷ್ಪ್ರಯೋಜಕ-ವಾದು-ದೆಂದು
ನಿಸಿತಾಸಿಯ
ನಿಸಿದಿ
ನಿಸಿದಿ-ಗಲ್ಲನ್ನು
ನಿಸಿದಿ-ಗಲ್ಲು-ಗಳು
ನಿಸಿದಿಗೆ
ನಿಸಿದಿ-ಗೆ-ಯನ್ನು
ನಿಸ್ಸಂಕ
ನಿಸ್ಸಂಕ-ರೆ-ನಿಪ್ಪ
ನಿಸ್ಸೀಮ
ನಿಸ್ಸೀಮಂ
ನೀಡ-ದಿದ್ದರೂ
ನೀಡದೆ
ನೀಡದೇ
ನೀಡ-ಬಹುದು
ನೀಡಲಾ-ಗಿತ್ತು
ನೀಡ-ಲಾಗಿತ್ತೆಂದು
ನೀಡ-ಲಾಗಿದೆ
ನೀಡ-ಲಾಗಿ-ದೆಯೇ
ನೀಡ-ಲಾ-ಗಿದ್ದು
ನೀಡ-ಲಾಗುತ್ತಿತ್ತು
ನೀಡ-ಲಾಯಿತು
ನೀಡ-ಲಾ-ಯಿತೆಂದು
ನೀಡಲಾಯಿ-ತೆಂದೂ
ನೀಡಲು
ನೀಡಲ್ಪಟ್ಟಿದೆ
ನೀಡಲ್ಪಡುತ್ತಿದ್ದ
ನೀಡಿ
ನೀಡಿದ
ನೀಡಿದಂತೆ
ನೀಡಿದನ
ನೀಡಿದ-ನಂತೆ
ನೀಡಿದನು
ನೀಡಿದ-ನೆಂದು
ನೀಡಿದ-ನೆಂಬ
ನೀಡಿದರು
ನೀಡಿದ-ರೆಂದು
ನೀಡಿದಾಗ
ನೀಡಿದೆ
ನೀಡಿದ್ದ
ನೀಡಿದ್ದನು
ನೀಡಿದ್ದ-ನೆಂದು
ನೀಡಿದ್ದ-ನೆಂಬ
ನೀಡಿದ್ದರು
ನೀಡಿದ್ದ-ರೆಂದು
ನೀಡಿದ್ದಾನೆ
ನೀಡಿದ್ದಾನೆಂಬ
ನೀಡಿದ್ದಾರೆ
ನೀಡಿದ್ದು
ನೀಡಿ-ರ-ಬಹು-ದಾದ್ದನ್ನು
ನೀಡಿ-ರ-ಬಹುದು
ನೀಡಿ-ರ-ಬಹು-ದೆಂದು
ನೀಡಿ-ರ-ಬಹುದೇ
ನೀಡಿ-ರುತ್ತಾನೆ
ನೀಡಿ-ರುತ್ತಾರೆ
ನೀಡಿ-ರುವ
ನೀಡಿ-ರು-ವಂತೆ
ನೀಡಿ-ರು-ವಂತೆಯೇ
ನೀಡಿ-ರು-ವು-ದರ
ನೀಡಿ-ರು-ವು-ದರಿಂದ
ನೀಡಿ-ರು-ವು-ದಿಲ್ಲ
ನೀಡಿ-ರು-ವುದು
ನೀಡಿಲ್ಲ
ನೀಡಿವೆ
ನೀಡುತ್ತದೆ
ನೀಡುತ್ತವೆ
ನೀಡುತ್ತಾ
ನೀಡುತ್ತಾನೆ
ನೀಡುತ್ತಾ-ನೆಂದು
ನೀಡುತ್ತಾರೆ
ನೀಡುತ್ತಾಳೆ
ನೀಡುತ್ತಿದ್ದ
ನೀಡುತ್ತಿದ್ದ-ರೆಂದು
ನೀಡುವ
ನೀಡು-ವಂತೆ
ನೀಡು-ವಾಗ
ನೀಡು-ವು-ದನ್ನು
ನೀಡು-ವು-ದರ
ನೀಡು-ವುದು
ನೀತಿ
ನೀತಿಗೆ
ನೀತಿ-ಮಹಾ-ರಾಜ
ನೀತಿ-ಮಾರ್ಗ
ನೀತಿ-ಮಾರ್ಗ-ಎರೆ-ಗಂಗ-ನೆಂದು
ನೀತಿ-ಮಾರ್ಗನ
ನೀತಿ-ಮಾರ್ಗ-ನನ್ನು
ನೀತಿ-ಮಾರ್ಗ-ನಿಗೆ
ನೀತಿ-ಮಾರ್ಗನು
ನೀತಿ-ಮಾರ್ಗ-ನೆಂಬ
ನೀತಿ-ಮಾರ್ಗನೇ
ನೀತಿ-ಮಾರ್ಗ್ಗ
ನೀತಿ-ವಾಕ್ಯ
ನೀತಿ-ವಿ-ದರೂ
ನೀತಿ-ವಿಶಾ-ರದಃ
ನೀತಿ-ಶಾಸ್ತ್ರ
ನೀತಿ-ಶಾಸ್ತ್ರಸ್ಯ
ನೀನೂ
ನೀರಗುಂದ
ನೀರಾ-ವರಿ
ನೀರಾ-ವ-ರಿಗೆ
ನೀರಾ-ವರಿಯ
ನೀರಾ-ವರಿ-ಯಿಂದ
ನೀರಿಗೆ
ನೀರಿನ
ನೀರಿಲ್ಲದ
ನೀರು
ನೀರು-ಣಿ-ಸುವ
ನೀರು-ಹರಿ-ಸಲು
ನೀರ್ಗುಂದ
ನೀರ್ಗುಂದದ
ನೀರ್ಗ್ಗುಂದ
ನೀರ್ಗ್ಗುನ್ದೆಳಾ
ನೀರ್ನೆಲ-ವನ್ನು
ನೀಲ-ಕಂಠ
ನೀಲ-ಕಂಠ-ನ-ಹಳ್ಳಿ-ಗಳನ್ನು
ನೀಲ-ಕಂಠಾಚಾರ್ಯನ
ನೀಲ-ಗಿರಿ
ನೀಲ-ಗಿರಿಯ
ನೀಲ-ಗಿರಿ-ಸಾಧಾರ
ನೀಲ-ಚೊಟ್ಟ
ನೀಲ-ಮಸೂದ
ನೀಲಯ್ಯ
ನೀಲ-ಸಮುದ್ರ
ನೀಲಾಚಲ-ನೀಲ-ಗಿರಿ-ಯನ್ನು
ನೀಲಾಚಲ-ವನ್ನು
ನೀಳಾಚಳಮಂ
ನೀಳಾದ್ರಿಯಂ
ನುಂಗಿ
ನುಂಗುವ
ನುಕ-ರಾಜ-ನಿಗೆ
ನುಗು-ನಾಡು
ನುಗ್ಗಿ
ನುಗ್ಗಿತು
ನುಗ್ಗಿ-ತೆಂದೂ
ನುಗ್ಗಿದ
ನುಗ್ಗಿ-ದನು
ನುಗ್ಗಿ-ಲೂರು
ನುಗ್ಗಿ-ಹಳ್ಳಿಯ
ನುಡಿದಂನ್ತೆ
ನುಡಿದಂನ್ತೆ-ಗಂಡನುಂ
ನುಡಿದುದೇ
ನುಡಿದು-ಮತ್ತೆನ್ನನುಂ
ನುತ-ಬಲ್ಲಾಳ-ಭೂಪನ
ನುಳಮ್ಬನುಂ
ನೂಕಿ-ದನು
ನೂತನ
ನೂತನ-ವಾಗಿ
ನೂತ್ನ-ರತ್ನಮುಂ
ನೂತ್ನ-ರತ್ನ-ಮುಮಂ
ನೂರನ್ನು
ನೂರಾರು
ನೂರು
ನೂರ್ಮಡಿ
ನೂರ್ಮ್ಮಡಿ
ನೂಱನ್ನು
ನೂಳುಲು
ನೃಪಂ
ನೃಪ-ತುಂಗನ
ನೃಪ-ತುಂಗನು
ನೃಪನ
ನೃಪ-ನಿಂದೇವಣ್ನಿಪೆಂ
ನೃಪ-ಭೂಪನ
ನೃಪ-ಭೂಪ-ನೆಂದು
ನೃಪ-ರಾಜ್ಯ-ವಾರ್ದ್ಧಿ-ಸಂವರ್ದ್ಧನ
ನೃಪ-ರಿಂದೊಡ-ಗೂಡಿದ
ನೃಸಿಂಹ-ಭೂಪನೆಳೆಯಂ
ನೃಸಿಂಹ-ಭೂಪನೆಳೆಯಂದೋಸ್ಥಂಭದೊಲು
ನೆಂದು
ನೆಂಬು-ವ-ವನು
ನೆಗಳ್ದ
ನೆಗಳ್ದಂ
ನೆಗಳ್ದ-ತೆಂಕಣ-ರಾಯ-ನೆನಲ್ಕೆ-ನಿಪ್ಪ
ನೆಗಳ್ದ-ರೊಳ್
ನೆಗಳ್ದಾಧಿ-ರಾಜ-ಪದ-ವಿಗೆ
ನೆಗೆದು
ನೆಚ್ಚಿ-ಕೊಂಡಿದ್ದನು
ನೆಟ್ಟ-ಕಲ್ಲು
ನೆಟ್ಟನು
ನೆಟ್ಟೂರು
ನೆಡಿಸಿ-ದ-ರೆಂದು
ನೆಡಿ-ಸುತ್ತಾನೆ
ನೆಡುಮಾಮಿಟಿ
ನೆತ್ತರು-ಗೊಡಗೆ-ಯಾಗಿ
ನೆತ್ತಿ
ನೆನಪನ್ನು
ನೆನಪಿಗೆ
ನೆನೆಯ-ಬಹುದು
ನೆನೆವ
ನೆಮ್ಮೆದಿ-ಯನ್ನು
ನೆಯ
ನೆಯಾಮ
ನೆರನೆರಪಿ
ನೆರ-ವನ್ನು
ನೆರ-ವಾಗಿ
ನೆರ-ವಾಗಿ-ರ-ಬಹುದು
ನೆರ-ವಾಗಿರೆ
ನೆರ-ವಾಗುತ್ತವೆ
ನೆರ-ವಾಗುತ್ತಿದ್ದನು
ನೆರ-ವಾಗುವುದಲ್ಲದೆ
ನೆರ-ವಾ-ದನು
ನೆರ-ವಿಗೆ
ನೆರವಿ-ನಿಂದ
ನೆರವು
ನೆರವೇ-ರದೇ
ನೆರವೇರಿಸಿ
ನೆರವೇರಿ-ಸಿದ
ನೆರೆದಿದ್ದ-ರೆಂದು
ನೆರೆದು
ನೆರೆಯ
ನೆರೆಯೆ
ನೆಲಂ
ನೆಲನಂ
ನೆಲ-ಮಂಗಲ
ನೆಲಮನೆ
ನೆಲಸಮ-ವಾಗಿದೆ
ನೆಲಾ-ಪುರ
ನೆಲುಮನೆ
ನೆಲೆ
ನೆಲೆ-ಗಳನ್ನು
ನೆಲೆ-ಗಳು
ನೆಲೆ-ನಿಂತರು
ನೆಲೆ-ನಿಂತ-ರೆಂಬುದೂ
ನೆಲೆ-ಬೀಡಂ
ನೆಲೆ-ಬೀಡನ್ನು
ನೆಲೆ-ಬೀಡಾಗಿ
ನೆಲೆ-ಬೀಡಾಗಿತ್ತು
ನೆಲೆ-ಬೀಡಾಗಿ-ರ-ಬಹುದು
ನೆಲೆ-ಬೀ-ಡಿಗೆ
ನೆಲೆ-ಬೀಡಿ-ನಲ್ಲಿ
ನೆಲೆ-ಬೀ-ಡಿನಲ್ಲಿದ್ದ-ನೆಂದು
ನೆಲೆ-ಬೀ-ಡಿನಿಂದ
ನೆಲೆ-ಬೀಡು-ಗಳನ್ನು
ನೆಲೆ-ಬೀಡು-ಗಳೂ
ನೆಲೆ-ಯಾಗಿ
ನೆಲೆ-ಯಾಗಿತ್ತೆಂದು
ನೆಲೆ-ಯಾಗಿದೆ
ನೆಲೆ-ಯಾಗಿದ್ದಿತು
ನೆಲೆ-ವೀಡಾಗಿತ್ತೆಂಬು-ದನ್ನು
ನೆಲೆ-ವೀಡಾಗಿದ್ದ
ನೆಲೆ-ವೀಡಿ-ನಲ್ಲಿ
ನೆಲೆ-ವೀಡಿ-ನಲ್ಲಿದ್ದಾಗ
ನೆಲೆ-ವೀಡಿ-ನಿಂದ
ನೆಲೆ-ವೀ-ಡಿನೊಳ್ಸಮುತ್ತುಂಗ
ನೆಲೆ-ವೀಡು-ಗಳಲ್ಲಿ
ನೆಲೆ-ವೃತ್ತಿ-ಗಳ
ನೆಲೆಸಿ
ನೆಲೆ-ಸಿದ
ನೆಲೆ-ಸಿ-ದರು
ನೆಲೆ-ಸಿ-ದ-ವ-ರಿಂದ
ನೆಲೆ-ಸಿದ್ದನು
ನೆಲೆ-ಸಿದ್ದ-ನೆಂದು
ನೆಲೆ-ಸಿದ್ದ-ರೆಂದು
ನೆಲೆ-ಸಿದ್ದ-ರೆಂಬ
ನೆಲೆ-ಸಿದ್ದ-ರೆಂಬುದು
ನೆಲೆ-ಸಿ-ರುವ
ನೆಲೆ-ಸಿ-ರು-ವಂತೆ
ನೆಲ್ಲ
ನೆಲ್ಲ-ಕೂಳಣ
ನೆಲ್ಲ-ಕೂಳಣ-ವಾಗಿ
ನೆಲ್ಲ-ಕೂೞಣಲಾಗೊದೆನ್ದು
ನೆಲ್ಲು
ನೇ
ನೇತೃತ್ವ
ನೇತೃತ್ವ-ದಲ್ಲಿ
ನೇತೃತ್ವ-ವನ್ನು
ನೇತ್ರ
ನೇತ್ರಃ
ನೇತ್ರ-ನೆಂದುಮೀ
ನೇಮ
ನೇಮಕ
ನೇಮ-ಕ-ಮಾಡ-ಲಾಗುತ್ತಿತ್ತು
ನೇಮ-ಕ-ಮಾಡಿದನು
ನೇಮ-ಕ-ವಾಗಿದ್ದರು
ನೇಮ-ಕ-ವಾದ-ವ-ರೆಂದು
ನೇಮ-ಕಾತಿ
ನೇಮ-ದಂಡೇಶ
ನೇಮ-ದಂಡೇ-ಸದಿಕ್ಕುಂ
ನೇಮ-ವೆರ್ಗಡೆ
ನೇಮ-ಹೆರ್ಗಡೆ
ನೇಮಿ-ಸ-ಲಾಗುತ್ತಿತ್ತು
ನೇಮಿಸ-ಲಾಗುತ್ತಿತ್ತೆಂದು
ನೇಮಿ-ಸ-ಲಾಯಿತು
ನೇಮಿಸಲ್ಪಟ್ಟ
ನೇಮಿಸಲ್ಪಟ್ಟನು
ನೇಮಿಸಲ್ಪಡುತ್ತಿದ್ದ
ನೇಮಿಸಿ
ನೇಮಿ-ಸಿದ
ನೇಮಿಸಿ-ದನು
ನೇಮಿಸಿ-ದ-ನೆಂದು
ನೇಮಿಸಿ-ದನೇ
ನೇಮಿಸಿ-ದರು
ನೇಮಿಸಿ-ದುದಂತೂ
ನೇಮಿಸಿದ್ದನು
ನೇಮಿಸಿದ್ದರೂ
ನೇಮಿಸಿದ್ದಾನೆ
ನೇಮಿಸಿ-ರ-ಬೇಕು
ನೇಮಿ-ಸುತ್ತಾನೆ
ನೇಮಿ-ಸುತ್ತಿದ್ದನು
ನೇಮಿ-ಸುತ್ತಿದ್ದರು
ನೇಮಿ-ಸುತ್ತಿದ್ದ-ರೆಂದು
ನೇಮಿ-ಸುತ್ತಿದ್ದುರು
ನೇಮಿ-ಸುವ
ನೇಮೀಶ್ವರ
ನೇಯ್ಗೆಗೆ
ನೇರ
ನೇರ-ಲ-ಕಟ್ಟೆ
ನೇರ-ಲ-ಕೆರೆಯ
ನೇರ-ಲಿಗೆ
ನೇರ-ವಾಗಿ
ನೇಶ
ನೈಜ-ವಾದ
ನೈಜಾಮ-ನಿಗೆ
ನೈಜಾಮ್
ನೈವೇದ್ಯ
ನೈವೇದ್ಯಕ್ಕಾಗಿ
ನೈವೇದ್ಯಕ್ಕೆ
ನೈವೇದ್ಯದ
ನೊಣಂಬ-ವಾಡಿ
ನೊಬೆಯ
ನೊಳಂಬ
ನೊಳಂಬ-ಕುಲಾಂತಕ-ದೇವನ
ನೊಳಂಬನ
ನೊಳಂಬ-ನಿಗೆ
ನೊಳಂಬನು
ನೊಳಂಬರ
ನೊಳಂಬ-ರನ್ನು
ನೊಳಂಬ-ರ-ಸ-ರಾಗಿ-ರ-ಬಹುದು
ನೊಳಂಬ-ರಾಜನ
ನೊಳಂಬ-ರಾಜಾನ್ವ-ಯದ
ನೊಳಂಬ-ರಿಂದ
ನೊಳಂಬರು
ನೊಳಂಬರೂ
ನೊಳಂಬ-ರೊಡನೆ
ನೊಳಂಬ-ಳಿಗೆ
ನೊಳಂಬ-ವಾಡಿ
ನೊಳಂಬ-ವಾಡಿ-ಯನ್ನು
ನೊಳಂಬಾದಿ-ರಾಜ
ನೊಳಂಬಾದಿ-ರಾಜನು
ನೊಳಂಬಾದಿ-ರಾಜ-ನುಕ್ರಿಶ
ನೊಳಂಬಾಧಿ-ರಾಜನು
ನೊಳಂಬಾಧಿ-ರಾಜ-ರನ್ನು
ನೊಳಂಬಿ
ನೊಳಬಂನೂ
ನೊೞಂಬ
ನೋಂತು
ನೋಡ-ಬಹುದು
ನೋಡಿ
ನೋಡಿ-ಕೊಂಡು
ನೋಡಿ-ಕೊಳ್ಳುತ್ತಿದ್ದ
ನೋಡಿ-ಕೊಳ್ಳುತ್ತಿದ್ದರು
ನೋಡಿ-ಕೊಳ್ಳುತ್ತಿದ್ದ-ರೆಂದು
ನೋಡಿ-ಕೊಳ್ಳುತ್ತಿದ್ದ-ರೆಂಬುದು
ನೋಡಿ-ಕೊಳ್ಳುತ್ತಿದ್ದ-ವನೇ
ನೋಡಿ-ಕೊಳ್ಳುತ್ತಿದ್ದವು
ನೋಡಿ-ಕೊಳ್ಳುವ-ವರು
ನೋಡಿದ
ನೋಡಿದರೆ
ನೋಡಿದಾಗ
ನೋಡೆ
ನೋವಿನ
ನೌಕಾ-ಸೇನೆಯ
ನ್ನು
ನ್ಯಾಯ-ತೀರ್ಮಾನ
ನ್ಯಾಯ-ತೀರ್ಮಾನ-ವನ್ನು
ನ್ಯೂನಿಸ್ರು
ಪ
ಪಂಕ್ತಿಯ
ಪಂಗಡಕ್ಕೆ
ಪಂಚಂಪಲ್ಲಿ
ಪಂಚ-ಕಂಬಿಮೇಳ
ಪಂಚ-ಗೊಂಡ
ಪಂಚದ
ಪಂಚ-ನೇತ್ರಧ್ವಜ
ಪಂಚಪ್ರಧಾನ
ಪಂಚಪ್ರಧಾನ-ರಲ್ಲಿ
ಪಂಚ-ಬ-ಸದಿ-ಯೊಳಗೆ
ಪಂಚ-ಬಾಣ-ಕವಿಯ
ಪಂಚ-ಮಠ-ಗ-ಳಿಗೆ
ಪಂಚಮಠಸ್ಥಾನ-ಪತಿ
ಪಂಚಮಠಸ್ಥಾನ-ಪತಿ-ಗಳ
ಪಂಚಮಠಸ್ಥಾನ-ಪತಿ-ಗಳು
ಪಂಚ-ಮಹಾಪ್ರಧಾನರ
ಪಂಚ-ಮಹಾಪ್ರಧಾನ-ರಲ್ಲಿ
ಪಂಚ-ಮಹಾಪ್ರಧಾನ-ರೆಂದರೆ
ಪಂಚ-ಮಹಾ-ಶಬ್ದ
ಪಂಚಮಿ
ಪಂಚಮುಖ-ವಿಭಾಡ
ಪಂಚ-ರಿಗೆ
ಪಂಚರು
ಪಂಚಲಿಂಗ
ಪಂಚಲಿಂಗ-ಗ-ಳಿಗೆ
ಪಂಚ-ಲಿಂಗೇಶ್ವರ
ಪಂಚ-ವನ್
ಪಂಚ-ವನ್ಮಹಾ-ರಾಯ-ನೆಂಬ
ಪಂಚವಮಾ-ರಾಯ-ನಾದ
ಪಂಚಾನನಂ
ಪಂಚಾರತಿ
ಪಂಚಿಕೇಶ್ವರ
ಪಂಜದ
ಪಂಡಿ-ತನ
ಪಂಡಿ-ತನೆಂದು
ಪಂಡಿತರ
ಪಂಡಿತರಿಗೂ
ಪಂಡಿತರಿಗೆ
ಪಂಡಿತರು
ಪಂಡಿತ-ವರ್ಯರು
ಪಂಡಿತ-ಹಳ್ಳಿ
ಪಂತಳೆದಂ
ಪಂದಲ-ದೇವ
ಪಂದಲ-ದೇವನು
ಪಂನಗವೈನತೇಯ
ಪಂನಾಯ-ವನ್ನು
ಪಂಪನ
ಪಂಪ-ಭಾರ-ತ-ದಲ್ಲಿ
ಪಂಪ-ರಾಜ
ಪಕ್ಕ
ಪಕ್ಕದ
ಪಕ್ಕ-ದಲ್ಲಿ
ಪಕ್ಕ-ದಲ್ಲಿದ್ದ
ಪಕ್ಕ-ದಲ್ಲಿಯೇ
ಪಕ್ಕ-ದಲ್ಲಿ-ರುವ
ಪಕ್ಕ-ದಲ್ಲೇ
ಪಕ್ಷ
ಪಕ್ಷ-ದ-ವ-ರನ್ನು
ಪಕ್ಷ-ಪಾ-ತಿ-ಗಳಾಗಿದ್ದರು
ಪಕ್ಷ-ಪಾ-ತಿ-ಗಳಾಗಿದ್ದರೂ
ಪಕ್ಷ-ವಹಿಸಿ
ಪಕ್ಷಾರ್ಧ-ದಲ್ಲಿ
ಪಗೋಡಾ
ಪಟೇಲ
ಪಟೇಲ್
ಪಟ್ಟ
ಪಟ್ಟಂಗಟ್ಟಿದ
ಪಟ್ಟಂಗಟ್ಟಿದ-ನೆಂದು
ಪಟ್ಟ-ಕಟ್ಟಿದ
ಪಟ್ಟ-ಕಟ್ಟಿ-ದನು
ಪಟ್ಟ-ಕಟ್ಟಿ-ದರು
ಪಟ್ಟ-ಕಟ್ಟಿ-ದ-ರೆಂದು
ಪಟ್ಟ-ಕಟ್ಟಿ-ಸಿದ
ಪಟ್ಟಕ್ಕೆ
ಪಟ್ಟಕ್ಕೇರಿ-ದರು
ಪಟ್ಟಣ
ಪಟ್ಟಣಕ್ಕೆ
ಪಟ್ಟಣ-ಗಳ
ಪಟ್ಟಣ-ಗಳನ್ನು
ಪಟ್ಟಣ-ಗಳಲ್ಲಿ
ಪಟ್ಟಣ-ಗ-ಳಿಗೆ
ಪಟ್ಟಣ-ಗಳು
ಪಟ್ಟಣ-ಗೆರೆ
ಪಟ್ಟಣದ
ಪಟ್ಟಣ-ದಲ್ಲಿ
ಪಟ್ಟಣ-ದಿಂದ
ಪಟ್ಟಣ-ಪುರ
ಪಟ್ಟಣ-ವನ್ನಾಗಿ
ಪಟ್ಟಣ-ವನ್ನು
ಪಟ್ಟಣ-ವಾಗಿದ್ದಿರ-ಬಹು-ದೆಂದು
ಪಟ್ಟಣವು
ಪಟ್ಟಣ-ವೆಂದು
ಪಟ್ಟಣ-ಸೆಟ್ಟಿ
ಪಟ್ಟಣ-ಸೆಟ್ಟಿ-ಗ-ಳಿಗೆ
ಪಟ್ಟಣಸ್ವಾಮಿ
ಪಟ್ಟಣಸ್ವಾಮಿ-ಗಳ
ಪಟ್ಟಣಸ್ವಾಮಿ-ಗ-ಳಿಗೆ
ಪಟ್ಟಣಸ್ವಾಮಿ-ಯಾಗಿದ್ದ
ಪಟ್ಟ-ದ-ರಸಿ
ಪಟ್ಟ-ದಾನೆ-ಯಂತೆ
ಪಟ್ಟ-ಬಂಧ
ಪಟ್ಟಮಂ
ಪಟ್ಟ-ಮಹಾ-ದೇವಿ
ಪಟ್ಟ-ಯಙ್ಗನ್
ಪಟ್ಟ-ಯಾಂಗ-ನಿಗೆ
ಪಟ್ಟ-ಯೆ-ಲೆಯ
ಪಟ್ಟ-ವನು
ಪಟ್ಟ-ವನ್ನು
ಪಟ್ಟ-ವರ್ಧನರ
ಪಟ್ಟ-ವಾಗಿ
ಪಟ್ಟ-ವಾಯಿತು
ಪಟ್ಟ-ವಾಯಿ-ತೆಂದೂ
ಪಟ್ಟವೂ
ಪಟ್ಟ-ವೇರಿ-ದನು
ಪಟ್ಟ-ಸಾಲೆ-ಯನ್ನು
ಪಟ್ಟ-ಸಾಹಣಿ
ಪಟ್ಟ-ಸಾಹಣಿ-ಯಾಗಿ
ಪಟ್ಟ-ಸಾಹಣಿಯು
ಪಟ್ಟ-ಸೋಮ-ನ-ಹಳ್ಳಿ
ಪಟ್ಟಾಭಿಷಿಕ್ತ-ನಾ-ದನು
ಪಟ್ಟಾಭಿಷಿಕ್ತನಾ-ದ-ನೆಂದು
ಪಟ್ಟಾಭಿಷೇಕ
ಪಟ್ಟಾಭಿಷೇಕ-ವನ್ನು
ಪಟ್ಟಾಭಿಷೇಕ-ವನ್ನೇ
ಪಟ್ಟಾಭಿಷೇಕ-ವಾದ-ಕೂಡಲೇ
ಪಟ್ಟಿದ್ದಾರೆ
ಪಟ್ಟಿದ್ದಾರೆಂದು
ಪಟ್ಟಿದ್ದಾರೆೆ
ಪಟ್ಟಿ-ಮಾಡಿದ್ದಾರೆ
ಪಟ್ಟಿ-ಮಾಡುತ್ತದೆ
ಪಟ್ಟಿ-ಯನ್ನು
ಪಟ್ಟಿ-ಯನ್ನೂ
ಪಟ್ಟಿ-ಯಲ್ಲಿ
ಪಟ್ಟಿ-ರುವ
ಪಟ್ಟಿ-ರು-ವುದು
ಪಟ್ಟೆ
ಪಟ್ಟೆ-ಯಾಂಗ
ಪಟ್ಟೆ-ಯಾಂಗ-ನೆಂಬುವವ-ನಿಗೆ
ಪಟ್ಟೆ-ಯೆ-ಲೆಯ
ಪಠಿ-ಸಲು
ಪಠಿಸಿ
ಪಡಿಯ
ಪಡಿಯಾರ್ದ್ದಕ್ಷಿಣ-ಚಕ್ರ-ವರ್ತ್ತಿ
ಪಡಿಯಿಪ್ಪಂತೆ
ಪಡಿ-ಸಿದ್ದು
ಪಡಿಹಾರ
ಪಡುತ್ತಾರೆ
ಪಡುವ
ಪಡುವಂತಹ-ವರು
ಪಡುವಣ
ಪಡುವ-ನಾಡ
ಪಡುವ-ಲ-ಪಟ್ಟಣದ
ಪಡುವಲು
ಪಡುವಲ್
ಪಡುವೆಣ್ಣೆ
ಪಡೆ
ಪಡೆ-ಗಳ
ಪಡೆ-ಗ-ಳಿಗೆ
ಪಡೆದ
ಪಡೆದಂ
ಪಡೆ-ದಂತೆ
ಪಡೆ-ದ-ನಂತೆ
ಪಡೆ-ದನು
ಪಡೆ-ದ-ನೆಂದು
ಪಡೆ-ದರು
ಪಡೆ-ದ-ವನು
ಪಡೆ-ದ-ವ-ರಾಗಿದ್ದರು
ಪಡೆ-ದ-ವ-ರಾಗಿ-ರ-ಬಹುದು
ಪಡೆ-ದ-ವ-ರಾಗಿ-ರುತ್ತಿದ್ದರು
ಪಡೆ-ದ-ವರು
ಪಡೆ-ದ-ವರೂ
ಪಡೆ-ದಿತ್ತು
ಪಡೆ-ದಿದ್ದ
ಪಡೆ-ದಿದ್ದನು
ಪಡೆ-ದಿದ್ದರು
ಪಡೆ-ದಿದ್ದ-ರೆಂದು
ಪಡೆ-ದಿದ್ದ-ವರು
ಪಡೆ-ದಿದ್ದಾನೆ
ಪಡೆ-ದಿದ್ದಾನೆಂಬುದು
ಪಡೆ-ದಿದ್ದು
ಪಡೆ-ದಿರ-ಬಹುದು
ಪಡೆ-ದಿ-ರು-ವು-ದನ್ನು
ಪಡೆದು
ಪಡೆ-ದು-ಕೊಂಡ
ಪಡೆ-ದು-ಕೊಂಡನು
ಪಡೆ-ದು-ಕೊಂಡಿದ್ದರು
ಪಡೆ-ದು-ಕೊಂಡು
ಪಡೆ-ದು-ಕೊಳ್ಳುತ್ತಾರೆ
ಪಡೆದೇ
ಪಡೆ-ಮೆಚ್ಚೆ-ಗಂಡ
ಪಡೆಯ
ಪಡೆಯಂ
ಪಡೆ-ಯದೇ
ಪಡೆ-ಯನ್ನು
ಪಡೆ-ಯ-ಲಾಗಿದೆ
ಪಡೆ-ಯ-ಲಿಲ್ಲ
ಪಡೆ-ಯಲು
ಪಡೆ-ಯ-ವ-ರಾಗಿದ್ದು
ಪಡೆ-ಯುತ್ತಾನೆ
ಪಡೆ-ಯುತ್ತಿದ್ದು-ದ-ರಿಂದ
ಪಡೆ-ಯುವ
ಪಡೆ-ವಳ
ಪಡೆ-ವಳರ
ಪಡೈಕ್ಕಣಕ್ಕನ್
ಪಣ-ವನ್ನು
ಪಣ್ಯಾ-ಗುಣಂ
ಪತನ
ಪತನ-ಗೊಂಡ-ನಂತರ
ಪತ-ನದ
ಪತನಾ
ಪತಿ
ಪತಿ-ಭಕ್ತಂ
ಪತಿ-ಹಿ-ತದೆ-ಯೊಳು
ಪತ್ತಿಯ
ಪತ್ತಿಸೆ
ಪತ್ತೆ
ಪತ್ತೆ-ಯಾಗಿ-ರುವ
ಪತ್ತೆ-ಹಚ್ಚಿ
ಪತ್ತೊಂದಿ
ಪತ್ನಿ
ಪತ್ನಿಯ
ಪತ್ನಿ-ಯನ್ನು
ಪತ್ನಿ-ಯರ
ಪತ್ನಿ-ಯ-ರಾದ
ಪತ್ನಿ-ಯ-ರಿಗೂ
ಪತ್ನಿ-ಯರು
ಪತ್ನಿ-ಯಾದ
ಪತ್ನಿ-ಯೊರಡನೆ
ಪತ್ಮಾ-ವಸುಂಧರಾಭ್ಯಾಮಾಕಲ್ಪಂ
ಪತ್ರ
ಪತ್ರ-ಗಳನ್ನು
ಪತ್ರ-ವನ್ನು
ಪದ
ಪದ-ಕ-ವನ್ನು
ಪದ-ಗಳನ್ನು
ಪದ-ಗಳು
ಪದ-ಗಳೂ
ಪದದ
ಪದಪಿಂ
ಪದ-ವನ್ನು
ಪದ-ವನ್ನೈದಿ-ದ-ನೆಂದೂ
ಪದ-ವಾಗಿ-ರ-ಲಿಲ್ಲ
ಪದವಿ
ಪದ-ವಿ-ಗಳನ್ನು
ಪದ-ವಿ-ಗ-ಳನ್ನೂ
ಪದ-ವಿ-ಗಳು
ಪದ-ವಿಗೆ
ಪದ-ವಿ-ಗೇರುತ್ತಿದ್ದ-ರೆಂದು
ಪದ-ವಿಯ
ಪದ-ವಿ-ಯನ್ನು
ಪದ-ವಿ-ಯನ್ನೂ
ಪದ-ವಿ-ಯಲ್ಲಿ
ಪದ-ವಿ-ಯಾ-ಗಿತ್ತು
ಪದ-ವಿ-ಯಿಂದ
ಪದ-ವಿಯು
ಪದ-ವಿ-ಯೆಂದು
ಪದವೀ
ಪದವು
ಪದಾಕ್ರಾಂತ-ವಾದ
ಪದಾತಿ
ಪದಾದ್ವಿಪ್ರಗಣೇ
ಪದಾಬ್ಜಂಗಳ-ವರ್ಗ್ಗೆ
ಪದಾಬ್ಜ-ಗಳು
ಪದಾರಾಧ-ಕನುಂ
ಪದಿ-ನಾಲ್ಕು
ಪದಿ-ನಾಲ್ಕು-ನಾ-ಡಿನ
ಪದುಮಂಣನ
ಪದುಮಣ್ಣ
ಪದುಮಣ್ಣನ
ಪದುಮಣ್ಣ-ನ-ವರ
ಪದುಮಣ್ಣನು
ಪದುಮಣ್ಣ-ಸೆಟ್ಟರ
ಪದುಮನಾಭ-ಪುರದ
ಪದು-ಮಲ-ದೇವಿ
ಪದು-ಮಲ-ದೇವಿ-ಯರ
ಪದುಳಂ
ಪದುಳದಿಂ
ಪದೇ
ಪದೋನ್ನತಿ
ಪದ್ಧತಿ
ಪದ್ಧತಿಗೆ
ಪದ್ಧತಿಯ
ಪದ್ಧತಿ-ಯನ್ನು
ಪದ್ಧತಿ-ಯಿಂದ
ಪದ್ಧತಿಯು
ಪದ್ಮನಾಭನ
ಪದ್ಮಪ್ರಭ-ನೆಂದು
ಪದ್ಮಲ-ದೇವಿ
ಪದ್ಮಾವ-ತಿ-ದೇವೀಲಬ್ಧ-ವರಪ್ರಸಾದಂ
ಪದ್ಮೋಪಜೀವಿ-ಯಾಗಿ
ಪದ್ಯ
ಪದ್ಯ-ಗಳ
ಪದ್ಯ-ಗಳನ್ನು
ಪದ್ಯ-ಗಳನ್ನೊಳ-ಗೊಂಡ
ಪದ್ಯ-ಗಳಲ್ಲಿ
ಪದ್ಯ-ಗಳಿವೆ
ಪದ್ಯದ
ಪದ್ಯ-ದಲ್ಲಿ
ಪದ್ಯ-ರೂಪ-ದಲ್ಲಿ
ಪದ್ಯವೂ
ಪನಃ
ಪನರ್
ಪನೆ-ಕೊಳ
ಪನ್ನಗವೈನತೇಯ-ನೆನಿಸಿದ್ದ
ಪನ್ನಗಶಾಯೀ
ಪನ್ನರ್ವ್ವರ್ಗ್ಗಂ
ಪನ್ನಾಯ
ಪನ್ನಾಯ-ವನ್ನು
ಪನ್ನಿರ್ಚ್ಛಾಸಿರ
ಪನ್ನಿರ್ಛಾಸಿರ
ಪನ್ನಿರ್ವ್ವರ್ಗೆ
ಪನ್ನೆರಡಕ್ಕೆ
ಪನ್ನೆರಡನ್ನು
ಪನ್ನೆರಡರ
ಪನ್ನೆರಡರಲ್ಲಿದ್ದ
ಪನ್ನೆ-ರಡು
ಪಯಣ-ಬೆಳೆಸಿ
ಪಯೋಜಭಾನು
ಪಯೋಧಿ
ಪರ
ಪರಂಪರಾಗತ
ಪರಂಪರಾನುಗತ-ವಾ-ಗಿತ್ತು
ಪರಂಪರೆ
ಪರಂಪ-ರೆಗೆ
ಪರಂಪ-ರೆಯ
ಪರಂಪರೆ-ಯ-ವರು
ಪರಂಪರೆ-ಯಿಂದ
ಪರಂಪ-ರೆಯೂ
ಪರಕ-ಪಾಂಡ್ಯರು
ಪರಕೇ-ಸರಿ
ಪರದ-ರ-ಕುಲದ
ಪರದಾರ-ಸ-ಹೋದರಃ
ಪರ-ನಾರಿ
ಪರನಾರೀದೂ-ರನುಂ
ಪರನಾರೀ-ಪುತ್ರನುಂ
ಪರನಾರೀ-ಸೋ-ದರ
ಪರ-ಬಲ-ಭೀಮ
ಪರಬಳ-ಕಕ್ಷ
ಪರಬಳಕೃ-ತಾಂತ
ಪರಬಳಜಳಧಿಬಡಬಾನಳಂ
ಪರಬಳಭಕ್ಷಕ
ಪರ-ಭಾರೆ
ಪರಮ
ಪರಮ-ಗೂಳ
ಪರಮ-ಗೂಳನ
ಪರಮ-ಗೂಳನು
ಪರಮ-ಗೂಳನೂ
ಪರಮ-ಗೂಳ-ನೆಂದು
ಪರಮ-ಗೂಳ-ನೆಂದೂ
ಪರಮ-ಗೂಳ-ಪತ್ನಿ
ಪರಮ-ಜೈನರೂ
ಪರಮ-ತಸಹಿಷ್ಣತೆ
ಪರಮಬ್ಬೆ
ಪರಮಬ್ಬೆಯು
ಪರಮ-ಭಾಗ-ವತ
ಪರಮ-ವಿಶ್ವಾಸಿ
ಪರಮಾರ
ಪರಮಾ-ರರ
ಪರಮೇಶ್ವರ
ಪರಮೇಶ್ವರ-ವರ್ಮ-ನಿರ-ಬಹುದು
ಪರಮೋಚ್ಛ
ಪರ-ರಾಜ-ಭಯಂಕರ
ಪರ-ರಾಜ-ಭಯಂಕರಃ
ಪರ-ರಾಯ
ಪರ-ರಾಯ-ಭಯಂಕರಃ
ಪರ-ರಾಷ್ಟ್ರ
ಪರ-ವಾಗಿ
ಪರ-ವಾದಿ-ಮಲ್ಲ
ಪರವೆಂಡಿರಣ್ಣ
ಪರವೆಣ್ಡಿರಣ್ನನೀ-ಸರಯ್ಯ
ಪರಸ್ಪರ
ಪರಾಂತಕನ
ಪರಾಂತಕ-ನಿಗೆ
ಪರಾಕು
ಪರಾಕ್ರಮ
ಪರಾಕ್ರಮಕ್ಕೆ
ಪರಾಕ್ರಮ-ಗಳಿಂದ
ಪರಾಕ್ರಮ-ದಿಂದ
ಪರಾಕ್ರಮ-ವನ್ನು
ಪರಾಕ್ರಮಿ-ಯಾದ
ಪರಾಜಿತ-ಗೊಂಡ
ಪರಾಧೀನತೆ-ಯಿಂದ
ಪರಾಭವ-ಗೊಳಿಸಿ
ಪರಾರಿ-ಯಾದರು
ಪರಿ-ಗಣಿಸಿ-ದರೆ
ಪರಿ-ಗಣಿಸಿ-ರುತ್ತಾರೆ
ಪರಿ-ಗೆರೆ
ಪರಿಚಯಾತ್ಮಕ
ಪರಿ-ಜನ-ಪರಿವೃತ-ನಾಗಿ
ಪರಿಣತ-ನೆಂದು
ಪರಿಣಾಮ-ವಾಗಿ
ಪರಿತಾಪಹತ
ಪರಿ-ಪಾಲಿತ-ವಾದ
ಪರಿ-ಪಾಲಿಸು-ವಂತೆ
ಪರಿ-ವಂತೆ
ಪರಿ-ವರ್ತನೆ
ಪರಿ-ವರ್ತನೆ-ಯಾಗಿ-ದೆ-ಯೆಂದು
ಪರಿ-ವರ್ತಿ-ತ-ವಾಗಿದೆ
ಪರಿ-ವರ್ತಿ-ತವಾಯಿತೆಂಬುದು
ಪರಿ-ವರ್ತಿ-ಸ-ಲಾ-ಯಿತೆಂದು
ಪರಿ-ವಾರ
ಪರಿವೃಢಃ
ಪರಿವೇಷ್ಟಿತೇ
ಪರಿಶೀಲನೆ
ಪರಿಶೀಲನೆಗೆ
ಪರಿಶೀಲನೆ-ಯಿಂದ
ಪರಿಶೀಲಿಸ-ಬಹುದು
ಪರಿಶೀಲಿಸಬೇಕಾಗುತ್ತದೆ
ಪರಿಶೀಲಿ-ಸ-ಲಾಗಿದೆ
ಪರಿಶೀಲಿ-ಸಿದಾಗ
ಪರಿಶೀಲಿ-ಸಿದ್ದು
ಪರಿಷತ್ತಿ-ನಲ್ಲಿ
ಪರಿಷತ್ತಿ-ನ-ವರು
ಪರಿಷ್ಕರಿಸಿ
ಪರಿಷ್ಕೃತ
ಪರಿಷ್ಕೃತ-ಗೊಂಡ
ಪರಿ-ಸರಕ್ಕೆ
ಪರಿ-ಸರ-ದಲ್ಲಿರುವ
ಪರಿ-ಸರದ್ದೆಂದು
ಪರಿಸೇವ್ಯ-ಮಾನಃ
ಪರಿಸ್ಥಿತಿ
ಪರಿಸ್ಥಿತಿ-ಗಳ
ಪರಿಸ್ಥಿತಿ-ಯನ್ನು
ಪರುವಿ
ಪರುಷೆ
ಪರುಷೆಯ
ಪರೋಕ್ಷ
ಪರೋಕ್ಷ-ವಿನ-ಯ-ವಾಗಿ
ಪರ್ಯಾಯ
ಪರ್ಯಾಯ-ದಲು
ಪರ್ವಿ-ಯಲ್ಲಿ
ಪರ್ಷಿ-ಯನ್
ಪಲಕ್ಕಿ
ಪಲಗೈಯಾನುಮ್
ಪಲರ್ಪ್ಪೊಗೞ್ದೆ-ನದಟಿಂ
ಪಲಸಿಗೆ
ಪಲಾ-ಯನ
ಪಲಾ-ಯನಂ
ಪಲಾ-ಯನ-ಮಾಡಿ
ಪಲ್ಲಕ್ಕಿ
ಪಲ್ಲಪಂಡಿತ
ಪಲ್ಲ-ಪಂಡಿತರಿಗೆ
ಪಲ್ಲಪೆರಿಯೂರಿ-ನಲ್ಲಿ
ಪಲ್ಲವ
ಪಲ್ಲವ-ತಟಾಕ-ವೆಂಬ
ಪಲ್ಲವ-ತಟಾಕಾ
ಪಲ್ಲವತ್ರಿಣೇತ್ರ
ಪಲ್ಲವನ್ವಾಯ
ಪಲ್ಲವ-ಮಲ್ಲ
ಪಲ್ಲವರ
ಪಲ್ಲವ-ರನ್ನೇ
ಪಲ್ಲವ-ರಾಯ
ಪಲ್ಲವ-ರಾಯ-ನನ್ನು
ಪಲ್ಲವ-ರಾಯ-ನೆಂಬ
ಪಲ್ಲವರು
ಪಲ್ಲವ-ರೆಂದು
ಪಲ್ಲವ-ರೊಡನೆ
ಪಲ್ಲವ-ವಂಶ-ದ-ವ-ರೆಂದು
ಪಲ್ಲವಾದಿತ್ಯ
ಪಲ್ಲವಾಧಿ-ರಾಜ
ಪಲ್ಲವಾಧಿ-ರಾಜನ
ಪಲ್ಲವಾಧಿ-ರಾಜನು
ಪಲ್ಲವಾಧಿ-ರಾಜನೂ
ಪಲ್ಲವಾಧಿ-ರಾಜರ
ಪಲ್ಲವೆಂದ್ರ-ನನ್ನು
ಪಳಗಿದ-ವ-ರೆಂಬ
ಪಳಗಿಸಿ
ಪಳಗಿ-ಸುವ-ವರ
ಪಳಗಿ-ಸುವ-ವ-ರಿಗೆ
ಪಳಗಿ-ಸುವುದ-ರಲ್ಲಿ
ಪಳಿಯುಲನು
ಪಳಿಯು-ಳನ
ಪವಿತ್ರ
ಪವಿತ್ರ-ಜಲ-ವನ್ನು
ಪವಿತ್ರಾರೋಹಣ
ಪವಿತ್ರೀಕೃ-ತೋತ್ತಮಾಂಗನುಂ
ಪವಿತ್ರೀಕ್ರಿ-ತೋತ್ತಮಾಂಗ
ಪಶ್ಚಿಮ
ಪಶ್ಚಿಮಕ್ಕೂ
ಪಶ್ಚಿಮಕ್ಕೆ
ಪಶ್ಚಿಮದ
ಪಶ್ಚಿಮ-ಭಾಗ
ಪಶ್ಚಿಮ-ರಂಗ
ಪಶ್ಚಿಮ-ರಂಗ-ರಾಜ-ನ-ಗರೀ
ಪಶ್ಚಿಮ-ರಂಗೇ
ಪಶ್ಚಿಮ-ವಾ-ಹಿನಿ
ಪಶ್ಚಿಮೋತ್ತರ-ವಾಗಿ
ಪಸಯಿತ-ನಿಗೆ
ಪಸಾಯತರು
ಪಸಾಯಿತ
ಪಸಾಯಿತ-ನಿಗೆ
ಪಸಾಯ್ತ
ಪಹ-ರದು
ಪಾಂಚಾಲ-ದೇವ
ಪಾಂಚಾಲ-ದೇವ-ನನ್ನು
ಪಾಂಚಾಳ-ದ-ವರು
ಪಾಂಡವ-ಪುರ
ಪಾಂಡವ-ಪುರ-ದಲ್ಲಿವೆ
ಪಾಂಡ-ವರ
ಪಾಂಡ-ವರ-ಗುಹೆಯ
ಪಾಂಡವ-ರಿಂದ
ಪಾಂಡ-ವರು
ಪಾಂಡ್ಯ
ಪಾಂಡ್ಯ-ಕುಲ
ಪಾಂಡ್ಯನ
ಪಾಂಡ್ಯನು
ಪಾಂಡ್ಯ-ಪಾಡಿ
ಪಾಂಡ್ಯ-ಬಲ
ಪಾಂಡ್ಯರ
ಪಾಂಡ್ಯ-ರನ್ನು
ಪಾಂಡ್ಯ-ರಾಜ-ನೊಡನೆ
ಪಾಂಡ್ಯ-ರಾಜರ
ಪಾಂಡ್ಯ-ರಾಜ್ಯ
ಪಾಂಡ್ಯ-ರಾಜ್ಯ-ವನ್ನು
ಪಾಂಡ್ಯ-ರಾಯ
ಪಾಂಡ್ಯ-ರಾಯಪ್ರತಿಷ್ಟಾಚಾರ್ಯ್ಯ
ಪಾಂಡ್ಯ-ರಿಂದ
ಪಾಂಡ್ಯ-ರಿಗೂ
ಪಾಂಡ್ಯರು
ಪಾಂಬಬ್ಬೆ
ಪಾಚ-ಯಪ್ಪ
ಪಾಚಿಯಪ್ಪನ
ಪಾಛಾ-ಬಾದ-ಶಹಾ-ರ-ವರ
ಪಾಠ-ದಿಂದ
ಪಾಠ-ವನ್ನು
ಪಾಠವು
ಪಾಠಶಾಲೆ
ಪಾಠ-ಶಾಲೆ-ಗಳನ್ನು
ಪಾಡಿ
ಪಾಡಿ-ಗಳೆತ್ತಿ
ಪಾತ್ರ
ಪಾತ್ರಕ್ಕೆ
ಪಾತ್ರ-ನಾಗಿದ್ದಾನೆ
ಪಾತ್ರ-ವನ್ನು
ಪಾತ್ರ-ವಹಿಸಿದ್ದ-ನೆಂಬುದು
ಪಾತ್ರವೂ
ಪಾತ್ರೆ-ಗಳನ್ನು
ಪಾದ
ಪಾದ-ಗಳನ್ನು
ಪಾದ-ಚಾರಕ-ನಾದ
ಪಾದ-ದ-ಬಳಿ
ಪಾದ-ಪದ್ಮಾರಾಧಕ
ಪಾದ-ಪದ್ಮಾರಾಧಕ-ರು-ಮಪ್ಪ
ಪಾದ-ಪದ್ಮೋಪಜೀವಿ
ಪಾದ-ಪದ್ಮೋಪಜೀವಿ-ಯಾಗಿ
ಪಾದ-ಪದ್ಮೋಪಜೀವಿ-ಯಾಗಿದ್ದ
ಪಾದ-ಪದ್ಮೋಪಜೀವಿ-ಯಾಗಿದ್ದ-ನೆಂದು
ಪಾದ-ಪದ್ಮೋಪಜೀವಿ-ಯೆಂದು
ಪಾದ-ಪೂಜೆ-ಯನ್ನು
ಪಾದ-ಪೂಜೆ-ಯಾಗಿ
ಪಾದ-ಸೇವಕ-ನಾದ
ಪಾದ-ಸೇವಕ-ಳಾದ
ಪಾದಾಬ್ಜಕೃಕಟಾಯಿತಚೇತಸಃ
ಪಾದಾರಾಧ-ಕನುಂ
ಪಾದಾರಾಧ-ಕನು-ಮಪ್ಪ
ಪಾದಾರ್ಚನೆಗೆ
ಪಾಪಣ್ಣನು
ಪಾಪಯ್ಯ-ನ-ಕೊಪ್ಪಲು
ಪಾಪರ್
ಪಾಪಾರ-ಪಟ್ಟಿ
ಪಾಪಾರ್ಪಟ್ಟಿ
ಪಾಯಾತ್
ಪಾಯೆಸ್
ಪಾರಂಪರಿಕ
ಪಾರರಾ
ಪಾರಾದ-ನೆಂದು
ಪಾರಿಜಾತ
ಪಾರಿಜಾತಾಪ-ಹರ-ಣಮು
ಪಾರಿ-ಭಾಷಿಕ
ಪಾರೀಷ-ದೇವರ
ಪಾರುಪತ್ತೇಗಾರ್
ಪಾರ್ಥಸಾರಥಿ
ಪಾರ್ಥಿವಸ್ಯಾಸ್ಯ
ಪಾರ್ವತಿ-ಯನ್ನು
ಪಾರ್ಶ್ವ
ಪಾರ್ಶ್ವ-ಜಿನಗೃಹ-ವನ್ನು
ಪಾರ್ಶ್ವ-ಜಿ-ನೇಶ್ವರ
ಪಾರ್ಶ್ವ-ದಲ್ಲಿ
ಪಾರ್ಶ್ವ-ದೇವ
ಪಾರ್ಶ್ವ-ದೇವನು
ಪಾರ್ಶ್ವ-ದೇವರ
ಪಾರ್ಶ್ವ-ನಾಥ
ಪಾರ್ಶ್ವ-ನಾಥ-ದೇವರು
ಪಾರ್ಶ್ವ-ಪುರ-ವನ್ನಾಗಿ
ಪಾಲ-ಯನ-ಖಿಳಾಂ
ಪಾಲ-ಯನ್
ಪಾಲ-ಹಳ್ಳಿ
ಪಾಲಿತ
ಪಾಲಿನ
ಪಾಲಿಸ-ಬೇಕೆಂದು
ಪಾಲಿ-ಸಲು
ಪಾಲಿ-ಸಿದ
ಪಾಲಿ-ಸಿದ್ದ
ಪಾಲಿ-ಸಿದ್ದನು
ಪಾಲಿ-ಸಿದ್ದ-ನೆಂದು
ಪಾಲಿ-ಸಿದ್ದ-ನೆಂದೂ
ಪಾಲಿ-ಸುತ್ತಿದ್ದರು
ಪಾಲು
ಪಾಲು-ದಾರಿ-ಕೆಯ
ಪಾಲ್ಗೊಂಡಿರ-ಬಹುದು
ಪಾಲ್ಯಂ
ಪಾಳು
ಪಾಳು-ಬ-ಸದಿ-ಯಲ್ಲಿರುವ
ಪಾಳೆಗಾರ-ನಾಗಿ
ಪಾಳೆಯಗಾ-ರನ
ಪಾಳೆಯಗಾರ-ನಾಗಿದ್ದ
ಪಾಳೆಯಗಾರ-ನಾಗಿದ್ದ-ನೆಂದು
ಪಾಳೆಯಗಾ-ರರ
ಪಾಳೆಯ-ಗಾರ-ರನ್ನು
ಪಾಳೆಯ-ಗಾ-ರರು
ಪಾಳೆಯ-ಪಟ್ಟನ್ನು
ಪಾಳೆಯ-ಪಟ್ಟೆಂದು
ಪಾಳ್ಯ
ಪಾವಗಡ
ಪಿಂಛಾತ-ಪತ್ರಾನ್ವಿತಾ-ಸನ
ಪಿಡಿದಂ
ಪಿತಾಮಹ
ಪಿತಾಮ-ಹನ
ಪಿತಾಮಹ-ರೆನಿಸಿ-ಕೊಂಡ
ಪಿತ್ರಾರ್ಜಿತ-ವಾಗಿ
ಪಿನಾಕಿನಿಯ
ಪಿಬಿ
ಪಿಬಿ-ದೇಸಾಯಿ
ಪಿಬಿ-ದೇಸಾಯಿ-ಯ-ವರು
ಪಿರಿಯ
ಪಿರಿಯ-ಆಳ್ವಿಕೆ
ಪಿರಿಯ-ಕಳಲೆಯ
ಪಿರಿಯ-ದಂಡ-ನಾಯಕ
ಪಿರಿಯ-ಪಟ್ಟದ
ಪಿರಿಯ-ರಸಿ
ಪಿರಿಯೊಡೆ-ಯನ
ಪಿರಿಯೊಡೆಯ-ನನ್ನು
ಪಿರಿಯೊಡೆಯ-ನಿಗೂ
ಪಿರಿಯೊಡೆ-ಯನು
ಪಿರಿಯೊಡೆ-ಯನೂ
ಪಿರಿಯೊಡೆಯ-ನೆಂಬ
ಪಿರಿಯೊಡೆ-ಯರು
ಪೀಠದ
ಪೀಠಿ-ಕೆ-ಗಳನ್ನು
ಪೀಠಿಕೆ-ಗಳು
ಪೀಠಿಕೆ-ಯನ್ನು
ಪೀಠಿಕೆ-ಯಲ್ಲಿ
ಪೀಠೋಪ-ಕರ-ಣ-ಗಳು
ಪುಂಡರಿಕ
ಪುಂಡಾಟಿಕೆ-ಯನ್ನು
ಪುಂಣ್ಯಕ್ರುತ
ಪುಂಣ್ಯ-ಚರಿತಂ
ಪುಗಿರಿ-ನಾ-ಡಿನ
ಪುಗಿರಿ-ನಾಡು-ಪೊ-ಗರ್ನಾಡು
ಪುಗಿಸಿ-ದಾರ್
ಪುಟಕ್ಕೆ
ಪುಣಸ-ಮಯ್ಯ-ನನ್ನು
ಪುಣಸಿ-ಮಯ್ಯನು
ಪುಣಸೆ-ಪಟ್ಟಿ
ಪುಣಿಗದ
ಪುಣಿಸ
ಪುಣಿಸ-ದಂಡ-ನಾಥನು
ಪುಣಿಸ-ಮಯ್ಯ
ಪುಣಿಸ-ಮಯ್ಯನ
ಪುಣಿಸ-ಮಯ್ಯನು
ಪುಣಿಸ-ಮಯ್ಯ-ನೆಂದು-ಹೇಳಿದೆ
ಪುಣಿಸಮ್ಮ
ಪುಣಿಸಶ್ರೀ
ಪುಣ್ಯ
ಪುಣ್ಯ-ಜನ-ಧಾಮ
ಪುಣ್ಯ-ವಾಗ-ಬೇಕೆಂದು
ಪುಣ್ಯಾಹವೈಃ
ಪುಣ್ಯೇ
ಪುತ್ತೂರಿ-ನಲ್ಲಿ
ಪುತ್ತೂರು
ಪುತ್ರ
ಪುತ್ರ-ನೆಂದು
ಪುತ್ರ-ರತ್ನ-ರನ್ನೂ
ಪುತ್ರ-ರಲ್ಲಿ
ಪುತ್ರ-ರಾದ
ಪುತ್ರರು
ಪುತ್ರರೂ
ಪುತ್ರ-ರೆಂಬ
ಪುತ್ರರೋ
ಪುತ್ರ-ಸ-ಮಾನ-ನಾಗಿ
ಪುತ್ರ-ಸ-ಮಾನ-ನಾಗಿದ್ದ
ಪುತ್ರಿ
ಪುತ್ರಿಯ
ಪುತ್ರಿ-ಯ-ರನ್ನು
ಪುತ್ರೋತ್ಸವ-ಮಾಗಲ್
ಪುತ್ರೋತ್ಸವ-ವಾ-ದಾಗ
ಪುನಃ
ಪುನರಪಿ
ಪುನರುಜ್ಜೀ-ವನ-ಗೊಳಿಸು-ವಂತೆ
ಪುನರ್
ಪುನರ್ದತ್ತ-ಯಾಗಿ
ಪುನರ್ದತ್ತಿ-ಯಾಗಿ
ಪುನರ್ದಾನ
ಪುನರ್ದ್ಧಾರಾ-ಪೂರ್ವಕಂ
ಪುನರ್ಧಾರಾ-ಪೂರ್ವ-ಕ-ವಾಗಿ
ಪುನರ್ನಿಗದಿ-ಪಡಿಸಿ
ಪುನರ್ರಚಿಸ-ಬಹುದು
ಪುನರ್ರಚಿ-ಸ-ಲಾಯಿತು
ಪುನರ್ರಚಿಸಿ
ಪುಮಾನೇಷಃ
ಪುರ
ಪುರಃ
ಪುರ-ಗಳನ್ನು
ಪುರ-ಗಳು
ಪುರ-ಣೋಕ್ತ
ಪುರದ
ಪುರ-ದ-ಮಾಗಣಿಗೆ
ಪುರ-ದಾನ-ವಾಗಿ
ಪುರ-ವನ್ನಾಗಿ
ಪುರ-ವರಾಧೀಶ್ವರ
ಪುರ-ವರ್ಗ-ದಾನ
ಪುರ-ವಾದ
ಪುರ-ಷಾಶ್ಚ
ಪುರಾಣ
ಪುರಾತತ್ತ್ವ
ಪುರಾ-ತತ್ವ
ಪುರಾ-ತತ್ವದ
ಪುರಾ-ತನ
ಪುರಾ-ತನ-ವಾ-ದುದು
ಪುರಾಧಿಪನ
ಪುರಾಯ-ತತರಂ
ಪುರಿ-ಗೆರೆ
ಪುರಿ-ಗೆರೆನ್ನು
ಪುರಿಶೈ
ಪುರುಷ
ಪುರುಷ-ನಾದ
ಪುರುಷ-ಮಾಣಿಕ್ಯ-ಸೆಟ್ಟಿಯು
ಪುರುಷ-ರಿಗೆ
ಪುರುಷರು
ಪುರುಷಾರ್ತ್ಥ
ಪುರುಷಾರ್ಥ
ಪುರುಷೋತ್ತಮ-ದೇವ
ಪುರೋಹಿತ
ಪುರೋಹಿತ-ರಿಗೆ
ಪುಲಿಗೆ-ರೆನ್ನು
ಪುಲಿ-ಯಣ್ಣ
ಪುಲಿ-ಯಣ್ಣನ
ಪುಳ್ಳೈ-ಲೋ-ಕಾಚಾರ್ಯರ
ಪುವ-ಗಾಮ-ವನ್ನು
ಪುಷ್ಕರಣಿ
ಪುಷ್ಟಿ
ಪುಷ್ಟಿ-ಯನ್ನು
ಪುಷ್ಟಿರ್ಜ್ವಯಶ್ಚ
ಪುಷ್ಪಮಾಲೆ-ಯನ್ನು
ಪುಷ್ಪೋತ್ಗಮನ
ಪುಸಿ-ವರ-ಗಂಡ
ಪುಸ್ತ-ಕಕ್ಕೆ
ಪೂಜಾ
ಪೂಜಾ-ಕಾರ್ಯ-ಗ-ಳಿಗೆ
ಪೂಜಾ-ಕೈಂಕರ್ಯ-ಗ-ಳಿಗೆ
ಪೂಜಾ-ದಿ-ಕಾರ್ಯ-ಗ-ಳಿಗೆ
ಪೂಜಾ-ರರ
ಪೂಜಿಸಿ
ಪೂಜಿಸಿ-ದನು
ಪೂಜಿ-ಸುತ್ತಿದ್ದನು
ಪೂಜಿ-ಸುವ
ಪೂಜೆಗೆ
ಪೂಜೆ-ಪುನಸ್ಕಾರ-ಗಳು
ಪೂಜೆಯ
ಪೂಜೆ-ಯನ್ನು
ಪೂಜ್ಯ-ನಲ್ತೆ
ಪೂನಾಡು-ಪುನ್ನಾಡು-ಗಳನ್ನು
ಪೂರ-ಕ-ವಾಗಿ
ಪೂರಿಗಾಲಿ
ಪೂರೈಸಿ
ಪೂರೈ-ಸಿದ
ಪೂರೈಸಿ-ದರು
ಪೂರ್ಣಯ್ಯನ-ವರ
ಪೂರ್ಣಯ್ಯ-ನೆಂಬ
ಪೂರ್ಣ-ವಾಗಿ
ಪೂರ್ತಿ
ಪೂರ್ತಿ-ಯಾಗಿ
ಪೂರ್ಬ್ಬಾಯ
ಪೂರ್ವ
ಪೂರ್ವ-ಕವೇ-ಕಚ್ಛತ್ರಚ್ಛಾಯೆ
ಪೂರ್ವಕ್ಕೂ
ಪೂರ್ವಕ್ಕೆ
ಪೂರ್ವ-ಜರು
ಪೂರ್ವದ
ಪೂರ್ವ-ದಕ್ಷಿಣ-ಪಶ್ಚಿಮ
ಪೂರ್ವ-ದಲ್ಲಿ
ಪೂರ್ವ-ದಲ್ಲಿದ್ದ
ಪೂರ್ವ-ದಿಂದಲೂ
ಪೂರ್ವ-ದಿಕ್ಕಿನ
ಪೂರ್ವ-ಪಶ್ಚಿಮ
ಪೂರ್ವ-ಪಶ್ಚಿಮ-ಗಳಲ್ಲಿ
ಪೂರ್ವ-ಪಶ್ಚಿಮ-ಸಮದ್ರಾಧಿ-ಪತಿ
ಪೂರ್ವ-ಭಾಗ-ದಲ್ಲಿ
ಪೂರ್ವ-ಮರಿ-ಯಾದೆ-ಯಲು
ಪೂರ್ವಾಚಲ-ಮಾರ್ತಾಂಡ
ಪೂರ್ವಾನ್ವಯ
ಪೂರ್ವಾರ್ಧದ-ವ-ರೆಗೆ
ಪೂರ್ವೋಕ್ತ
ಪೂವಗಾಮವೇ
ಪೂವಗಾಮೆ-ಯನ್ನು
ಪೃಥಿವೀ
ಪೃಥಿವೀ-ನೀರ್ಗುಂದ
ಪೃಥುವೀ
ಪೃಥುವೀ-ಗಾಮುಣ್ಡರು
ಪೃಥುವೀ-ನೀರ್ಗುಂದ
ಪೃಥ್ವೀ
ಪೃಥ್ವೀಂ
ಪೃಥ್ವೀ-ಗಂಗನ
ಪೃಥ್ವೀ-ಗಂಗ-ನಿದ್ದು
ಪೃಥ್ವೀ-ಪತಿ
ಪೃಥ್ವೀ-ಪತಿ-ಗಳು
ಪೃಥ್ವೀ-ಪತಿಗೆ
ಪೃಥ್ವೀ-ಪತಿ-ಯಾಗಿದ್ದಾ-ನೆಂದು
ಪೃಥ್ವೀ-ಪತಿಯು
ಪೃಥ್ವೀ-ರಾಜ್ಯಂಗೆಯುತ್ತಿರೆ
ಪೃಥ್ವೀ-ರಾಜ್ಯಂಗೆಯ್ಯುತ್ತಿದ್ದನು
ಪೆಂ
ಪೆಂಗೆ-ನಾಯ-ಕನ
ಪೆಂಪಿನ
ಪೆಗ್ಗಡೆ-ನಾಯ್ಕ
ಪೆತ್ತ
ಪೆದ್ದ
ಪೆದ್ದ-ಗವುಡು-ಗಳ
ಪೆದ್ದಣ್ಣನು
ಪೆದ್ದಿ-ರಾಜುಗೆ
ಪೆದ್ದಿ-ರಾಜುವು
ಪೆನು-ಗೊಂಡೆ
ಪೆನು-ಗೊಂಡೆಗೆ
ಪೆನು-ಗೊಂಡೆ-ದುರ್ಗ-ದಲ್ಲಿ
ಪೆನು-ಗೊಂಡೆಯ
ಪೆನು-ಗೊಂಡೆ-ಯಲ್ಲಿದ್ದ
ಪೆನು-ಗೊಂಡೆ-ಯಲ್ಲಿದ್ದ-ನೆಂದು
ಪೆನು-ಗೊಂಡೆ-ಯ-ವ-ರೆಗೆ
ಪೆನು-ಗೊಂಡೆ-ಯ-ವರೋ
ಪೆನು-ಗೊಂಡೆ-ಯಿಂದ
ಪೆನು-ಗೊಂಡೆ-ಯೊಳು
ಪೆಮ್ಮೋ-ಜನೂ
ಪೆರಂಗೂರು
ಪೆರಮ-ಗಾವುಂಡನ
ಪೆರ-ಮಾನುಡನ್
ಪೆರಮಾಳ-ದೇವ
ಪೆರಮಾಳು
ಪೆರಮಾಳೆ
ಪೆರಮಾಳೆ-ದೇವ
ಪೆರಮಾಳೆ-ದೇವನ
ಪೆರಮಾಳೆ-ದೇವ-ನಿಗೆ
ಪೆರಮಾಳೆ-ದೇವನು
ಪೆರ-ಮಾಳೆಯು
ಪೆರಾಳ್ಕೆ
ಪೆರಾಳ್ಕೆಯೇ
ಪೆರಿಯಮನೈ
ಪೆರಿ-ರಾಜು
ಪೆರುಂಕೋಟೆ-ರಾಜ್ಯದ
ಪೆರುಮಾಳ
ಪೆರುಮಾಳ-ದೇವನು
ಪೆರುಮಾಳ-ದೇವ-ರ-ಸ-ನಿಗೆ
ಪೆರುಮಾಳ-ರಸ
ಪೆರುಮಾಳಿಗೆ
ಪೆರುಮಾಳು-ಸಮುದ್ರ
ಪೆರುಮಾಳೆ
ಪೆರುಮಾಳೆ-ಚಮೂಪ-ತಿ-ಗಿಂತು
ಪೆರುಮಾಳೆ-ದೇವ
ಪೆರುಮಾಳೆ-ದೇವನ
ಪೆರುಮಾಳೆ-ದೇವ-ನಿಗಿದ್ದ
ಪೆರುಮಾಳೆ-ದೇವ-ನಿಗೆ
ಪೆರುಮಾಳೆ-ದೇವನು
ಪೆರುಮಾಳೆ-ದೇವ-ರಸನು
ಪೆರುಮಾಳೆ-ಪುರ-ವೆಂಬ
ಪೆರುಮಾಳೆಯ
ಪೆರುಮಾಳೆಯು
ಪೆರುಮಾಳೆಯೂ
ಪೆರುಮಾಳೆ-ಲಕ್ಷ್ಮೀ-ನಾ-ರಾಯಣ
ಪೆರ್ಗಡೆ
ಪೆರ್ಗಡೆ-ಯಾಗಿ-ರುತ್ತಿದ್ದನು
ಪೆರ್ಗಡೆಯು
ಪೆರ್ಗಡೆ-ಯೊಳ್
ಪೆರ್ಗಡೆ-ಹೆಗ್ಗಡೆ
ಪೆರ್ಗ್ಗಡೆ
ಪೆರ್ಗ್ಗಡೆ-ಗಳ
ಪೆರ್ಗ್ಗಡೆ-ಗಳನ್ನು
ಪೆರ್ಗ್ಗಡೆ-ಗಳು
ಪೆರ್ಗ್ಗಡೆಗೆ
ಪೆರ್ಗ್ಗಡೆ-ನಾಯಕ
ಪೆರ್ಗ್ಗಡೆ-ಯಾಗಿದ್ದ
ಪೆರ್ಗ್ಗಡೆ-ಯಾಗಿದ್ದು
ಪೆರ್ಗ್ಗಡೆಯು
ಪೆರ್ಗ್ಗಡೆಯೂ
ಪೆರ್ಗ್ಗಡೆ-ಹಿರಿ-ಯ-ಹೆಗ್ಗಡೆ-ಗಳು
ಪೆರ್ದ್ದೊರೆ
ಪೆರ್ಬಾಣ-ನ-ಹಳ್ಳಿ-ಯನ್ನು
ಪೆರ್ಬ್ಬೞ
ಪೆರ್ಮಾಡಿ
ಪೆರ್ಮಾಡಿದೇವನ
ಪೆರ್ಮಾಡಿದೇವನು
ಪೆರ್ಮಾನಡಿ
ಪೆರ್ಮಾನ-ಡಿ-ಗಳು
ಪೆರ್ಮಾನ-ಡಿಯ
ಪೆರ್ಮಾನ-ಡಿ-ಯ-ಎರಡನೇ
ಪೆರ್ಮಾನ-ಡಿ-ಯನ್ನು
ಪೆರ್ಮಾನ-ಡಿಯು
ಪೆರ್ಮಾನ-ಡಿ-ಯೆಂಬ
ಪೆರ್ಮ್ಮನಡಿ-ಗಳ
ಪೆರ್ಮ್ಮಾನಡಿ
ಪೆರ್ಮ್ಮಾನ-ಡಿಯ
ಪೆರ್ಮ್ಮೆ
ಪೆಸಾಳಿ
ಪೆಸಾಳಿ-ಹನುಮ
ಪೇಟಿ-ರಾಜಯ್ಯನು
ಪೇಟೆ
ಪೇಟೆ-ಯನ್ನಾಗಿ
ಪೇದ
ಪೇರಾಳ್ಕೆ
ಪೇರೂರಿ-ನಲ್ಲಿ
ಪೇರೂರಿ-ನಲ್ಲಿದ್ದ
ಪೇಳ್ವೆ
ಪೈಕಿ
ಪೈಗಂಬರ್
ಪೊ
ಪೊಂನಣ್ಣ
ಪೊಕ್ಕು
ಪೊಕ್ಕುಮೆ
ಪೊಗಳೆ
ಪೊಗಳ್ಗು
ಪೊಗಳ್ವಿನಂ
ಪೊಟ್ಟಳಿ-ಸುವ
ಪೊಡರ್ಪ್ಪವೇವೇಳ್ವುದೋ
ಪೊಡ-ವಿಗೆ
ಪೊನ್ನ
ಪೊನ್ನ-ಗಾವುಣ್ಡ
ಪೊನ್ನಡಿ
ಪೊನ್ನಪ್ಪ
ಪೊನ್ನ-ಲ-ದೇವಿ-ಯರ
ಪೊನ್ನಳ್ಳಿ
ಪೊನ್ನೆತ್ತಿ-ಕೊಳೆ
ಪೊನ್ನೆತ್ತಿ-ಕೊಳ್ಳುವಂತೆ
ಪೊನ್ವಿಟ್ಟು
ಪೊಯ್
ಪೊಯ್ದಿರಿದಂ
ಪೊಯ್ದು
ಪೊಯ್ಸಳ
ಪೊಯ್ಸಳ-ದೇವ
ಪೊಯ್ಸಳ-ದೇವ-ರಸ
ಪೊಯ್ಸಳ-ದೇವ-ರ-ಸರು
ಪೊಯ್ಸಳ-ದೇವ-ರಾಜ್ಯ-ದಲ್ಲಿ
ಪೊಯ್ಸಳ-ದೇವರು
ಪೊಯ್ಸಳ-ನ-ರಾಜ್ಯ
ಪೊಯ್ಸಳನು
ಪೊಯ್ಸಳ-ನೆಂಬ
ಪೊಯ್ಸಳನೇ
ಪೊರರ
ಪೊರಳ್ಚಿ
ಪೊರೆ-ದ-ನೆಂದೂ
ಪೊಲು-ವರಂ
ಪೊಳಲ
ಪೊಳಲ-ಸೆಟ್ಟಿಗೆ
ಪೊಳಲು
ಪೊಳಲ್ಚೋ-ರನ
ಪೊಳಲ್ಚೋ-ರನು
ಪೋಗಿ
ಪೋಚಲ-ದೇವಿಯು
ಪೋಚವ್ವೆ-ಗಾಗಿ
ಪೋಚಾಂಬಿಕೋ-ದರೋ-ದನ್ವತ್ಪಾರಿಜಾತಂ
ಪೋಚಿಕಬ್ಬೆ
ಪೋಚಿಕಬ್ಬೆ-ಯರ
ಪೋಚಿಕಬ್ಬೆಯು
ಪೋತ-ನಾಯಕ
ಪೋದಲಶರ್ಮ
ಪೋಮನ್
ಪೋರಿಲಿಇಭದೆ
ಪೋರಿಲಿಭದೆ
ಪೋಷಕ-ನಾಗಿ
ಪೋಷಕ-ರಲ್ಲಿ
ಪೋಷ-ಣನಿರ್ಭರ-ಭೂನವಖಂಡಃ
ಪೋಷಿ-ತನಾದ
ಪೋಸಳ-ದೇವರು
ಪೋಸಳನು
ಪೌತ್ರ
ಪೌತ್ರ-ರಾದ
ಪೌತ್ರರೂ
ಪೌರಾಣಿಕ
ಪೌರಾಣಿ-ಕ-ರಾಗಿದ್ದ-ವರು
ಪ್ಪುತ್ರ-ಮಿತ್ರಸ್ತೋಮಂ
ಪ್ರಉಡ-ದೇವ-ರಾಯ-ರಾದ
ಪ್ರಕಟ-ಗೊಂಡು
ಪ್ರಕಟಣಾ
ಪ್ರಕಟಣೆ
ಪ್ರಕಟಣೆ-ಯಾಗುತ್ತಿ-ರುವ
ಪ್ರಕಟ-ವಾಗಿದೆ
ಪ್ರಕಟ-ವಾಗಿದ್ದು
ಪ್ರಕಟ-ವಾಗಿ-ರುವ
ಪ್ರಕಟ-ವಾಗಿವೆ
ಪ್ರಕಟ-ವಾಗುತ್ತಿದ್ದ
ಪ್ರಕಟ-ವಾಗುತ್ತಿದ್ದವು
ಪ್ರಕಟ-ವಾದ
ಪ್ರಕಟ-ವಾ-ದವು
ಪ್ರಕಟಿ-ಸ-ಲಾಗಿದೆ
ಪ್ರಕಟಿ-ಸಿದರು
ಪ್ರಕಟಿ-ಸಿದೆ
ಪ್ರಕಟಿ-ಸಿದ್ದಾರೆ
ಪ್ರಕಟಿ-ಸುತ್ತಿದೆ
ಪ್ರಕಟಿ-ಸುವ
ಪ್ರಕಾರ
ಪ್ರಕಾರವೂ
ಪ್ರಕಾಶಂ
ಪ್ರಕೃತ
ಪ್ರಕೃತಃ
ಪ್ರಕೃತಯಃ
ಪ್ರಕೃತಿ-ಜನ್ಯ
ಪ್ರಕೃತೀನಾಂ
ಪ್ರಕ್ಷುಬ್ದ-ವಾಗಿ
ಪ್ರಖ್ಯಾತ
ಪ್ರಖ್ಯಾತಂ
ಪ್ರಖ್ಯಾತ-ನಾಗಿ
ಪ್ರಖ್ಯಾತ-ನಾದ
ಪ್ರಖ್ಯಾತ-ನಾ-ದನು
ಪ್ರಖ್ಯಾತ-ರಾಗಿದ್ದು
ಪ್ರಖ್ಯಾತೌ
ಪ್ರಗತಿ
ಪ್ರಚಂಡ
ಪ್ರಚಂಡ-ದಂಡ-ನಾಯಕ
ಪ್ರಚಂಡ-ದೇವ
ಪ್ರಚಲಿತ-ದಲ್ಲಿದ್ದವು
ಪ್ರಚಾರ
ಪ್ರಚಾರ-ಕರು
ಪ್ರಜಾಃ
ಪ್ರಜಾ-ಧರ್ಮ-ಪರಿ-ಪಾಲ-ನಾದಿ
ಪ್ರಜೆ
ಪ್ರಜೆ-ಗ-ಗೌಂಡ-ಗಳು
ಪ್ರಜೆ-ಗಳ
ಪ್ರಜೆ-ಗಳಿಂದ
ಪ್ರಜೆ-ಗ-ಳಿಗೆ
ಪ್ರಜೆ-ಗಳು
ಪ್ರಜೆ-ಗಾವುಂಡ
ಪ್ರಜೆ-ಗಾವುಂಡ-ಗಳು
ಪ್ರಜೆ-ಗಾವುಂಡ-ನನ್ನಾಗಿ
ಪ್ರಜೆ-ಗಾವುಂಡರ
ಪ್ರಜೆ-ಗಾವುಂಡ-ರನ್ನು
ಪ್ರಜೆ-ಗಾವುಂಡ-ರನ್ನೇ
ಪ್ರಜೆ-ಗಾವುಂಡ-ರಿಗೆ
ಪ್ರಜೆ-ಗಾವುಂಡ-ರಿರ-ಬಹುದು
ಪ್ರಜೆ-ಗಾವುಂಡರು
ಪ್ರಜೆ-ಗಾವುಂಡು-ಗಳು
ಪ್ರಜೆ-ಗೌ-ಡಿನ
ಪ್ರಜೆ-ನಾಯಕ
ಪ್ರಜೆ-ನಾಯ-ಕ-ರಿಗೆ
ಪ್ರಜೆ-ಬೀ-ಡಿನ-ವರು
ಪ್ರಜೆ-ಮೆಚ್ಚೆ-ಗಂಡ
ಪ್ರತಾನಂಗಳೊಳು
ಪ್ರತಾಪ
ಪ್ರತಾಪ-ಕಂಠೀರವ
ಪ್ರತಾಪ-ಚಕ್ರ-ವರ್ತಿ
ಪ್ರತಾಪ-ದೇವ-ರಾಯ
ಪ್ರತಾಪ-ದೇವ-ರಾಯನ
ಪ್ರತಾಪ-ದೇವ-ರಾಯ-ನೆಂಬ
ಪ್ರತಾಪ-ದೇವ-ರಾಯ-ಪುರ
ಪ್ರತಾಪ-ನಾರ-ಸಿಂಹ
ಪ್ರತಾಪ-ನಿಳಯಂ
ಪ್ರತಾಪ-ಮೆಂತೆಂದಡೆ
ಪ್ರತಾಪ-ವಾನ್
ಪ್ರತಾಪ-ವಿಜಯ-ಮ-ದನ-ಪುರ
ಪ್ರತಾಪ-ಸಮೇತರ್
ಪ್ರತಾಪ-ಹೊಯ್ಸಳ
ಪ್ರತಾಪಿ-ಗಳೂ
ಪ್ರತಿ
ಪ್ರತಿ-ಕೂಲ
ಪ್ರತಿ-ದಿನ
ಪ್ರತಿ-ನಾಕ-ಮಲ್ಲ-ನೆಂಬ
ಪ್ರತಿ-ನಾಮ-ಕರಣ
ಪ್ರತಿ-ನಾಮ-ಧೇಯ-ವಾದ
ಪ್ರತಿ-ನಾಮ-ಧೇಯ-ವಿತ್ತು
ಪ್ರತಿ-ನಾಮ-ಧೇ-ಯವುಳ್ಳ
ಪ್ರತಿ-ನಿಧಿ
ಪ್ರತಿ-ನಿಧಿ-ಗಳನ್ನು
ಪ್ರತಿ-ನಿಧಿ-ಗ-ಳಿಗೆ
ಪ್ರತಿ-ನಿಧಿ-ಗಳು
ಪ್ರತಿ-ನಿಧಿಯ
ಪ್ರತಿ-ನಿಧಿ-ಯಾಗಿ
ಪ್ರತಿ-ನಿಧಿ-ಸುತ್ತಾ
ಪ್ರತಿ-ಪನ್ನದಿ
ಪ್ರತಿ-ಬಂಧ-ಕ-ಗ-ಳಾದ
ಪ್ರತಿಮೆ
ಪ್ರತಿ-ಮೆ-ಗಳನ್ನು
ಪ್ರತಿ-ಮೆ-ಯನ್ನು
ಪ್ರತಿ-ಯನ್ನು
ಪ್ರತಿ-ಯೊಂದು
ಪ್ರತಿ-ರೋಧದ
ಪ್ರತಿ-ರೋಧ-ವನ್ನು
ಪ್ರತಿಷ್ಟಾಚಾರ್ಯ
ಪ್ರತಿಷ್ಟಾಚಾರ್ಯ್ಯ
ಪ್ರತಿಷ್ಠಾಚಾರ್ಯ
ಪ್ರತಿಷ್ಠಾಪನಾಚಾರ್ಯ
ಪ್ರತಿಷ್ಠಾಪನಾಚಾರ್ಯ-ರಾದ
ಪ್ರತಿಷ್ಠಾಪಿ-ಸ-ಲಾ-ಯಿತೆಂದು
ಪ್ರತಿಷ್ಠಾಪಿಸಿ
ಪ್ರತಿಷ್ಠಾಪಿಸಿ-ದನು
ಪ್ರತಿಷ್ಠಾಪಿಸಿ-ದ-ನೆಂದು
ಪ್ರತಿಷ್ಠಾಪಿಸಿ-ದ-ನೆಂದೂ
ಪ್ರತಿಷ್ಠಾಪಿಸಿ-ದ-ರೆಂದು
ಪ್ರತಿಷ್ಠಾಪಿಸಿ-ದ-ರೆಂದೂ
ಪ್ರತಿಷ್ಠಾಪಿಸಿ-ದಳು
ಪ್ರತಿಷ್ಠೆ
ಪ್ರತಿಷ್ಠೆ-ಮಾಡಿ
ಪ್ರತಿಷ್ಠೆ-ಯನ್ನು
ಪ್ರತಿಷ್ಠೆ-ಯಾಗಿ
ಪ್ರತಿಷ್ಠೆ-ಯಾದ
ಪ್ರತಿಸ್ಪರ್ಧಿ-ಗಳಾಗಿದ್ದ
ಪ್ರತೀ-ಕ-ವಾಗಿ
ಪ್ರತೀತಿ
ಪ್ರತ್ಯಕ್ಷ
ಪ್ರತ್ಯಕ್ಷ-ದರ್ಶಿಯೊಬ್ಬ
ಪ್ರತ್ಯರ್ತ್ಥಿಕ್ಷಿತಿ-ಪಾಲ-ರತ್ನ-ಮಕುಟೀನೀ-ರಾಜಿ-ತಾಂಘ್ರಿಶ್ಚಿರಂ
ಪ್ರತ್ಯೇಕ
ಪ್ರತ್ಯೇಕ-ವಾಗಿ
ಪ್ರಥಮ
ಪ್ರದಾನ-ವಾಗಿದ್ದವು
ಪ್ರದೇಶ
ಪ್ರದೇಶಕ್ಕೆ
ಪ್ರದೇಶ-ಗಳ
ಪ್ರದೇಶ-ಗಳನ್ನು
ಪ್ರದೇಶ-ಗ-ಳನ್ನೂ
ಪ್ರದೇಶ-ಗಳಲ್ಲಿ
ಪ್ರದೇಶ-ಗಳಲ್ಲಿ-ರುವ
ಪ್ರದೇಶ-ಗ-ಳಿಗೆ
ಪ್ರದೇಶ-ಗಳು
ಪ್ರದೇಶ-ಗಳೂ
ಪ್ರದೇಶದ
ಪ್ರದೇಶ-ದಕ್ಕೆ
ಪ್ರದೇಶ-ದಲ್ಲಿ
ಪ್ರದೇಶ-ದಲ್ಲಿತ್ತು
ಪ್ರದೇಶ-ದಲ್ಲಿದ್ದ
ಪ್ರದೇಶ-ದಲ್ಲಿಯೇ
ಪ್ರದೇಶ-ದಲ್ಲಿ-ರುವ
ಪ್ರದೇಶ-ದಲ್ಲಿ-ರುವು-ದ-ರಿಂದ
ಪ್ರದೇಶ-ದ-ವರೇ
ಪ್ರದೇಶ-ವನ್ನು
ಪ್ರದೇಶ-ವಾ-ಗಿತ್ತು
ಪ್ರದೇಶ-ವಾ-ದರೂ
ಪ್ರದೇಶವು
ಪ್ರದೇಶವೂ
ಪ್ರದೇಶವೇ
ಪ್ರಧಾನ
ಪ್ರಧಾನ-ನಾಗಿ
ಪ್ರಧಾನ-ನಾಗಿ-ಮಂತ್ರಿ-ಯಾಗಿ
ಪ್ರಧಾನ-ನಾಗಿ-ರುತ್ತಾನೆ
ಪ್ರಧಾನ-ಪಾತ್ರ
ಪ್ರಧಾನ-ಮಂತ್ರಿ
ಪ್ರಧಾನರು
ಪ್ರಧಾನರೇ
ಪ್ರಧಾನ-ವಾಗಿ
ಪ್ರಧಾನಿ
ಪ್ರಧಾನಿ-ಗಳಾಗ-ಲಿಲ್ಲ
ಪ್ರಧಾನಿ-ಯಾಗಿದ್ದ
ಪ್ರಧಾನಿ-ಯಾಗಿದ್ದರೂ
ಪ್ರಬಂಧ
ಪ್ರಬಂಧ-ಗಳನ್ನು
ಪ್ರಬಂಧ-ಗಳಲ್ಲಿ
ಪ್ರಬಂಧ-ಗಳು
ಪ್ರಬಂಧ-ವಾಗಿದೆ
ಪ್ರಬಂಧವು
ಪ್ರಬಲ
ಪ್ರಬಲ-ಮ-ಭೂತ್ತುಷ್ಟಿಃ
ಪ್ರಬಲ-ವಾ-ಗಿತ್ತು
ಪ್ರಬಲ-ವಾದ
ಪ್ರಬಲಿ-ಸಲು
ಪ್ರಬಲಿಸಿ
ಪ್ರಭತ್ವ-ವನ್ನು
ಪ್ರಭಾ-ಚಂದ್ರ
ಪ್ರಭಾ-ಚಂದ್ರ-ಸಿದ್ಧಾಂತ
ಪ್ರಭಾ-ಚಂದ್ರ-ಸಿದ್ಧಾಂತ-ದೇವ-ರಿಗೆ
ಪ್ರಭಾವದ
ಪ್ರಭಾವ-ದಿಂದ
ಪ್ರಭಾವ-ನೆನಿಸಿ
ಪ್ರಭಾವಾ-ವತಾರಿತ
ಪ್ರಭಾವಿ
ಪ್ರಭು
ಪ್ರಭು-ಗಳ
ಪ್ರಭು-ಗಳಾಗಿ
ಪ್ರಭು-ಗಳಾಗಿದ್ದ
ಪ್ರಭು-ಗಳು
ಪ್ರಭು-ಗವುಡ-ಗಳು
ಪ್ರಭು-ಗಾವುಂಡ
ಪ್ರಭು-ಗಾವುಂಡ-ಗಳ
ಪ್ರಭು-ಗಾವುಂಡ-ಗಳು
ಪ್ರಭು-ಗಾವುಂಡ-ನೆಂದು
ಪ್ರಭು-ಗಾವುಂಡರ
ಪ್ರಭು-ಗಾವುಂಡ-ರಿಗೆ
ಪ್ರಭು-ಗಾವುಂಡ-ರಿದ್ದರು
ಪ್ರಭು-ಗಾವುಂಡರು
ಪ್ರಭು-ಗಾವುಂಡ-ರು-ಗಳಾಗಿದ್ದ-ರೆಂದು
ಪ್ರಭು-ಗಾವುಂಡ-ರುಪ್ರಜೆ-ಗಾವುಂಡರು
ಪ್ರಭು-ಗಾವುಂಡ-ರುಪ್ರಜೆ-ಗಾವುಂಡ-ರು-ಗಾವುಂಡರು
ಪ್ರಭು-ಗಾವುಂಡ-ರೆಂದು
ಪ್ರಭು-ಗಾವುಂಡುಗ-ಗಳು
ಪ್ರಭು-ಗಾವುಂಡು-ಗಳ
ಪ್ರಭು-ಗಾವುಂಡು-ಗಳು
ಪ್ರಭು-ಗಾವುಡು-ಗಳು
ಪ್ರಭು-ತನಕ್ಕೆ
ಪ್ರಭುತ್ವ
ಪ್ರಭುತ್ವ-ವನ್ನು
ಪ್ರಭುತ್ವ-ಸಹಿತ
ಪ್ರಭುತ್ವ-ಸಹಿತಂ
ಪ್ರಭುತ್ವ-ಸಹಿತ-ವಾಗಿ
ಪ್ರಭು-ಪೆರ್ಗ್ಗಡೆ
ಪ್ರಭು-ವರ್ಗದ-ವರ
ಪ್ರಭು-ವರ್ಗದ-ವ-ರಿದ್ದರು
ಪ್ರಭು-ವರ್ಗದ-ವರು
ಪ್ರಭು-ವರ್ಗದ-ವ-ರೆಂದು
ಪ್ರಭು-ವಾಗಿದ್ದ
ಪ್ರಭು-ವಾಗಿದ್ದ-ನೆಂದು
ಪ್ರಭು-ವಾಗಿದ್ದ-ವ-ನನ್ನು
ಪ್ರಭು-ವಾಗಿದ್ದು
ಪ್ರಭು-ವಾದ
ಪ್ರಭು-ಶಕ್ತಿ
ಪ್ರಭು-ಶಕ್ತಿ-ಯನಾಂತ
ಪ್ರಮುಖ
ಪ್ರಮುಖಃ
ಪ್ರಮುಖ-ನಾಗಿದ್ದ
ಪ್ರಮುಖ-ನೆಂದು
ಪ್ರಮುಖ-ಪಾತ್ರ
ಪ್ರಮುಖ-ಪಾತ್ರ-ವಹಿಸಿದ್ದ
ಪ್ರಮುಖ-ಪಾತ್ರ-ವಹಿಸಿದ್ದನು
ಪ್ರಮುಖ-ಮುಖ್ಯ
ಪ್ರಮುಖ-ಮುಖ್ಯರು
ಪ್ರಮುಖ-ರನ್ನು
ಪ್ರಮುಖ-ರಾಗಿದ್ದ-ರೆಂಬುದು
ಪ್ರಮುಖ-ರಾಗಿದ್ದು
ಪ್ರಮುಖ-ರಿಗೆ
ಪ್ರಮುಖರು
ಪ್ರಮುಖ-ವಾಗಿ
ಪ್ರಮುಖ-ವಾಗಿತ್ತೆಂದು
ಪ್ರಮುಖ-ವಾಗಿತ್ತೆಂಬುದು
ಪ್ರಮುಖ-ವಾಗಿವೆ
ಪ್ರಮುಖ-ವಾದ
ಪ್ರಮುಖ-ವಾದು-ವೆಂದರೆ
ಪ್ರಮುಖ-ವೆಂದೂ
ಪ್ರಯತ್ನ
ಪ್ರಯತ್ನ-ದಲ್ಲೂ
ಪ್ರಯತ್ನ-ವನ್ನು
ಪ್ರಯತ್ನವು
ಪ್ರಯತ್ನಿ-ಸಿದರು
ಪ್ರಯತ್ನಿ-ಸಿದಾಗ
ಪ್ರಯತ್ನಿಸಿರ-ಬಹುದು
ಪ್ರಯತ್ನಿಸಿರ-ಬಹು-ದೆಂದು
ಪ್ರಯತ್ನಿ-ಸುತ್ತಿದ್ದನು
ಪ್ರಯತ್ನಿ-ಸುತ್ತಿದ್ದರು
ಪ್ರಯಾಗ-ಪೆರುಮಾಳೆ
ಪ್ರಯಾಣ
ಪ್ರಯೋಕ್ತೃ-ಕುಶಲೋ
ಪ್ರಯೋಗ
ಪ್ರಯೋಗ-ಗಳನ್ನು
ಪ್ರಯೋಗ-ವಾಗಿದೆ
ಪ್ರಯೋಗ-ವಿದೆ
ಪ್ರಯೋಗ-ವಿದ್ದು
ಪ್ರಯೋಗಿ-ಸ-ಲಾಗಿದೆ
ಪ್ರವಚನ
ಪ್ರವರ್ತಿ-ಸುತ್ತಿದ್ದ-ನೆಂದು
ಪ್ರವರ್ಧ-ಮಾನಕ್ಕೆ
ಪ್ರವಾಸಿ
ಪ್ರವಾಸಿ-ಗ-ರಾದ
ಪ್ರವಿಷ್ಠದ
ಪ್ರವೀಣ-ನಾಗಿದ್ದ-ನೆಂದು
ಪ್ರವೀಣ-ನಾದ
ಪ್ರವುಡ-ದೇವ-ರಾಯ
ಪ್ರವುಢಪ್ರತಾಪ
ಪ್ರವೇಶ
ಪ್ರಶಸ್ತಿ
ಪ್ರಶಸ್ತಿಯ
ಪ್ರಶಸ್ತಿ-ಯನ್ನು
ಪ್ರಶಸ್ತಿ-ಸಹಿತ
ಪ್ರಶಿದ್ಧಃ
ಪ್ರಶ್ನೆ
ಪ್ರಸಕ್ತ
ಪ್ರಸನ್ನ
ಪ್ರಸನ್ನ-ಕೇಶವ-ಪುರ
ಪ್ರಸನ್ನ-ಮಾಧವ-ಪುರ-ಸತ್ಯಾಗಾಲ
ಪ್ರಸಸ್ತಿ-ಯನ್ನು
ಪ್ರಸಾದ-ವನ್ನು
ಪ್ರಸಾದಿತ
ಪ್ರಸಿದ್ಧ
ಪ್ರಸಿದ್ಧಃ
ಪ್ರಸಿದ್ಧ-ನಾಗಿದ್ದ
ಪ್ರಸಿದ್ಧ-ನಾಗಿದ್ದನು
ಪ್ರಸಿದ್ಧ-ನಾದ
ಪ್ರಸಿದ್ಧ-ನಾ-ದನು
ಪ್ರಸಿದ್ಧನು
ಪ್ರಸಿದ್ಧ-ವಾ-ಗಿತ್ತು
ಪ್ರಸಿದ್ಧ-ವಾಗಿತ್ತೆಂದು
ಪ್ರಸಿದ್ಧ-ವಾಗಿದೆ
ಪ್ರಸಿದ್ಧ-ವಾಗಿದ್ದು
ಪ್ರಸಿದ್ಧ-ವಾದ
ಪ್ರಸಿದ್ಧಿಗೆ
ಪ್ರಸಿದ್ಧಿ-ಯನ್ನು
ಪ್ರಸಿದ್ಧಿ-ಯಾ-ಗಿತ್ತು
ಪ್ರಸಿದ್ಧಿ-ಯಾಗಿದೆ
ಪ್ರಸ್ತಾಪ
ಪ್ರಸ್ತಾಪ-ವನ್ನು
ಪ್ರಸ್ತಾಪ-ವಾಗಿದೆ
ಪ್ರಸ್ತಾಪ-ವಾಗಿದ್ದು
ಪ್ರಸ್ತಾಪ-ವಿದೆ
ಪ್ರಸ್ತಾಪ-ವಿದ್ದು
ಪ್ರಸ್ತಾಪ-ವಿ-ರುವ
ಪ್ರಸ್ತಾಪ-ವಿಲ್ಲ
ಪ್ರಸ್ತಾಪಿತ-ವಾಗಿ-ರುವ
ಪ್ರಸ್ತಾಪಿ-ಸ-ಲಾಗಿದೆ
ಪ್ರಸ್ತಾಪಿ-ಸಿದೆ
ಪ್ರಸ್ತಾಪಿಸಿ-ರುವ
ಪ್ರಸ್ತಾಪಿ-ಸುತ್ತದೆ
ಪ್ರಸ್ತಾಪಿ-ಸುವ
ಪ್ರಸ್ತಾವ
ಪ್ರಸ್ತಾವವೂ
ಪ್ರಸ್ತುತ
ಪ್ರಹುಡ-ದೇವ-ರಾಯ
ಪ್ರಾಂತ-ಗಳ
ಪ್ರಾಂತ-ಗಳನ್ನು
ಪ್ರಾಂತ-ಗಳಾಗಿ
ಪ್ರಾಂತ-ಗ-ಳಿಗೆ
ಪ್ರಾಂತ-ಗಳು
ಪ್ರಾಂತದ
ಪ್ರಾಂತ-ದಿಂದ
ಪ್ರಾಂತ-ವನ್ನು
ಪ್ರಾಂತಾಧಿ-ಕಾರಿ-ಯಾಗಿ
ಪ್ರಾಂತೀಯ
ಪ್ರಾಂತ್ಯ
ಪ್ರಾಂತ್ಯಕ್ಕೆ
ಪ್ರಾಂತ್ಯ-ಗಳ
ಪ್ರಾಂತ್ಯ-ಗಳಲ್ಲಿ
ಪ್ರಾಂತ್ಯದ
ಪ್ರಾಂತ್ಯ-ದಲ್ಲಿ
ಪ್ರಾಂತ್ಯ-ದಿಂದ
ಪ್ರಾಂತ್ಯ-ವನ್ನಾಗಿ
ಪ್ರಾಕೃತಿ
ಪ್ರಾಗಿ-ತಿ-ಹಾಸ
ಪ್ರಾಗೈತಿ-ಹಾಸಿಕ
ಪ್ರಾಚೀನ
ಪ್ರಾಚೀನತೆ
ಪ್ರಾಚೀನ-ತೆ-ಯನ್ನು
ಪ್ರಾಚೀನ-ವಾದ
ಪ್ರಾಚ್ಯವಸ್ತು
ಪ್ರಾಣತ್ಯಾಗ
ಪ್ರಾಣ-ದೇವರ
ಪ್ರಾಣಾಧಿ-ಕಾರಿ-ಗಳೂ
ಪ್ರಾಣಾರ್ಪಣೆ
ಪ್ರಾಣಿ-ಗಳಿಗೂ
ಪ್ರಾಣಿ-ಗಳು
ಪ್ರಾತಿನಿಧ್ಯ
ಪ್ರಾದುದಭೂದ್ಗುಣಾಢ್ಯೋ-ನಾಮ್ನಾ
ಪ್ರಾದೇಶಿಕ
ಪ್ರಾಪ್ತ-ನಾಗುತ್ತಾನೆ
ಪ್ರಾಪ್ತ-ನಾದ-ನೆಂದು
ಪ್ರಾಪ್ತ-ನಾ-ದಾಗ
ಪ್ರಾಪ್ತ-ವಾಗಿ-ರು-ವು-ದನ್ನು
ಪ್ರಾಪ್ತ-ವಾದ
ಪ್ರಾಪ್ತ-ವಾಯಿತೋ
ಪ್ರಾಪ್ತೈಃ
ಪ್ರಾಬಲ್ಯ
ಪ್ರಾಭವ
ಪ್ರಾಮುಖ್ಯ-ವಾಗಿ
ಪ್ರಾಯಶಃ
ಪ್ರಾಯಶ್ಚಿತ್ತ
ಪ್ರಾಯಶ್ಚಿತ್ತ-ವನ್ನು
ಪ್ರಾರಂಭದ
ಪ್ರಾರಂಭ-ವಾಗಿ-ರುವ
ಪ್ರಾರಂಭ-ವಾಗಿ-ರುವುದು
ಪ್ರಾರಂಭ-ವಾಗುತ್ತದೆ
ಪ್ರಾರಂಭ-ವಾ-ಯಿತೆಂದು
ಪ್ರಾರಂಭಿ-ಸಿದಾಗ
ಪ್ರಾರ್ಥಿ-ಸಲು
ಪ್ರಾರ್ಥಿಸಿ
ಪ್ರಿಥುವೀ
ಪ್ರಿಥ್ವೀ
ಪ್ರಿಯ-ತನ-ಯರು
ಪ್ರಿಯ-ತನ-ಯರೂ
ಪ್ರಿಯನೂ
ಪ್ರಿಯ-ಪುತ್ರಂ
ಪ್ರಿಯ-ರಾಗಿ
ಪ್ರಿಯ-ವಾದ
ಪ್ರಿಯ-ಸುತ
ಪ್ರಿಯ-ಸುತ-ನೆಂದು
ಪ್ರಿಯ-ಸೇವಕ-ನಾದ
ಪ್ರೀಣನ
ಪ್ರೀತಿಯ
ಪ್ರೀತಿ-ಯಿಂದ
ಪ್ರೀತ್ಯರ್ಥ-ವಾಗಿ
ಪ್ರುಥ್ವೀ
ಪ್ರುಥ್ವೀ-ರಾಜ್ಯಂಗೆಯುತ್ತಿ-ರಲು
ಪ್ರುಥ್ವೀ-ವಲ್ಲಭ
ಪ್ರೇಮಂ
ಪ್ರೇಮಾಲಯ-ಸುತ
ಪ್ರೇರೇಪಿಸ-ದ-ವನು
ಪ್ರೇರೇಪಿಸಿ-ದ-ವನು
ಪ್ರೊ
ಪ್ರೊಃ
ಪ್ರೌಢ-ದೇವ-ರಾಯ
ಪ್ರೌಢ-ದೇವ-ರಾಯನ
ಪ್ರೌಢ-ದೇವ-ರಾಯನು
ಪ್ರೌಢ-ದೇವ-ರಾಯ-ಮಲ್ಲಿ-ಕಾರ್ಜುನ-ನ-ನಿಗೆ
ಪ್ರೌಢಪ್ರತಾಪ
ಪ್ರೌಢಪ್ರಧಾನ
ಪ್ರೌಢರೇಖಾ
ಪ್ಲವ
ಫರಿಷ್ತಾನು
ಫರ್ಲಾಂಗ್
ಫಲ
ಫಲಂ
ಫಲಪ್ರದಂ
ಫಲಪ್ರದ-ವಾಗಿ
ಫಲ-ವತ್ತತೆ-ಯನ್ನೂ
ಫಲ-ವತ್ತಾದ
ಫಲ-ವತ್ತಾ-ದುದು
ಫಲಾಕೃತೇಃ
ಫಲಾತಿಶಯಃ
ಫಸ-ಲಿನ
ಫಾತಿಮಾ
ಫಾದರ್
ಫಾದರ್ಹೆರಾಸ್
ಫಿಲಿಯೋಜಾ
ಫೆಬ್ರ-ವರಿ
ಫೌಜ್ದಾರಿಯ
ಫೌಜ್ದಾರ್
ಫ್ರಾನ್ಸಿಸ್
ಫ್ರೆಂಚರ
ಫ್ರೆಂಚರು
ಫ್ರೆಂಚ್ರಾಕ್ಸ್
ಫ್ರೆಂಚ್ರಾಕ್ಸ್ನಲ್ಲಿ
ಫ್ರೆಂಚ್ರಾಕ್ಸ್ನಲ್ಲಿದ್ದ
ಫ್ರೆಂಚ್ರಾಕ್ಸ್ಹಿರೋಡೆ
ಫ್ಲೀಟ್
ಬ
ಬಂಕನ-ಹಳ್ಳಿ
ಬಂಕಾ-ಪುರಕ್ಕೆ
ಬಂಕಾ-ಪುರದ
ಬಂಕಾ-ಪುರ-ದಲ್ಲಿ
ಬಂಕಾ-ಪುರ-ದಲ್ಲಿದ್ದ
ಬಂಕಾ-ಪುರ-ದಿಂದ
ಬಂಕಾ-ಪುರವೋ
ಬಂಕಿ-ನಾಡ
ಬಂಕಿ-ನಾಡನ್ನು
ಬಂಕಿ-ನಾಡು
ಬಂಕೆಯನ
ಬಂಕೆಯ-ನನ್ನು
ಬಂಕೆಯ-ನಿಗೆ
ಬಂಕೆಯನು
ಬಂಕೆಯುನು
ಬಂಕೇಶನ
ಬಂಕೇಶನು
ಬಂಕೇಶನೇ
ಬಂಗಲಿ
ಬಂಗಾ-ರದ
ಬಂಙ್ಕೆ-ಯನು
ಬಂಟ
ಬಂಟ-ರ-ಬಾ-ವನುಂ
ಬಂಟ-ರಭಾವ
ಬಂಟ-ರಭಾವನುಂ
ಬಂಡಮಾ-ರನ-ಹಳ್ಳಿ
ಬಂಡ-ವಾಳ
ಬಂಡಾಯ
ಬಂಡಿ
ಬಂಡಿ-ಹೊಳೆ
ಬಂಡಿ-ಹೊಳೆ-ಮಡು-ವಿನ-ಕೋಡಿ
ಬಂಡಿ-ಹೊಳೆಯ
ಬಂಡೂರು
ಬಂಡೆ-ಗಳಿಂದ
ಬಂಡೆದ್ದಿದ್ದ
ಬಂಡೆದ್ದು
ಬಂಡೆಯ
ಬಂಣಗಟ್ಟ
ಬಂಣಗಟ್ಟಿ
ಬಂತು
ಬಂದ
ಬಂದ-ಣಿಕೆ
ಬಂದದ್ದು
ಬಂದ-ನಂತರ
ಬಂದ-ನಂತರವೂ
ಬಂದನು
ಬಂದ-ನೆಂದು
ಬಂದ-ನೆಂಬುದು
ಬಂದರು
ಬಂದರೂ
ಬಂದಲ್ಲಿ
ಬಂದ-ವ-ರಾಗಿದ್ದು
ಬಂದ-ವರು
ಬಂದ-ವ-ರೆಂದು
ಬಂದ-ವರೇ
ಬಂದವು
ಬಂದಾಗ
ಬಂದಿತಂತೆ
ಬಂದಿತು
ಬಂದಿತೆಂದು
ಬಂದಿ-ತೆಂದೂ
ಬಂದಿತ್ತು
ಬಂದಿತ್ತೆಂದು
ಬಂದಿತ್ತೆಂಬುದು
ಬಂದಿದೆ
ಬಂದಿದ್ದ
ಬಂದಿದ್ದನು
ಬಂದಿದ್ದ-ನೆಂದು
ಬಂದಿದ್ದರೂ
ಬಂದಿದ್ದಾಗ
ಬಂದಿದ್ದಾರೆ
ಬಂದಿದ್ದು
ಬಂದಿರ-ಬಹುದು
ಬಂದಿ-ರುವ
ಬಂದಿ-ರು-ವುದು
ಬಂದು
ಬಂದು-ದನ್ನು
ಬಂದುದು
ಬಂದೊಡನೆ
ಬಂಧ-ನ-ದಲ್ಲಿ-ರಿಸಿದ್ದ-ನಷ್ಟೆ
ಬಂಧಿ-ಯಾಗಿದ್ದ
ಬಂಧಿ-ಯಾಗಿದ್ದಾಗ
ಬಂಧಿಸಿ
ಬಂಧಿಸಿ-ದಾಗ
ಬಂಧು-ಗಳು
ಬಂಧು-ಜನಂಗಳು
ಬಂಧು-ಜನ-ಧ-ವಳ
ಬಂಧುಬಾಂಧ-ವರು
ಬಂನಿಯೂರ
ಬಂನೂರು
ಬಂಮಚ
ಬಕರಿಪು
ಬಕಾಡೇ-ಹಳ್ಳಿ
ಬಗೆಗಿನ
ಬಗೆಗೆ
ಬಗೆಯ
ಬಗೆ-ಯನ್ನು
ಬಗೆಯೊಡೆ
ಬಗೆ-ಹರಿ-ಸುತ್ತಾನೆ
ಬಗ್ಗ-ವಳ್ಳಿ
ಬಗ್ಗ-ವಳ್ಳಿ-ಯನ್ನು
ಬಗ್ಗು
ಬಗ್ಗೆ
ಬಗ್ಗೆಯೂ
ಬಟ್ಟ-ಲಿನ
ಬಟ್ಟಲು-ಗಳನ್ನು
ಬಡ-ಗರ-ನಾಡ
ಬಡ-ಗರ-ನಾಡು
ಬಡ-ಗರೆ
ಬಡ-ಗರೆ-ನಾಡ
ಬಡ-ಗರೆ-ನಾಡೊಳಗಣ
ಬಡಗಲು
ಬಡಗುಂಡ
ಬಡಗುಂದ
ಬಡಗುಂದ-ನಾಡ
ಬಡಗುಂದ-ನಾಡನ್ನು
ಬಡಗುಡ-ನಾಡ
ಬಡಗುಡ-ನಾಡು
ಬಡಗು-ನಾಡು
ಬಡ-ಗೆರೆ
ಬಡ-ಗೆರೆ-ನಾಡ
ಬಡ-ಗೆರೆ-ನಾ-ಡಿನ-ವ-ರೊಡನೆ
ಬಡ-ಗೆರೆ-ನಾಡು
ಬಡ-ಗೆರೆ-ನಾಡೊಳಗಣ
ಬಡ-ವ-ನಾದೆ
ಬಡ-ವಾರ
ಬಡಿಕೋಲ
ಬಡಿಕೋಲ-ಭಟ್ಟ
ಬಡಿಯಬೇಕೆಂಬ
ಬಡು-ವಾರ
ಬಡ್ತಿ
ಬಡ್ತಿ-ಯನ್ನು
ಬಣಜಿಗ
ಬಣ್ಣಂಗಟ್ಟಿ-ಬನ್ನಂಗಾಡಿ
ಬಣ್ಣಂಗಟ್ಟಿಯು
ಬಣ್ಣಿಗದೆರೆ-ಹಳ್ಳಿ-ಯನ್ನು
ಬಣ್ಣಿ-ಸ-ಲಾಗಿದೆ
ಬಣ್ಣಿ-ಸಿದೆ
ಬಣ್ಣಿ-ಸುತ್ತದೆ
ಬಣ್ಣಿಸೆ
ಬಣ್ನಿಗ-ದರೆ-ಯನ್ನು
ಬಣ್ನಿಸಲ್ಬಿಂಡಿಗವಿ-ಲೆಯೊಳಾ
ಬತ್ತದ
ಬದನ-ಗುಪ್ಪೆ
ಬದ-ಲಾಗಿ
ಬದ-ಲಾಯಿತು
ಬದಲಾಯಿ-ಸ-ಲಾಯಿತು
ಬದಲಾಯಿಸಿ
ಬದಲಾಯಿಸಿ-ಕೊಂಡು
ಬದಲಾವಣೆ-ಗಳೊಂದಿಗೆ
ಬದಲಾವಣೆ-ಯಾದವು
ಬದ-ಲಿಗೆ
ಬದಲಿ-ಸುತ್ತಾನೆ
ಬದಲು
ಬದಿ-ಗಿರಿಸಿ
ಬದಿಗೊತ್ತಿ
ಬದಿಗೊತ್ತಿ-ದನು
ಬದಿಯ
ಬದುಕಿದ್ದ-ನೆಂದು
ಬದುಕಿದ್ದ-ರೆಂದು
ಬದುಕಿದ್ದಾಗಲೇ
ಬದುಕಿದ್ದಿರ-ಬಹುದು
ಬದುಕಿದ್ದು
ಬದುಕಿ-ರುವಾಗಲೇ
ಬದ್ದೆಗ
ಬದ್ದೆಗನ
ಬದ್ದೆ-ಗನು
ಬನದ
ಬನವಸೆ
ಬನವಸೆ-ಕಾರರ
ಬನವಸೆ-ಗಳನ್ನು
ಬನ-ವಾಸಿ
ಬನ-ವಾಸಿ-ಪಟ್ಟಣ-ದಲ್ಲಿ
ಬನ-ವಾಸಿ-ಯಲ್ಲಿ
ಬನ-ವಾಸಿ-ಯಿಂದ
ಬನ-ವಾಸೆ
ಬನ್ನಂಗಾಡಿ
ಬನ್ನಿಯೂರ
ಬನ್ನೂರು
ಬಪ್ಪ
ಬಪ್ಪಡೆ
ಬಪ್ಪ-ದೇವಿ-ಯ-ರನ್ನು
ಬಬಿ-ನಾಡಾಳ್ವರು
ಬಬೆಯ-ನಾಡಾಂಕಿ-ಯಾದೀ
ಬಬ್ಬ
ಬಬ್ಬೆಯ
ಬಬ್ಬೆಯ-ನಾಯ-ಕನ
ಬಬ್ಬೆಯ-ನಾಯ-ಕ-ನಿಗೆ
ಬಬ್ಬೆಯ-ನಾಯ-ಕನು
ಬಬ್ಬೆಯ-ನಾಯ-ಕ-ನೆಂಬ
ಬಭೈರಯ-ನಾಯಕ
ಬಮ್ಮ
ಬಮ್ಮ-ಗವುಡನ
ಬಮ್ಮಚ
ಬಮ್ಮ-ಚ-ನಧಿಕಬಳಂ
ಬಮ್ಮಣ
ಬಮ್ಮನು
ಬಮ್ಮ-ನೆಂಬು-ವ-ವನು
ಬಮ್ಮಲ
ಬಮ್ಮ-ಲ-ದೇವಿ
ಬಮ್ಮ-ಲ-ದೇವಿ-ಯನ್ನು
ಬಮ್ಮ-ಲ-ದೇವಿಯು
ಬಮ್ಮ-ಲೆಯ
ಬಮ್ಮವ್ವೆ
ಬಯಲ-ಮಾರ್ತಾಂಡ
ಬಯ-ಲಲ್ಲಿ
ಬಯಲ-ಹುಲಿ
ಬಯಲಿನ
ಬಯಲಿನ-ವರೆಗೂ
ಬಯಲ್ನಾಡ
ಬಯಲ್ನಾಡನಂ
ಬಯಲ್ನಾಡು
ಬಯ-ಸುತ್ತಾರೆ
ಬಯ-ಸುವ
ಬಯಿಚಕ್ಕ
ಬಯಿಚಣ್ಣ
ಬಯಿಚೆಯ
ಬಯಿರ
ಬಯಿರ-ರಸ
ಬಯಿರ-ರಾಜ
ಬಯಿ-ರೆಯ
ಬಯಿ-ರೆಯ-ದಂಡ-ನಾಯ-ಕನ
ಬಯ್ಯಪ್ಪ-ನಾಯ-ಕರ
ಬರಗಾಲ
ಬರಡು
ಬರಡು-ಭೂಮಿ-ಯಾ-ಗಿತ್ತು
ಬರಮಣ್ಣ
ಬರ-ಲಾಗಿದೆ
ಬರ-ಲಿಲ್ಲ
ಬರಹ
ಬರಹ-ಗಾರ-ರಾದ
ಬರಹದ
ಬರಹ-ವಿದೆ
ಬರಹ-ವಿಲ್ಲದ
ಬರೀದಸಪ್ತಾಂಗ-ಹರಣ
ಬರುತ್ತದೆ
ಬರುತ್ತ-ದೆಂದು
ಬರುತ್ತವೆ
ಬರುತ್ತಾನೆ
ಬರುತ್ತಾರೆ
ಬರುತ್ತಿದ್ದ
ಬರುತ್ತಿದ್ದ-ನೆಂಬುದು
ಬರುತ್ತಿದ್ದರು
ಬರುತ್ತಿದ್ದ-ರೆಂದೂ
ಬರುತ್ತಿದ್ದ-ರೆಂಬುದು
ಬರುತ್ತಿದ್ದ-ವೆಂದು
ಬರುತ್ತಿ-ರುವಾಗ
ಬರುತ್ತಿಲ್ಲ
ಬರುತ್ತೀನಿ
ಬರುತ್ತೆ
ಬರುವ
ಬರು-ವಂತೆ
ಬರುವ-ವ-ನನ್ನು
ಬರುವಾಗ
ಬರುವುದ-ರಿಂದ
ಬರು-ವು-ದಿಲ್ಲ
ಬರುವುದೇ
ಬರೂಲ್
ಬರೆದ
ಬರೆದ-ನೆಂದೂ
ಬರೆ-ದಿದೆ
ಬರೆ-ದಿದ್ದ
ಬರೆ-ದಿದ್ದಾನೆ
ಬರೆದಿರುತ್ತಾನೆ
ಬರೆದಿ-ರುತ್ತಾರೆ
ಬರೆ-ದಿ-ರುವ
ಬರೆ-ದಿ-ರು-ವು-ದನ್ನು
ಬರೆದು
ಬರೆದು-ದಕ್ಕೆ
ಬರೆಯ-ಲಾಗಿದೆ
ಬರೆಯಿಸಿ-ಕೊಂಡಿದ್ದಾನೆ
ಬರೆಯಿ-ಸಿದ್ದು
ಬರೆಯುತ್ತಾನೆ
ಬರೆಯುತ್ತಿದ್ದರು
ಬರೆಯುವಾಗ
ಬರೆಯುವುದು
ಬರೆ-ಸಿದ-ನೆಂದು
ಬರೆ-ಸಿದರು
ಬರ್ತೀನಿ
ಬರ್ಮಯ್ಯ
ಬರ್ಮಯ್ಯ-ನನ್ನು
ಬರ್ಮಯ್ಯ-ನಾಯ-ಕನ
ಬರ್ಮ್ಮಯ್ಯ
ಬರ್ಮ್ಮಯ್ಯನು
ಬಲಂಬು-ತೀರ್ಥ-ದಲ್ಲಿ
ಬಲ-ಗಯ್ಯ
ಬಲಗೈ
ಬಲ-ಗೈ-ಬಲ-ಭಾಗ
ಬಲ-ಗೈಯ
ಬಲ-ಗೈ-ಯ-ಸೇನಾ-ಧಿ-ಪತಿ
ಬಲಙ್ಗ-ಳನಟ್ಟಿಮುಟ್ಟಿ
ಬಲದ
ಬಲ-ದ-ಕಯ್ಯ
ಬಲ-ದೇವಣ್ಣ
ಬಲ-ದೇವನು
ಬಲ-ಪಡಿ-ಸುವ
ಬಲ-ಭಾಗದ
ಬಲ-ಭಾಗ-ದಲ್ಲಿ-ರುವ
ಬಲ-ಮುರಿ
ಬಲ-ಮುರಿಯ
ಬಲ-ರನ್ನು
ಬಲ-ರಾಮ-ಕೃಷ್ಣ-ರಂತೆ
ಬಲ-ವಂಕ
ಬಲ-ವಂಕ-ದಲ್ಲಿ
ಬಲ-ವಂಕಪ್ಪ
ಬಲ-ವನ್ನು
ಬಲ-ಸಮುದ್ರ
ಬಲಿಗೆಯ್ದ-ರಂತೆ
ಬಲಿ-ದಾನ
ಬಲಿಪೀಠ
ಬಲಿಯ-ಕೆರೆ-ಯನ್ನು
ಬಲಿಷ್ಠ-ರಾದ
ಬಲೀಂದ್ರ
ಬಲು
ಬಲು-ಫೌಜು
ಬಲು-ಮನುಷ
ಬಲು-ಮನುಷ್ಯ
ಬಲೆ-ಯ-ನಾಯ-ಕರ
ಬಲ್ಲಪ
ಬಲ್ಲಪಂ
ಬಲ್ಲಪನು
ಬಲ್ಲಪನೇ
ಬಲ್ಲಪ್ಪ
ಬಲ್ಲಪ್ಪ-ದಂಡ-ನಾಯ-ಕನ
ಬಲ್ಲಪ್ಪನು
ಬಲ್ಲಪ್ಪ-ಬಿಲ್ಲಪ್ಪ
ಬಲ್ಲ-ಯನ
ಬಲ್ಲಯ್ಯ
ಬಲ್ಲಯ್ಯನ
ಬಲ್ಲಯ್ಯ-ನಯ್ಯನೀ
ಬಲ್ಲಯ್ಯನು
ಬಲ್ಲಯ್ಯನೇ
ಬಲ್ಲ-ವರು
ಬಲ್ಲಹಂ
ಬಲ್ಲ-ಹನ
ಬಲ್ಲ-ಹನು
ಬಲ್ಲಾನು
ಬಲ್ಲಾಳ
ಬಲ್ಲಾಳಈ
ಬಲ್ಲಾಳ-ದಾಸರ
ಬಲ್ಲಾಳ-ದೇವಂ
ಬಲ್ಲಾಳ-ದೇವನ
ಬಲ್ಲಾಳ-ದೇವ-ನತ್ಯಂತ-ವಾಗಿ
ಬಲ್ಲಾಳ-ದೇವ-ನಿಗೆ
ಬಲ್ಲಾಳ-ದೇವನು
ಬಲ್ಲಾಳ-ದೇವ-ನೊಡನೆ
ಬಲ್ಲಾಳ-ದೇವರ
ಬಲ್ಲಾಳ-ದೇವ-ರ-ಸನು
ಬಲ್ಲಾಳ-ದೇವ-ರ-ಸರ
ಬಲ್ಲಾಳ-ದೇವ-ರ-ಸರು
ಬಲ್ಲಾಳ-ದೇವರು
ಬಲ್ಲಾಳನ
ಬಲ್ಲಾಳ-ನ-ದಲ್ಲಿ
ಬಲ್ಲಾಳ-ನ-ಪುರ-ವಾದ
ಬಲ್ಲಾಳ-ನಲ್ಲಿ
ಬಲ್ಲಾಳ-ನಲ್ಲಿದ್ದರು
ಬಲ್ಲಾಳ-ನ-ವರೆ-ಗಿನ
ಬಲ್ಲಾಳ-ನಾಗುತ್ತಾನೆ
ಬಲ್ಲಾಳ-ನಿಗೆ
ಬಲ್ಲಾಳನು
ಬಲ್ಲಾಳನೂ
ಬಲ್ಲಾಳ-ನೆಂದು
ಬಲ್ಲಾಳನೇ
ಬಲ್ಲಾಳ-ಪುರ
ಬಲ್ಲಾಳ-ಪುರದ
ಬಲ್ಲಾಳ-ಪುರ-ದಲ್ಲಿ
ಬಲ್ಲಾಳ-ಪುರಸ್ಥಳ
ಬಲ್ಲಾಳ-ಭೂಪಾಳಂ
ಬಲ್ಲಾಳ-ಮಹೀಕಾಂತನ
ಬಲ್ಲಾಳ-ಮಹೀ-ಪಾಲ
ಬಲ್ಲಾಳ-ಮಹೀ-ಪಾಲಯಂ
ಬಲ್ಲಾಳ-ರಾಯನ
ಬಲ್ಲಾಳ-ರಾಯ-ನಿಗೆ
ಬಲ್ಲಾಳ-ರಾಯ್ಯ
ಬಲ್ಲಾಳು
ಬಲ್ಲಾ-ಳೇಶ್ವರ
ಬಲ್ಲೆ-ಕೆರೆ
ಬಲ್ಲೆಯ
ಬಲ್ಲೆಯ-ನಾಯಕ
ಬಲ್ಲೆಯ-ನಾಯ-ಕನ
ಬಲ್ಲೆಯ-ನಾಯ-ಕನು
ಬಲ್ಲೆಯ-ಬಲ್ಲಪ್ಪ
ಬಲ್ಲೇ-ಗೌಡ
ಬಲ್ಲೇನ-ಪಲ್ಲಿ
ಬಲ್ಲೇನ-ಹಳ್ಳಿ
ಬಲ್ಲೇನ-ಹಳ್ಳಿ-ಯಾಗಿದೆ
ಬಳಕೆ
ಬಳಕೆ-ಗೊಂಡಿವೆ
ಬಳ-ಕೆಯ
ಬಳಕೆ-ಯಾಗಿದೆ
ಬಳಕೆ-ಯಾಗಿ-ರು-ವುದು
ಬಳಕೆ-ಯಾಗಿವೆ
ಬಳಗಾರ
ಬಳಗುಂದಿಯ
ಬಳಗುಳ
ಬಳಗುಳಕ್ಕೆ
ಬಳಗುಳದ
ಬಳಗುಳ-ದಲ್ಲಿ
ಬಳಗುಳ-ವನ್ನು
ಬಳಗೊಳ
ಬಳಗೊಳ-ದಲ್ಲಿ
ಬಳಗೊಳವು
ಬಳ-ಘಟ್ಟ-ವಾಗಿದೆ
ಬಳ-ಪದ-ಕಲ್ಲು-ಮಂಟಿ
ಬಳ-ಮಡು-ನಾ-ಡಿನಲ್ಲಿತ್ತೆಂದು
ಬಳ-ಮಡು-ನಾಡು
ಬಳ-ಸ-ಲಾಗಿದೆ
ಬಳ-ಸ-ಲಾಗಿ-ದೆಯೇ
ಬಳಸಿ-ಕೊಂಡಿದ್ದಾರೆ
ಬಳಸಿ-ಕೊಳ್ಳ-ಲಾಗಿದೆ
ಬಳ-ಸಿದ್ದಾನೆ
ಬಳಸಿ-ರುವು-ದ-ರಿಂದ
ಬಳಸಿಲ್ಲ
ಬಳ-ಸುವ
ಬಳ-ಸುವ-ವ-ನನ್ನು
ಬಳಿ
ಬಳಿಕ
ಬಳಿ-ಕುಂದೂರು
ಬಳಿಗ
ಬಳಿ-ಗ-ಗಟ್ಟದ
ಬಳಿಯ
ಬಳಿಯೂ
ಬಳಿಯೆ
ಬಳಿಯೇ
ಬಳುವ-ಳಿಯಾಗಲ್ಲ
ಬಳು-ವಳಿ-ಯಾಗಿ
ಬಳ್ಳಾರಿ
ಬಳ್ಳಿ-ಯ-ಕೆರೆ
ಬಳ್ಳೆಗೊಳ
ಬಳ್ಳೆಗೊಳಕ್ಕೆ
ಬಳ್ಳೆಗೊಳದ
ಬಳ್ಳೇ-ಕೆರೆ
ಬವರ-ದಲ್ಲಿ
ಬಸದಿ
ಬಸದಿ-ಗಳ
ಬಸದಿ-ಗಳನ್ನು
ಬಸದಿ-ಗ-ಳಿಗೆ
ಬಸದಿ-ಗಳೂ
ಬಸದಿಗೆ
ಬಸದಿ-ಗೆ-ಶಾ-ಸನ-ಬ-ಸದಿ
ಬಸದಿಯ
ಬಸದಿ-ಯನ್ನು
ಬಸದಿ-ಯನ್ನು-ಎ-ರಡು-ಕಟ್ಟೆ
ಬಸದಿ-ಯಲ್ಲಿದೆ
ಬಸದಿ-ಯಲ್ಲಿರುವ
ಬಸದಿ-ಯಾಗಿತ್ತೆಂದು
ಬಸದಿ-ಯಾಗಿ-ರ-ಬಹುದು
ಬಸದಿಯು
ಬಸರಾ-ಳನ್ನು
ಬಸರಾ-ಳಿನ
ಬಸರಾ-ಳಿ-ನಲ್ಲಿ
ಬಸರಾಳು
ಬಸ-ರುವಾಣು
ಬಸವ
ಬಸವಂತ
ಬಸವಂತ-ಪಟ್ಟಣ-ವನ್ನು
ಬಸವಂತ-ಪಟ್ಟಣ-ವಾಗಿ-ರ-ಬಹುದು
ಬಸವ-ಗವುಡನ
ಬಸವಟ್ಟಿಗೆ
ಬಸವಣ್ಣನ-ವರು
ಬಸ-ವನ
ಬಸವ-ನ-ಕೋಟೆ
ಬಸವ-ನ-ಬೆಟ್ಟ
ಬಸವ-ನ-ಬೆಟ್ಟದ
ಬಸವ-ನ-ಹಳ್ಳಿ
ಬಸ-ವನು
ಬಸ-ವನೂ
ಬಸವ-ಪಟ್ಟಣದ
ಬಸ-ವಪ್ಪ
ಬಸವಪ್ಪ-ನಾಯ-ಕನು
ಬಸವ-ಮಾತ್ಯ
ಬಸವ-ಮಾತ್ಯನ
ಬಸವ-ಮಾತ್ಯನು
ಬಸವ-ಮಾತ್ಯ-ಬಸವ-ರಸ
ಬಸವಯ್ಯ
ಬಸವ-ರಸ
ಬಸವ-ರಸನ
ಬಸವ-ರಸನೂ
ಬಸವ-ರಾಜಯ್ಯ-ದೇವ
ಬಸವಾ-ಪಟ್ಟಣದ
ಬಸವಾ-ಪಟ್ಟಣ-ವನ್ನು
ಬಸವಾ-ಪುರ
ಬಸವಿ-ಯಕ್ಕ
ಬಸುರ-ಬಂದ
ಬಸುರಿ-ವಾಳ
ಬಸುರಿ-ವಾಳ-ದೊಳು
ಬಸುರಿ-ವಾಳು
ಬಸು-ರುವಾಣು
ಬಸು-ರುವಾಣುಸ್ಥಳದ
ಬಸ್ತಿ
ಬಸ್ತಿ-ಯಲ್ಲಿ
ಬಸ್ತಿ-ಯ-ಹಳ್ಳಿ
ಬಸ್ತಿ-ಹಳ್ಳಿ
ಬಸ್ತಿ-ಹಳ್ಳಿ-ಯಲ್ಲಿರುವ
ಬಸ್ತೀ-ಪುರ
ಬಹತ್ತರ
ಬಹದ್ದೂರ್
ಬಹಮನಿ
ಬಹಲ್ಲಿ
ಬಹಳ
ಬಹಳ-ವಾಗಿ
ಬಹಾದೂರ್
ಬಹಾದ್ದೂ-ರನು
ಬಹಾದ್ದೂರು
ಬಹಾದ್ದೂರ್
ಬಹಿತ್ರ
ಬಹಿತ್ರದ
ಬಹಿತ್ರ-ದಂತೆ
ಬಹಿತ್ರರು
ಬಹು
ಬಹು-ಕಾಲ
ಬಹು-ತೆಕ
ಬಹು-ತೇಕ
ಬಹುದು
ಬಹು-ದೊಡ್ಡ
ಬಹು-ಪಾಲು
ಬಹು-ಭಾಗ
ಬಹು-ಭಾಗ-ದಲ್ಲಿ
ಬಹು-ಮಟ್ಟಿಗೆ
ಬಹು-ರಾಜ್ಯ-ಕಾರ್ಯ್ಯಂ
ಬಹುಳ
ಬಹು-ವಾಗಿ
ಬಹುಶ
ಬಹುಶಃ
ಬಹು-ಸಂಖ್ಯೆ-ಯಲ್ಲಿ
ಬಹು-ಸಂಖ್ಯೆ-ಯಲ್ಲಿದ್ದು
ಬಾಂಧವ್ಯಕ್ಕೆ
ಬಾಂಧವ್ಯ-ವಿದ್ದಿತು
ಬಾಗಣಬ್ಬೆ
ಬಾಗಣಬ್ಬೆಯ
ಬಾಗಣಬ್ಬೆಯು
ಬಾಗ-ಸೆಟ್ಟಿ-ಹಳ್ಳಿ-ಗಳನ್ನು
ಬಾಗಿ-ನಾಡೆಪ್ಪತ್ತು
ಬಾಗಿಲಿಗೆ
ಬಾಗಿಲಿನ
ಬಾಗಿಲು-ವಾಡ-ವನ್ನು
ಬಾಗಿವಾ-ಳನ್ನು
ಬಾಗಿ-ವಾಳು
ಬಾಗೆ
ಬಾಗೆ-ನಾಡುನ್ನು
ಬಾಚ-ಗವುಡ
ಬಾಚ-ಜೀಯಂಗೆ
ಬಾಚನ-ಹಳ್ಳಿಯೇ
ಬಾಚ-ಪಟ್ಟಣದ
ಬಾಚಪ್ಪ
ಬಾಚಪ್ಪನ
ಬಾಚಪ್ಪ-ನ-ಕೆರೆ
ಬಾಚಪ್ಪ-ನನ್ನು
ಬಾಚಪ್ಪನೇ
ಬಾಚಹ-ಳಿಯ
ಬಾಚ-ಹಳ್ಳಿ
ಬಾಚ-ಹಳ್ಳಿಗೆ
ಬಾಚ-ಹಳ್ಳಿಯ
ಬಾಚ-ಹಳ್ಳಿ-ಯನ್ನು
ಬಾಚಿಗ
ಬಾಚಿ-ಪಟ್ಟಣ-ವನು
ಬಾಚಿ-ಪಟ್ಟಣ-ವಾಗಿ-ರ-ಬಹುದು
ಬಾಚಿ-ಯಪ್ಪ
ಬಾಚಿಯಪ್ಪನ
ಬಾಚಿ-ಯಪ್ಪ-ನ-ವಿಗೆ
ಬಾಚಿ-ಯಪ್ಪ-ನಿಗೆ
ಬಾಚಿಯಪ್ಪನು
ಬಾಚಿಯ-ಹಳ್ಳಿಯ
ಬಾಚಿ-ರಾಜನ
ಬಾಚಿ-ರಾಜನಂ
ಬಾಚಿ-ರಾಜಾಭಿ-ದಾನಃ
ಬಾಚಿ-ರಾಜಾಹ್ವಯ
ಬಾಚಿಹ-ಳಿಯ
ಬಾಚಿ-ಹಳ್ಳಿಯ
ಬಾಚಿ-ಹಳ್ಳಿ-ಯನು
ಬಾಚಿ-ಹಳ್ಳಿ-ಯನ್ನು
ಬಾಚೆಯ
ಬಾಚೆಯ-ಹಳ್ಳಿಯ
ಬಾಚೆ-ಹಳ್ಳಿಯ
ಬಾಚೆ-ಹಳ್ಳಿಯು
ಬಾಜ-ದನ-ಮಲಕ
ಬಾಜ-ದನ-ಮಲುಕ
ಬಾಣ
ಬಾಣ-ಕುಲ-ಕ-ಲಾಕಲಃ
ಬಾಣ-ಕುಲದ
ಬಾಣದ
ಬಾಣ-ದಿಂದ
ಬಾಣರ
ಬಾಣ-ರ-ಮೇಲೆ
ಬಾಣ-ರ-ಸರ
ಬಾಣರು
ಬಾಣ-ವಂಶದ
ಬಾಣ-ವಂಶೋದ್ಭವ
ಬಾಣ-ವಂಶೋದ್ಭವ-ನಾದ
ಬಾಣ-ವಾಡಿ-ಯಲ್ಲಿ
ಬಾಣ-ಸಂದಾ-ಪುರ-ವನ್ನು
ಬಾಣಾ-ವರ-ದಲ್ಲಿ
ಬಾದಶ-ಹನನ್ನಾಗಿ
ಬಾದಶ-ಹರು
ಬಾದ-ಶಹಾ
ಬಾದಶಾಹ
ಬಾದಾಮಿ
ಬಾಪ್ಪೆಂಬಿನಂ
ಬಾಬ
ಬಾಬ-ಚಾ-ಮುಂಡ-ರಾಯ
ಬಾಬ-ಚಾವುಂಡ-ರಾಯ
ಬಾಬ-ಸಿಂಘೂರ-ಹರು
ಬಾಬು-ಸೆಟ್ಟಿಯು
ಬಾಬೆಯ-ನಾಯಕ
ಬಾಮುಲ
ಬಾಯಾನು-ಮತ-ದಿಂದ
ಬಾಯಿ-ದೇವಿ
ಬಾಯೊಳಕ್ಕೆ
ಬಾಯ್ಬೆಣ್ಣೆಗೆ
ಬಾರಂದರ
ಬಾರಕೂರು
ಬಾರಾ
ಬಾರಾ-ಗೋ-ಪಾಲ್
ಬಾರಾ-ಗೋ-ಪಾಲ್ರ-ವರು
ಬಾರಿ
ಬಾರಿ-ಕರು
ಬಾಲಕ-ನಾಗಿ
ಬಾಲಕ-ನಾಗಿದ್ದು-ದ-ರಿಂದ
ಬಾಲ-ಚಂದ್ರ
ಬಾಲಶಿಕ್ಷೆ
ಬಾಲಶಿಕ್ಷೆ-ಯನ್ನೂ
ಬಾಲಾತಪಂ
ಬಾಲೂರು
ಬಾಲ್ಯವೆಲ್ಲ
ಬಾಳಗುಂಚಿ
ಬಾಳಿದ-ನೆಂದು
ಬಾಳಿ-ದ-ರೆಂದು
ಬಾಳೆ
ಬಾಳೆ-ಹಳ್ಳಿ
ಬಾಳ್ಗಚ್ಚನ್ನು
ಬಾಳ್ಗಚ್ಚಾಗಿ
ಬಾಳ್ಗಳ್ಚಾಗಿ
ಬಾವ
ಬಾವಿ-ಯನ್ನು
ಬಾವಿ-ಸೆಟ್ಟಿ
ಬಾಷೆಯ
ಬಾಸಣಯ್ಯಂ
ಬಾಸಣಯ್ಯನ
ಬಾಸಣಯ್ಯ-ನೆಂಬು-ವ-ವನು
ಬಾಸಿ-ಮಯ್ಯ-ನು-ಬಾ-ಸಣಯ್ಯ
ಬಾಸೆಗೆ
ಬಾಸೆಯ
ಬಾಸೆ-ಯನ್ನು
ಬಾಹತ್ತರ
ಬಾಹತ್ತರ-ನಿಯೋಗಾಧಿ-ಪತಿ-ಪದ-ವಿ-ಗಳನ್ನು
ಬಾಹುಃ
ಬಾಹುತು
ಬಾಹು-ಬಲಿ
ಬಾಹು-ಬಲಿ-ಗಳು
ಬಾಹು-ಬ-ಲಿಯ
ಬಾಹು-ಬಲಿ-ಯರ
ಬಾಹು-ಸೌರ್ಯ್ಯಂ
ಬಿಂಕಂ
ಬಿಂಡಿಗನವಿಲೆ
ಬಿಂಡಿಗನವಿ-ಲೆಯ
ಬಿಂಡಿಗನವಿಲೆ-ಯೊಳಗೆ
ಬಿಂಡೇನ-ಹಳ್ಳಿ
ಬಿಂದಾರ-ಪತಿ
ಬಿಂದಾರೊ-ಪತಿ
ಬಿಂನಹ
ಬಿಂಮ-ನಾಯಕ
ಬಿಎಲ್
ಬಿಎಲ್ರೈಸ್
ಬಿಎಲ್ರೈಸ್ರ-ವರು
ಬಿಎಲ್ರೈಸ್ರ-ವರೂ
ಬಿಎ-ಸಾಲೆ-ತೂರ್
ಬಿಕ-ಸಮುದ್ರದ
ಬಿಕೆಯ-ನಾಯಕ
ಬಿಕೆಯ-ನಾಯ-ಕನೂ
ಬಿಕೆಯ-ನಾಯ-ಕರು
ಬಿಕ್ಕ-ಸಂದ್ರ
ಬಿಕ್ಕ-ಸಮುದ್ರ
ಬಿಗಡಾಯಿಸಿ-ರ-ಲಿಲ್ಲ-ವೆಂದು
ಬಿಜಯಂಗೈದು
ಬಿಜಯಂಗೈಯುತ್ತಿರ್ದು
ಬಿಜಯ-ಮಾಡಿ
ಬಿಜಾ-ಪುರದ
ಬಿಜೆಯ-ಮಾಡಿದ್ದ
ಬಿಜ್ಜಲ-ದೇವಿ-ಯರ
ಬಿಜ್ಜಲಾ-ಪುರ-ಹಾನು-ಗಲ್ಲು
ಬಿಜ್ಜ-ಲೇಶ್ವರ-ಪುರ-ವಾದ
ಬಿಜ್ಜೈ-ಯನು
ಬಿಟ್ಟ
ಬಿಟ್ಟಂತೆ
ಬಿಟ್ಟ-ಗೊಂಡ-ನ-ಹಳ್ಳಿ-ಯನ್ನು
ಬಿಟ್ಟ-ದತ್ತಿ
ಬಿಟ್ಟದ್ದು
ಬಿಟ್ಟ-ನಾಯ-ಕ-ನ-ಹಳ್ಳಿ
ಬಿಟ್ಟನು
ಬಿಟ್ಟ-ನೆಂದು
ಬಿಟ್ಟ-ನೆಂದೂ
ಬಿಟ್ಟ-ನೆಂಬುದು
ಬಿಟ್ಟರು
ಬಿಟ್ಟರೆ
ಬಿಟ್ಟ-ರೆಂದು
ಬಿಟ್ಟ-ರೆಂದು-ಹೇಳಿದೆ
ಬಿಟ್ಟರ್
ಬಿಟ್ಟ-ಳೆಂದು
ಬಿಟ್ಟ-ಶಾ-ಸನ
ಬಿಟ್ಟಾಗ
ಬಿಟ್ಟಿ
ಬಿಟ್ಟಿ-ಗನು-ವಿಷ್ಣ
ಬಿಟ್ಟಿ-ಗಾವುಂಡ
ಬಿಟ್ಟಿ-ಗಾವುಡ
ಬಿಟ್ಟಿ-ಗಾವುಡನ
ಬಿಟ್ಟಿ-ಗಾವುಡನು
ಬಿಟ್ಟಿ-ದ-ದಾನೆ
ಬಿಟ್ಟಿ-ದೇವ
ಬಿಟ್ಟಿ-ದೇವ-ನನ್ನು
ಬಿಟ್ಟಿ-ದೇವ-ನಿಂದ
ಬಿಟ್ಟಿ-ದೇವನು
ಬಿಟ್ಟಿ-ದೇವ-ನೆಂದು
ಬಿಟ್ಟಿದ್ದ
ಬಿಟ್ಟಿದ್ದನು
ಬಿಟ್ಟಿದ್ದ-ನೆಂದು
ಬಿಟ್ಟಿದ್ದ-ನೆಂದೂ
ಬಿಟ್ಟಿದ್ದಾನೆ
ಬಿಟ್ಟಿದ್ದಾನೆಂದು
ಬಿಟ್ಟಿದ್ದಾರೆ
ಬಿಟ್ಟಿದ್ದು
ಬಿಟ್ಟಿದ್ದುದು
ಬಿಟ್ಟಿನು
ಬಿಟ್ಟಿ-ಮಯ್ಯ
ಬಿಟ್ಟಿ-ಮಯ್ಯ-ಗಳ
ಬಿಟ್ಟಿ-ಮಯ್ಯನ
ಬಿಟ್ಟಿ-ಮಯ್ಯನು
ಬಿಟ್ಟಿ-ಮಯ್ಯನೂ
ಬಿಟ್ಟಿ-ಮಯ್ಯರು
ಬಿಟ್ಟಿ-ಯಣ್ಣ
ಬಿಟ್ಟಿ-ಯಣ್ಣನ
ಬಿಟ್ಟಿ-ಯಣ್ಣ-ನೆಂದೂ
ಬಿಟ್ಟಿ-ಯ-ದೇವನು
ಬಿಟ್ಟಿಯು
ಬಿಟ್ಟಿ-ರ-ಬಹುದು
ಬಿಟ್ಟಿ-ರ-ಬಹು-ದೆಂದು
ಬಿಟ್ಟಿ-ರುತ್ತಾನೆ
ಬಿಟ್ಟಿ-ರುವ
ಬಿಟ್ಟಿ-ರು-ವಂತೆ
ಬಿಟ್ಟಿ-ರು-ವು-ದನ್ನು
ಬಿಟ್ಟಿ-ರುವು-ದ-ರಿಂದ
ಬಿಟ್ಟಿ-ರು-ವುದು
ಬಿಟ್ಟೀ-ದೇವ
ಬಿಟ್ಟೀ-ದೇವ-ನು-ಇಮ್ಮಡಿ-ಬಲ್ಲಾಳ
ಬಿಟ್ಟು
ಬಿಟ್ಟು-ಕೊಟ್ಟ
ಬಿಟ್ಟು-ಕೊಡಲು
ಬಿಟ್ಟು-ಹೋಗಿ-ರು-ವಂತೆ
ಬಿಟ್ಟು-ಹೋಗಿವೆ
ಬಿಡದೆ
ಬಿಡಲಾ-ಗಿತ್ತು
ಬಿಡ-ಲಾಗಿತ್ತೆಂದು
ಬಿಡ-ಲಾಗಿದೆ
ಬಿಡ-ಲಾಗಿದ್ೆ
ಬಿಡ-ಲಾಗುತ್ತಿತ್ತು
ಬಿಡಲಾಗುತ್ತಿದ್ದ
ಬಿಡ-ಲಾಯಿತು
ಬಿಡ-ಲಾ-ಯಿತೆಂದು
ಬಿಡಲಾಯಿತೆಂಬ
ಬಿಡಲು
ಬಿಡಿಸ-ಲಾಗಿ-ರುವ
ಬಿಡಿಸಿ-ಕೊಂಡು
ಬಿಡಿ-ಸಿದ
ಬಿಡಿ-ಸಿದನು
ಬಿಡಿ-ಸಿದ-ನೆಂದು
ಬಿಡಿ-ಸಿದರೂ
ಬಿಡಿ-ಸಿದ್ದಾ-ನೆಂದು
ಬಿಡಿಸಿರ-ಬಹುದು
ಬಿಡಿಸಿ-ರು-ವುದು
ಬಿಡಿ-ಸುತ್ತಾನೆ
ಬಿಡು-ಗಡೆ
ಬಿಡು-ತಾರೆ
ಬಿಡುತ್ತಾನೆ
ಬಿಡುತ್ತಾರೆ
ಬಿಡುತ್ತಾಳೆ
ಬಿಡುತ್ತಿದ್ದನು
ಬಿಡುತ್ತಿದ್ದರು
ಬಿಡುವ
ಬಿಡು-ವಾಗ
ಬಿಣ್ಣಮ್ಮನ
ಬಿತ್ತವಾಟ್ಟಕ್ಕೆ
ಬಿತ್ತುವಟ್ಟಂ
ಬಿತ್ತುವಟ್ಟ-ವನ್ನೂ
ಬಿತ್ತು-ವಟ್ಟ-ವಾಗಿ
ಬಿತ್ತುವಟ್ಟಾಗಿ
ಬಿತ್ತುವಾಟಕ್ಕೆ
ಬಿತ್ತುವಾಟನ್ನು
ಬಿದಿಗೆ
ಬಿದಿ-ಯರ
ಬಿದಿರ-ಕೋಟೆ
ಬಿದಿರು-ಕೋಟೆ-ಯನ್ನು
ಬಿದ್ದನು
ಬಿದ್ದ-ನೆಂದು
ಬಿದ್ದಿದ್ದ-ನೆಂದು
ಬಿದ್ದಿ-ರಲು
ಬಿದ್ದು
ಬಿದ್ದು-ಹೋಗಿ
ಬಿನುಗುದೆರೆ
ಬಿನ್ನಪಂ
ಬಿನ್ನಹ
ಬಿನ್ನಹ-ಮಾಡಿ
ಬಿನ್ನಹ-ಮಾಡಿ-ಕೊಂಡ
ಬಿಮಿ-ಸೆಟ್ಟಿ
ಬಿಯಳಮ್ಮ-ನಿಗೆ
ಬಿರಿದು
ಬಿರಿಯು-ವಂತೆ
ಬಿರಿಯೆ
ಬಿರುಕು
ಬಿರುದ
ಬಿರುದಂತೆಂಬರ
ಬಿರುದಂತೆಂಬರ-ಗಂಡ
ಬಿರು-ದನ್ನು
ಬಿರುದನ್ನು-ಹುದ್ದೆ-ಯನ್ನು
ಬಿರುದ-ರಗೋವ
ಬಿರುದಾಂಕಿತ
ಬಿರುದಾಂಕಿತ-ನಾದ
ಬಿರುದಾದ
ಬಿರುದಾ-ವಳಿ
ಬಿರುದಾ-ವಳಿ-ಗಳನ್ನು
ಬಿರುದಾ-ವಳಿ-ಯನ್ನು
ಬಿರುದಿತ್ತ-ದೆಂದು
ಬಿರು-ದಿತ್ತು
ಬಿರುದಿತ್ತೆಂದು
ಬಿರುದಿತ್ತೆಂದೂ
ಬಿರುದಿತ್ತೆಂಬುದು
ಬಿರುದಿದೆ
ಬಿರುದಿದ್ದುದು
ಬಿರು-ದಿದ್ದುನ್ನು
ಬಿರುದಿನ
ಬಿರುದಿ-ನಂತೆ
ಬಿರುದಿರ-ಬಹುದು
ಬಿರು-ದಿಲ್ಲ
ಬಿರುದು
ಬಿರುದು-ಗಳ
ಬಿರುದು-ಗಳನ್ನು
ಬಿರುದು-ಗಳನ್ನುದ್ಧರಿಸಿ
ಬಿರುದು-ಗ-ಳನ್ನೂ
ಬಿರುದು-ಗಳಲ್ಲಿ
ಬಿರುದು-ಗಳಾಗಿದ್ದು
ಬಿರುದು-ಗಳಾವುವೂ
ಬಿರುದು-ಗಳಿಂದ
ಬಿರುದು-ಗಳಿದ್ದವು
ಬಿರುದು-ಗಳಿದ್ದು
ಬಿರುದು-ಗಳಿವೆ
ಬಿರುದು-ಗಳು
ಬಿರುದು-ಗಳೂ
ಬಿರುದು-ಗಳೇ
ಬಿರುದು-ಬಾವ-ಲಿ-ಗಳನ್ನು
ಬಿರುದುಳ್ಳ
ಬಿರುದೂ
ಬಿರುದೆಂತೆಂಬರ
ಬಿರುದೆಂಬರ-ಗಂಡ
ಬಿರುದೇ
ಬಿರುದೈರ್ವಂದಿತತ್ಯಾ-ನಿತ್ಯ-ಮಭಿಷ್ಟತಃ
ಬಿಲ್ಲಂಗೆರೆಯ
ಬಿಲ್ಲ-ಗೊಂಡ-ನ-ಹಳ್ಳಿ
ಬಿಲ್ಲಪ್ಪ
ಬಿಲ್ಲಬೆಳಗುಂದ
ಬಿಲ್ಲ-ಮೂಲೂನೂರ್ಪ್ಪಬ್ಬರು
ಬಿಲ್ಲಯ್ಯ-ನಿಗೆ
ಬಿಲ್ಲ-ರಾಮ-ನ-ಹಳ್ಳಿ
ಬಿಲ್ಲ-ವರ
ಬಿಲ್ಲ-ವರಿ-ರ-ಬೇಕು
ಬಿಲ್ವಿದ್ಯೆ-ಯಲ್ಲಿ
ಬಿಲ್ಹ-ಣನು
ಬಿಳಿ-ಕಲ್ಲು-ಮಂಠಿ
ಬಿಳಿ-ಕೆರೆ-ಯನ್ನು
ಬಿಳಿಗೆರೆ
ಬಿಷ್ಣುನೃ-ಪತಿ
ಬಿಸಾಡಿ-ಕೊಪ್ಪಲು
ಬಿಸು-ಗೆಯ
ಬೀಚನ-ಹಳ್ಳಿ
ಬೀಚವ್ವೆ
ಬೀಚೆಯ
ಬೀಚೇನ-ಹಳ್ಳಿ
ಬೀಡಿನ
ಬೀಡಿ-ನಲ್ಲಿ
ಬೀಡಿನಲ್ಲಿದ್ದಾಗ
ಬೀಡಿನಿಂದ
ಬೀಡಿನೊಳಗೆ
ಬೀಡು
ಬೀಡು-ಬಿಟ್ಟನು
ಬೀಡು-ಬಿಟ್ಟಲ್ಲಿ
ಬೀಡು-ಬಿಟ್ಟಿದ್ದ
ಬೀಡು-ಬಿಟ್ಟಿದ್ದ-ನೆಂದು
ಬೀಡು-ಬಿಟ್ಟಿದ್ದಾಗ
ಬೀಡು-ಬಿಟ್ಟಿ-ರ-ಬಹುದು
ಬೀಡು-ಬಿಟ್ಟು
ಬೀಡು-ಬಿಡಲು
ಬೀದಿ-ಯಲ್ಲಿರುವ
ಬೀದಿಯು
ಬೀದಿಯೇ
ಬೀಮಣ್ಣನ
ಬೀರ
ಬೀರಂ
ಬೀರಂಮಲೆ-ಯಲ್ಲಿ
ಬೀರಕ್ಕ
ಬೀರಕ್ಕನ
ಬೀರ-ಗಲ್ಲನ್ನು
ಬೀರ-ಗವುಂಡನು
ಬೀರ-ಗಾವುಂಡನು
ಬೀರ-ನ-ಹಳ್ಳಿ
ಬೀರಯ್ಯ
ಬೀರಯ್ಯ-ನನ್ನು
ಬೀರಯ್ಯ-ನೆಂದು
ಬೀರರಂ
ಬೀರ-ಲಕ್ಷ್ಮಿ
ಬೀರ-ಸೆಟ್ಟಿ
ಬೀರಿ-ಸೆಟ್ಟಿ-ಹಳ್ಳಿ
ಬೀರುಗ
ಬೀರು-ಬಳ್ಳಿ
ಬೀರು-ಬಳ್ಳಿ-ಯನ್ನು
ಬೀರು-ವಳ್ಳಿ
ಬೀರೆಯ-ನಾಯಕ
ಬೀರೆಯ-ನಾಯ-ಕನ
ಬೀರೆಯ್ಯ
ಬೀಳಲು
ಬೀಳ-ವೃತ್ತಿ
ಬೀಳ-ವೃತ್ತಿ-ಯಿಂದ
ಬೀಸುವ
ಬೀೞಅ್ಗುಂಡಿಕ್ಕಿದ
ಬೀೞ-ವೃತ್ತಿ
ಬೀೞ-ವೃತ್ತಿ-ಯಂತಹದೇ
ಬೀೞ-ವೃತ್ತಿಯು
ಬೀೞಾನು-ವೃತ್ತಿ-ಯಿಂದ
ಬುಕಂಣ
ಬುಕ್ಕ
ಬುಕ್ಕಂಣ
ಬುಕ್ಕಣ್ಣ
ಬುಕ್ಕಣ್ಣನ
ಬುಕ್ಕಣ್ಣನು
ಬುಕ್ಕಣ್ಣ-ವೊಡೆಯರ
ಬುಕ್ಕನ
ಬುಕ್ಕ-ನನ್ನೇ
ಬುಕ್ಕನು
ಬುಕ್ಕ-ನೃ-ಪತಿ-ನೊಳಂದತಿಶ-ಯದಿಂ
ಬುಕ್ಕನೇ
ಬುಕ್ಕಮಾ
ಬುಕ್ಕ-ರಾಜ-ರಾಯಾಬಾಹೂತ
ಬುಕ್ಕ-ರಾಯ
ಬುಕ್ಕ-ರಾಯ-ತನೂಭವ
ಬುಕ್ಕ-ರಾಯನ
ಬುಕ್ಕ-ರಾಯ-ನನ್ನು
ಬುಕ್ಕ-ರಾಯ-ನಿಗೆ
ಬುಕ್ಕ-ರಾಯನು
ಬುಕ್ಕ-ರಾಯನೇ
ಬುಕ್ಕ-ರಾಯರು
ಬುಕ್ಕ-ರಾಯ-ಸಮುದ್ರ
ಬುಕ್ಕರು
ಬುಕ್ನಾನ್
ಬುಧೈ-ಕಕಲ್ಪಭೂ
ಬುರಾನುದ್ದೀನ್
ಬುಳ್ಳಪ್ಪ-ನಾಯ-ಕರ
ಬೂಕನ
ಬೂಕನ-ಕೆರೆ
ಬೂಕಿನ
ಬೂಕಿನ-ಕೆರೆ
ಬೂಕಿನ-ಕೆರೆಯು
ಬೂಚಲೆ
ಬೂಚಿ-ಯಣ್ಣ
ಬೂಚಿ-ರಾಜ
ಬೂಚಿ-ರಾಜನು
ಬೂತ-ಗನು
ಬೂತ-ರ-ಸರು
ಬೂತಾರ್ಯನು
ಬೂತುಗ
ಬೂತುಗನ
ಬೂತುಗ-ನನ್ನು
ಬೂತುಗ-ನ-ರಸಿ
ಬೂತುಗ-ನಿಗೆ
ಬೂತುಗನು
ಬೂತುಗನೇ
ಬೂತುಗ-ನೊಂದಿಗೆ
ಬೂತುಗರ
ಬೂತುಗ-ಸತ್ಯ-ವಾಕ್ಯ
ಬೂದನೂರ
ಬೂದನೂ-ರಾದ
ಬೂವನ-ಹಳ್ಳಿ
ಬೂವನ-ಹಳ್ಳಿ-ಗಳ
ಬೂವನ-ಹಳ್ಳಿ-ಯನ್ನು
ಬೃಹತ್
ಬೆಂಕಿ
ಬೆಂಕಿ-ನವಾಬ-ನೆಂದು
ಬೆಂಕೊಂಡು
ಬೆಂಗಳೂ-ರನ್ನು
ಬೆಂಗಳೂ-ರಿನ
ಬೆಂಗಳೂರು
ಬೆಂಗಿ
ಬೆಂನಟ್ಟಿದಂ
ಬೆಂನೂರ
ಬೆಂನ್ನಂ
ಬೆಂಬಲ
ಬೆಂಬಲಕ್ಕೆ
ಬೆಂಬೆತ್ತಿ
ಬೆಂಬೆತ್ತಿ-ಹೋದನು
ಬೆಕ್ಕದ
ಬೆಗೆ-ಗವುಡ-ನೆಂದು
ಬೆಗೆವಂದಕ್ಕೆ
ಬೆಗೆ-ವಡೆದ
ಬೆಗೆ-ವನ್ದ
ಬೆಟಾ-ಲಿಯನ್ಗೆ
ಬೆಟ್ಟ
ಬೆಟ್ಟಕ್ಕೆ
ಬೆಟ್ಟ-ಗಳ
ಬೆಟ್ಟ-ಗಳಅ
ಬೆಟ್ಟ-ಗ-ಳಿಗೆ
ಬೆಟ್ಟ-ಗಳಿವೆ
ಬೆಟ್ಟ-ಗಳು
ಬೆಟ್ಟ-ಗುಡ್ಡ-ಗಳೂ
ಬೆಟ್ಟದ
ಬೆಟ್ಟ-ದ-ಕೋಟೆ
ಬೆಟ್ಟ-ದ-ಚಾಮ-ರಾಜ
ಬೆಟ್ಟ-ದ-ಚಾಮ-ರಾಜನು
ಬೆಟ್ಟ-ದ-ಪುರ
ಬೆಟ್ಟ-ದ-ಮೇಲೆ
ಬೆಟ್ಟ-ದಲ್ಲಿ
ಬೆಟ್ಟ-ದಲ್ಲಿ-ರುವ
ಬೆಟ್ಟ-ದ-ಹಳ್ಳಿ
ಬೆಟ್ಟ-ದ-ಹಳ್ಳಿ-ಯಿಂದ
ಬೆಟ್ಟ-ದಿಂದ
ಬೆಟ್ಟಮ್ಮೇಲ್ಕಾಲಂ
ಬೆಟ್ಟಯ್ಯ
ಬೆಟ್ಟ-ಹಳ್ಳಿ
ಬೆಟ್ಟ-ಹಳ್ಳಿ-ಯನ್ನು
ಬೆಣಚು-ಕಲ್ಲಿನ
ಬೆಣ್ಣೆ-ಗೆರೆಯ
ಬೆಣ್ಣೆದೊ-ಣೆ-ಯಲ್ಲಿ
ಬೆಣ್ಣೆ-ಸಿದ್ದ-ನ-ಗುಡ್ಡದ
ಬೆದರಿಸಿ
ಬೆದರೆ
ಬೆದ್ದ-ಲನ್ನು
ಬೆದ್ದಲು
ಬೆದ್ದ-ಲು-ಗಳನ್ನು
ಬೆದ್ದ-ಲೆ-ಯನ್ನು
ಬೆನ-ಕನ-ಕೆರೆ
ಬೆನ್ನಂ
ಬೆನ್ನಚರ್ಮವೇ
ಬೆನ್ನಟ್ಟಿ
ಬೆನ್ನ-ಹಿಂದೆಯೇ
ಬೆನ್ನಾ-ವರದ
ಬೆನ್ನು-ಹತ್ತಿ
ಬೆಮತೂರ-ಕಲ್ಲ
ಬೆಮ್ಬಮ್ಪಾಳ್
ಬೆರ್ರಡಿಯಾನ್
ಬೆಲತೂರು
ಬೆಲತ್ತೂರು
ಬೆಲದ-ತಾಲ
ಬೆಲ-ವತ್ತ
ಬೆಲಹುರ-ಬೇ-ಲೂರು
ಬೆಲಹೂರ-ಬೇ-ಲೂರು-ಅಧಿ-ಕಾರಿಯು
ಬೆಲುಹೂರಲಿ
ಬೆಲೂರಿನ
ಬೆಲೂರು-ಬೆಳ್ಳೂರು
ಬೆಲೆ-ಕೆರೆ
ಬೆಲ್ಲೂರು
ಬೆಳ-ಕನ್ನು
ಬೆಳಕ-ವಾಡಿ
ಬೆಳಕ-ವಾಡಿಗೆ
ಬೆಳಕ-ವಾಡಿಯ
ಬೆಳಕ-ವಾಡಿ-ಯನ್ನು
ಬೆಳಕ-ವಾಡಿಯು
ಬೆಳಕು
ಬೆಳಗಿ-ಸಿದಳು
ಬೆಳಗುಳದ
ಬೆಳಗೊಳ
ಬೆಳಗೊಳದ
ಬೆಳಗ್ಗೆ
ಬೆಳತೂರ
ಬೆಳತೂರಿನ
ಬೆಳತೂರು
ಬೆಳವಡಿ-ಯಲಿ
ಬೆಳವಡಿ-ಯಲ್ಲಿ
ಬೆಳವಡಿ-ಯಿಂದ
ಬೆಳವಣಿ-ಗೆಯ
ಬೆಳ-ವಾಡಿ
ಬೆಳ-ವಾಡಿ-ಯಲ್ಲಿ
ಬೆಳುಗಲಿ-ಯಲ್ಲಿದ್ದ
ಬೆಳುವೊ-ಲದ
ಬೆಳೆ
ಬೆಳೆಯ
ಬೆಳೆ-ಯನ
ಬೆಳೆ-ಯುವ
ಬೆಳೆಸಿ
ಬೆಳೆ-ಸಿ-ಕೊಳ್ಳುತ್ತಾ
ಬೆಳೆ-ಸಿದ
ಬೆಳೆ-ಸಿ-ದಳು
ಬೆಳೆ-ಸಿದ್ದ-ನಷ್ಟೆ
ಬೆಳೆ-ಸಿದ್ದನೋ
ಬೆಳೆ-ಸಿದ್ದವು
ಬೆಳ್ಕೆರೆ
ಬೆಳ್ಗುಪ್ಪ
ಬೆಳ್ಗೊಳ
ಬೆಳ್ಗೊಳದ
ಬೆಳ್ಗೊಳ-ದಲ್ಲಿ
ಬೆಳ್ಗೊಳ-ದೊಳ್ಜನ-ಮೆಲ್ಲಂ
ಬೆಳ್ಗೊಳ-ವಾಗುತ್ತದೆ
ಬೆಳ್ಳಾಲೆ
ಬೆಳ್ಳಿ
ಬೆಳ್ಳಿ-ಕೊಡ-ವನ್ನೂ
ಬೆಳ್ಳಿ-ಬಟ್ಟಲು-ಗಳ
ಬೆಳ್ಳಿ-ಬೆಟ್ಟದ
ಬೆಳ್ಳಿ-ಮಾಣಿಯ
ಬೆಳ್ಳಿ-ಮುಲಾಮಿನ
ಬೆಳ್ಳಿಯ
ಬೆಳ್ಳೂರ
ಬೆಳ್ಳೂ-ರನ್ನು
ಬೆಳ್ಳೂ-ರಿಗೆ
ಬೆಳ್ಳೂರಿನ
ಬೆಳ್ಳೂರಿ-ನಲ್ಲಿ
ಬೆಳ್ಳೂರು
ಬೆಳ್ವಲ
ಬೆಳ್ವೊಲ
ಬೆಳ್ವೊ-ಲದ
ಬೆಳ್ವೊಲ-ನಾ-ಡಿನ
ಬೆವ-ಹರ-ಪೂಜಾ-ಕೈಂಕರ್ಯ-ಗ-ಳಿಗೆ
ಬೆಸ-ಗರ-ಹಳ್ಳಿ
ಬೆಸ-ಗರ-ಹಳ್ಳಿ-ಯನ್ನು
ಬೆಸಟೆಯ
ಬೆಸಣಿಪೆಸಾಣಿ
ಬೆಸ-ದಿಂದ
ಬೆಸ-ದೊಳು
ಬೆಸದೊಳೆ
ಬೆಸನಂ
ಬೆಸ-ನನ್ನು
ಬೆಸ-ಸಲು
ಬೆಸಸಿ
ಬೆಸಸಿ-ದ-ನೆಂದು
ಬೆಸೆ-ಸಿದನು
ಬೆಸ್ತರ
ಬೇಂಟೆ-ಕಾರ
ಬೇಕಾಗಿತ್ತು
ಬೇಕಾಗಿತ್ತೆಂದು
ಬೇಕಾಗುತ್ತಿತ್ತು
ಬೇಕಾಗುತ್ತಿತ್ತೆಂಬುದು
ಬೇಕಾದ
ಬೇಕಾದವು
ಬೇಕಾದಷ್ಟಿವೆ
ಬೇಕಾ-ದು-ದನ್ನು
ಬೇಕು
ಬೇಗ
ಬೇಗಂ
ಬೇಗ-ಮಂಗಲ
ಬೇಗ-ಮಂಗಲ-ವಾಗಿ-ರ-ಬಹುದು
ಬೇಚರಾಕ್
ಬೇಚಿರಾಕ್
ಬೇಟೆಗೆ
ಬೇಟೆ-ಯನ್ನಾಡು-ವುದು
ಬೇಟೆ-ಯಾಡುವುದ-ರಲ್ಲಿ
ಬೇಡದೆ
ಬೇಡ-ರನ್ನು
ಬೇಡರ-ಹಳ್ಳಿ
ಬೇಡರ-ಹಳ್ಳಿ-ಯನ್ನು
ಬೇಡವ್ವೆ
ಬೇಡವ್ವೆಯ
ಬೇಡಿ
ಬೇಡಿ-ಕೊಂಡು
ಬೇಡಿ-ಕೊಳ್ಳಿ-ಮೆನೆ
ಬೇಡಿ-ಕೊಳ್ಳೆನೆ
ಬೇಡಿ-ಕೊಳ್ಳೆನ್ದೊಡೆ
ಬೇಡಿಕೋ
ಬೇಡಿ-ಪಡೆದ
ಬೇಡಿ-ಪಡೆ-ದನು
ಬೇಡಿ-ಪಡೆದು
ಬೇಡೆ
ಬೇಬಿ
ಬೇಬಿ-ಬೆಟ್ಟ
ಬೇರಂಬಾಡಿ
ಬೇರಾರು
ಬೇರೆ
ಬೇರೆ-ಬೇರ-ಯಾಗಿಯೇ
ಬೇರೆ-ಬೇರೆ
ಬೇರೆ-ಬೇರೆ-ಯಾಗಿ
ಬೇರೆ-ಯದೇ
ಬೇರೆ-ಯ-ವ-ರಿಗೆ
ಬೇರೆ-ಯಾಗಿಯೇ
ಬೇರೆಯೇ
ಬೇರ್ಪಡಿಸಿ
ಬೇಲೂ-ರನ್ನು
ಬೇಲೂ-ರಿಗೆ
ಬೇಲೂರಿನ
ಬೇಲೂರಿನಲ್ಲಿ
ಬೇಲೂರು
ಬೇವಿನ-ಕುಪ್ಪೆ
ಬೇವಿನ-ಕುಪ್ಪೆಯ
ಬೇವು-ಕಲ್ಲು
ಬೇಹಾರಿ
ಬೇಹಾರಿ-ಅಧಿ-ಕಾರಿ-ರಾಜ-ವರ್ತಕ
ಬೈಚ
ಬೈಚ-ದಂಡಾಧೀಶಂ
ಬೈಚ-ದಂಣಾಯಕ
ಬೈಚ-ದಂಣಾಯಕರ
ಬೈಚಪ್ಪ
ಬೈಚೆ-ದಂಡೇಶ-ನಿಗೆ
ಬೈಚೆಯ-ಬೀಚೆಯ-ಬೈಚ
ಬೈತ್ರಂ
ಬೊಂಮ-ದೇವ
ಬೊಂಮೋಜ-ನೊಳಗಾದ
ಬೊಕ್ಕಸ
ಬೊಕ್ಕಸಕ್ಕೆ
ಬೊಟ್ಟೈಯ್ಯ
ಬೊಪ್ಪ
ಬೊಪ್ಪ-ಗೌಡ-ನ-ಪುರ
ಬೊಪ್ಪಣ್ಣ
ಬೊಪ್ಪಣ್ಣ-ಪಂಡಿ-ತನ
ಬೊಪ್ಪ-ದಂಡಾಧೀಶ
ಬೊಪ್ಪ-ದೇವ
ಬೊಪ್ಪ-ದೇವನ
ಬೊಪ್ಪ-ದೇವನು
ಬೊಪ್ಪ-ನ-ಹಳ್ಳಿ
ಬೊಪ್ಪನು
ಬೊಪ್ಪ-ನೆಂಬ
ಬೊಪ್ಪ-ಸಂದ್ರ
ಬೊಪ್ಪ-ಸಮುದ್ರ
ಬೊಪ್ಪ-ಸಮುದ್ರ-ಗಳು
ಬೊಪ್ಪ-ಸಮುದ್ರವು
ಬೊಪ್ಪಾ-ದೇವಿ-ಯ-ರನ್ನು
ಬೊಪ್ಪಾ-ದೇವಿ-ಯ-ರಿನ್ತೀ
ಬೊಮನ-ಹಳ್ಳಿ
ಬೊಮ್ಮಣ್ಣ
ಬೊಮ್ಮಣ್ಣನ
ಬೊಮ್ಮಣ್ಣ-ನನ್ನು
ಬೊಮ್ಮನ-ಹಳ್ಳಿ
ಬೊಮ್ಮನ-ಹಳ್ಳಿ-ಗಳು
ಬೊಮ್ಮನ-ಹಳ್ಳಿಯ
ಬೊಮ್ಮನ-ಹಳ್ಳಿ-ಯನ್ನೂ
ಬೊಮ್ಮ-ನಾಯ-ಕ-ನ-ಹಳ್ಳಿ-ಯನ್ನು
ಬೊಮ್ಮ-ರಸ-ನ-ಕೊಪ್ಪಲು
ಬೊಮ್ಮವ್ವೆ
ಬೊಮ್ಮೆ-ಯನ-ಹಳ್ಳಿ-ಯನ್ನು
ಬೊಮ್ಮೇನ-ಹಳ್ಳಿ
ಬೊಯ್ಸಿ-ಕಟ್ಟೆ-ಯನ್ನು
ಬೊರಹ
ಬೋಕಂಣ
ಬೋಕಂಣನು
ಬೋಕಣ
ಬೋಕಣ್ಣ
ಬೋಕಣ್ಣನು
ಬೋಕಣ್ಣ-ರಲ್ಲದೆ
ಬೋಕಿ-ಮಯ್ಯನು
ಬೋಗ-ನ-ಹಳ್ಳಿ-ಯನ್ನು
ಬೋಗ-ವದಿಯ
ಬೋಗಾದಿ
ಬೋಗೇ-ಗೌಡನು
ಬೋಗೈಯ
ಬೋಗೈಯ್ಯ
ಬೋಯೆ-ಗನು
ಬೋರ-ಯನ-ಹಳ್ಳಿ
ಬೋಳ-ಚಾಮ-ರಾಜ
ಬೋಳ-ಚಾಮ-ರಾಜನ
ಬೋವ
ಬೋವರು
ಬ್ಯಾಡ-ರ-ಹಳ್ಳಿ
ಬ್ಯಾಲದ-ಕೆರೆ
ಬ್ರಣವಿಭೂಷಿತ
ಬ್ರಹ್ಮ-ಕುಲ-ದೀಪಕ-ನಪ್ಪ
ಬ್ರಹ್ಮಕ್ಷತ್ರಿಯ
ಬ್ರಹ್ಮಣ್ಯ-ತೀರ್ಥರ
ಬ್ರಹ್ಮದೇ-ಯವನ್ನಾಗಿ
ಬ್ರಹ್ಮದೇ-ಯ-ವಾಗಿ
ಬ್ರಹ್ಮ-ದೇವರ
ಬ್ರಹ್ಮ-ದೇವ-ರಿಗೆ
ಬ್ರಹ್ಮಧೇ-ಯ-ವಾಗಿ
ಬ್ರಹ್ಮ-ಪುರಿ-ಯನ್ನು
ಬ್ರಹ್ಮರಾಶಿ
ಬ್ರಹ್ಮೇಶ್ವರ
ಬ್ರಾಹ್ಮಣ
ಬ್ರಾಹ್ಮಣ-ನಿಗೆ
ಬ್ರಾಹ್ಮಣರ
ಬ್ರಾಹ್ಮಣ-ರಲ್ಲಿ
ಬ್ರಾಹ್ಮಣ-ರಾಗಿದ್ದ-ರೆಂದು
ಬ್ರಾಹ್ಮಣ-ರಿಂದ
ಬ್ರಾಹ್ಮಣ-ರಿ-ಗಾಗಿ
ಬ್ರಾಹ್ಮಣ-ರಿಗೆ
ಬ್ರಾಹ್ಮಣರು
ಬ್ರಿಟಿಷರ
ಬ್ರಿಟಿಷ-ರನ್ನು
ಬ್ರಿಟಿಷ-ರಿಗೂ
ಬ್ರಿಟಿಷರು
ಬ್ರಿಟಿಷ್
ಬ್ರಿಟೀಷ್
ಬ್ರೂಸ್ಫೂಟ್
ಭಂಗಿ-ಕರ
ಭಂಡಾರ
ಭಂಡಾರಕ್ಕೆ
ಭಂಡಾರದ
ಭಂಡಾರ-ಬ-ಸದಿಯ
ಭಂಡಾರ-ವನ್ನು
ಭಂಡಾರ-ವೆನಿಪ
ಭಂಡಾರಿ
ಭಂಡಾರಿ-ಗ-ನಾಗಿದ್ದನು
ಭಂಡಾರಿ-ಗಳಾಗಿದ್ದು
ಭಂಡಾರಿ-ಗಳು
ಭಂಡಾರಿ-ಗಳು-ಹಿರಿ-ಯ-ಭಂಡಾರಿ-ಮಾಣಿ-ಕ-ಭಂಡಾರಿ
ಭಂಡಾರಿ-ಗೌಂಡ
ಭಂಡಾರಿ-ಯಾಗಿದ್ದ
ಭಂಡಾರಿ-ಯಾಗಿದ್ದನು
ಭಂಡಾರಿಯು
ಭಂಡಿ-ವಾಳ
ಭಂಢಾರಿಯ
ಭಕ್ತ-ನಾಗಿದ್ದು
ಭಕ್ತ-ರಾಗಿ
ಭಕ್ತ-ರಿಗೆ
ಭಕ್ತಿ
ಭಕ್ತಿ-ಯಿಂದ
ಭಕ್ತಿ-ಯುಳ್ಳ-ವ-ನಾಗಿದ್ದ-ನೆಂದು
ಭಗೀರಥ
ಭಟ-ಭೀಮೆಯ-ನಾಯಕ
ಭಟಾ-ರರ
ಭಟಾರ-ರಿಗೆ
ಭಟ್ಟ
ಭಟ್ಟಂಗಿ
ಭಟ್ಟಂಗಿ-ಗಳಾಗಿ
ಭಟ್ಟಂಗಿ-ಗ-ಳೆಂದು
ಭಟ್ಟ-ನೆಂಬ
ಭಟ್ಟರ
ಭಟ್ಟ-ರ-ಬಾಚಪ್ಪನ
ಭಟ್ಟ-ರ-ಬಾಚಪ್ಪ-ನ-ವರು
ಭಟ್ಟ-ರ-ಬಾಚಪ್ಪ-ರಲ್ಲದೆ
ಭಟ್ಟ-ರ-ಬಾಚಿಯಪ್ಪನ
ಭಟ್ಟ-ರ-ಬಾಚಿ-ಯಪ್ಪ-ನಿಗೆ
ಭಟ್ಟ-ರ-ಬಾಚಿಯಪ್ಪನು
ಭಟ್ಟ-ರ-ಬಾಚಿಯಪ್ಪನೂ
ಭಟ್ಟಾರಕ
ಭಟ್ಟಾರ-ಕರ
ಭತ್ತ
ಭತ್ತಾಯ
ಭದ್ರ-ಕಾಳಮ್ಮ
ಭದ್ರ-ಕಾಳಿ-ಯಣ್ಣ
ಭದ್ರ-ನ-ಕೊಪ್ಪಲು
ಭದ್ರ-ಪಡಿ-ಸಲು
ಭದ್ರ-ಬಾಹು
ಭಯಂಕರ
ಭಯಂಕರ-ನಾಗಿ
ಭಯ-ದಿಂದ
ಭಯಲೋಭದುರ್ಲ್ಲಭಂ
ಭಯಿರಮೇಶ್ವರ
ಭಯಿರಮೇಶ್ವರ-ಪುರ
ಭರತ
ಭರ-ತ-ಚಮೂಪ-ತಿಯ
ಭರ-ತ-ಜೀಯ
ಭರ-ತ-ದಂಡ-ನಾಯ-ಕನು
ಭರ-ತನ
ಭರ-ತನೂ
ಭರ-ತನೇ
ಭರ-ತ-ಮಯ್ಯ
ಭರ-ತ-ರನ್ನು
ಭರ-ತಿ-ಮಯ್ಯ
ಭರ-ತಿ-ಮಯ್ಯ-ಗಳು
ಭರ-ತಿ-ಮಯ್ಯನ
ಭರ-ತಿ-ಮಯ್ಯರ
ಭರ-ತಿ-ಮಯ್ಯರು
ಭರ-ತೆಯ
ಭರ-ತೆಯ-ನಾಯಕ
ಭರ-ತೆಯ-ನಾಯಕಂ
ಭರ-ತೇಶ-ದಂಡ-ನಾಯ-ಕನ
ಭರ-ತೇಶ್ವರ
ಭರ್ತಿ-ಯಾಗಿತ್ತೆಂದು
ಭವತ್ಪ್ರತಾಪ
ಭವ-ನದಂತಿದ್ದ
ಭಾಗ
ಭಾಗಕ್ಕೆ
ಭಾಗ-ಗಳನ್ನು
ಭಾಗ-ಗಳಲ್ಲಿ
ಭಾಗ-ಗಳಾಗಿ
ಭಾಗ-ಗಳಾಗಿದ್ದ-ವೆಂದು
ಭಾಗ-ಗಳಿಗೂ
ಭಾಗ-ಗಳು
ಭಾಗ-ಗಳೂ
ಭಾಗದ
ಭಾಗ-ದಲ್ಲಿ
ಭಾಗ-ದಲ್ಲಿದ್ದ
ಭಾಗ-ದಲ್ಲಿದ್ದ-ನೆಂದು
ಭಾಗ-ದಲ್ಲಿ-ರುವ
ಭಾಗ-ದಲ್ಲೇ
ಭಾಗ-ದ-ವಳಾಗಿರ-ಬಹುದು
ಭಾಗ-ದಿಂದ
ಭಾಗ-ವತೋತ್ತಮೆ
ಭಾಗ-ವನ್ನು
ಭಾಗ-ವನ್ನೆಲ್ಲಾ
ಭಾಗ-ವಹಿಸಿ
ಭಾಗ-ವಹಿಸಿದ
ಭಾಗ-ವಹಿಸಿದ್ದ
ಭಾಗ-ವಹಿಸಿದ್ದ-ನೆಂದು
ಭಾಗ-ವಹಿಸಿದ್ದ-ರೆಂದು
ಭಾಗ-ವಹಿಸಿದ್ದಾರೆ
ಭಾಗ-ವಹಿಸಿ-ರ-ಬಹುದು
ಭಾಗ-ವಹಿ-ಸುತ್ತಿದ್ದ-ನೆಂಬು-ದನ್ನು
ಭಾಗ-ವಾಗಿ
ಭಾಗ-ವಾ-ಗಿತ್ತು
ಭಾಗ-ವಾಗಿತ್ತೆಂದು
ಭಾಗ-ವಾಗಿ-ರ-ಬಹುದು
ಭಾಗ-ವಿದ್ದು
ಭಾಗವು
ಭಾಗ-ವೆಂದು
ಭಾಗವೇ
ಭಾಗಶಃ
ಭಾಗಿ-ಯಾಗಿದ್ದ-ರೆಂದು
ಭಾಗ್ಯ-ವಾಗಲಿ
ಭಾನು-ಕೀರ್ತಿ
ಭಾನು-ಕೀರ್ತಿ-ದೇವನು
ಭಾನು-ಕೀರ್ತಿ-ದೇವರ
ಭಾನುಕೀರ್ತ್ತಿ
ಭಾನು-ಮತಿ-ಯ-ವರು
ಭಾನು-ಮತಿ-ಯ-ವರೂ
ಭಾರ
ಭಾರತ
ಭಾರ-ತದ
ಭಾರದ್ವಾಜ-ಗೋತ್ರ-ದ-ವನು
ಭಾರೀ
ಭಾರ್ಯೆ
ಭಾವನೆ-ಗಿಂತ
ಭಾವಮೈದ
ಭಾವಮೈದುನ
ಭಾವಿಸ-ಬಹುದು
ಭಾವಿ-ಸಿದ
ಭಾವಿಸಿ-ರುವಂತಿದೆ
ಭಾಷಾಪ್ರಯೋಗ
ಭಾಷಿಕ
ಭಾಷೆ
ಭಾಷೆ-ಗಳ
ಭಾಷೆ-ಗಳಲ್ಲಿ-ರು-ವು-ದನ್ನು
ಭಾಷೆ-ಗಳೆರಡ-ರಲ್ಲೂ
ಭಾಷೆಗೆ
ಭಾಷೆ-ಗೆ-ತಪ್ಪುವ
ಭಾಷೆಯ
ಭಾಷೆ-ಯನ್ನೇ
ಭಾಷೆ-ಯಲ್ಲಿ
ಭಾಷೆ-ಯವು
ಭಾಷ್ಯ-ಕಾರರು
ಭಾಸ್ವದ್ಬೃಹ
ಭಿತ್ತಿಯ
ಭಿನ್ನ
ಭಿನ್ನನು
ಭಿನ್ನ-ನೆಂದು
ಭಿನ್ನರು
ಭಿನ್ನ-ರೆಂದು
ಭಿನ್ನ-ವಾ-ಗಿತ್ತು
ಭಿನ್ನ-ವಾಗಿದೆ
ಭಿನ್ನಾಭಿಪ್ರಾಯ-ವನ್ನು
ಭಿಲ್ಲಮ-ನಿಗೂ
ಭೀಕರ-ತೆ-ಯನ್ನು
ಭೀಕರ-ವಾದ
ಭೀತ-ರಾಗಿ
ಭೀತಿ
ಭೀಮ
ಭೀಮ-ಗಾಮುಣ್ಡರು
ಭೀಮಣ್ಣ
ಭೀಮಣ್ಣನು
ಭೀಮ-ದೇವ
ಭೀಮ-ನ-ಕಂಡಿ-ಬೆಟ್ಟ
ಭೀಮ-ನ-ಕೆರೆ
ಭೀಮ-ನ-ಕೆರೆಗೆ
ಭೀಮ-ನ-ಹಳ್ಳಿ
ಭೀಮ-ರಾಯ
ಭೀಮ-ರಾಯನು
ಭೀಮಾರ್ಜುನರು
ಭೀಮೆಯ
ಭೀಮೆಯ-ನಾಯಕ
ಭೀಮೆಯ-ನಾಯ-ಕ-ನಾಗಿ-ರುವ
ಭೀಮೆಯ-ನಾಯ-ಕನು
ಭೀಮೇಶ್ವರ
ಭೀಮೇಶ್ವರಿ
ಭೀಷ್ಮ-ಪರ್ವ-ದಲ್ಲಿ
ಭುಕ್ತಿ
ಭುಜಪ್ರತಾಪದಿ
ಭುಜ-ಬಲ
ಭುಜ-ಬಲಪ್ರತಾಪ
ಭುಜ-ಬಲ-ರಾಯ-ನೆಂಬ
ಭುಜ-ಬಲ-ವೀರ-ಗಂಗ
ಭುಜ-ಬಲಿ
ಭುಜ-ಬಲಿ-ಚರಿತೆ-ಯೆಂಬ
ಭುಜಬಳ
ಭುಜಬಳ-ವೀರ-ಗಂಗ
ಭುಜಬಳಾವಷ್ಟಂಭ
ಭುಜ-ವಿಜಯ
ಭುಜ-ಸಾಹ-ಸದಿಂ
ಭುಜಾ-ದಂಡ
ಭುಜಾ-ದಂಡ-ವೆನಿ-ಸಿದ್ದ
ಭುವ-ನದೊಳಾಂತು
ಭುವ-ನೇಶ್ವರಿ
ಭುವನೈಕ-ವೀರ-ನೆಂಬ
ಭುವಿ
ಭೂಕಾಮಿನಿಯಿರ್ದ್ದಳಾ
ಭೂಗೋಳ
ಭೂಗೋಳ-ವನ್ನು
ಭೂಚಕ್ರ-ವಲಯ
ಭೂತಾ-ನಾಮ
ಭೂದಾನ
ಭೂದಾನಗ್ರಾಮ-ಧರ್ಮ-ಸಾಧನ-ವಾಗಿ
ಭೂದೇವತಾ
ಭೂನ್ರಿಪಂ
ಭೂಪತಿ
ಭೂಪತಿಃ
ಭೂಪತಿಯು
ಭೂಪನಾ
ಭೂಪರಿಮಿತೇ
ಭೂಪಸ್ಥಾನ-ರಂಜಿತೇ
ಭೂಪಸ್ಯ
ಭೂಪಾಲ
ಭೂಪಾಲಂ
ಭೂಪಾಲ-ಚಿರ-ಪುಣ್ಯ
ಭೂಪಾಲನ
ಭೂಬುಜಂ
ಭೂಭಾಗದ
ಭೂಭಾಗವು
ಭೂಭಾರ-ವನ್ನು
ಭೂಭುಜಂ
ಭೂಭು-ವನಂ
ಭೂಭೂಜಿ
ಭೂಭ್ರುನ್ನಿಳಯ
ಭೂಮಿ-ಗಳ
ಭೂಮಿ-ಗಳನ್ನು
ಭೂಮಿ-ಗ-ಳಿಗೆ
ಭೂಮಿಗೆ
ಭೂಮಿ-ದಾನ
ಭೂಮಿಪನ
ಭೂಮಿ-ಭಾಗ-ದೊಳ-ದನ್ಯ-ರದೇಕೆ
ಭೂಮಿಯ
ಭೂಮಿ-ಯನ್ನು
ಭೂಮಿ-ಯಾ-ಗಿತ್ತು
ಭೂಮಿ-ಯಾಗಿದೆ
ಭೂಮಿಯು
ಭೂವಲ್ಲಭ-ನಿಗೆ-ಬೂತುಗ
ಭೂವಿಕ್ರಮ-ನನ್ನು
ಭೂವಿಕ್ರಮನು
ಭೂಶಿ-ರದ-ವರೆ-ಗಿನ
ಭೂಷಿತಂ
ಭೇಟಿ
ಭೇಟಿ-ನೀಡಿದ್ದ-ನೆಂದು
ಭೇಟಿ-ನೀಡಿ-ರ-ಬಹುದು
ಭೇಟಿ-ನೀಡಿ-ರುತ್ತಾನೆ
ಭೇಟಿ-ನೀಡಿ-ರು-ವಂತೆ
ಭೇಟಿ-ಯಾಗಿ-ರ-ಬಹುದು
ಭೇಟಿ-ಯಾಗಿ-ರ-ಬಹು-ದೆಂದು
ಭೇದಿಸಿ
ಭೇರುಂಡ-ವರ್ಗ-ವನ್ನು
ಭೈತ್ರ
ಭೈರ-ಕಂಬೆಯ
ಭೈರಮೇಶ್ವರ
ಭೈರವ-ದಂಣಾಯ-ಕಿತ್ತಿ-ಯರ
ಭೈರವ-ಪುರ-ವೆಂಬ
ಭೈರವಾಪರು-ವೆಂಬ
ಭೈರವಾ-ಪುರ-ವೆಂಬ
ಭೈರವೇಶ್ವರ
ಭೈರವ್ವೆ
ಭೈರಾ-ಪುರ
ಭೈರಾ-ಪುರ-ದಲ್ಲಿರುವ
ಭೈರಾ-ಪುರ-ವೆಂಬ
ಭೈರೇ-ದೇವರ
ಭೋಗ
ಭೋಗ-ನ-ಹಳ್ಳಿ
ಭೋಗಯ್ಯ-ದೇವ
ಭೋಗ-ರಾಜ-ಭೂ-ಪಾಲನು
ಭೋಗ-ರಾಜ-ವರ-ತಲ್ಪಃ
ಭೋಗ-ವತಿ-ಯಲ್ಲಿ
ಭೋಗ-ವದಿಯ-ಬೋಗಾದಿ
ಭೋಗ-ವ-ಸದಿ-ಯೊಳು
ಭೋಗಾನುಭಾವಿ
ಭೋಗೈಯ್ಯ
ಭೋಜಃ
ಭೋಜನಕ್ಕೆ
ಭೋಜ-ರರು
ಭೋಜ-ರಾಜ-ನಿಗೆ
ಭೌಗೋ-ಳಿಕ
ಭೌಗೋ-ಳಿಕ-ವಾಗಿ
ಮ
ಮಂ
ಮಂಗಪ್ಪ
ಮಂಗಲ
ಮಂಗಲಕ್ಕೆ
ಮಂಗಲದ
ಮಂಗಲ-ದಲ್ಲಿ
ಮಂಗಲಮ್
ಮಂಗಲ-ವಾದ
ಮಂಗಲವು
ಮಂಗಳೂರು
ಮಂಚ-ಗಾವುಂಡನು
ಮಂಚ-ಗೌಂಡನ
ಮಂಚ-ಗೌಡ
ಮಂಚನ-ಹಳ್ಳಿ
ಮಂಚನ-ಹಳ್ಳಿ-ಯನ್ನು
ಮಂಚಯ-ದಂಡ-ನಾಯಕ
ಮಂಚಲಾ-ದೇವಿ
ಮಂಚವ್ವೆ
ಮಂಚಿ-ಗೌಡ
ಮಂಚೆ-ಗಾವುಂಡ
ಮಂಚೆ-ಗೌಡ
ಮಂಚೇ-ಗೌಂಡನ
ಮಂಚೇ-ಗೌಡನ
ಮಂಜಯ್ಯ
ಮಂಜಯ್ಯ-ನನ್ನು
ಮಂಜಯ್ಯನು
ಮಂಜು-ನಾಥ್
ಮಂಟಪ
ಮಂಟಪ-ಗಳನ್ನು
ಮಂಟಪ-ಗಳು
ಮಂಟಪದ
ಮಂಟಪ-ದಲ್ಲಿದೆ
ಮಂಟಪ-ವನ್ನು
ಮಂಟಪ-ವನ್ನು-ರಂಗ-ಮಂಟಪ
ಮಂಟಿ
ಮಂಟಿ-ಗಳಿಂದ
ಮಂಟಿಗೆ
ಮಂಠಿ
ಮಂಠೆ
ಮಂಠೆದ
ಮಂಠೆಯ
ಮಂಠೆ-ಯದ
ಮಂಠೆ-ಯ-ಮಂಡ್ಯ
ಮಂಠೆ-ಯವೇ
ಮಂಠೇದ
ಮಂಠೇದಯ್ಯ
ಮಂಠೇದಯ್ಯ-ನ-ವರು
ಮಂಡ
ಮಂಡ-ಗೌಡ-ನೆಂಬ
ಮಂಡ-ಮಂಡೆ-ಮಂಡೇವು-ಮಂಡ್ಯ
ಮಂಡಯಂ
ಮಂಡ-ರಿ-ವರ್ಮ-ರಾಜ
ಮಂಡಲ
ಮಂಡ-ಲಕ್ಕೂ
ಮಂಡ-ಲ-ಗ-ಳನ್ನಾಗಿ
ಮಂಡ-ಲ-ಗಳಾಗಿ
ಮಂಡ-ಲ-ಗಳಿದ್ದವು
ಮಂಡ-ಲ-ವನ್ನು
ಮಂಡ-ಲ-ವನ್ನೂ
ಮಂಡ-ಲ-ವಿಷಯ-ದೇಶ-ನಾಡು-ಕಂಪಣ
ಮಂಡ-ಲಸ್ವಾಮಿ
ಮಂಡ-ಲಸ್ವಾಮಿಗೆ
ಮಂಡ-ಲಸ್ವಾಮಿಯ
ಮಂಡ-ಲಸ್ವಾಮಿಯು
ಮಂಡ-ಲಾಧಿ-ಪತಿ-ಯನ್ನಾಗಿ
ಮಂಡ-ಲಾಧಿ-ಪತಿ-ಯಾಗಿದ್ದ
ಮಂಡ-ಲಿಕ
ಮಂಡ-ಲಿಕರು
ಮಂಡ-ಲೀಕ
ಮಂಡ-ಲೇಶ್ವರ
ಮಂಡ-ಲೇಶ್ವರ-ರನ್ನು
ಮಂಡ-ಲೇಶ್ವರ-ರಾಗಿ
ಮಂಡ-ಲೇಶ್ವರರು
ಮಂಡ-ಳಿಕ
ಮಂಡ-ಳಿಕ-ಜೂಬು
ಮಂಡ-ಳಿಕ-ನಾಗಿ
ಮಂಡ-ಳಿಕ-ನಾದ
ಮಂಡ-ಳಿ-ಕರು
ಮಂಡ-ಳೀಕ-ಜೂಬು
ಮಂಡ-ಳೀ-ಕರ-ಗಂಡ
ಮಂಡ-ಳೇಶ್ವರ-ನಾಗಿ
ಮಂಡ-ಳೇಶ್ವರರು
ಮಂಡಸ್ವಾಮಿಗೆ
ಮಂಡಿತ
ಮಂಡಿಸಲ್ಪಟ್ಟ
ಮಂಡೆಯ
ಮಂಡೆಯಂ
ಮಂಡೆ-ಯದ
ಮಂಡೆವೇಮು
ಮಂಡೇವು
ಮಂಡೇವುಕೆ
ಮಂಡೇವುಕ್ಕೆ
ಮಂಡೇವು-ಮಂಡ್ಯ
ಮಂಡ್ಯ
ಮಂಡ್ಯಂ
ಮಂಡ್ಯಂಣ
ಮಂಡ್ಯಕ್ಕಿಂತಲೂ
ಮಂಡ್ಯಕ್ಕೆ
ಮಂಡ್ಯ-ಗೋಪ-ಣನ
ಮಂಡ್ಯ-ಜಿಲ್ಲೆಯ
ಮಂಡ್ಯ-ಜಿಲ್ಲೆ-ಯಲ್ಲಿ
ಮಂಡ್ಯ-ಜಿಲ್ಲೆಯು
ಮಂಡ್ಯದ
ಮಂಡ್ಯ-ದಲ್ಲಿ
ಮಂಡ್ಯ-ವನ್ನು
ಮಂಡ್ಯವು
ಮಂತ-ಲಲಾಮನೀ
ಮಂತ್ರಚಿನ್ತಾಮಣಿ
ಮಂತ್ರ-ವಿದ್ಯಾ-ವಿಕಾಶಂ
ಮಂತ್ರಿ
ಮಂತ್ರಿ-ಗಳ
ಮಂತ್ರಿ-ಗಳಾಗಿದ್ದ
ಮಂತ್ರಿ-ಗಳಾಗಿದ್ದರು
ಮಂತ್ರಿ-ಗಳಾಗಿದ್ದ-ರೆಂದು
ಮಂತ್ರಿ-ಗಳಾಗಿದ್ದಾಗ
ಮಂತ್ರಿ-ಗಳಾಗಿದ್ದಿರ-ಬಹುದು
ಮಂತ್ರಿ-ಗಳಾಗಿದ್ದು
ಮಂತ್ರಿ-ಗ-ಳಾದ
ಮಂತ್ರಿ-ಗಳು
ಮಂತ್ರಿ-ಗಳೂ
ಮಂತ್ರಿ-ಗ-ಳೆಂದೂ
ಮಂತ್ರಿ-ಚೂಡಾಮಣಿ
ಮಂತ್ರಿ-ಣಾವಭ-ವತಾಂ
ಮಂತ್ರಿಣೇ
ಮಂತ್ರಿ-ತಿಳಕಂ
ಮಂತ್ರಿ-ಪದ-ವಿ-ಯಲ್ಲಿದ್ದಿರ-ಬಹುದು
ಮಂತ್ರಿ-ಪರಿಷತ್ತಿ-ನಲ್ಲಿ
ಮಂತ್ರಿಭಿಃ
ಮಂತ್ರಿ-ಮಂಡಲ
ಮಂತ್ರಿ-ಮಂಡ-ಲದ
ಮಂತ್ರಿ-ಮಾಣಿಕ್ಯ
ಮಂತ್ರಿ-ಮಾಣಿಕ್ಯಂ
ಮಂತ್ರಿ-ಮುಖದರ್ಪಣ
ಮಂತ್ರಿಯ
ಮಂತ್ರಿ-ಯಾಗಿದ್ದ
ಮಂತ್ರಿ-ಯಾಗಿದ್ದಂತೆ
ಮಂತ್ರಿ-ಯಾಗಿದ್ದನು
ಮಂತ್ರಿ-ಯಾಗಿದ್ದ-ನೆಂದು-ಗೋವಿಂದಯ್ಯಾಖ್ಯ
ಮಂತ್ರಿ-ಯಾಗಿದ್ದು-ದರ
ಮಂತ್ರಿ-ಯಾಗಿ-ರ-ಬಹುದು
ಮಂತ್ರಿ-ಯಾದಂ
ಮಂತ್ರಿ-ಯಾದ-ನೆಂದು
ಮಂತ್ರಿಯು
ಮಂತ್ರಿಯೂ
ಮಂತ್ರಿ-ಯೂ-ಥಾಗ್ರಣಿ
ಮಂತ್ರಿ-ಯೊಡನೆ
ಮಂತ್ರೀಶ
ಮಂತ್ರೀಶ್ವ-ರನಾ-ದಂತೆ
ಮಂದ-ಗೆರೆ
ಮಂದಿ
ಮಂದಿರಂ
ಮಂದಿ-ರಲ್ಲಿದೆ
ಮಂನನ
ಮಂನಿತಿ
ಮಂನೆಯ
ಮಂನೆಯ-ಗಜ-ಪತಿ
ಮಂನೆಯ-ಜೂಬು
ಮಂನೆ-ಯರು
ಮಕರ
ಮಕರ-ರಾಜ್ಯ
ಮಕರ-ರಾಯ
ಮಕುಟ-ಮಂಡ-ಲಿಕರ
ಮಕ್ಕಳ
ಮಕ್ಕ-ಳನ್ನು
ಮಕ್ಕ-ಳನ್ನೋ
ಮಕ್ಕಳಾಗಿದ್ದ
ಮಕ್ಕಳಾ-ಗಿದ್ದು
ಮಕ್ಕಳಾಗಿರ-ಬಹುದು
ಮಕ್ಕ-ಳಾದ
ಮಕ್ಕಳಿಗೂ
ಮಕ್ಕ-ಳಿಗೆ
ಮಕ್ಕಳಿದ್ದರು
ಮಕ್ಕಳಿದ್ದ-ರೆಂದು
ಮಕ್ಕಳಿದ್ದ-ರೆಂದೂ
ಮಕ್ಕಳಿದ್ದು
ಮಕ್ಕಳಿದ್ದುದು
ಮಕ್ಕಳಿಬ್ಬರೂ
ಮಕ್ಕಳಿಲ್ಲದ
ಮಕ್ಕಳಿಲ್ಲದೇ
ಮಕ್ಕಳು
ಮಕ್ಕಳು-ಗಳಿಗೂ
ಮಕ್ಕಳೂ
ಮಕ್ಕ-ಳೆಂದು
ಮಕ್ಕ-ಳೆಂದೂ
ಮಕ್ಕಳೆಂಬುದು
ಮಕ್ಕಳೇ
ಮಕ್ಕಳೊಡನೆ
ಮಖ್ಖಳು
ಮಗ
ಮಗಂ
ಮಗ-ದೊಬ್ಬ
ಮಗನ
ಮಗ-ನನ್ನು
ಮಗ-ನನ್ನೂ
ಮಗ-ನಾಗಿ
ಮಗ-ನಾಗಿದ್ದನು
ಮಗ-ನಾಗಿದ್ದು
ಮಗ-ನಾಗಿ-ರ-ಬಹುದು
ಮಗ-ನಾಗಿ-ರಲು
ಮಗ-ನಾಗಿ-ರುವ
ಮಗ-ನಾಗುತ್ತಾನೆ
ಮಗ-ನಾದ
ಮಗ-ನಿಗೆ
ಮಗ-ನಿದ್ದನು
ಮಗ-ನಿದ್ದ-ನೆಂದು
ಮಗ-ನಿದ್ದು
ಮಗ-ನಿರ-ಬಹುದು
ಮಗ-ನಿರ-ಬಹು-ದೆಂದು
ಮಗನು
ಮಗನೂ
ಮಗ-ನೆಂದು
ಮಗ-ನೆಂದೂ
ಮಗನೇ
ಮಗನೋ
ಮಗರ
ಮಗ-ರನ
ಮಗರ-ರಾಜ್ಯ
ಮಗರ-ರಾಯ
ಮಗರಾಧಿ-ರಾಯ
ಮಗಳ
ಮಗಳನ್ನು
ಮಗ-ಳನ್ನೂ
ಮಗಳಾಗಿದ್ದು
ಮಗ-ಳಾದ
ಮಗ-ಳಿದ್ದ
ಮಗಳು
ಮಗಳು-ಶಾ-ಸನ
ಮಗ-ಶಿಷ್ಯ-ನಾದ್ದ-ರಿಂದ
ಮಗು
ಮಗುರ್ದಡೆರೆಪ್ಪುವ
ಮಗ್ಗ-ತೆರೆ
ಮಗ್ಗದ
ಮಗ್ಗ-ದೆರೆ-ಯನ್ನು
ಮಗ್ಗ-ನ-ಹಳ್ಳಿಯ
ಮಗ್ಗ-ನ-ಹಳ್ಳಿ-ಯನ್ನು
ಮಗ್ಗ-ವನ್ನು
ಮಗ್ನ-ನಾಗಿದ್ದ
ಮಗ್ನ-ನಾದ
ಮಚ್ಚರಿಪ-ನಾಯ-ಕ-ರ-ಗಂಣ್ಡ
ಮಜ್ಜ-ನದ
ಮಟ್ಟದ
ಮಟ್ಟಿಗೂ
ಮಟ್ಟಿಗೆ
ಮಠಕ್ಕೆ
ಮಠ-ಗ-ಳಿಗೆ
ಮಠ-ಗಳು
ಮಠದ
ಮಠದ-ಕೇರಿ
ಮಠ-ಪತಿ-ದಾಸ-ವೈಷ್ಣವರ
ಮಠ-ಮಾನ್ಯ-ಗಳನ್ನು
ಮಠ-ವನ್ನು
ಮಠ-ವಿದೆ
ಮಠಾಧಿ-ಪತಿ
ಮಡಕೆ-ಪಟ್ಟಣ
ಮಡ-ವನ-ಕೋಡಿ
ಮಡಿದ
ಮಡಿದನು
ಮಡಿದ-ನೆಂದು
ಮಡಿದ-ರೆಂದು
ಮಡಿದ-ವನು
ಮಡಿದ-ವರ
ಮಡಿದಾಗ
ಮಡಿದಿದ್ದು
ಮಡಿದಿರ-ಬಹುದು
ಮಡಿದಿರ-ಬಹು-ದೆಂದು
ಮಡಿದಿರ-ವುದು
ಮಡಿದಿ-ರುವ
ಮಡಿ-ಯನ-ಹಳ್ಳಿಯ
ಮಡಿ-ಯಲು
ಮಡಿಯುತ್ತಾನೆ
ಮಡಿಯುತ್ತಾ-ನೆಂದಿದೆ
ಮಡಿಯುತ್ತಾರೆ
ಮಡಿ-ವಳ್ಳ
ಮಡು
ಮಡು-ವನ್ನು
ಮಡು-ವಿನ-ಕೋಡಿ
ಮಡು-ವಿನ-ಕೋಡಿಯ
ಮಡು-ವಿ-ನಲ್ಲಿ
ಮಡು-ಹಿನ
ಮಣಲಯ-ರನ
ಮಣಲೆ
ಮಣಲೆ-ಅ-ರಸನು
ಮಣ-ಲೆಯ
ಮಣಲೆ-ಯರ
ಮಣಲೆ-ಯ-ರನ
ಮಣಲೆ-ಯ-ರನು
ಮಣಲೆ-ಯ-ರರು
ಮಣಲೆ-ಯ-ರ-ಸರ
ಮಣಲೆ-ಯ-ರ-ಸರಾ
ಮಣಲೆ-ಯಾ-ರನಿರ-ಬಹುದು
ಮಣಲೆ-ಯಾ-ರನು
ಮಣಲೆರ
ಮಣಲೆ-ರಙ್ಗೆ
ಮಣಲೆ-ರನ
ಮಣಲೆ-ರನು
ಮಣಲೇರ
ಮಣಲೇರನ
ಮಣಲೇರ-ನಿಗೆ
ಮಣಲೇರನು
ಮಣಲೇರ-ರನ್ನು
ಮಣಲೇರ-ರನ್ನು-ಮರು-ವರ್ಮ
ಮಣಲೇರರು
ಮಣ-ಳೇಶ್ವರ
ಮಣ-ವಾಳ
ಮಣಾಲ-ರನ
ಮಣಾಲರ-ನನ್ನು
ಮಣಾಲ-ರನು
ಮಣಿ-ಕರ್ಣಿಕಾ
ಮಣಿಕೊನಖ-ಗಳ
ಮಣಿ-ನಾಗ-ಪುರ
ಮಣಿ-ನಾಗ-ಪುರ-ವರಾಧೀಶ್ವರ
ಮಣಿ-ನಾಗ-ಪುರ-ವರಾಧೀಶ್ವರನೂ
ಮಣಿ-ನಾಗ-ರಗ್ರಾಮ-ವೆಂದು
ಮಣಿಪ್ರದೀಪ
ಮಣಿ-ಯೂರು
ಮಣ್ಡ-ಳೀಕ-ಜೂಬು
ಮಣ್ಣನ್ನು
ಮಣ್ಣು
ಮಣ್ಣೆ
ಮಣ್ಣೆ-ಯನ್ನು
ಮಣ್ಣೆ-ಯಲ್ಲಿ
ಮಣ್ಣೆ-ಯಿಂದ
ಮತ
ಮತಾನು-ಯಾಯಿ-ಗಳು
ಮತಿ-ಸಾ-ಗರ
ಮತೀಯ
ಮತು
ಮತ್ತಮಾತಂಗ
ಮತ್ತ-ಯರ
ಮತ್ತರು
ಮತ್ತಿ-ಕೆರೆ-ಯನ್ನು
ಮತ್ತಿ-ದಾಗ
ಮತ್ತು
ಮತ್ತೆ
ಮತ್ತೆ-ಗೆರೆ
ಮತ್ತೊಂದು
ಮತ್ತೊಬ್ಬ
ಮಥ್ಚ-ಮತ್ಸ್ಯ
ಮದವದುಗ್ರ-ವೈರಿಮದ-ಮರ್ದ್ಧನ
ಮದವದುದಗ್ರ
ಮದವ-ಳಿಗೆ
ಮದಿ-ಸಿದ
ಮದುಗನ್ದೂರ
ಮದುವೆ
ಮದುವೆ-ನಿಂದು
ಮದುವೆಯ
ಮದುವೆ-ಯನ್ನು
ಮದುವೆ-ಯಾಗಿದ್ದನು
ಮದುವೆ-ಯಾದ-ನೆಂದು
ಮದ್ದಿನ-ಮನೆ-ಗಳು
ಮದ್ದಿ-ಯಕ್ಕರ
ಮದ್ದೂರ
ಮದ್ದೂರನ್ನು
ಮದ್ದೂರಾದ
ಮದ್ದೂರಿಗೆ
ಮದ್ದೂರಿನ
ಮದ್ದೂರಿ-ನಲ್ಲಿ
ಮದ್ದೂರು
ಮದ್ರಾಸಿ-ನಲ್ಲಿದ್ದ
ಮದ್ರಾಸಿ-ನ-ವರೆಗೂ
ಮದ್ರಾಸ್
ಮಧುಕೇಶ್ವರ
ಮಧುರ-ಮಂಡಲ
ಮಧುರ-ವಾಗಿದ್ದವು
ಮಧುರ-ವಾದ
ಮಧುರಾ
ಮಧುರೆ
ಮಧು-ರೆಯ
ಮಧುರೆ-ಯನ್ನು
ಮಧುವಂಣ
ಮಧುಸೂಧನ
ಮಧುಸೂಧನನ
ಮಧ್ಯ-ಕಾಲೀನ
ಮಧ್ಯಗತ-ವಾಗಿ-ರುವುದ-ರಿಂದ
ಮಧ್ಯದ
ಮಧ್ಯ-ದಲ್ಲಿ
ಮಧ್ಯದೇ-ಸಮುದ್ದಂಡ-ವಿನಾಳ್ದು
ಮಧ್ಯ-ದೊಳಗಣ
ಮಧ್ಯ-ದೊಳಗೆ
ಮಧ್ಯ-ಭಾಗ-ದಲ್ಲಿ
ಮಧ್ಯಸ್ಥಿಕೆ
ಮಧ್ಯೆ
ಮನಗಾಣ-ಲಿಲ್ಲ
ಮನಗಾಣಲು
ಮನದನ್ನ-ನಪ್ಪ
ಮನಮೊಸೆದು
ಮನಾಲರ
ಮನಿ-ಗಳು
ಮನಿಷಿಯ
ಮನು-ಚರಿತ
ಮನು-ಚರಿತರು
ಮನು-ಚಾರಿತ್ರ್ಯರೂ
ಮನು-ಧರ್ಮ-ಶಾಸ್ತ್ರ
ಮನುಪ್ರತಿಮಂ
ಮನುಬ್ರೋಲು
ಮನು-ಮಾರ್ಗನು
ಮನು-ಮಾರ್ಗಾಗ್ರಣಿ-ಗಳು
ಮನು-ಮುನಿ-ಚರಿ-ತನೂ
ಮನುಷ
ಮನೆ-ಗ-ಳನ್ನೂ
ಮನೆ-ಗಳಲ್ಲಿ
ಮನೆ-ತನ
ಮನೆ-ತನಕ್ಕೆ
ಮನೆ-ತ-ನದ
ಮನೆ-ತನ-ದಲ್ಲಿ
ಮನೆ-ತನ-ದ-ವ-ನಾಗಿದ್ದು
ಮನೆ-ತನ-ದ-ವ-ನೆಂದು
ಮನೆ-ತನ-ದ-ವರು
ಮನೆ-ತನ-ದ-ವ-ರೆಂದು
ಮನೆ-ತನ-ದ-ವಳೇ
ಮನೆ-ತನ-ದಿಂದ
ಮನೆ-ನದ-ವ-ರೆಂದು
ಮನೆ-ಮಗ
ಮನೆಯ
ಮನೆಯ-ಬಲೆ
ಮನೆಯ-ಮಗ
ಮನೆ-ಯಲ್ಲಿ
ಮನೆರ್ದೋಡಿ-ಸುತಂ
ಮನೆ-ವೆಗ್ಗಡೆ
ಮನೆ-ವೆಗ್ಗಡೆ-ಅರಮ-ನೆಯ
ಮನೆವೆರ್ಗ್ಗಡೆ
ಮನೋಜಃ
ಮನೋಜ-ಭಯಂಕರ
ಮನೋಭಾವಕ್ಕೆ
ಮನೋಭಾವವು
ಮನೋಭೀಷ್ಟ
ಮನೋಮಿತ್ರ-ನಾಗಿದ್ದ-ನೆಂದು
ಮನೋರಮೆ-ಪತ್ನಿ
ಮನೋ-ವಲ್ಲಭ
ಮನೋ-ಹರ-ವಾಗಿ
ಮನ್ತ್ರಿ-ಚಾಮುಣ್ಡನ
ಮನ್ನಣೆ
ಮನ್ನಾ
ಮನ್ನಾ-ಮಾಡಲು
ಮನ್ನಾ-ಮಾಡುವ
ಮನ್ನಾ-ರು-ಕೃಷ್ಣಸ್ವಾಮಿ
ಮನ್ನೆಯ
ಮನ್ನೆಯ-ರಿಗೆಲ್ಲಾ
ಮನ್ನೆಯ-ಸೂನು
ಮನ್ನೆ-ಯೊಳಗೆ
ಮನ್ಮಥ-ಪುಷ್ಕರ-ಣಿ-ಗಳನ್ನು
ಮಯಿಲನ-ಹಳ್ಳಿ
ಮಯಿಲನ-ಹಳ್ಳಿ-ಯಲ್ಲಿ
ಮಯಿಸೂರ
ಮಯಿಸೂರು
ಮಯೂರಾ-ಸನ
ಮಯ್ದುನ
ಮಯ್ದುನ-ನಾಗಿದ್ದ-ನೆಂದು
ಮಯ್ದುನನೂ
ಮರ
ಮರಗೆಲಸ-ವೆಲ್ಲಾ
ಮರಡಿ
ಮರಡಿಗೆ
ಮರಡಿ-ಪುರ
ಮರಡಿ-ಪುರವೂ
ಮರಡಿ-ಯೊಳ್
ಮರಣ
ಮರಣ-ಕಾಲ-ದಲ್ಲಿ
ಮರ-ಣದ
ಮರಣ-ವನ್ನಪ್ಪಿ-ದಾಗ
ಮರಣ-ವನ್ನು
ಮರಣ-ವನ್ನೇ
ಮರಣ-ಶಾ-ಸನ-ದಲ್ಲಿ
ಮರಣ-ಹೊಂದಿದ
ಮರಣ-ಹೊಂದಿ-ದನು
ಮರಣ-ಹೊಂದಿ-ದ-ನೆಂದು
ಮರಣ-ಹೊಂದಿ-ದ-ವರ
ಮರಣ-ಹೊಂದಿ-ನೆಂದು
ಮರಣಾ
ಮರಣಾ-ನಂತರ
ಮರದು-ರಾದ
ಮರದೂರ-ಮದ್ದೂರ
ಮರದೂ-ರಾದ-ಮದ್ದೂರು
ಮರಲ-ಹಳ್ಳಿ
ಮರಲ-ಹಳ್ಳಿಯ
ಮರಲೆ
ಮರಳಿ-ಕೆರೆ
ಮರಳಿ-ಬಿಟ್ಟು
ಮರಳಿಸಿ
ಮರ-ವೂರ
ಮರ-ಸೆಯ
ಮರ-ಹಳ್ಳಿ
ಮರಾಠರ
ಮರಾಠ-ರಿಗೆ
ಮರಾಠರು
ಮರಾಠಾ
ಮರಿ-ದೇವ
ಮರಿ-ದೇವ-ರಾಜ-ವಡೆಯ-ನೆಂಬ
ಮರಿ-ಯಣ್ಣನ-ವರ
ಮರಿಯ-ನಾಯ-ಕನ
ಮರಿಯಾನೆ
ಮರಿಯಾನೆ-ಕಿರಿಯ
ಮರಿಯಾನೆಗೆ
ಮರಿಯಾನೆಯ
ಮರಿಯಾನೆಯೂ
ಮರಿಯಾನೆಯೇ
ಮರಿಯಾನೆ-ಸಮುದ್ರದ
ಮರಿಯಾ-ಯನೆ-ಯನ್ನು
ಮರುದ-ಗಾ-ಮುಂಡ
ಮರು-ದೇವಿ-ಯರ
ಮರು-ಪರಿಶೀಲಿಸಬೇಕಾಗುತ್ತದೆ
ಮರುಪರಿಶೀಲಿ-ಸಿದ
ಮರುಳ
ಮರುಳ-ದೇವ
ಮರು-ಳನು
ಮರುಳ-ಸಿದ್ಧ
ಮರು-ವರ್ಮನ
ಮರು-ವರ್ಮ-ನಿಗೂ
ಮರು-ವರ್ಮನು
ಮರೆಯದೆ
ಮರೆವೊಕ್ಕಡೆ-ಕಾವ
ಮರ್ಕುಲಿ
ಮರ್ದ್ದಸ
ಮರ್ದ್ಧನ
ಮರ್ಯಾದೆ
ಮರ್ಯಾದೆ-ಉಂಬಳಿ-ಗವುಡು-ಗೊಡಗೆ
ಮರ್ಯಾದೆ-ಗಾಗಿ
ಮರ್ಯಾದೆಯ
ಮರ್ಯಾದೆ-ಯನ್ನು
ಮಱೆವೊಕ್ಕರ-ಕಾವರುಂ
ಮಱೆವೊಕ್ಕರೆ
ಮಲ-ಗಿರು-ವ-ವನು
ಮಲ-ತಮ್ಮ
ಮಲ-ತಮ್ಮ-ನಾದ
ಮಲ-ತಮ್ಮ-ನಿದ್ದನು
ಮಲ-ತಮ್ಮ-ಮಲ್ಲಿ-ತಮ್ಮ
ಮಲಯಾ
ಮಲ-ಯಾ-ಳನ
ಮಲ-ಸ-ಹೋದ-ರರ
ಮಲ-ಹ-ಗಳ್ಳಿಮಳ-ಗಳಲಿ
ಮಲಿ-ದೇವ
ಮಲಿ-ಯೂರಿನ
ಮಲಿ-ಯೂರು
ಮಲುಕಬ್ಬೆ-ಪುರ-ಇಂದಿನ
ಮಲು-ನಾಯ-ಕ-ನ-ಹಳ್ಳಿಯ
ಮಲೆ
ಮಲೆ-ನಾಡಿ-ನಲ್ಲಿ
ಮಲೆ-ನಾಡು-ಗಳು
ಮಲೆ-ನಾಡೇಳು
ಮಲೆ-ಪ-ನಾಯಕ
ಮಲೆ-ಪ-ನಾಯ-ಕನು
ಮಲೆ-ಪರ-ಮಲ್ಲ
ಮಲೆ-ಮಹ-ದೇಶ್ವರ
ಮಲೆಯ
ಮಲೆ-ಯ-ಕಡೆಗೆ
ಮಲೆ-ಯ-ನ-ಹಳ್ಳಿ-ಗಳು
ಮಲೆ-ಯ-ನಾಯ-ಕನ
ಮಲೆ-ಯ-ನಾಯ-ಕ-ನ-ಹಳ್ಳಿ
ಮಲೆ-ಯನು
ಮಲೆ-ಯಾಂಡನ್
ಮಲೆ-ಯಾಲಗಮಿಡಿಪಿ
ಮಲೆ-ಯಾಳ
ಮಲೆ-ಯಾಳನ
ಮಲೆ-ಯಾಳರಂ
ಮಲೆ-ರಾಜ-ರಾಜ
ಮಲೆ-ರಾಜ್ಯಕ್ಕೆ
ಮಲೆವ
ಮಲ್ಲ
ಮಲ್ಲ-ಗವುಡ
ಮಲ್ಲ-ಘಟ್ಟ
ಮಲ್ಲ-ಘಟ್ಟದ
ಮಲ್ಲ-ಘಟ್ಟ-ದಲ್ಲಿ
ಮಲ್ಲ-ಜೀಯ-ನಿಗೆ
ಮಲ್ಲ-ನಾಯಕ
ಮಲ್ಲ-ನಾಯ-ಕನ
ಮಲ್ಲನು
ಮಲ್ಲಯ್ಯ
ಮಲ್ಲಯ್ಯ-ನ-ಹಳ್ಳಿ
ಮಲ್ಲ-ರಸ
ಮಲ್ಲ-ರಸನೂ
ಮಲ್ಲ-ರಾಜ
ಮಲ್ಲ-ರಾಜ-ಗಳೆಂಬ
ಮಲ್ಲ-ರಾಜನ
ಮಲ್ಲ-ರಾಜ-ನೆಂಬ
ಮಲ್ಲ-ರಾಜ-ನೆಂಬು-ದಾಗಿ
ಮಲ್ಲ-ರಾಜ-ವೊಡ-ಯನೂ
ಮಲ್ಲ-ರಾಯ
ಮಲ್ಲ-ರಾಯ-ನೆಂಬ
ಮಲ್ಲ-ಸೆಟ್ಟಿ-ಯಾದ
ಮಲ್ಲಾಂಬಿಕೆ
ಮಲ್ಲಾಂಬಿಕೆಗೆ
ಮಲ್ಲಾಂಬಿ-ಕೆಯ
ಮಲ್ಲಾ-ದೇವಿ-ಯಿಂದ
ಮಲ್ಲಾ-ಪುರ
ಮಲ್ಲಿ-ಕಾರ್ಜುನ
ಮಲ್ಲಿ-ಕಾರ್ಜುನ-ದೇವ-ರಿಗೆ
ಮಲ್ಲಿ-ಕಾರ್ಜುನನ
ಮಲ್ಲಿ-ಕಾರ್ಜುನ-ನ-ವ-ರೆಗೆ
ಮಲ್ಲಿ-ಕಾರ್ಜುನ-ನಿಗೆ
ಮಲ್ಲಿ-ಕಾರ್ಜುನನು
ಮಲ್ಲಿ-ಕಾರ್ಜುನನ್ನು
ಮಲ್ಲಿ-ಕಾರ್ಜುನ-ರಾಯನು
ಮಲ್ಲಿ-ಕಾರ್ಜುನೋ
ಮಲ್ಲಿಕ್
ಮಲ್ಲಿಕ್ಕಾಫುರ-ನಿಂದ
ಮಲ್ಲಿಗೆರೆ
ಮಲ್ಲಿ-ತಮ್ಮ
ಮಲ್ಲಿ-ದೇವ
ಮಲ್ಲಿ-ದೇವನ
ಮಲ್ಲಿ-ನಾಥ
ಮಲ್ಲಿ-ನಾಥ-ದೇವರ
ಮಲ್ಲಿ-ನಾಥನ
ಮಲ್ಲಿ-ನಾಥನು
ಮಲ್ಲಿಯಣ
ಮಲ್ಲಿ-ಯಣ್ಣ
ಮಲ್ಲಿ-ಯಣ್ಣನು
ಮಲ್ಲೆ-ನಾಯಕ
ಮಲ್ಲೆ-ನಾಯ-ಕನ
ಮಲ್ಲೆ-ನಾಯ-ಕ-ನಿಗೆ
ಮಲ್ಲೆ-ನಾಯ-ಕರು
ಮಲ್ಲೆಯ
ಮಲ್ಲೆಯ-ನಾಯಕ
ಮಲ್ಲೆಯ-ನಾಯ-ಕನ
ಮಲ್ಲೆಯ-ನಾಯ-ಕ-ನಿಗೆ
ಮಲ್ಲೆ-ಸಾಮಂತನು
ಮಲ್ಲೇನ-ಹಳ್ಳಿ
ಮಲ್ಲೇಶ್ವರ
ಮಳಲಿಯ
ಮಳಲೂ-ರನ್ನು
ಮಳ-ಲೂರು
ಮಳ-ವಳ್ಳಿ
ಮಳ-ವಳ್ಳಿಯ
ಮಳ-ವಳ್ಳಿ-ಯನ್ನು
ಮಳ-ವಳ್ಳಿ-ಯಲ್ಲಿ
ಮಳ-ವಳ್ಳಿಯು
ಮಳೂರಿನಲ್ಲಿಯೂ
ಮಳೂರು
ಮಳೂರು-ಪಟ್ಟಣ
ಮಸಅಣಯ್ಯನು
ಮಸಣಯ್ಯನ
ಮಸಣಯ್ಯ-ನನ್ನು
ಮಸಣಯ್ಯ-ನಿಗೆ
ಮಸಣಯ್ಯ-ನೆಂಬು-ವ-ವ-ನನ್ನು
ಮಸಣಿ-ತಂಮನ
ಮಸಣೆ-ನಾಯಕ
ಮಸ-ಣೆಯ
ಮಸಣೈಯ
ಮಸಣೈ-ಯನ
ಮಸಣೈ-ಯನು
ಮಸೀದಿಗೆ
ಮಸೀದಿ-ಯನ್ನು
ಮಸುನಿ-ದೇಶ
ಮಸ್ಜಿದ್
ಮಸ್ತಕಶೂಲ
ಮಸ್ತಕಸೂಲ
ಮಸ್ತಕಾಸೂಲ
ಮಹತ್ತರ
ಮಹತ್ವ
ಮಹತ್ವದ
ಮಹತ್ವದ್ದಾಗಿದೆ
ಮಹತ್ವ-ಪೂರ್ಣ
ಮಹತ್ವ-ವಾದ
ಮಹತ್ವಾಕಾಂಕ್ಷಿ-ಯಾದ
ಮಹತ್ವಾಕಾಂಕ್ಷಿಯೂ
ಮಹತ್ವಾಕಾಂಕ್ಷೆಗೆ
ಮಹ-ದೇವ
ಮಹ-ದೇವಣ್ಣ
ಮಹ-ದೇವಣ್ಣನ
ಮಹ-ದೇವಣ್ಣ-ನಿಂದ
ಮಹ-ದೇವಣ್ಣನು
ಮಹ-ದೇವಣ್ಣನೇ
ಮಹ-ದೇವ-ದಂಡ-ನಾಯ-ಕ-ನನ್ನು
ಮಹ-ದೇವನ
ಮಹ-ದೇವ-ನನ್ನು
ಮಹ-ದೇವ-ನಾಯ-ಕ-ನನ್ನು
ಮಹ-ದೇವ-ನಾಯ-ಕ-ನೆಂಬ
ಮಹ-ದೇವ-ನಿಗೆ
ಮಹ-ದೇವನು
ಮಹ-ದೇವ-ಪುರದ
ಮಹ-ದೇವ-ರಾಣೆ
ಮಹ-ದೇವ-ರಾಣೆಯಂ
ಮಹ-ದೇವ-ರಾಣೆಯಿಂ
ಮಹ-ದೇವ-ವನೀ
ಮಹನೀಯ
ಮಹಪ್ರಧಾನ-ರೆಂದೂ
ಮಹ-ಮಂಡ-ಲೇಶ್ವರರು
ಮಹಮದೀ-ಯರ
ಮಹಮದೀಯ-ರಿಗೆ
ಮಹಮದ್ರಿಜಾ
ಮಹಮ್ಮದ್
ಮಹಲು
ಮಹಾ
ಮಹಾಂಗಮಂತ್ರ
ಮಹಾಂಡ-ಳೇಶ್ವರರ
ಮಹಾಂಭೋದಿ
ಮಹಾ-ಅ-ರಸ
ಮಹಾ-ಅ-ರಸನ
ಮಹಾ-ಅ-ರಸ-ನಾಗಿದ್ದಾನೆ
ಮಹಾ-ಅ-ರಸ-ನಿಗೂ
ಮಹಾ-ಅ-ರಸ-ನಿಗೆ
ಮಹಾ-ಅ-ರಸನು
ಮಹಾ-ಅ-ರಸರು
ಮಹಾ-ಅ-ರಸು
ಮಹಾ-ಅ-ರಸು-ಕೊನೇಟಿ-ರಾಜ
ಮಹಾ-ಅ-ರಸು-ಗಳ
ಮಹಾ-ಅ-ರಸು-ಗಳ-ವರು
ಮಹಾ-ಅ-ರಸು-ಗಳು
ಮಹಾ-ಅ-ರಸು-ರಾಮ-ರಾಜಯ್ಯ-ದೇವ
ಮಹಾ-ಕರ್ಣಾಟ
ಮಹಾ-ಕಾಳಿ
ಮಹಾಕ್ರತು-ಗಳನ್ನು
ಮಹಾ-ಜನಂಗಳು
ಮಹಾ-ಜನಂಗಳ್ಅ
ಮಹಾ-ಜನ-ಗಳ
ಮಹಾ-ಜನ-ಗಳಿಂದ
ಮಹಾ-ಜನ-ಗ-ಳಿಗೆ
ಮಹಾ-ಜನ-ಗಳು
ಮಹಾ-ಜನ-ರಂತೆ
ಮಹಾ-ಜನ-ರನ್ನು
ಮಹಾ-ಜನ-ರಿಂದ
ಮಹಾ-ಜನ-ರಿಗೆ
ಮಹಾ-ಜನರು
ಮಹಾ-ಜನ-ರು-ಗಳು
ಮಹಾ-ತಟಾ-ಕಾದಿ
ಮಹಾ-ದಂಡ-ನಾಯಕ
ಮಹಾ-ದಂಡಾಧಿ-ಕಾರಿ
ಮಹಾ-ದಾನ
ಮಹಾ-ದಾನದ
ಮಹಾ-ದಾನ-ದೊಳ್
ಮಹಾ-ದಾನ-ವನ್ನು
ಮಹಾ-ದೇವ
ಮಹಾ-ದೇವನ
ಮಹಾ-ದೇವನು
ಮಹಾ-ದೇವ-ನೆಂಬ
ಮಹಾ-ದೇವರ
ಮಹಾ-ದೇವ-ರ-ಇಂದಿನ
ಮಹಾ-ದೇವ-ರಿಗೆ
ಮಹಾ-ದೇವ-ಶಕ್ತಿ
ಮಹಾ-ದೇವಿ
ಮಹಾ-ದೇವಿಯ
ಮಹಾ-ದೇವಿ-ಯರ
ಮಹಾ-ದೇವಿಯು
ಮಹಾ-ದೇ-ವೋತ್ತಮ
ಮಹಾ-ದೇಸಿ-ಗರು
ಮಹಾ-ನಾಡು
ಮಹಾ-ನಾಯಂಕಾಚಾರ್ಯ
ಮಹಾ-ನಾಯಕ
ಮಹಾ-ನಾಯ-ಕರು
ಮಹಾ-ನಾಯ-ಕಾಚಾರ್ಯ
ಮಹಾ-ನುಭಾ-ವನ
ಮಹಾ-ನುಭಾ-ವನಿಂ
ಮಹಾ-ಪಸಾಯತ
ಮಹಾ-ಪಸಾಯಿತ
ಮಹಾ-ಪಸಾಯಿತರೂ
ಮಹಾ-ಪಸಾಯ್ತ
ಮಹಾ-ಪಸಾಯ್ತರು
ಮಹಾ-ಪಸಾಯ್ತ-ರು-ಪಸಾಯಿತರು
ಮಹಾ-ಪಸಾಯ್ತರೂ
ಮಹಾಪ್ರಚಂಡ
ಮಹಾಪ್ರಚಂಡ-ದಂಡ-ನಾಯಕ
ಮಹಾಪ್ರಚಂಡ-ದಂಡ-ನಾಯ-ಕರ
ಮಹಾಪ್ರ-ದಾನ
ಮಹಾಪ್ರಧನ
ಮಹಾಪ್ರಧಾನ
ಮಹಾಪ್ರಧಾನಂ
ಮಹಾಪ್ರಧಾನ-ದಂಡ-ನಾಯ-ಕನೇ
ಮಹಾಪ್ರಧಾನನ
ಮಹಾಪ್ರಧಾನ-ನಾಗಿ
ಮಹಾಪ್ರಧಾನ-ನಾಗಿದ್ದ
ಮಹಾಪ್ರಧಾನ-ನಾಗಿದ್ದ-ನೆಂದು
ಮಹಾಪ್ರಧಾನ-ನಾದ-ನೆಂದು
ಮಹಾಪ್ರಧಾನನು
ಮಹಾಪ್ರಧಾನನೂ
ಮಹಾಪ್ರಧಾನರ
ಮಹಾಪ್ರಧಾನ-ರನ್ನು
ಮಹಾಪ್ರಧಾನ-ರಾಗಿ
ಮಹಾಪ್ರಧಾನ-ರಾಗಿದ್ದರೂ
ಮಹಾಪ್ರಧಾನ-ರಾದ
ಮಹಾಪ್ರಧಾನ-ರಿಗೆ
ಮಹಾಪ್ರಧಾನರು
ಮಹಾಪ್ರಧಾನ-ರೆಂದು
ಮಹಾಪ್ರಧಾನ-ರೆನಿಸಿ-ಕೊಂಡಿದ್ದ-ರೆಂದು
ಮಹಾಪ್ರಧಾನ-ರೆನಿಸಿದ್ದ-ರೆಂದು
ಮಹಾಪ್ರಧಾನರೇ
ಮಹಾಪ್ರಧಾನಿ
ಮಹಾಪ್ರಧಾನಿ-ಗಳ
ಮಹಾಪ್ರಧಾನೂ
ಮಹಾಪ್ರಭು
ಮಹಾಪ್ರಭು-ಗಳಾಗಿ
ಮಹಾಪ್ರಭು-ಗಳು
ಮಹಾಪ್ರಭುಪ್ರಭು-ವಿಭು-ಗಳು
ಮಹಾಪ್ರಭು-ವಾಗಿದ್ದ-ನೆಂದು
ಮಹಾಪ್ರಭು-ವಾಗಿ-ರ-ಬಹುದು
ಮಹಾಪ್ರಭು-ವಿನ
ಮಹಾಪ್ರಭುವು
ಮಹಾಪ್ರಭೋತ್ತ-ಮೂರ್ತಿ-ರನೇಕ
ಮಹಾ-ಬ-ಲೇಶ್ವರ
ಮಹಾ-ಬ-ಲೇಶ್ವರ-ನಿಗೆ
ಮಹಾ-ಬಿರುದ
ಮಹಾ-ಭಾರ-ತದ
ಮಹಾ-ಭಾರ-ತ-ದಲ್ಲಿ
ಮಹಾ-ಭಾರ-ತ-ವನ್ನು
ಮಹಾ-ಭೂತ
ಮಹಾ-ಮಂಡ-ಲಿಕ
ಮಹಾ-ಮಂಡ-ಲೇಶ್ವರ
ಮಹಾ-ಮಂಡ-ಲೇಶ್ವರಃ
ಮಹಾ-ಮಂಡ-ಲೇಶ್ವರ-ನನ್ನಾಗಿ
ಮಹಾ-ಮಂಡ-ಲೇಶ್ವರ-ನಾಗಿ
ಮಹಾ-ಮಂಡ-ಲೇಶ್ವರ-ನಾಗಿದ್ದ
ಮಹಾ-ಮಂಡ-ಲೇಶ್ವರ-ನಾಗಿದ್ದು
ಮಹಾ-ಮಂಡ-ಲೇಶ್ವರ-ನಾದ
ಮಹಾ-ಮಂಡ-ಲೇಶ್ವರ-ನಿರ-ಬಹುದು
ಮಹಾ-ಮಂಡ-ಲೇಶ್ವರನು
ಮಹಾ-ಮಂಡ-ಲೇಶ್ವರರ
ಮಹಾ-ಮಂಡ-ಲೇಶ್ವರ-ರನ್ನು
ಮಹಾ-ಮಂಡ-ಲೇಶ್ವರ-ರಲ್ಲಿ
ಮಹಾ-ಮಂಡ-ಲೇಶ್ವರ-ರಾದ
ಮಹಾ-ಮಂಡ-ಲೇಶ್ವರ-ರಿಂದ
ಮಹಾ-ಮಂಡ-ಲೇಶ್ವರರು
ಮಹಾ-ಮಂಡ-ಲೇಶ್ವರ-ರು-ಮಹಾ-ಸಾಮಂತರು
ಮಹಾ-ಮಂಡ-ಲೇಶ್ವರ-ರೆಂದು
ಮಹಾ-ಮಂಡ-ಲೇಶ್ವರ-ರೆಲ್ಲರೂ
ಮಹಾ-ಮಂಡ-ಲೇಶ್ವರರೇ
ಮಹಾ-ಮಂಡ-ಳಿಕ
ಮಹಾ-ಮಂಡ-ಳೇಶ್ವ-ನಾಗಿ
ಮಹಾ-ಮಂಡ-ಳೇಶ್ವರ
ಮಹಾ-ಮಂಡ-ಳೇಶ್ವರಂ
ಮಹಾ-ಮಂಡ-ಳೇಶ್ವರ-ನಾಗಿ
ಮಹಾ-ಮಂಡ-ಳೇಶ್ವರರ
ಮಹಾ-ಮಂಡ-ಳೇಶ್ವರ-ರಾಗಿದ್ದ
ಮಹಾ-ಮಂಡ-ಳೇಶ್ವರ-ರಾಗಿದ್ದ-ರಿಂದ
ಮಹಾ-ಮಂಡ-ಳೇಶ್ವರ-ರಾಗಿದ್ದರು
ಮಹಾ-ಮಂಡ-ಳೇಶ್ವರ-ರಾದರೂ
ಮಹಾ-ಮಂಡ-ಳೇಶ್ವರರು
ಮಹಾ-ಮಂಡ-ಳೇಶ್ವರ-ರು-ಮಂಡ-ಳೇಶ್ವರರ
ಮಹಾ-ಮಂತ್ರಿ
ಮಹಾ-ಮತ್ಯ-ಪದ
ಮಹಾ-ಮ-ಮಂಡ-ಲೇಶ್ವರರು
ಮಹಾ-ಮ-ಹತ್ತಿನ
ಮಹಾ-ಮಹಿಪಂ
ಮಹಾ-ಮಾತ್ಯ
ಮಹಾ-ಮಾತ್ಯನು
ಮಹಾ-ಮಾತ್ಯ-ಪದವಿ
ಮಹಾ-ಮಾತ್ಯ-ಪದವೀ
ಮಹಾ-ಯಶಾಃ
ಮಹಾ-ರಾಜ
ಮಹಾ-ರಾಜ-ನಾದ
ಮಹಾ-ರಾಜ-ರಿಗೆ
ಮಹಾ-ರಾಜಾಧಿ-ರಾಜ
ಮಹಾ-ರಾಜಾಧಿ-ರಾಜ-ನೆಂದು
ಮಹಾ-ರಾಜ್ಯ
ಮಹಾ-ರಾಜ್ಯಕ್ಕೆ
ಮಹಾ-ರಾಜ್ಯ-ದಲ್ಲಿ
ಮಹಾ-ರಾಣಿ
ಮಹಾ-ರಾಯನ
ಮಹಾ-ರಾಯನು
ಮಹಾ-ರಾಯರ
ಮಹಾ-ರಾಯ-ರಿಗೆ
ಮಹಾ-ರಾಯರು
ಮಹಾ-ರಾಯರೂ
ಮಹಾ-ರಾಯೋ
ಮಹಾ-ರಾಷ್ಟ್ರ-ಕ-ಗಳಾಗಿ
ಮಹಾರ್ನ್ನವ
ಮಹಾ-ಲಕ್ಷ್ಮಿ
ಮಹಾ-ಲಕ್ಷ್ಮಿಯ
ಮಹಾ-ವಡ್ಡವ್ಯವ-ಹಾರಿ
ಮಹಾ-ವಿಭವದಿ
ಮಹಾ-ವೀರ
ಮಹಾ-ಶಾ-ಸನ-ವನ್ನು
ಮಹಾ-ಸಮಾಂತಾಧಿ-ಪತಿ
ಮಹಾ-ಸಮುದ್ರಮಾ
ಮಹಾ-ಸಮುದ್ರ-ವೆಂದು
ಮಹಾ-ಸಾಂತಾಧಿ-ಪತಿ
ಮಹಾ-ಸಾಮಂತ
ಮಹಾ-ಸಾಮಂತ-ನನ್ನಾಗಿ
ಮಹಾ-ಸಾಮಂತ-ನಾಗಿ
ಮಹಾ-ಸಾಮಂತ-ನಾಗಿದ್ದ
ಮಹಾ-ಸಾಮಂತ-ನಾಗಿದ್ದರೂ
ಮಹಾ-ಸಾಮಂತ-ನಾಗಿದ್ದುದು
ಮಹಾ-ಸಾಮಂತ-ನಾಗುತ್ತಿದ್ದನು
ಮಹಾ-ಸಾಮಂತ-ನೆಂದು
ಮಹಾ-ಸಾಮಂತನೋ
ಮಹಾ-ಸಾಮಂತರ
ಮಹಾ-ಸಾಮಂತ-ರನ್ನು
ಮಹಾ-ಸಾಮಂತ-ರಾಗಿ
ಮಹಾ-ಸಾಮಂತ-ರಾಗಿದ್ದ
ಮಹಾ-ಸಾಮಂತ-ರಾಗಿದ್ದ-ರೆಂಬು-ದನ್ನ
ಮಹಾ-ಸಾಮಂತ-ರಿಗೆ
ಮಹಾ-ಸಾಮಂತರು
ಮಹಾ-ಸಾಮಂತಾಧಿ-ಪತಿ
ಮಹಾ-ಸಾಮಂತಾಧಿ-ಪತಿ-ಗಳು
ಮಹಾ-ಸಾಮ್ರಾಜ್ಯ-ವಾ-ಗಿತ್ತು
ಮಹಾ-ಸೇನಾ
ಮಹಾ-ಸೇನಾ-ಸಮುದ್ರ
ಮಹಾ-ಸೋಪಾನ-ವನ್ನು
ಮಹಾಸ್ವಾಮಿ-ಯ-ರಿಗೆ
ಮಹಾಸ್ವಾಮಿ-ಯ-ವ-ರಿಂದ
ಮಹಾಸ್ವಾಮಿ-ಯ-ವರು
ಮಹಾ-ಹೋಸಲ
ಮಹಾ-ಹೋಸಲ-ನಾಡ
ಮಹಾ-ಹೋಸಲ-ನಾಡು
ಮಹಿಪನ
ಮಹಿಮೆ
ಮಹಿಶುರ
ಮಹಿ-ಶೂರ
ಮಹಿ-ಶೂರ-ನ-ಗರದ
ಮಹಿ-ಶೂರ-ನ-ಗರ-ದಲ್ಲಿ
ಮಹಿಷಿ
ಮಹಿಷಿಕಾ
ಮಹಿಷೀ
ಮಹಿಸೂರ
ಮಹೀಂ
ಮಹೀಕ-ರನೂ
ಮಹೀ-ಪಾಲ-ಕರು
ಮಹೀಪಾಳಕಃ
ಮಹೀಪೇ
ಮಹೀಭು-ಜನು-ವಿಷ್ಣು-ವರ್ಧನ
ಮಹೀ-ಮಂಡ-ಳ-ವನ್ನು
ಮಹೀ-ವಲ್ಲಭ
ಮಹೀಶೂರ-ದಳವಾಯಿ
ಮಹೂರ್ತ-ದಲ್ಲಿ
ಮಹೂರ್ತ-ವನ್ನೂ
ಮಹೇಂದ್ರ
ಮಹೇಂದ್ರ-ನನ್ನು
ಮಹೇಂದ್ರನು
ಮಹೇಂದ್ರನೇ
ಮಹೇಶ್ವರಿ
ಮಹೋಗ್ರಾಜಿಯೊಳಾಂತಿದಿರ್ಚಿದದಟಿಂ
ಮಹೋಗ್ರಾಜಿಯೊಳಾಂತಿದಿರ್ಚ್ಚಿದದಟಿಂ
ಮಹೋತ್ತಮ-ನಾಗಿದ್ದನು
ಮಹೋತ್ಸವವ-ಅನ್ನು
ಮಹೋತ್ಸವ-ವಾ-ಯಿತೆಂದು
ಮಾ
ಮಾಂಡ-ಲಿಕ
ಮಾಂಡ-ಲಿಕ-ನಾಗಿ
ಮಾಂಡ-ಲಿಕ-ನಾಗಿದ್ದನು
ಮಾಂಡ-ಲಿಕ-ನಾದ
ಮಾಂಡಲಿ-ಕನೂ
ಮಾಂಡ-ಲಿಕರ
ಮಾಂಡ-ಲಿಕ-ರನ್ನಾಗಿ
ಮಾಂಡ-ಲಿಕ-ರಾಗಿ
ಮಾಂಡ-ಲಿಕ-ರಾಗಿದ್ದ
ಮಾಂಡ-ಲಿಕ-ರಾಗಿದ್ದ-ರೆಂದು
ಮಾಂಡ-ಲಿಕ-ರಾಗಿದ್ದ-ರೆಂಬುದು
ಮಾಂಡ-ಲಿಕರು
ಮಾಂಡ-ಲಿಕ-ರೊಡನೆ
ಮಾಂಡ-ಲೀಕ-ನಾಗಿ
ಮಾಂಡಲೀ-ಕನೂ
ಮಾಂಡ-ಲೀಕ-ರಾಗಿ
ಮಾಂಡ-ಲೀಕ-ರಾಗಿದ್ದ
ಮಾಂಡಲೀ-ಕರು
ಮಾಂಡವ್ಯ
ಮಾಂಬಳ್ಳಿ
ಮಾಕಣಬ್ಬೆ
ಮಾಕಣಬ್ಬೆ-ಯರ
ಮಾಕಲೆ
ಮಾಕಲೆ-ಯರ
ಮಾಕುಂದಮುಕುಂದ
ಮಾಕು-ಬಳ್ಳಿ-ಮಾಕ-ವಳ್ಳಿ
ಮಾಕೇಶ್ವರ
ಮಾಗಡಿ
ಮಾಗಡಿಯ
ಮಾಗಣಿ
ಮಾಗಣಿಗೆ
ಮಾಗಣಿಯ
ಮಾಗಣಿ-ಯಾ-ಗಿತ್ತು
ಮಾಗಣಿ-ಯೊಳಗೆ
ಮಾಗಣೆಗೆ
ಮಾಗ-ಣೆಯ
ಮಾಗ-ನೂರು
ಮಾಗಲ
ಮಾಗಳಿ
ಮಾಗುಂಡರಕಿಲ್ಲ
ಮಾಘ
ಮಾಘ-ಣಂದಿ
ಮಾಚ-ಗವುಡ
ಮಾಚ-ಗೌಂಡ
ಮಾಚಣ
ಮಾಚ-ಣನ
ಮಾಚ-ದಂಡಾಧೀಶನು
ಮಾಚನ
ಮಾಚ-ನ-ಕಟ್ಟ
ಮಾಚ-ನ-ಕಟ್ಟಕ್ಕೆ
ಮಾಚ-ನ-ಕಟ್ಟದ
ಮಾಚ-ನ-ಕಟ್ಟ-ದಲ್ಲಿ
ಮಾಚ-ನ-ಕಟ್ಟೆಯ
ಮಾಚನು
ಮಾಚ-ಮಯ್ಯ
ಮಾಚ-ಮಯ್ಯನು
ಮಾಚಯ್ಯ
ಮಾಚಯ್ಯ-ದಂಡ-ನಾಯಕ
ಮಾಚಯ್ಯನ
ಮಾಚಯ್ಯ-ನನ್ನು
ಮಾಚಯ್ಯ-ನಿಗೆ
ಮಾಚಯ್ಯನು
ಮಾಚ-ಲಗಅಟ್ಟ
ಮಾಚ-ಲ-ರಾಣಿ
ಮಾಚಲೆ
ಮಾಚ-ಲೆ-ನಾರಿ
ಮಾಚ-ಳೇಶ್ವರ
ಮಾಚ-ವಳಲು
ಮಾಚವ್ವೆ
ಮಾಚವ್ವೆ-ಯನ್ನೂ
ಮಾಚ-ಸಮುದ್ರ
ಮಾಚಿಕೆ
ಮಾಚಿಗೆ-ಹಳ್ಳಿ-ಮಾಚ-ನ-ಹಳ್ಳಿ
ಮಾಚಿ-ದೇವ
ಮಾಚಿ-ದೇವನ
ಮಾಚಿ-ದೇವನು
ಮಾಚಿ-ನಾಯ-ಕ-ನ-ಹಳ್ಳಿ
ಮಾಚಿ-ಮಯ್ಯನು
ಮಾಚಿ-ರಾಜ
ಮಾಚಿ-ರಾಜಂ
ಮಾಚಿ-ರಾಜಂಗೆ
ಮಾಚಿ-ರಾಜನ
ಮಾಚಿ-ರಾಜ-ನನ್ನು
ಮಾಚಿ-ರಾಜನು
ಮಾಚಿ-ರಾಜನೂ
ಮಾಚಿ-ರಾಜ-ರಾಗಿದ್ದಾರೆಂದು
ಮಾಚಿ-ರಾಜ-ರೆಲ್ಲರೂ
ಮಾಚೀ-ದೇವ
ಮಾಚೀ-ದೇವನ
ಮಾಚೆ-ಗವುಡ
ಮಾಚೆ-ಗೌಡನ
ಮಾಚೆಯ-ನಾಯಕ
ಮಾಚೆಯ-ನಾಯ-ಕನ
ಮಾಚೆಯ-ನಾಯ-ಕ-ನಿಗೆ
ಮಾಚೆಯ-ನಾಯ-ಕನು
ಮಾಚೆಯ-ನಾಯ-ಕ-ನೆಂಬ
ಮಾಚೆ-ಯನು
ಮಾಚೋ-ಜನ
ಮಾಡ-ತಕ್ಕ
ಮಾಡ-ತೊಡಗಿ-ದರು
ಮಾಡದೇ
ಮಾಡ-ಬಹುದು
ಮಾಡ-ಬೇಕಾದ
ಮಾಡಲ
ಮಾಡ-ಲಾಗಿದೆ
ಮಾಡ-ಲಾಗಿ-ದೆ-ಯೆಂದು
ಮಾಡಲಾಯಿತಾ-ದರೂ
ಮಾಡ-ಲಾಯಿತು
ಮಾಡ-ಲಾ-ಯಿತೆಂದು
ಮಾಡ-ಲಿಲ್ಲ
ಮಾಡಲು
ಮಾಡಲೋಸುಗ
ಮಾಡಲ್ಪಟ್ಟ
ಮಾಡವ
ಮಾಡಿ
ಮಾಡಿ-ಕೊಂಡ
ಮಾಡಿ-ಕೊಂಡನು
ಮಾಡಿ-ಕೊಂಡ-ನೆಂದು
ಮಾಡಿ-ಕೊಂಡರು
ಮಾಡಿ-ಕೊಂಡ-ರೆಂದು
ಮಾಡಿ-ಕೊಂಡಿರ-ಬಹುದು
ಮಾಡಿ-ಕೊಂಡಿ-ರುವ-ವರ
ಮಾಡಿ-ಕೊಂಡಿ-ರು-ವು-ದಿಲ್ಲ
ಮಾಡಿ-ಕೊಂಡು
ಮಾಡಿ-ಕೊಟ್ಟ-ನೆಂದೂ
ಮಾಡಿ-ಕೊಟ್ಟಿದ್ದನು
ಮಾಡಿ-ಕೊಡಲಾ-ಗಿತ್ತು
ಮಾಡಿ-ಕೊಳ್ಳುತ್ತಾರೆ
ಮಾಡಿದ
ಮಾಡಿದಂ
ಮಾಡಿದಂತೆ
ಮಾಡಿದಂಥಾ
ಮಾಡಿದನು
ಮಾಡಿದ-ನೆಂದು
ಮಾಡಿದ-ನೆಂದೂ
ಮಾಡಿದರು
ಮಾಡಿದ-ರೆಂದಿದೆ
ಮಾಡಿದ-ರೆಂದು
ಮಾಡಿದ-ರೆಂದೂ
ಮಾಡಿದಲ್ಲಿ
ಮಾಡಿದ-ವನು
ಮಾಡಿದ-ವ-ನೆಂದರೆ
ಮಾಡಿದಾಗ
ಮಾಡಿದು-ದಕ್ಕಾಗಿ
ಮಾಡಿದು-ದರ
ಮಾಡಿದೆ
ಮಾಡಿದ್ದ
ಮಾಡಿದ್ದಕ್ಕಾಗಿ
ಮಾಡಿದ್ದ-ನೆಂದು
ಮಾಡಿದ್ದನ್ನು
ಮಾಡಿದ್ದನ್ನು-ಹೇಳಿದೆ
ಮಾಡಿದ್ದ-ರಿಂದ
ಮಾಡಿದ್ದ-ರೆಂದು
ಮಾಡಿದ್ದಾನೆ
ಮಾಡಿದ್ದಾರೆ
ಮಾಡಿಯೇ
ಮಾಡಿ-ರ-ಬಹುದು
ಮಾಡಿ-ರುವ
ಮಾಡಿ-ರು-ವುದು
ಮಾಡಿಸಿ
ಮಾಡಿ-ಸಿ-ಕೊಟಿ-ರುವ
ಮಾಡಿ-ಸಿ-ಕೊಟ್ಟಂತೆ
ಮಾಡಿ-ಸಿ-ಕೊಟ್ಟನು
ಮಾಡಿ-ಸಿ-ಕೊಟ್ಟರು
ಮಾಡಿ-ಸಿ-ಕೊಟ್ಟ-ರೆಂದು
ಮಾಡಿ-ಸಿ-ಕೊಟ್ಟಳು
ಮಾಡಿ-ಸಿ-ಕೊಟ್ಟಿದ್ದಾನೆ
ಮಾಡಿ-ಸಿದ
ಮಾಡಿ-ಸಿದಂ
ಮಾಡಿ-ಸಿ-ದಂತೆ
ಮಾಡಿ-ಸಿ-ದನ
ಮಾಡಿ-ಸಿ-ದ-ನಿನ್ತೀ
ಮಾಡಿ-ಸಿ-ದನು
ಮಾಡಿ-ಸಿ-ದ-ನೆಂದಿದೆ
ಮಾಡಿ-ಸಿ-ದ-ನೆಂದು
ಮಾಡಿ-ಸಿ-ದರು
ಮಾಡಿ-ಸಿ-ದ-ರೆಂದು
ಮಾಡಿ-ಸಿ-ದಳು
ಮಾಡಿ-ಸಿ-ದಾಗ
ಮಾಡಿ-ಸಿ-ದು-ದರ
ಮಾಡಿ-ಸಿದ್ದ-ನಂತೆ
ಮಾಡಿ-ಸಿದ್ದಾರೆ
ಮಾಡಿ-ಸುತ್ತಾರೆ
ಮಾಡಿ-ಸುತ್ತಿದ್ದಾಗ
ಮಾಡು
ಮಾಡುತ್ತಾ
ಮಾಡುತ್ತಾನೆ
ಮಾಡುತ್ತಾರೆ
ಮಾಡುತ್ತಿ-ದ-ರೆಂದು
ಮಾಡುತ್ತಿದೆ
ಮಾಡುತ್ತಿದ್ದಂತೆ
ಮಾಡುತ್ತಿದ್ದನು
ಮಾಡುತ್ತಿದ್ದ-ನೆಂದು
ಮಾಡುತ್ತಿದ್ದರು
ಮಾಡುತ್ತಿದ್ದ-ರೆಂದು
ಮಾಡುತ್ತಿದ್ದ-ರೆಂದೂ
ಮಾಡುತ್ತಿದ್ದಾಗ
ಮಾಡುತ್ತಿದ್ದು-ದಕ್ಕಾಗಿ
ಮಾಡುತ್ತಿದ್ದುದು
ಮಾಡುತ್ತಿ-ರಲು
ಮಾಡುತ್ತೇ-ನೆಂದು
ಮಾಡುವ
ಮಾಡು-ವಂತಹ
ಮಾಡು-ವಂತೆ
ಮಾಡು-ವಲ್ಲಿ
ಮಾಡು-ವ-ವ-ರನ್ನು
ಮಾಡು-ವಾಗ
ಮಾಡು-ವಾಗಲೂ
ಮಾಡು-ವುದ-ರಿಂದ
ಮಾಡು-ವುದು
ಮಾಣದ
ಮಾಣಿ
ಮಾಣಿಕ
ಮಾಣಿ-ಕ-ಭಂಡಾರಿ-ಗಳಾಗಿ-ರುತ್ತಿದ್ದ-ರೆಂದು
ಮಾಣಿ-ಕ-ಭಂಡಾರಿ-ಗಳು
ಮಾಣಿಕ್ಯ
ಮಾಣಿಕ್ಯ-ಭಂಡಾರದ
ಮಾಣಿಕ್ಯ-ಭಂಡಾರಿ
ಮಾಣಿಕ್ಯ-ವೊಳಲ
ಮಾಣಿಕ್ಯ-ವೊಳಲು
ಮಾಣಿಕ್ಯ-ವೊಳಲೆಂಬ
ಮಾಣಿ-ಯೊಳಗಣ
ಮಾತನ್ನು
ಮಾತ್ರ
ಮಾತ್ರಕ್ಕೆ
ಮಾದ-ಗವುಡಿಯ
ಮಾದಣ್ಣ
ಮಾದಣ್ಣ-ನಿಗೆ
ಮಾದಣ್ಣನು
ಮಾದಪ್ಪ
ಮಾದಪ್ಪ-ದಂಡ-ನಾಯ-ಕನೂ
ಮಾದಪ್ಪ-ದಣ್ನಾಯ-ಕರ
ಮಾದಪ್ಪನು
ಮಾದರ-ಗವುಡಿ
ಮಾದಲ-ಗೆರೆ
ಮಾದಲಮಹ-ದೇವಿ-ಯರು
ಮಾದಲ-ಮಹಾ-ದೇವಿ
ಮಾದಲ-ಮಹಾ-ದೇವಿ-ಯರು
ಮಾದಳ
ಮಾದಾ-ಪುರ
ಮಾದಿ-ಗಉಡ
ಮಾದಿ-ಗರುಳ
ಮಾದಿ-ಗವುಡನ
ಮಾದಿ-ಗವುಡ-ನಿಗೆ
ಮಾದಿ-ಗೌಡ
ಮಾದಿ-ದೇವ
ಮಾದಿ-ದೇವನ
ಮಾದಿ-ಯಣ್ಣ
ಮಾದಿ-ರಾಜ
ಮಾದಿ-ರಾಜನ
ಮಾದಿ-ರಾಜ-ನನ್ನು
ಮಾದಿ-ರಾಜನು
ಮಾದಿ-ವೆಗ್ಗಡೆ
ಮಾದಿ-ವೆಗ್ಗಡೆಯು
ಮಾದಿ-ಹಳ್ಳಿ
ಮಾದಿ-ಹಳ್ಳಿಯ
ಮಾದಿ-ಹಳ್ಳಿ-ಯನ್ನು
ಮಾದೆಯ
ಮಾದೆಯ-ನಾಯಕ
ಮಾದೆಯ-ನಾಯ-ಕನು
ಮಾದೆ-ಹಳ್ಳಿ
ಮಾದೇ-ಗೌಡನ
ಮಾದೇವ
ಮಾಧವ
ಮಾಧವ-ಚೋಳ-ನ-ಹಳ್ಳಿಯ
ಮಾಧವ-ಚೋಳ-ಯನ-ಹಳ್ಳಿಯ
ಮಾಧವತ್ತಿ-ಮಾಧವ-ಶಕ್ತಿ
ಮಾಧವ-ದಂಡ-ನಾಯಕ
ಮಾಧವ-ದಂಡ-ನಾಯ-ಕನ
ಮಾಧವ-ದಂಡ-ನಾಯ-ಕ-ನಿಗೆ
ಮಾಧವ-ದಂಣಾಯಕರುಂ
ಮಾಧವ-ದೇವರ
ಮಾಧವ-ದೇವ-ರಿಗೆ
ಮಾಧವ-ನನ್ನು
ಮಾಧವ-ನಿಗೆ
ಮಾಧ-ವನು
ಮಾಧ್ಯಮ-ವನ್ನಾಗಿ
ಮಾಧ್ವರ
ಮಾನ-ಗಳಿಂದ
ಮಾನದ
ಮಾನವ
ಮಾನ-ವ-ದುರ್ಗ-ವನ್ನು-ಮಾನವಿ
ಮಾನ-ವನ
ಮಾನ-ವ-ರೊಳು
ಮಾನ-ವ-ರೊಳ್
ಮಾನ-ವಾ-ಕಾರ
ಮಾನ-ವಾ-ಕಾರ-ವನ್ನು
ಮಾನ-ವಾಗಿದ್ದು
ಮಾನ-ಸ-ರೂಪ-ವಾದುದೋ
ಮಾನಸ್ತಂಭ
ಮಾನಸ್ತಂಭದ
ಮಾನಸ್ಥಂಬ
ಮಾನಿಸ
ಮಾನಿ-ಸೆಟ್ಟಿಗೆ
ಮಾನ್ಯ
ಮಾನ್ಯ-ಖೇಟ-ವನ್ನ
ಮಾನ್ಯ-ಗಳನ್ನು
ಮಾನ್ಯಗ್ರಾಮ-ಗಳಲ್ಲಿ
ಮಾನ್ಯತೆ
ಮಾನ್ಯ-ಪುರ-ದಲ್ಲಿ
ಮಾನ್ಯ-ವನ್ನು
ಮಾನ್ಯ-ವಾಗಿ
ಮಾಬಲಯ್ಯ
ಮಾಬಲಯ್ಯಂ
ಮಾಬಲಯ್ಯನ
ಮಾಬಲಯ್ಯ-ನಿಗೆ
ಮಾಬಲಯ್ಯನು
ಮಾಬಲಯ್ಯ-ನೆಂದೊಗೞದ-ರಾರ್
ಮಾಬಳಯ್ಯ
ಮಾಬಳ್ಳಿ
ಮಾಬ-ಹಳ್ಳಿ
ಮಾಮಲೆ-ದಾರ-ನಾಗಿದ್ದಂತೆ
ಮಾಮಲೆ-ದಾರ್
ಮಾಯ-ಣನ-ಪುರ
ಮಾಯಣ್ಣ
ಮಾಯಣ್ಣನ
ಮಾಯಣ್ಣ-ನಿಗೆ
ಮಾಯಣ್ಣನು
ಮಾಯಣ್ಣನೂ
ಮಾಯಣ್ಣ-ನೆಂಬುವವ-ನಿಗೆ
ಮಾಯಪ್ಪ-ನಿಗೆ
ಮಾಯಪ್ಪ-ಹಳ್ಳಿ-ದೇಪ-ಸಾ-ಗರ
ಮಾಯ-ಸಂದ್ರಕ್ಕೆ
ಮಾಯ-ಸಂದ್ರವೇ
ಮಾಯ-ಸಮುದ್ರ
ಮಾಯ-ಸಮುದ್ರ-ವಾಗಿದೆ
ಮಾಯಿ-ಗೌಡನ
ಮಾಯಿ-ಲಂಗೆ
ಮಾಯಿ-ಲಂಗೆ-ಯಲ್ಲಿ
ಮಾರಗವುಂಡನ
ಮಾರ-ಗಾ-ಮುಂಡ
ಮಾರ-ಗಾ-ಮುಂಡನ
ಮಾರ-ಗೂಳಿ
ಮಾರ-ಗೊಂಡ-ನ-ಹಳ್ಳಿ
ಮಾರ-ಗೊಂಡ-ನ-ಹಳ್ಳಿ-ಯನ್ನು
ಮಾರ-ಗೊಂಡ-ಹಳ್ಳಿ
ಮಾರ-ಗೌಂಡ
ಮಾರ-ಗೌಡ
ಮಾರ-ಗೌಡನ
ಮಾರಣ್ಣ
ಮಾರಣ್ಣನು
ಮಾರಥ
ಮಾರ-ದೇವನ
ಮಾರ-ದೇವನು
ಮಾರ-ನಾಯಕ
ಮಾರ-ನಾಯ-ಕನ
ಮಾರ-ನಾಯ-ಕನು
ಮಾರಪ್ಪ
ಮಾರಯ್ಯ
ಮಾರಯ್ಯನ
ಮಾರವ್ವೆಯ್ವೆ
ಮಾರ-ಸಿಂಗ
ಮಾರ-ಸಿಂಗ-ಗಾವುಂಡನು
ಮಾರ-ಸಿಂಹ
ಮಾರ-ಸಿಂಹ-ದೇವ
ಮಾರ-ಸಿಂಹ-ದೇವನ
ಮಾರ-ಸಿಂಹನ
ಮಾರ-ಸಿಂಹ-ನನ್ನು
ಮಾರ-ಸಿಂಹ-ನಲ್ಲಿ
ಮಾರ-ಸಿಂಹ-ನಲ್ಲೇ
ಮಾರ-ಸಿಂಹ-ನಿಗೆ
ಮಾರ-ಸಿಂಹನು
ಮಾರಾಟ
ಮಾರಾಟಕ್ಕೆ
ಮಾರಾಟ-ಮಾಡುತ್ತಾರೆ
ಮಾರಾಯನ್
ಮಾರಿ-ಕೊಳ್ಳುತ್ತಾರೆ
ಮಾರಿರ-ಬಹುದು
ಮಾರು-ಗ-ನೆಂದು
ಮಾರು-ಗ-ನೆಂಬು-ವ-ವನು
ಮಾರು-ಗನೇ
ಮಾರು-ಗೋನ-ಹಳ್ಳಿ
ಮಾರೂರ
ಮಾರೂರು-ಗಳ
ಮಾರೆಯ
ಮಾರೆಯ-ನಾಯಕ
ಮಾರೆಯ-ನಾಯ-ಕ-ನಿಗೂ
ಮಾರೆಯ-ನಾಯ್ಕನ
ಮಾರೆಯ್ಯ-ನೆಂಬು-ವ-ವನೂ
ಮಾರೆ-ಹಳ್ಳಿ
ಮಾರೆ-ಹಳ್ಳಿಯ
ಮಾರೇ-ಹಳ್ಳಿ
ಮಾರೇ-ಹಳ್ಳಿಯ
ಮಾರ್ಕ್ಕೊಂಡು
ಮಾರ್ಕ್ಕೋಲಭೈರವಂ
ಮಾರ್ಗ
ಮಾರ್ಗ-ದರ್ಶನ
ಮಾರ್ಗ-ವಾಗಿ
ಮಾರ್ಗ್ಗ-ಸಿರ
ಮಾರ್ಚನ-ಹಳ್ಳಿ-ಯನ್ನು
ಮಾರ್ಚ-ಹಳ್ಳಿ-ಯಲ್ಲೂ
ಮಾರ್ಚ್
ಮಾರ್ಚ್ರ
ಮಾರ್ತಾಂಡ
ಮಾರ್ತಾಂಡನುಂ
ಮಾರ್ತಾಂಡ-ನೆಂದು
ಮಾರ್ಪಟ್ಟಿತು
ಮಾರ್ಪಟ್ಟಿತ್ತು
ಮಾರ್ಪಡಿಸಬೇಕಾಗುತ್ತದೆ
ಮಾರ್ಪಡಿಸಿ
ಮಾರ್ಪಾಡಾಗಿದ್ದವು
ಮಾರ್ಪ್ಪಾಡಿ
ಮಾರ್ಪ್ಪೆನಾ-ನೆಂದೀ-ಗಳು
ಮಾಲಗಾ-ರನ-ಹಳ್ಳಿಯ
ಮಾಲನ-ಹಳ್ಳಿ
ಮಾಲೂರು
ಮಾಲೆಯ-ಹಳ್ಳಿ-ವೊಳಗಾದ
ಮಾಲ್ಯದ
ಮಾಳಗುಂದ-ಮಾಳಗೂರು
ಮಾಳಗೂರಿನ
ಮಾಳಗೂರು
ಮಾಳಗೂರೇ
ಮಾಳವ
ಮಾಳವ-ರಾಜ್ಯ
ಮಾಳವ್ವೆ
ಮಾಳವ್ವೆಯ
ಮಾಳಾನ-ಹಳ್ಳಿಯು
ಮಾಳಿಗೆ
ಮಾಳಿಗೆಯ
ಮಾಳಿಗೆ-ಯನ್ನು
ಮಾಳಿಗೆ-ಯಲ್ಲಿ
ಮಾಳಿಗೆ-ಯೂ-ರನ್ನು
ಮಾಳಿಗೆ-ಯೂರಿನ
ಮಾಳುಗಾಳ
ಮಾಳೆಯ
ಮಾಳೆಯ-ನ-ಹಳ್ಳಿ-ಯನ್ನು
ಮಾಳೇನ-ಹಳ್ಳಿ
ಮಾವ
ಮಾವಂ
ಮಾವಂದಿ-ರಾಗಿದ್ದ-ರೆಂದು
ಮಾವಂದಿ-ರೆಂದು
ಮಾವ-ನಂಕ-ಕಾರ
ಮಾವ-ನಾಗುತ್ತಾನೆ
ಮಾವ-ನಾದ
ಮಾವಳ್ಳಿ
ಮಾವಿನ-ಕೆರೆ
ಮಾವಿನ-ಕೆರೆ-ಯನ್ನು
ಮಾವಿನ-ಕೆರೆ-ಯನ್ನೂ
ಮಾವಿನಬನ-ವನ್ನು
ಮಾಸಮದೇಭಾವ-ಳಿಯಂ
ಮಾಸ-ವಾಡಿ
ಮಾಸ-ವೆಗ್ಗಡೆ
ಮಾಸ-ವೆಗ್ಗಡೆ-ಗಳ
ಮಾಸ್ತಮ್ಮನ
ಮಾಸ್ತಿ-ಕಲ್ಲಿನಲ್ಲಿ
ಮಾಸ್ತಿ-ಕಲ್ಲು
ಮಾಹಾ-ಸಾಮನ್ತ
ಮಾಹಿತಿ
ಮಾಹಿತಿ-ಗಳನ್ನು
ಮಾಹಿತಿ-ಗಳಿ-ರುವುದ-ರಿಂದ
ಮಾಹಿತಿ-ಗಳು
ಮಾಹಿತಿ-ಯನ್ನು
ಮಾಹಿತಿ-ಯಿಂದ
ಮಾಹೇಶ್ವರ-ನಾಗಿದ್ದು
ಮಿಂಚಗವುಂಡನ
ಮಿಂದು
ಮಿಕ್ಕದ್ದನ್ನು
ಮಿಗಿ-ಲಾಗಿ
ಮಿಗಿಲಾ-ದುದು
ಮಿಗಿಲೆನಿಪಂ
ಮಿಗೆ
ಮಿತಿ-ಯಲ್ಲಿಯೇ
ಮಿತಿಯೇ
ಮಿತ್ರನಂತಿದ್ದನು
ಮಿತ್ರ-ನಾಗಿದ್ದಿರ-ಬಹುದು
ಮಿತ್ರ-ರಾಜ್ಯ-ವಾದ
ಮಿರಾನ್
ಮಿರು-ಹನ-ಗಣ್ಯ
ಮಿರ್ಲೆ
ಮಿರ್ಲೆ-ಶಾ-ಸನೋಕ್ತ-ನಾಗಿದ್ದಾನೆ
ಮಿಸುಪೆಸೆವ
ಮೀಟರ್
ಮೀನು-ಗಳಿವೆ
ಮೀರಿ-ಸಿದನು
ಮೀರ್
ಮೀರ್ಇ-ಮಿರಾನ್
ಮೀರ್ಜೈನ್ಉನ್
ಮೀರ್ಮಹಮದ್
ಮೀಸರ-ಗಂಡ
ಮೀಸ-ಲಾಗಿಟ್ಟು
ಮೀಸಲು
ಮುಂಗೊಳ
ಮುಂಚೆ
ಮುಂಚೆಯೇ
ಮುಂಜಿ-ಯನೂ
ಮುಂಡಿಗೈ
ಮುಂತಾಗಿ
ಮುಂತಾದ
ಮುಂತಾದ-ವ-ರನ್ನು
ಮುಂತಾದ-ವರು
ಮುಂತಾ-ದವು
ಮುಂತಾದ-ವು-ಗಳನ್ನು
ಮುಂತಾದ-ವು-ಗಳು
ಮುಂತಿದಿರಾಂತ-ನಂತರಿಪು
ಮುಂದಕ್ಕೆ
ಮುಂದಣ
ಮುಂದಾಗುತ್ತಿದ್ದರು
ಮುಂದಿಟ್ಟ-ಕೊಂಡು
ಮುಂದಿಟ್ಟು-ಕೊಂಡು
ಮುಂದಿದೆ
ಮುಂದಿನ
ಮುಂದಿ-ನಂತೆ
ಮುಂದಿರೆ
ಮುಂದುರಿಯಿತು
ಮುಂದು-ವರಿದ
ಮುಂದು-ವರಿ-ದನು
ಮುಂದು-ವರಿ-ದರೂ
ಮುಂದು-ವರಿ-ದವು
ಮುಂದು-ವರಿ-ದಿತ್ತು
ಮುಂದು-ವರಿ-ದಿತ್ತೆಂದು
ಮುಂದು-ವರಿ-ದಿದೆ
ಮುಂದು-ವರಿ-ದಿದ್ದ-ನೆಂದು
ಮುಂದು-ವರಿ-ದಿದ್ದರೆ
ಮುಂದು-ವರಿ-ದಿದ್ದಾನೆ
ಮುಂದು-ವರಿ-ದಿದ್ದು
ಮುಂದು-ವರಿ-ದಿರ-ಬಹುದು
ಮುಂದು-ವರಿ-ದಿರ-ಬಹು-ದೆಂದು
ಮುಂದು-ವರಿ-ದಿ-ರುವ
ಮುಂದು-ವರಿ-ದಿ-ರು-ವುದು
ಮುಂದುವ-ರಿದು
ಮುಂದು-ವರಿ-ಯಿತು
ಮುಂದು-ವರಿ-ಯಿತೆಂದು
ಮುಂದು-ವರಿ-ಯುತ್ತದೆ
ಮುಂದು-ವರಿ-ಸ-ಲಾ-ಯಿತೆಂದು
ಮುಂದು-ವರಿ-ಸಲು
ಮುಂದು-ವರಿಸಿ
ಮುಂದು-ವರಿ-ಸಿ-ಕೊಳ್ಳುತ್ತಾರೆ
ಮುಂದು-ವರಿ-ಸಿದ
ಮುಂದು-ವರಿ-ಸಿ-ದಂತೆ
ಮುಂದು-ವರಿ-ಸಿ-ದನು
ಮುಂದು-ವರಿ-ಸಿ-ದರು
ಮುಂದು-ವರಿ-ಸಿದ್ದ-ರೆಂದು
ಮುಂದು-ವರಿ-ಸಿದ್ದಾನೆ
ಮುಂದು-ವರಿ-ಸಿ-ರುವ
ಮುಂದು-ವರಿ-ಸುತ್ತಾರೆ
ಮುಂದು-ವರೆ-ದರು
ಮುಂದು-ವರೆ-ದಿದ್ದನ್ನು
ಮುಂದು-ವರೆ-ಸಿದ-ನೆಂಬುದು
ಮುಂದೆ
ಮುಂದೊಡ್ಡಿ
ಮುಂನಂ
ಮುಂಬ-ರಿದು
ಮುಕುಳಿ-ಕೆರೆಯ
ಮುಕ್ತ-ಗೊಳಿಸಿ
ಮುಕ್ತ-ಹಸ್ತ-ದಿಂದ
ಮುಕ್ತಾಯ-ವಾಯಿತು
ಮುಕ್ತ್ಯಾಂಗ-ನ-ವಲ್ಲಭೋ
ಮುಖ-ವಾಗಿದ್ದ
ಮುಖಸುರ-ರತ್ನ-ದರ್ಪ್ಪಣಂ
ಮುಖಾಂತರ
ಮುಖ್ಯ
ಮುಖ್ಯ-ಕೇಂದ್ರ-ಗಳನ್ನು
ಮುಖ್ಯ-ಕೇಂದ್ರ-ವನ್ನಾಗಿ
ಮುಖ್ಯ-ಕೋಟೆ
ಮುಖ್ಯ-ನಾಗಿದ್ದ-ನೆಂದು
ಮುಖ್ಯ-ನಾಗಿದ್ದ-ನೆಂದೂ
ಮುಖ್ಯ-ಪಟ್ಟಣ-ವಾಗಿತ್ತೆಂದು
ಮುಖ್ಯ-ಪಾತ್ರ
ಮುಖ್ಯ-ಪಾತ್ರ-ವನ್ನು
ಮುಖ್ಯ-ಮಂತ್ರಿಗೆ
ಮುಖ್ಯ-ಮಂತ್ರಿಯ
ಮುಖ್ಯ-ಮಪ್ಪ
ಮುಖ್ಯ-ರಪ್ಪ
ಮುಖ್ಯ-ಲಕ್ಷ-ಣ-ವಾ-ಗಿತ್ತು
ಮುಖ್ಯ-ವಪ್ಪ
ಮುಖ್ಯ-ವಾಗಿ
ಮುಖ್ಯ-ವಾ-ಗಿತ್ತು
ಮುಖ್ಯ-ವಾಗಿತ್ತೆಂಬು-ದನ್ನು
ಮುಖ್ಯ-ವಾದ
ಮುಖ್ಯ-ವಾದುದೆಂದರೆ
ಮುಖ್ಯಸ್ಥ
ಮುಖ್ಯಸ್ಥ-ನಾಗಿ
ಮುಖ್ಯಸ್ಥ-ನಾಗಿದ್ದನು
ಮುಖ್ಯಸ್ಥ-ನಾಗಿದ್ದ-ನೆಂದು
ಮುಖ್ಯಸ್ಥ-ನಾಗಿದ್ದು
ಮುಖ್ಯಸ್ಥ-ನಾಗಿ-ರ-ಬಹುದು
ಮುಖ್ಯಸ್ಥ-ನೆಂದು
ಮುಖ್ಯಸ್ಥ-ರನ್ನಾಗಿ
ಮುಖ್ಯಸ್ಥ-ರನ್ನು
ಮುಖ್ಯಸ್ಥ-ರಾಗಿ
ಮುಖ್ಯಸ್ಥ-ರಾಗಿದ್ದ
ಮುಖ್ಯಸ್ಥ-ರಾಗಿದ್ದರು
ಮುಖ್ಯಸ್ಥ-ರಾಗಿದ್ದ-ರೆಂದು
ಮುಖ್ಯಸ್ಥ-ರಾಗಿದ್ದ-ರೆಂದೂ
ಮುಖ್ಯಸ್ಥ-ರಾಗಿ-ರುತ್ತಿದ್ದರು
ಮುಖ್ಯಸ್ಥ-ರಾದ
ಮುಖ್ಯಸ್ಥರು
ಮುಖ್ಯಸ್ಥಳ
ಮುಖ್ಯಸ್ಥ-ಳ-ವಾಗಿ
ಮುಖ್ಯಸ್ಥ-ಳ-ವಾ-ಗಿತ್ತು
ಮುಖ್ಯಸ್ಥ-ಳ-ವಾಗಿತ್ತೆಂದು
ಮುಖ್ಯಸ್ಥ-ಳ-ವಾಗಿದ್ದ
ಮುಖ್ಯಸ್ಥ-ಳ-ವಾಗಿದ್ದು
ಮುಖ್ಯಸ್ಥ-ಳ-ವಾದ
ಮುಖ್ಯಾಧಿ-ಕಾರಿ-ಗ-ಳಾಗಿ
ಮುಗಿದ
ಮುಗಿ-ದಿತ್ತು
ಮುಗುಳ್ನಗೆ-ಯೊಂದಿಗೆ
ಮುಘಲ್
ಮುಟ್ಟನ-ಹಳ್ಳಿ
ಮುಟ್ಟಿದಂ
ಮುಟ್ಣ-ಹಳ್ಳಿ
ಮುಡಿ-ಗೊಂಡ
ಮುಡಿಪಿ
ಮುಡಿಪಿ-ದ-ನೆಂದು
ಮುಡಿಪಿ-ರಬೇಕೆಂದೂಆ
ಮುತಹಡೆಯ-ರಾಯನ
ಮುತ್ತತ್ತಿ
ಮುತ್ತತ್ತಿಯ
ಮುತ್ತ-ರಸ
ಮುತ್ತ-ರಸನು
ಮುತ್ತಲು
ಮುತ್ತಿ
ಮುತ್ತಿಗ
ಮುತ್ತಿಗೆ
ಮುತ್ತಿ-ಗೆ-ಯಲ್ಲಿ
ಮುತ್ತಿದ
ಮುತ್ತಿ-ದ-ನೆಂದು
ಮುತ್ತಿ-ದಲ್ಲಿ
ಮುತ್ತಿ-ದಾಗ
ಮುತ್ತೆಗೆರೆ
ಮುತ್ತೆತ್ತಿ
ಮುತ್ತೇ-ಗೆರೆಯ
ಮುತ್ಸಂದ್ರ-ಬೇಚಿರಾಕ್
ಮುದಗಂದೂರಿನ
ಮುದಗಂದೂರಿ-ನಲ್ಲಿ
ಮುದಗಂದೂರು
ಮುದಗನ್ದೂರು
ಮುದಗಲ್
ಮುದ-ಗಾವುಂಡನ
ಮುದಗುಂದೂರಿ-ನಲ್ಲಿ
ಮುದಗುಂದೂರಿ-ನಲ್ಲಿ-ನಡೆದ
ಮುದಗುಂದೂರು
ಮುದ-ಗೆರೆ
ಮುದ-ಸಮುದ್ರ
ಮುದಿತ-ಮೂರ್ತಿರ್ಲೋಕ-ವಿಖ್ಯಾತ
ಮುದಿ-ಬೆಟ್ಟದ
ಮುದಿ-ಬೆಟ್ಟ-ದ-ಸಾತೇನ-ಹಳ್ಳಿ
ಮುದಿ-ಮಲೆ
ಮುದಿಮಾ-ರನ-ಹಳ್ಳಿ
ಮುದು-ಕೊಂಗಣಿ
ಮುದುಗ-ನೂರು-ದುರ್ಗ-ಗಳನ್ನು
ಮುದುಗುಂದೂರು
ಮುದು-ಗುಪ್ಪೆ
ಮುದು-ಗುಪ್ಪೆಯ
ಮುದು-ಗುಪ್ಪೆಯು
ಮುದುಡಿಯ
ಮುದುರಾಚಯ್ಯ
ಮುದುರಾಚಯ್ಯ-ನನ್ನು
ಮುದೇನ-ಹಳ್ಳಿ-ಯನ್ನು
ಮುದ್ದ-ಗೌಡನ
ಮುದ್ದ-ರಸಿ
ಮುದ್ದಿ-ಯಕ್ಕರ
ಮುದ್ದು-ಲಿಂಗಮ್ಮ-ನ-ವರು
ಮುದ್ದೆಯ
ಮುದ್ದೆಯ-ನಾಯ-ಕನು
ಮುದ್ರಿ-ಕೆಯ-ನೊಲ-ವಿನಿನೀ
ಮುದ್ರಿಕೆ-ಯನ್ನು
ಮುದ್ರೆ-ಯನ್ನು
ಮುನಿ
ಮುನಿ-ಗ-ಳಿಗೆ
ಮುನಿ-ಗಳು
ಮುನಿ-ಚಂದ್ರ-ದೇವರ
ಮುನಿಯ
ಮುನಿ-ಯನ್ನು
ಮುನಿ-ಯಿಂದ
ಮುನಿಯು
ಮುನಿ-ಯೊಬ್ಬನು
ಮುನಿ-ರಾಜಪ್ಪನ-ವರು
ಮುನಿ-ವರ-ನೆಂದು
ಮುನಿ-ಸಿ-ಕೊಂಡು
ಮುನೀಂದ್ರನು
ಮುನ್ನ
ಮುನ್ನಂರೂಢಿಯ
ಮುನ್ನಡೆ-ಸುತ್ತಿದ್ದ-ರೆಂದು
ಮುನ್ನ-ಸಂದ
ಮುನ್ನುಗ್ಗಿ
ಮುನ್ನೂ-ರನ್ನು
ಮುನ್ನೂರು
ಮುನ್ನೂರುಮ್
ಮುಪ್ಪಿನ
ಮುಮ್ಮಡಿ
ಮುಮ್ಮಡಿ-ಚೋಳ
ಮುಮ್ಮಡಿ-ನರ-ಸಿಂಹ
ಮುಮ್ಮಡಿ-ಬಲ್ಲಾಳನು
ಮುಮ್ಮುರಿ
ಮುರಾರಿ
ಮುರಿದ
ಮುರಿದು
ಮುರು-ಕನ-ಹಳ್ಳಿ
ಮುರು-ಕನ-ಹಳ್ಳಿಯ
ಮುರು-ಯನ
ಮುಲ್ಕ್
ಮುಳಬಾಗಲ್
ಮುಳ-ಬಾ-ಗಿಲು
ಮುಳುಗಿ
ಮುಳು-ಗಿದ್ದು
ಮುಳ್ಗುಂದ-ವನ್ನು
ಮುವರು-ರಾಯ-ರ-ಗಂಡ
ಮುಷೀರ್
ಮುಸಲ್ಮಾನರ
ಮುಸುಕ-ಮಾದೆ-ಗೊಂಡನ
ಮುಸುಕು
ಮುಸ್ತಫಾ-ಖಾನ್
ಮುಸ್ತೈದೆ
ಮುಸ್ಲಿಂ
ಮುಸ್ಲಿಮರ
ಮೂಕು-ತಿ-ಗಳುಂ
ಮೂಗನೂ
ಮೂಗನ್ನು
ಮೂಗರ
ಮೂಗರ-ನಾಡಾಳುವ
ಮೂಗರ-ನಾಡು
ಮೂಗರ-ನಾಡು-ಮೂಗೂರು-ಮೂರು-ನಾಡು
ಮೂಗರಿವೋನ್
ಮೂಗೂ-ರನ್ನು
ಮೂಗೂರು
ಮೂಡಣ
ಮೂಡ-ರಾಜ್ಯ
ಮೂಡ-ರಾಜ್ಯಕ್ಕೆ
ಮೂಡ-ರಾಜ್ಯದ
ಮೂಡಲು
ಮೂಡಿಗೆರೆ
ಮೂರನೆ
ಮೂರ-ನೆಯ
ಮೂರನೆ-ಯ-ವನು
ಮೂರನೇ
ಮೂರರ
ಮೂರು
ಮೂರು-ಜನ
ಮೂರು-ನಾಲ್ಕು
ಮೂರು-ರಾಉ-ಯರ
ಮೂರು-ರಾಯರ
ಮೂರು-ರಾಯ-ರ-ಗಂಡಾಂಕಃ
ಮೂರು-ರಾಯ-ರ-ಗಂಡಾಂಕೋ
ಮೂರು-ಲೋಕ-ಜಗದಳಂ
ಮೂರ್ಛಿತ-ವಾಗು-ವಂತೆ
ಮೂರ್ತಸ್ಯ
ಮೂರ್ತಿ
ಮೂರ್ತಿ-ಗಳನ್ನು
ಮೂರ್ತಿ-ಯನ್ನು
ಮೂರ್ತಿ-ಯ-ವರು
ಮೂರ್ತಿ-ಶಿಲ್ಪ-ಗಳ
ಮೂರ್ತಿ-ಶಿಲ್ಪವು
ಮೂಲ
ಮೂಲಕ
ಮೂಲ-ಕಥೆ
ಮೂಲ-ಕ-ಥೆಯು
ಮೂಲ-ಕವೇ
ಮೂಲತಃ
ಮೂಲದ
ಮೂಲ-ದ-ವನೆನ್ನು-ವುದು
ಮೂಲ-ಪುರ-ಷನ
ಮೂಲ-ಪುರ-ಷ-ನೆಂದು
ಮೂಲ-ಪುರುಷ
ಮೂಲ-ಪುರುಷ-ನಾಗಿ-ರುವ
ಮೂಲ-ಪುರುಷ-ನೆಂದೂ
ಮೂಲ-ಪುರುಷರು
ಮೂಲ-ರಾಜರು
ಮೂಲ-ರೂಪ-ವಾಗಿ-ರಲಾ-ರದು
ಮೂಲ-ವನ್ನು
ಮೂಲ-ವಾಗಿವೆ
ಮೂಲ-ವಾದ
ಮೂಲ-ವೆನಿಪಗ್ಗದ
ಮೂಲ-ಸಂಘದ
ಮೂಲಸ್ಥಾನ
ಮೂಲಸ್ಥಾನ-ದೇವ-ರಿಗೆ
ಮೂಲಿಗ
ಮೂಳೇನ-ಹಳ್ಳಿ-ಯನ್ನು
ಮೂವಡಿ
ಮೂವಡಿ-ಚೋಳ
ಮೂವತ್ತರ್ಛಾಸಿರ
ಮೂವತ್ತು
ಮೂವತ್ತು-ಕೊಳಗ
ಮೂವತ್ತೂರ
ಮೂವ-ರಲ್ಲದೆ
ಮೂವರು
ಮೂವರು-ರಾಯರ
ಮೂವರೂ
ಮೂಷಕಸ್ತಥಾ
ಮೂಷವ
ಮೂಷಿಕ
ಮೂಷಿಕ-ಮೂಷಕ
ಮೃಗಯಾಂ
ಮೃತ-ನಾಗಿ
ಮೃತ-ನಾಗಿದ್ದನು
ಮೃತ-ನಾಗಿದ್ದ-ನೆಂದು
ಮೃತ-ನಾಗಿ-ರ-ಬಹುದು
ಮೃತನಾದ
ಮೃತ-ನಾ-ದನು
ಮೃತನಾ-ದಾಗ
ಮೃತ-ಪಟ್ಟ
ಮೃತ-ಪಟ್ಟ-ನೆಂದು
ಮೃತ-ಪಟ್ಟಾಗ
ಮೃತ-ಪಟ್ಟಿದ್ದರು
ಮೃತ-ಪಟ್ಟಿದ್ದು
ಮೃತ-ಪಟ್ಟಿ-ರ-ಬಹುದು
ಮೃತರ
ಮೃತ-ರಾಗಿದ್ದ-ರೆಂಬುದು
ಮೃತರಾ-ದ-ರೆಂದು
ಮೃಷ್ಟಾಂನ-ದಾನ
ಮೆಂಟೆ-ಯದ
ಮೆಂಡೆ-ಯದ
ಮೆಕೆಂಝಿ
ಮೆಚ್ಚ-ದ-ವರು
ಮೆಚ್ಚ-ದೋ-ರಾರ್
ಮೆಚ್ಚಿ
ಮೆಚ್ಚಿದ
ಮೆಚ್ಚಿದೆ
ಮೆಚ್ಚಿದೆಂ
ಮೆಚ್ಚಿದ್ದನ್ನು
ಮೆಚ್ಚಿಸಿ
ಮೆಚ್ಚುಗೆ
ಮೆಚ್ಚುಗೆ-ಯಾಗಿ
ಮೆಚ್ಚೆ
ಮೆಟ್ಟಿ
ಮೆಣಸ
ಮೆಯಿ-ಸಿರಿ-ವಟ್ಟ-ವನೆ
ಮೆಯೆ-ದೇವನ
ಮೆಯ್ಯೊಳೆಯ್ದಿ
ಮೆರೆ-ದನು
ಮೆರೆವೊಳ್ಳೆಮ್ಬ
ಮೆಲು-ಕೋಟೆ-ಯನ್ನು
ಮೆಲ್ಲಗೆ
ಮೆಳ-ಹಳ್ಳಿ
ಮೇ
ಮೇಘ-ಚಂದ್ರ
ಮೇಘ-ಚಂದ್ರ-ಸಿದ್ಧಾಂತ
ಮೇಡು
ಮೇದಿನಿ
ಮೇದಿನೀ
ಮೇನಾ-ಗರ
ಮೇನಾ-ಪುರ
ಮೇರು
ಮೇರು-ಲಂಘಿಯಶೋ-ಭರಃ
ಮೇರು-ವಿನ
ಮೇರು-ವೆನಿ-ಸಿದ
ಮೇರೆ-ಗಳನ್ನು
ಮೇರೆ-ಗಳಾಗಿ
ಮೇರೆ-ಗಳಾಗಿದ್ದವು
ಮೇರೆಗೆ
ಮೇರೆ-ಯನ್ನು
ಮೇರೆ-ಯಾಗಿ
ಮೇರೆ-ಯಾಗಿದ್ದವು
ಮೇರೆಯೂ
ಮೇಲಕ್ಕೆ
ಮೇಲಧಿ-ಕಾರಿ
ಮೇಲಾಟಕ್ಕೆ
ಮೇಲಾ-ಳಿಕೆ
ಮೇಲಾ-ಳಿಕೆ-ಯನ್ನು
ಮೇಲಾಳ್ಕೆ
ಮೇಲಿಂದ
ಮೇಲಿದೆ
ಮೇಲಿದ್ದ
ಮೇಲಿನ
ಮೇಲಿಪಿ-ಳತ್ತೂರು
ಮೇಲಿರು
ಮೇಲಿ-ರುವ
ಮೇಲು-ಕೋಟೆ
ಮೇಲು-ಕೋಟೆಗೂ
ಮೇಲು-ಕೋಟೆಗೆ
ಮೇಲು-ಕೋಟೆಯ
ಮೇಲು-ಕೋಟೆ-ಯಲ್ಲಿ
ಮೇಲುಗೈ
ಮೇಲುಸ್ತು-ವಾರಿ-ಯಲ್ಲಿ
ಮೇಲೆ
ಮೇಲೆ-ರಗಿ
ಮೇಲೆ-ವನ್ದು
ಮೇಲೇರಿ
ಮೇಲೇರಿ-ರು-ವುದು
ಮೇಲ್
ಮೇಲ್ಕಂಡ
ಮೇಲ್ಕಂಡಂತೆ
ಮೇಲ್ಬಾಗ-ದಲ್ಲಿ
ಮೇಲ್ಮಟ್ಟದ
ಮೇಲ್ಮೆ-ಯನ್ನೇ
ಮೇಲ್ವಿಚಾರ-ಕನ
ಮೇಲ್ವಿಚಾರಣೆ
ಮೇಲ್ವಿಚಾರ-ಣೆ-ಯಲ್ಲಿ
ಮೇಳಯ್ಯನು
ಮೇಳಾ-ದೇವಿ-ಯರ
ಮೇಳಾ-ಪುರ
ಮೇಳಿ
ಮೈತುಂಬಾ
ಮೈದಾನ-ವಿದ್ದು
ಮೈದುನ
ಮೈದುನ-ರಾದ
ಮೈಮರೆತ
ಮೈಮೆಟ್ಟಿ
ಮೈಲನ-ಹಳ್ಳಿ
ಮೈಲಳ-ದೇವಿಯು
ಮೈಲಿ
ಮೈಸು-ನಾಡು-ಮೈಸೆ-ನಾಡು
ಮೈಸೂರ
ಮೈಸೂ-ರಿಗೆ
ಮೈಸೂರಿನ
ಮೈಸೂರಿ-ನಲ್ಲಿ
ಮೈಸೂರಿನಲ್ಲಿ-ರುವ
ಮೈಸೂರಿ-ನಿಂದ
ಮೈಸೂರು
ಮೊಗ-ವನ್ನು
ಮೊಗವಾಡವೋ
ಮೊಟ್ಟ-ಮೊದ-ಲಿಗೆ
ಮೊಟ್ಟೆನವಿಲೆಮಟ್ಟನೋಲೆ
ಮೊಟ್ಟೆನವಿಲೆ-ಯನ್ನು
ಮೊಡ-ವನ-ಕೋಡಿಯ
ಮೊಡ-ವಿನ-ಕೋಡಿ
ಮೊಡ-ವಿನ-ಕೋಡಿಯ
ಮೊತ್ತ
ಮೊತ್ತಕ
ಮೊತ್ತದ
ಮೊತ್ತ-ವನ್ನು
ಮೊತ್ತ-ಹಳ್ಳಿಯ
ಮೊದ-ನೆಯ
ಮೊದ-ಮೊದಲ
ಮೊದ-ಮೊದಲು
ಮೊದಲ
ಮೊದಲನೆ
ಮೊದಲ-ನೆಯ
ಮೊದಲ-ನೆ-ಯ-ದಾಗಿ
ಮೊದಲ-ನೆ-ಯ-ವ-ನಾದ
ಮೊದಲ-ನೆ-ಯ-ವನು
ಮೊದಲನೇ
ಮೊದಲ-ಬಾರಿಗೆ
ಮೊದಲ-ಮಗ-ನಿಗೆ
ಮೊದ-ಲಾಗಿ
ಮೊದಲಾದ
ಮೊದಲಾದಂ
ಮೊದಲಾದ-ವ-ರನ್ನು
ಮೊದಲಾದ-ವ-ರಿಗೆ
ಮೊದಲಾದ-ವರು
ಮೊದಲಾದ-ವು-ಗಳನ್ನು
ಮೊದ-ಲಾಯಿತು
ಮೊದಲಿ
ಮೊದಲಿಗ
ಮೊದಲಿ-ಗ-ನಲ್ಲ
ಮೊದಲಿ-ಗ-ರೆಂದು
ಮೊದ-ಲಿಗೆ
ಮೊದಲಿನ
ಮೊದಲಿ-ನಿಂದಲೂ
ಮೊದಲಿ-ಯಳ್ಳಿಯ
ಮೊದಲಿ-ಹಳ್ಳಿ
ಮೊದಲು
ಮೊದಲೇ
ಮೊದಲ್ಗೊಂಡು
ಮೊನೆಮಟ್ಟರ-ಹಳ್ಳಿ
ಮೊನೆಮುಟ್ಟರ-ಹಳ್ಳಿ
ಮೊನೆ-ಯಾಳು
ಮೊನೆ-ಯಾಳ್ತನಂಗೆಯ್ವ-ರಿಗೆ
ಮೊನೆ-ಯಾಳ್ತನ-ವನ್ನು
ಮೊನೆ-ಯಾಳ್ವ
ಮೊನೆ-ಯೊಳ್
ಮೊಮ್ಮಗ
ಮೊಮ್ಮಗ-ನಾಗಿ-ರ-ಬಹುದು
ಮೊಮ್ಮಗ-ನಾದ
ಮೊಮ್ಮಗ-ನಿರ-ಬಹುದು
ಮೊಮ್ಮಗ-ನೆಂದೂ
ಮೊಮ್ಮಗನೇ
ಮೊಮ್ಮ-ಗಳು
ಮೊರಸಾದಿ-ರಾಯರು
ಮೊರಸಾಧಿ-ರಾಯ
ಮೊರಸಾಧಿ-ರಾಯ-ರೆಂದು
ಮೊರಸು
ಮೊರಸು-ನಾಡು
ಮೊಲ
ಮೊಲಿಗೆ
ಮೊಲೆವಾಲ
ಮೊಳ-ನಾಡ
ಮೊಳ-ನಾಡಸ್ಥಳದ
ಮೊಹಮದ್
ಮೋಕ್ಷ-ತಿಳಕ-ವೆಂಬ
ಮೋಡ-ಕುಲ-ದ-ವನು
ಮೋಡ-ಕುಳ-ಕಮಳ
ಮೋದಿಖಾನೆ
ಮೋದು-ನಾ-ಡಿನ
ಮೋದು-ನಾಡುಕೇ
ಮೋದೂ-ರನ್ನು
ಮೋದೂರಿ-ನಲ್ಲಿ
ಮೋದೂರು
ಮೋದೂರು-ನಾಡಿ-ನಲ್ಲಿ
ಮೋದೂರು-ನಾಡು
ಮೌರ್ಯ
ಮ್ಯೂಸಿಯಂನಲ್ಲಿದೆ
ಮ್ಲೇಚ್ಛರು-ಗಳ
ಯಕ್ಷನಂ
ಯಕ್ಷ-ರಾಜ
ಯಕ್ಷ-ರಾಜನು
ಯಜು-ಶಾಖೆಯ
ಯಜುಶ್ಶಾಖೆಯ
ಯಜ್ಞ-ಗಳನ್ನು
ಯಜ್ಞಯಾಗಾದಿ-ಗಳನ್ನು
ಯಡ-ಗೋಡಿಯ
ಯಡ-ಹಳ್ಳಿ
ಯತಿಗೆ
ಯತಿ-ಭಿಕ್ಷೆ-ಗಾಗಿ
ಯತಿಯ
ಯತಿ-ರಾಜ-ಮಠ-ದಲ್ಲಿ
ಯತಿ-ರಾಜ-ಮಠ-ವನು
ಯತಿ-ರಾಜ-ಸಪ್ತತಿ-ಯನ್ನು
ಯತಿ-ರಾಜ-ಸಪ್ತಶತಿ-ಯನ್ನು
ಯತೀಶ್ವ-ರನು
ಯತ್ನದಿಂ
ಯತ್ನಿಸಿ
ಯತ್ನಿಸಿ-ದನು
ಯತ್ನಿಸಿದ್ದಾರೆ
ಯದು
ಯದು-ಗಿರಿ
ಯದು-ಗಿರಿ-ಯಲ್ಲಿ
ಯದು-ರಾಜನ
ಯದು-ವಂಶ
ಯದು-ವಂಶದ
ಯದು-ವಂಶ-ದಲ್ಲಿ
ಯದು-ವಂಶ-ವರ್ಧನ-ಕರಂ
ಯನ್ನು
ಯಮ್ಮದೂರು
ಯರ-ಹ-ಳಿಯ
ಯರ-ಹಳ್ಳ
ಯರ-ಹಳ್ಳಿ
ಯಲಬುರ್ಗಿಯ-ಸಿಂಧ
ಯಲವದ
ಯಲವದ-ಪಲ್ಲಿ
ಯಲವದ-ಹಳ್ಳಿ
ಯಲವದ-ಹಳ್ಳಿ-ಗಳನ್ನು
ಯಲಹಂಕ-ನಾಯ-ಕರು
ಯಲಾದ-ಹಳ್ಳಿ
ಯಲು
ಯಲೆ-ಚಾ-ಕನ-ಹಳ್ಳಿಯ
ಯಲ್ಲಪ್ಪಯ್ಯ-ನೆಂಬು-ವ-ವನು
ಯಲ್ಲಾದ-ಹಳ್ಳಿ
ಯಲ್ಲಾ-ಪುರ
ಯಳಂದೂರಿನ
ಯಶಸ್ವಿ-ಯಾಗಿ
ಯಶಸ್ಸಾಗ-ಬೇಕೆಂದು
ಯಶಸ್ಸು
ಯಶಸ್ಸು-ಗಳನ್ನು
ಯಶೋಧನ
ಯಶೋ-ಧರ
ಯಸ್ಮನ್ರಂಜ-ಯತಿ
ಯಸ್ಮಿನ್
ಯಸ್ಯ
ಯಾಗಿ
ಯಾಗಿ-ರ-ಬಹುದು
ಯಾಗಿ-ರುತ್ತಿದ್ದನು
ಯಾಚಕ-ಜನಾಭಿವ್ರಿದ್ಧಿ
ಯಾಚನ-ಘಟ್ಟ
ಯಾಚನ-ಹಳ್ಳಿ
ಯಾಡ
ಯಾತ್ರೆ-ಯನ್ನು
ಯಾದ-ಗಿರಿ
ಯಾದವ
ಯಾದ-ವ-ಕುಲಾಂಬುಧಿ
ಯಾದ-ವ-ಗಿರಿ
ಯಾದ-ವ-ಗಿರಿಗೆ
ಯಾದ-ವ-ನಾ-ರಾಯಣ
ಯಾದ-ವ-ನಾ-ರಾಯ-ಣ-ಪುರ-ವಾದ
ಯಾದ-ವ-ಪುರ
ಯಾದ-ವ-ಪುರ-ದಲ್ಲಿ
ಯಾದ-ವ-ಪುರ-ದಲ್ಲಿದ್ದ-ನೆಂದು
ಯಾದ-ವ-ಪುರ-ವಾದ
ಯಾದ-ವ-ಪುರಿ
ಯಾದ-ವ-ಪುರಿ-ಮೇಲು-ಕೋಟೆ
ಯಾದ-ವ-ಪುರಿಯ
ಯಾದ-ವ-ರಾಜ-ಧಾನಿ-ಯಾದ
ಯಾದ-ವ-ರಾಜ-ನಾದ
ಯಾದ-ವ-ರಾಜ್ಯ-ಲಕ್ಷ್ಮೀ
ಯಾದ-ವಾಚಲ-ಪತಿ-ಯಾದ
ಯಾದ-ವಾದ್ರಿ-ಪತಿಯು
ಯಾದ-ವಾದ್ರಿಯ
ಯಾರಾದ-ರೊಬ್ಬರ
ಯಾರಿಂದಲೂ
ಯಾರು
ಯಾರೆಂಬು-ದನ್ನು
ಯಾರೆಂಬುದು
ಯಾರೋ
ಯಾಲಾದ-ಹಳ್ಳಿ
ಯಾವ
ಯಾವ-ಕಾರ-ಣಕ್ಕೋ
ಯಾವ-ಯಾವ
ಯಾವ-ರೀತಿ
ಯಾವು-ದನ್ನೂ
ಯಾವು-ದಾದರೂ
ಯಾವುದು
ಯಾವುದೇ
ಯಾವುದೋ
ಯಾವುವು
ಯಾವುವೂ
ಯಿಂದ-ವರದ
ಯಿಂದಾಳ್ದನಾ
ಯಿಂಮಡಿ
ಯಿರಪ-ವಿ-ರೂಪಾಕ್ಷ-ದೇವ-ರಿಗೆ
ಯೀಚಣ
ಯುಗದಲಿ
ಯುತಾ-ನಾಮಾಭಿವಂದಿ-ತಾಂಹತ್ತು-ಸಾವಿರ
ಯುದ್ಧ
ಯುದ್ಧ-ಕಾಲ-ದಲ್ಲಿ
ಯುದ್ಧಕ್ಕೂ
ಯುದ್ಧಕ್ಕೆ
ಯುದ್ಧ-ಗಳನ್ನು
ಯುದ್ಧ-ಗ-ಳನ್ನೂ
ಯುದ್ಧ-ಗಳಲ್ಲಿ
ಯುದ್ಧ-ಗ-ಳಾದವು
ಯುದ್ಧ-ಗಳು
ಯುದ್ಧದ
ಯುದ್ಧ-ದಲ್ಲಿ
ಯುದ್ಧ-ದಲ್ಲೇ
ಯುದ್ಧ-ಪಟು-ವಾಗುವ
ಯುದ್ಧ-ಭೂಮಿ-ಗಳಲ್ಲಿ
ಯುದ್ಧ-ಭೂಮಿ-ಯಲ್ಲೇ
ಯುದ್ಧ-ಮಾಡಿ
ಯುದ್ಧ-ಮಾಡಿದನು
ಯುದ್ಧ-ಮಾಡಿದು-ದಕ್ಕೆ
ಯುದ್ಧ-ಮಾಡಿದು-ದನ್ನು
ಯುದ್ಧ-ಮಾಡುತ್ತಿದ್ದಾಗ
ಯುದ್ಧ-ರಂಗ-ದಲ್ಲಿ
ಯುದ್ಧ-ವನ್ನು
ಯುದ್ಧ-ವ-ಮಾಡಿ
ಯುದ್ಧ-ವಾಗಿ-ರ-ಬಹುದು
ಯುದ್ಧ-ವಾಗಿ-ರುತ್ತದೆ
ಯುದ್ಧ-ವಾದ
ಯುದ್ಧ-ವಾ-ದಾಗ
ಯುದ್ಧ-ವಿರ-ಬಹುದು
ಯುದ್ಧವು
ಯುದ್ಧಾರ-ಚರ-ಣೆಯು
ಯುಧಿಷ್ಟಿರಾಭಿಷೇಕ
ಯುವ-ಕ-ನಾಗಿದ್ದಾಗಲೇ
ಯುವ-ರಾಜ
ಯುವ-ರಾಜ-ನನ್ನಾಗಿ
ಯುವ-ರಾಜ-ನನ್ನು
ಯುವ-ರಾಜ-ನಾಗಿ
ಯುವ-ರಾಜ-ನಾಗಿದ್ದ
ಯುವ-ರಾಜ-ನಾಗಿದ್ದನು
ಯುವ-ರಾಜ-ನಾಗಿದ್ದ-ನೆಂದು
ಯುವ-ರಾಜ-ನಾಗಿದ್ದಾಗಲೇ
ಯುವ-ರಾಜ-ನಾಗಿದ್ದು-ಕೊಂಡು
ಯುವ-ರಾಜ-ನಾದ
ಯುವ-ರಾಜ-ನೆಂದು
ಯುವ-ರಾಜ-ರನ್ನು
ಯುವ-ರಾಜರು
ಯುವ-ರಾಜರೂ
ಯೆಂಕಟಪ್ಪ
ಯೆಂಣೆ-ನಾಡನ್ನು
ಯೆಡ-ತೊರೆ-ತಾಲ್ಲೂಕು
ಯೆಡ-ದೊರೆ
ಯೆಡವಾಣೆ
ಯೆಡೂರಿ
ಯೆಪ್ಪತ್ತಕ್ಕೆ
ಯೆಪ್ಪತ್ತು
ಯೆರೆ-ಗಂಗ
ಯೆಲೆಗನೂರ
ಯೆಲ್ಲಪ್ಪಯ್ಯನು
ಯೆಳಂದೂರು
ಯೇಕಾದಶಿವ್ರತ-ನಿರತ
ಯೇಚಿ-ರಾಜ
ಯೋ
ಯೋಗ-ಗೌಡ
ಯೋಗಭಂಗಿ-ಯಲ್ಲಿ
ಯೋಗಾ-ನರ-ಸಿಂಹಸ್ವಾಮಿಗೆ
ಯೋಗಿಪ್ರ-ವರ
ಯೋಗೀಂದ್ರರ
ಯೋಗ್ಯ
ಯೋಗ್ಯ-ತೆಗೆ
ಯೋಗ್ಯ-ತೆಯ
ಯೋಧ-ರಿದ್ದ-ರೆಂದು
ಯೋಧರು-ಗಳ
ಯೋಧ-ಸೈನಿಕ-ರಾದ
ಯ್ದೆಡೆ
ರ
ರಂಗ
ರಂಗಕ್ಷಿತೀಂದ್ರ
ರಂಗ-ಗವುಡನ
ರಂಗಣ್ಣ-ನೆಂಬ
ರಂಗದ
ರಂಗ-ದಲ್ಲಿ
ರಂಗ-ಧಾಮಸ್ವಾಮಿಗೆ
ರಂಗನ
ರಂಗ-ನ-ಕೊಪ್ಪಲು
ರಂಗ-ನ-ತಿಟ್ಟು
ರಂಗ-ನಾಥ
ರಂಗ-ನಾಥ-ದೇವರ
ರಂಗ-ನಾಥ-ದೇವ-ರಿಗೆ
ರಂಗ-ನಾಥ-ದೇವಾಲಯ-ದಿಂದ
ರಂಗ-ನಾಥನ
ರಂಗ-ನಾಥ-ನ-ಗರ-ದಲ್ಲಿ
ರಂಗ-ನಾಥ-ನಿಗೆ
ರಂಗ-ನಾಥಸ್ವಾಮಿ
ರಂಗ-ನಾಯಕಿ
ರಂಗ-ಪಯ್ಯ
ರಂಗ-ಪಯ್ಯನು
ರಂಗ-ಪಯ್ಯನೂ
ರಂಗಪ್ಪ-ನಾಯ-ಕನು
ರಂಗ-ಭೋಗ
ರಂಗ-ಮಂಟಪ
ರಂಗ-ಮಂಟಪ-ವನ್ನು
ರಂಗ-ಮಾಂಬಿಕೆ-ಯರು
ರಂಗ-ಮಾಂಬೆ-ಯ-ವರು
ರಂಗಮ್ಮ-ನ-ವರು
ರಂಗಯ್ಯ
ರಂಗಯ್ಯನ
ರಂಗಯ್ಯ-ನಾಯಕ
ರಂಗಯ್ಯ-ನಾಯ-ಕನು
ರಂಗ-ರಾಜಯ್ಯನ
ರಂಗ-ಹಳ್ಳಿ
ರಂಗಾಂಬಿಕಾ
ರಂಗಾಚಾರಿ
ರಂಗಾಚಾರಿಯು
ರಂಗಾ-ಪುರ
ರಂಗೆಯ-ನಾಯಕ
ರಂಜ-ಯನ್ನ-ಖಿಲಾಃ
ರಂಜಿತ-ವಾಗಿ
ರಂದು
ರಕ
ರಕ್ಕಸ-ಗಂಗ
ರಕ್ಕಸ-ಗಂಗನು
ರಕ್ಕಸ-ಗಂಗ-ನೆಂಬ
ರಕ್ಕಸಗಿ
ರಕ್ತ-ಕೊಡುಗೆ-ಯನ್ನು
ರಕ್ತಪಾತ-ವಾಗ-ಲಿಲ್ಲ
ರಕ್ತ-ಸಂಬಂಧಿ-ಗಳು
ರಕ್ಷಕ
ರಕ್ಷಣಾ
ರಕ್ಷಣಾಂಗ
ರಕ್ಷಣಾಯ
ರಕ್ಷಣೆ
ರಕ್ಷಣೆಗೆ
ರಕ್ಷಣೆಯ
ರಕ್ಷಣೆ-ಯಲ್ಲಿ
ರಕ್ಷಣೆ-ಯಲ್ಲಿಟ್ಟು
ರಕ್ಷಾ-ಕರಃ
ರಕ್ಷಾಪಾಳಕ-ರಾಗಿದ್ದರು
ರಕ್ಷಿಪಂ
ರಕ್ಷಿಪ್ಪ
ರಕ್ಷಿ-ಸಲು
ರಕ್ಷಿಸಿ
ರಕ್ಷಿ-ಸುವ
ರಚನೆ
ರಚನೆ-ಗಳು
ರಚನೆಗೆ
ರಚ-ನೆಯ
ರಚನೆ-ಯಲ್ಲಿ
ರಚನೆ-ಯಾಗಿದೆ
ರಚನೆ-ಯಾಗಿ-ರ-ಬಹುದು
ರಚನೆ-ಯಾಗಿ-ರ-ಲಿಲ್ಲ
ರಚನೆ-ಯಾದವು
ರಚಿತ-ವಾಗಿದ್ದ
ರಚಿತ-ವಾಗಿದ್ದು
ರಚಿತ-ವಾಗಿ-ರುವ
ರಚಿತ-ವಾದ
ರಚಿ-ಸ-ಲಾಗಿದೆ
ರಚಿ-ಸ-ಲಾಯಿತು
ರಚಿಸಿದ
ರಚಿಸಿ-ದ-ನೆಂದೂ
ರಚಿ-ಸಿದರು
ರಚಿಸಿದ್ದಾನೆ
ರಚಿ-ಸಿದ್ದಾರೆ
ರಚಿಸಿ-ರ-ಬಹು-ದಾದ
ರಚಿಸಿ-ರುವ
ರಜತಪರಿಯಂಕ
ರಜಾಕ್
ರಟ್ಟ
ರಟ್ಟ-ಪಾಡಿ
ರಟ್ಟ-ಪಾ-ವಾಡಿ
ರಟ್ಟಿ-ಹಳ್ಳಿ-ಗಳು
ರಟ್ಟೆ
ರಠ-ಹಳ್ಳಿಯ
ರಣ
ರಣಕಲ-ಕೇತ
ರಣದುಲ್ಲಾ-ಖಾನ್
ರಣ-ಧೀರ-ಕಂಠೀರವನು
ರಣಪಾರ
ರಣಪಾ-ರರ್
ರಣಮುಖ
ರಣ-ರಂಗ-ಕೇ-ಸರಿ
ರಣ-ರಂಗ-ದಲ್ಲಿ
ರಣ-ರಂಗ-ದಿಂದ
ರಣ-ರಂಗ-ಧೀರನುಂ
ರಣವಿಕ್ರಮಾರ್ಯ
ರಣಾವ-ಲೋಕ
ರಣಿತಗವುಂಡ
ರಣಿತಗವುಂಡನ
ರಣಿತ-ಗವುಂಡನು
ರಣಿತಗವುಂಡ-ನುದ್ಭವಿಸಿ
ರಣಿ-ಭಾಟು
ರಣಿರಾಉ-ಪದ-ಭೋಗ
ರತ್ನ
ರತ್ನತ್ರಯ
ರತ್ನತ್ರಯ-ದಂತೆ
ರತ್ನತ್ರಯಾ-ಕರಂ
ರತ್ನ-ಪಾಲ
ರತ್ನ-ಭಾರಣ
ರತ್ನ-ಸಿಂಹಾಸ-ನ-ದಿಂದ
ರತ್ನ-ಸಿಂಹಾಸ-ನಾರೂಢ-ನಾಗಿ
ರತ್ನ-ಸಿಂಹಾಸ-ನಾರೂಢ-ನಾಗಿದ್ದ-ನೆಂದು
ರತ್ನ-ಸಿಂಹಾಸ-ನಾರೂಢ-ರಾಗಿ
ರತ್ನ-ಸಿಂಹಾಸನೇ
ರತ್ನಾ-ಕರ
ರತ್ನಾಯಿಗೆ
ರಥೋತ್ಸವಕ್ಕೆ
ರದ್ದು-ಗೊಳಿಸಿ
ರದ್ದು-ಪಡಿಸಿ
ರದ್ದು-ಮಾಡಿ
ರನ್ನ-ಕವಿ
ರನ್ನನು
ರಮಣೀಯ
ರಮ್ಯ
ರಮ್ಯಂ
ರಮ್ಯಃ
ರಮ್ಯ-ವಾದ
ರಮ್ಯೇ
ರಲ್ಲಾಯಿತು
ರಲ್ಲಿ
ರಲ್ಲಿಯೂ
ರಲ್ಲೇ
ರವರ
ರವರೆ-ಗಿನ
ರವ-ರೆಗೆ
ರವೆಗೆ
ರಸಾವಾತ್ರೇಯ
ರಸ್ತೆ-ಯಲ್ಲಿರುವ
ರಹ-ಗೌಡ
ರಾಂಪುರ
ರಾಕ್ಷಸಿ
ರಾಗಮುಣ-ಗಾ-ಮುಂಡ
ರಾಗಿದ್ದ-ರೆಂದು
ರಾಘಣ್ಣ-ದೇವನ
ರಾಘವಾ-ಪುರ
ರಾಚನ-ಹಳ್ಳಿ-ಯನ್ನು
ರಾಚಪ್ಪಾಜೀ
ರಾಚ-ಮಲ್ಲ
ರಾಚ-ಮಲ್ಲಂ
ರಾಚ-ಮಲ್ಲನ
ರಾಚ-ಮಲ್ಲ-ನನ್ನು
ರಾಚ-ಮಲ್ಲ-ನೀತಿ-ಮಾರ್ಗ
ರಾಚ-ಮಲ್ಲನು
ರಾಚಯ್ಯ-ನಾಯ-ಕನ
ರಾಚವಲ್ಲ
ರಾಚ-ವೂರು
ರಾಚೆಯ-ನಾಯಕ
ರಾಜ
ರಾಜ-ಒಡೆ-ಯನು
ರಾಜ-ಕಂಠೀರವೇಂದ್ರ
ರಾಜ-ಕಾಂಅð-ವನ್ನು
ರಾಜ-ಕಾರ-ಣ-ದಲ್ಲೇ
ರಾಜ-ಕಾರ್ಯ
ರಾಜ-ಕಾರ್ಯಕ್ಕಾಗಿ
ರಾಜ-ಕಾರ್ಯದ
ರಾಜ-ಕೀಯ
ರಾಜ-ಕೀಯ-ದಲ್ಲಿ
ರಾಜ-ಕೀಯ-ದಿಂದ
ರಾಜ-ಕುಂಜರ
ರಾಜ-ಕುಮಾರ
ರಾಜ-ಕುಮಾರ-ರನ್ನು
ರಾಜ-ಕುಮಾರರೇ
ರಾಜ-ಕುಮಾರಿ
ರಾಜ-ಕೇ-ಸರಿ-ವರ್ಮ
ರಾಜ-ಗುರು
ರಾಜ-ಗುರು-ಗಳ
ರಾಜ-ಗುರು-ಗಳಿದ್ದ-ರೆಂಬು-ದನ್ನು
ರಾಜ-ಗುರು-ಗಳು
ರಾಜ-ಚಿಹ್ನೆ-ಗಳನ್ನು
ರಾಜ-ಧನತ್ವಕ್ಕೆ
ರಾಜ-ಧರ್ಮ್ಮೇಣ
ರಾಜ-ಧಾನಿ
ರಾಜ-ಧಾನಿ-ಗ-ಳಾದ
ರಾಜ-ಧಾನಿ-ಗಳು
ರಾಜ-ಧಾ-ನಿಗೆ
ರಾಜ-ಧಾನಿಯ
ರಾಜ-ಧಾನಿ-ಯನ್ನ
ರಾಜ-ಧಾನಿ-ಯನ್ನಾಗಿ
ರಾಜ-ಧಾನಿ-ಯನ್ನು
ರಾಜ-ಧಾನಿ-ಯಲ್ಲಿ
ರಾಜ-ಧಾನಿ-ಯ-ವ-ರೆಗೆ
ರಾಜ-ಧಾನಿ-ಯಾ-ಗಿತ್ತು
ರಾಜ-ಧಾನಿ-ಯಾಗಿದ್ದ
ರಾಜ-ಧಾನಿ-ಯಾಗಿದ್ದರೂ
ರಾಜ-ಧಾನಿ-ಯಾಗಿದ್ದು
ರಾಜ-ಧಾನಿ-ಯಾದ
ರಾಜ-ಧಾನಿ-ಯಾಯಿತು
ರಾಜ-ಧಿ-ರಾಜ
ರಾಜನ
ರಾಜ-ನಂತಹ
ರಾಜ-ನನ್ನು
ರಾಜ-ನ-ಸಂಬಂಧಿ-ಕರು
ರಾಜ-ನ-ಹಿರಿಯ
ರಾಜ-ನ-ಹೆ-ಸರಿಲ್ಲ
ರಾಜ-ನ-ಹೆ-ಸರು
ರಾಜ-ನಾಗಿದ್ದ
ರಾಜ-ನಾಗಿದ್ದನು
ರಾಜ-ನಾಗಿದ್ದು
ರಾಜ-ನಾದ
ರಾಜ-ನಾ-ದನು
ರಾಜ-ನಿಂದ
ರಾಜ-ನಿಗೆ
ರಾಜ-ನಿಗೇ
ರಾಜ-ನೀ-ತಿಯ
ರಾಜನು
ರಾಜ-ನೃಪ
ರಾಜ-ನೆಂದು
ರಾಜ-ನೆನಿಸಿದ
ರಾಜನೇ
ರಾಜ-ನೊಡನೆ
ರಾಜ-ಪರಮೇಶ್ವರ
ರಾಜ-ಪರಮೇಶ್ವರಂಯಾದ-ವ-ಕುಲಾಂಬುದಿ
ರಾಜ-ಪರಮೇಶ್ವರಃ
ರಾಜ-ಪರಮೇಶ್ವರ-ನೆಂದು
ರಾಜಪ್ರತಿ-ನಿಧಿಯ
ರಾಜ-ಬೆವಹಾರಿ
ರಾಜ-ಭಟರೇ
ರಾಜ-ಮನೆ-ತನ-ಗಳ
ರಾಜ-ಮನೆ-ತನ-ಗಳಿಗೂ
ರಾಜ-ಮನೆ-ತನ-ಗಳು
ರಾಜ-ಮನೆ-ತ-ನದ
ರಾಜ-ಮನೆ-ತನ-ದ-ವ-ರನ್ನು
ರಾಜ-ಮನೆ-ತನ-ದ-ವ-ರಿಗೂ
ರಾಜ-ಮನೆ-ತನ-ದ-ವರು
ರಾಜ-ಮನೆ-ತನ-ವನ್ನು
ರಾಜ-ಮಲ್ಲ
ರಾಜ-ಮಲ್ಲನ
ರಾಜ-ಮಲ್ಲ-ನನ್ನು
ರಾಜ-ಮಲ್ಲ-ನಿಗೆ
ರಾಜ-ಮಲ್ಲನು
ರಾಜ-ಮಲ್ಲರ
ರಾಜ-ಮಹಾ-ರಾಜ-ರೆಂದಲ್ಲ
ರಾಜ-ಮಹೇಂದ್ರಿ-ಯನ್ನು
ರಾಜ-ಮಾರ್ತಾಂಡ-ನೆಂಬ
ರಾಜಯ್ಯ
ರಾಜರ
ರಾಜ-ರದು
ರಾಜ-ರನ್ನಾಗಿ
ರಾಜ-ರನ್ನು
ರಾಜ-ರಾಗಿದ್ದರು
ರಾಜ-ರಾಜ
ರಾಜ-ರಾಜ-ಚೋಳನ
ರಾಜ-ರಾಜ-ಚೋಳ-ನಿಗೆ
ರಾಜ-ರಾಜ-ಚೋಳ-ನೆಂದು
ರಾಜ-ರಾಜ-ದೇವನ
ರಾಜ-ರಾಜ-ದೇವನು
ರಾಜ-ರಾಜ-ನನ್ನು
ರಾಜ-ರಾಜ-ಪುರ
ರಾಜ-ರಾಜ-ಪುರದ
ರಾಜ-ರಾಜ-ಪುರ-ದಲ್ಲಿರು-ವಾಗ
ರಾಜ-ರಾಜ-ಪುರ-ವಾದ
ರಾಜ-ರಾಜಶ್ರೀ
ರಾಜ-ರಾಜ-ಸಮಾಂಹತಿಃ
ರಾಜ-ರಿಂದ
ರಾಜರು
ರಾಜ-ರೆಲ್ಲರಂ
ರಾಜ-ವಂಶಕ್ಕೆ
ರಾಜ-ವಂಶ-ಗ-ಳಿಗೆ
ರಾಜ-ವಂಶದ
ರಾಜ-ವಂಶ-ವಾ-ಗಿತ್ತು
ರಾಜ-ವಡೇರ
ರಾಜ-ವರ್ತಕ
ರಾಜ-ವಿದ್ಯಾ-ಧರ
ರಾಜ-ವೊಡೆಯನ
ರಾಜ-ವೊಡೆಯ-ರಿಗೆ
ರಾಜ-ವೊಳಲ
ರಾಜ-ವೊಳಲಾಗಿರ-ಬಹುದು
ರಾಜಶ್ರೀ
ರಾಜ-ಸಭಾ-ಯೋಗ್ಯ-ನಾಗಿದ್ದನು
ರಾಜ-ಸಿಂಹಾಸ-ನ-ವನ್ನು
ರಾಜಾದಿತ್ಯನ
ರಾಜಾದಿತ್ಯ-ನನ್ನು
ರಾಜಾದಿತ್ಯನು
ರಾಜಾದಿತ್ಯನೂ
ರಾಜಾದಿತ್ಯರು
ರಾಜಾಧಿ-ರಾಜ
ರಾಜಾಧಿ-ರಾಜಃ
ರಾಜಾಧಿ-ರಾಜ-ಬಿರುದೋ
ರಾಜಾಧಿ-ರಾಜ-ಯಿತ್ಯುಕ್ತೋ
ರಾಜಾಧಿ-ರಾಜೇಂದ್ರ-ನೆಂದು
ರಾಜಾಧ್ಯಕ್ಷ
ರಾಜಾನ್ವ-ಯದೊಕ್ಕಲ
ರಾಜಾರ-ಮಡು
ರಾಜಾ-ರಾಮ
ರಾಜಾ-ವಳಿ
ರಾಜಾಸ್ಥಾನ-ದಲ್ಲಿ
ರಾಜೇಂದ್ರ
ರಾಜೇಂದ್ರ-ಚೋಳ
ರಾಜೇಂದ್ರ-ಚೋಳನ
ರಾಜೇಂದ್ರ-ಚೋಳನೇ
ರಾಜೇ-ಗೌಡ
ರಾಜೇನ್ದ್ರಚೋೞ
ರಾಜೇಶ್ವರಿ
ರಾಜೊಡೆಯರ
ರಾಜೋದ್ಯಾನ-ವನ-ವನ್ನು
ರಾಜ್ಯ
ರಾಜ್ಯಂ
ರಾಜ್ಯಂಗೆಯು-ತಂಮಿ-ರಲು
ರಾಜ್ಯಂಗೆಯುತ್ತಮಿರೆ
ರಾಜ್ಯಂಗೆಯುತ್ತಿರಲಿಕ್ಕಾಗಿ
ರಾಜ್ಯಂಗೆಯೆ
ರಾಜ್ಯಂಗೆಯ್ಯುತ್ತಮಿರೆ
ರಾಜ್ಯಂಗೆಯ್ಯುತ್ತಿದ್ದನು
ರಾಜ್ಯಂಗೆಯ್ಯುತ್ತಿದ್ದ-ರೆಂದು
ರಾಜ್ಯಂಗೆಯ್ಯೆ
ರಾಜ್ಯಂಗೈ-ಉತ್ತಮಿ-ರಲು
ರಾಜ್ಯ-ಕಾರ್ಯಕ್ಕಾಗಿ
ರಾಜ್ಯ-ಕಾರ್ಯ-ವನ್ನು
ರಾಜ್ಯಕ್ಕೀತಂ
ರಾಜ್ಯಕ್ಕೆ
ರಾಜ್ಯ-ಗತಂ
ರಾಜ್ಯ-ಗಳ
ರಾಜ್ಯ-ಗಳನ್ನಾಗಿ
ರಾಜ್ಯ-ಗಳನ್ನು
ರಾಜ್ಯ-ಗಳಲ್ಲಿ
ರಾಜ್ಯ-ಗಳಾಗಿ
ರಾಜ್ಯ-ಗಳಿಗೂ
ರಾಜ್ಯ-ಚಿಹ್ನೆಯೂ
ರಾಜ್ಯದ
ರಾಜ್ಯ-ದಲ್ಲಿ
ರಾಜ್ಯ-ದಲ್ಲಿತ್ತೆಂದು
ರಾಜ್ಯ-ದಿಂದ
ರಾಜ್ಯ-ದೇಶ-ಸೀಮೆ
ರಾಜ್ಯ-ದೊಳ್
ರಾಜ್ಯ-ಪಾಲ-ನಾಗಿದ್ದನು
ರಾಜ್ಯ-ಪಾಲ-ರು-ಗಳು
ರಾಜ್ಯ-ಭರ
ರಾಜ್ಯ-ಭಾರ
ರಾಜ್ಯ-ಭಾರಕ್ಕೆ
ರಾಜ್ಯ-ಭಾರದ
ರಾಜ್ಯಭ್ರಷ್ಟನನ್ನಾಗಿ
ರಾಜ್ಯಭ್ರಷ್ಟ-ನಾದ
ರಾಜ್ಯ-ಮಧ್ಯೇ
ರಾಜ್ಯ-ಲಕ್ಷ್ಮಿ
ರಾಜ್ಯ-ಲಕ್ಷ್ಮಿಯ
ರಾಜ್ಯ-ಲಕ್ಷ್ಮಿ-ಯನ್ನು
ರಾಜ್ಯ-ವನಾಳುತ್ತಮಿದ್ದ
ರಾಜ್ಯ-ವನ್ನಾಗಿ
ರಾಜ್ಯ-ವನ್ನು
ರಾಜ್ಯ-ವಾಳ-ತೊಡಗಿದ-ನೆಂದು
ರಾಜ್ಯ-ವಾಳದೇ
ರಾಜ್ಯ-ವಾಳ-ಲಿಲ್ಲ
ರಾಜ್ಯ-ವಾಳ-ಲಿಲ್ಲ-ವೆಂದು
ರಾಜ್ಯ-ವಾಳಲು
ರಾಜ್ಯ-ವಾಳಿದ
ರಾಜ್ಯ-ವಾಳಿ-ದನು
ರಾಜ್ಯ-ವಾಳಿ-ದ-ನೆಂದು
ರಾಜ್ಯ-ವಾಳಿ-ದ-ನೆಂದೂ
ರಾಜ್ಯ-ವಾಳಿ-ದ-ವನು
ರಾಜ್ಯ-ವಾಳುತ್ತಿದ್ದ
ರಾಜ್ಯ-ವಾಳುತ್ತಿದ್ದನು
ರಾಜ್ಯ-ವಾಳುತ್ತಿದ್ದ-ನೆಂದು
ರಾಜ್ಯ-ವಾಳುತ್ತಿದ್ದ-ರೆಂದು
ರಾಜ್ಯ-ವಾಳುತ್ತಿದ್ದಾಗ
ರಾಜ್ಯ-ವಿತ್ತು
ರಾಜ್ಯ-ವಿದ್ದಿತು
ರಾಜ್ಯ-ವಿಸ್ತರಣೆ-ಯನ್ನು
ರಾಜ್ಯವು
ರಾಜ್ಯ-ವೆಂದು
ರಾಜ್ಯ-ವೆಂಬ
ರಾಜ್ಯಶ್ರೀ
ರಾಜ್ಯ-ಸಂವತ್ಸರ-ದಲ್ಲಿ
ರಾಜ್ಯಸ್ಥಳಕ್ಕೆ
ರಾಜ್ಯಸ್ಥಾಪನೆ
ರಾಜ್ಯಾಡಳಿತ
ರಾಜ್ಯಾಡಳಿತ-ವನ್ನು
ರಾಜ್ಯಾಡಳಿ-ತವು
ರಾಜ್ಯಾಧಿಪ
ರಾಜ್ಯಾಧಿ-ಪತಿ
ರಾಜ್ಯಾಧಿಪ-ತಿ-ಗಳಾಗಿದ್ದ
ರಾಜ್ಯಾಧಿಪ-ತಿ-ಗಳು
ರಾಜ್ಯಾಧಿಪ-ತಿ-ಯಾಗಿದ್ದ
ರಾಜ್ಯಾಧಿಪ-ತಿ-ಯಾಗಿದ್ದ-ನೆಂದು
ರಾಜ್ಯಾಪಹಾರಕ್ಕೆ
ರಾಜ್ಯಾಭಿಷಿಕ್ತನಾ-ದ-ನೆಂದು
ರಾಜ್ಯಾಭಿಷೇಕ
ರಾಜ್ಯಾಭ್ಯುದ
ರಾಜ್ಯಾರೋಹ-ಣಕ್ಕೆ
ರಾಜ್ಯಾ-ವಾಳುತ್ತಿದ್ದನು
ರಾಣಾ-ಜಗ-ದೇವ-ರಾಯ
ರಾಣಾ-ಜಗ-ದೇವ-ರಾಯನ
ರಾಣಾ-ಜಗ-ದೇವ-ರಾಯನು
ರಾಣಾ-ಪೆದ್ದ
ರಾಣಾ-ವಂಶ-ದ-ವರು
ರಾಣಿ
ರಾಣಿ-ಮುಖಜ್ಯೋತಿ
ರಾಣಿಯ
ರಾಣಿ-ಯರ
ರಾಣಿ-ಯರು
ರಾಣಿಯು
ರಾಣಿ-ವಾಸ
ರಾಣೋಜಿ-ರಾವ್
ರಾತ್ರಿಯೊಳ್
ರಾಧನ-ಪುರ
ರಾಧಾ
ರಾಧಾ-ಪಟೇಲ್
ರಾಧೇಯ
ರಾಧೇಯ-ಕುಲ
ರಾಧೇಯನ
ರಾಮ
ರಾಮ-ಅ-ರಸಿ-ಕೆರೆಯ
ರಾಮ-ಕೃಷ್ಣ
ರಾಮ-ಗಉಡ
ರಾಮ-ಚಂದ್ರ-ದೇವರು
ರಾಮಣ್ಣ
ರಾಮಣ್ಣನು
ರಾಮ-ದೇವ
ರಾಮ-ದೇವನ
ರಾಮ-ದೇವನು
ರಾಮ-ದೇವ-ಮಹಾ-ರಾಯ
ರಾಮ-ದೇವರ
ರಾಮ-ದೇವ-ರಾಯ
ರಾಮ-ದೇವ-ರಾಯನು
ರಾಮ-ದೇವ-ರಿಗೆ
ರಾಮನ
ರಾಮ-ನ-ಗರ
ರಾಮ-ನ-ಬೆಂಕೊಂಡ-ಗಂಡ
ರಾಮ-ನ-ರಾಮ-ರಾ-ಜಯ್ಯ
ರಾಮ-ನ-ಹಳ್ಳಿ
ರಾಮ-ನಾಥ
ರಾಮ-ನಾಥ-ದೇವ-ರಿಗೆ
ರಾಮ-ನಾಥನ
ರಾಮ-ನಾಥ-ನಿಗೆ
ರಾಮ-ನಾಥ-ನೊಡನೆ
ರಾಮ-ನಾಥ-ಪುರದ
ರಾಮನು
ರಾಮನೇ
ರಾಮನ್ರಿಪ
ರಾಮ-ಪುರ
ರಾಮಪ್ಪ
ರಾಮ-ಭಟಯ್ಯ
ರಾಮ-ಭಟ್ಟ
ರಾಮ-ಭದ್ರಾ-ದೇವಿ
ರಾಮ-ಮಾತ್ಯ
ರಾಮ-ಯ-ರಾಯ
ರಾಮಯ್ಯ
ರಾಮ-ರಾಜ
ರಾಮ-ರಾಜ-ಅಯ್ಯ
ರಾಮ-ರಾಜ-ಅ-ಳಿಯ
ರಾಮ-ರಾಜ-ತಿರು-ಮಲ-ರಾಜಯ್ಯ-ನ-ವರ
ರಾಮ-ರಾಜ-ನ-ಸೇನಾನಿ
ರಾಮ-ರಾಜ-ನಾಯ-ಕ-ನಿಗೆ
ರಾಮ-ರಾಜ-ನಿಗೆ
ರಾಮ-ರಾಜ-ಯರ್ಸರು
ರಾಮ-ರಾ-ಜಯ್ಯ
ರಾಮ-ರಾಜಯ್ಯ-ದೇವ
ರಾಮ-ರಾಜಯ್ಯನ
ರಾಮ-ರಾಜಯ್ಯ-ನನ್ನು
ರಾಮ-ರಾಜಯ್ಯ-ನ-ವರ
ರಾಮ-ರಾಜಯ್ಯ-ನ-ವರು
ರಾಮ-ರಾಜಯ್ಯ-ನಿಗೆ
ರಾಮ-ರಾಜಯ್ಯನು
ರಾಮ-ರಾಜಯ್ಯನೂ
ರಾಮ-ರಾಜಯ್ಯ-ನೆಂದು
ರಾಮ-ರಾಜಯ್ಯನೇ
ರಾಮ-ರಾಜ-ವೊಡೆಯರು
ರಾಮ-ರಾಜು
ರಾಮ-ರಾಯ
ರಾಮ-ರಾಯನ
ರಾಮ-ರಾಯ-ನಾಯ-ಕ-ನಿಗೆ
ರಾಮ-ರಾಯನು
ರಾಮ-ರಾಯರೇ
ರಾಮ-ಲಕ್ಷ್ಮ-ಣರಂತಿದ್ದು
ರಾಮ-ಲಕ್ಷ್ಮ-ಣ-ರಂತೆ
ರಾಮ-ಲಕ್ಷ್ಮ-ಣರಿದ್ದಂತೆ
ರಾಮ-ಲಿಂಗ-ದೇವ-ರಿಗೆ
ರಾಮ-ಲಿಂಗೇಶ್ವರ
ರಾಮ-ಸಮುದ್ರ-ಗಳನ್ನು
ರಾಮಾ
ರಾಮಾ-ಜಯ್ಯ
ರಾಮಾ-ನುಜ
ರಾಮಾ-ನುಜ-ಕೂಟಕ್ಕೆ
ರಾಮಾ-ನುಜ-ಕೂಟ-ವನ್ನು
ರಾಮಾ-ನುಜ-ಜೀಯರ್
ರಾಮಾ-ನುಜನ
ರಾಮಾ-ನುಜ-ಪುರಂ
ರಾಮಾ-ನುಜಯ್ಯ-ನಿಗೆ
ರಾಮಾ-ನುಜರ
ರಾಮಾ-ನುಜಾ-ಚಾರ್ಯರ
ರಾಮಾ-ನುಜಾ-ಚಾರ್ಯರನ್ನು
ರಾಮಾ-ನುಜಾ-ಚಾರ್ಯರಿಗೆ
ರಾಮಾ-ನುಜಾ-ಚಾರ್ಯರು
ರಾಮಾ-ಭಟನ
ರಾಮಾ-ಭಟ-ನಿಗೆ
ರಾಮಾ-ಭಟಯ್ಯನ-ವರಿಗೆ
ರಾಮಾ-ಭಟಯ್ಯ-ನಿಗೆ
ರಾಮಾ-ಭಟಯ್ಯನು
ರಾಮಾ-ಭಟ್ಟ
ರಾಮಾ-ಭಟ್ಟನು
ರಾಮಾ-ಭಟ್ಟಯ್ಯನ
ರಾಮಾ-ಭಟ್ಟಯ್ಯನ-ವ-ರಿಂದ
ರಾಮಾ-ಭಟ್ಟಯ್ಯ-ನಿಗೆ
ರಾಮಾ-ಭಟ್ಟಯ್ಯನು
ರಾಮಾ-ಭಟ್ಟರ
ರಾಮಾ-ಯಣ-ಪೂರ್ವ್ವಕ
ರಾಮೆಯ-ನಾಯ-ಕನು
ರಾಮೆಯ-ನಾಯ್ಕ
ರಾಮೇಶ್ವರ
ರಾಮೇಶ್ವರದ
ರಾಮೇಶ್ವರ-ದಲ್ಲಿ
ರಾಮೇಶ್ವರ-ದ-ವರೆಗೂ
ರಾಯ
ರಾಯಂಣ
ರಾಯ-ಕುಮಾರರ
ರಾಯ-ಣ-ನಾಯ-ಕನು
ರಾಯಣ್ಣ
ರಾಯಣ್ಣ-ದಂಡ-ನಾಥನ
ರಾಯಣ್ಣ-ನಾಯ-ಕನ
ರಾಯನ
ರಾಯ-ನಿಂದ
ರಾಯ-ನಿಗೆ
ರಾಯನು
ರಾಯಪ್ಪ
ರಾಯ-ಭಾಟ
ರಾಯರ
ರಾಯ-ರ-ಕುಮಾರರ
ರಾಯ-ರ-ಗಂಡ
ರಾಯ-ರಾಯ
ರಾಯ-ರಿಗೆ
ರಾಯರು
ರಾಯ-ರೊಳು
ರಾಯ-ವೊಡೆಯ
ರಾಯ-ವೊಡೆಯರು
ರಾಯಸ
ರಾಯ-ಸದ
ರಾಯ-ಸ-ದ-ವನು
ರಾಯ-ಸ-ದ-ವರ
ರಾಯ-ಸ-ದ-ವ-ರಾದ
ರಾಯ-ಸ-ದ-ವ-ರಿಗೆ
ರಾಯ-ಸ-ದ-ವರು
ರಾಯ-ಸಮುದ್ರ
ರಾಯ-ಸವು
ರಾಯ-ಸಸ್ವಾಮಿ
ರಾಯ-ಸೆಟ್ಟಿ-ಪುರ
ರಾಯಾಂಬಾಮುಲ
ರಾಯೊಡೆ-ಯರ
ರಾವಂದೂ-ರಿಗೆ
ರಾವಂದೂರಿನ
ರಾವಂದೂರು
ರಾವಂದೂರು-ಗಳು
ರಾವಿಯ-ಹಾಳೆಯ-ಮಲ್ಲಿ-ನಾಥ-ಪುರ
ರಾವುತ
ರಾವು-ತರ
ರಾವು-ತ-ರಾಯ-ನು-ದಗ್ರದೊರ್ವ್ವಳಂ
ರಾವು-ತರು
ರಾವುತ್ತ-ರಾಯ
ರಾವುಳ
ರಾವ್ಬಹದ್ದೂರ್
ರಾಷ್ಟ್ರ
ರಾಷ್ಟ್ರಕ
ರಾಷ್ಟ್ರ-ಕೂಟ
ರಾಷ್ಟ್ರ-ಕೂಟರ
ರಾಷ್ಟ್ರ-ಕೂಟ-ರನ್ನು
ರಾಷ್ಟ್ರ-ಕೂಟ-ರಿಂದ
ರಾಷ್ಟ್ರ-ಕೂಟ-ರಿಗೂ
ರಾಷ್ಟ್ರ-ಕೂಟರು
ರಾಷ್ಟ್ರ-ಕೂಟರೇ
ರಾಷ್ಟ್ರ-ಕೂಟ-ರೊಡನೆ
ರಾಷ್ಟ್ರದ
ರಾಷ್ಟ್ರ-ವನ್ನು
ರಾಷ್ಟ್ರ-ವೆಂದು
ರಾಸಕ್ಕಲಿನ
ರಿಂದ
ರಿಂದಲೂ
ರಿಂದಲೇ
ರಿಣಕ್ಕೆ
ರಿಪು-ರಾಮ-ಗಾಮುಣ್ಡರು
ರಿಪುಸ್ತೋಮ
ರಿಪುಸ್ತೋಮ-ಕರಿ
ರಿಪೋರ್ಟ್ನಲ್ಲಿ
ರಿಪೋರ್ಟ್ರಲ್ಲಿ
ರಿಯಾಯಿತಿ
ರೀತಿ
ರೀತಿಯ
ರೀತಿ-ಯಲ್ಲಿ
ರೀತಿ-ಯಾಗಿ
ರುಕ್ಶಾಖಾಧ್ಯಾಯಿ-ಗ-ಳಾದ
ರುಖಯ್ಯಾಬೀಬಿಯ
ರುದ್ರಣ್ಣ
ರುದ್ರ-ದಂಡಾಧೀಶ-ನಿಗೆ
ರುದ್ರ-ದೇವಾತ್ಮಜಂ
ರುದ್ರ-ಭಟ್ಟ
ರುದ್ರ-ಮುನಿ
ರುದ್ರ-ಸಮುದ್ರ
ರುಧಿರೋದ್ಗಾರಿ
ರೂಡಿ
ರೂಡಿ-ವಡೆದ
ರೂಢಿ
ರೂಢಿಯ
ರೂಢಿ-ಯಲ್ಲಿದೆ
ರೂಪ
ರೂಪ-ಗಳು
ರೂಪ-ಗಳೇ
ರೂಪದ
ರೂಪ-ದಲ್ಲಿ
ರೂಪ-ದಲ್ಲಿತ್ತೆಂದು
ರೂಪ-ವುಳ್ಳ-ವನೂ
ರೂಪವೇ
ರೂಪಾಂತರ-ವಾಯಿತು
ರೂಪಾಂತರ-ವಾ-ಯಿತೆಂದು
ರೂಪಿ-ಸ-ಲಾ-ಯಿತೆಂದು
ರೂಪಿ-ಸಿದನು
ರೂಪು-ಗೊಂಡ
ರೂಪು-ಗೊಳ್ಳುವು-ದಕ್ಕೆ
ರೆಂಡಿ-ಷನ್
ರೆಜಿಮೆಂಟ್
ರೆಡ್ಡಿ-ಯ-ವರು
ರೆಸಿಡೆಂಟ್
ರೇಕವ್ವೆ
ರೇಕವ್ವೆಯ
ರೇಕಾ-ದೇವಿ
ರೇಚಣ್ಣ
ರೇಚೆಯ
ರೇಮಟಿ-ವೆಂಕಟ-ನನ್ನು
ರೇಮೇ
ರೇವಂತ
ರೇವಕ
ರೇವಕ-ನಿಮ್ಮಡಿ-ಯನ್ನು
ರೇವಕ್ಕ
ರೇವಕ್ಕ-ನಿರ್ಮಡಿ-ಯನ್ನು
ರೇವಣಯ್ಯ
ರೇವಣಾರಾಧ್ಯ
ರೇವಲಾಕಲ್ಪ-ವಲ್ಲಿ
ರೇವಲಾ-ದೇವಿಯ
ರೇವಲಾ-ದೇವಿ-ಯನ್ನು
ರೇವ-ಲೇಶ್ವರ
ರೈಟ್ಹ್ಯಾಂಡ್
ರೈತ-ರನ್ನು
ರೈತರು
ರೈಲ್ವೆ
ರೈಸ್
ರೈಸ್ರ-ವರ
ರೈಸ್ರ-ವರು
ರೋಣ-ಶಾ-ಸನವು
ಲಂಕಪ್ಪ
ಲಂಕೆಯ-ವ-ರೆಗೆ
ಲಂಭ-ಹಸ್ತ-ಗಳಅ
ಲಕು-ಮಯ್ಯ
ಲಕು-ಮಯ್ಯ-ಗಳ
ಲಕು-ಮಯ್ಯನು
ಲಕು-ಮಯ್ಯರು
ಲಕ್ಕಣ್ಣ
ಲಕ್ಕಣ್ಣ-ದಂಡ-ನಾಯ-ಕನು
ಲಕ್ಕಣ್ಣ-ದಂಡ-ನಾಯ-ಕರ
ಲಕ್ಕಣ್ಣ-ದಂಡೇಶ
ಲಕ್ಕಣ್ಣ-ದಂಡೇಶನ
ಲಕ್ಕಣ್ಣ-ದಂಡೇಶ-ನನ್ನು
ಲಕ್ಕಣ್ಣ-ನನ್ನು
ಲಕ್ಕಣ್ಣ-ನಾಯ-ಕರ
ಲಕ್ಕಮ್ಮ
ಲಕ್ಕವ್ವೆ
ಲಕ್ಕಿದೊಣೆಜಿ-ನದೊ-ಣೆಯ
ಲಕ್ಕಿ-ಯೂರು
ಲಕ್ಕುಂಡಿಯ-ತನಕ
ಲಕ್ಕೂರು
ಲಕ್ಷ
ಲಕ್ಷಣ
ಲಕ್ಷಮ್ಮಮ್ಮಣಿಯು
ಲಕ್ಷೋಪ-ಲಕ್ಷ
ಲಕ್ಷ್ಮ
ಲಕ್ಷ್ಮ-ಣ-ದಾಸ
ಲಕ್ಷ್ಮ-ಣಾಧ್ವರಿಯ
ಲಕ್ಷ್ಮನ
ಲಕ್ಷ್ಮಾ-ದೇವಿ-ಯರ
ಲಕ್ಷ್ಮಿ
ಲಕ್ಷ್ಮೀ-ಕಾಂತ
ಲಕ್ಷ್ಮೀ-ಕಾಂತ-ದೇವರ
ಲಕ್ಷ್ಮೀ-ಕಾಂತ-ದೇವಾಲಯದ
ಲಕ್ಷ್ಮೀ-ಜನಾರ್ದನ
ಲಕ್ಷ್ಮೀ-ದೇವರ
ಲಕ್ಷ್ಮೀ-ದೇವಿ
ಲಕ್ಷ್ಮೀ-ದೇ-ವಿಗೆ
ಲಕ್ಷ್ಮೀ-ದೇವಿಯೇ
ಲಕ್ಷ್ಮೀ-ನರ-ಸಿಂಹ
ಲಕ್ಷ್ಮೀ-ನಾ-ರಾಯಣ
ಲಕ್ಷ್ಮೀ-ನಾ-ರಾಯ-ಣ-ದಂಡ-ನಾಯಕ
ಲಕ್ಷ್ಮೀ-ನಾ-ರಾಯ-ಣ-ದೇ-ವ-ರಿಗೆ
ಲಕ್ಷ್ಮೀ-ನಾ-ರಾಯ-ಣ-ರಾವ್
ಲಕ್ಷ್ಮೀ-ಪತಿಯ
ಲಕ್ಷ್ಮೀ-ಭೂ-ವರಾಹ-ನಾಥ
ಲಕ್ಷ್ಮೀ-ಮತಿ-ದಂಡ-ನಾಯ-ಕಿತ್ತಿ
ಲಕ್ಷ್ಮೀ-ಮತಿ-ಯಿಂದ
ಲಕ್ಷ್ಮೀ-ಮುದೇ
ಲಕ್ಷ್ಮೀ-ಸಾ-ಗರ
ಲಕ್ಷ್ಮೀ-ಸಾ-ಗರ-ವನ್ನು
ಲಕ್ಷ್ಮೀ-ಸೇನ
ಲಖಂಣ
ಲಖಂಣನ
ಲಖಂಣ-ವೊಡೆಯರ
ಲಖಣ್ಣ-ವೊಡೆಯ
ಲಖಪ-ನಾಯ-ಕರ
ಲಖ್ಖೆಯ
ಲಗ್ಗೆ
ಲಚ್ಚಣ್ಣ
ಲಚ್ಚಿಯ-ನಾಯಕ
ಲಭ್ಯ-ವಾಗಿ
ಲಭ್ಯ-ವಾಗಿದ್ದು
ಲಭ್ಯ-ವಾಗಿವೆ
ಲಭ್ಯ-ವಾಗುತ್ತಿತ್ತು
ಲಭ್ಯ-ವಾದ
ಲಯಕಾಳ
ಲಲಾಮ
ಲಲಾಮಂ
ಲಲ್ಲ
ಲವಕುಶ-ರಂತೆ
ಲಸದ್ದೋರ್ದಣ್ಡ-ದೊಳ್ಸಂತೋಷಂ
ಲಾಕ್ಷಾಗೃಹೋಪಾಯಮುಂ
ಲಾಭ
ಲಾಭ-ಪಡೆ-ಯಲು
ಲಾಳ-ನ-ಕೆರೆ
ಲಾಳ-ನ-ಕೆರೆಯ
ಲಾವಣಿ-ಗಳನ್ನು
ಲಾವಣಿ-ಗಳು
ಲಿಂಗಕ್ಕೆ
ಲಿಂಗ-ಗವುಡನು
ಲಿಂಗಣ್ಣ
ಲಿಂಗಣ್ಣೊಡೆ-ಯನು
ಲಿಂಗ-ದೇವರು
ಲಿಂಗ-ಪಯ್ಯ
ಲಿಂಗ-ಪಯ್ಯನು
ಲಿಂಗಪ್ಪ-ಗವುಡ
ಲಿಂಗಪ್ಪ-ನಾಯ-ಕನ
ಲಿಂಗಾಚಾರಿಯ
ಲಿಂಗಾಜಮ್ಮಣ್ಣಿ-ಯ-ವರು
ಲಿಂಗಾ-ಪುರ
ಲಿಂಗಾ-ಭಟ್ಟರ
ಲಿಖಿತ-ಮೂಲ-ಗಳಿಂದ
ಲಿಪಿ
ಲಿಪಿಯ
ಲಿಪಿ-ಯಲ್ಲಿದೆ
ಲಿಪಿ-ಯಲ್ಲಿದ್ದು
ಲಿಪಿ-ಯಲ್ಲಿರುವ
ಲಿಪಿ-ಯಲ್ಲಿವೆ
ಲಿಪಿ-ಸಂಸ್ಕೃತ-ತಮಿಳು
ಲು
ಲೆಂಕ
ಲೆಂಕಂಕ
ಲೆಂಕ-ನಿಸ್ಸಂಕ-ನಾದ
ಲೆಂಕ-ಮಹಾ-ದೇವ
ಲೆಂಕರ
ಲೆಂಕ-ರಾಗಿ
ಲೆಂಕ-ರಾಗಿದ್ದರು
ಲೆಂಕ-ರಾಗಿದ್ದ-ರೆಂಬ
ಲೆಂಕ-ರಿದ್ದು
ಲೆಂಕರು
ಲೆಂಕ-ವಾಳಿ-ಯನ್ನು
ಲೆಂಕಿತಿ-ಯರು
ಲೆಕ್ಕ
ಲೆಕ್ಕದ
ಲೆಕ್ಕ-ದಲ್ಲಿ
ಲೆಕ್ಕ-ಪತ್ರ
ಲೆಕ್ಕ-ಪತ್ರ-ಗಳ
ಲೆಕ್ಕ-ಪತ್ರ-ಗಳನ್ನು
ಲೆಕ್ಕ-ಪತ್ರದ
ಲೆಕ್ಕ-ಪತ್ರ-ದಲ್ಲಿ
ಲೆಕ್ಕ-ಬ-ರೆಯುವ
ಲೆಕ್ಕ-ವನ್ನು
ಲೆಕ್ಕ-ಹಾಕಿ
ಲೆಕ್ಕ-ಹಾಕಿ-ದರೆ
ಲೆಕ್ಕ-ಹಾಕಿದ್ದಾರೆ
ಲೆಕ್ಕಾಚಾರ
ಲೆಕ್ಕಾಚಾರ-ವನ್ನು
ಲೆಕ್ಕಾಚಾರ-ಹಾಕಿ
ಲೆಕ್ಕಾಧಿ-ಕಾರಿ-ಗ-ಳೆಂದು
ಲೆಕ್ಕಿ-ಸದೆ
ಲೇಖ-ಕನು
ಲೇಖ-ಗಳು
ಲೇಖ-ನ-ಗಳಾಗಿವೆ
ಲೇಖ-ನ-ಗಳು
ಲೇಖ-ನ-ಗಳೆಲ್ಲ-ವನ್ನೂ
ಲೇಖ-ನ-ದಲ್ಲಿ
ಲೊಕ್ಕಾನೆ
ಲೊಕ್ಕಿಗುಂಡಿಯ
ಲೊಕ್ಕಿಗುಂಡಿ-ಯನ್ನು
ಲೊಕ್ಕಿಗುಂಡಿ-ಯಲ್ಲಿ
ಲೊಕ್ಕಿಯ-ಹಳ್ಳಿ
ಲೋಕ
ಲೋಕಕ್ಕೆ
ಲೋಕ-ತಿಲಕ-ಜಿನಭ-ವನಕ್ಕೆ
ಲೋಕ-ತಿಲಕ-ಭ-ವನ-ವೆಂಬ
ಲೋಕ-ನ-ಹಳ್ಳಿ
ಲೋಕ-ಪಾ-ವನಿ
ಲೋಕ-ಪಾ-ವನೆಗೆ
ಲೋಕಪ್ರ-ಸಿದ್ಧ-ನಾಗಿದ್ದ
ಲೋಕ-ವಿದ್ಯಾ-ಧರ
ಲೋಕ-ವಿದ್ಯಾ-ಧ-ರನು
ಲೋಕಾಂತಸಿಮ್ನಿ
ಲೋಕಾಂಬಿಕೆ
ಲೋಕಾನಾಂಚ
ಲೋಕೋಪ-ಕಾರ-ದಲ್ಲಿ
ಲೋಚೆರ್ಲ
ಲೋಭಿ-ರಾಯ
ಲೋಹದ
ಲೋಹಿತ
ಲೋಹಿತ-ಕುಲ-ಶೇಖರ
ಲ್ದಾತಂ
ಲ್ದೊರೆ-ವಿತ್ತೀ
ಳಂಬಿತ
ಳೆಂದುಂ
ಳ್ತಿರಿದುದ
ವಂಕಣ-ಪಲ್ಲಿ
ವಂಗರು
ವಂದಿಸಲ್ಪಡುತ್ತಿದ್ದನು
ವಂಶ
ವಂಶಕ್ಕೆ
ವಂಶ-ಗಳು
ವಂಶ-ಜ-ನಾಗಿ-ರುವ
ವಂಶ-ಜನಿರ-ಬಹು-ದೆಂದು
ವಂಶ-ಜ-ರನ್ನು
ವಂಶ-ಜ-ರಾದ
ವಂಶ-ಜರು
ವಂಶದ
ವಂಶ-ದಲ್ಲಿ
ವಂಶ-ದ-ವ-ನಾಗಿದ್ದಾನೆ
ವಂಶ-ದ-ವ-ನಾಗಿ-ರ-ಬಹುದು
ವಂಶ-ದ-ವನಿರ-ಬಹುದು
ವಂಶ-ದ-ವನಿರ-ಬಹು-ದೆಂದು
ವಂಶ-ದ-ವನು
ವಂಶ-ದ-ವ-ನೆಂದು
ವಂಶ-ದ-ವನೇ
ವಂಶ-ದ-ವರ
ವಂಶ-ದ-ವ-ರಾಗಿ-ರ-ಬಹುದು
ವಂಶ-ದ-ವ-ರಾದ
ವಂಶ-ದ-ವ-ರಿಂದಲೇ
ವಂಶ-ದ-ವ-ರಿಗೂ
ವಂಶ-ದ-ವ-ರಿಗೆ
ವಂಶ-ದ-ವರಿ-ರ-ಬಹುದು
ವಂಶ-ದ-ವರು
ವಂಶ-ದ-ವ-ರೆಂದು
ವಂಶ-ದ-ವ-ರೆಂಬ
ವಂಶ-ದ-ವರೇ
ವಂಶ-ದೊಡನೆ
ವಂಶ-ಪಾರಂಪರ್ಯ
ವಂಶ-ಪಾರಂಪರ್ಯ-ದಿಂದ
ವಂಶ-ಪಾರಂಪರ್ಯ-ವಾಗಿ
ವಂಶ-ಪಾರಂಪರ್ಯ-ವಾ-ಗಿತ್ತು
ವಂಶ-ಪಾರಂಪರ್ಯ-ವಾಗಿತ್ತೆಂದು
ವಂಶ-ಪಾರಂಪರ್ಯ-ವಾಗಿದ್ದಂತೆ
ವಂಶ-ಪಾರಂಪರ್ಯ-ವಾಗಿದ್ದರೂ
ವಂಶ-ಪಾರಂಪರ್ಯ-ವಾಗಿಯೂ
ವಂಶ-ಪಾರಂಪರ್ಯ-ವಾದ
ವಂಶ-ಮೌಕ್ತಿಕ
ವಂಶ-ವನ್ನು
ವಂಶವು
ವಂಶ-ವೃಕ್ಷ
ವಂಶ-ವೃಕ್ಷ-ಗಳ
ವಂಶ-ವೃಕ್ಷ-ದಲ್ಲಿ
ವಂಶ-ವೃಕ್ಷ-ವನ್ನು
ವಂಶ-ವೆಂದರೆ
ವಂಶ-ವೊಂದು
ವಂಶಸ್ಥ-ನಾದ
ವಂಶಸ್ಥನಿರ-ಬಹುದು
ವಂಶಸ್ಥ-ರನ್ನು
ವಂಶಸ್ಥ-ರಾದ
ವಂಶಸ್ಥರಿರ-ಬಹುದು
ವಂಶಸ್ಥರು
ವಂಶಸ್ಥರೋ
ವಂಶಾ-ವಳಿ
ವಂಶಾ-ವಳಿ-ಗಳನ್ನು
ವಂಶಾವ-ಳಿಯ
ವಂಶಾ-ವಳಿ-ಯನ್ನು
ವಂಶಾ-ವಳಿ-ಯನ್ನೂ
ವಂಶಾ-ವಳಿ-ಯಲ್ಲಿ
ವಂಶಾ-ವಳಿ-ಯಲ್ಲಿಯೂ
ವಂಶಾ-ವಳಿ-ಯಿಂದ
ವಂಶಾವ-ಳಿಯು
ವಂಶೋದ್ಭವ
ವಂಶೋದ್ಭವರು
ವಕ್ತೃ
ವಕ್ತೃಪ್ರಯೋಕ್ತೃ
ವಕ್ತ್ರಾಬ್ಜ
ವಕ್ಷಸ್ಥಲ
ವಚಃ
ವಚನ-ದಂತೆ
ವಚನ-ವನ್ನೂ
ವಚನಶತ-ಸಹಸ್ರ
ವಜ್ಜಲ-ದೇವ
ವಜ್ರದ-ಪುಡಿ-ಯನ್ನು
ವಜ್ರಪಂಜರ
ವಜ್ರಪಂಜರರುಂ
ವಜ್ರಬೈ-ಸಣಿ-ಗೆಯ-ನಿಕ್ಕಿ
ವಟವಾಪಿ
ವಡ-ಗೆರೆ
ವಡ-ಗೆರೆ-ನಾಡು
ವಡುಗ-ಪಿಳ್ಳೆಯು
ವಡುಗಪಿಳ್ಳೈ
ವಡುಗ-ವೇಳೆ-ಕಾರ
ವಡೇರ
ವಡೈ-ಯರೈಯ-ನ-ವರು
ವಡ್ಡರ
ವಡ್ಡವ್ಯವ-ಹಾರಿ
ವಡ್ರಬಿಳಿ-ಕೆರೆ
ವದಾನ್ಯತಃ
ವದ್ದೆಗ
ವದ್ದೆಗ-ನೆಂದು
ವನ-ಜಾ-ತಾಯತ
ವನ-ಮಾಲೆಗೆ
ವನ-ವಾಸಿ
ವನ-ವೇಲಿ
ವನಾಂತರ-ದಲ್ಲಿ
ವನಿತಾದೂರಂ
ವನಿ-ವಾರ್ದ್ಧಿಸುಧಾ-ಕರ
ವನ್ನು
ವನ್ಯಧಾಮ
ವಯಸ್ಸಿನ
ವಯಸ್ಸಿ-ನಲ್ಲಿ
ವಯೋವೃದ್ಧ-ನಾಗಿ-ರ-ಬಹುದು
ವರ-ಕೀರ್ತ್ತಿಯಂ
ವರದ
ವರ-ದಣ್ಣ-ನಾಯ-ಕನು
ವರ-ದಯ್ಯ
ವರ-ದ-ರಾಜ
ವರ-ದ-ರಾಜ-ದೇವರ
ವರ-ದ-ರಾಜ-ಪುರ-ವೆಂಬ
ವರ-ದ-ರಾಜ-ಪೆರುಮಾಳ್
ವರ-ದ-ರಾಜಯ್ಯ-ನೆಂಬ
ವರ-ದ-ರಾಜ-ಸಮುದ್ರ-ವೆಂಬ
ವರ-ದ-ರಾಜಸ್ವಾಮಿ
ವರ-ದ-ರಾಜಸ್ವಾಮಿ-ಯ-ವರ
ವರ-ದಾಚಾರ್ಯನ
ವರ-ದಿ-ಯಂತೆ
ವರ-ದೆಯ
ವರ-ದೆಯ-ನಾಯಕ
ವರ-ದೆಯ-ನಾಯ-ಕನು
ವರ-ದೆಯ-ನಾಯ-ಕನೂ
ವರ-ದೆಯ-ನಾಯು-ಕನೆಂದೂ
ವರನೊಳು
ವರಪ್ರಸಾದ
ವರ-ಭುಜ
ವರ-ಮಂತ್ರ-ಶಕ್ತಿ-ಯುತ-ನಿಂದ್ರಗೆಂತು
ವರ-ಮಂತ್ರಿ-ವಲ್ಲಭ
ವರಹ
ವರ-ಹಕ್ಕೆ
ವರ-ಹ-ಗಳನ್ನು
ವರ-ಹ-ಗ-ಳಿಗೆ
ವರ-ಹ-ವನ್ನು
ವರ-ಹಾ-ನಾಥ
ವರ-ಹಾ-ನಾಥ-ಕಲ್ಲ-ಹಳ್ಳಿಗೆ
ವರ-ಹೀ-ಳನ-ಹಳ್ಳಿ
ವರಾ-ಹನ-ಕಲ್ಲ-ಹಳ್ಳಿ
ವರಾಹ-ನಾಥ
ವರಾಹ-ನಾಥ-ಕಲ್ಲ-ಹಳ್ಳಿಯ
ವರಾಹ-ನಾಥನ
ವರಾಹಮುದ್ರೆಯ
ವರಾಹಸ್ತುತಿ
ವರಿ-ಸ-ದನ್ದು
ವರಿಸಿ
ವರಿ-ಸಿದ್ದ
ವರಿ-ಸಿದ್ದನು
ವರಿ-ಸಿದ್ದು
ವರೆಗೂ
ವರೆಗೆ
ವರ್ಗಕ್ಕೆ
ವರ್ಗದ
ವರ್ಗದ-ವರೂ
ವರ್ಗ-ವನ್ನಾಗಿ
ವರ್ಗ-ವಾಗಿ
ವರ್ಗಾಯಿ-ಸ-ಲಾಯಿತು
ವರ್ಗಾಯಿಸಿ
ವರ್ಗಾಯಿಸಿ-ದ-ನೆಂದು
ವರ್ಗಾಯಿಸಿ-ರ-ಬಹು-ದೆಂದು
ವರ್ಣದ-ವ-ರಾದ
ವರ್ಣನಾ
ವರ್ಣನೆ
ವರ್ಣ-ನೆಯ
ವರ್ಣನೆ-ಯನ್ನು
ವರ್ಣನೆ-ಯಲ್ಲಿ
ವರ್ಣನೆ-ಯಿಂದಲೇ
ವರ್ಣ-ನೆಯು
ವರ್ಣನೆ-ಯೊಂದಿಗೆ
ವರ್ಣಿ-ಸ-ಲಾಗಿದೆ
ವರ್ಣಿ-ಸಿದೆ
ವರ್ಣಿ-ಸಿದ್ದು
ವರ್ಣಿಸಿವೆ
ವರ್ಣಿ-ಸುತ್ತದೆ
ವರ್ಣಿ-ಸುತ್ತವೆ
ವರ್ಣಿ-ಸುವ
ವರ್ತಕ
ವರ್ತಕರು
ವರ್ತಕ-ಸಂಘದ
ವರ್ತಕ-ಸಮುದಾ-ಯದ-ವ-ರಾಗಿ
ವರ್ತ್ತ-ಮಾನ-ರಾಯ
ವರ್ಧ-ಮಾನಾಪ-ದಾನಃ
ವರ್ಮ-ನನ್ನು
ವರ್ಷ
ವರ್ಷಂಪ್ರತಿ
ವರ್ಷಕ್ಕಿಂತ
ವರ್ಷಕ್ಕೂ
ವರ್ಷಕ್ಕೆ
ವರ್ಷ-ಗಳ
ವರ್ಷ-ಗಳಲ್ಲಿ
ವರ್ಷ-ಗಳಿಂದ
ವರ್ಷದ
ವರ್ಷ-ದಂತೆ
ವರ್ಷ-ದಲ್ಲಿ
ವರ್ಷ-ದಲ್ಲಿಯೇ
ವರ್ಷ-ವನ್ನು
ವರ್ಷ-ವಾಗಿ-ಬಿಡುತ್ತದೆ
ವರ್ಷ-ವೆಂದು
ವರ್ಷವೇ
ವರ್ಷೆ
ವರ್ಷೇ
ವಲಯ-ಗಳು
ವಲಸೆ
ವಲ್ಲಭನು
ವಳ-ನಾ-ಡಿನಲ್ಲಿದ್ದ
ವಳ-ಬಾ-ಗಿಲ
ವಳ-ಭೀ-ಪುರ-ವರೇಶ್ವರ
ವಳ-ಭೀ-ಪುರೇಶ್ವರ
ವಳಿತ
ವಳಿ-ತಕ್ಕೆ
ವಳಿತ-ಗ-ಳೆಂದು
ವಳಿತದ
ವಶಕ್ಕೆ
ವಶ-ದಲ್ಲಿ
ವಶ-ದಲ್ಲಿತ್ತು
ವಶಪಡಿಸ-ಕೊಂಡ
ವಶ-ಪಡಿಸಿ-ಕೊಂಡ
ವಶ-ಪಡಿಸಿ-ಕೊಂಡದ್ದು
ವಶ-ಪಡಿಸಿ-ಕೊಂಡನು
ವಶ-ಪಡಿಸಿ-ಕೊಂಡ-ನೆಂದು
ವಶ-ಪಡಿಸಿ-ಕೊಂಡರು
ವಶ-ಪಡಿಸಿ-ಕೊಂಡಿದ್ದನು
ವಶ-ಪಡಿಸಿ-ಕೊಂಡಿದ್ದ-ನೆಂದು
ವಶ-ಪಡಿಸಿ-ಕೊಂಡಿದ್ದ-ನೆಂಬುದು
ವಶ-ಪಡಿಸಿ-ಕೊಂಡಿದ್ದ-ರಿಂದ
ವಶ-ಪಡಿಸಿ-ಕೊಂಡಿದ್ದು
ವಶ-ಪಡಿಸಿ-ಕೊಂಡಿರ-ಬಹುದು
ವಶ-ಪಡಿಸಿ-ಕೊಂಡು
ವಶ-ಪಡಿಸಿ-ಕೊಳ್ಳಲು
ವಶ-ಪಡಿಸಿ-ಕೊಳ್ಳುವುದ-ರಲ್ಲಿ
ವಶ-ವಾಗಿ-ರ-ಬಹುದು
ವಶ-ವಾಗಿ-ರ-ಲಿಲ್ಲ
ವಸಂತ-ಲಕ್ಷ್ಮಿ-ಯ-ವರ
ವಸಂತೋತ್ಸವ
ವಸಿಷ್ಠ-ಗೋತ್ರೋದ್ಭ-ವನೂ
ವಸುಂಧರಾ
ವಸುಂಧರಾ-ಫಿ-ಲಿಯೋಜಾ
ವಸುಂಧರೆ-ಯನ್ನು
ವಸೂಲಿ
ವಸೂಲು
ವಸೂಲು-ಮಾಡಿ
ವಸ್ತು-ಗಳ
ವಸ್ತು-ಗಳಿ-ಗಾಗಿ
ವಸ್ತುವಂ
ವಸ್ತು-ವನ್ನು
ವಸ್ತು-ವಾ-ಹನ
ವಸ್ತುವಿಸ್ತಾ-ರನುಂ
ವಸ್ತ್ರ-ವನ್ನು
ವಹನ್
ವಹಿ-ಸ-ಲಾಯಿತು
ವಹಿಸಿ
ವಹಿಸಿ-ಕೊಂಡನು
ವಹಿಸಿ-ಕೊಂಡ-ನೆಂದು
ವಹಿಸಿ-ಕೊಂಡ-ಮೇಲೆ
ವಹಿಸಿ-ಕೊಂಡಿದ್ದ-ನೆಂದು
ವಹಿಸಿ-ಕೊಂಡಿರ-ಬಹುದು
ವಹಿಸಿ-ಕೊಂಡಿರ-ಬಹು-ದೆಂದು
ವಹಿಸಿದ
ವಹಿಸಿದ್ದ
ವಹಿಸಿದ್ದನು
ವಹಿಸಿದ್ದ-ನೆಂದು
ವಹಿಸಿದ್ದಾರೆ
ವಹಿಸಿ-ರ-ಬಹು-ದೆಂದು
ವಹಿಸಿ-ರು-ವಂತೆ
ವಹಿಸಿ-ರು-ವುದು
ವಹಿ-ಸುತ್ತಿದ್ದ
ವಹಿ-ಸುತ್ತಿದ್ದರು
ವಾಂಡಿವಾಷ್ನಲ್ಲಿ
ವಾಂತಿಭ್ರಾಂತಿ
ವಾಕ್ಯವು
ವಾಕ್ಯವೂ
ವಾಗೀಶ್ವರ-ಮಂಗಲ
ವಾಗೀಶ್ವರ-ಮಂಗಲದ
ವಾಚಕ
ವಾಚಕ-ದಿಂದಅ
ವಾಜಿ
ವಾಜಿ-ಕುಲ-ತಿಲಕ-ನಾಗಿದ್ದು
ವಾಜಿ-ಕುಲ-ತಿಲಕ-ನಾದ
ವಾಜಿ-ಕುಲದ
ವಾಜಿ-ವಂಶ-ದ-ವರೇ
ವಾಡಕ್ಕೆ-ಘಟ್ಟ-ಹೊಡಾ-ಘಟ್ಟ
ವಾಡಿಕೆ
ವಾಣ-ಸತ್ತಿ
ವಾಣಿಜ್ಯ
ವಾತಾ-ವರ-ಣವು
ವಾದ
ವಾದಕ್ಕೆ
ವಾದ-ಗಳಿವೆ
ವಾದ-ವಿ-ವಾದ-ಗಳನ್ನು
ವಾದ-ವಿ-ವಾದ-ಗಳು
ವಾದಿ-ರಾಜ-ದೇವನ
ವಾದ್ಯ-ಗಳು
ವಾದ್ಯ-ವಿಶೇಷ-ಗಳಿರ-ಬಹುದು
ವಾದ್ಯ-ವಿಶೇಷವೋ
ವಾನ-ವನ್
ವಾನ-ವನ್ಮಾ-ದೇವಿ
ವಾಯುವ್ಯ
ವಾರ-ಣಾಸಿ
ವಾರ-ಣಾಸಿಗೆ
ವಾರದ
ವಾರದ್
ವಾರ-ಸುದಾರ-ರನ್ನು
ವಾರ-ಸು-ದಾರಿ-ಕೆಯು
ವಾರ್ತೆ-ಯನ್ನು
ವಾರ್ದ್ಧಿ-ವರ್ಧನ
ವಾಲಗ
ವಾಲಗ-ದ-ವರು
ವಾಸಂತಿಕಾ
ವಾಸ-ಮದೇಭಾವ-ಳಿಯಂ
ವಾಸ-ಮಾಡಿ-ಕೊಂಡು
ವಾಸ-ಮಾಡುತ್ತಿದ್ದನು
ವಾಸವ
ವಾಸ-ವನ
ವಾಸ-ವ-ನಿಗೆ
ವಾಸಿ-ಯಾಗಿ
ವಾಸಿ-ಸುತ್ತಿದ್ದ
ವಾಸು
ವಾಸು-ದೇವ
ವಾಸು-ವಿನ
ವಾಸ್ತವ-ವಾಗಿ
ವಾಸ್ತು
ವಾಸ್ತು-ದೃಷ್ಟಿ-ಯಿಂದ
ವಾಸ್ತು-ವಿನ
ವಾಸ್ತು-ಶಿಲ್ಪ
ವಾಹನ-ವಸ್ತು-ಗಳು
ವಿ
ವಿಂಗಡಿ-ಸದೇ
ವಿಂಗಡಿ-ಸ-ಬಹುದು
ವಿಂಗಡಿ-ಸ-ಬಹು-ದೆಂದು
ವಿಂಗಡಿ-ಸ-ಲಾಗಿತ್ತೆಂದು
ವಿಂಗಡಿ-ಸ-ಲಾಗಿತ್ತೆಂದೂ
ವಿಂಗಡಿ-ಸ-ಲಾಗುತ್ತಿತ್ತು
ವಿಂಗಡಿ-ಸ-ಲಾಯಿತು
ವಿಂಗಡಿ-ಸಲ್ಪಟ್ಟಿತ್ತು
ವಿಂಗಡಿಸಿ
ವಿಂಗಡಿ-ಸಿ-ದರೆ
ವಿಂಗಡಿ-ಸಿದ್ದರು
ವಿಕಲ್ಪ
ವಿಕ್ರ-ಮ-ಗಂಗ
ವಿಕ್ರಮನ
ವಿಕ್ರಮ-ನನ್ನು
ವಿಕ್ರಮ-ಯುತರೂ
ವಿಕ್ರಮ-ರಾಯ
ವಿಕ್ರಮ-ರಾಯ-ವಿಗಡ
ವಿಕ್ರಮಾಂಕ-ದೇವ-ಚರಿತ-ದಲ್ಲಿ
ವಿಕ್ರಮಾದಿತ್ಯ
ವಿಕ್ರಮಾದಿತ್ಯನ
ವಿಕ್ರಮಾದಿತ್ಯ-ನಿಗೆ
ವಿಕ್ರಮಾದಿತ್ಯನು
ವಿಕ್ರಮಾರ್ಜಿತ-ವಾಗಿ
ವಿಕ್ರಮಾರ್ಜಿತ-ವಾದ
ವಿಖ್ಯಾತ
ವಿಖ್ಯಾತಗ್ರಾಮಂ
ವಿಖ್ಯಾತೋ
ವಿಗ್ರಹದ
ವಿಗ್ರಹ-ವನ್ನು
ವಿಗ್ರಹ-ವಿದ್ದು
ವಿಗ್ರಹ-ವಿನೋದ
ವಿಘಟನ
ವಿಘ್ನೇಶ್ವರ
ವಿಚಾರ
ವಿಚಾರಕ್ಕೆ
ವಿಚಾರ-ಗಳ
ವಿಚಾರ-ಗಳನ್ನು
ವಿಚಾರ-ಗಳನ್ನುಳ್ಳ
ವಿಚಾರ-ಗ-ಳಿಗೆ
ವಿಚಾರ-ಗಳು
ವಿಚಾರ-ಗಳೂ
ವಿಚಾರ-ಚಾವಡಿ
ವಿಚಾರ-ಣೆಯ
ವಿಚಾ-ರದ
ವಿಚಾರ-ದಲ್ಲಿ
ವಿಚಾರ-ವನ್ನು
ವಿಚಾರ-ವಾಗಲೀ
ವಿಚಾರ-ವಿದೆ
ವಿಚಾರವೂ
ವಿಚಾ-ರಾರ್ಹ
ವಿಚಾರಿ-ಸಲು
ವಿಚಿತ್ರ-ವಾದ
ವಿಜಗೀಷು-ವೃತ್ತಿಯಿಂ
ವಿಜಯ
ವಿಜಯಕ್ಕಾಗಿ
ವಿಜಯ-ಗಳ
ವಿಜಯ-ಗಳನ್ನು
ವಿಜಯ-ಗಳಾಗಿದ್ದು
ವಿಜಯದ
ವಿಜಯ-ದಂತಹ
ವಿಜಯ-ದಲ್ಲಿ
ವಿಜಯ-ನ-ಗರ
ವಿಜಯ-ನ-ಗರಕ್ಕೆ
ವಿಜಯ-ನ-ಗರದ
ವಿಜಯ-ನ-ಗರ-ದಿಂದ
ವಿಜಯ-ನ-ಗರ-ದೊರೆ
ವಿಜಯ-ನ-ಗರ-ವಾದ
ವಿಜಯ-ನ-ಗರವು
ವಿಜಯ-ನ-ಗರಿ-ಯಲ್ಲಿ
ವಿಜಯ-ನರ-ಸಿಂಹ
ವಿಜಯ-ನಾರ-ಸಿಂಹ
ವಿಜಯ-ನಾರ-ಸಿಂಹ-ದೇವ-ರಿಗೆ
ವಿಜಯ-ನಾರ-ಸಿಂಹನ
ವಿಜಯ-ನಾರ-ಸಿಂಹನು
ವಿಜಯ-ನಾ-ರಾಯಣ
ವಿಜಯ-ಪಾಂಡ್ಯ-ನುಚ್ಚಂಗಿ
ವಿಜಯ-ಪುರ
ವಿಜಯ-ಪುರದ
ವಿಜಯ-ಪುರ-ದಲ್ಲಿರುವ
ವಿಜಯ-ಬುಕ್ಕ
ವಿಜಯ-ಬುಕ್ಕ-ರಾಯನ
ವಿಜಯ-ಯಾತ್ರೆಯ
ವಿಜಯ-ರಾಜ-ಧಾನಿ
ವಿಜಯ-ರಾಯ
ವಿಜಯ-ವನ್ನು
ವಿಜಯ-ವನ್ನೇ
ವಿಜಯ-ವಿ-ರೂಪಾಕ್ಷ
ವಿಜಯ-ವೆಂದು
ವಿಜಯವೇ
ವಿಜಯಶ್ರೀ
ವಿಜಯ-ಸಂವತ್ಸರದ
ವಿಜಯ-ಸಮುದ್ರ-ವೆನಿ-ಸಿದ
ವಿಜಯ-ಸೋಮ-ನಾಥ-ಪುರ
ವಿಜಯಸ್ಕಾಂದ-ವಾರ-ವಾದ
ವಿಜಯಸ್ತಂಭ-ವನ್ನು
ವಿಜಯಾದಿತ್ಯ
ವಿಜಯಾದಿತ್ಯ-ನಾಗಿದ್ದು
ವಿಜಯಾದಿತ್ಯ-ನಿಗೆ
ವಿಜಯಾದಿತ್ಯನು
ವಿಜಯೋತ್ತುಂಗ
ವಿಜಯೋತ್ಸವ
ವಿಜಯೋತ್ಸವ-ವನ್ನಾಚರಿ-ಸಲು
ವಿಜಯೋತ್ಸವ-ವನ್ನು
ವಿಜೆಯ-ನರ-ಸಿಂಹಂ
ವಿಜ್ಞಾಪನೆ
ವಿಜ್ಞಾಪ-ನೆಯ
ವಿಜ್ಞಾಪನೆ-ಯನ್ನು
ವಿಟ್ಟಿಯಣ್ಣ
ವಿಠಂಣ್ಣ
ವಿಠಂಣ್ಣ-ಗಳ
ವಿಠಣ್ಣ-ಹೆಗ್ಗಡೆಯು
ವಿಠ-ಲೇಶ್ವರ
ವಿಣ್ಣಯಾಂಡರ
ವಿಣ್ನಘರಂ
ವಿತ್ತಿ
ವಿತ್ತಿ-ರುಂದ-ವಿರ್ರಿರುಂದ
ವಿದೇಶ
ವಿದೇಶಾಂಗ
ವಿದೇಶಾಂಗಕ್ಕೆ
ವಿದ್ಯಾ
ವಿದ್ಯಾ-ಕೆಂದ್ರ-ಗಳನ್ನು
ವಿದ್ಯಾ-ದಾನ-ಗಳಿಂದ
ವಿದ್ಯಾ-ಧರ
ವಿದ್ಯಾ-ಧ-ರನು
ವಿದ್ಯಾ-ಧ-ರನೂ
ವಿದ್ಯಾ-ನ-ಗರ-ದಿಂದ
ವಿದ್ಯಾ-ನ-ಗರಿ
ವಿದ್ಯಾ-ನ-ಗರಿಯ
ವಿದ್ಯಾ-ನ-ಗರಿ-ಯಿಂದ
ವಿದ್ಯಾ-ನ-ಗರ್ಯ್ಯಾಂ
ವಿದ್ಯಾ-ನಿಧಿ
ವಿದ್ಯಾಭ್ಯಾಸ
ವಿದ್ಯಾ-ರಾಜು
ವಿದ್ಯಾರ್ಥಿ-ಗಳ
ವಿದ್ಯಾ-ವಿಶಾ-ರದ-ರಪ್ಪ
ವಿದ್ರಾವಣಂ
ವಿದ್ವ-ಜನ-ಪೋಷ-ಕನೂ
ವಿದ್ವ-ಜನ-ವಿ-ಪದ-ಳನ
ವಿದ್ವತ್
ವಿದ್ವನ್ಮಂಡ-ಲಿಯ
ವಿದ್ವಾಂಸ
ವಿದ್ವಾಂಸರ
ವಿದ್ವಾಂಸ-ರ-ಮತ
ವಿದ್ವಾಂಸ-ರಲ್ಲಿ
ವಿದ್ವಾಂಸ-ರಾಗಿದ್ದ
ವಿದ್ವಾಂಸ-ರಾದ
ವಿದ್ವಾಂಸ-ರಿಂದ
ವಿದ್ವಾಂಸ-ರಿಗೆ
ವಿದ್ವಾಂಸರು
ವಿದ್ವಾಂಸ-ರು-ಗಳು
ವಿದ್ವಿಷ್ಟ
ವಿಧ
ವಿಧ-ವೆ-ಯಾಗಿದ್ದರೂ
ವಿಧಾ-ನ-ದಿಂದ
ವಿಧಿ-ಗಳನ್ನು
ವಿಧಿಗೆ
ವಿಧಿ-ಯಿಂದ
ವಿಧೇಯ-ರಾಗಿ
ವಿನಂತಿ
ವಿನಂತಿ-ಯ-ಮೇ-ರೆಗೆ
ವಿನ-ಯಪರಂ
ವಿನ-ಯ-ವಾಗಿ
ವಿನ-ಯವಿಭೂಷಿ-ತನೂ
ವಿನ-ಯಸ್ಯೇವ
ವಿನ-ಯಾದಿತ್ಯ
ವಿನ-ಯಾದಿತ್ಯನ
ವಿನ-ಯಾದಿತ್ಯ-ನನ್ನು
ವಿನ-ಯಾದಿತ್ಯ-ನಾಗಿದ್ದಾನೆ
ವಿನ-ಯಾದಿತ್ಯನು
ವಿನ-ಯಾದಿತ್ಯನೇ
ವಿನಾ
ವಿನುತ
ವಿನೇಜತ್ಸಾಮ್ರ
ವಿನೇ-ಯವಿಳಾಸಂ
ವಿನೋದ-ದಿಂದ
ವಿನೋದನೂ
ವಿನೋದಿ
ವಿನೋದಿಂದಾಳೆ
ವಿಪರೀತ-ವಾಗಿದ್ದಂತೆ
ವಿಪ್ರ-ಕುಲ-ತಿಲಕನೂ
ವಿಪ್ರರು
ವಿಪ್ರೋತ್ತಮ-ನಿಗೆ
ವಿಫಲ-ಗೊಳಿಸಿ
ವಿಫಲ-ವಾಯಿತು
ವಿಬುಧ
ವಿಬುಧ-ಜನ-ಫಳಪ್ರದಾಯಕಂ
ವಿಬುಧಪ್ರಸನ್ನನುಂ
ವಿಭ-ಜನೆ-ಯಾಗಿದ್ದ-ರಿಂದ
ವಿಭಜಿತ-ವಾಗಿದ್ದು
ವಿಭಜಿ-ಸಲಾ-ಗಿತ್ತು
ವಿಭಜಿ-ಸ-ಲಾಗಿತ್ತೆಂದು
ವಿಭಜಿಸಲ್
ವಿಭಜಿಸಿ
ವಿಭಜಿಸಿತ್ತು
ವಿಭಜಿಸಿ-ದನು
ವಿಭಜಿ-ಸಿದ್ದು
ವಿಭವಪ್ರಭಾವ-ತೆ-ಯಿಂದಂ
ವಿಭಾಗ
ವಿಭಾಗಕ್ಕೆ
ವಿಭಾಗ-ಗಳ
ವಿಭಾಗ-ಗಳನ್ನಾಗಿ
ವಿಭಾಗ-ಗಳನ್ನು
ವಿಭಾಗ-ಗಳಲ್ಲಿ
ವಿಭಾಗ-ಗಳಾಗಿ
ವಿಭಾಗ-ಗಳಾಗಿದ್ದವು
ವಿಭಾಗ-ಗಳಿಗೂ
ವಿಭಾಗ-ಗ-ಳಿಗೆ
ವಿಭಾಗ-ಗಳಿದ್ದವು
ವಿಭಾಗ-ಗಳು
ವಿಭಾಗ-ಗಳೂ
ವಿಭಾಗ-ಗ-ಳೆಂದು
ವಿಭಾಗ-ಗಳೆನ್ನ-ಬಹುದು
ವಿಭಾಗ-ಗಳೇ
ವಿಭಾಗದ
ವಿಭಾಗ-ದಲ್ಲಿ
ವಿಭಾಗ-ವನ್ನು
ವಿಭಾಗ-ವಾ-ಗಿತ್ತು
ವಿಭಾಗ-ವಾಗಿತ್ತೆಂದು
ವಿಭಾಗ-ವಾಗಿದ್ದ
ವಿಭಾಗ-ವಾಗಿದ್ದಿರ-ಬಹುದು
ವಿಭಾಗ-ವಾಗಿದ್ದು
ವಿಭಾಗವು
ವಿಭಾಗವೂ
ವಿಭಾಗ-ವೆಂದು
ವಿಭಾಗವೇ
ವಿಭಾಗಿಸ-ಬಹುದು
ವಿಭಾಗಿ-ಸಲಾ-ಗಿತ್ತು
ವಿಭಾಗಿಸಿ
ವಿಭಾಡ-ರೆನಿ-ಸಿದ
ವಿಭಿನ್ನ-ವಾಗಿದೆ
ವಿಭಿನ್ನ-ವಾಗಿವೆ
ವಿಭು
ವಿಭು-ಗಳು
ವಿಭು-ದೇವ-ರಾಜನಂ
ವಿಭು-ದೇಶಂ
ವಿಭುಪ್ರಭು
ವಿಭು-ಬಲ್ಲಯ್ಯ-ನಾಯಕ
ವಿಭೂತಿ-ಯನ್ನು
ವಿಮಲ
ವಿಮಲ-ನಾಥ
ವಿಮಳ-ಗಂಗಾನ್ವಯ
ವಿಯಷ-ವನ್ನು
ವಿರಚಿತ
ವಿರಾಜ-ಮಾನ
ವಿರಾಜ-ಮಾನಂತಂತ್ರ-ರಕ್ಷಾಮಣಿ
ವಿರಾಜಿತ
ವಿರಾಜಿತ-ನಾಗಿದ್ದ-ನೆಂದು
ವಿರುದಯ-ರಾಯ
ವಿರುದ್ಧ
ವಿರುದ್ಧದ
ವಿರುದ್ಧ-ವಾಗಿ-ರಲು
ವಿರುದ್ಧವೇ
ವಿರುಪಂಣ
ವಿರುಪಣ್ಣ
ವಿರುಪಣ್ಣ-ನ-ವರ
ವಿರುಪನ-ಪುರ
ವಿರುಪ-ರಾಜ
ವಿರುಪಾಕ್ಷ-ದೇವ-ಅಣ್ಣನು
ವಿರೂಪಾಕ್ಷ
ವಿರೂಪಾಕ್ಷ-ದಲಿ
ವಿರೂಪಾಕ್ಷ-ದೇವ
ವಿರೂಪಾಕ್ಷ-ದೇವರ
ವಿರೂಪಾಕ್ಷನ
ವಿರೂಪಾಕ್ಷ-ನನ್ನು
ವಿರೂಪಾಕ್ಷ-ನಿಗೆ
ವಿರೂಪಾಕ್ಷನು
ವಿರೂಪಾಕ್ಷ-ನೆಂಬ
ವಿರೂಪಾಕ್ಷ-ಪುರ
ವಿರೂಪಾಕ್ಷ-ಪುರ-ವೆಂದು
ವಿರೂಪಾಕ್ಷ-ಪುರ-ವೆಂಬ
ವಿರೂಪಾಕ್ಷಯ್ಯ
ವಿರೂಪಾಕ್ಷಿ-ಪುರ-ಗಳಲ್ಲಿ
ವಿರೋಧಿ-ಗಳಾಗಿದ್ದುದೂ
ವಿರೋಧಿ-ಗಳಾಗಿ-ರ-ಲಿಲ್ಲ
ವಿರೋಧಿ-ಸಂವತ್ಸರದ
ವಿರೋಧಿಸಿ
ವಿರೋಧಿಸಿ-ಚಿವೋನ್ಮತ್ತೇಭ
ವಿರ್ರಿರುಂದ
ವಿಲಾ-ಸದರ್ಪಣ-ದಂತೆ
ವಿಲೀನಗೊಳಿ-ಸ-ಲಾಯಿತು
ವಿಳಂದೆ
ವಿಳಸತ್
ವಿಳ-ಸದ್ಬಲ್ಲಾಳ-ದೇವಾ-ವನೀ-ಪತಿಗೀ
ವಿವರ
ವಿವರ-ಗಳನ್ನು
ವಿವರ-ಗಳಿದ್ದು
ವಿವರ-ಗಳಿವೆ
ವಿವರ-ಗಳು
ವಿವರ-ಗಳೂ
ವಿವರ-ಣೆ-ಯನ್ನು
ವಿವರ-ಣೆ-ಯಿಂದ
ವಿವರ-ವನ್ನು
ವಿವರ-ವಾಗಿ
ವಿವರ-ವಾಗಿದ್ದು
ವಿವರ-ವಿದೆ
ವಿವರಿ-ಸ-ಲಾಗಿದೆ
ವಿವರಿ-ಸಿದೆ
ವಿವರಿ-ಸಿದ್ದಾರೆ
ವಿವರಿ-ಸುತ್ತದೆ
ವಿವರಿ-ಸುವ
ವಿವಾಹ
ವಿವಾಹಂ
ವಿವಾಹ-ಕಾಲ-ದಲ್ಲಿ
ವಿವಾಹದ
ವಿವಾಹ-ಮಾಡಿ-ಕೊಟ್ಟು
ವಿವಾಹ-ವಾಗಿದ್ದ
ವಿವಾಹ-ವಾದ
ವಿವಾಹ-ವಾ-ದನು
ವಿವಿಧ
ವಿವೇಚ-ನೆಯ
ವಿವೇಚಿ-ಸ-ಲಾಗಿದೆ
ವಿವೇಚಿಸಿ-ದಲ್ಲಿ
ವಿವೇಚಿ-ಸಿದ್ದಾರೆ
ವಿವೇಚಿ-ಸಿದ್ದು
ವಿಶಾಲಮುದ್ರಿ
ವಿಶಾಲ-ವಾದ
ವಿಶಿ-ವಾನಂದ್
ವಿಶಿಷ್ಟ
ವಿಶಿಷ್ಟ-ವಾದ
ವಿಶುದ್ಧ
ವಿಶೇಷ
ವಿಶೇಷಣ
ವಿಶೇಷ-ಣ-ಗಳನ್ನು
ವಿಶೇಷ-ಣ-ದಿಂದ
ವಿಶೇಷ-ಣ-ವನ್ನು
ವಿಶೇಷದ
ವಿಶೇಷ-ದ-ವರು
ವಿಶೇಷ-ವಾಗಿ
ವಿಶೇಷ-ವಾಗಿದೆ
ವಿಶೇಷ-ವಾಗಿದ್ದು
ವಿಶ್ಲೇಷಣೆ
ವಿಶ್ಲೇಷಣೆ-ಗಳಿಂದ
ವಿಶ್ಲೇಷಣೆಗೆ
ವಿಶ್ಲೇಷಣೆ-ಯಾಗಲೀ
ವಿಶ್ಲೇಷಣೆ-ಯಿಂದ
ವಿಶ್ಲೇಷಿ-ಸಿದ್ದಾರೆ
ವಿಶ್ವ-ಕರ್ಮ
ವಿಶ್ವಕೋಶದಂತಿದೆ
ವಿಶ್ವಣ್ಣ
ವಿಶ್ವ-ನಾಥ-ಪುರ-ವಾದ
ವಿಶ್ವಭೂಪಾಳ-ಕರ್
ವಿಶ್ವ-ವಿದ್ಯಾ-ನಿ-ಲದಯ
ವಿಶ್ವ-ವಿದ್ಯಾ-ನಿಲಯ-ಗಳ
ವಿಶ್ವ-ಸಣ್ಣ
ವಿಶ್ವಾ-ವನಿ
ವಿಶ್ವಾಸಾರ್ಹವೂ
ವಿಶ್ವಾಸಾ-ವಾಸ-ವೇಶ್ಮನಃ
ವಿಶ್ವಾಸಿಕ
ವಿಶ್ವಾಸಿ-ಕ-ರಾಗಿ
ವಿಶ್ವೇಶ್ವರ-ದೇವರ
ವಿಶ್ವೇಶ್ವರಯ್ಯ
ವಿಷಯ
ವಿಷಯ-ಕೆರೆ-ಗೋಡು-ನಾಡು
ವಿಷ-ಯಕ್ಕೆ
ವಿಷಯ-ಗಳ
ವಿಷಯ-ಗಳನ್ನು
ವಿಷಯ-ಗಳಾಗಿ
ವಿಷಯ-ಗಳಿದ್ದ-ವೆಂದು
ವಿಷಯ-ಗಳು
ವಿಷಯ-ಗಳೇ
ವಿಷಯದ
ವಿಷಯ-ದಲ್ಲಿ
ವಿಷಯ-ದಲ್ಲಿದ್ದ
ವಿಷಯ-ವನ್ನು
ವಿಷಯ-ವನ್ನೊಳ-ಗೊಂಡ
ವಿಷಯ-ವಾಗಿದೆ
ವಿಷಯ-ವಾಗಿ-ರ-ಬಹುದು
ವಿಷಯವು
ವಿಷಯಾಧೀಶ-ರನ್ನು
ವಿಷ್ಟ-ಪತ್ರಯ
ವಿಷ್ಣ-ವರ್ಧನನ
ವಿಷ್ಣ-ವರ್ಧನ-ನಿಗೆ
ವಿಷ್ಣ-ವಿಗೂ
ವಿಷ್ಣು
ವಿಷ್ಣು-ಚಮೂಪತಿ
ವಿಷ್ಣು-ದಂಡಾಧೀಶ
ವಿಷ್ಣು-ದಂಡಾಧೀಶ-ನನ್ನು
ವಿಷ್ಣು-ದಂಡಾಧೀಶನು
ವಿಷ್ಣು-ದಂಡಾಧೀಶನುಂ
ವಿಷ್ಣು-ದಂಡಾಧೀಶನೇ
ವಿಷ್ಣು-ದಂಡಾಧೀಶರುಃ
ವಿಷ್ಣು-ದೇವ
ವಿಷ್ಣು-ದೇವ-ನಿಗೆ
ವಿಷ್ಣು-ಪುರ
ವಿಷ್ಣು-ಪುರಾಣದ
ವಿಷ್ಣು-ಭಟ್ಟಯ್ಯನ
ವಿಷ್ಣು-ಭೂಪನ
ವಿಷ್ಣು-ರಾಯ
ವಿಷ್ಣು-ರಾಯ-ಮಹಾ-ರಾಯ-ನೆಂದರೆ
ವಿಷ್ಣು-ವರ್ಧನ
ವಿಷ್ಣು-ವರ್ಧನ-ದೇವ-ರು-ದುಷ್ಟನಿಗ್ರಹ
ವಿಷ್ಣು-ವರ್ಧನನ
ವಿಷ್ಣು-ವರ್ಧನ-ನಿಂದ
ವಿಷ್ಣು-ವರ್ಧನ-ನಿಗೆ
ವಿಷ್ಣು-ವರ್ಧನನು
ವಿಷ್ಣು-ವರ್ಧನನೇ
ವಿಷ್ಣು-ವರ್ಧನನ್ನು
ವಿಷ್ಣು-ವರ್ಧನಿನ-ಗಾಗಿ
ವಿಷ್ಣು-ವರ್ಧನು
ವಿಷ್ಣು-ವರ್ಧ್ಧನ
ವಿಷ್ಣು-ವಿಗೆ
ವಿಷ್ಣು-ವಿನ
ವಿಸ್ತರಣೆ
ವಿಸ್ತರಣೆ-ಯನ್ನು
ವಿಸ್ತರಣೆ-ಯಲ್ಲಿಯೂ
ವಿಸ್ತರಿಸಿತ್ತೆಂದು
ವಿಸ್ತರಿಸಿ-ದಾಗ
ವಿಸ್ತರಿಸಿದೆ
ವಿಸ್ತರಿ-ಸುತ್ತಾನೆ
ವಿಸ್ತಾರ-ವನ್ನು
ವಿಸ್ತಾರ-ವಾಗಿದೆ
ವಿಸ್ತಾರ-ವಾಗಿದ್ದ
ವಿಸ್ತಾರ-ವಾದ
ವಿಸ್ತಾರ-ವಾ-ದಂತೆಲ್ಲಾ
ವಿಸ್ತೀರ್ಣ
ವಿಸ್ತೀರ್ಣ-ವನ್ನು
ವಿಸ್ತೀರ್ಣ-ವಿ-ರುವ
ವಿಸ್ತೃತ
ವಿಹಂಗ
ವೀಕ್ಷಣೆ
ವೀಡು
ವೀತಾಯುಧವ್ರತ-ಧಾರಿ-ಯಾಗಿ
ವೀನರ-ಸಿಂಹನ
ವೀಬಲ್ಲಾಳ
ವೀರ
ವೀರ-ಅಚ್ಯುತ-ರಾಯ
ವೀರ-ಅಚ್ಯುತ-ರಾಯನ
ವೀರ-ಕೃಷ್ಣ-ರಾಯ-ಮಹಾ-ರಾಯ
ವೀರ-ಕೆಕ್ಕಾಯಿ
ವೀರ-ಕೇ-ತೆಯ
ವೀರ-ಕೊಂಗಾಳ್ವ
ವೀರ-ಗಂಗ
ವೀರ-ಗಂಗ-ಪೆರ್ಮಾನ-ಡಿಯು
ವೀರ-ಗಜ-ಬೇಂಟೆ-ಕಾರ
ವೀರ-ಗಲ್ಲನ್ನು
ವೀರ-ಗಲ್ಲಾ-ಗಿದ್ದು
ವೀರ-ಗಲ್ಲಿದೆ
ವೀರ-ಗಲ್ಲಿನಲ್ಲಿ
ವೀರ-ಗಲ್ಲಿನಲ್ಲಿದೆ
ವೀರ-ಗಲ್ಲಿನಿಂದ
ವೀರ-ಗಲ್ಲು
ವೀರ-ಗಲ್ಲು-ಗಳಲ್ಲಿ
ವೀರ-ಗಲ್ಲು-ಗಳಿಂದ
ವೀರ-ಗಲ್ಲು-ಗಳು
ವೀರ-ಗಲ್ಲು-ಶಾ-ಸನ
ವೀರ-ಗಲ್ಲು-ಶಾ-ಸನ-ಗಳಿವೆ
ವೀರ-ಗಲ್ಲು-ಶಾ-ಸನ-ಗಳು
ವೀರ-ಗಲ್ಲು-ಶಾ-ಸನ-ದಿಂದ
ವೀರ-ಗಲ್ಲೂ
ವೀರಗ್ರಾಣಿ-ಯಾಗಿದ್ದ-ನಂತೆ
ವೀರ-ಚಿಕ-ರಾಯನು
ವೀರ-ಚಿಕ-ವೊಡೆಯರ
ವೀರ-ಚಿಕ್ಕ-ಕೇತಯ್ಯ
ವೀರ-ಚಿಕ್ಕ-ಕೇ-ತೆಯ
ವೀರ-ಚಿಕ್ಕ-ಕೇ-ತೆಯ್ಯನ
ವೀರ-ಚಿಕ್ಕ-ರಾಯ
ವೀರ-ಚಿಕ್ಕ-ರಾಯ-ನಿ-ಗಿಂತ
ವೀರ-ಚಿಕ್ಕ-ರಾಯನು
ವೀರ-ಚಿಕ್ಕ-ರಾಯನೇ
ವೀರ-ಣ-ನಾಯ-ಕರು
ವೀರಣ್ಣ
ವೀರಣ್ಣ-ನಾಯ-ಕ-ನಿಗೆ
ವೀರಣ್ಣ-ನಾಯ-ಕನು
ವೀರತಃ
ವೀರತ್ವ-ದಿಂದ
ವೀರತ್ವ-ವನ್ನು
ವೀರ-ದೇವನ
ವೀರ-ದೇವ-ನ-ಹಳ್ಳಿ
ವೀರ-ದೇವ-ನ-ಹಳ್ಳಿಯ
ವೀರ-ದೇವ-ರಾಯನ
ವೀರನ
ವೀರ-ನಂಜ-ರಾಜ
ವೀರ-ನಂಜ-ರಾಜೊಡೆಯರ
ವೀರ-ನಂಜ-ರಾಯ
ವೀರ-ನಂಜ-ರಾಯನ
ವೀರ-ನನ್ನು
ವೀರ-ನರ-ಪತಿ
ವೀರ-ನರ-ಸಿಂಹ
ವೀರ-ನರ-ಸಿಂಹನ
ವೀರ-ನರ-ಸಿಂಹ-ನಿಂದ
ವೀರ-ನರ-ಸಿಂಹನು
ವೀರ-ನರ-ಸಿಂಹ-ರಾಯರ
ವೀರ-ನರ-ಸಿಂಹೇಂದ್ರ-ಪುರ-ವೆಂಬ
ವೀರ-ನ-ಹಳ್ಳಿ
ವೀರ-ನಾಯಕ
ವೀರ-ನಾರ-ಸಿಂಹ
ವೀರ-ನಾರ-ಸಿಂಹ-ದೇವನು
ವೀರ-ನಾರ-ಸಿಂಹ-ದೇವರ
ವೀರ-ನಾರ-ಸಿಂಹ-ದೇವ-ರ-ಸರ
ವೀರ-ನಾರ-ಸಿಂಹ-ದೇವ-ರ-ಸರು
ವೀರ-ನಾರ-ಸಿಂಹನ
ವೀರ-ನಾರ-ಸಿಂಹ-ನನ್ನು
ವೀರ-ನಾರ-ಸಿಂಹನು
ವೀರ-ನಾರ-ಸಿಂಹ-ಪುರ-ವಾದ
ವೀರ-ನಾ-ರಾಯಣ
ವೀರ-ನಾ-ರಾಯ-ಣದ
ವೀರ-ನಾ-ರಾಯ-ಣ-ದೇವ-ನೆಂಬ
ವೀರ-ನಾ-ರಾಯ-ಣ-ದೇವರ
ವೀರ-ನಾ-ರಾಯ-ಣ-ದೇ-ವ-ರಿಗೆ
ವೀರ-ನಿಗೆ
ವೀರನು
ವೀರನೂ
ವೀರ-ನೃ-ಸಿಂಹ
ವೀರ-ನೆಂದು
ವೀರ-ನೊಬ್ಬ-ನಿಗೆ
ವೀರ-ನೊಬ್ಬನು
ವೀರ-ಪಟ್ಟ
ವೀರ-ಪಟ್ಟಮಂ
ವೀರ-ಪಟ್ಟ-ವನ್ನು
ವೀರ-ಪಾಂಡ್ಯ
ವೀರ-ಪಾಂಡ್ಯನ
ವೀರ-ಪಾಂಡ್ಯ-ನನ್ನು
ವೀರ-ಪುರುಷ-ನಾಗಿದ್ದನು
ವೀರ-ಪೆರ್ಮಾಡಿ
ವೀರಪ್ಪ
ವೀರಪ್ಪ-ವೊಡ-ಯರ
ವೀರಪ್ರತಾಪ
ವೀರ-ಬಂಕೆಯನ
ವೀರ-ಬಂಕೆಯನು
ವೀರ-ಬಲ್ಲಾಳ
ವೀರ-ಬಲ್ಲಾಳ-ದೇವನ
ವೀರ-ಬಲ್ಲಾಳ-ದೇವ-ನಿಗೆ
ವೀರ-ಬಲ್ಲಾಳ-ದೇವನು
ವೀರ-ಬಲ್ಲಾಳ-ದೇವರ
ವೀರ-ಬಲ್ಲಾಳ-ದೇವ-ರ-ಸರ
ವೀರ-ಬಲ್ಲಾಳ-ದೇವ-ರ-ಸರು
ವೀರ-ಬಲ್ಲಾಳನ
ವೀರ-ಬಲ್ಲಾಳ-ನನ್ನು
ವೀರ-ಬಲ್ಲಾಳ-ನಲ್ಲಿ
ವೀರ-ಬಲ್ಲಾಳ-ನಿಗೂ
ವೀರ-ಬಲ್ಲಾಳ-ನಿಗೆ
ವೀರ-ಬಲ್ಲಾಳನು
ವೀರ-ಬಲ್ಲಾಳ-ಪುರ-ವನ್ನು
ವೀರ-ಬಲ್ಲಾಳ-ರಾಯ
ವೀರ-ಬಳಂಜು-ಧರ್ಮಕ್ಕೆ
ವೀರ-ಬುಕ-ರಾಜ
ವೀರ-ಬುಕ್ಕ
ವೀರ-ಬುಕ್ಕಣ್ಣ
ವೀರ-ಬುಕ್ಕಣ್ಣೊಡೆ-ಯರ
ವೀರ-ಭಟಲಲಾಟ-ಪಟ್ಟಂ
ವೀರ-ಭಟಾ-ವಳಿ
ವೀರ-ಭದ್ರ
ವೀರ-ಭದ್ರ-ದುರ್ಗ
ವೀರ-ಭದ್ರ-ದೇವರ
ವೀರ-ಭದ್ರ-ದೇವ-ರಿಗೆ
ವೀರ-ಭದ್ರಸ್ವಾಮಿ
ವೀರ-ಭುಜಕ್ಕನ್ದೈ-ಯರ್
ವೀರ-ಮಂಗಪ್ಪ
ವೀರ-ಮಯ್ದುನ
ವೀರ-ಮರಣಸ್ಮಾರಕ-ಗಳು
ವೀರ-ಮಸ-ಣನು
ವೀರಯ್ಯ
ವೀರಯ್ಯ-ದಂಡ-ನಾಯ-ಕನ
ವೀರಯ್ಯ-ದಂಡ-ನಾಯ-ಕನು
ವೀರಯ್ಯ-ನನ್ನು
ವೀರಯ್ಯನು
ವೀರರ
ವೀರ-ರ-ಗುಡಿ-ಗಳು
ವೀರ-ರನ್ನಾಗಲೀ
ವೀರ-ರ-ಸರು
ವೀರ-ರಾಗಿ-ರುವು-ದ-ರಿಂದ
ವೀರ-ರಾಜನ
ವೀರ-ರಾಜ-ನಿಗೆ
ವೀರ-ರಾಜೇಂದ್ರ
ವೀರ-ರಾಜೇಂದ್ರನ
ವೀರ-ರಾಜೇಂದ್ರ-ಹೊಯ್ಸಳ
ವೀರ-ರಾಜೈಯ್ಯನ-ವರ
ವೀರ-ರಾಮ-ದೇವ
ವೀರ-ರಾಮ-ದೇವ-ರಾಯನ
ವೀರ-ರಾಮ-ದೇವ-ರಾಯ-ನಿಂದ
ವೀರ-ರಾಮ-ನಾಥನು
ವೀರರು
ವೀರ-ರುಮ್
ವೀರ-ಲಕ್ಷ್ಮೀ-ಭುಜಂಗ
ವೀರ-ವಿಜಯ-ರಾಯ
ವೀರ-ವಿಜಯ-ರಾಯನೇ
ವೀರ-ವಿ-ರೂಪಾಕ್ಷ-ನನ್ನು
ವೀರ-ವಿ-ರೂಪಾಕ್ಷನು
ವೀರ-ವಿ-ರೂಪಾಕ್ಷ-ಬಲ್ಲಾಳ
ವೀರ-ವಿಷ್ಣು-ವರ್ಧನ
ವೀರ-ವಿಷ್ಣು-ವರ್ಧನ-ದೇವ
ವೀರ-ವೆಂದೊಡೀ
ವೀರ-ವೈಷ್ಣವಿ
ವೀರ-ಶಾ-ಸನ-ವನ್ನು
ವೀರ-ಶೆಟ್ಟಿ-ಹಳ್ಳಿ
ವೀರ-ಶೈವ
ವೀರ-ಶೈವ-ಧರ್ಮ
ವೀರ-ಶೈವ-ಧರ್ಮಕ್ಕೆ
ವೀರಶ್ರೀ
ವೀರಶ್ರೀ-ನಾರ-ಸಿಂಹೇಂದ್ರ-ಪುರ-ವಾದ
ವೀರ-ಸಂಗಮೇಶ್ವರ-ರಾಯ
ವೀರ-ಸಿದ್ಧಿ-ವೆ-ರಸು
ವೀರ-ಸೇವುಣರ
ವೀರ-ಸೋಮೇಶ್ವರ
ವೀರ-ಸೋಮೇಶ್ವರ-ದೇವನ
ವೀರ-ಸೋಮೇಶ್ವರನ
ವೀರ-ಸೋಮೇಶ್ವರನು
ವೀರ-ಹನು-ಮಪ್ಪ
ವೀರ-ಹರಿ-ಯಪ್ಪ-ವೊಡೆಯರು
ವೀರ-ಹರಿ-ಹರ
ವೀರ-ಹರಿ-ಹರ-ರಾಯನ
ವೀರ-ಹರಿ-ಹರ-ವೊಡೆಯರ
ವೀರ-ಹರಿ-ಹರೇಶ್ವರ
ವೀರ-ಹರ್ಯಣ
ವೀರ-ಹರ್ಯಣನ
ವೀರಾಂಬಿ-ಕೆಯರ
ವೀರಾಂಬುಧಿ
ವೀರಾವೇಶ-ದಿಂದ
ವೀರು
ವೀರೇಶ್ವರ
ವೀರೊಡೆಯ-ನಿಗೆ
ವೀರೋ
ವೀಳೆಯ-ವನ್ನು
ವುಂಡಿಗೆಯ
ವೃಂದಾ-ವನಕ್ಕೆ
ವೃಂದಾ-ವ-ನದ
ವೃಂದಾ-ವ-ನನ್ನು
ವೃತಿಯಂ
ವೃತಿ-ಯನ್ನು
ವೃತ್ತವು
ವೃತ್ತಿ
ವೃತ್ತಿ-ಗಳ
ವೃತ್ತಿ-ಗಳನ್ನಾಗಿ
ವೃತ್ತಿ-ಗಳನ್ನು
ವೃತ್ತಿ-ಗ-ಳಿಗೆ
ವೃತ್ತಿಗೆ
ವೃತ್ತಿಯ
ವೃತ್ತಿ-ಯ-ನಾಯ-ಕ-ನಾಗಿದ್ದನು
ವೃತ್ತಿ-ಯ-ನಾಯ-ಕ-ನಾಗಿದ್ದ-ನೆಂದು
ವೃತ್ತಿ-ಯನ್ನು
ವೃತ್ತಿ-ಯಲ್ಲಿ
ವೃತ್ತಿ-ಯವ್ರಿತ್ತಿಯ
ವೃತ್ತಿ-ಯಾಗಿ
ವೃತ್ತಿ-ಯಿಂದ
ವೃತ್ತಿಯು
ವೃದ್ಧರು
ವೃದ್ಧಿ
ವೃದ್ಧಿ-ಗಂತವಾ-ಗುತ್ತಲಿ-ರುವ
ವೃದ್ಧಿ-ಸಿ-ಕೊಂಡನು
ವೃದ್ಧ್ಯರ್ತ್ಥ-ವಾಗಿ
ವೆಂಕಟ
ವೆಂಕಟ-ಕೃಷ್ಣ
ವೆಂಕಟ-ಕೃಷ್ಣ-ರ-ವರು
ವೆಂಕಟನ
ವೆಂಕಟ-ನನ್ನು
ವೆಂಕಟ-ನಿಗೆ
ವೆಂಕಟ-ಪತಯ್ಯ
ವೆಂಕಟ-ಪತಿ
ವೆಂಕಟ-ಪತಿಗೆ
ವೆಂಕಟ-ಪತಿ-ಮಹಾ-ರಾಯ
ವೆಂಕಟ-ಪತಿ-ಮಹಾ-ರಾಯನ
ವೆಂಕಟ-ಪತಿ-ಮಹಾ-ರಾಯರ
ವೆಂಕಟ-ಪತಿ-ಯಾ-ರನ
ವೆಂಕಟ-ಪತಿಯು
ವೆಂಕಟ-ಪತಿಯೇ
ವೆಂಕಟ-ಪತಿ-ರಾಯ
ವೆಂಕಟ-ಪತಿ-ರಾಯ-ದೇವ
ವೆಂಕಟ-ಪತಿ-ರಾಯನ
ವೆಂಕಟ-ಪತಿ-ರಾಯನು
ವೆಂಕಟ-ಪತಿ-ರಾಯರ
ವೆಂಕಟಪ್ಪ-ನಾಯ-ಕನು
ವೆಂಕಟಪ್ಪನು
ವೆಂಕಟ-ರತ್ನಂ
ವೆಂಕಟ-ರತ್ನಮ್
ವೆಂಕಟ-ರಮಣಯ್ಯನ-ವರು
ವೆಂಕಟ-ರಮಣಸ್ವಾಮಿ
ವೆಂಕಟ-ರಾವ್
ವೆಂಕಟ-ಲಕ್ಷ್ಮಮ್ಮನು
ವೆಂಕಟ-ವರ-ದಾಚಾರ್ಯ-ನಿಗೆ
ವೆಂಕಟಾದ್ರಿ
ವೆಂಕಟಾದ್ರಿಗೆ
ವೆಂಕಟಾದ್ರಿ-ನಾಯಕ
ವೆಂಕಟಾದ್ರಿ-ನಾಯ-ಕ-ನಿಂದ
ವೆಂಕಟಾದ್ರಿ-ನಾಯ-ಕ-ನಿಗೆ
ವೆಂಕಟಾದ್ರಿ-ನಾಯ-ಕನು
ವೆಂಕಟಾದ್ರಿ-ನಾಯ-ಕನೂ
ವೆಂಕಟಾದ್ರಿಯು
ವೆಂಕಟಾದ್ರಿ-ಸಮುದ್ರ-ವೆಂಬ
ವೆಂಕಟಾದ್ರೀಶ-ನಾಯ-ಕಸ್ಯ
ವೆಂಕಟೇಶ
ವೆಂಗಳ-ರಾಜಯ್ಯನು
ವೆಂಗಿ-ಮಂಡಲ
ವೆಂಗೇನ-ಹಳ್ಳಿ-ಗಳನ್ನು
ವೆಂಗೇನ-ಹಳ್ಳಿಯು
ವೆಂಜಿ-ಮಲೈ
ವೆಟ್ಟದುಳ್
ವೆಲ್ಲೂರು-ಬೆಳ್ಳೂರು
ವೆಲ್ಲೆಸ್ಲಿಗೆ
ವೆಲ್ಲೆಸ್ಲಿಯು
ವೆಲ್ಲೆಸ್ಲಿಯೊಂದಿ
ವೇಂಕಟಾದ್ರೀಶ
ವೇಂಟೆ
ವೇಂಟೆಯ
ವೇಂಟೆ-ಯದ
ವೇಂಟೆ-ಯ-ದೊಳಗೆ
ವೇಂಠಕ
ವೇಂಠೆ
ವೇಂಠೆಯ
ವೇಂಠೆ-ಯಕ್ಕೆ
ವೇಂಠೆ-ಯದ
ವೇಂಠೆ-ಯ-ದಲ್ಲಿ
ವೇಂಠೆ-ಯ-ಮಾಗಣಿ-ವಳಿತ
ವೇಗ-ಮಂಗಲ-ಇಂದಿನ
ವೇತನ
ವೇದ
ವೇದ-ಪಾಠಶಾಲೆ-ಯನ್ನು
ವೇದ-ಪುಷ್ಕರ-ಣಿ-ಯನ್ನು
ವೇದ-ಮಾರ್ಗ
ವೇದ-ವನ್ನು
ವೇದ-ವಲ್ಲಿ
ವೇದ-ಶಾಸ್ತ್ರ
ವೇದಾಂತದ
ವೇದಾಂತಿ
ವೇದಾರಣ್ಯ
ವೇದಾರಣ್ಯ-ವೆಂದೂ
ವೇದಾರಣ್ಯ-ವೆಂಬ
ವೇಳಗೆ
ವೇಳಾ-ಪುರಂ
ವೇಳೆ
ವೇಳೆ-ಗಾಗಲೇ
ವೇಳೆಗೆ
ವೇಳೆಯ
ವೇಳೆ-ವಾಳಿ
ವೇಳೆ-ವಾಳಿ-ಯಾಗಿ
ವೇಳೆ-ವಾಳಿ-ಯಾಗಿದ್ದ
ವೈ
ವೈಜ-ನಾಥ
ವೈಜ-ನಾಥಂಗೊಲವಿಂ
ವೈಜ-ನಾಥ-ದೇವರ
ವೈಜ-ನಾಥ-ದೇವ-ರಿಗೆ
ವೈದಿಕ
ವೈದಿ-ಕರೂ
ವೈದ್ಯ-ನಾಥ
ವೈದ್ಯ-ನಾಥ-ದೇವ-ರಿಗೆ
ವೈದ್ಯ-ನಾಥ-ನಿಗೆ
ವೈದ್ಯ-ನಾಥ-ಪುರ
ವೈದ್ಯ-ನಾಥ-ಪುರದ
ವೈದ್ಯ-ನಾಥ-ಮುಡೆಯಾರ್
ವೈದ್ಯನು
ವೈಭೋಗ-ವುಳ್ಳವ-ನಾಗಿಯೂ
ವೈಮ-ನಸ್ಯ
ವೈರತ್ವವು
ವೈರಮುಡಿಯ
ವೈರವು
ವೈರಿ-ಗಳ
ವೈರಿ-ಗಳನ್ನು
ವೈರಿದಿಕ್ಕುಂಜರರುಂ
ವೈರಿ-ಮಂಡ-ಳಿಕ
ವೈರಿಮದ-ಮರ್ದ್ಧನ
ವೈರಿ-ರಾಜರ
ವೈರಿಸಂಹಾರ
ವೈರಿಸಮೂಹಮಿಲ್ಲಿ
ವೈರಿ-ಸಾಮಂತ
ವೈರಿ-ಸೇನೆಯು
ವೈವಾಹಿ
ವೈವಾಹಿಕ
ವೈವಿಧ್ಯಮಯ
ವೈಶಾಖ
ವೈಶಿಷ್ಟ್ಯ
ವೈಷಮ್ಯ
ವೈಷ್ಣವ
ವೈಷ್ಣವ-ಕೇಂದ್ರ
ವೈಷ್ಣವ-ಕೇಂದ್ರ-ಗಳು
ವೈಷ್ಣವ-ದೇವಾಲಯ-ಗ-ಳಿಗೆ
ವೈಷ್ಣವ-ಧರ್ಮದ
ವೈಷ್ಣವ-ಮಹಾ-ಜನ-ಗಳು
ವೈಷ್ಣವರ
ವೈಷ್ಣವ-ರನ್ನು
ವೈಸಿ
ವೈಸ್ರಾಯ್
ವೊಂದೆ
ವೊಡ-ಸಂದರು
ವೊಡೆಯ-ಅಪ್ಪಣ್ಣ
ವೊಡೆಯ-ನಿಗೆ
ವೊಡೆಯರ
ವೊಡೆಯ-ರ-ಕೂಡೆ
ವೊಡೆಯ-ರಿಗೆ
ವೊಡೇರ
ವೊಮ್ಮಯ್ಯಮ್ಮ
ವೊಮ್ಮಾಯಮ್ಮ
ವೋಜ-ಮಂಗಲ
ವೋಡೆ
ವೋಣ-ಮಯ್ಯನ
ವೋಣ-ಮಯ್ಯ-ನೆಂದು
ವೋಲಗಿ-ಸುತ್ತಿದ್ದನು
ವ್ಯಕ್ತಪಡಿ-ಸಿದ್ದಾರೆ
ವ್ಯಕ್ತ-ವಾಗಿ-ರುವ
ವ್ಯಕ್ತ-ವಾಗುತ್ತದೆ
ವ್ಯಕ್ತ-ವಾಗುತ್ತದೆಂದು
ವ್ಯಕ್ತ-ವಾಗು-ವುದು
ವ್ಯಕ್ತಿ
ವ್ಯಕ್ತಿ-ಗಳ
ವ್ಯಕ್ತಿ-ಗಳು
ವ್ಯಕ್ತಿ-ಗಿಂತಲೂ
ವ್ಯಕ್ತಿತ್ವ
ವ್ಯಕ್ತಿ-ಯಾಗಿ-ರ-ಬಹುದು
ವ್ಯಕ್ತಿ-ಯೊಬ್ಬ-ನಿಗೆ
ವ್ಯತ್ಯಾಸ
ವ್ಯತ್ಯಾಸ-ಗಳೊಂದಿಗೆ
ವ್ಯತ್ಯಾಸ-ಗಳೊಡನೆ
ವ್ಯತ್ಯಾಸ-ವನ್ನು
ವ್ಯತ್ಯಾಸ-ವೆನ್ನ-ಬಹುದು
ವ್ಯತ್ಯಾಸ-ವೇನೂ
ವ್ಯಯಿಸಿ
ವ್ಯವ-ಸಾ-ಯಕ್ಕೆ
ವ್ಯವಸ್ಥೆ
ವ್ಯವಸ್ಥೆ-ಗಳನ್ನು
ವ್ಯವಸ್ಥೆಗೂ
ವ್ಯವಸ್ಥೆ-ಗೊಳಿಸಿ
ವ್ಯವಸ್ಥೆಯ
ವ್ಯವಸ್ಥೆ-ಯನ್ನು
ವ್ಯವಸ್ಥೆ-ಯಲ್ಲಿ
ವ್ಯವಸ್ಥೆ-ಯಲ್ಲೂ
ವ್ಯವಸ್ಥೆಯು
ವ್ಯವಸ್ಥೆಯೂ
ವ್ಯವಸ್ಥೆಯೇ
ವ್ಯವ-ಹಾರ
ವ್ಯವ-ಹಾರಕ್ಕೆ
ವ್ಯವ-ಹಾರ-ಗಳನ್ನು
ವ್ಯವ-ಹಾರ-ಗಳಲ್ಲಿ
ವ್ಯವ-ಹಾರ-ಗಳಲ್ಲೇ
ವ್ಯವ-ಹಾರ-ಗ-ಳಿಗೆ
ವ್ಯವ-ಹಾರ-ವನ್ನು
ವ್ಯಾಖ್ಯಾನಿ-ಸಿದ್ದಾರೆ
ವ್ಯಾಪ-ಕ-ವಾಗಿ
ವ್ಯಾಪಾ-ರದ
ವ್ಯಾಪಾರಿ-ಗಳ
ವ್ಯಾಪಾರಿ-ಗಳಾಗಿದ್ದಾರೆ
ವ್ಯಾಪಾರಿ-ಗಳಿಂದ
ವ್ಯಾಪಾರಿ-ಗಳು
ವ್ಯಾಪಾರಿಯು
ವ್ಯಾಪಾರಿ-ವರ್ಗದ-ವ-ರಿಗೂ
ವ್ಯಾಪಿ-ಸಿದ್ದ
ವ್ಯಾಪ್ತಿ
ವ್ಯಾಪ್ತಿಗೆ
ವ್ಯಾಪ್ತಿ-ಯನ್ನು
ವ್ಯಾಪ್ತಿ-ಯಲ್ಲಿ
ವ್ಯಾಪ್ತಿ-ಯಲ್ಲಿದ್ದ
ವ್ಯಾಪ್ತಿ-ಯೊಳಗೆ
ವ್ಯಾಸ-ತೀರ್ಥ-ರಿಗೆ
ವ್ಯಾಸ-ತೀರ್ಥರು
ವ್ಯಾಸ-ರಾಯ-ರಿಗೆ
ವ್ರಣೋಪಲಬ್ದ
ವ್ರತ-ದಿಂದ
ವ್ರತ-ದೀಕ್ಷಿತ
ವ್ರತ-ವನ್ನು
ವ್ರಿತ್ತಿ
ವೞ್ದರೆ
ವೞ್ದರೆ-ಯನ್ಯಮ್ಮೂರೊಳೆ
ಶ
ಶಂಕ-ಚಕ್ರದ
ಶಂಕರ
ಶಂಕರ-ನ-ಹಳ್ಳಿ-ಯನ್ನು
ಶಂಕರ-ನಾಯ-ಕನೇ
ಶಂಕರ-ನಾ-ರಾಯಣ
ಶಂಕರ-ಪುರ
ಶಂಕರ-ರಸ
ಶಂಕರ-ರಸ-ಸಂಕರ-ರ-ಸರ
ಶಂಖ-ಚಕ್ರ
ಶಂಖ-ಚಕ್ರದ
ಶಂತನುವು
ಶಂಭವ-ರಾಯನ
ಶಂಭು
ಶಂಭು-ದೇವ
ಶಂಭು-ದೇವ-ನಿಗೆ
ಶಂಭು-ವನ್ನು
ಶಂಭು-ವ-ರಾಯರು
ಶಂಭೂನ-ಹಳ್ಳಿ
ಶಂಭೂನ-ಹಳ್ಳಿಯ
ಶಕ
ಶಕ-ವರುಷ
ಶಕ-ವರ್ಷ
ಶಕ-ವರ್ಷ-ವನ್ನು
ಶಕ್ತತ್ರಯ-ಸಮನ್ವಿತಂ
ಶಕ್ತಿ
ಶಕ್ತಿಯ
ಶಕ್ತಿ-ಯನ್ನು
ಶಕ್ತಿ-ಸಾಮರ್ಥ್ಯ
ಶತ-ಮಾನದ
ಶತ-ಮಾನ-ದಲ್ಲಿ
ಶತ-ಮಾನ-ದ-ವ-ರೆಗೆ
ಶತ-ಮಾ-ನ-ದಿಂದಲೇ
ಶತ್ರು-ಗಳ
ಶತ್ರು-ಗಳನ್ನು
ಶತ್ರು-ಗ-ಳಿಗೆ
ಶತ್ರು-ರಾಜ-ರನ್ನು
ಶತ್ರು-ರಾಜ-ರು-ಗ-ಳಿಗೆ
ಶತ್ರುವಿನ
ಶತ್ರು-ಸೇನೆ
ಶತ್ರು-ಸೇನೆ-ಯನ್ನು
ಶನಿ-ವಾರ
ಶನಿ-ವಾರ-ಸಿದ್ಧಿ
ಶಬದ್
ಶಬ್ದ
ಶಬ್ದಕ್ಕೆ
ಶಬ್ದ-ಗಳನ್ನು
ಶಬ್ದ-ಗ-ಳಿಗೆ
ಶಬ್ದ-ಗಳು
ಶಬ್ದದ
ಶಬ್ದ-ದಿಂದ
ಶಬ್ದನ್ನು
ಶಬ್ದ-ವನ್ನು
ಶಬ್ದವು
ಶರಣಾಗತ-ನಾಗಲು
ಶರಣಾಗತ-ವಜ್ರಪಂಜರ
ಶರಣಾಗತ-ವಜ್ರಪಂಜರಂ
ಶರಧಿಗಂಭೀರ-ನೆಂದು
ಶರಭ
ಶರಾಗತಮಂದಾರಃ
ಶಶಕ-ಪುರದ
ಶಶ-ಪುರದ
ಶಶಿ-ವಂಶ-ತಿಲಕ
ಶಸ್ತ್ರಾಸ್ತ್ರ
ಶಹಾ
ಶಾಂತಲ-ದೇವಿ
ಶಾಂತಲ-ದೇವಿ-ಯರ
ಶಾಂತಲೆ-ಗಿಂತ
ಶಾಂತಲೆಯ
ಶಾಂತಲೆ-ಯರ
ಶಾಂತಲೆಯು
ಶಾಂತಿ
ಶಾಂತಿಗ್ರಾಮ
ಶಾಂತಿಗ್ರಾಮದ
ಶಾಂತಿ-ನಾಥ
ಶಾಂತಿ-ನಾಥ-ದೇವರ
ಶಾಂತಿ-ನಾಥ-ದೇವ-ರಿಗೆ
ಶಾಂತೀಶ್ವರ
ಶಾಕೇಭ್ರೇಷು
ಶಾಖೆ
ಶಾಖೆಗೆ
ಶಾಖೆಯ
ಶಾಖೆ-ಯನ್ನು
ಶಾಖೆ-ಯ-ವನಿರ-ಬಹುದು
ಶಾತ-ವಾ-ಹನರ
ಶಾನು-ಭಾಗ
ಶಾರ್ದೂಲ
ಶಾರ್ವರಿ
ಶಾಲಿ-ವಾ-ಹನ
ಶಾಲೆ-ಗಳನ್ನು
ಶಾಶ್ವತ-ವಾಗಿ
ಶಾಸ-ಗಳಲ್ಲಿ
ಶಾಸತಿ
ಶಾಸ-ದಲ್ಲಿ
ಶಾಸ-ದಿಂದಿ
ಶಾಸನ
ಶಾಸನ-ಕಾರ
ಶಾಸನ-ಕಾರನು
ಶಾಸನ-ಗಳ
ಶಾಸನ-ಗಳಂತೆ
ಶಾಸನ-ಗ-ಳನ್ನ
ಶಾಸನ-ಗಳನ್ನು
ಶಾಸನ-ಗ-ಳನ್ನೂ
ಶಾಸನ-ಗಳಲಿ
ಶಾಸನ-ಗಳಲ್ಲಂತೂ
ಶಾಸನ-ಗಳಲ್ಲಿ
ಶಾಸನ-ಗಳಲ್ಲಿದೆ
ಶಾಸನ-ಗಳಲ್ಲಿದ್ದು
ಶಾಸನ-ಗಳಲ್ಲಿಯೂ
ಶಾಸನ-ಗಳಲ್ಲಿ-ರುವ
ಶಾಸನ-ಗ-ಳಲ್ಲೂ
ಶಾಸನ-ಗಳಾಗಿದ್ದು
ಶಾಸನ-ಗಳಾಗಿವೆ
ಶಾಸನ-ಗಳಾವುವೂ
ಶಾಸನ-ಗಳಿಂದ
ಶಾಸನ-ಗಳಿಗೂ
ಶಾಸನ-ಗಳಿದ್ದು
ಶಾಸನ-ಗಳಿವೆ
ಶಾಸನ-ಗಳು
ಶಾಸನ-ಗಳೂ
ಶಾಸನ-ಗಳೇ
ಶಾಸನ-ತಜ್ಞರು
ಶಾಸನದ
ಶಾಸನ-ದಲಿ
ಶಾಸನ-ದಲ್ಲಂತೂ
ಶಾಸನ-ದಲ್ಲಿ
ಶಾಸನ-ದಲ್ಲಿ-ದಲ್ಲಿ
ಶಾಸನ-ದಲ್ಲಿದೆ
ಶಾಸನ-ದಲ್ಲಿ-ದೆೆ
ಶಾಸನ-ದಲ್ಲಿದ್ದು
ಶಾಸನ-ದಲ್ಲಿಯೂ
ಶಾಸನ-ದಲ್ಲಿರು
ಶಾಸನ-ದಲ್ಲಿ-ರುವ
ಶಾಸನ-ದಲ್ಲಿ-ರು-ವಂತೆ
ಶಾಸನ-ದಲ್ಲೂ
ಶಾಸನ-ದಲ್ಲೇ
ಶಾಸನ-ದ-ವ-ರೆಗೆ
ಶಾಸನ-ದಿಂದ
ಶಾಸನ-ದಿದ
ಶಾಸನ-ಧಾರ-ಗಳಿಂದ
ಶಾಸನ-ಪದ್ಯ-ಗಳು
ಶಾಸ-ನಲ್ಲಿ
ಶಾಸನ-ವನ್ನು
ಶಾಸನ-ವನ್ನೂ
ಶಾಸನ-ವಾಗಿದೆ
ಶಾಸನ-ವಾಗಿದ್ದು
ಶಾಸನ-ವಾಗಿ-ರ-ಬಹುದು
ಶಾಸನ-ವಾಚಕ-ಚಕ್ರ-ವರ್ತಿ
ಶಾಸನ-ವಾ-ದರೆ
ಶಾಸನ-ವಿದೆ
ಶಾಸನ-ವಿದ್ದು
ಶಾಸನ-ವಿ-ರುವ
ಶಾಸನವು
ಶಾಸನ-ವು-ನರ-ಸಿಂಹ-ನನ್ನು
ಶಾಸನ-ವು-ವೀರ-ಬಲ್ಲಾಳ
ಶಾಸನವೂ
ಶಾಸನ-ವೆಂದರೆ
ಶಾಸನ-ವೆಂದು
ಶಾಸನವೇ
ಶಾಸನ-ವೊಂದು
ಶಾಸನೋಕ್ತ
ಶಾಸನೋಕ್ತ-ನಾಗಿದ್ದಾ-ನೆಂದು
ಶಾಸನೋಕ್ತ-ನಾಗಿದ್ದು
ಶಾಸನೋಕ್ತ-ನಾದ
ಶಾಸನೋಕ್ತ-ರಾಗಿದ್ದಾರೆ
ಶಾಸನೋಕ್ತ-ರಾದ
ಶಾಸನೋಕ್ತ-ವಲ್ಲದ
ಶಾಸನೋಕ್ತ-ವಾಗಿ
ಶಾಸನೋಕ್ತ-ವಾಗಿದೆ
ಶಾಸನೋಕ್ತ-ವಾಗಿಲ್ಲ
ಶಾಸನೋಕ್ತ-ವಾಗಿವೆ
ಶಾಸನೋಕ್ತ-ವಾದ
ಶಾಸನ್ದ-ದಲ್ಲಿ
ಶಾಸ-ವನವು
ಶಾಸ-ವನವೇ
ಶಾಸ-ವನು
ಶಾಸ-ಸವು
ಶಾಸೋಕ್ತ-ವಾಗಿವೆ
ಶಾಸ್ತ್ರಾರ್ಥ
ಶಿಂಗಂಣ-ಗಳ
ಶಿಂಗಣ್ಣ
ಶಿಂಗಪ್ಪ-ನಾಯಕ
ಶಿಂಗಪ್ಪ-ನಾಯ-ಕನು
ಶಿಂಗಪ್ಪ-ನಾಯ-ಕರು
ಶಿಂಗಮಾ-ರನ-ಹಳ್ಳಿ-ಯನ್ನು
ಶಿಂಗರೈಯ್ಯಂಗಾರ
ಶಿಂಗರೈಯ್ಯಂಗಾ-ರರ
ಶಿಂಘಣ
ಶಿಂಶಾ
ಶಿಂಷಾ
ಶಿತ-ಕರ-ಗಂಡ
ಶಿಥಿಲಬೆಂಕೊಂಬರುಂ
ಶಿರಚ್ಛೇದ
ಶಿರಪ್ರಧಾನ
ಶಿರ-ಶಾ-ಸನ-ವನ್ನು
ಶಿರಸ್ತೆ-ದಾರ್
ಶಿರಸ್ತೇ-ದಾರ್
ಶಿರಸ್ಸನ್ನು
ಶಿರೋಗ್ರಮಂ
ಶಿರೋಮಣಿ
ಶಿರೋಮಣಿ-ಯಂತಿದ್ದ
ಶಿರೋಮಣಿ-ಯಂತೆ
ಶಿಲಾ
ಶಿಲಾ-ಯುಗದ
ಶಿಲಾ-ಶಾ-ಸನ
ಶಿಲಾ-ಶಾ-ಸನ-ಗಳಲ್ಲಿ
ಶಿಲಾ-ಶಾ-ಸನ-ಗಳು
ಶಿಲಾ-ಶಾ-ಸನದ
ಶಿಲಾ-ಶಾ-ಸನ-ದಲ್ಲಿ
ಶಿಲಾ-ಶಾ-ಸನ-ದಲ್ಲೂ
ಶಿಲಾ-ಶಾ-ಸನ-ವನ್ನು
ಶಿಲಾ-ಶಾ-ಸನವು
ಶಿಲಾ-ಶಾ-ಸನವೇ
ಶಿಲಾ-ಶಾಸ-ವನೇ
ಶಿಲ್ಪ-ಕಲೆ
ಶಿಲ್ಪ-ಕಲೆ-ಯಲ್ಲಿ
ಶಿಲ್ಪ-ಗಳ
ಶಿಲ್ಪ-ಗಳಲ್ಲಿ
ಶಿಲ್ಪ-ಗಳಿವೆ
ಶಿಲ್ಪ-ಗಳು
ಶಿಲ್ಪದ
ಶಿಲ್ಪವು
ಶಿಲ್ಪಾಚಾರಿ-ಯರು
ಶಿಲ್ಪಿ-ಗಳೂ
ಶಿವಃ
ಶಿವ-ಗಂಗೆಗೂ
ಶಿವ-ಗಂಗೆಗೆ
ಶಿವ-ಗಂಗೆಯ
ಶಿವ-ದೇವ
ಶಿವ-ದೇವನು
ಶಿವ-ದೇವಾಲಯ
ಶಿವನ
ಶಿವನ-ಸಮುದ್ರ
ಶಿವನ-ಸಮುದ್ರ-ಗಳನ್ನು
ಶಿವನ-ಸಮುದ್ರದ
ಶಿವನ-ಸಮುದ್ರ-ದಲ್ಲಿ
ಶಿವನ-ಸಮುದ್ರ-ದಲ್ಲಿ-ರುವ
ಶಿವನ-ಸಮುದ್ರ-ದಿಂದ
ಶಿವನ-ಸಮುದ್ರ-ದಿಂದಲೂ
ಶಿವ-ನೆಂದು
ಶಿವ-ಪುರ
ಶಿವ-ಪುರದ
ಶಿವ-ಪುರ-ದೊಳಗಣ
ಶಿವ-ಪುರ-ವನ್ನಾಗಿ
ಶಿವ-ಪುರ-ವನ್ನು
ಶಿವಭಕ್ತ
ಶಿವಮಾರ
ಶಿವಮಾ-ರನ
ಶಿವಮಾರ-ನನ್ನು
ಶಿವಮಾರ-ನಿಗೆ
ಶಿವಮಾ-ರನು
ಶಿವ-ಮಾರ-ಸಿಂಹ
ಶಿವಮಾ-ರಸ್ಯ
ಶಿವಮೊಗ್ಗ
ಶಿವರ-ಮಂಡ್ಯ-ತಾಲ್ಲೂಕಿನ
ಶಿವ-ರಾಜ
ಶಿವ-ರಾಜನು
ಶಿವರುದ್ರಸ್ವಾಮಿ
ಶಿವಶರಣ-ರಿಗೆ
ಶಿವಶರಣರು
ಶಿವಶೋಧ
ಶಿವ-ಸನ್ನಿಧಿ-ಯಲ್ಲಿ
ಶಿವಾಚಾರ
ಶಿವಾಚಾ-ರದ
ಶಿವಾಜಿಯ
ಶಿವಾರ
ಶಿವಾಲಯಕ್ಕೆ
ಶಿವಾಲಯದ
ಶಿಶಿ-ಲದ
ಶಿಷ್ಟಪ್ರತಿ-ಪಾ-ಳನ
ಶಿಷ್ಟಪ್ರಿಯ
ಶಿಷ್ಯ
ಶಿಷ್ಯನೂ
ಶಿಷ್ಯ-ರಾದ
ಶಿಷ್ಯ-ರೊಡ-ಗೂಡಿ
ಶಿಹ್ವ
ಶೀ
ಶೀಘ್ರವೇ
ಶೀರಂಗ-ದೇವ
ಶೀರ್ಯ-ಪೇಟೆ
ಶೀಳುನೆರೆ
ಶು
ಶುಕ್ರ-ವಾರ
ಶುದ್ಧ
ಶುದ್ಧೋಭಯಾನ್ವಯ
ಶುಭ
ಶುಭ-ಚಂದ್ರ
ಶುಭ-ಚಂದ್ರ-ಸಿದ್ಧಾಂತ
ಶುಭ-ದೀಯಾರಭ-ವತ್ಸದಾ
ಶುಭ-ಯಸಿ
ಶುಭಾ-ವತೈಃ
ಶುಭೈಃ
ಶೂದ್ರಕಂ
ಶೂದ್ರಕ-ನೆಂದು
ಶೂದ್ರ-ಕುಲದ
ಶೂದ್ರರು
ಶೂದ್ರವಾಡ-ವಾಗಿದ್ದ
ಶೂರನು
ಶೂರ-ಯತಾ
ಶೂರರು
ಶೃಂಗಾರಹಾರ
ಶೃಂಗೇರಿ
ಶೃಂಗೇ-ರಿಗೆ
ಶೃಂಗೇರಿ-ಯಲ್ಲಿ
ಶೆಟ್ಟಿ
ಶೆಟ್ಟಿ-ಹಳ್ಳಿ
ಶೆಟ್ಟಿ-ಹಳ್ಳಿ-ಗಳಲ್ಲಿ
ಶೇಖರ-ಮಣಿ
ಶೇಖ್
ಶೇಖ್ದಾರ್
ಶೇಲೆಯ-ಪುರ-ಸೇಲಂದ
ಶೇವೆ
ಶೈವ
ಶೈವಕ್ಷೇತ್ರ-ವಾದ
ಶೈವ-ಧರ್ಮ-ವನ್ನು
ಶೈವ-ಧರ್ಮವು
ಶೈವ-ನಾ-ದರೂ
ಶೈವ-ನೊಬ್ಬನ
ಶೈವ-ಪಂಗಡಕ್ಕೆ
ಶೈವ-ಯತಿ
ಶೈವ-ರಾಗಿದ್ದ
ಶೈವ-ಳೆಂದೂ
ಶೈವ-ಶಿಲ್ಪ-ಗಳು
ಶೈವ-ಸಂಸ್ಥೆ-ಗ-ಳಿಗೆ
ಶೋಬಾರ್ಥ-ವಾಗಿಯೂ
ಶೋಭಾ
ಶೋಭಿಸು-ವಂತೆ
ಶೌಚ
ಶೌಚ-ಮಣಲೆ-ಯರುಂ
ಶೌರ್ಯ
ಶೌರ್ಯದಿಂ
ಶೌರ್ಯ-ದಿಂದ
ಶೌರ್ಯ-ವನ್ನು
ಶೌರ್ಯಾಟೋಪ-ದೊಳು
ಶೌರ್ಯ್ಯದಿಂ
ಶ್ಯಾನು-ಭೋಗ
ಶ್ಯಾವೆ
ಶ್ರತಿ-ಯೊಳಗಣ
ಶ್ರದ್ಧೆ
ಶ್ರಮಿ-ಸಿದ
ಶ್ರವಣ-ಕಾಲ-ದಲ್ಲಿ
ಶ್ರವ-ಣನ-ಹಳ್ಳಿ
ಶ್ರವಣ-ಬೆಳಗೊಳ
ಶ್ರವಣ-ಬೆಳಗೊಳಕ್ಕೆ
ಶ್ರವಣ-ಬೆಳಗೊಳ-ಗಳು
ಶ್ರವಣ-ಬೆಳಗೊಳದ
ಶ್ರವಣ-ಬೆಳಗೊಳ-ದಲ್ಲಿ
ಶ್ರವಣ-ಬೆಳಗೊಳ-ವನ್ನು
ಶ್ರವಣ-ಬೆಳಗೊಳವು
ಶ್ರವಣ-ಬೆಳಗೊಳವೇ
ಶ್ರವಣ-ಬೆಳಗೊಳ-ಶಾ-ಸನ-ದಲ್ಲಿ
ಶ್ರಾವಣ
ಶ್ರಿ
ಶ್ರಿಮದ್ರಾಜ-ಗುರು
ಶ್ರಿಮನ್ಮಹಾ
ಶ್ರಿಮನ್ಮಹಾಪ್ರಧಾನ
ಶ್ರಿವೈಷ್ಣವ-ರಿಗೆ
ಶ್ರೀ
ಶ್ರೀಕಂಠ-ದೇವ
ಶ್ರೀಕಂಠ-ಶಾಸ್ತ್ರಿ-ಯ-ವರ
ಶ್ರೀಕರಣ
ಶ್ರೀಕರ-ಣಂಗಳು
ಶ್ರೀಕರ-ಣದ
ಶ್ರೀಕರ-ಣ-ದ-ಧಿಷ್ಟಯಕ
ಶ್ರೀಕರ-ಣ-ದ-ಹೆಗ್ಗಡೆ
ಶ್ರೀಕರ-ಣಪ್ರಮುಖ
ಶ್ರೀಕರ-ಣರ
ಶ್ರೀಕರ-ಣರು
ಶ್ರೀಕರ-ಣಾಗ್ರಗಣ್ಯ
ಶ್ರೀಕರ-ಣಾಗ್ರಗಣ್ಯ-ನಾಗಿದ್ದನು
ಶ್ರೀಕರ-ಣಾಗ್ರಗಣ್ಯನೂ
ಶ್ರೀಕರ-ಣಾಗ್ರಗಣ್ಯರು
ಶ್ರೀಕರ-ಣಾಧಿ-ಕಾರಿ
ಶ್ರೀಕರ-ಣಾಧಿ-ಪತಿ
ಶ್ರೀಕಾರ್ಯಕ್ಕೆ
ಶ್ರೀಕೃಷ್ಣ
ಶ್ರೀಕೃಷ್ಣ-ರಾಯ
ಶ್ರೀಕೋವಿ-ರಾಜ
ಶ್ರೀಗೂ-ರನ-ಮಠ
ಶ್ರೀಚಾಮುಣ್ಡ-ರಾಜಂ
ಶ್ರೀದೇವಿ
ಶ್ರೀಧರ
ಶ್ರೀಧರಯ್ಯನ
ಶ್ರೀನಾರ-ಸಿಂಹ-ದೇವರು
ಶ್ರೀನಾ-ರಾಯ-ಣ-ದೇವರ
ಶ್ರೀನಿ-ವಾಸ
ಶ್ರೀನಿ-ವಾಸನ
ಶ್ರೀನಿ-ವಾಸ-ನಿಗೆ
ಶ್ರೀನಿ-ವಾಸ-ರಾವು
ಶ್ರೀನಿ-ವಾಸಾಚಾರಿಯ
ಶ್ರೀನಿ-ವಾಸಾ-ಚಾರ್ಯರೆಂಬ
ಶ್ರೀನೊಳಂಬ
ಶ್ರೀಪರು-ಷನು
ಶ್ರೀಪಾದ-ಗಳ
ಶ್ರೀಪಾದ-ವನ್ನು
ಶ್ರೀಪಾಲ
ಶ್ರೀಪಾಳತ್ರೈವಿದ್ಯ-ದೇವ-ರೆಂದು
ಶ್ರೀಪುರದ
ಶ್ರೀಪುರ-ದಲ್ಲಿ
ಶ್ರೀಪುರ-ವಾಗಿ-ರ-ಬಹು-ದೆಂದು
ಶ್ರೀಪುರ-ಷನ
ಶ್ರೀಪುರುಷ
ಶ್ರೀಪುರುಷನ
ಶ್ರೀಪುರುಷ-ನನ್ನು
ಶ್ರೀಪುರುಷ-ನಿಗೆ
ಶ್ರೀಪುರುಷನು
ಶ್ರೀಪೃಥ್ವೀ
ಶ್ರೀಪೆರುಮಾಳೆ-ದೇವ-ದಂಣಾಯಕರ
ಶ್ರೀಬಾ-ಣದ
ಶ್ರೀಬಾಸ
ಶ್ರೀಬಾಸಣ್ಣನ
ಶ್ರೀಭಂಡಾರಕ್ಕೆ
ಶ್ರೀಭಂಡಾರದ
ಶ್ರೀಭಂಡಾರವೂ
ಶ್ರೀಭಾಷ್ಯ
ಶ್ರೀಭೂಮಿ
ಶ್ರೀಮಂತ
ಶ್ರೀಮಂತಿಕೆ-ಗಾಗಿ
ಶ್ರೀಮಂನ್
ಶ್ರೀಮತು
ಶ್ರೀಮತ್
ಶ್ರೀಮತ್ಪಶ್ಚಿಮ-ರಂಗ-ನಾಥ
ಶ್ರೀಮತ್ಪೆರ್ಗ್ಗಡೆ
ಶ್ರೀಮದ-ನಾದಿಯಗ್ರಹಾರಂ
ಶ್ರೀಮದ್
ಶ್ರೀಮದ್ರಾಜಾಧಿ-ರಾಜ
ಶ್ರೀಮದ್ರಾ-ರಾಜಾಧಿ-ರಾಜ
ಶ್ರೀಮನು
ಶ್ರೀಮನು-ಮಹಾಪ್ರಧಾನ
ಶ್ರೀಮನು-ಮಹಾ-ಸಾಮಂತ
ಶ್ರೀಮನ್
ಶ್ರೀಮನ್ನೊಳಂಬ
ಶ್ರೀಮನ್ಮಣಲಯ-ರನ
ಶ್ರೀಮನ್ಮಹಾ
ಶ್ರೀಮನ್ಮಹಾ-ನಾಯಂಕಾಚಾರ್ಯ
ಶ್ರೀಮನ್ಮಹಾ-ನಾಯಕ
ಶ್ರೀಮನ್ಮಹಾ-ನಾಯ-ಕಾಚಾರ್ಯ
ಶ್ರೀಮನ್ಮಹಾ-ಪಸಾಯ್ತ
ಶ್ರೀಮನ್ಮಹಾಪ್ರಧಾನ
ಶ್ರೀಮನ್ಮಹಾಪ್ರಧಾನ-ನೆಂದು
ಶ್ರೀಮನ್ಮಹಾಪ್ರಧಾನಿ
ಶ್ರೀಮನ್ಮಹಾಪ್ರಧಾನೆ-ರೆಂದು
ಶ್ರೀಮನ್ಮಹಾಪ್ರಭು
ಶ್ರೀಮನ್ಮಹಾ-ಮಂಡ-ಲೇಶ್ವರ
ಶ್ರೀಮನ್ಮಹಾ-ಮಂಡ-ಲೇಶ್ವರ-ನೆಂದೇ
ಶ್ರೀಮನ್ಮಹಾ-ಮಂಡ-ಳೇಶ್ವರ
ಶ್ರೀಮನ್ಮಹಾ-ರಾಜಾಧಿ-ರಾಜ
ಶ್ರೀಮನ್ಮಹಾ-ವೀರ-ರಾಜೇಂದ್ರ
ಶ್ರೀಮನ್ಮಹಾ-ಸಾಮಂತ
ಶ್ರೀಮನ್ಮಹಾ-ಸಾಮಂತನ
ಶ್ರೀಮನ್ಮಹಾ-ಸಾಮಂತ-ರಾದ
ಶ್ರೀಮನ್ಮಹಾ-ಸಾಮಂತಾಧಿ-ಪತಿ
ಶ್ರೀಮಪ್ರತಿಷ್ಟ-ವೀರಪ್ರಾಜ್ಯ-ರಾಜ್ಯ
ಶ್ರೀಮಲ್ಲಿ-ಕಾರ್ಜುನ-ಮಹಾ-ರಾಯರ
ಶ್ರೀಮಾರ-ಮಯ್ಯ
ಶ್ರೀಯ-ಗಾಮುಣ್ಡರು
ಶ್ರೀಯುಳ್ಳಿನ
ಶ್ರೀರಂಗ
ಶ್ರೀರಂಗಂ
ಶ್ರೀರಂಗ-ಐ-ದನೇ
ಶ್ರೀರಂಗಕ್ಕೆ
ಶ್ರೀರಂಗ-ಗಳ
ಶ್ರೀರಂಗದ
ಶ್ರೀರಂಗ-ದಿಂದ
ಶ್ರೀರಂಗ-ದೇವನ
ಶ್ರೀರಂಗ-ದೇವ-ರಾಯ
ಶ್ರೀರಂಗನ
ಶ್ರೀರಂಗ-ನಾಥ-ದೇವ-ರಿಗೆ
ಶ್ರೀರಂಗ-ನಾಯ-ಕಿ-ದೇವಿ-ಯರ
ಶ್ರೀರಂಗನು
ಶ್ರೀರಂಗ-ನೆಂದು
ಶ್ರೀರಂಗ-ಪಟಕ್ಕೆ
ಶ್ರೀರಂಗ-ಪಟ್ಟ
ಶ್ರೀರಂಗ-ಪಟ್ಟಣ
ಶ್ರೀರಂಗ-ಪಟ್ಟಣಕ್ಕೆ
ಶ್ರೀರಂಗ-ಪಟ್ಟಣ-ಗಳನ್ನು
ಶ್ರೀರಂಗ-ಪಟ್ಟಣ-ಗಳು
ಶ್ರೀರಂಗ-ಪಟ್ಟಣದ
ಶ್ರೀರಂಗ-ಪಟ್ಟಣ-ದಲು
ಶ್ರೀರಂಗ-ಪಟ್ಟಣ-ದಲ್ಲಿ
ಶ್ರೀರಂಗ-ಪಟ್ಟಣ-ದಲ್ಲಿದ್ದಾಗ
ಶ್ರೀರಂಗ-ಪಟ್ಟಣ-ದಿಂದ
ಶ್ರೀರಂಗ-ಪಟ್ಟಣ-ರಾಜ್ಯ-ಗಳು
ಶ್ರೀರಂಗ-ಪಟ್ಟಣ-ವನ್ನು
ಶ್ರೀರಂಗ-ಪಟ್ಟಣವು
ಶ್ರೀರಂಗ-ಪಟ್ಟಣ-ಸೀಮೆಯ
ಶ್ರೀರಂಗ-ಪಟ್ಟಣಸ್ಥಳದ
ಶ್ರೀರಂಗ-ಪಟ್ಟಣೇ
ಶ್ರೀರಂಗ-ಪುರ
ಶ್ರೀರಂಗ-ಪುರದ
ಶ್ರೀರಂಗ-ಪುರ-ದಶ್ರೀ-ರಂಗ-ಪಟ್ಟಣ
ಶ್ರೀರಂಗ-ಪುರ-ವಾದ
ಶ್ರೀರಂಗ-ಮಂಟಪ-ವನ್ನು
ಶ್ರೀರಂಗ-ಮಹಾ-ರಾಯರು
ಶ್ರೀರಂಗಮ್
ಶ್ರೀರಂಗ-ರಾಜ
ಶ್ರೀರಂಗ-ರಾಜ-ದೇವ
ಶ್ರೀರಂಗ-ರಾಜನ
ಶ್ರೀರಂಗ-ರಾಜನು
ಶ್ರೀರಂಗ-ರಾಮ-ದೇವ-ರಾಯ
ಶ್ರೀರಂಗ-ರಾಯ-ದೇವರು
ಶ್ರೀರಂಗ-ರಾಯನ
ಶ್ರೀರಂಗ-ರಾಯನು
ಶ್ರೀರಂಗ-ರಾಯ-ಮಹಾ-ರಾಯರು
ಶ್ರೀರಂಗ-ರಾಯರ
ಶ್ರೀರಂಗೇ
ಶ್ರೀರರ್ದ್ಧನಾರೀನಟೇಶ್ವರಃ
ಶ್ರೀರಾಜ್ಯ-ವೆಂಬ
ಶ್ರೀರಾಮ-ಕೃಷ್ಣ-ದೇವರ
ಶ್ರೀರಾಮ-ನಾಥ-ದೇವರ
ಶ್ರೀರಾಮಾ-ನುಜ-ರಲ್ಲಿ
ಶ್ರೀರಾಮೇಶ್ವರ
ಶ್ರೀಲಕುಮಿ
ಶ್ರೀವಲ್ಲಭ-ನೆಂಬ
ಶ್ರೀವಿಕ್ರಮನ
ಶ್ರೀವಿಜಯ
ಶ್ರೀವಿನ-ಯಾದಿತ್ಯ-ಪೊಯ್ಸಳ-ನೆ-ರೆಯಂಗ
ಶ್ರೀವಿಷ್ಣು-ಭೂಪಾಳಕಂ
ಶ್ರೀವೀರಪ್ರತಾಪ
ಶ್ರೀವೀರ-ರಾಮ-ದೇವ-ರಾಯರು
ಶ್ರೀವೈಷ್ಣವ
ಶ್ರೀವೈಷ್ಣವಕ್ಷೇತ್ರ-ವಾದ
ಶ್ರೀವೈಷ್ಣವರ
ಶ್ರೀವೈಷ್ಣವ-ರಾದ
ಶ್ರೀವೈಷ್ಣವ-ರಿ-ಗಾಗಿ
ಶ್ರೀವೈಷ್ಣವ-ರಿಗೆ
ಶ್ರೀವೈಷ್ಣವರು
ಶ್ರೀವೈಷ್ಣವಸ್ಥಳ-ವಾದ
ಶ್ರೀಶೈಲ
ಶ್ರೀಶೈಲದ
ಶ್ರೀಶೈಲ-ದಲ್ಲಿ
ಶ್ರೀಹೋಸಲ-ನಾ-ಡಿನ
ಶ್ರೀಹ್ರೀಧೃತಿದ್ಧಾರ್ಯತಾಂ
ಶ್ರುತ
ಶ್ರುತಿಯ
ಶ್ರುತಿಯು
ಶ್ರುತಿಶ್ರೋತ್ರಿಯೂರು
ಶ್ರೇಣಿ
ಶ್ರೇಣಿ-ಗಳಾಗಿದ್ದು
ಶ್ರೇಣಿಯ
ಶ್ರೇಣೀಕೃತ
ಶ್ರೇಷ್ಠ-ನಾಗಿದ್ದನು
ಶ್ರೇಷ್ಠ-ನಾದ
ಶ್ರೇಷ್ಠ-ನಾದ-ವನು
ಶ್ರೋತ್ರೀಯ
ಶ್ರೋತ್ರೀಯ-ವಾಗಿ
ಶ್ರೋತ್ರೀ-ಯಸ್ಯ
ಶ್ಲೋಕ-ದಲ್ಲಿ
ಶ್ಲೋಕ-ವಿದೆ
ಶ್ಲೋಕಾನ್
ಷಣ್ಣ-ವತಿ
ಷಣ್ಣ-ವತಿ-ಸಹಸ್ರ
ಷಣ್ಣ-ವತಿ-ಸಹಸ್ರ-ವಿಷಯ
ಷಣ್ಮುಖ
ಷರತ್ತಿನ
ಷಷ್ಠಿಯಂದು
ಷಹಬಾಜ್
ಷೇಕ್
ಸಂ
ಸಂಕಡಿಸಂನಾಹ
ಸಂಕಮ-ದೇವನ
ಸಂಕ-ಮನು
ಸಂಕ-ರಪ್ಪ
ಸಂಕಲ್ಪಿ-ಸಿದನು
ಸಂಕ-ಹಳ್ಳಿ
ಸಂಕಿ-ಯರ
ಸಂಕಿ-ಯರ-ಕುಲ-ತಿಲಕ
ಸಂಕಿರಣ-ವನ್ನು
ಸಂಕೀರ್ಣ
ಸಂಕೀರ್ಣ-ಶಾ-ಸನ
ಸಂಕೋಚದಾಯಿ
ಸಂಕ್ರಾಂತಿಯ
ಸಂಕ್ಷಿಪ್ತ
ಸಂಕ್ಷಿಪ್ತ-ವಾಗಿ
ಸಂಕ್ಷಿ-ಸುತ್ತಾರೆ
ಸಂಖ್ಯಾ
ಸಂಖ್ಯೆ
ಸಂಖ್ಯೆ-ಗಳನ್ನು
ಸಂಖ್ಯೆ-ಗಳು
ಸಂಖ್ಯೆ-ಯನ್ನು
ಸಂಖ್ಯೆ-ಯಲಿ
ಸಂಖ್ಯೆ-ಯಲ್ಲಿ
ಸಂಖ್ಯೆ-ಯಲ್ಲಿವೆ
ಸಂಗತಿ
ಸಂಗತಿ-ಯಾಗುತ್ತದೆ
ಸಂಗನ-ಬಸ-ವನ
ಸಂಗಮ
ಸಂಗಮ-ಕುಮಾರ-ನಾದ
ಸಂಗಮದ
ಸಂಗಮನ
ಸಂಗಮ-ನಿಂದ
ಸಂಗಮರ
ಸಂಗಮ-ರಾಯ-ಬುಕ್ಕ-ರಾಯ-ಹರಿ-ಹರ-ರಾಯ-ದೇವ-ರಾಯ-ವಿಜೆಯ-ರಾಯ-ಗಜಬೇಂಟೆಕಾಱ
ಸಂಗಮ-ವಂಶಾ-ವಳಿ-ಯನ್ನು
ಸಂಗಮ-ಸೋ-ದರರು
ಸಂಗಮೇಶ್ವರ-ಪುರ-ವಾದ
ಸಂಗಮೇಶ್ವರ-ಪುರ-ವೆಂಬ
ಸಂಗಮೇಶ್ವರ-ರಾಯ
ಸಂಗರಕೆ
ಸಂಗೀತ
ಸಂಗ್ರಹಕ್ಕೆ
ಸಂಗ್ರ-ಹಣೆ
ಸಂಗ್ರಹ-ವಾಗಿ
ಸಂಗ್ರಹಾಲ-ಯವು
ಸಂಗ್ರಹಿಸಿ
ಸಂಗ್ರಹಿಸಿ-ದರು
ಸಂಗ್ರಹಿಸಿದ್ದಾರೆ
ಸಂಗ್ರಹಿ-ಸುವ
ಸಂಗ್ರಾಮ
ಸಂಗ್ರಾಮ-ಭೀಮ-ಯೆಂಬ
ಸಂಗ್ರಾಮ-ರಂಗ
ಸಂಗ್ರಾಮ-ರಾಮ
ಸಂಘಕ್ಕೆ
ಸಂಘಟಿ-ಸಲು
ಸಂಘಟ್ಟ
ಸಂಘಡಿಸ್ಟ್ರಿಕ್ಟ್
ಸಂಘದ
ಸಂಘ-ವೆಂದೂ
ಸಂಘಸಂಸ್ಥೆ-ಗ-ಳಿಗೆ
ಸಂಚರಿಸಿ
ಸಂಚಾರ
ಸಂಚಾರ-ಗಳಿಗೂ
ಸಂಚಿಕೆ-ಗಳಲ್ಲಿ
ಸಂಚಿಗ
ಸಂಚಿತ
ಸಂಚಿಯ
ಸಂಜಾತಂ
ಸಂಜಾತ-ನಾಗಿದ್ದು
ಸಂಜ್ಞಿಕೇ
ಸಂತತಂ
ಸಂತತಿ
ಸಂತ-ತಿಯ
ಸಂತತಿ-ಯಲ್ಲಿ
ಸಂತತಿ-ಯ-ವನೋ
ಸಂತಾನಕ್ಕಾಗಿ
ಸಂತಾನ-ವಾಗಿ
ಸಂತೆ
ಸಂತೆ-ಯ-ಕರದ
ಸಂತೆ-ಯನ್ನು
ಸಂತೆ-ಯಾ-ಗಿತ್ತು
ಸಂತೆಯು
ಸಂತೆ-ಶಿ-ವರ
ಸಂತೇ-ಬಾಚ-ಹಳ್ಳಿ
ಸಂತೇ-ಬಾಚ-ಹಳ್ಳಿಯ
ಸಂತೇ-ಬಾಚ-ಹಳ್ಳಿ-ಯನ್ನು
ಸಂತೇ-ಬಾಚ-ಹಳ್ಳಿ-ಯಲ್ಲಿ
ಸಂತೇ-ಬಾಚ-ಹಳ್ಳಿ-ಯಲ್ಲಿದೆ
ಸಂತೋಷ-ದಿಂದ
ಸಂತ್ರಾಸಿ-ನೃಪಾ-ಪದಃ
ಸಂಥೆ-ಶಾ-ಸನ
ಸಂದ
ಸಂದರ್ಭಕ್ಕೆ
ಸಂದರ್ಭ-ಗಳಲ್ಲಿ
ಸಂದರ್ಭ-ಗಳಲ್ಲಿಯೂ
ಸಂದರ್ಭ-ದಲಿ
ಸಂದರ್ಭ-ದಲ್ಲಿ
ಸಂದರ್ಭೋಚಿತ-ವಾಗಿ
ಸಂದರ್ಶಿಸಿರ
ಸಂದ-ಳಾದಿವಂ
ಸಂದಾಯ
ಸಂದಾಯ-ವಾ-ಯಿತೆಂದು
ಸಂಧಿ-ವಿಗ್ರಹಿ
ಸಂನಾಹ-ಮಾವ-ನಂಕಕಾಱ
ಸಂಪಂನ-ರು-ಮಪ್ಪ
ಸಂಪಟಕ್ಕೆ
ಸಂಪಟು-ಗಳ
ಸಂಪತ್ಕರ
ಸಂಪತ್ಕರ-ನಾ-ರಾಯ-ಣ-ದೇವರು
ಸಂಪತ್ತಿಗೆ
ಸಂಪದಂ
ಸಂಪದ್ಭರಿತ
ಸಂಪನನ್ನನುಂ
ಸಂಪನ್ನಂ
ಸಂಪನ್ನ-ನಪ್ಪ
ಸಂಪನ್ನ-ರು-ಮಪ್ಪ
ಸಂಪನ್ಮೂಲ-ಗಳಿ-ರುವ
ಸಂಪರ್ಕ
ಸಂಪಾಕ-ದರು
ಸಂಪಾದ-ಕರು
ಸಂಪಾದ-ಕರೂ
ಸಂಪಾದ-ನೆ-ಯಿಂದ
ಸಂಪಾದಿಸಿ
ಸಂಪಾದಿ-ಸಿದ
ಸಂಪಾದಿ-ಸಿದ್ದ
ಸಂಪಾದಿಸಿ-ರುವ
ಸಂಪುಟ
ಸಂಪುಟ-ಗಳ
ಸಂಪುಟ-ಗಳನ್ನು
ಸಂಪುಟ-ಗಳಲ್ಲಿ
ಸಂಪುಟ-ಗಳಲ್ಲಿ-ರುವ
ಸಂಪುಟ-ಗಳಿಂದ
ಸಂಪುಟ-ಗ-ಳಿಗೆ
ಸಂಪುಟ-ಗಳು
ಸಂಪುಟ-ಗಳೂ
ಸಂಪುಟದ
ಸಂಪುಟ-ದಲ್ಲಿ
ಸಂಪೂರ್ಣ
ಸಂಪೂರ್ಣ-ವಾಗಿ
ಸಂಪ್ರದಾಯ-ವನ್ನು
ಸಂಪ್ರದಾ-ಯವೂ
ಸಂಪ್ರಾಪ್ಯ
ಸಂಪ್ರೀ-ತನಾದ
ಸಂಬಂಧ
ಸಂಬಂಧ-ಗಳ
ಸಂಬಂಧ-ಗಳನ್ನು
ಸಂಬಂಧ-ಗಳಿದ್ದವು
ಸಂಬಂಧ-ಗಳು
ಸಂಬಂಧ-ದಲ್ಲಿ
ಸಂಬಂಧ-ದಿಂದ
ಸಂಬಂಧ-ಪಟ್ಟಂತೆ
ಸಂಬಂಧ-ಪಟ್ಟ-ವ-ರಲ್ಲ-ವೆಂದೂ
ಸಂಬಂಧ-ಪಟ್ಟ-ವ-ರಾಗಿದ್ದಾರೆ
ಸಂಬಂಧ-ಪಟ್ಟಿ-ರ-ಬೇಕು
ಸಂಬಂಧ-ವನ್ನು
ಸಂಬಂಧ-ವಿಲ್ಲ-ವೆಂದು
ಸಂಬಂಧ-ವೇನು
ಸಂಬಂಧ-ವೇ-ನೆಂದು
ಸಂಬಂಧಿ-ಗ-ಳಾದ
ಸಂಬಂಧಿ-ಯಾದ
ಸಂಬಂಧಿ-ಸ-ಸಿದ
ಸಂಬಂಧಿ-ಸಿದ
ಸಂಬಂಧಿ-ಸಿ-ದಂತಹ
ಸಂಬಂಧಿ-ಸಿ-ದಂತೆ
ಸಂಬಂಧಿ-ಸಿದೆ
ಸಂಬಂಧಿ-ಸಿದ್ದಾಗಿದೆ
ಸಂಬಂಧಿ-ಸಿದ್ದು
ಸಂಬಂಧಿ-ಸಿ-ರ-ಬಹು-ದೆಂದು
ಸಂಬಳ-ಕೊಡಲಾಗದೆ
ಸಂಬುವ-ಗಉಡ
ಸಂಬೋಧನೆಗೆ
ಸಂಬೋಧಿ-ಸ-ಲಾಗಿದೆ
ಸಂಬೋಧಿಸಿ
ಸಂಬೋಧಿಸಿ-ರುವ
ಸಂಬೋಧಿಸಿವೆ
ಸಂಭವ
ಸಂಭವ-ನೀ-ಯವಲ್ಲ
ಸಂಭವ-ರಾಯ
ಸಂಭವಿ-ಸಿದ
ಸಂಭಾಜಿಯು
ಸಂಭಾ-ವನೆ
ಸಂಭಾ-ವನೆ-ಯಲ್ಲಿ
ಸಂಭು-ರಾಯ
ಸಂಭುವ-ಗವುಡ
ಸಂಭುವ-ರಾಯ
ಸಂಭೂತದ
ಸಂಮಟಿ-ಭಾಗ
ಸಂಮಟಿ-ಭಾಗ-ಭೂ-ಪತಿ
ಸಂಯುಕ್ತೋ
ಸಂರಕ್ಷಿತ
ಸಂರಕ್ಷಿಸಿ
ಸಂವ
ಸಂವತ್ಸರದ
ಸಂವತ್ಸರ-ದಲ್ಲಿ
ಸಂವತ್ಸರ-ವನ್ನು
ಸಂವ-ಸನ್
ಸಂಶಂಕರ
ಸಂಶೋಧ-ಕನು
ಸಂಶೋಧನಾ
ಸಂಶೋಧನಾತ್ಮಕ
ಸಂಶೋಧನೆ
ಸಂಶೋಧ-ನೆಯ
ಸಂಸ್ಕೃತ
ಸಂಸ್ಕೃತ-ಕನ್ನಡ
ಸಂಸ್ಕೃ-ತದ
ಸಂಸ್ಕೃತ-ಶಾ-ಸನ
ಸಂಸ್ಕೃತಿ
ಸಂಸ್ಕೃತಿಯ
ಸಂಸ್ಕೃತಿಯು
ಸಂಸ್ಕೃತೀ-ಕರ-ಣದ
ಸಂಸ್ಥಂ
ಸಂಸ್ಥತೋ-ನೃಪಃ
ಸಂಸ್ಥಾನ
ಸಂಸ್ಥಾನಕ್ಕೆ
ಸಂಸ್ಥಾನದ
ಸಂಸ್ಥಾನ-ದಲ್ಲಿ
ಸಂಸ್ಥಾನ-ದಲ್ಲಿದ್ದ
ಸಂಸ್ಥಾನ-ದಲ್ಲಿ-ರುವ
ಸಂಸ್ಥಾನ-ವನ್ನು
ಸಂಸ್ಥಾಪಿತ-ರಾದರು
ಸಂಸ್ಥೆಯ
ಸಂಸ್ಥೆಯಲ್ಲಿರುವ
ಸಂಸ್ಥೆ-ಯಾ-ಗಿತ್ತು
ಸಂಸ್ಥೆಯು
ಸಂಹರಿ-ಸಿದ
ಸಂಹರಿ-ಸಿದ-ನೆಂದು
ಸಂಹರಿ-ಸಿದ-ನೆಂದೂ
ಸಂಹರಿ-ಸು-ವಲ್ಲಿ
ಸಕಲ
ಸಕಲ-ಧರ್ಮ-ಗಳಂ
ಸಕಲ-ರಾಜ್ಯಾಧಿಪ-ತಿ-ಗ-ಳಾದ
ಸಕಲ-ವಿದ್ಯಾ-ನಿಧಿ
ಸಕಲ-ವಿದ್ಯಾ-ವಿಶಾ-ರದ-ರಾದ
ಸಕ-ಲೇಶ್ವರ
ಸಕಳಕಳಾ-ವಿಧಾನ-ಪದ್ಮಾ-ಸನ
ಸಕಳ-ಧರ್ಮ್ಮೋದ್ಧಾರಕ
ಸಕಳಿ-ಗವುಡ-ನನ್ನು
ಸಕಳಿ-ಗವುಡನು
ಸಕುಟುಂಬ-ಸಮೇತ-ನಾಗಿ
ಸಕ್ಕರೆ
ಸಕ್ಕರೆ-ಪಟ್ಟಣ-ದಿಂದ
ಸಕ್ಕ-ರೆಯ
ಸಕ್ಕರೆ-ಶೆಟ್ಟಿಯು
ಸಕ್ಕಿಯಾಗೆ
ಸಕ್ರಿ-ಯ-ವಾಗಿ
ಸಖ್ಯ-ವನ್ನು
ಸಗರ
ಸಗರ-ಕುಲ
ಸಗರ-ಕುಲ-ತಿಲಕ-ನೆಂಬ
ಸಗರ-ಕುಲದ
ಸಗರತ್ರಿಣೇತ್ರ
ಸಗರ-ವಂಶ
ಸಗರ-ವಂಶ-ಜರು
ಸಗರ-ವಂಶದ
ಸಗರ-ವಂಶ-ದ-ವ-ರಾಗಿದ್ದು
ಸಗರ-ವಂಶ-ದ-ವರು
ಸಚಿವ
ಸಚಿವಂ
ಸಚಿವ-ನಾಗಿ
ಸಚಿವ-ನಿಗೆ
ಸಚಿ-ವನೂ
ಸಚಿವ-ಮಂತ್ರಿ
ಸಚಿ-ವರು
ಸಚಿವಾಧೀಶ್ವರ
ಸಚಿವೋಭ-ವತ್
ಸಜ್ಜನ-ರಾದ
ಸಜ್ಜನರು
ಸಜ್ಜ-ನಾಮೋದ
ಸಜ್ಜು-ಗೊಳಿಸಿ
ಸಡಿಲಿಸಿ
ಸಣಬ
ಸಣ-ಬಿನ-ಹಳ್ಳಿ
ಸಣ-ಬಿಮುಕ-ಳಿಯ
ಸಣ್ಣ
ಸಣ್ಣ-ಕಟ್ಟೆ-ಯಿದ್ದು
ಸಣ್ಣ-ಪುಟ್ಟ
ಸಣ್ಣ-ಸಣ್ಣ
ಸಣ್ಣೇನ-ಹಳ್ಳಿ
ಸಣ್ನೆ-ನಾಡನ್ನು
ಸಣ್ನೆ-ನಾ-ಡಿಗೆ
ಸಣ್ನೆ-ನಾಡು
ಸತತ
ಸತತಂ
ಸತತ-ವಾಗಿ
ಸತಾದೃಗ್ಗುಣ
ಸತಿ
ಸತೀಶ್
ಸತ್ಕರಿ-ಸುತ್ತಿದ್ದಾಗ
ಸತ್ಕಾರ್ಯ-ನಿರತ-ವಾಗು-ವಂತೆ
ಸತ್ತ
ಸತ್ತನು
ಸತ್ತ-ನೆಂದ
ಸತ್ತ-ನೆಂದಿದೆ
ಸತ್ತ-ನೆಂದು
ಸತ್ತ-ರೆಂದು
ಸತ್ತ-ವರ
ಸತ್ತಾಗ
ಸತ್ತಿ-ಗನ-ಹಳ್ಳದ
ಸತ್ತಿದ್ದಾ-ನೆಂದು
ಸತ್ತಿ-ರ-ಬಹುದು
ಸತ್ತ್ಯ-ದಲಿ
ಸತ್ಯಕೆ
ಸತ್ಯದ
ಸತ್ಯ-ನಾ-ರಾಯಣ
ಸತ್ಯಭಾಮಾ
ಸತ್ಯ-ಮಂಗಲ
ಸತ್ಯ-ರಾಧೇಯ
ಸತ್ಯ-ರಾಧೇಯನುಂ
ಸತ್ಯ-ವಾಕ್ಯ
ಸತ್ಯ-ವಾಕ್ಯನ
ಸತ್ಯ-ವಾಕ್ಯನು
ಸತ್ಯ-ವಾಕ್ಯ-ಪೆರ್ಮಾನ-ಡಿಯ
ಸತ್ಯ-ವಾಕ್ಯ-ಪೆರ್ಮಾನ-ಡಿ-ಯ-ಇಮ್ಮಡಿ
ಸತ್ಯ-ವಾಕ್ಯ-ಪೆರ್ಮಾನ-ಡಿಯು
ಸತ್ರಾಧಿ-ಕಾರಿ-ಯಾಗಿದ್ದ-ನೆಂದು
ಸದರಿ
ಸದ-ಸಿವ-ರಾಯರು
ಸದಸ್ಯ-ರನ್ನು
ಸದಸ್ಯರೇ
ಸದಾಚಾರಿ
ಸದಾ-ಶಿವ-ದೇವ
ಸದಾ-ಶಿವ-ದೇವ-ರಾಯನ
ಸದಾ-ಶಿವ-ದೇವ-ರಾಯ-ನಿಂದ
ಸದಾ-ಶಿವ-ನಿಗೆ
ಸದಾಶಿವ-ಮಹಾ-ರಾಯಕ್ಷಮಾ-ನಾಯಕಃ
ಸದಾಶಿವ-ಮಹಾ-ರಾಯನ
ಸದಾಶಿವ-ಮಹಾ-ರಾಯನು
ಸದಾಶಿವ-ಮಹಾ-ರಾಯರು
ಸದಾ-ಶಿ-ವರ
ಸದಾ-ಶಿ-ವರಾಯ
ಸದಾ-ಶಿ-ವರಾ-ಯನ
ಸದಾ-ಶಿ-ವರಾಯ-ನನ್ನು
ಸದಾ-ಶಿ-ವರಾಯ-ನಿಗೆ
ಸದಾಶಿವ-ರಾಯನು
ಸದಾ-ಶಿ-ವರಾಯ-ನೆಂದೂ
ಸದಾ-ಶಿ-ವರಾಯ-ರಿಗೆ
ಸದಿಯು
ಸದು-ಗುಣ-ಸಮೇತ
ಸದೆಬಡಿದು
ಸದ್ಗುಣ
ಸದ್ಗುಣ-ದೊಳಧಿಕ-ತೇಜಂ
ಸನು-ಮಂತ್ರಿ-ಗಳೆನಿಸಿ
ಸನ್ತೊನಮಾಳೆ
ಸನ್ನಿಧಿ-ಯಲಿ
ಸನ್ನಿಧಿ-ಯಲ್ಲಿ
ಸನ್ನಿವೇಶ-ದಲ್ಲಿ
ಸನ್ನೆ-ಯನ್ನು
ಸನ್ಮಥ-ದಿಂದ
ಸನ್ಮಾರ್ಗ
ಸನ್ಯ-ಸನ
ಸನ್ಯ-ಸನ-ದಿಂದ
ಸನ್ಯಾಸಿ-ಪುರ
ಸಪ್ತಮ-ಭಾಗೆ-ಯನು
ಸಪ್ತ-ಸಾ-ಗರ
ಸಪ್ತಾಂಗ-ಲಕ್ಷ್ಮೀ
ಸಪ್ತಾಷ್ಟ
ಸಫಲ-ನಾ-ದನು
ಸಬಂಧ-ವೇ-ನೆಂದು
ಸಬ್ಡಿವಿ-ಜನ್
ಸಬ್ಡಿವಿ-ಜನ್ಗಳನ್ನು
ಸಬ್ಡಿವಿ-ಜನ್ನ್ನು
ಸಬ್ತಾಲ್ಲೂಕು
ಸಬ್ಬಗೊಡುಗೆ-ಯಾಗಿ
ಸಭೆ
ಸಭೆಯ
ಸಭೆ-ಯಾ-ಗಿತ್ತು
ಸಭೆ-ಯಾಗಿ-ರ-ಬಹುದು
ಸಭೆ-ಯೋ-ಜನ
ಸಭೆ-ಸೇರಿ
ಸಮ
ಸಮ-ಕಾಲಿನ
ಸಮ-ಕಾಲೀನ
ಸಮ-ಕಾಲೀನ-ನಾಗಿದ್ದ
ಸಮ-ಕಾಲೀನ-ನಾಗಿದ್ದ-ನೆಂದು
ಸಮ-ಕಾಲೀನ-ನಾಗಿದ್ದು
ಸಮ-ಕಾಲೀನ-ರಾಗಿ
ಸಮ-ಕಾಲೀನರು
ಸಮ-ಕಾಲೀನ-ರೆಂದು
ಸಮ-ಕಾಲೀನವೂ
ಸಮಕ್ಷಮ
ಸಮಗ್ರ
ಸಮಗ್ರ-ಬಲದ
ಸಮಗ್ರ-ಬಲ-ನಿಲಯೇ
ಸಮಗ್ರ-ಬಲ-ವನ್ನು
ಸಮಗ್ರ-ವಾಗಿ
ಸಮಚಿತ್ತ-ದಿಂದ
ಸಮತಟ್ಟು
ಸಮಧಿಗತ
ಸಮ-ನಾದ
ಸಮ-ನೆಂಬೊಂದು
ಸಮನ್ವಯತೆ-ಯನ್ನು
ಸಮಯ-ಗಳಲ್ಲಿಯೂ
ಸಮಯ-ಜೈನ-ಧರ್ಮ
ಸಮಯ-ದಲ್ಲಿ
ಸಮಯ-ದ-ವರು
ಸಮರ
ಸಮರ-ಗಳಲ್ಲಿ
ಸಮರ-ಧಾರ-ಧರಂ
ಸಮರ-ಧುರೀಣ-ರಾಗಿ
ಸಮರ-ಮುಖಲ-ಸದ್
ಸಮರಾಧಿತತ್ರಿ-ವರ್ಗ್ಗ
ಸಮರ್ಥ-ನಲ್ಲದ
ಸಮರ್ಥನೆ
ಸಮರ್ಥನೆ-ಯನ್ನು
ಸಮರ್ಥ-ವಾಗಿ
ಸಮರ್ಥಿ-ಸಿದ್ದಾರೆ
ಸಮರ್ಥಿ-ಸುತ್ತದೆ
ಸಮರ್ಥಿ-ಸುತ್ತವೆ
ಸಮರ್ಥಿ-ಸುತ್ತವೆಂದು
ಸಮರ್ಪ-ಕ-ವಾಗಿ
ಸಮರ್ಪಕ-ವಾಗುತ್ತದೆ
ಸಮಸ್ತ
ಸಮಸ್ತ-ಗವುಡು-ಗಳು
ಸಮಸ್ತ-ಗುಣ-ಸಂಪನ್ನ
ಸಮಸ್ತ-ಗುಣ-ಸಂಪನ್ನ-ರು-ಮಪ್ಪ
ಸಮಸ್ತ-ಭಾಗ್ಯೈಃ
ಸಮಸ್ತ-ರಾಜ್ಯ-ಭಾರ
ಸಮಸ್ತರು
ಸಮಸ್ಯೆ-ಯನ್ನು
ಸಮಾಜಃ
ಸಮಾಜಕ್ಕೆ
ಸಮಾಜದ
ಸಮಾಜಶ್ಚಾಮ-ರಾಜೇಂದ್ರ
ಸಮಾಧಿ
ಸಮಾಧಿ-ಗಳ
ಸಮಾಧಿ-ಗಳಿದ್ದು
ಸಮಾಧಿ-ಗಳು
ಸಮಾಧಿ-ಗಳುಳ್ಳ
ಸಮಾಧಿ-ಗುಹೆ
ಸಮಾಧಿ-ಮರಣ-ವನ್ನಪ್ಪಿ-ದನು
ಸಮಾಧಿ-ಮರಣ-ವನ್ನು
ಸಮಾಧಿಯ
ಸಮಾಧಿ-ಯನ್ನು
ಸಮಾಧಿಯೆ
ಸಮಾನ
ಸಮಾನತೆ
ಸಮಾನ-ನಾದ
ಸಮಾನ-ರಾಗಿಯೂ
ಸಮಾನ-ರೂಪ
ಸಮಾನ-ಳಾಗಿದ್ದ-ಳೆಂದು
ಸಮಾನ-ವಾಗಿ
ಸಮಾನ-ವಾಗಿಯೂ
ಸಮಾನ-ವಾದ
ಸಮಾನ-ವಾದುದೇ
ಸಮಾನಾರ್ಥಕ-ಗಳು
ಸಮಾಪ್ತಿ-ಗೊಳಿಸಿ-ದನು
ಸಮಾಯಯೌ
ಸಮಾಲೋಚಕ-ರಲ್ಲಿ
ಸಮಾಶ್ರಿತ
ಸಮಾಹೂಯ
ಸಮಿತಿಗೆ
ಸಮೀದಲ್ಲಿದೆ
ಸಮೀಪ
ಸಮೀ-ಪದ
ಸಮೀಪ-ದಲ್ಲಿ
ಸಮೀಪ-ದಲ್ಲಿದ್ದ
ಸಮೀಪ-ದಲ್ಲಿಯೇ
ಸಮೀಪ-ದಲ್ಲಿ-ರುವ
ಸಮೀಪ-ದಲ್ಲೇ
ಸಮೀಪ-ವಿ-ರುವ
ಸಮುಂನ
ಸಮುದಾಯ-ಗ-ಳಿಗೆ
ಸಮುದಾಯ-ದ-ವರು
ಸಮುದ್ಧರಣ
ಸಮುದ್ಧರಣಂ
ಸಮುದ್ಧರ-ಣನುಂ
ಸಮುದ್ಧರ-ಣನೂ
ಸಮುದ್ಧರ-ಣ-ವನ್ನು
ಸಮುದ್ರ
ಸಮುದ್ರ-ಪಾಂಡ್ಯ
ಸಮುದ್ರಾಧಿ-ಪತಿ
ಸಮುದ್ರಾಧೀಶ್ವರ
ಸಮೂಹದ
ಸಮೂಹ-ವನ್ನು
ಸಮೃದ್ಧ
ಸಮೃದ್ಧ-ವಾಗಿತ್ತೆಂದು
ಸಮೃದ್ಧ-ವಾಗು-ವು-ದಕ್ಕೆ
ಸಮೃದ್ಧಿ-ಯನ್ನು
ಸಮೃದ್ಧಿ-ಯನ್ನೂ
ಸಮೃದ್ಧಿ-ಯಾಗಿದ್ದಿತೆಂದು
ಸಮೇತ
ಸಮೇತ-ನಾಗಿ
ಸಮೇತ-ರಾದ
ಸಮ್ಮಸ್ತ
ಸಮ್ಮುಖ-ದಲ್ಲಿ
ಸಮ್ಯಕ್ತ್ವ
ಸಮ್ಯಕ್ತ್ವ-ಚೂಡಾಮಣಿ
ಸಯಿಗೋಲಪಾರ್ತನುಂ
ಸಯ್ಯದ್
ಸರ-ಗೂರ
ಸರ-ಗೂರಿನ
ಸರ-ಗೂರು
ಸರ-ಣಾಗತ
ಸರ-ದಾರ-ರಾದ
ಸರ-ಬ-ರಾಜು
ಸರಸ್ವತಿ
ಸರಿ
ಸರಿ-ಗಟ್ಟುವ
ಸರಿ-ಮಕ್ಕನ-ಹಳ್ಳಿ
ಸರಿ-ಯಲ್ಲ
ಸರಿ-ಯಾಗಿ
ಸರಿ-ಯಾಗಿದೆ
ಸರಿ-ಸ-ಮಾನ-ರಲ್ಲ-ವೆಂದು
ಸರಿ-ಸ-ಮಾನ-ರಾಗಿ
ಸರಿ-ಸ-ಮಾನ-ವಾದ
ಸರಿ-ಹೊಂದಿಸಿ
ಸರಿ-ಹೊಂದುತ್ತದೆ
ಸರಿ-ಹೊಂದುವ
ಸರೀರ
ಸರೀರ-ಸಂಪತ್ತಿಗೆ
ಸರೋ-ವರ
ಸರೋ-ವರ-ವನ್ನು
ಸರ್
ಸರ್ಇ-ಅಸ್ಕರ್
ಸರ್ಕಾರಕ್ಕೆ
ಸರ್ಕಾರದ
ಸರ್ಕಾರ-ದಲ್ಲಿ
ಸರ್ಕಾರೆ
ಸರ್ವ
ಸರ್ವಜ್ಞ
ಸರ್ವಜ್ಞ-ನಪರಿಮಿತ-ದಾನ-ವಿನೋದ-ಶೀಳ
ಸರ್ವಜ್ಞ-ಪುರ-ವೆಂಬ
ಸರ್ವಜ್ಞ-ವಿಷ್ಣು-ಭಟ್ಟಯ್ಯನು
ಸರ್ವ-ಧರ್ಮ
ಸರ್ವ-ನಮಸ್ಯ-ವಾಗಿ
ಸರ್ವ-ಬಾಧಾಪರಿಹಾರ-ವಾಗಿ
ಸರ್ವ-ಭೂ-ತಾನು-ಕಂಪಿನಃ
ಸರ್ವ-ಮಾನ್ಯ
ಸರ್ವ-ಮಾನ್ಯ-ವಾಗಿ
ಸರ್ವ-ಮಾನ್ಯ-ವಾದ
ಸರ್ವ-ವಿದ್ಯಾ-ವಿಚಕ್ಷ-ಣನು
ಸರ್ವ-ವಿದ್ಯಾ-ಸುವೈಚಕ್ಷಣ್ಯಂ
ಸರ್ವಸ್ವ-ವನ್ನೂ
ಸರ್ವಸ್ವಾಧೀನ-ವಾ-ದರೆ
ಸರ್ವಾಧಿ-ಕಾರಿ
ಸರ್ವಾಧಿ-ಕಾರಿ-ಗಳು
ಸರ್ವಾಧಿ-ಕಾರಿ-ಗಳೂ
ಸರ್ವಾಧಿ-ಕಾರಿಯ
ಸರ್ವಾಧಿ-ಕಾರಿ-ಯಾಗಿ
ಸರ್ವಾಧಿ-ಕಾರಿ-ಯಾಗಿದ್ದ
ಸರ್ವಾಧಿ-ಕಾರಿ-ಯಾದನು
ಸರ್ವಾಧಿ-ಕಾರಿ-ಯಾದರೂ
ಸರ್ವಾಧಿ-ಕಾರಿಯೂ
ಸರ್ವಾಧ್ಯಕ್ಷ-ನಾಗಿದ್ದ-ನೆಂದು
ಸರ್ವಾಧ್ಯಕ್ಷನೂ
ಸರ್ವಾಧ್ಯಕ್ಷ-ನೆಂದು
ಸರ್ವೋತ್ತಮ
ಸರ್ವೋರ್ವೀಶನತಃ
ಸರ್ವ್ವ-ಧರ್ಮ್ಮರಹಸ್ಯಸ್ಯ
ಸರ್ವ್ವಧಾರ್ಯಾಹ್ವಯೇ
ಸಲ
ಸಲಕ-ರಾಜು
ಸಲಗೆ
ಸಲಾಕೆ-ಯನ್ನು
ಸಲಾಕೆ-ಯಿಂದ
ಸಲಾಕೆ-ಯಿಂದಲೇ
ಸಲಾಕೆ-ಯೊಂದನ್ನು
ಸಲಿ-ಸುತ್ತ-ಮಿರೆ
ಸಲುವ
ಸಲೆ
ಸಲ್ಯದ-ಚಲ್ಯ
ಸಲ್ಲಿಸ-ಬೇಕಾಗುತ್ತಿತ್ತು
ಸಲ್ಲಿಸಿ
ಸಲ್ಲಿಸಿದ
ಸಲ್ಲಿಸಿ-ದನು
ಸಲ್ಲಿಸಿದ್ದಾರೆ
ಸಲ್ಲಿ-ಸುತ್ತಿದ್ದರು
ಸಲ್ಲಿ-ಸುತ್ತಿದ್ದ-ರೆಂದು
ಸಲ್ಲಿ-ಸುವ
ಸಲ್ಲುತ್ತಿತ್ತೆಂದು
ಸಲ್ಲುತ್ತಿದ್ದ
ಸಲ್ಲುವ
ಸಲ್ಲು-ವಂತೆ
ಸಲ್ಲುವು-ದೆಂದು
ಸಲ್ಲೇಖನ
ಸಳ
ಸಳನ
ಸಳ-ನನ್ನು
ಸಳ-ನಿಗೆ
ಸಳನು
ಸಳ-ನೆಂದು
ಸಳ-ನೆಂಬ
ಸಳ-ಪಯ್ಯ
ಸಳಿ-ಗವುಡ
ಸವಿಲಾಸಮಾಸ
ಸಶ್ಯಾಲ-ಪುರದ
ಸಶ್ಯಾಲ-ಪುರ-ವನ್ನುಯ
ಸಸಿ-ಯಾ-ಲದ
ಸಸಿ-ಯಾ-ಲದ-ಪುರ
ಸಸಿ-ಯಾ-ಲದ-ಪುರ-ವನ್ನು
ಸಸಿಯಾಲ-ಪುರ
ಸಸಿಯಾಲ-ಪುರಕ್ಕೆ
ಸಸಿ-ರರ್ಬ್ಬಲ್ಲ-ವರೆಮ್ಮೞ್ದಕ್ಕೆ
ಸಸ್ಯ
ಸಹ
ಸಹ-ಕರಿ-ಸಿದರು
ಸಹ-ಕಾರ-ದಿಂದ
ಸಹಜ
ಸಹಜ-ವಾಗಿದೆ
ಸಹಸ್ರ
ಸಹಸ್ರ-ಕಿಳಲೆ-ನಾಡು-ಕೆಳಲೆ-ನಾಡು
ಸಹಸ್ರ-ಗಳು
ಸಹಸ್ರ-ದೊಳಗೆ
ಸಹಸ್ರ-ನಾ-ಮ-ಗಳನ್ನು
ಸಹಸ್ರ-ಬಾಹು
ಸಹಸ್ರ-ವಿಷಯ
ಸಹಾಯ
ಸಹಾಯಕ
ಸಹಾಯ-ಕ-ನಾಗಿ
ಸಹಾಯ-ಕ-ನಾಗಿದ್ದನು
ಸಹಾಯ-ಕ-ರಾಗಿ
ಸಹಾಯ-ಕ-ರಾಗಿದ್ದ-ರೆಂದು
ಸಹಾಯ-ಕ-ರಾಗಿದ್ದು
ಸಹಾ-ಯಕ್ಕೆ
ಸಹಾಯ-ದಿಂದ
ಸಹಿತ
ಸಹಿತಂ
ಸಹಿತಃ
ಸಹಿತ-ವಾಗಿ
ಸಹೋದರ
ಸಹೋದ-ರ-ನಾಗಿ-ರ-ಬಹುದು
ಸಹೋದ-ರ-ನಾದ
ಸಹೋದ-ರರ
ಸಹೋದ-ರ-ರಾಗಿದ್ದ-ರೆಂಬುದು
ಸಹೋದ-ರ-ರಾಗಿ-ರ-ಬಹುದು
ಸಹೋದ-ರ-ರಾದ
ಸಹೋದ-ರ-ರಿಗೂ
ಸಹೋದ-ರರು
ಸಹೋದ-ರಿಗೆ
ಸಾ
ಸಾಂಕೇತಿ-ಸುತ್ತಿದೆ
ಸಾಂಗತ್ಯ-ದಲ್ಲಿ
ಸಾಂತತ್ಯ
ಸಾಂತಿ-ಯಕ್ಕ
ಸಾಂದರ್ಭಿ-ಕ-ವಾಗಿ
ಸಾಂದರ್ಭೋಚಿತ-ವಾಗಿ
ಸಾಂಪ್ಪನ-ಹಳ್ಳಿ
ಸಾಂಪ್ರದಾಯಕ
ಸಾಂಪ್ರದಾಯಕ-ವಾಗಿ
ಸಾಂಪ್ರದಾಯಿ-ಕ-ವಾಗಿ
ಸಾಂಬ್ರಾಜ್ಯಂಗಯಿಉತ
ಸಾಂಬ್ರಾಜ್ಯಂಗೈಯುತ್ತಿ-ರಲು
ಸಾಂಸ್ಕೃತಿ
ಸಾಂಸ್ಕೃತಿಕ
ಸಾಂಸ್ಕೃತಿ-ಕ-ವಾಗಿ
ಸಾಕಲ್ಯ-ವಾಗಿ
ಸಾಕಷ್ಟು
ಸಾಕ್ಷಾತ್
ಸಾಕ್ಷಿ
ಸಾಕ್ಷಿ-ಗಳಾಗಿ
ಸಾಕ್ಷಿ-ಗಳಾಗಿದ್ದಾರೆ
ಸಾಕ್ಷಿ-ಗಳಾಗಿ-ರುತ್ತಾರೆ
ಸಾಕ್ಷಿಣಃ
ಸಾಕ್ಷಿ-ಯಾಗಿ
ಸಾಕ್ಷಿ-ಯಾಗಿದೆ
ಸಾಕ್ಷಿ-ಯಾಗಿದ್ದ-ನೆಂದು
ಸಾಕ್ಷಿ-ಯಾಗಿದ್ದ-ರೆಂದು
ಸಾಕ್ಷಿ-ಯಾಗಿ-ರಲು
ಸಾಕ್ಷಿ-ಯಾಗಿ-ರುತ್ತಾನೆ
ಸಾಕ್ಷಿ-ಯಾಗಿ-ರುತ್ತಾರೆ
ಸಾಕ್ಷಿ-ಯಾಗಿ-ರು-ವುದು
ಸಾಕ್ಷಿ-ಯಾಗಿವೆ
ಸಾಗಿ-ಸುತ್ತಿದ್ದನು
ಸಾಡತಿ-ಕಾತಿ
ಸಾಣೆ-ಹಳ್ಳಿ
ಸಾಣೆ-ಹಳ್ಳಿಯ
ಸಾತ-ನೂರು
ಸಾತ-ನೂರು-ಗಳು
ಸಾತಿಗ್ರಾಮ
ಸಾತಿ-ಸೆಟ್ಟಿ
ಸಾದಿಪ್ಪ
ಸಾದಿ-ಯಪ್ಪ
ಸಾದಿಯಪ್ಪನ
ಸಾದಿ-ಯಪ್ಪ-ನಿಗೆ
ಸಾದು-ಗೊಂಡ-ನ-ಹಳ್ಳಿ
ಸಾದು-ಪುರ
ಸಾದೊಳಲು
ಸಾಧನ-ಗಳು
ಸಾಧನೆ
ಸಾಧನೆ-ಗಳ
ಸಾಧನೆ-ಗಳನ್ನು
ಸಾಧನೆ-ಗಳಲ್ಲಿ
ಸಾಧನೆ-ಗಳು
ಸಾಧ-ನೆಯ
ಸಾಧಾರ
ಸಾಧಾರ-ವಾಗಿ
ಸಾಧಿ-ಸಲು
ಸಾಧಿಸಿ
ಸಾಧಿಸಿ-ಕೊಟ್ಟ-ನೆಂದು
ಸಾಧಿ-ಸಿದ
ಸಾಧಿಸಿ-ದ-ರಾರ್ಪ್ಪಾಂಡ್ಯೇಶನಂ
ಸಾಧಿಸಿ-ದರು
ಸಾಧಿಸಿವೆ
ಸಾಧ್ಯ
ಸಾಧ್ಯತೆ
ಸಾಧ್ಯ-ತೆಯೂ
ಸಾಧ್ಯ-ವಾಗುತ್ತದೆ
ಸಾಧ್ಯ-ವಾದ
ಸಾಧ್ಯ-ವಿಲ್ಲ
ಸಾಧ್ಯವೇ
ಸಾಮಂತ
ಸಾಮಂತಂ
ಸಾಮಂತ-ದೇವ
ಸಾಮಂತನ
ಸಾಮಂತ-ನನ್ನಾಗಿ
ಸಾಮಂತ-ನಾಗಿ
ಸಾಮಂತ-ನಾಗಿದ್ದ
ಸಾಮಂತ-ನಾಗಿದ್ದನು
ಸಾಮಂತ-ನಾಗಿದ್ದ-ನೆಂದು
ಸಾಮಂತ-ನಾಗಿದ್ದು
ಸಾಮಂತ-ನಾಗಿ-ರ-ಬಹುದು
ಸಾಮಂತ-ನಾದ
ಸಾಮಂತ-ನಾ-ದರೂ
ಸಾಮಂತ-ನಿರ-ಬಹುದು
ಸಾಮಂತನು
ಸಾಮಂತ-ನೆಂದರೆ
ಸಾಮಂತ-ನೆಂದು
ಸಾಮಂತ-ನೆಂದೂ
ಸಾಮಂತ-ನೆನಿಸಿದ
ಸಾಮಂತನೋ
ಸಾಮಂತ-ಪದ-ವಿ-ಗೇರಿ-ರು-ವುದು
ಸಾಮಂತ-ಪದ-ವಿ-ಯನ್ನು
ಸಾಮಂತರ
ಸಾಮಂತ-ರ-ಗಂಡ
ಸಾಮಂತ-ರನ್ನಾಗಿ
ಸಾಮಂತ-ರನ್ನು
ಸಾಮಂತ-ರಾಗಿ
ಸಾಮಂತ-ರಾಗಿದ್ದ
ಸಾಮಂತ-ರಾಗಿದ್ದರು
ಸಾಮಂತ-ರಾಗಿದ್ದು
ಸಾಮಂತ-ರಾಗಿ-ರ-ಬಹುದು
ಸಾಮಂತ-ರಾಜ-ನೆಂದೂ
ಸಾಮಂತ-ರಾದ
ಸಾಮಂತ-ರಿಗೆ
ಸಾಮಂತರು
ಸಾಮಂತ-ರು-ಗಳು
ಸಾಮಂತರೂ
ಸಾಮಂತ-ರೆಲ್ಲರಂ
ಸಾಮಂತ-ರೊಡ-ಗೂಡಿದ
ಸಾಮಂತರೋ
ಸಾಮಂತ-ಸೋಮನ
ಸಾಮಂತ-ಸೋಮನು
ಸಾಮಂತಾಧಿ-ಪತಿ
ಸಾಮಂತಾಧಿ-ಪತಿ-ಯಾಗಿದ್ದ-ನೆಂದು
ಸಾಮಗ್ರಿ-ಯನ್ನಾಗಿ
ಸಾಮನ್ತ
ಸಾಮ-ಮತರು
ಸಾಮರ್ಥ್ಯ-ಗಳಿ-ಗನು-ಗುಣ-ವಾಗಿ
ಸಾಮರ್ಥ್ಯ-ದಿಂದ
ಸಾಮಾಜಿ
ಸಾಮಾಜಿಕ
ಸಾಮಾಜ್ಯದ
ಸಾಮಾನ್ಯ
ಸಾಮಾನ್ಯ-ವಾಗಿ
ಸಾಮಾನ್ಯ-ವಾ-ಗಿತ್ತು
ಸಾಮಿಸಂಕಡಿ
ಸಾಮ್ಯ
ಸಾಮ್ಯ-ದಿಂದ
ಸಾಮ್ಯ-ವನು
ಸಾಮ್ರಾಜ್ಯ
ಸಾಮ್ರಾಜ್ಯಂಗೈಯುತ್ತಿ-ರುವಲ್ಲಿ
ಸಾಮ್ರಾಜ್ಯಕ್ಕೆ
ಸಾಮ್ರಾಜ್ಯ-ಗ-ಳಿಗೆ
ಸಾಮ್ರಾಜ್ಯದ
ಸಾಮ್ರಾಜ್ಯ-ದಲ್ಲಿ
ಸಾಮ್ರಾಜ್ಯ-ದಿಂದ
ಸಾಮ್ರಾಜ್ಯ-ಮಮ್
ಸಾಮ್ರಾಜ್ಯ-ರಮಾಮಣಿಯ
ಸಾಮ್ರಾಜ್ಯ-ವನ್ನಾಗಿ
ಸಾಮ್ರಾಜ್ಯ-ವನ್ನು
ಸಾಮ್ರಾಜ್ಯ-ವನ್ನೂ
ಸಾಮ್ರಾಜ್ಯವು
ಸಾಮ್ರಾಟ-ರಲ್ಲಿಯೇ
ಸಾಮ್ರಾಟ-ರಾದ
ಸಾಯಣ್ಣ
ಸಾಯಣ್ಣನು
ಸಾಯಿರ
ಸಾಯಿರವೂ
ಸಾಯುತ್ತಾನೆ
ಸಾರಂಗಿ
ಸಾರಿ-ದ-ರೆಂದು
ಸಾರಿ-ದ-ವನು
ಸಾರೆಯ್ಕ
ಸಾರ್ದ್ದುದಾ
ಸಾರ್ವಭೌಮ
ಸಾಲ-ಗಾವುಂಡ
ಸಾಲ-ಗಾವುಂಡನು
ಸಾಲನ್ನೂ
ಸಾಲ-ಮಂನೆಯ
ಸಾಲಿಗ್ರಾಮ
ಸಾಲಿಗ್ರಾಮದ
ಸಾಲಿಗ್ರಾಮ-ವಾಗಿದೆ
ಸಾಲಿದೆ
ಸಾಲು
ಸಾಲು-ಗಳ
ಸಾಲು-ಗಳಲ್ಲಿ
ಸಾಲು-ಗಳಿವೆ
ಸಾಲು-ಗಳು
ಸಾಲೂರು-ಮಠದ
ಸಾಲೆ
ಸಾಳ
ಸಾಳುವ
ಸಾಳುವ-ಗಜ-ಸಿಂಹ
ಸಾಳುವ-ತಿಕ್ಕಮನ
ಸಾಳುವನ
ಸಾಳುವ-ನರ-ಸಿಂಗನ
ಸಾಳುವ-ನರ-ಸಿಂಗನು
ಸಾಳ್ವ
ಸಾವಂತ
ಸಾವಂತ-ಗೌಡ
ಸಾವಂತನ
ಸಾವಂತನು
ಸಾವಕಾಶ-ವಾಗಿ
ಸಾವನ್ತ
ಸಾವಿನ
ಸಾವಿ-ಮಲೆ
ಸಾವಿ-ಯಣ್ಣ
ಸಾವಿಯಬ್ಬೇಶ್ವರಕ್ಕೆ
ಸಾವಿರ
ಸಾವಿರ-ಕೊಳಗ
ಸಾವಿರಕ್ಕೂ
ಸಾವಿರ-ವನ್ನು
ಸಾವು
ಸಾವು-ಕ-ಗವುಡ
ಸಾವೆ-ಗಿರಿ-ಯ-ವ-ರೆಗೆ
ಸಾಶನಮ
ಸಾಸ-ನದ
ಸಾಸ-ಲಿನ
ಸಾಸಲು
ಸಾಸಿರ
ಸಾಸಿರ-ಕಬ್ಬಹು
ಸಾಸಿರ-ಕಳ್ವಪ್ಪು
ಸಾಸಿರಕ್ಕೆ
ಸಾಸಿರದ
ಸಾಸಿರ-ದಲ್ಲಿ
ಸಾಸಿರ-ದೊಳಗೆ
ಸಾಸಿರ-ದೊಳು
ಸಾಸ್ತ್ರ-ವಿನೋದನುಂ
ಸಾಹಣಿರು
ಸಾಹಳ್ಳಿ
ಸಾಹಸ
ಸಾಹಸಕ್ಕಾಗಿ
ಸಾಹಸ-ಗಳನ್ನು
ಸಾಹಸ-ಗಳಿಂದ
ಸಾಹಸ-ವನ್ನು
ಸಾಹಸಿ-ಯಾದ
ಸಾಹಿತ್ಯ
ಸಾಹಿತ್ಯದ
ಸಾಹಿತ್ಯಿಕ
ಸಾಹಿರ್
ಸಾಹೇಬ್
ಸಿ
ಸಿಂ
ಸಿಂಗ-ಟ-ಗೆರೆಯ
ಸಿಂಗಡಿ
ಸಿಂಗ-ಣಾಖ್ಯ
ಸಿಂಗಣ್ಣ
ಸಿಂಗದಂ
ಸಿಂಗ-ನಪಳ್ಳಿ
ಸಿಂಗ-ನ-ಹಳ್ಳಿ
ಸಿಂಗ-ಪೆರುಮಾಳೆ
ಸಿಂಗ-ಪೆರುಮಾಳ್
ಸಿಂಗ-ಪೆರುಮಾಳ್ಗೆ
ಸಿಂಗಪ್ಪ
ಸಿಂಗಪ್ಪ-ನಾಯ-ಕನು
ಸಿಂಗಮ್ಮ
ಸಿಂಗಯ್ಯ
ಸಿಂಗಯ್ಯನು
ಸಿಂಗ-ರಾಜಯ್ಯ-ನೆಂದೂ
ಸಿಂಗ-ರಾರ್ಯ-ನನ್ನು
ಸಿಂಗಾಡಿದೇವನ
ಸಿಂಗಾಡಿದೇವ-ನೆಂದು
ಸಿಂಗಾಡಿದೇವ-ನೆಂಬ
ಸಿಂಗಾಡಿ-ನಾಯಕ
ಸಿಂಗಾ-ರಾರ್ಯನ
ಸಿಂಗಾ-ರಾರ್ಯನು
ಸಿಂಗಾ-ರಾರ್ಯ-ರಿಂದ
ಸಿಂಗೆಯ
ಸಿಂಗೆಯ-ದಂಡ-ನಾಯ-ಕನ
ಸಿಂಗೆಯ-ದಂಡ-ನಾಯ-ಕನು
ಸಿಂಗೆಯ-ದಂಡ-ನಾಯ-ಕನೂ
ಸಿಂಗೆಯ-ದಂಡ-ನಾಯ-ಕ-ರು-ಗಳು
ಸಿಂಗೆಯ-ದಂಣ್ನಾಯ-ಕರು
ಸಿಂಗೆಯ-ನಾಯಕ
ಸಿಂಘ-ಣನ
ಸಿಂದ-ಗೆರೆ
ಸಿಂದ-ಘಟ್ಟ
ಸಿಂದ-ಘಟ್ಟಕ್ಕೆ
ಸಿಂದ-ಘಟ್ಟದ
ಸಿಂದ-ಘಟ್ಟ-ವನ್ನು
ಸಿಂದ-ಘಟ್ಟವು
ಸಿಂದ-ಘಟ್ಟ-ಸೀಮೆಯ
ಸಿಂದು-ಘಟ್ಟ
ಸಿಂಧ
ಸಿಂಧ-ಗಿರಿ-ಸಿಂಧ-ಗೆರೆ
ಸಿಂಧ-ಗೆರೆ
ಸಿಂಧ-ಗೆರೆಯ
ಸಿಂಧ-ಗೆರೆ-ಯನ್ನು
ಸಿಂಧ-ಗೋವಿಂದ
ಸಿಂಧ-ಘಟ್ಟ
ಸಿಂಧ-ಘಟ್ಟದ
ಸಿಂಧ-ಘಟ್ಟಸ್ಯ
ಸಿಂಧ-ರ-ಸರು
ಸಿಂಧು
ಸಿಂಧುಃ
ಸಿಂಧು-ಗೋವಿಂದ
ಸಿಂಧು-ರದ
ಸಿಂಧು-ರ-ರಾಜ-ಗಭೀರಧೀಃ
ಸಿಂಧೆಯ
ಸಿಂಧೆಯ-ನಾಯಕ
ಸಿಂಧೆಯ-ನಾಯ-ಕನ
ಸಿಂಧೆಯ-ನಾಯ-ಕ-ನಿಗೆ
ಸಿಂಧೇಶ್ವರ
ಸಿಂಫ್ತ್
ಸಿಂಫ್ತ್ಗ-ಳಿಗೆ
ಸಿಂಫ್ತ್ದಾರ-ರಿದ್ದು
ಸಿಂಹ
ಸಿಂಹ-ಗಳ
ಸಿಂಹ-ದಂತೆ
ಸಿಂಹ-ನೆನಿಸಿದ್ದ-ವನೂ
ಸಿಂಹ-ಪೋತ
ಸಿಂಹ-ಪೋತ-ಕಲಿ-ನೊಳಂಬಾದಿ-ರಾಜನು
ಸಿಂಹ-ಪೋತನು
ಸಿಂಹಪ್ರಾಯ-ನೆಂದು
ಸಿಂಹ-ಳ-ದೇವಿಯ
ಸಿಂಹ-ಳ-ದೇವಿ-ಯರ
ಸಿಂಹಾಸಕ್ಕೆ
ಸಿಂಹಾ-ಸನ
ಸಿಂಹಾ-ಸನಕೆ
ಸಿಂಹಾ-ಸನಕ್ಕಾಗಿ
ಸಿಂಹಾ-ಸನಕ್ಕಿದ್ದ
ಸಿಂಹಾ-ಸನಕ್ಕೆ
ಸಿಂಹಾಸ-ನದ
ಸಿಂಹಾ-ಸನ-ದಲ್ಲಿ
ಸಿಂಹಾ-ಸನ-ದಿಂದ
ಸಿಂಹಾಸ-ನದು
ಸಿಂಹಾ-ಸನ-ವನ್ನು
ಸಿಂಹಾ-ಸನ-ವನ್ನೇರಿ-ದನು
ಸಿಂಹಾ-ಸನ-ವನ್ನೇ-ರಿದ್ದು
ಸಿಂಹಾ-ಸನವು
ಸಿಂಹಾ-ಸನಾಧಿಪ-ತಿ-ಗಳಲ್ಲಿ
ಸಿಂಹಾ-ಸನಾಧೀಶ್ವರ
ಸಿಂಹಾ-ಸನಾಧೀಶ್ವರ-ರಾಗಿ
ಸಿಂಹಾಸ-ನಾರೂಢ-ರಾಗಿ
ಸಿಂಹಾ-ಸನಾರೋಹಣ
ಸಿಂಹಾ-ಸನಾಸೀನ-ನಾಗಿ
ಸಿಂಹಾ-ಸನೋಚಿತ
ಸಿಂಹಾಸಾನೋರಹ-ಣಕ್ಕೆ
ಸಿಕ್ಕಿತು
ಸಿಕ್ಕಿವೆ
ಸಿಗುತ್ತವೆ
ಸಿಗುತ್ತಿದ್ದ
ಸಿಗುವ
ಸಿಗು-ವು-ದಿಲ್ಲ
ಸಿಡಿದೇಳಲು
ಸಿಡಿಲಂತೆ
ಸಿಡಿಲು-ಕಲ್ಲು
ಸಿಡುಬುರೋಗಕ್ಕೆ
ಸಿತ-ಕರ-ಗಂಡ
ಸಿತ-ಗರ
ಸಿದ್ದೇಶ್ವರ
ಸಿದ್ಧ-ನಾಥ-ದೇವ-ರಿಗೆ
ಸಿದ್ಧ-ನಾ-ದನು
ಸಿದ್ಧ-ಪಡಿ-ಸ-ಲಾಗಿದೆ
ಸಿದ್ಧ-ಪಡಿಸಿ
ಸಿದ್ಧ-ಪಡಿಸಿದ್ದಕ್ಕಾಗಿ
ಸಿದ್ಧ-ಪಡಿ-ಸುವ-ವ-ನೆಂದೂ
ಸಿದ್ಧ-ಪುರುಷ-ರು-ಗಳು
ಸಿದ್ಧಪ್ಪಾಜೀ
ಸಿದ್ಧಯ್ಯ-ಗವುಡ
ಸಿದ್ಧರ್ದಪ್ಪಣ್ಣ
ಸಿದ್ಧ-ಲ-ದೇವಿಯ
ಸಿದ್ಧ-ವಾಗಿ
ಸಿದ್ಧಾಂತ
ಸಿದ್ಧಾಂತ-ದೇವರ
ಸಿದ್ಧಾನ್ತ-ದೇವರ
ಸಿದ್ಧಾಯ
ಸಿದ್ಧಾಯ-ದಿಂದ
ಸಿದ್ಧಾಯ-ವನ್ನು
ಸಿದ್ಧಿ-ಗಳನ್ನು
ಸಿದ್ಧಿ-ಸಿರಲಾ-ರದು
ಸಿನ್ನಿ-ವರ
ಸಿಬ್ಬಾಲ್
ಸಿರಕು-ಬಳ್ಳಿ
ಸಿರಗುಪ್ಪಿ
ಸಿರಿ
ಸಿರಿಗ
ಸಿರಿಯ
ಸಿರಿ-ಯ-ಕಲ-ಸತ್ತು-ಪಾಡಿ-ಯಾದ
ಸಿರಿ-ಯಣ್ಣ-ನನ್ನು
ಸಿರಿ-ರಂಗ-ನಾಯಕ
ಸಿರಿ-ರಂಗ-ನಾಯ-ಕನ
ಸಿರಿ-ರಂಗ-ಪುರದ
ಸಿರೋಂಬುಜಮಂ
ಸಿವ-ದೇವಂ
ಸಿವನೆ-ನಾಯ-ಕನ
ಸಿವ-ನೆಯ-ನಾಯಕ
ಸಿವ-ಮಯ್ಯ-ಗವುಡ
ಸಿವ-ರಮ್ಯ-ಗೇಹ-ವನ್ನು-ಶಿವಾಲಯ
ಸಿವಾಲಯ-ಗಳನ್ನು
ಸಿವೋಜಿ-ನಾಯ-ಕನು
ಸೀಗೆ
ಸೀಗೆ-ಯ-ನಾಡು
ಸೀತಾಂಬಿಕಾ
ಸೀತಾ-ಪುರ
ಸೀತಾ-ಪುರ-ವೆಂಬ
ಸೀತಾಯಂಮ-ನ-ವರ
ಸೀತಾಯಂಮ-ವನರ
ಸೀತಾ-ರಾಮ-ಜಾಗಿರ್ದಾರ್
ಸೀನಣ್ಣನು
ಸೀಪನ-ಮರಡಿ
ಸೀಮಾಂತರ-ದಲ್ಲಿ
ಸೀಮಾಂತ-ವರ್ತಿನ
ಸೀಮಾರೇಖೆ-ಗಳೂ
ಸೀಮಾ-ಸಂಬಂಧಿ
ಸೀಮಾ-ಸಹಿತ-ವಾಗಿ
ಸೀಮಿತ-ವಾಗಿ
ಸೀಮಿತ-ವಾಗಿದ್ದ-ರೆಂದು
ಸೀಮಿತ-ವಾಗಿವೆ
ಸೀಮೆ
ಸೀಮೆ-ಗಳನ್ನು
ಸೀಮೆ-ಗಳು
ಸೀಮೆಗೆ
ಸೀಮೆ-ನಾಡು
ಸೀಮೆ-ಮೈಸೂರು
ಸೀಮೆಯ
ಸೀಮೆ-ಯನ್ನು
ಸೀಮೆ-ಯನ್ನು-ಇಂದಿನ
ಸೀಮೆ-ಯಲ್ಲಿ
ಸೀಮೆ-ಯಲ್ಲಿದ್ದ
ಸೀಮೆ-ಯ-ವನಿರ-ಬಹು-ದೆಂದು
ಸೀಮೆ-ಯಾಗಿ
ಸೀಮೆ-ಯಾ-ಗಿತ್ತು
ಸೀಮೆ-ಯಾಗಿದ್ದ
ಸೀಮೆ-ಯಾಗಿದ್ದವು
ಸೀಮೆಯಿಂ
ಸೀಮೆಯು
ಸೀಮೆಯೇ
ಸೀಮೆ-ಯೊಳಗಣ
ಸೀಮೆ-ಯೊಳಗಿನ
ಸೀಮೆ-ಯೊಳಗೆ
ಸೀಮೆಸ್ಥಳ
ಸೀಯ-ಕನು
ಸೀರೆ
ಸೀರೇ-ಹಳ್ಳಿ
ಸೀರ್ಯದ
ಸೀಳನೆರೆ
ಸೀಳಿದ-ನೆಂದು
ಸು
ಸುಂಕ
ಸುಂಕಕ್ಕೆ
ಸುಂಕ-ಗಳ
ಸುಂಕ-ಗಳನ್ನು
ಸುಂಕ-ಗ-ಳನ್ನೂ
ಸುಂಕ-ಗಳಲ್ಲಿ
ಸುಂಕ-ತೆ-ರಿಗೆ-ಗಳನ್ನು
ಸುಂಕ-ತೆ-ರಿಗೆ-ಯನ್ನು
ಸುಂಕದ
ಸುಂಕ-ದ-ವರು
ಸುಂಕ-ದ-ವರೂ
ಸುಂಕ-ದ-ಹೆಗ್ಗಡೆ
ಸುಂಕ-ದ-ಹೆಗ್ಗಡೆ-ಗಳ
ಸುಂಕ-ದ-ಹೆಗ್ಗಡೆ-ಗಳು
ಸುಂಕ-ದೊಳಗೆ
ಸುಂಕ-ವನ್ನು
ಸುಂಕ-ವನ್ನೂ
ಸುಂಕ-ವಿತ್ತೆಂದು
ಸುಂಕ-ವೆಂಬ
ಸುಂಕಾ-ತೊಂಡನೂರಿನ
ಸುಂಕಾ-ತೊಂಡನೂರಿನಲ್ಲಿ
ಸುಂಕಾ-ತೊಂಡ-ನೂರು
ಸುಂದರ
ಸುಂದರ-ಪಾಂಡ್ಯನ
ಸುಂದರ-ಪಾಂಡ್ಯ-ನಿಗೆ
ಸುಂದರ-ವಾಗಿದೆ
ಸುಂದರ-ವಾದ
ಸುಂದರವೂ
ಸುಂದರೀ
ಸುಕ
ಸುಕದಿಂ
ಸುಕ-ದೋರಾ-ಇಂದಿನ
ಸುಕ-ನಾಸಿ
ಸುಕವಿ-ಜನ
ಸುಕ-ಸಂಕಥಾ-ವಿನೋದದಿಂ
ಸುಕ್ಕು-ಧರೆ
ಸುಖದಿಂ
ಸುಖದಿ-ನ-ರಸು-ಗೆಯ್ಯುತ್ತಮಿರೆ
ಸುಖ-ರಾಜ್ಯಂಗೈಯುತ್ತಿದ್ದರು
ಸುಖ-ರಾಜ್ಯ-ಗೆಯುತ್ತಿದ್ದ-ರೆಂದು
ಸುಖಸಂಕಥಾ
ಸುಖಸಂಕಥಾ-ವಿನೋದದಿಂ
ಸುಖೀಭವ
ಸುಗ-ಧರೆ
ಸುಗ್ಗಗವುಂಡನ
ಸುಗ್ಗ-ಗೌಂಡ
ಸುಗ್ಗಲ-ದೇವಿ
ಸುಚಿಸುಧನ್ವ
ಸುಜ್ಜ-ಲೂರಿನ
ಸುಜ್ಜ-ಲೂರು
ಸುಣ್ಣ-ದಲ್ಲಿ
ಸುಣ್ಣಬಣ್ಣ-ವನ್ನು
ಸುತ
ಸುತಂ
ಸುತರು
ಸುತೆ
ಸುತ್ತಣ
ಸುತ್ತ-ಮುತ್ತ
ಸುತ್ತ-ಮುತ್ತಲ
ಸುತ್ತ-ಮುತ್ತ-ಲಿನ
ಸುತ್ತ-ಮುತ್ತಲೂ
ಸುತ್ತ-ಲಿನ
ಸುತ್ತಲೂ
ಸುತ್ತಾ-ಲಯ-ವನ್ನು
ಸುತ್ತಿನ
ಸುತ್ತು-ವರೆ-ದನು
ಸುತ್ತೂರು
ಸುದ
ಸುದತ್ತಾಚಾರ್ಯ-ನೆಂದೇ
ಸುದತ್ತಾ-ಚಾರ್ಯರ
ಸುಧರ್ಮ
ಸುಧರ್ಮ-ಪುರ-ವಾಗಿ
ಸುಧಾಂಶುರಿವ
ಸುಧಾ-ಕರಂ
ಸುಧಾಕ-ರರುಂ
ಸುಧಾನಿಧೇಃ
ಸುಧಾರಿತ
ಸುನಾರ್ಖಾನೆ
ಸುನಾರ್ಖಾನೆ-ಯಒಡವೆ
ಸುಪತ್ರ
ಸುಪುತ್ರ-ನಪ್ಪ
ಸುಪ್ರತಿಷ್ಠೆಯಂ
ಸುಪ್ರತೀಕ
ಸುಪ್ರತೀಕ-ಗಜದ
ಸುಪ್ರ-ಸಿದ್ಧ
ಸುಪ್ರ-ಸಿದ್ಧ-ನಾದ
ಸುಬ್ಬಯ್ಯನ
ಸುಬ್ಬ-ರಾಯ
ಸುಬ್ಬ-ರಾಯನ
ಸುಬ್ಬ-ರಾಯ-ನ-ಕೊಪ್ಪಲು
ಸುಬ್ರಹ್ಮಣ್ಯ
ಸುಭಟರ
ಸುಭಟ-ರನ್ನು
ಸುಭಟ-ರಾದಿತ್ಯನುಂ
ಸುಮಾ-ರಿಗೆ
ಸುಮಾರಿ-ನಲ್ಲಿ
ಸುಮಾರು
ಸುಮಿತ್ರೆಯ-ರಂತೆ
ಸುಮ್ಮನೆ
ಸುರಕ್ಷಿತ-ವಾಗಿ
ಸುರಗಿ
ಸುರಗಿಯ
ಸುರತರುಸ್ಪರ್ಧಾಳುವಿಶ್ರಾ-ಣನಃ
ಸುರತ್ರಾಣ
ಸುರ-ಪುರ
ಸುರ-ರಾಜ-ಪೂಜ್ಯ
ಸುರ-ಲೋಕ
ಸುರ-ಲೋಕಂ
ಸುರ-ಲೋಕಪ್ರಾಪ್ತ
ಸುರ-ಲೋಕಪ್ರಾಪ್ತ-ನಾಗುತ್ತಾನೆ
ಸುರ-ಲೋಕಪ್ರಾಪ್ತ-ನಾ-ದನು
ಸುರ-ಲೋಕಪ್ರಾಪ್ತ-ನಾದ-ನೆಂದು
ಸುರಿಗೆ
ಸುರಿಗೆ-ಕಾರ
ಸುರಿಗೆ-ನಾಗಯ್ಯನ
ಸುರಿಗೆಯ
ಸುರಿಗೆ-ಯಿಂದ
ಸುರಿಗೆ-ವಿಡಿವ
ಸುರಿ-ತಾಣ-ಭೂಪರಂ
ಸುರಿದ
ಸುರುಚಿರ
ಸುರೇಂದ್ರ-ತೀರ್ಥ
ಸುಲಭ-ವಾಗಿ
ಸುಲ್ತಾನನ
ಸುಲ್ತಾನನು
ಸುಲ್ತಾನ್
ಸುವರ್ಣ
ಸುವರ್ಣ-ದಾನ-ಶೂರ
ಸುವ್ಯವಸ್ಥೆಗೆ
ಸುಹೃತ್ವ
ಸೂಕ್ತವಲ್ಲ-ವೆಂದು
ಸೂಕ್ತ-ವಾಗಿದೆ
ಸೂಕ್ತ-ವಾಗುತ್ತದೆ
ಸೂಕ್ತಿಸುಧಾರ್ಣವದ
ಸೂಕ್ರ-ವಾಗಿದೆ
ಸೂಕ್ಷ್ಮ-ವಾಗಿ
ಸೂಗುರು
ಸೂಗೂರಿನ
ಸೂಚನೆ
ಸೂಚನೆ-ಗಳಿವೆ
ಸೂಚನೆ-ಗಾಗಿ
ಸೂಚಿ
ಸೂಚಿ-ಗ-ಳಾದರೆ
ಸೂಚಿತ
ಸೂಚಿ-ತ-ವಾಗಿ-ರುವ
ಸೂಚಿ-ಸ-ಬಹುದು
ಸೂಚಿ-ಸಲು
ಸೂಚಿ-ಸುತ್ತದ
ಸೂಚಿ-ಸುತ್ತದೆ
ಸೂಚಿ-ಸುತ್ತವೆ
ಸೂಚಿ-ಸುತ್ತವೆಂದು
ಸೂಚಿ-ಸುತ್ತವೆ-ಯೆಂದು
ಸೂಚಿ-ಸುತ್ತಿದೆ
ಸೂಚಿ-ಸುತ್ತಿದ್ದು
ಸೂಚಿ-ಸುವ
ಸೂತ
ಸೂತ-ಕುಲದ
ಸೂತ್ರ-ಗಳನ್ನು
ಸೂತ್ರಗುತ್ತಗೆ-ಯಾಗಿ
ಸೂತ್ರದ
ಸೂತ್ರ-ವನ್ನು
ಸೂದ್ರಕ
ಸೂರನ-ಹಳ್ಳಿಯ
ಸೂರನ-ಹಳ್ಳಿ-ಯನ್ನು-ಮೆಚ್ಚುಗೆ-ಯಾಗಿ
ಸೂರನ-ಹಳ್ಳಿಯು
ಸೂರಸ್ತಗ-ಣದ
ಸೂರೆ
ಸೂರೆ-ಗೊಂಡು
ಸೂರೆ-ಮಾಡಿದನು
ಸೂರೆ-ಮಾಡಿದ-ರೆಂದು
ಸೂರ್ಯ-ನಾಥ
ಸೂರ್ಯ-ನಾಥ-ಕಾ-ಮತ್
ಸೂರ್ಯಪ್ರತಿಷ್ಠೆ-ಯನ್ನು
ಸೂಳೆ
ಸೆಕೆಯು
ಸೆಜ್ಜೆೆ-ಮ-ನೆ-ಯಲ್ಲಿ
ಸೆಟ್ಟಿ
ಸೆಟ್ಟಿ-ಗಳು
ಸೆಟ್ಟಿ-ಗವುಡನ
ಸೆಟ್ಟಿ-ಗೆರೆ
ಸೆಟ್ಟಿ-ಗೆರೆಯು
ಸೆಟ್ಟಿ-ತಿಗೂ
ಸೆಟ್ಟಿ-ಪುರ
ಸೆಟ್ಟಿಯ
ಸೆಟ್ಟಿ-ಯನ್ನು
ಸೆಟ್ಟಿ-ಯರು
ಸೆಟ್ಟಿಯು
ಸೆಟ್ಟಿ-ವಟ್ಟ-ವನ್ನು
ಸೆಟ್ಟಿ-ಹಳ್ಳಿ
ಸೆಟ್ಟಿ-ಹಳ್ಳಿಯ
ಸೆಣಸಿ
ಸೆಣಸೆ
ಸೆಪ್ಟೆಂಬರ್
ಸೆರಗು-ವಾರ್ದಪೊರಿನ್ನಿರ-ನೆಂದು
ಸೆರೆ-ಯಲ್ಲಿಟ್ಟನು
ಸೆರೆ-ಯಲ್ಲಿಟ್ಟಿದ್ದನು
ಸೆರೆಯಲ್ಲಿಟ್ಟು
ಸೆರೆ-ಯಲ್ಲಿ-ಡಿ-ಸಿದ-ನೆಂದು
ಸೆರೆ-ಯಾಗಿ-ರ-ಬೇಕೆಂದು
ಸೆರೆಹಿಡಿದ
ಸೆರೆ-ಹಿಡಿದು
ಸೆಳೆ-ದಿ-ರುವ
ಸೆಳೆದು-ಕೊಂಡ-ನೆಂದು
ಸೇಂದ್ರಕ
ಸೇಂದ್ರಕ-ವಂಶದ
ಸೇಡು
ಸೇತು-ಬಂಧ-ಮಾಡಿ
ಸೇತು-ವರಂ
ಸೇತು-ವಿನ
ಸೇತುವೆ
ಸೇತುವೆ-ಗಳನ್ನು
ಸೇತುವೆಯ
ಸೇತುವೆ-ಯನ್ನು
ಸೇತುವೆ-ಯೊಂದನ್ನು
ಸೇನ
ಸೇನ-ಬೋಗ
ಸೇನ-ಬೋವ
ಸೇನ-ಬೋವ-ಅ-ನಿದ್ದನು
ಸೇನ-ಬೋವ-ನನ್ನೇ
ಸೇನ-ಬೋವ-ನಾಗಿ-ರ-ಬಹುದು
ಸೇನ-ಬೋವ-ನಿ-ರುತ್ತಿದ್ದನು
ಸೇನ-ಬೋವನು
ಸೇನ-ಬೋವ-ನೊಳಗಾದ
ಸೇನ-ಬೋವರ
ಸೇನ-ಬೋವ-ರಂತಹ
ಸೇನ-ಬೋವ-ರನ್ನು
ಸೇನ-ಬೋವ-ರ-ವರೆ-ಗಿನ
ಸೇನ-ಬೋವ-ರಿದ್ದ-ರೆಂದು
ಸೇನ-ಬೋವರು
ಸೇನ-ಬೋವರೇ
ಸೇನಯ
ಸೇನ-ವಾರ
ಸೇನಾ
ಸೇನಾ-ಠಾಣ್ಯವಿದ್ದ
ಸೇನಾ-ತುಕಡಿ-ಯನ್ನು
ಸೇನಾ-ತುಕಡಿ-ರೆಜಿಮೆಂಟ್
ಸೇನಾ-ದಳದ
ಸೇನಾ-ಧಿ-ಕಾರಿ-ಗಳ
ಸೇನಾ-ಧಿ-ಕಾರಿ-ಗಳೂ
ಸೇನಾ-ಧಿ-ಕಾರಿ-ಯಾಗಿದ್ದ-ನೆಂದು
ಸೇನಾ-ಧಿ-ಕಾರಿ-ಯೆಂದೇ
ಸೇನಾ-ಧಿ-ಪತಿ
ಸೇನಾ-ಧಿ-ಪತಿ-ಆಗಿದ್ದನು
ಸೇನಾ-ಧಿ-ಪತಿ-ಗಳ
ಸೇನಾ-ಧಿ-ಪತಿ-ಗಳನ್ನು
ಸೇನಾ-ಧಿ-ಪತಿ-ಗಳಲ್ಲಿ
ಸೇನಾ-ಧಿ-ಪತಿ-ಗಳಾಗಿದ್ದರು
ಸೇನಾ-ಧಿ-ಪತಿ-ಗಳಿರುತ್ತಿದ್ದ-ರೆಂದು
ಸೇನಾ-ಧಿ-ಪತಿ-ಗಳು
ಸೇನಾ-ಧಿ-ಪತಿಯ
ಸೇನಾ-ಧಿ-ಪತಿ-ಯಾಗಿದ್ದನು
ಸೇನಾ-ಧಿ-ಪತಿಯೂ
ಸೇನಾ-ಧಿ-ಪತಿ-ಯೆಂದು
ಸೇನಾ-ಧಿ-ಪತಿಯೋ
ಸೇನಾ-ನ-ನಾಯ-ಕರು
ಸೇನಾ-ನಾಥ
ಸೇನಾ-ನಾಯಕ
ಸೇನಾ-ನಾಯ-ಕ-ತನ
ಸೇನಾ-ನಾಯ-ಕ-ನಪ್ಪ
ಸೇನಾ-ನಾಯ-ಕ-ನಾಗಿ
ಸೇನಾ-ನಾಯ-ಕ-ನಾಗಿದ್ದ-ನೆಂದು
ಸೇನಾ-ನಾಯ-ಕ-ನಾದ
ಸೇನಾ-ನಾಯ-ಕ-ನೆಂದು
ಸೇನಾ-ನಾಯ-ಕ-ನೆನಿಸಿ-ದ-ಅನು
ಸೇನಾ-ನಾಯ-ಕರ
ಸೇನಾ-ನಾಯ-ಕ-ರಪ್ಪ
ಸೇನಾ-ನಾಯ-ಕ-ರಾಗಿದ್ದರು
ಸೇನಾ-ನಾಯ-ಕರು
ಸೇನಾ-ನಾಯ-ಕ-ರು-ಬಲ-ಗೈಯ
ಸೇನಾ-ನಾಯ-ಕ-ರೆಂದೇ
ಸೇನಾನಿ
ಸೇನಾ-ನಿ-ಗಳಲ್ಲಿ
ಸೇನಾ-ನಿ-ಗಳಾಗಿದ್ದ
ಸೇನಾ-ನಿ-ಯಾಗಿದ್ದ
ಸೇನಾ-ನೆಲೆ-ಯನ್ನು
ಸೇನಾ-ಪಡೆ
ಸೇನಾ-ಪಡೆಗೆ
ಸೇನಾ-ಪಡೆಯ
ಸೇನಾ-ಪಡೆ-ಯನ್ನು
ಸೇನಾ-ಪಡೆ-ಯೆಂದೂ
ಸೇನಾ-ಪತಿ
ಸೇನಾ-ಪತಿ-ಗಳಾಗಲೀ
ಸೇನಾ-ಪತಿ-ಗ-ಳಿಗೆ
ಸೇನಾ-ಪತಿ-ಗಳು-ಸೇನಾ-ಧಿ-ಪತಿ-ಗಳು-ಚಮೂಪರು
ಸೇನಾ-ಪತಿ-ಯಾಗಿ-ರ-ಬಹುದು
ಸೇನಾ-ಪತಿಯು
ಸೇನಾ-ಬಲ
ಸೇನಾ-ಬಲಕ್ಕೆ
ಸೇನಾ-ಬಲದ
ಸೇನಾ-ವೀರ-ರಾಗಿದ್ದರು
ಸೇನಾ-ವೀ-ರರು
ಸೇನು-ಬೋವ
ಸೇನು-ಬೋವನು
ಸೇನು-ಬೋವ-ರಿದ್ದ-ರೆಂಬುದು
ಸೇನು-ಬೋವರೂ
ಸೇನೆ
ಸೇನೆ-ಗ-ಳಿಗೆ
ಸೇನೆಗೆ
ಸೇನೆಯ
ಸೇನೆ-ಯನ್ನು
ಸೇನೆ-ಯಲ್ಲಿ
ಸೇನೆಯಿಂ
ಸೇನೆ-ಯಿಂದ
ಸೇನೆಯು
ಸೇನೆ-ಯೊಂದಿಗೆ
ಸೇನೆ-ಯೊಡ-ಗೂಡಿ
ಸೇನೆ-ಯೊಡನೆ
ಸೇರನ-ಹಳ್ಳಿ
ಸೇರಿ
ಸೇರಿ-ಕೊಂಡು
ಸೇರಿತ್ತು
ಸೇರಿತ್ತೆಂದು
ಸೇರಿತ್ತೇ
ಸೇರಿದ
ಸೇರಿ-ದಂತೆ
ಸೇರಿ-ದ-ವ-ನಲ್ಲ-ವೆಂದೂ
ಸೇರಿ-ದ-ವ-ನಾಗಿದ್ದು
ಸೇರಿ-ದ-ವನಿರ-ಬಹುದು
ಸೇರಿ-ದ-ವನು
ಸೇರಿ-ದ-ವ-ನೆಂದು
ಸೇರಿ-ದ-ವ-ನೆಂದೂ
ಸೇರಿ-ದ-ವ-ರಾಗಿದ್ದರು
ಸೇರಿ-ದ-ವ-ರಾಗಿದ್ದಾರೆ
ಸೇರಿ-ದ-ವ-ರಾಗಿದ್ದು
ಸೇರಿ-ದ-ವ-ರಾಗಿ-ರ-ಬಹು-ದೆಂದೂ
ಸೇರಿ-ದ-ವ-ರಾಗಿ-ರುತ್ತಾರೆ
ಸೇರಿ-ದ-ವರು
ಸೇರಿ-ದ-ವರು-ಇಂದಿನ
ಸೇರಿ-ದ-ವ-ರೆಂದು
ಸೇರಿ-ದ-ವ-ರೆಂದೂ
ಸೇರಿ-ದ-ವಾಗಿದ್ದು
ಸೇರಿ-ದಾಗ
ಸೇರಿದೆ
ಸೇರಿದ್ದ
ಸೇರಿದ್ದ-ರೆಂದು
ಸೇರಿದ್ದವು
ಸೇರಿದ್ದ-ವೆಂದು
ಸೇರಿದ್ದ-ವೆಂದೂ
ಸೇರಿದ್ದಾನೆ
ಸೇರಿದ್ದು
ಸೇರಿ-ರ-ಬಹು-ದಾದ
ಸೇರಿ-ರ-ಬಹು-ದೆಂದು
ಸೇರಿ-ರುವ
ಸೇರಿ-ರುವು-ದ-ರಿಂದ
ಸೇರಿಲ್ಲ
ಸೇರಿಲ್ಲ-ವೆಂದಾಯಿತು
ಸೇರಿವೆ
ಸೇರಿ-ಸಲಾ-ಗಿತ್ತು
ಸೇರಿ-ಸ-ಲಾಗಿದೆ
ಸೇರಿ-ಸ-ಲಾಯಿತು
ಸೇರಿಸಿ
ಸೇರಿ-ಸಿ-ಕೊಂಡ-ನೆಂದು
ಸೇರಿ-ಸಿ-ಕೊಂಡರು
ಸೇರಿ-ಸಿ-ಕೊಳ್ಳುತ್ತಿ-ದುದು
ಸೇರಿ-ಸಿದ
ಸೇರಿ-ಸಿ-ದ-ರೆಂದು
ಸೇರಿ-ಸಿದೆ
ಸೇರಿ-ಸಿದ್ದಾರೆ
ಸೇರಿ-ಸಿ-ರ-ಬಹುದು
ಸೇರಿ-ಸುವಾಗ
ಸೇರಿ-ಹೋಗಿದೆ
ಸೇರಿ-ಹೋಗಿ-ರುವ
ಸೇರುತ್ತ-ದೆಂದು
ಸೇರುತ್ತವೆ
ಸೇರುತ್ತಾರೆ
ಸೇರುವ
ಸೇರ್ಪಡೆ
ಸೇರ್ಪಡೆ-ಯಾಗಿದ್ದ-ವೆಂದು
ಸೇವಂತನ-ಹಳ್ಳಿ
ಸೇವಕ-ರಾಗಿ
ಸೇವಾರ್ಥ-ವಾಗಿ
ಸೇವುಣ
ಸೇವುಣ-ದಳದ
ಸೇವುಣರ
ಸೇವುಣ-ರನ್ನು
ಸೇವುಣ-ರಾಯ-ದರ್ಪದ-ಳನ
ಸೇವುಣ-ರಿಗೆ
ಸೇವುಣರು
ಸೇವುಣ-ರೊಡನೆ
ಸೇವುಣ-ಸೇನೆ-ಯನ್ನು
ಸೇವುಣಾಧಿಪ
ಸೇವುಳ-ರಾಯ
ಸೇವೆ
ಸೇವೆ-ಎಂದು
ಸೇವೆ-ಗಾಗಿ
ಸೇವೆಗೆ
ಸೇವೆ-ಯನ್ನು
ಸೇವೆ-ಯಲ್ಲಿ
ಸೇವೆ-ಯಿಂದ
ಸೇವೆ-ಸಲ್ಲಿಸಿ
ಸೇವೆ-ಸಲ್ಲಿ-ಸುತ್ತಿದ್ದರೂ
ಸೇವ್ಯ-ನೆಂದು
ಸೈಗೊಟ್ಟ
ಸೈಗೋಲ-ಮಾತ್ಯ
ಸೈದ್ಧಾಂತಿಕ
ಸೈನಿ-ಕರ
ಸೈನಿ-ಕರಂ
ಸೈನಿ-ಕ-ರನ್ನು
ಸೈನಿ-ಕ-ರಿಗೆ
ಸೈನಿ-ಕರು
ಸೈನ್ಯ
ಸೈನ್ಯಕ್ಕೆ
ಸೈನ್ಯದ
ಸೈನ್ಯ-ದೊಡನೆ
ಸೈನ್ಯ-ವನ್ನು
ಸೈನ್ಯ-ಸಮೇತ-ನಾಗಿ
ಸೈನ್ಯಾಧಿ-ಕಾರಿ-ಗಳಾಗಿದ್ದರು
ಸೈನ್ಯಾನೀಕಮಂ
ಸೈಯದ್
ಸೊಂಡೆ-ಕೊಪ್ಪ
ಸೊಂನಾಕೋತ್ಸತಿ
ಸೊಗಯಿಪನೆನೆಸುಂ
ಸೊದರ-ಳಿಯಂದಿ-ರಾದ
ಸೊಮೆಯ-ದಂಡ-ನಾಯಕ
ಸೊಮೇಶ್ವರ-ದೇವ-ನೊಡನೆ
ಸೊರಟೂರು
ಸೊರಬ
ಸೊಲಿ-ಸಿದನು
ಸೊಸಿ-ಯಪ್ಪ
ಸೊಸಿಯಪ್ಪನ
ಸೊಸಿ-ಯಪ್ಪ-ನಾಯಕ
ಸೊಸೆವೂ-ರನ್ನು
ಸೊಸೆ-ವೂರಿನಿಂದ
ಸೋತ
ಸೋತ-ನೆಂದು
ಸೋತ-ವರ
ಸೋತು
ಸೋದರ
ಸೋದರ-ನಾಗಿ-ರ-ಬಹುದು
ಸೋದರ-ಮಾವ
ಸೋದರರ
ಸೋದರರು
ಸೋದರ-ರೆಂದು
ಸೋದರ-ಳಿಯ
ಸೋದರ-ಳಿಯಂದಿ-ರಾದ
ಸೋದರ-ಳಿಯಂದಿರು
ಸೋದರ-ಳಿಯ-ನಾದ
ಸೋದರಿ
ಸೋಧೆವೆ-ಸನಂ
ಸೋಪಾನ-ವನ್ನು
ಸೋಭಿಸೆ
ಸೋಮ
ಸೋಮಂ
ಸೋಮಂಗ-ಳಿಯಂ
ಸೋಮಣ್ಣ
ಸೋಮಣ್ಣ-ದಂಡ-ನಾಯ-ಕರು
ಸೋಮ-ದಂಡ-ನಾಥನ
ಸೋಮ-ದಂಡ-ನಾಯ-ಕನ
ಸೋಮ-ದಂಡ-ನಾಯ-ಕ-ನನ್ನು
ಸೋಮ-ದಂಡ-ನಾಯ-ಕ-ನಿಗೆ
ಸೋಮ-ದಂಡ-ನಾಯ-ಕನು
ಸೋಮನ
ಸೋಮ-ನನ್ನು
ಸೋಮ-ನ-ವ-ರೆಗೆ
ಸೋಮ-ನ-ಹಳ್ಳಿ
ಸೋಮ-ನಾಥ
ಸೋಮ-ನಾಥ-ದೇವರ
ಸೋಮ-ನಾಥ-ಪುರ
ಸೋಮ-ನಾಥ-ಪುರದ
ಸೋಮ-ನಾಥ-ಪುರ-ದಲ್ಲಿ
ಸೋಮ-ನಾಥ-ಪುರ-ವಾದ
ಸೋಮ-ನಾಥ-ಪುರ-ವೆಂಬ
ಸೋಮ-ನಿನಿ-ಸಿದನು
ಸೋಮನು
ಸೋಮ-ನೃಪನ
ಸೋಮಯ
ಸೋಮ-ಯ-ನಾಯ-ಕ-ನೊಳಗಾದ
ಸೋಮಯೆ
ಸೋಮಯ್ಯ
ಸೋಮಯ್ಯನು
ಸೋಮ-ವಂಶಾಧೀಶ್ವರ-ನೆಂದು
ಸೋಮ-ವರ್ಮನು
ಸೋಮ-ವರ್ಮ-ನೆಂಬ
ಸೋಮಿ-ದೇವ
ಸೋಮಿ-ಸೆಟ್ಟಿ
ಸೋಮಿ-ಸೆಟ್ಟಿಯು
ಸೋಮೆಯ
ಸೋಮೆಯಜ
ಸೋಮೆಯ-ದಂಡ-ನಾಯ-ಕನ
ಸೋಮೆಯ-ದಂಡ-ನಾಯ-ಕನು
ಸೋಮೆಯ-ದಂಡ-ನಾಯ-ಕ-ಮಲ್ಲಿ-ದೇವ
ಸೋಮೆಯ-ದಂಡ-ನಾಯ-ಕರ
ಸೋಮೆಯನ
ಸೋಮೆಯ-ನಾಯಕ
ಸೋಮೆಯ-ನಾಯ-ಕನ
ಸೋಮೆಯ-ನಾಯ-ಕನು
ಸೋಮೇಶ್ವರ
ಸೋಮೇಶ್ವರ-ದೇವ-ರ-ಸರು
ಸೋಮೇಶ್ವರನ
ಸೋಮೇಶ್ವರ-ನನ್ನು
ಸೋಮೇಶ್ವರ-ನಿಗೆ
ಸೋಮೇಶ್ವರನು
ಸೋಮೇಶ್ವರನೇ
ಸೋಯಂ
ಸೋಯಕ್ಕ
ಸೋಯಕ್ಕ-ನಿಗೂ
ಸೋಲನ್ನು
ಸೋಲಾಯಿತು
ಸೋಲಿಸಲಾಗದ
ಸೋಲಿಸ-ಲಾಗಿ
ಸೋಲಿ-ಸಲು
ಸೋಲಿಸಲ್ಪಟ್ಟ
ಸೋಲಿಸಿ
ಸೋಲಿಸಿ-ಕೊಂದು
ಸೋಲಿ-ಸಿದ
ಸೋಲಿಸಿ-ದಂತೆ
ಸೋಲಿಸಿ-ದನು
ಸೋಲಿಸಿ-ದ-ನೆಂದು
ಸೋಲಿಸಿ-ದರು
ಸೋಲಿಸಿ-ದಾಗ
ಸೋಲಿಸಿದ್ದಕ್ಕಾಗಿಯೂ
ಸೋಲಿಸಿದ್ದ-ನೆಂದು
ಸೋಲಿಸಿದ್ದಲ್ಲದೆ
ಸೋಲಿಸಿ-ರ-ಬಹುದು
ಸೋಲುಂಟಾ-ಯಿತೆಂದು
ಸೋಲೂರು
ಸೋವಣ
ಸೋವಣ್ಣ
ಸೋವಣ್ಣನೂ
ಸೋವಿ-ದೇವ
ಸೋವಿ-ದೇವ-ಘಟೆಯಂ
ಸೋವಿ-ದೇವ-ನಿಗೂ
ಸೋವಿ-ದೇವ-ನೊಡನೆ
ಸೋವಿ-ಸೆಟ್ಟಿಯು
ಸೋವೆಯ-ದಂಡ-ನಾಯ-ಕ-ಸೋಮೆಯ
ಸೋವೆಯ-ನಾಯಕ
ಸೋವೆಯ-ನಾಯಕಂ
ಸೋವೆಯ-ನಾಯ-ಕನ
ಸೋಸಲಿ
ಸೋಸಲಿ-ಯ-ಇಂದಿನ
ಸೋಸಲೆ
ಸೋಸ-ಲೆಯೇ
ಸೌಜನ-ಬಾಂಧವ
ಸೌಮ್ಯ-ಕೇಶವ-ದೇವಾಲಯ-ವನ್ನು
ಸೌಮ್ಯಜಾಮಾತೃ
ಸೌಮ್ಯ-ರಾಜ
ಸೌರಾಷ್ಟ್ರ-ಗಳನ್ನು
ಸೌರಾಷ್ಟ್ರ-ದಿಂದ
ಸೌವಿದಲ್ಲ-ಪದಂ
ಸೌಹಾರ್ದ-ವನ್ನು
ಸ್ಟೇಷನ್
ಸ್ತನ-ಹಾರ
ಸ್ತಳ
ಸ್ತೀರೋಮಣಿ
ಸ್ತುತಿ
ಸ್ತುತಿಗೆ
ಸ್ತುತಿ-ಯ-ನಂತರ
ಸ್ತುತಿ-ಯಿಂದ
ಸ್ತುತಿ-ಯಿಂದಲೇ
ಸ್ತುತಿ-ಯೊಂದಿಗೆ
ಸ್ತುತಿ-ಸಿದು
ಸ್ತುತಿ-ಸಿದೆ
ಸ್ತುತಿ-ಸಿದ್ದು
ಸ್ತುತಿ-ಸುತ್ತದೆ
ಸ್ತ್ರೀಯರ
ಸ್ತ್ರೀಯರು
ಸ್ತ್ರೀಸಮಾಜ
ಸ್ಥಂಬನು
ಸ್ಥಂಭನನ್ನು-ರಣಾವ-ಲೋಕ
ಸ್ಥಂಭನು
ಸ್ಥಂಭೀ-ತನನ್ನಾಗಿ
ಸ್ಥಳ
ಸ್ಥಳಕ್ಕೆ
ಸ್ಥಳ-ಗಳ
ಸ್ಥಳ-ಗಳನ್ನು
ಸ್ಥಳ-ಗ-ಳನ್ನೇ
ಸ್ಥಳ-ಗಳು
ಸ್ಥಳ-ಗಳೆಂಬ
ಸ್ಥಳದ
ಸ್ಥಳ-ದಲು
ಸ್ಥಳ-ದಲ್ಲಿ
ಸ್ಥಳ-ದಲ್ಲಿದ್ದವು
ಸ್ಥಳ-ದಿಂದ
ಸ್ಥಳ-ದೊಳಗಣ
ಸ್ಥಳ-ನಾಮ
ಸ್ಥಳ-ನಾಮ-ಗಳ
ಸ್ಥಳ-ನಾಮ-ಗಳನ್ನು
ಸ್ಥಳ-ನಾಮ-ಗಳು
ಸ್ಥಳ-ನಾಮದ
ಸ್ಥಳ-ನಾಮ-ವನ್ನು
ಸ್ಥಳ-ಪರಿ-ವೀಕ್ಷ-ಣೆಯ
ಸ್ಥಳ-ಪುರಾಣ
ಸ್ಥಳ-ಪುರಾಣ-ಗಳ
ಸ್ಥಳ-ಪುರಾಣದ
ಸ್ಥಳ-ವನ್ನು
ಸ್ಥಳ-ವಾಗಿ
ಸ್ಥಳ-ವಾ-ಗಿತ್ತು
ಸ್ಥಳ-ವಾಗಿದೆ
ಸ್ಥಳ-ವಾಗಿದ್ದು
ಸ್ಥಳ-ವಾಯಿತು
ಸ್ಥಳವು
ಸ್ಥಳ-ವು-ಇಂದಿನ
ಸ್ಥಳಾಂತ-ರದ
ಸ್ಥಳಾಂತರ-ನಾದ-ನೆಂದು
ಸ್ಥಳಾಂತರ-ವಾ-ಗಿತ್ತು
ಸ್ಥಳಾಂತರಿ-ಸ-ಲಾಗಿದೆ
ಸ್ಥಳಾಂತರಿ-ಸ-ಲಾಯಿತು
ಸ್ಥಳಾಂತರಿಸಿ-ದ-ನೆಂದು
ಸ್ಥಳೀಯ
ಸ್ಥಳೀ-ಯರ
ಸ್ಥಳೀ-ಯರು
ಸ್ಥಳೀ-ಯರೇ
ಸ್ಥಳೀ-ಯವೇ
ಸ್ಥಳೇ
ಸ್ಥಾನ
ಸ್ಥಾನ-ಗಳಿದ್ದುದು
ಸ್ಥಾನ-ಗಳು
ಸ್ಥಾನದ
ಸ್ಥಾನ-ಪಡೆ-ದಿದ್ದರು
ಸ್ಥಾನ-ಪಡೆದು
ಸ್ಥಾನ-ಪತಿ
ಸ್ಥಾನ-ಪತಿಗೆ
ಸ್ಥಾನ-ಪತಿ-ಯಾಗಿ
ಸ್ಥಾನ-ಪತಿ-ಯಾಗಿದ್ದನು
ಸ್ಥಾನ-ಪತಿ-ಯಾಗಿದ್ದ-ನೆಂದು
ಸ್ಥಾನ-ಪತಿಯೂ
ಸ್ಥಾನ-ಮಾನ
ಸ್ಥಾನ-ಮಾನ-ಗಳನ್ನು
ಸ್ಥಾನ-ಮಾನ-ವನ್ನೋ
ಸ್ಥಾನ-ವನ್ನು
ಸ್ಥಾನ-ವಾಗಿ-ರ-ಬೇಕು
ಸ್ಥಾನಿ-ಕ-ನಾಗಿದ್ದನು
ಸ್ಥಾನಿ-ಕ-ರಾಗಿದ್ದ-ರೆಂದು
ಸ್ಥಾನಿ-ಕರು
ಸ್ಥಾನೀಕ
ಸ್ಥಾಪಕ
ಸ್ಥಾಪಕ-ನೆಂದು
ಸ್ಥಾಪಕ-ರಾದ
ಸ್ಥಾಪ-ಕರು
ಸ್ಥಾಪನಾಚಾರ್ಯ
ಸ್ಥಾಪನೆ
ಸ್ಥಾಪನೆಗೆ
ಸ್ಥಾಪನೆಯ
ಸ್ಥಾಪನೆ-ಯಲ್ಲಿ
ಸ್ಥಾಪನೆ-ಯಾದ
ಸ್ಥಾಪಿ-ಸಲು
ಸ್ಥಾಪಿಸಿ
ಸ್ಥಾಪಿಸಿ-ದನು
ಸ್ಥಾಪಿಸಿ-ದ-ನೆಂದು
ಸ್ಥಾಪಿಸಿ-ದ-ರೆಂದು
ಸ್ಥಾಪ್ಯಂತೇ
ಸ್ಥಾಫನಾಚಾರ್ಯ
ಸ್ಥಾವರ
ಸ್ಥಿತಿ-ಯನ್ನು
ಸ್ಥಿತ್ಯಂತ-ರದ
ಸ್ಥಿರ
ಸ್ಥಿರಂ
ಸ್ಥಿರ-ಜೀವಿ-ಗ-ಳಾದ-ರೆಂದು
ಸ್ಥಿರ-ತಾಟಂಕ-ವತ್ಯ-ಭೂತ್
ಸ್ಥಿರ-ನಾ-ರಾಯಣಂ
ಸ್ಥಿರ-ಪಡಿಸತೊಡಗಿ-ದನು
ಸ್ಥಿರ-ವಾಗಿ
ಸ್ಥಿರ-ವಾ-ಯಿತೆಂದು
ಸ್ಥಿರ-ವೈಭವಸ್ತಸ್ಯ
ಸ್ಥೂಲ
ಸ್ಥೂಲ-ವಾಗಿ
ಸ್ಥೈರ್ಯಮಂದರಂ
ಸ್ನೇಹ
ಸ್ನೇಹ-ವನ್ನು
ಸ್ಪತಿ
ಸ್ಪಷ್ಟ
ಸ್ಪಷ್ಟತೆ
ಸ್ಪಷ್ಟ-ಪಡಿ-ಸುತ್ತದೆ
ಸ್ಪಷ್ಟ-ಪಡಿ-ಸುತ್ತ-ದೆಂದು
ಸ್ಪಷ್ಟ-ವಾಗಿ
ಸ್ಪಷ್ಟ-ವಾಗಿದೆ
ಸ್ಪಷ್ಟ-ವಾಗಿಲ್ಲ
ಸ್ಪಷ್ಟ-ವಾಗುತ್ತದೆ
ಸ್ಪಷ್ಟ-ವಾಗು-ವುದು
ಸ್ಪಷ್ಟ-ವಿಲ್ಲ
ಸ್ಫಾರಪ್ರತಾಪ
ಸ್ಮರಣ
ಸ್ಮರಣಾರ್ಥ
ಸ್ಮರಣಾರ್ಥ-ವಾಗಿ
ಸ್ಮಾರಕ
ಸ್ಮಾರ್ತ
ಸ್ಮಾರ್ತಬ್ರಾಹ್ಮಣನು
ಸ್ಯಮ್ಯಕ್ತ್ವ
ಸ್ವಂತ
ಸ್ವಕೀಯ
ಸ್ವಕೀಯ-ಕರ್ನಾಟ-ಕಕ
ಸ್ವಕೀಯೈಕಾದಶ-ಪಲ್ಲಿ
ಸ್ವಜನಂ
ಸ್ವಜನ-ಗೋತ್ರ
ಸ್ವತಂತ್ರ
ಸ್ವತಂತ್ರ-ನಾಗ-ಬೇಕೆಂದು
ಸ್ವತಂತ್ರ-ನಾಗಲು
ಸ್ವತಂತ್ರ-ನಾಗಿ
ಸ್ವತಂತ್ರ-ನಾದ
ಸ್ವತಂತ್ರ-ರಾಗಿ
ಸ್ವತಂತ್ರ-ರಾಜ-ನಂತೆ
ಸ್ವತಂತ್ರ-ವಾಗಿ
ಸ್ವತಂತ್ರ-ವಾದ
ಸ್ವತಃ
ಸ್ವಧರ್ಮ-ದಿಂದ
ಸ್ವಭಾನು
ಸ್ವಯಂ
ಸ್ವಯಂಭು
ಸ್ವಯಂಭೂ
ಸ್ವಯಂಭೂ-ನಾಥ-ನಿಗೆ
ಸ್ವರ-ವನ-ಹಳ್ಳಿ
ಸ್ವರೂ-ಪದ್ದಾ-ಗಿದ್ದು
ಸ್ವರೂ-ಪ-ವನ್ನು
ಸ್ವರ್ಗಮರ್ತ್ಯಪಾತಾಳ
ಸ್ವರ್ಗ-ಲೋಕ-ಸುಕಪ್ರಾಪ್ತ-ನೆಂದು
ಸ್ವರ್ಗಸ್ಥ-ನಾಗಿ
ಸ್ವರ್ಗಸ್ಥ-ನಾಗುತ್ತಾನೆ
ಸ್ವರ್ಗಸ್ಥ-ನಾದ-ನೆಂದು
ಸ್ವರ್ಗಸ್ಥ-ನಾದಾಗ
ಸ್ವರ್ಗಸ್ಥ-ರಾದ-ರೆಂದು
ಸ್ವರ್ಗ್ಗಕ್ಕೆ
ಸ್ವರ್ಣಕಿರೀಟ
ಸ್ವಲ್ಪ
ಸ್ವಲ್ಪ-ಕಾಲ
ಸ್ವಲ್ಪ-ಭಾಗ-ವನ್ನು
ಸ್ವಲ್ಪ-ಮಟ್ಟಿಗೆ
ಸ್ವಲ್ಪ-ಮಟ್ಟಿನ
ಸ್ವಷ್ಟ-ವಾಗಿ
ಸ್ವಸ್ತಾನೇಕ
ಸ್ವಸ್ತಿ
ಸ್ವಸ್ತಿ-ಪುರ-ವರಾಧೀಶ್ವರ
ಸ್ವಸ್ತಿ-ಳಕೈಃ
ಸ್ವಸ್ತ್ಯ-ನ-ವರತ
ಸ್ವಸ್ವಾಮಿನಂ
ಸ್ವಹಸ್ತ-ದಿಂದ
ಸ್ವಾತಂತ್ರ್ಯ
ಸ್ವಾತಂತ್ರ್ಯ-ವನ್ನು
ಸ್ವಾತಿ
ಸ್ವಾಧೀನನಯಸಂಪದಃ
ಸ್ವಾಭಾವಿಕ
ಸ್ವಾಮಿ
ಸ್ವಾಮಿ-ಕಾರ್ಯ
ಸ್ವಾಮಿಗೆ
ಸ್ವಾಮಿದ್ರೋ-ಹರ-ಗಂಣ್ಡನುಂ
ಸ್ವಾಮಿ-ಭಕ್ತಿಗೆ
ಸ್ವಾಮಿ-ಭೃತ್ಯಂ
ಸ್ವಾಮಿಯ
ಸ್ವಾಮಿ-ಯಂಗ-ಸನ್ನಾಹ
ಸ್ವಾಮಿ-ಯನ್ನು
ಸ್ವಾಮಿ-ಯರ
ಸ್ವಾಮಿ-ಯಾದ
ಸ್ವಾಮಿ-ವಂಚ-ಕರ-ಗಂಡ
ಸ್ವಾಮ್ಯವಂತರು
ಸ್ವಾರಸ್ಯ-ಕರ-ವಾಗಿವೆ
ಸ್ವಾರಾಜ-ರಾಜ-ಮಾನಶ್ರೀ
ಸ್ವೀಕರಿ-ಸದೇ
ಸ್ವೀಕರಿಸಿ
ಸ್ವೀಕರಿ-ಸಿದ್ದ-ನೆಂದು
ಸ್ವೀಕರಿ-ಸಿದ್ದ-ರೆಂದು
ಸ್ವೀಕರಿ-ಸಿ-ರ-ಬಹುದು
ಸ್ವೀಕಾರ-ಸಾರೋದಯ
ಸ್ವೀಯಪ್ರತಾಪೋದಯೌ
ಸ್ವೋರ-ನಾ-ಡಿನ
ಸ್ವೋರೆ-ನಾಡು
ಸ್ಷಷ್ಟ-ವಾಗಿ
ಸ್ಷಷ್ಟ-ವಾಗುತ್ತದೆ
ಹ
ಹಂಗಾಮು
ಹಂಚಿಕೆ
ಹಂಚಿಕೆ-ಗ-ಳಿಗೆ
ಹಂಚಿಕೆ-ಯನ್ನು
ಹಂಚಿಕೆ-ಯಾಗಿದೆ
ಹಂಚಿ-ಪುರ
ಹಂಚಿಯ
ಹಂಣೆಚೌ-ಕನ-ಹಳ್ಳಿ-ಅಣ್ಣೆಚಾ-ಕನ-ಹಳ್ಳಿ
ಹಂಣೆಚೌ-ಕನ-ಹಳ್ಳಿ-ಅಣ್ಣೆಚಾ-ಕನ-ಹಳ್ಳಿ-ಚಿಕ್ಕ-ಕಳಲೆ
ಹಂತ
ಹಂತ-ಗಳನ್ನು
ಹಂತದ
ಹಂತ-ದಲ್ಲಿ
ಹಂತ-ವಾ-ದರೆ
ಹಂತ-ಹಂತ-ವಾಗಿ
ಹಂಪನಾ
ಹಂಪನಾ-ಗ-ರಾಜಯ್ಯ-ನ-ವರು
ಹಂಪ-ರಾಜರ
ಹಂಪಾ-ಪುರ
ಹಂಪೆಯ
ಹಂಪೆ-ಯನ್ನು
ಹಂಪೆ-ಯಲ್ಲಿ
ಹಂಪೆಯಲ್ಲಿಯೇ
ಹಂಪೆಯೇ
ಹಕ
ಹಕ್ಕನ್ನು
ಹಕ್ಕಿಯೂ
ಹಕ್ಕೀ-ಮಂಚನ-ಹಳ್ಳಿ
ಹಕ್ಕು
ಹಕ್ಕು-ದಾರ-ನೆಂದು
ಹಕ್ಕುಸ್ಥಾಪಿಸಿ-ದ-ರೆಂದೂ
ಹಗವಮ-ಗೆರೆ
ಹಗವಮ-ಗೆರೆ-ಯನ್ನು
ಹಗೆ-ತನ-ವಿ-ರ-ಲಿಲ್ಲ
ಹಚ್ಚಿ-ಕೊಂಡಿರುತ್ತಿದ್ದ
ಹಚ್ಚಿದ್ದ-ನೆಂದೂ
ಹಟ್ಟಣ
ಹಟ್ಟ-ಣದ
ಹಟ್ಟಣ-ದಲ್ಲಿ
ಹಟ್ಟಣ-ವನ್ನು
ಹಟ್ಟಿ
ಹಟ್ಟಿ-ಗ-ಳನ್ನೂ
ಹಟ್ಟಿ-ಗಾಲಗಕ್ಕೆಕ
ಹಟ್ಟಿ-ಗಾಳಗ-ದಲ್ಲಿ
ಹಟ್ಟಿ-ಗಾಳೆಗ
ಹಟ್ಟಿ-ಗಾಳೆಗಕ್ಕೆ
ಹಟ್ಟಿ-ಗಾಳೆಗ-ದಲ್ಲಿ
ಹಟ್ಟಿ-ಗಾಳೆಗ-ದೊಳ್
ಹಟ್ಟಿಯ
ಹಠ-ಹರಣ
ಹಡಗಿನ
ಹಡಗು
ಹಡ-ದಕ್ಷೇತ್ರದ
ಹಡದು
ಹಡ-ಪದ
ಹಡಪ-ವಳ
ಹಡಪ-ವಳ-ರಾಗಿದ್ದರೂ
ಹಡಬಳ
ಹಡವಪಳ
ಹಡ-ವಳ
ಹಡ-ವಳದ
ಹಡ-ವಳ-ದೇವ
ಹಡ-ವಳ-ನಾಗಿದ್ದ
ಹಡ-ವಳರು
ಹಡುವಂಗಲ
ಹಡು-ವಳ
ಹಡು-ವಳದ
ಹಡು-ವಳರ
ಹಡುವ-ಳರು
ಹಡು-ವಳರೇ
ಹಡು-ವಳ-ಹಡೆ-ವಳ-ಹಡ-ಪದ
ಹಡೆದು-ಪಡೆದು
ಹಡೆಪ-ವಳ
ಹಡೆ-ವಳ
ಹಡೆ-ವಳನ
ಹಣ
ಹಣಕಾಸಿನ
ಹಣಕಾಸು
ಹಣದ
ಹಣ-ದಿಂದ
ಹಣ-ಪಡೆದು
ಹಣ-ವನ್ನು
ಹಣವು
ಹಣೆ
ಹಣ್ಣೆಯ
ಹಣ್ನೆಯ
ಹತ-ನಾಗಿ-ರುವುದು
ಹತ-ನಾ-ದನು
ಹತನಾ-ದಾಗ
ಹತಾಶ-ನಾಗಿ
ಹತಾಶೆ-ಗೊಂಡನು
ಹತೋಟಿ-ಯನ್ನು
ಹತೋಟಿ-ಯಲ್ಲಿದ್ದರು
ಹತ್ತ-ನೆಯ
ಹತ್ತನೇ
ಹತ್ತಾರು
ಹತ್ತಿಕ್ಕಲು
ಹತ್ತಿಕ್ಕಿ
ಹತ್ತಿಕ್ಕಿ-ದನು
ಹತ್ತಿಕ್ಕು-ವಂತೆ
ಹತ್ತಿರ
ಹತ್ತಿ-ರದ
ಹತ್ತಿ-ರ-ದಿಂದಲೇ
ಹತ್ತಿ-ರ-ವಾದ
ಹತ್ತಿ-ರ-ವಿ-ರುವ
ಹತ್ತು
ಹತ್ತು-ಸಲಗೆ
ಹತ್ತು-ಸಾವಿರ
ಹದಗೆಟ್ಟಿದ್ದ-ರಿಂದ
ಹದರ-ಹಳಿ-ವಿನ
ಹದಿನಾಡ
ಹದಿ-ನಾಡನ್ನು
ಹದಿನಾ-ಡ-ಸೀಮೆಯ
ಹದಿನಾ-ಡಿಗೆ
ಹದಿನಾ-ಡಿನ
ಹದಿ-ನಾಡು
ಹದಿನಾ-ರ-ನೆಯ
ಹದಿನಾರು
ಹದಿ-ನಾಲ್ಕ-ನೆಯ
ಹದಿ-ನಾಲ್ಕು-ಮಂದಿ
ಹದಿ-ನಾಲ್ಕು-ಹ-ದಿನಾ-ಡು-ನಾಡು
ಹದಿನೆಂಟು
ಹದಿನೈದು
ಹದಿ-ಮೂರು
ಹನ-ಸೋ-ಗೆಯ
ಹನುಂತನು
ಹನುಮ
ಹನುಮಂತ
ಹನುಮಂತ-ರಾಯಸ್ವಾಮಿಗೆ
ಹನುಮಂತೇಶ್ವರ
ಹನು-ಮದ್ಗರುಡ
ಹನು-ಮನ
ಹನು-ಮ-ನ-ಕಟ್ಟೆ
ಹನು-ಮನೇ
ಹನ್ನೆರಡನೇ
ಹನ್ನೆ-ರಡು
ಹನ್ನೊಂದು
ಹನ್ಮನೆನೀ
ಹಪ್ಪ-ಳಿಗೆ-ಯನ್ನು
ಹಬ್ಬ
ಹಮೀದ್
ಹಯಪ್ರತ-ತಿಯಂ
ಹಯವ-ದನ-ರಾವ್
ಹಯಾರೂಢ
ಹಯಾರೂಢ-ನಾದ-ನೆಂದು
ಹರ-ಈಶ್ವರ
ಹರಕೆ
ಹರ-ಗನ-ಹಳ್ಳಿ
ಹರ-ಡಿತ್ತೆಂದು
ಹರ-ಡಿದೆ
ಹರ-ಡಿದ್ದ
ಹರ-ಡಿದ್ದು
ಹರ-ಡಿ-ರುವ
ಹರತಿ
ಹರ-ತಿ-ಸಿರಿ-ಯಲ್ಲಿ
ಹರದ-ನ-ಹಳ್ಳಿ
ಹರದ-ನ-ಹಳ್ಳಿಯ
ಹರದ-ನ-ಹಳ್ಳಿ-ಯನ್ನು
ಹರ-ಪನ-ಹಳ್ಳಿಯ
ಹರ-ಪಾಲ
ಹರ-ಳು-ಹಳ್ಳಿ
ಹರವು
ಹರಸಿ
ಹರ-ಹಿನ
ಹರಿ
ಹರಿ-ಕಾರ್ಭಕ್ಷಿ-ಯಾಗಿದ್ದ
ಹರಿ-ಗಿಲ
ಹರಿಣ
ಹರಿ-ದಾಸ
ಹರಿ-ದಿನ
ಹರಿ-ದೇವ
ಹರಿ-ನೀಲ
ಹರಿ-ಪಾಲನ
ಹರಿ-ಪಾಳಯ್ಯ
ಹರಿ-ಭಕ್ತಿ-ಸುಧಾ-ನಿಧಿಃ
ಹರಿ-ಯಂಣ-ನೆನಿಸಿ-ದನು
ಹರಿ-ಯಣ್ಣ
ಹರಿ-ಯಣ್ಣನ
ಹರಿ-ಯಣ್ಣನು
ಹರಿ-ಯಲೆ
ಹರಿ-ಯುತ್ತದೆ
ಹರಿ-ಯುತ್ತವೆ
ಹರಿ-ಯುತ್ತಿದ್ದ
ಹರಿ-ಯುತ್ತಿದ್ದವು
ಹರಿ-ಯುತ್ತಿದ್ದು
ಹರಿ-ಯುವ
ಹರಿ-ವರ್ಮನ
ಹರಿ-ಹರ
ಹರಿ-ಹರಂ
ಹರಿ-ಹರ-ದಂಡ-ನಾಯಕ
ಹರಿ-ಹರ-ದಂಡ-ನಾಯ-ಕನ
ಹರಿ-ಹರ-ದಂಡ-ನಾಯ-ಕ-ನಿಗೆ
ಹರಿ-ಹರ-ದಂಡ-ನಾಯ-ಕನು
ಹರಿ-ಹರ-ದಂಡಾಯ-ಕನು
ಹರಿ-ಹರ-ದೇವ
ಹರಿ-ಹರ-ದೇವನು
ಹರಿ-ಹರ-ಧರ-ಣೀ-ಪಾಲಕ
ಹರಿ-ಹ-ರನ
ಹರಿ-ಹರ-ನನ್ನು
ಹರಿ-ಹರ-ನಾಗಿದ್ದು
ಹರಿ-ಹರ-ನಿಗೆ
ಹರಿ-ಹ-ರನು
ಹರಿ-ಹರ-ನೃಪ-ನ-ನುಜಾತಂ
ಹರಿ-ಹ-ರನೇ
ಹರಿ-ಹರ-ಪಟ್ಟಣ
ಹರಿ-ಹರ-ಪಟ್ಟಣ-ದಲ್ಲಿ
ಹರಿ-ಹರ-ಪುರ
ಹರಿ-ಹರ-ಪುರಕ್ಕೆ
ಹರಿ-ಹರ-ಪುರ-ಗಳು
ಹರಿ-ಹರ-ಪುರದ
ಹರಿ-ಹರ-ಪುರ-ದಲ್ಲಿದ್ದ
ಹರಿ-ಹರ-ಪುರವು
ಹರಿ-ಹರ-ಪುರ-ವೆಂಬ
ಹರಿ-ಹರಬ್ರಹ್ಮಾದಿ-ಗಳೇ
ಹರಿ-ಹರ-ಭಟ್ಟೋಪಾಧ್ಯಾಯ-ರಿಗೆ
ಹರಿ-ಹರ-ಮಹಾ-ರಾಯರ
ಹರಿ-ಹರ-ರಾಯನ
ಹರುವ-ನ-ಹಳ್ಳಿಯ
ಹರೂರು
ಹರೆದು
ಹರೋ-ಜನು
ಹರ್ಮ್ಮ್ಯ-ಕುಲಕ್ಕೆ
ಹರ್ಮ್ಮ್ಯ-ಮಾಣಿಕ್ಯ
ಹರ್ಯಣ
ಹರ್ಯಣನ
ಹರ್ಯಣ-ನನ್ನು
ಹರ್ಯಣ-ನಿಂದಾಗಿ
ಹರ್ಯಣನು
ಹರ್ಯಣಾತ್ಮಜಃ
ಹರ್ಯಣೋ
ಹಲಕೂ-ರನ್ನು
ಹಲಕೂರು
ಹಲಗೂ-ರನ್ನು
ಹಲಗೂರು
ಹಲ-ನಾಡೊಳ-ಗಳ
ಹಲರು
ಹಲ-ವ-ರನ್ನು
ಹಲ-ವಾರು
ಹಲವಿವೆ
ಹಲವು
ಹಲವು-ಮಾ-ರಾದಿ
ಹಲ-ಸನ-ಹಳ್ಳಿ
ಹಲ-ಸನ-ಹಳ್ಳಿ-ಯನ್ನು
ಹಲಸ-ಹಳ್ಳಿ
ಹಲಸಿತಾಳ-ಹಳ್ಳಿಯ
ಹಲಸಿನತಾಳ
ಹಲಸಿನ-ಹಳ್ಳಿ
ಹಲ್ಮಿಡಿ
ಹಲ್ಲೆ-ಗೆರೆ
ಹಳಿಕಾಱ
ಹಳುವು
ಹಳೆಯ
ಹಳೆಯ-ಬಿಡು
ಹಳೆಯ-ಬೀ-ಡಿಗೆ
ಹಳೆಯ-ಬೆಳ್ಗೊಳವೇ
ಹಳೇ-ಬೀ-ಡಿನ
ಹಳೇ-ಬೀಡು
ಹಳೇ-ಬೂ-ದನೂರಿನ
ಹಳೇ-ಬೂ-ದನೂರಿನಲ್ಲಿದೆ
ಹಳೇ-ಬೂದ-ನೂರು
ಹಳೇಮನೆ
ಹಳ್ಳ-ಕೆರೆ-ಇಂದಿನ
ಹಳ್ಳದ
ಹಳ್ಳದ-ಬೀ-ಡಿನಲು
ಹಳ್ಳದ-ಬೀಡು
ಹಳ್ಳ-ಬೀಡಾಗಿ-ರ-ಬಹುದು
ಹಳ್ಳ-ಬೀಡಿ-ನಲ್ಲಿ
ಹಳ್ಳ-ವೂರ
ಹಳ್ಳ-ವೂರು
ಹಳ್ಳಿ
ಹಳ್ಳಿ-ಕೆರೆ
ಹಳ್ಳಿ-ಗಳ
ಹಳ್ಳಿ-ಗಳನ್ನು
ಹಳ್ಳಿ-ಗ-ಳನ್ನೂ
ಹಳ್ಳಿ-ಗಳಾಗಿ-ರ-ಬಹುದು
ಹಳ್ಳಿ-ಗಳಾಗಿ-ರ-ಬಹು-ದೆಂದು
ಹಳ್ಳಿ-ಗಳಾಗಿವೆ
ಹಳ್ಳಿ-ಗಳಿಗೂ
ಹಳ್ಳಿ-ಗ-ಳಿಗೆ
ಹಳ್ಳಿ-ಗಳಿದ್ದು
ಹಳ್ಳಿ-ಗಳು
ಹಳ್ಳಿ-ಗಳೆಲ್ಲಾ
ಹಳ್ಳಿ-ಗಳೊಡ-ಗೂಡಿದ್ದ
ಹಳ್ಳಿಗೂ
ಹಳ್ಳಿಗೆ
ಹಳ್ಳಿಯ
ಹಳ್ಳಿ-ಯನ್ನು
ಹಳ್ಳಿ-ಯನ್ನು-ಹೊನ್ನೇನ-ಹಳ್ಳಿ
ಹಳ್ಳಿ-ಯನ್ನೇ
ಹಳ್ಳಿ-ಯ-ವರು
ಹಳ್ಳಿ-ಯಾಗಿದೆ
ಹಳ್ಳಿಯು
ಹಳ್ಳಿಯೇ
ಹಳ್ಳಿ-ಸೀಮೆ-ಯಾಗಿ
ಹಳ್ಳಿ-ಹಳ್ಳಿ-ಗಳಲ್ಲಿ
ಹಳ್ಳೆ-ಗೆರೆ
ಹವ-ಣಿಕೆ-ಯಲ್ಲಿದ್ದನು
ಹವಣಿಸಿ
ಹವಣಿಸಿ-ದನು
ಹವಣಿಸಿ-ದಾಗ
ಹವದೆಡೆಗಾ-ಗಳುಂ
ಹವಾಲಿಸಿ-ಕೊಡುತ್ತಾನೆ
ಹವೆ-ಯನ್ನು
ಹಸಿ-ಯಪ್ಪಂಗೆ
ಹಸೆಮ-ಣೆ-ಯಲ್ಲಿ
ಹಸೆ-ಯೊಳ್
ಹಸ್ತ
ಹಸ್ತ-ದಿಂದ
ಹಸ್ತಾಂತ-ರಿ-ಸ-ಲಾಯಿತು
ಹಸ್ತಾಕ್ಷರ-ಗಳ
ಹಸ್ತಾಕ್ಷರ-ವಿದೆ
ಹಸ್ತಾಕ್ಷರ-ವಿದ್ದು
ಹಸ್ತಾಕ್ಷರವಿರ-ಬಹುದು
ಹಸ್ತಾಕ್ಷರವೇ
ಹಸ್ತಿ-ಶೈಲೇಂದ್ರ-ಮಹಾತ್ಮೆ-ಯನ್ನು
ಹಾಕ-ಬಹುದು
ಹಾಕಲು
ಹಾಕ-ಸ-ಲಾಗಿದೆ
ಹಾಕಿ
ಹಾಕಿ-ಕೊಟ್ಟ
ಹಾಕಿ-ಕೊಟ್ಟಂತೆ
ಹಾಕಿ-ಕೊಟ್ಟನು
ಹಾಕಿ-ಕೊಟ್ಟ-ನೆಂದು
ಹಾಕಿ-ಕೊಟ್ಟ-ನೆಂದೂ
ಹಾಕಿ-ಕೊಟ್ಟ-ರೆಂದು
ಹಾಕಿ-ಕೊಟ್ಟ-ಳೆಂದು
ಹಾಕಿ-ಕೊಟ್ಟಿದ್ದ-ನೆಂದು
ಹಾಕಿ-ಕೊಟ್ಟಿರ-ಬಹು-ದೆಂದು
ಹಾಕಿ-ಕೊಟ್ಟಿ-ರುವ
ಹಾಕಿ-ಕೊಟ್ಟು
ಹಾಕಿ-ಕೊಡ-ಲಾಗುತ್ತಿತ್ತು
ಹಾಕಿ-ಕೊಡುತ್ತಾನೆ
ಹಾಕಿ-ಕೊಡುತ್ತಾರೆ
ಹಾಕಿ-ದನು
ಹಾಕಿ-ದರೂ
ಹಾಕಿ-ದಾಗ
ಹಾಕಿದೆ
ಹಾಕಿದ್ದಾರೆ
ಹಾಕಿ-ರ-ಬಹು-ದಾದ
ಹಾಕಿ-ರು-ವುದು
ಹಾಕಿ-ಸ-ಲಾಗಿದೆ
ಹಾಕಿಸಿ
ಹಾಕಿ-ಸಿ-ದನ
ಹಾಕಿ-ಸಿ-ದನು
ಹಾಕಿ-ಸಿದ್ದಾನೆ
ಹಾಕಿ-ಸಿ-ರ-ಬಹುದು
ಹಾಕಿ-ಸಿ-ರ-ಬಹು-ದೆಂದು
ಹಾಕಿ-ಸಿ-ರುವ
ಹಾಕಿ-ಸಿ-ರು-ವುದು
ಹಾಕಿ-ಸುತ್ತಾನೆ
ಹಾಕಿ-ಸುತ್ತಾರೆ
ಹಾಕಿ-ಸುವ
ಹಾಗಲ-ಹಳ್ಳಿ
ಹಾಗಲ-ಹಳ್ಳಿ-ಯನ್ನು
ಹಾಗ-ವನ್ನು
ಹಾಗಾಗಿ
ಹಾಗಿದ್ದಲ್ಲಿ
ಹಾಗು
ಹಾಗೂ
ಹಾಗೆ
ಹಾಗೆಯೇ
ಹಾಜರಿದ್ದ-ನೆಂದು
ಹಾಜ-ರಿದ್ದ-ರೆಂದು
ಹಾಜ-ರಿದ್ದು-ದರ
ಹಾಡಿ-ಮಂಡಲ
ಹಾಡಿಹೊ-ಗಳಿವೆ
ಹಾಥಿ-ದರ-ವಾಜ
ಹಾದ-ನೂರು
ಹಾದರ-ವಾಗಿಲ
ಹಾದರ-ವಾಗಿ-ಲನ್ನು
ಹಾದರ-ವಾಗಿಲು
ಹಾದಿ-ಯಲ್ಲಿರುವ
ಹಾದಿರ-ವಾಗಿ-ಲನ್ನು
ಹಾನುಂಗಲಯ್ನೂರು-ಗಳನ್ನು
ಹಾನುಂಗಲ್ಲು
ಹಾನುಂಗಲ್ಲು-ಗೊಂಡ
ಹಾನುಗಲ್ಲಿನ
ಹಾರಪ್ಪಂಗಳ
ಹಾರುವ
ಹಾರುವಳ್ಳಿ-ಯನ್ನು
ಹಾರುವ-ಹಳ್ಳಿ-ಹಾರೋ-ಹಳ್ಳಿ
ಹಾಲತಿ
ಹಾಲಾಳು
ಹಾಲಿ-ಮೊತ್ತದ
ಹಾಲಿಯ-ಮೊತ್ತದ
ಹಾಲು-ಗಂಗ-ಕೆರೆ
ಹಾಲು-ಗಂಗ-ಕೆರೆಗೆ
ಹಾಲು-ಗಂಗ-ಕೆರೆ-ಯನ್ನು
ಹಾಳ-ಹಾಳು-ಇಂದಿನ
ಹಾಳಾಗಿದ್ದ
ಹಾಳಾಯಿತು
ಹಾಳು-ಗೆಡವಿದ್ದ
ಹಾಳೆ-ಗಳ
ಹಾಳೆಯ
ಹಾಳೆ-ಹಳ್ಳಿ
ಹಾಸನ
ಹಾಸ-ನ-ಸೀಮೆಯ
ಹಾಹನ-ವಾಡಿ-ಯ-ಹನಿಯಂಬಾಡಿ
ಹಿ
ಹಿಂಡನ್ನು
ಹಿಂತೆಗೆ-ದ-ನೆಂದು
ಹಿಂದಕ್ಕೆ
ಹಿಂದಣ
ಹಿಂದಿಕ್ಕಿ
ಹಿಂದಿನ
ಹಿಂದಿನ-ವನು
ಹಿಂದಿರುಗಬೇಕಾ-ಯಿತೆಂದು
ಹಿಂದಿರುಗಿ
ಹಿಂದಿರುಗಿ-ರ-ಬಹುದು
ಹಿಂದಿರುಗಿ-ಸಿದರು
ಹಿಂದಿರುಗುತ್ತಿದ್ದ
ಹಿಂದೂ
ಹಿಂದೂ-ಪುರ
ಹಿಂದೂ-ರಾಯ
ಹಿಂದೂ-ರಾಯ-ಸುರತ್ರಾಣ
ಹಿಂದೂಸ್ಥಾನಿ
ಹಿಂದೆ
ಹಿಂದೆಯೂ
ಹಿಂದೆಯೇ
ಹಿಂದೆ-ಹಬ್ಬಿ
ಹಿಂಭಾಗ-ದಲ್ಲಿಯೇ
ಹಿಜರಿ
ಹಿಜಾಜ್
ಹಿಡಿತ-ದಿಂದ
ಹಿಡಿದಂತೆ
ಹಿಡಿದನು
ಹಿಡಿದ-ನೆಂದೂ
ಹಿಡಿದರೆ
ಹಿಡಿದಾಗ
ಹಿಡಿದು
ಹಿಡಿದು-ಕೊಂಡಿದ್ದರು
ಹಿಡಿದು-ದಕ್ಕೆ
ಹಿಡಿ-ಯದೇ
ಹಿಡಿಸಿ
ಹಿತ-ಕರ-ವಾದ
ಹಿತ-ವನ್ನೇ
ಹಿತವೇ
ಹಿತೇ
ಹಿತ್ತಾಳೆ
ಹಿನ್ನೀರಿ-ನಲ್ಲಿ
ಹಿನ್ನೆಲೆ
ಹಿನ್ನೆಲೆ-ಗಳೊಡನೆ
ಹಿನ್ನೆಲೆಯ
ಹಿನ್ನೆಲೆ-ಯನ್ನು
ಹಿನ್ನೆಲೆ-ಯಲ್ಲಿ
ಹಿನ್ನೆಲೆ-ಯಲ್ಲಿಯೇ
ಹಿಮದಿಂ
ಹಿಮವದ್
ಹಿಮ್ಮೆಟ್ಟಿಸಿ
ಹಿರಣ್ಣಯ್ಯನ
ಹಿರಣ್ಯ-ಗರ್ಭ
ಹಿರಿ
ಹಿರಿ-ಕಳಲೆ
ಹಿರಿ-ಕೊಂಡ-ರಾಜ
ಹಿರಿ-ತ-ನ-ದಿಂದ
ಹಿರಿ-ದಾದ
ಹಿರಿದು
ಹಿರಿ-ಮಂಡ-ಳಿಕ-ಮಾನ
ಹಿರಿ-ಮೆ-ಗಳನ್ನು
ಹಿರಿಯ
ಹಿರಿ-ಯ-ಅಗ್ರಹಾರ
ಹಿರಿ-ಯ-ಅಡವೆ-ಹಿರೋಡೆ
ಹಿರಿ-ಯ-ಕಂನೆಯ-ನ-ಹಳ್ಳಿ
ಹಿರಿ-ಯ-ಕೆರೆಯ
ಹಿರಿ-ಯ-ಕೆರೆ-ಯ-ಕೆಳಗೆ
ಹಿರಿ-ಯ-ಚಾಮ-ರಸ
ಹಿರಿ-ಯ-ಜೀಯ
ಹಿರಿ-ಯಣ್ಣ
ಹಿರಿ-ಯ-ತಮ್ಮನ
ಹಿರಿ-ಯ-ದಂಡ-ನಾಯಕ
ಹಿರಿ-ಯ-ದಂಡ-ನಾಯಕಂ
ಹಿರಿ-ಯ-ದಂಡ-ನಾಯ-ಕ-ನಾಗಿದ್ದನು
ಹಿರಿ-ಯ-ದೇವನು
ಹಿರಿ-ಯ-ನನ್ನು
ಹಿರಿ-ಯ-ನೀರಗುಂದ
ಹಿರಿ-ಯಪ್ಪನು
ಹಿರಿ-ಯಪ್ರಧಾನ
ಹಿರಿ-ಯ-ಬಯಿಚಪ್ಪ
ಹಿರಿ-ಯ-ಬಲ್ಲಾಳ
ಹಿರಿ-ಯ-ಬೆಟ್ಟದ
ಹಿರಿ-ಯ-ಬೆಟ್ಟ-ದ-ಚಾಮ-ರಾಜನು
ಹಿರಿ-ಯ-ಭಂಡಾರಿ
ಹಿರಿ-ಯ-ಭಂಡಾರಿ-ಯಾಗಿದ್ದು-ದರ
ಹಿರಿ-ಯ-ಭೇರುಂಡನ
ಹಿರಿ-ಯ-ಮಗ
ಹಿರಿ-ಯ-ಮರಳಿ
ಹಿರಿ-ಯ-ಮರಳಿ-ಇಂದಿನ
ಹಿರಿ-ಯ-ಮರಿಯಾನೆ
ಹಿರಿ-ಯ-ಮಾಚ
ಹಿರಿ-ಯ-ರ-ಸುತ-ನ-ವನ್ನು
ಹಿರಿ-ಯ-ರಾ-ದ-ವರು
ಹಿರಿ-ಯರು
ಹಿರಿ-ಯ-ಹಡ-ವಳ
ಹಿರಿ-ಯ-ಹಡೆ-ವಳ
ಹಿರಿ-ಯ-ಹೆಗ್ಗಡೆ
ಹಿರಿ-ಯೂರು
ಹಿರಿ-ವೋಡೆ
ಹಿರಿ-ಸಾವೆ
ಹಿರೀ-ಕಳಲೆ
ಹಿರೆ-ಕೊಲೆ
ಹಿರೆಜಂತಕಲ್
ಹಿರೆ-ಮರಳಿಯ
ಹಿರೇಜಟ್ಟಿಗ
ಹಿರೇ-ಬೆಟ್ಟದ
ಹಿರೇಮಠ್ರ-ವರ
ಹಿರೇ-ಮರಳಿ
ಹಿರೋಡೆ
ಹಿರೋಡೆಗೆ
ಹಿರೋಡೆಯ
ಹಿರೋಡೆ-ಯನ್ನು
ಹಿಳ-ಪಲ್ಲಿ
ಹಿಳ್ಳ-ಹಳ್ಳಿ
ಹೀಗಾಗಿ
ಹೀಗಿದೆ
ಹೀಗೆ
ಹುಂಗೇನ-ಹಳ್ಳಿ
ಹುಂಚ
ಹುಂಚದ
ಹುಚ್ಚನ-ಹಳ್ಳಿ
ಹುಚ್ಚ-ಮಾರುಡು
ಹುಜೂರ್
ಹುಜೂರ್ನಾಯಕ
ಹುಟ್ಟಿಗೆ
ಹುಟ್ಟಿದ
ಹುಟ್ಟಿದನು
ಹುಟ್ಟಿದ-ಹಳ್ಳಿ
ಹುಟ್ಟಿ-ಬೆಳೆದು
ಹುಟ್ಟುತ್ತದೆ
ಹುಟ್ಟು-ವಳಿ
ಹುಟ್ಟು-ವಳಿ-ಗಳನ್ನು
ಹುಟ್ಟು-ವಳಿ-ಯುಳ್ಳ
ಹುಣಸೂರು
ಹುತಾತ್ಮ
ಹುತಾತ್ಮ-ನಾದದ್ದು
ಹುತಾತ್ಮ-ನಾದ-ನೆಂದು
ಹುತಾತ್ಮ-ನಾ-ದು-ದನ್ನು
ಹುತಾತ್ಮರ
ಹುದ್ದೆ
ಹುದ್ದೆ-ಗಳ
ಹುದ್ದೆ-ಗಳನ್ನು
ಹುದ್ದೆ-ಗಳಲ್ಲಿ
ಹುದ್ದೆ-ಗ-ಳಿಗೆ
ಹುದ್ದೆ-ಗಳಿದ್ದುದು
ಹುದ್ದೆ-ಗಳಿಲ್ಲ
ಹುದ್ದೆ-ಗಳು
ಹುದ್ದೆ-ಗ-ಳೆಂದು
ಹುದ್ದೆಗೂ
ಹುದ್ದೆಗೆ
ಹುದ್ದೆ-ಗೇ-ರಿದ್ದು
ಹುದ್ದೆ-ಗೇರಿ-ರು-ವುದು
ಹುದ್ದೆಯ
ಹುದ್ದೆ-ಯನ್ನು
ಹುದ್ದೆ-ಯನ್ನೂ
ಹುದ್ದೆ-ಯನ್ನೋ
ಹುದ್ದೆ-ಯಲ್ಲ
ಹುದ್ದೆ-ಯಲ್ಲಿ
ಹುದ್ದೆ-ಯಲ್ಲಿದ್ದ-ನೆಂದು
ಹುದ್ದೆ-ಯಲ್ಲಿದ್ದು
ಹುದ್ದೆ-ಯಾಗಿತ್ತೆಂದು
ಹುದ್ದೆ-ಯಾಗಿದ್ದು
ಹುದ್ದೆ-ಯಾಗಿ-ರ-ಬಹುದು
ಹುದ್ದೆ-ಯಾಗಿ-ರ-ಬಹು-ದು-ಮಹಾಪ್ರಧಾನ
ಹುದ್ದೆ-ಯಿಂದ
ಹುದ್ದೆಯು
ಹುದ್ದೆಯೂ
ಹುದ್ದೆ-ಯೆಂದು
ಹುದ್ದೆಯೇ
ಹುಬ್ಬನ-ಹಳ್ಳಿ
ಹುಬ್ಬನ-ಹಳ್ಳಿ-ಯಲ್ಲಿ
ಹುಯ್ಯ-ಲಾ-ಯಿತೆಂದು
ಹುರಗಲ-ವಾಡಿ
ಹುರುಗಲ-ವಾಡಿ
ಹುಲಗೂರ
ಹುಲಗೂರು
ಹುಲಿ
ಹುಲಿ-ಕಲ್ಲು
ಹುಲಿ-ಕೆರೆ
ಹುಲಿ-ಗಳಿಲ್ಲ
ಹುಲಿ-ನ-ವನ-ಇಂದಿನ
ಹುಲಿ-ಮುಖದ
ಹುಲಿ-ಮುಖ-ವನಿಕ್ಕಿ-ಸಿದನು
ಹುಲಿ-ಮೊಗವಾಡ-ವನ್ನು
ಹುಲಿಯ
ಹುಲಿ-ಯ-ಜಂಗುಳಿ
ಹುಲಿ-ಯ-ಜಂಗುಳಿಯ
ಹುಲಿ-ಯ-ನನ್ನು
ಹುಲಿ-ಯನ್ನು
ಹುಲಿಯು
ಹುಲಿ-ಯೊಂದು
ಹುಲಿ-ರಾಯ
ಹುಲಿ-ವನ
ಹುಲಿ-ವಾನ
ಹುಲಿ-ವಾನದ
ಹುಲಿ-ವಾನ-ವನ್ನು
ಹುಲ್ಲಂಬಳ್ಳಿಯ
ಹುಲ್ಲವಂಗ-ಲದ
ಹುಲ್ಲವಂಗಲ-ವನ್ನು
ಹುಲ್ಲ-ಹಳ್ಳಿ
ಹುಲ್ಲೇಗಾಲ
ಹುಲ್ಲೇಗಾ-ಲದ
ಹುಲ್ಲೇಗಾಲ-ವನ್ನು
ಹುಳ್ಳ
ಹುಳ್ಳಂಬಳ್ಳಿ
ಹುಳ್ಳಂಬಳ್ಳಿಯ
ಹುಳ್ಳ-ಗಾವುಂಡನ
ಹುಳ್ಳ-ಚಮೂಪ
ಹುಳ್ಳ-ಚಮೂಪನ
ಹುಳ್ಳ-ಚಮೂಪನು
ಹುಳ್ಳನ
ಹುಳ್ಳನೂ
ಹುಳ್ಳ-ಮಯ್ಯನೂ
ಹುಳ್ಳಯ್ಯ
ಹುಳ್ಳ-ರಾಜಂಗೆ
ಹುಳ್ಳೆಯ
ಹುಳ್ಳೆಯ-ನಾಯ-ಕನು
ಹುಳ್ಳೆಯ-ಹಳ್ಳಿ
ಹುಳ್ಳೆಯ-ಹಳ್ಳಿ-ಯಲ್ಲಿ
ಹುಳ್ಳೇನ-ಹಳ್ಳಿ
ಹುಳ್ಳೇನ-ಹಳ್ಳಿಗೆ
ಹುಳ್ಳೋ-ಹಳ್ಳಿ-ಹುಳ್ಳೇನ-ಹಳ್ಳಿ
ಹುಸಕೂರು
ಹುಸೇನ್
ಹುಸೈನ್
ಹುಸ್ಕೂರಿ-ನಲ್ಲಿರುವ
ಹುಸ್ಕೂರು
ಹೂಡಿದ
ಹೂಡಿದ್ದನು
ಹೂಣ-ರಾಜ-ನಾದ
ಹೂಣರು
ಹೂರದಹಳ್ಳಿ-ಯನ್ನು
ಹೂಲಿ-ಕೆರೆಯ
ಹೂಲಿಯಕೆರೆ
ಹೂಳು-ವಂತೆ
ಹೂವಿನಬಾಗೆ-ಯಲ್ಲಿ
ಹೃದಯಶಲ್ಯ
ಹೃದಯಸ್ಥಂಗಳ್
ಹೃದಯಸ್ಥ-ವಾಗಿದ್ದವು
ಹೃದುವನಕೆರೆಗೂ
ಹೃದುವನಕೆರೆ-ಯಲ್ಲಿ
ಹೆಂಡತಿ
ಹೆಂಡತಿಯ
ಹೆಂಡತಿ-ಯರು
ಹೆಂಡತಿ-ಯ-ರೆಂದು
ಹೆಂಡತಿ-ಯ-ರೊಡನೆ
ಹೆಂಡಿರನ್ನು
ಹೆಂಬೆಟ್ಟ
ಹೆಂಮ
ಹೆಂಮನ
ಹೆಂಮ-ನ-ಗೌಡನ
ಹೆಂಮನಹಳ್ಳಿ
ಹೆಂಮನಾಜಿಗುಂಮನಂ
ಹೆಂಮನೂ
ಹೆಂಮಯ್ಯಂಗ-ಳಿಯನ್
ಹೆಂಮಹೆಂಮಯ್ಯ
ಹೆಂಮೆಯ
ಹೆಂಮೇಶ್ವರದೇವರು
ಹೆಕ್ಟೇರ್
ಹೆಗ್ಗಡದೇವನಕೋಟೆ
ಹೆಗ್ಗಡಿಕೆ-ಯಲಿ
ಹೆಗ್ಗಡೆ
ಹೆಗ್ಗಡೆ-ಗಳ
ಹೆಗ್ಗಡೆ-ಗಳನ್ನು
ಹೆಗ್ಗಡೆ-ಗ-ಳಿಗೆ
ಹೆಗ್ಗಡೆ-ಗಳಿತ್ತು
ಹೆಗ್ಗಡೆ-ಗಳಿದ್ದರು
ಹೆಗ್ಗಡೆ-ಗಳಿದ್ದ-ರೆಂದು
ಹೆಗ್ಗಡೆ-ಗಳು
ಹೆಗ್ಗಡೆ-ಗಳೂ
ಹೆಗ್ಗಡೆ-ಗ-ಳೆಂದೂ
ಹೆಗ್ಗಡೆ-ಗಳೆಲ್ಲರೂ
ಹೆಗ್ಗಡೆಗೆ
ಹೆಗ್ಗಡೆತಿಲೆ-ನಾಯಕ
ಹೆಗ್ಗಡೆದೇವ
ಹೆಗ್ಗಡೆ-ಪೆರ್ಗ್ಗಡೆ-ಪೆರಾಳ್ಕೆ
ಹೆಗ್ಗಡೆ-ಮೇಲಾ-ಳಿಕೆ
ಹೆಗ್ಗಡೆಯ
ಹೆಗ್ಗಡೆ-ಯರು
ಹೆಗ್ಗಡೆ-ಯವರ
ಹೆಗ್ಗಡೆ-ಯವ-ರೆಗೆ
ಹೆಗ್ಗಡೆ-ಯಾಗಿದ್ದ
ಹೆಗ್ಗಡೆ-ಯಾಗಿದ್ದ-ನೆಂದು
ಹೆಗ್ಗಡೆಯು
ಹೆಗ್ಗಪ್ಪ
ಹೆಗ್ಗಪ್ಪ-ಗಳು
ಹೆಗ್ಗೆಡಯೇ
ಹೆಚ್ಚಾಗಿ
ಹೆಚ್ಚಾಗಿದ್ದು-ಕೊಂಡು
ಹೆಚ್ಚಾಗಿ-ರುವು-ದ-ರಿಂದ
ಹೆಚ್ಚಾಯಿತು
ಹೆಚ್ಚಿನ
ಹೆಚ್ಚಿನ-ದಾಗಿತ್ತು
ಹೆಚ್ಚಿನ-ವರು
ಹೆಚ್ಚು
ಹೆಚ್ಚು-ಕಡಿಮೆ
ಹೆಚ್ಚು-ವರಿ
ಹೆಚ್ಚು-ವರಿ-ಯಾಗಿ
ಹೆಜ್ಜಾಜಿ
ಹೆಜ್ಜುಂಕ-ವನ್ನು
ಹೆಡ-ತಲೆಯ
ಹೆಣಗಾಡಿ
ಹೆಣ್ಣಾನೆ-ಗಳನ್ನು
ಹೆಣ್ಣು-ಮಕ್ಕಳಾದ
ಹೆಣ್ಣು-ಮಕ್ಕಳಿದ್ದರು
ಹೆಣ್ಣು-ಮಕ್ಕಳೂ
ಹೆಣ್ಣುಸೆರೆ
ಹೆತ್ತಗೋನಹಳ್ಳಿ
ಹೆದ್ದಾರಿಯ
ಹೆದ್ದೊರೆ-ಯಾದಿ-ಯಾಗಿ
ಹೆಬ್ಬಕವಾಡಿ-ಯನ್ನು
ಹೆಬ್ಬಟ್ಟದ
ಹೆಬ್ಬಟ್ಟವು
ಹೆಬ್ಬಟ್ಟು
ಹೆಬ್ಬಳ್ಳದ
ಹೆಬ್ಬಾ-ಗಿಲಿ
ಹೆಬ್ಬಾ-ಗಿಲಿ-ನಲ್ಲಿ
ಹೆಬ್ಬಾ-ಗಿಲು
ಹೆಬ್ಬಾ-ರುವ
ಹೆಬ್ಬಾ-ರುವರ
ಹೆಬ್ಬಾಳ
ಹೆಬ್ಬಾಳು
ಹೆಬ್ಬಾವು
ಹೆಬ್ಬಾವು-ಗಳಿದ್ದವು
ಹೆಬ್ಬಾವು-ಗಳೂ
ಹೆಬ್ಬಿದರ-ವಾಡಿಯ
ಹೆಬ್ಬಿದಿರವಾಡಿಯ
ಹೆಬ್ಬಿದಿರವಾಡಿ-ಯಲ್ಲಿ
ಹೆಬ್ಬಿದಿರೂರ್ವಾಡಿ-ಯಲಿ
ಹೆಬ್ಬಿದಿರೂರ್ವಾಡಿ-ಯಲ್ಲಿ-ಇದೇ
ಹೆಬ್ಬಿದಿರೂರ್ವಾಡಿಯೇ
ಹೆಬ್ಬೆಟ್ಟುನಾಡು
ಹೆಬ್ಬೊಳಲ
ಹೆಬ್ಬೊಳಲ-ಇಂದಿನ
ಹೆಮೆಯ-ದಂಡ-ನಾಥ
ಹೆಮ್ಮಣ್ಣ
ಹೆಮ್ಮನಹಳ್ಳಿ
ಹೆಮ್ಮ-ನಿಂದ
ಹೆಮ್ಮಯ್ಯನ
ಹೆಮ್ಮಯ್ಯ-ನನ್ನು
ಹೆಮ್ಮಯ್ಯ-ನೆಂದಿದೆ
ಹೆಮ್ಮರಗಾಲ
ಹೆಮ್ಮವ್ವೆ
ಹೆಮ್ಮಾಡಿಯಣ್ಣನು
ಹೆಮ್ಮೆಪಡ-ತಕ್ಕಂತಹ
ಹೆಮ್ಮೆಯನಾಯಕಂ
ಹೆಮ್ಮೆಯನಾಯಕ-ನನ್ನು
ಹೆರಾಸ್
ಹೆರುಳಹಳ್ಳಿ
ಹೆರ್ಮ್ಮಾಡಿದೇವ-ನೆಂಬುವವ-ನಿಗೆ
ಹೆಸರ
ಹೆಸರನ್ನು
ಹೆಸರನ್ನೂ
ಹೆಸರನ್ನೇ
ಹೆಸರಲಿ
ಹೆಸ-ರಲು
ಹೆಸರಾಂತ
ಹೆಸರಾ-ಗಿತ್ತು
ಹೆಸ-ರಾಗಿದ್ದನು
ಹೆಸ-ರಾಗಿದ್ದಿರ-ಬಹು-ದೆಂದು
ಹೆಸ-ರಾಗಿ-ರ-ಬಹುದು
ಹೆಸ-ರಾಯಿತು
ಹೆಸ-ರಿಂದ
ಹೆಸ-ರಿಗೆ
ಹೆಸರಿಟ್ಟನು
ಹೆಸರಿಟ್ಟ-ನೆಂದೂ
ಹೆಸರಿಟ್ಟರು
ಹೆಸರಿಟ್ಟು
ಹೆಸರಿಟ್ಟು-ಕೊಂಡು
ಹೆಸರಿಟ್ಟು-ಕೊಳ್ಳುತ್ತಿದ್ದರು
ಹೆಸರಿ-ಡಲಾ-ಗಿತ್ತು
ಹೆಸರಿ-ಡು-ವು-ದಕ್ಕೆ
ಹೆಸರಿತ್ತು
ಹೆಸರಿತ್ತೆಂದು
ಹೆಸರಿದೆ
ಹೆಸರಿದ್ದರೂ
ಹೆಸ-ರಿದ್ದು
ಹೆಸರಿನ
ಹೆಸರಿ-ನಲಿ
ಹೆಸರಿ-ನಲ್ಲಿ
ಹೆಸರಿ-ನಲ್ಲಿಯೇ
ಹೆಸರಿ-ನಲ್ಲೇ
ಹೆಸರಿ-ನಿಂದ
ಹೆಸರಿ-ರ-ಬಹುದು
ಹೆಸರಿ-ರುವ
ಹೆಸರಿಲ್ಲ
ಹೆಸರಿ-ಸ-ಲಾಗಿದೆ
ಹೆಸರಿ-ಸಿದೆ
ಹೆಸರಿ-ಸಿಲ್ಲ
ಹೆಸರಿ-ಸುತ್ತದೆ
ಹೆಸರಿ-ಸುವ
ಹೆಸರು
ಹೆಸರು-ಗಳ
ಹೆಸರು-ಗಳನ್ನು
ಹೆಸರು-ಗಳಾಗಿವೆ
ಹೆಸರು-ಗಳಿಂದ
ಹೆಸರು-ಗಳಿದ್ದು
ಹೆಸರು-ಗಳಿ-ಸಿದ್ದನು
ಹೆಸರು-ಗಳಿಸಿರ-ಬಹುದು
ಹೆಸರು-ಗಳು
ಹೆಸರುನ್ನು
ಹೆಸರು-ಬಂದಿದೆ
ಹೆಸರುಳ್ಳ
ಹೆಸರು-ವಾಸಿ-ಯಾದನು
ಹೆಸರೂ
ಹೇಗಾಯಿತೆಂಬು-ದನ್ನು
ಹೇಗಿದ್ದರೂ
ಹೇಮಗಿರಿಯ
ಹೇಮದ
ಹೇಮಾದ್ರಿಯೇ
ಹೇಮಾವತಿ
ಹೇಮಾವತಿಯ
ಹೇಮೇಶ್ವರ
ಹೇಮೇಸ್ವರ-ದೇವರ
ಹೇರಿಗೆ
ಹೇರಿನ
ಹೇರಿನಷ್ಟು
ಹೇಳಬಹದು
ಹೇಳ-ಬಹುದ
ಹೇಳ-ಬಹು-ದಾಗಿದೆ
ಹೇಳ-ಬಹುದು
ಹೇಳ-ಬಹು-ದೆಂದು
ಹೇಳಬೇಕಾಗುತ್ತದೆ
ಹೇಳ-ಲಾಗಿದೆ
ಹೇಳ-ಲಾಗುತ್ತದೆ
ಹೇಳ-ಲಾಗುತ್ತಿತ್ತು
ಹೇಳಲಾದ
ಹೇಳಲು
ಹೇಳಹುದು
ಹೇಳಿ
ಹೇಳಿ-ಕಳುಹಿಸಿ-ದ-ನಂತೆ
ಹೇಳಿ-ಕಳುಹಿಸಿ-ರ-ಬಹುದು
ಹೇಳಿಕೆ-ಗಳನ್ನು
ಹೇಳಿಕೆ-ಗಳು
ಹೇಳಿಕೆ-ಯನ್ನು
ಹೇಳಿಕೆ-ಯನ್ನೂ
ಹೇಳಿ-ಕೊಂಡಿದ್ದರೂ
ಹೇಳಿ-ಕೊಂಡಿದ್ದಾನೆ
ಹೇಳಿ-ಕೊಂಡಿದ್ದಾರೆ
ಹೇಳಿ-ಕೊಂಡಿದ್ದು
ಹೇಳಿ-ಕೊಂಡಿರ-ಬಹುದು
ಹೇಳಿ-ಕೊಂಡು
ಹೇಳಿ-ಕೊಳ್ಳುತ್ತಾ
ಹೇಳಿದ
ಹೇಳಿ-ದನು
ಹೇಳಿ-ದರೂ
ಹೇಳಿ-ದಾರೆ
ಹೇಳಿದೆ
ಹೇಳಿದ್ದ
ಹೇಳಿದ್ದರೂ
ಹೇಳಿದ್ದರೆ
ಹೇಳಿದ್ದಾನೆ
ಹೇಳಿದ್ದಾರೆ
ಹೇಳಿದ್ದಾರೆಂದು
ಹೇಳಿದ್ದಾರೋ
ಹೇಳಿದ್ದಾರ್ೆ
ಹೇಳಿದ್ದು
ಹೇಳಿ-ರುವ
ಹೇಳಿ-ರು-ವಂತೆ
ಹೇಳಿ-ರು-ವಂತೆಯೇ
ಹೇಳಿ-ರು-ವು-ದಕ್ಕೆ
ಹೇಳಿ-ರು-ವು-ದನ್ನು
ಹೇಳಿ-ರು-ವು-ದರ
ಹೇಳಿ-ರು-ವು-ದರಿಂದ
ಹೇಳಿ-ರುವು-ದಿರಂದ
ಹೇಳಿ-ರು-ವು-ದಿಲ್ಲ
ಹೇಳಿ-ರು-ವುದು
ಹೇಳಿ-ರುವುದೇ
ಹೇಳಿಲ್ಲ
ಹೇಳಿವೆ
ಹೇಳುತ್ತದೆ
ಹೇಳುತ್ತವೆ
ಹೇಳುತ್ತಾ
ಹೇಳುತ್ತಾನೆ
ಹೇಳುತ್ತಾರೆ
ಹೇಳುತ್ತಿದ್ದರು
ಹೇಳುತ್ತಿ-ರುವಂತಿದೆ
ಹೇಳುತ್ತಿ-ರುವಷ್ಟ-ರಲ್ಲಿ
ಹೇಳುವ
ಹೇಳುವಂತೆ
ಹೇಳುವಾಗ
ಹೇಳುವುದಕ್ಕಿಂತ
ಹೇಳು-ವು-ದಿಲ್ಲ
ಹೇಳುವುದು
ಹೈದರನ
ಹೈದರನು
ಹೈದರಾಬಾದಿನ
ಹೈದರಾಲಿಯು
ಹೈದರ್
ಹೈದರ್ಅಲಿ
ಹೈದರ್ಅಲಿಖಾನ್
ಹೈದರ್ಅಲಿಯ
ಹೈದರ್ಅಲಿಯು
ಹೈದರ್ನ
ಹೈದರ್ನನ್ನು
ಹೈದರ್ನೊಂದಿಗೆ
ಹೈಹಯ
ಹೈಹಯರ
ಹೊಂಕುಂದದ
ಹೊಂಗ-ನೂರು
ಹೊಂಡರಬಾಳು
ಹೊಂದಲ-ಗೆರೆ
ಹೊಂದಲ-ಗೆರೆಯ
ಹೊಂದಿ
ಹೊಂದಿ-ಕೆಯಾಗ-ದಿ-ರಲು
ಹೊಂದಿ-ಕೊಂಡ
ಹೊಂದಿ-ಕೊಂಡಂತೆ
ಹೊಂದಿ-ಕೊಂಡ-ಹಾಗೆ
ಹೊಂದಿ-ಕೊಂಡ-ಹಾಗೇ
ಹೊಂದಿ-ಕೊಂಡಿತ್ತು
ಹೊಂದಿ-ಕೊಂಡಿದೆ
ಹೊಂದಿ-ಕೊಂಡಿದ್ದರೆ
ಹೊಂದಿ-ಕೊಂಡಿ-ರುವ
ಹೊಂದಿ-ಕೊಂಡಿವೆ
ಹೊಂದಿತು
ಹೊಂದಿದ
ಹೊಂದಿ-ದನು
ಹೊಂದಿ-ದ-ನೆಂದು
ಹೊಂದಿ-ದ-ನೆಂದೂ
ಹೊಂದಿ-ದಾಗ
ಹೊಂದಿದೆ
ಹೊಂದಿದ್ದ
ಹೊಂದಿದ್ದನು
ಹೊಂದಿದ್ದ-ನೆಂದು
ಹೊಂದಿದ್ದನ್ನು
ಹೊಂದಿದ್ದ-ರಿಂದ
ಹೊಂದಿದ್ದ-ರಿಂದಲೇ
ಹೊಂದಿದ್ದರು
ಹೊಂದಿದ್ದರೂ
ಹೊಂದಿದ್ದರೆ
ಹೊಂದಿದ್ದ-ರೆಂದು
ಹೊಂದಿದ್ದ-ವರು
ಹೊಂದಿದ್ದವು
ಹೊಂದಿದ್ದು
ಹೊಂದಿ-ರ-ಬಹುದು
ಹೊಂದಿ-ರ-ಬೇಕು
ಹೊಂದಿ-ರುತ್ತಿದ್ದರೆ
ಹೊಂದಿ-ರುವ
ಹೊಂದಿ-ರುವ-ವರು
ಹೊಂದಿ-ರು-ವು-ದನ್ನು
ಹೊಂದಿವೆ
ಹೊಂನಕಹಳ್ಳಿ
ಹೊಂನಯನಹಳ್ಳಿ-ಯನ್ನು
ಹೊಂನಿ-ಸೆಟ್ಟಿಯು
ಹೊಂನೆಯ
ಹೊಂನೇಹಳ್ಳಿ
ಹೊಂಪುರ
ಹೊಇಸಳಮಂಡಲೇ
ಹೊಗರನಾಡಿಗೆ
ಹೊಗರ್ನಾಡಿನ
ಹೊಗರ್ನ್ನಾಡಿನ
ಹೊಗರ್ನ್ನಾಡು
ಹೊಗಳ-ಲಾಗಿದೆ
ಹೊಗಳಲು
ಹೊಗ-ಳಿಕೆಗೆ
ಹೊಗಳಿದೆ
ಹೊಗಳಿದ್ದಾನೆ
ಹೊಗಳಿದ್ದು
ಹೊಗಳಿ-ರುವುದ-ರಿಂದ
ಹೊಗಳುತ್ತವೆ
ಹೊಗಳುತ್ತಿತ್ತೆಂದು
ಹೊಗಳು-ಭಟ್ಟ-ರಲ್ಲ
ಹೊಗಳು-ಭಟ್ಟ-ರಾಗಿದ್ದ-ರೆಂದು
ಹೊಡುಕೇ-ಕಟ್ಟ-ಹೊಡೆ-ಘಟ್ಟ
ಹೊಡೆ-ತಕ್ಕೆ
ಹೊಡೆ-ದಟ್ಟಿ-ದನು
ಹೊಡೆದಾಟಕ್ಕೆ
ಹೊಡೆದಾಡಿ-ಕೊಂಡು
ಹೊಡೆದು
ಹೊಡೆದೋಡಿ-ಸು-ವಲ್ಲಿ
ಹೊಣಕನ-ಹಳ್ಳಿ
ಹೊಣಕನ-ಹಳ್ಳಿಯೋ
ಹೊಣೆಗಾರಿಕೆ
ಹೊಣೆ-ಯನ್ನು
ಹೊತ್ತಿ-ಗಾಗಲೇ
ಹೊತ್ತಿಗೆ
ಹೊತ್ತಿಗೇ
ಹೊತ್ತಿದ್ದ
ಹೊತ್ತು
ಹೊದಕೆ
ಹೊನಗನಹಳ್ಳಿ
ಹೊನಗನಹಳ್ಳಿಯ
ಹೊನಗನಹಳ್ಳಿಯೋ
ಹೊನಗಾನಹಳ್ಳಿ
ಹೊನಗುಂಟಾ
ಹೊನಗುಂದ
ಹೊನ್ನ-ಕಳಸ-ಗಳನ್ನು
ಹೊನ್ನ-ಕಳಸ-ವನ್ನು
ಹೊನ್ನಗೌಡನು
ಹೊನ್ನನ್ನು
ಹೊನ್ನಯನ-ಹಳ್ಳಿಯು
ಹೊನ್ನಯ್ಯ
ಹೊನ್ನಯ್ಯ-ನನ್ನು
ಹೊನ್ನಯ್ಯ-ನಿಂದಲೇ
ಹೊನ್ನಯ್ಯನು
ಹೊನ್ನಲಗೆರೆ
ಹೊನ್ನಲಗೆರೆಗೆ
ಹೊನ್ನಲಗೆರೆಯು
ಹೊನ್ನವ್ವೆ
ಹೊನ್ನಾಜಿ
ಹೊನ್ನಾ-ವರ
ಹೊನ್ನಾ-ವರದ
ಹೊನ್ನಿರ-ಬಹುದು
ಹೊನ್ನಿಸೆಟ್ಟಿ
ಹೊನ್ನು
ಹೊನ್ನು-ಡಿಗೆ-ಯೂ-ಹೊನ್ನು-ಡಿಕೆ-ಶಾ-ಸನೋಕ್ತ-ವಾಗಿದೆ
ಹೊನ್ನೂರನ್ನು
ಹೊನ್ನೂರಿನ
ಹೊನ್ನೆಯ
ಹೊನ್ನೆಯ-ನ-ಹಳ್ಳಿ
ಹೊನ್ನೇಗೌಡನು
ಹೊನ್ನೇನಹಳ್ಳಿ
ಹೊನ್ನೇನಹಳ್ಳಿ-ಯಲ್ಲಿ
ಹೊನ್ನೇನಹಳ್ಳಿಯು
ಹೊನ್ನೊಳಗೆ
ಹೊಮ್ಮ
ಹೊಯಿಕು
ಹೊಯಿದು
ಹೊಯಿ-ಸಣ
ಹೊಯಿಸಣ-ದೇಶದ
ಹೊಯಿಸಣ-ರಾಜ್ಯದ
ಹೊಯಿಸಣಾಭಿಧೇ
ಹೊಯಿಸಳ
ಹೊಯಿಸಳ-ದೇವರು
ಹೊಯಿಸಳ-ರಾಜ್ಯದ
ಹೊಯಿಸಳ-ರಾಜ್ಯ-ಲಕ್ಷ್ಮೀಪ್ರಾ-ಕಾರ
ಹೊಯಿಸಳಲೆಂಕ
ಹೊಯಿಸಳೇಶ್ವರ
ಹೊಯ್
ಹೊಯ್ದು
ಹೊಯ್ಸಣ
ಹೊಯ್ಸಣ-ದೇವ
ಹೊಯ್ಸಣ-ದೇಶದ
ಹೊಯ್ಸಣ-ನಾಡು
ಹೊಯ್ಸಣ-ರಾಯ
ಹೊಯ್ಸಣಾಖ್ಯಸ್ಯ
ಹೊಯ್ಸಲ-ನಾಡ
ಹೊಯ್ಸಲ-ನಾಡು
ಹೊಯ್ಸಲಾಹ್ವ-ಯವತ
ಹೊಯ್ಸಳ
ಹೊಯ್ಸಳ-ಕರ್ನಾಟಕ
ಹೊಯ್ಸಳ-ಕಾಲದ
ಹೊಯ್ಸಳಖ್ಯಾತರಂ
ಹೊಯ್ಸಳ-ದೇವನ
ಹೊಯ್ಸಳ-ದೇವನು
ಹೊಯ್ಸಳ-ದೇವರ
ಹೊಯ್ಸಳ-ದೇವರು
ಹೊಯ್ಸಳ-ದೇಶ
ಹೊಯ್ಸಳ-ದೇಶದ
ಹೊಯ್ಸಳ-ದೇಶ-ವನ್ನು
ಹೊಯ್ಸಳ-ದೇಶೇತ್ವಸ್ಮಿನ್
ಹೊಯ್ಸಳ-ನಾಡ
ಹೊಯ್ಸಳ-ನಾಡಾಗಿ
ಹೊಯ್ಸಳ-ನಾ-ಡಿನ
ಹೊಯ್ಸಳ-ನಾಡು
ಹೊಯ್ಸಳನು
ಹೊಯ್ಸಳ-ಮಹಾ-ಸಾಮಂನ್ತ
ಹೊಯ್ಸಳ-ಮಹೀಶ
ಹೊಯ್ಸಳರ
ಹೊಯ್ಸಳರ-ಕಾಲ-ದಿಂದ
ಹೊಯ್ಸಳರನ್ನು
ಹೊಯ್ಸಳರ-ರಾಜ್ಯಕ್ಕೆ
ಹೊಯ್ಸಳ-ರಲ್ಲಿಯೇ
ಹೊಯ್ಸಳ-ರ-ವ-ರೆಗೆ
ಹೊಯ್ಸಳ-ರಾಜ್ಯ
ಹೊಯ್ಸಳ-ರಾಜ್ಯ-ದಲ್ಲಿ
ಹೊಯ್ಸಳ-ರಾಜ್ಯ-ಪಯೋಜಭಾನು
ಹೊಯ್ಸಳ-ರಾಜ್ಯ-ವನ್ನು
ಹೊಯ್ಸಳ-ರಾಜ್ಯಾಧಿ-ಪತಿ
ಹೊಯ್ಸಳ-ರಿಂದ
ಹೊಯ್ಸಳ-ರಿಗೂ
ಹೊಯ್ಸಳ-ರಿಗೆ
ಹೊಯ್ಸಳರು
ಹೊಯ್ಸಳ-ರೆಂದು
ಹೊಯ್ಸಳರೇ
ಹೊಯ್ಸಳ-ವಂಶದ
ಹೊಯ್ಸಳ-ಸಣ್ನೆ-ನಾಡಾಳ್ವ
ಹೊಯ್ಸಳ-ಸಾಮ್ರಾಜ್ಯ
ಹೊಯ್ಸಳ-ಸಾಮ್ರಾಜ್ಯದ
ಹೊಯ್ಸಳಸೆಟಿ
ಹೊಯ್ಸಳಸೆಟ್ಟಿ
ಹೊಯ್ಸಳೇಶ್ವರ
ಹೊಯ್ಸಿಳ
ಹೊಯ್ಸೆಯ
ಹೊಯ್ಸೆಯ-ನಾಯ-ಕನ
ಹೊಯ್ಸೊಳಲು
ಹೊರಗಿನ
ಹೊರಗುತ್ತಿಗೆ-ಯಾಗಿ
ಹೊರಟ
ಹೊರಟಂತೆ
ಹೊರಟನು
ಹೊರಟರು
ಹೊರಟಾಗ
ಹೊರಟಿದೆ
ಹೊರಟಿದ್ದು
ಹೊರಟಿ-ರುವ
ಹೊರಟಿ-ರುವು
ಹೊರಟಿ-ರು-ವುದು
ಹೊರಟಿವೆ
ಹೊರಡಿ-ಸ-ಲಾಗಿದೆ
ಹೊರಡಿಸಿ
ಹೊರಡಿಸಿ-ದ-ನೆಂದೂ
ಹೊರಡಿಸಿದ್ದರೂ
ಹೊರಡಿಸಿ-ರುವ
ಹೊರಡಿ-ಸುವ
ಹೊರಡುತ್ತದೆ
ಹೊರಡುತ್ತಿದ್ದರು
ಹೊರಡುವ
ಹೊರತಾದ
ಹೊರತು
ಹೊರತು-ಪಡಿಸಿ
ಹೊರತು-ಪಡಿಸಿ-ದರೆ
ಹೊರದೂಡಲು
ಹೊರದೂಡು-ವಲ್ಲಿ
ಹೊರವಲೆ-ನಾ-ಡಿನ
ಹೊರ-ವಾರು
ಹೊರ-ವೃತ್ತಿಯ
ಹೊರುವ
ಹೊಲ-ಕುಪ್ಪೆ
ಹೊಲಗನ-ಹಳ್ಳಿ
ಹೊಲಗನ-ಹಳ್ಳಿ-ಯನ್ನು
ಹೊಲಗಾಹು
ಹೊಲ-ದಲ್ಲಿರುವ
ಹೊಲ-ವನ್ನು
ಹೊಲಿಯ-ಜಂಗುಲಿ-ಹುಲಿಯ
ಹೊಲೆ-ಮಗ್ಗ
ಹೊಲೆ-ಸುಂಕ
ಹೊಳಲಕೆರೆಗೆ
ಹೊಳಲಕೆರೆಯ
ಹೊಳಲಕೆರೆ-ಯಲ್ಲಿ
ಹೊಳಲಕೆರೆ-ಯ-ವ-ರೆಗೆ
ಹೊಳಲಗುಂದ
ಹೊಳಲಯದ
ಹೊಳಲಯನಾಡ
ಹೊಳಲಯನಾಡು-ಗಳು
ಹೊಳಲಿನ
ಹೊಳಲಿಯ
ಹೊಳಲು
ಹೊಳೆ
ಹೊಳೆನರಸಿಪುರ
ಹೊಳೆನರಸೀಪುರ
ಹೊಳೆಯ
ಹೊಳೆಯು
ಹೊಸ
ಹೊಸಕನ್ನಂಬಾಡಿ
ಹೊಸಕೆರಯೂ
ಹೊಸಕೋಟೆ
ಹೊಸಕೋಟೆಯ
ಹೊಸಗ್ರಾಮದ
ಹೊಸಣಮಂಡಲಧೃತಃರಾಜಶ್ರೀ
ಹೊಸ-ದಾಗಿ
ಹೊಸದುರ್ಗ
ಹೊಸನಾಡು
ಹೊಸಪಟ್ಟಣ
ಹೊಸಪಟ್ಟಣದ
ಹೊಸಪಟ್ಟಣ-ದಲ್ಲಿ
ಹೊಸಪಟ್ಟಣ-ದಿಂದ
ಹೊಸಪಟ್ಟಣ-ವನ್ನು
ಹೊಸಪಟ್ಟಣ-ವನ್ನೇ
ಹೊಸಪಟ್ಟಣ-ವಾಗಿ-ರುವ
ಹೊಸಪಟ್ಟಣ-ವೆಂದಾಯಿತು
ಹೊಸಪಟ್ಟಣ-ವೆಂಬ
ಹೊಸಪಟ್ಟಣವೇ
ಹೊಸಪುರ
ಹೊಸ-ಬಿರುದರ
ಹೊಸಬೂದನೂರು
ಹೊಸಮಲೆ
ಹೊಸಲಹೊಳಲು
ಹೊಸವಾಡದ
ಹೊಸವೀಡು
ಹೊಸವೊಳಲ
ಹೊಸಹಳ್ಳಿ
ಹೊಸಹಳ್ಳಿಪುರ
ಹೊಸಹಳ್ಳಿ-ಯನ್ನು
ಹೊಸಹೊಳಲ
ಹೊಸಹೊಳಲಿಗೆ
ಹೊಸಹೊಳಲಿನ
ಹೊಸಹೊಳಲು
ಹೊಸಹೊಳಲೇ
ಹೊಸೂರು
ಹೋಗ-ಬೇಕಾಯಿತು
ಹೋಗಲಾಡಿ-ಸಿದ-ರೆಂಬುದು
ಹೋಗಲು
ಹೋಗಿ
ಹೋಗಿದೆ
ಹೋಗಿದ್
ಹೋಗಿದ್ದನು
ಹೋಗಿದ್ದ-ನೆಂದು
ಹೋಗಿದ್ದು
ಹೋಗಿ-ಬಂದ-ನೆಂಬುದು
ಹೋಗಿರ-ಬಹುದು
ಹೋಗಿರ-ಬಹು-ದೆಂದು
ಹೋಗಿ-ರಲು
ಹೋಗಿ-ರುವ
ಹೋಗಿವೆ
ಹೋಗು
ಹೋಗುತ್ತದೋ
ಹೋಗುತ್ತಾನೆ
ಹೋಗುತ್ತಿದ್ದ
ಹೋಗುತ್ತಿದ್ದರು
ಹೋಗುತ್ತಿದ್ದರೆ
ಹೋಗುತ್ತಿದ್ದಾಗ
ಹೋಗುವ
ಹೋತ-ನಡಕೆಯ
ಹೋತ್ತಮನಂ
ಹೋದ
ಹೋದನು
ಹೋದರು
ಹೋದರೂ
ಹೋದ-ರೆಂದು
ಹೋದಾಗ
ಹೋಬಲಾ
ಹೋಬಳಿ
ಹೋಬಳಿ-ಗಳ
ಹೋಬಳಿ-ಗಳು
ಹೋಬಳಿಗೂ
ಹೋಬಳಿಗೆ
ಹೋಬಳಿಯ
ಹೋಬಳಿ-ಯಂತಹ
ಹೋಬಳಿ-ಯನ್ನು
ಹೋಬಳಿ-ಯಲ್ಲಿ
ಹೋಬಳಿ-ಯಲ್ಲಿದೆ
ಹೋಬಳಿ-ಯಲ್ಲಿದ್ದರೆ
ಹೋಬಳಿ-ಯಾಗಿ
ಹೋಯಿತಂತೆ
ಹೋಯಿತು
ಹೋರಾಟ
ಹೋರಾಟಕ್ಕೆ
ಹೋರಾಟ-ಗಳನ್ನು
ಹೋರಾಟ-ಗಳಲ್ಲಿ
ಹೋರಾಟ-ದಲ್ಲಿ
ಹೋರಾಟ-ನಡೆಸಿ
ಹೋರಾಟ-ವನ್ನು
ಹೋರಾಡಿ
ಹೋರಾಡಿದ
ಹೋರಾಡಿದಂತೆ
ಹೋರಾಡಿದ-ನೆಂದು
ಹೋರಾಡಿದ-ನೆಂದೂ
ಹೋರಾಡಿದರು
ಹೋರಾಡಿದರೆ
ಹೋರಾಡಿ-ದವ-ರಲ್ಲಿ
ಹೋರಾಡಿದಾಗ
ಹೋರಾಡಿದು-ದನ್ನು
ಹೋರಾಡಿರ-ಬಹುದು
ಹೋರಾಡುತ್ತಾ
ಹೋರಾಡುತ್ತಿದ್ದ
ಹೋರಾಡುತ್ತಿದ್ದರು
ಹೋರಾಡುತ್ತಿದ್ದ-ರೆಂದು
ಹೋರಾಡು-ವಲ್ಲಿ
ಹೋರಾಡು-ವಾಗ
ಹೋರಿನಿ-ದೇವ-ನಿಗೆ
ಹೋರ್ಷಣಾಹ್ವಯ
ಹೋಲಿಸ-ಬಹು-ದೆಂದು
ಹೋಲಿಸ-ಲಾಗಿದೆ
ಹೋಲಿಸಿ-ರು-ವುದು
ಹೋಲುತ್ತ-ದೆಂದೂ
ಹೋಸಣ
ಹೋಸಣ-ದೇಶದ
ಹೋಸಣಾಖ್ಯ
ಹೋಸಲ-ನಾಡ
ಹೋಸಲ-ನಾ-ಡಿನ
ಹೋಸಲ-ನಾಡು
ಹೋಸಲ-ನಾಡು-ಹೊಯ್ಸಳ
ಹ್ಯಾರಿಸ್
ಹ್ರಸ್ವ-ರೂಪ
ಹ್ವೈಸಣ
ೞೋಡವ-ನಾಯಕ
ೞೋಡವ-ನಾಯ-ಕರು
}
