\sethyphenation{kannada}{
ಅಂಕಕಾರ
ಅಂಕಕಾರನಾಗಿದ್ದನೆಂದಯ
ಅಂಕಕಾರನೆಂದು
ಅಂಕಕಾರಸೇನಾಧಿಪತಿ
ಅಂಕಗಾವುಂಡ
ಅಂಕನಹಳ್ಳಿ
ಅಂಕನಹಳ್ಳಿಯನ್ನು
ಅಂಕನಾಥಪುರದ
ಅಂಕಿಗಳು
ಅಂಕುಶರಾಯ
ಅಂಕುಶೇಂದ್ರನ
ಅಂಕುಶೇಂದ್ರನಿರಬಹುದು
ಅಂಕೆಯ
ಅಂಕೆಯದಂಣ್ನಾಯಕರ
ಅಂಕೆಯನಾಯಕನು
ಅಂಗಭೋಗ
ಅಂಗರಕ್ಕ
ಅಂಗರಕ್ಕನೆಂದು
ಅಂಗವಾಗಿದ್ದ
ಅಂಘರಕರಿಗೆ
ಅಂಡಲೆಯುತ್ತಿದ್ದ
ಅಂಣ
ಅಂಣನೆಂದೆನಿಸಿ
ಅಂಣ್ನಯ್ಯನು
ಅಂತಃಕಲಹದ
ಅಂತಃಪುರಾಧ್ಯಕ್ಷ
ಅಂತರ
ಅಂತರಂಗದ
ಅಂತರವಳ್ಳಿ
ಅಂತರವಳ್ಳಿಯ
ಅಂತರವಳ್ಳಿಯಲ್ಲಿ
ಅಂತರಿಕ
ಅಂತರ್ಗತವಾಗಿತ್ತು
ಅಂತರ್ಗತವಾಗಿದ್ದಂತೆ
ಅಂತರ್ಗತವಾಗಿದ್ದವು
ಅಂತರ್ಗತವಾಗಿದ್ದವೆಂದು
ಅಂತಸ್ತುಗಳನ್ನು
ಅಂತಹ
ಅಂತಹವರಲ್ಲಿ
ಅಂತೆಂಬರಗಂಡ
ಅಂತ್ಯಕಾಲದಲ್ಲಿ
ಅಂತ್ಯದಲ್ಲಿ
ಅಂದಮೇಲೆ
ಅಂದರೆ
ಅಂದರೆರಲ್ಲಿ
ಅಂದಿನ
ಅಂದಿನಿಂದ
ಅಂದೇ
ಅಂಬಲಿ
ಅಂಬಾರಿಯ
ಅಂಶ
ಅಂಶಗಳ
ಅಂಶಗಳನ್ನು
ಅಂಶಗಳಾವುವೂ
ಅಂಶಗಳು
ಅಂಶವಾಗಿದೆ
ಅಂಶವಾಗುವುದಿಲ್ಲ
ಅಂಶವು
ಅಉಬಳದೇವ
ಅಕಜಾಪುರವು
ಅಕರವಾಗಿ
ಅಕಳಂಕನ
ಅಕಸಾಲೆಗಳು
ಅಕಾಡೆಮಿಯಿಂದ
ಅಕಾರಾದಿಯಾಗಿ
ಅಕಾಲವರ್ಷ
ಅಕಾಲವರ್ಷನು
ಅಕಾಲವರ್ಷನುಇಮ್ಮಡಿ
ಅಕ್ಕ
ಅಕ್ಕಜಾಪುರದ
ಅಕ್ಕನ
ಅಕ್ಕಪಕ್ಕದ
ಅಕ್ಕಪಕ್ಕದಲ್ಲಿ
ಅಕ್ಕಪಕ್ಕದಲ್ಲಿದ್ದ
ಅಕ್ಕಪ್ಕದಲ್ಲಿ
ಅಕ್ಕಮಾ
ಅಕ್ಕರಸಾಕ್ಷಿ
ಅಕ್ಕಸಾಲಿಗರು
ಅಕ್ಕಿಯೆಬ್ಬಾಳು
ಅಕ್ಕಿಹೆಬ್ಬಾಳಿಗೆ
ಅಕ್ಕಿಹೆಬ್ಬಾಳು
ಅಕ್ಟೋಬರ್
ಅಕ್ಷತೃತೀಯದಂದು
ಅಕ್ಷರವನ್ನು
ಅಖಂಡ
ಅಖಂಡಬಾಗಿಲಿನ
ಅಖಂಡಿತ
ಅಖಿಲಭಾರತ
ಅಖಿಳಗುಣಧಾರೆ
ಅಗಣ್ಯಪುಣ್ಯವೇ
ಅಗತಿಯಪ್ಪ
ಅಗತಿಯಪ್ಪನ
ಅಗತಿಯಪ್ಪನಿಗೆ
ಅಗತ್ತಿ
ಅಗತ್ಯ
ಅಗತ್ಯಗಳಿಗೆ
ಅಗತ್ಯವಾಗಿದೆ
ಅಗತ್ಯವಿಲ್ಲವೆಂದು
ಅಗಮ್ಯವಾಗಿತ್ತೆಂದು
ಅಗರದಿಂದ
ಅಗರದುರ್ಗಅಗ್ರಹಾರ
ಅಗಸರಹಳ್ಳಿ
ಅಗಸ್ತ್ಯೇಶ್ವರ
ಅಗ್ರಅಹಾರ
ಅಗ್ರಗಣ್ಯನೆನಿಸಿ
ಅಗ್ರಹಾರ
ಅಗ್ರಹಾರಕ್ಕೆ
ಅಗ್ರಹಾರಗಳ
ಅಗ್ರಹಾರಗಳನ್ನಾಗಿ
ಅಗ್ರಹಾರಗಳನ್ನು
ಅಗ್ರಹಾರಗಳಲ್ಲಿ
ಅಗ್ರಹಾರಗಳು
ಅಗ್ರಹಾರದ
ಅಗ್ರಹಾರದಮಜ್ಜಿಗೆಪುರದ
ಅಗ್ರಹಾರದಲ್ಲಿ
ಅಗ್ರಹಾರದಲ್ಲಿದ್ದ
ಅಗ್ರಹಾರದವರೆಂದು
ಅಗ್ರಹಾರದಿಂದ
ಅಗ್ರಹಾರದ್ವಯವನ್ನಾಗಿ
ಅಗ್ರಹಾರನವನ್ನು
ಅಗ್ರಹಾರಬಾಚಹಳ್ಳಿ
ಅಗ್ರಹಾರಬಾಚಹಳ್ಳಿಯ
ಅಗ್ರಹಾರಬಾಚಹಳ್ಳಿಯನ್ನು
ಅಗ್ರಹಾರಬೆಳಗಲಿ
ಅಗ್ರಹಾರಬೆಳಗಲಿಯಲ್ಲಿ
ಅಗ್ರಹಾರಬೆಳಗುಲಿ
ಅಗ್ರಹಾರಬೆಳುಗಲಿಯಲ್ಲಿ
ಅಗ್ರಹಾರವ
ಅಗ್ರಹಾರವನ್ನಾಗಿ
ಅಗ್ರಹಾರವನ್ನಾಗಿಯೂ
ಅಗ್ರಹಾರವನ್ನು
ಅಗ್ರಹಾರವಾಗಿ
ಅಗ್ರಹಾರವಾಗಿತ್ತು
ಅಗ್ರಹಾರವಾಗಿದ್ದು
ಅಗ್ರಹಾರವಾಗಿಯೂ
ಅಗ್ರಹಾರವಾದ
ಅಗ್ರಹಾರವು
ಅಚ್ಚುತರಾಯ
ಅಚ್ಚೊತ್ತಿದ
ಅಚ್ಯತರಾಯವೀರಣ್ಣ
ಅಚ್ಯತೇಂದ್ರ
ಅಚ್ಯುತ
ಅಚ್ಯುತದೇವ
ಅಚ್ಯುತದೇವನ
ಅಚ್ಯುತದೇವನನ್ನು
ಅಚ್ಯುತದೇವನು
ಅಚ್ಯುತದೇವಮಹಾರಾಯನು
ಅಚ್ಯುತನ
ಅಚ್ಯುತಪುರವೆಂಬ
ಅಚ್ಯುತಮಹಾರಾಯರಿಗೆ
ಅಚ್ಯುತರಾಯ
ಅಚ್ಯುತರಾಯನ
ಅಚ್ಯುತರಾಯನನ್ನು
ಅಚ್ಯುತರಾಯನಿಗೆ
ಅಚ್ಯುತರಾಯನು
ಅಚ್ಯುತರಾಯರಿಗೆ
ಅಚ್ಯುತಸಮುದ್ರ
ಅಚ್ಯುತಾಖ್ಯೋ
ಅಚ್ಯುತಿಮಯ್ಯ
ಅಚ್ಯುತೇಂದ್ರ
ಅಚ್ಯುತೇಂದ್ರನು
ಅಜಿತಮುನಿ
ಅಜಿತಸೇನರೆಂದು
ಅಜಿತಸೇನಾಚಾರ್ಯರ
ಅಜಿತಾದೇವಿ
ಅಜ್ಜ
ಅಜ್ಜನಹಳ್ಳಿ
ಅಜ್ಜನಾಯಕನಹಳ್ಳಿ
ಅಜ್ಜನಿಗೆ
ಅಜ್ಜವೂರನ್ನೇ
ಅಜ್ಜವೂರು
ಅಜ್ಜಿ
ಅಜ್ಞಾತ
ಅಜ್ಞೆ
ಅಜ್ಞೆಯ
ಅಟ್ಟಾಡಿಸಿಕೊಂಡು
ಅಟ್ಟಿದನು
ಅಟ್ಟಿದನೆಂದು
ಅಟ್ಟಿದನೆಂದೂ
ಅಟ್ಟಿಸಿಕೊಂಡು
ಅಟ್ಟುಣ್ಣಲೀಯದೆ
ಅಠವಣ
ಅಠವಣೆ
ಅಠವಣೆಯ
ಅಡಕೆ
ಅಡಕೆಮರದ
ಅಡಕೆಯತೊಟ
ಅಡಕೆಯತೋಟ
ಅಡಗಿದರೂ
ಅಡಗಿದ್ದನ್ನು
ಅಡಗಿಸಲು
ಅಡಗಿಸಿ
ಅಡಗಿಸಿದ
ಅಡಗಿಸಿದನೆಂದೂ
ಅಡಗಿಸುವುದಕ್ಕಾಗಿ
ಅಡಪ
ಅಡಳಿತದಲ್ಲಿ
ಅಡಿ
ಅಡಿಕೆಮರದ
ಅಡಿಗೈಮಾನ್
ಅಡೆತಡೆ
ಅಡೆತಡೆಗಳನ್ನು
ಅಡೆತಡೆಗಳೂ
ಅಡ್ಡಲಾಗಿ
ಅಡ್ಡಹೆಸರು
ಅಡ್ಡಾಯಿದದ
ಅಡ್ಡಾಯುಧಅಢಾಯುಧ
ಅಡ್ಡಾಯ್ದದ
ಅಣತಿ
ಅಣಿಲೇಶ್ವರ
ಅಣುವಸಮುದ್ರದ
ಅಣುವಸಮುದ್ರದಲ್ಲಿಇಂದಿನ
ಅಣೆಕಟ್ಟನ್ನು
ಅಣೆಕಟ್ಟೆಯನ್ನು
ಅಣ್ಣ
ಅಣ್ಣಂದಿರು
ಅಣ್ಣಂದಿರೂ
ಅಣ್ಣತಮ್ಮಂದಿರ
ಅಣ್ಣತಮ್ಮಂದಿರಂತೆ
ಅಣ್ಣತಮ್ಮಂದಿರಾಗಿದ್ದು
ಅಣ್ಣತಮ್ಮಂದಿರೊಳಗೆ
ಅಣ್ಣನ
ಅಣ್ಣನಂಕುರ
ಅಣ್ಣನನ್ನು
ಅಣ್ಣನವರ
ಅಣ್ಣನಾದ
ಅಣ್ಣನಿಗಿಂತ
ಅಣ್ಣನು
ಅಣ್ಣನೂ
ಅಣ್ಣನೆಂದು
ಅಣ್ಣನೆಂಬುದೂ
ಅಣ್ಣಪ್ಪ
ಅಣ್ಣಬೂಚಣನಿಗೆ
ಅಣ್ಣಯ್ಯನವರ
ಅಣ್ಣಾಮಲೆ
ಅಣ್ನಯ್ಯ
ಅಣ್ನಾಮಲೆ
ಅತಿಕುಪ್ಪೆ
ಅತಿರಾತ್ರಿಗೆ
ಅತಿವಿಷಮ
ಅತಿಶಯವಾಗಿ
ಅತಿಶಯವಾದ
ಅತಿಸಣ್ಣ
ಅತೀ
ಅತೀತನಾದನು
ಅತುವಾಸುವಿನ
ಅತ್ತಿಕುಪ್ಪೆ
ಅತ್ತಿಕುಪ್ಪೆಗಳನ್ನು
ಅತ್ತಿಕುಪ್ಪೆಯೆಂಬ
ಅತ್ತಿಗುಪ್ಪೆ
ಅತ್ತಿಗುಪ್ಪೆಕೃಷ್ಣರಾಜಪೇಟೆ
ಅತ್ತಿಗುಪ್ಪೆಗೆ
ಅತ್ತಿಗೆ
ಅತ್ತಿಗೊಂಡನಹಳ್ಳಿ
ಅತ್ತಿತಾಳಾದತ್ತಲಪುರವನ್ನು
ಅತ್ತಿಮಬ್ಬೆಯನ್ನು
ಅತ್ಯಂತ
ಅತ್ಯಮನಾಯಕ
ಅಥವ
ಅಥವಾ
ಅದಕೆ
ಅದಕ್ಕಾಗಿ
ಅದಕ್ಕೂ
ಅದಕ್ಕೆ
ಅದನ್ನ
ಅದನ್ನು
ಅದನ್ನೂ
ಅದನ್ನೇ
ಅದರ
ಅದರಂತೆ
ಅದರಂತೆಯೇ
ಅದರಲ್ಲಿ
ಅದರಲ್ಲೂ
ಅದರಿಂದ
ಅದರಿಂದಲೇ
ಅದರಿಂದಾಗಿ
ಅದರಿಂದಾಗಿಯೇ
ಅದರೊಡನೆ
ಅದರೊಳಗೆ
ಅದಲಗೆರೆ
ಅದಲ್ಲ
ಅದಾಗುತ್ತಲೂ
ಅದಾದನಂತರ
ಅದಿರದ
ಅದು
ಅದೂ
ಅದೇ
ಅದೇರೀತಿ
ಅದೇವುದೋ
ಅದೊಂದು
ಅದ್ದಿಯಾಪಳ್ತಿಯ
ಅದ್ದಿಹಳ್ಳಿ
ಅದ್ಯತನ
ಅದ್ಯಾಪಿ
ಅದ್ವಿತೀಯವಾದುದೆಂದರೆ
ಅಧಿಕಬಳ
ಅಧಿಕಾರ
ಅಧಿಕಾರಕ್ಕಾಗಿ
ಅಧಿಕಾರಕ್ಕೆ
ಅಧಿಕಾರಗಳನ್ನು
ಅಧಿಕಾರದ
ಅಧಿಕಾರದಲ್ಲಿ
ಅಧಿಕಾರದಲ್ಲಿದ್ದನು
ಅಧಿಕಾರದಲ್ಲಿದ್ದು
ಅಧಿಕಾರದಲ್ಲಿರುವಾಗ
ಅಧಿಕಾರದಿಂದ
ಅಧಿಕಾರಪದದ
ಅಧಿಕಾರವನ್ನು
ಅಧಿಕಾರವರ್ಗಗಳ
ಅಧಿಕಾರವರ್ಗದವರು
ಅಧಿಕಾರವಿರಲಿಲ್ಲ
ಅಧಿಕಾರವು
ಅಧಿಕಾರವೂ
ಅಧಿಕಾರವೆಲ್ಲಾ
ಅಧಿಕಾರಿ
ಅಧಿಕಾರಿಗಳ
ಅಧಿಕಾರಿಗಳನ್ನು
ಅಧಿಕಾರಿಗಳಾಗಿ
ಅಧಿಕಾರಿಗಳಾಗಿದ್ದರು
ಅಧಿಕಾರಿಗಳಾಗಿದ್ದರೆಂದು
ಅಧಿಕಾರಿಗಳಾಗಿದ್ದರೆಂಬುದು
ಅಧಿಕಾರಿಗಳಾಗಿದ್ದಿರಬಹುದು
ಅಧಿಕಾರಿಗಳಾಗಿದ್ದು
ಅಧಿಕಾರಿಗಳಾಗಿರಲಿಲ್ಲ
ಅಧಿಕಾರಿಗಳಾದ
ಅಧಿಕಾರಿಗಳಿಂದ
ಅಧಿಕಾರಿಗಳಿಗಿರುವುದರಿಂದ
ಅಧಿಕಾರಿಗಳಿಗೆ
ಅಧಿಕಾರಿಗಳಿದ್ದರು
ಅಧಿಕಾರಿಗಳಿರಬಹುದು
ಅಧಿಕಾರಿಗಳಿರುತ್ತಿದ್ದರು
ಅಧಿಕಾರಿಗಳು
ಅಧಿಕಾರಿಗಳೆಂದು
ಅಧಿಕಾರಿಗೇ
ಅಧಿಕಾರಿಯ
ಅಧಿಕಾರಿಯನ್ನು
ಅಧಿಕಾರಿಯನ್ನೂ
ಅಧಿಕಾರಿಯಾಗಿ
ಅಧಿಕಾರಿಯಾಗಿದ್ದ
ಅಧಿಕಾರಿಯಾಗಿದ್ದಂತೆ
ಅಧಿಕಾರಿಯಾಗಿದ್ದನು
ಅಧಿಕಾರಿಯಾಗಿದ್ದನೆಂದು
ಅಧಿಕಾರಿಯಾಗಿದ್ದನೆಂಬ
ಅಧಿಕಾರಿಯಾಗಿದ್ದರೂ
ಅಧಿಕಾರಿಯಾಗಿದ್ದಿರಬಹುದು
ಅಧಿಕಾರಿಯಾಗಿದ್ದಿರಬಹುದೆಂದು
ಅಧಿಕಾರಿಯಾಗಿದ್ದು
ಅಧಿಕಾರಿಯಾಗಿರಬಹುದು
ಅಧಿಕಾರಿಯಾಗಿರಬೇಕು
ಅಧಿಕಾರಿಯಾಗಿರುತ್ತಾನೆ
ಅಧಿಕಾರಿಯಾದ
ಅಧಿಕಾರಿಯು
ಅಧಿಕಾರಿಯೂ
ಅಧಿಕಾರಿಯೇ
ಅಧಿಕಾರಿಯೊಬ್ಬನ
ಅಧಿಕಾರಿಯೋ
ಅಧಿಕೃತಗೊಳಿಸಿರಬಹುದೆಂದು
ಅಧಿಕೃತಗೊಳಿಸುತ್ತವೆ
ಅಧಿದೈವವಾಗಿ
ಅಧಿನನಾಗಿದ್ದ
ಅಧಿನರಾಗಿ
ಅಧಿಪತಿಗಳಾಗಿದ್ದ
ಅಧಿಪತಿಗಳಾಗಿದ್ದರು
ಅಧಿಪತಿಗಳಾಗಿದ್ದರೆಂದು
ಅಧಿಪತಿಯನ್ನಾಗಿ
ಅಧಿಪತಿಯಾಗಿದ್ದು
ಅಧಿಪತಿಯಾದನೆಂಬ
ಅಧಿರಾಜ
ಅಧಿರಾಜತ್ವದ
ಅಧಿರಾಜತ್ವವನ್ನು
ಅಧಿರಾಜರು
ಅಧಿವಸಿತನಾಗಿದ್ದನೆಂದು
ಅಧೀನ
ಅಧೀನತೆಯನ್ನು
ಅಧೀನದಲ್ಲಿ
ಅಧೀನದಲ್ಲಿತ್ತು
ಅಧೀನದಲ್ಲಿದ್ದ
ಅಧೀನನಾಗಿ
ಅಧೀನರಾಗಿ
ಅಧೀನರಾಗಿದ್ದರೆಂದು
ಅಧೀನವಾಗಿತ್ತೆಂದು
ಅಧ್ಯಕ್ಷತೆಯಲ್ಲಿ
ಅಧ್ಯಯನ
ಅಧ್ಯಯನಕ್ಕೆ
ಅಧ್ಯಯನಗಳನ್ನೂ
ಅಧ್ಯಯನಗಳು
ಅಧ್ಯಯನದ
ಅಧ್ಯಯನದಲ್ಲಿ
ಅಧ್ಯಯನದಿಂದ
ಅಧ್ಯಯನವನ್ನು
ಅಧ್ಯಯನವಾಗಲೀ
ಅಧ್ಯಾಯ
ಅಧ್ಯಾಯದಲ್ಲಿ
ಅನಂತ
ಅನಂತಪುರ
ಅನಂತರ
ಅನಂತಾಚಾರ್ಯರ
ಅನಂತಾಚಾರ್ಯರು
ಅನಂತೋಜಿ
ಅನತಿ
ಅನತಿದೂರದಲ್ಲಿರುವ
ಅನವರತ
ಅನಾದಿ
ಅನಾದಿಕಾಲದಿಂದ
ಅನಾಯಕತ್ವ
ಅನಾಹುತವನ್ನು
ಅನಿರುದ್ಧನಂತಹ
ಅನುಕರಿಸಿ
ಅನುಕರಿಸಿದನೆಂದು
ಅನುಕೂಲ
ಅನುಕೂಲಕ್ಕಾಗಿ
ಅನುಕೂಲಕ್ಕೋಸ್ಕರ
ಅನುಕೂಲತೆಯ
ಅನುಗುಣವಾಗಿ
ಅನುಜ
ಅನುಜನಾದ
ಅನುನಯದಿಂ
ಅನುಪಮತೇಜಂ
ಅನುಪಮದಾನಿ
ಅನುಪಮವಾದ
ಅನುಬಂಧದಲ್ಲಿ
ಅನುಭವ
ಅನುಭವವನ್ನು
ಅನುಭವಿಸಬೇಕಾಗಿ
ಅನುಮತದಿಂದ
ಅನುಮತಿ
ಅನುಮತಿಯ
ಅನುಮತಿಯನ್ನು
ಅನುಮತಿಯಿಂದ
ಅನುಮತಿಯೂ
ಅನುಮಾನ
ಅನುಯಾಯಿಗಳಾಗಿದ್ದು
ಅನುಯಾಯಿಗಳೂ
ಅನುಯಾಯಿಯಾಗಿದ್ದನು
ಅನುಯಾಯಿಯಾಗಿದ್ದನೆಂದು
ಅನುಯಾಯಿಯಾದ
ಅನುಲಕ್ಷಿಸಿ
ಅನುವಂಶಿಕವಾದ
ಅನುವಂಶೀಯ
ಅನುವರದೊಳ್
ಅನುವಾದವಿದ್ದಂತಿದೆ
ಅನುಸರಿಸಿ
ಅನುಸರಿಸಿದಂತೆ
ಅನುಸರಿಸುತ್ತಾ
ಅನುಸರಿಸುತ್ತಿದ್ದರೂ
ಅನುಸರಿಸುತ್ತಿದ್ದರೆಂಬುದು
ಅನೇಕ
ಅನೇಕಕಡೆ
ಅನೇಕರಿಗೆ
ಅನೇಕರು
ಅನೇಕವೇಳೆ
ಅನ್ನ
ಅನ್ನಛತ್ರ
ಅನ್ನದಾನಕ್ಕೆಂದು
ಅನ್ನದಾನಪಲ್ಲಿಯಲ್ಲಿ
ಅನ್ನದಾನಪಳ್ಳಿಯನ್ನು
ಅನ್ನದಾನಪಳ್ಳಿಯನ್ನುಇಂದಿನ
ಅನ್ನದಾನವಿನೋದ
ಅನ್ನಸತ್ರವನ್ನು
ಅನ್ಯ
ಅನ್ಯಗ್ರಂಥಗಳಿಗೆ
ಅನ್ಯರೂಪಗಳು
ಅನ್ಯೋನ್ಯತೆಯಿಂದ
ಅನ್ವಯ
ಅನ್ವಯಕ್ಕೆ
ಅನ್ವಯವನ್ನು
ಅನ್ವಯಾಗತ
ಅನ್ವಯಾಗತವಾಗಿ
ಅನ್ವಯಾಗದ
ಅನ್ವಯಾವತಾರವೆಂತೆಂದಡೆ
ಅನ್ವಯಿಸುತ್ತದೆಂದು
ಅಪತಿಯ
ಅಪಭ್ರಂಶ
ಅಪಮಾನಗಳು
ಅಪರಿಮಿತದಾನಸಾರವೃಷ್ಟಿ
ಅಪರೂಪಕ್ಕೆ
ಅಪರೂಪದ
ಅಪವಾದಾತ್ಮಕವಾಗಿ
ಅಪಹರಣ
ಅಪಾಯ
ಅಪಾರ
ಅಪಾರಪ್ರಮಾಣದ
ಅಪಾರವಾಗಿ
ಅಪಾರವಾದ
ಅಪಾರಸೈನ್ಯ
ಅಪೂರ್ಬ್ಬಾಯ
ಅಪ್ಪ
ಅಪ್ಪಣೆ
ಅಪ್ಪಣೆಯ
ಅಪ್ಪಣೆಯಂತೆ
ಅಪ್ಪಣೆಯನ್ನು
ಅಪ್ಪಣ್ಣ
ಅಪ್ಪಣ್ಣನಾಯಕನು
ಅಪ್ಪಣ್ಣಭೂಪತಿಯು
ಅಪ್ಪನಿಂದ
ಅಪ್ಪಳಕ್ಕನಹಳ್ಳಿ
ಅಪ್ಪಾಜಿ
ಅಪ್ಪಿದನೆಂದು
ಅಪ್ಪೆನಾಯಕ
ಅಪ್ಪೆಯ
ಅಪ್ರತಿಕಮಲ್ಲ
ಅಪ್ರತಿಮ
ಅಪ್ರತಿಮತೇಜಂ
ಅಪ್ರತಿಮನಾಗಿದ್ದು
ಅಪ್ರತಿಮವೀರ
ಅಪ್ರತಿಮವೀರನರಪತಿ
ಅಪ್ರಮೇಯ
ಅಪ್ರಮೇಯನ
ಅಪ್ರಮೇಯನು
ಅಫ್ತಬ್ಖಾನ್
ಅಬಲವಾಡಿ
ಅಬಲವಾಡಿಯಲ್ಲಿ
ಅಬಸಮುದ್ರಅಹೋಬಲಸಮುದ್ರ
ಅಬ್ದುಲ್
ಅಬ್ಬಗಂಜೂರು
ಅಬ್ಬರಾಜಗಳ
ಅಬ್ಬರಾಜನ
ಅಬ್ಬರಾಜುಗಳ
ಅಬ್ಬಾಸ್ಗಾರ್ಡನ್
ಅಬ್ಬೂರು
ಅಭಯಾರಣ್ಯವಾಗಿದ್ದು
ಅಭಿನನ್ನೆಂದು
ಅಭಿನವ
ಅಭಿನವಕುಲಶೇಖರರಾದ
ಅಭಿನವಮದನಾವತಾರ
ಅಭಿನ್ನರಾಗಿದ್ದು
ಅಭಿನ್ನರಿದ್ದು
ಅಭಿನ್ನರಿರಬಹುದು
ಅಭಿನ್ನರು
ಅಭಿನ್ನರೆಂದು
ಅಭಿಪ್ರಾಯ
ಅಭಿಪ್ರಾಯಗಳನ್ನು
ಅಭಿಪ್ರಾಯಪಟಿದ್ದಾರೆ
ಅಭಿಪ್ರಾಯಪಟ್ಟಿದ್ದಾರೆ
ಅಭಿಪ್ರಾಯಪಟ್ಟಿರುವುದು
ಅಭಿಪ್ರಾಯಪಡಲಾಗಿದೆ
ಅಭಿಪ್ರಾಯಪಡುತ್ತಾರೆ
ಅಭಿಪ್ರಾಯವನ್ನು
ಅಭಿಪ್ರಾಯವಾಗಿದೆ
ಅಭಿಪ್ರಾಯವು
ಅಭಿಪ್ರಾಯವೂ
ಅಭಿಮತ
ಅಭಿಮನ್ಯುವು
ಅಭಿಮಾನದಿಂದ
ಅಭಿಮಾನಿಯಾಗಿದ್ದನೆಂದು
ಅಭಿಯೋಗ
ಅಭಿವೃದ್ಧಿ
ಅಭಿವೃದ್ಧಿಗೆ
ಅಭಿವೃದ್ಧಿಯಾಗಬೇಕೆಂದು
ಅಭಿವ್ಯಕ್ತಿಗೆ
ಅಭಿಷಿಕ್ತ
ಅಭಿಷೇಕಕ್ಕೆ
ಅಭ್ಯುದಯದಲ್ಲಿ
ಅಭ್ಯುದಯಾರ್ಥ
ಅಮರ
ಅಮರಂಬೋದು
ಅಮರನಾಯಕತನಕ್ಕೆ
ಅಮರನಾಯಕತನಕ್ಕೆಸೇರಿತ್ತು
ಅಮರನಾಯಕನಾಗಿರುತ್ತಾನೆ
ಅಮರನಾಯಕರ
ಅಮರನಾಯಕರಿಗೆ
ಅಮರನಾಯಕರು
ಅಮರಪಡೆಯ
ಅಮರಮಹಲೆ
ಅಮರಮಾಗಣಿ
ಅಮರಮಾಗಣಿಗೆ
ಅಮರಮಾಗಣಿಯಾಗಿ
ಅಮರಮಾಗಣೆಗೆ
ಅಮರಮಾಗಣೆಯಾಗಿ
ಅಮರೇಂದ್ರ
ಅಮಲ್ದಾರ್
ಅಮಾತ್ಯ
ಅಮಾತ್ಯನಾಗಿದ್ದ
ಅಮಾತ್ಯಪದ
ಅಮಾತ್ಯರು
ಅಮಿಲ್
ಅಮೀಲ
ಅಮೀಲ್ದಾರನನ್ನು
ಅಮುಕ್ತ
ಅಮುದಸಮುದ್ರ
ಅಮುಲ್ದಾರ್
ಅಮೃತನಾಥಪುರವಾದ
ಅಮೃತಪಡಿ
ಅಮೃತಪಡಿಗೆ
ಅಮೃತಾಂಬಾ
ಅಮೃತಾಪುರ
ಅಮೃತಿ
ಅಮೃತೂರಿನ
ಅಮೃತೂರು
ಅಮೃತೇಶ್ವರ
ಅಮೋಘವರ್ಷ
ಅಮೋಘವರ್ಷನ
ಅಮೋಘವರ್ಷನು
ಅಮ್ಮನಪುರದ
ಅಮ್ಮನವರ
ಅಮ್ಮನವರಗುಡಿ
ಅಮ್ಮನವರಿಗೆ
ಅಮ್ಮನವರು
ಅಮ್ಮನವರೆಂಬ
ಅಮ್ಮಮ್ಮ
ಅಮ್ರಿತಪಡಿಗೆ
ಅಯಪ
ಅಯಿರಮೆನಾಯಕ
ಅಯ್ಕಣಂ
ಅಯ್ಕಣದ
ಅಯ್ಕಣನ
ಅಯ್ಕಣನನ್ನು
ಅಯ್ಕಣನಿಗೆ
ಅಯ್ಕಣನು
ಅಯ್ಕಣನೆಂಬ
ಅಯ್ಯ
ಅಯ್ಯಂಗಾರ್
ಅಯ್ಯಗೊಂಡನಪಲ್ಲಿ
ಅಯ್ಯಣ
ಅಯ್ಯದೇವ
ಅಯ್ಯನ
ಅಯ್ಯನವರ
ಅಯ್ಯನವರಿಗೆ
ಅಯ್ಯನವರಿಗೆಕೃಷ್ಣದೇವರಾಯ
ಅಯ್ಯನವರು
ಅಯ್ಯನು
ಅಯ್ಯಪ್ಪನಹಳ್ಳಿ
ಅಯ್ಯರವೀರ
ಅಯ್ಯಾವೊಳೆ
ಅಯ್ಯಾವೊಳೆಯ
ಅರಕನಕೆರೆ
ಅರಕಲಗೂಡು
ಅರಕೆರೆ
ಅರಕೆರೆಯ
ಅರಕೆರೆಯನ್ನು
ಅರಕೆಲ್ಲ
ಅರಕೇಸಿ
ಅರಕೇಸಿಯ
ಅರಕೇಸಿಯರಅ
ಅರಕೇಸಿಯು
ಅರಣ್ಯ
ಅರಣ್ಯವನ್ನು
ಅರನಕೆರೆಯ
ಅರಬ್ಬೀ
ಅರಮನೆ
ಅರಮನೆಯ
ಅರಮನೆಯಲ್ಲಿ
ಅರಮನೆಯಲ್ಲಿದ್ದ
ಅರಮನೆಯಲ್ಲಿಯೂ
ಅರಮನೆಯಾನ್ತ
ಅರಮನೆಯಿಂದ
ಅರಲುಕುಪ್ಪೆ
ಅರಳಿಮರಗಳಿಗೆ
ಅರವತ್ತೊಕ್ಕಲಿನ
ಅರವಮನೆಯಲ್ಲಿ
ಅರವೀಟಿ
ಅರವೀಟಿರಂಗರಾಜ
ಅರವೀಡು
ಅರಸ
ಅರಸಂಕಸೂನೆಗಾರ
ಅರಸನ
ಅರಸನಕೆರೆ
ಅರಸನಕೆರೆಯ
ಅರಸನನ್ನಾಗಿ
ಅರಸನನ್ನು
ಅರಸನಲ್ಲಿ
ಅರಸನಾಗಿ
ಅರಸನಾಗಿದ್ದ
ಅರಸನಾಗಿದ್ದಲ್ಲದೆ
ಅರಸನಾಗಿದ್ದು
ಅರಸನಾಗಿರಬಹುದು
ಅರಸನಾದ
ಅರಸನು
ಅರಸನೆಂದು
ಅರಸನೇ
ಅರಸನೋ
ಅರಸರ
ಅರಸರನ್ನು
ಅರಸರಲ್ಲಿ
ಅರಸರಲ್ಲಿಯೇ
ಅರಸರಿಗೂ
ಅರಸರಿಗೆ
ಅರಸರು
ಅರಸರುಗಳು
ಅರಸರುಮಣಲೇರ
ಅರಸರೂ
ಅರಸರೆಂದು
ಅರಸರೆಲ್ಲರೂ
ಅರಸರ್
ಅರಸರ್ಮ್ಮಹಾಸಾಮನ್ತಾಧಿಪತೀ
ಅರಸಾದಿತ್ಯ
ಅರಸಿ
ಅರಸಿಕೆರೆ
ಅರಸಿಕೆರೆಯ
ಅರಸಿಕೆರೆಯಲ್ಲಿ
ಅರಸಿದಂತಾದುದು
ಅರಸಿನ
ಅರಸಿನಕೆರೆ
ಅರಸಿನಕೆರೆಯ
ಅರಸಿಯಕೆರೆ
ಅರಸಿಯಕೆರೆಯ
ಅರಸೀಕೆರೆ
ಅರಸು
ಅರಸುಗಂಡ
ಅರಸುಗಳ
ಅರಸುಗಳು
ಅರಸುಮಕ್ಕಳು
ಅರಸೊತ್ತಿಗೆಯ
ಅರಸೊತ್ತಿಗೆಯನ್ನು
ಅರಾಜಕತೆಯನ್ನು
ಅರಿಕನಕಟ್ಟ
ಅರಿಕನಕಟ್ಟವನ್ನು
ಅರಿಕುಂಟೆ
ಅರಿಕುಠಾರಪುರದವನು
ಅರಿಕುಶಕುಠಾರ
ಅರಿಕೆ
ಅರಿಗೋಧೂಮಘರಟ್ಟ
ಅರಿನೃಪರ
ಅರಿಬಿರುದರ
ಅರಿಬಿರುದರದಂಡನಾಥ
ಅರಿಯಪ್ಪನು
ಅರಿರಾಯದಟ್ಟ
ಅರಿರಾಯವಿಭಾಡ
ಅರಿರೂಪಸಿಂಗ
ಅರಿವರ್ಮ
ಅರುಮುಳಿ
ಅರುಮುಳಿದೇವ
ಅರುಮುಳಿದೇವನ
ಅರುಮುಳಿದೇವನು
ಅರುಮೋಳಿದೇವ
ಅರುಳ್ನಾದನಿಗೆ
ಅರುವನಹಳ್ಳಿ
ಅರುವನಹಳ್ಳಿಯ
ಅರುವನಹಳ್ಳಿಯೇ
ಅರುಹನಹಳಿ್ಳ
ಅರುಹನಹಳ್ಳಿ
ಅರುಹನಹಳ್ಳಿಯ
ಅರುಹನಹಳ್ಳಿಯನ್ನು
ಅರುಹನಹಳ್ಳಿಯಲ್ಲಿರುವ
ಅರುಹನಹಳ್ಳಿಯವರಿಗೂ
ಅರುಹಳಿಅರುಹನಹಳ್ಳಿ
ಅರೆ
ಅರೆಕೊಠಾರದ
ಅರೆಕೊಠಾರದಲ್ಲಿ
ಅರೆತಿಪ್ಪೂರಿನ
ಅರೆತಿಪ್ಪೂರು
ಅರೆತಿಪ್ಪೂರೇ
ಅರೆಬಂಡೆಗಳು
ಅರೆಬೊಪ್ಪನಹಳ್ಳಿ
ಅರೆಯಹಳ್ಳಿ
ಅರೇಬಿಕ್
ಅರ್ಕಗುಪ್ತಿಪುರವೆಂದು
ಅರ್ಕಾವತಿ
ಅರ್ಕೇಶ್ವರ
ಅರ್ಕೇಶ್ವರನ
ಅರ್ಕೇಶ್ವರಸ್ವಾಮಿ
ಅರ್ಕ್ಕರದುರ್ಕ್ಕೆಯಿಂದ
ಅರ್ಕ್ಕೇಶ್ವರ
ಅರ್ಚಕರಂಗಸ್ವಾಮಿಯವರು
ಅರ್ಚಕರು
ಅರ್ಚನಾವೃತ್ತಿಗೆ
ಅರ್ಚನಾವೃತ್ತಿಯಾಗಿ
ಅರ್ಚನೆಗೆ
ಅರ್ಜುನಹಳ್ಳಿ
ಅರ್ತಿಗಳಿಗೂ
ಅರ್ಥ
ಅರ್ಥಗಳೂ
ಅರ್ಥದ
ಅರ್ಥದಲ್ಲಿ
ಅರ್ಥಪೂರ್ಣವಾಗಿ
ಅರ್ಥವನ್ನು
ಅರ್ಥವಾಗುತ್ತದೆ
ಅರ್ಥವಾಗುವುದಿಲ್ಲ
ಅರ್ಥವಿದೆ
ಅರ್ಥವಿದೆಯೇ
ಅರ್ಥವಿವೇಚನೆಯನ್ನು
ಅರ್ಥವೆಂದು
ಅರ್ಥೈಸಬಹುದು
ಅರ್ಥೈಸಬಹುದೇ
ಅರ್ಥೈಸಿ
ಅರ್ಥೈಸಿದ್ದಾರೆ
ಅರ್ಥೈಸಿದ್ದಾರೆಂದು
ಅರ್ಧ
ಅರ್ಧಕ್ಕೇ
ಅರ್ಧಭಾಗ
ಅರ್ಧಾಂಗ
ಅರ್ಪಿಸಲಾಯಿತೆಂದು
ಅರ್ಪಿಸಲು
ಅರ್ಪಿಸುವ
ಅರ್ಪೊಳೆಯ
ಅರ್ಯಾಬಿಕ್
ಅರ್ವಾಚೀನ
ಅರ್ಹಗೇಹಗಳನ್ನು
ಅರ್ಹತೆಯೂ
ಅರ್ಹನಲ್ಲವೆಂದು
ಅರ್ಹಪೂಜೆಗೆ
ಅರ್ಹರನ್ನು
ಅರ್ಹವಾಗಿವೆ
ಅರ್ಹವಾದ
ಅಲಂಕರಿಸಿದ
ಅಲಂಕರಿಸಿದ್ದ
ಅಲಂಕರಿಸಿದ್ದನು
ಅಲಂಕರಿಸಿದ್ದರು
ಅಲಂಕರಿಸಿರುವುದನ್ನು
ಅಲಂಕರಿಸುತ್ತಿದ್ದರು
ಅಲಂಘ್ಯ
ಅಲಸಿಂಗರಾರ್ಯಸ್ಯ
ಅಲಾ
ಅಲಿ
ಅಲಿಖಾನ್
ಅಲಿಯ
ಅಲಿಯವರು
ಅಲಿಯು
ಅಲಿಹೈದರನು
ಅಲೀ
ಅಲ್
ಅಲ್ಪಕಾಲ
ಅಲ್ಪಕಾಲದವರೆಗೆ
ಅಲ್ಲ
ಅಲ್ಲದೆ
ಅಲ್ಲಪ್ಪ
ಅಲ್ಲಪ್ಪದಂಡನಾಯಕನು
ಅಲ್ಲಪ್ಪನಹಳ್ಳಿ
ಅಲ್ಲಲ್ಲಿ
ಅಲ್ಲವೆಂದು
ಅಲ್ಲಾಂಬಾ
ಅಲ್ಲಾಳ
ಅಲ್ಲಾಳದೇವ
ಅಲ್ಲಾಳನಾಥ
ಅಲ್ಲಾಳನಾಥದೇವರಿಗೆ
ಅಲ್ಲಾಳಪೆರುಮಾಳ
ಅಲ್ಲಾಳಪೆರುಮಾಳದೇವರಿಗೆವರದರಾಜ
ಅಲ್ಲಾಳಪೆರುಮಾಳೆ
ಅಲ್ಲಾಳಸಮುದ್ರವೆಂಬ
ಅಲ್ಲಿ
ಅಲ್ಲಿಂದ
ಅಲ್ಲಿಂದಲೇ
ಅಲ್ಲಿಗೆ
ಅಲ್ಲಿಡಲಾಯಿತೆಂದು
ಅಲ್ಲಿದ್ದ
ಅಲ್ಲಿದ್ದವರು
ಅಲ್ಲಿದ್ದು
ಅಲ್ಲಿನ
ಅಲ್ಲಿಯ
ಅಲ್ಲಿಯವರೆಗೆ
ಅಲ್ಲಿಯೇ
ಅಲ್ಲಿರುವ
ಅಲ್ಲೂ
ಅಲ್ಲೇ
ಅಲ್ಲೋಲಕಲ್ಲೋಲ
ಅಳಗಿಯ
ಅಳಗುವಂಣನು
ಅಳತೆಯ
ಅಳವಡಿಸಿಕೊಂಡರೂ
ಅಳವಡಿಸಿಕೊಂಡು
ಅಳಹ
ಅಳಹಿಯ
ಅಳಿಯ
ಅಳಿಯಂದಿರಾದ
ಅಳಿಯನಾಗಿದ್ದು
ಅಳಿಯನಾಗಿರಬಹುದು
ಅಳಿಯನಾದ
ಅಳಿಯನೂ
ಅಳಿಯನೆಂದು
ಅಳಿಯನೇ
ಅಳಿಯರಾಮರಾಯ
ಅಳಿಯರಾಮರಾಯನ
ಅಳಿಯರಾಮರಾಯನೂ
ಅಳಿಸಂದ್ರ
ಅಳಿಸಂದ್ರಶಾಸನದಲ್ಲಿ
ಅಳಿಸಿ
ಅಳಿಸಿಹೋಗಿ
ಅಳಿಸಿಹೋಗಿದೆ
ಅಳಿಸಿಹೋಗಿವೆ
ಅಳೀಸಂದ್ರ
ಅಳೀಸಂದ್ರದ
ಅವಕಾಶವಾಯಿತೆಂದು
ಅವಕಾಶವಿದೆ
ಅವಧಿಗೆ
ಅವಧಿಯ
ಅವಧಿಯಲ್ಲಿ
ಅವನ
ಅವನನ್ನು
ಅವನಾದ
ಅವನಿಂದ
ಅವನಿಗೂ
ಅವನಿಗೆ
ಅವನಿಪನೆನಗಿತ್ತಪನೆಂದವರಿವರವೊಲುಳಿದ
ಅವನು
ಅವನೇ
ಅವನೊಬ್ಬ
ಅವರ
ಅವರದ್ದೇ
ಅವರನ್ನು
ಅವರಲ್ಲಿ
ಅವರವರ
ಅವರಿಂದ
ಅವರಿಂದಲೇ
ಅವರಿಗಾಗಿ
ಅವರಿಗಿಂತ
ಅವರಿಗೆ
ಅವರಿಬ್ಬರಿಗೂ
ಅವರಿಬ್ಬರೂ
ಅವರಿವರಂತೆ
ಅವರೀರ್ವರಲ್ಲಿ
ಅವರು
ಅವರುಹೇಳಿದ್ದಾರೆ
ಅವರೂ
ಅವರೆಗೆರೆಯ
ಅವರೆಲ್ಲರ
ಅವರೇ
ಅವರೊಡನೆ
ಅವಲಂಬಿಸಿತ್ತು
ಅವಲಂಬಿಸಿದ್ದಿತು
ಅವಲೋಕನ
ಅವಲೋಕಿಸಿದಾಗ
ಅವಳ
ಅವಶೇಷಗಳಿಂದ
ಅವಶೇಷಗಳಿವೆ
ಅವಶೇಷಗಳು
ಅವಸಾನದ
ಅವಾರ್ಯವೀರ್ಯನಾದ
ಅವು
ಅವುಗಳ
ಅವುಗಳನ್ನು
ಅವುಗಳಲ್ಲಿ
ಅವುಗಳಿಗೆ
ಅವುಗಳು
ಅವುಬಳರಾಜಯ್ಯದೇವ
ಅವೆಲ್ಲಾ
ಅವ್ವೆಯರಕೆರೆ
ಅವ್ವೆಯರಾಣೆ
ಅವ್ವೇರಹಳ್ಳಿ
ಅಶರೀರವಾಣಿಯಾಯಿತು
ಅಶೇಷ
ಅಶೇಷರಾಜ್ಯಭಾರ
ಅಶ್ವಪತಿ
ಅಶ್ವಸೇನೆಯು
ಅಶ್ವಾರೋಹಿ
ಅಷ್ಟಗ್ರಾಮ
ಅಷ್ಟಗ್ರಾಮಗಳ
ಅಷ್ಟಗ್ರಾಮದ
ಅಷ್ಟದಿಕ್ಕುರಾಯ
ಅಷ್ಟದಿಕ್ಪಾಲಕರ
ಅಷ್ಟವಿಧಾರ್ಚನೆಗೆ
ಅಷ್ಟಾದಶಪ್ರಧಾನರಲ್ಲಿ
ಅಷ್ಟೇ
ಅಸಮ
ಅಸರಿಸ್ವಯಂಭು
ಅಸರು
ಅಸವಯ್ಯನುಂ
ಅಸಿವರದಲಿ
ಅಸುನೀಗಿದಂತೆ
ಅಸೂಯೆ
ಅಸೆಂಬ್ಲಿಎಂದೂ
ಅಸೋಫನಿದ್ದನು
ಅಸೋಫಿಗಳಾಗಿ
ಅಸೋಫಿಗೆ
ಅಸೋಫ್
ಅಸ್ಕರ್
ಅಸ್ತಮಾನವಾದಾಗ
ಅಸ್ತಿತ್ವ
ಅಸ್ತಿತ್ವಕ್ಕೆ
ಅಸ್ತಿತ್ವದಲ್ಲಿತ್ತು
ಅಸ್ತಿತ್ವದಲ್ಲಿದ್ದ
ಅಸ್ತಿತ್ವದಲ್ಲಿದ್ದವು
ಅಸ್ತಿತ್ವದಲ್ಲಿದ್ದು
ಅಸ್ತಿತ್ವದಲ್ಲಿರುವ
ಅಸ್ತಿತ್ವದಲ್ಲಿವೆ
ಅಸ್ತಿತ್ವವನ್ನು
ಅಸ್ತಿಯನ್ನು
ಅಸ್ಥಿರತೆಯನ್ನು
ಅಸ್ಪಷ್ಟವಾಗಿವೆ
ಅಸ್ಯ
ಅಹುಬಳ
ಅಹುಬಳದೇವರಾಜಯ್ಯದೇವ
ಅಹುಬಳರಾಜಯ್ಯನಿಗೆ
ಅಹೊಬಲದೇವರಾಜಯ್ಯನು
ಅಹೋಬಲ
ಅಹೋಬಲದೇವ
ಅಹೋಬಲದೇವಗಳ
ಅಹೋಬಲದೇವನ
ಅಹೋಬಲದೇವರಾಜಯ್ಯ
ಅಹೋಬಲಯ್ಯನ
ಅಹೋಬಲವಾಡಿ
ಅಹೋಬಳ
ಅಹೋಬಳಪುರವೆಂಬ
ಅಹೋಬಳರಾಜನ
ಆ
ಆಂಗೀರಸ
ಆಂಗ್ಲ
ಆಂಗ್ಲಭಾಷೆಯಲ್ಲಿ
ಆಂಗ್ಲಭಾಷೆಯಲ್ಲೂ
ಆಂಗ್ಲೋ
ಆಂಜನೇಯ
ಆಂಜನೇಯನ
ಆಂಡಾನ್
ಆಂಧ್ರ
ಆಂಧ್ರನಾಡಿನಲ್ಲಿ
ಆಂಧ್ರಪ್ರದೇಶದ
ಆಂಧ್ರರಾಜಮದಗಜಗಳಿಗೆ
ಆಕರಗಳು
ಆಕೆಯ
ಆಕ್ರಮಣ
ಆಕ್ರಮಣಕ್ಕೆ
ಆಕ್ರಮಣಗಳನ್ನು
ಆಕ್ರಮಣದ
ಆಕ್ರಮಣದಲ್ಲಿ
ಆಕ್ರಮಣನಡೆಸಲು
ಆಕ್ರಮಣವನ್ನು
ಆಕ್ರಮಣವಾಗಿರಬಹುದು
ಆಕ್ರಮಿತವಾಗಿ
ಆಕ್ರಮಿಸಿ
ಆಕ್ರಮಿಸಿಕೊಂಡನು
ಆಕ್ರಮಿಸಿಕೊಂಡರು
ಆಕ್ರಮಿಸಿಕೊಂಡು
ಆಕ್ರಮಿಸಿದ
ಆಕ್ರಮಿಸಿದನು
ಆಕ್ರಮಿಸಿದಮೇಲೂ
ಆಕ್ರಮಿಸಿದ್ದು
ಆಗ
ಆಗಂತುಕ
ಆಗತಾನೆ
ಆಗಬೇಕೆಂದು
ಆಗವಹಾಳ
ಆಗಸ್ಟ್
ಆಗಸ್ಟ್ನಿಂದ
ಆಗಾಗ್ಗೆ
ಆಗಿ
ಆಗಿತ್ತು
ಆಗಿತ್ತೆಂದು
ಆಗಿದೆ
ಆಗಿದ್ದ
ಆಗಿದ್ದನು
ಆಗಿದ್ದನೆಂದು
ಆಗಿದ್ದರಿಂದ
ಆಗಿದ್ದರು
ಆಗಿದ್ದರೂ
ಆಗಿದ್ದರೆಂದು
ಆಗಿದ್ದಲ್ಲಿ
ಆಗಿದ್ದಾಗ
ಆಗಿದ್ದಾನೆ
ಆಗಿದ್ದಾನೆಂದು
ಆಗಿದ್ದಿರಬಹುದು
ಆಗಿದ್ದು
ಆಗಿನ
ಆಗಿನ್ನೂ
ಆಗಿರಬಹದು
ಆಗಿರಬಹುದು
ಆಗಿರಬಹುದೆಂದು
ಆಗಿರಲಿಲ್ಲ
ಆಗಿರುತ್ತಾನೆ
ಆಗಿರುತ್ತಾನೆಂದು
ಆಗಿರುತ್ತಿದ್ದರು
ಆಗಿರುವ
ಆಗಿರುವಂತೆ
ಆಗಿರುವುದನ್ನು
ಆಗಿರುವುದು
ಆಗಿವೆ
ಆಗಿಹೋದರೆಂದೂ
ಆಗುತ್ತದ
ಆಗುತ್ತದೆ
ಆಗುವುದಿಲ್ಲ
ಆಗ್ನೇಯ
ಆಗ್ನೇಯದಲ್ಲಿ
ಆಚನಹಳ್ಳಿ
ಆಚಮಂಗೆ
ಆಚಮಆಚಮ್ಮ
ಆಚಮನು
ಆಚರಾಜ
ಆಚರಿಸಿ
ಆಚರಿಸಿದ
ಆಚರಿಸಿರಬಹುದು
ಆಚಾಂಬಿಕೆ
ಆಚಾಂಬಿಕೆಗೆ
ಆಚಾಂಬಿಕೆಯ
ಆಚಾರ್ಯ
ಆಚಾರ್ಯರ
ಆಚಿಕಬ್ಬೆ
ಆಚಿಯಕ್ಕನು
ಆಚೆ
ಆಚೆಗೆ
ಆಚೆಯೇ
ಆಜ್ಞಾಪಿಸಿದನೆಂದಿದೆ
ಆಜ್ಞಾಪಿಸಿದನೆಂದು
ಆಜ್ಞಾಪಿಸಿದವನು
ಆಜ್ಞಾಪಿಸುವ
ಆಜ್ಞೆ
ಆಜ್ಞೆಯ
ಆಜ್ಞೆಯಂತೆ
ಆಠವಣೆಯ
ಆಡಳಿಗಾರನನ್ನಾಗಿರಾಜ್ಯಪಾಲ
ಆಡಳಿತ
ಆಡಳಿತಕ್ಕೆ
ಆಡಳಿತಗಾರರನ್ನು
ಆಡಳಿತಗಾರರು
ಆಡಳಿತದ
ಆಡಳಿತದಲ್ಲಿ
ಆಡಳಿತದಲ್ಲಿದ್ದ
ಆಡಳಿತದಲ್ಲಿದ್ದರೆಂದು
ಆಡಳಿತದಲ್ಲೂ
ಆಡಳಿತನವನ್ನು
ಆಡಳಿತವನ್ನು
ಆಡಳಿತವನ್ನೂ
ಆಡಳಿತವರ್ಷದಲ್ಲಿ
ಆಡಳಿತವಿಭಾಗವಾಗಿತ್ತೆಂದು
ಆಡಳಿತವೆಲ್ಲವೂ
ಆಡಳಿತವ್ಯವಹಾರಗಳು
ಆಡಳಿತಸೂತ್ರಗಳನ್ನೂ
ಆಡಳಿತಾಧಿಕಾರಿ
ಆಡಳಿತಾಧಿಕಾರಿಗಳ
ಆಡಳಿತಾಧಿಕಾರಿಗಳೊಡನೆ
ಆಡಳಿತಾವಧಿಯಲ್ಲಿ
ಆಡು
ಆಡುಂಬೊಲವಾದ
ಆಣೆ
ಆತ
ಆತಕೂರು
ಆತಕೂರುರಲ್ಲಿ
ಆತನ
ಆತನನ್ನು
ಆತನಿಗಿರಲಿಲ್ಲ
ಆತನಿಗೆ
ಆತನು
ಆತನೇ
ಆತಿಶ್
ಆತಿಶ್ಖಾನ್
ಆತೂರು
ಆತ್ಕೂರುಆತಕೂರು
ಆತ್ಮಭಕ್ತಿಯಿಂದ
ಆತ್ಮಹತ್ಯೆ
ಆತ್ಮಾಗ್ರಜ
ಆತ್ಮಾರ್ಪಣೆ
ಆತ್ರೇಯ
ಆತ್ರೇಯಗೋತ್ರದ
ಆತ್ರೇಯಸ
ಆಥವಾ
ಆದ
ಆದಕಾರಣ
ಆದರು
ಆದರೂ
ಆದರೆ
ಆದಾಯದಲ್ಲಿ
ಆದಾಯವನ್ನು
ಆದಾಯವಿರುವ
ಆದಾಯವುಳ್ಳ
ಆದಿಗುಂಜನರಸಿಂಹದೇವರಿಗೆ
ಆದಿಗುಂಜೆಯ
ಆದಿಚುಂಚನಗಿರಿ
ಆದಿಚುಂಚನಗಿರಿಒಂದು
ಆದಿಚುಂಚನಗಿರಿಯ
ಆದಿದೇವನ
ಆದಿಪುರಾಣದಲ್ಲಿ
ಆದಿಯಮ
ಆದಿಯಮನನ್ನು
ಆದಿಯಮನು
ಆದಿಯಮನೋಡಿದೋಟ
ಆದಿಲ್ನು
ಆದಿವರಾಹನಿಗೂ
ಆದಿವಾರ
ಆದಿಸಿಂಗೆಯ
ಆದಿಸಿಂಗೆಯದಣ್ಣಾಯಕರುಮ
ಆದುದರಿಂದ
ಆದುದರಿಂದಲೇ
ಆದುರಿಂದ
ಆದೇಶ
ಆದೇಶದ
ಆದೇಶದಂತೆ
ಆದೇಶದಿಂದ
ಆದೇಶವಾಗುವುದು
ಆದ್ಯತೆ
ಆಧರಿಸಿ
ಆಧಾರ
ಆಧಾರಗಳನ್ನು
ಆಧಾರಗಳಲ್ಲಿ
ಆಧಾರಗಳಿಂದ
ಆಧಾರಗಳಿಲ್ಲ
ಆಧಾರಗಳು
ಆಧಾರಗಳೂ
ಆಧಾರದ
ಆಧಾರದಿಂದ
ಆಧಾರವಾಗಿ
ಆಧಾರವಾಗಿಟ್ಟುಕೊಂಡು
ಆಧಾರವೂ
ಆಧಿಕ್ಯವನ್ನು
ಆಧಿಪತ್ಯದಲ್ಲಿ
ಆಧಿಪತ್ಯವನ್ನು
ಆಧುನಿಕ
ಆನಂತರ
ಆನಂದ
ಆನಂದಾನ್ಪುಳ್ಳೆ
ಆನಂದೂರಿನಲ್ಲಿ
ಆನತರಾಗುವಂತೆ
ಆನೆ
ಆನೆಕೆರೆ
ಆನೆಗನಕೆರಿಆನೆಕೆರೆ
ಆನೆಗಳ
ಆನೆಗಳನ್ನು
ಆನೆಗಳು
ಆನೆಗೊಂದಿಯಲ್ಲಿ
ಆನೆಬಸದಿಗೆ
ಆನೆಬಸದಿಯ
ಆನೆಬಸದಿಯು
ಆನೆಯ
ಆನೆಯಂ
ಆನೆಯನು
ಆನೆಯನ್ನು
ಆನೆಯು
ಆನೆಯೊಡನೆ
ಆನೆವಾಳ
ಆನೆಸಲಗ
ಆನೆಸಾಸಲು
ಆನೆಹಾಳು
ಆಪಸ್ತಂಭ
ಆಪಸ್ತಂಭಸೂತ್ರದ
ಆಪ್ತ
ಆಪ್ತರ
ಆಪ್ತಸಹಾಯಕರ
ಆಪ್ತೇಷ್ಟರಲ್ಲಿ
ಆಪ್ತೇಷ್ಟರು
ಆಭರಣ
ಆಭರಣಗಳನ್ನು
ಆಮೇಲೆ
ಆಯಆಯತ
ಆಯಕಟ್ಟಿನ
ಆಯಗಳನ್ನು
ಆಯಗಾರರು
ಆಯತದ
ಆಯವನ್ನು
ಆಯಸ್ಸು
ಆಯಾ
ಆಯಿತು
ಆಯಿತೆಂದು
ಆಯಿದು
ಆಯಿದುಮೊತ್ತದ
ಆಯುಧ
ಆಯುಧಗಳನ್ನು
ಆಯುಧಗಳು
ಆಯುರಾರೋಗ್ಯ
ಆಯ್ಕೆ
ಆಯ್ಕೆಮಾಡಿಕೊಳ್ಳುತ್ತಿದ್ದರೆಂದು
ಆಯ್ಕೆಯಾಗುತ್ತಿದ್ದರು
ಆಯ್ಕೆಯಾದವರು
ಆಯ್ಕೆಯಾದವರೆಂದು
ಆಯ್ದುಕೊಂಡಿದೆ
ಆರಂಭ
ಆರಂಭಗೊಳ್ಳುತ್ತದೆ
ಆರಂಭದ
ಆರಂಭದಲ್ಲಿ
ಆರಂಭಲ್ಲನು
ಆರಂಭಲ್ಲವನು
ಆರಂಭವಾಗಿದೆ
ಆರಂಭವಾಗಿರುವುದು
ಆರಂಭವಾಗುತ್ತದೆ
ಆರಂಭವಾದುದು
ಆರಂಭವಾಯಿತು
ಆರಂಭವಾಯಿತೆಂದು
ಆರಂಭವಾಯಿತೆಂದೂ
ಆರಂಭವಾಯಿತೆನ್ನಬಹುದು
ಆರಂಭಿಸಿದನು
ಆರಂಭಿಸಿದರು
ಆರಂಭಿಸಿದ್ದನ್ನು
ಆರಣಿ
ಆರಣಿಯ
ಆರಣಿಯು
ಆರಣಿಸಯಸ್ಥಳದ
ಆರನೆಯ
ಆರನೇ
ಆರಾಧಿಸಿ
ಆರಾಧಿಸುತ್ತಿದ್ದ
ಆರಾಧ್ಯ
ಆರಾಧ್ಯದೈವ
ಆರಿದವಾಳಿಕೆಯ
ಆರಿಸಿಕೊಂಡನು
ಆರಿಸಿಕೊಳ್ಳುತ್ತಿದ್ದರು
ಆರು
ಆರುಗ್ರಾಮಗಳನ್ನು
ಆರುತಲೆಮಾರುಗಳ
ಆರೆಂಟು
ಆರೋಗಣೆಯನ್ನು
ಆರೋಪಿಸಲಾಗಿರುವುದನ್ನು
ಆರೋಪಿಸಿಲ್ಲ
ಆರೋಪಿಸುವುದು
ಆರ್ಎಸ್ಪಂಚಮುಖಿ
ಆರ್ಕಾಟಿನ
ಆರ್ಕಾಟಿನವನಾದ
ಆರ್ಕಿಯಾಲಾಜಿಕಲ್
ಆರ್ಕಿಯೋಲಜಿಕಲ್
ಆರ್ಥರ್
ಆರ್ಥಿಕ
ಆರ್ಯಮಂಡುನದ
ಆರ್ಶೇಷಶಾಸ್ತ್ರಿಯವರ
ಆಲಂಬಾಡಿ
ಆಲಗಾವುಂಡ
ಆಲತಿ
ಆಲತ್ತೂರನಿಱಿದು
ಆಲತ್ತೂರಿನ
ಆಲದಹಳ್ಳಿ
ಆಲದಹಳ್ಳಿಯ
ಆಲಪ್ಪ
ಆಲುಗೋಡನ್ನು
ಆಲುಗೋಡು
ಆಲುಗೋಡುರಾಜ್ಯ
ಆಲೂರಿನ
ಆಲೂರಿನವರಿಗೂ
ಆಲೂರು
ಆಲೇನಹಳ್ಳಿ
ಆಲ್ಗೋಡು
ಆಳತ್ತಿದ್ದನೆಂದು
ಆಳದೇ
ಆಳಲು
ಆಳವಾಗಿ
ಆಳವಾದ
ಆಳಿಕೊಂಡು
ಆಳಿದ
ಆಳಿದನು
ಆಳಿದನೆಂದು
ಆಳಿದನೆಂದೂ
ಆಳಿದರು
ಆಳಿದರೆಂದೂ
ಆಳುಗೋಡೀ
ಆಳುಗೋಡು
ಆಳುತಿದ್ದನು
ಆಳುತಿದ್ದರು
ಆಳುತಿದ್ದಾಗ
ಆಳುತ್ತಾ
ಆಳುತ್ತಿದ್ದ
ಆಳುತ್ತಿದ್ದಂತೆ
ಆಳುತ್ತಿದ್ದನು
ಆಳುತ್ತಿದ್ದನೆಂದಿದೆ
ಆಳುತ್ತಿದ್ದನೆಂದು
ಆಳುತ್ತಿದ್ದನೆಂದೂ
ಆಳುತ್ತಿದ್ದನೆಂಬುದು
ಆಳುತ್ತಿದ್ದರು
ಆಳುತ್ತಿದ್ದರೂ
ಆಳುತ್ತಿದ್ದರೆಂದು
ಆಳುತ್ತಿದ್ದರೆಂದೂ
ಆಳುತ್ತಿದ್ದರೆಂಬ
ಆಳುತ್ತಿದ್ದಳು
ಆಳುತ್ತಿದ್ದಳೆಂದು
ಆಳುತ್ತಿದ್ದವನು
ಆಳುತ್ತಿದ್ದವರಿಗೆ
ಆಳುತ್ತಿದ್ದವರೆಂದರೆ
ಆಳುತ್ತಿದ್ದಾಗ
ಆಳುತ್ತಿದ್ದಿರಬಹುದು
ಆಳುತ್ತಿದ್ದು
ಆಳುತ್ತಿದ್ದುದನ್ನು
ಆಳುತ್ತಿದ್ದುದರಿಂದ
ಆಳುತ್ತಿದ್ದುದು
ಆಳುತ್ತಿರಲು
ಆಳುತ್ತಿರುತ್ತಾನೆ
ಆಳುವ
ಆಳುವಖೇಡ
ಆಳ್ತನವನ್ನು
ಆಳ್ದನ
ಆಳ್ವಕೆ
ಆಳ್ವಖೇಡ
ಆಳ್ವಾರ್
ಆಳ್ವಿಕೆ
ಆಳ್ವಿಕೆಗೆ
ಆಳ್ವಿಕೆಯ
ಆಳ್ವಿಕೆಯನ್ನು
ಆಳ್ವಿಕೆಯನ್ನೇ
ಆಳ್ವಿಕೆಯಲ್ಲಿ
ಆಳ್ವಿಕೆಯಲ್ಲೂ
ಆಳ್ವಿಕೆಯೇ
ಆವರಣದಲ್ಲಿಯೇ
ಆವರಣದಲ್ಲಿರುವ
ಆವೃತವಾದ
ಆಶಾದಾಯಕ
ಆಶ್ಚರ್ಯಕರ
ಆಶ್ಚರ್ಯಕರವಾಗಿದೆ
ಆಶ್ರಯ
ಆಶ್ರಯದಲ್ಲಿ
ಆಶ್ರಯವರ್ತಿಯಾಗಿದ್ದುಕೊಂಡು
ಆಶ್ರಯಿಸಿದನೆಂದು
ಆಶ್ರಯಿಸಿದ್ದು
ಆಶ್ರಿತಜನಕಲ್ಪವೃಕ್ಷ
ಆಶ್ರಿತನಾಗಿದ್ದು
ಆಶ್ವಲಾಯನ
ಆಶ್ವಲಾಯನಸೂತ್ರದ
ಆಶ್ವೀಜ
ಆಸಂದಿ
ಆಸಂದಿನಾಡ
ಆಸಂಧಿನಾಡ
ಆಸಂನ್ನ
ಆಸನ್ನ
ಆಸೆ
ಆಸೆಮಾಡುವ
ಆಸೇತುಮೇರುಪರ್ಯಂತಂ
ಆಸ್ತಿ
ಆಸ್ತಿಯನ್ನು
ಆಸ್ತಿಹಂಚಿಕೆ
ಆಸ್ಥಾನ
ಆಸ್ಥಾನಕವಿಯಾಗಿದ್ದ
ಆಸ್ಥಾನಕ್ಕೆ
ಆಸ್ಥಾನಜಗಜೆಟಿ
ಆಸ್ಥಾನದ
ಆಸ್ಥಾನದಲ್ಲಿ
ಆಸ್ಥಾನದಲ್ಲಿದ್ದ
ಆಸ್ಥಾನದಲ್ಲಿದ್ದು
ಆಸ್ಥಾನವನ್ನು
ಆಸ್ಥಾಯಿಕಾ
ಆಸ್ಪದ
ಆಹಾರ
ಆಹಾರದಾನಕ್ಕಾಗಿ
ಆಹಾರದಾನಕ್ಕೆ
ಆಹಾರಮಂಡಲಭುಕ್ತಿವಿಷಯದೇಶ
ಆಹಾರಾಭಯ
ಆಹಾರಾಭಯನುಂ
ಆಹಾರಾಭಯಭೈಷಜ್ಯಶಾಸ್ತ್ರವಿನೋದನುಂ
ಆಹ್ವಾನಿಸುತ್ತಿದ್ದನು
ಇ
ಇಂಗಲಗುಪ್ಪೆ
ಇಂಗಲಗುಪ್ಪೆಯ
ಇಂಗ್ಲಿಷ್
ಇಂತಹ
ಇಂತಿವರ
ಇಂತೀ
ಇಂಥ
ಇಂದಿಗೂ
ಇಂದಿನ
ಇಂದು
ಇಂದುಕೊಟ್ಟು
ಇಂದ್ರನಂತೆ
ಇಂದ್ರನಾಗಿದ್ದಾನೆಂದು
ಇಂದ್ರನಿಗೆ
ಇಂದ್ರನು
ಇಂದ್ರರಾಜನ
ಇಂದ್ರವರ್ಮನೆಂಬ
ಇಂಮಡಿದೇವ
ಇಕ್ಕಿದಂಥಾ
ಇಕ್ಕಿಸಿದೆವು
ಇಕ್ಕೇರಿಯ
ಇಗ್ಗಲೂರು
ಇಚ್ಚಿಸದೇ
ಇಜ್ಜಲಘಟ್ಟವೆಂಬ
ಇಜ್ಜಲನ್ನು
ಇಟಗಿ
ಇಟ್ಟನು
ಇಟ್ಟರೆಂದೂ
ಇಟ್ಟಾಡಿ
ಇಟ್ಟಿಗೆಯಲ್ಲಿ
ಇಟ್ಟಿದ್ದನು
ಇಟ್ಟಿದ್ದಾನೆಂದು
ಇಟ್ಟಿರಬೇಕಾಗುತ್ತಿತ್ತು
ಇಟ್ಟುಕೊಂಡರು
ಇಟ್ಟುಕೊಂಡರೂ
ಇಟ್ಟುಕೊಂಡರೆ
ಇಟ್ಟುಕೊಂಡರೆಂದು
ಇಟ್ಟುಕೊಂಡಿದ್ದನೆಂದೂ
ಇಟ್ಟುಕೊಂಡಿದ್ದರು
ಇಟ್ಟುಕೊಂಡಿದ್ದರೆಂದು
ಇಟ್ಟುಕೊಂಡು
ಇಟ್ಟುಕೊಳ್ಳಬಹುದು
ಇಟ್ಟುಕೊಳ್ಳುತ್ತಿದ್ದರು
ಇಟ್ಟುಕೊಳ್ಳುತ್ತಿದ್ದರೆಂದು
ಇಡಗೂರು
ಇಡಲಾಗುತ್ತಿತ್ತು
ಇಡೀ
ಇಡುಗೂರ
ಇಡುಗೂರು
ಇಡುತುರೈನಾಟ್ಟು
ಇಡುತ್ತಾನೆ
ಇಡುತ್ತಿದ್ದುದು
ಇಡುದುರೈ
ಇಡುವ
ಇಡೆಯನಾಡು
ಇಡೈತುರೈನಾಡನ್ನು
ಇಡೈತುರೈನಾಡುಕಾವೇರಿ
ಇಡೈಮುನೂರುಇಡೈಕುನ್ದನಾಡು
ಇತರ
ಇತರರನ್ನು
ಇತರೆ
ಇತರೆಉಳಿದವರು
ಇತಿ
ಇತಿಹಾಸ
ಇತಿಹಾಸಕಾರರ
ಇತಿಹಾಸಕಾರರು
ಇತಿಹಾಸಕ್ಕಿಂತ
ಇತಿಹಾಸಕ್ಕೆ
ಇತಿಹಾಸಗಳನ್ನು
ಇತಿಹಾಸದ
ಇತಿಹಾಸದಲ್ಲಿ
ಇತಿಹಾಸದಿಂದ
ಇತಿಹಾಸಪ್ರಸಿದ್ಧ
ಇತಿಹಾಸವನ್ನು
ಇತಿಹಾಸವಿದ್ವಾಂಸರು
ಇತಿಹಾಸವು
ಇತ್ತ
ಇತ್ತಣ
ಇತ್ತೀಚಿನವರೆಗೂ
ಇತ್ತೀಚೆಗೆ
ಇತ್ತು
ಇತ್ತೆಂದು
ಇತ್ತೆಂಬುದು
ಇತ್ತೇ
ಇತ್ಯಾದಿ
ಇತ್ಯಾದಿಯಾಗಿ
ಇದಕ್ಕಾಗಿ
ಇದಕ್ಕೂ
ಇದಕ್ಕೆ
ಇದನ್ನು
ಇದನ್ನೂ
ಇದನ್ನೇ
ಇದರ
ಇದರಲ್ಲಿ
ಇದರಲ್ಲಿದೆ
ಇದರಲ್ಲಿದ್ದು
ಇದರಿಂದ
ಇದರಿಂದಾಗಿ
ಇದಾಗಿದೆ
ಇದಾಗಿವೆ
ಇದಾದ
ಇದು
ಇದುಳೆಯನ್ನು
ಇದುವರೆಗಿನ
ಇದುವರೆಗೆ
ಇದೂ
ಇದೆ
ಇದೆಯೇ
ಇದೇ
ಇದೊಂದು
ಇದೊಂದೇ
ಇದ್ದ
ಇದ್ದಂತಹ
ಇದ್ದಂತೆ
ಇದ್ದಂತೆಯೂ
ಇದ್ದಕ್ಕಿದ್ದಹಾಗೆ
ಇದ್ದನಂತೆ
ಇದ್ದನು
ಇದ್ದನೆಂದು
ಇದ್ದನೆಂದೂ
ಇದ್ದನೆನ್ನುವುದರ
ಇದ್ದರು
ಇದ್ದರೂ
ಇದ್ದರೆ
ಇದ್ದರೆಂದು
ಇದ್ದರೆಂದೂ
ಇದ್ದಳು
ಇದ್ದವು
ಇದ್ದವೆಂದು
ಇದ್ದಹಾಗೆ
ಇದ್ದಾಗ
ಇದ್ದಾರೆ
ಇದ್ದಿತು
ಇದ್ದಿತೆಂದು
ಇದ್ದಿತೆಂದೂ
ಇದ್ದಿತೆಂಬ
ಇದ್ದಿತೆಂಬುದು
ಇದ್ದಿತೇ
ಇದ್ದಿರಬಹುದಾದ
ಇದ್ದಿರಬಹುದು
ಇದ್ದಿರಬೇಕು
ಇದ್ದು
ಇದ್ದುಕೊಂಡು
ಇದ್ದುದನ್ನು
ಇದ್ದುದರಿಂದ
ಇದ್ದುದು
ಇದ್ದುದೇ
ಇನನ
ಇನಾಮಾಗಿ
ಇನಿಗೆ
ಇನ್ದರ
ಇನ್ನಿಬ್ಬರು
ಇನ್ನು
ಇನ್ನೂ
ಇನ್ನೂರಹನ್ನೊಂದ
ಇನ್ನೂರು
ಇನ್ನೂರೆಂಬತ್ತು
ಇನ್ನೊಂದು
ಇನ್ನೊಬ್ಬ
ಇನ್ನೊಬ್ಬರಿಗೆ
ಇನ್ನೊಬ್ಬಳು
ಇಪ್ಪತ್ತನಾಲ್ಕು
ಇಪ್ಪತ್ತು
ಇಪ್ಪತ್ತೈದು
ಇಪ್ಪತ್ತೊಂದನೆಯ
ಇಬ್ಬರ
ಇಬ್ಬರನ್ನೂ
ಇಬ್ಬರಿಗೂ
ಇಬ್ಬರಿಗೆ
ಇಬ್ಬರು
ಇಬ್ಬರೂ
ಇಭಾಟು
ಇಮ್ಮಡಿ
ಇಮ್ಮಡಿದೇವನದೇರಾಯನ
ಇಮ್ಮಡಿದೇವರಾಯನನ್ನು
ಇಮ್ಮಡಿದೇವರಾಯನೆಂಬ
ಇಮ್ಮಡಿಬಲ್ಲಾಳನ
ಇಮ್ಮಡಿಬಲ್ಲಾಳನಿಂದ
ಇಮ್ಮಡಿಬೀರ
ಇಮ್ಮಡಿಬೂತುಗನು
ಇಮ್ಮಡಿಯಾಯಿತು
ಇಮ್ಮಡಿರಾಯ
ಇಮ್ಮಡಿರಾವುತ್ತರಾಯ
ಇರಣ್ಡುಕರೈ
ಇರಬಹುದು
ಇರಬಹುದೆಂದು
ಇರಬೇಕಾಗಿತ್ತೆಂಬುದನ್ನು
ಇರಲಿಲ್ಲ
ಇರಲಿಲ್ಲವೆಂದು
ಇರಲಿಲ್ಲವೆಂದೂ
ಇರಲಿಲ್ಲವೆಂಬುದು
ಇರಲು
ಇರಲೂಬಹುದು
ಇರಲೇ
ಇರಾಮನ್
ಇರಿದಂ
ಇರಿದನು
ಇರಿದನುಯುದ್ಧಮಾಡಿದನು
ಇರಿದನೆಂದು
ಇರಿದು
ಇರಿವಬೆಡಂಗ
ಇರಿಸಲಾಗಿತ್ತು
ಇರಿಸಿಕೊಂಡಿದ್ದನೆಂದು
ಇರಿಸಿದ್ದನೆಂದು
ಇರಿಸಿದ್ದರು
ಇರುಂಗೋಳನ
ಇರುಂಗೋಳನಕೋಟೆ
ಇರುಂಗೋಳನೂ
ಇರುಗಂಗಣ್ಣ
ಇರುಗಪ್ಪನು
ಇರುತಿದ್ದನೆಂದು
ಇರುತ್ತಿತತ್ತೆಂದು
ಇರುತ್ತಿತ್ತು
ಇರುತ್ತಿತ್ತೆಂದು
ಇರುತ್ತಿದ್ದ
ಇರುತ್ತಿದ್ದನು
ಇರುತ್ತಿದ್ದನೆಂದು
ಇರುತ್ತಿದ್ದರು
ಇರುತ್ತಿದ್ದರೆಂದು
ಇರುತ್ತಿದ್ದುದು
ಇರುಮುಡಿಚೋಳ
ಇರುವ
ಇರುವಂತೆ
ಇರುವುದನ್ನು
ಇರುವುದನ್ನೂ
ಇರುವುದರಿಂದ
ಇರುವುದಿಲ್ಲ
ಇರುವುದು
ಇರ್ರಾಜೇಂದ್ರ
ಇಲಾಖೆ
ಇಲಾಖೆಗಳ
ಇಲಾಖೆಗಳನ್ನು
ಇಲಾಖೆಗಳಿಗೆ
ಇಲಾಖೆಗೆ
ಇಲಾಖೆಯ
ಇಲಾಖೆಯನ್ನು
ಇಲಾಖೆಯು
ಇಲ್ಲ
ಇಲ್ಲದಿರಲು
ಇಲ್ಲದಿರುವ
ಇಲ್ಲದಿರುವಂಶ
ಇಲ್ಲದಿರುವುದು
ಇಲ್ಲದಿಲ್ಲ
ಇಲ್ಲದೇ
ಇಲ್ಲವೇ
ಇಲ್ಲಾದೆರೆ
ಇಲ್ಲಿ
ಇಲ್ಲಿಂದ
ಇಲ್ಲಿಗೆ
ಇಲ್ಲಿದ್ದ
ಇಲ್ಲಿದ್ದನೆಂದು
ಇಲ್ಲಿನ
ಇಲ್ಲಿಯೂ
ಇಲ್ಲಿಯೇ
ಇಲ್ಲಿರುವ
ಇಲ್ಲಿಲ್ಲ
ಇಲ್ಲೇ
ಇಲ್ಲೊಂದು
ಇಳಿಯಿತೆಂದು
ಇಳೆಯ
ಇಳೈಯಾಳ್ವಾನ್
ಇವನ
ಇವನದೇ
ಇವನನಿಗೆ
ಇವನನ್ನು
ಇವನಿಂದಲೇ
ಇವನಿಗೂ
ಇವನಿಗೆ
ಇವನಿಗೇ
ಇವನಿರಬೇಕೆಂದು
ಇವನು
ಇವನುಮರಿಯಾನೆ
ಇವನೂ
ಇವನೇ
ಇವರ
ಇವರಗಳು
ಇವರನ್ನು
ಇವರನ್ನೂ
ಇವರನ್ನೆಲಾ
ಇವರನ್ನೇ
ಇವರಲ್ಲಿ
ಇವರಲ್ಲೂ
ಇವರಲ್ಲೇ
ಇವರವನ್ನು
ಇವರಾರೂ
ಇವರಿಂದ
ಇವರಿಗಿಲ್ಲದಿರುವುದು
ಇವರಿಗೂ
ಇವರಿಗೆ
ಇವರಿಗೆಲ್ಲಾ
ಇವರಿಗೇ
ಇವರಿಬ್ಬರ
ಇವರಿಬ್ಬರಲ್ಲಿ
ಇವರಿಬ್ಬರಿಗೆ
ಇವರಿಬ್ಬರು
ಇವರಿಬ್ಬರೂ
ಇವರು
ಇವರುಗಳ
ಇವರುಗಳನ್ನು
ಇವರುಗಳಿಗೆ
ಇವರುಗಳು
ಇವರುಗಳೂ
ಇವರೂ
ಇವರೆಲ್ಲರೂ
ಇವರೆಲ್ಲಾ
ಇವರೇ
ಇವರೊಡಗೂಡಿ
ಇವರೊಳಗಾದ
ಇವಳ
ಇವಳು
ಇವು
ಇವುಗಳ
ಇವುಗಳನೂ
ಇವುಗಳನ್ನು
ಇವುಗಳಲ್ಲಿ
ಇವುಗಳಿಗೂ
ಇವುಗಳಿಗೆ
ಇವುಗಳು
ಇವುಗಳೆಲ್ಲವನ್ನೂ
ಇವುಗಳೆಲ್ಲಾ
ಇವುಗಳೇ
ಇವೆ
ಇವೆರಡನ್ನೂ
ಇವೆರಡು
ಇವೆರಡೂ
ಇವೆಲ್ಲವನ್ನೂ
ಇವೆಲ್ಲವೂ
ಇವೆಲ್ಲಾ
ಇವೇ
ಇಶಾಮುದ್ರ
ಇಷ್ಟಲ್ಲದೆ
ಇಷ್ಟೂ
ಇಷ್ಟೊಂದು
ಇಸವಿ
ಇಸವಿಯ
ಇಸ್ಲಾಂನ
ಈ
ಈಕೆ
ಈಕೆಯು
ಈಗ
ಈಗಲೂ
ಈಗಾಗಲೇ
ಈಗಿನ
ಈಚೆಗೆ
ಈಡೇರಿದುದರಿಂದ
ಈತ
ಈತನ
ಈತನನ್ನು
ಈತನಿಗೆ
ಈತನು
ಈತನೂ
ಈತನೇ
ಈತನೇನಾನದರೂ
ಈರೀತಿ
ಈರೋಡು
ಈವರೆಗೆ
ಈಶಾನ್ಯ
ಈಶಾನ್ಯದ
ಈಶಾನ್ಯದಲ್ಲಿ
ಈಶ್ವರ
ಈಶ್ವರದೇವ
ಈಶ್ವರನನ್ನು
ಈಶ್ವರಭಕ್ತನಾದರೂ
ಈಶ್ವರಯ್ಯಈಸರಯ್ಯ
ಈಶ್ವರಯ್ಯನ
ಈಶ್ವರಯ್ಯನೂ
ಈಶ್ವರಾಂಕ
ಈಸರಗಂಡನ
ಈಸರಗಂಡನಿಗೆ
ಈಸರಗಂಡನು
ಈಸರಯ್ಯಈಶ್ವರಯ್ಯ
ಈಸರಯ್ಯನ
ಈಸರಯ್ಯನೆಂಬ
ಈಸರಯ್ಯನೇಈಶ್ವರಯ್ಯ
ಉಂಟಾಗಲು
ಉಂಟಾಗಿದೆ
ಉಂಟಾಗಿದ್ದ
ಉಂಟಾಗಿದ್ದು
ಉಂಟಾಗುತ್ತದೆ
ಉಂಟಾದ
ಉಂಟಾಯಿತು
ಉಂಟಾಯಿತೆಂದು
ಉಂಟು
ಉಂಟುಮಾಡುತ್ತವೆ
ಉಂಡಿಗನಹಾಳು
ಉಂಡಿಗೆಯಾಗಿ
ಉಂಬಳಿ
ಉಂಬಳಿಗಳನ್ನೂ
ಉಂಬಳಿಯಾಗಿ
ಉಂಮರಹಳ್ಳಿ
ಉಂಮರಹಳ್ಳಿಉಮ್ಮಡಹಳ್ಳಿ
ಉಕ್ತನಾಗಿದ್ದು
ಉಕ್ತನಾಗಿರುವ
ಉಕ್ತನಾದ
ಉಕ್ತರಾಗಿರುವ
ಉಕ್ತವಾಗಿದೆ
ಉಕ್ತವಾಗಿರುವ
ಉಕ್ತವಾದ
ಉಗತ್ತಿ
ಉಗಮದ
ಉಗಮವಾಗಿ
ಉಚ್ಚಂಗಿ
ಉಚ್ಚಂಗಿಗಳನ್ನು
ಉಚ್ಚರಿಸುತ್ತಿದರು
ಉಚ್ಛಂಗಿ
ಉಚ್ಛಂಗಿಯೋ
ಉಚ್ಛರಿಸುತ್ತಾರೆ
ಉಚ್ಛಾರಣಾ
ಉಡಿಯಗಾಲ
ಉಡುಗೊರೆಗಳನ್ನು
ಉಡುಪು
ಉಡುವ
ಉಡುವಂಕನಾಡ
ಉಡೈಯಾರ್
ಉತ್ತಮ
ಉತ್ತಮಚೋಳ
ಉತ್ತರ
ಉತ್ತರಕರ್ನಾಟಕದವರೋ
ಉತ್ತರಕರ್ನಾಟದ
ಉತ್ತರಕ್ಕಿರುವ
ಉತ್ತರಕ್ಕೂ
ಉತ್ತರಕ್ಕೆ
ಉತ್ತರದ
ಉತ್ತರದಲ್ಲಿ
ಉತ್ತರದಿಂದ
ಉತ್ತರಭಾಗಗಳನ್ನು
ಉತ್ತರಭಾಗಗಳು
ಉತ್ತರಭಾಗದಲ್ಲಿ
ಉತ್ತರಭಾಗದಲ್ಲಿಯೂ
ಉತ್ತರಭಾಗವನ್ನು
ಉತ್ತರಾಧಿಕಾರಕ್ಕೆ
ಉತ್ತರಾಧಿಕಾರತ್ವವು
ಉತ್ತರಾಧಿಕಾರಿ
ಉತ್ತರಾಧಿಕಾರಿಯ
ಉತ್ತರಾಧಿಕಾರಿಯಾದ
ಉತ್ತರಾರ್ಧದಲ್ಲಿ
ಉತ್ತರಾರ್ಧದಲ್ಲಿಯೇ
ಉತ್ತುಂಗ
ಉತ್ತುಮ
ಉತ್ಪನ್ನದ
ಉತ್ಪ್ರೇಕ್ಷೆಯಿಂದ
ಉತ್ಸವ
ಉತ್ಸವಮಂಟಪ
ಉತ್ಸಾಹದಿಂದ
ಉತ್ಸಾಹಿಯೂ
ಉದಯಗಿರಿ
ಉದಯಗಿರಿಯ
ಉದಯಗಿರಿಯಲ್ಲಿ
ಉದಯಮಯ್ಯ
ಉದಯವಾಯಿತೆಂದು
ಉದಯಾದಿತ್ಯ
ಉದಯಾದಿತ್ಯರು
ಉದಾರ
ಉದಾರವಾಗಿ
ಉದಾರವಾರಿನಿಧಿ
ಉದಾರಿಯೂ
ಉದಾಹಣೆಗಳಿವೆ
ಉದಾಹರಣೆಗಳೂ
ಉದಾಹರಣೆಗೆ
ಉದಾಹರಣೆಯನ್ನು
ಉದಾಹರಣೆಯನ್ನೂ
ಉದಾಹರಣೆಯೂ
ಉದಿತೋದಿತ
ಉದಿಸಿದ
ಉದ್ಗರಿಸುವಂತೆ
ಉದ್ಗಾರವೆತ್ತಿದೆ
ಉದ್ಘಾಟನೆಗೆ
ಉದ್ದಂಡ
ಉದ್ದಕ್ಕೂ
ಉದ್ದೇಶ
ಉದ್ದೇಶದಿಂದ
ಉದ್ದೇಶವೂ
ಉದ್ದೇಶಿಸಿದ್ದ
ಉದ್ಭವ
ಉದ್ಭವನರಸಿಂಹಪುರವೆಂಬ
ಉದ್ಭವವಿಶ್ವನಾಥಪುರಬಾಳಗಂಚಿ
ಉದ್ಭವಸರ್ವಜ್ಞ
ಉದ್ಯಮಗಳಿಗೆ
ಉದ್ಯೋಗಮಲ್ಲನೆನಿಸಿದನು
ಉಧಾರಣ
ಉನ್ನತ
ಉನ್ನತದರ್ಜೆಯ
ಉಪ
ಉಪಕಾರದ
ಉಪಕ್ರಮಿಸಿರಬಹುದೆಂದು
ಉಪಗ್ರಾಮ
ಉಪಗ್ರಾಮಗಳ
ಉಪಗ್ರಾಮಗಳನ್ನು
ಉಪಗ್ರಾಮಗಳಾಗಿ
ಉಪಗ್ರಾಮಗಳಾಗಿದ್ದವು
ಉಪಗ್ರಾಮಗಳಾಗಿದ್ದವೆಂದು
ಉಪಗ್ರಾಮಗಳಾದ
ಉಪಗ್ರಾಮವನ್ನು
ಉಪಟಳ
ಉಪತಾಲ್ಲೂಕನ್ನು
ಉಪದೇಶ
ಉಪದ್ರವದಿಂದ
ಉಪನದಿಗಳಲ್ಲಿ
ಉಪನದಿಗಳಾದ
ಉಪನಯನ
ಉಪನಾಡುಗಳಾಗಿದ್ದವೆಂದು
ಉಪಪಂಗಡವಿದ್ದು
ಉಪಭೋಗಕ್ಕಾಗಿ
ಉಪಯೋಗಕ್ಕಾಗಿ
ಉಪಯೋಗವಾಗಿದೆ
ಉಪಯೋಗಿಸಿಕೊಂಡು
ಉಪಯೋಗಿಸಿದ್ದಾರೆ
ಉಪಯೋಗಿಸುವ
ಉಪಯೋಗಿಸುವವರು
ಉಪವಿಭಾಗಕ್ಕೆ
ಉಪವಿಭಾಗಗಳನ್ನಾಗಿ
ಉಪವಿಭಾಗಗಳಿದ್ದವು
ಉಪವಿಭಾಗಗಳಿದ್ದು
ಉಪವಿಭಾಗದಲ್ಲಿ
ಉಪವಿಭಾಗವಿದ್ದು
ಉಪಶಾಂತವಾಯಿತು
ಉಪಸ್ಥಿತಿ
ಉಬ್ಬು
ಉಬ್ಬುಶಿಲ್ಪಗಳಿವೆ
ಉಬ್ಬುಶಿಲ್ಪವಿದೆ
ಉಭಯ
ಉಭಯತ್ರರೂ
ಉಭಯದೇಸಿ
ಉಭಯನಾಚ್ಚಿಯಾರರುಗಳಿಗೆ
ಉಭಯಬಲಸುಭಟ
ಉಭಯರಾಯ
ಉಭಯವೇದಾಂತಾಚಾರ್ಯ
ಉಭಯಾನ್ವಯ
ಉಮೆಯಕ್ಕನೆಂದೂ
ಉಮ್ಮತೂರ
ಉಮ್ಮತ್ತೂರ
ಉಮ್ಮತ್ತೂರನ್ನು
ಉಮ್ಮತ್ತೂರಿಗೆ
ಉಮ್ಮತ್ತೂರಿನ
ಉಮ್ಮತ್ತೂರಿನಿಂದ
ಉಮ್ಮತ್ತೂರಿನಿಂದಲೂ
ಉಮ್ಮತ್ತೂರು
ಉಮ್ಮತ್ತೂರುಗಳನ್ನು
ಉಯ್ಯಕೊಂಡಪಿಳ್ಳೆ
ಉರದಿದಿರಾನ್ತ
ಉರವಣೆಯನ್ನು
ಉರವಣೆಯಿಂದ
ಉರಿಯುವ
ಉರೋದ್ಗುಕರತೆ
ಉರ್ಕಣೆ
ಉರ್ದುಶಾಸದಿಂದ
ಉಲಗಾಮುಂಡನ
ಉಲ್
ಉಲ್ಲೆಖಿಸಿದ
ಉಲ್ಲೆಖಿಸುತ್ತದೆ
ಉಲ್ಲೇಖ
ಉಲ್ಲೇಖಗಳನ್ನು
ಉಲ್ಲೇಖಗಳಲ್ಲಿ
ಉಲ್ಲೇಖಗಳಿಂದ
ಉಲ್ಲೇಖಗಳು
ಉಲ್ಲೇಖಗೊಳ್ಳುವ
ಉಲ್ಲೇಖದೆ
ಉಲ್ಲೇಖವನ್ನು
ಉಲ್ಲೇಖವಾಗಿದೆ
ಉಲ್ಲೇಖವಾಗಿರುತ್ತಾರೆ
ಉಲ್ಲೇಖವಾಗಿರುವುದನ್ನು
ಉಲ್ಲೇಖವಾಗಿವೆ
ಉಲ್ಲೇಖವಿದೆ
ಉಲ್ಲೇಖವಿದೆಯೇ
ಉಲ್ಲೇಖವಿದ್ದು
ಉಲ್ಲೇಖವಿರಬಹುದೆಂದು
ಉಲ್ಲೇಖವಿರುವ
ಉಲ್ಲೇಖವಿಲ್ಲ
ಉಲ್ಲೇಖವಿಲ್ಲದ
ಉಲ್ಲೇಖವು
ಉಲ್ಲೇಖವೂ
ಉಲ್ಲೇಖವೇ
ಉಲ್ಲೇಖಿತ
ಉಲ್ಲೇಖಿತನಾಗಿದ್ದಾನೆ
ಉಲ್ಲೇಖಿತನಾಗಿರುವ
ಉಲ್ಲೇಖಿತನಾದ
ಉಲ್ಲೇಖಿತವಾಗಿದೆ
ಉಲ್ಲೇಖಿತವಾದ
ಉಲ್ಲೇಖಿಸದೇ
ಉಲ್ಲೇಖಿಸಬಹುದು
ಉಲ್ಲೇಖಿಸಲಾಗಿದೆ
ಉಲ್ಲೇಖಿಸಲಾಗಿದ್ದು
ಉಲ್ಲೇಖಿಸಿ
ಉಲ್ಲೇಖಿಸಿದ
ಉಲ್ಲೇಖಿಸಿದೆ
ಉಲ್ಲೇಖಿಸಿದ್ದರೂ
ಉಲ್ಲೇಖಿಸಿದ್ದಾರೆ
ಉಲ್ಲೇಖಿಸಿದ್ದು
ಉಲ್ಲೇಖಿಸಿರಬಹುದೆಂದು
ಉಲ್ಲೇಖಿಸಿರುವ
ಉಲ್ಲೇಖಿಸಿಲ್ಲ
ಉಲ್ಲೇಖಿಸಿವೆ
ಉಲ್ಲೇಖಿಸುತ್ತದೆ
ಉಲ್ಲೇಖಿಸುತ್ತವೆ
ಉಲ್ಲೇಖಿಸುತ್ತಾ
ಉಲ್ಲೇಖಿಸುತ್ತಾರೆ
ಉಲ್ಲೇಖಿಸುವ
ಉಳಿದ
ಉಳಿದಂತೆ
ಉಳಿದನು
ಉಳಿದವರ
ಉಳಿದವರಲ್ಲಿ
ಉಳಿದವು
ಉಳಿದಿತ್ತು
ಉಳಿದಿದೆ
ಉಳಿದಿರುವ
ಉಳಿದಿವೆ
ಉಳಿದುದನ್ನು
ಉಳಿದೆಲ್ಲವೂ
ಉಳಿಯುವಂತೆ
ಉಳಿಸಲು
ಉಳಿಸಿಕೊಂಡನು
ಉಳಿಸಿಕೊಂಡನೆಂದು
ಉಳಿಸಿಕೊಂಡಿತ್ತು
ಉಳಿಸಿಕೊಂಡಿವೆ
ಉಳಿಸಿಕೊಂಡು
ಉಳಿಸಿಕೊಳ್ಳಲಾಯಿತು
ಉಳುಮೆ
ಉಳುವ
ಉಳುವರ್ತಿ
ಉಳುವವರಿಗೆ
ಊಂಚಹಳ್ಳಿ
ಊರ
ಊರಗಾವುಂಡನಿರುತ್ತಿದ್ದನು
ಊರಗಾವುಂಡನೇ
ಊರಗಾವುಂಡರನ್ನು
ಊರನ್ನಾಗಿ
ಊರನ್ನು
ಊರನ್ನೇ
ಊರಳಿವಿನ
ಊರಳಿವಿನಲ್ಲಿ
ಊರಳಿವಿಲ್ಲಿ
ಊರಳಿವು
ಊರವರಿಂದ
ಊರಸೇನಬೋವ
ಊರಾಗಿತ್ತು
ಊರಾಗಿದೆ
ಊರಾಗಿದ್ದರಿಂದ
ಊರಾಗಿದ್ದು
ಊರಾಗಿರಬಹುದು
ಊರಾದ
ಊರಿಗೆ
ಊರಿದೆ
ಊರಿದ್ದು
ಊರಿನ
ಊರಿನಲ್ಲಿ
ಊರಿನಲ್ಲಿರುವ
ಊರಿನವನಾದ
ಊರಿನವರು
ಊರು
ಊರುಗಳ
ಊರುಗಳನ್ನು
ಊರುಗಳಲ್ಲಿ
ಊರುಗಳಾಗಿವೆ
ಊರುಗಳಾದ
ಊರುಗಳಿಗೆ
ಊರುಗಳಿವೆ
ಊರುಗಳು
ಊರುಗಳೂ
ಊರುಗಳೆಲ್ಲಾ
ಊರೂ
ಊರೆ
ಊರೇ
ಊರೊಡಯರೆಂದ
ಊರೊಡೆಯರು
ಊರೊಳಗಿನ
ಊಳಿಗಕ್ಕೆ
ಊಳಿಗದ
ಊಳಿಗದವರು
ಊಳಿಗವನ್ನು
ಊಹಿಸಬಹುದ
ಊಹಿಸಬಹುದು
ಊಹಿಸಬಹುದುಜಿನಗೃಹಮಂ
ಊಹಿಸಬುದು
ಊಹಿಸಲಾಗಿದೆ
ಊಹಿಸಲು
ಊಹಿಸಹುದು
ಊಹಿಸಿ
ಊಹಿಸಿದ್ದಾರೆ
ಊಹಿಸುವುದು
ಊಹೆ
ಊಹೆಗಿಂತ
ಊಹೆಯನ್ನು
ಊಹೆಯಾಗಿದೆ
ಋಕ್ಶಾಖೆಗೆ
ಋಣಕ್ಕೆ
ಋಷಿಗಳ
ಋಷಿಗಳು
ಋಷಿಗಳೇ
ಋಷಿಯರ
ಋಷಿಯು
ಎ
ಎಂ
ಎಂಎಂ
ಎಂಎಚ್
ಎಂಎಚ್ನಾಗರಾಜರಾವ್
ಎಂಟು
ಎಂದ
ಎಂದಮೇಲೆ
ಎಂದರೆ
ಎಂದರೇನು
ಎಂದಾಗ
ಎಂದಾಗಿರಬಹುದು
ಎಂದಾಗಿರುವ
ಎಂದಾಗುತ್ತದೆ
ಎಂದಿದೆ
ಎಂದಿದ್ದರೂ
ಎಂದಿದ್ದು
ಎಂದಿರಬಹುದು
ಎಂದಿರಬೇಕು
ಎಂದು
ಎಂದೂ
ಎಂದೆಂದಿಗೂ
ಎಂದೆನಿಸಿಕೊಳ್ಳುವುದರಲ್ಲಿಯೇ
ಎಂದೆಲ್ಲಾ
ಎಂದೇ
ಎಂಬ
ಎಂಬಂತೆ
ಎಂಬಲ್ಲಿ
ಎಂಬವು
ಎಂಬಾತನಿಗೆ
ಎಂಬಿವು
ಎಂಬು
ಎಂಬುದಕ್ಕೆ
ಎಂಬುದನ್ನು
ಎಂಬುದರ
ಎಂಬುದರಿಂದ
ಎಂಬುದಾಗಿ
ಎಂಬುದಾಗಿಯೂ
ಎಂಬುದಾಗಿಯೇ
ಎಂಬುದು
ಎಂಬುದೂ
ಎಂಬುದೇ
ಎಂಬುವನನ್ನು
ಎಂಬುವರು
ಎಂಬುವವನ
ಎಂಬುವವನನ್ನು
ಎಂಬುವವನಿಗೆ
ಎಂಬುವವನು
ಎಂಬುವವರನ್ನು
ಎಂಬುವವರು
ಎಂಬುವವಳ
ಎಂಬುವವಳು
ಎಂಬೆರುಮಾನರು
ಎಂಭತ್ತುವರಹವ
ಎಂಮ
ಎಂವಿ
ಎಂವಿಕೃಷ್ಣರಾವ್
ಎಕಾಎಕುವಾಘನಮಾರಿತ್ಯಾರಾಯಾಂಬಾಮುಲಾ
ಎಕ್ಕವ್ವೆಯರ
ಎಕ್ಕೋಟಿ
ಎಡಗಯ್ಯ
ಎಡಗೆಯ್ಯ
ಎಡಗೈ
ಎಡಗೈಎಡಭಾಗ
ಎಡಗೈಯ
ಎಡಗೈಯ್ಯ
ಎಡತಲೆಯ
ಎಡತಲೆಯಹೆಡತಲೆ
ಎಡತೊರೆಮಠದ
ಎಡದರೆ
ಎಡದರೆಸಾಯಿರ
ಎಡದೊರೆ
ಎಡದೊರೆನಾಡಿಗೆ
ಎಡದೊರೆನಾಡು
ಎಡಬಲಗಳಲ್ಲಿ
ಎಡಬಲದಲ್ಲಿರುವ
ಎಡವಂಕದಲ್ಲಿ
ಎಡವದ
ಎಡವಾರಯ
ಎಡೂರು
ಎಣಿಸುಮಗ
ಎಣ್ಣೆನಾಡ
ಎತ್ತರದ
ಎತ್ತರದಲ್ಲಿದ್ದು
ಎತ್ತರವಿದೆ
ಎತ್ತಿಕಟ್ಟಿದ
ಎತ್ತಿಕೊಂಡು
ಎತ್ತಿಸಿದನೆಂದು
ಎತ್ತಿಸಿದರೆಂದು
ಎತ್ತುತ್ತಿದ್ದ
ಎದುರಾಗಿ
ಎದುರಾದ
ಎದುರಿಗೆ
ಎದುರಿಸ
ಎದುರಿಸಬೇಕಾಗಿ
ಎದುರಿಸಬೇಕಾಯಿತು
ಎದುರಿಸಲು
ಎದುರಿಸಿ
ಎದುರಿಸಿದಾಗ
ಎದುರಿಸುತ್ತಾನೆ
ಎದುರಿಸುತ್ತಿದ್ದಾಗ
ಎದುರಿಸುವಂತ
ಎದುರಿಸುವಂತೆ
ಎದ್ದನು
ಎದ್ದರು
ಎದ್ದಾಗ
ಎದ್ದಿದ್ದ
ಎದ್ದು
ಎದ್ದುಹೋಗಲು
ಎನ್
ಎನ್ನಬಹುದು
ಎನ್ನಲು
ಎನ್ನುತ್ತಾರೆ
ಎನ್ನುತ್ತಿದ್ದರು
ಎನ್ನುವ
ಎನ್ನುವವನ
ಎನ್ನುವವರು
ಎನ್ನುವಷ್ಟರಲ್ಲಿ
ಎನ್ನುವುದನ್ನು
ಎನ್ನುವುದು
ಎಪಿಗ್ರಾಪಿಯಾ
ಎಪಿಗ್ರಾಫಿಯಾ
ಎಪ್ಪತ್ತಕ್ಕೆ
ಎಪ್ಪತ್ತೆರಡು
ಎಪ್ಪತ್ತೆರೆಡು
ಎಬ್ಬೆಬಸವನು
ಎಬ್ಬೆಯ
ಎಮ್ಎಆರ್
ಎಮ್ಜಿ
ಎಮ್ಮದೂರ
ಎಮ್ಮಳ್ದಕ್ಕೆ
ಎಮ್ಮೆಯ
ಎಮ್ಮೆಯಕೇತನಹಟ್ಟಿಯನ್ನು
ಎರಗನಹಳ್ಳಿಯನ್ನು
ಎರಗಿತು
ಎರಡನೆ
ಎರಡನೆಯ
ಎರಡನೆಯವನು
ಎರಡನೇ
ಎರಡನ್ನೂ
ಎರಡರ
ಎರಡರಲ್ಲೂ
ಎರಡರುನೂರು
ಎರಡು
ಎರಡುಕಟ್ಟೆ
ಎರಡುಬಾರಿ
ಎರಡುಮೂರು
ಎರಡೂ
ಎರದಿಂಮರಾಜಯ್ಯ
ಎರೆಗಂಗ
ಎರೆಗಂಗನ
ಎರೆಗಂಗನದು
ಎರೆಗಂಗನಿಗೆ
ಎರೆಗಂಗನು
ಎರೆಗಂಗನೆಂದು
ಎರೆದು
ಎರೆಯಂಗ
ಎರೆಯಂಗದೇವ
ಎರೆಯಂಗದೇವನ
ಎರೆಯಂಗನ
ಎರೆಯಂಗನಕಾಲದಿಂದ
ಎರೆಯಂಗನಿಗೆ
ಎರೆಯಂಗನು
ಎರೆಯಂಗನೆಂಬ
ಎರೆಯಂಗನೇ
ಎರೆಯಣ್ಣ
ಎರೆಯಣ್ಣನ
ಎರೆಯಪ್ಪ
ಎರೆಯಪ್ಪನ
ಎರೆಯಪ್ಪನನ್ನು
ಎರೆಯಪ್ಪನು
ಎರೆಯಪ್ಪನೆಂಬ
ಎರೆಯಪ್ಪನೊಂದಿಗೆ
ಎರೆಯಪ್ಪನೊಡನೆ
ಎರೆಯಪ್ಪರಸ
ಎರೆಯಪ್ಪರಸನನ್ನು
ಎರೆಯಪ್ಪರಸನು
ಎರೆಯಪ್ಪರಸನೆಂಬ
ಎರೆಯಪ್ಪರಸರು
ಎರೆಯಮಂಗಲದ
ಎರೆಯಮ್ಮನ
ಎರೆಯಮ್ಮನೆಂಬುವವನು
ಎಱೆಯಂಗ
ಎಲಿಗಾರ
ಎಲಿಗಾರರ
ಎಲೆ
ಎಲೆಕೊಪ್ಪದ
ಎಲೆಚಾಕನಹಳ್ಳಿಯ
ಎಲೆಯ
ಎಲೆಯಗುಳಿಯನ್ನು
ಎಲ್ಲ
ಎಲ್ಲರನ್ನೂ
ಎಲ್ಲರೂ
ಎಲ್ಲವನ್ನೂ
ಎಲ್ಲವೂ
ಎಲ್ಲಾ
ಎಲ್ಲಾದರೂ
ಎಲ್ಲಿಂದ
ಎಲ್ಲಿತ್ತು
ಎಲ್ಲಿದೆ
ಎಲ್ಲಿಯೂ
ಎಲ್ಲೂ
ಎಲ್ಲೆಗಳನ್ನು
ಎಲ್ಲೆಗಳಾಗಿದ್ದು
ಎಲ್ಲೆಯನ್ನು
ಎಲ್ಲೆಯಲ್ಲಿ
ಎಳಂದೂರು
ಎಳಗ
ಎಳಗನೆಂಬ
ಎಳವಾರೆ
ಎಳೆಯ
ಎಳೆಯನೆಂದು
ಎವಿ
ಎಷ್ಟೊಂದು
ಎಷ್ಟೋ
ಎಷ್ಟೋವೇಳೆ
ಎಸಗೂರು
ಎಸುವರಾದಿತ್ಯ
ಎಸ್ಕೆ
ಎಸ್ಕೆಮೋಹನ್ರವರ
ಎಸ್ವುರಾದಿತ್ಯನುಂ
ಎಸ್ಶಿವಣ್ಣ
ಏಕಭೋಗ
ಏಕಾಂಗವೀರ
ಏಕಾಂಗಿಯಾದನು
ಏಕಾದಶಪಲ್ಲೀ
ಏಕೀಕರಣ
ಏಕೆಂದರೆ
ಏಕೈಕ
ಏಕೋಜಿಯು
ಏಗಗವುಂಡನ
ಏಚಣ್ಣ
ಏಚಣ್ಣದಂಡನಾಯಕ
ಏಚಣ್ಣನ
ಏಚಣ್ಣನು
ಏಚದಂಡಾಧೀಶ
ಏಚನೆಂಬ
ಏಚಬ್ಬೆ
ಏಚಲದೇವಿ
ಏಚಲದೇವಿಯರ
ಏಚಲದೇವಿರು
ಏಚವ್ವೆ
ಏಚಿಕಬ್ಬೆ
ಏಚಿದಂಡಾಧಿಪನನ್ನು
ಏಚಿಮಯ್ಯ
ಏಚಿಮಯ್ಯದಂಡನಾಯಕ
ಏಚಿರಾಜ
ಏಚಿರಾಜದಂಡಾಧೀಶನ
ಏಚಿರಾಜನ
ಏಚಿರಾಜನನ್ನು
ಏಚಿರಾಜನಿಗೆ
ಏಚಿರಾಜನು
ಏಚಿರಾಜನೂ
ಏಚಿರಾಜಹಿರಿಯ
ಏಟೂರುನಿವಾಸಿ
ಏತಕ್ಕೆ
ಏನಾದರೂ
ಏನಿದ್ದರೂ
ಏನು
ಏನುಬೇಕಾದರೂ
ಏಪ್ರಿಲ್
ಏರಲು
ಏರಿದ
ಏರಿದನು
ಏರಿದನೆಂದು
ಏರಿದರೆಂದು
ಏರಿದ್ದನು
ಏರಿದ್ದರೆಂದು
ಏರಿದ್ದಾನೆ
ಏರಿರಬಹುದು
ಏರಿರಬಹುದೆಂದು
ಏರಿರುವ
ಏರುತ್ತಿದ್ದರೆಂದು
ಏರ್ಪಡಿಸಲು
ಏರ್ಪಡಿಸಿ
ಏರ್ಪಡಿಸಿದಂತೆ
ಏರ್ಪಡಿಸಿದನು
ಏರ್ಪಡಿಸಿರಬಹುದು
ಏರ್ಪಡಿಸುವ
ಏರ್ಪಡಿಸುವಲ್ಲಿ
ಏರ್ಪಡಿಸುವುದು
ಏರ್ಪಾಡುಮಾಡುತ್ತಾನೆ
ಏಳನೆ
ಏಳನೇ
ಏಳನೇಬಾರಿಗೆ
ಏಳರಲಕ್ಕಏಳೂವರೆಲಕ್ಷ
ಏಳು
ಏಳುತ್ತದೆ
ಏಳುದಿನಗಳಲ್ಲಿ
ಏಳುನೂರು
ಏಳುನೂರುಮ್
ಏಳುಪುರ
ಏಳುಪುರದ
ಏಳುಬತ್ತೆಟ್ಟು
ಏಳುಬೀಳುಗಳು
ಏಳುಬೀಳುಗಳೇ
ಏಳುಮಲೆ
ಏಳುಹಣದ
ಏಳೂವರೆಲಕ್ಷ
ಏಳೂವರೆಲಕ್ಷವನ್ನು
ಏಳೆಂಟು
ಏವೊಗಳ್ವೆನುನ್ನತಿಯಂ
ಏೞನೆಯ
ಐಂದ್ರಪರ್ವಕ್ಕೆ
ಐಕ್ಯವಾದಂತೆ
ಐತಪಾರ್ಯನ
ಐತಿಹಾಸಿಕ
ಐತಿಹಾಸಿಕವಾಗಿ
ಐತಿಹ್ಯ
ಐತಿಹ್ಯದ
ಐತಿಹ್ಯದಂತೆ
ಐತಿಹ್ಯದಿಂದಲೂ
ಐದನೆಯ
ಐದನೇ
ಐದು
ಐದುಜನ
ಐದುಜನರಿಗೆ
ಐದುಹಳ್ಳಿಗಳನ್ನು
ಐನೂರು
ಐನ್
ಐನ್ಉಲ್ಮುಲ್ಕ್
ಐಮಂಗಳ
ಐವತ್ತು
ಐವತ್ತೊಕ್ಕಲು
ಐಶ್ವರ್ಯ
ಐಶ್ವರ್ಯವನ್ನು
ಒಂಟೆ
ಒಂದಕ್ಕಿಂತ
ಒಂದಕ್ಕೆ
ಒಂದನೆಯ
ಒಂದನೆಯಬಲ್ಲಾಳನ
ಒಂದನೇ
ಒಂದರ
ಒಂದಾಗಿ
ಒಂದಾಗಿತ್ತು
ಒಂದಾಗಿದೆ
ಒಂದಾಗಿರುವುದರಿಂದ
ಒಂದಾದ
ಒಂದು
ಒಂದುಕಡೆ
ಒಂದುಸಲಗೆ
ಒಂದುಸಾವಿರ
ಒಂದುಸಾವಿರಹೊನ್ನನ್ನು
ಒಂದೂ
ಒಂದೂವರೆ
ಒಂದೆರಡು
ಒಂದೇ
ಒಂದೊಂದು
ಒಂದೋ
ಒಂಬತ್ತನೆಯ
ಒಂಬತ್ತು
ಒಕ್ಕಣೆ
ಒಕ್ಕಣೆಗಿಂತ
ಒಕ್ಕಣೆಯನ್ನೇ
ಒಕ್ಕಣೆಯು
ಒಕ್ಕಣೆಯೂ
ಒಕ್ಕಲಾದ
ಒಕ್ಕಲಿಕ್ಕಿದನೆಂದು
ಒಕ್ಕಲಿಗ
ಒಕ್ಕಲು
ಒಕ್ಕಲುಗಳಿಂದ
ಒಕ್ಕಲುಗಳು
ಒಕ್ಕಲುಗೂಡಿದ್ದರೆಂದು
ಒಕ್ಕಲ್ಗೆ
ಒಕ್ಕುವ
ಒಕ್ಕೂಟವನ್ನು
ಒಗೊಂಡಿತ್ತು
ಒಟ್ಟಾಗಿ
ಒಟ್ಟಾಗಿಯೇ
ಒಟ್ಟಾರೆ
ಒಟ್ಟಿಗೆ
ಒಟ್ಟಿನಲ್ಲಿ
ಒಟ್ಟು
ಒಡಂಬಟ್ಟು
ಒಡಗೂಡಿ
ಒಡಗೆರೆ
ಒಡಗೆರೆಮಲ್ಲಇದೂ
ಒಡಗೆರೆಮಲ್ಲನುಂ
ಒಡನೆಯೇ
ಒಡಹುಟ್ಟಿದ
ಒಡೆತನ
ಒಡೆತನಕ್ಕೆ
ಒಡೆತನವನ್ನು
ಒಡೆತನವೂ
ಒಡೆದು
ಒಡೆಯ
ಒಡೆಯನ
ಒಡೆಯನಂಬಿಯಾದ
ಒಡೆಯನನ್ನು
ಒಡೆಯನನ್ನೇ
ಒಡೆಯನಾಗಿದ್ದನೆಂದು
ಒಡೆಯನಾದ
ಒಡೆಯನಿಗೂ
ಒಡೆಯನಿಗೆ
ಒಡೆಯನು
ಒಡೆಯನುದೊಡ್ಡದೇವರಾಜ
ಒಡೆಯನೂಮಹಾಪ್ರಧಾನಿ
ಒಡೆಯನೆಂದರೆ
ಒಡೆಯನೆಂದು
ಒಡೆಯನೆಂಬ
ಒಡೆಯನೇ
ಒಡೆಯನೊಬ್ಬನಿಗೆ
ಒಡೆಯರ
ಒಡೆಯರಕಟ್ಟೆ
ಒಡೆಯರನ್ನು
ಒಡೆಯರಾಗಿ
ಒಡೆಯರಾಗಿದ್ದರು
ಒಡೆಯರಾಗಿದ್ದರೆಂದು
ಒಡೆಯರಾದ
ಒಡೆಯರಿಂದ
ಒಡೆಯರಿಗೆ
ಒಡೆಯರು
ಒಡೆಯರ್
ಒಡೆಯಾರ
ಒಡೆಯುನು
ಒಡೆಯುರ
ಒಡೆರಯರು
ಒಡ್ಡಗಲ್ಲುರಂಗಸ್ವಾಮಿಬೆಟ್ಟ
ಒಡ್ಡು
ಒತ್ತರಿಸಿದವನು
ಒತ್ತಿ
ಒತ್ತೆ
ಒತ್ತೆಇಟ್ಟಿದ್ದ
ಒತ್ತೆಯಾಗಿ
ಒತ್ತೆಯಾಗಿರಿಸಿಕೊಂಡಿದ್ದನೆಂದೂ
ಒದಗಿದಾಗ
ಒದಗಿಸಿದರೆ
ಒದಗಿಸುತ್ತದೆ
ಒದಗಿಸುತ್ತವೆ
ಒದಗಿಸುತ್ತಿದ್ದವರೇ
ಒದಗಿಸುವುದನ್ನು
ಒದೆಯೂರು
ಒಪ್ಪ
ಒಪ್ಪಂದ
ಒಪ್ಪಂದಅವನ್ನು
ಒಪ್ಪಂದಕ್ಕೆ
ಒಪ್ಪಂದದ
ಒಪ್ಪಂದವನ್ನು
ಒಪ್ಪಂದವಾಗಿ
ಒಪ್ಪತಕ್ಕದ್ದೇ
ಒಪ್ಪದ
ಒಪ್ಪಬಹುದು
ಒಪ್ಪಿ
ಒಪ್ಪಿಕೊಳ್ಳದೆ
ಒಪ್ಪಿಕೊಳ್ಳಬೇಕಾಯಿತು
ಒಪ್ಪಿಗೆ
ಒಪ್ಪಿಗೆಗೆ
ಒಪ್ಪಿಗೆಯನ್ನು
ಒಪ್ಪಿಸಲಾಯಿತೆಂದು
ಒಪ್ಪಿಸಿ
ಒಪ್ಪಿಸಿದ
ಒಪ್ಪಿಸಿದನಂತೆ
ಒಪ್ಪಿಸಿದನೆಂದು
ಒಪ್ಪಿಸಿದಾಗ
ಒಪ್ಪುತ್ತಾರೆ
ಒಬ್ಬ
ಒಬ್ಬನಾಗಿದ್ದನು
ಒಬ್ಬನಾಗಿದ್ದು
ಒಬ್ಬನಾಗಿರಬಹುದು
ಒಬ್ಬನಾಗಿರುವ
ಒಬ್ಬನಾದ
ಒಬ್ಬನೇ
ಒಬ್ಬರ
ಒಬ್ಬರಾಗಿದ್ದರು
ಒಬ್ಬರಾಗಿದ್ದರೆಂದು
ಒಬ್ಬರಾಜನಾಗಿ
ಒಬ್ಬರಾದರು
ಒಬ್ಬರಿಂದ
ಒಬ್ಬರಿಗಿಂತ
ಒಬ್ಬರಿಗೊಬ್ಬರಿಗೆ
ಒಬ್ಬರೆಂದು
ಒಬ್ಬರೇ
ಒಬ್ಬೊಬ್ಬ
ಒಮಲೂರು
ಒಮ್ಮತವಿಲ್ಲ
ಒಮ್ಮೆಲೇ
ಒಯ್ದರು
ಒರಟೂರು
ಒರಿಸ್ಸಾ
ಒರಿಸ್ಸಾದ
ಒಲವು
ಒಳ
ಒಳಕೇರಿಯ
ಒಳಕೇರಿಯಲ್ಲಿ
ಒಳಕ್ಕೆ
ಒಳಗಣ
ಒಳಗಾದ
ಒಳಗಾದರು
ಒಳಗಿತ್ತೆಂದು
ಒಳಗೆ
ಒಳಗೊಂಡ
ಒಳಗೊಂಡಿತ್ತು
ಒಳಗೊಂಡಿತ್ತೆಂದು
ಒಳಗೊಂಡಿವೆ
ಒಳನಾಡಾಗಿರಬಹುದು
ಒಳಪಟ್ಟ
ಒಳಪಟ್ಟರೂ
ಒಳಪಟ್ಟಿತ್ತು
ಒಳಪಟ್ಟಿತ್ತೆಂದು
ಒಳಪಟ್ಟಿದ್ದಂತೆ
ಒಳಪಡಿಸಲಾಗಿದೆ
ಒಳಪಡಿಸಿ
ಒಳಪಡಿಸಿಕೊಳ್ಳಲು
ಒಳಪಡಿಸಿದನು
ಒಳಪಡುವುದಕ್ಕೆ
ಒಳಪ್ರಾಕಾರದ
ಒಳಭಾಗದಲ್ಲಿದ್ದು
ಒಳಮುಟ್ಟನಹಳ್ಳಿಗಳನ್ನು
ಒಳಲು
ಒಳವಾರು
ಒಳಹೊಕ್ಕನೆಂದು
ಒಳಹೊಕ್ಕವರಲ್ಲಿ
ಒಳಹೊಕ್ಕು
ಒಳ್ಳೆಯ
ಓಕದಕಲ್ಲು
ಓಡಾಡುತ್ತಿದ್ದನಷ್ಟೆ
ಓಡಿ
ಓಡಿಸಲು
ಓಡಿಸಿ
ಓಡಿಸಿಕೊಂಡು
ಓಡಿಸಿದಂತೆ
ಓಡಿಸಿದನೆಂದೂ
ಓಡಿಸಿದುದು
ಓಡಿಹೋಗಿದ್ದು
ಓಡಿಹೋದದ್ದು
ಓಡಿಹೋದನಂತೆ
ಓಡಿಹೋದನು
ಓಡಿಹೋಯಿತು
ಓದಿ
ಓದಿಕೊಂಡರೆ
ಓಬಾಂಬಿಕೆ
ಓಬಾಂಬಿಕೆಯಿಂದ
ಓಬಾಂಬೆಯರ
ಓರಂಗಲ್
ಓರಪಣಪುರ
ಓಲಗಸಾಲೆಯನ್ನು
ಓಲೆ
ಓಷಧಿಪತ್ಯುಪಮಾಯಿತಗಂಡಸ್ತೋಷಣರೂಪಜಿತಾಸಮಕಾಂಡಃ
ಔದಾರ್ಯಕ್ಕೆ
ಕಂಗು
ಕಂಚಲದೇವಿ
ಕಂಚಿ
ಕಂಚಿಗಹಳ್ಳಿಯ
ಕಂಚಿಗುರಿಯಪ್ಪನಮೋಡಿದ
ಕಂಚಿಗೆ
ಕಂಚಿಗೊಂಡ
ಕಂಚಿನಕೆರೆ
ಕಂಚಿಪಟ್ಟಣದತ್ತ
ಕಂಚಿಮಠದ
ಕಂಚಿಯ
ಕಂಚಿಯತ್ತ
ಕಂಚಿಯನ್ನೇ
ಕಂಚಿಯಿಂದ
ಕಂಚುಗಹಳ್ಳಿಗಳು
ಕಂಟಕಗಳನ್ನು
ಕಂಟಿಮಯ್ಯ
ಕಂಟಿಮಯ್ಯಂ
ಕಂಟಿಮಯ್ಯನ
ಕಂಟಿಮಯ್ಯನು
ಕಂಟಿಮಯ್ಯನೂ
ಕಂಠೀರವ
ಕಂಠೀರವಗುಳಿಗೆ
ಕಂಠೀರವನರಸನೃಪಾಂಬೋಧಿ
ಕಂಠೀರವನರಸರಾಜ
ಕಂಠೀರವನರಸರಾಜನು
ಕಂಠೀರವನು
ಕಂಠೀರವನೆನಿಸಿ
ಕಂಠೀರವಾಕೃತಿಃ
ಕಂಠೀರವಾಕ್ರುತಿಃ
ಕಂಠೀರಾಯವರಹವನ್ನು
ಕಂಠೀವರ
ಕಂಡ
ಕಂಡನು
ಕಂಡರಿಸಲ್ಪಟ್ಟಿದೆ
ಕಂಡು
ಕಂಡುಂದಿದ್ದು
ಕಂಡುಗ
ಕಂಡುಬಂದಿದ್ದು
ಕಂಡುಬಂದಿರುವ
ಕಂಡುಬರುತ್ತದೆ
ಕಂಡುಬರುತ್ತವೆ
ಕಂಡುಬರುತ್ತಾರೆ
ಕಂಡುಬರುತ್ತಿದೆ
ಕಂಡುಬರುವ
ಕಂಡುಬರುವುದಿಲ್ಲ
ಕಂಡುಹಿಡಿದು
ಕಂಡೆವು
ಕಂಣಂಬಾಡಿಯ
ಕಂಣನೂರ
ಕಂಣ್ನಂಬಿನಾತನುಂ
ಕಂಣ್ನಯ
ಕಂಣ್ನಯನಾಯಕನು
ಕಂದ
ಕಂದಂ
ಕಂದಕುದ್ದಾಳ
ಕಂದಾಚಾರ
ಕಂದಾಚಾರದ
ಕಂದಾಡಿ
ಕಂದಾಯ
ಕಂದಾಯಕ್ಕೆ
ಕಂದಾಯದ
ಕಂದಾಯವನ್ನು
ಕಂಧರ
ಕಂನಂಬಾಡಿ
ಕಂನಗೊಂಡೇಶ್ವರ
ಕಂನಡಿ
ಕಂನಾರದೇವ
ಕಂನಾರದೇವನೆಂದು
ಕಂನೆಯನಾಯಕನು
ಕಂಪಂಣಗಳು
ಕಂಪಣ
ಕಂಪಣಕ್ಕೆ
ಕಂಪಣಗಳೆಂಬ
ಕಂಪಣದ
ಕಂಪಣನ
ಕಂಪಣ್ಣ
ಕಂಪಣ್ಣನು
ಕಂಪನಿಯವರು
ಕಂಪನು
ಕಂಪಮಂತ್ರಿಯಿದ್ದನೆಂದು
ಕಂಪರಾಜನ
ಕಂಪರಾಜನು
ಕಂಪಿಲದೇವನೊಡನೆ
ಕಂಪಿಲನ
ಕಂಪಿಲನು
ಕಂಪಿಲನೊಡನೆ
ಕಂಪಿಲಿದೇವನು
ಕಂಪಿಲಿಯ
ಕಂಪೆಲನ
ಕಂಬದ
ಕಂಬದಹಳ್ಳಿ
ಕಂಬದಹಳ್ಳಿಯ
ಕಂಬದಹಳ್ಳಿಯಲ್ಲಿ
ಕಂಬದಿಂದ
ಕಂಬನ
ಕಂಬಯ್ಯ
ಕಂಬಯ್ಯನನ್ನು
ಕಂಬಯ್ಯನು
ಕಂಬರಾಜನು
ಕಂಬವನ್ನು
ಕಂಬೇಶ್ವರ
ಕಂಭದ
ಕಂಮಗಾರ
ಕಂಸರ
ಕಇವಾರ
ಕಇವಾರಕ
ಕಇವಾರಜಗದ್ಧಳ
ಕಕುದ್ಗಿರಿ
ಕಕ್ಕಗ
ಕಗ್ಗಲೀಪುರ
ಕಗ್ಗಲೀಹಳ್ಳಿ
ಕಚ
ಕಚೇರಿ
ಕಚ್ಚವರದ
ಕಚ್ಚಾಣಗದ್ಯಾಣವನ್ನು
ಕಚ್ಚೆಗ
ಕಟ
ಕಟಕಕತ್ತಿ
ಕಟಕತೋಟಿಕಾರ
ಕಟಕದ
ಕಟಕದೊಡನೆ
ಕಟಕರಕ್ಷಣೆ
ಕಟಕವು
ಕಟವಪ್ರ
ಕಟವಪ್ರಗಿರಿ
ಕಟವಪ್ರಶೈಲ
ಕಟಿಸಿದನೆಂದು
ಕಟೆಯೇರಿ
ಕಟ್ಟಡದೊಳಕ್ಕೆ
ಕಟ್ಟಲಾಗಿದೆ
ಕಟ್ಟಲಾಯಿತು
ಕಟ್ಟಲು
ಕಟ್ಟಳೆಯ
ಕಟ್ಟಿ
ಕಟ್ಟಿಕೊಂಡು
ಕಟ್ಟಿಕೊಟಬಹುದು
ಕಟ್ಟಿಕೊಟ್ಟರು
ಕಟ್ಟಿಕೊಟ್ಟಿದ್ದಾರೆ
ಕಟ್ಟಿಕೊಡಅಬಹುದು
ಕಟ್ಟಿಕೊಡಬಹದ್ದು
ಕಟ್ಟಿಕೊಡಬಹುದು
ಕಟ್ಟಿಕೊಡಲಾಗಿದೆ
ಕಟ್ಟಿದನು
ಕಟ್ಟಿದನೆಂದು
ಕಟ್ಟಿದರಂದು
ಕಟ್ಟಿದಲಗಿನಂತಿದ್ದ
ಕಟ್ಟಿರುವ
ಕಟ್ಟಿಸಿ
ಕಟ್ಟಿಸಿಕೊಡುತ್ತಾನೆ
ಕಟ್ಟಿಸಿದ
ಕಟ್ಟಿಸಿದನು
ಕಟ್ಟಿಸಿದನೆಂದು
ಕಟ್ಟಿಸಿದನೆಂದೂ
ಕಟ್ಟಿಸಿದರು
ಕಟ್ಟಿಸಿದಳು
ಕಟ್ಟಿಸಿದಾಗ
ಕಟ್ಟಿಸಿದೆವು
ಕಟ್ಟಿಸಿದ್ದ
ಕಟ್ಟಿಸಿದ್ದನು
ಕಟ್ಟಿಸಿದ್ದನೆಂದು
ಕಟ್ಟಿಸಿದ್ದಾನೆ
ಕಟ್ಟಿಸಿರಬಹುದು
ಕಟ್ಟಿಸಿರುವ
ಕಟ್ಟಿಸುತ್ತಾನೆ
ಕಟ್ಟಿಸುವ
ಕಟ್ಟು
ಕಟ್ಟುಕಾಲುವೆಗಳು
ಕಟ್ಟುಕಾಲುವೆಯೊಳಗಾದ
ಕಟ್ಟುಕಾಲುವೆಯೊಳಗಿನ
ಕಟ್ಟುಕಾಲುವೆಯೊಳಗೆ
ಕಟ್ಟುಪಾಡುಗಳನ್ನು
ಕಟ್ಟುವುದಕ್ಕಾಗಿ
ಕಟ್ಟುವುದಕ್ಕೆ
ಕಟ್ಟೆಕಾಲುವೆಗಳನ್ನು
ಕಟ್ಟೆಕೇತನಹಳ್ಳಿ
ಕಟ್ಟೆಯ
ಕಟ್ಟೆಯನ್ನು
ಕಟ್ಟೆಯಲ್ಲಿ
ಕಟ್ಟೇನಹಳ್ಳಿ
ಕಟ್ಟೇರಿ
ಕಟ್ಟೇರಿನ
ಕಟ್ಟೇಸೋಮನಹಳ್ಳಿ
ಕಠಾರಿ
ಕಠಾರಿರಾಯ
ಕಠಾರಿರಾಯರಾದ
ಕಠಾರಿಸಾಳುವ
ಕಡತವನ್ನು
ಕಡಬದ
ಕಡಲ
ಕಡಲವಾಗಿಲ
ಕಡಲವಾಗಿಲು
ಕಡಲುವಾಗಿಲು
ಕಡವು
ಕಡವುಅಪ್ಪು
ಕಡವೂರ
ಕಡಾರಂನ್ನು
ಕಡಿತಕ್ಕೇರಿತು
ಕಡಿದು
ಕಡಿದುಕೊಂಡನೆಂದು
ಕಡಿದುಕೊಂಡು
ಕಡಿಮೆ
ಕಡಿಮೆಯಾಗುತ್ತಾ
ಕಡುಕಲಿ
ಕಡುಪಿಂ
ಕಡೂರು
ಕಡೆ
ಕಡೆಗಣಿಸಿ
ಕಡೆಗಳಲ್ಲಿ
ಕಡೆಗಳಲ್ಲಿರುವ
ಕಡೆಗೆ
ಕಡೆಯ
ಕಡೆಯಲ್ಲಿ
ಕಡೆಯವರು
ಕಡೆಯಿಂದ
ಕಡ್ವಪ್ಪು
ಕಣ
ಕಣಗಳ
ಕಣಿವೆ
ಕಣಿವೆಗಳಿವೆ
ಕಣಿವೆಯ
ಕಣಿವೆಯಲ್ಲಿ
ಕಣಿವೆಯಲ್ಲಿದ್ದು
ಕಣಿವೆಯು
ಕಣ್ಡೆವೆನೆ
ಕಣ್ಣಂಬಾಡಿ
ಕಣ್ಣನೆಂಬುವವನು
ಕಣ್ಣನ್ನಿಟ್ಟೇ
ಕಣ್ಣವಂಗಲವನು
ಕಣ್ಣಾನೂರ
ಕಣ್ಣಾನೂರನ್ನೂ
ಕಣ್ಣಾನೂರಿನಲ್ಲಿ
ಕಣ್ಣಾನೂರಿನಿಂದಲೇ
ಕಣ್ಣಾನೂರೇ
ಕಣ್ಣಾರೆಕಂಡು
ಕಣ್ಣೆಗಾಲದಲ್ಲಿ
ಕಣ್ಣೇಗಾಲದಲ್ಲಿ
ಕಣ್ನಂಬಾಡಿ
ಕಣ್ನಂಬಾಡಿಯ
ಕಣ್ನಂಬಿನೊಡೆಯ
ಕಣ್ವ
ಕಣ್ವೇಶ್ವರ
ಕಣ್ವೇಶ್ವರಕ್ಕೆ
ಕತ್ತರಿಗಟ್ಟ
ಕತ್ತರಿಗಟ್ಟದ
ಕತ್ತರಿಗಟ್ಟದಲ್ಲಿ
ಕತ್ತರಿಘಟ್ಟ
ಕತ್ತರಿಘಟ್ಟದ
ಕತ್ತರಿಘಟ್ಟದಲ್ಲಿ
ಕತ್ತರಿಸಿ
ಕತ್ತಿ
ಕಥನಮನ್ತೆಂದಡೆ
ಕಥೆ
ಕಥೆಗಳನ್ನು
ಕಥೆಯ
ಕಥೆಯನ್ನು
ಕಥೆಯನ್ನುಹೇಳುವ
ಕಥೆಯಲ್ಲಿ
ಕಥೆಯಲ್ಲಿಯೂ
ಕಥೆಯು
ಕದಂಬ
ಕದಂಬರ
ಕದಂಬರು
ಕದಂಬರೆಂದರೆ
ಕದಂಬೆಹಳ್ಳಿ
ಕದಂವ
ಕದನಗಳಲ್ಲಿ
ಕದನಗಳು
ಕದನತ್ರಿಣೇತ್ರನುಂ
ಕದನದಲ್ಲಿ
ಕದನದೊಳಾಂತು
ಕದನದೊಳ್
ಕದನಪ್ರಚಂಡ
ಕದನವಾಗಿ
ಕದನವಾಗಿರಬಹುದು
ಕದನವು
ಕದನೈಕ
ಕದನೈಕಸೂದ್ರಕ
ಕದಪನಾಯಕ
ಕದಬಳ್ಳಿ
ಕದಬಳ್ಳಿಯು
ಕದಬಹಳ್ಳಿಯನ್ನು
ಕದಬೆ
ಕದಬೆಹಳ್ಳಿಯ
ಕದರಪ್ಪ
ಕದರೂರು
ಕದರೆಯನಾಯಕ
ಕದರೆಯನಾಯಕನು
ಕದಲಗೆರೆ
ಕದಲಗೆರೆಯಿಂದ
ಕದಳಗೆರೆಯ
ಕದವಿ
ಕದವೆಹಳ್ಳಿ
ಕದ್ದಳಗೆರಇಂದಿನ
ಕದ್ದಳಗೆರೆ
ಕದ್ದಳಗೆರೆಯ
ಕದ್ದಳಗೆರೆಯುಇಂದಿನ
ಕನಕಕರ್ಪ್ಪೂರಧಾರಾಪ್ರವಾಹ
ಕನಕಗಿರಿ
ಕನಕಗಿರಿಯು
ಕನಕಚಾಮರ
ಕನಕಛತ್ರಂಗಳಂ
ಕನಕದಂಡಿಗೆ
ಕನಕನಘಟ್ಟ
ಕನಕಪುರ
ಕನಕಸೇನ
ಕನಗನಮರಡಿ
ಕನೂರು
ಕನೂರ್
ಕನ್ನಂಬಾಡಿ
ಕನ್ನಂಬಾಡಿಯ
ಕನ್ನಂಬಾಡಿಯನ್ನು
ಕನ್ನಂಬಾಡಿಯಲ್ಲಿ
ಕನ್ನಂಬಾಡಿಯಲ್ಲಿದ್ದ
ಕನ್ನಂಬಾಡಿಯವರೆಗೆ
ಕನ್ನಂಬಾಡಿಯು
ಕನ್ನಡ
ಕನ್ನಡದ
ಕನ್ನಡದಲ್ಲಿ
ಕನ್ನಡನಾಡನ್ನು
ಕನ್ನಡನಾಡಿನ
ಕನ್ನಡನಾಡು
ಕನ್ನಡವನ್ನು
ಕನ್ನಡಿಗ
ಕನ್ನಡಿಗನಾಗಿದ್ದ
ಕನ್ನಡಿಗರ
ಕನ್ನಡಿಗರಾದ
ಕನ್ನಯನಪಳ್ಳಿಯಲ್ಲಿ
ಕನ್ನಯ್ಯನ
ಕನ್ನರ
ಕನ್ನರದೇವ
ಕನ್ನರದೇವನು
ಕನ್ನರನು
ಕನ್ನಲ್ಲಿ
ಕನ್ನಸತ್ತಿ
ಕನ್ನಿಕೇಶ್ವರ
ಕನ್ನೆಗೆರೆಮಲ್ಲ
ಕನ್ನೆಗೆರೆಯನ್ನು
ಕನ್ನೆಗೆರೆಯಾಗಿ
ಕನ್ನೆಯನ
ಕನ್ನೆವಸದಿಯನ್ನು
ಕನ್ಯಕೆಯರನೊಂದೆ
ಕನ್ಯಾಕುಮಾರಿ
ಕನ್ಯಾದಾನ
ಕನ್ಯಾದಾನಗಳನ್ನು
ಕಪಿಲಾ
ಕಪಿಲಾನದಿಗಳ
ಕಪಿಲೆಯ
ಕಪಿಳೆಯ
ಕಪ್ಪಕಾಣಿಕೆಗಳನ್ನು
ಕಪ್ಪವನ್ನು
ಕಬರಸ್ಥಾನಕ್ಕಾಗಿ
ಕಬಾಹು
ಕಬಾಹುನಾಡಾಳುವ
ಕಬ್ಬಪ್ಪು
ಕಬ್ಬಪ್ಪುನಾಡ
ಕಬ್ಬರೆ
ಕಬ್ಬಹಲಿನ
ಕಬ್ಬಹಲು
ಕಬ್ಬಹು
ಕಬ್ಬಹುನಾಡಾಳುವ
ಕಬ್ಬಹುನಾಡಿನ
ಕಬ್ಬಾರೆ
ಕಬ್ಬಾಳ
ಕಬ್ಬಾಳುದುರ್ಗ
ಕಬ್ಬಾಳುದುರ್ಗವು
ಕಬ್ಬಾಹು
ಕಬ್ಬಾಹುಕಬ್ಬಹು
ಕಬ್ಬಾಹುನಾಡನ್ನು
ಕಬ್ಬಾಹುನಾಡಾಳುವ
ಕಬ್ಬಾಹುನಾಡಿನ
ಕಬ್ಬಾಹುನಾಡು
ಕಬ್ಬಾಹುನಾಡೇ
ಕಬ್ಬಾಹುಸಾಸಿರದ
ಕಬ್ಬಿಣಯುಗ
ಕಬ್ಬಿನಹಳ್ಳಿ
ಕಬ್ಬುನಾಡ
ಕಬ್ಬುನಾಡಾಗಿರಬಹುದು
ಕಬ್ಬುನಾಡಿನ
ಕಬ್ಬುಹುನಾಡ
ಕಮನೀಯಾ
ಕಮಲನಾದ
ಕಮಲವನ
ಕಮಳರಾಜ
ಕಮೀಷನರ್ಗಳ
ಕಮ್ಮಗಾರ
ಕಮ್ಮಟದ
ಕಮ್ಮಾರ
ಕಮ್ಮೆಕುಲಕ್ಕೆ
ಕಯಿಸೆರೆಯನಿಕ್ಕಿದ
ಕಯ್ಯಲು
ಕರಂಡ
ಕರಗುಂದ
ಕರಗ್ರಾಮಗಳು
ಕರಜಿತಸುರಭೂಜಃ
ಕರಡಹಳ್ಳಿ
ಕರಡಿಕೊಪ್ಪಲು
ಕರಡಿಯಹಳ್ಳಿ
ಕರಡು
ಕರಣ
ಕರಣಕರು
ಕರಣದ
ಕರಣನಾಗಿದ್ದನೆಂದು
ಕರಣನಾದ
ಕರಣನೆಂದರೆ
ಕರಣರಾಗಿದ್ದು
ಕರಣರು
ಕರಣರೂ
ಕರಣಿ
ಕರಣಿಕ
ಕರಣಿಕರಾದ
ಕರಣಿಕರು
ಕರಣಿಕರೆಂದೇ
ಕರಣಿಕಸೇನಬೋವಕುಲಕರಣಿ
ಕರಣಿಕಸೇನಬೋವಕುಳಕರಣಿ
ಕರದಾಖಿಲಭೂಪಾಲಃ
ಕರಬಸಾಣಿ
ಕರಮೆಸೆದಂ
ಕರಮೆಸೆಯೆ
ಕರಾರುವಕ್ಕಾಗಿ
ಕರಾವಳಿ
ಕರಿ
ಕರಿಅಯ್ಕಣನ
ಕರಿಎಮ್ಮಾಉರ
ಕರಿಕಲ್ಲುಮಂಟಿ
ಕರಿಘಟ್ಟ
ಕರಿತುರಕಗಪಟ್ಟಸಾಹಣಿ
ಕರಿತುರಗ
ಕರಿಪಗವುಡ
ಕರಿಯ
ಕರಿಯಅಯ್ಕಣನೆಂದು
ಕರಿಯನೆಚ್ಚಡೆ
ಕರಿಯಯ್ಕಣನೆಂದು
ಕರಿಲ
ಕರೀಘಟ್ಟದ
ಕರೀಜೀರಹಳ್ಳಿ
ಕರುಣಿಸುತ್ತಾನೆ
ಕರುಣೈಕ
ಕರೆಕಂಠಜೀಯನು
ಕರೆದರೆ
ಕರೆದಿದೆ
ಕರೆದಿದ್ದಾನೆ
ಕರೆದಿದ್ದಾರೆ
ಕರೆದಿದ್ದು
ಕರೆದಿರಬಹುದು
ಕರೆದಿರಬಹುದುಪ್ರಿಯಸುತ
ಕರೆದಿರುಬಹುದು
ಕರೆದಿರುವ
ಕರೆದಿರುವುದನ್ನು
ಕರೆದಿರುವುದರಿಂದ
ಕರೆದಿರುವುದಿಲ್ಲ
ಕರೆದಿಲ್ಲ
ಕರೆದಿವೆ
ಕರೆದಿವೆಮೇಲ್ಕಂಡ
ಕರೆದೀವದಾನಿವುಂ
ಕರೆದು
ಕರೆದುಕೊಂಡ
ಕರೆದುಕೊಂಡರೇ
ಕರೆದುಕೊಂಡಿದ್ದಾನೆ
ಕರೆದುಕೊಂಡಿದ್ದಾರೆ
ಕರೆದುಕೊಂಡಿರುವುದರಲ್ಲಿ
ಕರೆದುಕೊಂಡಿರುವುದರಿಂದ
ಕರೆದೇ
ಕರೆಯ
ಕರೆಯತೊಡಗಿದರು
ಕರೆಯತ್ತಿದ್ದರೆಂದು
ಕರೆಯಬಹುದು
ಕರೆಯಲಾಗತ್ತಿತ್ತು
ಕರೆಯಲಾಗಿ
ಕರೆಯಲಾಗಿದೆ
ಕರೆಯಲಾಗಿದೆೆ
ಕರೆಯಲಾಗಿದ್ದು
ಕರೆಯಲಾಗುತ್ತದೆ
ಕರೆಯಲಾಗುತ್ತಿತು
ಕರೆಯಲಾಗುತ್ತಿತೆಂದೂ
ಕರೆಯಲಾಗುತ್ತಿತ್ತು
ಕರೆಯಲಾಗುತ್ತಿತ್ತೆಂದು
ಕರೆಯಲಾಗುತ್ತಿತ್ತೆಂದೂ
ಕರೆಯಲಾಯಿತು
ಕರೆಯಲಾಯಿತೆಂದು
ಕರೆಯಲ್ಪಡುತ್ತಿದ್ದ
ಕರೆಯಲ್ಪಡುವ
ಕರೆಯಿಸಿಕೊಂಡು
ಕರೆಯುತ್ತಾರೆ
ಕರೆಯುತ್ತಿದ್ದನೆಂದು
ಕರೆಯುತ್ತಿದ್ದರು
ಕರೆಯುತ್ತಿದ್ದರೆಂದು
ಕರೆಯುವುದರ
ಕರೆಯುವುದು
ಕರೆಸಿ
ಕರ್ಕನನ್ನು
ಕರ್ಣನು
ಕರ್ಣವೃತ್ತಾಂತದ
ಕರ್ಣಾಟ
ಕರ್ಣಾಟಕ
ಕರ್ಣಾಟಕದ
ಕರ್ಣಾಟಕರ್ಣಾಟಕಕರ್ನಾಟಕರ್ನಾಟಕ
ಕರ್ಣಾಟೇಶ್ವರರಾಯ
ಕರ್ಣ್ನಾಟಧರಾಮರೋತ್ತಂಸಂ
ಕರ್ತನಾದ
ಕರ್ತನಾದಅಧಿಕಾರಿ
ಕರ್ತರಾದ
ಕರ್ತವ್ಯ
ಕರ್ತವ್ಯಗಳನ್ನು
ಕರ್ತವ್ಯವಾಗಿತ್ತು
ಕರ್ತೃ
ಕರ್ನಲ್
ಕರ್ನಾಟ
ಕರ್ನಾಟಕ
ಕರ್ನಾಟಕದ
ಕರ್ನಾಟಕದಲ್ಲಿ
ಕರ್ನಾಟಕದಲ್ಲಿದ್ದ
ಕರ್ನಾಟಕರ್ಣ್ನಾಟಕ
ಕರ್ನಾಟಕವನ್ನು
ಕರ್ನಾಟಕೇ
ಕರ್ನಾಟಕೇಂದುವಾಗಿದ್ದನೆಂದು
ಕರ್ನಾಟಲಕ್ಷ್ಮೀ
ಕರ್ನಾಟಿಕಾ
ಕರ್ನಾಟಿಕಾದ
ಕರ್ನಾಟಿಕ್
ಕರ್ನಾಟೇಶ್ವರರಾಯ
ಕರ್ನ್ನರುಂಮಪ
ಕರ್ನ್ನಾಟ
ಕರ್ನ್ನಾಟಕ
ಕರ್ಪೂರದಾರತಿಯ
ಕರ್ಬಪ್ಪುಕಳ್ಬಪ್ಪು
ಕರ್ಮಗರಾಜ
ಕರ್ಮಟೇಶ್ವರ
ಕರ್ಮವಿಪಾಕ
ಕರ್ಮಾಚ್ಯುತೇಂದ್ರಃ
ಕರ್ಮ್ಮಗರಾಚ
ಕರ್ಮ್ಮಗರಾಚನು
ಕರ್ಮ್ಮಠೇಶ್ವರ
ಕರ್ಮ್ಮನ
ಕರ್ಮ್ಮವಿಪಾಕ
ಕಱಿಗಟ್ಟಿದ
ಕಲಕುಣಿ
ಕಲಚುರಿ
ಕಲಬುರ್ಗಿಯವರ
ಕಲಬುರ್ಗಿಯವರು
ಕಲವೂರ
ಕಲಶ
ಕಲಸ್ತವಾಡಿ
ಕಲಹದ
ಕಲಹದಲ್ಲಿ
ಕಲಹಳಿಯನು
ಕಲಹವಾಗಿ
ಕಲಿ
ಕಲಿಕಣಿ
ಕಲಿಕಣಿನಾಡ
ಕಲಿಕಾಲಧರ್ಮ್ಮರಾಜ
ಕಲಿಕಾಳೇಸ್ಮಿನ್ಗಂಗಮಂಡಲ
ಕಲಿಗಳಂಕುಸ
ಕಲಿಗಳನ್ನು
ಕಲಿತನದ
ಕಲಿತು
ಕಲಿದೇವ
ಕಲಿದೇವನ
ಕಲಿದೇವನಹಳ್ಳಿ
ಕಲಿದೇವರ
ಕಲಿದೇವರಿಗೆ
ಕಲಿನಾಥಪುರ
ಕಲಿನೊಳಂಬಾದಿ
ಕಲಿನೊಳಂಬಾದಿರಾಜ
ಕಲಿನೊಳಂಬಾದಿರಾಜನ
ಕಲಿನೊಳಂಬಾದಿರಾಜನಾಗಿದ್ದು
ಕಲಿನೊಳಂಬಾದಿರಾಜನು
ಕಲಿನೊಳಂಬಾದಿರಾಜನೆಂದು
ಕಲಿನೊಳಂಬಾಧಿರಾಜನು
ಕಲಿಯ
ಕಲಿಯಂಣನ
ಕಲಿಯಣ್ಣ
ಕಲಿಯಣ್ಣನು
ಕಲಿಯರಗಂಡ
ಕಲಿಯುಗಭೀಮನೆಂದು
ಕಲಿಯುಗಭೀಮಾರ್ಹಗೇಹಾದಿ
ಕಲಿಯುಗಮಾತ್ತಂಡನುಂ
ಕಲಿಯೂರಿನ
ಕಲಿಯೂರು
ಕಲಿರತ್ನಪಾಲನ
ಕಲಿರತ್ನಪಾಲನನ್ನು
ಕಲಿರತ್ನಪಾಲನು
ಕಲಿರತ್ನಪಾಳನ
ಕಲಿವರ್ಷವನ್ನು
ಕಲಿಹೃದುವ
ಕಲುಕಣಿ
ಕಲುಕಣಿಕುಣಿ
ಕಲುಕಣಿನಾಡ
ಕಲುಕಣಿನಾಡಿಗೆ
ಕಲುಕಣಿನಾಡಿನ
ಕಲುಕಣಿನಾಡು
ಕಲುಕಣಿಯ
ಕಲುಕಣಿಯೆಪ್ಪತ್ತಕ್ಕೆ
ಕಲುಕರೆ
ಕಲುಕುಣಿ
ಕಲೆಗಾರರಿಗೆಇವರು
ಕಲ್ಕಣಿ
ಕಲ್ಕಣಿನಾಡ
ಕಲ್ಕಣಿನಾಡು
ಕಲ್ಕರೆಕಲ್ಕುಣಿ
ಕಲ್ಕಱೆ
ಕಲ್ಕುಣಿ
ಕಲ್ಕುಣಿಕಾಲುಕಣಿ
ಕಲ್ಕುಣಿಯ
ಕಲ್ಕುಣಿಯಲ್ಲಿ
ಕಲ್ನಾಟ್ಟಾಗಿ
ಕಲ್ನಾಡಾಗಿ
ಕಲ್ಪನೆ
ಕಲ್ಪನೆಯಾಗಿದೆ
ಕಲ್ಪಯತ್
ಕಲ್ಪಿಸುವುದಿಲ್ಲ
ಕಲ್ಯ
ಕಲ್ಯದಲ್ಲಿ
ಕಲ್ಯಾಣ
ಕಲ್ಯಾಣಚಾಲುಕ್ಯ
ಕಲ್ಯಾಣಚಾಲುಕ್ಯರ
ಕಲ್ಯಾಣದ
ಕಲ್ಯಾಣಿ
ಕಲ್ಯಾಣಿಯ
ಕಲ್ಲ
ಕಲ್ಲಕೆರೆಯ
ಕಲ್ಲಗುಂಡಿಯಹಳ್ಳಿಯನು
ಕಲ್ಲದೇಗುಲವನ್ನು
ಕಲ್ಲನೆಟ್ಟು
ಕಲ್ಲನ್ನು
ಕಲ್ಲಬಾಗಿಲ
ಕಲ್ಲಬ್ಬರಸಿಯ
ಕಲ್ಲಯ್ಯಕಾಳಯ್ಯ
ಕಲ್ಲಹಳ್ಳಿ
ಕಲ್ಲಹಳ್ಳಿಯ
ಕಲ್ಲಹಳ್ಳಿಯನ್ನು
ಕಲ್ಲಹಳ್ಳಿಯೇ
ಕಲ್ಲಹಳ್ಳಿಶಾಸನದಲ್ಲಿ
ಕಲ್ಲಾಗಿರಬಹದು
ಕಲ್ಲಿಂದ
ಕಲ್ಲಿದೇವನ
ಕಲ್ಲಿನ
ಕಲ್ಲಿನಕ್ರಮ
ಕಲ್ಲಿನಬೆಟ್ಟ
ಕಲ್ಲಿನಲ್ಲಿ
ಕಲ್ಲು
ಕಲ್ಲುಗಳಿಂದ
ಕಲ್ಲುಗುಂಡಿಯನ್ನು
ಕಲ್ಲುಗುಡ್ಡೆ
ಕಲ್ಲುಬಂಡೆಗಳಿಂದ
ಕಲ್ಲುಬಂಡೆಗಳು
ಕಲ್ಲುಮಂಟಿ
ಕಲ್ಲುಮಂಟಿಗಳಿಂದ
ಕಲ್ಲುಮರಡಿಯೊಳಾಡುವುದೇ
ಕಲ್ಲುಮಸೀತಿಗೆ
ಕಲ್ಲುಮಸೀದಿಗೆ
ಕಲ್ಲೆಯನಾಯಕ
ಕಳಚುರಿ
ಕಳಚುರ್ಯರ
ಕಳನಾಗಿ
ಕಳನಿ
ಕಳಭ್ರ
ಕಳಲದ
ಕಳಲೆ
ಕಳಲೆಯ
ಕಳಶನೂ
ಕಳಸ
ಕಳಸತ್ತು
ಕಳಸದಂತಿದ್ದನು
ಕಳಸನಿರ್ವಾಣಗೆಯ್ಸಿ
ಕಳಸ್ತವಾಡಿ
ಕಳಾಭ್ಯಸ್ತರಿಗೆ
ಕಳಿಂಗ
ಕಳಿಯೂರ
ಕಳುಹಿಸಲಾಗಿತ್ತು
ಕಳುಹಿಸಿ
ಕಳುಹಿಸಿದ
ಕಳುಹಿಸಿದನು
ಕಳುಹಿಸಿದನೆಂದು
ಕಳುಹಿಸಿದ್ದ
ಕಳುಹಿಸುತ್ತಾನೆ
ಕಳುಹಿಸುತ್ತಾರೆ
ಕಳೆದ
ಕಳೆದಿರಬಹುದು
ಕಳೆದುಕೊಂಡಿರಬಹುದು
ಕಳ್
ಕಳ್ಬಪ್ಪ
ಕಳ್ಬಪ್ಪಿನಾ
ಕಳ್ಬಪ್ಪು
ಕಳ್ಳನಕೆರೆ
ಕಳ್ಳರು
ಕಳ್ಳರ್ವಾಡಿ
ಕಳ್ಳರ್ವಾಡಿಯು
ಕಳ್ವಪ್ಪಿನಾ
ಕಳ್ವಪ್ಪು
ಕಳ್ವಪ್ಪುನಾಡು
ಕವ
ಕವಡಯ್ಯ
ಕವನಗಳಿದ್ದು
ಕವಿ
ಕವಿಗಳಿಗೂ
ಕವಿಗಳು
ಕವಿಗೆ
ಕವಿಚರಿತೆಯನ್ನು
ಕವಿದಿದ್ದ
ಕವಿಬುಧಾರ್ತಿಂ
ಕವಿಯಲು
ಕವಿಯೂ
ಕವುಂಗಿನ
ಕವುಂಗು
ಕಷ್ಟ
ಕಷ್ಟವಾಗುತ್ತದೆ
ಕಸಬ
ಕಸಬಾ
ಕಸಲಗೆರೆ
ಕಸಲಗೆರೆಗೆ
ಕಸಲಗೆರೆಯ
ಕಸಲಗೆರೆಯೇ
ಕಸವಯ್ಯ
ಕಸಿದನು
ಕಸಿದುಕೊಂಡು
ಕಹಿನ
ಕಾಂಚಿಕಾಂಚನ
ಕಾಂಚಿಗೊಂಡ
ಕಾಂಚಿಪುರವನ್ನು
ಕಾಂಚೀಪುರದ
ಕಾಂತಯ್ಯನವರ
ಕಾಂತಾ
ಕಾಂತಿರಾಜಿಷ್ಣು
ಕಾಂತೈಯ್ಯನವರ
ಕಾಂದು
ಕಾಂಭೋಜಭೋಜಕಾಲಿಂಗಕರಹಾತಾದಿಪಾರ್ಥಿವೈಃ
ಕಾಕಡೆ
ಕಾಕತೀಯ
ಕಾಕನಹಳ್ಳಿ
ಕಾಗದಪತ್ರ
ಕಾಗೆ
ಕಾಗೆಗಳು
ಕಾಚಿದೇವ
ಕಾಚಿದೇವನ
ಕಾಚಿದೇವನು
ಕಾಚಿನಾಯಕ
ಕಾಚಿಯ
ಕಾಚೀದೇವ
ಕಾಚೀದೇವನ
ಕಾಚೀದೇವನನ್ನು
ಕಾಚೀದೇವನು
ಕಾಚೀದೇವನೆಂದು
ಕಾಚೆಯ
ಕಾಟ
ಕಾಡಂಕಾಖ್ಯಪುರವನ್ನು
ಕಾಡಕ್ಕಿಯನ್ನು
ಕಾಡನ್ನು
ಕಾಡಯನಾಯಕನು
ಕಾಡವರಾಯ
ಕಾಡಾನೆಯು
ಕಾಡಿಗೆ
ಕಾಡಿನ
ಕಾಡಿನಲ್ಲಿ
ಕಾಡಿನಲ್ಲಿದ್ದ
ಕಾಡಿಯೂರನ್ನು
ಕಾಡಿಯೂರಿನತ್ತ
ಕಾಡು
ಕಾಡುಕೊತ್ತನಹಳ್ಳಿ
ಕಾಡುಕೊತ್ತನಹಳ್ಳಿಯ
ಕಾಡುಕೊತ್ತಹಳ್ಳಿ
ಕಾಡುಗಳಲ್ಲಿ
ಕಾಡುಗಳಾಗಿ
ಕಾಡುಗಳಿದ್ದು
ಕಾಡುಪ್ರಾಣಿಗಳಿವೆ
ಕಾಡುಮೆಣಸಿಗೆಕಾಡುಮೆಣಸ
ಕಾಡುವಿಟ್ಟಿಯ
ಕಾಡುವಿಟ್ಟಿಯನ್ನು
ಕಾಡುವಿಟ್ಟಿಯು
ಕಾಡುಹಂದಿ
ಕಾಡೆಮ್ಮೆ
ಕಾಡೇ
ಕಾಣಬಹುದು
ಕಾಣಲಿಲ್ಲವಲ್ಲ
ಕಾಣಿಕೆಗಳನ್ನು
ಕಾಣಿಕೆಯ
ಕಾಣಿಕೆಯನ್ನು
ಕಾಣಿಸಕೊಂಡಾಗ
ಕಾಣಿಸಕೊಂಡಿದೆ
ಕಾಣಿಸಕೊಳ್ಳುತ್ತದೆ
ಕಾಣಿಸಕೊಳ್ಳುತ್ತವೆ
ಕಾಣಿಸಿ
ಕಾಣಿಸಿಕೊಂಡವು
ಕಾಣಿಸಿಕೊಂಡಿತು
ಕಾಣಿಸಿಕೊಂಡಿದೆ
ಕಾಣಿಸಿಕೊಂಡಿದ್ದು
ಕಾಣಿಸಿಕೊಂಡಿಲ್ಲ
ಕಾಣಿಸಿಕೊಳ್ಳುತ್ತದೆ
ಕಾಣಿಸಿಕೊಳ್ಳುತ್ತದೆಂದು
ಕಾಣಿಸಿಕೊಳ್ಳುತ್ತಾರೆ
ಕಾಣಿಸಿಕೊಳ್ಳುವ
ಕಾಣಿಸಿಕೊಳ್ಳುವಾತ
ಕಾಣಿಸಿಕೊಳ್ಳುವುದಿಲ್ಲ
ಕಾಣಿಸಿಕೊಳ್ಳುವುದು
ಕಾಣಿಸುವುದಿಲ್ಲವೆಂದು
ಕಾಣುತ್ತದೆ
ಕಾಣುತ್ತವೆ
ಕಾಣುತ್ತಿದ್ದರು
ಕಾಣುತ್ತಿದ್ದವು
ಕಾಣೆಮೆ
ಕಾದಾಡಿ
ಕಾದಾಡಿದ
ಕಾದಾಡಿದ್ದಾರೆ
ಕಾದಿ
ಕಾದಿಕೊಂದು
ಕಾದಿದರೆಂದು
ಕಾದಿಬಿದ್ದಾಗ
ಕಾದುವಲಿ
ಕಾದುವುದೆಂದು
ಕಾನನ
ಕಾನನಂ
ಕಾನೀನನೆನಿಸಿ
ಕಾನ್ಸ್ಟಾಂಟಿನೋಪಲ್ಗೆ
ಕಾಪಾಡಿದವನು
ಕಾಫುರನು
ಕಾಬೈಯನು
ಕಾಮ
ಕಾಮಕೋಟಿದೇವಿ
ಕಾಮಗವುಡನ
ಕಾಮಗೆರೆ
ಕಾಮಗೆರೆಯ
ಕಾಮತಮ್ಮ
ಕಾಮತ್
ಕಾಮತ್ರವರು
ಕಾಮದೇವನ
ಕಾಮನಹಳ್ಳಿ
ಕಾಮಪ್ಪನಾಯಕನೆಂಬ
ಕಾಮಯ್ಯ
ಕಾಮಯ್ಯನು
ಕಾಮಲದೇವಿ
ಕಾಮಿಕಬ್ಬೆ
ಕಾಮಿನೀನಾಂ
ಕಾಮಿಯಕ್ಕ
ಕಾಮೆನಾಯಕನಹಳ್ಳಿ
ಕಾಮೆಯ
ಕಾಮೆಯದಂಡನಾಯಕನು
ಕಾಮೆಯದಣ್ನಾಯಕರ
ಕಾಮೆಯನಾಯಕ
ಕಾಮೆಯನಾಯಕನ
ಕಾಮೆಯನಾಯಕನಹಳ್ಳಿ
ಕಾಮೆಯನಾಯಕನು
ಕಾಮೆಯನಾಯಕರ
ಕಾಯಕದಲ್ಲಿದ್ದರು
ಕಾಯುತ್ತಾ
ಕಾಯುತ್ತಿದ್ದ
ಕಾರ
ಕಾರಕೂನನೂ
ಕಾರಕೂನರು
ಕಾರಕ್ಕೆ
ಕಾರಗನಹಳ್ಳಿ
ಕಾರಣ
ಕಾರಣಕ್ಕಾಗಿ
ಕಾರಣಕ್ಕಾಗಿಯೆ
ಕಾರಣದಿಂದ
ಕಾರಣದಿಂದಾಗಿಯೇ
ಕಾರಣನಾದನೆಂದೂ
ಕಾರಣರಾದ
ಕಾರಣವಾಗಿತ್ತೆಂಬುದು
ಕಾರಣವಾಗಿರಬಹುದು
ಕಾರಣವಾದಂತಿದೆ
ಕಾರಣವಿರಬಹುದು
ಕಾರಬಯಲಿನ
ಕಾರಬಯಲು
ಕಾರಯಿತ್ವಾಜಜಾಖ್ಯಕಾಂ
ಕಾರಸವಾಡಿ
ಕಾರಾಗೃಹದಲ್ಲಿಟ್ಟು
ಕಾರಿಕುಡಿ
ಕಾರಿಮಂಗಲನಾಡ
ಕಾರು
ಕಾರುಕನಕೊಳ್ಳ
ಕಾರುಗಹಳ್ಳಿಯ
ಕಾರುಣ್ಯಂಗೆಯ್ದು
ಕಾರುಣ್ಯದಿಂದ
ಕಾರೈಕುಡಿ
ಕಾರೈಕುಡಿಯ
ಕಾರ್ತೀಕ
ಕಾರ್ಮನ
ಕಾರ್ಯಕರ್ತ
ಕಾರ್ಯಕರ್ತನ
ಕಾರ್ಯಕರ್ತನಾದ
ಕಾರ್ಯಕರ್ತನೂ
ಕಾರ್ಯಕೆ
ಕಾರ್ಯಕೆಕರ್ತ
ಕಾರ್ಯಕೆಕರ್ತನಾದ
ಕಾರ್ಯಕೆಕರ್ತರನ್ನು
ಕಾರ್ಯಕೆಕರ್ತರಾದ
ಕಾರ್ಯಕೆಕರ್ತರು
ಕಾರ್ಯಕೆಕರ್ತಹೆಸರು
ಕಾರ್ಯಕ್ಕಾಗಿ
ಕಾರ್ಯಕ್ಕೆ
ಕಾರ್ಯಕ್ಷೇತ್ರವನ್ನಾಗಿ
ಕಾರ್ಯಗಳ
ಕಾರ್ಯಗಳನ್ನು
ಕಾರ್ಯಗಳಲ್ಲಿ
ಕಾರ್ಯಗಳು
ಕಾರ್ಯದ
ಕಾರ್ಯದಲ್ಲಿ
ಕಾರ್ಯನಿಮಿತ್ತ
ಕಾರ್ಯನಿರ್ವಹಣೆಗೆ
ಕಾರ್ಯನಿರ್ವಹಣೆಯ
ಕಾರ್ಯನಿರ್ವಹಿಸುತ್ತಿದ್ದರೆಂದು
ಕಾರ್ಯಮಠದ
ಕಾರ್ಯವನಿರ್ವಹಿಸುತ್ತಿದ್ದರೆಂದು
ಕಾರ್ಯವನ್ನು
ಕಾರ್ಯಸ್ಥಾನವನ್ನಾಗಿ
ಕಾರ್ಯಾವಧಿಯು
ಕಾಲ
ಕಾಲಆಂಗ್ಲಭಾಷೆಯ
ಕಾಲಕ್ಕಾಗಲೇ
ಕಾಲಕ್ಕಿಂತ
ಕಾಲಕ್ಕೂ
ಕಾಲಕ್ಕೆ
ಕಾಲಕ್ಕೇ
ಕಾಲಗಳಲ್ಲಿ
ಕಾಲಜ್ಞಾನ
ಕಾಲಜ್ಞಾನವನ್ನು
ಕಾಲದ
ಕಾಲದಂದು
ಕಾಲದಲಿ
ಕಾಲದಲ್ಲಿ
ಕಾಲದಲ್ಲಿದ್ದ
ಕಾಲದಲ್ಲಿದ್ದನು
ಕಾಲದಲ್ಲಿದ್ದವನೆಂದು
ಕಾಲದಲ್ಲಿಯೂ
ಕಾಲದಲ್ಲಿಯೇ
ಕಾಲದಲ್ಲೂ
ಕಾಲದಲ್ಲೇ
ಕಾಲದವನಿರಬಹುದು
ಕಾಲದವನೆಂದು
ಕಾಲದವರೆಗೂ
ಕಾಲದವರೆಗೆ
ಕಾಲದಸುಪ್ರಸಿದ್ಧ
ಕಾಲದಿಂದ
ಕಾಲದಿಂದಲೂ
ಕಾಲದಿಂದಲೇ
ಕಾಲದ್ದಾಗಿದ್ದು
ಕಾಲದ್ದಿರಬಹುದೆಂದು
ಕಾಲದ್ಲಲಿ
ಕಾಲನಿರೂಪಣೆಯಲ್ಲಿ
ಕಾಲನ್ನು
ಕಾಲಮ್ರಿತ್ತು
ಕಾಲರಾಜ
ಕಾಲವನ್ನು
ಕಾಲವಾಗಿದೆ
ಕಾಲವಾದ
ಕಾಲವಾದನೆಂದು
ಕಾಲವು
ಕಾಲವೂ
ಕಾಲವೆ
ಕಾಲವೇ
ಕಾಲಸ
ಕಾಲಾಳುಗಳ
ಕಾಲಾವಧಿಯ
ಕಾಲುಕಣಿಯ
ಕಾಲುಕುಣಿಯ
ಕಾಲುಕುಣಿಯಲ್ಲಿ
ಕಾಲುವಲಿಗಳು
ಕಾಲುವಳಿ
ಕಾಲುವಳಿಯಾಗಿ
ಕಾಲುವಳ್ಳಿ
ಕಾಲುವಳ್ಳಿಗ
ಕಾಲುವಳ್ಳಿಗಳ
ಕಾಲುವಳ್ಳಿಗಳನ್ನು
ಕಾಲುವಳ್ಳಿಗಳಾದ
ಕಾಲುವಳ್ಳಿಯ
ಕಾಲುವಳ್ಳಿಯಾದ
ಕಾಲುವೆ
ಕಾಲುವೆಗಳಿಗೆ
ಕಾಲುವೆಯ
ಕಾಲುವೆಯನ್ನು
ಕಾಲುವೆಯೊಳಗೆ
ಕಾಲೈಯನಾಯಕ
ಕಾಲ್ಗಾಹಿನ
ಕಾಲ್ದಳದ
ಕಾಲ್ವೆ
ಕಾಳಗದಲಿ
ಕಾಳಗದಲ್ಲಿ
ಕಾಳಗವಾಗಿರಬಹದು
ಕಾಳನ
ಕಾಳನ್ರಿಪಾಳನ
ಕಾಳಬೋವನಹಳ್ಳಿಯನ್ನು
ಕಾಳಯ್ಯ
ಕಾಳಯ್ಯಂ
ಕಾಳಯ್ಯನು
ಕಾಳರಾಜನನ್ನು
ಕಾಳಲದೇವಿ
ಕಾಳಲದೇವಿಯರ
ಕಾಳಲೇಶ್ವರ
ಕಾಳಾಂಚಿಯ
ಕಾಳಾಂತಕನಲುತೆ
ಕಾಳಿ
ಕಾಳಿಂಗನಹಳ್ಳಿ
ಕಾಳಿಂಗರಾಮನಹಳ್ಳಿಯ
ಕಾಳಿಯಂ
ಕಾಳಿಯಕ್ಕ
ಕಾಳಿಯೆಂಬ
ಕಾಳುಪಳ್ಳಿಗಳನ್ನು
ಕಾಳುಪಳ್ಳಿಗಳು
ಕಾಳೆಯ
ಕಾಳೆಯನಾಯಕ
ಕಾಳೆಯನಾಯಕನ
ಕಾಳೆಯನಾಯಕನು
ಕಾವ
ಕಾವಣ್ಣ
ಕಾವಣ್ಣನೆಂಬುವವನೂ
ಕಾವನಹಳ್ಳಿ
ಕಾವನಹಳ್ಳಿಯನ್ನು
ಕಾವಪ್ಪ
ಕಾವರಾಜ
ಕಾವಲುಗಾರ
ಕಾವಿಧಾರಿಯಾಗಿ
ಕಾವೇಟಿರಂಗ
ಕಾವೇರಿ
ಕಾವೇರಿಗೆ
ಕಾವೇರಿನದಿ
ಕಾವೇರಿನದಿಗೆ
ಕಾವೇರಿಯ
ಕಾವೇರಿಯಿಂದ
ಕಾವೇರೀ
ಕಾವ್ಯದ
ಕಾವ್ಯದಿಂದ
ಕಾವ್ಯವನ್ನು
ಕಾಶಿಯಲ್ಲಿ
ಕಾಶೀರಾವ್ಗೆ
ಕಾಶ್ಯಪ
ಕಾಶ್ಯಪಗೋತ್ರದ
ಕಿಕ್ಕೇರಿ
ಕಿಕ್ಕೇರಿಕ್ಕೆ
ಕಿಕ್ಕೇರಿಗೆ
ಕಿಕ್ಕೇರಿನ್ನು
ಕಿಕ್ಕೇರಿಪುರದಲ್ಲಿ
ಕಿಕ್ಕೇರಿಯ
ಕಿಕ್ಕೇರಿಯನ್ನು
ಕಿಕ್ಕೇರಿಯಪುರದಲ್ಲಿ
ಕಿಟ್ಟೆಲ್ರವರು
ಕಿಡದಂತೆ
ಕಿಡಿಸಿದ
ಕಿತ್ತನಕೆರೆ
ಕಿತ್ತನಕೆರೆಯನ್ನು
ಕಿತ್ತಪ್ಪ
ಕಿತ್ತಪ್ಪದಂಡನಾಯಕನು
ಕಿತ್ತಿಪ್ಪ
ಕಿತ್ತುಕೊಂಡನು
ಕಿತ್ತುಕೊಂಡು
ಕಿತ್ತೂರನ್ನು
ಕಿಮೀ
ಕಿರಂಗೂರಿಗೆ
ಕಿರಂಗೂರಿನ
ಕಿರಂಗೂರೇ
ಕಿರಗತೂರ
ಕಿರಗಸೂರು
ಕಿರಾತಕರನ್ನು
ಕಿರಿಯ
ಕಿರಿಯದು
ಕಿರಿಯನಾದುದರಿಂದ
ಕಿರಿಯಯ್ಯ
ಕಿರಿಯವಯಸ್ಸಿನಲ್ಲೇ
ಕಿರುಕಾವಲನ್ನುಸೇನೆ
ಕಿರುಕಾವಲು
ಕಿರುಗಣಬ್ಬೆ
ಕಿರುಗವರ
ಕಿರುಗವರೆ
ಕಿರುಗವರೆಅ
ಕಿರುಗಾವಲನ್ನು
ಕಿರುಗಾವಲಾಗಿದೆ
ಕಿರುಗಾವಲಿನ
ಕಿರುಗಾವಲಿರಬಹುದು
ಕಿರುಗಾವಲು
ಕಿರುನಗರ
ಕಿರುನಗರವು
ಕಿರುನಗರವೇ
ಕಿರುಭಾಗ
ಕಿರುವೆಳ್ನಗರ
ಕಿರುವೆಳ್ನಗರದ
ಕಿರುವೆಳ್ನಗರವನ್ನು
ಕಿರುವೆಳ್ನಗರವು
ಕಿರುವೆಳ್ನನಗರ
ಕಿಳಲೆ
ಕಿಳಲೆನಾಡ
ಕಿಳಲೆನಾಡನ್ನು
ಕಿಳಲೆನಾಡು
ಕಿಳಲೆಸಹಸ್ರದ
ಕಿಳಲೈ
ಕಿಳಲೈನಾಟ್ಟ
ಕಿಳುವನಹಳ್ಳಿ
ಕಿಳುವನಹಳ್ಳಿಕೆರೆ
ಕಿವುಡನೂ
ಕಿಸುಕಾಡು
ಕಿಸುಕಾಡೆಪ್ಪತ್ತು
ಕೀರ್ತಿ
ಕೀರ್ತಿಅರಸರ
ಕೀರ್ತಿತನಾಗಿದ್ದಾನೆ
ಕೀರ್ತಿತನೂಭವ
ಕೀರ್ತಿದೇವ
ಕೀರ್ತಿದೇವನ
ಕೀರ್ತಿದೇವನಿಗೆ
ಕೀರ್ತಿದೇವನು
ಕೀರ್ತಿನಾರಾಯಣದೇವರ
ಕೀರ್ತಿನಾರಾಯಣದೇವರಿಗೆ
ಕೀರ್ತಿನಾರಾಯಣರಾಯ
ಕೀರ್ತಿಮಾನ್
ಕೀರ್ತಿಯರಸ
ಕೀರ್ತಿಯರಸನ
ಕೀರ್ತಿಯರಸನನನ್ನು
ಕೀರ್ತಿಯರಸನಿಗೆ
ಕೀರ್ತಿಯರಸರ
ಕೀರ್ತಿಯು
ಕೀರ್ತಿರಾಜುವಿಗೆ
ಕೀರ್ತಿರ್ಹರತಿ
ಕೀರ್ತಿವಂತನಾಗಿದ್ದನೆಂದು
ಕೀರ್ತಿವಂತನೆಂದೂ
ಕೀರ್ತಿವಿಲಾಸಿಯಾಗಿ
ಕೀರ್ತಿಸಮುದ್ರ
ಕೀರ್ತೌ
ಕೀರ್ತ್ತಿಗ
ಕೀರ್ತ್ತಿದೇವ
ಕೀರ್ತ್ತಿದೇವಂಗಳು
ಕೀರ್ತ್ತಿರಾಜನ
ಕೀರ್ತ್ಯಂಗನವಲ್ಲಭ
ಕೀರ್ತ್ಯಾವತಾರವೆನ್ತೆಂದಡೆ
ಕೀರ್ತ್ರಯಾಂ
ಕೀಲಾರ
ಕೀಳಿನಿ
ಕೀಳ್
ಕುಂಚಗನಹಳ್ಳಿ
ಕುಂಚದಹಳ್ಳಿ
ಕುಂಚನಹಳ್ಳಿಗಳ
ಕುಂಚನಹಳ್ಳಿಯನ್ನು
ಕುಂಚಿಕೊಂಡ
ಕುಂಚಿಗನಹಳ್ಳಿ
ಕುಂಚಿಗನಹಳ್ಳಿಬೇಚಿರಾಕ್
ಕುಂಚಿಗರ
ಕುಂಚಿಯ
ಕುಂಜರ
ಕುಂಟ
ಕುಂತಲ
ಕುಂತಲಗಳ
ಕುಂತಲಗಳು
ಕುಂತಲವು
ಕುಂತಲವೆಂದೂ
ಕುಂತಲೇಂದ್ರ
ಕುಂತಿ
ಕುಂತಿಬೆಟ್ಟ
ಕುಂತಿಬೆಟ್ಟದ
ಕುಂತಿಬೆಟ್ಟದಲ್ಲಿ
ಕುಂತೂರನ್ನು
ಕುಂತೂರುಮಠದ
ಕುಂದಘಟ್ಟ
ಕುಂದಣ
ಕುಂದನಹಳ್ಳಿ
ಕುಂದನ್ನಾಡಿನ
ಕುಂದನ್ನಾಡು
ಕುಂದನ್ನಾಡುಗಳಾದ್ದಿರಬಹುದೆಂದು
ಕುಂದಸತ್ತಿ
ಕುಂದಾಚಿ
ಕುಂದಾಚಿಯು
ಕುಂದಾಚ್ಚಿ
ಕುಂದಾಚ್ಚಿಯರ
ಕುಂದೂರನ್ನು
ಕುಂದೂರು
ಕುಂದೇಂದುಮಂದಾಕಿನೀವಿಶದಯಶಂ
ಕುಂನೆಯನಾಯಕ
ಕುಂನ್ದ
ಕುಂಪೆನಾಡಾಳುವವನನ್ನು
ಕುಂಪೆನಾಡು
ಕುಂಬೇನಹಳ್ಳಿಯಲ್ಲಿ
ಕುಂಬೇನಹಳ್ಳಿಯು
ಕುಂಭಸ್ಥಳಕ್ಕೆ
ಕುಂಭಸ್ಥಳವನ್ನು
ಕುಂಭಸ್ಥಳವು
ಕುಂಮಟ
ಕುಕನೂರು
ಕುಕ್ಕನೂರು
ಕುಟುಂಬದಲ್ಲಿ
ಕುಟುಂಬದವರೂ
ಕುಟುಂಬನಾಮವನ್ನು
ಕುಟ್ಟಾಡಿ
ಕುಡಗಬಾಳ
ಕುಡುಗುಕೋಲಾಹಲ
ಕುಡುಗುನಾಡ
ಕುಡುಗುನಾಡನ್ನು
ಕುಡುಗುನಾಡುಗಳನ್ನು
ಕುಡುಗುನಾಡೊಳಗಣ
ಕುಡುಗುಬಾಳು
ಕುಡೆ
ಕುಣರಪಾಕಂ
ಕುಣಿಂಗಲ
ಕುಣಿಂಗಲವೂರ
ಕುಣಿಂಗಲವೂರಇಂದಿನ
ಕುಣಿಂಗಲವೂರಿನ
ಕುಣಿಂಗಲವೂರಿನಲ್ಲಿ
ಕುಣಿಂಗಲಾಚಾರ
ಕುಣಿಗಲ
ಕುಣಿಗಲು
ಕುಣಿಗಲ್
ಕುಣಿಗಲ್ನಲ್ಲಿ
ಕುಣಿಗಲ್ನಾಡಿನ
ಕುಣಿಗಲ್ಲು
ಕುತನೀಯೆ
ಕುತುಬ್ಷಾ
ಕುತೂಹಲಕರ
ಕುತೂಹಲಭರಿತ
ಕುತ್ತಾಲ
ಕುತ್ತುವ
ಕುತ್ತೂರು
ಕುದಿಹೆರುಕುದೇರು
ಕುದುರೆಗಳನ್ನು
ಕುದುರೆಗುಂಡಿಯನ್ನು
ಕುದುರೆಯ
ಕುದ್ದಾಳ
ಕುದ್ಧಾಳ
ಕುನ್ದಗಾಮುಣ್ಡರು
ಕುನ್ದನ್ನಾಡನ್ನು
ಕುನ್ದನ್ನಾಡು
ಕುನ್ದಸತ್ತಿ
ಕುನ್ದಸತ್ತಿಅರಸನು
ಕುನ್ದುನಾಟ್ಟು
ಕುನ್ದುನಾಡಾಳ್ವ
ಕುನ್ದೂರು
ಕುನ್ನಂಪಾಕಂ
ಕುನ್ನಪಾಕಂ
ಕುನ್ನಲಬೊಪ್ಪ
ಕುನ್ನಿಯ
ಕುಪಂಣ
ಕುಪ್ಪಣ್ಣ
ಕುಪ್ಪಾಲ್ಮಾದವರುಂ
ಕುಪ್ಪೆಮಂಚನಹಳ್ಳಿ
ಕುಪ್ಪೆಮದ್ದೂರನ್ನು
ಕುಪ್ಪೆಯನ್ನು
ಕುಬೇರಪುರ
ಕುಮಾರ
ಕುಮಾರಗೋವಿಯಂಣ್ನನ
ಕುಮಾರನಾದ
ಕುಮಾರನೆಂದು
ಕುಮಾರರ
ಕುಮಾರರಗಂಡ
ಕುಮಾರರಾದ
ಕುಮಾರರಾಮನ
ಕುಮಾರರು
ಕುಮಾರವೃತ್ತಿಯಿಂದ
ಕುಮಾರಸ್ವಾಮಿಯವರ
ಕುಮಾರಹೆಗ್ಗಡೆದೇವ
ಕುಮ್ಮಟದುರ್ಗದ
ಕುರವಂಕನಾಡ
ಕುರಿತಂತೆ
ಕುರಿತು
ಕುರುಕಿಮಾಳೆಯರ
ಕುರುಕ್ಕಿ
ಕುರುಚಲು
ಕುರುಣೆಯನಹಳ್ಳಿಯನ್ನು
ಕುರುಣೆಯನಹಳ್ಳಿಯು
ಕುರುಣೇನಹಳ್ಳಿ
ಕುರುನಂದನರ
ಕುರುಬರಕಾಳೇನಹಳ್ಳಿ
ಕುರುಭೂಮಿಯಲ್ಲಿ
ಕುರುವಂಕ
ಕುರುವಂಕದ
ಕುರುವಂಕನಾಡ
ಕುರುವಂಕನಾಡಿಗೆ
ಕುರುವಂಕನಾಡಿನ
ಕುರುವಂಕನಾಡು
ಕುರುವಂದ
ಕುರುವಂದಕುಲ
ಕುರುವಂದಕುಲಕಮಲಮಾರ್ತಾಂಡ
ಕುರುವಂದಕುಲೈಕಭೂಷಣನೆನಿಸಿದ
ಕುರುವಮಕ
ಕುರುವೈ
ಕುರುಹುಗಳಿವೆ
ಕುರುಹುಗಳು
ಕುರುಹೆಂದು
ಕುರ್ತಕೋಟಿ
ಕುರ್ರಬೂಲಾಡು
ಕುರ್ವಂಕನಾಡಿನ
ಕುರ್ವಂಕಸ್ಥಳದ
ಕುರ್ವ್ವಂಕನಾಡ
ಕುಱಡುವ
ಕುಱುವಂದೇಶ್ವರ
ಕುಲ
ಕುಲಕ
ಕುಲಕಮಲ
ಕುಲಕರಣಿಗಳು
ಕುಲಕರ್ಣಿ
ಕುಲಕೆ
ಕುಲಕ್ಕೆ
ಕುಲಕ್ಷತ್ರಿಯ
ಕುಲಗಳು
ಕುಲಗಾಣಾ
ಕುಲತಿಲಕ
ಕುಲತಿಲಕನಾಗಿದ್ದಾನೆ
ಕುಲತಿಲಕರಾದ
ಕುಲದ
ಕುಲದಲ್ಲಿ
ಕುಲದವನಾಗಿದ್ದು
ಕುಲದವರು
ಕುಲದೀಪಕನಾದ
ಕುಲದೀಪನುಮೆನಿಪ
ಕುಲದೇವರಾದ
ಕುಲಧವಳ
ಕುಲವನಹಳ್ಳದಲ್ಲಿ
ಕುಲವೆಂಬ
ಕುಲಶೇಖರನೆಂಬ
ಕುಲಾಂತಕ
ಕುಲಾನ್ವಯದ
ಕುಲಾನ್ವಯರಾದ
ಕುಲಾನ್ವಯರುಂ
ಕುಲುಮೆಯಲ್ಲಿ
ಕುಲೋತ್ತುಂಗ
ಕುಲೋತ್ತುಂಗಚೋಳನ
ಕುಲೋತ್ತುಂಗನಿಗೆ
ಕುಲೋದ್ಧರಣ
ಕುಲೋದ್ಭವನೂ
ಕುಳ
ಕುಳಕರಣಿ
ಕುಳಕರಣಿಕರನ್ನು
ಕುಳಕರಣಿಗಳ
ಕುಳಕರ್ಣಿ
ಕುಳತಿಳಕ
ಕುಳತ್ತೂರು
ಕುಳವ
ಕುಳವಕಟ್ಟಿಸಿ
ಕುಳಸುಂಕವನ್ನು
ಕುಳಾಂತಕ
ಕುಳಿ
ಕುಳಿಗಳಿಂದ
ಕುಳಿತ
ಕುಳಿತಿದ್ದ
ಕುಳಿತಿರುತ್ತಿದ್ದರು
ಕುಳಿತಿರುವ
ಕುಳಿಯಿಂದ
ಕುಳ್ಳಿರ್ದ್ದು
ಕುವರ
ಕುವಳಾಲಪುರವರೇಶ್ವರ
ಕುವೆಂಪು
ಕುಶಲಮೆಂದಿರದೋಡಿದನೊಂದೆ
ಕುಶಲಮೆಂದೋಡಿದನೊಂದೆ
ಕುಶೇಶೆಯನ
ಕುಹುಯೋಗವನ್ನು
ಕೂಂಡಿ
ಕೂಂಡಿನಾಡಕುಹುಂಡಿ
ಕೂಂಬಡಿ
ಕೂಗಿದನು
ಕೂಚಿತಂದೆ
ಕೂಟ
ಕೂಟಕ್ಕೆ
ಕೂಟದೊಳು
ಕೂಟವನ್ನು
ಕೂಡಲುಕುಪ್ಟೆ
ಕೂಡಲುಕುಪ್ಪೆ
ಕೂಡಲೂರ
ಕೂಡಲೂರು
ಕೂಡಲೇ
ಕೂಡಾ
ಕೂಡಿತಪ್ಪುನಾಯಕರಗಂಡ
ಕೂಡಿತಪ್ಪುವ
ಕೂಡಿದ
ಕೂಡಿದ್ದ
ಕೂಡಿಸಂದರು
ಕೂಡುವನಾಯಕರ
ಕೂಡೆ
ಕೂಡೆಸೋಮೆಯ
ಕೂಡ್ಲಿ
ಕೂಡ್ಲುಕುಪ್ಪೆ
ಕೂತ್ತಗಾವುಂಡ
ಕೂತ್ತಾಂಡಿ
ಕೂತ್ತಾನ್
ಕೂಮ್ಬಡಿ
ಕೂರಯ
ಕೂರಯನಾಯಕನ
ಕೂರಯನಾಯಕನು
ಕೂರಲಗನ್ನು
ಕೂರಿಗಿಹಳ್ಳಿಯ
ಕೂರಿಸಲ್ಪಟ್ಟನು
ಕೂರಿಸಿದನು
ಕೂರೆಯನಾಯಕ
ಕೂರೆಯನಾಯಕನ
ಕೂರೆಯನಾಯಕರು
ಕೂರ್ಗಲ್ಲನ್ನು
ಕೂಲಿಗ್ಗೆರೆ
ಕೂಲಿಗ್ಗೆರೆಕೂಳಗೆರೆ
ಕೂಲಿಗ್ಗೆರೆಯು
ಕೂಳಣವಾಗಿ
ಕೂಸಂ
ಕೂಸಅಪರ್ಣ
ಕೂಸುಗಳೆಂದು
ಕೃತಕ
ಕೃತಜ್ಞಂ
ಕೃತಜ್ಞನಾದ
ಕೃತಯುಗದಲ್ಲಿ
ಕೃತವತಿ
ಕೃತಿ
ಕೃತಿಗಳ
ಕೃತಿಗಳನ್ನು
ಕೃತಿಗಳಲ್ಲಿ
ಕೃತಿಗಳಿಂದ
ಕೃತಿಗಳು
ಕೃತಿಯ
ಕೃತಿಯನ್ನು
ಕೃತಿಯಲ್ಲಿ
ಕೃತಿಯಾಗಿದೆ
ಕೃಪೆಯನ್ನು
ಕೃಷಿಪದ್ಧತಿ
ಕೃಷ್ಣ
ಕೃಷ್ಣಕಂಧರ
ಕೃಷ್ಣಕಂಧರನನ್ನು
ಕೃಷ್ಣಕಂಧರನುಮಂ
ಕೃಷ್ಣದೇವ
ಕೃಷ್ಣದೇವರಾಯ
ಕೃಷ್ಣದೇವರಾಯನ
ಕೃಷ್ಣದೇವರಾಯನನ್ನು
ಕೃಷ್ಣದೇವರಾಯನಿಂದ
ಕೃಷ್ಣದೇವರಾಯನಿಗೆ
ಕೃಷ್ಣದೇವರಾಯನು
ಕೃಷ್ಣದೇವರಾಯನೆಂದು
ಕೃಷ್ಣದೇವರಾಯನೇ
ಕೃಷ್ಣದೇವರಾಯಪಟ್ಟಣಕ್ಕೆ
ಕೃಷ್ಣದೇವಾಲಯದ
ಕೃಷ್ಣದೇವೊಡೆಯರ
ಕೃಷ್ಣನ
ಕೃಷ್ಣನನ್ನೇ
ಕೃಷ್ಣನಾಗುತ್ತಾನೆ
ಕೃಷ್ಣನಿಗಿಂತಲೂ
ಕೃಷ್ಣನಿಗೆ
ಕೃಷ್ಣನು
ಕೃಷ್ಣನೂ
ಕೃಷ್ಣನೇ
ಕೃಷ್ಣಪ್ಪ
ಕೃಷ್ಣಪ್ಪನವರ
ಕೃಷ್ಣಪ್ಪನಾಯಕನ
ಕೃಷ್ಣಪ್ಪನಾಯಕನಿಗೆ
ಕೃಷ್ಣಪ್ಪನಾಯಕನು
ಕೃಷ್ಣಪ್ಪನಾಯಕನೂ
ಕೃಷ್ಣಪ್ಪನಾಯಕರಿಗೆ
ಕೃಷ್ಣಮಹಾಧಿರಾಜನ
ಕೃಷ್ಣಮೂರ್ತಿ
ಕೃಷ್ಣಯ್ಯ
ಕೃಷ್ಣರಾಜ
ಕೃಷ್ಣರಾಜಒಡೆಯನು
ಕೃಷ್ಣರಾಜನ
ಕೃಷ್ಣರಾಜನಗರ
ಕೃಷ್ಣರಾಜನನ್ನು
ಕೃಷ್ಣರಾಜನು
ಕೃಷ್ಣರಾಜಪೇಟೆ
ಕೃಷ್ಣರಾಜಪೇಟೆಯ
ಕೃಷ್ಣರಾಜರು
ಕೃಷ್ಣರಾಜವಡರೈಯನವರು
ಕೃಷ್ಣರಾಜವಡೆಯರೈಯ್ಯಾನವರು
ಕೃಷ್ಣರಾಜಸಾಗರ
ಕೃಷ್ಣರಾಜಸಾಗರದ
ಕೃಷ್ಣರಾಜಸಾಗರದೊಳಗೆ
ಕೃಷ್ಣರಾಜು
ಕೃಷ್ಣರಾಜೊಡೆಯರ
ಕೃಷ್ಣರಾಜೊಡೆಯರು
ಕೃಷ್ಣರಾಯ
ಕೃಷ್ಣರಾಯನಾಯಕ
ಕೃಷ್ಣರಾಯನಾಯಕನು
ಕೃಷ್ಣರಾಯನು
ಕೃಷ್ಣರಾಯಪುರಗಳೆಂಬ
ಕೃಷ್ಣರಾಯಪುರವನ್ನಾಗಿ
ಕೃಷ್ಣರಾಯಪುರವೆಂಬ
ಕೃಷ್ಣರಾಯಮಹಾರಾಯನ
ಕೃಷ್ಣರಾಯರಕೆರೆಯ
ಕೃಷ್ಣರಾಯಸಮುದ್ರವೆಂದು
ಕೃಷ್ಣರಾಯೇ
ಕೃಷ್ಣರಾವ್
ಕೃಷ್ಣವರ್ಮಮಹಾಧಿರಾಜನು
ಕೃಷ್ಣವಿಲಾಸದ
ಕೃಷ್ಣವೇಣಿತೀರದಲ್ಲಿ
ಕೃಷ್ಣಾನದಿ
ಕೆಂಗಲ್ಕೊಪ್ಪಲಿನ
ಕೆಂಚಪನಾಯಕ
ಕೆಂಚಪನಾಯಕರು
ಕೆಂದನಹಾಳು
ಕೆಂದನಹಾಳುಕೆನ್ನಾಳು
ಕೆಂಪ
ಕೆಂಪದೇವಯ್ಯರಸನಿಗೆ
ಕೆಂಪನಂಜಮ್ಮಣ್ಣಿಗೆ
ಕೆಂಪನಂಜೇದೇವರಿಗೆ
ಕೆಂಪಬಯಿರರಸ
ಕೆಂಪುನಾಯಕ
ಕೆಂಪೇಗೌಡನಕೊಪ್ಪಲು
ಕೆಂಬಾಳಿಗೆ
ಕೆಂಬಾವಿ
ಕೆಂಬಾವಿಯ
ಕೆಂಬೊಳಲನ್ನು
ಕೆಂಬೊಳಲಿಗೆ
ಕೆಂಬೊಳಲು
ಕೆಅನಂತರಾಮು
ಕೆಆರ್ನಗರ
ಕೆಎಸ್ಶಿವಣ್ಣ
ಕೆಡಿಸಿ
ಕೆತ್ತಿಸಿದ
ಕೆನರಾ
ಕೆನ್ನ
ಕೆಬೆಟ್ಟಹಳ್ಳಿ
ಕೆಬ್ಬೆಹಳ್ಳಿ
ಕೆಯ್ದಾರ್
ಕೆರಗೋಡಿಗೆ
ಕೆರಗೋಡು
ಕೆರೆ
ಕೆರೆಏರಿಯ
ಕೆರೆಕಟ್ಟೆಗಳ
ಕೆರೆಕಟ್ಟೆಗಳನ್ನು
ಕೆರೆಕೋಡಿಕೆರಗೋಡು
ಕೆರೆಗಳನ್ನು
ಕೆರೆಗಳನ್ನೂ
ಕೆರೆಗೆ
ಕೆರೆಗೊಡಗೆಯಾಗಿ
ಕೆರೆಗೋಡ
ಕೆರೆಗೋಡಿನಾಡ
ಕೆರೆಗೋಡಿನಾಡಿನ
ಕೆರೆಗೋಡು
ಕೆರೆಯ
ಕೆರೆಯಕೆಳಗೆ
ಕೆರೆಯನ್ನು
ಕೆರೆಯನ್ನೂ
ಕೆರೆಯಲ್ಲಿ
ಕೆರೆಯಾಗಿರಬಹುದು
ಕೆರೆಯೂ
ಕೆರೆಹಳ್ಳಿ
ಕೆಲಗೆರೆಯ
ಕೆಲಬಲಗಳಲ್ಲಿದ್ದ
ಕೆಲವರ
ಕೆಲವರಂತೂ
ಕೆಲವರನ್ನು
ಕೆಲವರಿಗೆ
ಕೆಲವರು
ಕೆಲವು
ಕೆಲವುಕಾಲ
ಕೆಲವುಭಾಗ
ಕೆಲವೆಡೆ
ಕೆಲವೇ
ಕೆಲವೊಂದರಲ್ಲಿ
ಕೆಲವೊಂದು
ಕೆಲವೊಮ್ಮೆ
ಕೆಲಸ
ಕೆಲಸಕ್ಕಾಗಿ
ಕೆಲಸಗಳನ್ನು
ಕೆಲಸವನ್ನು
ಕೆಲಸವಾಗಿರುವುದು
ಕೆಲ್ಲಂಗೆರೆ
ಕೆಲ್ಲಂಗೆರೆಯ
ಕೆಲ್ಲಂಗೆರೆಯನು
ಕೆಲ್ಲಂಗೆರೆಯನ್ನು
ಕೆಲ್ಲಬಸದಿಯ
ಕೆಲ್ಲವತ್ತಿ
ಕೆಳಕಂಡ
ಕೆಳಕಂಡಂತೆ
ಕೆಳಗಣ
ಕೆಳಗಿನ
ಕೆಳಗಿನಂತಿದೆ
ಕೆಳಗಿನಂತಿವೆ
ಕೆಳಗಿನಂತೆ
ಕೆಳಗಿರುವ
ಕೆಳಗಿಳಿಸಿ
ಕೆಳಗಿಳಿಸಿದರು
ಕೆಳಗಿಳಿಸಿದುದಕ್ಕಾಗಿ
ಕೆಳಗೆ
ಕೆಳಗೆರೆ
ಕೆಳತಿರುಪತಿಯ
ಕೆಳದರ್ಜೆ
ಕೆಳದಿ
ಕೆಳದಿಯ
ಕೆಳದಿರಾಜರ
ಕೆಳಭಾಗದಲ್ಲಿಯೇ
ಕೆಳಲಿ
ಕೆಳಲಿನಾಡ
ಕೆಳಲೆ
ಕೆಳಲೆಕಿಳಲೆ
ಕೆಳಲೆನಾಡ
ಕೆಳಲೆನಾಡನ್ನು
ಕೆಳಲೆನಾಡಿನ
ಕೆಳಲೆನಾಡಿನಲ್ಲಿದ್ದವು
ಕೆಳಲೆನಾಡು
ಕೆಳಲೆಯ
ಕೆಳಲೆಯನಾಡ
ಕೆಳವಾಡಿ
ಕೆಳಸ್ಥರದಲ್ಲಿ
ಕೆಳಹಂತದ
ಕೆಳಹಂತದಲ್ಲಿ
ಕೆಳೆಯಬ್ಬರಸಿ
ಕೆಳೆಯಬ್ಬರಸಿಯು
ಕೆಸವಿನಕಟ್ಟೆ
ಕೆಸ್ತೂರು
ಕೇಂದ್ರ
ಕೇಂದ್ರಗಳಾಗಿದ್ದವು
ಕೇಂದ್ರಗಳಾಗಿರುವುದನ್ನು
ಕೇಂದ್ರಗಳು
ಕೇಂದ್ರಗಳೂ
ಕೇಂದ್ರವನ್ನಾಗಿ
ಕೇಂದ್ರವನ್ನಾಗಿರಿಸಿಕೊಂಡು
ಕೇಂದ್ರವನ್ನಾಗಿಸಿಕೊಂಡು
ಕೇಂದ್ರವನ್ನು
ಕೇಂದ್ರವಾಗಿ
ಕೇಂದ್ರವಾಗಿತ್ತು
ಕೇಂದ್ರವಾಗಿದೆ
ಕೇಂದ್ರವಾಗಿದ್ದ
ಕೇಂದ್ರವಾಗಿದ್ದರಿಂದ
ಕೇಂದ್ರವಾಗಿದ್ದು
ಕೇಂದ್ರವಾಗಿಸಿಕೊಂಡಿದ್ದ
ಕೇಂದ್ರವಾದ
ಕೇಂದ್ರವಾಯಿತು
ಕೇಂದ್ರಸ್ಥಳಗಳಲ್ಲಿ
ಕೇಂದ್ರಸ್ಥಳವನ್ನಾಗಿ
ಕೇಂದ್ರಸ್ಥಳವಾಗಿ
ಕೇಂದ್ರೀಯ
ಕೇಂದ್ರೀಯವೇ
ಕೇತ
ಕೇತಗಉಡ
ಕೇತಚಮೂಪತಿ
ಕೇತಣ
ಕೇತಣವಾಹಿನೀ
ಕೇತಣ್ಣ
ಕೇತನಹಟ್ಟಿ
ಕೇತನಹಳ್ಳಿ
ಕೇತನಹಳ್ಳಿಇಂದಿನ
ಕೇತನಹಳ್ಳಿಯನ್ನು
ಕೇತನು
ಕೇತಪ್ಪ
ಕೇತಮಗೆರೆ
ಕೇತಮಲ್ಲ
ಕೇತಯ್ಯ
ಕೇತಯ್ಯದಂಡನಾಯಕ
ಕೇತಯ್ಯನೂ
ಕೇತಲದೇವಿ
ಕೇತಲೇಶ್ವರ
ಕೇತವ್ವೆ
ಕೇತಿ
ಕೇತಿಗಾವುಂಡ
ಕೇತಿಸೆಟ್ಟಿ
ಕೇತೆಮಾದೆಯನಾಯಕ
ಕೇತೆಯ
ಕೇತೆಯಕೇತಚಮೂಪತಿ
ಕೇತೆಯದಂಡನಾಯಕನು
ಕೇತೆಯನಾಯಕ
ಕೇತ್ರ
ಕೇರಳವಡ್ಡಿಯ
ಕೇರಳಾಧಿಪತಿಯಾಗಿರ್ದೆ
ಕೇರಳಾಪುರವೆಂಬ
ಕೇರಳೇ
ಕೇರಳೇನಾಡಿನ
ಕೇರಹಳ್ಳಿಯ
ಕೇರಾಳನಾಯಕ
ಕೇರಾಳನಾಯಕನು
ಕೇರಾಳನಾಯಕನೆಂದಿದೆ
ಕೇರಾಳಪುರವು
ಕೇಳಲು
ಕೇಳಿ
ಕೇಳಿದ
ಕೇಳಿದನೆಂದು
ಕೇಳಿಪಡೆದನು
ಕೇಳಿಬರುವುದಿಲ್ಲವೆಂದೂ
ಕೇಳುತ್ತಿದ್ದನೆಂದು
ಕೇಳ್ದಿದಿರುವಂದು
ಕೇವಲ
ಕೇವಲಿಗಳ
ಕೇಶವ
ಕೇಶವದೇವರ
ಕೇಶವದೇವರಿಗೆ
ಕೇಶವದೇವರು
ಕೇಶವದೇವಾಲಯದ
ಕೇಶವನಾಥ
ಕೇಶವಾಪುರ
ಕೇಶಾಲಂಕಾರಗಳನ್ನೂ
ಕೇಶಿಯಣ್ಣ
ಕೇಶಿಯಣ್ಣನು
ಕೇಸವಯ್ಯನಿಗೆ
ಕೇಸಿಗ
ಕೇಸಿಮಯ್ಯ
ಕೇಸಿಯಣ್ಣನು
ಕೈಂಕರ್ಯಕ್ಕೆ
ಕೈಂಕರ್ಯಗಳನ್ನು
ಕೈಂಕರ್ಯಗಳಿಗೆ
ಕೈಕಾಲು
ಕೈಕೆಳಗಿನ
ಕೈಕೆಳಗೆ
ಕೈಕೊಂಡರಾರ್
ಕೈಕೊಂಡರಾರ್ಚ್ಚೋಳನಂ
ಕೈಕೊಂಡು
ಕೈಕೊಳೆ
ಕೈಗಿತ್ತ
ಕೈಗೆ
ಕೈಗೊಂಡನಪಲ್ಲಿಯನ್ನು
ಕೈಗೊಂಡನಪಲ್ಲಿಯು
ಕೈಗೊಂಡು
ಕೈಗೊಳ್ಳುವ
ಕೈಗೋನಹಳ್ಳಿ
ಕೈತಪ್ಪಿಹೋಗಿದ್ದವು
ಕೈದಾಳದ
ಕೈದೀವಿಗೆಗೆ
ಕೈಫಿಯತ್ತು
ಕೈಫಿಯತ್ತುಗಳನ್ನು
ಕೈಫಿಯತ್ತುಗಳಲ್ಲಿ
ಕೈಫಿಯತ್ತುಗಳು
ಕೈಬಿಟ್ಟರು
ಕೈಬಿಟ್ಟುಹೋಗಿದ್ದ
ಕೈಬಿಟ್ಟುಹೋದವು
ಕೈಯ
ಕೈಯಲಿ
ಕೈಯಲು
ಕೈಯಲ್ಲಿ
ಕೈಯಲ್ಲಿತ್ತು
ಕೈಯಲ್ಲಿಯೇ
ಕೈಯಲ್ಲೇ
ಕೈಯಿಂದ
ಕೈಯ್ಯಲ್ಲಿ
ಕೈಲಾಸನಾಥ
ಕೈಲಾಸಪ್ರಾಪ್ತರಾಗುತ್ತಾರೆ
ಕೈಲಾಸಸ್ಥಾನದಲ್ಲಿ
ಕೈಲಿ
ಕೈವಲ್ಯೇಶ್ವರ
ಕೈವಶವಾಗದಾಯಿತು
ಕೈವಾರಕರನಿರೋಧಕ
ಕೈವಾರನಿಸಂಕಮಲ್ಲ
ಕೈಸಾರ್ವ್ವಿನಂ
ಕೈಸೆರೆಯಾಗಿದ್ದ
ಕೈಸೇರಿ
ಕೈಹಾಕಿ
ಕೊಂಕಣ
ಕೊಂಗಣಿ
ಕೊಂಗಣಿವರ್ಮ
ಕೊಂಗನಾಡನ್ನಾಳುತ್ತಿದ್ದಾಗ
ಕೊಂಗಮಾರಿ
ಕೊಂಗಯರ್
ಕೊಂಗರ
ಕೊಂಗರದಿಶಾಪಟ್ಟ
ಕೊಂಗರನಡಗಿಸಿ
ಕೊಂಗರನ್ನು
ಕೊಂಗರಿಳಂಚಿಂಗರು
ಕೊಂಗಲ್ನಾಡಿನ
ಕೊಂಗಲ್ನಾಡೊಳಗಣ
ಕೊಂಗಳ್ನಾಡಿಗೆ
ಕೊಂಗಳ್ನಾಡು
ಕೊಂಗಳ್ನಾಡೆಂದು
ಕೊಂಗಸೇನೆಯನ್ನು
ಕೊಂಗಾಳೇಶ್ವರ
ಕೊಂಗಾಳ್ನಾಡ
ಕೊಂಗಾಳ್ನಾಡಿನಲ್ಲಿ
ಕೊಂಗಾಳ್ನಾಡಿನಲ್ಲಿದ್ದ
ಕೊಂಗಾಳ್ನಾಡು
ಕೊಂಗಾಳ್ವದೇವನು
ಕೊಂಗಾಳ್ವರ
ಕೊಂಗಾಳ್ವರನ್ನು
ಕೊಂಗಾಳ್ವರಾಜಕುಮಾರಿ
ಕೊಂಗಾಳ್ವರಿಗಿದ್ದ
ಕೊಂಗಾಳ್ವರು
ಕೊಂಗಾಳ್ವರುಚೆಂಗಾಳ್ವರ
ಕೊಂಗು
ಕೊಂಗುಣಿ
ಕೊಂಗುಣಿಮುತ್ತರಸ
ಕೊಂಗುಣಿವರ್ಮ
ಕೊಂಗುದೇಶ
ಕೊಂಗುದೇಶೈಕ
ಕೊಂಗುನಾಡನ್ನು
ಕೊಂಗುನಾಡಿನ
ಕೊಂಗುನಾಡಿನಿಂದ
ಕೊಂಗುನಾಡು
ಕೊಂಗುರಿಳಅಂಚಿಂಗರುಮ್
ಕೊಂಚನಿರಾಶನಾದ
ಕೊಂಚಭಾಗ
ಕೊಂಡ
ಕೊಂಡನಸಮ
ಕೊಂಡನಿಂತು
ಕೊಂಡಯ್ಯದೇವ
ಕೊಂಡರಾಜಯ್ಯದೇವ
ಕೊಂಡಹಾಗೆ
ಕೊಂಡಾನ್
ಕೊಂಡು
ಕೊಂತದ
ಕೊಂತಿದೇವಿಗಧಿಕಂ
ಕೊಂತಿಯ
ಕೊಂದ
ಕೊಂದನಲ್ಲದೆ
ಕೊಂದನು
ಕೊಂದನೆಂದು
ಕೊಂದರೆ
ಕೊಂದಿಕ್ಕಿ
ಕೊಂದಿಕ್ಕಿದನೊಕ್ಕಿಲಿಕ್ಕಿ
ಕೊಂದಿರುವ
ಕೊಂದು
ಕೊಂದುದಕ್ಕಾಗಿ
ಕೊಂದುದಕ್ಕಾಗಿಯೇ
ಕೊಂದುದಕ್ಕೆ
ಕೊಂದುಹಾಕಿದ
ಕೊಂದುಹಾಕಿದಂತೆ
ಕೊಂಬಾಳೆಯಲ್ಲಿ
ಕೊಂಬುದುಮಾತನ
ಕೊಂಮೆಯರ
ಕೊಟ
ಕೊಟಟ್ಟ
ಕೊಟ್ಟ
ಕೊಟ್ಟಂತಹ
ಕೊಟ್ಟಂತೆ
ಕೊಟ್ಟನು
ಕೊಟ್ಟನೆಂದು
ಕೊಟ್ಟರ
ಕೊಟ್ಟರದ
ಕೊಟ್ಟರವೆಗ್ಗಡೆ
ಕೊಟ್ಟರೆಂದು
ಕೊಟ್ಟಲಿಗೆಎಂದು
ಕೊಟ್ಟಾಗ
ಕೊಟ್ಟಿದೆ
ಕೊಟ್ಟಿದ್ದ
ಕೊಟ್ಟಿದ್ದನೆಂದು
ಕೊಟ್ಟಿದ್ದರಿಂದ
ಕೊಟ್ಟಿದ್ದಾನೆ
ಕೊಟ್ಟಿದ್ದಾರೆ
ಕೊಟ್ಟಿದ್ದೇ
ಕೊಟ್ಟಿರಬಹುದೆಂದು
ಕೊಟ್ಟಿರುವ
ಕೊಟ್ಟು
ಕೊಠಾರ
ಕೊಡಗನ್ನು
ಕೊಡಗಹಳ್ಳಿ
ಕೊಡಗಹಳ್ಳಿಯ
ಕೊಡಗಿಯ
ಕೊಡಗು
ಕೊಡಗೆಹಳ್ಳಿ
ಕೊಡದೇ
ಕೊಡಬಹುದು
ಕೊಡಬೇಕು
ಕೊಡಲಾಗದ
ಕೊಡಲು
ಕೊಡಿಸಿದನೆಂದು
ಕೊಡಿಸಿದರೆಂದು
ಕೊಡುಗೆ
ಕೊಡುಗೆಗಳು
ಕೊಡುಗೆಗೆ
ಕೊಡುಗೆಯನ್ನು
ಕೊಡುಗೆಯಲ್ಲಿ
ಕೊಡುಗೆಯಾಗಿ
ಕೊಡುಗೆಹಳ್ಳಿ
ಕೊಡುತ್ತದೆಂದು
ಕೊಡುತ್ತಾನೆ
ಕೊಡುತ್ತಾರೆ
ಕೊಡುತ್ತಿದ್ದರು
ಕೊಡುವ
ಕೊಡುವುದರಲ್ಲಿ
ಕೊಡೆ
ಕೊಡೆಹಾಳ
ಕೊಣನೂರು
ಕೊಣೆಹಳ್ಳಿ
ಕೊಣ್ಡನಾ
ಕೊತ್ತತ್ತಿ
ಕೊತ್ತತ್ತಿಯ
ಕೊತ್ತತ್ತಿಯನ್ನು
ಕೊತ್ತಲವಾಡಿ
ಕೊತ್ತಲುಗಳಿವೆ
ಕೊತ್ತಾಗಾಲ
ಕೊತ್ತಿವರದಹಳ್ಳಿಯನ್ನು
ಕೊನಯ
ಕೊನೆಗಾಲದಲ್ಲಿ
ಕೊನೆಗೆ
ಕೊನೆಗೊಳಿಸಿದನು
ಕೊನೆಯ
ಕೊನೆಯಬಾರಿಗೆ
ಕೊನೆಯಲ್ಲಿ
ಕೊನೆಯವರ್ಷದ
ಕೊನೆಯಾದ
ಕೊನೆಯುಸಿರೆಳೆಯುತ್ತಾನೆ
ಕೊನೆರಿ
ಕೊನ್ತದ
ಕೊನ್ದಡಿಯು
ಕೊನ್ದು
ಕೊಪಣಾದಿತೀರ್ಥ
ಕೊಪ್ಪ
ಕೊಪ್ಪದ
ಕೊಪ್ಪಲಿನ
ಕೊಪ್ಪಲಿನಲ್ಲಿಯೂ
ಕೊಪ್ಪಲಿನವರು
ಕೊಪ್ಪಲು
ಕೊಪ್ಪಳ
ಕೊಪ್ಪಳದಲ್ಲಿ
ಕೊಮಾರ
ಕೊಮಾರತಿಯನ್ನು
ಕೊಮಾರರು
ಕೊಮಾರಸೇನ
ಕೊಮ್ಮಣ್ಣ
ಕೊಮ್ಮಣ್ಣನು
ಕೊಮ್ಮರಾಜ
ಕೊಮ್ಮರಾಜಂ
ಕೊಮ್ಮರಾಜನಿರಬಹುದು
ಕೊಮ್ಮೇಶ್ವರ
ಕೊಯಳರಸನು
ಕೊಯಿಲೋ
ಕೊರತೆ
ಕೊರಳಹಾರದ
ಕೊಲುವಲ್ಲಿ
ಕೊಲೆ
ಕೊಲ್ಲಲು
ಕೊಲ್ಲಿಪಲ್ಲವ
ಕೊಲ್ಲಿಪಲ್ಲವನೊಳಂಬನೆಂಬ
ಕೊಲ್ಲಿಪೊಲ್ಲವ
ಕೊಲ್ಲಿಯರಸ
ಕೊಲ್ಲಿಯರಸನು
ಕೊಲ್ಲಿಸಿದನೆಂದು
ಕೊಲ್ಲುತ್ತಿರುವ
ಕೊಲ್ಲುವ
ಕೊಳಗ
ಕೊಳಗಳು
ಕೊಳತೂರುಇಂದಿನ
ಕೊಳವನ್ನು
ಕೊಳಾಲದ
ಕೊಳುಗುಂದದ
ಕೊಳುವಲ್ಲಿ
ಕೊಳೆಗೊಳು
ಕೊಳೆತು
ಕೊಳ್ಳಿ
ಕೊಳ್ಳಿಅಯ್ಯ
ಕೊಳ್ಳಿಅಯ್ಯನ
ಕೊಳ್ಳಿಪಾಕೆ
ಕೊಳ್ಳಿಯಮ್ಮನ
ಕೊಳ್ಳಿಯಮ್ಮೆಯಂಗಳ
ಕೊಳ್ಳಿಯಮ್ಮೆಯ್ಯ
ಕೊಳ್ಳಿಯಮ್ಮೆಯ್ಯನ
ಕೊಳ್ಳೆಗಾಲ
ಕೊಳ್ಳೇಗಾಲ
ಕೊಳ್ಳೇಗಾಲದ
ಕೊವಳೆವೆಟ್ಟು
ಕೋಗಳಿ
ಕೋಗಳಿನಾಡು
ಕೋಗಿಲಲಿ
ಕೋಟಗಾರರ
ಕೋಟೆ
ಕೋಟೆಕುರ
ಕೋಟೆಕೊತ್ತಲುಗಳ
ಕೋಟೆಕೊತ್ತಲುಗಳನ್ನು
ಕೋಟೆಗಳನ್ನು
ಕೋಟೆಗೆ
ಕೋಟೆಬೆಟ್ಟ
ಕೋಟೆಯ
ಕೋಟೆಯನ್ನು
ಕೋಟೆಯಬಯಲ
ಕೋಟೆಯಲ್ಲಿ
ಕೋಟೆಯೆಂಬ
ಕೋಟೆಯೊಳಗಿರುವ
ಕೋಟ್ಟ
ಕೋಡಾಲ
ಕೋಡಾಲದ
ಕೋಡಾಲವನ್ನು
ಕೋಡಾಲವು
ಕೋಡಿ
ಕೋಡಿನಕೊಪ್ಪ
ಕೋಡಿನಕೊಪ್ಪಕೋಡಿಹಳ್ಳಿ
ಕೋಡಿಪುರ
ಕೋಣನಕಲ್ಲು
ಕೋಣೆಯ
ಕೋತನಪುರ
ಕೋದಂಡರಾಮ
ಕೋದಂಡರಾಮಸ್ವಾಮಿಯ
ಕೋನಾಪುರ
ಕೋನೇಟಿ
ಕೋನೇರಿನ್ಮೈ
ಕೋನೇರಿಮ್ಮೈ
ಕೋಮಟಿಗಳ
ಕೋಮನಹಳ್ಳಿ
ಕೋರವಂಗಲ
ಕೋರವಂಗಲದ
ಕೋರಿಕೆಯ
ಕೋರಿದನು
ಕೋರೆಗಾಲ
ಕೋರೆಗಾಲದ
ಕೋಲಾರ
ಕೋಳಾಲ
ಕೋಳಾಲಪುರದ
ಕೋಳಾಲಪುರಪರಮೇಶ್ವರ
ಕೋಳಾಹಳ
ಕೋಳಿಗಾಲ
ಕೋಳೋಗಾಲ
ಕೋಳೋಗಾಲವನ್ನು
ಕೋಶವು
ಕೋಶಸ್ಯ
ಕೋಶಾಧಿಕಾರಿ
ಕೌಂಡಿಣ್ಯ
ಕೌಂಡಿನ್ಯ
ಕೌಂಡಿಲ್ಯ
ಕೌಡಲಿ
ಕೌಡಲಿಇಂದಿನ
ಕೌಡ್ಲೆ
ಕೌಡ್ಲೆಯು
ಕೌಣ್ಡಿಲ್ಯ
ಕೌಲನ್ನು
ಕೌಶಿಕ
ಕೌಶಿಕಕುಳಾಂಬರ
ಕೌಶಿಕಗೋತ್ರಅಪವಿತ್ರನೂ
ಕೌಶಿಕಗೋತ್ರದ
ಕೌಸಲ್ಯ
ಕೌಸಲ್ಯಾ
ಕ್ಕೂ
ಕ್ಕೆ
ಕ್ಕೋವಿದಂ
ಕ್ಯಾತನಹಳ್ಳಿ
ಕ್ಯಾತುಂಗೆರೆ
ಕ್ರಮಬದ್ಧವಾಗಿ
ಕ್ರಮವಾಗಿ
ಕ್ರಮೇಣ
ಕ್ರಯದ
ಕ್ರಯದಾನವಾಗಿ
ಕ್ರಯಪತ್ರವನ್ನು
ಕ್ರಯವಾಗಿ
ಕ್ರಯವಾಗಿಕೊಂಡುಅದನ್ನು
ಕ್ರಯಶಾಸನದ
ಕ್ರಯಶಾಸನವನ್ನು
ಕ್ರಿ
ಕ್ರಿಪೂದಲ್ಲಿಯೇ
ಕ್ರಿರ
ಕ್ರಿಶ
ಕ್ರಿಶಕ್ಕೆ
ಕ್ರಿಶಗಂಗರಾಷ್ಟ್ರಕೂಟರ
ಕ್ರಿಶನೆಯ
ಕ್ರಿಶನೇ
ಕ್ರಿಶರ
ಕ್ರಿಶರದ್ದೇ
ಕ್ರಿಶರನಂತರ
ಕ್ರಿಶರರಲ್ಲಿ
ಕ್ರಿಶರಲ್ಲಿ
ಕ್ರಿಶರಲ್ಲಿಯೂ
ಕ್ರಿಶರಲ್ಲೇ
ಕ್ರಿಶರವರೆಗಿನ
ಕ್ರಿಶರವರೆಗೆ
ಕ್ರಿಶರಿಂದ
ಕ್ರಿಶರಿಂದಲೇ
ಕ್ರಿಶಸುಮಾರು
ಕ್ರಿಶಹಿಜರಿ
ಕ್ಲಪ್ತವಿಷ್ಣ್ವೀಶಪೂಜಃ
ಕ್ವ್ಮಾಪತೇಃ
ಕ್ಷತಮಱೆವಾತಂ
ಕ್ಷತ್ರಿಯರಾದ
ಕ್ಷತ್ರಿಯರು
ಕ್ಷತ್ರಿಲಾಡರಿ
ಕ್ಷಮಾಧೀಶ
ಕ್ಷಮಾಧೀಶತಾ
ಕ್ಷಮಿಸಿ
ಕ್ಷಾಮಢಾಮರಗಳು
ಕ್ಷಾಮಿಸ್ಫುರನ್
ಕ್ಷಿತಿನಾಥ
ಕ್ಷಿತಿಪಾಲಕನು
ಕ್ಷಿತಿಪಾಲಮೌಳಿರ್ವದಾನ್ಯಮೂರ್ತಿಃ
ಕ್ಷಿತೀಂದ್ರ
ಕ್ಷಿತೀಂದ್ರನ
ಕ್ಷಿತೀಂದ್ರನು
ಕ್ಷಿತೀಂದ್ರನೆಂಬ
ಕ್ಷಿತೀಶ್ವರನು
ಕ್ಷೇತ್ರಕಾರ್ಯದ
ಕ್ಷೇತ್ರಗಳಾದ
ಕ್ಷೇತ್ರದಲ್ಲಿ
ಕ್ಷೇತ್ರವಾದ
ಕ್ಷೇತ್ರಾಯ
ಕ್ಷೋಣೀವಧೂಭೂಷಣೇ
ಕ್ಷೋಭೆಗಳನ್ನು
ಕ್ಷೋಭೆಯನ್ನು
ಕ್ಷ್ಮಾಯಾಂರಾಜ್ಯ
ಕೞಅ್ಬಹುಕೞ್ಬಾಹುಕಬ್ಬಾಹು
ಕೞ್ಬಪ್ಪು
ಖಂಡಸ್ಫುಟಿತ
ಖಂಡಿಸಿ
ಖಂಡುಗ
ಖಂಡೆಯರಾಯ
ಖಂತಿಕಾರ
ಖಗರಾಜನಿನೇಕದೊಡನಿಂ
ಖಚಿತ
ಖಚಿತಗುಂಪೇ
ಖಚಿತಪಡಿಸುತ್ತದೆ
ಖಚಿತಪಡಿಸುತ್ತವೆ
ಖಚಿತಪಡಿಸುವಲ್ಲಿ
ಖಚಿತಪಡುತ್ತದೆ
ಖಚಿತವಾಗಿ
ಖಚಿತವಾಗುತ್ತದೆ
ಖಚಿತವಿಲ್ಲ
ಖಜಾನೆಗೆ
ಖಜಾನೆಯ
ಖಡಿಲೆಗೊಂಡು
ಖದೀರ್
ಖರ
ಖರದೂಷಣರನ್ನು
ಖರೀದಿ
ಖರೀದಿಸಿ
ಖರೀದಿಸುತ್ತಾರೆ
ಖರ್ಚುವೆಚ್ಚಗಳ
ಖಲೀಫನಾದ
ಖಳತ್ರಿಣೇತ್ರ
ಖಳ್ಗದಿಂ
ಖಾತಿಧರಿತಿತೆ
ಖಾದ್ರಿ
ಖಾಯಿಲೆ
ಖಾಲಿ
ಖಾಸ
ಖಾಸಗಿ
ಖಾಸಾ
ಖಾಸಾಬೊಕ್ಕಸದ
ಖಿಡಿಜ್ಚಿಣ್ಡಿಜ್ಡಿ
ಖಿಲವಾಗಿದ್ದ
ಖಿಲವಾಗಿರಲು
ಖುದಾದಾದ್
ಖೈಬರದ
ಖೊಟ್ಟಿಗನು
ಖ್ಯಾತ
ಖ್ಯಾತನಾಗಿದ್ದನು
ಖ್ಯಾತನಾಗಿದ್ದು
ಖ್ಯಾತನಾದನೆಂದು
ಖ್ಯಾತಸ್ಯಾನಸ್ಯ
ಖ್ಯಾತಿಯ
ಖ್ಯಾತೆಯಾಗ
ಖ್ಯಾತೆಯಾದ
ಖ್ವಾಜಾ
ಗಂಗ
ಗಂಗಕುಳಚಂದ್ರಂ
ಗಂಗಗಾಮುಂಡ
ಗಂಗಚಮೂಪಂ
ಗಂಗಡಿಕಾರ
ಗಂಗಣ್ಣ
ಗಂಗದಂಡಾಧೀಶ
ಗಂಗದಂಡಾಧೀಶನ
ಗಂಗದಂಡಾಧೀಶನನ್ನು
ಗಂಗದಂಡಾಧೀಶನು
ಗಂಗದಂಡೇಶ
ಗಂಗದೇಶಾಧಿಪ
ಗಂಗದೊರೆಯ
ಗಂಗನನ್ನು
ಗಂಗನಹಳ್ಳಿ
ಗಂಗನಾರಾಯಣ
ಗಂಗನೃಪನು
ಗಂಗನೊಳಂಬರ
ಗಂಗಪಯ್ಯನ
ಗಂಗಪಲ್ಲವರ
ಗಂಗಪೆರ್ಮಾನಡಿ
ಗಂಗಪೆರ್ಮಾನಡಿಯನ್ನು
ಗಂಗಪೆರ್ಮಾನಡಿಯು
ಗಂಗಪೆರ್ಮ್ಮಾನಡಿಯು
ಗಂಗಪ್ಪಯ್ಯ
ಗಂಗಪ್ರವಾಹೋದಾರ
ಗಂಗಮಂಡಲ
ಗಂಗಮಂಡಲಗಳಿಗೆ
ಗಂಗಮಂಡಲವೆಂದು
ಗಂಗಮಂಡಲಾಧಿಪತ್ಯವನ್ನು
ಗಂಗಮಂಡಳ
ಗಂಗಮಂಡಳವನ್ನು
ಗಂಗಮಂಡಳೇಶ್ವರ
ಗಂಗರ
ಗಂಗರಅ
ಗಂಗರಕಾಲ
ಗಂಗರಕಾಲದ
ಗಂಗರಕಾಲದಲ್ಲಿ
ಗಂಗರಮೇಲಿನ
ಗಂಗರಸರಾಗಿ
ಗಂಗರಸರು
ಗಂಗರಾಜ
ಗಂಗರಾಜಂ
ಗಂಗರಾಜಕುಮಾರ
ಗಂಗರಾಜನ
ಗಂಗರಾಜನಕಾಲದಲ್ಲಿಯೇ
ಗಂಗರಾಜನನ್ನು
ಗಂಗರಾಜನನ್ನೂ
ಗಂಗರಾಜನಾದ
ಗಂಗರಾಜನಿಗೆ
ಗಂಗರಾಜನಿರಬಹುದು
ಗಂಗರಾಜನಿರಬಹುದೆಂಬ
ಗಂಗರಾಜನು
ಗಂಗರಾಜನೆಂಬುದು
ಗಂಗರಾಜನೊಡನೆ
ಗಂಗರಾಜನ್ನು
ಗಂಗರಾಜ್ಯ
ಗಂಗರಾಜ್ಯಕ್ಕೆ
ಗಂಗರಾಜ್ಯವನ್ನು
ಗಂಗರಾಷ್ಟ್ರಕೂಟರ
ಗಂಗರಿಂದ
ಗಂಗರಿಗೂ
ಗಂಗರಿಗೆ
ಗಂಗರು
ಗಂಗರೊಂದಿಗೆ
ಗಂಗರೊಡನೆ
ಗಂಗವಂಶದ
ಗಂಗವಂಶದರೆೆಂದೂ
ಗಂಗವಂಶದವನೆಂದು
ಗಂಗವಾಡಿ
ಗಂಗವಾಡಿಗೆ
ಗಂಗವಾಡಿತೊಂಭತ್ತರುಸಾಸಿರದ
ಗಂಗವಾಡಿಯ
ಗಂಗವಾಡಿಯನ್ನು
ಗಂಗವಾಡಿಯಲ್ಲಿ
ಗಂಗವಾಡಿಯಲ್ಲೇ
ಗಂಗವಾಡಿಯಿಂದ
ಗಂಗವಾಡಿಯು
ಗಂಗವಾಡಿಯೊಳಕ್ಕೆ
ಗಂಗಸಂದ್ರ
ಗಂಗಸಮುದ್ರ
ಗಂಗಸೇನಾಪತಿಯ
ಗಂಗಾದೇವಿ
ಗಂಗಾಧರ
ಗಂಗಾಧರನೆಂಬ
ಗಂಗಾಧರಪುರವೆಂದು
ಗಂಗಾಧರಯ್ಯನು
ಗಂಗಾಧರೇಶ್ವರ
ಗಂಗಾಧರೇಶ್ವರನ
ಗಂಗಾಧರೇಶ್ವರಸ್ವಾಮಿಯ
ಗಂಗಾನದಿ
ಗಂಗಾನದಿಯ
ಗಂಗಾನ್ವಯ
ಗಂಗಾವನಿರಟ್ಟ
ಗಂಗಾವನಿರಟ್ಟವಾಡಿ
ಗಂಗಾಸಮುದ್ರ
ಗಂಗಿಗವುಂಡನ
ಗಂಗೇಗೌಡ
ಗಂಗೇಶ್ವರ
ಗಂಗೈಕೊಂಡ
ಗಂಗೈಕೊಂಡಚೋಳಪುರಕ್ಕೆ
ಗಂಜಾಂ
ಗಂಜಾಮ್
ಗಂಜಾಮ್ನಲ್ಲಿರುವ
ಗಂಟು
ಗಂಡ
ಗಂಡಃ
ಗಂಡಗೂಳಿ
ಗಂಡನಾದ
ಗಂಡನಾರಾಯಣ
ಗಂಡನಾರಾಯಣಸೆಟ್ಟಿ
ಗಂಡನಾರಾಯಣಸೆಟ್ಟಿಗೆ
ಗಂಡನಾರಾಯಣಸೆಟ್ಟಿಯ
ಗಂಡನಾರಾಯಣಸೆಟ್ಟಿಯರು
ಗಂಡನು
ಗಂಡನೆನಿಸಿದ
ಗಂಡಪೆಂಡಾರ
ಗಂಡಪೆಂಡಾರಗೊಂಡನೆಂದು
ಗಂಡಪೆಂಡಾರವನ್ನು
ಗಂಡಬೇರುಂಡ
ಗಂಡಭೇರುಂಡ
ಗಂಡರ
ಗಂಡರಗಂಡಮುಂಡ
ಗಂಡರನಾಂಪೆವೆಂದು
ಗಂಡರಾಗಿದ್ದರೆಂದು
ಗಂಡರಾಜ
ಗಂಡರುಂ
ಗಂಡವಿಮುಕ್ತ
ಗಂಡಾಂತರದಿಂದ
ಗಂಡಾನೆ
ಗಂಡುಗಲಿ
ಗಂಡುಮಕ್ಕಳೂ
ಗಂಧಗೋಡಿ
ಗಂಧನಹಳ್ಳಿ
ಗಂಧವಾರಣ
ಗಂಧವಾರಣನೆಂದು
ಗಂಭೀರಪ್ಪ
ಗಂಭೀರವೂ
ಗಉಡು
ಗಉಡುಕುಲತಿಲಕರುಂ
ಗಉಡುಗಳು
ಗಜಖೇಟಕ
ಗಜಗಾರ
ಗಜಗಾರಕುಪ್ಪೆ
ಗಜಗಾರಕುಪ್ಪೆಯೇ
ಗಜಗಾರಗುಪ್ಪೆ
ಗಜಗಾರರೆಂದು
ಗಜಪತಿ
ಗಜಪತಿಗಳು
ಗಜಪತಿಯ
ಗಜಪತಿಯು
ಗಜಬೇಂಟೆಕಾರ
ಗಜರಾಜಗಿರಿ
ಗಜವು
ಗಜಸಿಂಹ
ಗಜಸೇನೆಯೊಂದಿಗೆ
ಗಜಾಂಕುಶ
ಗಜಾಖೇಟಕಮತ್ಯುಗ್ರಂ
ಗಜಾರಣ್ಯ
ಗಜೇಂದ್ರಮಂಟಪವನ್ನು
ಗಜೌಘ
ಗಜೌಘಗಂಡಭೇರುಂಡೋ
ಗಟ್ಟಿಗೊಳ್ಳಬೇಕಾಗಿದೆ
ಗಟ್ಟಿಯಾದ
ಗಟ್ಟೇಶ್ವರ
ಗಡ
ಗಡದ
ಗಡದಗಡ್ಡದ
ಗಡಿ
ಗಡಿಗೆ
ಗಡಿದ
ಗಡಿದತಿರುಮಲಾಪುರ
ಗಡಿಮೂಡಲು
ಗಡಿಯ
ಗಡಿಯಲ್ಲಿ
ಗಡಿಯಲ್ಲಿರುವ
ಗಡಿಯಲ್ಲೇ
ಗಡಿಯಾಗಿದ್ದಿರಬಹುದು
ಗಡಿಯಿಂದ
ಗಡಿಯು
ಗಡ್ಡದ
ಗಡ್ಡದದಾಡಿಯ
ಗಡ್ಡದದಾಡಿಯಸೋಮೆಯದಂಡನಾಯಕ
ಗಣಕರು
ಗಣಪತಿ
ಗಣಹಳ್ಳಿ
ಗಣಹಳ್ಳಿಯನ್ನು
ಗಣ್ಡಪೆಣ್ಡಾರ
ಗಣ್ದಪೆಣ್ಡಾರ
ಗಣ್ಯದೊರೆ
ಗತಿಸಿದನೆಂದು
ಗತಿಸಿರಬಹುದು
ಗತಿಸಿರಬಹುದೆಂಬುದು
ಗದಗ
ಗದಿರದೆ
ಗದ್ದೆ
ಗದ್ದೆಗಳನ್ನು
ಗದ್ದೆಗಳಿಗೆ
ಗದ್ದೆಬೆದ್ದಲು
ಗದ್ದೆಬೆದ್ದಲುಗಳನ್ನು
ಗದ್ದೆಯನ್ನು
ಗದ್ದೆಯನ್ನೂ
ಗದ್ಯಾಣ
ಗದ್ಯಾಣಗಳನ್ನು
ಗನ್ನು
ಗಮನವನ್ನು
ಗಮನಾರ್ಹ
ಗಮನಿಸತಕ್ಕದ್ದಾಗಿದೆ
ಗಮನಿಸತಕ್ಕದ್ದಾಗಿವೆ
ಗಮನಿಸಬಹುದು
ಗಮನಿಸಬೇಕಾಗುತ್ತದೆ
ಗಮನಿಸಬೇಕಾದುದು
ಗಮನಿಸಲಾಗಿದೆ
ಗಮನಿಸಿ
ಗಮನಿಸಿಬಹುದು
ಗರುಜೆ
ಗರುಡ
ಗರುಡಗಂಬದ
ಗರುಡನಂತೆ
ಗರುಡನನಪ್ಪಿ
ಗರುಡನನ್ನು
ಗರುಡನಹಳ್ಳಿಯನ್ನು
ಗರುಡನಾದ
ಗರುಡರಾಗಿ
ಗರುಡರಾಗಿದ್ದವರು
ಗರುಡರು
ಗರುಡಲೆಂಕರಾಗಿ
ಗರುಡಶಾಸನದ
ಗರ್ಬ್ಭಸರ್ಬ್ಬಸ್ವಾಪಹಾರ
ಗರ್ವೋದ್ಧತನಾದ
ಗಳಹನ
ಗಳಿಗೆ
ಗಳಿಸಿಕೊಟ್ಟಿದ್ದಕ್ಕಾಗಿಯೇ
ಗಳಿಸಿದ
ಗಳಿಸಿದಂತೆ
ಗಳಿಸಿದವನಾಗಿರಬೇಕು
ಗಳಿಸಿದ್ದನು
ಗಳು
ಗವರೆ
ಗವರೇಶ್ವರ
ಗವುಡ
ಗವುಡನಾಗಿದ್ದ
ಗವುಡರನ್ನು
ಗವುಡರಿಗೆ
ಗವುಡರು
ಗವುಡರುಗಳು
ಗವುಡಿಕೆಗೆ
ಗವುಡಿಕೆಯನ್ನು
ಗವುಡಿಗೆರೆ
ಗವುಡು
ಗವುಡುಗಳ
ಗವುಡುಗಳಾದ
ಗವುಡುಗಳಾದಗಾವುಂಡರು
ಗವುಡುಗಳಿಗೆ
ಗವುಡುಗಳು
ಗವುಡುಗಳೆಂದರೆ
ಗವುಡುಪ್ರಜೆಗಳ
ಗವುಡುಪ್ರಜೆಗಳನ್ನು
ಗವುಡುಪ್ರಜೆಗಳು
ಗಾಂಗೇಯ
ಗಾಂಚನೂರನ್ನು
ಗಾಂನ್ಧವರಾನೆ
ಗಾಣದ
ಗಾಣದೆರೆಗಳನ್ನು
ಗಾಣವನ್ನು
ಗಾಣಿಗನಪುರ
ಗಾತ್ರ
ಗಾತ್ರಂ
ಗಾತ್ರಃ
ಗಾಮಬ್ಬೆಯನ್ನು
ಗಾಮವನ್ನು
ಗಾಮುಂಡ
ಗಾಮುಂಡನ
ಗಾಮುಂಡರ
ಗಾಮುಂಡರನ್ನು
ಗಾಮುಂಡರಾಗಿರುತ್ತಿದ್ದರೆಂದು
ಗಾಮುಂಡರಿಗೆಲ್ಲ
ಗಾಮುಂಡರು
ಗಾಮುಂಡಸ್ವಾಮಿ
ಗಾಮುಂಡಸ್ವಾಮಿಗಳ
ಗಾಮುಂಡಿಯ
ಗಾಮುಂಡಿಯರೆಂದು
ಗಾಮುಣ್ಡ
ಗಾಮುಣ್ಡರು
ಗಾಮುಣ್ಡಸ್ವಾಮಿಗಳ
ಗಾಮುಣ್ಡಸ್ವಾಮಿಯು
ಗಾಯಿಗೋಪಾಳ
ಗಾಯಿಗೋವಳ
ಗಾಳಿಯಂದೊಡಿಂ
ಗಾವುಂಡ
ಗಾವುಂಡತನಕ್ಕೆ
ಗಾವುಂಡನ
ಗಾವುಂಡನದು
ಗಾವುಂಡನನ್ನೇ
ಗಾವುಂಡನಿರುತ್ತಿದ್ದನು
ಗಾವುಂಡನು
ಗಾವುಂಡರ
ಗಾವುಂಡರನ್ನು
ಗಾವುಂಡರನ್ನೇ
ಗಾವುಂಡರಲ್ಲಿ
ಗಾವುಂಡರಾದ
ಗಾವುಂಡರಿಂದ
ಗಾವುಂಡರಿಗಿಂತ
ಗಾವುಂಡರಿಗೆ
ಗಾವುಂಡರಿರುತ್ತಿದ್ದರೂ
ಗಾವುಂಡರು
ಗಾವುಂಡರುಗಳ
ಗಾವುಂಡರುಗಳು
ಗಾವುಂಡರೇ
ಗಾವುಂಡುಗಳ
ಗಾವುಂಡುಗಳು
ಗಾವುಡ
ಗಾವುಡಂರು
ಗಾವುಡನಿಗೆ
ಗಾವುಡನು
ಗಿಜಿಹಳ್ಳಿ
ಗಿಡದ
ಗಿರಿದುರ್ಗಮಲ್ಲ
ಗಿರಿಯಣ್ಣನಾಯಕ
ಗಿರಿಯಣ್ಣನಾಯಕರ
ಗಿರಿಯಲಲ್ಲದೆ
ಗಿರಿಶ್ರೇಣಿಗಳ
ಗಿರಿಶ್ರೇಣಿಯಿದೆ
ಗೀತೆಗಳಲ್ಲಿ
ಗು
ಗುಂಡಲುಪೇಟೆ
ಗುಂಡೇನಹಳ್ಳಿ
ಗುಂಡ್ಲುಪೇಟೆ
ಗುಂಡ್ಲುಪೇಟೆಯಲ್ಲಿ
ಗುಂಪನ್ನು
ಗುಂಪಿಗೆ
ಗುಂಪಿತ್ತು
ಗುಂಪಿನ
ಗುಂಪು
ಗುಂಬಜ್
ಗುಂಬಜ್ನಲ್ಲಿ
ಗುಂಬಜ್ನಲ್ಲಿರುವ
ಗುಂಬದ್ಇಅಲಾ
ಗುಂಮಂಣನು
ಗುಜರಾಥಿ
ಗುಜ್ಜಯನಾಯ್ಕನ
ಗುಜ್ಜರರೊಡನೆ
ಗುಜ್ಜಲೆ
ಗುಜ್ಜೆಯ
ಗುಡಿಗಳು
ಗುಡಿಯ
ಗುಡಿಯನ್ನು
ಗುಡಿಯಲ್ಲಿ
ಗುಡಿಯಲ್ಲಿರುವ
ಗುಡಿಯಾಗಿ
ಗುಡಿಯಾಗಿದೆ
ಗುಡೇನಹಳ್ಳಿ
ಗುಡ್ಡ
ಗುಡ್ಡಂ
ಗುಡ್ಡಗಳಿರುವ
ಗುಡ್ಡಗಳು
ಗುಡ್ಡದ
ಗುಡ್ಡಿಯಾಗಿದ್ದಳು
ಗುಡ್ಡುಗಳಾದ
ಗುಡ್ಡೆಹಳ್ಳಿ
ಗುಣ
ಗುಣಗಣದಿನಾತನೆಣೆಯಪ್ಪಂನಂ
ಗುಣಗನೆಯಿಂದ
ಗುಣಗಾನ
ಗುಣಗಾನವನ್ನು
ಗುಣದಭಿ
ಗುಣದಿನಾದನದಾವಂ
ಗುಣಸಂಪಂನ್ನ
ಗುಣಸಂಪನ್ನ
ಗುಣಸಂಪನ್ನನುಂ
ಗುಣಸಂಪನ್ನರಪ್ಪ
ಗುಣೋದಗ್ರ
ಗುತ್ತಲ
ಗುತ್ತಲನ್ನು
ಗುತ್ತಲನ್ನೂ
ಗುತ್ತಲಲ್ಲಿ
ಗುತ್ತಲಿನ
ಗುತ್ತಲು
ಗುತ್ತಿಗೆಗೆ
ಗುತ್ತಿಯ
ಗುತ್ತಿಯಗಂಗ
ಗುತ್ತಿಯಗಂಗನೆಂದು
ಗುದ್ಲಿಕಲ್ಲುಮಂಠಿ
ಗುಬ್ಬಿಯ
ಗುಬ್ಬಿಹಳ್ಳಿ
ಗುಮ್ಮಟದೇವನು
ಗುಮ್ಮಣ್ಣ
ಗುಮ್ಮನವೃತ್ತಿ
ಗುಮ್ಮನವೃತ್ತಿಯ
ಗುಮ್ಮನಹಳ್ಳಿ
ಗುಮ್ಮನಹಳ್ಳಿಯನ್ನು
ಗುಮ್ಮಳಾಪುರದ
ಗುರಿಕಾರ
ಗುರಿಯನ್ನಾಗಿಸಿಕೊಂಡು
ಗುರು
ಗುರುಕವಿಪ್ರಾಜ್ಞೈಃವೃತೇ
ಗುರುಗಳ
ಗುರುಗಳಾಗಿದ್ದು
ಗುರುಗಳಾದ
ಗುರುಗಳಾದರು
ಗುರುಗಳಿಗೆ
ಗುರುಗಳಿರಬಹುದು
ಗುರುಗಳು
ಗುರುಗಳೊಡನೆ
ಗುರುತಿದೆ
ಗುರುತಿಸಬಹುದಾಗಿದೆ
ಗುರುತಿಸಬಹುದು
ಗುರುತಿಸಬಹುದೆಂದು
ಗುರುತಿಸಲಾಗದೆಂದು
ಗುರುತಿಸಲಾಗಿದೆ
ಗುರುತಿಸಲು
ಗುರುತಿಸಲ್ಲ
ಗುರುತಿಸಿ
ಗುರುತಿಸಿದೆ
ಗುರುತಿಸಿದ್ದಾರೆ
ಗುರುತಿಸಿರುವ
ಗುರುತಿಸುವುದು
ಗುರುಪರಂಪರೆಯನ್ನು
ಗುರುಪೀಠದ
ಗುರುಲಿಂಗಜಂಗಮ
ಗುರುವಿಗೆ
ಗುರ್ಜ್ಜರ
ಗುಲಬರ್ಗಾ
ಗುಲಾಮ್
ಗುಲ್ಲಯ್ಯನು
ಗುಳಿಯ
ಗುಹೆ
ಗೂಬೆಕಲ್ಲುಮಂಠಿ
ಗೂರ್ಜರ
ಗೂರ್ಜರರು
ಗೂಳಿಗೌಡ
ಗೂಳೂರನ್ನು
ಗೂಳೂರು
ಗೂಳೂರೇ
ಗೃಹೋಪಕರಣಗಳು
ಗೆ
ಗೆಜಗಾರಕುಪ್ಪೆಯಾಗಿರ
ಗೆಜ್ಜಗಾರಗುಪ್ಪೆಯ
ಗೆಜ್ಜಗಾರಗುಪ್ಪೆಯಲ್ಲಿಯೂ
ಗೆದೆಗಾಂತುಕಂಬಳ
ಗೆದ್ದ
ಗೆದ್ದನು
ಗೆದ್ದನೆಂದು
ಗೆದ್ದಾಗ
ಗೆದ್ದಿರಬಹುದೆಂದು
ಗೆದ್ದು
ಗೆದ್ದುಕೊಂಡನೆಂದು
ಗೆದ್ದುಕೊಂಡರು
ಗೆದ್ದುಕೊಟ್ಟ
ಗೆದ್ದುಕೊಟ್ಟದ್ದಕ್ಕಾಗಿ
ಗೆದ್ದುಕೊಟ್ಟನು
ಗೆದ್ದುಕೊಟ್ಟನೆಂದು
ಗೆದ್ದುದು
ಗೆಯಾದವಗಂ
ಗೆಯ್ಸಿದಂ
ಗೆಲವು
ಗೆಲಿದು
ಗೆಲುವು
ಗೆಲ್ದಡೆ
ಗೆಲ್ಲಲು
ಗೆಲ್ಲುತ್ತಿದ್ದನಂತೆ
ಗೇಟ್ನ
ಗೇಣಾಂಕಚಕ್ರೇಶ್ವರ
ಗೇರಹಳ್ಳಿ
ಗೇಹದ
ಗೇಹವನ್ನು
ಗೈಯಲು
ಗೊಂದಲವನ್ನು
ಗೊಂದಲವಾಗುವುದು
ಗೊಂದಲವಿದೆ
ಗೊಂಮಟೇಶ್ವರ
ಗೊಡಗೆ
ಗೊತ್ತಾಗುತ್ತದೆ
ಗೊತ್ತಾಗುವುದಿಲ್ಲ
ಗೊತ್ತಿರುವ
ಗೊತ್ತಿಲ್ಲ
ಗೊಮ್ಮಟಜಿನಸ್ತುತಿ
ಗೊಮ್ಮಟದೆವನಿಗೆ
ಗೊಮ್ಮಟದೇವರ
ಗೊಮ್ಮಟದೇವರು
ಗೊಮ್ಮಟನನನ್ನು
ಗೊಮ್ಮಟನಲ್ತೆ
ಗೊಮ್ಮಟನಿಗೆ
ಗೊಮ್ಮಟೇಶ್ವರ
ಗೊಮ್ಮಟೇಶ್ವರನ
ಗೊಮ್ಮಟೇಶ್ವರನಿಗೆ
ಗೊರಊರು
ಗೊರವಂಕ
ಗೊರೂರು
ಗೊಲ್ಲರಚೆಟ್ಟನಹಳ್ಳಿ
ಗೊಲ್ಲರಚೆಟ್ಟನಹಳ್ಳಿಗಳ
ಗೊಲ್ಲರಚೆಟ್ಟನಹಳ್ಳಿಯ
ಗೊಲ್ಲರಹೊಸಹಳ್ಳಿ
ಗೋಂವಿದರನೆಂದೂ
ಗೋಗ್ರಹಣವಾಗಿರಬಹುದು
ಗೋಜಲುಗಳಿವೆ
ಗೋಡೆಯ
ಗೋಣಿಸೋಮನಹಳ್ಳಿ
ಗೋತ್ರಕ್ಕೆ
ಗೋತ್ರಚಿಂತಾಮಣಿ
ಗೋತ್ರದ
ಗೋತ್ರದವನು
ಗೋತ್ರದವರು
ಗೋತ್ರಪವಿತ್ರಂ
ಗೋತ್ರಪವಿತ್ರನುಂ
ಗೋತ್ರವನ್ನು
ಗೋತ್ರಸೂತ್ರಗಳನ್ನು
ಗೋತ್ರೋದಯಃ
ಗೋದಾನ
ಗೋಪಣ್ಣ
ಗೋಪಯ್ಯನನ್ನು
ಗೋಪಾಲ
ಗೋಪಾಲಕೃಷ್ಣ
ಗೋಪಾಲಕೃಷ್ಣದೇವಾಲಯವನ್ನು
ಗೋಪಾಲಕೃಷ್ಣದೇವಾಲಯವಾಗಿದೆ
ಗೋಪಾಲರಾಜ
ಗೋಪಾಲರಾವ್
ಗೋಪಾಲಸ್ವಾಮಿ
ಗೋಪಾಲ್
ಗೋಪಾಳ
ಗೋಪಾಳದೇವ
ಗೋಪಾಳದೇವನು
ಗೋಪಾಳದೇವರ
ಗೋಪಾಳದೇವರು
ಗೋಪಾಳರಾಜನ
ಗೋಪಿನಾಥದೇವರ
ಗೋಪಿನಾಥದೇವರನ್ನು
ಗೋಪಿನಾಥದೇವರಿಗೆ
ಗೋಪಿಯ
ಗೋಪಿಯನಾಯಕ
ಗೋಪಿಯನಾಯಕನ
ಗೋಪೀನಾಥದೇವರಿಗೆ
ಗೋಪುರಗಳಿಗೆ
ಗೋಪುರವತಿ
ಗೋಬ್ರಾಹ್ಮಣಪ್ರಿಯ
ಗೋಬ್ರಾಹ್ಮಣಹಯಧೂಳಿಧೂಸರ
ಗೋಮಹಿಷಿಗಳ
ಗೋಮಹಿಷಿಗಳನ್ನು
ಗೋಯರ
ಗೋಯರನು
ಗೋಯಿಂದರ
ಗೋಯಿಂದರನ
ಗೋಯಿಗ
ಗೋಲೂರು
ಗೋಲ್ಕೊಂಡದವರ
ಗೋಳಗವುಡನು
ಗೋವರ್ಧನಗಿರಿಯಲ್ಲಿ
ಗೋವಿಂದ
ಗೋವಿಂದನ
ಗೋವಿಂದನನ್ನು
ಗೋವಿಂದನಹಳ್ಳಿ
ಗೋವಿಂದನಹಳ್ಳಿಯ
ಗೋವಿಂದನು
ಗೋವಿಂದಮಯ್ಯ
ಗೋವಿಂದಮಯ್ಯನ
ಗೋವಿಂದಮಯ್ಯನೆಂಬ
ಗೋವಿಂದಯ್ಯ
ಗೋವಿಂದಯ್ಯನವರಿಗೆ
ಗೋವಿಂದಯ್ಯನು
ಗೋವಿಂದಯ್ಯಾಖ್ಯ
ಗೋವಿಂದರ
ಗೋವಿಂದರದೇವ
ಗೋವಿಂದರದೇವನ
ಗೋವಿಂದರದೇವನಿಗೂ
ಗೋವಿಂದರದೇವನಿಗೆ
ಗೋವಿಂದರದೇವನು
ಗೋವಿಂದರನ
ಗೋವಿಂದರನಾಗಿರಬಹುದು
ಗೋವಿಂದರನಿರಬಹುದು
ಗೋವಿಂದರನು
ಗೋವಿಂದರನೂ
ಗೋವಿಂದರನೇ
ಗೋವಿಂದರರಕ್ಕಸಗಂಗ
ಗೋವಿಂದರರಲ್ಲಿ
ಗೋವಿಂದರಸ
ಗೋವಿಂದರಸನು
ಗೋವಿಂದರಸನುಗೋವಿಂದಮಯ್ಯ
ಗೋವಿಂದರಾಜ
ಗೋವಿಂದರಾಜಗುರುಗಳ
ಗೋವಿಂದರಾಜಗುರುವಿಗೆ
ಗೋವಿಂದರಾಜಗುರುವಿನ
ಗೋವಿಂದರಾಜನನ್ನು
ಗೋವಿಂದರಾಜಯ್ಯನವರ
ಗೋವಿಂದರಾಜರ
ಗೋವಿಂದರಾಜರಿಗೆ
ಗೋವಿಂದವಾಡಿಯನ್ನು
ಗೋವಿದೇವಿಯರು
ಗೋವಿನಾಯಕ
ಗೋವುಗಳನ್ನು
ಗೌಂಡ
ಗೌಡ
ಗೌಡಗೆರೆ
ಗೌಡರನ್ನು
ಗೌಡರಿಗೆ
ಗೌಡರು
ಗೌಡರೇ
ಗೌಡಿಕೆ
ಗೌಡಿಕೆಯ
ಗೌಡಿಕೆಯನ್ನು
ಗೌಡಿಕೆಯು
ಗೌಡುಗಳ
ಗೌಡುಗಳು
ಗೌಡುಗೆರೆಯ
ಗೌಡುಪ್ರಜೆಗಳು
ಗೌಣವಾಗಿವೆ
ಗೌತಮ
ಗೌತಮಗೋತ್ರದ
ಗೌತಮಿ
ಗೌತಮೇಶ್ವರ
ಗೌರವ
ಗೌರವಕ್ಕೆ
ಗೌರವದಿಂದ
ಗೌರವಪೂರ್ವಕವಾಗಿ
ಗೌರವವನ್ನು
ಗೌರವವು
ಗೌರವಸೂಚಕ
ಗೌರವಸೂಚಕವಾದ
ಗೌರವಸೂಚಿ
ಗೌರಿ
ಗೌರಿಯಹಳ್ಳಿ
ಗೌರ್ನರ್
ಗ್ರಂಥ
ಗ್ರಂಥಗಳ
ಗ್ರಂಥಗಳನ್ನು
ಗ್ರಂಥಗಳಿಂದ
ಗ್ರಂಥಗಳು
ಗ್ರಂಥಲಿಪಿ
ಗ್ರಂಥಲಿಪಿಕನ್ನಡ
ಗ್ರಹಗತಿಗಳ
ಗ್ರಾಮ
ಗ್ರಾಮಂ
ಗ್ರಾಮಕ್ಕೆ
ಗ್ರಾಮಕ್ಕೆಸ್ಥಳ
ಗ್ರಾಮಗಳ
ಗ್ರಾಮಗಳನ್ನು
ಗ್ರಾಮಗಳನ್ನೇ
ಗ್ರಾಮಗಳಲ್ಲಿ
ಗ್ರಾಮಗಳಾಗಿದ್ದವು
ಗ್ರಾಮಗಳಾಗಿದ್ದವೆಂದು
ಗ್ರಾಮಗಳಾಗಿವೆ
ಗ್ರಾಮಗಳಿಂದ
ಗ್ರಾಮಗಳಿಗೆ
ಗ್ರಾಮಗಳಿದ್ದವು
ಗ್ರಾಮಗಳು
ಗ್ರಾಮಗಳುಳ್ಳ
ಗ್ರಾಮಗಳೂ
ಗ್ರಾಮಗಳೆಂದು
ಗ್ರಾಮಗಳೆಲ್ಲ
ಗ್ರಾಮಗಳೆಲ್ಲವೂ
ಗ್ರಾಮಗಳೇ
ಗ್ರಾಮಗಾಮುಂಡ
ಗ್ರಾಮಗೊಡಗೆಯನ್ನು
ಗ್ರಾಮಗೊಡುಗೆಯಾಗಿ
ಗ್ರಾಮದ
ಗ್ರಾಮದಲ್ಲಿ
ಗ್ರಾಮದಲ್ಲಿದ್ದ
ಗ್ರಾಮದಸೇನಬೋವ
ಗ್ರಾಮದೇವತಾಪುರದ
ಗ್ರಾಮದೇವತೆಯು
ಗ್ರಾಮನಾಮಗಳನ್ನು
ಗ್ರಾಮಪಂಚಕವೆಂದು
ಗ್ರಾಮಮಟ್ಟದಲ್ಲಿ
ಗ್ರಾಮವನು
ಗ್ರಾಮವನ್ನು
ಗ್ರಾಮವನ್ನುಧನಗೂರು
ಗ್ರಾಮವಾಗಿತು
ಗ್ರಾಮವಾಗಿತ್ತು
ಗ್ರಾಮವಾಗಿತ್ತೆಂದು
ಗ್ರಾಮವಾಗಿದೆ
ಗ್ರಾಮವಾಗಿದ್ದ
ಗ್ರಾಮವಾಗಿರಬಹುದು
ಗ್ರಾಮವಿತ್ತೆಂದು
ಗ್ರಾಮವು
ಗ್ರಾಮವೆಂದು
ಗ್ರಾಮವೇ
ಗ್ರಾಮಸಂಖ್ಯಾಧಾರಿತ
ಗ್ರಾಮಸಭೆಗಳು
ಗ್ರಾಮಸಭೆಗೆ
ಗ್ರಾಮಸಭೆಯ
ಗ್ರಾಮಾಧಿ
ಗ್ರಾಹ್ಯರಾಹುಃ
ಘಟಕ
ಘಟಕಗಳ
ಘಟಕಗಳಾಗಿದ್ದ
ಘಟಕಗಳಾಗಿದ್ದವು
ಘಟಕಗಳಾದ
ಘಟಕಗಳಿದ್ದವು
ಘಟಕಗಳು
ಘಟಕವಾಗಿ
ಘಟನೆ
ಘಟನೆಗಳನ್ನು
ಘಟನೆಗಳು
ಘಟನೆಯನ್ನು
ಘಟನೆಯು
ಘಟನೆಯೊಂದನ್ನು
ಘಟವೆಂಬ
ಘಟಿಸಿರಬೇಕೆಂದು
ಘಟೆಯಂ
ಘಟ್ಟ
ಘಟ್ಟವು
ಘಣ್ಟಮ್ಮ
ಘಣ್ಟಮ್ಮನು
ಘನಗಿರಿ
ಘನಗಿರಿಗೆ
ಘನಘೋರ
ಘನತೆವೆತ್ತ
ಘನವೃತ್ತ
ಘಮ್ಮನಾಯಕ
ಘರ್ಜಿಸುತ್ತಾ
ಘರ್ಷಣೆಗಳನ್ನು
ಘರ್ಷಣೆಗಳು
ಘರ್ಷಣೆಯ
ಘರ್ಷಣೆಯನ್ನು
ಘರ್ಷಣೆಯಲ್ಲಿ
ಘೃತಪರ್ವತದಾನವನ್ನು
ಘೇಣಾಂಕ
ಘೋಷಿಸಲಾಯಿತು
ಘೋಷಿಸಿ
ಘೋಷಿಸಿಕೊಂಡಿದ್ದನು
ಚ
ಚಂಗ
ಚಂಗಭೂಪನ
ಚಂಗವಾಡಿ
ಚಂಗವಾಡಿಯನ್ನು
ಚಂಗವಾಡಿಯಲ್ಲಿ
ಚಂಗಾಳ್ವನಂ
ಚಂಗಿಕುಳ
ಚಂಗಿಕುಳಕಮಳ
ಚಂಗಿಕುಳಕಮಳಮಾರ್ತ್ರಂಡನತುಳ
ಚಂದಪ್ಪವೊಡೆಯನ
ಚಂದಯ್ಯ
ಚಂದಯ್ಯನು
ಚಂದಲದೇವಿಯನ್ನು
ಚಂದಲದೇವಿಯರು
ಚಂದಹಳ್ಳಿಯ
ಚಂದಹಳ್ಳಿಯನ್ನು
ಚಂದಿಗ
ಚಂದಿಗಾಲು
ಚಂದಿಯಕ್ಕರ
ಚಂದ್ರ
ಚಂದ್ರಗಿರಿಗಳನ್ನು
ಚಂದ್ರಗುಪ್ತ
ಚಂದ್ರನಂತಹ
ಚಂದ್ರನಂದಿಯ
ಚಂದ್ರನನ್ನು
ಚಂದ್ರಮಾಶ್ಚಂದ್ರಕೀರ್ತಿಮಾನ್
ಚಂದ್ರಮೌಳಿ
ಚಂದ್ರಮೌಳಿಯಣ್ಣ
ಚಂದ್ರಮೌಳಿಯಣ್ಣನ
ಚಂದ್ರಮೌಳಿಯು
ಚಂದ್ರಮೌಳೀಶ್ವರ
ಚಂದ್ರಮೌಳೇಶ್ವರ
ಚಂದ್ರರೂಪೋ
ಚಂದ್ರವನದ
ಚಂದ್ರಶೇಖರ
ಚಂದ್ರೊಬ್ಬಲಬ್ಬೆಯನ್ನು
ಚಂನರಾಜ
ಚಂಪೂಕಾವ್ಯವನ್ನಾಗಿ
ಚಂಬಲ್ಲೀಪುರ
ಚಂಬಿನ
ಚಉಗಾವೆಯ
ಚಕಿತ
ಚಕ್ಕೆರೆ
ಚಕ್ತವರ್ತಿ
ಚಕ್ರ
ಚಕ್ರಕೊಳ
ಚಕ್ರಗೊಟ್ಟ
ಚಕ್ರವರ್ತಿ
ಚಕ್ರವರ್ತಿಗಳ
ಚಕ್ರವರ್ತಿಗಳೆಂದು
ಚಕ್ರವರ್ತಿಗೆ
ಚಕ್ರವರ್ತಿಯ
ಚಕ್ರವರ್ತಿಯಾಗಿದ್ದ
ಚಕ್ರವರ್ತಿಯಾದ
ಚಕ್ರವರ್ತಿಯಿಂದ
ಚಕ್ರವರ್ತಿಯು
ಚಕ್ರವರ್ತಿಯೂ
ಚಕ್ರಾಧಿಪತಿಯಾದ
ಚಕ್ರೇಶನ
ಚಕ್ರೇಶ್ವರ
ಚಟಯನಾಯಕನ
ಚಟುವಟಿಕೆಗಳ
ಚಟುವಟಿಕೆಗಳನ್ನು
ಚಟ್ಟಂಗೆರೆ
ಚಟ್ಟಣಕೆರೆಚಟ್ಟಂಗೆರೆ
ಚಟ್ಟಣಕೆರೆಚಟ್ಟಮೆರೆ
ಚಟ್ಟದೇವ
ಚಟ್ಟಪಯ್ಯ
ಚಟ್ಟಮಗೆರೆ
ಚಟ್ಟಮಗೆರೆಯ
ಚಟ್ಟಯ
ಚಟ್ಟಯ್ಯನಹಳ್ಳಿ
ಚಟ್ಟಲದೇವಿ
ಚಟ್ಟೇನಹಳ್ಳಿಯನ್ನು
ಚಟ್ಟೊಡೆಯ
ಚಟ್ಟೊಡೆಯನು
ಚತುರಂಗಬಲವನ್ನು
ಚತುರಙ್ಗ
ಚತುರಳಾಗಿದ್ದು
ಚತುರ್ತ್ಥವಂಶರು
ಚತುರ್ತ್ಥವಂಶರೊಳು
ಚತುರ್ಥ
ಚತುರ್ಥಕುಲದವರು
ಚತುರ್ಥಗೋತ್ರ
ಚತುರ್ಥಗೋತ್ರದ
ಚತುರ್ದ್ಧಶವಿದ್ಯಾಸ್ಥಾನಾಧಿಗಮವಿಮಲಮತಿಃ
ಚತುರ್ವೇದಿ
ಚತುರ್ವೇದಿಮಂಗಲದ
ಚತುರ್ವೇದಿಮಂಗಲವೆಂಬ
ಚತುರ್ವ್ವಿಧಾನೂನದಾನವಿನೋದಂ
ಚತುಷಷ್ಟಿಕಳಾಕಳಿತ
ಚತುಸಮಯಸಮುದ್ಧರಣಂ
ಚತುಸಮುದ್ರಾಧಿಪತಿ
ಚತುಸ್ಸಮುದ್ರಾಧಿಪತಿ
ಚತುಸ್ಸೀಮೆಗೆ
ಚತುಸ್ಸೀಮೆಯನ್ನು
ಚತುಸ್ಸೀಮೆಯಲಿ
ಚತುಸ್ಸೀಮೆಯಾಗಿ
ಚತುಸ್ಸೀಮೆಯೊಳಗಾದ
ಚತುಸ್ಸೀಮೆಯೊಳಗೆ
ಚತ್ರರಿಗೆಬಹುಶಃ
ಚತ್ರಾಧಿಕಾರಿ
ಚತ್ರಾಧಿಕಾರಿಯಾಗಿದ್ದನು
ಚನ್ದಯ್ಯನು
ಚನ್ನಕೇಶವ
ಚನ್ನಕೇಶವದೇವಾಲಯವನ್ನು
ಚನ್ನಕೇಶವಪುರವೆಂಬ
ಚನ್ನನಂಜರಾಜನಿಗೇ
ಚನ್ನನಂಜರಾಜನೆಂದು
ಚನ್ನಪಟ್ಟಣ
ಚನ್ನಪಟ್ಟಣದ
ಚನ್ನಪಟ್ಟಣರಾಜ್ಯದ
ಚನ್ನಪಟ್ಟಣರಾಜ್ಯವನ್ನು
ಚನ್ನಪಟ್ಟಣವನ್ನು
ಚನ್ನಪಟ್ಟಣಸ್ಥಳಕ್ಕೆ
ಚನ್ನಪ್ಪನದೊಡ್ಡಿಯಲ್ಲಿದೆ
ಚನ್ನಪ್ಪನು
ಚನ್ನಯ್ಯನು
ಚನ್ನರಾಜ
ಚನ್ನರಾಯಪಟ್ಟಣ
ಚಪೂತ
ಚಮದ್ರಮೌಳಿಯ
ಚಮೂಧರ
ಚಮೂಪ
ಚಮೂಪತಿ
ಚಮೂಪತಿಯ
ಚಮೂಪನ
ಚಮೂಪನು
ಚಮೂಪನೋ
ಚಮೂಪರೆಂದೂ
ಚಮ್ಮಾವುಗೆಯ
ಚರಣಗಳಿಗೆ
ಚರಣಾಕ್ಯನೆನಲು
ಚರಣಾರವಿಂದ
ಚರಪಿಗೆ
ಚರಮಗೀತೆಯಂತಿದೆ
ಚರಿತಂ
ಚರಿತಃ
ಚರಿತೆಯಲ್ಲಿ
ಚರಿತ್ರೆಯ
ಚರಿತ್ರೆಯನ್ನು
ಚರಿತ್ರೆಯಲ್ಲಿ
ಚರಿತ್ರೆಯಿಂದ
ಚರುಪಿಗೆ
ಚರ್ಪನ್ನು
ಚಲಕದೇವ
ಚಲಕೆಬಲುಗಂಡ
ಚಲದ
ಚಲದುತ್ತರಂಗ
ಚಲನವಲನಗಳನ್ನು
ಚಲನೆ
ಚಲಾಯಿಸಲು
ಚಲುಕ್ಯ
ಚಲುಕ್ಯರ
ಚಲುವವ್ವೆಯರ
ಚವರಂ
ಚವರಬಂಬಾಳು
ಚವುಗಾವುಗಳ
ಚವುಗಾವುಗಳು
ಚವುಗಾವೆ
ಚವುಡಪ್ಪ
ಚವುಡಪ್ಪನ
ಚವುಡಯ್ಯ
ಚವುಡಯ್ಯನಹಳ್ಳಿ
ಚವುಡಿಗೌಡ
ಚವುಡೆಗೊಂಡನಿಗೆ
ಚವುಡೆಗೌಡನಿಗೆ
ಚವುಡೋಜನ
ಚಾಕಲೆ
ಚಾಕಲೆಯ
ಚಾಕಳಹಳ್ಳಿಯ
ಚಾಕೇನಹಳ್ಳಿಯ
ಚಾಗಿಪೆರ್ಮಾನಡಿಗಳ
ಚಾಣಕ್ಯನೆನಿಪ
ಚಾತುರ್ವರ್ಣದಲ್ಲಿ
ಚಾತುರ್ವ್ವೈದ್ಯ
ಚಾಮ
ಚಾಮಂಡಹಳ್ಳಿ
ಚಾಮಡಹಳ್ಳಿ
ಚಾಮಣ್ಣನು
ಚಾಮದೇವನಿರಬಹುದೆಂದು
ಚಾಮನೃಪನಿಗೆ
ಚಾಮನೃಪನುಬೋಳುಚಾಮರಾಜ
ಚಾಮಪ್ಪನು
ಚಾಮಯ್ಯ
ಚಾಮರ
ಚಾಮರಸ
ಚಾಮರಸಗೌಡನು
ಚಾಮರಸನು
ಚಾಮರಸವೊಡೆಯನು
ಚಾಮರಸವೊಡೆಯರ
ಚಾಮರಸವೊಡೆಯರಿಗೆ
ಚಾಮರಸವೊಡೆಯರು
ಚಾಮರಸೊಡೆಯರವರ
ಚಾಮರಸೊಡೆಯರೈಯನವರ
ಚಾಮರಾಜ
ಚಾಮರಾಜನ
ಚಾಮರಾಜನಗರ
ಚಾಮರಾಜನಗರದ
ಚಾಮರಾಜನಗರದಲ್ಲಿ
ಚಾಮರಾಜನನ್ನುಹತ್ತನೆಯ
ಚಾಮರಾಜನಿಗೆ
ಚಾಮರಾಜನು
ಚಾಮರಾಜನೆಂದು
ಚಾಮರಾಜನೆಂಬ
ಚಾಮರಾಜನೇ
ಚಾಮರಾಜವೊಡೇರ
ಚಾಮರಾಜೇಂದ್ರ
ಚಾಮರಾಜೊಡೆಯರ
ಚಾಮರಾಜೊಡೆಯರಿಗೆ
ಚಾಮರಾಜೊಡೆಯರು
ಚಾಮಲದೇವಿ
ಚಾಮಲಾಪುರ
ಚಾಮಲಾಪುರದ
ಚಾಮಲಾಪುರವನ್ನು
ಚಾಮಲೆಯು
ಚಾಮವ್ವೆ
ಚಾಮಾಂಬಿಕೆಗೆ
ಚಾಮುಂಡನು
ಚಾಮುಂಡರಾಯ
ಚಾಮುಂಡರಾಯಂ
ಚಾಮುಂಡರಾಯನ
ಚಾಮುಂಡರಾಯನಿಗೆ
ಚಾಮುಂಡರಾಯನಿರಬಹುದು
ಚಾಮುಂಡರಾಯನು
ಚಾಮುಂಡರಾಯನೆಂಬ
ಚಾಮುಂಡರಾಯನೆಂಬುವವನು
ಚಾಮುಣ್ಡಯ್ಯನೂ
ಚಾಮುಣ್ಡರಿಬ್ಬರೂ
ಚಾಮೇಂದ್ರನಿಗೆ
ಚಾರಗೌಂಡ
ಚಾರಿತ್ರ
ಚಾರಿತ್ರನುಂ
ಚಾರಿತ್ರಲಕ್ಷ್ಮೀಕರ್ಣ್ನಪೂರಂ
ಚಾರಿತ್ರಿಕ
ಚಾರುಪೊನ್ನೇರ
ಚಾಲುಕ್ಯ
ಚಾಲುಕ್ಯರ
ಚಾಲುಕ್ಯರನ್ನು
ಚಾಲುಕ್ಯರಿಗೆ
ಚಾಳಿಸಿ
ಚಾಳುಕ್ಯ
ಚಾಳುಕ್ಯವಿಕ್ರಮಕಾಲದ
ಚಾವಗೌಂಡ
ಚಾವಡಿ
ಚಾವಡಿಗಳನ್ನಾಗಿ
ಚಾವಡಿಯ
ಚಾವಡಿಯಲ್ಲಿ
ಚಾವಡಿಯಾಗಿರಬಹುದು
ಚಾವಡಿಯು
ಚಾವಯ್ಯ
ಚಾವಲದೇವಿ
ಚಾವಾಟ
ಚಾವುಂಡ
ಚಾವುಂಡರಾಯಚಾಮುಂಡರಾಯ
ಚಾವುಂಡರಾಯನ
ಚಾವುಂಡವ್ವೆಯರ
ಚಾವುಣ್ಡ
ಚಾವುಣ್ಡನು
ಚಾವುಣ್ಡನೆಂಬ
ಚಿಂಣ್ನಂ
ಚಿಂತಿಸಿದನೆಂದು
ಚಿಂಮತೂರು
ಚಿಂಮತ್ತೂರು
ಚಿಕಕೇತಯ
ಚಿಕಗವುಡ
ಚಿಕದೇವರಾಜ
ಚಿಕದೇವರಾಜನ
ಚಿಕದೇವರಾಜನವರೆಗೆ
ಚಿಕದೇವರಾಜನು
ಚಿಕದೇವರಾಜವಿಜಯದಲ್ಲಿ
ಚಿಕದೇವರಾಯನೃಪತೀ
ಚಿಕದೇವರಾಯವಂಶಾವಳಿಯಲ್ಲೂ
ಚಿಕರಾಜನ
ಚಿಕವಂಗಲವಕ್ಕೆ
ಚಿಕವಡೆಯ
ಚಿಕವಡೆಯರು
ಚಿಕವೀರಗವುಡ
ಚಿಕಸಿದ್ಧಯ್ಯಗವುಡ
ಚಿಕ್ಕ
ಚಿಕ್ಕಅಬ್ಬಾಗಿಲಿನ
ಚಿಕ್ಕಅರಸಿನಕೆರೆ
ಚಿಕ್ಕಅಲ್ಲಪ್ಪ
ಚಿಕ್ಕಅಲ್ಲಪ್ಪನಾಯಕ
ಚಿಕ್ಕಅಲ್ಲಪ್ಪನಾಯಕನು
ಚಿಕ್ಕಅಲ್ಲಪ್ಪನಾಯಕರ
ಚಿಕ್ಕಕಂನೆಯ
ಚಿಕ್ಕಕಂನೆಯನಹಳ್ಳಿಯನ್ನು
ಚಿಕ್ಕಕಂಪಣ್ಣನು
ಚಿಕ್ಕಕಳಲೆ
ಚಿಕ್ಕಕೇತಣ್ಣ
ಚಿಕ್ಕಕೇತಯ
ಚಿಕ್ಕಕೇತಯನು
ಚಿಕ್ಕಕೇತಯ್ಯ
ಚಿಕ್ಕಕೇತಯ್ಯದಂಡನಾಯಕ
ಚಿಕ್ಕಕೇತಯ್ಯನು
ಚಿಕ್ಕಕೇತಯ್ಯನೇ
ಚಿಕ್ಕಕೇತೆಯ
ಚಿಕ್ಕಕೇತೆಯನು
ಚಿಕ್ಕಕೇತೆಯ್ಯನಾಯಕನು
ಚಿಕ್ಕಗಂಗವಾಡಿ
ಚಿಕ್ಕಗಂಗವಾಡಿಯ
ಚಿಕ್ಕಗ್ರಾಮಗಳಿಗೆ
ಚಿಕ್ಕಜಟಕ
ಚಿಕ್ಕಜಟ್ಟಿಗಹಳ್ಳಿಇಂದಿನ
ಚಿಕ್ಕದೇವರಾಜ
ಚಿಕ್ಕದೇವರಾಜನ
ಚಿಕ್ಕದೇವರಾಜನು
ಚಿಕ್ಕದೇವರಾಜರ
ಚಿಕ್ಕದೇವರಾಜೇಂದ್ರ
ಚಿಕ್ಕದೇವರಾಯನಿಗಿಂತ
ಚಿಕ್ಕದೇವೇಂದ್ರನು
ಚಿಕ್ಕನಹಳ್ಳಿ
ಚಿಕ್ಕನಾಯಕನಪುರದಿಂದ
ಚಿಕ್ಕನಾಯಕನಹಳ್ಳಿ
ಚಿಕ್ಕನಾಯಕರು
ಚಿಕ್ಕಪ್ಪ
ಚಿಕ್ಕಪ್ಪನ
ಚಿಕ್ಕಪ್ಪನಹಳ್ಳಿ
ಚಿಕ್ಕಪ್ಪನಿಂದ
ಚಿಕ್ಕಪ್ರದೇಶದ
ಚಿಕ್ಕಬಯಿಚಪ್ಪ
ಚಿಕ್ಕಬಳ್ಳಿ
ಚಿಕ್ಕಬಾಗಿಲು
ಚಿಕ್ಕಬೆಟ್ಟಕ್ಕೆ
ಚಿಕ್ಕಬೆಟ್ಟದ
ಚಿಕ್ಕಬೆಟ್ಟದಲ್ಲಿ
ಚಿಕ್ಕಬೆಳೂರ
ಚಿಕ್ಕಬೆಳೂರು
ಚಿಕ್ಕಬ್ಬೆಹಳ್ಳಿ
ಚಿಕ್ಕಮಂಟೆಯಕ್ಕೆ
ಚಿಕ್ಕಮಂಟೆಯಚಿಕ್ಕಮಂಡ್ಯ
ಚಿಕ್ಕಮಂಠೆಯಚಿಕ್ಕಮಂಡ್ಯ
ಚಿಕ್ಕಮಂಠೆಯವೇ
ಚಿಕ್ಕಮಂಡ್ಯ
ಚಿಕ್ಕಮಗಳೂರು
ಚಿಕ್ಕಮರಲಿ
ಚಿಕ್ಕಮಲ್ಲಯ್ಯನಾಯಕನ
ಚಿಕ್ಕಮಲ್ಲೆಯನಾಯಕನ
ಚಿಕ್ಕಮಲ್ಲೆಯನಾಯಕನು
ಚಿಕ್ಕಮಲ್ಲೆಯನಾಯಕನೆಂಬ
ಚಿಕ್ಕಮಳಲಿ
ಚಿಕ್ಕಮಾಯಿ
ಚಿಕ್ಕಯಗಟಿ
ಚಿಕ್ಕರಸ
ಚಿಕ್ಕರಾಜ
ಚಿಕ್ಕರಾಮರಾಜನು
ಚಿಕ್ಕರಾಯ
ಚಿಕ್ಕರಾಯನ
ಚಿಕ್ಕರಾಯನಿಗೆ
ಚಿಕ್ಕರಾಯನು
ಚಿಕ್ಕರಾಯನೂ
ಚಿಕ್ಕರಾಯಪಟ್ಟಣ
ಚಿಕ್ಕರಾಯಪಟ್ಟಣವನಾಳುವ
ಚಿಕ್ಕರಾಯಪಟ್ಟಣವನ್ನಾಳುತ್ತಿದ್ದನೆಂದು
ಚಿಕ್ಕರಾಯಪಟ್ಟವೇ
ಚಿಕ್ಕರಾಯಪುರವೆಂಬ
ಚಿಕ್ಕರಾಯಪ್ಪನವರಿಗೆ
ಚಿಕ್ಕರಾಯರು
ಚಿಕ್ಕರಾಯಸಾಗರ
ಚಿಕ್ಕಲಿಂಗನಕೊಪ್ಪಲು
ಚಿಕ್ಕವಡ್ಡರಗುಡಿ
ಚಿಕ್ಕವನಹಳ್ಳಿ
ಚಿಕ್ಕವನಾಗಿರುವಾಗಲೇ
ಚಿಕ್ಕವೊಡೆಯ
ಚಿಕ್ಕವೊಡೆಯನೆಂಬ
ಚಿಕ್ಕವೋಡೆ
ಚಿಕ್ಕಸಾದಿಪ್ಪ
ಚಿಕ್ಕಸಾದಿಪ್ಪನಿಗೆ
ಚಿಕ್ಕಸಾದಿಪ್ಪನು
ಚಿಕ್ಕಸಾದಿಯಪ್ಪನಿಗೆ
ಚಿಕ್ಕಸಾದಿಯಪ್ಪನೂ
ಚಿಕ್ಕಹಡೆವಳ್ಳ
ಚಿಕ್ಕಹರಿಯಲೆ
ಚಿಕ್ಕಹೊಸಹಳ್ಳಿ
ಚಿಕ್ಕಾಡೆ
ಚಿಕ್ಕೆಯನಾಯಕ
ಚಿಕ್ಕೆಹಳ್ಳಿಮಂಡ್ಯ
ಚಿಕ್ಕೊಡೆಯನಿಗೆ
ಚಿಕ್ಕೊಡೆಯರ
ಚಿಕ್ಕೊಲೆ
ಚಿಗುಡಹಳ್ಳಿ
ಚಿಗುಡಹಳ್ಳಿತಿಗಡಹಳ್ಳಿ
ಚಿಗುಲಿಹಳ್ಳಿ
ಚಿಟ್ಟನಹಳ್ಳಿ
ಚಿಟ್ಟನಹಳ್ಳಿಯ
ಚಿಣ್ಣ
ಚಿಣ್ಣನನ್ನು
ಚಿಣ್ಣನು
ಚಿಣ್ಣಯ್ಯ
ಚಿಣ್ಣಯ್ಯನು
ಚಿಣ್ನಂ
ಚಿಣ್ನಯ
ಚಿಣ್ನಯನಾಡ
ಚಿತ್ತಮಂ
ಚಿತ್ತವಲ್ಲಭೆಯಾದ
ಚಿತ್ತೈಸಿ
ಚಿತ್ರಕಾರರೂ
ಚಿತ್ರಣ
ಚಿತ್ರಣವನ್ನು
ಚಿತ್ರದುರ್ಗ
ಚಿತ್ರದುರ್ಗದಲ್ಲಿ
ಚಿತ್ರದುರ್ಗವು
ಚಿತ್ರಭಾನು
ಚಿತ್ರಮಕೊಂಡನಾಯಕನ
ಚಿತ್ರವನ್ನು
ಚಿತ್ರವಿದೆ
ಚಿದಾನಂದ
ಚಿದಾನಂದಮೂರ್ತಿಯವರ
ಚಿದಾನಂದಮೂರ್ತಿಯವರು
ಚಿನಕುರಳಿ
ಚಿನಕುರಳಿಗೆ
ಚಿನಕುರಳಿಬೆಟ್ಟ
ಚಿನಕುರಳಿಯ
ಚಿನಕುರಳಿಯಲ್ಲಿ
ಚಿನ್ನದೇವಚೋಡಮಹಾಅರಸನು
ಚಿನ್ನಮ
ಚಿನ್ನಾದೇವಿಪುರ
ಚಿನ್ನಾದೇವಿಯ
ಚಿಮತೂರಕಲ್ಲ
ಚಿಮತೂರಬೆಮತೂರ
ಚಿಮತೂರಿನ
ಚಿಮತ್ತೂರಕಲ್ಲು
ಚಿಮ್ಮತ್ತನಕಲ್ಲು
ಚಿಮ್ಮತ್ತನೂರಿನ
ಚಿಮ್ಮತ್ತೂರಕಲ್ಲ
ಚಿಮ್ಮತ್ತೂರು
ಚಿರತೆ
ಚಿರತೆಗಳು
ಚಿರತೆಗಳೂ
ಚಿರಸ್ಥಾಯಿಯಾಗಿ
ಚಿಲಕುರ್ಲಿಚಿನಕುರಳಿ
ಚಿಹ್ನೆಯಾಗಿ
ಚಿಹ್ನೆಯಾದ
ಚೀಣ್ಯ
ಚುಂಚನ
ಚುಂಚನಕೋಟೆ
ಚುಂಚನಕೋಟೆಯ
ಚುಂಚನಗಿರಿ
ಚುಂಚನಗಿರಿಯ
ಚುಂಚನಭಯಿರವ
ಚುಂಚನಹಳ್ಳಿ
ಚುಂಚನಹಳ್ಳಿಯನ್ನು
ಚೂಡಾಮಣಿ
ಚೂರ್ಣೀಕರಿಸಿದ
ಚೆಂಗವಾಡಿಯ
ಚೆಂಗವಾಡಿಯನ್ನು
ಚೆಂಗವಾಡಿಯಲ್ಲಿ
ಚೆಂಗಾಳ್ವನಂ
ಚೆಂಗಾಳ್ವನಾಗಿರಬಹುದೆಂದು
ಚೆಂಗಾಳ್ವನು
ಚೆಂಗಾಳ್ವರ
ಚೆಂಗಾಳ್ವರನ್ನು
ಚೆಂಗಾಳ್ವರು
ಚೆಂಗಿರಿಯನ್ನು
ಚೆಂಡಾಡಿದನೆಂದು
ಚೆಂದಾಪುರವೆಂಬ
ಚೆಂಬಿನ
ಚೆಂಬೊಂಗಳಂ
ಚೆಟ್ಟಹಳ್ಳಿ
ಚೆಟ್ಟಿಯರನ್ನು
ಚೆನ್ನಕೇಶವ
ಚೆನ್ನಕೇಶವದೇವರ
ಚೆನ್ನಕೇಶವದೇವಾಲಯದ
ಚೆನ್ನಕೇಶವನಿಗೆ
ಚೆನ್ನಕೇಶವಪದಾಂಭೋಜಕಮಳಿನೀಕಳಹಂಸಭಿನವಪ್ರಹರಾಜ
ಚೆನ್ನಕೇಶವಪುರವೆಂಬ
ಚೆನ್ನಕ್ಕ
ಚೆನ್ನದೀಕ್ಷಿತನಿಗೆ
ಚೆನ್ನದೇವಚೋಡಮಹಾಅರಸನು
ಚೆನ್ನದೇವಚೋಡಮಹಾಅರಸು
ಚೆನ್ನನಂಜರಾಜ
ಚೆನ್ನಯ್ಯನೆಂಬುವವನ
ಚೆನ್ನರಸ
ಚೆನ್ನರಾಜಯ್ಯನು
ಚೆನ್ನರಾಯಪಟ್ಟಣ
ಚೆನ್ನರಾಯ್ಯನವರಿಗೆ
ಚೆನ್ನಾದೇವಿಪುರವೆಂಬ
ಚೆನ್ನಿಸೆಟ್ಟಿಯರ
ಚೆಲುಪಿಳ್ಳೆದೇವರ
ಚೆಲುವದೇವಾಂಬುಧಿ
ಚೆಲುವನಾರಾಯಣ
ಚೆಲುವನಾರಾಯಣನ
ಚೆಲುವನಾರಾಯಸ್ವಾಮಿ
ಚೆಲುವಪಿಳ್ಳೆ
ಚೆಲುವಪಿಳ್ಳೆದೇವರಿಗೆ
ಚೆಲುವಪಿಳ್ಳೆರಾಯರ
ಚೆಲುವಪಿಳ್ಳೆರಾಯರಿಗೆ
ಚೆಲುವವ್ವೆಯರ
ಚೆಲುವಾಂಬಾ
ಚೆಲ್ಲಬಹುದಾಗಿದೆ
ಚೆಲ್ಲುತ್ತದೆ
ಚೆಲ್ವಡರಾಯನೆಂಬ
ಚೆಲ್ವಾಜಮಾಂಬ
ಚೆಲ್ವಾಜಮಾಂಬೆಯ
ಚೇರ
ಚೇರಮನಹಳ್ಳಿಯನ್ನು
ಚೈತ್ಯಾಲಯವನ್ನು
ಚೈತ್ಯಾಲಯವೆಂದೂ
ಚೊಕ್ಕಜಿನಾಲಯಕ್ಕೆ
ಚೊಕ್ಕಣ್ಣನ
ಚೊಕ್ಕನಾಥನ
ಚೊಟ್ಟನಹಳ್ಳಿ
ಚೊತ್ತರಳಿ
ಚೊಳನಿಗೆ
ಚೋಕಲ
ಚೋಕಲದೇವಿ
ಚೋಕಲದೇವಿಗೆ
ಚೋಕವ್ವೆ
ಚೋಳ
ಚೋಳಕೊಟ್ಟ
ಚೋಳಗಂಗ
ಚೋಳಗವುಂಡನು
ಚೋಳಗವುಡ
ಚೋಳಗೌಂಡನು
ಚೋಳತುರುನಾಡನ್ನು
ಚೋಳದೇಶದಲ್ಲಿ
ಚೋಳನ
ಚೋಳನಕೋಟೆ
ಚೋಳನನ್ನು
ಚೋಳನರಾಜಾದಿತ್ಯ
ಚೋಳನಾಡನ್ನಾಗಿ
ಚೋಳನಾಡನ್ನು
ಚೋಳನಿಗೆ
ಚೋಳನು
ಚೋಳನೆ
ಚೋಳನೆಂಬ
ಚೋಳಪರಾಂತಕ
ಚೋಳಪರಾಂತಕನ
ಚೋಳಪುರದ
ಚೋಳಪ್ಪಯ್ಯನ
ಚೋಳಬಲವನ್ನು
ಚೋಳಭೂಮಿಯ
ಚೋಳಮಂಡಲದ
ಚೋಳಮಹಾಅರಸ
ಚೋಳಮಹಾಅರಸುಗಳ
ಚೋಳರ
ಚೋಳರಕಾಲದ
ಚೋಳರನ್ನು
ಚೋಳರಾಜನನ್ನು
ಚೋಳರಾಜನೇ
ಚೋಳರಾಜೇಂದ್ರನು
ಚೋಳರಾಜ್ಯ
ಚೋಳರಾಜ್ಯಕ್ಕೆ
ಚೋಳರಾಜ್ಯದ
ಚೋಳರಾಜ್ಯದಲ್ಲಿ
ಚೋಳರಾಯ
ಚೋಳರಾಯಸ್ಥಾಪನಾಚಾರ್ಯ್ಯ
ಚೋಳರಿಂದ
ಚೋಳರಿಗೂ
ಚೋಳರು
ಚೋಳರೊಡನೆ
ಚೋಳಲಾಳಾದಿಗಳು
ಚೋಳಸಿಂಹಾಸನದ
ಚೋಳಸಿಂಹಾಸನದಲ್ಲಿ
ಚೋಳಸೇನೆಯನ್ನು
ಚೋಳಿ
ಚೋಳಿಕರನ್ನು
ಚೋಳೆಯನಹಳ್ಳಿ
ಚೋಳೋರ್ವ್ವಿಯಂ
ಚೌಕಟ್ಟನ್ನು
ಚೌಗಾವೆಯ
ಚೌಡಪ್ಪನ
ಚೌಡಹಳ್ಳಿಯಲ್ಲಿರುವ
ಚೌತ
ಚೌತನ್ನು
ಚೌತವನ್ನು
ಚ್ಚರಿಸುವರೆಲ್ಲರುಂ
ಛತ್ರ
ಛತ್ರಗಳ
ಛತ್ರಗಳನ್ನು
ಛತ್ರಛಾಯೆಯಿಂ
ಛತ್ರವನ್ನು
ಛಾಯಾಚಿತ್ರಗಳನ್ನು
ಜಂಗಮರು
ಜಂಗಮೊಡೆಯರ
ಜಂಗುಳಿ
ಜಂಗುಳಿಯವರು
ಜಂಟಿ
ಜಂಟಿಯಾಗಿ
ಜಕಗೌಡನ
ಜಕ್ಕಣಬ್ಬೆ
ಜಕ್ಕಣಬ್ಬೆಯ
ಜಕ್ಕಣಬ್ಬೆಯರ
ಜಕ್ಕಣ್ಣನಾಯಕನಿಗೆ
ಜಕ್ಕನಹಳ್ಳಿ
ಜಕ್ಕಯ್ಯ
ಜಕ್ಕಲದೇವಿ
ಜಕ್ಕಲೆಗೆ
ಜಕ್ಕಿಕಟ್ಟೆ
ಜಕ್ಕಿಮವ್ವೆಯು
ಜಗ
ಜಗಕಾರಿಗ
ಜಗತೀಸಾಮ್ರಾಜ್ಯದೀಕ್ಷಾಂ
ಜಗದಾಳಮೊನೆಯೊಳು
ಜಗದುಗೋಪಾಳ
ಜಗದೇಕ
ಜಗದೇಕಮಲ್ಲ
ಜಗದೇಕಮಲ್ಲನು
ಜಗದೇಕವೀರನು
ಜಗದೇಕವೊಡೆಯನ
ಜಗದೇವಕಅರಾಯ
ಜಗದೇವನ
ಜಗದೇವರಾಯ
ಜಗದೇವರಾಯನ
ಜಗದೇವರಾಯನಿಗೆ
ಜಗದೇವರಾಯನು
ಜಗದೇವರಾಯನೆಂಬ
ಜಗನ್ನಾಥವಿಜಯವೆಂಬ
ಜಗಳದಲ್ಲಿ
ಜಗಳವು
ಜಗಳೂರು
ಜಗ್ಗಗವುಡನ
ಜಟಕ
ಜಟಾವರ್ಮಸುಂದರಪಾಂಡ್ಯನ
ಜಟ್ಟಿಗವನ್ನು
ಜಡೆಯ
ಜಡೆಯದ
ಜತ್ತ
ಜನ
ಜನಕ
ಜನಕನ
ಜನಗನ್ನುತನಾ
ಜನಗಳಿಗೋಸ್ಕರ
ಜನಜನಿತವಾದ
ಜನಜೀವನ
ಜನಜೀವನವು
ಜನತಾಧಾರನುದಾರನನ್ಯ
ಜನತಾಪ್ರಿಯೇಣ
ಜನತೆಯನ್ನು
ಜನನ
ಜನನದ
ಜನನವಾಗಿದೆ
ಜನನಿ
ಜನನಿಯ
ಜನಪದ
ಜನಪ್ರಿಯ
ಜನರ
ಜನರಲ್
ಜನರಲ್ಲಿ
ಜನರಿಂದ
ಜನರಿಗೆ
ಜನರು
ಜನರೂ
ಜನವರಿ
ಜನಸಂಖ್ಯೆಯು
ಜನಹಳ್ಳಿಯ
ಜನಾಂಗದಲ್ಲಿ
ಜನಾಂಗದವರು
ಜನಾರ್ದನ
ಜನಾರ್ದನದೇವರ
ಜನಿಸಿ
ಜನಿಸಿದ
ಜನಿಸಿದನು
ಜನಿಸಿದನೆಂದು
ಜನಿಸಿಬಂದು
ಜನಿಸುತ್ತೇನೆ
ಜನೋಪಕಾರಿ
ಜನ್ನನ
ಜನ್ಮಕ್ಷೇತ್ರವನ್ನಾಗಿ
ಜನ್ಮತಪಃ
ಜನ್ಮೋತ್ಸವದಂದು
ಜಮರಂಣನು
ಜಮರಣ್ಣನು
ಜಮೀನಿನ
ಜಯ
ಜಯಜೀಯವರ್ದ್ಧನಕರಂ
ಜಯತಿ
ಜಯತ್ಯಸೌ
ಜಯದ
ಜಯದುತ್ತರಂಗ
ಜಯಪತ್ರದ
ಜಯಮ್ಮನ
ಜಯರೇಖೆಯನ್ನು
ಜಯಲಕ್ಷ್ಮಿಗಿತ್ತು
ಜಯಲಕ್ಷ್ಮಿಗೆ
ಜಯವನ್ನು
ಜಯಸಿಂಹನನ್ನು
ಜಯಸ್ಥಂಭವನೆತ್ತಿಸಿ
ಜಯಾಂಗನಾವಲ್ಲಭಂ
ಜಯಿಸಿ
ಜಯಿಸಿಕೊಳ್ಳಿ
ಜಯಿಸಿದ
ಜಯಿಸಿದನಂತೆ
ಜಯಿಸಿದನೆಂದು
ಜಯಿಸಿದ್ದ
ಜರಯ್ಯಂ
ಜಲ
ಜಲಕ್ರೀಡೆಯಾಡುತ್ತಿದ್ದನೆಂದು
ಜಲದುರ್ಗ
ಜಲಧಾಮ
ಜಲಾಶಯದ
ಜಲಾಶಯವು
ಜಲ್ಮೋತ್ಸವ
ಜಲ್ಲ
ಜಳಧಿಯೊಳ್
ಜವನಿಕೆ
ಜವನಿಕೆನಾರಾಯಣ
ಜವನಿಕೆಯೊಡಲಿರ್ವ್ವಲದ
ಜವನೊಡನಾದಡಂ
ಜವಾದಿ
ಜವಾದಿಕೋಳಾಹಳ
ಜವಾಬ್ದಾರಿಯನ್ನು
ಜವಾಬ್ದಾರಿಯುತವಾದ
ಜಸವತರ
ಜಸಹಿತದೇವ
ಜಹಾಂಗೀರ್
ಜಾಗ
ಜಾಗದಲ್ಲಿ
ಜಾಗನಕೆರೆ
ಜಾಗವನ್ನು
ಜಾಗವಿದ್ದು
ಜಾಗವು
ಜಾತಿ
ಜಾತಿಯ
ಜಾತ್ಯಶ್ವದಿಂ
ಜಾತ್ರೆ
ಜಾನಪದ
ಜಾನಪದೀಯ
ಜಾರಿಗೆ
ಜಾರಿಯಲ್ಲಿತ್ತು
ಜಾಸ್ತಿ
ಜಾಸ್ತಿಯಾಗಿ
ಜಾಸ್ತಿಯಾಗಿದ್ದ
ಜಾಸ್ತಿಯಾದಾಗ
ಜಿಂಕೆ
ಜಿಂಕೆಯಂತೆ
ಜಿಂಜಿಯ
ಜಿಆರ್ಕುಪ್ಪುಸ್ವಾಮಿ
ಜಿಎಸ್ದೀಕ್ಷಿತ್
ಜಿಎಸ್ದೀಕ್ಷಿತ್ರವರು
ಜಿತಪಾರ್ಶ್ವಂ
ಜಿನಗನ್ಧೋದಕ
ಜಿನಚಂದ್ರಪಂಡಿತನಿಗೆ
ಜಿನದೇವನ
ಜಿನದೇವನನಜಿತಸೇನಮುನಿಪವರ
ಜಿನದೊಣೆಲಕ್ಕದೊಣೆಯ
ಜಿನಧರ್ಮಾಗ್ರಣಿ
ಜಿನನಾಥಪುರ
ಜಿನನಾಥಪುರದಲ್ಲಿ
ಜಿನನಾಥಪುರವನ್ನು
ಜಿನಪಾದಪಂಕಜ
ಜಿನಪಾರ್ಶ್ವದೇವರ
ಜಿನಪಾರ್ಶ್ವನಾಥ
ಜಿನಪೂಜೆಗೆ
ಜಿನಭಕ್ತ
ಜಿನಮುಖಚಂದ್ರವಾಕ್ಚಂದ್ರಿಕಾಚಕೋರಂ
ಜಿನಮುನಿ
ಜಿನಮುನಿಯಲ್ಲಿ
ಜಿನರಾಜರಾಜತ್ಪೂಜಾಪುರಂದರಂ
ಜಿನಶಾಸನರಕ್ಷಾಮಣಿ
ಜಿನಸಮಯ
ಜಿನಾರ್ಚನಲುಬ್ಧಂ
ಜಿನಾರ್ಚನೆಗೆ
ಜಿನಾಲಯ
ಜಿನಾಲಯಕ್ಕೆ
ಜಿನಾಲಯದ
ಜಿನಾಲಯವನ್ನು
ಜಿನಾಲಯವನ್ನೂ
ಜಿನಾಲಯವೆಂದು
ಜಿಫ್ರಿ
ಜಿಲೆಯಲ್ಲಿದ್ದು
ಜಿಲ್ಲಾ
ಜಿಲ್ಲಾವಾರು
ಜಿಲ್ಲೆ
ಜಿಲ್ಲೆಗಳ
ಜಿಲ್ಲೆಗಳನ್ನು
ಜಿಲ್ಲೆಗಳಲ್ಲಿ
ಜಿಲ್ಲೆಗಳಲ್ಲಿರುವ
ಜಿಲ್ಲೆಗಳಾಗಿ
ಜಿಲ್ಲೆಗಳಿಗೆ
ಜಿಲ್ಲೆಗಳಿದ್ದವು
ಜಿಲ್ಲೆಗೆ
ಜಿಲ್ಲೆಯ
ಜಿಲ್ಲೆಯನ್ನು
ಜಿಲ್ಲೆಯಲ್ಲಿ
ಜಿಲ್ಲೆಯಲ್ಲಿಯೂ
ಜಿಲ್ಲೆಯಲ್ಲಿಯೇ
ಜಿಲ್ಲೆಯಲ್ಲಿರುವ
ಜಿಲ್ಲೆಯಲ್ಲಿವೆ
ಜಿಲ್ಲೆಯಲ್ಲೂ
ಜಿಲ್ಲೆಯಾಗಿದೆ
ಜಿಲ್ಲೆಯಿಂದ
ಜಿಲ್ಲೆಯು
ಜೀಗುಂಡಿಪಟ್ಟಣ
ಜೀಗುಂಡಿಪಟ್ಟಣದಲ್ಲೂ
ಜೀಯ
ಜೀಯನಿಗೆ
ಜೀಯನೆಂಬ
ಜೀಯಾತು
ಜೀಯಾದಾಚ್ಚಂದ್ರತಾರಕಂ
ಜೀರಹಳ್ಳಿ
ಜೀರಹಳ್ಳಿಗಳುಅಂಬಲಜೀರಹಳ್ಳಿ
ಜೀರಿಗೆಯೊಕ್ಕಲಿಕ್ಕಿ
ಜೀರ್ಣವಾಗಿದೆ
ಜೀರ್ಣವಾಗಿದ್ದಾಗ
ಜೀರ್ಣವಾಗಿರಲು
ಜೀರ್ಣವಾಗಿವೆ
ಜೀರ್ಣವಾದ
ಜೀರ್ಣೋ
ಜೀರ್ಣೋದ್ಧಾರ
ಜೀರ್ಣೋದ್ಧಾರಕ
ಜೀರ್ಣೋದ್ಧಾರಕ್ಕೆ
ಜೀರ್ಣೋದ್ಧಾರವನ್ನು
ಜೀವಂತವಾಗಿ
ಜೀವಿತ
ಜೀವಿತವನ್ನು
ಜೀವಿತವಾಗಿ
ಜೀವಿತವೂ
ಜೀವಿತಾವಧಿಯ
ಜೀವಿಸಿದ್ದನು
ಜೀವಿಸಿದ್ದನೆಂದು
ಜೀವಿಸಿರುವವರೆಗೆ
ಜುಂಜಾಪುರ
ಜುಮ್ಮಾಮಸೀದಿ
ಜುಲೈ
ಜೂನ್
ಜೂನ್ಜುಲೈ
ಜೆಂನಿಗೆ
ಜೆಎಂನಾಗಯ್ಯನವರು
ಜೆಟ್ಟಿಗದ
ಜೆಟ್ಟಿಗದಲ್ಲಿ
ಜೆಡಲ
ಜೇನುಗುಡ್ಡ
ಜೈತಾಜಿಕಾಟ್ಕರ್
ಜೈತಾಜಿಯರನ್ನು
ಜೈತುಗಿಯ
ಜೈದೇವ
ಜೈನ
ಜೈನಕೇಂದ್ರ
ಜೈನತೀರ್ಥಗಳಿಗೆ
ಜೈನಧರ್ಮಕ್ಕೆ
ಜೈನಧರ್ಮದ
ಜೈನಧರ್ಮನಿರ್ಮಳಾಂಬರ
ಜೈನಧರ್ಮವನ್ನು
ಜೈನಪುರಾಣಗಳಲ್ಲಿ
ಜೈನಬಸದಿಗಳನ್ನು
ಜೈನಬಸದಿಗಳಿಗೆ
ಜೈನಬಸದಿಗೆ
ಜೈನಬಸದಿಯು
ಜೈನಬಸ್ತಿಯ
ಜೈನಮತಾವಲಂಬಿಯಾಗಿದ್ದಳು
ಜೈನಮುನಿ
ಜೈನಮುನಿಗಳು
ಜೈನಯತಿಗಳಿಗೆ
ಜೈನರ
ಜೈನವೈಷ್ಣವ
ಜೊತಗೆ
ಜೊತೆ
ಜೊತೆಗಿದ್ದು
ಜೊತೆಗೂಡಿ
ಜೊತೆಗೆ
ಜೊತೆಗೆಹೋಗಿ
ಜೊತೆಗೇ
ಜೊತೆಯಲ್ಲಿ
ಜೊತೆಯಲ್ಲಿದ್ದ
ಜೊತೆಯಲ್ಲಿಯೇ
ಜೊತೆಯಾಗಿ
ಜೊರೆಗ
ಜೊಸೆಫ್
ಜೋಗುಂಡಯ್ಯ
ಜೋಗುಣ್ಡಯ್ಯನಿಗೆ
ಜೋಡು
ಜೋಸೆಫ್
ಜ್ಞಾಪಕಾರ್ಥವಾಗಿ
ಜ್ವರದ
ಝರಾಧರಾ
ಝುಮ್ರಾ
ಟಂಕಸಾಲೆ
ಟಂಕಸಾಲೆಗೆ
ಟನಿಟುರ
ಟಪ್ಪುಸುಲ್ತಾನನು
ಟಿಪು
ಟಿಪುವಿನ
ಟಿಪ್ಪಣಿಗಳನ್ನು
ಟಿಪ್ಪವಿನ
ಟಿಪ್ಪು
ಟಿಪ್ಪುವನ್ನು
ಟಿಪ್ಪುವಿಗ
ಟಿಪ್ಪುವಿನ
ಟಿಪ್ಪುವು
ಟಿಪ್ಪುಸುಲ್ತಾನನು
ಟಿಪ್ಪುಸುಲ್ತಾನ್
ಟಿಪ್ಪೂ
ಟಿಪ್ಪೂವಿನ
ಟಿಪ್ಪೂಸುಲ್ತಾನನ
ಟಿಪ್ಪೂಸುಲ್ತಾನನು
ಟಿಪ್ಪೂಸುಲ್ತಾನ್
ಟೇಕಲ್
ಠಾಣೆಯ
ಡಂಕನ್
ಡಂಕನ್ಡೆರೆಟ್
ಡಂಕನ್ಡೆರೆಟ್ರವರು
ಡಾ
ಡಾಅಲನರಸಿಂಹನ್
ಡಾಎಂ
ಡಾಎಂಎಂ
ಡಾಎಂಚಿದಾನಂದಮೂರ್ತಿಯವರು
ಡಾಎಂಬಿಪದ್ಮ
ಡಾಎಚ್ಎಸ್
ಡಾಎವಿನರಸಿಂಹಮೂರ್ತಿಯವರು
ಡಾಎಸತ್ಯನಾರಾಯಣ
ಡಾಎಸ್
ಡಾಎಸ್ಎನ್
ಡಾಎಸ್ಗುರುರಾಜಾಚಾರ್ಯರ
ಡಾಎಸ್ನಾಗರಾಜು
ಡಾಎಸ್ರಂಗರಾಜು
ಡಾಕರಸ
ಡಾಕರಸನನ್ನು
ಡಾಕರಸನೆಂದು
ಡಾದೇವರಕೊಂಡಾರೆಡ್ಡಿ
ಡಾದೇವರಕೊಂಡಾರೆಡ್ಡಿಯವರು
ಡಾಬಿಆರ್
ಡಾಬಿಶೇಕ್ಅಲಿ
ಡಾರಾಧಾಪಟೇಲ್
ಡಾಸಾಲೆತೂರ್
ಡಾಸೂರ್ಯನಾಥಕಾಮತ್
ಡಿಂಕ
ಡಿಡಗದ
ಡಿವಿಜನ್
ಡಿವಿಜನ್ಗೆ
ಡಿವಿಜನ್ನಲ್ಲಿ
ಡಿಸೆಂಬರ್
ಡೆಂಕಣಿಕೋಟೆ
ಡೆಕ್ಕನ್
ಡೆರೆಟ್
ಡೆರೆಟ್ರವರು
ಡೊಣೆ
ಡೊಣೆಗಳು
ಡೊಣೆಯಿದ್ದು
ಢಣಾಯಕನ
ತ
ತಂಗಡಗಿ
ತಂಗಿ
ತಂಗಿದ್ದ
ತಂಗಿದ್ದು
ತಂಗಿಯ
ತಂಗಿಯನ್ನೇನಾದರೂ
ತಂಗಿರಬಹುದೆಂದು
ತಂಜಾವೂರನ್ನು
ತಂಜಾವೂರು
ತಂಡವತಂದು
ತಂಡಸೇಹಳ್ಳಿ
ತಂತಕನಂತೆಸಂಗರದೊಳೋವದೆ
ತಂತ್ರಗಳ
ತಂತ್ರವೆಗ್ಗಡೆ
ತಂತ್ರವೆಗ್ಗಡೆತನವನ್ನು
ತಂತ್ರವೆಗ್ಗಡೆಯು
ತಂತ್ರಾಧಿಷ್ಟಾಯಕ
ತಂತ್ರಾಧಿಷ್ಠಾಯಕ
ತಂದ
ತಂದಂತೆ
ತಂದಿದ್ದ
ತಂದು
ತಂದೆ
ತಂದೆಗಳ
ತಂದೆಗೆ
ತಂದೆತಾಯಿಗಳೆಂದು
ತಂದೆಯ
ತಂದೆಯಕಾಲದಲ್ಲೇ
ತಂದೆಯಗಂಧವಾರಣ
ತಂದೆಯನ್ನು
ತಂದೆಯಾಗಿದ್ದು
ತಂದೆಯಾದ
ತಂದೆಯೊಲ್
ತಂನ
ತಂನನುಜಾತರ್ಬ್ಬೋಕಣಂ
ತಂನಾಮ್ನಾಯದ
ತಂಮ
ತಂಮಂಗೆ
ತಂಮಡಿಗಟ್ಟೆಯ
ತಂಮಯ
ತಂಮೆಯದೇವ
ತಕ್ಕ
ತಕ್ಕಂತೆ
ತಕ್ಕೋಲಂ
ತಕ್ಕೋಲಂನಲ್ಲಿ
ತಕ್ಕೋಲದಲ್ಲಿ
ತಕ್ಕೋಲದೊಲ್ಕಾದಿ
ತಕ್ಕೋಲಮ್
ತಕ್ಷಣ
ತಕ್ಷಣದಲ್ಲಿ
ತಕ್ಷಣದಲ್ಲಿಯೇ
ತಗಚೆಗೆರೆಗಳನ್ನು
ತಗಡೂರಿನಲ್ಲಿ
ತಗಡೂರು
ತಗರೆ
ತಗ್ಗಲೂರು
ತಗ್ಗಿನ
ತಗ್ಗಿಳೂರು
ತಗ್ಗುಪ್ರದೇಶದಲ್ಲಿದ್ದುದರಿಂದ
ತಜ್ಜವನಿಕೆಗೊಂಡಗಂಡ
ತಜ್ಞರಾದ
ತಜ್ಞರು
ತಟದ
ತಟ್ಟೇಹಳ್ಳಿ
ತಡರೆಬಲ್ಗಂಡನುಂ
ತಡಿಮಾಲಿಂಗಿ
ತಡಿಯಲ್ಲಿದ್ದ
ತಡೆಗಟ್ಟಲು
ತಡೆದಿರಬಹುದೆಂದು
ತಡೆದು
ತಡೆಯಲಾಗದೆ
ತಡೆಯುವ
ತತೇಜೋನಿಳಯಂ
ತತ್ಪುತ್ರಃ
ತತ್ರಯಿಕ
ತತ್ವ
ತತ್ವೈಕನಿಷ್ಠುರ
ತತ್ಸಾಹಸಾಭ್ಯುದಯಂ
ತದನು
ತದನುಜನ್ಮಾ
ತನಂಗಾಡಿ
ತನಂಗಾಡಿಯಾದ
ತನಕ
ತನಗೆ
ತನದಕಯ್ಯ
ತನದಟ್ಟಿಬಡಿವಂ
ತನಯ
ತನಯಃ
ತನಯರ
ತನುಜರು
ತನೂಭವ
ತನೆಯರು
ತನ್ನ
ತನ್ನನ್ನು
ತನ್ನೂರಾದ
ತನ್ನೊಂದೆ
ತಪಃಫಲ
ತಪಸಿಯ
ತಪಸಿಯತಪಸಿಹಳ್ಳಿ
ತಪಸೀರಾಯನ
ತಪಸ್ಸು
ತಪುವರಾಯರ
ತಪೋಧನನಿಗೆ
ತಪ್ಪದೆ
ತಪ್ಪದೇ
ತಪ್ಪಾಗಿ
ತಪ್ಪಾಗಿರುವ
ತಪ್ಪಾಗುವುದು
ತಪ್ಪಿದ್ದು
ತಪ್ಪುವ
ತಪ್ಪುವನಾಯಕರ
ತಪ್ಪುವರಾಯರ
ತಪ್ಪೆತಪ್ಪುವಂ
ತಮಗೆ
ತಮಿಳು
ತಮಿಳುಗ್ರಂಥಲಿಪಿಯ
ತಮಿಳುನಾಡನ್ನು
ತಮಿಳುನಾಡಿನ
ತಮಿಳುನಾಡಿನಲ್ಲಿ
ತಮಿಳುನಾಡಿನಲ್ಲಿದ್ದನು
ತಮಿಳುನಾಡಿನಲ್ಲೇ
ತಮಿಳುನಾಡು
ತಮಿಳುರೂಪ
ತಮಿಳುಶಾಸನದಲ್ಲಿ
ತಮ್ಮ
ತಮ್ಮಂದಿರಿರಬಹುದು
ತಮ್ಮಂದಿರು
ತಮ್ಮಂದಿರೆಂದು
ತಮ್ಮಂದಿರೋ
ತಮ್ಮಡಿಹಳ್ಳಿ
ತಮ್ಮತೀರ್ವ್ವರ್ಗ್ಗೆ
ತಮ್ಮನ
ತಮ್ಮನಂತೆ
ತಮ್ಮನನ್ನೋ
ತಮ್ಮನಾದ
ತಮ್ಮನಿದ್ದನು
ತಮ್ಮನಿದ್ದನೆಂದು
ತಮ್ಮನಿರಬಹುದು
ತಮ್ಮನೀರ್ವ್ವರ್ಗೆ
ತಮ್ಮನೂ
ತಮ್ಮನೆಂದು
ತಮ್ಮನೆಂದೂ
ತಮ್ಮನ್ನು
ತಮ್ಮಯಣ್ಣ
ತಮ್ಮವ್ವೆ
ತಮ್ಮೊಳಗೆ
ತಮ್ಮೋಜಿ
ತಯಾರಿಸಿ
ತಯಾರಿಸುತ್ತಿದ್ದರೆಂದು
ತಯಾರು
ತರಲಾಗುತ್ತಿತ್ತು
ತರಿದಿಕ್ಕಿದನು
ತರಿದುಹಾಕಿದ
ತರಿಸಿ
ತರಿಸಿಕೊಡುತ್ತಾರೆ
ತರುಣದಲ್ಲಿಯೇ
ತರುವ
ತರುವಾಯ
ತರೈ
ತರ್ದವಾಡಿ
ತಲಕಾಡ
ತಲಕಾಡನ್ನು
ತಲಕಾಡಾದ
ತಲಕಾಡಿಗೆ
ತಲಕಾಡಿನ
ತಲಕಾಡಿನಲ್ಲಿ
ತಲಕಾಡಿನಲ್ಲಿದ್ದ
ತಲಕಾಡಿನಲ್ಲಿದ್ದನೆಂದು
ತಲಕಾಡಿನಲ್ಲಿದ್ದಾಗ
ತಲಕಾಡಿನವರೆಗೂ
ತಲಕಾಡಿನಿಂದ
ತಲಕಾಡು
ತಲಕಾಡುಗೊಂಡ
ತಲಗವಾಡಿ
ತಲೆ
ತಲೆಕಾಡಿನ
ತಲೆಗಳಿಯಿಸಿಕೊಂಡು
ತಲೆಗೆ
ತಲೆಗೊಂಡನಿರದೆ
ತಲೆದೋರಿತೆಂದು
ತಲೆನೆರೆಯಲ
ತಲೆಬಾಗಿರಲಿಲ್ಲವೆಂದು
ತಲೆಮಾರಿಗೆ
ತಲೆಮಾರಿನ
ತಲೆಮಾರಿನವರು
ತಲೆಮಾರು
ತಲೆಮಾರುಗಳ
ತಲೆಯ
ತಲೆಯನ್ನು
ತಲೆಯಮಾಳೆಯ
ತಳಕಾಡ
ತಳಕಾಡಂ
ತಳಕಾಡಅಧಿಕಾರಿ
ತಳಕಾಡನಾಡ
ತಳಕಾಡನಾಡನ್ನು
ತಳಕಾಡನಾಡಪ್ರಭು
ತಳಕಾಡಪ್ರಭು
ತಳಕಾಡಿನ
ತಳಕಾಡು
ತಳಕಾಡುಗೊಂಡನೆಂದು
ತಳಕಾಡುನಾಡು
ತಳಪಾದಿಯ
ತಳವನಪುರ
ತಳವನಪುರತಲಕಾಡು
ತಳವನಪುರತಲಕಾಡುವಿಜಯಸ್ಕಾಂದವಾರದಲ್ಲಿದ್ದಾಗರಾಜಧಾನಿ
ತಳವನಪುರವೇ
ತಳವಾರ
ತಳವಾರರಿಗೆ
ತಳವಾರರಿದ್ದರು
ತಳವಾರಿಕೆ
ತಳವಾರಿಕೆಗಳಿಂದ
ತಳವಾರಿಕೆಗೆ
ತಳವಾರಿಕೆಯ
ತಳವಾರಿಕೆಯಿಂದ
ತಳವೃತ್ತಿಯನ್ನು
ತಳವ್ರಿತ್ತಿಯಂ
ತಳಾರ
ತಳಾರರು
ತಳಾರಿ
ತಳಾರಿಕೆಯು
ತಳಿಗ್ರಾಮದ
ತಳಿಯಣ್ಣ
ತಳಿಯೂರನ್ನು
ತಳೆಕಾಡಂ
ತಳೆಕಾಡಸೀಮೆಯ
ತಳೆಕಾಡುಪಟ್ಟಣ
ತಳೆದನೆಂದು
ತಳೆದಿದ್ದರು
ತಳೆದು
ತಳ್ತಿಯದ
ತಳ್ತಿರಿದು
ತಳ್ತಿರಿವೆಡೆಗೊರ್ವ್ವರಪ್ಪೊಡಮಿದಿರ್ಚುವ
ತಳ್ತಿರ್ದ್ದ
ತಳ್ಳಿಯದ
ತಳ್ಳಿಹಾಕ
ತವಿಸಿದನು
ತಸ್ಯ
ತಸ್ಯಾ
ತಸ್ಯಾತ್ಮಜೋ
ತಾ
ತಾಂ
ತಾಂಜಂ
ತಾಂಡವೇಶ್ವರ
ತಾಂಬೂಲ
ತಾಂಬೂಲಸೇವೆಯ
ತಾಗಲಾರಂಭಿಸಿತು
ತಾಗಿ
ತಾಗಿತ್ತು
ತಾಣ
ತಾಣಪನಹಳ್ಳಿ
ತಾತ
ತಾತಾಚಾರ್ಯನಿಗೆ
ತಾತಾಚಾರ್ಯರಿಗೆ
ತಾನಿತ್ತು
ತಾನು
ತಾನೂ
ತಾನೆ
ತಾನೇ
ತಾಮೊನೆಬೆನ್ನಬಾರನೆ
ತಾಮ್ರ
ತಾಮ್ರಪಟಗಳಲ್ಲಿ
ತಾಮ್ರಪಟಗಳಿಂದ
ತಾಮ್ರಪಟಗಳು
ತಾಮ್ರಪಟದಲ್ಲಿ
ತಾಮ್ರಪಟದಲ್ಲಿದೆ
ತಾಮ್ರಶಾಸಗಳು
ತಾಮ್ರಶಾಸನ
ತಾಮ್ರಶಾಸನಗಳ
ತಾಮ್ರಶಾಸನಗಳನ್ನು
ತಾಮ್ರಶಾಸನಗಳಲ್ಲಿ
ತಾಮ್ರಶಾಸನಗಳಲ್ಲೂ
ತಾಮ್ರಶಾಸನಗಳು
ತಾಮ್ರಶಾಸನದ
ತಾಮ್ರಶಾಸನದಅಲ್ಲಿ
ತಾಮ್ರಶಾಸನದಲ್ಲಿ
ತಾಮ್ರಶಾಸನದಲ್ಲಿದೆ
ತಾಮ್ರಶಾಸನದಲ್ಲೂ
ತಾಮ್ರಶಾಸನದಿಂದ
ತಾಮ್ರಶಾಸನವನ್ನು
ತಾಮ್ರಶಾಸನವಾಗಿದ್ದು
ತಾಮ್ರಶಾಸನವಿದ್ದು
ತಾಮ್ರಶಾಸನವು
ತಾಮ್ರಶಾಸನವೇ
ತಾಮ್ರಶಾಸನ್
ತಾಯ
ತಾಯಣ್ಣನು
ತಾಯಲೂರಿನ
ತಾಯಲೂರಿನಲ್ಲಿರುವ
ತಾಯಲೂರು
ತಾಯಿ
ತಾಯಿಗಳ
ತಾಯಿಗಳೇ
ತಾಯಿಮಂಚಿಯಕ್ಕನ
ತಾಯಿಯ
ತಾಯಿಸಾಕುತಾಯಿ
ತಾಯೂರು
ತಾಯ್ಮುದ್ದರಸಿ
ತಾರೀಖನ್ನು
ತಾರೀಖಿನ
ತಾರೀಖಿನಂದು
ತಾರೀಖಿನಂದೇ
ತಾರೀಖು
ತಾರೀಖೆಂದು
ತಾಲೂಕುಗಳು
ತಾಲ್ಲೂಕನ್ನಾಗಿ
ತಾಲ್ಲೂಕನ್ನು
ತಾಲ್ಲೂಕನ್ನೂ
ತಾಲ್ಲೂಕಾಗಿದೆ
ತಾಲ್ಲೂಕಾಗಿದ್ದು
ತಾಲ್ಲೂಕಿ
ತಾಲ್ಲೂಕಿಗೂ
ತಾಲ್ಲೂಕಿಗೆ
ತಾಲ್ಲೂಕಿನ
ತಾಲ್ಲೂಕಿನಅ
ತಾಲ್ಲೂಕಿನಲ್ಲಿ
ತಾಲ್ಲೂಕಿನಲ್ಲಿದೆ
ತಾಲ್ಲೂಕಿನಲ್ಲಿದ್ದರೆ
ತಾಲ್ಲೂಕಿನಲ್ಲಿದ್ದವೆಂದು
ತಾಲ್ಲೂಕಿನಲ್ಲಿರುವ
ತಾಲ್ಲೂಕು
ತಾಲ್ಲೂಕುಗಳ
ತಾಲ್ಲೂಕುಗಳನ್ನು
ತಾಲ್ಲೂಕುಗಳನ್ನೊಳಗೊಂಡ
ತಾಲ್ಲೂಕುಗಳಲ್ಲಿ
ತಾಲ್ಲೂಕುಗಳಲ್ಲಿದ್ದ
ತಾಲ್ಲೂಕುಗಳವೆರೆಗೆ
ತಾಲ್ಲೂಕುಗಳಾಗಿ
ತಾಲ್ಲೂಕುಗಳಾದ
ತಾಲ್ಲೂಕುಗಳಿದ್ದವು
ತಾಲ್ಲೂಕುಗಳು
ತಾಲ್ಲೂಕುಗಳೂ
ತಾಲ್ಲೂಕುಗಳೆಂಬ
ತಾಲ್ಲೂಕುವಾರು
ತಾಲ್ಲೂಕ್ನಲ್ಲಿ
ತಾಳಗುಂದ
ತಾಳತಿಟ್ಟು
ತಾಳ್ದಿದಂ
ತಾವರೆಕೆರೆಯ
ತಾವಾಗಿಯೇ
ತಾವು
ತಾವೂ
ತಾವೇ
ತಿಂಗಳ
ತಿಂಗಳಲ್ಲಿ
ತಿಂಗಳವರೆಗೆ
ತಿಂಗಳಿನಲ್ಲಿ
ತಿಂಗಳು
ತಿಂಮಂಣ್ನ
ತಿಂಮಣ
ತಿಂಮಪನಾಯಕ
ತಿಂಮರಾಜಗಳ
ತಿಕ್ಕಮ
ತಿಕ್ಕಮನನ್ನು
ತಿಕ್ಕಾಟ
ತಿಗಡಹಳ್ಳಿ
ತಿಗಳ
ತಿಗಳನು
ತಿಗಳರು
ತಿಗುಳನಕೆರೆ
ತಿಟ್ಟು
ತಿನರಸಿಪುರ
ತಿನರಸೀಪುರ
ತಿನಿರಾಯ
ತಿನಿರಾಯಾಂಬಾಮುಲ
ತಿಪಟೂರು
ತಿಪ್ಪಣ್ಣ
ತಿಪ್ಪಣ್ಣನಾಯಕನೆಂಬುನನೂ
ತಿಪ್ಪಣ್ಣನಾಯಕರಿಗೆ
ತಿಪ್ಪಯ್ಯನ
ತಿಪ್ಪಯ್ಯನು
ತಿಪ್ಪರಸರು
ತಿಪ್ಪರಸಾರ್ಯನಪುತ್ರ
ತಿಪ್ಪಳಿನಾಯಕನು
ತಿಪ್ಪವ್ವೆ
ತಿಪ್ಪವ್ವೆಗೂ
ತಿಪ್ಪಾಜಿ
ತಿಪ್ಪಾಜಿಯ
ತಿಪ್ಪೂರ
ತಿಪ್ಪೂರಿಗೆ
ತಿಪ್ಪೂರು
ತಿಪ್ಪೂರೇ
ತಿಪ್ಪೆಯೂರಿನ
ತಿಪ್ಪೆಯೂರು
ತಿಪ್ಪೆರುವಳ್ಳಿಯ
ತಿಪ್ಪೆರೂರನ್ನು
ತಿಪ್ಪೆರೂರಿನ
ತಿಪ್ಪೆರೂರು
ತಿಬ್ಬನಹಳ್ಳಿ
ತಿಬ್ಬನಹಳ್ಳಿಗೆ
ತಿಬ್ಬನಹಳ್ಳಿಯ
ತಿಬ್ಬನಹಳ್ಳಿಯನ್ನು
ತಿಬ್ಬಸೆಟ್ಟಿಯು
ತಿಮ್ಮ
ತಿಮ್ಮಅರಸಯ್ಯ
ತಿಮ್ಮಕವಿಯ
ತಿಮ್ಮಜಗದೇವರಾಯನ
ತಿಮ್ಮಣ್ಣ
ತಿಮ್ಮಣ್ಣದಂಡನಾಯಕನ
ತಿಮ್ಮಣ್ಣನಾಯಕ
ತಿಮ್ಮಣ್ಣನು
ತಿಮ್ಮಣ್ಣನ್ನು
ತಿಮ್ಮನ
ತಿಮ್ಮನಾಯಕನ
ತಿಮ್ಮನಾಯಕನು
ತಿಮ್ಮನು
ತಿಮ್ಮಪ್ಪ
ತಿಮ್ಮಪ್ಪನಾಯಕ
ತಿಮ್ಮಪ್ಪಯ್ಯ
ತಿಮ್ಮಯ್ಯದೇವ
ತಿಮ್ಮಯ್ಯನ
ತಿಮ್ಮರಸನು
ತಿಮ್ಮರಸಯ್ಯನು
ತಿಮ್ಮರಸರು
ತಿಮ್ಮರಾಜ
ತಿಮ್ಮರಾಜನ
ತಿಮ್ಮರಾಜನಿಗೆ
ತಿಮ್ಮರಾಜನು
ತಿಮ್ಮರಾಜಯ್ಯ
ತಿಮ್ಮರಾಜು
ತಿಮ್ಮಸಮುದ್ರ
ತಿಮ್ಮಾಂಬೆಯರ
ತಿರಸ್ಕರಿಸಿದ್ದಾರೆ
ತಿರಿಮಣ್ಣ
ತಿರುಕುಡಿ
ತಿರುಗಾಡುತ್ತಾ
ತಿರುಗಿ
ತಿರುಗಿಬಿದ್ದ
ತಿರುಗಿಬಿದ್ದನು
ತಿರುಗಿಬಿದ್ದರೆಂದೂ
ತಿರುಗಿಬಿದ್ದಿರುವ
ತಿರುಗಿಬಿದ್ದು
ತಿರುಚಿರಾಪಳ್ಳಿ
ತಿರುಣನಾಯಕ
ತಿರುನಂದಾದೀಪಕ್ಕೆ
ತಿರುನಾರಾಯಣ
ತಿರುನಾರಾಯಣದೇವರಿಗೆ
ತಿರುನಾರಾಯಣಪುರ
ತಿರುನಾರಾಯಣಪುರಕ್ಕೆ
ತಿರುನಾಳಿಗೆ
ತಿರುಪತಿ
ತಿರುಪತಿಯಲ್ಲಿ
ತಿರುಪ್ರತಿಷ್ಠೆಯನ್ನು
ತಿರುಮಂಜನಕ್ಕೆ
ತಿರುಮಕೂಡಲು
ತಿರುಮಕೂಡಲುನರಸೀಪುರ
ತಿರುಮಕೂಡು
ತಿರುಮಲ
ತಿರುಮಲಗಿರಿಯ
ತಿರುಮಲದೇವ
ತಿರುಮಲದೇವರ
ತಿರುಮಲದೇವರಿಗೆ
ತಿರುಮಲನ
ತಿರುಮಲನನ್ನು
ತಿರುಮಲನಾಥ
ತಿರುಮಲನಾಥನ
ತಿರುಮಲನಾಥನು
ತಿರುಮಲನಾಯಕನ
ತಿರುಮಲನಿಂದ
ತಿರುಮಲನು
ತಿರುಮಲನೆಂಬ
ತಿರುಮಲನೇ
ತಿರುಮಲಮ್ಮ
ತಿರುಮಲಯಾರ್ಯೋವ್ಯತಾನೀತ್ತಾಂಬ್ರ
ತಿರುಮಲರಾಜ
ತಿರುಮಲರಾಜನ
ತಿರುಮಲರಾಜನನ್ನು
ತಿರುಮಲರಾಜನಾಯಕಗೆ
ತಿರುಮಲರಾಜನಿಗೂ
ತಿರುಮಲರಾಜನು
ತಿರುಮಲರಾಜನೆಂದು
ತಿರುಮಲರಾಜಯದೇವ
ತಿರುಮಲರಾಜಯ್ಯ
ತಿರುಮಲರಾಜಯ್ಯದೇವ
ತಿರುಮಲರಾಜಯ್ಯನ
ತಿರುಮಲರಾಜಯ್ಯನವರು
ತಿರುಮಲರಾಜಯ್ಯನು
ತಿರುಮಲರಾಜಯ್ಯನೆಂಬ
ತಿರುಮಲರಾಜಯ್ಯನೇ
ತಿರುಮಲರಾಜರು
ತಿರುಮಲರಾಜು
ತಿರುಮಲರಾಯ
ತಿರುಮಲರಾಯರ
ತಿರುಮಲಾಚಾರ್ಯರು
ತಿರುಮಲಾರ್ಯ
ತಿರುಮಲಾರ್ಯನಿಂದ
ತಿರುಮಲಾರ್ಯನು
ತಿರುಮಲೆ
ತಿರುಮಲೆಯಾರ್ಯರ
ತಿರುಮಲೈಯ್ಯಂಗಾರರ
ತಿರುಮಾಲೆ
ತಿರುಳುನಾಡು
ತಿರುವಣ್ಣಾಮಲೆಗೂ
ತಿರುವಣ್ಣಾಮಲೆಯನ್ನೇ
ತಿರುವಣ್ಣಾಮಲೆಯಲ್ಲಿದ್ದುಕೊಂಡು
ತಿರುವನಂತಪುರದ
ತಿರುವರಂಗದಾಸನು
ತಿರುವರುಂಗದಾಸನ
ತಿರುವವ್ವೆ
ತಿರುವಿಂದಳೂರ
ತಿರುವಿಡಿಯಾಟಕ್ಕೆ
ತಿರುವಿಡಿಯಾಟದ
ತಿರುವಿಡಿಯಾಟ್ಟಕ್ಕೆ
ತಿರುವೆಂಕಟನಾಯಕ
ತಿರುವೆಂಕಟಾದ್ರಿ
ತಿರುವೆಂಗಟಯ್ಯನ
ತಿರುವೆಂಗಳನಾಥ
ತಿರುವೆಂಗಳನಾಥನ
ತಿರುವೇಂಕಟನಾಯಕನು
ತಿರೆ
ತಿಲಕನೆನಿಸಿದ
ತಿಲಕರಂತೆ
ತಿಲಕರು
ತಿಲಿಕೂತ್ತಾಂಡಿ
ತಿಲೆ
ತಿಲ್ಲೆಕೂತ್ತ
ತಿಳಂಕಂಗೀ
ತಿಳಕ
ತಿಳದುಬರುತ್ತದೆ
ತಿಳಿದ
ತಿಳಿದಬರುತ್ತದೆ
ತಿಳಿದಿದೆ
ತಿಳಿದಿದೆಯಾಗಿ
ತಿಳಿದಿರುವ
ತಿಳಿದು
ತಿಳಿದುಕೊಂಡಿದ್ದರು
ತಿಳಿದುಕೊಳ್ಳಬಹುದು
ತಿಳಿದುಕೊಳ್ಳುವುದು
ತಿಳಿದುಬರುತ್ತದೆ
ತಿಳಿದುಬರುತ್ತದೆಲಾಳನಕೆರೆಯ
ತಿಳಿದುಬರುತ್ತದೆೆ
ತಿಳಿದುಬರುತ್ತವೆ
ತಿಳಿದುಬರುತ್ತೆ
ತಿಳಿದುಬರುವ
ತಿಳಿದುಬರುವುದರಿಂದ
ತಿಳಿದುಬರುವುದಿಲ್ಲ
ತಿಳಿಯದ
ತಿಳಿಯದು
ತಿಳಿಯಬಹುದು
ತಿಳಿಯಬಾರದು
ತಿಳಿಯುತ್ತದೆ
ತಿಳಿಯುವುದು
ತಿಳಿವಳಿಕೆ
ತಿಳಿಸಿದ
ತಿಳಿಸಿದಂತೆ
ತಿಳಿಸಿದನು
ತಿಳಿಸಿದನೆಂದೂ
ತಿಳಿಸುತ್ತದೆ
ತಿಳಿಸುತ್ತದೆಂದು
ತಿಳಿಸುತ್ತವೆ
ತಿಳಿಸುತ್ತಿದೆ
ತಿಳಿಸುವ
ತಿಳಿಸುವಂತೆ
ತಿಳಿಸುವಲ್ಲಿ
ತೀತಳಮರಿವನ್ತು
ತೀರ
ತೀರದ
ತೀರದಲ್ಲಿ
ತೀರದಲ್ಲಿತ್ತು
ತೀರದಲ್ಲಿರುವ
ತೀರದವರೆಗೂ
ತೀರಪ್ರದೇಶ
ತೀರಾ
ತೀರಿಕೊಂಡನು
ತೀರಿಕೊಂಡನೆಂದು
ತೀರಿಕೊಂಡಾಗ
ತೀರಿಕೊಂಡಿದ್ದನು
ತೀರಿಕೊಂಡಿರಬೇಕು
ತೀರಿಕೊಳ್ಳಲು
ತೀರಿಸಿಕೊಳ್ಳುವ
ತೀರ್ಥಂಕರರ
ತೀರ್ಥಕ್ಕೆ
ತೀರ್ಥದ
ತೀರ್ಥದಲ್ಲಿ
ತೀರ್ಥಯಾತ್ರಾದಿಗಳಲ್ಲಿ
ತೀರ್ಥವನ್ನು
ತೀರ್ಥವಾಗಿತ್ತೆಂದು
ತೀರ್ಥವು
ತೀರ್ಥವುಇಂದಿನ
ತೀರ್ಥವೆಂದು
ತೀರ್ಮಾನಿಸಬಹುದು
ತೀವ್ರ
ತು
ತುಂಗಭದ್ರಾ
ತುಂಗಭದ್ರಾತೀರ
ತುಂಗಭದ್ರಾತೀರದ
ತುಂಗಭದ್ರಾತೀರದಲ್ಲಿ
ತುಂಗಭದ್ರಾತೀರಲ್ಲಿದ್ದಾಗ
ತುಂಗಭದ್ರಾತೀರ್ಥದಲ್ಲಿ
ತುಂಗಭದ್ರಾನದಿಗೆ
ತುಂಗಭದ್ರೆಯ
ತುಂಗಭದ್ರೆಯನ್ನು
ತುಂಡನುಂನತಂ
ತುಂಡು
ತುಂಬದೇವನಹಳ್ಳಿಯ
ತುಂಬಲ
ತುಂಬಿದನು
ತುಂಬಿರಬಹುದು
ತುಂಬಿಹರಿಯುತ್ತಿದ್ದ
ತುಕಡಿಯನ್ನು
ತುಕಡಿಯಲ್ಲಿ
ತುಗಲಕ್
ತುತ್ತಾಗಿ
ತುಪದ
ತುಮಕೂರಿನ
ತುಮಕೂರು
ತುರಂಗಮಂ
ತುರಂಗಮಂಗಳಂ
ತುರಗ
ತುರಗಕಳನಿರಿದು
ತುರಗಕಳವನಿರಿದು
ತುರಗಗಳನ್ನು
ತುರಗಗಳನ್ನೆಲ್ಲಾ
ತುರಲೋಭತು
ತುರುಕರ
ತುರುಕರು
ತುರುಕಳಗನಿರಿದು
ತುರುಕಳವನಿರಿದು
ತುರುಗಮಂಗಳಂ
ತುರುಗಳ
ತುರುಗಳನ್ನು
ತುರುಗೋಳಿನ
ತುರುಗೋಳಿನಲ್ಲಿ
ತುರುಗೋಳು
ತುರುಗೋಳುಗಳ
ತುರುಪರಿವಿನಲ್ಲಿತುರುಗೊಳ್
ತುರುವನಿಕ್ಕಿಸಿ
ತುರುವನ್ನು
ತುರುವೆಕೆರೆ
ತುರುಷ್ಕತುರಗಾರೂಢ
ತುರುಷ್ಕಮುಸ್ಲಿಂ
ತುರುಷ್ಕರ
ತುರುಷ್ಕರಾಜ
ತುರ್ಕಿಯ
ತುಲಗಣ್ಡ
ತುಲಾಪುರಷಾದಿ
ತುಲಾಪುರುಷಾದಿ
ತುಲುವೇಂದ್ರನಾದ
ತುಳಿದಂ
ತುಳುವ
ತುಳುವನರಸಿಂಹ
ತುಳುವಲ
ತುಳುವಲದೇವಿ
ತುಳುವಲೇಶ್ವರ
ತುಳುವವಂಶದ
ತುವ್ವಲೇಶ್ವರ
ತುಷ್ಟಾಶೇಷದ್ವಿಜನ್ಮನಃ
ತೂಬನ್ನಿಡಿಸಿ
ತೂಬನ್ನು
ತೂಬಿನಕೆರೆ
ತೂರ್ಯ
ತೃಪ್ತಿ
ತೆಂಕಣ
ತೆಂಕಣಭಾಗ
ತೆಂಕಣರಾಯ
ತೆಂಕಭಾಗದಲ್ಲಿದ್ದ
ತೆಂಕಲಂಕದ
ತೆಂಕಲು
ತೆಂಕಳಣ
ತೆಂಗಿನಕಟ್ಟ
ತೆಂಗಿನಕಟ್ಟಇಂದಿನ
ತೆಂಗಿನಕಟ್ಟದ
ತೆಂಗಿನಕಟ್ಟದಲ್ಲಿ
ತೆಂಗಿನಕಟ್ಟವನು
ತೆಂಗಿನಘಟ್ಟ
ತೆಂಗಿನಘಟ್ಟವನ್ನು
ತೆಂಗು
ತೆಂಪಾಗೈ
ತೆಗಡರಹಳ್ಳಿ
ತೆಗೆದಿಡುತ್ತಾರೆ
ತೆಗೆದಿರಿಸುವುದು
ತೆಗೆದು
ತೆಗೆದುಕೊಂಡನು
ತೆಗೆದುಕೊಂಡರೆಂದು
ತೆಗೆದುಕೊಂಡಿದ್ದನು
ತೆಗೆದುಕೊಂಡಿದ್ದನೆಂದೂ
ತೆಗೆದುಕೊಂಡು
ತೆಗೆದುಕೊಳ್ಳುತ್ತಿದ್ದರೆಂಬುದು
ತೆತ್ತಿಗನೆನೆವುದು
ತೆನದಂಕಾನ್ವಯ
ತೆನದಂಕಾನ್ವಯದ
ತೆಪ್ಪಕೊಳದ
ತೆಪ್ಪಕೊಳವನ್ನು
ತೆಪ್ಪಣ್ಣತೇಪಣ್ಣದೇವಣ್ಣ
ತೆಪ್ಪತಿರುನಾಳು
ತೆಪ್ಪದ
ತೆಪ್ಪದನಾಗಣ್ಣನು
ತೆರಕಣಾಂಬಿ
ತೆರಕಣಾಂಬಿಯ
ತೆರಕಣಾಂಬಿಯಲ್ಲಿ
ತೆರಕಣಾಂಬಿಯಲ್ಲಿದ್ದ
ತೆರಕಣಾಂಬಿಯಿಂದ
ತೆರಕಣಾಂಬಿಸೀಮೆಯ
ತೆರಕಣಾಂಬೆ
ತೆರಕಣಾಂಬೆಯ
ತೆರಕಣಾಂಬೆಯನ್ನು
ತೆರಣೆನಹಳ್ಳಿ
ತೆರದಿಂದವೆ
ತೆರನಾಗಿದ್ದರೂ
ತೆರನಾದ
ತೆರಲು
ತೆರಳಿ
ತೆರಾಯಾಂಬಾಮುಲ
ತೆರಿಗೆ
ತೆರಿಗೆಗಳ
ತೆರಿಗೆಗಳನ್ನು
ತೆರಿಗೆಗಳನ್ನುದತ್ತಿಗಳನ್ನು
ತೆರಿಗೆಗಳಲ್ಲಿ
ತೆರಿಗೆಯ
ತೆರಿಗೆಯನ್ನು
ತೆರೆಯನ್ನು
ತೆಲುಂಗನ
ತೆಲುಂಗರಾಯಸ್ಥಾಪನಾಚಾರ್ಯ
ತೆಲುಗು
ತೆಲುಗುಚೋಡ
ತೆಲುಗುಚೋಡರ
ತೆಲುಗುಚೋಳರನ್ನು
ತೆಲುಗುಭಾಷೆ
ತೆಲುಗುಭಾಷೆಯ
ತೆಲುಗುಮೂಲದ
ತೆಲುಗುಮೂಲವೆಂದು
ತೆಲುಗುಸೀಮೆಯವರು
ತೆಲ್ಲಿಗ
ತೆಳರಕುಲತಿಲಕ
ತೆಳರಕುಲದ
ತೆಳ್ಳರ
ತೆಳ್ಳರಕುಲದ
ತೇಕಲ್ಲು
ತೇಗಿನಹಳ್ಳಿ
ತೇಜೋಮೂರ್ತಿ
ತೇದಿ
ತೇದಿಗಳನ್ನು
ತೇದಿಗಳಿವೆ
ತೇದಿಗಳು
ತೇದಿಯ
ತೇದಿಯನ್ನು
ತೇದಿಯಲ್ಲಿ
ತೇದಿಯಿಲ್ಲದ
ತೇದಿಯು
ತೇದಿಯುಕ್ತ
ತೇದಿಯುಳ್ಳ
ತೇದಿರಹಿತ
ತೇದಿರಹಿತವಾಗಿದೆ
ತೇದಿರಹಿತವಾದ
ತೇರಣ್ಯ
ತೈರೂರ
ತೈಲನ
ತೈಲನು
ತೈಲಪನು
ತೈಲೂರು
ತೊಂಡನೂರ
ತೊಂಡನೂರಿಗೂ
ತೊಂಡನೂರಿನ
ತೊಂಡನೂರಿನಲ್ಲಿ
ತೊಂಡನೂರಿನಲ್ಲಿದ್ದಾಗ
ತೊಂಡನೂರು
ತೊಂಡಮಂಡಲದ
ತೊಂಡೇಹಳ್ಳಿ
ತೊಂಡೈಮಂಡಲಮ್
ತೊಂದರೆ
ತೊಂದರೆಯಾದಾಗ
ತೊಂಬತ್ತರುಸಾವಿರ
ತೊಂಬತ್ತರುಸಾಸಿರ
ತೊಂಬತ್ತರುಸಾಸಿರದ
ತೊಂಬತ್ತರುಸಾಸಿರಮಂ
ತೊಂಬತ್ತರುಸಾಸಿರಮನು
ತೊಂಬತ್ತರುಸಾಸಿರಮನೇಕ
ತೊಂಬತ್ತಾರು
ತೊಂಬತ್ತಾರುಸಾವಿರ
ತೊಂಬತ್ತಾರುಸಾವಿರದ
ತೊಂಬತ್ತಾರುಸಾವಿರನ್ನು
ತೊಂಬತ್ತಾರುಸಾವಿರವನ್ನು
ತೊಗರವಾಡಿ
ತೊಟ್ಟಿಲು
ತೊಡಗಿಕೊಂಡಿದ್ದನು
ತೊಡಗಿದ್ದನು
ತೊಡಗಿದ್ದುದನ್ನು
ತೊಡಗಿಸಲು
ತೊಡಗಿಸಿಕೊಂಡರು
ತೊಡಗುತ್ತಿದ್ದ
ತೊಡರ್ದರಡೊಂಕಿಯುಂ
ತೊಡರ್ದ್ದರಂಕುಸ
ತೊಣಚಿ
ತೊಣ್ಣೂರಿನ
ತೊಣ್ಣೂರಿನಲ್ಲಿ
ತೊಣ್ಣೂರಿನಲ್ಲಿರುವ
ತೊಣ್ಣೂರಿನಿಂದ
ತೊಣ್ಣೂರು
ತೊಣ್ಣೂರುಗಳಲ್ಲಿ
ತೊರೆ
ತೊರೆಕಾಡನಹಳ್ಳಿ
ತೊರೆಗಳಾಗಿವೆ
ತೊರೆಗಳು
ತೊರೆದು
ತೊರೆನಾಡು
ತೊರೆಬೊಮ್ಮನಹಳ್ಳಿ
ತೊರೆಮಗ್ಗ
ತೊರೆಯ
ತೊಱಗಲೆಯ
ತೊಲಗಂಡ
ತೊಲಗದ
ತೊಳಂಚೆಯ
ತೊಳಂಚೆಯಲ್ಲಿ
ತೊಳಲ್ದು
ತೊಳಸಿ
ತೊಳಸಿಯ
ತೊಳೆದು
ತೋಟ
ತೋಟಗಳನ್ನು
ತೋಟದ
ತೋಟದಲ್ಲಿದ್ದು
ತೋಟನವನ್ನು
ತೋಟವನ್ನು
ತೋಟವೃತ್ತಿ
ತೋಟಸ್ಥಳಗಳನು
ತೋಟಿ
ತೋಟಿಗರು
ತೋಡಿಸಿದ್ದನೆಂದೂ
ತೋಡಿಸುತ್ತಾರೆ
ತೋರಣವನ್ನು
ತೋರಿನಾಡ
ತೋರಿನಾಡು
ತೋರಿಸಲೆಂದು
ತೋರಿಸಿದರೆ
ತೋರಿಸಿದ್ದಾರೆ
ತೋರಿಸುತ್ತದೆ
ತೋರಿಸುತ್ತವೇನೋ
ತೋರಿಸುತ್ತಿದೆ
ತೋರಿಸುವುದೇ
ತೋರುತ್ತದೆ
ತೋರುತ್ತದೆಎಂದು
ತೋರುತ್ತಿರುವಂತಿದ್ದರೆ
ತೋಳ
ತೋಳಬಿಂಕಮಂ
ತೌಳಿಯಮ್ಮ
ತ್ತುತ್ತಿರೆ
ತ್ತೊಮ್ಭತ್ತಱು
ತ್ಯಂತವಾಗಿ
ತ್ಯಾಗದ
ತ್ಯಾಗದಕೊಡುಗೆಯಾಗಿ
ತ್ಯಾಗನಹಳ್ಳಿ
ತ್ಯಾಗವಾಗಿ
ತ್ರಿಕೂಟ
ತ್ರಿಕೂಟಬ
ತ್ರಿಕೂಟರತ್ನತ್ರಯ
ತ್ರಿಣೇತ್ರ
ತ್ರಿಣೇತ್ರನಂ
ತ್ರಿಣೇತ್ರನೆಂದು
ತ್ರಿಭುವಚಕ್ರವರ್ತಿ
ತ್ರಿಭುವನ
ತ್ರಿಭುವನಕಠಾರಿರಾಯನೂ
ತ್ರಿಭುವನಚಕ್ರವರ್ತಿ
ತ್ರಿಭುವನತೀರ್ಥದ
ತ್ರಿಭುವನಮಲ್ಲ
ತ್ರಿಭುವನೀರಾಯ
ತ್ರಿಮೂರ್ತಿಗಳ
ತ್ರಿಯಂಬಕೇಶ್ವರ
ತ್ರುಟಿತ
ತ್ರುಟಿತಭಾಗ
ತ್ರುಟಿತವಾಗಿತ್ತೆಂದು
ತ್ರುಟಿತವಾಗಿದೆ
ತ್ರುಟಿತವಾಗಿದ್ದು
ತ್ರುಟಿತವಾಗಿರುವ
ತ್ರುಟಿತವಾಗಿರುವುದರಿಂದ
ತ್ರುಟಿತವಾದ
ತ್ರುಟಿದ
ತ್ರೈಲೋಕ್ಯರಂಜನ
ತ್ರೈವಿದ್ಯದೇವರ
ತ್ವಂ
ಥಾಣ
ದ
ದಂಗೆ
ದಂಗೆಯ
ದಂಗೆಯನ್ನು
ದಂಡ
ದಂಡಂಗಳು
ದಂಡಗಿ
ದಂಡಡನಾಯಕನ
ದಂಡದಧಿಷ್ಠಾಯಕ
ದಂಡದಧಿಷ್ಠಾಯಕರು
ದಂಡನಾತಾಂಬರಾರ್ಕ್ಕಂ
ದಂಡನಾಥ
ದಂಡನಾಥನ
ದಂಡನಾಥನನ್ನು
ದಂಡನಾಥನಾಗಿ
ದಂಡನಾಥನು
ದಂಡನಾಥಾಧಿಪ
ದಂಡನಾಥೋ
ದಂಡನಾಯಕ
ದಂಡನಾಯಕಂಗೆ
ದಂಡನಾಯಕತ್ವದಲ್ಲಿ
ದಂಡನಾಯಕನ
ದಂಡನಾಯಕನದ್ದೇ
ದಂಡನಾಯಕನನೂ
ದಂಡನಾಯಕನನ್ನಾಗಿ
ದಂಡನಾಯಕನನ್ನು
ದಂಡನಾಯಕನಾಗಿ
ದಂಡನಾಯಕನಾಗಿದ್ದ
ದಂಡನಾಯಕನಾಗಿದ್ದಂತೆ
ದಂಡನಾಯಕನಾಗಿದ್ದನು
ದಂಡನಾಯಕನಾಗಿದ್ದನೆಂದು
ದಂಡನಾಯಕನಾಗಿದ್ದನೆಂಬುದು
ದಂಡನಾಯಕನಾಗಿದ್ದು
ದಂಡನಾಯಕನಾಗಿರಬಹುದಾದ
ದಂಡನಾಯಕನಾಗಿರಲು
ದಂಡನಾಯಕನಾಗಿರುತ್ತಿದ್ದನು
ದಂಡನಾಯಕನಾದ
ದಂಡನಾಯಕನಿಕ್ಕಯಣ್ಣನು
ದಂಡನಾಯಕನಿಗಿಂತ
ದಂಡನಾಯಕನಿಗಿದ್ದ
ದಂಡನಾಯಕನಿಗೂ
ದಂಡನಾಯಕನಿಗೆ
ದಂಡನಾಯಕನು
ದಂಡನಾಯಕನೂ
ದಂಡನಾಯಕನೂಸೋಮದಂಡನಾಯಕ
ದಂಡನಾಯಕನೆಂದು
ದಂಡನಾಯಕನೆಂಬ
ದಂಡನಾಯಕನೆನಿಸಿದನು
ದಂಡನಾಯಕನೇ
ದಂಡನಾಯಕರ
ದಂಡನಾಯಕರನ್ನು
ದಂಡನಾಯಕರಲ್ಲಿ
ದಂಡನಾಯಕರಾಗಲೀ
ದಂಡನಾಯಕರಾಗಿ
ದಂಡನಾಯಕರಾಗಿದ್ದ
ದಂಡನಾಯಕರಾಗಿದ್ದರು
ದಂಡನಾಯಕರಾಗಿದ್ದರೆಂದು
ದಂಡನಾಯಕರಾಗಿದ್ದವರು
ದಂಡನಾಯಕರಾಗಿದ್ದವರೇ
ದಂಡನಾಯಕರಾದ
ದಂಡನಾಯಕರಿಂದ
ದಂಡನಾಯಕರಿಗಿಂತ
ದಂಡನಾಯಕರಿಗೆ
ದಂಡನಾಯಕರು
ದಂಡನಾಯಕರುಗಳ
ದಂಡನಾಯಕರುಗಳನ್ನು
ದಂಡನಾಯಕರುಗಳಾಗಿ
ದಂಡನಾಯಕರುಗಳಾಗಿದ್ದ
ದಂಡನಾಯಕರುಗಳಿಗಿಂತ
ದಂಡನಾಯಕರುಗಳು
ದಂಡನಾಯಕರುಗಳೂ
ದಂಡನಾಯಕರುದಂಡಾಧೀಶರು
ದಂಡನಾಯಕರುಮಂತ್ರಿಗಳು
ದಂಡನಾಯಕರುಮಹಾಪ್ರಧಾನ
ದಂಡನಾಯಕರೂ
ದಂಡನಾಯಕರೆಂದು
ದಂಡನಾಯಕರೆಂಬ
ದಂಡನಾಯಕರೇ
ದಂಡನಾಯಕವೀರಯ್ಯ
ದಂಡನಾಯಕಸು
ದಂಡನಾಯಕಸುರಿಗೆ
ದಂಡನಾಯಕಿತಿ
ದಂಡನಾಯಕಿತ್ತಿ
ದಂಡನಾಯಕಿತ್ತಿಗೆ
ದಂಡನಾಯಕಿತ್ತಿಯ
ದಂಡನಾಯಕಿತ್ತಿಯರ
ದಂಡನಾಯಕಿತ್ತಿಯು
ದಂಡನಾಯನ
ದಂಡನಾಯುಕನ
ದಂಡನ್ನು
ದಂಡಯಾತ್ರೆ
ದಂಡಯಾತ್ರೆಗಳ
ದಂಡಯಾತ್ರೆಗಳಲ್ಲಿ
ದಂಡಯಾತ್ರೆಗಳಿಂದ
ದಂಡಯಾತ್ರೆಗಳಿಗೂ
ದಂಡಯಾತ್ರೆಯ
ದಂಡಯಾತ್ರೆಯನ್ನು
ದಂಡಯಾತ್ರೆಯಲ್ಲಿ
ದಂಡವನ್ನು
ದಂಡಾಧಿಪನದ್ದಲ್ಲವೆಂದು
ದಂಡಾಧಿಪರೊಳತಿಶಯಂ
ದಂಡಾಧೀಶ
ದಂಡಾಧೀಶದಾವಾನಲನೂ
ದಂಡಾಧೀಶನ
ದಂಡಾಧೀಶನಾಗಿದ್ದು
ದಂಡಾಧೀಶನು
ದಂಡಾಧೀಶನೆಂದು
ದಂಡಾಧೀಶರ
ದಂಡಿಗೆ
ದಂಡಿಗೆತ್ತಿ
ದಂಡಿನಹಳ್ಳಿ
ದಂಡಿನೊಡನೆ
ದಂಡು
ದಂಡುಅಹೋಬಲದೇವನ
ದಂಡೆಗಳಿಗೂ
ದಂಡೆತ್ತಿ
ದಂಡೆತ್ತಿಬಂದನು
ದಂಡೆತ್ತಿಬಂದು
ದಂಡೆತ್ತಿಹೋಗಿ
ದಂಡೆತ್ತಿಹೋದ
ದಂಡೆತ್ತಿಹೋದರು
ದಂಡೆಯ
ದಂಡೆಯಗುಂಟ
ದಂಡೆಯಲ್ಲಿ
ದಂಡೆಯಲ್ಲಿರುವ
ದಂಡೇಶ
ದಂಡೇಶನು
ದಂಡೇಶನೇ
ದಂಣಾಯಕ
ದಂಣಾಯಕರ
ದಂಣಾಯಕರಿಗೆ
ದಂಣಾಯಕರು
ದಂಣ್ನಾಯಕ
ದಂಣ್ನಾಯಕನನ್ನು
ದಂಣ್ನಾಯಕನೂ
ದಂಣ್ನಾಯಕರ
ದಂಣ್ನಾಯಕರು
ದಂಪತಿಗಳಿಗೆ
ದಕ್ಷ
ದಕ್ಷತೆಗಳಿಂದ
ದಕ್ಷಿಣ
ದಕ್ಷಿಣಕ್ಕಿರುವ
ದಕ್ಷಿಣಕ್ಕೂ
ದಕ್ಷಿಣಕ್ಕೆ
ದಕ್ಷಿಣಚಕ್ರವರ್ತಿ
ದಕ್ಷಿಣದ
ದಕ್ಷಿಣದಕಡೆಗೆ
ದಕ್ಷಿಣದಲ್ಲಿ
ದಕ್ಷಿಣಭಾಗದಲ್ಲಿದ್ದ
ದಕ್ಷಿಣಭಾಗದಲ್ಲಿಯೂ
ದಕ್ಷಿಣಭಾಗದಲ್ಲಿರುವ
ದಕ್ಷಿಣಭಾರತದಲ್ಲೆಲ್ಲಾ
ದಕ್ಷಿಣಭಾರತವನ್ನು
ದಕ್ಷಿಣಭುಜಾದಂಡನೆನಿಸಿದ್ದ
ದಕ್ಷಿಣಾಪಥದ
ದಕ್ಷಿಣಾಮೂರ್ತಿ
ದಗಂಡಪೆಂಡಾರ
ದಟ್ಟವಾದ
ದಡಗ
ದಡಗದ
ದಡಗದಡಿಗನಕೆರೆ
ದಡಗಳಲ್ಲೂ
ದಡದಲ್ಲಿ
ದಡದಹಳ್ಳಿ
ದಡಿಗ
ದಡಿಗದೀಡಿಗನ
ದಡಿಗನಕೆರೆ
ದಡಿಗನಕೆರೆಗೆ
ದಡಿಗನಕೆರೆಯ
ದಡಿಗನಕೆರೆಯಇಂದಿನ
ದಡಿಗವಾಡಿ
ದಡಿಗೇಶ್ವರ
ದಡಿಘಟ್ಟ
ದಣ್ಡೆ
ದಣ್ಣಾಯಕನಪುರ
ದಣ್ಣಾಯಕನು
ದಣ್ನಾಯಕನೂ
ದಣ್ನಾಯಕಿತಿ
ದತ್ತಂ
ದತ್ತಕ
ದತ್ತಕಪಡೆದಳು
ದತ್ತಕಪಡೆದಳೆಂದು
ದತ್ತಯಾಗಿ
ದತ್ತಿ
ದತ್ತಿಕೊಟ್ಟಿರುವುದು
ದತ್ತಿಗಳ
ದತ್ತಿಗಳನ್ನು
ದತ್ತಿಗಳನ್ನೂ
ದತ್ತಿಗಳಾಗಿವೆ
ದತ್ತಿಗೆ
ದತ್ತಿನೀಡಲಾಗಿದೆ
ದತ್ತಿನೀಡುತ್ತಾನೆ
ದತ್ತಿಪಡೆದು
ದತ್ತಿಬಿಟ್ಟ
ದತ್ತಿಬಿಟ್ಟನು
ದತ್ತಿಬಿಟ್ಟನೆಂದು
ದತ್ತಿಬಿಟ್ಟನೆಂದೂ
ದತ್ತಿಬಿಟ್ಟರು
ದತ್ತಿಬಿಟ್ಟರೆಂದು
ದತ್ತಿಬಿಟ್ಟಾಗ
ದತ್ತಿಬಿಟ್ಟಿದ್ದಾರೆ
ದತ್ತಿಬಿಟ್ಟಿನೆಂದು
ದತ್ತಿಬಿಟ್ಟಿರುವ
ದತ್ತಿಬಿಡಲಾಗಿದೆ
ದತ್ತಿಬಿಡಲಾಯಿತು
ದತ್ತಿಬಿಡುತ್ತಾನೆ
ದತ್ತಿಬಿಡುತ್ತಾರೆ
ದತ್ತಿಬಿಡುತ್ತಾಳೆ
ದತ್ತಿಬಿಡುವ
ದತ್ತಿಬಿಡುವುದು
ದತ್ತಿಯ
ದತ್ತಿಯನ್ನು
ದತ್ತಿಯಾಗಿ
ದತ್ತಿಯಾಗಿಬಿಟ್ಟನು
ದತ್ತಿಯಾಗಿಬಿಟ್ಟನೆಂದು
ದತ್ತಿಯು
ದತ್ತಿಶಾಸನ
ದತ್ತಿಶಾಸನಗಳೆಂದು
ದತ್ತಿಹಾಕಿಕೊಟ್ಟನೆಂದು
ದತ್ತಿಹಾಕಿಕೊಟ್ಟಿರುತ್ತಾನೆ
ದತ್ತಿಹಾಕಿಕೊಡಲಾಗಿತ್ತು
ದತ್ತಿಹಾಕಿಕೊಡುತ್ತಾನೆ
ದತ್ತಿಹಾಕಿಕೊಡುತ್ತಾರೆ
ದತ್ತಿಹಾಕಿಕೊಡುತ್ತಾಳೆ
ದತ್ತು
ದತ್ತುಪುತ್ರನೇ
ದದತಾ
ದನಗೂರು
ದನುಗೂರು
ದಬಗ
ದಬಗಾವುಡ
ದಬ್ಬಾಳಿಕೆಯನ್ನು
ದಮ್ಮಿಸೆಟ್ಟಿಯರ
ದಯಂಣ
ದಯಂಣದೇವಣ್ಣ
ದಯಣ್ಣ
ದಯಪಾಲಿಸಿ
ದಯಪಾಲಿಸಿದ್ದ
ದಯಾಂಬುಧಿಸೋಮ
ದಯೆಗೆಯ್ಯೆನ್ದು
ದಯೆಯ
ದಯೆಯನ್ನು
ದರಗಿರಿತುಂಗನಿಂದುಕುಮುದೋಜ್ವಳಕೀರ್ತ್ತಿ
ದರಸಗುಪ್ಪೆಯಾಗಿರುವ
ದರಸಿಕುಪ್ಪೆ
ದರಿದ್ರರ
ದರಿಯಾದೌಲತ್ನಲ್ಲಿ
ದರ್ಗಾಕ್ಕೆ
ದರ್ಗಾದಲ್ಲಿ
ದರ್ಪ್ಪದಳನ
ದರ್ವೇಷನಿಗೆ
ದರ್ಶನ
ದರ್ಶನದ
ದರ್ಶನಾರ್ಥವಾಗಿ
ದಳದಳವಾಗಿ
ದಳಪತಿ
ದಳಪತಿಗಳ
ದಳಪತಿಗಳಲ್ಲಿ
ದಳಪತಿಗಳಲ್ಲೊಬ್ಬನು
ದಳಪತಿಗಳಾದ
ದಳಪತಿಗಳು
ದಳಭಾರಸಹಿತ
ದಳವಾಯಿ
ದಳವಾಯಿಗಳ
ದಳವಾಯಿಗಳಿಂದ
ದಳವಾಯಿಗಳು
ದಳವಿಟ್ಟಿರಿದಲ್ಲಿ
ದಳೇ
ದವಳವಾಯಿ
ದವೆ
ದಶಮೀ
ದಶವಂದವನ್ನು
ದಶೇಕಾದಶವರ್ಷದಲ್ಲಿ
ದಾಖಲಿಸಲಾಗಿದೆ
ದಾಖಲಿಸಿದೆ
ದಾಖಲಿಸಿವೆ
ದಾಖಲುಪತ್ರ
ದಾಖಲೆ
ದಾಖಲೆಗಳ
ದಾಖಲೆಗಳನ್ನು
ದಾಖಲೆಗಳಲ್ಲಿ
ದಾಖಲೆಯಾಗಿದೆ
ದಾಖಲೆಯಾಗಿದ್ದು
ದಾಟಿ
ದಾಡಿಯ
ದಾಡಿಯಗಡ್ಡದ
ದಾಡಿಯಸೋಮೆಯ
ದಾದಾಜಿ
ದಾದಿ
ದಾದೋಜಿ
ದಾನ
ದಾನಕ್ಕೆ
ದಾನಗಳನ್ನು
ದಾನಗಳನ್ನೂ
ದಾನಗುಣ
ದಾನದ
ದಾನದಂನಪುರದ
ದಾನದತ್ತಿಗಳನ್ನು
ದಾನದುನ್ನತಿಯಿಂದ
ದಾನದೊಳು
ದಾನಧರ್ಮಗಳನ್ನು
ದಾನಧರ್ಮ್ಮದ
ದಾನಬಿಟ್ಟಿದ್ದಾನೆ
ದಾನಮದ್ಭುತಂ
ದಾನವನ್ನು
ದಾನವಾಗಿ
ದಾನಶಾಸಣ
ದಾನಶ್ರೇಯಾಂಸಂ
ದಾನಸ್ಯ
ದಾಮ
ದಾಮಣ್ಣ
ದಾಮಣ್ಣನು
ದಾಮಣ್ಣನೆಂದು
ದಾಮನೂ
ದಾಮನೆಂಬ
ದಾಮನೆಯ್ದನೆ
ದಾಮಪಯ್ಯನನ್ನು
ದಾಮಪಯ್ಯನೆಂಬುವನನ್ನು
ದಾಮರಲೈಯಪೇಂದ್ರ
ದಾಮಾದ್
ದಾಮೋದರ
ದಾಮೋದರನ
ದಾಮೋದರನು
ದಾಯಾದ
ದಾಯಾದಿ
ದಾಯಾದಿಗಳು
ದಾಯಾದಿಯಾಗಿರಬಹುದು
ದಾಯಾದ್ಯಕ್ಕೆ
ದಾಯಿಗಬೇಂಟೆಕಾರ
ದಾಯ್ಗರು
ದಾರಿ
ದಾರಿಯಲ್ಲಿಯೇ
ದಾರುಸ್ಸಲ್ತನತ್
ದಾಳಿ
ದಾಳಿಂಅ
ದಾಳಿಗಳಲ್ಲಿ
ದಾಳಿಮಾಡಿ
ದಾಳಿಯ
ದಾಳಿಯನ್ನು
ದಾಳಿಯಲ್ಲಿ
ದಾವಹವಿ
ದಾವಾನಳ
ದಾಶರಾಜನಲ್ಲಿಗೆ
ದಾಸನದೊಡ್ಡಿ
ದಾಸನಪುರ
ದಾಸಪನಾಯಕರ
ದಾಸಾನುದಾಸನೆಂದು
ದಾಸೋಹಕ್ಕೆ
ದಾಸ್ತಾನು
ದಿಂಡಕ
ದಿಂಡಿಕ
ದಿಂಡಿಕಗ
ದಿಂಡಿಗ
ದಿಂಡಿಗನ
ದಿಂಡಿಗನಕೆರೆಯ
ದಿಂಡಿಗನಾಡಿಯರು
ದಿಂಡಿಗನು
ದಿಂಡಿಗಮಹಾಪ್ರಭುವೇ
ದಿಂಡಿಗರಾಜ
ದಿಂಡಿಗರಾಜನು
ದಿಂಡಿಗರು
ದಿಂಡಿಗಲ್ಲಿನ
ದಿಂಮರಾಜಯ್ಯನು
ದಿಕ್ಕಿಗೆ
ದಿಕ್ಕಿನಲ್ಲಿ
ದಿಕ್ಕಿನಿಂದ
ದಿಕ್ಕೆಟ್ಟನು
ದಿಗ್ವಿಜಯ
ದಿಗ್ವಿಜಯಗಳನ್ನು
ದಿಗ್ವಿಜಯದ
ದಿಗ್ವಿಜಯಾರ್ಥವಾಗಿ
ದಿಡಗ
ದಿಡಗದ
ದಿಡಗವಾಗಿದೆ
ದಿಡಗವು
ದಿಡುಗವನ್ನು
ದಿಣ್ಡಿಗ
ದಿಣ್ಡಿಗಕೂಡಲೂರು
ದಿಣ್ಡಿಗೋ
ದಿಣ್ಣೆಯ
ದಿನ
ದಿನಕ್ಕೆ
ದಿನಗಳ
ದಿನದಲಿ
ದಿನದಿನದ
ದಿನಸಿಗಳ
ದಿನಾ
ದಿನಾಂಕ
ದಿನಾಂಕದ
ದಿನ್
ದಿವಂಗತನಾದಾಗ
ದಿವಾಕರನೆನಿಸಿದ
ದಿವಾನರ
ದಿವಾನ್
ದಿವಿಜಲಲನೆಯರು
ದಿವ್ಯ
ದಿವ್ಯದೇಶವಾದ
ದಿವ್ಯಮುನಿವರನೆಂದು
ದಿವ್ಯವಾಹನ
ದಿವ್ಯವ್ರತ
ದಿವ್ಯಶ್ರೀ
ದಿವ್ಯಶ್ರೀಪಾದಪದ್ಮದ
ದಿಶಾಪಟ್ಟನುಂ
ದಿಸೆಯಲ್ಲಿ
ದೀಕ್ಷಿತ
ದೀಕ್ಷಿತ್
ದೀಕ್ಷಿತ್ರವರು
ದೀಕ್ಷೆ
ದೀಪಮಾಲೆ
ದೀಪಮಾಲೆಕಂಬ
ದೀಪಮಾಲೆಯ
ದೀಪಾಂಕುರನಂ
ದೀಪಿತ
ದೀರ್ಘ
ದೀರ್ಘಕಾಲ
ದೀರ್ಘಕಾಲದಿಂದ
ದೀರ್ಘವಾಗಿ
ದೀರ್ಘವಾದ
ದೀವಿಗೆಗೆ
ದುಂಡು
ದುಂಡುವಿನ
ದುಂಡುವು
ದುಇಪಹರರಾಉತು
ದುಗ್ಗಮಾರನ
ದುಗ್ಗಮಾರನು
ದುಗ್ಗಯ್ಯಂ
ದುಗ್ಗಲೆ
ದುಗ್ಗವೆ
ದುಗ್ಗವ್ವೆಯರಿಗೆ
ದುದ್ದ
ದುದ್ದದ
ದುದ್ದಮಲ್ಲದೇವನ
ದುಬಿಗಾವುಂಡಿ
ದುಮ್ಮೆತನಕ
ದುಮ್ಮೆಯ
ದುಮ್ಮೆಯನಾಯಕ
ದುಮ್ಮೆಯನಾಯಕನ
ದುಮ್ಮೆಯನಾಯಕನು
ದುಮ್ಮೆಯನಾಯಕರು
ದುಮ್ಮೆವರೆಗೆ
ದುರಂಧರನೆಂದು
ದುರಂಧರೋ
ದುರಸ್ತೆ
ದುರಾದೃಷ್ಟವಶಾತ್
ದುರಿತದೂರಂ
ದುರ್ಗಗಳನ್ನು
ದುರ್ಗದ
ದುರ್ಗದೊಳಗೆ
ದುರ್ಗಮನುರವಣೆಯಿಂ
ದುರ್ಗವನಾಳುವಲ್ಲಿ
ದುರ್ಗವನ್ನು
ದುರ್ಗಾಧಿಪತಿ
ದುರ್ಗಾಧಿಪತಿಗಳು
ದುರ್ಜನ
ದುರ್ಬಲ
ದುರ್ಬಲವಾಗಿದ್ದ
ದುರ್ಬಲವಾಯಿತು
ದುರ್ಯೋಧನನು
ದುಷ್ಟಜನದುರ್ಲಭ
ದುಷ್ಟನಿರ್ಮೂಲನೆಗಾಗಿ
ದುಷ್ಟಶಾರ್ದೂಲಮರ್ದನಃ
ದುಸ್ಸಾಧ್ಯವಾದ
ದೂರದ
ದೂರದಲ್ಲಿ
ದೂರದಲ್ಲಿರುವ
ದೃಢಂ
ದೃಢಪಡಿಸಿ
ದೃಢಪಡಿಸುತ್ತದೆ
ದೃಢಪಡಿಸುತ್ತವೆ
ದೃಷ್ಟಿಕೋನಗಳಿಂದ
ದೃಷ್ಟಿಯಿಂದ
ದೆತ್ತಿದ
ದೆಲೆಗೌಡ
ದೆವರಾಜೊಡೆಯರ
ದೆಸೆ
ದೇಕವೆದಂಡನಾಯಕಿತಿಯ
ದೇಕವ್ವೆ
ದೇಕೆಯನಾಯಕ
ದೇಕೆಯನಾಯಕನ
ದೇಕೆಯನಾಯಕನು
ದೇಕೆಯನಾಯಕರ
ದೇಕೆಯನಾಯಕರು
ದೇಗುಲಗೌಣ್ಡಿ
ದೇಗುಲಗೌಣ್ಡಿಯನ್ನು
ದೇಪಂಣೊಡೆಯರ
ದೇಪಣ್ಣ
ದೇಪಯನಾಮಧೇಯೋ
ದೇಪಯಸ್ತು
ದೇಪಯ್ಯ
ದೇಪಯ್ಯನನ್ನು
ದೇಪಯ್ಯನು
ದೇಮಲದೇವಿದೇವಲದೇವಿ
ದೇಮಲದೇವಿಯು
ದೇಮಲಾದೇವಿಯ
ದೇಮಲಾದೇವಿಯನ್ನು
ದೇಮಲಾಪುರವೆಂಬ
ದೇಮಸಮುದ್ರ
ದೇಮಾಂಬಿಕೆಯರ
ದೇಮಿಕಬ್ಬೆ
ದೇಮಿಕಬ್ಬೆಯರು
ದೇಯಾತು
ದೇವ
ದೇವಕಿ
ದೇವಕೀರ್ತಿಪಂಡಿತರ
ದೇವಕುಮಾರ
ದೇವಕ್ಷೇತ್ರವಾದ
ದೇವಚಂದ್ರನ
ದೇವಚಂದ್ರನು
ದೇವಣ
ದೇವಣ್ಣನು
ದೇವತಾ
ದೇವತಾಗೃಹಮಲ್ಲಿಕಾರ್ಜುನ
ದೇವತಾಗ್ರಾಮಂ
ದೇವತಾಪೂಜೆಗೆ
ದೇವತಾಮಂದಿರ
ದೇವತೆಗಳನ್ನು
ದೇವತೆಯ
ದೇವತೆಯಾಗಿರಬಹುದು
ದೇವದಂಡನಾಯಕನಿಗೆ
ದೇವದಾನವನ್ನು
ದೇವದ್ವಿಜಬಂಧುಮಿತ್ರವರ್ಗ್ಗಾಣಾಂ
ದೇವನ
ದೇವನಂ
ದೇವನಣುಗಿನರ್ಕ್ಕರಿನ
ದೇವನಿಂ
ದೇವನಿಗೆ
ದೇವನು
ದೇವನೂ
ದೇವನೂರನ್ನು
ದೇವಪರ್ವ
ದೇವಪುರಿ
ದೇವಪೂಜೆ
ದೇವಪ್ಪ
ದೇವಪ್ಪನಾಯಕ
ದೇವಪ್ಪನಾಯಕನ
ದೇವಪ್ಪನಾಯಕನು
ದೇವಪ್ಪನಾಯಕರು
ದೇವಪ್ಪನು
ದೇವಬ್ರಾಹ್ಮಣ
ದೇವಭಟ್ಟರಿಗೆ
ದೇವಭುವನೇ
ದೇವಮಾಂಬ
ದೇವಮಾನ್ಯವನ್ನು
ದೇವಯ್ಯನು
ದೇವರ
ದೇವರಕೊಂಡಾ
ದೇವರಕೊಂಡಾರೆಡ್ಡಿಯವರ
ದೇವರಕೊಂಡಾರೆಡ್ಡಿಯವರು
ದೇವರದರ್ಶನ
ದೇವರನ್ನು
ದೇವರಸ
ದೇವರಸಗವುಡ
ದೇವರಸನ
ದೇವರಸನನ್ನು
ದೇವರಸನು
ದೇವರಸರ
ದೇವರಸರಿಗೆ
ದೇವರಸರು
ದೇವರಹಳ್ಳಿ
ದೇವರಹಳ್ಳಿಗಳು
ದೇವರಹಳ್ಳಿಯ
ದೇವರಹಳ್ಳಿಯನ್ನು
ದೇವರಾಜ
ದೇವರಾಜದೇವರಾಜ
ದೇವರಾಜನ
ದೇವರಾಜನನ್ನು
ದೇವರಾಜನಿಗೆ
ದೇವರಾಜನು
ದೇವರಾಜಪುರ
ದೇವರಾಜಪುರವೆಂಬ
ದೇವರಾಜಭೂಪಾಲನು
ದೇವರಾಜಮಹೀಪಾಲಕರು
ದೇವರಾಜಮಹೀಪಾಲರು
ದೇವರಾಜಯ್ಯ
ದೇವರಾಜಯ್ಯದೇವನ
ದೇವರಾಜಯ್ಯನ
ದೇವರಾಜರ
ದೇವರಾಜರು
ದೇವರಾಜವೊಡೆಯನು
ದೇವರಾಜೇಂದ್ರ
ದೇವರಾಜೇಂದ್ರನಿಗೆ
ದೇವರಾಜೊಡೆಯನೂ
ದೇವರಾಜೊಡೆಯರ
ದೇವರಾಜೊಡೆಯರು
ದೇವರಾಜ್ಯಂ
ದೇವರಾಯ
ದೇವರಾಯನ
ದೇವರಾಯನನ್ನು
ದೇವರಾಯನಿಂದ
ದೇವರಾಯನಿಗೆ
ದೇವರಾಯನು
ದೇವರಾಯಪಟ್ಟಣ
ದೇವರಾಯಪ್ರೌಢದೇವರಾಯ
ದೇವರಾಯಮಹಾರಾಯರ
ದೇವರಿಗೆ
ದೇವರು
ದೇವರುಗಳ
ದೇವರುಗಳನ್ನು
ದೇವರುಗಳಿಗೆ
ದೇವರೆಂದೇ
ದೇವರೆಂಬ
ದೇವರೇ
ದೇವರ್
ದೇವರ್ವಲ್ಲವನ್
ದೇವಲ
ದೇವಲಪುರವಾಗಿ
ದೇವಲಮಹಾಸಮುದ್ರ
ದೇವಲಾಪುರ
ದೇವಲಾಪುರದ
ದೇವಲಾಪುರವನ್ನು
ದೇವಲಾಪುರವು
ದೇವಲಾಪುರವೂ
ದೇವಲಾಪುರವೆಂಬ
ದೇವಲಾಪುರವೇ
ದೇವವೃಂದ
ದೇವಸತ್ತಿ
ದೇವಸ್ಥಾನಗಳಿಗೆ
ದೇವಾಜಮಾಂಬ
ದೇವಾಜಮ್ಮ
ದೇವಾಜಮ್ಮಣ್ಣಿಯು
ದೇವಾಪುರ
ದೇವಾರಾಧ್ಯರು
ದೇವಾಲಪುರದ
ದೇವಾಲಯ
ದೇವಾಲಯಕ್ಕೆ
ದೇವಾಲಯಗಳ
ದೇವಾಲಯಗಳನ್ನು
ದೇವಾಲಯಗಳಲ್ಲಿ
ದೇವಾಲಯಗಳಾಗಿವೆ
ದೇವಾಲಯಗಳಿಗೂ
ದೇವಾಲಯಗಳಿಗೆ
ದೇವಾಲಯಗಳು
ದೇವಾಲಯಗಳುಒಂದು
ದೇವಾಲಯಗಳೂ
ದೇವಾಲಯದ
ದೇವಾಲಯದಲ್ಲಿ
ದೇವಾಲಯದಲ್ಲಿರುವ
ದೇವಾಲಯನ್ನು
ದೇವಾಲಯನ್ನೂ
ದೇವಾಲಯವನ್ನು
ದೇವಾಲಯವನ್ನೂ
ದೇವಾಲಯವಾಗಿದೆ
ದೇವಾಲಯವಾದ
ದೇವಾಲಯವಿದೆ
ದೇವಾಲಯವಿದ್ದು
ದೇವಾಲಯವು
ದೇವಾಲಯವೂ
ದೇವಾಲಯವೆಂದರೆ
ದೇವಾಲಯವೇ
ದೇವಾಲವೂ
ದೇವಿ
ದೇವಿಯ
ದೇವಿಯನ್ನು
ದೇವಿಯರಿಗೆ
ದೇವೀರಮ್ಮಣ್ಣಿ
ದೇವೀರಮ್ಮನ
ದೇವೇಂದ್ರ
ದೇಶ
ದೇಶಕಾವಲುಗಾರರಿದ್ದರು
ದೇಶಗಳನ್ನು
ದೇಶದ
ದೇಶದಲ್ಲಿ
ದೇಶದಲ್ಲಿರುವ
ದೇಶರಾಜ್ಯನಾಡುಮಂಡಲ
ದೇಶವಾಗಿತ್ತೆಂದು
ದೇಶಶ್ರೀ
ದೇಶಸ್ಥಂ
ದೇಶಸ್ಯ
ದೇಶಾಂತರ
ದೇಶಾಂತ್ರಿ
ದೇಶಾಖ್ಯೇ
ದೇಶೇ
ದೇಸಾಯಿ
ದೇಸಾಯಿಯವರು
ದೇಸಿಯ
ದೇಸಿಯಪ್ಪನ
ದೇಸಿಯರು
ದೇಹವನ್ನು
ದೈವ
ದೈವಕೃಪೆಯಿಂದ
ದೈವದತ್ತವಾಗಿ
ದೈವದತ್ತವಾದ
ದೈವಭಕ್ತರಾಗಿದ್ದರು
ದೊಡ್ಡ
ದೊಡ್ಡಅರಸಿನಕೆರೆ
ದೊಡ್ಡಉಳುವರ್ತಿ
ದೊಡ್ಡಕೃಷ್ಣರಾಯರು
ದೊಡ್ಡಗದ್ದವಳ್ಳಿ
ದೊಡ್ಡಗದ್ದವಳ್ಳಿಯ
ದೊಡ್ಡಗರುಡನಹಳ್ಳಿ
ದೊಡ್ಡಗಾಡಿಗನಹಳ್ಳಿ
ದೊಡ್ಡಜಟಕ
ದೊಡ್ಡಜಟಕಾ
ದೊಡ್ಡದಾಗಿದ್ದು
ದೊಡ್ಡದು
ದೊಡ್ಡದೇವಯ್ಯನ
ದೊಡ್ಡದೇವಯ್ಯನವರು
ದೊಡ್ಡದೇವರಾಜ
ದೊಡ್ಡದೇವರಾಜನ
ದೊಡ್ಡದೇವರಾಜನು
ದೊಡ್ಡದೇವರಾಜನೂ
ದೊಡ್ಡದೇವರಾಜರಿಗೆ
ದೊಡ್ಡದೇವರಾಯರು
ದೊಡ್ಡದೊಂದು
ದೊಡ್ಡದ್ಯಾಮಗೌಡನಿಗೆ
ದೊಡ್ಡಪ್ಪ
ದೊಡ್ಡಪ್ಪಂದಿರು
ದೊಡ್ಡಪ್ಪನೊಡನೆ
ದೊಡ್ಡಬೆಟ್ಟದ
ದೊಡ್ಡಮಸೀದಿ
ದೊಡ್ಡಮ್ಮ
ದೊಡ್ಡಯಗಟಿ
ದೊಡ್ಡಯ್ಯ
ದೊಡ್ಡಯ್ಯನಹಳ್ಳಿ
ದೊಡ್ಡಯ್ಯನು
ದೊಡ್ಡರಸಿನಕೆರೆ
ದೊಡ್ಡವಡ್ಡಅರಗುಡಿ
ದೊಡ್ಡಹುಂಡಿ
ದೊಡ್ಡಾದಣ್ಣನ
ದೊಡ್ಡಿ
ದೊಡ್ಡಿಘಟ್ಟ
ದೊಡ್ಡಿಯೇ
ದೊಡ್ಡೈಯನವರ
ದೊರಕಿದೆ
ದೊರಕಿದ್ದು
ದೊರಕಿರುವ
ದೊರಕಿಲ್ಲ
ದೊರಕಿವೆ
ದೊರಕುತ್ತವೆ
ದೊರಕುವ
ದೊರಭಕ್ಕೆರೆ
ದೊರೆ
ದೊರೆಗಳ
ದೊರೆಗಳಲ್ಲಿ
ದೊರೆಗಳಾಗಿರಬಹುದು
ದೊರೆಗಳು
ದೊರೆತ
ದೊರೆತನದಿರವನ್ನು
ದೊರೆತವು
ದೊರೆತಿದ್ದು
ದೊರೆತಿರುವ
ದೊರೆತಿವೆ
ದೊರೆತು
ದೊರೆಯದಿರುವುದು
ದೊರೆಯದೇ
ದೊರೆಯವ
ದೊರೆಯಿತೆಂದು
ದೊರೆಯು
ದೊರೆಯುತ್ತದೆ
ದೊರೆಯುತ್ತವೆ
ದೊರೆಯುತ್ತಿತ್ತು
ದೊರೆಯುವ
ದೊರೆಯುವುದಿಲ್ಲ
ದೊರೆವರು
ದೋರ
ದೋರನು
ದೋರಸಮುದ್ರ
ದೋರಸಮುದ್ರಕ್ಕೆ
ದೋರಸಮುದ್ರದ
ದೋರಸಮುದ್ರದದಲ್ಲಿದ್ದನು
ದೋರಸಮುದ್ರದಲು
ದೋರಸಮುದ್ರದಲ್ಲಿ
ದೋರಸಮುದ್ರದಲ್ಲೇ
ದೋರಸಮುದ್ರದಿಂದ
ದೋರಸಮುದ್ರವನ್ನು
ದೋರಸಮುದ್ರಹಳೆಯಬೀಡು
ದೋರಸಮುದ್ರಾಖ್ಯಾಂ
ದೋರಸಮುದ್ರಿಂದ
ದೋಸ್ತಂಭದೊಳು
ದೌರ್ಬಲ್ಯಗಳ
ದ್ಧರಣಂ
ದ್ಯಾವಣ್ಣನು
ದ್ಯಾವಣ್ಣಹೆಮ್ಮಾಡಿಯಣ್ಣ
ದ್ಯಾವರಹಳ್ಳಿ
ದ್ಯಾವರಹಳ್ಳಿಯಲ್ಲೂ
ದ್ರಮಿಳ
ದ್ರಮಿಳಸಂಘದ
ದ್ರಾವಿಡಾನ್ವಯದ
ದ್ರುವನು
ದ್ರೋಹಘರಟ್ಟ
ದ್ರೋಹಘರಟ್ಟಂ
ದ್ರೋಹಘರಟ್ಟನೆಂಬ
ದ್ವಾಪರಯುಗದ
ದ್ವಾರಕೆಯಿಂದ
ದ್ವಾರದಲ್ಲಿರುವ
ದ್ವಾರಪಕ್ಷದ
ದ್ವಾರಸಮುದ್ರದಿಂದ
ದ್ವಾರಾವತಿ
ದ್ವಾರಾವತಿನಾಥನ
ದ್ವಾವೇತಾವಥ
ದ್ವಿಗುಣಮತ್ರಿಗುಣಂಚತುರ್ಗಣಂ
ದ್ವಿಗುಣೀಕೃತ
ದ್ವಿಜವಂಶತಿಲಕನೂ
ದ್ವಿತೀಯವಿಭವಂ
ದ್ವೀಪವನ್ನು
ದ್ವೀಪವಿದ್ದು
ದ್ವೇಷವನ್ನು
ಧಕ್ಕೆಯಾದ
ಧನ
ಧನಂಜಯಪುರವನ್ನಾಗಿ
ಧನಂಜಯರಾಯ
ಧನಂಜಯರಾಯವೊಡೆಯನು
ಧನಗೂರನ್ನು
ಧನಗೂರಿಗೆ
ಧನಗೂರಿನ
ಧನಗೂರು
ಧನಗೂರುಸ್ಥಳದ
ಧನವೆಲ್ಲ
ಧನಸಹಾಯವನ್ನು
ಧನುಗೂರನ್ನು
ಧನುಗೂರು
ಧನುರು
ಧನುರ್ಮಾಸ
ಧನುರ್ವಿದ್ಯಾಪರಿಣತರುಂ
ಧರಣಿದೇವತಾರುದ್ರನುಂ
ಧರಣೀತೇಃ
ಧರಣೀದೇವ
ಧರಣೀವರಾಹ
ಧರಾಮರೋತ್ತಂಸನಾಗಿದ್ದನೆಂದು
ಧರಾರಾಜ್ಯವಾಳುತ್ತಿದ್ದಾಗ
ಧರಿಸದೇ
ಧರಿಸಿ
ಧರಿಸಿದ
ಧರಿಸಿದಂತೆ
ಧರಿಸಿದನು
ಧರಿಸಿದನೆಂದು
ಧರಿಸಿದ್ದ
ಧರಿಸಿದ್ದನು
ಧರಿಸಿದ್ದನೆಂದು
ಧರಿಸಿದ್ದನ್ನು
ಧರಿಸಿದ್ದರಿಂದ
ಧರಿಸಿದ್ದರೆಂದೂ
ಧರಿಸಿದ್ದಾನೆ
ಧರಿಸಿರುವ
ಧರಿಸುತ್ತಿದ್ದ
ಧರಿಸುವ
ಧರೆ
ಧರೆಕೂರ್ತ್ತುಕೀರ್ತ್ತಿಕುಂ
ಧರೆಗೆಲ್ಲಂ
ಧರೆತನ್ನಂ
ಧರೆಯನ್ನು
ಧರೆಯೊಳ್
ಧರ್ಮ
ಧರ್ಮಂಗಳನ್ನು
ಧರ್ಮಕಾರ್ಯಗಳನ್ನು
ಧರ್ಮಕಾರ್ಯಗಳಿಗೆ
ಧರ್ಮಕಾರ್ಯಗಳು
ಧರ್ಮಕ್ಕೆ
ಧರ್ಮಗಳನ್ನು
ಧರ್ಮದ
ಧರ್ಮದಿಂದ
ಧರ್ಮದೊಳು
ಧರ್ಮಪತ್ನಿ
ಧರ್ಮಪರಾಯಣರಾದ
ಧರ್ಮಪ್ರಸಂಗದ
ಧರ್ಮಬುದ್ಧಿ
ಧರ್ಮಬೊಜ್ಜವಿಷ್ಣುವರ್ಧನ
ಧರ್ಮಭೂಮಿಯಾಗಿದ್ದ
ಧರ್ಮಮಹಾಧಿರಾಜ
ಧರ್ಮಮಹಾಧಿರಾಜರೆಂದು
ಧರ್ಮಮಹಾರಾಜಾಧಿರಾಜ
ಧರ್ಮಮಹಾರಾಜಾಧಿರಾಜನು
ಧರ್ಮವನ್ನು
ಧರ್ಮವನ್ನೇ
ಧರ್ಮವಾಗಬೇಕೆಂದು
ಧರ್ಮವಾಗಲೆಂದು
ಧರ್ಮವಾಗಿ
ಧರ್ಮವು
ಧರ್ಮವೇ
ಧರ್ಮಶಾಸ್ತ್ರಕ್ಕನುಗುಣವಾಗಿ
ಧರ್ಮಶ್ಚ
ಧರ್ಮಸತ್ರಗಳನ್ನು
ಧರ್ಮಾಗ್ರಹಾರವಾಗಿ
ಧರ್ಮಾನುಯಾಯಿಗಳಾದ
ಧರ್ಮಾಪುರ
ಧರ್ಮಾಪುರವನ್ನು
ಧರ್ಮಾಪುರವೆಂಬ
ಧರ್ಮೇಶ್ವರದೇವರೇ
ಧರ್ಮ್ಮಪ್ರತಿಪಾಳಕರುಮಪ್ಪ
ಧರ್ಮ್ಮಶೀಲಾಕ್ಕಮಾಗರ್ಭಶುಕ್ತಿಮುಕ್ತಾಫಲಾತ್ಮನಃ
ಧವಳಂಕಭೀಮ
ಧವಳಾಂಕಭೀಮ
ಧಾನ್ಯದ
ಧಾನ್ಯವನ್ನು
ಧಾರಾಧತ್ತವಾಗಿ
ಧಾರಾನಗರವನ್ನು
ಧಾರಾಪುರಗಳನ್ನು
ಧಾರಾಪುರದಿಂದ
ಧಾರಾಪೂರ್ವಕ
ಧಾರಾಪೂರ್ವಕವಾಗಿ
ಧಾರಿಣಿಯು
ಧಾರೆ
ಧಾರೆಯಂ
ಧಾರೆಯನಾತ್ಮ
ಧಾರೆಯನೆರೆದು
ಧಾರೆಯನೆರೆಸಿ
ಧಾರೆಯೆರೆದು
ಧಾರೆಯೆರೆದುಕೊಡುತ್ತಾರೆ
ಧಾರೆಯೆರೆಸಿ
ಧಾರ್ಮಿಕ
ಧಾರ್ಮಿಸ್ಥಳವಾದ
ಧಾವಿಸಿದ
ಧೀರ
ಧುರಂಧರಂಮಮಾತ್ಯ
ಧುರದ
ಧುರೀಣ
ಧುರೀಣಸ್ಯ
ಧುರ್ಮ್ಮಣ್ಣ
ಧೂಳೀಪಟಮಾಡಿ
ಧೃತಸತ್ಯವಾಕ್ಯಂ
ಧೈರ್ಯಸುರಗಾತ್ರ
ಧ್ರುವ
ಧ್ರುವಉಂಡಿಗೆಯ
ಧ್ರುವನ
ಧ್ರುವನು
ಧ್ವಂಸಮಾಡಿದನು
ಧ್ವಜವನ್ನು
ಧ್ವಜಸ್ಥಂಭವನ್ನು
ಧ್ವಜಿನೀಪತಿ
ಧ್ವಜಿನೀಪತಿಯಾದ
ಧ್ವನಿಯೂ
ನಂಗಲಿ
ನಂಗಲಿಯ
ನಂಗಲಿಯನ್ನು
ನಂಜನಗೂಡಿನ
ನಂಜನಗೂಡು
ನಂಜಯ
ನಂಜಯದೇವರಿಗೆ
ನಂಜರಾಜ
ನಂಜರಾಜನ
ನಂಜರಾಜನು
ನಂಜರಾಜನೆಂದೇ
ನಂಜರಾಜನೇ
ನಂಜರಾಜಯ್ಯ
ನಂಜರಾಜಯ್ಯನ
ನಂಜರಾಜಯ್ಯನು
ನಂಜರಾಜಸಮುದ್ರವೆಂಬ
ನಂಜರಾಜೈಯ್ಯನವರ
ನಂಜರಾಜೊಡೆಯನ
ನಂಜರಾಯ
ನಂಜರಾಯಜ
ನಂಜರಾಯನು
ನಂಜರಾಯಪಟ್ಟಣದ
ನಂಜರಾಯೊಡೆಯನ
ನಂಜರಾಯ್ಯನನ್ನು
ನಂಜವೊಡೇರಿಗೆ
ನಂಜಿನ
ನಂಜೀನಾಥನಿಗೆ
ನಂಜುಂಡಸ್ವಾಮಿ
ನಂಜುಂಡೇಶ್ವರ
ನಂಜುಡೇಶ್ವರ
ನಂಜೇಗವುಡ
ನಂಜೇಹೆಬ್ಬಾರುವನಿಗೆ
ನಂಟರಂಗರಕನುಂ
ನಂತರ
ನಂತರದ
ನಂತರದಲ್ಲಿ
ನಂತರವಷ್ಟೇ
ನಂತರವೂ
ನಂತರವೇ
ನಂದಗಿರಿ
ನಂದಗಿರಿನಾಥ
ನಂದನ
ನಂದಾದೀಪ
ನಂದಾದೀಪಕ್ಕೆ
ನಂದಾದೀವಿಗೆ
ನಂದಾದೀವಿಗೆಗೆ
ನಂದಾದೀವಿಗೆಯು
ನಂದಿ
ನಂದಿನ್ತವರಿವರಳವೆ
ನಂದಿಯಾಲದ
ನಂದಿವರ್ಮನ
ನಂದಿವರ್ಮನಿಂದ
ನಂದೀಧ್ವಜವನ್ನು
ನಂದ್ಯಾಲ
ನಂದ್ಯಾಲದ
ನಂನಿಯಮೇರು
ನಂನಿಯಮೇರುವನ್ನು
ನಂಬಲಾಗಿದೆ
ನಂಬಿ
ನಂಬಿಕೆ
ನಂಬಿನಾಯಕನಹಳ್ಳಿ
ನಂಬಿನಾಯಕನಹಳ್ಳಿಯ
ನಂಬಿಪಿಳ್ಳೆ
ನಂಬುಗೆಯ
ನಕರಗಳ
ನಖರ
ನಖರಗಳೆಲ್ಲರೂ
ನಖರೇಶ್ವರ
ನಗರ
ನಗರಕ್ಕೆ
ನಗರದ
ನಗರದಲ್ಲಿ
ನಗರವಿರುವ
ನಗರುರ
ನಗರೂರು
ನಗರೇರ
ನಗರೇಶ್ವರದೇವರಿಗೆ
ನಗುಲನ
ನಗುಲನಹಳ್ಳಿ
ನಗುವನಹಳ್ಳಿ
ನಟಿ
ನಟಿವೆಂಕಟಪ್ಪನಾಯಕ
ನಟ್ಟಕಲ್ಲು
ನಡದ
ನಡಸಿದಂ
ನಡುನಾಡಿನಲ್ಲಿ
ನಡುವಣ
ನಡುವಿನ
ನಡುವೆ
ನಡುವೆಯೂ
ನಡೆದ
ನಡೆದಲ್ಲಿ
ನಡೆದವು
ನಡೆದಾಗ
ನಡೆದಿದೆ
ನಡೆದಿರಬಹುದಾದ
ನಡೆದಿರಬಹುದು
ನಡೆದಿರಬಹುದೆಂದು
ನಡೆದಿರಬೇಕೆಂದು
ನಡೆದಿರುವ
ನಡೆದಿರುವುದನ್ನು
ನಡೆದಿರುವುದು
ನಡೆದು
ನಡೆದುದರ
ನಡೆದುದು
ನಡೆದುಬಂದು
ನಡೆದುಹೋಗಬೇಕಾಯಿತು
ನಡೆಯದೇ
ನಡೆಯಬೇಕಾಗಿದೆ
ನಡೆಯಲಿಎಂದು
ನಡೆಯಿತು
ನಡೆಯಿತೆಂದು
ನಡೆಯು
ನಡೆಯುತ
ನಡೆಯುತ್ತದೆಂದು
ನಡೆಯುತ್ತಲೇ
ನಡೆಯುತ್ತಿತ್ತು
ನಡೆಯುತ್ತಿದ್ದ
ನಡೆಯುತ್ತಿದ್ದುದನ್ನು
ನಡೆಯುತ್ತಿರುವಾಗಲೇ
ನಡೆಯುವ
ನಡೆಯುವಂತೆ
ನಡೆವಂದು
ನಡೆವಲ್ಲಿ
ನಡೆವಲ್ಲಿಗೆಪುರದಲ್ಲಿದ್ದ
ನಡೆಸತೊಡಗಿದರು
ನಡೆಸತೊಡಗಿದಳು
ನಡೆಸಬೇಕು
ನಡೆಸಲಾಗಿದೆ
ನಡೆಸಲಿಲ್ಲ
ನಡೆಸಲು
ನಡೆಸಿ
ನಡೆಸಿಕೊಂಡು
ನಡೆಸಿಕೊಡಬೇಕೆಂದು
ನಡೆಸಿದ
ನಡೆಸಿದನು
ನಡೆಸಿದನೆಂದು
ನಡೆಸಿದನೆಂದೂ
ನಡೆಸಿದರು
ನಡೆಸಿದರೂ
ನಡೆಸಿದರೆಂದು
ನಡೆಸಿದ್ದಾರೆಂದು
ನಡೆಸಿದ್ದು
ನಡೆಸಿರುವ
ನಡೆಸಿರುವು
ನಡೆಸಿರುವುದು
ನಡೆಸುತ್ತ
ನಡೆಸುತ್ತಾ
ನಡೆಸುತ್ತಿದರೆಂಬುದು
ನಡೆಸುತ್ತಿದ್ದ
ನಡೆಸುತ್ತಿದ್ದಂತೆ
ನಡೆಸುತ್ತಿದ್ದನು
ನಡೆಸುತ್ತಿದ್ದನೆಂದು
ನಡೆಸುತ್ತಿದ್ದನೆಂದೂ
ನಡೆಸುತ್ತಿದ್ದನೆಂಬುದು
ನಡೆಸುತ್ತಿದ್ದರು
ನಡೆಸುತ್ತಿದ್ದರೂ
ನಡೆಸುತ್ತಿದ್ದರೆಂದು
ನಡೆಸುತ್ತಿದ್ದರೆಂಬುದು
ನಡೆಸುತ್ತಿದ್ದರೇ
ನಡೆಸುತ್ತಿದ್ದಾಗ
ನಡೆಸುತ್ತಿದ್ದಿರಬಹುದು
ನಡೆಸುತ್ತಿದ್ದು
ನಡೆಸುತ್ತಿದ್ದುದನ್ನು
ನಡೆಸುತ್ತಿದ್ದುದರಿಂದ
ನಡೆಸುತ್ತಿರುತ್ತಾನೆ
ನಡೆಸುತ್ತೇವೆಂದು
ನಡೆಸುವ
ನಡೆಸುವಂತಹ
ನಡೆಸುವುದು
ನದಿ
ನದಿಗಳ
ನದಿಗಳನ್ನು
ನದಿಗಳಾಗಿವೆ
ನದಿಗಳು
ನದಿಗೆ
ನದಿಯ
ನದಿಯನ್ನು
ನದಿಯನ್ನೂ
ನದಿಯಮಡು
ನದಿಯಲ್ಲಿ
ನದಿಯವರೆಗೂ
ನದಿಯಾಚೆ
ನದಿಯು
ನನ್ದನನೊಲವಿಂ
ನನ್ನವ್ವೆ
ನನ್ನಿ
ನನ್ನಿಕಂದರ್ಪನೆಂಬುವವನು
ನನ್ನಿಗ
ನನ್ನಿನೊಳಂಬನು
ನನ್ನಿಮಳಲೂರಂ
ನನ್ನಿಯ
ನನ್ನಿಯಗಂಗ
ನನ್ನಿಯಮೇರು
ನನ್ನಿಯಸೇಕರ
ನಮಗೆ
ನಮಸ್ಕಾರವನ್ನು
ನಮೂದಾಗಿರುವ
ನಮೂದಾಗಿರುವಂತೆ
ನಮೂದಿಸಲಾಗಿದೆ
ನಮೂದಿಸಿದ್ದು
ನಮೂದಿಸಿರಬಹುದು
ನಮೂದಿಸಿರುವುದಿಲ್ಲ
ನಮೂದಿಸಿಲ್ಲ
ನಮ್ಮ
ನಮ್ಮಾಳ್ವಾರ್
ನಯಕೀರ್ತಿ
ನಯಣದ
ನಯಧೀರ
ನಯಧೀರರೊಡಗೂಡಿ
ನಯೋಂನತೇಃ
ನರಅಸೀಪುರ
ನರಗ
ನರಗಲು
ನರಗುಂದ
ನರಣನಾರಾಯಣ
ನರಪತಿ
ನರಪತಿಬೆನ್ನೊಳಿರ್ದೊನಿದಿರಾಂತುದು
ನರಭಕ್ಷಕ
ನರಮನಕಟ್ಟೆ
ನರಸ
ನರಸಂಣನಾಯ್ಕರು
ನರಸಣ್ಣ
ನರಸಣ್ಣನಾಯಕರ
ನರಸನ
ನರಸನಾಯಕ
ನರಸನಾಯಕನ
ನರಸನಾಯಕನಿಗೆ
ನರಸನಾಯಕನು
ನರಸನಿಗೆ
ನರಸನು
ನರಸಯ್ಯ
ನರಸಯ್ಯನು
ನರಸರಾಜ
ನರಸರಾಜನ
ನರಸರಾಜನು
ನರಸರಾಜನೆಂದೇ
ನರಸರಾಜರ
ನರಸರಾಜೊಡೆಯ
ನರಸರಾಜೊಡೆಯರ
ನರಸಾಕ್ಷಿಯಾಗಿರುತ್ತಾರೆ
ನರಸಾವನಿಪಾಲಜ
ನರಸಿಂಗ
ನರಸಿಂಗಣ್ಣಗಳಿಗೆ
ನರಸಿಂಗದೇವ
ನರಸಿಂಗದೇವನು
ನರಸಿಂಗದೇವರು
ನರಸಿಂಗನ
ನರಸಿಂಗನಾಯಕ
ನರಸಿಂಗನಾಯಕಂ
ನರಸಿಂಗನಾಯಕನ
ನರಸಿಂಗನಾಯಕನು
ನರಸಿಂಗಯ್ಯದೇವ
ನರಸಿಂಗಯ್ಯನ
ನರಸಿಂಗರಾಜವೊಡೆಯರ
ನರಸಿಂಗರಾಯ
ನರಸಿಂಗವರ್ಮ
ನರಸಿಂಗವರ್ಮನು
ನರಸಿಂಹ
ನರಸಿಂಹದೇವರ
ನರಸಿಂಹದೇವರಿಗೆ
ನರಸಿಂಹನ
ನರಸಿಂಹನಂತೆಯೇ
ನರಸಿಂಹನಕಾಲದಲ್ಲಿ
ನರಸಿಂಹನಕಾಲದವರೆಗೆ
ನರಸಿಂಹನನ್ನು
ನರಸಿಂಹನನ್ನೂ
ನರಸಿಂಹನರಪಾಳಂ
ನರಸಿಂಹನರಸುಗೆಯ್ಯುತ್ತಿರ್ದ್ದಂ
ನರಸಿಂಹನಲ್ಲಿ
ನರಸಿಂಹನಿಗೂ
ನರಸಿಂಹನಿಗೆ
ನರಸಿಂಹನು
ನರಸಿಂಹನೂ
ನರಸಿಂಹಮಹಾರಾಯರು
ನರಸಿಂಹರನ್ನು
ನರಸಿಂಹರಾಯನ
ನರಸಿಂಹರಾಯರು
ನರಸಿಂಹವರ್ಮನ
ನರಸಿಂಹವರ್ಮನೂ
ನರಸಿಂಹವರ್ಮ್ಮನೋಡಿದ
ನರಸಿಂಹಸ್ವಾಮಿ
ನರಸಿಂಹಸ್ವಾಮಿಗೆ
ನರಸಿಂಹಸ್ವಾಮಿಯ
ನರಸಿಂಹಸ್ವಾಮಿಯವರ
ನರಸಿಂಹಾಚಾರ್ಯರು
ನರಸಿಂಹೋರ್ವ್ವೀಶನ
ನರಸೀಪುರ
ನರಸೀಪುರದ
ನರಸೇಂದ್ರನೆಂಬ
ನರಿ
ನರಿಹಳ್ಳಿ
ನರಿಹಳ್ಳಿಯಲ್ಲೂ
ನರೆಗನಹಳ್ಳಿ
ನಲಗೌಡ
ನಲವತ್ತು
ನಲ್ಲತಂಬಿಯು
ನಳನಹುಷಾದಿಗಳಂತೆ
ನಳಮಾರುಡು
ನಳಸಂವತ್ಸರ
ನಳಸಂವತ್ಸರದ
ನಳಸಂವತ್ಸರವು
ನಳಿನಕೆರೆಯನ್ನು
ನವಗ್ರಹಗಳೆಂದು
ನವದಂಡನಾಯಕರುಗಳೆಂದು
ನವದಣ್ಣಾಯಕರೆಂಬ
ನವನಿಧಿಕುಲಪರ್ವತದ
ನವರಂಗದ
ನವರತ್ನಕಿರೀಟವನ್ನು
ನವಲೆನಾಡು
ನವಶಿಲಾಯುಗದ
ನವಾಬ
ನವಾಬ್
ನವಿಲು
ನವೀಕರಿಸಿದನು
ನವೆಂಬರ್
ನವ್ಯ
ನವ್ವಾಬ್
ನಾಕಣ
ನಾಕನಿಪ್ಪ
ನಾಕಯ್ಯನು
ನಾಕಹಳ್ಳಿ
ನಾಗ
ನಾಗಂಣ
ನಾಗಂಣವೊಡೆಯ
ನಾಗಂಣವೊಡೆಯನು
ನಾಗಂಣ್ಣ
ನಾಗಣ್ಣ
ನಾಗಣ್ಣನ
ನಾಗಣ್ಣನವರು
ನಾಗಣ್ಣನು
ನಾಗಣ್ಣನ್ನನು
ನಾಗದೇವ
ನಾಗದೇವನ
ನಾಗದೇವನು
ನಾಗದೇವಭಟ್ಟರಿಗೆ
ನಾಗನ
ನಾಗನಹಳ್ಳಿ
ನಾಗನಾಯಕನ
ನಾಗನಿಂದ
ನಾಗನಿಂದಲೇ
ನಾಗನು
ನಾಗಪ್ಪ
ನಾಗಪ್ಪನ
ನಾಗಪ್ಪನಾಗರಸ
ನಾಗಭಟ್ಟನಿಗೆ
ನಾಗಭಟ್ಟನು
ನಾಗಮಂಗಲ
ನಾಗಮಂಗಲಕೆ
ನಾಗಮಂಗಲಕ್ಕೆ
ನಾಗಮಂಗಲಕ್ಕೆರಾಜ್ಯಕ್ಕೆ
ನಾಗಮಂಗಲದ
ನಾಗಮಂಗಲದಲ್ಲಿ
ನಾಗಮಂಗಲರಾಜ್ಯ
ನಾಗಮಂಗಲರಾಜ್ಯದ
ನಾಗಮಂಗಲವನ್ನು
ನಾಗಮಂಗಲವು
ನಾಗಮಂಗಲಶ್ರವಣಬೆಳಗೊಳ
ನಾಗಮಂಗಲಸ್ಥಳದ
ನಾಗಮಯ್ಯ
ನಾಗಮಯ್ಯನಿಗೆ
ನಾಗಮಯ್ಯನು
ನಾಗಮರ್ವ
ನಾಗಮಾರ್ಜಿತಮದಾತ್ತಿಂಮಕ್ಷಿತೀಂದ್ರಾತ್ಮಜಃ
ನಾಗಯ್ಯ
ನಾಗಯ್ಯಗಳ
ನಾಗಯ್ಯನ
ನಾಗಯ್ಯನನ್ನು
ನಾಗಯ್ಯನವರ
ನಾಗಯ್ಯನವರು
ನಾಗಯ್ಯನವರೂ
ನಾಗಯ್ಯನಿಗೆ
ನಾಗಯ್ಯನು
ನಾಗಯ್ಯನೂ
ನಾಗಯ್ಯನೆಂಬುವವನು
ನಾಗಯ್ಯನೇ
ನಾಗರಖಂಡ
ನಾಗರಖಂಡನವನ್ನು
ನಾಗರಘಟ್ಟದ
ನಾಗರದಮೊಲೆಗೋಡನ್ನು
ನಾಗರಸ
ನಾಗರಸನ
ನಾಗರಸನು
ನಾಗರಸರ
ನಾಗರಸರು
ನಾಗರಸರೆಂದು
ನಾಗರಹಾಳ
ನಾಗರಾಜರಾವ್
ನಾಗರಾಸಿ
ನಾಗರಿಲಿಪಿ
ನಾಗಲದೇವಿ
ನಾಗಲದೇವಿಯಿಂದ
ನಾಗಲಾಂಬಿಕಾ
ನಾಗಲಾಂಬಿಕೆಯ
ನಾಗಲಾಂಬಿಕೆಯರು
ನಾಗಲಾದೇವಿ
ನಾಗಲಾದೇವಿಯಿಂದ
ನಾಗಲಾಪುರವೆಂದು
ನಾಗಲಾಪುರವೆಂಬ
ನಾಗಲೆ
ನಾಗವರ್ಮ
ನಾಗವರ್ಮ್ಮಯ್ಯ
ನಾಗವಲ್ಲಿ
ನಾಗಶಕ್ತಿ
ನಾಗಾಂಬಾ
ನಾಗಾಭಟ್ಟರಿಗೆ
ನಾಗಿದ್ದನೆಂದು
ನಾಗಿಯಕ್ಕನು
ನಾಗಿಯಣ್ಣನೆಂಬುವವನು
ನಾಗುತ್ತಾನೆ
ನಾಗೆಯ
ನಾಗೆಯನಾಯಕ
ನಾಗೆಯನಾಯಕನ
ನಾಗೆಯನಾಯಕರು
ನಾಗೇಣ
ನಾಗೇಶ್ವರ
ನಾಗೊಡೆಯನಿಗೆ
ನಾಗೊಡೆಯನು
ನಾಗೋಹಳ್ಳಿ
ನಾಚಿಯಾರಮ್ಮನಿಗೆ
ನಾಚ್ಚಾರಮ್ಮನವರ
ನಾಚ್ಚಿಯಾರ್ಗೆ
ನಾಟನಹಳ್ಳಿ
ನಾಟ್ಟು
ನಾಟ್ಟೋರ್ಗಳ್
ನಾಡ
ನಾಡಅನ್ನು
ನಾಡಗವುಡ
ನಾಡಗವುಡಗಳ
ನಾಡಗವುಡರು
ನಾಡಗಾವುಂಡನನ್ನು
ನಾಡಗಾವುಂಡರು
ನಾಡಗೌಡರೆಂದು
ನಾಡಗೌಡಿಕೆ
ನಾಡನ್ನು
ನಾಡನ್ನೂ
ನಾಡಪ್ರಭು
ನಾಡಪ್ರಭುಗಳು
ನಾಡಬೋಯನಹಳ್ಳಿ
ನಾಡಬೋಯನಹಳ್ಳಿನಾಡಬೋವನಹಳ್ಳಿ
ನಾಡಬೋವನಹಳ್ಳಿ
ನಾಡಮಂಡಳಿಕ
ನಾಡಮಾಣಿಕದೊಡಲೂರನ್ನು
ನಾಡಮಾಣಿಕದೊಡಲೂರಿನಲ್ಲಿ
ನಾಡಮಾಣಿಕದೊಡಲೂರಿನಲ್ಲಿದ್ದ
ನಾಡಮಾಣಿಕದೊಡಲೂರು
ನಾಡರಸರಾದ
ನಾಡವಂಡಳೀಕರು
ನಾಡಸುಂಕವನ್ನು
ನಾಡಾಗಿತ್ತು
ನಾಡಾಗಿತ್ತೆಂದು
ನಾಡಾಗಿದ್ದ
ನಾಡಾಗಿದ್ದು
ನಾಡಾಗಿರಬಹುದು
ನಾಡಾಗಿರುವ
ನಾಡಾದುದೆಲ್ಲವಮನೇಕಚ್ಛತ್ರಂ
ನಾಡಾಳುತ್ತಿದ್ದನೆಂದು
ನಾಡಾಳುವ
ನಾಡಾಳ್ವ
ನಾಡಾಳ್ವಂ
ನಾಡಾಳ್ವನು
ನಾಡಾಳ್ವರು
ನಾಡಿಗರು
ನಾಡಿಗವುಡನವರ
ನಾಡಿಗವುಡನವರು
ನಾಡಿಗವುಡರ
ನಾಡಿಗೂ
ನಾಡಿಗೆ
ನಾಡಿಗೇ
ನಾಡಿತ್ತೆಂದು
ನಾಡಿನ
ನಾಡಿನಲ್ಲಿ
ನಾಡಿನಲ್ಲಿತ್ತು
ನಾಡಿನಲ್ಲಿತ್ತೆಂದು
ನಾಡಿನಲ್ಲಿದ್ದ
ನಾಡಿನಲ್ಲಿದ್ದವು
ನಾಡಿನಲ್ಲಿಯೇ
ನಾಡಿನವರ
ನಾಡಿನಿಂದ
ನಾಡಿನೊಳಗಿರುವ
ನಾಡಿನೊಳಗೇ
ನಾಡಿಯರು
ನಾಡು
ನಾಡುಕಬ್ಬಪ್ಪು
ನಾಡುಕಲ್ಕಣಿನಾಡುಕಲಿಕಣಿನಾಡು
ನಾಡುಗಳ
ನಾಡುಗಳನ್ನು
ನಾಡುಗಳಾಗಿ
ನಾಡುಗಳಿಗೆ
ನಾಡುಗಳು
ನಾಡುಗಳೆಂದು
ನಾಡುಗಳೆಂಬ
ನಾಡುನೀರ್ಗುಂದ
ನಾಡುಬಡಗುಂದ
ನಾಡುಬಡಗುನಾಡುವಡಗೆರೆ
ನಾಡುವಟ್ಟವಾಗಿ
ನಾಡೆಂದರೆ
ನಾಡೆಂದು
ನಾಡೆಹಳ್ಳಿಗಳನ್ನು
ನಾಡೇ
ನಾಡೊಳಗಣ
ನಾಡೊಳಗಿನ
ನಾಣ್ಯಗಳನ್ನು
ನಾತನಸುತರಗಣಿತ
ನಾಥ
ನಾಥನನ್ನು
ನಾನಲ
ನಾನಲಕೆರೆ
ನಾನಲಕೆರೆಯ
ನಾನಲಕೆರೆಯನ್ನು
ನಾನಲಕೆರೆಯಲ್ಲಿ
ನಾನಲಕೆರೆಯುಇಂದಿನ
ನಾನಲಕೆರೆಲಾಳನಕೆರೆಯನ್ನು
ನಾನಲಕೆಱೆಯ
ನಾನಾ
ನಾನಾದೇಸಿ
ನಾನಾದೇಸಿಯಿಂದ
ನಾನಾವರ್ನ
ನಾನು
ನಾಮ
ನಾಮಕರಣ
ನಾಮಕರಣಮಾಡಿ
ನಾಮಗ್ರಾಮ
ನಾಮದ
ನಾಮಾಯಂ
ನಾಮಾವಳಿ
ನಾಮಾವಳಿಸಮಾಲಂಕೃತರುಂ
ನಾಮಾವಶೇಷಗೊಳಿಸಿದನೆಂದು
ನಾಮೆ
ನಾಮ್ನಾ
ನಾಯ
ನಾಯಂಕರ
ನಾಯಕ
ನಾಯಕಂ
ನಾಯಕಅರಸ
ನಾಯಕತನ
ನಾಯಕತನಕೆ
ನಾಯಕತನಕ್ಕೆ
ನಾಯಕತನದ
ನಾಯಕತನದಿಂದ
ನಾಯಕತನಮಂ
ನಾಯಕತನವನ್ನು
ನಾಯಕತನವೆಂಬ
ನಾಯಕತವನ್ನು
ನಾಯಕತ್ವ
ನಾಯಕತ್ವಕ್ಕೆ
ನಾಯಕದೇವ
ನಾಯಕದೇವಪಿಳ್ಳೆ
ನಾಯಕನ
ನಾಯಕನನ್ನಾಗಿ
ನಾಯಕನನ್ನು
ನಾಯಕನಹಳ್ಳಿ
ನಾಯಕನಾಗಿ
ನಾಯಕನಾಗಿದ್ದ
ನಾಯಕನಾಗಿರಬಹುದು
ನಾಯಕನಾದ
ನಾಯಕನಿಗೆ
ನಾಯಕನಿರಬಹುದು
ನಾಯಕನು
ನಾಯಕನೂ
ನಾಯಕನೆಂಬ
ನಾಯಕನೇ
ನಾಯಕಮಕ್ಕಳು
ನಾಯಕರ
ನಾಯಕರಗಂಡ
ನಾಯಕರನ್ನು
ನಾಯಕರಲ್ಲಿ
ನಾಯಕರಾಗಿದ್ದರು
ನಾಯಕರಿಗೂ
ನಾಯಕರಿಗೆ
ನಾಯಕರಿದ್ದರೆಂದು
ನಾಯಕರಿದ್ದು
ನಾಯಕರು
ನಾಯಕರುಗಳಿಗೆ
ನಾಯಕರುಗಳು
ನಾಯಕರೂ
ನಾಯಕಹೆಗ್ಗಡೆ
ನಾಯಕಿತ್ತಿ
ನಾಯಕಿತ್ತಿಗೆ
ನಾಯಕಿತ್ತಿಯ
ನಾಯಕಿತ್ತಿಯರ
ನಾಯಕಿತ್ತಿಯರು
ನಾಯಕಿತ್ತಿಯು
ನಾಯಿಂದರಿಗೆ
ನಾಯಿಂದರುಗಳಿಗೆ
ನಾಯಿಯನ್ನು
ನಾಯಿಯನ್ನೂ
ನಾಯಿಯನ್ನೇ
ನಾಯ್ಕ
ನಾಯ್ಕರಸನು
ನಾಯ್ಕರಸರನ
ನಾರಣದೇವಿ
ನಾರಣವೆಗ್ಗಡೆ
ನಾರಣವೆಗ್ಗಡೆಯಾಗಿರುವಂತೆ
ನಾರಣವೆಗ್ಗಡೆಯು
ನಾರಣವೆಗ್ಗಡೆಯೇ
ನಾರಣವೆರ್ಗಡೆ
ನಾರಣವೆರ್ಗ್ಗಡೆ
ನಾರಣವೆರ್ಗ್ಗಡೆಯಯು
ನಾರಣಾಂಕವಿದು
ನಾರಪ್ಪರಾಜ
ನಾರಪ್ಪರಾಜಯ್ಯನ
ನಾರಪ್ಪರಾಜಯ್ಯನು
ನಾರಯದೇವ
ನಾರಯದೇವನು
ನಾರಯ್ಯದೇವ
ನಾರಯ್ಯದೇವನ
ನಾರಸಿಂಗ
ನಾರಸಿಂಗಚತುರ್ವೇದಿ
ನಾರಸಿಂಗದೇವನ
ನಾರಸಿಂಗದೇವನು
ನಾರಸಿಂಗದೇವರು
ನಾರಸಿಂಗನು
ನಾರಸಿಂಗಯ್ಯದೇವನ
ನಾರಸಿಂಘ
ನಾರಸಿಂಘದೇವನ
ನಾರಸಿಂಘದೇವನು
ನಾರಸಿಂಘಯದೇವ
ನಾರಸಿಂಹ
ನಾರಸಿಂಹಚತುರ್ವೇದಿಮಂಗಲದ
ನಾರಸಿಂಹದೇವ
ನಾರಸಿಂಹದೇವನ
ನಾರಸಿಂಹದೇವರಸ
ನಾರಸಿಂಹದೇವರಸರು
ನಾರಸಿಂಹದೇವರಿಗೆ
ನಾರಸಿಂಹದೇವರು
ನಾರಸಿಂಹನ
ನಾರಸಿಂಹನನ್ನು
ನಾರಸಿಂಹನಿಗೂ
ನಾರಸಿಂಹನಿಗೆ
ನಾರಸಿಂಹನು
ನಾರಸಿಂಹರಸ
ನಾರಸಿಂಹರಾಯನ
ನಾರಸಿಂಹಸ್ವಾಮಿಗೆ
ನಾರಾಯಣ
ನಾರಾಯಣಗಿರಿ
ನಾರಾಯಣಗಿರಿದುರ್ಗ
ನಾರಾಯಣಗಿರಿದುರ್ಗದ
ನಾರಾಯಣದ
ನಾರಾಯಣದೇವರ
ನಾರಾಯಣದೇವರಿಗೆ
ನಾರಾಯಣದೇವರು
ನಾರಾಯಣದೇವಾಲಯದಲ್ಲಿರುವ
ನಾರಾಯಣನ
ನಾರಾಯಣನಿಗೆ
ನಾರಾಯಣನೆಂದೂ
ನಾರಾಯಣಪಾದಪಙ್ಕಜಯುಗೀವನ್ಯಸ್ತವಿಪ್ಪಗ್ಭರಃ
ನಾರಾಯಣಪುರವನ್ನು
ನಾರಾಯಣರಾಯರೆಂಬುವವರ
ನಾರಾಯಣಶೈಲಕ್ಕೆ
ನಾರಾಯಣಸ್ವಾಮಿ
ನಾರಾಯಣಾಂಬಿಕೆಯರ
ನಾರಿ
ನಾರಿಯಪ್ಪ
ನಾರಿವಾಳವನು
ನಾರ್ತ್ಬ್ಯಾಂಕ್
ನಾಲಯ್ಯನಿಗೆ
ನಾಲೂರಿನ
ನಾಲೆಯಿಂದ
ನಾಲ್ಕನೆಯ
ನಾಲ್ಕನೆಯವನು
ನಾಲ್ಕನೇ
ನಾಲ್ಕರಲ್ಲಿ
ನಾಲ್ಕು
ನಾಲ್ಕುಜನ
ನಾಲ್ಕೂ
ನಾಲ್ದೆಸೆಗೆ
ನಾಲ್ಮಡಿ
ನಾಲ್ವತ್ತು
ನಾಲ್ವರು
ನಾಲ್ವರೂ
ನಾಳನಕೆರೆ
ನಾಳೆಯಿಲ್ಲೆಂದ
ನಾಳ್ಗಾವುಂಡರು
ನಾವಿದ್ದೇವೆ
ನಾವು
ನಾವೇ
ನಾಶಪಡಿಸಿತೆಂದು
ನಾಶಪಡಿಸಿದನೆಂದು
ನಾಶಮಾಡಿ
ನಾಶವಾಗಿ
ನಾಸಿರ್ಜಂಗ್
ನಾಸಿರ್ಜಂಗ್ನ
ನಾೞ್ಪ್ರಭು
ನಿಂತರು
ನಿಂತರುಎಂದು
ನಿಂತಿದ್ದಾರೆ
ನಿಂತಿರಬಹುದು
ನಿಂತಿರುವುದು
ನಿಂತು
ನಿಂದರಿನ್ರಿಪಾಳರ
ನಿಂನಂತಾರೊಳವೊಕ್ಕು
ನಿಕ್ಕಿಯಣ್ಣ
ನಿಕ್ಕಿಯರಸ
ನಿಕ್ಕಿಯರಸನ
ನಿಕ್ಕಿರಸನಿಕ್ಕರಸ
ನಿಕ್ಕೀಶ್ವರ
ನಿಕ್ಕೇಶ್ವರ
ನಿಖರವಾಗಿ
ನಿಖಿಳಲಕ್ಷ್ಮೀ
ನಿಖಿಳಾಂ
ನಿಗದಿ
ನಿಗದಿಪಡಿಸಲಾಗಿದೆ
ನಿಗದಿಪಡಿಸಲು
ನಿಗದಿಪಡಿಸಿ
ನಿಗದಿಪಡಿಸಿದ್ದನ್ನು
ನಿಗದಿಪಡಿಸಿದ್ದಾರೆ
ನಿಗದಿಪಡಿಸುವ
ನಿಗೆ
ನಿಗ್ರಹಿಸಿದನು
ನಿಜ
ನಿಜಕಳತ್ರ
ನಿಜಕ್ಕೂ
ನಿಜಪ್ರತಾಪದಿಂದ
ನಿಜಪ್ರತಾಪಾದಧಿಗತ್ಯ
ನಿಜಪ್ರಧಾನ
ನಿಜರಾಜಧಾನಿ
ನಿಜರಾಧಾನಿಮಧಿವಸನ್
ನಿಜರಾಮ
ನಿಜವಿಜಯ
ನಿಜಸ್ವಾಮಿ
ನಿಜಾಂ
ನಿಜಾಂಶಂ
ನಿಜಾಮರು
ನಿಜಾಮರೊಂದಿಗೆ
ನಿಡದವೋಲು
ನಿಡುಗಲ್ಲಿನ
ನಿಡುವುಟೆಯನ್ನು
ನಿಡುವುಟೆಯು
ನಿಡುವೊಳಲಾಗಿರಬಹುದು
ನಿತ್ತರಿಸಲಾರದೆ
ನಿತ್ಯ
ನಿತ್ಯಂ
ನಿತ್ಯೋತ್ಸಾಹಗಳಿಗೆ
ನಿದರ್ಶನವನ್ನು
ನಿಧನನಾಗಿದ್ದನೆಂದು
ನಿಧನಾನಂತರ
ನಿಧಾನಃ
ನಿಧಾನವಾಗಿ
ನಿನಗೆಂದು
ನಿನಗೇನು
ನಿನ್ನ
ನಿಪ್ಪ
ನಿಬಂಧ
ನಿಬಂಧಿಯಾಗಿ
ನಿಮಿತ್ತ
ನಿಮ್ಮಡಿಯನ್ನು
ನಿಯಗಾಮುಂಡನ
ನಿಯತಕಾಲಿಕಗಳಲ್ಲಿ
ನಿಯಮದಂತೆ
ನಿಯುಕ್ತರಾದರೆ
ನಿಯುಕ್ತಳಾದಳು
ನಿಯೋಗಗಳ
ನಿಯೋಗದಿಂದ
ನಿಯೋಗದುರಂಧರ
ನಿಯೋಗನವನು
ನಿಯೋಗಾಧಿಪತಿ
ನಿಯೋಗಾಧಿಪತಿಗಳ
ನಿಯೋಗಾಧಿಪತಿಗಳು
ನಿಯೋಗಾಧಿಪತಿಗಳೂ
ನಿಯೋಗಿ
ನಿರಂತರ
ನಿರಂತರಂ
ನಿರಂತರವೆನ್ನಲು
ನಿರತನಾದನೆಂದು
ನಿರನಾಗಿದ್ದನೆಂದು
ನಿರವದ್ಯ
ನಿರಾಕುಳದಿಂದ
ನಿರಾತಂಕವಾಗಿ
ನಿರಿಸಿದರ್ಸ್ಸೋಮಾನ್ವಯೋರ್ವ್ವೀಶ್ವರರ್
ನಿರೀಕ್ಷಿಸುತ್ತಿದ್ದ
ನಿರೀಕ್ಷೆ
ನಿರುಪಾಧಿಕ
ನಿರುಪಾಯನಾಗಿ
ನಿರೂಪ
ನಿರೂಪಗಳನ್ನು
ನಿರೂಪಣೆಗಳನ್ನು
ನಿರೂಪದಂತೆ
ನಿರೂಪದಲಿ
ನಿರೂಪದಿಂದ
ನಿರೂಪವನ್ನು
ನಿರೂಪಿತ
ನಿರೂಪಿತವಾಗಿದ್ದು
ನಿರೂಪಿಸಲಾಗಿದೆ
ನಿರೂಪಿಸಲಾಗಿದ್ದು
ನಿರೂಪಿಸಿ
ನಿರೂಪಿಸಿದ್ದಾರೆ
ನಿರೂಪಿಸುವ
ನಿರ್ಗ್ಗುಂದ
ನಿರ್ಜಿತ್ಯ
ನಿರ್ಣಯ
ನಿರ್ದಿಷ್ಟಪ್ರಮಾಣದ
ನಿರ್ದೇಶಿಸಲು
ನಿರ್ಧರಿತವಾಗುತ್ತಿದ್ದವು
ನಿರ್ಧರಿಸಿದನು
ನಿರ್ಧರಿಸುತ್ತಿರಬಹುದು
ನಿರ್ಮಾಣ
ನಿರ್ಮಾಣಕ್ಕೆ
ನಿರ್ಮಾಣದ
ನಿರ್ಮಾಣವನ್ನು
ನಿರ್ಮಾಣವಾಗಿ
ನಿರ್ಮಾಣವಾಗಿರುವ
ನಿರ್ಮಾಣವಾಯಿತು
ನಿರ್ಮಾಣವೂ
ನಿರ್ಮಿತವಾಗಿದ್ದ
ನಿರ್ಮಿತವಾಗಿರಬಹುದು
ನಿರ್ಮಿತವಾದ
ನಿರ್ಮಿತವಾದವು
ನಿರ್ಮಿಸಲಾಗಿದೆ
ನಿರ್ಮಿಸಲು
ನಿರ್ಮಿಸಿ
ನಿರ್ಮಿಸಿಕೊಳ್ಳುವಂತಿರಲಿಲ್ಲವೆಂದು
ನಿರ್ಮಿಸಿದ
ನಿರ್ಮಿಸಿದನಷ್ಟೆ
ನಿರ್ಮಿಸಿದನು
ನಿರ್ಮಿಸಿದನೆಂದು
ನಿರ್ಮಿಸಿದರು
ನಿರ್ಮಿಸಿದರೆಂದು
ನಿರ್ಮಿಸಿದಳು
ನಿರ್ಮಿಸಿದಾಗ
ನಿರ್ಮಿಸಿದ್ದಾನೆ
ನಿರ್ಮಿಸಿರಬಹುದು
ನಿರ್ಮಿಸಿರಬಹುದೆಂದು
ನಿರ್ಮಿಸಿರುವ
ನಿರ್ಮಿಸಿರುವುದರಿಂದ
ನಿರ್ಮಿಸುತ್ತಾನೆ
ನಿರ್ಮಿಸುವುದರಲ್ಲಿ
ನಿರ್ಮೂಲ
ನಿರ್ಮೂಲನ
ನಿರ್ಮೂಲನೆ
ನಿರ್ಮೂಳನ
ನಿರ್ಮ್ಮಡಿಯನ್ನು
ನಿರ್ವಹಣೆ
ನಿರ್ವಹಣೆಗೆ
ನಿರ್ವಹಣೆಚಾರಿತ್ರಿಕ
ನಿರ್ವಹಣೆಯ
ನಿರ್ವಹಿಸುತ್ತಿದ್ದರು
ನಿರ್ವಹಿಸುತ್ತಿದ್ದರೆಂದು
ನಿರ್ವಹಿಸುತ್ತಿದ್ದುದನ್ನು
ನಿರ್ವಹಿಸುತ್ತಿದ್ದುದು
ನಿರ್ವಹಿಸುವ
ನಿರ್ವಹಿಸುವವರು
ನಿರ್ವಹಿಸುವವರೂ
ನಿಲವಂದದೇವರಿಗೆ
ನಿಲಿಸಿದಂ
ನಿಲಿಸಿಳೆಯಂ
ನಿಲ್ಲಿಸದರು
ನಿಲ್ಲಿಸಿ
ನಿಲ್ಲಿಸಿದ
ನಿಲ್ಲಿಸಿದನು
ನಿಲ್ಲಿಸಿದನೆಂಬುದು
ನಿಲ್ಲಿಸಿದರೆಂದು
ನಿಲ್ಲಿಸುತ್ತಾನೆ
ನಿಲ್ಲಿಸುತ್ತಾರೆ
ನಿಲ್ಲಿಸುತ್ತಾಳೆ
ನಿಲ್ಲುವ
ನಿವನ
ನಿವಾರಣೆಗಾಗಿ
ನಿವಾರಿಸಿಕೊಂಡು
ನಿವಾರಿಸಿದನು
ನಿವಾಸಾಶ್ರಾಯಾಂ
ನಿವಾಸಿಗಳ
ನಿವಾಸಿಗಳಾದ
ನಿವೃತ್ತನಾಗಲಿಚ್ಚಿಸಿ
ನಿವೃತ್ತಿಗೊಳಿಸಿ
ನಿವೇದ್ಯ
ನಿಶಂಕಮಲ್ಲು
ನಿಶಿದಿಗೆಯನ್ನು
ನಿಶಿಧಿಯಂ
ನಿಶ್ಶಂಕ
ನಿಶ್ಶಂಕಪ್ರತಾಪ
ನಿಷ್ಕಂಟಕಂ
ನಿಷ್ಕಂಟಕವನ್ನಾಗಿ
ನಿಷ್ಕಂಟಕವಾದ
ನಿಷ್ಕಾಮೇಶ್ವರ
ನಿಷ್ಕ್ರಿಯತೆಯನ್ನು
ನಿಷ್ಟ
ನಿಷ್ಠನಾಗಿ
ನಿಷ್ಠನಾಗಿದ್ದರೂ
ನಿಷ್ಠರಾಗಿ
ನಿಷ್ಠರಾಗಿದ್ದರೆಂಬುದನ್ನು
ನಿಷ್ಠರಾಗಿದ್ದರೆಂಬುದು
ನಿಷ್ಠಸಾಮಂತರು
ನಿಷ್ಠಾವಂತರಾಗಿದ್ದವರು
ನಿಷ್ಠೆಯಿಂದ
ನಿಷ್ಪತ್ತಿ
ನಿಷ್ಪತ್ತಿಯ
ನಿಷ್ಪತ್ತಿಯನ್ನು
ನಿಷ್ಪನ್ನಗೊಳಿಸಲು
ನಿಷ್ಪ್ರಯೋಜಕವಾದುದೆಂದು
ನಿಸಿತಾಸಿಯ
ನಿಸಿದಿ
ನಿಸಿದಿಗಲ್ಲನ್ನು
ನಿಸಿದಿಗಲ್ಲುಗಳು
ನಿಸಿದಿಗೆ
ನಿಸಿದಿಗೆಯನ್ನು
ನಿಸ್ಸಂಕ
ನಿಸ್ಸಂಕರೆನಿಪ್ಪ
ನಿಸ್ಸೀಮ
ನಿಸ್ಸೀಮಂ
ನೀಡದಿದ್ದರೂ
ನೀಡದೆ
ನೀಡದೇ
ನೀಡಬಹುದು
ನೀಡಲಾಗಿತ್ತು
ನೀಡಲಾಗಿತ್ತೆಂದು
ನೀಡಲಾಗಿದೆ
ನೀಡಲಾಗಿದೆಯೇ
ನೀಡಲಾಗಿದ್ದು
ನೀಡಲಾಗುತ್ತಿತ್ತು
ನೀಡಲಾಯಿತು
ನೀಡಲಾಯಿತೆಂದು
ನೀಡಲಾಯಿತೆಂದೂ
ನೀಡಲು
ನೀಡಲ್ಪಟ್ಟಿದೆ
ನೀಡಲ್ಪಡುತ್ತಿದ್ದ
ನೀಡಿ
ನೀಡಿದ
ನೀಡಿದಂತೆ
ನೀಡಿದನ
ನೀಡಿದನಂತೆ
ನೀಡಿದನು
ನೀಡಿದನೆಂದು
ನೀಡಿದನೆಂಬ
ನೀಡಿದರು
ನೀಡಿದರೆಂದು
ನೀಡಿದಾಗ
ನೀಡಿದೆ
ನೀಡಿದ್ದ
ನೀಡಿದ್ದನು
ನೀಡಿದ್ದನೆಂದು
ನೀಡಿದ್ದನೆಂಬ
ನೀಡಿದ್ದರು
ನೀಡಿದ್ದರೆಂದು
ನೀಡಿದ್ದಾನೆ
ನೀಡಿದ್ದಾನೆಂಬ
ನೀಡಿದ್ದಾರೆ
ನೀಡಿದ್ದು
ನೀಡಿರಬಹುದಾದ್ದನ್ನು
ನೀಡಿರಬಹುದು
ನೀಡಿರಬಹುದೆಂದು
ನೀಡಿರಬಹುದೇ
ನೀಡಿರುತ್ತಾನೆ
ನೀಡಿರುತ್ತಾರೆ
ನೀಡಿರುವ
ನೀಡಿರುವಂತೆ
ನೀಡಿರುವಂತೆಯೇ
ನೀಡಿರುವುದರ
ನೀಡಿರುವುದರಿಂದ
ನೀಡಿರುವುದಿಲ್ಲ
ನೀಡಿರುವುದು
ನೀಡಿಲ್ಲ
ನೀಡಿವೆ
ನೀಡುತ್ತದೆ
ನೀಡುತ್ತವೆ
ನೀಡುತ್ತಾ
ನೀಡುತ್ತಾನೆ
ನೀಡುತ್ತಾನೆಂದು
ನೀಡುತ್ತಾರೆ
ನೀಡುತ್ತಾಳೆ
ನೀಡುತ್ತಿದ್ದ
ನೀಡುತ್ತಿದ್ದರೆಂದು
ನೀಡುವ
ನೀಡುವಂತೆ
ನೀಡುವಾಗ
ನೀಡುವುದನ್ನು
ನೀಡುವುದರ
ನೀಡುವುದು
ನೀತಿ
ನೀತಿಗೆ
ನೀತಿಮಹಾರಾಜ
ನೀತಿಮಾರ್ಗ
ನೀತಿಮಾರ್ಗಎರೆಗಂಗನೆಂದು
ನೀತಿಮಾರ್ಗನ
ನೀತಿಮಾರ್ಗನನ್ನು
ನೀತಿಮಾರ್ಗನಿಗೆ
ನೀತಿಮಾರ್ಗನು
ನೀತಿಮಾರ್ಗನೆಂಬ
ನೀತಿಮಾರ್ಗನೇ
ನೀತಿಮಾರ್ಗ್ಗ
ನೀತಿವಾಕ್ಯ
ನೀತಿವಿದರೂ
ನೀತಿವಿಶಾರದಃ
ನೀತಿಶಾಸ್ತ್ರ
ನೀತಿಶಾಸ್ತ್ರಸ್ಯ
ನೀನೂ
ನೀರಗುಂದ
ನೀರಾವರಿ
ನೀರಾವರಿಗೆ
ನೀರಾವರಿಯ
ನೀರಾವರಿಯಿಂದ
ನೀರಿಗೆ
ನೀರಿನ
ನೀರಿಲ್ಲದ
ನೀರು
ನೀರುಣಿಸುವ
ನೀರುಹರಿಸಲು
ನೀರ್ಗುಂದ
ನೀರ್ಗುಂದದ
ನೀರ್ಗ್ಗುಂದ
ನೀರ್ಗ್ಗುನ್ದೆಳಾ
ನೀರ್ನೆಲವನ್ನು
ನೀಲಕಂಠ
ನೀಲಕಂಠನಹಳ್ಳಿಗಳನ್ನು
ನೀಲಕಂಠಾಚಾರ್ಯನ
ನೀಲಗಿರಿ
ನೀಲಗಿರಿಯ
ನೀಲಗಿರಿಸಾಧಾರ
ನೀಲಚೊಟ್ಟ
ನೀಲಮಸೂದ
ನೀಲಯ್ಯ
ನೀಲಸಮುದ್ರ
ನೀಲಾಚಲನೀಲಗಿರಿಯನ್ನು
ನೀಲಾಚಲವನ್ನು
ನೀಳಾಚಳಮಂ
ನೀಳಾದ್ರಿಯಂ
ನುಂಗಿ
ನುಂಗುವ
ನುಕರಾಜನಿಗೆ
ನುಗುನಾಡು
ನುಗ್ಗಿ
ನುಗ್ಗಿತು
ನುಗ್ಗಿತೆಂದೂ
ನುಗ್ಗಿದ
ನುಗ್ಗಿದನು
ನುಗ್ಗಿಲೂರು
ನುಗ್ಗಿಹಳ್ಳಿಯ
ನುಡಿದಂನ್ತೆ
ನುಡಿದಂನ್ತೆಗಂಡನುಂ
ನುಡಿದುದೇ
ನುಡಿದುಮತ್ತೆನ್ನನುಂ
ನುತಬಲ್ಲಾಳಭೂಪನ
ನುಳಮ್ಬನುಂ
ನೂಕಿದನು
ನೂತನ
ನೂತನವಾಗಿ
ನೂತ್ನರತ್ನಮುಂ
ನೂತ್ನರತ್ನಮುಮಂ
ನೂರನ್ನು
ನೂರಾರು
ನೂರು
ನೂರ್ಮಡಿ
ನೂರ್ಮ್ಮಡಿ
ನೂಱನ್ನು
ನೂಳುಲು
ನೃಪಂ
ನೃಪತುಂಗನ
ನೃಪತುಂಗನು
ನೃಪನ
ನೃಪನಿಂದೇವಣ್ನಿಪೆಂ
ನೃಪಭೂಪನ
ನೃಪಭೂಪನೆಂದು
ನೃಪರಾಜ್ಯವಾರ್ದ್ಧಿಸಂವರ್ದ್ಧನ
ನೃಪರಿಂದೊಡಗೂಡಿದ
ನೃಸಿಂಹಭೂಪನೆಳೆಯಂ
ನೃಸಿಂಹಭೂಪನೆಳೆಯಂದೋಸ್ಥಂಭದೊಲು
ನೆಂದು
ನೆಂಬುವವನು
ನೆಗಳ್ದ
ನೆಗಳ್ದಂ
ನೆಗಳ್ದತೆಂಕಣರಾಯನೆನಲ್ಕೆನಿಪ್ಪ
ನೆಗಳ್ದರೊಳ್
ನೆಗಳ್ದಾಧಿರಾಜಪದವಿಗೆ
ನೆಗೆದು
ನೆಚ್ಚಿಕೊಂಡಿದ್ದನು
ನೆಟ್ಟಕಲ್ಲು
ನೆಟ್ಟನು
ನೆಟ್ಟೂರು
ನೆಡಿಸಿದರೆಂದು
ನೆಡಿಸುತ್ತಾನೆ
ನೆಡುಮಾಮಿಟಿ
ನೆತ್ತರುಗೊಡಗೆಯಾಗಿ
ನೆತ್ತಿ
ನೆನಪನ್ನು
ನೆನಪಿಗೆ
ನೆನೆಯಬಹುದು
ನೆನೆವ
ನೆಮ್ಮೆದಿಯನ್ನು
ನೆಯ
ನೆಯಾಮ
ನೆರನೆರಪಿ
ನೆರವನ್ನು
ನೆರವಾಗಿ
ನೆರವಾಗಿರಬಹುದು
ನೆರವಾಗಿರೆ
ನೆರವಾಗುತ್ತವೆ
ನೆರವಾಗುತ್ತಿದ್ದನು
ನೆರವಾಗುವುದಲ್ಲದೆ
ನೆರವಾದನು
ನೆರವಿಗೆ
ನೆರವಿನಿಂದ
ನೆರವು
ನೆರವೇರದೇ
ನೆರವೇರಿಸಿ
ನೆರವೇರಿಸಿದ
ನೆರೆದಿದ್ದರೆಂದು
ನೆರೆದು
ನೆರೆಯ
ನೆರೆಯೆ
ನೆಲಂ
ನೆಲನಂ
ನೆಲಮಂಗಲ
ನೆಲಮನೆ
ನೆಲಸಮವಾಗಿದೆ
ನೆಲಾಪುರ
ನೆಲುಮನೆ
ನೆಲೆ
ನೆಲೆಗಳನ್ನು
ನೆಲೆಗಳು
ನೆಲೆನಿಂತರು
ನೆಲೆನಿಂತರೆಂಬುದೂ
ನೆಲೆಬೀಡಂ
ನೆಲೆಬೀಡನ್ನು
ನೆಲೆಬೀಡಾಗಿ
ನೆಲೆಬೀಡಾಗಿತ್ತು
ನೆಲೆಬೀಡಾಗಿರಬಹುದು
ನೆಲೆಬೀಡಿಗೆ
ನೆಲೆಬೀಡಿನಲ್ಲಿ
ನೆಲೆಬೀಡಿನಲ್ಲಿದ್ದನೆಂದು
ನೆಲೆಬೀಡಿನಿಂದ
ನೆಲೆಬೀಡುಗಳನ್ನು
ನೆಲೆಬೀಡುಗಳೂ
ನೆಲೆಯಾಗಿ
ನೆಲೆಯಾಗಿತ್ತೆಂದು
ನೆಲೆಯಾಗಿದೆ
ನೆಲೆಯಾಗಿದ್ದಿತು
ನೆಲೆವೀಡಾಗಿತ್ತೆಂಬುದನ್ನು
ನೆಲೆವೀಡಾಗಿದ್ದ
ನೆಲೆವೀಡಿನಲ್ಲಿ
ನೆಲೆವೀಡಿನಲ್ಲಿದ್ದಾಗ
ನೆಲೆವೀಡಿನಿಂದ
ನೆಲೆವೀಡಿನೊಳ್ಸಮುತ್ತುಂಗ
ನೆಲೆವೀಡುಗಳಲ್ಲಿ
ನೆಲೆವೃತ್ತಿಗಳ
ನೆಲೆಸಿ
ನೆಲೆಸಿದ
ನೆಲೆಸಿದರು
ನೆಲೆಸಿದವರಿಂದ
ನೆಲೆಸಿದ್ದನು
ನೆಲೆಸಿದ್ದನೆಂದು
ನೆಲೆಸಿದ್ದರೆಂದು
ನೆಲೆಸಿದ್ದರೆಂಬ
ನೆಲೆಸಿದ್ದರೆಂಬುದು
ನೆಲೆಸಿರುವ
ನೆಲೆಸಿರುವಂತೆ
ನೆಲ್ಲ
ನೆಲ್ಲಕೂಳಣ
ನೆಲ್ಲಕೂಳಣವಾಗಿ
ನೆಲ್ಲಕೂೞಣಲಾಗೊದೆನ್ದು
ನೆಲ್ಲು
ನೇ
ನೇತೃತ್ವ
ನೇತೃತ್ವದಲ್ಲಿ
ನೇತೃತ್ವವನ್ನು
ನೇತ್ರ
ನೇತ್ರಃ
ನೇತ್ರನೆಂದುಮೀ
ನೇಮ
ನೇಮಕ
ನೇಮಕಮಾಡಲಾಗುತ್ತಿತ್ತು
ನೇಮಕಮಾಡಿದನು
ನೇಮಕವಾಗಿದ್ದರು
ನೇಮಕವಾದವರೆಂದು
ನೇಮಕಾತಿ
ನೇಮದಂಡೇಶ
ನೇಮದಂಡೇಸದಿಕ್ಕುಂ
ನೇಮವೆರ್ಗಡೆ
ನೇಮಹೆರ್ಗಡೆ
ನೇಮಿಸಲಾಗುತ್ತಿತ್ತು
ನೇಮಿಸಲಾಗುತ್ತಿತ್ತೆಂದು
ನೇಮಿಸಲಾಯಿತು
ನೇಮಿಸಲ್ಪಟ್ಟ
ನೇಮಿಸಲ್ಪಟ್ಟನು
ನೇಮಿಸಲ್ಪಡುತ್ತಿದ್ದ
ನೇಮಿಸಿ
ನೇಮಿಸಿದ
ನೇಮಿಸಿದನು
ನೇಮಿಸಿದನೆಂದು
ನೇಮಿಸಿದನೇ
ನೇಮಿಸಿದರು
ನೇಮಿಸಿದುದಂತೂ
ನೇಮಿಸಿದ್ದನು
ನೇಮಿಸಿದ್ದರೂ
ನೇಮಿಸಿದ್ದಾನೆ
ನೇಮಿಸಿರಬೇಕು
ನೇಮಿಸುತ್ತಾನೆ
ನೇಮಿಸುತ್ತಿದ್ದನು
ನೇಮಿಸುತ್ತಿದ್ದರು
ನೇಮಿಸುತ್ತಿದ್ದರೆಂದು
ನೇಮಿಸುತ್ತಿದ್ದುರು
ನೇಮಿಸುವ
ನೇಮೀಶ್ವರ
ನೇಯ್ಗೆಗೆ
ನೇರ
ನೇರಲಕಟ್ಟೆ
ನೇರಲಕೆರೆಯ
ನೇರಲಿಗೆ
ನೇರವಾಗಿ
ನೇಶ
ನೈಜವಾದ
ನೈಜಾಮನಿಗೆ
ನೈಜಾಮ್
ನೈವೇದ್ಯ
ನೈವೇದ್ಯಕ್ಕಾಗಿ
ನೈವೇದ್ಯಕ್ಕೆ
ನೈವೇದ್ಯದ
ನೊಣಂಬವಾಡಿ
ನೊಬೆಯ
ನೊಳಂಬ
ನೊಳಂಬಕುಲಾಂತಕದೇವನ
ನೊಳಂಬನ
ನೊಳಂಬನಿಗೆ
ನೊಳಂಬನು
ನೊಳಂಬರ
ನೊಳಂಬರನ್ನು
ನೊಳಂಬರಸರಾಗಿರಬಹುದು
ನೊಳಂಬರಾಜನ
ನೊಳಂಬರಾಜಾನ್ವಯದ
ನೊಳಂಬರಿಂದ
ನೊಳಂಬರು
ನೊಳಂಬರೂ
ನೊಳಂಬರೊಡನೆ
ನೊಳಂಬಳಿಗೆ
ನೊಳಂಬವಾಡಿ
ನೊಳಂಬವಾಡಿಯನ್ನು
ನೊಳಂಬಾದಿರಾಜ
ನೊಳಂಬಾದಿರಾಜನು
ನೊಳಂಬಾದಿರಾಜನುಕ್ರಿಶ
ನೊಳಂಬಾಧಿರಾಜನು
ನೊಳಂಬಾಧಿರಾಜರನ್ನು
ನೊಳಂಬಿ
ನೊಳಬಂನೂ
ನೊೞಂಬ
ನೋಂತು
ನೋಡಬಹುದು
ನೋಡಿ
ನೋಡಿಕೊಂಡು
ನೋಡಿಕೊಳ್ಳುತ್ತಿದ್ದ
ನೋಡಿಕೊಳ್ಳುತ್ತಿದ್ದರು
ನೋಡಿಕೊಳ್ಳುತ್ತಿದ್ದರೆಂದು
ನೋಡಿಕೊಳ್ಳುತ್ತಿದ್ದರೆಂಬುದು
ನೋಡಿಕೊಳ್ಳುತ್ತಿದ್ದವನೇ
ನೋಡಿಕೊಳ್ಳುತ್ತಿದ್ದವು
ನೋಡಿಕೊಳ್ಳುವವರು
ನೋಡಿದ
ನೋಡಿದರೆ
ನೋಡಿದಾಗ
ನೋಡೆ
ನೋವಿನ
ನೌಕಾಸೇನೆಯ
ನ್ನು
ನ್ಯಾಯತೀರ್ಮಾನ
ನ್ಯಾಯತೀರ್ಮಾನವನ್ನು
ನ್ಯೂನಿಸ್ರು
ಪ
ಪಂಕ್ತಿಯ
ಪಂಗಡಕ್ಕೆ
ಪಂಚಂಪಲ್ಲಿ
ಪಂಚಕಂಬಿಮೇಳ
ಪಂಚಗೊಂಡ
ಪಂಚದ
ಪಂಚನೇತ್ರಧ್ವಜ
ಪಂಚಪ್ರಧಾನ
ಪಂಚಪ್ರಧಾನರಲ್ಲಿ
ಪಂಚಬಸದಿಯೊಳಗೆ
ಪಂಚಬಾಣಕವಿಯ
ಪಂಚಮಠಗಳಿಗೆ
ಪಂಚಮಠಸ್ಥಾನಪತಿ
ಪಂಚಮಠಸ್ಥಾನಪತಿಗಳ
ಪಂಚಮಠಸ್ಥಾನಪತಿಗಳು
ಪಂಚಮಹಾಪ್ರಧಾನರ
ಪಂಚಮಹಾಪ್ರಧಾನರಲ್ಲಿ
ಪಂಚಮಹಾಪ್ರಧಾನರೆಂದರೆ
ಪಂಚಮಹಾಶಬ್ದ
ಪಂಚಮಿ
ಪಂಚಮುಖವಿಭಾಡ
ಪಂಚರಿಗೆ
ಪಂಚರು
ಪಂಚಲಿಂಗ
ಪಂಚಲಿಂಗಗಳಿಗೆ
ಪಂಚಲಿಂಗೇಶ್ವರ
ಪಂಚವನ್
ಪಂಚವನ್ಮಹಾರಾಯನೆಂಬ
ಪಂಚವಮಾರಾಯನಾದ
ಪಂಚಾನನಂ
ಪಂಚಾರತಿ
ಪಂಚಿಕೇಶ್ವರ
ಪಂಜದ
ಪಂಡಿತನ
ಪಂಡಿತನೆಂದು
ಪಂಡಿತರ
ಪಂಡಿತರಿಗೂ
ಪಂಡಿತರಿಗೆ
ಪಂಡಿತರು
ಪಂಡಿತವರ್ಯರು
ಪಂಡಿತಹಳ್ಳಿ
ಪಂತಳೆದಂ
ಪಂದಲದೇವ
ಪಂದಲದೇವನು
ಪಂನಗವೈನತೇಯ
ಪಂನಾಯವನ್ನು
ಪಂಪನ
ಪಂಪಭಾರತದಲ್ಲಿ
ಪಂಪರಾಜ
ಪಕ್ಕ
ಪಕ್ಕದ
ಪಕ್ಕದಲ್ಲಿ
ಪಕ್ಕದಲ್ಲಿದ್ದ
ಪಕ್ಕದಲ್ಲಿಯೇ
ಪಕ್ಕದಲ್ಲಿರುವ
ಪಕ್ಕದಲ್ಲೇ
ಪಕ್ಷ
ಪಕ್ಷದವರನ್ನು
ಪಕ್ಷಪಾತಿಗಳಾಗಿದ್ದರು
ಪಕ್ಷಪಾತಿಗಳಾಗಿದ್ದರೂ
ಪಕ್ಷವಹಿಸಿ
ಪಕ್ಷಾರ್ಧದಲ್ಲಿ
ಪಗೋಡಾ
ಪಟೇಲ
ಪಟೇಲ್
ಪಟ್ಟ
ಪಟ್ಟಂಗಟ್ಟಿದ
ಪಟ್ಟಂಗಟ್ಟಿದನೆಂದು
ಪಟ್ಟಕಟ್ಟಿದ
ಪಟ್ಟಕಟ್ಟಿದನು
ಪಟ್ಟಕಟ್ಟಿದರು
ಪಟ್ಟಕಟ್ಟಿದರೆಂದು
ಪಟ್ಟಕಟ್ಟಿಸಿದ
ಪಟ್ಟಕ್ಕೆ
ಪಟ್ಟಕ್ಕೇರಿದರು
ಪಟ್ಟಣ
ಪಟ್ಟಣಕ್ಕೆ
ಪಟ್ಟಣಗಳ
ಪಟ್ಟಣಗಳನ್ನು
ಪಟ್ಟಣಗಳಲ್ಲಿ
ಪಟ್ಟಣಗಳಿಗೆ
ಪಟ್ಟಣಗಳು
ಪಟ್ಟಣಗೆರೆ
ಪಟ್ಟಣದ
ಪಟ್ಟಣದಲ್ಲಿ
ಪಟ್ಟಣದಿಂದ
ಪಟ್ಟಣಪುರ
ಪಟ್ಟಣವನ್ನಾಗಿ
ಪಟ್ಟಣವನ್ನು
ಪಟ್ಟಣವಾಗಿದ್ದಿರಬಹುದೆಂದು
ಪಟ್ಟಣವು
ಪಟ್ಟಣವೆಂದು
ಪಟ್ಟಣಸೆಟ್ಟಿ
ಪಟ್ಟಣಸೆಟ್ಟಿಗಳಿಗೆ
ಪಟ್ಟಣಸ್ವಾಮಿ
ಪಟ್ಟಣಸ್ವಾಮಿಗಳ
ಪಟ್ಟಣಸ್ವಾಮಿಗಳಿಗೆ
ಪಟ್ಟಣಸ್ವಾಮಿಯಾಗಿದ್ದ
ಪಟ್ಟದರಸಿ
ಪಟ್ಟದಾನೆಯಂತೆ
ಪಟ್ಟಬಂಧ
ಪಟ್ಟಮಂ
ಪಟ್ಟಮಹಾದೇವಿ
ಪಟ್ಟಯಙ್ಗನ್
ಪಟ್ಟಯಾಂಗನಿಗೆ
ಪಟ್ಟಯೆಲೆಯ
ಪಟ್ಟವನು
ಪಟ್ಟವನ್ನು
ಪಟ್ಟವರ್ಧನರ
ಪಟ್ಟವಾಗಿ
ಪಟ್ಟವಾಯಿತು
ಪಟ್ಟವಾಯಿತೆಂದೂ
ಪಟ್ಟವೂ
ಪಟ್ಟವೇರಿದನು
ಪಟ್ಟಸಾಲೆಯನ್ನು
ಪಟ್ಟಸಾಹಣಿ
ಪಟ್ಟಸಾಹಣಿಯಾಗಿ
ಪಟ್ಟಸಾಹಣಿಯು
ಪಟ್ಟಸೋಮನಹಳ್ಳಿ
ಪಟ್ಟಾಭಿಷಿಕ್ತನಾದನು
ಪಟ್ಟಾಭಿಷಿಕ್ತನಾದನೆಂದು
ಪಟ್ಟಾಭಿಷೇಕ
ಪಟ್ಟಾಭಿಷೇಕವನ್ನು
ಪಟ್ಟಾಭಿಷೇಕವನ್ನೇ
ಪಟ್ಟಾಭಿಷೇಕವಾದಕೂಡಲೇ
ಪಟ್ಟಿದ್ದಾರೆ
ಪಟ್ಟಿದ್ದಾರೆಂದು
ಪಟ್ಟಿದ್ದಾರೆೆ
ಪಟ್ಟಿಮಾಡಿದ್ದಾರೆ
ಪಟ್ಟಿಮಾಡುತ್ತದೆ
ಪಟ್ಟಿಯನ್ನು
ಪಟ್ಟಿಯನ್ನೂ
ಪಟ್ಟಿಯಲ್ಲಿ
ಪಟ್ಟಿರುವ
ಪಟ್ಟಿರುವುದು
ಪಟ್ಟೆ
ಪಟ್ಟೆಯಾಂಗ
ಪಟ್ಟೆಯಾಂಗನೆಂಬುವವನಿಗೆ
ಪಟ್ಟೆಯೆಲೆಯ
ಪಠಿಸಲು
ಪಠಿಸಿ
ಪಡಿಯ
ಪಡಿಯಾರ್ದ್ದಕ್ಷಿಣಚಕ್ರವರ್ತ್ತಿ
ಪಡಿಯಿಪ್ಪಂತೆ
ಪಡಿಸಿದ್ದು
ಪಡಿಹಾರ
ಪಡುತ್ತಾರೆ
ಪಡುವ
ಪಡುವಂತಹವರು
ಪಡುವಣ
ಪಡುವನಾಡ
ಪಡುವಲಪಟ್ಟಣದ
ಪಡುವಲು
ಪಡುವಲ್
ಪಡುವೆಣ್ಣೆ
ಪಡೆ
ಪಡೆಗಳ
ಪಡೆಗಳಿಗೆ
ಪಡೆದ
ಪಡೆದಂ
ಪಡೆದಂತೆ
ಪಡೆದನಂತೆ
ಪಡೆದನು
ಪಡೆದನೆಂದು
ಪಡೆದರು
ಪಡೆದವನು
ಪಡೆದವರಾಗಿದ್ದರು
ಪಡೆದವರಾಗಿರಬಹುದು
ಪಡೆದವರಾಗಿರುತ್ತಿದ್ದರು
ಪಡೆದವರು
ಪಡೆದವರೂ
ಪಡೆದಿತ್ತು
ಪಡೆದಿದ್ದ
ಪಡೆದಿದ್ದನು
ಪಡೆದಿದ್ದರು
ಪಡೆದಿದ್ದರೆಂದು
ಪಡೆದಿದ್ದವರು
ಪಡೆದಿದ್ದಾನೆ
ಪಡೆದಿದ್ದಾನೆಂಬುದು
ಪಡೆದಿದ್ದು
ಪಡೆದಿರಬಹುದು
ಪಡೆದಿರುವುದನ್ನು
ಪಡೆದು
ಪಡೆದುಕೊಂಡ
ಪಡೆದುಕೊಂಡನು
ಪಡೆದುಕೊಂಡಿದ್ದರು
ಪಡೆದುಕೊಂಡು
ಪಡೆದುಕೊಳ್ಳುತ್ತಾರೆ
ಪಡೆದೇ
ಪಡೆಮೆಚ್ಚೆಗಂಡ
ಪಡೆಯ
ಪಡೆಯಂ
ಪಡೆಯದೇ
ಪಡೆಯನ್ನು
ಪಡೆಯಲಾಗಿದೆ
ಪಡೆಯಲಿಲ್ಲ
ಪಡೆಯಲು
ಪಡೆಯವರಾಗಿದ್ದು
ಪಡೆಯುತ್ತಾನೆ
ಪಡೆಯುತ್ತಿದ್ದುದರಿಂದ
ಪಡೆಯುವ
ಪಡೆವಳ
ಪಡೆವಳರ
ಪಡೈಕ್ಕಣಕ್ಕನ್
ಪಣವನ್ನು
ಪಣ್ಯಾಗುಣಂ
ಪತನ
ಪತನಗೊಂಡನಂತರ
ಪತನದ
ಪತನಾ
ಪತಿ
ಪತಿಭಕ್ತಂ
ಪತಿಹಿತದೆಯೊಳು
ಪತ್ತಿಯ
ಪತ್ತಿಸೆ
ಪತ್ತೆ
ಪತ್ತೆಯಾಗಿರುವ
ಪತ್ತೆಹಚ್ಚಿ
ಪತ್ತೊಂದಿ
ಪತ್ನಿ
ಪತ್ನಿಯ
ಪತ್ನಿಯನ್ನು
ಪತ್ನಿಯರ
ಪತ್ನಿಯರಾದ
ಪತ್ನಿಯರಿಗೂ
ಪತ್ನಿಯರು
ಪತ್ನಿಯಾದ
ಪತ್ನಿಯೊರಡನೆ
ಪತ್ಮಾವಸುಂಧರಾಭ್ಯಾಮಾಕಲ್ಪಂ
ಪತ್ರ
ಪತ್ರಗಳನ್ನು
ಪತ್ರವನ್ನು
ಪದ
ಪದಕವನ್ನು
ಪದಗಳನ್ನು
ಪದಗಳು
ಪದಗಳೂ
ಪದದ
ಪದಪಿಂ
ಪದವನ್ನು
ಪದವನ್ನೈದಿದನೆಂದೂ
ಪದವಾಗಿರಲಿಲ್ಲ
ಪದವಿ
ಪದವಿಗಳನ್ನು
ಪದವಿಗಳನ್ನೂ
ಪದವಿಗಳು
ಪದವಿಗೆ
ಪದವಿಗೇರುತ್ತಿದ್ದರೆಂದು
ಪದವಿಯ
ಪದವಿಯನ್ನು
ಪದವಿಯನ್ನೂ
ಪದವಿಯಲ್ಲಿ
ಪದವಿಯಾಗಿತ್ತು
ಪದವಿಯಿಂದ
ಪದವಿಯು
ಪದವಿಯೆಂದು
ಪದವೀ
ಪದವು
ಪದಾಕ್ರಾಂತವಾದ
ಪದಾತಿ
ಪದಾದ್ವಿಪ್ರಗಣೇ
ಪದಾಬ್ಜಂಗಳವರ್ಗ್ಗೆ
ಪದಾಬ್ಜಗಳು
ಪದಾರಾಧಕನುಂ
ಪದಿನಾಲ್ಕು
ಪದಿನಾಲ್ಕುನಾಡಿನ
ಪದುಮಂಣನ
ಪದುಮಣ್ಣ
ಪದುಮಣ್ಣನ
ಪದುಮಣ್ಣನವರ
ಪದುಮಣ್ಣನು
ಪದುಮಣ್ಣಸೆಟ್ಟರ
ಪದುಮನಾಭಪುರದ
ಪದುಮಲದೇವಿ
ಪದುಮಲದೇವಿಯರ
ಪದುಳಂ
ಪದುಳದಿಂ
ಪದೇ
ಪದೋನ್ನತಿ
ಪದ್ಧತಿ
ಪದ್ಧತಿಗೆ
ಪದ್ಧತಿಯ
ಪದ್ಧತಿಯನ್ನು
ಪದ್ಧತಿಯಿಂದ
ಪದ್ಧತಿಯು
ಪದ್ಮನಾಭನ
ಪದ್ಮಪ್ರಭನೆಂದು
ಪದ್ಮಲದೇವಿ
ಪದ್ಮಾವತಿದೇವೀಲಬ್ಧವರಪ್ರಸಾದಂ
ಪದ್ಮೋಪಜೀವಿಯಾಗಿ
ಪದ್ಯ
ಪದ್ಯಗಳ
ಪದ್ಯಗಳನ್ನು
ಪದ್ಯಗಳನ್ನೊಳಗೊಂಡ
ಪದ್ಯಗಳಲ್ಲಿ
ಪದ್ಯಗಳಿವೆ
ಪದ್ಯದ
ಪದ್ಯದಲ್ಲಿ
ಪದ್ಯರೂಪದಲ್ಲಿ
ಪದ್ಯವೂ
ಪನಃ
ಪನರ್
ಪನೆಕೊಳ
ಪನ್ನಗವೈನತೇಯನೆನಿಸಿದ್ದ
ಪನ್ನಗಶಾಯೀ
ಪನ್ನರ್ವ್ವರ್ಗ್ಗಂ
ಪನ್ನಾಯ
ಪನ್ನಾಯವನ್ನು
ಪನ್ನಿರ್ಚ್ಛಾಸಿರ
ಪನ್ನಿರ್ಛಾಸಿರ
ಪನ್ನಿರ್ವ್ವರ್ಗೆ
ಪನ್ನೆರಡಕ್ಕೆ
ಪನ್ನೆರಡನ್ನು
ಪನ್ನೆರಡರ
ಪನ್ನೆರಡರಲ್ಲಿದ್ದ
ಪನ್ನೆರಡು
ಪಯಣಬೆಳೆಸಿ
ಪಯೋಜಭಾನು
ಪಯೋಧಿ
ಪರ
ಪರಂಪರಾಗತ
ಪರಂಪರಾನುಗತವಾಗಿತ್ತು
ಪರಂಪರೆ
ಪರಂಪರೆಗೆ
ಪರಂಪರೆಯ
ಪರಂಪರೆಯವರು
ಪರಂಪರೆಯಿಂದ
ಪರಂಪರೆಯೂ
ಪರಕಪಾಂಡ್ಯರು
ಪರಕೇಸರಿ
ಪರದರಕುಲದ
ಪರದಾರಸಹೋದರಃ
ಪರನಾರಿ
ಪರನಾರೀದೂರನುಂ
ಪರನಾರೀಪುತ್ರನುಂ
ಪರನಾರೀಸೋದರ
ಪರಬಲಭೀಮ
ಪರಬಳಕಕ್ಷ
ಪರಬಳಕೃತಾಂತ
ಪರಬಳಜಳಧಿಬಡಬಾನಳಂ
ಪರಬಳಭಕ್ಷಕ
ಪರಭಾರೆ
ಪರಮ
ಪರಮಗೂಳ
ಪರಮಗೂಳನ
ಪರಮಗೂಳನು
ಪರಮಗೂಳನೂ
ಪರಮಗೂಳನೆಂದು
ಪರಮಗೂಳನೆಂದೂ
ಪರಮಗೂಳಪತ್ನಿ
ಪರಮಜೈನರೂ
ಪರಮತಸಹಿಷ್ಣತೆ
ಪರಮಬ್ಬೆ
ಪರಮಬ್ಬೆಯು
ಪರಮಭಾಗವತ
ಪರಮವಿಶ್ವಾಸಿ
ಪರಮಾರ
ಪರಮಾರರ
ಪರಮೇಶ್ವರ
ಪರಮೇಶ್ವರವರ್ಮನಿರಬಹುದು
ಪರಮೋಚ್ಛ
ಪರರಾಜಭಯಂಕರ
ಪರರಾಜಭಯಂಕರಃ
ಪರರಾಯ
ಪರರಾಯಭಯಂಕರಃ
ಪರರಾಷ್ಟ್ರ
ಪರವಾಗಿ
ಪರವಾದಿಮಲ್ಲ
ಪರವೆಂಡಿರಣ್ಣ
ಪರವೆಣ್ಡಿರಣ್ನನೀಸರಯ್ಯ
ಪರಸ್ಪರ
ಪರಾಂತಕನ
ಪರಾಂತಕನಿಗೆ
ಪರಾಕು
ಪರಾಕ್ರಮ
ಪರಾಕ್ರಮಕ್ಕೆ
ಪರಾಕ್ರಮಗಳಿಂದ
ಪರಾಕ್ರಮದಿಂದ
ಪರಾಕ್ರಮವನ್ನು
ಪರಾಕ್ರಮಿಯಾದ
ಪರಾಜಿತಗೊಂಡ
ಪರಾಧೀನತೆಯಿಂದ
ಪರಾಭವಗೊಳಿಸಿ
ಪರಾರಿಯಾದರು
ಪರಿಗಣಿಸಿದರೆ
ಪರಿಗಣಿಸಿರುತ್ತಾರೆ
ಪರಿಗೆರೆ
ಪರಿಚಯಾತ್ಮಕ
ಪರಿಜನಪರಿವೃತನಾಗಿ
ಪರಿಣತನೆಂದು
ಪರಿಣಾಮವಾಗಿ
ಪರಿತಾಪಹತ
ಪರಿಪಾಲಿತವಾದ
ಪರಿಪಾಲಿಸುವಂತೆ
ಪರಿವಂತೆ
ಪರಿವರ್ತನೆ
ಪರಿವರ್ತನೆಯಾಗಿದೆಯೆಂದು
ಪರಿವರ್ತಿತವಾಗಿದೆ
ಪರಿವರ್ತಿತವಾಯಿತೆಂಬುದು
ಪರಿವರ್ತಿಸಲಾಯಿತೆಂದು
ಪರಿವಾರ
ಪರಿವೃಢಃ
ಪರಿವೇಷ್ಟಿತೇ
ಪರಿಶೀಲನೆ
ಪರಿಶೀಲನೆಗೆ
ಪರಿಶೀಲನೆಯಿಂದ
ಪರಿಶೀಲಿಸಬಹುದು
ಪರಿಶೀಲಿಸಬೇಕಾಗುತ್ತದೆ
ಪರಿಶೀಲಿಸಲಾಗಿದೆ
ಪರಿಶೀಲಿಸಿದಾಗ
ಪರಿಶೀಲಿಸಿದ್ದು
ಪರಿಷತ್ತಿನಲ್ಲಿ
ಪರಿಷತ್ತಿನವರು
ಪರಿಷ್ಕರಿಸಿ
ಪರಿಷ್ಕೃತ
ಪರಿಷ್ಕೃತಗೊಂಡ
ಪರಿಸರಕ್ಕೆ
ಪರಿಸರದಲ್ಲಿರುವ
ಪರಿಸರದ್ದೆಂದು
ಪರಿಸೇವ್ಯಮಾನಃ
ಪರಿಸ್ಥಿತಿ
ಪರಿಸ್ಥಿತಿಗಳ
ಪರಿಸ್ಥಿತಿಯನ್ನು
ಪರುವಿ
ಪರುಷೆ
ಪರುಷೆಯ
ಪರೋಕ್ಷ
ಪರೋಕ್ಷವಿನಯವಾಗಿ
ಪರ್ಯಾಯ
ಪರ್ಯಾಯದಲು
ಪರ್ವಿಯಲ್ಲಿ
ಪರ್ಷಿಯನ್
ಪಲಕ್ಕಿ
ಪಲಗೈಯಾನುಮ್
ಪಲರ್ಪ್ಪೊಗೞ್ದೆನದಟಿಂ
ಪಲಸಿಗೆ
ಪಲಾಯನ
ಪಲಾಯನಂ
ಪಲಾಯನಮಾಡಿ
ಪಲ್ಲಕ್ಕಿ
ಪಲ್ಲಪಂಡಿತ
ಪಲ್ಲಪಂಡಿತರಿಗೆ
ಪಲ್ಲಪೆರಿಯೂರಿನಲ್ಲಿ
ಪಲ್ಲವ
ಪಲ್ಲವತಟಾಕವೆಂಬ
ಪಲ್ಲವತಟಾಕಾ
ಪಲ್ಲವತ್ರಿಣೇತ್ರ
ಪಲ್ಲವನ್ವಾಯ
ಪಲ್ಲವಮಲ್ಲ
ಪಲ್ಲವರ
ಪಲ್ಲವರನ್ನೇ
ಪಲ್ಲವರಾಯ
ಪಲ್ಲವರಾಯನನ್ನು
ಪಲ್ಲವರಾಯನೆಂಬ
ಪಲ್ಲವರು
ಪಲ್ಲವರೆಂದು
ಪಲ್ಲವರೊಡನೆ
ಪಲ್ಲವವಂಶದವರೆಂದು
ಪಲ್ಲವಾದಿತ್ಯ
ಪಲ್ಲವಾಧಿರಾಜ
ಪಲ್ಲವಾಧಿರಾಜನ
ಪಲ್ಲವಾಧಿರಾಜನು
ಪಲ್ಲವಾಧಿರಾಜನೂ
ಪಲ್ಲವಾಧಿರಾಜರ
ಪಲ್ಲವೆಂದ್ರನನ್ನು
ಪಳಗಿದವರೆಂಬ
ಪಳಗಿಸಿ
ಪಳಗಿಸುವವರ
ಪಳಗಿಸುವವರಿಗೆ
ಪಳಗಿಸುವುದರಲ್ಲಿ
ಪಳಿಯುಲನು
ಪಳಿಯುಳನ
ಪವಿತ್ರ
ಪವಿತ್ರಜಲವನ್ನು
ಪವಿತ್ರಾರೋಹಣ
ಪವಿತ್ರೀಕೃತೋತ್ತಮಾಂಗನುಂ
ಪವಿತ್ರೀಕ್ರಿತೋತ್ತಮಾಂಗ
ಪಶ್ಚಿಮ
ಪಶ್ಚಿಮಕ್ಕೂ
ಪಶ್ಚಿಮಕ್ಕೆ
ಪಶ್ಚಿಮದ
ಪಶ್ಚಿಮಭಾಗ
ಪಶ್ಚಿಮರಂಗ
ಪಶ್ಚಿಮರಂಗರಾಜನಗರೀ
ಪಶ್ಚಿಮರಂಗೇ
ಪಶ್ಚಿಮವಾಹಿನಿ
ಪಶ್ಚಿಮೋತ್ತರವಾಗಿ
ಪಸಯಿತನಿಗೆ
ಪಸಾಯತರು
ಪಸಾಯಿತ
ಪಸಾಯಿತನಿಗೆ
ಪಸಾಯ್ತ
ಪಹರದು
ಪಾಂಚಾಲದೇವ
ಪಾಂಚಾಲದೇವನನ್ನು
ಪಾಂಚಾಳದವರು
ಪಾಂಡವಪುರ
ಪಾಂಡವಪುರದಲ್ಲಿವೆ
ಪಾಂಡವರ
ಪಾಂಡವರಗುಹೆಯ
ಪಾಂಡವರಿಂದ
ಪಾಂಡವರು
ಪಾಂಡ್ಯ
ಪಾಂಡ್ಯಕುಲ
ಪಾಂಡ್ಯನ
ಪಾಂಡ್ಯನು
ಪಾಂಡ್ಯಪಾಡಿ
ಪಾಂಡ್ಯಬಲ
ಪಾಂಡ್ಯರ
ಪಾಂಡ್ಯರನ್ನು
ಪಾಂಡ್ಯರಾಜನೊಡನೆ
ಪಾಂಡ್ಯರಾಜರ
ಪಾಂಡ್ಯರಾಜ್ಯ
ಪಾಂಡ್ಯರಾಜ್ಯವನ್ನು
ಪಾಂಡ್ಯರಾಯ
ಪಾಂಡ್ಯರಾಯಪ್ರತಿಷ್ಟಾಚಾರ್ಯ್ಯ
ಪಾಂಡ್ಯರಿಂದ
ಪಾಂಡ್ಯರಿಗೂ
ಪಾಂಡ್ಯರು
ಪಾಂಬಬ್ಬೆ
ಪಾಚಯಪ್ಪ
ಪಾಚಿಯಪ್ಪನ
ಪಾಛಾಬಾದಶಹಾರವರ
ಪಾಠದಿಂದ
ಪಾಠವನ್ನು
ಪಾಠವು
ಪಾಠಶಾಲೆ
ಪಾಠಶಾಲೆಗಳನ್ನು
ಪಾಡಿ
ಪಾಡಿಗಳೆತ್ತಿ
ಪಾತ್ರ
ಪಾತ್ರಕ್ಕೆ
ಪಾತ್ರನಾಗಿದ್ದಾನೆ
ಪಾತ್ರವನ್ನು
ಪಾತ್ರವಹಿಸಿದ್ದನೆಂಬುದು
ಪಾತ್ರವೂ
ಪಾತ್ರೆಗಳನ್ನು
ಪಾದ
ಪಾದಗಳನ್ನು
ಪಾದಚಾರಕನಾದ
ಪಾದದಬಳಿ
ಪಾದಪದ್ಮಾರಾಧಕ
ಪಾದಪದ್ಮಾರಾಧಕರುಮಪ್ಪ
ಪಾದಪದ್ಮೋಪಜೀವಿ
ಪಾದಪದ್ಮೋಪಜೀವಿಯಾಗಿ
ಪಾದಪದ್ಮೋಪಜೀವಿಯಾಗಿದ್ದ
ಪಾದಪದ್ಮೋಪಜೀವಿಯಾಗಿದ್ದನೆಂದು
ಪಾದಪದ್ಮೋಪಜೀವಿಯೆಂದು
ಪಾದಪೂಜೆಯನ್ನು
ಪಾದಪೂಜೆಯಾಗಿ
ಪಾದಸೇವಕನಾದ
ಪಾದಸೇವಕಳಾದ
ಪಾದಾಬ್ಜಕೃಕಟಾಯಿತಚೇತಸಃ
ಪಾದಾರಾಧಕನುಂ
ಪಾದಾರಾಧಕನುಮಪ್ಪ
ಪಾದಾರ್ಚನೆಗೆ
ಪಾಪಣ್ಣನು
ಪಾಪಯ್ಯನಕೊಪ್ಪಲು
ಪಾಪರ್
ಪಾಪಾರಪಟ್ಟಿ
ಪಾಪಾರ್ಪಟ್ಟಿ
ಪಾಯಾತ್
ಪಾಯೆಸ್
ಪಾರಂಪರಿಕ
ಪಾರರಾ
ಪಾರಾದನೆಂದು
ಪಾರಿಜಾತ
ಪಾರಿಜಾತಾಪಹರಣಮು
ಪಾರಿಭಾಷಿಕ
ಪಾರೀಷದೇವರ
ಪಾರುಪತ್ತೇಗಾರ್
ಪಾರ್ಥಸಾರಥಿ
ಪಾರ್ಥಿವಸ್ಯಾಸ್ಯ
ಪಾರ್ವತಿಯನ್ನು
ಪಾರ್ಶ್ವ
ಪಾರ್ಶ್ವಜಿನಗೃಹವನ್ನು
ಪಾರ್ಶ್ವಜಿನೇಶ್ವರ
ಪಾರ್ಶ್ವದಲ್ಲಿ
ಪಾರ್ಶ್ವದೇವ
ಪಾರ್ಶ್ವದೇವನು
ಪಾರ್ಶ್ವದೇವರ
ಪಾರ್ಶ್ವನಾಥ
ಪಾರ್ಶ್ವನಾಥದೇವರು
ಪಾರ್ಶ್ವಪುರವನ್ನಾಗಿ
ಪಾಲಯನಖಿಳಾಂ
ಪಾಲಯನ್
ಪಾಲಹಳ್ಳಿ
ಪಾಲಿತ
ಪಾಲಿನ
ಪಾಲಿಸಬೇಕೆಂದು
ಪಾಲಿಸಲು
ಪಾಲಿಸಿದ
ಪಾಲಿಸಿದ್ದ
ಪಾಲಿಸಿದ್ದನು
ಪಾಲಿಸಿದ್ದನೆಂದು
ಪಾಲಿಸಿದ್ದನೆಂದೂ
ಪಾಲಿಸುತ್ತಿದ್ದರು
ಪಾಲು
ಪಾಲುದಾರಿಕೆಯ
ಪಾಲ್ಗೊಂಡಿರಬಹುದು
ಪಾಲ್ಯಂ
ಪಾಳು
ಪಾಳುಬಸದಿಯಲ್ಲಿರುವ
ಪಾಳೆಗಾರನಾಗಿ
ಪಾಳೆಯಗಾರನ
ಪಾಳೆಯಗಾರನಾಗಿದ್ದ
ಪಾಳೆಯಗಾರನಾಗಿದ್ದನೆಂದು
ಪಾಳೆಯಗಾರರ
ಪಾಳೆಯಗಾರರನ್ನು
ಪಾಳೆಯಗಾರರು
ಪಾಳೆಯಪಟ್ಟನ್ನು
ಪಾಳೆಯಪಟ್ಟೆಂದು
ಪಾಳ್ಯ
ಪಾವಗಡ
ಪಿಂಛಾತಪತ್ರಾನ್ವಿತಾಸನ
ಪಿಡಿದಂ
ಪಿತಾಮಹ
ಪಿತಾಮಹನ
ಪಿತಾಮಹರೆನಿಸಿಕೊಂಡ
ಪಿತ್ರಾರ್ಜಿತವಾಗಿ
ಪಿನಾಕಿನಿಯ
ಪಿಬಿ
ಪಿಬಿದೇಸಾಯಿ
ಪಿಬಿದೇಸಾಯಿಯವರು
ಪಿರಿಯ
ಪಿರಿಯಆಳ್ವಿಕೆ
ಪಿರಿಯಕಳಲೆಯ
ಪಿರಿಯದಂಡನಾಯಕ
ಪಿರಿಯಪಟ್ಟದ
ಪಿರಿಯರಸಿ
ಪಿರಿಯೊಡೆಯನ
ಪಿರಿಯೊಡೆಯನನ್ನು
ಪಿರಿಯೊಡೆಯನಿಗೂ
ಪಿರಿಯೊಡೆಯನು
ಪಿರಿಯೊಡೆಯನೂ
ಪಿರಿಯೊಡೆಯನೆಂಬ
ಪಿರಿಯೊಡೆಯರು
ಪೀಠದ
ಪೀಠಿಕೆಗಳನ್ನು
ಪೀಠಿಕೆಗಳು
ಪೀಠಿಕೆಯನ್ನು
ಪೀಠಿಕೆಯಲ್ಲಿ
ಪೀಠೋಪಕರಣಗಳು
ಪುಂಡರಿಕ
ಪುಂಡಾಟಿಕೆಯನ್ನು
ಪುಂಣ್ಯಕ್ರುತ
ಪುಂಣ್ಯಚರಿತಂ
ಪುಗಿರಿನಾಡಿನ
ಪುಗಿರಿನಾಡುಪೊಗರ್ನಾಡು
ಪುಗಿಸಿದಾರ್
ಪುಟಕ್ಕೆ
ಪುಣಸಮಯ್ಯನನ್ನು
ಪುಣಸಿಮಯ್ಯನು
ಪುಣಸೆಪಟ್ಟಿ
ಪುಣಿಗದ
ಪುಣಿಸ
ಪುಣಿಸದಂಡನಾಥನು
ಪುಣಿಸಮಯ್ಯ
ಪುಣಿಸಮಯ್ಯನ
ಪುಣಿಸಮಯ್ಯನು
ಪುಣಿಸಮಯ್ಯನೆಂದುಹೇಳಿದೆ
ಪುಣಿಸಮ್ಮ
ಪುಣಿಸಶ್ರೀ
ಪುಣ್ಯ
ಪುಣ್ಯಜನಧಾಮ
ಪುಣ್ಯವಾಗಬೇಕೆಂದು
ಪುಣ್ಯಾಹವೈಃ
ಪುಣ್ಯೇ
ಪುತ್ತೂರಿನಲ್ಲಿ
ಪುತ್ತೂರು
ಪುತ್ರ
ಪುತ್ರನೆಂದು
ಪುತ್ರರತ್ನರನ್ನೂ
ಪುತ್ರರಲ್ಲಿ
ಪುತ್ರರಾದ
ಪುತ್ರರು
ಪುತ್ರರೂ
ಪುತ್ರರೆಂಬ
ಪುತ್ರರೋ
ಪುತ್ರಸಮಾನನಾಗಿ
ಪುತ್ರಸಮಾನನಾಗಿದ್ದ
ಪುತ್ರಿ
ಪುತ್ರಿಯ
ಪುತ್ರಿಯರನ್ನು
ಪುತ್ರೋತ್ಸವಮಾಗಲ್
ಪುತ್ರೋತ್ಸವವಾದಾಗ
ಪುನಃ
ಪುನರಪಿ
ಪುನರುಜ್ಜೀವನಗೊಳಿಸುವಂತೆ
ಪುನರ್
ಪುನರ್ದತ್ತಯಾಗಿ
ಪುನರ್ದತ್ತಿಯಾಗಿ
ಪುನರ್ದಾನ
ಪುನರ್ದ್ಧಾರಾಪೂರ್ವಕಂ
ಪುನರ್ಧಾರಾಪೂರ್ವಕವಾಗಿ
ಪುನರ್ನಿಗದಿಪಡಿಸಿ
ಪುನರ್ರಚಿಸಬಹುದು
ಪುನರ್ರಚಿಸಲಾಯಿತು
ಪುನರ್ರಚಿಸಿ
ಪುಮಾನೇಷಃ
ಪುರ
ಪುರಃ
ಪುರಗಳನ್ನು
ಪುರಗಳು
ಪುರಣೋಕ್ತ
ಪುರದ
ಪುರದಮಾಗಣಿಗೆ
ಪುರದಾನವಾಗಿ
ಪುರವನ್ನಾಗಿ
ಪುರವರಾಧೀಶ್ವರ
ಪುರವರ್ಗದಾನ
ಪುರವಾದ
ಪುರಷಾಶ್ಚ
ಪುರಾಣ
ಪುರಾತತ್ತ್ವ
ಪುರಾತತ್ವ
ಪುರಾತತ್ವದ
ಪುರಾತನ
ಪುರಾತನವಾದುದು
ಪುರಾಧಿಪನ
ಪುರಾಯತತರಂ
ಪುರಿಗೆರೆ
ಪುರಿಗೆರೆನ್ನು
ಪುರಿಶೈ
ಪುರುಷ
ಪುರುಷನಾದ
ಪುರುಷಮಾಣಿಕ್ಯಸೆಟ್ಟಿಯು
ಪುರುಷರಿಗೆ
ಪುರುಷರು
ಪುರುಷಾರ್ತ್ಥ
ಪುರುಷಾರ್ಥ
ಪುರುಷೋತ್ತಮದೇವ
ಪುರೋಹಿತ
ಪುರೋಹಿತರಿಗೆ
ಪುಲಿಗೆರೆನ್ನು
ಪುಲಿಯಣ್ಣ
ಪುಲಿಯಣ್ಣನ
ಪುಳ್ಳೈಲೋಕಾಚಾರ್ಯರ
ಪುವಗಾಮವನ್ನು
ಪುಷ್ಕರಣಿ
ಪುಷ್ಟಿ
ಪುಷ್ಟಿಯನ್ನು
ಪುಷ್ಟಿರ್ಜ್ವಯಶ್ಚ
ಪುಷ್ಪಮಾಲೆಯನ್ನು
ಪುಷ್ಪೋತ್ಗಮನ
ಪುಸಿವರಗಂಡ
ಪುಸ್ತಕಕ್ಕೆ
ಪೂಜಾ
ಪೂಜಾಕಾರ್ಯಗಳಿಗೆ
ಪೂಜಾಕೈಂಕರ್ಯಗಳಿಗೆ
ಪೂಜಾದಿಕಾರ್ಯಗಳಿಗೆ
ಪೂಜಾರರ
ಪೂಜಿಸಿ
ಪೂಜಿಸಿದನು
ಪೂಜಿಸುತ್ತಿದ್ದನು
ಪೂಜಿಸುವ
ಪೂಜೆಗೆ
ಪೂಜೆಪುನಸ್ಕಾರಗಳು
ಪೂಜೆಯ
ಪೂಜೆಯನ್ನು
ಪೂಜ್ಯನಲ್ತೆ
ಪೂನಾಡುಪುನ್ನಾಡುಗಳನ್ನು
ಪೂರಕವಾಗಿ
ಪೂರಿಗಾಲಿ
ಪೂರೈಸಿ
ಪೂರೈಸಿದ
ಪೂರೈಸಿದರು
ಪೂರ್ಣಯ್ಯನವರ
ಪೂರ್ಣಯ್ಯನೆಂಬ
ಪೂರ್ಣವಾಗಿ
ಪೂರ್ತಿ
ಪೂರ್ತಿಯಾಗಿ
ಪೂರ್ಬ್ಬಾಯ
ಪೂರ್ವ
ಪೂರ್ವಕವೇಕಚ್ಛತ್ರಚ್ಛಾಯೆ
ಪೂರ್ವಕ್ಕೂ
ಪೂರ್ವಕ್ಕೆ
ಪೂರ್ವಜರು
ಪೂರ್ವದ
ಪೂರ್ವದಕ್ಷಿಣಪಶ್ಚಿಮ
ಪೂರ್ವದಲ್ಲಿ
ಪೂರ್ವದಲ್ಲಿದ್ದ
ಪೂರ್ವದಿಂದಲೂ
ಪೂರ್ವದಿಕ್ಕಿನ
ಪೂರ್ವಪಶ್ಚಿಮ
ಪೂರ್ವಪಶ್ಚಿಮಗಳಲ್ಲಿ
ಪೂರ್ವಪಶ್ಚಿಮಸಮದ್ರಾಧಿಪತಿ
ಪೂರ್ವಭಾಗದಲ್ಲಿ
ಪೂರ್ವಮರಿಯಾದೆಯಲು
ಪೂರ್ವಾಚಲಮಾರ್ತಾಂಡ
ಪೂರ್ವಾನ್ವಯ
ಪೂರ್ವಾರ್ಧದವರೆಗೆ
ಪೂರ್ವೋಕ್ತ
ಪೂವಗಾಮವೇ
ಪೂವಗಾಮೆಯನ್ನು
ಪೃಥಿವೀ
ಪೃಥಿವೀನೀರ್ಗುಂದ
ಪೃಥುವೀ
ಪೃಥುವೀಗಾಮುಣ್ಡರು
ಪೃಥುವೀನೀರ್ಗುಂದ
ಪೃಥ್ವೀ
ಪೃಥ್ವೀಂ
ಪೃಥ್ವೀಗಂಗನ
ಪೃಥ್ವೀಗಂಗನಿದ್ದು
ಪೃಥ್ವೀಪತಿ
ಪೃಥ್ವೀಪತಿಗಳು
ಪೃಥ್ವೀಪತಿಗೆ
ಪೃಥ್ವೀಪತಿಯಾಗಿದ್ದಾನೆಂದು
ಪೃಥ್ವೀಪತಿಯು
ಪೃಥ್ವೀರಾಜ್ಯಂಗೆಯುತ್ತಿರೆ
ಪೃಥ್ವೀರಾಜ್ಯಂಗೆಯ್ಯುತ್ತಿದ್ದನು
ಪೆಂ
ಪೆಂಗೆನಾಯಕನ
ಪೆಂಪಿನ
ಪೆಗ್ಗಡೆನಾಯ್ಕ
ಪೆತ್ತ
ಪೆದ್ದ
ಪೆದ್ದಗವುಡುಗಳ
ಪೆದ್ದಣ್ಣನು
ಪೆದ್ದಿರಾಜುಗೆ
ಪೆದ್ದಿರಾಜುವು
ಪೆನುಗೊಂಡೆ
ಪೆನುಗೊಂಡೆಗೆ
ಪೆನುಗೊಂಡೆದುರ್ಗದಲ್ಲಿ
ಪೆನುಗೊಂಡೆಯ
ಪೆನುಗೊಂಡೆಯಲ್ಲಿದ್ದ
ಪೆನುಗೊಂಡೆಯಲ್ಲಿದ್ದನೆಂದು
ಪೆನುಗೊಂಡೆಯವರೆಗೆ
ಪೆನುಗೊಂಡೆಯವರೋ
ಪೆನುಗೊಂಡೆಯಿಂದ
ಪೆನುಗೊಂಡೆಯೊಳು
ಪೆಮ್ಮೋಜನೂ
ಪೆರಂಗೂರು
ಪೆರಮಗಾವುಂಡನ
ಪೆರಮಾನುಡನ್
ಪೆರಮಾಳದೇವ
ಪೆರಮಾಳು
ಪೆರಮಾಳೆ
ಪೆರಮಾಳೆದೇವ
ಪೆರಮಾಳೆದೇವನ
ಪೆರಮಾಳೆದೇವನಿಗೆ
ಪೆರಮಾಳೆದೇವನು
ಪೆರಮಾಳೆಯು
ಪೆರಾಳ್ಕೆ
ಪೆರಾಳ್ಕೆಯೇ
ಪೆರಿಯಮನೈ
ಪೆರಿರಾಜು
ಪೆರುಂಕೋಟೆರಾಜ್ಯದ
ಪೆರುಮಾಳ
ಪೆರುಮಾಳದೇವನು
ಪೆರುಮಾಳದೇವರಸನಿಗೆ
ಪೆರುಮಾಳರಸ
ಪೆರುಮಾಳಿಗೆ
ಪೆರುಮಾಳುಸಮುದ್ರ
ಪೆರುಮಾಳೆ
ಪೆರುಮಾಳೆಚಮೂಪತಿಗಿಂತು
ಪೆರುಮಾಳೆದೇವ
ಪೆರುಮಾಳೆದೇವನ
ಪೆರುಮಾಳೆದೇವನಿಗಿದ್ದ
ಪೆರುಮಾಳೆದೇವನಿಗೆ
ಪೆರುಮಾಳೆದೇವನು
ಪೆರುಮಾಳೆದೇವರಸನು
ಪೆರುಮಾಳೆಪುರವೆಂಬ
ಪೆರುಮಾಳೆಯ
ಪೆರುಮಾಳೆಯು
ಪೆರುಮಾಳೆಯೂ
ಪೆರುಮಾಳೆಲಕ್ಷ್ಮೀನಾರಾಯಣ
ಪೆರ್ಗಡೆ
ಪೆರ್ಗಡೆಯಾಗಿರುತ್ತಿದ್ದನು
ಪೆರ್ಗಡೆಯು
ಪೆರ್ಗಡೆಯೊಳ್
ಪೆರ್ಗಡೆಹೆಗ್ಗಡೆ
ಪೆರ್ಗ್ಗಡೆ
ಪೆರ್ಗ್ಗಡೆಗಳ
ಪೆರ್ಗ್ಗಡೆಗಳನ್ನು
ಪೆರ್ಗ್ಗಡೆಗಳು
ಪೆರ್ಗ್ಗಡೆಗೆ
ಪೆರ್ಗ್ಗಡೆನಾಯಕ
ಪೆರ್ಗ್ಗಡೆಯಾಗಿದ್ದ
ಪೆರ್ಗ್ಗಡೆಯಾಗಿದ್ದು
ಪೆರ್ಗ್ಗಡೆಯು
ಪೆರ್ಗ್ಗಡೆಯೂ
ಪೆರ್ಗ್ಗಡೆಹಿರಿಯಹೆಗ್ಗಡೆಗಳು
ಪೆರ್ದ್ದೊರೆ
ಪೆರ್ಬಾಣನಹಳ್ಳಿಯನ್ನು
ಪೆರ್ಬ್ಬೞ
ಪೆರ್ಮಾಡಿ
ಪೆರ್ಮಾಡಿದೇವನ
ಪೆರ್ಮಾಡಿದೇವನು
ಪೆರ್ಮಾನಡಿ
ಪೆರ್ಮಾನಡಿಗಳು
ಪೆರ್ಮಾನಡಿಯ
ಪೆರ್ಮಾನಡಿಯಎರಡನೇ
ಪೆರ್ಮಾನಡಿಯನ್ನು
ಪೆರ್ಮಾನಡಿಯು
ಪೆರ್ಮಾನಡಿಯೆಂಬ
ಪೆರ್ಮ್ಮನಡಿಗಳ
ಪೆರ್ಮ್ಮಾನಡಿ
ಪೆರ್ಮ್ಮಾನಡಿಯ
ಪೆರ್ಮ್ಮೆ
ಪೆಸಾಳಿ
ಪೆಸಾಳಿಹನುಮ
ಪೇಟಿರಾಜಯ್ಯನು
ಪೇಟೆ
ಪೇಟೆಯನ್ನಾಗಿ
ಪೇದ
ಪೇರಾಳ್ಕೆ
ಪೇರೂರಿನಲ್ಲಿ
ಪೇರೂರಿನಲ್ಲಿದ್ದ
ಪೇಳ್ವೆ
ಪೈಕಿ
ಪೈಗಂಬರ್
ಪೊ
ಪೊಂನಣ್ಣ
ಪೊಕ್ಕು
ಪೊಕ್ಕುಮೆ
ಪೊಗಳೆ
ಪೊಗಳ್ಗು
ಪೊಗಳ್ವಿನಂ
ಪೊಟ್ಟಳಿಸುವ
ಪೊಡರ್ಪ್ಪವೇವೇಳ್ವುದೋ
ಪೊಡವಿಗೆ
ಪೊನ್ನ
ಪೊನ್ನಗಾವುಣ್ಡ
ಪೊನ್ನಡಿ
ಪೊನ್ನಪ್ಪ
ಪೊನ್ನಲದೇವಿಯರ
ಪೊನ್ನಳ್ಳಿ
ಪೊನ್ನೆತ್ತಿಕೊಳೆ
ಪೊನ್ನೆತ್ತಿಕೊಳ್ಳುವಂತೆ
ಪೊನ್ವಿಟ್ಟು
ಪೊಯ್
ಪೊಯ್ದಿರಿದಂ
ಪೊಯ್ದು
ಪೊಯ್ಸಳ
ಪೊಯ್ಸಳದೇವ
ಪೊಯ್ಸಳದೇವರಸ
ಪೊಯ್ಸಳದೇವರಸರು
ಪೊಯ್ಸಳದೇವರಾಜ್ಯದಲ್ಲಿ
ಪೊಯ್ಸಳದೇವರು
ಪೊಯ್ಸಳನರಾಜ್ಯ
ಪೊಯ್ಸಳನು
ಪೊಯ್ಸಳನೆಂಬ
ಪೊಯ್ಸಳನೇ
ಪೊರರ
ಪೊರಳ್ಚಿ
ಪೊರೆದನೆಂದೂ
ಪೊಲುವರಂ
ಪೊಳಲ
ಪೊಳಲಸೆಟ್ಟಿಗೆ
ಪೊಳಲು
ಪೊಳಲ್ಚೋರನ
ಪೊಳಲ್ಚೋರನು
ಪೋಗಿ
ಪೋಚಲದೇವಿಯು
ಪೋಚವ್ವೆಗಾಗಿ
ಪೋಚಾಂಬಿಕೋದರೋದನ್ವತ್ಪಾರಿಜಾತಂ
ಪೋಚಿಕಬ್ಬೆ
ಪೋಚಿಕಬ್ಬೆಯರ
ಪೋಚಿಕಬ್ಬೆಯು
ಪೋತನಾಯಕ
ಪೋದಲಶರ್ಮ
ಪೋಮನ್
ಪೋರಿಲಿಇಭದೆ
ಪೋರಿಲಿಭದೆ
ಪೋಷಕನಾಗಿ
ಪೋಷಕರಲ್ಲಿ
ಪೋಷಣನಿರ್ಭರಭೂನವಖಂಡಃ
ಪೋಷಿತನಾದ
ಪೋಸಳದೇವರು
ಪೋಸಳನು
ಪೌತ್ರ
ಪೌತ್ರರಾದ
ಪೌತ್ರರೂ
ಪೌರಾಣಿಕ
ಪೌರಾಣಿಕರಾಗಿದ್ದವರು
ಪ್ಪುತ್ರಮಿತ್ರಸ್ತೋಮಂ
ಪ್ರಉಡದೇವರಾಯರಾದ
ಪ್ರಕಟಗೊಂಡು
ಪ್ರಕಟಣಾ
ಪ್ರಕಟಣೆ
ಪ್ರಕಟಣೆಯಾಗುತ್ತಿರುವ
ಪ್ರಕಟವಾಗಿದೆ
ಪ್ರಕಟವಾಗಿದ್ದು
ಪ್ರಕಟವಾಗಿರುವ
ಪ್ರಕಟವಾಗಿವೆ
ಪ್ರಕಟವಾಗುತ್ತಿದ್ದ
ಪ್ರಕಟವಾಗುತ್ತಿದ್ದವು
ಪ್ರಕಟವಾದ
ಪ್ರಕಟವಾದವು
ಪ್ರಕಟಿಸಲಾಗಿದೆ
ಪ್ರಕಟಿಸಿದರು
ಪ್ರಕಟಿಸಿದೆ
ಪ್ರಕಟಿಸಿದ್ದಾರೆ
ಪ್ರಕಟಿಸುತ್ತಿದೆ
ಪ್ರಕಟಿಸುವ
ಪ್ರಕಾರ
ಪ್ರಕಾರವೂ
ಪ್ರಕಾಶಂ
ಪ್ರಕೃತ
ಪ್ರಕೃತಃ
ಪ್ರಕೃತಯಃ
ಪ್ರಕೃತಿಜನ್ಯ
ಪ್ರಕೃತೀನಾಂ
ಪ್ರಕ್ಷುಬ್ದವಾಗಿ
ಪ್ರಖ್ಯಾತ
ಪ್ರಖ್ಯಾತಂ
ಪ್ರಖ್ಯಾತನಾಗಿ
ಪ್ರಖ್ಯಾತನಾದ
ಪ್ರಖ್ಯಾತನಾದನು
ಪ್ರಖ್ಯಾತರಾಗಿದ್ದು
ಪ್ರಖ್ಯಾತೌ
ಪ್ರಗತಿ
ಪ್ರಚಂಡ
ಪ್ರಚಂಡದಂಡನಾಯಕ
ಪ್ರಚಂಡದೇವ
ಪ್ರಚಲಿತದಲ್ಲಿದ್ದವು
ಪ್ರಚಾರ
ಪ್ರಚಾರಕರು
ಪ್ರಜಾಃ
ಪ್ರಜಾಧರ್ಮಪರಿಪಾಲನಾದಿ
ಪ್ರಜೆ
ಪ್ರಜೆಗಗೌಂಡಗಳು
ಪ್ರಜೆಗಳ
ಪ್ರಜೆಗಳಿಂದ
ಪ್ರಜೆಗಳಿಗೆ
ಪ್ರಜೆಗಳು
ಪ್ರಜೆಗಾವುಂಡ
ಪ್ರಜೆಗಾವುಂಡಗಳು
ಪ್ರಜೆಗಾವುಂಡನನ್ನಾಗಿ
ಪ್ರಜೆಗಾವುಂಡರ
ಪ್ರಜೆಗಾವುಂಡರನ್ನು
ಪ್ರಜೆಗಾವುಂಡರನ್ನೇ
ಪ್ರಜೆಗಾವುಂಡರಿಗೆ
ಪ್ರಜೆಗಾವುಂಡರಿರಬಹುದು
ಪ್ರಜೆಗಾವುಂಡರು
ಪ್ರಜೆಗಾವುಂಡುಗಳು
ಪ್ರಜೆಗೌಡಿನ
ಪ್ರಜೆನಾಯಕ
ಪ್ರಜೆನಾಯಕರಿಗೆ
ಪ್ರಜೆಬೀಡಿನವರು
ಪ್ರಜೆಮೆಚ್ಚೆಗಂಡ
ಪ್ರತಾನಂಗಳೊಳು
ಪ್ರತಾಪ
ಪ್ರತಾಪಕಂಠೀರವ
ಪ್ರತಾಪಚಕ್ರವರ್ತಿ
ಪ್ರತಾಪದೇವರಾಯ
ಪ್ರತಾಪದೇವರಾಯನ
ಪ್ರತಾಪದೇವರಾಯನೆಂಬ
ಪ್ರತಾಪದೇವರಾಯಪುರ
ಪ್ರತಾಪನಾರಸಿಂಹ
ಪ್ರತಾಪನಿಳಯಂ
ಪ್ರತಾಪಮೆಂತೆಂದಡೆ
ಪ್ರತಾಪವಾನ್
ಪ್ರತಾಪವಿಜಯಮದನಪುರ
ಪ್ರತಾಪಸಮೇತರ್
ಪ್ರತಾಪಹೊಯ್ಸಳ
ಪ್ರತಾಪಿಗಳೂ
ಪ್ರತಿ
ಪ್ರತಿಕೂಲ
ಪ್ರತಿದಿನ
ಪ್ರತಿನಾಕಮಲ್ಲನೆಂಬ
ಪ್ರತಿನಾಮಕರಣ
ಪ್ರತಿನಾಮಧೇಯವಾದ
ಪ್ರತಿನಾಮಧೇಯವಿತ್ತು
ಪ್ರತಿನಾಮಧೇಯವುಳ್ಳ
ಪ್ರತಿನಿಧಿ
ಪ್ರತಿನಿಧಿಗಳನ್ನು
ಪ್ರತಿನಿಧಿಗಳಿಗೆ
ಪ್ರತಿನಿಧಿಗಳು
ಪ್ರತಿನಿಧಿಯ
ಪ್ರತಿನಿಧಿಯಾಗಿ
ಪ್ರತಿನಿಧಿಸುತ್ತಾ
ಪ್ರತಿಪನ್ನದಿ
ಪ್ರತಿಬಂಧಕಗಳಾದ
ಪ್ರತಿಮೆ
ಪ್ರತಿಮೆಗಳನ್ನು
ಪ್ರತಿಮೆಯನ್ನು
ಪ್ರತಿಯನ್ನು
ಪ್ರತಿಯೊಂದು
ಪ್ರತಿರೋಧದ
ಪ್ರತಿರೋಧವನ್ನು
ಪ್ರತಿಷ್ಟಾಚಾರ್ಯ
ಪ್ರತಿಷ್ಟಾಚಾರ್ಯ್ಯ
ಪ್ರತಿಷ್ಠಾಚಾರ್ಯ
ಪ್ರತಿಷ್ಠಾಪನಾಚಾರ್ಯ
ಪ್ರತಿಷ್ಠಾಪನಾಚಾರ್ಯರಾದ
ಪ್ರತಿಷ್ಠಾಪಿಸಲಾಯಿತೆಂದು
ಪ್ರತಿಷ್ಠಾಪಿಸಿ
ಪ್ರತಿಷ್ಠಾಪಿಸಿದನು
ಪ್ರತಿಷ್ಠಾಪಿಸಿದನೆಂದು
ಪ್ರತಿಷ್ಠಾಪಿಸಿದನೆಂದೂ
ಪ್ರತಿಷ್ಠಾಪಿಸಿದರೆಂದು
ಪ್ರತಿಷ್ಠಾಪಿಸಿದರೆಂದೂ
ಪ್ರತಿಷ್ಠಾಪಿಸಿದಳು
ಪ್ರತಿಷ್ಠೆ
ಪ್ರತಿಷ್ಠೆಮಾಡಿ
ಪ್ರತಿಷ್ಠೆಯನ್ನು
ಪ್ರತಿಷ್ಠೆಯಾಗಿ
ಪ್ರತಿಷ್ಠೆಯಾದ
ಪ್ರತಿಸ್ಪರ್ಧಿಗಳಾಗಿದ್ದ
ಪ್ರತೀಕವಾಗಿ
ಪ್ರತೀತಿ
ಪ್ರತ್ಯಕ್ಷ
ಪ್ರತ್ಯಕ್ಷದರ್ಶಿಯೊಬ್ಬ
ಪ್ರತ್ಯರ್ತ್ಥಿಕ್ಷಿತಿಪಾಲರತ್ನಮಕುಟೀನೀರಾಜಿತಾಂಘ್ರಿಶ್ಚಿರಂ
ಪ್ರತ್ಯೇಕ
ಪ್ರತ್ಯೇಕವಾಗಿ
ಪ್ರಥಮ
ಪ್ರದಾನವಾಗಿದ್ದವು
ಪ್ರದೇಶ
ಪ್ರದೇಶಕ್ಕೆ
ಪ್ರದೇಶಗಳ
ಪ್ರದೇಶಗಳನ್ನು
ಪ್ರದೇಶಗಳನ್ನೂ
ಪ್ರದೇಶಗಳಲ್ಲಿ
ಪ್ರದೇಶಗಳಲ್ಲಿರುವ
ಪ್ರದೇಶಗಳಿಗೆ
ಪ್ರದೇಶಗಳು
ಪ್ರದೇಶಗಳೂ
ಪ್ರದೇಶದ
ಪ್ರದೇಶದಕ್ಕೆ
ಪ್ರದೇಶದಲ್ಲಿ
ಪ್ರದೇಶದಲ್ಲಿತ್ತು
ಪ್ರದೇಶದಲ್ಲಿದ್ದ
ಪ್ರದೇಶದಲ್ಲಿಯೇ
ಪ್ರದೇಶದಲ್ಲಿರುವ
ಪ್ರದೇಶದಲ್ಲಿರುವುದರಿಂದ
ಪ್ರದೇಶದವರೇ
ಪ್ರದೇಶವನ್ನು
ಪ್ರದೇಶವಾಗಿತ್ತು
ಪ್ರದೇಶವಾದರೂ
ಪ್ರದೇಶವು
ಪ್ರದೇಶವೂ
ಪ್ರದೇಶವೇ
ಪ್ರಧಾನ
ಪ್ರಧಾನನಾಗಿ
ಪ್ರಧಾನನಾಗಿಮಂತ್ರಿಯಾಗಿ
ಪ್ರಧಾನನಾಗಿರುತ್ತಾನೆ
ಪ್ರಧಾನಪಾತ್ರ
ಪ್ರಧಾನಮಂತ್ರಿ
ಪ್ರಧಾನರು
ಪ್ರಧಾನರೇ
ಪ್ರಧಾನವಾಗಿ
ಪ್ರಧಾನಿ
ಪ್ರಧಾನಿಗಳಾಗಲಿಲ್ಲ
ಪ್ರಧಾನಿಯಾಗಿದ್ದ
ಪ್ರಧಾನಿಯಾಗಿದ್ದರೂ
ಪ್ರಬಂಧ
ಪ್ರಬಂಧಗಳನ್ನು
ಪ್ರಬಂಧಗಳಲ್ಲಿ
ಪ್ರಬಂಧಗಳು
ಪ್ರಬಂಧವಾಗಿದೆ
ಪ್ರಬಂಧವು
ಪ್ರಬಲ
ಪ್ರಬಲಮಭೂತ್ತುಷ್ಟಿಃ
ಪ್ರಬಲವಾಗಿತ್ತು
ಪ್ರಬಲವಾದ
ಪ್ರಬಲಿಸಲು
ಪ್ರಬಲಿಸಿ
ಪ್ರಭತ್ವವನ್ನು
ಪ್ರಭಾಚಂದ್ರ
ಪ್ರಭಾಚಂದ್ರಸಿದ್ಧಾಂತ
ಪ್ರಭಾಚಂದ್ರಸಿದ್ಧಾಂತದೇವರಿಗೆ
ಪ್ರಭಾವದ
ಪ್ರಭಾವದಿಂದ
ಪ್ರಭಾವನೆನಿಸಿ
ಪ್ರಭಾವಾವತಾರಿತ
ಪ್ರಭಾವಿ
ಪ್ರಭು
ಪ್ರಭುಗಳ
ಪ್ರಭುಗಳಾಗಿ
ಪ್ರಭುಗಳಾಗಿದ್ದ
ಪ್ರಭುಗಳು
ಪ್ರಭುಗವುಡಗಳು
ಪ್ರಭುಗಾವುಂಡ
ಪ್ರಭುಗಾವುಂಡಗಳ
ಪ್ರಭುಗಾವುಂಡಗಳು
ಪ್ರಭುಗಾವುಂಡನೆಂದು
ಪ್ರಭುಗಾವುಂಡರ
ಪ್ರಭುಗಾವುಂಡರಿಗೆ
ಪ್ರಭುಗಾವುಂಡರಿದ್ದರು
ಪ್ರಭುಗಾವುಂಡರು
ಪ್ರಭುಗಾವುಂಡರುಗಳಾಗಿದ್ದರೆಂದು
ಪ್ರಭುಗಾವುಂಡರುಪ್ರಜೆಗಾವುಂಡರು
ಪ್ರಭುಗಾವುಂಡರುಪ್ರಜೆಗಾವುಂಡರುಗಾವುಂಡರು
ಪ್ರಭುಗಾವುಂಡರೆಂದು
ಪ್ರಭುಗಾವುಂಡುಗಗಳು
ಪ್ರಭುಗಾವುಂಡುಗಳ
ಪ್ರಭುಗಾವುಂಡುಗಳು
ಪ್ರಭುಗಾವುಡುಗಳು
ಪ್ರಭುತನಕ್ಕೆ
ಪ್ರಭುತ್ವ
ಪ್ರಭುತ್ವವನ್ನು
ಪ್ರಭುತ್ವಸಹಿತ
ಪ್ರಭುತ್ವಸಹಿತಂ
ಪ್ರಭುತ್ವಸಹಿತವಾಗಿ
ಪ್ರಭುಪೆರ್ಗ್ಗಡೆ
ಪ್ರಭುವರ್ಗದವರ
ಪ್ರಭುವರ್ಗದವರಿದ್ದರು
ಪ್ರಭುವರ್ಗದವರು
ಪ್ರಭುವರ್ಗದವರೆಂದು
ಪ್ರಭುವಾಗಿದ್ದ
ಪ್ರಭುವಾಗಿದ್ದನೆಂದು
ಪ್ರಭುವಾಗಿದ್ದವನನ್ನು
ಪ್ರಭುವಾಗಿದ್ದು
ಪ್ರಭುವಾದ
ಪ್ರಭುಶಕ್ತಿ
ಪ್ರಭುಶಕ್ತಿಯನಾಂತ
ಪ್ರಮುಖ
ಪ್ರಮುಖಃ
ಪ್ರಮುಖನಾಗಿದ್ದ
ಪ್ರಮುಖನೆಂದು
ಪ್ರಮುಖಪಾತ್ರ
ಪ್ರಮುಖಪಾತ್ರವಹಿಸಿದ್ದ
ಪ್ರಮುಖಪಾತ್ರವಹಿಸಿದ್ದನು
ಪ್ರಮುಖಮುಖ್ಯ
ಪ್ರಮುಖಮುಖ್ಯರು
ಪ್ರಮುಖರನ್ನು
ಪ್ರಮುಖರಾಗಿದ್ದರೆಂಬುದು
ಪ್ರಮುಖರಾಗಿದ್ದು
ಪ್ರಮುಖರಿಗೆ
ಪ್ರಮುಖರು
ಪ್ರಮುಖವಾಗಿ
ಪ್ರಮುಖವಾಗಿತ್ತೆಂದು
ಪ್ರಮುಖವಾಗಿತ್ತೆಂಬುದು
ಪ್ರಮುಖವಾಗಿವೆ
ಪ್ರಮುಖವಾದ
ಪ್ರಮುಖವಾದುವೆಂದರೆ
ಪ್ರಮುಖವೆಂದೂ
ಪ್ರಯತ್ನ
ಪ್ರಯತ್ನದಲ್ಲೂ
ಪ್ರಯತ್ನವನ್ನು
ಪ್ರಯತ್ನವು
ಪ್ರಯತ್ನಿಸಿದರು
ಪ್ರಯತ್ನಿಸಿದಾಗ
ಪ್ರಯತ್ನಿಸಿರಬಹುದು
ಪ್ರಯತ್ನಿಸಿರಬಹುದೆಂದು
ಪ್ರಯತ್ನಿಸುತ್ತಿದ್ದನು
ಪ್ರಯತ್ನಿಸುತ್ತಿದ್ದರು
ಪ್ರಯಾಗಪೆರುಮಾಳೆ
ಪ್ರಯಾಣ
ಪ್ರಯೋಕ್ತೃಕುಶಲೋ
ಪ್ರಯೋಗ
ಪ್ರಯೋಗಗಳನ್ನು
ಪ್ರಯೋಗವಾಗಿದೆ
ಪ್ರಯೋಗವಿದೆ
ಪ್ರಯೋಗವಿದ್ದು
ಪ್ರಯೋಗಿಸಲಾಗಿದೆ
ಪ್ರವಚನ
ಪ್ರವರ್ತಿಸುತ್ತಿದ್ದನೆಂದು
ಪ್ರವರ್ಧಮಾನಕ್ಕೆ
ಪ್ರವಾಸಿ
ಪ್ರವಾಸಿಗರಾದ
ಪ್ರವಿಷ್ಠದ
ಪ್ರವೀಣನಾಗಿದ್ದನೆಂದು
ಪ್ರವೀಣನಾದ
ಪ್ರವುಡದೇವರಾಯ
ಪ್ರವುಢಪ್ರತಾಪ
ಪ್ರವೇಶ
ಪ್ರಶಸ್ತಿ
ಪ್ರಶಸ್ತಿಯ
ಪ್ರಶಸ್ತಿಯನ್ನು
ಪ್ರಶಸ್ತಿಸಹಿತ
ಪ್ರಶಿದ್ಧಃ
ಪ್ರಶ್ನೆ
ಪ್ರಸಕ್ತ
ಪ್ರಸನ್ನ
ಪ್ರಸನ್ನಕೇಶವಪುರ
ಪ್ರಸನ್ನಮಾಧವಪುರಸತ್ಯಾಗಾಲ
ಪ್ರಸಸ್ತಿಯನ್ನು
ಪ್ರಸಾದವನ್ನು
ಪ್ರಸಾದಿತ
ಪ್ರಸಿದ್ಧ
ಪ್ರಸಿದ್ಧಃ
ಪ್ರಸಿದ್ಧನಾಗಿದ್ದ
ಪ್ರಸಿದ್ಧನಾಗಿದ್ದನು
ಪ್ರಸಿದ್ಧನಾದ
ಪ್ರಸಿದ್ಧನಾದನು
ಪ್ರಸಿದ್ಧನು
ಪ್ರಸಿದ್ಧವಾಗಿತ್ತು
ಪ್ರಸಿದ್ಧವಾಗಿತ್ತೆಂದು
ಪ್ರಸಿದ್ಧವಾಗಿದೆ
ಪ್ರಸಿದ್ಧವಾಗಿದ್ದು
ಪ್ರಸಿದ್ಧವಾದ
ಪ್ರಸಿದ್ಧಿಗೆ
ಪ್ರಸಿದ್ಧಿಯನ್ನು
ಪ್ರಸಿದ್ಧಿಯಾಗಿತ್ತು
ಪ್ರಸಿದ್ಧಿಯಾಗಿದೆ
ಪ್ರಸ್ತಾಪ
ಪ್ರಸ್ತಾಪವನ್ನು
ಪ್ರಸ್ತಾಪವಾಗಿದೆ
ಪ್ರಸ್ತಾಪವಾಗಿದ್ದು
ಪ್ರಸ್ತಾಪವಿದೆ
ಪ್ರಸ್ತಾಪವಿದ್ದು
ಪ್ರಸ್ತಾಪವಿರುವ
ಪ್ರಸ್ತಾಪವಿಲ್ಲ
ಪ್ರಸ್ತಾಪಿತವಾಗಿರುವ
ಪ್ರಸ್ತಾಪಿಸಲಾಗಿದೆ
ಪ್ರಸ್ತಾಪಿಸಿದೆ
ಪ್ರಸ್ತಾಪಿಸಿರುವ
ಪ್ರಸ್ತಾಪಿಸುತ್ತದೆ
ಪ್ರಸ್ತಾಪಿಸುವ
ಪ್ರಸ್ತಾವ
ಪ್ರಸ್ತಾವವೂ
ಪ್ರಸ್ತುತ
ಪ್ರಹುಡದೇವರಾಯ
ಪ್ರಾಂತಗಳ
ಪ್ರಾಂತಗಳನ್ನು
ಪ್ರಾಂತಗಳಾಗಿ
ಪ್ರಾಂತಗಳಿಗೆ
ಪ್ರಾಂತಗಳು
ಪ್ರಾಂತದ
ಪ್ರಾಂತದಿಂದ
ಪ್ರಾಂತವನ್ನು
ಪ್ರಾಂತಾಧಿಕಾರಿಯಾಗಿ
ಪ್ರಾಂತೀಯ
ಪ್ರಾಂತ್ಯ
ಪ್ರಾಂತ್ಯಕ್ಕೆ
ಪ್ರಾಂತ್ಯಗಳ
ಪ್ರಾಂತ್ಯಗಳಲ್ಲಿ
ಪ್ರಾಂತ್ಯದ
ಪ್ರಾಂತ್ಯದಲ್ಲಿ
ಪ್ರಾಂತ್ಯದಿಂದ
ಪ್ರಾಂತ್ಯವನ್ನಾಗಿ
ಪ್ರಾಕೃತಿ
ಪ್ರಾಗಿತಿಹಾಸ
ಪ್ರಾಗೈತಿಹಾಸಿಕ
ಪ್ರಾಚೀನ
ಪ್ರಾಚೀನತೆ
ಪ್ರಾಚೀನತೆಯನ್ನು
ಪ್ರಾಚೀನವಾದ
ಪ್ರಾಚ್ಯವಸ್ತು
ಪ್ರಾಣತ್ಯಾಗ
ಪ್ರಾಣದೇವರ
ಪ್ರಾಣಾಧಿಕಾರಿಗಳೂ
ಪ್ರಾಣಾರ್ಪಣೆ
ಪ್ರಾಣಿಗಳಿಗೂ
ಪ್ರಾಣಿಗಳು
ಪ್ರಾತಿನಿಧ್ಯ
ಪ್ರಾದುದಭೂದ್ಗುಣಾಢ್ಯೋನಾಮ್ನಾ
ಪ್ರಾದೇಶಿಕ
ಪ್ರಾಪ್ತನಾಗುತ್ತಾನೆ
ಪ್ರಾಪ್ತನಾದನೆಂದು
ಪ್ರಾಪ್ತನಾದಾಗ
ಪ್ರಾಪ್ತವಾಗಿರುವುದನ್ನು
ಪ್ರಾಪ್ತವಾದ
ಪ್ರಾಪ್ತವಾಯಿತೋ
ಪ್ರಾಪ್ತೈಃ
ಪ್ರಾಬಲ್ಯ
ಪ್ರಾಭವ
ಪ್ರಾಮುಖ್ಯವಾಗಿ
ಪ್ರಾಯಶಃ
ಪ್ರಾಯಶ್ಚಿತ್ತ
ಪ್ರಾಯಶ್ಚಿತ್ತವನ್ನು
ಪ್ರಾರಂಭದ
ಪ್ರಾರಂಭವಾಗಿರುವ
ಪ್ರಾರಂಭವಾಗಿರುವುದು
ಪ್ರಾರಂಭವಾಗುತ್ತದೆ
ಪ್ರಾರಂಭವಾಯಿತೆಂದು
ಪ್ರಾರಂಭಿಸಿದಾಗ
ಪ್ರಾರ್ಥಿಸಲು
ಪ್ರಾರ್ಥಿಸಿ
ಪ್ರಿಥುವೀ
ಪ್ರಿಥ್ವೀ
ಪ್ರಿಯತನಯರು
ಪ್ರಿಯತನಯರೂ
ಪ್ರಿಯನೂ
ಪ್ರಿಯಪುತ್ರಂ
ಪ್ರಿಯರಾಗಿ
ಪ್ರಿಯವಾದ
ಪ್ರಿಯಸುತ
ಪ್ರಿಯಸುತನೆಂದು
ಪ್ರಿಯಸೇವಕನಾದ
ಪ್ರೀಣನ
ಪ್ರೀತಿಯ
ಪ್ರೀತಿಯಿಂದ
ಪ್ರೀತ್ಯರ್ಥವಾಗಿ
ಪ್ರುಥ್ವೀ
ಪ್ರುಥ್ವೀರಾಜ್ಯಂಗೆಯುತ್ತಿರಲು
ಪ್ರುಥ್ವೀವಲ್ಲಭ
ಪ್ರೇಮಂ
ಪ್ರೇಮಾಲಯಸುತ
ಪ್ರೇರೇಪಿಸದವನು
ಪ್ರೇರೇಪಿಸಿದವನು
ಪ್ರೊ
ಪ್ರೊಃ
ಪ್ರೌಢದೇವರಾಯ
ಪ್ರೌಢದೇವರಾಯನ
ಪ್ರೌಢದೇವರಾಯನು
ಪ್ರೌಢದೇವರಾಯಮಲ್ಲಿಕಾರ್ಜುನನನಿಗೆ
ಪ್ರೌಢಪ್ರತಾಪ
ಪ್ರೌಢಪ್ರಧಾನ
ಪ್ರೌಢರೇಖಾ
ಪ್ಲವ
ಫರಿಷ್ತಾನು
ಫರ್ಲಾಂಗ್
ಫಲ
ಫಲಂ
ಫಲಪ್ರದಂ
ಫಲಪ್ರದವಾಗಿ
ಫಲವತ್ತತೆಯನ್ನೂ
ಫಲವತ್ತಾದ
ಫಲವತ್ತಾದುದು
ಫಲಾಕೃತೇಃ
ಫಲಾತಿಶಯಃ
ಫಸಲಿನ
ಫಾತಿಮಾ
ಫಾದರ್
ಫಾದರ್ಹೆರಾಸ್
ಫಿಲಿಯೋಜಾ
ಫೆಬ್ರವರಿ
ಫೌಜ್ದಾರಿಯ
ಫೌಜ್ದಾರ್
ಫ್ರಾನ್ಸಿಸ್
ಫ್ರೆಂಚರ
ಫ್ರೆಂಚರು
ಫ್ರೆಂಚ್ರಾಕ್ಸ್
ಫ್ರೆಂಚ್ರಾಕ್ಸ್ನಲ್ಲಿ
ಫ್ರೆಂಚ್ರಾಕ್ಸ್ನಲ್ಲಿದ್ದ
ಫ್ರೆಂಚ್ರಾಕ್ಸ್ಹಿರೋಡೆ
ಫ್ಲೀಟ್
ಬ
ಬಂಕನಹಳ್ಳಿ
ಬಂಕಾಪುರಕ್ಕೆ
ಬಂಕಾಪುರದ
ಬಂಕಾಪುರದಲ್ಲಿ
ಬಂಕಾಪುರದಲ್ಲಿದ್ದ
ಬಂಕಾಪುರದಿಂದ
ಬಂಕಾಪುರವೋ
ಬಂಕಿನಾಡ
ಬಂಕಿನಾಡನ್ನು
ಬಂಕಿನಾಡು
ಬಂಕೆಯನ
ಬಂಕೆಯನನ್ನು
ಬಂಕೆಯನಿಗೆ
ಬಂಕೆಯನು
ಬಂಕೆಯುನು
ಬಂಕೇಶನ
ಬಂಕೇಶನು
ಬಂಕೇಶನೇ
ಬಂಗಲಿ
ಬಂಗಾರದ
ಬಂಙ್ಕೆಯನು
ಬಂಟ
ಬಂಟರಬಾವನುಂ
ಬಂಟರಭಾವ
ಬಂಟರಭಾವನುಂ
ಬಂಡಮಾರನಹಳ್ಳಿ
ಬಂಡವಾಳ
ಬಂಡಾಯ
ಬಂಡಿ
ಬಂಡಿಹೊಳೆ
ಬಂಡಿಹೊಳೆಮಡುವಿನಕೋಡಿ
ಬಂಡಿಹೊಳೆಯ
ಬಂಡೂರು
ಬಂಡೆಗಳಿಂದ
ಬಂಡೆದ್ದಿದ್ದ
ಬಂಡೆದ್ದು
ಬಂಡೆಯ
ಬಂಣಗಟ್ಟ
ಬಂಣಗಟ್ಟಿ
ಬಂತು
ಬಂದ
ಬಂದಣಿಕೆ
ಬಂದದ್ದು
ಬಂದನಂತರ
ಬಂದನಂತರವೂ
ಬಂದನು
ಬಂದನೆಂದು
ಬಂದನೆಂಬುದು
ಬಂದರು
ಬಂದರೂ
ಬಂದಲ್ಲಿ
ಬಂದವರಾಗಿದ್ದು
ಬಂದವರು
ಬಂದವರೆಂದು
ಬಂದವರೇ
ಬಂದವು
ಬಂದಾಗ
ಬಂದಿತಂತೆ
ಬಂದಿತು
ಬಂದಿತೆಂದು
ಬಂದಿತೆಂದೂ
ಬಂದಿತ್ತು
ಬಂದಿತ್ತೆಂದು
ಬಂದಿತ್ತೆಂಬುದು
ಬಂದಿದೆ
ಬಂದಿದ್ದ
ಬಂದಿದ್ದನು
ಬಂದಿದ್ದನೆಂದು
ಬಂದಿದ್ದರೂ
ಬಂದಿದ್ದಾಗ
ಬಂದಿದ್ದಾರೆ
ಬಂದಿದ್ದು
ಬಂದಿರಬಹುದು
ಬಂದಿರುವ
ಬಂದಿರುವುದು
ಬಂದು
ಬಂದುದನ್ನು
ಬಂದುದು
ಬಂದೊಡನೆ
ಬಂಧನದಲ್ಲಿರಿಸಿದ್ದನಷ್ಟೆ
ಬಂಧಿಯಾಗಿದ್ದ
ಬಂಧಿಯಾಗಿದ್ದಾಗ
ಬಂಧಿಸಿ
ಬಂಧಿಸಿದಾಗ
ಬಂಧುಗಳು
ಬಂಧುಜನಂಗಳು
ಬಂಧುಜನಧವಳ
ಬಂಧುಬಾಂಧವರು
ಬಂನಿಯೂರ
ಬಂನೂರು
ಬಂಮಚ
ಬಕರಿಪು
ಬಕಾಡೇಹಳ್ಳಿ
ಬಗೆಗಿನ
ಬಗೆಗೆ
ಬಗೆಯ
ಬಗೆಯನ್ನು
ಬಗೆಯೊಡೆ
ಬಗೆಹರಿಸುತ್ತಾನೆ
ಬಗ್ಗವಳ್ಳಿ
ಬಗ್ಗವಳ್ಳಿಯನ್ನು
ಬಗ್ಗು
ಬಗ್ಗೆ
ಬಗ್ಗೆಯೂ
ಬಟ್ಟಲಿನ
ಬಟ್ಟಲುಗಳನ್ನು
ಬಡಗರನಾಡ
ಬಡಗರನಾಡು
ಬಡಗರೆ
ಬಡಗರೆನಾಡ
ಬಡಗರೆನಾಡೊಳಗಣ
ಬಡಗಲು
ಬಡಗುಂಡ
ಬಡಗುಂದ
ಬಡಗುಂದನಾಡ
ಬಡಗುಂದನಾಡನ್ನು
ಬಡಗುಡನಾಡ
ಬಡಗುಡನಾಡು
ಬಡಗುನಾಡು
ಬಡಗೆರೆ
ಬಡಗೆರೆನಾಡ
ಬಡಗೆರೆನಾಡಿನವರೊಡನೆ
ಬಡಗೆರೆನಾಡು
ಬಡಗೆರೆನಾಡೊಳಗಣ
ಬಡವನಾದೆ
ಬಡವಾರ
ಬಡಿಕೋಲ
ಬಡಿಕೋಲಭಟ್ಟ
ಬಡಿಯಬೇಕೆಂಬ
ಬಡುವಾರ
ಬಡ್ತಿ
ಬಡ್ತಿಯನ್ನು
ಬಣಜಿಗ
ಬಣ್ಣಂಗಟ್ಟಿಬನ್ನಂಗಾಡಿ
ಬಣ್ಣಂಗಟ್ಟಿಯು
ಬಣ್ಣಿಗದೆರೆಹಳ್ಳಿಯನ್ನು
ಬಣ್ಣಿಸಲಾಗಿದೆ
ಬಣ್ಣಿಸಿದೆ
ಬಣ್ಣಿಸುತ್ತದೆ
ಬಣ್ಣಿಸೆ
ಬಣ್ನಿಗದರೆಯನ್ನು
ಬಣ್ನಿಸಲ್ಬಿಂಡಿಗವಿಲೆಯೊಳಾ
ಬತ್ತದ
ಬದನಗುಪ್ಪೆ
ಬದಲಾಗಿ
ಬದಲಾಯಿತು
ಬದಲಾಯಿಸಲಾಯಿತು
ಬದಲಾಯಿಸಿ
ಬದಲಾಯಿಸಿಕೊಂಡು
ಬದಲಾವಣೆಗಳೊಂದಿಗೆ
ಬದಲಾವಣೆಯಾದವು
ಬದಲಿಗೆ
ಬದಲಿಸುತ್ತಾನೆ
ಬದಲು
ಬದಿಗಿರಿಸಿ
ಬದಿಗೊತ್ತಿ
ಬದಿಗೊತ್ತಿದನು
ಬದಿಯ
ಬದುಕಿದ್ದನೆಂದು
ಬದುಕಿದ್ದರೆಂದು
ಬದುಕಿದ್ದಾಗಲೇ
ಬದುಕಿದ್ದಿರಬಹುದು
ಬದುಕಿದ್ದು
ಬದುಕಿರುವಾಗಲೇ
ಬದ್ದೆಗ
ಬದ್ದೆಗನ
ಬದ್ದೆಗನು
ಬನದ
ಬನವಸೆ
ಬನವಸೆಕಾರರ
ಬನವಸೆಗಳನ್ನು
ಬನವಾಸಿ
ಬನವಾಸಿಪಟ್ಟಣದಲ್ಲಿ
ಬನವಾಸಿಯಲ್ಲಿ
ಬನವಾಸಿಯಿಂದ
ಬನವಾಸೆ
ಬನ್ನಂಗಾಡಿ
ಬನ್ನಿಯೂರ
ಬನ್ನೂರು
ಬಪ್ಪ
ಬಪ್ಪಡೆ
ಬಪ್ಪದೇವಿಯರನ್ನು
ಬಬಿನಾಡಾಳ್ವರು
ಬಬೆಯನಾಡಾಂಕಿಯಾದೀ
ಬಬ್ಬ
ಬಬ್ಬೆಯ
ಬಬ್ಬೆಯನಾಯಕನ
ಬಬ್ಬೆಯನಾಯಕನಿಗೆ
ಬಬ್ಬೆಯನಾಯಕನು
ಬಬ್ಬೆಯನಾಯಕನೆಂಬ
ಬಭೈರಯನಾಯಕ
ಬಮ್ಮ
ಬಮ್ಮಗವುಡನ
ಬಮ್ಮಚ
ಬಮ್ಮಚನಧಿಕಬಳಂ
ಬಮ್ಮಣ
ಬಮ್ಮನು
ಬಮ್ಮನೆಂಬುವವನು
ಬಮ್ಮಲ
ಬಮ್ಮಲದೇವಿ
ಬಮ್ಮಲದೇವಿಯನ್ನು
ಬಮ್ಮಲದೇವಿಯು
ಬಮ್ಮಲೆಯ
ಬಮ್ಮವ್ವೆ
ಬಯಲಮಾರ್ತಾಂಡ
ಬಯಲಲ್ಲಿ
ಬಯಲಹುಲಿ
ಬಯಲಿನ
ಬಯಲಿನವರೆಗೂ
ಬಯಲ್ನಾಡ
ಬಯಲ್ನಾಡನಂ
ಬಯಲ್ನಾಡು
ಬಯಸುತ್ತಾರೆ
ಬಯಸುವ
ಬಯಿಚಕ್ಕ
ಬಯಿಚಣ್ಣ
ಬಯಿಚೆಯ
ಬಯಿರ
ಬಯಿರರಸ
ಬಯಿರರಾಜ
ಬಯಿರೆಯ
ಬಯಿರೆಯದಂಡನಾಯಕನ
ಬಯ್ಯಪ್ಪನಾಯಕರ
ಬರಗಾಲ
ಬರಡು
ಬರಡುಭೂಮಿಯಾಗಿತ್ತು
ಬರಮಣ್ಣ
ಬರಲಾಗಿದೆ
ಬರಲಿಲ್ಲ
ಬರಹ
ಬರಹಗಾರರಾದ
ಬರಹದ
ಬರಹವಿದೆ
ಬರಹವಿಲ್ಲದ
ಬರೀದಸಪ್ತಾಂಗಹರಣ
ಬರುತ್ತದೆ
ಬರುತ್ತದೆಂದು
ಬರುತ್ತವೆ
ಬರುತ್ತಾನೆ
ಬರುತ್ತಾರೆ
ಬರುತ್ತಿದ್ದ
ಬರುತ್ತಿದ್ದನೆಂಬುದು
ಬರುತ್ತಿದ್ದರು
ಬರುತ್ತಿದ್ದರೆಂದೂ
ಬರುತ್ತಿದ್ದರೆಂಬುದು
ಬರುತ್ತಿದ್ದವೆಂದು
ಬರುತ್ತಿರುವಾಗ
ಬರುತ್ತಿಲ್ಲ
ಬರುತ್ತೀನಿ
ಬರುತ್ತೆ
ಬರುವ
ಬರುವಂತೆ
ಬರುವವನನ್ನು
ಬರುವಾಗ
ಬರುವುದರಿಂದ
ಬರುವುದಿಲ್ಲ
ಬರುವುದೇ
ಬರೂಲ್
ಬರೆದ
ಬರೆದನೆಂದೂ
ಬರೆದಿದೆ
ಬರೆದಿದ್ದ
ಬರೆದಿದ್ದಾನೆ
ಬರೆದಿರುತ್ತಾನೆ
ಬರೆದಿರುತ್ತಾರೆ
ಬರೆದಿರುವ
ಬರೆದಿರುವುದನ್ನು
ಬರೆದು
ಬರೆದುದಕ್ಕೆ
ಬರೆಯಲಾಗಿದೆ
ಬರೆಯಿಸಿಕೊಂಡಿದ್ದಾನೆ
ಬರೆಯಿಸಿದ್ದು
ಬರೆಯುತ್ತಾನೆ
ಬರೆಯುತ್ತಿದ್ದರು
ಬರೆಯುವಾಗ
ಬರೆಯುವುದು
ಬರೆಸಿದನೆಂದು
ಬರೆಸಿದರು
ಬರ್ತೀನಿ
ಬರ್ಮಯ್ಯ
ಬರ್ಮಯ್ಯನನ್ನು
ಬರ್ಮಯ್ಯನಾಯಕನ
ಬರ್ಮ್ಮಯ್ಯ
ಬರ್ಮ್ಮಯ್ಯನು
ಬಲಂಬುತೀರ್ಥದಲ್ಲಿ
ಬಲಗಯ್ಯ
ಬಲಗೈ
ಬಲಗೈಬಲಭಾಗ
ಬಲಗೈಯ
ಬಲಗೈಯಸೇನಾಧಿಪತಿ
ಬಲಙ್ಗಳನಟ್ಟಿಮುಟ್ಟಿ
ಬಲದ
ಬಲದಕಯ್ಯ
ಬಲದೇವಣ್ಣ
ಬಲದೇವನು
ಬಲಪಡಿಸುವ
ಬಲಭಾಗದ
ಬಲಭಾಗದಲ್ಲಿರುವ
ಬಲಮುರಿ
ಬಲಮುರಿಯ
ಬಲರನ್ನು
ಬಲರಾಮಕೃಷ್ಣರಂತೆ
ಬಲವಂಕ
ಬಲವಂಕದಲ್ಲಿ
ಬಲವಂಕಪ್ಪ
ಬಲವನ್ನು
ಬಲಸಮುದ್ರ
ಬಲಿಗೆಯ್ದರಂತೆ
ಬಲಿದಾನ
ಬಲಿಪೀಠ
ಬಲಿಯಕೆರೆಯನ್ನು
ಬಲಿಷ್ಠರಾದ
ಬಲೀಂದ್ರ
ಬಲು
ಬಲುಫೌಜು
ಬಲುಮನುಷ
ಬಲುಮನುಷ್ಯ
ಬಲೆಯನಾಯಕರ
ಬಲ್ಲಪ
ಬಲ್ಲಪಂ
ಬಲ್ಲಪನು
ಬಲ್ಲಪನೇ
ಬಲ್ಲಪ್ಪ
ಬಲ್ಲಪ್ಪದಂಡನಾಯಕನ
ಬಲ್ಲಪ್ಪನು
ಬಲ್ಲಪ್ಪಬಿಲ್ಲಪ್ಪ
ಬಲ್ಲಯನ
ಬಲ್ಲಯ್ಯ
ಬಲ್ಲಯ್ಯನ
ಬಲ್ಲಯ್ಯನಯ್ಯನೀ
ಬಲ್ಲಯ್ಯನು
ಬಲ್ಲಯ್ಯನೇ
ಬಲ್ಲವರು
ಬಲ್ಲಹಂ
ಬಲ್ಲಹನ
ಬಲ್ಲಹನು
ಬಲ್ಲಾನು
ಬಲ್ಲಾಳ
ಬಲ್ಲಾಳಈ
ಬಲ್ಲಾಳದಾಸರ
ಬಲ್ಲಾಳದೇವಂ
ಬಲ್ಲಾಳದೇವನ
ಬಲ್ಲಾಳದೇವನತ್ಯಂತವಾಗಿ
ಬಲ್ಲಾಳದೇವನಿಗೆ
ಬಲ್ಲಾಳದೇವನು
ಬಲ್ಲಾಳದೇವನೊಡನೆ
ಬಲ್ಲಾಳದೇವರ
ಬಲ್ಲಾಳದೇವರಸನು
ಬಲ್ಲಾಳದೇವರಸರ
ಬಲ್ಲಾಳದೇವರಸರು
ಬಲ್ಲಾಳದೇವರು
ಬಲ್ಲಾಳನ
ಬಲ್ಲಾಳನದಲ್ಲಿ
ಬಲ್ಲಾಳನಪುರವಾದ
ಬಲ್ಲಾಳನಲ್ಲಿ
ಬಲ್ಲಾಳನಲ್ಲಿದ್ದರು
ಬಲ್ಲಾಳನವರೆಗಿನ
ಬಲ್ಲಾಳನಾಗುತ್ತಾನೆ
ಬಲ್ಲಾಳನಿಗೆ
ಬಲ್ಲಾಳನು
ಬಲ್ಲಾಳನೂ
ಬಲ್ಲಾಳನೆಂದು
ಬಲ್ಲಾಳನೇ
ಬಲ್ಲಾಳಪುರ
ಬಲ್ಲಾಳಪುರದ
ಬಲ್ಲಾಳಪುರದಲ್ಲಿ
ಬಲ್ಲಾಳಪುರಸ್ಥಳ
ಬಲ್ಲಾಳಭೂಪಾಳಂ
ಬಲ್ಲಾಳಮಹೀಕಾಂತನ
ಬಲ್ಲಾಳಮಹೀಪಾಲ
ಬಲ್ಲಾಳಮಹೀಪಾಲಯಂ
ಬಲ್ಲಾಳರಾಯನ
ಬಲ್ಲಾಳರಾಯನಿಗೆ
ಬಲ್ಲಾಳರಾಯ್ಯ
ಬಲ್ಲಾಳು
ಬಲ್ಲಾಳೇಶ್ವರ
ಬಲ್ಲೆಕೆರೆ
ಬಲ್ಲೆಯ
ಬಲ್ಲೆಯನಾಯಕ
ಬಲ್ಲೆಯನಾಯಕನ
ಬಲ್ಲೆಯನಾಯಕನು
ಬಲ್ಲೆಯಬಲ್ಲಪ್ಪ
ಬಲ್ಲೇಗೌಡ
ಬಲ್ಲೇನಪಲ್ಲಿ
ಬಲ್ಲೇನಹಳ್ಳಿ
ಬಲ್ಲೇನಹಳ್ಳಿಯಾಗಿದೆ
ಬಳಕೆ
ಬಳಕೆಗೊಂಡಿವೆ
ಬಳಕೆಯ
ಬಳಕೆಯಾಗಿದೆ
ಬಳಕೆಯಾಗಿರುವುದು
ಬಳಕೆಯಾಗಿವೆ
ಬಳಗಾರ
ಬಳಗುಂದಿಯ
ಬಳಗುಳ
ಬಳಗುಳಕ್ಕೆ
ಬಳಗುಳದ
ಬಳಗುಳದಲ್ಲಿ
ಬಳಗುಳವನ್ನು
ಬಳಗೊಳ
ಬಳಗೊಳದಲ್ಲಿ
ಬಳಗೊಳವು
ಬಳಘಟ್ಟವಾಗಿದೆ
ಬಳಪದಕಲ್ಲುಮಂಟಿ
ಬಳಮಡುನಾಡಿನಲ್ಲಿತ್ತೆಂದು
ಬಳಮಡುನಾಡು
ಬಳಸಲಾಗಿದೆ
ಬಳಸಲಾಗಿದೆಯೇ
ಬಳಸಿಕೊಂಡಿದ್ದಾರೆ
ಬಳಸಿಕೊಳ್ಳಲಾಗಿದೆ
ಬಳಸಿದ್ದಾನೆ
ಬಳಸಿರುವುದರಿಂದ
ಬಳಸಿಲ್ಲ
ಬಳಸುವ
ಬಳಸುವವನನ್ನು
ಬಳಿ
ಬಳಿಕ
ಬಳಿಕುಂದೂರು
ಬಳಿಗ
ಬಳಿಗಗಟ್ಟದ
ಬಳಿಯ
ಬಳಿಯೂ
ಬಳಿಯೆ
ಬಳಿಯೇ
ಬಳುವಳಿಯಾಗಲ್ಲ
ಬಳುವಳಿಯಾಗಿ
ಬಳ್ಳಾರಿ
ಬಳ್ಳಿಯಕೆರೆ
ಬಳ್ಳೆಗೊಳ
ಬಳ್ಳೆಗೊಳಕ್ಕೆ
ಬಳ್ಳೆಗೊಳದ
ಬಳ್ಳೇಕೆರೆ
ಬವರದಲ್ಲಿ
ಬಸದಿ
ಬಸದಿಗಳ
ಬಸದಿಗಳನ್ನು
ಬಸದಿಗಳಿಗೆ
ಬಸದಿಗಳೂ
ಬಸದಿಗೆ
ಬಸದಿಗೆಶಾಸನಬಸದಿ
ಬಸದಿಯ
ಬಸದಿಯನ್ನು
ಬಸದಿಯನ್ನುಎರಡುಕಟ್ಟೆ
ಬಸದಿಯಲ್ಲಿದೆ
ಬಸದಿಯಲ್ಲಿರುವ
ಬಸದಿಯಾಗಿತ್ತೆಂದು
ಬಸದಿಯಾಗಿರಬಹುದು
ಬಸದಿಯು
ಬಸರಾಳನ್ನು
ಬಸರಾಳಿನ
ಬಸರಾಳಿನಲ್ಲಿ
ಬಸರಾಳು
ಬಸರುವಾಣು
ಬಸವ
ಬಸವಂತ
ಬಸವಂತಪಟ್ಟಣವನ್ನು
ಬಸವಂತಪಟ್ಟಣವಾಗಿರಬಹುದು
ಬಸವಗವುಡನ
ಬಸವಟ್ಟಿಗೆ
ಬಸವಣ್ಣನವರು
ಬಸವನ
ಬಸವನಕೋಟೆ
ಬಸವನಬೆಟ್ಟ
ಬಸವನಬೆಟ್ಟದ
ಬಸವನಹಳ್ಳಿ
ಬಸವನು
ಬಸವನೂ
ಬಸವಪಟ್ಟಣದ
ಬಸವಪ್ಪ
ಬಸವಪ್ಪನಾಯಕನು
ಬಸವಮಾತ್ಯ
ಬಸವಮಾತ್ಯನ
ಬಸವಮಾತ್ಯನು
ಬಸವಮಾತ್ಯಬಸವರಸ
ಬಸವಯ್ಯ
ಬಸವರಸ
ಬಸವರಸನ
ಬಸವರಸನೂ
ಬಸವರಾಜಯ್ಯದೇವ
ಬಸವಾಪಟ್ಟಣದ
ಬಸವಾಪಟ್ಟಣವನ್ನು
ಬಸವಾಪುರ
ಬಸವಿಯಕ್ಕ
ಬಸುರಬಂದ
ಬಸುರಿವಾಳ
ಬಸುರಿವಾಳದೊಳು
ಬಸುರಿವಾಳು
ಬಸುರುವಾಣು
ಬಸುರುವಾಣುಸ್ಥಳದ
ಬಸ್ತಿ
ಬಸ್ತಿಯಲ್ಲಿ
ಬಸ್ತಿಯಹಳ್ಳಿ
ಬಸ್ತಿಹಳ್ಳಿ
ಬಸ್ತಿಹಳ್ಳಿಯಲ್ಲಿರುವ
ಬಸ್ತೀಪುರ
ಬಹತ್ತರ
ಬಹದ್ದೂರ್
ಬಹಮನಿ
ಬಹಲ್ಲಿ
ಬಹಳ
ಬಹಳವಾಗಿ
ಬಹಾದೂರ್
ಬಹಾದ್ದೂರನು
ಬಹಾದ್ದೂರು
ಬಹಾದ್ದೂರ್
ಬಹಿತ್ರ
ಬಹಿತ್ರದ
ಬಹಿತ್ರದಂತೆ
ಬಹಿತ್ರರು
ಬಹು
ಬಹುಕಾಲ
ಬಹುತೆಕ
ಬಹುತೇಕ
ಬಹುದು
ಬಹುದೊಡ್ಡ
ಬಹುಪಾಲು
ಬಹುಭಾಗ
ಬಹುಭಾಗದಲ್ಲಿ
ಬಹುಮಟ್ಟಿಗೆ
ಬಹುರಾಜ್ಯಕಾರ್ಯ್ಯಂ
ಬಹುಳ
ಬಹುವಾಗಿ
ಬಹುಶ
ಬಹುಶಃ
ಬಹುಸಂಖ್ಯೆಯಲ್ಲಿ
ಬಹುಸಂಖ್ಯೆಯಲ್ಲಿದ್ದು
ಬಾಂಧವ್ಯಕ್ಕೆ
ಬಾಂಧವ್ಯವಿದ್ದಿತು
ಬಾಗಣಬ್ಬೆ
ಬಾಗಣಬ್ಬೆಯ
ಬಾಗಣಬ್ಬೆಯು
ಬಾಗಸೆಟ್ಟಿಹಳ್ಳಿಗಳನ್ನು
ಬಾಗಿನಾಡೆಪ್ಪತ್ತು
ಬಾಗಿಲಿಗೆ
ಬಾಗಿಲಿನ
ಬಾಗಿಲುವಾಡವನ್ನು
ಬಾಗಿವಾಳನ್ನು
ಬಾಗಿವಾಳು
ಬಾಗೆ
ಬಾಗೆನಾಡುನ್ನು
ಬಾಚಗವುಡ
ಬಾಚಜೀಯಂಗೆ
ಬಾಚನಹಳ್ಳಿಯೇ
ಬಾಚಪಟ್ಟಣದ
ಬಾಚಪ್ಪ
ಬಾಚಪ್ಪನ
ಬಾಚಪ್ಪನಕೆರೆ
ಬಾಚಪ್ಪನನ್ನು
ಬಾಚಪ್ಪನೇ
ಬಾಚಹಳಿಯ
ಬಾಚಹಳ್ಳಿ
ಬಾಚಹಳ್ಳಿಗೆ
ಬಾಚಹಳ್ಳಿಯ
ಬಾಚಹಳ್ಳಿಯನ್ನು
ಬಾಚಿಗ
ಬಾಚಿಪಟ್ಟಣವನು
ಬಾಚಿಪಟ್ಟಣವಾಗಿರಬಹುದು
ಬಾಚಿಯಪ್ಪ
ಬಾಚಿಯಪ್ಪನ
ಬಾಚಿಯಪ್ಪನವಿಗೆ
ಬಾಚಿಯಪ್ಪನಿಗೆ
ಬಾಚಿಯಪ್ಪನು
ಬಾಚಿಯಹಳ್ಳಿಯ
ಬಾಚಿರಾಜನ
ಬಾಚಿರಾಜನಂ
ಬಾಚಿರಾಜಾಭಿದಾನಃ
ಬಾಚಿರಾಜಾಹ್ವಯ
ಬಾಚಿಹಳಿಯ
ಬಾಚಿಹಳ್ಳಿಯ
ಬಾಚಿಹಳ್ಳಿಯನು
ಬಾಚಿಹಳ್ಳಿಯನ್ನು
ಬಾಚೆಯ
ಬಾಚೆಯಹಳ್ಳಿಯ
ಬಾಚೆಹಳ್ಳಿಯ
ಬಾಚೆಹಳ್ಳಿಯು
ಬಾಜದನಮಲಕ
ಬಾಜದನಮಲುಕ
ಬಾಣ
ಬಾಣಕುಲಕಲಾಕಲಃ
ಬಾಣಕುಲದ
ಬಾಣದ
ಬಾಣದಿಂದ
ಬಾಣರ
ಬಾಣರಮೇಲೆ
ಬಾಣರಸರ
ಬಾಣರು
ಬಾಣವಂಶದ
ಬಾಣವಂಶೋದ್ಭವ
ಬಾಣವಂಶೋದ್ಭವನಾದ
ಬಾಣವಾಡಿಯಲ್ಲಿ
ಬಾಣಸಂದಾಪುರವನ್ನು
ಬಾಣಾವರದಲ್ಲಿ
ಬಾದಶಹನನ್ನಾಗಿ
ಬಾದಶಹರು
ಬಾದಶಹಾ
ಬಾದಶಾಹ
ಬಾದಾಮಿ
ಬಾಪ್ಪೆಂಬಿನಂ
ಬಾಬ
ಬಾಬಚಾಮುಂಡರಾಯ
ಬಾಬಚಾವುಂಡರಾಯ
ಬಾಬಸಿಂಘೂರಹರು
ಬಾಬುಸೆಟ್ಟಿಯು
ಬಾಬೆಯನಾಯಕ
ಬಾಮುಲ
ಬಾಯಾನುಮತದಿಂದ
ಬಾಯಿದೇವಿ
ಬಾಯೊಳಕ್ಕೆ
ಬಾಯ್ಬೆಣ್ಣೆಗೆ
ಬಾರಂದರ
ಬಾರಕೂರು
ಬಾರಾ
ಬಾರಾಗೋಪಾಲ್
ಬಾರಾಗೋಪಾಲ್ರವರು
ಬಾರಿ
ಬಾರಿಕರು
ಬಾಲಕನಾಗಿ
ಬಾಲಕನಾಗಿದ್ದುದರಿಂದ
ಬಾಲಚಂದ್ರ
ಬಾಲಶಿಕ್ಷೆ
ಬಾಲಶಿಕ್ಷೆಯನ್ನೂ
ಬಾಲಾತಪಂ
ಬಾಲೂರು
ಬಾಲ್ಯವೆಲ್ಲ
ಬಾಳಗುಂಚಿ
ಬಾಳಿದನೆಂದು
ಬಾಳಿದರೆಂದು
ಬಾಳೆ
ಬಾಳೆಹಳ್ಳಿ
ಬಾಳ್ಗಚ್ಚನ್ನು
ಬಾಳ್ಗಚ್ಚಾಗಿ
ಬಾಳ್ಗಳ್ಚಾಗಿ
ಬಾವ
ಬಾವಿಯನ್ನು
ಬಾವಿಸೆಟ್ಟಿ
ಬಾಷೆಯ
ಬಾಸಣಯ್ಯಂ
ಬಾಸಣಯ್ಯನ
ಬಾಸಣಯ್ಯನೆಂಬುವವನು
ಬಾಸಿಮಯ್ಯನುಬಾಸಣಯ್ಯ
ಬಾಸೆಗೆ
ಬಾಸೆಯ
ಬಾಸೆಯನ್ನು
ಬಾಹತ್ತರ
ಬಾಹತ್ತರನಿಯೋಗಾಧಿಪತಿಪದವಿಗಳನ್ನು
ಬಾಹುಃ
ಬಾಹುತು
ಬಾಹುಬಲಿ
ಬಾಹುಬಲಿಗಳು
ಬಾಹುಬಲಿಯ
ಬಾಹುಬಲಿಯರ
ಬಾಹುಸೌರ್ಯ್ಯಂ
ಬಿಂಕಂ
ಬಿಂಡಿಗನವಿಲೆ
ಬಿಂಡಿಗನವಿಲೆಯ
ಬಿಂಡಿಗನವಿಲೆಯೊಳಗೆ
ಬಿಂಡೇನಹಳ್ಳಿ
ಬಿಂದಾರಪತಿ
ಬಿಂದಾರೊಪತಿ
ಬಿಂನಹ
ಬಿಂಮನಾಯಕ
ಬಿಎಲ್
ಬಿಎಲ್ರೈಸ್
ಬಿಎಲ್ರೈಸ್ರವರು
ಬಿಎಲ್ರೈಸ್ರವರೂ
ಬಿಎಸಾಲೆತೂರ್
ಬಿಕಸಮುದ್ರದ
ಬಿಕೆಯನಾಯಕ
ಬಿಕೆಯನಾಯಕನೂ
ಬಿಕೆಯನಾಯಕರು
ಬಿಕ್ಕಸಂದ್ರ
ಬಿಕ್ಕಸಮುದ್ರ
ಬಿಗಡಾಯಿಸಿರಲಿಲ್ಲವೆಂದು
ಬಿಜಯಂಗೈದು
ಬಿಜಯಂಗೈಯುತ್ತಿರ್ದು
ಬಿಜಯಮಾಡಿ
ಬಿಜಾಪುರದ
ಬಿಜೆಯಮಾಡಿದ್ದ
ಬಿಜ್ಜಲದೇವಿಯರ
ಬಿಜ್ಜಲಾಪುರಹಾನುಗಲ್ಲು
ಬಿಜ್ಜಲೇಶ್ವರಪುರವಾದ
ಬಿಜ್ಜೈಯನು
ಬಿಟ್ಟ
ಬಿಟ್ಟಂತೆ
ಬಿಟ್ಟಗೊಂಡನಹಳ್ಳಿಯನ್ನು
ಬಿಟ್ಟದತ್ತಿ
ಬಿಟ್ಟದ್ದು
ಬಿಟ್ಟನಾಯಕನಹಳ್ಳಿ
ಬಿಟ್ಟನು
ಬಿಟ್ಟನೆಂದು
ಬಿಟ್ಟನೆಂದೂ
ಬಿಟ್ಟನೆಂಬುದು
ಬಿಟ್ಟರು
ಬಿಟ್ಟರೆ
ಬಿಟ್ಟರೆಂದು
ಬಿಟ್ಟರೆಂದುಹೇಳಿದೆ
ಬಿಟ್ಟರ್
ಬಿಟ್ಟಳೆಂದು
ಬಿಟ್ಟಶಾಸನ
ಬಿಟ್ಟಾಗ
ಬಿಟ್ಟಿ
ಬಿಟ್ಟಿಗನುವಿಷ್ಣ
ಬಿಟ್ಟಿಗಾವುಂಡ
ಬಿಟ್ಟಿಗಾವುಡ
ಬಿಟ್ಟಿಗಾವುಡನ
ಬಿಟ್ಟಿಗಾವುಡನು
ಬಿಟ್ಟಿದದಾನೆ
ಬಿಟ್ಟಿದೇವ
ಬಿಟ್ಟಿದೇವನನ್ನು
ಬಿಟ್ಟಿದೇವನಿಂದ
ಬಿಟ್ಟಿದೇವನು
ಬಿಟ್ಟಿದೇವನೆಂದು
ಬಿಟ್ಟಿದ್ದ
ಬಿಟ್ಟಿದ್ದನು
ಬಿಟ್ಟಿದ್ದನೆಂದು
ಬಿಟ್ಟಿದ್ದನೆಂದೂ
ಬಿಟ್ಟಿದ್ದಾನೆ
ಬಿಟ್ಟಿದ್ದಾನೆಂದು
ಬಿಟ್ಟಿದ್ದಾರೆ
ಬಿಟ್ಟಿದ್ದು
ಬಿಟ್ಟಿದ್ದುದು
ಬಿಟ್ಟಿನು
ಬಿಟ್ಟಿಮಯ್ಯ
ಬಿಟ್ಟಿಮಯ್ಯಗಳ
ಬಿಟ್ಟಿಮಯ್ಯನ
ಬಿಟ್ಟಿಮಯ್ಯನು
ಬಿಟ್ಟಿಮಯ್ಯನೂ
ಬಿಟ್ಟಿಮಯ್ಯರು
ಬಿಟ್ಟಿಯಣ್ಣ
ಬಿಟ್ಟಿಯಣ್ಣನ
ಬಿಟ್ಟಿಯಣ್ಣನೆಂದೂ
ಬಿಟ್ಟಿಯದೇವನು
ಬಿಟ್ಟಿಯು
ಬಿಟ್ಟಿರಬಹುದು
ಬಿಟ್ಟಿರಬಹುದೆಂದು
ಬಿಟ್ಟಿರುತ್ತಾನೆ
ಬಿಟ್ಟಿರುವ
ಬಿಟ್ಟಿರುವಂತೆ
ಬಿಟ್ಟಿರುವುದನ್ನು
ಬಿಟ್ಟಿರುವುದರಿಂದ
ಬಿಟ್ಟಿರುವುದು
ಬಿಟ್ಟೀದೇವ
ಬಿಟ್ಟೀದೇವನುಇಮ್ಮಡಿಬಲ್ಲಾಳ
ಬಿಟ್ಟು
ಬಿಟ್ಟುಕೊಟ್ಟ
ಬಿಟ್ಟುಕೊಡಲು
ಬಿಟ್ಟುಹೋಗಿರುವಂತೆ
ಬಿಟ್ಟುಹೋಗಿವೆ
ಬಿಡದೆ
ಬಿಡಲಾಗಿತ್ತು
ಬಿಡಲಾಗಿತ್ತೆಂದು
ಬಿಡಲಾಗಿದೆ
ಬಿಡಲಾಗಿದ್ೆ
ಬಿಡಲಾಗುತ್ತಿತ್ತು
ಬಿಡಲಾಗುತ್ತಿದ್ದ
ಬಿಡಲಾಯಿತು
ಬಿಡಲಾಯಿತೆಂದು
ಬಿಡಲಾಯಿತೆಂಬ
ಬಿಡಲು
ಬಿಡಿಸಲಾಗಿರುವ
ಬಿಡಿಸಿಕೊಂಡು
ಬಿಡಿಸಿದ
ಬಿಡಿಸಿದನು
ಬಿಡಿಸಿದನೆಂದು
ಬಿಡಿಸಿದರೂ
ಬಿಡಿಸಿದ್ದಾನೆಂದು
ಬಿಡಿಸಿರಬಹುದು
ಬಿಡಿಸಿರುವುದು
ಬಿಡಿಸುತ್ತಾನೆ
ಬಿಡುಗಡೆ
ಬಿಡುತಾರೆ
ಬಿಡುತ್ತಾನೆ
ಬಿಡುತ್ತಾರೆ
ಬಿಡುತ್ತಾಳೆ
ಬಿಡುತ್ತಿದ್ದನು
ಬಿಡುತ್ತಿದ್ದರು
ಬಿಡುವ
ಬಿಡುವಾಗ
ಬಿಣ್ಣಮ್ಮನ
ಬಿತ್ತವಾಟ್ಟಕ್ಕೆ
ಬಿತ್ತುವಟ್ಟಂ
ಬಿತ್ತುವಟ್ಟವನ್ನೂ
ಬಿತ್ತುವಟ್ಟವಾಗಿ
ಬಿತ್ತುವಟ್ಟಾಗಿ
ಬಿತ್ತುವಾಟಕ್ಕೆ
ಬಿತ್ತುವಾಟನ್ನು
ಬಿದಿಗೆ
ಬಿದಿಯರ
ಬಿದಿರಕೋಟೆ
ಬಿದಿರುಕೋಟೆಯನ್ನು
ಬಿದ್ದನು
ಬಿದ್ದನೆಂದು
ಬಿದ್ದಿದ್ದನೆಂದು
ಬಿದ್ದಿರಲು
ಬಿದ್ದು
ಬಿದ್ದುಹೋಗಿ
ಬಿನುಗುದೆರೆ
ಬಿನ್ನಪಂ
ಬಿನ್ನಹ
ಬಿನ್ನಹಮಾಡಿ
ಬಿನ್ನಹಮಾಡಿಕೊಂಡ
ಬಿಮಿಸೆಟ್ಟಿ
ಬಿಯಳಮ್ಮನಿಗೆ
ಬಿರಿದು
ಬಿರಿಯುವಂತೆ
ಬಿರಿಯೆ
ಬಿರುಕು
ಬಿರುದ
ಬಿರುದಂತೆಂಬರ
ಬಿರುದಂತೆಂಬರಗಂಡ
ಬಿರುದನ್ನು
ಬಿರುದನ್ನುಹುದ್ದೆಯನ್ನು
ಬಿರುದರಗೋವ
ಬಿರುದಾಂಕಿತ
ಬಿರುದಾಂಕಿತನಾದ
ಬಿರುದಾದ
ಬಿರುದಾವಳಿ
ಬಿರುದಾವಳಿಗಳನ್ನು
ಬಿರುದಾವಳಿಯನ್ನು
ಬಿರುದಿತ್ತದೆಂದು
ಬಿರುದಿತ್ತು
ಬಿರುದಿತ್ತೆಂದು
ಬಿರುದಿತ್ತೆಂದೂ
ಬಿರುದಿತ್ತೆಂಬುದು
ಬಿರುದಿದೆ
ಬಿರುದಿದ್ದುದು
ಬಿರುದಿದ್ದುನ್ನು
ಬಿರುದಿನ
ಬಿರುದಿನಂತೆ
ಬಿರುದಿರಬಹುದು
ಬಿರುದಿಲ್ಲ
ಬಿರುದು
ಬಿರುದುಗಳ
ಬಿರುದುಗಳನ್ನು
ಬಿರುದುಗಳನ್ನುದ್ಧರಿಸಿ
ಬಿರುದುಗಳನ್ನೂ
ಬಿರುದುಗಳಲ್ಲಿ
ಬಿರುದುಗಳಾಗಿದ್ದು
ಬಿರುದುಗಳಾವುವೂ
ಬಿರುದುಗಳಿಂದ
ಬಿರುದುಗಳಿದ್ದವು
ಬಿರುದುಗಳಿದ್ದು
ಬಿರುದುಗಳಿವೆ
ಬಿರುದುಗಳು
ಬಿರುದುಗಳೂ
ಬಿರುದುಗಳೇ
ಬಿರುದುಬಾವಲಿಗಳನ್ನು
ಬಿರುದುಳ್ಳ
ಬಿರುದೂ
ಬಿರುದೆಂತೆಂಬರ
ಬಿರುದೆಂಬರಗಂಡ
ಬಿರುದೇ
ಬಿರುದೈರ್ವಂದಿತತ್ಯಾನಿತ್ಯಮಭಿಷ್ಟತಃ
ಬಿಲ್ಲಂಗೆರೆಯ
ಬಿಲ್ಲಗೊಂಡನಹಳ್ಳಿ
ಬಿಲ್ಲಪ್ಪ
ಬಿಲ್ಲಬೆಳಗುಂದ
ಬಿಲ್ಲಮೂಲೂನೂರ್ಪ್ಪಬ್ಬರು
ಬಿಲ್ಲಯ್ಯನಿಗೆ
ಬಿಲ್ಲರಾಮನಹಳ್ಳಿ
ಬಿಲ್ಲವರ
ಬಿಲ್ಲವರಿರಬೇಕು
ಬಿಲ್ವಿದ್ಯೆಯಲ್ಲಿ
ಬಿಲ್ಹಣನು
ಬಿಳಿಕಲ್ಲುಮಂಠಿ
ಬಿಳಿಕೆರೆಯನ್ನು
ಬಿಳಿಗೆರೆ
ಬಿಷ್ಣುನೃಪತಿ
ಬಿಸಾಡಿಕೊಪ್ಪಲು
ಬಿಸುಗೆಯ
ಬೀಚನಹಳ್ಳಿ
ಬೀಚವ್ವೆ
ಬೀಚೆಯ
ಬೀಚೇನಹಳ್ಳಿ
ಬೀಡಿನ
ಬೀಡಿನಲ್ಲಿ
ಬೀಡಿನಲ್ಲಿದ್ದಾಗ
ಬೀಡಿನಿಂದ
ಬೀಡಿನೊಳಗೆ
ಬೀಡು
ಬೀಡುಬಿಟ್ಟನು
ಬೀಡುಬಿಟ್ಟಲ್ಲಿ
ಬೀಡುಬಿಟ್ಟಿದ್ದ
ಬೀಡುಬಿಟ್ಟಿದ್ದನೆಂದು
ಬೀಡುಬಿಟ್ಟಿದ್ದಾಗ
ಬೀಡುಬಿಟ್ಟಿರಬಹುದು
ಬೀಡುಬಿಟ್ಟು
ಬೀಡುಬಿಡಲು
ಬೀದಿಯಲ್ಲಿರುವ
ಬೀದಿಯು
ಬೀದಿಯೇ
ಬೀಮಣ್ಣನ
ಬೀರ
ಬೀರಂ
ಬೀರಂಮಲೆಯಲ್ಲಿ
ಬೀರಕ್ಕ
ಬೀರಕ್ಕನ
ಬೀರಗಲ್ಲನ್ನು
ಬೀರಗವುಂಡನು
ಬೀರಗಾವುಂಡನು
ಬೀರನಹಳ್ಳಿ
ಬೀರಯ್ಯ
ಬೀರಯ್ಯನನ್ನು
ಬೀರಯ್ಯನೆಂದು
ಬೀರರಂ
ಬೀರಲಕ್ಷ್ಮಿ
ಬೀರಸೆಟ್ಟಿ
ಬೀರಿಸೆಟ್ಟಿಹಳ್ಳಿ
ಬೀರುಗ
ಬೀರುಬಳ್ಳಿ
ಬೀರುಬಳ್ಳಿಯನ್ನು
ಬೀರುವಳ್ಳಿ
ಬೀರೆಯನಾಯಕ
ಬೀರೆಯನಾಯಕನ
ಬೀರೆಯ್ಯ
ಬೀಳಲು
ಬೀಳವೃತ್ತಿ
ಬೀಳವೃತ್ತಿಯಿಂದ
ಬೀಸುವ
ಬೀೞಅ್ಗುಂಡಿಕ್ಕಿದ
ಬೀೞವೃತ್ತಿ
ಬೀೞವೃತ್ತಿಯಂತಹದೇ
ಬೀೞವೃತ್ತಿಯು
ಬೀೞಾನುವೃತ್ತಿಯಿಂದ
ಬುಕಂಣ
ಬುಕ್ಕ
ಬುಕ್ಕಂಣ
ಬುಕ್ಕಣ್ಣ
ಬುಕ್ಕಣ್ಣನ
ಬುಕ್ಕಣ್ಣನು
ಬುಕ್ಕಣ್ಣವೊಡೆಯರ
ಬುಕ್ಕನ
ಬುಕ್ಕನನ್ನೇ
ಬುಕ್ಕನು
ಬುಕ್ಕನೃಪತಿನೊಳಂದತಿಶಯದಿಂ
ಬುಕ್ಕನೇ
ಬುಕ್ಕಮಾ
ಬುಕ್ಕರಾಜರಾಯಾಬಾಹೂತ
ಬುಕ್ಕರಾಯ
ಬುಕ್ಕರಾಯತನೂಭವ
ಬುಕ್ಕರಾಯನ
ಬುಕ್ಕರಾಯನನ್ನು
ಬುಕ್ಕರಾಯನಿಗೆ
ಬುಕ್ಕರಾಯನು
ಬುಕ್ಕರಾಯನೇ
ಬುಕ್ಕರಾಯರು
ಬುಕ್ಕರಾಯಸಮುದ್ರ
ಬುಕ್ಕರು
ಬುಕ್ನಾನ್
ಬುಧೈಕಕಲ್ಪಭೂ
ಬುರಾನುದ್ದೀನ್
ಬುಳ್ಳಪ್ಪನಾಯಕರ
ಬೂಕನ
ಬೂಕನಕೆರೆ
ಬೂಕಿನ
ಬೂಕಿನಕೆರೆ
ಬೂಕಿನಕೆರೆಯು
ಬೂಚಲೆ
ಬೂಚಿಯಣ್ಣ
ಬೂಚಿರಾಜ
ಬೂಚಿರಾಜನು
ಬೂತಗನು
ಬೂತರಸರು
ಬೂತಾರ್ಯನು
ಬೂತುಗ
ಬೂತುಗನ
ಬೂತುಗನನ್ನು
ಬೂತುಗನರಸಿ
ಬೂತುಗನಿಗೆ
ಬೂತುಗನು
ಬೂತುಗನೇ
ಬೂತುಗನೊಂದಿಗೆ
ಬೂತುಗರ
ಬೂತುಗಸತ್ಯವಾಕ್ಯ
ಬೂದನೂರ
ಬೂದನೂರಾದ
ಬೂವನಹಳ್ಳಿ
ಬೂವನಹಳ್ಳಿಗಳ
ಬೂವನಹಳ್ಳಿಯನ್ನು
ಬೃಹತ್
ಬೆಂಕಿ
ಬೆಂಕಿನವಾಬನೆಂದು
ಬೆಂಕೊಂಡು
ಬೆಂಗಳೂರನ್ನು
ಬೆಂಗಳೂರಿನ
ಬೆಂಗಳೂರು
ಬೆಂಗಿ
ಬೆಂನಟ್ಟಿದಂ
ಬೆಂನೂರ
ಬೆಂನ್ನಂ
ಬೆಂಬಲ
ಬೆಂಬಲಕ್ಕೆ
ಬೆಂಬೆತ್ತಿ
ಬೆಂಬೆತ್ತಿಹೋದನು
ಬೆಕ್ಕದ
ಬೆಗೆಗವುಡನೆಂದು
ಬೆಗೆವಂದಕ್ಕೆ
ಬೆಗೆವಡೆದ
ಬೆಗೆವನ್ದ
ಬೆಟಾಲಿಯನ್ಗೆ
ಬೆಟ್ಟ
ಬೆಟ್ಟಕ್ಕೆ
ಬೆಟ್ಟಗಳ
ಬೆಟ್ಟಗಳಅ
ಬೆಟ್ಟಗಳಿಗೆ
ಬೆಟ್ಟಗಳಿವೆ
ಬೆಟ್ಟಗಳು
ಬೆಟ್ಟಗುಡ್ಡಗಳೂ
ಬೆಟ್ಟದ
ಬೆಟ್ಟದಕೋಟೆ
ಬೆಟ್ಟದಚಾಮರಾಜ
ಬೆಟ್ಟದಚಾಮರಾಜನು
ಬೆಟ್ಟದಪುರ
ಬೆಟ್ಟದಮೇಲೆ
ಬೆಟ್ಟದಲ್ಲಿ
ಬೆಟ್ಟದಲ್ಲಿರುವ
ಬೆಟ್ಟದಹಳ್ಳಿ
ಬೆಟ್ಟದಹಳ್ಳಿಯಿಂದ
ಬೆಟ್ಟದಿಂದ
ಬೆಟ್ಟಮ್ಮೇಲ್ಕಾಲಂ
ಬೆಟ್ಟಯ್ಯ
ಬೆಟ್ಟಹಳ್ಳಿ
ಬೆಟ್ಟಹಳ್ಳಿಯನ್ನು
ಬೆಣಚುಕಲ್ಲಿನ
ಬೆಣ್ಣೆಗೆರೆಯ
ಬೆಣ್ಣೆದೊಣೆಯಲ್ಲಿ
ಬೆಣ್ಣೆಸಿದ್ದನಗುಡ್ಡದ
ಬೆದರಿಸಿ
ಬೆದರೆ
ಬೆದ್ದಲನ್ನು
ಬೆದ್ದಲು
ಬೆದ್ದಲುಗಳನ್ನು
ಬೆದ್ದಲೆಯನ್ನು
ಬೆನಕನಕೆರೆ
ಬೆನ್ನಂ
ಬೆನ್ನಚರ್ಮವೇ
ಬೆನ್ನಟ್ಟಿ
ಬೆನ್ನಹಿಂದೆಯೇ
ಬೆನ್ನಾವರದ
ಬೆನ್ನುಹತ್ತಿ
ಬೆಮತೂರಕಲ್ಲ
ಬೆಮ್ಬಮ್ಪಾಳ್
ಬೆರ್ರಡಿಯಾನ್
ಬೆಲತೂರು
ಬೆಲತ್ತೂರು
ಬೆಲದತಾಲ
ಬೆಲವತ್ತ
ಬೆಲಹುರಬೇಲೂರು
ಬೆಲಹೂರಬೇಲೂರುಅಧಿಕಾರಿಯು
ಬೆಲುಹೂರಲಿ
ಬೆಲೂರಿನ
ಬೆಲೂರುಬೆಳ್ಳೂರು
ಬೆಲೆಕೆರೆ
ಬೆಲ್ಲೂರು
ಬೆಳಕನ್ನು
ಬೆಳಕವಾಡಿ
ಬೆಳಕವಾಡಿಗೆ
ಬೆಳಕವಾಡಿಯ
ಬೆಳಕವಾಡಿಯನ್ನು
ಬೆಳಕವಾಡಿಯು
ಬೆಳಕು
ಬೆಳಗಿಸಿದಳು
ಬೆಳಗುಳದ
ಬೆಳಗೊಳ
ಬೆಳಗೊಳದ
ಬೆಳಗ್ಗೆ
ಬೆಳತೂರ
ಬೆಳತೂರಿನ
ಬೆಳತೂರು
ಬೆಳವಡಿಯಲಿ
ಬೆಳವಡಿಯಲ್ಲಿ
ಬೆಳವಡಿಯಿಂದ
ಬೆಳವಣಿಗೆಯ
ಬೆಳವಾಡಿ
ಬೆಳವಾಡಿಯಲ್ಲಿ
ಬೆಳುಗಲಿಯಲ್ಲಿದ್ದ
ಬೆಳುವೊಲದ
ಬೆಳೆ
ಬೆಳೆಯ
ಬೆಳೆಯನ
ಬೆಳೆಯುವ
ಬೆಳೆಸಿ
ಬೆಳೆಸಿಕೊಳ್ಳುತ್ತಾ
ಬೆಳೆಸಿದ
ಬೆಳೆಸಿದಳು
ಬೆಳೆಸಿದ್ದನಷ್ಟೆ
ಬೆಳೆಸಿದ್ದನೋ
ಬೆಳೆಸಿದ್ದವು
ಬೆಳ್ಕೆರೆ
ಬೆಳ್ಗುಪ್ಪ
ಬೆಳ್ಗೊಳ
ಬೆಳ್ಗೊಳದ
ಬೆಳ್ಗೊಳದಲ್ಲಿ
ಬೆಳ್ಗೊಳದೊಳ್ಜನಮೆಲ್ಲಂ
ಬೆಳ್ಗೊಳವಾಗುತ್ತದೆ
ಬೆಳ್ಳಾಲೆ
ಬೆಳ್ಳಿ
ಬೆಳ್ಳಿಕೊಡವನ್ನೂ
ಬೆಳ್ಳಿಬಟ್ಟಲುಗಳ
ಬೆಳ್ಳಿಬೆಟ್ಟದ
ಬೆಳ್ಳಿಮಾಣಿಯ
ಬೆಳ್ಳಿಮುಲಾಮಿನ
ಬೆಳ್ಳಿಯ
ಬೆಳ್ಳೂರ
ಬೆಳ್ಳೂರನ್ನು
ಬೆಳ್ಳೂರಿಗೆ
ಬೆಳ್ಳೂರಿನ
ಬೆಳ್ಳೂರಿನಲ್ಲಿ
ಬೆಳ್ಳೂರು
ಬೆಳ್ವಲ
ಬೆಳ್ವೊಲ
ಬೆಳ್ವೊಲದ
ಬೆಳ್ವೊಲನಾಡಿನ
ಬೆವಹರಪೂಜಾಕೈಂಕರ್ಯಗಳಿಗೆ
ಬೆಸಗರಹಳ್ಳಿ
ಬೆಸಗರಹಳ್ಳಿಯನ್ನು
ಬೆಸಟೆಯ
ಬೆಸಣಿಪೆಸಾಣಿ
ಬೆಸದಿಂದ
ಬೆಸದೊಳು
ಬೆಸದೊಳೆ
ಬೆಸನಂ
ಬೆಸನನ್ನು
ಬೆಸಸಲು
ಬೆಸಸಿ
ಬೆಸಸಿದನೆಂದು
ಬೆಸೆಸಿದನು
ಬೆಸ್ತರ
ಬೇಂಟೆಕಾರ
ಬೇಕಾಗಿತ್ತು
ಬೇಕಾಗಿತ್ತೆಂದು
ಬೇಕಾಗುತ್ತಿತ್ತು
ಬೇಕಾಗುತ್ತಿತ್ತೆಂಬುದು
ಬೇಕಾದ
ಬೇಕಾದವು
ಬೇಕಾದಷ್ಟಿವೆ
ಬೇಕಾದುದನ್ನು
ಬೇಕು
ಬೇಗ
ಬೇಗಂ
ಬೇಗಮಂಗಲ
ಬೇಗಮಂಗಲವಾಗಿರಬಹುದು
ಬೇಚರಾಕ್
ಬೇಚಿರಾಕ್
ಬೇಟೆಗೆ
ಬೇಟೆಯನ್ನಾಡುವುದು
ಬೇಟೆಯಾಡುವುದರಲ್ಲಿ
ಬೇಡದೆ
ಬೇಡರನ್ನು
ಬೇಡರಹಳ್ಳಿ
ಬೇಡರಹಳ್ಳಿಯನ್ನು
ಬೇಡವ್ವೆ
ಬೇಡವ್ವೆಯ
ಬೇಡಿ
ಬೇಡಿಕೊಂಡು
ಬೇಡಿಕೊಳ್ಳಿಮೆನೆ
ಬೇಡಿಕೊಳ್ಳೆನೆ
ಬೇಡಿಕೊಳ್ಳೆನ್ದೊಡೆ
ಬೇಡಿಕೋ
ಬೇಡಿಪಡೆದ
ಬೇಡಿಪಡೆದನು
ಬೇಡಿಪಡೆದು
ಬೇಡೆ
ಬೇಬಿ
ಬೇಬಿಬೆಟ್ಟ
ಬೇರಂಬಾಡಿ
ಬೇರಾರು
ಬೇರೆ
ಬೇರೆಬೇರಯಾಗಿಯೇ
ಬೇರೆಬೇರೆ
ಬೇರೆಬೇರೆಯಾಗಿ
ಬೇರೆಯದೇ
ಬೇರೆಯವರಿಗೆ
ಬೇರೆಯಾಗಿಯೇ
ಬೇರೆಯೇ
ಬೇರ್ಪಡಿಸಿ
ಬೇಲೂರನ್ನು
ಬೇಲೂರಿಗೆ
ಬೇಲೂರಿನ
ಬೇಲೂರಿನಲ್ಲಿ
ಬೇಲೂರು
ಬೇವಿನಕುಪ್ಪೆ
ಬೇವಿನಕುಪ್ಪೆಯ
ಬೇವುಕಲ್ಲು
ಬೇಹಾರಿ
ಬೇಹಾರಿಅಧಿಕಾರಿರಾಜವರ್ತಕ
ಬೈಚ
ಬೈಚದಂಡಾಧೀಶಂ
ಬೈಚದಂಣಾಯಕ
ಬೈಚದಂಣಾಯಕರ
ಬೈಚಪ್ಪ
ಬೈಚೆದಂಡೇಶನಿಗೆ
ಬೈಚೆಯಬೀಚೆಯಬೈಚ
ಬೈತ್ರಂ
ಬೊಂಮದೇವ
ಬೊಂಮೋಜನೊಳಗಾದ
ಬೊಕ್ಕಸ
ಬೊಕ್ಕಸಕ್ಕೆ
ಬೊಟ್ಟೈಯ್ಯ
ಬೊಪ್ಪ
ಬೊಪ್ಪಗೌಡನಪುರ
ಬೊಪ್ಪಣ್ಣ
ಬೊಪ್ಪಣ್ಣಪಂಡಿತನ
ಬೊಪ್ಪದಂಡಾಧೀಶ
ಬೊಪ್ಪದೇವ
ಬೊಪ್ಪದೇವನ
ಬೊಪ್ಪದೇವನು
ಬೊಪ್ಪನಹಳ್ಳಿ
ಬೊಪ್ಪನು
ಬೊಪ್ಪನೆಂಬ
ಬೊಪ್ಪಸಂದ್ರ
ಬೊಪ್ಪಸಮುದ್ರ
ಬೊಪ್ಪಸಮುದ್ರಗಳು
ಬೊಪ್ಪಸಮುದ್ರವು
ಬೊಪ್ಪಾದೇವಿಯರನ್ನು
ಬೊಪ್ಪಾದೇವಿಯರಿನ್ತೀ
ಬೊಮನಹಳ್ಳಿ
ಬೊಮ್ಮಣ್ಣ
ಬೊಮ್ಮಣ್ಣನ
ಬೊಮ್ಮಣ್ಣನನ್ನು
ಬೊಮ್ಮನಹಳ್ಳಿ
ಬೊಮ್ಮನಹಳ್ಳಿಗಳು
ಬೊಮ್ಮನಹಳ್ಳಿಯ
ಬೊಮ್ಮನಹಳ್ಳಿಯನ್ನೂ
ಬೊಮ್ಮನಾಯಕನಹಳ್ಳಿಯನ್ನು
ಬೊಮ್ಮರಸನಕೊಪ್ಪಲು
ಬೊಮ್ಮವ್ವೆ
ಬೊಮ್ಮೆಯನಹಳ್ಳಿಯನ್ನು
ಬೊಮ್ಮೇನಹಳ್ಳಿ
ಬೊಯ್ಸಿಕಟ್ಟೆಯನ್ನು
ಬೊರಹ
ಬೋಕಂಣ
ಬೋಕಂಣನು
ಬೋಕಣ
ಬೋಕಣ್ಣ
ಬೋಕಣ್ಣನು
ಬೋಕಣ್ಣರಲ್ಲದೆ
ಬೋಕಿಮಯ್ಯನು
ಬೋಗನಹಳ್ಳಿಯನ್ನು
ಬೋಗವದಿಯ
ಬೋಗಾದಿ
ಬೋಗೇಗೌಡನು
ಬೋಗೈಯ
ಬೋಗೈಯ್ಯ
ಬೋಯೆಗನು
ಬೋರಯನಹಳ್ಳಿ
ಬೋಳಚಾಮರಾಜ
ಬೋಳಚಾಮರಾಜನ
ಬೋವ
ಬೋವರು
ಬ್ಯಾಡರಹಳ್ಳಿ
ಬ್ಯಾಲದಕೆರೆ
ಬ್ರಣವಿಭೂಷಿತ
ಬ್ರಹ್ಮಕುಲದೀಪಕನಪ್ಪ
ಬ್ರಹ್ಮಕ್ಷತ್ರಿಯ
ಬ್ರಹ್ಮಣ್ಯತೀರ್ಥರ
ಬ್ರಹ್ಮದೇಯವನ್ನಾಗಿ
ಬ್ರಹ್ಮದೇಯವಾಗಿ
ಬ್ರಹ್ಮದೇವರ
ಬ್ರಹ್ಮದೇವರಿಗೆ
ಬ್ರಹ್ಮಧೇಯವಾಗಿ
ಬ್ರಹ್ಮಪುರಿಯನ್ನು
ಬ್ರಹ್ಮರಾಶಿ
ಬ್ರಹ್ಮೇಶ್ವರ
ಬ್ರಾಹ್ಮಣ
ಬ್ರಾಹ್ಮಣನಿಗೆ
ಬ್ರಾಹ್ಮಣರ
ಬ್ರಾಹ್ಮಣರಲ್ಲಿ
ಬ್ರಾಹ್ಮಣರಾಗಿದ್ದರೆಂದು
ಬ್ರಾಹ್ಮಣರಿಂದ
ಬ್ರಾಹ್ಮಣರಿಗಾಗಿ
ಬ್ರಾಹ್ಮಣರಿಗೆ
ಬ್ರಾಹ್ಮಣರು
ಬ್ರಿಟಿಷರ
ಬ್ರಿಟಿಷರನ್ನು
ಬ್ರಿಟಿಷರಿಗೂ
ಬ್ರಿಟಿಷರು
ಬ್ರಿಟಿಷ್
ಬ್ರಿಟೀಷ್
ಬ್ರೂಸ್ಫೂಟ್
ಭಂಗಿಕರ
ಭಂಡಾರ
ಭಂಡಾರಕ್ಕೆ
ಭಂಡಾರದ
ಭಂಡಾರಬಸದಿಯ
ಭಂಡಾರವನ್ನು
ಭಂಡಾರವೆನಿಪ
ಭಂಡಾರಿ
ಭಂಡಾರಿಗನಾಗಿದ್ದನು
ಭಂಡಾರಿಗಳಾಗಿದ್ದು
ಭಂಡಾರಿಗಳು
ಭಂಡಾರಿಗಳುಹಿರಿಯಭಂಡಾರಿಮಾಣಿಕಭಂಡಾರಿ
ಭಂಡಾರಿಗೌಂಡ
ಭಂಡಾರಿಯಾಗಿದ್ದ
ಭಂಡಾರಿಯಾಗಿದ್ದನು
ಭಂಡಾರಿಯು
ಭಂಡಿವಾಳ
ಭಂಢಾರಿಯ
ಭಕ್ತನಾಗಿದ್ದು
ಭಕ್ತರಾಗಿ
ಭಕ್ತರಿಗೆ
ಭಕ್ತಿ
ಭಕ್ತಿಯಿಂದ
ಭಕ್ತಿಯುಳ್ಳವನಾಗಿದ್ದನೆಂದು
ಭಗೀರಥ
ಭಟಭೀಮೆಯನಾಯಕ
ಭಟಾರರ
ಭಟಾರರಿಗೆ
ಭಟ್ಟ
ಭಟ್ಟಂಗಿ
ಭಟ್ಟಂಗಿಗಳಾಗಿ
ಭಟ್ಟಂಗಿಗಳೆಂದು
ಭಟ್ಟನೆಂಬ
ಭಟ್ಟರ
ಭಟ್ಟರಬಾಚಪ್ಪನ
ಭಟ್ಟರಬಾಚಪ್ಪನವರು
ಭಟ್ಟರಬಾಚಪ್ಪರಲ್ಲದೆ
ಭಟ್ಟರಬಾಚಿಯಪ್ಪನ
ಭಟ್ಟರಬಾಚಿಯಪ್ಪನಿಗೆ
ಭಟ್ಟರಬಾಚಿಯಪ್ಪನು
ಭಟ್ಟರಬಾಚಿಯಪ್ಪನೂ
ಭಟ್ಟಾರಕ
ಭಟ್ಟಾರಕರ
ಭತ್ತ
ಭತ್ತಾಯ
ಭದ್ರಕಾಳಮ್ಮ
ಭದ್ರಕಾಳಿಯಣ್ಣ
ಭದ್ರನಕೊಪ್ಪಲು
ಭದ್ರಪಡಿಸಲು
ಭದ್ರಬಾಹು
ಭಯಂಕರ
ಭಯಂಕರನಾಗಿ
ಭಯದಿಂದ
ಭಯಲೋಭದುರ್ಲ್ಲಭಂ
ಭಯಿರಮೇಶ್ವರ
ಭಯಿರಮೇಶ್ವರಪುರ
ಭರತ
ಭರತಚಮೂಪತಿಯ
ಭರತಜೀಯ
ಭರತದಂಡನಾಯಕನು
ಭರತನ
ಭರತನೂ
ಭರತನೇ
ಭರತಮಯ್ಯ
ಭರತರನ್ನು
ಭರತಿಮಯ್ಯ
ಭರತಿಮಯ್ಯಗಳು
ಭರತಿಮಯ್ಯನ
ಭರತಿಮಯ್ಯರ
ಭರತಿಮಯ್ಯರು
ಭರತೆಯ
ಭರತೆಯನಾಯಕ
ಭರತೆಯನಾಯಕಂ
ಭರತೇಶದಂಡನಾಯಕನ
ಭರತೇಶ್ವರ
ಭರ್ತಿಯಾಗಿತ್ತೆಂದು
ಭವತ್ಪ್ರತಾಪ
ಭವನದಂತಿದ್ದ
ಭಾಗ
ಭಾಗಕ್ಕೆ
ಭಾಗಗಳನ್ನು
ಭಾಗಗಳಲ್ಲಿ
ಭಾಗಗಳಾಗಿ
ಭಾಗಗಳಾಗಿದ್ದವೆಂದು
ಭಾಗಗಳಿಗೂ
ಭಾಗಗಳು
ಭಾಗಗಳೂ
ಭಾಗದ
ಭಾಗದಲ್ಲಿ
ಭಾಗದಲ್ಲಿದ್ದ
ಭಾಗದಲ್ಲಿದ್ದನೆಂದು
ಭಾಗದಲ್ಲಿರುವ
ಭಾಗದಲ್ಲೇ
ಭಾಗದವಳಾಗಿರಬಹುದು
ಭಾಗದಿಂದ
ಭಾಗವತೋತ್ತಮೆ
ಭಾಗವನ್ನು
ಭಾಗವನ್ನೆಲ್ಲಾ
ಭಾಗವಹಿಸಿ
ಭಾಗವಹಿಸಿದ
ಭಾಗವಹಿಸಿದ್ದ
ಭಾಗವಹಿಸಿದ್ದನೆಂದು
ಭಾಗವಹಿಸಿದ್ದರೆಂದು
ಭಾಗವಹಿಸಿದ್ದಾರೆ
ಭಾಗವಹಿಸಿರಬಹುದು
ಭಾಗವಹಿಸುತ್ತಿದ್ದನೆಂಬುದನ್ನು
ಭಾಗವಾಗಿ
ಭಾಗವಾಗಿತ್ತು
ಭಾಗವಾಗಿತ್ತೆಂದು
ಭಾಗವಾಗಿರಬಹುದು
ಭಾಗವಿದ್ದು
ಭಾಗವು
ಭಾಗವೆಂದು
ಭಾಗವೇ
ಭಾಗಶಃ
ಭಾಗಿಯಾಗಿದ್ದರೆಂದು
ಭಾಗ್ಯವಾಗಲಿ
ಭಾನುಕೀರ್ತಿ
ಭಾನುಕೀರ್ತಿದೇವನು
ಭಾನುಕೀರ್ತಿದೇವರ
ಭಾನುಕೀರ್ತ್ತಿ
ಭಾನುಮತಿಯವರು
ಭಾನುಮತಿಯವರೂ
ಭಾರ
ಭಾರತ
ಭಾರತದ
ಭಾರದ್ವಾಜಗೋತ್ರದವನು
ಭಾರೀ
ಭಾರ್ಯೆ
ಭಾವನೆಗಿಂತ
ಭಾವಮೈದ
ಭಾವಮೈದುನ
ಭಾವಿಸಬಹುದು
ಭಾವಿಸಿದ
ಭಾವಿಸಿರುವಂತಿದೆ
ಭಾಷಾಪ್ರಯೋಗ
ಭಾಷಿಕ
ಭಾಷೆ
ಭಾಷೆಗಳ
ಭಾಷೆಗಳಲ್ಲಿರುವುದನ್ನು
ಭಾಷೆಗಳೆರಡರಲ್ಲೂ
ಭಾಷೆಗೆ
ಭಾಷೆಗೆತಪ್ಪುವ
ಭಾಷೆಯ
ಭಾಷೆಯನ್ನೇ
ಭಾಷೆಯಲ್ಲಿ
ಭಾಷೆಯವು
ಭಾಷ್ಯಕಾರರು
ಭಾಸ್ವದ್ಬೃಹ
ಭಿತ್ತಿಯ
ಭಿನ್ನ
ಭಿನ್ನನು
ಭಿನ್ನನೆಂದು
ಭಿನ್ನರು
ಭಿನ್ನರೆಂದು
ಭಿನ್ನವಾಗಿತ್ತು
ಭಿನ್ನವಾಗಿದೆ
ಭಿನ್ನಾಭಿಪ್ರಾಯವನ್ನು
ಭಿಲ್ಲಮನಿಗೂ
ಭೀಕರತೆಯನ್ನು
ಭೀಕರವಾದ
ಭೀತರಾಗಿ
ಭೀತಿ
ಭೀಮ
ಭೀಮಗಾಮುಣ್ಡರು
ಭೀಮಣ್ಣ
ಭೀಮಣ್ಣನು
ಭೀಮದೇವ
ಭೀಮನಕಂಡಿಬೆಟ್ಟ
ಭೀಮನಕೆರೆ
ಭೀಮನಕೆರೆಗೆ
ಭೀಮನಹಳ್ಳಿ
ಭೀಮರಾಯ
ಭೀಮರಾಯನು
ಭೀಮಾರ್ಜುನರು
ಭೀಮೆಯ
ಭೀಮೆಯನಾಯಕ
ಭೀಮೆಯನಾಯಕನಾಗಿರುವ
ಭೀಮೆಯನಾಯಕನು
ಭೀಮೇಶ್ವರ
ಭೀಮೇಶ್ವರಿ
ಭೀಷ್ಮಪರ್ವದಲ್ಲಿ
ಭುಕ್ತಿ
ಭುಜಪ್ರತಾಪದಿ
ಭುಜಬಲ
ಭುಜಬಲಪ್ರತಾಪ
ಭುಜಬಲರಾಯನೆಂಬ
ಭುಜಬಲವೀರಗಂಗ
ಭುಜಬಲಿ
ಭುಜಬಲಿಚರಿತೆಯೆಂಬ
ಭುಜಬಳ
ಭುಜಬಳವೀರಗಂಗ
ಭುಜಬಳಾವಷ್ಟಂಭ
ಭುಜವಿಜಯ
ಭುಜಸಾಹಸದಿಂ
ಭುಜಾದಂಡ
ಭುಜಾದಂಡವೆನಿಸಿದ್ದ
ಭುವನದೊಳಾಂತು
ಭುವನೇಶ್ವರಿ
ಭುವನೈಕವೀರನೆಂಬ
ಭುವಿ
ಭೂಕಾಮಿನಿಯಿರ್ದ್ದಳಾ
ಭೂಗೋಳ
ಭೂಗೋಳವನ್ನು
ಭೂಚಕ್ರವಲಯ
ಭೂತಾನಾಮ
ಭೂದಾನ
ಭೂದಾನಗ್ರಾಮಧರ್ಮಸಾಧನವಾಗಿ
ಭೂದೇವತಾ
ಭೂನ್ರಿಪಂ
ಭೂಪತಿ
ಭೂಪತಿಃ
ಭೂಪತಿಯು
ಭೂಪನಾ
ಭೂಪರಿಮಿತೇ
ಭೂಪಸ್ಥಾನರಂಜಿತೇ
ಭೂಪಸ್ಯ
ಭೂಪಾಲ
ಭೂಪಾಲಂ
ಭೂಪಾಲಚಿರಪುಣ್ಯ
ಭೂಪಾಲನ
ಭೂಬುಜಂ
ಭೂಭಾಗದ
ಭೂಭಾಗವು
ಭೂಭಾರವನ್ನು
ಭೂಭುಜಂ
ಭೂಭುವನಂ
ಭೂಭೂಜಿ
ಭೂಭ್ರುನ್ನಿಳಯ
ಭೂಮಿಗಳ
ಭೂಮಿಗಳನ್ನು
ಭೂಮಿಗಳಿಗೆ
ಭೂಮಿಗೆ
ಭೂಮಿದಾನ
ಭೂಮಿಪನ
ಭೂಮಿಭಾಗದೊಳದನ್ಯರದೇಕೆ
ಭೂಮಿಯ
ಭೂಮಿಯನ್ನು
ಭೂಮಿಯಾಗಿತ್ತು
ಭೂಮಿಯಾಗಿದೆ
ಭೂಮಿಯು
ಭೂವಲ್ಲಭನಿಗೆಬೂತುಗ
ಭೂವಿಕ್ರಮನನ್ನು
ಭೂವಿಕ್ರಮನು
ಭೂಶಿರದವರೆಗಿನ
ಭೂಷಿತಂ
ಭೇಟಿ
ಭೇಟಿನೀಡಿದ್ದನೆಂದು
ಭೇಟಿನೀಡಿರಬಹುದು
ಭೇಟಿನೀಡಿರುತ್ತಾನೆ
ಭೇಟಿನೀಡಿರುವಂತೆ
ಭೇಟಿಯಾಗಿರಬಹುದು
ಭೇಟಿಯಾಗಿರಬಹುದೆಂದು
ಭೇದಿಸಿ
ಭೇರುಂಡವರ್ಗವನ್ನು
ಭೈತ್ರ
ಭೈರಕಂಬೆಯ
ಭೈರಮೇಶ್ವರ
ಭೈರವದಂಣಾಯಕಿತ್ತಿಯರ
ಭೈರವಪುರವೆಂಬ
ಭೈರವಾಪರುವೆಂಬ
ಭೈರವಾಪುರವೆಂಬ
ಭೈರವೇಶ್ವರ
ಭೈರವ್ವೆ
ಭೈರಾಪುರ
ಭೈರಾಪುರದಲ್ಲಿರುವ
ಭೈರಾಪುರವೆಂಬ
ಭೈರೇದೇವರ
ಭೋಗ
ಭೋಗನಹಳ್ಳಿ
ಭೋಗಯ್ಯದೇವ
ಭೋಗರಾಜಭೂಪಾಲನು
ಭೋಗರಾಜವರತಲ್ಪಃ
ಭೋಗವತಿಯಲ್ಲಿ
ಭೋಗವದಿಯಬೋಗಾದಿ
ಭೋಗವಸದಿಯೊಳು
ಭೋಗಾನುಭಾವಿ
ಭೋಗೈಯ್ಯ
ಭೋಜಃ
ಭೋಜನಕ್ಕೆ
ಭೋಜರರು
ಭೋಜರಾಜನಿಗೆ
ಭೌಗೋಳಿಕ
ಭೌಗೋಳಿಕವಾಗಿ
ಮ
ಮಂ
ಮಂಗಪ್ಪ
ಮಂಗಲ
ಮಂಗಲಕ್ಕೆ
ಮಂಗಲದ
ಮಂಗಲದಲ್ಲಿ
ಮಂಗಲಮ್
ಮಂಗಲವಾದ
ಮಂಗಲವು
ಮಂಗಳೂರು
ಮಂಚಗಾವುಂಡನು
ಮಂಚಗೌಂಡನ
ಮಂಚಗೌಡ
ಮಂಚನಹಳ್ಳಿ
ಮಂಚನಹಳ್ಳಿಯನ್ನು
ಮಂಚಯದಂಡನಾಯಕ
ಮಂಚಲಾದೇವಿ
ಮಂಚವ್ವೆ
ಮಂಚಿಗೌಡ
ಮಂಚೆಗಾವುಂಡ
ಮಂಚೆಗೌಡ
ಮಂಚೇಗೌಂಡನ
ಮಂಚೇಗೌಡನ
ಮಂಜಯ್ಯ
ಮಂಜಯ್ಯನನ್ನು
ಮಂಜಯ್ಯನು
ಮಂಜುನಾಥ್
ಮಂಟಪ
ಮಂಟಪಗಳನ್ನು
ಮಂಟಪಗಳು
ಮಂಟಪದ
ಮಂಟಪದಲ್ಲಿದೆ
ಮಂಟಪವನ್ನು
ಮಂಟಪವನ್ನುರಂಗಮಂಟಪ
ಮಂಟಿ
ಮಂಟಿಗಳಿಂದ
ಮಂಟಿಗೆ
ಮಂಠಿ
ಮಂಠೆ
ಮಂಠೆದ
ಮಂಠೆಯ
ಮಂಠೆಯದ
ಮಂಠೆಯಮಂಡ್ಯ
ಮಂಠೆಯವೇ
ಮಂಠೇದ
ಮಂಠೇದಯ್ಯ
ಮಂಠೇದಯ್ಯನವರು
ಮಂಡ
ಮಂಡಗೌಡನೆಂಬ
ಮಂಡಮಂಡೆಮಂಡೇವುಮಂಡ್ಯ
ಮಂಡಯಂ
ಮಂಡರಿವರ್ಮರಾಜ
ಮಂಡಲ
ಮಂಡಲಕ್ಕೂ
ಮಂಡಲಗಳನ್ನಾಗಿ
ಮಂಡಲಗಳಾಗಿ
ಮಂಡಲಗಳಿದ್ದವು
ಮಂಡಲವನ್ನು
ಮಂಡಲವನ್ನೂ
ಮಂಡಲವಿಷಯದೇಶನಾಡುಕಂಪಣ
ಮಂಡಲಸ್ವಾಮಿ
ಮಂಡಲಸ್ವಾಮಿಗೆ
ಮಂಡಲಸ್ವಾಮಿಯ
ಮಂಡಲಸ್ವಾಮಿಯು
ಮಂಡಲಾಧಿಪತಿಯನ್ನಾಗಿ
ಮಂಡಲಾಧಿಪತಿಯಾಗಿದ್ದ
ಮಂಡಲಿಕ
ಮಂಡಲಿಕರು
ಮಂಡಲೀಕ
ಮಂಡಲೇಶ್ವರ
ಮಂಡಲೇಶ್ವರರನ್ನು
ಮಂಡಲೇಶ್ವರರಾಗಿ
ಮಂಡಲೇಶ್ವರರು
ಮಂಡಳಿಕ
ಮಂಡಳಿಕಜೂಬು
ಮಂಡಳಿಕನಾಗಿ
ಮಂಡಳಿಕನಾದ
ಮಂಡಳಿಕರು
ಮಂಡಳೀಕಜೂಬು
ಮಂಡಳೀಕರಗಂಡ
ಮಂಡಳೇಶ್ವರನಾಗಿ
ಮಂಡಳೇಶ್ವರರು
ಮಂಡಸ್ವಾಮಿಗೆ
ಮಂಡಿತ
ಮಂಡಿಸಲ್ಪಟ್ಟ
ಮಂಡೆಯ
ಮಂಡೆಯಂ
ಮಂಡೆಯದ
ಮಂಡೆವೇಮು
ಮಂಡೇವು
ಮಂಡೇವುಕೆ
ಮಂಡೇವುಕ್ಕೆ
ಮಂಡೇವುಮಂಡ್ಯ
ಮಂಡ್ಯ
ಮಂಡ್ಯಂ
ಮಂಡ್ಯಂಣ
ಮಂಡ್ಯಕ್ಕಿಂತಲೂ
ಮಂಡ್ಯಕ್ಕೆ
ಮಂಡ್ಯಗೋಪಣನ
ಮಂಡ್ಯಜಿಲ್ಲೆಯ
ಮಂಡ್ಯಜಿಲ್ಲೆಯಲ್ಲಿ
ಮಂಡ್ಯಜಿಲ್ಲೆಯು
ಮಂಡ್ಯದ
ಮಂಡ್ಯದಲ್ಲಿ
ಮಂಡ್ಯವನ್ನು
ಮಂಡ್ಯವು
ಮಂತಲಲಾಮನೀ
ಮಂತ್ರಚಿನ್ತಾಮಣಿ
ಮಂತ್ರವಿದ್ಯಾವಿಕಾಶಂ
ಮಂತ್ರಿ
ಮಂತ್ರಿಗಳ
ಮಂತ್ರಿಗಳಾಗಿದ್ದ
ಮಂತ್ರಿಗಳಾಗಿದ್ದರು
ಮಂತ್ರಿಗಳಾಗಿದ್ದರೆಂದು
ಮಂತ್ರಿಗಳಾಗಿದ್ದಾಗ
ಮಂತ್ರಿಗಳಾಗಿದ್ದಿರಬಹುದು
ಮಂತ್ರಿಗಳಾಗಿದ್ದು
ಮಂತ್ರಿಗಳಾದ
ಮಂತ್ರಿಗಳು
ಮಂತ್ರಿಗಳೂ
ಮಂತ್ರಿಗಳೆಂದೂ
ಮಂತ್ರಿಚೂಡಾಮಣಿ
ಮಂತ್ರಿಣಾವಭವತಾಂ
ಮಂತ್ರಿಣೇ
ಮಂತ್ರಿತಿಳಕಂ
ಮಂತ್ರಿಪದವಿಯಲ್ಲಿದ್ದಿರಬಹುದು
ಮಂತ್ರಿಪರಿಷತ್ತಿನಲ್ಲಿ
ಮಂತ್ರಿಭಿಃ
ಮಂತ್ರಿಮಂಡಲ
ಮಂತ್ರಿಮಂಡಲದ
ಮಂತ್ರಿಮಾಣಿಕ್ಯ
ಮಂತ್ರಿಮಾಣಿಕ್ಯಂ
ಮಂತ್ರಿಮುಖದರ್ಪಣ
ಮಂತ್ರಿಯ
ಮಂತ್ರಿಯಾಗಿದ್ದ
ಮಂತ್ರಿಯಾಗಿದ್ದಂತೆ
ಮಂತ್ರಿಯಾಗಿದ್ದನು
ಮಂತ್ರಿಯಾಗಿದ್ದನೆಂದುಗೋವಿಂದಯ್ಯಾಖ್ಯ
ಮಂತ್ರಿಯಾಗಿದ್ದುದರ
ಮಂತ್ರಿಯಾಗಿರಬಹುದು
ಮಂತ್ರಿಯಾದಂ
ಮಂತ್ರಿಯಾದನೆಂದು
ಮಂತ್ರಿಯು
ಮಂತ್ರಿಯೂ
ಮಂತ್ರಿಯೂಥಾಗ್ರಣಿ
ಮಂತ್ರಿಯೊಡನೆ
ಮಂತ್ರೀಶ
ಮಂತ್ರೀಶ್ವರನಾದಂತೆ
ಮಂದಗೆರೆ
ಮಂದಿ
ಮಂದಿರಂ
ಮಂದಿರಲ್ಲಿದೆ
ಮಂನನ
ಮಂನಿತಿ
ಮಂನೆಯ
ಮಂನೆಯಗಜಪತಿ
ಮಂನೆಯಜೂಬು
ಮಂನೆಯರು
ಮಕರ
ಮಕರರಾಜ್ಯ
ಮಕರರಾಯ
ಮಕುಟಮಂಡಲಿಕರ
ಮಕ್ಕಳ
ಮಕ್ಕಳನ್ನು
ಮಕ್ಕಳನ್ನೋ
ಮಕ್ಕಳಾಗಿದ್ದ
ಮಕ್ಕಳಾಗಿದ್ದು
ಮಕ್ಕಳಾಗಿರಬಹುದು
ಮಕ್ಕಳಾದ
ಮಕ್ಕಳಿಗೂ
ಮಕ್ಕಳಿಗೆ
ಮಕ್ಕಳಿದ್ದರು
ಮಕ್ಕಳಿದ್ದರೆಂದು
ಮಕ್ಕಳಿದ್ದರೆಂದೂ
ಮಕ್ಕಳಿದ್ದು
ಮಕ್ಕಳಿದ್ದುದು
ಮಕ್ಕಳಿಬ್ಬರೂ
ಮಕ್ಕಳಿಲ್ಲದ
ಮಕ್ಕಳಿಲ್ಲದೇ
ಮಕ್ಕಳು
ಮಕ್ಕಳುಗಳಿಗೂ
ಮಕ್ಕಳೂ
ಮಕ್ಕಳೆಂದು
ಮಕ್ಕಳೆಂದೂ
ಮಕ್ಕಳೆಂಬುದು
ಮಕ್ಕಳೇ
ಮಕ್ಕಳೊಡನೆ
ಮಖ್ಖಳು
ಮಗ
ಮಗಂ
ಮಗದೊಬ್ಬ
ಮಗನ
ಮಗನನ್ನು
ಮಗನನ್ನೂ
ಮಗನಾಗಿ
ಮಗನಾಗಿದ್ದನು
ಮಗನಾಗಿದ್ದು
ಮಗನಾಗಿರಬಹುದು
ಮಗನಾಗಿರಲು
ಮಗನಾಗಿರುವ
ಮಗನಾಗುತ್ತಾನೆ
ಮಗನಾದ
ಮಗನಿಗೆ
ಮಗನಿದ್ದನು
ಮಗನಿದ್ದನೆಂದು
ಮಗನಿದ್ದು
ಮಗನಿರಬಹುದು
ಮಗನಿರಬಹುದೆಂದು
ಮಗನು
ಮಗನೂ
ಮಗನೆಂದು
ಮಗನೆಂದೂ
ಮಗನೇ
ಮಗನೋ
ಮಗರ
ಮಗರನ
ಮಗರರಾಜ್ಯ
ಮಗರರಾಯ
ಮಗರಾಧಿರಾಯ
ಮಗಳ
ಮಗಳನ್ನು
ಮಗಳನ್ನೂ
ಮಗಳಾಗಿದ್ದು
ಮಗಳಾದ
ಮಗಳಿದ್ದ
ಮಗಳು
ಮಗಳುಶಾಸನ
ಮಗಶಿಷ್ಯನಾದ್ದರಿಂದ
ಮಗು
ಮಗುರ್ದಡೆರೆಪ್ಪುವ
ಮಗ್ಗತೆರೆ
ಮಗ್ಗದ
ಮಗ್ಗದೆರೆಯನ್ನು
ಮಗ್ಗನಹಳ್ಳಿಯ
ಮಗ್ಗನಹಳ್ಳಿಯನ್ನು
ಮಗ್ಗವನ್ನು
ಮಗ್ನನಾಗಿದ್ದ
ಮಗ್ನನಾದ
ಮಚ್ಚರಿಪನಾಯಕರಗಂಣ್ಡ
ಮಜ್ಜನದ
ಮಟ್ಟದ
ಮಟ್ಟಿಗೂ
ಮಟ್ಟಿಗೆ
ಮಠಕ್ಕೆ
ಮಠಗಳಿಗೆ
ಮಠಗಳು
ಮಠದ
ಮಠದಕೇರಿ
ಮಠಪತಿದಾಸವೈಷ್ಣವರ
ಮಠಮಾನ್ಯಗಳನ್ನು
ಮಠವನ್ನು
ಮಠವಿದೆ
ಮಠಾಧಿಪತಿ
ಮಡಕೆಪಟ್ಟಣ
ಮಡವನಕೋಡಿ
ಮಡಿದ
ಮಡಿದನು
ಮಡಿದನೆಂದು
ಮಡಿದರೆಂದು
ಮಡಿದವನು
ಮಡಿದವರ
ಮಡಿದಾಗ
ಮಡಿದಿದ್ದು
ಮಡಿದಿರಬಹುದು
ಮಡಿದಿರಬಹುದೆಂದು
ಮಡಿದಿರವುದು
ಮಡಿದಿರುವ
ಮಡಿಯನಹಳ್ಳಿಯ
ಮಡಿಯಲು
ಮಡಿಯುತ್ತಾನೆ
ಮಡಿಯುತ್ತಾನೆಂದಿದೆ
ಮಡಿಯುತ್ತಾರೆ
ಮಡಿವಳ್ಳ
ಮಡು
ಮಡುವನ್ನು
ಮಡುವಿನಕೋಡಿ
ಮಡುವಿನಕೋಡಿಯ
ಮಡುವಿನಲ್ಲಿ
ಮಡುಹಿನ
ಮಣಲಯರನ
ಮಣಲೆ
ಮಣಲೆಅರಸನು
ಮಣಲೆಯ
ಮಣಲೆಯರ
ಮಣಲೆಯರನ
ಮಣಲೆಯರನು
ಮಣಲೆಯರರು
ಮಣಲೆಯರಸರ
ಮಣಲೆಯರಸರಾ
ಮಣಲೆಯಾರನಿರಬಹುದು
ಮಣಲೆಯಾರನು
ಮಣಲೆರ
ಮಣಲೆರಙ್ಗೆ
ಮಣಲೆರನ
ಮಣಲೆರನು
ಮಣಲೇರ
ಮಣಲೇರನ
ಮಣಲೇರನಿಗೆ
ಮಣಲೇರನು
ಮಣಲೇರರನ್ನು
ಮಣಲೇರರನ್ನುಮರುವರ್ಮ
ಮಣಲೇರರು
ಮಣಳೇಶ್ವರ
ಮಣವಾಳ
ಮಣಾಲರನ
ಮಣಾಲರನನ್ನು
ಮಣಾಲರನು
ಮಣಿಕರ್ಣಿಕಾ
ಮಣಿಕೊನಖಗಳ
ಮಣಿನಾಗಪುರ
ಮಣಿನಾಗಪುರವರಾಧೀಶ್ವರ
ಮಣಿನಾಗಪುರವರಾಧೀಶ್ವರನೂ
ಮಣಿನಾಗರಗ್ರಾಮವೆಂದು
ಮಣಿಪ್ರದೀಪ
ಮಣಿಯೂರು
ಮಣ್ಡಳೀಕಜೂಬು
ಮಣ್ಣನ್ನು
ಮಣ್ಣು
ಮಣ್ಣೆ
ಮಣ್ಣೆಯನ್ನು
ಮಣ್ಣೆಯಲ್ಲಿ
ಮಣ್ಣೆಯಿಂದ
ಮತ
ಮತಾನುಯಾಯಿಗಳು
ಮತಿಸಾಗರ
ಮತೀಯ
ಮತು
ಮತ್ತಮಾತಂಗ
ಮತ್ತಯರ
ಮತ್ತರು
ಮತ್ತಿಕೆರೆಯನ್ನು
ಮತ್ತಿದಾಗ
ಮತ್ತು
ಮತ್ತೆ
ಮತ್ತೆಗೆರೆ
ಮತ್ತೊಂದು
ಮತ್ತೊಬ್ಬ
ಮಥ್ಚಮತ್ಸ್ಯ
ಮದವದುಗ್ರವೈರಿಮದಮರ್ದ್ಧನ
ಮದವದುದಗ್ರ
ಮದವಳಿಗೆ
ಮದಿಸಿದ
ಮದುಗನ್ದೂರ
ಮದುವೆ
ಮದುವೆನಿಂದು
ಮದುವೆಯ
ಮದುವೆಯನ್ನು
ಮದುವೆಯಾಗಿದ್ದನು
ಮದುವೆಯಾದನೆಂದು
ಮದ್ದಿನಮನೆಗಳು
ಮದ್ದಿಯಕ್ಕರ
ಮದ್ದೂರ
ಮದ್ದೂರನ್ನು
ಮದ್ದೂರಾದ
ಮದ್ದೂರಿಗೆ
ಮದ್ದೂರಿನ
ಮದ್ದೂರಿನಲ್ಲಿ
ಮದ್ದೂರು
ಮದ್ರಾಸಿನಲ್ಲಿದ್ದ
ಮದ್ರಾಸಿನವರೆಗೂ
ಮದ್ರಾಸ್
ಮಧುಕೇಶ್ವರ
ಮಧುರಮಂಡಲ
ಮಧುರವಾಗಿದ್ದವು
ಮಧುರವಾದ
ಮಧುರಾ
ಮಧುರೆ
ಮಧುರೆಯ
ಮಧುರೆಯನ್ನು
ಮಧುವಂಣ
ಮಧುಸೂಧನ
ಮಧುಸೂಧನನ
ಮಧ್ಯಕಾಲೀನ
ಮಧ್ಯಗತವಾಗಿರುವುದರಿಂದ
ಮಧ್ಯದ
ಮಧ್ಯದಲ್ಲಿ
ಮಧ್ಯದೇಸಮುದ್ದಂಡವಿನಾಳ್ದು
ಮಧ್ಯದೊಳಗಣ
ಮಧ್ಯದೊಳಗೆ
ಮಧ್ಯಭಾಗದಲ್ಲಿ
ಮಧ್ಯಸ್ಥಿಕೆ
ಮಧ್ಯೆ
ಮನಗಾಣಲಿಲ್ಲ
ಮನಗಾಣಲು
ಮನದನ್ನನಪ್ಪ
ಮನಮೊಸೆದು
ಮನಾಲರ
ಮನಿಗಳು
ಮನಿಷಿಯ
ಮನುಚರಿತ
ಮನುಚರಿತರು
ಮನುಚಾರಿತ್ರ್ಯರೂ
ಮನುಧರ್ಮಶಾಸ್ತ್ರ
ಮನುಪ್ರತಿಮಂ
ಮನುಬ್ರೋಲು
ಮನುಮಾರ್ಗನು
ಮನುಮಾರ್ಗಾಗ್ರಣಿಗಳು
ಮನುಮುನಿಚರಿತನೂ
ಮನುಷ
ಮನೆಗಳನ್ನೂ
ಮನೆಗಳಲ್ಲಿ
ಮನೆತನ
ಮನೆತನಕ್ಕೆ
ಮನೆತನದ
ಮನೆತನದಲ್ಲಿ
ಮನೆತನದವನಾಗಿದ್ದು
ಮನೆತನದವನೆಂದು
ಮನೆತನದವರು
ಮನೆತನದವರೆಂದು
ಮನೆತನದವಳೇ
ಮನೆತನದಿಂದ
ಮನೆನದವರೆಂದು
ಮನೆಮಗ
ಮನೆಯ
ಮನೆಯಬಲೆ
ಮನೆಯಮಗ
ಮನೆಯಲ್ಲಿ
ಮನೆರ್ದೋಡಿಸುತಂ
ಮನೆವೆಗ್ಗಡೆ
ಮನೆವೆಗ್ಗಡೆಅರಮನೆಯ
ಮನೆವೆರ್ಗ್ಗಡೆ
ಮನೋಜಃ
ಮನೋಜಭಯಂಕರ
ಮನೋಭಾವಕ್ಕೆ
ಮನೋಭಾವವು
ಮನೋಭೀಷ್ಟ
ಮನೋಮಿತ್ರನಾಗಿದ್ದನೆಂದು
ಮನೋರಮೆಪತ್ನಿ
ಮನೋವಲ್ಲಭ
ಮನೋಹರವಾಗಿ
ಮನ್ತ್ರಿಚಾಮುಣ್ಡನ
ಮನ್ನಣೆ
ಮನ್ನಾ
ಮನ್ನಾಮಾಡಲು
ಮನ್ನಾಮಾಡುವ
ಮನ್ನಾರುಕೃಷ್ಣಸ್ವಾಮಿ
ಮನ್ನೆಯ
ಮನ್ನೆಯರಿಗೆಲ್ಲಾ
ಮನ್ನೆಯಸೂನು
ಮನ್ನೆಯೊಳಗೆ
ಮನ್ಮಥಪುಷ್ಕರಣಿಗಳನ್ನು
ಮಯಿಲನಹಳ್ಳಿ
ಮಯಿಲನಹಳ್ಳಿಯಲ್ಲಿ
ಮಯಿಸೂರ
ಮಯಿಸೂರು
ಮಯೂರಾಸನ
ಮಯ್ದುನ
ಮಯ್ದುನನಾಗಿದ್ದನೆಂದು
ಮಯ್ದುನನೂ
ಮರ
ಮರಗೆಲಸವೆಲ್ಲಾ
ಮರಡಿ
ಮರಡಿಗೆ
ಮರಡಿಪುರ
ಮರಡಿಪುರವೂ
ಮರಡಿಯೊಳ್
ಮರಣ
ಮರಣಕಾಲದಲ್ಲಿ
ಮರಣದ
ಮರಣವನ್ನಪ್ಪಿದಾಗ
ಮರಣವನ್ನು
ಮರಣವನ್ನೇ
ಮರಣಶಾಸನದಲ್ಲಿ
ಮರಣಹೊಂದಿದ
ಮರಣಹೊಂದಿದನು
ಮರಣಹೊಂದಿದನೆಂದು
ಮರಣಹೊಂದಿದವರ
ಮರಣಹೊಂದಿನೆಂದು
ಮರಣಾ
ಮರಣಾನಂತರ
ಮರದುರಾದ
ಮರದೂರಮದ್ದೂರ
ಮರದೂರಾದಮದ್ದೂರು
ಮರಲಹಳ್ಳಿ
ಮರಲಹಳ್ಳಿಯ
ಮರಲೆ
ಮರಳಿಕೆರೆ
ಮರಳಿಬಿಟ್ಟು
ಮರಳಿಸಿ
ಮರವೂರ
ಮರಸೆಯ
ಮರಹಳ್ಳಿ
ಮರಾಠರ
ಮರಾಠರಿಗೆ
ಮರಾಠರು
ಮರಾಠಾ
ಮರಿದೇವ
ಮರಿದೇವರಾಜವಡೆಯನೆಂಬ
ಮರಿಯಣ್ಣನವರ
ಮರಿಯನಾಯಕನ
ಮರಿಯಾನೆ
ಮರಿಯಾನೆಕಿರಿಯ
ಮರಿಯಾನೆಗೆ
ಮರಿಯಾನೆಯ
ಮರಿಯಾನೆಯೂ
ಮರಿಯಾನೆಯೇ
ಮರಿಯಾನೆಸಮುದ್ರದ
ಮರಿಯಾಯನೆಯನ್ನು
ಮರುದಗಾಮುಂಡ
ಮರುದೇವಿಯರ
ಮರುಪರಿಶೀಲಿಸಬೇಕಾಗುತ್ತದೆ
ಮರುಪರಿಶೀಲಿಸಿದ
ಮರುಳ
ಮರುಳದೇವ
ಮರುಳನು
ಮರುಳಸಿದ್ಧ
ಮರುವರ್ಮನ
ಮರುವರ್ಮನಿಗೂ
ಮರುವರ್ಮನು
ಮರೆಯದೆ
ಮರೆವೊಕ್ಕಡೆಕಾವ
ಮರ್ಕುಲಿ
ಮರ್ದ್ದಸ
ಮರ್ದ್ಧನ
ಮರ್ಯಾದೆ
ಮರ್ಯಾದೆಉಂಬಳಿಗವುಡುಗೊಡಗೆ
ಮರ್ಯಾದೆಗಾಗಿ
ಮರ್ಯಾದೆಯ
ಮರ್ಯಾದೆಯನ್ನು
ಮಱೆವೊಕ್ಕರಕಾವರುಂ
ಮಱೆವೊಕ್ಕರೆ
ಮಲಗಿರುವವನು
ಮಲತಮ್ಮ
ಮಲತಮ್ಮನಾದ
ಮಲತಮ್ಮನಿದ್ದನು
ಮಲತಮ್ಮಮಲ್ಲಿತಮ್ಮ
ಮಲಯಾ
ಮಲಯಾಳನ
ಮಲಸಹೋದರರ
ಮಲಹಗಳ್ಳಿಮಳಗಳಲಿ
ಮಲಿದೇವ
ಮಲಿಯೂರಿನ
ಮಲಿಯೂರು
ಮಲುಕಬ್ಬೆಪುರಇಂದಿನ
ಮಲುನಾಯಕನಹಳ್ಳಿಯ
ಮಲೆ
ಮಲೆನಾಡಿನಲ್ಲಿ
ಮಲೆನಾಡುಗಳು
ಮಲೆನಾಡೇಳು
ಮಲೆಪನಾಯಕ
ಮಲೆಪನಾಯಕನು
ಮಲೆಪರಮಲ್ಲ
ಮಲೆಮಹದೇಶ್ವರ
ಮಲೆಯ
ಮಲೆಯಕಡೆಗೆ
ಮಲೆಯನಹಳ್ಳಿಗಳು
ಮಲೆಯನಾಯಕನ
ಮಲೆಯನಾಯಕನಹಳ್ಳಿ
ಮಲೆಯನು
ಮಲೆಯಾಂಡನ್
ಮಲೆಯಾಲಗಮಿಡಿಪಿ
ಮಲೆಯಾಳ
ಮಲೆಯಾಳನ
ಮಲೆಯಾಳರಂ
ಮಲೆರಾಜರಾಜ
ಮಲೆರಾಜ್ಯಕ್ಕೆ
ಮಲೆವ
ಮಲ್ಲ
ಮಲ್ಲಗವುಡ
ಮಲ್ಲಘಟ್ಟ
ಮಲ್ಲಘಟ್ಟದ
ಮಲ್ಲಘಟ್ಟದಲ್ಲಿ
ಮಲ್ಲಜೀಯನಿಗೆ
ಮಲ್ಲನಾಯಕ
ಮಲ್ಲನಾಯಕನ
ಮಲ್ಲನು
ಮಲ್ಲಯ್ಯ
ಮಲ್ಲಯ್ಯನಹಳ್ಳಿ
ಮಲ್ಲರಸ
ಮಲ್ಲರಸನೂ
ಮಲ್ಲರಾಜ
ಮಲ್ಲರಾಜಗಳೆಂಬ
ಮಲ್ಲರಾಜನ
ಮಲ್ಲರಾಜನೆಂಬ
ಮಲ್ಲರಾಜನೆಂಬುದಾಗಿ
ಮಲ್ಲರಾಜವೊಡಯನೂ
ಮಲ್ಲರಾಯ
ಮಲ್ಲರಾಯನೆಂಬ
ಮಲ್ಲಸೆಟ್ಟಿಯಾದ
ಮಲ್ಲಾಂಬಿಕೆ
ಮಲ್ಲಾಂಬಿಕೆಗೆ
ಮಲ್ಲಾಂಬಿಕೆಯ
ಮಲ್ಲಾದೇವಿಯಿಂದ
ಮಲ್ಲಾಪುರ
ಮಲ್ಲಿಕಾರ್ಜುನ
ಮಲ್ಲಿಕಾರ್ಜುನದೇವರಿಗೆ
ಮಲ್ಲಿಕಾರ್ಜುನನ
ಮಲ್ಲಿಕಾರ್ಜುನನವರೆಗೆ
ಮಲ್ಲಿಕಾರ್ಜುನನಿಗೆ
ಮಲ್ಲಿಕಾರ್ಜುನನು
ಮಲ್ಲಿಕಾರ್ಜುನನ್ನು
ಮಲ್ಲಿಕಾರ್ಜುನರಾಯನು
ಮಲ್ಲಿಕಾರ್ಜುನೋ
ಮಲ್ಲಿಕ್
ಮಲ್ಲಿಕ್ಕಾಫುರನಿಂದ
ಮಲ್ಲಿಗೆರೆ
ಮಲ್ಲಿತಮ್ಮ
ಮಲ್ಲಿದೇವ
ಮಲ್ಲಿದೇವನ
ಮಲ್ಲಿನಾಥ
ಮಲ್ಲಿನಾಥದೇವರ
ಮಲ್ಲಿನಾಥನ
ಮಲ್ಲಿನಾಥನು
ಮಲ್ಲಿಯಣ
ಮಲ್ಲಿಯಣ್ಣ
ಮಲ್ಲಿಯಣ್ಣನು
ಮಲ್ಲೆನಾಯಕ
ಮಲ್ಲೆನಾಯಕನ
ಮಲ್ಲೆನಾಯಕನಿಗೆ
ಮಲ್ಲೆನಾಯಕರು
ಮಲ್ಲೆಯ
ಮಲ್ಲೆಯನಾಯಕ
ಮಲ್ಲೆಯನಾಯಕನ
ಮಲ್ಲೆಯನಾಯಕನಿಗೆ
ಮಲ್ಲೆಸಾಮಂತನು
ಮಲ್ಲೇನಹಳ್ಳಿ
ಮಲ್ಲೇಶ್ವರ
ಮಳಲಿಯ
ಮಳಲೂರನ್ನು
ಮಳಲೂರು
ಮಳವಳ್ಳಿ
ಮಳವಳ್ಳಿಯ
ಮಳವಳ್ಳಿಯನ್ನು
ಮಳವಳ್ಳಿಯಲ್ಲಿ
ಮಳವಳ್ಳಿಯು
ಮಳೂರಿನಲ್ಲಿಯೂ
ಮಳೂರು
ಮಳೂರುಪಟ್ಟಣ
ಮಸಅಣಯ್ಯನು
ಮಸಣಯ್ಯನ
ಮಸಣಯ್ಯನನ್ನು
ಮಸಣಯ್ಯನಿಗೆ
ಮಸಣಯ್ಯನೆಂಬುವವನನ್ನು
ಮಸಣಿತಂಮನ
ಮಸಣೆನಾಯಕ
ಮಸಣೆಯ
ಮಸಣೈಯ
ಮಸಣೈಯನ
ಮಸಣೈಯನು
ಮಸೀದಿಗೆ
ಮಸೀದಿಯನ್ನು
ಮಸುನಿದೇಶ
ಮಸ್ಜಿದ್
ಮಸ್ತಕಶೂಲ
ಮಸ್ತಕಸೂಲ
ಮಸ್ತಕಾಸೂಲ
ಮಹತ್ತರ
ಮಹತ್ವ
ಮಹತ್ವದ
ಮಹತ್ವದ್ದಾಗಿದೆ
ಮಹತ್ವಪೂರ್ಣ
ಮಹತ್ವವಾದ
ಮಹತ್ವಾಕಾಂಕ್ಷಿಯಾದ
ಮಹತ್ವಾಕಾಂಕ್ಷಿಯೂ
ಮಹತ್ವಾಕಾಂಕ್ಷೆಗೆ
ಮಹದೇವ
ಮಹದೇವಣ್ಣ
ಮಹದೇವಣ್ಣನ
ಮಹದೇವಣ್ಣನಿಂದ
ಮಹದೇವಣ್ಣನು
ಮಹದೇವಣ್ಣನೇ
ಮಹದೇವದಂಡನಾಯಕನನ್ನು
ಮಹದೇವನ
ಮಹದೇವನನ್ನು
ಮಹದೇವನಾಯಕನನ್ನು
ಮಹದೇವನಾಯಕನೆಂಬ
ಮಹದೇವನಿಗೆ
ಮಹದೇವನು
ಮಹದೇವಪುರದ
ಮಹದೇವರಾಣೆ
ಮಹದೇವರಾಣೆಯಂ
ಮಹದೇವರಾಣೆಯಿಂ
ಮಹದೇವವನೀ
ಮಹನೀಯ
ಮಹಪ್ರಧಾನರೆಂದೂ
ಮಹಮಂಡಲೇಶ್ವರರು
ಮಹಮದೀಯರ
ಮಹಮದೀಯರಿಗೆ
ಮಹಮದ್ರಿಜಾ
ಮಹಮ್ಮದ್
ಮಹಲು
ಮಹಾ
ಮಹಾಂಗಮಂತ್ರ
ಮಹಾಂಡಳೇಶ್ವರರ
ಮಹಾಂಭೋದಿ
ಮಹಾಅರಸ
ಮಹಾಅರಸನ
ಮಹಾಅರಸನಾಗಿದ್ದಾನೆ
ಮಹಾಅರಸನಿಗೂ
ಮಹಾಅರಸನಿಗೆ
ಮಹಾಅರಸನು
ಮಹಾಅರಸರು
ಮಹಾಅರಸು
ಮಹಾಅರಸುಕೊನೇಟಿರಾಜ
ಮಹಾಅರಸುಗಳ
ಮಹಾಅರಸುಗಳವರು
ಮಹಾಅರಸುಗಳು
ಮಹಾಅರಸುರಾಮರಾಜಯ್ಯದೇವ
ಮಹಾಕರ್ಣಾಟ
ಮಹಾಕಾಳಿ
ಮಹಾಕ್ರತುಗಳನ್ನು
ಮಹಾಜನಂಗಳು
ಮಹಾಜನಂಗಳ್ಅ
ಮಹಾಜನಗಳ
ಮಹಾಜನಗಳಿಂದ
ಮಹಾಜನಗಳಿಗೆ
ಮಹಾಜನಗಳು
ಮಹಾಜನರಂತೆ
ಮಹಾಜನರನ್ನು
ಮಹಾಜನರಿಂದ
ಮಹಾಜನರಿಗೆ
ಮಹಾಜನರು
ಮಹಾಜನರುಗಳು
ಮಹಾತಟಾಕಾದಿ
ಮಹಾದಂಡನಾಯಕ
ಮಹಾದಂಡಾಧಿಕಾರಿ
ಮಹಾದಾನ
ಮಹಾದಾನದ
ಮಹಾದಾನದೊಳ್
ಮಹಾದಾನವನ್ನು
ಮಹಾದೇವ
ಮಹಾದೇವನ
ಮಹಾದೇವನು
ಮಹಾದೇವನೆಂಬ
ಮಹಾದೇವರ
ಮಹಾದೇವರಇಂದಿನ
ಮಹಾದೇವರಿಗೆ
ಮಹಾದೇವಶಕ್ತಿ
ಮಹಾದೇವಿ
ಮಹಾದೇವಿಯ
ಮಹಾದೇವಿಯರ
ಮಹಾದೇವಿಯು
ಮಹಾದೇವೋತ್ತಮ
ಮಹಾದೇಸಿಗರು
ಮಹಾನಾಡು
ಮಹಾನಾಯಂಕಾಚಾರ್ಯ
ಮಹಾನಾಯಕ
ಮಹಾನಾಯಕರು
ಮಹಾನಾಯಕಾಚಾರ್ಯ
ಮಹಾನುಭಾವನ
ಮಹಾನುಭಾವನಿಂ
ಮಹಾಪಸಾಯತ
ಮಹಾಪಸಾಯಿತ
ಮಹಾಪಸಾಯಿತರೂ
ಮಹಾಪಸಾಯ್ತ
ಮಹಾಪಸಾಯ್ತರು
ಮಹಾಪಸಾಯ್ತರುಪಸಾಯಿತರು
ಮಹಾಪಸಾಯ್ತರೂ
ಮಹಾಪ್ರಚಂಡ
ಮಹಾಪ್ರಚಂಡದಂಡನಾಯಕ
ಮಹಾಪ್ರಚಂಡದಂಡನಾಯಕರ
ಮಹಾಪ್ರದಾನ
ಮಹಾಪ್ರಧನ
ಮಹಾಪ್ರಧಾನ
ಮಹಾಪ್ರಧಾನಂ
ಮಹಾಪ್ರಧಾನದಂಡನಾಯಕನೇ
ಮಹಾಪ್ರಧಾನನ
ಮಹಾಪ್ರಧಾನನಾಗಿ
ಮಹಾಪ್ರಧಾನನಾಗಿದ್ದ
ಮಹಾಪ್ರಧಾನನಾಗಿದ್ದನೆಂದು
ಮಹಾಪ್ರಧಾನನಾದನೆಂದು
ಮಹಾಪ್ರಧಾನನು
ಮಹಾಪ್ರಧಾನನೂ
ಮಹಾಪ್ರಧಾನರ
ಮಹಾಪ್ರಧಾನರನ್ನು
ಮಹಾಪ್ರಧಾನರಾಗಿ
ಮಹಾಪ್ರಧಾನರಾಗಿದ್ದರೂ
ಮಹಾಪ್ರಧಾನರಾದ
ಮಹಾಪ್ರಧಾನರಿಗೆ
ಮಹಾಪ್ರಧಾನರು
ಮಹಾಪ್ರಧಾನರೆಂದು
ಮಹಾಪ್ರಧಾನರೆನಿಸಿಕೊಂಡಿದ್ದರೆಂದು
ಮಹಾಪ್ರಧಾನರೆನಿಸಿದ್ದರೆಂದು
ಮಹಾಪ್ರಧಾನರೇ
ಮಹಾಪ್ರಧಾನಿ
ಮಹಾಪ್ರಧಾನಿಗಳ
ಮಹಾಪ್ರಧಾನೂ
ಮಹಾಪ್ರಭು
ಮಹಾಪ್ರಭುಗಳಾಗಿ
ಮಹಾಪ್ರಭುಗಳು
ಮಹಾಪ್ರಭುಪ್ರಭುವಿಭುಗಳು
ಮಹಾಪ್ರಭುವಾಗಿದ್ದನೆಂದು
ಮಹಾಪ್ರಭುವಾಗಿರಬಹುದು
ಮಹಾಪ್ರಭುವಿನ
ಮಹಾಪ್ರಭುವು
ಮಹಾಪ್ರಭೋತ್ತಮೂರ್ತಿರನೇಕ
ಮಹಾಬಲೇಶ್ವರ
ಮಹಾಬಲೇಶ್ವರನಿಗೆ
ಮಹಾಬಿರುದ
ಮಹಾಭಾರತದ
ಮಹಾಭಾರತದಲ್ಲಿ
ಮಹಾಭಾರತವನ್ನು
ಮಹಾಭೂತ
ಮಹಾಮಂಡಲಿಕ
ಮಹಾಮಂಡಲೇಶ್ವರ
ಮಹಾಮಂಡಲೇಶ್ವರಃ
ಮಹಾಮಂಡಲೇಶ್ವರನನ್ನಾಗಿ
ಮಹಾಮಂಡಲೇಶ್ವರನಾಗಿ
ಮಹಾಮಂಡಲೇಶ್ವರನಾಗಿದ್ದ
ಮಹಾಮಂಡಲೇಶ್ವರನಾಗಿದ್ದು
ಮಹಾಮಂಡಲೇಶ್ವರನಾದ
ಮಹಾಮಂಡಲೇಶ್ವರನಿರಬಹುದು
ಮಹಾಮಂಡಲೇಶ್ವರನು
ಮಹಾಮಂಡಲೇಶ್ವರರ
ಮಹಾಮಂಡಲೇಶ್ವರರನ್ನು
ಮಹಾಮಂಡಲೇಶ್ವರರಲ್ಲಿ
ಮಹಾಮಂಡಲೇಶ್ವರರಾದ
ಮಹಾಮಂಡಲೇಶ್ವರರಿಂದ
ಮಹಾಮಂಡಲೇಶ್ವರರು
ಮಹಾಮಂಡಲೇಶ್ವರರುಮಹಾಸಾಮಂತರು
ಮಹಾಮಂಡಲೇಶ್ವರರೆಂದು
ಮಹಾಮಂಡಲೇಶ್ವರರೆಲ್ಲರೂ
ಮಹಾಮಂಡಲೇಶ್ವರರೇ
ಮಹಾಮಂಡಳಿಕ
ಮಹಾಮಂಡಳೇಶ್ವನಾಗಿ
ಮಹಾಮಂಡಳೇಶ್ವರ
ಮಹಾಮಂಡಳೇಶ್ವರಂ
ಮಹಾಮಂಡಳೇಶ್ವರನಾಗಿ
ಮಹಾಮಂಡಳೇಶ್ವರರ
ಮಹಾಮಂಡಳೇಶ್ವರರಾಗಿದ್ದ
ಮಹಾಮಂಡಳೇಶ್ವರರಾಗಿದ್ದರಿಂದ
ಮಹಾಮಂಡಳೇಶ್ವರರಾಗಿದ್ದರು
ಮಹಾಮಂಡಳೇಶ್ವರರಾದರೂ
ಮಹಾಮಂಡಳೇಶ್ವರರು
ಮಹಾಮಂಡಳೇಶ್ವರರುಮಂಡಳೇಶ್ವರರ
ಮಹಾಮಂತ್ರಿ
ಮಹಾಮತ್ಯಪದ
ಮಹಾಮಮಂಡಲೇಶ್ವರರು
ಮಹಾಮಹತ್ತಿನ
ಮಹಾಮಹಿಪಂ
ಮಹಾಮಾತ್ಯ
ಮಹಾಮಾತ್ಯನು
ಮಹಾಮಾತ್ಯಪದವಿ
ಮಹಾಮಾತ್ಯಪದವೀ
ಮಹಾಯಶಾಃ
ಮಹಾರಾಜ
ಮಹಾರಾಜನಾದ
ಮಹಾರಾಜರಿಗೆ
ಮಹಾರಾಜಾಧಿರಾಜ
ಮಹಾರಾಜಾಧಿರಾಜನೆಂದು
ಮಹಾರಾಜ್ಯ
ಮಹಾರಾಜ್ಯಕ್ಕೆ
ಮಹಾರಾಜ್ಯದಲ್ಲಿ
ಮಹಾರಾಣಿ
ಮಹಾರಾಯನ
ಮಹಾರಾಯನು
ಮಹಾರಾಯರ
ಮಹಾರಾಯರಿಗೆ
ಮಹಾರಾಯರು
ಮಹಾರಾಯರೂ
ಮಹಾರಾಯೋ
ಮಹಾರಾಷ್ಟ್ರಕಗಳಾಗಿ
ಮಹಾರ್ನ್ನವ
ಮಹಾಲಕ್ಷ್ಮಿ
ಮಹಾಲಕ್ಷ್ಮಿಯ
ಮಹಾವಡ್ಡವ್ಯವಹಾರಿ
ಮಹಾವಿಭವದಿ
ಮಹಾವೀರ
ಮಹಾಶಾಸನವನ್ನು
ಮಹಾಸಮಾಂತಾಧಿಪತಿ
ಮಹಾಸಮುದ್ರಮಾ
ಮಹಾಸಮುದ್ರವೆಂದು
ಮಹಾಸಾಂತಾಧಿಪತಿ
ಮಹಾಸಾಮಂತ
ಮಹಾಸಾಮಂತನನ್ನಾಗಿ
ಮಹಾಸಾಮಂತನಾಗಿ
ಮಹಾಸಾಮಂತನಾಗಿದ್ದ
ಮಹಾಸಾಮಂತನಾಗಿದ್ದರೂ
ಮಹಾಸಾಮಂತನಾಗಿದ್ದುದು
ಮಹಾಸಾಮಂತನಾಗುತ್ತಿದ್ದನು
ಮಹಾಸಾಮಂತನೆಂದು
ಮಹಾಸಾಮಂತನೋ
ಮಹಾಸಾಮಂತರ
ಮಹಾಸಾಮಂತರನ್ನು
ಮಹಾಸಾಮಂತರಾಗಿ
ಮಹಾಸಾಮಂತರಾಗಿದ್ದ
ಮಹಾಸಾಮಂತರಾಗಿದ್ದರೆಂಬುದನ್ನ
ಮಹಾಸಾಮಂತರಿಗೆ
ಮಹಾಸಾಮಂತರು
ಮಹಾಸಾಮಂತಾಧಿಪತಿ
ಮಹಾಸಾಮಂತಾಧಿಪತಿಗಳು
ಮಹಾಸಾಮ್ರಾಜ್ಯವಾಗಿತ್ತು
ಮಹಾಸೇನಾ
ಮಹಾಸೇನಾಸಮುದ್ರ
ಮಹಾಸೋಪಾನವನ್ನು
ಮಹಾಸ್ವಾಮಿಯರಿಗೆ
ಮಹಾಸ್ವಾಮಿಯವರಿಂದ
ಮಹಾಸ್ವಾಮಿಯವರು
ಮಹಾಹೋಸಲ
ಮಹಾಹೋಸಲನಾಡ
ಮಹಾಹೋಸಲನಾಡು
ಮಹಿಪನ
ಮಹಿಮೆ
ಮಹಿಶುರ
ಮಹಿಶೂರ
ಮಹಿಶೂರನಗರದ
ಮಹಿಶೂರನಗರದಲ್ಲಿ
ಮಹಿಷಿ
ಮಹಿಷಿಕಾ
ಮಹಿಷೀ
ಮಹಿಸೂರ
ಮಹೀಂ
ಮಹೀಕರನೂ
ಮಹೀಪಾಲಕರು
ಮಹೀಪಾಳಕಃ
ಮಹೀಪೇ
ಮಹೀಭುಜನುವಿಷ್ಣುವರ್ಧನ
ಮಹೀಮಂಡಳವನ್ನು
ಮಹೀವಲ್ಲಭ
ಮಹೀಶೂರದಳವಾಯಿ
ಮಹೂರ್ತದಲ್ಲಿ
ಮಹೂರ್ತವನ್ನೂ
ಮಹೇಂದ್ರ
ಮಹೇಂದ್ರನನ್ನು
ಮಹೇಂದ್ರನು
ಮಹೇಂದ್ರನೇ
ಮಹೇಶ್ವರಿ
ಮಹೋಗ್ರಾಜಿಯೊಳಾಂತಿದಿರ್ಚಿದದಟಿಂ
ಮಹೋಗ್ರಾಜಿಯೊಳಾಂತಿದಿರ್ಚ್ಚಿದದಟಿಂ
ಮಹೋತ್ತಮನಾಗಿದ್ದನು
ಮಹೋತ್ಸವವಅನ್ನು
ಮಹೋತ್ಸವವಾಯಿತೆಂದು
ಮಾ
ಮಾಂಡಲಿಕ
ಮಾಂಡಲಿಕನಾಗಿ
ಮಾಂಡಲಿಕನಾಗಿದ್ದನು
ಮಾಂಡಲಿಕನಾದ
ಮಾಂಡಲಿಕನೂ
ಮಾಂಡಲಿಕರ
ಮಾಂಡಲಿಕರನ್ನಾಗಿ
ಮಾಂಡಲಿಕರಾಗಿ
ಮಾಂಡಲಿಕರಾಗಿದ್ದ
ಮಾಂಡಲಿಕರಾಗಿದ್ದರೆಂದು
ಮಾಂಡಲಿಕರಾಗಿದ್ದರೆಂಬುದು
ಮಾಂಡಲಿಕರು
ಮಾಂಡಲಿಕರೊಡನೆ
ಮಾಂಡಲೀಕನಾಗಿ
ಮಾಂಡಲೀಕನೂ
ಮಾಂಡಲೀಕರಾಗಿ
ಮಾಂಡಲೀಕರಾಗಿದ್ದ
ಮಾಂಡಲೀಕರು
ಮಾಂಡವ್ಯ
ಮಾಂಬಳ್ಳಿ
ಮಾಕಣಬ್ಬೆ
ಮಾಕಣಬ್ಬೆಯರ
ಮಾಕಲೆ
ಮಾಕಲೆಯರ
ಮಾಕುಂದಮುಕುಂದ
ಮಾಕುಬಳ್ಳಿಮಾಕವಳ್ಳಿ
ಮಾಕೇಶ್ವರ
ಮಾಗಡಿ
ಮಾಗಡಿಯ
ಮಾಗಣಿ
ಮಾಗಣಿಗೆ
ಮಾಗಣಿಯ
ಮಾಗಣಿಯಾಗಿತ್ತು
ಮಾಗಣಿಯೊಳಗೆ
ಮಾಗಣೆಗೆ
ಮಾಗಣೆಯ
ಮಾಗನೂರು
ಮಾಗಲ
ಮಾಗಳಿ
ಮಾಗುಂಡರಕಿಲ್ಲ
ಮಾಘ
ಮಾಘಣಂದಿ
ಮಾಚಗವುಡ
ಮಾಚಗೌಂಡ
ಮಾಚಣ
ಮಾಚಣನ
ಮಾಚದಂಡಾಧೀಶನು
ಮಾಚನ
ಮಾಚನಕಟ್ಟ
ಮಾಚನಕಟ್ಟಕ್ಕೆ
ಮಾಚನಕಟ್ಟದ
ಮಾಚನಕಟ್ಟದಲ್ಲಿ
ಮಾಚನಕಟ್ಟೆಯ
ಮಾಚನು
ಮಾಚಮಯ್ಯ
ಮಾಚಮಯ್ಯನು
ಮಾಚಯ್ಯ
ಮಾಚಯ್ಯದಂಡನಾಯಕ
ಮಾಚಯ್ಯನ
ಮಾಚಯ್ಯನನ್ನು
ಮಾಚಯ್ಯನಿಗೆ
ಮಾಚಯ್ಯನು
ಮಾಚಲಗಅಟ್ಟ
ಮಾಚಲರಾಣಿ
ಮಾಚಲೆ
ಮಾಚಲೆನಾರಿ
ಮಾಚಳೇಶ್ವರ
ಮಾಚವಳಲು
ಮಾಚವ್ವೆ
ಮಾಚವ್ವೆಯನ್ನೂ
ಮಾಚಸಮುದ್ರ
ಮಾಚಿಕೆ
ಮಾಚಿಗೆಹಳ್ಳಿಮಾಚನಹಳ್ಳಿ
ಮಾಚಿದೇವ
ಮಾಚಿದೇವನ
ಮಾಚಿದೇವನು
ಮಾಚಿನಾಯಕನಹಳ್ಳಿ
ಮಾಚಿಮಯ್ಯನು
ಮಾಚಿರಾಜ
ಮಾಚಿರಾಜಂ
ಮಾಚಿರಾಜಂಗೆ
ಮಾಚಿರಾಜನ
ಮಾಚಿರಾಜನನ್ನು
ಮಾಚಿರಾಜನು
ಮಾಚಿರಾಜನೂ
ಮಾಚಿರಾಜರಾಗಿದ್ದಾರೆಂದು
ಮಾಚಿರಾಜರೆಲ್ಲರೂ
ಮಾಚೀದೇವ
ಮಾಚೀದೇವನ
ಮಾಚೆಗವುಡ
ಮಾಚೆಗೌಡನ
ಮಾಚೆಯನಾಯಕ
ಮಾಚೆಯನಾಯಕನ
ಮಾಚೆಯನಾಯಕನಿಗೆ
ಮಾಚೆಯನಾಯಕನು
ಮಾಚೆಯನಾಯಕನೆಂಬ
ಮಾಚೆಯನು
ಮಾಚೋಜನ
ಮಾಡತಕ್ಕ
ಮಾಡತೊಡಗಿದರು
ಮಾಡದೇ
ಮಾಡಬಹುದು
ಮಾಡಬೇಕಾದ
ಮಾಡಲ
ಮಾಡಲಾಗಿದೆ
ಮಾಡಲಾಗಿದೆಯೆಂದು
ಮಾಡಲಾಯಿತಾದರೂ
ಮಾಡಲಾಯಿತು
ಮಾಡಲಾಯಿತೆಂದು
ಮಾಡಲಿಲ್ಲ
ಮಾಡಲು
ಮಾಡಲೋಸುಗ
ಮಾಡಲ್ಪಟ್ಟ
ಮಾಡವ
ಮಾಡಿ
ಮಾಡಿಕೊಂಡ
ಮಾಡಿಕೊಂಡನು
ಮಾಡಿಕೊಂಡನೆಂದು
ಮಾಡಿಕೊಂಡರು
ಮಾಡಿಕೊಂಡರೆಂದು
ಮಾಡಿಕೊಂಡಿರಬಹುದು
ಮಾಡಿಕೊಂಡಿರುವವರ
ಮಾಡಿಕೊಂಡಿರುವುದಿಲ್ಲ
ಮಾಡಿಕೊಂಡು
ಮಾಡಿಕೊಟ್ಟನೆಂದೂ
ಮಾಡಿಕೊಟ್ಟಿದ್ದನು
ಮಾಡಿಕೊಡಲಾಗಿತ್ತು
ಮಾಡಿಕೊಳ್ಳುತ್ತಾರೆ
ಮಾಡಿದ
ಮಾಡಿದಂ
ಮಾಡಿದಂತೆ
ಮಾಡಿದಂಥಾ
ಮಾಡಿದನು
ಮಾಡಿದನೆಂದು
ಮಾಡಿದನೆಂದೂ
ಮಾಡಿದರು
ಮಾಡಿದರೆಂದಿದೆ
ಮಾಡಿದರೆಂದು
ಮಾಡಿದರೆಂದೂ
ಮಾಡಿದಲ್ಲಿ
ಮಾಡಿದವನು
ಮಾಡಿದವನೆಂದರೆ
ಮಾಡಿದಾಗ
ಮಾಡಿದುದಕ್ಕಾಗಿ
ಮಾಡಿದುದರ
ಮಾಡಿದೆ
ಮಾಡಿದ್ದ
ಮಾಡಿದ್ದಕ್ಕಾಗಿ
ಮಾಡಿದ್ದನೆಂದು
ಮಾಡಿದ್ದನ್ನು
ಮಾಡಿದ್ದನ್ನುಹೇಳಿದೆ
ಮಾಡಿದ್ದರಿಂದ
ಮಾಡಿದ್ದರೆಂದು
ಮಾಡಿದ್ದಾನೆ
ಮಾಡಿದ್ದಾರೆ
ಮಾಡಿಯೇ
ಮಾಡಿರಬಹುದು
ಮಾಡಿರುವ
ಮಾಡಿರುವುದು
ಮಾಡಿಸಿ
ಮಾಡಿಸಿಕೊಟಿರುವ
ಮಾಡಿಸಿಕೊಟ್ಟಂತೆ
ಮಾಡಿಸಿಕೊಟ್ಟನು
ಮಾಡಿಸಿಕೊಟ್ಟರು
ಮಾಡಿಸಿಕೊಟ್ಟರೆಂದು
ಮಾಡಿಸಿಕೊಟ್ಟಳು
ಮಾಡಿಸಿಕೊಟ್ಟಿದ್ದಾನೆ
ಮಾಡಿಸಿದ
ಮಾಡಿಸಿದಂ
ಮಾಡಿಸಿದಂತೆ
ಮಾಡಿಸಿದನ
ಮಾಡಿಸಿದನಿನ್ತೀ
ಮಾಡಿಸಿದನು
ಮಾಡಿಸಿದನೆಂದಿದೆ
ಮಾಡಿಸಿದನೆಂದು
ಮಾಡಿಸಿದರು
ಮಾಡಿಸಿದರೆಂದು
ಮಾಡಿಸಿದಳು
ಮಾಡಿಸಿದಾಗ
ಮಾಡಿಸಿದುದರ
ಮಾಡಿಸಿದ್ದನಂತೆ
ಮಾಡಿಸಿದ್ದಾರೆ
ಮಾಡಿಸುತ್ತಾರೆ
ಮಾಡಿಸುತ್ತಿದ್ದಾಗ
ಮಾಡು
ಮಾಡುತ್ತಾ
ಮಾಡುತ್ತಾನೆ
ಮಾಡುತ್ತಾರೆ
ಮಾಡುತ್ತಿದರೆಂದು
ಮಾಡುತ್ತಿದೆ
ಮಾಡುತ್ತಿದ್ದಂತೆ
ಮಾಡುತ್ತಿದ್ದನು
ಮಾಡುತ್ತಿದ್ದನೆಂದು
ಮಾಡುತ್ತಿದ್ದರು
ಮಾಡುತ್ತಿದ್ದರೆಂದು
ಮಾಡುತ್ತಿದ್ದರೆಂದೂ
ಮಾಡುತ್ತಿದ್ದಾಗ
ಮಾಡುತ್ತಿದ್ದುದಕ್ಕಾಗಿ
ಮಾಡುತ್ತಿದ್ದುದು
ಮಾಡುತ್ತಿರಲು
ಮಾಡುತ್ತೇನೆಂದು
ಮಾಡುವ
ಮಾಡುವಂತಹ
ಮಾಡುವಂತೆ
ಮಾಡುವಲ್ಲಿ
ಮಾಡುವವರನ್ನು
ಮಾಡುವಾಗ
ಮಾಡುವಾಗಲೂ
ಮಾಡುವುದರಿಂದ
ಮಾಡುವುದು
ಮಾಣದ
ಮಾಣಿ
ಮಾಣಿಕ
ಮಾಣಿಕಭಂಡಾರಿಗಳಾಗಿರುತ್ತಿದ್ದರೆಂದು
ಮಾಣಿಕಭಂಡಾರಿಗಳು
ಮಾಣಿಕ್ಯ
ಮಾಣಿಕ್ಯಭಂಡಾರದ
ಮಾಣಿಕ್ಯಭಂಡಾರಿ
ಮಾಣಿಕ್ಯವೊಳಲ
ಮಾಣಿಕ್ಯವೊಳಲು
ಮಾಣಿಕ್ಯವೊಳಲೆಂಬ
ಮಾಣಿಯೊಳಗಣ
ಮಾತನ್ನು
ಮಾತ್ರ
ಮಾತ್ರಕ್ಕೆ
ಮಾದಗವುಡಿಯ
ಮಾದಣ್ಣ
ಮಾದಣ್ಣನಿಗೆ
ಮಾದಣ್ಣನು
ಮಾದಪ್ಪ
ಮಾದಪ್ಪದಂಡನಾಯಕನೂ
ಮಾದಪ್ಪದಣ್ನಾಯಕರ
ಮಾದಪ್ಪನು
ಮಾದರಗವುಡಿ
ಮಾದಲಗೆರೆ
ಮಾದಲಮಹದೇವಿಯರು
ಮಾದಲಮಹಾದೇವಿ
ಮಾದಲಮಹಾದೇವಿಯರು
ಮಾದಳ
ಮಾದಾಪುರ
ಮಾದಿಗಉಡ
ಮಾದಿಗರುಳ
ಮಾದಿಗವುಡನ
ಮಾದಿಗವುಡನಿಗೆ
ಮಾದಿಗೌಡ
ಮಾದಿದೇವ
ಮಾದಿದೇವನ
ಮಾದಿಯಣ್ಣ
ಮಾದಿರಾಜ
ಮಾದಿರಾಜನ
ಮಾದಿರಾಜನನ್ನು
ಮಾದಿರಾಜನು
ಮಾದಿವೆಗ್ಗಡೆ
ಮಾದಿವೆಗ್ಗಡೆಯು
ಮಾದಿಹಳ್ಳಿ
ಮಾದಿಹಳ್ಳಿಯ
ಮಾದಿಹಳ್ಳಿಯನ್ನು
ಮಾದೆಯ
ಮಾದೆಯನಾಯಕ
ಮಾದೆಯನಾಯಕನು
ಮಾದೆಹಳ್ಳಿ
ಮಾದೇಗೌಡನ
ಮಾದೇವ
ಮಾಧವ
ಮಾಧವಚೋಳನಹಳ್ಳಿಯ
ಮಾಧವಚೋಳಯನಹಳ್ಳಿಯ
ಮಾಧವತ್ತಿಮಾಧವಶಕ್ತಿ
ಮಾಧವದಂಡನಾಯಕ
ಮಾಧವದಂಡನಾಯಕನ
ಮಾಧವದಂಡನಾಯಕನಿಗೆ
ಮಾಧವದಂಣಾಯಕರುಂ
ಮಾಧವದೇವರ
ಮಾಧವದೇವರಿಗೆ
ಮಾಧವನನ್ನು
ಮಾಧವನಿಗೆ
ಮಾಧವನು
ಮಾಧ್ಯಮವನ್ನಾಗಿ
ಮಾಧ್ವರ
ಮಾನಗಳಿಂದ
ಮಾನದ
ಮಾನವ
ಮಾನವದುರ್ಗವನ್ನುಮಾನವಿ
ಮಾನವನ
ಮಾನವರೊಳು
ಮಾನವರೊಳ್
ಮಾನವಾಕಾರ
ಮಾನವಾಕಾರವನ್ನು
ಮಾನವಾಗಿದ್ದು
ಮಾನಸರೂಪವಾದುದೋ
ಮಾನಸ್ತಂಭ
ಮಾನಸ್ತಂಭದ
ಮಾನಸ್ಥಂಬ
ಮಾನಿಸ
ಮಾನಿಸೆಟ್ಟಿಗೆ
ಮಾನ್ಯ
ಮಾನ್ಯಖೇಟವನ್ನ
ಮಾನ್ಯಗಳನ್ನು
ಮಾನ್ಯಗ್ರಾಮಗಳಲ್ಲಿ
ಮಾನ್ಯತೆ
ಮಾನ್ಯಪುರದಲ್ಲಿ
ಮಾನ್ಯವನ್ನು
ಮಾನ್ಯವಾಗಿ
ಮಾಬಲಯ್ಯ
ಮಾಬಲಯ್ಯಂ
ಮಾಬಲಯ್ಯನ
ಮಾಬಲಯ್ಯನಿಗೆ
ಮಾಬಲಯ್ಯನು
ಮಾಬಲಯ್ಯನೆಂದೊಗೞದರಾರ್
ಮಾಬಳಯ್ಯ
ಮಾಬಳ್ಳಿ
ಮಾಬಹಳ್ಳಿ
ಮಾಮಲೆದಾರನಾಗಿದ್ದಂತೆ
ಮಾಮಲೆದಾರ್
ಮಾಯಣನಪುರ
ಮಾಯಣ್ಣ
ಮಾಯಣ್ಣನ
ಮಾಯಣ್ಣನಿಗೆ
ಮಾಯಣ್ಣನು
ಮಾಯಣ್ಣನೂ
ಮಾಯಣ್ಣನೆಂಬುವವನಿಗೆ
ಮಾಯಪ್ಪನಿಗೆ
ಮಾಯಪ್ಪಹಳ್ಳಿದೇಪಸಾಗರ
ಮಾಯಸಂದ್ರಕ್ಕೆ
ಮಾಯಸಂದ್ರವೇ
ಮಾಯಸಮುದ್ರ
ಮಾಯಸಮುದ್ರವಾಗಿದೆ
ಮಾಯಿಗೌಡನ
ಮಾಯಿಲಂಗೆ
ಮಾಯಿಲಂಗೆಯಲ್ಲಿ
ಮಾರಗವುಂಡನ
ಮಾರಗಾಮುಂಡ
ಮಾರಗಾಮುಂಡನ
ಮಾರಗೂಳಿ
ಮಾರಗೊಂಡನಹಳ್ಳಿ
ಮಾರಗೊಂಡನಹಳ್ಳಿಯನ್ನು
ಮಾರಗೊಂಡಹಳ್ಳಿ
ಮಾರಗೌಂಡ
ಮಾರಗೌಡ
ಮಾರಗೌಡನ
ಮಾರಣ್ಣ
ಮಾರಣ್ಣನು
ಮಾರಥ
ಮಾರದೇವನ
ಮಾರದೇವನು
ಮಾರನಾಯಕ
ಮಾರನಾಯಕನ
ಮಾರನಾಯಕನು
ಮಾರಪ್ಪ
ಮಾರಯ್ಯ
ಮಾರಯ್ಯನ
ಮಾರವ್ವೆಯ್ವೆ
ಮಾರಸಿಂಗ
ಮಾರಸಿಂಗಗಾವುಂಡನು
ಮಾರಸಿಂಹ
ಮಾರಸಿಂಹದೇವ
ಮಾರಸಿಂಹದೇವನ
ಮಾರಸಿಂಹನ
ಮಾರಸಿಂಹನನ್ನು
ಮಾರಸಿಂಹನಲ್ಲಿ
ಮಾರಸಿಂಹನಲ್ಲೇ
ಮಾರಸಿಂಹನಿಗೆ
ಮಾರಸಿಂಹನು
ಮಾರಾಟ
ಮಾರಾಟಕ್ಕೆ
ಮಾರಾಟಮಾಡುತ್ತಾರೆ
ಮಾರಾಯನ್
ಮಾರಿಕೊಳ್ಳುತ್ತಾರೆ
ಮಾರಿರಬಹುದು
ಮಾರುಗನೆಂದು
ಮಾರುಗನೆಂಬುವವನು
ಮಾರುಗನೇ
ಮಾರುಗೋನಹಳ್ಳಿ
ಮಾರೂರ
ಮಾರೂರುಗಳ
ಮಾರೆಯ
ಮಾರೆಯನಾಯಕ
ಮಾರೆಯನಾಯಕನಿಗೂ
ಮಾರೆಯನಾಯ್ಕನ
ಮಾರೆಯ್ಯನೆಂಬುವವನೂ
ಮಾರೆಹಳ್ಳಿ
ಮಾರೆಹಳ್ಳಿಯ
ಮಾರೇಹಳ್ಳಿ
ಮಾರೇಹಳ್ಳಿಯ
ಮಾರ್ಕ್ಕೊಂಡು
ಮಾರ್ಕ್ಕೋಲಭೈರವಂ
ಮಾರ್ಗ
ಮಾರ್ಗದರ್ಶನ
ಮಾರ್ಗವಾಗಿ
ಮಾರ್ಗ್ಗಸಿರ
ಮಾರ್ಚನಹಳ್ಳಿಯನ್ನು
ಮಾರ್ಚಹಳ್ಳಿಯಲ್ಲೂ
ಮಾರ್ಚ್
ಮಾರ್ಚ್ರ
ಮಾರ್ತಾಂಡ
ಮಾರ್ತಾಂಡನುಂ
ಮಾರ್ತಾಂಡನೆಂದು
ಮಾರ್ಪಟ್ಟಿತು
ಮಾರ್ಪಟ್ಟಿತ್ತು
ಮಾರ್ಪಡಿಸಬೇಕಾಗುತ್ತದೆ
ಮಾರ್ಪಡಿಸಿ
ಮಾರ್ಪಾಡಾಗಿದ್ದವು
ಮಾರ್ಪ್ಪಾಡಿ
ಮಾರ್ಪ್ಪೆನಾನೆಂದೀಗಳು
ಮಾಲಗಾರನಹಳ್ಳಿಯ
ಮಾಲನಹಳ್ಳಿ
ಮಾಲೂರು
ಮಾಲೆಯಹಳ್ಳಿವೊಳಗಾದ
ಮಾಲ್ಯದ
ಮಾಳಗುಂದಮಾಳಗೂರು
ಮಾಳಗೂರಿನ
ಮಾಳಗೂರು
ಮಾಳಗೂರೇ
ಮಾಳವ
ಮಾಳವರಾಜ್ಯ
ಮಾಳವ್ವೆ
ಮಾಳವ್ವೆಯ
ಮಾಳಾನಹಳ್ಳಿಯು
ಮಾಳಿಗೆ
ಮಾಳಿಗೆಯ
ಮಾಳಿಗೆಯನ್ನು
ಮಾಳಿಗೆಯಲ್ಲಿ
ಮಾಳಿಗೆಯೂರನ್ನು
ಮಾಳಿಗೆಯೂರಿನ
ಮಾಳುಗಾಳ
ಮಾಳೆಯ
ಮಾಳೆಯನಹಳ್ಳಿಯನ್ನು
ಮಾಳೇನಹಳ್ಳಿ
ಮಾವ
ಮಾವಂ
ಮಾವಂದಿರಾಗಿದ್ದರೆಂದು
ಮಾವಂದಿರೆಂದು
ಮಾವನಂಕಕಾರ
ಮಾವನಾಗುತ್ತಾನೆ
ಮಾವನಾದ
ಮಾವಳ್ಳಿ
ಮಾವಿನಕೆರೆ
ಮಾವಿನಕೆರೆಯನ್ನು
ಮಾವಿನಕೆರೆಯನ್ನೂ
ಮಾವಿನಬನವನ್ನು
ಮಾಸಮದೇಭಾವಳಿಯಂ
ಮಾಸವಾಡಿ
ಮಾಸವೆಗ್ಗಡೆ
ಮಾಸವೆಗ್ಗಡೆಗಳ
ಮಾಸ್ತಮ್ಮನ
ಮಾಸ್ತಿಕಲ್ಲಿನಲ್ಲಿ
ಮಾಸ್ತಿಕಲ್ಲು
ಮಾಹಾಸಾಮನ್ತ
ಮಾಹಿತಿ
ಮಾಹಿತಿಗಳನ್ನು
ಮಾಹಿತಿಗಳಿರುವುದರಿಂದ
ಮಾಹಿತಿಗಳು
ಮಾಹಿತಿಯನ್ನು
ಮಾಹಿತಿಯಿಂದ
ಮಾಹೇಶ್ವರನಾಗಿದ್ದು
ಮಿಂಚಗವುಂಡನ
ಮಿಂದು
ಮಿಕ್ಕದ್ದನ್ನು
ಮಿಗಿಲಾಗಿ
ಮಿಗಿಲಾದುದು
ಮಿಗಿಲೆನಿಪಂ
ಮಿಗೆ
ಮಿತಿಯಲ್ಲಿಯೇ
ಮಿತಿಯೇ
ಮಿತ್ರನಂತಿದ್ದನು
ಮಿತ್ರನಾಗಿದ್ದಿರಬಹುದು
ಮಿತ್ರರಾಜ್ಯವಾದ
ಮಿರಾನ್
ಮಿರುಹನಗಣ್ಯ
ಮಿರ್ಲೆ
ಮಿರ್ಲೆಶಾಸನೋಕ್ತನಾಗಿದ್ದಾನೆ
ಮಿಸುಪೆಸೆವ
ಮೀಟರ್
ಮೀನುಗಳಿವೆ
ಮೀರಿಸಿದನು
ಮೀರ್
ಮೀರ್ಇಮಿರಾನ್
ಮೀರ್ಜೈನ್ಉನ್
ಮೀರ್ಮಹಮದ್
ಮೀಸರಗಂಡ
ಮೀಸಲಾಗಿಟ್ಟು
ಮೀಸಲು
ಮುಂಗೊಳ
ಮುಂಚೆ
ಮುಂಚೆಯೇ
ಮುಂಜಿಯನೂ
ಮುಂಡಿಗೈ
ಮುಂತಾಗಿ
ಮುಂತಾದ
ಮುಂತಾದವರನ್ನು
ಮುಂತಾದವರು
ಮುಂತಾದವು
ಮುಂತಾದವುಗಳನ್ನು
ಮುಂತಾದವುಗಳು
ಮುಂತಿದಿರಾಂತನಂತರಿಪು
ಮುಂದಕ್ಕೆ
ಮುಂದಣ
ಮುಂದಾಗುತ್ತಿದ್ದರು
ಮುಂದಿಟ್ಟಕೊಂಡು
ಮುಂದಿಟ್ಟುಕೊಂಡು
ಮುಂದಿದೆ
ಮುಂದಿನ
ಮುಂದಿನಂತೆ
ಮುಂದಿರೆ
ಮುಂದುರಿಯಿತು
ಮುಂದುವರಿದ
ಮುಂದುವರಿದನು
ಮುಂದುವರಿದರೂ
ಮುಂದುವರಿದವು
ಮುಂದುವರಿದಿತ್ತು
ಮುಂದುವರಿದಿತ್ತೆಂದು
ಮುಂದುವರಿದಿದೆ
ಮುಂದುವರಿದಿದ್ದನೆಂದು
ಮುಂದುವರಿದಿದ್ದರೆ
ಮುಂದುವರಿದಿದ್ದಾನೆ
ಮುಂದುವರಿದಿದ್ದು
ಮುಂದುವರಿದಿರಬಹುದು
ಮುಂದುವರಿದಿರಬಹುದೆಂದು
ಮುಂದುವರಿದಿರುವ
ಮುಂದುವರಿದಿರುವುದು
ಮುಂದುವರಿದು
ಮುಂದುವರಿಯಿತು
ಮುಂದುವರಿಯಿತೆಂದು
ಮುಂದುವರಿಯುತ್ತದೆ
ಮುಂದುವರಿಸಲಾಯಿತೆಂದು
ಮುಂದುವರಿಸಲು
ಮುಂದುವರಿಸಿ
ಮುಂದುವರಿಸಿಕೊಳ್ಳುತ್ತಾರೆ
ಮುಂದುವರಿಸಿದ
ಮುಂದುವರಿಸಿದಂತೆ
ಮುಂದುವರಿಸಿದನು
ಮುಂದುವರಿಸಿದರು
ಮುಂದುವರಿಸಿದ್ದರೆಂದು
ಮುಂದುವರಿಸಿದ್ದಾನೆ
ಮುಂದುವರಿಸಿರುವ
ಮುಂದುವರಿಸುತ್ತಾರೆ
ಮುಂದುವರೆದರು
ಮುಂದುವರೆದಿದ್ದನ್ನು
ಮುಂದುವರೆಸಿದನೆಂಬುದು
ಮುಂದೆ
ಮುಂದೊಡ್ಡಿ
ಮುಂನಂ
ಮುಂಬರಿದು
ಮುಕುಳಿಕೆರೆಯ
ಮುಕ್ತಗೊಳಿಸಿ
ಮುಕ್ತಹಸ್ತದಿಂದ
ಮುಕ್ತಾಯವಾಯಿತು
ಮುಕ್ತ್ಯಾಂಗನವಲ್ಲಭೋ
ಮುಖವಾಗಿದ್ದ
ಮುಖಸುರರತ್ನದರ್ಪ್ಪಣಂ
ಮುಖಾಂತರ
ಮುಖ್ಯ
ಮುಖ್ಯಕೇಂದ್ರಗಳನ್ನು
ಮುಖ್ಯಕೇಂದ್ರವನ್ನಾಗಿ
ಮುಖ್ಯಕೋಟೆ
ಮುಖ್ಯನಾಗಿದ್ದನೆಂದು
ಮುಖ್ಯನಾಗಿದ್ದನೆಂದೂ
ಮುಖ್ಯಪಟ್ಟಣವಾಗಿತ್ತೆಂದು
ಮುಖ್ಯಪಾತ್ರ
ಮುಖ್ಯಪಾತ್ರವನ್ನು
ಮುಖ್ಯಮಂತ್ರಿಗೆ
ಮುಖ್ಯಮಂತ್ರಿಯ
ಮುಖ್ಯಮಪ್ಪ
ಮುಖ್ಯರಪ್ಪ
ಮುಖ್ಯಲಕ್ಷಣವಾಗಿತ್ತು
ಮುಖ್ಯವಪ್ಪ
ಮುಖ್ಯವಾಗಿ
ಮುಖ್ಯವಾಗಿತ್ತು
ಮುಖ್ಯವಾಗಿತ್ತೆಂಬುದನ್ನು
ಮುಖ್ಯವಾದ
ಮುಖ್ಯವಾದುದೆಂದರೆ
ಮುಖ್ಯಸ್ಥ
ಮುಖ್ಯಸ್ಥನಾಗಿ
ಮುಖ್ಯಸ್ಥನಾಗಿದ್ದನು
ಮುಖ್ಯಸ್ಥನಾಗಿದ್ದನೆಂದು
ಮುಖ್ಯಸ್ಥನಾಗಿದ್ದು
ಮುಖ್ಯಸ್ಥನಾಗಿರಬಹುದು
ಮುಖ್ಯಸ್ಥನೆಂದು
ಮುಖ್ಯಸ್ಥರನ್ನಾಗಿ
ಮುಖ್ಯಸ್ಥರನ್ನು
ಮುಖ್ಯಸ್ಥರಾಗಿ
ಮುಖ್ಯಸ್ಥರಾಗಿದ್ದ
ಮುಖ್ಯಸ್ಥರಾಗಿದ್ದರು
ಮುಖ್ಯಸ್ಥರಾಗಿದ್ದರೆಂದು
ಮುಖ್ಯಸ್ಥರಾಗಿದ್ದರೆಂದೂ
ಮುಖ್ಯಸ್ಥರಾಗಿರುತ್ತಿದ್ದರು
ಮುಖ್ಯಸ್ಥರಾದ
ಮುಖ್ಯಸ್ಥರು
ಮುಖ್ಯಸ್ಥಳ
ಮುಖ್ಯಸ್ಥಳವಾಗಿ
ಮುಖ್ಯಸ್ಥಳವಾಗಿತ್ತು
ಮುಖ್ಯಸ್ಥಳವಾಗಿತ್ತೆಂದು
ಮುಖ್ಯಸ್ಥಳವಾಗಿದ್ದ
ಮುಖ್ಯಸ್ಥಳವಾಗಿದ್ದು
ಮುಖ್ಯಸ್ಥಳವಾದ
ಮುಖ್ಯಾಧಿಕಾರಿಗಳಾಗಿ
ಮುಗಿದ
ಮುಗಿದಿತ್ತು
ಮುಗುಳ್ನಗೆಯೊಂದಿಗೆ
ಮುಘಲ್
ಮುಟ್ಟನಹಳ್ಳಿ
ಮುಟ್ಟಿದಂ
ಮುಟ್ಣಹಳ್ಳಿ
ಮುಡಿಗೊಂಡ
ಮುಡಿಪಿ
ಮುಡಿಪಿದನೆಂದು
ಮುಡಿಪಿರಬೇಕೆಂದೂಆ
ಮುತಹಡೆಯರಾಯನ
ಮುತ್ತತ್ತಿ
ಮುತ್ತತ್ತಿಯ
ಮುತ್ತರಸ
ಮುತ್ತರಸನು
ಮುತ್ತಲು
ಮುತ್ತಿ
ಮುತ್ತಿಗ
ಮುತ್ತಿಗೆ
ಮುತ್ತಿಗೆಯಲ್ಲಿ
ಮುತ್ತಿದ
ಮುತ್ತಿದನೆಂದು
ಮುತ್ತಿದಲ್ಲಿ
ಮುತ್ತಿದಾಗ
ಮುತ್ತೆಗೆರೆ
ಮುತ್ತೆತ್ತಿ
ಮುತ್ತೇಗೆರೆಯ
ಮುತ್ಸಂದ್ರಬೇಚಿರಾಕ್
ಮುದಗಂದೂರಿನ
ಮುದಗಂದೂರಿನಲ್ಲಿ
ಮುದಗಂದೂರು
ಮುದಗನ್ದೂರು
ಮುದಗಲ್
ಮುದಗಾವುಂಡನ
ಮುದಗುಂದೂರಿನಲ್ಲಿ
ಮುದಗುಂದೂರಿನಲ್ಲಿನಡೆದ
ಮುದಗುಂದೂರು
ಮುದಗೆರೆ
ಮುದಸಮುದ್ರ
ಮುದಿತಮೂರ್ತಿರ್ಲೋಕವಿಖ್ಯಾತ
ಮುದಿಬೆಟ್ಟದ
ಮುದಿಬೆಟ್ಟದಸಾತೇನಹಳ್ಳಿ
ಮುದಿಮಲೆ
ಮುದಿಮಾರನಹಳ್ಳಿ
ಮುದುಕೊಂಗಣಿ
ಮುದುಗನೂರುದುರ್ಗಗಳನ್ನು
ಮುದುಗುಂದೂರು
ಮುದುಗುಪ್ಪೆ
ಮುದುಗುಪ್ಪೆಯ
ಮುದುಗುಪ್ಪೆಯು
ಮುದುಡಿಯ
ಮುದುರಾಚಯ್ಯ
ಮುದುರಾಚಯ್ಯನನ್ನು
ಮುದೇನಹಳ್ಳಿಯನ್ನು
ಮುದ್ದಗೌಡನ
ಮುದ್ದರಸಿ
ಮುದ್ದಿಯಕ್ಕರ
ಮುದ್ದುಲಿಂಗಮ್ಮನವರು
ಮುದ್ದೆಯ
ಮುದ್ದೆಯನಾಯಕನು
ಮುದ್ರಿಕೆಯನೊಲವಿನಿನೀ
ಮುದ್ರಿಕೆಯನ್ನು
ಮುದ್ರೆಯನ್ನು
ಮುನಿ
ಮುನಿಗಳಿಗೆ
ಮುನಿಗಳು
ಮುನಿಚಂದ್ರದೇವರ
ಮುನಿಯ
ಮುನಿಯನ್ನು
ಮುನಿಯಿಂದ
ಮುನಿಯು
ಮುನಿಯೊಬ್ಬನು
ಮುನಿರಾಜಪ್ಪನವರು
ಮುನಿವರನೆಂದು
ಮುನಿಸಿಕೊಂಡು
ಮುನೀಂದ್ರನು
ಮುನ್ನ
ಮುನ್ನಂರೂಢಿಯ
ಮುನ್ನಡೆಸುತ್ತಿದ್ದರೆಂದು
ಮುನ್ನಸಂದ
ಮುನ್ನುಗ್ಗಿ
ಮುನ್ನೂರನ್ನು
ಮುನ್ನೂರು
ಮುನ್ನೂರುಮ್
ಮುಪ್ಪಿನ
ಮುಮ್ಮಡಿ
ಮುಮ್ಮಡಿಚೋಳ
ಮುಮ್ಮಡಿನರಸಿಂಹ
ಮುಮ್ಮಡಿಬಲ್ಲಾಳನು
ಮುಮ್ಮುರಿ
ಮುರಾರಿ
ಮುರಿದ
ಮುರಿದು
ಮುರುಕನಹಳ್ಳಿ
ಮುರುಕನಹಳ್ಳಿಯ
ಮುರುಯನ
ಮುಲ್ಕ್
ಮುಳಬಾಗಲ್
ಮುಳಬಾಗಿಲು
ಮುಳುಗಿ
ಮುಳುಗಿದ್ದು
ಮುಳ್ಗುಂದವನ್ನು
ಮುವರುರಾಯರಗಂಡ
ಮುಷೀರ್
ಮುಸಲ್ಮಾನರ
ಮುಸುಕಮಾದೆಗೊಂಡನ
ಮುಸುಕು
ಮುಸ್ತಫಾಖಾನ್
ಮುಸ್ತೈದೆ
ಮುಸ್ಲಿಂ
ಮುಸ್ಲಿಮರ
ಮೂಕುತಿಗಳುಂ
ಮೂಗನೂ
ಮೂಗನ್ನು
ಮೂಗರ
ಮೂಗರನಾಡಾಳುವ
ಮೂಗರನಾಡು
ಮೂಗರನಾಡುಮೂಗೂರುಮೂರುನಾಡು
ಮೂಗರಿವೋನ್
ಮೂಗೂರನ್ನು
ಮೂಗೂರು
ಮೂಡಣ
ಮೂಡರಾಜ್ಯ
ಮೂಡರಾಜ್ಯಕ್ಕೆ
ಮೂಡರಾಜ್ಯದ
ಮೂಡಲು
ಮೂಡಿಗೆರೆ
ಮೂರನೆ
ಮೂರನೆಯ
ಮೂರನೆಯವನು
ಮೂರನೇ
ಮೂರರ
ಮೂರು
ಮೂರುಜನ
ಮೂರುನಾಲ್ಕು
ಮೂರುರಾಉಯರ
ಮೂರುರಾಯರ
ಮೂರುರಾಯರಗಂಡಾಂಕಃ
ಮೂರುರಾಯರಗಂಡಾಂಕೋ
ಮೂರುಲೋಕಜಗದಳಂ
ಮೂರ್ಛಿತವಾಗುವಂತೆ
ಮೂರ್ತಸ್ಯ
ಮೂರ್ತಿ
ಮೂರ್ತಿಗಳನ್ನು
ಮೂರ್ತಿಯನ್ನು
ಮೂರ್ತಿಯವರು
ಮೂರ್ತಿಶಿಲ್ಪಗಳ
ಮೂರ್ತಿಶಿಲ್ಪವು
ಮೂಲ
ಮೂಲಕ
ಮೂಲಕಥೆ
ಮೂಲಕಥೆಯು
ಮೂಲಕವೇ
ಮೂಲತಃ
ಮೂಲದ
ಮೂಲದವನೆನ್ನುವುದು
ಮೂಲಪುರಷನ
ಮೂಲಪುರಷನೆಂದು
ಮೂಲಪುರುಷ
ಮೂಲಪುರುಷನಾಗಿರುವ
ಮೂಲಪುರುಷನೆಂದೂ
ಮೂಲಪುರುಷರು
ಮೂಲರಾಜರು
ಮೂಲರೂಪವಾಗಿರಲಾರದು
ಮೂಲವನ್ನು
ಮೂಲವಾಗಿವೆ
ಮೂಲವಾದ
ಮೂಲವೆನಿಪಗ್ಗದ
ಮೂಲಸಂಘದ
ಮೂಲಸ್ಥಾನ
ಮೂಲಸ್ಥಾನದೇವರಿಗೆ
ಮೂಲಿಗ
ಮೂಳೇನಹಳ್ಳಿಯನ್ನು
ಮೂವಡಿ
ಮೂವಡಿಚೋಳ
ಮೂವತ್ತರ್ಛಾಸಿರ
ಮೂವತ್ತು
ಮೂವತ್ತುಕೊಳಗ
ಮೂವತ್ತೂರ
ಮೂವರಲ್ಲದೆ
ಮೂವರು
ಮೂವರುರಾಯರ
ಮೂವರೂ
ಮೂಷಕಸ್ತಥಾ
ಮೂಷವ
ಮೂಷಿಕ
ಮೂಷಿಕಮೂಷಕ
ಮೃಗಯಾಂ
ಮೃತನಾಗಿ
ಮೃತನಾಗಿದ್ದನು
ಮೃತನಾಗಿದ್ದನೆಂದು
ಮೃತನಾಗಿರಬಹುದು
ಮೃತನಾದ
ಮೃತನಾದನು
ಮೃತನಾದಾಗ
ಮೃತಪಟ್ಟ
ಮೃತಪಟ್ಟನೆಂದು
ಮೃತಪಟ್ಟಾಗ
ಮೃತಪಟ್ಟಿದ್ದರು
ಮೃತಪಟ್ಟಿದ್ದು
ಮೃತಪಟ್ಟಿರಬಹುದು
ಮೃತರ
ಮೃತರಾಗಿದ್ದರೆಂಬುದು
ಮೃತರಾದರೆಂದು
ಮೃಷ್ಟಾಂನದಾನ
ಮೆಂಟೆಯದ
ಮೆಂಡೆಯದ
ಮೆಕೆಂಝಿ
ಮೆಚ್ಚದವರು
ಮೆಚ್ಚದೋರಾರ್
ಮೆಚ್ಚಿ
ಮೆಚ್ಚಿದ
ಮೆಚ್ಚಿದೆ
ಮೆಚ್ಚಿದೆಂ
ಮೆಚ್ಚಿದ್ದನ್ನು
ಮೆಚ್ಚಿಸಿ
ಮೆಚ್ಚುಗೆ
ಮೆಚ್ಚುಗೆಯಾಗಿ
ಮೆಚ್ಚೆ
ಮೆಟ್ಟಿ
ಮೆಣಸ
ಮೆಯಿಸಿರಿವಟ್ಟವನೆ
ಮೆಯೆದೇವನ
ಮೆಯ್ಯೊಳೆಯ್ದಿ
ಮೆರೆದನು
ಮೆರೆವೊಳ್ಳೆಮ್ಬ
ಮೆಲುಕೋಟೆಯನ್ನು
ಮೆಲ್ಲಗೆ
ಮೆಳಹಳ್ಳಿ
ಮೇ
ಮೇಘಚಂದ್ರ
ಮೇಘಚಂದ್ರಸಿದ್ಧಾಂತ
ಮೇಡು
ಮೇದಿನಿ
ಮೇದಿನೀ
ಮೇನಾಗರ
ಮೇನಾಪುರ
ಮೇರು
ಮೇರುಲಂಘಿಯಶೋಭರಃ
ಮೇರುವಿನ
ಮೇರುವೆನಿಸಿದ
ಮೇರೆಗಳನ್ನು
ಮೇರೆಗಳಾಗಿ
ಮೇರೆಗಳಾಗಿದ್ದವು
ಮೇರೆಗೆ
ಮೇರೆಯನ್ನು
ಮೇರೆಯಾಗಿ
ಮೇರೆಯಾಗಿದ್ದವು
ಮೇರೆಯೂ
ಮೇಲಕ್ಕೆ
ಮೇಲಧಿಕಾರಿ
ಮೇಲಾಟಕ್ಕೆ
ಮೇಲಾಳಿಕೆ
ಮೇಲಾಳಿಕೆಯನ್ನು
ಮೇಲಾಳ್ಕೆ
ಮೇಲಿಂದ
ಮೇಲಿದೆ
ಮೇಲಿದ್ದ
ಮೇಲಿನ
ಮೇಲಿಪಿಳತ್ತೂರು
ಮೇಲಿರು
ಮೇಲಿರುವ
ಮೇಲುಕೋಟೆ
ಮೇಲುಕೋಟೆಗೂ
ಮೇಲುಕೋಟೆಗೆ
ಮೇಲುಕೋಟೆಯ
ಮೇಲುಕೋಟೆಯಲ್ಲಿ
ಮೇಲುಗೈ
ಮೇಲುಸ್ತುವಾರಿಯಲ್ಲಿ
ಮೇಲೆ
ಮೇಲೆರಗಿ
ಮೇಲೆವನ್ದು
ಮೇಲೇರಿ
ಮೇಲೇರಿರುವುದು
ಮೇಲ್
ಮೇಲ್ಕಂಡ
ಮೇಲ್ಕಂಡಂತೆ
ಮೇಲ್ಬಾಗದಲ್ಲಿ
ಮೇಲ್ಮಟ್ಟದ
ಮೇಲ್ಮೆಯನ್ನೇ
ಮೇಲ್ವಿಚಾರಕನ
ಮೇಲ್ವಿಚಾರಣೆ
ಮೇಲ್ವಿಚಾರಣೆಯಲ್ಲಿ
ಮೇಳಯ್ಯನು
ಮೇಳಾದೇವಿಯರ
ಮೇಳಾಪುರ
ಮೇಳಿ
ಮೈತುಂಬಾ
ಮೈದಾನವಿದ್ದು
ಮೈದುನ
ಮೈದುನರಾದ
ಮೈಮರೆತ
ಮೈಮೆಟ್ಟಿ
ಮೈಲನಹಳ್ಳಿ
ಮೈಲಳದೇವಿಯು
ಮೈಲಿ
ಮೈಸುನಾಡುಮೈಸೆನಾಡು
ಮೈಸೂರ
ಮೈಸೂರಿಗೆ
ಮೈಸೂರಿನ
ಮೈಸೂರಿನಲ್ಲಿ
ಮೈಸೂರಿನಲ್ಲಿರುವ
ಮೈಸೂರಿನಿಂದ
ಮೈಸೂರು
ಮೊಗವನ್ನು
ಮೊಗವಾಡವೋ
ಮೊಟ್ಟಮೊದಲಿಗೆ
ಮೊಟ್ಟೆನವಿಲೆಮಟ್ಟನೋಲೆ
ಮೊಟ್ಟೆನವಿಲೆಯನ್ನು
ಮೊಡವನಕೋಡಿಯ
ಮೊಡವಿನಕೋಡಿ
ಮೊಡವಿನಕೋಡಿಯ
ಮೊತ್ತ
ಮೊತ್ತಕ
ಮೊತ್ತದ
ಮೊತ್ತವನ್ನು
ಮೊತ್ತಹಳ್ಳಿಯ
ಮೊದನೆಯ
ಮೊದಮೊದಲ
ಮೊದಮೊದಲು
ಮೊದಲ
ಮೊದಲನೆ
ಮೊದಲನೆಯ
ಮೊದಲನೆಯದಾಗಿ
ಮೊದಲನೆಯವನಾದ
ಮೊದಲನೆಯವನು
ಮೊದಲನೇ
ಮೊದಲಬಾರಿಗೆ
ಮೊದಲಮಗನಿಗೆ
ಮೊದಲಾಗಿ
ಮೊದಲಾದ
ಮೊದಲಾದಂ
ಮೊದಲಾದವರನ್ನು
ಮೊದಲಾದವರಿಗೆ
ಮೊದಲಾದವರು
ಮೊದಲಾದವುಗಳನ್ನು
ಮೊದಲಾಯಿತು
ಮೊದಲಿ
ಮೊದಲಿಗ
ಮೊದಲಿಗನಲ್ಲ
ಮೊದಲಿಗರೆಂದು
ಮೊದಲಿಗೆ
ಮೊದಲಿನ
ಮೊದಲಿನಿಂದಲೂ
ಮೊದಲಿಯಳ್ಳಿಯ
ಮೊದಲಿಹಳ್ಳಿ
ಮೊದಲು
ಮೊದಲೇ
ಮೊದಲ್ಗೊಂಡು
ಮೊನೆಮಟ್ಟರಹಳ್ಳಿ
ಮೊನೆಮುಟ್ಟರಹಳ್ಳಿ
ಮೊನೆಯಾಳು
ಮೊನೆಯಾಳ್ತನಂಗೆಯ್ವರಿಗೆ
ಮೊನೆಯಾಳ್ತನವನ್ನು
ಮೊನೆಯಾಳ್ವ
ಮೊನೆಯೊಳ್
ಮೊಮ್ಮಗ
ಮೊಮ್ಮಗನಾಗಿರಬಹುದು
ಮೊಮ್ಮಗನಾದ
ಮೊಮ್ಮಗನಿರಬಹುದು
ಮೊಮ್ಮಗನೆಂದೂ
ಮೊಮ್ಮಗನೇ
ಮೊಮ್ಮಗಳು
ಮೊರಸಾದಿರಾಯರು
ಮೊರಸಾಧಿರಾಯ
ಮೊರಸಾಧಿರಾಯರೆಂದು
ಮೊರಸು
ಮೊರಸುನಾಡು
ಮೊಲ
ಮೊಲಿಗೆ
ಮೊಲೆವಾಲ
ಮೊಳನಾಡ
ಮೊಳನಾಡಸ್ಥಳದ
ಮೊಹಮದ್
ಮೋಕ್ಷತಿಳಕವೆಂಬ
ಮೋಡಕುಲದವನು
ಮೋಡಕುಳಕಮಳ
ಮೋದಿಖಾನೆ
ಮೋದುನಾಡಿನ
ಮೋದುನಾಡುಕೇ
ಮೋದೂರನ್ನು
ಮೋದೂರಿನಲ್ಲಿ
ಮೋದೂರು
ಮೋದೂರುನಾಡಿನಲ್ಲಿ
ಮೋದೂರುನಾಡು
ಮೌರ್ಯ
ಮ್ಯೂಸಿಯಂನಲ್ಲಿದೆ
ಮ್ಲೇಚ್ಛರುಗಳ
ಯಕ್ಷನಂ
ಯಕ್ಷರಾಜ
ಯಕ್ಷರಾಜನು
ಯಜುಶಾಖೆಯ
ಯಜುಶ್ಶಾಖೆಯ
ಯಜ್ಞಗಳನ್ನು
ಯಜ್ಞಯಾಗಾದಿಗಳನ್ನು
ಯಡಗೋಡಿಯ
ಯಡಹಳ್ಳಿ
ಯತಿಗೆ
ಯತಿಭಿಕ್ಷೆಗಾಗಿ
ಯತಿಯ
ಯತಿರಾಜಮಠದಲ್ಲಿ
ಯತಿರಾಜಮಠವನು
ಯತಿರಾಜಸಪ್ತತಿಯನ್ನು
ಯತಿರಾಜಸಪ್ತಶತಿಯನ್ನು
ಯತೀಶ್ವರನು
ಯತ್ನದಿಂ
ಯತ್ನಿಸಿ
ಯತ್ನಿಸಿದನು
ಯತ್ನಿಸಿದ್ದಾರೆ
ಯದು
ಯದುಗಿರಿ
ಯದುಗಿರಿಯಲ್ಲಿ
ಯದುರಾಜನ
ಯದುವಂಶ
ಯದುವಂಶದ
ಯದುವಂಶದಲ್ಲಿ
ಯದುವಂಶವರ್ಧನಕರಂ
ಯನ್ನು
ಯಮ್ಮದೂರು
ಯರಹಳಿಯ
ಯರಹಳ್ಳ
ಯರಹಳ್ಳಿ
ಯಲಬುರ್ಗಿಯಸಿಂಧ
ಯಲವದ
ಯಲವದಪಲ್ಲಿ
ಯಲವದಹಳ್ಳಿ
ಯಲವದಹಳ್ಳಿಗಳನ್ನು
ಯಲಹಂಕನಾಯಕರು
ಯಲಾದಹಳ್ಳಿ
ಯಲು
ಯಲೆಚಾಕನಹಳ್ಳಿಯ
ಯಲ್ಲಪ್ಪಯ್ಯನೆಂಬುವವನು
ಯಲ್ಲಾದಹಳ್ಳಿ
ಯಲ್ಲಾಪುರ
ಯಳಂದೂರಿನ
ಯಶಸ್ವಿಯಾಗಿ
ಯಶಸ್ಸಾಗಬೇಕೆಂದು
ಯಶಸ್ಸು
ಯಶಸ್ಸುಗಳನ್ನು
ಯಶೋಧನ
ಯಶೋಧರ
ಯಸ್ಮನ್ರಂಜಯತಿ
ಯಸ್ಮಿನ್
ಯಸ್ಯ
ಯಾಗಿ
ಯಾಗಿರಬಹುದು
ಯಾಗಿರುತ್ತಿದ್ದನು
ಯಾಚಕಜನಾಭಿವ್ರಿದ್ಧಿ
ಯಾಚನಘಟ್ಟ
ಯಾಚನಹಳ್ಳಿ
ಯಾಡ
ಯಾತ್ರೆಯನ್ನು
ಯಾದಗಿರಿ
ಯಾದವ
ಯಾದವಕುಲಾಂಬುಧಿ
ಯಾದವಗಿರಿ
ಯಾದವಗಿರಿಗೆ
ಯಾದವನಾರಾಯಣ
ಯಾದವನಾರಾಯಣಪುರವಾದ
ಯಾದವಪುರ
ಯಾದವಪುರದಲ್ಲಿ
ಯಾದವಪುರದಲ್ಲಿದ್ದನೆಂದು
ಯಾದವಪುರವಾದ
ಯಾದವಪುರಿ
ಯಾದವಪುರಿಮೇಲುಕೋಟೆ
ಯಾದವಪುರಿಯ
ಯಾದವರಾಜಧಾನಿಯಾದ
ಯಾದವರಾಜನಾದ
ಯಾದವರಾಜ್ಯಲಕ್ಷ್ಮೀ
ಯಾದವಾಚಲಪತಿಯಾದ
ಯಾದವಾದ್ರಿಪತಿಯು
ಯಾದವಾದ್ರಿಯ
ಯಾರಾದರೊಬ್ಬರ
ಯಾರಿಂದಲೂ
ಯಾರು
ಯಾರೆಂಬುದನ್ನು
ಯಾರೆಂಬುದು
ಯಾರೋ
ಯಾಲಾದಹಳ್ಳಿ
ಯಾವ
ಯಾವಕಾರಣಕ್ಕೋ
ಯಾವಯಾವ
ಯಾವರೀತಿ
ಯಾವುದನ್ನೂ
ಯಾವುದಾದರೂ
ಯಾವುದು
ಯಾವುದೇ
ಯಾವುದೋ
ಯಾವುವು
ಯಾವುವೂ
ಯಿಂದವರದ
ಯಿಂದಾಳ್ದನಾ
ಯಿಂಮಡಿ
ಯಿರಪವಿರೂಪಾಕ್ಷದೇವರಿಗೆ
ಯೀಚಣ
ಯುಗದಲಿ
ಯುತಾನಾಮಾಭಿವಂದಿತಾಂಹತ್ತುಸಾವಿರ
ಯುದ್ಧ
ಯುದ್ಧಕಾಲದಲ್ಲಿ
ಯುದ್ಧಕ್ಕೂ
ಯುದ್ಧಕ್ಕೆ
ಯುದ್ಧಗಳನ್ನು
ಯುದ್ಧಗಳನ್ನೂ
ಯುದ್ಧಗಳಲ್ಲಿ
ಯುದ್ಧಗಳಾದವು
ಯುದ್ಧಗಳು
ಯುದ್ಧದ
ಯುದ್ಧದಲ್ಲಿ
ಯುದ್ಧದಲ್ಲೇ
ಯುದ್ಧಪಟುವಾಗುವ
ಯುದ್ಧಭೂಮಿಗಳಲ್ಲಿ
ಯುದ್ಧಭೂಮಿಯಲ್ಲೇ
ಯುದ್ಧಮಾಡಿ
ಯುದ್ಧಮಾಡಿದನು
ಯುದ್ಧಮಾಡಿದುದಕ್ಕೆ
ಯುದ್ಧಮಾಡಿದುದನ್ನು
ಯುದ್ಧಮಾಡುತ್ತಿದ್ದಾಗ
ಯುದ್ಧರಂಗದಲ್ಲಿ
ಯುದ್ಧವನ್ನು
ಯುದ್ಧವಮಾಡಿ
ಯುದ್ಧವಾಗಿರಬಹುದು
ಯುದ್ಧವಾಗಿರುತ್ತದೆ
ಯುದ್ಧವಾದ
ಯುದ್ಧವಾದಾಗ
ಯುದ್ಧವಿರಬಹುದು
ಯುದ್ಧವು
ಯುದ್ಧಾರಚರಣೆಯು
ಯುಧಿಷ್ಟಿರಾಭಿಷೇಕ
ಯುವಕನಾಗಿದ್ದಾಗಲೇ
ಯುವರಾಜ
ಯುವರಾಜನನ್ನಾಗಿ
ಯುವರಾಜನನ್ನು
ಯುವರಾಜನಾಗಿ
ಯುವರಾಜನಾಗಿದ್ದ
ಯುವರಾಜನಾಗಿದ್ದನು
ಯುವರಾಜನಾಗಿದ್ದನೆಂದು
ಯುವರಾಜನಾಗಿದ್ದಾಗಲೇ
ಯುವರಾಜನಾಗಿದ್ದುಕೊಂಡು
ಯುವರಾಜನಾದ
ಯುವರಾಜನೆಂದು
ಯುವರಾಜರನ್ನು
ಯುವರಾಜರು
ಯುವರಾಜರೂ
ಯೆಂಕಟಪ್ಪ
ಯೆಂಣೆನಾಡನ್ನು
ಯೆಡತೊರೆತಾಲ್ಲೂಕು
ಯೆಡದೊರೆ
ಯೆಡವಾಣೆ
ಯೆಡೂರಿ
ಯೆಪ್ಪತ್ತಕ್ಕೆ
ಯೆಪ್ಪತ್ತು
ಯೆರೆಗಂಗ
ಯೆಲೆಗನೂರ
ಯೆಲ್ಲಪ್ಪಯ್ಯನು
ಯೆಳಂದೂರು
ಯೇಕಾದಶಿವ್ರತನಿರತ
ಯೇಚಿರಾಜ
ಯೋ
ಯೋಗಗೌಡ
ಯೋಗಭಂಗಿಯಲ್ಲಿ
ಯೋಗಾನರಸಿಂಹಸ್ವಾಮಿಗೆ
ಯೋಗಿಪ್ರವರ
ಯೋಗೀಂದ್ರರ
ಯೋಗ್ಯ
ಯೋಗ್ಯತೆಗೆ
ಯೋಗ್ಯತೆಯ
ಯೋಧರಿದ್ದರೆಂದು
ಯೋಧರುಗಳ
ಯೋಧಸೈನಿಕರಾದ
ಯ್ದೆಡೆ
ರ
ರಂಗ
ರಂಗಕ್ಷಿತೀಂದ್ರ
ರಂಗಗವುಡನ
ರಂಗಣ್ಣನೆಂಬ
ರಂಗದ
ರಂಗದಲ್ಲಿ
ರಂಗಧಾಮಸ್ವಾಮಿಗೆ
ರಂಗನ
ರಂಗನಕೊಪ್ಪಲು
ರಂಗನತಿಟ್ಟು
ರಂಗನಾಥ
ರಂಗನಾಥದೇವರ
ರಂಗನಾಥದೇವರಿಗೆ
ರಂಗನಾಥದೇವಾಲಯದಿಂದ
ರಂಗನಾಥನ
ರಂಗನಾಥನಗರದಲ್ಲಿ
ರಂಗನಾಥನಿಗೆ
ರಂಗನಾಥಸ್ವಾಮಿ
ರಂಗನಾಯಕಿ
ರಂಗಪಯ್ಯ
ರಂಗಪಯ್ಯನು
ರಂಗಪಯ್ಯನೂ
ರಂಗಪ್ಪನಾಯಕನು
ರಂಗಭೋಗ
ರಂಗಮಂಟಪ
ರಂಗಮಂಟಪವನ್ನು
ರಂಗಮಾಂಬಿಕೆಯರು
ರಂಗಮಾಂಬೆಯವರು
ರಂಗಮ್ಮನವರು
ರಂಗಯ್ಯ
ರಂಗಯ್ಯನ
ರಂಗಯ್ಯನಾಯಕ
ರಂಗಯ್ಯನಾಯಕನು
ರಂಗರಾಜಯ್ಯನ
ರಂಗಹಳ್ಳಿ
ರಂಗಾಂಬಿಕಾ
ರಂಗಾಚಾರಿ
ರಂಗಾಚಾರಿಯು
ರಂಗಾಪುರ
ರಂಗೆಯನಾಯಕ
ರಂಜಯನ್ನಖಿಲಾಃ
ರಂಜಿತವಾಗಿ
ರಂದು
ರಕ
ರಕ್ಕಸಗಂಗ
ರಕ್ಕಸಗಂಗನು
ರಕ್ಕಸಗಂಗನೆಂಬ
ರಕ್ಕಸಗಿ
ರಕ್ತಕೊಡುಗೆಯನ್ನು
ರಕ್ತಪಾತವಾಗಲಿಲ್ಲ
ರಕ್ತಸಂಬಂಧಿಗಳು
ರಕ್ಷಕ
ರಕ್ಷಣಾ
ರಕ್ಷಣಾಂಗ
ರಕ್ಷಣಾಯ
ರಕ್ಷಣೆ
ರಕ್ಷಣೆಗೆ
ರಕ್ಷಣೆಯ
ರಕ್ಷಣೆಯಲ್ಲಿ
ರಕ್ಷಣೆಯಲ್ಲಿಟ್ಟು
ರಕ್ಷಾಕರಃ
ರಕ್ಷಾಪಾಳಕರಾಗಿದ್ದರು
ರಕ್ಷಿಪಂ
ರಕ್ಷಿಪ್ಪ
ರಕ್ಷಿಸಲು
ರಕ್ಷಿಸಿ
ರಕ್ಷಿಸುವ
ರಚನೆ
ರಚನೆಗಳು
ರಚನೆಗೆ
ರಚನೆಯ
ರಚನೆಯಲ್ಲಿ
ರಚನೆಯಾಗಿದೆ
ರಚನೆಯಾಗಿರಬಹುದು
ರಚನೆಯಾಗಿರಲಿಲ್ಲ
ರಚನೆಯಾದವು
ರಚಿತವಾಗಿದ್ದ
ರಚಿತವಾಗಿದ್ದು
ರಚಿತವಾಗಿರುವ
ರಚಿತವಾದ
ರಚಿಸಲಾಗಿದೆ
ರಚಿಸಲಾಯಿತು
ರಚಿಸಿದ
ರಚಿಸಿದನೆಂದೂ
ರಚಿಸಿದರು
ರಚಿಸಿದ್ದಾನೆ
ರಚಿಸಿದ್ದಾರೆ
ರಚಿಸಿರಬಹುದಾದ
ರಚಿಸಿರುವ
ರಜತಪರಿಯಂಕ
ರಜಾಕ್
ರಟ್ಟ
ರಟ್ಟಪಾಡಿ
ರಟ್ಟಪಾವಾಡಿ
ರಟ್ಟಿಹಳ್ಳಿಗಳು
ರಟ್ಟೆ
ರಠಹಳ್ಳಿಯ
ರಣ
ರಣಕಲಕೇತ
ರಣದುಲ್ಲಾಖಾನ್
ರಣಧೀರಕಂಠೀರವನು
ರಣಪಾರ
ರಣಪಾರರ್
ರಣಮುಖ
ರಣರಂಗಕೇಸರಿ
ರಣರಂಗದಲ್ಲಿ
ರಣರಂಗದಿಂದ
ರಣರಂಗಧೀರನುಂ
ರಣವಿಕ್ರಮಾರ್ಯ
ರಣಾವಲೋಕ
ರಣಿತಗವುಂಡ
ರಣಿತಗವುಂಡನ
ರಣಿತಗವುಂಡನು
ರಣಿತಗವುಂಡನುದ್ಭವಿಸಿ
ರಣಿಭಾಟು
ರಣಿರಾಉಪದಭೋಗ
ರತ್ನ
ರತ್ನತ್ರಯ
ರತ್ನತ್ರಯದಂತೆ
ರತ್ನತ್ರಯಾಕರಂ
ರತ್ನಪಾಲ
ರತ್ನಭಾರಣ
ರತ್ನಸಿಂಹಾಸನದಿಂದ
ರತ್ನಸಿಂಹಾಸನಾರೂಢನಾಗಿ
ರತ್ನಸಿಂಹಾಸನಾರೂಢನಾಗಿದ್ದನೆಂದು
ರತ್ನಸಿಂಹಾಸನಾರೂಢರಾಗಿ
ರತ್ನಸಿಂಹಾಸನೇ
ರತ್ನಾಕರ
ರತ್ನಾಯಿಗೆ
ರಥೋತ್ಸವಕ್ಕೆ
ರದ್ದುಗೊಳಿಸಿ
ರದ್ದುಪಡಿಸಿ
ರದ್ದುಮಾಡಿ
ರನ್ನಕವಿ
ರನ್ನನು
ರಮಣೀಯ
ರಮ್ಯ
ರಮ್ಯಂ
ರಮ್ಯಃ
ರಮ್ಯವಾದ
ರಮ್ಯೇ
ರಲ್ಲಾಯಿತು
ರಲ್ಲಿ
ರಲ್ಲಿಯೂ
ರಲ್ಲೇ
ರವರ
ರವರೆಗಿನ
ರವರೆಗೆ
ರವೆಗೆ
ರಸಾವಾತ್ರೇಯ
ರಸ್ತೆಯಲ್ಲಿರುವ
ರಹಗೌಡ
ರಾಂಪುರ
ರಾಕ್ಷಸಿ
ರಾಗಮುಣಗಾಮುಂಡ
ರಾಗಿದ್ದರೆಂದು
ರಾಘಣ್ಣದೇವನ
ರಾಘವಾಪುರ
ರಾಚನಹಳ್ಳಿಯನ್ನು
ರಾಚಪ್ಪಾಜೀ
ರಾಚಮಲ್ಲ
ರಾಚಮಲ್ಲಂ
ರಾಚಮಲ್ಲನ
ರಾಚಮಲ್ಲನನ್ನು
ರಾಚಮಲ್ಲನೀತಿಮಾರ್ಗ
ರಾಚಮಲ್ಲನು
ರಾಚಯ್ಯನಾಯಕನ
ರಾಚವಲ್ಲ
ರಾಚವೂರು
ರಾಚೆಯನಾಯಕ
ರಾಜ
ರಾಜಒಡೆಯನು
ರಾಜಕಂಠೀರವೇಂದ್ರ
ರಾಜಕಾಂಅðವನ್ನು
ರಾಜಕಾರಣದಲ್ಲೇ
ರಾಜಕಾರ್ಯ
ರಾಜಕಾರ್ಯಕ್ಕಾಗಿ
ರಾಜಕಾರ್ಯದ
ರಾಜಕೀಯ
ರಾಜಕೀಯದಲ್ಲಿ
ರಾಜಕೀಯದಿಂದ
ರಾಜಕುಂಜರ
ರಾಜಕುಮಾರ
ರಾಜಕುಮಾರರನ್ನು
ರಾಜಕುಮಾರರೇ
ರಾಜಕುಮಾರಿ
ರಾಜಕೇಸರಿವರ್ಮ
ರಾಜಗುರು
ರಾಜಗುರುಗಳ
ರಾಜಗುರುಗಳಿದ್ದರೆಂಬುದನ್ನು
ರಾಜಗುರುಗಳು
ರಾಜಚಿಹ್ನೆಗಳನ್ನು
ರಾಜಧನತ್ವಕ್ಕೆ
ರಾಜಧರ್ಮ್ಮೇಣ
ರಾಜಧಾನಿ
ರಾಜಧಾನಿಗಳಾದ
ರಾಜಧಾನಿಗಳು
ರಾಜಧಾನಿಗೆ
ರಾಜಧಾನಿಯ
ರಾಜಧಾನಿಯನ್ನ
ರಾಜಧಾನಿಯನ್ನಾಗಿ
ರಾಜಧಾನಿಯನ್ನು
ರಾಜಧಾನಿಯಲ್ಲಿ
ರಾಜಧಾನಿಯವರೆಗೆ
ರಾಜಧಾನಿಯಾಗಿತ್ತು
ರಾಜಧಾನಿಯಾಗಿದ್ದ
ರಾಜಧಾನಿಯಾಗಿದ್ದರೂ
ರಾಜಧಾನಿಯಾಗಿದ್ದು
ರಾಜಧಾನಿಯಾದ
ರಾಜಧಾನಿಯಾಯಿತು
ರಾಜಧಿರಾಜ
ರಾಜನ
ರಾಜನಂತಹ
ರಾಜನನ್ನು
ರಾಜನಸಂಬಂಧಿಕರು
ರಾಜನಹಿರಿಯ
ರಾಜನಹೆಸರಿಲ್ಲ
ರಾಜನಹೆಸರು
ರಾಜನಾಗಿದ್ದ
ರಾಜನಾಗಿದ್ದನು
ರಾಜನಾಗಿದ್ದು
ರಾಜನಾದ
ರಾಜನಾದನು
ರಾಜನಿಂದ
ರಾಜನಿಗೆ
ರಾಜನಿಗೇ
ರಾಜನೀತಿಯ
ರಾಜನು
ರಾಜನೃಪ
ರಾಜನೆಂದು
ರಾಜನೆನಿಸಿದ
ರಾಜನೇ
ರಾಜನೊಡನೆ
ರಾಜಪರಮೇಶ್ವರ
ರಾಜಪರಮೇಶ್ವರಂಯಾದವಕುಲಾಂಬುದಿ
ರಾಜಪರಮೇಶ್ವರಃ
ರಾಜಪರಮೇಶ್ವರನೆಂದು
ರಾಜಪ್ರತಿನಿಧಿಯ
ರಾಜಬೆವಹಾರಿ
ರಾಜಭಟರೇ
ರಾಜಮನೆತನಗಳ
ರಾಜಮನೆತನಗಳಿಗೂ
ರಾಜಮನೆತನಗಳು
ರಾಜಮನೆತನದ
ರಾಜಮನೆತನದವರನ್ನು
ರಾಜಮನೆತನದವರಿಗೂ
ರಾಜಮನೆತನದವರು
ರಾಜಮನೆತನವನ್ನು
ರಾಜಮಲ್ಲ
ರಾಜಮಲ್ಲನ
ರಾಜಮಲ್ಲನನ್ನು
ರಾಜಮಲ್ಲನಿಗೆ
ರಾಜಮಲ್ಲನು
ರಾಜಮಲ್ಲರ
ರಾಜಮಹಾರಾಜರೆಂದಲ್ಲ
ರಾಜಮಹೇಂದ್ರಿಯನ್ನು
ರಾಜಮಾರ್ತಾಂಡನೆಂಬ
ರಾಜಯ್ಯ
ರಾಜರ
ರಾಜರದು
ರಾಜರನ್ನಾಗಿ
ರಾಜರನ್ನು
ರಾಜರಾಗಿದ್ದರು
ರಾಜರಾಜ
ರಾಜರಾಜಚೋಳನ
ರಾಜರಾಜಚೋಳನಿಗೆ
ರಾಜರಾಜಚೋಳನೆಂದು
ರಾಜರಾಜದೇವನ
ರಾಜರಾಜದೇವನು
ರಾಜರಾಜನನ್ನು
ರಾಜರಾಜಪುರ
ರಾಜರಾಜಪುರದ
ರಾಜರಾಜಪುರದಲ್ಲಿರುವಾಗ
ರಾಜರಾಜಪುರವಾದ
ರಾಜರಾಜಶ್ರೀ
ರಾಜರಾಜಸಮಾಂಹತಿಃ
ರಾಜರಿಂದ
ರಾಜರು
ರಾಜರೆಲ್ಲರಂ
ರಾಜವಂಶಕ್ಕೆ
ರಾಜವಂಶಗಳಿಗೆ
ರಾಜವಂಶದ
ರಾಜವಂಶವಾಗಿತ್ತು
ರಾಜವಡೇರ
ರಾಜವರ್ತಕ
ರಾಜವಿದ್ಯಾಧರ
ರಾಜವೊಡೆಯನ
ರಾಜವೊಡೆಯರಿಗೆ
ರಾಜವೊಳಲ
ರಾಜವೊಳಲಾಗಿರಬಹುದು
ರಾಜಶ್ರೀ
ರಾಜಸಭಾಯೋಗ್ಯನಾಗಿದ್ದನು
ರಾಜಸಿಂಹಾಸನವನ್ನು
ರಾಜಾದಿತ್ಯನ
ರಾಜಾದಿತ್ಯನನ್ನು
ರಾಜಾದಿತ್ಯನು
ರಾಜಾದಿತ್ಯನೂ
ರಾಜಾದಿತ್ಯರು
ರಾಜಾಧಿರಾಜ
ರಾಜಾಧಿರಾಜಃ
ರಾಜಾಧಿರಾಜಬಿರುದೋ
ರಾಜಾಧಿರಾಜಯಿತ್ಯುಕ್ತೋ
ರಾಜಾಧಿರಾಜೇಂದ್ರನೆಂದು
ರಾಜಾಧ್ಯಕ್ಷ
ರಾಜಾನ್ವಯದೊಕ್ಕಲ
ರಾಜಾರಮಡು
ರಾಜಾರಾಮ
ರಾಜಾವಳಿ
ರಾಜಾಸ್ಥಾನದಲ್ಲಿ
ರಾಜೇಂದ್ರ
ರಾಜೇಂದ್ರಚೋಳ
ರಾಜೇಂದ್ರಚೋಳನ
ರಾಜೇಂದ್ರಚೋಳನೇ
ರಾಜೇಗೌಡ
ರಾಜೇನ್ದ್ರಚೋೞ
ರಾಜೇಶ್ವರಿ
ರಾಜೊಡೆಯರ
ರಾಜೋದ್ಯಾನವನವನ್ನು
ರಾಜ್ಯ
ರಾಜ್ಯಂ
ರಾಜ್ಯಂಗೆಯುತಂಮಿರಲು
ರಾಜ್ಯಂಗೆಯುತ್ತಮಿರೆ
ರಾಜ್ಯಂಗೆಯುತ್ತಿರಲಿಕ್ಕಾಗಿ
ರಾಜ್ಯಂಗೆಯೆ
ರಾಜ್ಯಂಗೆಯ್ಯುತ್ತಮಿರೆ
ರಾಜ್ಯಂಗೆಯ್ಯುತ್ತಿದ್ದನು
ರಾಜ್ಯಂಗೆಯ್ಯುತ್ತಿದ್ದರೆಂದು
ರಾಜ್ಯಂಗೆಯ್ಯೆ
ರಾಜ್ಯಂಗೈಉತ್ತಮಿರಲು
ರಾಜ್ಯಕಾರ್ಯಕ್ಕಾಗಿ
ರಾಜ್ಯಕಾರ್ಯವನ್ನು
ರಾಜ್ಯಕ್ಕೀತಂ
ರಾಜ್ಯಕ್ಕೆ
ರಾಜ್ಯಗತಂ
ರಾಜ್ಯಗಳ
ರಾಜ್ಯಗಳನ್ನಾಗಿ
ರಾಜ್ಯಗಳನ್ನು
ರಾಜ್ಯಗಳಲ್ಲಿ
ರಾಜ್ಯಗಳಾಗಿ
ರಾಜ್ಯಗಳಿಗೂ
ರಾಜ್ಯಚಿಹ್ನೆಯೂ
ರಾಜ್ಯದ
ರಾಜ್ಯದಲ್ಲಿ
ರಾಜ್ಯದಲ್ಲಿತ್ತೆಂದು
ರಾಜ್ಯದಿಂದ
ರಾಜ್ಯದೇಶಸೀಮೆ
ರಾಜ್ಯದೊಳ್
ರಾಜ್ಯಪಾಲನಾಗಿದ್ದನು
ರಾಜ್ಯಪಾಲರುಗಳು
ರಾಜ್ಯಭರ
ರಾಜ್ಯಭಾರ
ರಾಜ್ಯಭಾರಕ್ಕೆ
ರಾಜ್ಯಭಾರದ
ರಾಜ್ಯಭ್ರಷ್ಟನನ್ನಾಗಿ
ರಾಜ್ಯಭ್ರಷ್ಟನಾದ
ರಾಜ್ಯಮಧ್ಯೇ
ರಾಜ್ಯಲಕ್ಷ್ಮಿ
ರಾಜ್ಯಲಕ್ಷ್ಮಿಯ
ರಾಜ್ಯಲಕ್ಷ್ಮಿಯನ್ನು
ರಾಜ್ಯವನಾಳುತ್ತಮಿದ್ದ
ರಾಜ್ಯವನ್ನಾಗಿ
ರಾಜ್ಯವನ್ನು
ರಾಜ್ಯವಾಳತೊಡಗಿದನೆಂದು
ರಾಜ್ಯವಾಳದೇ
ರಾಜ್ಯವಾಳಲಿಲ್ಲ
ರಾಜ್ಯವಾಳಲಿಲ್ಲವೆಂದು
ರಾಜ್ಯವಾಳಲು
ರಾಜ್ಯವಾಳಿದ
ರಾಜ್ಯವಾಳಿದನು
ರಾಜ್ಯವಾಳಿದನೆಂದು
ರಾಜ್ಯವಾಳಿದನೆಂದೂ
ರಾಜ್ಯವಾಳಿದವನು
ರಾಜ್ಯವಾಳುತ್ತಿದ್ದ
ರಾಜ್ಯವಾಳುತ್ತಿದ್ದನು
ರಾಜ್ಯವಾಳುತ್ತಿದ್ದನೆಂದು
ರಾಜ್ಯವಾಳುತ್ತಿದ್ದರೆಂದು
ರಾಜ್ಯವಾಳುತ್ತಿದ್ದಾಗ
ರಾಜ್ಯವಿತ್ತು
ರಾಜ್ಯವಿದ್ದಿತು
ರಾಜ್ಯವಿಸ್ತರಣೆಯನ್ನು
ರಾಜ್ಯವು
ರಾಜ್ಯವೆಂದು
ರಾಜ್ಯವೆಂಬ
ರಾಜ್ಯಶ್ರೀ
ರಾಜ್ಯಸಂವತ್ಸರದಲ್ಲಿ
ರಾಜ್ಯಸ್ಥಳಕ್ಕೆ
ರಾಜ್ಯಸ್ಥಾಪನೆ
ರಾಜ್ಯಾಡಳಿತ
ರಾಜ್ಯಾಡಳಿತವನ್ನು
ರಾಜ್ಯಾಡಳಿತವು
ರಾಜ್ಯಾಧಿಪ
ರಾಜ್ಯಾಧಿಪತಿ
ರಾಜ್ಯಾಧಿಪತಿಗಳಾಗಿದ್ದ
ರಾಜ್ಯಾಧಿಪತಿಗಳು
ರಾಜ್ಯಾಧಿಪತಿಯಾಗಿದ್ದ
ರಾಜ್ಯಾಧಿಪತಿಯಾಗಿದ್ದನೆಂದು
ರಾಜ್ಯಾಪಹಾರಕ್ಕೆ
ರಾಜ್ಯಾಭಿಷಿಕ್ತನಾದನೆಂದು
ರಾಜ್ಯಾಭಿಷೇಕ
ರಾಜ್ಯಾಭ್ಯುದ
ರಾಜ್ಯಾರೋಹಣಕ್ಕೆ
ರಾಜ್ಯಾವಾಳುತ್ತಿದ್ದನು
ರಾಣಾಜಗದೇವರಾಯ
ರಾಣಾಜಗದೇವರಾಯನ
ರಾಣಾಜಗದೇವರಾಯನು
ರಾಣಾಪೆದ್ದ
ರಾಣಾವಂಶದವರು
ರಾಣಿ
ರಾಣಿಮುಖಜ್ಯೋತಿ
ರಾಣಿಯ
ರಾಣಿಯರ
ರಾಣಿಯರು
ರಾಣಿಯು
ರಾಣಿವಾಸ
ರಾಣೋಜಿರಾವ್
ರಾತ್ರಿಯೊಳ್
ರಾಧನಪುರ
ರಾಧಾ
ರಾಧಾಪಟೇಲ್
ರಾಧೇಯ
ರಾಧೇಯಕುಲ
ರಾಧೇಯನ
ರಾಮ
ರಾಮಅರಸಿಕೆರೆಯ
ರಾಮಕೃಷ್ಣ
ರಾಮಗಉಡ
ರಾಮಚಂದ್ರದೇವರು
ರಾಮಣ್ಣ
ರಾಮಣ್ಣನು
ರಾಮದೇವ
ರಾಮದೇವನ
ರಾಮದೇವನು
ರಾಮದೇವಮಹಾರಾಯ
ರಾಮದೇವರ
ರಾಮದೇವರಾಯ
ರಾಮದೇವರಾಯನು
ರಾಮದೇವರಿಗೆ
ರಾಮನ
ರಾಮನಗರ
ರಾಮನಬೆಂಕೊಂಡಗಂಡ
ರಾಮನರಾಮರಾಜಯ್ಯ
ರಾಮನಹಳ್ಳಿ
ರಾಮನಾಥ
ರಾಮನಾಥದೇವರಿಗೆ
ರಾಮನಾಥನ
ರಾಮನಾಥನಿಗೆ
ರಾಮನಾಥನೊಡನೆ
ರಾಮನಾಥಪುರದ
ರಾಮನು
ರಾಮನೇ
ರಾಮನ್ರಿಪ
ರಾಮಪುರ
ರಾಮಪ್ಪ
ರಾಮಭಟಯ್ಯ
ರಾಮಭಟ್ಟ
ರಾಮಭದ್ರಾದೇವಿ
ರಾಮಮಾತ್ಯ
ರಾಮಯರಾಯ
ರಾಮಯ್ಯ
ರಾಮರಾಜ
ರಾಮರಾಜಅಯ್ಯ
ರಾಮರಾಜಅಳಿಯ
ರಾಮರಾಜತಿರುಮಲರಾಜಯ್ಯನವರ
ರಾಮರಾಜನಸೇನಾನಿ
ರಾಮರಾಜನಾಯಕನಿಗೆ
ರಾಮರಾಜನಿಗೆ
ರಾಮರಾಜಯರ್ಸರು
ರಾಮರಾಜಯ್ಯ
ರಾಮರಾಜಯ್ಯದೇವ
ರಾಮರಾಜಯ್ಯನ
ರಾಮರಾಜಯ್ಯನನ್ನು
ರಾಮರಾಜಯ್ಯನವರ
ರಾಮರಾಜಯ್ಯನವರು
ರಾಮರಾಜಯ್ಯನಿಗೆ
ರಾಮರಾಜಯ್ಯನು
ರಾಮರಾಜಯ್ಯನೂ
ರಾಮರಾಜಯ್ಯನೆಂದು
ರಾಮರಾಜಯ್ಯನೇ
ರಾಮರಾಜವೊಡೆಯರು
ರಾಮರಾಜು
ರಾಮರಾಯ
ರಾಮರಾಯನ
ರಾಮರಾಯನಾಯಕನಿಗೆ
ರಾಮರಾಯನು
ರಾಮರಾಯರೇ
ರಾಮಲಕ್ಷ್ಮಣರಂತಿದ್ದು
ರಾಮಲಕ್ಷ್ಮಣರಂತೆ
ರಾಮಲಕ್ಷ್ಮಣರಿದ್ದಂತೆ
ರಾಮಲಿಂಗದೇವರಿಗೆ
ರಾಮಲಿಂಗೇಶ್ವರ
ರಾಮಸಮುದ್ರಗಳನ್ನು
ರಾಮಾ
ರಾಮಾಜಯ್ಯ
ರಾಮಾನುಜ
ರಾಮಾನುಜಕೂಟಕ್ಕೆ
ರಾಮಾನುಜಕೂಟವನ್ನು
ರಾಮಾನುಜಜೀಯರ್
ರಾಮಾನುಜನ
ರಾಮಾನುಜಪುರಂ
ರಾಮಾನುಜಯ್ಯನಿಗೆ
ರಾಮಾನುಜರ
ರಾಮಾನುಜಾಚಾರ್ಯರ
ರಾಮಾನುಜಾಚಾರ್ಯರನ್ನು
ರಾಮಾನುಜಾಚಾರ್ಯರಿಗೆ
ರಾಮಾನುಜಾಚಾರ್ಯರು
ರಾಮಾಭಟನ
ರಾಮಾಭಟನಿಗೆ
ರಾಮಾಭಟಯ್ಯನವರಿಗೆ
ರಾಮಾಭಟಯ್ಯನಿಗೆ
ರಾಮಾಭಟಯ್ಯನು
ರಾಮಾಭಟ್ಟ
ರಾಮಾಭಟ್ಟನು
ರಾಮಾಭಟ್ಟಯ್ಯನ
ರಾಮಾಭಟ್ಟಯ್ಯನವರಿಂದ
ರಾಮಾಭಟ್ಟಯ್ಯನಿಗೆ
ರಾಮಾಭಟ್ಟಯ್ಯನು
ರಾಮಾಭಟ್ಟರ
ರಾಮಾಯಣಪೂರ್ವ್ವಕ
ರಾಮೆಯನಾಯಕನು
ರಾಮೆಯನಾಯ್ಕ
ರಾಮೇಶ್ವರ
ರಾಮೇಶ್ವರದ
ರಾಮೇಶ್ವರದಲ್ಲಿ
ರಾಮೇಶ್ವರದವರೆಗೂ
ರಾಯ
ರಾಯಂಣ
ರಾಯಕುಮಾರರ
ರಾಯಣನಾಯಕನು
ರಾಯಣ್ಣ
ರಾಯಣ್ಣದಂಡನಾಥನ
ರಾಯಣ್ಣನಾಯಕನ
ರಾಯನ
ರಾಯನಿಂದ
ರಾಯನಿಗೆ
ರಾಯನು
ರಾಯಪ್ಪ
ರಾಯಭಾಟ
ರಾಯರ
ರಾಯರಕುಮಾರರ
ರಾಯರಗಂಡ
ರಾಯರಾಯ
ರಾಯರಿಗೆ
ರಾಯರು
ರಾಯರೊಳು
ರಾಯವೊಡೆಯ
ರಾಯವೊಡೆಯರು
ರಾಯಸ
ರಾಯಸದ
ರಾಯಸದವನು
ರಾಯಸದವರ
ರಾಯಸದವರಾದ
ರಾಯಸದವರಿಗೆ
ರಾಯಸದವರು
ರಾಯಸಮುದ್ರ
ರಾಯಸವು
ರಾಯಸಸ್ವಾಮಿ
ರಾಯಸೆಟ್ಟಿಪುರ
ರಾಯಾಂಬಾಮುಲ
ರಾಯೊಡೆಯರ
ರಾವಂದೂರಿಗೆ
ರಾವಂದೂರಿನ
ರಾವಂದೂರು
ರಾವಂದೂರುಗಳು
ರಾವಿಯಹಾಳೆಯಮಲ್ಲಿನಾಥಪುರ
ರಾವುತ
ರಾವುತರ
ರಾವುತರಾಯನುದಗ್ರದೊರ್ವ್ವಳಂ
ರಾವುತರು
ರಾವುತ್ತರಾಯ
ರಾವುಳ
ರಾವ್ಬಹದ್ದೂರ್
ರಾಷ್ಟ್ರ
ರಾಷ್ಟ್ರಕ
ರಾಷ್ಟ್ರಕೂಟ
ರಾಷ್ಟ್ರಕೂಟರ
ರಾಷ್ಟ್ರಕೂಟರನ್ನು
ರಾಷ್ಟ್ರಕೂಟರಿಂದ
ರಾಷ್ಟ್ರಕೂಟರಿಗೂ
ರಾಷ್ಟ್ರಕೂಟರು
ರಾಷ್ಟ್ರಕೂಟರೇ
ರಾಷ್ಟ್ರಕೂಟರೊಡನೆ
ರಾಷ್ಟ್ರದ
ರಾಷ್ಟ್ರವನ್ನು
ರಾಷ್ಟ್ರವೆಂದು
ರಾಸಕ್ಕಲಿನ
ರಿಂದ
ರಿಂದಲೂ
ರಿಂದಲೇ
ರಿಣಕ್ಕೆ
ರಿಪುರಾಮಗಾಮುಣ್ಡರು
ರಿಪುಸ್ತೋಮ
ರಿಪುಸ್ತೋಮಕರಿ
ರಿಪೋರ್ಟ್ನಲ್ಲಿ
ರಿಪೋರ್ಟ್ರಲ್ಲಿ
ರಿಯಾಯಿತಿ
ರೀತಿ
ರೀತಿಯ
ರೀತಿಯಲ್ಲಿ
ರೀತಿಯಾಗಿ
ರುಕ್ಶಾಖಾಧ್ಯಾಯಿಗಳಾದ
ರುಖಯ್ಯಾಬೀಬಿಯ
ರುದ್ರಣ್ಣ
ರುದ್ರದಂಡಾಧೀಶನಿಗೆ
ರುದ್ರದೇವಾತ್ಮಜಂ
ರುದ್ರಭಟ್ಟ
ರುದ್ರಮುನಿ
ರುದ್ರಸಮುದ್ರ
ರುಧಿರೋದ್ಗಾರಿ
ರೂಡಿ
ರೂಡಿವಡೆದ
ರೂಢಿ
ರೂಢಿಯ
ರೂಢಿಯಲ್ಲಿದೆ
ರೂಪ
ರೂಪಗಳು
ರೂಪಗಳೇ
ರೂಪದ
ರೂಪದಲ್ಲಿ
ರೂಪದಲ್ಲಿತ್ತೆಂದು
ರೂಪವುಳ್ಳವನೂ
ರೂಪವೇ
ರೂಪಾಂತರವಾಯಿತು
ರೂಪಾಂತರವಾಯಿತೆಂದು
ರೂಪಿಸಲಾಯಿತೆಂದು
ರೂಪಿಸಿದನು
ರೂಪುಗೊಂಡ
ರೂಪುಗೊಳ್ಳುವುದಕ್ಕೆ
ರೆಂಡಿಷನ್
ರೆಜಿಮೆಂಟ್
ರೆಡ್ಡಿಯವರು
ರೆಸಿಡೆಂಟ್
ರೇಕವ್ವೆ
ರೇಕವ್ವೆಯ
ರೇಕಾದೇವಿ
ರೇಚಣ್ಣ
ರೇಚೆಯ
ರೇಮಟಿವೆಂಕಟನನ್ನು
ರೇಮೇ
ರೇವಂತ
ರೇವಕ
ರೇವಕನಿಮ್ಮಡಿಯನ್ನು
ರೇವಕ್ಕ
ರೇವಕ್ಕನಿರ್ಮಡಿಯನ್ನು
ರೇವಣಯ್ಯ
ರೇವಣಾರಾಧ್ಯ
ರೇವಲಾಕಲ್ಪವಲ್ಲಿ
ರೇವಲಾದೇವಿಯ
ರೇವಲಾದೇವಿಯನ್ನು
ರೇವಲೇಶ್ವರ
ರೈಟ್ಹ್ಯಾಂಡ್
ರೈತರನ್ನು
ರೈತರು
ರೈಲ್ವೆ
ರೈಸ್
ರೈಸ್ರವರ
ರೈಸ್ರವರು
ರೋಣಶಾಸನವು
ಲಂಕಪ್ಪ
ಲಂಕೆಯವರೆಗೆ
ಲಂಭಹಸ್ತಗಳಅ
ಲಕುಮಯ್ಯ
ಲಕುಮಯ್ಯಗಳ
ಲಕುಮಯ್ಯನು
ಲಕುಮಯ್ಯರು
ಲಕ್ಕಣ್ಣ
ಲಕ್ಕಣ್ಣದಂಡನಾಯಕನು
ಲಕ್ಕಣ್ಣದಂಡನಾಯಕರ
ಲಕ್ಕಣ್ಣದಂಡೇಶ
ಲಕ್ಕಣ್ಣದಂಡೇಶನ
ಲಕ್ಕಣ್ಣದಂಡೇಶನನ್ನು
ಲಕ್ಕಣ್ಣನನ್ನು
ಲಕ್ಕಣ್ಣನಾಯಕರ
ಲಕ್ಕಮ್ಮ
ಲಕ್ಕವ್ವೆ
ಲಕ್ಕಿದೊಣೆಜಿನದೊಣೆಯ
ಲಕ್ಕಿಯೂರು
ಲಕ್ಕುಂಡಿಯತನಕ
ಲಕ್ಕೂರು
ಲಕ್ಷ
ಲಕ್ಷಣ
ಲಕ್ಷಮ್ಮಮ್ಮಣಿಯು
ಲಕ್ಷೋಪಲಕ್ಷ
ಲಕ್ಷ್ಮ
ಲಕ್ಷ್ಮಣದಾಸ
ಲಕ್ಷ್ಮಣಾಧ್ವರಿಯ
ಲಕ್ಷ್ಮನ
ಲಕ್ಷ್ಮಾದೇವಿಯರ
ಲಕ್ಷ್ಮಿ
ಲಕ್ಷ್ಮೀಕಾಂತ
ಲಕ್ಷ್ಮೀಕಾಂತದೇವರ
ಲಕ್ಷ್ಮೀಕಾಂತದೇವಾಲಯದ
ಲಕ್ಷ್ಮೀಜನಾರ್ದನ
ಲಕ್ಷ್ಮೀದೇವರ
ಲಕ್ಷ್ಮೀದೇವಿ
ಲಕ್ಷ್ಮೀದೇವಿಗೆ
ಲಕ್ಷ್ಮೀದೇವಿಯೇ
ಲಕ್ಷ್ಮೀನರಸಿಂಹ
ಲಕ್ಷ್ಮೀನಾರಾಯಣ
ಲಕ್ಷ್ಮೀನಾರಾಯಣದಂಡನಾಯಕ
ಲಕ್ಷ್ಮೀನಾರಾಯಣದೇವರಿಗೆ
ಲಕ್ಷ್ಮೀನಾರಾಯಣರಾವ್
ಲಕ್ಷ್ಮೀಪತಿಯ
ಲಕ್ಷ್ಮೀಭೂವರಾಹನಾಥ
ಲಕ್ಷ್ಮೀಮತಿದಂಡನಾಯಕಿತ್ತಿ
ಲಕ್ಷ್ಮೀಮತಿಯಿಂದ
ಲಕ್ಷ್ಮೀಮುದೇ
ಲಕ್ಷ್ಮೀಸಾಗರ
ಲಕ್ಷ್ಮೀಸಾಗರವನ್ನು
ಲಕ್ಷ್ಮೀಸೇನ
ಲಖಂಣ
ಲಖಂಣನ
ಲಖಂಣವೊಡೆಯರ
ಲಖಣ್ಣವೊಡೆಯ
ಲಖಪನಾಯಕರ
ಲಖ್ಖೆಯ
ಲಗ್ಗೆ
ಲಚ್ಚಣ್ಣ
ಲಚ್ಚಿಯನಾಯಕ
ಲಭ್ಯವಾಗಿ
ಲಭ್ಯವಾಗಿದ್ದು
ಲಭ್ಯವಾಗಿವೆ
ಲಭ್ಯವಾಗುತ್ತಿತ್ತು
ಲಭ್ಯವಾದ
ಲಯಕಾಳ
ಲಲಾಮ
ಲಲಾಮಂ
ಲಲ್ಲ
ಲವಕುಶರಂತೆ
ಲಸದ್ದೋರ್ದಣ್ಡದೊಳ್ಸಂತೋಷಂ
ಲಾಕ್ಷಾಗೃಹೋಪಾಯಮುಂ
ಲಾಭ
ಲಾಭಪಡೆಯಲು
ಲಾಳನಕೆರೆ
ಲಾಳನಕೆರೆಯ
ಲಾವಣಿಗಳನ್ನು
ಲಾವಣಿಗಳು
ಲಿಂಗಕ್ಕೆ
ಲಿಂಗಗವುಡನು
ಲಿಂಗಣ್ಣ
ಲಿಂಗಣ್ಣೊಡೆಯನು
ಲಿಂಗದೇವರು
ಲಿಂಗಪಯ್ಯ
ಲಿಂಗಪಯ್ಯನು
ಲಿಂಗಪ್ಪಗವುಡ
ಲಿಂಗಪ್ಪನಾಯಕನ
ಲಿಂಗಾಚಾರಿಯ
ಲಿಂಗಾಜಮ್ಮಣ್ಣಿಯವರು
ಲಿಂಗಾಪುರ
ಲಿಂಗಾಭಟ್ಟರ
ಲಿಖಿತಮೂಲಗಳಿಂದ
ಲಿಪಿ
ಲಿಪಿಯ
ಲಿಪಿಯಲ್ಲಿದೆ
ಲಿಪಿಯಲ್ಲಿದ್ದು
ಲಿಪಿಯಲ್ಲಿರುವ
ಲಿಪಿಯಲ್ಲಿವೆ
ಲಿಪಿಸಂಸ್ಕೃತತಮಿಳು
ಲು
ಲೆಂಕ
ಲೆಂಕಂಕ
ಲೆಂಕನಿಸ್ಸಂಕನಾದ
ಲೆಂಕಮಹಾದೇವ
ಲೆಂಕರ
ಲೆಂಕರಾಗಿ
ಲೆಂಕರಾಗಿದ್ದರು
ಲೆಂಕರಾಗಿದ್ದರೆಂಬ
ಲೆಂಕರಿದ್ದು
ಲೆಂಕರು
ಲೆಂಕವಾಳಿಯನ್ನು
ಲೆಂಕಿತಿಯರು
ಲೆಕ್ಕ
ಲೆಕ್ಕದ
ಲೆಕ್ಕದಲ್ಲಿ
ಲೆಕ್ಕಪತ್ರ
ಲೆಕ್ಕಪತ್ರಗಳ
ಲೆಕ್ಕಪತ್ರಗಳನ್ನು
ಲೆಕ್ಕಪತ್ರದ
ಲೆಕ್ಕಪತ್ರದಲ್ಲಿ
ಲೆಕ್ಕಬರೆಯುವ
ಲೆಕ್ಕವನ್ನು
ಲೆಕ್ಕಹಾಕಿ
ಲೆಕ್ಕಹಾಕಿದರೆ
ಲೆಕ್ಕಹಾಕಿದ್ದಾರೆ
ಲೆಕ್ಕಾಚಾರ
ಲೆಕ್ಕಾಚಾರವನ್ನು
ಲೆಕ್ಕಾಚಾರಹಾಕಿ
ಲೆಕ್ಕಾಧಿಕಾರಿಗಳೆಂದು
ಲೆಕ್ಕಿಸದೆ
ಲೇಖಕನು
ಲೇಖಗಳು
ಲೇಖನಗಳಾಗಿವೆ
ಲೇಖನಗಳು
ಲೇಖನಗಳೆಲ್ಲವನ್ನೂ
ಲೇಖನದಲ್ಲಿ
ಲೊಕ್ಕಾನೆ
ಲೊಕ್ಕಿಗುಂಡಿಯ
ಲೊಕ್ಕಿಗುಂಡಿಯನ್ನು
ಲೊಕ್ಕಿಗುಂಡಿಯಲ್ಲಿ
ಲೊಕ್ಕಿಯಹಳ್ಳಿ
ಲೋಕ
ಲೋಕಕ್ಕೆ
ಲೋಕತಿಲಕಜಿನಭವನಕ್ಕೆ
ಲೋಕತಿಲಕಭವನವೆಂಬ
ಲೋಕನಹಳ್ಳಿ
ಲೋಕಪಾವನಿ
ಲೋಕಪಾವನೆಗೆ
ಲೋಕಪ್ರಸಿದ್ಧನಾಗಿದ್ದ
ಲೋಕವಿದ್ಯಾಧರ
ಲೋಕವಿದ್ಯಾಧರನು
ಲೋಕಾಂತಸಿಮ್ನಿ
ಲೋಕಾಂಬಿಕೆ
ಲೋಕಾನಾಂಚ
ಲೋಕೋಪಕಾರದಲ್ಲಿ
ಲೋಚೆರ್ಲ
ಲೋಭಿರಾಯ
ಲೋಹದ
ಲೋಹಿತ
ಲೋಹಿತಕುಲಶೇಖರ
ಲ್ದಾತಂ
ಲ್ದೊರೆವಿತ್ತೀ
ಳಂಬಿತ
ಳೆಂದುಂ
ಳ್ತಿರಿದುದ
ವಂಕಣಪಲ್ಲಿ
ವಂಗರು
ವಂದಿಸಲ್ಪಡುತ್ತಿದ್ದನು
ವಂಶ
ವಂಶಕ್ಕೆ
ವಂಶಗಳು
ವಂಶಜನಾಗಿರುವ
ವಂಶಜನಿರಬಹುದೆಂದು
ವಂಶಜರನ್ನು
ವಂಶಜರಾದ
ವಂಶಜರು
ವಂಶದ
ವಂಶದಲ್ಲಿ
ವಂಶದವನಾಗಿದ್ದಾನೆ
ವಂಶದವನಾಗಿರಬಹುದು
ವಂಶದವನಿರಬಹುದು
ವಂಶದವನಿರಬಹುದೆಂದು
ವಂಶದವನು
ವಂಶದವನೆಂದು
ವಂಶದವನೇ
ವಂಶದವರ
ವಂಶದವರಾಗಿರಬಹುದು
ವಂಶದವರಾದ
ವಂಶದವರಿಂದಲೇ
ವಂಶದವರಿಗೂ
ವಂಶದವರಿಗೆ
ವಂಶದವರಿರಬಹುದು
ವಂಶದವರು
ವಂಶದವರೆಂದು
ವಂಶದವರೆಂಬ
ವಂಶದವರೇ
ವಂಶದೊಡನೆ
ವಂಶಪಾರಂಪರ್ಯ
ವಂಶಪಾರಂಪರ್ಯದಿಂದ
ವಂಶಪಾರಂಪರ್ಯವಾಗಿ
ವಂಶಪಾರಂಪರ್ಯವಾಗಿತ್ತು
ವಂಶಪಾರಂಪರ್ಯವಾಗಿತ್ತೆಂದು
ವಂಶಪಾರಂಪರ್ಯವಾಗಿದ್ದಂತೆ
ವಂಶಪಾರಂಪರ್ಯವಾಗಿದ್ದರೂ
ವಂಶಪಾರಂಪರ್ಯವಾಗಿಯೂ
ವಂಶಪಾರಂಪರ್ಯವಾದ
ವಂಶಮೌಕ್ತಿಕ
ವಂಶವನ್ನು
ವಂಶವು
ವಂಶವೃಕ್ಷ
ವಂಶವೃಕ್ಷಗಳ
ವಂಶವೃಕ್ಷದಲ್ಲಿ
ವಂಶವೃಕ್ಷವನ್ನು
ವಂಶವೆಂದರೆ
ವಂಶವೊಂದು
ವಂಶಸ್ಥನಾದ
ವಂಶಸ್ಥನಿರಬಹುದು
ವಂಶಸ್ಥರನ್ನು
ವಂಶಸ್ಥರಾದ
ವಂಶಸ್ಥರಿರಬಹುದು
ವಂಶಸ್ಥರು
ವಂಶಸ್ಥರೋ
ವಂಶಾವಳಿ
ವಂಶಾವಳಿಗಳನ್ನು
ವಂಶಾವಳಿಯ
ವಂಶಾವಳಿಯನ್ನು
ವಂಶಾವಳಿಯನ್ನೂ
ವಂಶಾವಳಿಯಲ್ಲಿ
ವಂಶಾವಳಿಯಲ್ಲಿಯೂ
ವಂಶಾವಳಿಯಿಂದ
ವಂಶಾವಳಿಯು
ವಂಶೋದ್ಭವ
ವಂಶೋದ್ಭವರು
ವಕ್ತೃ
ವಕ್ತೃಪ್ರಯೋಕ್ತೃ
ವಕ್ತ್ರಾಬ್ಜ
ವಕ್ಷಸ್ಥಲ
ವಚಃ
ವಚನದಂತೆ
ವಚನವನ್ನೂ
ವಚನಶತಸಹಸ್ರ
ವಜ್ಜಲದೇವ
ವಜ್ರದಪುಡಿಯನ್ನು
ವಜ್ರಪಂಜರ
ವಜ್ರಪಂಜರರುಂ
ವಜ್ರಬೈಸಣಿಗೆಯನಿಕ್ಕಿ
ವಟವಾಪಿ
ವಡಗೆರೆ
ವಡಗೆರೆನಾಡು
ವಡುಗಪಿಳ್ಳೆಯು
ವಡುಗಪಿಳ್ಳೈ
ವಡುಗವೇಳೆಕಾರ
ವಡೇರ
ವಡೈಯರೈಯನವರು
ವಡ್ಡರ
ವಡ್ಡವ್ಯವಹಾರಿ
ವಡ್ರಬಿಳಿಕೆರೆ
ವದಾನ್ಯತಃ
ವದ್ದೆಗ
ವದ್ದೆಗನೆಂದು
ವನಜಾತಾಯತ
ವನಮಾಲೆಗೆ
ವನವಾಸಿ
ವನವೇಲಿ
ವನಾಂತರದಲ್ಲಿ
ವನಿತಾದೂರಂ
ವನಿವಾರ್ದ್ಧಿಸುಧಾಕರ
ವನ್ನು
ವನ್ಯಧಾಮ
ವಯಸ್ಸಿನ
ವಯಸ್ಸಿನಲ್ಲಿ
ವಯೋವೃದ್ಧನಾಗಿರಬಹುದು
ವರಕೀರ್ತ್ತಿಯಂ
ವರದ
ವರದಣ್ಣನಾಯಕನು
ವರದಯ್ಯ
ವರದರಾಜ
ವರದರಾಜದೇವರ
ವರದರಾಜಪುರವೆಂಬ
ವರದರಾಜಪೆರುಮಾಳ್
ವರದರಾಜಯ್ಯನೆಂಬ
ವರದರಾಜಸಮುದ್ರವೆಂಬ
ವರದರಾಜಸ್ವಾಮಿ
ವರದರಾಜಸ್ವಾಮಿಯವರ
ವರದಾಚಾರ್ಯನ
ವರದಿಯಂತೆ
ವರದೆಯ
ವರದೆಯನಾಯಕ
ವರದೆಯನಾಯಕನು
ವರದೆಯನಾಯಕನೂ
ವರದೆಯನಾಯುಕನೆಂದೂ
ವರನೊಳು
ವರಪ್ರಸಾದ
ವರಭುಜ
ವರಮಂತ್ರಶಕ್ತಿಯುತನಿಂದ್ರಗೆಂತು
ವರಮಂತ್ರಿವಲ್ಲಭ
ವರಹ
ವರಹಕ್ಕೆ
ವರಹಗಳನ್ನು
ವರಹಗಳಿಗೆ
ವರಹವನ್ನು
ವರಹಾನಾಥ
ವರಹಾನಾಥಕಲ್ಲಹಳ್ಳಿಗೆ
ವರಹೀಳನಹಳ್ಳಿ
ವರಾಹನಕಲ್ಲಹಳ್ಳಿ
ವರಾಹನಾಥ
ವರಾಹನಾಥಕಲ್ಲಹಳ್ಳಿಯ
ವರಾಹನಾಥನ
ವರಾಹಮುದ್ರೆಯ
ವರಾಹಸ್ತುತಿ
ವರಿಸದನ್ದು
ವರಿಸಿ
ವರಿಸಿದ್ದ
ವರಿಸಿದ್ದನು
ವರಿಸಿದ್ದು
ವರೆಗೂ
ವರೆಗೆ
ವರ್ಗಕ್ಕೆ
ವರ್ಗದ
ವರ್ಗದವರೂ
ವರ್ಗವನ್ನಾಗಿ
ವರ್ಗವಾಗಿ
ವರ್ಗಾಯಿಸಲಾಯಿತು
ವರ್ಗಾಯಿಸಿ
ವರ್ಗಾಯಿಸಿದನೆಂದು
ವರ್ಗಾಯಿಸಿರಬಹುದೆಂದು
ವರ್ಣದವರಾದ
ವರ್ಣನಾ
ವರ್ಣನೆ
ವರ್ಣನೆಯ
ವರ್ಣನೆಯನ್ನು
ವರ್ಣನೆಯಲ್ಲಿ
ವರ್ಣನೆಯಿಂದಲೇ
ವರ್ಣನೆಯು
ವರ್ಣನೆಯೊಂದಿಗೆ
ವರ್ಣಿಸಲಾಗಿದೆ
ವರ್ಣಿಸಿದೆ
ವರ್ಣಿಸಿದ್ದು
ವರ್ಣಿಸಿವೆ
ವರ್ಣಿಸುತ್ತದೆ
ವರ್ಣಿಸುತ್ತವೆ
ವರ್ಣಿಸುವ
ವರ್ತಕ
ವರ್ತಕರು
ವರ್ತಕಸಂಘದ
ವರ್ತಕಸಮುದಾಯದವರಾಗಿ
ವರ್ತ್ತಮಾನರಾಯ
ವರ್ಧಮಾನಾಪದಾನಃ
ವರ್ಮನನ್ನು
ವರ್ಷ
ವರ್ಷಂಪ್ರತಿ
ವರ್ಷಕ್ಕಿಂತ
ವರ್ಷಕ್ಕೂ
ವರ್ಷಕ್ಕೆ
ವರ್ಷಗಳ
ವರ್ಷಗಳಲ್ಲಿ
ವರ್ಷಗಳಿಂದ
ವರ್ಷದ
ವರ್ಷದಂತೆ
ವರ್ಷದಲ್ಲಿ
ವರ್ಷದಲ್ಲಿಯೇ
ವರ್ಷವನ್ನು
ವರ್ಷವಾಗಿಬಿಡುತ್ತದೆ
ವರ್ಷವೆಂದು
ವರ್ಷವೇ
ವರ್ಷೆ
ವರ್ಷೇ
ವಲಯಗಳು
ವಲಸೆ
ವಲ್ಲಭನು
ವಳನಾಡಿನಲ್ಲಿದ್ದ
ವಳಬಾಗಿಲ
ವಳಭೀಪುರವರೇಶ್ವರ
ವಳಭೀಪುರೇಶ್ವರ
ವಳಿತ
ವಳಿತಕ್ಕೆ
ವಳಿತಗಳೆಂದು
ವಳಿತದ
ವಶಕ್ಕೆ
ವಶದಲ್ಲಿ
ವಶದಲ್ಲಿತ್ತು
ವಶಪಡಿಸಕೊಂಡ
ವಶಪಡಿಸಿಕೊಂಡ
ವಶಪಡಿಸಿಕೊಂಡದ್ದು
ವಶಪಡಿಸಿಕೊಂಡನು
ವಶಪಡಿಸಿಕೊಂಡನೆಂದು
ವಶಪಡಿಸಿಕೊಂಡರು
ವಶಪಡಿಸಿಕೊಂಡಿದ್ದನು
ವಶಪಡಿಸಿಕೊಂಡಿದ್ದನೆಂದು
ವಶಪಡಿಸಿಕೊಂಡಿದ್ದನೆಂಬುದು
ವಶಪಡಿಸಿಕೊಂಡಿದ್ದರಿಂದ
ವಶಪಡಿಸಿಕೊಂಡಿದ್ದು
ವಶಪಡಿಸಿಕೊಂಡಿರಬಹುದು
ವಶಪಡಿಸಿಕೊಂಡು
ವಶಪಡಿಸಿಕೊಳ್ಳಲು
ವಶಪಡಿಸಿಕೊಳ್ಳುವುದರಲ್ಲಿ
ವಶವಾಗಿರಬಹುದು
ವಶವಾಗಿರಲಿಲ್ಲ
ವಸಂತಲಕ್ಷ್ಮಿಯವರ
ವಸಂತೋತ್ಸವ
ವಸಿಷ್ಠಗೋತ್ರೋದ್ಭವನೂ
ವಸುಂಧರಾ
ವಸುಂಧರಾಫಿಲಿಯೋಜಾ
ವಸುಂಧರೆಯನ್ನು
ವಸೂಲಿ
ವಸೂಲು
ವಸೂಲುಮಾಡಿ
ವಸ್ತುಗಳ
ವಸ್ತುಗಳಿಗಾಗಿ
ವಸ್ತುವಂ
ವಸ್ತುವನ್ನು
ವಸ್ತುವಾಹನ
ವಸ್ತುವಿಸ್ತಾರನುಂ
ವಸ್ತ್ರವನ್ನು
ವಹನ್
ವಹಿಸಲಾಯಿತು
ವಹಿಸಿ
ವಹಿಸಿಕೊಂಡನು
ವಹಿಸಿಕೊಂಡನೆಂದು
ವಹಿಸಿಕೊಂಡಮೇಲೆ
ವಹಿಸಿಕೊಂಡಿದ್ದನೆಂದು
ವಹಿಸಿಕೊಂಡಿರಬಹುದು
ವಹಿಸಿಕೊಂಡಿರಬಹುದೆಂದು
ವಹಿಸಿದ
ವಹಿಸಿದ್ದ
ವಹಿಸಿದ್ದನು
ವಹಿಸಿದ್ದನೆಂದು
ವಹಿಸಿದ್ದಾರೆ
ವಹಿಸಿರಬಹುದೆಂದು
ವಹಿಸಿರುವಂತೆ
ವಹಿಸಿರುವುದು
ವಹಿಸುತ್ತಿದ್ದ
ವಹಿಸುತ್ತಿದ್ದರು
ವಾಂಡಿವಾಷ್ನಲ್ಲಿ
ವಾಂತಿಭ್ರಾಂತಿ
ವಾಕ್ಯವು
ವಾಕ್ಯವೂ
ವಾಗೀಶ್ವರಮಂಗಲ
ವಾಗೀಶ್ವರಮಂಗಲದ
ವಾಚಕ
ವಾಚಕದಿಂದಅ
ವಾಜಿ
ವಾಜಿಕುಲತಿಲಕನಾಗಿದ್ದು
ವಾಜಿಕುಲತಿಲಕನಾದ
ವಾಜಿಕುಲದ
ವಾಜಿವಂಶದವರೇ
ವಾಡಕ್ಕೆಘಟ್ಟಹೊಡಾಘಟ್ಟ
ವಾಡಿಕೆ
ವಾಣಸತ್ತಿ
ವಾಣಿಜ್ಯ
ವಾತಾವರಣವು
ವಾದ
ವಾದಕ್ಕೆ
ವಾದಗಳಿವೆ
ವಾದವಿವಾದಗಳನ್ನು
ವಾದವಿವಾದಗಳು
ವಾದಿರಾಜದೇವನ
ವಾದ್ಯಗಳು
ವಾದ್ಯವಿಶೇಷಗಳಿರಬಹುದು
ವಾದ್ಯವಿಶೇಷವೋ
ವಾನವನ್
ವಾನವನ್ಮಾದೇವಿ
ವಾಯುವ್ಯ
ವಾರಣಾಸಿ
ವಾರಣಾಸಿಗೆ
ವಾರದ
ವಾರದ್
ವಾರಸುದಾರರನ್ನು
ವಾರಸುದಾರಿಕೆಯು
ವಾರ್ತೆಯನ್ನು
ವಾರ್ದ್ಧಿವರ್ಧನ
ವಾಲಗ
ವಾಲಗದವರು
ವಾಸಂತಿಕಾ
ವಾಸಮದೇಭಾವಳಿಯಂ
ವಾಸಮಾಡಿಕೊಂಡು
ವಾಸಮಾಡುತ್ತಿದ್ದನು
ವಾಸವ
ವಾಸವನ
ವಾಸವನಿಗೆ
ವಾಸಿಯಾಗಿ
ವಾಸಿಸುತ್ತಿದ್ದ
ವಾಸು
ವಾಸುದೇವ
ವಾಸುವಿನ
ವಾಸ್ತವವಾಗಿ
ವಾಸ್ತು
ವಾಸ್ತುದೃಷ್ಟಿಯಿಂದ
ವಾಸ್ತುವಿನ
ವಾಸ್ತುಶಿಲ್ಪ
ವಾಹನವಸ್ತುಗಳು
ವಿ
ವಿಂಗಡಿಸದೇ
ವಿಂಗಡಿಸಬಹುದು
ವಿಂಗಡಿಸಬಹುದೆಂದು
ವಿಂಗಡಿಸಲಾಗಿತ್ತೆಂದು
ವಿಂಗಡಿಸಲಾಗಿತ್ತೆಂದೂ
ವಿಂಗಡಿಸಲಾಗುತ್ತಿತ್ತು
ವಿಂಗಡಿಸಲಾಯಿತು
ವಿಂಗಡಿಸಲ್ಪಟ್ಟಿತ್ತು
ವಿಂಗಡಿಸಿ
ವಿಂಗಡಿಸಿದರೆ
ವಿಂಗಡಿಸಿದ್ದರು
ವಿಕಲ್ಪ
ವಿಕ್ರಮಗಂಗ
ವಿಕ್ರಮನ
ವಿಕ್ರಮನನ್ನು
ವಿಕ್ರಮಯುತರೂ
ವಿಕ್ರಮರಾಯ
ವಿಕ್ರಮರಾಯವಿಗಡ
ವಿಕ್ರಮಾಂಕದೇವಚರಿತದಲ್ಲಿ
ವಿಕ್ರಮಾದಿತ್ಯ
ವಿಕ್ರಮಾದಿತ್ಯನ
ವಿಕ್ರಮಾದಿತ್ಯನಿಗೆ
ವಿಕ್ರಮಾದಿತ್ಯನು
ವಿಕ್ರಮಾರ್ಜಿತವಾಗಿ
ವಿಕ್ರಮಾರ್ಜಿತವಾದ
ವಿಖ್ಯಾತ
ವಿಖ್ಯಾತಗ್ರಾಮಂ
ವಿಖ್ಯಾತೋ
ವಿಗ್ರಹದ
ವಿಗ್ರಹವನ್ನು
ವಿಗ್ರಹವಿದ್ದು
ವಿಗ್ರಹವಿನೋದ
ವಿಘಟನ
ವಿಘ್ನೇಶ್ವರ
ವಿಚಾರ
ವಿಚಾರಕ್ಕೆ
ವಿಚಾರಗಳ
ವಿಚಾರಗಳನ್ನು
ವಿಚಾರಗಳನ್ನುಳ್ಳ
ವಿಚಾರಗಳಿಗೆ
ವಿಚಾರಗಳು
ವಿಚಾರಗಳೂ
ವಿಚಾರಚಾವಡಿ
ವಿಚಾರಣೆಯ
ವಿಚಾರದ
ವಿಚಾರದಲ್ಲಿ
ವಿಚಾರವನ್ನು
ವಿಚಾರವಾಗಲೀ
ವಿಚಾರವಿದೆ
ವಿಚಾರವೂ
ವಿಚಾರಾರ್ಹ
ವಿಚಾರಿಸಲು
ವಿಚಿತ್ರವಾದ
ವಿಜಗೀಷುವೃತ್ತಿಯಿಂ
ವಿಜಯ
ವಿಜಯಕ್ಕಾಗಿ
ವಿಜಯಗಳ
ವಿಜಯಗಳನ್ನು
ವಿಜಯಗಳಾಗಿದ್ದು
ವಿಜಯದ
ವಿಜಯದಂತಹ
ವಿಜಯದಲ್ಲಿ
ವಿಜಯನಗರ
ವಿಜಯನಗರಕ್ಕೆ
ವಿಜಯನಗರದ
ವಿಜಯನಗರದಿಂದ
ವಿಜಯನಗರದೊರೆ
ವಿಜಯನಗರವಾದ
ವಿಜಯನಗರವು
ವಿಜಯನಗರಿಯಲ್ಲಿ
ವಿಜಯನರಸಿಂಹ
ವಿಜಯನಾರಸಿಂಹ
ವಿಜಯನಾರಸಿಂಹದೇವರಿಗೆ
ವಿಜಯನಾರಸಿಂಹನ
ವಿಜಯನಾರಸಿಂಹನು
ವಿಜಯನಾರಾಯಣ
ವಿಜಯಪಾಂಡ್ಯನುಚ್ಚಂಗಿ
ವಿಜಯಪುರ
ವಿಜಯಪುರದ
ವಿಜಯಪುರದಲ್ಲಿರುವ
ವಿಜಯಬುಕ್ಕ
ವಿಜಯಬುಕ್ಕರಾಯನ
ವಿಜಯಯಾತ್ರೆಯ
ವಿಜಯರಾಜಧಾನಿ
ವಿಜಯರಾಯ
ವಿಜಯವನ್ನು
ವಿಜಯವನ್ನೇ
ವಿಜಯವಿರೂಪಾಕ್ಷ
ವಿಜಯವೆಂದು
ವಿಜಯವೇ
ವಿಜಯಶ್ರೀ
ವಿಜಯಸಂವತ್ಸರದ
ವಿಜಯಸಮುದ್ರವೆನಿಸಿದ
ವಿಜಯಸೋಮನಾಥಪುರ
ವಿಜಯಸ್ಕಾಂದವಾರವಾದ
ವಿಜಯಸ್ತಂಭವನ್ನು
ವಿಜಯಾದಿತ್ಯ
ವಿಜಯಾದಿತ್ಯನಾಗಿದ್ದು
ವಿಜಯಾದಿತ್ಯನಿಗೆ
ವಿಜಯಾದಿತ್ಯನು
ವಿಜಯೋತ್ತುಂಗ
ವಿಜಯೋತ್ಸವ
ವಿಜಯೋತ್ಸವವನ್ನಾಚರಿಸಲು
ವಿಜಯೋತ್ಸವವನ್ನು
ವಿಜೆಯನರಸಿಂಹಂ
ವಿಜ್ಞಾಪನೆ
ವಿಜ್ಞಾಪನೆಯ
ವಿಜ್ಞಾಪನೆಯನ್ನು
ವಿಟ್ಟಿಯಣ್ಣ
ವಿಠಂಣ್ಣ
ವಿಠಂಣ್ಣಗಳ
ವಿಠಣ್ಣಹೆಗ್ಗಡೆಯು
ವಿಠಲೇಶ್ವರ
ವಿಣ್ಣಯಾಂಡರ
ವಿಣ್ನಘರಂ
ವಿತ್ತಿ
ವಿತ್ತಿರುಂದವಿರ್ರಿರುಂದ
ವಿದೇಶ
ವಿದೇಶಾಂಗ
ವಿದೇಶಾಂಗಕ್ಕೆ
ವಿದ್ಯಾ
ವಿದ್ಯಾಕೆಂದ್ರಗಳನ್ನು
ವಿದ್ಯಾದಾನಗಳಿಂದ
ವಿದ್ಯಾಧರ
ವಿದ್ಯಾಧರನು
ವಿದ್ಯಾಧರನೂ
ವಿದ್ಯಾನಗರದಿಂದ
ವಿದ್ಯಾನಗರಿ
ವಿದ್ಯಾನಗರಿಯ
ವಿದ್ಯಾನಗರಿಯಿಂದ
ವಿದ್ಯಾನಗರ್ಯ್ಯಾಂ
ವಿದ್ಯಾನಿಧಿ
ವಿದ್ಯಾಭ್ಯಾಸ
ವಿದ್ಯಾರಾಜು
ವಿದ್ಯಾರ್ಥಿಗಳ
ವಿದ್ಯಾವಿಶಾರದರಪ್ಪ
ವಿದ್ರಾವಣಂ
ವಿದ್ವಜನಪೋಷಕನೂ
ವಿದ್ವಜನವಿಪದಳನ
ವಿದ್ವತ್
ವಿದ್ವನ್ಮಂಡಲಿಯ
ವಿದ್ವಾಂಸ
ವಿದ್ವಾಂಸರ
ವಿದ್ವಾಂಸರಮತ
ವಿದ್ವಾಂಸರಲ್ಲಿ
ವಿದ್ವಾಂಸರಾಗಿದ್ದ
ವಿದ್ವಾಂಸರಾದ
ವಿದ್ವಾಂಸರಿಂದ
ವಿದ್ವಾಂಸರಿಗೆ
ವಿದ್ವಾಂಸರು
ವಿದ್ವಾಂಸರುಗಳು
ವಿದ್ವಿಷ್ಟ
ವಿಧ
ವಿಧವೆಯಾಗಿದ್ದರೂ
ವಿಧಾನದಿಂದ
ವಿಧಿಗಳನ್ನು
ವಿಧಿಗೆ
ವಿಧಿಯಿಂದ
ವಿಧೇಯರಾಗಿ
ವಿನಂತಿ
ವಿನಂತಿಯಮೇರೆಗೆ
ವಿನಯಪರಂ
ವಿನಯವಾಗಿ
ವಿನಯವಿಭೂಷಿತನೂ
ವಿನಯಸ್ಯೇವ
ವಿನಯಾದಿತ್ಯ
ವಿನಯಾದಿತ್ಯನ
ವಿನಯಾದಿತ್ಯನನ್ನು
ವಿನಯಾದಿತ್ಯನಾಗಿದ್ದಾನೆ
ವಿನಯಾದಿತ್ಯನು
ವಿನಯಾದಿತ್ಯನೇ
ವಿನಾ
ವಿನುತ
ವಿನೇಜತ್ಸಾಮ್ರ
ವಿನೇಯವಿಳಾಸಂ
ವಿನೋದದಿಂದ
ವಿನೋದನೂ
ವಿನೋದಿ
ವಿನೋದಿಂದಾಳೆ
ವಿಪರೀತವಾಗಿದ್ದಂತೆ
ವಿಪ್ರಕುಲತಿಲಕನೂ
ವಿಪ್ರರು
ವಿಪ್ರೋತ್ತಮನಿಗೆ
ವಿಫಲಗೊಳಿಸಿ
ವಿಫಲವಾಯಿತು
ವಿಬುಧ
ವಿಬುಧಜನಫಳಪ್ರದಾಯಕಂ
ವಿಬುಧಪ್ರಸನ್ನನುಂ
ವಿಭಜನೆಯಾಗಿದ್ದರಿಂದ
ವಿಭಜಿತವಾಗಿದ್ದು
ವಿಭಜಿಸಲಾಗಿತ್ತು
ವಿಭಜಿಸಲಾಗಿತ್ತೆಂದು
ವಿಭಜಿಸಲ್
ವಿಭಜಿಸಿ
ವಿಭಜಿಸಿತ್ತು
ವಿಭಜಿಸಿದನು
ವಿಭಜಿಸಿದ್ದು
ವಿಭವಪ್ರಭಾವತೆಯಿಂದಂ
ವಿಭಾಗ
ವಿಭಾಗಕ್ಕೆ
ವಿಭಾಗಗಳ
ವಿಭಾಗಗಳನ್ನಾಗಿ
ವಿಭಾಗಗಳನ್ನು
ವಿಭಾಗಗಳಲ್ಲಿ
ವಿಭಾಗಗಳಾಗಿ
ವಿಭಾಗಗಳಾಗಿದ್ದವು
ವಿಭಾಗಗಳಿಗೂ
ವಿಭಾಗಗಳಿಗೆ
ವಿಭಾಗಗಳಿದ್ದವು
ವಿಭಾಗಗಳು
ವಿಭಾಗಗಳೂ
ವಿಭಾಗಗಳೆಂದು
ವಿಭಾಗಗಳೆನ್ನಬಹುದು
ವಿಭಾಗಗಳೇ
ವಿಭಾಗದ
ವಿಭಾಗದಲ್ಲಿ
ವಿಭಾಗವನ್ನು
ವಿಭಾಗವಾಗಿತ್ತು
ವಿಭಾಗವಾಗಿತ್ತೆಂದು
ವಿಭಾಗವಾಗಿದ್ದ
ವಿಭಾಗವಾಗಿದ್ದಿರಬಹುದು
ವಿಭಾಗವಾಗಿದ್ದು
ವಿಭಾಗವು
ವಿಭಾಗವೂ
ವಿಭಾಗವೆಂದು
ವಿಭಾಗವೇ
ವಿಭಾಗಿಸಬಹುದು
ವಿಭಾಗಿಸಲಾಗಿತ್ತು
ವಿಭಾಗಿಸಿ
ವಿಭಾಡರೆನಿಸಿದ
ವಿಭಿನ್ನವಾಗಿದೆ
ವಿಭಿನ್ನವಾಗಿವೆ
ವಿಭು
ವಿಭುಗಳು
ವಿಭುದೇವರಾಜನಂ
ವಿಭುದೇಶಂ
ವಿಭುಪ್ರಭು
ವಿಭುಬಲ್ಲಯ್ಯನಾಯಕ
ವಿಭೂತಿಯನ್ನು
ವಿಮಲ
ವಿಮಲನಾಥ
ವಿಮಳಗಂಗಾನ್ವಯ
ವಿಯಷವನ್ನು
ವಿರಚಿತ
ವಿರಾಜಮಾನ
ವಿರಾಜಮಾನಂತಂತ್ರರಕ್ಷಾಮಣಿ
ವಿರಾಜಿತ
ವಿರಾಜಿತನಾಗಿದ್ದನೆಂದು
ವಿರುದಯರಾಯ
ವಿರುದ್ಧ
ವಿರುದ್ಧದ
ವಿರುದ್ಧವಾಗಿರಲು
ವಿರುದ್ಧವೇ
ವಿರುಪಂಣ
ವಿರುಪಣ್ಣ
ವಿರುಪಣ್ಣನವರ
ವಿರುಪನಪುರ
ವಿರುಪರಾಜ
ವಿರುಪಾಕ್ಷದೇವಅಣ್ಣನು
ವಿರೂಪಾಕ್ಷ
ವಿರೂಪಾಕ್ಷದಲಿ
ವಿರೂಪಾಕ್ಷದೇವ
ವಿರೂಪಾಕ್ಷದೇವರ
ವಿರೂಪಾಕ್ಷನ
ವಿರೂಪಾಕ್ಷನನ್ನು
ವಿರೂಪಾಕ್ಷನಿಗೆ
ವಿರೂಪಾಕ್ಷನು
ವಿರೂಪಾಕ್ಷನೆಂಬ
ವಿರೂಪಾಕ್ಷಪುರ
ವಿರೂಪಾಕ್ಷಪುರವೆಂದು
ವಿರೂಪಾಕ್ಷಪುರವೆಂಬ
ವಿರೂಪಾಕ್ಷಯ್ಯ
ವಿರೂಪಾಕ್ಷಿಪುರಗಳಲ್ಲಿ
ವಿರೋಧಿಗಳಾಗಿದ್ದುದೂ
ವಿರೋಧಿಗಳಾಗಿರಲಿಲ್ಲ
ವಿರೋಧಿಸಂವತ್ಸರದ
ವಿರೋಧಿಸಿ
ವಿರೋಧಿಸಿಚಿವೋನ್ಮತ್ತೇಭ
ವಿರ್ರಿರುಂದ
ವಿಲಾಸದರ್ಪಣದಂತೆ
ವಿಲೀನಗೊಳಿಸಲಾಯಿತು
ವಿಳಂದೆ
ವಿಳಸತ್
ವಿಳಸದ್ಬಲ್ಲಾಳದೇವಾವನೀಪತಿಗೀ
ವಿವರ
ವಿವರಗಳನ್ನು
ವಿವರಗಳಿದ್ದು
ವಿವರಗಳಿವೆ
ವಿವರಗಳು
ವಿವರಗಳೂ
ವಿವರಣೆಯನ್ನು
ವಿವರಣೆಯಿಂದ
ವಿವರವನ್ನು
ವಿವರವಾಗಿ
ವಿವರವಾಗಿದ್ದು
ವಿವರವಿದೆ
ವಿವರಿಸಲಾಗಿದೆ
ವಿವರಿಸಿದೆ
ವಿವರಿಸಿದ್ದಾರೆ
ವಿವರಿಸುತ್ತದೆ
ವಿವರಿಸುವ
ವಿವಾಹ
ವಿವಾಹಂ
ವಿವಾಹಕಾಲದಲ್ಲಿ
ವಿವಾಹದ
ವಿವಾಹಮಾಡಿಕೊಟ್ಟು
ವಿವಾಹವಾಗಿದ್ದ
ವಿವಾಹವಾದ
ವಿವಾಹವಾದನು
ವಿವಿಧ
ವಿವೇಚನೆಯ
ವಿವೇಚಿಸಲಾಗಿದೆ
ವಿವೇಚಿಸಿದಲ್ಲಿ
ವಿವೇಚಿಸಿದ್ದಾರೆ
ವಿವೇಚಿಸಿದ್ದು
ವಿಶಾಲಮುದ್ರಿ
ವಿಶಾಲವಾದ
ವಿಶಿವಾನಂದ್
ವಿಶಿಷ್ಟ
ವಿಶಿಷ್ಟವಾದ
ವಿಶುದ್ಧ
ವಿಶೇಷ
ವಿಶೇಷಣ
ವಿಶೇಷಣಗಳನ್ನು
ವಿಶೇಷಣದಿಂದ
ವಿಶೇಷಣವನ್ನು
ವಿಶೇಷದ
ವಿಶೇಷದವರು
ವಿಶೇಷವಾಗಿ
ವಿಶೇಷವಾಗಿದೆ
ವಿಶೇಷವಾಗಿದ್ದು
ವಿಶ್ಲೇಷಣೆ
ವಿಶ್ಲೇಷಣೆಗಳಿಂದ
ವಿಶ್ಲೇಷಣೆಗೆ
ವಿಶ್ಲೇಷಣೆಯಾಗಲೀ
ವಿಶ್ಲೇಷಣೆಯಿಂದ
ವಿಶ್ಲೇಷಿಸಿದ್ದಾರೆ
ವಿಶ್ವಕರ್ಮ
ವಿಶ್ವಕೋಶದಂತಿದೆ
ವಿಶ್ವಣ್ಣ
ವಿಶ್ವನಾಥಪುರವಾದ
ವಿಶ್ವಭೂಪಾಳಕರ್
ವಿಶ್ವವಿದ್ಯಾನಿಲದಯ
ವಿಶ್ವವಿದ್ಯಾನಿಲಯಗಳ
ವಿಶ್ವಸಣ್ಣ
ವಿಶ್ವಾವನಿ
ವಿಶ್ವಾಸಾರ್ಹವೂ
ವಿಶ್ವಾಸಾವಾಸವೇಶ್ಮನಃ
ವಿಶ್ವಾಸಿಕ
ವಿಶ್ವಾಸಿಕರಾಗಿ
ವಿಶ್ವೇಶ್ವರದೇವರ
ವಿಶ್ವೇಶ್ವರಯ್ಯ
ವಿಷಯ
ವಿಷಯಕೆರೆಗೋಡುನಾಡು
ವಿಷಯಕ್ಕೆ
ವಿಷಯಗಳ
ವಿಷಯಗಳನ್ನು
ವಿಷಯಗಳಾಗಿ
ವಿಷಯಗಳಿದ್ದವೆಂದು
ವಿಷಯಗಳು
ವಿಷಯಗಳೇ
ವಿಷಯದ
ವಿಷಯದಲ್ಲಿ
ವಿಷಯದಲ್ಲಿದ್ದ
ವಿಷಯವನ್ನು
ವಿಷಯವನ್ನೊಳಗೊಂಡ
ವಿಷಯವಾಗಿದೆ
ವಿಷಯವಾಗಿರಬಹುದು
ವಿಷಯವು
ವಿಷಯಾಧೀಶರನ್ನು
ವಿಷ್ಟಪತ್ರಯ
ವಿಷ್ಣವರ್ಧನನ
ವಿಷ್ಣವರ್ಧನನಿಗೆ
ವಿಷ್ಣವಿಗೂ
ವಿಷ್ಣು
ವಿಷ್ಣುಚಮೂಪತಿ
ವಿಷ್ಣುದಂಡಾಧೀಶ
ವಿಷ್ಣುದಂಡಾಧೀಶನನ್ನು
ವಿಷ್ಣುದಂಡಾಧೀಶನು
ವಿಷ್ಣುದಂಡಾಧೀಶನುಂ
ವಿಷ್ಣುದಂಡಾಧೀಶನೇ
ವಿಷ್ಣುದಂಡಾಧೀಶರುಃ
ವಿಷ್ಣುದೇವ
ವಿಷ್ಣುದೇವನಿಗೆ
ವಿಷ್ಣುಪುರ
ವಿಷ್ಣುಪುರಾಣದ
ವಿಷ್ಣುಭಟ್ಟಯ್ಯನ
ವಿಷ್ಣುಭೂಪನ
ವಿಷ್ಣುರಾಯ
ವಿಷ್ಣುರಾಯಮಹಾರಾಯನೆಂದರೆ
ವಿಷ್ಣುವರ್ಧನ
ವಿಷ್ಣುವರ್ಧನದೇವರುದುಷ್ಟನಿಗ್ರಹ
ವಿಷ್ಣುವರ್ಧನನ
ವಿಷ್ಣುವರ್ಧನನಿಂದ
ವಿಷ್ಣುವರ್ಧನನಿಗೆ
ವಿಷ್ಣುವರ್ಧನನು
ವಿಷ್ಣುವರ್ಧನನೇ
ವಿಷ್ಣುವರ್ಧನನ್ನು
ವಿಷ್ಣುವರ್ಧನಿನಗಾಗಿ
ವಿಷ್ಣುವರ್ಧನು
ವಿಷ್ಣುವರ್ಧ್ಧನ
ವಿಷ್ಣುವಿಗೆ
ವಿಷ್ಣುವಿನ
ವಿಸ್ತರಣೆ
ವಿಸ್ತರಣೆಯನ್ನು
ವಿಸ್ತರಣೆಯಲ್ಲಿಯೂ
ವಿಸ್ತರಿಸಿತ್ತೆಂದು
ವಿಸ್ತರಿಸಿದಾಗ
ವಿಸ್ತರಿಸಿದೆ
ವಿಸ್ತರಿಸುತ್ತಾನೆ
ವಿಸ್ತಾರವನ್ನು
ವಿಸ್ತಾರವಾಗಿದೆ
ವಿಸ್ತಾರವಾಗಿದ್ದ
ವಿಸ್ತಾರವಾದ
ವಿಸ್ತಾರವಾದಂತೆಲ್ಲಾ
ವಿಸ್ತೀರ್ಣ
ವಿಸ್ತೀರ್ಣವನ್ನು
ವಿಸ್ತೀರ್ಣವಿರುವ
ವಿಸ್ತೃತ
ವಿಹಂಗ
ವೀಕ್ಷಣೆ
ವೀಡು
ವೀತಾಯುಧವ್ರತಧಾರಿಯಾಗಿ
ವೀನರಸಿಂಹನ
ವೀಬಲ್ಲಾಳ
ವೀರ
ವೀರಅಚ್ಯುತರಾಯ
ವೀರಅಚ್ಯುತರಾಯನ
ವೀರಕೃಷ್ಣರಾಯಮಹಾರಾಯ
ವೀರಕೆಕ್ಕಾಯಿ
ವೀರಕೇತೆಯ
ವೀರಕೊಂಗಾಳ್ವ
ವೀರಗಂಗ
ವೀರಗಂಗಪೆರ್ಮಾನಡಿಯು
ವೀರಗಜಬೇಂಟೆಕಾರ
ವೀರಗಲ್ಲನ್ನು
ವೀರಗಲ್ಲಾಗಿದ್ದು
ವೀರಗಲ್ಲಿದೆ
ವೀರಗಲ್ಲಿನಲ್ಲಿ
ವೀರಗಲ್ಲಿನಲ್ಲಿದೆ
ವೀರಗಲ್ಲಿನಿಂದ
ವೀರಗಲ್ಲು
ವೀರಗಲ್ಲುಗಳಲ್ಲಿ
ವೀರಗಲ್ಲುಗಳಿಂದ
ವೀರಗಲ್ಲುಗಳು
ವೀರಗಲ್ಲುಶಾಸನ
ವೀರಗಲ್ಲುಶಾಸನಗಳಿವೆ
ವೀರಗಲ್ಲುಶಾಸನಗಳು
ವೀರಗಲ್ಲುಶಾಸನದಿಂದ
ವೀರಗಲ್ಲೂ
ವೀರಗ್ರಾಣಿಯಾಗಿದ್ದನಂತೆ
ವೀರಚಿಕರಾಯನು
ವೀರಚಿಕವೊಡೆಯರ
ವೀರಚಿಕ್ಕಕೇತಯ್ಯ
ವೀರಚಿಕ್ಕಕೇತೆಯ
ವೀರಚಿಕ್ಕಕೇತೆಯ್ಯನ
ವೀರಚಿಕ್ಕರಾಯ
ವೀರಚಿಕ್ಕರಾಯನಿಗಿಂತ
ವೀರಚಿಕ್ಕರಾಯನು
ವೀರಚಿಕ್ಕರಾಯನೇ
ವೀರಣನಾಯಕರು
ವೀರಣ್ಣ
ವೀರಣ್ಣನಾಯಕನಿಗೆ
ವೀರಣ್ಣನಾಯಕನು
ವೀರತಃ
ವೀರತ್ವದಿಂದ
ವೀರತ್ವವನ್ನು
ವೀರದೇವನ
ವೀರದೇವನಹಳ್ಳಿ
ವೀರದೇವನಹಳ್ಳಿಯ
ವೀರದೇವರಾಯನ
ವೀರನ
ವೀರನಂಜರಾಜ
ವೀರನಂಜರಾಜೊಡೆಯರ
ವೀರನಂಜರಾಯ
ವೀರನಂಜರಾಯನ
ವೀರನನ್ನು
ವೀರನರಪತಿ
ವೀರನರಸಿಂಹ
ವೀರನರಸಿಂಹನ
ವೀರನರಸಿಂಹನಿಂದ
ವೀರನರಸಿಂಹನು
ವೀರನರಸಿಂಹರಾಯರ
ವೀರನರಸಿಂಹೇಂದ್ರಪುರವೆಂಬ
ವೀರನಹಳ್ಳಿ
ವೀರನಾಯಕ
ವೀರನಾರಸಿಂಹ
ವೀರನಾರಸಿಂಹದೇವನು
ವೀರನಾರಸಿಂಹದೇವರ
ವೀರನಾರಸಿಂಹದೇವರಸರ
ವೀರನಾರಸಿಂಹದೇವರಸರು
ವೀರನಾರಸಿಂಹನ
ವೀರನಾರಸಿಂಹನನ್ನು
ವೀರನಾರಸಿಂಹನು
ವೀರನಾರಸಿಂಹಪುರವಾದ
ವೀರನಾರಾಯಣ
ವೀರನಾರಾಯಣದ
ವೀರನಾರಾಯಣದೇವನೆಂಬ
ವೀರನಾರಾಯಣದೇವರ
ವೀರನಾರಾಯಣದೇವರಿಗೆ
ವೀರನಿಗೆ
ವೀರನು
ವೀರನೂ
ವೀರನೃಸಿಂಹ
ವೀರನೆಂದು
ವೀರನೊಬ್ಬನಿಗೆ
ವೀರನೊಬ್ಬನು
ವೀರಪಟ್ಟ
ವೀರಪಟ್ಟಮಂ
ವೀರಪಟ್ಟವನ್ನು
ವೀರಪಾಂಡ್ಯ
ವೀರಪಾಂಡ್ಯನ
ವೀರಪಾಂಡ್ಯನನ್ನು
ವೀರಪುರುಷನಾಗಿದ್ದನು
ವೀರಪೆರ್ಮಾಡಿ
ವೀರಪ್ಪ
ವೀರಪ್ಪವೊಡಯರ
ವೀರಪ್ರತಾಪ
ವೀರಬಂಕೆಯನ
ವೀರಬಂಕೆಯನು
ವೀರಬಲ್ಲಾಳ
ವೀರಬಲ್ಲಾಳದೇವನ
ವೀರಬಲ್ಲಾಳದೇವನಿಗೆ
ವೀರಬಲ್ಲಾಳದೇವನು
ವೀರಬಲ್ಲಾಳದೇವರ
ವೀರಬಲ್ಲಾಳದೇವರಸರ
ವೀರಬಲ್ಲಾಳದೇವರಸರು
ವೀರಬಲ್ಲಾಳನ
ವೀರಬಲ್ಲಾಳನನ್ನು
ವೀರಬಲ್ಲಾಳನಲ್ಲಿ
ವೀರಬಲ್ಲಾಳನಿಗೂ
ವೀರಬಲ್ಲಾಳನಿಗೆ
ವೀರಬಲ್ಲಾಳನು
ವೀರಬಲ್ಲಾಳಪುರವನ್ನು
ವೀರಬಲ್ಲಾಳರಾಯ
ವೀರಬಳಂಜುಧರ್ಮಕ್ಕೆ
ವೀರಬುಕರಾಜ
ವೀರಬುಕ್ಕ
ವೀರಬುಕ್ಕಣ್ಣ
ವೀರಬುಕ್ಕಣ್ಣೊಡೆಯರ
ವೀರಭಟಲಲಾಟಪಟ್ಟಂ
ವೀರಭಟಾವಳಿ
ವೀರಭದ್ರ
ವೀರಭದ್ರದುರ್ಗ
ವೀರಭದ್ರದೇವರ
ವೀರಭದ್ರದೇವರಿಗೆ
ವೀರಭದ್ರಸ್ವಾಮಿ
ವೀರಭುಜಕ್ಕನ್ದೈಯರ್
ವೀರಮಂಗಪ್ಪ
ವೀರಮಯ್ದುನ
ವೀರಮರಣಸ್ಮಾರಕಗಳು
ವೀರಮಸಣನು
ವೀರಯ್ಯ
ವೀರಯ್ಯದಂಡನಾಯಕನ
ವೀರಯ್ಯದಂಡನಾಯಕನು
ವೀರಯ್ಯನನ್ನು
ವೀರಯ್ಯನು
ವೀರರ
ವೀರರಗುಡಿಗಳು
ವೀರರನ್ನಾಗಲೀ
ವೀರರಸರು
ವೀರರಾಗಿರುವುದರಿಂದ
ವೀರರಾಜನ
ವೀರರಾಜನಿಗೆ
ವೀರರಾಜೇಂದ್ರ
ವೀರರಾಜೇಂದ್ರನ
ವೀರರಾಜೇಂದ್ರಹೊಯ್ಸಳ
ವೀರರಾಜೈಯ್ಯನವರ
ವೀರರಾಮದೇವ
ವೀರರಾಮದೇವರಾಯನ
ವೀರರಾಮದೇವರಾಯನಿಂದ
ವೀರರಾಮನಾಥನು
ವೀರರು
ವೀರರುಮ್
ವೀರಲಕ್ಷ್ಮೀಭುಜಂಗ
ವೀರವಿಜಯರಾಯ
ವೀರವಿಜಯರಾಯನೇ
ವೀರವಿರೂಪಾಕ್ಷನನ್ನು
ವೀರವಿರೂಪಾಕ್ಷನು
ವೀರವಿರೂಪಾಕ್ಷಬಲ್ಲಾಳ
ವೀರವಿಷ್ಣುವರ್ಧನ
ವೀರವಿಷ್ಣುವರ್ಧನದೇವ
ವೀರವೆಂದೊಡೀ
ವೀರವೈಷ್ಣವಿ
ವೀರಶಾಸನವನ್ನು
ವೀರಶೆಟ್ಟಿಹಳ್ಳಿ
ವೀರಶೈವ
ವೀರಶೈವಧರ್ಮ
ವೀರಶೈವಧರ್ಮಕ್ಕೆ
ವೀರಶ್ರೀ
ವೀರಶ್ರೀನಾರಸಿಂಹೇಂದ್ರಪುರವಾದ
ವೀರಸಂಗಮೇಶ್ವರರಾಯ
ವೀರಸಿದ್ಧಿವೆರಸು
ವೀರಸೇವುಣರ
ವೀರಸೋಮೇಶ್ವರ
ವೀರಸೋಮೇಶ್ವರದೇವನ
ವೀರಸೋಮೇಶ್ವರನ
ವೀರಸೋಮೇಶ್ವರನು
ವೀರಹನುಮಪ್ಪ
ವೀರಹರಿಯಪ್ಪವೊಡೆಯರು
ವೀರಹರಿಹರ
ವೀರಹರಿಹರರಾಯನ
ವೀರಹರಿಹರವೊಡೆಯರ
ವೀರಹರಿಹರೇಶ್ವರ
ವೀರಹರ್ಯಣ
ವೀರಹರ್ಯಣನ
ವೀರಾಂಬಿಕೆಯರ
ವೀರಾಂಬುಧಿ
ವೀರಾವೇಶದಿಂದ
ವೀರು
ವೀರೇಶ್ವರ
ವೀರೊಡೆಯನಿಗೆ
ವೀರೋ
ವೀಳೆಯವನ್ನು
ವುಂಡಿಗೆಯ
ವೃಂದಾವನಕ್ಕೆ
ವೃಂದಾವನದ
ವೃಂದಾವನನ್ನು
ವೃತಿಯಂ
ವೃತಿಯನ್ನು
ವೃತ್ತವು
ವೃತ್ತಿ
ವೃತ್ತಿಗಳ
ವೃತ್ತಿಗಳನ್ನಾಗಿ
ವೃತ್ತಿಗಳನ್ನು
ವೃತ್ತಿಗಳಿಗೆ
ವೃತ್ತಿಗೆ
ವೃತ್ತಿಯ
ವೃತ್ತಿಯನಾಯಕನಾಗಿದ್ದನು
ವೃತ್ತಿಯನಾಯಕನಾಗಿದ್ದನೆಂದು
ವೃತ್ತಿಯನ್ನು
ವೃತ್ತಿಯಲ್ಲಿ
ವೃತ್ತಿಯವ್ರಿತ್ತಿಯ
ವೃತ್ತಿಯಾಗಿ
ವೃತ್ತಿಯಿಂದ
ವೃತ್ತಿಯು
ವೃದ್ಧರು
ವೃದ್ಧಿ
ವೃದ್ಧಿಗಂತವಾಗುತ್ತಲಿರುವ
ವೃದ್ಧಿಸಿಕೊಂಡನು
ವೃದ್ಧ್ಯರ್ತ್ಥವಾಗಿ
ವೆಂಕಟ
ವೆಂಕಟಕೃಷ್ಣ
ವೆಂಕಟಕೃಷ್ಣರವರು
ವೆಂಕಟನ
ವೆಂಕಟನನ್ನು
ವೆಂಕಟನಿಗೆ
ವೆಂಕಟಪತಯ್ಯ
ವೆಂಕಟಪತಿ
ವೆಂಕಟಪತಿಗೆ
ವೆಂಕಟಪತಿಮಹಾರಾಯ
ವೆಂಕಟಪತಿಮಹಾರಾಯನ
ವೆಂಕಟಪತಿಮಹಾರಾಯರ
ವೆಂಕಟಪತಿಯಾರನ
ವೆಂಕಟಪತಿಯು
ವೆಂಕಟಪತಿಯೇ
ವೆಂಕಟಪತಿರಾಯ
ವೆಂಕಟಪತಿರಾಯದೇವ
ವೆಂಕಟಪತಿರಾಯನ
ವೆಂಕಟಪತಿರಾಯನು
ವೆಂಕಟಪತಿರಾಯರ
ವೆಂಕಟಪ್ಪನಾಯಕನು
ವೆಂಕಟಪ್ಪನು
ವೆಂಕಟರತ್ನಂ
ವೆಂಕಟರತ್ನಮ್
ವೆಂಕಟರಮಣಯ್ಯನವರು
ವೆಂಕಟರಮಣಸ್ವಾಮಿ
ವೆಂಕಟರಾವ್
ವೆಂಕಟಲಕ್ಷ್ಮಮ್ಮನು
ವೆಂಕಟವರದಾಚಾರ್ಯನಿಗೆ
ವೆಂಕಟಾದ್ರಿ
ವೆಂಕಟಾದ್ರಿಗೆ
ವೆಂಕಟಾದ್ರಿನಾಯಕ
ವೆಂಕಟಾದ್ರಿನಾಯಕನಿಂದ
ವೆಂಕಟಾದ್ರಿನಾಯಕನಿಗೆ
ವೆಂಕಟಾದ್ರಿನಾಯಕನು
ವೆಂಕಟಾದ್ರಿನಾಯಕನೂ
ವೆಂಕಟಾದ್ರಿಯು
ವೆಂಕಟಾದ್ರಿಸಮುದ್ರವೆಂಬ
ವೆಂಕಟಾದ್ರೀಶನಾಯಕಸ್ಯ
ವೆಂಕಟೇಶ
ವೆಂಗಳರಾಜಯ್ಯನು
ವೆಂಗಿಮಂಡಲ
ವೆಂಗೇನಹಳ್ಳಿಗಳನ್ನು
ವೆಂಗೇನಹಳ್ಳಿಯು
ವೆಂಜಿಮಲೈ
ವೆಟ್ಟದುಳ್
ವೆಲ್ಲೂರುಬೆಳ್ಳೂರು
ವೆಲ್ಲೆಸ್ಲಿಗೆ
ವೆಲ್ಲೆಸ್ಲಿಯು
ವೆಲ್ಲೆಸ್ಲಿಯೊಂದಿ
ವೇಂಕಟಾದ್ರೀಶ
ವೇಂಟೆ
ವೇಂಟೆಯ
ವೇಂಟೆಯದ
ವೇಂಟೆಯದೊಳಗೆ
ವೇಂಠಕ
ವೇಂಠೆ
ವೇಂಠೆಯ
ವೇಂಠೆಯಕ್ಕೆ
ವೇಂಠೆಯದ
ವೇಂಠೆಯದಲ್ಲಿ
ವೇಂಠೆಯಮಾಗಣಿವಳಿತ
ವೇಗಮಂಗಲಇಂದಿನ
ವೇತನ
ವೇದ
ವೇದಪಾಠಶಾಲೆಯನ್ನು
ವೇದಪುಷ್ಕರಣಿಯನ್ನು
ವೇದಮಾರ್ಗ
ವೇದವನ್ನು
ವೇದವಲ್ಲಿ
ವೇದಶಾಸ್ತ್ರ
ವೇದಾಂತದ
ವೇದಾಂತಿ
ವೇದಾರಣ್ಯ
ವೇದಾರಣ್ಯವೆಂದೂ
ವೇದಾರಣ್ಯವೆಂಬ
ವೇಳಗೆ
ವೇಳಾಪುರಂ
ವೇಳೆ
ವೇಳೆಗಾಗಲೇ
ವೇಳೆಗೆ
ವೇಳೆಯ
ವೇಳೆವಾಳಿ
ವೇಳೆವಾಳಿಯಾಗಿ
ವೇಳೆವಾಳಿಯಾಗಿದ್ದ
ವೈ
ವೈಜನಾಥ
ವೈಜನಾಥಂಗೊಲವಿಂ
ವೈಜನಾಥದೇವರ
ವೈಜನಾಥದೇವರಿಗೆ
ವೈದಿಕ
ವೈದಿಕರೂ
ವೈದ್ಯನಾಥ
ವೈದ್ಯನಾಥದೇವರಿಗೆ
ವೈದ್ಯನಾಥನಿಗೆ
ವೈದ್ಯನಾಥಪುರ
ವೈದ್ಯನಾಥಪುರದ
ವೈದ್ಯನಾಥಮುಡೆಯಾರ್
ವೈದ್ಯನು
ವೈಭೋಗವುಳ್ಳವನಾಗಿಯೂ
ವೈಮನಸ್ಯ
ವೈರತ್ವವು
ವೈರಮುಡಿಯ
ವೈರವು
ವೈರಿಗಳ
ವೈರಿಗಳನ್ನು
ವೈರಿದಿಕ್ಕುಂಜರರುಂ
ವೈರಿಮಂಡಳಿಕ
ವೈರಿಮದಮರ್ದ್ಧನ
ವೈರಿರಾಜರ
ವೈರಿಸಂಹಾರ
ವೈರಿಸಮೂಹಮಿಲ್ಲಿ
ವೈರಿಸಾಮಂತ
ವೈರಿಸೇನೆಯು
ವೈವಾಹಿ
ವೈವಾಹಿಕ
ವೈವಿಧ್ಯಮಯ
ವೈಶಾಖ
ವೈಶಿಷ್ಟ್ಯ
ವೈಷಮ್ಯ
ವೈಷ್ಣವ
ವೈಷ್ಣವಕೇಂದ್ರ
ವೈಷ್ಣವಕೇಂದ್ರಗಳು
ವೈಷ್ಣವದೇವಾಲಯಗಳಿಗೆ
ವೈಷ್ಣವಧರ್ಮದ
ವೈಷ್ಣವಮಹಾಜನಗಳು
ವೈಷ್ಣವರ
ವೈಷ್ಣವರನ್ನು
ವೈಸಿ
ವೈಸ್ರಾಯ್
ವೊಂದೆ
ವೊಡಸಂದರು
ವೊಡೆಯಅಪ್ಪಣ್ಣ
ವೊಡೆಯನಿಗೆ
ವೊಡೆಯರ
ವೊಡೆಯರಕೂಡೆ
ವೊಡೆಯರಿಗೆ
ವೊಡೇರ
ವೊಮ್ಮಯ್ಯಮ್ಮ
ವೊಮ್ಮಾಯಮ್ಮ
ವೋಜಮಂಗಲ
ವೋಡೆ
ವೋಣಮಯ್ಯನ
ವೋಣಮಯ್ಯನೆಂದು
ವೋಲಗಿಸುತ್ತಿದ್ದನು
ವ್ಯಕ್ತಪಡಿಸಿದ್ದಾರೆ
ವ್ಯಕ್ತವಾಗಿರುವ
ವ್ಯಕ್ತವಾಗುತ್ತದೆ
ವ್ಯಕ್ತವಾಗುತ್ತದೆಂದು
ವ್ಯಕ್ತವಾಗುವುದು
ವ್ಯಕ್ತಿ
ವ್ಯಕ್ತಿಗಳ
ವ್ಯಕ್ತಿಗಳು
ವ್ಯಕ್ತಿಗಿಂತಲೂ
ವ್ಯಕ್ತಿತ್ವ
ವ್ಯಕ್ತಿಯಾಗಿರಬಹುದು
ವ್ಯಕ್ತಿಯೊಬ್ಬನಿಗೆ
ವ್ಯತ್ಯಾಸ
ವ್ಯತ್ಯಾಸಗಳೊಂದಿಗೆ
ವ್ಯತ್ಯಾಸಗಳೊಡನೆ
ವ್ಯತ್ಯಾಸವನ್ನು
ವ್ಯತ್ಯಾಸವೆನ್ನಬಹುದು
ವ್ಯತ್ಯಾಸವೇನೂ
ವ್ಯಯಿಸಿ
ವ್ಯವಸಾಯಕ್ಕೆ
ವ್ಯವಸ್ಥೆ
ವ್ಯವಸ್ಥೆಗಳನ್ನು
ವ್ಯವಸ್ಥೆಗೂ
ವ್ಯವಸ್ಥೆಗೊಳಿಸಿ
ವ್ಯವಸ್ಥೆಯ
ವ್ಯವಸ್ಥೆಯನ್ನು
ವ್ಯವಸ್ಥೆಯಲ್ಲಿ
ವ್ಯವಸ್ಥೆಯಲ್ಲೂ
ವ್ಯವಸ್ಥೆಯು
ವ್ಯವಸ್ಥೆಯೂ
ವ್ಯವಸ್ಥೆಯೇ
ವ್ಯವಹಾರ
ವ್ಯವಹಾರಕ್ಕೆ
ವ್ಯವಹಾರಗಳನ್ನು
ವ್ಯವಹಾರಗಳಲ್ಲಿ
ವ್ಯವಹಾರಗಳಲ್ಲೇ
ವ್ಯವಹಾರಗಳಿಗೆ
ವ್ಯವಹಾರವನ್ನು
ವ್ಯಾಖ್ಯಾನಿಸಿದ್ದಾರೆ
ವ್ಯಾಪಕವಾಗಿ
ವ್ಯಾಪಾರದ
ವ್ಯಾಪಾರಿಗಳ
ವ್ಯಾಪಾರಿಗಳಾಗಿದ್ದಾರೆ
ವ್ಯಾಪಾರಿಗಳಿಂದ
ವ್ಯಾಪಾರಿಗಳು
ವ್ಯಾಪಾರಿಯು
ವ್ಯಾಪಾರಿವರ್ಗದವರಿಗೂ
ವ್ಯಾಪಿಸಿದ್ದ
ವ್ಯಾಪ್ತಿ
ವ್ಯಾಪ್ತಿಗೆ
ವ್ಯಾಪ್ತಿಯನ್ನು
ವ್ಯಾಪ್ತಿಯಲ್ಲಿ
ವ್ಯಾಪ್ತಿಯಲ್ಲಿದ್ದ
ವ್ಯಾಪ್ತಿಯೊಳಗೆ
ವ್ಯಾಸತೀರ್ಥರಿಗೆ
ವ್ಯಾಸತೀರ್ಥರು
ವ್ಯಾಸರಾಯರಿಗೆ
ವ್ರಣೋಪಲಬ್ದ
ವ್ರತದಿಂದ
ವ್ರತದೀಕ್ಷಿತ
ವ್ರತವನ್ನು
ವ್ರಿತ್ತಿ
ವೞ್ದರೆ
ವೞ್ದರೆಯನ್ಯಮ್ಮೂರೊಳೆ
ಶ
ಶಂಕಚಕ್ರದ
ಶಂಕರ
ಶಂಕರನಹಳ್ಳಿಯನ್ನು
ಶಂಕರನಾಯಕನೇ
ಶಂಕರನಾರಾಯಣ
ಶಂಕರಪುರ
ಶಂಕರರಸ
ಶಂಕರರಸಸಂಕರರಸರ
ಶಂಖಚಕ್ರ
ಶಂಖಚಕ್ರದ
ಶಂತನುವು
ಶಂಭವರಾಯನ
ಶಂಭು
ಶಂಭುದೇವ
ಶಂಭುದೇವನಿಗೆ
ಶಂಭುವನ್ನು
ಶಂಭುವರಾಯರು
ಶಂಭೂನಹಳ್ಳಿ
ಶಂಭೂನಹಳ್ಳಿಯ
ಶಕ
ಶಕವರುಷ
ಶಕವರ್ಷ
ಶಕವರ್ಷವನ್ನು
ಶಕ್ತತ್ರಯಸಮನ್ವಿತಂ
ಶಕ್ತಿ
ಶಕ್ತಿಯ
ಶಕ್ತಿಯನ್ನು
ಶಕ್ತಿಸಾಮರ್ಥ್ಯ
ಶತಮಾನದ
ಶತಮಾನದಲ್ಲಿ
ಶತಮಾನದವರೆಗೆ
ಶತಮಾನದಿಂದಲೇ
ಶತ್ರುಗಳ
ಶತ್ರುಗಳನ್ನು
ಶತ್ರುಗಳಿಗೆ
ಶತ್ರುರಾಜರನ್ನು
ಶತ್ರುರಾಜರುಗಳಿಗೆ
ಶತ್ರುವಿನ
ಶತ್ರುಸೇನೆ
ಶತ್ರುಸೇನೆಯನ್ನು
ಶನಿವಾರ
ಶನಿವಾರಸಿದ್ಧಿ
ಶಬದ್
ಶಬ್ದ
ಶಬ್ದಕ್ಕೆ
ಶಬ್ದಗಳನ್ನು
ಶಬ್ದಗಳಿಗೆ
ಶಬ್ದಗಳು
ಶಬ್ದದ
ಶಬ್ದದಿಂದ
ಶಬ್ದನ್ನು
ಶಬ್ದವನ್ನು
ಶಬ್ದವು
ಶರಣಾಗತನಾಗಲು
ಶರಣಾಗತವಜ್ರಪಂಜರ
ಶರಣಾಗತವಜ್ರಪಂಜರಂ
ಶರಧಿಗಂಭೀರನೆಂದು
ಶರಭ
ಶರಾಗತಮಂದಾರಃ
ಶಶಕಪುರದ
ಶಶಪುರದ
ಶಶಿವಂಶತಿಲಕ
ಶಸ್ತ್ರಾಸ್ತ್ರ
ಶಹಾ
ಶಾಂತಲದೇವಿ
ಶಾಂತಲದೇವಿಯರ
ಶಾಂತಲೆಗಿಂತ
ಶಾಂತಲೆಯ
ಶಾಂತಲೆಯರ
ಶಾಂತಲೆಯು
ಶಾಂತಿ
ಶಾಂತಿಗ್ರಾಮ
ಶಾಂತಿಗ್ರಾಮದ
ಶಾಂತಿನಾಥ
ಶಾಂತಿನಾಥದೇವರ
ಶಾಂತಿನಾಥದೇವರಿಗೆ
ಶಾಂತೀಶ್ವರ
ಶಾಕೇಭ್ರೇಷು
ಶಾಖೆ
ಶಾಖೆಗೆ
ಶಾಖೆಯ
ಶಾಖೆಯನ್ನು
ಶಾಖೆಯವನಿರಬಹುದು
ಶಾತವಾಹನರ
ಶಾನುಭಾಗ
ಶಾರ್ದೂಲ
ಶಾರ್ವರಿ
ಶಾಲಿವಾಹನ
ಶಾಲೆಗಳನ್ನು
ಶಾಶ್ವತವಾಗಿ
ಶಾಸಗಳಲ್ಲಿ
ಶಾಸತಿ
ಶಾಸದಲ್ಲಿ
ಶಾಸದಿಂದಿ
ಶಾಸನ
ಶಾಸನಕಾರ
ಶಾಸನಕಾರನು
ಶಾಸನಗಳ
ಶಾಸನಗಳಂತೆ
ಶಾಸನಗಳನ್ನ
ಶಾಸನಗಳನ್ನು
ಶಾಸನಗಳನ್ನೂ
ಶಾಸನಗಳಲಿ
ಶಾಸನಗಳಲ್ಲಂತೂ
ಶಾಸನಗಳಲ್ಲಿ
ಶಾಸನಗಳಲ್ಲಿದೆ
ಶಾಸನಗಳಲ್ಲಿದ್ದು
ಶಾಸನಗಳಲ್ಲಿಯೂ
ಶಾಸನಗಳಲ್ಲಿರುವ
ಶಾಸನಗಳಲ್ಲೂ
ಶಾಸನಗಳಾಗಿದ್ದು
ಶಾಸನಗಳಾಗಿವೆ
ಶಾಸನಗಳಾವುವೂ
ಶಾಸನಗಳಿಂದ
ಶಾಸನಗಳಿಗೂ
ಶಾಸನಗಳಿದ್ದು
ಶಾಸನಗಳಿವೆ
ಶಾಸನಗಳು
ಶಾಸನಗಳೂ
ಶಾಸನಗಳೇ
ಶಾಸನತಜ್ಞರು
ಶಾಸನದ
ಶಾಸನದಲಿ
ಶಾಸನದಲ್ಲಂತೂ
ಶಾಸನದಲ್ಲಿ
ಶಾಸನದಲ್ಲಿದಲ್ಲಿ
ಶಾಸನದಲ್ಲಿದೆ
ಶಾಸನದಲ್ಲಿದೆೆ
ಶಾಸನದಲ್ಲಿದ್ದು
ಶಾಸನದಲ್ಲಿಯೂ
ಶಾಸನದಲ್ಲಿರು
ಶಾಸನದಲ್ಲಿರುವ
ಶಾಸನದಲ್ಲಿರುವಂತೆ
ಶಾಸನದಲ್ಲೂ
ಶಾಸನದಲ್ಲೇ
ಶಾಸನದವರೆಗೆ
ಶಾಸನದಿಂದ
ಶಾಸನದಿದ
ಶಾಸನಧಾರಗಳಿಂದ
ಶಾಸನಪದ್ಯಗಳು
ಶಾಸನಲ್ಲಿ
ಶಾಸನವನ್ನು
ಶಾಸನವನ್ನೂ
ಶಾಸನವಾಗಿದೆ
ಶಾಸನವಾಗಿದ್ದು
ಶಾಸನವಾಗಿರಬಹುದು
ಶಾಸನವಾಚಕಚಕ್ರವರ್ತಿ
ಶಾಸನವಾದರೆ
ಶಾಸನವಿದೆ
ಶಾಸನವಿದ್ದು
ಶಾಸನವಿರುವ
ಶಾಸನವು
ಶಾಸನವುನರಸಿಂಹನನ್ನು
ಶಾಸನವುವೀರಬಲ್ಲಾಳ
ಶಾಸನವೂ
ಶಾಸನವೆಂದರೆ
ಶಾಸನವೆಂದು
ಶಾಸನವೇ
ಶಾಸನವೊಂದು
ಶಾಸನೋಕ್ತ
ಶಾಸನೋಕ್ತನಾಗಿದ್ದಾನೆಂದು
ಶಾಸನೋಕ್ತನಾಗಿದ್ದು
ಶಾಸನೋಕ್ತನಾದ
ಶಾಸನೋಕ್ತರಾಗಿದ್ದಾರೆ
ಶಾಸನೋಕ್ತರಾದ
ಶಾಸನೋಕ್ತವಲ್ಲದ
ಶಾಸನೋಕ್ತವಾಗಿ
ಶಾಸನೋಕ್ತವಾಗಿದೆ
ಶಾಸನೋಕ್ತವಾಗಿಲ್ಲ
ಶಾಸನೋಕ್ತವಾಗಿವೆ
ಶಾಸನೋಕ್ತವಾದ
ಶಾಸನ್ದದಲ್ಲಿ
ಶಾಸವನವು
ಶಾಸವನವೇ
ಶಾಸವನು
ಶಾಸಸವು
ಶಾಸೋಕ್ತವಾಗಿವೆ
ಶಾಸ್ತ್ರಾರ್ಥ
ಶಿಂಗಂಣಗಳ
ಶಿಂಗಣ್ಣ
ಶಿಂಗಪ್ಪನಾಯಕ
ಶಿಂಗಪ್ಪನಾಯಕನು
ಶಿಂಗಪ್ಪನಾಯಕರು
ಶಿಂಗಮಾರನಹಳ್ಳಿಯನ್ನು
ಶಿಂಗರೈಯ್ಯಂಗಾರ
ಶಿಂಗರೈಯ್ಯಂಗಾರರ
ಶಿಂಘಣ
ಶಿಂಶಾ
ಶಿಂಷಾ
ಶಿತಕರಗಂಡ
ಶಿಥಿಲಬೆಂಕೊಂಬರುಂ
ಶಿರಚ್ಛೇದ
ಶಿರಪ್ರಧಾನ
ಶಿರಶಾಸನವನ್ನು
ಶಿರಸ್ತೆದಾರ್
ಶಿರಸ್ತೇದಾರ್
ಶಿರಸ್ಸನ್ನು
ಶಿರೋಗ್ರಮಂ
ಶಿರೋಮಣಿ
ಶಿರೋಮಣಿಯಂತಿದ್ದ
ಶಿರೋಮಣಿಯಂತೆ
ಶಿಲಾ
ಶಿಲಾಯುಗದ
ಶಿಲಾಶಾಸನ
ಶಿಲಾಶಾಸನಗಳಲ್ಲಿ
ಶಿಲಾಶಾಸನಗಳು
ಶಿಲಾಶಾಸನದ
ಶಿಲಾಶಾಸನದಲ್ಲಿ
ಶಿಲಾಶಾಸನದಲ್ಲೂ
ಶಿಲಾಶಾಸನವನ್ನು
ಶಿಲಾಶಾಸನವು
ಶಿಲಾಶಾಸನವೇ
ಶಿಲಾಶಾಸವನೇ
ಶಿಲ್ಪಕಲೆ
ಶಿಲ್ಪಕಲೆಯಲ್ಲಿ
ಶಿಲ್ಪಗಳ
ಶಿಲ್ಪಗಳಲ್ಲಿ
ಶಿಲ್ಪಗಳಿವೆ
ಶಿಲ್ಪಗಳು
ಶಿಲ್ಪದ
ಶಿಲ್ಪವು
ಶಿಲ್ಪಾಚಾರಿಯರು
ಶಿಲ್ಪಿಗಳೂ
ಶಿವಃ
ಶಿವಗಂಗೆಗೂ
ಶಿವಗಂಗೆಗೆ
ಶಿವಗಂಗೆಯ
ಶಿವದೇವ
ಶಿವದೇವನು
ಶಿವದೇವಾಲಯ
ಶಿವನ
ಶಿವನಸಮುದ್ರ
ಶಿವನಸಮುದ್ರಗಳನ್ನು
ಶಿವನಸಮುದ್ರದ
ಶಿವನಸಮುದ್ರದಲ್ಲಿ
ಶಿವನಸಮುದ್ರದಲ್ಲಿರುವ
ಶಿವನಸಮುದ್ರದಿಂದ
ಶಿವನಸಮುದ್ರದಿಂದಲೂ
ಶಿವನೆಂದು
ಶಿವಪುರ
ಶಿವಪುರದ
ಶಿವಪುರದೊಳಗಣ
ಶಿವಪುರವನ್ನಾಗಿ
ಶಿವಪುರವನ್ನು
ಶಿವಭಕ್ತ
ಶಿವಮಾರ
ಶಿವಮಾರನ
ಶಿವಮಾರನನ್ನು
ಶಿವಮಾರನಿಗೆ
ಶಿವಮಾರನು
ಶಿವಮಾರಸಿಂಹ
ಶಿವಮಾರಸ್ಯ
ಶಿವಮೊಗ್ಗ
ಶಿವರಮಂಡ್ಯತಾಲ್ಲೂಕಿನ
ಶಿವರಾಜ
ಶಿವರಾಜನು
ಶಿವರುದ್ರಸ್ವಾಮಿ
ಶಿವಶರಣರಿಗೆ
ಶಿವಶರಣರು
ಶಿವಶೋಧ
ಶಿವಸನ್ನಿಧಿಯಲ್ಲಿ
ಶಿವಾಚಾರ
ಶಿವಾಚಾರದ
ಶಿವಾಜಿಯ
ಶಿವಾರ
ಶಿವಾಲಯಕ್ಕೆ
ಶಿವಾಲಯದ
ಶಿಶಿಲದ
ಶಿಷ್ಟಪ್ರತಿಪಾಳನ
ಶಿಷ್ಟಪ್ರಿಯ
ಶಿಷ್ಯ
ಶಿಷ್ಯನೂ
ಶಿಷ್ಯರಾದ
ಶಿಷ್ಯರೊಡಗೂಡಿ
ಶಿಹ್ವ
ಶೀ
ಶೀಘ್ರವೇ
ಶೀರಂಗದೇವ
ಶೀರ್ಯಪೇಟೆ
ಶೀಳುನೆರೆ
ಶು
ಶುಕ್ರವಾರ
ಶುದ್ಧ
ಶುದ್ಧೋಭಯಾನ್ವಯ
ಶುಭ
ಶುಭಚಂದ್ರ
ಶುಭಚಂದ್ರಸಿದ್ಧಾಂತ
ಶುಭದೀಯಾರಭವತ್ಸದಾ
ಶುಭಯಸಿ
ಶುಭಾವತೈಃ
ಶುಭೈಃ
ಶೂದ್ರಕಂ
ಶೂದ್ರಕನೆಂದು
ಶೂದ್ರಕುಲದ
ಶೂದ್ರರು
ಶೂದ್ರವಾಡವಾಗಿದ್ದ
ಶೂರನು
ಶೂರಯತಾ
ಶೂರರು
ಶೃಂಗಾರಹಾರ
ಶೃಂಗೇರಿ
ಶೃಂಗೇರಿಗೆ
ಶೃಂಗೇರಿಯಲ್ಲಿ
ಶೆಟ್ಟಿ
ಶೆಟ್ಟಿಹಳ್ಳಿ
ಶೆಟ್ಟಿಹಳ್ಳಿಗಳಲ್ಲಿ
ಶೇಖರಮಣಿ
ಶೇಖ್
ಶೇಖ್ದಾರ್
ಶೇಲೆಯಪುರಸೇಲಂದ
ಶೇವೆ
ಶೈವ
ಶೈವಕ್ಷೇತ್ರವಾದ
ಶೈವಧರ್ಮವನ್ನು
ಶೈವಧರ್ಮವು
ಶೈವನಾದರೂ
ಶೈವನೊಬ್ಬನ
ಶೈವಪಂಗಡಕ್ಕೆ
ಶೈವಯತಿ
ಶೈವರಾಗಿದ್ದ
ಶೈವಳೆಂದೂ
ಶೈವಶಿಲ್ಪಗಳು
ಶೈವಸಂಸ್ಥೆಗಳಿಗೆ
ಶೋಬಾರ್ಥವಾಗಿಯೂ
ಶೋಭಾ
ಶೋಭಿಸುವಂತೆ
ಶೌಚ
ಶೌಚಮಣಲೆಯರುಂ
ಶೌರ್ಯ
ಶೌರ್ಯದಿಂ
ಶೌರ್ಯದಿಂದ
ಶೌರ್ಯವನ್ನು
ಶೌರ್ಯಾಟೋಪದೊಳು
ಶೌರ್ಯ್ಯದಿಂ
ಶ್ಯಾನುಭೋಗ
ಶ್ಯಾವೆ
ಶ್ರತಿಯೊಳಗಣ
ಶ್ರದ್ಧೆ
ಶ್ರಮಿಸಿದ
ಶ್ರವಣಕಾಲದಲ್ಲಿ
ಶ್ರವಣನಹಳ್ಳಿ
ಶ್ರವಣಬೆಳಗೊಳ
ಶ್ರವಣಬೆಳಗೊಳಕ್ಕೆ
ಶ್ರವಣಬೆಳಗೊಳಗಳು
ಶ್ರವಣಬೆಳಗೊಳದ
ಶ್ರವಣಬೆಳಗೊಳದಲ್ಲಿ
ಶ್ರವಣಬೆಳಗೊಳವನ್ನು
ಶ್ರವಣಬೆಳಗೊಳವು
ಶ್ರವಣಬೆಳಗೊಳವೇ
ಶ್ರವಣಬೆಳಗೊಳಶಾಸನದಲ್ಲಿ
ಶ್ರಾವಣ
ಶ್ರಿ
ಶ್ರಿಮದ್ರಾಜಗುರು
ಶ್ರಿಮನ್ಮಹಾ
ಶ್ರಿಮನ್ಮಹಾಪ್ರಧಾನ
ಶ್ರಿವೈಷ್ಣವರಿಗೆ
ಶ್ರೀ
ಶ್ರೀಕಂಠದೇವ
ಶ್ರೀಕಂಠಶಾಸ್ತ್ರಿಯವರ
ಶ್ರೀಕರಣ
ಶ್ರೀಕರಣಂಗಳು
ಶ್ರೀಕರಣದ
ಶ್ರೀಕರಣದಧಿಷ್ಟಯಕ
ಶ್ರೀಕರಣದಹೆಗ್ಗಡೆ
ಶ್ರೀಕರಣಪ್ರಮುಖ
ಶ್ರೀಕರಣರ
ಶ್ರೀಕರಣರು
ಶ್ರೀಕರಣಾಗ್ರಗಣ್ಯ
ಶ್ರೀಕರಣಾಗ್ರಗಣ್ಯನಾಗಿದ್ದನು
ಶ್ರೀಕರಣಾಗ್ರಗಣ್ಯನೂ
ಶ್ರೀಕರಣಾಗ್ರಗಣ್ಯರು
ಶ್ರೀಕರಣಾಧಿಕಾರಿ
ಶ್ರೀಕರಣಾಧಿಪತಿ
ಶ್ರೀಕಾರ್ಯಕ್ಕೆ
ಶ್ರೀಕೃಷ್ಣ
ಶ್ರೀಕೃಷ್ಣರಾಯ
ಶ್ರೀಕೋವಿರಾಜ
ಶ್ರೀಗೂರನಮಠ
ಶ್ರೀಚಾಮುಣ್ಡರಾಜಂ
ಶ್ರೀದೇವಿ
ಶ್ರೀಧರ
ಶ್ರೀಧರಯ್ಯನ
ಶ್ರೀನಾರಸಿಂಹದೇವರು
ಶ್ರೀನಾರಾಯಣದೇವರ
ಶ್ರೀನಿವಾಸ
ಶ್ರೀನಿವಾಸನ
ಶ್ರೀನಿವಾಸನಿಗೆ
ಶ್ರೀನಿವಾಸರಾವು
ಶ್ರೀನಿವಾಸಾಚಾರಿಯ
ಶ್ರೀನಿವಾಸಾಚಾರ್ಯರೆಂಬ
ಶ್ರೀನೊಳಂಬ
ಶ್ರೀಪರುಷನು
ಶ್ರೀಪಾದಗಳ
ಶ್ರೀಪಾದವನ್ನು
ಶ್ರೀಪಾಲ
ಶ್ರೀಪಾಳತ್ರೈವಿದ್ಯದೇವರೆಂದು
ಶ್ರೀಪುರದ
ಶ್ರೀಪುರದಲ್ಲಿ
ಶ್ರೀಪುರವಾಗಿರಬಹುದೆಂದು
ಶ್ರೀಪುರಷನ
ಶ್ರೀಪುರುಷ
ಶ್ರೀಪುರುಷನ
ಶ್ರೀಪುರುಷನನ್ನು
ಶ್ರೀಪುರುಷನಿಗೆ
ಶ್ರೀಪುರುಷನು
ಶ್ರೀಪೃಥ್ವೀ
ಶ್ರೀಪೆರುಮಾಳೆದೇವದಂಣಾಯಕರ
ಶ್ರೀಬಾಣದ
ಶ್ರೀಬಾಸ
ಶ್ರೀಬಾಸಣ್ಣನ
ಶ್ರೀಭಂಡಾರಕ್ಕೆ
ಶ್ರೀಭಂಡಾರದ
ಶ್ರೀಭಂಡಾರವೂ
ಶ್ರೀಭಾಷ್ಯ
ಶ್ರೀಭೂಮಿ
ಶ್ರೀಮಂತ
ಶ್ರೀಮಂತಿಕೆಗಾಗಿ
ಶ್ರೀಮಂನ್
ಶ್ರೀಮತು
ಶ್ರೀಮತ್
ಶ್ರೀಮತ್ಪಶ್ಚಿಮರಂಗನಾಥ
ಶ್ರೀಮತ್ಪೆರ್ಗ್ಗಡೆ
ಶ್ರೀಮದನಾದಿಯಗ್ರಹಾರಂ
ಶ್ರೀಮದ್
ಶ್ರೀಮದ್ರಾಜಾಧಿರಾಜ
ಶ್ರೀಮದ್ರಾರಾಜಾಧಿರಾಜ
ಶ್ರೀಮನು
ಶ್ರೀಮನುಮಹಾಪ್ರಧಾನ
ಶ್ರೀಮನುಮಹಾಸಾಮಂತ
ಶ್ರೀಮನ್
ಶ್ರೀಮನ್ನೊಳಂಬ
ಶ್ರೀಮನ್ಮಣಲಯರನ
ಶ್ರೀಮನ್ಮಹಾ
ಶ್ರೀಮನ್ಮಹಾನಾಯಂಕಾಚಾರ್ಯ
ಶ್ರೀಮನ್ಮಹಾನಾಯಕ
ಶ್ರೀಮನ್ಮಹಾನಾಯಕಾಚಾರ್ಯ
ಶ್ರೀಮನ್ಮಹಾಪಸಾಯ್ತ
ಶ್ರೀಮನ್ಮಹಾಪ್ರಧಾನ
ಶ್ರೀಮನ್ಮಹಾಪ್ರಧಾನನೆಂದು
ಶ್ರೀಮನ್ಮಹಾಪ್ರಧಾನಿ
ಶ್ರೀಮನ್ಮಹಾಪ್ರಧಾನೆರೆಂದು
ಶ್ರೀಮನ್ಮಹಾಪ್ರಭು
ಶ್ರೀಮನ್ಮಹಾಮಂಡಲೇಶ್ವರ
ಶ್ರೀಮನ್ಮಹಾಮಂಡಲೇಶ್ವರನೆಂದೇ
ಶ್ರೀಮನ್ಮಹಾಮಂಡಳೇಶ್ವರ
ಶ್ರೀಮನ್ಮಹಾರಾಜಾಧಿರಾಜ
ಶ್ರೀಮನ್ಮಹಾವೀರರಾಜೇಂದ್ರ
ಶ್ರೀಮನ್ಮಹಾಸಾಮಂತ
ಶ್ರೀಮನ್ಮಹಾಸಾಮಂತನ
ಶ್ರೀಮನ್ಮಹಾಸಾಮಂತರಾದ
ಶ್ರೀಮನ್ಮಹಾಸಾಮಂತಾಧಿಪತಿ
ಶ್ರೀಮಪ್ರತಿಷ್ಟವೀರಪ್ರಾಜ್ಯರಾಜ್ಯ
ಶ್ರೀಮಲ್ಲಿಕಾರ್ಜುನಮಹಾರಾಯರ
ಶ್ರೀಮಾರಮಯ್ಯ
ಶ್ರೀಯಗಾಮುಣ್ಡರು
ಶ್ರೀಯುಳ್ಳಿನ
ಶ್ರೀರಂಗ
ಶ್ರೀರಂಗಂ
ಶ್ರೀರಂಗಐದನೇ
ಶ್ರೀರಂಗಕ್ಕೆ
ಶ್ರೀರಂಗಗಳ
ಶ್ರೀರಂಗದ
ಶ್ರೀರಂಗದಿಂದ
ಶ್ರೀರಂಗದೇವನ
ಶ್ರೀರಂಗದೇವರಾಯ
ಶ್ರೀರಂಗನ
ಶ್ರೀರಂಗನಾಥದೇವರಿಗೆ
ಶ್ರೀರಂಗನಾಯಕಿದೇವಿಯರ
ಶ್ರೀರಂಗನು
ಶ್ರೀರಂಗನೆಂದು
ಶ್ರೀರಂಗಪಟಕ್ಕೆ
ಶ್ರೀರಂಗಪಟ್ಟ
ಶ್ರೀರಂಗಪಟ್ಟಣ
ಶ್ರೀರಂಗಪಟ್ಟಣಕ್ಕೆ
ಶ್ರೀರಂಗಪಟ್ಟಣಗಳನ್ನು
ಶ್ರೀರಂಗಪಟ್ಟಣಗಳು
ಶ್ರೀರಂಗಪಟ್ಟಣದ
ಶ್ರೀರಂಗಪಟ್ಟಣದಲು
ಶ್ರೀರಂಗಪಟ್ಟಣದಲ್ಲಿ
ಶ್ರೀರಂಗಪಟ್ಟಣದಲ್ಲಿದ್ದಾಗ
ಶ್ರೀರಂಗಪಟ್ಟಣದಿಂದ
ಶ್ರೀರಂಗಪಟ್ಟಣರಾಜ್ಯಗಳು
ಶ್ರೀರಂಗಪಟ್ಟಣವನ್ನು
ಶ್ರೀರಂಗಪಟ್ಟಣವು
ಶ್ರೀರಂಗಪಟ್ಟಣಸೀಮೆಯ
ಶ್ರೀರಂಗಪಟ್ಟಣಸ್ಥಳದ
ಶ್ರೀರಂಗಪಟ್ಟಣೇ
ಶ್ರೀರಂಗಪುರ
ಶ್ರೀರಂಗಪುರದ
ಶ್ರೀರಂಗಪುರದಶ್ರೀರಂಗಪಟ್ಟಣ
ಶ್ರೀರಂಗಪುರವಾದ
ಶ್ರೀರಂಗಮಂಟಪವನ್ನು
ಶ್ರೀರಂಗಮಹಾರಾಯರು
ಶ್ರೀರಂಗಮ್
ಶ್ರೀರಂಗರಾಜ
ಶ್ರೀರಂಗರಾಜದೇವ
ಶ್ರೀರಂಗರಾಜನ
ಶ್ರೀರಂಗರಾಜನು
ಶ್ರೀರಂಗರಾಮದೇವರಾಯ
ಶ್ರೀರಂಗರಾಯದೇವರು
ಶ್ರೀರಂಗರಾಯನ
ಶ್ರೀರಂಗರಾಯನು
ಶ್ರೀರಂಗರಾಯಮಹಾರಾಯರು
ಶ್ರೀರಂಗರಾಯರ
ಶ್ರೀರಂಗೇ
ಶ್ರೀರರ್ದ್ಧನಾರೀನಟೇಶ್ವರಃ
ಶ್ರೀರಾಜ್ಯವೆಂಬ
ಶ್ರೀರಾಮಕೃಷ್ಣದೇವರ
ಶ್ರೀರಾಮನಾಥದೇವರ
ಶ್ರೀರಾಮಾನುಜರಲ್ಲಿ
ಶ್ರೀರಾಮೇಶ್ವರ
ಶ್ರೀಲಕುಮಿ
ಶ್ರೀವಲ್ಲಭನೆಂಬ
ಶ್ರೀವಿಕ್ರಮನ
ಶ್ರೀವಿಜಯ
ಶ್ರೀವಿನಯಾದಿತ್ಯಪೊಯ್ಸಳನೆರೆಯಂಗ
ಶ್ರೀವಿಷ್ಣುಭೂಪಾಳಕಂ
ಶ್ರೀವೀರಪ್ರತಾಪ
ಶ್ರೀವೀರರಾಮದೇವರಾಯರು
ಶ್ರೀವೈಷ್ಣವ
ಶ್ರೀವೈಷ್ಣವಕ್ಷೇತ್ರವಾದ
ಶ್ರೀವೈಷ್ಣವರ
ಶ್ರೀವೈಷ್ಣವರಾದ
ಶ್ರೀವೈಷ್ಣವರಿಗಾಗಿ
ಶ್ರೀವೈಷ್ಣವರಿಗೆ
ಶ್ರೀವೈಷ್ಣವರು
ಶ್ರೀವೈಷ್ಣವಸ್ಥಳವಾದ
ಶ್ರೀಶೈಲ
ಶ್ರೀಶೈಲದ
ಶ್ರೀಶೈಲದಲ್ಲಿ
ಶ್ರೀಹೋಸಲನಾಡಿನ
ಶ್ರೀಹ್ರೀಧೃತಿದ್ಧಾರ್ಯತಾಂ
ಶ್ರುತ
ಶ್ರುತಿಯ
ಶ್ರುತಿಯು
ಶ್ರುತಿಶ್ರೋತ್ರಿಯೂರು
ಶ್ರೇಣಿ
ಶ್ರೇಣಿಗಳಾಗಿದ್ದು
ಶ್ರೇಣಿಯ
ಶ್ರೇಣೀಕೃತ
ಶ್ರೇಷ್ಠನಾಗಿದ್ದನು
ಶ್ರೇಷ್ಠನಾದ
ಶ್ರೇಷ್ಠನಾದವನು
ಶ್ರೋತ್ರೀಯ
ಶ್ರೋತ್ರೀಯವಾಗಿ
ಶ್ರೋತ್ರೀಯಸ್ಯ
ಶ್ಲೋಕದಲ್ಲಿ
ಶ್ಲೋಕವಿದೆ
ಶ್ಲೋಕಾನ್
ಷಣ್ಣವತಿ
ಷಣ್ಣವತಿಸಹಸ್ರ
ಷಣ್ಣವತಿಸಹಸ್ರವಿಷಯ
ಷಣ್ಮುಖ
ಷರತ್ತಿನ
ಷಷ್ಠಿಯಂದು
ಷಹಬಾಜ್
ಷೇಕ್
ಸಂ
ಸಂಕಡಿಸಂನಾಹ
ಸಂಕಮದೇವನ
ಸಂಕಮನು
ಸಂಕರಪ್ಪ
ಸಂಕಲ್ಪಿಸಿದನು
ಸಂಕಹಳ್ಳಿ
ಸಂಕಿಯರ
ಸಂಕಿಯರಕುಲತಿಲಕ
ಸಂಕಿರಣವನ್ನು
ಸಂಕೀರ್ಣ
ಸಂಕೀರ್ಣಶಾಸನ
ಸಂಕೋಚದಾಯಿ
ಸಂಕ್ರಾಂತಿಯ
ಸಂಕ್ಷಿಪ್ತ
ಸಂಕ್ಷಿಪ್ತವಾಗಿ
ಸಂಕ್ಷಿಸುತ್ತಾರೆ
ಸಂಖ್ಯಾ
ಸಂಖ್ಯೆ
ಸಂಖ್ಯೆಗಳನ್ನು
ಸಂಖ್ಯೆಗಳು
ಸಂಖ್ಯೆಯನ್ನು
ಸಂಖ್ಯೆಯಲಿ
ಸಂಖ್ಯೆಯಲ್ಲಿ
ಸಂಖ್ಯೆಯಲ್ಲಿವೆ
ಸಂಗತಿ
ಸಂಗತಿಯಾಗುತ್ತದೆ
ಸಂಗನಬಸವನ
ಸಂಗಮ
ಸಂಗಮಕುಮಾರನಾದ
ಸಂಗಮದ
ಸಂಗಮನ
ಸಂಗಮನಿಂದ
ಸಂಗಮರ
ಸಂಗಮರಾಯಬುಕ್ಕರಾಯಹರಿಹರರಾಯದೇವರಾಯವಿಜೆಯರಾಯಗಜಬೇಂಟೆಕಾಱ
ಸಂಗಮವಂಶಾವಳಿಯನ್ನು
ಸಂಗಮಸೋದರರು
ಸಂಗಮೇಶ್ವರಪುರವಾದ
ಸಂಗಮೇಶ್ವರಪುರವೆಂಬ
ಸಂಗಮೇಶ್ವರರಾಯ
ಸಂಗರಕೆ
ಸಂಗೀತ
ಸಂಗ್ರಹಕ್ಕೆ
ಸಂಗ್ರಹಣೆ
ಸಂಗ್ರಹವಾಗಿ
ಸಂಗ್ರಹಾಲಯವು
ಸಂಗ್ರಹಿಸಿ
ಸಂಗ್ರಹಿಸಿದರು
ಸಂಗ್ರಹಿಸಿದ್ದಾರೆ
ಸಂಗ್ರಹಿಸುವ
ಸಂಗ್ರಾಮ
ಸಂಗ್ರಾಮಭೀಮಯೆಂಬ
ಸಂಗ್ರಾಮರಂಗ
ಸಂಗ್ರಾಮರಾಮ
ಸಂಘಕ್ಕೆ
ಸಂಘಟಿಸಲು
ಸಂಘಟ್ಟ
ಸಂಘಡಿಸ್ಟ್ರಿಕ್ಟ್
ಸಂಘದ
ಸಂಘವೆಂದೂ
ಸಂಘಸಂಸ್ಥೆಗಳಿಗೆ
ಸಂಚರಿಸಿ
ಸಂಚಾರ
ಸಂಚಾರಗಳಿಗೂ
ಸಂಚಿಕೆಗಳಲ್ಲಿ
ಸಂಚಿಗ
ಸಂಚಿತ
ಸಂಚಿಯ
ಸಂಜಾತಂ
ಸಂಜಾತನಾಗಿದ್ದು
ಸಂಜ್ಞಿಕೇ
ಸಂತತಂ
ಸಂತತಿ
ಸಂತತಿಯ
ಸಂತತಿಯಲ್ಲಿ
ಸಂತತಿಯವನೋ
ಸಂತಾನಕ್ಕಾಗಿ
ಸಂತಾನವಾಗಿ
ಸಂತೆ
ಸಂತೆಯಕರದ
ಸಂತೆಯನ್ನು
ಸಂತೆಯಾಗಿತ್ತು
ಸಂತೆಯು
ಸಂತೆಶಿವರ
ಸಂತೇಬಾಚಹಳ್ಳಿ
ಸಂತೇಬಾಚಹಳ್ಳಿಯ
ಸಂತೇಬಾಚಹಳ್ಳಿಯನ್ನು
ಸಂತೇಬಾಚಹಳ್ಳಿಯಲ್ಲಿ
ಸಂತೇಬಾಚಹಳ್ಳಿಯಲ್ಲಿದೆ
ಸಂತೋಷದಿಂದ
ಸಂತ್ರಾಸಿನೃಪಾಪದಃ
ಸಂಥೆಶಾಸನ
ಸಂದ
ಸಂದರ್ಭಕ್ಕೆ
ಸಂದರ್ಭಗಳಲ್ಲಿ
ಸಂದರ್ಭಗಳಲ್ಲಿಯೂ
ಸಂದರ್ಭದಲಿ
ಸಂದರ್ಭದಲ್ಲಿ
ಸಂದರ್ಭೋಚಿತವಾಗಿ
ಸಂದರ್ಶಿಸಿರ
ಸಂದಳಾದಿವಂ
ಸಂದಾಯ
ಸಂದಾಯವಾಯಿತೆಂದು
ಸಂಧಿವಿಗ್ರಹಿ
ಸಂನಾಹಮಾವನಂಕಕಾಱ
ಸಂಪಂನರುಮಪ್ಪ
ಸಂಪಟಕ್ಕೆ
ಸಂಪಟುಗಳ
ಸಂಪತ್ಕರ
ಸಂಪತ್ಕರನಾರಾಯಣದೇವರು
ಸಂಪತ್ತಿಗೆ
ಸಂಪದಂ
ಸಂಪದ್ಭರಿತ
ಸಂಪನನ್ನನುಂ
ಸಂಪನ್ನಂ
ಸಂಪನ್ನನಪ್ಪ
ಸಂಪನ್ನರುಮಪ್ಪ
ಸಂಪನ್ಮೂಲಗಳಿರುವ
ಸಂಪರ್ಕ
ಸಂಪಾಕದರು
ಸಂಪಾದಕರು
ಸಂಪಾದಕರೂ
ಸಂಪಾದನೆಯಿಂದ
ಸಂಪಾದಿಸಿ
ಸಂಪಾದಿಸಿದ
ಸಂಪಾದಿಸಿದ್ದ
ಸಂಪಾದಿಸಿರುವ
ಸಂಪುಟ
ಸಂಪುಟಗಳ
ಸಂಪುಟಗಳನ್ನು
ಸಂಪುಟಗಳಲ್ಲಿ
ಸಂಪುಟಗಳಲ್ಲಿರುವ
ಸಂಪುಟಗಳಿಂದ
ಸಂಪುಟಗಳಿಗೆ
ಸಂಪುಟಗಳು
ಸಂಪುಟಗಳೂ
ಸಂಪುಟದ
ಸಂಪುಟದಲ್ಲಿ
ಸಂಪೂರ್ಣ
ಸಂಪೂರ್ಣವಾಗಿ
ಸಂಪ್ರದಾಯವನ್ನು
ಸಂಪ್ರದಾಯವೂ
ಸಂಪ್ರಾಪ್ಯ
ಸಂಪ್ರೀತನಾದ
ಸಂಬಂಧ
ಸಂಬಂಧಗಳ
ಸಂಬಂಧಗಳನ್ನು
ಸಂಬಂಧಗಳಿದ್ದವು
ಸಂಬಂಧಗಳು
ಸಂಬಂಧದಲ್ಲಿ
ಸಂಬಂಧದಿಂದ
ಸಂಬಂಧಪಟ್ಟಂತೆ
ಸಂಬಂಧಪಟ್ಟವರಲ್ಲವೆಂದೂ
ಸಂಬಂಧಪಟ್ಟವರಾಗಿದ್ದಾರೆ
ಸಂಬಂಧಪಟ್ಟಿರಬೇಕು
ಸಂಬಂಧವನ್ನು
ಸಂಬಂಧವಿಲ್ಲವೆಂದು
ಸಂಬಂಧವೇನು
ಸಂಬಂಧವೇನೆಂದು
ಸಂಬಂಧಿಗಳಾದ
ಸಂಬಂಧಿಯಾದ
ಸಂಬಂಧಿಸಸಿದ
ಸಂಬಂಧಿಸಿದ
ಸಂಬಂಧಿಸಿದಂತಹ
ಸಂಬಂಧಿಸಿದಂತೆ
ಸಂಬಂಧಿಸಿದೆ
ಸಂಬಂಧಿಸಿದ್ದಾಗಿದೆ
ಸಂಬಂಧಿಸಿದ್ದು
ಸಂಬಂಧಿಸಿರಬಹುದೆಂದು
ಸಂಬಳಕೊಡಲಾಗದೆ
ಸಂಬುವಗಉಡ
ಸಂಬೋಧನೆಗೆ
ಸಂಬೋಧಿಸಲಾಗಿದೆ
ಸಂಬೋಧಿಸಿ
ಸಂಬೋಧಿಸಿರುವ
ಸಂಬೋಧಿಸಿವೆ
ಸಂಭವ
ಸಂಭವನೀಯವಲ್ಲ
ಸಂಭವರಾಯ
ಸಂಭವಿಸಿದ
ಸಂಭಾಜಿಯು
ಸಂಭಾವನೆ
ಸಂಭಾವನೆಯಲ್ಲಿ
ಸಂಭುರಾಯ
ಸಂಭುವಗವುಡ
ಸಂಭುವರಾಯ
ಸಂಭೂತದ
ಸಂಮಟಿಭಾಗ
ಸಂಮಟಿಭಾಗಭೂಪತಿ
ಸಂಯುಕ್ತೋ
ಸಂರಕ್ಷಿತ
ಸಂರಕ್ಷಿಸಿ
ಸಂವ
ಸಂವತ್ಸರದ
ಸಂವತ್ಸರದಲ್ಲಿ
ಸಂವತ್ಸರವನ್ನು
ಸಂವಸನ್
ಸಂಶಂಕರ
ಸಂಶೋಧಕನು
ಸಂಶೋಧನಾ
ಸಂಶೋಧನಾತ್ಮಕ
ಸಂಶೋಧನೆ
ಸಂಶೋಧನೆಯ
ಸಂಸ್ಕೃತ
ಸಂಸ್ಕೃತಕನ್ನಡ
ಸಂಸ್ಕೃತದ
ಸಂಸ್ಕೃತಶಾಸನ
ಸಂಸ್ಕೃತಿ
ಸಂಸ್ಕೃತಿಯ
ಸಂಸ್ಕೃತಿಯು
ಸಂಸ್ಕೃತೀಕರಣದ
ಸಂಸ್ಥಂ
ಸಂಸ್ಥತೋನೃಪಃ
ಸಂಸ್ಥಾನ
ಸಂಸ್ಥಾನಕ್ಕೆ
ಸಂಸ್ಥಾನದ
ಸಂಸ್ಥಾನದಲ್ಲಿ
ಸಂಸ್ಥಾನದಲ್ಲಿದ್ದ
ಸಂಸ್ಥಾನದಲ್ಲಿರುವ
ಸಂಸ್ಥಾನವನ್ನು
ಸಂಸ್ಥಾಪಿತರಾದರು
ಸಂಸ್ಥೆಯ
ಸಂಸ್ಥೆಯಲ್ಲಿರುವ
ಸಂಸ್ಥೆಯಾಗಿತ್ತು
ಸಂಸ್ಥೆಯು
ಸಂಹರಿಸಿದ
ಸಂಹರಿಸಿದನೆಂದು
ಸಂಹರಿಸಿದನೆಂದೂ
ಸಂಹರಿಸುವಲ್ಲಿ
ಸಕಲ
ಸಕಲಧರ್ಮಗಳಂ
ಸಕಲರಾಜ್ಯಾಧಿಪತಿಗಳಾದ
ಸಕಲವಿದ್ಯಾನಿಧಿ
ಸಕಲವಿದ್ಯಾವಿಶಾರದರಾದ
ಸಕಲೇಶ್ವರ
ಸಕಳಕಳಾವಿಧಾನಪದ್ಮಾಸನ
ಸಕಳಧರ್ಮ್ಮೋದ್ಧಾರಕ
ಸಕಳಿಗವುಡನನ್ನು
ಸಕಳಿಗವುಡನು
ಸಕುಟುಂಬಸಮೇತನಾಗಿ
ಸಕ್ಕರೆ
ಸಕ್ಕರೆಪಟ್ಟಣದಿಂದ
ಸಕ್ಕರೆಯ
ಸಕ್ಕರೆಶೆಟ್ಟಿಯು
ಸಕ್ಕಿಯಾಗೆ
ಸಕ್ರಿಯವಾಗಿ
ಸಖ್ಯವನ್ನು
ಸಗರ
ಸಗರಕುಲ
ಸಗರಕುಲತಿಲಕನೆಂಬ
ಸಗರಕುಲದ
ಸಗರತ್ರಿಣೇತ್ರ
ಸಗರವಂಶ
ಸಗರವಂಶಜರು
ಸಗರವಂಶದ
ಸಗರವಂಶದವರಾಗಿದ್ದು
ಸಗರವಂಶದವರು
ಸಚಿವ
ಸಚಿವಂ
ಸಚಿವನಾಗಿ
ಸಚಿವನಿಗೆ
ಸಚಿವನೂ
ಸಚಿವಮಂತ್ರಿ
ಸಚಿವರು
ಸಚಿವಾಧೀಶ್ವರ
ಸಚಿವೋಭವತ್
ಸಜ್ಜನರಾದ
ಸಜ್ಜನರು
ಸಜ್ಜನಾಮೋದ
ಸಜ್ಜುಗೊಳಿಸಿ
ಸಡಿಲಿಸಿ
ಸಣಬ
ಸಣಬಿನಹಳ್ಳಿ
ಸಣಬಿಮುಕಳಿಯ
ಸಣ್ಣ
ಸಣ್ಣಕಟ್ಟೆಯಿದ್ದು
ಸಣ್ಣಪುಟ್ಟ
ಸಣ್ಣಸಣ್ಣ
ಸಣ್ಣೇನಹಳ್ಳಿ
ಸಣ್ನೆನಾಡನ್ನು
ಸಣ್ನೆನಾಡಿಗೆ
ಸಣ್ನೆನಾಡು
ಸತತ
ಸತತಂ
ಸತತವಾಗಿ
ಸತಾದೃಗ್ಗುಣ
ಸತಿ
ಸತೀಶ್
ಸತ್ಕರಿಸುತ್ತಿದ್ದಾಗ
ಸತ್ಕಾರ್ಯನಿರತವಾಗುವಂತೆ
ಸತ್ತ
ಸತ್ತನು
ಸತ್ತನೆಂದ
ಸತ್ತನೆಂದಿದೆ
ಸತ್ತನೆಂದು
ಸತ್ತರೆಂದು
ಸತ್ತವರ
ಸತ್ತಾಗ
ಸತ್ತಿಗನಹಳ್ಳದ
ಸತ್ತಿದ್ದಾನೆಂದು
ಸತ್ತಿರಬಹುದು
ಸತ್ತ್ಯದಲಿ
ಸತ್ಯಕೆ
ಸತ್ಯದ
ಸತ್ಯನಾರಾಯಣ
ಸತ್ಯಭಾಮಾ
ಸತ್ಯಮಂಗಲ
ಸತ್ಯರಾಧೇಯ
ಸತ್ಯರಾಧೇಯನುಂ
ಸತ್ಯವಾಕ್ಯ
ಸತ್ಯವಾಕ್ಯನ
ಸತ್ಯವಾಕ್ಯನು
ಸತ್ಯವಾಕ್ಯಪೆರ್ಮಾನಡಿಯ
ಸತ್ಯವಾಕ್ಯಪೆರ್ಮಾನಡಿಯಇಮ್ಮಡಿ
ಸತ್ಯವಾಕ್ಯಪೆರ್ಮಾನಡಿಯು
ಸತ್ರಾಧಿಕಾರಿಯಾಗಿದ್ದನೆಂದು
ಸದರಿ
ಸದಸಿವರಾಯರು
ಸದಸ್ಯರನ್ನು
ಸದಸ್ಯರೇ
ಸದಾಚಾರಿ
ಸದಾಶಿವದೇವ
ಸದಾಶಿವದೇವರಾಯನ
ಸದಾಶಿವದೇವರಾಯನಿಂದ
ಸದಾಶಿವನಿಗೆ
ಸದಾಶಿವಮಹಾರಾಯಕ್ಷಮಾನಾಯಕಃ
ಸದಾಶಿವಮಹಾರಾಯನ
ಸದಾಶಿವಮಹಾರಾಯನು
ಸದಾಶಿವಮಹಾರಾಯರು
ಸದಾಶಿವರ
ಸದಾಶಿವರಾಯ
ಸದಾಶಿವರಾಯನ
ಸದಾಶಿವರಾಯನನ್ನು
ಸದಾಶಿವರಾಯನಿಗೆ
ಸದಾಶಿವರಾಯನು
ಸದಾಶಿವರಾಯನೆಂದೂ
ಸದಾಶಿವರಾಯರಿಗೆ
ಸದಿಯು
ಸದುಗುಣಸಮೇತ
ಸದೆಬಡಿದು
ಸದ್ಗುಣ
ಸದ್ಗುಣದೊಳಧಿಕತೇಜಂ
ಸನುಮಂತ್ರಿಗಳೆನಿಸಿ
ಸನ್ತೊನಮಾಳೆ
ಸನ್ನಿಧಿಯಲಿ
ಸನ್ನಿಧಿಯಲ್ಲಿ
ಸನ್ನಿವೇಶದಲ್ಲಿ
ಸನ್ನೆಯನ್ನು
ಸನ್ಮಥದಿಂದ
ಸನ್ಮಾರ್ಗ
ಸನ್ಯಸನ
ಸನ್ಯಸನದಿಂದ
ಸನ್ಯಾಸಿಪುರ
ಸಪ್ತಮಭಾಗೆಯನು
ಸಪ್ತಸಾಗರ
ಸಪ್ತಾಂಗಲಕ್ಷ್ಮೀ
ಸಪ್ತಾಷ್ಟ
ಸಫಲನಾದನು
ಸಬಂಧವೇನೆಂದು
ಸಬ್ಡಿವಿಜನ್
ಸಬ್ಡಿವಿಜನ್ಗಳನ್ನು
ಸಬ್ಡಿವಿಜನ್ನ್ನು
ಸಬ್ತಾಲ್ಲೂಕು
ಸಬ್ಬಗೊಡುಗೆಯಾಗಿ
ಸಭೆ
ಸಭೆಯ
ಸಭೆಯಾಗಿತ್ತು
ಸಭೆಯಾಗಿರಬಹುದು
ಸಭೆಯೋಜನ
ಸಭೆಸೇರಿ
ಸಮ
ಸಮಕಾಲಿನ
ಸಮಕಾಲೀನ
ಸಮಕಾಲೀನನಾಗಿದ್ದ
ಸಮಕಾಲೀನನಾಗಿದ್ದನೆಂದು
ಸಮಕಾಲೀನನಾಗಿದ್ದು
ಸಮಕಾಲೀನರಾಗಿ
ಸಮಕಾಲೀನರು
ಸಮಕಾಲೀನರೆಂದು
ಸಮಕಾಲೀನವೂ
ಸಮಕ್ಷಮ
ಸಮಗ್ರ
ಸಮಗ್ರಬಲದ
ಸಮಗ್ರಬಲನಿಲಯೇ
ಸಮಗ್ರಬಲವನ್ನು
ಸಮಗ್ರವಾಗಿ
ಸಮಚಿತ್ತದಿಂದ
ಸಮತಟ್ಟು
ಸಮಧಿಗತ
ಸಮನಾದ
ಸಮನೆಂಬೊಂದು
ಸಮನ್ವಯತೆಯನ್ನು
ಸಮಯಗಳಲ್ಲಿಯೂ
ಸಮಯಜೈನಧರ್ಮ
ಸಮಯದಲ್ಲಿ
ಸಮಯದವರು
ಸಮರ
ಸಮರಗಳಲ್ಲಿ
ಸಮರಧಾರಧರಂ
ಸಮರಧುರೀಣರಾಗಿ
ಸಮರಮುಖಲಸದ್
ಸಮರಾಧಿತತ್ರಿವರ್ಗ್ಗ
ಸಮರ್ಥನಲ್ಲದ
ಸಮರ್ಥನೆ
ಸಮರ್ಥನೆಯನ್ನು
ಸಮರ್ಥವಾಗಿ
ಸಮರ್ಥಿಸಿದ್ದಾರೆ
ಸಮರ್ಥಿಸುತ್ತದೆ
ಸಮರ್ಥಿಸುತ್ತವೆ
ಸಮರ್ಥಿಸುತ್ತವೆಂದು
ಸಮರ್ಪಕವಾಗಿ
ಸಮರ್ಪಕವಾಗುತ್ತದೆ
ಸಮಸ್ತ
ಸಮಸ್ತಗವುಡುಗಳು
ಸಮಸ್ತಗುಣಸಂಪನ್ನ
ಸಮಸ್ತಗುಣಸಂಪನ್ನರುಮಪ್ಪ
ಸಮಸ್ತಭಾಗ್ಯೈಃ
ಸಮಸ್ತರಾಜ್ಯಭಾರ
ಸಮಸ್ತರು
ಸಮಸ್ಯೆಯನ್ನು
ಸಮಾಜಃ
ಸಮಾಜಕ್ಕೆ
ಸಮಾಜದ
ಸಮಾಜಶ್ಚಾಮರಾಜೇಂದ್ರ
ಸಮಾಧಿ
ಸಮಾಧಿಗಳ
ಸಮಾಧಿಗಳಿದ್ದು
ಸಮಾಧಿಗಳು
ಸಮಾಧಿಗಳುಳ್ಳ
ಸಮಾಧಿಗುಹೆ
ಸಮಾಧಿಮರಣವನ್ನಪ್ಪಿದನು
ಸಮಾಧಿಮರಣವನ್ನು
ಸಮಾಧಿಯ
ಸಮಾಧಿಯನ್ನು
ಸಮಾಧಿಯೆ
ಸಮಾನ
ಸಮಾನತೆ
ಸಮಾನನಾದ
ಸಮಾನರಾಗಿಯೂ
ಸಮಾನರೂಪ
ಸಮಾನಳಾಗಿದ್ದಳೆಂದು
ಸಮಾನವಾಗಿ
ಸಮಾನವಾಗಿಯೂ
ಸಮಾನವಾದ
ಸಮಾನವಾದುದೇ
ಸಮಾನಾರ್ಥಕಗಳು
ಸಮಾಪ್ತಿಗೊಳಿಸಿದನು
ಸಮಾಯಯೌ
ಸಮಾಲೋಚಕರಲ್ಲಿ
ಸಮಾಶ್ರಿತ
ಸಮಾಹೂಯ
ಸಮಿತಿಗೆ
ಸಮೀದಲ್ಲಿದೆ
ಸಮೀಪ
ಸಮೀಪದ
ಸಮೀಪದಲ್ಲಿ
ಸಮೀಪದಲ್ಲಿದ್ದ
ಸಮೀಪದಲ್ಲಿಯೇ
ಸಮೀಪದಲ್ಲಿರುವ
ಸಮೀಪದಲ್ಲೇ
ಸಮೀಪವಿರುವ
ಸಮುಂನ
ಸಮುದಾಯಗಳಿಗೆ
ಸಮುದಾಯದವರು
ಸಮುದ್ಧರಣ
ಸಮುದ್ಧರಣಂ
ಸಮುದ್ಧರಣನುಂ
ಸಮುದ್ಧರಣನೂ
ಸಮುದ್ಧರಣವನ್ನು
ಸಮುದ್ರ
ಸಮುದ್ರಪಾಂಡ್ಯ
ಸಮುದ್ರಾಧಿಪತಿ
ಸಮುದ್ರಾಧೀಶ್ವರ
ಸಮೂಹದ
ಸಮೂಹವನ್ನು
ಸಮೃದ್ಧ
ಸಮೃದ್ಧವಾಗಿತ್ತೆಂದು
ಸಮೃದ್ಧವಾಗುವುದಕ್ಕೆ
ಸಮೃದ್ಧಿಯನ್ನು
ಸಮೃದ್ಧಿಯನ್ನೂ
ಸಮೃದ್ಧಿಯಾಗಿದ್ದಿತೆಂದು
ಸಮೇತ
ಸಮೇತನಾಗಿ
ಸಮೇತರಾದ
ಸಮ್ಮಸ್ತ
ಸಮ್ಮುಖದಲ್ಲಿ
ಸಮ್ಯಕ್ತ್ವ
ಸಮ್ಯಕ್ತ್ವಚೂಡಾಮಣಿ
ಸಯಿಗೋಲಪಾರ್ತನುಂ
ಸಯ್ಯದ್
ಸರಗೂರ
ಸರಗೂರಿನ
ಸರಗೂರು
ಸರಣಾಗತ
ಸರದಾರರಾದ
ಸರಬರಾಜು
ಸರಸ್ವತಿ
ಸರಿ
ಸರಿಗಟ್ಟುವ
ಸರಿಮಕ್ಕನಹಳ್ಳಿ
ಸರಿಯಲ್ಲ
ಸರಿಯಾಗಿ
ಸರಿಯಾಗಿದೆ
ಸರಿಸಮಾನರಲ್ಲವೆಂದು
ಸರಿಸಮಾನರಾಗಿ
ಸರಿಸಮಾನವಾದ
ಸರಿಹೊಂದಿಸಿ
ಸರಿಹೊಂದುತ್ತದೆ
ಸರಿಹೊಂದುವ
ಸರೀರ
ಸರೀರಸಂಪತ್ತಿಗೆ
ಸರೋವರ
ಸರೋವರವನ್ನು
ಸರ್
ಸರ್ಇಅಸ್ಕರ್
ಸರ್ಕಾರಕ್ಕೆ
ಸರ್ಕಾರದ
ಸರ್ಕಾರದಲ್ಲಿ
ಸರ್ಕಾರೆ
ಸರ್ವ
ಸರ್ವಜ್ಞ
ಸರ್ವಜ್ಞನಪರಿಮಿತದಾನವಿನೋದಶೀಳ
ಸರ್ವಜ್ಞಪುರವೆಂಬ
ಸರ್ವಜ್ಞವಿಷ್ಣುಭಟ್ಟಯ್ಯನು
ಸರ್ವಧರ್ಮ
ಸರ್ವನಮಸ್ಯವಾಗಿ
ಸರ್ವಬಾಧಾಪರಿಹಾರವಾಗಿ
ಸರ್ವಭೂತಾನುಕಂಪಿನಃ
ಸರ್ವಮಾನ್ಯ
ಸರ್ವಮಾನ್ಯವಾಗಿ
ಸರ್ವಮಾನ್ಯವಾದ
ಸರ್ವವಿದ್ಯಾವಿಚಕ್ಷಣನು
ಸರ್ವವಿದ್ಯಾಸುವೈಚಕ್ಷಣ್ಯಂ
ಸರ್ವಸ್ವವನ್ನೂ
ಸರ್ವಸ್ವಾಧೀನವಾದರೆ
ಸರ್ವಾಧಿಕಾರಿ
ಸರ್ವಾಧಿಕಾರಿಗಳು
ಸರ್ವಾಧಿಕಾರಿಗಳೂ
ಸರ್ವಾಧಿಕಾರಿಯ
ಸರ್ವಾಧಿಕಾರಿಯಾಗಿ
ಸರ್ವಾಧಿಕಾರಿಯಾಗಿದ್ದ
ಸರ್ವಾಧಿಕಾರಿಯಾದನು
ಸರ್ವಾಧಿಕಾರಿಯಾದರೂ
ಸರ್ವಾಧಿಕಾರಿಯೂ
ಸರ್ವಾಧ್ಯಕ್ಷನಾಗಿದ್ದನೆಂದು
ಸರ್ವಾಧ್ಯಕ್ಷನೂ
ಸರ್ವಾಧ್ಯಕ್ಷನೆಂದು
ಸರ್ವೋತ್ತಮ
ಸರ್ವೋರ್ವೀಶನತಃ
ಸರ್ವ್ವಧರ್ಮ್ಮರಹಸ್ಯಸ್ಯ
ಸರ್ವ್ವಧಾರ್ಯಾಹ್ವಯೇ
ಸಲ
ಸಲಕರಾಜು
ಸಲಗೆ
ಸಲಾಕೆಯನ್ನು
ಸಲಾಕೆಯಿಂದ
ಸಲಾಕೆಯಿಂದಲೇ
ಸಲಾಕೆಯೊಂದನ್ನು
ಸಲಿಸುತ್ತಮಿರೆ
ಸಲುವ
ಸಲೆ
ಸಲ್ಯದಚಲ್ಯ
ಸಲ್ಲಿಸಬೇಕಾಗುತ್ತಿತ್ತು
ಸಲ್ಲಿಸಿ
ಸಲ್ಲಿಸಿದ
ಸಲ್ಲಿಸಿದನು
ಸಲ್ಲಿಸಿದ್ದಾರೆ
ಸಲ್ಲಿಸುತ್ತಿದ್ದರು
ಸಲ್ಲಿಸುತ್ತಿದ್ದರೆಂದು
ಸಲ್ಲಿಸುವ
ಸಲ್ಲುತ್ತಿತ್ತೆಂದು
ಸಲ್ಲುತ್ತಿದ್ದ
ಸಲ್ಲುವ
ಸಲ್ಲುವಂತೆ
ಸಲ್ಲುವುದೆಂದು
ಸಲ್ಲೇಖನ
ಸಳ
ಸಳನ
ಸಳನನ್ನು
ಸಳನಿಗೆ
ಸಳನು
ಸಳನೆಂದು
ಸಳನೆಂಬ
ಸಳಪಯ್ಯ
ಸಳಿಗವುಡ
ಸವಿಲಾಸಮಾಸ
ಸಶ್ಯಾಲಪುರದ
ಸಶ್ಯಾಲಪುರವನ್ನುಯ
ಸಸಿಯಾಲದ
ಸಸಿಯಾಲದಪುರ
ಸಸಿಯಾಲದಪುರವನ್ನು
ಸಸಿಯಾಲಪುರ
ಸಸಿಯಾಲಪುರಕ್ಕೆ
ಸಸಿರರ್ಬ್ಬಲ್ಲವರೆಮ್ಮೞ್ದಕ್ಕೆ
ಸಸ್ಯ
ಸಹ
ಸಹಕರಿಸಿದರು
ಸಹಕಾರದಿಂದ
ಸಹಜ
ಸಹಜವಾಗಿದೆ
ಸಹಸ್ರ
ಸಹಸ್ರಕಿಳಲೆನಾಡುಕೆಳಲೆನಾಡು
ಸಹಸ್ರಗಳು
ಸಹಸ್ರದೊಳಗೆ
ಸಹಸ್ರನಾಮಗಳನ್ನು
ಸಹಸ್ರಬಾಹು
ಸಹಸ್ರವಿಷಯ
ಸಹಾಯ
ಸಹಾಯಕ
ಸಹಾಯಕನಾಗಿ
ಸಹಾಯಕನಾಗಿದ್ದನು
ಸಹಾಯಕರಾಗಿ
ಸಹಾಯಕರಾಗಿದ್ದರೆಂದು
ಸಹಾಯಕರಾಗಿದ್ದು
ಸಹಾಯಕ್ಕೆ
ಸಹಾಯದಿಂದ
ಸಹಿತ
ಸಹಿತಂ
ಸಹಿತಃ
ಸಹಿತವಾಗಿ
ಸಹೋದರ
ಸಹೋದರನಾಗಿರಬಹುದು
ಸಹೋದರನಾದ
ಸಹೋದರರ
ಸಹೋದರರಾಗಿದ್ದರೆಂಬುದು
ಸಹೋದರರಾಗಿರಬಹುದು
ಸಹೋದರರಾದ
ಸಹೋದರರಿಗೂ
ಸಹೋದರರು
ಸಹೋದರಿಗೆ
ಸಾ
ಸಾಂಕೇತಿಸುತ್ತಿದೆ
ಸಾಂಗತ್ಯದಲ್ಲಿ
ಸಾಂತತ್ಯ
ಸಾಂತಿಯಕ್ಕ
ಸಾಂದರ್ಭಿಕವಾಗಿ
ಸಾಂದರ್ಭೋಚಿತವಾಗಿ
ಸಾಂಪ್ಪನಹಳ್ಳಿ
ಸಾಂಪ್ರದಾಯಕ
ಸಾಂಪ್ರದಾಯಕವಾಗಿ
ಸಾಂಪ್ರದಾಯಿಕವಾಗಿ
ಸಾಂಬ್ರಾಜ್ಯಂಗಯಿಉತ
ಸಾಂಬ್ರಾಜ್ಯಂಗೈಯುತ್ತಿರಲು
ಸಾಂಸ್ಕೃತಿ
ಸಾಂಸ್ಕೃತಿಕ
ಸಾಂಸ್ಕೃತಿಕವಾಗಿ
ಸಾಕಲ್ಯವಾಗಿ
ಸಾಕಷ್ಟು
ಸಾಕ್ಷಾತ್
ಸಾಕ್ಷಿ
ಸಾಕ್ಷಿಗಳಾಗಿ
ಸಾಕ್ಷಿಗಳಾಗಿದ್ದಾರೆ
ಸಾಕ್ಷಿಗಳಾಗಿರುತ್ತಾರೆ
ಸಾಕ್ಷಿಣಃ
ಸಾಕ್ಷಿಯಾಗಿ
ಸಾಕ್ಷಿಯಾಗಿದೆ
ಸಾಕ್ಷಿಯಾಗಿದ್ದನೆಂದು
ಸಾಕ್ಷಿಯಾಗಿದ್ದರೆಂದು
ಸಾಕ್ಷಿಯಾಗಿರಲು
ಸಾಕ್ಷಿಯಾಗಿರುತ್ತಾನೆ
ಸಾಕ್ಷಿಯಾಗಿರುತ್ತಾರೆ
ಸಾಕ್ಷಿಯಾಗಿರುವುದು
ಸಾಕ್ಷಿಯಾಗಿವೆ
ಸಾಗಿಸುತ್ತಿದ್ದನು
ಸಾಡತಿಕಾತಿ
ಸಾಣೆಹಳ್ಳಿ
ಸಾಣೆಹಳ್ಳಿಯ
ಸಾತನೂರು
ಸಾತನೂರುಗಳು
ಸಾತಿಗ್ರಾಮ
ಸಾತಿಸೆಟ್ಟಿ
ಸಾದಿಪ್ಪ
ಸಾದಿಯಪ್ಪ
ಸಾದಿಯಪ್ಪನ
ಸಾದಿಯಪ್ಪನಿಗೆ
ಸಾದುಗೊಂಡನಹಳ್ಳಿ
ಸಾದುಪುರ
ಸಾದೊಳಲು
ಸಾಧನಗಳು
ಸಾಧನೆ
ಸಾಧನೆಗಳ
ಸಾಧನೆಗಳನ್ನು
ಸಾಧನೆಗಳಲ್ಲಿ
ಸಾಧನೆಗಳು
ಸಾಧನೆಯ
ಸಾಧಾರ
ಸಾಧಾರವಾಗಿ
ಸಾಧಿಸಲು
ಸಾಧಿಸಿ
ಸಾಧಿಸಿಕೊಟ್ಟನೆಂದು
ಸಾಧಿಸಿದ
ಸಾಧಿಸಿದರಾರ್ಪ್ಪಾಂಡ್ಯೇಶನಂ
ಸಾಧಿಸಿದರು
ಸಾಧಿಸಿವೆ
ಸಾಧ್ಯ
ಸಾಧ್ಯತೆ
ಸಾಧ್ಯತೆಯೂ
ಸಾಧ್ಯವಾಗುತ್ತದೆ
ಸಾಧ್ಯವಾದ
ಸಾಧ್ಯವಿಲ್ಲ
ಸಾಧ್ಯವೇ
ಸಾಮಂತ
ಸಾಮಂತಂ
ಸಾಮಂತದೇವ
ಸಾಮಂತನ
ಸಾಮಂತನನ್ನಾಗಿ
ಸಾಮಂತನಾಗಿ
ಸಾಮಂತನಾಗಿದ್ದ
ಸಾಮಂತನಾಗಿದ್ದನು
ಸಾಮಂತನಾಗಿದ್ದನೆಂದು
ಸಾಮಂತನಾಗಿದ್ದು
ಸಾಮಂತನಾಗಿರಬಹುದು
ಸಾಮಂತನಾದ
ಸಾಮಂತನಾದರೂ
ಸಾಮಂತನಿರಬಹುದು
ಸಾಮಂತನು
ಸಾಮಂತನೆಂದರೆ
ಸಾಮಂತನೆಂದು
ಸಾಮಂತನೆಂದೂ
ಸಾಮಂತನೆನಿಸಿದ
ಸಾಮಂತನೋ
ಸಾಮಂತಪದವಿಗೇರಿರುವುದು
ಸಾಮಂತಪದವಿಯನ್ನು
ಸಾಮಂತರ
ಸಾಮಂತರಗಂಡ
ಸಾಮಂತರನ್ನಾಗಿ
ಸಾಮಂತರನ್ನು
ಸಾಮಂತರಾಗಿ
ಸಾಮಂತರಾಗಿದ್ದ
ಸಾಮಂತರಾಗಿದ್ದರು
ಸಾಮಂತರಾಗಿದ್ದು
ಸಾಮಂತರಾಗಿರಬಹುದು
ಸಾಮಂತರಾಜನೆಂದೂ
ಸಾಮಂತರಾದ
ಸಾಮಂತರಿಗೆ
ಸಾಮಂತರು
ಸಾಮಂತರುಗಳು
ಸಾಮಂತರೂ
ಸಾಮಂತರೆಲ್ಲರಂ
ಸಾಮಂತರೊಡಗೂಡಿದ
ಸಾಮಂತರೋ
ಸಾಮಂತಸೋಮನ
ಸಾಮಂತಸೋಮನು
ಸಾಮಂತಾಧಿಪತಿ
ಸಾಮಂತಾಧಿಪತಿಯಾಗಿದ್ದನೆಂದು
ಸಾಮಗ್ರಿಯನ್ನಾಗಿ
ಸಾಮನ್ತ
ಸಾಮಮತರು
ಸಾಮರ್ಥ್ಯಗಳಿಗನುಗುಣವಾಗಿ
ಸಾಮರ್ಥ್ಯದಿಂದ
ಸಾಮಾಜಿ
ಸಾಮಾಜಿಕ
ಸಾಮಾಜ್ಯದ
ಸಾಮಾನ್ಯ
ಸಾಮಾನ್ಯವಾಗಿ
ಸಾಮಾನ್ಯವಾಗಿತ್ತು
ಸಾಮಿಸಂಕಡಿ
ಸಾಮ್ಯ
ಸಾಮ್ಯದಿಂದ
ಸಾಮ್ಯವನು
ಸಾಮ್ರಾಜ್ಯ
ಸಾಮ್ರಾಜ್ಯಂಗೈಯುತ್ತಿರುವಲ್ಲಿ
ಸಾಮ್ರಾಜ್ಯಕ್ಕೆ
ಸಾಮ್ರಾಜ್ಯಗಳಿಗೆ
ಸಾಮ್ರಾಜ್ಯದ
ಸಾಮ್ರಾಜ್ಯದಲ್ಲಿ
ಸಾಮ್ರಾಜ್ಯದಿಂದ
ಸಾಮ್ರಾಜ್ಯಮಮ್
ಸಾಮ್ರಾಜ್ಯರಮಾಮಣಿಯ
ಸಾಮ್ರಾಜ್ಯವನ್ನಾಗಿ
ಸಾಮ್ರಾಜ್ಯವನ್ನು
ಸಾಮ್ರಾಜ್ಯವನ್ನೂ
ಸಾಮ್ರಾಜ್ಯವು
ಸಾಮ್ರಾಟರಲ್ಲಿಯೇ
ಸಾಮ್ರಾಟರಾದ
ಸಾಯಣ್ಣ
ಸಾಯಣ್ಣನು
ಸಾಯಿರ
ಸಾಯಿರವೂ
ಸಾಯುತ್ತಾನೆ
ಸಾರಂಗಿ
ಸಾರಿದರೆಂದು
ಸಾರಿದವನು
ಸಾರೆಯ್ಕ
ಸಾರ್ದ್ದುದಾ
ಸಾರ್ವಭೌಮ
ಸಾಲಗಾವುಂಡ
ಸಾಲಗಾವುಂಡನು
ಸಾಲನ್ನೂ
ಸಾಲಮಂನೆಯ
ಸಾಲಿಗ್ರಾಮ
ಸಾಲಿಗ್ರಾಮದ
ಸಾಲಿಗ್ರಾಮವಾಗಿದೆ
ಸಾಲಿದೆ
ಸಾಲು
ಸಾಲುಗಳ
ಸಾಲುಗಳಲ್ಲಿ
ಸಾಲುಗಳಿವೆ
ಸಾಲುಗಳು
ಸಾಲೂರುಮಠದ
ಸಾಲೆ
ಸಾಳ
ಸಾಳುವ
ಸಾಳುವಗಜಸಿಂಹ
ಸಾಳುವತಿಕ್ಕಮನ
ಸಾಳುವನ
ಸಾಳುವನರಸಿಂಗನ
ಸಾಳುವನರಸಿಂಗನು
ಸಾಳ್ವ
ಸಾವಂತ
ಸಾವಂತಗೌಡ
ಸಾವಂತನ
ಸಾವಂತನು
ಸಾವಕಾಶವಾಗಿ
ಸಾವನ್ತ
ಸಾವಿನ
ಸಾವಿಮಲೆ
ಸಾವಿಯಣ್ಣ
ಸಾವಿಯಬ್ಬೇಶ್ವರಕ್ಕೆ
ಸಾವಿರ
ಸಾವಿರಕೊಳಗ
ಸಾವಿರಕ್ಕೂ
ಸಾವಿರವನ್ನು
ಸಾವು
ಸಾವುಕಗವುಡ
ಸಾವೆಗಿರಿಯವರೆಗೆ
ಸಾಶನಮ
ಸಾಸನದ
ಸಾಸಲಿನ
ಸಾಸಲು
ಸಾಸಿರ
ಸಾಸಿರಕಬ್ಬಹು
ಸಾಸಿರಕಳ್ವಪ್ಪು
ಸಾಸಿರಕ್ಕೆ
ಸಾಸಿರದ
ಸಾಸಿರದಲ್ಲಿ
ಸಾಸಿರದೊಳಗೆ
ಸಾಸಿರದೊಳು
ಸಾಸ್ತ್ರವಿನೋದನುಂ
ಸಾಹಣಿರು
ಸಾಹಳ್ಳಿ
ಸಾಹಸ
ಸಾಹಸಕ್ಕಾಗಿ
ಸಾಹಸಗಳನ್ನು
ಸಾಹಸಗಳಿಂದ
ಸಾಹಸವನ್ನು
ಸಾಹಸಿಯಾದ
ಸಾಹಿತ್ಯ
ಸಾಹಿತ್ಯದ
ಸಾಹಿತ್ಯಿಕ
ಸಾಹಿರ್
ಸಾಹೇಬ್
ಸಿ
ಸಿಂ
ಸಿಂಗಟಗೆರೆಯ
ಸಿಂಗಡಿ
ಸಿಂಗಣಾಖ್ಯ
ಸಿಂಗಣ್ಣ
ಸಿಂಗದಂ
ಸಿಂಗನಪಳ್ಳಿ
ಸಿಂಗನಹಳ್ಳಿ
ಸಿಂಗಪೆರುಮಾಳೆ
ಸಿಂಗಪೆರುಮಾಳ್
ಸಿಂಗಪೆರುಮಾಳ್ಗೆ
ಸಿಂಗಪ್ಪ
ಸಿಂಗಪ್ಪನಾಯಕನು
ಸಿಂಗಮ್ಮ
ಸಿಂಗಯ್ಯ
ಸಿಂಗಯ್ಯನು
ಸಿಂಗರಾಜಯ್ಯನೆಂದೂ
ಸಿಂಗರಾರ್ಯನನ್ನು
ಸಿಂಗಾಡಿದೇವನ
ಸಿಂಗಾಡಿದೇವನೆಂದು
ಸಿಂಗಾಡಿದೇವನೆಂಬ
ಸಿಂಗಾಡಿನಾಯಕ
ಸಿಂಗಾರಾರ್ಯನ
ಸಿಂಗಾರಾರ್ಯನು
ಸಿಂಗಾರಾರ್ಯರಿಂದ
ಸಿಂಗೆಯ
ಸಿಂಗೆಯದಂಡನಾಯಕನ
ಸಿಂಗೆಯದಂಡನಾಯಕನು
ಸಿಂಗೆಯದಂಡನಾಯಕನೂ
ಸಿಂಗೆಯದಂಡನಾಯಕರುಗಳು
ಸಿಂಗೆಯದಂಣ್ನಾಯಕರು
ಸಿಂಗೆಯನಾಯಕ
ಸಿಂಘಣನ
ಸಿಂದಗೆರೆ
ಸಿಂದಘಟ್ಟ
ಸಿಂದಘಟ್ಟಕ್ಕೆ
ಸಿಂದಘಟ್ಟದ
ಸಿಂದಘಟ್ಟವನ್ನು
ಸಿಂದಘಟ್ಟವು
ಸಿಂದಘಟ್ಟಸೀಮೆಯ
ಸಿಂದುಘಟ್ಟ
ಸಿಂಧ
ಸಿಂಧಗಿರಿಸಿಂಧಗೆರೆ
ಸಿಂಧಗೆರೆ
ಸಿಂಧಗೆರೆಯ
ಸಿಂಧಗೆರೆಯನ್ನು
ಸಿಂಧಗೋವಿಂದ
ಸಿಂಧಘಟ್ಟ
ಸಿಂಧಘಟ್ಟದ
ಸಿಂಧಘಟ್ಟಸ್ಯ
ಸಿಂಧರಸರು
ಸಿಂಧು
ಸಿಂಧುಃ
ಸಿಂಧುಗೋವಿಂದ
ಸಿಂಧುರದ
ಸಿಂಧುರರಾಜಗಭೀರಧೀಃ
ಸಿಂಧೆಯ
ಸಿಂಧೆಯನಾಯಕ
ಸಿಂಧೆಯನಾಯಕನ
ಸಿಂಧೆಯನಾಯಕನಿಗೆ
ಸಿಂಧೇಶ್ವರ
ಸಿಂಫ್ತ್
ಸಿಂಫ್ತ್ಗಳಿಗೆ
ಸಿಂಫ್ತ್ದಾರರಿದ್ದು
ಸಿಂಹ
ಸಿಂಹಗಳ
ಸಿಂಹದಂತೆ
ಸಿಂಹನೆನಿಸಿದ್ದವನೂ
ಸಿಂಹಪೋತ
ಸಿಂಹಪೋತಕಲಿನೊಳಂಬಾದಿರಾಜನು
ಸಿಂಹಪೋತನು
ಸಿಂಹಪ್ರಾಯನೆಂದು
ಸಿಂಹಳದೇವಿಯ
ಸಿಂಹಳದೇವಿಯರ
ಸಿಂಹಾಸಕ್ಕೆ
ಸಿಂಹಾಸನ
ಸಿಂಹಾಸನಕೆ
ಸಿಂಹಾಸನಕ್ಕಾಗಿ
ಸಿಂಹಾಸನಕ್ಕಿದ್ದ
ಸಿಂಹಾಸನಕ್ಕೆ
ಸಿಂಹಾಸನದ
ಸಿಂಹಾಸನದಲ್ಲಿ
ಸಿಂಹಾಸನದಿಂದ
ಸಿಂಹಾಸನದು
ಸಿಂಹಾಸನವನ್ನು
ಸಿಂಹಾಸನವನ್ನೇರಿದನು
ಸಿಂಹಾಸನವನ್ನೇರಿದ್ದು
ಸಿಂಹಾಸನವು
ಸಿಂಹಾಸನಾಧಿಪತಿಗಳಲ್ಲಿ
ಸಿಂಹಾಸನಾಧೀಶ್ವರ
ಸಿಂಹಾಸನಾಧೀಶ್ವರರಾಗಿ
ಸಿಂಹಾಸನಾರೂಢರಾಗಿ
ಸಿಂಹಾಸನಾರೋಹಣ
ಸಿಂಹಾಸನಾಸೀನನಾಗಿ
ಸಿಂಹಾಸನೋಚಿತ
ಸಿಂಹಾಸಾನೋರಹಣಕ್ಕೆ
ಸಿಕ್ಕಿತು
ಸಿಕ್ಕಿವೆ
ಸಿಗುತ್ತವೆ
ಸಿಗುತ್ತಿದ್ದ
ಸಿಗುವ
ಸಿಗುವುದಿಲ್ಲ
ಸಿಡಿದೇಳಲು
ಸಿಡಿಲಂತೆ
ಸಿಡಿಲುಕಲ್ಲು
ಸಿಡುಬುರೋಗಕ್ಕೆ
ಸಿತಕರಗಂಡ
ಸಿತಗರ
ಸಿದ್ದೇಶ್ವರ
ಸಿದ್ಧನಾಥದೇವರಿಗೆ
ಸಿದ್ಧನಾದನು
ಸಿದ್ಧಪಡಿಸಲಾಗಿದೆ
ಸಿದ್ಧಪಡಿಸಿ
ಸಿದ್ಧಪಡಿಸಿದ್ದಕ್ಕಾಗಿ
ಸಿದ್ಧಪಡಿಸುವವನೆಂದೂ
ಸಿದ್ಧಪುರುಷರುಗಳು
ಸಿದ್ಧಪ್ಪಾಜೀ
ಸಿದ್ಧಯ್ಯಗವುಡ
ಸಿದ್ಧರ್ದಪ್ಪಣ್ಣ
ಸಿದ್ಧಲದೇವಿಯ
ಸಿದ್ಧವಾಗಿ
ಸಿದ್ಧಾಂತ
ಸಿದ್ಧಾಂತದೇವರ
ಸಿದ್ಧಾನ್ತದೇವರ
ಸಿದ್ಧಾಯ
ಸಿದ್ಧಾಯದಿಂದ
ಸಿದ್ಧಾಯವನ್ನು
ಸಿದ್ಧಿಗಳನ್ನು
ಸಿದ್ಧಿಸಿರಲಾರದು
ಸಿನ್ನಿವರ
ಸಿಬ್ಬಾಲ್
ಸಿರಕುಬಳ್ಳಿ
ಸಿರಗುಪ್ಪಿ
ಸಿರಿ
ಸಿರಿಗ
ಸಿರಿಯ
ಸಿರಿಯಕಲಸತ್ತುಪಾಡಿಯಾದ
ಸಿರಿಯಣ್ಣನನ್ನು
ಸಿರಿರಂಗನಾಯಕ
ಸಿರಿರಂಗನಾಯಕನ
ಸಿರಿರಂಗಪುರದ
ಸಿರೋಂಬುಜಮಂ
ಸಿವದೇವಂ
ಸಿವನೆನಾಯಕನ
ಸಿವನೆಯನಾಯಕ
ಸಿವಮಯ್ಯಗವುಡ
ಸಿವರಮ್ಯಗೇಹವನ್ನುಶಿವಾಲಯ
ಸಿವಾಲಯಗಳನ್ನು
ಸಿವೋಜಿನಾಯಕನು
ಸೀಗೆ
ಸೀಗೆಯನಾಡು
ಸೀತಾಂಬಿಕಾ
ಸೀತಾಪುರ
ಸೀತಾಪುರವೆಂಬ
ಸೀತಾಯಂಮನವರ
ಸೀತಾಯಂಮವನರ
ಸೀತಾರಾಮಜಾಗಿರ್ದಾರ್
ಸೀನಣ್ಣನು
ಸೀಪನಮರಡಿ
ಸೀಮಾಂತರದಲ್ಲಿ
ಸೀಮಾಂತವರ್ತಿನ
ಸೀಮಾರೇಖೆಗಳೂ
ಸೀಮಾಸಂಬಂಧಿ
ಸೀಮಾಸಹಿತವಾಗಿ
ಸೀಮಿತವಾಗಿ
ಸೀಮಿತವಾಗಿದ್ದರೆಂದು
ಸೀಮಿತವಾಗಿವೆ
ಸೀಮೆ
ಸೀಮೆಗಳನ್ನು
ಸೀಮೆಗಳು
ಸೀಮೆಗೆ
ಸೀಮೆನಾಡು
ಸೀಮೆಮೈಸೂರು
ಸೀಮೆಯ
ಸೀಮೆಯನ್ನು
ಸೀಮೆಯನ್ನುಇಂದಿನ
ಸೀಮೆಯಲ್ಲಿ
ಸೀಮೆಯಲ್ಲಿದ್ದ
ಸೀಮೆಯವನಿರಬಹುದೆಂದು
ಸೀಮೆಯಾಗಿ
ಸೀಮೆಯಾಗಿತ್ತು
ಸೀಮೆಯಾಗಿದ್ದ
ಸೀಮೆಯಾಗಿದ್ದವು
ಸೀಮೆಯಿಂ
ಸೀಮೆಯು
ಸೀಮೆಯೇ
ಸೀಮೆಯೊಳಗಣ
ಸೀಮೆಯೊಳಗಿನ
ಸೀಮೆಯೊಳಗೆ
ಸೀಮೆಸ್ಥಳ
ಸೀಯಕನು
ಸೀರೆ
ಸೀರೇಹಳ್ಳಿ
ಸೀರ್ಯದ
ಸೀಳನೆರೆ
ಸೀಳಿದನೆಂದು
ಸು
ಸುಂಕ
ಸುಂಕಕ್ಕೆ
ಸುಂಕಗಳ
ಸುಂಕಗಳನ್ನು
ಸುಂಕಗಳನ್ನೂ
ಸುಂಕಗಳಲ್ಲಿ
ಸುಂಕತೆರಿಗೆಗಳನ್ನು
ಸುಂಕತೆರಿಗೆಯನ್ನು
ಸುಂಕದ
ಸುಂಕದವರು
ಸುಂಕದವರೂ
ಸುಂಕದಹೆಗ್ಗಡೆ
ಸುಂಕದಹೆಗ್ಗಡೆಗಳ
ಸುಂಕದಹೆಗ್ಗಡೆಗಳು
ಸುಂಕದೊಳಗೆ
ಸುಂಕವನ್ನು
ಸುಂಕವನ್ನೂ
ಸುಂಕವಿತ್ತೆಂದು
ಸುಂಕವೆಂಬ
ಸುಂಕಾತೊಂಡನೂರಿನ
ಸುಂಕಾತೊಂಡನೂರಿನಲ್ಲಿ
ಸುಂಕಾತೊಂಡನೂರು
ಸುಂದರ
ಸುಂದರಪಾಂಡ್ಯನ
ಸುಂದರಪಾಂಡ್ಯನಿಗೆ
ಸುಂದರವಾಗಿದೆ
ಸುಂದರವಾದ
ಸುಂದರವೂ
ಸುಂದರೀ
ಸುಕ
ಸುಕದಿಂ
ಸುಕದೋರಾಇಂದಿನ
ಸುಕನಾಸಿ
ಸುಕವಿಜನ
ಸುಕಸಂಕಥಾವಿನೋದದಿಂ
ಸುಕ್ಕುಧರೆ
ಸುಖದಿಂ
ಸುಖದಿನರಸುಗೆಯ್ಯುತ್ತಮಿರೆ
ಸುಖರಾಜ್ಯಂಗೈಯುತ್ತಿದ್ದರು
ಸುಖರಾಜ್ಯಗೆಯುತ್ತಿದ್ದರೆಂದು
ಸುಖಸಂಕಥಾ
ಸುಖಸಂಕಥಾವಿನೋದದಿಂ
ಸುಖೀಭವ
ಸುಗಧರೆ
ಸುಗ್ಗಗವುಂಡನ
ಸುಗ್ಗಗೌಂಡ
ಸುಗ್ಗಲದೇವಿ
ಸುಚಿಸುಧನ್ವ
ಸುಜ್ಜಲೂರಿನ
ಸುಜ್ಜಲೂರು
ಸುಣ್ಣದಲ್ಲಿ
ಸುಣ್ಣಬಣ್ಣವನ್ನು
ಸುತ
ಸುತಂ
ಸುತರು
ಸುತೆ
ಸುತ್ತಣ
ಸುತ್ತಮುತ್ತ
ಸುತ್ತಮುತ್ತಲ
ಸುತ್ತಮುತ್ತಲಿನ
ಸುತ್ತಮುತ್ತಲೂ
ಸುತ್ತಲಿನ
ಸುತ್ತಲೂ
ಸುತ್ತಾಲಯವನ್ನು
ಸುತ್ತಿನ
ಸುತ್ತುವರೆದನು
ಸುತ್ತೂರು
ಸುದ
ಸುದತ್ತಾಚಾರ್ಯನೆಂದೇ
ಸುದತ್ತಾಚಾರ್ಯರ
ಸುಧರ್ಮ
ಸುಧರ್ಮಪುರವಾಗಿ
ಸುಧಾಂಶುರಿವ
ಸುಧಾಕರಂ
ಸುಧಾಕರರುಂ
ಸುಧಾನಿಧೇಃ
ಸುಧಾರಿತ
ಸುನಾರ್ಖಾನೆ
ಸುನಾರ್ಖಾನೆಯಒಡವೆ
ಸುಪತ್ರ
ಸುಪುತ್ರನಪ್ಪ
ಸುಪ್ರತಿಷ್ಠೆಯಂ
ಸುಪ್ರತೀಕ
ಸುಪ್ರತೀಕಗಜದ
ಸುಪ್ರಸಿದ್ಧ
ಸುಪ್ರಸಿದ್ಧನಾದ
ಸುಬ್ಬಯ್ಯನ
ಸುಬ್ಬರಾಯ
ಸುಬ್ಬರಾಯನ
ಸುಬ್ಬರಾಯನಕೊಪ್ಪಲು
ಸುಬ್ರಹ್ಮಣ್ಯ
ಸುಭಟರ
ಸುಭಟರನ್ನು
ಸುಭಟರಾದಿತ್ಯನುಂ
ಸುಮಾರಿಗೆ
ಸುಮಾರಿನಲ್ಲಿ
ಸುಮಾರು
ಸುಮಿತ್ರೆಯರಂತೆ
ಸುಮ್ಮನೆ
ಸುರಕ್ಷಿತವಾಗಿ
ಸುರಗಿ
ಸುರಗಿಯ
ಸುರತರುಸ್ಪರ್ಧಾಳುವಿಶ್ರಾಣನಃ
ಸುರತ್ರಾಣ
ಸುರಪುರ
ಸುರರಾಜಪೂಜ್ಯ
ಸುರಲೋಕ
ಸುರಲೋಕಂ
ಸುರಲೋಕಪ್ರಾಪ್ತ
ಸುರಲೋಕಪ್ರಾಪ್ತನಾಗುತ್ತಾನೆ
ಸುರಲೋಕಪ್ರಾಪ್ತನಾದನು
ಸುರಲೋಕಪ್ರಾಪ್ತನಾದನೆಂದು
ಸುರಿಗೆ
ಸುರಿಗೆಕಾರ
ಸುರಿಗೆನಾಗಯ್ಯನ
ಸುರಿಗೆಯ
ಸುರಿಗೆಯಿಂದ
ಸುರಿಗೆವಿಡಿವ
ಸುರಿತಾಣಭೂಪರಂ
ಸುರಿದ
ಸುರುಚಿರ
ಸುರೇಂದ್ರತೀರ್ಥ
ಸುಲಭವಾಗಿ
ಸುಲ್ತಾನನ
ಸುಲ್ತಾನನು
ಸುಲ್ತಾನ್
ಸುವರ್ಣ
ಸುವರ್ಣದಾನಶೂರ
ಸುವ್ಯವಸ್ಥೆಗೆ
ಸುಹೃತ್ವ
ಸೂಕ್ತವಲ್ಲವೆಂದು
ಸೂಕ್ತವಾಗಿದೆ
ಸೂಕ್ತವಾಗುತ್ತದೆ
ಸೂಕ್ತಿಸುಧಾರ್ಣವದ
ಸೂಕ್ರವಾಗಿದೆ
ಸೂಕ್ಷ್ಮವಾಗಿ
ಸೂಗುರು
ಸೂಗೂರಿನ
ಸೂಚನೆ
ಸೂಚನೆಗಳಿವೆ
ಸೂಚನೆಗಾಗಿ
ಸೂಚಿ
ಸೂಚಿಗಳಾದರೆ
ಸೂಚಿತ
ಸೂಚಿತವಾಗಿರುವ
ಸೂಚಿಸಬಹುದು
ಸೂಚಿಸಲು
ಸೂಚಿಸುತ್ತದ
ಸೂಚಿಸುತ್ತದೆ
ಸೂಚಿಸುತ್ತವೆ
ಸೂಚಿಸುತ್ತವೆಂದು
ಸೂಚಿಸುತ್ತವೆಯೆಂದು
ಸೂಚಿಸುತ್ತಿದೆ
ಸೂಚಿಸುತ್ತಿದ್ದು
ಸೂಚಿಸುವ
ಸೂತ
ಸೂತಕುಲದ
ಸೂತ್ರಗಳನ್ನು
ಸೂತ್ರಗುತ್ತಗೆಯಾಗಿ
ಸೂತ್ರದ
ಸೂತ್ರವನ್ನು
ಸೂದ್ರಕ
ಸೂರನಹಳ್ಳಿಯ
ಸೂರನಹಳ್ಳಿಯನ್ನುಮೆಚ್ಚುಗೆಯಾಗಿ
ಸೂರನಹಳ್ಳಿಯು
ಸೂರಸ್ತಗಣದ
ಸೂರೆ
ಸೂರೆಗೊಂಡು
ಸೂರೆಮಾಡಿದನು
ಸೂರೆಮಾಡಿದರೆಂದು
ಸೂರ್ಯನಾಥ
ಸೂರ್ಯನಾಥಕಾಮತ್
ಸೂರ್ಯಪ್ರತಿಷ್ಠೆಯನ್ನು
ಸೂಳೆ
ಸೆಕೆಯು
ಸೆಜ್ಜೆೆಮನೆಯಲ್ಲಿ
ಸೆಟ್ಟಿ
ಸೆಟ್ಟಿಗಳು
ಸೆಟ್ಟಿಗವುಡನ
ಸೆಟ್ಟಿಗೆರೆ
ಸೆಟ್ಟಿಗೆರೆಯು
ಸೆಟ್ಟಿತಿಗೂ
ಸೆಟ್ಟಿಪುರ
ಸೆಟ್ಟಿಯ
ಸೆಟ್ಟಿಯನ್ನು
ಸೆಟ್ಟಿಯರು
ಸೆಟ್ಟಿಯು
ಸೆಟ್ಟಿವಟ್ಟವನ್ನು
ಸೆಟ್ಟಿಹಳ್ಳಿ
ಸೆಟ್ಟಿಹಳ್ಳಿಯ
ಸೆಣಸಿ
ಸೆಣಸೆ
ಸೆಪ್ಟೆಂಬರ್
ಸೆರಗುವಾರ್ದಪೊರಿನ್ನಿರನೆಂದು
ಸೆರೆಯಲ್ಲಿಟ್ಟನು
ಸೆರೆಯಲ್ಲಿಟ್ಟಿದ್ದನು
ಸೆರೆಯಲ್ಲಿಟ್ಟು
ಸೆರೆಯಲ್ಲಿಡಿಸಿದನೆಂದು
ಸೆರೆಯಾಗಿರಬೇಕೆಂದು
ಸೆರೆಹಿಡಿದ
ಸೆರೆಹಿಡಿದು
ಸೆಳೆದಿರುವ
ಸೆಳೆದುಕೊಂಡನೆಂದು
ಸೇಂದ್ರಕ
ಸೇಂದ್ರಕವಂಶದ
ಸೇಡು
ಸೇತುಬಂಧಮಾಡಿ
ಸೇತುವರಂ
ಸೇತುವಿನ
ಸೇತುವೆ
ಸೇತುವೆಗಳನ್ನು
ಸೇತುವೆಯ
ಸೇತುವೆಯನ್ನು
ಸೇತುವೆಯೊಂದನ್ನು
ಸೇನ
ಸೇನಬೋಗ
ಸೇನಬೋವ
ಸೇನಬೋವಅನಿದ್ದನು
ಸೇನಬೋವನನ್ನೇ
ಸೇನಬೋವನಾಗಿರಬಹುದು
ಸೇನಬೋವನಿರುತ್ತಿದ್ದನು
ಸೇನಬೋವನು
ಸೇನಬೋವನೊಳಗಾದ
ಸೇನಬೋವರ
ಸೇನಬೋವರಂತಹ
ಸೇನಬೋವರನ್ನು
ಸೇನಬೋವರವರೆಗಿನ
ಸೇನಬೋವರಿದ್ದರೆಂದು
ಸೇನಬೋವರು
ಸೇನಬೋವರೇ
ಸೇನಯ
ಸೇನವಾರ
ಸೇನಾ
ಸೇನಾಠಾಣ್ಯವಿದ್ದ
ಸೇನಾತುಕಡಿಯನ್ನು
ಸೇನಾತುಕಡಿರೆಜಿಮೆಂಟ್
ಸೇನಾದಳದ
ಸೇನಾಧಿಕಾರಿಗಳ
ಸೇನಾಧಿಕಾರಿಗಳೂ
ಸೇನಾಧಿಕಾರಿಯಾಗಿದ್ದನೆಂದು
ಸೇನಾಧಿಕಾರಿಯೆಂದೇ
ಸೇನಾಧಿಪತಿ
ಸೇನಾಧಿಪತಿಆಗಿದ್ದನು
ಸೇನಾಧಿಪತಿಗಳ
ಸೇನಾಧಿಪತಿಗಳನ್ನು
ಸೇನಾಧಿಪತಿಗಳಲ್ಲಿ
ಸೇನಾಧಿಪತಿಗಳಾಗಿದ್ದರು
ಸೇನಾಧಿಪತಿಗಳಿರುತ್ತಿದ್ದರೆಂದು
ಸೇನಾಧಿಪತಿಗಳು
ಸೇನಾಧಿಪತಿಯ
ಸೇನಾಧಿಪತಿಯಾಗಿದ್ದನು
ಸೇನಾಧಿಪತಿಯೂ
ಸೇನಾಧಿಪತಿಯೆಂದು
ಸೇನಾಧಿಪತಿಯೋ
ಸೇನಾನನಾಯಕರು
ಸೇನಾನಾಥ
ಸೇನಾನಾಯಕ
ಸೇನಾನಾಯಕತನ
ಸೇನಾನಾಯಕನಪ್ಪ
ಸೇನಾನಾಯಕನಾಗಿ
ಸೇನಾನಾಯಕನಾಗಿದ್ದನೆಂದು
ಸೇನಾನಾಯಕನಾದ
ಸೇನಾನಾಯಕನೆಂದು
ಸೇನಾನಾಯಕನೆನಿಸಿದಅನು
ಸೇನಾನಾಯಕರ
ಸೇನಾನಾಯಕರಪ್ಪ
ಸೇನಾನಾಯಕರಾಗಿದ್ದರು
ಸೇನಾನಾಯಕರು
ಸೇನಾನಾಯಕರುಬಲಗೈಯ
ಸೇನಾನಾಯಕರೆಂದೇ
ಸೇನಾನಿ
ಸೇನಾನಿಗಳಲ್ಲಿ
ಸೇನಾನಿಗಳಾಗಿದ್ದ
ಸೇನಾನಿಯಾಗಿದ್ದ
ಸೇನಾನೆಲೆಯನ್ನು
ಸೇನಾಪಡೆ
ಸೇನಾಪಡೆಗೆ
ಸೇನಾಪಡೆಯ
ಸೇನಾಪಡೆಯನ್ನು
ಸೇನಾಪಡೆಯೆಂದೂ
ಸೇನಾಪತಿ
ಸೇನಾಪತಿಗಳಾಗಲೀ
ಸೇನಾಪತಿಗಳಿಗೆ
ಸೇನಾಪತಿಗಳುಸೇನಾಧಿಪತಿಗಳುಚಮೂಪರು
ಸೇನಾಪತಿಯಾಗಿರಬಹುದು
ಸೇನಾಪತಿಯು
ಸೇನಾಬಲ
ಸೇನಾಬಲಕ್ಕೆ
ಸೇನಾಬಲದ
ಸೇನಾವೀರರಾಗಿದ್ದರು
ಸೇನಾವೀರರು
ಸೇನುಬೋವ
ಸೇನುಬೋವನು
ಸೇನುಬೋವರಿದ್ದರೆಂಬುದು
ಸೇನುಬೋವರೂ
ಸೇನೆ
ಸೇನೆಗಳಿಗೆ
ಸೇನೆಗೆ
ಸೇನೆಯ
ಸೇನೆಯನ್ನು
ಸೇನೆಯಲ್ಲಿ
ಸೇನೆಯಿಂ
ಸೇನೆಯಿಂದ
ಸೇನೆಯು
ಸೇನೆಯೊಂದಿಗೆ
ಸೇನೆಯೊಡಗೂಡಿ
ಸೇನೆಯೊಡನೆ
ಸೇರನಹಳ್ಳಿ
ಸೇರಿ
ಸೇರಿಕೊಂಡು
ಸೇರಿತ್ತು
ಸೇರಿತ್ತೆಂದು
ಸೇರಿತ್ತೇ
ಸೇರಿದ
ಸೇರಿದಂತೆ
ಸೇರಿದವನಲ್ಲವೆಂದೂ
ಸೇರಿದವನಾಗಿದ್ದು
ಸೇರಿದವನಿರಬಹುದು
ಸೇರಿದವನು
ಸೇರಿದವನೆಂದು
ಸೇರಿದವನೆಂದೂ
ಸೇರಿದವರಾಗಿದ್ದರು
ಸೇರಿದವರಾಗಿದ್ದಾರೆ
ಸೇರಿದವರಾಗಿದ್ದು
ಸೇರಿದವರಾಗಿರಬಹುದೆಂದೂ
ಸೇರಿದವರಾಗಿರುತ್ತಾರೆ
ಸೇರಿದವರು
ಸೇರಿದವರುಇಂದಿನ
ಸೇರಿದವರೆಂದು
ಸೇರಿದವರೆಂದೂ
ಸೇರಿದವಾಗಿದ್ದು
ಸೇರಿದಾಗ
ಸೇರಿದೆ
ಸೇರಿದ್ದ
ಸೇರಿದ್ದರೆಂದು
ಸೇರಿದ್ದವು
ಸೇರಿದ್ದವೆಂದು
ಸೇರಿದ್ದವೆಂದೂ
ಸೇರಿದ್ದಾನೆ
ಸೇರಿದ್ದು
ಸೇರಿರಬಹುದಾದ
ಸೇರಿರಬಹುದೆಂದು
ಸೇರಿರುವ
ಸೇರಿರುವುದರಿಂದ
ಸೇರಿಲ್ಲ
ಸೇರಿಲ್ಲವೆಂದಾಯಿತು
ಸೇರಿವೆ
ಸೇರಿಸಲಾಗಿತ್ತು
ಸೇರಿಸಲಾಗಿದೆ
ಸೇರಿಸಲಾಯಿತು
ಸೇರಿಸಿ
ಸೇರಿಸಿಕೊಂಡನೆಂದು
ಸೇರಿಸಿಕೊಂಡರು
ಸೇರಿಸಿಕೊಳ್ಳುತ್ತಿದುದು
ಸೇರಿಸಿದ
ಸೇರಿಸಿದರೆಂದು
ಸೇರಿಸಿದೆ
ಸೇರಿಸಿದ್ದಾರೆ
ಸೇರಿಸಿರಬಹುದು
ಸೇರಿಸುವಾಗ
ಸೇರಿಹೋಗಿದೆ
ಸೇರಿಹೋಗಿರುವ
ಸೇರುತ್ತದೆಂದು
ಸೇರುತ್ತವೆ
ಸೇರುತ್ತಾರೆ
ಸೇರುವ
ಸೇರ್ಪಡೆ
ಸೇರ್ಪಡೆಯಾಗಿದ್ದವೆಂದು
ಸೇವಂತನಹಳ್ಳಿ
ಸೇವಕರಾಗಿ
ಸೇವಾರ್ಥವಾಗಿ
ಸೇವುಣ
ಸೇವುಣದಳದ
ಸೇವುಣರ
ಸೇವುಣರನ್ನು
ಸೇವುಣರಾಯದರ್ಪದಳನ
ಸೇವುಣರಿಗೆ
ಸೇವುಣರು
ಸೇವುಣರೊಡನೆ
ಸೇವುಣಸೇನೆಯನ್ನು
ಸೇವುಣಾಧಿಪ
ಸೇವುಳರಾಯ
ಸೇವೆ
ಸೇವೆಎಂದು
ಸೇವೆಗಾಗಿ
ಸೇವೆಗೆ
ಸೇವೆಯನ್ನು
ಸೇವೆಯಲ್ಲಿ
ಸೇವೆಯಿಂದ
ಸೇವೆಸಲ್ಲಿಸಿ
ಸೇವೆಸಲ್ಲಿಸುತ್ತಿದ್ದರೂ
ಸೇವ್ಯನೆಂದು
ಸೈಗೊಟ್ಟ
ಸೈಗೋಲಮಾತ್ಯ
ಸೈದ್ಧಾಂತಿಕ
ಸೈನಿಕರ
ಸೈನಿಕರಂ
ಸೈನಿಕರನ್ನು
ಸೈನಿಕರಿಗೆ
ಸೈನಿಕರು
ಸೈನ್ಯ
ಸೈನ್ಯಕ್ಕೆ
ಸೈನ್ಯದ
ಸೈನ್ಯದೊಡನೆ
ಸೈನ್ಯವನ್ನು
ಸೈನ್ಯಸಮೇತನಾಗಿ
ಸೈನ್ಯಾಧಿಕಾರಿಗಳಾಗಿದ್ದರು
ಸೈನ್ಯಾನೀಕಮಂ
ಸೈಯದ್
ಸೊಂಡೆಕೊಪ್ಪ
ಸೊಂನಾಕೋತ್ಸತಿ
ಸೊಗಯಿಪನೆನೆಸುಂ
ಸೊದರಳಿಯಂದಿರಾದ
ಸೊಮೆಯದಂಡನಾಯಕ
ಸೊಮೇಶ್ವರದೇವನೊಡನೆ
ಸೊರಟೂರು
ಸೊರಬ
ಸೊಲಿಸಿದನು
ಸೊಸಿಯಪ್ಪ
ಸೊಸಿಯಪ್ಪನ
ಸೊಸಿಯಪ್ಪನಾಯಕ
ಸೊಸೆವೂರನ್ನು
ಸೊಸೆವೂರಿನಿಂದ
ಸೋತ
ಸೋತನೆಂದು
ಸೋತವರ
ಸೋತು
ಸೋದರ
ಸೋದರನಾಗಿರಬಹುದು
ಸೋದರಮಾವ
ಸೋದರರ
ಸೋದರರು
ಸೋದರರೆಂದು
ಸೋದರಳಿಯ
ಸೋದರಳಿಯಂದಿರಾದ
ಸೋದರಳಿಯಂದಿರು
ಸೋದರಳಿಯನಾದ
ಸೋದರಿ
ಸೋಧೆವೆಸನಂ
ಸೋಪಾನವನ್ನು
ಸೋಭಿಸೆ
ಸೋಮ
ಸೋಮಂ
ಸೋಮಂಗಳಿಯಂ
ಸೋಮಣ್ಣ
ಸೋಮಣ್ಣದಂಡನಾಯಕರು
ಸೋಮದಂಡನಾಥನ
ಸೋಮದಂಡನಾಯಕನ
ಸೋಮದಂಡನಾಯಕನನ್ನು
ಸೋಮದಂಡನಾಯಕನಿಗೆ
ಸೋಮದಂಡನಾಯಕನು
ಸೋಮನ
ಸೋಮನನ್ನು
ಸೋಮನವರೆಗೆ
ಸೋಮನಹಳ್ಳಿ
ಸೋಮನಾಥ
ಸೋಮನಾಥದೇವರ
ಸೋಮನಾಥಪುರ
ಸೋಮನಾಥಪುರದ
ಸೋಮನಾಥಪುರದಲ್ಲಿ
ಸೋಮನಾಥಪುರವಾದ
ಸೋಮನಾಥಪುರವೆಂಬ
ಸೋಮನಿನಿಸಿದನು
ಸೋಮನು
ಸೋಮನೃಪನ
ಸೋಮಯ
ಸೋಮಯನಾಯಕನೊಳಗಾದ
ಸೋಮಯೆ
ಸೋಮಯ್ಯ
ಸೋಮಯ್ಯನು
ಸೋಮವಂಶಾಧೀಶ್ವರನೆಂದು
ಸೋಮವರ್ಮನು
ಸೋಮವರ್ಮನೆಂಬ
ಸೋಮಿದೇವ
ಸೋಮಿಸೆಟ್ಟಿ
ಸೋಮಿಸೆಟ್ಟಿಯು
ಸೋಮೆಯ
ಸೋಮೆಯಜ
ಸೋಮೆಯದಂಡನಾಯಕನ
ಸೋಮೆಯದಂಡನಾಯಕನು
ಸೋಮೆಯದಂಡನಾಯಕಮಲ್ಲಿದೇವ
ಸೋಮೆಯದಂಡನಾಯಕರ
ಸೋಮೆಯನ
ಸೋಮೆಯನಾಯಕ
ಸೋಮೆಯನಾಯಕನ
ಸೋಮೆಯನಾಯಕನು
ಸೋಮೇಶ್ವರ
ಸೋಮೇಶ್ವರದೇವರಸರು
ಸೋಮೇಶ್ವರನ
ಸೋಮೇಶ್ವರನನ್ನು
ಸೋಮೇಶ್ವರನಿಗೆ
ಸೋಮೇಶ್ವರನು
ಸೋಮೇಶ್ವರನೇ
ಸೋಯಂ
ಸೋಯಕ್ಕ
ಸೋಯಕ್ಕನಿಗೂ
ಸೋಲನ್ನು
ಸೋಲಾಯಿತು
ಸೋಲಿಸಲಾಗದ
ಸೋಲಿಸಲಾಗಿ
ಸೋಲಿಸಲು
ಸೋಲಿಸಲ್ಪಟ್ಟ
ಸೋಲಿಸಿ
ಸೋಲಿಸಿಕೊಂದು
ಸೋಲಿಸಿದ
ಸೋಲಿಸಿದಂತೆ
ಸೋಲಿಸಿದನು
ಸೋಲಿಸಿದನೆಂದು
ಸೋಲಿಸಿದರು
ಸೋಲಿಸಿದಾಗ
ಸೋಲಿಸಿದ್ದಕ್ಕಾಗಿಯೂ
ಸೋಲಿಸಿದ್ದನೆಂದು
ಸೋಲಿಸಿದ್ದಲ್ಲದೆ
ಸೋಲಿಸಿರಬಹುದು
ಸೋಲುಂಟಾಯಿತೆಂದು
ಸೋಲೂರು
ಸೋವಣ
ಸೋವಣ್ಣ
ಸೋವಣ್ಣನೂ
ಸೋವಿದೇವ
ಸೋವಿದೇವಘಟೆಯಂ
ಸೋವಿದೇವನಿಗೂ
ಸೋವಿದೇವನೊಡನೆ
ಸೋವಿಸೆಟ್ಟಿಯು
ಸೋವೆಯದಂಡನಾಯಕಸೋಮೆಯ
ಸೋವೆಯನಾಯಕ
ಸೋವೆಯನಾಯಕಂ
ಸೋವೆಯನಾಯಕನ
ಸೋಸಲಿ
ಸೋಸಲಿಯಇಂದಿನ
ಸೋಸಲೆ
ಸೋಸಲೆಯೇ
ಸೌಜನಬಾಂಧವ
ಸೌಮ್ಯಕೇಶವದೇವಾಲಯವನ್ನು
ಸೌಮ್ಯಜಾಮಾತೃ
ಸೌಮ್ಯರಾಜ
ಸೌರಾಷ್ಟ್ರಗಳನ್ನು
ಸೌರಾಷ್ಟ್ರದಿಂದ
ಸೌವಿದಲ್ಲಪದಂ
ಸೌಹಾರ್ದವನ್ನು
ಸ್ಟೇಷನ್
ಸ್ತನಹಾರ
ಸ್ತಳ
ಸ್ತೀರೋಮಣಿ
ಸ್ತುತಿ
ಸ್ತುತಿಗೆ
ಸ್ತುತಿಯನಂತರ
ಸ್ತುತಿಯಿಂದ
ಸ್ತುತಿಯಿಂದಲೇ
ಸ್ತುತಿಯೊಂದಿಗೆ
ಸ್ತುತಿಸಿದು
ಸ್ತುತಿಸಿದೆ
ಸ್ತುತಿಸಿದ್ದು
ಸ್ತುತಿಸುತ್ತದೆ
ಸ್ತ್ರೀಯರ
ಸ್ತ್ರೀಯರು
ಸ್ತ್ರೀಸಮಾಜ
ಸ್ಥಂಬನು
ಸ್ಥಂಭನನ್ನುರಣಾವಲೋಕ
ಸ್ಥಂಭನು
ಸ್ಥಂಭೀತನನ್ನಾಗಿ
ಸ್ಥಳ
ಸ್ಥಳಕ್ಕೆ
ಸ್ಥಳಗಳ
ಸ್ಥಳಗಳನ್ನು
ಸ್ಥಳಗಳನ್ನೇ
ಸ್ಥಳಗಳು
ಸ್ಥಳಗಳೆಂಬ
ಸ್ಥಳದ
ಸ್ಥಳದಲು
ಸ್ಥಳದಲ್ಲಿ
ಸ್ಥಳದಲ್ಲಿದ್ದವು
ಸ್ಥಳದಿಂದ
ಸ್ಥಳದೊಳಗಣ
ಸ್ಥಳನಾಮ
ಸ್ಥಳನಾಮಗಳ
ಸ್ಥಳನಾಮಗಳನ್ನು
ಸ್ಥಳನಾಮಗಳು
ಸ್ಥಳನಾಮದ
ಸ್ಥಳನಾಮವನ್ನು
ಸ್ಥಳಪರಿವೀಕ್ಷಣೆಯ
ಸ್ಥಳಪುರಾಣ
ಸ್ಥಳಪುರಾಣಗಳ
ಸ್ಥಳಪುರಾಣದ
ಸ್ಥಳವನ್ನು
ಸ್ಥಳವಾಗಿ
ಸ್ಥಳವಾಗಿತ್ತು
ಸ್ಥಳವಾಗಿದೆ
ಸ್ಥಳವಾಗಿದ್ದು
ಸ್ಥಳವಾಯಿತು
ಸ್ಥಳವು
ಸ್ಥಳವುಇಂದಿನ
ಸ್ಥಳಾಂತರದ
ಸ್ಥಳಾಂತರನಾದನೆಂದು
ಸ್ಥಳಾಂತರವಾಗಿತ್ತು
ಸ್ಥಳಾಂತರಿಸಲಾಗಿದೆ
ಸ್ಥಳಾಂತರಿಸಲಾಯಿತು
ಸ್ಥಳಾಂತರಿಸಿದನೆಂದು
ಸ್ಥಳೀಯ
ಸ್ಥಳೀಯರ
ಸ್ಥಳೀಯರು
ಸ್ಥಳೀಯರೇ
ಸ್ಥಳೀಯವೇ
ಸ್ಥಳೇ
ಸ್ಥಾನ
ಸ್ಥಾನಗಳಿದ್ದುದು
ಸ್ಥಾನಗಳು
ಸ್ಥಾನದ
ಸ್ಥಾನಪಡೆದಿದ್ದರು
ಸ್ಥಾನಪಡೆದು
ಸ್ಥಾನಪತಿ
ಸ್ಥಾನಪತಿಗೆ
ಸ್ಥಾನಪತಿಯಾಗಿ
ಸ್ಥಾನಪತಿಯಾಗಿದ್ದನು
ಸ್ಥಾನಪತಿಯಾಗಿದ್ದನೆಂದು
ಸ್ಥಾನಪತಿಯೂ
ಸ್ಥಾನಮಾನ
ಸ್ಥಾನಮಾನಗಳನ್ನು
ಸ್ಥಾನಮಾನವನ್ನೋ
ಸ್ಥಾನವನ್ನು
ಸ್ಥಾನವಾಗಿರಬೇಕು
ಸ್ಥಾನಿಕನಾಗಿದ್ದನು
ಸ್ಥಾನಿಕರಾಗಿದ್ದರೆಂದು
ಸ್ಥಾನಿಕರು
ಸ್ಥಾನೀಕ
ಸ್ಥಾಪಕ
ಸ್ಥಾಪಕನೆಂದು
ಸ್ಥಾಪಕರಾದ
ಸ್ಥಾಪಕರು
ಸ್ಥಾಪನಾಚಾರ್ಯ
ಸ್ಥಾಪನೆ
ಸ್ಥಾಪನೆಗೆ
ಸ್ಥಾಪನೆಯ
ಸ್ಥಾಪನೆಯಲ್ಲಿ
ಸ್ಥಾಪನೆಯಾದ
ಸ್ಥಾಪಿಸಲು
ಸ್ಥಾಪಿಸಿ
ಸ್ಥಾಪಿಸಿದನು
ಸ್ಥಾಪಿಸಿದನೆಂದು
ಸ್ಥಾಪಿಸಿದರೆಂದು
ಸ್ಥಾಪ್ಯಂತೇ
ಸ್ಥಾಫನಾಚಾರ್ಯ
ಸ್ಥಾವರ
ಸ್ಥಿತಿಯನ್ನು
ಸ್ಥಿತ್ಯಂತರದ
ಸ್ಥಿರ
ಸ್ಥಿರಂ
ಸ್ಥಿರಜೀವಿಗಳಾದರೆಂದು
ಸ್ಥಿರತಾಟಂಕವತ್ಯಭೂತ್
ಸ್ಥಿರನಾರಾಯಣಂ
ಸ್ಥಿರಪಡಿಸತೊಡಗಿದನು
ಸ್ಥಿರವಾಗಿ
ಸ್ಥಿರವಾಯಿತೆಂದು
ಸ್ಥಿರವೈಭವಸ್ತಸ್ಯ
ಸ್ಥೂಲ
ಸ್ಥೂಲವಾಗಿ
ಸ್ಥೈರ್ಯಮಂದರಂ
ಸ್ನೇಹ
ಸ್ನೇಹವನ್ನು
ಸ್ಪತಿ
ಸ್ಪಷ್ಟ
ಸ್ಪಷ್ಟತೆ
ಸ್ಪಷ್ಟಪಡಿಸುತ್ತದೆ
ಸ್ಪಷ್ಟಪಡಿಸುತ್ತದೆಂದು
ಸ್ಪಷ್ಟವಾಗಿ
ಸ್ಪಷ್ಟವಾಗಿದೆ
ಸ್ಪಷ್ಟವಾಗಿಲ್ಲ
ಸ್ಪಷ್ಟವಾಗುತ್ತದೆ
ಸ್ಪಷ್ಟವಾಗುವುದು
ಸ್ಪಷ್ಟವಿಲ್ಲ
ಸ್ಫಾರಪ್ರತಾಪ
ಸ್ಮರಣ
ಸ್ಮರಣಾರ್ಥ
ಸ್ಮರಣಾರ್ಥವಾಗಿ
ಸ್ಮಾರಕ
ಸ್ಮಾರ್ತ
ಸ್ಮಾರ್ತಬ್ರಾಹ್ಮಣನು
ಸ್ಯಮ್ಯಕ್ತ್ವ
ಸ್ವಂತ
ಸ್ವಕೀಯ
ಸ್ವಕೀಯಕರ್ನಾಟಕಕ
ಸ್ವಕೀಯೈಕಾದಶಪಲ್ಲಿ
ಸ್ವಜನಂ
ಸ್ವಜನಗೋತ್ರ
ಸ್ವತಂತ್ರ
ಸ್ವತಂತ್ರನಾಗಬೇಕೆಂದು
ಸ್ವತಂತ್ರನಾಗಲು
ಸ್ವತಂತ್ರನಾಗಿ
ಸ್ವತಂತ್ರನಾದ
ಸ್ವತಂತ್ರರಾಗಿ
ಸ್ವತಂತ್ರರಾಜನಂತೆ
ಸ್ವತಂತ್ರವಾಗಿ
ಸ್ವತಂತ್ರವಾದ
ಸ್ವತಃ
ಸ್ವಧರ್ಮದಿಂದ
ಸ್ವಭಾನು
ಸ್ವಯಂ
ಸ್ವಯಂಭು
ಸ್ವಯಂಭೂ
ಸ್ವಯಂಭೂನಾಥನಿಗೆ
ಸ್ವರವನಹಳ್ಳಿ
ಸ್ವರೂಪದ್ದಾಗಿದ್ದು
ಸ್ವರೂಪವನ್ನು
ಸ್ವರ್ಗಮರ್ತ್ಯಪಾತಾಳ
ಸ್ವರ್ಗಲೋಕಸುಕಪ್ರಾಪ್ತನೆಂದು
ಸ್ವರ್ಗಸ್ಥನಾಗಿ
ಸ್ವರ್ಗಸ್ಥನಾಗುತ್ತಾನೆ
ಸ್ವರ್ಗಸ್ಥನಾದನೆಂದು
ಸ್ವರ್ಗಸ್ಥನಾದಾಗ
ಸ್ವರ್ಗಸ್ಥರಾದರೆಂದು
ಸ್ವರ್ಗ್ಗಕ್ಕೆ
ಸ್ವರ್ಣಕಿರೀಟ
ಸ್ವಲ್ಪ
ಸ್ವಲ್ಪಕಾಲ
ಸ್ವಲ್ಪಭಾಗವನ್ನು
ಸ್ವಲ್ಪಮಟ್ಟಿಗೆ
ಸ್ವಲ್ಪಮಟ್ಟಿನ
ಸ್ವಷ್ಟವಾಗಿ
ಸ್ವಸ್ತಾನೇಕ
ಸ್ವಸ್ತಿ
ಸ್ವಸ್ತಿಪುರವರಾಧೀಶ್ವರ
ಸ್ವಸ್ತಿಳಕೈಃ
ಸ್ವಸ್ತ್ಯನವರತ
ಸ್ವಸ್ವಾಮಿನಂ
ಸ್ವಹಸ್ತದಿಂದ
ಸ್ವಾತಂತ್ರ್ಯ
ಸ್ವಾತಂತ್ರ್ಯವನ್ನು
ಸ್ವಾತಿ
ಸ್ವಾಧೀನನಯಸಂಪದಃ
ಸ್ವಾಭಾವಿಕ
ಸ್ವಾಮಿ
ಸ್ವಾಮಿಕಾರ್ಯ
ಸ್ವಾಮಿಗೆ
ಸ್ವಾಮಿದ್ರೋಹರಗಂಣ್ಡನುಂ
ಸ್ವಾಮಿಭಕ್ತಿಗೆ
ಸ್ವಾಮಿಭೃತ್ಯಂ
ಸ್ವಾಮಿಯ
ಸ್ವಾಮಿಯಂಗಸನ್ನಾಹ
ಸ್ವಾಮಿಯನ್ನು
ಸ್ವಾಮಿಯರ
ಸ್ವಾಮಿಯಾದ
ಸ್ವಾಮಿವಂಚಕರಗಂಡ
ಸ್ವಾಮ್ಯವಂತರು
ಸ್ವಾರಸ್ಯಕರವಾಗಿವೆ
ಸ್ವಾರಾಜರಾಜಮಾನಶ್ರೀ
ಸ್ವೀಕರಿಸದೇ
ಸ್ವೀಕರಿಸಿ
ಸ್ವೀಕರಿಸಿದ್ದನೆಂದು
ಸ್ವೀಕರಿಸಿದ್ದರೆಂದು
ಸ್ವೀಕರಿಸಿರಬಹುದು
ಸ್ವೀಕಾರಸಾರೋದಯ
ಸ್ವೀಯಪ್ರತಾಪೋದಯೌ
ಸ್ವೋರನಾಡಿನ
ಸ್ವೋರೆನಾಡು
ಸ್ಷಷ್ಟವಾಗಿ
ಸ್ಷಷ್ಟವಾಗುತ್ತದೆ
ಹ
ಹಂಗಾಮು
ಹಂಚಿಕೆ
ಹಂಚಿಕೆಗಳಿಗೆ
ಹಂಚಿಕೆಯನ್ನು
ಹಂಚಿಕೆಯಾಗಿದೆ
ಹಂಚಿಪುರ
ಹಂಚಿಯ
ಹಂಣೆಚೌಕನಹಳ್ಳಿಅಣ್ಣೆಚಾಕನಹಳ್ಳಿ
ಹಂಣೆಚೌಕನಹಳ್ಳಿಅಣ್ಣೆಚಾಕನಹಳ್ಳಿಚಿಕ್ಕಕಳಲೆ
ಹಂತ
ಹಂತಗಳನ್ನು
ಹಂತದ
ಹಂತದಲ್ಲಿ
ಹಂತವಾದರೆ
ಹಂತಹಂತವಾಗಿ
ಹಂಪನಾ
ಹಂಪನಾಗರಾಜಯ್ಯನವರು
ಹಂಪರಾಜರ
ಹಂಪಾಪುರ
ಹಂಪೆಯ
ಹಂಪೆಯನ್ನು
ಹಂಪೆಯಲ್ಲಿ
ಹಂಪೆಯಲ್ಲಿಯೇ
ಹಂಪೆಯೇ
ಹಕ
ಹಕ್ಕನ್ನು
ಹಕ್ಕಿಯೂ
ಹಕ್ಕೀಮಂಚನಹಳ್ಳಿ
ಹಕ್ಕು
ಹಕ್ಕುದಾರನೆಂದು
ಹಕ್ಕುಸ್ಥಾಪಿಸಿದರೆಂದೂ
ಹಗವಮಗೆರೆ
ಹಗವಮಗೆರೆಯನ್ನು
ಹಗೆತನವಿರಲಿಲ್ಲ
ಹಚ್ಚಿಕೊಂಡಿರುತ್ತಿದ್ದ
ಹಚ್ಚಿದ್ದನೆಂದೂ
ಹಟ್ಟಣ
ಹಟ್ಟಣದ
ಹಟ್ಟಣದಲ್ಲಿ
ಹಟ್ಟಣವನ್ನು
ಹಟ್ಟಿ
ಹಟ್ಟಿಗಳನ್ನೂ
ಹಟ್ಟಿಗಾಲಗಕ್ಕೆಕ
ಹಟ್ಟಿಗಾಳಗದಲ್ಲಿ
ಹಟ್ಟಿಗಾಳೆಗ
ಹಟ್ಟಿಗಾಳೆಗಕ್ಕೆ
ಹಟ್ಟಿಗಾಳೆಗದಲ್ಲಿ
ಹಟ್ಟಿಗಾಳೆಗದೊಳ್
ಹಟ್ಟಿಯ
ಹಠಹರಣ
ಹಡಗಿನ
ಹಡಗು
ಹಡದಕ್ಷೇತ್ರದ
ಹಡದು
ಹಡಪದ
ಹಡಪವಳ
ಹಡಪವಳರಾಗಿದ್ದರೂ
ಹಡಬಳ
ಹಡವಪಳ
ಹಡವಳ
ಹಡವಳದ
ಹಡವಳದೇವ
ಹಡವಳನಾಗಿದ್ದ
ಹಡವಳರು
ಹಡುವಂಗಲ
ಹಡುವಳ
ಹಡುವಳದ
ಹಡುವಳರ
ಹಡುವಳರು
ಹಡುವಳರೇ
ಹಡುವಳಹಡೆವಳಹಡಪದ
ಹಡೆದುಪಡೆದು
ಹಡೆಪವಳ
ಹಡೆವಳ
ಹಡೆವಳನ
ಹಣ
ಹಣಕಾಸಿನ
ಹಣಕಾಸು
ಹಣದ
ಹಣದಿಂದ
ಹಣಪಡೆದು
ಹಣವನ್ನು
ಹಣವು
ಹಣೆ
ಹಣ್ಣೆಯ
ಹಣ್ನೆಯ
ಹತನಾಗಿರುವುದು
ಹತನಾದನು
ಹತನಾದಾಗ
ಹತಾಶನಾಗಿ
ಹತಾಶೆಗೊಂಡನು
ಹತೋಟಿಯನ್ನು
ಹತೋಟಿಯಲ್ಲಿದ್ದರು
ಹತ್ತನೆಯ
ಹತ್ತನೇ
ಹತ್ತಾರು
ಹತ್ತಿಕ್ಕಲು
ಹತ್ತಿಕ್ಕಿ
ಹತ್ತಿಕ್ಕಿದನು
ಹತ್ತಿಕ್ಕುವಂತೆ
ಹತ್ತಿರ
ಹತ್ತಿರದ
ಹತ್ತಿರದಿಂದಲೇ
ಹತ್ತಿರವಾದ
ಹತ್ತಿರವಿರುವ
ಹತ್ತು
ಹತ್ತುಸಲಗೆ
ಹತ್ತುಸಾವಿರ
ಹದಗೆಟ್ಟಿದ್ದರಿಂದ
ಹದರಹಳಿವಿನ
ಹದಿನಾಡ
ಹದಿನಾಡನ್ನು
ಹದಿನಾಡಸೀಮೆಯ
ಹದಿನಾಡಿಗೆ
ಹದಿನಾಡಿನ
ಹದಿನಾಡು
ಹದಿನಾರನೆಯ
ಹದಿನಾರು
ಹದಿನಾಲ್ಕನೆಯ
ಹದಿನಾಲ್ಕುಮಂದಿ
ಹದಿನಾಲ್ಕುಹದಿನಾಡುನಾಡು
ಹದಿನೆಂಟು
ಹದಿನೈದು
ಹದಿಮೂರು
ಹನಸೋಗೆಯ
ಹನುಂತನು
ಹನುಮ
ಹನುಮಂತ
ಹನುಮಂತರಾಯಸ್ವಾಮಿಗೆ
ಹನುಮಂತೇಶ್ವರ
ಹನುಮದ್ಗರುಡ
ಹನುಮನ
ಹನುಮನಕಟ್ಟೆ
ಹನುಮನೇ
ಹನ್ನೆರಡನೇ
ಹನ್ನೆರಡು
ಹನ್ನೊಂದು
ಹನ್ಮನೆನೀ
ಹಪ್ಪಳಿಗೆಯನ್ನು
ಹಬ್ಬ
ಹಮೀದ್
ಹಯಪ್ರತತಿಯಂ
ಹಯವದನರಾವ್
ಹಯಾರೂಢ
ಹಯಾರೂಢನಾದನೆಂದು
ಹರಈಶ್ವರ
ಹರಕೆ
ಹರಗನಹಳ್ಳಿ
ಹರಡಿತ್ತೆಂದು
ಹರಡಿದೆ
ಹರಡಿದ್ದ
ಹರಡಿದ್ದು
ಹರಡಿರುವ
ಹರತಿ
ಹರತಿಸಿರಿಯಲ್ಲಿ
ಹರದನಹಳ್ಳಿ
ಹರದನಹಳ್ಳಿಯ
ಹರದನಹಳ್ಳಿಯನ್ನು
ಹರಪನಹಳ್ಳಿಯ
ಹರಪಾಲ
ಹರಳುಹಳ್ಳಿ
ಹರವು
ಹರಸಿ
ಹರಹಿನ
ಹರಿ
ಹರಿಕಾರ್ಭಕ್ಷಿಯಾಗಿದ್ದ
ಹರಿಗಿಲ
ಹರಿಣ
ಹರಿದಾಸ
ಹರಿದಿನ
ಹರಿದೇವ
ಹರಿನೀಲ
ಹರಿಪಾಲನ
ಹರಿಪಾಳಯ್ಯ
ಹರಿಭಕ್ತಿಸುಧಾನಿಧಿಃ
ಹರಿಯಂಣನೆನಿಸಿದನು
ಹರಿಯಣ್ಣ
ಹರಿಯಣ್ಣನ
ಹರಿಯಣ್ಣನು
ಹರಿಯಲೆ
ಹರಿಯುತ್ತದೆ
ಹರಿಯುತ್ತವೆ
ಹರಿಯುತ್ತಿದ್ದ
ಹರಿಯುತ್ತಿದ್ದವು
ಹರಿಯುತ್ತಿದ್ದು
ಹರಿಯುವ
ಹರಿವರ್ಮನ
ಹರಿಹರ
ಹರಿಹರಂ
ಹರಿಹರದಂಡನಾಯಕ
ಹರಿಹರದಂಡನಾಯಕನ
ಹರಿಹರದಂಡನಾಯಕನಿಗೆ
ಹರಿಹರದಂಡನಾಯಕನು
ಹರಿಹರದಂಡಾಯಕನು
ಹರಿಹರದೇವ
ಹರಿಹರದೇವನು
ಹರಿಹರಧರಣೀಪಾಲಕ
ಹರಿಹರನ
ಹರಿಹರನನ್ನು
ಹರಿಹರನಾಗಿದ್ದು
ಹರಿಹರನಿಗೆ
ಹರಿಹರನು
ಹರಿಹರನೃಪನನುಜಾತಂ
ಹರಿಹರನೇ
ಹರಿಹರಪಟ್ಟಣ
ಹರಿಹರಪಟ್ಟಣದಲ್ಲಿ
ಹರಿಹರಪುರ
ಹರಿಹರಪುರಕ್ಕೆ
ಹರಿಹರಪುರಗಳು
ಹರಿಹರಪುರದ
ಹರಿಹರಪುರದಲ್ಲಿದ್ದ
ಹರಿಹರಪುರವು
ಹರಿಹರಪುರವೆಂಬ
ಹರಿಹರಬ್ರಹ್ಮಾದಿಗಳೇ
ಹರಿಹರಭಟ್ಟೋಪಾಧ್ಯಾಯರಿಗೆ
ಹರಿಹರಮಹಾರಾಯರ
ಹರಿಹರರಾಯನ
ಹರುವನಹಳ್ಳಿಯ
ಹರೂರು
ಹರೆದು
ಹರೋಜನು
ಹರ್ಮ್ಮ್ಯಕುಲಕ್ಕೆ
ಹರ್ಮ್ಮ್ಯಮಾಣಿಕ್ಯ
ಹರ್ಯಣ
ಹರ್ಯಣನ
ಹರ್ಯಣನನ್ನು
ಹರ್ಯಣನಿಂದಾಗಿ
ಹರ್ಯಣನು
ಹರ್ಯಣಾತ್ಮಜಃ
ಹರ್ಯಣೋ
ಹಲಕೂರನ್ನು
ಹಲಕೂರು
ಹಲಗೂರನ್ನು
ಹಲಗೂರು
ಹಲನಾಡೊಳಗಳ
ಹಲರು
ಹಲವರನ್ನು
ಹಲವಾರು
ಹಲವಿವೆ
ಹಲವು
ಹಲವುಮಾರಾದಿ
ಹಲಸನಹಳ್ಳಿ
ಹಲಸನಹಳ್ಳಿಯನ್ನು
ಹಲಸಹಳ್ಳಿ
ಹಲಸಿತಾಳಹಳ್ಳಿಯ
ಹಲಸಿನತಾಳ
ಹಲಸಿನಹಳ್ಳಿ
ಹಲ್ಮಿಡಿ
ಹಲ್ಲೆಗೆರೆ
ಹಳಿಕಾಱ
ಹಳುವು
ಹಳೆಯ
ಹಳೆಯಬಿಡು
ಹಳೆಯಬೀಡಿಗೆ
ಹಳೆಯಬೆಳ್ಗೊಳವೇ
ಹಳೇಬೀಡಿನ
ಹಳೇಬೀಡು
ಹಳೇಬೂದನೂರಿನ
ಹಳೇಬೂದನೂರಿನಲ್ಲಿದೆ
ಹಳೇಬೂದನೂರು
ಹಳೇಮನೆ
ಹಳ್ಳಕೆರೆಇಂದಿನ
ಹಳ್ಳದ
ಹಳ್ಳದಬೀಡಿನಲು
ಹಳ್ಳದಬೀಡು
ಹಳ್ಳಬೀಡಾಗಿರಬಹುದು
ಹಳ್ಳಬೀಡಿನಲ್ಲಿ
ಹಳ್ಳವೂರ
ಹಳ್ಳವೂರು
ಹಳ್ಳಿ
ಹಳ್ಳಿಕೆರೆ
ಹಳ್ಳಿಗಳ
ಹಳ್ಳಿಗಳನ್ನು
ಹಳ್ಳಿಗಳನ್ನೂ
ಹಳ್ಳಿಗಳಾಗಿರಬಹುದು
ಹಳ್ಳಿಗಳಾಗಿರಬಹುದೆಂದು
ಹಳ್ಳಿಗಳಾಗಿವೆ
ಹಳ್ಳಿಗಳಿಗೂ
ಹಳ್ಳಿಗಳಿಗೆ
ಹಳ್ಳಿಗಳಿದ್ದು
ಹಳ್ಳಿಗಳು
ಹಳ್ಳಿಗಳೆಲ್ಲಾ
ಹಳ್ಳಿಗಳೊಡಗೂಡಿದ್ದ
ಹಳ್ಳಿಗೂ
ಹಳ್ಳಿಗೆ
ಹಳ್ಳಿಯ
ಹಳ್ಳಿಯನ್ನು
ಹಳ್ಳಿಯನ್ನುಹೊನ್ನೇನಹಳ್ಳಿ
ಹಳ್ಳಿಯನ್ನೇ
ಹಳ್ಳಿಯವರು
ಹಳ್ಳಿಯಾಗಿದೆ
ಹಳ್ಳಿಯು
ಹಳ್ಳಿಯೇ
ಹಳ್ಳಿಸೀಮೆಯಾಗಿ
ಹಳ್ಳಿಹಳ್ಳಿಗಳಲ್ಲಿ
ಹಳ್ಳೆಗೆರೆ
ಹವಣಿಕೆಯಲ್ಲಿದ್ದನು
ಹವಣಿಸಿ
ಹವಣಿಸಿದನು
ಹವಣಿಸಿದಾಗ
ಹವದೆಡೆಗಾಗಳುಂ
ಹವಾಲಿಸಿಕೊಡುತ್ತಾನೆ
ಹವೆಯನ್ನು
ಹಸಿಯಪ್ಪಂಗೆ
ಹಸೆಮಣೆಯಲ್ಲಿ
ಹಸೆಯೊಳ್
ಹಸ್ತ
ಹಸ್ತದಿಂದ
ಹಸ್ತಾಂತರಿಸಲಾಯಿತು
ಹಸ್ತಾಕ್ಷರಗಳ
ಹಸ್ತಾಕ್ಷರವಿದೆ
ಹಸ್ತಾಕ್ಷರವಿದ್ದು
ಹಸ್ತಾಕ್ಷರವಿರಬಹುದು
ಹಸ್ತಾಕ್ಷರವೇ
ಹಸ್ತಿಶೈಲೇಂದ್ರಮಹಾತ್ಮೆಯನ್ನು
ಹಾಕಬಹುದು
ಹಾಕಲು
ಹಾಕಸಲಾಗಿದೆ
ಹಾಕಿ
ಹಾಕಿಕೊಟ್ಟ
ಹಾಕಿಕೊಟ್ಟಂತೆ
ಹಾಕಿಕೊಟ್ಟನು
ಹಾಕಿಕೊಟ್ಟನೆಂದು
ಹಾಕಿಕೊಟ್ಟನೆಂದೂ
ಹಾಕಿಕೊಟ್ಟರೆಂದು
ಹಾಕಿಕೊಟ್ಟಳೆಂದು
ಹಾಕಿಕೊಟ್ಟಿದ್ದನೆಂದು
ಹಾಕಿಕೊಟ್ಟಿರಬಹುದೆಂದು
ಹಾಕಿಕೊಟ್ಟಿರುವ
ಹಾಕಿಕೊಟ್ಟು
ಹಾಕಿಕೊಡಲಾಗುತ್ತಿತ್ತು
ಹಾಕಿಕೊಡುತ್ತಾನೆ
ಹಾಕಿಕೊಡುತ್ತಾರೆ
ಹಾಕಿದನು
ಹಾಕಿದರೂ
ಹಾಕಿದಾಗ
ಹಾಕಿದೆ
ಹಾಕಿದ್ದಾರೆ
ಹಾಕಿರಬಹುದಾದ
ಹಾಕಿರುವುದು
ಹಾಕಿಸಲಾಗಿದೆ
ಹಾಕಿಸಿ
ಹಾಕಿಸಿದನ
ಹಾಕಿಸಿದನು
ಹಾಕಿಸಿದ್ದಾನೆ
ಹಾಕಿಸಿರಬಹುದು
ಹಾಕಿಸಿರಬಹುದೆಂದು
ಹಾಕಿಸಿರುವ
ಹಾಕಿಸಿರುವುದು
ಹಾಕಿಸುತ್ತಾನೆ
ಹಾಕಿಸುತ್ತಾರೆ
ಹಾಕಿಸುವ
ಹಾಗಲಹಳ್ಳಿ
ಹಾಗಲಹಳ್ಳಿಯನ್ನು
ಹಾಗವನ್ನು
ಹಾಗಾಗಿ
ಹಾಗಿದ್ದಲ್ಲಿ
ಹಾಗು
ಹಾಗೂ
ಹಾಗೆ
ಹಾಗೆಯೇ
ಹಾಜರಿದ್ದನೆಂದು
ಹಾಜರಿದ್ದರೆಂದು
ಹಾಜರಿದ್ದುದರ
ಹಾಡಿಮಂಡಲ
ಹಾಡಿಹೊಗಳಿವೆ
ಹಾಥಿದರವಾಜ
ಹಾದನೂರು
ಹಾದರವಾಗಿಲ
ಹಾದರವಾಗಿಲನ್ನು
ಹಾದರವಾಗಿಲು
ಹಾದಿಯಲ್ಲಿರುವ
ಹಾದಿರವಾಗಿಲನ್ನು
ಹಾನುಂಗಲಯ್ನೂರುಗಳನ್ನು
ಹಾನುಂಗಲ್ಲು
ಹಾನುಂಗಲ್ಲುಗೊಂಡ
ಹಾನುಗಲ್ಲಿನ
ಹಾರಪ್ಪಂಗಳ
ಹಾರುವ
ಹಾರುವಳ್ಳಿಯನ್ನು
ಹಾರುವಹಳ್ಳಿಹಾರೋಹಳ್ಳಿ
ಹಾಲತಿ
ಹಾಲಾಳು
ಹಾಲಿಮೊತ್ತದ
ಹಾಲಿಯಮೊತ್ತದ
ಹಾಲುಗಂಗಕೆರೆ
ಹಾಲುಗಂಗಕೆರೆಗೆ
ಹಾಲುಗಂಗಕೆರೆಯನ್ನು
ಹಾಳಹಾಳುಇಂದಿನ
ಹಾಳಾಗಿದ್ದ
ಹಾಳಾಯಿತು
ಹಾಳುಗೆಡವಿದ್ದ
ಹಾಳೆಗಳ
ಹಾಳೆಯ
ಹಾಳೆಹಳ್ಳಿ
ಹಾಸನ
ಹಾಸನಸೀಮೆಯ
ಹಾಹನವಾಡಿಯಹನಿಯಂಬಾಡಿ
ಹಿ
ಹಿಂಡನ್ನು
ಹಿಂತೆಗೆದನೆಂದು
ಹಿಂದಕ್ಕೆ
ಹಿಂದಣ
ಹಿಂದಿಕ್ಕಿ
ಹಿಂದಿನ
ಹಿಂದಿನವನು
ಹಿಂದಿರುಗಬೇಕಾಯಿತೆಂದು
ಹಿಂದಿರುಗಿ
ಹಿಂದಿರುಗಿರಬಹುದು
ಹಿಂದಿರುಗಿಸಿದರು
ಹಿಂದಿರುಗುತ್ತಿದ್ದ
ಹಿಂದೂ
ಹಿಂದೂಪುರ
ಹಿಂದೂರಾಯ
ಹಿಂದೂರಾಯಸುರತ್ರಾಣ
ಹಿಂದೂಸ್ಥಾನಿ
ಹಿಂದೆ
ಹಿಂದೆಯೂ
ಹಿಂದೆಯೇ
ಹಿಂದೆಹಬ್ಬಿ
ಹಿಂಭಾಗದಲ್ಲಿಯೇ
ಹಿಜರಿ
ಹಿಜಾಜ್
ಹಿಡಿತದಿಂದ
ಹಿಡಿದಂತೆ
ಹಿಡಿದನು
ಹಿಡಿದನೆಂದೂ
ಹಿಡಿದರೆ
ಹಿಡಿದಾಗ
ಹಿಡಿದು
ಹಿಡಿದುಕೊಂಡಿದ್ದರು
ಹಿಡಿದುದಕ್ಕೆ
ಹಿಡಿಯದೇ
ಹಿಡಿಸಿ
ಹಿತಕರವಾದ
ಹಿತವನ್ನೇ
ಹಿತವೇ
ಹಿತೇ
ಹಿತ್ತಾಳೆ
ಹಿನ್ನೀರಿನಲ್ಲಿ
ಹಿನ್ನೆಲೆ
ಹಿನ್ನೆಲೆಗಳೊಡನೆ
ಹಿನ್ನೆಲೆಯ
ಹಿನ್ನೆಲೆಯನ್ನು
ಹಿನ್ನೆಲೆಯಲ್ಲಿ
ಹಿನ್ನೆಲೆಯಲ್ಲಿಯೇ
ಹಿಮದಿಂ
ಹಿಮವದ್
ಹಿಮ್ಮೆಟ್ಟಿಸಿ
ಹಿರಣ್ಣಯ್ಯನ
ಹಿರಣ್ಯಗರ್ಭ
ಹಿರಿ
ಹಿರಿಕಳಲೆ
ಹಿರಿಕೊಂಡರಾಜ
ಹಿರಿತನದಿಂದ
ಹಿರಿದಾದ
ಹಿರಿದು
ಹಿರಿಮಂಡಳಿಕಮಾನ
ಹಿರಿಮೆಗಳನ್ನು
ಹಿರಿಯ
ಹಿರಿಯಅಗ್ರಹಾರ
ಹಿರಿಯಅಡವೆಹಿರೋಡೆ
ಹಿರಿಯಕಂನೆಯನಹಳ್ಳಿ
ಹಿರಿಯಕೆರೆಯ
ಹಿರಿಯಕೆರೆಯಕೆಳಗೆ
ಹಿರಿಯಚಾಮರಸ
ಹಿರಿಯಜೀಯ
ಹಿರಿಯಣ್ಣ
ಹಿರಿಯತಮ್ಮನ
ಹಿರಿಯದಂಡನಾಯಕ
ಹಿರಿಯದಂಡನಾಯಕಂ
ಹಿರಿಯದಂಡನಾಯಕನಾಗಿದ್ದನು
ಹಿರಿಯದೇವನು
ಹಿರಿಯನನ್ನು
ಹಿರಿಯನೀರಗುಂದ
ಹಿರಿಯಪ್ಪನು
ಹಿರಿಯಪ್ರಧಾನ
ಹಿರಿಯಬಯಿಚಪ್ಪ
ಹಿರಿಯಬಲ್ಲಾಳ
ಹಿರಿಯಬೆಟ್ಟದ
ಹಿರಿಯಬೆಟ್ಟದಚಾಮರಾಜನು
ಹಿರಿಯಭಂಡಾರಿ
ಹಿರಿಯಭಂಡಾರಿಯಾಗಿದ್ದುದರ
ಹಿರಿಯಭೇರುಂಡನ
ಹಿರಿಯಮಗ
ಹಿರಿಯಮರಳಿ
ಹಿರಿಯಮರಳಿಇಂದಿನ
ಹಿರಿಯಮರಿಯಾನೆ
ಹಿರಿಯಮಾಚ
ಹಿರಿಯರಸುತನವನ್ನು
ಹಿರಿಯರಾದವರು
ಹಿರಿಯರು
ಹಿರಿಯಹಡವಳ
ಹಿರಿಯಹಡೆವಳ
ಹಿರಿಯಹೆಗ್ಗಡೆ
ಹಿರಿಯೂರು
ಹಿರಿವೋಡೆ
ಹಿರಿಸಾವೆ
ಹಿರೀಕಳಲೆ
ಹಿರೆಕೊಲೆ
ಹಿರೆಜಂತಕಲ್
ಹಿರೆಮರಳಿಯ
ಹಿರೇಜಟ್ಟಿಗ
ಹಿರೇಬೆಟ್ಟದ
ಹಿರೇಮಠ್ರವರ
ಹಿರೇಮರಳಿ
ಹಿರೋಡೆ
ಹಿರೋಡೆಗೆ
ಹಿರೋಡೆಯ
ಹಿರೋಡೆಯನ್ನು
ಹಿಳಪಲ್ಲಿ
ಹಿಳ್ಳಹಳ್ಳಿ
ಹೀಗಾಗಿ
ಹೀಗಿದೆ
ಹೀಗೆ
ಹುಂಗೇನಹಳ್ಳಿ
ಹುಂಚ
ಹುಂಚದ
ಹುಚ್ಚನಹಳ್ಳಿ
ಹುಚ್ಚಮಾರುಡು
ಹುಜೂರ್
ಹುಜೂರ್ನಾಯಕ
ಹುಟ್ಟಿಗೆ
ಹುಟ್ಟಿದ
ಹುಟ್ಟಿದನು
ಹುಟ್ಟಿದಹಳ್ಳಿ
ಹುಟ್ಟಿಬೆಳೆದು
ಹುಟ್ಟುತ್ತದೆ
ಹುಟ್ಟುವಳಿ
ಹುಟ್ಟುವಳಿಗಳನ್ನು
ಹುಟ್ಟುವಳಿಯುಳ್ಳ
ಹುಣಸೂರು
ಹುತಾತ್ಮ
ಹುತಾತ್ಮನಾದದ್ದು
ಹುತಾತ್ಮನಾದನೆಂದು
ಹುತಾತ್ಮನಾದುದನ್ನು
ಹುತಾತ್ಮರ
ಹುದ್ದೆ
ಹುದ್ದೆಗಳ
ಹುದ್ದೆಗಳನ್ನು
ಹುದ್ದೆಗಳಲ್ಲಿ
ಹುದ್ದೆಗಳಿಗೆ
ಹುದ್ದೆಗಳಿದ್ದುದು
ಹುದ್ದೆಗಳಿಲ್ಲ
ಹುದ್ದೆಗಳು
ಹುದ್ದೆಗಳೆಂದು
ಹುದ್ದೆಗೂ
ಹುದ್ದೆಗೆ
ಹುದ್ದೆಗೇರಿದ್ದು
ಹುದ್ದೆಗೇರಿರುವುದು
ಹುದ್ದೆಯ
ಹುದ್ದೆಯನ್ನು
ಹುದ್ದೆಯನ್ನೂ
ಹುದ್ದೆಯನ್ನೋ
ಹುದ್ದೆಯಲ್ಲ
ಹುದ್ದೆಯಲ್ಲಿ
ಹುದ್ದೆಯಲ್ಲಿದ್ದನೆಂದು
ಹುದ್ದೆಯಲ್ಲಿದ್ದು
ಹುದ್ದೆಯಾಗಿತ್ತೆಂದು
ಹುದ್ದೆಯಾಗಿದ್ದು
ಹುದ್ದೆಯಾಗಿರಬಹುದು
ಹುದ್ದೆಯಾಗಿರಬಹುದುಮಹಾಪ್ರಧಾನ
ಹುದ್ದೆಯಿಂದ
ಹುದ್ದೆಯು
ಹುದ್ದೆಯೂ
ಹುದ್ದೆಯೆಂದು
ಹುದ್ದೆಯೇ
ಹುಬ್ಬನಹಳ್ಳಿ
ಹುಬ್ಬನಹಳ್ಳಿಯಲ್ಲಿ
ಹುಯ್ಯಲಾಯಿತೆಂದು
ಹುರಗಲವಾಡಿ
ಹುರುಗಲವಾಡಿ
ಹುಲಗೂರ
ಹುಲಗೂರು
ಹುಲಿ
ಹುಲಿಕಲ್ಲು
ಹುಲಿಕೆರೆ
ಹುಲಿಗಳಿಲ್ಲ
ಹುಲಿನವನಇಂದಿನ
ಹುಲಿಮುಖದ
ಹುಲಿಮುಖವನಿಕ್ಕಿಸಿದನು
ಹುಲಿಮೊಗವಾಡವನ್ನು
ಹುಲಿಯ
ಹುಲಿಯಜಂಗುಳಿ
ಹುಲಿಯಜಂಗುಳಿಯ
ಹುಲಿಯನನ್ನು
ಹುಲಿಯನ್ನು
ಹುಲಿಯು
ಹುಲಿಯೊಂದು
ಹುಲಿರಾಯ
ಹುಲಿವನ
ಹುಲಿವಾನ
ಹುಲಿವಾನದ
ಹುಲಿವಾನವನ್ನು
ಹುಲ್ಲಂಬಳ್ಳಿಯ
ಹುಲ್ಲವಂಗಲದ
ಹುಲ್ಲವಂಗಲವನ್ನು
ಹುಲ್ಲಹಳ್ಳಿ
ಹುಲ್ಲೇಗಾಲ
ಹುಲ್ಲೇಗಾಲದ
ಹುಲ್ಲೇಗಾಲವನ್ನು
ಹುಳ್ಳ
ಹುಳ್ಳಂಬಳ್ಳಿ
ಹುಳ್ಳಂಬಳ್ಳಿಯ
ಹುಳ್ಳಗಾವುಂಡನ
ಹುಳ್ಳಚಮೂಪ
ಹುಳ್ಳಚಮೂಪನ
ಹುಳ್ಳಚಮೂಪನು
ಹುಳ್ಳನ
ಹುಳ್ಳನೂ
ಹುಳ್ಳಮಯ್ಯನೂ
ಹುಳ್ಳಯ್ಯ
ಹುಳ್ಳರಾಜಂಗೆ
ಹುಳ್ಳೆಯ
ಹುಳ್ಳೆಯನಾಯಕನು
ಹುಳ್ಳೆಯಹಳ್ಳಿ
ಹುಳ್ಳೆಯಹಳ್ಳಿಯಲ್ಲಿ
ಹುಳ್ಳೇನಹಳ್ಳಿ
ಹುಳ್ಳೇನಹಳ್ಳಿಗೆ
ಹುಳ್ಳೋಹಳ್ಳಿಹುಳ್ಳೇನಹಳ್ಳಿ
ಹುಸಕೂರು
ಹುಸೇನ್
ಹುಸೈನ್
ಹುಸ್ಕೂರಿನಲ್ಲಿರುವ
ಹುಸ್ಕೂರು
ಹೂಡಿದ
ಹೂಡಿದ್ದನು
ಹೂಣರಾಜನಾದ
ಹೂಣರು
ಹೂರದಹಳ್ಳಿಯನ್ನು
ಹೂಲಿಕೆರೆಯ
ಹೂಲಿಯಕೆರೆ
ಹೂಳುವಂತೆ
ಹೂವಿನಬಾಗೆಯಲ್ಲಿ
ಹೃದಯಶಲ್ಯ
ಹೃದಯಸ್ಥಂಗಳ್
ಹೃದಯಸ್ಥವಾಗಿದ್ದವು
ಹೃದುವನಕೆರೆಗೂ
ಹೃದುವನಕೆರೆಯಲ್ಲಿ
ಹೆಂಡತಿ
ಹೆಂಡತಿಯ
ಹೆಂಡತಿಯರು
ಹೆಂಡತಿಯರೆಂದು
ಹೆಂಡತಿಯರೊಡನೆ
ಹೆಂಡಿರನ್ನು
ಹೆಂಬೆಟ್ಟ
ಹೆಂಮ
ಹೆಂಮನ
ಹೆಂಮನಗೌಡನ
ಹೆಂಮನಹಳ್ಳಿ
ಹೆಂಮನಾಜಿಗುಂಮನಂ
ಹೆಂಮನೂ
ಹೆಂಮಯ್ಯಂಗಳಿಯನ್
ಹೆಂಮಹೆಂಮಯ್ಯ
ಹೆಂಮೆಯ
ಹೆಂಮೇಶ್ವರದೇವರು
ಹೆಕ್ಟೇರ್
ಹೆಗ್ಗಡದೇವನಕೋಟೆ
ಹೆಗ್ಗಡಿಕೆಯಲಿ
ಹೆಗ್ಗಡೆ
ಹೆಗ್ಗಡೆಗಳ
ಹೆಗ್ಗಡೆಗಳನ್ನು
ಹೆಗ್ಗಡೆಗಳಿಗೆ
ಹೆಗ್ಗಡೆಗಳಿತ್ತು
ಹೆಗ್ಗಡೆಗಳಿದ್ದರು
ಹೆಗ್ಗಡೆಗಳಿದ್ದರೆಂದು
ಹೆಗ್ಗಡೆಗಳು
ಹೆಗ್ಗಡೆಗಳೂ
ಹೆಗ್ಗಡೆಗಳೆಂದೂ
ಹೆಗ್ಗಡೆಗಳೆಲ್ಲರೂ
ಹೆಗ್ಗಡೆಗೆ
ಹೆಗ್ಗಡೆತಿಲೆನಾಯಕ
ಹೆಗ್ಗಡೆದೇವ
ಹೆಗ್ಗಡೆಪೆರ್ಗ್ಗಡೆಪೆರಾಳ್ಕೆ
ಹೆಗ್ಗಡೆಮೇಲಾಳಿಕೆ
ಹೆಗ್ಗಡೆಯ
ಹೆಗ್ಗಡೆಯರು
ಹೆಗ್ಗಡೆಯವರ
ಹೆಗ್ಗಡೆಯವರೆಗೆ
ಹೆಗ್ಗಡೆಯಾಗಿದ್ದ
ಹೆಗ್ಗಡೆಯಾಗಿದ್ದನೆಂದು
ಹೆಗ್ಗಡೆಯು
ಹೆಗ್ಗಪ್ಪ
ಹೆಗ್ಗಪ್ಪಗಳು
ಹೆಗ್ಗೆಡಯೇ
ಹೆಚ್ಚಾಗಿ
ಹೆಚ್ಚಾಗಿದ್ದುಕೊಂಡು
ಹೆಚ್ಚಾಗಿರುವುದರಿಂದ
ಹೆಚ್ಚಾಯಿತು
ಹೆಚ್ಚಿನ
ಹೆಚ್ಚಿನದಾಗಿತ್ತು
ಹೆಚ್ಚಿನವರು
ಹೆಚ್ಚು
ಹೆಚ್ಚುಕಡಿಮೆ
ಹೆಚ್ಚುವರಿ
ಹೆಚ್ಚುವರಿಯಾಗಿ
ಹೆಜ್ಜಾಜಿ
ಹೆಜ್ಜುಂಕವನ್ನು
ಹೆಡತಲೆಯ
ಹೆಣಗಾಡಿ
ಹೆಣ್ಣಾನೆಗಳನ್ನು
ಹೆಣ್ಣುಮಕ್ಕಳಾದ
ಹೆಣ್ಣುಮಕ್ಕಳಿದ್ದರು
ಹೆಣ್ಣುಮಕ್ಕಳೂ
ಹೆಣ್ಣುಸೆರೆ
ಹೆತ್ತಗೋನಹಳ್ಳಿ
ಹೆದ್ದಾರಿಯ
ಹೆದ್ದೊರೆಯಾದಿಯಾಗಿ
ಹೆಬ್ಬಕವಾಡಿಯನ್ನು
ಹೆಬ್ಬಟ್ಟದ
ಹೆಬ್ಬಟ್ಟವು
ಹೆಬ್ಬಟ್ಟು
ಹೆಬ್ಬಳ್ಳದ
ಹೆಬ್ಬಾಗಿಲಿ
ಹೆಬ್ಬಾಗಿಲಿನಲ್ಲಿ
ಹೆಬ್ಬಾಗಿಲು
ಹೆಬ್ಬಾರುವ
ಹೆಬ್ಬಾರುವರ
ಹೆಬ್ಬಾಳ
ಹೆಬ್ಬಾಳು
ಹೆಬ್ಬಾವು
ಹೆಬ್ಬಾವುಗಳಿದ್ದವು
ಹೆಬ್ಬಾವುಗಳೂ
ಹೆಬ್ಬಿದರವಾಡಿಯ
ಹೆಬ್ಬಿದಿರವಾಡಿಯ
ಹೆಬ್ಬಿದಿರವಾಡಿಯಲ್ಲಿ
ಹೆಬ್ಬಿದಿರೂರ್ವಾಡಿಯಲಿ
ಹೆಬ್ಬಿದಿರೂರ್ವಾಡಿಯಲ್ಲಿಇದೇ
ಹೆಬ್ಬಿದಿರೂರ್ವಾಡಿಯೇ
ಹೆಬ್ಬೆಟ್ಟುನಾಡು
ಹೆಬ್ಬೊಳಲ
ಹೆಬ್ಬೊಳಲಇಂದಿನ
ಹೆಮೆಯದಂಡನಾಥ
ಹೆಮ್ಮಣ್ಣ
ಹೆಮ್ಮನಹಳ್ಳಿ
ಹೆಮ್ಮನಿಂದ
ಹೆಮ್ಮಯ್ಯನ
ಹೆಮ್ಮಯ್ಯನನ್ನು
ಹೆಮ್ಮಯ್ಯನೆಂದಿದೆ
ಹೆಮ್ಮರಗಾಲ
ಹೆಮ್ಮವ್ವೆ
ಹೆಮ್ಮಾಡಿಯಣ್ಣನು
ಹೆಮ್ಮೆಪಡತಕ್ಕಂತಹ
ಹೆಮ್ಮೆಯನಾಯಕಂ
ಹೆಮ್ಮೆಯನಾಯಕನನ್ನು
ಹೆರಾಸ್
ಹೆರುಳಹಳ್ಳಿ
ಹೆರ್ಮ್ಮಾಡಿದೇವನೆಂಬುವವನಿಗೆ
ಹೆಸರ
ಹೆಸರನ್ನು
ಹೆಸರನ್ನೂ
ಹೆಸರನ್ನೇ
ಹೆಸರಲಿ
ಹೆಸರಲು
ಹೆಸರಾಂತ
ಹೆಸರಾಗಿತ್ತು
ಹೆಸರಾಗಿದ್ದನು
ಹೆಸರಾಗಿದ್ದಿರಬಹುದೆಂದು
ಹೆಸರಾಗಿರಬಹುದು
ಹೆಸರಾಯಿತು
ಹೆಸರಿಂದ
ಹೆಸರಿಗೆ
ಹೆಸರಿಟ್ಟನು
ಹೆಸರಿಟ್ಟನೆಂದೂ
ಹೆಸರಿಟ್ಟರು
ಹೆಸರಿಟ್ಟು
ಹೆಸರಿಟ್ಟುಕೊಂಡು
ಹೆಸರಿಟ್ಟುಕೊಳ್ಳುತ್ತಿದ್ದರು
ಹೆಸರಿಡಲಾಗಿತ್ತು
ಹೆಸರಿಡುವುದಕ್ಕೆ
ಹೆಸರಿತ್ತು
ಹೆಸರಿತ್ತೆಂದು
ಹೆಸರಿದೆ
ಹೆಸರಿದ್ದರೂ
ಹೆಸರಿದ್ದು
ಹೆಸರಿನ
ಹೆಸರಿನಲಿ
ಹೆಸರಿನಲ್ಲಿ
ಹೆಸರಿನಲ್ಲಿಯೇ
ಹೆಸರಿನಲ್ಲೇ
ಹೆಸರಿನಿಂದ
ಹೆಸರಿರಬಹುದು
ಹೆಸರಿರುವ
ಹೆಸರಿಲ್ಲ
ಹೆಸರಿಸಲಾಗಿದೆ
ಹೆಸರಿಸಿದೆ
ಹೆಸರಿಸಿಲ್ಲ
ಹೆಸರಿಸುತ್ತದೆ
ಹೆಸರಿಸುವ
ಹೆಸರು
ಹೆಸರುಗಳ
ಹೆಸರುಗಳನ್ನು
ಹೆಸರುಗಳಾಗಿವೆ
ಹೆಸರುಗಳಿಂದ
ಹೆಸರುಗಳಿದ್ದು
ಹೆಸರುಗಳಿಸಿದ್ದನು
ಹೆಸರುಗಳಿಸಿರಬಹುದು
ಹೆಸರುಗಳು
ಹೆಸರುನ್ನು
ಹೆಸರುಬಂದಿದೆ
ಹೆಸರುಳ್ಳ
ಹೆಸರುವಾಸಿಯಾದನು
ಹೆಸರೂ
ಹೇಗಾಯಿತೆಂಬುದನ್ನು
ಹೇಗಿದ್ದರೂ
ಹೇಮಗಿರಿಯ
ಹೇಮದ
ಹೇಮಾದ್ರಿಯೇ
ಹೇಮಾವತಿ
ಹೇಮಾವತಿಯ
ಹೇಮೇಶ್ವರ
ಹೇಮೇಸ್ವರದೇವರ
ಹೇರಿಗೆ
ಹೇರಿನ
ಹೇರಿನಷ್ಟು
ಹೇಳಬಹದು
ಹೇಳಬಹುದ
ಹೇಳಬಹುದಾಗಿದೆ
ಹೇಳಬಹುದು
ಹೇಳಬಹುದೆಂದು
ಹೇಳಬೇಕಾಗುತ್ತದೆ
ಹೇಳಲಾಗಿದೆ
ಹೇಳಲಾಗುತ್ತದೆ
ಹೇಳಲಾಗುತ್ತಿತ್ತು
ಹೇಳಲಾದ
ಹೇಳಲು
ಹೇಳಹುದು
ಹೇಳಿ
ಹೇಳಿಕಳುಹಿಸಿದನಂತೆ
ಹೇಳಿಕಳುಹಿಸಿರಬಹುದು
ಹೇಳಿಕೆಗಳನ್ನು
ಹೇಳಿಕೆಗಳು
ಹೇಳಿಕೆಯನ್ನು
ಹೇಳಿಕೆಯನ್ನೂ
ಹೇಳಿಕೊಂಡಿದ್ದರೂ
ಹೇಳಿಕೊಂಡಿದ್ದಾನೆ
ಹೇಳಿಕೊಂಡಿದ್ದಾರೆ
ಹೇಳಿಕೊಂಡಿದ್ದು
ಹೇಳಿಕೊಂಡಿರಬಹುದು
ಹೇಳಿಕೊಂಡು
ಹೇಳಿಕೊಳ್ಳುತ್ತಾ
ಹೇಳಿದ
ಹೇಳಿದನು
ಹೇಳಿದರೂ
ಹೇಳಿದಾರೆ
ಹೇಳಿದೆ
ಹೇಳಿದ್ದ
ಹೇಳಿದ್ದರೂ
ಹೇಳಿದ್ದರೆ
ಹೇಳಿದ್ದಾನೆ
ಹೇಳಿದ್ದಾರೆ
ಹೇಳಿದ್ದಾರೆಂದು
ಹೇಳಿದ್ದಾರೋ
ಹೇಳಿದ್ದಾರ್ೆ
ಹೇಳಿದ್ದು
ಹೇಳಿರುವ
ಹೇಳಿರುವಂತೆ
ಹೇಳಿರುವಂತೆಯೇ
ಹೇಳಿರುವುದಕ್ಕೆ
ಹೇಳಿರುವುದನ್ನು
ಹೇಳಿರುವುದರ
ಹೇಳಿರುವುದರಿಂದ
ಹೇಳಿರುವುದಿರಂದ
ಹೇಳಿರುವುದಿಲ್ಲ
ಹೇಳಿರುವುದು
ಹೇಳಿರುವುದೇ
ಹೇಳಿಲ್ಲ
ಹೇಳಿವೆ
ಹೇಳುತ್ತದೆ
ಹೇಳುತ್ತವೆ
ಹೇಳುತ್ತಾ
ಹೇಳುತ್ತಾನೆ
ಹೇಳುತ್ತಾರೆ
ಹೇಳುತ್ತಿದ್ದರು
ಹೇಳುತ್ತಿರುವಂತಿದೆ
ಹೇಳುತ್ತಿರುವಷ್ಟರಲ್ಲಿ
ಹೇಳುವ
ಹೇಳುವಂತೆ
ಹೇಳುವಾಗ
ಹೇಳುವುದಕ್ಕಿಂತ
ಹೇಳುವುದಿಲ್ಲ
ಹೇಳುವುದು
ಹೈದರನ
ಹೈದರನು
ಹೈದರಾಬಾದಿನ
ಹೈದರಾಲಿಯು
ಹೈದರ್
ಹೈದರ್ಅಲಿ
ಹೈದರ್ಅಲಿಖಾನ್
ಹೈದರ್ಅಲಿಯ
ಹೈದರ್ಅಲಿಯು
ಹೈದರ್ನ
ಹೈದರ್ನನ್ನು
ಹೈದರ್ನೊಂದಿಗೆ
ಹೈಹಯ
ಹೈಹಯರ
ಹೊಂಕುಂದದ
ಹೊಂಗನೂರು
ಹೊಂಡರಬಾಳು
ಹೊಂದಲಗೆರೆ
ಹೊಂದಲಗೆರೆಯ
ಹೊಂದಿ
ಹೊಂದಿಕೆಯಾಗದಿರಲು
ಹೊಂದಿಕೊಂಡ
ಹೊಂದಿಕೊಂಡಂತೆ
ಹೊಂದಿಕೊಂಡಹಾಗೆ
ಹೊಂದಿಕೊಂಡಹಾಗೇ
ಹೊಂದಿಕೊಂಡಿತ್ತು
ಹೊಂದಿಕೊಂಡಿದೆ
ಹೊಂದಿಕೊಂಡಿದ್ದರೆ
ಹೊಂದಿಕೊಂಡಿರುವ
ಹೊಂದಿಕೊಂಡಿವೆ
ಹೊಂದಿತು
ಹೊಂದಿದ
ಹೊಂದಿದನು
ಹೊಂದಿದನೆಂದು
ಹೊಂದಿದನೆಂದೂ
ಹೊಂದಿದಾಗ
ಹೊಂದಿದೆ
ಹೊಂದಿದ್ದ
ಹೊಂದಿದ್ದನು
ಹೊಂದಿದ್ದನೆಂದು
ಹೊಂದಿದ್ದನ್ನು
ಹೊಂದಿದ್ದರಿಂದ
ಹೊಂದಿದ್ದರಿಂದಲೇ
ಹೊಂದಿದ್ದರು
ಹೊಂದಿದ್ದರೂ
ಹೊಂದಿದ್ದರೆ
ಹೊಂದಿದ್ದರೆಂದು
ಹೊಂದಿದ್ದವರು
ಹೊಂದಿದ್ದವು
ಹೊಂದಿದ್ದು
ಹೊಂದಿರಬಹುದು
ಹೊಂದಿರಬೇಕು
ಹೊಂದಿರುತ್ತಿದ್ದರೆ
ಹೊಂದಿರುವ
ಹೊಂದಿರುವವರು
ಹೊಂದಿರುವುದನ್ನು
ಹೊಂದಿವೆ
ಹೊಂನಕಹಳ್ಳಿ
ಹೊಂನಯನಹಳ್ಳಿಯನ್ನು
ಹೊಂನಿಸೆಟ್ಟಿಯು
ಹೊಂನೆಯ
ಹೊಂನೇಹಳ್ಳಿ
ಹೊಂಪುರ
ಹೊಇಸಳಮಂಡಲೇ
ಹೊಗರನಾಡಿಗೆ
ಹೊಗರ್ನಾಡಿನ
ಹೊಗರ್ನ್ನಾಡಿನ
ಹೊಗರ್ನ್ನಾಡು
ಹೊಗಳಲಾಗಿದೆ
ಹೊಗಳಲು
ಹೊಗಳಿಕೆಗೆ
ಹೊಗಳಿದೆ
ಹೊಗಳಿದ್ದಾನೆ
ಹೊಗಳಿದ್ದು
ಹೊಗಳಿರುವುದರಿಂದ
ಹೊಗಳುತ್ತವೆ
ಹೊಗಳುತ್ತಿತ್ತೆಂದು
ಹೊಗಳುಭಟ್ಟರಲ್ಲ
ಹೊಗಳುಭಟ್ಟರಾಗಿದ್ದರೆಂದು
ಹೊಡುಕೇಕಟ್ಟಹೊಡೆಘಟ್ಟ
ಹೊಡೆತಕ್ಕೆ
ಹೊಡೆದಟ್ಟಿದನು
ಹೊಡೆದಾಟಕ್ಕೆ
ಹೊಡೆದಾಡಿಕೊಂಡು
ಹೊಡೆದು
ಹೊಡೆದೋಡಿಸುವಲ್ಲಿ
ಹೊಣಕನಹಳ್ಳಿ
ಹೊಣಕನಹಳ್ಳಿಯೋ
ಹೊಣೆಗಾರಿಕೆ
ಹೊಣೆಯನ್ನು
ಹೊತ್ತಿಗಾಗಲೇ
ಹೊತ್ತಿಗೆ
ಹೊತ್ತಿಗೇ
ಹೊತ್ತಿದ್ದ
ಹೊತ್ತು
ಹೊದಕೆ
ಹೊನಗನಹಳ್ಳಿ
ಹೊನಗನಹಳ್ಳಿಯ
ಹೊನಗನಹಳ್ಳಿಯೋ
ಹೊನಗಾನಹಳ್ಳಿ
ಹೊನಗುಂಟಾ
ಹೊನಗುಂದ
ಹೊನ್ನಕಳಸಗಳನ್ನು
ಹೊನ್ನಕಳಸವನ್ನು
ಹೊನ್ನಗೌಡನು
ಹೊನ್ನನ್ನು
ಹೊನ್ನಯನಹಳ್ಳಿಯು
ಹೊನ್ನಯ್ಯ
ಹೊನ್ನಯ್ಯನನ್ನು
ಹೊನ್ನಯ್ಯನಿಂದಲೇ
ಹೊನ್ನಯ್ಯನು
ಹೊನ್ನಲಗೆರೆ
ಹೊನ್ನಲಗೆರೆಗೆ
ಹೊನ್ನಲಗೆರೆಯು
ಹೊನ್ನವ್ವೆ
ಹೊನ್ನಾಜಿ
ಹೊನ್ನಾವರ
ಹೊನ್ನಾವರದ
ಹೊನ್ನಿರಬಹುದು
ಹೊನ್ನಿಸೆಟ್ಟಿ
ಹೊನ್ನು
ಹೊನ್ನುಡಿಗೆಯೂಹೊನ್ನುಡಿಕೆಶಾಸನೋಕ್ತವಾಗಿದೆ
ಹೊನ್ನೂರನ್ನು
ಹೊನ್ನೂರಿನ
ಹೊನ್ನೆಯ
ಹೊನ್ನೆಯನಹಳ್ಳಿ
ಹೊನ್ನೇಗೌಡನು
ಹೊನ್ನೇನಹಳ್ಳಿ
ಹೊನ್ನೇನಹಳ್ಳಿಯಲ್ಲಿ
ಹೊನ್ನೇನಹಳ್ಳಿಯು
ಹೊನ್ನೊಳಗೆ
ಹೊಮ್ಮ
ಹೊಯಿಕು
ಹೊಯಿದು
ಹೊಯಿಸಣ
ಹೊಯಿಸಣದೇಶದ
ಹೊಯಿಸಣರಾಜ್ಯದ
ಹೊಯಿಸಣಾಭಿಧೇ
ಹೊಯಿಸಳ
ಹೊಯಿಸಳದೇವರು
ಹೊಯಿಸಳರಾಜ್ಯದ
ಹೊಯಿಸಳರಾಜ್ಯಲಕ್ಷ್ಮೀಪ್ರಾಕಾರ
ಹೊಯಿಸಳಲೆಂಕ
ಹೊಯಿಸಳೇಶ್ವರ
ಹೊಯ್
ಹೊಯ್ದು
ಹೊಯ್ಸಣ
ಹೊಯ್ಸಣದೇವ
ಹೊಯ್ಸಣದೇಶದ
ಹೊಯ್ಸಣನಾಡು
ಹೊಯ್ಸಣರಾಯ
ಹೊಯ್ಸಣಾಖ್ಯಸ್ಯ
ಹೊಯ್ಸಲನಾಡ
ಹೊಯ್ಸಲನಾಡು
ಹೊಯ್ಸಲಾಹ್ವಯವತ
ಹೊಯ್ಸಳ
ಹೊಯ್ಸಳಕರ್ನಾಟಕ
ಹೊಯ್ಸಳಕಾಲದ
ಹೊಯ್ಸಳಖ್ಯಾತರಂ
ಹೊಯ್ಸಳದೇವನ
ಹೊಯ್ಸಳದೇವನು
ಹೊಯ್ಸಳದೇವರ
ಹೊಯ್ಸಳದೇವರು
ಹೊಯ್ಸಳದೇಶ
ಹೊಯ್ಸಳದೇಶದ
ಹೊಯ್ಸಳದೇಶವನ್ನು
ಹೊಯ್ಸಳದೇಶೇತ್ವಸ್ಮಿನ್
ಹೊಯ್ಸಳನಾಡ
ಹೊಯ್ಸಳನಾಡಾಗಿ
ಹೊಯ್ಸಳನಾಡಿನ
ಹೊಯ್ಸಳನಾಡು
ಹೊಯ್ಸಳನು
ಹೊಯ್ಸಳಮಹಾಸಾಮಂನ್ತ
ಹೊಯ್ಸಳಮಹೀಶ
ಹೊಯ್ಸಳರ
ಹೊಯ್ಸಳರಕಾಲದಿಂದ
ಹೊಯ್ಸಳರನ್ನು
ಹೊಯ್ಸಳರರಾಜ್ಯಕ್ಕೆ
ಹೊಯ್ಸಳರಲ್ಲಿಯೇ
ಹೊಯ್ಸಳರವರೆಗೆ
ಹೊಯ್ಸಳರಾಜ್ಯ
ಹೊಯ್ಸಳರಾಜ್ಯದಲ್ಲಿ
ಹೊಯ್ಸಳರಾಜ್ಯಪಯೋಜಭಾನು
ಹೊಯ್ಸಳರಾಜ್ಯವನ್ನು
ಹೊಯ್ಸಳರಾಜ್ಯಾಧಿಪತಿ
ಹೊಯ್ಸಳರಿಂದ
ಹೊಯ್ಸಳರಿಗೂ
ಹೊಯ್ಸಳರಿಗೆ
ಹೊಯ್ಸಳರು
ಹೊಯ್ಸಳರೆಂದು
ಹೊಯ್ಸಳರೇ
ಹೊಯ್ಸಳವಂಶದ
ಹೊಯ್ಸಳಸಣ್ನೆನಾಡಾಳ್ವ
ಹೊಯ್ಸಳಸಾಮ್ರಾಜ್ಯ
ಹೊಯ್ಸಳಸಾಮ್ರಾಜ್ಯದ
ಹೊಯ್ಸಳಸೆಟಿ
ಹೊಯ್ಸಳಸೆಟ್ಟಿ
ಹೊಯ್ಸಳೇಶ್ವರ
ಹೊಯ್ಸಿಳ
ಹೊಯ್ಸೆಯ
ಹೊಯ್ಸೆಯನಾಯಕನ
ಹೊಯ್ಸೊಳಲು
ಹೊರಗಿನ
ಹೊರಗುತ್ತಿಗೆಯಾಗಿ
ಹೊರಟ
ಹೊರಟಂತೆ
ಹೊರಟನು
ಹೊರಟರು
ಹೊರಟಾಗ
ಹೊರಟಿದೆ
ಹೊರಟಿದ್ದು
ಹೊರಟಿರುವ
ಹೊರಟಿರುವು
ಹೊರಟಿರುವುದು
ಹೊರಟಿವೆ
ಹೊರಡಿಸಲಾಗಿದೆ
ಹೊರಡಿಸಿ
ಹೊರಡಿಸಿದನೆಂದೂ
ಹೊರಡಿಸಿದ್ದರೂ
ಹೊರಡಿಸಿರುವ
ಹೊರಡಿಸುವ
ಹೊರಡುತ್ತದೆ
ಹೊರಡುತ್ತಿದ್ದರು
ಹೊರಡುವ
ಹೊರತಾದ
ಹೊರತು
ಹೊರತುಪಡಿಸಿ
ಹೊರತುಪಡಿಸಿದರೆ
ಹೊರದೂಡಲು
ಹೊರದೂಡುವಲ್ಲಿ
ಹೊರವಲೆನಾಡಿನ
ಹೊರವಾರು
ಹೊರವೃತ್ತಿಯ
ಹೊರುವ
ಹೊಲಕುಪ್ಪೆ
ಹೊಲಗನಹಳ್ಳಿ
ಹೊಲಗನಹಳ್ಳಿಯನ್ನು
ಹೊಲಗಾಹು
ಹೊಲದಲ್ಲಿರುವ
ಹೊಲವನ್ನು
ಹೊಲಿಯಜಂಗುಲಿಹುಲಿಯ
ಹೊಲೆಮಗ್ಗ
ಹೊಲೆಸುಂಕ
ಹೊಳಲಕೆರೆಗೆ
ಹೊಳಲಕೆರೆಯ
ಹೊಳಲಕೆರೆಯಲ್ಲಿ
ಹೊಳಲಕೆರೆಯವರೆಗೆ
ಹೊಳಲಗುಂದ
ಹೊಳಲಯದ
ಹೊಳಲಯನಾಡ
ಹೊಳಲಯನಾಡುಗಳು
ಹೊಳಲಿನ
ಹೊಳಲಿಯ
ಹೊಳಲು
ಹೊಳೆ
ಹೊಳೆನರಸಿಪುರ
ಹೊಳೆನರಸೀಪುರ
ಹೊಳೆಯ
ಹೊಳೆಯು
ಹೊಸ
ಹೊಸಕನ್ನಂಬಾಡಿ
ಹೊಸಕೆರಯೂ
ಹೊಸಕೋಟೆ
ಹೊಸಕೋಟೆಯ
ಹೊಸಗ್ರಾಮದ
ಹೊಸಣಮಂಡಲಧೃತಃರಾಜಶ್ರೀ
ಹೊಸದಾಗಿ
ಹೊಸದುರ್ಗ
ಹೊಸನಾಡು
ಹೊಸಪಟ್ಟಣ
ಹೊಸಪಟ್ಟಣದ
ಹೊಸಪಟ್ಟಣದಲ್ಲಿ
ಹೊಸಪಟ್ಟಣದಿಂದ
ಹೊಸಪಟ್ಟಣವನ್ನು
ಹೊಸಪಟ್ಟಣವನ್ನೇ
ಹೊಸಪಟ್ಟಣವಾಗಿರುವ
ಹೊಸಪಟ್ಟಣವೆಂದಾಯಿತು
ಹೊಸಪಟ್ಟಣವೆಂಬ
ಹೊಸಪಟ್ಟಣವೇ
ಹೊಸಪುರ
ಹೊಸಬಿರುದರ
ಹೊಸಬೂದನೂರು
ಹೊಸಮಲೆ
ಹೊಸಲಹೊಳಲು
ಹೊಸವಾಡದ
ಹೊಸವೀಡು
ಹೊಸವೊಳಲ
ಹೊಸಹಳ್ಳಿ
ಹೊಸಹಳ್ಳಿಪುರ
ಹೊಸಹಳ್ಳಿಯನ್ನು
ಹೊಸಹೊಳಲ
ಹೊಸಹೊಳಲಿಗೆ
ಹೊಸಹೊಳಲಿನ
ಹೊಸಹೊಳಲು
ಹೊಸಹೊಳಲೇ
ಹೊಸೂರು
ಹೋಗಬೇಕಾಯಿತು
ಹೋಗಲಾಡಿಸಿದರೆಂಬುದು
ಹೋಗಲು
ಹೋಗಿ
ಹೋಗಿದೆ
ಹೋಗಿದ್
ಹೋಗಿದ್ದನು
ಹೋಗಿದ್ದನೆಂದು
ಹೋಗಿದ್ದು
ಹೋಗಿಬಂದನೆಂಬುದು
ಹೋಗಿರಬಹುದು
ಹೋಗಿರಬಹುದೆಂದು
ಹೋಗಿರಲು
ಹೋಗಿರುವ
ಹೋಗಿವೆ
ಹೋಗು
ಹೋಗುತ್ತದೋ
ಹೋಗುತ್ತಾನೆ
ಹೋಗುತ್ತಿದ್ದ
ಹೋಗುತ್ತಿದ್ದರು
ಹೋಗುತ್ತಿದ್ದರೆ
ಹೋಗುತ್ತಿದ್ದಾಗ
ಹೋಗುವ
ಹೋತನಡಕೆಯ
ಹೋತ್ತಮನಂ
ಹೋದ
ಹೋದನು
ಹೋದರು
ಹೋದರೂ
ಹೋದರೆಂದು
ಹೋದಾಗ
ಹೋಬಲಾ
ಹೋಬಳಿ
ಹೋಬಳಿಗಳ
ಹೋಬಳಿಗಳು
ಹೋಬಳಿಗೂ
ಹೋಬಳಿಗೆ
ಹೋಬಳಿಯ
ಹೋಬಳಿಯಂತಹ
ಹೋಬಳಿಯನ್ನು
ಹೋಬಳಿಯಲ್ಲಿ
ಹೋಬಳಿಯಲ್ಲಿದೆ
ಹೋಬಳಿಯಲ್ಲಿದ್ದರೆ
ಹೋಬಳಿಯಾಗಿ
ಹೋಯಿತಂತೆ
ಹೋಯಿತು
ಹೋರಾಟ
ಹೋರಾಟಕ್ಕೆ
ಹೋರಾಟಗಳನ್ನು
ಹೋರಾಟಗಳಲ್ಲಿ
ಹೋರಾಟದಲ್ಲಿ
ಹೋರಾಟನಡೆಸಿ
ಹೋರಾಟವನ್ನು
ಹೋರಾಡಿ
ಹೋರಾಡಿದ
ಹೋರಾಡಿದಂತೆ
ಹೋರಾಡಿದನೆಂದು
ಹೋರಾಡಿದನೆಂದೂ
ಹೋರಾಡಿದರು
ಹೋರಾಡಿದರೆ
ಹೋರಾಡಿದವರಲ್ಲಿ
ಹೋರಾಡಿದಾಗ
ಹೋರಾಡಿದುದನ್ನು
ಹೋರಾಡಿರಬಹುದು
ಹೋರಾಡುತ್ತಾ
ಹೋರಾಡುತ್ತಿದ್ದ
ಹೋರಾಡುತ್ತಿದ್ದರು
ಹೋರಾಡುತ್ತಿದ್ದರೆಂದು
ಹೋರಾಡುವಲ್ಲಿ
ಹೋರಾಡುವಾಗ
ಹೋರಿನಿದೇವನಿಗೆ
ಹೋರ್ಷಣಾಹ್ವಯ
ಹೋಲಿಸಬಹುದೆಂದು
ಹೋಲಿಸಲಾಗಿದೆ
ಹೋಲಿಸಿರುವುದು
ಹೋಲುತ್ತದೆಂದೂ
ಹೋಸಣ
ಹೋಸಣದೇಶದ
ಹೋಸಣಾಖ್ಯ
ಹೋಸಲನಾಡ
ಹೋಸಲನಾಡಿನ
ಹೋಸಲನಾಡು
ಹೋಸಲನಾಡುಹೊಯ್ಸಳ
ಹ್ಯಾರಿಸ್
ಹ್ರಸ್ವರೂಪ
ಹ್ವೈಸಣ
ೞೋಡವನಾಯಕ
ೞೋಡವನಾಯಕರು
}
