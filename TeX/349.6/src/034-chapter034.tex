
\chapter{ವೇದಾಂತ ಮತ್ತು ಹಕ್ಕು ಬಾಧ್ಯತೆಗಳು \protect\footnote{\enginline{* C.W, Vol. I, P. 417}}\\(ಲಂಡನ್ನಿನಲ್ಲಿ ನೀಡಿದ ಉಪನ್ಯಾಸ)}

ಅದ್ವೈತವೇದಾಂತದ ತಾತ್ತ್ವಿಕ ಭಾಗವನ್ನು ಬಹುಪಾಲು ಪೂರೈಸಿರುವೆವು. ಬಹುಶಃ ತಿಳಿದುಕೊಳ್ಳುವುದಕ್ಕೆ ಅತಿ ಕಷ್ಟವಾದ ಭಾಗವೊಂದು ಉಳಿದಿದೆ. ಅದ್ವೈತ ವೇದಾಂತದ ಪ್ರಕಾರ ನಮ್ಮ ಸುತ್ತಲಿರುವುದೆಲ್ಲ, ಇಡೀ ಬ್ರಹ್ಮಾಂಡವೆ, ಆ ನಿರಪೇಕ್ಷವಾದ ವಸ್ತುವಿನ ವಿಕಾಸ. ಇದನ್ನು ಸಂಸ್ಕೃತದಲ್ಲಿ ಬ್ರಹ್ಮ ಎನ್ನುವರು. ಬ್ರಹ್ಮನೇ ಈ ಪ್ರಕೃತಿಯಾಗಿರುವನು. ಆದರೆ ಇಲ್ಲೊಂದು ತೊಡಕು ಬರುವುದು, ನಿರಪೇಕ್ಷವಾದ ಬ್ರಹ್ಮ ಬದಲಾಗುವುದು ಹೇಗೆ? ಈ ಬ್ರಹ್ಮವನ್ನು ಬದಲಾಗುವಂತೆ ಮಾಡುವುದು ಯಾವುದು? ತತ್ತ್ವತಃ ಬ್ರಹ್ಮ ಅವಿಕಾರಿ. ಅವಿಕಾರಿಯ ವಿಕಾರವೆಂಬುದು ವಿರೋಧಾಭಾಸ. ಸಗುಣ ದೇವರನ್ನು ನಂಬುವವರಿಗೂ ಒಂದು ತೊಡಕು ಇದೆ. ಈ ಸೃಷ್ಟಿ ಹೇಗೆ ಬಂತು? ಇದು ಶೂನ್ಯದಿಂದ ಬಂದಿರಲಾರದು. ಇದು ಯುಕ್ತಿಗೆ ವಿರೋಧ. ಶೂನ್ಯದಿಂದ ಎಂದೂ ಯಾವುದೂ ಬರಲಾರದು. ಕಾರ್ಯವೇ ಕಾರಣದ ಮತ್ತೊಂದು ಅವಸ್ಥೆ. ಬೀಜದಿಂದ ದೊಡ್ಡ ಮರ ಬೆಳೆಯುವುದು. ಮರವೆಂದರೆ, ಬೀಜ+ಅದು ತೆಗೆದುಕೊಂಡ ನೀರು ಗಾಳಿ ಇತ್ಯಾದಿ. ಅದು ತೆಗೆದುಕೊಂಡ ಗಾಳಿಯನ್ನು ಮತ್ತು ನೀರನ್ನು ಅಳೆಯುವುದಕ್ಕೆ ಒಂದು ಉಪಕರಣವಿದ್ದಿದ್ದರೆ ಅದು ಕಾರ್ಯವಾದ ಮರವೇ ಎಂಬುದನ್ನು ತೋರಿಸಬಹುದಾಗಿತ್ತು. ಆಧುನಿಕ ವಿಜ್ಞಾನ ನಿಸ್ಸಂಶಯವಾಗಿ ಕಾರ್ಯವು ಕಾರಣದ ಬೇರೊಂದು ಅವಸ್ಥೆ ಎಂಬುದನ್ನು ತೋರಿದೆ. ಕಾರಣವು ಭಿನ್ನರೂಪ ತಾಳಿ ಕಾರ್ಯವಾಗುವುದು. ಯಾವ ಕಾರಣವೂ ಇಲ್ಲದೆ ಪ್ರಪಂಚ ಆಯಿತು ಎಂಬ ತೊಡಕಿನಿಂದ ಪಾರಾಗಬೇಕಾಗಿದೆ. ಬ್ರಹ್ಮನೇ ಈ ಪ್ರಪಂಚ ಆಗಿರುವನು ಎಂದು ಒಪ್ಪಿಕೊಳ್ಳಲೇಬೇಕಾಗಿದೆ.

ನಾವು ಒಂದು ಕಷ್ಟದಿಂದ ಪಾರಾಗಿ ಮತ್ತೊಂದು ಕಷ್ಟದ ಬಲೆಗೆ ಬಿದ್ದಿರುವೆವು. ಎಲ್ಲಾ ಸಿದ್ದಾಂತಗಳೂ ದೇವರು ಎಂದರೆ ಅವಿಕಾರಿ ಎಂದು ಒಪ್ಪಿಕೊಳ್ಳುವುವು. ದೇವರನ್ನು ಹುಡುಕುವಾಗಲೆಲ್ಲ ಮಾನವನಲ್ಲಿ ಇದ್ದ ಏಕಮಾತ್ರ ಭಾವನೆ ಸ್ವಾತಂತ್ರ್ಯ ಎಂಬುದನ್ನು ಚಾರಿತ್ರಿಕವಾಗಿ ತೋರಿರುವೆವು. ಅನ್ವೇಷಣೆ ಇನ್ನೂ ಅತಿ ಕೆಳಗಿನ ಮಟ್ಟದಲ್ಲಿರುವಾಗಲೂ ಅದು ಸ್ವಾತಂತ್ರ್ಯಕ್ಕಾಗಿ ಪ್ರಯತ್ನಿಸುತ್ತಿರುತ್ತದೆ. ಸ್ವಾತಂತ್ರ್ಯ ಮತ್ತು ನಿರ್ವಿಕಾರ ಭಾವನೆ ಎರಡೂ ಒಂದೇ. ಸ್ವತಂತ್ರವಾಗಿರುವುದು ಮಾತ್ರ ಎಂದಿಗೂ ಬದಲಾಗುವುದಿಲ್ಲ. ನಿರ್ವಿಕಾರಿಯಾಗಿರುವುದು ಮಾತ್ರ ಸ್ವತಂತ್ರ. ಏಕೆಂದರೆ ಬದಲಾವಣೆ ಯಾವುದೊ ಒಂದು ಹೊರಗಿನ ವಸ್ತುವಿನಿಂದ ಆಗುವುದು; ಇಲ್ಲವೆ ತಾನೇ ಸುತ್ತಲಿನ ವಾತಾವರಣಕ್ಕಿಂತ ಬಲವಾಗಿರುವುದರಿಂದ ತನ್ನಲ್ಲಿಯೇ ಬದಲಾವಣೆಯಾಗುವುದು. ಬದಲಾಗುವುದೆಲ್ಲ ಯಾವುದೋ ಒಂದು ಕಾರಣ ಅಥವಾ ಅನೇಕ ಕಾರಣಗಳಿಂದ ಬದ್ಧವಾಗಿದೆ. ಆದಕಾರಣ ಅದು ನಿರ್ವಿಕಾರಿಯಾಗಲಾರದು. ದೇವರೇ ಪ್ರಪಂಚವಾಗಿದ್ದರೆ ಅವನು ನಮ್ಮ ಕಣ್ಣೆದುರಿಗೆ ಇರುವನು, ಅವನು ಬದಲಾಗಿರುವನು. ಅನಂತಬ್ರಹ್ಮ ನಮ್ಮ ಕಣ್ಣೆದುರಿಗೆ ಇರುವ ಸಾಂತ ಪ್ರಪಂಚವಾದರೆ ಅವನ ಅನಂತತೆಯಲ್ಲಿ ಇಷ್ಟು ಕಡಮೆ ಆದಂತೆ ಆಯಿತು. ಆದಕಾರಣ ಬ್ರಹ್ಮವು ಅನಂತದಿಂದ ವಿಶ್ವವನ್ನು ತೆಗೆದಷ್ಟು ಆಗುವನು. ಬದಲಾಗುವ ದೇವರು ದೇವರೆ ಅಲ್ಲ. ಪ್ರಪಂಚವೇ ದೇವರು ಎಂಬ ಸಿದ್ಧಾಂತದಿಂದ ಪಾರಾಗುವುದಕ್ಕೆ ವೇದಾಂತಿಗಳ ಬಳಿ ಧೈರ್ಯವಾದ ಒಂದು ಸಿದ್ದಾಂತವಿದೆ. ಅದೇ ನಾವು ತಿಳಿದಿರುವ ಮತ್ತು ಆಲೋಚಿಸುವ ಪ್ರಪಂಚ ಇಲ್ಲವೆನ್ನುವುದು. ನಿರ್ವಿಕಾರವು ಬದಲಾಗಿಲ್ಲ. ಈ ಪ್ರಪಂಚವೆಲ್ಲ ಒಂದು ತೋರಿಕೆ, ಸತ್ಯವಲ್ಲ. ಭಿನ್ನ ಭಿನ್ನ ಭಾಗಗಳು, ಅಲ್ಪವಸ್ತುಗಳು. ಈ ವ್ಯತ್ಯಾಸಗಳೆಲ್ಲ ಕೇವಲ ತೋರಿಕೆಯವು. ಇವು ವಸ್ತುವಿನ ಸ್ವಭಾವದಲ್ಲಿಲ್ಲ. ಬ್ರಹ್ಮ ಎಂದಿಗೂ ಪ್ರಪಂಚವಾಗಿಲ್ಲ, ಅವನೆಂದಿಗೂ ಬದಲಾಗಿಲ್ಲ. ನಾವು ಬ್ರಹ್ಮನನ್ನು ಪ್ರಪಂಚದಂತೆ ಕಾಣುತ್ತಿರುವೆವು; ಏಕೆಂದರೆ ನಾವು ದೇಶ-ಕಾಲ-ನಿಮಿತ್ತಗಳ ಮೂಲಕ ನೋಡಬೇಕಾಗಿರುವುದರಿಂದ. ಈ ತೋರಿಕೆಯ ವ್ಯತ್ಯಾಸಕ್ಕೆ ಕಾರಣ ದೇಶ-ಕಾಲ-ನಿಮಿತ್ತಗಳು. ಆದರೆ ನಿಜವಾಗಿಯೂ ಈ ವ್ಯತ್ಯಾಸವಿಲ್ಲ. ಇದೊಂದು ಧೈರ್ಯವಾದ ಸಿದ್ದಾಂತವೇನೋ ನಿಜ. ನಾವು ಈ ಸಿದ್ದಾಂತವನ್ನು ಮತ್ತೂ ಸ್ಪಷ್ಟವಾಗಿ ವಿವರಿಸಬೇಕಾಗಿದೆ. ಇದು ಸಾಧಾರಣವಾಗಿ ಭಾವಿಸುವ ಭಾವಸತ್ತಾವಾದ (idealism) ಅಲ್ಲ. ಇದು ಪ್ರಪಂಚ ಇಲ್ಲ ಎಂದು ಹೇಳುವುದಿಲ್ಲ, ಅದು ಇದೆ ಆದರೆ ನಾವು ಈಗ ಭಾವಿಸುವಂತೆ ಇಲ್ಲ. ಇದನ್ನು ವಿವರಿಸುವುದಕ್ಕೆ ಅದ್ವೈತ ವೇದಾಂತಿಗಳು ಕೊಡುವ ಉದಾಹರಣೆ ಪ್ರಖ್ಯಾತವಾಗಿದೆ. ರಾತ್ರಿಯಲ್ಲಿ ಒಂದು ಮೋಟು ಮರವನ್ನು ಕೆಲವು ಮೂಢರು ದೆವ್ವವೆಂದು ತಿಳಿಯುವರು; ಕಳ್ಳನು ಪೋಲೀಸಿನವನೆಂದು ತಿಳಿಯುವನು, ಸ್ನೇಹಿತನಿಗಾಗಿ ಕಾಯುತ್ತಿರುವವನಿಗೆ ಅದು ಸ್ನೇಹಿತನಂತೆ ಕಾಣುವುದು. ಆದರೆ ಆ ಮೋಟುಮರ ಮಾತ್ರ ಬದಲಾಗಲಿಲ್ಲ. ಆದರೆ ಬದಲಾವಣೆಯೆಲ್ಲ ನೋಡುವವರ ಮನಸ್ಸಿನಲ್ಲಿ ಆದದ್ದು. ನಾವು ಇದನ್ನು ಮನಶ್ಶಾಸ್ತ್ರದ ಮೂಲಕ ಚೆನ್ನಾಗಿ ಅರ್ಥಮಾಡಿಕೊಳ್ಳಬಲ್ಲೆವು. ನಮ್ಮಿಂದ ಹೊರಗೆ ಏನೋ ಇದೆ. ನೈಜಸ್ವಭಾವ ನಮಗೆ ಗೊತ್ತಿಲ್ಲ, ಗೊತ್ತಾಗುವಂತೆಯೂ ಇಲ್ಲ. ಅದನ್ನು x ಎಂದು ಕರೆಯೋಣ. ನಮ್ಮ ಒಳಗೆ ಒಂದು ಇದೆ. ಅದೂ ಕೂಡ ನಮಗೆ ಗೊತ್ತಿಲ್ಲ ಮತ್ತು ಗೊತ್ತಾಗುವಂತೆಯೂ ಇಲ್ಲ. ಅದನ್ನು y ಎಂದು ಕರೆಯೋಣ. ನಮಗೆ ಗೊತ್ತಿರುವುದು x+y ಗಳ ಸಮ್ಮಿಶ್ರಣ. ನಮಗೆ ತಿಳಿದಿರುವುದಕ್ಕೆ ಎರಡು ಭಾಗಗಳು ಇರಬೇಕು. ಅದೇ ಹೊರಗಿನ x ಮತ್ತು ಒಳಗಿನ y. ನಮಗೆ ಗೊತ್ತಿರುವುದೇ x+y. ಪ್ರಪಂಚದಲ್ಲಿರುವ ಪ್ರತಿಯೊಂದು ರೂಪವೂ ಸ್ವಲ್ಪ ಬಾಹ್ಯಪ್ರಪಂಚಕ್ಕೆ ಸೇರಿದ್ದು, ಸ್ವಲ್ಪ ನಮ್ಮದು. ವೇದಾಂತ ಹೇಳುವುದೇ ಈ x ಮತ್ತು y ಗಳು ಒಂದೇ ಎಂದು.

ಕೆಲವು ಪಾಶ್ಚಾತ್ಯ ತತ್ತ್ವಜ್ಞರು ಕೂಡ ಇದೇ ನಿರ್ಧಾರಕ್ಕೆ ಬಂದಿರುವರು. ಅವರಲ್ಲಿ ಹರ್ಬರ್ಟ್ ಸ್ಪೆನ್ಸರ್ ಮತ್ತು ಇತರ ಆಧುನಿಕ ತತ್ತ್ವಜ್ಞರು ಇದ್ದಾರೆ. ಒಂದು ಹೂವಿನಲ್ಲಿ ವಿಕಾಸವಾಗುತ್ತಿರುವ ಶಕ್ತಿಯೇ ನನ್ನ ಚೇತನದ ಅಂತರಾಳದಲ್ಲಿಯೂ ವ್ಯಕ್ತವಾಗುತ್ತಿರುವುದು ಎಂದರೆ ವೇದಾಂತಿಗಳು ಬೋಧಿಸುವುದೂ ಇದೇ ಭಾವನೆಯನ್ನೇ. ಅಂದರೆ ಬಾಹ್ಯ ಮತ್ತು ಆಂತರಿಕ ಸತ್ಯಗಳು ಒಂದೇ ಎಂಬುದು. ಕೇವಲ ಭಿನ್ನತೆಯಿಂದ ಮಾತ್ರ ಬಾಹ್ಯ ಮತ್ತು ಆಂತರಿಕ ಎಂಬ ಭಾವನೆಗಳು ಇರುವುವು, ವಸ್ತುಗಳಲ್ಲಿ ಅಲ್ಲ. ಉದಾಹರಣೆಗೆ ನಮಗೆ ಮತ್ತೊಂದು ಇಂದ್ರಿಯ ಬಂದರೆ ಇಡೀ ಜಗತ್ತು ಬದಲಾಗುವುದು. ದೃಕ್ ದೃಶ್ಯವನ್ನು ಬದಲಾಯಿಸುವುದು ಎಂಬುದನ್ನು ಇದು ತೋರುವುದು. ನಾನು ಬದಲಾದರೆ ಬಾಹ್ಯಪ್ರಪಂಚವೂ ಬದಲಾಗುವುದು. ಆದಕಾರಣವೆ ವೇದಾಂತ ಈ ನಿರ್ಧಾರಕ್ಕೆ ಬರುವುದು: ನಾನು ನೀವು ಮತ್ತು ಪ್ರಪಂಚದಲ್ಲಿರುವ ಎಲ್ಲವೂ, ಅಂಶಗಳಲ್ಲ, ನಾವೆಲ್ಲ ಅದೇ ಪೂರ್ಣ ಆಗಿದ್ದೇವೆ. ನೀವೇ ಆ ಪೂರ್ಣ ನಿರಪೇಕ್ಷ, ಅದರಂತೆಯೇ ಎಲ್ಲರೂ ಕೂಡ. ಏಕೆಂದರೆ ಇಲ್ಲಿ ಅಂಶದ ಭಾವನೆಯೇ ಬರುವುದಿಲ್ಲ. ಈ ಭಿನ್ನತೆಗಳು, ನಿರ್ಬಂಧಗಳು ಕೇವಲ ತೋರಿಕೆಗೆ, ಅವು ವಸ್ತುತಃ ಇಲ್ಲ. ನಾನು ಪೂರ್ಣ, ಪರಿಶುದ್ದ, ನಾನೆಂದೂ ಬದ್ದನಾಗಿರಲಿಲ್ಲ. ನೀನು ಬದ್ದನೆಂದು ಭಾವಿಸಿದರೆ ಬದ್ದನಾಗಿಯೇ ಉಳಿಯುವೆ, ಮುಕ್ತನೆಂದು ಭಾವಿಸಿದರೆ ಮುಕ್ತನಾಗುವೆ ಎಂದು ವೇದಾಂತ ಧೈರ್ಯವಾಗಿ ಸಾರುವುದು. ವೇದಾಂತದ ಮುಖ್ಯ ಗುರಿಯೆ, ಆದರ್ಶವೆ, ನಾವು ಹಿಂದೆ ಯಾವಾಗಲೂ ಮುಕ್ತರಾಗಿದ್ದೆವು, ಮತ್ತು ಎಂದೆಂದಿಗೂ ಮುಕ್ತರಾಗಿಯೇ ಇರುವೆವು ಎಂಬುದನ್ನು ತೋರುವುದಾಗಿದೆ. ನಾವೆಂದಿಗೂ ಬದಲಾಗುವುದಿಲ್ಲ, ನಾವೆಂದಿಗೂ ಜನಿಸಿಲ್ಲ. ಹಾಗಾದರೆ ಈ ಬದಲಾವಣೆಗಳೆಲ್ಲ ಏನು? ಹಾಗಾದರೆ ಈ ಬಾಹ್ಯಪ್ರಪಂಚ ಏನಾಗುವುದು? ಕಾಲ-ದೇಶ-ನಿಮಿತ್ತಗಳಿಂದ ಬದ್ದವಾದ ತೋರಿಕೆಯ ಪ್ರಪಂಚ ಇದು. ಇದನ್ನು ಸಂಸ್ಕೃತದಲ್ಲಿ ವಿವರ್ತವಾದ ಎನ್ನುವರು. ಅಂದರೆ ಪ್ರಕೃತಿಯ ವಿಕಾಸ ಮತ್ತು ಬ್ರಹ್ಮನ ಆವಿರ್ಭಾವ ಎಂದು. ನಿರಪೇಕ್ಷ ಬ್ರಹ್ಮ ಬದಲಾಗುವುದಿಲ್ಲ, ಅದು ಹೊಸದಾಗಿ ವಿಕಾಸವಾಗುವುದಿಲ್ಲ. ಒಂದು ಸಣ್ಣ ಅಮಿಾಬದಲ್ಲಿ ಅನಂತವಾದ ಪೂರ್ಣತೆ ಸುಪ್ತವಾಗಿದೆ. ಅದಕ್ಕೆ ಒಂದು ಅಮಿಾಬದ ಆವರಣ ಇರುವುದರಿಂದ ಅದನ್ನು ಅಮಿಾಬ ಎಂದು ಕರೆಯುತ್ತಾರೆ. ಅಮಿಾಬದಿಂದ ಹಿಡಿದು ಪೂರ್ಣ ಮಾನವನಾಗುವವರೆಗೆ ಆಗುವ ಬದಲಾವಣೆ ಒಳಗೆ ಅಲ್ಲ. ಚೈತನ್ಯ ಯಾವಾಗಲೂ ಒಂದೇ ಸಮನಾಗಿರುವುದು, ಅವಿಕಾರಿಯಾಗಿರುವುದು. ಆವರಣದಲ್ಲಿ ಮಾತ್ರ ಬದಲಾವಣೆ ಆಗುವುದು.

ಇಲ್ಲೊಂದು ತೆರೆ ಇದೆ, ಹೊರಗೆ ಒಂದು ರಮ್ಯವಾದ ದೃಶ್ಯ ಇದೆ. ಈ ತೆರೆಯಲ್ಲಿ ಒಂದು ಸಣ್ಣ ರಂಧ್ರವಿದೆ. ಇದರ ಮೂಲಕ ಹೊರಗೆ ಇರುವುದು ಸ್ವಲ್ಪ ಕಾಣುವುದು. ಈ ರಂಧ್ರ ವಿಸ್ತಾರವಾಗುತ್ತ ಬಂದಂತೆಲ್ಲ ಹೆಚ್ಚು ಹೆಚ್ಚು ದೃಶ್ಯಗಳು ಕಾಣುತ್ತ ಬರುವುವು. ಈ ತೆರೆ ಸಂಪೂರ್ಣ ಮಾಯವಾದರೆ ನಾವು ದೃಶ್ಯದ ಎದುರಿಗೆ ಸಾಕ್ಷಾತ್ತಾಗಿ ನಿಲ್ಲುವೆವು. ಹೊರಗಿರುವ ದೃಶ್ಯವೇ ಆತ್ಮ, ಆದರೆ ಮುಂದೆ ಇರುವ ತೆರೆಯೆ ಕಾಲ-ದೇಶನಿಮಿತ್ತವೆಂಬ ಮಾಯೆ. ಅದರಲ್ಲಿ ಎಲ್ಲೋ ಒಂದು ರಂಧ್ರವಿದೆ, ಅದರ ಮೂಲಕ ನಮಗೆ ದೃಶ್ಯ ಸ್ವಲ್ಪ ಕಾಣುವುದು. ರಂಧ್ರ ದೊಡ್ಡದಾದರೆ ಹೆಚ್ಚು ಕಾಣುವುದು, ಅದು ಸಂಪೂರ್ಣ ಮಾಯವಾದರೆ ನಾನೇ ಆತ್ಮನೆಂಬುದು ಗೊತ್ತಾಗುವುದು. ಆದಕಾರಣ ಬದಲಾವಣೆಗಳು ಪ್ರಕೃತಿಯಲ್ಲಿ ಮಾತ್ರ; ಆತ್ಮನಲ್ಲಿ ಅಲ್ಲ. ಕೊನೆಗೆ ಆತ್ಮವು ಸಂಪೂರ್ಣ ವ್ಯಕ್ತವಾಗುವವರೆಗೆ ಪ್ರಕೃತಿ ವಿಕಾಸವಾಗುತ್ತ ಹೋಗುವುದು. ಎಲ್ಲರಲ್ಲಿಯೂ ಆತ್ಮವಿರುವುದು. ಕೆಲವರಲ್ಲಿ ಅದು ಇತರರಿಗಿಂತ ಹೆಚ್ಚಾಗಿ ಪ್ರಕಟವಾಗುವುದು. ಇಡೀ ಪ್ರಪಂಚ ಒಂದು ಆತ್ಮವನ್ನು ಕುರಿತು ಮಾತನಾಡುವಾಗ ಒಂದು ಮತ್ತೊಂದಕ್ಕಿಂತ ಉತ್ತಮ ಎಂದು ಹೇಳುವುದರಲ್ಲಿ ಅರ್ಥವಿಲ್ಲ. ಆತ್ಮವನ್ನು ಕುರಿತು ಮಾತನಾಡುವಾಗ ಮಾನವನು ಪ್ರಾಣಿಗಿಂತ ಅಥವಾ ಸಸ್ಯಕ್ಕಿಂತ ಮೇಲು ಎಂದರೆ ಅದಕ್ಕೆ ಅರ್ಥವಿಲ್ಲ. ಇಡೀ ಬ್ರಹ್ಮಾಂಡವೇ ಒಂದಾಗಿರುವುದು. ಸಸ್ಯಗಳಲ್ಲಿ ಆತ್ಮನ ಪ್ರಕಟನೆಗೆ ಆತಂಕಗಳು ಹೆಚ್ಚು. ಪ್ರಾಣಿಗಳಲ್ಲಿ ಅದಕ್ಕಿಂತ ಕಡಮೆ, ಮನುಷ್ಯನಲ್ಲಿ ಅದಕ್ಕಿಂತಲೂ ಕಡಮೆ, ಸುಸಂಸ್ಕೃತರಾದ ಮನುಷ್ಯರಲ್ಲಿ ಅದು ಇನ್ನೂ ಕಡಮೆ, ಆಧ್ಯಾತ್ಮಿಕ ಜೀವಗಳಲ್ಲಿ ಅದು ಮತ್ತೂ ಕಡಮೆ, ಪೂರ್ಣಾತ್ಮನಲ್ಲಿ ಅದೆಲ್ಲ ಸಂಪೂರ್ಣ ಮಾಯವಾಗಿದೆ. ನಮ್ಮ ಹೋರಾಟ ಸಾಧನೆಗಳೆಲ್ಲ, ಸುಖದುಃಖಗಳೆಲ್ಲ, ಅಳುನಗುಗಳೆಲ್ಲ, ನಾವು ಮಾಡುವ ಕಾರ್ಯಗಳೆಲ್ಲ ಮತ್ತು ಆಲೋಚನೆಗಳೆಲ್ಲ ನಮ್ಮನ್ನು ಆ ಗುರಿಯೆಡೆಗೆ ಒಯ್ಯುತ್ತಿವೆ. ಅವು ಆವರಣವನ್ನು ಹರಿಯುವುದಕ್ಕೆ, ರಂಧ್ರವನ್ನು ದೊಡ್ಡದು ಮಾಡುವುದಕ್ಕೆ, ಸತ್ಯಕ್ಕೂ ತೋರಿಕೆಯ ಪ್ರಕೃತಿಗೂ ಇರುವ ತೆರೆಯನ್ನು ತೆಳ್ಳಗೆ ಮಾಡುವುದಕ್ಕೆ ಪ್ರಯತ್ನಿಸುತ್ತಿವೆ. ಆತ್ಮವನ್ನು ಮುಕ್ತಗೊಳಿಸುವುದಲ್ಲ ನಮ್ಮ ಕರ್ತವ್ಯ, ಇರುವ ಬಂಧನದಿಂದ ಪಾರಾಗುವುದು ನಮ್ಮ ಕರ್ತವ್ಯ. ಸೂರ್ಯನನ್ನು ಹಲವು ಮೋಡಗಳ ತೆರೆಗಳು ಮುತ್ತಿವೆ. ಆದರೆ ಸೂರ್ಯ ಇದರಿಂದ ಬಾಧಿತನಾಗುವುದಿಲ್ಲ. ಗಾಳಿಯ ಕರ್ತವ್ಯ ಮೋಡವನ್ನು ಚೆದರಿಸುವುದು. ಎಷ್ಟು ಹೆಚ್ಚು ಮೋಡಗಳು ಹೋಗುವುವೋ ಅಷ್ಟು ಚೆನ್ನಾಗಿ ಸೂರ್ಯ ಪ್ರಾಕಾಶಿಸುವನು. ಸನಾತನವಾದ ನಿರಪೇಕ್ಷನಾದ ಸಚ್ಚಿದಾನಂದಸ್ವರೂಪನಾದ ಆತ್ಮನಲ್ಲಿ ಏನೂ ವ್ಯತ್ಯಾಸವಾಗುವುದಿಲ್ಲ. ಆತ್ಮನಿಗೆ ಜನನಮರಣಗಳು ಇಲ್ಲ. ಜನನಮರಣಗಳು, ಪುನರ್ಜನ್ಮ, ಸ್ವರ್ಗಕ್ಕೆ ಹೋಗುವುದು ಮುಂತಾದುವು ಆತ್ಮನಿಗಲ್ಲ. ಇವೆಲ್ಲ ವಿವಿಧ ತೋರಿಕೆಗಳು, ಮರೀಚಿಕೆಗಳು. ಕೆಟ್ಟ ಕೃತ್ಯಗಳನ್ನು ಆಲೋಚಿಸುತ್ತಿದ್ದರೆ, ಕೆಲವು ಕಾಲದ ಮೇಲೆ ಇಂತಹ ಆಲೋಚನೆಗಳೇ ಇಂತಹ ಕನಸಿಗೆ ಕಾರಣವಾಗುವುವು. ಮಾನವನು ತಾನು ಒಂದು ದೊಡ್ಡ ನರಕದಲ್ಲಿರುವೆನು ಎಂದು ಭಾವಿಸುವನು; ದೊಡ್ಡ ಯಾತನೆಗೆ ಗುರಿಯಾಗುತ್ತಿರುವೆನು ಎಂದು ತಿಳಿಯುವನು. ಕೆಲವು ವೇಳೆ ಒಳ್ಳೆಯ ಆಲೋಚನೆಗಳನ್ನು ಮತ್ತು ಕೆಲಸಗಳನ್ನು ಕಾಣುವನು. ಅವನಿಗೆ ಈಗಿನ ಕನಸು ಆದಮೇಲೆ ತಾನೊಂದು ಒಳ್ಳೆಯ ಲೋಕದಲ್ಲಿರುವೆನು, ಎಂದು ಭಾವಿಸುವನು. ಹೀಗೆಯೇ ಕನಸಿನಿಂದ ಕನಸಿಗೆ ಹೋಗುತ್ತಿರುವನು. ಆದರೆ ಒಂದು ಸಮಯ ಬರುವುದು. ಆಗ ಕನಸೆಲ್ಲ ಮಾಯವಾಗುವುದು. ನಮ್ಮಲ್ಲಿ ಪ್ರತಿಯೊಬ್ಬರಿಗೂ ಈ ಪ್ರಪಂಚವೇ ಒಂದು ಕನಸು ಎನ್ನುವ ಸ್ಥಿತಿ ಬರುವುದು. ಆಗ ಆತ್ಮವು ಸುತ್ತಲಿರುವ ವಾತಾವರಣಕ್ಕಿಂತ ಸಹಸ್ರ ಪಾಲು ಮೇಲು ಎಂದು ಗೊತ್ತಾಗುವುದು. ಸುತ್ತಲಿರುವ ವಾತಾವರಣದಿಂದ ಪಾರಾಗುವುದಕ್ಕೆ ನಾವು ಪ್ರಯತ್ನಿಸುತ್ತಿರುವಾಗ ಒಂದು ಸಮಯ ಬರುವುದು, ಆಗ ಆತ್ಮನೊಂದಿಗೆ ಹೋಲಿಸಿದರೆ ವಾತಾವರಣಕ್ಕೆ ಏನೂ ಬೆಲೆಯಿಲ್ಲ ಎಂದು ಗೊತ್ತಾಗುವುದು. ಇದಕ್ಕೆಲ್ಲ ಒಂದು ಕಾಲ ಬರಬೇಕು. ಅನಂತತೆಯಲ್ಲಿ ಕಾಲಕ್ಕೆ ಬೆಲೆಯಿಲ್ಲ. ಕಾಲವೆನ್ನುವುದು ಮಹಾಸಿಂಧುವಿನಲ್ಲಿ ಒಂದು ಬಿಂದು. ನಾವು ಆ ಸಮಯ ಬರುವವರೆಗೆ ಕಾಯಬಹುದು, ಅಲ್ಲಿಯವರೆಗೆ ಉದ್ವಿಗ್ನತೆಯಿಲ್ಲದೆ ಇರಬಹುದು.

ತಿಳಿದೋ ತಿಳಿಯದೆಯೋ ಇಡೀ ಪ್ರಪಂಚ ಆ ಗುರಿಯಡೆಗೆ ಹೋಗುತ್ತಿದೆ. ಚಂದ್ರ ಇತರ ಗ್ರಹಗಳ ಆಕರ್ಷಣೆಯಿಂದ ಪಾರಾಗಲು ಯತ್ನಿಸುತ್ತಿರುವನು, ಕೊನೆಗೆ ಅವನು ಅದರಿಂದ ಪಾರಾಗಿ ಬರುವನು. ಆದರೆ ಯಾರು ತಾವೇ ಬೇಗ ಪಾರಾಗಬೇಕೆಂದು ಇಚ್ಛೆಪಡುವರೋ ಅವರು ಬೇಗ ಪಾರಾಗುವರು. ಈ ಸಿದ್ಧಾಂತದಲ್ಲಿ ನಾವೇ ಕಣ್ಣಾರೆ ಕಾಣುವ ಒಂದು ಪ್ರಯೋಜನವೇ, ವಿಶ್ವ ಪ್ರೇಮ ಎಂಬ ಭಾವನೆ ಈ ದೃಷ್ಟಿಯಿಂದ ಮಾತ್ರ ಸಾಧ್ಯ ಎಂಬುದು. ಎಲ್ಲರೂ ಸಹಯಾತ್ರಿಕರು, ಎಲ್ಲಾ ಜೀವಿಗಳು, ಸಸ್ಯ, ಪ್ರಾಣಿಗಳು ಕೂಡ, ನನ್ನ ಸಹೋದರರು-ಮಾನವ ಮಾತ್ರ ಅಲ್ಲ. ನನ್ನ ಸಹೋದರನಾದ ಮೂರ್ಖನೂ, ನನ್ನ ಸಹೋದರನಾದ ಸಸಿಯೂ ಕೂಡ; ನನ್ನ ಒಳ್ಳೆಯ ಸಹೋದರ ಮಾತ್ರವಲ್ಲ, ನನ್ನ ಕೆಟ್ಟ ಸಹೋದರ ಕೂಡ; ನನ್ನ ಧರ್ಮಾತ್ಮನಾದ ಸಹೋದರ ಮಾತ್ರವಲ್ಲ, ದುರಾತ್ಮನಾದ ಸಹೋದರ ಕೂಡ - ಎಲ್ಲರೂ ಸಹ ಯಾತ್ರಿಕರು ಇವರೆಲ್ಲ ಒಂದೇ ಗುರಿಯೆಡೆಗೆ ಹೋಗುತ್ತಿರುವರು. ಎಲ್ಲರೂ ಒಂದೇ ಪ್ರವಾಹದಲ್ಲಿ ತೇಲುತ್ತಿರುವರು. ಪ್ರತಿಯೊಬ್ಬರೂ ಆ ಅನಂತ ಸ್ವಾತಂತ್ರ್ಯದೆಡೆಗೆ ಧಾವಿಸುತ್ತಿರುವರು. ನಾವು ಈ ಪ್ರವಾಹವನ್ನು ತಡೆಯುವ ಹಾಗಿಲ್ಲ, ಯಾರೂ ಇದನ್ನು ತಡೆಯಲಾರರು. ಯಾರು ಎಷ್ಟು ಪ್ರಯತ್ನಪಟ್ಟರೂ ಪ್ರವಾಹಕ್ಕೆ ಎದುರಾಗಿ ಹಿಂದಕ್ಕೆ ಹೋಗಲು ಆಗುವುದಿಲ್ಲ. ಅವನನ್ನು ಪ್ರವಾಹ ಮುಂದಕ್ಕೆ ನೂಕುವುದು, ಕೊನೆಗೆ ಅವನು ಮುಕ್ತನಾಗುವನು. ಸೃಷ್ಟಿಯೆಂದರೆ ಮುಕ್ತಿಯ ಕಡೆಗೆ ಹೋಗಲು ನಡೆಸುವ ಹೋರಾಟ; ನಮ್ಮ ವ್ಯಕ್ತಿತ್ವದ ಕೇಂದ್ರಕ್ಕೆ ಪುನಃ ಸೇರುವುದಕ್ಕೆ ಪ್ರಯತ್ನ. ಈಗ ನಾವು ಅಲ್ಲಿಂದ ಈಚೆಗೆ ಎಸೆಯಲ್ಪಟ್ಟವರಂತೆ ಇರುವೆವು. ನಾವಿಲ್ಲಿರುವುದೇ ನಾವು ಕೇಂದ್ರದ ಕಡೆ ಹೋಗುತ್ತೇವೆ ಎನ್ನುವುದನ್ನು ತೋರುವುದು. ಈ ಕೇಂದ್ರದ ಆಕರ್ಷಣೆಯ ಅಭಿವ್ಯಕ್ತಿಯನ್ನೇ ನಾವು ಪ್ರೇಮ ಎನ್ನುವುದು.

ಈ ಪ್ರಪಂಚ ಎಲ್ಲಿಂದ ಬರುವುದು, ಎಲ್ಲಿರುವುದು, ಎಲ್ಲಿಗೆ ಹೋಗುವುದು ಎಂಬ ಪ್ರಶ್ನೆಯನ್ನು ಹಾಕುವರು. ಇದಕ್ಕೆ ಉತ್ತರವೇ ಪ್ರೇಮದಿಂದ ಬಂತು, ಪ್ರೇಮದಲ್ಲಿದೆ, ಪುನಃ ಪ್ರೇಮಕ್ಕೆ ಹೋಗುವುದು ಎಂಬುದು. ಆದಕಾರಣ ನಮಗೆ ಇಚ್ಛೆ ಇರಲಿ ಇಲ್ಲದಿರಲಿ, ಅಂತೂ ನಾವು ಹಿಂದಕ್ಕೆ ಹೋಗುವಂತಿಲ್ಲ. ಪ್ರತಿಯೊಬ್ಬರೂ ಎಷ್ಟೇ ವಿರೋಧವಾಗಿ ಹೋರಾಡಲಿ, ಕೇಂದ್ರಕ್ಕೆ ಹೋಗಬೇಕಾಗಿದೆ. ತಿಳಿದು ಇದಕ್ಕಾಗಿ ಯಾರು ಪ್ರಯತ್ನಿಸುವರೊ ಅವರ ಮಾರ್ಗ ಸುಗಮವಾಗುವುದು, ಘರ್ಷಣೆ ಕಡಮೆಯಾಗುವುದು. ಗುರಿಯನ್ನು ಬೇಗ ಸೇರುವರು. ಸ್ವಾಭಾವಿಕವಾಗಿ ಇದರಿಂದ ಬರುವ ಮತ್ತೊಂದು ನಿರ್ಣಯವೇ ಜ್ಞಾನ ಮತ್ತು ಶಕ್ತಿಯೆಲ್ಲ ಆಗಲೇ ಎಲ್ಲರಲ್ಲಿಯೂ ಸುಪ್ತವಾಗಿದೆ, ಅದೆಲ್ಲೂ ಹೊರಗಿನಿಂದ ಬರುವುದಿಲ್ಲ ಎಂಬುದು. ಯಾವುದನ್ನು ಪ್ರಕೃತಿ ಎನ್ನುವೆವೊ ಅದು ಪ್ರತಿಬಿಂಬಿಸುವ ಕನ್ನಡಿ ಅಷ್ಟೆ; ಪ್ರಕೃತಿಯ ಪ್ರಯೋಜನ ಅಷ್ಟೇ. ಜ್ಞಾನವೆಂದರೆ ಪ್ರಕೃತಿ ಕನ್ನಡಿಯ ಮೇಲೆ ಉಂಟಾಗುವ ಅಂತರಂಗದ ಪ್ರತಿಬಿಂಬ. ಯಾವುದನ್ನು ಶಕ್ತಿ, ಪ್ರಕೃತಿರಹಸ್ಯ ಎನ್ನುವೆವೊ ಅವೆಲ್ಲ ಒಳಗಿವೆ. ಬಾಹ್ಯ ಪ್ರಪಂಚದಲ್ಲಿರುವುದೆಲ್ಲ ಒಂದು ತುದಿಮೊದಲಿಲ್ಲದ ಬದಲಾವಣೆ, ಪ್ರಕೃತಿಯಲ್ಲಿ ಯಾವ ಜ್ಞಾನವೂ ಇಲ್ಲ. ಜ್ಞಾನವೆಲ್ಲ ಆತ್ಮನಿಂದ ಬರುವುದು. ಮಾನವನು ಜ್ಞಾನವನ್ನು ವ್ಯಕ್ತಗೊಳಿಸುವನು. ತನ್ನಲ್ಲಿಯೇ ಅದನ್ನು ಕಂಡುಹಿಡಿಯುವನು. ಅನಂತ ಕಾಲದಿಂದಲೂ ಅದು ಅಲ್ಲಿಯೇ ಇದೆ. ಪ್ರತಿಯೊಬ್ಬರೂ ಜ್ಞಾನಘನ, ಪ್ರತಿಯೊಬ್ಬರೂ ಆನಂದಮೂರ್ತಿಗಳು, ಪ್ರತಿಯೊಬ್ಬರೂ ಸತ್ ಸ್ವರೂಪರು. ಸಮಾನತೆಗೆ ಸಂಬಂಧಿಸಿದಂತೆ ಇತರ ಕಡೆ ನೋಡಿದಂತೆಯೇ, ನೈತಿಕ ಪರಿಣಾಮವೂ ಹೀಗೆಯೇ ಇರುತ್ತದೆ.

ಹಕ್ಕುಬಾಧ್ಯತೆಗಳಿಗಾಗಿ ಹೋರಾಡುವುದೇ ಮಾನವಕೋಟಿಯ ಒಂದು ಮಹಾದುಃಖವಾಗಿರುವುದು. ಎರಡು ಶಕ್ತಿಗಳು ಯಾವಾಗಲೂ ಕೆಲಸ ಮಾಡುತ್ತಿವೆ. ಒಂದು ಜಾತಿಯನ್ನು ನಿರ್ಮಿಸುತ್ತಿದೆ, ಮತ್ತೊಂದು ಜಾತಿಯನ್ನು ಧ್ವಂಸಮಾಡುತ್ತಿದೆ. ಆದರೆ ಒಂದು ಹಕ್ಕುಬಾಧ್ಯತೆಗೆ ಕಾರಣವಾಗುತ್ತಿದೆ, ಮತ್ತೊಂದು ಅದನ್ನು ನಿರ್ಮೂಲ ಮಾಡುತ್ತಿದೆ. ಹಕ್ಕುಬಾಧ್ಯತೆಗಳು ನಾಶವಾದಾಗಲೆಲ್ಲ ಜನಾಂಗಕ್ಕೆ ಹೆಚ್ಚು ಹೆಚ್ಚು ಜ್ಞಾನ ಬರುವುದು, ಅದು ಪ್ರಗತಿಪರ ವಾಗುವುದು. ನಮ್ಮ ಸುತ್ತಲೂ ಈ ಹೋರಾಟವನ್ನು ನೋಡುತ್ತಿರುವೆವು. ಮೊದಲು ಹಕ್ಕುಬಾಧ್ಯತೆಯ ಮೃಗೀಯ ಸ್ವರೂಪ ಇದೆ. ಅದೇ ಬಲಾಡ್ಯನಿಗೆ ಬಲಹೀನನ ಮೇಲೆ ಇರುವ ಹಕ್ಕುಬಾಧ್ಯತೆ. ಅನಂತರ ಐಶ್ವರ್ಯದ ಹಕ್ಕುಗಳಿವೆ. ಒಬ್ಬನಿಗೆ ಮತ್ತೊಬ್ಬನಿಗಿಂತ ಹೆಚ್ಚು ಐಶ್ವರ್ಯವಿದ್ದರೆ ಅವನಿಗೆ ಹೆಚ್ಚು ಹಕ್ಕುಗಳು ಬೇಕು. ಇದಕ್ಕಿಂತ ಸೂಕ್ಷ್ಮವಾದ ಆದರೆ ಇದಕ್ಕಿಂತಲೂ ಪ್ರಬಲವಾದ ಬುದ್ದಿಯ ಹಕ್ಕಿದೆ. ಒಬ್ಬನಿಗೆ ಇತರರಿಗಿಂತ ಹೆಚ್ಚು ಗೊತ್ತಿದೆ ಎಂದು ಹೆಚ್ಚು ಹಕ್ಕುಗಳನ್ನು ಆಶಿಸುವನು. ಕೊನೆಯದೆ, ಅತಿ ದುಷ್ಟವಾದುದೆ, ಆಧ್ಯಾತ್ಮಿಕತೆಯ ಹಕ್ಕು. ಕೆಲವರಿಗೆ ದೇವರು ಧರ್ಮದ ವಿಷಯ ಹೆಚ್ಚು ಗೊತ್ತಿದೆ ಎಂದು ಎಲ್ಲರಿಗಿಂತ ಹೆಚ್ಚು ಹಕ್ಕುಗಳನ್ನು ಕೇಳುವರು. 'ಕುರಿಮಂದೆಯಂತೆ ಇರುವ ಜನರೇ, ಬಂದು ನಮ್ಮನ್ನು ಪೂಜಿಸಿ, ನಾವು ದೇವದೂತರು, ನೀವು ನಮ್ಮನ್ನು ಪೂಜಿಸಬೇಕು' ಎನ್ನುವರು. ಇತರರಿಗಿಂತ ತಮಗೆ ಯಾವುದೇ ವಿಧವಾದ ದೈಹಿಕ, ಮಾನಸಿಕ, ಆಧ್ಯಾತ್ಮಿಕವಾದ ವಿಶೇಷ ಹಕ್ಕುಗಳನ್ನು ಒಪ್ಪುವವರು ಯಾರೂ ನಿಜವಾದ ವೇದಾಂತಿಗಳಾಗಿರಲಾರರು. ಯಾರಿಗೂ ಯಾವ ವಿಧವಾದ ಹಕ್ಕುಗಳೂ ಇಲ್ಲ. ಒಂದೇ ಶಕ್ತಿ ಎಲ್ಲರಲ್ಲಿಯೂ ಇದೆ. ಒಬ್ಬ ಹೆಚ್ಚು ಅದನ್ನು ವ್ಯಕ್ತಗೊಳಿಸಿರುವನು, ಮತ್ತೊಬ್ಬ ಅದನ್ನು ಕಡಮೆ ವ್ಯಕ್ತಗೊಳಿಸಿರುವನು. ಈ ಶಕ್ತಿ ಎಲ್ಲರಲ್ಲಿಯೂ ಸುಪ್ತವಾಗಿದೆ. ವಿಶೇಷ ಹಕ್ಕಿಗೆ ಇಲ್ಲಿ ಅವಕಾಶವೆಲ್ಲಿದೆ? ಜ್ಞಾನವೆಲ್ಲ ಎಲ್ಲರಲ್ಲಿಯೂ ಇದೆ, ಅತಿ ಮೂಢನಲ್ಲಿಯೂ ಇದೆ, ಅವನು ಅದನ್ನು ವ್ಯಕ್ತಗೊಳಿಸಿಲ್ಲ ಅಷ್ಟೆ. ಬಹುಶಃ ಅದನ್ನು ವ್ಯಕ್ತಗೊಳಿಸಲು ಅವಕಾಶ ಸಿಕ್ಕಿಲ್ಲ, ವಾತಾವರಣ ಅದಕ್ಕೆ ಸೂಕ್ತವಾಗಿರಲಿಲ್ಲ. ಅವನಿಗೆ ಅವಕಾಶ ಸಿಕ್ಕಿದರೆ ಅದನ್ನು ವ್ಯಕ್ತಗೊಳಿಸುವನು. ಒಬ್ಬ ಜನ್ಮತಃ ಮತ್ತೊಬ್ಬನಿಗಿಂತ ಮೇಲು ಎನ್ನುವುದಕ್ಕೆ ವೇದಾಂತದಲ್ಲಿ ಅರ್ಥವಿಲ್ಲ. ಎರಡು ಜನಾಂಗಗಳಲ್ಲಿ ಒಂದು ಮೇಲು ಮತ್ತೊಂದು ಕೀಳು ಎನ್ನುವುದಕ್ಕೂ ಅರ್ಥವಿಲ್ಲ. ಅವರಿಬ್ಬರನ್ನೂ ಒಂದೇ ವಾತಾವರಣದಲ್ಲಿಟ್ಟು ನೋಡಿ, ಅದೇ ಬುದ್ದಿ ಪ್ರಕಟವಾಗುವುದೋ ಇಲ್ಲವೋ ಎನ್ನುವುದನ್ನು. ಒಂದು ರಾಷ್ಟ್ರವು ಮತ್ತೊಂದಕ್ಕಿಂತ ಮೇಲು ಎನ್ನುವುದಕ್ಕೆ ನಿಮಗೆ ಅಧಿಕಾರವಿಲ್ಲ. ಆಧ್ಯಾತ್ಮಿಕ ದೃಷ್ಟಿಯಿಂದಂತೂ ಯಾರೂ ವಿಶೇಷ ಹಕ್ಕನ್ನು ಕೇಳಲೇಬಾರದು. ಮಾನವ ಕೋಟಿಗೆ ಸೇವೆ ಮಾಡುವುದೇ ಒಂದು ಮಹಾಭಾಗ್ಯ, ಏಕೆಂದರೆ ಇದೇ ದೇವರ ಪೂಜೆ, ದೇವರು ಇಲ್ಲಿ ಎಲ್ಲರಲ್ಲಿಯೂ ಇರುವನು. ಅವನೇ ಮಾನವನ ಆತ್ಮನಾಗಿರುವನು. ಮಾನವರು ಏನು ಹಕ್ಕನ್ನು ಕೇಳಬಲ್ಲರು? ಭಗವಂತನ ವಿಶೇಷ ದೇವದೂತರೆಂಬುವವರು ಹಿಂದೆ ಇರಲಿಲ್ಲ, ಮುಂದೆಯೂ ಇರುವುದಿಲ್ಲ. ದೊಡ್ಡವರು ಚಿಕ್ಕವರು ಎಲ್ಲರೂ ಭಗವಂತನ ಅಭಿವ್ಯಕ್ತಿಗಳು. ವ್ಯತ್ಯಾಸವಿರುವುದು ಅದರ ಪ್ರಕಟಣೆಯಲ್ಲಿ ಮಾತ್ರ. ಶಾಶ್ವತವಾಗಿ ಮಾನವನಿಗೆ ಕೊಡಲ್ಪಟ್ಟ ಸನಾತನ ಸಂದೇಶವೇ ಸ್ವಲ್ಪಸ್ವಲ್ಪವಾಗಿ ಅವರಿಗೆ ಬರುತ್ತದೆ. ಆ ಸನಾತನ ಸಂದೇಶ ಪ್ರತಿಯೊಬ್ಬರ ಹೃದಯಾಂತರಾಳದಲ್ಲಿಯೂ ಆಗಲೆ ಬರೆದಿದೆ. ಅದನ್ನು ವ್ಯಕ್ತಗೊಳಿಸಲು ಎಲ್ಲರೂ ಪ್ರಯತ್ನಿಸುತ್ತಿರುವರು. ಸೂಕ್ತ ವಾತಾವರಣದಲ್ಲಿರುವ ಕೆಲವರು ಇತರರಿಗಿಂತ ಹೆಚ್ಚು ಅದನ್ನು ಪ್ರಕಟಗೊಳಿಸುವರು. ಆದರೆ ಸಂದೇಶವಾಹಕರ ದೃಷ್ಟಿಯಿಂದ ಎಲ್ಲರೂ ಒಂದೆ. ಮೇಲೆಂದು ಹೆಮ್ಮೆ ಕೊಚ್ಚಿಕೊಳ್ಳುವುದಕ್ಕೆ ಇಲ್ಲಿ ಏನಿದೆ? ಅಜ್ಞಾನಿಯಾದ ವ್ಯಕ್ತಿ, ಕೊನೆಗೆ ಏನೂ ತಿಳಿಯದ ಒಂದು ಮಗು ಕೂಡ, ಪ್ರಪಂಚದಲ್ಲಿ ಹಿಂದೆ ಎಂದಾದರೂ ಇದ್ದ ಶ್ರೇಷ್ಠ ದೇವದೂತನಷ್ಟೇ, ಅಥವಾ ಮುಂದೆ ಬರುವ ಶ್ರೇಷ್ಠ ದೇವದೂತನಷ್ಟೇ ಶ್ರೇಷ್ಠವಾಗಿರುವುದು. ಏಕೆಂದರೆ ಅನಂತ ಸಂದೇಶವು ಆಗಲೇ ಎಲ್ಲರಲ್ಲಿಯೂ ಸುಪ್ತವಾಗಿದೆ. ಎಲ್ಲಿಯಾದರೂ ಒಬ್ಬನಿದ್ದರೆ ಈ ಸನಾತನವಾದ ಶ್ರೇಷ್ಠವಾದ ಸಂದೇಶ ಆಗಲೇ ಅವನಲ್ಲಿರುವುದು. ಅದ್ವೈತದ ಕೆಲಸವೇ ಈ ಹಕ್ಕುಗಳನ್ನು ನಿರ್ಮೂಲ ಮಾಡುವುದು. ಇದೇ ಎಲ್ಲಕ್ಕಿಂತ ಬಹಳ ಕಷ್ಟದ ಕೆಲಸ. ಈ ಕೆಲಸ ಎಲ್ಲಾ ಕಡೆಗಳಿಗಿಂತ ಕಡಮೆ ನಡೆಯುತ್ತಿರುವುದು ವೇದಾಂತದ ಜನ್ಮ ಭೂಮಿಯಲ್ಲಿ ಎಂದು ಹೇಳಲು ಸೋಜಿಗವಾಗುವುದು. ಎಲ್ಲಿಯಾದರೂ ವಿಶೇಷ ಹಕ್ಕುಗಳು ಇರುವ ಒಂದು ದೇಶವಿದ್ದರೆ ಅದೇ ಈ ತತ್ತ್ವದ ತೌರೂರು. ಮೇಲಿನ ಜಾತಿ ಮತ್ತು ಆಧ್ಯಾತ್ಮಿಕ ಸಂಪತ್ತು ಉಳ್ಳವನಿಗೆ ಹಕ್ಕುಗಳು ಅಲ್ಲಿ ಅಧಿಕ. ಅಲ್ಲಿ ಐಶ್ವರ್ಯದ ಹಕ್ಕು ಅಷ್ಟು ಇಲ್ಲ. (ಇದೊಂದು ಅನುಕೂಲ ಎಂದು ಭಾವಿಸುತ್ತೇನೆ. ) ಜನ್ಮದ ಮತ್ತು ಆಧ್ಯಾತ್ಮಿಕತೆಯ ಹಕ್ಕುಗಳು ಎಲ್ಲೆಲ್ಲಿಯೂ ಇರುವುವು.

ಹಿಂದೆ ಒಮ್ಮೆ ವೇದಾಂತದ ನೀತಿಯನ್ನು ಜನರಿಗೆ ಬೋಧಿಸುವುದಕ್ಕೆ ಅದ್ಭುತವಾದ ಪ್ರಯತ್ನ ನಡೆಯಿತು. ಅದು ಸ್ವಲ್ಪ ಮಟ್ಟಿಗೆ ಕೆಲವು ನೂರು ವರ್ಷಗಳವರೆಗೆ ಫಲಿಸಿತು. ಚಾರಿತ್ರಿಕ ದೃಷ್ಟಿಯಿಂದ ಆ ದೇಶ ಆ ಕಾಲದಲ್ಲಿ ಬಹಳ ಉತ್ತಮಸ್ಥಿತಿಯಲ್ಲಿತ್ತೆಂದು ನಮಗೆ ಗೊತ್ತಿದೆ. ಅದು ಯಾವಾಗ ಎಂದರೆ ಬುದ್ದನು ಹಕ್ಕುಬಾಧ್ಯತೆಗಳನ್ನು ನಿರ್ಮೂಲ ಮಾಡಲು ಪ್ರಯತ್ನ ಪಟ್ಟಾಗ. ಬುದ್ಧನಿಗೆ ಆರೋಪಿಸಿರುವ ಅತ್ಯಂತ ಸುಂದರವಾದ ಗುಣಗಳಲ್ಲಿ ಇದೊಂದು: “ನೀನೇ ಜಾತಿಯನ್ನು ಧ್ವಂಸ ಮಾಡಿದವನು. ಹಕ್ಕುಬಾಧ್ಯತೆಗಳನ್ನು ನಾಶಮಾಡಿದವನು, ಎಲ್ಲರಿಗೂ ಸಮತ್ವವನ್ನು ಬೋಧಿಸಿದವನು.” ಬುದ್ಧನು ಸಮತ್ವದ ಒಂದು ಭಾವನೆಯನ್ನು ಸಾರಿದನು. ಶ್ರಮಣರ ಸಂಘದಲ್ಲಿ ಇದನ್ನು ಸರಿಯಾಗಿ ಅರ್ಥಮಾಡಿಕೊಂಡಂತೆ ಕಾಣಲಿಲ್ಲ. ಅಲ್ಲಿ ಮೇಲಿನ ಮತ್ತು ಕೆಳಗಿನ ಅಂತಸ್ತಿನಲ್ಲಿರುವ ವ್ಯಕ್ತಿಗಳನ್ನು ಒಟ್ಟಿಗೆ ಸೇರಿಸಲು ಹಲವು ವೇಳೆ ಯತ್ನಿಸಿರುವರು. ಜನರಿಗೆ ನೀವೆಲ್ಲ ದೇವಾಂಶರು ಎಂದು ಹೇಳಿದರೆ ಒಂದು ಸಂಘಬದ್ಧ ಧರ್ಮದ ಆವಶ್ಯಕತೆ ಅಷ್ಟು ಉಳಿಯುವುದಿಲ್ಲ. ವೇದಾಂತದ ಒಂದು ಒಳ್ಳೆಯ ಪರಿಣಾಮವೆ ಧಾರ್ಮಿಕ ವಿಷಯದಲ್ಲಿ ಒಂದು ಸ್ವಾತಂತ್ರ್ಯವನ್ನು ಜನರಿಗೆ ಕೊಟ್ಟಿರುವುದು. ಭರತಖಂಡದಲ್ಲಿ ಇದು ಯಾವಾಗಲೂ ಇರುವುದು. ಧರ್ಮದ ಹೆಸರಿನಲ್ಲಿ ಭರತಖಂಡದಲ್ಲಿ ಎಂದಿಗೂ ಹಿಂಸೆಯಾಗಿಲ್ಲ. ಜನರಿಗೆ ಧಾರ್ಮಿಕ ವಿಷಯದಲ್ಲಿ ಸಂಪೂರ್ಣ ಸ್ವಾತಂತ್ರ್ಯವಿದೆ. ಇದೊಂದು ಹೆಮ್ಮೆಯ ವಿಷಯ.

ವೇದಾಂತದ ಅನುಷ್ಠಾನಮುಖವಾದ ನೈತಿಕತೆ ಹಿಂದಿನಷ್ಟೇ ಈಗಲೂ ಮುಖ್ಯವಾಗಿದೆ. ಬಹುಶಃ ಹಿಂದಿಗಿಂತ ಈಗ ಹೆಚ್ಚು ಮುಖ್ಯವಾಗಿದೆ. ಜ್ಞಾನ ಹೆಚ್ಚಿದಂತೆಲ್ಲಾ ಹಕ್ಕುಬಾಧ್ಯತೆಗಳ ಕೂಗು ಹೆಚ್ಚಿದೆ. ದೇವನಿಗೂ ಮತ್ತು ಸೈತಾನನಿಗೂ, ಅಹುರಮಜದ (ದೇವರು) ನಿಗೂ ಮತ್ತು ಅಹರಿಮಾನ್ (ಸೈತಾನ್)ನಿಗೂ ಇರುವ ವ್ಯತ್ಯಾಸ ಒಳ್ಳೆಯ ಕಾವ್ಯಮಯವಾಗಿದೆ. ದೇವರಿಗೂ ಸೈತಾನನಿಗೂ ಇರುವ ವ್ಯತ್ಯಾಸವೆ ನಿಃಸ್ವಾರ್ಥ ಮತ್ತು ಸ್ವಾರ್ಥಗಳಲ್ಲಿದೆ. ಸೈತಾನನಿಗೆ ದೇವರಷ್ಟೇ ಜ್ಞಾನವಿದೆ, ಆದರ ಪಾವಿತ್ರ್ಯತೆ ಇಲ್ಲ. ಅದಕ್ಕೇ ಅವನು ಸೈತಾನನಾಗಿರುವುದು. ಆಧುನಿಕ ಪ್ರಪಂಚಕ್ಕೂ ಇದನ್ನೇ ಅನ್ವಯಿಸಿ. ಪಾವಿತ್ರ್ಯತೆಯಿಲ್ಲದೆ ಜ್ಞಾನ ಮತ್ತು ಶಕ್ತಿ ಪ್ರಬಲವಾಗಿದ್ದಷ್ಟೂ ಅವು ಜನರನ್ನು ಸೈತಾನರನ್ನಾಗಿ ಮಾಡುವುವು. ಯಂತ್ರಗಳ ಮತ್ತು ಇತರ ವಸ್ತುಗಳ ತಯಾರಿಕೆಯಿಂದ ಅದ್ಭುತ ಶಕ್ತಿಯನ್ನು ಮಾನವ ಇಂದು ತನ್ನ ಸ್ವಾಧೀನದಲ್ಲಿಟ್ಟುಕೊಂಡಿರುವನು. ಜಗತ್ತಿನ ಇತಿಹಾಸದಲ್ಲಿ ಹಿಂದೆ ಎಲ್ಲೂ ಇಲ್ಲದ ಹಕ್ಕುಬಾಧ್ಯತೆಗಳನ್ನು ಮಾನವ ಇಂದು ಕೇಳುತ್ತಿರುವನು. ಆದಕಾರಣವೆ ವೇದಾಂತವು ಇದಕ್ಕೆ ವಿರೋಧವಾಗಿ ಬೋಧಿಸುವುದು; ಮಾನವರನ್ನು ನಾಶಮಾಡುತ್ತಿರುವ ಸ್ವಭಾವವನ್ನು ನಿರ್ಮೂಲ ಮಾಡಲೆತ್ನಿಸುವುದು.

ಯಾರು ಭಗವದ್ಗೀತೆಯನ್ನು ಓದಿರುವರೊ ಅವರು ಈ ಗಮನಾರ್ಹವಾದ ಶ್ಲೋಕವನ್ನು ನೆನಪಿನಲ್ಲಿಟ್ಟಿರಬಹುದು: “ಯಾರು ಪಂಡಿತ ಹಸು ಆನೆ ನಾಯಿ ಚಂಡಾಲ - ಎಲ್ಲರನ್ನೂ ಒಂದೇ ಸಮನಾಗಿ ಕಾಣುವರೋ, ಅವರೇ ಋಷಿಗಳು, ಅವರೇ ಜ್ಞಾನಿಗಳು.” “ಯಾರು ಸಮತ್ವದಲ್ಲಿ ಪ್ರತಿಷ್ಠಿತರಾಗಿರುವರೊ ಅವರು ಈ ಜನ್ಮದಲ್ಲಿಯೇ ಮುಕ್ತರು, ಏಕೆಂದರೆ ದೇವರು ಎಲ್ಲರನ್ನೂ ಸಮದೃಷ್ಟಿಯಿಂದ ನೋಡುವನು, ಅವನು ಪರಿಶುದ್ದನು. ಆದಕಾರಣ ಯಾರು ಎಲ್ಲರನ್ನೂ ಒಂದೇ ಸಮನಾಗಿ ಕಾಣುವರೊ ಮತ್ತು ಯಾರು ಪರಿಶುದ್ದರೊ ಅವರು ದೇವರಲ್ಲಿ ನೆಲಸಿರುವರು.” ಈ ಸಮತ್ವವೇ ವೇದಾಂತನೀತಿಯ ಸಾರ. ದೃಶ್ಯವನ್ನು ಆಳುತ್ತಿರುವುದು ದೃಕ್ ಎಂಬುದನ್ನು ನೋಡಿದೆವು. ನೀವು ದೃಕ್ ಅನ್ನು ಬದಲಾಯಿಸಿದರೆ ದೃಶ್ಯವೂ ಬದಲಾಗುವುದು. ನೀವು ಪರಿಶುದ್ಧರಾಗಿ, ಅನಂತರ ಜಗತ್ತು ಪರಿಶುದ್ಧವಾಗುವುದರಲ್ಲಿ ಸಂದೇಹವಿಲ್ಲ. ಈ ಒಂದು ಭಾವನೆಯನ್ನು ಎಂದಿಗಿಂತ ಹೆಚ್ಚಾಗಿ ಬೋಧಿಸಬೇಕಾಗಿದೆ. ನಾವು ಇತರರನ್ನು ತಿದ್ದುವುದರಲ್ಲಿ ಹೆಚ್ಚು ಕಾತರರಾಗುತ್ತಿರುವೆವು, ನಮ್ಮನ್ನು ಗಮನಕ್ಕೆ ತೆಗೆದುಕೊಳ್ಳುವುದಿಲ್ಲ. ನಾವು ಬದಲಾದರೆ ಜಗತ್ತೂ ಬದಲಾಗುವುದು. ನಾವು ಪರಿಶುದ್ಧರಾದರೆ ಜಗತ್ತೂ ಪರಿಶುದ್ಧವಾಗುವುದು. ನಾನು ಇತರರಲ್ಲಿ ಏತಕ್ಕೆ ಪಾಪವನ್ನು ನೋಡಬೇಕು? ನಾನು ಪಾಪಿಯಲ್ಲದೇ ಇದ್ದರೆ ಪಾಪವನ್ನು ನೋಡಲಾರೆ. ನಾನು ದುರ್ಬಲನಾಗುವವರೆಗೆ ದುಃಖಿಯಾಗಲಾರೆ. ನಾನು ಮಗುವಾಗಿದ್ದಾಗ ಯಾವುದು ನನ್ನನ್ನು ದುಃಖಿಯಾನ್ನಾಗಿ ಮಾಡಿತೊ ಅದು ಈಗ ನನ್ನನ್ನು ದುಃಖಿಯನ್ನಾಗಿ ಮಾಡಲಾರದು. ದೃಕ್ ಬದಲಾಗಿದೆ, ಆದಕಾರಣ ದೃಶ್ಯವೂ ಬದಲಾಗಬೇಕಾಯಿತು ಎನ್ನುವುದು ವೇದಾಂತ. ಸಮತ್ವ ಎಂಬ ಆ ಅದ್ಭುತ ಸ್ಥಿತಿಯಲ್ಲಿ ನೆಲೆಗೊಂಡಾಗ ಯಾವುದನ್ನು ದುಃಖಕ್ಕೆ ಮತ್ತು ಪಾಪಕ್ಕೆ ಕಾರಣ ಎಂದು ಹೇಳುತ್ತೇವೆಯೋ ಅದನ್ನೆಲ್ಲ ನೋಡಿ ನಾವು ನಗುತ್ತೇವೆ. ಸಮತ್ವದಲ್ಲಿ ಪ್ರತಿಷ್ಠಿತರಾಗುವುದನ್ನೇ ವೇದಾಂತದಲ್ಲಿ ಮುಕ್ತಿ ಎನ್ನುವುದು. ಮುಕ್ತಿಯನ್ನು ಸಮೀಪಿಸಿದಂತೆಲ್ಲಾ ಈ ಸಮದರ್ಶಿತ್ವ ಹೆಚ್ಚು ಹೆಚ್ಚು ಪ್ರಕಟವಾಗುವುದು. ಯಾರು ಸುಖದುಃಖಗಳನ್ನು ಸಮವಾಗಿ ನೋಡುವರೊ, ಜಯಾಪಜಯಗಳಲ್ಲಿ ಯಾರು ಸಮರಾಗಿರುವರೋ, ಅವರು ಮುಕ್ತಿಯ ಸಮಿಾಪಕ್ಕೆ ಬರುತ್ತಿರುವರು.

ಮನಸ್ಸನ್ನು ಸುಲಭವಾಗಿ ಗೆಲ್ಲುವುದಕ್ಕೆ ಆಗುವುದಿಲ್ಲ. ಪ್ರತಿಯೊಂದು ಸಣ್ಣ ವಸ್ತುವಿನಲ್ಲಿ ಆಸಕ್ತರಾಗಿ, ಸ್ವಲ್ಪ ಅಪಾಯಕ್ಕೆ ಮತ್ತು ಕೋಪಕ್ಕೆ ತುತ್ತಾದರೂ ಮನಸ್ಸು ಅಲ್ಲೋಲಕಲ್ಲೋಲವಾದರೆ ಅವರು ಎಂತಹ ಸ್ಥಿತಿಯಲ್ಲಿರುವರು? ಮನಸ್ಸು ಇಂತಹ ಕ್ಷೋಭೆಗೆ ತುತ್ತಾಗುತ್ತಿದ್ದ ಶ್ರೇಷ್ಠತೆ ಅಥವಾ ಆಧ್ಯಾತ್ಮಿಕತೆಯ ವಿಷಯದಲ್ಲಿ ಹೆಮ್ಮೆ ಪಟ್ಟುಕೊಳ್ಳುವುದರಲ್ಲೇನಿದೆ? ಇಂತಹ ಮಾನಸಿಕ ಚಂಚಲತೆಯನ್ನು ತಡೆಯಬೇಕು. ನಮಗೆ ಎಷ್ಟೇ ವಿರೋಧವಿದ್ದರೂ ನಾವು ಎಷ್ಟು ಮಟ್ಟಿಗೆ ಸ್ವತಂತ್ರರಾಗಿ ನಿಲ್ಲಬಲ್ಲೆವು, ಮತ್ತು ಮನಸ್ಸು ಎಷ್ಟರ ಮಟ್ಟಿಗೆ ಅದರ ಪ್ರಭಾವಕ್ಕೆ ಬೀಳುವುದಿಲ್ಲ, ಎಂಬುದನ್ನು ನಾವೇ ನೋಡಿಕೊಳ್ಳಬೇಕಾಗಿದೆ. ಬಾಹ್ಯಘಟನೆಗಳಾವುವೂ ನಮ್ಮ ಮನಸ್ಸಿನ ಸ್ವಾಸ್ಥ್ಯವನ್ನು ಕೆಡಿಸದಾಗ ಮಾತ್ರ ನಾವು ಸ್ವತಂತ್ರರಾದಂತೆ, ಅದಕ್ಕೆ ಮುಂಚೆ ಇಲ್ಲ. ಇದೇ ಮುಕ್ತಿ. ಅದು ಇಲ್ಲೇ ಇರುವುದು, ಇನ್ನೆಲ್ಲಿಯೂ ಇಲ್ಲ. ಈ ಕ್ಷಣ ಅದು ಇರುವುದು. ಈ ಭಾವನೆಯಿಂದ ಮಾತ್ರ, ಈ ಮೂಲದಿಂದ ಮಾತ್ರ ಎಲ್ಲಾ ಸುಂದರ ಭಾವನೆಗಳೂ ಪ್ರಪಂಚಕ್ಕೆ ಹರಿದು ಬಂದಿವೆ. ಸಾಧಾರಣವಾಗಿ ಈ ಭಾವನೆಗಳನ್ನು ತಪ್ಪು ತಿಳಿದುಕೊಂಡಿರುವರು. ಇವು ತೋರಿಕೆಗೆ ಮಾತ್ರ ಪರಸ್ಪರ ವಿರೋಧವಾಗಿರುವಂತೆ ಕಾಣುವುವು. ಹಲವು ಧೀರರಾದ ಆಧ್ಯಾತ್ಮಿಕ ವ್ಯಕ್ತಿಗಳು ಪ್ರತಿಯೊಂದು ದೇಶದಲ್ಲಿಯೂ ಬಾಹ್ಯಪ್ರಪಂಚದ ಸಂಬಂಧವನ್ನೆಲ್ಲ ತ್ಯಜಿಸಿ ಧ್ಯಾನಮಾಡುವುದಕ್ಕಾಗಿ ಕಾಡಿಗೊ ಗುಹೆಗೂ ತೆರಳುವರು. ಇದೊಂದು ಭಾವನೆ. ಮತ್ತೊಂದೇ ಪ್ರಖ್ಯಾತರಾದ ಮಹಾ ಮೇಧಾವಿಗಳು ಸಮಾಜಕ್ಕೆ ಬಂದು ಅಲ್ಲಿರುವ ದೀನ ದರಿದ್ರರನ್ನು ಮೇಲೆ ಎತ್ತಲು ಪ್ರಯತ್ನಿಸುತ್ತಿರುವರು ಎಂಬುದು. ತೋರಿಕೆಗೆ ಈ ಎರಡು ಮಾರ್ಗಗಳೂ ವಿರೋಧವಾಗಿವೆ. ಇತರರಿಂದ ದೂರವಾಗಿದ್ದು ಗುಹೆಯಲ್ಲಿರುವವನು ಇತರರ ಮೇಲ್ಮೆಗೆ ದುಡಿಯುತ್ತಿರುವವನನ್ನು ನೋಡಿದಾಗ ನಿಕೃಷ್ಟವಾಗಿ ನಗುವನು. "ಎಂತಹ ಮೌಢ್ಯ! ಏನಿದೆ ಕೆಲಸಮಾಡುವುದಕ್ಕೆ? ಮಾಯಾಪ್ರಪಂಚ ಯಾವಾಗಲೂ ಒಂದೇ ಸಮನಾಗಿರುವುದು. ಇದನ್ನು ಬದಲಾಯಿಸುವುದಕ್ಕೆ ಆಗುವುದಿಲ್ಲ" ಎನ್ನುವನು. ನಾನು ಇಂಡಿಯಾ ದೇಶದಲ್ಲಿ ಯಾರಾದರೂ ಬ್ರಾಹ್ಮಣರನ್ನು ನೀವು ವೇದಾಂತದಲ್ಲಿ ನಂಬುತ್ತೀರಾ ಎಂದು ಕೇಳಿದರೆ, ಅವರು ಹೌದು, ಅದೇ ನಮ್ಮ ಧರ್ಮ ಎನ್ನುವರು. ಹಾಗಾದರೆ ಎಲ್ಲರೂ ಸಮ ಎನ್ನುವುದನ್ನು ಒಪ್ಪಿಕೊಳ್ಳುವಿರಾ ಎಂದು ಕೇಳಿದರೆ, ಹೌದು ನಿಜವಾಗಿಯೂ ಒಪ್ಪಿಕೊಳ್ಳುತ್ತೇವೆ ಎನ್ನುವರು. ಮರುಕ್ಷಣವೇ ಯಾವ ಚಂಡಾಲನಾದರೂ ಅವರ ಸಮೀಪಕ್ಕೆ ಬಂದರೆ, ಅವನಿಂದ ದೂರವಿರುವುದಕ್ಕೆ ದಾರಿಯ ಕೊನೆಗೆ ನೆಗೆದು ಓಡುವರು. ಏತಕ್ಕೆ ಓಡುವಿರಿ ಎಂದರೆ, ಅವನ ಸ್ಪರ್ಶವೇ ನಮ್ಮನ್ನು ಮೈಲಿಗೆ ಮಾಡುವುದು ಎನ್ನುವರು. ಆದರೆ ನೀವೂ ಎಲ್ಲರೂ ಒಂದೇ, ಯಾರೂ ಹೆಚ್ಚು ಕಡಮೆಯಲ್ಲ ಎಂದು ಈಗ ತಾನೆ ಹೇಳುತ್ತಿದ್ದೀರಲ್ಲ ಎಂದರೆ, ಗೃಹಸ್ಥರಿಗೆ ಅದೆಲ್ಲ ಬರಿಯ ಬಾಯಿಮಾತು, ನಾವು ಕಾಡಿಗೆ ಹೋದ ಮೇಲೆ ಎಲ್ಲರೂ ಒಂದೇ ಎಂದು ನೋಡುತ್ತೇವೆ, ಎನ್ನುವರು. ಇಂಗ್ಲೆಂಡಿನಲ್ಲಿ ಒಂದು ಕುಲೀನ ವಂಶದಲ್ಲಿ ಜನಿಸಿದ್ದ ಶ‍್ರೀಮಂತನನ್ನು, ನಾವೆಲ್ಲ ದೇವರಿಂದ ಬಂದುದರಿಂದ ವಿಶ್ವದ ಸಹೋದರತ್ವದಲ್ಲಿ ನಂಬುವೆಯಾ ಎಂದು ಕೇಳಿದರೆ, ಅವನು ಹೌದು ಎನ್ನುವನು. ಆದರೆ ಐದು ನಿಮಿಷ ಕಳೆಯಲಿಲ್ಲ, ಆಗಲೇ ಸಾಧಾರಣ ಮನುಷ್ಯನನ್ನು ನಿಕೃಷ್ಟ ದೃಷ್ಟಿಯಿಂದ ನೋಡುವನು, ಅವನ ಬಗ್ಗೆ ಕೆಟ್ಟ ಮಾತಾಡುವನು. ಆದಕಾರಣ ಸಾವಿರಾರು ವರ್ಷಗಳಿಂದ ಇದು ಕೇವಲ ಒಂದು ಸಿದ್ದಾಂತ ಮಾತ್ರವಾಗಿದೆ, ಅನುಷ್ಠಾನಕ್ಕೆ ಬಂದಿಲ್ಲ. ಎಲ್ಲರೂ ಇದನ್ನು ಅರ್ಥಮಾಡಿಕೊಳ್ಳುವರು, ಸತ್ಯವೆಂದು ಒಪ್ಪಿಕೊಳ್ಳುವರು. ಆದರೆ ಅದರಂತೆ ನಡೆಯಿರಿ ಎಂದರೆ, ಓ ಅದಕ್ಕೆ ಕೋಟ್ಯಂತರ ವರ್ಷಗಳು ಹಿಡಿಯುವುವು ಎನ್ನುವರು.

ರಾಜನೊಬ್ಬನಿಗೆ ಹಲವು ಜನರು ಆಸ್ಥಾನದ ಹೊಗಳುಭಟ್ಟರಿದ್ದರು. ಅವರಲ್ಲಿ ಪ್ರತಿಯೊಬ್ಬರೂ ರಾಜನಿಗಾಗಿ ತಾವು ಪ್ರಾಣವನ್ನು ಬೇಕಾದರೂ ಅರ್ಪಿಸಬಲ್ಲೆವು, ತಾವೇ ಅತ್ಯಂತ ಪ್ರಾಮಾಣಿಕರು ಎಂದು ಹೇಳಿಕೊಳ್ಳುತ್ತಿದ್ದರು. ಒಂದು ದಿನ ಸಂನ್ಯಾಸಿಯೊಬ್ಬನು ರಾಜನ ಆಸ್ಥಾನಕ್ಕೆ ಬಂದನು. ರಾಜನು, ಇನ್ನಾವ ದೊರೆಗೂ ಪ್ರಪಂಚದಲ್ಲಿ ಇಷ್ಟು ಪ್ರಾಮಾಣಿಕರಾದ ಆಸ್ಥಾನಿಕರಿಲ್ಲ ಎಂದನು. ಸಂನ್ಯಾಸಿ ನಕ್ಕು ನಾನು ಇದನ್ನು ನಂಬುವುದಿಲ್ಲ ಎಂದನು. ರಾಜ ಸಂನ್ಯಾಸಿಗೆ ಬೇಕಾದರೆ ನೀವು ಪರೀಕ್ಷೆ ಮಾಡಬಹುದು ಎಂದನು. ಆಗ ಸಂನ್ಯಾಸಿ ರಾಜನ ಆಳ್ವಿಕೆ ದೀರ್ಘಕಾಲ ನಿಷ್ಕಳಂಕವಾಗಿರುವುದಕ್ಕೆ ತಾನು ದೊಡ್ಡದೊಂದು ಯಾಗವನ್ನು ಮಾರನೆಯ ದಿನ ಮಾಡಬೇಕಾಗಿದೆ, ಅದಕ್ಕಾಗಿ ಆಸ್ಥಾನದ ಪ್ರತಿಯೊಬ್ಬರೂ ರಾತ್ರಿ ಒಂದು ಚೊಂಬಿನಲ್ಲಿ ಹಾಲನ್ನು ತಂದು ದೊಡ್ಡದೊಂದು ಹಂಡೆಯಲ್ಲಿ ಹಾಕಬೇಕು ಎಂದನು. ರಾಜ ನಕ್ಕು ಇದೇನೆ ಪರೀಕ್ಷೆ ಎಂದನು. ರಾಜನು ಆಸ್ಥಾನದವರನ್ನೆಲ್ಲ ತನ್ನ ಬಳಿಗೆ ಕರೆದು ಅವರು ಏನು ಮಾಡಬೇಕೋ ಅದನ್ನು ಹೇಳಿದನು. ಆಸ್ಥಾನಿಕರೆಲ್ಲ ಇದಕ್ಕೆ ಸಂತೋಷದಿಂದ ಒಪ್ಪಿ ಹಿಂತಿರುಗಿದರು. ಅವರು ಅರ್ಧರಾತ್ರಿ ಸಮಯದಲ್ಲಿ ಒಂದೊಂದು ಚೆಂಬನ್ನು ತಂದು ಹಂಡೆಗೆ ಸುರಿದರು. ಆದರೆ ಮಾರನೆಯಬೆಳಿಗ್ಗೆ ನೋಡಿದಾಗ ಅದೆಲ್ಲ ಬರಿಯ ನೀರಾಗಿತ್ತು. ಆಸ್ಥಾನದವರನ್ನೆಲ್ಲ ಸೇರಿಸಿ ರಾಜ ಏತಕ್ಕೆ ಹೀಗಾಯಿತು ಎಂದು ಪ್ರಶ್ನಿಸಿದನು. ಅದರಲ್ಲಿ ಪ್ರತಿಯೊಬ್ಬರೂ 'ಇತರರೆಲ್ಲ ಹಾಲನ್ನು ಹಾಕುವರು, ನಾನು ಬರೀ ನೀರನ್ನು ಹಾಕಿದರೆ ಯಾರಿಗೂ ಗೊತ್ತಾಗುವುದಿಲ್ಲ ಎಂದು ಭಾವಿಸಿದ್ದೆ' ಎಂದರು. ದುರದೃಷ್ಟವಶಾತ್ ನಮ್ಮಲ್ಲಿ ಹಲವರು ಹೀಗೆ ಇರುವೆವು. ನಾವೆಲ್ಲ ನಮ್ಮ ಪಾಲಿಗೆ ಬಂದ ಕೆಲಸವನ್ನು ಆಸ್ಥಾನಿಕರು ಮಾಡಿದಂತೆ ಮಾಡುವೆವು.

ಈ ಜಗತ್ತಿನಲ್ಲಿ ಎಷ್ಟೋ ಸಮತ್ವದ ಭಾವನೆ ಇದೆ, ನನ್ನ ಸ್ವಲ್ಪ ಹಕ್ಕು - ಭಾಧ್ಯತೆ ಗೊತ್ತಾಗುವುದಿಲ್ಲ ಎಂದು ಬ್ರಾಹ್ಮಣ ಭಾವಿಸುವನು. ಹಾಗೆಯೇ ಶ‍್ರೀಮಂತರು ಮತ್ತು ಪ್ರತಿ ದೇಶದ ಪ್ರಜೆಗಳನ್ನು ಪೀಡಿಸುವವರು ಕೂಡ. ಯಾರು ದೌರ್ಜನ್ಯಕ್ಕೆ ತುತ್ತಾಗುವರೋ ಅವರಿಗಾದರೂ ಭರವಸೆಯಿದೆ; ಆದರೆ ದುರ್ಜನರಿಗೆ ಇಲ್ಲ. ದುರ್ಜನರು ಮುಕ್ತರಾಗಬೇಕಾದರೆ ಬಹಳ ಕಾಲ ಬೇಕು. ಆದರೆ ಇತರರು ಬೇಗ ಮುಕ್ತಿಯನ್ನು ಪಡೆಯುವರು. ಸಿಂಹದ ಕ್ರೌರ್ಯಕ್ಕಿಂತ ನರಿಯ ಕ್ರೌರ್ಯ ಭಯಾನಕವಾದುದು. ಸಿಂಹ ಒಂದು ಸಲ ಕೊಂದ ಮೇಲೆ ಕೆಲವು ಕಾಲ ಶಾಂತವಾಗಿರುವುದು. ಆದರೆ ಯಾವಾಗಲೂ ತನ್ನ ಬಲೆಗೆ ಬೀಳುವವರನ್ನು ಹುಡುಕುತ್ತಿರುವ ನರಿಯಾದರೋ ಒಂದು ಕ್ಷಣವೂ ಸುಮ್ಮನಿರುವುದಿಲ್ಲ. ಸ್ವಭಾವತಃ ಪುರೋಹಿತ ಕ್ರೂರಿ, ನಿರ್ದಯನು. ಆದಕಾರಣವೇ ಎಲ್ಲಿ ಪುರೋಹಿತರು ಮೇಲೇಳುವರೋ ಅಲ್ಲಿ ಧರ್ಮ ಅವನತಿಗೆ ಹೋಗುವುದು. ನಾವು ಹಕ್ಕು ಬಾಧ್ಯತೆಗಳನ್ನು ತ್ಯಜಿಸಬೇಕು, ಆಗ ಮಾತ್ರ ಧರ್ಮ ಸಾಧ್ಯ ಎಂದು ವೇದಾಂತ ಹೇಳುವುದು. ಅದಕ್ಕೆ ಮುಂಚೆ ಯಾವ ಧರ್ಮವೂ ಇಲ್ಲ.

ಕ್ರಿಸ್ತ ಹೇಳುವುದನ್ನು ನೀವು ನಂಬುತ್ತೀರ? 'ನಿಮ್ಮಲ್ಲಿರುವುದನ್ನೆಲ್ಲ ಮಾರಿ ಬಡವರಿಗೆ ಕೊಡಿ.” ಅನುಷ್ಠಾನಕ್ಕೆ ಯೋಗ್ಯವಾದ ಸಮತ್ವ ಇಲ್ಲಿದೆ. ಗ್ರಂಥ ಪೀಡನೆಯ ತಂಟೆಯೇ ಇಲ್ಲ. ಸತ್ಯವನ್ನು ಇದ್ದಂತೆ ತೆಗೆದುಕೊಳ್ಳಬೇಕು. ಗ್ರಂಥ ಪೀಡನೆಗೆ ಯತ್ನಿಸಬೇಡಿ. ಇದನ್ನು ಎಲ್ಲೋ ಕೆಲವು ಯೆಹೂದ್ಯರಿಗೆ ಮಾತ್ರ ಕ್ರಿಸ್ತ ಹೇಳಿದ್ದು ಎನ್ನುವರು ಕೆಲವರು. ಇದೇ ವಾದ ಉಳಿದುದಕ್ಕೂ ಅನ್ವಯಿಸುವುದು. ಸುಮ್ಮನೆ ಗ್ರಂಥವನ್ನು ಹಿಸುಕಾಡಿದರೆ ಪ್ರಯೋಜನವಿಲ್ಲ. ಸತ್ಯ ಹೇಗಿದೆಯೋ ಹಾಗೆ ಅದನ್ನು ಎದುರಿಸಲು ಧೈರ್ಯವಿರಲಿ, ನಮಗೆ ಅದು ಸಾಧ್ಯವಿಲ್ಲದೆ ಇದ್ದರೆ ನಮ್ಮ ದೌರ್ಬಲ್ಯವನ್ನು ಒಪ್ಪಿಕೊಳ್ಳೋಣ, ಆದರೆ ಆದರ್ಶವನ್ನು ಕೆಳಗೆ ಎಳೆಯದಿರೋಣ. ಎಂದಾದರೊಂದು ದಿನ ಆದರ್ಶವನ್ನು ಸೇರಬಹುದೆಂದು ಅದಕ್ಕಾಗಿ ಹೋರಾಡೋಣ. “ನಿಮ್ಮಲ್ಲಿರುವುದನ್ನೆಲ್ಲಾ ಮಾರಿ ದರಿದ್ರರಿಗೆ ಕೊಟ್ಟು ನನ್ನನ್ನು ಅನುಸರಿಸಿ.'' ಇದನ್ನು ಗಮನಿಸಿ. ಹಕ್ಕುಬಾಧ್ಯತೆಗಳನ್ನೆಲ್ಲ ಧ್ವಂಸಮಾಡಿ, ಯಾವುದು ಹಕ್ಕುಬಾಧ್ಯತೆಗೆ ಕಾರಣವಾಗುವುದೊ ಅದನ್ನೆಲ್ಲ ನಿರ್ಮೂಲ ಮಾಡಿ. ಮಾನವರನ್ನೆಲ್ಲ ಸಮತ್ವದ ಆದರ್ಶದೆಡೆಗೆ ಒಯ್ಯುವುದಕ್ಕಾಗಿ ನಾವು ಯತ್ನಿಸೋಣ. ನೀವು ಮತ್ತೊಬ್ಬನಿಗಿಂತ ಚೆನ್ನಾಗಿ ಮಾತನಾಡುತ್ತೀರಿ ಎಂದು ದಾರಿಹೋಕನಿಗಿಂತ ಮೇಲು ಎಂದು ಭಾವಿಸುವಿರೇನು? ಹೀಗೆ ನೀವು ಮಾಡಿದರೆ ಮುಕ್ತಿಯ ಕಡೆ ಹೋಗುತ್ತಿಲ್ಲ, ಹೆಚ್ಚು ಬಂಧನಕ್ಕೆ ಈಡಾಗುತ್ತಿರುವಿರಿ. ಎಲ್ಲಕ್ಕಿಂತ ಹೆಚ್ಚಾಗಿ ಆಧ್ಯಾತ್ಮಿಕ ಅಹಂಕಾರ ನಿಮ್ಮನ್ನು ಮೆಟ್ಟಿ ಕೊಂಡರೆ ಉಳಿಗಾಲವೇ ಇಲ್ಲ. ಬಂಧನಗಳೆಲ್ಲಕ್ಕಿಂತ ಭಯಾನಕ ಇದು. ಐಶ್ಚರ್ಯದ ಮತ್ತು ಮತ್ತಾವ ಪ್ರೇಮದ ಬಂಧನವೂ ಇದರಷ್ಟು ಆತ್ಮನನ್ನು ಬಂಧಿಸುವುದಿಲ್ಲ. 'ನಾನು ಇತರರಿಗಿಂತ ಪರಿಶುದ್ದನು' ಎಂಬುದೇ ನಿಮ್ಮ ಹೃದಯವನ್ನು ಪ್ರವೇಶಿಸುವ ಮಹಾ ಭಯಂಕರ ಭಾವನೆ. ನೀವು ಯಾವ ದೃಷ್ಟಿಯಲ್ಲಿ ಇತರರಿಗಿಂತ ಹೆಚ್ಚು ಪುಣ್ಯವಂತರು? ನಿಮ್ಮಲ್ಲಿರುವ ದೇವರೇ ಎಲ್ಲರಲ್ಲಿಯೂ ಇರುವನು. ಇದು ನಿಮಗೆ ಗೊತ್ತಿಲ್ಲದೆ ಇದ್ದರೆ ನಿಮಗೆ ಏನೂ ಗೊತ್ತಿಲ್ಲ. ವ್ಯತ್ಯಾಸ ಹೇಗೆ ಇರಬಲ್ಲದು? ಇರುವುದೆಲ್ಲ ಒಂದೇ, ಎಲ್ಲರೂ ಪರಮಾತ್ಮನ ದೇವಾಲಯಗಳೆ, ನೀವು ಅದನ್ನು ನೋಡಿದರೆ ಒಳ್ಳೆಯದು, ಇಲ್ಲದೇ ಇದ್ದರೆ ಆಧ್ಯಾತ್ಮಿಕತೆ ನಿಮಗೆ ಇನ್ನೂ ಬಂದಿಲ್ಲ.

