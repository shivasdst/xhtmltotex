
\chapter[ಮನಶ್ಶಾಸ್ತ್ರದ ಪ್ರಾಮುಖ್ಯ]{ಮನಶ್ಶಾಸ್ತ್ರದ ಪ್ರಾಮುಖ್ಯ\protect\footnote{\engfoot{C.W, Vol. VI, P. 28}}}

ಪಾಶ್ಚಾತ್ಯರಲ್ಲಿ ಮನಶ್ಶಾಸ್ತ್ರ ಅತಿ ಅಧೋಗತಿಗೆ ಇಳಿದಿದೆ. ಶಾಸ್ತ್ರಗಳಿಗೆಲ್ಲ ಶಾಸ್ತ್ರ ಮನಶ್ಶಾಸ್ತ್ರ. ಆದರೆ ಪಾಶ್ಚಾತ್ಯರಲ್ಲಿ ಅದನ್ನು ಇತರ ವಿಜ್ಞಾನಶಾಸ್ತ್ರಗಳಿಗೆ ಸಮನಾದ ಸ್ಥಳದಲ್ಲಿ ಇಟ್ಟಿರುವರು. ಅಂದರೆ ಅದನ್ನು ಪ್ರಯೋಜನದ ದೃಷ್ಟಿಯಿಂದ ಅಳೆಯುವರು.

ಇದರಿಂದ ಮನುಷ್ಯನಿಗೆ ಎಷ್ಟು ಮಟ್ಟಿಗೆ ನಿತ್ಯಜೀವನದಲ್ಲಿ ಉಪಯೋಗವಾಗುವುದು? ಪ್ರತಿದಿನ ವೃದ್ಧಿಯಾಗುತ್ತಿರುವ ಸುಖಕ್ಕೆ ಇದು ಮತ್ತೆಷ್ಟನ್ನು ಕೂಡಿಸಬಲ್ಲದು? ಪ್ರತಿದಿನ ವೃದ್ಧಿಯಾಗುತ್ತಿರುವ ಯಾತನೆಯಲ್ಲಿ ಇದರಿಂದ ಎಷ್ಟು ಪಾಲು ಕಡಮೆಯಾಗಬಲ್ಲುದು? ಪಾಶ್ಚಾತ್ಯರಲ್ಲಿ ಪ್ರತಿಯೊಂದನ್ನೂ ಅಳೆಯುವ ದೃಷ್ಟಿಯೇ ಇದು.

ನಮ್ಮ ಜ್ಞಾನದಲ್ಲಿ ಶೇಕಡಾ ತೊಂಬತ್ತರಷ್ಟು ನಮ್ಮ ಸುಖವನ್ನು ವೃದ್ಧಿ ಮಾಡುವುದಕ್ಕಾಗಲಿ ಅಥವಾ ನಮ್ಮ ವ್ಯಥೆಯನ್ನು ಕಡಿಮೆಮಾಡುವುದಕ್ಕಾಗಲಿ ಉಪಯೋಗಿಸಲು ಸಾಧ್ಯವಿಲ್ಲವೆಂಬುದನ್ನು ಜನ ಮರೆಯುವರು. ವಿಜ್ಞಾನದ ಯಾವುದೋ ಒಂದು ಅಲ್ಪಾಂಶ ಮಾತ್ರ ನಮ್ಮ ನಿತ್ಯ ಜೀವನಕ್ಕೆ ಸಹಕಾರಿಯಾಗಬಲ್ಲದು. ಇದು ಹೀಗೆ ಇರುವುದಕ್ಕೆ ಕಾರಣ ನಮ್ಮ ಮನಸ್ಸಿನ ಯಾವುದೋ ಒಂದು ಅಲ್ಪಾಂಶ ಮಾತ್ರ ವಿಷಯ ವಸ್ತುಗಳ ಕ್ಷೇತ್ರದಲ್ಲಿ ಇರುವುದೇ ಆಗಿದೆ. ನಮಗೆ ಇರುವ ವಿಷಯ ಪ್ರಜ್ಞೆ ಅತ್ಯಲ್ಪ ಮಾತ್ರ. ಅದೇ ನಮ್ಮ ಸಂಪೂರ್ಣ ಮನಸ್ಸು ಮತ್ತು ಜೀವನ ಎಂದು ಕಲ್ಪಿಸಿಕೊಳ್ಳುತ್ತೇವೆ. ಆದರೆ ಇದು ನಮ್ಮ ಅಪ್ರಜ್ಞಾ ಸಾಗರದಲ್ಲಿ ಒಂದು ಬಿಂದು ಮಾತ್ರ. ನಮ್ಮಲ್ಲಿರುವುದೆಲ್ಲ ಬರಿಯ ಇಂದ್ರಿಯಗ್ರಹಣದ ಕಂತೆಯಾದರೆ, ಇದರ ಮೂಲಕ ಪಡೆಯುವ ಜ್ಞಾನವೆಲ್ಲವನ್ನು ಇಂದ್ರಿಯತೃಪ್ತಿಗಾಗಿ ಉಪಯೋಗಿಸಿಕೊಳ್ಳಬಹುದು. ಆದರೆ ಅದೃಷ್ಟವಶಾತ್ ಅದು ಹಾಗಿಲ್ಲ. ನಾವು ಮೃಗೀಯ ಸ್ಥಿತಿಯಿಂದ ಮೇಲೆ ಮೇಲೆ ಹೋದಷ್ಟೂ ನಮ್ಮ ಇಂದ್ರಿಯ ಸುಖ ಕಡಮೆ ಕಡಮೆಯಾಗುತ್ತಾ ಬರುವುದು. ಮತ್ತು ದಿನೇ ದಿನೇ ವೃದ್ಧಿಯಾಗುತ್ತಿರುವ ವಿಜ್ಞಾನದಲ್ಲಿ ಮತ್ತು ಮನಶ್ಶಾಸ್ತ್ರದಲ್ಲಿ ಅಭಿರುಚಿ ಹೆಚ್ಚುತ್ತಾ ಹೋಗುವುದು. ಜ್ಞಾನಕ್ಕಾಗಿ ಜ್ಞಾನವೇ, (ಅದು ಎಷ್ಟು ಇಂದ್ರಿಯ ಸುಖವನ್ನು ಕೊಡಬಹುದೆಂಬುದರ ಗಮನವಿಲ್ಲದೆ) ಮನಸ್ಸಿಗೆ ಅತಿ ಶ್ರೇಷ್ಠ ಆನಂದವಾಗಿ\break ಪರಿಣಮಿಸುವುದು.

ಪಾಶ್ಚಾತ್ಯರ ಪ್ರಯೋಜನ ದೃಷ್ಟಿಯಿಂದಲೇ ಮನಶ್ಶಾಸ್ತ್ರವನ್ನು ಅಳೆದರೂ ಅದು ವಿಜ್ಞಾನಗಳ ವಿಜ್ಞಾನವಾಗುವುದು. ಇದು ಹೇಗೆ? ನಾವೆಲ್ಲ ಪಂಚೇಂದ್ರಿಯಗಳ\break ಗುಲಾಮರು, ಸುಪ್ತ ಮತ್ತು ಜಾಗೃತ ಮನಸ್ಸಿನ ಗುಲಾಮರು. ಒಬ್ಬ ದುರಾತ್ಮನಾಗಿರುವುದಕ್ಕೆ ಕಾರಣ ಅವನು ಇಚ್ಛೆಪಟ್ಟು ಹಾಗೆ ಆಗಿದ್ದಾನೆ ಎಂದಲ್ಲ. ಆದರೆ ಅವನ ಮನಸ್ಸು ಅವನ ವಶದಲ್ಲಿ ಇಲ್ಲದೆ ಹಾಗೆ ಆಗಿರುವನು. ಆದಕಾರಣವೆ ಅವನು ತನ್ನ ಸುಪ್ತ ಮತ್ತು ಜಾಗೃತ ಮನಸ್ಸಿಗೆ ಮತ್ತು ಇತರರೆಲ್ಲರ ಮನಸ್ಸಿಗೆ ಅಡಿಯಾಳಾಗಿರುವನು. ಅವನು ತನ್ನ ಮನಸ್ಸಿನ ಪ್ರಮುಖವಾದ ಸ್ವಭಾವವನ್ನು ಅನುಸರಿಸಬೇಕಾಗಿದೆ. ಅವನು ಹಾಗೆ ಮಾಡದೆ ವಿಧಿಯಿಲ್ಲ. ತಾನು ತನ್ನ ಸ್ವಾಧೀನದಲ್ಲಿಲ್ಲದೆ, ತನ್ನ ಉತ್ತಮ ಸ್ವಭಾವ ಬೇಡ ಎಂದು ಹೇಳುತ್ತಿದ್ದರೂ, ಅವನು ತನ್ನ ಮನಸ್ಸು ಯಾವುದನ್ನು ಒತ್ತಾಯ ಮಾಡುವುದೋ ಅದರಂತೆ ಮಾಡಬೇಕಾಗಿದೆ. ಪಾಪ, ಅವನು ಹಾಗೆ ಮಾಡದೆ ವಿಧಿಯಿಲ್ಲ. ನಾವು ನಮ್ಮ ಜೀವನದಲ್ಲಿ ಇದನ್ನು ನಿತ್ಯ ನೋಡುತ್ತಿರುವೆವು. ನಾವು ಅನೇಕ ವೇಳೆ ನಮ್ಮ ಉತ್ತಮ ಸ್ವಭಾವಕ್ಕೆ ವಿರೋಧವಾಗಿ ಕೆಲಸ ಮಾಡುತ್ತಿರುವೆವು. ಅನಂತರ ಹೀಗೆ ಮಾಡಿದ್ದಕ್ಕೆ ನಮ್ಮನ್ನು ನಾವೇ ನಿಂದಿಸಿಕೊಳ್ಳುವೆವು. ಇದನ್ನು ಮಾಡಲು ನಮಗೆ ಹೇಗೆ ಸಾಧ್ಯವಾಯಿತು ಎಂದು ನಾವೇ ಆಶ್ಚರ್ಯಪಡುವೆವು. ಆದರೆ ಪುನಃ ಪುನಃ ನಾವು ಅದನ್ನೇ ಮಾಡುವೆವು, ಅದಕ್ಕಾಗಿ ವ್ಯಥೆಪಡುವೆವು, ಅದಕ್ಕಾಗಿ ನಮ್ಮನ್ನು ನಾವು ತಿರಸ್ಕರಿಸಿಕೊಳ್ಳುವೆವು. ಆ ಕಾಲದಲ್ಲಿ ಹಾಗೆ ಮಾಡಬೇಕು ಎಂದು ಅನ್ನಿಸುವುದು ಎಂದು ಭಾವಿಸುವೆವು. ಆದರೆ ಹಾಗೆ ಇಚ್ಛೆಪಡುವಂತೆ ಮತ್ತಾವುದೊ ನಮ್ಮನ್ನು ಬಲಾತ್ಕರಿಸುವುದು, ನಮಗೆ ಹಾಗೆ ಮಾಡದೆ ವಿಧಿಯಿಲ್ಲ. ನಾವೆಲ್ಲ ನಮ್ಮ ಮತ್ತು ಇತರರ ಮನಸ್ಸಿಗೆ ದಾಸರು. ನಾವು ಒಳ್ಳೆಯವರಾಗಲಿ, ಕೆಟ್ಟವರಾಗಲಿ, ಏನೂ ವ್ಯತ್ಯಾಸವಿಲ್ಲ. ನಮಗೆ ವಿಧಿ ಇಲ್ಲದೆ ಇರುವುದರಿಂದ ಅಲ್ಲಿ ಇಲ್ಲಿ ಮನಸ್ಸು ಹೇಳಿದಂತೆ ಕೇಳಬೇಕಾಗಿದೆ. ನಾವು ಆಲೋಚಿಸುತ್ತೇವೆ, ನಾವು ಕೆಲಸ ಮಾಡುತ್ತೇವೆ, ಎಂದು ಭಾವಿಸುವೆವು. ಇದು ಹಾಗಲ್ಲ. ನಾವು ಆಲೋಚಿಸಲೇಬೇಕಾಗಿರುವುದರಿಂದ ಆಲೋಚಿಸುತ್ತೇವೆ, ನಾವು ಕೆಲಸ ಮಾಡಲೇಬೇಕಾಗಿರುವುದರಿಂದ ಮಾಡುತ್ತೇವೆ. ನಾವೆಲ್ಲ ನಮಗೆ ಮತ್ತು ಇತರರಿಗೆ ದಾಸರು. ನಮ್ಮ ಸುಪ್ತಚೇತನದಲ್ಲಿ ಈ ಒಂದು ಜನ್ಮದ ಮಾತ್ರವಲ್ಲ ಹಲವು ಜನ್ಮಗಳ ಕ್ರಿಯೆಗಳು, ಆಲೋಚನೆಗಳು ಮತ್ತು ಭಾವನೆಗಳೆಲ್ಲ ಶೇಖರವಾಗಿವೆ. ಈ ಅಸೀಮವಾದ ನಾನೆಂಬ ಮನಸ್ಸಾಗರದಲ್ಲಿ ನಮ್ಮ ಹಿಂದಿನ ಆಲೋಚನಾಕ್ರಿಯೆಗಳೆಲ್ಲ ಇವೆ. ಇವುಗಳಲ್ಲಿ ಪ್ರತಿಯೊಂದೂ ಒಂದು ಅಲೆಯಂತೆ ನನ್ನ ಜಾಗೃತ ಚಿತ್ರದ ಹಿಂದೆ ಬಂದು ನಿಂತು, ತನ್ನನ್ನು ಒಪ್ಪಿಕೊಳ್ಳುವಂತೆ ಬಲಾತ್ಕರಿಸುತ್ತಿದೆ. ಈ ಸಂಗ್ರಹವಾದ ಶಕ್ತಿಯನ್ನೇ, ಆಲೋಚನೆಯನ್ನೇ ನಮ್ಮ ಸ್ವಭಾವ ಎನ್ನುವೆವು. ಅವುಗಳ ನಿಜವಾದ ಮೂಲ ನಮಗೆ ಗೊತ್ತಾಗುವುದಿಲ್ಲ. ಅವುಗಳನ್ನು ಕಣ್ಣು ಮುಚ್ಚಿಕೊಂಡು ಪ್ರಶ್ನಿಸದೆ ನಾವು ಅನುಸರಿಸುತ್ತೇವೆ. ಇದರ ಪರಿಣಾಮವೇ ಅತಿ ದಾರುಣವಾದ ಗುಲಾಮಗಿರಿ. ಆದರೂ ನಾವು ಮುಕ್ತರೆಂದು ಹೆಮ್ಮೆ ಕೊಚ್ಚಿಕೊಳ್ಳುತ್ತೇವೆ. ಮುಕ್ತರೆ! ನಮ್ಮ ಮನಸ್ಸನ್ನು ಕ್ಷಣಕಾಲ ನಿಗ್ರಹಿಸಲಾರೆವು, ಉಳಿದ ವಸ್ತುಗಳನ್ನು ತೊರೆದು ಒಂದು ವಿಷಯದ ಮೇಲೆ ನಿಲ್ಲಿಸಲಾರೆವು! ಆದರೂ ನಾವು ಸ್ವತಂತ್ರರು ಎಂದು ಕರೆದುಕೊಳ್ಳುತ್ತೇವೆ. ಇದನ್ನು ಆಲೋಚಿಸಿ ನೋಡಿ! ಒಂದು ಕ್ಷಣಕಾಲವೂ ನಾವು ಇಚ್ಛಿಸಿದಂತೆ ಮಾಡಲಾರೆವು. ಯಾವುದೋ ಒಂದು ಆಸೆ ನಮ್ಮನ್ನು ಮೆಟ್ಟಿಕೊಳ್ಳುವುದು. ತಕ್ಷಣ ಅದು ಹೇಳಿದಂತೆ ಕೇಳುವೆವು. ನಮ್ಮ ಅಂತರಾತ್ಮ ಆ ದುರ್ಬಲತೆ ತಪ್ಪು ಎಂದು ಸಾರುವುದು. ಆದರೆ ಅದನ್ನು ಪದೇ ಪದೇ ಮಾಡುತ್ತೇವೆ. ಯಾವಾಗಲೂ ಅದನ್ನೆ ಮಾಡುತ್ತಿರುವೆವು. ನಾವು ಎಷ್ಟು ಪ್ರಯತ್ನಪಟ್ಟರೂ ಉತ್ತಮ ಆದರ್ಶಕ್ಕೆ ತಕ್ಕಂತೆ ಬಾಳಲಾರೆವು. ನಮ್ಮ ಹಿಂದಿನ ಜನ್ಮಗಳ ಆಲೋಚನೆ ಮತ್ತು ಕರ್ಮ ಭೂತ ನಮ್ಮನ್ನು ಹಾಗೆ ಮಾಡಗೊಡುವುದಿಲ್ಲ. ಈ ಜಗತ್ತಿನ ದುಃಖಕ್ಕೆಲ್ಲ ಕಾರಣ ಇಂದ್ರಿಯದ ಗುಲಾಮಗಿರಿ. ಈ ಇಂದ್ರಿಯ ಪ್ರಪಂಚವನ್ನು ಮೀರಿ ಹೋಗಲು ಆಗದೆ ಇರುವುದು. ಕೇವಲ ದೈಹಿಕ ಮತ್ತು ಮಾನಸಿಕ ಆಸೆ ಆಕಾಂಕ್ಷೆಗಳನ್ನು ಅರಸಿಕೊಂಡು ಹೋಗುವುದೇ ಪ್ರಪಂಚದ ದುಃಖಕ್ಕೆಲ್ಲ ಮತ್ತು ವ್ಯಥೆಗೆಲ್ಲ ಕಾರಣ.

ಈ ಮನಶ್ಶಾಸ್ತ್ರವು, ಮನಸ್ಸಿನ ಅಲೆತವನ್ನು ನಿಗ್ರಹಿಸಿ, ನಿಮ್ಮ ಇಚ್ಛೆಯ ಸ್ವಾಧೀನಕ್ಕೆ ತನ್ನಿ, ನಿಮ್ಮ ಪೂರ್ವ ಸಂಸ್ಕಾರಗಳ ಸೆಳೆತದಿಂದ ಪಾರಾಗಿ ಎಂದು ಬೋಧಿಸುವುದು. ಆದಕಾರಣವೆ ಮನಶ್ಶಾಸ್ತ್ರವೇ ವಿಜ್ಞಾನಗಳಿಗೆಲ್ಲ ವಿಜ್ಞಾನವಾಗಿರುವುದು. ಇದಿಲ್ಲದೆ ವಿಜ್ಞಾನ ಮತ್ತು ಇತರ ಶಾಸ್ತ್ರಗಳಿಂದ ಪ್ರಯೋಜನವಿಲ್ಲ.

ನಿಗ್ರಹಕ್ಕೆ ಒಳಗಾಗದ ಮನಸ್ಸು ಎಂದೆಂದಿಗೂ ನಮ್ಮನ್ನು ಅಧಃಪತನಕ್ಕೆ ಎಳೆಯುವುದು, ನಮ್ಮನ್ನು ಹಿಂಸಿಸುವುದು, ನಮ್ಮನ್ನೇ ನಾಶ ಮಾಡುವುದು. ನಿಗ್ರಹಿಸಿದ, ಒಳ್ಳೆಯ ದಾರಿಗೆ ತಿರುಗಿಸಿದ ಮನಸ್ಸು ನಮ್ಮನ್ನು ಉದ್ದಾರ ಮಾಡುವುದು, ಮುಕ್ತರನ್ನಾಗಿ ಮಾಡುವುದು. ಆದಕಾರಣವೇ ಮನಸ್ಸನ್ನು ನಿಗ್ರಹಿಸಬೇಕು. ಮನಶ್ಶಾಸ್ತ್ರವು ಅದನ್ನು ಹೇಗೆ ನಿಗ್ರಹಿಸಬೇಕು ಎಂಬುದನ್ನು ತೋರುವುದು.

ನಾವು ಯಾವುದೇ ಭೌತವಿಜ್ಞಾನವನ್ನು ಓದಬೇಕಾದರೂ ಸಾಧ್ಯವಾದಷ್ಟು ಅಂಕಿಅಂಶಗಳನ್ನೆಲ್ಲ ಸಂಗ್ರಹಿಸಬೇಕು. ಅವುಗಳನ್ನು ಅಧ್ಯಯನಮಾಡಿ ವಿಶ್ಲೇಷಿಸಬೇಕು. ಇವುಗಳ ಮೂಲಕ ಒಂದು ಶಾಸ್ತ್ರಜ್ಞಾನ ಸಿದ್ಧಿಸುವುದು. ಆದರೆ ಮನಸ್ಸನ್ನು ಪರೀಕ್ಷಿಸಲು, ಅದನ್ನು ವಿಶ್ಲೇಷಿಸಲು ಅಂಕಿ ಅಂಶಗಳಾಗಲಿ, ಹೊರಗಿನಿಂದ ದೊರಕುವ ವಾಸ್ತವಾಂಶಗಳಾಗಲಿ ಎಲ್ಲರಿಗೂ ದೊರಕುವಂತಿಲ್ಲ. ಮನಸ್ಸೇ ಮನಸ್ಸನ್ನು ವಿಶ್ಲೇಷಣೆ ಮಾಡಬೇಕಾಗಿದೆ. ಆದಕಾರಣವೇ ಶ್ರೇಷ್ಠ ವಿಜ್ಞಾನವೇ ಮನೋವಿಜ್ಞಾನ, ಮನಶ್ಶಾಸ್ತ್ರ.

ಪಾಶ್ಚಾತ್ಯರಲ್ಲಿ ಮನಸ್ಸಿನ ಶಕ್ತಿಯನ್ನು, ಅದರಲ್ಲೂ ಅದರ ಅಸಾಧಾರಣ ಶಕ್ತಿಯನ್ನು, ಒಂದು ಮಾಟವಿದ್ಯೆ ಮತ್ತು ರಹಸ್ಯ ವಿದ್ಯೆ ಎಂದು ಭಾವಿಸುವರು. ಹಿಂದೂ ದೇಶದ ಕೆಲವು ಫಕೀರರು ಆಚರಿಸುತ್ತಾರೆ ಎಂದು ಭಾವಿಸಲಾಗಿರುವ ರಹಸ್ಯ ವಿದ್ಯೆಗಳೊಂದಿಗೆ ಮನಶ್ಶಾಸ್ತ್ರವನ್ನು ಗಂಟುಹಾಕಿರುವುದರಿಂದ ನಿಜವಾದ ಮನಶ್ಶಾಸ್ತ್ರದ ಪ್ರಗತಿ ಕುಂಠಿತವಾಗಿದೆ.

ಭೌತಶಾಸ್ತ್ರಜ್ಞರಿಗೆ ಪ್ರಪಂಚದಲ್ಲೆಲ್ಲಾ ಒಂದೇ ರೀತಿಯ ಪರಿಣಾಮ ಕಾಣುವುದು. ಅವರು ಸಂಗ್ರಹಿಸುವ ಅಂಕಿಅಂಶಗಳಲ್ಲಿ ವ್ಯತ್ಯಾಸವಿರುವಿರುವುದಿಲ್ಲ. ಆದಕಾರಣ ಅವರ ಪ್ರಯೋಗದ ಫಲದಲ್ಲಿಯೂ ಯಾವ ವ್ಯತ್ಯಾಸವೂ ಕಾಣುವುದಿಲ್ಲ. ಇದಕ್ಕೆ ಕಾರಣವೇನೆಂದರೆ, ವಿಜ್ಞಾನಶಾಸ್ತ್ರಕ್ಕೆ ಬೇಕಾದ ಅಂಕಿಅಂಶಗಳನ್ನು ಎಲ್ಲರೂ ಸಂಗ್ರಹಿಸಬಹುದು ಮತ್ತು ಅವು ಸಾರ್ವತ್ರಿಕವಾಗಿ ಒಪ್ಪಿಗೆಯನ್ನು ಪಡೆದಿವೆ. ಇವುಗಳ ಆಧಾರದ ಮೇಲೆ ಪಡೆದ ಪ್ರತಿಫಲಗಳು ತರ್ಕಬದ್ಧವಾಗಿರುತ್ತದೆ. ಎಲ್ಲರಿಗೂ ಕಾಣುವ ವಸ್ತುಗಳ ಆಧಾರದ ಮೇಲೆ ಮಾಡಿದ ಪ್ರಯೋಗ ನ್ಯಾಯ ಸಮ್ಮತವೆಂದು ಎಲ್ಲರೂ ಒಪ್ಪುವರು. ಆದರೆ ಮನಸ್ಸಿನ ಕ್ಷೇತ್ರದಲ್ಲಿ ಇದು ಹಾಗಿಲ್ಲ. ಇಲ್ಲಿ ಯಾವ ಅಂಕಿ ಅಂಶಗಳೂ ಇಲ್ಲ. ಇಂದ್ರಿಯಗಳಿಗೆ ಸ್ಥೂಲವಾಗಿ ಕಾಣುವ ಯಾವ ಘಟನೆಗಳೂ ಇಲ್ಲ. ಎಲ್ಲರಿಗೂ ಕಾಣುವ ವಸ್ತುಗಳೂ ಇಲ್ಲ. ಆದಕಾರಣ ಎಲ್ಲರೂ ಮನಸ್ಸನ್ನು ಪರೀಕ್ಷೆಮಾಡಿ ಒಂದು ಮನಶ್ಶಾಸ್ತ್ರೀಯ ಸಿದ್ಧಾಂತಕ್ಕೆ ಬರಲು\break ಸಾಧ್ಯವಿಲ್ಲ.

ಆಂತರ್ಯದಲ್ಲಿ ನಿಜವಾದ ಮಾನವನಿರುವನು, ಅದೇ ಆತ್ಮ. ಮನಸ್ಸನ್ನು\break ಅಂತರ್ಮುಖಗೊಳಿಸಿ ಆತ್ಮನಲ್ಲಿ ಐಕ್ಯರಾಗಿ, ಇಂತಹ ಸ್ಥಿರವಾದ ನೆಲೆಯಲ್ಲಿ ನಿಂತುಕೊಂಡು ಮನಸ್ಸಿನ ಚಂಚಲತೆಯನ್ನು ಗಮನಿಸಿ, ಎಲ್ಲ ವ್ಯಕ್ತಿಗಳಲ್ಲಿಯೂ ಇರುವ ವಾಸ್ತವಾಂಶಗಳನ್ನು ಅರಿಯಬಹುದು. ಯಾರು ಆಳಕ್ಕೆ ಹೋಗಬಲ್ಲರೋ ಅಂಥವರು ಮಾತ್ರ ಮಾನಸಿಕ ಕ್ಷೇತ್ರದ ವಾಸ್ತವಾಂಶಗಳನ್ನು ನೋಡಬಲ್ಲರು. ಜಗತ್ತಿನಲ್ಲೆಲ್ಲಾ ತಾವು ಯೋಗಿಗಳು ಎಂದು ತಮ್ಮನ್ನು ತಾವು ಕರೆಸಿಕೊಳ್ಳುವವರಲ್ಲಿ, ಮನಸ್ಸಿನ ಸ್ವಭಾವ ಮತ್ತು ಅದರ ಶಕ್ತಿಸಾಧ್ಯತೆಗಳ\break ವಿಷಯಗಳಲ್ಲಿ ಎಷ್ಟೋ ಭಿನ್ನಾಭಿಪ್ರಾಯ ಇರುವುದು. ಏಕೆಂದರೆ ಅವರು ಆಳಕ್ಕೆ ಹೋಗಿರುವುದಿಲ್ಲ. ಅಂಥವರು ತಮ್ಮ ಮತ್ತು ಇತರರ ಮನಸ್ಸನ್ನು ಸ್ವಲ್ಪ ತಿಳಿದುಕೊಂಡಿರುವರು. ಆದರೆ ಇಂತಹ ತೋರಿಕೆಯ ಆವಿರ್ಭಾವದ ಹಿಂದೆ ಇರುವ ಮುಖ್ಯ ವಿಷಯಗಳನ್ನು ಅರಿಯದೆ, ತಮಗೆ ಯಾವುದು ತೋರುವುದೊ ಅದೇ ಎಲ್ಲರಿಗೂ ಅನ್ವಯಿಸುವುದು ಎಂದು ಸಾರುವರು. ಪ್ರತಿಯೊಬ್ಬ ಮತಿಗೆಟ್ಟ ಯೋಗಿಯೂ ಮನಸ್ಸಿಗೆ ಅನ್ವಯಿಸುವ ಕೆಲವು ವಿಷಯಗಳು ತನಗೆ ಗೊತ್ತಿವೆ, ಬೇಕಾದರೆ ಅವನ್ನು ಪರೀಕ್ಷಿಸಬಹುದು ಎನ್ನುವನು. ಆದರೆ ಇದು ಅವನ ಕಲ್ಪನೆಯಲ್ಲದೆ ಮತ್ತೇನೂ ಅಲ್ಲ.

ನೀವು ಮನಸ್ಸನ್ನು ಅಧ್ಯಯನ ಮಾಡಬೇಕಾದರೆ ಅದಕ್ಕೆ ಸರಿಯಾದ ತರಬೇತಿ ಆವಶ್ಯಕ. ಮನಸ್ಸಿನ ಚಂಚಲತೆಗೆ ಸಿಕ್ಕಿ ಬೀಳದೆ ಅದನ್ನು ಸ್ಥಿರವಾಗಿ ನಿಂತು ನೋಡಬೇಕಾದರೆ ಮನಸ್ಸನ್ನು ತನ್ನ ಸ್ವಾಧೀನಕ್ಕೆ ತಂದಿರಬೇಕು. ಇಲ್ಲದೆ ಇದ್ದರೆ ನಾವು ನೋಡಿದುದನ್ನು ನೆಚ್ಚುವುದಕ್ಕೆ\break ಆಗುವುದಿಲ್ಲ. ಅದು ಎಲ್ಲರಿಗೂ ಅನ್ವಯಿಸುವುದಿಲ್ಲ. ಆದಕಾರಣ ಅವು ವಾಸ್ತವಿಕ ಅಂಶಗಳಾಗಲಾರವು.

ಯಾರು ಮನಸ್ಸನ್ನು ಪರೀಕ್ಷಿಸಲು ತುಂಬಾ ಆಳಕ್ಕೆ ಹೋಗುವರೋ ಅಂತಹವರು ಪ್ರಪಂಚದ ಯಾವ ಮೂಲೆಯಲ್ಲಿರಲಿ, ಯಾವ ಧರ್ಮಕ್ಕೆ ಸೇರಿರಲಿ ಅವರು ನೋಡುವುದರಲ್ಲಿ ಯಾವ ವ್ಯತ್ಯಾಸವೂ ಇರುವುದಿಲ್ಲ. ಮನಸ್ಸಿನ ಆಳಕ್ಕೆ ಹೋಗುವವರೆಲ್ಲ ನೋಡುವುದು ಒಂದೇ ಬಗೆಯಾಗಿರುವುದು.

ಮನಸ್ಸು ಇಂದ್ರಿಯ ಗ್ರಹಣದಿಂದ ಮತ್ತು ಪ್ರೇರಣೆಯಿಂದ ಕೆಲಸ ಮಾಡುವುದು. ಉದಾಹರಣೆಗೆ ಬೆಳಕಿನ ಕಿರಣ ನನ್ನ ಕಣ್ಣಿನ ಮೇಲೆ ಬೀಳುವುದು, ಅನಂತರ ನರಗಳು ಅದನ್ನು ಮೆದುಳಿಗೆ ಒಯ್ಯುವುವು, ಆದರೂ ಬೆಳಕನ್ನು ನಾನು ಕಾಣುವುದಿಲ್ಲ. ಮೆದುಳು ಅನಂತರ ಸಂವೇದನೆಯನ್ನು ಮನಸ್ಸಿಗೆ ತಿಳಿಸುವುದು. ಆದರೂ ಬೆಳಕನ್ನು ನಾನು ಕಾಣುವುದಿಲ್ಲ. ಆಗ ಮನಸ್ಸು ಪ್ರತಿಕ್ರಿಯೆ ತೋರಿಸುತ್ತದೆ. ಆಗ ನನಗೆ ಬೆಳಕು ಮನಸ್ಸಿನಲ್ಲಿ ಹೊಳೆಯುವುದು. ಮನಸ್ಸಿನ ಪ್ರತಿಕ್ರಿಯೆಯೇ ಪ್ರೇರಣೆ, ಇದರ ಪರಿಣಾಮವಾಗಿ ಕಣ್ಣು ಬೆಳಕನ್ನು ನೋಡುವುದು.

ಮನಸ್ಸನ್ನು ನಿಗ್ರಹಿಸಬೇಕಾದರೆ ಸುಪ್ತಚೇತನದ ಆಳಕ್ಕೆ ಹೋಗಬೇಕು. ಅಲ್ಲಿರುವ ಸಂಸ್ಕಾರಗಳನ್ನೆಲ್ಲಾ, ಆಲೋಚನೆಗಳನ್ನೆಲ್ಲಾ, ನಾವು ಸರಿಯಾಗಿ ಜೋಡಿಸಿ ಅನಂತರ ಅವನ್ನು ನಿಗ್ರಹಿಸಬೇಕು. ಇದೇ ಮೊದಲನೆಯ ಹೆಜ್ಜೆ. ನೀವು ಸುಪ್ತಮನಸ್ಸನ್ನು ನಿಗ್ರಹಿಸಿದರೆ ಆಗ ಜಾಗ್ರತಾವಸ್ಥೆಯ ನಿಮ್ಮ ಮನಸ್ಸು ನಿಮ್ಮ ವಶಕ್ಕೆ ಬರುವುದು.

