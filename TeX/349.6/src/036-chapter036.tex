
\chapter{ವೇದಾಂತದ ಆದರ್ಶ ಮತ್ತು ಪ್ರಭಾವ \protect\footnote{\enginline{* C.W, Vol. I, P. 387}}\\(ಬಾಸ್ಟನ್ನಿನ ಟ್ವೆಂಟಿಯತ್ ಸೆಂಚುರಿ ಕ್ಲಬ್‌ನಲ್ಲಿ ನೀಡಿದ ಪ್ರವಚನ)}

ನಾನು ಇಂದಿನ ವಿಷಯವನ್ನು ಕುರಿತು ಮಾತನಾಡುವುದಕ್ಕೆ ಮುಂಚೆ ಸ್ವಲ್ಪ ಕಾಲಾವಕಾಶವಿರುವುದರಿಂದ, ನಿಮಗೆ ಧನ್ಯವಾದವನ್ನು ಅರ್ಪಿಸುವುದಕ್ಕೆ ನನಗೆ ಅವಕಾಶ ಕೊಡುವಿರಿ ಎಂದು ಭಾವಿಸುತ್ತೇನೆ. ನಾನು ನಿಮ್ಮೊಡನೆ ಮೂರುವರ್ಷಗಳನ್ನು ಕಳೆದೆ. ನಾನು ಮುಕ್ಕಾಲು ಪಾಲು ಅಮೆರಿಕ ದೇಶವನ್ನೆಲ್ಲ ಸಂಚರಿಸಿರುವೆನು. ನಾನು ಇಲ್ಲಿಂದ ನನ್ನ ದೇಶಕ್ಕೆ ಹಿಂದಿರುಗಿಹೋಗುವುದರಲ್ಲಿರುವೆನು. ಇಂತಹ ಸಮಯದಲ್ಲಿ ಅಮೆರಿಕದ ಅಥೆನ್ಸ್‌ನಂತಿರುವ ಈ ನಗರದಲ್ಲಿ (ಬಾಸ್ಟನ್) ನಿಮಗೆ ಧನ್ಯವಾದವನ್ನು ಅರ್ಪಿಸುವುದು ಸೂಕ್ತವಾಗಿದೆ. ನಾನು ಈ ದೇಶಕ್ಕೆ ಮೊದಲು ಬಂದಾಗ, ಕೆಲವು ದಿನಗಳ ಮೇಲೆ ಈ ದೇಶದ ವಿಚಾರವಾಗಿ ಒಂದು ಪುಸ್ತಕ ಬರೆಯಬೇಕೆಂದು ಇದ್ದೆ. ಆದರೆ ಮೂರು ವರ್ಷ ಈ ದೇಶದಲ್ಲಿ ಇದ್ದ ಮೇಲೆಯೂ ನನಗೆ ಒಂದು ಪುಟವನ್ನೂ ಬರೆಯಲು ಆಗಲಿಲ್ಲ. ನಾನು ಹಲವು ದೇಶಗಳನ್ನು ಸಂಚರಿಸಿದ ಮೇಲೆ, ಆಹಾರ ಬಟ್ಟೆ ಬರೆ ಮತ್ತು ನಡವಳಿಕೆ ಇವುಗಳಲ್ಲಿ ಎಷ್ಟೇ ವ್ಯತ್ಯಾಸ ಕಂಡರೂ ಮಾನವನು ಎಲ್ಲಾ ಕಡೆಗಳಲ್ಲಿಯೂ ಒಂದೇ ಎಂದು ತೋರುವುದು. ಅದ್ಭುತವಾದ ಮಾನವ ಸ್ವಭಾವ ಎಲ್ಲಾ ಕಡೆಗಳಲ್ಲಿಯೂ ಒಂದೇ ಆಗಿದೆ. ಆದರೂ ಕೆಲವು ವೈಶಿಷ್ಟ್ಯಗಳಿವೆ. ಕೆಲವು ಮಾತುಗಳಲ್ಲಿ ನನ್ನ ಅನುಭವವನ್ನು ವ್ಯಕ್ತಪಡಿಸುವೆನು. ಅಮೆರಿಕಾ ದೇಶದಲ್ಲಿ ವ್ಯಕ್ತಿಗೆ ಸಂಬಂಧಿಸಿದ ಯಾವ ಪ್ರಶ್ನೆಯನ್ನೂ ಹಾಕುವುದಿಲ್ಲ. ಒಬ್ಬನು ನಿಜವಾಗಿ ಮಾನವನಾಗಿದ್ದರೆ ಸಾಕು, ಅವನನ್ನು ಸಂಪೂರ್ಣ ಪ್ರೀತಿಸುವರು. ಈ ವಿಶೇಷವನ್ನು ಪ್ರಪಂಚದ ಬೇರೆ ಯಾವ ಭಾಗದಲ್ಲಿಯೂ ನಾನು ಕಂಡಿಲ್ಲ.

ನಾನು ಈ ದೇಶಕ್ಕೆ ಭಾರತೀಯ ತತ್ತ್ವಶಾಸ್ತ್ರದ ಪ್ರತಿನಿಧಿಯಾಗಿ ಬಂದೆ. ಅದನ್ನು ವೇದಾಂತ ಎನ್ನುವರು. ಇದು ಬಹಳ ಪುರಾತನವಾದುದು. ವೇದಗಳು ಎಂದು ಕರೆಯುವ ಪುರಾತನ ಆರ್ಯರ ಸಾಹಿತ್ಯದಿಂದ ಬಂದುದು ಇದು. ಹಲವು ಶತಮಾನಗಳಿಂದ ಸಂಗ್ರಹಿಸಲ್ಪಟ್ಟ ಆ ವೇದರಾಶಿಯಿಂದ ಬಂದುದು ಇದು. ಅವರ ಕಲ್ಪನೆ, ಅನುಭವ, ಮತ್ತು ವಿಶ್ಲೇಷಣೆಯ ಸಾರವೇ ಇದರಲ್ಲಿದೆ. ಈ ವೇದಾಂತ ತತ್ತ್ವದಲ್ಲಿ ಕೆಲವು ವಿಶೇಷಗಳಿವೆ. ಮೊದಲನೆಯದಾಗಿ ಇದು ಯಾವ ವ್ಯಕ್ತಿಯ ಮೇಲೂ ನಿಂತಿಲ್ಲ. ಇದು ಯಾವ ದೇವದೂತನಿಂದಲೂ ಅಥವಾ ವ್ಯಕ್ತಿಯಿಂದಲೂ ಬಂದುದಲ್ಲ. ಇದು ಯಾವ ಒಂದು ವ್ಯಕ್ತಿಯ ಕೇಂದ್ರದ ಸುತ್ತಲೂ ಬೆಳೆಯುವುದಿಲ್ಲ. ಕಾಲಾನಂತರ ಬೌದ್ದ ಮತ್ತು ಇತರ ಧರ್ಮಗಳು ಆಯಾಯ ವ್ಯಕ್ತಿಯ ಆಧಾರದ ಮೇಲೆ ಬೆಳೆದವು. ಇವರಿಗೆಲ್ಲ ಮಹಮ್ಮದೀಯರಲ್ಲಿ ಮತ್ತು ಕ್ರೈಸ್ತರಲ್ಲಿ ಇರುವಂತೆ ಗೌರವಿಸುವುದಕ್ಕೆ ಒಂದು ವ್ಯಕ್ತಿ ಇದೆ. ಆದರೆ ವೇದಾಂತ ತತ್ತ್ವವು ಈ ಧರ್ಮಗಳಿಗೆಲ್ಲ ಹಿನ್ನೆಲೆಯಂತೆ ಇದೆ. ವೇದಾಂತಕ್ಕೆ ಜಗತ್ತಿನ ಯಾವ ಧರ್ಮದೊಂದಿಗೂ ವಿರೋಧವಿಲ್ಲ.

ವೇದಾಂತ ಒಂದು ಭಾವನೆಯನ್ನು ಒತ್ತಿ ಹೇಳುವುದು, ಆ ಭಾವನೆ ಪ್ರತಿಯೊಂದು ಧರ್ಮದಲ್ಲಿಯೂ ಇದೆ ಎಂದು ಸಾರುವುದು. ಅದೇ, ಮಾನವ ಪವಿತ್ರ ಎಂಬುದು. ನಮ್ಮ ಸುತ್ತಲೂ ಇರುವುದೆಲ್ಲ ಆ ಪವಿತ್ರಾತ್ಮನ ಭಾವನೆಯಿಂದ ಬಂದುದು ಎಂಬುದು. ಮಾನವ ಸ್ವಭಾವದಲ್ಲಿ ದೃಢವಾಗಿರುವುದೆಲ್ಲ, ಒಳ್ಳೆಯದೆಲ್ಲ, ಬಲಶಾಲಿಯಾಗಿರುವುದೆಲ್ಲ ಆ ಪಾವಿತ್ರ್ಯದಿಂದ ಬಂದುದು. ಇದು ಹಲವರಲ್ಲಿ ಸುಪ್ತವಾಗಿದ್ದರೂ, ಮಾನವರಲ್ಲಿ ಒಬ್ಬರಿಗೂ ಮತ್ತೊಬ್ಬರಿಗೂ ಯಾವ ವ್ಯತ್ಯಾಸವೂ ಇಲ್ಲ - ಮೂಲತಃ ಎಲ್ಲರೂ ಪವಿತ್ರಾತ್ಮರು. ನಮ್ಮ ಹಿನ್ನೆಲೆಯಾಗಿ ಒಂದು ಅಸೀಮ ಸಾಗರವಿರುವಂತೆ ಇದೆ. ನೀವು ನಾವುಗಳೆಲ್ಲ ಆ ಅನಂತ ಸಾಗರದಿಂದ ಏಳುವ ಅಲೆಗಳಂತೆ. ನಮ್ಮಲ್ಲಿ ಪ್ರತಿಯೊಬ್ಬರೂ ಆ ಪಾವಿತ್ರ್ಯವನ್ನು ಹೊರಗೆ ವ್ಯಕ್ತಗೊಳಿಸಲು ಸಾಧ್ಯವಾದಷ್ಟೂ ಪ್ರಯತ್ನಿಸುತ್ತಿರುವೆವು. ಪ್ರತಿಯೊಬ್ಬರಲ್ಲಿಯೂ ಆ ಸಚ್ಚಿದಾನಂದ ಸುಪ್ತವಾಗಿದೆ. ಅದೇ ನಮ್ಮ ಸ್ವಭಾವ, ನಮ್ಮ ಆಜನ್ಮಸಿದ್ಧ ಹಕ್ಕು. ನಮ್ಮಲ್ಲಿರುವ ವ್ಯತ್ಯಾಸಗಳಿಗೆ ನಾವು ಅದನ್ನು ವ್ಯಕ್ತಗೊಳಿಸುತ್ತಿರುವುದರಲ್ಲಿ ಇರುವ ತಾರತಮ್ಯವೇ ಕಾರಣ. ಮಾನವನನ್ನು ಅಳೆಯುವಾಗ ಅವನು ಎಷ್ಟನ್ನು ವ್ಯಕ್ತಗೊಳಿಸುತ್ತಿರುವನು ಎಂಬ ದೃಷ್ಟಿಯಿಂದಲ್ಲ, ಆದರೆ ಅವನಲ್ಲಿ ಯಾವ ಹಿನ್ನೆಲೆ ಇದೆ ಎಂಬ ದೃಷ್ಟಿಯಿಂದ ಅಳೆಯಬೇಕು. ಪ್ರತಿಯೊಬ್ಬ ಮಾನವನೂ ಪಾವಿತ್ರ್ಯದ ಪ್ರತಿನಿಧಿ. ಆದಕಾರಣ ಪ್ರತಿಯೊಬ್ಬ ಗುರುವೂ ಮತ್ತೊಬ್ಬರಿಗೆ ಸಹಾಯಕನಾಗಿರಬೇಕು, ಮಾನವನನ್ನು ದೂರುವುದರಿಂದ ಅಲ್ಲ, ಅವನಲ್ಲಿ ಅಂತರ್ಗತವಾಗಿರುವ ಪಾವಿತ್ರ್ಯವನ್ನು ವ್ಯಕ್ತಪಡಿಸಲು ಸಹಾಯ ಮಾಡುವುದರಿಂದ.

ಸಮಾಜದ ಪ್ರತಿಯೊಂದು ಕಾರ್ಯಕ್ಷೇತ್ರದಲ್ಲಿ ವ್ಯಕ್ತವಾಗುತ್ತಿರುವ ಶಕ್ತಿಯೆಲ್ಲವೂ ಒಳಗಿನಿಂದ ಹೊರಹೊಮ್ಮಿರುವುದಾಗಿದೆ ಎಂದು ವೇದಾಂತ ಸಾರುವುದು. ಆದಕಾರಣ ಇತರರು ಯಾವುದನ್ನು ಸ್ಫೂರ್ತಿ ಎನ್ನುವರೋ ಅದನ್ನು ವೇದಾಂತಿಯು ಮಾನವನಿಂದ ಬಾಹ್ಯಮುಖವಾಗಿ ಹೊರಟಿದ್ದು ಎಂದು ಹೇಳುವನು. ಅದೇ ಸಮಯದಲ್ಲಿ ಅವನು ಇತರ ಪಂಥದವರೊಂದಿಗೆ ವ್ಯಾಜ್ಯ ಮಾಡುವುದಿಲ್ಲ. ಮಾನವನಲ್ಲಿರುವ ಪಾವಿತ್ರ್ಯವನ್ನು ಯಾರು ಒಪ್ಪಿಕೊಳ್ಳುವುದಿಲ್ಲವೋ ಅಂತಹವರೊಂದಿಗೆ ವೇದಾಂತಕ್ಕೆ ಮನಸ್ತಾಪವಿಲ್ಲ. ತಿಳಿದೋ ತಿಳಿಯದೆಯೋ ಪ್ರತಿಯೊಬ್ಬರೂ ಈ ಪಾವಿತ್ರ್ಯವನ್ನು ವ್ಯಕ್ತಗೊಳಿಸಲು ಯತ್ನಿಸುತ್ತಿರುವರು.

ಮಾನವನು ಒಂದು ಸಣ್ಣ ಪೆಟ್ಟಿಗೆಯಲ್ಲಿ ಅಮುಕಿ ಇಟ್ಟ ದೊಡ್ಡದೊಂದು ಸ್ಪ್ರಿಂಗಿನಂತೆ ಇರುವನು. ಈ ಸ್ಪ್ರಿಂಗ್ ತನ್ನ ನೈಜಸ್ತಿತಿಗೆ ಬರಲು ಯತ್ನಿಸುತ್ತದೆ. ನಮಗೆ ಕಾಣುವ ಸಾಮಾಜಿಕ ಚಟುವಟಿಕೆಗಳೆಲ್ಲ, ಇದರ ಪರಿಣಾಮವಾಗಿದೆ. ನಮ್ಮ ಸುತ್ತಲೂ ಕಾಣುವ ಪೋಟಾಪೋಟಿ, ಹೋರಾಟ, ಪಾಪ, ಇವೆಲ್ಲ ಸ್ಪ್ರಿಂಗ್ ತನ್ನ ನೈಜಸ್ಥಿತಿಗೆ ತೆರಳುತ್ತಿರುವುದರ ಕಾರ್ಯವೂ ಅಲ್ಲ, ಅದಕ್ಕೆ ಕಾರಣವೂ ಅಲ್ಲ. ನಮ್ಮ ದೊಡ್ಡ ದಾರ್ಶನಿಕರೊಬ್ಬರು ಹೇಳುವಂತೆ, ವ್ಯವಸಾಯಮಾಡುವಾಗ ಕೆರೆ ಎಲ್ಲೋ ಮೇಲೆ ಇದೆ, ನೀರು ಗದ್ದೆಗೆ ಬರಬೇಕಾದರೆ ಮಧ್ಯೆ ತೂಬಿನ ಒಂದು ಆತಂಕವಿದೆ. ತೂಬನ್ನು. ತೆಗೆದೊಡನೆ ನೀರು ಸ್ವಭಾವತಃ ಹರಿದುಬರುವುದು. ದಾರಿಯಲ್ಲಿ ಮರಳು ಕೊಳೆ ಇದ್ದರೆ ನೀರು ಅದರ ಮೇಲೆ ಹರಿದುಹೋಗುವುದು. ಆದರೆ ಈ ಕೊಳೆ ಅಥವಾ ಕಶ್ಮಲ ಮಾನವ ತನ್ನ ಪಾವಿತ್ರ್ಯವನ್ನು ವ್ಯಕ್ತಗೊಳಿಸುವುದಕ್ಕೆ ಕಾರಣವೂ ಅಲ್ಲ, ಅದರ ಕಾರ್ಯವೂ ಅಲ್ಲ. ಅವೆಲ್ಲ ಮಾನವಸ್ವಭಾವದೊಡನಿರುವ ಪರಿಸರಗಳು, ಅವನ್ನೆಲ್ಲ ನಿವಾರಿಸಬಹುದು.

ಈ ಭಾವನೆ ಭರತಖಂಡದ ಮತ್ತು ಇತರ ದೇಶಗಳ ಧರ್ಮಗಳಲ್ಲೆಲ್ಲ ಇದೆ ಎನ್ನುತ್ತದೆ ವೇದಾಂತ. ಕೆಲವು ಕಡೆ ಇದನ್ನು ಒಂದು ಪುರಾಣದ ಮೂಲಕ ವಿವರಿಸುವರು, ಮತ್ತೆ ಕೆಲವು ಕಡೆ ಇದನ್ನು ಸಂಕೇತಗಳ ಮೂಲಕ ವಿವರಿಸುವರು. ಧಾರ್ಮಿಕ ಕ್ಷೇತ್ರದಲ್ಲಿರುವ ಪ್ರತಿಯೊಂದು ಸ್ಫೂರ್ತಿಯಲ್ಲಿಯೂ, ಪ್ರತಿಯೊಬ್ಬ ಮಹಾವ್ಯಕ್ತಿಯ ಹಿಂದೆಯೂ ಮಾನವನಲ್ಲಿರುವ ಅನಂತ ಏಕತೆಯೆ ವ್ಯಕ್ತವಾಗುತ್ತಿದೆ ಎಂದು ವೇದಾಂತ ಸಾರುವುದು. ನೀತಿ, ಧರ್ಮ, ಇತರರಿಗೆ ಒಳ್ಳೆಯದನ್ನು ಮಾಡಬೇಕು ಎನ್ನುವುದರ ಹಿಂದೆಲ್ಲ ಈ ಏಕತೆಯ ಅಭಿವ್ಯಕ್ತಿ ಇದೆ. ಪ್ರತಿಯೊಬ್ಬರೂ ತಾವು ವಿಶ್ವದಲ್ಲಿ ಒಂದು ಎಂದು ಅನುಭವಿಸುವ ಕ್ಷಣಗಳು ಕೆಲವು ಇರುವುವು. ಅವನಿಗೆ ಅದು ಗೊತ್ತಿದೆಯೋ ಇಲ್ಲವೊ ಅದನ್ನು ವ್ಯಕ್ತಗೊಳಿಸಲು ಮಾನವನು ಧಾವಿಸುವನು. ಈ ಏಕತೆಯ ಹೊರಸೂಸುವಿಕೆಯನ್ನೇ ನಾವು ಪ್ರೀತಿ, ಸಹಾನುಭೂತಿ ಮುಂತಾದ್ದಾಗಿ ಹೇಳುವುದು. ಇದೇ ನಮ್ಮ ನೀತಿಗೆ ಮತ್ತು ಧರ್ಮಕ್ಕೆ ಮೂಲ, ತತ್ತ್ವಮಸಿ ಎಂಬ ವೇದಾಂತದ ಅತಿಮುಖ್ಯ ವಾಕ್ಯದಲ್ಲಿ ಇದನ್ನು ಸಂಗ್ರಹಿಸಿ ಹೇಳಿರುವರು.

- ಪ್ರತಿಯೊಬ್ಬರಿಗೂ ಇದನ್ನು ವಿವರಿಸುವರು: ನೀನು ವಿಶ್ವಾತ್ಮನೊಂದಿಗೆ ಒಂದು, ಇರುವ ಪ್ರತಿಯೊಬ್ಬರೂ ನಿನ್ನ ಆತ್ಮವೆ, ಇರುವ ಪ್ರತಿಯೊಂದು ದೇಹವೂ ನಿನ್ನ ದೇಹ, ಇತರರನ್ನು ಹಿಂಸಿಸಿದರೆ ನಿನ್ನನ್ನು ನೀನೆ ಹಿಂಸಿಸಿಕೊಳ್ಳುವೆ, ಇತರರನ್ನು ಪ್ರೀತಿಸಿದರೆ, ನಿನ್ನನ್ನು ನೀನೇ ಪ್ರೀತಿಸುವೆ. ದ್ವೇಷಭಾವನೆಯೊಂದನ್ನು ಹೊರಗೆಡವಿದೊಡನೆ ಅದು ಇತರರನ್ನು ಹಾನಿಮಾಡಿದರೂ ನಿನ್ನನ್ನೂ ಹಾನಿಗೆ ಗುರಿಪಡಿಸುವುದು. ನನ್ನಿಂದ ಪ್ರೀತಿ ಹೊರಬಂದರೆ ಪುನಃ ಅದೇ ಪ್ರೀತಿ ನನ್ನಲ್ಲಿಗೆ ಬಂದೇ ಬರುವುದು. ಏಕೆಂದರೆ ನಾನೇ ವಿಶ್ವವಾಗಿರುವವನು. ಈ ವಿಶ್ವವೇ ನನ್ನ ದೇಹ. ನಾನು ಅನಂತ, ಆದರೆ ನನಗೆ ಈಗ ಅದು ಗೊತ್ತಿಲ್ಲದೆ ಇರುವುದು. ಆದರೆ ನಾನು ಅನಂತದ ಅರಿವನ್ನು ಪಡೆಯಲು ಯತ್ನಿಸುತ್ತಿರುವೆನು. ಈ ಅನಂತತೆಯ ಭಾವನೆ ಪೂರ್ಣ ವ್ಯಕ್ತವಾದೊಡನೆ ನಾವು ಪರಿಪೂರ್ಣತೆಯನ್ನು ಪಡೆದಂತೆ.

ವೇದಾಂತದ ಮತ್ತೊಂದು ವೈಶಿಷ್ಟ್ಯವೇ ಧರ್ಮದಲ್ಲಿ ಅನಂತ ವೈಶಿಷ್ಟ್ಯಕ್ಕೆ ಅವಕಾಶವಿರಬೇಕು ಎನ್ನುವುದು. ಗುರಿ ಎಲ್ಲರಿಗೂ ಒಂದೇ ಎಂದು ಎಲ್ಲರೂ ಒಂದೇ ಅಭಿಪ್ರಾಯವನ್ನು ಒಪ್ಪಿಕೊಳ್ಳುವಂತೆ ಮಾಡುವುದಲ್ಲ. ವೇದಾಂತಿ ಕಾವ್ಯಮಯವಾದ ಭಾಷೆಯಲ್ಲಿ ಹೀಗೆ ಹೇಳುವನು: ''ಬೇರೆ ಬೇರೆ ಬೆಟ್ಟಗಳಲ್ಲಿ ಹುಟ್ಟಿದ ಹಲವು ನದಿಗಳು ನೇರವಾಗಿಯೋ ಡೊಂಕು ಡೊಂಕಾಗಿಯೋ ಹರಿದು ಕೊನೆಗೆ ಸಾಗರವನ್ನು ಸೇರುವಂತೆ ಹಲವು ಮತಗಳು ಮತ್ತು ಪಂಥಗಳು ಭಿನ್ನ ಭಿನ್ನ ಭಾವಗಳಿಂದ ಉದಯಿಸಿ ನೇರವಾಗಿಯೋ ಡೊಂಕು ಡೊಂಕಾಗಿಯೋ, ಹರಿದು ಕೊನೆಗೆ ದೇವರನ್ನು ಸೇರುತ್ತವೆ.”

ಈ ಅತಿ ಪುರಾತನ ತತ್ತ್ವವು ಪ್ರಪಂಚದ ಪ್ರಪ್ರಥಮ ಮಿಷನರಿ ಧರ್ಮವಾದ ಬೌದ್ಧ ಧರ್ಮದ ಮೇಲೆ ತನ್ನ ಪ್ರಭಾವವನ್ನು ನೇರವಾಗಿ ಬೀರಿದೆ. ಪರೋಕ್ಷವಾಗಿ ಅಲೆಕ್ಸಾಂಡ್ರಿಯದವರು, ನೋಗ್ಟಿಕರು ಮತ್ತು ಮಧ್ಯಕಾಲದ ಯೂರೋಪಿನ ತಾತ್ತ್ವಿಕರು ಇವರ ಮೂಲಕ ಕೈಸ್ತರ ಮೇಲೆ ತನ್ನ ಪ್ರಭಾವವನ್ನು ಬೀರಿದೆ. ಅನಂತರ ಜರ್ಮನಿಯ ಭಾವನೆಯ ಮೇಲೆ ತನ್ನ ಪ್ರಭಾವವನ್ನು ಬೀರಿ ತತ್ತ್ವ ಪ್ರಪಂಚದಲ್ಲಿ ಮತ್ತು ಧರ್ಮ ಪ್ರಪಂಚದಲ್ಲಿ ಒಂದು ದೊಡ್ಡ ಕ್ರಾಂತಿಯನ್ನು ಎಬ್ಬಿಸಿದೆ. ಆದರೆ ಇಷ್ಟೊಂದು ಪ್ರಭಾವವನ್ನು ಯಾರಿಗೂ ಗೊತ್ತಾಗದಂತೆ ಬೀರಿದೆ. ರಾತ್ರಿಯಲ್ಲಿ ಸದ್ದು ಗದ್ದಲಗಳಿಲ್ಲದೆ ಬೀಳುವ ಹಿಮಮಣಿಯು ಇಡೀ ಸಸ್ಯಪ್ರಪಂಚಕ್ಕೆ ಜೀವದಾನ ಮಾಡುವಂತೆ, ನಿಧಾನವಾಗಿ ಯಾರಿಗೂ ಗೊತ್ತಾಗದಂತೆ ಮಾನವಕೋಟಿಯ ಹಿತಕ್ಕಾಗಿ ಈ ಪವಿತ್ರವಾದ ವೇದಾಂತವು ಜಗತ್ತಿನಲ್ಲೆಲ್ಲಾ ಹಬ್ಬಿದೆ. ಈ ಧರ್ಮದ ಬೋಧನೆಗಾಗಿ ಯಾವ ಸೈನ್ಯವನ್ನೂ ಉಪಯೋಗಿಸಿಲ್ಲ. ಪ್ರಪಂಚದಲ್ಲೆಲ್ಲಾ ಅಗ್ರಗಣ್ಯ ಮಿಷನರಿ ಧರ್ಮವಾದ ಬೌದ್ಧ ಧರ್ಮಕ್ಕೆ ಸೇರಿದ ಅಶೋಕನ ಶಾಸನಗಳಲ್ಲಿ, ಆಗಿನ ಕಾಲದಲ್ಲಿ ಬೌದ್ಧ ಭಿಕ್ಷುಗಳನ್ನು ಅಲೆಕ್ಸಾಂಡ್ರಿಯ, ಆಂಟಿಯೋಕ್, ಪರ್ಷಿಯಾ, ಚೈನಾ ಮುಂತಾದ ನಾಗರಿಕ ದೇಶಕ್ಕೆ ಕಳುಹಿಸುತ್ತಿದ್ದರೆಂದು ಓದುತ್ತೇವೆ. ಕ್ರಿಸ್ತನಿಗೆ ಮುನ್ನೂರು ವರ್ಷಗಳ ಮುಂಚೆ ಇತರ ಧರ್ಮಗಳನ್ನು ದೂರದಂತೆ ಅಲ್ಲಿ ಕಟ್ಟಪ್ಪಣೆ ಆಗಿತ್ತು: “ಧರ್ಮಗಳ ಮೂಲವೆಲ್ಲ ಒಂದೇ, ಅವು ಯಾವ ದೇಶಕಾಲಗಳಿಗೆ ಸೇರಿದ್ದರೂ ಚಿಂತೆಯಿಲ್ಲ. ನಿಮ್ಮ ಕೈಲಾದ ಮಟ್ಟಿಗೆ ಅವರಿಗೆ ಬೋಧಿಸಿ, ಆದರೆ ಅವರನ್ನು ಹಿಂಸಿಸಲು ಯತ್ನಿಸಬೇಡಿ.”

ಭರತಖಂಡದಲ್ಲಿ ಹಿಂದೂಗಳು ಧರ್ಮದ ಹೆಸರಿನಲ್ಲಿ ಮತ್ತೊಬ್ಬರನ್ನು ಹಿಂಸಿಸಲಿಲ್ಲ. ಇತರ ಧರ್ಮಗಳಿಗೆಲ್ಲ ಅದ್ಭುತವಾದ ಗೌರವವನ್ನು ತೋರಿದರು. ತಮ್ಮ ದೇಶದಿಂದ ಓಡಿಸಲ್ಪಟ್ಟ ಹಿಬ್ರೂಗಳಿಗೆ ಭರತಖಂಡದಲ್ಲಿ ರಕ್ಷಣೆಯನ್ನು ಕೊಟ್ಟರು. ಇದರ ಪರಿಣಾಮವಾಗಿ ಈಗ ಮಲಬಾರಿನಲ್ಲಿ ಯೆಹೂದ್ಯರು ಇರುವರು. ಪಾರ್ಸಿಯನ್ನರು ಇನ್ನೇನು ನಿರ್ನಾಮವಾಗುವ ಸ್ಥಿತಿಯಲ್ಲಿದ್ದಾಗ ಅವರಿಗೆ ರಕ್ಷಣೆಯನ್ನು ಕೊಟ್ಟರು. ಈಗಲೂ ಅವರು ನಮ್ಮ ಪ್ರೀತಿಗೆ ಪಾತ್ರರಾಗಿ ನಮ್ಮಲ್ಲಿ ಒಬ್ಬರಾಗಿ ಬೊಂಬಾಯಿಯಲ್ಲಿ ಇರುವರು. ಏಸುವಿನ ಸೆಂಟ್ ಥಾಮಸ್‌ನೊಂದಿಗೆ ಬಂದವರು ತಾವೆಂದು ಹೇಳಿಕೊಳ್ಳುವ ಕ್ರೈಸ್ತರು ಇರುವರು. ಅವರಿಗೆ ಇಂಡಿಯಾದೇಶದಲ್ಲಿ ನೆಲಸಿ ತಮ್ಮ ಧರ್ಮವನ್ನು ಅನುಸರಿಸಲು ಅವಕಾಶ ಕೊಟ್ಟರು. ಈಗಲೂ ಕೂಡ ಅವರ ಒಂದು ಶಾಖೆ ಭರತಖಂಡದಲ್ಲಿದೆ. ಈ ಅನ್ಯಮತಸಹಿಷ್ಣುತೆಯ ಭಾವನೆ ನಾಶವಾಗಿಲ್ಲ, ಅದೆಂದಿಗೂ ನಾಶವಾಗುವಂತೆಯೂ ಇಲ್ಲ.

ವೇದಾಂತವು ನಮಗೆ ಬೋಧಿಸಬಲ್ಲ ದೊಡ್ಡ ನೀತಿಯೆ ಇದು: ತಿಳಿದೊ ತಿಳಿಯದೆಯೋ ನಾವೆಲ್ಲ ಒಂದೇ ಗುರಿಯನ್ನು ಸೇರಲೇಬೇಕಾಗಿರುವಾಗ ನಾವು ತಾಳ್ಮೆಗೆಡಬೇಕಾಗಿಲ್ಲ. ಒಬ್ಬನು ಮತ್ತೊಬ್ಬನಿಗಿಂತ ನಿಧಾನವಾಗಿ ಮುಂದುವರಿದರೆ ನಾವು ತಾಳ್ಮೆಗೆಡಬೇಕಾಗಿಲ್ಲ. ಅವರನ್ನು ನಾವು ದೂರಬೇಕಾಗಿಲ್ಲ, ದ್ವೇಷಿಸಬೇಕಾಗಿಲ್ಲ. ನಮ್ಮಲ್ಲಿರುವ ಅಜ್ಞಾನದ ತೆರೆ ಜಾರಿದರೆ, ಹೃದಯ ಪರಿಶುದ್ಧವಾದಾಗ, ಆ ಪುಣ್ಯಪ್ರಭಾವ ಪ್ರತಿಯೊಬ್ಬರಲ್ಲಿಯೂ ವ್ಯಕ್ತವಾಗುವುದನ್ನು ನೋಡುತ್ತೇವೆ. ಆಗ ಮಾತ್ರ ನಾವು ಸಹೋದರರು ಎಂದು ಹೇಳಿಕೊಳ್ಳಬಹುದು.

ಒಬ್ಬನು ಇಂತಹ ಅನುಭವಶಿಖರಕ್ಕೆ ಏರಿ, ಸ್ತ್ರೀಪುರುಷರನ್ನು ನೋಡದೆ, ಲಿಂಗ, ಜಾತಿ, ಬಣ್ಣ, ಕುಲ ಮುಂತಾದ ಯಾವುದನ್ನೂ ನೋಡದೆ, ಇವನ್ನೆಲ್ಲ ಮೀರಿ ಹೋಗಿ ಪ್ರತಿಯೊಬ್ಬರಲ್ಲಿಯೂ ಮಾನವನ ನಿಜವಾದ ಸ್ವಭಾವವಾದ ಪಾವಿತ್ರ್ಯವನ್ನು ಯಾವಾಗ ನೋಡುವನೋ ಆಗ ಮಾತ್ರ ಅವನು ವಿಶ್ವ ಸಹೋದರನಾಗುವನು. ಅವನು ಮಾತ್ರ ವೇದಾಂತಿ..

ಅನುಷ್ಠಾನ ವೇದಾಂತದ ಫಲಗಳು ಇವು.

