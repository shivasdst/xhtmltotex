
\chapter[ಪರಾಭಕ್ತಿ]{ಪರಾಭಕ್ತಿ\protect\footnote{\engfoot{C.W, Vol. VI, P. 70}}}

\begin{center}
(೧೯೦೦ರ ಏಪ್ರಿಲ್ ೧೨ ರಂದು ಸ್ಯಾನ್‌ಫ್ರಾನ್ಸಿಸ್ಕೋ ಪ್ರದೇಶದಲ್ಲಿ ನೀಡಿದ ಪ್ರವಚನ)
\end{center}

ಪ್ರೇಮವನ್ನು ತ್ರಿಕೋನಕ್ಕೆ ಹೋಲಿಸಬಹುದು. ಅದರಲ್ಲಿ ಮೊದಲನೆಯ ಕೋನವೆ ಪ್ರೀತಿ. ಏನನ್ನೂ ಕೇಳುವುದಿಲ್ಲ. ಪ್ರೇಮ ಭಿಕ್ಷುಕನಲ್ಲ. ಭಿಕ್ಷುಕನ ಪ್ರೀತಿ ಪ್ರೀತಿಯೇ ಅಲ್ಲ. ಏನನ್ನೂ ಕೇಳದೆ ಇರುವುದೇ ಪ್ರೀತಿಯ ಮೊದಲನೆಯ ಚಿಹ್ನೆ. ಅದು ತನ್ನಲ್ಲಿರುವುದನ್ನೆಲ್ಲ ಕೊಡುವುದು. ಪ್ರೀತಿಯ ಮೂಲಕ ಪೂಜಿಸುವುದೇ ನಿಜವಾದ ಆಧ್ಯಾತ್ಮಿಕ ಪೂಜೆ. ದೇವರು ದಯಾಮಯನೆ ಮುಂತಾದ ಪ್ರಶ್ನೆಗಳನ್ನೆಲ್ಲ ಇನ್ನು ಮೇಲೆ ಹಾಕುವುದಿಲ್ಲ. ಅವನು ದೇವರು. ಅವನು ನನ್ನ ಪ್ರೇಮಮೂರ್ತಿ. ದೇವರು ಸರ್ವಶಕ್ತ, ಸರ್ವೆಶ್ವರ, ಅವನೇನು ಸಾಂತನೊ ಅಥವಾ ಅನಂತನೊ ಎಂಬುದನ್ನು ಇನ್ನು ಮೇಲೆ ಪ್ರಶ್ನಿಸುವುದೇ ಇಲ್ಲ. ಅವನು ಒಳ್ಳೆಯದನ್ನು ಕೊಟ್ಟರೆ ಸರಿ, ಇಲ್ಲದೆ ಅವನು ಕೆಟ್ಟದ್ದನ್ನು ಕೊಟ್ಟರೆ ತಾನೆ ಏನು? ಅವನು ಅನಂತ ಪ್ರೇಮಸ್ವರೂಪ ಎಂಬುದಲ್ಲದೆ ಬೇರೆ ಗುಣಗಳೆಲ್ಲ ಮಾಯವಾಗುವುವು.

ಹಿಂದೆ ಭರತಖಂಡದಲ್ಲಿ ಒಬ್ಬ ಚಕ್ರವರ್ತಿ ಇದ್ದ. ಅವನು ಒಂದು ಸಲ ಬೇಟೆಗೆ ಹೋಗಿದ್ದಾಗ ಒಬ್ಬ ಮಹಾತ್ಮನನ್ನು ಕಂಡನು. ಚಕ್ರವರ್ತಿಗೆ ಇವನ ಮೇಲೆ ತುಂಬಾ ವಿಶ್ವಾಸ ಮೂಡಿತು. ಚಕ್ರವರ್ತಿ ಮಹಾತ್ಮನನ್ನು ತನ್ನ ರಾಜಧಾನಿಗೆ ಬಂದು ತನ್ನಿಂದ ಏನನ್ನಾದರೂ ಬಹುಮಾನವಾಗಿ ಸ್ವೀಕರಿಸಬೇಕೆಂದು ಕೇಳಿಕೊಂಡನು. ಮೊದಲು\break ಮಹಾತ್ಮನು ಅದಕ್ಕೆ ಒಪ್ಪಲಿಲ್ಲ. ಚಕ್ರವರ್ತಿ ಒತ್ತಾಯಮಾಡಿದುದರಿಂದ ಮಹಾತ್ಮನು ವಿಧಿಯಿಲ್ಲದೆ ಒಪ್ಪಬೇಕಾಯಿತು. ಮಹಾತ್ಮನು ರಾಜಧಾನಿಗೆ ಬಂದಾಗ ಚಾರರು ಚಕ್ರವರ್ತಿಗೆ ಹೋಗಿ ಹೇಳಿದರು. ಚಕ್ರವರ್ತಿ, “ಒಂದು ನಿಮಿಷ ನಿಲ್ಲಿ, ನಾನು ಪ್ರಾರ್ಥನೆಯನ್ನು ಪೂರೈಸಿಕೊಂಡು ಬರುವೆ'' ಎಂದು ಹೇಳಿ ಕಳುಹಿಸಿದನು. ಚಕ್ರವರ್ತಿ ಹೀಗೆ ಪ್ರಾರ್ಥಿಸತೊಡಗಿದನು: “ದೇವರೇ, ನನಗೆ ಇನ್ನೂ ಹೆಚ್ಚು ಐಶ್ವರ್ಯವನ್ನು ಕೊಡು, ಇನ್ನೂ ಹೆಚ್ಚು ರಾಜ್ಯವನ್ನು ಕೊಡು, ಆರೋಗ್ಯವನ್ನು ಕೊಡು, ಮಕ್ಕಳನ್ನು ಕೊಡು.” ಮಹಾತ್ಮನು ಎದ್ದು ನಿಂತು ಕೋಣೆಯಿಂದ ಹೊರಡುವುದಕ್ಕೆ ಅಣಿಯಾದನು. ಆಗ ಚಕ್ರವರ್ತಿ, “ನೀವು ನನ್ನಿಂದ ಯಾವ ಬಹುಮಾನವನ್ನೂ ಸ್ವೀಕರಿಸಲಿಲ್ಲವಲ್ಲ!'' ಎಂದನು. ಅದಕ್ಕೆ ಮಹಾತ್ಮನು, “ನಾನು ಭಿಕ್ಷುಕರಿಂದ ಬೇಡುವುದಿಲ್ಲ. ನೀನೇ ಇದುವರೆಗೆ ಹೆಚ್ಚು ರಾಜ್ಯಕ್ಕಾಗಿ, ಹಣಕ್ಕಾಗಿ, ಅದಕ್ಕಾಗಿ ಇದಕ್ಕಾಗಿ ಬೇಡುತ್ತಿರುವೆ. ನೀನು ನನಗೆ ಏನನ್ನು ಕೊಡಬಲ್ಲೆ? ಮೊದಲು ನಿನ್ನ ಬಯಕೆಗಳನ್ನು ಈಡೇರಿಸಿಕೊ” ಎಂದನು.

ಪ್ರೀತಿ ಎಂದಿಗೂ ಕೇಳುವುದಿಲ್ಲ. ಅದು ಯಾವಾಗಲೂ ಕೊಡುವುದು. ಯುವಕ ತನ್ನ ಪ್ರಿಯತಮೆಯ ಹತ್ತಿರ ಹೋದಾಗ, ಅವರಿಬ್ಬರಲ್ಲಿ ಯಾವ ವ್ಯಾಪಾರದ ಸಂಬಂಧವೂ ಇರುವುದಿಲ್ಲ. ಅವರಲ್ಲಿ ಪ್ರೇಮದ ಸಂಬಂಧ ಮಾತ್ರ ಇರುವುದು. ಪ್ರೀತಿ ಎಂದಿಗೂ ಬೇಡುವುದಿಲ್ಲ. ಇದರಂತೆಯೇ ನಿಜವಾದ ಆಧ್ಯಾತ್ಮಿಕ ಪೂಜೆಯ ಪ್ರಾರಂಭದಲ್ಲಿ ಯಾವ ಬೇಡುವುದೂ ಇರಲಾರದೆಂದು ನಾನು ಭಾವಿಸುತ್ತೇನೆ. ದೇವರೆ, ಇದನ್ನು ಕೊಡು, ಅದನ್ನು ಕೊಡು ಎಂಬುದನ್ನೆಲ್ಲ ನಾವು ಪೂರೈಸಿರುವೆವು. ಆಗ ಮಾತ್ರ ನಿಜವಾದ ಧರ್ಮ ಪ್ರಾರಂಭವಾಗುವುದು.

ಪ್ರೇಮದ ತ್ರಿಕೋನದಲ್ಲಿ ಎರಡನೆಯ ಕೋನವೇ ಪ್ರೇಮದಲ್ಲಿ ಭಯವಿಲ್ಲ\break ಎಂಬುದು. ನೀನು ನನ್ನನ್ನು ಚೂರುಚೂರಾಗಿ ಕತ್ತರಿಸಬಹುದು. ಆದರೂ ನಾನು ನಿನ್ನನ್ನು ಪ್ರೀತಿಸುತ್ತೇನೆ. ದುರ್ಬಲ ಹೆಂಗಸರಾದ ನಿಮ್ಮಲ್ಲಿ ಒಬ್ಬರ ಮಗುವನ್ನು ಹುಲಿಯೊಂದು ಹಿಡಿದುಕೊಂಡಿದೆ ಎಂದು ಭಾವಿಸೋಣ. ಆಗ ನೀವು ಎಲ್ಲಿ ಇರುತ್ತೀರಿ ಎಂಬುದು ಗೊತ್ತು. ನೀವು ಹುಲಿಯನ್ನು ಎದುರಿಸುತ್ತೀರಿ. ಮತ್ತೊಂದು ಸಲ ನಾಯಿಯೊಂದು ಬೀದಿಯಲ್ಲಿ ಬೊಗಳುತ್ತಾ ಬಂದರೂ ನೀವು ಓಡಿ ಹೋಗುವಿರಿ. ಆದರೆ ಮಗು ಇದ್ದಾಗ ಒಂದು ಹುಲಿಯೇ ಬಂದು ಅದನ್ನು ಹಿಡಿದುಕೊಂಡರೂ, ನೀವು ಅದರ ಬಾಯಿಂದ ಮಗುವನ್ನು ಕೀಳಲು ಯತ್ನಿಸುವಿರಿ. ಪ್ರೀತಿಗೆ ಭಯ ಇಲ್ಲ. ಅದು ಪಾಪವನ್ನೆಲ್ಲ ಗೆಲ್ಲುವುದು. ದೇವರ ಮೇಲೆ ಇರುವ ಅಂಜಿಕೆ ಧರ್ಮದ ಪ್ರಾರಂಭ ಮಾತ್ರ. ದೇವರ ಮೇಲೆ ಇರುವ ಪ್ರೀತಿ ಧರ್ಮದ ಪರಾಕಾಷ್ಠೆ. ಎಲ್ಲಾ ಅಂಜಿಕೆಗಳೂ ಅಲ್ಲಿ ನಾಶವಾಗಿ ಹೋಗಿವೆ.

ಪ್ರೇಮದ ಮೂರನೆಯ ಕೋನವೇ, ಪ್ರೇಮವೇ ತನ್ನ ಗುರಿ ಎಂಬುದು. ಅದನ್ನು ಮತ್ತಾವುದನ್ನೂ ಪಡೆದುಕೊಳ್ಳುವುದಕ್ಕೆ ಉಪಯೋಗಿಸಿಕೊಳ್ಳಲಾರೆವು. ನಾನು ನಿನ್ನನ್ನು ಇದಕ್ಕಾಗಿ ಪ್ರೀತಿಸುತ್ತೇನೆ ಎಂದು ಹೇಳಿದರೆ ಅವನು ಪ್ರೀತಿಸುವುದಿಲ್ಲ. ಪ್ರೇಮ ಮತ್ಯಾವುದನ್ನೂ ಪಡೆಯುವುದಕ್ಕೆ ಒಂದು ಸಾಧನವಾಗಲಾರದು. ಪ್ರೇಮವೇ ಅದರ ಗುರಿಯಾಗಬೇಕು. ಪ್ರೇಮದ ಉದ್ದೇಶವೇನು? ಗುರಿಯೇನು? ದೇವರನ್ನು ಪ್ರೀತಿಸುವುದು, ಅಷ್ಟೆ. ಒಬ್ಬ ದೇವರನ್ನು ಏತಕ್ಕೆ ಪ್ರೀತಿಸಬೇಕು? ಅಲ್ಲಿ ಯಾವ ಉದ್ದೇಶವೂ ಇಲ್ಲ. ಏಕೆಂದರೆ ಪ್ರೀತಿಯೊಂದು ಸಾಧನವಲ್ಲ. ಒಬ್ಬನು ಪ್ರೀತಿಸಬಲ್ಲವನಾದರೆ, ಅದೇ ಮುಕ್ತಿ, ಅದೇ ಪರಿಪೂರ್ಣತೆ, ಅದೇ ಚರಮಗುರಿ. ಪ್ರೀತಿಗಿಂತ ಶ್ರೇಷ್ಠವಾಗಿರುವುದು ನಿಮಗೆ ಮತ್ತೇನು ಸಿಕ್ಕಬಲ್ಲದು?

ಸಾಧಾರಣ ಜನರು ಪ್ರೀತಿ ಎಂದರೆ ಏನು ಭಾವಿಸಿಕೊಂಡಿರುವರೋ ಆ ವಿಷಯವನ್ನು ಕುರಿತು ನಾನು ಇಲ್ಲಿ ಮಾತನಾಡುತ್ತಿಲ್ಲ. ಅಲ್ಪವಾದ ದುರ್ಬಲವಾದ ಪ್ರೀತಿ ಮನೋಹರವಾಗಿರುವುದು. ಗಂಡಸು ಒಬ್ಬ ಹೆಂಗಸನ್ನು ಪ್ರೀತಿಸುತ್ತಾನೆ. ಹೆಂಗಸು ಗಂಡಸಿಗಾಗಿ ತನ್ನ ಪ್ರಾಣವನ್ನು ಕೊಡಲು ಸಿದ್ದಳಾಗಿರುವಳು. ಐದು ನಿಮಿಷಗಳಲ್ಲಿ, ಜಾನ್ ಜೇನಳನ್ನು ಒದೆಯುವುದು, ಜೇನಳು ಜಾನನ್ನು ಒದೆಯುವುದು ಸಂಭವನೀಯ. ಇದು ಕೇವಲ ಭೌತಿಕವಾದುದು, ಇಲ್ಲಿ ಯಾವ ಪ್ರೇಮವೂ ಇಲ್ಲ. ಜಾನ್‌ಗೆ ಜೇನಳನ್ನು ಹೃತೂರ್ವಕವಾಗಿ ಪ್ರೀತಿಸಲು ಸಾಧ್ಯವಾದರೆ, ಆ ಕ್ಷಣವೇ ಅವನು ಪೂರ್ಣಾತ್ಮನಾಗಬಲ್ಲ. ಅವನ ನಿಜವಾದ ಸ್ವಭಾವವೇ ಪ್ರೀತಿ. ಅವನು ತನ್ನಲ್ಲಿ ಪೂರ್ಣನಾಗಿರುವನು. ಸುಮ್ಮನೇ ಜೇನಳನ್ನು ಪ್ರೀತಿಸುತ್ತಿದ್ದರೆ ಅವನಿಗೆ ಯೋಗಶಕ್ತಿಯೆಲ್ಲ ಬರುವುದು. ಅವನಿಗೆ ಯಾವ ಧರ್ಮವಾಗಲಿ ಮನಶ್ಶಾಸ್ತ್ರವಾಗಲಿ ಶ್ರುತಿಸ್ಮೃತಿಗಳಾಗಲಿ ತಿಳಿದಿರಬೇಕಾಗಿಲ್ಲ. ಸ್ತ್ರೀಪುರುಷರು ನಿಜವಾಗಿ ಒಬ್ಬರು ಇನ್ನೊಬ್ಬರನ್ನು ಪ್ರೀತಿಸಿದರೆ, ಯೋಗಿಗಳಿಗೆ ಇರುವ ಶಕ್ತಿಯನ್ನೆಲ್ಲ ಅವರು ಗಳಿಸಬಹುದೆಂದು ನಾನು ಭಾವಿಸುತ್ತೇನೆ. ಏಕೆಂದರೆ ಪ್ರೇಮವೇ ಸಾಕ್ಷಾತ್ ಭಗವಂತ, ದೇವರು ಸರ್ವಾಂತರ್ಯಾಮಿ. ಆದಕಾರಣವೇ ನಿಮಗೆ ಅದು ಗೊತ್ತಿರಲಿ ಬಿಡಲಿ, ನಿಮ್ಮಲ್ಲಿ ಆ ಪ್ರೀತಿ ಇದೆ.

ನಾನು ಒಂದು ಸಂಜೆ, ಹುಡುಗನೊಬ್ಬ ತನ್ನ ಹುಡುಗಿಗಾಗಿ ಕಾಯುತ್ತಿರುವುದನ್ನು ನೋಡಿದೆ. ಈ ಹುಡುಗನನ್ನು ಪರೀಕ್ಷಿಸುವುದು ಒಂದು ಒಳ್ಳೆಯ ಪ್ರಯೋಗ ಎಂದು ನಾನು ಭಾವಿಸಿದೆ. ಅವನು ತನ್ನ ಪ್ರೇಮದ ತೀವ್ರತೆಯಿಂದ ದೂರದಲ್ಲಿ ಏನಾಗುವುದೋ ಅದನ್ನು ನೋಡುವ ಮತ್ತು ಕೇಳುವ ಶಕ್ತಿಯನ್ನು ಪಡೆದುಕೊಂಡನು. ಅವನು ಅರವತ್ತು ಎಪ್ಪತ್ತು ವೇಳೆ ಯಾವ ತಪ್ಪನ್ನೂ ಮಾಡಲಿಲ್ಲ. ಆ ಹುಡುಗಿಯಾದರೂ ಇನ್ನೂರು ಮೈಲಿಗಳ ಆಚೆಯಲ್ಲಿ ಇದ್ದಳು. ಅವನು, ಅವಳು ಈಗ ಹೀಗೆ ಬಟ್ಟೆ ಹಾಕಿಕೊಂಡಿರುವಳು ಅಥವಾ ಅವಳು ಈಗ ಅಲ್ಲಿಗೆ ಹೋಗುತ್ತಿರುವಳು ಎಂದು ಹೇಳುತ್ತಿದ್ದ. ನಾನು ನನ್ನ ಕಣ್ಣುಗಳಿಂದಲೇ ಇದನ್ನು ನೋಡಿರುವೆನು.

ಈಗ ಬರುವ ಪ್ರಶ್ನೆಯೇ ಇದು. ನಿಮ್ಮ ಗಂಡ ದೇವರಲ್ಲವೇ? ಮಗು ದೇವರಲ್ಲವೆ? ನೀವು ನಿಮ್ಮ ಹೆಂಡತಿಯನ್ನು ಪ್ರೀತಿಸಬಲ್ಲಿರಾದರೆ, ಈ ಪ್ರಪಂಚದ ಧರ್ಮವೆಲ್ಲ ನಿಮ್ಮದಾಗುವುದು. ಧರ್ಮದ ಮತ್ತು ಯೋಗದ ಸಾರವೇ ನಿಮಗೆ ದೊರಕುವುದು. ಆದರೆ ನೀವು ನಿಜವಾಗಿ ಪ್ರೀತಿಸಬಲ್ಲಿರಾ? ಅದೇ ಪ್ರಶ್ನೆ: ನೀವು “ಓ, ನಾನು ನಿನ್ನನ್ನು ಪ್ರೀತಿಸುತ್ತೇನೆ. ಮೇರಿ, ನಿನಗಾಗಿ ನಾನು ಸಾಯುತ್ತಿರುವೆ" ಎನ್ನುತ್ತೀರಿ. ಆದರೆ ನೀವು, ಮೇರಿ ಮತ್ತೊಬ್ಬ ಪುರುಷನನ್ನು ಚುಂಬಿಸುತ್ತಿರುವುದನ್ನು ಕಂಡರೆ, ಅವನ ಕೊರಳನ್ನು ಕತ್ತರಿಸಲೆತ್ನಿಸುವಿರಿ. ಮೇರಿ, ತನ್ನ ಜಾನ್ ಬೇರೊಬ್ಬ ಹುಡುಗಿಯೊಂದಿಗೆ ಮಾತನಾಡುತ್ತಿರುವುದನ್ನು ಕಂಡರೆ, ಅವಳು ಆ ರಾತ್ರಿ ನಿದ್ರಿಸಲಾರಳು, ಅವಳು ಜಾನನ ಜೀವನವನ್ನು ನರಕಸದೃಶವನ್ನಾಗಿ ಮಾಡುವಳು. ಇದು ಪ್ರೀತಿಯಲ್ಲ. ಇದೊಂದು ಕಾಮ ವ್ಯಾಪಾರ. ಇದನ್ನು ಪ್ರೇಮ ಎನ್ನುವುದು ಪಾಷಂಡಿತನ. ಹಗಲು ರಾತ್ರಿ ಪ್ರಪಂಚವು ದೇವರ ಮತ್ತು ಧರ್ಮದ ವಿಷಯವಾಗಿ ಮಾತನಾಡುತ್ತಿದೆ. ಅದರಂತೆಯೇ ಪ್ರೀತಿ. ನೀವು ಎಲ್ಲವನ್ನೂ ಹಾಸ್ಯಾಸ್ಪದವನ್ನಾಗಿ ಮಾಡುತ್ತಿರುವಿರಿ. ನೀವು ಮಾಡುತ್ತಿರುವುದು ಅದನ್ನೇ. ಪ್ರತಿಯೊಬ್ಬರೂ ಪ್ರೀತಿಯ ವಿಷಯವನ್ನು ಕುರಿತು ಮಾತನಾಡುವರು. ಆದರೂ ವೃತ್ತ ಪತ್ರಿಕೆಗಳಲ್ಲಿ ವಿವಾಹ ವಿಚ್ಛೇದನಗಳನ್ನು ನಾವು ಓದುತ್ತೇವೆ. ನೀವು ಜಾನ್‌ನನ್ನು. ಪ್ರೀತಿಸಿದರೆ, ನೀವು ಅವನಿಗಾಗಿ ಪ್ರೀತಿಸುವಿರೋ, ಅಥವಾ ನಿಮಗಾಗಿ ಪ್ರೀತಿಸುವಿರೂ? ನೀವು ಅವನನ್ನು ನಿಮಗಾಗಿ ಪ್ರೀತಿಸಿದರೆ, ನೀವು ಜಾನನಿಂದ ಯಾವುದನ್ನಾದರೂ ನಿರೀಕ್ಷಿಸುವಿರಿ. ನೀವು ಅವನನ್ನು ಅವನಿಗಾಗಿ ಪ್ರೀತಿಸಿದರೆ, ಅವನಿಂದ ನೀವು ಯಾವುದನ್ನೂ ನಿರೀಕ್ಷಿಸುವುದಿಲ್ಲ. ಅವನು ತನಗೆ ತೋರಿದಂತೆ ಮಾಡಬಹುದು. ಆದರೂ ನೀವು ಅವನನ್ನು ಎಂದಿನಂತೆ ಪ್ರೀತಿಸುತ್ತೀರಿ.

ತ್ರಿಭುಜದ ಪ್ರೇಮ ತ್ರಿಕೋನಗಳು ಇವು. ಪ್ರೇಮವಿಲ್ಲದೆ ಇದ್ದರೆ ತತ್ತ್ವ ನೀರಸವಾಗುವುದು, ಮನಶ್ಶಾಸ್ತ್ರ ಕೇವಲ ಒಂದು ಸಿದ್ಧಾಂತವಾಗುವುದು, ಕರ್ಮ ಕೇವಲ ಒಂದು ದುಡಿತವಾಗುವುದು. ಪ್ರೇಮವಿದ್ದರೆ, ತತ್ತ್ವವು ಕಾವ್ಯಮಯವಾಗುವುದು, ಮನಶ್ಶಾಸ್ತ್ರವು ಅನುಭಾವವಾಗುವುದು. ಕರ್ಮವು ಸೃಷ್ಟಿಯಲ್ಲೇ ಅತ್ಯಂತ ರಸಭರಿತವಾದ ಕೆಲಸವಾಗುವುದು. ಬರಿಯ ಗ್ರಂಥಗಳನ್ನು ಓದುತ್ತಿದ್ದರೆ ಒಬ್ಬನು ಶುಷ್ಕನಾಗುವನು. ಯಾರು ಜ್ಞಾನಿಯಾಗುವವನು? ಯಾರು ಪ್ರೇಮದ ಒಂದು ಬಿಂದುವನ್ನಾದರೂ ರುಚಿ ನೋಡಿರುವನೊ ಅವನು. ಪ್ರೀತಿಯೇ ದೇವರು, ದೇವರೇ ಪ್ರೀತಿ. ಭಗವಂತನು ಸರ್ವಾಂತರ್ಯಾಮಿಯಾಗಿರುವನು. ದೇವರು ಪ್ರೇಮಸ್ವರೂಪನು ಮತ್ತು ವಿಶ್ವವ್ಯಾಪಿ ಎಂದು ಅರಿತವನಿಗೆ ತಾನು ಎಲ್ಲಿ ನಿಂತಿರುವನೋ ಅದು ಅವನಿಗೆ ಗೊತ್ತಾಗುವುದಿಲ್ಲ – ತನ್ನ ತಲೆಯ ಮೇಲೆ ನಿಂತಿರುವನೋ ಅಥವಾ ತನ್ನ ಕಾಲಮೇಲೆ ನಿಂತಿರುವನೊ ಗೊತ್ತಾಗುವುದಿಲ್ಲ. ಒಬ್ಬನ ಕೈಗೆ ಒಂದು ಮದ್ಯದ ಬುಡ್ಡಿ ಸಿಕ್ಕಿದರೆ ಅವನಿಗೆ ಎಲ್ಲಿ ನಿಂತಿರುವನೊ ಎಂಬುದು ಗೊತ್ತಾಗದೆ ಇರುವಂತೆ ನಾವು ನಿಜವಾಗಿ ದೇವರಿಗಾಗಿ ಹತ್ತು ನಿಮಿಷಗಳು ಅತ್ತರೆ, ಅನಂತರ ಮುಂದಿನ ಎರಡು ತಿಂಗಳು ನಾವು ಎಲ್ಲಿರುವೆವೊ ಅದು ನಮಗೆ ಗೊತ್ತಾಗುವುದಿಲ್ಲ; ನಾವು ಏನು ಊಟಮಾಡುತ್ತಿರುವೆವೋ ಅದನ್ನು ಕೂಡ ಗಮನಿಸುವುದಿಲ್ಲ. ನೀವು ದೇವರನ್ನು ಪ್ರೀತಿಸಿ, ಜೊತೆಗೆ ವ್ಯವಹಾರ ಜ್ಞಾನದಲ್ಲಿ ಅಷ್ಟೇ ಚುರುಕಾಗಿ ಚೆನ್ನಾಗಿ ಇರಲು ಹೇಗೆ ಸಾಧ್ಯ? ವಿಶ್ವವಿಜಯಿಯಾದ, ಸರ್ವಶಕ್ತವಾದ ಪ್ರೇಮಶಕ್ತಿ ಹೇಗೆ ಬರಬಲ್ಲದು?~।

ಜನರನ್ನು ನೀವು ಪರೀಕ್ಷಿಸಲು ಯತ್ನಿಸಬೇಡಿ. ಅವರೆಲ್ಲ ಹುಚ್ಚರು, ಮಕ್ಕಳು ಆಟದಲ್ಲಿ ಹುಚ್ಚರಾಗಿರುವರು. ಯುವಕರು ಯುವತಿಯರಲ್ಲಿ ಹುಚ್ಚರಾಗಿರುವರು ವೃದ್ದರು ಚಿಂತೆಯಲ್ಲಿ ಮಗ್ನರಾಗಿರುವರು. ಕೆಲವರು ಹೊನ್ನಿಗಾಗಿ ಹುಚ್ಚರಾಗಿರುವರು. ಕೆಲವರು ಏತಕ್ಕೆ ದೇವರಿಗಾಗಿ ಹುಚ್ಚರಾಗಬಾರದು? ಜಾನ್ ಮತ್ತು ಜೇನಳಿಗಾಗಿ ಹುಚ್ಚರಾಗುವಂತೆ ದೇವರಿಗಾಗಿ ಹುಚ್ಚರಾಗಿ, ಅವರು ಯಾರು ಎಂದು ಜನರು ಕೇಳುವರು. ನಾನು ಇದನ್ನು ಬಿಡಲೆ, ಅದನ್ನು ಬಿಡಲೆ ಎಂದು ಜನರು ಕೇಳುವರು. ಒಬ್ಬ ತಾನು ಮದುವೆಮಾಡಿಕೊಳ್ಳುವುದನ್ನು ಬಿಡಲೆ ಎಂದು ಕೇಳಿದ. ನೀವು ಯಾವುದನ್ನೂ ಬಿಡಬೇಕಾಗಿಲ್ಲ. ಅದೇ ನಿಮ್ಮನ್ನು ಬಿಡುತ್ತದೆ. ತಾಳಿ, ನೀವೇ ಅದನ್ನು ಮರೆಯುತ್ತೀರಿ.

ಸಂಪೂರ್ಣವಾಗಿ ಭಗವತ್ ಪ್ರೇಮದಲ್ಲೇ ತನ್ಮಯನಾಗುವುದಾದರೆ ಅದೇ ನಿಜವಾದ ಪೂಜೆ. ರೋಮನ್ ಕ್ಯಾಥೋಲಿಕ್ ಚರ್ಚಿನಲ್ಲಿ ನಿಮಗೆ ಕೆಲವು ವೇಳೆ ಅಂತಹ ಪಕ್ಷಿನೋಟಗಳು ಸಿಕ್ಕುವುವು. ಕೆಲವು ಮಂದಿ ಅಪೂರ್ವ ಸಾಧುಗಳು ಮತ್ತು ಸಾಧ್ವಿಗಳು ಪರಮಪ್ರೀತಿಯಿಂದ ಹುಚ್ಚರಾಗಿ ಹೋಗುವರು. ಅಂತಹ ಪ್ರೀತಿಯನ್ನು ನೀವು ಪಡೆಯಬೇಕು. ಪ್ರೀತಿ ಹೀಗೆ ಇರಬೇಕು. ಏನನ್ನೂ ಕೇಳಕೂಡದು. ಏನನ್ನೂ ಅರಸಕೂಡದು.

ಅವನನ್ನು ಹೇಗೆ ಪೂಜೆಮಾಡುವುದು ಎಂಬ ಪ್ರಶ್ನೆ ಏಳುವುದು. ಅವನನ್ನು ಅತ್ಯಂತ ಪ್ರಿಯತಮವಾದ ವಸ್ತುವಿನಂತೆ ಪ್ರೀತಿಸಿ. ಎಲ್ಲಾ ಬಂಧುಗಳಿಗಿಂತಲೂ, ಮಕ್ಕಳಿಗಿಂತಲೂ ಪ್ರಿಯನೆಂದು ಪ್ರೀತಿಸಿ. ನಿಮಗೆ ಅತ್ಯಂತ ಪ್ರಿಯರಾದವರನ್ನು ನೀವು ಪ್ರೀತಿಸುವ ಹಾಗೆ ದೇವರನ್ನು ಪ್ರೀತಿಸಿ. ಇದೊಂದೇ ದೇವರ ವಿಷಯದ ವಿವರಣೆ. ಈ ಪ್ರಪಂಚ ನಾಶವಾದರೂ ಚಿಂತಿಸಬೇಡಿ. ಎಲ್ಲಿಯವರೆಗೆ ಅವನು ಪರಮ ಪ್ರೇಮವಾಗಿ ಇರುವನೋ ಅಲ್ಲಿಯವರೆಗೆ ಯಾವುದನ್ನು ನಾವು ಲೆಕ್ಕಿಸಬೇಕು? ಪೂಜೆ ಎಂದರೆ ಏನು ಎಂಬುದು ನಿಮಗೆ ಗೊತ್ತೆ? ಎಲ್ಲಾ ಆಲೋಚನೆಗಳೂ ಹೋಗಬೇಕು. ದೇವರಲ್ಲದೆ ಉಳಿದ ಭಾವನೆಗಳೆಲ್ಲ ಮಾಯವಾಗಬೇಕು. ತಾಯಿತಂದೆಗಳಿಗೆ ತಮ್ಮ ಮಕ್ಕಳ ಮೇಲೆ ಇರುವ ಪ್ರೇಮ, ಹೆಂಡತಿಗೆ ಗಂಡನ ಮೇಲೆ ಇರುವ ಪ್ರೇಮ, ಗಂಡನಿಗೆ ಹೆಂಡತಿಯ ಮೇಲೆ ಇರುವ ಪ್ರೇಮ, ಸ್ನೇಹಿತನಿಗೆ ಸ್ನೇಹಿತನ ಮೇಲೆ ಇರುವ ಪ್ರೇಮ, ಈ ಪ್ರೇಮವನ್ನೆಲ್ಲ ಸೇರಿಸಿ ದೇವರಿಗೆ ಅರ್ಪಿಸಬೇಕು. ಹೆಂಗಸು ಒಬ್ಬ ಗಂಡಸನ್ನು ಪ್ರೀತಿಸಿದರೆ, ಅವಳು ಬೇರೊಬ್ಬನನ್ನು ಪ್ರೀತಿಸಲಾರಳು. ಗಂಡಸು ಒಬ್ಬ ಹೆಂಗಸನ್ನು ಪ್ರೀತಿಸಿದರೆ ಅವನು ಬೇರೊಬ್ಬ ಹೆಂಗಸನ್ನು ಪ್ರೀತಿಸಲಾರನು. ಪ್ರೀತಿಯ ಸ್ವಭಾವವೇ ಇದು.

ಹಿರಿಯರಾದ ನನ್ನ ಗುರುಗಳು ಹೀಗೆ ಹೇಳುತ್ತಿದ್ದರು: “ಈ ಕೋಣೆಯಲ್ಲಿ ಚಿನ್ನದ ನಾಣ್ಯಗಳಿರುವ ಒಂದು ಚೀಲವಿದೆ ಎಂದು ಭಾವಿಸೋಣ. ಅದರ ಪಕ್ಕದ ಕೋಣೆಯಲ್ಲಿಯೇ ಒಬ್ಬ ಕಳ್ಳನಿರುವನು. ಆ ಕಳ್ಳನಿಗೆ ಪಕ್ಕದ ಕೋಣೆಯಲ್ಲಿಯೇ ಹಣವಿದೆ ಎಂದು ಚೆನ್ನಾಗಿ ಗೊತ್ತಿದೆ. ಆ ಕಳ್ಳನಿಗೆ ಆಗ ನಿದ್ರೆ ಬರುವುದೆ? ಎಂದಿಗೂ ಸಾಧ್ಯವಿಲ್ಲ. ಅವನು ಯಾವಾಗಲೂ ಆ ಚಿನ್ನವನ್ನು ಹೇಗೆ ಅಪಹರಿಸಬಲ್ಲೆ ಎಂಬ ಯೋಚನೆಯಲ್ಲಿಯೇ ತಲ್ಲೀನನಾಗಿರುವನು''. ಇದರಂತೆಯೇ ಒಬ್ಬ ದೇವರನ್ನು ಪ್ರೀತಿಸಿದರೆ ಅವನು ಇನ್ನು ಬೇರೆ ವಸ್ತುಗಳನ್ನು ಹೇಗೆ ಪ್ರೀತಿಸಲು ಸಾಧ್ಯ? ಭಗವಂತನ ಅದ್ಭುತ ಪ್ರೇಮಕ್ಕೆ ವಿರೋಧವಾಗಿ ಯಾವುದು ನಿಲ್ಲಬಹುದು? ಉಳಿದುದೆಲ್ಲವೂ ಮಾಯವಾಗುವುದು. ಆ ಪರಮಪ್ರೇಮವನ್ನು ಹುಡುಕುವುದರಲ್ಲಿ, ಅದನ್ನು ಸಾಕ್ಷಾತ್ಕಾರ ಮಾಡಿಕೊಳ್ಳುವುದರಲ್ಲಿ, ಅನುಭವಿಸುವುದರಲ್ಲಿ, ಅದರಲ್ಲೆ ಬಾಳುವುದರಲ್ಲಿ ಉನ್ಮತ್ತನಾಗದೆ ಹೇಗೆ ಇರಬಲ್ಲನು?

ನಾವು ದೇವರನ್ನು ಹೇಗೆ ಪ್ರೀತಿಸಬೇಕು. “ನನಗೆ ದ್ರವ್ಯ ಬೇಕಾಗಿಲ್ಲ, ಸ್ನೇಹಿತರು ಬೇಕಾಗಿಲ್ಲ, ಸುಂದರಿಯರು ಬೇಕಾಗಿಲ್ಲ, ಪದವಿ ಬೇಕಾಗಿಲ್ಲ, ಪಾಂಡಿತ್ಯ ಬೇಕಾಗಿಲ್ಲ, ಯುಕ್ತಿಯೂ ಬೇಕಾಗಿಲ್ಲ. ನಿನ್ನ ಇಚ್ಚೆಯಾದರೆ ಸಹಸ್ರಾರು ಮರಣಗಳನ್ನು ನೀಡು. ಆದರೆ ಇದನ್ನು ಮಾತ್ರ ನೀನು ಅನುಗ್ರಹಿಸು: ನಾನು ನಿನ್ನನ್ನು ಮಾತ್ರ ಪ್ರೀತಿಸಬೇಕು. ಅದು ಪ್ರೀತಿಗಾಗಿ ಪ್ರೀತಿಸುವುದಾಗಬೇಕು. ಪ್ರಾಪಂಚಿಕರು ವಿಷಯವಸ್ತುಗಳ ಮೇಲೆ ಯಾವ ಪ್ರೀತಿಯನ್ನು ಇಟ್ಟುಕೊಂಡಿರುವರೊ ಅಂತಹ ನಿಕಟವಾದ ಪ್ರೀತಿ ನಿನ್ನ ಸೌಂದರ್ಯದ ಮೇಲೆ ಇರಲಿ.” ಭಗವಂತನನ್ನು ಕೊಂಡಾಡಿ, ಭಗವದ್ ಭಕ್ತರನ್ನು ಕೊಂಡಾಡಿ, ದೇವರು ಇದಲ್ಲದೆ ಬೇರೆ ಅಲ್ಲ. ಹಲವು ಯೋಗಿಗಳು ಮಾಡಬಲ್ಲ ಅದ್ಭುತಗಳನ್ನು ಇವನು ಲೆಕ್ಕಿಸುವುದಿಲ್ಲ. ಸಣ್ಣ ಮಂತ್ರವಾದಿಗಳು ಸಣ್ಣ ಕೈಚಳಕಗಳನ್ನು ಮಾಡುತ್ತಾರೆ. ದೇವರು ದೊಡ್ಡ ಮಂತ್ರವಾದಿ. ಅವನು ದೊಡ್ಡ ಕೈಚಳಕಗಳನ್ನು ಮಾಡುತ್ತಾನೆ. ಈ ವಿಶ್ವದಲ್ಲಿ ಎಷ್ಟು ಲೋಕಗಳಿವೆಯೋ ಅದು ಯಾರಿಗೆ ಬೇಕಾಗಿದೆ?

\newpage

ಬೇರೊಂದು ಮಾರ್ಗವಿದೆ. ಅದೇ ಪ್ರತಿಯೊಂದನ್ನೂ ಗೆಲ್ಲುವುದು, ಪ್ರತಿಯೊಂದನ್ನೂ ಸೋಲಿಸುವುದು, ದೇಹವನ್ನು ಮತ್ತು ಮನಸ್ಸುಗಳನ್ನು ಗೆಲ್ಲುವುದು. ಆದರೆ ಭಕ್ತನಾದರೊ “ಇವುಗಳನ್ನೆಲ್ಲ ಗೆದ್ದು ಏನು ಪ್ರಯೋಜನ? ನನ್ನ ಸಂಬಂಧ ದೇವರೊಡನೆ ಇರುವುದು" ಎಂದು ಹೇಳುತ್ತಾನೆ.

ಒಬ್ಬ ಯೋಗಿಗಳಿದ್ದರು. ಮಹಾ ಭಕ್ತರವರು. ಅವರಿಗೆ ಗಂಟಲಿನಲ್ಲಿ ವ್ರಣವಾಗಿ ಸಾಯುವ ಸ್ಥಿತಿಗೆ ಬಂದರು. ತತ್ತ್ವಜ್ಞಾನಿಯಾದ ಮತ್ತೊಬ್ಬ ಯೋಗಿ ಅವರ ಬಳಿಗೆ ಬಂದನು. ಆತ, “ನೋಡು ನನ್ನ ಸ್ನೇಹಿತನೆ, ನೀನು ಏತಕ್ಕೆ ಆ ವ್ರಣದ ಮೇಲೆ ನಿನ್ನ ಮನಸ್ಸನ್ನು ಏಕಾಗ್ರಮಾಡಿ ವ್ರಣದಿಂದ ಪಾರಾಗಬಾರದು?” ಎಂದು ಕೇಳಿದನು. ಮೂರನೆಯ ಬಾರಿಯೂ ಈ ಪ್ರಶ್ನೆಯನ್ನು ಆತ ಹಾಕಿದನು. ಆಗ ಇವರು, “ಯಾವ ಮನಸ್ಸನ್ನು ಪೂರ್ಣವಾಗಿ ಭಗವಂತನಿಗೆ ಅರ್ಪಿಸಿರುವೆನೊ, ಅದನ್ನು ರಕ್ತ ಮಾಂಸಗಳ ಒಂದು ಚೀಲದ ಮೇಲೆ ಇಡಲು ಸಾಧ್ಯವೆ?” ಎಂದು ಕೇಳಿದರು. ಕ್ರಿಸ್ತ, ದೇವತೆಗಳ ಸೈನ್ಯವನ್ನು ತನ್ನ ಸಹಾಯಕ್ಕೆ ತರಬಹುದಾಗಿತ್ತು, ಆದರೆ ತರಲಿಲ್ಲ. ಏನು ಇಪ್ಪತ್ತು ಸಾವಿರ ದೇವತೆಗಳ ಸಹಾಯದಿಂದ ಈ ದೇಹವನ್ನು ಎರಡು ಮೂರು ದಿನಗಳು ಹೆಚ್ಚಾಗಿ ಬಾಳುವಂತೆ ಮಾಡುವಷ್ಟು ಶ್ರೇಷ್ಠವಾದುದೇ ಈ ದೇಹ?

ಪ್ರಾಪಂಚಿಕ ದೃಷ್ಟಿಯಿಂದ ದೇಹವೇ ನನ್ನ ಸರ್ವಸ್ವ. ಈ ದೇಹವೇ ನನ್ನ ಪ್ರಪಂಚ. ದೇಹವೇ ನನ್ನ ದೇವರು. ನಾನು ದೇಹ, ನೀವೇನಾದರೂ ನನ್ನ ದೇಹವನ್ನು ಜಿಗುಟಿದರೆ, ನನ್ನನ್ನು ಜಿಗುಟುತ್ತೀರಿ. ನನಗೆ ಸ್ವಲ್ಪ ತಲೆನೋವು ಬಂದರೆ ಸಾಕು ದೇವರನ್ನು ನಾನು ಮರೆಯುತ್ತೇನೆ: ನಾನು ದೇಹ! ಈ ಪರಮ ಆದರ್ಶವಾದ ದೇಹಕ್ಕಾಗಿ ದೇವರು ಮತ್ತು ಇತರರೆಲ್ಲರೂ ಬರಬೇಕು. ಈ ದೃಷ್ಟಿಯಿಂದ, ಕ್ರಿಸ್ತ ಶಿಲುಬೆಯ ಮೇಲೆ ಸತ್ತಾಗ, ದೇವತೆಗಳ ಸಮೂಹವನ್ನು ತನ್ನ ಸಹಾಯಕ್ಕೆ ತರದೆ ಹೋದುದರಿಂದ ಅವನು ಒಬ್ಬ ಮೂರ್ಖನಾಗುವನು. ಅವನು ದೇವತೆಗಳ ಸಹಾಯದಿಂದ ಶಿಲುಬೆಯಿಂದ ಪಾರಾಗಬೇಕಾಗಿತ್ತು. ಆದರೆ ಭಕ್ತನ ದೃಷ್ಟಿಯಿಂದ ಈ ದೇಹ ಕೆಲಸಕ್ಕೆ ಬಾರದುದು, ಈ ಅನಿಷ್ಟ ಇದ್ದರೆ ಎಷ್ಟು, ಹೋದರೆ ಎಷ್ಟು? ಬಂದು ಹೋಗುವ ಈ ದೇಹವನ್ನು ಕುರಿತು ಯಾರು ಚಿಂತಿಸುವರು? ರೋಮನ್ ಸಿಪಾಯಿಗಳು ಲಾಟರಿ ಹಾಕುವುದಕ್ಕೆ ಉಪಯೋಗಿಸುವ ಒಂದು ಚಿಂದಿ ಬಟ್ಟೆಗಿಂತ ಮೇಲಲ್ಲ ಇದು..

ಪ್ರಾಪಂಚಿಕನ ದೃಷ್ಟಿಗೂ ಭಕ್ತನ ದೃಷ್ಟಿಗೂ ಧ್ರುವದಷ್ಟು ಅಂತರವಿದೆ. ಪ್ರೀತಿಸುತ್ತ ಹೋಗಿ, ಯಾರೋ ಕೋಪಗೊಡರೆ ನೀವು ಏತಕ್ಕೆ ಕೋಪಗೊಳ್ಳಬೇಕು? “ಯಾರೋ ಅಧೋಗತಿಗೆ ಬಂದರೆ ನಾನೇತಕ್ಕೆ ಅಧೋಗತಿಗೆ ಬರಬೇಕು? ಯಾರೋ ಒಬ್ಬ ಮೂರ್ಖನಾದರೆ ನಾನು ಏತಕ್ಕೆ ಕೋಪಗೊಂಡು ಮೂರ್ಖನಾಗಬೇಕು? ಪಾಪವನ್ನು ಎದುರಿಸುವುದಲ್ಲ.” ಭಗವತ್‌ಭಕ್ತರು ಹೇಳುವುದು ಹೀಗೆ. ಪ್ರಪಂಚ ಏನನ್ನಾದರೂ ಮಾಡಲಿ, ಅದು ಎಲ್ಲಿಗೆ ಬೇಕಾದರೂ ಹೋಗಲಿ. ಇವರ ಮೇಲೆ ಯಾವ ಪ್ರಭಾವವನ್ನೂ ಬೀರಲಾರದು. ಅದು.

\newpage

ಒಬ್ಬ ಯೋಗಿ ಅದ್ಭುತ ಶಕ್ತಿಯನ್ನು ಪಡೆದನು. ಅವನು ಹೇಳಿದನು “ನೋಡು ನನ್ನ ಶಕ್ತಿಯನ್ನು. ಮೇಲೆ ಆಕಾಶವನ್ನು ನೋಡು. ಅದನ್ನೆಲ್ಲ ಮೋಡಗಳಿಂದ ಮುಚ್ಚುತ್ತೇನೆ'' ಎಂದನು. ಆಮೇಲೆ ಮಳೆ ಬರಲು ಆರಂಭವಾಯಿತು. “ಸ್ವಾಮಿ ನೀವು ಒಂದು ಅದ್ಭುತವನ್ನು ಮಾಡಿರುವಿರಿ. ಆದರೆ ನಾನು ಯಾವುದನ್ನು ತಿಳಿದರೆ, ಮತ್ತೆ ಯಾವುದನ್ನೂ ಕೋರುವುದಿಲ್ಲವೊ ಅದನ್ನು ಬೋಧಿಸಿ'' ಎಂದು ಯಾರೋ ಕೇಳಿಕೊಂಡರು. ಶಕ್ತಿಯನ್ನು ಬಯಸಕೂಡದು, ಏನನ್ನೂ ಬಯಸಕೂಡದು. ಇದನ್ನು ಕೇವಲ ಬುದ್ಧಿಶಕ್ತಿಯಿಂದಲೇ ಗ್ರಹಿಸುವುದಕ್ಕೆ ಆಗುವುದಿಲ್ಲ, ಸಹಸ್ರಾರು ಗ್ರಂಥಗಳನ್ನು ಓದಿದರೂ ನಿಮಗೆ ಇದು ಅರ್ಥವಾಗಲಾರದು. ನಾವು ಇದನ್ನು ಅರ್ಥಮಾಡಿಕೊಳ್ಳುವುದಕ್ಕೆ ಪ್ರಾರಂಭಿಸಿದರೆ ಒಂದು ಹೊಸ ಪ್ರಪಂಚ ಬೆಳಕಿಗೆ ಬರುವುದು. ಹುಡುಗಿ ತನ್ನ ಗೊಂಬೆಯೊಂದಿಗೆ ಆಟವಾಡುತ್ತಿರುವಾಗ ಆ ಗೊಂಬೆಗೆ ಯಾವಾಗಲೂ ಹೊಸ ಗಂಡನನ್ನು ಹೊಂಚುತ್ತಿರುವಳು. ಆದರೆ ಅವಳಿಗೆ ತನ್ನ ನಿಜವಾದ ಗಂಡ ಸಿಕ್ಕಿದರೆ, ಗೊಂಬೆಗಳನ್ನೆಲ್ಲ ಒಂದೇ ಸಲ ಮೂಲೆಗೆ ಎಸೆಯುವಳು. ಇದರಂತೆಯೇ ನಮ್ಮ ಆಟವೂ ಕೂಡ. ಭಾಸ್ಕರನು ಪ್ರೇಮ ಉದಯಿಸಿದಾಗ ಆಟದ ಸೂರ್ಯರು, ಎಂದರೆ ನಮ್ಮ ಇತರ ಬಯಕೆಗಳೆಲ್ಲ ಒಂದೇ ಸಲ ಮಾಯವಾಗುವುವು. ನಾವು ಶಕ್ತಿಯಿಂದ ಏನು ಮಾಡಬಲ್ಲೆವು? ನಮ್ಮಲ್ಲಿರುವ ಶಕ್ತಿ ಹೊರಟುಹೋದರೆ ದೇವರಿಗೆ ಧನ್ಯವಾದವನ್ನು ಅರ್ಪಿಸಿ, ಪ್ರೀತಿಸುವುದಕ್ಕೆ ಪ್ರಾರಂಭಿಸಿ. ನನಗೂ ದೇವರಿಗೂ ಮಧ್ಯದಲ್ಲಿ ಪ್ರೇಮ ವಿನಃ ಮತ್ತಾವುದೂ ಇರಕೂಡದು. ದೇವರು ಪ್ರೇಮಪಾತ್ರನು, ಅವನು ಮತ್ತೇನೂ ಅಲ್ಲ. ಪ್ರಾರಂಭದಲ್ಲಿ ಪ್ರೀತಿ, ಮಧ್ಯದಲ್ಲಿ ಪ್ರೀತಿ, ಕೊನೆಯಲ್ಲಿ ಪ್ರೀತಿ ಇರಬೇಕು.

ರಾಣಿಯೊಬ್ಬಳು ಬೀದಿಯಲ್ಲಿ ಭಗವಂತನ ಭಕ್ತಿಯನ್ನು ಜನರಿಗೆ ಬೋಧಿಸುತ್ತಿದ್ದ ಕಥೆಯೊಂದಿದೆ. ಕೋಪಗೊಂಡ ಅವಳ ಗಂಡನು ಅವಳನ್ನು ಹಿಂಸಿಸಲು ಪ್ರಾರಂಭಿಸಿದನು. ಅವಳನ್ನು ದೇಶದಲ್ಲೆಲ್ಲ ಓಡಾಡಿಸಿ ತೊಂದರೆಗೆ ಈಡುಮಾಡಿದರು. ಅವಳು ಭಗವಂತನ ಮೇಲೆ ತನಗಿರುವ ಪ್ರೀತಿಯನ್ನು ವಿವರಿಸಿ ಹಾಡುತ್ತಿದ್ದಳು. ಅವಳ ಹಾಡನ್ನು ಈಗ ದೇಶದಲ್ಲೆಲ್ಲ ಹಾಡುತ್ತಿರುವರು. “ಕಂಬನಿಯ ನೀರನ್ನು ಎರೆದು ಶಾಶ್ವತವಾದ ಪ್ರೇಮದ ಬಳ್ಳಿಯನ್ನು ನಾನು ಆರೈಕೆ ಮಾಡಿದೆನು.” ಇದೇ ಪರಮಾದರ್ಶ. ಇದನ್ನು ಬಿಟ್ಟು ಇನ್ನೇನು ಉಳಿದಿರುವುದು? ಜನರಿಗೆ ಕೆಲಸಕ್ಕೆ ಬಾರದುದೆಲ್ಲ ಬೇಕು. ಅದನ್ನು ಪಡೆಯಬೇಕೆಂದು ಆಶಿಸುವರು. ಆದಕಾರಣವೇ ಆ ಪರಮ ಪ್ರೇಮವನ್ನು ಅರ್ಥಮಾಡಿ ಕೊಳ್ಳುವವರ ಸಂಖ್ಯೆ ಅತಿ ವಿರಳ, ಅದರ ಸಮೀಪಕ್ಕೆ ಬರುವ ಮಂದಿಗಳು ಮತ್ತೂ ವಿರಳ, ಅವರನ್ನು ಜಾಗೃತರನ್ನಾಗಿ ಮಾಡಿ, ಅವರಿಗೆ ಕೆಲವು ಸಲಹೆಗಳು ಸಿಕ್ಕಲಿ.

ಪ್ರೇಮವೇ ಆದಿ–ಅಂತ್ಯವಿಲ್ಲದ ಒಂದು ಯೋಗ. ನೀವು ಎಲ್ಲವನ್ನೂ ತ್ಯಜಿಸಬೇಕಾಗಿದೆ. ನೀವು ಯಾವುದನ್ನೂ ಸ್ವೀಕರಿಸಲಾರಿರಿ. ಪ್ರೇಮ ನಿಮಗೆ ದೊರೆತ ಮೇಲೆ ನಿಮಗೆ ಮತ್ತಾವುದೂ ಬೇಕಾಗುವುದಿಲ್ಲ. “ದೇವರೆ, ನೀನು ಮಾತ್ರ ಎಂದೆಂದಿಗೂ ನನ್ನ ಪ್ರೇಮೇಶ್ವರನಾಗು.” ಪ್ರೇಮ ಬಯಸುವುದು ಇದನ್ನು. “ನನ್ನ ಪ್ರೇಮವೇ, ನಿನ್ನ ತುಟಿಗಳಿಂದ ಒಂದು ಮುತ್ತು! ಯಾರು ನಿನ್ನಿಂದ ಆ ಚುಂಬನವನ್ನು ಪಡೆದಿರುವನೊ, ಅವನು ಎಲ್ಲಾ ದುಃಖಗಳಿಂದಲೂ ಪಾರಾಗುವನು. ಒಮ್ಮೆ ನಿನ್ನಿಂದ ಚುಂಬಿಸಿಕೊಂಡರೆ ಮನುಷ್ಯ ಎಂದೆಂದಿಗೂ ಧನ್ಯನಾಗುವನು, ಉಳಿದ ವಸ್ತುಗಳ ಮೇಲೆ ಇರುವ ಪ್ರೀತಿಯೆಲ್ಲ ಮಾಯವಾಗುವುದು. ಅವನು ನಿನ್ನನ್ನು ಮಾತ್ರ ಕೊಂಡಾಡುವನು, ನಿನ್ನನ್ನು ಮಾತ್ರ ನೋಡುವನು.” ಮಾನವನ ಪ್ರೀತಿಯ ಸ್ವಭಾವದಲ್ಲಿಯೂ ಕೂಡ ಆ ಪರಮ ಪವಿತ್ರ ಪ್ರೇಮದ ಕಣಗಳಿವೆ. ತೀವ್ರ ಪ್ರೇಮದ ಪ್ರಾರಂಭದಲ್ಲಿ, ಇಡಿಯ ವಿಶ್ವ ನಿಮ್ಮ ಹೃದಯಕ್ಕೆ ಸರಿಯಾಗಿ ಹೊಂದಿಕೊಂಡಿರುವಂತೆ ತೋರುವುದು. ಪ್ರಪಂಚದ ಪ್ರತಿಯೊಂದು ಹಕ್ಕಿಯೂ, ನಿಮ್ಮ ಪ್ರೇಮವನ್ನೆ ಕುರಿತು ಗಾನಮಾಡುವುದು, ಹೂವುಗಳು ನಿಮಗಾಗಿ ಅರಳುತ್ತವೆ. ಮಾನವ ಪ್ರೀತಿ ಬರುವುದು ಆದಿ ಅಂತ್ಯವಿಲ್ಲದ ಅನಂತ ಪ್ರೇಮಾಂಬುಧಿಯಿಂದ.

ಭಗವದ್ಭಕ್ತನು ಯಾವುದಕ್ಕಾದರೂ ಏತಕ್ಕೆ ಅಂಜಬೇಕು? ಅವನು ಕಳ್ಳರಿಗೆ ಅಂಜುವನೆ? ದುಃಖಕ್ಕೆ ಅಂಜುವನೆ? ಅವನು ತನ್ನ ಪ್ರಾಣಕ್ಕಾದರೂ ಅಂಜುವನೇನು? ನಾವೆಲ್ಲ ಈ ಸ್ವರ್ಗ–ನರಕಗಳ ಭಾವನೆಯನ್ನು ತ್ಯಜಿಸಿ ಹೆಚ್ಚು ಪ್ರೀತಿಯನ್ನು ಪಡೆಯಬೇಕಾಗಿದೆ. ಈ ಪ್ರೇಮದ ಉನ್ಮತ್ತತೆಯನ್ನು ಪಡೆಯುವುದಕ್ಕಾಗಿ ನೂರಾರು ಜನರು ಹುಡುಕಾಡುತ್ತಿರುವರು. ಇವರ ಮುಂದೆ ದೇವರು ವಿನಃ ಉಳಿದೆಲ್ಲವೂ ಮಾಯವಾಗುವುದು.

ಕೊನೆಗೆ ಭಕ್ತ ಭಗವಾನ್ ಭಕ್ತಿ–ಇವೆಲ್ಲವೂ ಒಂದೇ ಆಗುವುವು. ಅದೇ ಗುರಿ. ಆತ್ಮಕ್ಕೂ ಮಾನವನಿಗೂ ಏತಕ್ಕೆ ವ್ಯತ್ಯಾಸವಿದೆ? ಆತ್ಮಕ್ಕೂ ದೇವರಿಗೂ ಏತಕ್ಕೆ ವ್ಯತ್ಯಾಸವಿದೆ? ಈ ಪ್ರೇಮವನ್ನು ಅನುಭವಿಸುವುದಕ್ಕಾಗಿ. ಅವನು ತನ್ನನ್ನು ತಾನೆ ಪ್ರೀತಿಸಬಯಸಿದನು. ಅದಕ್ಕಾಗಿ ಅವನು ಹಲವಾದನು. ಸೃಷ್ಟಿಗೆ ಮೂಲಕಾರಣ ಇದೇ ಎನ್ನುವನು ಭಕ್ತ. ನಾವೆಲ್ಲ ಒಂದೇ. “ನಾನು ಮತ್ತು ನನ್ನ ತಂದೆ ಇಬ್ಬರೂ ಒಂದೇ.” ಸದ್ಯಕ್ಕೆ ನಾನು ದೇವರನ್ನು ಪ್ರೀತಿಸುವುದಕ್ಕಾಗಿ ಬೇರೆಯಾಗಿರುವೆನು. ಸಕ್ಕರೆಯಾಗುವುದು ಅಥವಾ ಸಕ್ಕರೆಯನ್ನು ರುಚಿ ನೋಡುವುದು ಇವೆರಡರಲ್ಲಿ ಯಾವುದು ಒಳ್ಳೆಯದು? ಸಕ್ಕರೆ ಆಗುವುದರಲ್ಲಿ ಏನು ಆನಂದವಿದೆ? ಸಕ್ಕರೆ ರುಚಿ ನೋಡುವುದರಲ್ಲಿ ಪ್ರೀತಿಯ ಅನಂತಾನಂದ ಇರುವುದು.

ದೇವರನ್ನು ತಂದೆ, ತಾಯಿ, ಸಖ ಮತ್ತು ಮಗುವೆಂದು ಭಾವಿಸುವುದು, ದೇವರ ಮೇಲೆ ನಮ್ಮಲ್ಲಿ ಇರುವ ಭಕ್ತಿಯನ್ನು ವೃದ್ಧಿಗೊಳಿಸುವುದಕ್ಕೆ ಮತ್ತು ದೇವರ ಬಳಿಗೆ ನಾವು ಹೋಗುತ್ತಿರುವೆವು, ಅವನು ನಮಗೆ ಹೆಚ್ಚು ಪ್ರಿಯನಾಗುತ್ತ ಬರುತ್ತಿರುವನು ಎಂಬುದನ್ನು ತೋರುವುದಕ್ಕೆ. ಅತ್ಯಂತ ತೀವ್ರವಾದ ಪ್ರೇಮವೇ ಸ್ತ್ರೀ ಪುರುಷರಲ್ಲಿ ಇರುವುದು. ದೇವರನ್ನು ಅದೇ ತೀವ್ರತೆಯಿಂದ ಪ್ರೀತಿಸಬೇಕು. ಹೆಂಗಸು ತನ್ನ ತಂದೆಯನ್ನು ಪ್ರೀತಿಸುತ್ತಾಳೆ. ಅವಳು ತನ್ನ ತಾಯಿಯನ್ನು ಪ್ರೀತಿಸುತ್ತಾಳೆ. ಅವಳು ತನ್ನ ಮಗುವನ್ನು ಪ್ರೀತಿಸುತ್ತಾಳೆ. ಅವಳು ತನ್ನ ಸ್ನೇಹಿತಳನ್ನು ಪ್ರೀತಿಸುತ್ತಾಳೆ. ಆದರೆ ಅವಳು ಮುಚ್ಚುಮರೆಯಿಲ್ಲದೆ ತನ್ನಲ್ಲಿರುವ ಭಾವನೆಗಳನ್ನೆಲ್ಲ ತಂದೆ, ತಾಯಿ, ಸ್ನೇಹಿತರು ಮತ್ತು ಮಕ್ಕಳಲ್ಲಿ ಹೇಳಿಕೊಳ್ಳುವುದಿಲ್ಲ. ಅವಳು ಒಬ್ಬನಿಂದ ಮಾತ್ರ ತನ್ನಲ್ಲಿರುವುದನ್ನು ಬಚ್ಚಿಡುವುದಿಲ್ಲ. ಅದರಂತೆಯೆ ಪುರುಷನು ಕೂಡ. ಸತಿಪತಿಯರ ಸಂಬಂಧದಲ್ಲಿ ಇತರ ಸಂಬಂಧಗಳೆಲ್ಲ ಸಮಾಗಮವಾಗಿವೆ. ಗಂಡನಲ್ಲಿ ಹೆಂಡತಿಗೆ ತನ್ನ ತಂದೆ ಇರುವನು, ಸ್ನೇಹಿತ ಇರುವನು, ಮಗ ಇರುವನು; ಹೆಂಡತಿಯಲ್ಲಿ ಗಂಡನಿಗೆ ತಾಯಿ, ಮಗಳು, ಮತ್ತು ಬೇರೆ ಎಲ್ಲ ಇರುವರು. ಸತಿಪತಿಯರಲ್ಲಿರುವ ತೀವ್ರವಾದ ಪ್ರೇಮಾಕರ್ಷಣೆ ದೇವರ ಕಡೆಗೆ ಹರಿಯಬೇಕು. ಹೆಂಡತಿಗೆ ತನ್ನ ಗಂಡನೊಡನೆ ಯಾವ ರಕ್ತಸಂಬಂಧ ಇಲ್ಲದೇ ಇದ್ದರೂ, ತನ್ನ ಮನಸ್ಸಿನಲ್ಲಿ ಇರುವ ಭಾವನೆಗಳನ್ನೆಲ್ಲ ಪೂರ್ಣವಾಗಿ, ನಿರ್ಭಯವಾಗಿ, ನಾಚಿಕೆ ಇಲ್ಲದೆ ಹೇಳಿಕೊಳ್ಳುವಳು. ಅಲ್ಲಿ ಯಾವ ಅಜ್ಞಾನವೂ ಇಲ್ಲ. ತನ್ನ ಗಂಡ ತನ್ನಿಂದ ಹೇಗೆ ಏನನ್ನೂ ಬಚ್ಚಿಡುವುದಿಲ್ಲವೋ ಹಾಗೆಯೇ ಅವಳು ಏನನ್ನೂ ಅವನಿಂದ ಬಚ್ಚಿಡುವುದಿಲ್ಲ. ಇಂತಹ ಪ್ರೀತಿಯು ದೇವರ ಮೇಲೆ ಇರಬೇಕಾಗಿದೆ. ಇವುಗಳನ್ನು ತಿಳಿದುಕೊಳ್ಳುವುದು ಬಹಳ ಕಷ್ಟ, ನೀವು ಕ್ರಮೇಣ ಇವುಗಳನ್ನು ಅರ್ಥಮಾಡಿಕೊಳ್ಳುವಿರಿ. ಆಗ ಲಿಂಗಭಾವನೆಯೆಲ್ಲ ಮರೆತುಹೋಗುವುದು. “ಬೇಸಗೆಯ ಕಾಲದಲ್ಲಿ ನದಿಯ ಮರಳ ದಿಣ್ಣೆಯ ಮೇಲೆ ನೀರಿನ ಹನಿಯೊಂದು ಮಾಯವಾಗುವಂತೆ, ಈ ಜೀವನ ಮತ್ತು ಇದಕ್ಕೆ ಸೇರಿದ ಸಂಬಂಧಗಳು.”

ಅವನು ಸೃಷ್ಟಿಕರ್ತ ಎಂಬ ಭಾವನೆಗಳೆಲ್ಲ ಮಕ್ಕಳಿಗೆ. ಅವನು ನನ್ನ ಪ್ರೇಮದ ಸಾರ, ಜೀವದ ಜೀವ, ಇದೇ ನನ್ನ ಹೃದಯದ ಕ್ರಂದನವಾಗಬೇಕು!

“ನನಗೆ ಒಂದು ಆಸೆ ಇದೆ. ಅವರು ನಿನ್ನನ್ನು ವಿಶ್ವೇಶ್ವರ ಎನ್ನುವರು. ನಿನ್ನನ್ನು ಒಳ್ಳೆಯವನು ಅಥವಾ ಕೆಟ್ಟವನು, ಸಣ್ಣವನು ಅಥವಾ ದೊಡ್ಡವನು ಎನ್ನುವರು. ನಾನು ಪ್ರಪಂಚದ ಒಂದು ಅಂಶ, ನೀನು ಕೂಡ ನನ್ನ ಪ್ರೇಮವೆ. ನನ್ನ ದೇಹ ಮನಸ್ಸು ಆತ್ಮ ಎಲ್ಲಾ ನಿನ್ನ ವೇದಿಕೆಯ ಬಳಿ ಇರುವುದು. ಪ್ರೇಮೇಶ್ವರನೇ, ಇವುಗಳನ್ನು ನಿರಾಕರಿಸಬೇಡ.''

