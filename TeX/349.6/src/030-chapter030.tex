
\chapter[ಜೀವ, ಈಶ್ವರ ಮತ್ತು ಧರ್ಮ]{ಜೀವ, ಈಶ್ವರ ಮತ್ತು ಧರ್ಮ\protect\footnote{\engfoot{C.W, Vol. 1, P. 317}}}

ಭೂತಕಾಲದ ಗರ್ಭದೊಳಗಿನಿಂದ ಶತಶತಮಾನಗಳ ಹಿಂದಿನ ವಾಣಿಯೊಂದು\break ನಮಗೆ ಇಂದು ಕೇಳಿಬರುತ್ತಿದೆ. ಅದು ಹಿಮಾಲಯದಲ್ಲಿ ನೆಲಸಿದ ಋಷಿಗಳ ವಾಣಿ, ಕಾನನಾಂತರಗಳಲ್ಲಿ ಪರ್ಣಕುಟೀರಗಳಲ್ಲಿ ನೆಲಸಿದ ಸಾಧುಸಂತರ ವಾಣಿ; ಸೆಮಿಟಿಕ್ ಜನಾಂಗಕ್ಕೆ ಕೇಳಿದ ವಾಣಿ, ಬುದ್ಧ ಮುಂತಾದ ಧಾರ್ಮಿಕ ವಿಭೂತಿ ಪುರುಷರ ವಾಣಿ, ಸೃಷ್ಟಿಯ ಆದಿಯಲ್ಲಿ ಮಾನವನ ಒಡನೆ ಬಂದ ಅವನು ಹೋದಡೆಯೆಲ್ಲ ಸದಾ ಅವನೊಡನೆ ನೆಲೆಸಿರುವ ಜ್ಯೋತಿಯಲ್ಲಿ ವಿಹರಿಸುತ್ತಿರುವವರ ವಾಣಿ. ಅದು ಈಗಲೂ ನಮಗೆ ಕೇಳಿಬರುತ್ತಿದೆ. ಆ ವಾಣಿ ಬೆಟ್ಟಗಳಿಂದ ಹರಿದು ಬರುತ್ತಿರುವ ಝರಿಗಳಂತೆ ಇದೆ. ಒಮ್ಮೆ ಕಣ್ಣಿಗೆ ಕಾಣಿಸುವುದು, ಮತ್ತೊಮ್ಮೆ ಮಾಯವಾಗಿ ಕೊನೆಗೆ ಒಂದು ಅದ್ಭುತ ಪ್ರವಾಹವಾಗಿ ಮೈದೋರುವುದು. ಎಲ್ಲಾ ದೇಶಗಳ ಎಲ್ಲಾ ಧರ್ಮಗಳ ಮಹಾತ್ಮರ, ಸಾಧುಗಳ, ಸಂತರ, ಸ್ತ್ರೀಪುರುಷರ ವಾಣಿಗಳ ಮಿಲನವಾಗಿ ಭೂತಕಾಲದ ಘನತೂರ್ಯವಾಣಿ ಕೇಳಿಬರುತ್ತಿದೆ. ಅದರ ಪ್ರಥಮ ಸಂದೇಶವೆ, ನಿಮ್ಮಲ್ಲಿ ಮತ್ತು ಉಳಿದ ಎಲ್ಲಾ ಧರ್ಮಗಳಲ್ಲಿಯೂ ಶಾಂತಿ ನೆಲಸಲಿ ಎಂಬುದು. ಈ ಸಂದೇಶ ಧರ್ಮಗಳ ಪರಸ್ಪರ ವೈಮನಸ್ಯದಿಂದ ಪ್ರೇರಿತವಾದುದಲ್ಲ; ಅದು ಸಾಮರಸ್ಯವನ್ನೊಳಗೊಂಡಿರುವುದು.

ಮೊದಲು ಈ ಸಂದೇಶವನ್ನು ಗ್ರಹಿಸೋಣ. ಈ ಶತಮಾನದ ಆದಿಯಲ್ಲಿ, ಇನ್ನೇನು ಧರ್ಮದ ಅವಸಾನಕಾಲ ಪ್ರಾಪ್ತವಾಯಿತು ಎಂಬ ಅಂಜಿಕೆ ಮೂಡಿತ್ತು. ವೈಜ್ಞಾನಿಕರ ನವನವಾನ್ವೇಷಣೆಯ ವಜ್ರಾಘಾತಕ್ಕೆ ಸಿಕ್ಕಿ ಮಣ್ಣಿನ ಪಾತ್ರೆಗಳಂತೆ ಹಳೆಯ ಮೂಢನಂಬಿಕೆಗಳು ಧ್ವಂಸವಾಗುತ್ತಿದ್ದವು. ಯಾರು ಧರ್ಮವೆಂದರೆ ಕೆಲವು ಮತಗಳ ಹೊರೆ ಎಂದು, ಅರ್ಥವಿಲ್ಲದ ಆಚಾರಗಳ ಕಂತೆಯೆಂದು ಭಾವಿಸಿದ್ದರೋ, ಅವರು ಗತಿಕಾಣದಾದರು. ಅವರಿಗೆ ದಿಕ್ಕೇ ತೋಚದೆ ಹೋಯಿತು. ಎಲ್ಲಾ ನಂಬಿಕೆಗಳೂ ತಮ್ಮ ಕೈಗಳಿಂದ ನುಸುಳಿಕೊಳ್ಳುತ್ತಿದ್ದವು. ಅಜೇಯವಾದ ಮತ್ತು ಚಾರ್ವಾಕ ಮತದ ಪ್ರಚಂಡ ಪ್ರವಾಹ ಎಲ್ಲವನ್ನೂ ಕೊಚ್ಚಿಕೊಂಡು ಹೋಗುವುದೆಂಬ ಭೀತಿ ಕೆಲಕಾಲ ಮೂಡಿತು. ತಮ್ಮ ಮನಸ್ಸಿಗೆ ತೋಚಿದುದನ್ನು ವ್ಯಕ್ತಪಡಿಸುವುದಕ್ಕೆ ಜನರು ಅಂಜುತ್ತಿದ್ದರು. ಅನೇಕರು, ಸ್ಥಿತಿ ತಮ್ಮ ಕೈಮೀರಿ ಹೋಗಿದೆ, ಧರ್ಮವನ್ನು ಇನ್ನು ಯಾರೂ ಕೇಳುವಂತೆಯೇ ಇಲ್ಲ ಎಂದು ಭಾವಿಸಿದರು. ಆದರೆ ಈಗ ಸ್ಥಿತಿ ಬದಲಾಗಿದೆ. ಧರ್ಮದ ನೆರವಿಗೆ ಮತ್ತೊಂದು ಬಂದಿದೆ. ಅದಾವುದು? ಅದೇ ಅನ್ಯಮತಗಳ ತೌಲನಿಕ ಅಧ್ಯಯನ. ನಾವು ಬಗೆಬಗೆಯ ಧರ್ಮಗಳನ್ನು ನೋಡಿದರೆ ಅವುಗಳ ಸಾರ ಒಂದೇ ಎಂದು ಕಾಣುವುದು. ನಾನು ಹುಡುಗನಾಗಿದ್ದಾಗ ಈ ಸಂದೇಹ ನನ್ನಲ್ಲಿ ಮೂಡಿತ್ತು. ಧರ್ಮದ ಮೇಲಿನ ಭರವಸೆಯನ್ನೆಲ್ಲ ತ್ಯಜಿಸಬೇಕಾಗಬಹುದೆಂದು ತತ್ಕಾಲದಲ್ಲಿ ಭಾವಿಸಿದ್ದೆ. ಆದರೆ ದೇವರ ದಯೆಯಿಂದ ನಾನು ಕ್ರೈಸ್ತ, ಬೌದ್ಧ, ಮಹಮ್ಮದೀಯ ಮತ್ತು ಇತರ ಧರ್ಮಗಳನ್ನು ಓದತೊಡಗಿದೆ. ನನ್ನ ಧರ್ಮದಲ್ಲಿ ಸಾರಿದ ಮೂಲತತ್ತ್ವಗಳನ್ನೇ ಇತರರ ಧರ್ಮಗಳೂ ಕೂಡ ಸಾರುತ್ತಿವೆ ಎಂಬುದನ್ನು ತಿಳಿದು ನನಗೆ ಆಶ್ಚರ್ಯವಾಯಿತು.\break ಸತ್ಯವೆಂದರೆ ಏನು ಎಂದು ನಾನು ಪ್ರಶ್ನೆಯನ್ನು ಹಾಕಿಕೊಂಡೆ. ಈ ಪ್ರಪಂಚ ಸತ್ಯವೇ? ಹೌದು, ಏಕೆ? ಏಕೆಂದರೆ ನಾನು ಅದನ್ನು ನೋಡುತ್ತೇನೆ. ನಾನು ಈಗ ತಾನೆ ಕೇಳಿದ ಹಾಡುಗಳ ಮತ್ತು ವಾದ್ಯಗಳ ಧ್ವನಿ ಸತ್ಯವೇ? ಹೌದು, ಏಕೆಂದರೆ ನಾನೀಗ ಅದನ್ನು ಕೇಳಿದೆ. ತನಗೆ ದೇಹ ಕಣ್ಣು ಕಿವಿಗಳಿವೆ, ತಾನು ನೋಡಲಾಗದ ಒಂದು ಆಧ್ಯಾತ್ಮಿಕ ಸ್ವಭಾವವೂ ಇದೆ ಎಂದು ಮನುಷ್ಯನಿಗೆ ಗೊತ್ತಿದೆ. ಈ ಆಧ್ಯಾತ್ಮಿಕ ಸ್ವಭಾವದ ಮೂಲಕ ಅವನು ಧರ್ಮಗಳನ್ನು ಪರೀಕ್ಷಿಸಿದರೆ, ಅದು ಭರತಖಂಡದ ಅರಣ್ಯಗಳಲ್ಲಿ ಬೋಧಿಸಿದ ಧರ್ಮವಾಗಲಿ ಅಥವಾ ಕ್ರೈಸ್ತ ದೇಶಕ್ಕೆ ಸೇರಿದ ಧರ್ಮವಾಗಲಿ, ಮೂಲದಲ್ಲಿ ಧರ್ಮಗಳೆಲ್ಲ ಒಂದೇ ಎಂಬುದು ಅವನಿಗೆ ಗೊತ್ತಾಗುವುದು. ಇದರಿಂದ ಧರ್ಮವು ಮಾನವ ಮನಸ್ಸಿನ ಅನಿವಾರ್ಯವಾದ ಆವಶ್ಯಕತೆ ಎಂಬುದು ಗೊತ್ತಾಗುತ್ತದೆ. ಒಂದು ಧರ್ಮದ ಸತ್ಯ ಇತರ ಧರ್ಮದ ಸತ್ಯಗಳ ಮೇಲೆ ನಿಂತಿದೆ. ನಿದರ್ಶನಕ್ಕೆ ನನ್ನ ಕೈಗಳಲ್ಲಿ ಆರು ಬೆರಳುಗಳಿದ್ದು ಇತರರ ಕೈಯಲ್ಲಿ ಐದೇ ಬೆರಳುಗಳಿದ್ದರೆ ಅದು ಅಸ್ವಾಭಾವಿಕ ಎನ್ನಬಹುದು. ಒಂದು ಧರ್ಮವು ಮಾತ್ರ ಸತ್ಯ, ಇತರ ಧರ್ಮಗಳು ಅಸತ್ಯ ಎಂಬ ವಾದಕ್ಕೆ ಇದೇ ಯುಕ್ತಿಯನ್ನು ಆರೋಪಿಸಬಹುದು. ಆರು ಬೆರಳಿನವನಂತೆ ಒಂದೇ ಧರ್ಮ ಸತ್ಯವಾಗಿರುವುದು ಅಸ್ವಾಭಾವಿಕವಾಗಿರುವುದು. ಆದಕಾರಣ ಒಂದು ಧರ್ಮ ಸತ್ಯವಾದರೆ ಇತರ ಧರ್ಮಗಳೂ ಕೂಡ ಸತ್ಯವಾಗಿರಲೇ ಬೇಕು. ಗೌಣ ವಿಷಯಗಳಲ್ಲಿ ಕೆಲವು ವ್ಯತ್ಯಾಸಗಳಿರಬಹುದು. ಆದರೆ ಸಾರದಲ್ಲಿ ಎಲ್ಲಾ ಒಂದೇ. ನನ್ನ ಕೈಯ ಐದು ಬೆರಳುಗಳು ಸತ್ಯವಾದರೆ ನಿಮ್ಮ ಕೈಯ ಐದು ಬೆರಳುಗಳೂ ಕೂಡ ಸತ್ಯವೆನ್ನುವುದನ್ನು ದೃಢಪಡಿಸುವುದು. ಮನುಷ್ಯ ಎಲ್ಲಿಯಾದರೂ ಇರಲಿ, ಅವನು ಒಂದು ನಂಬಿಕೆಯನ್ನು ರೂಢಿಸಬೇಕು, ಧಾರ್ಮಿಕ ಸ್ವಭಾವವನ್ನು ರೂಢಿಸಬೇಕು.

ಹಲವು ಧರ್ಮಗಳನ್ನು ಅಧ್ಯಯನ ಮಾಡಿದ ಮೇಲೆ ನನಗೆ ತೋರುವ ಮತ್ತೊಂದು ವಿಷಯವೆ, ಜೀವ ಮತ್ತು ಈಶ್ವರರಿಗೆ ಸಂಬಂಧಪಟ್ಟ ಭಾವನೆಯಲ್ಲಿ ಮೂರು ವಿಧಗಳಿರುವುವು. ಮೊದಲನೆಯದಾಗಿ ಎಲ್ಲಾ ಧರ್ಮಗಳೂ, ನಾಶವಾಗುವ ದೇಹದಲ್ಲಿ ಅದರಂತೆ ವಿಕಾರವಾಗದ, ಬದಲಾಗದ, ಸನಾತನವಾದ, ನಾಶವಾಗದ ಮತ್ತೊಂದು ಇದೆ ಎಂದು ಒಪ್ಪಿಕೊಳ್ಳುವುವು. ಅನಂತರ ಬಂದ ಕೆಲವು ಧರ್ಮಗಳು, ನಮ್ಮಲ್ಲಿ ನಾಶವಾಗದ ಯಾವುದೋ ಒಂದು ಅಂಶವಿದ್ದರೂ ಅದು ಯಾವುದೋ ಒಂದು ಕಾಲದಲ್ಲಿ ಪ್ರಾರಂಭವಾಯಿತು ಎನ್ನುವುವು. ಆದರೆ ಯಾವುದಕ್ಕೆ ಒಂದು ಪ್ರಾರಂಭವಿದೆಯೋ ಅದಕ್ಕೆ ಒಂದು ಅಂತ್ಯ ಇರಲೇಬೇಕಾಗುವುದು. ನಮ್ಮ ನೈಜ ಸ್ಥಿತಿಗೆ ಒಂದು ಆದಿ ಇರಲಿಲ್ಲ, ಅದಕ್ಕೆ ಒಂದು ಅಂತ್ಯ ಇರುವಂತೆ ಇಲ್ಲ. ನಮ್ಮೆಲ್ಲರನ್ನೂ ಮೀರಿದ, ಈ ಆದಿ ಅಂತ್ಯಗಳುಳ್ಳ ವಸ್ತುವನ್ನು ಮೀರಿದ ಮತ್ತೊಂದು ಆದ್ಯಂತರಹಿತ ವಸ್ತುವಿದೆ, ಅದೇ ದೇವರು. ಜನರು ಪ್ರಪಂಚದ ಮತ್ತು ಮಾನವನ ಆದಿಯನ್ನು ಕುರಿತು ಮಾತನಾಡುವರು. ಇಲ್ಲಿ ಆದಿ ಎಂದರೆ ಈ ಕಲ್ಪದ ಆದಿ ಎಂದು ಅರ್ಥ. ಅದಕ್ಕೆ ಎಲ್ಲೂ ಇಡಿಯ ಬ್ರಹ್ಮಾಂಡದ ಆದಿ ಎಂದು ಅರ್ಥವಿಲ್ಲ. ಸೃಷ್ಟಿಗೆ ಒಂದು ಪ್ರಾರಂಭವಿರುವುದು ಅಸಂಗತ. ನಿಮಗೆ ಯಾರಿಗೂ ಪ್ರಾರಂಭವನ್ನು\break ಊಹಿಸಲು ಸಾಧ್ಯವಿಲ್ಲ. ಯಾವುದಕ್ಕೆ ಒಂದು ಆದಿ ಇದೆಯೊ ಅದಕ್ಕೆ ಒಂದು ಅಂತ್ಯವಿರಲೇಬೇಕು. “ನಾನು ಇಲ್ಲದ ಕಾಲವೇ, ನೀವೂ ಅಥವಾ ಯಾರೇ ಆಗಲಿ ಮುಂದೆ ಇಲ್ಲದ ಕಾಲವೇ ಇರುವುದಿಲ್ಲ” ಎಂದು ಗೀತೆಯು ಸಾರುವುದು. ಎಲ್ಲಿ ಸೃಷ್ಟಿಯ ಪ್ರಾರಂಭವಾಯಿತು ಎಂದು ಹೇಳುವರೋ ಅಲ್ಲೆಲ್ಲ ಆ ಕಲ್ಪದ ಆದಿ ಎಂದು ಅರ್ಥ. ನಮ್ಮ ದೇಹವು ನಾಶವಾಗುವುದು, ಆದರೆ ಆತ್ಮವು ಎಂದಿಗೂ ನಾಶವಾಗುವುದಿಲ್ಲ.

ಈ ಆತ್ಮಭಾವನೆಯೊಡನೆ, ಆತ್ಮದ ಪೂರ್ಣ ಸ್ಥಿತಿಗೆ ಸಂಬಂಧಪಟ್ಟ ಇತರ ಕೆಲವು ಭಾವನೆಗಳು ಬರುವುವು. ಆತ್ಮವು ಸ್ವಭಾವತಃ ಪೂರ್ಣ. ಹಿಬ್ರೂಗಳ ನ್ಯೂ ಟೆಸ್ಟಮೆಂಟಿನಲ್ಲಿ ಮಾನವ ಆದಿಯಲ್ಲಿ ಪೂರ್ಣನಾಗಿದ್ದ ಎಂದು ಹೇಳಿದೆ. ಮಾನವ ತನ್ನ ಕರಗಳಿಂದಲೇ ಅಪವಿತ್ರನಾದ. ಆದರೆ ಅವನು ತನ್ನ ಹಿಂದಿನ ಸ್ವಭಾವವನ್ನು ಪಡೆಯಬೇಕಾಗಿದೆ. ತನ್ನ ಪರಿಶುದ್ಧತೆಯನ್ನು ಪಡೆಯಬೇಕಾಗಿದೆ. ಕೆಲವರು ಇದನ್ನು ರೂಪಕ ಕಥೆಗಳ ಮೂಲಕ, ಸಂಕೇತಗಳ ಮೂಲಕ ಹೇಳುತ್ತಾರೆ. ನಾವು ಇದನ್ನೆಲ್ಲ ಚೆನ್ನಾಗಿ ಪರೀಕ್ಷಿಸಿದರೆ, ಮಾನವನ ಆತ್ಮ ಸ್ವಭಾವತಃ ಪರಿಪೂರ್ಣವಾದುದೆಂದೂ, ಅವನು ಅದನ್ನು ಪುನಃ ಸಾಧಿಸಬೇಕೆಂದೂ ತಿಳಿಯುವುದು. ಇದು ಹೇಗೆ? ಭಗವತ್ ಸಾಕ್ಷಾತ್ಕಾರದಿಂದ. ಹಿಬ್ರೂ ಬೈಬಲಿನಲ್ಲಿ “ಮಗನ ಮೂಲಕ ಅಲ್ಲದೆ ಯಾರೂ ದೇವರನ್ನು ಪ್ರತ್ಯಕ್ಷ ನೋಡಲಾರರು'' ಎಂದಿದೆ. ಹಾಗೆಂದರೇನು? ಭಗವತ್ ಸಾಕ್ಷಾತ್ಕಾರವೇ ಜೀವನದ ಚರಮಗುರಿ. ನಾವು ತಂದೆಯೊಡನೆ ಐಕ್ಯವಾಗುವುದಕ್ಕೆ ಮುಂಚೆ ಪುತ್ರತ್ವವನ್ನು ಪಡೆಯಬೇಕು. ಮನುಷ್ಯನು ತನ್ನ ಕರ್ಮದಿಂದಲೇ ತನ್ನ ಪಾವಿತ್ರ್ಯವನ್ನು ಕಳೆದುಕೊಂಡನು ಎಂಬುದನ್ನು ನೆನಪಿನಲ್ಲಿಡಿ. ನಾವು ವ್ಯಥೆಗೀಡಾದರೆ ಅದು ನಮ್ಮ ಕರ್ಮಗಳಿಂದ. ಅದಕ್ಕಾಗಿ ದೇವರನ್ನು ದೂರಬೇಕಾಗಿಲ್ಲ.

ಇದರೊಡನೆ ಮತ್ತೊಂದು ಭಾವನೆಯು– ಅದನ್ನು ಯುರೋಪಿಯನ್ನರು ವಿಕೃತಗೊಳಿಸುವ ಮುನ್ನ –ವಿಶ್ವದಲ್ಲೆಲ್ಲಾ, ಪ್ರಚಲಿತವಾಗಿತ್ತು. ಅದೇ ಪುನರ್ಜನ್ಮ ಸಿದ್ದಾಂತ. ನಿಮ್ಮಲ್ಲಿ ಕೆಲವರು ಇದನ್ನು ಕೇಳಿ ಉಪೇಕ್ಷಿಸಿರಬಹುದು. ಪುನರ್ಜನ್ಮ ಸಿದ್ಧಾಂತ, ಮಾನವನ ಆತ್ಮ ಆದ್ಯಂತರಹಿತವಾದುದು ಎಂಬ ಸಿದ್ಧಾಂತಕ್ಕೆ ಸಂವಾದಿಯಾಗಿ ಇದ್ದೇ ಇರುವುದು. ಒಮ್ಮೆ ಯಾವುದು ನಾಶವಾಗುವುದೋ ಅದಕ್ಕೆ ಆದಿ ಮತ್ತೆಲ್ಲಿಯಾದರೂ ಇರಲೇಬೇಕು. ಒಮ್ಮೆ ಯಾವುದು ಪ್ರಾರಂಭವಾಗಿದೆಯೋ ಅದಕ್ಕೆ ಅಂತ್ಯ ಇರಲೇಬೇಕು. ಮಾನವಜೀವನದ ಆದಿ ಎಂಬ ಅಸಾಧ್ಯವಾದ ಆಸುರೀ ಅಸಾಧ್ಯತೆಯನ್ನು ನಾವು ನಂಬಲಾರೆವು. ಪುನರ್ಜನ್ಮ ಸಿದ್ಧಾಂತವು ಮಾನವನ ಆತ್ಮದ ಸ್ವಾತಂತ್ರ್ಯವನ್ನು ಘೋಷಿಸುವುದು. ಜೀವ ಎಲ್ಲೋ ಒಂದು ಕಾಲದಲ್ಲಿ ಪ್ರಾರಂಭವಾಯಿತು ಎಂದು ನಂಬೋಣ ಎನ್ನಿ. ಆಗ ಮಾನವನಲ್ಲಿರುವ ಪಾಪಕ್ಕೆಲ್ಲ ದೇವರು ಹೊಣೆಯಾಗಬೇಕಾಗುವುದು. ದಯಾಮಯನಾದ ತಂದೆ ಪ್ರಪಂಚದ ಪಾಪಕ್ಕೆಲ್ಲ ಗುರಿಯಾಗಬೇಕಾಗುವುದು! ಹೀಗೆ ಆದರೆ ಒಬ್ಬ ಏತಕ್ಕೆ ಮತ್ತೊಬ್ಬನಿಗಿಂತ ಹೆಚ್ಚು ವ್ಯಥೆಗೆ ಈಡಾಗಬೇಕು? ದಯಾಮಯನಾದ ತಂದೆಯಿಂದಲೇ ಇದೆಲ್ಲ ಬರುವ ಹಾಗೆ ಇದ್ದರೆ, ಅವನಲ್ಲಿ ಏತಕ್ಕೆ ಇಷ್ಟೊಂದು ಪಕ್ಷಪಾತ! ಕೋಟ್ಯಂತರ ಜನರು ಏತಕ್ಕೆ ದೌರ್ಜನ್ಯಕ್ಕೆ ತುತ್ತಾಗುತ್ತಾರೆ? ಜನರು ತಾವು ಕಾರಣಕರ್ತರಲ್ಲದಿದ್ದರೆ ಏತಕ್ಕೆ ಉಪವಾಸವಿರಬೇಕು? ಇದಕ್ಕೆ ಯಾರು ಕಾರಣಕರ್ತರು? ಜೀವಿಗಳು ಇದಕ್ಕೆ ಜವಾಬ್ದಾರರಲ್ಲದೆ ಇದ್ದರೆ ನಿಜವಾಗಿ ದೇವರೇ ಇದಕ್ಕೆ ಹೊಣೆಗಾರನಾಗಬೇಕು. ಆದಕಾರಣ ಇದಕ್ಕಿಂತ ಮೇಲಾದ ವಿವರಣೆಯೆ, ಒಬ್ಬ ತಾನು ಪಡುವ ಕಷ್ಟಕ್ಕೆಲ್ಲ ತಾನೇ ಹೊಣೆಗಾರನೆಂಬುದು. ನಾನೊಂದು ಕೆಲಸವನ್ನು ಪ್ರಾರಂಭ ಮಾಡಿದರೆ ಅದರ ಪರಿಣಾಮವನ್ನು ಅನುಭವಿಸಲೇಬೇಕು. ನಾನು ದುಃಖಕ್ಕೆ ಕಾರಣನಾದರೆ ಅದನ್ನು ನಿಲ್ಲಿಸುವುದಕ್ಕೂ ಕಾರಣನಾಗಬೇಕು. ಇದರಿಂದ ನಾವು ಸ್ವತಂತ್ರರು ಎಂಬುದನ್ನು ಸಾಧಿಸಿದಂತೆ ಆಯಿತು. ಅದೃಷ್ಟವೆಂಬುದಿಲ್ಲ. ಯಾವುದೂ ನಮ್ಮನ್ನು ಬಲಾತ್ಕರಿಸಲಾರದು. ನಾವು ಹಾಕಿದ ಗಂಟನ್ನು ನಾವೇ ಬಿಚ್ಚಲೂಬಹುದು.

ಈ ಸಿದ್ಧಾಂತಕ್ಕೆ ಸಂಬಂಧಪಟ್ಟ ಒಂದು ವಿಷಯವನ್ನು ನೀವು ಮನಸ್ಸಿಟ್ಟು ಕೇಳಬೇಕಾಗಿದೆ. ಏಕೆಂದರೆ ಇದು ಸ್ವಲ್ಪ ಜಟಿಲವಾದ ವಿಷಯ. ನಾವು ಜ್ಞಾನವನ್ನು ಅನುಭವದ ಮೂಲಕ ಗ್ರಹಿಸುವೆವು. ಅದೊಂದೆ ನಮಗೆ ಇರುವ ಮಾರ್ಗ. ನಾವು ಯಾವುದನ್ನು ಅನುಭವ ಎನ್ನುವೆವೊ ಅದು ನಮ್ಮ ಪ್ರಜ್ಞಾವಸ್ಥೆಯ ಸ್ತರದಲ್ಲಿದೆ. ಉದಾಹರಣೆಗೆ, ಒಬ್ಬ ಪಿಯಾನೊ ಮೇಲೆ ಒಂದು ರಾಗವನ್ನು ನುಡಿಸಿದ ಎಂದು ಭಾವಿಸಿ. ಅವನು ಬೆರಳನ್ನು ಆಯಾಯಾ ಮನೆಯ ಮೇಲೆ ಇಡುವನು. ಅವನು ಇದನ್ನು ಚೆನ್ನಾಗಿ ಅಭ್ಯಾಸವಾಗುವವರೆಗೆ ಮಾಡುತ್ತ ಹೋಗುವನು. ಅವನು ಅನಂತರ ಒಂದು ರಾಗವನ್ನು ನುಡಿಸುವಾಗ ಪಿಯಾನೊದಲ್ಲಿ ಪ್ರತಿಯೊಂದು ಮನೆಯ ಮೇಲೆ ಅಷ್ಟೊಂದು ಗಮನವನ್ನು ಕೊಡಬೇಕಾಗಿಲ್ಲ. ಇದರಂತೆಯೆ ನಾವು ಕೂಡ, ನಮ್ಮ ಈಗಿನ ಪ್ರಭಾವ ನಾವು ಹಿಂದೆ ಪ್ರಜ್ಞಾಪೂರ್ವಕವಾಗಿ ಮಾಡಿದುದರ ಪರಿಣಾಮ. ಮಗು ಕೆಲವು ಸ್ವಭಾವಗಳೊಡನೆ ಜನಿಸುವುದು. ಈ ಸ್ವಭಾವ ಎಲ್ಲಿಂದ ಬರುವುದು? ಯಾವ ಮಗುವೂ ಖಾಲಿ ಮನಸ್ಸಿನಿಂದ ಜಗತ್ತಿಗೆ ಬರುವುದಿಲ್ಲ. ಅದರಲ್ಲಿ ಆಗಲೇ ಏನೋ ಸಂಸ್ಕಾರಗಳಿರುತ್ತವೆ. ಹಿಂದಿನ ಕಾಲದ ಈಜಿಪ್ಟಿನ ಮತ್ತು ಗ್ರೀಸಿನ ಶಾಸ್ತ್ರಜ್ಞರು ಖಾಲಿ ಮನಸ್ಸಿನಿಂದ ಯಾವ ಮಗುವೂ ಪ್ರಪಂಚಕ್ಕೆ ಬರುವುದಿಲ್ಲ ಎಂದು ಬೋಧಿಸಿದರು. ಹಿಂದೆ ತಾನು ಇಚ್ಚಾಪೂರ್ವಕವಾಗಿ ಮಾಡಿದ ಕರ್ಮಗಳ ಪರಿಣಾಮವಾದ ನೂರಾರು ಸಂಸ್ಕಾರಗಳೊಡನೆ ಮಗು ಜನಿಸುವುದು. ಅದು ಈ ಜನ್ಮದಲ್ಲಿ ಇದನ್ನು ಗಳಿಸಲಿಲ್ಲ; ಅದು ಹಿಂದಿನ ಜನ್ಮದಲ್ಲಿ ಸಂಗ್ರಹಿಸಿರಬೇಕು. ಕಟ್ಟಾ ಭೌತವಾದಿಗಳೂ ಕೂಡ ಈ ಪ್ರವೃತ್ತಿಗಳು ಹಿಂದಿನ ಕ್ರಿಯೆಗಳ ಪರಿಣಾಮ ಎಂಬುದನ್ನು ಒಪ್ಪಿಕೊಳ್ಳಲೇಬೇಕಾಗುವುದು. ಆದರೆ ಅವರು ಇಂತಹ ಪ್ರಭಾವ ಆನುವಂಶಿಕವಾದುದು ಎನ್ನುವರು. ನಮ್ಮ ತಂದೆ ತಾತ ಅಜ್ಜಂದಿರೆಲ್ಲ ಈ ಆನುವಂಶಿಕ ನಿಯಾಮಾನುಸಾರ ಹುಟ್ಟಿ ಬಂದಿರುವರು. ಆನುವಂಶಿಕತೆಯೇ ಇದನ್ನು ವಿವರಿಸುವ ಹಾಗಿದ್ದರೆ ನಾವು ಆತ್ಮನಲ್ಲಿ ನಂಬಲೇಬೇಕಾಗಿಲ್ಲ. ಏಕೆಂದರೆ, ಶರೀರವೇ ಇದನ್ನೆಲ್ಲ ವಿವರಿಸುವುದು. ನಾಸ್ತಿಕತನಕ್ಕೆ ಮತ್ತು ಆಸ್ತಿಕತನಕ್ಕೆ ಸಂಬಂಧಪಟ್ಟ ವಾದಗಳನ್ನು ಕುರಿತು ನಾವು ಇಲ್ಲಿ ಚರ್ಚಿಸಬೇಕಾಗಿಲ್ಲ. ಇಲ್ಲಿಯವರೆಗೆ ಯಾರು ಒಂದು ಆತ್ಮನನ್ನು ನಂಬುವರೋ ಅವರ ದಾರಿ ಸ್ಪಷ್ಟವಾಗಿದೆ. ತರ್ಕಬದ್ಧವಾದ ಒಂದು ನಿರ್ಣಯಕ್ಕೆ ಬರಬೇಕಾದರೆ ನಮಗೆ ಪೂರ್ವಜನ್ಮ ಇತ್ತು ಎಂಬುದನ್ನು ನಂಬಬೇಕಾಗುತ್ತದೆ. ಇದು ಹಿಂದಿನ ಮಹಾತತ್ತ್ವಜ್ಞಾನಿಗಳ, ಹಿಂದಿನ ಮತ್ತು ಈಗಿನ ಋಷಿಗಳ ಅಭಿಮತ. ಯೆಹೂದ್ಯರು ಇದನ್ನು ನಂಬಿದ್ದರು. ಏಸುಕ್ರಿಸ್ತ ಇದನ್ನು ನಂಬಿದ್ದನು. ಅವನು ಬೈಬಲ್ಲಿನಲ್ಲಿ ಎಬ್ರಹಾಮನಿಗೆ ಮುನ್ನ ನಾನಿದ್ದೆ ಎನ್ನುವನು. ಮತ್ತೊಂದು ಕಡೆ, ಅದೇ ಎಲಿಯಾಸನೆ ಈಗ ಬಂದಿರುವುದು\break ಎನ್ನುವನು.

ಭಿನ್ನ ಭಿನ್ನ ರಾಷ್ಟ್ರಗಳಲ್ಲಿ, ಬೇರೆ ಬೇರೆ ಸನ್ನಿವೇಶಗಳಲ್ಲಿ ವೃದ್ಧಿಗೊಂಡ ಧರ್ಮಗಳಿಗೆಲ್ಲ ಮೂಲ ಏಷ್ಯಾಖಂಡ. ಏಷ್ಯನರು ಇದನ್ನು ಚೆನ್ನಾಗಿ ಅರ್ಥ ಮಾಡಿಕೊಂಡರು. ಧರ್ಮಗಳು ತಮ್ಮ ಮೂಲಸ್ಥಾನದಿಂದ ಬಂದ ಮೇಲೆ ರೂಢಿಯಲ್ಲಿದ್ದ ತಪ್ಪು ಅಭಿಪ್ರಾಯಗಳೊಂದಿಗೆ ಮಿಶ್ರವಾಗಿ ಹೋದವು. ಕೈಸ್ತ ಧರ್ಮದ ಉದಾತ್ತವಾದ ಭವ್ಯಭಾವನೆಗಳನ್ನು ಯೂರೋಪಿಯನರು ಎಂದಿಗೂ ಅರ್ಥ ಮಾಡಿಕೊಳ್ಳಲಿಲ್ಲ. ಏಕೆಂದರೆ ಬೈಬಲಿನಲ್ಲಿ ಬರುವ ಭಾವನೆಗಳು, ರೂಪಕಗಳು ಅವರಿಗೆ ದೂರವಾಗಿದ್ದವು. ಉದಾಹರಣೆಗೆ ಮಡೋನಳ (ಕ್ರಿಸ್ತನ ತಾಯಿ) ಚಿತ್ರಗಳನ್ನು ತೆಗೆದುಕೊಳ್ಳಿ. ಪ್ರತಿಯೊಬ್ಬ ಕಲಾವಿದನೂ ಮಡೋನಳನ್ನು ತನ್ನ ಪೂರ್ವಕಲ್ಪಿತ ರೀತಿಯಲ್ಲಿ ಚಿತ್ರಿಸುವನು. ನಾನು ಇದುವರೆಗೆ ಜೀಸಸ್ಸಿನ ಕೊನೆಯ ಊಟದ ನೂರಾರು ಚಿತ್ರಗಳನ್ನು ನೋಡಿರುವೆನು. ಅದರಲ್ಲಿ ಅವನನ್ನು ಒಂದು ಮೇಜಿನ ಹಿಂದೆ ಕುಳಿತಂತೆ ಚಿತ್ರಿಸಿರುವರು. ಕ್ರಿಸ್ತ ಎಂದಿಗೂ ಮೇಜಿನ ಹತ್ತಿರ ಕುಳಿತವನಲ್ಲ. ಇತರರೊಡನೆ ಅವನು ನೆಲದ ಮೇಲೆ ಕುಳಿತುಕೊಳ್ಳುತ್ತಿದ್ದನು. ರೊಟ್ಟಿಯನ್ನು ಅದ್ದಿ ತಿನ್ನುವುದಕ್ಕೆ ಒಂದು ಬೋಗುಣಿ ಇರುತ್ತಿತ್ತು. ಆ ರೊಟ್ಟಿ ಈಗಿನ ಬ್ರೆಡ್ ಆಗಿರಲಿಲ್ಲ. ಒಂದು ರಾಷ್ಟ್ರಕ್ಕೆ ಇತರ ಜನಾಂಗಗಳ ಆಚಾರ ವ್ಯವಹಾರಗಳನ್ನು ಊಹಿಸುವುದು ಬಹಳ ಕಷ್ಟ. ಹೀಗಿರುವಾಗ ನೂರಾರು ವರುಷಗಳು ಗ್ರೀಕ್ ರೋಮನ್ ಮುಂತಾದವರ ಪ್ರಭಾವಕ್ಕೆ ಒಳಗಾದ ಯೂರೋಪಿಯನ್ನರಿಗೆ ಸಹಸ್ರಾರು ವರುಷಗಳ ಹಿಂದೆ ಇದ್ದ ಯೆಹೂದ್ಯರ ಆಚಾರವ್ಯವಹಾರಗಳನ್ನು ಊಹಿಸುವುದು ಎಷ್ಟು ಕಷ್ಟವಾಗಿರಬೇಕು! ನೂರಾರು ಕಥೆಗಳಿಂದ ಮತ್ತು ಪುರಾಣಗಳಿಂದ ಆವರಿಸಲ್ಪಟ್ಟ ಯೂರೋಪಿಯನ್ನರಿಗೆ ಜೀಸಸ್ಸಿನ ಅತಿ ಭವ್ಯವಾದ ಕ್ರೈಸ್ತಧರ್ಮ ಅಷ್ಟು ಕಡಮೆ ತಿಳಿದಿರುವುದರಲ್ಲಿ ಆಶ್ಚರ್ಯವೇನು? ಈಗ ಅದೊಂದು ಆಧುನಿಕ ವ್ಯಾಪಾರದ ಧರ್ಮವಾಗಿರುವುದರಲ್ಲಿ ಕೌತುಕವೇನೂ ಇಲ್ಲ.

ಪುನಃ ನಮ್ಮ ವಿಷಯಕ್ಕೆ ಬರೋಣ. ಎಲ್ಲಾ ಧರ್ಮಗಳೂ ಆತ್ಮವು ಆದ್ಯಂತ ರಹಿತವೆಂದೂ, ಅದರ ಕಾಂತಿ ಈಗ ಕುಗ್ಗಿದೆಯೆಂದೂ, ಹಿಂದಿನ ಕಾಂತಿಯನ್ನು ಭಗವತ್ಸಾಕ್ಷಾತ್ಕಾರದಿಂದ ಪಡೆಯಬೇಕೆಂದೂ ಸಾರುವುವು. ಬೇರೆ ಬೇರೆ ಧರ್ಮಗಳ ಪ್ರಕಾರ ಭಗವಂತ ಎಂದರೇನು? ಆದಿಯಲ್ಲಿ ಭಗವತ್ ಭಾವನೆ ಅಸ್ಪಷ್ಟವಾಗಿತ್ತು. ಅತಿ ಪುರಾತನ ದೇಶಗಳಲ್ಲಿ ಸೂರ್ಯ, ಭೂಮಿ, ಬೆಂಕಿ, ನೀರು ಮುಂತಾದ ಹಲವು ದೇವರುಗಳಿದ್ದರು. ಆದಿಯ ಯೆಹೂದ್ಯ ಜನಾಂಗದಲ್ಲಿ ಈ ದೇವರುಗಳು ಒಬ್ಬರೊಡನೊಬ್ಬರು ಉಗ್ರವಾಗಿ ಕಾದಾಡುತ್ತಿದ್ದರು. ಅನಂತರ ಯೆಹೂದ್ಯರು ಮತ್ತು ಬ್ಯಾಬಿಲೋನಿಯಾದವರು ಇಲೋಹಿಂನನ್ನು ಪೂಜಿಸುವುದನ್ನು ನೋಡುತ್ತೇವೆ. ಇದಾದ ಮೇಲೆ ಒಬ್ಬ ದೇವರು ಸರ್ವ ಶ್ರೇಷ್ಠನಾಗಿರುವುದನ್ನು ನೋಡುತ್ತೇವೆ. ಆದರೆ ಬೇರೆ ಬೇರೆ ಪಂಗಡಗಳಲ್ಲಿ ಬೇರೆ ಬೇರೆ ಭಾವನೆಗಳು ಇರುವುವು. ಅವರಲ್ಲಿ ಪ್ರತಿಯೊಬ್ಬರೂ ತಮ್ಮ ದೇವರೇ ಶ್ರೇಷ್ಠ ಎಂದು ಸಾಧಿಸಲು ಹೊರಟರು. ಹೋರಾಡಿ ಇದನ್ನು ನಿರ್ಧರಿಸಲು ಯತ್ನಿಸಿದರು. ಯಾರು ಚೆನ್ನಾಗಿ ಕಾದಾಡುವರೋ ಅವರ ದೇವರೇ ಶ್ರೇಷ್ಠ ಎಂದು ತೀರ್ಮಾನಿಸಿದರು. ಆ ಜನಾಂಗಗಳು ಇನ್ನೂ ಸುಸಂಸ್ಕೃತರಾಗಿರಲಿಲ್ಲ. ಆದರೆ ಕ್ರಮೇಣ ಉತ್ತಮ ಭಾವನೆಗಳು ಅವರಲ್ಲಿ ಬೇರೂರಿದವು. ಪುರಾತನ ಭಾವನೆಗಳೆಲ್ಲ ಹಾರಿಹೋದವು, ಮರವಿನ ಕಸದ ಬುಟ್ಟಿಗೆ ಬಿದ್ದವು. ಈ\break ಧರ್ಮಗಳೆಲ್ಲ ಶತಶತಮಾನಗಳಿಂದ ಬೆಳೆದು ಬಂದವುಗಳು. ಯಾವುದೂ ಆಕಾಶದಿಂದ ಏನೂ ಉದುರಲಿಲ್ಲ. ಪ್ರತಿಯೊಂದು ಧರ್ಮವೂ ಕ್ರಮೇಣ ಬೆಳೆಯುತ್ತ ಹೋಗಿದೆ. ಅನಂತರ ಸರ್ವಶಕ್ತನಾದ ಸರ್ವಜ್ಞನಾದ ಸರ್ವೇಶ್ವರನ ಭಾವನೆ ಬರುವುದು. ಈ ದೇವರು ಪ್ರಪಂಚಕ್ಕೆ ಅತೀತನು, ಸ್ವರ್ಗದಲ್ಲಿ ನೆಲಸಿರುವನು ಎಂದು ಭಾವಿಸಿದರು. ಅವನಿಗೆ ಒಂದು ಸ್ಥೂಲ ಭಾವನೆಯನ್ನು ಆರೋಪ ಮಾಡುವರು. ಅವನಿಗೆ ಎಡಬಲಗಳಿವೆ, ಕೈಯಲ್ಲಿ ಒಂದು ಪಕ್ಷಿ ಇತ್ಯಾದಿಗಳಿವೆ. ಆದರೆ ಒಂದು ಸಂಗತಿಯನ್ನು ನಾವು ಇಲ್ಲಿ ನೋಡುತ್ತೇವೆ: ಅದೇ ಆಯಾಯಾ ಬುಡಕಟ್ಟಿನ ದೇವರುಗಳು ಎಲ್ಲ ಕಾಲಕ್ಕೂ ಮಾಯವಾಗಿ ಹೋಗಿ, ಅದರ ಬದಲು ದೇವದೇವನಾದ ಒಬ್ಬ ಈಶ್ವರ ಬರುವನು. ಆದರೂ ಅವನು ಸೃಷ್ಟಿಗೆ ಅತೀತನಾದ ದೇವರು, ಅವನನ್ನು ಸಮೀಪಿಸುವುದಕ್ಕೆ ಆಗುವುದಿಲ್ಲ. ಯಾವುದೂ ಅವನ ಹತ್ತಿರ ಬರಲಾರದು. ಕ್ರಮೇಣ ಈ ಭಾವನೆಯೂ ಬದಲಾಗಿ ಕೊನೆಗೆ ಸೃಷ್ಟಿಯಲ್ಲೆಲ್ಲ ಅಂತರ್ಯಾಮಿಯಾದ ಈಶ್ವರನ ಭಾವನೆಯು ಬರುವುದು.

ಹೊಸ ಟೆಸ್ಟಮೆಂಟಿನಲ್ಲಿ “ಸ್ವರ್ಗದಲ್ಲಿರುವ ನಮ್ಮ ತಂದೆ, '' ಮನುಷ್ಯನಿಂದ ಬೇರೆಯಾಗಿ ಸ್ವರ್ಗದಲ್ಲಿ ವಿಹರಿಸುವ ದೇವರು ಎಂದು ಬೋಧಿಸಿದೆ. ನಾವು ಮರ್ತ್ಯಲೋಕದಲ್ಲಿರುವೆವು, ಅವನು ಸ್ವರ್ಗದಲ್ಲಿರುವನು. ಅವನು ಸೃಷ್ಟಿಯಲ್ಲಿ ಅಂತರ್ಯಾಮಿಯಾದ ದೇವರು ಎಂಬ ಭಾವನೆ ಅನಂತರ ಬರುವುದು. ಅವನು ಸ್ವರ್ಗದಲ್ಲಿ ಮಾತ್ರ ದೇವರು ಅಲ್ಲ, ಭೂಮಿಯಲ್ಲೂ ದೇವರು; ಅವನೇ ನಮ್ಮ ಅಂತರಾಳದಲ್ಲಿರುವವನು. ಹಿಂದೂ ತತ್ತ್ವಶಾಸ್ತ್ರದಲ್ಲಿ ಇವುಗಳನ್ನು ಹೋಲುವ ಭಾವನೆಗಳೆಲ್ಲ ಇವೆ. ಆದರೆ ಇಲ್ಲೇ ನಿಲ್ಲುವುದಿಲ್ಲ. ಒಂದು ಅದ್ವೈತ ಸ್ಥಿತಿಯಿದೆ, ಅಲ್ಲಿ ತಾನು ಪೂಜಿಸುತ್ತಿದ್ದ ದೇವರು, ಸ್ವರ್ಗಲೋಕದ ಮತ್ತು ಮರ್ತ್ಯಲೋಕದ ತಂದೆ ಮಾತ್ರವಲ್ಲ, ತಾನೂ ಆ ತಂದೆಯೂ ಒಂದೇ ಎಂಬ ಭಾವವೂ ಇದೆ. ಅವನು ತನ್ನ ಅಂತರಾಳದಲ್ಲಿ ತಾನೇ ಆ ದೇವರು, ಆ ದೇವರ ಒಂದು ಕೆಳಗಿನ ಆವಿರ್ಭಾವ ಎಂದು ಅರಿಯುತ್ತಾನೆ. ನನ್ನಲ್ಲಿ ಸತ್ಯವಾಗಿರುವುದೆಲ್ಲ ಅವನೆ, ಅವನಲ್ಲಿ ಸತ್ಯವಾಗಿರುವುದೆಲ್ಲ ನಾನೆ. ದೇವರು ಮತ್ತು ಮಾನವರಿಗೆ ಇರುವ ವ್ಯತ್ಯಾಸ ಹೀಗೆ ಮಾಯವಾಗುವುದು. ದೇವರನ್ನು ಅರಿತರೆ ಸ್ವರ್ಗ ನಮ್ಮಲ್ಲಿಯೇ ಹೇಗೆ ಇರುವುದು ಎಂಬುದನ್ನು ಈಗ ತಿಳಿದಂತೆ ಆಯಿತು.

ಮೊದಲ ಅಥವಾ ದ್ವೈತದ ಹಂತದಲ್ಲಿ ಮಾನವ ತಾನು ನಾಣಿ, ಶೀನ, ವೆಂಕ ಎಂಬ ಸಣ್ಣ ವ್ಯಕ್ತಿಯೆಂದು ತಿಳಿದುಕೊಳ್ಳುವನು. ತಾನು ಕೊನೆಯತನಕ ಹಾಗೆಯೇ ಇರುತ್ತೇನೆ, ಇನ್ನೇನೂ ಆಗುವುದಿಲ್ಲ ಎನ್ನುವನು. ಒಬ್ಬ ಕೊಲೆಪಾತಕಿ ಬಂದು ತಾನು ಎಂದೆಂದಿಗೂ ಕೊಲೆಪಾತಕಿಯಾಗಿಯೇ ಇರುತ್ತೇನೆ ಎಂದಂತೆ ಇದು. ಆದರೆ ಕಾಲ ಕಳೆದಂತೆ ಅವನು ಮಾಯವಾಗಿ ತನ್ನ ಆದಿ ಸ್ಥಿತಿಗೆ ಹೋಗುತ್ತಾನೆ.

\newpage

ಪವಿತ್ರಾತ್ಮರೇ ಧನ್ಯರು. ಏಕೆಂದರೆ ಅವರಿಗೆಯೇ ಭಗವತ್ ದರ್ಶನ ಲಭಿಸುವುದು. ನಾವು ದೇವರನ್ನು ನೋಡಬಲ್ಲೆವೆ? ಅದೇನೂ ಸಾಧ್ಯವಿಲ್ಲ. ನಾವು ದೇವರನ್ನು ಅರಿಯಬಲ್ಲೆವೆ? ಅದೂ ಸಾಧ್ಯವಿಲ್ಲ. ದೇವರನ್ನು ಅರಿತರೆ ಅವನು ಇನ್ನು ಮೇಲೆ ದೇವರೇ ಆಗಿ ಉಳಿಯುವುದಿಲ್ಲ. ಜ್ಞಾನವೇ ಒಂದು ಮಿತಿ. ಆದರೆ ನಾನು ಮತ್ತು ನನ್ನ ದೇವರು ಒಂದೆ. ಈ ಸತ್ಯವನ್ನು ನಾನು ನನ್ನಲ್ಲಿ ಅರಿಯುತ್ತೇನೆ. ಕೆಲವು ಧರ್ಮಗಳಲ್ಲಿ ಇದನ್ನು ವ್ಯಕ್ತಪಡಿಸುವರು, ಮತ್ತೆ ಕೆಲವದರಲ್ಲಿ ಇದನ್ನು ಸೂಚಿಸುವರು, ಮತ್ತೆ ಕೆಲವದರಲ್ಲಿ ಈ ಭಾವನೆಯನ್ನು ಬಿಟ್ಟು ಬಿಟ್ಟಿರುವರು. ಕ್ರಿಸ್ತನ ಉಪದೇಶಗಳನ್ನು ಈ ದೇಶದಲ್ಲಿ ಅರ್ಥ ಮಾಡಿಕೊಂಡಿಲ್ಲವೆಂದೇ ಹೇಳಬಹುದು. ನೀವು ನನ್ನನ್ನು ಕ್ಷಮಿಸುವಿರಾದರೆ, ಎಂದೂ ಅದನ್ನು ಚೆನ್ನಾಗಿ ಅರ್ಥಮಾಡಿಕೊಂಡಿಲ್ಲ ಎಂದೇ ಹೇಳಬಲ್ಲೆ.

ಪರಿಪೂರ್ಣನಾಗಬೇಕಾದರೆ, ಪವಿತ್ರಾತ್ಮನಾಗಬೇಕಾದರೆ ಜೀವಿಯು ಬೆಳವಣಿಗೆಯ ಹಲವು ಹಂತಗಳ ಮೂಲಕ ಸಾಗಿಹೋಗಬೇಕು. ವಿಭಿನ್ನ ಧರ್ಮಗಳ ಮೂಲದಲ್ಲೆಲ್ಲಾ ಒಂದೇ ಸಾಮಾನ್ಯ ಭಾವನೆ ಇದೆ. ಜೀಸಸ್, “ಸ್ವರ್ಗ ನಿನ್ನಲ್ಲಿಯೇ ಇದೆ" ಎನ್ನುತ್ತಾನೆ; ಪುನಃ “ಸ್ವರ್ಗದಲ್ಲಿರುವ ತಂದೆ'' ಎನ್ನುತ್ತಾನೆ. ಇವೆರಡಕ್ಕೂ ಹೇಗೆ ಹೊಂದಾಣಿಕೆ ಮಾಡಿಸುವಿರಿ? ಈ ರೀತಿಯಲ್ಲಿ: ಸ್ವರ್ಗದಲ್ಲಿರುವ ತಂದೆ ಎಂದು ಅವನು ಹೇಳಿದಾಗ ಧಾರ್ಮಿಕ ಜೀವನದಲ್ಲಿ ಅಷ್ಟು ಮುಂದುವರಿಯದ ಜನಸಾಮಾನ್ಯರ ಹತ್ತಿರ ಮಾತನಾಡುತ್ತಿದ್ದನು. ಅವರ ಭಾಷೆಯಲ್ಲಿಯೇ ಅವನು ಮಾತನಾಡಬೇಕಾಗಿತ್ತು. ಜನಸಾಮಾನ್ಯರಿಗೆ ತಮ್ಮ ಇಂದ್ರಿಯಗಳು ಗ್ರಹಿಸಬಲ್ಲ ಸ್ಥೂಲ ಭಾವನೆಗಳು ಬೇಕಾಗಿವೆ. ಒಬ್ಬನು ಪ್ರಪಂಚದಲ್ಲೆಲ್ಲಾ ಅತ್ಯಂತ ದೊಡ್ಡ ತತ್ತ್ವಜ್ಞಾನಿಯಾಗಿರಬಹುದು. ಆದರೂ ಅವನು ಧಾರ್ಮಿಕ ಜೀವನದಲ್ಲಿ ಏನೂ ಅರಿಯದ ಹಸುಳೆಯಾಗಿರಬಹುದು. ಒಬ್ಬನು ಧಾರ್ಮಿಕ ಜೀವನದಲ್ಲಿ ಪರಿಣತನಾದಮೇಲೆ ಸ್ವರ್ಗ ತನ್ನಲ್ಲಿಯೇ ಇದೆ ಎಂದರೆ ಏನು ಎಂಬುದನ್ನು ಅರ್ಥಮಾಡಿಕೊಳ್ಳಬಹುದು. ಅದೇ ನಿಜವಾದ ಮಾನಸಿಕ ಸ್ವರ್ಗ. ಪ್ರತಿಯೊಂದು ಧರ್ಮದಲ್ಲಿರುವ ಭಿನ್ನಾಭಿಪ್ರಾಯಗಳು, ಸಮಸ್ಯೆಗಳು, ಆತ್ಮವಿಕಾಸ ದಲ್ಲಿರುವ ಹಲವು ಹಂತಗಳು. ಆದಕಾರಣ ನಮಗೆ ಯಾರ ಧರ್ಮವನ್ನೂ ದೂರುವುದಕ್ಕೆ ಅಧಿಕಾರವಿಲ್ಲ. ಕೆಲವು ಸ್ಥಿತಿಗಳಲ್ಲಿ ಬಾಹ್ಯ ಆಕಾರಗಳು, ಸಂಕೇತಗಳು ಆವಶ್ಯಕ. ಆ ಸ್ಥಿತಿಗಳಲ್ಲಿರುವ ಜೀವಿಗಳು ಅರ್ಥಮಾಡಿಕೊಳ್ಳಬಲ್ಲಂತಹ ಭಾಷೆಯೇ ಅದು.

ಅನಂತರ ನಾನು ನಿಮಗೆ ಹೇಳಬೇಕೆಂದಿರುವ ಭಾವನೆ, ಧರ್ಮವೆಂದರೆ ಅದು ಕೇವಲ ಸಿದ್ದಾಂತಗಳಲ್ಲ ಎಂಬುದು. ನೀವು ಏನನ್ನು ಓದುವಿರೋ ಅಥವಾ ಏನನ್ನು ನಂಬುವಿರೋ ಅದಲ್ಲ ಮುಖ್ಯ, ನೀವು ಏನನ್ನು ಸಾಕ್ಷಾತ್ಕಾರಮಾಡಿಕೊಂಡಿರುವಿರಿ ಅದು ಮುಖ್ಯ. “ಯಾರು ಪರಿಶುದ್ಧಾತ್ಮರೊ, ಅವರೇ ದೇವರನ್ನು ನೋಡುವರು.” ಹೌದು, ಈ ಜೀವನದಲ್ಲಿ ಅವರು ದೇವರನ್ನು ನೋಡುವರು. ಅದೇ ಮೋಕ್ಷ. ಕೆಲವು ಮಂತ್ರಗಳ ಉಚ್ಚಾರಣೆಯಿಂದ ಇದು ಸಿದ್ಧಿಸುವುದು ಎಂದು ಕೆಲವರು ಹೇಳುವರು. ಆದರೆ ಯಾವ ಮಹಾ ಗುರುವೂ ಬಾಹ್ಯಾಚಾರಗಳು ಮುಕ್ತಿಗೆ ಆವಶ್ಯಕ ಎಂದು ಎಂದಿಗೂ ಸಾರಲಿಲ್ಲ. ಮುಕ್ತಿಯನ್ನು ಸಾಧಿಸುವ ಶಕ್ತಿ ನಮ್ಮಲ್ಲಿಯೇ ಇರುವುದು. ನಾವು ದೇವರಲ್ಲಿಯೇ ಜೀವಿಸುತ್ತಿರುವೆವು, ಚಲಿಸುತ್ತಿರುವೆವು. ದೇವತಾ ಸಿದ್ಧಾಂತಗಳಿಂದಲೂ ಮತಗಳಿಂದಲೂ ಪ್ರಯೋಜನ ಇದೆ ಎಂಬುದು ನಿಜ. ಆದರೆ ಅವು ಕೇವಲ ಮಕ್ಕಳಿಗೆ ಮಾತ್ರ, ಅವು ತಾತ್ಕಾಲಿಕ. ಗ್ರಂಥ ಎಂದಿಗೂ ಧರ್ಮಗಳನ್ನು ಸೃಷ್ಟಿಸಲಾರದು, ಧರ್ಮಗಳು ಮಾತ್ರ ಗ್ರಂಥವನ್ನು ಸೃಷ್ಟಿಸಬಲ್ಲವು. ನಾವು ಇದನ್ನು ಮರೆಯಕೂಡದು. ಯಾವ ಶಾಸ್ತ್ರವೂ ದೇವರನ್ನು ಸೃಷ್ಟಿಸಲಿಲ್ಲ. ಆದರೆ ದೇವರೇ ಎಲ್ಲಾ ಪವಿತ್ರ ಗ್ರಂಥಗಳ ಪ್ರೇರಕ. ಯಾವ ಶಾಸ್ತ್ರವೂ ಒಂದು ಜೀವಿಯನ್ನು ಸೃಷ್ಟಿಸಲಿಲ್ಲ. ನಾವು ಇದನ್ನು ಎಂದಿಗೂ ಮರೆಯಕೂಡದು.\break ಧರ್ಮಗಳ ಗುರಿಯೇ ಆತ್ಮನಲ್ಲಿ ಪರಮಾತ್ಮನನ್ನು ಕಾಣುವುದು. ಇದೇ ಏಕ ಮಾತ್ರ ವಿಶ್ವಧರ್ಮ. ಎಲ್ಲಾ ಧರ್ಮಗಳಲ್ಲಿಯೂ ಸರ್ವವ್ಯಾಪಿಯಾದ ಸತ್ಯ ಒಂದಿದ್ದರೆ ಅದೇ ಭಗವತ್ ಸಾಕ್ಷಾತ್ಕಾರ ಎನ್ನುತ್ತೇನೆ. ಆದರ್ಶಗಳು, ಮಾರ್ಗಗಳು ಬದಲಾಗಬಹುದು. ಆದರೆ ಇದೇ ಕೇಂದ್ರ ಭಾವನೆ. ಒಂದು ವೃತ್ತದಲ್ಲಿ ವ್ಯಾಸಾರ್ಧರೇಖೆಗಳು ಎಷ್ಟೋ ಇರಬಹುದು. ಆದರೆ ಅವೆಲ್ಲ ಒಂದು ಕೇಂದ್ರದಲ್ಲಿ ಸಂಧಿಸುವುವು. ಇದೇ ಭಗವತ್ ಸಾಕ್ಷಾತ್ಕಾರ. ಇದು ಇಂದ್ರಿಯ ಪ್ರಪಂಚದ ಹಿಂದೆ, ತುದಿಮೊದಲಿಲ್ಲದ ಆಮೋದ ಪ್ರಮೋದಗಳ ಹಿಂದೆ, ಈ ಭ್ರಾಂತಿಯುಕ್ತ ಜಗತ್ತಿನ ಹಿಂದೆ, ಸ್ವಾರ್ಥದ ಹಿಂದೆ ಇದೆ. ಎಲ್ಲ ಶಾಸ್ತ್ರಗಳ, ಎಲ್ಲ ಮತತತ್ತ್ವಗಳ ಹಿಂದೆ ಇರುವುದೇ ನಿನ್ನ ಹೃದಯಾಂತರಾಳದಲ್ಲಿ ಭಗವಂತನನ್ನು ಸಾಕ್ಷಾತ್ಕಾರ ಮಾಡಿಕೊಳ್ಳಬೇಕು ಎಂಬುದು. ಒಬ್ಬನು ಪ್ರಪಂಚದಲ್ಲಿರುವ ಚರ್ಚುಗಳನ್ನೆಲ್ಲಾ ನಂಬಬಹುದು, ಪ್ರಪಂಚದ ಪುಣ್ಯನದಿಗಳಲ್ಲೆಲ್ಲ ಅವನು ಮಿಂದಿರಬಹುದು, ಆದರೂ ಅವನಿಗೆ ಭಗವಂತನ ಅನುಭವ ಇಲ್ಲದೇ ಇದ್ದರೆ ಅವನನ್ನು ಶುದ್ದ ನಾಸ್ತಿಕರ ಗುಂಪಿಗೆ ಸೇರಿಸುತ್ತೇನೆ. ಒಬ್ಬ ಯಾವ ಚರ್ಚಿನೊಳಗೂ ಅಥವಾ ಮಸೀದಿಯೊಳಗೂ ಕಾಲಿಡದೆ ಇರಬಹುದು; ಯಾವ ವ್ರತಾಚಾರಗಳನ್ನೂ ಮಾಡದೆ ಇರಬಹುದು; ಆದರೆ ಅವನು ಭಗವಂತನನ್ನು ತನ್ನ ಹೃದಯದಲ್ಲಿ ಅನುಭವಿಸಿ ಅದರಿಂದ ಅವನು ಪ್ರಪಂಚದ ಕ್ಷಣಿಕ ಸುಖದ ಆಕರ್ಷಣೆಯಿಂದ ಪಾರಾಗಿದ್ದರೆ, ಅವನೇ ಪವಿತ್ರಾತ್ಮ, ಅವನೇ ಮಹಾತ್ಮ. ನೀವು ಅವನನ್ನು ಯಾವ ಹೆಸರಿನಿಂದ ಬೇಕಾದರೂ ಕರೆಯಿರಿ, ಯಾವಾಗ ಒಬ್ಬ ಎದ್ದು ನಿಂತು, ತಾನು ಸರಿ ಅಥವಾ ತನ್ನ ಚರ್ಚು ಸರಿ, ಉಳಿದವರೆಲ್ಲ ತಪ್ಪು ಎನ್ನುವನೋ ಆಗ ಅವನೇ ತಪ್ಪುಗಳ ಒಂದು ಕಂತೆ ಆಗಿರುವನು. ಇತರ ಧರ್ಮಗಳ ಸತ್ಯಗಳ ಮೇಲೆ ತನ್ನ ಧರ್ಮದ ಸತ್ಯವೂ ನಿಂತಿದೆ ಎಂಬುದು ಅವನಿಗೆ ಅರಿಯದು. ಇಡಿಯ ಮಾನವಕೋಟಿಗೆ ಪ್ರೀತಿಯನ್ನು ಮತ್ತು ಅನುಕಂಪವನ್ನು ತೋರಿಸಬೇಕು. ಅದೇ ನಿಜವಾದ ಧರ್ಮದ ಪ್ರಮಾಣ. ಮಾನವರೆಲ್ಲ ನಿಜವಾಗಿ ಸಹೋದರರು ಎಂದು ಕೇವಲ ಉದ್ವೇಗವಶರಾಗಿ ಹೇಳುವುದಲ್ಲ. ಒಬ್ಬನು ಮಾನವ ಜೀವನದ ಏಕತೆಯನ್ನು ಅನುಭವಿಸಬೇಕು. ಎಲ್ಲಿಯವರೆಗೆ ಧರ್ಮಗಳು ಇತರರನ್ನು ಬಹಿಷ್ಕರಿಸುವುದಿಲ್ಲವೋ ಅಲ್ಲಿಯವರೆಗೆ ಎಲ್ಲಾ ಪಂಗಡಗಳೂ, ಮತಗಳೂ ನನ್ನವೆಂದು ಭಾವಿಸುತ್ತೇನೆ. ಅಲ್ಲಿಯವರೆಗೆ ಅವೆಲ್ಲ ಸೊಗಸಾಗಿವೆ. ಅವೆಲ್ಲ ನಿಜವಾದ ಧರ್ಮಕ್ಕೆ ಮಾನವನನ್ನು ಒಯ್ಯಲು ಸಹಾಯಮಾಡುತ್ತವೆ. ನಾನು ಜೊತೆಗೆ ಇದನ್ನೂ\break ಸೇರಿಸುತ್ತೇನೆ; ಒಂದು ಚರ್ಚಿನಲ್ಲಿ ಹುಟ್ಟುವುದು ಒಳ್ಳೆಯದು, ಆದರೆ ಅಲ್ಲಿ\break ಸಾಯುವುದು ಕೆಟ್ಟದ್ದು. ಶಿಶುವಾಗಿ ಹುಟ್ಟುವುದು ಒಳ್ಳೆಯದು, ಆದರೆ ಶಿಶುವಾಗಿಯೇ ಉಳಿಯುವುದು ಕೆಟ್ಟದ್ದು. ಚರ್ಚುಗಳು, ಆಚಾರಗಳು, ಸಂಕೇತಗಳು, ಎಲ್ಲಾ ಮಕ್ಕಳಿಗೆ ಒಳ್ಳೆಯವು. ಆದರೆ ಮಗು ಬೆಳೆದರೆ ಅವನ್ನು ಒಡೆದು ಹೊರಗೆ ಬರಬೇಕು, ಇಲ್ಲವೇ ತಾನೇ ನಾಶವಾಗಬೇಕು. ನಾವು ಎಂದೆಂದಿಗೂ ಮಕ್ಕಳಾಗಿರಕೂಡದು. ಇದು ಒಂದು ಅಂಗಿಯನ್ನು ಎಲ್ಲಾ ವಯಸ್ಸಿನವರಿಗೂ ಎಲ್ಲಾ ಗಾತ್ರದವರಿಗೂ ಕೊಡಿಸಲು ಯತ್ನಿಸುವಂತೆ. ಪ್ರಪಂಚದಲ್ಲಿರುವ ಮತಗಳನ್ನು ನಾನು ಅಲ್ಲಗಳೆಯುವುದಿಲ್ಲ. ಭಗವಂತನ ದಯೆಯಿಂದ ಇನ್ನೂ ಇಪ್ಪತ್ತು ಲಕ್ಷ ಮತಗಳು ಬೇಕಾದರೂ ಇರಲಿ, ಹೆಚ್ಚು ಇದ್ದಷ್ಟೂ ನಮಗೆ ಆರಿಸಿಕೊಳ್ಳಲು ಹೆಚ್ಚು\break ಅವಕಾಶವಿರುವುದು. ಆದರೆ ನಾನು ವಿರೋಧಿಸುವುದು ಯಾವುದನ್ನೆಂದರೆ ಎಲ್ಲರಿಗೂ ಒಂದೇ ಧರ್ಮವನ್ನು ಕೊಡಲು ಯತ್ನಿಸುವುದನ್ನು. ಧರ್ಮಗಳೆಲ್ಲ ಸಾರದೃಷ್ಟಿಯಿಂದ ಒಂದೇ ಆದರೂ ಬೇರೆ ಬೇರೆ ಜನಾಂಗಗಳ ವಿವಿಧ ವಾತಾವರಣಗಳಿಗೆ ತಕ್ಕಂತೆ ಅವುಗಳಲ್ಲಿ ವೈವಿಧ್ಯತೆ ಇರಬೇಕು. ನಮ್ಮಲ್ಲಿ ಪ್ರತಿಯೊಬ್ಬರಿಗೂ ಪ್ರತ್ಯೇಕವಾದ ಧರ್ಮ ಬೇಕಾಗಿದೆ. ಬಾಹ್ಯದೃಷ್ಟಿಯಿಂದ ಪ್ರತ್ಯೇಕತೆ ಇರಬೇಕು.

ಹಲವು ವರುಷಗಳ ಹಿಂದೆ ನನ್ನ ಸ್ವಂತ ದೇಶದ ಮಹಾತ್ಮರೊಬ್ಬರನ್ನು ಸಂದರ್ಶಿಸಿದೆ. ಅತಿ ಪವಿತ್ರಾತ್ಮರವರು, ಭಗವದ್ವಾಣಿ ಎನ್ನುವ ವೇದ, ಖುರಾನ್, ಬೈಬಲ್ ಮುಂತಾದು\-ವುಗಳ ವಿಷಯಗಳನ್ನು ಕುರಿತು ಮಾತನಾಡಿದೆವು. ಮಾತು ಮುಗಿದ ಮೇಲೆ ಆ ಮಹಾತ್ಮರು ಮೇಜಿನ ಮೇಲಿರುವ ಒಂದು ಪುಸ್ತಕವನ್ನು ತಾ ಎಂದರು. ಆ ಪುಸ್ತಕದಲ್ಲಿ (ಪಂಚಾಂಗದಲ್ಲಿ) ಇತರ ವಿಷಯಗಳೊಂದಿಗೆ ಆ ವರುಷ ಎಷ್ಟು ಮಳೆ ಬೀಳುವುದು ಎಂಬುದನ್ನು ಬರೆದಿತ್ತು. ಅದನ್ನು ಓದು ಎಂದರು ಆ ಮಹಾತ್ಮರು. ಎಷ್ಟು ಕೊಳಗ ಮಳೆ ಬರುವುದು ಎಂಬುದನ್ನು ಓದಿದೆ. ಅವರು ಅನಂತರ ನನಗೆ ಆ ಪುಸ್ತಕವನ್ನು ಹಿಂಡು ಎಂದರು. ನಾನು ಹಾಗೆಯೇ ಮಾಡಿದೆ. ಆಗ ಅವರು ಹೇಳಿದರು: “ಏತಕ್ಕೆ ಮಗು, ಒಂದು ತೊಟ್ಟು ನೀರೂ ಬೀಳುವುದಿಲ್ಲವಲ್ಲ? ನೀರು ಬೀಳದೆ ಇದ್ದರೆ ಅದು ಬರಿಯ ಪುಸ್ತಕ ಮಾತ್ರ. ಹಾಗೆಯೆ ಧರ್ಮದಿಂದ ಭಗವತ್ ಸಾಕ್ಷಾತ್ಕಾರವಾಗದೆ ಇದ್ದರೆ ಅದರಿಂದ ಏನೂ ಪ್ರಯೋಜನವಿಲ್ಲ. ಧರ್ಮವನ್ನು ತಿಳಿದುಕೊಳ್ಳಬೇಕೆಂದು ಬರಿಯ ಗ್ರಂಥಪಠನೆಯನ್ನು ಮಾಡಿದವನು ಸಕ್ಕರೆಯ ಮೂಟೆಯನ್ನು ಹೊತ್ತರೂ ಅದರ ಸವಿಯನ್ನು ಅರಿಯದ ಕತ್ತೆಯ ಕಥೆಯನ್ನು ಜ್ಞಾಪಕಕ್ಕೆ ತರುತ್ತಾನೆ.”

ನಾವು ಮನುಷ್ಯರಿಗೆ “ಎಂತಹ ದುರ್ದೈವವಶರಾದ ಪಾಪಿಗಳು ನಾವು'' ಎಂದು ಮೊಳಕಾಲೂರಿ ಪ್ರಾರ್ಥಿಸಿ ಎಂದು ಬೋಧಿಸಬೇಕೆ? ಇಲ್ಲ, ಅದರ ಬದಲು ಅವರ ಪವಿತ್ರ ಸ್ವಭಾವವನ್ನು ನೆನಪಿಗೆ ತನ್ನಿ. ನಾನೊಂದು ಕಥೆಯನ್ನು ಹೇಳುತ್ತೇನೆ. ಒಂದು ಹೆಣ್ಣು ಸಿಂಹ ಆಹಾರಕ್ಕಾಗಿ ಪ್ರಾಣಿಗಳನ್ನು ಹುಡುಕಾಡುತ್ತಿದ್ದಾಗ ಒಂದು ಕುರಿಯ ಮಂದೆ ಎದುರಿಗೆ ಕಂಡುಬಂತು. ಅದರಲ್ಲಿ ಒಂದು ಕುರಿಯನ್ನು ಹಿಡಿಯಬೇಕೆಂದು ನೆಗೆದಾಗ ಸಿಂಹ ಒಂದು ಮರಿಯನ್ನು ಹೆತ್ತು ಅಲ್ಲೇ ತೀರಿಕೊಂಡಿತು. ಆ ಸಿಂಹದ ಮರಿಯನ್ನು ಕುರಿಯಂತೆ ಮಂದೆಯ ಜೊತೆಯಲ್ಲಿ ಬೆಳೆಸಿದರು. ಅದು ಹುಲ್ಲು ತಿಂದು ಕುರಿಯಂತೆಯೆ ಅರಚಿಕೊಳ್ಳುತ್ತಿತ್ತು. ಒಂದು ದಿನ ಒಂದು ಸಿಂಹ ಆ ಮಂದೆಯ ಸಮೀಪಕ್ಕೆ ಬಂತು. ಆ ಕುರಿಯ ಮಂದೆಯಲ್ಲಿ ದೊಡ್ಡದೊಂದು ಸಿಂಹ ಕುರಿಯಂತೆ ಹುಲ್ಲು ತಿಂದು, ಅದರಂತೆ ಅರಚಿಕೊಳ್ಳುತ್ತಿದ್ದುದ್ದನ್ನು ನೋಡಿ ಅದಕ್ಕೆ ಆಶ್ಚರ್ಯವಾಯಿತು. ಸಿಂಹವನ್ನು ಕಂಡೊಡನೆಯೆ ಕುರಿಗಳೆಲ್ಲ ಕುರಿಯಂತೆ ಇರುವ ಸಿಂಹದೊಂದಿಗೆ ಪರಾರಿಯಾದವು. ಸಿಂಹ ಕಾದಿದ್ದು ಒಂದು ದಿನ ಮಲಗಿದ ಸಿಂಹದ ಕುರಿಯನ್ನು ಕಂಡಿತು. ಅದನ್ನು ಎಬ್ಬಿಸಿ ನೀನು ಸಿಂಹ ಎಂದಿತು. ಅದು, ಇಲ್ಲ ನಾನು ಕುರಿ ಎಂದು ಕುರಿಯಂತೆ ಅರಚಿತು. ನಿಜವಾದ ಸಿಂಹ ಅದನ್ನು ಒಂದು ಕೊಳಕ್ಕೆ ಕರೆದುಕೊಂಡು ಹೋಗಿ, “ನನ್ನ ನೆರಳನ್ನು ನೋಡು; ಅದು ನಿನ್ನ ನೆರಳಿನಂತೆಯೆ ಇದೆಯೋ ಇಲ್ಲವೋ?'' ಎಂದಿತು. ಕುರಿಯ ಮಂದೆಯಲ್ಲಿದ್ದ ಸಿಂಹ ಅದನ್ನು ನೋಡಿ ಒಪ್ಪಿಕೊಂಡಿತು. ಆಗ ಸಿಂಹ ಗರ್ಜಿಸಿ ಕುರಿಯ ಮಂದೆಯಲ್ಲಿದ್ದ ಸಿಂಹಕ್ಕೂ ಗರ್ಜಿಸು ಎಂದಿತು. ಅದು ಪ್ರಯತ್ನಿಸಿ ಮತ್ತೊಂದು ಸಿಂಹದಷ್ಟೇ ಭಯಾನಕವಾಗಿ ಗರ್ಜಿಸಿತು. ಅದು ಇನ್ನು ಮೇಲೆ ಕುರಿಯಾಗಿ ಉಳಿಯಲಿಲ್ಲ.

ನನ್ನ ಸ್ನೇಹಿತರೆ, ನೀವೆಲ್ಲ ಸಿಂಹದಂತೆ ಪೌರುಷವಂತರು ಎಂದು ಹೇಳಬಯಸುತ್ತೇನೆ.

ಕೋಣೆ ಕತ್ತಲೆಯಾಗಿದ್ದರೆ “ಅಯ್ಯೋ! ಕತ್ತಲೆ, ಕತ್ತಲೆ'' ಎಂದು ಅರಚುತ್ತೀರೇನು? ಬೆಳಕನ್ನು ತರಲು ಇರುವ ಏಕ ಮಾತ್ರ ಮಾರ್ಗವೆ ಒಂದು ದೀಪವನ್ನು ಹತ್ತಿಸುವುದು. ಆಗ ಕತ್ತಲೆ ತಕ್ಷಣ ಮಾಯವಾಗುವುದು. ಈಗ ನಿಮಗೆ ಅತೀತವಾಗಿರುವ ಜ್ಯೋತಿಯನ್ನು ತಿಳಿದುಕೊಳ್ಳಲು ಏಕಮಾತ್ರ ಮಾರ್ಗವೆ ನಿಮ್ಮ ಅಂತರಾಳದಲ್ಲಿರುವ ಆಧ್ಯಾತ್ಮಿಕ ಜ್ಯೋತಿಯನ್ನು ಹಚ್ಚುವುದು. ಆಗ ಪಾಪದ ಮತ್ತು ಅಶುದ್ಧ ಭಾವನೆಯ ಅಂಧಕಾರವೆಲ್ಲ ಮಾಯವಾಗುವುದು. ನಿಮ್ಮ ಉತ್ತಮ ಸ್ವಭಾವವನ್ನು ಕುರಿತು ಯೋಚಿಸಿ, ಕೀಳು ಸ್ವಭಾವವನ್ನು ಕುರಿತು ಯೋಚಿಸಬೇಡಿ.

\delimiter

ಉಪನ್ಯಾಸಾನಂತರ ಶ್ರೋತೃಗಳಲ್ಲಿ ಕೆಲವರು ಕೆಲವು ಪ್ರಶ್ನೆಗಳನ್ನು ಕೇಳಿದರು.\break ಅವರಲ್ಲಿ ಒಬ್ಬರು, “ಪಾದ್ರಿಗಳು ನರಕಜ್ವಾಲೆಯ ಭಯವನ್ನು ಬೋಧಿಸದೆ ಇದ್ದರೆ ಜನರ ಮೇಲೆ ಹತೋಟಿಯೆ ಇರುವುದಿಲ್ಲ” ಎಂದರು.

ಸ್ವಾಮೀಜಿ: “ಆಗ ಹತೋಟಿ ಕಳೆದುಕೊಳ್ಳುವುದು ಒಳ್ಳೆಯದು. ಯಾರು ಭಯದಿಂದ ಧಾರ್ಮಿಕರಾಗುತ್ತಾರೆಯೊ ಅವರಲ್ಲಿ ಯಾವ ಧರ್ಮವೂ ಇಲ್ಲ. ಮಾನವರಿಗೆ ಅವರ ಮೃಗೀಯ ಸ್ವಭಾವದ ಬದಲು ದೈವೀ ಸ್ವಭಾವವನ್ನು ಬೋಧಿಸಿ.”

ಪ್ರಶ್ನೆ: “ಸ್ವರ್ಗದ ರಾಜ್ಯವು ಇಹಕ್ಕೆ ಸಂಬಂಧಪಟ್ಟಿದ್ದಲ್ಲ ಎಂದು ಕ್ರಿಸ್ತ ಹೇಳಿದುದರ\break ಅರ್ಥವೇನು?''

ಉತ್ತರ: “ಸ್ವರ್ಗದ ರಾಜ್ಯ ನಮ್ಮಲ್ಲಿಯೇ ಇರುವುದು, ಯೆಹೂದ್ಯರು ಈ ಪ್ರಪಂಚವನ್ನು ಮೀರಿದ ಸ್ವರ್ಗ ಇದೆ ಎಂದು ನಂಬಿದ್ದರು. ಏಸುವಿನ ಭಾವನೆ ಇದಲ್ಲ.”

ಪ್ರಶ್ನೆ: “ನಾವು ಪ್ರಾಣಿಗಳಿಂದ ವಿಕಾಸವಾದವರು ಎಂಬುದನ್ನು ನೀವು ನಂಬುತ್ತೀರಾ?''

ಉತ್ತರ: “ನಾನು, ವಿಕಾಸ ಸಿದ್ದಾಂತದ ಪ್ರಕಾರ ಜೀವಿಗಳು ಕೆಳವರ್ಗದಿಂದ ಮೇಲಿನ ವರ್ಗಕ್ಕೆ ವಿಕಾಸಹೊಂದುತ್ತವೆ ಎಂಬುದನ್ನು ನಂಬುತ್ತೇನೆ.”

ಪ್ರಶ್ನೆ: “ತನ್ನ ಹಿಂದಿನ ಜನ್ಮಗಳು ಜ್ಞಾಪಕದಲ್ಲಿರುವ ಯಾರನ್ನಾದರೂ ನೀವು\break ನೋಡಿರುವಿರಾ?”

ಉತ್ತರ: “ತಮ್ಮ ಹಿಂದಿನ ಜೀವನದ ನೆನಪು ಇದೆ ಎಂದು ಹೇಳಿದ ಕೆಲವರನ್ನು ನಾನು ನೋಡಿರುವೆನು. ಅವರು ತಮ್ಮ ಹಿಂದಿನ ಜೀವನವನ್ನೆಲ್ಲ ತಿಳಿದುಕೊಳ್ಳಬಲ್ಲಂತಹ ಒಂದು ಸ್ಥಿತಿಗೆ ಬಂದಿದ್ದರು.''

ಪ್ರಶ್ನೆ: “ನೀವು ಕ್ರಿಸ್ತನನ್ನು ಶಿಲುಬೆಗೆ ಏರಿಸಿದರು ಎಂದು ನಂಬುತ್ತೀರಾ?”

ಉತ್ತರ: “ಭಗವಂತನ ಅವತಾರವೆ ಕ್ರಿಸ್ತ, ಜನರಿಗೆ ಅವನನ್ನು ಕೊಲ್ಲಲು ಸಾಧ್ಯವಾಗಲಿಲ್ಲ. ಶಿಲುಬೆಗೆ ಏರಿಸಿದ್ದು ಒಂದು ಮಿಥ್ಯಾವಸ್ತುವನ್ನು.”

ಪ್ರಶ್ನೆ: “ಅವನು ಅಂತಹ ಮಿಥ್ಯಾವಸ್ತುವನ್ನು ಉಂಟುಮಾಡಿದ್ದರೆ ಅದು ಪವಾಡಗಳಲ್ಲೆಲ್ಲಾ ಅತಿ ದೊಡ್ಡ ಪವಾಡವಲ್ಲವೆ?”

ಉತ್ತರ: “ಆಧ್ಯಾತ್ಮಿಕ ಜೀವನದಲ್ಲಿ ಪವಾಡಗಳು ಒಂದು ದೊಡ್ಡ ಆತಂಕ ಎಂದು ನಾನು ಪರಿಗಣಿಸುತ್ತೇನೆ. ಬುದ್ಧನ ಶಿಷ್ಯನೊಬ್ಬನು, ಬಹಳ ಮೇಲಿರುವ ಕಮಂಡಲ ಒಂದನ್ನು, ಒಬ್ಬನು ಅದನ್ನು ಮುಟ್ಟದೆ ತೆಗೆದ ಎಂದು ಹೇಳಿ, ಆ ಕಮಂಡಲವನ್ನು ಅವನಿಗೆ ತೋರಿದನು. ಬುದ್ದ ಆ ಕಮಂಡಲವನ್ನು ತೆಗೆದುಕೊಂಡು ಅದನ್ನು ಚೂರುಚೂರುಮಾಡಿ `ಇಂತಹ ಪವಾಡದ ಮೇಲೆ ನಿಮ್ಮ ಶ್ರದ್ದೆಯನ್ನು ರೂಪಿಸಬೇಡಿ. ಎಂದೆಂದಿಗೂ ಇರುವ ಸನಾತನ ತತ್ತ್ವದಲ್ಲಿ ಸತ್ಯವನ್ನು ಕಾಣಿರಿ' ಎಂದನು. ಬುದ್ದನು ಜನರಿಗೆ ನಿಜವಾದ ಅಂತರ್ಜ್ಯೋತಿಯನ್ನು ಕೊಟ್ಟನು, ಆತ್ಮಜ್ಯೋತಿಯನ್ನು ಬೋಧಿಸಿದನು. ಇದೊಂದೆ ಸುರಕ್ಷಿತವಾದ ಮಾರ್ಗ. ಪವಾಡಗಳು ಆತಂಕಗಳು. ನಾವು ಅವನ್ನು ಆಚೆಗೆ ಕಿತ್ತೊಗೆಯಬೇಕು.”

ಪ್ರಶ್ನೆ: “ಜೀಸಸ್ ಗುಡ್ಡದ ಮೇಲಿನ ಉಪದೇಶವನ್ನು ಮಾಡಿದ ಎಂದು ನೀವು ನಂಬುತ್ತೀರಾ?”

ಉತ್ತರ: “ಅವನು ಮಾಡಿದ ಎಂದು ನಾನು ನಂಬುತ್ತೇನೆ. ಆದರೆ ಇಲ್ಲಿ ನಾನು ಇತರರಂತೆ ಪುಸ್ತಕವನ್ನು ಅನುಸರಿಸಬೇಕಾಗಿದೆ. ಬರಿಯ ಪುಸ್ತಕ ಪ್ರಮಾಣ ಅಷ್ಟೊಂದು ಪ್ರಬಲವಲ್ಲ ಎಂದು ನನಗೆ ಗೊತ್ತು. ಆದರೆ ಗುಡ್ಡದ ಮೇಲೆ ಮಾಡಿದ ಉಪದೇಶವನ್ನು ನಮ್ಮ ಜೀವನದಲ್ಲಿ ಮಾರ್ಗದರ್ಶಿಯಂತೆ ತೆಗೆದುಕೊಂಡರೆ ನಾವು ಸುರಕ್ಷಿತರಾಗಿರುವೆವು. ಯಾವುದು ನಮ್ಮ ಅಂತರಾತ್ಮನಿಗೆ ಸೂಕ್ತವೆಂದು ತೋರುವುದೋ ಅದನ್ನು ನಾವು ಸ್ವೀಕರಿಸಬೇಕು. ಬುದ್ದ ಕ್ರಿಸ್ತಪೂರ್ವ ಐನೂರು ವರ್ಷಕ್ಕೆ ಮುಂಚೆ ಬೋಧಿಸಿದ ಅವನ ಉಪದೇಶವೆಲ್ಲ ಆಶೀರ್ವಾದಮಯವಾಗಿದೆ. ಅವನ ಬಾಯಿಂದ ಒಂದು ಶಾಪವೂ ಬರಲಿಲ್ಲ. ಅವನು ಜೀವನದಲ್ಲಿ ಎಂದೂ ಇತರರನ್ನು ಶಪಿಸಿದವನಲ್ಲ. ಅದರಂತೆಯೆ ಜೊರಾಸ್ಟರ್‌ ಮತ್ತು ಕನ್ಫೂಶಿಯಸ್‌ರು ಕೂಡ.”

