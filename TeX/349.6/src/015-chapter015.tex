
\chapter[ಪ್ರಕೃತಿ ಮತ್ತು ಪುರುಷ]{ಪ್ರಕೃತಿ ಮತ್ತು ಪುರುಷ\protect\footnote{\engfoot{C.W, Vol. VI, P. 33}}}

ಆಧುನಿಕರ ಭಾವನೆಯಲ್ಲಿ ಪ್ರಕೃತಿ ಎಂದರೆ ಕೇವಲ ಭೌತಿಕ ಆವಿರ್ಭಾವ ಮಾತ್ರ. ಯಾವುದನ್ನು ಸಾಧಾರಣವಾಗಿ ಮನಸ್ಸು ಎನ್ನುವೆವೊ ಅದು ಪ್ರಕೃತಿಯಲ್ಲ ಎಂದು ಭಾವಿಸಿದ್ದೇವೆ. ಇಚ್ಛಾಸ್ವಾತಂತ್ರ್ಯ ಎಂಬುದೊಂದಿದೆ ಎಂದು ಸಾಧಿಸಲು ಯತ್ನಿಸುವ ತಾತ್ತ್ವಿಕರು ಮನಸ್ಸನ್ನು ಪ್ರಕೃತಿಗೆ ಸೇರಿಸುವುದಿಲ್ಲ. ಏಕೆಂದರೆ ಪ್ರಕೃತಿ ಖಂಡಿತವಾಗಿ ಯಾರಿಗೂ ಜಗ್ಗದ ಒಂದು ನಿಯಮಕ್ಕೆ ಒಳಪಟ್ಟಿದೆ. ಪ್ರಕೃತಿಯಲ್ಲಿ ಮನಸ್ಸು ಇದೆ ಎಂದರೆ ಮನಸ್ಸೂ ಕೂಡ ಆ ನಿಯಮಕ್ಕೆ ಒಳಪಡಬೇಕಾಗುವುದು. ಮನಸ್ಸು ಪ್ರಕೃತಿಯ ಅಧೀನಕ್ಕೆ ಒಳಪಟ್ಟರೆ\break ಸ್ವಾತಂತ್ರ್ಯೇಚ್ಛೆಯ ಸಿದ್ದಾಂತ ಮಣ್ಣುಪಾಲಾದಂತೆಯೆ. ಏಕೆಂದರೆ ಯಾವುದು ನಿಯಮಕ್ಕೆ ಒಳಪಟ್ಟಿದೆಯೋ ಅದು ಹೇಗೆ ಸ್ವತಂತ್ರವಾಗಬಲ್ಲದು?

ಭರತಖಂಡದ ದಾರ್ಶನಿಕರು ಇದಕ್ಕೆ ವಿರುದ್ಧವಾದ ನಿಲುವನ್ನು ತಾಳಿರುವರು. ವ್ಯಕ್ತವಾದ ಅಥವಾ ಅವ್ಯಕ್ತವಾದ ಎಲ್ಲಾ ವಿಧವಾದ ಜೀವಗಳೂ ನಿಯಮಕ್ಕೆ ಒಳಪಟ್ಟಿರುವುವೆಂದು ಅವರು ಹೇಳುವರು. ಮನಸ್ಸು ಮತ್ತು ಬಾಹ್ಯ ಪ್ರಕೃತಿ ಎಲ್ಲಾ ನಿಯಮಕ್ಕೆ ಒಳಪಟ್ಟಿವೆ, ಎಲ್ಲಾ ಒಂದೇ ಬಗೆಯ ನಿಯಮಕ್ಕೆ ಅಧೀನವಾಗಿವೆ ಎನ್ನುತ್ತಾರೆ ಅವರು. ಮನಸ್ಸು ಒಂದು ನಿಯಮಕ್ಕೆ ಒಳಗಾಗದಿದ್ದರೆ, ನಾವು ಆಲೋಚಿಸುವ ಯೋಚನೆಗಳು, ಅವುಗಳ ಹಿಂದೆ ಮಾಡಿದ ಯೋಚನೆಯ ಪ್ರತಿಫಲಗಳಲ್ಲದಿದ್ದರೆ, ಒಂದು ಮಾನಸಿಕ ಸ್ಥಿತಿ ಮತ್ತೊಂದು ಮಾನಸಿಕ ಸ್ಥಿತಿಯ ಪರಿಣಾಮವಲ್ಲದೇ ಇದ್ದರೆ, ಮನಸ್ಸು ತರ್ಕರಹಿತವಾಗುತ್ತದೆ. ಇಚ್ಛೆ ಸ್ವತಂತ್ರವೆಂದೂ ಮತ್ತು ಯುಕ್ತಿ ಅಲ್ಲಿ ಕೆಲಸಮಾಡಲಾರದು ಎಂದೂ ಯಾರು ಹೇಳಬಲ್ಲರು? ಮನಸ್ಸು ಕಾರ್ಯಕಾರಣ ನಿಯಮಕ್ಕೆ ಒಳಪಟ್ಟಿದ್ದರೆ ಇಚ್ಛೆ ಸ್ವತಂತ್ರ ಎಂದು ಅವರು ಹೇಗೆ ಸಾಧಿಸಬಲ್ಲರು? ನಿಯಮವೆಂಬುದು ಕಾರ್ಯಕಾರಣ ಸಂಬಂಧವನ್ನು ಅವಲಂಬಿಸಿದೆ. ಕೆಲವು ಘಟನೆಗಳು ಆದ ಮೇಲೆ ಅವಕ್ಕೆ ಸರಿಯಾಗಿ ಮತ್ತೆ ಕೆಲವು ಘಟನೆಗಳು ಆಗುವುವು; ಪ್ರತಿಯೊಂದು ನಡೆದುದಕ್ಕೂ ಮುಂದೆ ನಡೆಯುವುದೊಂದು ಇದ್ದೇ ಇರುವುದು. ಪ್ರಕೃತಿಯಲ್ಲೇ ಹೀಗೆ. ಮನಸ್ಸಿನಲ್ಲಿ ಕಾರ್ಯಕಾರಣ ನಿಯಮ ಜಾರಿಯಲ್ಲಿದ್ದರೆ, ಮನಸ್ಸು ಬದ್ಧವಾದುದು; ಅದು ಸ್ವತಂತ್ರವಲ್ಲ. ಇಲ್ಲಿ ಇಚ್ಛೆ ಎಂದಿಗೂ ಸ್ವತಂತ್ರವಲ್ಲ, ಅದು ಹೇಗೆ ಸ್ವತಂತ್ರವಾಗಬಲ್ಲುದು? ಆದರೆ ನಾವೆಲ್ಲ ಮುಕ್ತರೆಂದು ಗೊತ್ತಿದೆ. ಮುಕ್ತರೆಂದು ಅನುಭವಿಸುತ್ತೇವೆ. ನಾವು ಸ್ವತಂತ್ರರಲ್ಲದೆ ಇದ್ದರೆ, ಜೀವನಕ್ಕೆ ಅರ್ಥವೇ ಇರುತ್ತಿರಲಿಲ್ಲ; ಬಾಳಿ ಪ್ರಯೋಜನವಿರುತ್ತಿರಲಿಲ್ಲ.

ಪ್ರಾಚ್ಯ ದಾರ್ಶನಿಕರು ಈ ಸಿದ್ದಾಂತವನ್ನು ಒಪ್ಪಿಕೊಂಡರು, ಅಥವಾ ಈ ಸಿದ್ದಾಂತವನ್ನು ಸಾರಿದರು. ಮನಸ್ಸು ಮತ್ತು ಇಚ್ಛೆ ಇವೆರಡೂ ಪ್ರಕೃತಿಯಂತೆಯೇ ಕಾಲದೇಶನಿಮಿತ್ತದೊಳಗೆ ಇವೆ. ಆದಕಾರಣ ಅವು ಕಾರ್ಯಕಾರಣ ನಿಯಮಕ್ಕೆ ಬಂಧಿಗಳು. ನಾವು ಕಾಲದಲ್ಲಿ ಆಲೋಚಿಸುವೆವು, ನಮ್ಮ ಆಲೋಚನೆ ಕಾಲಕ್ಕೆ ಅಧೀನವಾಗಿರುವುದು. ಇರುವುದೆಲ್ಲ ಕಾಲದೇಶಗಳಲ್ಲಿವೆ. ಎಲ್ಲವೂ ಕಾರ್ಯಕಾರಣ ನಿಯಮಕ್ಕೆ ಅಡಿಯಾಳು.

ನಾವು ಯಾವುದನ್ನು ದ್ರವ್ಯ (\enginline{matter}) ಮತ್ತು ಮನಸ್ಸು ಎನ್ನುವೆವೊ ಅವೆರಡೂ ಒಂದೇ. ಅವುಗಳಿಗೆ ಇರುವ ವ್ಯತ್ಯಾಸ ಸ್ಪಂದನದ ತರತಮದಲ್ಲಿ ಮಾತ್ರ. ಮನಸ್ಸು ಅತಿ ಕಡಮೆ ಸ್ಪಂದಿಸಿದಾಗ ಅದನ್ನು ದ್ರವ್ಯ (\enginline{matter}) ಎನ್ನುವರು. ದ್ರವ್ಯ ಅತಿ ಹೆಚ್ಚಾಗಿ ಸ್ಪಂದಿಸಿದಾಗ ಅದನ್ನು ಮನಸ್ಸು ಎನ್ನುವರು. ಎರಡೂ ಒಂದೇ ವಸ್ತುವಿನಿಂದ ಆಗಿವೆ. ಹೇಗೆ ದ್ರವ್ಯವು ದೇಶ–ಕಾಲ ನಿಮಿತ್ತಗಳಿಗೆ ಬಂದಿಯೋ ಹಾಗೆಯೇ ಮನಸ್ಸು ಕೂಡ ದೇಶ–ಕಾಲ ನಿಮಿತ್ತಗಳಿಗೆ ಬಂದಿ.

ಪ್ರಕೃತಿ ಎಲ್ಲ ಕಡೆಗಳಲ್ಲಿಯೂ ಸಮರೂಪವಾಗಿರುವುದು. (\enginline{homogeneous}). ವೈವಿಧ್ಯ ಇರುವುದು ಆವಿರ್ಭಾವದಲ್ಲಿ ಮಾತ್ರ. ಪ್ರಕೃತಿ ಎಂದರೆ ವೈವಿಧ್ಯ. ಎಲ್ಲಾ ಒಂದೇ ವಸ್ತು, ಆದರೆ ಅದು ವಿಧವಿಧವಾಗಿ ವ್ಯಕ್ತಗೊಂಡಿದೆ. ಮನಸ್ಸು ದ್ರವ್ಯವಾಗುವುದು, ದ್ರವ್ಯ ಮನಸ್ಸಾಗುವುದು. ಇದು ಕೇವಲ ಸ್ಪಂದನದ ಹೆಚ್ಚು ಕಡಿಮೆ ಮಾತ್ರ. ಒಂದು ಉಕ್ಕಿನ ಸಲಾಕಿಯನ್ನು ತೆಗೆದುಕೊಂಡು, ಶಕ್ತಿಯನ್ನು ಅದರ ಮೂಲಕ ಹರಿಸಿ ಅದು ಸ್ಪಂದಿಸುವಂತೆ ಮಾಡಿ. ಆಗ ಏನಾಗುವುದೆಂಬುದನ್ನು ನೋಡಿ. ಇದನ್ನು ಒಂದು ಕತ್ತಲೆಕೋಣೆಯಲ್ಲಿ ಮಾಡಿದರೆ ನಿಮಗೆ ಮೊದಲು ನಿಮ್ಮ ಅರಿವಿಗೆ ಬರುವುದೇ ಗುಂಯ್‌ಗುಟ್ಟುವ ಶಬ್ದ. ನೀವು ಸ್ಪಂದನವನ್ನು ಹೆಚ್ಚಿಸಿದರೆ ಆ ಉಕ್ಕಿನ ಸಲಾಕಿ ಕಾಂತಿಯಿಂದ ಕೋರೈಸುವುದು. ಸ್ಪಂದನವನ್ನು ಇನ್ನೂ ಹೆಚ್ಚು ಮಾಡಿದರೆ ಉಕ್ಕಿನ ಸಲಾಕಿ ಮಾಯವಾಗಿ ಹೋಗುವುದು. ಅದು ಮನಸ್ಸಾಗುವುದು.

ಮತ್ತೊಂದು ಉದಾಹರಣೆಯನ್ನು ತೆಗೆದುಕೊಳ್ಳಿ. ನಾನು ಹತ್ತು ದಿನ ಊಟ ಮಾಡದೆ ಇದ್ದರೆ ಆಲೋಚಿಸಲಾರದವನಾಗುತ್ತೇನೆ. ಎಲ್ಲೋ ಕೆಲವು ಆಲೋಚನೆಗಳು ಮಾತ್ರ ನನ್ನ ಮನಸ್ಸಿನಲ್ಲಿರುವುವು. ನಾನು ಬಹಳ ದುರ್ಬಲನಾಗುತ್ತೇನೆ. ನನ್ನ ಹೆಸರೆ ನನಗೆ ಗೊತ್ತಿರುವುದಿಲ್ಲ. ನಾನು ಸ್ವಲ್ಪ ಆಹಾರವನ್ನು ತೆಗೆದುಕೊಂಡರೆ ಸ್ವಲ್ಪ ಹೊತ್ತಿನ ಮೇಲೆ ಆಲೋಚಿಸುವುದಕ್ಕೆ ಮೊದಲು ಮಾಡುತ್ತೇನೆ. ನನ್ನ ಮನಸ್ಸಿನ ಶಕ್ತಿ ಪುನಃ ಬರುವುದು. ಆಹಾರ ಆಲೋಚನೆಯಾಯಿತು. ಇದರಂತೆಯೇ ಮನಸ್ಸು ತನ್ನ ಸ್ಪಂದನವನ್ನು ಕಡಮೆಮಾಡಿ ದೇಹದಲ್ಲಿ ವ್ಯಕ್ತವಾಗಿ ದ್ರವ್ಯವಾಗುವುದು.

ಇದರಲ್ಲಿ ದ್ರವ ಮೊದಲೆ, ಆಲೋಚನೆ ಮೊದಲೆ. ಒಂದು ಉದಾಹರಣೆಯನ್ನು ತೆಗೆದುಕೊಳ್ಳೋಣ: ಕೋಳಿ ಮೊಟ್ಟೆ ಇಡುವುದು, ಮೊಟ್ಟೆ ಮತ್ತೊಂದು ಕೋಳಿಯಾಗುವುದು. ಆ ಕೋಳಿ ಮತ್ತೊಂದು ಮೊಟ್ಟೆಯನ್ನು ಇಡುವುದು. ಹೀಗೆ ಒಂದು ಮತ್ತೊಂದಾಗುತ್ತಿರುವುದು. ಈಗ ಮೊಟ್ಟೆ ಮೊದಲೆ, ಕೋಳಿ ಮೊದಲೆ? ನೀವು ಯಾವ ಕೋಳಿಯೂ ಹಾಕದ ಮೊಟ್ಟೆಯನ್ನು ಕಲ್ಪಿಸಿಕೊಳ್ಳಲಾರಿರಿ. ಇಲ್ಲವೇ ಯಾವ ಮೊಟ್ಟೆಯಿಂದಲೂ ಬರದ ಕೋಳಿಯನ್ನು ಕಲ್ಪಿಸಿಕೊಳ್ಳಲಾರಿರಿ. ಇದರಲ್ಲಿ ಯಾವುದು ಮೊದಲಾದರೂ ಏನೂ ವ್ಯತ್ಯಾಸವಾಗುವುದಿಲ್ಲ. ನಮ್ಮ ಆಲೋಚನೆಯೆಲ್ಲ ಕೋಳಿಯ ಮತ್ತು ಮೊಟ್ಟೆಯ ನ್ಯಾಯದಂತೆ ಇದೆ.

ಜೀವನದ ಅತಿ ಶ್ರೇಷ್ಠ ಸತ್ಯಗಳು ಮರೆತುಹೋಗಿವೆ. ಕಾರಣ ಅವು ಅಷ್ಟು ಸರಳವಾಗಿರುವುದೇ ಆಗಿರುವುದು. ಶ್ರೇಷ್ಠ ಸತ್ಯಗಳು ಬಹಳ ಸರಳವಾಗಿವೆ ಏಕೆಂದರೆ ಅವು ಸರ್ವಕ್ಕೂ ಅನ್ವಯಿಸುವುವುಗಳಾಗಿವೆ. ಸತ್ಯವೇ ಯಾವಾಗಲೂ ಸರಳ ವಾಗಿರುವುದು, ಮಾನವನ ಅಜ್ಞಾನದಿಂದಲೇ ಅದು ಜಟಿಲವಾಗುವುದು.

ಮನುಷ್ಯನಲ್ಲಿ ಸ್ವತಂತ್ರವಾಗಿರುವುದು ಮನಸ್ಸಲ್ಲ. ಏಕೆಂದರೆ ಅದು ನಿಯಮಕ್ಕೆ ಒಳಗಾಗಿರುವುದು. ಅಲ್ಲಿ ಸ್ವಾತಂತ್ರ್ಯವಿಲ್ಲ. ಮನುಷ್ಯ ಮನಸ್ಸಲ್ಲ; ಅವನು ಆತ್ಮ. ಆತ್ಮ ನಿತ್ಯಮುಕ್ತ, ಯಾವಾಗಲೂ ಬಂಧನಕ್ಕೆ ಸಿಕ್ಕಿಬಿದ್ದುದಲ್ಲ, ಸನಾತನವಾದುದು. ಮಾನವ ಆತ್ಮದೃಷ್ಟಿಯಿಂದ ಸ್ವತಂತ್ರನು. ಆತ್ಮ ನಿತ್ಯಮುಕ್ತವಾದುದು. ಆದರೆ ಮನಸ್ಸು ತಾತ್ಕಾಲಿಕವಾಗಿ ಏಳುವ ಅಲೆಗಳೇ ತಾನೆಂದು ಭಾವಿಸಿ, ಆತ್ಮನನ್ನು ಮರೆತು, ಮಾಯೆಯೆಂಬ ಕಾಲ–ದೇಶ ನಿಮಿತ್ತದ ಜಾಲದಲ್ಲಿ ಮುಳುಗಿರುವುದು. ಇದೇ ನಮ್ಮ ಬಂಧನಕ್ಕೆ ಕಾರಣ. ನಾವು ಯಾವಾಗಲೂ ಮನಸ್ಸು ಮತ್ತು ಅದರಲ್ಲಿ ಆಗುತ್ತಿರುವ ಬದಲಾವಣೆಗಳೊಂದಿಗೆ ತಾದಾತ್ಮ್ಯಭಾವವನ್ನು ತಾಳುವೆವು. ಮಾನವನ ಸ್ವಾತಂತ್ರ್ಯವೇ ಆತ್ಮನಲ್ಲಿ ಪ್ರತಿಷ್ಠಿತವಾಗಿದೆ. ಆತ್ಮ ತಾನು ಮುಕ್ತವೆಂದು ಯಾವಾಗಲೂ ಅರಿತು ಮನಸ್ಸು ಬಂಧನದಲ್ಲಿದ್ದರೂ ತನ್ನ ಸ್ವಾತಂತ್ರ್ಯವನ್ನು ವ್ಯಕ್ತಗೊಳಿಸಲು ಯತ್ನಿಸುತ್ತಿರುವುದು. `ನಾನು ಮುಕ್ತ, ನನ್ನ ನಿಜರೂಪವೆ ನಾನಾಗಿರುವೆನು.' ಇದೇ ನಮ್ಮ ಮುಕ್ತಿ, ನಿತ್ಯ ಮುಕ್ತವಾದ ಅನಿರ್ಬಂಧವಾದ ಸನಾತನವಾದ ಆತ್ಮ ಯುಗಯುಗಗಳಿಂದಲೂ ತನ್ನ ಯಂತ್ರವಾದ ಮನಸ್ಸಿನ ಮೂಲಕ ವ್ಯಕ್ತವಾಗಲು ಯತ್ನಿಸುತ್ತಿದೆ.

ಹಾಗಾದರೆ ಮನುಷ್ಯನಿಗೂ ಪ್ರಕೃತಿಗೂ ಏನು ಸಂಬಂಧವಿದೆ? ಅತಿಕ್ಷುದ್ರತಮ ಕೀಟದಿಂದ ಹಿಡಿದು ಮಾನವನವರೆಗೆ ಆತ್ಮವು ಪ್ರಕೃತಿಯ ಮೂಲಕ ವ್ಯಕ್ತವಾಗಲು ಯತ್ನಿಸುತ್ತಿದೆ. ಅತಿ ಕ್ಷುದ್ರ ಜೀವದ ಆವಿರ್ಭಾವದ ಹಿಂದೆಯೂ ಆತ್ಮನ ಅನಂತ ಮಹಿಮೆ ಸುಪ್ತವಾಗಿದೆ. ಅದು ಕ್ರಮೇಣ ವಿಕಾಸದ ವಿಧಾನದ ಮೂಲಕ ವ್ಯಕ್ತವಾಗುತ್ತಾ ಬರುತ್ತಿದೆ. ಆತ್ಮ ತನ್ನ ಮಹಿಮೆಯನ್ನು ವ್ಯಕ್ತಗೊಳಿಸುವುದಕ್ಕಾಗಿಯೆ ವಿಕಾಸದ ಅನಂತ ಮೆಟ್ಟಲುಗಳು ಇರುವುದು. ಪ್ರತಿ ಕ್ಷಣದಲ್ಲಿಯೂ ಅದು ಪ್ರಕೃತಿಗೆ ವಿರೋಧವಾಗಿ ಹೋರಾಡುತ್ತಿದೆ. ಮನುಷ್ಯ ಈಗ ಇರುವ ಸ್ಥಿತಿಗೆ ಕಾರಣ ಪ್ರಕೃತಿಯೊಡನೆ ಅವನು ಮಾಡಿಕೊಳ್ಳುವ ರಾಜಿಯಲ್ಲ. ಪ್ರಕೃತಿಯೊಂದಿಗೆ ಸೌಹಾರ್ದದಿಂದ, ಪ್ರಕೃತಿಯೊಂದಿಗೆ ಸಮರಸದಿಂದ ಬಾಳಿ ಎಂದು ನಾವು ಬೇಕಾದಷ್ಟು ಕೇಳುತ್ತೇವೆ. ಇದೊಂದು ತಪ್ಪು. ಮೇಜು ಮಡಕೆ ಲೋಹ ಗಿಡಮರ ಇವೆಲ್ಲ ಪ್ರಕೃತಿಯೊಂದಿಗೆ ಸಮರಸದಿಂದ ಇವೆ. ಅಲ್ಲಿ ಪೂರ್ಣಸಾಮರಸ್ಯವಿದೆ. ಅಸಾಮರಸ್ಯ ಇಲ್ಲವೇ ಇಲ್ಲ. ಪ್ರಕೃತಿಯೊಂದಿಗೆ ಸಾಮರಸ್ಯದಿಂದ ಇರುವುದೇ ತಾಟಸ್ಥ್ಯ, ಅದೇ ಮೃತ್ಯು. ಮನುಷ್ಯ, ಈ ಮನೆಯನ್ನು ಹೇಗೆ ಕಟ್ಟಿದ? ಪ್ರಕೃತಿಯ ಸ್ನೇಹದಿಂದಲೆ? ಇಲ್ಲ, ಪ್ರಕೃತಿಯೊಡನೆ ಹೋರಾಡಿ. ಪ್ರಕೃತಿಯೊಡನೆ ಅವಿರಳವಾಗಿ ನಡೆಸಿದ ಹೋರಾಟವೇ ಮಾನವನ ಪ್ರಗತಿಗೆ ಕಾರಣವೆ ಹೊರತು ಅದಕ್ಕೆ ಬಾಗಿ ಶರಣಾಗುವುದಲ್ಲ.

