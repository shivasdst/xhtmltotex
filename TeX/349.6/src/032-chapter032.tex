
\chapter[ವೇದಾಂತ ತತ್ತ್ವ]{ವೇದಾಂತ ತತ್ತ್ವ\protect\footnote{\engfoot{C.W. Vol. I, P. 357}}}

ಯಾವುದನ್ನು ಈಗ ಸಾಧಾರಣವಾಗಿ ವೇದಾಂತ ತತ್ತ್ವ ಎನ್ನುವರೋ ಅದರಲ್ಲಿ ಈಗ ಭರತಖಂಡದಲ್ಲಿರುವ ಎಲ್ಲಾ ಪಂಥಗಳೂ ಸೇರಿವೆ. ಆದಕಾರಣ ಅದನ್ನು ಹಲವು ರೀತಿಗಳಲ್ಲಿ ವಿವರಿಸುವರು. ನನ್ನ ದೃಷ್ಟಿಗೆ ಈ ವಿವರಣೆ ಪ್ರಗತಿಪರವಾಗಿ ಕಾಣುವುದು. ಇವೆಲ್ಲ ದ್ವೈತದಲ್ಲಿ ಪ್ರಾರಂಭವಾಗಿ ಅದ್ವೈತದಲ್ಲಿ ಪರ್ಯವಸಾನವಾಗುವುವು. ವೇದಾಂತ ಎಂದರೆ ವೇದಗಳ ಕೊನೆ ಎಂದು ಅರ್ಥ. ವೇದಗಳೇ ಹಿಂದೂಗಳ ಶಾಸ್ತ್ರ. ಕೆಲವು ವೇಳೆ ಪಾಶ್ಚಾತ್ಯದೇಶದಲ್ಲಿ ವೇದವೆಂದರೆ ಮಂತ್ರ ಮತ್ತು ಯಜ್ಞ ಎಂದು ಮಾತ್ರ ಭಾವಿಸುವರು. ಆದರೆ ಈಗ ಅವುಗಳು ಮುಕ್ಕಾಲುಪಾಲು ಉಪಯೋಗದಲ್ಲೇ ಇಲ್ಲ. ಈಗ ಭರತಖಂಡದಲ್ಲಿ ವೇದ ಎಂದರೆ ವೇದಾಂತ ಎಂದು ರೂಢಿಯಾಗಿ ಹೋಗಿದೆ. ನಮ್ಮ ಭಾಷ್ಯಕಾರರೆಲ್ಲ ಶಾಸ್ತ್ರದಿಂದ ಏನನ್ನಾದರೂ ಉದಾಹರಿಸಬೇಕಾದರೆ ಯಾವಾಗಲೂ ವೇದಾಂತದಿಂದ ಮಾತ್ರ ಹೇಳುವರು. ಇದಕ್ಕೆ ಇನ್ನೊಂದು ಪರಿಭಾಷೆ ಇದೆ; ಅದೇ ಶ್ರುತಿ. ಯಾವುದನ್ನು ಈಗ ವೇದಾಂತವೆನ್ನುವೆವೋ ಅವೆಲ್ಲ ಕರ್ಮ ಕಾಂಡವಾದ ಮೇಲೆ ಬರೆದವುಗಳಲ್ಲ; ಉದಾಹರಣೆಗೆ ಈಶೋಪನಿಷತ್ತು ಯಜುರ್ವೇದದ ನಲವತ್ತನೇ ಅಧ್ಯಾಯ. ಇದು ವೇದಗಳ ಅತಿ ಪುರಾತನ ಭಾಗಗಳಲ್ಲಿ ಒಂದು. ಕೆಲವು ಉಪನಿಷತ್ತುಗಳು ಬ್ರಾಹ್ಮಣದ ಅಥವಾ ಕರ್ಮಕಾಂಡದ ಭಾಗಗಳಾಗಿವೆ. ಮತ್ತೆ ಕೆಲವು ಉಪನಿಷತ್ತುಗಳು ಸ್ವತಂತ್ರವಾಗಿವೆ. ಅವು ಯಾವ ಬ್ರಾಹ್ಮಣದ ಅಥವಾ ಮಂತ್ರದ ಭಾಗವೂ ಆಗಿಲ್ಲ. ಆದರೆ ಅವು ಮತ್ತಾವುದರೊಂದಿಗೂ ಸಂಬಂಧವಿಲ್ಲದೆ ಸ್ವತಂತ್ರವಾಗಿದ್ದುವು ಎಂದು ಹೇಳಲಾಗುವುದಿಲ್ಲ. ಏಕೆಂದರೆ ವೇದದ ಎಲ್ಲಾ ಭಾಗಗಳೂ ಈಗ ಇಲ್ಲ. ಎಷ್ಟೋ ಬ್ರಾಹ್ಮಣಗಳು ನಾಶವಾಗಿ ಹೋಗಿವೆ. ಈ ಉಪನಿಷತ್ತುಗಳು ಹಿಂದೆ ಯಾವುದೋ ಒಂದು ಬ್ರಾಹ್ಮಣದ ಭಾಗವಾಗಿದ್ದು ಕಾಲಕ್ರಮೇಣ ಆ ಬ್ರಾಹ್ಮಣ ಉಪಯೋಗಕ್ಕೆ ಬರದೆ ಅಥವಾ ಮತ್ತಾವುದೊ ಕಾರಣಾಂತರದಿಂದ ಮಾಯವಾಗಿ ಹೋದಮೇಲೆ ಉಪನಿಷತ್ತುಗಳು ಮಾತ್ರ ಉಳಿದುಕೊಂಡಿವೆ. ಈ ಉಪನಿಷತ್ತುಗಳನ್ನು ಆರಣ್ಯಕ ಎಂತಲೂ ಕರೆಯುವರು.

ಆದಕಾರಣ ವೇದಾಂತವೇ ಹಿಂದೂಗಳ ಧರ್ಮಶಾಸ್ತ್ರವಾಗಿರುವುದು. ಎಲ್ಲಾ ಆಸ್ತಿಕ ಸಿದ್ಧಾಂತಗಳಿಗೂ ಅದು ತಳಹದಿಯಾಗಿದೆ. ಬೌದ್ಧರು ಮತ್ತು ಜೈನರು ಕೂಡ ತಮಗೆ ಅನುಕೂಲವೆಂದು ತೋರಿದಾಗ ವೇದಾಂತದಿಂದ ಉದಾಹರಿಸುವರು. ಭರತಖಂಡದ ದರ್ಶನಗಳೆಲ್ಲ ತಾವು ವೈದಿಕವೆಂದು ಹೇಳಿಕೊಂಡರೂ ಬೇರೆ ಬೇರೆ ಹೆಸರುಗಳನ್ನು ಪಡೆಯುವುವು. ಕೊನೆಗೆ ಬಂದ ವ್ಯಾಸರ ಸಿದ್ಧಾಂತ ಇತರ ದರ್ಶನಗಳಿಗಿಂತ ಹೆಚ್ಚಾಗಿ ವೇದಾಂತವನ್ನು ಆಶ್ರಯಿಸಿ ಅದಕ್ಕಿಂತ ಹಿಂದೆ ಬಂದ ಸಾಂಖ್ಯ ನ್ಯಾಯ ಮುಂತಾದುವನ್ನು ವೇದಾಂತದೊಂದಿಗೆ ಒಂದುಗೂಡಿಸಲು ಯತ್ನಿಸಿತು. ಆದಕಾರಣವೇ ಅದನ್ನು ಪ್ರತ್ಯೇಕವಾಗಿ ವೇದಾಂತ ದರ್ಶನ ಎನ್ನುವರು. ವ್ಯಾಸಸೂತ್ರವೇ ಈಗಿನ ಕಾಲದಲ್ಲಿ ವೇದಾಂತ ದರ್ಶನಕ್ಕೆ ತಳಹದಿ. ವ್ಯಾಸಸೂತ್ರಕ್ಕೆ ಹಲವು ಜನರು ಹಲವು ಭಾಷ್ಯಗಳನ್ನು ಬರೆದಿರುವರು. ಈಗ ಸಾಧಾರಣವಾಗಿ ಭಾರತ ದೇಶದಲ್ಲಿ ಮೂರು ವರ್ಗದ ಭಾಷ್ಯಗಳಿವೆ. ಅವುಗಳಿಂದ ಮೂರು ತತ್ತ್ವಶಾಖೆಗಳೂ ಮತ್ತು ಮೂರು ಪಂಥಗಳೂ ಉದಿಸಿವೆ. ಮೊದಲನೆಯದು ದ್ವೈತ, ಎರಡನೆಯದು ವಿಶಿಷ್ಟಾದ್ವೈತ, ಮೂರನೆಯದು ಅದ್ವೈತ. ದ್ವೈತಿಗಳು ಮತ್ತು ವಿಶಿಷ್ಟಾದ್ವೈತಿಗಳ ಸಂಖ್ಯೆ ಹೆಚ್ಚು. ಅದ್ವೈತಿಗಳ ಸಂಖ್ಯೆ ಬಹಳ ಕಡಮೆ. ಈ ಮೂರು ಸಿದ್ಧಾಂತಗಳ ಭಾವನೆಯನ್ನು ಈಗ ನಿಮ್ಮ ಮುಂದೆ ಇಡಲು ನಾನು ಯತ್ನಿಸುತ್ತೇನೆ. ಆದರೆ ಹಾಗೆ ಮಾಡುವುದಕ್ಕೆ ಮುಂಚೆ ಈ ಮೂರು ಸಿದ್ಧಾಂತಗಳಿಗೂ ಸಾಮಾನ್ಯವಾದ ಒಂದು ಮನಃಶ್ಶಾಸ್ತ್ರ ಇದೆ. ಅದು ಸಾಂಖ್ಯರ ಮನಶ್ಶಾಸ್ತ್ರ ಎಂದು ಹೇಳಬಯಸುತ್ತೇನೆ. ಅದು ನ್ಯಾಯ ವೈಶೇಷಿಕಗಳಂತೆಯೇ ಬಹುಮಟ್ಟಿಗೆ ಇದೆ. ಆದರೆ ಎಲ್ಲೊ ಗೌಣವಾದ ವಿಷಯಗಳಲ್ಲಿ ಮಾತ್ರ ಸ್ವಲ್ಪ ವ್ಯತ್ಯಾಸವಿದೆ.

ವೇದಾಂತಿಗಳು ಈ ಮೂರು ವಿಷಯಗಳನ್ನು ಒಪ್ಪಿಕೊಳ್ಳುವರು. ಅವರೆಲ್ಲ ಈಶ್ವರನನ್ನು ನಂಬುವರು. ವೇದಗಳು ಅಪೌರುಷೇಯ ಎನ್ನುವರು ಮತ್ತು ಕಲ್ಪಗಳನ್ನು ಒಪ್ಪುವರು. ವೇದಗಳನ್ನು ಕುರಿತು ಹೇಳಿದ್ದಾಗಿದೆ. ಕಲ್ಪಗಳ ವಿಷಯದಲ್ಲಿ ಅವರ ಭಾವನೆ ಹೀಗಿದೆ: ಪ್ರಪಂಚದಲ್ಲಿರುವ ದ್ರವ್ಯವೆಲ್ಲ ಆಕಾಶವೆಂಬ ಮೂಲದ್ರವ್ಯದಿಂದ ಬಂದಿದೆ. ಎಲ್ಲಾ ವಿಧವಾದ ಶಕ್ತಿಗಳು, ಆಕರ್ಷಣೆ, ವಿಕರ್ಷಣೆ, ಜೀವನ ಮುಂತಾದುವೆಲ್ಲ ಪ್ರಾಣವೆಂಬ ಆದಿಶಕ್ತಿಯ ಆವಿರ್ಭಾವಗಳು. ಪ್ರಾಣವು ಆಕಾಶದ ಮೇಲೆ ತನ್ನ ಪ್ರಭಾವವನ್ನು ಬೀರಿ ವಿಶ್ವವನ್ನು ಸೃಷ್ಟಿಸುತ್ತದೆ. ಕಲ್ಪದ ಆದಿಯಲ್ಲಿ ಆಕಾಶವು ಸ್ಪಂದನೆಯಿಲ್ಲದೆ ಇರುವುದು, ಅವ್ಯಕ್ತವಾಗಿರುವುದು. ಆಗ ಪ್ರಾಣ ಹೆಚ್ಚು ಹೆಚ್ಚು ಕ್ರಿಯಾಶೀಲವಾಗುವುದು. ಆಕಾಶದಿಂದ ಸ್ಥೂಲವಾದ ಸಸ್ಯ ಪ್ರಾಣಿ ಮನುಷ್ಯ ಗ್ರಹ ತಾರೆ ಇವುಗಳನ್ನು ಸೃಷ್ಟಿಸುವುದು. ಬಹಳ ಕಾಲವಾದ ಮೇಲೆ ಈ ವಿಕಸನ ನಿಂತು ತಿರೋಧಾನ \enginline{(involution)} ಪ್ರಾರಂಭವಾಗುವುದು. ಎಲ್ಲವೂ ಆಕಾಶದ ಮತ್ತು ಪ್ರಾಣದ ಸೂಕ್ಷ್ಮಸ್ಥಿತಿಗೆ ಹಿಂತಿರುಗುವುವು. ಆಗ ಹೊಸದೊಂದು ಕಲ್ಪ ಪ್ರಾರಂಭವಾಗುವುದು. ಆಕಾಶಕ್ಕೆ ಮತ್ತು ಪ್ರಾಣಕ್ಕೆ ಅತೀತವಾಗಿರುವ ವಸ್ತು ಒಂದಿದೆ. ಇವೆರಡೂ ಮಹತ್ತಿನಲ್ಲಿ ಅಥವಾ ವಿಶ್ವಮನಸ್ಸಿನಲ್ಲಿ ಲೀನವಾಗುವುವು. ಈ ಮಹತ್ತು ಆಕಾಶವನ್ನು ಮತ್ತು ಪ್ರಾಣವನ್ನು ಸೃಷ್ಟಿಸುವುದಿಲ್ಲ, ಆದರೆ ಅದೇ ಅವುಗಳಾಗಿ ಪರಿವರ್ತನಗೊಳ್ಳುವುದು.

ಈಗ ನಾವು ಮನಸ್ಸು, ಆತ್ಮ ಮತ್ತು ದೇವರೆಂಬ ಭಾವನೆಗಳನ್ನು ತೆಗೆದುಕೊಳ್ಳೋಣ. ಸಾಂಖ್ಯರ ಮನಃಶ್ಶಾಸ್ತ್ರದಲ್ಲಿ ಎಲ್ಲರೂ ಒಪ್ಪಿಕೊಂಡಿರುವ ಈ ಭಾವನೆ ಇದೆ: ದೃಶ್ಯ – ಗ್ರಹಣದ ಹಿಂದೆ ಮೊದಲು ಅದನ್ನು ನೋಡುವ ಕರಣಗಳಾದ ಕಣ್ಣುಗಳಿವೆ. ಈ ಕಣ್ಣುಗಳ ಹಿಂದೆ ನೋಡುವುದಕ್ಕೆ ಸಹಾಯಕವಾದ ಇಂದ್ರಿಯ ಅಥವಾ ಆಪ್ಟಿಕ್ ನರ ಮತ್ತು ಅದರ ಕೇಂದ್ರವಿದೆ. ಅದು ಹೊರಗೆ ಇಲ್ಲ, ಆದರೆ ಅದಿಲ್ಲದೇ ಕಣ್ಣು ನೋಡಲಾರದು. ನಾವು ಒಂದು ವಸ್ತುವನ್ನು ನೋಡಬೇಕಾದರೆ ಮತ್ತೊಂದು ಬೇಕಾಗುವುದು – ಮನಸ್ಸು ಆ ಇಂದ್ರಿಯದೊಂದಿಗೆ ಸಂಬಂಧವನ್ನು ಕಲ್ಪಿಸಿಕೊಳ್ಳಬೇಕು. ಇದಲ್ಲದೆ ಈ ಸಂವೇದನೆಯನ್ನು \enginline{(sensation)} ಬುದ್ಧಿಗೆ ಒಯ್ಯಬೇಕು. ಬುದ್ಧಿಯೇ ನಿಶ್ಚಯಾತ್ಮಕ ಮತ್ತು ಪ್ರತಿಕ್ರಿಯಾತ್ಮಕ ಮನಃಸ್ಥಿತಿ. ಅಲ್ಲಿಂದ ಪ್ರತಿಕ್ರಿಯೆ ಬಂದೊಡನೆಯೆ ಬಾಹ್ಯಜಗತ್ತು ಮತ್ತು ಅಹಂಕಾರ ತೋರುವುವು. ಇಲ್ಲಿಯೇ ಇಚ್ಛೆ ಇರುವುದು. ಇಲ್ಲಿಗೇ ಎಲ್ಲಾ ಪೂರ್ಣವಾಗಲಿಲ್ಲ. ಪ್ರತಿಯೊಂದು ಚಿತ್ರವೂ ಹಲವು ಜ್ಯೋತಿಸ್ಪಂದನಗಳಿಂದ ಆಗುವುದರಿಂದ ಆ ಚಿತ್ರ ನಮಗೆ ಕಾಣಬೇಕಾದರೆ ಅವೆಲ್ಲ ತಟಸ್ಥವಾದ ಒಂದು ತೆರೆಯ ಮೇಲೆ ಬೀಳುವುದು ಅವಶ್ಯಕವಾಗಿರುವಂತೆ, ಮನಸ್ಸಿನ ಭಾವನೆಗಳೆಲ್ಲ ಒಂದು ತಟಸ್ಥವಾದ ತೆರೆಯ ಮೇಲೆ ಬೀಳಬೇಕಾಗಿದೆ. ದೇಹದ ಮತ್ತು ಮನಸ್ಸಿನ ದೃಷ್ಟಿಯಿಂದ ತಟಸ್ಥವಾದುದೇ ಪುರುಷ ಅಥವಾ ಆತ್ಮ.

ಸಾಂಖ್ಯದರ್ಶನದ ಪ್ರಕಾರ ಬುದ್ದಿಯು ಮಹತ್ತಿನ ಆವಿರ್ಭಾವ. ಮಹತ್ತು\break ಸ್ಪಂದಿಸುವ ಆಲೋಚನೆಯಾಗುವುದು. ಅದರ ಒಂದು ಭಾಗ ಇಂದ್ರಿಯವಾಗುವುದು, ಮತ್ತೊಂದು ಭಾಗ ಸೂಕ್ಷ್ಮದ್ರವ್ಯವಾಗುವುದು. ಇವುಗಳ ಸಂಯೋಗದಿಂದಲೇ\break ಬ್ರಹ್ಮಾಂಡವೆಲ್ಲ ಸೃಷ್ಟಿಯಾಗಿರುವುದು. ಮಹತ್ತಿಗಿಂತಲೂ ಹಿಂದೆ ಅವ್ಯಕ್ತ ಇದೆ ಎಂದು ಸಾಂಖ್ಯರು ಹೇಳುವರು. ಅಲ್ಲಿ ಮನಸ್ಸು ಕೂಡ ಇರುವುದಿಲ್ಲ. ಅಲ್ಲಿ ಕೇವಲ ಕಾರಣಗಳು ಮಾತ್ರ ಇರುವುವು. ಅದನ್ನೇ ಪ್ರಕೃತಿ ಎನ್ನುವರು. ಈ ಪ್ರಕೃತಿಯಾಚೆ ಅದಕ್ಕಿಂತ ಎಂದೆಂದಿಗೂ ಬೇರೆಯಾಗಿ ಸಾಂಖ್ಯರ ಪುರುಷನು ಇರುವನು. ಅವನು ಗುಣಾತೀತ ಮತ್ತು ವಿಭು. ಪುರುಷ ಕರ್ತೃವಲ್ಲ, ಕೇವಲ ಸಾಕ್ಷಿ. ಪುರುಷನನ್ನು ವಿವರಿಸುವುದಕ್ಕೆ ಸ್ಪಟಿಕದ ಉದಾಹರಣೆಯನ್ನು ತರುವರು. ಪುರುಷ ಯಾವ ಬಣ್ಣವೂ ಇಲ್ಲದ ಸ್ಪಟಿಕಶಿಲೆಯಂತೆ ಇರುವನು. ಅದರ ಮುಂದೆ ಹಲವು ಬಣ್ಣಗಳು ಇರುವುದರಿಂದ ಸ್ಪಟಿಕಕ್ಕೆ ಆ ಬಣ್ಣಗಳಿವೆ ಎಂಬ ಭ್ರಾಂತಿ ಬರುವುದು. ಆದರೆ ನಿಜವಾಗಿ ಅದಕ್ಕೆ ಯಾವ ಬಣ್ಣವೂ ಇಲ್ಲ. ವೇದಾಂತಿಗಳು ಸಾಂಖ್ಯರ ಪುರುಷ ಮತ್ತು ಪ್ರಕೃತಿಗಳ ಭಾವನೆಗಳನ್ನು ಸ್ವೀಕರಿಸುವುದಿಲ್ಲ. ಇವೆರಡರ ಮಧ್ಯೆ ದೊಡ್ಡ ಅಂತರವಿದೆ ಎಂದು ಅವರು ಹೇಳುತ್ತಾರೆ. ಸಾಂಖ್ಯಸಿದ್ಧಾಂತ ಒಮ್ಮೆ ಪ್ರಕೃತಿಯ ಕಡೆ ಸೇರುವುದು, ಅನಂತರ ತಕ್ಷಣ ಅದು ಪ್ರಕೃತಿಗಿಂತ ಬೇರೆಯಾಗಿರುವ ಪುರುಷನ ಕಡೆ ನೆಗೆಯುವುದು. ಈ ಹಲವು ಬಣ್ಣಗಳು ಬಣ್ಣವೇ ಇಲ್ಲದ ಆತ್ಮವೆಂಬ ಆ ಸ್ಪಟಿಕದ ಮೇಲೆ ಹೇಗೆ ತಮ್ಮ ಪ್ರಭಾವವನ್ನು ಬೀರಬಲ್ಲವು? ಆದಕಾರಣ ವೇದಾಂತಿಗಳು ಪ್ರಾರಂಭದಲ್ಲೇ ಈ ಪ್ರಕೃತಿ ಮತ್ತು ಪುರುಷ ಎರಡೂ ಒಂದೇ ಎಂದು ಹೇಳುವರು. ದ್ವೈತಿಗಳು ಕೂಡ ದೇವರು ನಿಮಿತ್ತಕಾರಣ ಮಾತ್ರ ಅಲ್ಲ, ಉಪಾದಾನಕಾರಣ ಕೂಡ ಎನ್ನುವರು. ಅವರು ಸುಮ್ಮನೆ ಹೀಗೆ ಹೇಳುವರು. ಆದರೆ ನಿಜವಾದ ಅರ್ಥದಲ್ಲಿ ಹಾಗೆ ಹೇಳುವುದಿಲ್ಲ. ಅವರು ಮುಂದೆ ಹೇಳುವ ರೀತಿಯಲ್ಲಿ ಈ ನಿರ್ಣಯದಿಂದ ಪಾರಾಗಲು ಯತ್ನಿಸುವರು: ಈ ಪ್ರಪಂಚದಲ್ಲಿ ಮೂರು ವಸ್ತುಗಳಿವೆ. ಅವೇ ಈಶ್ವರ, ಜೀವ ಮತ್ತು ಜಗತ್ತು. ಜೀವ ಮತ್ತು ಜಗತ್ತು ದೇವರ ದೇಹವಿದ್ದಂತೆ. ಈ ಅರ್ಥದಲ್ಲಿ ದೇವರು ಮತ್ತು ಜಗತ್ತು ಒಂದು ಎನ್ನಬಹುದು. ಆದರೆ ಪ್ರಕೃತಿ ಮತ್ತು ಜೀವ ಎಂದೆಂದಿಗೂ ಬೇರೆ ಬೇರೆ ಇರುವುವು. ಕಲ್ಪದ ಆದಿಯಲ್ಲಿ ಅವು ವ್ಯಕ್ತವಾಗುವುವು. ಕಲ್ಪದ ಅಂತ್ಯದಲ್ಲಿ ಸೂಕ್ಷ್ಮವಾಗಿ ಆ ಸ್ಥಿತಿಯಲ್ಲೇ ಇರುವುವು. ಅದ್ವೈತ ವೇದಾಂತಿಗಳು ಆತ್ಮದ ಈ ಸಿದ್ಧಾಂತವನ್ನು ನಿರಾಕರಿಸುವರು. ಉಪನಿಷತ್ತುಗಳೆಲ್ಲಾ ಅದರ ಪಕ್ಷದಲ್ಲೇ ಇರುವುದರಿಂದ ಅವರು ಉಪನಿಷತ್ತುಗಳ ಆಧಾರದ ಮೇಲೆಯೇ ತಮ್ಮ ದರ್ಶನವನ್ನು ಸ್ಥಾಪಿಸಿರುವರು. ಉಪನಿಷತ್ತುಗಳಲ್ಲೆಲ್ಲ ಒಂದು ವಿಷಯವಿದೆ, ಅವು ಮಾಡಬೇಕಾದ ಒಂದು ಕೆಲಸವಿದೆ. ಅದೇ ಇದು: “ನಾವು ಹೇಗೆ ಒಂದು ಹಿಡಿ ಮೃತ್ತಿಕೆಯನ್ನು ಅರಿತರೆ ಪ್ರಪಂಚದ ಮೃತ್ತಿಕೆಯನ್ನೆಲ್ಲ ತಿಳಿಯಬಹುದೋ, ಹಾಗೆಯೇ ಯಾವುದನ್ನು ಅರಿತರೆ ಈ ಪ್ರಪಂಚವನ್ನೆಲ್ಲ ಅರಿಯಬಹುದು?” ಅದ್ವೈತಿಗಳು ಈ ಪ್ರಪಂಚವನ್ನೆಲ್ಲ ಒಂದರ ಅಡಿಗೆ ತರಲು ಯತ್ನಿಸುವರು. ಅದೇ ಈ ವಿಶ್ವದ ಸಮಷ್ಟಿ. ಈ ವಿಶ್ವವೆಲ್ಲ ಒಂದು; ಆ ಒಂದು ಹಲವು ವಿಧಗಳಲ್ಲಿ ಕಾಣುತ್ತಿದೆ. ಸಾಂಖ್ಯರು ಹೇಳುವ ಪ್ರಕೃತಿಯನ್ನು ವೇದಾಂತಿಗಳು ಒಪ್ಪುವರು. ಆದರೆ ವೇದಾಂತಿಗಳು ಪ್ರಕೃತಿಯನ್ನು ದೇವರು ಎನ್ನುವರು. ಆ ದೇವರೇ ಜಗತ್ತು. ಆತ್ಮ, ಜೀವ, ಮನುಷ್ಯ ಎಂದು ಅನಂತ ರೂಪಗಳನ್ನು ಧರಿಸಿರುವುದು. ಮನಸ್ಸು ಮತ್ತು ಮಹತ್ತು ಆ ಒಂದರ ಅಭಿವ್ಯಕ್ತಿಗಳು. ಆದರೆ ಇಲ್ಲೊಂದು ತೊಂದರೆ ಬರುವುದು. ಅದೇ ವಿಶ್ವವನ್ನೇ ದೇವರು ಎಂದಂತೆ ಆಗುವುದು. ಅವಿಕಾರಿಯಾಗಿರುವ ಸತ್ ವಿಕಾರಿಯಾದ ಮತ್ತು ನಾಶವಾಗುವ ಪ್ರಪಂಚ ಹೇಗೆ ಆಯಿತು? ಅದ್ವೈತಿಗಳು ಇಲ್ಲಿ ಒಂದು ಸಿದ್ಧಾಂತವನ್ನು ತರುವರು, ಅದೇ ವಿವರ್ತವಾದ. ದ್ವೈತಿಗಳ ಮತ್ತು ಸಾಂಖ್ಯರ ದೃಷ್ಟಿಯಲ್ಲಿ ಈ ವಿಶ್ವವೆಲ್ಲ ಮೂಲ ಪ್ರಕೃತಿಯ ವಿಕಸನ. ಕೆಲವು ಅದ್ವೈತಿಗಳ ಮತ್ತು ದ್ವೈತಿಗಳ ದೃಷ್ಟಿಯಲ್ಲಿ ಈ ಪ್ರಪಂಚವೆಲ್ಲ ದೇವರಿಂದ ಬಂದಿದೆ. ಆದರೆ ಶಂಕರಾಚಾರ್ಯರ ಅನುಯಾಯಿಗಳಾದ ನಿಜವಾದ ಅದ್ವೈತಿಗಳ ದೃಷ್ಟಿಯಲ್ಲಿ ಈ ಪ್ರಪಂಚವೆಲ್ಲ ದೇವರ ಒಂದು ತೋರಿಕೆಯ ಸೃಷ್ಟಿ ಮಾತ್ರ. ದೇವರು ಈ ಪ್ರಪಂಚದ ಉಪಾದಾನ ಕಾರಣ; ಆದರೆ ನಿಜವಾಗಿ ಅಲ್ಲ. ಕೇವಲ ತೋರಿಕೆಯ ದೃಷ್ಟಿಯಿಂದ. ಇದನ್ನು ವಿವರಿಸುವುದಕ್ಕೆ ಅವರು ಕೊಡುವ ಪ್ರಖ್ಯಾತವಾದ ಉಪಮಾನವೆ ರಜ್ಜು – ಸರ್ಪಗಳ ಉಪಮಾನ. ಹಗ್ಗವು ಸರ್ಪದಂತೆ ಕಾಣುತ್ತದೆಯೇ ಹೊರತು ಸರ್ಪವಾಗಿಲ್ಲ. ಹಗ್ಗ ಎಂದಿಗೂ ಹಾವಾಗಲಿಲ್ಲ. ಹಗ್ಗವು ಹಾವಿನಂತೆ ಕಂಡಿತು, ಅಷ್ಟೆ. ಆದರೆ ನಿಜವಾಗಿಯೂ ಹಾಗೆ ಆಗಲಿಲ್ಲ. ಇದರಂತೆಯೇ ಈ ವಿಶ್ವವು ಕೂಡ ದೇವರಿಂದ ಆಗಿದೆ. ಇದು ಯಾವಾಗಲೂ ಅವಿಕಾರಿ, ಕಾಣುವ ವಿಕಾರಗಳೆಲ್ಲ ಕೇವಲ ತೋರಿಕೆ. ಇವುಗಳೆಲ್ಲ ಕಾಲದೇಶನಿಮಿತ್ತಗಳಿಂದ, ಅಥವಾ ಮತ್ತೂ ಶ್ರೇಷ್ಠವಾದ ಮನಶ್ಶಾಸ್ತ್ರದ ಪದವನ್ನು ಉಪಯೋಗಿಸುವುದಾದರೆ, ನಾಮರೂಪಗಳಿಂದ ಆಗಿವೆ. ನಾಮರೂಪಗಳೇ ಒಂದಕ್ಕೂ ಮತ್ತೊಂದಕ್ಕೂ ಇರುವ ವ್ಯತ್ಯಾಸಕ್ಕೆ ಕಾರಣ. ಸತ್ಯವಾಗಿ ಇರುವುದೊಂದೆ. ಪುನಃ, ವೇದಾಂತಿಗಳು ಕಾರ್ಯ ಬೇರೆ, ಕಾರಣ ಬೇರೆ ಇದೆ ಎಂದು ಒಪ್ಪಿಕೊಳ್ಳುವುದಿಲ್ಲ. ಕೇವಲ ತೋರಿಕೆಗೆ ಮಾತ್ರ ಹಗ್ಗ ಹಾವಿನಂತೆ ಆಗಿದೆ. ಭ್ರಾಂತಿ ಹೋದರೆ ಸರ್ಪವು ತೊಲಗುವುದು. ಒಬ್ಬನು ಅಜ್ಞಾನದಲ್ಲಿರುವಾಗ ಕೇವಲ ಪ್ರಪಂಚವನ್ನು ಮಾತ್ರ ನೋಡುತ್ತಾನೆ, ದೇವರನ್ನು ನೋಡುವುದಿಲ್ಲ. ಅವನು ದೇವರನ್ನು ನೋಡಿದಾಗ ಈ ಪ್ರಪಂಚ ಸಂಪೂರ್ಣ ಮಾಯವಾಗುವುದು. ಮಾಯೆಯೇ ಸೃಷ್ಟಿಗೆ ಕಾರಣ. ಅವಿಕಾರಿಯಾದುದನ್ನು ಮತ್ತು ನಿರಪೇಕ್ಷವಾದುದನ್ನು ಈ ಪ್ರಪಂಚವೆಂದು ತಪ್ಪು ತಿಳಿಯುವೆವು. ಮಾಯೆ ಸಂಪೂರ್ಣ ಶೂನ್ಯವಲ್ಲ, ನಾಸ್ತಿಯಲ್ಲ. ಅದನ್ನು ಇದೆ ಮತ್ತು ಇಲ್ಲ ಎನ್ನುವರು. ಅದು ಅವ್ಯಕ್ತದ ದೃಷ್ಟಿಯಿಂದ ಇಲ್ಲ. ಅಖಂಡದ ದೃಷ್ಟಿಯಿಂದ ಬ್ರಹ್ಮ ಒಂದೇ ಇರಲು ಸಾಧ್ಯ, ಮಿಕ್ಕ ಯಾವುದೂ ಇರಲಾರದು. ಆದರೆ ಅದನ್ನು ಇಲ್ಲವೆಂತಲೂ ಹೇಳಲಾಗುವುದಿಲ್ಲ. ಏಕೆಂದರೆ ಅದು ಇಲ್ಲದೆ ಇದ್ದರೆ ಈ ಪ್ರಪಂಚವೇ ಇರುತ್ತಿರಲಿಲ್ಲ. ಆದಕಾರಣ ಅದನ್ನು ಇದೆ ಅಥವಾ ಇಲ್ಲ ಎನ್ನಲಾಗುವುದಿಲ್ಲ. ಅದನ್ನು ವೇದಾಂತದಲ್ಲಿ ಅನಿರ್ವಚನೀಯ ಎನ್ನುವರು. ನಿಜವಾಗಿ ಮಾಯೆಯೇ ಈ ಪ್ರಪಂಚಕ್ಕೆ ಕಾರಣವಾಯಿತು. ದೇವರು ಯಾವುದಕ್ಕೆ ಉಪಾದಾನವನ್ನು ಕೊಡುವನೋ ಅದಕ್ಕೆ ಮಾಯೆ ನಾಮರೂಪಗಳನ್ನು ನೀಡುವುದು. ಬ್ರಹ್ಮವೇ ಅನಂತರ ಇದೆಲ್ಲ ಆದಂತೆ ನಮಗೆ ತೋರುವುದು. ಅದ್ವೈತಿಗಳ ದೃಷ್ಟಿಯಲ್ಲಿ ಪ್ರತ್ಯೇಕ ಆತ್ಮನಿಗೆ ಸ್ಥಳವಿಲ್ಲ. ಪ್ರತ್ಯೇಕ ಆತ್ಮರೆಲ್ಲ ಮಾಯಾಜನಿತರು. ನಿಜವಾಗಿ ಅವು ಇರಲಾರವು. ಇರುವುದು ಒಂದೇ. ಆದರೆ, ನಾವು ನೀವು ಎಲ್ಲಾ ಬೇರೆ ಬೇರೆ ಇರುವುದಕ್ಕೆ ಹೇಗೆ ಸಾಧ್ಯ? ನಾವೆಲ್ಲ ಒಂದು. ಎರಡನ್ನು ಕಾಣುವುದೇ ಅಜ್ಞಾನಕ್ಕೆಲ್ಲ ಮೂಲ. ನಾನು ಪ್ರಪಂಚದಿಂದ ಬೇರೆ ಎಂದು ತಿಳಿದೊಡನೆಯೆ ಮೊದಲು ಅಂಜಿಕೆ, ಅನಂತರ ದುಃಖ ಬರುವುದು. ಎಲ್ಲಿ ಒಬ್ಬ ಮತ್ತೊಂದನ್ನು ಕೇಳುವನೊ ಮತ್ತೊಂದನ್ನು ನೋಡುವನೊ ಅದು ಅಲ್ಪ; ಎಲ್ಲಿ ಮತ್ತೊಂದನ್ನು ನೋಡುವುದಿಲ್ಲವೊ ಮತ್ತೊಂದನ್ನು ಕೇಳುವುದಿಲ್ಲವೋ ಅದೇ ಭೂಮ, ಅದೇ ಬ್ರಹ್ಮ. ಆ ಭೂಮದಲ್ಲಿ ಮಾತ್ರ ಸುಖ. ಅಲ್ಪದಲ್ಲಿ ಸುಖವಿಲ್ಲ.”

ಅದ್ವೈತ ಸಿದ್ಧಾಂತದ ಪ್ರಕಾರ ನಾಮರೂಪಗಳ ವೈವಿಧ್ಯತೆ ಮನುಷ್ಯನ ನೈಜಸ್ವಭಾವವನ್ನು ಮರೆಸಿರುವುದು. ಆದರೆ ನಿಜವಾಗಿ ಮನುಷ್ಯನ ನಿಜ ಸ್ವಭಾವವು ವಿಕಾರ ಹೊಂದಿಲ್ಲ. ಕ್ಷುದ್ರತಮ ಕೀಟದಿಂದ ಹಿಡಿದು ಶ್ರೇಷ್ಠತಮ ಮಾನವನವರೆಗೆ ಅದೇ ದಿವ್ಯ ಸ್ವರೂಪವು ಇರುವುದು. ಕೀಟದ ರೂಪ ಅತಿ ಕೀಳು. ಅಲ್ಲಿ ದಿವ್ಯತೆಯ ಬಹುಭಾಗ ಮಾಯೆಯಿಂದ ಮರೆಯಾಗಿರುವುದು. ಎಲ್ಲಿ ದಿವ್ಯತೆಗೆ ಆತಂಕ ಬಹಳ ಕಡಮೆ ಇದೆಯೋ ಅದೇ ಶ್ರೇಷ್ಠ. ಪ್ರತಿಯೊಂದು ವಸ್ತುವಿನ ಹಿಂದೆ ಒಂದೇ ದಿವ್ಯತೆ ಹುದುಗಿದೆ. ಇದೇ ನೀತಿಗೆ ಆಧಾರ. ಮತ್ತೊಬ್ಬನನ್ನು ಹಿಂಸಿಸಬೇಡಿ, ಎಲ್ಲರನ್ನೂ ನಿಮ್ಮಂತೆಯೇ ನೋಡಿ, ಏಕೆಂದರೆ ಪ್ರಪಂಚವೆಲ್ಲ ಒಂದು. ಮತ್ತೊಬ್ಬನನ್ನು ಹಿಂಸಿಸಿದರೆ ನಾನೇ ಹಿಂಸೆಮಾಡಿಕೊಳ್ಳುತ್ತಿರುವೆನು, ಮತ್ತೊಬ್ಬನನ್ನು ಪ್ರೀತಿಸಿದರೆ ನಾನೇ ಪ್ರೀತಿಸಿಕೊಳ್ಳುತ್ತಿರುವೆನು. ಇದರಿಂದಲೇ ಅದ್ವೈತನೀತಿಯ ಮೂಲಸಿದ್ಧಾಂತ ವ್ಯಕ್ತವಾಗುವುದು – ಅದನ್ನು 'ಆತ್ಮ ನಿರಾಕರಣೆ' ಎಂದು ಒಂದು ಪದದಲ್ಲಿಡಬಹುದು. ಈ ಅಲ್ಪವಾದ ಕ್ಷುದ್ರ ವ್ಯಕ್ತಿತ್ವವೇ ನನ್ನ ದುಃಖಕ್ಕೆಲ್ಲ ಕಾರಣ ಎನ್ನುವನು ಅದ್ವೈತಿ. ನಾನು ಇತರರಿಗಿಂತ ಬೇರೆ ಎಂದು ಭಾವಿಸಿರುವುದೇ ದ್ವೇಷ, ಅಸೂಯೆ, ದುಃಖ, ಹೋರಾಟ ಮುಂತಾದ ಪಾಪಗಳಿಗೆಲ್ಲ ಕಾರಣ. ನಾವು ಈ ಭಾವನೆಯಿಂದ ಪಾರಾದರೆ, ಆಗ ಹೋರಾಟವೆಲ್ಲ ನಿಲ್ಲುವುದು, ದುಃಖವೆಲ್ಲ ಕೊನೆಗಾಣುವುದು. ಆದಕಾರಣವೇ ಈ ಅಲ್ಪ ವ್ಯಕ್ತಿತ್ವವನ್ನು ತ್ಯಜಿಸಬೇಕು. ನಮ್ಮ ಪ್ರಾಣವನ್ನು ಎಂತಹ ತುಚ್ಛ ಪ್ರಾಣಿಗಾಗಿಯಾದರೂ ಸಮರ್ಪಿಸಲು ಸದಾ ಸಿದ್ಧವಾಗಿರಬೇಕು. ಒಂದು ಸಣ್ಣ ಕೀಟಕ್ಕಾಗಿಯೂ ತನ್ನ ಪ್ರಾಣವನ್ನು ಬಲಿದಾನಮಾಡಲು ಸಿದ್ದರಾದರೆ ಆಗ ಅದ್ವೈತಿಯ ಪರಮ ಗುರಿಯನ್ನು ಮುಟ್ಟಿದಂತೆ. ಅವನು ಹೀಗೆ ಸಿದ್ದನಾದಾಗ ಅಜ್ಞಾನದ ತೆರೆ ಅವನಿಂದ ಕಳಚಿ ಬೀಳುವುದು. ಅವನು ತನ್ನ ನೈಜಸ್ವಭಾವವನ್ನು ಅರಿಯುವನು. ಅವನು ಬದುಕಿರುವಾಗಲೇ ಜಗತ್ತೆಲ್ಲ ತಾನು ಎಂದು ಭಾವಿಸುವನು. ಅವನಿಗೆ ಆಗ ಬಾಹ್ಯ ವಿಶ್ವವೆಲ್ಲ ಮಾಯವಾಗಿ ಅವನು ತನ್ನ ನೈಜಸ್ವಭಾವವನ್ನು ಅರಿಯುವನು. ಆದರೆ ಈ ದೇಹದ ಕರ್ಮ ಇರುವವರೆಗೆ ಅವನು ಜೀವಿಸಿರಬೇಕಾಗಿದೆ. ಅಜ್ಞಾನ ಹೋದ ಮೇಲೆ, ಅವನು ಕೆಲವು ಕಾಲ ದೇಹದಲ್ಲಿರುವ ಸ್ಥಿತಿಯನ್ನು ಜೀವನ್ಮುಕ್ತಿ ಎನ್ನುವರು. ಒಬ್ಬ ಸ್ವಲ್ಪಕಾಲ ಮರೀಚಿಕೆಯಿಂದ ಭ್ರಾಂತನಾಗಿ, ಒಂದು ಸಲ ಆ ಭ್ರಾಂತಿ ಮಾಯವಾಗಿ ಅದು ಪುನಃ ಮಾರನೆಯ ದಿನ ಬಂದರೆ, ಅವನು ಪುನಃ ಅದರಿಂದ ಭ್ರಾಂತನಾಗುವುದಿಲ್ಲ. ಭ್ರಾಂತಿ ಹೋಗುವುದಕ್ಕಿಂತ ಮುಂಚೆ ನಿಜ ಯಾವುದು ಸುಳ್ಳು ಯಾವುದು ಎನ್ನುವುದನ್ನು ಅವನು ತಿಳಿಯಲಾರ. ಆದರೆ ಒಮ್ಮೆ ಅಜ್ಞಾನ ಹೊದರೆ ಅವನ ಇಂದ್ರಿಯಗಳು ಇರುವವರೆಗೆ ಅವನು ಭ್ರಾಂತಿಯನ್ನು ನೋಡುತ್ತಾನೆ, ಆದರೆ ಅದರ ಬಲೆಗೆ ಬೀಳುವುದಿಲ್ಲ. ಮರೀಚಿಕೆ ಯಾವುದು, ಮರಳುಗಾಡು ಯಾವುದು ಎಂಬ ತಾರತಮ್ಯವನ್ನು ಅವನು ಚೆನ್ನಾಗಿ ಅರಿತಿರುವನು. ಮರೀಚಿಕೆ ಪುನಃ ಅವನನ್ನು ಭ್ರಾಂತಿಗೊಳಿಸಲಾರದು. ವೇದಾಂತಿ ತನ್ನ ಸ್ವಭಾವವನ್ನು ಅರಿತ ಮೇಲೆ ಈ ಪ್ರಪಂಚವೇ ಅವನಿಗೆ ಮಾಯವಾಗುತ್ತದೆ. ಜಗತ್ತು ಅವನಿಗೆ ಪುನಃ ಕಂಡರೂ ಅದು ಹಿಂದಿನಂತೆ ದುಃಖದ ಜಗತ್ತಾಗಿರುವುದಿಲ್ಲ. ದುಃಖದ ಸೆರೆಮನೆ ಸಚ್ಚಿದಾನಂದವಾಗುವುದು. ಅದನ್ನು ಪಡೆಯುವುದೇ ಅದ್ವೈತ ದರ್ಶನದ ಗುರಿ.\footnote{ ಮೇಲಿನ ಉಪನ್ಯಾನವು 1896, ಮಾರ್ಚ್ 25ರಂದು ಹಾರ್ವರ್ಡ್ ವಿಶ್ವವಿದ್ಯಾಲಯದ ಸ್ನಾತಕ ತತ್ತ್ವಶಾಸ್ತ್ರ ಸೊಸೈಟಿಯಲ್ಲಿ ನೀಡಿದುದಾಗಿದೆ}

