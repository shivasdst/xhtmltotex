
\chapter[ಮನಶ್ಶಕ್ತಿ]{ಮನಶ್ಶಕ್ತಿ\protect\footnote{\engfoot{* C.W, Vol. II, P. 10}}}

\begin{center}
(೧೯೦೦ರ ಜನವರಿ ೮ರಂದು ಲಾಸ್‌ಏಂಜಲೀಸ್‌ನಲ್ಲಿ ನೀಡಿದ ಪ್ರವಚನ)
\end{center}

ಪ್ರಪಂಚದಲ್ಲೆಲ್ಲಾ, ಎಲ್ಲಾ ಕಾಲದಲ್ಲಿಯೂ ಭೌತಿಕ ಅತಿ ಶಕ್ತಿಯಲ್ಲಿ ನಂಬಿಕೆ ಇದೆ. ನಾವೆಲ್ಲ ಅಸಾಮಾನ್ಯ ಘಟನೆಗಳ ವಿಷಯವನ್ನು ಕೇಳಿರುವೆವು. ನಮ್ಮಲ್ಲಿ ಕೆಲವರಿಗೆ\break ಅದರ ಪ್ರತ್ಯಕ್ಷ ಅನುಭವವೂ ಇದೆ. ನನ್ನ ಅನುಭವಕ್ಕೆ ಬಂದ ಕೆಲವು ವಿಷಯಗಳನ್ನು ಹೇಳಿ ಇಂದಿನ ವಿಷಯವನ್ನು ಪ್ರಾರಂಭಿಸುವೆನು. ಯಾವುದಾದರೂ ಪ್ರಶ್ನೆಯನ್ನು ಮನಸ್ಸಿನಲ್ಲಿ ಇಟ್ಟುಕೊಂಡು ಹೋದರೆ ಅದಕ್ಕೆ ತಕ್ಷಣ ಉತ್ತರ ಕೊಡುವವನೊಬ್ಬನಿದ್ದ. ಆತನು ಭವಿಷ್ಯವನ್ನೂ ಹೇಳುವನೆಂಬುದನ್ನು ಕೇಳಿದ್ದೆ. ನನಗೆ ಕುತೂಹಲವಾಗಿ ಕೆಲವು ಸ್ನೇಹಿತರೊಂದಿಗೆ ನೋಡಲು ಹೋದೆ. ನಮ್ಮಲ್ಲಿ ಪ್ರತಿಯೊಬ್ಬರ ಮನಸ್ಸಿನಲ್ಲಿಯೂ ಅವನನ್ನು ಕೇಳುವುದಕ್ಕೆ ಪ್ರಶ್ನೆಗಳಿದ್ದುವು. ಮರೆತುಹೋಗದೆ ಇರಲಿ ಎಂದು ಅದನ್ನು ಒಂದು ಕಾಗದದಲ್ಲಿ ಬರೆದು ಜೇಬಿನಲ್ಲಿ ಇಟ್ಟುಕೊಂಡೆವು. ಆತ ನಮ್ಮಲ್ಲಿ ಒಬ್ಬೊಬ್ಬರನ್ನೂ ನೋಡಿದೊಡನೆಯೆ ಪ್ರಶ್ನೆಯನ್ನು ಅವನೇ ಹೇಳಿ ಅದಕ್ಕೆ ಉತ್ತರ ಕೊಟ್ಟನು. ಅನಂತರ ಆತ ಒಂದು ಕಾಗದದ ಮೇಲೆ ಏನನ್ನೋ ಬರೆದು ಅದನ್ನು ಮಡಿಸಿ ಅದರ ಹಿಂದೆ ನನ್ನ ಹೆಸರನ್ನು ಬರೆಯುವಂತೆ ಹೇಳಿದನು. "ಅದನ್ನು ತೆಗೆದು ನೋಡಬೇಡ. ಜೇಬಿನಲ್ಲಿ ಇಟ್ಟುಕೊ. ನಿಮ್ಮ ಪ್ರಶ್ನೆ ಮತ್ತು ಅದಕ್ಕೆ ಉತ್ತರ ಅದರಲ್ಲಿದೆ'' ಎಂದನು. ಹೀಗೆಯೇ ನಮ್ಮಲ್ಲಿ ಪ್ರತಿಯೊಬ್ಬರಿಗೂ ಮಾಡಿದನು. ಅನಂತರ ಆತ ನಮ್ಮ ಮುಂದಿನ ಜೀವನದಲ್ಲಿ ನಡೆಯಬಹುದಾದ ಕೆಲವು ಘಟನೆಗಳನ್ನು ಹೇಳಿದನು. ಅನಂತರ ಆತ 'ನಿಮ್ಮ ಮನಸ್ಸಿಗೆ ಬಂದ ಭಾಷೆಯಲ್ಲಿ ಯಾವುದಾದರೂ ಪದವನ್ನೊ, ವಾಕ್ಯವನ್ನೊ ಆಲೋಚಿಸಿ'' ಎಂದನು. ನಾನು ಅವನಿಗೆ ಪರಿಚಯವಿಲ್ಲದ ಸಂಸ್ಕೃತದಿಂದ ದೊಡ್ಡ ವಾಕ್ಯವನ್ನು ಜ್ಞಾಪಿಸಿಕೊಂಡೆ. ಅನಂತರ ಆತ ನಿಮ್ಮ ಜೇಬಿನಲ್ಲಿರುವ ಕಾಗದವನ್ನು ತೆಗೆದುನೋಡಿ ಎಂದನು. ಸಂಸ್ಕೃತ ವಾಕ್ಯ ಅಲ್ಲಿ ಬರೆದಿತ್ತು. ಆತ ಇದನ್ನು ಒಂದು ಗಂಟೆ ಮುಂಚೆ "ಈ ಮನುಷ್ಯ ನಾನು ಇಲ್ಲಿ ಬರೆದಿರುವುದನ್ನು ಆಲೋಚಿಸುತ್ತಾನೆ'' ಎಂದು ಬರೆದಿದ್ದನು. ಇದು ನಿಜವಾಗಿತ್ತು. ಮತ್ತೊಬ್ಬನಿಗೆ ಹೀಗೆಯೆ ಒಂದು ಕಾಗದ ಕೊಟ್ಟಿದ್ದನು. ಆತನು ತನ್ನ ಹೆಸರನ್ನು ರುಜುಮಾಡಿ ಆ ಪತ್ರವನ್ನು ಮಡಿಸಿ ಜೇಬಿನಲ್ಲಿ ಇಟ್ಟುಕೊಂಡಿದ್ದನು. ಆತನಿಗೂ ಯಾವುದಾದರೂ ವಾಕ್ಯವನ್ನು ಜ್ಞಾಪಿಸಿಕೊ ಎಂದು ಹೇಳಿದನು. ಆತ ಮಂತ್ರವಾದಿಗೆ ತಿಳಿದುಕೊಳ್ಳುವುದಕ್ಕೆ ಮತ್ತೂ ಕಷ್ಟವಾದ ಅರಬ್ಬಿ ಭಾಷೆಯಲ್ಲಿ ಒಂದು ವಾಕ್ಯವನ್ನು ಆಲೋಚಿಸಿಕೊಂಡನು. ಕೊರಾನಿನಿಂದ ಯಾವುದೊ ಕೆಲವು ಸುರ ಅದು. ನನ್ನ ಸ್ನೇಹಿತನಿಗೆ ತನ್ನ ಕಾಗದದಲ್ಲಿ ಇದನ್ನೇ ಬರೆದಿರುವುದು ಕಂಡಿತು.

ನಮ್ಮಲ್ಲಿ ಮತ್ತೊಬ್ಬ ವೈದ್ಯ ಜರ್ಮನ್ ವೈದ್ಯಗ್ರಂಥದಿಂದ ಒಂದು ವಾಕ್ಯವನ್ನು ಸ್ಮರಿಸಿಕೊಂಡನು. ಇದೂ ಕೂಡ ಅವನ ಕಾಗದದಲ್ಲಿ ಬರೆದಿತ್ತು.

ಹೇಗೊ ನಾನು ಹಿಂದೆ ಮೋಸ ಹೋಗಿದ್ದಿರಬಹುದು ಎಂದು ಕೆಲವು ದಿನಗಳ ಮೇಲೆ ಪುನಃ ನಾನು ಈ ಮನುಷ್ಯನ ಬಳಿಗೆ ಹೋದೆ, ಕೆಲವು ಇತರ ಸ್ನೇಹಿತರನ್ನೂ ಕರೆದುಕೊಂಡು ಹೋದೆ. ಈ ವೇಳೆಯೂ ಆತ ಅದ್ಭುತ ಆಶ್ಚರ್ಯಕರವಾಗಿಯೆ ವಿಜಯಿಯಾದನು.

ಮತ್ತೊಂದು ವೇಳೆ ನಾನು ಹೈದರಾಬಾದಿನಲ್ಲಿದ್ದೆ. ಅಲ್ಲಿ ಒಬ್ಬ ಬ್ರಾಹ್ಮಣ ಏನನ್ನು ಬೇಕಾದರೂ ತರಿಸಬಲ್ಲ, ಎಲ್ಲಿಂದ ಎಂಬುದು ಗೊತ್ತಾಗುವುದಿಲ್ಲ ಎಂಬುದನ್ನು ಕೇಳಿದೆ. ಆತ ಕೆಲಸದಲ್ಲಿದ್ದ. ಗೌರವಾನ್ವಿತ ಮನುಷ್ಯ. ಆತನ ಚತುರತೆಯನ್ನು ತೋರಿಸಬೇಕೆಂದು ಕೇಳಿಕೊಂಡೆ. ಆತನಿಗೆ ಜ್ವರ ಬಂದಿತ್ತು. ಭರತಖಂಡದಲ್ಲಿ ರೋಗಿಯ ತಲೆಯಮೇಲೆ ಸಾಧು ಕೈ ಇಟ್ಟರೆ ರೋಗಿ ಗುಣವಾಗುವನು ಎಂಬ ಸಾಧಾರಣ ನಂಬಿಕೆ ಇದೆ. ಈ ಬ್ರಾಹ್ಮಣನು ನನ್ನ ಹತ್ತಿರ ಬಂದು "ಸ್ವಾಮಿ, ನಿಮ್ಮ ಕೈಯನ್ನು ನನ್ನ ತಲೆಯ ಮೇಲೆ ಇಡಿ, ಜ್ವರ ಗುಣವಾಗಲಿ'' ಎಂದು ಕೇಳಿಕೊಂಡ. ಅದಕ್ಕೆ ನಾನು “ಆಗಲಿ, ನೀನು ನಿನ್ನ ಮಾಯಾಮಂತ್ರವನ್ನು ತೋರಬೇಕು" ಎಂದೆ. ಆತ ಆಗಲಿ ಎಂದು ಮಾತು ಕೊಟ್ಟನು. ನಾನು ಆತನ ತಲೆಯ ಮೇಲೆ ಕೈಯನ್ನು ಇಟ್ಟೆನು. ಅನಂತರ ಆತ ಮಾತಿನಂತೆ ತನ್ನ ಚಾತುರ್ಯವನ್ನು ತೋರಿಸಲು ಬಂದ. ಆತ ಸೊಂಟಕ್ಕೆ ಒಂದು ಬಟ್ಟೆಯನ್ನು ಮಾತ್ರ ಕಟ್ಟಿಕೊಂಡಿದ್ದ. ಅವನಿಂದ ಉಳಿದ ಬಟ್ಟೆಯನ್ನೆಲ್ಲ ತೆಗೆದುಕೊಂಡೆವು. ಚಳಿಯಾಗಿದ್ದುದ್ದರಿಂದ ನನ್ನ ಹತ್ತಿರ ಇದ್ದ ಕಂಬಳಿಯನ್ನು ಆತನಿಗೆ ಹೊದೆದು ಕೊಳ್ಳುವುದಕ್ಕೆ ಕೊಟ್ಟು ಒಂದು ಮೂಲೆಯಲ್ಲಿ ಅವನನ್ನು ಕೂಡಿಸಿದೆವು. ಇಪ್ಪತ್ತೈದು ಜನರ ಕಣ್ಣುಗಳು ಅವನನ್ನು ನೋಡುತ್ತಿದ್ದವು. ಆತ "ಈಗ ನೀವು ಏನುಬೇಕಾದರೂ ಬರೆಯಿರಿ'' ಎಂದನು. ನಾವೆಲ್ಲ ಆ ದೇಶದಲ್ಲಿ ಎಂದಿಗೂ ಬೆಳೆಯದ ದ್ರಾಕ್ಷಿಗೊನೆ, ಕಿತ್ತಳೇಹಣ್ಣು ಮುಂತಾದುವನ್ನು ಬರೆದೆವು. ಆ ಕಾಗದದ ಚೂರನ್ನು ಅವನಿಗೆ ಕೊಟ್ಟೆವು. ಆತ ಕಂಬಳಿಯ ಒಳಗಿನಿಂದ ರಾಶಿರಾಶಿ ದ್ರಾಕ್ಷಿ ಕಿತ್ತಳೆ ಮೊದಲಾದುವನ್ನೆಲ್ಲ ತಂದನು. ಆ ಹಣ್ಣಿನ ರಾಶಿಯನ್ನು ತೂಗಿದರೆ ಈ ಮನುಷ್ಯನ ಎರಡರಷ್ಟು ಇರುವಂತೆ ಇತ್ತು. ಆ ಹಣ್ಣನ್ನು ತಿನ್ನುವಂತೆ ಹೇಳಿದನು. ನಮ್ಮಲ್ಲಿ ಕೆಲವರಿಗೆ ಅನುಮಾನ - ಇದೆಲ್ಲ ಇಂದ್ರಜಾಲವೇನೊ ಎಂದು. ಆದರೆ ಈ ಮಂತ್ರವಾದಿಯೆ ಹಣ್ಣನ್ನು ತಿನ್ನಲು ಶುರುಮಾಡಿದನು. ಆಗ ನಾವೆಲ್ಲ ತಿಂದೆವು. ಅವು ಚೆನ್ನಾಗಿದ್ದುವು. ಕೊನೆಗೆ ಒಂದು ರಾಶಿ ಗುಲಾಬಿ ಹೂವನ್ನು ತೋರಿಸಿ ಆಟವನ್ನು ಪೂರೈಸಿದನು. ಪ್ರತಿಯೊಂದು ಗುಲಾಬಿ ಹೂವೂ ಹೊಸದಾಗಿತ್ತು, ದಳದ ಮೇಲೆ ಹಿಮಮಣಿ ಹೊಳೆಯುತ್ತಿತ್ತು. ಒಂದೂ ಬಾಡಿಹೋಗಿರಲಿಲ್ಲ. ಇವು ಒಂದು ರಾಶಿ ಇತ್ತು! ಇದು ಹೇಗೆ ಸಾಧ್ಯವೆಂದು ನಾನು ಆತನನ್ನು ಕೇಳಿದಾಗ “ಇದೆಲ್ಲ ಕೈಚಳಕ'' ಎಂದನು. ಅದು ಏನಾದರೂ ಆಗಲಿ ಬರೀ ಕೈಚಳಕವಾಗಿರಲಿಕ್ಕಿಲ್ಲ. ಎಲ್ಲಿಂದ ಆತನಿಗೆ ಇಷ್ಟೊಂದು ತರಲು ಸಾಧ್ಯವಾಗುತ್ತಿತ್ತು?

ಇಂತಹ ಎಷ್ಟೋ ಘಟನೆಗಳನ್ನು ನಾನು ನೋಡಿರುವನು. ಇಂಡಿಯಾ ದೇಶದಲ್ಲಿ ಸಂಚರಿಸುತಿದ್ದರೆ ಇಂತಹ ನೂರಾರು ವಿಷಯಗಳು ಕಾಣುವುವು. ಇವು ಪ್ರತಿಯೊಂದು ದೇಶದಲ್ಲಿಯೂ ಇವೆ. ಈ ದೇಶದಲ್ಲಿಯೂ ಇವೆ. ಈ ದೇಶದಲ್ಲಿಯೂ ಇಂತಹ ಕೆಲವು ಅದ್ಭುತ ವಿಷಯಗಳಿವೆ. ನಿಸ್ಸಂದೇಹವಾಗಿ ಎಷ್ಟೋ ಮೋಸವಿದೆ. ಆದರೆ ನೀವು ಯಾವಾಗ ಮೋಸವನ್ನು ನೋಡುತ್ತೀರೋ ಅದೊಂದು ಅನುಕರಣೆ ಎನ್ನಬೇಕಾಗಿದೆ. ಎಲ್ಲೋ ಒಂದು ಸತ್ಯವಿರಬೇಕು, ಅದನ್ನು ಅನುಕರಿಸುತ್ತಿರುವರು. ಎಂದಿಗೂ ಇಲ್ಲದೆ ಇರುವುದನ್ನು ಅನುಕರಿಸುವುದಕ್ಕೆ ಆಗುವುದಿಲ್ಲ. ಯಾವುದೋ ಸತ್ಯವಾಗಿರುವುದನ್ನು ಅನುಕರಿಸಬೇಕಾಗುವುದು.

ಭರತಖಂಡದಲ್ಲಿ ಅತಿ ಪುರಾತನ ಕಾಲದಲ್ಲಿ ಸಾವಿರಾರು ವರುಷಗಳಿಗೆ ಮುಂಚೆ ಇಂತಹ ಘಟನೆಗಳು ಈಚೆಗಿಂತಲೂ ಹೆಚ್ಚಾಗಿ ಆಗುತ್ತಿದ್ದವು. ಒಂದು ದೇಶದಲ್ಲಿ ಜನಸಂಖ್ಯೆ ಹೆಚ್ಚಾದಂತೆ ಅವರ ಮಾನಸಿಕ ಶಕ್ತಿ ಕ್ಷೀಣವಾಗುವುದು ಎಂದು ತೋರುತ್ತದೆ. ಬಹಳ ಅಲ್ಪ ಜನರಿರುವ ದೊಡ್ಡ ದೇಶದಲ್ಲಿ ಬಹುಶಃ ಹೆಚ್ಚು ಮಾನಸಿಕ ಶಕ್ತಿ ಇರಬಹುದು. ಹಿಂದೂಗಳು ವಿಶ್ಲೇಷಣಾತ್ಮಕ ಮನಸ್ಸಿನವರಾದುದರಿಂದ ಇಂತಹ ಘಟನೆಗಳನ್ನು ತೆಗೆದುಕೊಂಡು ಪರೀಕ್ಷಿಸತೊಡಗಿದರು. ಇದರಿಂದ ಕೆಲವು ಅದ್ಭುತ ನಿರ್ಣಯಗಳಿಗೆ ಬಂದರು. ಅಂದರೆ ಅದನ್ನೇ ಒಂದು ಶಾಸ್ತ್ರ ಮಾಡಿದರು. ಇದೆಲ್ಲ ಅದ್ಭುತವಾದರೂ ಸ್ವಾಭಾವಿಕ. ಪ್ರಕೃತಿಯಲ್ಲಿ ಅತಿಭೌತಿಕ (\enginline{Supernatural}) ಎಂಬುದಿಲ್ಲ. ಉಳಿದೆಲ್ಲ ಭೌತಿಕಘಟನೆಗಳು ಒಂದು ನಿಯಮವನ್ನು ಅನುಸರಿಸುವಂತೆಯೇ ಇವೂ ಒಂದು ನಿಯಮಾವಳಿಯ ಕೆಳಗೆ ಬರುವುದೆಂಬುದನ್ನು ಕಂಡುಹಿಡಿದರು. ಇಂತಹ ಅದ್ಭುತ ಶಕ್ತಿಯೊಂದಿಗೆ ಮನುಷ್ಯ ಜನಿಸುವುದು ಆಕಸ್ಮಿಕವಲ್ಲ. ಇದನ್ನು ಶಾಸ್ತ್ರೀಯವಾಗಿ ಅಧ್ಯಯನ ಮಾಡಬಹುದು, ಅನುಷ್ಠಾನ ಮಾಡಬಹುದು, ಪಡೆಯಬಹುದು. ಈ ಶಾಸ್ತ್ರವನ್ನೇ ಅವರು ರಾಜಯೋಗವೆನ್ನುವರು. ಸಹಸ್ರಾರು ಮಂದಿ ಈ ಶಾಸ್ತ್ರವನ್ನು ಅಧ್ಯಯನ ಮಾಡುತ್ತಾರೆ. ಇಡೀ ಜನಾಂಗಕ್ಕೆ ಇದೊಂದು ಅವರ ನಿತ್ಯಪೂಜೆಯ ಅಂಗವಾಗಿದೆ.

ಇಂತಹ ಅದ್ಭುತ ಶಕ್ತಿಯೆಲ್ಲ ಮಾನವನ ಮನಸ್ಸಿನಲ್ಲಿದೆ ಎಂಬುದೆ ಅವರ ನಿರ್ಣಯ. ಈ ಮನಸ್ಸು ವಿಶ್ವಮನಸ್ಸಿನ ಒಂದು ಅಂಶ. ಪ್ರತಿಯೊಂದು ಮನಸ್ಸೂ ಮತ್ತೊಂದು ಮನಸ್ಸಿನೊಂದಿಗೆ ಸಂಬಂಧ ಪಡೆದಿದೆ. ಪ್ರತಿಯೊಂದು ಮನಸ್ಸೂ ಅದು ಎಲ್ಲಿಯಾದರೂ ಇರಲಿ ವಿಶ್ವಮನಸ್ಸಿನೊಂದಿಗೆ ಸಂಪರ್ಕ ಬೆಳೆಸಿಕೊಂಡಿದೆ. ನಮ್ಮ ಆಲೋಚನೆಯನ್ನು ಮತ್ತೊಬ್ಬನಿಗೆ ರವಾನೆಮಾಡುವ ಘಟನೆಯನ್ನು ನೀವು ಎಂದಾದರೂ ನೋಡಿರುವಿರಾ? ಒಬ್ಬ ಇಲ್ಲಿ ಕುಳಿತುಕೊಂಡು ಏನನ್ನೋ ಆಲೋಚಿಸುತ್ತಿರುವನು. ಬೇರೊಂದು ಕಡೆ ಇನ್ನೊಬ್ಬರಲ್ಲಿ ಆ ಆಲೋಚನೆ ಮೂಡುತ್ತಿದೆ. ಇದು ಕೇವಲ ಅಕಸ್ಮಾತ್ತಾಗಿ ಆಗುವುದಿಲ್ಲ, ಸಿದ್ದತೆಯಿಂದ ಆಗುವುದು. ದೂರದಲ್ಲಿ ಒಬ್ಬನು ತನ್ನ ಮನಸ್ಸಿನ ಆಲೋಚನೆಯನ್ನು ಕಳಿಸಬೇಕೆಂದಿರುವನು. ಮತ್ತೊಬ್ಬನಿಗೆ ಇಂತಹ ಒಂದು ಆಲೋಚನೆ ಬರುತ್ತಿದೆ ಎಂಬುದು ಗೊತ್ತಿದೆ. ಹೇಗೆ ಕಳುಹಿಸುವವನು ಆಲೋಚನೆಯನ್ನು ಹೊರಗೆಡಹುವನೊ ಹಾಗೆಯೆ ಸ್ವೀಕರಿಸುವವನು ಅದನ್ನು ಪಡೆಯುವನು. ದೂರ ಅಥವಾ ಅಂತರವು ಯಾವ ವ್ಯತ್ಯಾಸವನ್ನೂ ಮಾಡುವುದಿಲ್ಲ. ಆ ಆಲೋಚನೆ ಮತ್ತೊಬ್ಬನಿಗೆ ಸೇರುವುದು. ಆತನಿಗೆ ಇದು ಗೊತ್ತಾಗುವುದು. ನಿಮ್ಮ ಮನಸ್ಸು ಪ್ರತ್ಯೇಕವಾಗಿದ್ದು, ನನ್ನ ಮನಸ್ಸು ಅದರೊಡನೆ ಯಾವ ಸಂಪರ್ಕವೂ ಇಲ್ಲದೆ ಪ್ರತ್ಯೇಕವಾಗಿದ್ದರೆ, ಇಬ್ಬರಿಗೂ ಸಂಬಂಧ ಇಲ್ಲದೆ ಇದ್ದರೆ ನನ್ನ ಆಲೋಚನೆ ನಿಮಗೆ ಹೇಗೆ ತಲಪುವುದಕ್ಕೆ ಸಾಧ್ಯವಾಗುತ್ತಿತ್ತು?\break ಸಾಧಾರಣ ಘಟನೆಗಳಲ್ಲಿ ನನ್ನ ಆಲೋಚನೆ ನಿಮಗೆ ನೇರವಾಗಿ ಬರುತ್ತಿಲ್ಲ. ನನ್ನ ಆಲೋಚನೆ ಆಕಾಶದ ಸ್ಪಂದನಗಳಾಗಿ, ಆ ಸ್ಪಂದನ ನಿಮ್ಮ ಮೆದುಳಿಗೆ ಹೋಗಿ ಪುನಃ ಆಲೋಚನೆಯ ರೂಪವನ್ನು ತಾಳುವುವು. ಈ ಕಡೆಯಿಂದ ಆಲೋಚನೆ ಅವ್ಯಕ್ತವಾಗುವುದು. ಆ ಕಡೆಯಿಂದ ಆಲೋಚನೆ ವ್ಯಕ್ತವಾಗುವುದು. ಇದು ನೇರವಾದ ಮಾರ್ಗವಲ್ಲ. ಟೆಲಿಪತಿಯಲ್ಲಿ (ದೂರ ಮನಸ್ಪರ್ಶನ) ಹೀಗಿಲ್ಲ. ಇದೆಲ್ಲ ನೇರವಾಗಿ ನಡೆಯುವುದು.

ಯೋಗಿಗಳು ಹೇಳುವಂತೆ ಮನಸ್ಸು ಅಖಂಡ ಎಂಬುದನ್ನು ಇದು ತೋರುವುದು. ಮನಸ್ಸು ವಿಶ್ವವ್ಯಾಪಿ. ನಿಮ್ಮ, ನಮ್ಮ ಮತ್ತು ಇತರರ ಕಿರುಮನಸ್ಸೆಲ್ಲ ವಿಶ್ವಮನಸ್ಸಿನ ಅಂಶಗಳು, ಸಾಗರದ ಮೇಲೆ ಇರುವ ತೆರೆಗಳಂತೆ. ಹೀಗೆ ಒಂದು ಸಂಬಂಧ ವಿರುವುದರಿಂದ ಆಲೋಚನೆಯನ್ನು ಬೇರೊಬ್ಬರಿಗೆ ಪ್ರತ್ಯಕ್ಷವಾಗಿ ಕಳುಹಿಸಬಹುದು.

ನಿಮ್ಮ ಸುತ್ತಲೂ ಆಗುತ್ತಿರುವುದನ್ನು ನೀವು ನೋಡುತ್ತಿರುವಿರಿ. ಜಗತ್ತಿನಲ್ಲಿ ಪ್ರತಿಯೊಂದೂ ಪರಸ್ಪರ ಪ್ರಭಾವವುಳ್ಳವುಗಳು. ನಮ್ಮ ದೇಹಸಂರಕ್ಷಣೆಗೆ ಮನಸ್ಸಿನ ಸ್ವಲ್ಪ ಭಾಗವನ್ನು ಉಪಯೋಗಿಸುವೆವು. ಇದಲ್ಲದೆ ಉಳಿದ ಮನಶ್ಶಕ್ತಿಯೆಲ್ಲ ಹಗಲೂ ರಾತ್ರಿಯೂ ಇತರರ ಮೇಲೆ ತಮ್ಮ ಪರಿಣಾಮವನ್ನು ಬೀರುವುದರಲ್ಲಿ ಉಪಯೋಗಿಸಲ್ಪಡುತ್ತಿದೆ. ನಮ್ಮ ದೇಹ, ಗುಣ, ಬುದ್ದಿ ನಮ್ಮ ಆಧ್ಯಾತ್ಮಿಕತೆಯೆಲ್ಲ ನಿರಂತರ ಇತರರ ಮೇಲೆ ತಮ್ಮ ಪರಿಣಾಮವನ್ನು ಬೀರುತ್ತಿವೆ. ಅದರಂತೆಯೆ ನಾವೂ ಕೂಡ ಇತರರ ಪರಿಣಾಮಕ್ಕೆ ಒಳಗಾಗುತ್ತಿರುವೆವು. ಇದು ನಮ್ಮ ಸುತ್ತಲೂ ಆಗುತ್ತದೆ. ಒಂದು ಪ್ರತ್ಯಕ್ಷ ಉದಾಹರಣೆಯನ್ನು ತೆಗೆದುಕೊಳ್ಳೋಣ. ಒಬ್ಬ ನಿಮ್ಮ ಬಳಿಗೆ ಬರುವನು. ಅವನು ಬುದ್ದಿವಂತನೆಂಬುದು ನಿಮಗೆ ಗೊತ್ತಿದೆ. ಅವನು ಮಾತನಾಡುವ ಭಾಷೆ ಸುಂದರವಾಗಿದೆ. ಗಂಟೆಗಟ್ಟಲೆ ಅವನು ನಿಮ್ಮ ಹತ್ತಿರ ಮಾತನಾಡುತ್ತಿರುವನು. ಅವನು ನಿಮ್ಮ ಮನಸ್ಸಿನ ಮೇಲೆ ಯಾವ ಪ್ರಭಾವವನ್ನೂ ಬೀರುವುದಿಲ್ಲ. ಮತ್ತೊಬ್ಬ ಬರುವನು. ಅವನು ಎಲ್ಲೋ ಸ್ವಲ್ಪ ಮಾತನಾಡುವನು, ವಿಷಯಗಳನ್ನು ಸರಿಯಾಗಿ ಜೋಡಿಸಿಲ್ಲ, ಭಾಷೆ ವ್ಯಾಕರಣ ಶುದ್ಧವಾಗಿಲ್ಲ. ಆದರೂ ನಿಮ್ಮ ಮನಸ್ಸಿನ ಮೇಲೆ ಅದ್ಭುತ ಪರಿಣಾಮವನ್ನು ಉಂಟುಮಾಡುವನು. ನಿಮ್ಮಲ್ಲಿ ಹಲವರು ಇದನ್ನು ನೋಡಿರುವಿರಿ. ಆದಕಾರಣ ಕೇವಲ ಮಾತೇ ಪರಿಣಾಮಕಾರಿಯಲ್ಲ ಎಂಬುದು ನಿಜವಾಯಿತು. ಮತ್ತೊಬ್ಬರ ಮೇಲೆ ಪರಿಣಾಮವನ್ನು ಬೀರುವುದರಲ್ಲಿ ಭಾಷೆ, ಆಲೋಚನೆ ಕೂಡ, ಕೇವಲ ಮೂರನೆಯ ಒಂದು ಭಾಗ. ಅದರ ಹಿಂದೆ ಇರುವ ವ್ಯಕ್ತಿತ್ವ ಮೂರನೆಯ ಎರಡು ಭಾಗ, ವ್ಯಕ್ತಿತ್ವದ ಆಕರ್ಷಣೆಯೆಂಬುದು - ಅದೇ ನಿಮ್ಮನ್ನು ಮುಗ್ಧರನ್ನಾಗಿ ಮಾಡುವುದು.

ನಮ್ಮ ಮನೆಗಳಲ್ಲಿ ಮುಖ್ಯಸ್ಥರಿರುವರು. ಅವರಲ್ಲಿ ಕೆಲವರು ಜಯಶೀಲರಾಗುವರು. ಮತ್ತೆ ಕೆಲವರು ಇಲ್ಲ. ಏತಕ್ಕೆ? ನಾವು ಸೋತರೆ ಇತರರನ್ನು ದೂರುವೆವು. ನಾನು ಎಂದು ಜಯಶೀಲನಾಗುವುದಿಲ್ಲವೋ ಆಗ ಇಂಥವನಿಂದ ಆಗಲಿಲ್ಲ ಎನ್ನುವೆನು. ಸೋತಾಗ ತನ್ನ ತಪ್ಪನ್ನು, ದುರ್ಬಲತೆಯನ್ನು ಒಪ್ಪಿಕೊಳ್ಳುವುದಕ್ಕೆ ಇಚ್ಛೆ ಇಲ್ಲ. ಪ್ರತಿಯೊಬ್ಬನೂ ತನ್ನಲ್ಲಿ ಏನೂ ತಪ್ಪಿಲ್ಲ, ಅದಕ್ಕೆ ಮತ್ತಾರೊ, ಮತ್ತಾವುದೊ ಕಾರಣವೆಂದು ದೂರುವನು. ಕೊನೆಗೆ ಗ್ರಹಚಾರವೆಂದು ಹಳಿಯುವನು. ಮನೆಯ ಯಜಮಾನರು ಸೋತಾಗ, ಏತಕ್ಕೆ\break ಕೆಲವರು ಉದ್ಯಮದಲ್ಲಿ ಅಷ್ಟು ಯಶಸ್ವಿಗಳಾಗಿರುವರು ಮತ್ತೆ ಕೆಲವರು ಇಲ್ಲ ಎಂಬ ಪ್ರಶ್ನೆಯನ್ನು ಕೇಳಬೇಕು. ಇದಕ್ಕೆ ಕಾರಣ ಆ ಮನುಷ್ಯನ ವ್ಯಕ್ತಿತ್ವದಲ್ಲಿದೆ.

ಜನರ ಪ್ರಮುಖ ಮುಂದಾಳುಗಳನ್ನು ತೆಗೆದುಕೊಂಡರೆ ಯಾವಾಗಲೂ ಅವರ ವ್ಯಕ್ತಿತ್ವ ಬಹಳ ಮುಖ್ಯವೆಂಬುದನ್ನು ನೋಡುತ್ತೇವೆ. ಹಿಂದಿನ ಮಹಾಕವಿಗಳು,\break ಆಲೋಚನಾಪರರು ಎಲ್ಲರನ್ನೂ ತೆಗೆದುಕೊಳ್ಳಿ. ನಿಜವಾಗಿ ನೋಡಿದರೆ ಅವರು ಎಷ್ಟು ಹೊಸ ಆಲೋಚನೆಯನ್ನು ಮಾಡಿರುವರು? ಹಿಂದಿನ ಮಾನವಕೋಟಿಯ ಪ್ರಮುಖರು ಬಿಟ್ಟಿರುವ ಬರವಣಿಗೆಯನ್ನೆಲ್ಲ ತೆಗೆದುಕೊಳ್ಳಿ, ಅವರ ಪ್ರತಿಯೊಂದು ಗ್ರಂಥವನ್ನೂ ತೆಗೆದುಕೊಂಡು ಪರೀಕ್ಷೆ ಮಾಡಿ ನೋಡಿ, ಜಗತ್ತಿನ ಇದುವರೆಗೆ ಮಾಡಿದ ಹೊಸದಾದ ನಿಜವಾದ ಆಲೋಚನೆಗಳು ಬಹಳ ಅಲ್ಪ. ಅವರು ನಮಗೆ ಬಿಟ್ಟಿರುವ ಆಲೋಚನೆಯನ್ನು ಅವರ ಗ್ರಂಥಗಳಲ್ಲಿ ಓದಿ ನೋಡಿ. ಅದರ ಗ್ರಂಥಕರ್ತರು ಮಹಾವಿಭೂತಿಗಳಂತೆ ತೋರುವುದಿಲ್ಲ. ಆದರೂ ಅವರ ಕಾಲದಲ್ಲಿ ಅವರು ದೊಡ್ಡವರಾಗಿದ್ದರು. ಹೀಗೆ ಅವರನ್ನು ಮಾಡಿದುದು ಯಾವುದು? ಅವರು ಮಾಡಿದ ಆಲೋಚನೆ ಮಾತ್ರವಲ್ಲ, ಬರೆದ ಗ್ರಂಥಮಾತ್ರವಲ್ಲ, ಕೊಟ್ಟ ಭಾಷಣವಲ್ಲ, ಅದು ಈಗ ನಮಗೆ ದೊರಕದ ಮತ್ತಾವುದೊ ಒಂದು, ಅದೇ ವ್ಯಕ್ತಿತ್ವ. ನಾನು ಮುಂಚೆಯೇ ಹೇಳಿದಂತೆ ಮನುಷ್ಯನ ವ್ಯಕ್ತಿತ್ವ ಮೂರನೆಯ ಎರಡುಭಾಗ. ಅವನ ಬುದ್ದಿ ಮಾತುಕತೆಯೆಲ್ಲ ಮೂರನೆಯ ಒಂದುಭಾಗ. ಮನುಷ್ಯನ ನಿಜವಾದ ಸ್ವರೂಪವೆ, ಮನುಷ್ಯನ ವ್ಯಕ್ತಿತ್ವವೆ ನಮ್ಮಲ್ಲಿ ಸಂಚರಿಸುತ್ತಿದೆ. ನಾವು ಮಾಡುವ ಕೆಲಸಗಳೆಲ್ಲ ಪರಿಣಾಮಗಳು ಅಷ್ಟೆ. ಮನುಷ್ಯ ಅಲ್ಲಿದ್ದರೆ ಕರ್ಮವೂ ನಡೆಯಲೇಬೇಕು. ಕಾರ್ಯವು ಕಾರಣವನ್ನು ಅನುಸರಿಸಲೇಬೇಕು.

ಎಲ್ಲಾ ವಿದ್ಯಾಭ್ಯಾಸದ ಗುರಿ, ತರಭೇತಿನ ಗುರಿ ಪುರುಷಸಿಂಹರನ್ನು ಮಾಡುವುದು. ಹಾಗೆ ಮಾಡುವುದನ್ನು ಬಿಟ್ಟು ಹೊರಗೆ ನಯಮಾಡುವುದಕ್ಕೆ ನಾವು ಯಾವಾಗಲೂ ಯತ್ನಿಸುತ್ತಿರುವೆವು. ಒಳಗೆ ಏನೂ ಇಲ್ಲದೆ ಹೊರಗಿನದರಲ್ಲಿ ನಯಮಾಡಿದರೆ ಬಂದ ಪ್ರಯೋಜನವೇನು? ಎಲ್ಲಾ ತರಬೇತಿಯ ಗುರಿ ಮನುಷ್ಯನ ಬೆಳವಣಿಗೆಗೆ ಸಹಾಯ ಮಾಡುವುದು. ಯಾರು ಮತ್ತೊಬ್ಬರ ಮೇಲೆ ತನ್ನ ಪ್ರಭಾವವನ್ನು ಬೀರಬಲ್ಲನೋ, ಯಾರು ತನ್ನ ಮಂತ್ರಶಕ್ತಿಯನ್ನು ತನ್ನ ಸಹೋದರರ ಮೇಲೆ ಎಸೆಯಬಲ್ಲನೊ, ಅವನು ಶಕ್ತಿಯ ಮಹಾಯಂತ್ರ. ಆತ ಸಿದ್ದನಾದರೆ ಏನನ್ನು ಬೇಕಾದರೂ ಮಾಡಬಲ್ಲ. ಇಚ್ಛಿಸಿದುದೆಲ್ಲವನ್ನೂ ಮಾಡಬಲ್ಲ. ಯಾವ ಕೆಲಸವನ್ನು ಅವನಿಗೆ ಕೊಟ್ಟರೂ ಅದನ್ನು ಅವನು ಸಾಧಿಸಬಲ್ಲ.

ಇದು ನಿಜವಾದರೂ ನಮಗೆ ತಿಳಿದಿರುವ ಯಾವ ಭೌತಿಕ ನಿಯಮಗಳೂ ಇಂಥವನ್ನು ವಿವರಿಸಲಾರವು. ಇದನ್ನು ನಾವು ರಾಸಾಯನಿಕ ಮತ್ತು ಭೌತಿಕ ಜ್ಞಾನದಿಂದ ಹೇಗೆ ವಿವರಿಸಲು ಸಾಧ್ಯ? ದೇಹದಲ್ಲಿರುವ ಎಷ್ಟು ಕಾರ್ಬನ್, ಹೈಡೋಜನ್, ಆಕ್ಸಿಜನ್ ಕಣಗಳು, ದೇಹದಲ್ಲಿ ಬೇರೆ ಬೇರೆ ಸ್ಥಾನಗಳಲ್ಲಿರುವ ಎಷ್ಟು ಜೀವಕಣಗಳು ಮನುಷ್ಯನ ವ್ಯಕ್ತಿತ್ವವೆಂಬ ರಹಸ್ಯವನ್ನು ವಿವರಿಸಬಲ್ಲವು? ಆದರೂ ಅದು ನಿಜವೆನ್ನುವುದನ್ನು ನಾವು ನೋಡುತ್ತೇವೆ. ಅದು ಮಾತ್ರವಲ್ಲ, ಅವನೇ ನಿಜವಾದ ಮನುಷ್ಯ. ಆ ಮನುಷ್ಯನೆ ಬಾಳುವುದು, ಬದುಕುವುದು, ಚಲಿಸುವುದು. ಆತನೆ ಇತರರ ಮೇಲೆ ತನ್ನ ಪ್ರಭಾವವನ್ನು ಬೀರುವುದು, ಅವರಲ್ಲಿ ಕ್ರಿಯೊತ್ಸಾಹವನ್ನು ತುಂಬುವುದು. ಅನಂತರ ಅವನು ಮಾಯವಾಗಿ ಹೋಗುವನು. ಅವನ ಬುದ್ದಿವಂತಿಕೆ, ಬರೆದ ಪುಸ್ತಕ, ಮಾಡಿದ ಕೆಲಸವೆಲ್ಲ ಹಿಂದೆ ಬಿಟ್ಟುಹೋದ ಕೆಲವು ಚಿಹ್ನೆಗಳು ಅಷ್ಟೆ. ಇದನ್ನು ಆಲೋಚಿಸಿ ನೋಡಿ. ದೊಡ್ಡ ಧಾರ್ಮಿಕ ಬೋಧಕರನ್ನು ದೊಡ್ಡ ತಾತ್ವಿಕರೊಂದಿಗೆ ಹೋಲಿಸಿ ನೋಡಿ. ತಾತ್ವಿಕರಿಗೆ ಯಾರ ಅಂತರ್ಜೀವನವನ್ನೂ ಮುಟ್ಟಲಾಗಲಿಲ್ಲ. ಆದರೂ ಅದ್ಭುತ ಗ್ರಂಥಗಳನ್ನು ಬರೆದರು. ಆದರೆ ಧಾರ್ಮಿಕ ಬೋಧಕರಾದರೋ, ಬದುಕಿರುವಾಗಲೆ ಇಡೀ ದೇಶದ ಮೇಲೆ ತಮ್ಮ ಪ್ರಭಾವವನ್ನು ಬೀರಿದರು. ಈ ವ್ಯತ್ಯಾಸಕ್ಕೆ ಕಾರಣ ಅವರ ವ್ಯಕ್ತಿತ್ವ. ತಾತ್ವಿಕರಲ್ಲಿ ಬಹಳ ಅಸ್ಪಷ್ಟ ವಾಗಿರುವ ವ್ಯಕ್ತಿತ್ವ ಮತ್ತೊಬ್ಬರನ್ನು ಪ್ರಚೋದಿಸುವುದು. ಮಹಾಪುರುಷರಲ್ಲಿ ಆ ವ್ಯಕ್ತಿತ್ವ ಅದ್ಭುತವಾಗಿರುವುದು. ತತ್ವದ ಮೂಲಕ ನಾವು ಜನರ ಬುದ್ದಿಯನ್ನು ಮುಟ್ಟುವೆವು. ಧರ್ಮದ ಮೂಲಕ ಅವರ ಇಡೀ ಜೀವನವನ್ನು ಮುಟ್ಟುವೆವು. ಮೊದಲನೆಯದು ಕೇವಲ ಒಂದು ರಾಸಾಯನಿಕ ಮಾರ್ಗ. ಕೆಲವು ರಾಸಾಯನಿಕ ದ್ರವ್ಯಗಳನ್ನು ಒಟ್ಟು ತರುವುದು. ಅದು ಕ್ರಮೇಣ ಮಿಶ್ರವಾಗುವುದು. ಸಮಯ ಸರಿಯಾಗಿ ಒದಗಿದರೆ ಅದು ಜ್ಞಾನದಲ್ಲಿ ಪರ್ಯವಸಾನವಾದರೂ ಆಗಬಹುದು, ಇಲ್ಲದೆ ಇದ್ದರೂ ಇರಬಹುದು. ಆದರೆ ಮತ್ತೊಂದಾದರೊ ಪಂಜಿನಂತೆ ಬೇಗ ಬೆಂಕಿಯನ್ನು ಹಚ್ಚುತ್ತಾ, ಎಂದರೆ ಇತರರಿಗೆ ಬೆಳಕನ್ನು ನೀಡುತ್ತಾ ಹೋಗುವುದು.

ಯೋಗಶಾಸ್ತ್ರ ಈ ವ್ಯಕ್ತಿತ್ವವನ್ನು ಅಭಿವೃದ್ಧಿಗೊಳಿಸುವ ನಿಯಮವನ್ನು ಕಂಡುಹಿಡಿದಿರುವೆ ಎನ್ನುವುದು. ಈ ನಿಯಮಗಳಿಗೂ ಮತ್ತು ಮಾರ್ಗಗಳಿಗೂ ಸಾಕಾದಷ್ಟು ಗಮನಕೊಟ್ಟರೆ ಪ್ರತಿಯೊಬ್ಬರೂ ತಮ್ಮ ವ್ಯಕ್ತಿತ್ವವನ್ನು ಬಲಪಡಿಸಿಕೊಳ್ಳ ಬಹುದು. ಅನುಷ್ಠಾನದಲ್ಲಿ ಬಹಳ ಪ್ರಮುಖವಾದುದು ಇದು. ಇದೇ ಎಲ್ಲಾ ವಿದ್ಯೆಯ ರಹಸ್ಯ. ಇದು ಎಲ್ಲರಿಗೂ ಅನ್ವಯಿಸುವುದು, ಗೃಹಸ್ಥರಾಗಲಿ ದರಿದ್ರರಾಗಲಿ ಶ‍್ರೀಮಂತರಾಗಲಿ ವರ್ತಕನಾಗಲಿ ಧಾರ್ಮಿಕಪ್ರವೃತ್ತಿಯುಳ್ಳವನಾಗಲಿ ಪ್ರತಿಯೊಬ್ಬರ ಜೀವನದಲ್ಲಿಯೂ ವ್ಯಕ್ತಿತ್ವವನ್ನು ರೂಢಿಸುವುದು ಬಹಳ ಮುಖ್ಯವಾದುದು. ಭೌತಿಕ ನಿಯಮಗಳ ಹಿಂದೆ ನಮಗೆ ತಿಳಿದಿರುವಂತೆಯೇ ಅತಿಸೂಕ್ಷ್ಮ ನಿಯಮಾವಳಿಗಳಿವೆ. ಭೌತಿಕ ಮಾನಸಿಕ ಆಧ್ಯಾತ್ಮಿಕಗಳೆಂಬ ಬೇರೆಬೇರೆ ಸತ್ಯಗಳಿಲ್ಲ. ಏನಿದೆಯೋ ಅದೆಲ್ಲ ಒಂದು. ಇದೊಂದು ಸೂಕ್ಷ್ಮತರವಾಗುತ್ತಾ ಹೋಗುವ ಕ್ರಿಯೆ. ಅತಿ ಸ್ಥೂಲವಾಗಿರುವುದು ಇಲ್ಲಿದೆ. ಅದು ಸೂಕ್ಷ್ಮ ಸೂಕ್ಷ್ಮವಾಗುತ್ತಾ ಬರುವುದು. ಅತಿ ಸೂಕ್ಷ್ಮವಾದುದನ್ನು ಅಧ್ಯಾತ್ಮ ಎನ್ನುತ್ತೇವೆ. ಸ್ಥೂಲವಾದುದೆ ದೇಹ. ಇದು ವ್ಯಷ್ಟಿಯಲ್ಲಿ ಹೇಗಿದೆಯೋ ಹಾಗೆಯೇ ಸಮಷ್ಟಿಯಲ್ಲಿ. ನಮ್ಮ ವಿಶ್ವವೂ ಕೂಡ ಹಾಗೆಯೇ ಇರುವುದು. ಪ್ರಪಂಚ ಹೊರಗಿರುವ ಸ್ಥೂಲಕವಚ. ಅದು ಸೂಕ್ಷ್ಮ ಸೂಕ್ಷ್ಮವಾಗಿ ಕೊನೆಗೆ ಈಶ್ವರನಾಗುವುದು.

ಅದ್ಭುತ ಶಕ್ತಿ ಸೂಕ್ಷ್ಮದಲ್ಲಿ ಮಾತ್ರ ಅಡಗಿದೆ ಸ್ಥೂಲದಲ್ಲಿ ಅಲ್ಲ ಎಂಬುದು ನಮಗೆ ಗೊತ್ತಿದೆ. ಒಬ್ಬ ಭಾರ ಹೊರುವುದನ್ನು ನೋಡುತ್ತೇವೆ. ಅವನ ಮಾಂಸಖಂಡಗಳು\break ಉಬ್ಬುವುದನ್ನು ನೋಡುತ್ತೇವೆ. ದೇಹಾದ್ಯಂತವೂ ಅವನ ಶ್ರಮದ ಚಿಹ್ನೆಗಳಿವೆ. ಮಾಂಸಖಂಡಗಳು ಬಹಳ ಬಲವಾದುದೆಂದು ನಾವು ಯೋಚಿಸುವೆವು. ಆದರೆ ದಾರದಂತೆ ಸಣ್ಣಗಿರುವ ನರಗಳು ಮಾಂಸಖಂಡಕ್ಕೆ ಶಕ್ತಿಯನ್ನೊದಗಿಸುವುವು. ಮಾಂಸಖಂಡಗಳಿಗೆ ಬರದಂತೆ ಈ ನರಗಳನ್ನು ಕಡಿದೊಡನೆ ಇವು ಯಾವ ಕೆಲಸವನ್ನೂ ಮಾಡಲಾರವು. ಈ ಸಣ್ಣ ನರಗಳು ಮತ್ತಾವುದೊ ತಮಗಿಂತಲೂ ಸೂಕ್ಷ್ಮವಾದುದರಿಂದ ಶಕ್ತಿಯನ್ನು ತರುವುದು. ಅದು ತನಗಿಂತಲೂ ಸೂಕ್ಷ್ಮತರವಾದುದರಿಂದ ಶಕ್ತಿಯನ್ನು ತರುವುದು. ಅದೇ ಚಿಂತನೆ. ಆದಕಾರಣ ನಿಜವಾದ ಶಕ್ತಿಯ ಕೇಂದ್ರವೇ ಸೂಕ್ಷ್ಮ, ಸ್ಥೂಲದ ಚಲನೆಯನ್ನು ನಾವು ನೋಡಬಹುದು. ಸೂಕ್ಷ್ಮದ ಚಲನೆಯನ್ನು ನಾವು ನೋಡಲಾರೆವು. ಸ್ಥೂಲವಾಗಿರುವುದು ಚಲಿಸಿದರೆ ನಾವು ಅದನ್ನು ಗ್ರಹಿಸುವೆವು. ಆದಕಾರಣವೆ ನಾವು ಸ್ವಭಾವತಃ ಚಲನೆಯನ್ನು ಸ್ಥೂಲದೊಂದಿಗೆ ಅನ್ವಯಿಸುವೆವು. ಆದರೆ ನಿಜವಾದ ಶಕ್ತಿಯೆಲ್ಲ ಸೂಕ್ಷ್ಮದಲ್ಲಿರುವುದು. ಸೂಕ್ಷದ ಯಾವ ಚಲನೆಯೂ ನಮಗೆ ಕಾಣುವುದಿಲ್ಲ. ಏಕೆಂದರೆ ಬಹುಶಃ ಅವೆಲ್ಲ ಅಷ್ಟು ತೀವ್ರವಾಗಿರಬಹುದು. ಯಾವುದಾದರೂ ವಿಜ್ಞಾನದ ಸಹಾಯದಿಂದ, ಸಂಶೋಧನೆಯ ಸಹಾಯದಿಂದ ಇವುಗಳ ಹಿಂದೆ ಇರುವ ಸೂಕ್ಷ್ಮ ಶಕ್ತಿಗಳ ಚಲನೆ ನಮ್ಮ ವಶವಾದರೆ ಅದರ ಸ್ಥೂಲ ಬಾಹ್ಯ ಅಭಿವ್ಯಕ್ತಿ ಕೂಡ ನಮ್ಮ ವಶವಾಗುವುದು. ಸರೋವರದ ಕೆಳಗಿನಿಂದ ಸಣ್ಣದೊಂದು ಗುಳ್ಳೆ ಬರುತ್ತಿದೆ. ಅದು ಯಾವಾಗಲೂ ನಮಗೆ ಕಾಣುವುದಿಲ್ಲ. ಮೇಲೆ ಬಂದು ಒಡೆದಾಗ ಮಾತ್ರ ನಮಗೆ ಗೊತ್ತಾಗುವುದು. ಹಾಗೆಯೆ ಆಲೋಚನೆ ಕೂಡ ಪ್ರವರ್ಧಮಾನಕ್ಕೆ ಬಂದಮೇಲೆ ಅಥವಾ ಅದು ಕ್ರಿಯೆಯಲ್ಲಿ ಪರ್ಯವಸಾನವಾದ ಮೇಲೆ ಮಾತ್ರ ನಾವು ಅದನ್ನು ನೋಡುವೆವು. ನಮಗೆ ನಮ್ಮ ಕ್ರಿಯೆಗಳ ಮತ್ತು ಆಲೋಚನೆಗಳ ಮೇಲೆ ಸ್ವಲ್ಪವೂ ಹತೋಟಿಯಿಲ್ಲ ಎಂದು ಅನೇಕ ವೇಳೆ ಗೊಣಗಾಡುವೆವು. ಇದು ಹೇಗೆ ಸಾಧ್ಯ? ಅದರ ಸೂಕ್ಷ್ಮಚಲನೆ ನಮ್ಮ ವಶವಾದಾಗ ನಾವು ಎಲ್ಲವನ್ನೂ ಬೇಕಾದರೆ ನಿಗ್ರಹಿಸಬಹುದು. ಸೂಕ್ಷ್ಮಶಕ್ತಿಗಳನ್ನು ವಿಶ್ಲೇಷಣೆ ಮಾಡಿ, ಪರೀಕ್ಷೆ ಮಾಡಿ, ತಿಳಿದುಕೊಳ್ಳುವುದಕ್ಕೆ ಒಂದು ಮಾರ್ಗವಿದ್ದರೆ ಆಗ ಮಾತ್ರ ನಮ್ಮನ್ನು ನಾವು ನಿಗ್ರಹಿಸಬಹುದು. ಯಾರು ತಮ್ಮ ಮನಸ್ಸನ್ನು ನಿಗ್ರಹಿಸುವರೋ ಅವರಿಗೆ ಇತರರ ಮನಸ್ಸಿನ ಮೇಲೆ ನಿಗ್ರಹ ನಿಜವಾಗಿ ದೊರಕುವುದು. ಆದಕಾರಣವೆ ಪರಿಶುದ್ಧತೆ ಮತ್ತು ನೀತಿಯ ಜೀವನವೆ ಯಾವಾಗಲೂ ಧರ್ಮದ ಗುರಿ. ಒಬ್ಬ ಪರಿಶುದ್ದ ನೀತಿವಂತನಿಗೆ ಅವನ ಜೀವನದ ಮೇಲೆ ನಿಗ್ರಹವಿದೆ. ಎಲ್ಲಾ ಮನಸ್ಸು ಒಂದೇ ವಿಶ್ವಮನಸ್ಸಿನ ಭಿನ್ನ ಅಂಶಗಳು. ಯಾರು ಒಂದು ಹಿಡಿ ಮೃತ್ತಿಕೆಯನ್ನು ಅರಿತಿರುವರೊ ಅವರು ಜಗದ ಮೃತ್ತಿಕೆಯನ್ನೆಲ್ಲಾ ಅರಿತಿರುವರು. ಯಾರಿಗೆ ತನ್ನ ಮನಸ್ಸು ಅರಿವಾಗಿರುವುದೋ, ಅದನ್ನು ತನ್ನ ವಶದಲ್ಲಿಟ್ಟುಕೊಂಡಿರುವನೋ ಅವನಿಗೆ ಎಲ್ಲ ಮನಸ್ಸುಗಳ ರಹಸ್ಯವೂ ಗೊತ್ತು, ಎಲ್ಲರ ಮೇಲೂ ಅವನಿಗೆ ಅಧಿಕಾರವಿದೆ.

ಸೂಕ್ಷ್ಮದ ಮೇಲೆ ಹತೋಟಿ ಇದ್ದರೆ ನಮ್ಮ ಎಷ್ಟೋ ಭೌತಿಕ ಕಷ್ಟಗಳು ಮಾಯವಾಗಬಹುದಾಗಿತ್ತು. ಸೂಕ್ಷ್ಮಚಲನೆಯ ಮೇಲೆ ನಮಗೆ ನಿಗ್ರಹವಿದ್ದರೆ ಎಷ್ಟೋ ವ್ಯಥೆಯಿಂದ ಪಾರಾಗಬಹುದು, ಎಷ್ಟೊ ಸೋಲಿನಿಂದ ಪಾರಾಗಬಹುದು. ಇಲ್ಲಿಯವರೆಗೆ ಅದರ ಪ್ರಯೋಜನವಾಯಿತು. ಇದರಾಚೆ ಮತ್ತೂ ಶ್ರೇಷ್ಠವಾದದು ಇದೆ.

ನಾನೊಂದು ಸಿದ್ದಾಂತವನ್ನು ನಿಮಗೆ ಈಗ ಹೇಳುವೆನು. ಅದನ್ನು ಈಗ ಚರ್ಚಿಸುವುದಿಲ್ಲ; ಅದರ ನಿರ್ಣಯಗಳನ್ನು ಮಾತ್ರ ಹೇಳುತ್ತೇನೆ. ಪ್ರತಿಯೊಬ್ಬನೂ ತನ್ನ\break ಬಾಲ್ಯಾವಸ್ಥೆಯಲ್ಲಿ ತನ್ನ ಜನಾಂಗದ ಬೆಳವಣಿಗೆಯ ಎಲ್ಲಾ ಹಂತಗಳ ಅನುಭವವನ್ನೂ ಪಡೆಯುತ್ತಾನೆ. ಒಂದು ಜನಾಂಗವು ಆ ಸ್ಥಿತಿಗೆ ಬರಲು ಸಹಸ್ರಾರು ವರುಷಗಳನ್ನು ತೆಗೆದುಕೊಂಡಿತು. ಆದರೆ ಮಗು ಕೆಲವು ವರುಷಗಳಲ್ಲಿ ಇದನ್ನೆಲ್ಲಾ ಪೂರೈಸುವುದು, ಮಗು ಮೊದಲು ಹಳೆಯ ಕಾಡುಮನುಷ್ಯ. ಚಿಟ್ಟೆಯನ್ನು ತನ್ನ ಕಾಲಕೆಳಗೆ ತುಳಿಯುವುದು. ಮಗು ಮೊದಲು ತನ್ನ ಜನಾಂಗದ ಪೂರ್ವಿಕರ ಆದಿಮಾನವನಂತೆ ಇರುವುದು. ಬೆಳೆಯುತ್ತಾ ಹೋದಂತೆ ಅದು ಹಲವು ಅವಸ್ಥೆಗಳನ್ನು ದಾಟಿ ಕೊನೆಗೆ ತನ್ನ ಜನಾಂಗ ಈಗ ಇರುವ ಸ್ಥಿತಿಗೆ ಬರುವುದು. ಅದು ವೇಗವಾಗಿ ಬೇಗ ಪೂರೈಸುವುದು ಅಷ್ಟೆ. ಈಗ ಇಡೀ ಮಾನವಕೋಟಿಯನ್ನು ಒಂದು ಜನಾಂಗದಂತೆ ಭಾವಿಸಿ ಅಥವಾ ಪ್ರಾಣಿವರ್ಗವನ್ನೆಲ್ಲ (ಮಾನವ ಮತ್ತು ಮೃಗ) ಒಂದು ವರ್ಗದಂತೆ ಭಾವಿಸಿ. ಇದೆಲ್ಲ ಒಂದು ಗುರಿಯ ಕಡೆಗೆ ಧಾವಿಸುತ್ತಿದೆ. ಅದನ್ನು ಪೂರ್ಣತೆ ಎಂದು ಕರೆಯೋಣ. ಕೆಲವು ಸ್ತ್ರೀ ಪುರುಷರು ಜನಿಸುವರು. ಅವರಿಗೆ ಇಡೀ ಮಾನವಕೋಟಿ ಎತ್ತ ಸಾಗುತ್ತಿದೆ ಎಂಬುದು ಗೊತ್ತಾಗುವುದು. ಪುನಃ ಪುನಃ ಹುಟ್ಟಿ ಇಡೀ ಮಾನವಕೋಟಿ ಪೂರ್ಣತೆಯ ಕಡೆ ಹೋಗುವ ತನಕ ಕಾಯುವುದಕ್ಕಿಂತ ಅವರು ತಮ್ಮ ಜೀವನದ ಅಲ್ಪ ಅವಧಿಯಲ್ಲಿ ಇದನ್ನೆಲ್ಲಾ ಪೂರೈಸಲು ಧಾವಿಸುವರು. ನಾವು ಹೃತ್ಪೂರ್ವಕ ಪ್ರಯತ್ನಪಟ್ಟರೆ ಈ ಗತಿಯನ್ನು ತ್ವರಿತಗೊಳಿಸಬಹುದೆಂದು ಗೊತ್ತಿದೆ. ಯಾವ ಸಂಸ್ಕೃತಿಯೂ ಇಲ್ಲದ ಜನರನ್ನು ಒಂದು ದ್ವೀಪದಲ್ಲಿ ಬಿಟ್ಟರೆ, ಅವರಿಗೆ ಊಟ ಬಟ್ಟೆ ವಸತಿ ಮಾತ್ರ ಕಲ್ಪಿಸಿಕೊಟ್ಟರೆ ಅವರು ಕ್ರಮಕ್ರಮವಾಗಿ ನಾಗರಿಕತೆಯಲ್ಲಿ ಮುಂದುವರಿಯುವರು. ಉತ್ತಮ ಸಂಸ್ಕೃತಿಯನ್ನು ಅಭಿವೃದ್ಧಿ ಮಾಡಿಕೊಳ್ಳುವರು. ಈ ಅಭಿವೃದ್ಧಿಯನ್ನು ಬೇಕಾದರೆ ಇತರ ಬಾಹ್ಯ ಸಹಾಯದಿಂದ ತ್ವರಿತಗೊಳಿಸಬಹುದೆಂಬುದೂ ನಮಗೆ ಗೊತ್ತಿದೆ. ನಾವು ಗಿಡಗಳ ಬೆಳವಣಿಗೆಗೆ ಸಹಾಯ ಮಾಡುವೆವು, ಇಲ್ಲವೆ? ಕೇವಲ ಅವನ್ನು ಪ್ರಕೃತಿಗೇ ಬಿಟ್ಟಿದ್ದರೆ ಬೆಳೆಯುತ್ತಿದ್ದವು. ಆದರೆ ಹೆಚ್ಚು ಹೊತ್ತು ಹಿಡಿಯುತ್ತಿತ್ತು. ಅವು ಎಷ್ಟು ಕಾಲ ತೆಗೆದುಕೊಳ್ಳುತ್ತಿದ್ದವೋ ಅದಕ್ಕಿಂತ ಬೇಗ ಬೆಳೆಯುವಂತೆ ಮಾಡಬಹುದು. ನಾವು ಯಾವಾಗಲೂ ಇದನ್ನೆ ಮಾಡುತ್ತಿರುವೆವು. ಕೃತಕವಾಗಿ ಅವುಗಳ ಬೆಳವಣಿಗೆಯನ್ನು ತ್ವರಿತಗೊಳಿಸುತ್ತಿರುವೆವು, ಮಾನವನ ಬೆಳವಣಿಗೆಯನ್ನು ಏತಕ್ಕೆ ನಾವು ತ್ವರಿತಗೊಳಿಸಕೂಡದು? ಒಂದು ಇಡೀ ಜನಾಂಗ ಬೇಕಾದರೆ ಇದನ್ನು ಮಾಡಬಹುದು. ಅನ್ಯ ದೇಶಗಳಿಗೆ ಬೋಧಕರನ್ನು ಏತಕ್ಕೆ ಕಳುಹಿಸುವರು? ಇದರಿಂದ ಅನ್ಯ ಜನಾಂಗದ ಬೆಳವಣಿಗೆಯನ್ನು ಜಾಗೃತಗೊಳಿಸಬಹುದು. ಈಗ, ವ್ಯಕ್ತಿಯ ಬೆಳವಣಿಗೆಯನ್ನು ತ್ವರಿತಗೊಳಿಸಲಾರೆವೆ? ಇದು ಸಾಧ್ಯ. ನಾವು ತ್ವರಿತಗೊಳಿಸುವ ಕ್ರಿಯೆಗೆ ಒಂದು ಮಿತಿಯನ್ನು ಕಲ್ಪಿಸಲು ಸಾಧ್ಯವೆ? ಒಬ್ಬ ತನ್ನ ಜನ್ಮದಲ್ಲಿ ಎಷ್ಟು ಮುಂದುವರಿಯಬಲ್ಲ ಎಂಬುದನ್ನು ಹೇಳಲಾರೆವು. ಒಟ್ಟಿಗೆ ಇಷ್ಟು ಮಾತ್ರ ಸಾಧ್ಯ, ಹೆಚ್ಚು ಇಲ್ಲ ಎಂಬುದನ್ನು ನಿಮಗೆ ಹೇಳುವುದಕ್ಕೆ ಸ್ವಲ್ಪವೂ ಅಧಿಕಾರವಿಲ್ಲ. ಸನ್ನಿವೇಶಗಳು, ಘಟನೆಗಳು ಅದ್ಭುತವಾಗಿ ಮುಂದುವರಿಯುವುದಕ್ಕೆ ಅವನಿಗೆ ಸಹಾಯ ಮಾಡಬಲ್ಲವು. ಪೂರ್ಣತೆ ಮುಟ್ಟುವತನಕ ನಿಮಗೆ ಯಾವುದಾದರೂ ಅಡ್ಡಿ ಇದೆಯೆ? ಇದರಿಂದ ಏನು ಸಿದ್ದಿಸಿದ ಹಾಗಾಯಿತು? ಕೋಟ್ಯಂತರ ವರ್ಷಗಳ ಮೇಲೆ ಈ ಜನಾಂಗದಿಂದ ಉದ್ಭವಿಸಬಹುದಾದ ಪೂರ್ಣಾತ್ಮನೊಬ್ಬ, ಈಗ ಬರಲು ಸಾಧ್ಯ. ಇದನ್ನೇ ಯೋಗಿಗಳು ಹೇಳುವುದು. ಎಲ್ಲಾ ಮಹಾ ಅವತಾರಗಳೂ ದೇವದೂತರೂ ಇಂತಹ ಅದ್ಭುತ ವ್ಯಕ್ತಿಗಳು. ಅವರು ಈ ಒಂದು ಜನ್ಮದಲ್ಲಿ ಪೂರ್ಣಾತ್ಮರಾದವರು. ಜಗದ ಇತಿಹಾಸದಲ್ಲಿ ಎಲ್ಲಾ ಕಾಲದಲ್ಲಿಯೂ ಇಂತಹ ವ್ಯಕ್ತಿಗಳು ಇದ್ದರು. ಇತ್ತೀಚೆಗೆ ಇಂತಹವರೊಬ್ಬರು ಇದ್ದರು. ಇಡೀ ಮಾನವ ಜನಾಂಗದ ಬಾಳುವೆಯನ್ನು ತಮ್ಮ ಒಂದು ಜನ್ಮದಲ್ಲೇ ಬಾಳಿ ಮುಕ್ತಿಯನ್ನು ಪಡೆದಿದ್ದರು. ಈ ಬೆಳವಣಿಗೆಯ ತ್ವರಿತ ಕೂಡ ಒಂದು ನಿಯಮವನ್ನು ಅನುಸರಿಸುವುದು. ಈ ನಿಯಮವನ್ನು ಪರೀಕ್ಷಿಸಿ ಅದರ ರಹಸ್ಯವನ್ನು ಅರಿತು ನಮ್ಮ ಆವಶ್ಯಕತೆಗೆ ತಕ್ಕಂತೆ ಅದನ್ನು ಉಪಯೋಗಿಸಿದರೆ ನಾವು ಅಭಿವೃದ್ಧಿಯಾಗಬಹುದು. ನಮ್ಮ ಬೆಳವಣಿಗೆ ತ್ವರಿತವಾಗುವುದು. ನಾವು ಬೇಗ ಮುಂದುವರಿಯುವೆವು. ಈ ಜನ್ಮದಲ್ಲೇ ನಾವು ಪೂರ್ಣಾತ್ಮರಾಗಬಹುದು. ಇದೇ ನಮ್ಮ ಜೀವನದ ಉತ್ತಮ ಭಾಗ. ಮನಸ್ಸಿನ ವಿಷಯವನ್ನು ತಿಳಿಯುವ, ಅದರ ಶಕ್ತಿರಹಸ್ಯವನ್ನು ಅರಿಯುವ ಯೋಗದ ಪರಮ ಗುರಿಯೆ ಈ ಪೂರ್ಣತೆ, ಇತರರಿಗೆ ದ್ರವ್ಯ ಮತ್ತು ಇತರ ಪ್ರಾಪಂಚಿಕ ವಸ್ತುಗಳನ್ನು ಕೊಟ್ಟು ತಮ್ಮ ನಿತ್ಯ ಜೀವನದಲ್ಲಿಯೆ ಘರ್ಷಣೆಯಿಲ್ಲದೆ ಮುಂದುವರಿಯುವಂತೆ ಬೋಧಿಸುವುದು ಅಷ್ಟೇನೂ ಮುಖ್ಯವಲ್ಲ, ಗೌಣ.

ಈ ಯೋಗದ ಪ್ರಯೋಜನವೆ ಪೂರ್ಣಾತ್ಮನ ಅಭಿವ್ಯಕ್ತಿಗೆ ಸಹಾಯ ಮಾಡುವುದು. ಯುಗಯುಗಗಳು ಸಾಗುತ್ತ, ಸಾಗರದಲ್ಲಿ ಅಲೆಯ ಮೇಲೆ ತೇಲುತ್ತ, ಒಂದು ಕಡೆಯಿಂದ ಮತ್ತೊಂದು ಕಡೆಗೆ ತಾಡಿತವಾಗಿರುವ ಮರದ ತುಂಡಿನಂತೆ ಜೀವಿ, ಭೌತಿಕ ನಿಯಮಗಳಿಗೆ ಸಿಕ್ಕಿ ಅವು ಹೇಳಿದಂತೆ ಕೇಳುವುದಲ್ಲ. ನೀವು ಶಕ್ತಿವಂತರಾಗಬೇಕು, ನಿಮ್ಮ ಬೆಳವಣಿಗೆಯನ್ನು ಪ್ರಕೃತಿಗೆ ಬಿಡುವ ಬದಲು ಅದಕ್ಕೆ ನೀವೆ ಹೊಣೆಯಾಗಿ, ಈ ಕ್ಷುದ್ರ ಜನ್ಮದಿಂದ ಪಾರಾಗಬೇಕೆಂದು ಯೋಗ ಸಾರುವುದು.

ಮನುಷ್ಯ ಜ್ಞಾನ, ಶಕ್ತಿ ಆನಂದದಲ್ಲಿ ಮುಂದುವರಿಯುತ್ತಿರುವನು. ನಿರಂತರವಾಗಿ ನಾವು ಒಂದು ಜನಾಂಗವಾಗಿ ಬೆಳೆಯುತ್ತಿರುವೆವು. ಇದು ನಿಜ. ಪೂರ್ಣಸತ್ಯವೆಂಬುದನ್ನು ನಾವು ನೋಡುವೆವು, ಇದು ವ್ಯಷ್ಟಿ ಜೀವನದಲ್ಲಿಯೂ ಸತ್ಯವೆ? ಸ್ವಲ್ಪಮಟ್ಟಿಗೆ ನಿಜ. ಆದರೆ ಪುನಃ ಮಿತಿಯನ್ನು ಎಲ್ಲಿ ಕಲ್ಪಿಸಬಹುದು? ನನಗೆ ಮುಂದೆ ಕೆಲವು ಅಡಿಗಳು ಮಾತ್ರ ಕಾಣುವುದು. ಕಣ್ಣು ಮುಚ್ಚಿಕೊಂಡು ಬೇರೊಂದು ಕೋಣೆಯಲ್ಲಿ ಏನಾಗುತ್ತಿದೆ ಎಂಬುದನ್ನು ನೋಡುವವನೊಬ್ಬನನ್ನು ನಾನು ಬಲ್ಲೆ. ನೀವು ಅದನ್ನು ನಂಬುವುದಿಲ್ಲವೆಂದರೆ ಮೂರು ವಾರಗಳಲ್ಲಿ ನಿಮಗೂ ಅದು ಸಾಧ್ಯವಾಗುವಂತೆ ಅವನು ಮಾಡಬಲ್ಲ. ಇದನ್ನು ಯಾರಿಗೆ ಬೇಕಾದರೂ ಕಲಿಸಬಹುದು. ಕೆಲವರಿಗೆ ಐದು ನಿಮಿಷದಲ್ಲಿ ಮತ್ತೊಬ್ಬರ ಮನಸ್ಸಿನಲ್ಲಿ ಏನಾಗುತ್ತಿದೆ ಎಂಬುದನ್ನು ಕಲಿಸುವುದಕ್ಕೆ ಸಾಧ್ಯ. ಈ ವಿಷಯಗಳನ್ನು ಬೇಕಾದರೆ ಪ್ರತ್ಯಕ್ಷವಾಗಿ ತೋರಿಸಬಹುದು.

ಇವು ಸತ್ಯವಾಗಿದ್ದರೆ ಇದಕ್ಕೆ ಮಿತಿಯನ್ನು ನಾವು ಎಲ್ಲಿ ಹಾಕುವುದು? ಈ ಕೋಣೆಯ ಮೂಲೆಯಲ್ಲಿ ಒಬ್ಬನ ಮನಸ್ಸಿನಲ್ಲಿ ಆಗುತ್ತಿರುವುದನ್ನು ತಿಳಿಯಬಹುದಾದರೆ ಬೇರೊಂದು ಕೋಣೆಯಿಂದ ಅದನ್ನು ತಿಳಿಯಲಾರೆವೆ? ಎಲ್ಲಿಂದ ಬೇಕಾದರೂ ಏತಕ್ಕೆ ತಿಳಿಯಬಾರದು? ಸಾಧ್ಯವಿಲ್ಲ ಎಂದು ನಾವು ಹೇಳಲಾರೆವು. ಇದು ಸಾಧ್ಯವಿಲ್ಲವೆಂದು ಹೇಳುವುದಕ್ಕೆ ನಮಗೆ ಧೈರ್ಯವಿಲ್ಲ. ಇದು ಹೇಗೆ ಆಗುತ್ತದೆಯೋ ಅದು ನನಗೆ ಗೊತ್ತಿಲ್ಲ ಎಂದು ಮಾತ್ರ ಹೇಳಬಹುದು. ಭೌತಿಕಶಾಸ್ತ್ರಗಳಿಗೆ ಇಂತಹ ವಿಷಯ ಅಸಾಧ್ಯವೆಂದು ಹೇಳಲು ಅಧಿಕಾರವಿಲ್ಲ. ಇದು ನಮಗೆ ಗೊತ್ತಿಲ್ಲ ಎಂದು ಮಾತ್ರ ಹೇಳಬಹುದು. ವಿಜ್ಞಾನವು ಘಟನೆಗಳನ್ನು ಆಯ್ದು ಅವುಗಳ ಸಾಮಾನ್ಯ ನಿಯಮವನ್ನು ತಿಳಿದು ಅದನ್ನು ಒಂದು ಸಿದ್ದಾಂತಮಾಡಿ ಸತ್ಯವನ್ನು ಹೇಳಬೇಕು. ನಾವು ವಾಸ್ತವಿಕ ಅಂಶಗಳನ್ನು ಅಲ್ಲಗಳೆದರೆ ಅದು ಒಂದು ವಿಜ್ಞಾನಶಾಸ್ತ್ರ ಹೇಗೆ ಆಗಬಲ್ಲುದು?

ಮನುಷ್ಯ ಗಳಿಸಬಲ್ಲ ಶಕ್ತಿಗೆ ಒಂದು ಅಂತ್ಯವಿಲ್ಲ. ಇದು ಭಾರತೀಯನ ವೈಶಿಷ್ಟ್ಯ: ಯಾವುದರಲ್ಲಾದರೂ ಅವನಿಗೆ ಆಸಕ್ತಿ ಹುಟ್ಟಿದರೆ, ಅದರಲ್ಲೇ ನಿರತನಾಗಿ ಉಳಿದೆಲ್ಲವನ್ನು ಮರೆಯುವನು. ಭರತಖಂಡ ಎಷ್ಟು ವಿಜ್ಞಾನಶಾಸ್ತ್ರಗಳಿಗೆ ತೌರೂರು ಎಂಬುದು ನಿಮಗೆ ಗೊತ್ತಿದೆ. ಗಣಿತದಲ್ಲಿ ಪ್ರಾರಂಭವಾಯಿತು. ಈಗಲೂ ನೀವು ೧, ೨, ೩ ಮುಂತಾದುವನ್ನು ಸೊನ್ನೆಯವರೆಗೆ ಸಂಸ್ಕೃತದ ರೀತಿ ಹೇಳುತ್ತಿರುವಿರಿ. ಜೀಜಗಣಿತ ಕೂಡ ಇಂಡಿಯಾದಲ್ಲಿ ಹುಟ್ಟಿತು ಎಂಬುದು ನಿಮಗೆಲ್ಲ ಗೊತ್ತಿದೆ. ನ್ಯೂಟನ್ ಹುಟ್ಟುವುದಕ್ಕೆ ಸಹಸ್ರಾರು ವರುಷಗಳು ಮುಂಚೆ ಆಕರ್ಷಣವೆಂದರೇನೆಂಬುದು ಭಾರತೀಯರಿಗೆ ತಿಳಿದಿತ್ತು.

ಈ ವಿಚಿತ್ರವನ್ನು ನೋಡಿ, ಭರತಖಂಡದಲ್ಲಿ, ಒಂದು ಕಾಲದಲ್ಲಿ, ಮನುಷ್ಯ ಮತ್ತು ಅವನ ಮನಸ್ಸು ಅವರ ಆಸಕ್ತಿಯನ್ನೆಲ್ಲ ಆವರಿಸಿತು. ಅದು ಅಷ್ಟೊಂದು ಆಕರ್ಷಣೀಯವಾಗಿತ್ತು. ತಮ್ಮ ಗುರಿಯನ್ನು ಸಾಧಿಸುವುದಕ್ಕೆ ಬಹಳ ಸುಲಭವಾದ ಮಾರ್ಗದಂತೆ ತೋರಿತು. ನಿಯಮಾನುಸಾರ ಮಾನವನ ಮನಸ್ಸು ಏನನ್ನು ಬೇಕಾದರೂ ಸಾಧಿಸಬಹುದೆಂಬುದು ಅವನಿಗೆ ಇತ್ಯರ್ಥವಾಯಿತು. ಆದಕಾರಣವೆ ಮನಸ್ಸಿನ ಶಕ್ತಿಯು ಅಧ್ಯಯನದ ಮುಖ್ಯ ವಿಷಯವಾಯಿತು. ಯಂತ್ರ, ತಂತ್ರ, ಮಂತ್ರಗಳೆಲ್ಲ ಅದ್ಭುತವಾದುವಲ್ಲ; ವಿಜ್ಞಾನವನ್ನು ಇವಕ್ಕೆ ಮುಂಚೆ ಹೇಗೆ ಬೋಧಿಸುತ್ತಿದ್ದರೊ, ಹಾಗೆ ಈ ವಿಷಯಗಳನ್ನೂ ಬೋಧಿಸುತ್ತಿದ್ದರು. ಇಡೀ ಜನಾಂಗಕ್ಕೆ ಇದರ ಮೇಲೆ ಅಷ್ಟು ನಂಬಿಕೆ ಹುಟ್ಟಿತು. ಭೌತಿಕ ವಿಜ್ಞಾನ ಬಹುಪಾಲು ಮಾಯವಾಯಿತು. ಇದೊಂದು ಅವರ ಮನಸ್ಸಿನಲ್ಲಿತ್ತು. ಹಲವು ಬಗೆಯ ಯೋಗಿಗಳು ವಿಧವಿಧ ಪ್ರಯೋಗಗಳನ್ನು ಮಾಡಲೆತ್ನಿಸಿದರು. ಕೆಲವರು ಬೆಳಕಿನೊಂದಿಗೆ ಪರೀಕ್ಷೆ ನಡೆಸಿದರು; ಬಗೆಬಗೆಯ ಬಣ್ಣಗಳು ಯಾವರೀತಿ ಬದಲಾವಣೆಯನ್ನು ದೇಹದ ಮೇಲೆ ಉಂಟುಮಾಡುವುವು ಎಂಬುದನ್ನು ನೋಡಿದರು. ಕೆಲವು ಬಣ್ಣದ ಬಟ್ಟೆಯುಟ್ಟು, ಕೆಲವು ಬಣ್ಣದ ಕೆಳಗೆ ವಾಸಿಸಿ, ಕೆಲವು ಬಣ್ಣದ ಆಹಾರವನ್ನು ತಿನ್ನುತ್ತಿದ್ದರು. ಹೀಗೆ ಎಷ್ಟೋ ಬಗೆಯ ಪ್ರಯೋಗಗಳನ್ನು ನಡೆಸಿದರು. ಕೆಲವರು ಧ್ವನಿಯ ವಿಷಯದಲ್ಲಿ ಪ್ರಯೋಗ ನಡೆಸಿದರು. ಕಿವಿಯನ್ನು ಮುಚ್ಚಿ ತೆರೆದು ನೋಡಿದರು. ಮತ್ತೆ ಕೆಲವರು ವಾಸನೆ ಮುಂತಾದುವುಗಳ ಮೇಲೆ ಪ್ರಯೋಗ ನಡೆಸಿದರು.

ಇದರ ಮುಖ್ಯ ಉದ್ದೇಶವೆಲ್ಲ ಮೂಲಕ್ಕೆ ಹೋಗಿ ಸೂಕ್ಷ್ಮವನ್ನು ನಿಗ್ರಹಿಸುವುದಾಗಿತ್ತು. ಕೆಲವರು ನಿಜವಾಗಿ ಅತ್ಯದ್ಭುತ ಶಕ್ತಿಯನ್ನು ತೋರಿದರು. ಕೆಲವರು ಆಕಾಶದಲ್ಲಿ ತೇಲುವುದು, ಅದರಲ್ಲಿ ಸಂಚರಿಸುವುದು ಮುಂತಾದುವನ್ನು ಮಾಡಲುಪಕ್ರಮಿಸಿದರು. ಪಾಶ್ಚಾತ್ಯ ವಿದ್ವಾಂಸನೊಬ್ಬನಿಂದ ಕೇಳಿದ ಒಂದು ಕಥೆಯನ್ನು ನಿಮಗೆ ಹೇಳುತ್ತೇನೆ. ಈ ಒಂದು ಘಟನೆಯನ್ನು ನೋಡಿದ ಸಿಲೋನಿನ ಗೌರ್ನರ್‌ ಇದನ್ನು ಅವನಿಗೆ ಹೇಳಿದರಂತೆ. ಒಂದು ಹುಡುಗಿಯನ್ನು ಕೋಲುಗಳಿಂದ ಮಾಡಿದ ಒಂದು ಸ್ಟೂಲಿನ ಮೇಲೆ ಕಾಲುಮಡಿಸಿ ಕೂಡಿಸಿದರು. ಅವಳು ಕುಳಿತ ಸ್ವಲ್ಪ ಹೊತ್ತಾದ ಮೇಲೆ ಅವಳು ಕುಳಿತ ಸ್ಟೂಲಿನ ಕೋಲುಗಳನ್ನು ಒಂದಾದಮೇಲೊಂದನ್ನು ತೆಗೆಯಲು ಪ್ರಾರಂಭಿಸಿದರು. ಕೋಲನ್ನೆಲ್ಲಾ ತೆಗೆದಾದಮೇಲೆ ಆ ಹುಡುಗಿ ಮಧ್ಯ ಆಕಾಶದಲ್ಲಿ ತೇಲುತ್ತಿದ್ದಳು. ಇದರಲ್ಲಿ ಏನೋ ಉಪಾಯವಿರಬೇಕೆಂದು ಗೌರ್ನರ್‌ ತನ್ನ ಕತ್ತಿಯನ್ನು ತೆಗೆದು ಆ ಹುಡುಗಿಯ ಕೆಳಗೆ ಅದನ್ನು ವೇಗವಾಗಿ ಬೀಸಿದನು. ಅಲ್ಲೇನೂ ಇರಲಿಲ್ಲ. ಈಗ ಇದನ್ನು ಏನೆನ್ನುವಿರಿ? ಇದು ಮಾಯೆ ಗೀಯೆ ಅಲ್ಲ. ಇದೇ ವಿಚಿತ್ರ. ಭರತಖಂಡದಲ್ಲಿ ಯಾರೂ ಇದನ್ನು ಸಾಧ್ಯವಿಲ್ಲ ಎಂದು ಹೇಳಲಾರರು. ಹಿಂದೂಗಳಿಗೆ ಇದು ಸರ್ವಸಾಧಾರಣ. ಅವರು ತಮ್ಮ ವೈರಿಗಳೊಂದಿಗೆ ಹೋರಾಡಬೇಕಾದಾಗ ಅನೇಕ ವೇಳೆ ಅವರು ಹೀಗೆ ಹೇಳುವರು: “ಓ! ನಮ್ಮ ಯೋಗಿಗಳೊಬ್ಬರು ಬಂದು ಇವರನ್ನೆಲ್ಲಾ ಆಚೆಗೆ ಓಡಿಸುವರು.” ಇಡೀ ಜನಾಂಗ ಇದನ್ನು ಅಷ್ಟು ದೃಢವಾಗಿ ನಂಬುವುದು, ಕೋಲು ಕತ್ತಿಗಳಲ್ಲಿ ಏನು ಶಕ್ತಿ ಇದೆ? ಶಕ್ತಿಯೆಲ್ಲ ಆತ್ಮನಲ್ಲಿದೆ.

ಇದು ನಿಜವಾದರೆ ಇದನ್ನು ಹೇಗಾದರೂ ಮಾಡಿ ಸಾಧಿಸಬೇಕೆಂಬ ಪ್ರಲೋಭನೆ ಉಂಟಾಗುತ್ತದೆ. ಇತರ ವಿಜ್ಞಾನಗಳಂತೆ ಇಲ್ಲಿ ಕೂಡ ಮುಂದುವರಿದು ಹೋಗುವುದು ಬಹಳ ಕಷ್ಟ, ಅಲ್ಲ, ಅದಕ್ಕಿಂತಲೂ ಕಷ್ಟ. ಆದರೂ ಅನೇಕ ಜನ ಇಂತಹ ಶಕ್ತಿಯನ್ನೆಲ್ಲಾ ಸುಲಭವಾಗಿ ಸಾಧಿಸುವೆವು ಎಂದು ಭಾವಿಸುವರು. ನೀವು ಸ್ವಲ್ಪ ಆಸ್ತಿಯನ್ನು ಸಂಪಾದಿಸಬೇಕಾದರೆ ಎಷ್ಟು ಕಷ್ಟಪಡುವಿರಿ. ಇದನ್ನು ಕುರಿತು ಆಲೋಚಿಸಿ ನೋಡಿ! ವಿದ್ಯುಚ್ಛಕ್ತಿಯ ವಿಜ್ಞಾನವನ್ನು ತಿಳಿಯಬೇಕಾದರೆ, ಇಂಜನಿಯರಿಂಗ್ ವಿಷಯವನ್ನು ಕಲಿಯಬೇಕಾದರೆ, ಎಷ್ಟು ವರುಷಗಳನ್ನು ನಾವು ತೆಗೆದುಕೊಳ್ಳುತ್ತೇವೆ. ಅನಂತರ ಅದನ್ನು ಪ್ರಯೋಗಕ್ಕೆ ತರಬೇಕಾದರೆ ಇಡೀ ಜೀವನ ಕಷ್ಟಪಡಬೇಕು.

ಬಹುಪಾಲು ಇತರ ವಿಜ್ಞಾನಶಾಸ್ತ್ರಗಳು ಚಲಿಸದೆ ಇರುವ ವಸ್ತುವನ್ನು ಕುರಿತದ್ದು. ನೀವು ಕುರ್ಚಿಯನ್ನು ವಿಶ್ಲೇಷಣೆ ಮಾಡಬಹುದು. ಅದು ಹಾರಿಹೋಗುವುದಿಲ್ಲ. ಆದರೆ ಯೋಗ ಮನಸ್ಸನ್ನು ಕುರಿತದ್ದು. ಅದನ್ನು ತಿಳಿದುಕೊಳ್ಳಬೇಕೆಂದು ಯತ್ನಿಸಿದೊಡನೆಯೆ ಹಾರುವುದು. ಮನಸ್ಸು ಈಗ ಒಂದು ಸ್ಥಿತಿಯಲ್ಲಿರುವುದು, ಇನ್ನೊಂದು ಸಲ ಇನ್ನೊಂದು ಸ್ಥಿತಿಯಲ್ಲಿರುವುದು. ಹೀಗೆ ಯಾವಾಗಲೂ ಬದಲಾಯಿಸುತ್ತಿರುವುದು. ಈ ಬದಲಾವಣೆಯ ಮಧ್ಯದಲ್ಲಿಯೇ ನಾವು ಅದನ್ನು ಅರಿಯಬೇಕು, ನಿಗ್ರಹಿಸಬೇಕು. ಹಾಗಾದರೆ ಈ ಶಾಸ್ತ್ರ ಮತ್ತಷ್ಟು ಕಷ್ಟವಾಗಿರುವುದು! ಇದಕ್ಕೆ ಕಡುತರ ಸಾಧನೆ ಬೇಕು. ಅನುಷ್ಠಾನಕ್ಕೆ ಸಾಧ್ಯವಾದ ಕೆಲವು ಪ್ರವಚನಗಳನ್ನು ಏತಕ್ಕೆ ಕೊಡುವುದಿಲ್ಲವೆಂದು ಜನ ನನ್ನನ್ನು ಆಕ್ಷೇಪಿಸುವರು. ಇದು ತಮಾಷೆಯಲ್ಲ. ನಾನು ಈ ವೇದಿಕೆಯ ಮೇಲೆ ನಿಂತು ಮಾತನಾಡುವೆನು, ನೀವು ಮನೆಗೆ ಹೋಗಿ ಅದರಿಂದ ಏನೂ ಪ್ರಯೋಜನವಿಲ್ಲ ಎನ್ನುವಿರಿ. ಹೀಗೆ ಮಾತನಾಡಿ ಏನು ಪ್ರಯೋಜನ ಎಂದು ನಾನು ಭಾವಿಸುವೆನು. ಇದೆಲ್ಲಾ ಕೆಲಸಕ್ಕೆ ಬಾರದುದು ಎಂದು ನೀವು ಅನಂತರ ಹೇಳುವಿರಿ. ಏಕೆಂದರೆ ನೀವು ಇದನ್ನು ಕೆಲಸಕ್ಕೆ ಬರದಂತೆ ಮಾಡಬೇಕೆಂದಿರುವಿರಿ. ನನಗೆ ಈ ಯೋಗದ ವಿಷಯವಾಗಿ ಗೊತ್ತಿರುವುದು ಎಲ್ಲೋ ಅಲ್ಪ. ಆದರೆ ಆ ಅಲ್ಪವನ್ನು ಕಲಿಯಲು ನಾನು ಮೂವತ್ತು ವರುಷ ಪ್ರಯತ್ನ ಪಟ್ಟೆ. ನನಗೆ ತಿಳಿದ ಅಲ್ಪವನ್ನು ಕಳೆದ ಆರು ವರುಷಗಳಿಂದ ಹೇಳುತ್ತಿರುವೆ. ಅದನ್ನು ಕಲಿಯುವುದಕ್ಕೆ ಮೂವತ್ತು ವರುಷ ಹಿಡಿಯಿತು. ಮೂವತ್ತು ವರುಷ ಕಠೋರವಾಗಿ ಸಾಧನೆ ಮಾಡಬೇಕಾಯಿತು. ಕೆಲವು ವೇಳೆ ನಾನು ದಿನಕ್ಕೆ ಇಪ್ಪತ್ತು ಗಂಟೆ ಅದಕ್ಕಾಗಿ ಹೆಣಗಿರುವೆನು. ಕೆಲವು ವೇಳೆ ನಾನು ರಾತ್ರಿ ಹೊತ್ತು ಕೇವಲ ಒಂದು ಗಂಟೆ ನಿದ್ದೆ ಮಾಡುತ್ತಿದ್ದೆ. ಕೆಲವು ವೇಳೆ ರಾತ್ರಿಯಲ್ಲಿ ಕೆಲಸ ಮಾಡುತ್ತಿದ್ದೆ. ಕೆಲವು ವೇಳೆ ನಿಶ್ಶಬ್ದವಾದ ಸ್ಥಳದಲ್ಲಿ ಇರುತ್ತಿದ್ದೆ. ಕೆಲವು ವೇಳೆ ಗುಹೆಗಳಲ್ಲಿ ವಾಸಮಾಡ ಬೇಕಾಗಿತ್ತು. ಇದನ್ನು ಯೋಚಿಸಿ ನೋಡಿ! ಆದರೂ ನನಗೆ ತಿಳಿದಿರುವುದು ಬಹಳ ಅಲ್ಪ ಅಥವಾ ಏನೂ ಇಲ್ಲವೆಂದೇ ಹೇಳಬಹುದು. ಈ ಯೋಗಶಾಸ್ತ್ರದ ಹೊರವಲಯವನ್ನು ಕೂಡ ಮುಟ್ಟಿಲ್ಲವೆನ್ನಬಹುದು. ಆದರೆ ಅದು ಸತ್ಯ, ಅದ್ಭುತವಾಗಿದೆ, ವಿಶಾಲವಾಗಿದೆ ಎಂದು ತಿಳಿಯಬಲ್ಲೆ.

ನಿಮ್ಮಲ್ಲಿ ಯಾರಿಗಾದರೂ ಈ ಯೋಗವನ್ನು ತಿಳಿಯಬೇಕೆಂದು ಇಚ್ಚೆ ಇದ್ದರೆ ಇತರ ವ್ಯವಹಾರ ಜೀವನಕ್ಕೆ ಬೇಕಾದ ಶಪಥ, ಇಲ್ಲ, ಅದಕ್ಕಿಂತಲೂ ಹೆಚ್ಚಾಗಿ ಸಂಕಲ್ಪ ಮಾಡಿಕೊಂಡು ಹೊರಡಬೇಕು.

ನಮ್ಮ ವ್ಯವಹಾರಕ್ಕೆ ಎಷ್ಟು ಏಕಾಗ್ರತೆ ಬೇಕು! ಅದು ಎಷ್ಟು ಕಠಿಣವಾಗಿ ನಮ್ಮಿಂದ ದುಡಿಸಿಕೊಳ್ಳುವುದು! ತಂದೆ ತಾಯಿ, ಹೆಂಡತಿ ಮಕ್ಕಳು, ಯಾರು ಸತ್ತರೂ ಅದು ನಿಲ್ಲುವಂತಿಲ್ಲ. ನಮ್ಮ ಹೃದಯ ಶಿಥಿಲವಾಗುತ್ತಿದ್ದರೂ ಪ್ರತಿಯೊಂದು ಕ್ಷಣವೂ ಯಮಯಾತನೆ; ಆದರೂ ನಾವು ನಮ್ಮ ಕಾರ್ಯಕ್ಷೇತ್ರಕ್ಕೆ ತೊಡಗಬೇಕು. ಇದೇ ಕಾರ್ಯ, ಇದೇ ನ್ಯಾಯವಾದುದು, ಸರಿಯಾದುದು ಎನ್ನುವೆವು.

ಈ ಸಾಧನೆಗಾದರೋ ಇತರ ವ್ಯವಹಾರಕ್ಕಿಂತ ಹೆಚ್ಚು ಶ್ರಮ ನಮಗೆ ಬೇಕು. ವ್ಯಾಪಾರದಲ್ಲಿ ಎಷ್ಟೋ ಜನ ಮುಂದುವರಿಯಬಹುದು. ಆದರೆ ಇಲ್ಲಿ ಮುಂದುವರಿಯುವವರು ಬಹಳ ಅಲ್ಪ ಮಂದಿ. ಏಕೆಂದರೆ ಅದು ಇದನ್ನು ಅಧ್ಯಯನ ಮಾಡುವವನ ಪ್ರಕೃತಿ ವೈಶಿಷ್ಟ್ಯದ ಮೇಲೆ ನಿಂತಿದೆ. ವ್ಯಾಪಾರದಲ್ಲಿ ಎಲ್ಲರೂ ಶ‍್ರೀಮಂತರಾಗದೇ ಇದ್ದರೂ ಹೇಗೆ ಸ್ವಲ್ಪವನ್ನು ಸಂಪಾದಿಸಲು ಸಾಧ್ಯವೊ ಹಾಗೆಯೆ ಇದರಲ್ಲಿಯೂ ಪ್ರತಿಯೊಬ್ಬರೂ ಮುಕ್ತಾತ್ಮರಾಗದೆ ಇದ್ದರೂ ಒಂದು ಕ್ಷಣಿಕ ನೋಟ ದೊರಕಬಹುದು. ಇದರಿಂದ ಆ ಯೋಗ ಸತ್ಯ, ಅಲ್ಲಿಯ ವಿಷಯಗಳನ್ನು ಸಾಕ್ಷಾತ್ಕಾರಮಾಡಿಕೊಂಡವರು ಇರುವರು ಎಂಬುದನ್ನು ನಂಬಬಹುದು.

ಇದೇ ಯೋಗದ ಸಂಕ್ಷೇಪ ಪರಿಚಯ. ಇದು ತನ್ನ ಕಾಲಮೇಲೆ ನಿಂತಿರುವುದು. ತನ್ನ ಬೆಳಕಿನಲ್ಲಿ ಬೆಳಗುತ್ತಿರುವುದು. ಇತರ ವಿಜ್ಞಾನ ಶಾಸ್ತ್ರಗಳೊಂದಿಗೆ ಬೇಕಾದರೆ ಹೋಲಿಸಿ ನೋಡಿ ಎನ್ನುವುದು. ಇತರ ಕಡೆಗಿಂತ ಇಲ್ಲಿ ಹೆಚ್ಚಾಗಿ ಕಪಟಿಗಳು, ಮಂತ್ರವಾದಿಗಳು, ಮೋಸಗಾರರು ಇರಬಹುದು. ಏಕೆಂದರೆ ವ್ಯಾಪಾರ ಯಾವಾಗ ಹೆಚ್ಚು ಲಾಭದಾಯಕವಾಗುವುದೋ ಆಗ ಮೋಸಗಾರರೂ ಕಪಟಿಗಳೂ ಅಲ್ಲಿ ಹೆಚ್ಚು. ಆದರೆ ಇದರಿಂದ ವ್ಯಾಪಾರ ಒಳ್ಳೆಯದಲ್ಲ ಎನ್ನಲಾಗುವುದಿಲ್ಲ. ಮತ್ತೊಂದು ವಿಷಯ. ಈ ವಾದಗಳನ್ನೆಲ್ಲಾ ಕೇಳುವುದರಿಂದ ನಮ್ಮ ಬುದ್ದಿಗೆ ದೊಡ್ಡ ಕಸರತ್ತಾಗಬಹುದು. ಇಂತಹ ಅದ್ಭುತ ವಿಷಯಗಳನ್ನು ಕೇಳಿ ನಮಗೊಂದು ಬೌದ್ಧಿಕ ತೃಪ್ತಿಯಾಗಬಹುದು. ಆದರೆ ನಿಮ್ಮಲ್ಲಿ ಯಾರಾದರೂ ಇದಕ್ಕಿಂತ ಹೆಚ್ಚಾಗಿ ಕಲಿಯಲು ಇಚ್ಚಿಸಿದರೆ ಬರಿ ಉಪನ್ಯಾಸ ಕೇಳಿದರೆ ಸಾಲದು. ಇದನ್ನು ಉಪನ್ಯಾಸದಲ್ಲಿ ಕಲಿಸುವುದಕ್ಕೆ ಆಗುವುದಿಲ್ಲ. ಏಕೆಂದರೆ ಇದು ಜೀವನಕ್ಕೆ ಸಂಬಂಧಪಟ್ಟುದು. ಜೀವನ ಮಾತ್ರ ಮತ್ತೊಂದಕ್ಕೆ ಜೀವನವನ್ನು ನೀಡಬಲ್ಲದು. ನಿಮ್ಮಲ್ಲಿ ಯಾರಾದರೂ ಇದನ್ನು ಕಲಿಯಲು ದೃಢ ಸಂಕಲ್ಪಿಗಳಾಗಿದ್ದರೆ ಅವರ ಸಹಾಯಕ್ಕೆ ಬರಲು ನನಗೆ ತುಂಬಾ ಸಂತೋಷ.

