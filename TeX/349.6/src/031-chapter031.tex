
\chapter[ಅನೇಕದಂತೆ ಕಾಣುತ್ತಿರುವ ಏಕ]{ಅನೇಕದಂತೆ ಕಾಣುತ್ತಿರುವ ಏಕ\protect\footnote{\engfoot{C.W, Vol. III, P. 19}}}

\begin{center}
(೧೮೯೬ರಲ್ಲಿ ನ್ಯೂಯಾರ್ಕಿನಲ್ಲಿ ನೀಡಿದ ಪ್ರವಚನ)
\end{center}

ಎಲ್ಲಾ ಯೋಗಗಳಲ್ಲಿಯೂ ವೈರಾಗ್ಯವೇ ಬಹಳ ಮುಖ್ಯವಾದುದು. ಕರ್ಮಯೋಗಿ ಕರ್ಮಫಲಗಳನ್ನು ತ್ಯಜಿಸುವನು. ಭಕ್ತ ಪರಮೇಶ್ವರನಿಗಾಗಿ, ಆ ಸಾರ್ವಭೌಮ ಪ್ರೀತಿಗಾಗಿ, ಅಲ್ಪ ಪ್ರೀತಿಗಳನ್ನೆಲ್ಲ ತ್ಯಜಿಸುತ್ತಾನೆ. ರಾಜಯೋಗಿ ತನ್ನ ಅನುಭವವನ್ನೆಲ್ಲ ತ್ಯಜಿಸುತ್ತಾನೆ. ಏಕೆಂದರೆ ಅವನ ತತ್ತ್ವವೇ ಇಡೀ ಪ್ರಕೃತಿಯು ಜೀವಿಯ ಅನುಭವಕ್ಕಾಗಿ ಇದ್ದರೂ ಕೊನೆಗೆ, ತಾನು ಪ್ರಕೃತಿಯಲ್ಲಿ ಅಲ್ಲ ಇರುವುದು, ಅದಕ್ಕಿಂತ ಬೇರೆ ಎಂಬುದನ್ನು ಅರಿಯುವನು. ಜ್ಞಾನಯೋಗಿ ಎಲ್ಲವನ್ನೂ ತ್ಯಜಿಸುವನು. ಏಕೆಂದರೆ ಅವನ ತತ್ತ್ವದಲ್ಲಿ ಪ್ರಕೃತಿಗೆ ಹಿಂದೆಯಾಗಲೀ ಇಂದಾಗಲೀ ಮುಂದೆಯಾಗಲಿ ಸ್ಥಳವೇ ಇಲ್ಲ. ಈ ಮೇಲಿನ ಸಿದ್ದಾಂತಗಳಿಂದ, ಈ ಉನ್ನತ ವಿಷಯಗಳಲ್ಲಿ ಪ್ರಯೋಜನದ ಪ್ರಶ್ನೆಯೇ ಏಳುವುದಿಲ್ಲ. ಪ್ರಶ್ನೆಯನ್ನು ಹಾಕುವುದೇ ತಪ್ಪು. ಈ ಪ್ರಶ್ನೆಯನ್ನು ಕೇಳಿದರೂ ಇದನ್ನು ಸರಿಯಾಗಿ ವಿಶ್ಲೇಷಣೆ ಮಾಡಿದರೆ ಈ ಪ್ರಯೋಜನದ ಪ್ರಶ್ನೆಯಲ್ಲಿ ನಾವು ಏನನ್ನು ನೋಡುತ್ತೇವೆ? ಸುಖದ ಆದರ್ಶವನ್ನು. ಯಾವುದು ಮನುಷ್ಯನಿಗೆ ಹೆಚ್ಚು ಸುಖವನ್ನು ತರಬಲ್ಲದೋ, ಅದು ಅವನ ಪ್ರಾಪಂಚಿಕ ಸ್ಥಿತಿಗತಿಗಳನ್ನು ಉತ್ತಮಪಡಿಸಲಾರದ ಶ್ರೇಷ್ಠ ಸತ್ಯಗಳಿಗಿಂತ ಹೆಚ್ಚು ಪ್ರಯೋಜನಕಾರಿ ಎಂದು ಜನರು ಭಾವಿಸುವರು. ಎಲ್ಲಾ ಶಾಸ್ತ್ರಗಳಿರುವುದು ಈ ಗುರಿಗಾಗಿ, ಮಾನವನಿಗೆ ಸುಖವನ್ನು ತರುವುದಕ್ಕಾಗಿ. ಯಾವುದು ಅವನಿಗೆ ಹೆಚ್ಚು ಸುಖವನ್ನು ಕೊಡಬಲ್ಲದೋ, ಅದನ್ನು ಸ್ವೀಕರಿಸುತ್ತಾನೆ, ಯಾವುದು ಕಡಿಮೆ ಸುಖವನ್ನು ಕೊಡುವುದೊ ಅದನ್ನು ತ್ಯಜಿಸುತ್ತಾನೆ. ಸುಖ ದೇಹದಲ್ಲಾಗಲಿ, ಮನಸ್ಸಿನಲ್ಲಾಗಲಿ, ಆತ್ಮನಲ್ಲಾಗಲಿ ಇದೆ ಎಂಬುದನ್ನು ನೋಡಿರುವೆವು. ಪ್ರಾಣಿಗಳಲ್ಲಿ ಮತ್ತು ಪ್ರಾಣಿಗಳಂತಿರುವ ಅತಿ ಕೀಳು ಮಾನವರಲ್ಲಿ ಸುಖವಿರುವುದು ದೇಹದಲ್ಲಿ. ಯಾರೂ ಉಪವಾಸವಿರುವ ನಾಯಿಯಷ್ಟು ಅಥವಾ ತೋಳದಷ್ಟು ಆನಂದದಿಂದ ಊಟಮಾಡಲಾರರು. ನಾಯಿಗಳಲ್ಲಿ ಮತ್ತು ತೋಳಗಳಲ್ಲಿ ಸುಖ ದೇಹದಲ್ಲಿ ಅಂತರ್ಗತವಾಗಿದೆ. ಮನುಷ್ಯರಲ್ಲಿ ಆನಂದವನ್ನು ಇನ್ನೂ ಮೇಲಿನ ಮಟ್ಟದಲ್ಲಿ ಕಾಣುತ್ತೇವೆ; ಅದೇ ಆಲೋಚನೆಯಲ್ಲಿ. ಜ್ಞಾನಿಯ ಆನಂದ ಶ್ರೇಷ್ಠವಾದುದು, ಏಕೆಂದರೆ, ಅದು ಆತನಲ್ಲಿ ಪ್ರತಿಷ್ಟಿತವಾಗಿರುವುದು. ಜ್ಞಾನಿಗೆ ಆತ್ಮಜ್ಞಾನವೇ ಶ್ರೇಷ್ಠ ಪ್ರಯೋಜನ. ಏಕೆಂದರೆ ಇದೇ ಅವನಿಗೆ ಅತ್ಯಂತ ಹೆಚ್ಚಿನ ಆನಂದವನ್ನು ಈಯಬಲ್ಲದು. ಇಂದ್ರಿಯಗಳನ್ನು ತೃಪ್ತಿಪಡಿಸುವುದರಿಂದ ಅಥವಾ ಬಾಹ್ಯವಸ್ತುಗಳನ್ನು ಶೇಖರಿಸುವುದರಿಂದ ಅವನಿಗೆ ಯಾವ ಪ್ರಯೋಜನವೂ ಇಲ್ಲ. ಏಕೆಂದರೆ ಅವನಿಗೆ ಜ್ಞಾನದಲ್ಲಿ ದೊರಕುವ ಆನಂದ ಅದರಲ್ಲಿ ದೊರಕುವುದಿಲ್ಲ. ಅಂತೂ ಜ್ಞಾನವೇ ಪರಮಗುರಿ; ನಮ್ಮ ಅನುಭವಕ್ಕೆ ತಿಳಿದ ಪರಮಾನಂದವೇ ಅದು. ಯಾರು ಅಜ್ಞಾನದಲ್ಲಿ ಕೆಲಸ ಮಾಡುತ್ತಿರುವರೋ ಅವರೆಲ್ಲ ದೇವತೆಗಳ ಊಳಿಗದವರು. ಇಲ್ಲಿ ದೇವತೆಗಳನ್ನು ಜ್ಞಾನಿ ಎಂಬ ಅರ್ಥದಲ್ಲಿ ಉಪಯೋಗಿಸಿದೆ. ಯಾರು ದುಡಿಯುವರೊ, ಶ್ರಮಪಡುವರೋ, ಯಂತ್ರದಂತೆ ಕೆಲಸ ಮಾಡುವರೋ, ಅವರಾರೂ ಜೀವನವನ್ನು ಅನುಭವಿಸಲಾರರು. ನಿಜವಾಗಿ ಇದನ್ನು ಅನುಭವಿಸುವವನು ಜ್ಞಾನಿ. ಒಬ್ಬ ಶ‍್ರೀಮಂತ ಒಂದು ಲಕ್ಷ ಡಾಲರುಗಳನ್ನು ಖರ್ಚುಮಾಡಿ ಒಂದು ಚಿತ್ರವನ್ನು ಕೊಳ್ಳುತ್ತಾನೆ ಎಂದು ಭಾವಿಸಿ, ಆದರೆ ಅದನ್ನು ನೋಡಿ ಆನಂದಿಸಬಲ್ಲವನು ಕಲೆಗಾರ. ಶ‍್ರೀಮಂತನಿಗೆ ಕಲೆಯ ಅಭಿರುಚಿ ಇಲ್ಲದೆ ಇದ್ದರೆ ಆ ಚಿತ್ರದಿಂದ ಏನೂ ಪ್ರಯೋಜನವಿಲ್ಲ. ಅವನು ಕೇವಲ ಆ ಚಿತ್ರದ ಒಡೆಯ ಅಷ್ಟೆ. ಮೂರ್ಖ ಎಂದಿಗೂ ಆನಂದಿಸಲಾರ. ತನಗೆ ಗೊತ್ತಿಲ್ಲದೆ ಅವನು ಇತರರಿಗಾಗಿ ದುಡಿಯಬೇಕಾಗಿದೆ.

ಇರುವುದು ಒಂದು ಆತ್ಮ, ಎರಡಲ್ಲ ಎಂಬುದನ್ನು ಅದೈತ ಸಿದ್ದಾಂತದ ಮೂಲಕ ನೋಡಿದೆವು. ಈ ಪ್ರಪಂಚದಲ್ಲಿರುವುದೆಲ್ಲ ಒಂದು ಎಂಬುದನ್ನು ನೋಡಿದೆವು. ಆ ಒಂದನ್ನು ಪಂಚೇಂದ್ರಿಯಗಳ ಮೂಲಕ ನೋಡಿದಾಗ ಅದು ಭೌತಿಕ ಪ್ರಪಂಚವಾಗುವುದು. ಅದನ್ನು ಮನಸ್ಸಿನ ಮೂಲಕ ನೋಡಿದಾಗ ಅದು ಭಾವನೆಯಂತೆ ಮತ್ತು ಆಲೋಚನೆಯಂತೆ ಕಾಣುವುದು. ಅದು ಹೇಗಿದೆಯೋ ಹಾಗೆ ನೋಡಿದರೆ ಅದು ಅಖಂಡವಾದದು. ನೀವು ಇದನ್ನು ಗಮನಿಸಬೇಕು; ಮಾನವನಲ್ಲಿ ಒಂದು ಜೀವವಿದೆ ಎಂದು ಅಲ್ಲ, ಮೊದಲು ನಾನು ವಿವರಿಸಬೇಕಾದರೆ ಹಾಗೆ ಊಹಿಸುವುದು ಆವಶ್ಯಕವಾಗಿದ್ದಿರಬಹುದು. ಇರುವುದೆಲ್ಲ ಒಂದೇ, ಅದೇ ಆತ್ಮ. ನಾವು ಅದನ್ನು ಪಂಚೇಂದ್ರಿಯಗಳ ಮೂಲಕ ನೋಡಿದಾಗ, ಸ್ಥೂಲ ಕಲ್ಪನೆಗಳ ಮೂಲಕ ರೂಪಿಸಲು ಯತ್ನಿಸಿದಾಗ ಅದನ್ನು ಒಂದು ದೇಹ ಎನ್ನುತ್ತೇವೆ; ಅದರ ನೈಜಸ್ಥಿತಿಯಿಂದ ನೋಡಿದರೆ ಆತ್ಮವಾಗುವುದು. ಇರುವ ಏಕಮಾತ್ರ ವಸ್ತು ಅದೇ. ದೇಹ, ಮನಸ್ಸು, ಪ್ರಾಣ ಇವು ಒಂದರಲ್ಲಿ ಇರುವುವು ಎಂದಲ್ಲ. ನಾವು ವಿವರಿಸುವಾಗ ಹೀಗೆ ಹೇಳಿರಬಹುದು, ಅಷ್ಟೆ. ಇರುವುದೆಲ್ಲ ಆತ್ಮವೆ. ಅದನ್ನು ಕೆಲವು ವೇಳೆ ದೇಹ ಎನ್ನುವರು; ಕೆಲವು ವೇಳೆ ಮನಸ್ಸು ಎನ್ನುವರು; ಕೆಲವು ವೇಳೆ ಜೀವ ಎನ್ನುವರು; ನಾವು ಅದನ್ನು ಯಾವ ದೃಷ್ಟಿಯಿಂದ ನೋಡುತ್ತೇವೆಯೋ ಆ ದೃಷ್ಟಿಯಿಂದ ಹೇಳುವೆವು. ಇರುವುದೊಂದೆ. ಮೂಢರು ಅದನ್ನು ಪ್ರಪಂಚ ಎನ್ನುವರು. ಜ್ಞಾನದಲ್ಲಿ ಮುಂದುವರಿದು ಹೋದರೆ ಅದನ್ನೇ ಭಾವನಾ ಪ್ರಪಂಚ ಎನ್ನುವರು. ಪರಮಜ್ಞಾನ ಪ್ರಾಪ್ತವಾದಾಗ ಭ್ರಾಂತಿಯೆಲ್ಲ ನಾಶವಾಗಿ ಎಲ್ಲಾ ಆತ್ಮಮಯ ಎಂದು ಅರಿಯುವರು. ನಾನೇ ಅದು, ಅದೇ ಕೊನೆಯ ನಿರ್ಣಯ. ಪ್ರಪಂಚದಲ್ಲಿ ಮೂರೂ ಇಲ್ಲ, ಎರಡೂ ಇಲ್ಲ. ಇರುವುದೆಲ್ಲ ಒಂದೆ. ಆ ಒಂದು ಮಾಯೆಯಿಂದ ಹಲವದರಂತೆ ಕಾಣುತ್ತಿದೆ, ಹಗ್ಗ ಭ್ರಾಂತಿಯಿಂದ ಹಾವಾಗಿ ಕಾಣುವ ಹಾಗೆ. ಹಗ್ಗದಲ್ಲಿ ಹಾವೊಂದು ಮತ್ತು ಹಗ್ಗವೊಂದು ಎಂದು ಎರಡಿಲ್ಲ. ಯಾರೂ ಅಲ್ಲಿ ಏಕಕಾಲದಲ್ಲಿ ಎರಡನ್ನೂ ನೋಡುವುದಿಲ್ಲ. ದ್ವೈತ ಮತ್ತು ಅದ್ವೈತಗಳೆಂಬುವು ಒಳ್ಳೆಯ ತಾತ್ವಿಕ ಪದಗಳೇನೊ ನಿಜ. ಆದರೆ ನಾವು ನಿಜವಾಗಿ ನೋಡುವಾಗ ಏಕಕಾಲದಲ್ಲಿ ಸತ್ಯ ಮಿಥ್ಯಗಳೆರಡನ್ನೂ ನೋಡಲಾರೆವು. ನಾವೆಲ್ಲ ಹುಟ್ಟು\break ಅದ್ವೈತಿಗಳು. ಹಾಗಾಗದೆ ನಿರ್ವಾಹವಿಲ್ಲ. ನಾವು ಯಾವಾಗಲೂ ಒಂದನ್ನೇ ನೋಡುವುದು. ನಾವು ಹಗ್ಗವನ್ನು ನೋಡಿದಾಗ ಹಾವನ್ನು ನೋಡುವುದೇ ಇಲ್ಲ. ಅದು ಮಾಯವಾಗಿದೆ. ನೀವು ಭ್ರಾಂತಿಯನ್ನು ನೋಡಿದಾಗ ಸತ್ಯವನ್ನು ನೋಡಲಾರಿರಿ. ದೂರದಲ್ಲಿ ನಿಮ್ಮ ಸ್ನೇಹಿತನೊಬ್ಬ ಬರುತ್ತಿರುವುದನ್ನು ನೀವು ನೋಡಿದಿರಿ ಎಂದು ಇಟ್ಟುಕೊಳ್ಳಿ. ಅವನು ನಿಮಗೆ ಚೆನ್ನಾಗಿ ಗೊತ್ತಿದ್ದವನು. ಆದರೆ ಮಂಜಿನಿಂದಲೋ ಏನೋ ಅವನು ಸರಿಯಾಗಿ ಕಾಣದೆ, ಅವನು ಇನ್ನಾರೋ ಎಂದು ಭಾವಿಸುವಿರಿ. ನೀವು ಅವನನ್ನು ಮತ್ತೊಬ್ಬನಂತೆ ಭಾವಿಸಿದಾಗ ಅಲ್ಲಿ ನಿಮ್ಮ ಸ್ನೇಹಿತನನ್ನು ನೋಡುವುದೇ ಇಲ್ಲ, ಅವನು ಮಾಯವಾಗಿ ಹೋಗಿರುವನು. ನೀವು ಒಬ್ಬನನ್ನು ಮಾತ್ರ ನೋಡುತ್ತಿರುವಿರಿ. ನೀವು `ಎ' ಅನ್ನು `ಬಿ' ಎಂದು ತಿಳಿದಾಗ ಅಲ್ಲಿ `ಎ' ಅನ್ನು ನೋಡುವುದೇ ಇಲ್ಲ. ಪ್ರತಿ ಸಲವೂ ನೀವು ಒಬ್ಬನನ್ನು ಮಾತ್ರ ನೋಡುತ್ತೀರಿ. ನೀವು ದೇಹವೆಂದು ಭಾವಿಸಿದಾಗ ನೀವು ದೇಹ ಮಾತ್ರ, ಮತ್ತೇನೂ ಅಲ್ಲ. ಬಹುಪಾಲು ಮಂದಿ ಜನರು ನೋಡುವುದು ಹೀಗೆ; ಅವರು ಜೀವ, ಮನಸ್ಸು ಮುಂತಾದ ಹಲವು ಮಾತುಗಳನ್ನು ಆಡಬಹುದು. ಆದರೆ ಅವರು ಗ್ರಹಿಸುವುದು ಸ್ಥೂಲ ಆಕಾರವನ್ನು ಮಾತ್ರ, ನೋಡುವುದು ಶಬ್ದ ಸ್ಪರ್ಶ ರೂಪ ರಸ ಗಂಧ ಇವನ್ನು ಮಾತ್ರ. ಕೆಲವರು ಕೆಲವು ವೇಳೆ ತಾವು ಕೇವಲ ಆಲೋಚನೆ ಎಂದು ಭಾವಿಸುವರು. ನಿಮಗೆ ಸರ್ ಹಂಫ್ರಿ ಡೇವಿಯ ಕಥೆ ಗೊತ್ತಿರಬಹುದು. ಅವನು ನಗು ಅನಿಲದ ವಿಷಯವಾಗಿ ಗ್ಲಾಸಿನೊಳಗೆ ಪ್ರಯೋಗ ಮಾಡುತ್ತಿದ್ದ. ಆಗ ಒಂದು ಗ್ಲಾಸಿನ ಕೊಳವಿ ಒಡೆದುಹೋಗಿ ಅಲ್ಲಿಂದ ಬರುತ್ತಿದ್ದ ಗಾಳಿಯನ್ನು ಅವನು ಸೇವಿಸಿದ. ಅವನು ಅನಂತರ ಕೆಲವು ಕ್ಷಣಗಳವರೆಗೆ ಶಿಲಾ ಪ್ರತಿಮೆಯಂತೆ ನಿಂತ. ಅನಂತರ ಅವನು ತಾನು ಆ ಸ್ಥಿತಿಯಲ್ಲಿರುವವರೆಗೆ ಇಡೀ ಜಗತ್ತು ಆಲೋಚನೆಗಳಿಂದಾದದ್ದು ಎಂಬುದಾಗಿ ಕಂಡೆ ಎಂದು ಹೇಳಿದನು. ಆ ಅನಿಲ ತಾತ್ಕಾಲಿಕವಾಗಿ ಅವನ ದೇಹದ ಇರವನ್ನು ಮರೆಯುವಂತೆ ಮಾಡಿತು. ಯಾವುದನ್ನು ದೇಹರೂಪವಾಗಿ ನೋಡುತ್ತಿದ್ದನೋ ಅದನ್ನೇ ಆಲೋಚನಾರೂಪವಾಗಿ ನೋಡತೊಡಗಿದ. ಪ್ರಜ್ಞೆ ಇನ್ನೂ ಮೇಲೆ ಹೋದರೆ, ಈ ಅಲ್ಪ ಪ್ರಜ್ಞೆ ಒಂದೇ ಅಖಂಡ ಸಚ್ಚಿದಾನಂದ, ಸರ್ವವ್ಯಾಪಿ ಎಂದು ಗೊತ್ತಾಗುವುದು. ಯಾವುದು ಜ್ಞಾನ ಮಾತ್ರ ಆಗಿದೆಯೋ ಯಾವುದು ಆನಂದವೇ ಆಗಿದೆಯೊ ಅದು ಅಪ್ರಮೇಯ; ಅದು ಎಲ್ಲಾ ಮಿತಿಗಳನ್ನೂ ಮೀರಿ ಹೋಗಿದೆ. ಅದು ನಿತ್ಯಮುಕ್ತ, ಎಂದಿಗೂ ಬದ್ಧವಲ್ಲ. ಅದು ಆಕಾಶದಂತೆ ಬದಲಾವಣೆಗೆ ನಿಲುಕದುದಾಗಿದೆ. ಇಂತಹದು ಧ್ಯಾನದ ಸಮಯದಲ್ಲಿ ನಿಮ್ಮ ಹೃದಯಾಂತರಾಳದಲ್ಲಿ ಗೋಚರಿಸುವುದು.

ಈ ಎಲ್ಲಾ ಧರ್ಮಗಳಲ್ಲಿಯೂ ಇರುವ ಸ್ವರ್ಗ ನರಕಗಳೆಂಬ ಭಾವನೆಯನ್ನು ಅದ್ವೈತ ಹೇಗೆ ವಿವರಿಸಬಲ್ಲುದು? ಒಬ್ಬ ಕಾಲವಾದರೆ ಅವನು ಸ್ವರ್ಗಕ್ಕೊ ನರಕಕ್ಕೊ ಹೋಗುವನು; ಇಲ್ಲೊ ಅಲ್ಲಿಯೊ ಹೋಗುವನು ಎನ್ನುವರು; ಅಥವಾ ಅವನು ಬೇರೊಂದು ದೇಹದಲ್ಲಿ ಸ್ವರ್ಗದಲ್ಲೊ ನರಕದಲ್ಲೋ ಅಥವಾ ಇನ್ನೂ ಯಾವ ಲೋಕದಲ್ಲೊ ಜನಿಸುವನು ಎನ್ನುವರು. ಇದೆಲ್ಲ ಒಂದು ಭ್ರಾಂತಿ. ನಿಜವಾಗಿ ಹುಟ್ಟಿಯೂ ಇಲ್ಲ, ಸಾಯುವುದೂ ಇಲ್ಲ. ಸ್ವರ್ಗವೂ ಇಲ್ಲ, ನರಕವೂ ಇಲ್ಲ, ಈ ಜಗತ್ತೂ ಇಲ್ಲ. ಇವುಗಳು ಎಂದೂ ಇರಲೇ ಇಲ್ಲ. ನೀವು ಮಕ್ಕಳಿಗೆ ಬೇಕಾದಷ್ಟು ದೆವ್ವದ ಕಥೆಗಳನ್ನು ಹೇಳಿ ಅವನ್ನು ಸಂಜೆ ಹೊರಗೆ ಬಿಡಿ. ದಾರಿಯಲ್ಲಿ ಒಂದು ಮೋಟುಮರವಿದೆ. ಮಗು ಅಲ್ಲಿ ಏನು ನೋಡುವುದು? ಅವನನ್ನು ತಿನ್ನುವುದಕ್ಕೆ ಕೈ ಎತ್ತಿಕೊಂಡು ನಿಂತಿರುವ ದೆವ್ವವನ್ನು ಕಾಣುವುದು. ದಾರಿಯ ಮೂಲೆಯಿಂದ ಒಬ್ಬ ತನ್ನ ಪ್ರಿಯತಮಳನ್ನು ಹುಡುಕಿಕೊಂಡು ಬರುತ್ತಿರುವನು. ಅವನು ತನಗಾಗಿ ಕಾಯುತ್ತಿರುವ ಹುಡುಗಿ ಅವಳು ಎಂದು ಭಾವಿಸುವನು. ಪೋಲೀಸಿನವನು ಅದನ್ನು ಕಳ್ಳ ಎಂದು ಭಾವಿಸುವನು. ಬೇರೆ ಬೇರೆಯಾಗಿ ಕಂಡರೂ ಅಂದೊಂದು ಮೊಟುಮರವೇ ಸದಾ ಕಾಲವೂ ಆಗಿರುವುದು. ಆ ಮೋಟುಮರ ನಿಜ. ಬೇರೆ ಬೇರೆಯಂತೆ ಕಂಡುದೆಲ್ಲ ನೋಡುವವರು ಆರೋಪಿಸಿರುವ ಊಹೆಗಳು ಮಾತ್ರ. ಈ ಆತ್ಮವೊಂದೇ ಇರುವುದು. ಇದು ಬರುವುದೂ ಇಲ್ಲ, ಹೋಗುವುದೂ ಇಲ್ಲ. ಮನುಷ್ಯ ಅಜ್ಞಾನಿಯಾಗಿರುವಾಗ ಸ್ವರ್ಗಕ್ಕೋ ಅಥವಾ ಎಲ್ಲಿಗೊ ಹೋಗಬೇಕೆಂದು ಇಚ್ಛಿಸುವನು. ಅವನು ಇಡೀ ಜೀವನ ಇದನ್ನೇ ಆಲೋಚಿಸುತ್ತಿರುವನು. ಈ ಭೂಮಿಯ ಸ್ವಪ್ನ ಮಾಯವಾದ ಮೇಲೆ ಇದನ್ನೇ ದೇವದೇವತೆಗಳು ಹಾರಾಡುತ್ತಿರುವ ಸ್ವರ್ಗದಂತೆ ಕಾಣುವನು. ಒಬ್ಬ ಇಡೀ ಜೀವನ ತನ್ನ ಪಿತೃಗಳನ್ನು ನೋಡಬೇಕೆಂದು ಹಾರೈಸುತ್ತಿದ್ದರೆ, ಆಡಮ್ಮಿನಿಂದ ಹಿಡಿದು ಎಲ್ಲರನ್ನೂ ಅವನು ನೋಡುವನು. ಏಕೆಂದರೆ ಇವನೇ ಅವರನ್ನೆಲ್ಲ ಸೃಷ್ಟಿಸುವವನು. ಅವನು ಇನ್ನೂ ಮೂಢನಾಗಿ ಮತಭ್ರಾಂತರು ವಿವರಿಸುವ ನರಕವನ್ನು ಮತ್ತು ಅಲ್ಲಿ ಕೊಡುವ ಚಿತ್ರವಿಚಿತ್ರ ಹಿಂಸೆಗಳನ್ನು ಕೇಳಿ ಭಯಗೊಂಡಿದ್ದರೆ, ಇದನ್ನೇ ಒಂದು ನರಕದಂತೆ ಅವನು ಅನಂತರ ಕಾಣುವನು. ಹುಟ್ಟು ಸಾವುಗಳೆಂದರೆ ಕೇವಲ ನಾವು ನೋಡುತ್ತಿರುವ ದೃಷ್ಟಿಯ ಬದಲಾವಣೆ, ಅಷ್ಟೆ. ನೀವೂ ಚಲಿಸುವುದಿಲ್ಲ, ಯಾವುದರ ಮೇಲೆ ನಿಮ್ಮ ದೃಷ್ಟಿಯನ್ನು ಆರೋಪ ಮಾಡುವಿರೋ ಅದೂ ಚಲಿಸುವುದಿಲ್ಲ. ನೀವು ಸ್ಥಾಣು; ಅವಿಕಾರಿ. ಹೇಗೆ ಬಂದು ಹೋಗ ಬಲ್ಲಿರಿ? ಇದು ಅಸಾಧ್ಯ. ನೀವು ಸರ್ವವ್ಯಾಪಿಯಾಗಿರುವಿರಿ. ಆಕಾಶ ಎಂದಿಗೂ ಚಲಿಸುವುದಿಲ್ಲ. ಅದರ ಕೆಳಗೆ ಇರುವ ಮೋಡಗಳು ಚಲಿಸುತ್ತವೆ. ನಾವು ಆಕಾಶವೇ ಚಲಿಸುತ್ತದೆ ಎಂದು ಭಾವಿಸಬಹುದು. ನೀವು ರೈಲಿನ ಒಳಗೆ ಇರುವಾಗ ಭೂಮಿಯೇ ಚಲಿಸುತ್ತಿರುವಂತೆ ಕಾಣುತ್ತದೆ. ಚಲಿಸುತ್ತಿರುವುದು ರೈಲು. ನೀವು ಇರುವಲ್ಲಿಯೇ ಇರುವಿರಿ. ಹಲವು ಕನಸುಗಳೆಂಬ ಮೋಡಗಳು ಚಲಿಸುತ್ತಿವೆ. ಯಾವ ಸಂಬಂಧವೂ ಇಲ್ಲದೆ ಒಂದು ಕನಸು ಮತ್ತೊಂದು ಕನಸನ್ನು ಅನುಸರಿಸುತ್ತದೆ. ಈ ಪ್ರಪಂಚದಲ್ಲಿ ಒಂದು ನಿಯಮ, ಸಂಬಂಧ ಎಂಬುದು ಇಲ್ಲ. ಆದರೆ ನಾವು ಏನೋ ದೊಡ್ಡ ಸಂಬಂಧವಿದೆ ಎಂದು ಊಹಿಸುವೆವು. ನಿಮ್ಮಲ್ಲಿ ಎಲ್ಲರೂ `Alice in Wonderland' (ಪಾತಾಳದಲ್ಲಿ ಪಾಪಚ್ಚಿ) ಎಂಬ ಪುಸ್ತಕವನ್ನು ಓದಿರಬಹುದು. ಈ ಶತಮಾನದಲ್ಲಿ ಮಕ್ಕಳಿಗಾಗಿ ಬರೆದ ಅತಿ ಶ್ರೇಷ್ಠ ಗ್ರಂಥ ಅದು. ನಾನು ಅದನ್ನು ಓದಿದಾಗ ನನಗೆ ಅತ್ಯಾನಂದವಾಯಿತು. ಅಂತಹ ಪುಸ್ತಕವನ್ನು ಮಕ್ಕಳಿಗೆ ಬರೆಯಬೇಕೆಂದು ನಾನು ಯಾವಾಗಲೂ ಯೋಚಿಸುತ್ತಿದ್ದೆ. ನನಗೆ ಅದರಲ್ಲಿ ಅತ್ಯಾನಂದವನ್ನು ಉಂಟುಮಾಡಿದ್ದು ಯಾವುದೆಂದರೆ ನಾವು ಯಾವುದನ್ನು ಅಸಂಬದ್ಧ ಎನ್ನುವೆವೊ ಅದು. ಅಲ್ಲಿ ಯಾವ ಸಂಬಂಧವೂ ಇಲ್ಲ. ಒಂದು ಭಾವನೆ ಬರುತ್ತದೆ. ಅದು ಯಾವ ಕಾರಣವೂ ಇಲ್ಲದೆ ಮತ್ತಾವುದೊ ಒಂದಕ್ಕೆ ನೆಗೆಯುತ್ತದೆ. ನೀವು ಮಕ್ಕಳಾಗಿದ್ದಾಗ ಅಲ್ಲಿ ಯಾವುದೊ ಒಂದು ಅದ್ಭುತವಾದ ನಿಯಮಾವಳಿ ಇದೆ ಎಂದು ಭಾವಿಸಿದ್ದಿರಿ. ಗ್ರಂಥಕರ್ತೃ ತಾನು ಮಗುವಾಗಿದ್ದಾಗ ಹೇಗೆ ಭಾವಿಸಿದ್ದನೊ ಆ ಭಾವನೆಗಳನ್ನೆಲ್ಲಾ ಕೂಡಿಸಿ ಮಕ್ಕಳಿಗೆ ಒಂದು ಪುಸ್ತಕ ಬರೆದನು. ಇತರರು ತಮ್ಮ ಭಾವನೆಯನ್ನು ಮಕ್ಕಳು ಸ್ವೀಕರಿಸಬೇಕೆಂದು ಬರೆಯುವ ಪುಸ್ತಕಗಳೆಲ್ಲ ಕೆಲಸಕ್ಕೆ ಬಾರದವು. ನಾವು ದೊಡ್ಡವರಾದ ಮಕ್ಕಳು ಅಷ್ಟೆ. ಪ್ರಪಂಚ ಯಾವಾಗಲೂ ಒಂದೇ ಸಮನಾಗಿಯೇ ಇರುತ್ತದೆ, ಅಲ್ಲಿ ಯಾವ ಕಾರ್ಯಕಾರಣ ಸಂಬಂಧವೂ ಇಲ್ಲ. ಎಲ್ಲಾ ಆಲಿಸಳ ವಿಚಿತ್ರ ಲೋಕದಲ್ಲಿದ್ದಂತೆ. ಒಂದೇ ರೀತಿಯಲ್ಲಿ ಘಟನೆಗಳು ಕೆಲವು ವೇಳೆ ಸಂಭವಿಸಿದರೆ, ಅಲ್ಲೊಂದು ಕಾರ್ಯಕಾರಣ ಸಂಬಂಧವಿದೆ ಎಂದು ಭಾವಿಸಿ, ಮುಂದೆಯೂ ಇದು ಆಗುವುದು ಎನ್ನುವೆವು. ಈ ಕನಸು ಹೋಗಿ ಮತ್ತೊಂದು ಬಂದರೆ, ಅದೂ ಹಿಂದಿನಂತೆಯೇ ನಿಯಮಬದ್ಧವೆನ್ನುವೆವು. ನಾವು ಕನಸು ಕಾಣುವಾಗ ಅಲ್ಲಿರುವುದಕ್ಕೆ ಒಂದು ಸಂಬಂಧ ಇದೆ ಎಂದು ಊಹಿಸುವೆವು. ನಾವು ಕನಸು ಕಾಣುವಾಗ ಅವೆಲ್ಲ ಅಸಂಗತ ಎಂಬ ಭಾವನೆ ಹೊಳೆಯುವುದೇ ಇಲ್ಲ. ನಾವು ಜಾಗೃತಾವಸ್ಥೆಗೆ ಬಂದ ಮೇಲೆ ಮಾತ್ರ ಅದರಲ್ಲಿ ಒಂದು ಸಂಬಂಧವಿಲ್ಲ ಎಂದು ಗೊತ್ತಾಗುವುದು. ನಾವು ಪ್ರಪಂಚವೆಂಬ ಕನಸಿನಿಂದ ಜಾಗೃತರಾಗಿ ಸತ್ಯದೊಂದಿಗೆ ಹೋಲಿಸಿ ನೋಡಿದಾಗ ಕನಸೆಲ್ಲ ಅರ್ಥವಿಲ್ಲದ್ದು, ಅಸಮಂಜಸವಾದುದು, ಮತ್ತು ಕನಸಿನ ವಸ್ತುಗಳೆಲ್ಲ ನಮ್ಮ ಮುಂದೆ ಸಂಚರಿಸುತ್ತಿದ್ದ ಅಸಂಬದ್ದ ವಸ್ತುಗಳೆಂದು ಗೊತ್ತಾಗುವುದು. ಅವು ಎಲ್ಲಿಂದ ಹೇಗೆ ಬಂದವು ಎಂಬುದೇನೂ ನಮಗೆ ಗೊತ್ತಾಗುವುದಿಲ್ಲ. ಆದರೆ ಅವೆಲ್ಲ ಒಂದು ದಿನ ಕೊನೆಗಾಣುವುವು ಎಂದು ಗೊತ್ತಿದೆ. ಇದೇ ಮಾಯೆ. ಅವೆಲ್ಲ ಓಡುತ್ತಿರುವ ಮೇಘಮಾಲೆಯ ಮಂದೆಯಂತೆ; ಅದರಂತೆಯೇ ಬದಲಾಗುತ್ತಿದೆ ಜಗತ್ತು. ಸ್ಥಿರವಾಗಿರುವವನು ಸೂರ್ಯ ಮಾತ್ರ. ಅದೇ ನೀವು. ನೀವು ಹೊರಗಿನಿಂದ ಅಚಲವಾಗಿರುವುದನ್ನು ನೋಡಿದಾಗ ಅದನ್ನು ದೇವರು ಎನ್ನುವಿರಿ. ಒಳಗಿನಿಂದ ನೋಡಿದಾಗ ಅದೇ ನೀವಾಗಿರುವಿರಿ. ಅದೆಲ್ಲ ಒಂದೇ, ನಿಮಗಿಂತ ಬೇರೆ ದೇವರಿಲ್ಲ. ನಿಜವಾಗಿ ನಿಮಗಿಂತ ಮೇಲಾದ ದೇವರಿಲ್ಲ. ಇತರ ದೇವತೆಗಳೆಲ್ಲ ನಿಮ್ಮ ಪಾಲಿಗೆ ಅಲ್ಪ. ಸ್ವರ್ಗದಲ್ಲಿರುವ ತಂದೆ ಮುಂತಾದ ದೇವರ ಭಾವನೆಗಳೆಲ್ಲ ನಿಮ್ಮ ಪ್ರತಿಬಿಂಬಗಳಷ್ಟೆ. ನಿಮ್ಮ ವಿಗ್ರಹವೇ ದೇವರು. “ದೇವರು ತನ್ನಂತೆ ಮಾನವನನ್ನು ಸೃಷ್ಟಿಸಿದ'' ಎನ್ನುವುದು ತಪ್ಪು. ಮಾನವ ತನ್ನಂತೆ ದೇವರನ್ನು ಸೃಷ್ಟಿಸಿದ. ಇದು ಸರಿ. ಪ್ರಪಂಚದಲ್ಲಿ ನಮ್ಮಂತೆಯೆ ನಾವು ದೇವರನ್ನು ಸೃಷ್ಟಿಸುತ್ತಿರುವೆವು. ನಾವು ದೇವರನ್ನು ಸೃಷ್ಟಿಸಿ ಅವನ ಪಾದಕ್ಕೆ ಬಿದ್ದು ನಮಸ್ಕರಿಸುವೆವು. ಈ ಕನಸು ಬಿದ್ದಾಗ ನಾವು ಅದನ್ನು ಪ್ರೀತಿಸುವೆವು!

ನಾವು ತಿಳಿದುಕೊಳ್ಳಬೇಕಾದ ಮುಖ್ಯ ವಿಷಯ ಇದು: ಈ ಉಪನ್ಯಾಸದ\break ಸಾರಾಂಶವೆ, ಸಾರವೆ, ಇರುವುದು ಒಂದೇ ವಸ್ತು-ಎಂಬುದು. ನಾವು ಅದನ್ನು ಬೇರೆ ಬೇರೆ ದೃಷ್ಟಿಗಳಿಂದ ನೋಡಿದಾಗ ಅದು ಸ್ವರ್ಗ ಪೃಥ್ವಿ ನರಕಗಳಂತೆಯೋ; ದೇವತೆ ದೆವ್ವ ರಾಕ್ಷಸ ಮನುಷ್ಯರಂತೆಯೋ; ಅಥವಾ ಇವುಗಳೆಲ್ಲವನ್ನೂ ಒಳಗೊಂಡಿರುವಂತೆ ಕಾಣುತ್ತದೆ. “ಯಾರು ಅನಿತ್ಯದಲ್ಲಿ ನಿತ್ಯವನ್ನು ನೋಡುತ್ತಾರೆಯೋ, ಅಚೇತನದಲ್ಲಿ ಚೇತನವನ್ನು ನೋಡುತ್ತಾರೆಯೋ, ಅವಿಕಾರಿಯಾದ ಅವನನ್ನು ಕಾಣುವರೋ, ಅವರಿಗೆ ಮಾತ್ರ ಶಾಶ್ವತ ಶಾಂತಿ; ಇತರರಿಗೆ ಅಲ್ಲ.” ಈ ಏಕವಾದ ಅಸ್ತಿತ್ವವನ್ನು ನಾವೀಗ ಸಾಕ್ಷಾತ್ಕಾರ ಮಾಡಿಕೊಳ್ಳಬೇಕಾಗಿದೆ. ಅದನ್ನು ಹೇಗೆ ಸಾಕ್ಷಾತ್ಕಾರ ಮಾಡಿಕೊಳ್ಳಬೇಕು ಎಂಬುದೇ ಅನಂತರದ ಪ್ರಶ್ನೆ. ಈ ಕನಸನ್ನು ನಾವು ಒಡೆಯುವುದು ಹೇಗೆ? ನಾವು ಕೆಲಸಕ್ಕೆ ಬಾರದ ಸ್ತ್ರೀ ಪುರುಷರು ಎಂಬ ಕನಸಿನಿಂದ ಜಾಗೃತರಾಗುವುದು ಹೇಗೆ? ವಿಶ್ವವ್ಯಾಪಿಯಾದ ಆತ್ಮವಾಗಿರುವ ನಾವು ಕೆಲಸಕ್ಕೆ ಬಾರದ ಅಲ್ಪರಾಗಿರುವ ಸ್ತ್ರೀಪುರುಷರಾಗಿರುವೆವು. ಒಬ್ಬನ ಒಳ್ಳೆಯ ಅಥವಾ ಕೆಟ್ಟ ಮಾತನ್ನು ಆಶ್ರಯಿಸುತ್ತ ಅಂಜಿಕೊಂಡಿರುವೆವು. ಎಂತಹ ಭಯಂಕರವಾದ ಅನ್ಯಾಶ್ರಯ ಇದು! ಎಂತಹ ಘೋರವಾದ ಗುಲಾಮಗಿರಿ ಇದು! ಸುಖದುಃಖಗಳಿಗೆ ಅತೀತನು ನಾನು, ನನ್ನ ಒಂದು ಛಾಯೆಯೇ ಈ ಬ್ರಹ್ಮಾಂಡ, ನನ್ನ ಅಸ್ತಿತ್ವದ ಕೆಲವು ಬಿಂದುಗಳು ಈ ಸೂರ್ಯ ಚಂದ್ರ ತಾರೆಗಳು. ಇಂತಹವನಾದ ನಾನು ಗುಲಾಮಗಿರಿಯ ಭಯಾನಕ ಸ್ಥಿತಿಗೆ ಇಳಿದಿರುವೆನು. ನೀವು ನನ್ನನ್ನು ಜಿಗುಟಿದರೆ ಸಾಕು, ನನಗೆ ನೋವಾಗುವುದು. ಯಾರಾದರೂ ಸ್ವಲ್ಪ ಒಳ್ಳೆಯ ಮಾತನ್ನಾಡಿದರೆ ನಾನು ಸಂತೋಷಪಡುವೆನು. ನನ್ನ ಅವಸ್ಥೆಯನ್ನು ನೋಡಿ; ದೇಹದ ದಾಸ, ಮನಸ್ಸಿನ ದಾಸ, ಪ್ರಪಂಚದ ದಾಸ, ಒಳ್ಳೆಯ ಮತ್ತು ಕೆಟ್ಟ ಮನಸ್ಸಿನ ದಾಸ, ಕಾಮಕ್ಕೆ ದಾಸ, ಸುಖಕ್ಕೆ ದಾಸ, ಜೀವನಕ್ಕೆ ದಾಸ, ಮರಣಕ್ಕೆ ದಾಸ, ಎಲ್ಲಕ್ಕೂ ದಾಸನಾಗಿರುವೆನು. ಈ ಗುಲಾಮಗಿರಿಯಿಂದ ಪಾರಾಗಬೇಕಾಗಿದೆ. ಇದು ಹೇಗೆ? “ಮೊದಲು ಈ ಆತ್ಮನ ವಿಚಾರವನ್ನು ಕೇಳಬೇಕಾಗಿದೆ. ಅನಂತರ ಮನನ ಮಾಡಬೇಕಾಗಿದೆ. ಅನಂತರ ಧ್ಯಾನ ಮಾಡಬೇಕಾಗಿದೆ.'' ಅದ್ವೈತಜ್ಞಾನಿಯ ಮಾರ್ಗ ಇದು. ಸತ್ಯವನ್ನು ಮೊದಲು ಕೇಳಬೇಕು, ಅನಂತರ ಅದನ್ನು ಕುರಿತು ಯೋಚಿಸಬೇಕು, ಅದನ್ನು ಪುನಃ ಒತ್ತೊತ್ತಿ ಹೇಳಬೇಕಾಗಿದೆ. ನಾನೇ ಬ್ರಹ್ಮ ಎಂದು ಅನುಗಾಲವೂ ಯೋಚಿಸಿ, ಇತರ ಯೋಚನೆಗಳನ್ನೆಲ್ಲ ದೌರ್ಬಲ್ಯವೆಂದು ಆಚೆಗೆ ಕಿತ್ತೊಗೆಯಬೇಕು. ನೀವು ಸ್ತ್ರೀಪುರುಷರೆಂದು ಹೇಳುವ ಭಾವನೆಗಳನ್ನೆಲ್ಲವನ್ನು ಕಿತ್ತೊಗೆಯಿರಿ. ದೇಹ ಹೋಗಲಿ, ಮನಸ್ಸು ಹೋಗಲಿ, ದೇವರು ಹೋಗಲಿ, ದೆವ್ವಗಳು ಹೋಗಲಿ. ಆ ಏಕ ವಿನಃ ಉಳಿದವೆಲ್ಲ ಹೋಗಲಿ, “ಒಬ್ಬ ಎಲ್ಲಿ ಮತ್ತೊಂದನ್ನು ಕೇಳುವನೊ, ಮತ್ತೊಂದನ್ನು ನೋಡುವನೋ ಅದು ಅಲ್ಪ. ಎಲ್ಲಿ ಒಬ್ಬ ಮತ್ತೊಂದನ್ನು ಕೇಳುವುದಿಲ್ಲವೊ, ಮತ್ತೊಂದನ್ನು ನೋಡಿವುದಿಲ್ಲವೋ ಅದು ಭೂಮ.” ಅದೇ ಪರಾ. ಅಲ್ಲಿ ಜ್ಞಾತೃಜ್ಞೇಯಗಳೆರಡೂ ಒಂದಾಗುವುವು. ನಾನೇ ಕೇಳುವವನೂ ಮಾತನಾಡುವವನೂ ಆದಾಗ, ನಾನೇ ಗುರು ಮತ್ತು ಶಿಷ್ಯನಾದಾಗ, ನಾನೇ ಸೃಷ್ಟಿಕರ್ತ ಮತ್ತು ಸೃಷ್ಟಿ ಎರಡೂ ಆದಾಗ ಮಾತ್ರ ಅಂಜಿಕೆ ಮಾಯವಾಗುವುದು. ನಮ್ಮನ್ನು ಅಂಜಿಸುವುದಕ್ಕೆ ಬೇರೆ ಯಾವುದೂ ಇಲ್ಲ. ನಾನಲ್ಲದೆ ಬೇರೆ ಇಲ್ಲ. ಯಾವುದು ನನ್ನನ್ನು ಅಂಜಿಸಬಲ್ಲದು? ಇದನ್ನು ದಿನ ದಿನವೂ ಕೇಳುತ್ತಿರಬೇಕು. ಉಳಿದೆಲ್ಲ ಭಾವನೆಗಳಿಂದ ಪಾರಾಗಿ, ಎಲ್ಲವನ್ನೂ ಕಿತ್ತೊಗೆದು ಈ ಭಾವನೆ ಹೃದಯಕ್ಕೆ ನೇರವಾಗಿ ತೂರಿಹೋಗುವವರೆಗೆ ಕೇಳಬೇಕು. ಸೋಽಹಂ ಸೋಽಹಂ ಎಂಬ ಭಾವನೆಯಿಂದ ನಮ್ಮ ದೇಹದ ಪ್ರತಿಯೊಂದು ನರ, ಪ್ರತಿಯೊಂದು ಮಾಂಸಖಂಡವೂ ಮತ್ತು ಪ್ರತಿಯೊಂದು ಬಿಂದು ರಕ್ತವೂ ಅನುರಣಿತವಾಗುವ ತನಕ ಅದನ್ನು ಕುರಿತು ಚಿಂತಿಸುತ್ತಿರಬೇಕು. ಮೃತ್ಯುದ್ವಾರದಲ್ಲಿರುವಾಗಲೂ ಸೋಽಹಂ ಎನ್ನಿ, ಭರತಖಂಡದಲ್ಲಿ ಒಬ್ಬ ಸಂನ್ಯಾಸಿ ಇದ್ದ. ಅವನು ಯಾವಾಗಲೂ `ಶಿವೋಹಂ' ಎಂದು ಉಚ್ಚರಿಸುತ್ತಿದ್ದನು. ಒಂದು ದಿನ ವ್ಯಾಘ್ರ ಬಂದು ಅವನನ್ನು ಎಳೆದುಕೊಂಡು ಹೋಗಿ ಕೊಂದಿತು. ಆದರೆ ಅವನು ಬದುಕಿರುವ ಪರಿಯಂತ ಶಿವೋಹಂ ಎಂದು ಉಚ್ಚರಿಸುತ್ತಿದ್ದನು. ಮೃತ್ಯುದ್ವಾರದಲ್ಲಿ, ಮಹಾ ಅಪಾಯದಲ್ಲಿ, ಸಮರಾಂಗಣದ ಗಡಿಬಿಡಿಯಲ್ಲಿ, ಸಮುದ್ರದ ಆಳದಲ್ಲಿ, ಪರ್ವತದ ತುದಿಯಲ್ಲಿ, ಘೋರಾರಣ್ಯದಲ್ಲಿ ಇರುವಾಗ `ನಾನೇ ಅವನು, ನಾನೇ ಅವನು' ಎಂದು ಹೇಳಿ. ಹಗಲು ರಾತ್ರಿ ನಾನೇ ಅವನು, ನಾನೇ ಅವನು ಎಂದು ಹೇಳಿ. ಇದೇ ಪರಮಶಕ್ತಿ, ಇದೇ ಧರ್ಮ. “ಬಲಹೀನರಿಗೆ ಆತ್ಮ ಲಭಿಸದು.” “ದೇವರೇ, ನಾನು ಪಾಪಿ'' ಎಂದು ಎಂದಿಗೂ ಹೇಳಬೇಡಿ. ಯಾರು ನಿಮಗೆ ಸಹಾಯಮಾಡಬಲ್ಲರು? ನೀವೇ ವಿಶ್ವಕ್ಕೆ ಸಹಾಯಮಾಡುವವರು. ಈ ಪ್ರಪಂಚದಲ್ಲಿ ಯಾವುದು ನಿಮಗೆ ಸಹಾಯ ಮಾಡಬಲ್ಲದು? ನಿಮಗೆ ಸಹಾಯಮಾಡಬಲ್ಲ ದೇವನಾರು, ದಾನವನಾರು, ಮಾನವನಾರು? ಯಾರು ನಿಮ್ಮನ್ನು ಬಲಾತ್ಕರಿಸಬಲ್ಲರು? ನೀವೇ ವಿಶ್ವೇಶ್ವರ, ನೀವು ಸಹಾಯವನ್ನು ಹುಡುಕುವುದು ಎಲ್ಲಿ? ಎಂದಿಗೂ ಸಹಾಯ ನಿಮ್ಮಿಂದ ಅಲ್ಲದೆ ಬೇರೆ ಎಲ್ಲಿಂದಲೂ ಬರಲಿಲ್ಲ. ನೀವು ಅಜ್ಞಾನದಲ್ಲಿರುವಾಗ ಮಾಡಿದ ಪ್ರತಿಯೊಂದು ಪ್ರಾರ್ಥನೆಯೂ ಈಡೇರಿತು ಎಂದು ಭಾವಿಸಿದಾಗ, ಅದನ್ನು ಯಾರೋ ಹೊರಗಿರುವ ದೇವರು ಈಡೇರಿಸಿದನು ಎಂದು ಭಾವಿಸದಿರಿ. ಆದರೆ ನೀವೇ ಅರಿವಿಲ್ಲದೆ ಆ ಪ್ರಾರ್ಥನೆ ಈಡೇರುವಂತೆ ಮಾಡಿದಿರಿ, ಸಹಾಯ ನಿಮ್ಮಿಂದಲೇ ಬಂದಿತು. ಯಾರೋ ಹೊರಗಿನವರು ನಿಮಗೆ ಸಹಾಯ ಕಳುಹಿಸಿದರೆಂದು ನೀವು ಭ್ರಾಂತರಾಗಿದ್ದೀರಿ. ನಿಮ್ಮಿಂದ ಹೊರಗೆ ಯಾವ ಸಹಾಯವೂ ಇಲ, ನೀವೇ ವಿಶ್ವದ ಸೃಷ್ಟಿಕರ್ತರು. ರೇಷ್ಮೆ ಯ ಹುಳುವಿನಂತೆ ನೀವೇ ಸುತ್ತಲೂ ಗೂಡನ್ನು ಕಟ್ಟಿಕೊಂಡಿರುವಿರಿ. ಯಾರು ನಿಮ್ಮನ್ನು ರಕ್ಷಿಸುತ್ತಾರೆ? ನಿಮ್ಮ ಸುತ್ತಲೂ ಇರುವ ಗೂಡನ್ನು ಒಡೆದು ಸುಂದರವಾದ ಚಿಟ್ಟೆಯಂತೆ, ಮುಕ್ತಜೀವಿಯಂತೆ ಹೊರಗೆ ಬನ್ನಿ. ಆಗ ಮಾತ್ರ ನೀವು ಸತ್ಯವನ್ನು ನೋಡುತ್ತೀರಿ. ಸದಾ “ನಾನೇ ಅವನು” ಎಂದು ಸ್ಮರಿಸಿಕೊಳ್ಳಿ. ಈ ಪದಗಳು ಮನಸ್ಸಿನ ಕಲ್ಮಷವನ್ನು ದಹಿಸುವುವು. ಆಗಲೇ ನಿಮ್ಮಲ್ಲಿರುವ ಅದ್ಭುತ ಶಕ್ತಿಯನ್ನು ಅವು ಜಾಗೃತಗೊಳಿಸುವುವು. ನಾವು ಅನವರತ ಸತ್ಯವನ್ನು ಕೇಳಿ ಇದನ್ನು ವ್ಯಕ್ತಗೊಳಿಸಬೇಕು. ಬೇರೆ ದಾರಿಯೇ ಇಲ್ಲ. ದುರ್ಬಲವಾದ ಭಾವನೆಗಳಿರುವ ಕಡೆ ನೀವು ಸುಳಿಯಬೇಡಿ. ನೀವು ಜ್ಞಾನಿಯಾಗಬೇಕಾದರೆ ಎಲ್ಲಾ ದುರ್ಬಲತೆಯನ್ನು ಧಿಕ್ಕರಿಸಿ.

ನೀವು ಅಭ್ಯಾಸಮಾಡುವುದಕ್ಕೆ ಮುಂಚೆ ಮನಸ್ಸಿನಲ್ಲಿರುವ ಸಂಶಯಗಳನ್ನೆಲ್ಲಾ ದೂರಮಾಡಿ. ವಾದಮಾಡಿ, ತರ್ಕಿಸಿ, ಆಲೋಚನೆಮಾಡಿ. ಇದೊಂದೇ ಸತ್ಯ, ಉಳಿದುದೆಲ್ಲ ಮಿಥ್ಯ ಎಂಬ ಭಾವನೆಯಲ್ಲಿ ಪ್ರತಿಷ್ಠಿತರಾದ ಮೇಲೆ ಇನ್ನು ವಾದ ಮಾಡಬೇಡಿ; ಶಾಂತರಾಗಿ, ವಾದ ಮಾಡುವುದನ್ನು ಕೇಳಬೇಡಿ, ವಾದಮಾಡಲು ಹೋಗಬೇಡಿ. ಇನ್ನು ಮೇಲೆ ವಾದದಿಂದ ಏನು ಪ್ರಯೋಜನ? ನಿಮಗೆ ತೃಪ್ತಿಯಾಗಿದೆ. ನೀವು ಇನ್ನು ಏನು ಮಾಡಬೇಕೆಂದು ನಿರ್ಧರಿಸಿರುವಿರಿ? ಇನ್ನು ಉಳಿದಿರುವುದೇನು? ಈಗ ಸತ್ಯವನ್ನು ಸಾಕ್ಷಾತ್ಕಾರಮಾಡಿಕೊಳ್ಳಬೇಕಾಗಿದೆ. ಆದಕಾರಣ ಪ್ರಶಸ್ತವಾದ ಕಾಲವನ್ನು ಏತಕ್ಕೆ ವಾದದಲ್ಲಿ ಕಳೆಯುತ್ತೀರಿ? ಈಗ ಸತ್ಯವನ್ನು ಕುರಿತು ಧ್ಯಾನಿಸಬೇಕಾಗಿದೆ. ನಿಮಗೆ ಸಹಾಯಮಾಡುವ ಬಲವರ್ಧಕ ಭಾವನೆಗಳನ್ನು ಸ್ವೀಕರಿಸಬೇಕು, ದುರ್ಬಲಗೊಳಿಸುವ ಭಾವನೆಗಳನ್ನೆಲ್ಲ ತಿರಸ್ಕರಿಸಬೇಕು. ಭಕ್ತನು ದೇವರ ವಿಗ್ರಹ ಮುಂತಾದುವುಗಳ ಮೇಲೆಲ್ಲ ಧ್ಯಾನಿಸುತ್ತಾನೆ. ಇದೇ ಸ್ವಾಭಾವಿಕವಾದ ಮಾರ್ಗ. ಆದರೆ ಬಹಳ ನಿಧಾನವಾದುದು. ರಾಜಯೋಗಿ ತನ್ನ ದೇಹದಲ್ಲಿರುವ ಹಲವು ಚಕ್ರಗಳ ಮೇಲೆ ಧ್ಯಾನಿಸಿ ಶಕ್ತಿಯನ್ನು ಮನಸ್ಸಿನ ಮೂಲಕ ಪ್ರಯೋಗಿಸುತ್ತಾನೆ. ಜ್ಞಾನಿ ದೇಹವೂ ಇಲ್ಲ, ಮನಸ್ಸೂ ಇಲ್ಲ ಎನ್ನುತ್ತಾನೆ. ದೇಹದ ಮತ್ತು ಮನಸ್ಸಿನ ಭಾವನೆಗಳು ತೊಲಗಬೇಕು, ಅವನ್ನು ಹೊಡೆದಟ್ಟಬೇಕು. ಆದಕಾರಣ ಅವನ್ನು ಕುರಿತು ಚಿಂತಿಸುವುದು ನಿಷ್ಪ್ರಯೋಜಕ. ಒಂದು ರೋಗವನ್ನು ಗುಣಮಾಡಲು ಮತ್ತೊಂದು ರೋಗವನ್ನು ತಂದಂತೆ ಇದು. ಜ್ಞಾನಿಯ ಧ್ಯಾನ ಬಹಳ ಕಷ್ಟ. ಏಕೆಂದರೆ ಅದು ಅಭಾವ ಸೂಚಕವಾದುದು. ಅವನು ಎಲ್ಲವನ್ನೂ ಅಲ್ಲಗಳೆಯುತ್ತಾನೆ. ಕೊನೆಗೆ ಯಾವುದು ಉಳಿಯುವುದೋ ಅದೇ ಆತ್ಮ. ಇದೇ ಶ್ರೇಷ್ಠ ವಿಶ್ಲೇಷಣಾತ್ಮಕ ಮಾರ್ಗ. ಜ್ಞಾನಿಯಾದವನು ಕೇವಲ ವಿಶ್ಲೇಷಣಾ ಶಕ್ತಿಯಿಂದಲೇ ಜಗತ್ತನ್ನು ಆತ್ಮನಿಂದ ಕಿತ್ತೊಗೆಯಲು ಯತ್ನಿಸುವನು. ನಾನು ಜ್ಞಾನಿ ಎಂದು ಹೇಳಿಕೊಳ್ಳುವುದು ಸುಲಭ. ಆದರೆ ಹಾಗೆ ಆಗಬೇಕಾದರೆ ಬಹಳ ಕಷ್ಟ. `ದಾರಿ ಬಹುದೂರವಿದೆ. ಹಿರಿದ ಕತ್ತಿಯ ಅಲಗಿನ ಮೇಲೆ ನಡೆದಂತೆ ಇದು. ಆದರೂ ನಿರಾಶನಾಗಬೇಡ, ಜಾಗೃತನಾಗು, ಏಳು, ಗುರಿ ಸೇರುವವರೆಗೆ ನಿಲ್ಲಬೇಡ' ಎನ್ನುವುದು ವೇದ.

ಹಾಗಾದರೆ ಜ್ಞಾನಿಯ ಧ್ಯಾನವೆಂತಹುದು? ಅವನು ದೇಹದ ಮತ್ತು ಮನಸ್ಸಿನ ಭಾವನೆಗಳಿಗೆ ಅತೀತನಾಗಲು ಯತ್ನಿಸುವನು. ತಾನೊಂದು ದೇಹ ಎಂಬುದನ್ನು ಆಮೂಲಾಗ್ರವಾಗಿ ಕಿತ್ತೊಗೆಯಲು ಯತ್ನಿಸುವನು. “ನಾನು ಒಬ್ಬ ಸ್ವಾಮಿ" ಎಂದು ಕೇಳಿದೊಡನೆಯೆ ದೇಹದ ಭಾವನೆ ಬರುವುದು. ಆಗ ನಾನು ಏನು ಮಾಡಬೇಕು? ಆಗ ನಾನು ಮನಸ್ಸಿಗೆ ಒಂದು ಬಲವಾದ ಪೆಟ್ಟನ್ನು ಕೊಡಬೇಕು? “ಇಲ್ಲ, ನಾನು ದೇಹವಲ್ಲ, ಆತ್ಮ'' ಎನ್ನಬೇಕು. ಯಾವ ರೋಗ ಬಂದರೆ ನನಗೇನು? ಮೃತ್ಯು ಅತಿ ಭಯಂಕರವಾದ ಆಕಾರವನ್ನು ಧರಿಸಿ ಬಂದರೇನು? ನಾನು ದೇಹವಲ್ಲ, ದೇಹವನ್ನು ಏತಕ್ಕೆ ಚೆನ್ನಾಗಿ ಇಡಬೇಕು? ಈ ಮಿಥ್ಯಾ ಪ್ರಪಂಚವನ್ನು ಮತ್ತೊಮ್ಮೆ ಅನುಭವಿಸುವುದಕ್ಕೇನು? ಈ ಗುಲಾಮಗಿರಿಯನ್ನು ಮುಂದುವರಿಸುವುದಕ್ಕೇನು? ಇದು ಹೋಗಲಿ. ನಾನು ದೇಹವಲ್ಲ. ಜ್ಞಾನಿಯ ಮಾರ್ಗ ಇದು. “ಈ ಭವಸಾಗರವನ್ನು ದಾಟುವುದಕ್ಕೆ ದೇವರು ನನಗೊಂದು ದೇಹವನ್ನು ಕೊಟ್ಟಿರುವನು. ಗುರಿ ಸೇರುವವರೆಗೆ ಇದನ್ನು ನೋಡಿಕೊಳ್ಳಬೇಕು" ಎನ್ನುವನು ಭಕ್ತ. “ನಾನು ದೇಹವನ್ನು ಚೆನ್ನಾಗಿ ನೋಡಿಕೊಳ್ಳಬೇಕು, ಏಕೆಂದರೆ, ನಾನು ಎಡೆಬಿಡದೆ ಸಾಧನೆಮಾಡಿ ಮುಕ್ತಿಯನ್ನು ಗಳಿಸಬೇಕಾಗಿದೆ" ಎನ್ನುವನು ಯೋಗಿ. ಆದರೆ ಜ್ಞಾನಿ ತಾಳಲಾರ; ಈ ಕ್ಷಣ ಅವನು ಗುರಿಯನ್ನು ಸೇರಬೇಕೆಂದು ಇಚ್ಛಿಸುವನು. “ನಾನು ನಿತ್ಯಮುಕ್ತ, ಎಂದೂ ಬದ್ದನಲ್ಲ, ಎಂದೆಂದಿಗೂ ನಾನೇ ವಿಶ್ವೇಶ್ವರ. ಯಾರು ನನ್ನನ್ನು ಪೂರ್ಣನನ್ನಾಗಿ ಮಾಡಬಲ್ಲರು? ನಾನಾಗಲೇ ಪೂರ್ಣಾತ್ಮ" ಎನ್ನುವನು ಜ್ಞಾನಿ. ಮನುಷ್ಯ ಪೂರ್ಣಾನಾಗಿದ್ದರೆ ಇತರರಲ್ಲಿ ಪೂರ್ಣತೆಯನ್ನು ಕಾಣುತ್ತಾನೆ. ಅವನು ಇತರರಲ್ಲಿ ಅಪೂರ್ಣತೆಯನ್ನು ನೋಡಿದರೆ ಅದು ಕೇವಲ ತಾನೆ ಆರೋಪಮಾಡಿರುವುದು. ತನ್ನಲ್ಲಿ ಅಪೂರ್ಣತೆ ಇಲ್ಲದಿದ್ದರೆ ಇತರರಲ್ಲಿ ಅದನ್ನು ಹೇಗೆ ನೋಡಬಲ್ಲ? ಆದಕಾರಣ ಜ್ಞಾನಿ ಪೂರ್ಣ. ಅವನು ಅಪೂರ್ಣತೆಯನ್ನು ಲೆಕ್ಕಿಸುವುದೇ ಇಲ್ಲ. ಅವನಿಗೆ ಯಾವುದೂ ಇಲ್ಲವೇ ಇಲ್ಲ. ಅವನು ಮುಕ್ತನಾದೊಡನೆ ಒಳ್ಳೆಯದನ್ನು ಮತ್ತು ಕೆಟ್ಟದನ್ನು ನೋಡುವುದೇ ಇಲ್ಲ. ಯಾರು ಒಳ್ಳೆಯದನ್ನು ಕೆಟ್ಟದನ್ನು ನೋಡುವರು? ಯಾರಲ್ಲಿ ಅವು ಇವೆಯೋ ಅವನು ಅವನ್ನು ಹೊರಗೆ ನೋಡುವನು. ದೇಹವನ್ನು ಯಾರು ನೋಡುವರು? ಯಾರು ತಾವು ದೇಹವೆಂದು ಭಾವಿಸುವರೋ ಅವರು. ನಿಮ್ಮ ದೇಹವೆಂಬ ಭ್ರಾಂತಿ ಹೋದೊಡನೆ ಪ್ರಪಂಚವೇ ನಿಮಗೆ ಕಾಣುವುದಿಲ್ಲ. ಅದು ಎಂದೆಂದಿಗೂ ಮಾಯವಾಗುವುದು. ಜ್ಞಾನಿಯಾದವನು ಯುಕ್ತಿಯ ದೃಢನಿಶ್ಚಯದಿಂದ ದೇಹಭಾವನೆಯ ದಾಸ್ಯದಿಂದ ಪಾರಾಗಲು ಯತ್ನಿಸುವನು. ಇದೇ “ನೇತಿ, ನೇತಿ'' “ಇದಲ್ಲ, ಇದಲ್ಲ'' ಎಂಬ ನಿಷೇಧಾತ್ಮಕವಾದ ಮಾರ್ಗ.

