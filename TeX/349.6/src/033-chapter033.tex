
\chapter{ವೇದಗಳ ಧಾರ್ಮಿಕ ಆದರ್ಶಗಳು\protect\footnote{\enginline{* C.W, Vol. I, P. 417}}}

ನಮಗೆ ಅತಿ ಮುಖ್ಯವಾಗಿ ಬೇಕಾಗಿರುವುದೆ ಆತ್ಮ, ದೇವರು ಮತ್ತು ಧರ್ಮಕ್ಕೆ ಸಂಬಂಧಪಟ್ಟ ಭಾವನೆಗಳು. ನಾವು ಈಗ ಸಂಹಿತೆಗಳನ್ನು ತೆಗೆದುಕೊಳ್ಳೋಣ. ಇವೆಲ್ಲ ಒಂದು ಮಂತ್ರಗಳ ಸಂಗ್ರಹ. ಇವೇ ಆರ್ಯರ ಅತಿ ಪುರಾತನ ಸಾಹಿತ್ಯ ಅಥವಾ ಸರಿಯಾಗಿ ಹೇಳಬೇಕಾದರೆ ಜಗತ್ತಿನ ಪ್ರಾಚೀನತಮ ಸಾಹಿತ್ಯ. ಬೇರೆ ಕಡೆಗಳಲ್ಲಿ ಇವಕ್ಕಿಂತ ಪುರಾತನವಾದ ಕೆಲವು ಬರಹಗಳು ಸಿಕ್ಕಬಹುದು. ಆದರೆ ಅವು ಸಾಹಿತ್ಯವಲ್ಲ. ಸಂಗ್ರಹಗೊಂಡ ಪುಸ್ತಕ ಎಂಬ ಅರ್ಥದಲ್ಲಿ ಇವು ಜಗತ್ತಿನಲ್ಲಿ ಅತ್ಯಂತ ಪ್ರಾಚೀನವಾದವು. ಇಲ್ಲಿ ಆರ್ಯರ ಅತಿ ಪುರಾತನ ಭಾವನೆಗಳು, ಅವರ ಆಸೆ-ಆಕಾಂಕ್ಷೆಗಳು, ಅವರ ಮನಸ್ಸಿನಲ್ಲಿ ಮೂಡಿದ ಸಂದೇಹಗಳು, ಪ್ರಶ್ನೆಗಳು ಇವೆ. ಪ್ರಾರಂಭದಲ್ಲೇ ಇಲ್ಲೊಂದು ವಿಚಿತ್ರವನ್ನು ನೋಡುವೆವು. ಇಲ್ಲಿ ಬರುವ ಪ್ರಾರ್ಥನೆಗಳೆಲ್ಲ ಬೇರೆ ಬೇರೆ ದೇವತೆಗಳಿಗೆ ಸಲ್ಲಿಸಿದವುಗಳು. ಇವರಲ್ಲಿ ಹಲವು ದೇವತೆಗಳಿರುವರು. ಒಬ್ಬನನ್ನು ಇಂದ್ರ ಎನ್ನುವರು, ಮತ್ತೊಬ್ಬನನ್ನು ವರುಣ ಎನ್ನುವರು; ಹೀಗೆಯೇ ಮಿತ್ರ, ಪರ್ಜನ್ಯ ಮುಂತಾದ ಹೆಸರುಗಳಿಂದ ಕರೆಯುವರು. ಇವರಿಗೆ ಸಂಬಂಧಪಟ್ಟ ಹಲವು ಪೌರಾಣಿಕ ಮತ್ತು ರೂಪಕ ಕಥೆಗಳು ಒಂದಾದ ಮೇಲೊಂದು ಬರುವುವು. ಗುಡಿಗಿನಂತೆ ಗರ್ಜಿಸುವ ಇಂದ್ರ ಮಳೆಯನ್ನು ತಡೆದ ಸರ್ಪವನ್ನು ತನ್ನ ವಜ್ರಾಯುಧದಿಂದ ಹೊಡೆಯುವನು. ಆಗ ಸರ್ಪವು ನಾಶವಾಗಿ ವರ್ಷಾಕಾಲದಂತೆ ಮಳೆ ಬೀಳುವುದು. ಜನರಿಗೆ ತೃಪ್ತಿಯಾಗಿ, ಇಂದ್ರನಿಗೆ ಹವಿಸ್ಸನ್ನು ಅರ್ಪಿಸಿ ಪೂಜಿಸುವರು. ಒಂದು ಅಗ್ನಿಯನ್ನು ಹೊತ್ತಿಸಿ ಅದಕ್ಕೆ ಹಲವು ಪ್ರಾಣಿಗಳನ್ನು ಬಲಿಕೊಟ್ಟು ಅವುಗಳ ಮಾಂಸವನ್ನು ಅದರಲ್ಲಿ ಸುಟ್ಟು ಇಂದ್ರನಿಗೆ ಅದನ್ನು ಅರ್ಪಿಸುವರು. ಆಗಿನ ಕಾಲದಲ್ಲಿ ಸೋಮ ಎಂಬ ಬಳ್ಳಿಯಿತ್ತು. ಅದು ಎಂತಹ ಬಳ್ಳಿಯೆಂದು ಈಗ ಯಾರಿಗೂ ಗೊತ್ತಿಲ್ಲ. ಅದು ಈಗ ಸಂಪೂರ್ಣ ಮಾಯವಾಗಿದೆ. ಅದಕ್ಕೆ ಸಂಬಂಧಪಟ್ಟ ವಿಷಯಗಳನ್ನು ಪುಸ್ತಕಗಳಿಂದ ಗ್ರಹಿಸಬೇಕು. ಈ ಬಳ್ಳಿಯನ್ನು ಜಜ್ಜಿದಾಗ ಒಂದು ವಿಧವಾದ ಹಾಲಿನಂತಹ ರಸ ಬರುತ್ತಿತ್ತು. ಅದನ್ನು ಕೆಲವು ಕಾಲವಿಟ್ಟು ಹುಳಿ ಬರಿಸಿ ತೆಗೆದುಕೊಳ್ಳುತ್ತಿದ್ದರು. ಅದನ್ನು ಅತಿಯಾಗಿ ಸೇವಿಸಿದರೆ ಅಮಲು ಬರುತ್ತಿತ್ತು. ಅದನ್ನು ಇಂದ್ರ ಮುಂತಾದ ದೇವರುಗಳಿಗೆ ಕೊಟ್ಟು ತಾವೂ ಸೇವಿಸುತ್ತಿದ್ದರು. ಕೆಲವು ವೇಳೆ ತಾವು ಅದನ್ನು ಹೆಚ್ಚು ಸೇವಿಸಿ ದೇವರಿಗೂ ಹೆಚ್ಚು ಕುಡಿಸುತ್ತಿದ್ದರು. ಕೆಲವು ವೇಳೆ ಇಂದ್ರ ವಿಪರೀತ ಕುಡಿಯುತ್ತಿದ್ದನು. ಒಂದು ಸಲ ಇಂದ್ರ ವಿಪರೀತ ಕುಡಿದು ಬಾಯಿಗೆ ಬಂದಂತೆ ಎಚ್ಚರ ತಪ್ಪಿ ಮಾತನಾಡಿದನು ಎಂಬುದನ್ನು ಕೂಡ ನಾವು ಗ್ರಂಥದಲ್ಲಿ ಓದುವೆವು. ವರುಣ ಕೂಡ ಪ್ರಬಲನಾದ ದೇವರು. ಅವನು ತನ್ನ ಭಕ್ತರನ್ನು ರಕ್ಷಿಸುತ್ತಿದ್ದ. ಭಕ್ತರು ಹೋಮವನ್ನು ವರುಣನಿಗೆ ಅರ್ಪಿಸಿ ಪ್ರಾರ್ಥಿಸುತ್ತಿದ್ದರು. ಇದರಂತೆಯೇ ಯುದ್ಧ ಮುಂತಾದುವುಗಳ ದೇವತೆಗಳು. ಸಂಹಿತೆಯಲ್ಲಿ ಬರುವ ಪುರಾಣಕ್ಕೂ ಇತರ ಕಡೆಗಳಲ್ಲಿ ಬರುವ ಪುರಾಣಕ್ಕೂ ದೊಡ್ಡದೊಂದು ವ್ಯತ್ಯಾಸವಿದೆ. ಇಲ್ಲಿ ಪ್ರತಿಯೊಬ್ಬ ದೇವತೆಯೂ ಅನಂತ ಎಂಬ ಭಾವನೆ ಇದೆ. ಕೆಲವು ವೇಳೆ ಈ ಗುಣ ಒಂದು ಅಮೂರ್ತಭಾವನೆಯಾಗಿರುತ್ತಿತ್ತು - ಅದನ್ನು ಆದಿತ್ಯ ಎಂದು ಕರೆಯುತ್ತಿದ್ದರು. ಕೆಲವು ವೇಳೆ ಇದೇ ಗುಣವಾಚಕವನ್ನು ಇತರ ದೇವತೆಗಳಿಗೆಲ್ಲರಿಗೂ ಸೇರಿಸುವರು. ಉದಾಹರಣೆಗೆ ಇಂದ್ರನನ್ನು ತೆಗೆದುಕೊಳ್ಳಿ. ಕೆಲವು ಗ್ರಂಥಗಳಲ್ಲಿ ಇಂದ್ರನಿಗೆ ಒಂದು ದೇಹವಿದೆ ಎಂದು ಹೇಳುವರು. ಅವನು ಬಹಳ ಬಲಾಡ್ಯನು; ಸ್ವರ್ಣ ಕವಚಗಳನ್ನು ಧರಿಸಿಕೊಂಡಿರುವನು. ಅವನು ಕೆಲವು ವೇಳೆ ಧರೆಗೆ ಇಳಿದು ಬಂದು ಭಕ್ತರೊಡನೆ ಊಟಮಾಡಿ ಬೆರೆಯುವನು, ರಾಕ್ಷಸರೊಡನೆ, ಸರ್ಪಗಳೊಡನೆ ಯುದ್ಧ ಮಾಡುವನು. ಇನ್ನೊಂದು ಮಂತ್ರದಲ್ಲಿ ಇಂದ್ರನಿಗೆ ಬಹಳ ಉಚ್ಚ ಸ್ಥಾನವನ್ನು ಕೊಟ್ಟಿರುವರು. ಅವನು ಸರ್ವಶಕ್ತ ಮತ್ತು ಸರ್ವವ್ಯಾಪಿ ಯಾಗಿರುವನು. ಅವನು ಪ್ರತಿಯೊಬ್ಬರ ಹೃದಯವನ್ನೂ ಅರಿಯಬಲ್ಲ. ಹಾಗೆಯೇ ವರುಣ ಕೂಡ. ಈ ವರುಣ ಗಾಳಿಯ ದೇವತೆ ಮತ್ತು ಜಲದೇವತೆ. ಹಿಂದೆ ಇಂದ್ರನಿಗೆ ಇವನ ಸ್ಥಾನವಿತ್ತು. ಅನಂತರ ಇದ್ದಕ್ಕೆ ಇದ್ದಂತೆ ವರುಣನನ್ನು ಸರ್ವವ್ಯಾಪಿ ಸರ್ವಶಕ್ತನ ಸ್ಥಾನಕ್ಕೆ ಏರಿಸುವರು. ವರುಣನಿಗೆ ಸಂಬಂಧಪಟ್ಟ ಅತಿ ಶ್ರೇಷ್ಠ ಭಾವನೆಯ ಒಂದು ಶ್ಲೋಕವನ್ನು ಓದುತ್ತೇನೆ. ಅದೇನು ಎಂಬುದು ನಿಮಗೇ ಗೊತ್ತಾಗುವುದು. ಅದನ್ನು ಇಂಗ್ಲೀಷಿನಲ್ಲಿ ಭಾಷಾಂತರ ಮಾಡಿರುವರು. ಅದನ್ನೇ ನಿಮಗೆ ಓದಿಹೇಳಿದರೆ ಮೇಲು:

“ಮೇಲೆ ಇರುವ ಬಲಶಾಲಿಯಾದ ದೇವರು ನಾವು ಮಾಡುವ ಕೆಲಸಗಳನ್ನೆಲ್ಲಾ, ನಮ್ಮ ಹತ್ತಿರ ಇರುವವನಂತೆ ನೋಡುತ್ತಿರುವನು. ಮಾನವರು ಮಾಡುವುದೆಲ್ಲಾ ದೇವರಿಗೆ ಗೊತ್ತಿದೆ. ಆದರೆ ಮಾನವರು ದೇವರಿಗೆ ಗೊತ್ತಿಲ್ಲವೆಂದು ಭಾವಿಸಬಹುದು. ಯಾರು ನಿಂತಿರುವರೋ, ಚಲಿಸುತ್ತಿರುವರೋ, ಸ್ಥಳದಿಂದ ಸ್ಥಳಕ್ಕೆ ಕಾಣದಂತೆ ಹೋಗುತ್ತಿರುವರೋ, ಯಾವುದನ್ನು ತಮ್ಮ ಹೃದಯಾಂತರಾಳದಲ್ಲಿ ಬಚ್ಚಿಟ್ಟುಕೊಂಡಿರುವರೋ ಅವನ್ನೆಲ್ಲ ಅವನು ನೋಡಬಲ್ಲನು. ಎಲ್ಲಿ ಇಬ್ಬರು ಕಲೆತು ಏನನ್ನೊ ಆಲೋಚಿಸುತ್ತ, ಯಾರೂ ನೋಡುತ್ತಿಲ್ಲ ಎಂದು ಭಾವಿಸುವರೋ ಅಲ್ಲಿ ವರುಣ ಮೂರನೆಯವನಾಗಿರುವನು. ಅವರು ಯೋಚಿಸುವುದೆಲ್ಲ ಅವನಿಗೆ ಗೊತ್ತಾಗುವುದು. ಈ ಭೂಮಿ ಅವನದು, ಮೇಲಿರುವ ಅನಂತ ಅಸೀಮ ಆಕಾಶವೂ ಅವನದೆ. ಎರಡು ಸಾಗರಗಳೂ ಅವನಲ್ಲಿವೆ, ಆದರೂ ಅವನು ಅಲ್ಲಿರುವ ಸಣ್ಣ ಹಳ್ಳದಲ್ಲಿರುವನು. ಗರಿಗೆದರಿ ಆಕಾಶದಲ್ಲಿ ಎಲ್ಲಿ ಹಾರಿ ಹೋದರೂ ಯಾರೂ ವರುಣನ ಪಾಶದಿಂದ ತಪ್ಪಿಸಿಕೊಂಡು ಹೋಗಲಾರರು. ಆಕಾಶದಿಂದ ಬರುವ ಅವನ ಗೂಢಚಾರರು ಚತುರ್ದಿಕ್ಕುಗಳಲ್ಲಿಯೂ ಸಂಚರಿಸುತ್ತಿರುವರು. ಅವನ ನೂರಾರು ಕಣ್ಣುಗಳು ಜಗತ್ತಿನ ಯಾವ ಮೂಲೆಯಲ್ಲಿರುವುದನ್ನೂ ಕಂಡುಹಿಡಿಯುವುವು." - ಹೀಗೆಯೆ ಇತರ ದೇವತೆಗಳ ಉದಾಹರಣೆಯನ್ನು ಕೊಡುತ್ತಾ ಹೋಗಬಹುದು. ಇವರೆಲ್ಲ ಒಬ್ಬರಾದ ಮೇಲೆ ಒಬ್ಬರು ಬರುವರು. ಎಲ್ಲರಿಗೂ ಇದೇ ಗತಿ-ಮೊದಲು ಅವರು ದೇವತೆಗಳಾಗಿರುವರು; ಅನಂತರ ಅವರು ಪ್ರಪಂಚವೆಲ್ಲವನ್ನೂ ತಮ್ಮಲ್ಲಿ ಒಳಕೊಳ್ಳುವ, ಎಲ್ಲರ ಹೃದಯವನ್ನೂ ನೋಡುವ ವಿಶ್ವೇಶ್ವರನ ಸ್ಥಿತಿಗೆ ಏರುವರು. ವರುಣನ ವಿಷಯದಲ್ಲಿ ಮತ್ತೊಂದು ಭಾವನೆ ಬರುವುದು - ಭಾವನೆಯ ಸೂಚನೆಯೊಂದು ಮೊಳೆದೋರುವುದು. ಆದರೆ ಆರ್ಯರ ಮನಸ್ಸು ಅದನ್ನು ತಕ್ಷಣ ನಿರ್ಮೂಲ ಮಾಡುವುದು. ಅದೇ ಅಂಜಿಕೆಯ ಭಾವನೆ. ಒಂದು ಕಡೆ ಅವರಿಗೆ ಅಂಜಿಕೆಯಾಗುವುದು, ತಾವು ಪಾಪಮಾಡಿದೆವೆಂದು ಭಾವಿಸಿ ವರುಣನ ಕ್ಷಮೆಯನ್ನು ಬೇಡುವರು. ಆದರೆ ಈ ಭಾವನೆ ಭಾರತ ದೇಶದಲ್ಲಿ ಬೆಳೆಯಲು ಅವಕಾಶವನ್ನು ಕೊಡಲಿಲ್ಲ. ಅದಕ್ಕೆ ಕಾರಣ ನಿಮಗೆ ಅನಂತರ ಗೊತ್ತಾಗುವುದು. ಆದರೆ ಈ ಅಂಜಿಕೆಯ ಭಾವನೆ, ಪಾಪದ ಭಾವನೆ, ಮೊಳೆಯುತ್ತಿರುವುದನ್ನು ನಾವು ನೋಡುವೆವು. ನಿಮಗೆಲ್ಲ ತಿಳಿದಂತೆ ಏಕದೇವವಾದ ಎಂಬುದೊಂದು ಇದೆ. ಬಹಳ ಪೂರ್ವದಲ್ಲಿ ಭರತಖಂಡಕ್ಕೆ ಈ ಭಾವನೆ ಬಂದಿತು. ಸಂಹಿತೆಯಲ್ಲಿ ಅತಿ ಪುರಾತನವಾದ ಪ್ರಥಮ ಭಾಗದಲ್ಲಿ ಈ ಭಾವನೆಯನ್ನು ನೋಡುತ್ತೇವೆ. ಆದರೆ ಇದರಿಂದ ಆರ್ಯರಿಗೆ ತೃಪ್ತಿಯಾಗಲಿಲ್ಲವೆಂದು ಕಾಣುವುದು. ಅವರು ಅದನ್ನೆಲ್ಲ ಬಹಳ ಹಳೆಯ ಭಾವನೆ ಎಂದು ಬಿಸುಟು, ಹಿಂದೂಗಳು ಭಾವಿಸುವಂತೆ, ಮುಂದೆ ಮುಂದೆ ಹೊರಟರು. ಯೂರೋಪಿಯನ್ನರು ವೇದಗಳ ವಿಷಯವಾಗಿ ಬರೆದ ಗ್ರಂಥಗಳನ್ನು ಮತ್ತು ವಿಮರ್ಶೆಗಳನ್ನು ಓದುವಾಗ, ಅವರು ಹಿಂದೂಗಳಲ್ಲಿ ಈ ಭಾವನೆ ಆದಿಕಾಲದಿಂದಲೂ ಇತ್ತು ಎಂದು ಹೇಳುವಾಗ ನಾವು ನಗದೆ ಇರಲಾರೆವು. ಯಾರು ತಮ್ಮ ಜನನಾರಭ್ಯದಿಂದಲೂ ಸಗುಣ ದೇವರೇ ಅತಿ ಶ್ರೇಷ್ಠ ಎಂದು ತಿಳಿದುಕೊಂಡಿರುವರೋ ಅವರಿಗೆ ನಮ್ಮ ಪುರಾತನ ಮಹರ್ಷಿಗಳಂತೆ ಆಲೋಚಿಸುವುದಕ್ಕೆ ಧೈರ್ಯವಿಲ್ಲ. ಸಂಹಿತೆಯ ಕಾಲಾನಂತರ, ಅದರಲ್ಲಿ ಬರುವ ಏಕ - ಈಶ್ವರನ ಭಾವನೆ ಪ್ರಯೋಜನವಿಲ್ಲ, ಅದು ದಾರ್ಶನಿಕರಿಗೆ ಮತ್ತು ತಾತ್ತ್ವಿಕರಿಗೆ ಯೋಗ್ಯವಲ್ಲ ಎಂದು ಆರ್ಯರು ಹೇಳುವುದನ್ನು ನೋಡಿದಾಗ, ಯುರೋಪಿಯನ್ನರಿಗೆ ಆಶ್ಚರ್ಯವಾಗುವುದು. ಆರ್ಯರು ಹೆಚ್ಚು ತಾತ್ತ್ವಿಕವಾದ ಇಂದ್ರಿಯಾತೀತವಾದ ಸತ್ಯವನ್ನು ಅರಸಲು ತೊಡಗಿದರು. ಏಕದೇವವಾದ ಮಾನವಸಹಜವಾಗಿತ್ತು. ಅವರು ಆ ದೇವರನ್ನು ವಿಶ್ವವೆಲ್ಲ ನಿನ್ನಲ್ಲಿದೆ, ನೀನೆಲ್ಲರ ಹೃದಯಾಂತರ್ಯಾಮಿ” ಎಂದರೂ ಅವರಿಗೆ ತೃಪ್ತಿ ಇರಲಿಲ್ಲ. ಹಿಂದೂಗಳು ನಿರ್ಭೀತರಾಗಿದ್ದರು. ಇದು ಅವರ ಮಹಿಮೆ. ಎಲ್ಲವನ್ನೂ ಅವರು ನಿರ್ಭೀತಿಯಿಂದ ಆಲೋಚಿಸುತ್ತಿದ್ದರು. ಅವರು ನಿರ್ಭೀತರಾಗಿದ್ದರು ಎಂದರೆ ಅವರ ನಿರ್ಭೀತಿಯ ಒಂದು ಕಣವೆ ಸಾಕು ಮಹಾ ಧೈರ್ಯವಂತರೆಂದು ಹೆಮ್ಮೆ ಕೊಚ್ಚಿಕೊಳ್ಳುವ ಪಾಶ್ಚಾತ್ಯರನ್ನು ಅಂಜಿಸುವುದಕ್ಕೆ. ಮ್ಯಾಕ್ಸ್‌ಮುಲ್ಲರ್ ಅವರು ಈ ವಿಷಯವಾಗಿ ಚೆನ್ನಾಗಿಯೇ ಹೇಳಿರುವರು: “ಅವರು ಹೋದ ಎತ್ತರದಲ್ಲಿ ಅವರಿಗೆ ಮಾತ್ರ ಉಸಿರಾಡುವುದಕ್ಕೆ ಸಾಧ್ಯ; ಅಲ್ಲಿ ಇತರರ ಶ್ವಾಸಕೋಶಗಳು ಚೂರು ಚೂರಾಗಿ ಹೋಗುತ್ತಿದ್ದವು. ಈ ಸಾಹಸಿಗಳು ಯುಕ್ತಿ ಒಯ್ದೆಡೆ ಅದನ್ನು ಅನುಸರಿಸಿದರು, ಅದರ ಪರಿಣಾಮ ಏನಾದರೂ ಚಿಂತೆಯಿಲ್ಲ. ಬಾಲ್ಯದಿಂದ ಅಪ್ಪಿದ್ದ ತಮ್ಮ ಮೂಢ ನಂಬಿಕೆಗಳೆಲ್ಲ ನಾಶವಾದರೂ ಚಿಂತೆಯಿಲ್ಲ. ಸಮಾಜ ಅವರನ್ನು ಯಾವ ರೀತಿ ನೋಡುವುದು, ಹೇಗೆ ಟೀಕಿಸುವುದು ಎಂಬುದನ್ನು ಬಗೆಗೆ ತರಲಿಲ್ಲ. ಆದರೆ ಯಾವುದನ್ನು ತಾವು ಸತ್ಯವೆಂದು, ಧರ್ಮವೆಂದು ತಿಳಿದರೋ ಅದನ್ನು ಬೋಧಿಸಿದರು, ಅದರ ವಿಷಯವಾಗಿ ಮಾತನಾಡಿದರು.”

ಪುರಾತನ ವೈದಿಕ ಋಷಿಗಳು ಪರ್ಯಾಲೋಚಿಸಿರುವುದನ್ನು ನೋಡುವುದಕ್ಕೆ ಮುಂಚೆ ವೇದಗಳಲ್ಲಿ ಬರುವ ಒಂದೆರಡು ವಿಚಿತ್ರ ಪ್ರಸಂಗಗಳನ್ನು ತೆಗೆದುಕೊಳ್ಳೋಣ. ಇದರಲ್ಲಿ ಒಬ್ಬೊಬ್ಬ ದೇವರನ್ನೂ ತೆಗೆದುಕೊಂಡು ಅನಂತಾತ್ಮನಾದ ಪರಮೇಶ್ವರನ ಎತ್ತರಕ್ಕೆ ಅವನನ್ನು ಏರಿಸುವುದಕ್ಕೆ ಕಾರಣವನ್ನು ಕೊಡಬೇಕಾಗಿದೆ. ಪ್ರೊಫೆಸರ್‌ ಮ್ಯಾಕ್ಸ್ ಮುಲ್ಲರ್ ಇದಕ್ಕೆ ಬೇರೊಂದು ಪದವನ್ನು ಸೃಷ್ಟಿಸಿರುವರು, ಅದೇ ಹಿನೊಥೀಯಿಸಮ್ (henotheism) ಎಂದರೆ, ತಾನು ಉಪಾಸಿಸುವ ದೇವರು ಒಬ್ಬನು ಮಾತ್ರವೇ ಇರುವುದು ಎಂದು ಹೇಳದ ಏಕದೇವೋಪಾಸನೆ. ಅಂದರೆ ಇತರ ದೇವರುಗಳನ್ನು ದೂರದೆ ತಮ್ಮ ದೇವರಲ್ಲಿ ಇಟ್ಟಿರುವ ಅಚಲವಾದ ನಂಬಿಕೆ. ಇದನ್ನು ವಿವರಿಸುವುದಕ್ಕೆ ನಾವೇನು ಅಷ್ಟೊಂದು ತೊಂದರೆ ತೆಗೆದುಕೊಳ್ಳಬೇಕಾಗಿಲ್ಲ. ಇದು ಶಾಸ್ತ್ರದಲ್ಲಿಯೇ ಇದೆ. ಈ ದೇವರನ್ನು ಪರಮೇಶ್ವರನ ಶಿಖರಕ್ಕೆ ಏರಿಸಿದ ಶಾಸ್ತ್ರಭಾಗದ ಸಮಿಾಪದಲ್ಲಿಯೇ ಇದಕ್ಕೆ ತಕ್ಕ ವಿವರಣೆಯೂ ಇದೆ. ಹಿಂದೂ ಪುರಾಣಗಳಲ್ಲಿ ಈ ವೈಶಿಷ್ಟ್ಯ ಹೇಗೆ ಸಾಧ್ಯವಾಯಿತು, ಇತರರಲ್ಲಿ ಏಕೆ ಸಾಧ್ಯವಾಗಲಿಲ್ಲ ಎಂಬ ಪ್ರಶ್ನೆ ಉದಯಿಸುವುದು. ಬ್ಯಾಬಿಲೋನಿಯಾ ಅಥವಾ ಗ್ರೀಸ್ ಪುರಾಣಗಳಲ್ಲಿ ಒಬ್ಬ ದೇವರು ಮೇಲೇರಲು ಯತ್ನಿಸುವುದನ್ನು ನೋಡುವೆವು. ಮೇಲೆ ಬಂದ ಮೇಲೆ ಅವನು ಅಲ್ಲಿರುವನು, ಉಳಿದ ದೇವರುಗಳೆಲ್ಲ ಮಾಯವಾಗಿ ಹೋಗುವರು. ಮೊಲಾಕಿನಲ್ಲೆಲ್ಲಾ ಯಹೋವನೇ ಶ್ರೇಷ್ಠನಾಗಿ ಕೊನೆಗೆ ಇತರರೆಲ್ಲ ಮರತೇ ಹೋಗುವರು. ಯಹೋವನೇ ಸರ್ವಶ್ರೇಷ್ಠನಾಗಿ ದೇವ ದೇವನಾಗುವನು. ಇತರ ದೇವತೆಗಳೆಲ್ಲ ಮರೆವಿಗೆ ಗುರಿಯಾಗಿ ಎಲ್ಲ ಕಾಲಕ್ಕೂ ಕಣ್ಮರೆಯಾಗುವರು. ಗ್ರೀಕ್ ದೇವತೆಗಳಿಗೂ ಇದೇ ಗತಿ ಒದಗಿತು. ಸ್ಯೂಸ್ ದೇವತೆ ಪ್ರಧಾನನಾಗಿ, ವಿಶ್ವದ ಒಡೆಯನಾಗುತ್ತಾನೆ. ಉಳಿದ ದೇವರುಗಳೆಲ್ಲಾ ಸಣ್ಣಪುಟ್ಟ ದೇವತೆಗಳಾಗುತ್ತಾರೆ. ಇದೇ ರೀತಿ ಅನಂತರವೂ ಆಯಿತು. ಬೌದ್ಧರು ಮತ್ತು ಜೈನರು ತಮ್ಮ ಸಂತರಲ್ಲಿ ಒಬ್ಬನನ್ನು ಆ ಸ್ಥಾನಕ್ಕೆ ಏರಿಸುವರು. ಇತರ ದೇವತೆಗಳೆಲ್ಲ ಬುದ್ಧ ಅಥವಾ ಜಿನನ ಅಡಿಯಾಳುಗಳಾಗುವರು. ಪ್ರಪಂಚದಲ್ಲೆಲ್ಲ ನಡೆದಿರುವ ರೀತಿಯೇ ಇದು. ಆದರೆ ವೇದಗಳಲ್ಲಿ ಮಾತ್ರ ವ್ಯತ್ಯಾಸ. ಇಲ್ಲಿಯೂ ಒಬ್ಬ ದೇವರನ್ನು ಹೊಗಳುವರು; ಸದ್ಯಕ್ಕೆ ಇತರ ದೇವತೆಗಳೂ ಕೂಡ ಅವನ ಆಜ್ಞಾಧಾರಕರು ಎನ್ನುವರು. ಯಾರನ್ನು ವರುಣ ಮೇಲೆತ್ತಿದ್ದ ಎನ್ನುವರೋ, ಅವನನ್ನೇ ಮುಂದಿನ ಗ್ರಂಥದಲ್ಲಿ ಅತಿ ಶ್ರೇಷ್ಠ ಸ್ಥಾನಕ್ಕೆ ಒಯ್ಯುವರು. ಇಲ್ಲಿ ಒಬ್ಬರಾದ ಮೇಲೊಬ್ಬರು ಈಶ್ವರನ ಸ್ಥಾನಕ್ಕೆ ಏರುವರು. ಆದರೆ ಇದಕ್ಕೆ ವಿವರಣೆ ಗ್ರಂಥದಲ್ಲಿದೆ; ಇದೊಂದು ಅದ್ಭುತವಾದ ವಿವರಣೆ. ಭರತ ಖಂಡದಲ್ಲಿ ಅನಂತರ ಬಂದ ಉದಾತ್ತ ಭಾವನೆಗಳಿಗೆಲ್ಲ ಇದು ಮೂಲವಾಗಿದೆ. ಜಗತ್ತಿನ ಧರ್ಮಗಳಲ್ಲೆಲ್ಲ ಇದೇ ಪುಣ್ಯತಮ ಭಾವನೆಯಾಗಿದೆ: 'ಏಕಂ ಸತ್ ವಿಪ್ರಾ ಬಹುಧಾ ವದನ್ತಿ” – ಇರುವುದೊಂದು, ಋಷಿಗಳು ಅದನ್ನು ಭಿನ್ನ ಭಿನ್ನ ನಾಮಗಳಿಂದ ಕರೆಯುವರು. ಇತರ ದೇವರುಗಳ ವಿಷಯವಾಗಿ ರಚಿಸಿದ ಸ್ತೋತ್ರಗಳಲ್ಲೆಲ್ಲ ಕಂಡ ದೇವರು ಒಬ್ಬನೇ ಒಬ್ಬನು. ನೋಡುವವನು ಮಾತ್ರ ಈ ವ್ಯತ್ಯಾಸವನ್ನು ಮಾಡಿದನು. ಒಬ್ಬನೇ ಒಬ್ಬ ದೇವರನ್ನು, ಮಂತ್ರಕಾರರು, ಋಷಿಗಳು, ಕವಿಗಳು, ಭಿನ್ನ ಭಿನ್ನ ಭಾಷೆಗಳಲ್ಲಿ ವರ್ಣಿಸುವರು 'ಏಕಂ ಸತ್ ವಿಪ್ರಾ ಬಹುಧಾ ವದನ್ತಿ.'' ಈ ಒಂದು ಮಂತ್ರ ಅನಂತ ಅದ್ಭುತ ಪರಿಣಾಮಕಾರಿಯಾಗಿದೆ. ಧರ್ಮದ ಹೆಸರಿನಲ್ಲಿ ಬಲಾತ್ಕಾರಮಾಡದೆ ಇರುವುದು, ಧರ್ಮದ ಹೆಸರಿನಲ್ಲಿ ಹಿಂಸೆ ಮಾಡದೆ ಇರುವುದು, ಈ ಒಂದು ಧರ್ಮದಲ್ಲಿ ಮಾತ್ರ ಸಾಧ್ಯ ಎಂಬುದನ್ನು ಅರಿತರೆ ನಿಮಗೆ ಆಶ್ಚರ್ಯವಾಗಬಹುದು. ಇಲ್ಲಿ ಆಸ್ತಿಕರು ನಾಸ್ತಿಕರು ಇರುವರು, ದ್ವೈತಿಗಳು ವಿಶಿಷ್ಟಾದ್ವೈತಿಗಳು ಮತ್ತು ಅದ್ವೈತಿಗಳು ಇರುವರು. ಇವರಿಗೆ ಯಾರೂ ತೊಂದರೆ ಕೊಡುವುದಿಲ್ಲ. ನಾಸ್ತಿಕರಿಗೆ ದೇವಸ್ಥಾನದಲ್ಲಿ ದೇವದೇವರುಗಳ ವಿರೋಧವಾಗಿ ಬೋಧಿಸುವುದಕ್ಕೆ ಅವಕಾಶ ಸಿಕ್ಕಿತು. ಅವರಿಗೆ ಯಾರೂ ತೊಂದರೆ ಕೊಡುವುದಿಲ್ಲ. ದೇಶದಲ್ಲೆಲ್ಲ ದೇವರು, ವೇದ ಮುಂತಾದವೆಲ್ಲ ದೊಡ್ಡ ಒಂದು ಮೂಢ ನಂಬಿಕೆ; ಪುರೋಹಿತರೆ ಇದನ್ನೆಲ್ಲ ಸೃಷ್ಟಿಸಿರುವರು ಎಂದು ಸಾರುತ್ತಾ ಹೋಗುವುದಕ್ಕೆ ಅವರಿಗೆ ಅವಕಾಶ ಸಿಕ್ಕಿತು. ಅವರಿಗೆ ಯಾವ ವಿಧವಾದ ಆತಂಕವನ್ನೂ ತಂದೊಡ್ಡಲಿಲ್ಲ. ಹಿಂದೂಗಳಿಗೆ ಪವಿತ್ರವಾದ ಹಳೆಯದನ್ನೆಲ್ಲ ಮೂಢನಂಬಿಕೆಯೆಂದು ಬುದ್ಧ ಅಲ್ಲಗಳೆಯುತ್ತಾ ಹೋದ. ಆದರೂ ಅವನು ವೃದ್ದನಾಗುವವರೆಗೆ ಭರತಖಂಡದಲ್ಲಿ ಬದುಕಿರುವುದಕ್ಕೆ ಸಾಧ್ಯವಾಯಿತು. ದೇವರ ವಿಷಯವಾಗಿ ತಾತ್ಸಾರದಿಂದ ನೋಡಿದ ಜೈನರೂ ಕೂಡ ಅಂತೆಯೇ, 'ದೇವರೊಬ್ಬ ಇರಲು ಹೇಗೆ ಸಾಧ್ಯ? ಇದೊಂದು ಬರಿಯ ಮೂಢನಂಬಿಕೆ'' ಎಂದು ಅವರು ಹಳಿಯುತ್ತಿದ್ದರು. ಹೀಗೆಯೇ ಎಷ್ಟೋ ಉದಾಹರಣೆಗಳಿವೆ.

ಮಹಮ್ಮದೀಯರು ಭರತಖಂಡಕ್ಕೆ ಬರುವುದಕ್ಕೆ ಮುಂಚೆ ಧಾರ್ಮಿಕ ಹಿಂಸೆ ಏನೆಂಬುದು ಇಲ್ಲಿ ಗೊತ್ತಿರಲಿಲ್ಲ. ಹಿಂದೂಗಳು ಹೊರಗಿನವರಿಂದ ಅದನ್ನು ಅನುಭವಿಸಬೇಕಾಯಿತು. ಈಗಲೂ ಕೂಡ ಅನೇಕ ಕ್ರೈಸ್ತರಿಗೆ ಹಿಂದೂಗಳು ಚರ್ಚುಗಳನ್ನು ಕಟ್ಟಿಸಿಕೊಟ್ಟಿರುವರು. ಅವರಿಗೆ ಸಹಾಯ ಮಾಡುವುದಕ್ಕೆ ಈಗಲೂ ಸಿದ್ದರಾಗಿರುವರು. ಇಲ್ಲಿ ಧರ್ಮದ ಹೆಸರಿನಲ್ಲಿ ಎಂದಿಗೂ ರಕ್ತಪಾತವಿರಲಿಲ್ಲ. ಭರತಖಂಡದಲ್ಲಿ ಉದ್ಭವಿಸಿದ ಅವೈದಿಕ ಧರ್ಮಗಳಲ್ಲಿ ಕೂಡ ನಾವು ಈ ಗುಣವನ್ನು ನೋಡುತ್ತೇವೆ. ಉದಾಹರಣೆಗೆ ಬೌದ್ಧ ಧರ್ಮ. ಅದು ಬಹಳ ದೊಡ್ಡದು, ನಿಜ. ಆದರೆ ಇದನ್ನು ವೇದಾಂತವೆನ್ನುವುದು ಅರ್ಥವಿಲ್ಲದ್ದು. ಇದು ಕ್ರೈಸ್ತ ಧರ್ಮಕ್ಕೂ ಸಾಲ್ವೆಷನ್ ಆರ್ಮಿಗೂ ಇರುವ ವ್ಯತ್ಯಾಸದಂತೆ. ಬೌದರಲ್ಲಿ ಬಹಳ ಮುಖ್ಯವಾದ ಒಳ್ಳೆಯ ಭಾವನೆಗಳಿವೆ. ಆದರೆ ಈ ಒಳ್ಳೆಯ ಭಾವನೆಗಳು ಯಾರ ಕೈಗೆ ಬಿದ್ದವೋ ಅವರು ಅವನ್ನು ಸುರಕ್ಷಿತವಾಗಿಡುವಷ್ಟು ಯೋಗ್ಯರಾಗಿರಲಿಲ್ಲ. ತಾತ್ತ್ವಿಕರಿಂದ ಬಂದ ಅನರ್ಘ್ಯ ರತ್ನಗಳು ದೊಂಬಿಯ ಕೈಗೆ ಬಿದ್ದವು. ಆ ಜನರು ತಮಗೆ ಇಚ್ಛೆ ಬಂದ ರೀತಿಯಲ್ಲಿ ಅವನ್ನು ಅನುಸರಿಸಲು ಪ್ರಾರಂಭಿಸಿದರು. ಅವರಲ್ಲಿ ಬಹಳ ಉತ್ಸಾಹವಿತ್ತು, ಕೆಲವು ಅದ್ಭುತ ಭಾವನೆಗಳಿದ್ದುವು. ಆದರೆ ಎಲ್ಲವನ್ನೂ ಸುರಕ್ಷಿತವಾಗಿಡಬೇಕಾದರೆ ಮತ್ತೇನೋ ಬೇಕಾಗಿತ್ತು - ಅವೇ ವಿಚಾರ ಮತ್ತು ಯುಕ್ತಿ. ಎಲ್ಲಿ ದೊಂಬಿಯ ಕೈಗೆ ಸಾರ್ವಜನಿಕರ ಉಪಕಾರಾರ್ಥ ಭಾವನೆಗಳು ಬೀಳುವುವೋ ಅಲ್ಲಿ ನಮಗೆ ಮೊದಲಿಗೆ ಕಾಣುವುದೇ ಪತನದ ಚಿಹ್ನೆ. ಭಾವಗಳನ್ನು ಸುರಕ್ಷಿತವಾಗಿಡಬೇಕಾದರೆ ವಿದ್ಯೆ ಬುದ್ದಿ ಇವು ಆವಶ್ಯಕ. ಪ್ರಪಂಚದಲ್ಲಿ ಅನ್ಯರನ್ನು ತಮ್ಮ ಮತಕ್ಕೆ ಸೇರಿಸಿಕೊಳ್ಳುವ ಧರ್ಮದ ಪಾತ್ರದಲ್ಲಿ ಪ್ರಥಮತಃ ಬೌದ್ಧ ಧರ್ಮ ಹೊರಟಿತು. ಬೌದ್ಧರು ಆಗಿನ ಕಾಲದ ನಾಗರಿಕ ದೇಶವಿದೇಶಗಳ ಮೂಲೆಮೂಲೆಗಳಿಗೆ ಹೋದರು. ಆದರೂ ಧರ್ಮದ ಹೆಸರಿನಲ್ಲಿ ಒಂದು ಬಿಂದು ರಕ್ತವನ್ನೂ ಹರಿಸಲಿಲ್ಲ. ಚೈನಾದೇಶದವರು ಬೌದ್ಧ ಭಿಕ್ಷುಗಳನ್ನು ಮೊದಲು ಹಿಂಸಿಸತೊಡಗಿದರು. ಅವ್ಯಾಹತವಾಗಿ ಇಬ್ಬರು ಮೂವರು ಚಕ್ರವರ್ತಿಗಳು ಸಹಸ್ರಾರು ಭಿಕ್ಷುಗಳನ್ನು ಕೊಲ್ಲಿಸಿದರು. ಅನಂತರ ಬೌದ್ಧರಿಗೆ ದೆಸೆ ತಿರುಗಿತು. ಅನಂತರವೆ ಬಂದ ಚಕ್ರವರ್ತಿಯೊಬ್ಬ ಬೌದ್ಧರ ಪರವಾಗಿ ಸೇಡನ್ನು ತೀರಿಸಿಕೊಳ್ಳಲು ಭಿಕ್ಷುಗಳನ್ನು ಕೇಳಿದನು. ಆದರೆ ಭಿಕ್ಷುಗಳು ಅದನ್ನು ನಿರಾಕರಿಸಿದರು. ಅದಕ್ಕೆಲ್ಲ ಕಾರಣ ಈ ಒಂದು ಮಂತ್ರ. ಆದಕಾರಣವೆ ನೀವು ಇದನ್ನು ಜ್ಞಾಪಕದಲ್ಲಿಡಿ ಎನ್ನುವುದು: “ಯಾವುದನ್ನು ಇಂದ್ರ ಮಿತ್ರ ವರುಣ ಎನ್ನುವರೋ ಅವರಲ್ಲಿ ಇರುವುದೊಂದೆ; ಋಷಿಗಳು ಹಲವು ಹೆಸರುಗಳಿಂದ ಅದನ್ನು ಕರೆಯುತ್ತಿರುವರು.”

ಇದನ್ನು ಯಾವ ಕಾಲದಲ್ಲಿ ರಚಿಸಿದರೊ ಅದು ಗೊತ್ತಿಲ್ಲ. ಅದು ಬಹುಶಃ ಎಂಟು ಸಾವಿರ ವರುಷಗಳ ಹಿಂದೆ ಇರಬಹುದು. ಆಧುನಿಕ ವಿದ್ವಾಂಸರು ಏನೇ ಹೇಳಿದರೂ ಅದು ಒಂಭತ್ತು ಸಾವಿರ ವರ್ಷಗಳಾದರೂ ಇರಬೇಕು. ಈ ಧಾರ್ಮಿಕ ಭಾವನೆಗಳಾವುವೂ ಆಧುನಿಕವಲ್ಲ. ಅವು ಹಿಂದೆ ಬರೆದಾಗ ಹೇಗೆ ಇದ್ದವೋ ಈಗಲೂ ಹಾಗೆಯೇ ಇವೆ ಅಥವಾ ಅದಕ್ಕಿಂತ ಈಗ ಅಚ್ಚ ಹೊಸದಾಗಿವೆ. ಏಕೆಂದರೆ ಆಗ ಮಾನವ ಈಗಿನಂತೆ ನಾಗರಿಕನಾಗಿರಲಿಲ್ಲ. ಆಗ ಮಾನವ ಇತರರು ತನ್ನಂತೆ ಆಲೋಚಿಸುವುದಿಲ್ಲವೆಂದು ಅವರನ್ನು ಕೊಲ್ಲಲ್ಲು ಕಲಿತಿರಲಿಲ್ಲ; ಅವನಿನ್ನೂ ರಕ್ತದ ಕಾಲುವೆ ಹರಿಸಿರಲಿಲ್ಲ. ತನ್ನ ಸಹೋದರರಿಗೇ ಅವನು ಕಂಟಕಪ್ರಾಯನಾಗಿರಲಿಲ್ಲ. ದಯೆಯ ಸೋಗಿನಲ್ಲಿ ಮಾನವ ಕುಲವನ್ನು ನಿರ್ಮೂಲಮಾಡಲು ಯತ್ನಿಸಲಿಲ್ಲ. ಆದಕಾರಣವೇ ಈ ಪದಗಳು ನಮಗೆ ಈಗ ಹೊಸದಾಗಿ ಬಂದಂತೆ ಇವೆ, ಮಹಾಪ್ರಚೋದನಕಾರಿಗಳಾಗಿವೆ, ಜೀವದಾನ ಮಾಡುವಂತೆ ಇವೆ, ಆಗಿನ ಕಾಲಕ್ಕಿಂತ ಹೆಚ್ಚು ಹೊಸದಾಗಿರುವಂತೆ ಇವೆ. "ಏಕಂ ಸತ್ ವಿಪ್ರಾಃ ಬಹುಧಾ ವದನ್ತಿ." ಧರ್ಮಗಳು ಯಾವ ಹೆಸರಿನಲ್ಲಿರಲಿ, ಹಿಂದೂಗಳಾಗಲಿ, ಬೌದ್ದರಾಗಲಿ, ಮಹಮ್ಮದೀಯರಾಗಲಿ, ಕ್ರೈಸ್ತರಾಗಲಿ, ಎಲ್ಲರಿಗೂ ಇರುವ ದೇವರು ಒಬ್ಬನೇ. ಯಾರು ಇತರ ಧರ್ಮಗಳ ದೇವರನ್ನು ಅಲ್ಲಗಳೆಯುವನೊ, ಅವನು ತನ್ನ ದೇವರನ್ನೇ ಅಲ್ಲಗಳೆಯುತ್ತಿರುವನು ಎಂಬುದನ್ನು ಅರಿಯಬೇಕಾಗಿದೆ.

. ಅವರು ತಲುಪಿದ ನಿರ್ಣಯ ಅದು. ಆದರೆ ಈ ಪುರಾತನ ಏಕದೇವೇಶ್ವರವಾದ ಅವರನ್ನು ತೃಪ್ತಿಪಡಿಸಲಿಲ್ಲ. ಅದು ಬಹಳ ಮುಂದೆ ಹೋಗಲಿಲ್ಲ. ಅದು ಎದುರಿಗೆ ಇರುವ ಪ್ರಪಂಚವನ್ನು ವಿವರಿಸಲಿಲ್ಲ. ಪ್ರಪಂಚಕ್ಕೆ ಆಳುವವನೊಬ್ಬನಿದ್ದರೆ ಅವನು ಪ್ರಪಂಚವನ್ನು ವಿವರಿಸಿದಂತೆ ಆಗಲಿಲ್ಲ. ನಿಜವಾಗಿಯೂ ಅಲ್ಲ. ಸೃಷ್ಟಿಯ ಹೊರಗೆ ಇರುವ ದೇವರು ಇದನ್ನು ವಿವರಿಸುವುದಕ್ಕಂತೂ ಆಗುವುದೇ ಇಲ್ಲ. ಅವನು ನೀತಿಗೆ ಮಾರ್ಗದರ್ಶಕನಾಗಬಹುದು. ಆದರೆ ಅದು ಪ್ರಪಂಚದ ವಿವರಣೆ ಅಲ್ಲ. ಈಗ ಉದಯಿಸುವ, ಬರಬರುತ್ತಾ ಬೃಹದಾಕಾರವನ್ನು ತಾಳುವ, ಮೊದಲನೆ ಪ್ರಶ್ನೆಯೇ ವಿಶ್ವಕ್ಕೆ ಸಂಬಂಧಪಟ್ಟದ್ದು: ವಿಶ್ವ ಎಲ್ಲಿಂದ ಬಂತು? ಅದು ಹೇಗೆ ಇರುವುದು? ಇವೇ ಮೂಡಿದ ಹೊಸ ಪ್ರಶ್ನೆಗಳು. ಈ ಪ್ರಶ್ನೆಗೆ ಸಂಬಂಧಪಟ್ಟ ಹಲವು ಮಂತ್ರಗಳು ಈ ಭಾವವನ್ನು ವ್ಯಕ್ತಗೊಳಿಸಲು ರೂಪುಗೊಳ್ಳುತ್ತಿದ್ದವು. ಮತ್ತೆಲ್ಲಿಯೂ ಕೆಳಗೆ ಬರುವಷ್ಟು ಸುಂದರವಾಗಿ, ಕಾವ್ಯಮಯವಾಗಿ ಇದನ್ನು ಚಿತ್ರಿಸಿಲ್ಲ:

“ಆಗ ಅಸ್ತಿಯೂ ಇರಲಿಲ್ಲ, ನಾಸ್ತಿಯೂ ಇರಲಿಲ್ಲ, ವಾಯು ಆಕಾಶಗಳಾವುವೂ ಇರಲಿಲ್ಲ. ಏನೂ ಇರಲಿಲ್ಲ. ಎಲ್ಲವನ್ನೂ ಯಾವುದು ಆವರಿಸಿತ್ತು? ಎಲ್ಲವೂ ಎಲ್ಲಿತ್ತು? ಆಗ ಮರಣವೂ ಇರಲಿಲ್ಲ, ಅಮರಣವೂ ಇರಲಿಲ್ಲ, ಹಗಲು ರಾತ್ರಿಗಳಾವುವೂ ಇರಲಿಲ್ಲ.” ಭಾಷಾಂತರದಲ್ಲಿ ಅದರ ಕಾವ್ಯಸೌಂದರ್ಯವೆಷ್ಟೋ ಹಾಳಾಗುವುದು. “ಆಗ ಮೃತ್ಯುವಿರಲಿಲ್ಲ, ಅಮೃತ್ಯುವಿರಲಿಲ್ಲ. ಹಗಲು-ರಾತ್ರೆಗಳ ಬದಲಾವಣೆಗಳಾವುವೂ ಇರಲಿಲ್ಲ. ಸಂಸ್ಕೃತ ಭಾಷೆಯೇ ಸಂಗೀತದಂತೆ ಇದೆ. “ಅದೊಂದೇ ಇತ್ತು. ಇಡೀ ವಿಶ್ವವನ್ನೆಲ್ಲ ವ್ಯಾಪಿಸಿರುವ ಪ್ರಾಣವೊಂದೇ ಇತ್ತು. ಅದಿನ್ನೂ ಚಲಿಸಲು ಪ್ರಾರಂಭಿಸಿರಲಿಲ್ಲ. ಅದು ಚಲಿಸದೆ ಇತ್ತು ಎಂಬ ಒಂದು ಭಾವನೆಯನ್ನು ಜ್ಞಾಪಕದಲ್ಲಿಡುವುದು ಒಳ್ಳೆಯದು. ಏಕೆಂದರೆ ಇದೇ ಅನಂತರ ಹಿಂದೂಗಳ ಸೃಷ್ಟಿ ಸಿದ್ದಾಂತದಲ್ಲಿ ಮೊಳೆಯುವುದು. ಹಿಂದೂ ತತ್ತ್ವದ ಮತ್ತು ದಾರ್ಶನಿಕರ ದೃಷ್ಟಿಯಲ್ಲಿ ಈ ಸೃಷ್ಟಿಯೆಲ್ಲ ಚಲನೆಯ ರಾಶಿ. ಕೆಲವು ವೇಳೆ ಈ ಚಲನೆ ಕಡಮೆಯಾಗಿ ಸೂಕ್ಷ್ಮ ಸೂಕ್ಷ್ಮವಾಗಿ ಕೆಲವು ಕಾಲ ಆ ಸ್ಥಿತಿಯಲ್ಲಿರುವುದು. ಈ ಸೂಕ್ತದಲ್ಲಿ ವರ್ಣಿತವಾಗಿರುವ ಸ್ಥಿತಿ ಅದೇ. ಅದು ಚಲನೆಯಿಲ್ಲದ ಅಸ್ತಿತ್ವದಲ್ಲಿತ್ತು. ಸೃಷ್ಟಿ ಪ್ರಾರಂಭವಾದಾಗ ಅದು ಕಂಪಿಸಲು ತೊಡಗಿತು. ಅದರಿಂದಲೇ ಈ ಸೃಷ್ಟಿ ಸಮಸ್ತವೂ ಬಂದದ್ದು. ಆ ಒಂದು ಉಸಿರು, ಶಾಂತವಾಗಿ, ತನಗೆ ತಾನೇ ಆಲಂಬನವಾಗಿ ಇತ್ತು. ಅದರಾಚೆಗೆ ಏನೂ ಇರಲಿಲ್ಲ.

“ಆದಿಯಲ್ಲಿ ತಮಸ್ಸು ಇತ್ತು. ನಿಮ್ಮಲ್ಲಿ ಯಾರಾದರೂ ಇಂಡಿಯಾ ದೇಶಕ್ಕೆ ಹೋಗಿದ್ದರೆ, ಅಥವಾ ಉಷ್ಣವಲಯಗಳಲ್ಲಿದ್ದರೆ, ಅಲ್ಲಿ ಮಳೆಗಾಲ ಹೇಗೆ ಆರಂಭವಾಗುವುದು ಎಂಬುದನ್ನು ನೋಡಿದಾಗ, ಈ ಪದಗಳು ಹೇಗೆ ಧ್ವನಿಪೂರ್ಣವಾಗಿವೆ ಎಂಬುದು ಗೊತ್ತಾಗುವುದು. ಮೂರು ಮಂದಿ ಕವಿಗಳು ಇದನ್ನು ವರ್ಣಿಸಿರುವುದು ನನಗೆ ಜ್ಞಾಪಕವಿದೆ. ಮಿಲ್ಟನ್ (No light but darkness visible” 'ಬೆಳಕಲ್ಲ, ಕಣ್ಣಿಗೆ ಕಾಣುವ ಕಗ್ಗತ್ತಲೆ' ಎನ್ನುವನು. ಕಾಳಿದಾಸ 'ಸೂಜಿ ತೂರಿಹೋಗುವಷ್ಟು ಘನೀಭೂತವಾಗಿತ್ತು ಕತ್ತಲೆ' ಎನ್ನುವನು. ಆದರೆ ಯಾವುದೂ ವೇದದಲ್ಲಿ ಬರುವ ''ತಮಸ್ಸು ತಮಸ್ಸಿನಲ್ಲಿ ಆವೃತವಾಗಿತ್ತು" ಎಂಬುದರ ಸಮಾಪಕ್ಕೆ ಬರುವುದಿಲ್ಲ. ಎಲ್ಲ ಒಣಗಿಹೋಗಿದೆ, ಕಟಲ್ ಕಟಿಲ್ ಎಂದು ಶಬ್ದ ಮಾಡುತ್ತಿದೆ. ಸೃಷ್ಟಿಯೆಲ್ಲ ಉರಿದುಹೋಗುವಂತೆ ಕಾಣುತ್ತಿದೆ. ಹಲವು ಕಾಲ ಹೀಗಿತ್ತು. ಒಂದು ದಿನ ಮಧ್ಯಾಹ್ನ ದೂರದ ದಿಗಂತದಲ್ಲಿ ಒಂದು ಚೂರು ಮುಗಿಲಿತ್ತು. ಅರ್ಧ ಗಂಟೆಯೊಳಗೆ ಅದು ಭೂಮಿಯನ್ನೆಲ್ಲ ವ್ಯಾಪಿಸಿತು. ಎಲ್ಲಿ ನೋಡಿದರೂ ಮೋಡ, ಕೆಳಗೆ ಮೇಲೆಲ್ಲಾ ಮೋಡ. ಆಗ ಜಲಪ್ರಳಯದಂತೆ ಮಳೆ ಆರಂಭವಾಗುವುದು. ಇಚ್ಛೆಯೇ ಸೃಷ್ಟಿಗೆ ಕಾರಣ ಎನ್ನುವರು. ಆದಿಯಲ್ಲಿ ಇದ್ದದ್ದು ಇಚ್ಛಾಶಕ್ತಿಯಾಯಿತು. ಈ ಇಚ್ಛಾಶಕ್ತಿಯೇ ಆಸೆಯ ರೂಪವನ್ನು ತಾಳಿತು. ಇದನ್ನು ನಾವು ನೆನಪಿನಲ್ಲಿಡಬೇಕಾಗಿದೆ. ಏಕೆಂದರೆ, ಈ ಆಸೆಯೇ ನಮ್ಮಲ್ಲಿರುವುದಕ್ಕೆಲ್ಲ ಕಾರಣ ಎನ್ನುವರು. ಈ ಇಚ್ಛೆಯೇ ಬೌದ್ದ ಮತ್ತು ವೇದಾಂತ ದರ್ಶನಗಳ ಮೂಲಾಧಾರ. ಅನಂತರ ಇದು ಜರ್ಮನ್ ತತ್ತ್ವಶಾಸ್ತ್ರಕ್ಕೂ ಸೇರಿ ಶೋಪನ್‌ಹೋರ್‌ನ ದರ್ಶನದ ಮೂಲಸ್ತಂಭವಾಗಿದೆ. ಈ ಸೂಕ್ತದಲ್ಲಿ ನಾವು ಮೊದಲು ಈ ಭಾವನೆಯನ್ನು ನೋಡುತ್ತೇವೆ. |

'ಮೊದಲು ಆಸೆ ಅಂಕುರಿಸಿತು. ಇದೇ ಮನಸ್ಸಿಗೆ ಆದಿ, ಋಷಿಗಳು ಜ್ಞಾನದಿಂದ ತಮ್ಮ ಹೃದಯವನ್ನು ಪರಿಶೋಧಿಸಿ ಅಸ್ತಿ ನಾಸ್ತಿಗಳ ಮಧ್ಯೆ ಸೇತುವೆಯಂತೆ ಇರುವ ಇಚ್ಛೆಯನ್ನು ಕಂಡರು.”

ಇದೊಂದು ಅಪೂರ್ವವಾದ ಮಾತು - “ಅವನಿಗೂ ಕೂಡ ತಿಳಿಯದೋ ಏನೊ" ಎಂದು ಹೇಳಿ ಕವಿ ಮುಕ್ತಾಯ ಮಾಡುವನು. ನಾವು ಈ ಮಂತ್ರದಲ್ಲಿ ಕಾವ್ಯ ಸೌಂದರ್ಯವಲ್ಲದೆ, ಸೃಷ್ಟಿಗೆ ಸಂಬಂಧಪಟ್ಟ ಪ್ರಶ್ನೆಗಳು ಸ್ಪಷ್ಟರೂಪವನ್ನು ತಾಳುವುದನ್ನು ನೋಡುತ್ತೇವೆ. ಸಾಧಾರಣ ಉತ್ತರಗಳು ಋಷಿಗಳಿಗೆ ಯಾವ ತೃಪ್ತಿಯನ್ನೂ ಕೊಡುವ ಸ್ಥಿತಿಯಲ್ಲಿರಲಿಲ್ಲ. ಅವರು ಅಷ್ಟು ಮುಂದುವರಿದು ಹೋಗಿದ್ದರು. ನಮ್ಮನ್ನೆಲ್ಲ ಆಳುವ ಸರ್ವೆಶ್ವರನು ಕೂಡ ಅವರಿಗೆ ತೃಪ್ತಿಯನ್ನು ಕೊಡಲಿಲ್ಲ. ಇವೆಲ್ಲ ಹೇಗೆ ಬಂದವು ಎಂಬುದನ್ನು ವಿವರಿಸುವ ಹಲವು ಮಂತ್ರಗಳು ಬರುವುವು. ಅವರು ಒಬ್ಬ ಸರ್ವೆಶ್ವರನಾದ ಸಗುಣ ಭಗವಂತನನ್ನು ಹುಡುಕಾಡುತ್ತಿದ್ದಾಗ, ಒಬ್ಬರಾದ ಮೇಲೆ ಒಬ್ಬ ದೇವರನ್ನು ಆ ಪೀಠಕ್ಕೆ ಏರಿಸಿದಂತೆ, ಹಲವು ಮಂತ್ರಗಳಲ್ಲಿ ಹಲವು ಭಾವನೆಗಳನ್ನು ಒಂದಾದ ಮೇಲೊಂದರಂತೆ ತೆಗೆದುಕೊಂಡು ಅದರ ಮೂಲಕ ಸರ್ವವನ್ನೂ ವಿವರಿಸಲು ಯತ್ನಿಸುವರು. ಒಂದು ಭಾವನೆಯನ್ನು ತೆಗೆದುಕೊಂಡು ಅದನ್ನೇ ಮುಖ್ಯವಾಗಿ ಮಾಡಿ, ಉಳಿದುವೆಲ್ಲ ಅದಕ್ಕೆ ಆಶ್ರಯವೆಂದು ವಿವರಿಸಲು ಯತ್ನಿಸುವರು. ಇದರಂತೆಯೇ ವಿವಿಧ ಭಾವನೆಗಳು ಕೂಡ. ಮೊದಲು ಪ್ರಾಣವನ್ನು ಇದಕ್ಕಾಗಿ ತೆಗೆದುಕೊಂಡರು. ಪ್ರಾಣಭಾವನೆಯನ್ನು ಅನಂತವಾಗಿ ಸರ್ವವ್ಯಾಪಿಯಾಗಿ ಮಾಡಿದರು. ಈ ಪ್ರಾಣವೇ ಎಲ್ಲಕ್ಕೂ ಆಶ್ರಯವಾಗಿರುವುದು. ಮನುಷ್ಯನ ದೇಹಕ್ಕೆ ಮಾತ್ರವಲ್ಲ. ಇದೇ ಸೂರ್ಯ ಚಂದ್ರರಲ್ಲಿರುವ ಬೆಳಕು, ಎಲ್ಲವನ್ನೂ ಚಲಿಸುತ್ತಿರುವಂತೆ ಮಾಡುತ್ತಿರುವುದೇ ಇದು, ಇದೇ ವಿಶ್ವದ ಕ್ರಿಯೋತ್ತೇಜಕ ಶಕ್ತಿ. ಇವುಗಳಲ್ಲಿ ಕೆಲವು ಪ್ರಯತ್ನಗಳು ಅತಿ ಸುಂದರವಾಗಿವೆ, ಕಾವ್ಯಮಯವಾಗಿವೆ. “ಅವನಿಂದಲೇ ಸುಂದರವಾದ ಉಷಸ್ಸು” ಎಂದು ವಿವರಿಸುವ ಮಂತ್ರ ಭಾವಗೀತಾತ್ಮಕವಾಗಿದೆ. ನಾವು ಈಗ ತಾನೇ ಓದಿದ ಈ ಆಸೆಯೇ ಸೃಷ್ಟಿಯ ಆಶಾಬೀಜವಾಯಿತು ಎಂಬುದನ್ನು ಸರ್ವೆಶ್ವರನ ಸ್ಥಾನಕ್ಕೆ ಒಯ್ದರು. ಆದರೆ ಇವಾವುವೂ ಅವರಿಗೆ ತೃಪ್ತಿಯನ್ನು ಕೊಡಲಿಲ್ಲ.

ಇಲ್ಲಿ ಆ ಭಾವನೆಯನ್ನು ದೈವೀಪೀಠಕ್ಕೆ ಏರಿಸಿ ಅದಕ್ಕೆ ಒಂದು ವ್ಯಕ್ತಿಯ ರೂಪವನ್ನು ಕೊಡುವರು. ಆದಿಯಲ್ಲಿ ಅವನೊಬ್ಬನೇ ಇದ್ದುದು, ಇರುವುದಕ್ಕೆಲ್ಲ ಏಕಮಾತ್ರ ಒಡೆಯನೇ ಅವನು. ಅವನೇ ಈ ವಿಶ್ವಕ್ಕೆ ಆಶ್ರಯ. ಅವನೇ ಆತ್ಮಗಳನ್ನು ಸೃಷ್ಟಿಸಿದವನು, ಅವನೇ ಬಲಕ್ಕೆ ಕಾರಣ; ಎಲ್ಲ ದೇವರುಗಳೂ ಪೂಜಿಸುವುದೇ ಅವನನ್ನು, ಅವನ ಛಾಯೆಗಳೇ ಜನನ ಮತ್ತು ಮರಣ. ನಾವು ಮತ್ತಾರನ್ನು ಪೂಜಿಸಬೇಕು? ಯಾರ ಮಹಿಮೆಯನ್ನು ಹಿಮಾಲಯದ ಧವಳ ಶಿಖರಗಳು ಸಾರುತ್ತಿವೆಯೋ ಯಾರ ಮಹಿಮೆಯನ್ನು ಪೃಥ್ವಿಯ ಸಾಗರದ ನೀರೆಲ್ಲ ಸಾರುತ್ತಿದೆಯೋ'' - ಹೀಗೆ ಅದು ಮುಂದುವರಿಯುವುದು. ಆದರೆ ನಾನು ಈಗ ತಾನೆ ಹೇಳಿದಂತೆ ಅವರಿಗೆ ಇದರಿಂದ ತೃಪ್ತಿಯಾಗಲಿಲ್ಲ.

ಕೊನೆಗೆ ನಾವೊಂದು ಅಪೂರ್ವ ಪರಿಸ್ಥಿತಿಗೆ ಬರುವೆವು. ಆರ್ಯರು ಇದುವರೆಗೆ ಉತ್ತರವನ್ನು ಬಾಹ್ಯಪ್ರಪಂಚದಲ್ಲಿ ಅರಸುತ್ತಿದ್ದರು. ತಮಗೆ ಕಂಡುದನ್ನೆಲ್ಲ ವಿಚಾರಿಸುತ್ತ ಹೋದರು. ಸೂರ್ಯ ಚಂದ್ರ ತಾರಕೆಗಳನ್ನೆಲ್ಲಾ ಪ್ರಶ್ನಿಸಿದರು. ಈ ರೀತಿ ಬರುವ ವಿಷಯಗಳನ್ನೆಲ್ಲ ಸಂಗ್ರಹಿಸಿದರು. ಪ್ರಕೃತಿ ಹೆಚ್ಚು ಎಂದರೆ ಒಬ್ಬ ಸಾಕಾರ ದೇವರನ್ನು ಮಾತ್ರ ಬೋಧಿಸಬಲ್ಲುದಾಗಿತ್ತು. ಅದಕ್ಕಿಂತ ಹೆಚ್ಚು ಏನೂ ಇಲ್ಲ. ಬಾಹ್ಯಪ್ರಪಂಚದ ಅನ್ವೇಷಣೆಯಲ್ಲಿ ಈ ಪ್ರಪಂಚಕ್ಕೆಲ್ಲ ಕಾರಣನಾದ ಒಬ್ಬ ಶಿಲ್ಪಿ ಮಾತ್ರ ನಮಗೆ ದೊರಕುವನು. ಇದನ್ನೇ ರಚನಾ ಸಿದ್ದಾಂತ ಎನ್ನುವುದು. ನಮಗೆಲ್ಲ ತಿಳಿದಿರುವಂತೆ ಇದೇನೂ ಅಷ್ಟು ಯುಕ್ತಿಪೂರಿತವಾದ ಸಿದ್ದಾಂತವಲ್ಲ. ಇದು ಕೇವಲ ಬಾಲಿಶವಾದುದು. ಆದರೂ ಬಾಹ್ಯ ಪ್ರಪಂಚದ ಅನ್ವೇಷಣೆಯಲ್ಲಿ ಭಗವಂತನ ವಿಷಯವಾಗಿ ನಮಗೆ ದೊರಕಬಹುದಾದ ಸ್ವಲ್ಪವೇ ಇದು. ಅದು ಈ ಸೃಷ್ಟಿಯ ಕರ್ತೃವು ಒಬ್ಬನಿರಬೇಕು ಎನ್ನುವುದು. ಆದರೆ ಇದರಿಂದ ವಿಶ್ವವನ್ನು ವಿವರಿಸಿದಂತೆ ಆಗಲಿಲ್ಲ. ಈ ವಿಶ್ವಕ್ಕೆ ಬೇಕಾದ ಸಾಮಾನೆಲ್ಲ ಅವನ ಮುಂದೆ ಇತ್ತು. ದೇವರಿಗೆ ಇದೆಲ್ಲ ಬೇಕಾಗಿತ್ತು. ಇಲ್ಲಿ ಬರುವ ದೊಡ್ಡ ಆಕ್ಷೇಪಣೆಯೇ ದೇವರು ತನ್ನ ಎದುರಿಗೆ ಇರುವ ಸಾಮಾನಿನಿಂದಲೇ ಸೃಷ್ಟಿಸಬೇಕಾಗಿತ್ತು, ಬೇರೆ ವಿಧಿಯೇ ಇರಲಿಲ್ಲ. ಮನೆ ಕಟ್ಟುವವನು ಯಾವ ಸಾಮಾನಿನಿಂದ ಮನೆಕಟ್ಟಬೇಕೊ ಅದಿಲ್ಲದೆ ಮನೆ ಕಟ್ಟುವಂತೆಯೇ ಇಲ್ಲ. ಆದಕಾರಣ ಅವನು ಈ ವಸ್ತುಗಳಿಗೆ ಅಧೀನನಾದನು. ಅಂದರೆ ದೇವರು ಈ ವಸ್ತುಗಳ ಮೂಲಕ ಯಾವುದನ್ನು ರಚಿಸಲು ಸಾಧ್ಯವೊ ಅದನ್ನು ಮಾತ್ರ ಮಾಡಬಲ್ಲವನಾಗಿದ್ದನು. ಸೃಷ್ಟಿರಚನೆಯ ಸಿದ್ದಾಂತದ ಮೂಲಕ ನಮಗೆ ದೊರಕುವ ದೇವರು ಹೆಚ್ಚು ಎಂದರೆ ವಾಸ್ತುಶಿಲ್ಪಿ, ಒಂದು ಮಿತಿಯಲ್ಲಿರುವ ಸೃಷ್ಟಿಯ ರಚನಾಕಾರ. ಅವನು ವಸ್ತುಗಳಿಗೆ ಅಧೀನ. ಅವನು ಸ್ವತಂತ್ರನಲ್ಲ. ಇಷ್ಟನ್ನು ಅವರು ಆಗಲೇ ಕಂಡು ಹಿಡಿದಿದ್ದರು. ಇತರರಾದರೆ ಇಷ್ಟರಲ್ಲೇ ತೃಪ್ತರಾಗಿ ನಿಲ್ಲುತ್ತಿದ್ದರು. ಇತರ ದೇಶಗಳಲ್ಲಿ ಇದೇ ಆಯಿತು. ಆದರೆ ಮಾನವನ ಮನಸ್ಸು ಇಷ್ಟಕ್ಕೆ ತೃಪ್ತಿಗೊಳ್ಳಲಿಲ್ಲ. ಆಲೋಚಿಸುವ ಮತ್ತು ಗ್ರಹಿಸುವ ಮನಸ್ಸು ಮುಂದೆ ಹೋಗಲು ಇಚ್ಛಿಸುತ್ತಿತ್ತು. ಆದರೂ ಯಾರು ಹಿಂದೆ ಉಳಿದುಕೊಂಡರೋ ಅವರು ಮುಂದೆ ಇರುವವರ ಬೆಳವಣಿಗೆಗೆ ಅವಕಾಶ ಕೊಡಲಿಲ್ಲ. ಅದೃಷ್ಟವಶದಿಂದ ಆರ್ಯ ಋಷಿಗಳನ್ನು ಯಾವುದೂ ತಡೆಯುವಂತೆ ಇರಲಿಲ್ಲ. ಅವರಿಗೆ ಒಂದು ಪರಿಹಾರಬೇಕಾಗಿತ್ತು. ಈಗ ಅವರ ಅನ್ವೇಷಣೆ ಬಾಹ್ಯವನ್ನು ತ್ಯಜಿಸಿ ಅಂತರ್ಮುಖವಾಯಿತು. ಮೊದಲು ಅವರು ಅರಿತದ್ದೇ, ಕಣ್ಣು ಮತ್ತು ಉಳಿದ ಇಂದ್ರಿಯಗಳಿಂದ ಅಲ್ಲ ನಾವು ಬಾಹ್ಯ ಪ್ರಪಂಚವನ್ನು ಗ್ರಹಿಸುವುದು ಮತ್ತು ಧಾರ್ಮಿಕ ವಿಷಯವನ್ನು ತಿಳಿಯುವುದು, ಎಂಬುದು. ಈ ವಿವರಣೆ ಸಾಲದು ಎಂಬುದೇ ಅವರಿಗೆ ಹೊಳೆದ ಮೊದಲನೆಯ ಭಾವನೆ. ಈ ಕುಂದು ನಾವು ನೋಡುವಂತೆ ಭೌತಿಕ ಮತ್ತು ನೈತಿಕವಾಗಿತ್ತು. ಈ ಪ್ರಪಂಚಕ್ಕೆ ಕಾರಣ ನಿಮಗೆ ಗೊತ್ತಿಲ್ಲ ಎನ್ನುವನು ಒಬ್ಬ ಋಷಿ. “ನಿನಗೂ ನನಗೂ ದೊಡ್ಡದೊಂದು ವ್ಯತ್ಯಾಸವಿದೆ. ಇದಕ್ಕೆ ಕಾರಣವೇನು? ಏಕೆಂದರೆ ನೀನು ಇಂದ್ರಿಯಕ್ಕೆ ಸಂಬಂಧಪಟ್ಟ ವಿಷಯಗಳನ್ನು ಕುರಿತು ಮಾತನಾಡುತ್ತಿರುವೆ; ಇಂದ್ರಿಯಕ್ಕೆ ಸಂಬಂಧಪಟ್ಟ ವಿಷಯಗಳಲ್ಲಿ, ಕೇವಲ ಧರ್ಮದ ಬಾಹ್ಯಾಚಾರದಲ್ಲಿ ನೀನು ತೃಪ್ತನಾಗಿರುವೆ; ಆದರೆ ಇದಕ್ಕೆ ಅತೀತವಾಗಿರುವ ಪುರುಷ ತತ್ತ್ವವು ನನಗೆ ಗೊತ್ತಿದೆ.'

- ಆಧ್ಯಾತ್ಮಿಕ ಭಾವನೆಗಳು ಹೇಗೆ ಬೆಳೆಯುತ್ತ ಹೋದವು ಎಂಬುದನ್ನು ನಿಮ್ಮ ಮುಂದೆ ವಿವರಿಸುತ್ತಿರುವಾಗ, ಇದರೊಡನೆ ಬೆಳೆದ ಮತ್ತೊಂದು ಭಾವನೆಯ ವಿಷಯವಾಗಿ ಸ್ವಲ್ಪ ಸೂಚನೆಯನ್ನು ಮಾತ್ರ ಕೊಡುತ್ತೇನೆ. ಏಕೆಂದರೆ ನಾನು ಇಂದು ವಿವರಿಸುತ್ತಿರುವ ವಿಷಯಕ್ಕೂ ಅದಕ್ಕೂ ಏನೂ ಸಂಬಂಧವಿಲ್ಲ. ಆದ್ದರಿಂದ ಅದರ ವಿಷಯ ಹೆಚ್ಚು ಹೇಳಬೇಕಾಗಿಲ್ಲ. ಅದು ಕ್ರಿಯಾವಿಧಿಗಳ ವಿಷಯ. ಆಧ್ಯಾತ್ಮಿಕಭಾವನೆ ಒಂದು ವೇಗದಲ್ಲಿ ಬೆಳೆಯುತ್ತ ಹೋದರೆ ಅದಕ್ಕಿಂತ ಹೆಚ್ಚಿನ ವೇಗದಲ್ಲಿ ಕ್ರಿಯಾವಿಧಿಗಳ ವಿಷಯಗಳು ಬೆಳೆಯುತ್ತ ಹೋದುವು. ಹಳೆಯ ಮೂಢನಂಬಿಕೆಗಳು ಇಷ್ಟು ಹೊತ್ತಿಗೆ ಬೆಳೆದು ಬೆಳೆದು ಅದೊಂದು ದೊಡ್ಡ ಕ್ರಿಯಾವಿಧಿಗಳ ಮೊತ್ತವಾಗಿತ್ತು; ಹಿಂದೂ ಜೀವನವನ್ನು ನಿರ್ಮೂಲಮಾಡುವ ಸ್ಥಿತಿಯಲ್ಲಿತ್ತು. ಅದು ಈಗಲೂ ಇದೆ. ಅದು ನಮ್ಮನ್ನು ಮೆಟ್ಟಿಕೊಂಡು ಜೀವನದ ಪ್ರತಿಯೊಂದು ಕಾರ್ಯಕ್ಷೇತ್ರಕ್ಕೂ ಧಾಳಿಯಿಟ್ಟು ನಮ್ಮನ್ನು ಹುಟ್ಟು ಗುಲಾಮರನ್ನಾಗಿ ಮಾಡಿದೆ. ಆದರೂ ಬಹಳ ಪೂರ್ವದಲ್ಲೇ ಇದನ್ನು ವಿರೋಧಿಸುತ್ತಿದ್ದವರೂ ಇದ್ದರು. ಅವರ ಒಂದು ಆಕ್ಷೇಪಣೆಯೇ ಇದು: ಈ ಬಾಹ್ಯಾಚಾರದ ವ್ಯಾಮೋಹ - ಒಂದು ರೀತಿ ಉಡಿಗೆ ತೊಡುವುದು, ಒಂದು ರೀತಿ ಊಟ ಮಾಡುವುದು ಇವೆಲ್ಲ ಕೇವಲ ಬಾಹ್ಯಾಡಂಬರಗಳು. ಏಕೆಂದರೆ ನೀವು ಕೆಲವು ವಿಷಯವಸ್ತುಗಳಲ್ಲಿಯೇ ತೃಪ್ತರಾಗಿರುವುರಿಂದ ಮುಂದುವರಿಯಲು ನಿಮಗೆ ತವಕವೇ ಇರುವುದಿಲ್ಲ. ಇದೊಂದು ನಮ್ಮಲ್ಲಿ ಮತ್ತು ಪ್ರತಿಯೊಬ್ಬರಲ್ಲಿ ಇರುವ ದೊಡ್ಡ ಕಷ್ಟ. ಹೆಚ್ಚು ಎಂದರೆ ನಾವು ಆಧ್ಯಾತ್ಮಿಕ ವಿಷಯಗಳನ್ನು ಕೇಳುವಾಗ ಇಂದ್ರಿಯಗಳು ನಮ್ಮ ಪ್ರಮಾಣ ಆಗುವುವು. ಇಲ್ಲವೆ ಹಲವು ದಿನಗಳ ಕಾಲ ಒಬ್ಬನು ತತ್ತ್ವ, ದೇವರು, ಅತೀಂದ್ರಿಯ ಮುಂತಾದುವುಗಳನ್ನು ಕೇಳಿ ಆದ ಮೇಲೆ ಇದರಿಂದ ಎಷ್ಟು ಹಣ ಸಿಕ್ಕುವುದು, ಇದರಿಂದ ಎಷ್ಟು ವಿಷಯಸುಖ ದೊರಕುವುದು ಎಂದು ಪ್ರಶ್ನಿಸುವನು. ಸ್ವಾಭಾವಿಕವಾಗಿ ಅವನ ಆನಂದವೆಲ್ಲ ಇಂದ್ರಿಯಗಳಲ್ಲಿ ಇರುವುದು. ಆದರೆ ಋಷಿಗಳು ಈ ಇಂದ್ರಿಯ ತೃಪ್ತಿಯೇ ಸತ್ಯವು ನಮಗೆ ಕಾಣದಂತೆ ಮಾಡಿರುವುದು ಎಂದು ಹೇಳುವರು. ಬಾಹ್ಯಾಚಾರ, ಇಂದ್ರಿಯ ತೃಪ್ತಿ ಮತ್ತು ಹಲವು ಕಾಲ್ಪನಿಕ ಸಿದ್ಧಾಂತಗಳನ್ನು ನೇಯುವುದು ಇವೇ ಸತ್ಯವು ನಮಗೆ ಕಾಣದಂತೆ ಮರೆಮಾಡಿರುವ ತೆರೆ. ಇದೊಂದು ಅತಿ ಮುಖ್ಯ ವಿಷಯ. ಈ ಆದರ್ಶವನ್ನು ಕೊನೆಯವರೆಗೂ ವಿಮರ್ಶಿಸಿ, ಇದರಿಂದ ವೇದಾಂತದ ಅದ್ಭುತ ಮಾಯಾಸಿದ್ಧಾಂತ ಹೇಗೆ ಆಯಿತು ಎಂಬುದನ್ನು ನೋಡಬೇಕಾಗಿದೆ. ಸತ್ಯ ಯಾವಾಗಲೂ ಅಲ್ಲಿತ್ತು, ಆದರೆ ಈ ಮಾಯೆಯ ತೆರೆ ಅದನ್ನು ಮುಚ್ಚಿತ್ತು ಎಂಬುದೇ ಸರಿಯಾದ ವೇದಾಂತದ ವಿವರಣೆ.

ಪುರಾತನ ಆರ್ಯಧೀಮಂತರು ಹೊಸದೊಂದು ಅಧ್ಯಾಯವನ್ನು ಪ್ರಾರಂಭಿಸಿದರು. ಬಾಹ್ಯಪ್ರಪಂಚದ ಯಾವ ಅನ್ವೇಷಣೆಯೂ ಅವರ ಪ್ರಶ್ನೆಗೆ ಉತ್ತರ ಕೊಡಲಾರದು ಎಂಬುದನ್ನು ಅವರು ಅರಿತರು. ತಲೆತಲಾಂತರಗಳ ತನಕ ಇಲ್ಲಿ ಹುಡಕಾಡಬಹುದು, ಆದರೆ ಅವರ ಪ್ರಶ್ನೆಗೆ ಉತ್ತರ ದೊರಕಲಾರದು. ಆದಕಾರಣವೇ ಅವರು ಬೇರೊಂದು ಮಾರ್ಗಕ್ಕೆ ಕೈಹಾಕಿದರು. ಇಂದ್ರಿಯಾಭಿಲಾಷೆ, ಕ್ರಿಯಾವಿಧಿಗಳು, ಇವು ತಮಗೂ ಸತ್ಯಕ್ಕೂ ಮಧ್ಯೆ ಒಂದು ತೆರೆಯನ್ನು ತಂದೊಡ್ಡಿವೆ ಎಂಬುದನ್ನು ಈ ಮಾರ್ಗದಲ್ಲಿ ಕಲಿತರು. ಇದನ್ನು ಕೇವಲ ಕ್ರಿಯಾವಿಧಿಗಳಿಂದಲೂ ನಿವಾರಿಸಲಾಗುವುದಿಲ್ಲ. ಅವರು ತಮ್ಮ ಮನಸ್ಸನ್ನೇ ಗಮನಿಸಬೇಕು. ಸತ್ಯವನ್ನು ತಮ್ಮಲ್ಲಿಯೇ ಅರಿಯಬೇಕಾದರೆ ಮನಸ್ಸನ್ನು ವಿಶ್ಲೇಷಿಸಬೇಕು. ಬಾಹ್ಯಪ್ರಪಂಚದಿಂದ ಏನೂ ಪ್ರಯೋಜನವಾಗಲಿಲ್ಲ. ಅನಂತರ ಅವರು ಅಂತರ್ಮುಖಿಗಳಾದರು. ಆಗಲೇ ನಿಜವಾದ ವೇದಾಂತ ಪ್ರಾರಂಭವಾಗುವುದು. ಇದೇ ವೇದಾಂತ ತತ್ತ್ವಕ್ಕೆ ಮೂಲಾಧಾರ. ನಾವು ಮುಂದುವರಿದಂತೆಲ್ಲ ವಿಚಾರವೆಲ್ಲ ಅಂತರ್ಮುಖ ಜಗತ್ತಿಗೆ ಸಂಬಂಧ ಪಟ್ಟುದೆಂಬುದನ್ನು ನೋಡುವೆವು. ಆದಿಯಲ್ಲಿಯೇ ಅದು "ಯಾವ ಧರ್ಮದಲ್ಲಿಯೂ ಸತ್ಯವನ್ನು ಹುಡುಕಲು ಯತ್ನಿಸಬೇಡಿ. ಅದು ಇಲ್ಲಿದೆ, ಮಾನವನ ಅಂತರಾತ್ಮನಲ್ಲಿ. ಮಾನವನ ಆತ್ಮವೇ ಅತ್ಯಂತ ವಿಸ್ಮಯಕಾರಿಯಾದುದು, ಇದೇ ಎಲ್ಲಾ ಜ್ಞಾನ ನಿಧಿ, ಎಲ್ಲಾ ಜೀವನದ ಮೂಲ; ಸತ್ಯವನ್ನು ಇಲ್ಲಿ ಅರಸಿ'' ಎಂದು ಹೇಳುವಂತೆ ಇತ್ತು. ಯಾವುದು ಇಲ್ಲಿ ಇಲ್ಲವೋ ಅದು ಅಲ್ಲಿ ಇರಲಾರದು. ಯಾವುದು ಬಾಹ್ಯದಲ್ಲಿದೆಯೋ ಅದು ಹೆಚ್ಚು ಎಂದರೆ ಆಂತರ್ಯದಲ್ಲಿರುವುದರ ಒಂದು ಅಸ್ಪಷ್ಟ ಪ್ರತಿಬಿಂಬ ಎಂಬುದನ್ನು ಕ್ರಮೇಣ ಕಂಡುಹಿಡಿದರು. ವಿಶ್ವಕ್ಕೆ ಬಾಹ್ಯವಾಗಿರುವ ಸರ್ವೇಶ್ವರ ಎಂಬ ಹಳೆಯ ಭಾವನೆಯನ್ನು ಮೊದಲು ಸ್ವೀಕರಿಸಿ ಅದನ್ನು ವಿಶ್ವದ ಒಳಗೆ ಇಟ್ಟರು. ಅವನು ಬಾಹ್ಯದಲ್ಲಿರುವ ದೇವರಲ್ಲ, ಅವನು ವಿಶ್ವಾತ್ಮ ಎಂದರು. ಅನಂತರ ಅವನನ್ನು ತಮ್ಮ ಆತ್ಮನೊಳಗೆ ಸೇರಿಸಿದರು. ಅವನು ಇಲ್ಲಿ ಅಂತರ್ಯಾಮಿಯಾಗಿರುವನು, ನಮ್ಮ ಆತ್ಮನ ಆತ್ಮನಾಗಿರುವನು, ನಮ್ಮಲ್ಲಿರುವ ಸತ್ಯವಾಗಿರುವನು.

ವೇದಾಂತ ತತ್ತ್ವವನ್ನು ಚೆನ್ನಾಗಿ ಗ್ರಹಿಸಬೇಕಾದರೆ ಹಲವು ಮುಖ್ಯ ಭಾವನೆಗಳನ್ನು ಅರಿಯಬೇಕು. ಮೊದಲನೆಯದಾಗಿ ನಾವು ಯಾವುದನ್ನು ಕಾಂಟ್‌ನ ಅಥವಾ ಹೆಗೆಲ್‌ನ ಸಿದ್ದಾಂತ ಎನ್ನುವೆವೊ ಹಾಗೆ ಇದೊಂದು ಸಿದ್ಧಾಂತವಲ್ಲ. ಇದು ಒಂದು ಗ್ರಂಥವೂ ಅಲ್ಲ, ಒಬ್ಬ ಕರ್ತೃವಿನದೂ ಅಲ್ಲ. ವೇದಾಂತವೆಂದರೆ ಬೇರೆ ಬೇರೆ ಕಾಲಗಳಲ್ಲಿ ಬರೆದ ಹಲವು ಗ್ರಂಥಗಳು. ಕೆಲವು ವೇಳೆ ಒಂದು ಗ್ರಂಥದಲ್ಲಿ ಸುಮಾರು ಐವತ್ತು ಭಿನ್ನ ಭಿನ್ನ ಭಾವನೆಗಳು ಬರುತ್ತವೆ. ಅವನ್ನು ಸರಿಯಾಗಿ ಜೋಡಿಸಿಯೂ ಇರುವುದಿಲ್ಲ, ಸುಮ್ಮನೆ ಆಲೋಚನೆಗಳನ್ನು ಹಾಗೆಯೇ ಗುರುತು ಮಾಡಿ ಇಟ್ಟಂತೆ ಇರುತ್ತದೆ. ಕೆಲವು ವೇಳೆ ಮತ್ಯಾವುವೋ ಅನಾವಶ್ಯಕವಾದ ವಿಷಯಗಳ ಮಧ್ಯೆ ಒಂದು ಅದ್ಭುತ ಭಾವನೆ ಕಾಣುವುದು. ಆದರೆ ಅದರಲ್ಲಿ ಈ ಒಂದು ವಿಷಯ ಮುಖ್ಯವಾಗಿದೆ: ಅದೇನೆಂದರೆ ಉಪನಿಷತ್ತಿನಲ್ಲಿ ಬರುವ ಭಾವನೆ ಯಾವಾಗಲೂ ಪ್ರಗತಿಪರವಾಗಿರುವುದು. ಆ ಹಳೆಯ ಒರಟು ಭಾಷೆಯಲ್ಲಿ ಪ್ರತಿಯೊಬ್ಬ ಋಷಿಯ ಮನಸ್ಸು ಹೇಗೆ ಬೆಳೆಯುತ್ತ ಹೋಯಿತು. ಎಂಬುದನ್ನು ಚಿತ್ರಿಸಿರುವಂತಿದೆ. ಭಾವನೆಗಳು ಮೊದಮೊದಲು ಒರಟಾಗಿದ್ದು, ಕ್ರಮೇಣ ಅವು ಸೂಕ್ಷ್ಮವಾಗುತ್ತ ಬಂದು ವೇದಾಂತದ ಗುರಿ ಸೇರುವುದನ್ನು ವಿವರವಾಗಿ ನಿರೂಪಿಸಿರುವರು. ಕೊನೆಗೆ ಈ ಗುರಿ ಒಂದು ದರ್ಶನದ ಹೆಸರನ್ನು ಪಡೆಯುವುದು. ಪ್ರಾರಂಭದಲ್ಲಿ ಇದು ದೇವತೆಗಳ ಅನ್ವೇಷಣೆ, ಅನಂತರ ಸೃಷ್ಟಿ ಹೇಗೆ ಪ್ರಾರಂಭವಾಯಿತು ಎಂಬ ಪ್ರಶ್ನೆ, ಅನಂತರ ಈ ಪ್ರಶ್ನೆಯೇ ಹೆಚ್ಚು ತಾತ್ತ್ವಿಕವಾಗಿ ಸ್ಪಷ್ಟವಾಗಿ "ಯಾವುದನ್ನು ಅರಿತರೆ ನಾವು ಎಲ್ಲವನ್ನು ಅರಿಯಬಹುದೊ” ಅಂತಹ ಒಂದು ಏಕತೆಗೆ ನಮ್ಮನ್ನು ಒಯ್ಯುವುದು.

