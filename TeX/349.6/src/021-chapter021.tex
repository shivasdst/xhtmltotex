
\chapter[ವೇದಾಂತವು ಭವಿಷ್ಯದ ಧರ್ಮವಾಗಬಲ್ಲದೆ?]{ವೇದಾಂತವು ಭವಿಷ್ಯದ ಧರ್ಮವಾಗಬಲ್ಲದೆ?\protect\footnote{\engfoot{C.W, Vol. VIII, P. 122}}}

\begin{center}
(೧೯೦೦ ರ ಏಪ್ರಿಲ್ ೮ರಂದು ಸ್ಯಾನ್‌ಫ್ರಾನ್ಸಿಸ್ಕೋದಲ್ಲಿ ನೀಡಿದ ಪ್ರವಚನ.)
\end{center}

ಯಾರು ಕಳೆದ ಒಂದು ತಿಂಗಳಿನಿಂದ ನನ್ನ ಉಪನ್ಯಾಸವನ್ನು ಕೇಳುತ್ತಿದ್ದೀರೋ ಅವರಿಗೆಲ್ಲ ಇಷ್ಟುಹೊತ್ತಿಗೆ ವೇದಾಂತದ ಭಾವನೆಗಳ ಪರಿಚಯವಾಗಿರಬಹುದು. ಜಗತ್ತಿನ ಅತಿ ಪುರಾತನ ಧರ್ಮವೇ ವೇದಾಂತ. ಆದರೆ ಇದು ಜನಸಾಮಾನ್ಯರಿಗೆ ಅರ್ಥವಾಗಿದೆ ಎಂದು ಹೇಳಲಾಗುವುದಿಲ್ಲ. ಆದಕಾರಣ ಇದು ಭವಿಷ್ಯದ ಧರ್ಮವಾಗಬಲ್ಲದೆ ಎಂಬ ಪ್ರಶ್ನೆಗೆ ಉತ್ತರ ಕೊಡುವುದು ಬಹಳ ಕಷ್ಟ.

ಪ್ರಾರಂಭದಲ್ಲಿ ಇದು ವಿಶ್ವದ ಬಹುಪಾಲು ಜನರಿಗೆ ಒಂದು ಧರ್ಮವಾಗಬಲ್ಲದೋ ಇಲ್ಲವೋ ಎಂಬುದು ನನಗೆ ಗೊತ್ತಿಲ್ಲ ಎಂದು ಹೇಳಬೇಕಾಗಿದೆ. ಅಮೆರಿಕಾದ ಸಂಯುಕ್ತ ಸಂಸ್ಥಾನಗಳಂತಹ ಇಡೀ ರಾಷ್ಟ್ರವನ್ನು ಇದು ಆಕ್ರಮಿಸಬಲ್ಲುದೇ? ಸಾಧ್ಯವಾಗಬಹುದು. ನಾವಿಂದು ಚರ್ಚಿಸಬೇಕೆಂದಿರುವ ಪ್ರಶ್ನೆಯೇ ಇದು.

ಯಾವುದು ವೇದಾಂತವಲ್ಲ ಎಂಬುದನ್ನು ನಿಮಗೆ ಹೇಳಿ ಅನಂತರ ವೇದಾಂತ ಎಂದರೆ ಏನು ಎಂಬುದನ್ನು ಹೇಳುತ್ತೇನೆ. ವೇದಾಂತವು ವ್ಯಕ್ತಿಗತವಲ್ಲದ ಸಿದ್ದಾಂತಗಳಿಗೆ ಹೆಚ್ಚು ಪ್ರಾಮುಖ್ಯವನ್ನು ಕೊಟ್ಟರೂ, ಅದು ಯಾವುದನ್ನೂ ವಿರೋಧಿಸುವುದಿಲ್ಲ. ಆದರೂ ಅದು ತಾನು ಯಾವುದನ್ನು ಮೂಲಭೂತ ಸತ್ಯವೆಂದು ತಿಳಿಯುವುದೊ ಅದನ್ನು ತೊರೆಯಲಾರದು ಅಥವಾ ಇತರರೊಂದಿಗೆ ಸಂಧಿ ಮಾಡಿಕೊಳ್ಳಲಾರದು.

ಒಂದು ಧರ್ಮಕ್ಕೆ ಕೆಲವು ವಿಷಯಗಳು ಅತ್ಯಗತ್ಯ ಎಂಬುದು ನಿಮಗೆಲ್ಲ ಗೊತ್ತಿದೆ. ಮೊದಲನೆಯದೆ ಶಾಸ್ತ್ರ. ಈ ಶಾಸ್ತ್ರದ ಶಕ್ತಿ ಅದ್ಭುತವಾದುದು. ಅದು ಏನಾದರೂ ಆಗಿರಲಿ, ಮಾನವರೆಲ್ಲ ಒಟ್ಟು ಕಲೆತು ಗೌರವ ತೋರಬೇಕಾದರೆ ಅದು ಶಾಸ್ತ್ರಕ್ಕೆ ಮಾತ್ರ. ಶಾಸ್ತ್ರವಿಲ್ಲದ ಯಾವ ಧರ್ಮವೂ ಇಂದು ಉಳಿದಿಲ್ಲ. ಮಾನವಕೋಟಿ ಯುಕ್ತಿಯ ವಿಷಯವಾಗಿ ಬೇಕಾದಷ್ಟು ಮಾತನಾಡಿದರೂ ಅವರೆಲ್ಲ ಶಾಸ್ತ್ರವನ್ನು ಅಪ್ಪಿಕೊಂಡಿರುವರು. ನಿಮ್ಮ ದೇಶದಲ್ಲಿ ಶಾಸ್ತ್ರವಿಲ್ಲದೆ ಒಂದು ಹೊಸ ಧರ್ಮವನ್ನು ಜಾರಿಗೆ ತರಬೇಕೆಂಬ ಪ್ರಯತ್ನವೆಲ್ಲ ನಿಷ್ಪಲವಾಗಿದೆ. ಭರತಖಂಡದಲ್ಲಿ ಬಹಳ ಬೇಗ ಮತಗಳು ಏಳುವುವು. ಆದರೆ ಕೆಲವು ವರುಷಗಳಲ್ಲಿ ಅವು ಅಳಿದುಹೋಗುವುವು. ಏಕೆಂದರೆ ಅವಕ್ಕೆ ಒಂದು ಶಾಸ್ತ್ರವಿಲ್ಲ. ಇದರಂತೆಯೇ ಇತರ ದೇಶಗಳಲ್ಲಿ ಕೂಡ.

ಯೂನಿಟೆರಿಯನ್ ಚಳುವಳಿಯ ಉಗಮ ಮತ್ತು ಅವನತಿಗಳನ್ನು ಕುರಿತು\break ಆಲೋಚಿಸಿ ನೋಡಿ. ನಿಮ್ಮ ದೇಶದ ಶ್ರೇಷ್ಠ ಭಾವನೆಗಳು ಅಲ್ಲಿವೆ. ಅವರೇಕೆ ಮೆಥಾಡಿಸ್ಟ್, ಬ್ಯಾಪ್ಟಿಸ್ಟ್ ಮತ್ತು ಇತರ ಕ್ರೈಸ್ತ ಪಂಗಡಗಳಂತೆ ಹರಡಲಿಲ್ಲ? ಏಕೆಂದರೆ ಅವರಿಗೆ ಒಂದು ಶಾಸ್ತ್ರವಿರಲಿಲ್ಲ. ಆದರೆ ಯೆಹೂದ್ಯರನ್ನು ನೋಡಿ! ಅವರು ಎಲ್ಲೋ ಸ್ವಲ್ಪ ಜನರು; ದೇಶದಿಂದ ದೇಶಕ್ಕೆ ಅವರನ್ನು ಅಟ್ಟುತ್ತಿರುವರು. ಆದರೂ ಅವರೆಲ್ಲ ಒಟ್ಟಿಗೆ ಇರುವರು. ಏಕೆಂದರೆ ಅವರಿಗೆ ಒಂದು ಶಾಸ್ತ್ರವಿದೆ. ಪಾರ್ಸಿಗಳನ್ನು ನೋಡಿ. ಅವರೆಲ್ಲ ಪ್ರಪಂಚದಲ್ಲಿ ಎಲ್ಲೋ ಒಂದು ಲಕ್ಷ ಜನರು. ಭರತಖಂಡದಲ್ಲಿರುವ ಜೈನರ ಸಂಖ್ಯೆಯಾದರೂ ಎಲ್ಲೊ ಹತ್ತು ಲಕ್ಷಗಳು. ಆದರೆ ಈ ಅಲ್ಪ ಪಾರ್ಸಿಗಳ ಮತ್ತು ಜೈನರ ತಂಡಗಳು ಇನ್ನೂ ಇರುವುವು: ಏಕೆಂದರೆ ಅವರಿಗೆ ಒಂದು ಶಾಸ್ತ್ರವಿದೆ. ಇಂದು ಜೀವಂತವಾಗಿರುವ ಧರ್ಮಗಳಿಗೆಲ್ಲ ಶಾಸ್ತ್ರವಿದೆ.

ಧರ್ಮಕ್ಕೆ ಬೇಕಾದ ಎರಡನೆಯ ಆವಶ್ಯಕತೆಯೆ ಗೌರವ ತೋರುವುದಕ್ಕೆ\break ಯಾರಾದರೂ ಇರಬೇಕು ಎಂಬುದು. ಅವನನ್ನು ದೇವರೆಂದೂ ಮಹಾಗುರುವೆಂದೂ ಆರಾಧಿಸುತ್ತಾರೆ. ಮನುಷ್ಯರು ಯಾರಾದರೂ ಮನುಷ್ಯರನ್ನು ಆರಾಧಿಸಬೇಕಾಗಿದೆ. ಮಾನವರಿಗೆ ಅವತಾರವೋ, ದೇವದೂತನೋ, ಮಹಾನಾಯಕನೋ ಬೇಕಾಗಿದೆ. ಇಂದು ನೀವು ಇದನ್ನು ಎಲ್ಲಾ ಧರ್ಮಗಳಲ್ಲಿಯೂ ನೋಡುತ್ತೀರಿ. ಹಿಂದೂಗಳಲ್ಲಿ ಮತ್ತು ಕ್ರೈಸ್ತರಲ್ಲಿ ಅವತಾರಗಳಿವೆ. ಬೌದ್ಧರಲ್ಲಿ, ಮಹಮ್ಮದೀಯ ಮತ್ತು ಯೆಹೂದ್ಯರಲ್ಲಿ ದೇವದೂತರಿರುವರು. ಆದರೆ ಇದರ ಹಿಂದೆ ಇರುವ ಭಾವನೆ ಒಂದೆ. ಅದೇ ಯಾವುದಾದರೂ ವ್ಯಕ್ತಿಗೆ ತಮ್ಮ ಗೌರವವನ್ನು ಅರ್ಪಿಸುವುದು.

ಮೂರನೆಯ ಆವಶ್ಯಕತೆಯೆ, ಧರ್ಮ ಬಲಿಷ್ಠವಾಗಿರಬೇಕಾದರೆ, ತನ್ನಲ್ಲಿ ಒಂದು ಭರವಸೆ ಬರಬೇಕಾದರೆ, ತಾನು ಹೇಳುವುದು ಮಾತ್ರ ಸತ್ಯ ಎಂದು ನಂಬುವುದು. ಇಲ್ಲದೆ ಇದ್ದರೆ ಅದು ಇತರರ ಮೇಲೆ ತನ್ನ ಪ್ರಭಾವವನ್ನು ಬೀರಲಾರದು.

ಔದಾರ್ಯ ಬೇಗ ತಣ್ಣಗಾಗುವುದು. ಅದು ನಿಸ್ಸಾರ; ಏಕೆಂದರೆ ಅದು ಹೃದಯದಲ್ಲಿ ತಮ್ಮ ಮತವೇ ಶ್ರೇಷ್ಠ ಎಂಬ ಕೆಚ್ಚನ್ನು ಹುಟ್ಟಿಸಲಾರದು, ಇತರರ ಮೇಲೆ ದ್ವೇಷವನ್ನು ಎಬ್ಬಿಸಲಾರದು. ಆದಕಾರಣವೆ ಔದಾರ್ಯ ಪದೇ ಪದೇ ಹಿಂಬದಿಗೆ ಸರಿಯುವುದು. ಅದೆಲ್ಲೋ ಕೆಲವರ ಮೇಲೆ ತನ್ನ ಪ್ರಭಾವವನ್ನು ಬೀರಬಲ್ಲದು. ಇದಕ್ಕೆ ಕಾರಣ ಗೊತ್ತೇ ಇದೆ. ಔದಾರ್ಯ ನಮ್ಮನ್ನು ನಿಃಸ್ವಾರ್ಥರನ್ನಾಗಿ ಮಾಡಲು ಯತ್ನಿಸುವುದು. ಆದರೆ ನಮಗೆ ನಿಃಸ್ವಾರ್ಥರಾಗಲು ಇಚ್ಛೆಯಿಲ್ಲ. ನಿಃಸ್ವಾರ್ಥತೆಯಿಂದ ಸದ್ಯಕ್ಕೆ ಏನೂ ಪ್ರಯೋಜನ ಕಾಣುವುದಿಲ್ಲ. ಸ್ವಾರ್ಥಿಯಾದರೆಯೇ ಹೆಚ್ಚು ಪ್ರಯೋಜನ. ನಾವು ಬಡವರಾಗಿರುವ ತನಕ, ಏನೂ ಇಲ್ಲದಾಗ, ಔದಾರ್ಯವನ್ನು ಒಪ್ಪಿಕೊಳ್ಳುತ್ತೇವೆ. ಆದರೆ ಅಧಿಕಾರ, ಐಶ್ವರ್ಯ ಬಂದೊಡನೆಯೆ ನಾವು ಸಂಪ್ರದಾಯಬದ್ದರಾಗುವೆವು. ಬಡವ ಪ್ರಜಾಪ್ರಭುತ್ವವಾದಿ. ಅವನು ಹಣವಂತನಾದ ಮೇಲೆ ಶ‍್ರೀಮಂತ ಪ್ರಭುತ್ವದವನಾಗುತ್ತಾನೆ. ಧರ್ಮದಲ್ಲಿ ಕೂಡ ಮಾನವ ಸ್ವಭಾವ ಹೀಗೆಯೆ.

ಒಬ್ಬ ಪ್ರವಾದಿ ಉದಯಿಸುವನು. ಯಾರು ತನ್ನನ್ನು ಅನುಸರಿಸುವರೋ ಅವರಿಗೆ ಎಲ್ಲಾ ಬಗೆಯ ಬಹುಮಾನಗಳೂ ದೊರಕುತ್ತವೆ ಎಂದೂ, ಯಾರು ತನ್ನನ್ನು ಅನುಸರಿಸುವುದಿಲ್ಲವೋ ಅವರಿಗೆ ನಿತ್ಯ ನರಕ ಪ್ರಾಪ್ತಿ ಎಂದೂ ಸಾರುವನು. ಅವನ ಭಾವನೆ ಹರಡುವುದು ಹೀಗೆ ಹರಡುತ್ತಿರುವ ಧರ್ಮಗಳಲ್ಲೆಲ್ಲಾ ಮಂತಾಂಧತೆ ಬಹಳ ಉಗ್ರವಾಗಿದೆ. ಒಂದು ಪಂಗಡ ಇತರ ಪಂಗಡಗಳನ್ನು ಎಷ್ಟು ಉಗ್ರವಾಗಿ ದ್ವೇಷಿಸಿದರೆ ಅಷ್ಟೂ\break ಜಯಶಾಲಿಯಾಗುವುದು, ಅಷ್ಟೂ ಹೆಚ್ಚು ಜನರು ಅದರ ಗುಂಪಿಗೆ ಬರುವರು. ನಾನು ಪ್ರಪಂಚವನ್ನು ಬೇಕಾದಷ್ಟು ಸುತ್ತಿದ ಮೇಲೆ, ಹಲವು ಜನಾಂಗಗಳೊಂದಿಗೆ ವಾದಿಸಿದ ಮೇಲೆ, ಪ್ರಪಂಚದ ಸ್ಥಿತಿಗತಿಗಳನ್ನು ಕೂಲಂಕಷವಾಗಿ ವಿಚಾರಿಸಿದ ಮೇಲೆ, ನಾವು ಈಗ ಎಷ್ಟೇ ವಿಶ್ವ ಸಹೋದರತ್ವದ ವಿಷಯವನ್ನು ಮಾತಾನಾಡಿದರೂ ಸ್ಥಿತಿ ಈಗಿರುವಂತೆಯೇ ಬಹಳ ಕಾಲ ಮುಂದುವರಿಯುವುದು ಎಂಬುದು ನನ್ನ ಅಭಿಪ್ರಾಯ.

ವೇದಾಂತವು ಇದಾವುದನ್ನೂ ನಂಬುವುದಿಲ್ಲ. ಅದು ಯಾವ ಶಾಸ್ತ್ರವನ್ನೂ ನಂಬುವುದಿಲ್ಲ. ಅದೇ ಮೊದಲಿನ ತೊಡಕು. ಯಾವ ಒಂದು ಶಾಸ್ತ್ರವೂ ಮತ್ತೊಂದು ಶಾಸ್ತ್ರಕ್ಕಿಂತ ಮೇಲು ಎಂದು ಅದು ಹೇಳುವುದಿಲ್ಲ. ಯಾವುದೋ ಒಂದು ಗ್ರಂಥದಲ್ಲಿ ಮಾತ್ರ ದೇವರು, ಜೀವ ಮತ್ತು ಪ್ರಪಂಚ ಇವುಗಳ ರಹಸ್ಯತಮ ವಿಷಯ ಅಡಗಿದೆ ಎಂದು ಅದು ನಂಬುವುದಿಲ್ಲ. ಯಾರು ಉಪನಿಷತ್ತನ್ನು ಓದಿರುವರೋ ಅವರು, “ಶಾಸ್ತ್ರಾಧ್ಯಯನದಿಂದ ಆತ್ಮ ಸಾಕ್ಷಾತ್ಕಾರವಾಗುವುದಿಲ್ಲ” ಎಂದು ಮತ್ತೆ ಮತ್ತೆ ಸಾರುವುದನ್ನು ಗಮನಿಸಿರಬಹುದು.

ಎರಡನೆಯದಾಗಿ ಯಾರೋ ಒಬ್ಬರಿಗೆ ನಾವು ವಿಶೇಷ ಗೌರವವನ್ನು ತೋರಬೇಕೆಂಬುದನ್ನು ಸಮರ್ಥಿಸಲು ಅದಕ್ಕೆ ಆಗುವುದಿಲ್ಲ. ಯಾರು ನಿಮ್ಮಲ್ಲಿ ವೇದಾಂತವನ್ನು ಅಧ್ಯಯನ ಮಾಡಿರುವರೋ (ವೇದಾಂತವೆಂದರೆ ಉಪನಿಷತ್ತು) ಅವರಿಗೆ ಗೊತ್ತಿದೆ, ಪ್ರಪಂಚದಲ್ಲಿ ಇದೊಂದೇ ಯಾವ ವ್ಯಕ್ತಿಯ ಆಧಾರದ ಮೇಲೂ ನಿಂತಿಲ್ಲ ಎಂಬುದು. ವೇದಾಂತಿಗಳು ಯಾರೋ ಒಬ್ಬ ಸ್ತ್ರೀಯನ್ನು ಅಥವಾ ಪುರುಷನನ್ನು ಆರಾಧಿಸುವುದಿಲ್ಲ. ಇದು ಸಾಧ್ಯವೇ ಇಲ್ಲ. ಮನುಷ್ಯ ಯಾವ ಪ್ರಾಣಿಗಿಂತಲೂ ಅಥವಾ ಕೀಟಕ್ಕಿಂತಲೂ ಹೆಚ್ಚು ಆರಾಧನೆಗೆ ಯೋಗ್ಯನಲ್ಲ. ನಾವೆಲ್ಲ ಸಹೋದರರು, ಅಭಿವ್ಯಕ್ತಿಯ ತರತಮ ಮಾತ್ರ ವ್ಯತ್ಯಾಸಕ್ಕೆ ಕಾರಣ. ನಾನೊಂದು ಕ್ಷುದ್ರತಮ ಕೀಟಕ್ಕಿಂತ ಮೇಲಲ್ಲ. ಯಾರೋ ಒಬ್ಬನು ನಮಗಿಂತ ಮೇಲಿರುವುದು, ನಾವೆಲ್ಲ ಅವನನ್ನು ಪೂಜಿಸುವುದು, ಅವನು ನಮ್ಮನ್ನು ಉದ್ಧರಿಸುವುದು, ನಾವು ಅವನಿಂದ ಉದ್ಧಾರಗೊಳ್ಳುವುದು, ಇವಕ್ಕೆಲ್ಲ ವೇದಾಂತದಲ್ಲಿ ಸ್ಥಳವಿಲ್ಲ. ವೇದಾಂತ ನಿಮಗೆ ಇದನ್ನು ಕೊಡುವುದಿಲ್ಲ. ಶಾಸ್ತ್ರವಿಲ್ಲ, ಪೂಜಿಸುವುದಕ್ಕೆ ಯಾವ ಮನುಷ್ಯರೂ ಇಲ್ಲ, ಏನೂ ಇಲ್ಲ.

ಇದಕ್ಕಿಂತ ಮತ್ತೊಂದು ದೊಡ್ಡ ಕಷ್ಟವೆ ದೇವರಿಗೆ ಸಂಬಂಧಪಟ್ಟುದು. ನೀವು ಈ ದೇಶದಲ್ಲಿ ಪ್ರಜಾಪ್ರಭುತ್ವವಾದಿಗಳಾಗಬೇಕಾಗಿದೆ. ವೇದಾಂತ ಬೋಧಿಸುವುದೇ ಪ್ರಜಾಪ್ರಭುತ್ವದ ದೇವರನ್ನು.

ನಿಮಗೊಂದು ಸರ್ಕಾರವಿದೆ. ಆದರೆ ಅದು ಯಾವ ವ್ಯಕ್ತಿಗೂ ಸೇರಿದ್ದಲ್ಲ. ಇದು ಯಾರೋ ಒಬ್ಬರು ತನಗೆ ತೋಚಿದಂತೆ ಆಳುವ ಸರ್ಕಾರವಲ್ಲ. ಆದರೂ ಅದು ಪ್ರಪಂಚದ ಚಕ್ರಾಧಿಪತ್ಯಗಳೆಲ್ಲಕ್ಕಿಂತ ಬಲಶಾಲಿಯಾಗಿದೆ. ನಿಜವಾದ ಅಧಿಕಾರ, ಜೀವನ, ಶಕ್ತಿ ಇವು ಅವ್ಯಕ್ತದಲ್ಲಿ ನಿರ್ಗುಣ ನಿರಾಕಾರದಲ್ಲಿ ಅಡಗಿದೆ ಎಂಬುದು ಯಾರಿಗೂ ಗೊತ್ತಾಗುವಂತೆ ಇಲ್ಲ. ಇತರರಿಗಿಂತ ಬೇರೆಯಾದ ವ್ಯಕ್ತಿಯ ದೃಷ್ಟಿಯಿಂದ ನಿಮಗೆ ಯಾವ ಶಕ್ತಿಯೂ ಇಲ್ಲ. ಆದರೆ ಈ ದೇಶವನ್ನು ಆಳುತ್ತಿರುವ ಜನ ಸಾಮಾನ್ಯರಲ್ಲಿ ನೀವು ಒಂದಾದಾಗ ನಿಮ್ಮ ಶಕ್ತಿ ಅದ್ಭುತವಾದುದು. ಸರ್ಕಾರದಲ್ಲಿ ನೀವೆಲ್ಲ ಒಂದು.\break ಅಪರಿಮಿತವಾದ ಶಕ್ತಿ ನಿಮಗೆ ಇದೆ. ಆದರೆ ಶಕ್ತಿ ನಿಜವಾಗಿ ಎಲ್ಲಿದೆ? ಪ್ರತಿಯೊಬ್ಬನಲ್ಲಿಯೂ ಶಕ್ತಿ ಇದೆ. ರಾಜನಿಲ್ಲ. ನಿಮ್ಮಲ್ಲಿ ಎಲ್ಲರನ್ನೂ ಒಂದೇ ಸಮನಾಗಿ ಕಾಣುತ್ತೀರಿ. ಯಾರಿಗೂ ನಾನು ಟೋಪಿ ತೆಗೆದು ಬಾಗಿ ನಮಸ್ಕರಿಸಬೇಕಾಗಿಲ್ಲ. ಆದರೂ ಪ್ರತಿಯೊಬ್ಬನಲ್ಲಿಯೂ ಅದ್ಭುತವಾಗಿ ಶಕ್ತಿ ಇದೆ.

ಇದೇ ವೇದಾಂತ; ಎಂದರೆ, ಇಲ್ಲಿಯ ದೇವರು ಜನರಿಂದ ಬೇರೆಯಾಗಿ ಎಲ್ಲೋ ಸಿಂಹಾಸನದ ಮೇಲೆ ಮಂಡಿಸಿರುವ ಚಕ್ರವರ್ತಿಯಂತೆ ಅಲ್ಲ. ಕೆಲವರಿಗೆ ಇಂಥ ದೇವರನ್ನು ಕಂಡರೆ ಇಚ್ಛೆ - ಎಂದರೆ ಅಂಜಿಕೊಳ್ಳುವುದಕ್ಕೆ ಒಬ್ಬನಿರಬೇಕು, ತೃಪ್ತಿ ಪಡಿಸುವುದಕ್ಕೆ ಒಬ್ಬನಿರಬೇಕು. ಅವನಿಗೆ ಮಂಗಳಾರತಿ ಎತ್ತುವರು, ಧೂಳಿನಲ್ಲಿ ಅವನೆದುರಿಗೆ ಹೊರಳಾಡುವರು. ಅವರನ್ನು ಆಳುವುದಕ್ಕೆ ಒಬ್ಬ ರಾಜನಿರಬೇಕು. ಸ್ವರ್ಗದಲ್ಲಿರುವ ಯಾವನೋ ದೊರೆ ತಮ್ಮನ್ನೆಲ್ಲ ಆಳುತ್ತಾನೆ ಎಂದು ನಂಬುವರು. ಈ ದೇಶದಿಂದಲಾದರೂ ರಾಜ ಈಗ ಹೋಗಿರುವನು. ಈಗ ಸ್ವರ್ಗದ ರಾಜ ಎಲ್ಲಿರುವನು? ಪೃಥ್ವಿಯ ರಾಜನೆಲ್ಲಿರುವನೋ ಅಲ್ಲಿರುವನು. ಈ ದೇಶದಲ್ಲಿ ರಾಜ ನಿಮ್ಮಲ್ಲಿ ಪ್ರತಿಯೊಬ್ಬರಲ್ಲಿಯೂ ಪ್ರವೇಶಿಸಿರುವನು. ಈ ದೇಶದಲ್ಲಿ ನೀವೆಲ್ಲ ರಾಜರು. ಇದರಂತೆಯೇ ವೇದಾಂತದ ಧರ್ಮ ಕೂಡ. ನೀವೇ ದೇವತೆಗಳು. ಒಂದೇ ದೇವರು ಸಾಲುವುದಿಲ್ಲ, ನೀವೆಲ್ಲ ದೇವರು ಎನ್ನುವುದು ವೇದಾಂತ.

ಇದರಿಂದಲೇ ವೇದಾಂತ ಕಷ್ಟವಾಗುವುದು. ಇದು ಹಿಂದಿನ ದೇವರ ಭಾವನೆಯನ್ನು ಬೋಧಿಸುವುದಿಲ್ಲ. ಎಲ್ಲೋ ಮೋಡಗಳಾಚೆ ಕುಳಿತು, ನಮ್ಮ ಇಚ್ಛೆಯನ್ನು ಕೇಳದೆ ಪ್ರಪಂಚವನ್ನು ನಡೆಸುತ್ತಿರುವ, ಅವನಿಗೆ ಇಚ್ಛೆಯಾಯಿತೆಂದು ನಮ್ಮನ್ನೆಲ್ಲ ಶೂನ್ಯದಿಂದ ಸೃಷ್ಟಿಸಿ, ಈ ದುಃಖದ ಅನುಭವದ ಮೂಲಕ ಸಾಗಿ ಹೋಗುವಂತೆ ಮಾಡುವ ದೇವರನ್ನು ವೇದಾಂತ ಬೋಧಿಸದೆ, ಪ್ರತಿಯೊಬ್ಬರಲ್ಲಿಯೂ ಇರುವ, ಪ್ರತಿಯೊಂದು ವಸ್ತುವೂ, ಪ್ರತಿಯೊಂದು ವ್ಯಕ್ತಿಯೂ ಆದ ದೇವರನ್ನು ವೇದಾಂತ ಬೋಧಿಸುವುದು. ಚಕ್ರವರ್ತಿ ಈ ದೇಶದಿಂದ ಹೋದ. ಹಾಗೆಯೇ ವೇದಾಂತದಲ್ಲಿ ಸ್ವರ್ಗದ ರಾಜ ನೂರಾರು ವರುಷಗಳ ಹಿಂದೆಯೇ ಮಾಯವಾದ.

ಭರತಖಂಡವು ಭೂಮಿಯ ದೊರೆಯನ್ನು ತ್ಯಜಿಸಲಾರದು. ಆದ ಕಾರಣವೇ ವೇದಾಂತ ಭರತಖಂಡದ ಧರ್ಮವಾಗಲಾರದು. ಆದರೆ ನಿಮ್ಮ ದೇಶದಲ್ಲಿ ಅದಕ್ಕೆ ಅವಕಾಶವಿದೆ. ಏಕೆಂದರೆ ಇಲ್ಲಿ ಪ್ರಜಾಪ್ರಭುತ್ವವಿದೆ. ಆದರೆ ನೀವು ವೇದಾಂತವನ್ನು ಚೆನ್ನಾಗಿ ಅರ್ಥಮಾಡಿಕೊಂಡರೆ ಮಾತ್ರ ಸಾಧ್ಯ. ನೀವು ನಿಜವಾದ ಸ್ತ್ರೀ ಪುರುಷರಾಗಬೇಕು. ನೀವು ನಿಜವಾಗಿ ಆಧ್ಯಾತ್ಮಿಕ ಜೀವಿಗಳಾಗಬೇಕಾದರೆ ಅಸ್ಪಷ್ಟವಾದ ಭಾವನೆಗಳನ್ನು ಮತ್ತು ಮೂಢ ನಂಬಿಕೆಗಳನ್ನು ನಿಮ್ಮ ತಲೆಯಲ್ಲಿ ತುಂಬಿಕೊಂಡರೆ ಪ್ರಯೋಜನವಿಲ್ಲ. ಏಕೆಂದರೆ ವೇದಾಂತ ಅಧ್ಯಾತ್ಮಕ್ಕೆ ಮಾತ್ರ ಸಂಬಂಧಪಟ್ಟದ್ದು.

ಸ್ವರ್ಗದಲ್ಲಿರುವ ದೇವರ ಭಾವನೆ ಎಂದರೆ ಏನು? ಅದು ಕೇವಲ ಜಡವಾದ: ವೇದಾಂತದ ಭಾವನೆಯೇ, ನಮ್ಮಲ್ಲಿ ಪ್ರತಿಯೊಬ್ಬರಲ್ಲಿಯೂ ಅನಂತವಾದ ಅಧ್ಯಾತ್ಮ ತತ್ತ್ವ ಸುಪ್ತವಾಗಿದೆ ಎಂಬುದು. ದೇವರೆಲ್ಲೊ ಮೋಡದ ಮೇಲೆ ಕುಳಿತುಕೊಳ್ಳುವುದಂತೆ! ಇಂತಹ ಈಶ್ವರನಿಂದೆಯನ್ನು ಕುರಿತು ಯೋಚಿಸಿ ನೋಡಿ. ಇದು ಜಡವಾದ, ಶುದ್ದ ಭೌತಿಕತೆ. ಮಕ್ಕಳು ಹೀಗೆ ಯೋಚಿಸಿದರೆ ಚಿಂತೆಯಿಲ್ಲ, ಆದರೆ ಚೆನ್ನಾಗಿ ದೊಡ್ಡವರಾಗಿರುವವರು ಈ ರೀತಿ ಬೋಧಿಸಿದರೆ ಅದು ಜುಗುಪ್ಸಾಕಾರಕ. ಇದು ಹಾಗೆಯೇ ಆಗಿರುವುದು. ಇವೆಲ್ಲ ಜಡ, ಬರಿಯ ದೇಹಭಾವನೆ, ಸ್ಥೂಲವಾಗಿರುವುದು, ಪಂಚೇಂದ್ರಿಯಗಳಿಗೆ ಸಂಬಂಧಪಟ್ಟದ್ದು. ಇದರಲ್ಲಿ ಪ್ರತಿಯೊಂದು ಭಾಗವೂ ಮಣ್ಣು, ಬರೀ ಮಣ್ಣು. ಇದು ಧರ್ಮವೇ? ಇದು ಆಫ್ರಿಕಾ ದೇಶದ ಕಾಡುಜನರ ಧರ್ಮಕ್ಕಿಂತ ಮೇಲೇನೂ ಇಲ್ಲ. ದೇವರು ಆತ್ಮ, ಅವನನ್ನು ನಿಜವಾಗಿ ಆ ದೃಷ್ಟಿಯಿಂದ ಪೂಜಿಸಬೇಕು. ಆತ್ಮ ಸ್ವರ್ಗದಲ್ಲಿ ಮಾತ್ರ ಇರುವುದೆ? ಆತ್ಮವೆಂದರೇನು? ನಾವೆಲ್ಲ ಆತ್ಮ. ಆದರೆ ನಮಗೆ ಏತಕ್ಕೆ ಇದು ಗೊತ್ತಾಗುವುದಿಲ್ಲ. ನನಗೂ ನಿಮಗೂ ವ್ಯತ್ಯಾಸಕ್ಕೆ ಕಾರಣವೇನು? ದೇಹ-ಮತ್ತೇನೂ ಅಲ್ಲ. ದೇಹವನ್ನು ಮರೆಯಿರಿ. ಎಲ್ಲಾ ಆತ್ಮವೇ.

ವೇದಾಂತ ನೀಡುವ ಭಾವನೆಗಳು ಇವು. ಯಾವ ಗ್ರಂಥವೂ ಇಲ್ಲ. ಯಾವ ಒಬ್ಬ ವ್ಯಕ್ತಿಯೂ ಮಾನವ ಕೋಟಿಯಿಂದ ಪ್ರತ್ಯೇಕವಾಗಿ ನಿಂತು, “ನೀವೆಲ್ಲ ಕೀಟಗಳು, ನಾನು ಜಗದೊಡೆಯ, ಭಗವಂತ'' ಎಂದು ಹೇಳಬೇಕಾಗಿಲ್ಲ. ನೀನು ಈಶ್ವರನಾದರೆ ನಾನೂ ಈಶ್ವರ, ವೇದಾಂತ ಪಾಪವನ್ನು ಒಪ್ಪಿಕೊಳ್ಳುವುದಿಲ್ಲ. ತಪ್ಪುಗಳಿವೆ, ಆದರೆ ಪಾಪವಿಲ್ಲ. ಕೊನೆಗೆ ಎಲ್ಲಾ ಸರಿಯಾಗುವುದು. ಸೈತಾನ್ ಮುಂತಾದ ಅನಿಷ್ಟಗಳಾವುವೂ ಇಲ್ಲ. ವೇದಾಂತ ಒಂದು ಪಾಪವನ್ನು ಮಾತ್ರ ಒಪ್ಪಿಕೊಳ್ಳುವುದು, ಪ್ರಪಂಚದ ಏಕಮಾತ್ರ ಪಾಪವನ್ನು ಒಪ್ಪಿಕೊಳ್ಳುವುದು. ಅದೆ ಇದು: ನೀನು ಪಾಪಿ ಅಥವಾ ಇತರರು ಯಾರೊ ಪಾಪಿಗಳು ಎಂದು ಆಲೋಚಿಸಿದೊಡನೆ ನೀನು ಪಾಪಿಯಾಗುವೆ. ಈ ಪ್ರಥಮ ತಪ್ಪಿನಿಂದ ಅನಂತರ ಬರುವ ಪಾಪಗಳು ಅಥವಾ ತಪ್ಪುಗಳೆಲ್ಲ ಬರುವುವು. ನಮ್ಮ ಜೀವನದಲ್ಲಿ ಎಷ್ಟೋ ತಪ್ಪುಗಳಿವೆ. ಆದರೂ ನಾವು ಇನ್ನೂ ಇರುವೆವು. ಶಹಭಾಸ್, ನಾವು ತಪ್ಪು ಮಾಡಿದೆವು! ನಿಮ್ಮ ಹಿಂದಿನ ಜೀವನವನ್ನು ಸ್ವಲ್ಪ ದೀರ್ಘವಾಗಿ ಪರಿಶೀಲಿಸಿ. ನಿಮ್ಮ ಈಗಿನ ಸ್ಥಿತಿ ಒಳ್ಳೆಯದಾಗಿದ್ದರೆ ಅದು ಹಿಂದೆ ಮಾಡಿದ ಒಳ್ಳೆಯ ಮತ್ತು ಕೆಟ್ಟ ಪರಿಣಾಮಗಳ ಮೊತ್ತವಾಗಿದೆ. ಒಳ್ಳೆಯದಕ್ಕೆ ಧನ್ಯವಾದ, ಕೆಟ್ಟದಕ್ಕೂ ಧನ್ಯವಾದ. ಆದುದ್ದನ್ನು ಕುರಿತು ಚಿಂತಿಸಬೇಡಿ. ಮುಂದೆ ಹೋಗಿ.

ನೋಡಿ, ವೇದಾಂತ ಪಾಪವನ್ನೂ ಒಪ್ಪಿಕೊಳ್ಳುವುದಿಲ್ಲ, ಪಾಪಿಯನ್ನೂ ಒಪ್ಪಿಕೊಳ್ಳುವುದಿಲ್ಲ. ಅಂಜಿಕೊಳ್ಳುವಂತಹ ಯಾವ ದೇವರೂ ಇಲ್ಲ. ನಾವೆಂದಿಗೂ ದೇವರಿಗೆ ಅಂಜಬೇಕಾಗಿಲ್ಲ. ಏಕೆಂದರೆ ಅವನೇ ನಮ್ಮ ಆತ್ಮ. ನೀನು ಅಂಜಿಕೊಳ್ಳದವನು ಬಹುಶಃ ಒಬ್ಬನಿರಬಹುದು. ಅದೇ ಅವನು. ಹಾಗಾದರೆ ಈಶ್ವರನಿಗೆ ಅಂಜುವವನು ಬಹಳ ದೊಡ್ಡ ಮೂಢನಲ್ಲವೆ? ತಮ್ಮ ನೆರಳಿಗೆ ಅಂಜುವ ಕೆಲವರು ಇರಬಹುದು. ಅವನೂ ಕೂಡ ತನಗೇ ಅಂಜಿಕೊಳ್ಳುವುದಿಲ್ಲ. ಮಾನವನ ಆತ್ಮವೇ ದೇವರು. ನೀವು ಎಂದಿಗೂ ಅಂಜದವನು ಅವನೊಬ್ಬನೇ ಎನ್ನಬೇಕಾಗುವುದು. ದೇವರ ಭಯ ಮನುಷ್ಯನಿಗೆ ತಗುಲಿ ಅವನನ್ನು ತತ್ತರಿಸುವಂತೆ ಮಾಡುವುದೇನು? ಸದ್ಯಕ್ಕೆ ನಾವು ಹುಚ್ಚರಾಸ್ಪತ್ರೆಯಲ್ಲಿ ಇಲ್ಲದೇ ಇರುವುದಕ್ಕೆ ದೇವರಿಗೆ ಧನ್ಯವಾದ. ನಾವು ಅನೇಕರು ಹುಚ್ಚರಲ್ಲದೇ ಇದ್ದರೆ, ದೇವರ ಅಂಜಿಕೆ ಮುಂತಾದುವನ್ನೆಲ್ಲ ಏತಕ್ಕೆ ಸೃಷ್ಟಿಸುವುದು? ಭಗವಾನ್ ಬುದ್ದ ಬಹುಪಾಲು ಮಾನವ ಕೋಟಿ ಹುಚ್ಚರು ಎಂದು ಸಾರಿದ. ಇದು ಸಂಪೂರ್ಣ ನಿಜ ಎಂದು ತೋರುವುದು.

ಯಾವ ಶಾಸ್ತ್ರವೂ ಇಲ್ಲ, ವ್ಯಕ್ತಿಯೂ ಇಲ್ಲ, ಸಾಕಾರ ದೇವರೂ ಇಲ್ಲ. ಇವುಗಳೆಲ್ಲ ಹೋಗಬೇಕು. ಪುನಃ ಪಂಚೇಂದ್ರಿಯಗಳೂ ತೊಲಗಬೇಕು. ನಾವು ಇಂದ್ರಿಯಗಳಿಗೆ ದಾಸರಾಗಲಾರೆವು. ನಾವು ಸದ್ಯಕ್ಕೆ ಬದ್ಧರಾಗಿರುವೆವು. ನೀರ್ಗಲ್ಲಿನಲ್ಲಿ ಶೈತ್ಯಾಧಿಕ್ಯದಿಂದ ಸಾಯುತ್ತಿರುವವರಂತೆ ಇರುವೆವು. ಅವರಿಗೆ ಮಲಗಲು ಉತ್ಕಟ ಆಸೆ. ಅವರ ಸ್ನೇಹಿತರು ಹಾಗೆ ಮಲಗಿದರೆ ನೀವು ಸಾಯುವಿರಿ ಎಂದು ಎಚ್ಚರಿಸಿದರೆ, ನಾವು ಸತ್ತರೂ ಚಿಂತೆಯಿಲ್ಲ, ನಾವು ನಿದ್ರೆ ಮಾಡಬೇಕು ಎನ್ನುವರು. ನಾವೆಲ್ಲ ಇಂದ್ರಿಯಗಳಿಗೆ ಸಂಬಂಧಪಟ್ಟ ಅಲ್ಪ ವಸ್ತುಗಳನ್ನು ಬಾಚಿ ತಬ್ಬಿರುವೆವು. ನಾವು ಅದರಿಂದ ನಾಶವಾದರೂ ಚಿಂತೆಯಿಲ್ಲ. ಈ ಪ್ರಪಂಚದಲ್ಲಿ ಎಷ್ಟೋ ಮಹತ್ತಾದ ವಸ್ತುಗಳಿವೆ ಎಂಬುದನ್ನು ಮರೆಯುವೆವು.

ಒಂದು ಹಿಂದೂಗಳ ಕಥೆಯಿದೆ. ಹಿಂದೆ ಒಮ್ಮೆ ದೇವರು ಹಂದಿಯಂತೆ ಅವತಾರ ಮಾಡಿದನು. ಅವನು ಒಂದು ಹೆಣ್ಣು ಹಂದಿಯನ್ನು ಮದುವೆಯಾಗಿ ಕೆಲವು ದಿನಗಳಲ್ಲಿ ಅವನಿಗೆ ಹಲವು ಮರಿಗಳು ಆದವು. ಅವನು ತನ್ನ ಸಂಸಾರದೊಂದಿಗೆ ಕೆಸರಿನಲ್ಲಿ ಆಡುತ್ತ ಕಿರಿಚಿಕೊಳ್ಳುತ್ತ ತನ್ನ ಮಾಹಾತ್ಮಿಯನ್ನು ಮತ್ತು ಈಶ್ವರತ್ವವನ್ನು ಮರೆತು ಆನಂದದಲ್ಲಿದ್ದನು. ದೇವತೆಗಳಿಗೆ ತುಂಬಾ ಯೋಚನೆಯಾಯಿತು. ಅವರೆಲ್ಲ ಭೂಲೋಕಕ್ಕೆ ಬಂದು ತನ್ನ ಹಂದಿಯ ದೇಹವನ್ನು ತ್ಯಜಿಸಿ ದೇವಲೋಕಕ್ಕೆ ಬಾ ಎಂದು ಅವನನ್ನು ಬೇಡಿಕೊಂಡರು. ಆದರೆ ದೇವರಿಗೆ ಇದಾವುದೂ ಹಿಡಿಸಲಿಲ್ಲ. ಅವರನ್ನು ಓಡಿಸಿಬಿಟ್ಟನು. `ನಾನೀಗ ಆನಂದದಲ್ಲಿರುವೆನು, ನನ್ನ ಸುಖಕ್ಕೆ ಭಂಗ ತರಬೇಡಿ' ಎಂದನು. ಬೇರೆ ದಾರಿಯೇ ಇಲ್ಲದೆ ಆಗ ದೇವತೆಗಳು ದೇವರ ಹಂದಿಯ ರೂಪವನ್ನು ಸಂಹರಿಸಿದರು. ತಕ್ಷಣ ದೇವರಿಗೆ ತನ್ನ ಮಾಹಾತ್ಮ್ಯ ಅರಿವಾಯಿತು. ಹಂದಿಯ ಜನ್ಮದಲ್ಲಿ ತನಗೆ ಹೇಗೆ ತೃಪ್ತಿಯುಂಟಾಗಿತ್ತು ಎಂದು ಆಗ ಅವನಿಗೆ ಆಶ್ಚರ್ಯವಾಯಿತು.

ಜನರು ಹೀಗೆಯೇ ವ್ಯವಹರಿಸುವರು. ಅವರು ನಿರ್ಗುಣ ದೇವರ ಹೆಸರನ್ನು ಕೇಳಿದೊಡನೆಯೇ; “ಹಾಗಾದರೆ ನನ್ನ ವ್ಯಕ್ತಿತ್ವ ಏನಾಗುವುದು, ಅದು ಹೋಗುವುದಲ್ಲ!'' ಎಂದು ವ್ಯಥೆಪಡುವರು. ನಿಮಗೆ ಪುನಃ ಆ ಆಲೋಚನೆ ಬಂದಾಗ ಈ ಹಂದಿಯನ್ನು ಜ್ಞಾಪಿಸಿಕೊಳ್ಳಿ. ನಿಮ್ಮಲ್ಲಿ ಪ್ರತಿಯೊಬ್ಬರಲ್ಲಿಯೂ ಅನಂತಾನಂದದ ಗಣಿ ಇದೆ ಎಂಬುದನ್ನು ಕುರಿತು ಯೋಚಿಸಿ ನೋಡಿ. ನಿಮ್ಮ ಈಗಿನ ಸ್ಥಿತಿಯಲ್ಲಿ ಎಷ್ಟು ತೃಪ್ತರಾಗಿರುವಿರಿ! ಆದರೆ ನೀವು ನಿಜವಾಗಿ ಏನಾಗಿರುವಿರೋ ಅದನ್ನು ತಿಳಿದಾಗ ಈ ವಿಷಯ ಪ್ರಪಂಚವನ್ನು ತೊರೆಯಲು ನಿಮಗೆ ಆಸೆ ಇರದೇ ಇರುವುದು ಅತ್ಯಾಶ್ಚರ್ಯಕರವಾಗಿ ಕಾಣುವುದು. ನಿಮ್ಮ ವ್ಯಕ್ತಿತ್ವದಲ್ಲಿ ಏನಿದೆ? ಆ ಹಂದಿಗಿಂತ ಏನಾದರೂ ಮೇಲಾಗಿರುವುದೇನು? ಇದನ್ನು ನೀವು ತ್ಯಜಿಸಲಾರಿರಿ! ರಾಮನೇ ರಕ್ಷಿಸಬೇಕು!

ವೇದಾಂತ ಬೋಧಿಸುವುದೇನು? ಮೊದಲನೆಯದಾಗಿ ನೀವು ಸತ್ಯವನ್ನು ಅರಿಯಬೇಕಾದರೆ ನಿಮ್ಮಿಂದ ಹೊರಗೆ ಹೋಗಬೇಕಾಗಿಲ್ಲ ಎಂಬುದನ್ನು. ಭೂತ, ಭವಿಷ್ಯತ್ತುಗಳೆಲ್ಲ ಇಲ್ಲಿ ವರ್ತಮಾನದಲ್ಲಿವೆ. ಯಾರೂ ಹಿಂದಿನದನ್ನು ನೋಡಿಲ್ಲ. ನಿಮ್ಮಲ್ಲಿ ಯಾರಾದರೂ ಹಿಂದಿನದನ್ನು ನೋಡಿರುವಿರೇನು? ನಿಮಗೆ ಭೂತಕಾಲ ಗೊತ್ತಿದೆ ಎಂದು ವರ್ತಮಾನಕಾಲದಲ್ಲಿ ಅದನ್ನು ಕಲ್ಪಿಸಿಕೊಳ್ಳುತ್ತಿರುವಿರಿ. ಭವಿಷ್ಯತ್ತನ್ನು ನೋಡಬೇಕಾದರೂ ವರ್ತಮಾನಕ್ಕೆ ತರಬೇಕು. ಇದೊಂದೆ ಸತ್ಯ, ಉಳಿದುವೆಲ್ಲ ಕಲ್ಪನೆ. ಇರುವುದೆಲ್ಲ ವರ್ತಮಾನಕಾಲ ಮಾತ್ರ ಇರುವುದೊಂದೆ. ಎಲ್ಲಾ ಈಗ ಇಲ್ಲಿದೆ. ಅನಂತದಲ್ಲಿ ಪ್ರತಿ ಒಂದು ಕ್ಷಣವೂ ಮತ್ತೊಂದರಷ್ಟೇ ಪೂರ್ಣವಾಗಿದೆ, ಸರ್ವವ್ಯಾಪಿಯಾಗಿದೆ. ಹಿಂದೆ ಇದ್ದ, ಈಗ ಇರುವ, ಮುಂದೆ ಇರಬಲ್ಲದುದೆಲ್ಲ ಈಗ ಇಲ್ಲಿದೆ. ಇದಕ್ಕೆ ಅತೀತವಾದುದನ್ನು ಯಾರಾದರೂ ಕಲ್ಪಿಸಿಕೊಳ್ಳುವುದಕ್ಕೆ ಯತ್ನಿಸಲಿ. ಇದು ಅವನಿಗೆ ಸಾಧ್ಯವಿಲ್ಲ.

ಈ ಪ್ರಪಂಚದಂತೆ ಇಲ್ಲದೆ ಇರುವ ಸ್ವರ್ಗವನ್ನು ಯಾವ ಧರ್ಮ ಚಿತ್ರಿಸಬಲ್ಲದು? ಇದೆಲ್ಲ ಒಂದು ಕಲೆ. ಈ ಕಲೆ ಕ್ರಮೇಣ ನಮಗೆ ಗೊತ್ತಾಗುವುದು, ಅಷ್ಟೆ. ನಾವು ಪಂಚೇಂದ್ರಿಯಗಳ ಮೂಲಕ ಈ ಪ್ರಪಂಚವನ್ನು ನೋಡಿ ಇದು ಸ್ಥೂಲ ಎನ್ನುವೆವು. ಇದಕ್ಕೆ ಬಣ್ಣ ಆಕಾರ ಶಬ್ದ ಮುಂತಾದುವು ಇವೆ ಎನ್ನುವೆವು. ನನಗೆ ಒಂದು ವಿದ್ಯುತ್ತನ್ನು ಅರಿಯುವ ಒಂದು ಇಂದ್ರಿಯವೂ ಬಂತು ಎಂದು ಇಟ್ಟುಕೊಳ್ಳಿ. ಆಗ ಎಲ್ಲಾ ಬದಲಾಗುವುದು. ನನ್ನ ಇಂದ್ರಿಯ ಸೂಕ್ಷ್ಮವಾಗುತ್ತಾ ಬಂದಂತೆ ನೀವೆಲ್ಲ ಬದಲಾಗುತ್ತೀರಿ. ನಾನು ಬದಲಾದರೆ ನೀವೂ ಬದಲಾಗುವಿರಿ. ನಾನು ಇಂದ್ರಿಯಾತೀತನಾಗಿ ಹೋದರೆ ನೀವೆಲ್ಲ ಆತ್ಮದಂತೆ, ದೇವರಂತೆ ಕಾಣುತ್ತೀರಿ. ವಸ್ತುಗಳು, ಅವು ತೋರುವಂತೆ ಇಲ್ಲ.

ಇದು ನಮಗೆ ಕ್ರಮೇಣ ಗೊತ್ತಾಗುತ್ತಾ ಬರುವುದು. ಆಗ ಇದರ ಸತ್ಯದ ಅರಿವಾಗುವುದು. ಸ್ವರ್ಗಗಳೆಲ್ಲ ಇಲ್ಲಿ ಈಗ ಇರುವುವು. ಇವು ಪರಮಾತ್ಮನ ಅಸ್ತಿತ್ವದ ಮೇಲೆ ಕಾಣುವ ತೋರಿಕೆಯಲ್ಲದೆ ಬೇರೆಯಲ್ಲ. ಈ ಅಸ್ತಿತ್ವವೆ ಪೃಥ್ವಿ, ಸ್ವರ್ಗಗಳೆಲ್ಲಕ್ಕಿಂತ ಮಹತ್ತಾದುದು. ಜನರು ಈ ಪ್ರಪಂಚ ಎಲ್ಲಾ ಕೆಟ್ಟುಹೋಗಿದೆ, ಸ್ವರ್ಗ ಎಲ್ಲೋ ಬೇರೇ ಕಡೆ ಇದೆ ಎಂದು ಭಾವಿಸುವರು. ಈ ಪ್ರಪಂಚ ಏನೂ ಕೆಟ್ಟಿಲ್ಲ. ನಿಮಗೆ ಗೊತ್ತಾದರೆ ಇದು ದೇವರೇ ಆಗಿರುವುದು. ಇದನ್ನು ತಿಳಿದುಕೊಳ್ಳುವುದು ಬಹಳ ಕಷ್ಟ; ಇದನ್ನು ನಂಬುವಷ್ಟು ಸುಲಭವಲ್ಲ. ನಾಳೆ ಗಲ್ಲಿಗೆ ಏರಿಸಲ್ಪಡುವನು ದೇವರೇ ಆಗಿರುವನು, ಪೂರ್ಣ ದೇವರಾಗಿರುವನು. ನಿಜವಾಗಿ ಇದನ್ನು ತಿಳಿದುಕೊಳ್ಳುವುದು ಬಹಳ ಕಷ್ಟವೇ ಆಗಿದೆ. ಆದರೂ ಇದನ್ನು ತಿಳಿದುಕೊಳ್ಳಬಹುದು.

ಆದಕಾರಣವೇ ವೇದಾಂತ ವಿಶ್ವಸಹೋದರತ್ವದ ಭಾವನೆಯನ್ನು ತರುವುದಿಲ್ಲ. ವಿಶ್ವದ ಏಕತೆಯನ್ನು ತರುವುದು. ನಾನು ಕೂಡ ಇತರ ಪ್ರಾಣಿಗಳಂತೆ ಮತ್ತು ಮನುಷ್ಯರಂತೆ ಒಳ್ಳೆಯವನೂ ಕೆಟ್ಟವನೂ ಆಗಿರುವೆನು. ಎಲ್ಲಾ ಕಡೆಗಳಲ್ಲಿಯೂ ಇರುವುದು ಒಂದು ದೇಹ, ಒಂದು ಮನಸ್ಸು, ಒಂದು ಆತ್ಮ. ಆತ್ಮ ಎಂದಿಗೂ ನಾಶವಾಗುವುದಿಲ್ಲ. ಎಲ್ಲಿಯೂ ಸಾವಿಲ್ಲ, ದೇಹಕ್ಕೂ ಸಾವಿಲ್ಲ, ಮನಸ್ಸಿಗೂ ಸಾವಿಲ್ಲ. ದೇಹ ತಾನೇ ಹೇಗೆ ಸಾಯಬಲ್ಲದು? ಒಂದು ಎಲೆ ಬೀಳಬಹುದು, ಆದರೆ ಮರವೇ ಬಿತ್ತೇನು? ವಿಶ್ವವೇ ನನ್ನ ದೇಹ. ನೋಡಿ ಅದು ಹೇಗೆ ಮುಂದುವರಿಯುತ್ತಿದೆ ಎಂಬುದನ್ನು. ಮನಸ್ಸುಗಳೆಲ್ಲ ನನ್ನವೆ. ಎಲ್ಲಾ ಕಾಲುಗಳ ಮೂಲಕ ನಾನು ನಡೆಯುವೆನು, ಎಲ್ಲಾ ಬಾಯಿಗಳ ಮೂಲಕ ನಾನು ಮಾತನಾಡುವೆನು, ಎಲ್ಲಾ ದೇಹಗಳಲ್ಲಿ ನಾನೇ ಇರುವೆನು.

ಹಾಗಾದರೆ ನಾನು ಏತಕ್ಕೆ ಇದನ್ನು ಅನುಭವಿಸುವುದಿಲ್ಲ? ಆ ಹಂದಿಯಂತೆ ಇರುವ ವ್ಯಕ್ತಿತ್ವಕ್ಕೆ ಅಂಟಿಕೊಂಡಿರುವುದರಿಂದ. ನೀವು ಈ ಮನಸ್ಸಿಗೆ ಬದ್ದರಾಗಿರುವುದರಿಂದ ಇಲ್ಲಿ ಮಾತ್ರ ಇರಬಲ್ಲಿರಿ, ಅಲ್ಲಿ ಇರಲಾರಿರಿ. ಅಮೃತತ್ವ ಎಂದರೇನು ಎಂಬ ಪ್ರಶ್ನೆಗೆ “ಇದೇ, ನಮ್ಮ ಜೀವನವೇ ಅದು'' ಎಂದು ಎಷ್ಟು ಕಡಮೆ ಜನರು ಉತ್ತರಿಸಬಲ್ಲರು! ಅನೇಕ ಜನರು ಇದೆಲ್ಲ ಮರ್ತ್ಯ, ಸಾವು; ದೇವರು ಇಲ್ಲಿ ಇಲ್ಲ, ಸ್ವರ್ಗಕ್ಕೆ ಹೋದಾಗ ಮಾತ್ರ ಅಮೃತರಾಗುವೆವು ಎಂದು ಭಾವಿಸುವರು. ಸತ್ತ ಮೇಲೆ ದೇವರನ್ನು ನೋಡುತ್ತೇವೆ ಎಂದು ಊಹಿಸುವರು. ಆದರೆ ದೇವರನ್ನು ಇಲ್ಲಿ ನೋಡದೆ ಇದ್ದರೆ ಸತ್ತಮೇಲೆ ನೋಡುವುದಕ್ಕೆ ಆಗುವುದಿಲ್ಲ. ಅವರೆಲ್ಲ ಅಮರತ್ವದಲ್ಲಿ ನಂಬಿದ್ದರೂ, ಸತ್ತು ಸ್ವರ್ಗಕ್ಕೆ ಹೋದರೆ ಅದು ಸಿಕ್ಕುವುದಿಲ್ಲ ಎಂದು ಅವರಿಗೆ ಗೊತ್ತಿಲ್ಲ. ಅಮೃತತ್ವ ನಮಗೆ ಪ್ರಾಪ್ತವಾಗಬೇಕಾದರೆ, ನಮ್ಮ ಅಲ್ಪ ಹಂದಿಯ ವ್ಯಕ್ತಿತ್ವವನ್ನು ತ್ಯಜಿಸಬೇಕು; ಈ ದೇಹಕ್ಕೆ ನಾವು ಬದ್ಧರಾಗಿರಕೂಡದು. ಅಮೃತತ್ವ ಎಂದರೆ ನಾವೆಲ್ಲ ಒಂದು ಎಂದು ಭಾವಿಸುವುದು, ಎಲ್ಲ ದೇಹಗಳಲ್ಲಿ ವಾಸಿಸುವುದು, ಎಲ್ಲ ಮನಸ್ಸುಗಳ ಮೂಲಕ ಗ್ರಹಿಸುವುದು. ಈ ದೇಹದಲ್ಲಿ ಅಲ್ಲದೆ ಬೇರೆ ದೇಹಗಳಲ್ಲೂ ಅನುಭವಿಸಬೇಕಾಗಿದೆ. ಸಹಾನುಭೂತಿ ಎಂದರೇನು? ಇತರರಿಗೆ ತೋರಿಸುವ ಸಹಾನುಭೂತಿಗೆ ಒಂದು ಮಿತಿ ಇದೆಯೆ? ನಾನು ಇಡಿಯ ವಿಶ್ವದ ಮೂಲಕ ಅನುಭವಿಸಬಲ್ಲ ಒಂದು ಕಾಲ ಪ್ರಾಪ್ತವಾಗುವುದು ಸಾಧ್ಯ.

ಇದರಿಂದ ಪ್ರಯೋಜನವೇನು? ದೇಹವನ್ನು ತ್ಯಜಿಸುವುದೇ ಬಹಳ ಕಷ್ಟ. ನನ್ನ ಈ ಅಲ್ಪ ಹಂದಿಯ ದೇಹದ ಸುಖವನ್ನು ಕಳೆದುಕೊಳ್ಳುವುದಕ್ಕೆ ತುಂಬಾ ದುಃಖವಾಗುವುದು. ವೇದಾಂತ ಸುಖವನ್ನು ತ್ಯಜಿಸಿ ಎನ್ನುವುದಿಲ್ಲ. ಅದನ್ನು ಅತಿಕ್ರಮಿಸಿ ಎನ್ನುವುದು. ಇಲ್ಲೇನೂ ದೇಹದಂಡನೆಯಿಲ್ಲ. ಒಂದು ದೇಹಕ್ಕಿಂತ ಎರಡು ದೇಹಗಳ ಮೂಲಕ ಪಡುವ ಆನಂದ ಹೆಚ್ಚು. ಅದಕ್ಕಿಂತ ಮೂರರ ಆನಂದ ಹೆಚ್ಚು. ನಾನು ಇಡಿಯ ವಿಶ್ವದ ಮೂಲಕ ಅನುಭವಿಸಬಲ್ಲೆನಾದರೆ ಇಡಿಯ ವಿಶ್ವವೇ ನನ್ನ ದೇಹವಾಗುವುದು.~।

ಈ ಬೋಧನೆಗಳನ್ನು ಕೇಳಿದರೆ ಹಲವರಿಗೆ ಅಂಜಿಕೆಯಾಗುವುದು. ಅವರು ಒಬ್ಬ ಮತಾಂಧನಾದ ದೇವರಿಂದ ಸೃಷ್ಟಿಸಲ್ಪಟ್ಟ ಅಲ್ಪ ಹಂದಿ ದೇಹಗಳು ತಾವಲ್ಲ ಎಂದು ಕೇಳಲು ಇಚ್ಛಿಸುವುದಿಲ್ಲ. ಮೇಲೆ ಬನ್ನಿ ಎಂದು ನಾನು ಅವರಿಗೆ ಹೇಳುತ್ತೇನೆ. ಅಯ್ಯೊ ನಾವು ಪಾಪದಲ್ಲಿ ಹುಟ್ಟಿರುವೆವು, ಮತ್ತೊಬ್ಬರ ಕೃಪೆಯಿಲ್ಲದೆ ನಾವು ಬರಲಾರೆವು ಎನ್ನುವರು ಅವರು. ನೀವು ಪವಿತ್ರಾತ್ಮರು ಎಂದರೆ, “ಅಯ್ಯೋ ನಾಸ್ತಿಕನೇ, ಇದನ್ನು ಹೇಗೆ ನೀನು ಹೇಳಬಲ್ಲೆ? ನಮ್ಮಂತಹ ಕೆಲಸಕ್ಕೆ ಬಾರದವರು ದೇವರಾಗಬಲ್ಲರು ಎಂದು ಹೇಳಲು ಎಷ್ಟು ಧೈರ್ಯ ನಿನಗೆ? ನಾವೆಲ್ಲ ಪಾಪಿಗಳು" ಎನ್ನುವರು. ಕೆಲವು ವೇಳೆ ಇದನ್ನು ಕಂಡಾಗ ನಾನು ತುಂಬಾ ನಿರುತ್ಸಾಹಿಯಾಗುವೆನು. ನೂರಾರು ಮಂದಿ ಸ್ತ್ರೀಪುರುಷರು “ನರಕವಿಲ್ಲದೇ ಇದ್ದರೆ ಹೇಗೆ ಧರ್ಮ ಇರಬಲ್ಲದು?'' ಎಂದು ಪ್ರಶ್ನಿಸುವರು. ಈ ಜನರು ತಾವೇ ನರಕಕ್ಕೆ ಹೋಗಲು ಇಚ್ಛೆ ಪಟ್ಟರೆ ಯಾರು ಅವರನ್ನು ತಡೆಯಬಲ್ಲರು?

ನೀವು ಯಾವುದನ್ನು ಕನಸುಕಟ್ಟುತ್ತೀರೋ, ಯಾವುದನ್ನು ಕುರಿತು ಆಲೋಚಿಸುತ್ತಿರುವಿರೋ ಅವನ್ನು ಸೃಷ್ಟಿಸುತ್ತೀರಿ. ಅದು ನರಕವಾದರೆ ಸತ್ತ ಮೇಲೆ ನರಕವನ್ನು ನೋಡುತ್ತೀರಿ. ನೀವು ಪಾಪವನ್ನು ಅಂದರೆ ಸೈತಾನನನ್ನು ಆಲೋಚಿಸುತ್ತಿದ್ದರೆ ನಿಮಗೆ ಸೈತಾನ ಕಾಣುವನು. ಭೂತಪ್ರೇತಗಳನ್ನು ಆಲೋಚಿಸಿದರೆ ನೀವು ಅವನ್ನು ನೋಡುತ್ತೀರಿ. ನೀವು ಆಲೋಚಿಸಿದಂತೆ ಆಗುತ್ತೀರಿ. ನೀವು ಆಲೋಚಿಸಬೇಕಾದರೆ ಒಳ್ಳೆಯದನ್ನು ಆಲೋಚಿಸಿ, ಮಹದಾಲೋಚನೆಗಳನ್ನು ಮಾಡಿ. ನೀವು ಕೆಲಸಕ್ಕೆ ಬಾರದ ಸಣ್ಣ ಕೀಟಗಳೆಂದು ಭಾವಿಸುವುದು ಯೋಗ್ಯವಲ್ಲ. ನಾವು ದುರ್ಬಲರೆಂದು ಹೇಳಿಕೊಂಡರೆ ದುರ್ಬಲರಾಗುತ್ತೇವೆ, ನಾವು ಅದಕ್ಕಿಂತ ಮೇಲಾಗುವುದಿಲ್ಲ. ನಾವು ದೀಪವನ್ನು ಆರಿಸಿ, ಕದವನ್ನು ಮುಚ್ಚಿ, ಕೋಣೆ ಕತ್ತಲೆ ಎನ್ನುತ್ತೇವೆ. ಈ ಅವಿವೇಕವನ್ನು ಕುರಿತು ಯೋಚಿಸಿ ನೋಡಿ. ನಾನು ಪಾಪಿ ಎಂದುಕೊಂಡರೆ ಇದರಿಂದ ಏನು ಪ್ರಯೋಜನ? ನಾವು ಕತ್ತಲೆಯಲ್ಲಿದ್ದರೆ ಒಂದು ದೀಪವನ್ನು ಹಚ್ಚೋಣ. ಆಗ ಎಲ್ಲಾ ಸರಿ ಬರುವುದು. ಆದರೂ ಮನುಷ್ಯನ ಸ್ವಭಾವ ಎಷ್ಟು ವಿಚಿತ್ರವಾದುದು! ತಮ್ಮ ಹಿಂದೆ ಯಾವಾಗಲೂ ವಿಶ್ವಾತ್ಮನಿರುವನೆಂದು ಗೊತ್ತಿದ್ದರೂ ಅವರು ಸೈತಾನ್, ಪಾಪ, ಅಸತ್ಯ ಇವನ್ನು ಕುರಿತು ಯೋಚಿಸುತ್ತಿರುವರು. ನೀವು ಅವರಿಗೆ ಸತ್ಯ ಹೇಳಿದರೆ ಅದು ಕಾಣುವುದಿಲ್ಲ. ಅವರಿಗೆ ಕತ್ತಲೆಯೇ ಇಷ್ಟ.

ವೇದಾಂತಿ ಕೇಳುವ ದೊಡ್ಡದೊಂದು ಪ್ರಶ್ನೆ ಇದು: ಜನರು ಏತಕ್ಕೆ ಅಂಜುವರು? ಇದಕ್ಕೆ ಕಾರಣವೇ ಜನರು ಅನ್ಯರನ್ನು ಆಶ್ರಯಿಸಿ ತಾವು ಅಸಹಾಯಕರಾಗಿರುವುದು. ನಾವು ಶುದ್ಧ ಸೋಮಾರಿಗಳು. ಏನನ್ನೂ ನಮಗಾಗಿ ಮಾಡಿಕೊಳ್ಳಲು ಇಚ್ಛೆಯಿಲ್ಲ. ಅದನ್ನೆಲ್ಲಾ ಮಾಡುವುದಕ್ಕೆ ನಮಗೆ ಒಬ್ಬ ದೇವರು ಬೇಕು, ದೇವದೂತ ಬೇಕು. ತುಂಬಾ ಶ‍್ರೀಮಂತ ಎಂದೂ ನಡೆಯುವುದಿಲ್ಲ. ಯಾವಾಗಲೂ ಗಾಡಿಯಲ್ಲೇ ಹೋಗುತ್ತಾನೆ. ಆದರೆ ಕೆಲವು ವರುಷಗಳಾದ ಮೇಲೆ ಅವನಿಗೆ ಗೊತ್ತಾಗುವುದು, ಕಾಲುಗಳ ಮೇಲೆ ಸ್ವಾಧೀನವೇ ಅವನಿಗೆ ತಪ್ಪಿಹೋಗಿದೆ ಎಂದು. ಆಗ ತಾನು ಹಾಗೆ ಜೀವಿಸಿದ್ದು ತಪ್ಪು ಎಂದು ಭಾವಿಸುವನು. ಯಾರೂ ನನಗಾಗಿ ನಡೆಯಲಾರರು. ಎಂದಾದರೂ ಯಾರಾದರೂ ಹೀಗೆ ಮಾಡಿದ್ದರೆ ಇದರಿಂದ ನನಗೇ ಅಪಾಯ. ಒಬ್ಬನಿಗೆ ಎಲ್ಲವನ್ನೂ ಇತರರು ಮಾಡಿದರೆ ಅಂಗಾಂಗಗಳ ಪ್ರಯೋಜನವೇ ಇರುವುದಿಲ್ಲ. ನಾವೇ ನಮಗೆ ಮಾಡಿಕೊಂಡದ್ದು ಮಾತ್ರ ನಮ್ಮದು. ನಮಗಾಗಿ ಇತರರು ಮಾಡಿದುದಾವುದೂ ನಮ್ಮದಾಗಲಾರದು. ನನ್ನ ಉಪನ್ಯಾಸದಿಂದ ನೀವು ಆಧ್ಯಾತ್ಮಿಕ ಸತ್ಯಗಳನ್ನು ಅರಿಯಲಾರಿರಿ. ನೀವು ಏನಾದರೂ ಕಲಿತಿದ್ದರೆ, ನಾನೊಂದು ಕಿಡಿ ಮಾತ್ರ ಆಗಿದ್ದೆ; ಅದರಿಂದ ಕಿಚ್ಚು ನಿಮ್ಮಲ್ಲಿ ಹುಟ್ಟಿತು. ಮಹಾತ್ಮರು, ಗುರುಗಳು, ಮಾಡುವುದು ಇಷ್ಟು ಮಾತ್ರ. ಸಹಾಯಕ್ಕಾಗಿ ಅಲ್ಲಿ ಇಲ್ಲಿ ಅಲೆಯುವುದೆಲ್ಲ ನಿಷ್ಪ್ರಯೋಜಕ.

ಭರತಖಂಡದಲ್ಲಿ ಎತ್ತಿನ ಗಾಡಿಗಳಿವೆ. ಗಾಡಿಗೆ ಒಂದು ಜೊತೆ ಎತ್ತುಗಳನ್ನು ಕಟ್ಟುತ್ತಾರೆ. ಕೆಲವು ವೇಳೆ ನೊಗದ ಮುಂದೆ ಎತ್ತಿಗೆ ಕಾಣುವಂತೆ ಸ್ವಲ್ಪ ಹುಲ್ಲನ್ನು ಕಟ್ಟುತ್ತಾರೆ. ಆದರೆ ಅದರ ಬಾಯಿಗೆ ಅದು ಎಟಕುವುದಿಲ್ಲ. ಎತ್ತುಗಳು ಅದನ್ನು\break ಮೇಯಬೇಕೆಂದು ಎಷ್ಟೋ ಪ್ರಯತ್ನಪಡುತ್ತವೆ, ಆದರೆ ಸಾಧ್ಯವಾಗುವುದಿಲ್ಲ. ನಿಮಗೆ ನಿಜವಾಗಿ ಸಹಾಯ ಬರುವುದೂ ಹೀಗೆಯೇ! ನಮಗೆ ಬಾಹ್ಯ ಪ್ರಪಂಚದಿಂದ ರಕ್ಷಣೆ, ಬಲ, ಬುದ್ದಿ, ಸೌಖ್ಯ ಎಲ್ಲಾ ಬರುವುದೆಂದು ಆಶಿಸುವೆವು. ನಾವು ಯಾವಾಗಲೂ ಆಶಿಸುವೆವು. ಆದರೆ ಅದು ಎಂದಿಗೂ ಈಡೇರುವುದಿಲ್ಲ. ಎಂದಿಗೂ ಹೊರಗಿನಿಂದ ಯಾವ ಸಹಾಯವೂ ಬಂದಿಲ್ಲ.

ಮನುಷ್ಯನಿಗೆ ಸಹಾಯವಿಲ್ಲ. ಹಿಂದೆ ಎಂದೂ ಇರಲಿಲ್ಲ, ಈಗ ಇಲ್ಲ, ಮುಂದೆ ಎಂದೆಂದಿಗೂ ಇರಲಾರದು. ಏತಕ್ಕೆ ಬೇಕು? ನೀವು ಸ್ತ್ರೀಪುರುಷರಲ್ಲವೆ? ನಾಚಿಕೆಯಾಗುವುದಿಲ್ಲವೆ? ನೀವು ಧೂಳೀಕಣಗಳಾದರೆ ನಿಮಗೆ ಸಹಾಯ ಬರುವುದು. ಆದರೆ ನೀವು ಆತ್ಮ, ನಿಮ್ಮ ಕಷ್ಟಗಳಿಂದ ನೀವೇ ಪಾರಾಗಿ, ಆತ್ಮನನ್ನು ಆತ್ಮನಿಂದ ಉದ್ದರಿಸಿ. ನಿಮಗೆ ಸಹಾಯ ಮಾಡುವುದಕ್ಕೆ ಯಾರೂ ಇಲ್ಲ, ಎಂದೂ ಇರಲಿಲ್ಲ. ಇದ್ದರು ಎಂದು ಭಾವಿಸುವುದೊಂದು ಮಧುರ ಮೋಹ. ಇದರಿಂದ ಏನೂ ಪ್ರಯೋಜನವಿಲ್ಲ.

ಒಂದು ಸಲ ಕ್ರೈಸ್ತನೊಬ್ಬ ಬಂದು “ನೀನೊಬ್ಬ ಘೋರ ಪಾತಕಿ'' ಎಂದನು. ನಾನು “ಹೌದು” ಎಂದೆನು. ಅವನೊಬ್ಬ ಪಾದ್ರಿ. ಅವನು ನನಗೆ ಸ್ವಲ್ಪವೂ ವಿರಾಮ ಕೊಡುತ್ತಿರಲಿಲ್ಲ. ಅವನನ್ನು ಕಂಡರೆ ನಾನು ಓಡಿಹೋಗಲೆತ್ನಿಸುತ್ತಿದ್ದೆನು. ಅವನು ಮತ್ತೊಮ್ಮೆ “ನೋಡು, ನಿನಗೊಂದು ಒಳ್ಳೆಯ ಸಮಾಚಾರ ಇರುವುದು. ನೀನು ಪಾಪಿ, ನರಕಕ್ಕೆ ಹೋಗುವೆ?'' ಎಂದನು. ಅದಕ್ಕೆ ನಾನು “ಒಳ್ಳೆಯದು, ಇನ್ನೇನಿದೆ? ನೀನು ಎಲ್ಲಿಗೆ ಹೋಗುವೆ?” ಎಂದು ಕೇಳಿದೆ. ಆತ ತಾನು ಸ್ವರ್ಗಕ್ಕೆ ಹೋಗುತ್ತೇನೆ ಎಂದ. “ನಾನು ನರಕಕ್ಕೆ ಹೋಗುತ್ತೇನೆ'' ಎಂದೆ. ಅಂದಿನಿಂದ ಅವನು ನನ್ನನ್ನು ತ್ಯಜಿಸಿದ.

ಕ್ರೈಸ್ತನೊಬ್ಬ ಬಂದು, “ನೀವೆಲ್ಲ ಹಾಳಾಗುವಿರಿ, ಆದರೆ ನೀವು ಈ ಸಿದ್ದಾಂತವನ್ನು ನಂಬಿದರೆ ಕ್ರಿಸ್ತ ನಿಮ್ಮನ್ನು ಉದ್ಧರಿಸುತ್ತಾನೆ'' ಎನ್ನುವನು. ಇದು ಸತ್ಯವಾದರೆ (ಆದರೆ ಇದು ಬರಿಯ ಮೂಢನಂಬಿಕೆ) ಕೈಸ್ತ ದೇಶಗಳಲ್ಲಿ ಪಾಪವೇ ಇರುತ್ತಿರಲಿಲ್ಲ. ನಾವು ಇದನ್ನು ನಂಬೋಣ. ನಂಬುವುದಕ್ಕೆ ಏನೂ ಖರ್ಚುಮಾಡಬೇಕಾಗಿಲ್ಲ. ಆದರೆ ಇದರಿಂದ ಏನೂ ಪ್ರಯೋಜನವಿಲ್ಲ. ಏತಕ್ಕೆ ಕ್ರೈಸ್ತರಲ್ಲಿ ಇಷ್ಟೊಂದು ದುರ್ಜನರು ಇರುವರು ಎಂದು ಕೇಳಿದರೆ, ನಾವು ಇನ್ನೂ ಅದಕ್ಕೆ ಕೆಲಸ ಮಾಡಬೇಕಾಗಿದೆ ಎನ್ನುವರು. ದೇವರನ್ನು ನಂಬಿ, ಆದರೆ ಬಂದೂಕಿನ ಮದ್ದನ್ನು ಒಣಗಿಸಿಟ್ಟಿರಿ ಎನ್ನುವರು. ನೀವು ದೇವರನ್ನು ಪ್ರಾರ್ಥಿಸಿ, ಅವನು ಬಂದು ನಿಮಗೆ ಸಹಾಯಮಾಡಲಿ! ಆದರೆ ನಾನೇ ಹೋರಾಡಬೇಕಾಗಿದೆ, ಪ್ರಾರ್ಥಿಸಬೇಕಾಗಿದೆ, ಪೂಜಿಸಬೇಕಾಗಿದೆ. ನನ್ನ ಸಮಸ್ಯೆಗಳನ್ನೆಲ್ಲ ಪರಿಹರಿಸಿಕೊಳ್ಳುವವನು ನಾನೇ. ಆದರೆ ಕೀರ್ತಿಯನ್ನೆಲ್ಲ ದೇವರು ತೆಗೆದುಕೊಳ್ಳುವನು. ಇದು ಸರಿಯಲ್ಲ. ನಾನು ಎಂದಿಗೂ ಹಾಗೆ ಮಾಡುವುದಿಲ್ಲ.

ಒಂದು ಸಲ ಯಾರೋ ನನ್ನನ್ನು ಊಟಕ್ಕೆ ಕರೆದರು. ಗೃಹಿಣಿ “ಗ್ರೇಸ್ (grace) ಹೇಳಿ ಊಟಕ್ಕೆ ಮುಂಚೆ” ಎಂದಳು. “ತಾಯಿ ನಿನಗೆ ಗ್ರೇಸ್ ಹೇಳುತ್ತೇನೆ. ಗ್ರೇಸ್ ಮತ್ತು ಧನ್ಯವಾದ ಇವುಗಳನ್ನೆಲ್ಲ ನಿನಗೆ ಸಲ್ಲಿಸುತ್ತೇನೆ'' ಎಂದೆನು. ಕೆಲಸಮಾಡಿದಾಗ ನಾನೇ ಶ್ಲಾಘಿಸಿಕೊಳ್ಳುತ್ತೇನೆ. ನಾನು ಕಷ್ಟಪಟ್ಟು ಕೆಲಸಮಾಡಿ ನನ್ನಲ್ಲಿರುವುದನ್ನು ಸಂಪಾದಿಸಿದೆ. ಆದಕಾರಣ ನನಗೆಯೇ ಧನ್ಯವಾದ.

ನೀವು ಯಾವಾಗಲೂ ಕಷ್ಟಪಟ್ಟು ಕೆಲಸಮಾಡಿ, ಇನ್ನೊಬ್ಬರನ್ನು ಹೊಗಳುತ್ತೀರಿ. ಯಾಕೆಂದರೆ ನಿಮಗೆ ಅಂಜಿಕೆ, ಮೂಢನಂಬಿಕೆ. ಸಾವಿರಾರು ವರುಷಗಳಿಂದ ಬಂದ ಈ ಮೂಢನಂಬಿಕೆ ಇನ್ನು ಮೇಲೆ ಇರಕೂಡದು. ಆಧ್ಯಾತ್ಮಿಕರಾಗಬೇಕಾದರೆ ಸ್ವಲ್ಪ ಕಷ್ಟಪಟ್ಟು ಸಾಧನೆ ಮಾಡಬೇಕು. ಮೂಢನಂಬಿಕೆಗಳೆಲ್ಲ ಜಡವಾದಕ್ಕೆ ಸೇರಿದುವುಗಳು. ಏಕೆಂದರೆ ಅವೆಲ್ಲ ದೇಹದಿಂದ ಜನಿಸಿದುವುಗಳು. ದೇಹ, ದೇಹ! ಅಲ್ಲಿ ಆತ್ಮವೇನೂ ಇಲ್ಲ: ಆತ್ಮಕ್ಕೆ ಯಾವ ಮೂಢನಂಬಿಕೆಯೂ ಇಲ್ಲ. ಅದು ದೇಹದ ಕ್ಷಣಿಕ ಆಸೆಗೆ ಅತೀತವಾಗಿದೆ.

ಆದರೆ ಆಧ್ಯಾತ್ಮಿಕ ಕ್ಷೇತ್ರಕ್ಕೆ ಈ ಕ್ಷಣಿಕ ಆಸೆಗಳು ಕೆಲವು ಕಡೆ ಧಾಳಿ ಇಟ್ಟಿವೆ. ನಾನು ಹಲವು ಸ್ಪಿರಿಚುಯಲಿಸ್ಟರ ಸಭೆಗೆ ಹೋಗಿರುವೆನು. ಅಂತಹ ಒಂದರಲ್ಲಿ ಮುಂದಾಳು ಒಬ್ಬ ಮಹಿಳೆಯಾಗಿದ್ದಳು. ಆಕೆ `ನಿನ್ನ ತಾಯಿ ಮತ್ತು ಅಜ್ಜಿ ನನ್ನ ಬಳಿಗೆ ಬಂದು ಮಾತನಾಡಿದರು' ಎಂದಳು. ಆದರೆ ನನ್ನ ತಾಯಿ ಇನ್ನೂ ಬದುಕಿರುವಳು. ಜನರು ಸತ್ತರೂ ದೇಹಧಾರಿಗಳಾಗಿರುವರೆಂದು ಅವರ ಬಂಧುಗಳು ಭಾವಿಸುವರು. ಸ್ಪಿರಿಚುಯಲಿಸ್ಟರು ಈ ಮೂಢನಂಬಿಕೆಯನ್ನೇ ಉಪಯೋಗಿಸಿಕೊಳ್ಳುವರು. ನನ್ನ ಸತ್ತುಹೋದ ತಂದೆ ಇನ್ನೂ ಈ ಅನಿಷ್ಟ ಭೌತಿಕ ದೇಹವನ್ನು ಧರಿಸಿರುವನು ಎಂದು ಕೇಳಿದರೆ ನನಗೆ ತುಂಬಾ ವ್ಯಥೆಯಾಗುವುದು. ಜನರಿಗೆ ಇದರಿಂದ ಒಂದು ಸಮಾಧಾನ, ಅವರೆಲ್ಲ ದೇಹಧಾರಿಗಳಾಗಿರುವರು ಎಂದು. ಮತ್ತೊಂದು ಕಡೆ ಏಸು ಕ್ರಿಸ್ತನನ್ನು ಬರಮಾಡಿದರು. `ದೇವರೆ, ಆರೋಗ್ಯವೆ?' ಎಂದೆ ನಾನು. ನನಗೆ ಇದರಿಂದ ತುಂಬಾ ನಿರಾಸೆಯಾಗುವುದು. ಆ ಮಹಾಪುರುಷನೇ ಇನ್ನೂ ಒಂದು ದೇಹವನ್ನು ಧರಿಸಿದ್ದರೆ ನಮ್ಮಂತಹವರ ಪಾಡೇನು? ಸ್ಪಿರಿಚುಯಲಿಸ್ಟರು ಅವರಲ್ಲಿ ಯಾರನ್ನೂ ಮುಟ್ಟಲು ನನಗೆ ಬಿಡಲಿಲ್ಲ. ಅವರು ನಿಜವಾದರೂ ನನಗೆ ಬೇಕಾಗಿಲ್ಲ. “ತಾಯಿ, ನಿಜವಾಗಿ ನಾಸ್ತಿಕರು ಈ ಜನರು! ಇನ್ನೂ ಇಂದ್ರಿಯಗಳ ವ್ಯಾಮೋಹವಿದೆ. ಈಗ ಇಲ್ಲಿರುವುದು ಸಾಲದೆ? ಸತ್ತ ಅನಂತರವೂ ಇದೇ ಇರಬೇಕು ಇವರಿಗೆ!”

ವೇದಾಂತದ ದೇವರು ಯಾರು? ಅವನೊಂದು ತತ್ತ್ವ, ವ್ಯಕ್ತಿಯಲ್ಲ. ನಾವು, ನೀವೆಲ್ಲ ಸಗುಣ ದೇವರು. ಸೃಷ್ಟಿ ಸ್ಥಿತಿಪ್ರಳಯಕರ್ತೃವಾದ ನಿರಾಕಾರ ಈಶ್ವರನು ನಿರ್ಗುಣವಾದ ಒಂದು ತತ್ತ್ವ. ನಾವು ನೀವು, ಭೂತ ಪ್ರೇತ, ಬೆಕ್ಕು ಇಲಿ ಮುಂತಾದವುಗಳೆಲ್ಲ ಆ ನಿರಾಕಾರದ ಆವಿರ್ಭಾವಗಳು. ಎಲ್ಲರೂ ಸಗುಣ ದೇವರುಗಳು. ನೀವು ಸಗುಣ ದೇವರುಗಳನ್ನು ಪೂಜಿಸಲು ಇಚ್ಚಿಸುತ್ತೀರಿ. ಅದು ನಿಮ್ಮ ಆತ್ಮ ಪೂಜೆ. ನೀವು ನನ್ನ ಬುದ್ಧಿವಾದವನ್ನು ಕೇಳಿದರೆ ನೀವು ಯಾವ ಚರ್ಚಿಗೂ ಹೋಗುವುದಿಲ್ಲ; ಹೊರಗೆ ಬಂದು ಅವುಗಳಿಂದ ಪಾರಾಗಿ. ಹಲವು ಶತಮಾನಗಳಿಂದ ನಿಮಗೆ ಅಂಟಿಕೊಂಡಿರುವ ಮೂಢನಂಬಿಕೆಗಳನ್ನೆಲ್ಲ ತೊಳೆಯಿರಿ. ಅಥವಾ ನಿಮಗೆ ಹಾಗೆ ಮಾಡಲು ಇಚ್ಛೆ ಇಲ್ಲದೆ ಇರಬಹುದು. ಏಕೆಂದರೆ ನಿಮಗೆ ಈ ದೇಶದಲ್ಲಿ ದಿನವೂ ಸ್ನಾನಮಾಡಿ ಅಭ್ಯಾಸವಿಲ್ಲ. ಪದೇ ಪದೇ ಸ್ನಾನಮಾಡುವುದು ಹಿಂದೂಗಳಲ್ಲಿ ರೂಢಿ, ಈ ದೇಶದಲ್ಲಿ ಇಲ್ಲ.

“ನೀವು ಏತಕ್ಕೆ ಇಷ್ಟೊಂದು ನಗುವುದು, ಇಷ್ಟೊಂದು ತಮಾಷೆಮಾಡುವುದು?'' ಎಂದು ಅನೇಕ ವೇಳೆ ಜನರು ನನ್ನನ್ನು ಕೇಳುವರು. “ನಾನು ಸ್ವಲ್ಪ ಗಂಭೀರವಾಗುತ್ತೇನೆ, ಹೊಟ್ಟೆನೋವು ಬಂದಾಗ!” ಎಂದೆ. ಭಗವಂತ ಸಚ್ಚಿದಾನಂದಮಯ. ಇರುವ ವಸ್ತುಗಳಲ್ಲೆಲ್ಲಾ ಅವನೇ ಸತ್ಯ. ಅವನೇ ಎಲ್ಲದರಲ್ಲಿಯೂ ಸತ್ಯ ಮತ್ತು ಶುಭ. ನೀವೆಲ್ಲ ಅವನ ಅವತಾರಗಳು. ಇದೇ ಮಹಿಮಾಮಯವಾದುದು. ನೀವು ಹೆಚ್ಚು ಅವನ ಸಮೀಪದಲ್ಲಿದ್ದಷ್ಟೂ, ಅಳುವುದು ಕಡಮೆಯಾಗುವುದು. ದೂರವಿದ್ದಷ್ಟೂ ಗೋಳಿನ ಮುಖ ಜಾಸ್ತಿ. ಭಗವಂತನಲ್ಲಿ ಸದಾ ವಿಹರಿಸುವವನು ದುಃಖಿಯಾದರೆ, ಅವನಲ್ಲಿದ್ದು ಪ್ರಯೋಜನವೇನಾಯಿತು? ಇಂತಹ ದೇವರಿಂದ ಏನು ಪ್ರಯೋಜನ? ಅವನನ್ನು ಪೆಸಿಫಿಕ್ ಸಾಗರಕ್ಕೆ ನೂಕಿ ಬಿಡಿ. ನಮಗೆ ಅವನು ಬೇಡ.

ಆದರೆ ದೇವರು ಅನಂತ, ನಿರ್ಗುಣ, ಸದಾಕಾಲದಲ್ಲಿಯೂ ಇರುವವನು.\break ನಿರ್ವಿಕಾರಿ, ಅಮೃತನು, ನಿರ್ಭಯನು ಅವನು. ನೀವೆಲ್ಲ ಅವನ ಅಭಿವ್ಯಕ್ತಿಗಳು, ಅವನ ವಿಗ್ರಹಗಳು. ಇದೇ ವೇದಾಂತದ ದೇವರು. ಅವನ ಸ್ವರ್ಗ ಎಲ್ಲೆಲ್ಲಿಯೂ ಇರುವುದು. ಈ ಸ್ವರ್ಗದಲ್ಲಿ ಎಲ್ಲಾ ದೇವ ದೇವತೆಗಳೂ ಇರುವರು. ಆ ದೇವ ದೇವತೆಗಳು ನೀವೇ. ದೇವಾಲಯಗಳಲ್ಲಿ ಹೂವುಗಳನ್ನಿಟ್ಟು ಪ್ರಾರ್ಥಿಸುವುದನ್ನು ಬಿಟ್ಟುಬಿಡಿ.

ನೀವು ಯಾವುದಕ್ಕಾಗಿ ಪ್ರಾರ್ಥಿಸುವಿರಿ? ನೀವು ಸ್ವರ್ಗಕ್ಕೆ ಹೋಗಬೇಕೆ? ನಿಮಗೆ ಏನೋ ಸಿಕ್ಕಬೇಕು, ಇತರರಿಗೆ ಅದು ಸಿಕ್ಕಬಾರದೆಂದು ಪ್ರಾರ್ಥಿಸುವಿರೇನು? “ದೇವರೆ, ನನಗೆ ಹೆಚ್ಚು ಊಟ ಕೊಡು, ಇನ್ನೊಬ್ಬರು ಬೇಕಾದರೆ ಉಪವಾಸವಿರಲಿ.'' ಯಾರು ಸತ್ಯಸ್ವರೂಪನೊ, ಅನಂತನೋ, ಅಂಶವಿಲ್ಲದ ನಿಷ್ಕಳಂಕನಾದ ನಿತ್ಯಮುಕ್ತನೊ, ನಿತ್ಯಶುದ್ದನೋ, ನಿತ್ಯಪೂರ್ಣನೊ, ಅಂತಹ ದೇವರನ್ನು ನೀವು ಎಷ್ಟು ಅಲ್ಪವಾಗಿ ಭಾವಿಸುವಿರಿ! ನಾವು ಅವನಿಗೆ ಮಾನವಗುಣ, ಕ್ರಿಯೆ, ಮಿತಿ- ಇವುಗಳನ್ನೆಲ್ಲ ಆರೋಪಿಸುತ್ತೇವೆ. ಅವನು ನಮಗೆ ಅನ್ನ ಬಟ್ಟೆಗಳನ್ನು ಒದಗಿಸಬೇಕು. ಆದರೆ ನಿಜವಾಗಿ ನಾವೇ ಇದನ್ನೆಲ್ಲ ಮಾಡಿಕೊಳ್ಳಬೇಕಾಗಿದೆ. ಇತರರಾರೂ ನಮಗೆ ಇದನ್ನು ಮಾಡುವುದಿಲ್ಲ. ಇದೇ ಸತ್ಯಾಂಶ.

ಆದರೆ ಇದನ್ನು ನೀವು ಆಲೋಚಿಸುವುದು ಅಪರೂಪ, ಎಲ್ಲಿಯೋ ಒಬ್ಬನು ದೇವರಿರುವನೆಂದು, ಅವನಿಗೆ ನಿಮ್ಮ ಮೇಲೆ ಬಹಳ ಪ್ರೀತಿಯೆಂದು ಭಾವಿಸುವಿರಿ. ನೀವು ಕೇಳಿದ್ದನ್ನೆಲ್ಲಾ ಅವನು ಕೊಡುವನು. ನೀವು ಇತರರಿಗಾಗಿ ಏನನ್ನೂ ಕೇಳುವುದಿಲ್ಲ, ನೀವು ಬೇಡುವುದೆಲ್ಲ ನಿಮಗಾಗಿ, ನಿಮ್ಮ ಮನೆಯವರಿಗಾಗಿ, ನಿಮ್ಮ ನೆಂಟರಿಷ್ಟರಿಗಾಗಿ. ಹಿಂದೂ ಉಪವಾಸದಿಂದ ನರಳುತ್ತಿದ್ದರೆ ನೀವು ಮನಸ್ಸಿಗೆ ಹಾಕಿಕೊಳ್ಳುವುದಿಲ್ಲ. ಆಗ ಕ್ರೈಸ್ತರ ದೇವರೇ ಹಿಂದೂಗಳ ದೇವರೆಂಬ ಭಾವನೆ ನಿಮಗೆ ಬರುವುದಿಲ್ಲ. ನಮ್ಮ ಭಗವಂತನ ಭಾವನೆ, ನಮ್ಮ ಪೂಜೆ, ನಮ್ಮ ಪ್ರಾರ್ಥನೆ ಇವುಗಳಲ್ಲೆಲ್ಲಾ ನಾವು ದೇಹವೆಂಬ ಮೂಢ ಭಾವನೆ ಬೆರೆತುಹೋಗಿದೆ. ನಾನು ಹೇಳುತ್ತಿರುವುದು ನಿಮಗೆ ಇಷ್ಟವಾಗದಿರಬಹುದು. ನೀವು ನನ್ನನ್ನು ಇಂದು ಶಪಿಸಬಹುದು. ಆದರೆ ನಾಳೆ ನೀವೇ ನನ್ನನ್ನು ಆಶೀರ್ವದಿಸುತ್ತೀರಿ.

ನಾವು ಚಿಂತಕರಾಗಬೇಕಾಗಿದೆ. ಪ್ರತಿಯೊಂದು ಜನ್ಮದಲ್ಲಿಯೂ ದುಃಖವಿದೆ. ನಾವು ಈ ಜಡವಾದದಿಂದ ಪಾರಾಗಬೇಕು. ನಮ್ಮ ತಾಯಂದಿರಿಗೆ ನಮ್ಮ ಮೇಲೆ ಇರುವ ಅಧಿಕಾರವನ್ನು ಕಳೆದುಕೊಳ್ಳಲು ಇಚ್ಚೆಯಿಲ್ಲ. ಆದರೂ ನಾವು ಪ್ರಯತ್ನಪಡಬೇಕು. ಈ ಹೋರಾಟವೇ ನಿಜವಾದ ಪೂಜೆ. ಉಳಿದುದೆಲ್ಲ ನಿಷ್ಪ್ರಯೋಜನ. ನೀವೇ ಸಾಕಾರ ದೇವರು. ಈಗ ತಾನೇ ನಾನು ನಿಮ್ಮನ್ನು ಪೂಜಿಸುತ್ತಿರುವೆನು. ಇದೇ ಶ್ರೇಷ್ಠ ಪ್ರಾರ್ಥನೆ. ಇಡಿಯ ವಿಶ್ವವನ್ನು ಅದರ ಸೇವೆಯನ್ನು ಮಾಡುವುದರ ಮೂಲಕ ಪೂಜಿಸಿ. ಅದು ಸೇವೆಯಾದರೆ ಪೂಜೆಯಾಗುವುದು, ಎತ್ತರವಾದ ವೇದಿಕೆಯ ಮೇಲೆ ನಿಲ್ಲುವುದು ಪೂಜೆಯಾಗುವುದಿಲ್ಲ.

ನಾವು ಅನಂತ ಸತ್ಯವನ್ನು ಹೊಸದಾಗಿ ಎಂದಿಗೂ ಪಡೆಯುವುದಿಲ್ಲ. ಅದು ಯಾವಾಗಲೂ ಇಲ್ಲೇ ಇರುವುದು. ಅವಿನಾಶಿ, ಅಜ ಅದು. ವಿಶ್ವೇಶ್ವರನು ಎಲ್ಲರಲ್ಲಿಯೂ ಇರುವನು. ಇರುವುದೊಂದೇ ದೇವಾಲಯ, ಅದೇ ದೇಹ. ಇರುವ ಏಕಮಾತ್ರ ದೇವಸ್ಥಾನ ಅದೊಂದೆ. ಈ ದೇಹದಲ್ಲಿ ಆತ್ಮೇಶ್ವರನು ರಾಜರಾಜನಾಗಿ ಇರುವನು. ನಾವು ಅವನನ್ನು ನೋಡುವುದಿಲ್ಲ. ಆದಕಾರಣವೆ ಅವನ ವಿಗ್ರಹವನ್ನು ಮಾಡಿ ಅದಕ್ಕೊಂದು ದೇವಸ್ಥಾನವನ್ನು ಕಟ್ಟುವೆವು, ವೇದಾಂತವು ಭಾರತ ದೇಶದಲ್ಲಿ ಯಾವಾಗಲಿನಿಂದಲೂ ಇರುವುದು. ಆದರೂ ಅಲ್ಲಿ ಬೇಕಾದಷ್ಟು ದೇವಸ್ಥಾನಗಳಿವೆ. ದೇವಸ್ಥಾನಗಳು ಮಾತ್ರವಲ್ಲ, ವಿಗ್ರಹಗಳಿರುವ ಗುಹೆಗಳಿವೆ. ಗಂಗಾ ನದಿಯ ತೀರದಲ್ಲಿ ನೀರಿಗಾಗಿ ಮೂಢನು ಒಂದು ಭಾವಿಯನ್ನು ತೋಡುತ್ತಿರುವನು. ನಾವು ಅವನಂತೆ ಇರುವೆವು. ನಾವು ದೇವರೊಡನೆ ಇದ್ದರೂ ನಾವು ಹೊರಗೆ ವಿಗ್ರಹಗಳನ್ನು ಮಾಡಬೇಕಾಗಿದೆ. ಅವನು ಸದಾಕಾಲದಲ್ಲಿ ನಮ್ಮ ದೇಹ ಎಂಬ ದೇವಾಲಯದಲ್ಲಿರುವಾಗಲೂ ನಾವು ಅವನನ್ನು ಒಂದು ವಿಗ್ರಹದಲ್ಲಿ ಆರೋಪಿಸಿರುವೆವು. ನಾವು ಹುಚ್ಚರು; ಇದೇ ಮಹಾ ಮೋಹ.

ಪ್ರತಿಯೊಂದನ್ನೂ ದೇವರೆಂದು ಆರಾಧಿಸಿ, ಪ್ರತಿಯೊಂದು ಆಕಾರವೂ ಅವನ ದೇವಾಲಯವೆ. ಉಳಿದುದೆಲ್ಲಾ ಭ್ರಾಂತಿ. ಯಾವಾಗಲೂ ಒಳಗೆ ನೋಡಿ, ಹೊರಗೆ ನೋಡಬೇಡಿ. ವೇದಾಂತವು ಬೋಧಿಸುವ ದೇವರು ಇದು, ಬೋಧಿಸುವ ಪೂಜೆ ಇದು. ಆದಕಾರಣ ವೇದಾಂತದಲ್ಲಿ ಜಾತಿ-ಕುಲ-ವರ್ಣಬೇಧಗಳಿಲ್ಲ. ಈ ಧರ್ಮ ಇಡಿಯ ಭರತಖಂಡದ ರಾಷ್ಟ್ರೀಯ ಧರ್ಮ ಹೇಗೆ ಆಗಬಲ್ಲದು?

ಭರತಖಂಡದಲ್ಲಿ ನೂರಾರು ಜಾತಿಗಳಿವೆ. ಒಬ್ಬ ಮತ್ತೊಬ್ಬನ ಆಹಾರವನ್ನು ಮುಟ್ಟಿದರೆ “ದೇವರೇ, ನನ್ನನ್ನು ರಕ್ಷಿಸು, ನಾನು ಮೈಲಿಗೆಯಾದೆ" ಎಂದು ಅರಚುವನು. ನಾನು ಪಾಶ್ಚಾತ್ಯ ಪ್ರವಾಸವನ್ನು ಪೂರೈಸಿಕೊಂಡು ಭರತಖಂಡಕ್ಕೆ ಹೋದಾಗ ಅಲ್ಲಿ ಅನೇಕ ಸಂಪ್ರದಾಯಸ್ಥರು ನಾನು ಪಾಶ್ಚಾತ್ಯರೊಂದಿಗೆ ಬೆರೆತಿದ್ದಕ್ಕೆ ವಿಪರೀತ ಗಲಾಟೆ ಮಾಡಿದರು. ಪಾಶ್ಚಾತ್ಯರಿಗೆ ವೇದಾಂತವನ್ನು ಬೋಧಿಸಿದ್ದು ಅವರಿಗೆ ಆಗದು.

ಹಾಗಾದರೆ ಈ ಭಿನ್ನತೆ ಮತ್ತು ವ್ಯತ್ಯಾಸಗಳು ಹೇಗೆ ಇವೆ? ನಾವೆಲ್ಲ ಆತ್ಮವಾದರೆ, ಒಂದೇ ಸಮನಾಗಿದ್ದರೆ, ಶ‍್ರೀಮಂತ ಬಡವನನ್ನು, ಪಂಡಿತ ಪಾಮರರನ್ನು ಹೇಗೆ ನಿಕೃಷ್ಟ ದೃಷ್ಟಿಯಿಂದ ನೋಡಬಲ್ಲ? ಸಮಾಜ ಬದಲಾಗುವವರೆಗೆ ಇಂತಹ ಧರ್ಮ ಹೇಗೆ ಅಲ್ಲಿ ಇರಬಲ್ಲದು? ನಿಜವಾಗಿಯೂ ವಿಚಾರಪರರಾದ ಜನರು ಬಹಳ ಮಂದಿ ಪ್ರಪಂಚದಲ್ಲಿ ಇರಬೇಕಾದರೆ ಸಾವಿರಾರು ವರುಷಗಳು ಬೇಕಾಗುವುದು. ಹೊಸ ವಿಷಯಗಳನ್ನು ಜನರಿಗೆ ತೋರಿಸುವುದು, ಉದಾತ್ತ ಭಾವನೆಗಳನ್ನು ಜನರಿಗೆ ಕೊಡುವುದು ಬಹಳ ಕಷ್ಟ. ಹಳೆಯ ಮೂಢನಂಬಿಕೆಗಳನ್ನು ಹೊಡೆದೋಡಿಸುವುದು ಮತ್ತೂ ಕಷ್ಟ. ಅವು ಸುಲಭವಾಗಿ ನಾಶವಾಗುವುದಿಲ್ಲ. ಎಷ್ಟೇ ವಿದ್ಯಾವಂತನಾದರೂ ಕತ್ತಲಲ್ಲಿ ಅಂಜುವನು. ಬಾಲ್ಯದಲ್ಲಿ ಕೇಳಿದ್ದ ಕಟ್ಟು ಕತೆಗಳ ಭೂತಪ್ರೇತಗಳು ಅಲ್ಲಿ ಬಂದಂತೆ ಆಗುವುದು.

ವೇದ ಎಂಬ ಪದದಿಂದ ವೇದಾಂತ ಎಂಬ ಪದ ಬಂದಿದೆ. ವೇದ ಎಂದರೆ ಜ್ಞಾನ. ಜ್ಞಾನವೆಲ್ಲ ವೇದ, ದೇವರಷ್ಟೇ ಅನಂತವಾದುದು ಅದು. ಯಾರೂ ಜ್ಞಾನವನ್ನು ಹೊಸದಾಗಿ ಸೃಷ್ಟಿಸುವುದಿಲ್ಲ. ಜ್ಞಾನ ಸೃಷ್ಟಿಯಾಗುವುದನ್ನು ನೀವು ಯಾವಾಗಲಾದರೂ ನೋಡಿರುವಿರಾ? ಅದೊಂದು ಆವಿಷ್ಕಾರ; ಯಾವುದು ಮುಚ್ಚಿ ಹೋಗಿರುವುದೋ ಅದು ಬೆಳಕಿಗೆ ಬರುವುದು. ಅದು ಯಾವಾಗಲೂ ಇಲ್ಲಿರುವುದು. ಏಕೆಂದರೆ ಅದು ದೇವರೇ ಆಗಿದೆ. ಹಿಂದಿನ, ಈಗಿನ, ಮುಂದಿನ ಜ್ಞಾನವೆಲ್ಲ ಆಗಲೇ ನಮ್ಮೆಲ್ಲರಲ್ಲಿಯೂ ಇದೆ. ನಾವು ಅದನ್ನು ಕಂಡುಹಿಡಿಯುವೆವು, ಅಷ್ಟೆ. ಈ ಜ್ಞಾನವೆಲ್ಲ ಸ್ವಯಂ ಭಗವಂತನೆ. ವೇದಗಳು ದೊಡ್ಡ ಸಂಸ್ಕೃತ ಗ್ರಂಥಗಳು. ನಮ್ಮ ದೇಶದಲ್ಲಿ ಯಾರು ವೇದವನ್ನು ಓದುವರೊ ಅವರಿಗೆ ನಮಸ್ಕರಿಸುತ್ತೇವೆ. ಆದರೆ ಯಾರು ಭೌತಶಾಸ್ತ್ರವನ್ನು ಓದುತ್ತಿರುವರೊ ಅವರನ್ನು ಲೆಕ್ಕಿಸುವುದೇ ಇಲ್ಲ. ಇದೊಂದು ಮೌಡ್ಯ. ಇದನ್ನು ವೇದಾಂತ ಎಂದು ಹೇಳಲಾಗುವುದಿಲ್ಲ. ಇದು ಕೇವಲ ಜಡವಾದ. ದೇವರಿದ್ದರೆ ಪ್ರತಿಯೊಂದು ಜ್ಞಾನವೂ ಪವಿತ್ರವಾಗುವುದು. ಜ್ಞಾನವೇ ದೇವರು. ಅನಂತ ಜ್ಞಾನ ಸಂಪೂರ್ಣವಾಗಿ ಎಲ್ಲರಲ್ಲಿಯೂ ಇರುವುದು. ನಿಜವಾಗಿಯೂ ನೀವು ಅಜ್ಞರೆ? ತೋರಿಕೆಗೆ ಹಾಗೆ ನೀವು ಕಂಡರೂ ಕೂಡ ನೀವೆಲ್ಲ ಭಗವಂತನ ಅವತಾರಗಳು. ಸರ್ವಶಕ್ತನಾದ, ಸರ್ವಾಂತರ್ಯಾಮಿಯಾದ ಭಗವತ್ತತ್ತ್ವದ ಸಾಕಾರ ರೂಪಗಳು ನೀವು. ಈಗ ನೀವು ಅದನ್ನು ಕೇಳಿ ನಗಬಹುದು, ಆದರೆ ಅದನ್ನು ಅರ್ಥ ಮಾಡಿಕೊಳ್ಳುವ ಒಂದು ಸಮಯ ಬರುವುದು. ನೀವು ಅದನ್ನು ಅರಿಯಲೇ ಬೇಕಾಗುವುದು; ಯಾರೂ ಹಿಂದೆ ಉಳಿಯುವುದಿಲ್ಲ.

ಗುರಿಯೇನು? ನಾನು ಹೇಳಿದ ವೇದಾಂತ ಒಂದು ಹೊಸ ಧರ್ಮವಲ್ಲ. ಅದು ಬಹಳ ಹಳೆಯದು, ದೇವರಷ್ಟೇ ಹಳೆಯದು. ಅದು ಯಾವ ದೇಶಕಾಲಗಳಿಗೂ ಸೇರಿಲ್ಲ. ಅದು ಸರ್ವಾಂತರ್ಯಾಮಿ, ಪ್ರತಿಯೊಬ್ಬರಿಗೂ ಈ ಸತ್ಯ ಗೊತ್ತಿದೆ. ನಾವೆಲ್ಲ ಅದನ್ನು ಪಡೆಯಲು ಪ್ರಯತ್ನಿಸುತ್ತಿರುವೆವು. ಪ್ರಪಂಚದ ಗುರಿಯೇ ಅದು. ಅದು ಬಾಹ್ಯ ಪ್ರಪಂಚಕ್ಕೂ ಅನ್ವಯಿಸುವುದು. ಪ್ರತಿಯೊಂದು ಕಣವೂ ಆ ಗುರಿಯೆಡೆಗೆ ಧಾವಿಸುತ್ತಿದೆ. ಅನಂತವಾದ ಪರಿಶುದ್ಧಾತ್ಮರಲ್ಲಿ ಕೆಲವರು ಈ ಪರಮತತ್ತ್ವವನ್ನು ತಿಳಿಯದೆ ಉಳಿದುಕೊಳ್ಳುವರು ಎಂದು ಭಾವಿಸಿದಿರೇನು? ಸತ್ಯ ಎಲ್ಲರಲ್ಲಿಯೂ ಇದೆ. ಅವರೆಲ್ಲ ಒಂದೇ ಗುರಿಯೆಡೆಗೆ ಹೋಗುತ್ತಿರುವರು. ತಮ್ಮ ಸ್ವಭಾವಸಹಜವಾದ ಪವಿತ್ರತೆಯನ್ನು ಹುಡುಕುವುದಾಗಿದೆ ಆ ಗುರಿ. ಹುಚ್ಚ, ಕೊಲೆಪಾತಕಿ, ಮೂಢ, ಈ ದೇಶದಲ್ಲಿ ನೀವು ಯಾರನ್ನು (ನೀಗ್ರೊಗಳನ್ನು) ಮರಕ್ಕೆ ಕಟ್ಟಿ ಸುಡುವಿರೋ ಅವರೆಲ್ಲ ಒಂದೇ ಗುರಿಯೆಡೆಗೆ ಹೋಗುತ್ತಿರುವರು. ನಾವು ಯಾವುದನ್ನು ತಿಳಿಯದೆ ಮಾಡುವೆವೊ, ಅದನ್ನು ತಿಳಿದು ಮಾಡಬೇಕು, ಚೆನ್ನಾಗಿ ಮಾಡಬೇಕು ಅಷ್ಟೆ.

ವಿಶ್ವದ ಏಕತೆ -ಇದೆಲ್ಲ ಆಗಲೆ ನಿಮ್ಮಲ್ಲಿದೆ. ಇದಿಲ್ಲದೇ ಯಾರೂ ಜನ್ಮ ಗ್ರಹಣಮಾಡಲಿಲ್ಲ. ನೀವು ಅದನ್ನು ಎಷ್ಟು ಅಲ್ಲಗಳೆದರೂ ಅದು ಪದೇ ಪದೇ ಎದ್ದು ನಿಲ್ಲುವುದು. ಮಾನವಪ್ರೇಮವೆಂದರೇನು? ಅದು ಆ ಏಕತೆಯ ಸಮರ್ಥನೆಯಾಗಿದೆ. “ನನ್ನ ಹೆಂಡತಿ ಮಕ್ಕಳು ಬಂಧುಬಳಗವೆಲ್ಲ ಒಂದೆ'' ಎಂದು ಹೇಳಿದಾಗ ಅರಿಯದೆ ಏಕತೆಯನ್ನು ಸಮರ್ಥಿಸುತ್ತಿರುವಿರಿ. “ಗಂಡನಿಗಾಗಿ ಯಾರೂ ಗಂಡನನ್ನು ಪ್ರೀತಿಸಲಿಲ್ಲ. ಆದರೆ ಗಂಡನಲ್ಲಿರುವ ಆತ್ಮನಿಗಾಗಿ.” ಹೆಂಡತಿ ಅಲ್ಲಿ ಏಕತೆಯನ್ನು ಕಾಣುವಳು. ಗಂಡನು ಹೆಂಡತಿಯಲ್ಲಿ ತನ್ನನ್ನು ಕಾಣುವನು. ಸ್ವಾಭಾವಿಕವಾಗಿ ಇದನ್ನು ಮಾಡುವನು. ಇದನ್ನು ತಿಳಿದಂತೆ, ಜ್ಞಾನಪೂರ್ವಕವಾಗಿ ಮಾಡುತ್ತಿಲ್ಲ.

ವಿಶ್ವವೆಲ್ಲ ಏಕ, ಅಲ್ಲಿ ಮತ್ತೊಂದು ಇಲ್ಲ. ಭಿನ್ನತೆಯಿಂದ ನಾವೆಲ್ಲ ಏಕತ್ವದ ಕಡೆಗೆ ಹೋಗುತ್ತಿರುವೆವು. ಕುಲಗಳು ಪಂಗಡಗಳಲ್ಲಿ, ಪಂಗಡಗಳು ಜನಾಂಗಗಳಲ್ಲಿ, ಜನಾಂಗಗಳು ರಾಷ್ಟ್ರಗಳಲ್ಲಿ, ರಾಷ್ಟ್ರಗಳು ಮಾನವ ಕೋಟಿಯಲ್ಲಿ ಐಕ್ಯವಾಗುತ್ತಿವೆ. ಒಂದರಲ್ಲಿ ಎಷ್ಟೊಂದು ಇಚ್ಛೆಗಳು ಸೇರುತ್ತಿವೆ! ಈ ಏಕತೆಯ ಸಾಕ್ಷಾತ್ಕಾರವೇ ಜ್ಞಾನ ಮತ್ತು ವಿಜ್ಞಾನ.

ಏಕತೆಯೇ ಜ್ಞಾನ, ಭಿನ್ನತೆಯೇ ಅಜ್ಞಾನ. ಈ ಜ್ಞಾನವೇ ನಿಮ್ಮ ಆಜನ್ಮಸಿದ್ದ ಹಕ್ಕು. ನಾನು ಇದನ್ನು ನಿಮಗೆ ಕಲಿಸಬೇಕಾಗಿಲ್ಲ. ಪ್ರಪಂಚದಲ್ಲಿ ಎಂದೂ ವಿಭಿನ್ನ ಧರ್ಮಗಳು ಇರಲಿಲ್ಲ. ನಮಗೆಲ್ಲ ಮುಕ್ತಿ ಬರಲೇಬೇಕಾಗಿದೆ. ಏಕೆಂದರೆ ಮುಕ್ತರಾಗುವುದೇ ನಮ್ಮ ಸ್ವಭಾವ. ನಾವಾಗಲೇ ಮುಕರು, ಆದರೆ ನಮಗೆ ಅದು ಗೊತ್ತಿಲ್ಲ. ನಾವು ಏನು ಮಾಡುತ್ತಿರುವೆವೋ ಅದು ನಮಗೆ ಗೊತ್ತಿಲ್ಲ. ಎಲ್ಲಾ ಧರ್ಮಗಳಲ್ಲಿ ಮತ್ತು ಆದರ್ಶಗಳಲ್ಲಿ ಒಂದೇ ನೀತಿಯಿದೆ; ಅವುಗಳು ಒಂದನ್ನೇ ಬೋಧಿಸುವುವು. ಅದೇ ನಿಃಸ್ವಾರ್ಥರಾಗಿ, ಇತರರನ್ನು ಪ್ರೀತಿಸಿ ಎಂಬುದನ್ನು. ಒಬ್ಬ ಯಹೋವ ಹಾಗೆ ಹೇಳಿದ ಎನ್ನುತ್ತಾರೆ, ಮತ್ತೊಬ್ಬ ಅಲ್ಲಾ ಹೀಗೆ ಹೇಳಿದ ಎನ್ನುತ್ತಾರೆ, ಇನ್ನೊಬ್ಬ ಏಸು ಹೀಗೆ ಹೇಳಿದ ಎನ್ನುತ್ತಾರೆ. ಇದು ಕೇವಲ ಯಹೋವನ ಆಜ್ಞೆ ಮಾತ್ರವಾಗಿದ್ದರೆ ಯಹೋವನ ಪರಿಚಯವೇ ಇಲ್ಲದ ಇತರರಿಗೆ ಇದು ಹೇಗೆ ಬರುತ್ತಿತ್ತು? ಇದು ಏಸುವಿನ ಆಜ್ಞೆ ಮಾತ್ರವಾಗಿದ್ದರೆ ಅವನ ಪರಿಚಯವೇ ಇಲ್ಲದವರಿಗೆ ಇದು ಹೇಗೆ ಬರುತ್ತಿತ್ತು? ಇದು ವಿಷ್ಣುವಿನ ಆಜ್ಞೆ ಮಾತ್ರವಾಗಿದ್ದರೆ ಆ ಸಭ್ಯಪುರುಷನ ಪರಿಚಯವೇ ಇಲ್ಲದ ಯೆಹೂದ್ಯರಿಗೆ ಇದು ಹೇಗೆ ಗೊತ್ತಾಗುತ್ತಿತ್ತು? ಇವರೆಲ್ಲರಿಗಿಂತ ಮಿಗಿಲಾದ ಒಂದು ಮೂಲಸ್ಥಾನವಿದೆ. ಅದು ಎಲ್ಲಿದೆ? ಕ್ಷುದ್ರತಮ ಕೀಟದಿಂದ ಹಿಡಿದು ಶ್ರೇಷ್ಠತಮ ಮಾನವನವರೆಗೆ ಎಲ್ಲರ ಆತ್ಮನಲ್ಲಿರುವ ಭಗವಂತನ ಸನಾತನ ದೇವಾಲಯದಲ್ಲಿದೆ. ಅನಂತ ನಿಃಸ್ವಾರ್ಥ, ಅನಂತತ್ಯಾಗ ಅನಂತವನ್ನು ಸೇರುವ ಅನಂತ ಪ್ರೇರಣೆ ಎಲ್ಲರಲ್ಲೂ ಇವೆ.

ನಾವು ಅಜ್ಞಾನದಿಂದ ಕೇವಲ ತೋರಿಕೆಗೆ ಬೇರೆಯಾಗಿರುವೆವು, ಮಿತವಾಗಿರುವೆವು, ಇಂತಹ ಸ್ತ್ರೀಪುರುಷರಾಗಿರುವೆವು. ಪ್ರತಿಕ್ಷಣವೂ ಪ್ರಕೃತಿ ಈ ಭ್ರಾಂತಿಯನ್ನು ಅಲ್ಲಗಳೆಯುತ್ತಿದೆ. ಇತರರಿಂದ ಬೇರೆಯಾದ ಅಲ್ಪ ವ್ಯಕ್ತಿತ್ವದ ಸ್ತ್ರೀ ಅಥವಾ ಪುರುಷನಲ್ಲ ನಾನು. ನಾನೇ ವಿಶ್ವಾತ್ಮ. ಸ್ವಭಾವತಃ ಆತ್ಮವು ಪ್ರತಿಕ್ಷಣವೂ ತನ್ನ ಸ್ವಭಾವಸಿದ್ದ ಪವಿತ್ರತೆಯನ್ನು ಸಾರುತ್ತಿದೆ.

ಈ ವೇದಾಂತ ಎಲ್ಲಾ ಕಡೆಗಳಲ್ಲಿಯೂ ಇದೆ. ಆದರೆ ನಿಮ್ಮ ಅರಿವಿಗೆ ಅದು ಬರಬೇಕು ಅಷ್ಟೆ. ಈ ಮೂಢನಂಬಿಕೆಯ ರಾಶಿಯೇ ಪ್ರಗತಿಗೆ ಇರುವ ಆತಂಕ. ಸಾಧ್ಯವಾದರೆ ಅವನ್ನು ಆಚೆಗೆ ಒಗೆದುಬಿಡೋಣ. ದೇವರು ಆತ್ಮ; ಅವನನ್ನು ಆತ್ಮನಂತೆ, ಸತ್ಯದಂತೆ ಆರಾಧಿಸೋಣ. ಇನ್ನು ಮೇಲೆ ಜಡವಾದಿಗಳಾಗಬೇಡಿ. ಜಡವನ್ನೆಲ್ಲ ಆಚೆಗೆ ಒಗೆಯಿರಿ. ಭಗವಂತನ ಭಾವನೆ ನಿಜವಾಗಿಯೂ ಆಧ್ಯಾತ್ಮಿಕವಾಗಿರಬೇಕು. ಜಡವಾದದ ಭಾವನೆಗೆ ಸಂಬಂಧಪಟ್ಟ ದೇವರ ಆಲೋಚನೆಯೆಲ್ಲ ಹೋಗಬೇಕು. ಮನುಷ್ಯ ಹೆಚ್ಚು ಹೆಚ್ಚು ಆಧ್ಯಾತ್ಮಿಕನಾದಂತೆ ಇದನ್ನೆಲ್ಲ ಕಿತ್ತೊಗೆದು ಅವನು ಮುಂದೆ ಸಾಗಬೇಕು. ಪ್ರತಿಯೊಂದು ದೇಶದಲ್ಲಿಯೂ ಕೆಲವರು ಇರುವರು; ಭೌತವಸ್ತುವನ್ನೆಲ್ಲ ಕಿತ್ತೊಗೆಯುವ ಶಕ್ತಿ ಅವರಲ್ಲಿದೆ. ಆತ್ಮನನ್ನು ಆತ್ಮನ ಮೂಲಕ ಆರಾಧಿಸಿದ ಉಜ್ವಲ ಉದಾಹರಣೆಗಳವರು.

ಈ ವೇದಾಂತ, ಅಂದರೆ ಎಲ್ಲರೂ ಒಂದೇ ಆತ್ಮ ಎಂಬ ಭಾವನೆ ಹರಡಿದರೆ ಇಡಿಯ ಮಾನವಕೋಟಿ, ಆಧ್ಯಾತ್ಮಿಕ ಪ್ರಕೃತಿಯುಳ್ಳದ್ದಾಗುವುದು. ಆದರೆ ಇದು ಸಾಧ್ಯವೆ.? ನನಗೆ ಗೊತ್ತಿಲ್ಲ. ಅಂತೂ ಸಾವಿರಾರು ವರುಷಗಳಿಗಿಂತ ಮುಂಚೆ ಆಗುವಂತಿಲ್ಲ. ಹಳೆಯ ಮೂಢನಂಬಿಕೆಗಳು ನಾಶವಾಗಬೇಕಾಗಿದೆ. ನೀವೆಲ್ಲ ಆ ಮೂಢನಂಬಿಕೆಯನ್ನು ಉಳಿಸಿಕೊಂಡು ಬರುವುದಕ್ಕೆ ಆಸಕ್ತರಾಗಿರುವಿರಿ. ಅನಂತರ ಇವನು ನಮ್ಮ ಕುಲಕ್ಕೆ ಸೇರಿದವನು, ಜಾತಿಗೆ ಸೇರಿದವನು, ದೇಶಕ್ಕೆ ಸೇರಿದವನು ಎಂಬ ಭಾವನೆಗಳಿವೆ. ಇವೆಲ್ಲ ವೇದಾಂತದ ಆದರ್ಶಸಾಧನೆಗೆ ಆತಂಕಗಳು. ಧರ್ಮ ಎಲ್ಲೋ ಕೆಲವರಿಗೆ ಮಾತ್ರ ಧರ್ಮವಾಗಿ ಉಳಿದಿದೆ.

\newpage

ಪ್ರಪಂಚದಲ್ಲಿ ಧಾರ್ಮಿಕ ಕ್ಷೇತ್ರದಲ್ಲಿ ಕೆಲಸಮಾಡಿದವರೆಲ್ಲ ಬಹುಪಾಲು\break ರಾಜಕೀಯ ಮನೋಭಾವದವರು. ಇದೇ ಮಾನವಕೋಟಿಯ ಇತಿಹಾಸವಾಗಿದೆ. ಅವರಲ್ಲಿ ರಾಜಕೀಯ ಮನೋಭಾವವಿಲ್ಲದೆ, ಸತ್ಯದೊಂದಿಗೆ ಬಾಳಬೇಕೆಂದು ಯತ್ನಿಸಿದವರೇ ಬಹಳ ಅಪರೂಪ. ಅವರು ಯಾವಾಗಲೂ ಸಮಾಜವೆಂಬ ದೇವರನ್ನು ಪೂಜಿಸುತ್ತಿದ್ದರು. ಸಾಧಾರಣ ಜನರ ಮೂಢನಂಬಿಕೆ, ಅವರ ದುರ್ಬಲತೆ ಇವನ್ನೇ ಅವರು ಎತ್ತಿ ಹಿಡಿಯುತ್ತಿದ್ದರು. ಅವರು ಪ್ರಕೃತಿಯನ್ನು ಗೆಲ್ಲಲು ಪ್ರಯತ್ನಿಸಲಿಲ್ಲ. ಅದರೊಂದಿಗೆ ಹೊಂದಿಕೊಂಡು ಹೋಗಲು ಯತ್ನಿಸಿದರೇ ವಿನಃ ಬೇರೆಯಲ್ಲ. ನೀವು ಭರತಖಂಡಕ್ಕೆ ಹೋಗಿ ಹೊಸ ಮಾರ್ಗವೊಂದನ್ನು ಬೋಧಿಸಿ, ಅವರು ಅದನ್ನು ಕೇಳುವುದಿಲ್ಲ. ಆದರೆ ಅವರಿಗೆ ನೀವು ಇದು ವೇದದಲ್ಲಿದೆ ಎನ್ನಿ; ಆಗ ಅವರು ಇದು ಒಳ್ಳೆಯದು ಎನ್ನುತ್ತಾರೆ. ನಾನು ಇಲ್ಲಿ ಆ ಸಿದ್ದಾಂತವನ್ನೇ ಬೋಧಿಸಬಹುದು. ಆದರೆ ನಿಮ್ಮಲ್ಲಿ ಎಷ್ಟು ಜನರು ನಾನು ಹೇಳುತ್ತಿರುವುದು ಸತ್ಯ ಎಂದು ತಿಳಿಯಬಲ್ಲಿರಿ? ಆದರೆ ಇದು ಸತ್ಯ, ನಾನು ಸತ್ಯವನ್ನು ಹೇಳಬೇಕಾಗಿದೆ.

ಪ್ರಶ್ನೆಯ ಮತ್ತೊಂದು ಭಾಗವಿದೆ. ಶ್ರೇಷ್ಠವಾದ ಪರಮ ಸತ್ಯವನ್ನು ಎಲ್ಲರೂ ಏಕಕಾಲದಲ್ಲಿ ಸಾಕ್ಷಾತ್ಕಾರ ಮಾಡಿಕೊಳ್ಳಲಾರರು; ಅವರನ್ನು ಕ್ರಮೇಣ ಪೂಜೆ, ಪ್ರಾರ್ಥನೆ ಮುಂತಾದುವುಗಳ ಮೂಲಕ ಕರೆದುಕೊಂಡು ಹೋಗಬೇಕು ಎನ್ನುವರು. ಇದು ಸರಿಯಾದ ಮಾರ್ಗವೇ, ಅಲ್ಲವೆ ಎಂಬುದು ನನಗೆ ಗೊತ್ತಿಲ್ಲ. ನಾನು ಭರತಖಂಡದಲ್ಲಿ ಎರಡನ್ನೂ ಮಾಡುತ್ತೇನೆ. ದೇವರು, ವೇದ, ಬೈಬಲ್, ಕ್ರಿಸ್ತ, ಬುದ್ದ ಇವರ ಹೆಸರಿನಲ್ಲಿ ಕಲ್ಕತ್ತೆಯಲ್ಲಿ ವಿಗ್ರಹಗಳು ಮತ್ತು ದೇವಸ್ಥಾನಗಳು ಇವೆ. ಇದನ್ನು ಪ್ರಯತ್ನಿಸಿ ನೋಡೋಣ. ಆದರೆ ಹಿಮಾಲಯದಲ್ಲಿ ಒಂದು ಸ್ಥಳವಿದೆ. ಅಲ್ಲಿ ಸತ್ಯವಲ್ಲದೆ ಬಾಹ್ಯ ಪೂಜೆಗಳಾವುವೂ ಪ್ರವೇಶಿಸಕೂಡದೆಂದು ನಾನು ಸಂಕಲ್ಪ ಮಾಡಿರುವೆನು. ನಾನು ನಿಮಗೆ ಇಂದು ಹೇಳಿದ ವಿಷಯವನ್ನು ಅನುಷ್ಠಾನಕ್ಕೆ ತರಲು ಯತ್ನಿಸುವೆನು. ಇದು ಒಬ್ಬಳು ಆಂಗ್ಲ ಸ್ತ್ರೀ ಮತ್ತು ಒಬ್ಬ ಪುರುಷ ಇವರ ಸ್ವಾಧೀನದಲ್ಲಿದೆ. ಅಲ್ಲಿ ಭಗವಂತನ ಮಕ್ಕಳಾದ ಸತ್ಯೋಪಾಸಕರು ಯಾವ ಅಂಜಿಕೆ ಇಲ್ಲದೆ ಮತ್ತು ಮೂಢನಂಬಿಕೆಯ ಭಯವಿಲ್ಲದೆ ಮುಂದುವರಿಯಲಿ ಎಂಬುದೇ ನನ್ನ ಉದ್ದೇಶ. ಕ್ರಿಸ್ತ ಬುದ್ದ ಶಿವ ವಿಷ್ಣು ಇವರ ಹೆಸರನ್ನೇ ಅವರು ಕೇಳಕೂಡದು ಅಲ್ಲಿ. ಅಲ್ಲಿ ಪ್ರಾರಂಭದಿಂದಲೂ ಸತ್ಯದ ಆಸರೆಯ ಮೇಲೆ ನಿಲ್ಲುವುದನ್ನು ಕಲಿಯಬೇಕು. ಅಲ್ಲಿ ಬಾಲ್ಯಾರಭ್ಯದಿಂದಲೇ ದೇವರು ಆತ್ಮ, ಅವನನ್ನು ಆತ್ಮನಂತೆ ಸತ್ಯದ ಮೂಲಕ ಮಾತ್ರ ಪೂಜಿಸಬೇಕೆಂಬುದನ್ನು ಕಲಿಯುವರು. ಪ್ರತಿಯೊಬ್ಬರನ್ನೂ ಆತ್ಮ ದೃಷ್ಟಿಯಿಂದ ನೋಡಬೇಕು. ಇದೇ ನನ್ನ ಆದರ್ಶ. ಇದು ಎಷ್ಟರ ಮಟ್ಟಿಗೆ ಜಯಪ್ರದವಾಗುವುದೊ ನಾನು ಅರಿಯೆ. ನಾನಿಂದು ನನ್ನ ಇಚ್ಛೆಗೆ ಅನುಸಾರವಾಗಿ ಬೋಧಿಸುತ್ತಿರುವೆನು. ದ್ವೈತಭಾವನೆಗಳಿಲ್ಲದೆ ಬಾಲ್ಯದಿಂದಲೇ ನಾನು ಈ ವಾತಾವರಣದಲ್ಲಿಯೇ ಬೆಳೆದಿದ್ದರೆ ಎಷ್ಟು ಚೆನ್ನಾಗಿರುತ್ತಿತ್ತು!

\newpage

ಕೆಲವು ವೇಳೆ ದ್ವೈತಧರ್ಮದಲ್ಲಿ ಕೆಲವು ಒಳ್ಳೆಯ ಅಂಶಗಳಿವೆ ಎಂಬುದನ್ನು ಒಪ್ಪಿಕೊಳ್ಳುತ್ತೇನೆ. ಅನೇಕ ದುರ್ಬಲರಿಗೆ ಇವು ಸಹಾಯ ಮಾಡುವುವು. ನಿಮಗೆ ಯಾರಾದರೂ ಅರುಂಧತಿ ನಕ್ಷತ್ರವನ್ನು ತೋರು ಎಂದರೆ ನೀವು ಮೊದಲು ಪ್ರಕಾಶಮಾನವಾದ ನಕ್ಷತ್ರವನ್ನು ತೋರಿ, ಅದಾದ ಮೇಲೆ ಮಂದಕಾಂತಿಯ ತಾರೆಯೊಂದನ್ನು ತೋರಿ, ಕೊನೆಗೆ ಅರುಂಧತಿ ನಕ್ಷತ್ರವನ್ನು ತೋರುವಿರಿ. ಇದರಿಂದ ಅವನಿಗೆ ಅರುಂಧತಿ ನಕ್ಷತ್ರವನ್ನು ನೋಡಲು ಸಾಧ್ಯವಾಗುವುದು. ಬೈಬಲ್ಲು, ಸಗುಣ ದೇವರುಗಳು, ಇವೆಲ್ಲ ಧರ್ಮದ ಪ್ರಾಥಮಿಕ ಹಂತಗಳು, ಕಿಂಡರ್ ಗಾರ್ಟನ್ ಸ್ಕೂಲು ಇದ್ದಂತೆ, ಶಿಶುವಿಹಾರಗಳಂತೆ.

ಆದರೆ ನಾನು ಮತ್ತೊಂದು ವಿಷಯವನ್ನು ಕುರಿತು ಆಲೋಚಿಸುತ್ತೇನೆ. ಈ ನಿಧಾನವಾದ, ಕ್ರಮವಾದ, ಮಾರ್ಗವನ್ನು ಅನುಸರಿಸುತ್ತಿದ್ದರೆ ಜನರು ಯಾವಾಗ ನಿಜವಾದ ಸತ್ಯವನ್ನು ಸೇರುವುದು? ಎಷ್ಟು ಕಾಲ ತೆಗೆದುಕೊಳ್ಳುವುದು? ಈ ಮಾರ್ಗ ಸ್ವಲ್ಪವಾದರೂ ಯಶಸ್ವಿಯಾಗುವುದೆಂದು ಧೈರ್ಯವಾಗಿ ಹೇಳುವುದು ತಾನೆ ಹೇಗೆ? ಇಂದಿನವರೆಗೆ ಅದು ಮುಂದುವರಿದಿಲ್ಲ. ಕ್ರಮಕ್ರಮವಾಗಿಯೊ ಇಲ್ಲವೊ, ದುರ್ಬಲರಿಗೆ ಇದು ಸಾಧ್ಯವೊ ಇಲ್ಲವೊ, ಏನಾದರೂ ಆಗಲಿ ಈ ದ್ವೈತಮಾರ್ಗ ಅಸತ್ಯದ ಮೇಲೆ ತಾನೆ ನಿಂತಿರುವುದು? ಈಗಿರುವ ಧಾರ್ಮಿಕ ನಿಯಮಾವಳಿಗಳೆಲ್ಲ ನಮ್ಮನ್ನು ದುರ್ಬಲರನ್ನಾಗಿ ಮಾಡುತ್ತಿಲ್ಲವೆ? ಆದಕಾರಣ ಇವು ತಪ್ಪಲ್ಲವೆ? ಇವು ಮನುಷ್ಯನ ತಪ್ಪು ಭಾವನೆಯ ಮೇಲೆ ನಿಂತಿವೆ. ಎರಡು ತಪ್ಪುಗಳು ಒಂದು ಸರಿಯನ್ನು ಮಾಡಬಲ್ಲವೆ? ಸುಳ್ಳು ಸತ್ಯವಾಗಬಲ್ಲದೆ? ಕತ್ತಲೆ ಬೆಳಕಾಗಬಲ್ಲದೆ?

ಈಗ ದೇಹತ್ಯಾಗ ಮಾಡಿರುವ ವ್ಯಕ್ತಿಯೊಬ್ಬರ ದಾಸ ನಾನು. ನಾನು ಅವರ ಸಂದೇಶವನ್ನು ಒಯ್ಯುವವನು ಮಾತ್ರ. ನಾನು ಈ ಪ್ರಯೋಗವನ್ನು ಮಾಡಬೇಕೆಂದು ಇರುವೆನು. ನಾನು ನಿಮಗೆ ಹೇಳಿದ ವೇದಾಂತದ ಉಪದೇಶವನ್ನು ಇದುವರೆಗೆ ಅನುಷ್ಠಾನಕ್ಕೆ ತರಲು ಪ್ರಯತ್ನಿಸಿರಲೇ ಇಲ್ಲ. ಪ್ರಪಂಚದಲ್ಲಿ ಈ ವೇದಾಂತ ಅತಿ ಪುರಾತನ ತತ್ತ್ವವಾದರೂ ಅದು ಮೂಢನಂಬಿಕೆಗಳೊಡನೆ ಮಿಶ್ರವಾಗಿ ಹೋಗಿದೆ.

ಕ್ರಿಸ್ತ `ನಾನು ಮತ್ತು ನನ್ನ ತಂದೆ ಇಬ್ಬರೂ ಒಂದೇ' ಎಂದನು. ನೀವು ಅದನ್ನು ಉಚ್ಚರಿಸುತ್ತಿರುವಿರಿ. ಆದರೂ ಇದು ಮಾನವಕೋಟಿಗೆ ಸಹಾಯಮಾಡಲಿಲ್ಲ. ಕಳೆದ ಹತ್ತೊಂಬತ್ತುನೂರು ವರುಷಗಳಿಂದ ಜನರು ಅದನ್ನು ಅರ್ಥಮಾಡಿಕೊಂಡಿಲ್ಲ. ಮಾನವನನ್ನು ಉದ್ದರಿಸುವವನು ಕ್ರಿಸ್ತ ಎನ್ನುವರು. ಅವನು ಮಾತ್ರ ದೇವರು, ನಾವೆಲ್ಲ ಕೀಟಗಳು. ಇದರಂತೆಯೇ ಭಾರತ ಮತ್ತು ಇತರ ದೇಶಗಳಲ್ಲಿ ಕೂಡ. ಇಂತಹ ಭಾವನೆಗಳೇ ಪ್ರತಿಯೊಂದು ಪಂಥದ ಮೂಲಾಧಾರವಾಗಿವೆ. ಕಳೆದ ಸಹಸ್ರಾರು ವರುಷಗಳಿಂದ ಕೋಟ್ಯಂತರ ಜನರಿಗೆ ಸಗುಣ ದೇವರನ್ನು, ಅವತಾರಗಳನ್ನು, ದೇವದೂತರುಗಳನ್ನು, ಪ್ರವಾದಿಗಳನ್ನು ಪೂಜಿಸಿ ಎಂದು ಹೇಳುತ್ತಿರುವರು. ಕೆಲಸಕ್ಕೆ ಬಾರದ ವ್ಯಕ್ತಿಗಳು ತಾವು ತಮ್ಮ ಮುಕ್ತಿಗಾಗಿ ಯಾರ ಕೃಪೆಗೋ ಕಾದು ಕುಳಿತಿರಬೇಕು ಎಂದು ಅವರಿಗೆ ಬೋಧಿಸಿದ್ದಾಗಿದೆ. ಅಂತಹವುಗಳಲ್ಲಿ ಕೂಡ ಅನೇಕ ಉತ್ತಮವಾದ ಅಂಶಗಳಿವೆ. ಅವುಗಳ ಶ್ರೇಷ್ಠ ಸ್ಥಿತಿಯಲ್ಲಿಯೂ ಅವೆಲ್ಲ ಇನ್ನೂ ಧಾರ್ಮಿಕ ಶಿಶುವಿಹಾರಗಳು, ಇವುಗಳಿಂದ ಸಹಾಯ ಆಗಿರುವುದು ಬಹಳ ಸ್ವಲ್ಪ, ಮಾನವರು ತಾವು ಅತಿ ಹೀನವಾದ ಅವನತಿಯಲ್ಲಿ ಸಿಕ್ಕಿಬಿದ್ದಿರುವೆವು ಎಂದು ಭ್ರಾಂತರಾಗಿರುವರು. ಆದರೂ ಕೆಲವು ಧೀರಾತ್ಮರಿರುವರು, ಆ ಭ್ರಮೆಗಳನ್ನೆಲ್ಲಾ ಕಿತ್ತೊಗೆಯುತ್ತಾರೆ. ಈ ಧರ್ಮದ ಶಿಶುವಿಹಾರಗಳನ್ನೆಲ್ಲಾ ತ್ಯಜಿಸುವ ಮಹಾನ್ ವ್ಯಕ್ತಿಗಳು ಉದಿಸುವ ಕ್ಷಣವೊಂದು ಬರುತ್ತದೆ. ಆತ್ಮವನ್ನು ಆತ್ಮದಿಂದ ಪೂಜಿಸುವ ಸತ್ಯವಾದ ಧರ್ಮವನ್ನು ಅವರು ಸ್ಪಷ್ಟವೂ ಶಕ್ತಿಯುತವೂ ಆದುದನ್ನಾಗಿ ಮಾಡುತ್ತಾರೆ.

