
\chapter[ಕರ್ಮ ರಹಸ್ಯ]{ಕರ್ಮ ರಹಸ್ಯ\protect\footnote{\engfoot{C.W, Vol. II, P. 1}}}

\begin{center}
(೧೯೦೦ರ ಜನವರಿ ೪ನೇ ತಾರೀಖು ಲಾಸ್ ಏಂಜಲೀಸ್‌ನಲ್ಲಿ ನೀಡಿದ ಪ್ರವಚನ)
\end{center}

ನನ್ನ ಜೀವನದಲ್ಲಿ ಕಲಿತ ಅತ್ಯುತ್ತಮ ಪಾಠವೊಂದು, ಗುರಿಗೆ ಕೊಡುವ\break ಪ್ರಾಮುಖ್ಯತೆಯನ್ನೇ ದಾರಿಗೂ ಕೊಡಬೇಕೆನ್ನುವುದು. ನಾನು ಯಾರಿಂದ ಇದನ್ನು\break ಕಲಿತೆನೋ ಆತನೊಬ್ಬ ಮಹಾನುಭಾವ. ಅವರ ಇಡೀ ಜೀವನವೇ ಈ ಸಿದ್ಧಾಂತಕ್ಕೆ ಒಂದು ಉದಾಹರಣೆಯಾಗಿತ್ತು. ಈ ಒಂದು ಮಹಾಸಿದ್ಧಾಂತದಿಂದ ನಾನು ಹಲವು ಒಳ್ಳೆ ನೀತಿಗಳನ್ನು ಅಂದಿನಿಂದ ಕಲಿಯುತ್ತಿರುವೆನು. ಗುರಿಗೆ ಕೊಡುವಷ್ಟು ಗಮನವನ್ನೇ ದಾರಿಗೂ ಕೊಡುವುದರಲ್ಲಿ ಜಯದ ರಹಸ್ಯವೆಲ್ಲ ಇರುವಂತೆ ನನಗೆ ತೋರುವುದು.

ಜೀವನದಲ್ಲಿ ನಮ್ಮ ಒಂದು ದೊಡ್ಡ ದೋಷವೇ, ಗುರಿಯ ಕಡೆಗೆ ನಾವು ಹೆಚ್ಚು ಆಕರ್ಷಿತರಾಗುವುದು; ಗುರಿ ನಮ್ಮನ್ನು ಮೋಹಪರವಶರನ್ನಾಗಿ ಮಾಡುವುದು; ಮುಗ್ಧರನ್ನಾಗಿ ಮಾಡುವುದು; ಅದೇ ನಮ್ಮ ಮನಸ್ಸನ್ನೆಲ್ಲಾ ವ್ಯಾಪಿಸುವುದು. ಆಗ ಅದನ್ನು ಪಡೆಯುವ ವಿವರಗಳೆಲ್ಲಾ ಮರತೇ ಹೋಗುವುವು.

ಆದರೆ ನಮ್ಮ ಜೀವನದಲ್ಲಿ ಯಾವಾಗಲಾದರೂ ಸೋಲು ಬಂದಾಗ ಅದನ್ನು ಕುರಿತು ಕೂಲಂಕಷವಾಗಿ ವಿಶ್ಲೇಷಿಸಿದರೆ, ನೂರಕ್ಕೆ ತೊಂಬತ್ತೊಂಬತ್ತು ಕಡೆ ಅದಕ್ಕೆ ಕಾರಣ ದಾರಿಯ ಕಡೆಗೆ ಗಮನ ಕೊಡದೆ ಇದ್ದುದು ಎಂದು ನಮಗೆ ಗೊತ್ತಾಗುವುದು. ಅದನ್ನು ಪೂರ್ಣಗೊಳಿಸುವ ಕಡೆಗೆ, ಮಾರ್ಗವನ್ನು ಹೇಗೆ ಸುಗಮ ಮಾಡಬೇಕೆಂಬುದರ ಕಡೆಗೆ, ಸಾಕಾದಷ್ಟು ಗಮನ ಕೊಡುವುದು ನಮಗೆ ಆವಶ್ಯಕವಾಗಿರುವುದು. ಮಾರ್ಗ ಸರಿಯಾಗಿದ್ದರೆ ಗುರಿ ಸಿದ್ದಿಸಲೇಬೇಕು. ಕಾರ್ಯವನ್ನು ಸಾಧ್ಯಗೊಳಿಸುವುದು ಕಾರಣವೇ ಎಂಬುದನ್ನು ನಾವು ಮರೆಯುತ್ತೇವೆ. ಪರಿಣಾಮ ತನಗೆ ತಾನೆ ಬರಲಾರದು. ಕಾರಣಗಳು ಸರಿಯಾಗಿ, ಯೋಗ್ಯವಾಗಿ ಪ್ರಬಲವಾಗಿಲ್ಲದೆ ಇದ್ದರೆ ಯೋಗ್ಯ ಪರಿಣಾಮ, ಪ್ರತಿಫಲ ದೊರಕಲಾರವು. ಒಮ್ಮೆ ನಾವು ಆದರ್ಶವನ್ನು ಆರಿಸಿಕೊಂಡು, ಅದನ್ನು ಪಡೆಯುವ ರೀತಿಯನ್ನು ನಿರ್ಧರಿಸಿದ ಮೇಲೆ ಆದರ್ಶವನ್ನು ಬೇಕಾದರೆ ಸ್ವಲ್ಪ ಮರೆತರೂ ಚಿಂತೆ ಇಲ್ಲ. ಏಕೆಂದರೆ ಅದನ್ನು ಪಡೆಯುವ ಸಾಧನೆ ಪೂರ್ಣವಾದರೆ ಗುರಿ ಅಲ್ಲೇ ಇರುವುದು ನಮಗೆ ನಿಸ್ಸಂದೇಹವಾಗಿ ಗೊತ್ತಿದೆ. ಅದಕ್ಕೆ ಕಾರಣವಿದ್ದರೆ ಪರಿಣಾಮಕ್ಕೆ ಯಾವ ಒಂದು ತೊಡಕೂ ಇರುವುದಿಲ್ಲ. ಪರಿಣಾಮ ಬರಲೇಬೇಕಾಗಿದೆ. ನಾವು ಸಾಧನವನ್ನು ಗಮನಿಸಿದರೆ, ಸಿದ್ದಿ ತನ್ನನ್ನು ತಾನೇ ನೋಡಿಕೊಳ್ಳುವುದು. ನಮ್ಮ ಆದರ್ಶ ಸಾಕ್ಷಾತ್ಕಾರ ಸಿದ್ದಿ. ಅದಕ್ಕೆ ಮಾರ್ಗ ಸಾಧನೆ. ಆದಕಾರಣ ಸಾಧನೆಯ ಪ್ರಾಮುಖ್ಯತೆಯೆ ಜೀವನದ ಅತಿ ದೊಡ್ಡ ರಹಸ್ಯ. ನಾವು ಕರ್ಮ ಮಾಡಬೇಕು. ಎಡೆಬಿಡದೆ ನಮ್ಮ ಶಕ್ತಿಯನ್ನೆಲ್ಲ ಉಪಯೋಗಿಸಿ ಕರ್ಮ ಮಾಡಬೇಕು. ನಾವು ಮಾಡುವ ಎಲ್ಲಾ ಕಾರ್ಯದಲ್ಲಿ ನಮ್ಮ ಇಡೀ ಮನಸ್ಸನ್ನು ಅಲ್ಲಿ ಏಕಾಗ್ರಮಾಡಬೇಕೆಂದು ಗೀತೆಯಲ್ಲಿ ಓದುತ್ತೇವೆ. ಆದರೆ ನಾವು ಅದೇ ಕಾಲದಲ್ಲಿ ಆಸಕ್ತರಾಗಿರಬಾರದು. ಹಾಗೆಂದರೆ, ನಾವು ಮತ್ತೆ ಯಾವ ಆಕರ್ಷಣಕ್ಕೂ ಒಳಗಾಗಿ ಕೆಲಸದಿಂದ ದೂರ ಸರಿಯಬಾರದು; ಆದರೂ, ಕೆಲಸವನ್ನು ನಾವು ಇಚ್ಛಿಸಿದಾಗ ಬಿಡಲು ಸಿದ್ದರಾಗಿರಬೇಕು.

ನಮ್ಮ ಜೀವನವನ್ನು ನಾವು ಪರೀಕ್ಷೆ ಮಾಡಿಕೊಂಡರೆ, ನಮ್ಮ ದುಃಖಕ್ಕೆ ಅತಿ ದೊಡ್ಡ ಕಾರಣ ಇದು ಎಂದು ಗೊತ್ತಾಗುವುದು: ನಾವು ಯಾವುದಾದರೂ ಕಾರ್ಯದಲ್ಲಿ ತತ್ಪರರಾಗುತ್ತೇವೆ. ನಮ್ಮ ಮನಸ್ಸನ್ನೆಲ್ಲ ಅದರ ಮೇಲೆ ಬಿಡುತ್ತೇವೆ. ಆ ಕೆಲಸ ನೆರವೇರದೆ ಇದ್ದರೆ ಆಗಲೂ ಅದನ್ನು ಬಿಡುವುದಿಲ್ಲ. ಅದರಿಂದ ನಮಗೆ ವ್ಯಥೆಯಾಗುತ್ತಿದೆ ಎಂದು ಗೊತ್ತಿದೆ. ಹೆಚ್ಚುಕಾಲ ಅದರಲ್ಲಿ ಆಸಕ್ತರಾಗಿರುವುದರಿಂದ ದುಃಖ ಮತ್ತೂ ಹೆಚ್ಚಾಗುವುದು ಎಂಬುದೂ ಗೊತ್ತಿದೆ. ಆದರೂ ಅದರಿಂದ ನಾವು ಬಿಡಿಸಿಕೊಳ್ಳುವುದು ಅಸಾಧ್ಯ. ದುಂಬಿ ಜೇನನ್ನು ಆಸ್ವಾದಿಸಲು ಜೇನಿನ ಮಡಕೆಗೆ ಬಂದು ರೆಕ್ಕೆಗಳು ಜೇನಿಗೆ ಅಂಟಿಕೊಂಡು ಅದರಿಂದ ಪಾರಾಗಲು ಅಸಾಧ್ಯವಾಯಿತು. ಪುನಃ ಪುನಃ ನಾವು ಅದೇ ಸ್ಥಿತಿಗೆ ಬರುವೆವು. ಇದೇ ಜೀವನದ ರಹಸ್ಯ. ನಾವು ಇಲ್ಲಿರುವುದು ಏತಕ್ಕೆ? ನಾವು ಮಧುವನ್ನು ಕುಡಿಯಲು ಬಂದೆವು. ಆದರೆ ನಮ್ಮ ಕೈಕಾಲುಗಳು ಅದರಲ್ಲಿ ಸಿಕ್ಕಿಕೊಂಡಿರುವುದು ಕಾಣುವುದು. ನಾವು ಆನಂದವನ್ನು ಅನುಭವಿಸುವುದಕ್ಕೆ ಬಂದೆವು. ಆದರೆ ಮತ್ತಾರೊ ನಮ್ಮಿಂದ ಆನಂದವನ್ನು ಅನುಭವಿಸುತ್ತಿರುವರು. ನಾವು ಆಳುವುದಕ್ಕೆ ಬಂದೆವು, ಆದರೆ ಮತ್ತಾರೂ ನಮ್ಮನ್ನು ಆಳುತ್ತಿರುವರು. ನಾವು ಕೆಲಸ ಮಾಡುವುದಕ್ಕೆ ಬಂದೆವು, ಆದರೆ ಮತ್ತಾರೊ ನಮ್ಮಿಂದ ಅದನ್ನು ಮಾಡಿಸುತ್ತಿರುವರು. ಯಾವಾಗಲೂ ನಮಗೆ ಇದು ಕಾಣುವುದು. ನಮ್ಮ ಜೀವನದ ಪ್ರತಿಯೊಂದು ಕಾರ್ಯಕ್ಷೇತ್ರದಲ್ಲಿಯೂ ಇದನ್ನು ನೋಡುತ್ತೇವೆ. ಅನ್ಯರ ಮನಸ್ಸು ನಮ್ಮ ಮೇಲೆ ಕೆಲಸಮಾಡುತ್ತಿದೆ. ನಾವು ಮತ್ತೊಬ್ಬರ ಮನಸ್ಸಿನ ಮೇಲೆ ಕೆಲಸಮಾಡುವುದಕ್ಕೆ ಪ್ರಯತ್ನಿಸುತ್ತಿರುವೆವು. ನಾವು ಈ ಜೀವನದ ಸುಖವನ್ನು ಅನುಭವಿಸಬೇಕೆಂದು ಬಯಸುವೆವು. ಇದೇ ನಮ್ಮ ಜೀವನವನ್ನು ನಾಶಮಾಡುವುದು. ಪ್ರಕೃತಿಯಿಂದ ಎಲ್ಲವನ್ನೂ ಪಡೆಯಬಯಸುತ್ತೇವೆ. ಆದರೆ ಪ್ರಕೃತಿ ಎಲ್ಲವನ್ನೂ ಅಪಹರಿಸಿ, ನಮ್ಮನ್ನು ಸುಲಿದು ಆಚೆಗೆಸೆಯುವುದು.

ಇದು ಹೀಗಲ್ಲದೆ ಇದ್ದಿದ್ದರೆ ನಮ್ಮ ಜೀವನವೆಲ್ಲ ಸುಖಕರವಾಗುತ್ತಿತ್ತು. ಆದರೂ ಚಿಂತೆಯಿಲ್ಲ; ಅದರಲ್ಲಿ ಎಷ್ಟೇ ಜಯಾಪಜಯಗಳಿದ್ದರೂ, ಎಷ್ಟೇ ಸುಖದುಃಖಗಳಿದ್ದರೂ, ನಾವು ಬಲೆಗೆ ಬೀಳದೆ ಇದ್ದರೆ ಅದೊಂದು ಅಖಂಡ ಆನಂದ ಮಯವಾಗುವುದು.

ದುಃಖಕ್ಕೆ ಇದೊಂದೇ ಕಾರಣ; ನಾವು ಆಸಕ್ತರು. ಅದರಿಂದ ಬಲೆಗೆ ಬೀಳುತ್ತೇವೆ. ಅದಕ್ಕೇ ಗೀತೆ ಹೀಗೆ ಹೇಳುವುದು: ಎಡೆಬಿಡದೆ ಕರ್ಮ ಮಾಡಿ, ಆದರೆ ಆಸಕ್ತರಾಗಬೇಡಿ, ಬಲೆಗೆ ಬೀಳಬೇಡಿ. ಎಲ್ಲದರಿಂದಲೂ ನಿರ್ಲಿಪ್ತರಾಗುವ ಶಕ್ತಿಯನ್ನು ಮಾತ್ರ ಮೀಸಲಾಗಿಟ್ಟುಕೊಂಡಿರಿ. ಯಾವುದು ನಿಮಗೆ ಎಷ್ಟೇ ಪ್ರಿಯಕರವಾಗಿರಲಿ, ನಿಮ್ಮ ಜೀವ ಯಾವುದನ್ನೇ ಕುರಿತು ಹಂಬಲಿಸಲಿ, ಅದನ್ನು ಬಿಡುವುದರಿಂದ ಅತಿದಾರುಣ ವ್ಯಥೆಯಾಗಲಿ, ನಿಮಗೆ ಇಚ್ಛೆ ಬಂದಾಗ ಅದನ್ನು ತ್ಯಜಿಸುವ ಶಕ್ತಿಯನ್ನು ಮಾತ್ರ ಕಾಯ್ದಿಟ್ಟುಕೊಂಡಿರಿ. ದುರ್ಬಲರಿಗೆ ಈ ಜನ್ಮದಲ್ಲಿ ಆಗಲಿ ಅಥವಾ ಮತ್ತಾವುದಾದರೂ ಜನ್ಮದಲ್ಲಿ ಆಗಲಿ ಇಲ್ಲಿ ಸ್ಥಳವಿಲ್ಲ. ದುರ್ಬಲತೆಯೇ ಗುಲಾಮಗಿರಿಗೆ ದಾರಿ. ಹಲವಾರು ಶಾರೀರಿಕ ಮತ್ತು ಮಾನಸಿಕ ವ್ಯಥೆಗೆ ದುರ್ಬಲತೆಯೇ ಕಾರಣ. ದುರ್ಬಲತೆಯೆ ಸಾವು. ರೋಗಕಾರಕ ಸೂಕ್ಷ್ಮಕ್ರಿಮಿಗಳು ನಮ್ಮ ಸುತ್ತಲೂ ಸಹಸ್ರಾರು ಆವರಿಸಿಕೊಂಡಿವೆ. ಆದರೆ ನಾವು ದುರ್ಬಲರಾಗುವವರೆಗೆ, ದೇಹ ಸಿದ್ಧವಾಗಿ ಅವನ್ನು ಸ್ವಾಗತಿಸಲು ಸಂಕಲ್ಪಿಸುವವರೆಗೆ, ಅವು ನಮಗೆ ಯಾವ ಅಪಾಯವನ್ನೂ ತರಲಾರವು. ನಮ್ಮ ಸುತ್ತಲೂ ಕೋಟ್ಯಂತರ ದುಃಖಕಾರಕ ಕ್ರಿಮಿಗಳು ಹಾರಾಡುತ್ತಿರಬಹುದು. ಆದರೂ ಚಿಂತೆಯಿಲ್ಲ! ಅವು ನಮ್ಮ ಹತ್ತಿರ ಬರಲಾರವು. ಮನಸ್ಸು ದುರ್ಬಲವಾಗುವವರೆಗೆ, ನಮ್ಮನ್ನು ಮೆಟ್ಟಿಕೊಳ್ಳುವ ಶಕ್ತಿ ಅವಕ್ಕೆ ಇಲ್ಲ. ಇದೊಂದು ದೊಡ್ಡ ಸತ್ಯ; ಶಕ್ತಿಯೇ ಜೀವನ, ದುರ್ಬಲತೆಯೇ ಮರಣ. ಶಕ್ತಿಯೇ ಪರಮಾನಂದ, ಅಖಂಡಜೀವನ, ಅಮರತ್ವ. ದುರ್ಬಲತೆಯೇ ಅನವರತ ದುಃಖ, ವ್ಯಾಕುಲತೆ, ದುರ್ಬಲತೆಯೇ ಮರಣ..

\vskip 2pt

ಈಗ ಆಸಕ್ತಿಯೇ ನಮ್ಮ ಎಲ್ಲಾ ಸಂತೋಷಕ್ಕೂ ಮೂಲ. ನಾವು ನಮ್ಮ ಸ್ನೇಹಿತರಲ್ಲಿ ಆಸಕ್ತರು, ಬಂಧುಗಳಲ್ಲಿ ಆಸಕ್ತರು; ನಾವು ನಮ್ಮ ಬೌದ್ದಿಕ ಮತ್ತು ಆಧ್ಯಾತ್ಮಿಕ ಫಲಗಳಲ್ಲಿ ಆಸಕ್ತರು. ಬಾಹ್ಯ ವಸ್ತುವಿನಲ್ಲಿ ನಾವು ಆಸಕ್ತರು. ಏಕೆಂದರೆ ಅವುಗಳ ಮೂಲಕ ನಮಗೆ ಸಂತೋಷ ದೊರಕುವುದು. ಪುನಃ, ನಮಗೆ ದುಃಖ ತರುವುದು ಈ ಆಸಕ್ತಿಯಲ್ಲದೆ ಮತ್ತಾವುದು? ಆನಂದವನ್ನು ಪಡೆಯಬೇಕಾದರೆ ನಾವು ನಿರ್ಲಿಪ್ತರಾಗಬೇಕು. ಇಚ್ಛೆ ಬಂದಾಗ ನಮಗೆ ನಿರ್ಲಿಪ್ತರಾಗುವ ಶಕ್ತಿ ಇದ್ದರೆ ನಮಗೆ ಯಾವ ದುಃಖವೂ ಬರುತ್ತಿರಲಿಲ್ಲ. ಯಾರು ಒಂದು ಕಾರ್ಯದಲ್ಲಿ ತಮ್ಮ ಮನಸ್ಸನ್ನೆಲ್ಲಾ ಕೇಂದ್ರೀಕರಿಸಿ ಆಸಕ್ತರಾಗಬಲ್ಲರೋ, ಹಾಗೆಯೇ ಅವಶ್ಯಕವಾದಾಗ ಅದರಿಂದ ನಿರ್ಲಿಪ್ತರಾಗುವ ಶಕ್ತಿಯಿರುವುದೋ ಅವರಿಗೆ ಮಾತ್ರ ಪ್ರಪಂಚದ ಅತ್ಯುತ್ತಮ ಫಲ ಸಿಕ್ಕಬಲ್ಲದು. ಆಸಕ್ತರಾಗುವ ಶಕ್ತಿಯಷ್ಟೆ, ಆವಶ್ಯಕವಾದಾಗ, ಅನಾಸಕ್ತರಾಗುವ ಶಕ್ತಿಯೂ ಇರಬೇಕು. ಇದೇ ಕಷ್ಟ. ಯಾವ\break ಆಕರ್ಷಣೆಗೂ ಸಿಕ್ಕದ ಕೆಲವು ವ್ಯಕ್ತಿಗಳು ಇರುವರು. ಅವರು ಯಾರನ್ನೂ ಪ್ರೀತಿಸಲಾರರು, ನಿಷ್ಕರುಣಿಗಳು, ನಿರ್ಲಕ್ಷ್ಯರು. ಜೀವನದ ಎಷ್ಟೋ ದುಃಖಗಳಿಂದ ಅಂತಹವರು ಪಾರಾಗುವರು. ಗೋಡೆ ಎಂದಿಗೂ ದುಃಖಪಡುವುದಿಲ್ಲ, ಯಾವಾಗಲೂ ಪ್ರೀತಿಸುವುದಿಲ್ಲ, ಎಂದಿಗೂ ವ್ಯಥೆಪಡುವುದಿಲ್ಲ. ಆದರೆ ಅದು ಗೋಡೆಯಲ್ಲದೆ ಮತ್ತೇನೂ ಅಲ್ಲ. ನಿಶ್ಚಯವಾಗಿ ಆಸಕ್ತರಾಗಿ ಬಂಧನಕ್ಕೊಳಗಾಗುವುದು ಗೋಡೆಯಂತಿರುವುದಕ್ಕಿಂತ ಮೇಲು. ಆದಕಾರಣವೇ, ಯಾರು ಎಂದಿಗೂ ಮತ್ತೊಬ್ಬನನ್ನು ಪ್ರೀತಿಸಿಲ್ಲವೋ, ನಿರ್ದಯನಾಗಿರುವನೋ, ಅವನು ಜೀವನದ ಎಷ್ಟೋ ದುಃಖದಿಂದ ಪಾರಾಗಿರುವಂತೆ, ಸುಖವೂ ಅವನ ಪಾಲಿಗೆ ಇಲ್ಲದೇ ಹೋಗುವುದು. ನಮಗೆ ಅದು ಬೇಕಾಗಿರುವುದು. ಅದು ದುರ್ಬಲತೆ, ಅದು ಮೃತ್ಯು. ಯಾವ ಜೀವ ದುರ್ಬಲತೆಯನ್ನು ಅನುಭವಿಸಿಲ್ಲವೋ, ದುಃಖವನ್ನು ಅನುಭವಿಸಿಲ್ಲವೊ, ಅದೊಂದು ಔದಾಸೀನ್ಯ ಸ್ಥಿತಿ. ಅಲ್ಲಿ ಆತ್ಮ ಇನ್ನೂ ಜಾಗ್ರತವಾಗಿಲ್ಲ. ನಮಗೆ ಅದು ಬೇಕಾಗಿಲ್ಲ.

\vskip 2pt

ನಮಗೆ ಪ್ರೀತಿಯ ಅದ್ಭುತ ಶಕ್ತಿ, ಆಸಕ್ತರಾಗುವ ಪ್ರಚಂಡ ಶಕ್ತಿ, ಮನಸ್ಸನ್ನು ಒಂದು ವಸ್ತುವಿನಮೇಲೆ ಅಖಂಡವಾಗಿ ಏಕಾಗ್ರಮಾಡುವ ಶಕ್ತಿ, ಉಳಿದವರ ಹಿತಕ್ಕಾಗಿ ನಮ್ಮನ್ನೇ ಕಳೆದುಕೊಂಡು, ಸಾಯಲು ಸಿದ್ಧರಾಗಿರುವ, ದೇವತೆಗಳ ಶಕ್ತಿಗಿಂತ ಹೆಚ್ಚು ಶಕ್ತಿ ಬೇಕು. ನಾವು ದೇವತೆಗಳಿಗಿಂತ ಮೇಲೆ ಹೋಗಬೇಕು. ಪೂರ್ಣಾತ್ಮನು ತಾನು ಪ್ರೀತಿಸುವ ಏಕಮಾತ್ರ\break ವಸ್ತುವಿನ ಮೇಲೆ ತನ್ನ ಮನಸ್ಸನ್ನೆಲ್ಲಾ ಹರಿಸಬಲ್ಲ, ಆದರೂ ಆತ ನಿರ್ಲಿಪ್ತನು. ಇದು ಹೇಗೆ ಸಾಧ್ಯ? ನಾವು ಕಲಿಯಬೇಕಾದ ಮತ್ತೊಂದು ರಹಸ್ಯವಿದು.

ಭಿಕ್ಷುಕನಿಗೆ ಎಂದಿಗೂ ಸಂತೋಷವಿಲ್ಲ. ಭಿಕ್ಷುಕನಿಗೆ ಬರುವ ಭಿಕ್ಷೆಯ ಹಿಂದೆ, ಧಿಕ್ಕಾರ ಮತ್ತು ಅನುತಾಪವಿದೆ. ಭಿಕ್ಷುಕನೊಬ್ಬ ಕೀಳು ಜೀವಿ ಎಂಬ ಭಾವವಾದರೂ ಅದರ ಹಿಂದೆ ಇದೆ. ತನಗೆ ಸಿಕ್ಕಿರುವುದನ್ನು ಅವನು ಎಂದಿಗೂ ಅನುಭವಿಸಲಾರ.

ನಾವೆಲ್ಲ ಭಿಕ್ಷುಕರು. ನಾವು ಏನು ಮಾಡಿದರೂ ನಮಗೆ ಪ್ರತಿಫಲಬೇಕು. ನಾವೆಲ್ಲ ವ್ಯಾಪಾರಿಗಳು, ಜೀವನದಲ್ಲಿ ವ್ಯಾಪಾರಿಗಳು, ಸದ್ಗುಣಗಳಲ್ಲಿ ವ್ಯಾಪಾರಿಗಳು, ಧರ್ಮದಲ್ಲಿ ವ್ಯಾಪಾರಿಗಳು, ಅಯ್ಯೋ! ಪ್ರೀತಿಯಲ್ಲೂ ನಾವು ವ್ಯಾಪಾರಿಗಳು.

ನೀವು ವ್ಯಾಪಾರಕ್ಕೆ ಬಂದರೆ, ಕೊಟ್ಟು ತೆಗೆದುಕೊಳ್ಳುವುದಕ್ಕೆ ಬಂದರೆ, ಕೊಂಡುಮಾರುವ ನಿಯಮವನ್ನು ಅನುಸರಿಸಿ. ಕೆಲವು ವೇಳೆ ಅನುಕೂಲ, ಕೆಲವು ವೇಳೆ ಪ್ರತಿಕೂಲ. ಧಾರಣೆಯಲ್ಲಿ ಏರಿಳಿತಗಳಿವೆ. ಯಾವಾಗಲೂ ಪೆಟ್ಟನ್ನು ನಿರೀಕ್ಷಿಸಬೇಕಾಗಿದೆ. ನಾವು ಕನ್ನಡಿಯನ್ನು ನೋಡಿಕೊಂಡಂತೆ, ಅದು ನಮ್ಮ ಮುಖವನ್ನು ಪ್ರತಿಬಿಂಬಿಸುವುದು. ನೀವು ಅಣಕಿಸುವಿರಿ, ಅದೂ ನಿಮ್ಮನ್ನು ಅಣಕಿಸುವುದು. ನೀವು ನಕ್ಕರೆ ಕನ್ನಡಿಯಲ್ಲಿರುವುದೂ ನಗುವುದು. ಇದೇ ಕೊಂಡು ಮಾರುವುದು, ಕೊಟ್ಟು ತೆಗೆದುಕೊಳ್ಳುವುದು.

ನಾವು ಬಂಧನಕ್ಕೆ ಬೀಳುವೆವು. ಅದು ಹೇಗೆ? ನಾವು ಕೊಡುವುದರಿಂದ ಅಲ್ಲ, ನಾವು ನಿರೀಕ್ಷಿಸುವುದರಿಂದ ನಮ್ಮ ಪ್ರೀತಿಗೆ ಬದಲು ದುಃಖ ಬರುವುದು. ನಾವು ಪ್ರೀತಿಸುವುದರಿಂದ ಅಲ್ಲ; ಪುನಃ ಪ್ರೀತಿಗೆ ಫಲವನ್ನು ಅಪೇಕ್ಷಿಸಿದುದರಿಂದ. ಎಲ್ಲಿ ಬಯಕೆ ಇಲ್ಲವೋ ಅಲ್ಲಿ ದುಃಖವಿಲ್ಲ. ಬಯಕೆಯೇ ಎಲ್ಲ ದುಃಖಗಳ ಜನಕ. ಆಸೆ ನಿಯಮಗಳಿಂದ ಬದ್ದವಾಗಿದೆ. ಆಸೆಯು ಸೋಲು ಗೆಲುವುಗಳ ದುಃಖವನ್ನು ತಂದೇ ತರುವುದು.

ನಿಜವಾದ ಜಯವನ್ನು, ನಿಜವಾದ ಆನಂದವನ್ನು ಪಡೆಯಲು ಅಗತ್ಯವಾದ ಮಹಾ ರಹಸ್ಯವಿದು. ಯಾರು ಪ್ರತಿಫಲವನ್ನು ಅಪೇಕ್ಷಿಸುವುದಿಲ್ಲವೊ, ಪೂರ್ಣ ನಿಃಸ್ವಾರ್ಥಪರರೋ, ಅವರೇ ಹೆಚ್ಚು ಜಯಶಾಲಿಗಳು. ಇದೊಂದು ವಿರೋಧಾಭಾಸದಂತೆ ಕಾಣುವುದು. ಯಾರು ಜೀವನದಲ್ಲಿ ನಿಸ್ವಾರ್ಥಪರರೋ, ಅವರು ಮೋಸದ ಬಲೆಗೆ ಬೀಳುವುದು, ವ್ಯಥೆಪಡುವುದು ನಮಗೆ ಗೊತ್ತಿಲ್ಲವೆ? ತೋರಿಕೆಗೆ ಇದು ಸತ್ಯ. ಏಸುಕ್ರಿಸ್ತ ನಿಃಸ್ವಾರ್ಥನಾಗಿದ್ದ, ಆದರೂ ಅವನನ್ನು ಶಿಲುಬೆಗೆ ಏರಿಸಿದರು, ನಿಜ; ಆದರೆ ಅವನ ನಿಃಸ್ವಾರ್ಥತೆ ಮತ್ತೊಂದು ಜಯಕ್ಕೆ, ಕೋಟ್ಯಂತರ ಜನರಿಗೆ ನಿಜವಾದ ಶಾಂತಿ ಮತ್ತು ಮಂಗಳ ಲಭಿಸುವುದಕ್ಕೆ\break ಕಾರಣವಾಯಿತು.

ಏನನ್ನೂ ಕೇಳಬೇಡಿ; ಪ್ರತಿಯಾಗಿ ಏನನ್ನೂ ಆಶಿಸಬೇಡಿ. ಏನನ್ನು ನೀವು ಕೊಡಬೇಕೋ ಅದನ್ನು ಕೊಡಿ; ಅದು ನಿಮಗೆ ಹಿಂತಿರುಗಿಬರುವುದು. ಆದರೆ ಅದನ್ನು ನೀವು ಈಗ ಆಲೋಚಿಸಬೇಡಿ. ಅದು ಸಾವಿರ ಪಾಲಿನಷ್ಟು ಹೆಚ್ಚಾಗಿ ಪುನಃ ನಮಗೆ ಹಿಂತಿರುಗಿ ಬರುವುದು. ಆದರೆ ನಮ್ಮ ಗಮನ ಅದರ ಕಡೆಗೆ ಇರಕೂಡದು. ಆದರೂ ಕೊಡುವ ಶಕ್ತಿ ಇರಲಿ; ಕೊಡಿ; ಅದು ಅಲ್ಲಿ ಕೊನೆಗಾಣಲಿ. ಜೀವನವೆಲ್ಲ ಕೊಡುವುದಕ್ಕೆ ಇರುವುದು ಎಂಬುದನ್ನು ಕಲಿಯಿರಿ. ಪ್ರಕೃತಿ ಕೊಡುವಂತೆ ನಿಮ್ಮನ್ನು ಬಲಾತ್ಕರಿಸುವುದು. ಆದಕಾರಣ ಮನಃಪೂರ್ವಕವಾಗಿ ಕೊಡಿ. ಈಗಲೋ, ಆಗಲೋ, ನೀವು ಕೊಡಲೇಬೇಕಾಗಿದೆ. ಸಂಗ್ರಹಿಸುವುದಕ್ಕೆ ನೀವು ಪ್ರಪಂಚಕ್ಕೆ ಬರುವಿರಿ. ಮುಷ್ಟಿ ಹಿಡಿದು ಬಾಚಲೆತ್ನಿಸುವಿರಿ. ಆದರೆ ಪ್ರಕೃತಿ ನಿಮ್ಮ ಕತ್ತನ್ನು ಅದುಮಿ ಮುಷ್ಠಿ ತೆರೆಯುವಂತೆ ಮಾಡುವುದು. ನಿಮಗೆ ಇಚ್ಛೆ ಇರಲಿ ಇಲ್ಲದಿರಲಿ, ಕೊಡಲೇಬೇಕಾಗುವುದು. “ನಾನು ಕೊಡುವುದಿಲ್ಲ” ಎಂದು ಹೇಳಿದೊಡನೆಯೇ ಪೆಟ್ಟು ಬೀಳುವುದು, ವ್ಯಥೆಯಾಗುವುದು. ಯಾರೂ ಇಲ್ಲ; ಕೊನೆಗೆ ಎಲ್ಲರೂ ತಮ್ಮ ಸರ್ವಸ್ವವನ್ನು, ಬಲಾತ್ಕಾರದಿಂದ ತ್ಯಜಿಸಲೇಬೇಕಾಗುವುದು. ಈ ನಿಯಮಕ್ಕೆ ವಿರೋಧವಾಗಿ ಯಾರು ಎಷ್ಟು ಹೋರಾಡಿದರೆ ಅಷ್ಟೂ ಹೆಚ್ಚು ದುಃಖಕ್ಕೆ ಈಡಾಗಬೇಕಾಗುವುದು. ನಮಗೆ ತ್ಯಜಿಸುವುದಕ್ಕೆ ಧೈರ್ಯವಿಲ್ಲ. ಪ್ರಕೃತಿಯ ಈ ಆವಶ್ಯಕ ಮಹಾ ಕರೆಗೆ ನಾವು ಬಾಗಲಾರೆವು. ಆದಕಾರಣವೇ ನಾವು ದುಃಖಿಗಳಾಗಿರುವೆವು. ಕಾಡು ಹೋಯಿತು, ಆದರೆ ಕಾವು ಅದಕ್ಕೆ ಪ್ರತಿಯಾಗಿ ಬರುವುದು. ಸೂರ್ಯ ಸಾಗರದಿಂದ ನೀರನ್ನು ಹೀರುತ್ತಿರುವನು, ಪುನಃ ಮಳೆಯಂತೆ ಅದನ್ನು ಧರೆಗೆ ಕರೆಯುವುದಕ್ಕೆ. ಕೊಟ್ಟು ತೆಗೆದು ಕೊಳ್ಳುವುದಕ್ಕೆ ನೀವೊಂದು ಯಂತ್ರ. ಕೊಡುವುದಕ್ಕೆ ನೀವು ತೆಗೆದುಕೊಳ್ಳುವುದು. ಪ್ರತಿಯಾಗಿ ಏನನ್ನೂ ಕೋರಬೇಡಿ. ಹೆಚ್ಚು ನೀವು ಕೊಟ್ಟಷ್ಟೂ ಹೆಚ್ಚು ನಿಮಗೆ ಬರುವುದು. ಈ ಕೋಣೆಯಿಂದ ನೀವು ಎಷ್ಟು ಬೇಗ ಗಾಳಿಯನ್ನು ಹೊರಕ್ಕೆ ತಳ್ಳುವಿರೋ, ಅಷ್ಟು ಬೇಗ ಹೊರಗಿನಿಂದ ಗಾಳಿ ಒಳಗೆ ಬರುವುದು. ಎಲ್ಲಾ ಬಾಗಿಲುಗಳನ್ನೂ ಕಂಡಿಯನ್ನೂ ನೀವು ಮುಚ್ಚಿದರೆ ಯಾವುದು ಒಳಗೆ ಇರುವುದೋ ಅದು ಒಳಗೇ ಇರುವುದು. ಆದರೆ ಯಾವುದು ಹೊರಗೆ ಇರುವುದೋ ಅದು ಎಂದಿಗೂ ಒಳಗೆ ಬರುವುದಿಲ್ಲ. ಯಾವುದು ಒಳಗೆ ಇರುವುದೋ ಅದು ನಿಂತು ನಾರುವುದು, ವಿಷವಾಗುವುದು. ನದಿ ಅನವರತವೂ ಸಾಗರಕ್ಕೆ ನೀರನ್ನು ಕೊಡುವುದು, ಅನವರತವೂ ನೀರು ನದಿಗೆ ಬರುವುದು. ಅದು ಸಾಗರಕ್ಕೆ ಸೇರುವುದನ್ನು ತಡೆಯ ಬೇಡಿ. ಅದನ್ನು ಮಾಡಿದೊಡನೆಯೆ ಮೃತ್ಯು ನಿಮ್ಮನ್ನು ಹಿಡಿಯುವುದು.

ಭಿಕ್ಷುಕರಾಗಬೇಡಿ; ನಿರ್ಲಿಪ್ತರಾಗಿ; ಇದೇ ಜೀವನದ ಕಡು ಕಷ್ಟಕಾರ್ಯ. ದಾರಿಯಲ್ಲಿರುವ ಅಪಾಯಗಳನ್ನು ನೀವು ಗಮನಿಸುವುದಿಲ್ಲ. ಕಷ್ಟವನ್ನು ನಾವು ಯುಕ್ತಿಪೂರ್ವಕವಾಗಿ ಅರಿತರೂ ಅದನ್ನು ಅನುಭವಿಸುವವರೆಗೂ ಅದರ ಪರಿಚಯ ನಮಗೆ ಇರುವುದಿಲ್ಲ. ದೂರದಿಂದ ಉದ್ಯಾನವನದ ಪಕ್ಷಿನೋಟವೊಂದು ನಮಗೆ ದೊರೆಯಬಹುದು. ಆದರೆ ಅದರಿಂದ ಏನು ಪ್ರಯೋಜನ? ಅದನ್ನು ನಾವು ಸ್ವಂತ ತಿಳಿದುಕೊಳ್ಳುವುದು, ಅನುಭವಿಸುವುದು, ನಾವು ಅದರ ಮಧ್ಯದಲ್ಲಿ ಇರುವಾಗ. ಮುಂದೆ ಧಾವಿಸಿ ಹೋಗಿ, ರಕ್ತ ಸೋರುತ್ತಿದ್ದರೂ ಎದೆಗೆಡಕೂಡದು. ಇಂತಹ ಕಷ್ಟ ಪರಂಪರೆಯಲ್ಲಿಯೂ ನಮ್ಮ ದೈವೀ ಶಕ್ತಿಯನ್ನು ಮುಕ್ತಹಸ್ತದಿಂದ ಎತ್ತಿ ಹಿಡಿಯಬೇಕು. ನಮಗೆ ಶಕ್ತಿ ಇರುವಷ್ಟು ಪೆಟ್ಟಿಗೆ ಪೆಟ್ಟು, ಮೋಸಕ್ಕೆ ಮೋಸ, ಸುಳ್ಳಿಗೆ ಸುಳ್ಳನ್ನು ಪ್ರತಿಯಾಗಿ ಹಿಂತಿರುಗಿ ಕೊಡುವಂತೆ ಪ್ರಕೃತಿ ನಮ್ಮನ್ನು ಪ್ರೇರೇಪಿಸುವುದು. ಆಗ ಮುಯ್ಯಿಗೆ ಮುಯ್ಯಿ ಕೊಡದೆ ಶಾಂತರಾಗಿ ನಿರ್ಲಿಪ್ತರಾಗಿರಬೇಕಾದರೆ ಅತಿ ದೈವಿಕ ಶಕ್ತಿ ಬೇಕಾಗುವುದು.

ಪ್ರತಿದಿನವೂ ನಾವು ಅನಾಸಕ್ತರಾಗಿರಬೇಕೆಂಬ ಶಪಥ ಮರಳಿ ಮರಳಿ ಮಾಡುವೆವು. ಹಿಂದೆ ನಮ್ಮ ಪ್ರೀತಿ, ಆಸಕ್ತಿಗಳಿಗೆ ಪಾತ್ರವಾದ ವಿಷಯ ವಸ್ತುಗಳನ್ನು ಮತ್ತೊಮ್ಮೆ\break ನೋಡಿದಾಗ, ಅವುಗಳಲ್ಲಿ ಪ್ರತಿಯೊಂದೂ ನಮ್ಮನ್ನು ಎಷ್ಟು ವ್ಯಥೆಗೆ ಗುರಿಮಾಡಿದವು ಎಂಬುದು ಕಾಣುವುದು. ಪ್ರೀತಿಗೋಸುಗ ನಿರಾಶೆಯ ಪಾತಾಳಕ್ಕೆ ಹೋದೆವು! ಮತ್ತೊಬ್ಬರ ಗುಲಾಮರಾದೆವು, ಅಧಃಪಾತಾಳಕ್ಕೆ ನೂಕಲ್ಪಟ್ಟೆವು! ಪುನಃ ಹೊಸ ಶಪಥವನ್ನು ಮಾಡುವೆವು. “ಇಂದಿನಿಂದ ನಾನು ಯಜಮಾನನಾಗುವೆನು; ಇಂದಿನಿಂದ ನನ್ನನ್ನು ನಾನು ಜಯಿಸುವೆನು.'' ಆದರೆ ಕಾಲ ಬಂದಾಗ ಹಳೆಯ ಕಥೆಯೆ! ಪುನಃ ಜೀವವು ಬಂಧನಕ್ಕೆ ಸಿಕ್ಕಿ ಬೀಳುವುದು, ಪಾರಾಗಲಾರದು. ಹಕ್ಕಿ ಬಲೆಯಲ್ಲಿದೆ, ತಪ್ಪಿಸಿಕೊಳ್ಳಲು ಒದ್ದಾಡುತ್ತಿದೆ, ರೆಕ್ಕೆ ಬಡಿಯುತ್ತಿದೆ. ಇದೇ ನಮ್ಮ ಬಾಳು.

ನನಗೆ ಕಷ್ಟಗಳು ಗೊತ್ತಿವೆ. ಅವು ಭಯಂಕರವಾಗಿವೆ ಎಂಬುದು ಗೊತ್ತಿದೆ. ನಮ್ಮಲ್ಲಿ ನೂರಕ್ಕೆ ತೊಂಬತ್ತು ಮಂದಿ ನಿರಾಶೆಯಿಂದ ಎದೆಗೆಡುವೆವು. ಕ್ರಮೇಣ ಅನೇಕ ವೇಳೆ ಎಲ್ಲದರಲ್ಲಿಯೂ ಕೇಡನ್ನೇ ನಿರೀಕ್ಷಿಸುವ ಸ್ವಭಾವದವರಾಗಿ ವಿಶ್ವಾಸ, ಪ್ರೀತಿ, ಘನತೆ, ಪಾವಿತ್ರ್ಯ, ಇವುಗಳನ್ನು ನಂಬುವುದೇ ಇಲ್ಲ. ಆದಕಾರಣವೆ ಯಾರು ಬಾಲ್ಯದಲ್ಲಿ ಕ್ಷಮಾಶೀಲರಾಗಿದ್ದರೋ, ದಯಾಳುಗಳಾಗಿದ್ದರೋ, ಸರಳ ಸ್ವಭಾವದವರಾಗಿದ್ದರೋ, ನಿಷ್ಕಪಟಿಗಳಾಗಿದ್ದರೋ\break ಅವರು ವಯಸ್ಸಾದ ಮೇಲೆ ಮಿಥ್ಯಾಜೀವಿಗಳಾಗುವರು. ಅವರ ಮನಸ್ಸೊಂದು\break ಕಪಟವ್ಯೂಹ. ಅದರಲ್ಲಿ ಎಷ್ಟೋ ಬಾಹ್ಯ ನೈಪುಣ್ಯ ಇರಬಹುದು. ಬಹುಶಃ ಅವರು ಮುಂಗೋಪಿಗಳಲ್ಲದಿರಬಹುದು. ಅವರು ಹೆಚ್ಚು ಮಾತನಾಡದೆ ಇರಬಹುದು. ಆದರೆ ಅವರು ಹಾಗೆ ಮಾಡುವುದು ಒಳ್ಳೆಯದು. ಅವರ ಹೃದಯ ನಿರ್ಜೀವವಾಗಿದೆ. ಅದಕ್ಕೆ ಅವರು ಮಾತನಾಡುವುದಿಲ್ಲ. ಅವರು ನಿಂದಿಸುವುದಿಲ್ಲ, ಕೋಪಗೊಳ್ಳುವುದಿಲ್ಲ. ಆದರೆ ಅವರು ಕೋಪಗೊಳ್ಳುವುದು, ಒಳ್ಳೆಯದು. ಅವರು ಮತ್ತೊಬ್ಬರನ್ನು ನಿಂದಿಸುವುದು ಸಾವಿರಪಾಲು ಮೇಲು. ಅವರಿಗೆ ಇದು ಸಾಧ್ಯವಿಲ್ಲ. ಮೃತ್ಯು ಅವರ ಹೃದಯದಲ್ಲಿ ಮನೆಮಾಡಿಕೊಂಡಿದೆ. ಶೀತಲಕರಗಳು ಅದನ್ನು ವಶಮಾಡಿಕೊಂಡಿವೆ. ಅದು ಶಾಪಕೊಡುವುದಕ್ಕೂ, ನಿಂದಿಸುವುದಕ್ಕೂ ಚಲಿಸುವುದೇ ಇಲ್ಲ.

ಇವುಗಳೆಲ್ಲದರಿಂದ ನಾವು ಪಾರಾಗಬೇಕಾಗಿದೆ. ಆದಕಾರಣವೇ ಅತಿ ದೈವಿಕಶಕ್ತಿ ನಮಗೆ ಬೇಕೆಂದು ಹೇಳುವುದು. ಅತಿಮಾನವಶಕ್ತಿಗೆ ಸಾಕಾದಷ್ಟು ಬಲವಿಲ್ಲ. ಈ ದೌರ್ಬಲ್ಯದಿಂದ ಪಾರಾಗಬೇಕಾದರೆ ಅತಿದೈವಿಕಶಕ್ತಿಯೊಂದೇ ಮಾರ್ಗ. ಇದರ ಸಹಾಯದಿಂದ ಮಾತ್ರ ನಾವು ಇಂತಹ ಅದ್ಭುತವ್ಯೂಹದಿಂದ, ದಾರುಣ ದುಃಖವರ್ಷದಿಂದ, ಅಪಾಯಕ್ಕೆ ಬೀಳದೆ ಪಾರಾಗಬಲ್ಲೆವು. ನಮ್ಮನ್ನು ಛಿದ್ರಛಿದ್ರವಾಗಿ ತುಂಡುಮಾಡಬಹುದು, ಕತ್ತರಿಸಬಹುದು, ಆದರೂ ನಮ್ಮ ಹೃದಯ ಸದಾಕಾಲದಲ್ಲಿಯೂ ಉದಾರವಾಗಿರಬೇಕು, ಉಜ್ವಲವಾಗಿರಬೇಕು.

ಇದು ಕಡುಕಷ್ಟ. ಆದರೆ ನಿರಂತರ ಸಾಧನೆಯ ಬಲದಿಂದ ಅಂತಹ ಸ್ಥಿತಿಯಿಂದ ಪಾರಾಗಬಹುದು. ನಾವೇ ವಶರಾಗುವವರೆಗೆ ನಮಗೆ ಏನೂ ಆಗಲಾರದು ಎಂಬುದನ್ನು ತಿಳಿಯಬೇಕು. ದೇಹ ಸಿದ್ದವಾಗುವವರೆವಿಗೆ ಯಾವ ರೋಗವೂ ಬರುವುದಿಲ್ಲವೆಂಬುದನ್ನು ನಾನು ಈಗ ತಾನೇ ಹೇಳಿರುವೆನು. ಅದು ವಿಷಕ್ರಿಮಿಗಳ ಮೇಲೆ ಮಾತ್ರ ನಿಂತಿಲ್ಲ. ದೇಹ ಅದನ್ನು ಸ್ವೀಕರಿಸಲು ಸಿದ್ಧವಾಗಿರಬೇಕು. ನಾವು ಯಾವುದಕ್ಕೆ ಅರ್ಹರೋ ಅದು ನಮಗೆ ದೊರಕುವುದು. ದುಃಖ ಎಂದಿಗೂ ನನಗೆ ಯೋಗ್ಯವಿಲ್ಲದೇ ಬರಲಿಲ್ಲವೆಂಬುದನ್ನು, ಅಹಂಕಾರವನ್ನು ಬಿಟ್ಟು ಅರ್ಥ ಮಾಡಿಕೊಳ್ಳೋಣ. ಅರ್ಹರಾಗದೆ ನಮಗೆ ಯಾವ ಒಂದು ಪೆಟ್ಟು ಬೀಳಲಿಲ್ಲ. ನನ್ನ ಸ್ವಂತ ಕೈಗಳಿಂದ ಬರಮಾಡಿಕೊಳ್ಳದೆ ಪಾಪ ಒಂದೂ ಬರಲಿಲ್ಲ. ಇದನ್ನು ನಾವು ತಿಳಿದುಕೊಳ್ಳಬೇಕು. ನೀವು ಆಲೋಚನೆ ಮಾಡಿ ನೋಡಿ, ನಿಮಗೆ ಬಿದ್ದ ಪ್ರತಿಯೊಂದು ಪೆಟ್ಟೂ, ನೀವೇ ಅದಕ್ಕೆ ಸಿದ್ದರಾದುದರಿಂದ ಬಿತ್ತು. ನೀವು ಅರ್ಧ ಮಾಡಿದಿರಿ, ಬಾಹ್ಯಪ್ರಪಂಚ ಅರ್ಧ ಮಾಡಿತು. ಪೆಟ್ಟು ಬಂದುದು ಹೀಗೆ. ಇದು ನಮ್ಮ ಅಹಂಕಾರವನ್ನು ತಗ್ಗಿಸುವುದು. ಅದೇ ಕಾಲದಲ್ಲಿ ಈ ಆಲೋಚನೆಯಿಂದ ಒಂದು ಆಶಾಭಾವನೆಯು ಉದಿಸುವುದು. ಆ ಆಶಾಭಾವನೆಯೆ: “ನನಗೆ ಬಾಹ್ಯಪ್ರಪಂಚದ ಮೇಲೆ ಸ್ವಾಧೀನವಿಲ್ಲ; ಆದರೆ ಯಾವುದು ನನ್ನಲ್ಲಿದೆ, ನನ್ನ ಹತ್ತಿರವಿದೆ, ಅದರ ಮೇಲೆ ಮಾತ್ರ ಸ್ವಾತಂತ್ರ್ಯವಿದೆ. ನನ್ನ ಸ್ವಾಧೀನದಲ್ಲಿರುವ ಪ್ರಪಂಚ ಮತ್ತು ಬಾಹ್ಯ ಪ್ರಪಂಚ ಇವೆರಡೂ ಒಟ್ಟಿಗೆ ಸೇರುವುವು. ನನಗೆ ಪೆಟ್ಟು ಬೀಳುವುದಕ್ಕೆ ಕಾರಣವಾಗಬೇಕಾದರೆ, ಯಾವುದು ನನ್ನ ಸ್ವಾಧೀನದಲ್ಲಿರುವುದೋ ಅದನ್ನು ನಾನು ಕೊಡುವುದಿಲ್ಲ. ಆಗ ಪೆಟ್ಟು ಹೇಗೆ ಬೀಳಬಲ್ಲದು? ನನ್ನ ಮೇಲೆ ನನಗೆ ನಿಗ್ರಹವಿದ್ದರೆ ಪೆಟ್ಟು ಎಂದಿಗೂ ಬೀಳುವುದೇ ಇಲ್ಲ.”

ಬಾಲ್ಯದಿಂದಲೂ ನಾವು ಯಾವಾಗಲೂ ನಮ್ಮಿಂದ ಹೊರಗೆ ಯಾವುದರ ಮೇಲೋ ತಪ್ಪನ್ನು ಹೊರಿಸಲು ಪ್ರಯತ್ನಿಸುತ್ತಿರುವೆವು. ನಾವು ಯಾವಾಗಲೂ ಮತ್ತೊಬ್ಬರನ್ನು ತಿದ್ದುವುದರಲ್ಲಿ ನಿರತರಾಗಿರುವೆವು; ನಮ್ಮನ್ನೇ ತಿದ್ದಿಕೊಳ್ಳುವುದರಲ್ಲಿ ಅಲ್ಲ. ದುಃಖಿಗಳಾದರೆ,\break “ಅಯ್ಯೋ! ಈ ಪ್ರಪಂಚ ಒಂದು ಪೈಶಾಚಿಕ ಜಗತ್ತು!?” ನಾವು ಮತ್ತೊಬ್ಬರನ್ನು ದೂರಿ, “ಎಂತಹ ಮೂಢರು, ಮೋಹಪರವಶರು!'' ಎನ್ನುವೆವು. ನಾವು ನಿಜವಾಗಿಯೂ ಅಷ್ಟು ಒಳ್ಳೆಯವರಾಗಿದ್ದರೆ ಈ ಪ್ರಪಂಚದಲ್ಲಿ ಏಕೆ ಇರಬೇಕಾಗಿತ್ತು? ಇದೊಂದು ಪೈಶಾಚಿಕ ಜಗತ್ತಾದರೆ ನಾನು ಒಂದು ಪಿಶಾಚಿ. ಇಲ್ಲದೇ ಇದ್ದರೆ, ನಾನು ಇಲ್ಲಿಗೆ ಹೇಗೆ ಬರುತ್ತಿದ್ದೆ? “ಅಯ್ಯೋ ಪ್ರಪಂಚದ ಜನರು ಎಷ್ಟು ಸ್ವಾರ್ಥಪರರು!?” ಇದೇನೊ ಸತ್ಯ. ಆದರೆ ನಾವು ಅವರಿಗಿಂತ ಮೇಲಾದರೆ ಅವರ ಜೊತೆಯಲ್ಲಿ ಏತಕ್ಕೆ ಇರುವವು? ಇದನ್ನು ಆಲೋಚಿಸಿ ನೋಡಿ!?

ನಾವು ಯಾವುದಕ್ಕೆ ಯೋಗ್ಯರೊ ಅದು ನಮಗೆ ದೊರಕುವುದು. ಪ್ರಪಂಚ ಕೆಟ್ಟದ್ದು, ನಾನು ಒಳ್ಳೆಯವನು ಎಂದು ಹೇಳುವುದು ಸುಳ್ಳು. ಇದು ಎಂದಿಗೂ ಹಾಗೆ ಆಗಲಾರದು. ಇದು ನಮಗೆ ನಾವೇ ಹೇಳಿಕೊಳ್ಳುವ ಒಂದು ಘೋರ ಅಸತ್ಯ.

ನಾವು ಕಲಿಯಬೇಕಾದ ಮೊದಲನೆಯ ನೀತಿಯೇ ಇದು. ಶಪಥ ಮಾಡಿ,\break ಹೊರಗಿರುವುದಾವುದನ್ನೂ ನಿಂದಿಸಬೇಡಿ. ಹೊರಗಿನವರಾರನ್ನೂ ದೂರಬೇಡಿ. ಮನುಷ್ಯರಾಗಿ, ಎದ್ದು ನಿಲ್ಲಿ! ನೀವೇ ಹೊಣೆಗಾರರಾಗಿ; ಇದು ನಿತ್ಯ ಸತ್ಯವೆನ್ನುವುದು ಆಗ ಗೊತ್ತಾಗುವುದು. ಮೊದಲು ನಿಮ್ಮ ಮೇಲೆ ಸ್ವಾಧೀನವನ್ನು ಪಡೆಯಿರಿ.

ಒಂದು ಕ್ಷಣ ನಮ್ಮ ಪೌರುಷವನ್ನು ಕುರಿತು ಅಷ್ಟೊಂದು ಮಾತನಾಡುವೆವು.\break ದೇವತೆಗಳು ನಾವು, ಸರ್ವಜ್ಞರು ನಾವು, ಎಲ್ಲವನ್ನೂ ಸಾಧಿಸಬಲ್ಲೆವು; ಅನಿಂದ್ಯರು ನಾವು,\break ಕಳಂಕರಹಿತರು ನಾವು; ಪ್ರಪಂಚದ ಅತ್ಯಂತ ನಿಃಸ್ವಾರ್ಥಪರರು ನಾವು ಎಂದು\break ಹೆಮ್ಮೆ ಕೊಚ್ಚಿಕೊಳ್ಳುವೆವು. ಮರುಕ್ಷಣ ಒಂದು ಸಣ್ಣ ಕಲ್ಲು ನಮಗೆ ವ್ಯಥೆ ಕೊಡುವುದು. ಯಾವನೋ ಒಬ್ಬನ ಸ್ವಲ್ಪ ಕೋಪ ನಮ್ಮನ್ನು ಭಂಗಿಸುವುದು. ಯಾವ ಕೆಲಸಕ್ಕೂ ಬಾರದ ದಾರಿ ಹೋಕನು ಕೂಡ, ಈ ದೇವತೆಗಳನ್ನು ದುಃಖಿಗಳನ್ನಾಗಿ ಮಾಡುವನು! ನಾವು ಅಂತಹ ದೇವತೆಗಳಾಗಿದ್ದರೆ ಸ್ಥಿತಿ ಹೀಗಿರುತ್ತಿತ್ತೆ? ಇದಕ್ಕೆ ಪ್ರಪಂಚವನ್ನು ದೂರುವುದು ಸರಿಯೇ? ಪರಿಶುದ್ಧಾತ್ಮನೂ ಮಹಾಗುಣಾನ್ವಿತನೂ ಆದ ದೇವರು, ನಮ್ಮ ಉಪಾಯಗಳಿಂದ ದುಃಖಕ್ಕೆ ಈಡಾಗಬಲ್ಲನೆ? ನೀವು ಹಾಗೆ ನಿಃಸ್ವಾರ್ಥಪರರಾದರೆ ನೀವೇ ದೇವರ ಸಮಾನ. ಪ್ರಪಂಚ ಹೇಗೆ ನಿಮ್ಮನ್ನು ವ್ಯಥೆಗೆ ಈಡುಮಾಡಬಲ್ಲುದು? ಸಪ್ತಮಹಾನರಕದಲ್ಲಿಯೂ ಸ್ವಲ್ಪವೂ ಸೋಂಕದೆ ಪಾರಾಗಬಲ್ಲಿರಿ. ಆದರೆ ನಿಮ್ಮ ದೋಷಾರೋಪಣೆ ಮಾಡುವ ಸ್ವಭಾವವೇ, ಬಾಹ್ಯಪ್ರಪಂಚವನ್ನು ನಿಂದಿಸುವುದೇ, ನಿಮ್ಮಲ್ಲಿ ಬಾಹ್ಯಪ್ರಪಂಚದ ಇರವನ್ನು ತೋರುವುದು. ಅದರ ಇರವನ್ನು ನೀವು ಅನುಭವಿಸುವುದೇ, ನೀವು ಯಾರೆಂದು ಸಾಧಿಸಲು ಹೊರಟಿರುವಿರೋ ಅವರಾಗಿಲ್ಲವೆಂಬುದನ್ನು ಸಮರ್ಥಿಸುವುದು. ಬಾಹ್ಯಪ್ರಪಂಚ ನಿಮ್ಮನ್ನು ಹಿಂಸಿಸುತ್ತಿದೆ ಎಂದು ಭಾವಿಸಿ, “ಅಯ್ಯೋ, ಈ ಪೈಶಾಚಿಕ ಜಗತ್ತು! ಈ ಮನುಷ್ಯ ನನ್ನನ್ನು ಹಿಂಸಿಸುತ್ತಿರುವನು, ಆ ಮನುಷ್ಯ ನನ್ನನ್ನು ಹಿಂಸಿಸುತ್ತಿರುವನು” ಹೀಗೆ ಹೇಳುತ್ತ, ದುಃಖದ ಮೇಲೆ ದುಃಖವನ್ನು ಹೇರಿಕೊಳ್ಳುತ್ತ ನಿಮ್ಮ ದುಃಖವನ್ನು ಇನ್ನೂ ಹೆಚ್ಚು ಮಾಡಿಕೊಳ್ಳುತ್ತಿರುವಿರಿ. ಇದು ಆಗಲೇ ಇರುವ ದುಃಖಕ್ಕೆ ಸುಳ್ಳನ್ನು ಬೆರಸಿದಂತೆ.

ನಮ್ಮನ್ನು ನಾವು ರಕ್ಷಿಸಿಕೊಳ್ಳಬೇಕು. ಅದು ಮಾತ್ರ ನಮ್ಮಿಂದ ಸಾಧ್ಯ. ಕೆಲವು ಕಾಲ ಮತ್ತೊಬ್ಬರನ್ನು ತಿದ್ದುವ ಕೆಲಸಕ್ಕೆ ವಿರಾಮವಿರಲಿ, ದಾರಿಯನ್ನು ಶುದ್ಧ ಮಾಡೋಣ. ಗುರಿ ತನ್ನನ್ನು ತಾನೇ ನೋಡಿಕೊಳ್ಳುವುದು, ನಮ್ಮ ಜೀವನ ಒಳ್ಳೆಯದಾಗಿದ್ದರೆ, ಶುದ್ಧವಾಗಿದ್ದರೆ, ಆಗ ಮಾತ್ರ ಜಗತ್ತು ಒಳ್ಳೆಯದಾಗುವುದು; ಶುದ್ಧವಾಗುವುದು. ಇದೊಂದು ಪರಿಣಾಮ; ನಾವು ಅದಕ್ಕೆ ಕಾರಣ. ಮೊದಲು ನಾವು ಶುದ್ಧರಾಗೋಣ! ಪೂರ್ಣರಾಗೋಣ!

