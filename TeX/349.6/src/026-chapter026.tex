
\chapter[ಧನ್ಯತೆಗೆ ದಾರಿ]{ಧನ್ಯತೆಗೆ ದಾರಿ\protect\footnote{\engfoot{* C.W, Vol. II, P. 406}}}

ಈ ರಾತ್ರಿ ವೇದದಲ್ಲಿನ ಒಂದು ಕಥೆ ಹೇಳುವೆನು. ವೇದಗಳು ಹಿಂದೂಗಳ ಪವಿತ್ರಶಾಸ್ತ್ರ. ಅವು ಒಂದು ದೊಡ್ಡ ಸಾಹಿತ್ಯ ರಾಶಿ. ಅವುಗಳ ಕೊನೆಯ ಭಾಗಕ್ಕೆ ವೇದಾಂತವೆಂದು ಹೆಸರು. ಅದು ವೇದಗಳಲ್ಲಿ ಬರುವ ಸಿದ್ದಾಂತವನ್ನು ಕುರಿತು ವಿವರಿಸುವುದು. ಅದರಲ್ಲೂ ಮುಖ್ಯವಾಗಿ ನಮಗೆ ಬೇಕಾದ ತಾತ್ತ್ವಿಕ ವಿಷಯವನ್ನು ಹೇಳುವುದು. ಇದು ಬಹಳ ಪುರಾತನ ಸಂಸ್ಕೃತ ಭಾಷೆಯಲ್ಲಿದೆ. ಇದನ್ನು ಸಹಸ್ರಾರು ವರ್ಷಗಳ ಹಿಂದೆ ಬರೆದಿರುವರೆಂಬುದನ್ನು ನೀವು ಗಮನದಲ್ಲಿಡಬೇಕು. ಒಬ್ಬನು ದೊಡ್ಡದೊಂದು ಯಜ್ಞವನ್ನು ಮಾಡಬೇಕೆಂದು ಇದ್ದ. ಹಿಂದೂಧರ್ಮದಲ್ಲಿ ಯಜ್ಞಗಳು ಬಹಳ ಮುಖ್ಯ. ಹಲವು ಬಗೆಯ ಯಜ್ಞಗಳಿವೆ. ಅವರು ಯಜ್ಞವೇದಿಕೆ ಕಟ್ಟಿ ಅಗ್ನಿಗೆ ಆಹುತಿಯನ್ನು ಕೊಡುವರು. ಹಲವು ಮಂತ್ರಾದಿಗಳನ್ನು ಉಚ್ಚರಿಸುವರು. ಯಜ್ಞಾಂತ್ಯದಲ್ಲಿ ಬ್ರಾಹ್ಮಣರು ಮತ್ತು ದರಿದ್ರರಿಗೆ ದಾನಮಾಡುವರು. ಪ್ರತಿಯೊಂದು ಯಜ್ಞಕ್ಕೂ ಪ್ರತ್ಯೇಕ ದಾನ ಮಾಡಬೇಕಾಗಿದೆ. ಒಂದು ಯಜ್ಞವಿತ್ತು; ಅದರಲ್ಲಿ ಒಬ್ಬನ ಸರ್ವಸ್ವವನ್ನೂ ದಾನಮಾಡಬೇಕಾಗಿತ್ತು. ಈತ ಶ‍್ರೀಮಂತನಾದರೂ ಬಹಳ ಜಿಪುಣ. ಆದರೂ ದೊಡ್ಡ ಒಂದು ಯಾಗವನ್ನು ಮಾಡಿದೆ ಎಂಬ ಕೀರ್ತಿಯನ್ನು ಗಳಿಸಬೇಕೆಂದಿದ್ದನು. ಈ ಯಜ್ಞವನ್ನು ಮಾಡಿದಾಗ ತನ್ನ ಸರ್ವಸ್ವವನ್ನೂ ದಾನಮಾಡುವ ಬದಲು, ಇನ್ನೇನೂ ಹಾಲು ಕೊಡಲಾರದ ಕುಂಟು ಕುರುಡು ಹಸುಗಳನ್ನು ಕೊಟ್ಟನು. ಅವನಿಗೆ ನಚಿಕೇತನೆಂಬ ಒಬ್ಬ ಬುದ್ದಿವಂತ ಮಗನಿದ್ದ. ಆ ಹುಡುಗ ತಂದೆ ಮಾಡುತ್ತಿರುವ ಕೆಲಸಕ್ಕೆ ಬಾರದ ದಾನವನ್ನು ನೋಡಿ, ಇದರಿಂದ ತಂದೆಗೆ ಬರುವ ಕೆಟ್ಟ ಫಲವನ್ನು ಆಲೋಚಿಸಿ, ಅದಕ್ಕೆ ಪ್ರಾಯಶ್ಚಿತ್ತರೂಪವಾಗಿ ತನ್ನನ್ನೆ ದಾನಮಾಡಿಕೊಳ್ಳಬೇಕೆಂದು ಸಂಕಲ್ಪಿಸಿದನು. ತನ್ನ ತಂದೆ ಬಳಿಗೆ ಹೋಗಿ "ನನ್ನನ್ನು ಯಾರಿಗೆ ಕೊಡುವಿ?'' ಎಂದು ಕೇಳಿದನು. ತಂದೆ ಮಾತನಾಡಲಿಲ್ಲ. ಎರಡನೆಯ ಸಲ, ಮೂರನೆಯ ಸಲ ಅದೇ ಪ್ರಶ್ನೆಯನ್ನು ಹಾಕಿದಾಗ, ಕೋಪದಿಂದ "ನಿನ್ನನ್ನು ಯಮನಿಗೆ ಕೊಡುತ್ತೇನೆ, ನಿನ್ನನ್ನು ಮೃತ್ಯುವಿಗೆ ಕೊಡುತ್ತೇನೆ'' ಎಂದನು. ಆ ಹುಡುಗ ನೇರವಾಗಿ ಯಮಲೋಕಕ್ಕೆ ಹೋದನು. ಯಮ ಮನೆಯಲ್ಲಿರಲಿಲ್ಲ. ಅವನಿಗಾಗಿ ಕಾದನು. ಮೂರು ದಿನವಾದ ಮೇಲೆ ಯಮ ಬಂದು "ನೀನು ನನ್ನ ಅತಿಥಿ, ಮೂರುದಿನ ಊಟವಿಲ್ಲದೆ ಮನೆಯಲ್ಲಿದ್ದೆ. ನಿನಗೆ ನಮಸ್ಕಾರ. ನೀನು ಪಟ್ಟ ತೊಂದರೆಗಾಗಿ ನಿನಗೆ ಮೂರು ವರಗಳನ್ನು ಕೊಡುತ್ತೇನೆ” ಎಂದನು. ನಚಿಕೇತ ಮೊದಲನೆ ವರವಾಗಿ 'ತಂದೆಗೆ ನನ್ನ ಮೇಲಿನ ಕೋಪ ಶಾಂತವಾಗಲಿ'' ಎಂದು ಕೇಳಿಕೊಂಡನು. ಎರಡನೆಯದೆ ಯಾವುದೋ ಒಂದು ಯಾಗವನ್ನು ಹೇಗೆ ಮಾಡಬೇಕೆಂಬುದನ್ನು ಕೇಳಿದನು. ಮೂರನೆ ವರವೆ 'ಮನುಷ್ಯ ಕಾಲವಾದ ಮೇಲೆ ಏನಾಗುವನು? ಕೆಲವರು ಅವನಿಲ್ಲ ಎನ್ನುವರು. ಮತ್ತೆ ಕೆಲವರು ಅವನಿರುವನು ಎನ್ನುವರು. ಇದಕ್ಕೆ ಉತ್ತರವನ್ನು ದಯವಿಟ್ಟು ಹೇಳಿ, ಇದೇ ನನಗೆ ಬೇಕಾದ ಮೂರನೆ ವರ'' ಎಂದನು. ಅದಕ್ಕೆ ಮೃತ್ಯು, "ಹಿಂದಿನ ಕಾಲದಲ್ಲಿ\break ದೇವತೆಗಳು ಈ ಸಮಸ್ಯೆಯನ್ನು ಬಗೆಹರಿಸಲು ಯತ್ನಿಸಿದರು. ಇದು ಬಹಳ ಸೂಕ್ಷ್ಮ. ಇದನ್ನು ತಿಳಿದುಕೊಳ್ಳುವುದು ಬಹಳ ಕಷ್ಟ. ಬೇರಾವುದಾದರೂ ವರವನ್ನು ಕೇಳು. ಇದನ್ನು ಕೇಳಬೇಡ. ನೂರುವರುಷ ಬೇಕಾದರೆ ಬದುಕು. ಅಶ್ವ, ಗೋವುಗಳನ್ನು ಬೇಕಾದರೆ ಕೇಳು, ರಾಜ್ಯವನ್ನು ಬೇಕಾದರೆ ಕೇಳು. ಈ ಪ್ರಶ್ನೆಗೆ ಉತ್ತರ ಕೊಡುವಂತೆ ನನ್ನನ್ನು ಬಲಾತ್ಕರಿಸಬೇಡ. ಮಾನವ ಯಾವುದನ್ನು ಇಚ್ಛಿಸುವನೋ ಅದನ್ನೆಲ್ಲಾ ಬೇಕಾದರೆ ಕೇಳು, ಅದನ್ನು ಕೊಡುವೆನು. ಆದರೆ ಈ ಸಮಸ್ಯೆಗೆ ಉತ್ತರವನ್ನು ಮಾತ್ರ ತಿಳಿಯಲೆತ್ನಿಸಬೇಡ'' ಎಂದನು. ''ಇಲ್ಲ ಸ್ವಾಮಿ, ಮನುಷ್ಯ ಐಶ್ವರ್ಯದಿಂದ ತೃಪ್ತನಾಗಲಾರ. ಐಶ್ವರ್ಯ ಬೇಕಾದರೆ ನಿಮ್ಮನ್ನು ನೋಡಿದ ಮೇಲೆ ಬಂದೇ ಬರುವುದು. ನೀವಾಳುವವರೆಗೆ ನಾವು ಜೀವಿಸಿಯೇ ಜೀವಿಸುವೆವು, ಈ ಮರ್ತ್ಯಲೋಕದಲ್ಲಿ ವಾಸಿಸುವ ಸ್ವಲ್ಪ ಬುದ್ದಿಯುಳ್ಳ ಯಾವ ಮಾನವ ತಾನೆ, ನಾಶವಾಗದ ಅಮೃತಾತ್ಮನ ಸಂಗವನ್ನು ಪಡೆದು, ನೃತ್ಯಗೀತಗಳ ಆನಂದ ಯಾವ ರೀತಿ ಕೊನೆಗೊಳ್ಳುವುದೆಂದು ತಿಳಿದೂ, ದೀರ್ಘಾಯುಷ್ಯವನ್ನು ಇಚ್ಚಿಸುವನು? ನನಗೆ ಮರಣಾತೀತ ಮಹಾ ರಹಸ್ಯವನ್ನು ಮಾತ್ರ ಹೇಳಬೇಕು. ನನಗೆ ಮತ್ತೇನೂ ಬೇಕಿಲ್ಲ. ಮೃತ್ಯುರಹಸ್ಯವೇ ನಚಿಕೇತನಿಗೆ ಬೇಕಾಗಿರುವುದು” ಎಂದನು. ಆಗ ಯಮನಿಗೆ ಸಂತೋಷವಾಯಿತು. ಕಳೆದ ಎರಡುಮೂರು ಉಪನ್ಯಾಸಗಳಲ್ಲಿ ಈ ಜ್ಞಾನ ಮನಸ್ಸನ್ನು ಅಣಿಮಾಡುವುದೆಂಬುದನ್ನು ಹೇಳುತ್ತಿದ್ದೆವು. ಅದಕ್ಕೆ ಮೊದಲನೆ ಸಿದ್ಧತೆಯೆ, ಮಾನವ ಸತ್ಯವನ್ನಲ್ಲದೆ ಮತ್ತೇನನ್ನೂ ಆಶಿಸಕೂಡದು. ಆ ಸತ್ಯ ಕೂಡ ಕೇವಲ ಸತ್ಯಕ್ಕಾಗಿ. ಮೃತ್ಯು ಮುಂದೆ ಇಟ್ಟ ಆಸ್ತಿ ದ್ರವ್ಯ ದೀರ್ಘಾಯಸ್ಸು ಎಲ್ಲವನ್ನೂ ಹೇಗೆ ಆ ಹುಡುಗ ತಿರಸ್ಕರಿಸಿದ ನೋಡಿ! ಸತ್ಯಕ್ಕಾಗಿ, ಕೇವಲ ಜ್ಞಾನಕ್ಕಾಗಿ ಎಲ್ಲವನ್ನೂ ಸಮರ್ಪಿಸಲು ಸಿದ್ಧನಾಗಿದ್ದ. ಸತ್ಯ ಸಿದ್ಧಿಸುವುದು ಹೀಗೆ. ಯಮರಾಜನಿಗೆ ಸಂತೋಷವಾಯಿತು. ಹೀಗೆ ಹೇಳಿದನು: 'ಎರಡು ಮಾರ್ಗಗಳಿವೆ. ಒಂದು ಪ್ರೇಯಸ್ಸಿನ ಮಾರ್ಗ, ಮತ್ತೊಂದು ಶ್ರೇಯಸ್ಸಿನ ಮಾರ್ಗ. ಇವೆರಡೂ ಬೇರೆಬೇರೆ ರೀತಿಯಲ್ಲಿ ಜನರನ್ನು ಆಕರ್ಷಿಸುವುವು. ಶ್ರೇಯಸ್ಸಿನ ಪಥವನ್ನು ಯಾರು ಸ್ವೀಕರಿಸುವರೋ ಅವರೇ ಮಹಾತ್ಮರು. ಯಾರು ಪ್ರೇಯಸ್ಸಿನ ಮಾರ್ಗವನ್ನು ಸ್ವೀಕರಿಸುವರೋ ಅವರು ನಾಶವಾಗುವರು. ನಚಿಕೇತ, ನಾನು ನಿನ್ನನ್ನು ಮೆಚ್ಚುವೆನು. ನೀನು ಭೋಗವನ್ನು ಇಚ್ಚಿಸಲಿಲ್ಲ. ಹಲವು ರೀತಿ ನಾನು ನಿನಗೆ ಆಸೆ ತೋರಿಸಿದೆ. ಅದನ್ನೆಲ್ಲ ನೀನು ನಿಗ್ರಹಿಸಿದೆ. ಭೋಗಕ್ಕಿಂತ ಜ್ಞಾನ ಮೇಲೆಂದು ನೀನು ತಿಳಿದಿರುವೆ.”

'ಅಜ್ಞಾನದಲ್ಲಿ ವಿಹರಿಸುತ್ತಾ ಯಾರು ಭೋಗಿಸುವರೊ ಅವರು ಪ್ರಾಣಿಗಳಿಗಿಂತ ಮೇಲಲ್ಲ ಎಂಬುದನ್ನು ನೀನರಿತಿರುವೆ. ಆದರೂ ಹಲವರು ಅಜ್ಞಾನಾಂಧಕಾರದಲ್ಲಿದ್ದರೂ ಅಹಂಕಾರದಿಂದ ತಾವು ಸಂತರೆಂದು ಭಾವಿಸುವರು. ಕುರುಡನನ್ನು ಹಿಂಬಾಲಿಸುವ ಮತ್ತೊಬ್ಬ ಕುರುಡನಂತೆ ಅವರು ವಕ್ರಗತಿಯಲ್ಲಿ ಹೋಗುವರು. ನಚಿಕೇತ! ಯಾರು ಬಾಲರಂತೆ ಅಜ್ಞಾನಿಗಳೊ, ಪ್ರಪಂಚದ ಆಕರ್ಷಣೆಯಲ್ಲಿ ಮುಳುಗಿಹೋಗಿರುವರೊ ಅಂತಹವರ ಹೃದಯದಲ್ಲಿ ಈ ಸತ್ಯ ಎಂದಿಗೂ ಬೆಳಗುವುದಿಲ್ಲ. ಅವರು ಈ ಲೋಕವನ್ನೂ ತಿಳಿಯಲಾರರು, ಬೇರೊಂದು ಲೋಕವನ್ನೂ ತಿಳಿಯಲಾರರು. ಅವರು ಈ ಲೋಕವನ್ನು ಅಲ್ಲಗಳೆಯುವರು, ಪರಲೋಕವನ್ನೂ ಅಲ್ಲಗಳೆಯುವರು. ಪುನಃ ಪುನಃ ಅವರು ನನ್ನ ವಶವಾಗುವರು. ಅನೇಕರಿಗೆ ಇದನ್ನು ಕೇಳುವ ಅದೃಷ್ಟವೂ ಇಲ್ಲ. ಅನೇಕರು ಇದನ್ನು ಕೇಳಿದರೂ ತಿಳಿದುಕೊಳ್ಳಲಾರರು. ಏಕೆಂದರೆ ಹೇಳುವವನೂ ಕುಶಲಿಯಾಗಿರಬೇಕು. ಕೇಳುವವನೂ ಕುಶಲಿಯಾಗಿರಬೇಕು. ಹೇಳುವವನು ಆಧ್ಯಾತ್ಮಿಕ\break ಜೀವನದಲ್ಲಿ ಮುಂದುವರಿದ ವ್ಯಕ್ತಿ ಅಲ್ಲದೆ ಇದ್ದರೆ, ನೂರುವೇಳೆ ಅಂತಹವನ ಮಾತನ್ನು ಕೇಳಿದರೂ, ನೂರುವೇಳೆ ಅಂತಹವನಿಂದ ಕಲಿತರೂ ಜ್ಞಾನ ಲಭಿಸುವುದಿಲ್ಲ. ಕೆಲಸಕ್ಕೆ ಬಾರದ ತರ್ಕದಿಂದ ನಿನ್ನ ಮನಸ್ಸನ್ನು ಕೆಡಿಸಿಕೊಳ್ಳಬೇಡ, ನಚಿಕೇತ, ಎಲ್ಲಿ ಪರಿಶುದ್ಧ ಹೃದಯವಿದೆಯೊ ಅಲ್ಲಿ ಮಾತ್ರ ಈ ಜ್ಞಾನ ಬೆಳಗುವುದು. ಯಾರು ಗುಹ್ಯನಾಗಿರುವನೋ ಹೃದಯಾಂತರಾಳದಲ್ಲಿ ಅವಿತಿರುವನೊ ಅವನನ್ನು ಬಾಹ್ಯ ಚಕ್ಷುಗಳಿಂದ ನೋಡಲಾಗುವುದಿಲ್ಲ. ಅಂತಶ್ಚಕ್ಷುವಿನಿಂದ ನೋಡಿದವನು ಸುಖ ದುಃಖಗಳೆರಡನ್ನೂ ತ್ಯಜಿಸುವನು. ಯಾರಿಗೆ ಈ ರಹಸ್ಯ ಗೊತ್ತಿದೆಯೊ ಅವನು ಅನ್ಯ ಆಸೆಗಳನ್ನೆಲ್ಲಾ ತ್ಯಜಿಸಿ, ಈ ಅತೀಂದ್ರಿಯ ಅನುಭವವನ್ನು ಪಡೆದು ಧನ್ಯನಾಗುವನು. ನಚಿಕೇತ, ಇದೇ ಧನ್ಯತೆಗೆ ಮಾರ್ಗ. ಅವನು ಎಲ್ಲಾ ಪಾಪಪುಣ್ಯಗಳಾಚೆ ಇರುವನು. ಕರ್ತವ್ಯ, ಅಕರ್ತವ್ಯಗಳಾಚೆ ಇರುವನು. ಎಲ್ಲ ಅಸ್ತಿತ್ವಗಳಿಂದಲೂ, ಮುಂದೆ ಇರಬೇಕಾದ ಸ್ಥಿತಿಗಳಿಂದಲೂ ಮುಕ್ತನಾಗಿರುವನು. ಯಾರಿಗೆ ಇದು ಗೊತ್ತೊ ಅವನಿಗೆ ಮಾತ್ರ ಸತ್ಯ ತಿಳಿದಿರುವುದು. ಯಾವುದನ್ನು ವೇದಗಳು ಅರಸುತ್ತಿರುವುವೊ, ಯಾವುದನ್ನು ನೋಡುವುದಕ್ಕೆ ಜನ ತಪಸ್ಸನ್ನು ಆಚರಿಸುವರೊ, ಅದನ್ನು ನಿನಗೆ ಹೇಳುತ್ತೇನೆ. ಅದೇ 'ಓಂ'. ಈ ಅನಂತ ಓಂಕಾರವೇ ಬ್ರಹ್ಮ, ಇದೇ ಅಮೃತ. ಯಾರಿಗೆ ಇದರ ರಹಸ್ಯ ತಿಳಿದಿದೆಯೊ, ಅವನು ಯಾವುದನ್ನು ಇಚ್ಚಿಸಿದರೂ ಅದೆಲ್ಲವೂ ಅವನದಾಗುವುದು. ನಚಿಕೇತ, ನೀನು ತಿಳಿದುಕೊಳ್ಳಬೇಕೆಂಬ ಆತ್ಮ ಎಂದೂ ಹುಟ್ಟಿಲ್ಲ, ಸಾಯುವುದಿಲ್ಲ. ಅದಕ್ಕೆ ಆದಿಯಿಲ್ಲ. ಅದು ಯಾವಾಗಲೂ ಇರುವುದು. ಈ ಪುರಾಣ ವಸ್ತು ದೇಹ ನಾಶವಾದರೆ ನಾಶವಾಗುವುದಿಲ್ಲ. ಕೊಲ್ಲುವವನು ಕೊಂದೆ ಎಂದು ಭಾವಿಸಿದರೆ, ಕೊಲ್ಲಿಸಿಕೊಳ್ಳುವವನು ಸತ್ತೆ ಎಂದು ಭಾವಿಸಿದರೆ ಇಬ್ಬರೂ ತಪ್ಪು. ಅದು ಕೊಲ್ಲುವುದೂ ಇಲ್ಲ, ಕೊಲ್ಲಿಸಿಕೊಳ್ಳುವುದೂ ಇಲ್ಲ. ಅಣುವಿಗಿಂತ ಅಣು ಇದು, ವಿಭುವಿಗಿಂತ ವಿಭು. ಈ ಸರ್ವೆಶ್ವರನು ಎಲ್ಲರ ಹೃದಯಾಂತರಾಳದಲ್ಲಿಯೂ ಇರುವನು. ಯಾರು ಪಾಪ ದೂರರಾಗಿರುವರೊ ಅವರು ಆತ್ಮಪ್ರಸಾದದಿಂದ ಅವನ ದರ್ಶನವನ್ನು ಪಡೆಯುವರು. (ದೇವರ ಕೃಪೆ ಭಗವತ್ಸಾಕ್ಷಾತ್ಕಾರಕ್ಕೆ ಕಾರಣವೆಂಬುದು ಕಾಣುವುದು.) ಕುಳಿತುಕೊಂಡೆ ದೂರ ಹೋಗುವನು, ಮಲಗಿಯೆ ಎಲ್ಲಿ ಬೇಕಾದರೂ ಹೋಗುವನು. ಎಲ್ಲಿ ವಿರೋಧ ಗುಣಗಳೆಲ್ಲವೂ ಸೇರುವುವೋ ಅಂತಹ ಸತ್ಯವನ್ನು ನೋಡುವುದಕ್ಕೆ ಪರಿಶುದ್ಧ ಹೃದಯಿಗಳಾದ ಬುದ್ದಿವಂತರಿಗಲ್ಲದೆ ಅನ್ಯರಿಗೆ ಸಾಧ್ಯವಿಲ್ಲ. ನಿರ್ದೇಹಿಯಾದರೂ ದೇಹಧಾರಿಯಾಗಿ, ಅಸ್ಪರ್ಶವಾದರೂ ಎಲ್ಲದರೊಂದಿಗೆ ಸ್ಪರ್ಶವಾದಂತೆ ತೋರುತ್ತಿರುವ ಸರ್ವವ್ಯಾಪಿಯಾದ ಆತ್ಮನನ್ನು ಅರಿತ ಮುನಿಗಳು ದುಃಖದಿಂದ ಪಾರಾಗುವರು. ವೇದಾಧ್ಯಯನದಿಂದಾಗಲಿ ಸೂಕ್ಷ್ಮಬುದ್ದಿಯಿಂದಾಗಲಿ ಪಾಂಡಿತ್ಯದಿಂದಾಗಲಿ ಆತ್ಮಲಾಭವಾಗುವುದಿಲ್ಲ. ಯಾರಿಗೆ ಆತ್ಮ ಪ್ರಸಾದ ದೊರಕುವುದೊ ಅವರಿಗೆ ಆತ್ಮಲಾಭವಾಗುವುದು, ಆತ್ಮ ದರ್ಶನವಾಗುವುದು. ಯಾರು ಅನವರತ ಹೀನ ಕೃತ್ಯಗಳನ್ನು ಮಾಡುತ್ತಿರುವರೊ, ಯಾರ ಮನಸ್ಸು ಶುದ್ಧವಾಗಿಲ್ಲವೋ, ಯಾರು ಧ್ಯಾನಮಾಡಲಾರರೊ ಯಾರು ಅನವರತ ಚಂಚಲಚಿತ್ತರೋ ಅಂತಹವರಿಗೆ ಹೃದಯಗಹ್ವರದಲ್ಲಿರುವ ಆತ್ಮದರ್ಶನವಾಗುವುದಿಲ್ಲ. ನಚಿಕೇತ, ಈ ದೇಹವೇ ರಥ, ಇಂದ್ರಿಯಗಳೇ ಅಶ್ವಗಳು, ಮನಸ್ಸೇ ಲಗಾಮು, ಬುದ್ದಿಯೇ ಸಾರಥಿ, ಆತ್ಮನೇ ರಥದಲ್ಲಿ ಮಂಡಿಸಿರುವವನು. ಆತ್ಮವು ಸಾರಥಿಯಾದ ಬುದ್ದಿಯೊಂದಿಗೆ ಸೇರಿ ಅದರಿಂದ ಮನಸ್ಸೆಂಬ ಲಗಾಮನ್ನು ಹಿಡಿದು ಇಂದ್ರಿಯಗಳೆಂಬ ಅಶ್ವವನ್ನು ಸೇರಿದಾಗ ಅವನು ಭೋಕ್ತೃ ಎನ್ನಿಸಿಕೊಳ್ಳುವನು. ಆಗ ಅವನು ನೋಡುವನು, ಕೆಲಸ ಮಾಡುವನು. ಯಾರ ಮನಸ್ಸು ತಮ್ಮ ಸ್ವಾಧೀನದಲ್ಲಿಲ್ಲವೊ, ಯಾರಲ್ಲಿ ಯುಕ್ತಾಯುಕ್ತ ಪರಿಜ್ಞಾನವಿಲ್ಲವೊ ಅವನ ಇಂದ್ರಿಯಗಳನ್ನು ನಿಗ್ರಹಿಸಲು ಅಸಾಧ್ಯ - ಸಾರಥಿಯ ಕೈಯಲ್ಲಿರುವ ತುಂಟ ಕುದುರೆಗಳಂತೆ. ಯಾರಿಗೆ ಯುಕ್ತಾಯುಕ್ತ ಪರಿಜ್ಞಾನವಿದೆಯೋ, ಯಾರ ಮನಸ್ಸು ನಿಗ್ರಹದಲ್ಲಿದೆಯೊ ಅವರ ಇಂದ್ರಿಯಗಳು ಒಳ್ಳೆಯ ಕುದುರೆಗಳಂತೆ ಯಾವಾಗಲೂ ವಶದಲ್ಲಿವೆ. ಯಾರಿಗೆ ನಿತ್ಯಾನಿತ್ಯವಸ್ತು ವಿವೇಕವಿದೆಯೋ, ಯಾರ ಮನಸ್ಸು ಸತ್ಯವನ್ನು ಅರಿಯುವುದರಲ್ಲಿ ಉದ್ಯುಕ್ತವಾಗಿದೆಯೊ, ಯಾರು ನಿತ್ಯ ಪರಿಶುದ್ದರೊ ಅವರಿಗೆ ಸತ್ಯಸಾಕ್ಷಾತ್ಕಾರವಾಗುವುದು. ಇದನ್ನು ಪಡೆದಾದ ಮೇಲೆ ಪುನರ್ಜನ್ಮವಿಲ್ಲ. ಓ ನಚಿಕೇತ! ಇದು ಬಹಳ ಕಷ್ಟ, ದಾರಿ ತುಂಬಾ ದೂರ, ಗುರಿಯನ್ನು ಸೇರುವುದು ಅತಿ ಕಷ್ಟ. ಯಾರು ಸೂಕ್ಷ್ಮದರ್ಶಿಗಳೊ ಅವರು ಮಾತ್ರ ನೋಡಬಲ್ಲರು, ಅವರು ಮಾತ್ರ ತಿಳಿಯಬಲ್ಲರು. ಆದರೂ ಭೀತನಾಗಬೇಡ, ಜಾಗ್ರತನಾಗಿ ಪ್ರಯತ್ನಪಡು. ಗುರಿಸೇರುವವರೆಗೂ ನಿಲ್ಲಬೇಡ. ಇದು ಹರಿತವಾದ ಕತ್ತಿಯ ಮೇಲೆ ನಡೆಯುವಷ್ಟು ಕಷ್ಟ ಎಂದು ಋಷಿಗಳು ಸಾರುವರು. ಯಾರು ಶಬ್ದ ಸ್ಪರ್ಶ ರೂಪ ರಸ ಗಂಧಗಳಿಗೆ ಅತೀತನೊ, ಅವ್ಯಕ್ತನೊ, ಅನಂತನೊ, ಬುದ್ಧಿಗೆ ಅತೀತನೋ ಅವಿನಾಶಿಯೊ ಅವನನ್ನು ನೋಡಿದಾಗ ಮಾತ್ರ ಮೃತ್ಯುಮುಖದಿಂದ ಪಾರಾಗುವೆವು.”

ಇದುವರೆಗೆ ಯಮನು ಸೇರಬೇಕಾದ ಗುರಿಯನ್ನು ವಿವರಿಸಿದನು. ನಮಗೆ ತಿಳಿದ ಮೊದಲ ಭಾವನೆಯೆ, ನಾವು ಸತ್ಯವನ್ನು ಅರಿತಾಗ ಮಾತ್ರ ಜನನಮರಣ, ದುಃಖಕ್ಲೇಶಗಳಿಂದ ಪಾರಾಗಬಲ್ಲೆವೆಂಬುದು. ಸತ್ಯವೆಂದರೇನು? ಯಾವುದು ಬದಲಾಯಿಸುವುದಿಲ್ಲವೋ ಅಂತಹ ಮಾನವನ ಆತ್ಮ, ವಿಶ್ವದ ಹಿಂದೆ ಇರುವ ಆತ್ಮ. ಅವನನ್ನು ತಿಳಿಯುವುದು ಬಹಳ ಕಷ್ಟವೆಂದೂ ಹೇಳಿರುವೆವು. ತಿಳಿಯುವುದೆಂದರೆ ಬೌದ್ಧಿಕವಾಗಿ ಅರಿಯುವುದಲ್ಲ. ಅದು ಆತ್ಮಸಾಕ್ಷಾತ್ಕಾರ. ಪುನಃ ಪುನಃ ಈ ಆತ್ಮ ಎಂಬುದನ್ನು ನಾವು ನೋಡಬೇಕು. ಅದನ್ನು ಪಡೆಯಬೇಕೆಂಬುದನ್ನು ಓದಿರುವೆವು. ಅದನ್ನು ಕಣ್ಣಿನ ಮೂಲಕ ನೋಡಲಾಗುವುದಿಲ್ಲ. ನಮ್ಮ ಇಂದ್ರಿಯಗ್ರಹಣ ಸೂಕ್ಷ್ಮತರವಾಗಬೇಕು. ಗೋಡೆ ಪುಸ್ತಕ ಮುಂತಾದುವನ್ನು ನೋಡುವುದು ಸ್ಥೂಲಗ್ರಹಣ. ಆದರೆ ಸತ್ಯವನ್ನು ತಿಳಿಯಬೇಕಾದರೆ ಅದು ಸೂಕ್ಷ್ಮತರವಾಗಬೇಕು. ಇದೇ ಜ್ಞಾನ ರಹಸ್ಯವಲ್ಲ. ಒಬ್ಬನು ಪರಿಶುದ್ಧನಾಗಿರಬೇಕೆಂದು ಯಮನು ಹೇಳುವನು. ನಮ್ಮ ಇಂದ್ರಿಯಗ್ರಹಣ ಸೂಕ್ಷ್ಮವಾಗಬೇಕಾದರೆ ಇದೇ ದಾರಿ. ಅನಂತರ ಅವನು ಇತರ ವಿಷಯಗಳನ್ನೂ ಹೇಳುವನು. ಸ್ವಯಂಭೂ ಇಂದ್ರಿಯಾತೀತ. ಇಂದ್ರಿಯಗಳು ಹೊರಗೆ ನೋಡುವುವು. ಆದರೆ ಆತ್ಮ ಇರುವುದು ಅಂತರಾಳದಲ್ಲಿ. ನೋಡುವುದಕ್ಕೆ ಬೇಕಾದ 'ಅರ್ಹತೆಗಳನ್ನು ನಾವು ಜ್ಞಾಪಕದಲ್ಲಿಡಬೇಕು: ನಮ್ಮ ಮನಸ್ಸನ್ನು ಅಂತರ್ಮುಖಮಾಡಿ ಆತ್ಮನನ್ನು ನೋಡುವ ಬಯಕೆ ಇರಬೇಕು. ಪ್ರಕೃತಿಯಲ್ಲಿ ನಾವು ನೋಡುವ ಸುಂದರ ವಸ್ತುಗಳೆಲ್ಲ ಒಳ್ಳೆಯವೇ. ಆದರೆ ದೇವರನ್ನು ನೋಡುವ ರೀತಿ ಇದಲ್ಲ. ಅಂತರ್ಮುಖರಾಗುವುದನ್ನು ನಾವು ಕಲಿಯಬೇಕು. ಹೊರಗೆ ನೋಡುವ ಕುತೂಹಲವನ್ನು ತಗ್ಗಿಸಬೇಕು. ನೀವು ಜನಸಂದಣಿ ಇರುವ ದಾರಿಯಲ್ಲಿ ನಡೆದುಕೊಂಡು ಹೋಗುತ್ತಿರುವಾಗ ನಿಮ್ಮ ಸಂಗಡಿಗ ಮಾತನಾಡುವುದನ್ನು ನೀವು ಕೇಳಲಾರಿರಿ. ಏಕೆಂದರೆ ಓಡಾಡುತ್ತಿರುವ ಗಾಡಿಗಳ ಶಬ್ದದಿಂದ ದಾರಿ ತುಂಬಿದೆ. ಅವನಿಗೂ ನೀವು ಹೇಳುವುದು ಕೇಳಿಸುವುದಿಲ್ಲ. ಈ ಶಬ್ದದ ಕಾರಣದಿಂದ ಮನಸ್ಸು ಹೊರಗೆ ಹೋಗುತ್ತಿದೆ. ನಿಮ್ಮ ಪಕ್ಕದಲ್ಲೆ ಇರುವವನ ಮಾತು ಕೂಡ ಕೇಳಿಸುವುದಿಲ್ಲ. ಇದರಂತೆಯೇ ನಮ್ಮ ಸುತ್ತಲಿರುವ ಪ್ರಪಂಚದಲ್ಲಿ ಅಷ್ಟು ಗದ್ದಲವಿದೆ. ಮನಸ್ಸು ಇದರಿಂದ ಹೊರಗೆ ಹೋಗುವುದು. ನಾವು ಆತ್ಮನನ್ನು ಹೇಗೆ ಕಾಣುವುದು? ಹೊರಗೆ ಹೋಗುವುದನ್ನು ತಡೆಯಬೇಕು. ಮನಸ್ಸನ್ನು ಅಂತರ್ಮುಖಮಾಡುವುದೆಂದರೆ ಇದೇ ಅರ್ಥ. ಆಗ ಮಾತ್ರ ಅಂತರಾತ್ಮನ ಮಹಿಮೆ ಅರಿಯುವುದು.

ಆತ್ಮವೆಂದರೇನು? ಅದು ಬುದ್ದಿಗೂ ಅತೀತವೆಂಬುದನ್ನು ನಾವು ನೋಡಿದೆವು. ಆತ್ಮ ಅನಂತ, ವಿಭು, ನಾವು ನೀವೆಲ್ಲ ಸರ್ವವ್ಯಾಪಿಗಳು, ಆತ್ಮ ಅವಿಕಾರಿ ಎಂಬುದನ್ನು ಇದೇ ಉಪನಿಷತ್ತಿನಿಂದ ತಿಳಿದೆವು. ವಿಭುವಾದವನು ಏಕಮಾತ್ರ ಆಗಿರಬೇಕು. ಇಬ್ಬರು ವಿಭುಗಳಿರಲಾರರು. ಇದು ಹೇಗೆ ಸಾಧ್ಯ? ಅನಂತವಾಗಿರುವ ಎರಡು ಆತ್ಮಗಳಿರಲಾರವು. ಆದಕಾರಣ ಇರುವುದೊಂದೇ ಆತ್ಮ. ನಾನು, ನೀವು, ವಿಶ್ವವೆಲ್ಲ ಏಕ; ಅನೇಕದಂತೆ ತೋರುತ್ತಿದೆ ಅಷ್ಟೆ. “ಒಂದು ಬೆಂಕಿಯು ಪ್ರಪಂಚಕ್ಕೆ ಬಂದು ಹಲವು ರೀತಿಯಲ್ಲಿ ಕಾಣುತ್ತಿರುವಂತೆ, ಎಲ್ಲರ ಆತ್ಮವಾದ ಒಂದೇ ಒಂದು ಅನೇಕದಲ್ಲಿ ವ್ಯಕ್ತವಾಗುತ್ತಿದೆ.” ಈಗ ಬರುವ ಪ್ರಶ್ನೆಯೆ ವಿಶ್ಲೇಷಣೆ ಈ ಆತ್ಮ ಪರಿಶುದ್ಧವಾಗಿದ್ದರೆ, ಪರಿಪೂರ್ಣವಾಗಿದ್ದರೆ, ಇದೊಂದೆ ವಿಶ್ವಕ್ಕೆಲ್ಲಾ ಆತ್ಮವಾಗಿದ್ದರೆ, ಒಳ್ಳೆಯ ಕೆಟ್ಟ ದೇಹಗಳಿಗೆ ಹೋದಾಗ ಇದೇನಾಗುವುದು? ಅದು ಹೇಗೆ ಪರಿಶುದ್ಧವಾಗಿ ಉಳಿಯುವುದು? “ಎಲ್ಲರೂ ನೋಡುವುದಕ್ಕೆ ಒಂದೇ ಸೂರ್ಯ ಕಾರಣ. ಆದರೂ ಯಾರ ಕಣ್ಣಿನ ದೋಷವೂ ಅದಕ್ಕೆ ತಾಗುವುದಿಲ್ಲ.” ಕಾಮಾಲೆ ಇದ್ದರೆ ಎಲ್ಲಾ ಹಳದಿಯಂತೆ ನೋಡುವನು, ಅವನ ದೃಶ್ಯಕ್ಕೆ ಕಾರಣ ಸೂರ್ಯ. ಆದರೆ ಎಲ್ಲವನ್ನೂ ಹಳದಿಯಾಗಿ ನೋಡುವುದರಿಂದ ಸೂರ್ಯನಿಗೆ ದೋಷವಿಲ್ಲ. ಇದರಂತೆಯೆ ಒಂದು ಆತ್ಮ ಎಲ್ಲರ ಆತ್ಮವಾಗಿದ್ದರೂ ಹೊರಗಿನ ಗುಣದೋಷಗಳಿಗೆ ಒಳಗಾಗುವುದಿಲ್ಲ. “ಈ ಅನಿತ್ಯ ಪ್ರಪಂಚದಲ್ಲಿ ಯಾರು ನಿತ್ಯವನ್ನು ನೋಡುವರೊ, ಈ ಅಚೇತನ ಜಗತ್ತಿನಲ್ಲಿ ಯಾರು ಏಕಮಾತ್ರ ಚೇತನವನ್ನು ನೋಡುವರೋ, ಅನೇಕದರಲ್ಲಿ ಯಾರು ಏಕವನ್ನು ನೋಡುವರೋ, ಅದನ್ನು ತಮ್ಮ ಆತ್ಮದಲ್ಲಿ ನೋಡುವರೋ ಅವರಿಗೆ ಮಾತ್ರ ಅನಂತ ಧನ್ಯತೆ ಪ್ರಾಪ್ತಿ; ಇತರರಿಗಲ್ಲ. ಸೂರ್ಯನು ಅಲ್ಲಿ ಬೆಳಗುವುದಿಲ್ಲ. ತಾರೆ ಇಲ್ಲ, ವಿದ್ಯುತ್ ಬೆಳಗಲಾರದು. ಇನ್ನು ಬೆಂಕಿಯ ಮಾತೇನು? ಅವನ ಪ್ರಕಾಶದಿಂದ ಇವುಗಳೆಲ್ಲ ಪ್ರಕಾಶಿಸುತ್ತಿವೆ. ಅವನಿಂದ ಎಲ್ಲವೂ ಕಾಂತಿಯುತವಾಗಿವೆ. ಹೃದಯವನ್ನು ಬಾಧಿಸುವ ಆಸೆಗಳೆಲ್ಲ ನಿಂತಮೇಲೆ, ಮರ್ತ್ಯ ಅಮೃತನಾಗಿ ಬ್ರಹ್ಮನನ್ನು ಪಡೆಯುವನು. ಆಗ ಹೃದಯಗ್ರಂಥಿಗಳು ಮಾಯವಾಗಿ, ಬಂಧನಗಳೆಲ್ಲ ಬಿದ್ದು ಹೋಗುವುವು. ಆಗಲೆ ಮರ್ತ್ಯನು ಅಮೃತನಾಗುವನು. ಇದೇ ಮಾರ್ಗ.”

\begin{verse}
ಓಂ ಸಹ ನಾವವತು ಸಹ ನೌ ಭುನಕ್ತು ಸಹ ವೀರ್ಯಂ ಕರವಾವಹೈ~।\\ತೇಜಸ್ವಿನಾವಧೀತಮಸ್ತು ಮಾ ವಿದ್ವಿಷಾವಹೈ~॥\\ಓಂ ಶಾಂತಿಃ ಶಾಂತಿಃ ಶಾಂತಿಃ 
\end{verse}

ವೇದಾಂತ ತತ್ವದಲ್ಲಿ ನಮಗೆ ದೊರಕುವ ವಿಚಾರ ಸರಣಿ ಇದು. ಮೊದಲಲ್ಲಿ ನಮಗೆ ಪ್ರಪಂಚದಲ್ಲಿ ಮತ್ತೆಲ್ಲಿಯೂ ಕಾಣದ, ಸಂಪೂರ್ಣ ಬೇರೆಯಾದ ಆಲೋಚನೆ ಕಾಣುವುದು. ವೇದಗಳ ಕರ್ಮಕಾಂಡದಲ್ಲಿ ಅನ್ವೇಷಣೆ ಇತರ ಕಡೆಗಳಂತೆ ಹೊರಗೆಯೇ ಇರುವುದು. ಬಹಳ ಪುರಾತನ ಗ್ರಂಥದಲ್ಲಿ "ಪ್ರಪಂಚದಲ್ಲಿ ಮೊದಲು ಏನಿತ್ತು? ಅಲ್ಲಿ ಕಾಣುವುದಾವುದೂ ಇಲ್ಲದಿರುವಾಗ ತಮಸ್ಸು ತಮನಸ್ಸಿನಲ್ಲಿ ಆವೃತವಾದಾಗ ಇದನ್ನೆಲ್ಲಾ ಯಾರು ನಿರ್ಮಿಸಿದರು?' ಎಂಬ ಪ್ರಶ್ನೆ ಇದೆ. ಅನ್ವೇಷಣೆ ಹೀಗೆ ಆರಂಭವಾಯಿತು. ದೇವತೆಗಳು ದೇವದೂತರು ಮುಂತಾದುವನ್ನೆಲ್ಲಾ ಹೇಳತೊಡಗಿದರು. ಅನಂತರ ಇದರಿಂದ ಏನೂ ಪ್ರಯೋಜನವಿಲ್ಲ ಎಂದು ತ್ಯಜಿಸಿದರು. ಆ ಕಾಲದಲ್ಲಿ ಅನ್ವೇಷಣೆ ಹೊರಗಿತ್ತು. ಅಲ್ಲಿ ಏನೂ ಕಾಣಲಿಲ್ಲ. ಅನಂತರ ಆತ್ಮನನ್ನು ಆಂತರ್ಯದಲ್ಲಿ ಅರಸಬೇಕಾಯಿತೆಂಬುದನ್ನು ವೇದಗಳಲ್ಲಿ ಓದುವೆವು. ವೇದಗಳಲ್ಲಿ ಬರುವ ಒಂದು ಮುಖ್ಯ ಭಾವನೆ ಇದು. ತಾರೆ, ನೀಹಾರಿಕೆ, ಜ್ಯೋತಿಮಾರ್ಗ, ಇಡೀ ಬಾಹ್ಯ ಬ್ರಹ್ಮಾಂಡದ ಅನ್ವೇಷಣೆಯಿಂದ ಏನೂ ಪ್ರಯೋಜನವಿಲ್ಲ. ಅದು ಜನನ ಮರಣಗಳ ಸಮಸ್ಯೆಯನ್ನು ಪರಿಹರಿಸಲಾರದು. ಅದ್ಭುತ ಅಂತರಂಗವನ್ನು ವಿಶ್ಲೇಷಣೆ ಮಾಡಬೇಕು. ಆಗ ವಿಶ್ವ ರಹಸ್ಯ ತಿಳಿಯುವುದು. ಸೂರ್ಯ ತಾರಾವಳಿಗಳು ಇದನ್ನು ಮಾಡಲಾರವು. ಮಾನವನನ್ನು ವಿಶ್ಲೇಷಣೆ ಮಾಡಬೇಕು. ದೇಹವಲ್ಲ, ಮಾನವನ ಆತ್ಮವನ್ನು.ಅಲ್ಲಿ ಅವರಿಗೆ ಉತ್ತರ ದೊರಕಿತು. ದೊರೆತ ಉತ್ತರವೆಂತಹುದು? ದೇಹದ ಹಿಂದೆ, ಮನಸ್ಸಿನ ಹಿಂದೆ, ಸ್ವಯಂ ಜ್ಯೋತಿ ಆತ್ಮನಿರುವನೆಂಬುದು. ಅವನಿಗೆ ಜನನ ಮರಣಗಳಿಲ್ಲ. ಅವನಿಗೆ ಆಕಾರವಿಲ್ಲದುದರಿಂದ ಸರ್ವವ್ಯಾಪಿ. ಯಾವುದಕ್ಕೆ ರೂಪವಿಲ್ಲವೊ ಆಕಾರವಿಲ್ಲವೊ ಯಾವುದು ಕಾಲದೇಶಗಳಿಂದ ಮಿತವಾಗುವುದಿಲ್ಲವೊ, ಅದು ಒಂದು ಸ್ಥಳದಲ್ಲಿ ಇರಲಾರದು. ಅದು ಹೇಗೆ ಸಾಧ್ಯ? ಅದು ಎಲ್ಲದರಲ್ಲಿಯೂ ಇರುವುದು, ಸರ್ವವ್ಯಾಪಿ, ನಮ್ಮಲ್ಲೆಲ್ಲ ಒಂದೇ ಸಮನಾಗಿರುವುದು.

ಮಾನವ ಆತ್ಮವೆಂದರೇನು? ಒಬ್ಬ ಈಶ್ವರನೆಂಬುವನು ಇರುವನು, ಜೊತೆಗೆ ಬೇಕಾದಷ್ಟು ಆತ್ಮಗಳಿವೆ. ಅವು ರೂಪ, ಗುಣ ಇವುಗಳಲ್ಲೆಲ್ಲ ದೇವರಿಂದ ಬೇರೆ ಎನ್ನುವರು. ಇದೇ ದ್ವೈತ. ಇದೇ ಹಳೆಯದು. ಬಹಳ ಹಳೆಯ ಒರಟು ಭಾವನೆ. ಮತ್ತೊಂದು ಗುಂಪಿನವರು ಕೊಟ್ಟ ಉತ್ತರವೇ ಜೀವಾತ್ಮ, ದಿವ್ಯಾತ್ಮನಾದ ಭಗವಂತನ ಅಂಶವೆಂಬುದು. ಈ ದೇಹ ಒಂದು ಸಣ್ಣ ಪ್ರಪಂಚ, ಅದರ ಹಿಂದೆ ಮನಸ್ಸಿದೆ; ಅದರ ಹಿಂದೆ ಆತ್ಮವಿದೆ. ಇದರಂತೆಯೇ ಇಡೀ ಬ್ರಹ್ಮಾಂಡ ದೇಹ, ಅದರ ಹಿಂದೆ ಈಶ್ವರನ ಮನಸ್ಸಿದೆ, ಅದರ ಹಿಂದೆ ವಿಶ್ವಾತ್ಮನಿರುವನು. ಈ ದೇಹ ಹೇಗೆ ವಿಶ್ವದೇಹದ ಒಂದು ಅಂಶವೊ, ಈ ಮನಸ್ಸು ವಿಶ್ವಮನಸ್ಸಿನ ಒಂದು ಅಂಶವೊ, ಹಾಗೆಯೆ ಜೀವಾತ್ಮನೂ ವಿಶ್ವಾತ್ಮನ ಒಂದು ಅಂಶ. ಇದೇ ವಿಶಿಷ್ಟಾದ್ವೈತ, ವಿಶ್ವಾತ್ಮ ಅನಂತವೆಂದು ನಮಗೆ ಗೊತ್ತಿದೆ. ಅನಂತದಲ್ಲಿ ಹೇಗೆ ಭಾಗಗಳು ಇರುವುದಕ್ಕೆ ಸಾಧ್ಯ? ಅದು ಹೇಗೆ ಭಿನ್ನವಾಗುವುದು? ಭಾಜ್ಯವಾಗುವುದು? ಅನಂತಾತ್ಮನ ಒಂದು ಕಿಡಿ ನಾನು ಎಂದು ಹೇಳುವುದು ಕಾವ್ಯಮಯವಾಗಿರಬಹುದು. ಆದರೆ ವಿಚಾರಮತಿಗಳಿಗೆ ಇದರಲ್ಲಿ ಯಾವ ಅರ್ಥವೂ ಇಲ್ಲ. ಅನಂತವನ್ನು ಭಾಗಿಸುವುದೆಂದರೇನು? ಅದೇನು ಒಂದು ಭೌತಿಕ ವಸ್ತುವೆ, ಭಾಗಿಸುವುದಕ್ಕೆ? ಅನಂತವನ್ನು ನಾವು ಎಂದಿಗೂ ಭಾಗಿಸಲಾರೆವು. ಅದು ಸಾಧ್ಯವಾದರೆ ಇನ್ನು ಅನಂತವಾಗಲಾರದು. ಹಾಗಾದರೆ ಪರಿಣಾಮವೇನಾಯಿತು? ಉತ್ತರವೆ, ವಿಶ್ವಾತ್ಮನೆ ನೀನು, ಅದರ ಭಾಗವಲ್ಲ, ಪೂರ್ಣವೇ ನೀನು. ನೀನೆ ಪೂರ್ಣದೇವರು. ಹಾಗಾದರೆ ಈ ವೈವಿಧ್ಯಗಳೇನು? ಕೋಟ್ಯಂತರ ಜೀವಾತ್ಮರಿರುವರಲ್ಲ ಅವುಗಳೇನು? ಸೂರ್ಯನು ಕೋಟ್ಯಂತರ ಹಿಮಮಣಿಯ ಮೇಲೆ ಪ್ರಕಾಶಿಸಿದರೆ ಪ್ರತಿಯೊಂದು ಮಣಿಯಲ್ಲಿ ಸೂರ್ಯನ ಪೂರ್ಣ ಪ್ರತಿಬಿಂಬವಿರುತ್ತದೆ. ಆದರೆ ಅವು ಕೇವಲ ಪ್ರತಿಬಿಂಬ ಮಾತ್ರ. ನಿಜವಾಗಿರುವುದು ಒಂದೇ ಸೂರ್ಯ. ಪ್ರತಿಯೊಬ್ಬನಲ್ಲಿಯೂ ಇರುವ ತೋರಿಕೆಯ ಆತ್ಮವು ಭಗವಂತನ ಒಂದು ಪ್ರತಿಬಿಂಬ, ಅದಕ್ಕಿಂತ ಹೆಚ್ಚಲ್ಲ. ನಿಜವಾಗಿ ಹಿಂದಿರುವುದೊಂದೆ ದೇವರು. ನಾವೆಲ್ಲ ಅಲ್ಲಿ ಒಂದು. ವಿಶ್ವದಲ್ಲಿ ಇರುವ ಆತ್ಮ ಒಂದೆ, ಅದು ನನ್ನಲ್ಲಿದೆ, ನಿನ್ನಲ್ಲಿದೆ, ಅದೊಂದೆ. ಆ ಒಂದು ಆತ್ಮ ಹಲವು ದೇಹಗಳಲ್ಲಿ ಬೇರೆ ಬೇರೆ ಆತ್ಮಗಳಂತೆ ಕಾಣುತ್ತಿದೆ. ಆದರೆ ನಮಗೆ ಇದು ಗೊತ್ತಿಲ್ಲ. ನಾವೆಲ್ಲ ಬೇರೆ, ದೇವರಿಂದ ಬೇರೆ ಎಂದು ಭಾವಿಸುವೆವು. ಎಲ್ಲಿಯವರೆಗೆ ನಾವು ಹೀಗೆ ಭಾವಿಸುವೆವೋ ಅಲ್ಲಿಯವರೆಗೆ ಪ್ರಪಂಚದಲ್ಲಿ ನಮಗೆ ದುಃಖ ತಪ್ಪಿದ್ದಲ್ಲ. ಇದೇ ಭ್ರಾಂತಿ.

ನಮ್ಮ ದುಃಖದ ಇನ್ನೊಂದು ಮೂಲವೆ ಅಂಜಿಕೆ. ಒಬ್ಬ ಮತ್ತೊಬ್ಬನನ್ನು ಏತಕ್ಕೆ ಹಿಂಸಿಸುವನು? ಏಕೆಂದರೆ ಸಾಕಷ್ಟು ಸುಖ ತನಗೆ ಸಿಕ್ಕುವುದಿಲ್ಲ ಎಂದು ಅವನಿಗೆ ಅಂಜಿಕೆ. ಒಬ್ಬ ತನಗೆ ಸಾಕಾದಷ್ಟು ಹಣವಿಲ್ಲವೆಂದು ಅಂಜುವನು. ಆ ಅಂಜಿಕೆ ಇತರರನ್ನು ಹಿಂಸಿಸಿ ಹಣಸಂಪಾದನೆ ಮಾಡುವಂತೆ ಪ್ರೇರೇಪಿಸುವುದು; ಇರುವುದೊಂದೆ ಆದರೆ ಅಂಜಿಕೆ ಹೇಗೆ ಇರಬಲ್ಲದು? ನನ್ನ ತಲೆಯ ಮೇಲೆ ಒಂದು ಸಿಡಿಲು ಬಡಿದರೆ ನಾನೇ ಆ ಸಿಡಿಲು, ಏಕೆಂದರೆ ಇರುವವನು ನಾನೊಬ್ಬನೆ. ಪ್ಲೇಗು ಬಂದರೆ ನಾನೇ ಆ ಪ್ಲೇಗು, ವ್ಯಾಘ್ರ ಬಂದರೇ ನಾನೇ ವ್ಯಾಘ್ರ, ಮೃತ್ಯು ಬಂದರೇ ನಾನೆ ಆ ಮೃತ್ಯು. ನಾನೇ ಜನನ ಮರಣ. ಎರಡು ಇರುವುದೆಂಬ ಭಾವನೆ ಅಂಜಿಕೆಗೆ ಕಾರಣವೆಂದು ಗೊತ್ತಾಗುವುದು. ಮತ್ತೊಬ್ಬರನ್ನು ಪ್ರೀತಿಸಿ ಎಂಬುದನ್ನು ನಾವು ಯಾವಾಗಲೂ ಕೇಳಿರುವೆವು. ಏತಕ್ಕೆ ಪ್ರೀತಿಸಬೇಕು? ಆ ಬೋಧನೆಯನ್ನೇನೋ ಸಾರಿದರು. ಆದರೆ ಇದಕ್ಕೆ ವಿವರಣೆ ಇಲ್ಲಿದೆ. ನಾನೇಕೆ ಎಲ್ಲರನ್ನು ಪ್ರೀತಿಸಬೇಕು? ಏಕೆಂದರೆ ಅವರು ನಾನು ಇಬ್ಬರೂ ಒಂದೆ. ನಾನು ನನ್ನ ಸಹೋದರನನ್ನು ಏತಕ್ಕೆ ಪ್ರೀತಿಸಬೇಕು? ಏಕೆಂದರೆ ನಾವಿಬ್ಬರೂ ಒಂದೆ. ಏಕತ್ವವಿದೆ, ಇಡೀ ವಿಶ್ವವೆಲ್ಲ ಒಂದಾಗಿದೆ. ನಮ್ಮ ಕಾಲಕೆಳಗೆ ತೆವಳುತ್ತಿರುವ ಕೀಟದಿಂದ ಹಿಡಿದು, ಶ್ರೇಷ್ಠತಮ ಮಾನವನವರೆಗೆ ವಿವಿಧ ದೇಹಗಳಿವೆ. ಆದರೆ ಆತ್ಮ ಒಂದೇ. ಎಲ್ಲ ಬಾಯಿಗಳ ಮೂಲಕ ನೀನು ತಿನ್ನುವೆ. ಎಲ್ಲ ಕೈಗಳ ಮೂಲಕ ನೀನು ಕೆಲಸಮಾಡುವೆ. ಎಲ್ಲ ಕಣ್ಣುಗಳ ಮೂಲಕ ನೀನು ನೋಡುವೆ. ಕೋಟ್ಯಂತರ ದೇಹಗಳ ಮೂಲಕ ನೀನು ಆರೋಗ್ಯವನ್ನು ಅನುಭವಿಸುತ್ತಿರುವೆ. ಕೋಟ್ಯಂತರ ದೇಹಗಳ ಮೂಲಕ ನೀನು ರೋಗವನ್ನು ಅನುಭವಿಸುತ್ತಿರುವೆ. ಈ ಭಾವನೆ ಬಂದು, ನಾವು ಇದನ್ನು ಮನಗಂಡಾಗ, ನೋಡಿದಾಗ, ಅನುಭವಿಸಿದಾಗ ದುಃಖ ಮಾಯವಾಗುವುದು, ಅದರ ಜೊತೆಯಲ್ಲೇ ಅಂಜಿಕೆಯೂ ಕೂಡ. ನಾನು ಹೇಗೆ ಸಾಯಬಲ್ಲೆ? ನನ್ನನ್ನು ಮೀರಿ ಯಾವುದೂ ಇಲ್ಲ. ಅಂಜಿಕೆ ಮಾಯವಾಗುವುದು. ಆಗ ಮಾತ್ರ ಪೂರ್ಣ ಆನಂದ, ಪ್ರೀತಿ ಉದಿಸಬಲ್ಲದು. ಬದಲಾಯಿಸದ ಆ ವಿಶ್ವಾನುಕಂಪ, ವಿಶ್ವಪ್ರೇಮ, ವಿಶ್ವಾನಂದ ಮಾನವನನ್ನು ಎಲ್ಲರಿಗಿಂತ ಮೇಲಿನ ಮಟ್ಟಕ್ಕೆ ಒಯ್ಯುವುದು. ಅದಕ್ಕೆ ಯಾವ ಪ್ರತಿಕ್ರಿಯೆಯೂ ಇಲ್ಲ. ಯಾವ ದುಃಖವೂ ಅದನ್ನು ಸೋಂಕಲಾರದು. ಆದರೆ ಪ್ರಪಂಚದಲ್ಲಿ ಈ ಕುಡಿಯುವುದು, ತಿನ್ನುವುದು ಇವೆಲ್ಲವೂ ಪ್ರತಿಕ್ರಿಯೆಯನ್ನುಂಟುಮಾಡುವುವು. ಇದಕ್ಕೆಲ್ಲ ಕಾರಣ ದ್ವೈತಭಾವನೆ; ನಾನು ದೇವರಿಂದ ಬೇರೆ, ಪ್ರಪಂಚದಿಂದ ಬೇರೆ ಎನ್ನುವುದು. ಆದರೆ ಎಂದು 'ನಾನೇ ಅವನು, ನಾನೇ ವಿಶ್ವಾತ್ಮ, ನಿತ್ಯ ಬುದ್ಧ” ಎಂದು ಅರಿಯುವೆನೊ ಆಗ ನಿಜವಾದ ಪ್ರೇಮ ಉದಿಸುವುದು, ಅಂಜಿಕೆ ಮಾಯವಾಗುವುದು, ದುಃಖ ಕೊನೆಗಾಣುವುದು.

