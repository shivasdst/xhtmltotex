
\chapter[ಯೋಗ ಮಾರ್ಗ]{ಯೋಗ ಮಾರ್ಗ\protect\footnote{\engfoot{C.W, Vol. VIII, P 36}}}

ರಾಜಯೋಗ ಇತರ ವಿಜ್ಞಾನಶಾಸ್ತ್ರಗಳಂತೆಯೇ ಒಂದು ವಿಜ್ಞಾನಶಾಸ್ತ್ರ, ಇದೊಂದು ಮನಸ್ಸಿನ ವಿಶ್ಲೇಷಣೆ; ಅತೀಂದ್ರಿಯಕ್ಕೆ ಸಂಬಂಧಪಟ್ಟ ವಿಷಯಗಳನ್ನು ಸಂಗ್ರಹಿಸಿ ಅದರ ಆಧಾರದ ಮೇಲೆ ಆಧ್ಯಾತ್ಮಿಕ ಜೀವನವನ್ನು ಕಟ್ಟುವುದು. ಪ್ರಪಂಚದಲ್ಲಿ ಪ್ರಖ್ಯಾತರಾದ ಗುರುಗಳೆಲ್ಲರೂ “ನನಗೆ ಗೊತ್ತಿದೆ, ನಾನು ನೋಡಿರುವೆ” ಎಂದಿರುವರು. ಏಸು, ಪಾಲ್, ಪೀಟರ್ ಇವರೆಲ್ಲರೂ ಜನರಿಗೆ ಬೋಧಿಸಿದ ಆಧ್ಯಾತ್ಮಿಕ ಸತ್ಯಗಳನ್ನು ತಾವು ಪ್ರತ್ಯಕ್ಷ ಮನಗಂಡುದು ಎನ್ನುವರು.

ಈ ಅನುಭವ ಯೋಗದಿಂದ ಸಿದ್ಧಿಸುವುದು.

ಜೀವನದಲ್ಲಿ ಕೇವಲ ನೆನಪು ಮತ್ತು ಅರಿವು ಎರಡೇ ಅಲ್ಲ ಇರುವುದು. ಒಂದು ಪ್ರಜ್ಞಾತೀತಾವಸ್ಥೆಯೂ ಇದೆ. ಅಪ್ರಜ್ಞೆ (\enginline{Unconscious}) ಮತ್ತು ಅತಿಪ್ರಜ್ಞೆ (\enginline{Super conscious}) ಎರಡರಲ್ಲಿಯೂ ಅರಿವಿಲ್ಲ. ಆದರೆ ಅವೆರಡಕ್ಕೂ ಧ್ರುವದಷ್ಟು ಅಂತರ. ಒಂದು ಅಜ್ಞಾನ, ಮತ್ತೊಂದು ಜ್ಞಾನ. ಈಗ ಹೇಳುವ ಯೋಗ ಕೂಡ ವಿಜ್ಞಾನಶಾಸ್ತ್ರಗಳಂತೆ ಯುಕ್ತಿಯ ತಳಹದಿಯ ಮೇಲೆ ಇರುವುದು.

ಮನಸ್ಸಿನ ಏಕಾಗ್ರತೆಯೇ ಎಲ್ಲಾ ಜ್ಞಾನದ ಮೂಲ.

ಯೋಗವು, ಭೌತಿಕಪ್ರಪಂಚವನ್ನು ಹೇಗೆ ನಮ್ಮ ಸೇವಕರನ್ನಾಗಿ ಮಾಡಿಕೊಳ್ಳುವುದು ಎಂಬುದನ್ನು ಬೋಧಿಸುವುದು. ಯೋಗ ಎಂದರೆ ಒಂದುಗೂಡಿಸುವುದು ಎಂದು ಅರ್ಥ. ಅಂದರೆ ಜೀವಾತ್ಮನನ್ನು ಪರಮಾತ್ಮನೊಂದಿಗೆ ಸೇರಿಸುವುದು ಎಂದು.

ಮನಸ್ಸು ಪ್ರಜ್ಞೆಯಲ್ಲಿ, ಪ್ರಜ್ಞೆಯ ಮೂಲಕ ಕೆಲಸ ಮಾಡುವುದು. ಯಾವುದನ್ನು ಪ್ರಜ್ಞೆ ಎನ್ನುವೆವೊ ಅದು ನಮ್ಮ ಸ್ವಭಾವವೆಂಬ ಅನಂತ ಸರಪಳಿಯಲ್ಲಿ ಒಂದು ಕೊಂಡಿ ಮಾತ್ರ, 'ನಾನು' ಎಂಬುದರಲ್ಲಿ ಎಲ್ಲೋ ಸ್ವಲ್ಪ ಪ್ರಜ್ಞಾವಸ್ಥೆ (\enginline{Conscious life}) ಮತ್ತು ಬೇಕಾದಷ್ಟು ಅಪ್ರಜ್ಞೆ (\enginline{Unconscious}) ಇದೆ. ಅದರ ಮೇಲೆ ನಮಗೆ ಬಹುಪಾಲು ಗೊತ್ತಿಲ್ಲದ ಅತಿಪ್ರಜ್ಞೆ (\enginline{Super conscious}) ಇದೆ.

ನಾವು ಶ್ರದ್ದೆಯಿಂದ ಅಭ್ಯಾಸ ಮಾಡಿದರೆ ನಮ್ಮ ಮನಸ್ಸಿನ ಹಲವು ಪದರಗಳು ವ್ಯಕ್ತವಾಗುವುವು, ಪ್ರತಿಯೊಂದೂ ನಮಗೆ ಹೊಸ ವಿಷಯಗಳನ್ನು ತೋರುವುವು. ನಮ್ಮೆದುರಿಗೆ ಒಂದು ಹೊಸ ಜಗತ್ತನ್ನು ಸೃಷ್ಟಿಸಿದಂತೆ ಆಗುವುದು. ಹೊಸ ಶಕ್ತಿ ನಮ್ಮ ಕರಗತವಾಗುವುದು. ಆದರೆ ನಾವು ಅಲ್ಲೇ ನಿಲ್ಲಕೂಡದು, ಅಥವಾ ಅವುಗಳ ವ್ಯಾಮೋಹಕ್ಕೆ ಬೀಳಬಾರದು. ಇವೆಲ್ಲ ಮುಂದಿರುವ ವಜ್ರದ ಗಣಿಯೊಡನೆ ಹೋಲಿಸಿದರೆ ಗಾಜಿನ ಚೂರುಗಳಂತೆ.

ದೇವರೊಬ್ಬನೇ ನಮ್ಮ ಗುರಿ: ಅವನನ್ನು ಸೇರದೆ ಇದ್ದರೆ ನಾವು ನಾಶವಾಗುವೆವು.

ಈ ಜೀವನದಲ್ಲಿ ಮುಂದುವರಿಯಬೇಕಾದರೆ ಮೂರು ವಸ್ತುಗಳು ಆವಶ್ಯಕ. ಮೊದಲು ಇಹ ಮತ್ತು ಪರ ಫಲಭೋಗ ವಿರಾಗವಿರಬೇಕು. ದೇವರು ಮತ್ತು\break ಸತ್ಯವನ್ನು ಮಾತ್ರ ಲಕ್ಷ್ಯ ದಲ್ಲಿಡಬೇಕು. ನಾವಿಲ್ಲಿರುವುದು ಸತ್ಯವನ್ನು ಅರಿಯುವುದಕ್ಕೆ, ಭೋಗಕ್ಕಲ್ಲ. ಅದನ್ನೆಲ್ಲ ಮೂರ್ಖರಿಗೆ ಬಿಡಿ. ನಾವೆಂದಿಗೂ ಅವರಂತೆ ಅನುಭವಿಸಲಾರೆವು, ಮಾನವ ಆಲೋಚನಾಜೀವಿ. ಅವನು ಮೃತ್ಯುವನ್ನು ಜಯಿಸುವವರೆಗೆ ಮತ್ತು ಸತ್ಯವನ್ನು ಕಾಣುವವರೆಗೆ ಹೋರಾಡಬೇಕು. ನಿಷ್ಟ್ರಯೋಜನವಾದ ಮಾತಿನಲ್ಲಿ ಅವನು ಕಾಲಕಳೆಯಬಾರದು. ಸಮಾಜ ಮತ್ತು ಜನಸಾಧಾರಣರ ಮತವನ್ನು ಪೂಜಿಸುವುದೇ ವಿಗ್ರಹಾರಾಧನೆ. ಆತ್ಮನಿಗೆ ಲಿಂಗ ಕಾಲ ದೇಶ ಭೇದಗಳಿಲ್ಲ.

ಎರಡನೆಯದು, ಸತ್ಯ ಮತ್ತು ದೇವರನ್ನು ನೋಡಲು ತೀವ್ರ ಅಭಿಲಾಷೆ. ನೀರಿನಲ್ಲಿ ಮುಳುಗುತ್ತಿರುವ ಮನುಷ್ಯ ಗಾಳಿಗಾಗಿ ಕಾತರನಾಗುವಂತೆ ದೇವರಿಗೆ ಕಾತರರಾಗಿ, ಅವನಿಗಾಗಿ ಹಂಬಲಿಸಿ, ದೇವರನ್ನು ಮಾತ್ರ ಆಶಿಸಿ, ಮತ್ತಾವುದನ್ನೂ ಸ್ವೀಕರಿಸಬೇಡಿ. ಈ ತೋರಿಕೆಯ ಭ್ರಾಂತಿ ನಿಮ್ಮನ್ನು ಪುನಃ ಮೋಸಗೊಳಿಸದಿರಲಿ. ಎಲ್ಲರಿಂದಲೂ ವಿಮುಖರಾಗಿ ದೇವರನ್ನು ಮಾತ್ರ ಅರಸಿ.

ಮೂರನೆಯದು ಶಮದಮಾದಿ ಷಟ್‌ಸಂಪತ್ತಿ. ಇದರಲ್ಲಿ ಮೊದಲನೆಯದು ಶಮ. ಮನಸ್ಸನ್ನು ಹೊರಗೆ ಹೋಗದಂತೆ ತಡೆಯುವುದು. ಎರಡನೆಯದು ದಮ, ಇಂದ್ರಿಯಗಳನ್ನು ನಿಗ್ರಹಿಸುವುದು. ಮೂರನೆಯದೆ ಉಪರತಿ. ಮನಸ್ಸನ್ನು ಅಂತರ್ಮುಖ ಮಾಡುವುದು. ನಾಲ್ಕನೆಯದೆ ತಿತಿಕ್ಷೆ, ಎಲ್ಲವನ್ನೂ ಗೊಣಗಾಡದೆ ಸಹಿಸುವುದು. ಐದನೆಯದೆ ಸಮಾಧಾನ. ಅಂದರೆ ಮನಸ್ಸನ್ನು ಒಂದು ವಸ್ತುವಿನ ಮೇಲೆ ಕೇಂದ್ರೀಕರಿಸುವುದು. ನಿಮ್ಮ ಎದುರಿಗಿರುವ ವಸ್ತುವನ್ನು ಕುರಿತು ಯೋಚಿಸಿ, ಎಂದಿಗೂ ಅದನ್ನು ಮರೆಯಬೇಡಿ. ಕಾಲವನ್ನು ಗಣನೆಗೆ ತರಬೇಡಿ. ಆರನೆಯದೆ ನಿಮ್ಮ ಸ್ವಸ್ವರೂಪವನ್ನು ಕುರಿತು ಮನನಮಾಡುವುದು. ಭ್ರಾಂತಿಯಿಂದ ಪಾರಾಗಿ, ನೀವು ಯಾವುದಕ್ಕೂ ಯೋಗ್ಯರಲ್ಲ ಎಂದು ಭಾವಿಸಿ ಭ್ರಾಂತರಾಗಬೇಡಿ. ಹಗಲು ರಾತ್ರಿ ನಿಜಸ್ವರೂಪವನ್ನು ಕುರಿತು, ಕೊನೆಗೆ ದೇವರಲ್ಲಿ ನೀವು ಒಂದಾಗುವ ತನಕ ಅದನ್ನೇ ಒತ್ತಿ ಒತ್ತಿ ಹೇಳಿ.

ಮೇಲಿನ ಶಿಸ್ತು ಇಲ್ಲದೆ ಯಾವ ಪ್ರತಿಫಲವೂ ದೊರಕಲಾರದು. ಅಪ್ರಮೇಯ ನಮ್ಮ ಅರಿವಿಗೆ ಗೊತ್ತಿರಬಹುದು. ಆದರೆ ನಾವು ಅದನ್ನು ಎಂದಿಗೂ ವಿವರಿಸಲು ಆಗುವುದಿಲ್ಲ. ನಾವು ಅದನ್ನು ವಿವರಿಸಲು ಯತ್ನಿಸಿದೊಡನೆ ಅದನ್ನು ಮಿತಿಗೊಳಿಸುವೆವು. ಅದಿನ್ನು ಅಪ್ರಮೇಯವಾಗಲಾರದು.

ನಾವು ಇಂದ್ರಿಯಾತೀತರಾಗಬೇಕು. ಯುಕ್ತಿಗೂ ಅತೀತರಾಗಿ ಹೋಗಬೇಕು. ಇದನ್ನು ಮಾಡುವುದಕ್ಕೆ ನಮಗೆ ಶಕ್ತಿ ಇದೆ.

\begin{center}
೧
\end{center}

(ಪ್ರಾಣಾಯಾಮದ ಮೊದಲನೆಯ ಪಾಠವನ್ನು ಒಂದು ವಾರ ಅಭ್ಯಸಿಸಿದ ಮೇಲೆ ಶಿಷ್ಯನು ಗುರುವಿಗೆ ವರದಿ ಒಪ್ಪಿಸುತ್ತಾನೆ.)

ಇದು ವೈಯಕ್ತಿಕತೆಯನ್ನು ಅಭಿವ್ಯಕ್ತ ಪಡಿಸುವ ಸಾಧನ. ಪ್ರತಿಯೊಂದು ವೈಯಕ್ತಿಕತೆಯನ್ನೂ ನಾವು ರೂಢಿಸಬೇಕು. ಎಲ್ಲವೂ ಕೇಂದ್ರದಲ್ಲಿ ಸಂಧಿಸುವುವು. ಕಲ್ಪನೆಯೇ ಸ್ಫೂರ್ತಿಗೆ ದಾರಿ. ಕಲ್ಪನೆಯೇ ಎಲ್ಲಾ ಆಲೋಚನೆಗೂ ತಳಹದಿ. ಎಲ್ಲ ಪ್ರವಾದಿಗಳು, ಕವಿಗಳು, ಸಂಶೋಧಕರು ಎಲ್ಲರಿಗೂ ಮಹಾ ಕಲ್ಪನಾಶಕ್ತಿ ಇತ್ತು. ಪ್ರಕೃತಿಗೆ ವಿವರಣೆ ನಮ್ಮಲ್ಲಿಯೇ ಇದೆ. ಕಲ್ಲನ್ನು ಮೇಲಕ್ಕೆ ಎಸೆದರೆ ಕೆಳಗೆ ಬೀಳುವುದು. ಆದರೆ ಆಕರ್ಷಣ ಸಿದ್ದಾಂತ ನಮ್ಮಲ್ಲಿರುವುದು. ಹೊರಗೆ ಇಲ್ಲ. ಯಾರು ಅತಿಯಾಗಿ ಊಟ ಮಾಡುವರೋ ಅಥವಾ ಉಪವಾಸ ಮಾಡುವರೋ ಯಾರು ಅತಿಯಾಗಿ ನಿದ್ರಿಸುವರೋ ಅಥವಾ ಕಡಮೆ ನಿದ್ರಿಸುವರೋ ಅವರು ಎಂದಿಗೂ ಯೋಗಿಗಳಾಗಲಾರರು. ಅಜ್ಞಾನ, ಚಂಚಲತೆ, ಅಸೂಯೆ, ಸೋಮಾರಿತನ ಮತ್ತು ಪ್ರಾಪಂಚಿಕ ವಸ್ತುಗಳ ಮೇಲೆ ಅತಿ ಆಸಕ್ತಿ - ಇವೇ ಯೋಗ ಜೀವನದಲ್ಲಿ ಮುಂದುವರಿಯುವುದಕ್ಕೆ ಆತಂಕವಾಗಿರುವ ದೊಡ್ಡ ಶತ್ರುಗಳು.

ಆವಶ್ಯಕವಾಗಿ ಬೇಕಾದ ಮೂರು ಗುಣಗಳಿವು: ಮೊದಲು ಪರಿಶುದ್ಧತೆ, ಮಾನಸಿಕ ಮತ್ತು ದೈಹಿಕ ಪರಿಶುದ್ಧತೆ. ಎಲ್ಲಾ ಮಲಿನತೆಗಳನ್ನೂ, ಮನಸ್ಸನ್ನು ಅಧೋಗತಿಗೆ ಎಳೆವ ಆಲೋಚನೆಗಳೆಲ್ಲವನ್ನೂ ತ್ಯಜಿಸಬೇಕು. ಎರಡನೆಯದೆ ತಾಳ್ಮೆ. ಮೊದಮೊದಲು ಅದ್ಭುತವಾದ ದರ್ಶನಗಳಾಗುವುವು. ಆದರೆ ಕ್ರಮೇಣ ಅವೆಲ್ಲ ನಿಂತು ಹೋಗುವುವು. ಇದೇ ಬಹಳ ಕಷ್ಟದ ಕಾಲ. ಆ ಸಮಯದಲ್ಲಿ ನೆಚ್ಚುಗೆಡಬೇಡಿ. ತಾಳ್ಮೆ ಇದ್ದರೆ ಕೊನೆಗೆ ನೀವು ಜಯಿಸುವುದರಲ್ಲಿ ಸಂದೇಹವಿಲ್ಲ. ಮೂರನೆಯದೆ ಛಲ. ಸುಖ ದುಃಖಗಳಲ್ಲಿ, ಆರೋಗ್ಯ ಅನಾರೋಗ್ಯ ಕಾಲಗಳಲ್ಲಿಯೂ ಒಂದು ದಿನವೂ ಸಾಧನೆ ಮಾಡುವುದನ್ನು ಬಿಡಬೇಡಿ.

ಬೆಳಗಿನ ಝಾವ ಮತ್ತು ಸಂಧ್ಯಾಸಮಯ ಸಾಧನೆಗೆ ಅತಿ ಶ್ರೇಷ್ಠ ಕಾಲ. ಎರಡು ಕಾಲಗಳ ಸಂಧಿ ಸಮಯ ನಮ್ಮ ದೇಹದ ಚಲನೆಗಳಲ್ಲಿ ಅತಿ ಪ್ರಶಾಂತ ಸಮಯ. ಇದು ಸಾಧ್ಯವಿಲ್ಲದೆ ಇದ್ದರೆ ನಿದ್ರೆಯಿಂದ ಎದ್ದಾಗ ಮತ್ತು ನಿದ್ರೆಗೆ ಹೋಗುವಾಗ ಅಭ್ಯಾಸಮಾಡಿ. ದೇಹದ ನೈರ್ಮಲ್ಯ ಅತಿ ಆವಶ್ಯಕ. ಪ್ರತಿದಿನ ಸ್ನಾನಮಾಡಬೇಕು.

ಸ್ನಾನವಾದ ಮೇಲೆ ಸ್ಥಿರವಾಗಿ ಕುಳಿತುಕೊಳ್ಳಿ. ಬಂಡೆಯಂತೆ ಅಚಲವಾಗಿ ಕುಳಿತಿರುವೆ, ಯಾವುದೂ ನನ್ನನ್ನು ಕದಲಿಸಲಾರದು ಎಂದು ಭಾವಿಸಿ. ಶಿರ, ಭುಜ, ಸೊಂಟವನ್ನು ಒಂದೇ ಸಮನಾಗಿಡಿ. ಬೆನ್ನು ಮೂಳೆಯನ್ನು ನೇರವಾಗಿಡಿ. ಚಲನೆಯೆಲ್ಲ ಅದರ ಮೂಲಕ ಆಗಬೇಕಾಗಿದೆ. ಅದಕ್ಕೆ ಆತಂಕವನ್ನು ತರಬೇಡಿ.

ನಿಮ್ಮ ದೇಹದ ಪ್ರತಿಯೊಂದು ಅಂಗಾಂಗವೂ ಪರಿಪೂರ್ಣವಾಗಿದೆ ಎಂದು ಭಾವಿಸಿ. ನಿಮ್ಮ ಕಾಲಿನ ಹೆಬ್ಬೆರಳಿನಿಂದ ಪ್ರಾರಂಭಿಸಿ, ನಿಮ್ಮ ದೇಹದ ಪ್ರತಿಯೊಂದು ಅಂಗವನ್ನೂ ಮುಟ್ಟಿ ಇದು ದೃಢವಾಗಿದೆ ಎಂದು ಬೇಕಾದರೆ ಚಿತ್ರಿಸಿಕೊಳ್ಳಿ. ಹೀಗೆಯೇ ಸಹಸ್ರಾರದವರೆಗೆ ಹೋಗಿ. ಪ್ರತಿಯೊಂದು ಅಂಗವೂ ಸರಿಯಾಗಿದೆ, ಯಾವ ಊನವೂ ಇಲ್ಲ ಎಂದು ಭಾವಿಸಿ. ಅನಂತರ ಇಡೀ ದೇಹ ದೃಢವಾಗಿದೆ ಎಂದು ಭಾವಿಸಿ. ಸತ್ಯ ಸಾಕ್ಷಾತ್ಕಾರಕ್ಕೆ ಈಶ್ವರ ಕರುಣಿಸಿರುವ ಒಂದು ಯಂತ್ರ ದೇಹ. ಸಂಸಾರ ಸಾಗರವನ್ನು ದಾಟಿ ಸಚ್ಚಿದಾನಂದ ತೀರವನ್ನು ಸೇರಲು ಇದೊಂದು ದೋಣಿ ಎಂದು ಭಾವಿಸಿ. ಇದಾದ ಮೇಲೆ ಎರಡು ಮೂಗಿನ ಹೊಳ್ಳೆಗಳ ಮೂಲಕವಾಗಿ ದೀರ್ಘವಾಗಿ ಉಸಿರನ್ನು\break ಎಳೆದುಕೊಳ್ಳಿ. ಅನಂತರ ಉಸಿರನ್ನು ಹೊರಗೆ ಬಿಟ್ಟು ನಿಮಗೆ ಅನುಕೂಲವಾದಷ್ಟು ಕಾಲ ಉಸಿರನ್ನು ಹೊರಗೆ ಬಿಟ್ಟಿರಿ. ಹೀಗೆ ನಾಲ್ಕು ವೇಳೆ ಮಾಡಿ. ಅನಂತರ ಸ್ವಾಭಾವಿಕವಾಗಿ ಉಸಿರಾಡಿ, ಜ್ಞಾನಜ್ಯೋತಿಯನ್ನು ನೀಡೆಂದು ದೇವರನ್ನು ಪ್ರಾರ್ಥಿಸಿ.

“ಯಾರು ವಿಶ್ವವನ್ನು ಸೃಷ್ಟಿಸಿರುವನೋ ಅಂತಹ ದಿವ್ಯಜ್ಯೋತಿಯನ್ನು ಕುರಿತು ನಾನು ಧ್ಯಾನಿಸುತ್ತೇನೆ. ಅವನು ನನ್ನ ಬುದ್ಧಿಯನ್ನು ಪ್ರಚೋದಿಸಲಿ.'' ಹತ್ತು ಹದಿನೈದು ನಿಮಿಷಗಳು ಇದನ್ನೇ ಕುರಿತು ಚಿಂತಿಸಿ.

ಧ್ಯಾನದಲ್ಲಿ ಆಗುವ ಅನುಭವಗಳನ್ನು ಗುರುವಿಗಲ್ಲದೆ ಮತ್ತಾರಿಗೂ ಹೇಳಬೇಡಿ.

ಸಾಧ್ಯವಾದಷ್ಟು ಕಡಿಮೆ ಮಾತನಾಡಿ, ಶುದ್ದ ಚಾರಿತ್ರ್ಯವನ್ನು ರೂಢಿಸಿಕೊಳ್ಳಿ. ನಾವು ಆಲೋಚಿಸಿದಂತೆ ಆಗುವೆವು. ಪವಿತ್ರಧ್ಯಾನ ನಮ್ಮ ಮನೋಮಾಲಿನ್ಯವನ್ನೆಲ್ಲ ನಾಶಮಾಡಲು ಸಹಾಯಮಾಡುವುದು, ಯೋಗಿಗಳಲ್ಲದವರೆಲ್ಲ ದಾಸರು. ನಾವು ಮುಕ್ತರಾಗಬೇಕಾದರೆ ಬಂಧನಗಳ ಮೇಲೆ ಬಂಧನಗಳನ್ನು ತುಂಡರಿಸಬೇಕಾಗಿದೆ.

ಎಲ್ಲರೂ ಅತೀಂದ್ರಿಯ ಸತ್ಯವನ್ನು ಪಡೆಯಬಹುದು. ದೇವರು ಸತ್ಯವಾದರೆ ನಾವು ಅವನನ್ನು ನೋಡಬೇಕು. ಆತ್ಮವಿದ್ದರೆ ನಾವು ಅದನ್ನು ಅನುಭವಿಸಬೇಕು, ಸಾಕ್ಷಾತ್ಕಾರಮಾಡಿಕೊಳ್ಳಬೇಕು.

ಆತ್ಮನನ್ನು ಹುಡುಕಬೇಕಾದರೆ ನಾವು ದೇಹಾತೀತರಾಗಬೇಕಾಗಿದೆ. ಯೋಗಿಗಳು ನಮ್ಮ ಇಂದ್ರಿಯಗಳನ್ನು ಎರಡು ವಿಭಾಗ ಮಾಡುವರು. ಒಂದು ಜ್ಞಾನೇಂದ್ರಿಯ ಮತ್ತೊಂದು ಕರ್ಮೇಂದ್ರಿಯ.

ಅಂತಃಕರಣ ಅಥವಾ ಮನಸ್ಸಿಗೆ ನಾಲ್ಕು ಮುಖಗಳಿವೆ. ಮೊದಲನೆಯದೆ ಮನಸ್ಸು. ಅನುಮಾನಿಸುವುದು ಅಥವಾ ಆಲೋಚಿಸುವುದು ಇದರ ಪ್ರವೃತ್ತಿ. ಈ ಶಕ್ತಿ ಸಾಧಾರಣವಾಗಿ ನಷ್ಟವಾಗಿರುವುದು. ಏಕೆಂದರೆ ಇದನ್ನು ನಿಗ್ರಹಿಸಿರುವುದಿಲ್ಲ. ಇದನ್ನು ಸರಿಯಾಗಿ ನಿಗ್ರಹಿಸಿದರೆ ಅದ್ಭುತವಾದ ಶಕ್ತಿ ಕರಗತವಾಗುವುದು. ಎರಡನೆಯದೆ ಬುದ್ದಿ -ಅಂದರೆ ಇಚ್ಛೆ. ಮೂರನೆಯದೆ ಅಹಂಕಾರ, ನಾಲ್ಕನೆಯದೆ ಚಿತ್ತ. ಇತರ ಪ್ರವೃತ್ತಿಗಳೆಲ್ಲ ಇದರ ಮೂಲಕ ಕೆಲಸ ಮಾಡುವುದು. ಇದೇ ಇಡೀ ಮನಸ್ಸಿನ ಆಲಯ ಅಥವಾ ಹಲವು ಪ್ರವೃತ್ತಿಗಳೆಂಬ ಅಲೆಗಳೇಳುವ ಸರೋವರ.

ಯೋಗ ಎಂದರೆ ಚಿತ್ತವೃತ್ತಿಗಳನ್ನು ನಿಗ್ರಹಿಸುವ ಒಂದು ಶಾಸ್ತ್ರ. ಚಂದ್ರಕಾಂತಿಯು ಪ್ರಕ್ಷುಬ್ಧ ಸಾಗರದ ಮೇಲೆ ಬಿದ್ದರೆ ಅಲ್ಲಿ ಪ್ರತಿಬಿಂಬವು ಹೇಗೆ ಅಲೆಗಳಿಂದ ಕಾಂತಿಹೀನವಾಗಿ ಅಥವಾ ವಿಕಾರವಾಗಿ ಕಾಣುವುದೋ ಹಾಗೆಯೇ, ಆತ್ಮದ ಪ್ರತಿಬಿಂಬ ಚಿತ್ತವೃತ್ತಿಗಳಿಂದ ವಿಕಾರವಾಗಿ ಕಾಣುವುದು. ಕನ್ನಡಿಯಂತೆ ಸಾಗರವು ಶಾಂತವಾದಾಗ ಮಾತ್ರ ಚಂದ್ರನ ಪ್ರತಿಬಿಂಬವನ್ನು ನೋಡಬಹುದು. ಹಾಗೆಯೆ ಚಿತ್ತಸಾಗರ ಪ್ರಶಾಂತವಾದಾಗ ಮಾತ್ರ ಆತ್ಮವನ್ನು ನೋಡಬಹುದು.

\newpage

ಮನಸ್ಸು ಭೌತಿಕವಾದರೂ ಅದು ದೇಹವಲ್ಲ. ಅದು ಸೂಕ್ಷ್ಮಾವಸ್ಥೆಯಲ್ಲಿರುವ ಭೌತಿಕ ವಸ್ತು. ಮನಸ್ಸು ಯಾವಾಗಲೂ ದೇಹಕ್ಕೆ ದಾಸನಾಗಿಲ್ಲ. ಕೆಲವು ವೇಳೆ ದೇಹದಿಂದ ಬೇರೆಯಾದಾಗ ಇದು ನಮಗೆ ಗೊತ್ತಾಗುವುದು. ನಾವು ಇಂದ್ರಿಯಗಳನ್ನು ನಿಗ್ರಹಿಸಿದರೆ ನಮ್ಮ ಇಚ್ಛಾನುಸಾರ ಇದನ್ನು ಮಾಡಬಹುದು.

ಇದನ್ನು ನಾವು ಪೂರ್ಣವಾಗಿ ಮಾಡಿದರೆ ವಿಶ್ವವನ್ನೇ ನಿಗ್ರಹಿಸಬಹುದು. ಏಕೆಂದರೆ ಜಗತ್ತು ಎಂದರೆ ಇಂದ್ರಿಯಗಳು ನಮಗೆ ತರುವ ಸಂವೇದನೆಗಳು ಮಾತ್ರ. ಸ್ವಾತಂತ್ರ್ಯವೇ ಮೇಲ್ತರದ ಮಾನವನ ಲಕ್ಷಣ. ನಾವು ಇಂದ್ರಿಯವಶದಿಂದ ಪಾರಾದಾಗ ಮಾತ್ರ ಆಧ್ಯಾತ್ಮಿಕ ಜೀವನ ಪ್ರಾರಂಭವಾದಂತೆ. ಯಾರು ಇಂದ್ರಿಯಗಳ ಅಧೀನಕ್ಕೆ ಒಳಪಟ್ಟಿರುವರೋ ಅವರು ಗುಲಾಮರು.

ಚಿತ್ತವೃತ್ತಿಗಳನ್ನು ನಾವು ಸಂಪೂರ್ಣವಾಗಿ ನಿಗ್ರಹಿಸಿದೆವು ಎಂದರೆ ಅದು ನಮ್ಮ ದೇಹವನ್ನು ಕೊನೆಗಾಣಿಸುವುದು. ಕೋಟ್ಯಂತರ ವರುಷಗಳಿಂದ ನಾವು ಈ ಒಂದು ದೇಹ ಸೃಷ್ಟಿಗೆ ಅಷ್ಟೊಂದು ಕಷ್ಟಪಟ್ಟಿರುವೆವು. ಇದನ್ನು ಪಡೆಯುವ ಹೋರಾಟದಲ್ಲಿ ನಾವು ಇದನ್ನು ಏತಕ್ಕೆ ಪಡೆಯಬೇಕೆಂಬುದನ್ನು ಮರೆತಿರುವೆವು. ನಮಗೆ ದೇಹ ದೊರೆತಿರುವುದು ಪರಿಪೂರ್ಣರಾಗುವುದಕ್ಕೆ. ದೇಹ ರಚನೆಯೇ ಜೀವನದ ಗುರಿ ಎಂದು ಭಾವಿಸುವೆವು. ಇದೇ ಮಾಯೆ. ಈ ಭ್ರಾಂತಿಯನ್ನು ತೊಡೆದುಹಾಕಿ, ನಮ್ಮ ನಿಜವಾದ ಗುರಿಯ ಕಡೆ ಗಮನಕೊಡಬೇಕು. ನಾವು ದೇಹವಲ್ಲ, ಅದು ನಮ್ಮ ಗುಲಾಮ ಎಂಬುದನ್ನು ತಿಳಿಯಬೇಕು.

ನಾನು ದೇಹವಲ್ಲ ಎಂದು ಚಿಂತಿಸಲು ಯತ್ನಿಸಿ. ನಾವೇ ದೇಹಕ್ಕೆ ಪ್ರಾಣ ಮತ್ತು ಸಂವೇದನೆ ಇವುಗಳನ್ನು ಕೊಟ್ಟು, ಇದು ಸತ್ಯ, ಇದು ಬದುಕಿದೆ ಎಂದು ಭಾವಿಸುವೆವು. ನಾವು ದೇಹವನ್ನು ಬಹುಕಾಲ ಧರಿಸಿ ಈಗ ಅದು ನಮ್ಮಿಂದ ಬೇರೆ ಎಂಬುದನ್ನು ಮರೆತುಹೋಗಿರುವೆವು. ನಮ್ಮ ಇಚ್ಛೆ ಬಂದಾಗ ದೇಹವನ್ನು ಆಚೆಗೆ ಎಸೆದು, ಅದು ನಮ್ಮ ದಾಸ, ನಮ್ಮ ಯಂತ್ರ ಎಂದು ನೋಡುವಂತೆ ಮಾಡುವುದಕ್ಕೆ ಸಹಾಯ ಮಾಡುವುದು ಯೋಗ. ಯೋಗಾಭ್ಯಾಸದಲ್ಲಿ ಮಾನಸಿಕ ಶಕ್ತಿಯನ್ನು ನಿಗ್ರಹಿಸುವುದೇ ಪ್ರಥಮಗುರಿ. ಎರಡನೆಯದೆ ನಿಗ್ರಹಿಸಿದ ಶಕ್ತಿಯನ್ನು ಯಾವುದಾದರೊಂದು ವಸ್ತುವಿನ ಮೇಲೆ ಕೇಂದ್ರೀಕರಿಸುವುದು.

ನೀವು ಹೆಚ್ಚು ಮಾತಾಳಿಗಳಾದರೆ ಯೋಗಿಗಳಾಗಲಾರಿರಿ.

\begin{center}
೨
\end{center}

ಇದನ್ನು ಅಷ್ಟಾಂಗಯೋಗವೆನ್ನುವರು. ಎಂಟು ಮುಖ್ಯವಾದ ಭಾಗಗಳನ್ನಾಗಿ ಇದನ್ನು ಮಾಡಿರುವುದೇ ಅದಕ್ಕೆ ಕಾರಣ. ಇವುಗಳಲ್ಲಿ ಮೊದಲನೆಯದು ಯಮ. ಇದು ಬಹಳ ಮುಖ್ಯ. ಇದು ನಮ್ಮ ಜೀವನದಲ್ಲೆಲ್ಲಾ ಓತಪ್ರೋತವಾಗಿರಬೇಕು. ಇದರಲ್ಲಿ ಐದು ಭಾಗಗಳಿವೆ. ಮೊದಲನೆಯದು ಅಹಿಂಸೆ, ಕಾಯಾ ವಾಚಾ ಮನಸಾ ಇತರರಿಗೆ ಹಿಂಸೆ ಮಾಡಕೂಡದು. ಎರಡನೆಯದು ಸತ್ಯ. ಮೂರನೆಯದು ಅಸ್ತೇಯ, ನಾಲ್ಕನೆಯದು ಬ್ರಹ್ಮಚರ್ಯ. ಐದನೆಯದು ಅಪರಿಗ್ರಹ.

ಎರಡನೆ ಭಾಗವೆ ನಿಯಮ, ಎಂದರೆ ದೈಹಿಕ ಶೌಚಗಳಾದ ಸ್ನಾನ, ಆಹಾರ ಇತ್ಯಾದಿಗಳು. ಮೂರನೆಯದೆ ಆಸನ, ಕುಳಿತುಕೊಳ್ಳುವ ರೀತಿ, ಸೊಂಟ, ಭುಜ, ಶಿರಸ್ಸು ಕುಳಿತಾಗ ನೇರವಾಗಿ ಇರಬೇಕು. ಬೆನ್ನುಮೂಳೆ ಬಾಗಿರಬಾರದು. ಅದಕ್ಕೆ ಯಾವ ಆತಂಕವೂ ಇರಕೂಡದು. ನಾಲ್ಕನೆಯದೆ ಪ್ರಾಣಾಯಾಮ ಎಂದರೆ ಉಸಿರಾಟದ ಮೇಲೆ ನಿಗ್ರಹ. ಐದನೆಯದೇ ಪ್ರತ್ಯಾಹಾರ, ಮನಸ್ಸನ್ನು ಅಂತರ್ಮುಖ ಮಾಡಿ, ಬಹಿರ್ಮುಖವಾಗದಂತೆ ತಡೆಯುವುದು. ವಿಷಯಗಳನ್ನು ತಿಳಿದುಕೊಳ್ಳುವುದಕ್ಕಾಗಿ ಅದನ್ನು ಮನಸ್ಸಿನಲ್ಲೆ ವಿಶ್ಲೇಷಣೆ ಮಾಡುವುದು. ಆರನೆಯದೆ ಧಾರಣ, ಒಂದು ವಸ್ತುವಿನ ಮೇಲೆ ಏಕಾಗ್ರ ಮಾಡುವುದು. ಏಳನೆಯದೆ ಧ್ಯಾನ, ಎಂಟನೆಯದೆ ಸಮಾಧಿ. ಇದೇ ನಮ್ಮ ಎಲ್ಲ ಸಾಧನೆಯ ಗುರಿ.

ಯಮನಿಯಮಗಳನ್ನು ನಾವು ಜೀವಾವಧಿ ಅಭ್ಯಾಸ ಮಾಡಬೇಕು. ಉಳಿದುವುಗಳನ್ನು ಜಿಗಣೆ ಹೇಗೆ ಹುಲ್ಲಿನ ಮುಂದಿನ ಎಸಳನ್ನು ಹಿಡಿದುಕೊಂಡ ಮೇಲೆ ಮಾತ್ರವೇ ಹಿಂದಿನದನ್ನು ಬಿಡುವುದೋ ಹಾಗೆ ಮಾಡಬೇಕು. ಅಂದರೆ ಮತ್ತೊಂದು ಮೆಟ್ಟಲನ್ನು ಹತ್ತಿ ಮೇಲೆ ಹೋಗುವುದಕ್ಕೆ ಮುಂಚೆ ನಿಂತಿರುವ ಕಡೆಯಲ್ಲಿ ಮೊದಲು ಎಲ್ಲಾ ವಿಷಯಗಳನ್ನೂ ತಿಳಿದುಕೊಳ್ಳಬೇಕು.

ಈ ಪಾಠದ ವಿಷಯ ಪ್ರಾಣಾಯಾಮ ಅಥವಾ ಪ್ರಾಣವನ್ನು ನಿಗ್ರಹಿಸುವುದು. ರಾಜಯೋಗದಲ್ಲಿ ಪ್ರಾಣಾಯಾಮ ಮಾನಸಿಕ ಕ್ಷೇತ್ರಕ್ಕೆ ಒಯ್ದು ನಮ್ಮನ್ನು ಅನಂತರ ಆಧ್ಯಾತ್ಮಿಕ ಕ್ಷೇತ್ರದ ಕಡೆ ಒಯ್ಯುವುದು. ಇಡೀ ದೇಹಕ್ಕೆ ಇದು ಮೂಲಚಕ್ರದಂತೆ ಇದೆ. ಇದು ಮೊದಲು ಶ್ವಾಸಕೋಶಗಳ ಮೇಲೆ ಕೆಲಸಮಾಡುವುದು. ಶ್ವಾಸಕೋಶ ಹೃದಯದ ಮೇಲೆ ತನ್ನ ಪರಿಣಾಮವನ್ನು ಬೀರುವುದು. ಹೃದಯವು ರಕ್ತಚಲನೆ ಮೇಲೆ ಕೆಲಸ ಮಾಡುವುದು. ರಕ್ತಚಲನೆ ಮಿದುಳಿನ ಮೇಲೆ ಕೆಲಸ ಮಾಡುವುದು. ಮಿದುಳು ಮನಸ್ಸಿನ ಮೇಲೆ ಕೆಲಸ ಮಾಡುವುದು. ಇಚ್ಛೆ ಬಾಹ್ಯವೇದನೆಯನ್ನು ಉಂಟುಮಾಡಬಲ್ಲದು. ಹಾಗೆಯೆ ಬಾಹ್ಯವೇದನೆ ಇಚ್ಛೆಯನ್ನು ಉಂಟುಮಾಡಬಲ್ಲದು. ನಮ್ಮ ಇಚ್ಛಾಶಕ್ತಿಗಳು ಬಹಳ ದುರ್ಬಲವಾಗಿವೆ. ಅವುಗಳ ಶಕ್ತಿ ನಮಗೆ ಇನ್ನೂ ಗೊತ್ತಿಲ್ಲ. ನಾವು ಬಾಹ್ಯ ವಸ್ತುಗಳಿಗೆ ದಾಸರಾಗಿರುವೆವು. ನಮ್ಮ ಮುಕ್ಕಾಲುಪಾಲು ಕೆಲಸವೆಲ್ಲ ಹೊರಗಿನಿಂದ ಒಳಗೆ ತೆಗೆದುಕೊಳ್ಳುವುದಾಗಿದೆ. ಬಾಹ್ಯಪ್ರಕೃತಿಯು ನಮ್ಮ ಸಮತ್ವಕ್ಕೆ ಭಂಗವನ್ನು ಉಂಟುಮಾಡುತ್ತದೆ. ನಾವು, ಈಗ ಇರುವಂತೆ ಬಾಹ್ಯಪ್ರಕೃತಿಯ ಸಮತೋಲನಕ್ಕೆ ಭಂಗತರಲಾರೆವು. ಇವೆಲ್ಲ ತಪ್ಪು. ನಿಜವಾಗಿ ಅಧಿಕತರವಾದ ಶಕ್ತಿ ನಮ್ಮೊಳಗೇ ಇರುವುದು.

ಮಹಾಗುರುಗಳು, ಸಾಧುಸಂತರು ತಮ್ಮೊಳಗಿರುವ ಮನಸ್ಸನ್ನು ಗೆದ್ದು ಪ್ರಪಂಚಕ್ಕೆ ಅಧಿಕಾರವಾಣಿಯಿಂದ ಮಾತನಾಡಿದರು. ಮಂತ್ರಿಯೊಬ್ಬ ಮೇಲಿನ ಗೋಪುರ\break ಒಂದರಲ್ಲಿ ಬಂದಿಯಾಗಿದ್ದು ಅವನ ಹೆಂಡತಿಯ ಸಹಾಯದಿಂದ ಪಾರಾದ ಒಂದು ಕಥೆ\footnote{ಕಥೆ: ಮಂತ್ರಿಯೊಬ್ಬ ರಾಜನ ಅಸಮಾಧಾನಕ್ಕೆ ತುತ್ತಾಗಿ ಎತ್ತರವಾದ ಗೋಪುರ ಒಂದರಲ್ಲಿ ಬಂದಿಯಾದ. ರಾತ್ರಿ ಅವನ ಹೆಂಡತಿ ಬಂದು ಯಾವ ರೀತಿ ಗಂಡನ ಬಿಡುಗಡೆಗೆ ತಾನು ಸಹಾಯವಾದೇನು ಎಂದು ಕೇಳಿದಳು. ಮಂತ್ರಿ ಮಾರನೆಯ ದಿನ ರಾತ್ರಿ ಒಂದು ದುಂಬಿ, ಜೇನುತುಪ್ಪ, ರೇಷ್ಮೆದಾರ, ದೊಡ್ಡದಾರ ಮತ್ತು ದಪ್ಪ ಹಗ್ಗವನ್ನು ತರುವಂತೆ ಹೇಳಿದನು. ಮಾರನೆಯ ದಿನ ಹೆಂಡತಿ ಗಂಡ ಹೇಳಿದುದನ್ನೆಲ್ಲ ತಂದಳು. ಮಂತ್ರಿಯು ದುಂಬಿಯ ತುದಿಗೆ ಸ್ವಲ್ಪ ಜೇನು ಸವರಿ ಅದಕ್ಕೆ ಒಂದು ಸಣ್ಣ ರೇಷ್ಮೆದಾರವನ್ನು ಕಟ್ಟಿ ಸೆರೆಮನೆಯ ಗೋಪುರದ ಕೆಳಗೆ ಗೋಡೆಯ ಹತ್ತಿರ ಬಿಡು ಎಂದನು. ಆ ದುಂಬಿ ತನ್ನ ತುದಿಯಲ್ಲಿರುವ ಜೇನಿನ ಆಸೆಗೆ ಅದು ಮುಂದೆ ಸಿಕ್ಕುವುದೆಂದು ಮೇಲೆ ಮೇಲೆ ಹತ್ತಿಹೋಗಿ ಮಂತ್ರಿ ಸೆರೆಯಿದ್ದ ಕಿಟಕಿಯ ಹತ್ತಿರ ಹೋಯಿತು. ಅದರ ಜೊತೆಯಲ್ಲೇ ಸಣ್ಣ ರೇಷ್ಮೆ ತಂತುವೂ ಹೋಯಿತು. ಮಂತ್ರಿ ದುಂಬಿಗೆ ಕಟ್ಟಿದ್ದ ರೇಷ್ಮೆದಾರವನ್ನು ಕೈಗೆ ತೆಗೆದುಕೊಂಡು ಅದರ ಕೊನೆಗೆ ದಪ್ಪ ದಾರವನ್ನು ಕಟ್ಟೆಂದು ಕೆಳಗೆ ಇದ್ದ ಹೆಂಡತಿಗೆ ಹೇಳಿದನು. ರೇಷ್ಮೆದಾರವನ್ನು ಎಳೆದ ಮೇಲೆ ಅದಕ್ಕೆ ಕಟ್ಟಿದ ದಪ್ಪದಾರ ಕೈಗೆ ಬಂದಿತು. ಅನಂತರ ಆ ದಪ್ಪದಾರಕ್ಕೆ ದೊಡ್ಡ ಹಗ್ಗವನ್ನು ಕಟ್ಟಿದಳು. ಆ ದಪ್ಪದಾರವನ್ನು ಎಳೆದ ಮೇಲೆ ಅದಕ್ಕೆ ಕಟ್ಟಿದ್ದ ಹಗ್ಗ ಇವನ ಕೈಗೆ ಸಿಕ್ಕಿತು. ಆ ಹಗ್ಗವನ್ನು ಕಿಟಿಕಿಯ ಕಂಬಿಗೆ ಕಟ್ಟಿ ಅಲ್ಲಿಂದ ಕೆಳಗೆ ಇಳಿದು ಸೆರೆಯಿಂದ ಪಾರಾದನು.} ಇದೆ. ಅವಳು ಒಂದು ದುಂಬಿ, ಜೇನುತುಪ್ಪ, ರೇಷ್ಮೆದಾರ, ದೊಡ್ಡದಾರ ಮತ್ತು ಹಗ್ಗವನ್ನು ತಂದು ಗಂಡನನ್ನು ಬಿಡಿಸಿದಳು. ನಮ್ಮ ಮನಸ್ಸನ್ನು ಪ್ರಾಣಾಯಾಮದಂತಹ ರೇಷ್ಮೆದಾರದಿಂದ ಹೇಗೆ ಗೆಲ್ಲಬಹುದೆಂಬುದನ್ನು ಇದು ಉದಾಹರಿಸುವುದು. ಇದರಿಂದ ಒಂದಾದ ಮೇಲೆ ಮತ್ತೊಂದು ನಮ್ಮ ಸ್ವಾಧೀನಕ್ಕೆ ಬಂದು ಧ್ಯಾನವೆಂಬ ಹಗ್ಗದಿಂದ ಕೊನೆಗೆ ದೇಹವೆಂಬ ಸೆರೆಮನೆಯಿಂದ ಪಾರಾಗಿ ಮುಕ್ತರಾಗಬಹುದು. ಮುಕ್ತರಾದ ಮೇಲೆ ನಮಗೆ ಸಾಧನವಾದ ವಸ್ತುಗಳನ್ನೆಲ್ಲ ಬೇಕಾದರೆ ತ್ಯಜಿಸಬಹುದು.

ಪ್ರಾಣಾಯಾಮದಲ್ಲಿ ಮೂರು ಭಾಗಗಳಿವೆ. ಒಂದು ಪೂರಕ, ಒಳಗೆ ಉಸಿರನ್ನು ಸೆಳೆದುಕೊಳ್ಳುವುದು. ಎರಡನೆಯದು ಕುಂಭಕ, ಒಳಗೆ ಉಸಿರನ್ನು ಕಟ್ಟುವುದು. ಮೂರನೆಯದು ರೇಚಕ, ಉಸಿರನ್ನು ಬಿಡುವುದು.

ಎರಡು ಶಕ್ತಿಗಳು ಮಿದುಳಿನಲ್ಲಿ ಸಂಚರಿಸುತ್ತಿವೆ. ಅವು ಮಿದುಳಿನಿಂದ ಬೆನ್ನೆಲುಬುಗಳ ಮೂಲಕವಾಗಿ ಇಳಿದು ಮೂಲಾಧಾರಕ್ಕೆ ಹೋಗಿ ಪುನಃ ಮಿದುಳಿಗೆ ಬರುವುವು. ಇವುಗಳಲ್ಲಿ ಒಂದನ್ನೆ ಪಿಂಗಳ-ಸೂರ್ಯನಾಡಿ ಎನ್ನುವುದು. ಇದು ಮಿದುಳಿನ ಎಡಭಾಗದಿಂದ ಹೊರಟು ಮಿದುಳಿನ ಕೆಳಭಾಗದಲ್ಲಿ ಬಲಕ್ಕೆ ತಿರುಗಿ ಅಲ್ಲಿಂದ ಹೊರಟು ಮೂಲಾಧಾರದಲ್ಲಿ ಎಂದರೆ ಬೆನ್ನು ಮೂಳೆಯ ಕೆಳಭಾಗದಲ್ಲಿ ಅದನ್ನು ಸುತ್ತಿಕೊಂಡು ಮೂಲಾಧಾರವನ್ನು ಎಂಟು ಎಂಬ ಸಂಖ್ಯೆಯ ಅರ್ಧದ (\enginline{S}) ಆಕಾರದಲ್ಲಿ ಸೇರುವುದು.

ಮತ್ತೊಂದು ಇಡ ಅಥವಾ ಚಂದ್ರನಾಡಿ. ಇದು ಕೆಳಗಿನಿಂದ ಹೊರಟು ಮಿದುಳಿಗೆ ಸೇರಿ ಎಂಟು ಎಂಬ ಸಂಖ್ಯೆಯನ್ನು ಪೂರ್ಣಗೊಳಿಸುತ್ತದೆ. ಕೆಳಗಿನ ಭಾಗ ಮೇಲಿನ ಭಾಗಕ್ಕಿಂತ ಉದ್ದವಾಗಿಯೇನೋ ಇರುವುದು. ಈ ಶಕ್ತಿಗಳು ಹಗಲು ರಾತ್ರಿ ಸಂಚರಿಸುತ್ತ ಚಕ್ರಗಳೆಂಬ ಕೇಂದ್ರಗಳಿಗೆ ಪ್ರಾಣ ಶಕ್ತಿಯನ್ನು ತುಂಬುವುವು. ಆದರೆ ಇವು ನಮಗೆ ಗೊತ್ತೇ ಆಗುವುದಿಲ್ಲ. ಏಕಾಗ್ರತೆಯಿಂದ ನಾವು ಇವನ್ನು ಅನುಭವಿಸಬಹುದು. ದೇಹದಲ್ಲಿ ಎಲ್ಲೆಲ್ಲಿ ಇವು ಇವೆ ಎಂಬುದನ್ನು ಕಂಡುಹಿಡಿಯಬಹುದು. ಈ ಇಡಪಿಂಗಳ ಚಲನೆಗಳಿಗೂ ಉಸಿರಾಟಕ್ಕೂ ನಿಕಟ ಸಂಬಂಧವಿದೆ. ನಾವು ಇದನ್ನು ನಿಗ್ರಹಿಸಿದರೆ ದೇಹ ನಮ್ಮ ವಶವಾಗುವುದು.

ಕಠೋಪನಿಷತ್ತಿನಲ್ಲಿ ದೇಹವನ್ನು ಒಂದು ರಥಕ್ಕೆ ಹೋಲಿಸಿರುವರು. ಮನಸೇ ಕಡಿವಾಣ, ಬುದ್ದಿಯೇ ಸಾರಥಿ, ಇಂದ್ರಿಯಗಳೇ ಅಶ್ವಗಳು, ವಿಷಯಗಳೇ ಮಾರ್ಗ. ಆತ್ಮನೆ ರಥದಲ್ಲಿ ಕುಳಿತಿರುವ ರಥಿ. ರಥಿ ಹೇಳಿದಂತೆ ಕೇಳಿ ಸಾರಥಿ ಸರಿಯಾಗಿ ಕುದುರೆಗಳನ್ನು ನಿಗ್ರಹಿಸದೆ ಇದ್ದರೆ ಅವನು ಗುರಿಯನ್ನು ಸೇರಲಾರ. ದುಷ್ಟ ಅಶ್ವಗಳಂತೆ ಕುದುರೆಗಳು ತಮ್ಮ ಮನಸ್ಸಿಗೆ ಬಂದೆಡೆ ಅವನನ್ನು ಎಳೆದುಕೊಂಡು ಹೋಗಿ ರಥಿಯನ್ನು ಕೊನೆಗೆ ನಾಶ ಮಾಡಲೂಬಹುದು. ಸಾರಥಿಯ ಕೈಯಲ್ಲಿರುವ ಲಗಾಮು, ಇಡ ಮತ್ತು ಪಿಂಗಳ ಎಂಬ ಚಲನೆಗಳು. ಇಂದ್ರಿಯಗಳನ್ನು ನಿಗ್ರಹಿಸಬೇಕಾದರೆ ಸಾರಥಿ ಇವನ್ನು ವಶಗೊಳಿಸಿಕೊಳ್ಳಬೇಕು. ನೀತಿವಂತರಾಗಿ ಬಾಳುವುದಕ್ಕೆ ನಮಗೆ ಶಕ್ತಿ ಬೇಕು. ಹಾಗೆ ಮಾಡಿದಲ್ಲದೆ ನಾವು ನಮ್ಮ ಕ್ರಿಯೆಗಳನ್ನು ನಿಗ್ರಹಿಸಲಾರೆವು. ಯೋಗ ಒಂದೇ ನಾವು ನೀತಿವಂತರಾಗಿ ಇರಲು ಸಾಧ್ಯವಾಗುವಂತೆ ಮಾಡುವುದು. ಎಲ್ಲಾ ಮಹಾಗುರುಗಳೂ ಕೂಡ ದೊಡ್ಡ ಯೋಗಿಗಳಾಗಿದ್ದರು. ಅವರು ಶಕ್ತಿಯನ್ನೆಲ್ಲ ನಿಗ್ರಹಿಸಿದ್ದರು. ಯೋಗಿಗಳು ಮೂಲಾಧಾರದಲ್ಲಿ ಈ ಶಕ್ತಿಯನ್ನು ತಡೆಗಟ್ಟಿ ಸುಷುಮ್ನಾ ನಾಡಿಯ ಮೂಲಕ ಮೇಲಕ್ಕೆ ಹೋಗುವಂತೆ ಮಾಡುತ್ತಿದ್ದರು. ಆಗ ಅವು ಜ್ಞಾನಪ್ರವಾಹಗಳಾಗುತ್ತಿದ್ದುವು. ಅವು ಯೋಗಿಗಳಲ್ಲಿ ಮಾತ್ರ ಇವೆ.

ಪ್ರಾಣಾಯಾಮದಲ್ಲಿ ಎರಡನೆಯ ಪಾಠ: ಎಲ್ಲರಿಗೂ ಒಂದೇ ಮಾರ್ಗ ಅನ್ವಯಿಸುವುದಿಲ್ಲ. ಪ್ರಾಣಾಯಾಮವನ್ನು ಲಯಬದ್ಧವಾಗಿ ಮಾಡಬೇಕು. ಹೀಗೆ ಮಾಡುವುದಕ್ಕೆ ಬಹಳ ಸುಲಭವಾದ ಮಾರ್ಗವೆ ಎಣಿಕೆ. ಇದು ಕೇವಲ ಯಾಂತ್ರಿಕವಾಗಿರುವುದರಿಂದ ಒಂದು ಎರಡೆಂದು ಎಣಿಸುವುದಕ್ಕೆ ಬದಲಾಗಿ `ಓಂ' ಮಂತ್ರವನ್ನು ಉಚ್ಚರಿಸುವೆವು.

ಪ್ರಾಣಾಯಾಮ ಮಾಡುವ ರೀತಿ ಹೀಗೆ ಇರುವುದು: ಹೆಬ್ಬೆಟ್ಟಿನಿಂದ ಮೂಗಿನ ಬಲಹೊಳ್ಳೆಯನ್ನು ಮುಚ್ಚಿ ಎಡಹೊಳ್ಳೆಯಿಂದ ನಿಧಾನವಾಗಿ ಉಸಿರನ್ನು ನಾಲ್ಕು ವೇಳೆ `ಓಂ' ಎಂದು ಉಚ್ಚರಿಸುತ್ತ ಸೆಳೆದುಕೊಳ್ಳಿ. ಅನಂತರ ಮೂಗಿನ ಎರಡು ಹೊಳ್ಳೆಗಳನ್ನು ಬೆರಳಿನಿಂದ ಮುಚ್ಚಿ, `ಓಂ' ಎಂಬುದನ್ನು ಎಂಟು ವೇಳೆ ಎಣಿಸುವವರೆಗೆ ಉಸಿರನ್ನು ಕಟ್ಟಿ, ಅನಂತರ ಮೂಗಿನ ಬಲ ಹೊಳ್ಳೆಯಿಂದ ಹೆಬ್ಬೆಟ್ಟನ್ನು ತೆಗೆದು ಅದರ ಮೂಲಕ ನಾಲ್ಕುವೇಳೆ `ಓಂ' ಎಂದು ಹೇಳುತ್ತ ಉಸಿರನ್ನು ಹೊರಕ್ಕೆ ಬಿಡಿ. ನೀವು ಉಸಿರನ್ನು ಬಿಟ್ಟಾದ ಮೇಲೆ ನಿಮ್ಮ ಕಿಬ್ಬೊಟ್ಟೆಯನ್ನು ಒಳಕ್ಕೆ ಸೆಳೆದುಕೊಂಡು ಗಾಳಿಯನ್ನೆಲ್ಲ ಹೊರಕ್ಕೆ ನೂಕಿ. ಅನಂತರ ಎಡಹೊಳ್ಳೆಯನ್ನು ಮುಚ್ಚಿ, ಕ್ರಮೇಣ ಮೂಗಿನ ಬಲ ಹೊಳ್ಳೆಯ ಮೂಲಕ ಉಸಿರನ್ನು ನಾಲ್ಕುವೇಳೆ `ಓಂ' ಎಂದು ಹೇಳುತ್ತ ಸೆಳೆದುಕೊಳ್ಳಿ. ಅನಂತರ ಹೆಬ್ಬೆಟ್ಟಿನಿಂದ ಬಲಹೊಳ್ಳೆಯನ್ನು ಮುಚ್ಚಿ ಎಂಟು ವೇಳೆ `ಓಂ' ಎಂದು ಹೇಳುವ ಪರಿಯಂತರ ಉಸಿರನ್ನು ಒಳಗೆ ಇಟ್ಟುಕೊಳ್ಳಿ. ಅನಂತರ ಎಡಹೊಳ್ಳೆಯನ್ನು ಬಿಟ್ಟು ನಾಲ್ಕುವೇಳೆ `ಓಂ' ಎಂದು ಉಚ್ಚರಿಸುತ್ತ ಉಸಿರನ್ನು ಹೊರಗೆ ಬಿಡಿ. ಹಾಗೆ ಉಸಿರನ್ನು ಬಿಡುವಾಗ ಹಿಂದಿನಂತೆಯೇ ಕಿಬ್ಬೊಟ್ಟೆಯನ್ನು ಒಳಕ್ಕೆ ಸೆಳೆದುಕೊಂಡಿರಬೇಕು. ನೀವು ಪ್ರತಿಸಲ ಪ್ರಾಣಾಯಾಮಕ್ಕೆ ಕುಳಿತಾಗಲೂ ಎರಡು ವೇಳೆ ಹೀಗೆ ಮಾಡಿ. ಅಂದರೆ ಎಡಮೂಗಿನಿಂದ ಎರಡು ಬಲಮೂಗಿನಿಂದ ಎರಡು ಒಟ್ಟು ನಾಲ್ಕು ಪ್ರಾಣಾಯಾಮಗಳು. ನೀವು ಪ್ರಾಣಾಯಾಮಕ್ಕೆ ಕುಳಿತುಕೊಳ್ಳುವುದಕ್ಕೆ ಮುಂಚೆ ಪ್ರಾರ್ಥನೆಯಿಂದ ಪ್ರಾರಂಭಿಸಿ.

ಇದನ್ನು ಒಂದು ವಾರ ಅಭ್ಯಾಸಮಾಡಿ, ಕ್ರಮೇಣ ಉಸಿರಾಟದಲ್ಲಿ ಕಾಲವನ್ನು ಹೆಚ್ಚಿಸುತ್ತಾ ಹೋಗಿ. ಆದರೆ ಅದೇ ಪ್ರಮಾಣದಲ್ಲಿರಬೇಕು. ಉಸಿರನ್ನು ಸೆಳೆದುಕೊಳ್ಳುವಾಗ ಆರುವೇಳೆ `ಓಂ' ಉಚ್ಚರಿಸಿದರೆ ಕುಂಭಕದಲ್ಲಿ ಹನ್ನೆರಡುವೇಳೆ, ಅನಂತರ ಉಸಿರನ್ನು ಬಿಡುವಾಗ ಆರುವೇಳೆ ಉಚ್ಚರಿಸಬೇಕು. ಈ ಅಭ್ಯಾಸ ನಮ್ಮನ್ನು ಹೆಚ್ಚು ಆಧ್ಯಾತ್ಮಿಕ ವ್ಯಕ್ತಿಗಳನ್ನಾಗಿ ಮಾಡುವುದು, ಹೆಚ್ಚು ಪರಿಶುದ್ಧರನ್ನಾಗಿ ಮಾಡುವುದು, ಹೆಚ್ಚು ಪವಿತ್ರರನ್ನಾಗಿ ಮಾಡುವುದು. ಅಡ್ಡ ಹಾದಿಗೆ ಬೀಳಬೇಡಿ. ಯಾವ ಶಕ್ತಿಯ ಕಡೆಗೂ ಗಮನ ಕೊಡಬೇಡಿ. ಪ್ರೇಮ ಒಂದೇ ನಮ್ಮ ಹತ್ತಿರ ಇರಬಲ್ಲದು. ಅದೊಂದೇ ವೃದ್ಧಿಯಾಗಬಲ್ಲದು. ಯಾರು ರಾಜಯೋಗದ ಮೂಲಕ ಭಗವಂತನ ಸಮೀಪಕ್ಕೆ ಬರಲಿಚ್ಚಿಸುವರೋ ಅವರ ದೈಹಿಕ, ಮಾನಸಿಕ, ನೈತಿಕ ಮತ್ತು ಆಧ್ಯಾತ್ಮಿಕ ಶಕ್ತಿಗಳು ಪರಿಪುಷ್ಟವಾಗಿರಬೇಕು. ಮುಂದೆ ಇಡುವ ಪ್ರತಿಯೊಂದು ಹೆಜ್ಜೆಯೂ ವಿವೇಚನಾಪೂರ್ವಕವಾಗಿರಲಿ.

ಸಾವಿರಾರು ಜೀವಿಗಳಲ್ಲಿ ಎಲ್ಲೋ ಒಬ್ಬ “ನಾನು ಮಾಯೆಯನ್ನು ದಾಟುತ್ತೇನೆ. ದೇವರನ್ನು ಸೇರುತ್ತೇನೆ'' ಎನ್ನುವನು. ಸತ್ಯವನ್ನು ಇದಿರಿಸುವ ಸಾಮರ್ಥ್ಯವುಳ್ಳವರು ಬಹಳ ಅಪರೂಪ. ಆದರೆ ನಾವು ಏನನ್ನಾದರೂ ಸಾಧಿಸಬೇಕಾದರೆ ಸತ್ಯಕ್ಕಾಗಿ ಸಾಯಲು ಸಿದ್ದರಾಗಿರಬೇಕು.

\begin{center}
೩
\end{center}

ಆತ್ಮವನ್ನು ಜಡ ಎಂದು ಭಾವಿಸಬೇಡಿ. ಅದರ ನೈಜಸ್ಥಿತಿಯನ್ನು ಕುರಿತು ಯೋಚಿಸಿ. ನಾವು ಆತ್ಮನನ್ನು ದೇಹ ಎಂದು ಬಗೆದಿರುವೆವು. ಆದರೆ ಅದನ್ನು ಇಂದ್ರಿಯ ಮತ್ತು ಆಲೋಚನೆಗಳಿಂದ ಬೇರ್ಪಡಿಸಬೇಕು. ಆಗ ಮಾತ್ರ ನಾವು ಅಮೃತಾತ್ಮರು ಎಂದು ನಮಗೆ ಗೊತ್ತಾಗುವುದು. ಬದಲಾವಣೆಯಲ್ಲಿ ಕಾರ್ಯಕಾರಣಗಳ ದ್ವಂದ್ವವಿದೆ.\break ಯಾವುದು ಬದಲಾಯಿಸುವುದೋ ಅದೆಲ್ಲ ಮರ್ತ್ಯವಾಗಿರಬೇಕು. ಆದಕಾರಣ ದೇಹ ಮತ್ತು ಮನಸ್ಸು ಮರ್ತ್ಯವೆಂದು ಗೊತ್ತಾಗುವುದು. ಏಕೆಂದರೆ ಎರಡೂ ಬದಲಾಯಿಸುತ್ತಿವೆ. ಅವಿಕಾರಿಯಾಗಿರುವುದು ಮಾತ್ರ ಅಮೃತವಾಗಬಲ್ಲದು. ಏಕೆಂದರೆ ಯಾವುದೂ ಅದರ ಮೇಲೆ ತನ್ನ ಪ್ರಭಾವವನ್ನು ಬೀರಲಾರದು.

ನಾವು ಸತ್ಯವಾಗುವುದಿಲ್ಲ, ನಾವೇ ಅದು. ಆದರೆ ಸತ್ಯವನ್ನು ಮುಚ್ಚಿರುವ ಅಜ್ಞಾನದ ತೆರೆಯನ್ನು ಸರಿಸಬೇಕಾಗಿದೆ. ದೇಹ ಎಂದರೆ ಬಾಹ್ಯದಲ್ಲಿ ಘನೀಭೂತವಾದ ಆಲೋಚನೆ. ಇಡ ಪಿಂಗಳ ಚಲನೆಗಳು ನಮ್ಮ ದೇಹದ ಎಲ್ಲಾ ಭಾಗಕ್ಕೂ ಶಕ್ತಿಯನ್ನು ತರುತ್ತವೆ. ಹೆಚ್ಚಿಗೆ ಇರುವ ಶಕ್ತಿ ಸುಷುಮ್ನಾ ಕಾಲುವೆಯಲ್ಲಿ ಚಕ್ರಗಳೆಂಬ ಕೆಲವು ನರಕೇಂದ್ರಗಳಲ್ಲಿ ಸಂಗ್ರಹವಾಗಿರುತ್ತವೆ.

ಈ ಶಕ್ತಿ ಶವಗಳಲ್ಲಿ ಕಾಣುವುದಿಲ್ಲ. ಆರೋಗ್ಯವಾಗಿರುವ ದೇಹಗಳಲ್ಲಿ ಮಾತ್ರ ನಾವು ಇದನ್ನು ಗುರುತಿಸಬಹುದು.

ಯೋಗಿಗೆ ಒಂದು ಅನುಕೂಲವಿದೆ. ಅವು ಅವನಿಗೆ ಅನುಭವವಾಗುವುದು ಮಾತ್ರವಲ್ಲ, ಅವನು ಅವನ್ನು ನೋಡಬಲ್ಲನು. ಅವು ಅವನ ಜೀವನದಲ್ಲಿ ಜಾಜ್ವಲ್ಯಮಾನವಾಗಿವೆ. ಹಾಗೆಯೆ ಷಟ್‌ಚಕ್ರಗಳು ಕೂಡ.

ನಮ್ಮಲ್ಲಿ ಅರಿವಿಲ್ಲದೆ ಆಗತಕ್ಕ ಕ್ರಿಯೆಗಳು ಇವೆ. ಅರಿವಿನಿಂದ ಆಗತಕ್ಕ ಕ್ರಿಯೆಗಳೂ ಇವೆ. ಯೋಗಿಗಳಿಗೆ ಮೂರನೆಯದಾದ, ಅತಿಪ್ರಜ್ಞೆಗೆ ಸೇರಿದ ಒಂದು ಭಾಗವೂ ಇದೆ. ಇದೇ ಎಲ್ಲಾ ಕಾಲದೇಶಗಳಲ್ಲೂ ಧಾರ್ಮಿಕ ಜ್ಞಾನದ ಮೂಲವಾಗಿದೆ. ಅತಿಪ್ರಜ್ಞೆ ಎಂದಿಗೂ ತಪ್ಪುಮಾಡುವುದಿಲ್ಲ. ಆದರೆ ಹುಟ್ಟುಗುಣದ ಅಭ್ಯಾಸದ ಮೇಲೆ ಮಾಡತಕ್ಕವು ಕೇವಲ ಯಾಂತ್ರಿಕವಾಗಿವೆ. ಆದರೆ ಅತಿಪ್ರಜ್ಞೆಯಾದರೋ ಪ್ರಜ್ಞಾ ಕ್ಷೇತ್ರವನ್ನೇ ಮೀರಿದೆ. ಇದನ್ನು ಸ್ಫೂರ್ತಿ ಎನ್ನುವರು. ಆದರೆ ಯೋಗಿಗಳು, ಈ ಶಕ್ತಿ ಎಲ್ಲಾ ಜೀವಿಗಳಲ್ಲಿಯೂ ಸುಪ್ತವಾಗಿದೆ. ಕ್ರಮೇಣ ಎಲ್ಲರೂ ಅದನ್ನು ಅನುಭವಕ್ಕೆ ತಂದುಕೊಳ್ಳುವರು ಎನ್ನುತ್ತಾರೆ.

ಇಡಾ ಪಿಂಗಳಗಳ ಪ್ರವಾಹವನ್ನು ನಾವು ಒಂದು ಹೊಸ ದಿಕ್ಕಿನಲ್ಲಿ ಹರಿಸಬೇಕು. ಬೆನ್ನು ಮೂಳೆಯ ಮಧ್ಯದ ಮೂಲಕ ಅದಕ್ಕೆ ಒಂದು ಹೊಸ ಮಾರ್ಗವನ್ನು ಕಲ್ಪಿಸಬೇಕು. ಈ ಸುಷುಮ್ನಾ ಕಾಲುವೆಯ ಮೂಲಕ ನಾವು ಮಿದುಳಿಗೆ ಶಕ್ತಿಯನ್ನು ತಂದರೆ ಸದ್ಯಕ್ಕೆ ನಾವು ದೇಹದಿಂದ ಸಂಪೂರ್ಣ ಬೇರೆಯಾದಂತೆ,

ಬೆನ್ನು ಮೂಳೆಯ ಮೂಲದಲ್ಲಿ ಸೇಕ್ರಮ್ (\enginline{Sacrum}) ಎಂಬಲ್ಲಿ ಇರುವ ನರಗಳ ಕೇಂದ್ರವು ಬಹಳ ಮುಖ್ಯ. ಅದು ಲೈಂಗಿಕ ಶಕ್ತಿಯಾದ (\enginline{Sexual Energy}) ಜನನ ದ್ರವ್ಯದ (\enginline{Generative Substance}) ಮೂಲಸ್ಥಾನವಾಗಿದೆ. ಇದನ್ನು ಯೋಗದಲ್ಲಿ ಸುರುಳಿ ಸುತ್ತಿಕೊಂಡಿರುವ ಸರ್ಪವಿರುವ ತ್ರಿಕೋಣವೆಂದು ಚಿತ್ರಿಸುವರು. ಇಲ್ಲಿ ಸುಪ್ತವಾಗಿರುವ ಸರ್ಪವನ್ನೇ ಕುಂಡಲಿನಿ ಎನ್ನುವರು. ಕುಂಡಲಿನಿಯನ್ನು ಜಾಗ್ರತಗೊಳಿಸುವುದೇ ರಾಜಯೋಗದ ಮುಖ್ಯ ಗುರಿ.

ಲೈಂಗಿಕ ಶಕ್ತಿಯನ್ನು ಮೃಗೀಯ ಕ್ರಿಯೆಯಿಂದ ತಪ್ಪಿಸಿ ಮಿದುಳು ಎಂಬ ಅದ್ಭುತವಾದ ಧೀಯಂತ್ರಕ್ಕೆ ಕಳುಹಿಸಿ ಅಲ್ಲಿ ಅದನ್ನು ಸಂಗ್ರಹಿಸಿಟ್ಟರೆ ಅದು ಓಜಸ್ಸು ಅಥವಾ ಆಧ್ಯಾತ್ಮಿಕ ಶಕ್ತಿಯಾಗಿ ಪರಿವರ್ತನೆ ಹೊಂದುವುದು. ಎಲ್ಲ ಸದ್ಭಾವನೆಗಳೂ, ಎಲ್ಲ ಪ್ರಾರ್ಥನೆಗಳೂ, ಆ ಮೃಗೀಯ ಶಕ್ತಿಯ ಒಂದು ಭಾಗವನ್ನು ಓಜಸ್ಸಾಗಿ ಪರಿವರ್ತಿಸುತ್ತವೆ ಮತ್ತು ಆಧ್ಯಾತ್ಮಿಕ ಶಕ್ತಿಯನ್ನು ಪಡೆಯಲು ಸಹಾಯ ಮಾಡುತ್ತವೆ. ಈ ಓಜಸ್ಸೆ ನಿಜವಾದ ಮನುಷ್ಯ. ಈ ಓಜಸ್ಸನ್ನು ಪೂರ್ಣವಾಗಿ ಸಂಗ್ರಹಿಸುವುದು ಮಾನವ ಜನ್ಮದಲ್ಲಿ ಮಾತ್ರ ಸಾಧ್ಯ. ಯಾರಲ್ಲಿ ಮೃಗೀಯ ಲೈಂಗಿಕ ಶಕ್ತಿಯೆಲ್ಲ ಓಜಸ್ಸಾಗಿ ಪರಿವರ್ತನವಾಗಿದೆಯೋ ಅವನು ದೇವನಿಗೆ ಸಮ. ಅವನು ಅಧಿಕಾರವಾಣಿಯಿಂದ ಮಾತನಾಡುವನು. ಅವನ ಉಪದೇಶ ಜಗತ್ತಿನ ಶ್ರೇಯಸ್ಸಿಗೆ ಸಾಧನವಾಗುವುದು.

ಯೋಗಿಗಳು ಈ ಕುಂಡಲಿನಿ ಎಂಬ ಸರ್ಪವು ಮೇಲಮೇಲಕ್ಕೆ ಏರುತ್ತ ಕೊನೆಗೆ ಸಹಸ್ರಾರವನ್ನು ಸೇರುವುದು ಎಂದು ಕಲ್ಪಿಸಿಕೊಳ್ಳುವರು. ಮನುಷ್ಯನಲ್ಲಿರುವ ಅತಿ ಶ್ರೇಷ್ಠ ಶಕ್ತಿಯಾದ ಲೈಂಗಿಕ ಶಕ್ತಿಯನ್ನು ಓಜಸ್ಸನ್ನಾಗಿ ಮಾರ್ಪಡಿಸುವ ತನಕ ಯಾವ ಸ್ತ್ರೀ ಪುರುಷರನ್ನೂ ಆಧ್ಯಾತ್ಮಿಕ ವ್ಯಕ್ತಿಗಳು ಎನ್ನಲಾಗುವುದಿಲ್ಲ.

ಯಾವ ಶಕ್ತಿಯನ್ನೂ ಉತ್ಪನ್ನ ಮಾಡುವುದಕ್ಕೆ ಆಗುವುದಿಲ್ಲ. ಇರುವ ಶಕ್ತಿಯನ್ನು ಬೇರೆ ದಾರಿಗೆ ತಿರುಗಿಸಬಹುದು. ಆದಕಾರಣ ಆಗಲೆ ನಮ್ಮಲ್ಲಿರುವ ಅದ್ಭುತಶಕ್ತಿಯನ್ನು ನಿಗ್ರಹಿಸುವುದನ್ನು ಕಲಿತುಕೊಂಡು ನಮ್ಮ ಇಚ್ಛಾಶಕ್ತಿಯ ಮೂಲಕ ಅದು ಕೇವಲ ಮೃಗೀಯ ಆನಂದದಲ್ಲಿ ವ್ಯರ್ಥವಾಗದಂತೆ ಅದನ್ನು ಆಧ್ಯಾತ್ಮಿಕ ಶಕ್ತಿಯನ್ನಾಗಿ ಪರಿವರ್ತನಗೊಳಿಸಬೇಕು. ಎಲ್ಲಾ ನೀತಿ ಮತ್ತು ಧರ್ಮದ ತಳಹದಿಯೇ ಬ್ರಹ್ಮಚರ್ಯ ಎಂಬುದನ್ನು ನೋಡಿದೆವು. ರಾಜಯೋಗದಲ್ಲಂತೂ ಕಾಯಾ ವಾಚಾ ಮನಸಾ ಬ್ರಹ್ಮಚರ್ಯ ಅತ್ಯಾವಶ್ಯಕ. ಈ ನಿಯಮ ಮದುವೆಯಾದವರಿಗೆ ಮತ್ತು ಮದುವೆಯಾಗದವರಿಗೆ ಇಬ್ಬರಿಗೂ ಅನ್ವಯಿಸುವುದು. ಒಬ್ಬ ತನ್ನ ಅತಿ ಮುಖ್ಯ ಶಕ್ತಿಯನ್ನು ಪೋಲುಗಳೆದರೆ ಅವನು ಆಧ್ಯಾತ್ಮಿಕ ವ್ಯಕ್ತಿಯಾಗಲಾರ.

ಎಲ್ಲ ಕಾಲದ ಮಹಾತ್ಮರೂ ಗೃಹಸ್ಥಾಶ್ರಮವನ್ನು ತ್ಯಜಿಸಿದವರಾಗಿದ್ದರು. ಇಲ್ಲವೆ ಸಂನ್ಯಾಸಿಗಳಾಗಿದ್ದರೆಂಬುದನ್ನು ಜಗದ ಇತಿಹಾಸ ಸಾರುವುದು, ಪರಿಶುದ್ಧಾತ್ಮರು ಮಾತ್ರ ಭಗವಂತನನ್ನು ನೋಡಬಲ್ಲರು.

ಪ್ರಾಣಾಯಾಮವನ್ನು ಮಾಡುವುದಕ್ಕೆ ಮುಂಚೆ ತ್ರಿಕೋಣವನ್ನು ಕಲ್ಪಿಸಿಕೊಳ್ಳಲು ಯತ್ನಿಸಿ. ನಿಮ್ಮ ಕಣ್ಣನ್ನು ಮುಚ್ಚಿಕೊಂಡು ನಿಮ್ಮ ಬಗೆಗಣ್ಣೆದುರಿಗೆ ಇದನ್ನು ಚಿತ್ರಿಸಿಕೊಳ್ಳಿ. ಆ ತ್ರಿಕೋಣವು ಜ್ವಾಲೆಯಿಂದ ಆವೃತವಾಗಿದೆ, ಅದರ ಮಧ್ಯದಲ್ಲಿ ಸರ್ಪ ಸುರುಳಿ ಸುತ್ತಿಕೊಂಡು ಇದೆ ಎಂದು ಭಾವಿಸಿ. ಕುಂಡಲಿನಿಯನ್ನು ಸ್ಪಷ್ಟವಾಗಿ ಚಿತ್ರಿಸಿಕೊಂಡು ಆದಮೇಲೆ ಅದನ್ನು ಬೆನ್ನುಮೂಳೆಯ ಅಡಿಯಲ್ಲಿ ಕಲ್ಪನೆಯ ಮೂಲಕ ಇಡಿ. ಕುಂಭಕದಲ್ಲಿ ಉಸಿರನ್ನು ಹಿಡಿಯುವಾಗ ಕುಂಡಲಿನಿ ಸರ್ಪದ ಹೆಡೆಯ ಮೇಲೆ ಅದನ್ನು ಜಾಗ್ರತಗೊಳಿಸುವುದಕ್ಕೆ ವೇಗವಾಗಿ ತಳ್ಳಿ. ಕಲ್ಪನೆ ಪ್ರಬಲವಾದಷ್ಟೂ ನಿಜವಾದ ಪ್ರತಿಫಲ ಬೇಗ ಸಿದ್ದಿಸುವುದು ಮತ್ತು ಕುಂಡಲಿನಿ ಜಾಗ್ರತಗೊಳ್ಳುವುದು. ಅದು ಜಾಗ್ರತವಾಗುವವರೆಗೆ ಅದು ಜಾಗ್ರತವಾಗುತ್ತಿದೆ ಎಂದು ಭಾವಿಸಿ. ಆ ಶಕ್ತಿ ನಿಮ್ಮಲ್ಲಿದೆ. ಅದನ್ನು ಸುಷುಮ್ನಾ ಕಾಲುವೆಯಲ್ಲಿ ಮೇಲಕ್ಕೆ ನೂಕಿ. ಇದರಿಂದ ಅದು ಬೇಗ ಜಾಗ್ರತಗೊಳ್ಳುವುದು.

\begin{center}
೪
\end{center}

ಮನಸ್ಸನ್ನು ನಿಗ್ರಹಿಸುವುದಕ್ಕೆ ಮುಂಚೆ ಅದರ ವಿಷಯವಾಗಿ ತಿಳಿದುಕೊಳ್ಳಬೇಕು.

ಈ ಚಂಚಲವಾದ ಮನಸ್ಸನ್ನು ಅದರ ಅಲೆದಾಟದಿಂದ ಬಲಾತ್ಕಾರವಾಗಿ ಎಳೆದು ತಂದು ಅದನ್ನು ಯಾವುದಾದರೂ ಒಂದು ಭಾವನೆಯ ಮೇಲೆ ಏಕಾಗ್ರ ಮಾಡಬೇಕು. ಇದನ್ನೇ ಪದೇ ಪದೇ ಮಾಡುತ್ತಿರಬೇಕು. ಇಚ್ಛಾಶಕ್ತಿಯಿಂದ ಮನಸ್ಸನ್ನು ನಿಗ್ರಹಿಸಿ ಅದು ಭಗವಂತನ ಮಹಿಮೆಯನ್ನು ಕುರಿತು ಚಿಂತಿಸುವಂತೆ ಮಾಡಬೇಕು.

ಮನಸ್ಸನ್ನು ಸ್ವಾಧೀನಪಡಿಸಿಕೊಳ್ಳುವುದಕ್ಕೆ ಬಹಳ ಸುಲಭವಾದ ಮಾರ್ಗವೇ ಸುಮ್ಮನೆ ಕುಳಿತು ಮನಸ್ಸು ತನಗೆ ಮನಬಂದ ಕಡೆ ಹಾರಾಡಲು ಬಿಡುವುದು. “ನಾನು ಮನಸ್ಸಲ್ಲ, ಮನಸ್ಸಿನ ಚಂಚಲತೆಯನ್ನು ಈಕ್ಷಿಸುತ್ತಿರುವ ಸಾಕ್ಷಿ ನಾನು'' ಎಂದು ಭಾವಿಸಿ. ಅನಂತರ ಮನಸ್ಸು ನಿಮಗಿಂತ ಬೇರೆ ಎಂದು ಬಗೆದಂತೆ ಆಲೋಚಿಸಿ. ದೇವರೊಂದಿಗೆ ತಾದಾತ್ಮ್ಯಭಾವವನ್ನು ಹೊಂದಿ, ಎಂದಿಗೂ ದೇಹ ಅಥವಾ ಮನಸ್ಸು ನಾನು ಎನ್ನಬೇಡಿ.

ಮನಸ್ಸನ್ನು ನಿಮ್ಮ ಮುಂದಿರುವ ಪ್ರಶಾಂತಸಾಗರದಂತೆ ಭಾವಿಸಿ, ಬಂದುಹೋಗುವ ಆಲೋಚನೆಗಳೇ ಆ ಪ್ರಶಾಂತ ಸಾಗರದಿಂದ ಏಳುವ ಗುಳ್ಳೆಗಳು. ಆಲೋಚನೆಯನ್ನು ನಿಗ್ರಹಿಸಲು ಯಾವ ಪ್ರಯತ್ನವನ್ನೂ ಮಾಡಬೇಡಿ. ಅದು ಬಂದುಹೋಗುವುದನ್ನು ಗಮನಿಸಿ, ಅವುಗಳನ್ನು ಅನುಸರಿಸುತ್ತಿರುವಂತೆ ಭಾವಿಸಿ, ಇದು ಕ್ರಮೇಣ ಆಲೋಚನಾ ಕ್ಷೇತ್ರವನ್ನು ಕಿರಿಯದನ್ನಾಗಿ ಮಾಡುವುದು, ಮನಸ್ಸು ಸಾಮಾನ್ಯವಾಗಿ ಆಲೋಚನೆಯ ವಿಸ್ತಾರವಾದ ವೃತ್ತಗಳನ್ನು ಆವರಿಸುತ್ತದೆ. ಆ ವೃತ್ತಗಳು ಕ್ರಮೇಣ ಒಂದು ಕಲ್ಲನ್ನು ನೀರಿನಲ್ಲಿ ಎಸೆದಾಗ ಅಲೆಗಳು ವಿಸ್ತಾರವಾಗುತ್ತ ಹೋಗುವಂತೆ, ವಿಸ್ತಾರವಾಗುತ್ತಾ ಬರುವುದು. ನಾವು ಈ ಕ್ರಮವನ್ನು ವಿಪರ್ಯಯ ಮಾಡಬೇಕು. ವಿಸ್ತಾರವಾದ ವೃತ್ತದಿಂದ ಪ್ರಾರಂಭ ಮಾಡಿ, ಅದನ್ನು ಬರಬರುತ್ತ ಕಿರಿದುಗೊಳಿಸಿ, ಅದನ್ನು ಯಾವುದಾದರೂ ಒಂದು ವಸ್ತುವಿನ ಮೇಲೆ ಕೇಂದ್ರೀಕರಿಸಿ, ಅಲ್ಲೇ ನಿಲ್ಲುವಂತೆ ಮಾಡಬೇಕು. “ನಾನು ಮನಸ್ಸಲ್ಲ, ಆಲೋಚಿಸುವುದನ್ನು ನಾನು ನೋಡುತ್ತಿರುವೆನು, ನಾನು ನನ್ನ ಮನಸ್ಸು ಹೇಗೆ ಕೆಲಸಮಾಡುತ್ತಿದೆ ಎಂಬುದನ್ನು ನೋಡುತ್ತಿರುವೆ'' ಎಂದು ಭಾವಿಸಿ. ದಿನ ಕಳೆದಂತೆ ನಿಮ್ಮಲ್ಲಿ ಆಲೋಚನೆ ಮತ್ತು ಭಾವನೆಗಳಿಗೆ ಇರುವ ತಾದಾತ್ಮ್ಯಭಾವ ಕಡಿಮೆಯಾಗುತ್ತಾ ಬರುವುದು. ಕೊನೆಗೆ ನೀವು ಮನಸ್ಸಿನಿಂದ ಸಂಪೂರ್ಣ ಬೇರೆಯಾಗಿ, ಅದು ನಿಮ್ಮಿಂದ ಹೊರಗೆ ಇದೆ ಎಂದು ಗೊತ್ತಾಗುವುದು.

ಇದು ಸಿದ್ಧಿಸಿದರೆ ಮನಸ್ಸು ನಿಮ್ಮ ಗುಲಾಮನಾಗುವುದು. ನೀವು ಹೇಳಿದಂತೆ ಕೇಳುವುದು. ಯೋಗಿಯಾಗಬೇಕಾದರೆ ಪ್ರಥಮ ಲಕ್ಷಣವೇ ಇಂದ್ರಿಯಾತೀತನಾಗುವುದು. ಮನಸ್ಸನ್ನು ಜಯಿಸಿದ ಮೇಲೆ ವ್ಯಕ್ತಿಯು ಶ್ರೇಷ್ಠ ಅವಸ್ಥೆಗೆ ಏರಿದಂತೆ.

ಸಾಧ್ಯವಾದಷ್ಟು ಏಕಾಂಗಿಯಾಗಿರಿ. ಆಸನವು ಸಾಧ್ಯವಾದಷ್ಟು ಅನುಕೂಲಕರವಾದ ಎತ್ತರದಲ್ಲಿರಬೇಕು. ಮೊದಲೊಂದು ಚಾಪೆ ಹಾಕಿ, ಅದರ ಮೇಲೊಂದು ಚರ್ಮ, ಅನಂತರ ರೇಷ್ಮೆಯ ಬಟ್ಟೆ ಇರಲಿ, ಆಸನದ ಹಿಂದೆ ಒರಗಿಕೊಳ್ಳುವುದಕ್ಕೆ ಏನೂ ಇಲ್ಲದಿರಲಿ, ಆಸನ ಸ್ಥಿರವಾಗಿರಬೇಕು.

ಆಲೋಚನೆಗಳು ಚಿತ್ರಗಳಾಗಿರುವುದರಿಂದ ಅವನ್ನು ಸೃಷ್ಟಿಸಕೂಡದು. ಮನಸ್ಸಿನಿಂದ ಆಲೋಚನೆಯನ್ನೆಲ್ಲ ಓಡಿಸಿ ಅದನ್ನು ಬರಿದು ಮಾಡಬೇಕು. ಆಲೋಚನೆ ಬಂದೊಡನೆಯೇ ಅದನ್ನು ಓಡಿಸಬೇಕು. ನಾವು ಇದನ್ನು ಜಯಿಸಬೇಕಾದರೆ ವಿಷಯವಸ್ತುಗಳಿಂದ ಪಾರಾಗಬೇಕು. ನಮ್ಮ ದೇಹವನ್ನು ಮೀರಿ ಹೋಗಬೇಕು. ಮನುಷ್ಯನ ಜೀವನವೆಲ್ಲ ಇದನ್ನು ಸಾಧಿಸುವುದಕ್ಕೆ ಒಂದು ಪ್ರಯತ್ನ.

ಪ್ರತಿಯೊಂದು ಶಬ್ದಕ್ಕೂ ಒಂದು ಅರ್ಥವಿದೆ. ನಮ್ಮ ಸ್ವಭಾವದೊಡನೆ ಇವೆರಡೂ ಬೆರೆತಿವೆ.

ನಮ್ಮ ಶ್ರೇಷ್ಠ ಭಾವನೆಯೆ ದೇವರು. ಅವನ ಮೇಲೆ ಧ್ಯಾನಿಸಿ, ನಾವು ದೃಗ್ ಅನ್ನು ನೋಡುವುದಕ್ಕೆ ಆಗುವುದಿಲ್ಲ. ನಾವೇ ಅವನು.

ಪಾಪವನ್ನು ನಾವು ನೋಡಿದರೆ ನಾವೇ ಅದನ್ನು ಸೃಷ್ಟಿಸುತ್ತಿರುವೆವು. ನಾವೇನಾಗಿರುವೆವೊ ಅದನ್ನು ಹೊರಗೆ ನೋಡುತ್ತೇವೆ. ಪ್ರಪಂಚ ಕೇವಲ ಒಂದು ಕನ್ನಡಿಯಂತೆ, ಈ ಸಣ್ಣ ದೇಹ ನಾವು ಸೃಷ್ಟಿಸಿರುವ ಸಣ್ಣ ಕನ್ನಡಿ. ಆದರೆ ಇಡೀ ವಿಶ್ವವೇ ನಮ್ಮ ದೇಹ. ಇದನ್ನೇ ಸದಾ ಕಾಲದಲ್ಲಿಯೂ ಚಿಂತಿಸಬೇಕು. ಆಗ ನಾವು ಸಾಯಲಾರೆವು, ಮತ್ತೊಬ್ಬನಿಗೆ ವ್ಯಥೆಕೊಡಲಾರೆವು ಎಂಬುದು ಗೊತ್ತಾಗುವುದು. ಏಕೆಂದರೆ ಅವನೇ ನಾನಾಗಿರುವನು. ನಮಗೆ ಜನನ ಮರಣಗಳಿಲ್ಲ, ಪ್ರೀತಿಯೊಂದೇ ಇರುವುದು.

"ಈ ಬ್ರಹ್ಮಾಂಡವೇ ನಮ್ಮ ದೇಹ; ಸರ್ವ ಆರೋಗ್ಯ ಆನಂದ ನನ್ನದು. ಏಕೆಂದರೆ ಇವೆಲ್ಲ ಬ್ರಹ್ಮಾಂಡದಲ್ಲಿವೆ. ನಾನೇ ವಿಶ್ವ'' ಎಂದು ಸಾರಿ. ಕೊನೆಗೆ ಗೊತ್ತಾಗುವುದು, ನಮ್ಮ ಕೆಲಸವೆಲ್ಲ ನಮ್ಮನ್ನು ದೃಗ್ ಎಡೆಗೆ ತರುವುದು ಎಂಬುದು.

ನಾವು ಕಿರಿ ಅಲೆಗಳಂತೆ ಕಂಡರೂ ಇಡೀ ಸಾಗರ ನಮ್ಮ ಹಿಂಬದಿಯಲ್ಲಿದೆ. ನಾವು ಅದರಲ್ಲಿ ಐಕ್ಯರಾಗಿರುವೆವು. ಯಾವ ಅಲೆಯೂ ತಾನೇ ಪ್ರತ್ಯೇಕವಾಗಿ ಇರಲಾರದು.

ಕಲ್ಪನಾ ಶಕ್ತಿಯನ್ನು ನಾವು ಸರಿಯಾಗಿ ಉಪಯೋಗಿಸಿಕೊಂಡರೆ ಅದೇ ನಮ್ಮ ಪರಮಾಪ್ತ ಮಿತ್ರ. ಇದು ಯುಕ್ತಿಯನ್ನು ಮೀರಿ ಹೋಗುವುದು. ಈ ಜ್ಞಾನವೊಂದೇ ನಮ್ಮನ್ನು ಎಲ್ಲಾ ಕಡೆಗೂ ಒಯ್ಯುವುದು.

ಸ್ಫೂರ್ತಿ ಉಕ್ಕುವುದು ಅಂತರಾಳದಿಂದ. ನಮ್ಮ ಉದಾತ್ತ ಭಾವನೆಗಳಿಂದ ನಾವೇ ಸ್ಫೂರ್ತಿಯುತರಾಗಬೇಕಾಗಿದೆ.

\begin{center}
೫
\end{center}

ಪ್ರತ್ಯಾಹಾರ ಮತ್ತು ಧಾರಣ: “ಯಾವ ಮಾರ್ಗದ ಮೂಲಕವಾಗಲಿ ಯಾರು ನನ್ನನ್ನು ಅನುಸರಿಸುವರೋ ಅವರೆಲ್ಲರೂ ನನ್ನನ್ನು ಸೇರುವರು." “ಎಲ್ಲರೂ ನನ್ನನ್ನು ಸೇರಲೇಬೇಕು'' ಎಂದು ಕೃಷ್ಣ ಸಾರುವನು. ಪ್ರತ್ಯಾಹಾರ ಎಂದರೆ ಒಟ್ಟುಗೂಡಿಸುವುದು. ಮನಸ್ಸನ್ನು ವಶಪಡಿಸಿ ನಾವು ಇಚ್ಚಿಸಿದ ಯಾವುದಾದರೊಂದು ವಸ್ತುವಿನ ಮೇಲೆ ಕೇಂದ್ರೀಕರಿಸುವುದು ಎಂದು ಅರ್ಥ. ಮೊದಲನೆಯದಾಗಿ ಮನಸ್ಸನ್ನು ಅಲೆಯುವುದಕ್ಕೆ ಬಿಡುವುದು. ಅದನ್ನು ಗಮನಿಸುತ್ತಿರಿ. ಅದು ಏನು ಆಲೋಚಿಸುವುದೋ ಅದನ್ನು ಪರೀಕ್ಷಿಸಿ, ಕೇವಲ ಸಾಕ್ಷಿಯಂತೆ ಇರಿ. ಮನಸ್ಸು ಜೀವವೂ ಅಲ್ಲ, ಆತ್ಮವೂ ಅಲ್ಲ, ಇದು ಸೂಕ್ಷ್ಮವಾದ ವಸ್ತು ಅಷ್ಟೇ. ಇದು ನಮಗೆ ಸೇರಿದ್ದು. ನರಗಳ ಶಕ್ತಿಯ ಮೂಲಕ ನಾವು ಇದನ್ನು ಹೇಗೆ ಬೇಕಾದರೂ ನಡೆಸಿಕೊಳ್ಳುವಂತೆ ಮಾಡಬಹುದು.

ಮನಸ್ ಎಂಬ `ದೃಕ್'ಗೆ ದೇಹವೆಂಬುದು ದೃಶ್ಯ, ಆತ್ಮವಾದ ನಾವು ದೇಹ ಮನಸ್ಸುಗಳಿಗೆ ಅತೀತವಾಗಿರುವೆವು. ನಾವು ನಿತ್ಯವಾದ, ಅವಿಕಾರವಾದ ಸಾಕ್ಷಿ. ದೇಹವೆಂಬುದು ಘನೀಭೂತವಾದ ಆಲೋಚನೆ ಮಾತ್ರ.

ಉಸಿರು ಮೂಗಿನ ಎಡಹೊಳ್ಳೆಯ ಮೂಲಕ ಚಲಿಸುತ್ತಿದ್ದರೆ ಅದು ವಿಶ್ರಾಂತಿಯ ಸಮಯ, ಬಲ ಹೊಳ್ಳೆಯ ಮೂಲಕ ಚಲಿಸುತ್ತಿದ್ದರೆ ಕೆಲಸದ ಸಮಯ. ಎರಡೂ ಹೊಳ್ಳೆಗಳ ಮೂಲಕ ಚಲಿಸುತ್ತಿದರೆ ಧ್ಯಾನದ ಸಮಯ. ನಾವು ಶಾಂತರಾಗಿ ಮೂಗಿನ ಎರಡು ಹೊಳ್ಳೆಗಳ ಮೂಲಕ ಉಸಿರಾಡುತ್ತಿದ್ದರೆ ಅದೇ ಧ್ಯಾನಕ್ಕೆ ತಕ್ಕ ಸಮಯ. ಮೊದಲೇ ಚಿತ್ರವನ್ನು ಏಕಾಗ್ರಗೊಳಿಸಲು ಸಾಧ್ಯವಿಲ್ಲ. ಚಿತ್ತ ಸ್ವಾಧೀನ ಕ್ರಮೇಣ ತಾನಾಗಿ ಸಿದ್ಧಿಸುವುದು.

ಮೊದಲು ಹೆಬ್ಬೆರಳು ಮತ್ತು ತೋರು ಬೆರಳಿನ ಮೂಲಕ ಮೂಗಿನ ಹೊಳ್ಳೆಯನ್ನು ಮುಚ್ಚಿ, ರೂಢಿಯಾದ ಮೇಲೆ ಆಲೋಚನೆಯ ಮೂಲಕ ಇಚ್ಚಾಶಕ್ತಿಯಿಂದಲೇ ನಾವು ಇದನ್ನು ಮಾಡಬಹುದು. ಪ್ರಾಣಾಯಾಮವನ್ನು ಈಗ ಸ್ವಲ್ಪ ಬದಲಾಯಿಸಬೇಕು. ಸಾಧಕನಿಗೆ ಇಷ್ಟದೇವತಾಮಂತ್ರ ಗೊತ್ತಿದ್ದರೆ `ಓಂ' ಎಂಬುದಕ್ಕೆ ಬದಲಾಗಿ ಇದನ್ನು ಉಚ್ಛಾಸ ನಿಶ್ವಾಸಗಳ ಸಮಯದಲ್ಲಿ ಉಪಯೋಗಿಸಬೇಕು. ಕುಂಭಕದ ಸಮಯದಲ್ಲಿ “ಹೂಂ'' ಎಂದು ಉಚ್ಚರಿಸಬೇಕು.

ಕಟ್ಟಿದ ಉಸಿರನ್ನು, “ಹೂಂ” ಎಂದು ಬಲವಾಗಿ ಉಚ್ಚರಿಸುತ್ತ ಕುಂಡಲಿನಿಯ ಹೆಡೆಯ ಮೇಲೆ ಅಪ್ಪಳಿಸಿ, ಅದು ಜಾಗ್ರತವಾಗುತ್ತಿದೆ ಎಂದು ಭಾವಿಸಿ. ದೇವರೊಡನೆ ಮಾತ್ರ ತಾದಾತ್ಮ್ಯಭಾವನೆಯನ್ನು ಪಡೆಯಲು ಯತ್ನಿಸಿ. ಸ್ವಲ್ಪ ಸಮಯದ ಅನಂತರ ಆಲೋಚನೆಗಳು ಬರುವುದಕ್ಕೆ ಮುಂಚೆ ಸುಳುಹು ಕೊಡುವುವು. ಆಗ ಅವು ಹೇಗೆ ಪ್ರಾರಂಭವಾಗುವುವು, ನಾವು ಏನು ಆಲೋಚನೆ ಮಾಡುವೆವು, ಇವೆಲ್ಲ, ಈಗ ನಮಗೆ ದೂರದಲ್ಲಿ ಬರುತ್ತಿರುವವನು ಕಾಣುವಂತೆ ಗೊತ್ತಾಗುವುವು. ಮನಸ್ಸಿನಿಂದ ನಾವು ಬೇರೆಯಾಗಿ, ನಾನು ಬೇರೆ, ಆಲೋಚನೆ ಬೇರೆ ಎಂದು ಭಾವಿಸಿದಾಗ, ಈ ಅವಸ್ಥೆ ನಮಗೆ ಸಿದ್ಧಿಸುವುದು. ಆಲೋಚನೆಗಳು ನಿಮ್ಮನ್ನು ಮೆಟ್ಟಿಕೊಳ್ಳದಂತೆ ನೋಡಿಕೊಳ್ಳಿ. ನೀವು ದೂರ ಸರಿಯಿರಿ. ಅವು ಕ್ರಮೇಣ ನಂದಿಹೋಗುವುವು.

ಇಂತಹ ಪವಿತ್ರ ಭಾವನೆಗಳನ್ನು ಅನುಸರಿಸಿ ಅವುಗಳೊಡನೆ ಹೋಗಿ, ಅವು ಮಾಯವಾದಾಗ ಪರಮೇಶ್ವರನ ಪಾದಕಮಲ ಗೋಚರಿಸುವುದು. ಇದೇ ಅತೀಂದ್ರಿಯಾವಸ್ಥೆ, ಭಾವನೆಗಳನ್ನು ಅನುಸರಿಸಿ ಅವು ಕರಗಿದಾಗ ನೀವು ಅದರಲ್ಲಿ ಕರಗಿಹೋಗಿ.

ಕಾಂತಿಪರಿಧಿ ಅಂತರ್ಜೋತಿಯ ಚಿಹ್ನೆ. ಯೋಗಿಗಳು ಬೇಕಾದರೆ ಇದನ್ನು ನೋಡಬಹುದು. ಕೆಲವುವೇಳೆ ಜ್ಯೋತಿಯ ಪುಂಜದಿಂದ ಆವೃತವಾದ ಮುಖವನ್ನು ನೋಡಬಹುದು. ಅದು ಯಾರ ಮುಖ ಎಂಬುದನ್ನು ನಿಶ್ಚಿತವಾಗಿ ತಿಳಿಯಬಹುದು. ಇಷ್ಟದೇವತೆ ಒಂದು ದೃಶ್ಯದಂತೆ ನಮಗೆ ಗೋಚರಿಸಬಹುದು. ನಾವು ಸುಲಭವಾಗಿ ಆಶ್ರಯ ಪಡೆಯಬಲ್ಲ, ನಮ್ಮ ಮನಸ್ಸನ್ನು ಪೂರ್ಣವಾಗಿ ಕೇಂದ್ರೀಕರಿಸಬಲ್ಲ ಒಂದು ಸಂಕೇತವಾಗುತ್ತದೆ, ಅದು.

ನಾವು ಎಲ್ಲಾ ಇಂದ್ರಿಯಗಳ ಮೂಲಕ ಕಲ್ಪಿಸಿಕೊಳ್ಳಬಹುದು. ಆದರೆ ನಾವು ಕಣ್ಣಿನ ಮೂಲಕ ಹೆಚ್ಚು ಕಲ್ಪಿಸಿಕೊಳ್ಳುವೆವು. ಕಲ್ಪನೆ ಕೂಡ ಅರ್ಧ ಭೌತಿಕವಾದುದು. ಅಂದರೆ ಒಂದು ಪ್ರತೀಕದ ಆಸರೆ ಇಲ್ಲದೆ ನಾವು ಕಲ್ಪಿಸಿಕೊಳ್ಳಲಾರೆವು. ಪ್ರಾಣಿಗಳಲ್ಲಿಯೂ ಆಲೋಚನೆ ಇರುವಂತೆ ತೋರುವುದು. ಆದರೆ ಅವುಗಳಿಗೆ ಮಾತಿಲ್ಲ. ಆದುದರಿಂದ ಭಾವನೆ ಮತ್ತು ಪ್ರತೀಕಗಳಿಗೆ ಒಂದು ಅನ್ಯೋನ್ಯ ಆಶ್ರಯ ಇದ್ದೇ ಇದೆ ಎಂದು ಹೇಳಲು ಸಾಧ್ಯವಿಲ್ಲ.

ಯೋಗದಲ್ಲಿ ಕಲ್ಪನೆಯನ್ನು ಇಟ್ಟುಕೊಳ್ಳಿ. ಆದರೆ ಅದು ಪರಿಶುದ್ಧವಾಗಿರಲಿ. ಜೋಪಾನ, ಕಲ್ಪನಾಶಕ್ತಿಯಲ್ಲಿ ಪ್ರತಿಯೊಬ್ಬರಿಗೂ ಒಂದೊಂದು ವೈಶಿಷ್ಟ್ಯವಿದೆ.\break ಯಾವುದು ನಿಮಗೆ ತುಂಬಾ ಸಹಜವಾಗಿದೆಯೋ ಅದನ್ನು ಅನುಸರಿಸಿ. ಅದೇ ಬಹಳ ಸುಲಭವಾಗಿರುವುದು.

ನಮ್ಮ ಹಿಂದಿನ ಜನ್ಮಗಳೆಲ್ಲದರ ಕರ್ಮದ ಪರಿಣಾಮ ನಾವು - ಒಂದು ದೀಪದಿಂದ ಮತ್ತೊಂದು ದೀಪ ಹತ್ತಿಕೊಂಡಂತೆ. ದೀಪಗಳು ಹಲವು, ಆದರೆ ಬೆಳಕು ಒಂದೇ.

ಮಂದಹಾಸದಿಂದಿರಿ, ಧೀರರಾಗಿ. ಪ್ರತಿದಿನ ಸ್ನಾನಮಾಡಿ. ಸಮಾಧಾನ, ಪವಿತ್ರತೆ ಮತ್ತು ಎಂದಿಗೂ ಪ್ರಯತ್ನ ಬಿಡದ ಛಲ ಇದ್ದರೆ ನೀವು ನಿಜವಾಗಿಯೂ ಯೋಗಿಗಳಾಗುವಿರಿ. ಎಂದಿಗೂ ಅವಸರಪಡಬೇಡಿ. ಯಾವುದಾದರೂ ಅದ್ಭುತ ಶಕ್ತಿಗಳು ಬಂದರೆ ಅವೆಲ್ಲ ಗೌಣ ಎಂದು ಭಾವಿಸಿ. ಅವೆಲ್ಲ ನಿಮ್ಮನ್ನು ಮುಖ್ಯ ಗುರಿಯಿಂದ ಅಡ್ಡ ಹಾದಿಗೆ ಎಳೆಯದಿರಲಿ. ಅವನ್ನು ಬದಿಗೊಡ್ಡಿ. ನಿಮ್ಮ ಏಕಮಾತ್ರ ನಿಜವಾದ ಗುರಿಯಾದ ದೇವರಲ್ಲಿ ಮಾತ್ರ ಅಚಲರಾಗಿ ನಿಲ್ಲಿ. ಯಾವುದನ್ನು ಪಡೆದರೆ ನಾವು ಎಂದೆಂದಿಗೂ ಶಾಂತಿಯನ್ನು ಪಡೆಯಬಲ್ಲೆವೊ ಅಂತಹ ಶಾಶ್ವತವಾದುದನ್ನು ಮಾತ್ರ ಅರಸಿ, ಆ ಪೂರ್ಣ ವಸ್ತು ದೊರೆತರೆ ಪಡೆಯುವುದಕ್ಕೆ ಮತ್ತಾವುದೂ ಇರುವುದಿಲ್ಲ. ನಾವು ಎಂದೆಂದಿಗೂ ನಿತ್ಯಮುಕ್ತರು ನಿತ್ಯಪೂರ್ಣರು, ಸಚ್ಚಿದಾನಂದ ಸ್ವರೂಪರು.

\begin{center}
೬
\end{center}

ಸುಷುಮ್ನ: ಸುಷುಮ್ನದ ಮೇಲೆ ಧ್ಯಾನ ಮಾಡುವುದು ಅತಿ ಪ್ರಯೋಜನಕಾರಿ. ನಿಮಗೆ ಅದು ಪ್ರತ್ಯಕ್ಷವಾಗಬಹುದು. ಇದೇ ಉತ್ತಮ ಮಾರ್ಗ. ಅನಂತರ ದೀರ್ಘಕಾಲ ಅದರ ಮೇಲೆ ಧ್ಯಾನಿಸಿ. ಇದು ಅತಿ ಸೂಕ್ಷ್ಮವಾದ ಕಾಂತಿಯುತವಾದ ತಂತು, ಬೆನ್ನುಮೂಳೆಯ ಮೂಲಕ ಇರುವ ಜಾಗ್ರತ ಮಾರ್ಗ. ಇದೇ ಮುಕ್ತಿಮಾರ್ಗ. ಇದರ ಮೂಲಕವಾಗಿಯೆ ಕುಂಡಲಿನಿ ಜಾಗ್ರತವಾಗಿ ಮೇಲೇಳುವಂತೆ ಮಾಡಬೇಕು.

ಯೋಗಿಯ ಭಾಷೆಯಲ್ಲಿ ಸುಷುಮ್ನೆಯ ಎರಡು ತುದಿಗಳಲ್ಲಿ ಎರಡು ಪದ್ಮಗಳಿವೆ. ಕೆಳಗೆ ಕುಂಡಲಿನಿಯು ತ್ರಿಕೋಣಾಕಾರದಲ್ಲಿದೆ. ಮೇಲೆ ಮಿದುಳಿನಲ್ಲಿ ಪೀನಲ್‌ಗ್ರಾಂಡ್ (\enginline{Pineal gland}) ಹತ್ತಿರ ಸಹಸ್ರಾರ ಇದೆ. ಇವೆರಡರ ಮಧ್ಯದಲ್ಲಿ ಉಳಿದ ನಾಲ್ಕು ಕಮಲಗಳಿವೆ.

\begin{verse}
ಬೆನ್ನು ಮೂಳೆಯ ಕೆಳಗೆ\\ನಾಭಿಯ ಹತ್ತಿರ\\ಹೃದಯದ ಸಮೀಪದಲ್ಲಿ\\ಗಂಟಲಿನ ಹತ್ತಿರ\\ಭ್ರೂಮಧ್ಯೆ\\ಸಹಸ್ರಾರದ ಹತ್ತಿರ
\end{verse}

ನಾವು ಕುಂಡಲಿನಿಯನ್ನು ಜಾಗೃತಗೊಳಿಸಬೇಕು. ಅದನ್ನು ಕ್ರಮೇಣ ಚಕ್ರದಿಂದ ಚಕ್ರಕ್ಕೆ ಸಹಸ್ರಾರ ಸೇರುವವರೆಗೆ ತೆಗೆದುಕೊಂಡು ಹೋಗಬೇಕು. ಪ್ರತಿಯೊಂದು ಚಕ್ರವೂ ಮನಸ್ಸಿನ ಬೇರೆ ಬೇರೆ ಅವಸ್ಥೆಗೆ ಸಂಬಂಧಪಟ್ಟುದು.

