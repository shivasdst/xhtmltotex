
\chapter[ಶಿಷ್ಯನ ಲಕ್ಷಣ]{ಶಿಷ್ಯನ ಲಕ್ಷಣ\protect\footnote{\engfoot{C.W, Vol. VIII, P 106}}}

\begin{center}
(೧೯೦೦ರ ಮಾರ್ಚ್ ೨೯ರಂದು ಸ್ಯಾನ್‌ಫ್ರಾನ್ಸಿಸ್ಕೋದಲ್ಲಿ ನೀಡಿದ ಉಪನ್ಯಾಸ)
\end{center}

ನಾನಿಂದು ಮಾತನಾಡುವುದು ಶಿಷ್ಯನ ಲಕ್ಷಣಗಳನ್ನು ಕುರಿತು. ನಾನು ಹೇಳುವುದನ್ನು ನೀವು ಹೇಗೆ ಸ್ವೀಕರಿಸುವಿರೋ ತಿಳಿಯದು. ನಿಮಗೆ ಅದನ್ನು ಒಪ್ಪಿಕೊಳ್ಳುವುದಕ್ಕೆ ಸ್ವಲ್ಪ ಕಷ್ಟವಾಗುವುದು. ನಮ್ಮ ದೇಶಕ್ಕೂ ನಿಮ್ಮ ದೇಶಕ್ಕೂ ಗುರು–ಶಿಷ್ಯರ ಆದರ್ಶಗಳಲ್ಲಿ ಎಷ್ಟೋ ವ್ಯತ್ಯಾಸವಿರುವುದು. “ಲಕ್ಷ ಲಕ್ಷ ಗುರುಗಳು ದೊರಕುವರು. ಆದರೆ ನಿಜವಾದ ಶಿಷ್ಯ ದೊರಕುವುದು ಕಷ್ಟ" ಎಂಬ ನಮ್ಮ ದೇಶದ ನಾಣ್ಣುಡಿ ಜ್ಞಾಪಕಕ್ಕೆ ಬರುವುದು. ಇದು ನಿಜವೆಂದು ತೋರುವುದು. ಆಧ್ಯಾತ್ಮಿಕ ಜೀವನದಲ್ಲಿ ಮುಂದುವರಿಯಬೇಕಾದರೆ ಶಿಷ್ಯನ ಮನೋಧರ್ಮ ಅತಿಮುಖ್ಯ. ಶಿಷ್ಯನ ಮನೋಭಾವ ಸರಿಯಾದದ್ದಾದರೆ ಆತ್ಮಸಾಕ್ಷಾತ್ಕಾರ ಅವನಿಗೆ ಸುಲಭವಾಗಿ ದೊರಕುವುದು.

ಆತ್ಮಸಾಕ್ಷಾತ್ಕಾರವನ್ನು ಪಡೆಯಲು ಶಿಷ್ಯನಿಗೆ ಇರಬೇಕಾದದ್ದೇನು? ಮಹಾಋಷಿಗಳು, ಸತ್ಯವನ್ನು ತಿಳಿಯಬೇಕಾದರೆ ಒಂದು ಕ್ಷಣ ಸಾಕು, ಅದನ್ನು ಅರಿಯಬೇಕಾಗಿದೆ ಅಷ್ಟೆ, ಎನ್ನುವರು. ಕನಸು ಮಾಯವಾಗುವುದು. ಅದಕ್ಕೆ ಕಾಲವೆಷ್ಟು ಬೇಕು? ಒಂದು ಕ್ಷಣದಲ್ಲಿ ಕನಸು ಮಾಯವಾಗುವುದು. ಭ್ರಾಂತಿ ಮಾಯವಾಗುವುದಕ್ಕೆ ಕಾಲವೆಷ್ಟು ಬೇಕು? ಕಣ್ಣು ಮಿಟುಕಿಸುವಷ್ಟು ಕಾಲ ಸಾಕು. ನಾನು ಸತ್ಯವನ್ನು ಅರಿತಾಗ ಇನ್ನೇನೂ ಆಗುವುದಿಲ್ಲ, ಅಸತ್ಯ ಮಾಯವಾಗುವುದು. ನಾನು ಹಗ್ಗವನ್ನು ಹಾವೆಂದು ಭ್ರಮಿಸಿದೆ; ನಾನು ಈಗ ಅದನ್ನು ಹಗ್ಗವೆಂದು ನೋಡುತ್ತೇನೆ. ಇದಕ್ಕೆ ಅರೆಕ್ಷಣ ಸಾಕು. ಎಲ್ಲಾ ಅಷ್ಟರಲ್ಲೇ ಮುಗಿಯುವುದು, ನೀನು ಅದಾಗಿರುವೆ. ನೀನೇ ಸತ್ಯ. ಇದನ್ನು ತಿಳಿದುಕೊಳ್ಳಲು ಎಷ್ಟು ಸಮಯ ಬೇಕು? ನಾವು ದೇವರಾಗಿದ್ದರೆ, ಯಾವಾಗಲೂ ಆ ಸ್ಥಿತಿಯಲ್ಲೇ ಇದ್ದರೆ, ಅದನ್ನು ತಿಳಿಯದೆ ಇರುವುದೆ ಒಂದು ವಿಚಿತ್ರ. ಅದನ್ನು ಅರಿಯುವುದೊಂದೇ ಸ್ವಾಭಾವಿಕ ವಿಷಯ. ನಾವು ಯಾವಾಗಲೂ ಏನಾಗಿದ್ದೆವೋ, ಈಗ ಏನಾಗಿರುವೆವೋ, ಅದನ್ನು ತಿಳಿಯಲು ಸಹಸ್ರಾರು ವರುಷಗಳು ಬೇಕಾಗಿಲ್ಲ.

ಆದರೂ ಇಂತಹ ಸ್ವತಃಸಿದ್ದ ಸತ್ಯವನ್ನು ತಿಳಿದುಕೊಳ್ಳುವುದು ಬಹಳ ಕಷ್ಟವೆಂದು ಕಾಣುತ್ತದೆ. ಅದರ ಕ್ಷಣಿಕ ನೋಟವೊಂದು ದೊರಕಲು ಪ್ರಾರಂಭವಾಗುವುದಕ್ಕೆ\break ಯುಗಯುಗಗಳು ಕಳೆದುಹೋಗುವುವು. ದೇವರೇ ಜೀವನ, ದೇವರೇ ಸತ್ಯ. ನಾವು ಅದರ ವಿಷಯ ಬರೆಯುತ್ತೇವೆ, ನಮ್ಮ ಹೃದಯಾಂತರಾಳದಲ್ಲಿ ಇದೇ ಸತ್ಯವೆಂದು ಭಾವಿಸುತ್ತೇವೆ. ದೇವರಲ್ಲದೆ ಮತ್ತಾವುದೂ ಸತ್ಯದಂತೆ ಕಾಣಿಸುವುದಿಲ್ಲ. ಉಳಿದವೆಲ್ಲ ಇಂದು ಇವೆ, ನಾಳೆ ಮಂಗಮಾಯವಾಗುವುವು. ಆದರೂ ನಮ್ಮಲ್ಲಿ ಮುಕ್ಕಾಲುಪಾಲು ಜನರು ಜೀವನದಲ್ಲಿ ಹಿಂದಿನಂತೆಯೇ ಇರುವರು. ಅಸತ್ಯವನ್ನು ಅಪ್ಪುತ್ತೇವೆ, ಸತ್ಯದಿಂದ ವಿಮುಖರಾಗುತ್ತೇವೆ. ನಮಗೆ ಸತ್ಯ ಬೇಕಾಗಿಲ್ಲ. ನಮ್ಮ ಸ್ವಪ್ನಕ್ಕೆ ಯಾರೂ ಭಂಗ ತರಕೂಡದು. ಗುರುಗಳು ನಮಗೆ ಬೇಕಿಲ್ಲ. ಕಲಿಯಲು ಯಾರು ಇಚ್ಛಿಸುವರು? ಯಾರಾದರೂ ಭ್ರಾಂತಿಯಿಂದ ಪಾರಾಗಿ ಸತ್ಯವನ್ನು ಪಡೆಯಬೇಕೆಂದು ಆಶಿಸಿದರೆ, ಒಬ್ಬ ಗುರುವಿನಿಂದ ಅದನ್ನು ತಿಳಿಯಬೇಕೆಂದು ಇಚ್ಚಿಸಿದರೆ, ಅವನು ನಿಜವಾದ ಶಿಷ್ಯನಾಗಿರಬೇಕು.

ಶಿಷ್ಯನಾಗುವುದು ಸುಲಭವಲ್ಲ. ಅದಕ್ಕೆ ತುಂಬಾ ಸಿದ್ದತೆ ಬೇಕು. ಹಲವು ನಿಯಮಗಳನ್ನು ಪಾಲಿಸಬೇಕು. ನಾಲ್ಕು ಮುಖ್ಯ ಸಾಧನಾ ನಿಯಮಗಳನ್ನು ವೇದಾಂತಿಗಳು ಸಾರುವರು.

ಮೊದಲನೆಯ ನಿಯಮವೇ, ಇಹ – ಅಮುತ್ರ – ಫಲಭೋಗವಿರಾಗ, ಇಹದಲ್ಲಿ ಮತ್ತು ಪರದಲ್ಲಿ ಬರುವ ಸುಖಭೋಗಗಳ ಆಸೆಯನ್ನು ತೊರೆಯುವುದು.

ನಮ್ಮೆದುರಿಗೆ ಕಾಣಿಸುವಂತಹುದು ಸತ್ಯವಲ್ಲ. ನಮ್ಮ ಮನಸ್ಸಿನಲ್ಲಿ ಆಸೆ ಇರುವವರೆಗೆ ನಾವು ನೋಡುವುದು ಸತ್ಯವಲ್ಲ. ದೇವರೊಬ್ಬನೇ ಸತ್ಯ, ಪ್ರಪಂಚ ಮಿಥ್ಯ. ಎಲ್ಲಿಯವರೆಗೂ ನಮ್ಮ ಮನಸ್ಸಿನಲ್ಲಿ ಸ್ವಲ್ಪವಾದರೂ ಆಸೆ ಉಳಿದಿರುವುದೊ ಅಲ್ಲಿಯವರೆಗೆ ನಮಗೆ ಸತ್ಯ ದೊರಕುವುದಿಲ್ಲ. ನನ್ನ ಸುತ್ತಲಿರುವ ಪ್ರಪಂಚ ನಾಶವಾದರೂ ನಾನು ಲೆಕ್ಕಿಸುವುದಿಲ್ಲ. ಅದರಂತೆಯೇ ಸ್ವರ್ಗಕ್ಕೆ ಹೋಗಲೊಲ್ಲೆನು. ಸ್ವರ್ಗವನ್ನು ನಿಕೃಷ್ಟ ದೃಷ್ಟಿಯಿಂದ ನೋಡುವೆನು. ಸ್ವರ್ಗವೆಂದರೇನು? ಈ ಪ್ರಪಂಚದಂತೆಯೇ ಅದು. ಸ್ವರ್ಗವೇ ಇಲ್ಲದೆ ಇದ್ದರೆ, ಇಂತಹ ಮೂರ್ಖ ಜೀವನ ಮುಂದುವರಿಯದೆ ಇದ್ದರೆ, ನಾವು ಎಷ್ಟೋ ಮೇಲಾಗಿರುತ್ತಿದ್ದೆವು. ನಮ್ಮ ಕೆಲಸಕ್ಕೆ ಬಾರದ ಕನಸು ಎಂದೋ ಬೇಗ ಒಡೆಯುತ್ತಿತ್ತು. ಸ್ವರ್ಗಕ್ಕೆ ಹೋಗುವುದರಿಂದ ನಮ್ಮ ದುಃಖಕರವಾದ ಭ್ರಾಂತಿ ಜೀವನವನ್ನು ಮತ್ತೂ ಹೆಚ್ಚಿಸಿದಂತೆ ಆಗುವುದು.

ನಿಮಗೆ ಸ್ವರ್ಗದಲ್ಲಿ ದೊರಕುವುದೇನು? ನೀವು ಅಲ್ಲಿ ದೇವತೆಗಳಾಗುವಿರಿ, ಅಮೃತವನ್ನು ಕುಡಿಯುವಿರಿ, ವಾತರೋಗಿಗಳಾಗುವಿರಿ. ಅಲ್ಲಿ ಪ್ರಪಂಚದಲ್ಲಿರುವುದಕ್ಕಿಂತ ಕಡಮೆ ದುಃಖವಿದೆ. ಆದರೆ ಅಲ್ಲಿರುವ ಸತ್ಯವೂ ಕಡಮೆ. ಬಡವರಿಗಿಂತ ಅತಿ\break ಶ‍್ರೀಮಂತರಿಗೆ ಸತ್ಯ ತಿಳಿಯುವುದು ಬಹಳ ಕಷ್ಟ. “ಶ‍್ರೀಮಂತ ಭಗವಂತನ ರಾಜ್ಯಕ್ಕೆ ಹೋಗುವುದಕ್ಕಿಂತ ಒಂಟೆ ಸೂಜಿಯ ಕಣ್ಣಿನಲ್ಲಿ ತೂರುವುದು ಸುಲಭ.'' ಶ‍್ರೀಮಂತನಿಗೆ ತನ್ನ ಆಸ್ತಿ, ಅಧಿಕಾರ, ಸುಖ, ಭೋಗ ಇವನ್ನಲ್ಲದೆ ಬೇರೆ ಏನನ್ನೂ ಆಲೋಚಿಸುವುದಕ್ಕೆ ಸಮಯವೇ ಇಲ್ಲ. ಶ‍್ರೀಮಂತರು ಧಾರ್ಮಿಕರಾಗುವುದು ಬಹಳ ಅಪರೂಪ. ಅದು ಏತಕ್ಕೆ? ಧರ್ಮಾತ್ಮರಾದರೆ ಪ್ರಪಂಚವನ್ನು ಅನುಭವಿಸುವುದಕ್ಕೆ ಆಗುವುದಿಲ್ಲವೆಂದು ತಿಳಿಯುವರು. ಇದರಂತೆಯೇ ಸ್ವರ್ಗದಲ್ಲಿ ಧರ್ಮಾತ್ಮರಾಗುವುದಕ್ಕೆ ಅವಕಾಶ ಬಹಳ ಕಡಮೆ. ಅಲ್ಲಿ ಸುಖ, ಆನಂದ ವಿಪರೀತ. ಸ್ವರ್ಗವಾಸಿಗಳಿಗೆ ಅದನ್ನು ಬಿಡಲು ಇಚ್ಚೆಯೇ ಆಗುವುದಿಲ್ಲ.

ಸ್ವರ್ಗದಲ್ಲಿ ಅಳುವುದೇ ಇಲ್ಲವೆನ್ನುವರು. ಅಳದ ಮನುಷ್ಯನನ್ನು ನಾನೆಂದಿಗೂ ನೆಚ್ಚುವುದಿಲ್ಲ. ಅವನಿಗೆ ಹೃದಯವಿರಬೇಕಾದ ಕಡೆ ದೊಡ್ಡ ಕಲ್ಲು ಬಂಡೆಯಿದೆ.\break ಸ್ವರ್ಗವಾಸಿಗಳಿಗೆ ಕರುಣೆ ಇಲ್ಲವೆನ್ನುವುದು ಸ್ಪಷ್ಟ, ಸ್ವರ್ಗದಲ್ಲಿ ಎಷ್ಟೋ ಜನರು\break ವಿಹರಿಸುತ್ತಿರುವರು! ನಾವಿಲ್ಲಿ ಬಡಪಾಯಿಗಳು ತಾಪತ್ರಯಗಳಲ್ಲಿ ನರಳುತ್ತಿರುವೆವು. ನಮ್ಮನ್ನೆಲ್ಲ ಬೇಕಾದರೆ ಅವರು ಮೇಲಕ್ಕೆ ಎತ್ತಬಲ್ಲರು. ಆದರೆ ಅವರು ಹಾಗೆ ಮಾಡುವುದಿಲ್ಲ. ಅವರು ಅಳುವುದಿಲ್ಲ. ಅಲ್ಲಿ ದುಃಖಸಂಕಟಗಳಿಲ್ಲ. ಅದಕ್ಕೇ ಮತ್ತೊಬ್ಬರ ದುಃಖಸಂಕಟಗಳನ್ನು ಅವರು ಲೆಕ್ಕಿಸುವುದಿಲ್ಲ. ಅಮೃತವನ್ನು ಕುಡಿಯುವರು, ನೃತ್ಯ ಮಾಡುವರು; ಸುಂದರ ಯುವತಿಯರು ಅವರ ಸುತ್ತಲೂ ಮುತ್ತಿಕೊಂಡಿರುವರು.

ಈ ವಿಷಯಭೋಗವನ್ನು ಮೀರಿಹೋಗಿ ಶಿಷ್ಯನು ಹೀಗೆ ಹೇಳಬೇಕು: “ಈ ಪ್ರಪಂಚವನ್ನಾಗಲಿ ಅಥವಾ ಮತ್ತಾವ ಸ್ವರ್ಗವನ್ನಾಗಲಿ ನಾನು ಲೆಕ್ಕಿಸುವುದಿಲ್ಲ. ಅಲ್ಲಿಗೆ ಹೋಗುವುದಕ್ಕೆ ನನಗೆ ಇಚ್ಛೆ ಇಲ್ಲ. ಯಾವುದೇ ರೂಪದ ಇಂದ್ರಿಯಜೀವನವನ್ನೂ ನಾನು ಬಯಸುವುದಿಲ್ಲ. ಯಾವ ವಿಧದಲ್ಲಿಯೂ ದೇಹದೊಂದಿಗೆ ತಾದಾತ್ಮ್ಯ ಭಾವವನ್ನು ಪಡೆದ ಪಂಚೇಂದ್ರಿಯ ಸುಖ ನನಗೆ ಬೇಡ. ಈ ದೇಹವೇ ನಾನು, ಈ ಮಾಂಸರಾಶಿಯೇ ನಾನು ಎಂದು ಈಗ ಭಾವಿಸುವೆನು. ಆದರೆ ಅದನ್ನು ನಾನು ಒಪ್ಪುವುದಿಲ್ಲ.”

ಮರ್ತ್ಯ ಸ್ವರ್ಗಲೋಕಗಳೆರಡೂ ನಮ್ಮ ಪಂಚೇಂದ್ರಿಯಗಳಿಗೆ ಅಂಟಿಕೊಂಡಿವೆ. ನಿಮಗೆ ಪಂಚೇಂದ್ರಿಯಗಳಿಲ್ಲದೆ ಇದ್ದರೆ ಜಗತ್ತನ್ನೇ ನೀವು ಗಮನಿಸುತ್ತಿರಲಿಲ್ಲ. ಸ್ವರ್ಗವೂ ಕೂಡ ಒಂದು ಜಗತ್ತೇ. ಸ್ವರ್ಗ–ಮರ್ತ್ಯಲೋಕಗಳೆಲ್ಲ ಜಗತ್ತೇ.

ಶಿಷ್ಯನು, ಹಿಂದೆ ಆಗಿಹೋದುದು, ಈಗ ಆಗುತ್ತಿರುವುದು, ಮುಂದೆ ಆಗಲಿರುವುದು ಎಲ್ಲವನ್ನೂ ವಿಮರ್ಶಿಸುವನು. ಐಶ್ವರ್ಯ, ಸುಖ ಇವುಗಳನ್ನೆಲ್ಲ ಪರೀಕ್ಷಿಸಿ, ಸತ್ಯವನ್ನು ಮಾತ್ರ ಬಯಸುವನು. ಇದೇ ಮೊದಲನೆಯ ನಿಯಮ.

ಎರಡನೆಯ ನಿಯಮವೇ, ಶಿಷ್ಯನು ಬಾಹ್ಯ ಮತ್ತು ಆಂತರಿಕ ಇಂದ್ರಿಯಗಳನ್ನು ನಿಗ್ರಹಿಸಬೇಕು ಎಂಬುದು. ಅವನಲ್ಲಿ ಶಮದಮಾದಿ ಷಡ್ಗುಣಗಳಿರಬೇಕು. ಇತರ ಅನೇಕ ಆಧ್ಯಾತ್ಮಿಕ ಗುಣಗಳು ಅವನಿಗೆ ಇರಬೇಕು. ದೇಹದ ವಿವಿಧ ಭಾಗಗಳಲ್ಲಿ ಕಣ್ಣಿಗೆ ಕಾಣಿಸುವಂತಿರುವುದೆಲ್ಲ ಬಾಹ್ಯೇಂದ್ರಿಯಗಳು, ಆಂತರಿಕವಾಗಿರುವುದೆಲ್ಲ ಅತಿ ಸೂಕ್ಷ್ಮವಾದುವು. ನಿಮಗೆ ಬಾಹ್ಯ ಕಣ್ಣು, ಕಿವಿ, ಮೂಗು ಇತ್ಯಾದಿ ಇವೆ. ಇವಕ್ಕೆ ಸಂಬಂಧಪಟ್ಟ ಆಂತರಿಕ ಸೂಕ್ಷ್ಮೇಂದ್ರಿಯಗಳಿವೆ. ನಾವು ಪ್ರತಿಕ್ಷಣವೂ ಇವೆರಡೂ ಹೇಳಿದಂತೆ ಕೇಳುತ್ತಿರುವೆವು. ಇಂದ್ರಿಯಕ್ಕೂ ವಿಷಯ ವಸ್ತುಗಳಿಗೂ ನಿಕಟ ಸಂಬಂಧವಿರುವುದು. ಯಾವುದಾದರೂ ವಿಷಯವನ್ನು ಹತ್ತಿರವಿದ್ದರೆ, ಅದನ್ನು ನೋಡುವಂತೆ ಇಂದ್ರಿಯಗಳು ಬಲಾತ್ಕರಿಸುವುವು. ನಮಗೆ ಸ್ವಾತಂತ್ರ್ಯವಿಲ್ಲ. ದೊಡ್ಡದೊಂದು ಮೂಗು ಇದೆ. ಹತ್ತಿರದಲ್ಲೇ ಸುಗಂಧವಿದೆ. ನಾನು ಅದನ್ನು ಮೂಸಿನೋಡಲೇಬೇಕು. ದುರ್ಗಂಧವಿದ್ದರೆ ಅದನ್ನು ಮೂಸಿನೋಡಬಾರದೆಂದು ಅನ್ನಿಸುತ್ತದೆ. ಆದರೆ ಪ್ರಕೃತಿ ಮೂಸಿನೋಡು ಎನ್ನುವುದು. ನಾನು ಮೂಸಿನೋಡುವೆ. ನಮ್ಮ ಪಾಡೇನಾಗಿದೆ ನೋಡಿ! ನಾವು ಗುಲಾಮರಾಗಿರುವೆವು. ನನಗೆ ಕಣ್ಣುಗಳಿವೆ. ಒಳ್ಳೆಯದೋ ಕೆಟ್ಟದೋ ಸುತ್ತಲೂ ಏನಾದರೂ ಆಗುತ್ತಿರಲಿ, ಅದನ್ನು ನಾನು ನೋಡಲೇಬೇಕು. ಇದರಂತೆಯೇ ಕೇಳುವುದು ಕೂಡ. ಯಾರಾದರೂ ನನ್ನನ್ನು ನಿಂದಿಸಿದರೆ ನಾನು ಅದನ್ನು ಕೇಳಬೇಕು.\break ಶ್ರವಣೇಂದ್ರಿಯ ಕೇಳುವಂತೆ ಬಲಾತ್ಕರಿಸುವುದು. ನಾನೆಷ್ಟು ದುಃಖಕ್ಕೆ ಈಡಾಗುವೆ! ಅದು ನಿಂದೆಯೋ ಹೊಗಳಿಕೆಯೋ ಅದನ್ನು ಕೇಳಲೇಬೇಕು. ಕಿವಿ ಕೇಳಿಸದ ಕೆಲವು ಕಿವುಡರನ್ನು ನಾನು ನೋಡಿರುವೆನು. ಆದರೆ ತಮಗೆ ಸಂಬಂಧಪಟ್ಟದ್ದೆಲ್ಲ ಅವರಿಗೆ ಕೇಳಿಸುವುದು!

ಬಾಹ್ಯ ಮತ್ತು ಆಂತರಿಕ ಇಂದ್ರಿಯಗಳೆರಡೂ ಶಿಷ್ಯನ ಹತೋಟಿಯಲ್ಲಿರಬೇಕು. ಬಿಡುವಿಲ್ಲದ ಸಾಧನೆಯಿಂದ ಇಂದ್ರಿಯಗಳಿಗೆ ವಿರೋಧವಾಗಿ ನಿಲ್ಲುವುದನ್ನು ನಾವು ಕಲಿಯಬೇಕು. ಪ್ರಕೃತಿಯ ಅಪ್ಪಣೆಯನ್ನು ವಿರೋಧಿಸುವುದನ್ನು ಕಲಿಯಬೇಕು.\break ಶಿಷ್ಯನು ತನ್ನ ಮನಸ್ಸಿಗೆ “ನೀನು ನನ್ನವನು, ನೀನು ಏನನ್ನೂ ನೋಡಬೇಡ ಅಥವಾ ಕೇಳಬೇಡವೆಂದು ನಿನಗೆ ಅಪ್ಪಣೆಮಾಡುತ್ತೇನೆ'' ಎನ್ನಬೇಕು. ಮನಸ್ಸು ಯಾವುದನ್ನೂ ನೋಡಕೂಡದು, ಯಾವುದನ್ನೂ ಕೇಳಕೂಡದು. ಯಾವ ಧ್ವನಿಯಾಗಲೀ, ಆಕಾರವಾಗಲೀ, ನನ್ನ ಮನಸ್ಸಿನ ಮೇಲೆ ತನ್ನ ಪ್ರಭಾವವನ್ನು ಬೀರಬಾರದು. ಆಗ ಮನಸ್ಸು ಇಂದ್ರಿಯಗಳಿಂದ ಬೇರೆಯಾಗಿ, ಅವುಗಳ ವಶದಿಂದ ಪಾರಾಗುವುದು. ಬಾಹ್ಯವಸ್ತುಗಳು ಮನಸ್ಸನ್ನು ತಮ್ಮ ಅಧೀನಕ್ಕೆ ಸೆಳೆದುಕೊಳ್ಳಲಾರವು. ಮನಸ್ಸು ಬಾಹ್ಯವನ್ನು ಹೇಳಿದಂತೆ ಕೇಳುವುದಿಲ್ಲ. ಎದುರಿಗೆ ಸುಗಂಧವಿದೆ, ಸಾಧಕ ಮನಸ್ಸಿಗೆ ಅದನ್ನು ಮೂಸಿನೋಡಬೇಡ ಎನ್ನುವನು. ಮನಸ್ಸು ಗಂಧವನ್ನು ತಿಳಿಯುವುದಿಲ್ಲ. ನೀವು ಆ ಸ್ಥಿತಿಗೆ ಬಂದಾಗ ಮಾತ್ರ ಶಿಷ್ಯರಾಗುವಿರಿ. ಅದಕ್ಕೆ ಪ್ರತಿಯೊಬ್ಬರೂ ತಮಗೆ ಸತ್ಯ ಗೊತ್ತಿದೆ ಎಂದು ಹೇಳಿದಾಗ ನಾನು, “ನಿಮಗೆ ಸತ್ಯ ಗೊತ್ತಿದ್ದರೆ ಇಂದ್ರಿಯ ನಿಗ್ರಹವಿರಬೇಕು. ಅದಿದ್ದರೆ ಕಾರ್ಯತಃ ಅದನ್ನು ತೋರಿಸಿ” ಎನ್ನುತ್ತೇನೆ.

ಅನಂತರ ಮನಸ್ಸನ್ನು ಶಾಂತಗೊಳಿಸಬೇಕು. ಮನಸ್ಸು ಯಾವಾಗಲೂ ತಾರಾಡುತ್ತಿರುವುದು. ನಾನು ಧ್ಯಾನಕ್ಕೆ ಕುಳಿತಾಗಲೇ ಎಲ್ಲಾ ಕುಲಗೆಟ್ಟ ಆಲೋಚನೆಗಳೂ ಮೇಲೇಳುವುವು. ತುಂಬಾ ಜುಗುಪ್ಸಾಕಾರಕ ಇದು. ನನಗೆ ಬೇಡದ್ದನ್ನೇ ಮನಸ್ಸು ಏತಕ್ಕೆ ಆಲೋಚಿಸಬೇಕು? ನಾನು ಮನಸ್ಸಿನ ಗುಲಾಮನಂತೆ ಇರುವೆನು. ಮನಸ್ಸು ಚಂಚಲವಾಗಿ ನಿಗ್ರಹಕ್ಕೆ ಬರದಿದ್ದರೆ ಯಾವ ಆಧ್ಯಾತ್ಮಿಕತೆಯೂ ನಮಗೆ ದೊರಕುವುದಿಲ್ಲ. ಶಿಷ್ಯನು ಮನೋನಿಗ್ರಹವನ್ನು ಕಲಿಯಬೇಕು. ಹೌದು, ಮನಸ್ಸಿನ ಸ್ವಭಾವ ಆಲೋಚಿಸುವುದು. ಆದರೆ ಶಿಷ್ಯನಿಗೆ ಇಚ್ಛೆಯಿಲ್ಲದೇ ಇದ್ದರೆ ಅದು ಆಲೋಚಿಸಕೂಡದು. ಅದಕ್ಕೆ ಅಪ್ಪಣೆ ಮಾಡಿದಾಗ ಆಲೋಚಿಸುವುದನ್ನು ನಿಲ್ಲಿಸಬೇಕು. ಶಿಷ್ಯನಾಗಬೇಕಾದರೆ ಇಂತಹ ಮನೋವೃತ್ತಿ ಆವಶ್ಯಕ.

ಸಾಧಕನಿಗೆ ಹೆಚ್ಚು ಸಹನೆ ಇರಬೇಕು. ಎಲ್ಲಾ ಸರಿಯಾಗಿ ನಡೆಯುತ್ತಿದ್ದರೆ ಜೀವನ ಸುಖಕರವಾಗಿ ನಡೆಯುತ್ತಿರುವಂತೆ ತೋರುವುದು. ಮನಸ್ಸು ಸರಿಯಾಗಿರುವಂತೆ ಕಾಣುವುದು. ಆದರೆ ಅಕಸ್ಮಾತ್ತಾಗಿ ಏನಾದರೂ ಸಂಭವಿಸಿದರೆ ಮನಸ್ಸಿನ ಸ್ತಿಮಿತ ಕೆಡುವುದು. ಆದರೆ ಇದು ಸರಿಯಲ್ಲ. ಎಲ್ಲಾ ದುಃಖವನ್ನೂ ಪಾಪವನ್ನೂ ಗೊಣಗಾಡದೆ ಅನುಭವಿಸಬೇಕು. ಮುಯ್ಯಿಗೆ ಮುಯ್ಯಿ ತೀರಿಸಬೇಕೆಂಬ ಆಸೆ ಇಲ್ಲದೆ, ದುಃಖ ಮುಂತಾದುವನ್ನು ಪರಿಹರಿಸಬೇಕೆಂಬ ಆಲೋಚನೆಯಿಲ್ಲದೆ, ಕಷ್ಟ ದುಃಖಗಳನ್ನು ತಾಳ ಬೇಕು. ಅದೇ ನಿಜವಾದ ಸಹನೆ. ಅದನ್ನು ನೀವು ಪಡೆಯಬೇಕು.

ಪ್ರಪಂಚದಲ್ಲಿ ಒಳ್ಳೆಯದು ಮತ್ತು ಕೆಟ್ಟದು ಯಾವಾಗಲೂ ಇರುವುವು. ಪ್ರಪಂಚದಲ್ಲಿ ಕೆಟ್ಟದು ಇದೆಯೆಂದು ಅನೇಕರಿಗೆ ಗೊತ್ತಿರುವುದಿಲ್ಲ. ಅದನ್ನು ಮರೆಯುವುದಕ್ಕಾದರೂ ಪ್ರಯತ್ನಿಸುವರು. ಕೆಟ್ಟದು ಅವರನ್ನು ಕಾಡಿದಾಗ ಅದರ ಭಾರಕ್ಕೆ ಕುಗ್ಗಿ ಮನಸ್ಸನ್ನು ಕಹಿ ಮಾಡಿಕೊಳ್ಳುವರು. ಮತ್ತೆ ಕೆಲವರು ಇರುವರು, ಪ್ರಪಂಚದಲ್ಲಿ ಕೆಟ್ಟದು ಎಂಬುದೇ ಇಲ್ಲ, ಎಲ್ಲಾ ಒಳ್ಳೆಯದೇ ಎನ್ನುವರು. ಅದೂ ಒಂದು ದುರ್ಬಲತೆ, ಅದೂ ಕೂಡ ಪಾಪಭೀತಿಯಿಂದ ಜನಿಸಿದುದು. ಒಂದು ವಸ್ತು ಕೊಳೆತು ನಾರುತ್ತಿದ್ದರೆ ಅದರ ಮೇಲೆ ಪನ್ನೀರನ್ನು ಚೆಲ್ಲಿ ಅದನ್ನು ಸುಗಂಧವೆಂದು ಏತಕ್ಕೆ ಕರೆಯುವಿರಿ? ಹೌದು, ಪ್ರಪಂಚದಲ್ಲಿ ಒಳ್ಳೆಯದು, ಕೆಟ್ಟದು ಎರಡೂ ಇವೆ. ದೇವರೇ ಕೆಟ್ಟದ್ದನ್ನು ಇಟ್ಟಿರುವನು. ನೀವು ಅದಕ್ಕೇನೂ ಸುಣ್ಣ ಬಳಿಯಬೇಕಾಗಿಲ್ಲ. ಇಲ್ಲಿ ಏತಕ್ಕೆ ಕೆಟ್ಟದು ಇದೆಯೋ ಅದಕ್ಕೂ ನಿಮಗೂ ಸಂಬಂಧವಿಲ್ಲ. ದೇವರನ್ನು ನಂಬಿ ತೆಪ್ಪಗಿರಿ.

ನನ್ನ ಗುರುದೇವ ಶ‍್ರೀರಾಮಕೃಷ್ಣರು ಕಾಯಿಲೆಬಿದ್ದಾಗ ಯಾರೋ ಬ್ರಾಹ್ಮಣರು, ಶ‍್ರೀರಾಮಕೃಷ್ಣರು ತಮ್ಮ ಪ್ರಚಂಡ ಮಾನಸಿಕ ಶಕ್ತಿಯನ್ನು ಉಪಯೋಗಿಸಿ ಗುಣಮುಖರಾಗಬಹುದು; ಮನಸ್ಸನ್ನು ರೋಗದ ಭಾಗದ ಮೇಲೆ ಕೇಂದ್ರೀಕರಿಸಿದರೆ ಗುಣವಾಗಬಹುದು, ಎಂದರು. ಅದಕ್ಕೆ ಶ‍್ರೀರಾಮಕೃಷ್ಣರು “ಏನು? ಯಾವ ಮನಸ್ಸನ್ನು ಭಗವಂತನಿಗೆ ಅರ್ಪಿಸಿರುವೆನೊ ಅದನ್ನು ದೇಹದ ಮೇಲೆ ಇಡಲೆ!'' ಎಂದರು. ರೋಗವನ್ನು ಅವರು ಗಮನಿಸಲಿಲ್ಲ. ಅವರ ಮನಸ್ಸು ಯಾವಾಗಲೂ ಭಗವಂತನನ್ನು ಚಿಂತಿಸುತ್ತಿತ್ತು. ಅದನ್ನು ಸಂಪೂರ್ಣ ಅವನಿಗೆ ಅರ್ಪಿಸಿದ್ದರು. ಮತ್ತಾವುದಕ್ಕೂ ಅದನ್ನು ಉಪಯೋಗಿಸುತ್ತಿರಲಿಲ್ಲ.

ಆರೋಗ್ಯ, ಐಶ್ವರ್ಯ, ದೀರ್ಘಾಯುಸ್ಸು ಇವನ್ನೇ ಪ್ರಪಂಚದಲ್ಲಿ ಒಳ್ಳೆಯದೆನ್ನುವುದು. ಆದರೆ ಇವೆಲ್ಲ ಬರಿಯ ಭ್ರಾಂತಿ. ಇವುಗಳನ್ನು ಪಡೆಯುವುದಕ್ಕಾಗಿ ಮನಸ್ಸನ್ನು ವ್ಯರ್ಥಗೊಳಿಸುವುದೆಂದರೆ ನಮ್ಮ ಭ್ರಾಂತಿಯನ್ನು ದೃಢಪಡಿಸಿದಂತೆ. ಈ ಜೀವನದಲ್ಲಿ ಬೇಕಾದಷ್ಟು ಭ್ರಾಂತಿ, ಕನಸುಗಳಿವೆ. ಮುಂದೆ ಬರುವ ಸ್ವರ್ಗದಲ್ಲಿಯೂ ನಾವು ಇವನ್ನೇ ಆಶಿಸುತ್ತೇವೆ. ಇನ್ನೂ ಹೆಚ್ಚು ಹೆಚ್ಚು ಭ್ರಾಂತಿಯನ್ನು ಸಂಗ್ರಹಿಸುತ್ತಾ ಹೋಗುವೆವು, ನೀವು ಅವನ್ನು ಎದುರಿಸಿ, ನೀವು ಪಾಪವನ್ನು ಮೀರಿದವರು.

ಪ್ರಪಂಚದಲ್ಲಿ ದುಃಖವಿದೆ. ಪ್ರತಿಯೊಬ್ಬರೂ ಅದನ್ನು ಅನುಭವಿಸಲೇ ಬೇಕು. ಇತರರಿಗೆ ಕೇಡನ್ನು ಬಯಸದೆ ನಿಮಗೆ ಕೆಲಸ ಮಾಡುವುದಕ್ಕೆ ಆಗುವುದಿಲ್ಲ. ನೀವು ಯಾವುದಾದರೂ ಪ್ರಪಂಚದ ಸುಖವನ್ನು ಬಯಸಿದಾಗ ನೀವು ಅನುಭವಿಸಬೇಕಾದ ಕಷ್ಟದಿಂದ ಪಾರಾಗಲಿಚ್ಛಿಸುವಿರಿ. ಅದನ್ನು ಮತ್ತೊಬ್ಬ ಅನುಭವಿಸಬೇಕು. ಪ್ರತಿಯೊಬ್ಬರೂ ಮತ್ತೊಬ್ಬರ ಮೇಲೆ ಅದನ್ನು ಹಾಕಲು ಯತ್ನಿಸುತ್ತಿರುವರು. “ಪ್ರಪಂಚದ ದುಃಖ ನನ್ನ ಪಾಲಿಗೆ ಬರಲಿ. ನಾನು ಅದನ್ನೆಲ್ಲಾ ಅನುಭವಿಸುತ್ತೇನೆ. ಉಳಿದವರು ನಿರಾತಂಕವಾಗಿರಲಿ" ಎನ್ನುವನು ಶಿಷ್ಯ.

ಶಿಲುಬೆಯ ಮೇಲಿನ ವ್ಯಕ್ತಿಯನ್ನು ನೆನೆಸಿಕೊಳ್ಳಿ. ಬೇಕಾದರೆ ದೇವತೆಗಳ ಸಮೂಹವನ್ನೇ ಅವನು ತನ್ನ ಸಹಾಯಕ್ಕೆ ಕರೆಸಿಕೊಳ್ಳಬಹುದಾಗಿತ್ತು. ಆದರೆ ಅವನು ತನ್ನನ್ನು ಶಿಲುಬೆಗೆ ಏರಿಸಿದವರನ್ನು ಎದುರಿಸಲಿಲ್ಲ. ಅವರ ಮೇಲೆ ಕನಿಕರ ತೋರಿದನು. ಪ್ರತಿಯೊಂದು ಅವಮಾನವನ್ನೂ ಸಂಕಟವನ್ನೂ ಅನುಭವಿಸಿದನು. “ಕಷ್ಟಪಡುವವರೆಲ್ಲ ಬನ್ನಿ, ಭಾರ ಹೊರುವವರೆಲ್ಲ ಬನ್ನಿ, ನಾನು ನಿಮಗೆ ವಿಶ್ರಾಂತಿಯನ್ನು ನೀಡುವೆನು.” ಎಂದು ಎಲ್ಲರ ಜವಾಬ್ದಾರಿಯನ್ನು ಅವನು ಹೊತ್ತುಕೊಂಡನು. ಇದು ನಿಜವಾದ ಸಹನೆ. ನಮ್ಮ ಜೀವನಕ್ಕಿಂತ ಅವನು ಎಷ್ಟೋ ಮೇಲಿದ್ದನು. ನಾವು ಗುಲಾಮರು. ನಾವು ಅವನನ್ನು ತಿಳಿದುಕೊಳ್ಳಲಾರದಷ್ಟು ಅವನು ಮೇಲಿದ್ದನು. ಯಾರಾದರೂ ನನಗೆ ಒಂದು ಪೆಟ್ಟನ್ನು ಕೊಟ್ಟರೆ ತಕ್ಷಣವೇ ಅವನಿಗೆ ಪೆಟ್ಟು ಕೊಡಲು ನನ್ನ ಕೈ ಏಳುವುದು. ಆ ಮಹಾತ್ಮನ ಮಹಿಮೆಯನ್ನು, ಅವನ ಶಾಂತಿಯನ್ನು ನಾನು ಹೇಗೆ ತಿಳಿಯಬಲ್ಲೆ? ನಾನು ಅವನ ಮಹಾತ್ಮೆಯನ್ನು ಹೇಗೆ ಅರಿಯುವುದು.?

ನಾನು ಆದರ್ಶವನ್ನು ಕೆಳಗೆ ಎಳೆಯುವುದಿಲ್ಲ. ತಪ್ಪನ್ನು ಎದುರಿಸುತ್ತಿರುವ ದೇಹ ನಾನು ಎಂಬುದೇ ನನ್ನ ಭಾವನೆ. ನನಗೇನಾದರೂ ಸ್ವಲ್ಪ ತಲೆನೋವು ಆದರೆ ಅದರ ಪರಿಹಾರಕ್ಕೆ ಪ್ರಪಂಚದಲ್ಲೆಲ್ಲ ಅಲೆದಾಡುವೆನು. ಎರಡು ಸಾವಿರ ಸೀಸೆ ಔಷಧಿಯನ್ನು ಕುಡಿಯುವೆನು. ಇಂತಹ ನಾನು ಅಂತಹ ಅದ್ಭುತ ವ್ಯಕ್ತಿಗಳನ್ನು ಹೇಗೆ ಅರಿಯಬಲ್ಲೆ? ನನಗೇನೋ ಆ ಆದರ್ಶ ಗೋಚರಿಸುವುದು. ಆದರೆ ಎಷ್ಟನ್ನು ಅನುಷ್ಠಾನಕ್ಕೆ ತರಬಲ್ಲೆ? ಈ ದೇಹ ಬುದ್ದಿ ಸುಖ ದುಃಖ ಇವು ಮತ್ತು ಅನುಕೂಲ ಪ್ರತಿಕೂಲಗಳಿಂದ ಕೂಡಿದ ಯಾವುದೂ ಅಲ್ಲಿಗೆ ಹೋಗಲಾರದು. ಮನಸ್ಸನ್ನು ಜಡವಸ್ತುಗಳಿಂದ ಬೇರ್ಪಡಿಸಿ, ಅದು ಅನುಗಾಲವೂ ಆತ್ಮವನ್ನೇ ಚಿಂತಿಸುವಂತೆ ಮಾಡಿದರೆ ಆ ಆದರ್ಶದ ಕ್ಷಣಿಕ ದರ್ಶನ ನನಗೆ ದೊರಕಬಹುದು. ಪ್ರಾಪಂಚಿಕ ವ್ಯವಹಾರ, ವಿಷಯವಸ್ತು ಇವುಗಳಿಗೆ ಅಲ್ಲಿ ಎಡೆ ಇಲ್ಲ. ಮನಸ್ಸನ್ನು ಅದರಿಂದ ತೆಗೆದು ಆತ್ಮನ ಮೇಲೆ ನಿಲ್ಲಿಸಿ. ನಿಮ್ಮ ಜನನ ಮರಣ, ಸುಖ ದುಃಖ, ಹೆಸರು ಕೀರ್ತಿ, ಇವನ್ನು ತೊರೆದು, “ಈ ದೇಹ ನಾನಲ್ಲ, ಮನಸ್ಸು ನಾನಲ್ಲ, ಶುದ್ಧ ಆತ್ಮ ನಾನು." ಎಂಬುದನ್ನು ತಿಳಿಯಿರಿ..

ನಾನು ಎಂದರೆ ಆತ್ಮ, ನಿಮ್ಮ ಕಣ್ಣುಗಳನ್ನು ಮುಚ್ಚಿ `ನಾನು' ಎನ್ನುವುದು ಯಾವುದು ಎಂಬುದನ್ನು ಯೋಚಿಸಿ ನೋಡಿ. ಅದು ನಿಮ್ಮ ದೇಹವೇ ಅಥವಾ ಮನಸ್ಸಿನ ಸ್ವಭಾವವೇ? ಹಾಗಿದ್ದರೆ, ನಿಜವಾದ ನಿಮ್ಮನ್ನು ನೀವು ಇನ್ನೂ ತಿಳಿದುಕೊಂಡಿಲ್ಲ. ಒಂದು ಸಮಯ ಬರುವುದು. ಆಗ `ನಾನು' ಎಂದೊಡನೆಯೇ, ವಿಶ್ವವ್ಯಾಪಿಯಾದ ಅನಂತವನ್ನು\break ನೋಡುವೆ. ಇದೇ ಸತ್ಯ. ನೀವು ಚೇತನವಸ್ತು, ಜಡವಸ್ತುವಲ್ಲ. ಭ್ರಮೆ ಎಂಬುದೊಂದು ಇದೆ. ಒಂದನ್ನು ಮತ್ತೊಂದು ಎಂದು ಭ್ರಮಿಸುವೆವು. ಜಡವನ್ನು ಚೇತನವೆಂದೂ ದೇಹವನ್ನು ಆತ್ಮನೆಂದೂ ಭಾವಿಸುವೆವು. ಇದು ಮಹಾ ಮೋಹ, ಇದು ತೊಲಗಬೇಕು.

ಮುಂದಿನ ನಿಯಮವೇ, ಶಿಷ್ಯನಿಗೆ ಗುರುವಿನಲ್ಲಿರಬೇಕಾದ ಅಚಲ ಶ್ರದ್ದೆ. ಪಾಶ್ಚಾತ್ಯ ದೇಶಗಳಲ್ಲಿ ಗುರುವು ಶಿಷ್ಯನಿಗೆ ಕೆಲವು ಬೌದ್ದಿಕ ವಿಷಯಗಳನ್ನು ನೀಡುವನು, ಅಷ್ಟೆ. ಗುರುವಿನ ಸಂಬಂಧ ಶಿಷ್ಯನ ಜೀವನದಲ್ಲಿ ಅತಿ ನಿಕಟವಾದುದು. ನನ್ನ ಜೀವನದ ಅತಿ ನಿಕಟ ಪ್ರೀತಿಗೆ ಪಾತ್ರನಾದ ಬಂಧುವೆಂದರೆ ಗುರು, ಅನಂತರ ನನ್ನ ತಾಯಿ, ಅನಂತರ ತಂದೆ. ಗುರುವಿಗೆ ನನ್ನ ಪ್ರಥಮ ಗೌರವ. ನನ್ನ ತಂದೆ ಮಾಡು ಎಂದದ್ದನ್ನು, ಗುರು ಮಾಡಬೇಡ ಎಂದರೆ, ನಾನು ಅದನ್ನು ಮಾಡುವುದಿಲ್ಲ. ಗುರು ನನಗೆ ಮುಕ್ತಿಯನ್ನು ನೀಡುವನು. ತಂದೆ ತಾಯಿಗಳು ನನಗೆ ದೇಹವನ್ನು ಕೊಡುವರು. ಆದರೆ ಗುರು ನನಗೆ ಆತ್ಮನಲ್ಲಿ ಇನ್ನೊಂದು ಜನ್ಮವನ್ನೇ ಕೊಡುವನು.

ನಮ್ಮಲ್ಲಿ ಕೆಲವು ಅಸಾಧಾರಣವಾದ ನಂಬಿಕೆಗಳಿವೆ: ಅವುಗಳಲ್ಲಿ ಒಂದನೆಯದೆ, ಕೆಲವು ಅಪರೂಪದ ಮಹಾವ್ಯಕ್ತಿಗಳು ಇರುವರು ಎಂಬುದು. ಅವರು ಆಗಲೇ ಮುಕ್ತರಾಗಿರುವರು. ಅವರು ಜಗತ್ತಿನ ಹಿತಕ್ಕಾಗಿ ಜನ್ಮವೆತ್ತುವರು. ಅವರು ಆಗಲೇ ಜೀವನ್ಮುಕ್ತರು. ತಮ್ಮ ಮುಕ್ತಿಯನ್ನು ಅವರು ಪರಿಗಣಿಸುವುದಿಲ್ಲ. ಇತರರಿಗೆ ಸಹಾಯಮಾಡಲು ಇಚ್ಛಿಸುವರು. ಅವರಿಗೆ ಏನನ್ನೂ ಕಲಿಸಬೇಕಾಗಿಲ್ಲ. ಬಾಲ್ಯದಿಂದಲೂ ಅವರಿಗೆ ಎಲ್ಲಾ ಗೊತ್ತಿದೆ. ಅವರು ಆರು ತಿಂಗಳ ಹಸುಳೆಗಳಾಗಿರುವಾಗಲೇ ಪರಮಸತ್ಯವನ್ನಾಡುವರು.

ಇಂತಹ ಮಹಾತ್ಮರ ಮೇಲೆ ಮಾನವಕೋಟಿಯ ಆಧ್ಯಾತ್ಮಿಕ ಪುರೋಗಮನ ನಿಂತಿದೆ. ಅವರು ಪ್ರಥಮ ಹಣತೆಯಂತೆ, ಮಿಕ್ಕ ಹಣತೆಗಳ ರಾಶಿಗೆ ಬೆಳಕನ್ನು ದಾನಮಾಡುವರು. ನಿಜ, ಬೆಳಕು ಎಲ್ಲರಲ್ಲಿಯೂ ಇದೆ. ಆದರೆ ಮುಕ್ಕಾಲುಪಾಲು ಜನರಲ್ಲಿ ಅದು ಮುಚ್ಚಿಹೋಗಿದೆ. ಮಹಾಪುರುಷರು, ಬಾಲ್ಯದಿಂದಲೂ ಕೋರೈಸುತ್ತಿರುವ ಬೆಳಕಿನಂತೆ, ಅವರೊಂದಿಗೆ ಸಂಪರ್ಕವಿಟ್ಟುಕೊಳ್ಳುವವರು ತಮ್ಮ ಮನೆಯ ದೀವಿಗೆಯನ್ನು ಹೊತ್ತಿಸಿಕೊಂಡಂತೆ ಆಗುವುದು. ಇದರಿಂದ ಮೊದಲನೆಯ ಹಣತೆ ತನ್ನ ದೀಪವನ್ನು ಕಳೆದುಕೊಳ್ಳುವುದಿಲ್ಲ. ಆದರೂ ಅದು ಉಳಿದ ದೀವಿಗೆಗಳಿಗೆ ಬೆಳಕನ್ನು ದಾನಮಾಡಬಲ್ಲದು. ಕೋಟ್ಯಂತರ ದೀಪಗಳನ್ನು ಅದು ಹಚ್ಚುವುದು. ಆದರೂ ಮೊದಲನೆಯ ದೀಪ ಅಷ್ಟೇ ಕಾಂತಿಯಿಂದ ಬೆಳಗುತ್ತಿರುವುದು. ಮೊದಲನೆಯ ದೀವಿಗೆಯೆ ಗುರು, ಅನಂತರ ಹಚ್ಚಲ್ಪಟ್ಟುದೇ ಶಿಷ್ಯ. ಸರದಿಯ ಮೇಲೆ ಎರಡನೆಯದು ಗುರುವಾಗುವುದು. ಅವತಾರಗಳೆನ್ನುವ ಮಹಾವ್ಯಕ್ತಿಗಳು ಅದ್ಭುತ ಆಧ್ಯಾತ್ಮಿಕ ಗಿರಿಶಿಖರಗಳು. ಅವರು ಜಗತ್ತಿಗೆ ಬಂದು ತಮ್ಮ ಪ್ರಥಮ ಶಿಷ್ಯರಿಗೆ ತಪಸ್ಸನ್ನು ಧಾರೆ ಎರೆದು ಒಂದು ಮಹಾ ಆಧ್ಯಾತ್ಮಿಕ ಶಕ್ತಿ ತರಂಗವನ್ನು ಎಬ್ಬಿಸುವರು. ಶಿಷ್ಯರ ಮೂಲಕ ತಲೆತಲಾಂತರಗಳವರೆಗೆ ಈ ಆಧ್ಯಾತ್ಮಿಕ ಶಕ್ತಿಯು ಹರಿದುಹೋಗುವುದು.

 ಕ್ರೈಸ್ತರ ಚರ್ಚಿನಲ್ಲಿ ಒಬ್ಬ ಬಿಷಪ್ ಮತ್ತೊಬ್ಬರ ಮೇಲೆ ಕೈಯಿಟ್ಟು ತನ್ನ ಹಿಂದಿನ ಹಿರಿಯ ಬಿಷಪ್‌ಗಳಿಂದ ಬಂದ ತಪಃಶಕ್ತಿಯನ್ನು ಧಾರೆ ಎರೆಯುವೆನೆಂದು ಭಾವಿಸುವನು. ಏಸುಕ್ರಿಸ್ತ ತನ್ನ ಪ್ರತ್ಯಕ್ಷ ಶಿಷ್ಯರಿಗೆ ತನ್ನ ಶಕ್ತಿಯನ್ನು ಧಾರೆ ಎರೆದನೆಂದೂ, ಅವರು ತಮ್ಮ ಶಿಷ್ಯರಿಗೆ ಧಾರೆ ಎರೆದರೆಂದೂ, ಅವರಿಂದ ಹೀಗೆ ತಲೆ ತಲಾಂತರಗಳವರೆಗೂ ಬಂದ ಆ ಶಕ್ತಿ ಈಗ ತನ್ನಲ್ಲಿರುವುದೆಂದೂ ಪಾದ್ರಿ ಬಿಷಪ್ ಹೇಳುತ್ತಾನೆ. ಬಿಷಪ್‌ಗಳಲ್ಲಿ ಮಾತ್ರವಲ್ಲ, ಪ್ರತಿಯೊಬ್ಬರಲ್ಲಿಯೂ ಆ ಶಕ್ತಿ ಇರಲು ಸಾಧ್ಯವೆಂದು ನಾವು ನಂಬುತ್ತೇವೆ. ಪ್ರತಿಯೊಬ್ಬರಲ್ಲೂ ಪ್ರಚಂಡ ಆಧ್ಯಾತ್ಮಿಕ ತರಂಗ ಹರಿಯಲು ಏಕೆ ಸಾಧ್ಯವಾಗಬಾರದು?

ಮೊದಲು ನಿಮಗೆ ಒಬ್ಬ ಗುರು–ನಿಜವಾದ ಗುರು–ದೊರಕಬೇಕು. ಅವನು ಕೇವಲ ಮಾನವನಾಗಿರಕೂಡದು. ನಿಮಗೆ ಮಾನವ ದೇಹದಲ್ಲಿ ಗುರು ದೊರಕಬಹುದು. ಆದರೆ. ನಿಜವಾದ ಗುರು ದೇಹದಲ್ಲಿ ಇಲ್ಲ. ಅವನು ಈ ಭೌತಿಕ ದೇಹವಲ್ಲ, ಅವನು ನಿಮಗೆ ಕಾಣುವಂತೆ ಇಲ್ಲ. ಗುರು ನಿಮಗೆ ಮಾನವನಂತೆ ಬರಬಹುದು, ನೀವು ಅವನಿಂದ ಶಕ್ತಿಯನ್ನು ಪಡೆಯಬಹುದು. ಕೆಲವು ವೇಳೆ ಅವನು ಕನಸಿನಲ್ಲಿ ಬಂದು ಪ್ರಪಂಚಕ್ಕೆ ಶಕ್ತಿಯನ್ನು ಧಾರೆ ಎರೆಯಬಹುದು. ಗುರುಶಕ್ತಿ ನಮಗೆ ಹಲವು ವಿಧಗಳಲ್ಲಿ ಒದಗಬಹುದು. ಆದರೆ ನಮ್ಮಂತಹ ಸಾಧಾರಣ ಮನುಷ್ಯರಿಗೆ ಗುರು ಬರಬೇಕು. ಅವನು ಬರುವವರೆಗೂ ನಾವು ಸಿದ್ಧತೆ ನಡೆಸಬೇಕು.

ನಾವು ಉಪನ್ಯಾಸಗಳನ್ನು ಕೇಳುವೆವು, ಓದುವೆವು ನಾವು ದೇವರು, ಆತ್ಮ, ಧರ್ಮ, ಮುಕ್ತಿ ಮುಂತಾದುವನ್ನು ಕುರಿತು ಚರ್ಚಿಸುತ್ತೇವೆ. ಇವೆಲ್ಲ ಅಧ್ಯಾತ್ಮವಲ್ಲ. ಅಧ್ಯಾತ್ಮವು ಗ್ರಂಥದಲ್ಲಾಗಲಿ, ಸಿದ್ದಾಂತದಲ್ಲಾಗಲಿ, ತತ್ವದಲ್ಲಾಗಲಿ ಇಲ್ಲ. ಅದು ತರ್ಕದಲ್ಲೂ ಇಲ್ಲ, ಪಾಂಡಿತ್ಯದಲ್ಲೂ ಇಲ್ಲ. ಅದು ಇರುವುದು ಅಂತರಾತ್ಮನ ಜಾಗೃತಿಯಲ್ಲಿ. ಗಿಳಿಯೂ ಪಾಠವನ್ನೊಪ್ಪಿಸುವುದು. ನೀವು ಪಂಡಿತರಾದರೆ ಏನು? ಕತ್ತೆಯು ಗ್ರಂಥರಾಶಿಯನ್ನೇ ಹೊರಬಹುದು. ನಿಜವಾದ ಜ್ಞಾನ ಬಂದಾಗ ಇಂತಹ ಪುಸ್ತಕ ಪಾಂಡಿತ್ಯವಿರುವುದಿಲ್ಲ. ತನ್ನ ಹೆಸರನ್ನು ಕೂಡ ಬರೆಯಲಾರದವನು ಪೂರ್ಣ ಜ್ಞಾನಿಯಾಗಿರಬಲ್ಲ. ಜಗತ್ತಿನ ಪುಸ್ತಕಭಂಡಾರವನ್ನೆಲ್ಲ ಓದಿರುವವನು ಜ್ಞಾನಿ ಆಗಲಾರ. ಆಧ್ಯಾತ್ಮಿಕ ವಿಕಾಸಕ್ಕೆ ಪಾಂಡಿತ್ಯ ಆವಶ್ಯಕವಲ್ಲ. ಪಾಂಡಿತ್ಯ ಬೇಕೇ ಇಲ್ಲ. ಆಧ್ಯಾತ್ಮಿಕ ಶಕ್ತಿಯನ್ನು ಧಾರೆ ಎರೆಯುವ ಪವಿತ್ರ ಗುರುಸ್ಪರ್ಶ ನಿಮ್ಮ ಹೃದಯವನ್ನು ಜಾಗೃತಗೊಳಿಸುವುದು; ಅನಂತರ ಬೆಳವಣಿಗೆಯು ಪ್ರಾರಂಭವಾಗುವುದು. ಇದು ನಿಜವಾದ ಅಗ್ನಿದೀಕ್ಷೆ. ಇನ್ನು ನಿಲ್ಲುವಂತೆ ಇಲ್ಲ. ಮುಂದೆ ಮುಂದೆ ಸದಾ ಮುಂದುವರಿಯುತ್ತಿರಬೇಕು...

ಕೆಲವು ವರ್ಷಗಳ ಹಿಂದೆ ನನ್ನ ಸ್ನೇಹಿತರಾದ ನಿಮ್ಮ ಕ್ರೈಸ್ತ ಪಾದ್ರಿಯೊಬ್ಬರು, “ನೀವು ಕ್ರಿಸ್ತನನ್ನು ನಂಬುತ್ತೀರಾ?" ಎಂದು ಕೇಳಿದರು. “ಹೌದು, ಬಹುಶಃ ಸ್ವಲ್ಪ ಹೆಚ್ಚು ಗೌರವದಿಂದಲೇ ನಂಬುತ್ತೇನೆ" ಎಂದೆ. “ಹಾಗಾದರೆ, ನಮ್ಮ, ಮತದೀಕ್ಷೆಯನ್ನು ಏತಕ್ಕೆ ತೆಗೆದುಕೊಳ್ಳಬಾರದು" ಎಂದರು. ನನಗೆ ದೀಕ್ಷೆ ಕೊಡುವುದು! ಯಾರು ಕೊಡುವುದು? ದೀಕ್ಷೆ ಎಂದರೇನು? ಕೆಲವು ಮಂತ್ರಗಳನ್ನು ಉಚ್ಚರಿಸಿ ನೀರನ್ನು ದೇಹದ ಮೇಲೆ ಚೆಲ್ಲುವುದೇ ಅಥವಾ ನೀರಿನಲ್ಲಿ ನನ್ನನ್ನು ಮುಳುಗಿಸುವುದೇ?

ದೀಕ್ಷೆ ಎಂದರೆ ಆಧ್ಯಾತ್ಮಿಕ ಜೀವನದ ಪ್ರತ್ಯಕ್ಷ ಪರಿಚಯ. ನಿಮಗೆ ನಿಜವಾದ ದೀಕ್ಷೆ ದೊರೆತರೆ ನೀವು ದೇಹವಲ್ಲ, ಆತ್ಮ ಎಂದು ಗೊತ್ತಾಗುವುದು. ನಿಮಗೆ ಶಕ್ತಿಯಿದ್ದರೆ\break ದೀಕ್ಷೆಯನ್ನು ಕೊಡಿ, ಇಲ್ಲದೆ ಇದ್ದರೆ ನೀವು ಕ್ರೈಸ್ತರಲ್ಲ. ದೀಕ್ಷೆಯನ್ನು ಸ್ವೀಕರಿಸಿದ ಅನಂತರವೂ ನೀವು ಹಿಂದಿನಂತೆಯೇ ಇರುವಿರಿ. ಕ್ರಿಸ್ತನ ಹೆಸರಿನಲ್ಲಿ ನಾನು ದೀಕ್ಷೆಯನ್ನು ಸ್ವೀಕರಿಸಿದೆನು ಎಂದು ಹೇಳಿಕೊಳ್ಳುವುದರಿಂದ ಪ್ರಯೋಜನವೇನು? ಬರಿಯ ಮಾತು. ಮಾತಿನಿಂದ ಮತ್ತು ನಿಮ್ಮ ಮೂರ್ಖತನದಿಂದ ಪ್ರಪಂಚದ ಶಾಂತಿಭಂಗ. “ಅಜ್ಞಾನಾಂಧಕಾರದಲ್ಲಿ ಮುಳುಗಿದ್ದರೂ ತಾವು ಪಂಡಿತರೆಂದು ಭಾವಿಸುವರು. ಕುರುಡರನ್ನು ಅನುಸರಿಸುವ ಮತ್ತೊಬ್ಬ ಕುರುಡನಂತೆ ಸಂಸಾರ ಚಕ್ರದಲ್ಲಿ ಸುತ್ತುತ್ತಿರುವರು.” ಆದ್ದರಿಂದ “ನಾವು ಕ್ರೈಸ್ತರು'' ಎಂದು ನೀವು ಹೇಳಿಕೊಳ್ಳಬೇಡಿ. ಜ್ಞಾನದೀಕ್ಷೆ ಮುಂತಾದುವುಗಳನ್ನು ಕುರಿತು ಬಡಾಯಿಕೊಚ್ಚಿಕೊಳ್ಳಬೇಡಿ.

ನಿಜವಾಗಿ ದೀಕ್ಷೆ ಎಂಬುದು ಇದೆ. ಕ್ರಿಸ್ತನು, ಪ್ರಪಂಚಕ್ಕೆ ಬಂದು ಜನರಿಗೆ ಬೋಧಿಸಿದಾಗ ನಿಜವಾದ ದೀಕ್ಷೆ ಎಂಬುದು ಇತ್ತು. ಪ್ರಪಂಚಕ್ಕೆ ಆಗಿಂದಾಗ್ಗೆ ಬರುವ ಮುಕ್ತಾತ್ಮರಾದ ಮಹಾಪುರುಷರಿಗೆ ಆ ದಿವ್ಯದರ್ಶನವನ್ನು ನೀಡುವ ಶಕ್ತಿ ಇದೆ. ಅದು ನಿಜವಾದ ದೀಕ್ಷೆ, ಪ್ರತಿಯೊಂದು ಧರ್ಮದ ಮಂತ್ರತಂತ್ರಗಳ ಹಿಂದೆಯೂ ಸರ್ವವ್ಯಾಪಿಯಾದ ಸತ್ಯದ ಬೀಜವಿದೆ. ಕಾಲಕ್ರಮೇಣ ಈ ಸತ್ಯ ಮರೆತುಹೋಗುವುದು. ಬಾಹ್ಯ ಆಚಾರಗಳಿಂದ ಮಂತ್ರ–ತಂತ್ರಗಳಿಂದ ಇದು ಕ್ರಮೇಣ ಮುಚ್ಚಿಹೋಗುವುದು. ಬಾಹ್ಯ ಆಚಾರ ಮಾತ್ರ ಉಳಿಯುವುದು, ತಿರುಳಿಲ್ಲದೆ ಕರಟ ಮಾತ್ರ ಉಳಿಯುವುದು. ನಿಮ್ಮಲ್ಲಿ ದೀಕ್ಷೆಯ ಆಚಾರವಿದೆ. ಆದರೆ ನಿಜವಾದ ಚೈತನ್ಯದಿಂದ ತುಂಬಿ ತುಳುಕಾಡುವ ದೀಕ್ಷೆಯನ್ನು ಕೊಡಬಲ್ಲವರು ಅತ್ಯಲ್ಪ ಮಂದಿ. ಬಾಹ್ಯ ಆಚಾರವೇ ಸಾಲದು. ಸಜೀವ ಸತ್ಯಾನುಭವಕ್ಕೆ ನಮಗೆ ದೀಕ್ಷೆ ಬೇಕು. ಇದೇ ಆದರ್ಶ.

ಗುರು ನನಗೆ ಬೋಧಿಸಿ ನನ್ನನ್ನು ಜ್ಞಾನದೆಡೆಗೆ ಕರೆದೊಯ್ಯಬೇಕು. ಅವನು ಯಾವ ಒಂದು ಮಹಾ ಪರಂಪರೆಯ ಸರಪಳಿಯಲ್ಲಿ ಒಂದು ಕೊಂಡಿಯಾಗಿರುವನೋ ಹಾಗೆಯೇ ನನ್ನನ್ನೂ ಒಂದು ಕೊಂಡಿಯನ್ನಾಗಿ ಮಾಡಬೇಕು. ಬೀದಿಯಲ್ಲಿ ಹೋಗುವವನೊಬ್ಬ ಗುರುವಾಗಲಾರ, ಗುರು ಪರಮಸತ್ಯವನ್ನು ಪ್ರತ್ಯಕ್ಷ ಕಂಡಿರಬೇಕು, ಅವನಿಗೆ ತಾನು ಆತ್ಮವೆಂಬ ಅರಿವಾಗಿರಬೇಕು. ಬರಿಯ ಮಾತಾಳಿ ಗುರುವಾಗಲಾರ. ನನ್ನಂತಹ ಮೂರ್ಖ ಮಾತಾಳಿ ಬೇಕಾದಷ್ಟು ಮಾತನಾಡಬಹುದು; ಆದರೆ ಗುರುವಾಗಲಾರ. ನಿಜವಾದ ಗುರು ಶಿಷ್ಯನಿಗೆ, “ಹೋಗು, ಇನ್ನು ಪಾಪ ಮಾಡಬೇಡ'' ಎನ್ನುವನು. ಅವನು ಇನ್ನುಮುಂದೆ ಪಾಪ ಮಾಡಲಾರ. ಅವನಿಗೆ ಪಾಪ ಮಾಡುವುದಕ್ಕೆ ಶಕ್ತಿಯೇ ಇರುವುದಿಲ್ಲ.

ನಾನು ಈ ಜೀವನದಲ್ಲಿ ಅಂತಹ ವ್ಯಕ್ತಿಗಳನ್ನು ಕಂಡಿರುವೆನು. ನಾನು ಬೈಬಲ್ಲು ಮುಂತಾದ ಗ್ರಂಥಗಳನ್ನು ಓದಿರುವೆನು. ಅವು ಅದ್ಭುತವಾದುವು. ಆದರೆ ಪುಸ್ತಕಗಳಲ್ಲಿ ಸಜೀವ ಶಕ್ತಿ ಇರುವುದಿಲ್ಲ. ಕ್ಷಣದಲ್ಲಿ ಮತ್ತೊಬ್ಬರನ್ನು ಮಾರ್ಪಡಿಸಬಲ್ಲ ಶಕ್ತಿ ಜೀವಂತವಾಗಿರುವ ಆತ್ಮಜ್ಞಾನಿಗಳಲ್ಲಿ ಮಾತ್ರ ಇರಬಲ್ಲುದು. ಇಂತಹ ಮಹಾ ಜೀವಜ್ಯೋತಿಗಳು ಆಗಿಂದಾಗ್ಗೆ ನಮಗೆ ಗೋಚರಿಸುವರು. ಅಂತಹವರು ಮಾತ್ರ ಗುರುಗಳಾಗಲು ಸಮರ್ಥರು. ನಾನೂ ನೀವೂ ಬರಿಯ ಮಾತಾಳಿಗಳು, ಗುರುಗಳಲ್ಲ. ಮಾತನಾಡಿ ಕೆಲಸಕ್ಕೆ ಬಾರದ ಗದ್ದಲದಿಂದ ಪ್ರಪಂಚದ ಶಾಂತಿಗೆ ಭಂಗ ತರುತ್ತಿರುವೆವು. ನಾವು ಶ್ರದ್ದೆಯಿಂದ ಪ್ರಾರ್ಥಿಸಿ ಸಾಧನೆಯಲ್ಲಿ ನಿರತರಾಗೋಣ. ಕೊನೆಗೆ ಸತ್ಯವನ್ನು ಸಾಕ್ಷಾತ್ಕಾರಮಾಡಿಕೊಳ್ಳುವಂತಹ ಸುದಿನವೊಂದು ಬರುವುದು. ಆಗ ನಾವು ಮಾತಾಡಬೇಕಾಗಿಲ್ಲ.

“ಗುರು ಹದಿನಾರು ವರುಷದ ಯುವಕ; ಅವನು ಎಂಬತ್ತು ವರುಷದ ವೃದ್ಧನಾದ ಶಿಷ್ಯನಿಗೆ ಬೋಧಿಸಿದನು: ಮೌನವಾಗಿ ವ್ಯಾಖ್ಯಾನ ಮಾಡಿದನು. ಶಿಷ್ಯನ ಸಂಶಯಗಳು ಛಿನ್ನವಾದವು.” ಅವನೇ ನಿಜವಾದ ಗುರು. ಇಂತಹ ಒಬ್ಬ ಮನುಷ್ಯ ನಮ್ಮ ಪಾಲಿಗೆ ದೊರೆತರೆ, ಅವನ ಮೇಲೆ ಎಂತಹ ಪ್ರೀತಿ ಶ್ರದ್ದೆ ಇರಬೇಕು! ಅವನು ಪ್ರತ್ಯಕ್ಷ ದೇವರೇ, ಅವನಿಗಿಂತ ಕಡಮೆಯಲ್ಲ! ಅದಕ್ಕೇ ಕ್ರಿಸ್ತನ ಶಿಷ್ಯರು ಅವನನ್ನು ದೇವರೆಂದು ಆರಾಧಿಸಿದ್ದು. ಶಿಷ್ಯನು ಗುರುವನ್ನು ದೇವರೆಂದೇ ಆರಾಧಿಸಬೇಕು. ಶಿಷ್ಯನು ಪ್ರತ್ಯಕ್ಷ ಭಗವಂತನನ್ನು ಕಾಣುವವರೆಗೆ, ಮನುಷ್ಯನಲ್ಲಿ ಆವಿರ್ಭವಿಸಿರುವ ಜೀವಂತ ದೇವರನ್ನು ಮಾತ್ರ ತಿಳಿಯಬಲ್ಲ. ಇಲ್ಲದೇ ಇದ್ದರೆ ದೇವರನ್ನು ಮತ್ತೆ ಹೇಗೆ ತಿಳಿಯಬಲ್ಲ?

ಕ್ರಿಸ್ತನು ಹುಟ್ಟಿ ಹತ್ತೊಂಬತ್ತುನೂರು ವರುಷಗಳಾದ ಮೇಲೆ ಅಮೆರಿಕಾದಲ್ಲಿ ಇಂದು ಒಬ್ಬನು (ಅವನು ಕ್ರಿಸ್ತ ಜನಿಸಿದ ಯೆಹೂದ್ಯ ಮತದವನಲ್ಲ, ಏಸು ಅಥವಾ ಅವನ ವಂಶಜರನ್ನು ಕೂಡ ನೋಡಿದವನಲ್ಲ), “ಜೀಸಸ್ ದೇವರ ಅವತಾರ, ನೀನು ಅವನನ್ನು ನಂಬದೇ ಇದ್ದರೆ ನರಕಕ್ಕೆ ಹೋಗು" ಎನ್ನುವನು. ಕ್ರಿಸ್ತ ದೇವನೆಂಬುದನ್ನು ಅವನ ಪ್ರತ್ಯಕ್ಷ ಶಿಷ್ಯರು ನಂಬಿರುವುದನ್ನು ನಾವು ಅರ್ಥಮಾಡಿಕೊಳ್ಳಬಹುದು. ಏಕೆಂದರೆ ಅವನು ಅವರ ಗುರುಗಳಾಗಿದ್ದ. ಆದರೆ ಹತ್ತೊಂಬತ್ತುನೂರು ವರುಷಗಳಾದ ಮೇಲೆ ಹುಟ್ಟಿದ ಈ ಅಮೆರಿಕದವನಿಗೆ ಏನು ಅಧಿಕಾರವಿದೆ? ನಾನು ಕ್ರಿಸ್ತನನ್ನು ನಂಬದೇ ಇರುವುದರಿಂದ ನರಕಕ್ಕೆ ಹೋಗುತ್ತೇನೆ ಎಂದು ಈ ಯುವಕ ಹೇಳುವನು. ಏಸುವಿನ ವಿಷಯ ಇವನಿಗೆ ಏನು ಗೊತ್ತಿದೆ? ಇವನು ಹುಚ್ಚರ ಆಸ್ಪತ್ರೆಯನ್ನು ಸೇರಲು ಯೋಗ್ಯನಾಗಿರುವನು. ಇಂತಹ ನಂಬಿಕೆಯಿಂದ ಪ್ರಯೋಜನವಿಲ್ಲ. ಇವನು ತನ್ನ ಗುರುವನ್ನು ಹುಡುಕಬೇಕು.

ಏಸು ಪುನಃ ಹುಟ್ಟಬಹುದು, ನಿಮ್ಮ ಬಳಿಗೆ ಬರಬಹುದು. ಆಗ ನೀವು ಅವನನ್ನು ದೇವರೆಂದು ಆರಾಧಿಸಿದರೆ ಸರಿ. ಗುರು ಬರುವವರೆಗೆ ನಾವೆಲ್ಲ ಕಾಯಬೇಕು. ಗುರುವನ್ನು ದೇವರೆಂದೇ ಆರಾಧಿಸಬೇಕು. ಅವನು ದೇವರು, ಅವನಿಗಿಂತ ಕಡಮೆಯಲ್ಲ. ನೀವು ಅವನನ್ನು ನೋಡುತ್ತಿದ್ದರೆ ಅವನು ಕ್ರಮೇಣ ಮಾಯವಾಗುವನು. ಕೊನೆಗೆ ಉಳಿಯುವುದೇನು? ಗುರುವಿನ ಬದಲು ಅಲ್ಲಿ ದೇವರಿರುವನು. ದೇವರು ನಮ್ಮ ಬಳಿಗೆ ಬರುವುದಕ್ಕೆ ಗುರುವಿನ ಆಕಾರವನ್ನು ಧರಿಸುವನು. ನಾವು ಅವನನ್ನು ಸತತ ನೋಡುತ್ತಿದ್ದರೆ, ಕ್ರಮೇಣ ಆ ದೇಹ ಜಾರಿ ಭಗವಂತ ಗೋಚರಿಸುವನು.

ಬ್ರಹ್ಮಾನಂದಸ್ವರೂಪಿಯಾಗಿರುವ, ಜ್ಞಾನಮೂರ್ತಿಯಾಗಿರುವ, ಪರಮ ಸುಖವನ್ನು ನೀಡುವ, ಪರಿಶುದ್ಧನೂ, ಪರಿಪೂರ್ಣನೂ, ಅದ್ವಿತೀಯನೂ, ಅನಂತನೂ, ದ್ವಂದ್ವಾತೀತನೂ, ಭಾವಾತೀತನೂ, ಅವ್ಯಕ್ತನೂ ಆದ ಗುರುವಿಗೆ ನಮಸ್ಕಾರ, ಗುರುವಿನ ನಿಜಸ್ಥಿತಿ ಇದು. ಶಿಷ್ಯ ಅವನನ್ನು ದೇವರ ಸ್ವರೂಪವೆಂದು ನೋಡಿ, ಗೌರವಿಸಿ, ಆದರಿಸಿ, ಮರುಮಾತಿಲ್ಲದೆ ಅವನ ಆಣತಿಯನ್ನು ಪರಿಪಾಲಿಸುವುದರಲ್ಲಿ ಆಶ್ಚರ್ಯವಿಲ್ಲ. ಗುರು ಶಿಷ್ಯರ ಸಂಬಂಧ ಇಂತಹದು.

ಶಿಷ್ಯನ ಮತ್ತೊಂದು ಲಕ್ಷಣವೇ ಮುಕ್ತನಾಗಬೇಕೆಂಬ ಉತ್ಕಟ ಆಕಾಂಕ್ಷೆ ಅವನಲ್ಲಿರಬೇಕು.

ಉರಿಯುತ್ತಿರುವ ಬೆಂಕಿಗೆ ಧಾವಿಸುವ ಪತಂಗದ ಹುಳುಗಳಂತೆ ನಾವು, ಇಂದ್ರಿಯಗಳಿಂದ ಹಾನಿಯಾಗುವುದೆಂದು ತಿಳಿದಿದ್ದರೂ, ಅವು ನಮ್ಮನ್ನು ದಹಿಸುವುವು ಎಂದು ತಿಳಿದಿದ್ದರೂ, ಅವುಗಳಿಂದ ಆಸೆ ಇನ್ನೂ ಅಧಿಕವಾಗುವುದೆಂದು ತಿಳಿದಿದ್ದರೂ, ಅವುಗಳೆಡೆಗೆ\break ಧಾವಿಸುವೆವು. ಭೋಗದಿಂದ ಆಸೆ ತೃಪ್ತಿಗೊಳ್ಳುವುದಿಲ್ಲ. ಬೆಂಕಿಗೆ ತುಪ್ಪ ಹಾಕಿದಂತೆ ಅದು ದೊಡ್ಡದಾಗುವುದು; ಹಾಗೆಯೇ ಭೋಗದಿಂದ ನಮ್ಮ ಆಸೆ ಇನ್ನೂ ಹೆಚ್ಚುವುದು. ಆಸೆಯನ್ನು ತೃಪ್ತಿಗೊಳಿಸುವುದರಿಂದ ಆಸೆ ಹೆಚ್ಚುವುದು. ಇದು ಗೊತ್ತಿದ್ದರೂ ಜನರು ಅದರೆಡೆಗೆ ಧಾವಿಸುವರು. ಜನ್ಮಜನ್ಮಗಳಿಂದಲೂ ವಿಷಯವಸ್ತುಗಳೆಡೆಗೆ\break ಧಾವಿಸುತ್ತಿರುವೆವು. ಅದರ ಪರಿಣಾಮವಾಗಿ ಬೇಕಾದಷ್ಟು ವ್ಯಥೆಪಟ್ಟರೂ, ಆಸೆಯನ್ನು ತೊರೆಯುವುದಿಲ್ಲ. ಆಶಾಪಾಶದಿಂದ ತಮ್ಮನ್ನು ವಿಮೋಚನೆಮಾಡುವ ಧರ್ಮವನ್ನು ಕೂಡ ತಮ್ಮ ಆಶಾತೃಪ್ತಿಗೆ ಉಪಯೋಗಿಸಿಕೊಳ್ಳುವರು. ದೇಹ–ಇಂದ್ರಿಯಗಳ ಬಂಧನದಿಂದ, ಆಸೆಗಳ ದಾಸ್ಯದಿಂದ ತಮ್ಮನ್ನು ಪಾರುಮಾಡೆಂದು ದೇವರನ್ನು ಅವರು ಬೇಡುವುದು ಬಹಳ ಅಪರೂಪ. ಅದರ ಬದಲು ದೇವರನ್ನು ಆರೋಗ್ಯ, ಐಶ್ವರ್ಯ, ಆಯಸ್ಸು ಇವುಗಳಿಗಾಗಿ ಬೇಡುವರು. “ದೇವರೇ ನನ್ನ ತಲೆನೋವನ್ನು ಗುಣಮಾಡು, ನನಗೆ ಸ್ವಲ್ಪ ಹಣ ನೀಡು, ಮತ್ತೆ ಇನ್ನೇನಾದರೂ ಕೊಡು'' ಎನ್ನುವರು.

ನಮ್ಮ ದೃಷ್ಟಿ ಅಷ್ಟು ಅಲ್ಪವಾಗಿರುವುದು, ಹೀನವಾಗಿರುವುದು. ಈ ದೇಹವನ್ನು ಹೊರತು ಮತ್ತೇನನ್ನೂ ಯಾರೂ ಆಶಿಸರು. ಎಂತಹ ಅಧೋಗತಿ, ಎಂತಹ ದಾರುಣ ದುಃಖಪ್ರಾಪ್ತಿ ಇದರಿಂದ! ಪಂಚೇಂದ್ರಿಯಗಳು, ಹೊಟ್ಟೆ ಇಷ್ಟೆ. ಪ್ರಪಂಚವೆಂದರೆ ಹೊಟ್ಟೆ ಮತ್ತು ಜನನೇಂದ್ರಿಯ ಇವಲ್ಲದೆ ಮತ್ತೇನು? ಕೋಟ್ಯಂತರ ಸ್ತ್ರೀಪುರುಷರನ್ನು ನೋಡಿ, ಅವರು ಬದುಕಿರುವುದೇ ಇವಕ್ಕಾಗಿ, ಅವರ ಬಾಳಿನಿಂದ ಇವನ್ನು ಕಳೆಯಿರಿ, ಆಗ ಅವರ ಜೀವನ ನಿಸ್ಸಾರವಾಗುವುದು, ಶೂನ್ಯವಾಗುವುದು, ಸಹಿಸಲಸದಳವಾಗುವುದು. ನಮ್ಮ ಪಾಡೇ ಇದು. ಇದೇ ನಮ್ಮ ಪರಿಸ್ಥಿತಿ. ಪ್ರತಿಕ್ಷಣವೂ ಹೊಟ್ಟೆಗಾಗಿ ಮತ್ತು ಕಾಮಸುಖದ ತೃಪ್ತಿಗಾಗಿ ಹೊಸ ಮಾರ್ಗಗಳನ್ನು ಹುಡುಕುತ್ತಿರುವೆವು. ಇದು ಯಾವಾಗಲೂ ನಡೆಯುತ್ತಿರುವುದು. ನಮಗಿರುವ ದುಃಖಕ್ಕೆ ಅಂತ್ಯವಿಲ್ಲ. ದೇಹದ ಸುಖ ಕೇವಲ ತಾತ್ಕಾಲಿಕ ತೃಪ್ತಿಯನ್ನು ಮತ್ತು ಅನಂತ ದುಃಖವನ್ನು ತರುವುದು. ಸುಖ ಮೇಲೆ ಮಾತ್ರ. ಮೇಲೆ ತೇಲುವ ಅಮೃತ, ಕೆಳಗೆ ವಿಷವಿರುವ ಪಾನೀಯವನ್ನು ಕುಡಿದಂತೆ. ಆದರೂ ನಾವು ಇವೆಲ್ಲವನ್ನೂ ಇಚ್ಛಿಸುವೆವು.

ಏನು ಮಾಡಬೇಕು? ವಿಷಯವಸ್ತುಗಳ ಮತ್ತು ಇಂದ್ರಿಯಸುಖದ ತ್ಯಾಗವೇ ಈ ದುಃಖದಿಂದ ಪಾರಾಗುವುದಕ್ಕೆ ಇರುವ ಏಕಮಾತ್ರ ಮಾರ್ಗ. ನೀವು ಅಧ್ಯಾತ್ಮ ಸಂಪನ್ನರಾಗ ಬೇಕಾದರೆ ತ್ಯಾಗ ಮಾಡಬೇಕು. ಇದೇ ನಿಜವಾದ ಪರೀಕ್ಷೆ, ಪ್ರಪಂಚವನ್ನು ತ್ಯಜಿಸಿ, ಕೆಲಸಕ್ಕೆ ಬಾರದ ಇಂದ್ರಿಯವನ್ನು ತ್ಯಜಿಸಿ, ನಿಜವಾದ ಆಶೆಯೊಂದೇ ಇರುವುದು – ಅದೇ ಸತ್ಯವನ್ನು ತಿಳಿಯುವುದು, ಅಧ್ಯಾತ್ಮ ಸಂಪನ್ನರಾಗುವುದು. ಈ ಜಡಜೀವನ ಸಾಕು, ಅಹಂಕಾರ ಸಾಕು. ನಾನು ಅಧ್ಯಾತ್ಮ ಜೀವಿಯಾಗಬೇಕು. ಈ ಆಶೆ ಉತ್ಕಟವಾಗಿರಬೇಕು, ಪ್ರಚಂಡವಾಗಿರಬೇಕು. ಒಬ್ಬ ಮನುಷ್ಯನ ಕೈಕಾಲುಗಳು ಚಲಿಸದಂತೆ ಕಟ್ಟಿ ಕೆಂಡವನ್ನು ಅವನ ಮೇಲೆ ಇಟ್ಟರೆ, ಅದನ್ನು ಆಚೆಗೆಸೆಯಲು ಸರ್ವಪ್ರಯತ್ನವನ್ನು ಮಾಡುವನು. ಅಂತಹ ಉತ್ಕಟ ಆಕಾಂಕ್ಷೆ ನನ್ನಲ್ಲಿದ್ದರೆ, ದಹಿಸುತ್ತಿರುವ ಪ್ರಪಂಚವನ್ನು ಕಿತ್ತೊಗೆಯಬೇಕೆಂಬ ಹೋರಾಟ ಅವಿಶ್ರಾಂತವಾಗಿದ್ದರೆ, ಆಗ ಮಾತ್ರ. ಆ ಪರಮಸತ್ಯದ ಕ್ಷಣಿಕ ದರ್ಶನಕ್ಕೆ ಸಮಯ ಸನ್ನಿಹಿತವಾಗುತ್ತದೆ.

ನನ್ನನ್ನು ನೋಡಿ – ಎರಡು ಮೂರು ಡಾಲರುಗಳು ಇರುವ ಪಾಕೆಟ್ಟು – ಪುಸ್ತಕ ಕಳೆದುಹೋದರೆ ಅದನ್ನು ಇಪ್ಪತ್ತು ಸಲ ಹುಡುಕಲು ಹೋಗುವೆನು. ಅದಕ್ಕಾಗಿ ಎಷ್ಟು ಕಳವಳ, ಎಷ್ಟು ತೊಂದರೆ, ಎಷ್ಟು ಹೋರಾಟ! ನಿಮ್ಮಲ್ಲಿ ಯಾರಾದರೂ ನನ್ನನ್ನು ಸ್ವಲ್ಪ ದೂರಿದರೆ ಇಪ್ಪತ್ತು ವರುಷಗಳವರೆಗೆ ನಾನು ಅದನ್ನು ಮರೆಯುವುದಿಲ್ಲ. ನಾನು ಅದನ್ನು ಮರೆಯಲಾರೆ, ಕ್ಷಮಿಸಲಾರೆ. ಕೆಲಸಕ್ಕೆ ಬಾರದ ವಿಷಯವಸ್ತುಗಳಿಗೆ ನಾನು ಅಷ್ಟು ಹೋರಾಡುತ್ತೇನೆ. ದೇವರಿಗಾಗಿ ಹೀಗೆ ಹೋರಾಡುವವರು ಎಷ್ಟು ಮಂದಿ ಇರುವರು? ಮಕ್ಕಳು ಆಟದಲ್ಲಿ ಮಗ್ನವಾಗಿರುವಾಗ ಎಲ್ಲವನ್ನೂ ಮರೆಯುವರು. ಯುವಕರಿಗೆ ಭೋಗೇಚ್ಛೆ ಬಲವಾಗಿದೆ. ಅವರಿಗೆ ಮತ್ತೇನೂ ಬೇಡ, ಮುದುಕರು ಕಳೆದುಹೋದ ತಮ್ಮ ತಪ್ಪುಗಳನ್ನು ಕುರಿತು ಚಿಂತಿಸುತ್ತಿರುವರು. ಈಗ ಸುಖಪಡಲು ಆಗದ ವೃದ್ದರು, ಕಳೆದುಹೋದ ಸುಖದ ಸವಿನೆನಪನ್ನು ಮೆಲುಕುಹಾಕುತ್ತಿರುವರು. ಗತಕಾಲದ ಸುಖವನ್ನು ಮೆಲುಕುಹಾಕುವುದೊಂದೇ ಅವರಿಗೆ ಗೊತ್ತಿರುವುದು. ಜನರು ವಿಷಯವಸ್ತುಗಳಿಗೆ ಕಾತರರಾಗಿರುವಂತೆ ದೇವರಿಗೆ ಕಾತರರಾಗಿರುವುದು ಬಹಳ ಅಪರೂಪ.

ದೇವರೊಬ್ಬನೇ ಸತ್ಯ, ದೇವರೊಬ್ಬನೇ ನಿತ್ಯ, ಆತ್ಮ ಒಂದೇ ಇರುವುದು. ದೇಹ ಮಿಥ್ಯೆಯೆಂದು ಅನೇಕರು ಹೇಳುವರು. ಆದರೂ ಅವರು ದೇವರಿಂದ ಆತ್ಮಜ್ಞಾನವನ್ನು ಬಯಸುವುದು ಅಪರೂಪ. ಯಾವಾಗಲೂ ವಿಷಯವಸ್ತುಗಳನ್ನೇ ಆಶಿಸುವರು. ಅವರ ಪ್ರಾರ್ಥನೆಯಲ್ಲಿ, ಭಕ್ತಿ ಎಂಬುದು ಪ್ರಾಪಂಚಿಕ ಬಯಕೆಗಿಂತ ಬೇರೆ ಆಗಿಲ್ಲ. ಅಧೋಗತಿ! ಇಂತಹ ದುರವಸ್ಥೆಗೆ ಬಂದಿದೆ ಧರ್ಮ. ಎಲ್ಲಾ ನಟನೆ. ಆಯುಸ್ಸು ಕಳೆಯುತ್ತಿದೆ, ಆದರೆ ಆತ್ಮಶ‍್ರೀಯನ್ನು ಸಂಪಾದಿಸಲಿಲ್ಲ. ಮನುಷ್ಯ ಆತ್ಮಕ್ಕಾಗಿ ಹಾತೊರೆಯಬೇಕು. ಏಕೆಂದರೆ ಆತ್ಮ ಒಂದೇ ಸತ್ಯ, ಅದೇ ಆದರ್ಶ, ಅದನ್ನು ನಿಮಗೆ ಈಗ ಪಡೆಯಲು ಅಸಾಧ್ಯವಾದರೆ “ನನಗೆ ಅದು ಸಾಧ್ಯವಿಲ್ಲ. ಅದು ನನ್ನ ಆದರ್ಶವೆಂಬುದು ಗೊತ್ತಿದೆ, ಆದರೆ ನನಗೆ ಅದನ್ನು ಅನುಸರಿಸಲು ಇನ್ನೂ ಸಾಧ್ಯವಾಗಿಲ್ಲ” ಎನ್ನಿ. ಆದರೆ ನೀವು ಹೀಗೆ ಹೇಳುವುದಿಲ್ಲ. ಧರ್ಮವನ್ನು ಅಧೋಗತಿಗೆ ಎಳೆದು, ದೇವರ ಹೆಸರಿನಲ್ಲಿ ಪ್ರಾಪಂಚಿಕ ವಸ್ತುಗಳನ್ನು ಬಯಸುವಿರಿ. ನೀವೆಲ್ಲ ನಾಸ್ತಿಕರು. ಇಂದ್ರಿಯಗಳ ಸುಖವಲ್ಲದೆ ನಿಮಗೆ ಬೇರಾವುದರಲ್ಲಿಯೂ ನಂಬಿಕೆ ಇಲ್ಲ. “ಯಾರೋ ಏನನ್ನೋ ಹೇಳಿದರು – ಅದರಲ್ಲಿ ಏನೋ ಇರಬಹುದು. ಅಂತೂ ಸ್ವಲ್ಪ ತಮಾಷೆಮಾಡಿ ನೋಡೋಣ. ಅದರಿಂದ ಏನಾದರೂ ಉಪಯೋಗವಾಗಬಹುದು. ನನ್ನ ಮುರಿದ ಕಾಲು ಸರಿಯಾಗಬಹುದು!”

ರೋಗಗ್ರಸ್ತರು ಶೋಚನೀಯ ವ್ಯಕ್ತಿಗಳು; ಅವರು ದೇವರ ಪರಮಭಕ್ತರು.\break ದೇವರನ್ನು ಪ್ರಾರ್ಥಿಸಿದರೆ ಅವನು ತಮ್ಮ ರೋಗವನ್ನು ಗುಣಮಾಡುವನೆಂದು ನಂಬುವರು, ಇದೇನೂ ಕೆಟ್ಟದಲ್ಲ. ಆದರೆ ಈ ಪ್ರಾರ್ಥನೆ ಹೃತ್ಪೂರ್ವಕವಾಗಿರಬೇಕು ಮತ್ತು ಇದೇ ಧರ್ಮವೆಂದು ಭಾವಿಸಬಾರದು. ಶ‍್ರೀಕೃಷ್ಣ ಗೀತೆಯಲ್ಲಿ ಆರ್ತ, ಜಿಜ್ಞಾಸು, ಅರ್ಥಾರ್ಥಿ, ಜ್ಞಾನಿ ಎಂಬ ನಾಲ್ಕು ಬಗೆಯ ಜನರು ತನ್ನನ್ನು ಪೂಜಿಸುವರು ಎನ್ನುವನು. ಕಷ್ಟದಲ್ಲಿ ಇರುವವರು ಅದರಿಂದ ಪಾರಾಗುವುದಕ್ಕೆ ದೇವರ ಸಮೀಪಕ್ಕೆ ಬರುವರು.\break ರೋಗದಲ್ಲಿದ್ದರೆ ತಾವು ಗುಣವಾಗಲೆಂದು ಪ್ರಾರ್ಥಿಸುವರು. ದ್ರವ್ಯ ಕಳೆದುಕೊಂಡರೆ ಅದನ್ನು ಪಡೆಯಲು ಪ್ರಾರ್ಥಿಸುವರು. ಮತ್ತೆ ಕೆಲವರಲ್ಲಿ ಆಶೆ ತುಂಬಿತುಳುಕುತ್ತಿದೆ. ದೇವರನ್ನು, ಐಶ್ವರ್ಯ ಕೀರ್ತಿ ಪದವಿ ಮುಂತಾದುವುಗಳಿಗಾಗಿ ಬೇಡುವರು. “ಹೇ ಮೇರಿ ತಾಯಿ, ನನಗೆ ಬೇಕಾದುದು ದೊರೆತರೆ ನಿನಗೆ ಪೂಜೆ ಮಾಡಿಸುತ್ತೇನೆ. ನನ್ನ ಪ್ರಾರ್ಥನೆಯನ್ನು ಈಡೇರಿಸಿದರೆ ದೇವರನ್ನು ಪೂಜಿಸುತ್ತೇನೆ, ನಿನಗೂ ಎಲ್ಲದರಲ್ಲೂ ಒಂದು ಪಾಲು ಕೊಡುತ್ತೇನೆ" ಎನ್ನುವರು. ಮತ್ತೆ ಕೆಲವರು ಇರುವರು; ಅವರು ಪ್ರಾಪಂಚಿಕರಲ್ಲ. ಅವರಿಗೆ ದೇವರ ಮೇಲೆ ಅಷ್ಟು ಶ್ರದ್ದೆ ಇಲ್ಲ. ಆದರೆ ಅವನನ್ನು ತಿಳಿದುಕೊಳ್ಳಬೇಕೆಂದು ಬಯಸುವರು. ತತ್ತ್ವಶಾಸ್ತ್ರವನ್ನು ಓದುವರು, ಶಾಸ್ತ್ರವನ್ನು ಪಠಿಸುವರು, ಉಪನ್ಯಾಸಗಳನ್ನು ಕೇಳುವರು. ಅವರು ಜಿಜ್ಞಾಸುಗಳು. ಕೊನೆಯ ವರ್ಗದವರು ದೇವರನ್ನು ಪೂಜಿಸಿ ಜ್ಞಾನಿಗಳಾಗುವರು. ಈ ನಾಲ್ಕು ಬಗೆಯ ಜನರೂ ಒಳ್ಳೆಯವರೆ, ಕೆಟ್ಟವರಲ್ಲ. ಎಲ್ಲರೂ ದೇವರನ್ನು ಪೂಜಿಸುವವರು.

ಆದರೆ ನಾವು ಶಿಷ್ಯರಾಗಲು ಯತ್ನಿಸುತ್ತಿರುವೆವು. ನಮ್ಮ ಜೀವನದ ಗುರಿ ಪರಮಸತ್ಯವನ್ನು ತಿಳಿಯುವುದು. ಇದೇ ಸರ್ವಶ್ರೇಷ್ಠ ಧ್ಯೇಯ. ಪರಮ ಸಾಕ್ಷಾತ್ಕಾರ ಮುಂತಾದ ದೊಡ್ಡ ದೊಡ್ಡ ಪದಗಳನ್ನು ನಮಗೆ ನಾವೇ ಹೇಳಿಕೊಂಡದ್ದಾಯಿತು. ಈಗ ನಾವು ಹೇಳಿದಂತೆ ಆಗೋಣ. ಅಧ್ಯಾತ್ಮದ ಮೇಲೆ ನಿಂತು, ಅಧ್ಯಾತ್ಮವನ್ನು, ಅಧ್ಯಾತ್ಮದಲ್ಲಿ ಆರಾಧಿಸೋಣ. ತಳಹದಿ ಅಧ್ಯಾತ್ಮ, ಮಧ್ಯ ಅಧ್ಯಾತ್ಮ, ಶಿಖರ ಅಧ್ಯಾತ್ಮವಾಗಲಿ, ಪ್ರಪಂಚವೇ ಇಲ್ಲದಂತಾಗುವುದು. ಅದು ಆಕಾಶದಲ್ಲಿ ಲಯವಾಗಲಿ. ಅದರಿಂದ ಯಾರಿಗೇನು? ಅಧ್ಯಾತ್ಮದ ಮೇಲೆ ನಿಲ್ಲಿ! ಅದೇ ಗುರಿ; ಅದನ್ನು ನಾವಿನ್ನೂ ಸೇರಲಾರೆವೆಂಬುದು ಗೊತ್ತಿದೆ. ಚಿಂತೆಯಿಲ್ಲ, ನಿರಾಶರಾಗಬೇಕಾಗಿಲ್ಲ. ಆದರೆ ಆದರ್ಶವನ್ನು ಕೆಳಗೆ ಎಳೆಯಬೇಡಿ. ಮುಖ್ಯವಾಗಿ ಬೇಕಾಗಿರುವುದೇ, ನಾವು ದೇಹ, ನಾವು ಜಡ, ನಾವು ಸಾಯುವೆವು ಎಂಬ ಭಾವನೆಗಳನ್ನು ಕಡಿಮೆಮಾಡಿಕೊಳ್ಳುವುದು. ಕೋರೈಸುತ್ತಿರುವ\break ಅಮೃತಾತ್ಮ ನಾನು ಎಂದು ಹೆಚ್ಚು ಹೆಚ್ಚು ಭಾವಿಸಬೇಕು. ನೀವು, ಪರಿಶುದ್ದ ಅಮೃತಾತ್ಮ ನಾನು ಎಂದು ಎಷ್ಟು ಹೆಚ್ಚು ಹೆಚ್ಚು ಆಲೋಚಿಸುತ್ತಾ ಇದ್ದರೆ, ಅಷ್ಟು ಹೆಚ್ಚು ಹೆಚ್ಚು ವಿಷಯವಸ್ತು, ದೇಹ, ಇಂದ್ರಿಯಗಳು ಇವುಗಳಿಂದ ಪಾರಾಗಲು ಯತ್ನಿಸುವಿರಿ. ಇದೇ ಮುಕ್ತಿಯನ್ನು ಪಡೆಯಬೇಕೆಂಬ ಉತ್ಕಟಾಕಾಂಕ್ಷೆ.

ಶಿಷ್ಯನು ಅನುಸರಿಸಬೇಕಾದ ನಾಲ್ಕನೆಯ ಮತ್ತು ಕೊನೆಯ ನಿಯಮವೇ ನಿತ್ಯ–ಅನಿತ್ಯ –ವಸ್ತು–ವಿವೇಕ, ಪ್ರಪಂಚದಲ್ಲಿ ಸತ್ಯವಿರುವುದೊಂದೇ, ಅದೇ ದೇವರು.\break ಯಾವಾಗಲೂ ಮನಸ್ಸನ್ನು ಅವನಲ್ಲಿಡಬೇಕು, ಅವನಿಗೆ ನಿವೇದಿಸಬೇಕು. ದೇವರೊಬ್ಬನೇ ಇರುವವನು, ಮತ್ತಾವುದೂ ಇಲ್ಲ. ಉಳಿದುವೆಲ್ಲ ನಶ್ವರ, ಅವು ಬಂದು ಹೋಗುವುವು. ಪ್ರಪಂಚದ ಮೇಲಿನ ಆಸೆಯೆಲ್ಲಾ ಭ್ರಾಂತಿ, ಏಕೆಂದರೆ ಪ್ರಪಂಚ ಅಸತ್ಯ. ಉಳಿದವುಗಳೆಲ್ಲಾ ನಿಜವಾಗಿ ಭ್ರಾಂತಿ ಎಂದು ತೋರಿ ಮನಸ್ಸು ಹೆಚ್ಚು ಹೆಚ್ಚು ದೇವರನ್ನು ತಿಳಿದುಕೊಳ್ಳಬೇಕು.

ಶಿಷ್ಯನಾಗಬೇಕೆಂದು ಬಯಸುವವನು ಅನುಷ್ಠಾನಕ್ಕೆ ತರಬೇಕಾದ ನಾಲ್ಕು ನಿಯಮಗಳಿವು. ಇವುಗಳನ್ನು ಪರಿಪಾಲಿಸದೆ ಇದ್ದರೆ ನಿಜವಾದ ಗುರು ಅವನಿಗೆ ದೊರಕುವುದಿಲ್ಲ. ಅದೃಷ್ಟವಶಾತ್ ಅಂತಹವನು ದೊರಕಿದರೂ ಗುರುವು ಧಾರೆಯೆರೆಯುವ ತಪಶ್ಶಕ್ತಿಯಿಂದ ಜಾಗ್ರತನಾಗಲಾರ. ಈ ನಿಯಮಗಳ ವಿಷಯದಲ್ಲಿ ನಾವು ಚೌಕಾಸಿಮಾಡಲಾಗುವುದಿಲ್ಲ. ಸಾಧನೆಮಾಡಿ ಈ ನಿಯಮಗಳನ್ನು ಪರಿಪಾಲಿಸಿದರೆ ಶಿಷ್ಯನ ಹೃದಯಕಮಲ ಅರಳಿ, ದುಂಬಿ ತಾನಾಗಿ ಬರುವುದು. ಆಗ ಶಿಷ್ಯನಿಗೆ ಗುರು ತನ್ನ ದೇಹದಲ್ಲೇ, ತನ್ನ ಅಂತರಾಳದಲ್ಲಿಯೆ ಇದ್ದನೆಂಬುದು ಅರಿವಾಗುವುದು. ಆಗ ಅವನ ಮನಸ್ಸು ಅರಳುವುದು, ಅವನು ಆತ್ಮಸಾಕ್ಷಾತ್ಕಾರವನ್ನು ಪಡೆಯುವನು, ಭವಸಾಗರವನ್ನು ದಾಟುವನು, ಅದನ್ನು ಮಾರಿಹೋಗುವನು. ಭಯಾನಕ ಮಹಾಸಾಗರವನ್ನು ಅವನು ದಾಟಿ, ಲಾಭ ಕೀರ್ತಿಗಳ ಆಸೆ ಇಲ್ಲದೆ ಕೇವಲ ಕರುಣೆಯಿಂದ ಪ್ರೇರಿತನಾಗಿ ಇತರರಿಗೆ ಭವಸಾಗರವನ್ನು ದಾಟುವುದಕ್ಕೆ ಬೇಕಾದ ಸಹಾಯವನ್ನು ನೀಡುವನು.

