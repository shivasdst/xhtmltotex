
\chapter{ಧರ್ಮ: ಅದರ ಮಾರ್ಗಗಳು ಮತ್ತು ಉದ್ದೇಶ\protect\footnote{\enginline{* C.W.Vol. VI, P.3}}}

ನಾವು ವಿಶ್ವದ ಧರ್ಮಗಳನ್ನೆಲ್ಲ ಪರೀಕ್ಷಿಸಿ ನೋಡಿದರೆ ನಮಗೆ ಅಲ್ಲಿ ಅನ್ವೇಷಣೆಯ ಎರಡು ಮಾರ್ಗಗಳು ಕಾಣುವುವು. ಒಂದು ಮಾರ್ಗ ದೇವರಿಂದ ಮನುಷ್ಯನಿಗೆ ಬರುವುದು. ಸೆಮಿಟಿಕ್ ಜನಾಂಗದಲ್ಲಿ ದೇವರ ಭಾವನೆ ಮೊದಲಿನಿಂದಲೂ ಬರುವುದು. ಅದರಲ್ಲಿ ಜೀವದ ಭಾವನೆಯೇ ಇಲ್ಲದೆ ಇರುವುದು ಸೋಜಿಗವಾಗಿದೆ. ಪುರಾತನ ಹಿಬ್ರೂ ಜನಾಂಗದಲ್ಲಿ, ಇತ್ತೀಚೆಗಿನ ಕಾಲದವರೆಗೆ, ಮಾನವನ ಜೀವದ ವಿಷಯವಾಗಿ ಅವರು ಯಾವ ಭಾವನೆಯನ್ನೂ ವ್ಯಕ್ತಪಡಿಸದೆ ಇರುವುದು ಗಮನಾರ್ಹವಾದ ಅಂಶ. ಮಾನವನು ಕೆಲವು ಮಾನಸಿಕ ಮತ್ತು ಭೌತಿಕ ವಸ್ತುಗಳಿಂದ ಆದವನು ಮಾತ್ರ ಎಂದು ಅವರು ಭಾವಿಸಿದ್ದರು. ಸಾವಿನಲ್ಲಿ ಎಲ್ಲಾ ಕೊನೆಗಾಣುವುದು. ಆದರೆ ಅದೇ ಜನಾಂಗ ಅತಿ ಅದ್ಭುತವಾದ ಭಗವಂತನ ಭಾವನೆಯನ್ನು ಸೃಷ್ಟಿಸಿರುವುದು. ಇದೊಂದು ಅನ್ವೇಷಣಾ ಮಾರ್ಗ. ಮತ್ತೊಂದು ಮಾರ್ಗವೇ ಮಾನವನಿಂದ ದೇವರ ಕಡೆಗೆ ಹೋಗುವುದು. ಎರಡನೆಯದು ಆರ್ಯಜನಾಂಗದ ಒಂದು ವೈಶಿಷ್ಟ್ಯ, ಮೊದಲನೆಯದು ಸೆಮಿಟಿಕ್ ಜನಾಂಗದ ವೈಶಿಷ್ಟ್ಯ. ಆರ್ಯರು ಮೊದಲು ಆತ್ಮನಿಂದ ಪ್ರಾರಂಭಮಾಡಿದರು. ಅವರ ದೇವರ ಭಾವನೆ ಅಷ್ಟು ಸ್ಪಷ್ಟವಾಗಿರಲಿಲ್ಲ, ಮೊಬ್ಬು ಮೊಬ್ಬಾಗಿತ್ತು; ಇನ್ನೂ ತಿಳಿಗೊಂಡಿರಲಿಲ್ಲ. ಆದರೆ ಅವರ ಮಾನವ ಜೀವದ ಭಾವನೆ ಸ್ಪಷ್ಟವಾಗುತ್ತ ಬಂದಂತೆಲ್ಲ ಅವರ ದೇವರ ಭಾವನೆಯೂ ಸ್ಪಷ್ಟವಾಗುತ್ತ ಬಂತು. ಆದಕಾರಣವೆ ವೇದಗಳಲ್ಲಿ ವಿಚಾರಣೆ ಪ್ರಾರಂಭವಾಗುವುದು ಜೀವವನ್ನು ಕುರಿತ ವಿಚಾರಣೆಯಿಂದ. ಭಗವಂತನಿಗೆ ಸಂಬಂಧಪಟ್ಟ ಭಾವನೆಯೆಲ್ಲ ಆರ್ಯರಿಗೆ ಬಂದದ್ದು, ಮಾನವ ಜೀವವನ್ನು ಕುರಿತು ವಿಚಾರಿಸಿದ ಪರಿಣಾಮವಾಗಿ. ಆದಕಾರಣವೇ, ಭಾರತದ ದರ್ಶನಗಳಲ್ಲೆಲ್ಲ ದೈವತ್ವವನ್ನು ಅಂತರಂಗದಲ್ಲಿ ಹುಡುಕುವ ಪ್ರಯತ್ನ ನಡೆಯಿತು. ಇದರ ಮುದ್ರೆ ಅಲ್ಲಿಯ ದರ್ಶನಗಳ ಮೇಲೆ ಬಿದ್ದಿದೆ. ಆರ್ಯರು ಯಾವಾಗಲೂ ದೈವತ್ವವನ್ನು ತಮ್ಮ ಅಂತರಂಗದಲ್ಲಿ ಹುಡುಕುತ್ತಿದ್ದರು. ಕೆಲವು ಕಾಲದ ಮೇಲೆ ಹೀಗೆ ಹುಡುಕುವುದು ಅವರ ಸಹಜವಾದ ಲಕ್ಷಣವಾಯಿತು, ಅವರ ಹುಟ್ಟುಗುಣವಾಯಿತು. ಇದು ಅವರ ಕಲೆಯಲ್ಲಿ ಮತ್ತು ಅನುದಿನದ ಜೀವನದಲ್ಲೆಲ್ಲ ವ್ಯಕ್ತವಾಗುವುದು. ಈಗಲೂ ಕೂಡ ಯೂರೋಪಿಯನ್ ಚಿತ್ರಕಾರನು ಪ್ರಾರ್ಥನಾ ಭಂಗಿಯಲ್ಲಿರುವವನನ್ನು ಚಿತ್ರಿಸಬೇಕಾದರೆ, ಪ್ರಾರ್ಥಿಸುತ್ತಿರುವವನು ಮೇಲೆ ತಿರುಗಿ ನೋಡುವಂತೆ, ದೇವರನ್ನು ಎಲ್ಲೋ ಹೊರಗೆ ಹುಡುಕುವಂತೆ, ಆಕಾಶದಲ್ಲಿ ಹುಡುಕಾಡುತ್ತಿರುವಂತೆ ಚಿತ್ರಿಸುವನು. ಆದರೆ ಭರತಖಂಡದಲ್ಲಿ ಈ ಭಾವವನ್ನು ಚಿತ್ರಿಸಬೇಕಾದರೆ ಪ್ರಾರ್ಥಿಸುತ್ತಿರುವವನು ಕಣ್ಣುಗಳನ್ನು ಮುಚ್ಚಿಕೊಂಡಿರುವಂತೆ, ಅಂತರ್ಮುಖಿಯಾಗಿರುವಂತೆ ಚಿತ್ರಿಸುವರು.

ಮನುಷ್ಯ ಎರಡು ವಿಷಯಗಳನ್ನು ತಿಳಿದುಕೊಳ್ಳಬೇಕಾಗಿದೆ: ಒಂದು ಬಾಹ್ಯ ಸ್ವಭಾವ ಮತ್ತೊಂದು ಅಂತರಂಗ ಸ್ವಭಾವ. ಇವು ಮೊದಲು ಪರಸ್ಪರ ವಿರೋಧಿಗಳಾಗಿ ಕಂಡರೂ, ಬಾಹ್ಯಪ್ರಕೃತಿ ಕೂಡ ಸಾಧಾರಣ ಮಾನವನಿಗೆ ಅಂತರಂಗದಲ್ಲಿರುವ ಭಾವನಾಲೋಕದಿಂದ ಆದುದೆಂದು ತೋರುತ್ತದೆ. ಪ್ರತಿಯೊಂದು ದೇಶದ ತತ್ತ್ವಶಾಸ್ತ್ರದಲ್ಲೂ, ಅದರಲ್ಲೂ ಪಶ್ಚಿಮದಲ್ಲಿ, ಮನಸ್ಸು ಮತ್ತು ದ್ರವ್ಯ ಇವೆರಡೂ ಪರಸ್ಪರ ವಿರೋಧಿಗಳಾಗಿರುವಂತೆ ಊಹಿಸಲಾಗಿದೆ. ಆದರೆ ಕೊನೆಗೆ ಒಂದು ಮತ್ತೊಂದರಲ್ಲಿ ಐಕ್ಯವಾಗುವುದನ್ನು ನೋಡುತ್ತೇವೆ. ಕೊನೆಗೆ ಎರಡೂ ಸೇರಿ ಅನಂತವಾದ ಪೂರ್ಣವಾಗುವುದು. ಆದ್ದರಿಂದ ಈ ವಿಶ್ಲೇಷಣೆಯಿಂದ, ಒಂದು ವಿಷಯ ಮೇಲು ಮತ್ತೊಂದು ಕೀಳು ಎಂದು ಹೇಳುತ್ತಿಲ್ಲ. ಬಾಹ್ಯ ಪ್ರಕೃತಿಯ ಅನ್ವೇಷಣೆಯಿಂದ ಸತ್ಯವನ್ನು ತಿಳಿಯಲು ಯತ್ನಿಸುವವರನ್ನು ತಪ್ಪು ಎನ್ನುವುದಿಲ್ಲ. ಅಂತರ್ಮುಖಿಗಳಾಗಿ ಸತ್ಯವನ್ನು ಅರಸುವವರೇ ಸರಿ ಎಂತಲೂ ಹೇಳುವುದಿಲ್ಲ. ಇವೆರಡೂ ಬೇರೆ ಬೇರೆ ಅನ್ವೇಷಣಾ ಮಾರ್ಗಗಳು, ಅಷ್ಟೆ. ಇವೆರಡೂ ಇರಲೇಬೇಕಾಗಿವೆ. ಇವೆರಡನ್ನೂ ದಿನವೂ ಅಧ್ಯಯನ ಮಾಡಲೇಬೇಕಾಗಿದೆ. ಕೊನೆಗೆ ಇವೆರಡೂ ಸಂಧಿಸುವುದನ್ನು ನೋಡುತ್ತೇವೆ. ದೇಹವು ಮನಸ್ಸಿಗೆ ವಿರೋಧಿಯಲ್ಲ; ಮನಸ್ಸು ದೇಹಕ್ಕೆ ವಿರೋಧಿಯಲ್ಲ. ಅನೇಕರು ದೇಹದಿಂದ ಏನೂ ಪ್ರಯೋಜನವಿಲ್ಲವೆಂದು ಸಾರಬಹುದು. ಹಿಂದಿನ ಕಾಲದಲ್ಲಿ ಎಲ್ಲಾ ದೇಶಗಳಲ್ಲಿಯೂ ದೇಹವನ್ನು ಒಂದು ರೋಗ, ಒಂದು ಪಾಪ ಅಥವಾ ಒಂದು ಅನಿಷ್ಟವೆಂದು ಭಾವಿಸಿದವರು ಅನೇಕರು ಇದ್ದರು. ಅನಂತರ ಹೇಗೆ ದೇಹ ಮನಸ್ಸಾಗಿಯೂ, ಮನಸ್ಸು ದೇಹವಾಗಿಯೂ ಮಾರ್ಪಡುತ್ತವೆ ಎಂಬುದನ್ನು ವೇದದಲ್ಲಿ ಓದುತ್ತೇವೆ.

ವೇದಗಳಲ್ಲೆಲ್ಲಾ ಸಾಮಾನ್ಯವಾಗಿರುವ ಒಂದು ಭಾವನೆಯನ್ನು ನೀವು ಗಮನದಲ್ಲಿ ಇಡಬೇಕು. ಅದೇ “ಹೇಗೆ ನಾವು ಒಂದು ಹಿಡಿ ಜೇಡಿಮಣ್ಣನ್ನು ಅರಿತರೆ ಪ್ರಪಂಚದಲ್ಲಿರುವ ಜೇಡಿಮಣ್ಣನ್ನೆಲ್ಲಾ ಅರಿಯಬಹುದೋ ಹಾಗೆ, ಯಾವುದೊಂದನ್ನು ಅರಿತರೆ ಪ್ರಪಂಚದಲ್ಲಿರುವುದನ್ನೆಲ್ಲ ಅರಿಯಬಹುದು?” ಎಂಬುದು. ಸ್ವಲ್ಪ ಹೆಚ್ಚು ಕಡಮೆ ಹೇಳುವುದಾದರೆ ಮಾನವನ ಜ್ಞಾನದ ಸಾರವೆಲ್ಲ ಇದೇ. ನಾವೆಲ್ಲ ಯಾವ ಗುರಿಯ ಕಡೆ ಹೋಗುತ್ತಿರುವೆವೊ ಅದು ಈ ಏಕತೆಯನ್ನು ಕಂಡುಹಿಡಿಯುವುದೇ ಆಗಿದೆ. ನಾವು ಮಾಡುವ ಕೆಲಸವೆಲ್ಲ - ಅದು ಪ್ರಾಪಂಚಿಕವಾದುದಾಗಲಿ ಅಥವಾ ಅತಿ ಸೂಕ್ಷ್ಮವಾದುದಾಗಲಿ, ಅತಿ ಶ್ರೇಷ್ಠವಾದುದಾಗಲಿ ಎಲ್ಲವೂ ಏಕತೆಯ ಆದರ್ಶವನ್ನು ತಲುಪುವುದಕ್ಕೆ. ಮಾನವ ಒಂಟಿಯಾಗಿರುವನು. ಅವನು ಮದುವೆಯಾಗುವನು. ಇದು ಹೊರಗಿನಿಂದ ನೋಡಿದರೆ ಸ್ವಾರ್ಥವಾಗಿ ಕಂಡರೂ, ಅವನನ್ನು ಹೀಗೆ ಪ್ರಚೋದಿಸುವುದೆ, ಅದರ ಹಿಂದೆ ಇರುವ ಕ್ರಿಯೋತ್ತೇಜಕ ಶಕ್ತಿಯೆ, ಆ ಏಕತೆಯನ್ನು ಅರಸುವುದು ಆಗಿದೆ. ಅವನಿಗೆ ಮಕ್ಕಳಿವೆ, ಸ್ನೇಹಿತರು ಇರುವರು, ಅವನು ತನ್ನ ದೇಶವನ್ನು ಪ್ರೀತಿಸುವನು, ಅವನು ಜಗತ್ತನ್ನೇ ಪ್ರೀತಿಸುವನು. ಕೊನೆಗೆ ಇಡೀ ವಿಶ್ವಪ್ರೇಮದಲ್ಲಿ ಅವನ ಪ್ರೀತಿ ಪರ್ಯವಸಾನವಾಗುವುದು. ಏಕತೆಯನ್ನು ಕಂಡುಹಿಡಿಯಬೇಕೆಂಬ ಪೂರ್ಣತೆಯ ಕಡೆಗೆ ನಮ್ಮನ್ನೆಲ್ಲಾ ಯಾವುದೋ ಎದುರಿಸಲಾಗದ ಶಕ್ತಿ ನೂಕುತ್ತಿದೆ. ನಮ್ಮ ಅಲ್ಪ ಅಹಂಕಾರವನ್ನು ನಾಶಮಾಡಿ, ಅದನ್ನು ವಿಸ್ತಾರವಾಗುತ್ತಾ ಹೋಗುವಂತೆ ಮಾಡುವುದು. ಇದೇ ಗುರಿ. ವಿಶ್ವವೆಲ್ಲ ಧಾವಿಸುತ್ತಿರುವ ಅಂತಿಮ ಧ್ಯೇಯವೇ ಇದು. ಪ್ರತಿಯೊಂದು ಕಣವೂ ಮತ್ತೊಂದು ಕಣವನ್ನು ಸೇರುವುದಕ್ಕೆ ಧಾವಿಸುತ್ತಿರುವುದು. ಕಣಗಳಾದ ಮೇಲೆ ಕಣಗಳೆಲ್ಲ ಸೇರಿ ಪೃಥ್ವಿ ಗ್ರಹ ಚಂದ್ರ ಸೂರ್ಯ ತಾರಕೆಗಳೆಂಬ ಬೃಹತ್ ಗೋಳಗಳಾಗಿವೆ. ಅವುಗಳು ಕೂಡ ಪುನಃ ಒಂದಾಗಲು ಧಾವಿಸುತ್ತಿವೆ. ಕೊನೆಗೆ ಭೌತಿಕ ಮತ್ತು ಮಾನಸಿಕ ವಿಶ್ವಗಳೆರಡೂ ಕಲೆತು ಒಂದಾಗುವುದೆಂಬುದು ನಮಗೆ ಗೊತ್ತಿದೆ.

ಬ್ರಹ್ಮಾಂಡದಲ್ಲಿ ಯಾವುದು ಬೃಹದಾಕಾರದಲ್ಲಿ ಆಗುತ್ತಿದೆಯೋ ಅದೇ ಅಲ್ಪಪ್ರಮಾಣದಲ್ಲಿ ಪಿಂಡಾಂಡದಲ್ಲಿ ಆಗುತ್ತಿದೆ. ವೈವಿಧ್ಯದಲ್ಲಿ, ಪ್ರತ್ಯೇಕತೆಯಲ್ಲಿ, ಈ ವಿಶ್ವವು ಇದ್ದು, ಏಕತೆಯ ಕಡೆಗೆ, ಅದ್ವೈತದ ಕಡೆಗೆ ಸಾಗುತ್ತಿದೆ. ಹಾಗೆಯೇ ನಮ್ಮ ಸಣ್ಣ ಪ್ರಪಂಚದಲ್ಲಿ ಪ್ರತಿಯೊಂದು ಆತ್ಮವೂ ಜಗತ್ತಿನಿಂದ ಪ್ರತ್ಯೇಕವಾಗಿದೆ. ಜೀವಿ ಮೂಢನಾದಷ್ಟೂ ತಾನು ವಿಶ್ವದಿಂದ ಬೇರೆಯಾದವನು ಎಂದು ಭಾವಿಸುತ್ತಾನೆ. ಅವನು ಹೆಚ್ಚು ಮೂಢನಾದಷ್ಟೂ ತಾನು ಹುಟ್ಟುತ್ತೇನೆ, ಸಾಯುತ್ತೇನೆ ಎಂದು ಭಾವಿಸುವನು. ಇವೆಲ್ಲ ಪ್ರತ್ಯೇಕತೆಗೆ ಸಂಬಂಧಪಟ್ಟ ಭಾವನೆಗಳು. ಆದರೆ ಜ್ಞಾನ ಬಂದಂತೆ ಜೀವಿ ವಿಕಾಸವಾಗುವನು, ನೀತಿಯ ಕಲ್ಪನೆಯೂ ವಿಕಾಸವಾಗುವುದು. ನಾವೆಲ್ಲ ಒಂದು ಎಂಬ ಭಾವನೆ ಪ್ರಾರಂಭವಾಗುವುದು. ಜನರಿಗೆ ಇದು ಗೊತ್ತಾಗಲಿ, ಇಲ್ಲದೆ ಇರಲಿ, ನಮ್ಮ ಹಿಂದೆ ಇರುವ ಶಕ್ತಿ ನಮ್ಮನ್ನು ಸ್ವಾರ್ಥಹೀನರನ್ನಾಗಿ ಮಾಡಲು ಪ್ರೇರೇಪಿಸುತ್ತಿರುವುದು. ಇದೇ ಎಲ್ಲಾ ರೀತಿಯ ತಳಹದಿ. ಯಾವ ಭಾಷೆಯಲ್ಲಾಗಲಿ, ಯಾವ ಧರ್ಮದಲ್ಲಾಗಲಿ, ಜಗತ್ತಿನ ಯಾವ ಮಹಾತ್ಮನಾಗಲಿ, ಸಾರಿದ ನೀತಿಯ ಸಾರವೆ ಇದು. "ಸ್ವಾರ್ಥಹೀನರಾಗಿ', 'ನಾನಲ್ಲ ನೀನು'' - ಎಲ್ಲಾ ನೀತಿ ನಿಯಮಾವಳಿಗಳ ಹಿನ್ನೆಲೆಯೇ ಇದು. ನಾನು ನಿನ್ನ ಅಂಶ ಎಂಬ ಅದ್ವೈತ ಭಾವವನ್ನು ಒಪ್ಪಿಕೊಂಡಂತೆ ಆಗುವುದು ಇದು. ನಿನಗೆ ತೊಂದರೆ ಕೊಟ್ಟರೆ ನನಗೇ ತೊಂದರೆ ಕೊಟ್ಟಂತೆ ಆಗುವುದು; ನಿನಗೆ ಸಹಾಯಮಾಡಿದರೆ ನನಗೆ ಸಹಾಯಮಾಡಿಕೊಂಡಂತೆ ಎಂಬ ನಿಯಮವನ್ನು ಒಪ್ಪಿಕೊಂಡಂತೆ ಇದು. ನೀನು ಬದುಕಿರುವಾಗ ನಾನು ಸಾಯುವುದಿಲ್ಲ ಎಂಬುದನ್ನು ಒಪ್ಪಿಕೊಂಡಂತೆ ಆಯಿತು. ಒಂದು ಕೀಟ ಪ್ರಪಂಚದಲ್ಲಿರುವವರೆಗೆ ನಾನು ಹೇಗೆ ಸಾಯಬಲ್ಲೆ? ಏಕೆಂದರೆ ನನ್ನ ಜೀವವೇ ಆ ಕೀಟದ ಜೀವದಲ್ಲಿದೆ. ನಾವು ಯಾರಿಗೂ ಸಹಾಯ ಮಾಡದೆ ಇರಲಾರೆವು, ಇತರರ ಶ್ರೇಯಸ್ಸಿನಲ್ಲಿ ನಮ್ಮ ಶ್ರೇಯಸ್ಸು ಇದೆ ಎಂಬುದನ್ನು ಇದು ಸಾರುವುದು.

ವೇದಾಂತವನ್ನೆಲ್ಲಾ ವ್ಯಾಪಿಸಿರುವ ಭಾವನೆಯೇ ಇದು. ಮತ್ತೆ ಇದೇ ಎಲ್ಲಾ ಧರ್ಮಗಳಲ್ಲಿಯೂ ಇರುವುದು. ಸಾಧಾರಣ ಧರ್ಮದಲ್ಲಿ ಮೂರು ಭಾಗಗಳಿವೆ. ನೀವು ಇದನ್ನು ಗಮನಿಸಬೇಕು. ಮೊದಲನೆಯದೆ ತತ್ತ್ವ ಸಿದ್ದಾಂತ; ಇದೇ ಧರ್ಮದ ಸಾರ. ಈ ಸಿದ್ಧಾಂತವೇ ಪುರಾಣಗಳಲ್ಲಿ ಬರುವ ಸಾಧುಸಂತರ ಮತ್ತು ದೇವಮಾನವರ ಜೀವನಗಳಲ್ಲಿ ವ್ಯಕ್ತವಾಗಿದೆ. ಪುರಾಣಗಳಲ್ಲಿ ಇರುವ ಭಾವನೆಗಳೆಲ್ಲಾ ಶಕ್ತಿಗೆ ಸಂಬಂಧಪಟ್ಟವು. ಬಹಳ ಕೆಳಗಿನ ವರ್ಗದ ಪುರಾಣಗಳಲ್ಲಿ ಈ ಶಕ್ತಿ ಇನ್ನೂ ಅನಾಗರಿಕವಾಗಿದೆ. ಅಲ್ಲಿ ಶಕ್ತಿ ಕೇವಲ ದೇಹಶಕ್ತಿಯಾಗಿ ಪರಿಣಮಿಸಿದೆ. ಅಲ್ಲಿನ ನಾಯಕರು ಬಲಾಢ್ಯರು, ಭೀಮನಂತಹವರು. ಒಬ್ಬ ನಾಯಕ ಪ್ರಪಂಚವನ್ನೇ ಗೆಲ್ಲುತ್ತಾನೆ. ಮಾನವನು ಮುಂದುವರಿದಂತೆ ದೈಹಿಕಶಕ್ತಿಗೆ ಮೀರಿದುದನ್ನು ಅವನು ಆರಿಸಿಕೊಳ್ಳಬೇಕಾಗಿದೆ. ಆದಕಾರಣ ಅವನ ನಾಯಕರು ಕೂಡಾ ಉತ್ತಮವಾದುದನ್ನೇ ಆರಿಸಿಕೊಳ್ಳಬೇಕಾಗಿದೆ. ಉತ್ತಮ ವರ್ಗದ ಪುರಾಣದ ನಾಯಕರು ಪ್ರಚಂಡ ಧರ್ಮಾತ್ಮರಾಗಿರುವರು. ಶುದ್ದ ಚಾರಿತ್ರ್ಯದಲ್ಲಿ ಅವರ ಶಕ್ತಿ ವ್ಯಕ್ತವಾಗುತ್ತದೆ. ಅವರು ಒಬ್ಬರೇ ನಿಂತು ಸ್ವಾರ್ಥದ ಮತ್ತು ಅಧರ್ಮದ ಪ್ರವಾಹವನ್ನೆಲ್ಲ ಎದುರಿಸಬಲ್ಲರು. ಧರ್ಮದ ಮೂರನೆಯ ಭಾಗವೇ ಸಂಕೇತಗಳು. ಇವನ್ನೇ ಕ್ರಿಯಾವಿಧಿಗಳು ಕರ್ಮಗಳು ಎಂದು ಕರೆಯುವುದು. ಪುರಾಣಗಳಲ್ಲಿ ನಾಯಕರ ಮೂಲಕ ವ್ಯಕ್ತವಾಗುವುದು ಕೂಡ ಎಲ್ಲರಿಗೂ ಅರ್ಥವಾಗಲಾರದು. ಇದಕ್ಕಿಂತಲೂ ಕೆಳಗೆ ಜನ ಇರುವರು. ಅವರಿಗೆ ಮಕ್ಕಳಿಗೆ ಬೇಕಾಗುವಂತಹ ಶಿಶುವಿಹಾರಗಳು ಇರಬೇಕು. ಅದಕ್ಕೇ ಈ ಸಂಕೇತಗಳು ಹುಟ್ಟುವುವು. ಈ ಸ್ಥೂಲಚಿಹ್ನೆಗಳನ್ನು ಅವರು ಮುಟ್ಟಬಲ್ಲರು, ಗ್ರಹಿಸಬಲ್ಲರು.

ಹೀಗೆ ಪ್ರತಿಯೊಂದು ಧರ್ಮದಲ್ಲಿಯೂ ತತ್ತ್ವ, ಪುರಾಣ, ಕ್ರಿಯಾವಿಧಿಗಳೆಂಬ ಮೂರು ಹಂತಗಳಿವೆ. ವೇದಾಂತದಲ್ಲಿ ಒಂದು ಅನುಕೂಲವಿದೆ. ಅದೃಷ್ಟವಶಾತ್ ಭರತಖಂಡದಲ್ಲಿ ಈ ಮೂರನ್ನೂ ಬೇರೆ ಬೇರೆಯಾಗಿ ವರ್ಗೀಕರಿಸಿರುವರು. ಇತರ ಧರ್ಮಗಳಲ್ಲಿ ತತ್ತ್ವ ಮತ್ತು ಪುರಾಣಗಳೆರಡೂ ಬೆರೆತುಹೋಗಿವೆ. ಇವುಗಳಲ್ಲಿ ಯಾವುದು ತತ್ತ್ವ, ಯಾವುದು ಪುರಾಣ ಎಂದು ಗುರುತು ಹಚ್ಚುವುದೇ ಕಷ್ಟವಾಗಿ ಹೋಗಿದೆ. ತತ್ತ್ವವನ್ನೆಲ್ಲ ನುಂಗಿಕೊಂಡು ಪುರಾಣ ತಾನೇ ತಾನಾಗಿ ನಿಂತಿದೆ. ಕೆಲವು ಶತಮಾನಗಳಾದ ಮೇಲೆ ತತ್ತ್ವದ ಸುಳಿವೇ ಸಿಕ್ಕುವುದಿಲ್ಲ. ತತ್ತ್ವವನ್ನು ವಿವರಿಸುವುದಕ್ಕೆ ಕೊಟ್ಟ ದೃಷ್ಟಾಂತಗಳೇ ತತ್ತ್ವವನ್ನು ಮರೆಮಾಡಿ ನಿಲ್ಲುವುವು. ಜನರು ಬೋಧಕರು ಮತ್ತು ಮಹಾತ್ಮರು ಮುಂತಾದ ಉದಾಹರಣೆಗಳನ್ನು ಮಾತ್ರ ನೋಡುತ್ತಾರೆ. ತತ್ತ್ವಗಳು ಸಂಪೂರ್ಣವಾಗಿ ಮಾಯವಾಗಿ ಹೋಗಿವೆ. ಇದು ಎಷ್ಟು ಮಟ್ಟಿಗೆ ಈಗ ಆಗಿದೆ ಎಂದರೆ ಕ್ರಿಸ್ತನನ್ನು ಬಿಟ್ಟು ಅವನ ತತ್ತ್ವವನ್ನು ವಿವರಿಸಲು ಯಾರಾದರೂ ಪ್ರಯತ್ನಪಟ್ಟರೆ ಜನರು ಅವನ ಮೇಲೆ ತಿರುಗಿಬೀಳುವರು. ಅವನು ಮಾಡುವುದು ತಪ್ಪು, ಅವನು ಕ್ರೈಸ್ತಧರ್ಮವನ್ನೇ ಹಾಳುಮಾಡುತ್ತಿರುವನು ಎನ್ನುವರು. ಇದೇ ರೀತಿ ಮಹಮ್ಮದೀಯ ತತ್ತ್ವವನ್ನು ಬೋಧಿಸಲು ಪ್ರಯತ್ನಪಟ್ಟರೆ ಇದೇ ಪರಿಣಾಮ ಕಾದಿರುವುದು. ಏಕೆಂದರೆ ಸ್ಥೂಲವಾದ ಭಾವನೆಗಳು, ಮಹಾಪುರುಷರ ಮತ್ತು ಮಹಾತ್ಮರ ಜೀವನಗಳು, ಹಿಂದೆ ಇರುವ ತತ್ವವನ್ನೆಲ್ಲ ಸಂಪೂರ್ಣ ಮರೆಮಾಡಿ ನಿಂತಿವೆ.

ವೇದಾಂತದಲ್ಲಿ ಒಂದು ಅನುಕೂಲವೇನೆಂದರೆ ಅದು ಯಾವ ಒಂದು ವ್ಯಕ್ತಿಯ ಬರಹವೂ ಅಲ್ಲದೆ ಇರುವುದು. ಆದಕಾರಣ ಕ್ರೈಸ್ತ, ಬೌದ್ಧ ಮತ್ತು ಮಹಮ್ಮದೀಯ ಧರ್ಮಗಳಲ್ಲಿ ಆದಂತೆ ಮಹಾತ್ಮರು ತತ್ತ್ವವನ್ನೆಲ್ಲ ಮರೆಮಾಡಿ ನಿಲ್ಲಲಾಗಲಿಲ್ಲ. ತತ್ತ್ವಗಳು ಜೀವಂತವಾಗಿವೆ. ಮಹಾತ್ಮರು ಬೇರೆ ಗುಂಪಿನಲ್ಲಿ ಬರುವರು. ವೇದಾಂತಕ್ಕೆ ಅವರ ಪರಿಚಯವೇ ಇಲ್ಲ. ಉಪನಿಷತ್ತುಗಳು ಯಾವುದೊ ಒಬ್ಬ ಮಹಾತ್ಮನನ್ನು ಕುರಿತು ಹೇಳುವುದಿಲ್ಲ. ಅವು ಎಷ್ಟೋ ಮಂದಿ ಸ್ತ್ರೀ ಪುರುಷ ಮಹಾತ್ಮರನ್ನು ಕುರಿತು ಹೇಳುವುವು. ಹಳೆಯ ಹಿಬ್ರೂಗಳಲ್ಲಿ ಇಂತಹ ಕೆಲವು ಭಾವನೆಗಳು ಇದ್ದವು. ಆದರೂ ಹೆಬ್ರೂಗಳ ಗ್ರಂಥದಲ್ಲಿ ಮೋಸಸ್ ತಾನೇ ತಾನಾಗಿ ಬೆಳಗುತ್ತಿರುವನು. ಒಂದು ಜನಾಂಗದ ಧರ್ಮವನ್ನು ಇಂತಹ ಮಹಾತ್ಮರು ತಮ್ಮ ವಶಪಡಿಸಿಕೊಳ್ಳುವುದು ಕೆಟ್ಟದು ಎನ್ನುವುದಿಲ್ಲ. ಆದರೆ ತತ್ತ್ವವೆಲ್ಲ ಸಂಪೂರ್ಣ ಮರೆಯಾದರೆ ಬಹಳ ಅಪಾಯಕರ. ನಾವು ತತ್ತ್ವಗಳನ್ನು ಒಪ್ಪಬಹುದು. ಆದರೆ ಅಷ್ಟೇ ತೃಪ್ತಿಯನ್ನು ವ್ಯಕ್ತಿಗಳನ್ನು ಒಪ್ಪಿಕೊಳ್ಳುವುದರಲ್ಲಿ ತೋರಲಾರೆವು. ವ್ಯಕ್ತಿಗಳು ನಮ್ಮ ಭಾವಗಳಿಗೆ ತೃಪ್ತಿಯನ್ನು ತರುವರು. ತತ್ತ್ವಗಳು ಅವಕ್ಕಿಂತ ಸ್ವಲ್ಪ ಉತ್ತಮವಾದುದಕ್ಕೆ, ಉದ್ವೇಗರಹಿತ ವೈಚಾರಿಕತೆಗೆ ತೃಪ್ತಿಯನ್ನು ತರುವುವು. ಕೊನೆಗೆ ತತ್ತ್ವಗಳೇ ಜಯಿಸಬೇಕು. ಏಕೆಂದರೆ ಅವೇ ಮಾನವನ ಪುರುಷತ್ವ. ಭಾವಗಳು ಅನೇಕವೇಳೆ ನಮ್ಮನ್ನು ಮೃಗೀಯ ಸ್ಥಿತಿಗೆ ಎಳೆಯುವುವು. ಭಾವಗಳಿಗೆ ಯುಕ್ತಿಗಿಂತ ಹೆಚ್ಚಾಗಿ ಇಂದ್ರಿಯದೊಂದಿಗೆ ಸಂಬಂಧವಿದೆ. ಆದಕಾರಣ ತತ್ತ್ವಗಳನ್ನು ಸಂಪೂರ್ಣ ಮರೆತು ಕೇವಲ ಭಾವವೊಂದೇ ಪ್ರಧಾನವಾದರೆ ಧರ್ಮ ಒಂದು ಮತಭ್ರಾಂತಿಯಾಗುವುದು. ಆಗ ಅದು ಪಕ್ಷ-ರಾಜಕೀಯಕ್ಕಿಂತ ಮೇಲೇನೂ ಅಲ್ಲ. ಅತಿ ಭಯಂಕರವಾದ ಮೂಢಭಾವನೆಗಳನ್ನು ತೆಗೆದುಕೊಂಡು ಅವಕ್ಕಾಗಿ ಸಾವಿರಾರು ಸಹೋದರರ ಬಲಿ ಕೊಡಲು ಸಿದ್ದರಾಗುವರು. ಆದಕಾರಣವೇ ಮಹಾತ್ಮರ ಜೀವನ ಮತ್ತು ಉಪದೇಶಗಳು ಕಲ್ಯಾಣಕರವಾದ ಪ್ರಚಂಡವಾದ ಕ್ರಿಯೋತ್ತೇಜಕ ಶಕ್ತಿಸ್ವರೂಪಗಳಾಗಿವೆಯಾದರೂ, ಅವರು ಯಾವ ತತ್ತ್ವವನ್ನು ಉದಾಹರಿಸಲು ನಿಂತಿರುವರೋ ಅದಕ್ಕೆ ಗೌರವವನ್ನು ಕೊಡದಿದ್ದರೆ, ಅದರಿಂದ ಬಹಳ ಅಪಾಯವಿದೆ. ಇದರಿಂದಲೇ ಮತಭ್ರಾಂತಿಯೆಲ್ಲ ಹುಟ್ಟುವುದು, ಇಡೀ ಪ್ರಪಂಚವನ್ನು ರಕ್ತದಲ್ಲಿ ತೋಯಿಸುವುದು. ವೇದಾಂತವು ಈ ಕಷ್ಟವನ್ನು ನಿವಾರಿಸಬಲ್ಲದು. ಏಕೆಂದರೆ ಅದರಲ್ಲಿ ಇರುವವರು ಯಾರೋ ಒಬ್ಬ ಪ್ರವಾದಿಯಲ್ಲ. ಅದರಲ್ಲಿ ಹಲವು ಮಹಾತ್ಮರಿರುವರು. ಅವರೆಲ್ಲ ಋಷಿಗಳು, ಮಂತ್ರದ್ರಷ್ಟಾರರು ಎಂದರೆ ಸತ್ಯವನ್ನು ಸಾಕ್ಷಾತ್ತಾಗಿ ನೋಡಿರುವವರು.

ಮಂತ್ರ ಎಂದರೆ ಮನಸ್ಸಿನಿಂದ ಆಲೋಚಿಸಿದ್ದು. ಋಷಿಗಳು ಈ ಮಂತ್ರದ್ರಷ್ಟಾರರು. ಈ ಮಂತ್ರಗಳು ಯಾರೊಬ್ಬರ ಪ್ರತ್ಯೇಕ ಸ್ವತ್ತೂ ಅಲ್ಲ. ಒಬ್ಬ ಎಷ್ಟೇ ಪ್ರಖ್ಯಾತ ಸ್ತ್ರೀಯಾಗಲಿ ಪುರುಷನಾಗಲಿ, ಅವಳಿಗೆ ಅಥವಾ ಅವನಿಗೆ ಮಾತ್ರ ಇವು ಮೀಸಲಾಗಿರುವುದಿಲ್ಲ. ಪ್ರಪಂಚವು ಸೃಷ್ಟಿಸಿದ ಕ್ರಿಸ್ತ, ಬುದ್ಧರಂತಹ ಮಹಾ ಪುರುಷರಿಗೆ ಮಾತ್ರ ಸೇರಿದುವೂ ಅಲ್ಲ. ಇವು ಬುದ್ದನಿಗೆ ಎಷ್ಟು ಮಟ್ಟಿಗೆ ಸೇರಿರುವುವೋ ಅಷ್ಟೇ ಮಟ್ಟಿಗೆ ದೀನರಲ್ಲಿ ಅತಿ ದೀನರಿಗೆ ಕೂಡ ಸೇರಿವೆ. ಇವು ಕ್ರಿಸ್ತನಿಗೆ ಸೇರಿದಂತೆಯೇ ಅತಿ ಕ್ಷುದ್ರ ಜೀವಿಗೂ ಸೇರಿವೆ. ಏಕೆಂದರೆ ಇವು ವಿಶ್ವಸಾಮಾನ್ಯ ತತ್ತ್ವಗಳು. ಇವನ್ನು ಯಾರೂ ಸೃಷ್ಟಿಸಲಿಲ್ಲ. ಅನಾದಿ ಕಾಲದಿಂದಲೂ ಈ ಸಿದ್ಧಾಂತಗಳು ಇವೆ; ಅನಂತಕಾಲದವರೆಗೆ ಮುಂದೆಯೂ ಇರುವುವು. ಇವು ಅನಾದಿ, ವಿಜ್ಞಾನವು ಬೋಧಿಸುವ ಯಾವ ನಿಯಮಗಳೂ ಇವನ್ನು ಸೃಷ್ಟಿಸಲಿಲ್ಲ. ಕೆಲವು ವೇಳೆ ಇವು ಕಾಣಿಸದೆ ಹೋಗುವುವು; ಮತ್ತೆ ಕೆಲವು ವೇಳೆ ಬೆಳಕಿಗೆ ಬರುವುವು. ಆದರೆ ಪ್ರಕೃತಿಯಲ್ಲಿ ಇವು ಚಿರಕಾಲವೂ ಇರುವುವು. ನ್ಯೂಟನ್ನನು ಹುಟ್ಟದೆ ಇದ್ದಿದ್ದರೂ, ಆಕರ್ಷಣ ನಿಯಮವನ್ನು ಕಂಡುಹಿಡಿಯದೇ ಇದ್ದಿದ್ದರೂ, ಆ ನಿಯಮ ಇರುತ್ತಿತ್ತು, ಅದು ಕೆಲಸಮಾಡುತ್ತಲೇ ಇರುತ್ತಿತ್ತು. ನ್ಯೂಟನ್ನನ ಅದ್ಭುತ ಪ್ರತಿಭೆ ಇದನ್ನು ಕಂಡುಹಿಡಿದದ್ದು, ಇದಕ್ಕೆ ಒಂದುಸಿದ್ದಾಂತದ ರೂಪವನ್ನು ಕೊಟ್ಟಿದ್ದು, ಇದನ್ನು ಬೆಳಕಿಗೆ ತಂದದ್ದು, ಎಲ್ಲರಿಗೂ ತಿಳಿಯುವಂತೆ ಮಾಡಿದ್ದು. ಆಧ್ಯಾತ್ಮಿಕ ಮಹಾಸತ್ಯಗಳಾದ ಧಾರ್ಮಿಕ ನಿಯಮಗಳು ಕೂಡ ಹಾಗೆಯೇ, ಸದಾಕಾಲದಲ್ಲಿಯೂ ಇವು ಚಿರಜಾಗೃತವಾಗಿರುತ್ತವೆ. ವೇದ, ಬೈಬಲ್, ಖುರಾನುಗಳು ಯಾವುದೂ ಇಲ್ಲದೆ ಇದ್ದರೂ, ಮಹಾತ್ಮರು ಋಷಿಗಳು ಹುಟ್ಟದೇ ಇದ್ದರೂ, ಈ ಆಧ್ಯಾತ್ಮಿಕ ನಿಯಮಾವಳಿಗಳು ಜಾರಿಯಲ್ಲಿರುತ್ತಿದ್ದವು. ಅವು ವ್ಯಕ್ತವಾಗುವುದಕ್ಕೆ ಸ್ವಲ್ಪ ಕಾಲ ತಡವಾಗುವುದು, ಅಷ್ಟೆ. ನಿಧಾನವಾಗಿ ಆದರೂ ನಿಶ್ಚಯವಾಗಿ ಮಾನವಕೋಟಿಯನ್ನು ಮೇಲೆತ್ತಲು, ಅವರ ಸ್ವಭಾವವನ್ನು ತಿದ್ದಲು ಅವು ಯತ್ನಿಸುತ್ತವೆ. ಇಂತಹ ನಿಯಮಾವಳಿಗಳನ್ನು ನೋಡುವವರು, ಕಂಡುಹಿಡಿಯುವವರು ಪ್ರವಾದಿಗಳು. ಆಧ್ಯಾತ್ಮಿಕ ಪ್ರಪಂಚದಲ್ಲಿ ಪ್ರವಾದಿಗಳು ಆ ನಿಯಮಾವಳಿಗಳನ್ನು ಬೆಳಕಿಗೆ ತರುವರು. ವಿಜ್ಞಾನ ಪ್ರಪಂಚದಲ್ಲಿ ನ್ಯೂಟನ್, ಗೆಲಿಲಿಯೋ ಹೇಗೋ ಹಾಗೆಯೇ ಆಧ್ಯಾತ್ಮಿಕ ಪ್ರಪಂಚದಲ್ಲಿ ಈ ಪ್ರವಾದಿಗಳು. ಈ ಯಾವ ಒಂದು ನಿಯಮವೂ ತಮಗೇ ಸೇರಿದ್ದೆಂದು ಅವರು ಭಾವಿಸಲಾರರು. ಇವು ಪ್ರಕೃತಿಯಲ್ಲಿರುವವರೆಲ್ಲರ ಸರ್ವ ಸಾಮಾನ್ಯ ಸೊತ್ತು.

ಹಿಂದೂಗಳು ಹೇಳುವಂತೆ ವೇದ ಸನಾತನವಾದುದು. ಸನಾತನ ಎಂದರೆ ಏನು ಅರ್ಥ ಎಂಬುದು ನಮಗೆ ಈಗ ಗೊತ್ತಾಗುವುದು - ಅದೇ, ಪ್ರಕೃತಿಗೆ ಹೇಗೆ ಆದಿ ಅಂತ್ಯಗಳಿಲ್ಲವೋ ಹಾಗೆಯೇ ನಿಯಮಗಳಿಗೆ ಆದಿ ಅಂತ್ಯಗಳಿಲ್ಲವೆಂಬುದು. ಈ ಪೃಥ್ವಿಯಾದ ಮೇಲೆ ಇನ್ನೊಂದು ಪೃಥ್ವಿ, ವ್ಯೂಹವಾದ ಮೇಲೆ ವ್ಯೂಹ ಸೃಷ್ಟಿಯಾಗಿ ಕೆಲವು ಕಾಲವಿದ್ದು ಅವ್ಯಕ್ತದಲ್ಲಿ ಪುನಃ ಲೀನವಾಗಿ ಹೋಗುವುವು. ಆದರೆ ವಿಶ್ವ ಯಾವಾಗಲೂ ಹಾಗೆಯೇ ಇರುವುದು. ಕೋಟ್ಯಂತರ ಬ್ರಹ್ಮಾಂಡಗಳು ಸೃಷ್ಟಿಯಾಗುತ್ತಿವೆ. ಕೋಟ್ಯಂತರ ಬ್ರಹ್ಮಾಂಡಗಳು ನಾಶವಾಗುತ್ತಿವೆ. ಆದರೂ ವಿಶ್ವ ಹಾಗೆಯೇ ಇರುವುದು. ಯಾವುದೋ ಒಂದು ಗ್ರಹಕ್ಕೆ ಬೇಕಾದರೆ ಅದರ ಆದಿ ಅಂತ್ಯಗಳನ್ನು ಹೇಳಬಹುದು. ಆದರೆ ಇಡೀ ವಿಶ್ವದ ದೃಷ್ಟಿಯಿಂದ ಕಾಲವೆಂಬುದಕ್ಕೆ ಅರ್ಥವಿಲ್ಲ. ಇದರಂತೆಯೇ ಪ್ರಕೃತಿ ನಿಯಮಗಳು, ವೈಜ್ಞಾನಿಕ ನಿಯಮಗಳು, ಮಾನಸಿಕ ನಿಯಮಗಳು, ಆಧ್ಯಾತ್ಮಿಕ ನಿಯಮಗಳು, ಅವು ಆದಿ ಅಂತ್ಯಗಳಿಲ್ಲದೆ ಇವೆ. ಈ ನಿಯಮಗಳನ್ನೆಲ್ಲ ಎಲ್ಲೋ ಈಚೆಗೆ ಕೆಲವು ವರುಷಗಳಿಂದ, ಹೆಚ್ಚು ಎಂದರೆ ಕೆಲವು ಸಾವಿರ ವರುಷಗಳಿಂದ ಮಾನವನು ಕಂಡುಹಿಡಿಯಲು ಯತ್ನಿಸುತ್ತಿರುವನು. ಇನ್ನೂ ನಮಗೆ ತಿಳಿಯದಿರುವ ಅನಂತನಿಯಮಗಳಿವೆ. ವೇದಗಳಲ್ಲಿ ಪ್ರಾರಂಭದಲ್ಲೇ ನಾವು ಕಲಿಯವ ದೊಡ್ಡ ಪಾಠವೇ ಇದು - ಧರ್ಮ ಇದೀಗ ಪ್ರಾರಂಭವಾಗಿದೆ ಎಂಬುದು. ಆಧ್ಯಾತ್ಮಿಕ ಸತ್ಯದ ಅಸೀಮ ಮಹಾಸಾಗರ ನಮ್ಮ ಮುಂದಿದೆ. ನಾವು ಶ್ರಮಿಸಿ ಅದನ್ನು ಕಂಡುಹಿಡಿದು ನಮ್ಮ ಜೀವನದಲ್ಲಿ ಅನುಷ್ಠಾನಕ್ಕೆ ತರಬೇಕಾಗಿದೆ. ಸಹಸ್ರಾರು ಮಂದಿ ಪ್ರವಾದಿಗಳನ್ನು ಜಗತ್ತು ಈಗಾಗಲೇ ನೋಡಿದೆ. ಮುಂದೆ ಕೋಟ್ಯಂತರ ಪ್ರವಾದಿಗಳು ಬರಲಿರುವರು.

ಪ್ರತಿಯೊಂದು ಸಮಾಜದಲ್ಲಿಯೂ ಹಲವು ಮಹಾತ್ಮರಿದ್ದ ಒಂದು ಕಾಲವಿತ್ತು. ಜಗತ್ತಿನ ಪ್ರತಿಯೊಂದು ಊರಿನ ಪ್ರತಿಯೊಂದು ಬೀದಿಯಲ್ಲೂ ಮಹಾತ್ಮರು ಸಂಚರಿಸುವ ಕಾಲ ಒಂದು ಬರುವುದು. ಹಿಂದಿನ ಕಾಲದಲ್ಲಿ ಸಮಾಜವು ಎಲ್ಲೊ ಕೆಲವು ಪ್ರತ್ಯೇಕವಾದ ಅಪೂರ್ವ ವ್ಯಕ್ತಿಗಳನ್ನು ಮಾತ್ರ ಮಹಾತ್ಮರ ಪಾತ್ರಕ್ಕೆ ಆರಿಸುತ್ತಿತ್ತು. ಧಾರ್ಮಿಕನಾಗುವುದು ಎಂದರೆ ಒಬ್ಬ ಮಹಾತ್ಮನಾಗುವುದು; ಯಾರೂ ಮಹಾತ್ಮರಾಗದೆ, ಧಾರ್ಮಿಕರಾಗಲಾರರು ಎಂದು ತಿಳಿಯುವ ಒಂದು ಸಮಯ ಬರುವುದು. ಧಾರ್ಮಿಕ ರಹಸ್ಯವೆಂದರೆ ಧಾರ್ಮಿಕ ಭಾವನೆಗಳನ್ನು ಆಲೋಚಿಸು ವುದಾಗಲಿ ಮಾತನಾಡುವುದಾಗಲಿ ಅಲ್ಲ; ವೇದಗಳು ಬೋಧಿಸಿರುವಂತೆ ಅವನ್ನು ಸಾಕ್ಷಾತ್ಕಾರ ಮಾಡಿಕೊಳ್ಳುವುದು. ಇದುವರೆಗೆ ನಾವಿನ್ನೂ ಕಂಡುಹಿಡಿಯದ ಹೊಸ ಹೊಸ ಸತ್ಯಗಳನ್ನು, ಉತ್ತಮ ತರದ ಸತ್ಯಗಳನ್ನು ಕಂಡುಹಿಡಿದು ಅವನ್ನು ಸಮಾಜದಲ್ಲಿ ಹರಡಬೇಕು. ಧಾರ್ಮಿಕ ಅಧ್ಯಯನ ಒಬ್ಬನನ್ನು ಮಹಾತ್ಮನನ್ನಾಗಿ ಮಾಡುವ ಶಿಕ್ಷಣಾಲಯ ಎಂದು ಭಾವಿಸುವ ಸಮಯವೊಂದು ಬರುವುದು. ಶಾಲಾ ಕಾಲೇಜುಗಳು ಇಂತಹ ಮಹಾತ್ಮರ ಶಿಕ್ಷಣಾಲಯಗಳಾಗಬೇಕು. ಇಡೀ ಜಗತ್ತೇ ಮಹಾತ್ಮರಿಂದ ತುಂಬಿಹೋಗಬೇಕು. ಒಬ್ಬ ಮಹಾತ್ಮನಾಗುವವರೆಗೆ ಧರ್ಮ ಒಂದು ಅಣಕ, ಒಂದು ಸಾಮತಿ, ಅಷ್ಟೆ. ನಾವು ಎದುರಿಗೆ ಇರುವ ಗೊಡೆಯನ್ನು ನೋಡುವುದಕ್ಕಿಂತ ಸಾವಿರಪಾಲು ತೀವ್ರವಾಗಿ ಧರ್ಮವನ್ನು ನೋಡಬೇಕು, ಅನುಭವಿಸಬೇಕು.

ಹಲವು ಧರ್ಮಗಳಲ್ಲಿ ಒಂದು ಮೂಲಸಿದ್ದಾಂತ ಹರಿಯುತ್ತಿದೆ. ಅದು ಆಗಲೇ ನಮ್ಮ ಮುಂದೆ ಪ್ರಕಟವಾಗಿದೆ. ಪ್ರತಿಯೊಂದು ವಿಜ್ಞಾನವೂ ಎಲ್ಲಿ ಒಂದು ಏಕತೆಯನ್ನು ಕಾಣುವುದೊ ಅಲ್ಲಿ ಕೊನೆಗಾಣಬೇಕಾಗುವುದು. ಅಲ್ಲಿಂದ ಮುಂದಕ್ಕೆ ನಾವು ಹೋಗಲಾರೆವು. ಒಂದು ಪರಿಪೂರ್ಣ ಏಕತೆಯನ್ನು ಪಡೆದಾದ ಮೇಲೆ ಅದು ಯಾವ ಹೊಸ ಸಿದ್ದಾಂತವನ್ನೂ ಹೇಳಲಾರದು. ಧರ್ಮವು ಮಾಡಬೇಕಾದ ಕೆಲಸವೇ ಅದರ ವಿವರಗಳನ್ನು ಕುರಿತು ಆಲೋಚಿಸುವುದು. ಯಾವ ವಿಜ್ಞಾನಶಾಸ್ತ್ರವನ್ನು ಬೇಕಾದರೂ ತೆಗೆದುಕೊಳ್ಳಿ. ಉದಾಹರಣೆಗೆ ರಸಾಯನಶಾಸ್ತ್ರ; ಯಾವ ಒಂದು ವಸ್ತುವಿನಿಂದ ಉಳಿದ ಎಲ್ಲಾ ವಸ್ತುಗಳನ್ನೂ ನಾವು ಸೃಷ್ಟಿಸಬಹುದೊ ಅದು ನಮಗೆ ದೊರೆಯಿತು ಎಂದು ಇಟ್ಟುಕೊಳ್ಳೋಣ. ಆಗ ರಸಾಯನಶಾಸ್ತ್ರವು ವಿಜ್ಞಾನದ ದೃಷ್ಟಿಯಿಂದ ಪರಿಪೂರ್ಣವಾದಂತೆ. ಅನಂತರ ನಮಗೆ ಉಳಿಯುವುದೇ ಆ ಒಂದು ವಸ್ತುವಿನ ಹೊಸ ಹೊಸ ಸಂಯೋಗಗಳನ್ನು ಕಂಡುಹಿಡಿಯುವುದು ಮತ್ತು ಅದನ್ನು ಜೀವನದಲ್ಲಿ ಹೇಗೆ ಉಪಯೋಗಿಸಿಕೊಳ್ಳಬಹುದು ಎಂಬುದನ್ನು ನೋಡಬೇಕಾದುದು. ಇದರಂತೆಯೇ ಧರ್ಮ ಕೂಡ. ಧರ್ಮದ ಹಿಂದೆ ಇರುವ ಅದ್ಭುತ ನಿಯಮಾವಳಿ, ಅದರ ರಚನೆ ಮತ್ತು ವ್ಯಾಪ್ತಿ ಇವುಗಳನ್ನು ಶತಶತಮಾನಗಳ ಹಿಂದೆಯೇ ಕಂಡುಹಿಡಿದಿರುವರು. ವೇದಗಳಲ್ಲಿ ಉಕ್ತವಾಗಿರುವ 'ಸೋಽಹಂ' - ನಾನೇ ಅವನು ಎಂಬ ಉಚ್ಚತಮ ಭಾವನೆಯೇ ಅದು. ಇಲ್ಲಿ ಚೇತನ- ಅಚೇತನಗಳೆರಡೂ ಸಂಧಿಸುವುವು. ಇದನ್ನು ದೇವರು, ಅಲ್ಲ, ಬ್ರಹ್ಮ, ಜಿಹೋವ ಮುಂತಾದ ಹೆಸರುಗಳಿಂದ ಕರೆಯುವರು. ನಾವು ಇದನ್ನು ಅತಿಕ್ರಮಿಸಿ ಹೋಗಲಾರೆವು. ನಮಗಾಗಿ ಈ ಅದ್ಭುತ ನಿಯಮಾವಳಿ ಆಗಲೇ ಸಿದ್ದವಾಗಿದೆ. ಅದನ್ನು ನಮ್ಮ ನಿತ್ಯ ಜೀವನದಲ್ಲಿ ಅನುಷ್ಠಾನಕ್ಕೆ ತರುವುದೊಂದು ಉಳಿದಿದೆ. ನಮ್ಮಲ್ಲಿ ಪ್ರತಿಯೊಬ್ಬರೂ ಮಹಾತ್ಮರಾಗುವುದಕ್ಕಾಗಿ ಯತ್ನಿಸಬೇಕಾಗಿದೆ. ಇದೊಂದು ಮಹಾ ಕೆಲಸ ನಮ್ಮ ಮುಂದೆ ಇದೆ.

ಹಿಂದಿನ ಕಾಲದಲ್ಲಿ ಪ್ರವಾದಿಯೆಂದರೆ ಅನೇಕರಿಗೆ ಅರ್ಥವಾಗುತ್ತಿರಲಿಲ್ಲ. ಹೇಗೋ ಅದೃಷ್ಟದಿಂದಲೋ ಏನೋ, ಯಾವ ದೇವರ ಇಚ್ಛೆಯಿಂದಲೋ ಏನೋ, ಮಾನವನಿಗೆ ಆಧ್ಯಾತ್ಮಿಕ ಶಕ್ತಿ ಕರಗತವಾಗುವುದೆಂದು ಭಾವಿಸಿದ್ದರು. ಆದರೆ ಆಧುನಿಕ ಕಾಲದಲ್ಲಿ, ಈ ವಿದ್ಯೆ ಎಲ್ಲಾ ಮಾನವರ ಆಜನ್ಮಸಿದ್ಧ ಹಕ್ಕು - ಅವರು ಯಾರಾದರಾಗಲಿ, ಎಲ್ಲಿ ಬೇಕಾದರೂ ಇರಲಿ. ಈ ಜಗತ್ತಿನಲ್ಲಿ ಆಕಸ್ಮಿಕ ಎಂಬುದಿಲ್ಲ ಎಂಬುದನ್ನು ಪ್ರಮಾಣಪೂರ್ವಕವಾಗಿ ತೋರಲು ನಾವು ಸಿದ್ಧರಾಗಿರುವೆವು. ಯಾವನಾದರೂ ಅಕಸ್ಮಾತ್ ಏನೊ ಪಡೆದನು ಎಂದರೆ ಅವನು ಕಾಲಾಂತರಗಳಿಂದಲೂ ನಿಧಾನವಾಗಿ ಆದರೂ ನಿಶ್ಚಯವಾಗಿ ಇದಕ್ಕಾಗಿ ಶ್ರಮಿಸುತ್ತಿದ್ದನು. ಪ್ರಸ್ತುತ ಪ್ರಶ್ನೆ ಇದು; ನಾವು ಪ್ರವಾದಿಗಳಾಗಬೇಕೆ? ಇಚ್ಚೆ ಇದ್ದರೆ ಆಗುವೆವು.

ಪ್ರವಾದಿಗಳನ್ನು ಮಾಡುವುದೇ ನಮ್ಮ ಮುಂದೆ ಇರುವ ಮಹತ್ಕಾರ್ಯ. ತಿಳಿದೊ ತಿಳಿಯದೆಯೊ ಜಗತ್ತಿನ ಮಹಾಧರ್ಮಗಳೆಲ್ಲ ಈ ಒಂದು ಉದ್ದೇಶ ಸಾಧನೆಗಾಗಿ ಕೆಲಸಮಾಡುತ್ತಿವೆ. ಅವುಗಳಲ್ಲಿ ಈ ಒಂದು ವ್ಯತ್ಯಾಸ ಮಾತ್ರ ಇದೆ. ಅದೇ, ಪ್ರತ್ಯಕ್ಷ ಆತ್ಮಜ್ಞಾನ ಬದುಕಿರುವಾಗ ದೊರಕುವುದಿಲ್ಲ, ಮನುಷ್ಯ ಸಾಯಬೇಕು, ಸತ್ತಮೇಲೆ ಬೇರೊಂದು ಲೋಕದಲ್ಲಿ ಅವನಿಗೆ ಆತ್ಮಜ್ಞಾನ ಲಭಿಸುವುದು, ಯಾವುದನ್ನು ಅವನು ಈಗ ನಂಬುವನೊ ಅದನ್ನು ಆಗ ನೋಡುವ ಕಾಲ ಬರುವುದು ಎಂದು ಹಲವು ಧರ್ಮಗಳು ಸಾರುವುವು. ಹೀಗೆ ಹೇಳುವವರಿಗೆ ವೇದಾಂತವು, “ಹಾಗಾದರೆ ಆಧ್ಯಾತ್ಮವಿದ್ಯೆ ಇದೆ ಎಂದು ಹೇಗೆ ಗೊತ್ತು?'' ಎಂದು ಪ್ರಶ್ನಿಸುವುದು. ಆಗ ಅವರು ಅರಿಯದ ಮತ್ತು ಎಂದೆಂದಿಗೂ ಅರಿಯಲಾಗದುದರ ಅನುಭವಗಳನ್ನು ಈ ಜೀವನದಲ್ಲಿ ಪಡೆದ ವಿಶಿಷ್ಟ ವ್ಯಕ್ತಿಗಳು ಕೆಲವರು ಯಾವಾಗಲೂ ಇದ್ದೇ ತೀರಬೇಕು ಎಂದು ಒಪ್ಪಿಕೊಳ್ಳಬೇಕಾಗುವುದು.

ಆದರೆ ಇಲ್ಲಿಯೂ ಒಂದು ತೊಂದರೆ ಇದೆ. ಅವರು ಹೇಗೊ ಅಕಸ್ಮಾತ್ತಾಗಿ ಈ ಶಕ್ತಿಯನ್ನು ಪಡೆದ ವಿಶಿಷ್ಟ ವ್ಯಕ್ತಿಗಳಾಗಿದ್ದರೆ ನಮಗೆ ಅದನ್ನು ಒಪ್ಪಿಕೊಳ್ಳಲು ಅಧಿಕಾರವಿಲ್ಲ. ಕೇವಲ ಅಕಸ್ಮಾತ್ತಾಗಿ ಆಗುವುದನ್ನು ನಂಬುವುದು ಒಂದು ಪಾಪ. ಏಕೆಂದರೆ ಅಕಸ್ಮಾತ್ತಾಗಿ ನಡೆಯುವುದನ್ನು ನಾವು ಅರ್ಥಮಾಡಿಕೊಳ್ಳಲಾರೆವು. ಜ್ಞಾನವೆಂದರೇನು? ವಿಚಿತ್ರವಾಗಿರುವುದನ್ನು ನಾಶಗೊಳಿಸುವುದು. ಒಬ್ಬ ಹುಡುಗ ದಾರಿಯಲ್ಲೋ ಅಥವಾ ಸರ್ಕಸ್ಸಿನಲ್ಲೂ ಒಂದು ವಿಚಿತ್ರ ಪ್ರಾಣಿಯನ್ನು ನೋಡಿದ ಎಂದು ಭಾವಿಸಿ. ಅದು ಏನು ಎಂದು ಅವನಿಗೆ ಅರ್ಥವಾಗುವುದಿಲ್ಲ. ಅವನು ಅನಂತರ ಒಂದು ಬೇರೆ ದೇಶಕ್ಕೆ ಹೋದಾಗ, ಇಂತಹ ನೂರಾರು ಪ್ರಾಣಿಗಳನ್ನು ನೋಡಿದಾಗ, ಇವೆಲ್ಲ ಒಂದು ಜಾತಿಗೆ ಸೇರಿದುವು ಎಂದು ತೃಪ್ತನಾಗುವನು. ಯಾವುದಾದರೂ ಒಂದರ ಹಿಂದೆ ಇರುವ ನಿಯಮವನ್ನು ತಿಳಿಯುವುದೆ ಜ್ಞಾನ. ಅಜ್ಞಾನವೆಂದರೆ ಒಂದು ಘಟನೆಯನ್ನು ಅದರ ಹಿಂದೆ ಇರುವ ನಿಯಮದ ಆಧಾರವಿಲ್ಲದೆ ತಿಳಿಯಲು ಯತ್ನಿಸುವುದು. ಮೂಲ ನಿಯಮಕ್ಕೆ ವಿನಾಯಿತಿಯಾಗಿ ಇರುವ ಒಂದು ಅಥವಾ ಕೆಲವನ್ನು ನೋಡಿದಾಗ ನಾವು ಅಜ್ಞಾನದಲ್ಲಿರುವೆವು, ನಮಗೆ ಗೊತ್ತಿಲ್ಲ. ಈ ಮಹಾತ್ಮರು ವಿಶಿಷ್ಟ ವ್ಯಕ್ತಿಗಳಾಗಿದ್ದರೆ, ಈ ಅತೀಂದ್ರಿಯ ಅನುಭವಗಳು ಅವರಿಗೆ ಮಾತ್ರ ಕಂಡು ಇತರರಿಗೆ ಕಾಣದೇ ಇದ್ದರೆ, ನಾವು ಅವನ್ನು ನಂಬುವುದಿಲ್ಲ. ಏಕೆಂದರೆ ಅವು ಯಾವ ನಿಯಮಕ್ಕೂ ಒಳಪಡದ ವಿಚಿತ್ರ ಘಟನೆಗಳು. ನಾವೇ ಮಹಾತ್ಮರಾದಾಗ ಅವುಗಳನ್ನು ನಂಬಬಹುದು.

ಸಮುದ್ರದ ಹಾವಿನ ವಿಷಯವಾಗಿ ಬರುವ ಹಲವು ವಿಡಂಬನೆಗಳನ್ನು ನೀವೆಲ್ಲ ದಿನಪತ್ರಿಕೆಗಳಲ್ಲಿ ನೋಡಿರುವಿರಿ. ಇದು ಏತಕ್ಕೆ? ಏಕೆಂದರೆ ಎಲ್ಲೋ ಕೆಲವರು ಅಪರೂಪಕ್ಕೊಮ್ಮೆ ಸಮುದ್ರದ ಹಾವಿನ ವಿಷಯವಾಗಿ ಕಥೆಗಳನ್ನು ಹೇಳಿದ್ದಾರೆ. ಅದನ್ನು ಇತರರು ಯಾರೂ ನೋಡಿಲ್ಲ. ಇದು ಸರಿಯೆ ತಪ್ಪೆ ಎಂದು ಹೋಲಿಸಿ ನೋಡಲು ಅವರಿಗೆ ಯಾವ ಒಂದು ನಿರ್ದಿಷ್ಟವಾದ ನಿಯಮವೂ ಸಿಕ್ಕಿಲ್ಲ. ಆದ ಕಾರಣವೇ ಜಗತ್ತು ಅದನ್ನು ನಂಬುವುದಿಲ್ಲ. ಮಹಾತ್ಮನೊಬ್ಬ ಆಕಾಶದಲ್ಲಿ ಮಾಯವಾಗಿ ಹೋದ ಎಂದು ಯಾರಾದರೂ ನನಗೆ ಹೇಳಿದರೆ ನನಗೆ ಅದನ್ನು ನೋಡುವ ಅಧಿಕಾರವಿದೆ. ನಿನ್ನ ತಂದೆ ಅಥವಾ ನಿನ್ನ ತಾತ ಇದನ್ನು ನೋಡಿದ್ದಾನೆಯೆ ಎಂದು ನಾನು ಅವನನ್ನು ಕೇಳುತ್ತೇನೆ. ಇಲ್ಲ, ಐದುಸಾವಿರ ವರುಷಗಳ ಹಿಂದೆ ಇದು ಆಯಿತು ಎನ್ನುವರು. ನಾನು ಅದನ್ನು ನಂಬದೆ ಇದ್ದರೆ ಎಂದೆಂದಿಗೂ ನರಕದಲ್ಲಿ ಬೇಯಬೇಕಾಗುವುದು.

ಎಂತಹ ಮೂಢ ನಂಬಿಕೆಯ ಹೊರೆ ಇದು? ಇದರ ಪರಿಣಾಮವಾಗಿ ಮಾನವ ತನ್ನ ಸಹಜ ಸ್ವಭಾವವಾದ ಪಾವಿತ್ರ್ಯವನ್ನು ಕಳೆದುಕೊಂಡು ಮೂಢನಂತೆ ಆಗುವನು. ಬರೀ ನಾವು ಹೇಳಿದ್ದನ್ನು ನಂಬುವ ಹಾಗೆ ಇದ್ದರೆ ದೇವರು ಏತಕ್ಕೆ ನಮಗೆ ವಿವೇಚನೆಯನ್ನು ಕೊಟ್ಟಿರುವನು? ಯುಕ್ತಿಗೆ ವಿರೋಧವಾಗಿ ನಂಬುವುದು ಈಶ್ವರ ನಿಂದೆಯಲ್ಲವೆ? ಭಗವಂತ ಕೊಟ್ಟಿರುವ ಶ್ರೇಷ್ಠತಮ ಬಹುಮಾನವಾದ ಯುಕ್ತಿಯನ್ನು ಅಲ್ಲಗಳೆಯುವುದು ನ್ಯಾಯವೆ? ತನ್ನ ಯುಕ್ತಿಯನ್ನು ಉಪಯೋಗಿಸಿ ಯಾವ ನಿರ್ಧಾರಕ್ಕೂ ಬರಲಾರದೆ ನಂಬಲಾರದ ಮಾನವನನ್ನು ಬೇಕಾದರೆ ದೇವರು ಮೆಚ್ಚಿಯಾನು. ಆದರೆ ಯುಕ್ತಿ ಎಂಬ ಅವನ ವರವನ್ನು ನಿರಾಕರಿಸಿ ಸುಮ್ಮನೆ ನಂಬುವವನನ್ನು ಅವನು ಮೆಚ್ಚಲಾರನೆಂದು ಭಾವಿಸುತ್ತೇನೆ. ಅವನು ತನ್ನ ಸ್ವಭಾವವನ್ನು ಕಳೆದುಕೊಂಡು ಅಧೋಗತಿಗೆ ಇಳಿದು ಮೃಗಕ್ಕೆ ಸಮನಾಗುವನು. ಜ್ಞಾನೇಂದ್ರಿಯಗಳನ್ನು ಅಧೋಗತಿಗೆ ಎಳೆದು ನಾಶವಾಗುವನು. ನಾವು ವಿಚಾರ ಮಾಡಬೇಕು. ಪ್ರತಿಯೊಂದು ದೇಶದಲ್ಲಿಯೂ ಹಿಂದಿನ ಶಾಸ್ತ್ರಗಳಲ್ಲಿ ಬರುವ ಮಹಾತ್ಮರು ಸಾಧುಸಂತರು ಹೇಳುವುದು ಸತ್ಯ ಎಂದು ಯುಕ್ತಿ ಒಪ್ಪಿಕೊಂಡರೆ ಆಗ ನಾವು ಅವರನ್ನು ನಂಬುತ್ತೇವೆ. ನಮ್ಮಲ್ಲಿಯೇ ಅಂತಹ ಮಹಾತ್ಮರನ್ನು ನೋಡಿದಾಗ ನಾವು ನಂಬುತ್ತೇವೆ. ಆಗ ಅವರು ವಿಶಿಷ್ಟ ವ್ಯಕ್ತಿಗಳಲ್ಲಿ, ಕೆಲವು ತತ್ತ್ವಗಳ ಪ್ರತಿನಿಧಿಗಳು ಎಂಬುದು ಗೊತ್ತಾಗುತ್ತದೆ. ಅವರು ಸಾಧನೆ ಮಾಡಿದರು. ಇದರ ಪರಿಣಾಮವಾಗಿ ಆ ತತ್ತ್ವಗಳು ಅವರ ಮೂಲಕ ವ್ಯಕ್ತವಾದುವು. ನಾವು ಕೂಡ ಅವನ್ನು ವ್ಯಕ್ತಪಡಿಸಲು ಸಾಧನೆ ಮಾಡಬೇಕು. ಅವರು ಮಹಾತ್ಮರಾಗಿದ್ದರು ಎಂಬುದನ್ನು, ನಾವೂ ಮಹಾತ್ಮರಾದಾಗ ನಂಬುವೆವು. ಅವರು ಭಗವಂತನಿಗೆ ಸಂಬಂಧಪಟ್ಟ ವಿಷಯಗಳನ್ನು ಕಂಡ ಮಹಾಋಷಿಗಳು. ಅವರು ಇಂದ್ರಿಯಗಳನ್ನು ಅತಿಕ್ರಮಿಸಿ, ಅತೀಂದ್ರಿಯ ವಸ್ತುಗಳ ಕ್ಷಣಿಕ ನೋಟವನ್ನು ಪಡೆದರು. ನಾವು ಕೂಡ ಹಾಗೆ ಮಾಡಿದ ಮೇಲೆ ಅದನ್ನು ನಂಬುವೆವು. ಅದಕ್ಕೆ ಮುಂಚೆ ಅಲ್ಲ.

ಇದೇ ವೇದಾಂತದ ಏಕಮಾತ್ರ ತತ್ತ್ವ. ಧರ್ಮ ಎಂದರೆ ಇಲ್ಲಿ ಈಗ ಇರುವ ಸ್ಥಿತಿ ಎಂದು ವೇದಾಂತ ಸಾರುವುದು. ಏಕೆಂದರೆ ಈ ಜನ್ಮ, ಆ ಜನ್ಮ, ಜನನ, ಮರಣ, ಈ ಜಗತ್ತು, ಆ ಜಗತ್ತು ಇವೆಲ್ಲ ಮೂಢನಂಬಿಕೆ, ಪೂರ್ವಕಲ್ಪಿತ ಅಭಿಪ್ರಾಯ. ಭೂತ ಭವಿಷ್ಯ ಎಂಬ ಕಾಲವಿಭಾಗಗಳೆಲ್ಲ ಕೇವಲ ಕಾಲ್ಪನಿಕವಾದವುಗಳು. ಹತ್ತು ಗಂಟೆ ಮತ್ತು ಹನ್ನೆರಡು ಗಂಟೆ ಇವುಗಳಲ್ಲಿ, ಪ್ರಕೃತಿಯಲ್ಲಿ ನಾವು ಮಾಡಿದ ಕೆಲವು ಬದಲಾವಣೆಗಳಲ್ಲದೆ ಮತ್ತೇನು ವ್ಯತ್ಯಾಸವಿರುವುದು? ಕಾಲ ಯಥಾಪ್ರಕಾರವಾಗಿ ಹರಿಯುತ್ತಿರುವುದು. ಈ ಜೀವನ ಆ ಜೀವನ ಎಂದರೆ ಅರ್ಥವೇನು? ಇದೆಲ್ಲ ಕಾಲದ ಪ್ರಶ್ನೆ. ಯಾವುದನ್ನು ಕಾಲದಲ್ಲಿ ಕಳೆದುಕೊಳ್ಳುವೆವೊ ಅದನ್ನು ಹೆಚ್ಚು ವೇಗದಿಂದ ಕೆಲಸ ಮಾಡಿ ಸಾಧಿಸಬಹುದು. ಆದಕಾರಣವ ಧರ್ಮವನ್ನು ಈಗಲೇ ಸಾಕ್ಷಾತ್ಕಾರ ಮಾಡಿಕೊಳ್ಳಬೇಕು ಎನ್ನುವುದು ವೇದಾಂತ. ನೀವು ಧಾರ್ಮಿಕರಾಗಬೇಕು ಎಂದರೆ, ಮೊದಲು ಯಾವ ಧರ್ಮವೂ ಇಲ್ಲದೆ ಪ್ರಾರಂಭಿಸಬೇಕು; ಸಾಧನೆ ಮಾಡಿ ಅದನ್ನು ಸಾಕ್ಷಾತ್ಕಾರ ಮಾಡಿಕೊಳ್ಳಬೇಕು. ನೀವು ಅದನ್ನು ಮಾಡಿದಾಗ ಮಾತ್ರ ನಿಮ್ಮ ಪಾಲಿಗೆ ಧರ್ಮವಿರುವುದು. ನೀವು ಅದಕ್ಕೆ ಮುಂಚೆ ನಾಸ್ತಿಕರಿಗಿಂತ ಮೇಲೇನೂ ಅಲ್ಲ, ಅಥವಾ ಕೀಳು ಎಂದು ಬೇಕಾದರೂ ಹೇಳಬಹುದು. ಏಕೆಂದರೆ ನಾಸ್ತಿಕ ಪ್ರಾಮಾಣಿಕ, ಅವನು ಘಂಟಾಘೋಷವಾಗಿ ನನಗೆ ಇದು ಗೊತ್ತಿಲ್ಲ ಎನ್ನುತ್ತಾನೆ. ಆದರೆ ಇತರರಾದರೊ ತಮಗೆ ಏನೂ ಗೊತ್ತಿಲ್ಲದೆ ಇದ್ದರೂ ತಾವು ತುಂಬಾ ಧಾರ್ಮಿಕರು ಎಂದು ಸಾರುತ್ತಿರುವರು. ಅವರಿಗೆ ಯಾವ ಧರ್ಮವಿದೆಯೊ ಅದು ಯಾರಿಗೂ ಗೊತ್ತಿಲ್ಲ, ಏಕೆಂದರೆ ಯಾವುದೋ ಅಡಗೂಲಜ್ಜಿ ಕಥೆಯನ್ನೊ, ಪಾದ್ರಿ ಹೇಳಿರುವುದನ್ನೊ ಕಣ್ಣು ಮುಚ್ಚಿ ನಂಬುತ್ತಾರೆ. ಅದನ್ನು ನಂಬದಿದ್ದರೆ ಅವರಿಗೇ ಕೇಡು. ಧರ್ಮ ಎಂದರೆ ಇದೇ ಆಗಿದೆ.

ಧರ್ಮವನ್ನು ಸಾಕ್ಷಾತ್ಕಾರ ಮಾಡಿಕೊಳ್ಳುವುದೊಂದೇ ಮಾರ್ಗ. ಪ್ರತಿಯೊಬ್ಬರೂ ಅದನ್ನು ಕಂಡುಹಿಡಿಯಬೇಕು. ಹಾಗಾದರೆ ಧರ್ಮಗ್ರಂಥಗಳಿಂದ ಪ್ರಯೋಜನವೇನು? ಅದರಿಂದ ಬಹಳ ಪ್ರಯೋಜನವಿದೆ. ಅದು ಒಂದು ದೇಶದ ಭೂಪಟದಂತೆ. ನಾನು ಇಲ್ಲಿಗೆ ಬರುವುದಕ್ಕೆ ಮುಂಚೆ ಇಂಗ್ಲೆಂಡಿನ ಭೂಪಟವನ್ನು ನೋಡಿದ್ದೆ. ಅದರಿಂದ ಇಂಗ್ಲೆಂಡ್ ಹೇಗೆ ಇದೆ ಎಂದು ಕಲ್ಪಿಸಿಕೊಳ್ಳಲು ಬಹಳ ಸಹಾಯವಾಯಿತು. ಆದರೂ ನಾನು ಈ ದೇಶಕ್ಕೆ ಬಂದಮೇಲೆ ಆ ಭೂಪಟಕ್ಕೂ ನಿಜವಾದ ಈ ದೇಶಕ್ಕೂ ಎಷ್ಟೊಂದು ವ್ಯತ್ಯಾಸ! ಇದರಂತೆಯೇ ಸಾಕ್ಷಾತ್ಕಾರಕ್ಕೆ ಮತ್ತು ಶಾಸ್ತ್ರಗಳಿಗೆ ಇರುವ ವ್ಯತ್ಯಾಸ ಕೂಡ. ಈ ಗ್ರಂಥಗಳು ನಕ್ಷೆಯಂತೆ. ಹಿಂದಿನವರ ಅನುಭವ ಇವುಗಳಲ್ಲಿದೆ. ಅವರಂತೆ, ಇಲ್ಲವೆ ಅವರಿಗಿಂತ ಉತ್ತಮವಾಗಿ ಸತ್ಯವನ್ನು ಕಂಡುಹಿಡಿಯುವಂತೆ ಪ್ರೇರೇಪಿಸುವುದಕ್ಕೆ ದೊಡ್ಡ ಕ್ರಿಯೋತ್ತೇಜಕ ಶಕ್ತಿಗಳು ಇವು.

ವೇದಾಂತದ ಪ್ರಥಮ ಸಿದ್ದಾಂತವೆ ಇದು: ಸಾಕ್ಷಾತ್ಕಾರವೇ ಧರ್ಮ. ಯಾರು ಧರ್ಮವನ್ನು ಸಾಕ್ಷಾತ್ಕಾರ ಮಾಡಿಕೊಳ್ಳುವರೊ ಅವರು ಧಾರ್ಮಿಕರು, ಯಾರಿಗೆ ಸಾಕ್ಷಾತ್ಕಾರವಾಗಿಲ್ಲವೋ ಅವರು "ನನಗೆ ಇದು ಗೊತ್ತಿಲ್ಲ” ಎಂದು ಹೇಳುವವರಿಗಿಂತ ಏನೂ ಮೇಲಲ್ಲ, ಕೆಲವು ವೇಳೆ ಕೀಳು. ಏಕೆಂದರೆ ತಿಳಿಯದವನು ಸತ್ಯವಂತ, ತನಗೆ ಗೊತ್ತಿಲ್ಲದುದನ್ನು ಗೊತ್ತಿಲ್ಲ ಎಂದು ಹೇಳುತ್ತಾನೆ. ಈ ಗ್ರಂಥಗಳು ನಮ್ಮ ಸಾಕ್ಷಾತ್ಕಾರಕ್ಕೆ ಬಹಳ ಸಹಾಯ ಮಾಡುವುವು, ಅವು ಕೇವಲ ನಮ್ಮ ಮಾರ್ಗದರ್ಶಕಗಳಂತೆ ಮಾತ್ರ ಅಲ್ಲ; ನಾವು ಏನು ಮಾಡಬೇಕು, ಹೇಗೆ ಮಾಡಬೇಕು ಎಂಬುದನ್ನೂ ತೋರುವುವು. ಏಕೆಂದರೆ ಪ್ರತಿಯೊಂದು ವಿಜ್ಞಾನದಲ್ಲಿಯೂ ಒಂದು ಅನ್ವೇಷಣಾ ಮಾರ್ಗವಿದೆ. ಅನೇಕ ಜನರು ಹೀಗೆ ಹೇಳುವರು: “ನಾನು ಧಾರ್ಮಿಕನಾಗಬೇಕೆಂದು ಇದ್ದೆ, ನಾನು ಇವುಗಳನ್ನು ಸಾಕ್ಷಾತ್ಕಾರ ಮಾಡಿಕೊಳ್ಳಬೇಕೆಂದು ಇದ್ದೆ. ಆದರೆ ನನಗೆ ಸಾಧ್ಯವಾಗಲಿಲ್ಲ. ಆದಕಾರಣವೇ ನಾನು ಏನನ್ನೂ ನಂಬುವುದಿಲ್ಲ. ವಿದ್ಯಾವಂತರಲ್ಲಿಯೂ ಅನೇಕ ಜನರು ಹೀಗೆ ಹೇಳುವವರು ಇರುವರು. ಬಹಳ ಮಂದಿ ಹೀಗೂ ಹೇಳುವರು, 'ನಾನು ಇಡೀ ಜೀವನದಲ್ಲಿ ಧಾರ್ಮಿಕನಾಗಬೇಕೆಂದು ಯತ್ನಿಸಿದೆ, ಆದರೆ ಅದರಲ್ಲಿ ಏನೂ ಇಲ್ಲ.” ಆದರೆ ಈ ವಿಷಯವನ್ನೂ ನೀವು ಗಮನಿಸುತ್ತೀರಿ; ಒಬ್ಬ ರಸಾಯನಶಾಸ್ತ್ರಜ್ಞ, ದೊಡ್ಡ ವಿಜ್ಞಾನಿ ಇರುವನು ಎಂದು ಭಾವಿಸಿ, ಅವನ ಬಳಿಗೆ ನೀವು ಹೋಗಿ “ನಾನು ರಸಾಯನಶಾಸ್ತ್ರದಲ್ಲಿ ಏನನ್ನೂ ನಂಬುವುದಿಲ್ಲ, ಏಕೆಂದರೆ ನಾನು ಇದುವರೆಗೆ ರಸಾಯನಶಾಸ್ತ್ರಜ್ಞನಾಗಬೇಕೆಂದು ಪ್ರಯತ್ನಿಸಿದೆ, ಸಾಧ್ಯವಾಗಲಿಲ್ಲ; ಅದರಲ್ಲಿ ಏನೂ ಇಲ್ಲ” ಎಂದರೆ, ಆ ವಿಜ್ಞಾನಿ "ನೀನು ಯಾವಾಗ ಪ್ರಯತ್ನಿಸಿದೆ" ಎನ್ನುವನು. “ ನಾನು ಮಲಗಲು ಹೋಗುವಾಗ ಓ ರಸಾಯನಶಾಸ್ತ್ರವೆ, ಬಾ ನನ್ನ ಬಳಿಗೆ ಎಂದೆ. ಆದರೂ ಅದು ಬರಲಿಲ್ಲ” ಎನ್ನುವಿರಿ. ಹೀಗೆಯೇ ಧರ್ಮದಲ್ಲಿ ಕೂಡ. ರಸಾಯನಶಾಸ್ತ್ರಜ್ಞ ನಕ್ಕು ಹೀಗೆ ಹೇಳುತ್ತಾನೆ: “ಓ ಅದಲ್ಲ ಮಾರ್ಗ, ನೀನು ಪ್ರಯೋಗ ಶಾಲೆಗೆ ಹೋಗಿ, ಆಸಿಡ್, ಆಲ್ಕಲಿ ಇವುಗಳೊಡನೆ ಪ್ರಯೋಗಮಾಡು. ನಿನ್ನ ಕೈಯನ್ನು ಏಕೆ ಮಧ್ಯೆ ಮಧೈ ಸುಟ್ಟುಕೊಳ್ಳಲಿಲ್ಲ? ಅದೇ ತಾನಾಗಿ ನಿನಗೆ ಕಲಿಸುತ್ತಿತ್ತು.” ಧರ್ಮಕ್ಕೂ ಇಷ್ಟು ಶ್ರಮಪಡುತ್ತೀರಾ? ಪ್ರತಿಯೊಂದು ವಿಜ್ಞಾನ ಶಾಸ್ತ್ರವನ್ನು ಕಲಿಯಬೇಕಾದರೂ ಒಂದು ಮಾರ್ಗವಿದೆ. ಧರ್ಮವನ್ನು ಕೂಡ ಇದರಂತೆಯೇ ಅಭ್ಯಾಸ ಮಾಡಬೇಕು. ಇದರದೇ ಬೇರೊಂದು ಮಾರ್ಗವಿದೆ. ಇಲ್ಲಿ ಎಲ್ಲರೂ ಹಿಂದಿನ ಆಚಾರ್ಯರಿಂದ, ಅನುಭವಗಳಿಂದ, ಸಾಕ್ಷಾತ್ಕಾರವನ್ನು ಪಡೆದವರಿಂದ ಏನನ್ನಾದರೂ ಕಲಿತುಕೊಳ್ಳಬಹುದು, ಕಲಿತುಕೊಳ್ಳಲೇಬೇಕು. ಅವರು ನಮಗೊಂದು ಮಾರ್ಗವನ್ನು ಸೂಚಿಸುವರು. ಯಾವ ಮಾರ್ಗದ ಮೂಲಕ ಮಾತ್ರ ನಾವು ಸತ್ಯವನ್ನು ಸಾಕ್ಷಾತ್ಕಾರ ಮಾಡಿಕೊಳ್ಳಬಹುದೊ ಆ ಮಾರ್ಗವನ್ನು ತೋರುವರು. ಅವರು ಇಡೀ ಜೀವನ ಸಾಧನೆಮಾಡಿ ಮನಸ್ಸನ್ನು ಹೇಗೆ ನಿಗ್ರಹಿಸಬೇಕು ಎಂಬ ಮಾರ್ಗವನ್ನು ಕಂಡುಹಿಡಿದರು. ಮನಸ್ಸನ್ನು ಅತಿ ಸೂಕ್ಷ್ಮವಸ್ತುವನ್ನು ಗ್ರಹಿಸಬಲ್ಲ ಒಂದು ಸ್ಥಿತಿಗೆ ತಂದು, ಅದರ ಮೂಲಕ ಧಾರ್ಮಿಕ ತತ್ತ್ವಗಳನ್ನು ಮನಗಂಡರು. ನಾವೂ ಧಾರ್ಮಿಕರಾಗಬೇಕಾದರೆ, ಧರ್ಮವನ್ನು ನೋಡಬೇಕಾದರೆ, ಅದನ್ನು ಅನುಭವಿಸಬೇಕಾದರೆ, ಮಹಾತ್ಮರಾಗಬೇಕಾದರೆ ಈ ದಾರಿಯಲ್ಲಿ ಸಾಧನೆ ಮಾಡಬೇಕು. ಆಗ ನಮಗೆ ಏನೂ ತೋರದೆ ಇದ್ದರೆ ಆಗ 'ಧರ್ಮದಲ್ಲಿ ಏನೂ ಹುರುಳಿಲ್ಲ. ಏಕೆಂದರೆ ನಾನು ಇದನ್ನೆಲ್ಲ ಪರೀಕ್ಷೆ ಮಾಡಿ ನೋಡಿರುವೆನು” ಎನ್ನಬಹುದು.

ಎಲ್ಲ ಧರ್ಮಗಳ ಅನುಷ್ಠಾನದ ಭಾಗವೆ ಇದು. ಜಗತ್ತಿನ ಧರ್ಮಗ್ರಂಥಗಳಲ್ಲೆಲ್ಲಾ ನೀವು ಇದನ್ನು ನೋಡುತ್ತೀರಿ. ಅವು ಸಿದ್ಧಾಂತಗಳನ್ನು ಮತ್ತು ತತ್ತ್ವಗಳನ್ನು ಮಾತ್ರ ಬೋಧಿಸುವುದಿಲ್ಲ. ಮಹಾತ್ಮರ ಜೀವನದಲ್ಲಿ ಇವುಗಳ ಅನುಷ್ಠಾನವನ್ನೂ ನೋಡುತ್ತೇವೆ. ಏನೇನು ಮಾಡಬೇಕೋ ಅದನ್ನೆಲ್ಲ ಶಾಸ್ತ್ರದಲ್ಲಿ ಹೇಳದೆ ಇದ್ದರೂ, ಮಹಾತ್ಮರ ಜೀವನದಲ್ಲಿ ಕೆಲವು ವೇಳೆ ಅವರ ಆಹಾರ ಪಾನೀಯಗಳ ವಿಷಯದಲ್ಲಿ ಕೂಡ ವಿಧಿನಿಷೇಧಗಳಿದ್ದುವು ಎಂಬುದನ್ನು ನೋಡುತ್ತೇವೆ. ಅವರ ಇಡೀ ಬಾಳು, ಅವರ ಆಚಾರ ವ್ಯವಹಾರ ಮಾತುಕತೆ ಇವುಗಳೆಲ್ಲ ಅವರ ಸುತ್ತಲಿದ್ದ ಮನುಷ್ಯರಿಗಿಂತ ಬೇರೆಯಾಗಿದ್ದುವು. ಇದರಿಂದಲೇ ಅವರಿಗೆ ಜ್ಞಾನಜ್ಯೋತಿ, ಭಗವತ್ ದರ್ಶನ ಪ್ರಾಪ್ತವಾದುವು. ನಮಗೂ ಈ ಅನುಭವ ಬೇಕಾದರೆ ನಾವೂ ಈ ಮಾರ್ಗವನ್ನು ಅನುಸರಿಸಲು ಸಿದ್ದರಾಗಬೇಕು. ಸಾಧನೆಯಿಂದಲೇ ನಾವು ಇದನ್ನು ಪಡೆಯಬೇಕು. ವೇದಾಂತದ ಯೋಜನೆ ಇದು: ಮೊದಲು ಸಿದ್ದಾಂತವನ್ನು ನಿರೂಪಿಸುವುದು, ಗುರಿಯನ್ನು ನಿರ್ದೇಶಿಸುವುದು, ಅನಂತರ ಆ ಗುರಿಯನ್ನು ಸೇರುವುದು ಹೇಗೆ, ತಿಳಿದುಕೊಳ್ಳುವುದು ಹೇಗೆ ಪಡೆಯುವುದು ಹೇಗೆ ಎಂಬುದನ್ನು ವಿವರಿಸುವುದು.

ಹಲವು ಮಾರ್ಗಗಳು ಇರಬೇಕಾಗುವುವು. ನಮ್ಮ ಸ್ವಭಾವದಲ್ಲಿ ಎಷ್ಟೋ ವೈವಿಧ್ಯಗಳು ಇರುವುದರಿಂದ ಯಾವ ಒಂದು ಮಾರ್ಗವನ್ನೂ ಇಬ್ಬರಿಗೆ ಒಂದೇ ರೀತಿ ನಾವು ಕೊಡುವುದಕ್ಕೆ ಆಗುವುದಿಲ್ಲ. ಒಬ್ಬೊಬ್ಬರ ಮನಸ್ಸು ಒಂದೊಂದು ಬಗೆಯಾಗಿರುವುದು. ಆದಕಾರಣವೆ ಮಾರ್ಗವೂ ಬೇರೆಯಾಗಬೇಕಾಗಿರುವುದು. ಕೆಲವರು ತುಂಬಾ ಭಾವಜೀವಿಗಳು, ಮತ್ತೆ ಕೆಲವರು ವಿಚಾರಪರರು. ಕೆಲವರಿಗೆ ಎಲ್ಲಾ ಬಾಹ್ಯ ಆಚಾರಗಳೂ ಬೇಕು, ಸ್ಥೂಲವಾಗಿರುವುದರ ಆಸರೆ ಬೇಕು. ಮತ್ತೆ ಕೆಲವರಿಗೆ ಯಾವ ಆಚಾರವೂ, ಆಕಾರವೂ ಬೇಕಿಲ್ಲ. ಇವೆಲ್ಲ ಅವರಿಗೆ ಮೃತ್ಯುಸಮಾನ. ಮತ್ತೊಬ್ಬ ಮೈಮೇಲೆಲ್ಲ ಯಂತ್ರಗಳನ್ನು ಕಟ್ಟಿಕೊಂಡಿರುವನು. ಅವನಿಗೆ ಅವನ್ನು ಕಂಡರೆ ಬಹಳ ಆಸೆ. ಮತೊಬ್ಬ ಭಾವಜೀವಿಯಾದವನು ಎಲ್ಲರಿಗೂ ಒಳ್ಳೆಯದನ್ನು ಮಾಡಲಿಚ್ಚಿಸುವನು. ಅವನು ಅಳುವನು, ಪ್ರೀತಿಸುವನು, ಕುಣಿದಾಡುವನು. ನಿಜವಾಗಿ ಇವರಿಗೆಲ್ಲ ಒಂದೇ ಮಾರ್ಗವಿರಲಾರದು. ಸತ್ಯಕ್ಕೆ ಬರುವುದಕ್ಕೆ ಒಂದೇ ಮಾರ್ಗವಿದ್ದರೆ, ಯಾರಿಗೆ ಈ ಮಾರ್ಗ ಹೊಂದಿಕೊಳ್ಳುವುದಿಲ್ಲವೋ ಅವರಿಗೆಲ್ಲ ಇದರಿಂದ ಏನೂ ಪ್ರಯೋಜನವಿಲ್ಲ. ಆದಕಾರಣವೇ ಬೇರೆ ಬೇರೆ ಮಾರ್ಗಗಳು ಇವೆ. ನಿಮಗೆ ಯಾವುದು ಬೇಕೊ ಅದನ್ನು ತೆಗೆದುಕೊಳ್ಳಿ. ಒಂದು ನಿಮಗೆ ಸರಿಯಾಗದೆ ಇದ್ದರೆ ಮತ್ತೊಂದು ಸರಿಯಾಗಬಹುದು. ಈ ದೃಷ್ಟಿಯಿಂದ ನೋಡಿದಾಗ ಪ್ರಪಂಚದಲ್ಲಿ ಅಷ್ಟೊಂದು ಧರ್ಮಗಳು ಇರುವುದು ಎಷ್ಟೋ ಮೇಲು ಎಂಬುದು ಗೊತ್ತಾಗುವುದು. ಅನೇಕರು ಇಚ್ಚಿಸುವಂತೆ ಒಬ್ಬನೇ ಗುರು ಇರದೆ, ಹಲವು ಗುರುಗಳು ಮತ್ತು ಮಹಾತ್ಮರು ಇರುವುದರಿಂದ ಎಷ್ಟು ಪ್ರಯೋಜನ ಎನ್ನುವುದು ಗೊತ್ತಾಗುವುದು. ಮಹಮ್ಮದೀಯರು ಪ್ರಪಂಚದಲ್ಲಿರುವವರೆಲ್ಲ ಮಹಮ್ಮದೀಯರಾಗಬೇಕೆಂದು ಬಯಸುವರು, ಕ್ರೈಸ್ತರು ಎಲ್ಲರೂ ಕೈಸ್ತರಾಗಬೇಕೆನ್ನುವರು. ಬೌದ್ದರು ಎಲ್ಲರೂ ಬೌದ್ಧರಾಗಬೇಕೆನ್ನುವರು. ಆದರೆ ವೇದಾಂತಿ ಹೇಳುವುದು ಇದು: “ಸಾಧ್ಯವಾದರೆ ಪ್ರತಿಯೊಬ್ಬರೂ ಬೇರೆ ಬೇರೆ ಆಗಿರಲಿ, ಆದರೆ ಅವರ ಹಿಂದೆ ಸಾಮಾನ್ಯವಾದ ಒಂದು ನಿಯಮದ ಏಕತೆ ಇರುವುದು. ಹೆಚ್ಚು ಮಹಾತ್ಮರಿದ್ದಷ್ಟೂ, ಪುಸ್ತಕಗಳಿದ್ದಷ್ಟೂ, ಋಷಿಗಳಿದ್ದಷ್ಟೂ, ಮಾರ್ಗಗಳಿದ್ದಷ್ಟೂ ಸಮಾಜಕ್ಕೆ ಮೇಲು.” ಸಾಮಾಜಿಕ ಜೀವನದಲ್ಲಿ ಹೆಚ್ಚು ಔದ್ಯೋಗಿಕ ಅನುಕೂಲಗಳಿದ್ದಷ್ಟೂ ಆ ಸಮಾಜಕ್ಕೆ ಮೇಲು. ಅಲ್ಲಿರುವವರಿಗೆಲ್ಲ ಜೀವನೋಪಾಯಕ್ಕೆ ಮಾರ್ಗವಿರುವುದು. ಇದರಂತೆಯೇ ಭಾವನಾ ಪ್ರಪಂಚದಲ್ಲಿ ಮತ್ತು ಧಾರ್ಮಿಕ ಪ್ರಪಂಚದಲ್ಲಿ ಕೂಡ. ಹಲವು ಬಗೆಯ ವಿಜ್ಞಾನಗಳು ನಮ್ಮೆದುರಿಗೆ ಇರುವ ಈಗಿನ ಕಾಲ ಎಷ್ಟೋ ಮೇಲಾಗಿರುವುದು. ಇಷ್ಟೊಂದು ವೈವಿಧ್ಯ ಇರುವಾಗ ಅವರ ಮನೋವಿಕಾಸಕ್ಕೆ ಎಷ್ಟೊಂದು ಅನುಕೂಲವಿದೆ! ನಮಗೆ ಇಚ್ಛೆ ಬಂದುದನ್ನು, ನಮಗೆ ಶ್ರೇಷ್ಠವೆಂದು ತೋರಿದುದನ್ನು ನಾವು ತೆಗೆದುಕೊಳ್ಳಬಹುದು. ಇದರಂತೆಯೇ ಧಾರ್ಮಿಕ ಪ್ರಪಂಚದಲ್ಲಿ ಕೂಡ. ಪ್ರಪಂಚದಲ್ಲಿ ಇಷ್ಟೊಂದು ಧರ್ಮಗಳಿರುವುದು ಭಗವಂತನ ಅದ್ಭುತ ಔದಾರ್ಯವನ್ನು ತೋರುವುದು. ದೇವರ ದಯೆಯಿಂದ ಪ್ರಪಂಚದಲ್ಲಿ ಪ್ರತಿಯೊಬ್ಬರಿಗೂ ಒಂದೊಂದು ಧರ್ಮವಿರುವವರೆಗೆ ಧರ್ಮಗಳು ಹೆಚ್ಚುತ್ತಾ ಹೋಗಲಿ.

ವೇದಾಂತ ಇದನ್ನು ಮನಗಂಡು ಒಂದು ತತ್ತ್ವವನ್ನು ಸಾರಿ ಹಲವು ಮಾರ್ಗಗಳನ್ನು ಹೇಳುವುದು. ಅದು ಯಾರನ್ನೂ ದೂರುವುದಿಲ್ಲ. ಕ್ರೈಸ್ತರಾಗಲಿ, ಬೌದ್ಧರಾಗಲಿ, ಹಿಂದೂಗಳಾಗಲಿ ಚಿಂತೆಯಿಲ್ಲ. ನೀವು ಯಾವ ಪುರಾಣವನ್ನು ಬೇಕಾದರೂ ನಂಬಬಹುದು. ನೀವು ಕ್ರಿಸ್ತ ಮಹಮದ್, ಹಿಂದೂ ಅಥವಾ ಪ್ರಪಂಚದ ಮತ್ತಾವ ಮಹಾತ್ಮರನ್ನಾದರೂ ನಂಬಬಹುದು. ನೀವೇ ಬೇಕಾದರೂ ಮಹಾತ್ಮರಾಗಿರಬಹುದು. ಅದು ಯಾರನ್ನೂ ದೂರುವುದಿಲ್ಲ. ಅದು ಕೇವಲ ತತ್ತ್ವವನ್ನು ಮಾತ್ರ ಬೋಧಿಸುವುದು. ಇದೇ ಎಲ್ಲಾ ಧರ್ಮದ ಹಿನ್ನೆಲೆ, ಎಲ್ಲಾ ಮಹಾತ್ಮರು ಪ್ರವಾದಿಗಳು ಋಷಿಗಳು ಈ ಒಂದು ತತ್ತ್ವ ಕ್ಕೆ ಉದಾಹರಣೆ, ಈ ಒಂದು ತತ್ವ ದ ಆವಿರ್ಭಾವ. ನಿಮ್ಮ ಮಹಾತ್ಮರನ್ನು ಎಷ್ಟು ಬೇಕಾದರೂ ಹೆಚ್ಚಿಸಿಕೊಂಡು ಹೋಗಿ. ಇದಕ್ಕೇನೂ ಅಭ್ಯಂತರವಿಲ್ಲ. ವೇದಾಂತ ಕೇವಲ ತತ್ತ್ವವನ್ನು ಮಾತ್ರ ನಿಮಗೆ ಬೋಧಿಸಿ, ನೀವು ಯಾವ ಮಾರ್ಗವನ್ನು ಬೇಕಾದರೂ ಆರಿಸಿಕೊಳ್ಳಲು ಸ್ವಾತಂತ್ರ್ಯವನ್ನು ಕೊಡುವುದು. ಯಾವ ದಾರಿಯಾದರೂ ಹಿಡಿಯಿರಿ, ಯಾವ ಮಹಾತ್ಮನನ್ನಾದರೂ ಅನುಸರಿಸಿರಿ, ನಿಮ್ಮ ಸ್ವಭಾವಕ್ಕೆ ಅನುಗುಣವಾದ ಮಾರ್ಗವನ್ನು ಮಾತ್ರ ಹಿಡಿಯಿರಿ. ನೀವು ಅದರಿಂದ ಮುಂದುವರಿಯುವುದರಲ್ಲಿ ಸಂದೇಹವಿಲ್ಲ.

