
\chapter[ಮುಕ್ತಾತ್ಮ]{ಮುಕ್ತಾತ್ಮ\protect\footnote{\engfoot{C.W, Vol. III, P. 6}}}

\begin{center}
(೧೮೯೬ರಲ್ಲಿ ನ್ಯೂಯಾರ್ಕಿನಲ್ಲಿ ನೀಡಿದ ಪ್ರವಚನ.)
\end{center}

ಸಾಂಖ್ಯದ ವಿಶ್ಲೇಷಣೆ ಪ್ರಕೃತಿ–ಪುರುಷರೆಂಬ ದ್ವೈತದಲ್ಲಿ ಕೊನೆಗಾಣುವುದು.\break ಅಲ್ಲಿ ಅನಂತ ಜೀವಕೋಟಿಗಳಿವೆ. ಅವನ್ನು ಇನ್ನು ಮುಂದಕ್ಕೆ ವಿಭಜಿಸಲು ಆಗದೆ ಇರುವುದರಿಂದ ಅವು ನಾಶವಾಗಲಾರವು. ಅವು ಪ್ರಕೃತಿಯಿಂದ ಯಾವಾಗಲೂ ಬೇರೆ ಇರುವುವು. ಪ್ರಕೃತಿಯೇ ಬದಲಾವಣೆ ಹೊಂದಿ ಈ ಜಗದ್ರೂಪವನ್ನು ಅಭಿವ್ಯಕ್ತಿಗೊಳಿಸುವುದು. ಸಾಂಖ್ಯರ ದೃಷ್ಟಿಯಲ್ಲಿ ಪುರುಷ ನಿಷ್ಕ್ರಿಯ ಮತ್ತು ಕೇವಲ. ಪುರುಷನ ಮುಕ್ತಿಗಾಗಿ ಪ್ರಕೃತಿ ಈ ಸೃಷ್ಟಿಯನ್ನು ನಿರ್ಮಿಸಿದೆ. ಪುರುಷ ತಾನು ಪ್ರಕೃತಿಯಲ್ಲ ಎಂದು ತಿಳಿಯುವುದೇ ಮುಕ್ತಿ. ಸಾಂಖ್ಯರು, ಪ್ರತಿಯೊಂದು ಪುರುಷನೂ ಸರ್ವವ್ಯಾಪಿ ಎಂಬುದನ್ನು ಒಪ್ಪಲೇಬೇಕಾದುದನ್ನು ನೋಡುವೆವು. ಪುರುಷ ಕೇವಲವಾಗಿರುವುದರಿಂದ (\enginline{simple}) ಅದಕ್ಕೆ ಒಂದು\break ಮಿತಿ ಇಲ್ಲ. ಮಿತಿ ಬರುವುದು ದೇಶ–ಕಾಲ–ನಿಮಿತ್ತಗಳ ಮೂಲಕ. ಪುರುಷ ಇವುಗಳ ಆಚೆ ಇರುವುದರಿಂದ ಅವನಿಗೆ ಯಾವ ಮಿತಿಯೂ ಇಲ್ಲ. ಮಿತವಾಗಬೇಕಾದರೆ ಅದೊಂದು ದೇಶದಲ್ಲಿರಬೇಕು. ಅಂದರೆ ಒಂದು ದೇಹದಲ್ಲಿರಬೇಕು. ದೇಹವಾದುದು ಪ್ರಕೃತಿಯಲ್ಲಿರಬೇಕಾಗುವುದು. ಪುರುಷನಿಗೆ ಒಂದು ಆಕಾರವಿದ್ದರೆ ಅದು ಪ್ರಕೃತಿಯಲ್ಲಿ ಒಂದಾಗಬೇಕು. ಆದಕಾರಣ ಪುರುಷನು ಆಕಾರರಹಿತ. ಆಕಾರವಿಲ್ಲದುದನ್ನು ಅದು ಇಲ್ಲಿದೆ–ಅಲ್ಲಿದೆ ಎಂದು ಹೇಳಲಾರೆವು. ಅದು ಸರ್ವವ್ಯಾಪಿಯಾಗಬೇಕಾಯಿತು. ಸಾಂಖ್ಯ ತತ್ತ್ವ ಇದರಾಚೆ\break ಹೋಗುವುದಿಲ್ಲ.

ಇದಕ್ಕೆ ವಿರೋಧವಾಗಿ ವೇದಾಂತಿಗಳ ಪ್ರಥಮ ವಾದವೆ ಈ ವಿಶ್ಲೇಷಣೆ ಪೂರ್ಣವಾದುದಲ್ಲ ಎಂಬುದು. ಪ್ರಕೃತಿಯೊಂದು ನಿರಪೇಕ್ಷವಾಗಿ, ಪುರುಷನೂ ಒಂದು ನಿರಪೇಕ್ಷವಾದರೆ, ಎರಡು ನಿರಪೇಕ್ಷವಾದವುಗಳು ಇರಬೇಕಾಗುವುದು. ಪುರುಷನು ಸರ್ವವ್ಯಾಪಿ ಎಂದು ತೋರುವ ವಾದಸರಣಿಯೇ ಪ್ರಕೃತಿಗೆ ಅನ್ವಯಿಸುವುದು. ಆಗ ಪ್ರಕೃತಿಯ ಎಲ್ಲಾ ಕಾಲ–ದೇಶ–ನಿಮಿತ್ತಗಳ ಆಚೆ ಹೋಗುವುದು. ಇದರ ಪರಿಣಾಮವಾಗಿ ಯಾವ ಬದಲಾವಣೆಯಾಗಲಿ, ಅಭಿವ್ಯಕ್ತಿಯಾಗಲಿ ಇರಲಾರದು. ಆಗ ಎರಡು ನಿರಪೇಕ್ಷ ವಸ್ತುಗಳು ಇರುವ ಸಮಸ್ಯೆ ಬರುವುದು. ಇದು ಸಾಧ್ಯವೇ ಇಲ್ಲ. ವೇದಾಂತಿಗಳ ಪರಿಹಾರವೇನು? ಅವರ ಪರಿಹಾರವೆ, ಸಾಂಖ್ಯರು ಹೇಳುವಂತೆ ಮನಸ್ಸು ಆಲೋಚಿಸಬೇಕಾದರೆ, ಪ್ರಕೃತಿ ಕೆಲಸಮಾಡುವಂತೆ ಪ್ರೇರೇಪಿಸಬೇಕಾದರೆ ಒಂದು ಚೇತನ ಅದರ ಹಿಂದೆ ಇರಬೇಕು ಎಂಬುದು. ಏಕೆಂದರೆ ಪ್ರಕೃತಿ, ಸ್ಥೂಲವಸ್ತುವಿನಿಂದ ಹಿಡಿದು ಮಹತ್ತಿನವರೆಗೆ, ಎಲ್ಲಾ ಬದಲಾವಣೆಗಳ ಸ್ಥಿತಿಯಲ್ಲೂ ಜಡವಾದುದು. ವೇದಾಂತಿಗಳು ಈ ವಿಶ್ವದ ಹಿಂದೆ ಇರುವ ಚೇತನವನ್ನೇ ನಾವು ದೇವರು ಎನ್ನುವುದು ಎನ್ನುತ್ತಾರೆ. ಆದಕಾರಣ ಈ ಜಗತ್ತು ಅವನಿಂದ ಬೇರೆಯಾಗಿಲ್ಲ. ಈ ವಿಶ್ವವಾಗಿರುವವನೂ ಅವನೆ. ಅವನು ಕೇವಲ ನಿಮಿತ್ತಕಾರಣ ಮಾತ್ರವಲ್ಲ, ಉಪಾದಾನಕಾರಣವೂ ಆಗಿರುವನು. ಕಾರಣವು ಪರಿಣಾಮಕ್ಕಿಂತ ಬೇರೆಯಲ್ಲ. ಪರಿಣಾಮವೆ ಬೇರೊಂದು ರೀತಿಯಲ್ಲಿರುವ ಕಾರಣ. ನಾವು ಇದನ್ನು ಪ್ರತಿದಿನ ನೋಡುತ್ತಿರುವೆವು. ಈ ದೇವರೇ ಪ್ರಕೃತಿಗೆ ಕಾರಣ. ವೇದಾಂತಿಗಳಲ್ಲಿ, ಅವರು ಅದ್ವೈತಿಗಳಾಗಲಿ, ವಿಶಿಷ್ಟಾದ್ವೈತಿಗಳಾಗಲಿ, ದ್ವೈತಿಗಳಾಗಲಿ ಎಲ್ಲರೂ ಈ ನಿಲುವಿನ ಮೇಲೆ ನಿಲ್ಲುವರು: ದೇವರು ಪ್ರಪಂಚಕ್ಕೆ ನಿಮಿತ್ತಕಾರಣ ಮಾತ್ರವಲ್ಲ, ಉಪಾದಾನ ಕಾರಣವೂ ಆಗಿರುವನು; ಇರುವುದೆಲ್ಲ ಅವನೆ. ವೇದಾಂತದಲ್ಲಿ ಎರಡನೆಯ ಮೆಟ್ಟಿಲೆ, ಜೀವವೂ ಕೂಡ ಅವನ ಅಂಶ, ಆ ಅನಂತ ಜ್ಞಾನಾಗ್ನಿಯ ಒಂದು ಕಿಡಿ ಎಂಬುದು. “ಹೇಗೆ ಒಂದು ಅಗ್ನಿಕುಂಡದಿಂದ ಕೋಟ್ಯಂತರ ಕಿಡಿಗಳು ಹೊರಗೆ ಬರುವುವೋ ಹಾಗೆಯೇ ಆ ಸನಾತನ ವ್ಯಕ್ತಿಯಿಂದ ಆತ್ಮಗಳೆಲ್ಲ ಬಂದಿವೆ." ಇಲ್ಲಿಯವರೆಗೇನೊ ಸರಿಯಾಯಿತು. ಆದರೂ ಇದು ತೃಪ್ತಿಯನ್ನು ಕೊಡಲಾರದು. ಅನಂತದಲ್ಲಿ ಅಂಶ ಎಂದರೆ ಏನು ಅರ್ಥ? ಅನಂತ ಅವಿಭಾಜ್ಯ. ಅನಂತದಲ್ಲಿ ಭಾಗಗಳು ಇರಲಾರವು. ನಿರಪೇಕ್ಷವನ್ನು ನಾವು ವಿಭಜಿಸಲಾರೆವು. ಆದಕಾರಣ ಈ ಕಿಡಿಗಳೆಲ್ಲ ಅವನಿಂದ ಬಂದಿವೆ ಎಂದರೆ ಅರ್ಥವೇನು? ಅದ್ವೈತ ವೇದಾಂತಿಗಳು ಈ ಸಮಸ್ಯೆಯನ್ನು ಹೀಗೆ ಪರಿಹರಿಸುವರು: ನಿಜವಾಗಿಯೂ ಭಾಗವಿಲ್ಲ, ಆತ್ಮ ನಿಜವಾಗಿ ಬ್ರಹ್ಮನ ಅಂಶವಲ್ಲ, ಅದು ಅನಂತಬ್ರಹ್ಮವೇ ಆಗಿರುವುದು. ಹಾಗಾದರೆ ಇಷ್ಟೊಂದು ಆತ್ಮಗಳು ಹೇಗೆ ಇರಬಲ್ಲವು? ಕೋಟ್ಯಂತರ ಹಿಮಮಣಿಗಳಲ್ಲಿ ಪ್ರತಿಬಿಂಬಿಸುವ ಸೂರ್ಯನು ಕೋಟ್ಯಂತರ ಸೂರ್ಯರಂತೆಯೇ ಕಾಣುವನು. ಪ್ರತಿಯೊಂದು ಹಿಮಮಣಿಯಲ್ಲಿಯೂ ಸಣ್ಣ ಸೂರ್ಯನಿರುವನು. ಹಾಗೆಯೇ ಈ ಆತ್ಮಗಳೆಲ್ಲ ಪ್ರತಿಬಿಂಬಗಳೇ ಹೊರತು ನಿಜವಲ್ಲ. ಅವು ನಿಜವಾದ “ನಾನು'' ಅಲ್ಲ. ನಿಜವಾದ “ನಾನು'' ಈ ವಿಶ್ವದ ಈಶ್ವರ, ಈ ಜೀವದ ಅವಿಭಕ್ತ ಸತ್ಯ. ಈ ಎಲ್ಲ ಮಾನವರು, ಪ್ರಾಣಿಗಳು ಮುಂತಾದ ಸಣ್ಣ ಸಣ್ಣ ಜೀವಿಗಳೆಲ್ಲ ಪ್ರತಿಬಿಂಬಗಳೇ ಹೊರತು, ಸತ್ಯವಲ್ಲ. ಪ್ರಕೃತಿಯ ಮೇಲೆ ಕಾಣುವ ಭ್ರಾಂತಿಯ ಪ್ರತಿಬಿಂಬಗಳು. ವಿಶ್ವದಲ್ಲಿ ಒಬ್ಬ ಅನಂತಾತ್ಮನಿರುವನು, ಅವನೇ ನನ್ನಂತೆ ನಿಮ್ಮಂತೆ ಕಾಣುತ್ತಿರುವುದು. ಈ ತೋರಿಕೆಯ ವೈವಿಧ್ಯ ಭ್ರಾಂತಿಕಲ್ಪಿತ ಮಾತ್ರ. ಅವನು ನಿಜವಾಗಿಯೂ ಭಾಗವಾಗಿಲ್ಲ, ಭಾಗವಾದಂತೆ ಕಾಣುವನು ಅಷ್ಟೆ. ಈ ಭ್ರಾಂತಿ ರೂಪವಾದ ವೈವಿಧ್ಯಕ್ಕೆ ಕಾರಣ ಅವನನ್ನು ನಾವು ಕಾಲ–ದೇಶ–ನಿಮಿತ್ತಗಳ ಮೂಲಕ ನೋಡುವುದು. ಕಾಲ–ದೇಶ–ನಿಮಿತ್ತಗಳ ಮೂಲಕ ನಾವು ದೇವರನ್ನು ನೋಡಿದಾಗ ಅವನು ಸ್ಥೂಲ ಜಗತ್ತಿನಂತೆ ಕಾಣುವನು. ನಾವು ಅವನನ್ನು ಮತ್ತೂ ಸ್ವಲ್ಪ ಮೇಲಿನ ಸ್ತರದಲ್ಲಿ, ಆದರೂ ಅದೇ ಜಾಲದ ಮೂಲಕ ನೋಡಿದಾಗ ಅವನು ಪ್ರಾಣಿಯಂತೆ ಕಾಣುವನು, ಮತ್ತೂ ಸ್ವಲ್ಪ ಮೇಲಿನಿಂದ ನೋಡಿದಾಗ ಮಾನವನಂತೆ, ಅದಕ್ಕಿಂತಲೂ ಮೇಲಿನಿಂದ ನೋಡಿದಾಗ ದೇವನಂತೆ ಕಾಣುವನು. ಆದರೆ ಅವನೊಬ್ಬನೇ ಜಗತ್ತಿನಲ್ಲೆಲ್ಲ ವಿಭುವಾದವನು. ನಾವೇ ಅವನು. ನಾನೂ ಅದೇ, ನೀವೂ ಅದೇ. ಅದರ ಅಂಶವಲ್ಲ, ಅದರ ಪೂರ್ಣವೇ ನಾವು. “ಈ ಪ್ರಕೃತಿಯ ಹಿಂದೆ ನಿತ್ಯ ಸಾಕ್ಷಿಯಂತೆ ಇರುವವನು ಅವನೆ. ಅವನೇ ಪ್ರಕೃತಿಯೂ ಕೂಡ ಆಗಿರುವವನು.” ಅವನೇ ಜ್ಞಾತೃ ಮತ್ತು ಜ್ಞೇಯವಾಗಿರುವನು. ಅವನೇ `ನಾನು' ಮತ್ತು `ನೀನು'. ಇದು ಹೇಗೆ ಸಾಧ್ಯ? ಈ ಜ್ಞಾತೃವನ್ನು ಅರಿಯುವುದು ಹೇಗೆ? ಜ್ಞಾತೃ ತನ್ನನ್ನು ತಾನೇ ಅರಿಯಲಾರನು. ನಾನು ಎಲ್ಲವನ್ನೂ ನೋಡುತ್ತೇನೆ. ಆದರೆ ನನ್ನನ್ನು ನಾನು ನೋಡಲಾರೆ. ಆತ್ಮ ಜ್ಞಾತೃ; ಸರ್ವೇಶ್ವರನಾದ,\break ಸತ್ಯನಾದ ಅವನೇ ಪ್ರಪಂಚದ ದೃಶ್ಯಗಳಿಗೆಲ್ಲ ಕಾರಣ. ಆದರೆ ಪ್ರತಿಬಿಂಬದ ವಿನಃ ಅವನು ತನ್ನನ್ನು ತಾನೇ ನೋಡಲಾರ, ಅರಿತುಕೊಳ್ಳಲಾರ. ನೀವು ಕನ್ನಡಿಯಲ್ಲಿ ಅಲ್ಲದೆ ಬೇರೆ ಎಲ್ಲೂ ನಿಮ್ಮ ಮುಖವನ್ನು ನೋಡಿಕೊಳ್ಳಲಾರಿರಿ. ಆತ್ಮನೂ ಕೂಡ ಅದು ಪ್ರತಿಬಿಂಬಿತವಾಗದೆ ಇದ್ದರೆ ತನ್ನನ್ನು ತಾನು ಅರಿಯಲಾರದು. ಆದಕಾರಣ ಈ ವಿಶ್ವವೆಲ್ಲ ತನ್ನನ್ನು ತಾನು ಅರಿಯಲು ಪ್ರಯತ್ನಿಸುತ್ತಿರುವ ಆತ್ಮ. ಈ ಪ್ರತಿಬಿಂಬ ಮೊದಲು ಪ್ರೋಟೋಪ್ಲಾಸಂನಲ್ಲಿ ಕಾಣುವುದು. ಅನಂತರ ಗಿಡ ಮರ ಪ್ರಾಣಿಗಳು ಇವುಗಳಲ್ಲಿ. ಹೀಗೆ ಕೊನೆಗೆ ಪೂರ್ಣಾತ್ಮನೆಂಬ ಶ್ರೇಷ್ಠ ಪ್ರತಿಬಿಂಬವನ್ನು ಮುಟ್ಟುವವರೆಗೆ ಇದು ಮುಂದುವರಿಯುತ್ತಾ ಉತ್ತಮ, ಉತ್ತಮತರ ಪ್ರತಿಬಿಂಬಗಳನ್ನು ಹೊರಗೆಡಹುತ್ತ ಹೋಗುವುದು. ಒಬ್ಬನು ತನ್ನ ಪ್ರತಿಬಿಂಬವನ್ನು ಮೊದಲು ಬಗ್ಗಡದ ನೀರಿರುವ ಗುಂಡಿಯಲ್ಲಿ ನೋಡಿದಂತೆ ಇರುವುದು. ಅಲ್ಲಿ ತನ್ನ ಕೇವಲ ರೂಪರೇಖೆ ಮಾತ್ರ ಕಾಣುವುದು. ಅನಂತರ ಶುಭ್ರವಾದ ನೀರಿಗೆ ಬಂದರೆ ಅಲ್ಲಿ ಶುಭ್ರವಾದ ಪ್ರತಿಬಿಂಬವನ್ನು ಕಾಣುವನು. ಅನಂತರ ಹೊಳೆಯುತ್ತಿರುವ ಲೋಹದಲ್ಲಿ ನೋಡಿದಾಗ ಅವನ ಪ್ರತಿಬಿಂಬ ಇನ್ನೂ ಸ್ಪಷ್ಟವಾಗುವುದು. ಕೊನೆಯಲ್ಲಿ ಕನ್ನಡಿಯಲ್ಲಿ ನೋಡಿದಾಗ ತಾನು ನಿಜವಾಗಿ ಹೇಗೆ ಇರುವನೋ ಹಾಗೆ ಕಾಣುವನು. ಆದ ಕಾರಣ ಪೂರ್ಣಾತ್ಮನು, ಜ್ಞಾತೃಜ್ಞೇಯಗಳೆರಡೂ ಆದ ಪರಬ್ರಹ್ಮನ ಶ್ರೇಷ್ಠ ಆವಿರ್ಭಾವ. ಆದಕಾರಣವೇ ಮಾನವ ಏತಕ್ಕೆ ಎಲ್ಲವನ್ನೂ ಸ್ವಭಾವತಃ ಪೂಜಿಸುತ್ತಾನೆ ಎಂಬುದು ಗೊತ್ತಾಗುತ್ತದೆ. ಈ ಕಾರಣದಿಂದಲೇ ಪ್ರತಿಯೊಂದು ದೇಶದಲ್ಲಿಯೂ ಪೂರ್ಣಾತ್ಮರನ್ನು ದೇವರೆಂದು ಸ್ವಭಾವತಃ ಗೌರವಿಸುವರು. ನಿಮ್ಮ ಮನಸ್ಸಿಗೆ ತೋರಿದಂತೆ ನೀವು ಮಾತನಾಡಬಹುದು. ಆದರೆ ಅಂತಹ ವ್ಯಕ್ತಿಗಳನ್ನು ಮಾತ್ರ ಜನರು ಆರಾಧಿಸುವರು. ಆದಕಾರಣವೆ ಮನುಷ್ಯರು ಕ್ರಿಸ್ತ–ಬುದ್ಧರಂತಹ ಅವತಾರಗಳನ್ನು ಪೂಜಿಸುತ್ತಾರೆ. ಸನಾತನವಾಗಿರುವ ಆತ್ಮನ ಪರಿಪೂರ್ಣ ಆವಿರ್ಭಾವಗಳು ಇವರು. ನಾವು ನೀವು ಊಹಿಸಿಕೊಳ್ಳುವ ಭಗವಂತನ ಭಾವನೆಗಿಂತ ಅವರು ಎಷ್ಟೋ ಮೇಲು. ಇಂತಹ ಭಾವನೆಗಳಿಗಿಂತ ಪರಿಪೂರ್ಣಾತ್ಮ ಎಷ್ಟೋ ಮೇಲು. ಅವನಲ್ಲಿ ವೃತ್ತ ಪೂರ್ಣವಾಗುವುದು, ಜ್ಞಾತೃಜ್ಞೇಯಗಳೆರಡೂ ಒಂದಾಗುವುವು. ಅವನಲ್ಲಿ ಭ್ರಾಂತಿಯೆಲ್ಲ ಹೋಗಿ, ಅದರ ಬದಲು ತಾನು ಯಾವಾಗಲೂ ಆ ಪೂರ್ಣಾತ್ಮನೇ ಆಗಿದ್ದೆನೆಂಬ ಭಾವ ಇರುವುದು. ಹಾಗಾದರೆ ಈ ಬಂಧನ ಹೇಗೆ ಬಂತು? ಈ ಪೂರ್ಣಾತ್ಮ ಅಪೂರ್ಣ ಹೇಗೆ ಆದ? ಈ ಮುಕ್ತಾತ್ಮ ಬಂಧಿತನಾಗಲು ಹೇಗೆ ಸಾಧ್ಯವಾಯಿತು? ಅವನೆಂದಿಗೂ ಬಂಧಿತನಾಗಿರಲಿಲ್ಲ, ಯಾವಾಗಲೂ ಮುಕ್ತನೇ ಆಗಿದ್ದ ಎಂದು ಅದ್ವೈತಿಗಳು ಹೇಳುವರು. ಹಲವು ಬಗೆಯ ಮೋಡಗಳು ಆಕಾಶದಲ್ಲಿ ಬರುವುವು. ಅಲ್ಲಿ ಒಂದು ಕ್ಷಣ ಇದ್ದು ಅನಂತರ ಹೊರಟುಹೋಗುವುವು. ಗಗನ ಎಂದಿನಂತೆಯೇ ಇದ್ದ ಅಸೀಮ ನೀಲಿ ಆಕಾಶವೆ. ಆಕಾಶ ಎಂದಿಗೂ ಬದಲಾಗುವುದಿಲ್ಲ. ಬದಲಾಗುವುದು ಮೋಡಗಳು. ಆದಕಾರಣ ನೀವು ಯಾವಾಗಲೂ ಪರಿಶುದ್ಧರು,\break ನಿತ್ಯಶುದ್ಧರು. ಯಾವುದೂ ನಿಮ್ಮ ಸ್ವಭಾವವನ್ನು ಬದಲಾಯಿಸದು, ಮುಂದೆ ಎಂದಿಗೂ ಹಾಗೆ ಬದಲಾಯಿಸಲಾರದು. ನಾನು ಅಪೂರ್ಣ, ನಾನೊಬ್ಬ ಗಂಡಸು ಅಥವಾ ಹೆಂಗಸು ಅಥವಾ ನಾನೊಬ್ಬ ಪಾಪಿ ಎಂಬುದು ನನಗೆ ಒಂದು ಮನಸ್ಸಿದೆ, ನನಗೆ ಆಲೋಚನೆ ಇದೆ, ನಾನು ಆಲೋಚಿಸುತ್ತೇನೆ ಎಂಬುವುಗಳೆಲ್ಲ ಭ್ರಾಂತಿ. ನೀವು ಎಂದೂ ಆಲೋಚಿಸುವುದಿಲ್ಲ, ನಿಮಗೆ ಎಂದಿಗೂ ದೇಹವಿರಲಿಲ್ಲ, ನೀವು ಎಂದಿಗೂ ಅಪೂರ್ಣರಾಗಿರಲಿಲ್ಲ. ನೀವೇ ಪುಣ್ಯಾತ್ಮರಾದ ಪ್ರಪಂಚದ ಒಡೆಯರು, ಹಿಂದೆ ಇದ್ದ, ಈಗ ಇರುವ, ಮುಂದೆ ಇರಬಲ್ಲ, ಸರ್ವಕ್ಕೂ ಏಕಮಾತ್ರ ಸರ್ವಶಕ್ತನಾದ ಈಶ್ವರ; ಈ ಸೂರ್ಯ ನಕ್ಷತ್ರಗಳು, ಚಂದ್ರಪೃಥ್ವಿಗ್ರಹಗಳು ಮುಂತಾದ ವಿಶ್ವದ ಪ್ರತಿಯೊಂದು ಚೂರಿಗೂ ಅಪ್ರತಿಮ ಪರಾಕ್ರಮಶಾಲಿಯಾದ ಸಾಮ್ರಾಟನೆ ನೀವು. ಸೂರ್ಯ ಬೆಳಗುವುದು ನಿಮ್ಮಿಂದ, ತಾರೆಗಳು ಹೊಳೆಯುವುದು ನಿಮ್ಮಿಂದ, ವಸುಂಧರೆ ಸುಂದರವಾಗಿರುವುದು ನಿಮ್ಮಿಂದ. ನಿಮ್ಮ ಧನ್ಯತೆಯಿಂದ ಅವರೆಲ್ಲ ಪ್ರೀತಿಸುವುದು, ಒಬ್ಬರು ಮತ್ತೊಬ್ಬರಿಂದ ಆಕರ್ಷಿತರಾಗುವುದು. ನೀವು ಎಲ್ಲರಲ್ಲಿಯೂ ಇರುವಿರಿ, ನೀವೆ ಎಲ್ಲವೂ ಆಗಿರುವಿರಿ. ಸ್ವೀಕರಿಸುವುದು ಯಾರನ್ನು, ಬಹಿಷ್ಕರಿಸುವುದು ಯಾರನ್ನು? ನೀವೇ ಎಲ್ಲದರಲ್ಲಿರುವ ಸರ್ವವಾಗಿರುವಿರಿ. ಈ ಜ್ಞಾನ ಬಂದೊಡನೆ ಭ್ರಾಂತಿ ತಕ್ಷಣ ಅಳಿಯುವುದು.

ನಾನೊಮ್ಮೆ ಭರತಖಂಡದ ಮರುಳುಕಾಡಿನಲ್ಲಿ ಸಂಚರಿಸುತ್ತಿದ್ದೆ. ನಾನು ಒಂದು ತಿಂಗಳಿಗಿಂತ ಹೆಚ್ಚು ಸಂಚರಿಸುತ್ತಿದ್ದೆ. ಯಾವಾಗಲೂ ನನ್ನ ಮುಂದೆ ಸುಂದರವಾದ ಪ್ರಕೃತಿ ದೃಶ್ಯ ಕಾಣುತ್ತಿತ್ತು. ಸುಂದರವಾದ ಸರೋವರಗಳು ಇದ್ದವು. ಒಂದು ದಿನ ನನಗೆ ಬಹಳ ಬಾಯಾರಿಕೆ ಆಗಿ ಆ ಸರೋವರ ಒಂದರಲ್ಲಿ ನೀರು ಕುಡಿಯಬೇಕೆಂದು ಬಯಸಿದೆ. ಆದರೆ ನಾನು ಆ ಸರೋವರದ ಸಮೀಪಕ್ಕೆ ಬಂದೊಡನೆಯೆ ಅದು ಮಾಯವಾಯಿತು. ತಕ್ಷಣವೆ ನನಗೆ ಹೊಳೆಯಿತು, ಇದೊಂದು ಮರೀಚಿಕೆ ಎಂದು. ಯಾವುದರ ವಿಷಯವಾಗಿ ನಾನು ಇದುವರೆಗೆ ಪುಸ್ತಕದಲ್ಲಿ ಓದಿದ್ದೆನೋ ಅದೇ ಇದು ಎಂದು ಗೊತ್ತಾಯಿತು. ಇದನ್ನು ಜ್ಞಾಪಿಸಿಕೊಂಡು ನನ್ನ ಮೌಢ್ಯಕ್ಕೆ ನಾನೇ ನಕ್ಕೆ. ನಾನು ಕಳೆದ ತಿಂಗಳೆಲ್ಲ ನೋಡಿದ ಸುಂದರ ಪ್ರಕೃತಿ ದೃಶ್ಯಗಳು, ಸರೋವರಗಳು ಇವೆಲ್ಲ ಈ ಮರೀಚಿಕೆ ಆಗಿದ್ದುವು. ಆದರೆ ಆಗ ಇವೆಲ್ಲ ಭ್ರಾಂತಿ ಎಂದು ಗೊತ್ತಿರಲಿಲ್ಲ. ಮಾರನೆಯ ದಿನ ಎಂದಿನಂತೆ ಹೊರಟೆ. ಪುನಃ ಸುಂದರ ದೃಶ್ಯಗಳು ಮತ್ತು ಸರೋವರಗಳು ಕಂಡವು. ಆದರೆ ಅವುಗಳೊಡನೆ ಇವೆಲ್ಲ ಮರೀಚಿಕೆ ಎಂಬ ಭಾವನೆಯೂ ತಕ್ಷಣ ಹೊಳೆಯಿತು. ಒಮ್ಮೆ ಅದನ್ನು ಅರಿತ ಮೇಲೆ ಪುನಃ ಅದು ನಮ್ಮನ್ನು ಭ್ರಾಂತರನ್ನಾಗಿ ಮಾಡಲಾರದು. ಈ ವಿಶ್ವದ ಭ್ರಾಂತಿ ಒಂದು ದಿನ ಮಾಯವಾಗುವುದು. ಇದೆಲ್ಲ ಮಾಯವಾಗಿ ಕರಗಿಹೋಗುವುದು. ಇದೇ ಸಾಕ್ಷಾತ್ಕಾರ. ತತ್ತ್ವ ಎಂಬುದು ತಮಾಷೆಯಲ್ಲ, ಬರಿಯ ಮಾತಲ್ಲ. ಅದನ್ನು ಸಾಕ್ಷಾತ್ಕಾರ ಮಾಡಿಕೊಳ್ಳಬೇಕಾಗಿದೆ. ಈ ದೇಹ ಮಾಯವಾಗುವುದು, ಈ ಪೃಥ್ವಿ ಮತ್ತು ಎಲ್ಲವೂ ಮಾಯವಾಗುವುದು. ನಾನು ದೇಹ, ನನಗೊಂದು ಮನಸ್ಸಿದೆ ಎಂಬ ಭಾವನೆಯೂ ಕೆಲವು ಕಾಲದ ಮಟ್ಟಿಗೆ ಹೋಗುವುದು. ಕರ್ಮವೇನಾದರೂ ಸಂಪೂರ್ಣ ಕ್ಷಯವಾಗಿದ್ದರೆ ದೇಹದ ಮತ್ತು ಮನಸ್ಸಿನ ಭಾವನೆ ಸಂಪೂರ್ಣ ತೊಲಗಿಯೇ ಹೋಗುವುದು. ಆದರೆ ಕರ್ಮವೇನಾದರೂ ಇನ್ನು ಸ್ವಲ್ಪ ಉಳಿದಿದ್ದರೆ–ಕುಂಬಾರನ ಚಕ್ರ ಮಡಕೆಯನ್ನು ಮಾಡಿ ಆದ ಮೇಲೂ ಹಿಂದಿನವೇಗದಿಂದ ಚಲಿಸುತ್ತಿರುವಂತೆ – ಈ ದೇಹಕೂಡ ಅದರಂತೆಯೇ ಭ್ರಾಂತಿ ಅಳಿದ ಮೇಲೂ ಕೆಲವು ಕಾಲ ಇರುವುದು. ಆಗ ಪುನಃ ಈ ಪ್ರಪಂಚ, ಸ್ತ್ರೀ ಪುರುಷರು ಪ್ರಾಣಿಗಳಲ್ಲಿ ಆ ಮರೀಚಿಕೆ ಬಂದಂತೆ ಬರುವುವು. ಆದರೆ ಅಷ್ಟೇ ಪ್ರಬಲವಾಗಿ ಬರಲಾರವು. ಅವುಗಳೊಡನೆ ನನಗೆ ಈಗ ಅವುಗಳ ಸ್ವಭಾವ ಅರಿವಾಗಿದೆ ಎಂಬ ಭಾವವೂ ಬರುವುದು. ಇನ್ನು ಮೇಲೆ ಅವು ನಮ್ಮ ಬಂಧನಕ್ಕೆ, ದುಃಖಕ್ಕೆ ಅಥವಾ ವ್ಯಥೆಗೆ ಕಾರಣವಾಗುವುದಿಲ್ಲ. ಯಾವಾಗಲಾದರೂ ವ್ಯಥೆ ಬಂದರೆ “ನೀನೊಂದು ಭ್ರಾಂತಿ ಎಂದು ಗೊತ್ತಿದೆ'' ಎಂದು ಮನಸ್ಸು ಹೇಳಬಲ್ಲದು. ಮನುಷ್ಯ ಈ ಸ್ಥಿತಿಗೆ ಬಂದಾಗ ಜೀವನ್ಮುಕ್ತ. ನಿಸ್ಸಂಗನಾಗಿ ಪ್ರಪಂಚದಲ್ಲಿ ಬಾಳಬಲ್ಲಾತನೆ ಜೀವನ್ಮುಕ್ತ. ಅವನು ನೀರಿನಲ್ಲಿರುವ ಪದ್ಮಪತ್ರದಂತೆ, ನೀರಿನಲ್ಲಿದ್ದರೂ ಅದರಿಂದ ಒದ್ದೆಯಾಗುವುದಿಲ್ಲ. ಅವನೇ ಮಾನವ ಶ್ರೇಷ್ಠ. ಅಷ್ಟೆ ಅಲ್ಲ ಸೃಷ್ಟಿಯ ಎಲ್ಲ ಜೀವಿಗಳಿಗಿಂತಲೂ ಶ್ರೇಷ್ಠನಾದವನು. ಏಕೆಂದರೆ ಅವನು ಪರಬ್ರಹ್ಮನೊಂದಿಗೆ ತಾದಾತ್ಮ್ಯಭಾವವನ್ನು ಹೊಂದಿರುವನು. ತಾನೆ ಅವನು ಎಂಬುದನ್ನು ಅರಿತಿರುವನು. ಎಲ್ಲಿಯವರೆಗೆ ನೀವು ದೇವರಿಗೂ ನಿಮಗೂ ಸ್ವಲ್ಪವಾದರೂ ವ್ಯತ್ಯಾಸವಿದೆ ಎಂದು ತಿಳಿದಿರುವಿರೋ ಅಲ್ಲಿಯವರೆಗೆ ನೀವು ಭಯಗ್ರಸ್ತರಾಗಬೇಕಾಗಿದೆ. ನೀವೇ ಅವನಿಗೂ ವ್ಯತ್ಯಾಸವಿಲ್ಲ, ಎಳ್ಳಷ್ಟೂ ವ್ಯತ್ಯಾಸವಿಲ್ಲವೆಂದು ತಿಳಿದಾಗ ನೀವೆ ಅವನು, ಪೂರ್ಣ ಅವನೆ, ಸಂಪೂರ್ಣ ಅವನೆ–ಎಂದು ತಿಳಿದಾಗ, ಅಂಜಿಕೆಗಳೆಲ್ಲ ಓಡಿಹೋಗುವುವು. “ಅಲ್ಲಿ ಯಾರು ಯಾರನ್ನು ನೋಡುವರು? ಯಾರು ಯಾರನ್ನು ಪೂಜಿಸುವರು? ಯಾರು ಯಾರೊಂದಿಗೆ ಮಾತನಾಡುವರು? ಯಾರು ಯಾರನ್ನು ಕೇಳುವರು? ಎಲ್ಲಿ ಒಬ್ಬರು ಇನ್ನೊಬ್ಬರನ್ನು ನೋಡುತ್ತಾರೋ, ಒಬ್ಬರು ಇನ್ನೊಬ್ಬರೊಡನೆ ಮಾತನಾಡುತ್ತಾರೆಯೋ, ಒಬ್ಬರು ಇನ್ನೊಬ್ಬರನ್ನು ಕೇಳುತ್ತಾರೊ ಅದೆಲ್ಲ ಅಲ್ಪ. ಎಲ್ಲಿ ಒಬ್ಬರು ಇನ್ನೊಬ್ಬರನ್ನು ನೋಡುವುದಿಲ್ಲವೊ, ಒಬ್ಬರು ಇನ್ನೊಬ್ಬರೊಡನೆ ಮಾತನಾಡುವುದಿಲ್ಲವೊ, ಅದೇ ಮಹತ್, ಅದೇ ಭೂಮ, ಅದೇ ಬ್ರಹ್ಮ.” ನೀನು ಸರ್ವಕಾಲದಲ್ಲಿಯೂ ಅದೇ ಆಗಿರುವೆ. ಹಾಗಾದರೆ ಆಗ ಜಗತ್ತು ಏನಾಗುವುದು? ನಾವು ಜಗತ್ತಿಗೆ ಏನು ಒಳ್ಳೆಯದನ್ನು ಮಾಡಬಲ್ಲೆವು? ಇಂತಹ ಪ್ರಶ್ನೆಗಳೇ ಏಳುವುದಿಲ್ಲ. ನಾನು ವೃದ್ಧನಾದರೆ ಮಿಠಾಯಿ ಏನಾಗಬೇಕು ಎಂದು ಕೇಳುವುದು ಮಗು. ನಾನು ದೊಡ್ಡವನಾದರೆ ಗೋಲಿ ಗತಿ ಏನಾಗಬೇಕು, ಅದಕ್ಕೆಯೇ ನಾನು ದೊಡ್ಡವನೇ ಆಗುವುದಿಲ್ಲ ಎನ್ನುವುದು ಮಗು. ಪ್ರಪಂಚಕ್ಕೆ ಸಂಬಂಧಿಸಿದ ಪ್ರಶ್ನೆಯು ಇಂತಹುದೇ. ಈ ಪ್ರಪಂಚ ಹಿಂದೆಯಾಗಲಿ ಮುಂದೆಯಾಗಲಿ ಈಗಾಗಲಿ ಇಲ್ಲ. ಆತ್ಮನ ನೈಜ ಸ್ಥಿತಿಯನ್ನು ಅರಿತರೆ, ಆತ್ಮನಲ್ಲದೆ ಬೇರಾವುದೂ ಇಲ್ಲ ಎಂದು ಅರಿತರೆ, ಉಳಿದವೆಲ್ಲ ಬರಿಯ ಕನಸು, ಸತ್ಯವಲ್ಲ ಎಂದು ತಿಳಿದರೆ, ಆ ದಾರಿದ್ರ್ಯ ದುಃಖ ದೌರ್ಜನ್ಯ ಇವುಗಳಿಂದ ಮತ್ತು ಒಳ್ಳೆಯದರಿಂದ ಕೂಡಿದ ಪ್ರಪಂಚ ನಮ್ಮ ಸ್ವಾಸ್ಥ್ಯಕ್ಕೆ ಭಂಗ ತರದು. ಅದು ಇಲ್ಲದೆ ಇದ್ದರೆ ಯಾರಿಗಾಗಿ ಏತಕ್ಕಾಗಿ ನಾವು ತೊಂದರೆ ತೆಗೆದುಕೊಳ್ಳಬೇಕು? ಜ್ಞಾನ ಯೋಗಿಗಳು ಬೋಧಿಸುವುದು ಇದನ್ನು. ಆದಕಾರಣವೆ ಮುಕ್ತನಾಗಲು ಧೈರ್ಯತಾಳು, ನಿನ್ನ ಆಲೋಚನೆ ಒಯ್ದೆಡೆಗೆ ಹೋಗಲು ಸಾಹಸಪಡು, ಅದನ್ನು ನಿನ್ನ ಜೀವನದಲ್ಲಿ ಅನುಷ್ಠಾನಕ್ಕೆ ತರಲು ಧೀರನಾಗು. ಜ್ಞಾನವನ್ನು ಪಡೆಯುವುದು ಬಹಳ ಕಷ್ಟ. ಯಾರು ತುಂಬಾ ಧೈರ್ಯಶಾಲಿಗಳೊ, ಯಾರು ಪ್ರಚಂಡ ಸಾಹಸಿಗಳೊ, ವಿಗ್ರಹಗಳನ್ನೆಲ್ಲ ಧ್ವಂಸಮಾಡಬಲ್ಲರೊ, ಯಾರು ಬೌದ್ದಿಕ ವಿಗ್ರಹಗಳನ್ನು ಮಾತ್ರ ಅಲ್ಲ, ವಿಷಯ ಪ್ರಪಂಚಕ್ಕೆ ಸೇರಿದ ವಿಗ್ರಹಗಳನ್ನು ಧ್ವಂಸಮಾಡಬಲ್ಲರೊ ಅಂತಹ ಸಾಹಸಿಗಳಿಗೆ ಮಾತ್ರ ಸಾಧ್ಯ. ಈ ದೇಹ ನಾನಲ್ಲ, ಇದು ಹೋಗಬೇಕು. ಇದರ ಪರಿಣಾಮವಾಗಿ ಹಲವು ವಿಚಿತ್ರ ಸಂಗತಿಗಳು ಬರಬಹುದು. `ನಾನು ದೇಹವಲ್ಲ ಆದಕಾರಣ ನನ್ನ ತಲೆನೋವು ಹೋಗಬೇಕು' ಎಂದು ಒಬ್ಬ ಹೇಳಬಹುದು. ಇವನಿಗೆ ದೇಹವೇ ಇಲ್ಲದೆ ಇದ್ದರೆ ತಲೆನೋವು ಎಲ್ಲಿಂದ ಬರಬೇಕು? ಸಾವಿರಾರು ತಲೆನೋವುಗಳು ಬಂದು ಹೋದರೇನಂತೆ, ಸಾವಿರಾರು ದೇಹಗಳು ಬಂದು ಹೋದರೇನಂತೆ? ಇದರಿಂದ ನನಗೇನು? ನನಗೆ ಜನನ ಮರಣಗಳಿಲ್ಲ. ನನಗೆ ಎಂದಿಗೂ ತಾಯಿ ತಂದೆಗಳು ಇರಲಿಲ್ಲ. ಶತ್ರು ಮಿತ್ರರು ನನಗೆ ಎಂದಿಗೂ ಇರಲಿಲ್ಲ. ನನಗೆ ನಾನೇ ಶತ್ರು, ನನಗೆ ನಾನೇ ಮಿತ್ರ. ನಾನು ಅಖಂಡ ಸಚ್ಚಿದಾನಂದ. ನಾನೆ ಅವನು. ಸಾವಿರಾರು ದೇಹಗಳಲ್ಲಿ ನಾನು ದುಃಖ ಪಡುತ್ತಿದ್ದರೆ, ಇತರ ಸಾವಿರಾರು ದೇಹಗಳಲ್ಲಿ ನಾನು ಸುಖಪಡುತ್ತಿರುವೆನು. ಸಾವಿರ ದೇಹಗಳಲ್ಲಿ ನಾನು ಉಪವಾಸವಿದ್ದರೆ, ಸಾವಿರ ದೇಹಗಳಲ್ಲಿ ಭೂರಿ ಭೋಜನ ಮಾಡುತ್ತಿರುವೆನು. ಯಾರು ಯಾರನ್ನು ದೂರುವರು? ಯಾರು ಯಾರನ್ನು ಹೊಗಳುವರು? ಯಾರನ್ನು ಅರಸಬೇಕು, ಯಾರನ್ನು ಬಿಡಬೇಕು? ಯಾರನ್ನೂ ನಾನು ಹುಡುಕಿಕೊಂಡು ಹೋಗುವುದಿಲ್ಲ, ಯಾರು ಬಂದರೂ ಬೇಡವೆನ್ನುವುದಿಲ್ಲ. ಏಕೆಂದರೆ ನಾನೇ ವಿಶ್ವವೆಲ್ಲ ಆಗಿರುವೆನು. ನನ್ನನ್ನು ನಾನೇ ಹೊಗಳಿಕೊಳ್ಳುವೆನು, ನನ್ನನ್ನು ನಾನೇ ತೆಗಳಿಕೊಳ್ಳುವೆನು. ನನಗಾಗಿ ನಾನೇ ವ್ಯಥೆಪಡುವೆ, ನನಗಾಗಿ ನಾನೇ ಸಂತೋಷಪಡುವೆ. ನಾನು ನಿತ್ಯಮುಕ್ತ. ಇವನೇ ಜ್ಞಾನಿ, ಧೀರ, ಸಾಹಸಿ. ಇಡೀ ವಿಶ್ವ ಧೂಳಿಧೂಸರವಾದರೂ ಅವನು ನಗುತ್ತ ಪ್ರಪಂಚವೇ ಇಲ್ಲವೆನ್ನುವನು. ಇದೆಲ್ಲ ಒಂದು ಭ್ರಾಂತಿಯಾಗಿತ್ತು. ಅವನು ಪ್ರಪಂಚ ಪುಡಿಪುಡಿಯಾಗುವುದನ್ನು ನೋಡುವನು. ಅದೆಲ್ಲಿತ್ತು! ಎಲ್ಲಿಗೆ ಹೋಯಿತು!

ವ್ಯವಹಾರಯೋಗ್ಯ ವಿಚಾರಗಳಿಗೆ ಹೋಗುವುದಕ್ಕೆ ಮುಂಚೆ ಯುಕ್ತಿಗೆ ಸಂಬಂಧಪಟ್ಟ ಮತ್ತೊಂದು ಪ್ರಶ್ನೆಯನ್ನು ತೆಗೆದುಕೊಳ್ಳೋಣ. ಇಲ್ಲಿಯವರೆಗೆ ತರ್ಕ ಅದ್ಭುತವಾಗಿ ಯುಕ್ತಿ ಬದ್ದವಾಗಿದೆ. ಒಬ್ಬ ಯುಕ್ತಿಯ ಹಾದಿಯನ್ನು ಹಿಡಿದರೆ, ಇರುವುದೊಂದೆ, ಉಳಿದವುಗಳೆಲ್ಲ ಮಿಥ್ಯ ಎನ್ನುವ ನಿರ್ಧಾರಕ್ಕೆ ಬರುವವರೆಗೆ ಅವನು ಮಧ್ಯದಲ್ಲಿ ಎಲ್ಲಿಯೂ ನಿಲ್ಲಲಾರ. ಯುಕ್ತಿಯನ್ನು ಆಶ್ರಯಿಸುವ ಮಾನವನಿಗೆ ಇದಲ್ಲದೆ ಬೇರೆ ಮಾರ್ಗವೇ ಇಲ್ಲ. ಆದರೆ ಯಾವುದು ಅನಂತವಾದುದೊ ಪೂರ್ಣವಾದುದೊ ಧನ್ಯವಾದುದೋ ಅಂತಹ ಅಖಂಡ ಸಚ್ಚಿದಾನಂದವು ಇಂತಹ ಭ್ರಾಂತಿಗೆ ಏಕೆ ಬಂದಿತು? ಈ ಪ್ರಪಂಚದಲ್ಲೆಲ್ಲಾ ಕೇಳಿದ ಪ್ರಶ್ನೆಯೇ ಇದು. ಇದನ್ನು ಸ್ಥೂಲ ಭಾಷೆಯಲ್ಲಿ ಹೇಳಬೇಕಾದರೆ, ಈ ಪ್ರಶ್ನೆ, ಪಾಪ ಎಂಬುದು ಪ್ರಪಂಚಕ್ಕೆ ಹೇಗೆ ಬಂತು ಎಂದಾಗುವುದು. ಪ್ರಶ್ನೆಯ ಅತಿ ಹೀನವಾದ ಸ್ಥೂಲವಾದ ಭಾಗ ಇದು; ಅದರ ಮತ್ತೊಂದು ನಾಜೋಕಾದ ರೂಪವೆ ತಾತ್ವಿಕ ಭಾವ. ಆದರೆ ಉತ್ತರ ಎರಡಕ್ಕೂ ಒಂದೇ. ಇದೇ ಪ್ರಶ್ನೆಯನ್ನು ಹಲವು ರೀತಿಗಳಲ್ಲಿ ಹಲವು ಬಗೆಯಾಗಿ ಕೇಳಿರುವರು. ಆದರೆ ಈ ಪ್ರಶ್ನೆಯ ಸ್ಥೂಲ ಹಂತದಲ್ಲಿ ಇದಕ್ಕೆ ಉತ್ತರ ದೊರಕುವುದಿಲ್ಲ. ಏಕೆಂದರೆ ಹೆಣ್ಣು, ಸೇಬು, ಹಾವು ಮುಂತಾದ ಬೈಬಲ್ಲಿನಲ್ಲಿ ಬರುವ ಕಥೆಗಳು ತೃಪ್ತಿಯನ್ನು ಕೊಡಲಾರವು. ಈ ಸ್ಥಿತಿಯಲ್ಲಿ ಪ್ರಶ್ನೆ ಬಾಲಿಶವಾದುದು, ಅದರಂತೆಯೇ ಉತ್ತರ ಕೂಡ. ಆದರೆ ಪ್ರಶ್ನೆ ಈಗ ಬೃಹದಾಕಾರವನ್ನು ತಾಳಿದೆ. “ಈ ಭ್ರಾಂತಿ ಹೇಗೆ ಬಂತು?'' ಉತ್ತರವೂ ಚೆನ್ನಾಗಿಯೇ ಇದೆ. ಅದಕ್ಕೆ ಉತ್ತರವೇ, ಅಸಾಧ್ಯವಾದ ಪ್ರಶ್ನೆಗೆ ಉತ್ತರವನ್ನು ನಿರೀಕ್ಷಿಸಲಾರೆ ಎನ್ನುವುದು. ಈ ಪ್ರಶ್ನೆಯೇ ಅಸಾಧ್ಯ. ಈ ಪ್ರಶ್ನೆಯನ್ನು ಹಾಕುವುದಕ್ಕೆ ನಿಮಗೆ ಅಧಿಕಾರವಿಲ್ಲ. ಏತಕ್ಕೆ? ಪೂರ್ಣತೆ ಎಂದರೇನು? ಯಾವುದು ದೇಶ–ಕಾಲ–ನಿಮಿತ್ತಗಳಿಗೆ ಅತೀತವಾಗಿದೆಯೊ ಅದು ಮಾತ್ರ ಪರಿಪೂರ್ಣ. ಅನಂತರ ನೀವು ಪೂರ್ಣವಾದುದು ಅಪೂರ್ಣ ಹೇಗಾಯಿತು ಎಂದು ಕೇಳುವಿರಿ. ತರ್ಕದ ಭಾಷೆಯಲ್ಲಿ ನಾವು ಪ್ರಶ್ನೆಯನ್ನು ಹೀಗೆ ಕೇಳಬಹುದು: “ಯಾವುದು ಕಾರಣಕ್ಕೆ ನಿಲುಕದೊ ಅದು ಹೇಗೆ ಆಯಿತು?'' ನೀವೇ ಇದನ್ನು ವಿರೋಧಿಸುವಿರಿ. ಮೊದಲು ಇದನ್ನು ಕಾರ್ಯಕಾರಣ ನಿಯಮದೊಳಗೆ ಮಾತ್ರ ಹುಡುಕಬಹುದು, ಕಾಲ–ದೇಶ–ನಿಮಿತ್ತಗಳಿರುವ ಕಡೆ ಈ ಪ್ರಶ್ನೆಯನ್ನು ಹಾಕಬಹುದು. ಆದರೆ ಇದರಾಚೆ ಈ ಪ್ರಶ್ನೆಗೆ ಅರ್ಥವಿಲ್ಲ. ಏಕೆಂದರೆ ಪ್ರಶ್ನೆಯೇ ತರ್ಕಬದ್ಧವಾದುದಲ್ಲ. ಕಾಲ–ದೇಶ–ನಿಮಿತ್ತಗಳ ಎಲ್ಲೆಯೊಳಗೆ ಇರುವ ಪರಿಯಂತರ ನಾವು ಇದಕ್ಕೆ ಉತ್ತರ ಕೊಡಲಾರೆವು. ಇದರಾಚೆ ಪ್ರಶ್ನೆಗೆ ಯಾವ ಉತ್ತರವಿದೆಯೊ ಅದನ್ನು ನಾವು ಅಲ್ಲಿಗೆ ಹೋದಾಗ ಮಾತ್ರ ತಿಳಿಯಬಹುದು. ಆದ ಕಾರಣವೆ ಬುದ್ದಿವಂತರು ಪ್ರಶ್ನೆಗೆ ಉತ್ತರವನ್ನೇ ಕೊಡುವುದಿಲ್ಲ. ಮನುಷ್ಯನಿಗೆ ಖಾಯಿಲೆಯಾದಾಗ ಅದು ಹೇಗೆ ಆಯಿತು ಎಂದು ತಿಳಿಯುವುದಕ್ಕಿಂತ ಮೊದಲು ಅನಾರೋಗ್ಯದಿಂದ ಪಾರಾಗಲು ಯತ್ನಿಸುವನು.

ಇದಕ್ಕಿಂತ ಕೆಳಗೆ ಈ ಪ್ರಶ್ನೆಯ ಮತ್ತೊಂದು ಭಾಗವಿದೆ. ಆದರೆ ಅದು ಹೆಚ್ಚು ವ್ಯಾವಹಾರಿಕವಾದುದು ಮತ್ತು ಉದಾಹರಣೆಯಿಂದ ಕೂಡಿದುದು. ಈ ಭ್ರಾಂತಿಗೆ ಕಾರಣ ಯಾವುದು? ಯಾವುದಾದರೂ ನಿಜವಾಗಿರುವ ವಸ್ತು ಭ್ರಾಂತಿಯನ್ನು ಕಲಿಸಬಲ್ಲುದೆ? ಎಂದಿಗೂ ಇಲ್ಲ. ಒಂದು ಭ್ರಾಂತಿ ಮತ್ತೊಂದು ಭ್ರಾಂತಿಗೆ ಕಾರಣ. ಭ್ರಾಂತಿಯೆ\break ಯಾವಾಗಲೂ ಭ್ರಾಂತಿಗೆ ಕಾರಣ. ರೋಗ ರೋಗಕ್ಕೆ ಕಾರಣವೇ ಹೊರತು ಆರೋಗ್ಯವಲ್ಲ. ಅಲೆಯೇ ನೀರು, ಕಾರ್ಯವೇ ಕಾರಣದ ಬೇರೊಂದು ರೂಪ. ಕಾರ್ಯವೇ ಭ್ರಾಂತಿ. ಕಾರಣವೂ ಭ್ರಾಂತಿಯಾಗಿರಬೇಕು. ಈ ಭ್ರಾಂತಿಗೆ ಕಾರಣ ಯಾವುದು? ಮತ್ತೊಂದು ಭ್ರಾಂತಿ. ಹಾಗೆಯೆ ಕೊನೆಮೊದಲಿಲ್ಲದೆ ಹೋಗಬೇಕು. ನೀವು ಕೇಳಬೇಕಾದ ಮತ್ತೊಂದು ಪ್ರಶ್ನೆಯೇ ಇದು: ನಿಮ್ಮ ಅದ್ವೈತಕ್ಕೆ ಭಂಗ ಬರುವುದಿಲ್ಲವೆ ಎನ್ನುವುದು. ಏಕೆಂದರೆ ಇಲ್ಲಿ ಎರಡು ವಸ್ತುಗಳು ಇರುವುವು–ಒಂದು ನೀನು, ಮತ್ತೊಂದು ಭ್ರಾಂತಿ. ಇದಕ್ಕೆ ಉತ್ತರವೇ, ಭ್ರಾಂತಿ ಒಂದು ಅಸ್ತಿತ್ವವಲ್ಲ ಎಂಬುದು. ನಿಮ್ಮ ಜೀವನದಲ್ಲಿ ನೂರಾರು ಕನಸುಗಳಾಗಬಹುದು. ಅದಾವುದೂ ನಿಮ್ಮ ಭಾಗವಲ್ಲ. ಕನಸುಗಳು ಬಂದು ಹೋಗುವುವು. ಅವಕ್ಕೆ ಅಸ್ತಿತ್ವವಿಲ್ಲ. ಭ್ರಾಂತಿಯನ್ನು ಅಸ್ತಿತ್ವ ಎನ್ನುವುದು ಕುತರ್ಕ. ಆದಕಾರಣ ಪ್ರಪಂಚದಲ್ಲಿ ನಿತ್ಯಮುಕ್ತವಾದ, ನಿತ್ಯತೃಪ್ತವಾದ ಅಸ್ತಿತ್ವ ಒಂದೇ ಇರುವುದು. ನೀನೇ ಅದು. ಅದ್ವೈತ ಮುಟ್ಟುವ ಕೊನೆಯ ನಿರ್ಣಯ ಇದು.

ಹಾಗಾದರೆ ಹಲವು ಬಗೆಯ ಪೂಜೆಗಳ ಗತಿಯೇನಾಗುವುದು ಎಂದು ಕೇಳಬಹುದು. ಅವೆಲ್ಲ ಇರುವುವು, ಅವೆಲ್ಲ ಜ್ಞಾನಜ್ಯೋತಿಯನ್ನು ಹುಡುಕುವ ತಡಕಾಟ. ಹೀಗೆ\break ಹುಡುಕಾಡುವುದರಿಂದ ಬೆಳಕು ಬರುವುದು. ಆತ್ಮ ತನ್ನನ್ನೇ ನೋಡಲಾರದು ಎಂಬುದನ್ನು ನೋಡಿದೆವು. ನಮ್ಮ ಜ್ಞಾನವೆಲ್ಲ ಮಾಯಾವರಣದೊಳಗೆ ಇರುವುದು. ಅವುಗಳಾಚೆ\break ಸ್ವಾತಂತ್ರ್ಯವಿರುವುದು. ಈ ಮಾಯೆಯೊಳಗೆ ದಾಸ್ಯ, ಎಲ್ಲಾ ನಿಯಮದೊಳಗೆ ಬರುವುದು. ಇದರಾಚೆ ಯಾವ ನಿಯಮವೂ ಇಲ್ಲ. ಪ್ರಪಂಚದ ದೃಷ್ಟಿಯಿಂದ ನೋಡಿದರೆ, ಇದರಲ್ಲಿ ಇರುವುದೆಲ್ಲ ನಿಯಮಕ್ಕೆ ಬಾಗಿ ನಡೆಯುವುದು. ಇದರಾಚೆ ಸ್ವಾತಂತ್ರ್ಯವಿದೆ. ಕಾಲ–ದೇಶ–ನಿಮಿತ್ತಗಳ ವಲಯದೊಳಗೆ ಇರುವ ಪರಿಯಂತರ ನೀವು ಸ್ವತಂತ್ರರು ಎನ್ನುವುದು ಮೌಢ್ಯ. ಏಕೆಂದರೆ ಇಲ್ಲಿರುವುದೆಲ್ಲ ಕಾರ್ಯಕಾರಣಗಳ ಪರಿಣಾಮದ ವಜ್ರಮುಷ್ಠಿಯಲ್ಲಿರುವುದು. ನೀವು ಆಲೋಚಿಸುವ ಪ್ರತಿಯೊಂದು ಆಲೋಚನೆಯೂ ಉತ್ಪತ್ತಿಯಾಗಿದೆ. ನಿಮ್ಮ ಪ್ರತಿಯೊಂದು ಭಾವನೆಯೂ ಉತ್ಪತ್ತಿಯಾಗಿದೆ. ಇಚ್ಛೆ ಸ್ವತಂತ್ರ ಎನ್ನುವುದು ಅಸಂಬದ್ಧ. ಅನಂತ ವ್ಯಕ್ತಿತ್ವದ ಒಂದು ಅಂಶ ಮಾಯಾವರಣದೊಳಗೆ ಬಂದಂತೆ ಆದಾಗ, ಅದು ಇಚ್ಚೆಯ ಆಕಾರವನ್ನು ತಾಳುವುದು. ಮಾಯಾವರಣದೊಳಗೆ ಸಿಲುಕಿರುವ ಅಸ್ತಿತ್ವದ ಒಂದು ಅಂಶವೇ ಇಚ್ಛೆ. ಆದಕಾರಣ (\enginline{Free will}) ಇಚ್ಛಾ ಸ್ವಾತಂತ್ರ್ಯ ಎನ್ನುವುದಕ್ಕೆ ಅರ್ಥವಿಲ್ಲ, ಇದಕ್ಕೇನೂ ಬೆಲೆಯಿಲ್ಲ, ಬರಿಯ ಕುತರ್ಕ. ಸ್ವಾತಂತ್ರ್ಯಕ್ಕೆ ಸಂಬಂಧಪಟ್ಟ ಮಾತು ಕೂಡ ಹಾಗೆಯೇ. ಮಾಯೆಯಲ್ಲಿ ಸ್ವಾತಂತ್ರ್ಯವಿಲ್ಲ.

ಪ್ರತಿಯೊಬ್ಬರೂ ಒಂದು ಚೂರು ಕಲ್ಲಿನಂತೆ ಅಥವಾ ಎದುರಿಗೆ ಇರುವ ಮೇಜಿನಂತೆ ಕಾಯಾ ವಾಚಾ ಮನಸಾ ಬಂದಿಗಳು. ನಾನು ಈಗ ಮಾತನಾಡುತ್ತಿರುವುದು ನೀವು ಕೇಳುವಷ್ಟೇ ಕಾರ್ಯಕಾರಣ ನಿಯಮಕ್ಕೆ ಒಳಪಟ್ಟಿರುವುದು. ನೀವು ಮಾಯತೀತರಾಗುವವರೆಗೆ ಸ್ವಾತಂತ್ರ್ಯವಿಲ್ಲ. ಮಾಯಾತೀತರಾಗುವುದೇ ನಿಜವಾದ ಆತ್ಮಸ್ವಾತಂತ್ರ್ಯ. ಮನುಷ್ಯರು ಎಷ್ಟೇ ಕುಶಾಗ್ರಬುದ್ಧಿಯವರಾಗಲಿ, ಬುದ್ಧಿವಂತರಾಗಿರಲಿ, ಅವರು ಇಲ್ಲಿರುವುದಾವುದೂ ಸ್ವತಂತ್ರವಾಗಿರಲಾರದು ಎಂಬ ತರ್ಕಬದ್ದ ಸಿದ್ದಾಂತವನ್ನು ಒಪ್ಪಿಕೊಂಡರೂ, ತಾವು ಸ್ವತಂತ್ರರು ಎಂದು ಆಲೋಚಿಸುವಂತೆ ಅವರನ್ನು ಯಾವುದೋ ಬಲಾತ್ಕರಿಸುವುದು. ಹೀಗೆ ಆಲೋಚಿಸದೆ ವಿಧಿಯಿಲ್ಲ. ನಾವು ಸ್ವತಂತ್ರರು ಎಂದು ಯೋಚಿಸಿದಲ್ಲದೆ ಯಾವ ಕೆಲಸವೂ ಸಾಗುವಂತೆ ಇಲ್ಲ. ನಾವು ಹೇಳುವ ಸ್ವಾತಂತ್ರ್ಯ ಎಂಬುದು ಮೋಡಗಳ ಮೂಲಕ ಕಾಣುವ ನೀಲಾಕಾಶದ ಕ್ಷಣಿಕ ನೋಟ ಮಾತ್ರ. ಆ ನಿಜವಾದ ಸ್ವಾತಂತ್ರ್ಯ ಎಂಬ ನೀಲಿಯಾಕಾಶ ಅವುಗಳ ಹಿಂದೆ ಇದೆ. ನಿಜವಾದ ಸ್ವಾತಂತ್ರ್ಯ, ಈ ಭ್ರಾಂತಿಯ ಮಧ್ಯದಲ್ಲಿ, ಈ ಕಲ್ಪನೆಯ ಮಧ್ಯದಲ್ಲಿ, ಈ ನಿಷ್ಟ್ರಯೋಜಕ ಪ್ರಪಂಚದಲ್ಲಿ, ದೇಹೇಂದ್ರಿಯ ಮನಸ್ಸುಗಳಲ್ಲಿ ಇರಲಾರದು. ಆದಿ ಅಂತ್ಯಗಳಿಲ್ಲದ, ನಿಗ್ರಹಿಸದ, ನಿಗ್ರಹಕ್ಕೆ ಬಾರದ, ಸರಿಹೊಂದಿಕೊಳ್ಳಲಾರದ, ಚೂರುಚೂರಾದ, ಸಾಮರಸ್ಯವಿಲ್ಲದ ಕನಸುಗಳನ್ನೆ ಜಗತ್ತು ಎನ್ನುವುದು. ನಿಮ್ಮನ್ನು ಕನಸಿನಲ್ಲಿ ಇಪ್ಪತ್ತು ತಲೆಯ ರಾಕ್ಷಸನೊಬ್ಬನು ಅಟ್ಟಿಸಿಕೊಂಡು ಬರುವಾಗ ಅದು ನಿಜವಲ್ಲ ಎಂದು ಯೋಚಿಸುವುದಿಲ್ಲ; ನೀವು ಅವನಿಂದ ಓಡಿ ಹೋಗುವಿರಿ. ಅದೆಲ್ಲ ನಿಜ ಎಂದು ಭಾವಿಸುವಿರಿ. ಇದರಂತೆಯೇ ನಿಯಮ. ನೀವು ನಿಯಮವೆಂದು ಕರೆಯುವುದೆಲ್ಲ ಬರೀ ಅಕಸ್ಮಾತ್ತಾದುದು, ಅದಕ್ಕೆ ಅರ್ಥವಿಲ್ಲ. ಕನಸಿನಲ್ಲಿ ನೀವು ಇದನ್ನು ನಿಯಮ ಎಂದು ಕರೆಯುತ್ತೀರಿ. ಮಾಯೆಯೊಳಗೆ, ದೇಶ–ಕಾಲ–ನಿಮಿತ್ತಗಳು ಇರುವ ಪರಿಯಂತರ ಸ್ವಾತಂತ್ರ್ಯವಿಲ್ಲ. ಎಲ್ಲಾ ಬಗೆಯ ಉಪಾಸನೆಗಳೂ ಈ ಮಾಯಾಪ್ರಪಂಚದಲ್ಲಿವೆ. ದೇವರ ಭಾವನೆ, ಪ್ರಾಣಿಗಳ ಭಾವನೆ, ಮನುಷ್ಯನ ಭಾವನೆ, ಇವೆಲ್ಲ ಈ ಮಾಯಾವರಣದೊಳಗೆ ಇವೆ. ಆದಕಾರಣ ಇವುಗಳೆಲ್ಲ ಬರಿಯ ಭ್ರಾಂತಿ, ಬರಿಯ ಕನಸು. ಆದರೆ ಈಗಿನ\break ಕಾಲದಲ್ಲಿ ವಾದಮಾಡುವ ಕೆಲವು ಅಸಾಧಾರಣ ವ್ಯಕ್ತಿಗಳಂತೆ ನಾವು ಇರಕೂಡದು. ಅವರು ದೇವರ ಭಾವನೆಯನ್ನು ಬರಿಯ ಭ್ರಾಂತಿ ಎನ್ನುವರು; ಆದರೆ ಜಗತ್ತು ಅವರಿಗೆ ಸತ್ಯ. ಎರಡೂ ಒಂದೇ ತರ್ಕದಿಂದ ಉಳಿಯುವುವು, ಇಲ್ಲ ಅಳಿಯುವುವು. ದೇವರನ್ನು ಅಲ್ಲಗಳೆಯುವಂತೆ ಯಾರು ಜಗತ್ತನ್ನೂ ಅಲ್ಲಗಳೆಯುವನೊ ಅವನಿಗೆ ಮಾತ್ರ ನಾಸ್ತಿಕನಾಗಲು ಅಧಿಕಾರವಿದೆ. ಇಬ್ಬರಿಗೂ ಒಂದೇ ವಾದವಿದೆ. ದೇವರಿಂದ ಕ್ಷುದ್ರ ಕೀಟದವರೆಗೆ, ಒಂದು ಹುಲ್ಲಿನ ಎಸಳಿನಿಂದ ಸೃಷ್ಟಿಕರ್ತನವರೆಗೆ ಎಲ್ಲಾ ಕಡೆಗಳಲ್ಲಿಯೂ ಈ ಭ್ರಾಂತಿ ಇದೆ. ಇವೆಲ್ಲ ಒಂದೇ ತರ್ಕದಿಂದ ಉಳಿಯುವುವು, ಇಲ್ಲವೆ ಅಳಿಯುವುವು. ಭಗವಂತನ ಭಾವನೆ ತಪ್ಪು ಎಂದು ಭಾವಿಸುವವನು, ತನ್ನ ದೇಹವನ್ನು ಮತ್ತು ಮನಸ್ಸನ್ನು ಕೂಡ ಅದೇ ದೃಷ್ಟಿಯಿಂದ ನೋಡಬೇಕು. ದೇವರು ಮಾಯವಾದಾಗ ಅವನ ದೇಹ ಮತ್ತು ಮನಸ್ಸೂ ಕೂಡ ಮಾಯವಾಗುವುವು. ಇವೆರಡೂ ಮಾಯವಾದಾಗ ಯಾವುದು ನಿಜವಾಗಿ ಸತ್ಯವೊ ಅದು ಎಂದೆಂದಿಗೂ ಉಳಿಯುವುದು. “ಅದನ್ನು ಕಣ್ಣು ನೋಡಲಾರದು, ಮಾತು ವ್ಯಕ್ತಪಡಿಸಲಾರದು. ಮನಸ್ಸು ಕಲ್ಪಿಸಿಕೊಳ್ಳಲಾರದು; ನಾವು ಅದನ್ನು ನೋಡಲಾರೆವು ಅಥವಾ ಅರಿಯಲಾರೆವು.'' ಮಾತು, ಆಲೋಚನೆ, ಜ್ಞಾನ, ಬುದ್ದಿ ಇವೆಲ್ಲವೂ ಮಾಯೆಯಲ್ಲಿವೆ, ಬಂಧನದಲ್ಲಿವೆ ಎಂದು ಗೊತ್ತಾಗುವುದು. ಇದರಾಚೆಯೇ ಸತ್ಯವಿರುವುದು. ಅದನ್ನು ಮನಸ್ಸಾಗಲಿ, ಆಲೋಚನೆಯಾಗಲಿ, ಮಾತಾಗಲಿ ಮುಟ್ಟಲಾರದು.

ಇಲ್ಲಿಯವರೆಗೆ ಯುಕ್ತಿಯ ದೃಷ್ಟಿಯಿಂದ ಸರಿಯಾಯಿತು. ಅನಂತರ ಅನುಷ್ಠಾನ ಬರುವುದು. ಅನುಷ್ಠಾನವೇ ನಿಜವಾದ ಕಷ್ಟವಾಗಿರುವುದು. ಈ ಏಕತೆಯನ್ನು ಅರಿಯಬೇಕಾದರೆ ಏನಾದರೂ ಅಭ್ಯಾಸ ಆವಶ್ಯಕವೇ? ನಿಸ್ಸಂದೇಹವಾಗಿಯೂ ಆವಶ್ಯಕ. ನೀವು ಇನ್ನು ಮೇಲೆ ಬ್ರಹ್ಮನಾಗುವುದಲ್ಲ; ನೀವು ಆಗಲೇ ಅದಾಗಿರುವಿರಿ. ನೀವು ಇನ್ನು ಮೇಲೆ ದೇವರಾಗುವುದಲ್ಲ, ಪೂರ್ಣಾತ್ಮರಾಗುವುದಲ್ಲ; ನೀವಾಗಲೇ ಪೂರ್ಣಾತ್ಮರಾಗಿರುವಿರಿ. ನೀವು ಆಗಿಲ್ಲ ಎಂದು ಭಾವಿಸಿದರೆ ಅದು ಬರಿಯ ಭ್ರಾಂತಿ. ನೀವು ಇಂತಹ ಪುರುಷರು ಅಂತಹ ಸ್ತ್ರೀಯರೆಂದು ಸಾರುವ ಭ್ರಾಂತಿಯನ್ನು ಮತ್ತೊಂದು ಭ್ರಾಂತಿಯಿಂದ ಓಡಿಸಬೇಕಾಗಿದೆ. ಅದೇ ಅಭ್ಯಾಸ. ಬೆಂಕಿ ಬೆಂಕಿಯನ್ನು ಶಮನಮಾಡುವುದು. ಒಂದು ಭ್ರಾಂತಿಯಿಂದ ಪಾರಾಗಬೇಕಾದರೆ ಮತ್ತೊಂದು ಭ್ರಾಂತಿಯನ್ನು ಉಪಯೋಗಿಸಬೇಕು. ಒಂದು ಮೋಡ ಮತ್ತೊಂದನ್ನು ಆಚೆಗೆ ತಳ್ಳುವುದು. ಅನಂತರ ಎರಡೂ ಹೋಗುವುವು. ಹಾಗಾದರೆ ಈ ಅಭ್ಯಾಸಗಳು ಯಾವುವು? ನಾವು ಮುಂದೆ ಮುಕ್ತರಾಗುವುದಲ್ಲ, ನಾವು ಈಗಲೇ ಮುಕ್ತರು ಎನ್ನುವುದನ್ನು ಗಮನದಲ್ಲಿಡಬೇಕು. ನಾವು ಬಂದಿಗಳು ಎಂದು ಭಾವಿಸುವ ಭಾವನೆಗಳೆಲ್ಲ ಭ್ರಾಂತಿ. ನಾವು ಸುಖಿಗಳು ದುಃಖಿಗಳು ಎಂದು ಆಲೋಚಿಸುವುದೆಲ್ಲ ಅದ್ಭುತ ಭ್ರಾಂತಿ. ಮತ್ತೊಂದು ಭ್ರಾಂತಿ ಬರುವುದು. ಅದೇ ನಾವು ಕೆಲಸ ಮಾಡಬೇಕು, ಪೂಜೆ ಮಾಡಬೇಕು, ಮುಕ್ತರಾಗಲು ಹೋರಾಡಬೇಕು ಎನ್ನುವುದು. ಇವು ಮೊದಲಿನ ಭ್ರಾಂತಿಯನ್ನು ಓಡಿಸಿ ಕೊನೆಗೆ ಎರಡೂ ಲಯವಾಗುವುವು.

ಹಿಂದೂಗಳು ಮತ್ತು ಮಹಮ್ಮದೀಯರಿಬ್ಬರೂ ನರಿಯನ್ನು ಬಹಳ ಅಪವಿತ್ರ ಎಂದು ಭಾವಿಸುತ್ತಾರೆ. ನರಿಯೇನಾದರೂ ಅವರ ಅಡಿಗೆಯನ್ನು ಸ್ವಲ್ಪ ಮುಟ್ಟಿದರೂ ಅದನ್ನೆಲ್ಲ ಆಚೆಗೆ ಎಸೆಯಬೇಕು, ಯಾರೂ ಅದನ್ನು ಊಟಮಾಡುವುದಿಲ್ಲ. ಒಬ್ಬ ಮಹಮ್ಮದೀಯನ ಮನೆಗೆ ನರಿಯೊಂದು ನುಗ್ಗಿ ಮೇಜಿನ ಮೇಲೆ ಇಟ್ಟಿದ್ದ ಆಹಾರವನ್ನು ಸ್ವಲ್ಪ ತಿಂದು ಓಡಿಹೋಯಿತು. ಪಾಪ, ಆ ಮನುಷ್ಯ ಬಡವ. ತನ್ನ ಊಟಕ್ಕೆಂದು ಅಂದು ಬಹಳ ರುಚಿಕರವಾದ ಅಡಿಗೆಯನ್ನು ಮಾಡಿಕೊಂಡಿದ್ದ. ನರಿ ಅದನ್ನು ಮುಟ್ಟಿದ್ದರಿಂದ ಅದು ಎಂಜಲಾಯಿತು. ಅದನ್ನು ಅವನು ತಿನ್ನಲು ಆಗಲಿಲ್ಲ. ಅವನು ಮಹಮ್ಮದೀಯ ಪುರೋಹಿತನಾದ ಮುಲ್ಲನ ಬಳಿಗೆ ಹೋಗಿ ನೋಡಿ, “ನನಗೆ ಹೀಗಾಯಿತು. ನರಿಯೊಂದು ನನ್ನ ಊಟವನ್ನು ಸ್ವಲ್ಪ ತಿಂದುಹೋಯಿತು. ಈಗೇನು ಮಾಡುವುದು? ನಾನು ಹಬ್ಬದ ಅಡಿಗೆಯನ್ನು ಮಾಡಿ ಅದನ್ನು ತಿನ್ನಬೇಕೆಂದು ಕಾತರನಾಗಿದ್ದೆ. ಆಗ ನರಿಯೊಂದು ಬಂದು ಅದನ್ನೆಲ್ಲ ಹಾಳುಮಾಡಿತು'' ಎಂದ. ಮುಲ್ಲ ಒಂದು ಕ್ಷಣ ಯೋಚಿಸಿ ಏಕಮಾತ್ರ ಪರಿಹಾರವನ್ನು ಹೇಳಿದ: “ಇದಕ್ಕೆ ಪರಿಹಾರವೆ ಒಂದು ನಾಯಿಯನ್ನು ತಂದು ಅದನ್ನು ನರಿ ತಿಂದ ತಟ್ಟೆಯಿಂದಲೆ ತಿನ್ನುವಂತೆ ಮಾಡುವುದು. ಏಕೆಂದರೆ ನರಿನಾಯಿಗಳು ಯಾವಾಗಲೂ ಜಗಳ ಕಾಯುತ್ತಿರುವುವು. ನರಿಯ ಎಂಜಲು ಮತ್ತು ನಾಯಿಯ ಎಂಜಲು ಎರಡೂ ನಿನ್ನ ಹೊಟ್ಟೆಗೆ ಹೋದರೆ ಪರಿಶುದ್ಧವಾಗುವುದು". ನಾವೆಲ್ಲ ಇದೇ ಸ್ಥಿತಿಯಲ್ಲಿರುವೆವು. ನಾವು ಅಜ್ಞರು ಎಂದು ಭಾವಿಸುವುದೇ ಒಂದು ಭ್ರಾಂತಿ, ಆಗ ನಾವು ಮತ್ತೊಂದು ಭ್ರಾಂತಿಯನ್ನು ತೆಗೆದುಕೊಳ್ಳುತ್ತೇವೆ; ಅದೇ ನಾವು ಜ್ಞಾನಿಗಳಾಗಲು ಸಾಧನೆ ಮಾಡಬೇಕು ಎನ್ನುವುದು. ಆಗ ಒಂದು ಮತ್ತೊಂದನ್ನು ಆಚೆಗೆ ಓಡಿಸುವುದು. ಒಂದು ಮುಳ್ಳಿನಿಂದ ಮತ್ತೊಂದು ಮುಳ್ಳನ್ನು ತೆಗೆದುಹಾಕಿದಂತೆ. ಅನಂತರ ಎರಡು ಮುಳ್ಳುಗಳನ್ನೂ ಬಿಸಾಡಬೇಕು. ಕೆಲವರು ಇರುವರು. ಅವರು “ತತ್ತ್ವಮಸಿ'' ಎಂದು ಕೇಳಿದರೆ ಸಾಕು, ತಕ್ಷಣ ಜಗತ್ತು ಮಾಯವಾಗಿ ಸತ್ಯ ಅವರಿಗೆ ಹೊಳೆಯುವುದು. ಆದರೆ ಉಳಿದವರು ಈ ಭ್ರಾಂತಿಯಿಂದ ಪಾರಾಗಲು ಬಹಳ\break ಹೋರಾಡಬೇಕು.

ಮೊದಲನೆಯ ಪ್ರಶ್ನೆಯೇ ಯಾರು ಜ್ಞಾನಯೋಗಿಗಳಾಗಲು ಯೋಗ್ಯರು ಎಂಬುದು. ಯಾರಿಗೆ ಈ ಅರ್ಹತೆಗಳು ಇವೆಯೊ ಅವರು. ಮೊದಲನೆಯದೆ ಇಹಲೋಕದ ಮತ್ತು ಪರಲೋಕದ ಫಲಭೋಗದ ಮೇಲೆ ವಿರಕ್ತಿ. ನೀವು ಈ ಪ್ರಪಂಚದ ಸೃಷ್ಟಿಕರ್ತರಾದರೆ, ನೀವು ಇಚ್ಚಿಸುವುದೆಲ್ಲ ನಿಮಗೆ ಪ್ರಾಪ್ತವಾಗುವುದು. ಏಕೆಂದರೆ ನೀವೆ ಅದನ್ನು ಸೃಷ್ಟಿಸಿಕೊಳ್ಳುವಿರಿ. ಇದಕ್ಕೆಲ್ಲ ಸಕಾಲ ಬರಬೇಕು, ಅಷ್ಟೆ. ಕೆಲವರಿಗೆ ತಕ್ಷಣ ಅದು ಪ್ರಾಪ್ತವಾಗುವುದು. ಮತ್ತೆ ಕೆಲವರಿಗೆ ತಮ್ಮ ಹಿಂದಿನ ಸಂಸ್ಕಾರಗಳು ಆತಂಕವಾಗಿ ನಿಲ್ಲುವುವು. ಇಹಲೋಕದ ಮತ್ತು ಪರಲೋಕದ, ಭೋಗದ ಆಸೆಗಳಿಗೆ ನಾವು ಪ್ರಥಮಸ್ಥಾನವನ್ನು ಕೊಡುತ್ತೇವೆ. ಜೀವನ ಇದೆ ಎಂಬುದನ್ನು ಅಲ್ಲಗಳೆಯಿರಿ. ಏಕೆಂದರೆ, ಮರಣದ ಮತ್ತೊಂದು ಹೆಸರೇ ಜೀವನ. ನೀನೊಬ್ಬ ಬದುಕಿರುವ ಮನುಷ್ಯ ಎಂಬುದನ್ನು ಅಲ್ಲಗಳೆಯಿರಿ. ಈ ಜೀವನ ಯಾರಿಗೆ ಬೇಕು? ಜೀವನ ಒಂದು ಭ್ರಾಂತಿ, ಮರಣ ಇದಕ್ಕೆ ವಿರೋಧವಾದ ಇನ್ನೊಂದು ಭ್ರಾಂತಿ, ಸುಖ ಈ ಭ್ರಾಂತಿಯ ಮತ್ತೊಂದು ಭಾಗ, ದುಃಖವೂ ಈ ಭ್ರಾಂತಿಯ ಒಂದು ಭಾಗ; ಹೀಗೆಯೇ ಎಲ್ಲ. ಜನನ ಮರಣಗಳಿಂದ ನಿಮಗೆ ಆಗಬೇಕಾದುದೇನು? ಇವೆಲ್ಲ ಮನಸ್ಸಿನ ಕಲ್ಪನೆಗಳು. ಇದೇ ಇಹ ಪರ ಸುಖಗಳನ್ನು ತ್ಯಜಿಸುವುದು ಎಂಬುದು.

ಅನಂತರವೇ ಮನೋನಿಗ್ರಹ ಬರುವುದು: ಮನಸ್ಸನ್ನು ಶಾಂತಗೊಳಿಸಬೇಕು. ಪುನಃ ಚಿತ್ತವೃತ್ತಿಗಳೆದ್ದು ಆಸೆಯ ಅಲೆಗಳಿಂದ ತುಂಬಕೂಡದು. ಮನಸ್ಸನ್ನು ಏಕಾಗ್ರಗೊಳಿಸಬೇಕು. ಬಾಹ್ಯಕ್ಕೆ ಮತ್ತು ಆಂತರಿಕಕ್ಕೆ ಸಂಬಂಧಪಟ್ಟ ಯಾವ ಕಾರಣಗಳಿಂದಲೂ ಅದು ಪ್ರಕ್ಷುಬ್ಧವಾಗಕೂಡದು. ಇಚ್ಛಾಶಕ್ತಿಯಿಂದ ಮನಸ್ಸನ್ನು ಸಂಪೂರ್ಣ ನಿಗ್ರಹಿಸಬೇಕು. ಜ್ಞಾನಯೋಗಿ ಯಾವ ವಿಧವಾದ ಬಾಹ್ಯ ಅಥವಾ ಮಾನಸಿಕವಾದ ಸಹಾಯವನ್ನೂ ಅಪೇಕ್ಷಿಸುವುದಿಲ್ಲ. ಯುಕ್ತಿ, ಜ್ಞಾನ ಮತ್ತು ಇಚ್ಛಾಶಕ್ತಿ ಇವನ್ನೇ ಅವನು ನಂಬಿರುವುದು. ಅನಂತರವೆ ತಿತಿಕ್ಷೆ ಬರುವುದು. ಎಲ್ಲಾ ವ್ಯಥೆಯನ್ನೂ ಗೊಣಗಾಡದೆ, ದೂರದೆ ಸಹಿಸುವುದು.\break ಏನಾದರೂ ವ್ಯಥೆಯಾದರೆ ಅದನ್ನು ಲೆಕ್ಕಿಸಬೇಡಿ. ವ್ಯಾಘ್ರ ಬಂದರೆ ಅಲ್ಲಿ ನಿಲ್ಲಿ. ಯಾರು ಓಡಿಹೋಗುವರು? ತಿತಿಕ್ಷೆಯನ್ನು ಅಭ್ಯಾಸಮಾಡಿ ಅದರಲ್ಲಿ ಗೆದ್ದವರಿರುವರು. ಭರತಖಂಡದ ಉರಿಬೇಸಿಗೆಯಲ್ಲಿ ಗಂಗಾನದಿ ತೀರದ ಮೇಲೆ ಮಲಗಿರುವವರು ಇರುವರು. ಚಳಿಗಾಲದ ಮಧ್ಯದಲ್ಲಿ ಇಡೀ ದಿನ ನೀರಿನ ಮೇಲೆ ತೇಲುವವರು ಇರುವರು. ಅವರು ಅದನ್ನು ಲೆಕ್ಕಿಸುವುದೇ ಇಲ್ಲ. ಹಿಮಾಲಯದ ಮಂಜಿನ ಮೇಲೆ ಕುಳಿತುಕೊಳ್ಳುವರು. ವಸ್ತ್ರವನ್ನು ಧರಿಸಬೇಕೆಂದು ಇಚ್ಚೆಯೇ ಇರುವುದಿಲ್ಲ. ಚಳಿಯೇನು? ಬಿಸಿಲೇನು? ಇವು ಬಂದು ಹೋಗಲಿ, ಇವುಗಳಿಂದ ನನಗೇನು? ನಾನು ದೇಹವಲ್ಲ. ಪಾಶ್ಚಾತ್ಯ ದೇಶಗಳಲ್ಲಿ ಇವನ್ನು ನಂಬುವುದಕ್ಕೆ ಸಾಧ್ಯವಿಲ್ಲ. ಆದರೆ ಇವು ಸಾಧ್ಯ ಎಂಬುದನ್ನು ಅರಿತಿರುವುದು ಮೇಲು. ಹೇಗೆ ನಿಮ್ಮ ಜನರು ಗುಂಡಿನ ಬಾಯಿಗೆ ಎದೆಯೊಡ್ಡುವರೊ, ಸಮರಾಂಗಣಕ್ಕೆ ಧುಮುಕುವರೊ ಅದರಂತೆ ನಮ್ಮವರು ತಮ್ಮ ಧರ್ಮವನ್ನು ಮತ್ತು ತತ್ತ್ವವನ್ನು ಆಲೋಚಿಸಲು ಅದರಂತೆ ನಡೆಯಲು ಧೈರ್ಯವನ್ನು ತೋರುವರು. ಅವರು ತಮ್ಮ ಪ್ರಾಣವನ್ನು ಇದಕ್ಕೆ ಅರ್ಪಿಸುವರು. “ನಾನೇ ಅಖಂಡ ಸಚ್ಚಿದಾನಂದ, ಸೋಽಹಂ." ಪಾಶ್ಚಾತ್ಯರ ಆದರ್ಶ ಹೇಗೆ ನಿತ್ಯ ಜೀವನದಲ್ಲಿ ಭೋಗಸಾಮಗ್ರಿಗಳನ್ನು ಇಟ್ಟುಕೊಂಡಿರುವುದೊ ಹಾಗೆಯೆ ನಮ್ಮವರು ಶ್ರೇಷ್ಠ ಆಧ್ಯಾತ್ಮಿಕ ತತ್ತ್ವಗಳನ್ನು ನಿತ್ಯಜೀವನದಲ್ಲಿ ಇಟ್ಟುಕೊಂಡಿರುವರು. ಧರ್ಮ ಎಂದರೆ ಕೇವಲ ಬಾಯಿಮಾತಲ್ಲ. ಈ ಜೀವನದಲ್ಲಿ ಅದರ ಪ್ರತಿಯೊಂದು ಅಂಶವನ್ನು ಅನುಷ್ಠಾನಕ್ಕೆ ತರಬಹುದು ಎಂಬುದನ್ನು ತೋರುವರು. ಇದೇ ತಿತಿಕ್ಷೆ – ಎಲ್ಲವನ್ನೂ ಸಹಿಸುವುದು, ಏನನ್ನೂ ದೂರದೆ ಇರುವುದು. “ನಾನು ಆತ್ಮ. ಈ ವಿಶ್ವದಿಂದ ನನಗೇನು ಆಗಬೇಕಾಗಿದೆ? ಸುಖದುಃಖಗಳಾಗಲಿ, ಧರ್ಮ ಅಧರ್ಮಗಳಾಗಲಿ, ಶೀತೋಷ್ಣಗಳಾಗಲಿ ಇವುಗಳಿಂದ ನನಗೇನೂ ಇಲ್ಲ" ಎನ್ನುವವರನ್ನು ನಾನು ನೋಡಿರುವೆನು. ಇದೇ ತಿತಿಕ್ಷೆ. ದೇಹದ ಭೋಗವನ್ನು ಅರಸಿಕೊಂಡು ಹೋಗುವುದಲ್ಲ. ಧರ್ಮ ಎಂದರೇನು? ನನಗೆ ಅದು ಕೊಡು ಇದು ಕೊಡು ಎಂದು ಪ್ರಾರ್ಥಿಸುವುದೇ? ಇದೆಲ್ಲ ಮೂಢ ಧಾರ್ಮಿಕ ಭಾವನೆ. ಯಾರು ಇದನ್ನು ನಂಬುವರೋ ಅವರಿಗೆ ನಿಜವಾದ ಆತ್ಮ, ದೇವರು ಎಂದರೆ ಏನೆಂಬುದು ಗೊತ್ತಿಲ್ಲ. ನನ್ನ ಗುರುಗಳು ಹೇಳುತ್ತಿದ್ದರು. ರಣಹದ್ದು ಆಕಾಶದಲ್ಲಿ ಒಂದು ಸಣ್ಣ ಚುಕ್ಕೆಯಂತೆ ಕಾಣುವವರೆಗೆ ಮೇಲೆ ಮೇಲೆ ಹಾರಿಹೋಗುವುದು. ಆದರೆ ಅದರ ದೃಷ್ಟಿಯೆಲ್ಲ ಭೂಮಿಯ ಮೇಲೆ ಇರುವ ಕೊಳೆತು ನಾರುವ ಹೆಣದ ಕಡೆಗೆ. ನಿಮ್ಮ ಧಾರ್ಮಿಕ ಭಾವನೆಗಳ ಪರಿಣಾಮ ಏನಾಯಿತು? ರಸ್ತೆ ಗುಡಿಸುವುದು, ಹೆಚ್ಚು ಊಟ–ಬಟ್ಟೆ–ಬರೆ ಇವನ್ನು ಪಡೆಯುವುದೆ? ಯಾರು ಊಟ–ಬಟ್ಟೆಗಳನ್ನು ಲೆಕ್ಕಿಸುವರು? ಪ್ರತಿಕ್ಷಣ ಕೋಟ್ಯಂತರ ಜನರು ಬಂದು ಹೋಗುತ್ತಿರುವರು. ಇದನ್ನು ಯಾರು ಗಣನೆಗೆ ತರುವರು? ಈ ಕ್ಷುದ್ರ ಪ್ರಪಂಚದ ಕ್ಷಣಿಕ ಸುಖಗಳನ್ನು ಯಾರು ಲೆಕ್ಕಿಸುವರು? ಧೈರ್ಯವಿದ್ದರೆ ಇದನ್ನು ಮೀರಿ ಹೋಗಿ, ನಿಯಮವನ್ನು ದಾಟಿ. ಪ್ರಪಂಚವೆಲ್ಲ ಮಾಯವಾಗಲಿ, ನೀವೊಬ್ಬರೆ ನಿಲ್ಲಿ. “ನಾನೇ ಅಖಂಡ ಸಚ್ಚಿದಾನಂದ ಸೋಽಹಂ, ಸೋಽಹಂ.”

