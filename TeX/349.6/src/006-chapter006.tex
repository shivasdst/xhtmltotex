
\chapter[ವೇದಾಂತ ಮತ್ತು ಕ್ರೈಸ್ತ ಧರ್ಮ]{ವೇದಾಂತ ಮತ್ತು ಕ್ರೈಸ್ತ ಧರ್ಮ\protect\footnote{\enginline{* C.W, Vol. VI, P. 46}}}

\begin{center}
(೧೯೦೦ರ ಫೆಬ್ರವರಿ ೨೮ರಂದು ಓಕ್‌ಲ್ಯಾಂಡಿನ ಯೂನಿಟೇರಿಯನ್ ಚರ್ಚಿನಲ್ಲಿ ನೀಡಿದ ಪ್ರವಚನದ ಟಿಪ್ಪಣಿಗಳು)
\end{center}

ಪ್ರಪಂಚದ ಪ್ರಖ್ಯಾತ ಧರ್ಮಗಳಲ್ಲೆಲ್ಲಾ ಎಷ್ಟೋ ಸಮಾನವಾದ ಭಾವನೆಗಳು ಇವೆ. ಇವುಗಳಲ್ಲಿ ಕೆಲವು ವೇಳೆ ಎಷ್ಟು ಮಟ್ಟಿಗೆ ಸಾಮ್ಯ ಇರುತ್ತದೆ ಎಂದರೆ ಒಂದರಿಂದ ಅನೇಕ ಭಾವನೆಗಳನ್ನು ಮತ್ತೊಂದು ತೆಗೆದುಕೊಂಡಿದೆಯೋ ಎನ್ನುವಂತೆ ಕಾಣುವುದು. ಒಂದು ಧರ್ಮವು ಭಾವನೆಗಳನ್ನು ಇತರ ಧರ್ಮಗಳಿಂದ ತೆಗೆದುಕೊಂಡಿದೆ ಎಂಬ ವಾದಕ್ಕೆ ಹುರುಳಿಲ್ಲ ಎಂಬುದು ಕೆಳಗಿನ ವಾಸ್ತವಾಂಶಗಳಿಂದ ಸ್ಪಷ್ಟವಾಗುವುದು. ಮಾನವನ ಜೀವನದಲ್ಲಿ ಧರ್ಮ ಎಂಬುದು ಅವನ ಮುಖ್ಯ ಪ್ರವೃತ್ತಿಯಾಗಿರುವುದು. ಜೀವನವೆಂದರೆ ಯಾವುದು ಪ್ರತಿಯೊಬ್ಬ ಮಾನವನಲ್ಲಿಯೂ ಸುಪ್ತಾವಸ್ಥೆಯಲ್ಲಿರುವುದೋ ಅದರ ವಿಕಾಸ. ಆದುದರಿಂದ ಈ ಧರ್ಮ ಪ್ರವೃತ್ತಿ ಬಗೆಬಗೆಯ ಜನಗಳಲ್ಲಿ, ಬೇರೆ ಬೇರೆ ದೇಶಗಳಲ್ಲಿ ಅಭಿವ್ಯಕ್ತಗೊಳ್ಳುತ್ತಿದೆ.

ಆತ್ಮದ ಭಾಷೆ ಒಂದು, ಜನಾಂಗಗಳ ಭಾಷೆಗಳು ಹಲವು. ಅವರ ಆಚಾರ ವ್ಯವಹಾರಗಳಲ್ಲಿ ಎಷ್ಟೋ ವ್ಯತ್ಯಾಸವಿದೆ. ಧರ್ಮ ಎಂಬುದು ಆತ್ಮಕ್ಕೆ ಸಂಬಂಧಪಟ್ಟಿರುವುದು. ಇದು ಹಲವು ದೇಶಗಳ ಭಿನ್ನ ಭಿನ್ನ ಭಾಷೆಗಳ ಮತ್ತು ಆಚಾರಗಳ ಮೂಲಕ ವ್ಯಕ್ತವಾಗುತ್ತಿದೆ. ಆದಕಾರಣ ಪ್ರಪಂಚದ ಧರ್ಮಗಳಿಗೆ ಇರುವ ವ್ಯತ್ಯಾಸ ಕೇವಲ ಅದನ್ನು ವ್ಯಕ್ತಗೊಳಿಸುವ ರೀತಿಯಲ್ಲಿದೆಯೇ ವಿನಃ ಅವುಗಳ ಸಾರದಲ್ಲಿಲ್ಲ. ಧರ್ಮಗಳಲ್ಲಿರುವ ಸಾಮ್ಯ ಮತ್ತು ಏಕತೆ ಆತ್ಮನಿಗೆ ಸಂಬಂಧಪಟ್ಟಿದ್ದು. ಇದೇ ಧರ್ಮದ ಸಾರ. ಯಾವ ದೇಶದಲ್ಲಿಯಾಗಲಿ ಯಾವ ಕಾಲದಲ್ಲಿಯಾಗಲಿ ಮಾನವ ಮಾತನಾಡಿದರೂ ಆತ್ಮದ ಭಾಷೆ ಒಂದೇ ಆಗಿರುವುದು. ಹೇಗೆ ಬೇರೆ ಬೇರೆ ವಾದ್ಯಗಳಲ್ಲಿ ಒಂದೇ ಸ್ವರ ಮೇಳವಿದೆಯೋ ಹಾಗೆಯೇ ಬೇರೆ ಬೇರೆ ಧರ್ಮಗಳು ಸಾರುವುದೆಲ್ಲ ಒಂದೇ ಪಲ್ಲವಿ.

ಪ್ರಪಂಚದ ಪ್ರಮುಖ ಧರ್ಮಗಳಲ್ಲೆಲ್ಲ ಸಾಮಾನ್ಯವಾಗಿರುವ ಅಂಶವೆಂದರೆ ಪ್ರಮಾಣಭೂತವಾದ ಒಂದು ಶಾಸ್ತ್ರ. ಧರ್ಮಕ್ಕೆ ಇಂತಹ ಶಾಸ್ತ್ರವಿಲ್ಲದೆ ಇದ್ದರೆ ಅದು ನಿರ್ನಾಮವಾದಂತೆ. ಈಜಿಪ್ಟಿನ ಧರ್ಮಗಳ ವಿಷಯದಲ್ಲಿ ಆದದ್ದು ಹಾಗೆಯೇ. ಶಾಸ್ತ್ರವೇ ಧರ್ಮಕ್ಕೆ ತಳಹದಿ. ಇದರ ಸುತ್ತಲೂ ಆಯಾಯ ಧರ್ಮದ ಅನುಯಾಯಿಗಳು ನೆರೆಯುವರು. ಆ ಶಾಸ್ತ್ರದ ಮೂಲಕವೇ ಶಕ್ತಿ ಮತ್ತು ಸಿದ್ಧಾಂತ ಸಂಚರಿಸುವುದು.

ಪ್ರತಿಯೊಂದು ಧರ್ಮವೂ ತನ್ನ ಶಾಸ್ತ್ರವೇ ನಿಜವಾಗಿಯೂ ದೇವವಾಣಿ, ಉಳಿದ ಶಾಸ್ತ್ರಗಳಾವುವೂ ದೇವವಾಣಿಯಲ್ಲ, ಅವು ಮೂಢರ ತಲೆಯ ಮೇಲೆ ಹೇರಿರುವ ಒಂದು ಮೂಢ ನಂಬಿಕೆ ಎಂದು ಭಾವಿಸುವುದು. ಇತರ ಧರ್ಮಗಳನ್ನು ಅನುಸರಿಸುತ್ತಿರುವವನು ಅಜ್ಞಾನಿ ಮತ್ತು ಧರ್ಮಾಂಧ ಎನ್ನುವುದು.

ಎಲ್ಲಾ ಧರ್ಮಗಳ ಸಂಪ್ರದಾಯಬದ್ದರಲ್ಲಿ ಇರುವ ಮತಾಂಧತೆಯೇ ಇದು. ಉದಾಹರಣೆಗೆ ಸಂಪ್ರದಾಯ ಬದ್ಧ ವೈದಿಕರು ವೇದ ಒಂದೇ ಪ್ರಪಂಚದಲ್ಲಿ ಭಗವಂತನ ವಾಣಿ, ಅವನು ವೇದದ ಮೂಲಕ ಮಾತ್ರ ಪ್ರಪಂಚಕ್ಕೆ ಮಾತನಾಡಿರುವನು ಎನ್ನುವರು. ಇದು ಮಾತ್ರವಲ್ಲ, ಪ್ರಪಂಚ ಇರುವುದೇ ವೇದಗಳಿಂದ ಎನ್ನುವರು. (ಅವರ ಅಭಿಪ್ರಾಯದಲ್ಲಿ ಪ್ರಪಂಚಕ್ಕಿಂತ ಮುಂಚೆ ವೇದವಿತ್ತು). ವೇದದಲ್ಲಿರುವುದರಿಂದ ಪ್ರಪಂಚದಲ್ಲಿ ಪ್ರತಿಯೊಂದು ವಸ್ತುವೂ ಇರುವುದು. ಹಸು ಜಗತ್ತಿನಲ್ಲಿರುವುದಕ್ಕೆ ಕಾರಣ ಆ ಹೆಸರು ವೇದದಲ್ಲಿ ಬರುವುದರಿಂದ. ವೇದ ಭಾಷೆಯೇ ದೇವಭಾಷೆ, ಇತರ ಭಾಷೆಗಳು ಕೇವಲ ಆಡುನುಡಿಗಳೇ ಹೊರತು ಸಾಹಿತ್ಯ ಭಾಷೆಯಲ್ಲ. ವೇದದಲ್ಲಿ ಬರುವ ಪ್ರತಿಯೊಂದು ಅಕ್ಷರವನ್ನೂ ಸರಿಯಾಗಿ ಉಚ್ಚರಿಸಬೇಕು. ಪ್ರತಿಯೊಂದು ಸ್ವರವನ್ನೂ ಸರಿಯಾಗಿ ಉಚ್ಚರಿಸಬೇಕು. ಹೀಗೆ ಮಾಡದಿದ್ದರೆ ಇದೊಂದು ಮಹಾಪಾತಕ; ಅಕ್ಷಮ್ಯ.

ಇಂತಹ ಮತಾಂಧತೆ ಎಲ್ಲಾ ಸಾಂಪ್ರದಾಯಿಕ ಧರ್ಮಗಳಲ್ಲಿಯೂ ಇರುವುದು. ಆದರೆ ಕೇವಲ ಅಕ್ಷರಕ್ಕಾಗಿ ಹೋರಾಡುವವರು ಯಾರೆಂದರೆ ತಿಳಿವಳಿಕೆ ಇಲ್ಲದವರು ಮತ್ತು ಆಧ್ಯಾತ್ಮಿಕ ಸ್ವಭಾವವಿಲ್ಲದವರು ಮಾತ್ರ. ನಿಜವಾಗಿ ಧಾರ್ಮಿಕ ಜೀವನದಲ್ಲಿ ಮುಂದುವರಿದವರು ಎಂದಿಗೂ ಧರ್ಮದ ಹೊರ ಕವಚಕ್ಕಾಗಿ ಹೋರಾಡುವುದಿಲ್ಲ. ಎಲ್ಲಾ ಧರ್ಮದ ಸಾರ ಒಂದೇ ಎಂದು ಅವರಿಗೆ ಗೊತ್ತಿದೆ. ಆದಕಾರಣ ಅವರು ಬೇರೆ ಬೇರೆ ಭಾಷೆಗಳಲ್ಲಿ ಮಾತನಾಡಿದರೂ ಅವರಲ್ಲಿ ಎಂದಿಗೂ ವೈಮನಸ್ಯ ಇರುವುದಿಲ್ಲ.

ವೇದ ನಿಜವಾಗಿ ಪ್ರಪಂಚದ ಶಾಸ್ತ್ರಗಳಲ್ಲೆಲ್ಲ ಪುರಾತನವಾದುದು. ಇದನ್ನು ಯಾರು, ಯಾವ ಕಾಲದಲ್ಲಿ ಬರೆದರು ಎಂದು ಯಾರಿಗೂ ಹೇಳುವುದಕ್ಕೆ ಆಗುವುದಿಲ್ಲ. ಇದು ಹಲವು ಸಂಪುಟಗಳಲ್ಲಿದೆ. ಇವನ್ನೆಲ್ಲಾ ಯಾರಾದರೂ ಒಬ್ಬ ವ್ಯಕ್ತಿ ಸಂಪೂರ್ಣ ಓದಿರುವನೇ ಎಂದು ನನಗೆ ಅನುಮಾನವಿದೆ.

ವೇದ ಧರ್ಮವೇ ಹಿಂದೂಗಳ ಧರ್ಮ; ಇದೇ ಪ್ರಾಚ್ಯ ಧರ್ಮಗಳಿಗೆಲ್ಲಾ ತಳಹದಿ. ಅಂದರೆ ಇತರ ಪ್ರಾಚ್ಯ ಧರ್ಮಗಳೆಲ್ಲ ವೇದಮಹಾತರುವಿನ ಉಪಶಾಖೆಗಳು. ಪ್ರಾಚ್ಯಧರ್ಮಗಳಿಗೆಲ್ಲಾ ವೇದ ಪ್ರಮಾಣದ್ದಾಗಿದೆ.

ಏಸುಕ್ರಿಸ್ತನ ಉಪದೇಶಗಳನ್ನು ನಂಬಿಯೂ ಈಗಿನ ಕಾಲದಲ್ಲಿ ಅವುಗಳಲ್ಲಿ ಬಹುಭಾಗ ಅನುಷ್ಠಾನಕ್ಕೆ ಸಾಧ್ಯವಿಲ್ಲ ಎಂದು ಹೇಳುವುದು ಕುತರ್ಕ. ಈಗಿನ ಕ್ರೈಸ್ತರಲ್ಲಿ ಅಂತಹ ಶಕ್ತಿ ಏತಕ್ಕೆ ಇಲ್ಲ ಎಂದರೆ, ಅವರಲ್ಲಿ ಸಾಕಾದಷ್ಟು ಶ್ರದ್ದೆ ಇಲ್ಲ, ಚಿತ್ತಶುದ್ದಿಯಿಲ್ಲ ಎಂದು ಹೇಳಿದರೆ ಸರಿಯಾಗುವುದು. ಆದರೆ ಈಗಿನ ಕಾಲಕ್ಕೆ ಆ ಸತ್ಯಗಳು ಯೋಗ್ಯವಲ್ಲ ಎನ್ನುವುದು, ಕೆಲಸಕ್ಕೆ ಬಾರದ ಹರಟೆ.

ಕೊನೆಯಪಕ್ಷ ನನಗೆ ಸಮನಾಗಿಯಾದರೂ ಇಲ್ಲದೇ ಇರುವವರನ್ನು ನಾನು ಇದುವರೆಗೆ ಕಂಡಿಲ್ಲ. ನಾನು ಪ್ರಪಂಚವನ್ನೆಲ್ಲಾ ಸುತ್ತುತ್ತಿರುವೆನು. ನಾನು ಅಧಮಾಧಮರಾದ ನರಮಾಂಸಭಕ್ಷಕರೊಂದಿಗೂ ಇದ್ದೆ. ಎಲ್ಲಿಯೂ ನನಗೆ ಸಮನಾಗಿಲ್ಲದೇ ಇರುವವರನ್ನು ನೋಡಿಲ್ಲ. ನಾನು ಮೂರ್ಖನಾಗಿದ್ದಾಗ ಅವರು ಮಾಡಿದಂತೆ ಮಾಡಿರುವೆನು. ಆಗ ನನಗೆ ಉತ್ತಮವಾಗಿರುವುದು ಯಾವುದೂ ಗೊತ್ತಿರಲಿಲ್ಲ. ಈಗ ನನಗೆ ಗೊತ್ತಿದೆ. ಈಗ ಅವರಿಗೆ ಉತ್ತಮವಾಗಿರುವುದು ಗೊತ್ತಿಲ್ಲ. ಆದರೆ ಮುಂದೆ ಗೊತ್ತಾಗುವುದು. ಪ್ರತಿಯೊಬ್ಬನೂ ತನ್ನ ಸ್ವಭಾವಕ್ಕೆ ಅನುಗುಣವಾಗಿ ವರ್ತಿಸುವನು. ನಾವೆಲ್ಲಾ ಇನ್ನೂ ಬೆಳೆಯುತ್ತಿರುವೆವು. ಈ ದೃಷ್ಟಿಯಲ್ಲಿ ಒಬ್ಬನು ಮತ್ತೊಬ್ಬನಿಗಿಂತ ಮೇಲಲ್ಲ.

