
\chapter{ಸಾಂಖ್ಯ ತತ್ತ್ವ - ಒಂದು ಅಧ್ಯಯನ\protect\footnote{\enginline{* C.W, Vol. II, P. 442}}}

ಸಾಂಖ್ಯರು ಪ್ರಕೃತಿಯನ್ನು ಅವ್ಯಕ್ತ ಎನ್ನುವರು. ಅಲ್ಲಿ ವಸ್ತುಗಳೆಲ್ಲ ಸಂಪೂರ್ಣ ಸಮತ್ವದಲ್ಲಿರುವುವು. ಸಂಪೂರ್ಣ ಸಮತ್ವದಲ್ಲಿ ಯಾವ ವಿಧವಾದ ಚಲನೆಗೂ ಅವಕಾಶವಿಲ್ಲವೆನ್ನುವುದು ಇದರಿಂದ ಸ್ವಾಭಾವಿಕವಾಗಿ ಗೊತ್ತಾಗುವುದು. ಮೂಲಸ್ಥಿತಿಯಲ್ಲಿ, ಸೃಷ್ಟಿಯಾಗುವುದಕ್ಕಿಂತ ಮುಂಚೆ, ಯಾವ ಚಲನೆಯೂ ಇಲ್ಲದೆ ಇರುವಾಗ, ಎಲ್ಲ ಸಮತ್ವದಲ್ಲಿರುವಾಗ ಈ ಪ್ರಕೃತಿ ಅವಿನಾಶವಾಗಿತ್ತು. ಏಕೆಂದರೆ ಕ್ಷಯ ಅಥವಾ ಮರಣ ಬದಲಾವಣೆಯಿಂದ ಮತ್ತು ಸಮತ್ವನಾಶದಿಂದ ಮಾತ್ರ ಬರುವುದು. ಸಾಂಖ್ಯರ ಪ್ರಕಾರ ಕಣಗಳೆ ಮೂಲವಸ್ತುಗಳಲ್ಲ. ಈ ಸೃಷ್ಟಿಯು ಕಣಗಳಿಂದ ಬರುವುದಿಲ್ಲ. ಅವು ಎರಡನೆಯ ಅಥವಾ ಮೂರನೆಯ ಸ್ಥಿತಿಗಳು. ಮೂಲವಸ್ತುವು ಕಣಗಳಾಗಿ ಸೇರಿ ಸ್ಥೂಲಸ್ಥೂಲವಾಗಿ ದೊಡ್ಡದೊಡ್ಡದಾಗುವುದು. ಆಧುನಿಕ ಸಂಶೋಧನೆಯೂ ಇದನ್ನೇ ಅನುಮೋದಿಸುವಂತೆ ತೋರುವುದು. ಉದಾಹರಣೆಗೆ ಆಧುನಿಕ ಈಥರ್ ಸಿದ್ದಾಂತವಿದೆ; ಈಥರ್ ಎಂದರೆ ಕಣಗಳ ಸಮೂಹ ಎಂದು ಹೇಳಿದರೆ ಅದು ಸಮಸ್ಯೆಗೆ ಪರಿಹಾರವಾಗುವುದಿಲ್ಲ. ಇದನ್ನು ಸ್ಪಷ್ಟಪಡಿಸಬೇಕಾದರೆ, ಗಾಳಿ ಕಣಗಳಿಂದಾಗಿದೆ, ಈಥರ್‌ ಎಲ್ಲಾ ಕಡೆಯೂ ಇದೆ - ಸರ್ವವ್ಯಾಪಿಯಾಗಿ ಹಾಸು ಹೊಕ್ಕಾಗಿದೆ ಮತ್ತು ಗಾಳಿಯ ಕಣಗಳು ಈಥರ್‌ನಲ್ಲಿ ತೇಲುತ್ತಿವೆ ಎನ್ನಬೇಕು. ಈಥರ್ ಕಣಗಳಿಂದಾಗಿದ್ದ ಎರಡು ಈಥರ್ ಕಣಗಳ ಮಧ್ಯೆಯೂ ಅಂತರವಿರುತ್ತದೆ. ಇದನ್ನು ಯಾವುದು ಆವರಿಸುವುದು? ಇದಕ್ಕಿಂತಲೂ ಸೂಕ್ಷ್ಮವಾದ ಈಥರ್ ಕಣಗಳಿಂದ ಇದು ಕೂಡಿರುತ್ತದೆ ಎಂದರೆ, ಅಲ್ಲಿಯೂ ಎರಡು ಸೂಕ್ಷ್ಮ ಈಥರ್ ಕಣಗಳ ಮಧ್ಯೆ ಜಾಗವಿರುತ್ತದೆ. ಅದನ್ನು ಯಾವುದು ಆವರಿಸುವುದು ಎಂಬ ಪ್ರಶ್ನೆ ಏಳುವುದು. ಹೀಗೆಯೇ ಅಂತ್ಯವಿಲ್ಲದೆ ಹಿಂದೆ ಹಿಂದಕ್ಕೆ ಹೋಗುತ್ತಿರುತ್ತದೆ. ಇದು ಅನವಸ್ಥಾ ದೋಷವಾಗುವುದು (\enginline{regresses ad infinitum}). ಕಣ ಸಿದ್ದಾಂತವೇ ಕೊನೆ ಆಗಿರಲಾರದು. ಸಾಂಖ್ಯ ಸಿದ್ದಾಂತದ ಪ್ರಕಾರ ಪ್ರಕೃತಿ ಸರ್ವವ್ಯಾಪಿಯಾಗಿದೆ. ಪ್ರತಿಯೊಂದರ ಕಾರಣವೂ ಈ ಅಖಂಡ ಪ್ರಕೃತಿಯಲ್ಲಿ ಐಕ್ಯವಾಗಿದೆ. ಕಾರಣ ಎಂದರೆ ಅರ್ಥವೇನು? ವ್ಯಕ್ತಕ್ಕಿಂತ ಹಿಂದೆ ಇರುವ ಸೂಕ್ಷ್ಮಸ್ಥಿತಿ, ಅವ್ಯಕ್ತಸ್ಥಿತಿ. ನಾಶವೆಂದರೇನು? ಕಾರಣಕ್ಕೆ ಹಿಂತೆರಳುವುದು. ನಿಮ್ಮ ಹತ್ತಿರ ಒಂದು ಮಡಕೆ ಇದ್ದರೆ ನೀವು ಅದಕ್ಕೆ ಏಟು ಕೊಟ್ಟರೆ ಅದು ಒಡೆದು ಹೋಗುವುದು. ಕಾರ್ಯವು ತನ್ನ ಮೂಲಸ್ವಭಾವಕ್ಕೆ ಹಿಂತಿರುಗುವುದು ಎಂದು ಅರ್ಥ. ಯಾವ ವಸ್ತುವಿನಿಂದ ಮಡಕೆ ಆಗಿದೆಯೊ ಅದು ಮೂಲಸ್ಥಿತಿಗೆ ಹಿಂತಿರುಗುವುದು. ನಾಶವೆಂದರೆ ಇದಲ್ಲದೆ ಬೇರೆ ಯಾವ ಅರ್ಥವನ್ನು ಹೇಳಿದರೂ ಅದು ಅಸಂಬದ್ದವಾಗುತ್ತದೆ. ಆಧುನಿಕ ಭೌತಶಾಸ್ತ್ರದ ಪ್ರಕಾರ ಇದನ್ನು ಬೇಕಾದರೆ ತೋರಿಸಬಹುದು. ನಾಶ ಎಂದರೆ ಬಹಳ ಹಿಂದೆ ಕಪಿಲ ಹೇಳಿದಂತೆ ಕಾರಣಕ್ಕೆ ಹಿಂತಿರುಗುವುದು. ನಾಶವೆಂದರೆ ಹಿಂದಿನ ಸೂಕ್ಷ್ಮಸ್ಥಿತಿಗೆ ಹಿಂತಿರುಗುವುದು ಎಂದಷ್ಟೇ ಅರ್ಥ. ವಸ್ತು ಅವಿನಾಶಿ ಎಂಬುದನ್ನು ವೈಜ್ಞಾನಿಕ ಪ್ರಯೋಗಶಾಲೆಯಲ್ಲಿ ತೋರಿಸಬಹುದು. ಇದು ನಿಮಗೆ ಗೊತ್ತಿದೆ. ನಮಗೆ ಇಷ್ಟೊಂದು ವಿಷಯ ತಿಳಿದಿರುವ ಈಗಿನ ಕಾಲದಲ್ಲಿ, ಆತ್ಮ ಸರ್ವನಾಶವಾಗಿ ಹೋಗುವುದು (ಇಲ್ಲದೆ ಹೋಗುವುದು) ಎಂದರೆ, ಹಾಗೆ ಹೇಳುವವನು ನಗೆಗೀಡಾಗುವನು. ಅವಿದ್ಯಾವಂತರು, ಮೂಢರು, ಇಂತಹ ಸಿದ್ಧಾಂತಗಳನ್ನು ಮಂಡಿಸುವರು. ಆಧುನಿಕ ವಿಜ್ಞಾನವು ಹಿಂದಿನ ತತ್ತ್ವಶಾಸ್ತ್ರಗಳು ಏನನ್ನು ಹೇಳುವುವೊ ಅದನ್ನು ಅನುಮೋದಿಸುತ್ತಿರುವುದು ಆಶ್ಚರ್ಯವೇ ಸರಿ. ಇದೇ ಸರಿ, ಸತ್ಯಕ್ಕೆ ಇದೇ ಪ್ರಮಾಣ. ಮನಸ್ಸನ್ನು ಮೂಲ ವಸ್ತುವಾಗಿ ತೆಗೆದುಕೊಂಡು ಅವರು ಅನ್ವೇಷಣೆ ಪ್ರಾರಂಭಿಸಿದರು. ಈ ವಿಶ್ವದ ಮನೋವಿಭಾಗವನ್ನು ವಿಶ್ಲೇಷಣೆ ಮಾಡಿ ಕೆಲವು ನಿರ್ಣಯಗಳಿಗೆ ಬಂದರು. ನಾವು ಇಂದು ಭೌತಿಕಭಾಗವನ್ನು ವಿಶ್ಲೇಷಣೆ ಮಾಡಿದರೂ ಅದೇ ಸಿದ್ಧಾಂತಕ್ಕೆ ಬರಬೇಕು, ಏಕೆಂದರೆ ಎರಡೂ ಒಂದೇ ಕೇಂದ್ರಕ್ಕೆ ಒಯ್ಯಬೇಕು.

ವಿಶ್ವದಲ್ಲಿ ಪ್ರಕೃತಿಯ ಪ್ರಥಮ ಆವಿರ್ಭಾವವೇ ಸಾಂಖ್ಯರು ಹೇಳುವ “ಮಹತ್'' ಎಂಬುದನ್ನು ನೆನಪಿನಲ್ಲಿಟ್ಟಿರಬೇಕು. ಇದಕ್ಕೆ ಪ್ರಜ್ಞೆ, ಮಹತ್ತತ್ತ್ವ ಎಂಬ ಪದಶಃ ಅರ್ಥಕೊಡಬಹುದು. ಪ್ರಕೃತಿಯ ಪ್ರಥಮ ಆವಿರ್ಭಾವವೇ ಈ ಮಹತ್. ಇದನ್ನು ನಾನು ಚೇತನ ಎಂದು ಕರೆಯುವುದಿಲ್ಲ, ಇದು ತಪ್ಪಾಗುವುದು. ಚೇತನ ಮಹತ್ತಿನ ಒಂದು ಭಾಗ ಅಷ್ಟೆ. ಮಹತ್ತು ಸರ್ವವ್ಯಾಪಿ. ಇದು ನಮ್ಮ ಮನಸ್ಸಿನ, ಅಪ್ರಜ್ಞೆ ಮತ್ತು ಪ್ರಜ್ಞಾತೀತಸ್ಥಿತಿ ಎಲ್ಲಕ್ಕೂ ಅನ್ವಯಿಸುವುದು. ಪ್ರಜ್ಞೆಯ ಯಾವುದೇ ಒಂದು ಸ್ಥಿತಿಯನ್ನು ತೆಗೆದುಕೊಂಡು ಅದನ್ನು ಮಹತ್ತಿಗೆ ಅನ್ವಯಿಸಿದರೆ ಅದು ಸಾಕಾಗುವುದಿಲ್ಲ. ಉದಾಹರಣೆಗೆ ಪ್ರಕೃತಿಯಲ್ಲಿ ನಿಮ್ಮ ಕಣ್ಣ ಮುಂದೆ ಕೆಲವು ಬದಲಾವಣೆಗಳಾಗುತ್ತಿರುವುದನ್ನು ನೀವು ನೋಡಬಹುದು, ತಿಳಿದುಕೊಳ್ಳಬಹುದು. ಆದರೆ ಮತ್ತೆ ಕೆಲವು ಬದಲಾವಣೆಗಳಿವೆ; ಅವು ತುಂಬಾ ಸೂಕ್ಷ್ಮ. ಅವನ್ನು ಮಾನವ ಗ್ರಹಿಸಲಾರ. ಅದೇ ಮಹತ್ತು ಈ ಬದಲಾವಣೆಗಳನ್ನು ಉಂಟು ಮಾಡುತ್ತಿರುವುದು. ಆ ಮಹತ್ತಿನಿಂದ ವಿಶ್ವ ಅಹಂಕಾರ ಉದಿಸುವುದು. ಇವೆಲ್ಲ ದ್ರವ್ಯಗಳು ದ್ರವ್ಯಕ್ಕೂ ಮನಸ್ಸಿಗೂ ತರತಮವಲ್ಲದೆ ಬೇರೆ ವ್ಯತ್ಯಾಸವಿಲ್ಲ. ದ್ರವ ಒಂದೇ. ಒಂದು ಸೂಕ್ಷ್ಮಸ್ಥಿತಿಯಲ್ಲಿರುವುದು, ಮತ್ತೊಂದು ಸ್ಥೂಲಸ್ಥಿತಿಯಲ್ಲಿರುವುದು. ಒಂದು ಮತ್ತೊಂದಾಗುವುದು. ಆಧುನಿಕ ಶರೀರಶಾಸ್ತ್ರದ ಅನ್ವೇಷಣಾ ನಿರ್ಣಯಗಳನ್ನು ಇದು ಹೋಲುವುದು. ಮನಸ್ಸು ಮಿದುಳಿಗಿಂತ ಬೇರೆ ಅಲ್ಲ ಎಂಬುದನ್ನು ನಂಬಿದರೆ ನಾವು ಎಷ್ಟೋ ಹೊಡೆದಾಟದಿಂದ ಪಾರಾಗಬಹುದು. ಅಹಂಕಾರ ಮತ್ತೆರಡು ರೂಪವನ್ನು ಧಾರಣೆ ಮಾಡುವುದು. ಒಂದು ಇಂದ್ರಿಯವಾಗುವುದು. ಅವುಗಳಲ್ಲಿ ಒಂದು ಜ್ಞಾನೇಂದ್ರಿಯ ಮತ್ತೊಂದು ಕರ್ಮೇಂದ್ರಿಯ. ಅವು ಕಣ್ಣು ಕಿವಿಗಳಲ್ಲ, ಅವುಗಳ ಹಿಂದೆ ಇರುವ ಮಿದುಳಿನ ಕೇಂದ್ರಗಳು, ನರಕೇಂದ್ರಗಳು. ಈ ದ್ರವ್ಯವೇ, ಅಹಂಕಾರವೇ, ಬದಲಾಯಿಸುವುದು. ಈ ದ್ರವ್ಯದಿಂದ ಕೇಂದ್ರಗಳೆಲ್ಲಾ ಆಗುವುವು. ಇದೇ ದ್ರವ್ಯದಿಂದಲೇ ತನ್ಮಾತ್ರಗಳಾಗುವುವು. ಇವೇ ನಮ್ಮ ಇಂದ್ರಿಯಗಳಿಗೆ ತಾಕಿ ವಸ್ತುವಿನ ಅರಿವನ್ನು ಉಂಟುಮಾಡುವುದು. ನೀವು ಅವುಗಳನ್ನು ನೋಡಲಾರಿರಿ, ಅವು ಅಲ್ಲಿ ಇವೆ ಎಂದು ತಿಳಿಯಬಹುದು. ತನ್ಮಾತ್ರದಿಂದ ನಮಗೆ ಕಾಣುವ ಪಂಚಭೂತಗಳೆಲ್ಲ ಆಗುವುವು. ಇದನ್ನು ನಾನು ನಿಮಗೆ ಒತ್ತಿ ಹೇಳುತ್ತೇನೆ. ಇದನ್ನು ತಿಳಿದುಕೊಳ್ಳುವುದು ಬಹಳ ಕಷ್ಟ. ಏಕೆಂದರೆ ಪಾಶ್ಚಾತ್ಯರಲ್ಲಿ ದ್ರವ್ಯ ಮತ್ತು ಮನಸ್ಸಿನ ವಿಷಯವಾಗಿ ಹಲವು ವಿಚಿತ್ರ ಭಾವನೆಗಳಿವೆ. ಇಂತಹ ವಿಚಿತ್ರ ಭಾವನೆಗಳಿಂದ ಪಾರಾಗುವುದು ಬಹಳ ಕಷ್ಟ. ನಾನು ಬಾಲ್ಯದಲ್ಲಿ ಪಾಶ್ಚಾತ್ಯ ತತ್ತ್ವವನ್ನು ಅಧ್ಯಯನ ಮಾಡಿದ್ದುದರಿಂದ ನನಗೆ ಅದರಿಂದ ಪಾರಾಗಲು ಕಷ್ಟವಾಯಿತು. ಇವೆಲ್ಲ ಬ್ರಹ್ಮಾಂಡವಸ್ತುಗಳು. ಭೂತಗಳ ಮೂಲರೂಪವಾದ ಅಖಂಡ ಅವ್ಯಕ್ತ ಸ್ಥಿತಿಯನ್ನು ಕುರಿತು ಯೋಚಿಸಿ ನೋಡಿ. ಇದೇ ಎಲ್ಲದರ ಪ್ರಥಮ ಸ್ಥಿತಿ. ಈ ದ್ರವ್ಯವು ಹಾಲು ಮೊಸರಾದಂತೆ ಬದಲಾಯಿಸುವುದು. ಮೊದಲನೆ ಬದಲಾವಣೆಯ ಮಹತ್, ಮಹತ್ ಅಹಂಕಾರವಾಗುವುದು. ಮೂರನೆಯದೆ ಸರ್ವರಿಗೂ ಸಾಮಾನ್ಯವಾದ ಪಂಚೇಂದ್ರಿಯಗಳು, ಪಂಚತನ್ಮಾತ್ರಗಳು. ಈ ಪಂಚತನ್ಮಾತ್ರಗಳು ಮಿಶ್ರಣ ಹೊಂದಿ ನಾವು ಕೇಳುವ, ನೋಡುವ, ಮೂಸುವ ಸ್ಪರ್ಶಿಸುವ ಸ್ಥೂಲ ಜಗತ್ತಾಗಿ ಪರಿಣಮಿಸುತ್ತವೆ. ಸಾಂಖ್ಯ ತತ್ತ್ವದ ಪ್ರಕಾರ ಇದೇ ವಿಶ್ವರಚನಾರೀತಿ. ಯಾವುದು ಬ್ರಹ್ಮಾಂಡದಲ್ಲಿದೆಯೋ ಅದು ಪಿಂಡಾಂಡದಲ್ಲಿಯೂ ಇರಬೇಕು. ಒಬ್ಬ ವ್ಯಕ್ತಿಯನ್ನು ತೆಗೆದುಕೊಳ್ಳಿ. ಮೊದಲು ಅವನಲ್ಲಿ ಅವ್ಯಕ್ತ ಸ್ವಭಾವವಿರುವುದು. ಇದೇ ಮಹತ್ತಾಗುವುದು (ವಿಶ್ವಪ್ರಜ್ಞೆಯ ಒಂದು ಅಂಶ.) ಅದು ಅಹಂಕಾರವಾಗುವುದು, ಅನಂತರ ಪಂಚೇಂದ್ರಿಯಗಳು, ಅನಂತರ ನಮ್ಮದೇಹವಾಗುವ ತನ್ಮಾತ್ರಗಳು. ಇದನ್ನು ನೀವು ಸ್ಪಷ್ಟವಾಗಿ ತಿಳಿದುಕೊಳ್ಳಬೇಕು. ಏಕೆಂದರೆ ಇದೇ ಸಾಂಖ್ಯದರ್ಶನಕ್ಕೆ ಮೆಟ್ಟಿಲು. ಇದನ್ನು ನೀವು ತಿಳಿದುಕೊಳ್ಳುವುದು ಅತ್ಯಂತ ಆವಶ್ಯಕ. ಇಡೀ ಜಗತ್ತಿನ ತತ್ತ್ವಶಾಸ್ತ್ರಕ್ಕೆ ಇದೇ ತಳಹದಿ. ಕಪಿಲನಿಗೆ ಋಣಿಯಾಗದೆ ಇರುವ ಯಾವ ತತ್ತ್ವಶಾಸ್ತ್ರವೂ ಜಗತ್ತಿನಲ್ಲಿ ಇಲ್ಲ. ಪೈಥಾಗೊರಸ್ ಇಂಡಿಯಾದೇಶಕ್ಕೆ ಬಂದು ತತ್ತ್ವಶಾಸ್ತ್ರವನ್ನು ಅಧ್ಯಯನ ಮಾಡಿದನು. ಅದೇ ಗ್ರೀಕರ ತತ್ತ್ವಶಾಸ್ತ್ರಕ್ಕೆ ಆದಿಯಾಯಿತು. ಅನಂತರ ಅಲೆಕ್ಸಾಂಡ್ರಿಯದ ತತ್ತ್ವಶಾಸ್ತ್ರಕ್ಕೆ ಇದೇ ತಳಹದಿಯಾಯಿತು. ಇನ್ನು ಅನಂತರ ಇದೇ ಪಾಶ್ಚಾತ್ಯ ಅಧ್ಯಾತ್ಮ ರಹಸ್ಯ ವಿದ್ಯೆಗೆ ನಾಂದಿಯಾಯಿತು. ಅದು ಅನಂತರ ಎರಡು ಶಾಖೆಯಾಯಿತು. ಒಂದು ಯೂರೋಪು, ಅಲೆಕ್ಸಾಂಡ್ರಿಯಗಳಿಗೆ ಹೋಯಿತು. ಇನ್ನೊಂದು ಭಾರತದಲ್ಲಿ ಉಳಿಯಿತು. ಇದರಿಂದಲೇ ವ್ಯಾಸರ ಸಿದ್ದಾಂತ ಜನಿಸಿತು. ಕಪಿಲಮುನಿಗಳ ಸಾಂಖ್ಯದರ್ಶನವೇ ಜಗತ್ತು ಕಂಡ ಪ್ರಥಮ ಯುಕ್ತಿಯುಕ್ತವಾದ ಸಿದ್ಧಾಂತ. ಜಗತ್ತಿನ ಪ್ರತಿಯೊಬ್ಬ ತಾತ್ವಿಕನೂ ಕಪಿಲನಿಗೆ ಗೌರವ ತೋರಬೇಕು. ಅವನೇ ತತ್ತ್ವಶಾಸ್ತ್ರಗಳ ಪಿತಾಮಹ ಎಂಬ ದೃಷ್ಟಿಯಿಂದ ಅವನನ್ನು ನೋಡಬೇಕೆಂದು ನಿಮಗೆ ಹೇಳುವೆನು. ಈ ಅದ್ಭುತ ವ್ಯಕ್ತಿಯನ್ನು, ಪುರಾತನ ತತ್ತ್ವಶಾಸ್ತ್ರಜ್ಞನನ್ನು, ಶ್ರುತಿಗಳು ಕೂಡ ಕೊಂಡಾಡುವುವು. 'ಹೇ ದೇವರೆ ನೀನು. ಆದಿಯಲ್ಲಿ ಕಪಿಲಋಷಿಯನ್ನು ಸೃಷ್ಟಿಸಿದವನು.'' ಅವನ ಗ್ರಹಣಶಕ್ತಿ ಎಷ್ಟು ಅದ್ಭುತವಾಗಿತ್ತು! ಯೋಗಿಗಳ ಅದ್ಭುತ ಗ್ರಹಣ ಶಕ್ತಿಗೆ ನಮಗೆ ಏನಾದರೂ ಉದಾಹರಣೆ ಬೇಕಾದರೆ ಇಂತಹ ಋಷಿಗಳೇ ಅದಕ್ಕೆ ಪ್ರಮಾಣ. ಅವರ ಹತ್ತಿರ ದೂರದರ್ಶಕಯಂತ್ರ, ಸೂಕ್ಷ್ಮದರ್ಶಕ ಯಂತ್ರಗಳಿರಲಿಲ್ಲ. ಆದರೂ ಅವರ ಗ್ರಹಣಶಕ್ತಿ ಎಷ್ಟು ಸೂಕ್ಷ್ಮವಾಗಿತ್ತು! ಅವರ ವಿಶ್ಲೇಷಣೆ ಎಷ್ಟು ಅದ್ಭುತವಾಗಿತ್ತು, ಪೂರ್ಣವಾಗಿತ್ತು!

ಶೋಫನಿಯರ್‌ಗೂ ಭಾರತೀಯ ತತ್ತ್ವಶಾಸ್ತ್ರಕ್ಕೂ ಇರುವ ವ್ಯತ್ಯಾಸವನ್ನು ನಿಮಗೆ ಇಲ್ಲಿ ತೋರುವೆನು. ಶೋಫನಿಯರ್, ಇಚ್ಛೆಯೇ ಎಲ್ಲಕ್ಕೂ ಕಾರಣ ಎನ್ನುವನು. ಜೀವಿಸಬೇಕೆಂಬ ಇಚ್ಛೆ ನಾವು ಹುಟ್ಟುವುದಕ್ಕೆ ಕಾರಣ. ಆದರೆ ನಾವು ಇದನ್ನು ವಿರೋಧಿಸುತ್ತೇವೆ. ಇಚ್ಛೆ ಕರ್ಮೇಂದ್ರಿಯಕ್ಕೆ ಸಂಬಂಧಪಟ್ಟಿದೆ. ನಾನು ಒಂದು ವಸ್ತುವನ್ನು ನೋಡುವಾಗ ಇಚ್ಛೆ ಇರುವುದಿಲ್ಲ. ಈ ಸಂವೇದನೆಗಳು ಮೆದುಳಿಗೆ ಒಯ್ಯಲ್ಪಟ್ಟಾಗ 'ಇದನ್ನು ಮಾಡು, ಇದನ್ನು ಮಾಡಬೇಡ” ಎಂಬ ಪ್ರತಿಕ್ರಿಯೆಗಳುಂಟಾಗುವುದು. ಇಂತಹ ಅಹಂಕಾರದ ವಸ್ತುವನ್ನೇ ಇಚ್ಛೆ ಎನ್ನುವುದು. ಪ್ರತಿಕ್ರಿಯೆಯಿಲ್ಲದ ಯಾವ ಇಚ್ಛಾಶಕ್ತಿಯೂ ಇಲ್ಲ. ಇಚ್ಛೆಗಿಂತ ಮುಂಚೆ ಎಷ್ಟೋ ತತ್ತ್ವಗಳು ಬರುವುವು. ಇದೇ ಅಹಂಕಾರದಿಂದಾಗಿದೆ. ಅಹಂಕಾರ ಮಹತ್ತಿನಿಂದಾಗಿದೆ. ಮಹತ್ತೆ ಅವ್ಯಕ್ತ ಪ್ರಕೃತಿಯಿಂದಾಗಿದೆ. ನಮಗೆ ಕಾಣುವುದೆಲ್ಲ ಇಚ್ಛೆ ಎಂಬುದು ಬೌದ್ಧರ ಭಾವನೆ. ಮನಶ್ಶಾಸ್ತ್ರದ ದೃಷ್ಟಿಯಿಂದ ಇದು ಸಂಪೂರ್ಣ ತಪ್ಪು. ಇಚ್ಛೆ ಕೇವಲ ಕರ್ಮೇಂದ್ರಿಯಕ್ಕೆ ಮಾತ್ರ ಸಂಬಂಧಪಟ್ಟಿದೆ. ನೀವು ಕ್ರಿಯೆಗೆ ಕಾರಣವಾದ ನರಗಳನ್ನು ತೆಗೆದುಬಿಟ್ಟರೆ ಅವನಲ್ಲಿ ಇಚ್ಛೆಯೇ ಇರುವುದಿಲ್ಲ. ಬಹುಶಃ ನಿಮಗೆ ಚೆನ್ನಾಗಿ ವೇದ್ಯವಾಗಿರುವ ಈ ಅಂಶವನ್ನು ಪ್ರಾಣಿವರ್ಗದಲ್ಲಿ ಬೇಕಾದಷ್ಟು ಪ್ರಯೋಗಗಳನ್ನು ಮಾಡಿ ಆದ ಮೇಲೆ ಕಂಡುಹಿಡಿದಿರುವರು.

ನಾವು ಈ ಪ್ರಶ್ನೆಯನ್ನು ಈಗ ತೆಗೆದುಕೊಳ್ಳೋಣ. ಮನುಷ್ಯನಲ್ಲಿರುವ ಈ ಮಹತ್ತನ್ನು ತಿಳಿದುಕೊಳ್ಳುವುದು ಬಹಳ ಮುಖ್ಯ. ಈ ಮಹತ್ತೇ ಅಹಂಕಾರವಾಗಿ ರೂಪಾಂತರವಾಗುವುದು. ಈ ಮಹತ್ತೇ ದೇಹದಲ್ಲಿರುವ ಶಕ್ತಿಗೆಲ್ಲಾ ಮೂಲಕಾರಣ. ಇದು ಅಪ್ರಜ್ಞೆ, ಪ್ರಜ್ಞೆ ಮತ್ತು ಅತೀತಪ್ರಜ್ಞೆ ಇವನ್ನೆಲ್ಲ ಒಳಗೊಂಡಿರುವುದು. ಈ ಮೂರು ಸ್ಥಿತಿಗಳು ಏನು? ಅಪ್ರಜ್ಞೆ ನಮಗೆ ಪ್ರಾಣಿಗಳಲ್ಲಿ ದೊರಕುತ್ತದೆ. ಇದನ್ನೆ ಹುಟ್ಟುಗುಣ ಎನ್ನುವೆವು. ಇದು ತಪ್ಪುವುದು ಅಪರೂಪ. ಆದರೆ ಇದರ ಕಾರ್ಯಕ್ಷೇತ್ರ ತುಂಬಾ ಕಿರಿದು. ಅದು ಯಂತ್ರದಂತೆ ಕೆಲಸಮಾಡುತ್ತದೆ. ಪ್ರಾಣಿಗೆ ಯಾವ ಗಿಡ ವಿಷ ಮತ್ತು ಅಲ್ಲ ಎಂಬುದು ಹುಟ್ಟುಗುಣದಿಂದಲೇ ಗೊತ್ತಾಗುವುದು. ಆದರೆ ಇದಕ್ಕೆ ದೊಡ್ಡ ಮಿತಿ ಇದೆ. ಯಾವುದಾದರೂ ಹೊಸದು ಬಂದರೆ ಅದಕ್ಕೆ ಗೊತ್ತಾಗುವುದಿಲ್ಲ. ಇದಕ್ಕಿಂತ ಮೇಲಿರುವ ಜ್ಞಾನ ಅನಂತರ ಬರುವುದು. ಅದು ಆಗಾಗ ತಪ್ಪು ಮಾಡುವುದು, ಆದರೆ ಅದರ ಕಾರ್ಯಕ್ಷೇತ್ರ ವಿಶಾಲವಾಗಿದೆ. ಅದು ನಿಧಾನವಾಗಿ ಕೆಲಸ ಮಾಡುವುದು. ಇದನ್ನೇ ಯುಕ್ತಿ ಎನ್ನುವುದು. ಇದು ಹುಟ್ಟುಗುಣಕ್ಕಿಂತ ಹೆಚ್ಚು ವ್ಯಾಪ್ತಿಯುಳ್ಳದ್ದು. ಆದರೆ ಹುಟ್ಟುಗುಣ ಯುಕ್ತಿಗಿಂತ ಹೆಚ್ಚು ನಿಶ್ಚಿತವಾದ ರೀತಿಯಲ್ಲಿ ಕೆಲಸಮಾಡುತ್ತದೆ. ಯುಕ್ತಿಯಲ್ಲಿ ತಪ್ಪುವುದಕ್ಕೆ ಹೆಚ್ಚು ಅವಕಾಶವಿದೆ. ಆದರೆ ಹುಟ್ಟುಗುಣದಲ್ಲಿ ಇಲ್ಲ. ಯುಕ್ತಿಯನ್ನೂ ಮೀರಿದ ಒಂದು ಮಾನಸಿಕ ಅವಸ್ಥೆ ಇದೆ. ಅದು ಅತೀತ ಪ್ರಜ್ಞೆ - ಮನಸ್ಸನ್ನು ರೂಢಿಸಿರುವ ಯೋಗಿಗಳಿಗೆ ಮಾತ್ರ ಸಿದ್ಧಿಸುತ್ತದೆ. ಇದು ಯುಕ್ತಿಗೆ ಹೋಲಿಸಿದರೆ ತಪ್ಪನ್ನೇ ಮಾಡದೆ ಇರುವಂಥದು. ಇದರ ಕಾರ್ಯಕ್ಷೇತ್ರ ಅನಂತವಾದುದು. ಇದೇ ಅತಿ ಶ್ರೇಷ್ಠ ಸ್ಥಿತಿ. ಇಲ್ಲಿರುವುದಕ್ಕೆ ಮುಖ್ಯಕಾರಣ ಮಹತ್ ಎಂಬುದನ್ನು ನೆನಪಿನಲ್ಲಿಡಬೇಕು. ಇದೇ ಹಲವು ರೀತಿಯಲ್ಲಿ ಅಭಿವ್ಯಕ್ತಗೊಂಡಿದೆ. ಇದೇ ಮೇಲೆ ಸೂಚಿಸಿದ ಮನಸ್ಸಿನ ಮೂರು ಸ್ಥಿತಿಗಳನ್ನು ಆವರಿಸಿದೆ.

ಈಗ ಯಾವಾಗಲೂ ಕೇಳುವ ಒಂದು ಸೂಕ್ಷ್ಮ ಪ್ರಶ್ನೆ ಬರುವುದು. ಒಬ್ಬ ಪರಿಪೂರ್ಣನಾದ ದೇವರು ಈ ಪ್ರಪಂಚವನ್ನು ಸೃಷ್ಟಿಸಿದರೆ ಏತಕ್ಕೆ ಅಪೂರ್ಣತೆ ಪ್ರಪಂಚದಲ್ಲಿದೆ. ನಾವು ಯಾವುದನ್ನು ನೋಡುತ್ತೇವೆಯೋ ಅದನ್ನೇ ಪ್ರಪಂಚವೆನ್ನುವುದು. ಅದು ಪ್ರಜ್ಞೆ ಮತ್ತು ಆಲೋಚನೆಗಳ ಕಿರಿಯ ವಲಯ ಅಷ್ಟೆ. ಇದರಾಚೆ ನಮಗೆ ಕಾಣುವುದಿಲ್ಲ. ಈಗ ಈ ಪ್ರಶ್ನೆಯೇ ಸರಿಯಲ್ಲ. ನಾವು ಒಟ್ಟಿನಿಂದ ಸ್ವಲ್ಪ ಭಾಗವನ್ನು ತೆಗೆದುಕೊಂಡು ನೋಡಿದರೆ ಅದರಲ್ಲಿ ಯಾವ ಸಾಮರಸ್ಯವು ಕಾಣುವುದಿಲ್ಲ. ಇದು ಸ್ವಾಭಾವಿಕವೇ ಸರಿ, ಈ ಪ್ರಪಂಚದಲ್ಲಿ - ಸಾಮರಸ್ಯವಿಲ್ಲ. ಏತಕ್ಕೆಂದರೆ ನಾವೇ ಅದನ್ನು ಹಾಗೆ ಮಾಡುವೆವು. ಹೇಗೆ? ಇದಕ್ಕೆ ಕಾರಣವೇನು? ಜ್ಞಾನವೆಂದರೇನು? ಜ್ಞಾನವೆಂದರೆ ವಸ್ತುಗಳ ಪರಸ್ಪರ ಸಂಬಂಧವನ್ನು ಕಂಡುಹಿಡಿಯುವುದು. ನೀವು ಬೀದಿಗೆ ಹೋಗಿ ಒಬ್ಬ ಮನುಷ್ಯನನ್ನು ನೋಡಿ, ಇವನೊಬ್ಬ ಮನುಷ್ಯ ಎನ್ನುವಿರಿ. ಏಕೆಂದರೆ ಮನಸ್ಸಿನಲ್ಲಿ ಮನುಷ್ಯನಿಗೆ ಸಂಬಂಧ ಪಟ್ಟ ಲಕ್ಷಣಗಳನ್ನು ಜ್ಞಾಪಿಸಿಕೊಳ್ಳುವಿರಿ. ನೀವು ಹಲವು ಮನುಷ್ಯರನ್ನು ನೋಡಿರುವಿರಿ. ಪ್ರತಿಯೊಬ್ಬರೂ ನಿಮ್ಮ ಮನಸ್ಸಿನ ಮೇಲೆ ಒಂದು ಪರಿಣಾಮವನ್ನು ಬಿಟ್ಟಿರುವರು. ನೀವು ಹೊಸದಾಗಿ ಈಗ ಒಬ್ಬ ಮನುಷ್ಯನನ್ನು ನೋಡಿದಾಗ ಅವನನ್ನು ಹಿಂದಿನಿಂದ ಬಂದಿರುವ ಅಂತಹ ಚಿತ್ರಗಳೊಂದಿಗೆ ಹೋಲಿಸಿ ನೋಡಿದಾಗ ಸಮಾಧಾನವಾಗುವುದು. ಹೊಸ ಚಿತ್ರವನ್ನು ಹಳೆಯದರೊಂದಿಗೆ ಸೇರಿಸುವಿರಿ. ಒಂದು ಹೊಸ ಭಾವನೆ ಬಂದರೆ ಅದಕ್ಕೆ ಸಂಬಂಧಪಟ್ಟ ಹಳೆಯ ಚಿತ್ರಗಳಿದ್ದರೆ ನಿಮಗೆ ತೃಪ್ತಿಯಾಗುವುದು. ಇಂತಹ ಸಂಬಂಧವನ್ನೆ ಜ್ಞಾನ ಎನ್ನುವುದು. ಜ್ಞಾನ ಎಂದರೆ ಆಗಲೆ ಇರುವ - ಅನುಭವದೊಂದಿಗೆ ಒಂದು ಹೊಸದನ್ನು ಸೇರಿಸಿ ಅದಕ್ಕೊಂದು ಸ್ಥಾನವನ್ನು ಕೊಡುವಿರಿ. ನಿಮಗೆ ಆಗಲೇ ಒಂದು ಅನುಭವದ ನಿಧಿ ಇಲ್ಲದಿದ್ದರೆ ನಿಮಗೆ ಯಾವ ಜ್ಞಾನವೂ ಸಾಧ್ಯವಿಲ್ಲ ಎಂಬ ಸತ್ಯಕ್ಕೆ ಇದು ದೊಡ್ಡ ಪ್ರಮಾಣವಾಗಿದೆ. ಕೆಲವು ಐರೋಪ್ಯ ತಾತ್ವಿಕರು ಊಹಿಸುವಂತೆ, ನಿಮ್ಮಲ್ಲಿ ಪ್ರಾರಂಭದಲ್ಲಿ ಯಾವ ಅನುಭವವೂ ಇಲ್ಲದೆ ಇದ್ದರೆ ನಿಮಗೆ ಯಾವ ಜ್ಞಾನವೂ ಹೊಸದಾಗಿ ದೊರಕುವುದಿಲ್ಲ. ಏಕೆಂದರೆ ಜ್ಞಾನವೆಂದರೆ ಆಗಲೆ ಇರುವ ಹಳೆಯ ಅನುಭವದೊಂದಿಗೆ ಹೊಸದನ್ನು ತುಲನೆ ಮಾಡಿ ಅದಕ್ಕೆ ಒಂದು ಸ್ಥಳವನ್ನು ಕೊಡುವುದು. ಹೊಸ ಅನುಭವವನ್ನು ಹೋಲಿಸಬೇಕಾದರೆ ಆಗಲೆ ಒಂದು ಅನುಭವ ನನಗೆ ಇರಬೇಕು. ಇಂತಹ ಯಾವ ಅನುಭವದ ರಾಶಿಯೂ ಇಲ್ಲದೆ ಒಂದು ಮಗು ಹುಟ್ಟಿದರೆ ಆ ಮಗುವಿಗೆ ಜ್ಞಾನವನ್ನು ಪಡೆಯುವುದಕ್ಕೆ ಎಂದಿಗೂ ಸಾಧ್ಯವೆ ಇಲ್ಲ. ಆದಕಾರಣ ಮಗುವಿನಲ್ಲಿ ಆಗಲೆ ಸ್ವಲ್ಪ ಜ್ಞಾನವಿದ್ದ ಸ್ಥಿತಿ ಇರಬೇಕು. ಇದರಿಂದ ಜ್ಞಾನ ವೃದ್ದಿಯಾಗುತ್ತಾ ಬರುವುದು. ಈ ವಾದದಿಂದ ತಪ್ಪಿಸಿಕೊಳ್ಳುವುದಕ್ಕೆ ಒಂದು ಮಾರ್ಗವನ್ನು ತೋರಿಸಿ ನೋಡೋಣ. ಇದು ಎರಡು ಮತ್ತು ಎರಡು ನಾಲ್ಕು ಎನ್ನುವಷ್ಟು ಸತ್ಯ. ಕೆಲವು ಪಾಶ್ಚಾತ್ಯ ವಿದ್ವಾಂಸರ ಪಂಗಡದವರೂ ಕೂಡ ಆಗಲೆ ಒಂದು ಜ್ಞಾನದ ನಿಧಿ ಇಲ್ಲದೆ ಹೊಸದಾಗಿ ಜ್ಞಾನ ಸಾಧ್ಯವೇ ಇಲ್ಲ ಎನ್ನುವರು. ಮಗು ಜ್ಞಾನದೊಂದಿಗೆ ಹುಟ್ಟುವುದು ಎಂಬ ಭಾವನೆಯನ್ನು ಅವರೂ ತರುವರು. ಆದರೆ ಈ ಆಜನ್ಮ ಜ್ಞಾನ ಮಗುವಿನ ಪೂರ್ವಜನ್ಮದಲ್ಲ. ಇದು ಆ ಮಗುವಿನ ಪೂರ್ವಿಕರ ಅನುಭವ. ಇದು ಕೇವಲ ಆನುವಂಶಿಕವಾಗಿ ಬಂದದ್ದು ಎಂದು ಆ ಪಾಶ್ಚಾತ್ಯ ವಿದ್ವಾಂಸರು ಹೇಳುವರು. ಈ ಭಾವನೆ ತಪ್ಪು ಎಂದು ಅವರಿಗೆ ಬೇಗ ಗೊತ್ತಾಗುವುದು. ಕೆಲವು ಜರ್ಮನ್ ತಾತ್ವಿಕರು ಆನುವಂಶಿಕ ಸಿದ್ದಾಂತವನ್ನು ಅಲ್ಲಗಳೆಯುತ್ತಿರುವರು. ಆನುವಂಶಿಕತೆಯೇನೋ ಒಳ್ಳೆಯದು, ಆದರೆ ಸೂಕ್ತವಲ್ಲ. ಅದು ಕೇವಲ ಸ್ಥೂಲ ವಿಷಯವನ್ನು ಮಾತ್ರ ವಿವರಿಸುವುದು. ನಮ್ಮ ಸುತ್ತಲಿರುವ ವಾತಾವರಣ ನಮ್ಮ ಮೇಲೆ ತಮ್ಮ ಪ್ರಭಾವವನ್ನು ಬೀರುತ್ತಿರುವುದನ್ನು ನೀವು ಹೇಗೆ ವಿವರಿಸುತ್ತೀರಿ? ಹಲವು ಕಾರಣಗಳು ಒಂದು ಪರಿಣಾಮವನ್ನು ಉಂಟುಮಾಡುತ್ತವೆ. ವಾತಾವರಣವು ನಮ್ಮ ಮೇಲೆ ತನ್ನ ಪ್ರಭಾವವನ್ನು ಬೀರುತ್ತದೆ. ನಾವೇ ನಮ್ಮ ವಾತಾವರಣವನ್ನು ಸೃಷ್ಟಿಸಿಕೊಳ್ಳುವೆವು. ನಮ್ಮ ಹಿಂದಿನದಕ್ಕೆ ತಕ್ಕಂತೆ ಈಗಿನ ವಾತಾವರಣ. ಒಬ್ಬ ಕುಡುಕ ಸ್ವಾಭಾವಿಕವಾಗಿ ಅಂತಹ ಕುಡುಕರಿರುವಲ್ಲಿಗೆ ಆಕರ್ಷಿಸಲ್ಪಡುವನು.

ಜ್ಞಾನವೆಂದರೇನು ಎಂಬುದು ನಿಮಗೆ ಗೊತ್ತಾಯಿತು. ಜ್ಞಾನವೆಂದರೆ ಹೊಸ ಅನುಭವವನ್ನು ಹಳೆಯದರೊಂದಿಗೆ ಜೋಡಿಸುವುದು, ಹೊಸ ಅನುಭವವನ್ನು ಗುರುತಿಸುವುದು ಎಂದು ಅರ್ಥ. ಗುರುತಿಸುವುದು ಎಂದರೇನು? ಆಗಲೆ ಅವನಲ್ಲಿ ಇರುವ ಅನುಭವಗಳೊಂದಿಗೆ ಈಗಿನ ಅನುಭವದ ಸಂಬಂಧವನ್ನು ತಿಳಿದುಕೊಳ್ಳುವುದು ಎಂದು ಅರ್ಥ. ಜ್ಞಾನ ಎಂದರೆ ಇದಕ್ಕಿಂತ ಹೆಚ್ಚೇನೂ ಅಲ್ಲ. ಜ್ಞಾನ ಎಂದರೆ ಸಂಬಂಧವನ್ನು ತಿಳಿದುಕೊಳ್ಳುವುದು ಎಂದಾದರೆ ನಾವು ಯಾವುದನ್ನು ತಿಳಿದುಕೊಳ್ಳಬೇಕಾದರೂ ಅದರಂತೆ ಇರುವ ಎಲ್ಲ ವಸ್ತುಗಳನ್ನೂ ನಾವು ತಿಳಿದಿರಬೇಕಲ್ಲವೆ? ಉದಾಹರಣೆಗೆ, ನೀವು ಒಂದು ಕಲ್ಲನ್ನು ತೆಗೆದುಕೊಳ್ಳಿ. ಅದರ ಸಂಬಂಧವನ್ನು ತಿಳಿದುಕೊಳ್ಳಬೇಕಾದರೆ ಅದರಂತೆ ಇರುವ ಕಲ್ಲುಗಳನ್ನೆಲ್ಲಾ ನಾವು ನೋಡಬೇಕು. ಆದರೆ ಪೂರ್ಣ ವಿಶ್ವದ ಭಾವನೆಯ ವಿಷಯದಲ್ಲಿ ನಾವು ಹಾಗೆ ಮಾಡಲಾರೆವು. ನಮ್ಮ ಮನದ ಅನುಭವದ ಉಗ್ರಾಣದಲ್ಲಿ ಇಂತಹ ಒಂದು ಭಾವನೆ ಮಾತ್ರ ಇದೆ. ಈ ಗುಂಪಿಗೆ ಸೇರಿದ ಮತ್ತಾವುದೂ ಇಲ್ಲವೇ ಇಲ್ಲ. ನಾವು ಬೇರೆ ಅನುಭವದೊಂದಿಗೆ ಇದನ್ನು ಹೋಲಿಸಲಾರೆವು. ಅದಕ್ಕೆ ಸಂಬಂಧಪಟ್ಟ ವಸ್ತುಗಳೊಂದಿಗೆ ಹೋಲಿಸುವುದಕ್ಕೆ ಆಗುವುದಿಲ್ಲ. ನಮ್ಮ ಪ್ರಜ್ಞೆಯಿಂದ ಬೇರೆಯಾದ ಈ ವಿಶ್ವ ಒಂದು ವಿಚಿತ್ರವಾದ ಹೊಸ ವಸ್ತು. ಏಕೆಂದರೆ ನಮಗೆ ಅದರ ಸಂಬಂಧ ದೊರಕುವುದಿಲ್ಲ. ಆದಕಾರಣವೆ ನಾವು ಅದರೊಂದಿಗೆ ಹೆಣಗಾಡುತ್ತಿರುವೆವು. ಅದು ಬೀಭತ್ಸವಾದುದು, ಕೆಟ್ಟದು ಎಂದೆಲ್ಲ ಭಾವಿಸುತ್ತೇವೆ. ಕೆಲವು ವೇಳೆ ಒಳ್ಳೆಯದೆಂದು ಅದನ್ನು ಭಾವಿಸಬಹುದು, ಆದರೆ ಯಾವಾಗಲೂ ಅಪೂರ್ಣ ಎಂದೇ ತಿಳಿಯುವೆವು. ಅದರ ಸಂಬಂಧ ದೊರೆತಾಗ ಮಾತ್ರ ಅದರ ವಿವರಣೆ ದೊರಕುವುದು. ನಾವು ಬಾಹ್ಯ ಪ್ರಪಂಚ ಮತ್ತು ಪ್ರಜ್ಞೆಯನ್ನು ಮೀರಿಹೊದಾಗ ವಿಶ್ವದ ಯಥಾರ್ಥ ಸ್ಥಿತಿ ಅರಿವಾಗುವುದು. ನಾವು ಇದನ್ನು ಮಾಡದೆ ಸುಮ್ಮನೆ ಪ್ರಪಂಚಕ್ಕೆ ತಲೆ ಚಚ್ಚಿಕೊಂಡರೆ ಪ್ರಯೋಜನವಿಲ್ಲ. ಏಕೆಂದರೆ ಜ್ಞಾನವೆಂದರೆ ಸಾಂಬಂಧಿಕ ವಸ್ತುಗಳನ್ನು ಕಂಡುಹಿಡಿಯುವುದು. ನಮ್ಮ ಪ್ರಜ್ಞಾವಸ್ಥೆ ಇದಕ್ಕೆ ಸಂಬಂಧಿಸಿದ ಒಂದು ಅನುಭವವನ್ನು ಮಾತ್ರ ನಮಗೆ ಕೊಡುವುದು. ಇದರಂತೆಯೆ ದೇವರ ನಮ್ಮ ಭಾವನೆ ಕೂಡ. ದೇವರ ವಿಷಯವಾಗಿ ನಾವು ನೋಡುವುದೆಲ್ಲ ಒಂದು ಅಂಶ ಮಾತ್ರ. ನಮಗೆ ವಿಶ್ವದ ಯಾವುದೋ ಒಂದು ಭಾಗ ಮಾತ್ರ ಅರಿವಾಗಿ ಉಳಿದುದೆಲ್ಲವೂ ನಮ್ಮ ಪ್ರಜ್ಞೆಗೆ ಅತೀತವಾಗಿದೆ. "ವಿಶ್ವವ್ಯಾಪಿಯಾದ ನಾನು ಅಷ್ಟು ಭೂಮವಾಗಿರುವೆನು. ಈ ಪ್ರಪಂಚ ಕೂಡ ನನ್ನ ಒಂದು ಅಂಶ.'' ಆದಕಾರಣವೇ ನಾವು ದೇವರನ್ನು ಅಪೂರ್ಣವೆಂದು ನೋಡುತ್ತೇವೆ. ಅವನನ್ನು ನಾವು ತಿಳಿಯಲಾರೆವು. ಅವನನ್ನು ನಾವು ತಿಳಿಯಬೇಕಾದರೆ, ವಿಶ್ವವನ್ನು ನಾವು ತಿಳಿಯಬೇಕಾದರೆ, ಯುಕ್ತಿಯನ್ನು ಮೀರಿಹೋಗಬೇಕು, ಪ್ರಜ್ಞೆಯನ್ನು ಮೀರಿ ಹೋಗಬೇಕು. "ಎಂದು ನೀನು ಕೇಳಿದ ಮತ್ತು ಕೇಳುತ್ತಿರುವ ವಸ್ತುವನ್ನು ಮೀರಿ ಹೋಗುವೆಯೋ, ಆಲೋಚನೆ ಮತ್ತು ಆಲೋಚನೆಗೆ ಸಾಧ್ಯವಾದುದನ್ನು ಮೀರಿಹೋಗುವೆಯೋ, ಆಗ ಮಾತ್ರ ನೀನು ಸತ್ಯವನ್ನು ಅರಿಯುವೆ." ಶಾಸ್ತ್ರವನ್ನು ಮೀರಿಹೋಗಬೇಕು, ಶಾಸ್ತ್ರವು ಕೇವಲ ಪ್ರಕೃತಿಗೆ ಸಂಬಂಧಪಟ್ಟಿರುವುದನ್ನು ಮಾತ್ರ, ಮೂರು ಗುಣಗಳನ್ನು ಮಾತ್ರ ಹೇಳುವುದು. ನಾವು ಇದನ್ನು ಮೀರಿ ಹೋದಾಗ ಮಾತ್ರ ಸಾಮರಸ್ಯ ದೊರಕುವುದು. ಅದಕ್ಕೆ ಮುಂಚೆ ಅಲ್ಲ.

ಪಿಂಡಾಂಡ ಬ್ರಹ್ಮಾಂಡಗಳೆರಡೂ ಒಂದೇ ಯೋಜನೆಯ ಮೇಲೆ ರಚಿತವಾಗಿವೆ. ನಮಗೆ ಪಿಂಡಾಂಡದಲ್ಲಿ ಒಂದು ಭಾಗ ಮಾತ್ರ ಗೊತ್ತಿದೆ. ಅದೇ ಮಧ್ಯ ಭಾಗ. ಪ್ರಜ್ಞೆಗೆ ಕೆಳಗಿರುವುದೂ ಅರಿವಿಲ್ಲ, ಅದಕ್ಕೆ ಮೀರಿರುವುದೂ ಅರಿವಿಲ್ಲ. ಪ್ರಜ್ಞಾವಸ್ಥೆ ಮಾತ್ರ ನಮಗೆ ಗೊತ್ತಿದೆ. ಒಬ್ಬ ಮನುಷ್ಯ ತಾನು ಪಾಪಿ ಎಂದು ಹೇಳಿಕೊಂಡರೆ ಅದು ಸುಳ್ಳು. ಏಕೆಂದರೆ ತಾನು ಏನೆಂಬುದು ಇವನಿಗೆ ಗೊತ್ತಿಲ್ಲ. ಅವನೇ ಮೂಢಾಧಮ, ಅವನಿಗೆ ಸಂಬಂಧಪಟ್ಟಿರುವುದರಲ್ಲಿ ಎಲ್ಲೋ ಒಂದು ಭಾಗ ಮಾತ್ರ ಅವನಿಗೆ ಗೊತ್ತಿದೆ. ಇದರಂತೆಯೆ ವಿಶ್ವದ ವಿಷಯ ಕೂಡ. ಯುಕ್ತಿಯ ಮೂಲಕ ಯಾವುದೋ ಒಂದು ಅಂಶವನ್ನು ಮಾತ್ರ ನಾವು ತಿಳಿಯಬಹುದು, ಪೂರ್ಣವನ್ನು ತಿಳಿಯಲಾರೆವು. ವಿಶ್ವವೆಂದರೆ ಅದರಲ್ಲಿ ಅಪ್ರಜ್ಞೆ (\enginline{sub-conscious}), ಪ್ರಜ್ಞೆ (\enginline{conscious}), ಮತ್ತು ಪ್ರಜ್ಞಾತೀತ (\enginline{Super Conscious}), ವ್ಯಷ್ಟಿಯ ಮಹತ್ ಮತ್ತು ಸಮಷ್ಟಿಯ ಮಹತ್ ಮತ್ತು ಅದರ ವಿಕಾರಗಳೆಲ್ಲ ಸೇರಿವೆ.

ಪ್ರಕೃತಿಯ ವಿಕಾರಕ್ಕೆ ಕಾರಣವೇನು? ಪ್ರಕೃತಿಯೆಲ್ಲ ಜಡ ಎಂದು ನಮಗೆ ಇದುವರೆಗೆ ಗೊತ್ತಿದೆ. ಇದೆಲ್ಲ ಒಂದು ಮಿಶ್ರ, ಅಚೇತನ. ಪ್ರಕೃತಿ ನಿಯಮವು ಎಲ್ಲಿ ಜಾರಿಯಲ್ಲಿದೆಯೋ ಅದೆಲ್ಲ ಜಡ ಎನ್ನುವುದು ವೇದ್ಯ. ಮನಸ್ಸು, ಬುದ್ಧಿ, ಇಚ್ಛೆ ಎಲ್ಲಾ ಜಡ. ಆದರೆ ಇವುಗಳನ್ನು ಮೀರಿದ ಒಂದು ಶಕ್ತಿಯ ಚಿತ್ ಪ್ರಭೆಯನ್ನು ಇವೆಲ್ಲಾ ಪ್ರತಿಬಿಂಬಿಸುತ್ತಿವೆ. ಇದನ್ನೆ ಸಾಂಖ್ಯರು ಪುರುಷ ಎಂದು ಕರೆಯುವರು. ಪುರುಷನೆ ಪ್ರಕೃತಿಯ ಬದಲಾವಣೆಗಳಿಗೆಲ್ಲ ಅನೈಚ್ಛಿಕ ಕಾರಣ. ಸಮಷ್ಟಿ ದೃಷ್ಟಿಯಿಂದ ನಾವು ಈ ಪುರುಷನನ್ನು ತೆಗೆದುಕೊಂಡರೆ ಇವನೇ ವಿಶ್ವಕ್ಕೆ ದೇವರು. ದೇವರ ಇಚ್ಚೆಯೇ ಸೃಷ್ಟಿಗೆ ಕಾರಣ ಎನ್ನುವರು. ಇದು ಸಾಮಾನ್ಯವಾದ ಹೇಳಿಕೆಯಂತೆ ಚೆನ್ನಾಗೇನೊ ಇರುವುದು. ಆದರೆ ಇದು ನಿಜವಲ್ಲವೆಂಬುದನ್ನು ನೋಡುತ್ತೇವೆ. ಇದೇ ಹೇಗೆ ಇಚ್ಛೆಯಾಗಬಲ್ಲುದು? ಇಚ್ಛೆ ಪ್ರಕೃತಿಯಲ್ಲಿ ಮೂರನೆಯದೊ ನಾಲ್ಕನೆಯದೊ ಸ್ಥಾನಕ್ಕೆ ಬರುವುದು. ಇದಕ್ಕೆ ಮುಂಚೆ ಎಷ್ಟೋ ವಸ್ತುಗಳಿರುತ್ತವೆ. ಇವನ್ನೆಲ್ಲಾ ಯಾರು ಸೃಷ್ಟಿಸಿದರು? ಇಚ್ಛೆ ಒಂದು ಮಿಶ್ರ. ಮಿಶ್ರವಾಗಿರುವುದೆಲ್ಲ ಪ್ರಕೃತಿಯ ನಿರ್ಮಾಣ. ಇಚ್ಛೆಯೆ ಪ್ರಕೃತಿಯನ್ನು ಸೃಷ್ಟಿಸಿರಲಾರದು. ಈಶ್ಚರೇಚ್ಛೆ ಪ್ರಪಂಚವನ್ನು ಸೃಷ್ಟಿಸಿತು ಎಂಬುದಕ್ಕೆ ಅರ್ಥವಿಲ್ಲ ಎಂಬುದನ್ನು ನೋಡಿದಹಾಗಾಯಿತು. ನಮ್ಮ ಇಚ್ಛೆಯು ಪ್ರಜ್ಞೆಯಲ್ಲಿ ಒಂದು ಅಂಶವಷ್ಟೆ. ಅದು ಮೆದುಳಿನ ಕೆಲಸಕ್ಕೆ ಕಾರಣ. ನಿಮ್ಮ ದೇಹ ಅಥವಾ ಜಗತ್ತಿನಲ್ಲಿ ಕೆಲಸ ಮಾಡುತ್ತಿರುವುದು ಇಚ್ಛೆಯಲ್ಲ. ಈ ದೇಹವನ್ನು ಒಂದು ಶಕ್ತಿ ಚಲಿಸುತ್ತಿದೆ. ಇಚ್ಛೆ ಎಂಬುದು ಆ ಶಕ್ತಿಯ ಒಂದು ಅಂಶವಷ್ಟೆ. ಇದರಂತೆಯೆ ವಿಶ್ವದಲ್ಲಿಯೂ ಇಚ್ಛೆ ಇದೆ. ಆದರೆ ಇದು ವಿಶ್ವದ ಒಂದು ಅಂಶ ಮಾತ್ರ. ವಿಶ್ವವೆಲ್ಲ ಇಚ್ಛೆಯ ಅಧೀನಕ್ಕೆ ಒಳಪಟ್ಟಿಲ್ಲ. ಆದಕಾರಣವೆ ಇಚ್ಛಾಸಿದ್ಧಾಂತದ ಮೂಲಕ ಇದನ್ನು ವಿವರಿಸಲಾರೆವು. ಇಚ್ಛೆಯೇ ದೇಹವನ್ನು ನಡೆಸುತ್ತಿದೆ ಎಂದು ಭಾವಿಸೋಣ. ಆದರೆ ನನ್ನ ಇಚ್ಛಾನುಸಾರ ಅದನ್ನು ನಡೆಸುವುದಕ್ಕೆ ಆಗದೆ ಇದ್ದರೆ ನಾನು ಗೊಣಗಾಡುತ್ತೇನೆ. ಅದು ನನ್ನ ತಪ್ಪು. ಏಕೆಂದರೆ ಇಚ್ಛಾಸಿದ್ದಾಂತವನ್ನು ನಾನು ಒಪ್ಪಿಕೊಳ್ಳುವುದಕ್ಕೆ ನನಗೆ ಅಧಿಕಾರವಿಲ್ಲ. ಇದರಂತೆಯೆ ವಿಶ್ವವನ್ನೆಲ್ಲಾ ನಡೆಸುತ್ತಿರುವುದು ಇಚ್ಛೆ ಎಂದು ಭಾವಿಸಿ, ಅದಕ್ಕೆ ಸರಿಯಾಗಿ ಅದು ನಡೆಯದೆ ಇದ್ದರೆ ಅದು ನನ್ನ ತಪ್ಪು. ಆದ್ದರಿಂದ ಪುರುಷ ಇಚ್ಚೆಯಲ್ಲ. ಅವನು ಪ್ರಜ್ಞೆಯೂ ಅಲ್ಲ. ಪ್ರಜ್ಞೆಯೂ ಒಂದು ಮಿಶ್ರವಸ್ತು. ಮಿದುಳಿಗೆ ಸಂಬಂಧಪಟ್ಟದು ಯಾವುದಾದರೂ ಇಲ್ಲದೆ ಇದ್ದರೆ ಪ್ರಜ್ಞೆ ಇರುತ್ತಿರಲಿಲ್ಲ. ಎಲ್ಲಿ ಪ್ರಜ್ಞೆ ಇದೆಯೋ ಅಲ್ಲೆಲ್ಲಾ ಅದಕ್ಕೆ ಸಂಬಂಧಪಟ್ಟ ಮಿದುಳು ಎಂದು ನಾವು ಕರೆಯುವ ಯಾವುದಾದರೂ ವಸ್ತು ಇದ್ದೇ ಇರಬೇಕು. ಆದರೆ ಪ್ರಜ್ಞೆಯು ಒಂದು ಮಿಶ್ರ. ಹಾಗಾದರೆ ಪುರುಷ ಎಂದರೆ ಏನು? ಅದು ಪ್ರಜ್ಞೆಯೂ ಅಲ್ಲ, ಇಚ್ಛೆಯೂ ಅಲ್ಲ, ಇವೆರಡಕ್ಕೂ ಕಾರಣ. ಆ ಪುರುಷನ ಸಾನ್ನಿಧ್ಯದಿಂದ ಇವು ಕೆಲಸಮಾಡುವುವು. ಅದು ಪ್ರಕೃತಿಯೊಂದಿಗೆ ಬೆರೆಯುವುದಿಲ್ಲ. ಅದು ಪ್ರಜ್ಞೆಯಲ್ಲ, ಮಹತ್ತಲ್ಲ, ಅದೇ ಪರಿಶುದ್ಧಾತ್ಮ, ಪುರುಷ. “ನಾನು ಸಾಕ್ಷಿ. ನಾನು ಸಾಕ್ಷಿಯಾಗಿರುವುದರಿಂದ ಪ್ರಕೃತಿಯು ಚೇತನಾಚೇತನಗಳನ್ನು ಸೃಷ್ಟಿಸುತ್ತಿರುವುದು.”

ಪ್ರಕೃತಿಯಲ್ಲಿ ಚೇತನವೆನ್ನುವುದಾವುದು? ಈ ಚೇತನ ಎನ್ನುವುದು ಪ್ರಜ್ಞೆ, ಇದನ್ನೆ ನಾವು ಚಿತ್ ಎನ್ನುವುದು. ಈ ಚಿತ್ತಿನ ಮೂಲ ಪುರುಷನಲ್ಲಿದೆ. ಇದೇ ಪುರುಷನ ಸ್ವಭಾವ. ಇದನ್ನು ನಾವು ವಿವರಿಸುವುದಕ್ಕೆ ಆಗುವುದಿಲ್ಲ. ಆದರೆ ಜ್ಞಾನಕ್ಕೆಲ್ಲಾ ಇದೇ ಮೂಲಕಾರಣ. ಪುರುಷ ಪ್ರಜ್ಞೆಯಲ್ಲ. ಏಕೆಂದರೆ ಪ್ರಜ್ಞೆ ಒಂದು ಮಿಶ್ರ. ಆದರೆ ಪ್ರಜ್ಞೆಯಲ್ಲಿ ಯಾವುದು ಬೆಳಕಿನ ಅಂಶವೋ ಅದು ಪುರುಷನಿಗೆ ಸೇರಿದ್ದು, ಪ್ರಜ್ಞೆ ಪುರುಷನಲ್ಲಿದೆ. ಆದರೆ ಪುರುಷ ಪ್ರಜ್ಞೆಯೂ ಅಲ್ಲ, ಜ್ಞಾನವೂ ಅಲ್ಲ. ಪ್ರಕೃತಿ ಮತ್ತು ಪುರುಷನಲ್ಲಿರುವ ಚಿತ್ – ಇವೆರಡರ ಮಿಶ್ರಣವನ್ನು ನಾವು ಸುತ್ತಲೂ ನೋಡುವುದು. ವಿಶ್ವದಲ್ಲಿ ಯಾವುದು ಸುಖವಾಗಿದೆಯೋ ಆನಂದವಾಗಿದೆಯೊ, ಜ್ಞಾನದಂತೆ ಇದೆಯೋ ಅದೆಲ್ಲ ಪುರುಷನಿಗೆ ಸೇರಿದ್ದು. ಆದರೆ ಇದೆಲ್ಲ ಮಿಶ್ರ. ಏಕೆಂದರೆ ಇದರಲ್ಲಿ ಪುರುಷ ಪ್ರಕೃತಿಗಳೆರಡೂ ಸೇರಿವೆ. “ಎಲ್ಲಿ ಸಂತೋಷವಿದೆಯೊ, ಆನಂದವಿದೆಯೊ ಅಲ್ಲೆಲ್ಲಾ ಅಮೃತತ್ವದ ಅಂಶವಿದೆ. ಅದೇ ದೇವರು.'' ವಿಶ್ವಕ್ಕೆ ದೊಡ್ಡ ಆಕರ್ಷಣೆಯೆ ಪುರುಷ. ಪುರುಷ ಪ್ರಕೃತಿಗೆ ಸೇರಿಲ್ಲದೆ ಇದ್ದರೂ ಅವನು ಇಡೀ ವಿಶ್ವವನ್ನು ಆಕರ್ಷಿಸುವನು. ಚಿನ್ನಕ್ಕಾಗಿ ಧಾವಿಸುತ್ತಿರುವವನನ್ನು ನೀವು ನೋಡುವಿರಿ. ಅವನಲ್ಲಿ ಪುರುಷನ ಒಂದು ಅಂಶವಿದೆ. ಆದರೆ ಮೇಲೆ ಬೇಕಾದಷ್ಟು ಕೊಳೆ ಕುಳಿತಿರಬಹುದು. ಮನುಷ್ಯ ತನ್ನ ಮಕ್ಕಳನ್ನು ಪ್ರೀತಿಸುವಾಗ, ಹೆಂಡತಿ ತನ್ನ ಗಂಡನನ್ನು ಪ್ರೀತಿಸುವಾಗ, ಯಾವುದು ಆಕರ್ಷಿಸುವುದು? ಅದರ ಹಿಂದೆ ಇರುವ ಪುರುಷನ ಅಂಶ. ಅಲ್ಲಿ ಅದು ಕೊಳೆಯೊಂದಿಗೆ ಮಿಶ್ರವಾಗಿದೆ ಅಷ್ಟೆ. ಮತ್ತಾವುದೂ ಆಕರ್ಷಿಸಲಾರದು. “ಈ ಜಡ ಜಗತ್ತಿನಲ್ಲಿ ಪುರುಷನೊಬ್ಬನೇ ಚೇತನ'. ಇದೇ ಸಾಂಖ್ಯರ ಪುರುಷ. ಆದಕಾರಣ ಪುರುಷ ಸರ್ವ ವ್ಯಾಪಿಯಾಗಿರಬೇಕೆಂಬುದು ಇದರಿಂದ ಸಿದ್ಧಿಸಿದಂತೆ ಆಯಿತು. ಯಾವುದು ಸರ್ವವ್ಯಾಪಿಯಲ್ಲವೋ ಅದು ಮಿತವಾಗಿರಬೇಕು. ಎಲ್ಲ ಮಿತಿಯೂ ಕಾರಣ ಉಳ್ಳದ್ದು. ಯಾವುದು ಕಾರಣ ಉಳ್ಳದ್ದೋ ಅದಕ್ಕೆ ಆದಿ ಅಂತಗಳಿರಬೇಕು. ಪುರುಷ ಮಿತನಾಗಿದ್ದರೆ ಅವನು ಸಾಯುವನು, ಸ್ವತಂತ್ರನಾಗುವುದಿಲ್ಲ, ಅವನೇ ಕೊನೆಯಾಗುವುದಿಲ್ಲ. ಅವನಿಗೆ ಮತ್ತಾವುದೊ ಕಾರಣವಿರಬೇಕಾಗುತ್ತದೆ. ಆದಕಾರಣ ಅವನು ಸರ್ವವ್ಯಾಪಿ. ಕಪಿಲನ ಪ್ರಕಾರ ಪುರುಷರು ಅನೇಕ, ಒಬ್ಬನೇ ಅಲ್ಲ, ಅನಂತ ಸಂಖ್ಯೆಯ ಪುರುಷರಿರುವರು. ನಮಗೆ ನಿಮಗೆಲ್ಲಾ ಒಬ್ಬೊಬ್ಬ ಪುರುಷ ಇರುವನು. ಹಾಗೆಯೇ ಎಲ್ಲರಿಗೂ. ಅದರಂತೆಯೆ ಅನಂತ ಸಂಖ್ಯೆಯ ವೃತ್ತಗಳು. ಪ್ರತಿಯೊಂದೂ ಅನಂತವಾಗಿದೆ. ಇವೆಲ್ಲ ವಿಶ್ವದಲ್ಲಿವೆ. ಪುರುಷ ದ್ರವ್ಯವೂ ಅಲ್ಲ, ಮನಸ್ಸೂ ಅಲ್ಲ, ಅವನ ಪ್ರತಿಬಿಂಬದ ಕಾಂತಿಯೊಂದೆ ನಮಗೆ ಕಾಣುವುದು. ಅದು ಸರ್ವವ್ಯಾಪಿಯಾದರೆ ಅದಕ್ಕೂ ಜನನ ಮರಣಗಳಿಲ್ಲ ಎಂಬುದು ನಿಶ್ಚಿತವಾದಂತೆ ಆಯಿತು. ಪ್ರಕೃತಿ ತನ್ನ ಜನನಮರಣಗಳ ಛಾಯೆಯನ್ನು ಅದರ ಮೇಲೆ ಕೆಡಹುತ್ತಿದೆ ಅಷ್ಟೆ. ಆದರೆ ಪುರುಷ ಸ್ವಭಾವತಃ ಪರಿಶುದ್ದ. ಇಲ್ಲಿಯವರೆಗೆ ಸಾಂಖ್ಯರ ತತ್ತ್ವ ಅದ್ಭುತವಾಗಿರುವುದನ್ನು ನೋಡಿದೆವು.

ಅನಂತರ ಇದಕ್ಕೆ ವಿರೋಧವಾಗಿರುವ ಕೆಲವು ಪ್ರಮಾಣಗಳನ್ನು ತೆಗೆದುಕೊಳ್ಳೋಣ. ಇಲ್ಲಿಯವರೆಗೆ ಅವರ ವಿಶ್ಲೇಷಣೆ ಪರಿಪೂರ್ಣವಾಗಿದೆ. ಅವರ ಮನಶ್ಶಾಸ್ತ್ರವನ್ನು ಯಾರೂ ಪ್ರಶ್ನಿಸುವಂತೆಯೇ ಇಲ್ಲ. ಇಂದ್ರಿಯಗಳನ್ನು ಆಂತರಿಕ ಕೇಂದ್ರಗಳು ಮತ್ತು ಬಾಹ್ಯಕರಣಗಳು ಎಂದು ವಿಭಾಗಿಸುವುದರ ಮೂಲಕ ಅವು ಕೇವಲವಲ್ಲ, ಮಿಶ್ರ ಎಂಬುದನ್ನು ನೋಡಿದೆವು. ಅಹಂಕಾರವನ್ನು ಇಂದ್ರಿಯಗಳು ಮತ್ತು ಅದಕ್ಕೆ ಸಂಬಂಧಪಟ್ಟ ದ್ರವ್ಯ ಎಂದು ವಿಭಾಗಿಸಿ ಆಯಿತು. ಇದೂ ಕೂಡ ಭೌತಿಕ. ಮಹತ್ ಕೂಡ ವಸ್ತುವಿನ ಒಂದು ಸ್ಥಿತಿ ಎಂಬುದನ್ನು ನೋಡಿದೆವು. ಅನಂತರ ಪುರುಷನನ್ನು ನೋಡುವೆವು. ಇಲ್ಲಿಯವರೆಗೆ ಯಾವ ಅಭ್ಯಂತರವೂ ಇಲ್ಲ. ಆದರೆ ನಾವು ಸಾಂಖ್ಯರನ್ನು, ಪ್ರಕೃತಿಯನ್ನು ಯಾರು ಸೃಷ್ಟಿಸಿದರು ಎಂದು ಪ್ರಶ್ನಿಸಿದರೆ ಅವರು ಪುರುಷ ಮತ್ತು ಪ್ರಕೃತಿ ಇಬ್ಬರೂ ಅನಾದಿ ಮತ್ತು ಸರ್ವವ್ಯಾಪಿಗಳು ಮತ್ತು ಅನಂತ ಪುರುಷರು ಇರುವರು ಎನ್ನುವರು. ಅವರ ಈ ಹೇಳಿಕೆಗಳನ್ನು ನಾವು ವಿರೋಧಿಸಬೇಕಾಗಿದೆ. ಅದಕ್ಕೆ ಬೇರೊಂದು ಸಮರ್ಪಕವಾದ ಉತ್ತರವನ್ನು ಕಂಡುಹಿಡಿಯಬೇಕಾಗಿದೆ. ಹಾಗೆ ಮಾಡಿದರೆ ನಾವು ಅದ್ವೈತಕ್ಕೆ ಬರುವೆವು. ನಮ್ಮ ಮೊದಲನೆ ಅಭ್ಯಂತರವೆ ಈ ಎರಡು ಅನಂತಗಳು ಇರುವುದಕ್ಕೆ ಹೇಗೆ ಸಾಧ್ಯ ಎಂಬುವುದು. ಅನಂತರ ನಮ್ಮ ವಾದವೆ ಸಾಂಖ್ಯ ತತ್ತ್ವವು ಪೂರ್ಣ ಸಾಮಾನ್ಯೀಕರಣವಲ್ಲ, ಅದರಲ್ಲೂ ನಮಗೆ ತೃಪ್ತಿಕರವಾದ ಉತ್ತರ ದೊರೆಯುವುದಿಲ್ಲ ಎಂಬುದು. ವೇದಾಂತಿಗಳು ಹೇಗೆ ಈ ಸಮಸ್ಯೆಗಳನ್ನೆಲ್ಲಾ ಎದುರಿಸಿ ಒಂದು ಸಮಾಧಾನಕರವಾದ ಸಿದ್ದಾಂತವನ್ನು ರೂಪಿಸುವರು ಎಂಬುದನ್ನು ನೋಡುವೆವು. ಆದರೆ ನಿಜವಾಗಿ ಸಾಂಖ್ಯರಿಗೆ ಕೀರ್ತಿಯೆಲ್ಲಾ ಸೇರುವುದು. ಒಂದು ಮನೆಯನ್ನು ಕಟ್ಟಿದ ಮೇಲೆ ಅದಕ್ಕೆ ಸುಣ್ಣ ಬಳಿದು ಮುಗಿಸುವುದು ಸುಲಭ.

