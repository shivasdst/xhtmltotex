
\chapter[ಧರ್ಮದ ಹಕ್ಕುದಾರಿಕೆ]{ಧರ್ಮದ ಹಕ್ಕುದಾರಿಕೆ\protect\footnote{\engfoot{C.W, Vol. IV, P. 203}}}

\begin{center}
(ಭಾನುವಾರ, ೫ನೇ ಜನವರಿ)
\end{center}

ನೀವು ಮಕ್ಕಳಾಗಿದ್ದಾಗ ಮನೋಹರವಾದ ಸೂರ್ಯೋದಯವನ್ನು ನೋಡಿ\break ಎಂತಹ ಒಂದು ಆನಂದದಲ್ಲಿ ಭಾಗಿಗಳಾಗಿದ್ದಿರಿ ಎಂಬುದು ನಿಮ್ಮಲ್ಲಿ ಹಲವರಿಗೆ ನೆನಪಿರಬಹುದು. ನಿಮ್ಮಲ್ಲಿ ಎಲ್ಲರೂ ಜೀವನದಲ್ಲಿ ಒಂದಲ್ಲ ಒಂದು ಬಾರಿಯಾದರೂ ಮನೋಹರವಾದ ಸೂರ್ಯಾಸ್ತಮಯವನ್ನು ನೋಡಿರುವಿರಿ. ಕಲ್ಪನೆಯಲ್ಲಾದರೂ ಅದರ ಆಚೆ ಏನಿದೆ ಎಂದು ಊಹಿಸಲೂ ಯತ್ನಿಸಿರುವಿರಿ. ನಿಜವಾಗಿಯೂ ವಿಶ್ವದ ಹಿಂದೆಲ್ಲಾ ಇರುವುದು ಇದೇ. ಇದರಿಂದಲೇ ಎಲ್ಲವೂ ಉದ್ಭವಿಸಿ ಕೊನೆಗೆ ಇದರಲ್ಲಿಯೇ ಲೀನವಾಗುವುದು. ಇಡೀ ವಿಶ್ವವೂ ಅವ್ಯಕ್ತದಿಂದ ಅಭಿವ್ಯಕ್ತಗೊಂಡು ಅವ್ಯಕ್ತದಲ್ಲಿಯೇ ಲೀನವಾಗುವುದು. ಕತ್ತಲೆಯ ಗರ್ಭದಿಂದ ಮಗು ತೆವಳುತ್ತ ಬರುವುದು; ಪುನಃ ವೃದ್ಧನಾದಮೇಲೆ ಕತ್ತಲೆಯ ಕಡೆಗೆ ತೆವಳುತ್ತ ಹೋಗುವುದು.

ನಮ್ಮ ಜಗತ್ತು, ಪಂಚೇಂದ್ರಿಯಗಳಿಗೆ ಗೋಚರವಾದ ಜಗತ್ತು, ಯುಕ್ತಿ ಪೂರಿತವಾದ, ಬೌದ್ಧಿಕವಾದ ಈ ಜಗತ್ತು ಎರಡು ಕಡೆಗಳಲ್ಲಿಯೂ ಅನಂತವಾದ ಮತ್ತು ಅಜೇಯವಾದ ವಸ್ತುವಿನಿಂದ ಸುತ್ತುವರಿಯಲ್ಪಟ್ಟಿದೆ. ಇದನ್ನೇ ನಾವು ಅನ್ವೇಷಣೆ ಮಾಡಬೇಕಾಗಿದೆ. ಇಲ್ಲಿಯೇ ನಾವು ಪ್ರಶ್ನೆಗಳನ್ನು ಕೇಳಬೇಕಾಗಿದೆ. ಇಲ್ಲಿಯೇ ಸತ್ಯಾಂಶಗಳಿರುವುದು. ಇಲ್ಲಿಂದ ಬಂದ ಬೆಳಕನ್ನೇ ಪ್ರಪಂಚದಲ್ಲಿ ಧರ್ಮವೆಂದು ಕರೆಯುವುದು. ಮುಖ್ಯವಾಗಿ ಧರ್ಮವು ಅತೀಂದ್ರಿಯಕ್ಕೆ ಸಂಬಂಧಪಟ್ಟದ್ದು, ಎಂದಿಗೂ ಇಂದ್ರಿಯ ಪ್ರಪಂಚಕ್ಕೆ ಸಂಬಂಧಪಟ್ಟದಲ್ಲ. ಅದು ಯುಕ್ತಿಗೆ ಅತೀತವಾಗಿರುವುದು, ಬುದ್ದಿಯ ಕ್ಷೇತ್ರಕ್ಕೆ ಅತೀತವಾಗಿರುವುದು. ಇದುವರೆಗೆ ಅಜ್ಞೇಯವಾದ ಮತ್ತು ಎಂದೆಂದಿಗೂ ನಮಗೆ ಜ್ಞೇಯವಾಗಲಾರದಾದ ಒಂದು ದರ್ಶನವನ್ನು, ಸ್ಫೂರ್ತಿಯನ್ನು ಅನುಭವಿಸುವುದು ಆಗಿದೆ. ಧರ್ಮವು ಅಜ್ಞೇಯವಾದುದನ್ನು ಜ್ಞೇಯಕ್ಕಿಂತಲೂ ಶ್ರೇಷ್ಠವನ್ನಾಗಿ ಮಾಡುವುದು. ಏಕೆಂದರೆ ಅದು ಎಂದೆಂದಿಗೂ ಜ್ಞೇಯವಾಗುವುದಿಲ್ಲ. ಮಾನವ ಜನಾಂಗದ ಆದಿಯಿಂದಲೂ ಈ ಅನ್ವೇಷಣೆ ಮಾನವನಲ್ಲಿದೆ ಎಂದು ನಾನು ನಂಬುತ್ತೇನೆ. ಈ ಅತೀತವಾದುದರ ಸಂಶೋಧನೆ, ಇದನ್ನು ಅರಿಯುವುದಕ್ಕೆ ಮಾಡಿದ ಹೋರಾಟ, ಬುದ್ದಿಯ ಮೂಲಕ ಅವನು ಮಾಡಿದ ಸರ್ವಪ್ರಯತ್ನಗಳು ಮಾನವ ಜನಾಂಗದ ಇತಿಹಾಸದಲ್ಲಿ ಇಲ್ಲದ ಕಾಲವೇ ಇಲ್ಲ. ಮನುಷ್ಯನ ಮನಸ್ಸು ಎಂಬ ನಮ್ಮ ಈ ಸಣ್ಣ ಪ್ರಪಂಚದಲ್ಲಿ ಒಂದು ಆಲೋಚನೆಯು ಏಳುವುದನ್ನು ನಾವು ನೋಡುತ್ತೇವೆ. ಇದು ಎಲ್ಲಿಂದ ಏಳುವುದು ಎಂಬುದು ನಮಗೆ ಗೊತ್ತಿಲ್ಲ. ಅದು ಮಾಯವಾದಮೇಲೆ ಎಲ್ಲಿಗೆ ಹೋಗುವುದೋ ಅದೂ ನಮಗೆ ಗೊತ್ತಿಲ್ಲ. ಬ್ರಹ್ಮಾಂಡ, ಪಿಂಡಾಂಡಗಳು ಒಂದೇ ಜಾಡಿನಲ್ಲಿ ಇರುವಂತೆ ಕಾಣುವುವು; ಒಂದೇ ರೀತಿ ಇರುವಂತೆ ಒಂದೇ ರೀತಿ ಸ್ಪಂದಿಸುತ್ತಿರುವಂತೆ ನಮಗೆ ಕಾಣುವುದು.

ಧರ್ಮವು ಹೊರಗಿನಿಂದ ಬರುವುದಿಲ್ಲ, ಒಳಗಿನಿಂದ ಬರುವುದು ಎಂಬ ಹಿಂದೂಗಳ ಸಿದ್ದಾಂತವನ್ನು ನಿಮಗೆ ತಿಳಿಸಲು ಯತ್ನಿಸುತ್ತೇನೆ. ಧಾರ್ಮಿಕ ಭಾವನೆ ಮನುಷ್ಯನ ಸ್ವಭಾವದಲ್ಲೇ ಅಡಗಿದೆ. ಅವನು ತನ್ನ ಮನಸ್ಸು ಮತ್ತು ದೇಹಗಳನ್ನು ತ್ಯಜಿಸುವವರೆಗೆ, ಆಲೋಚನೆ ಮತ್ತು ಪ್ರಾಣವನ್ನು ನಿಲ್ಲಿಸುವವರೆಗೆ, ಧರ್ಮವನ್ನು ನಿಲ್ಲಿಸಲಾರ ಎಂಬುದು ನನ್ನ ನಂಬಿಕೆ. ಎಲ್ಲಿಯವರೆಗೆ ಮನುಷ್ಯ ಆಲೋಚಿಸುವ ಸ್ಥಿತಿಯಲ್ಲಿರುವನೋ ಅಲ್ಲಿಯವರೆಗೆ ಈ\break ಹೋರಾಟ ಮುಂದೆ ಸಾಗುವುದು. ಅಲ್ಲಿಯವರೆಗೆ ಅವನಿಗೆ ಯಾವುದಾದರೂ ಒಂದು ಬಗೆಯ ಧರ್ಮ ಇರುವುದು. ಆದಕಾರಣ ಈ ಪ್ರಪಂಚದಲ್ಲಿ ಹಲವು ಬಗೆಯ ಧರ್ಮಗಳು ಇರುವುವು. ಇದೊಂದು ದಿಗ್ಭ್ರಮೆ ಹಿಡಿಸುವಂತಹ ಅಧ್ಯಯನ. ಆದರೆ ನಮ್ಮಲ್ಲಿ ಅನೇಕರು ಭಾವಿಸುವಂತೆ ಇದೊಂದು ನಿಷ್ಪ್ರಯೋಜಕವಾದ ಊಹೆಯಲ್ಲ. ಈ ಗೊಂದಲದ ಮಧ್ಯದಲ್ಲಿ ಒಂದು ಸಾಮರಸ್ಯವಿದೆ; ಅಪಶ್ರುತಿಗಳ ಮಧ್ಯದಲ್ಲಿ ಒಂದು ಶ್ರುತಿ ಇದೆ. ಯಾರಿಗೆ ಇದನ್ನು ಕೇಳಲು ತಾಳ್ಮೆಯಿದೆಯೋ ಅವರಿಗೆ ಇದು ವೇದ್ಯವಾಗುವುದು.

ಈಗಿನ ಕಾಲದಲ್ಲಿ ಎಲ್ಲಕ್ಕಿಂತಲೂ ಮುಖ್ಯವಾದ ಪ್ರಶ್ನೆಯೇ ಇದು: ಈಗ ನಮಗೆ ತಿಳಿದಿರುವುದು ಮತ್ತು ಮುಂದೆ ತಿಳಿಯುವುದಕ್ಕೆ ಸಾಧ್ಯವಾಗಿರುವುದು – ಇವು ಅಜೇಯ\-ವಾದುದು ಮತ್ತು ಮುಂದೆ ಎಂದೆಂದಿಗೂ ವೇದ್ಯವೇ ಆಗದೆ ಇರುವುದರಿಂದ ಸುತ್ತುವರಿಯಲ್ಪಟ್ಟಿದೆ ಎಂದು ಇಟ್ಟುಕೊಂಡರೂ, ಅದನ್ನು ತಿಳಿಯುವುದಕ್ಕೆ ಏತಕ್ಕೆ ಪ್ರಯತ್ನಿಸುವುದು? ನಮಗೆ ತಿಳಿದಿರುವ ವಸ್ತುಗಳಲ್ಲೇ ಏತಕ್ಕೆ ತೃಪ್ತರಾಗಬಾರದು? ತಿನ್ನುವುದು, ಕುಡಿಯುವುದು, ಸ್ವಲ್ಪ ಸಮಾಜಸೇವೆ ಇವಿಷ್ಟರಲ್ಲೇ ನಾವೇಕೆ ತೃಪ್ತರಾಗಿರಬಾರದು? ಈ ಭಾವನೆ ಈಗ ಪ್ರಚಲಿತವಾಗಿದೆ. ಘನ ವಿದ್ವಾಂಸನಿಂದ ಹಿಡಿದು ಹರಟುವ ಹುಡುಗನವರೆಗೆ “ಪ್ರಪಂಚಕ್ಕೆ ಒಳ್ಳೆಯದನ್ನು ಮಾಡಿ, ಧರ್ಮವೆಂದರೆ ಇದೇ. ಅನಂತರ ಏನು ಎಂಬುದನ್ನು ಕುರಿತು ಚಿಂತಿಸಿ ನಿನ್ನ ತಲೆಯನ್ನು ಕೆಡಿಸಿಕೊಳ್ಳಬೇಡ” ಎಂಬುದು ಎಷ್ಟು ಪ್ರಚಲಿತವಾಗಿ\-ದೆಯೆಂದರೆ, ಇದೊಂದು ನಿರ್ವಿವಾದ ಸತ್ಯದಂತೆ ಇದೆ.

ಆದರೆ ಅದೃಷ್ಟವಶಾತ್ ಅತೀತವಾದುದನ್ನು ಕುರಿತು ವಿಚಾರಿಸದೆ ನಮಗೆ ಸಾಧ್ಯವೇ ಇಲ್ಲ. ಈಗ ನಮಗೆ ಗೊತ್ತಿರುವುದು, ವ್ಯಕ್ತವಾಗಿರುವುದು, ಅವ್ಯಕ್ತದ ಯಾವುದೋ ಒಂದು ಅಂಶವಾಗಿದೆ. ನಮಗೆ ತಿಳಿದಿರುವ ಇಂದ್ರಿಯ ಪ್ರಪಂಚ ಅನಂತವಾದ ಆಧ್ಯಾತ್ಮಿಕ\break ಪ್ರಪಂಚದ ಒಂದು ಸಣ್ಣ ಅಂಶ ಮಾತ್ರವಾಗಿದೆ. ಅದರ ಯಾವುದೋ ಒಂದು ಚೂರು ಮಾತ್ರ ಇಂದ್ರಿಯ ಪ್ರಪಂಚವಾಗಿ ತೋರಿಕೊಳ್ಳುತ್ತಿದೆ. ಯಾವುದು ಅತೀತವಾಗಿರುವುದೋ ಅದನ್ನು ತಿಳಿದುಕೊಳ್ಳದೆ ನಮಗೆ ತೋರುವ ಒಂದು ತುಣುಕನ್ನು ವಿವರಿಸುವುದು ಹೇಗೆ? ಒಂದು ದಿನ ಸಾಕ್ರಟೀಸ್, ಅಥೆನ್ಸ್ ನಗರದಲ್ಲಿ ಮಾತನಾಡುತ್ತಿದ್ದಾಗ ಭಾರತದಿಂದ ಅಲ್ಲಿಗೆ ಹೋಗಿದ್ದ ಒಬ್ಬ ಬ್ರಾಹ್ಮಣನನ್ನು ಕಂಡನು. ಸಾಕ್ರಟೀಸ್ ಬ್ರಾಹ್ಮಣನಿಗೆ, “ಮಾನವ ಕೋಟಿಯ ಶ್ರೇಷ್ಠವಾದ ಅಧ್ಯಯನ ಎಂದರೆ ಮಾನವನೇ ಆಗಿರುವನು” ಎಂದನು. ಅದಕ್ಕೆ ಬ್ರಾಹ್ಮಣನು ತಕ್ಷಣವೇ, “ದೇವರನ್ನು ಮೊದಲು ತಿಳಿಯದೆ ಮನುಷ್ಯನನ್ನು ಹೇಗೆ ತಿಳಿಯಬಲ್ಲೆ?'' ಎಂದು ಉತ್ತರ ಕೊಟ್ಟನು. ಈ ದೇವರು ನಿತ್ಯ ಅಜ್ಞೇಯವಾದುದು ಅಥವಾ ನಿರಪೇಕ್ಷವಾದುದು, ಅನಂತವಾದುದು ಅಥವಾ ನಾಮರಹಿತವಾದುದು. ಇದನ್ನು ನೀವು ಯಾವ ಹೆಸರಿನಿಂದ ಬೇಕಾದರೂಕರೆಯಬಹುದು. ಇದು ಮಾತ್ರ ನಿಮಗೆ ಈಗ ತಿಳಿದಿರುವ, ಮುಂದೆ ತಿಳಿಯಬಹುದಾದ ಜಗತ್ತಿಗೆ ಮತ್ತು ಈ ಜೀವನಕ್ಕೆ ಯುಕ್ತಿಪೂರಿತವಾದ ವಿವರಣೆಯನ್ನು ಕೊಡಬಲ್ಲದು. ನಿಮ್ಮ ಎದುರಿಗಿರುವ ಯಾವುದಾದರೂ ವಸ್ತುವನ್ನು ತೆಗೆದುಕೊಳ್ಳಿ. ತುಂಬಾ ಸ್ಥೂಲವಾಗಿರುವುದನ್ನು ಬೇಕಾದರೆ ತೆಗೆದುಕೊಳ್ಳಿ; ರಸಾಯನ ಶಾಸ್ತ್ರವನ್ನೇ ತೆಗೆದುಕೊಳ್ಳಿ, ಭೌತಶಾಸ್ತ್ರವನ್ನೇ ತೆಗೆದುಕೊಳ್ಳಿ, ಖಗೋಳಶಾಸ್ತ್ರವನ್ನೇ ತೆಗೆದುಕೊಳ್ಳಿ ಅಥವಾ ಜೀವ ವಿಜ್ಞಾನವನ್ನೇ ತೆಗೆದುಕೊಳ್ಳಿ. ಅದನ್ನು ವಿಶ್ಲೇಷಣೆ ಮಾಡಿ ತಿಳಿದುಕೊಳ್ಳುತ್ತ ಮುಂದೆ ಮುಂದೆ ಹೋಗಿ, ಅದರ ಸ್ಥೂಲವಾದ ಆಕಾರ ಮೊದಲು ಮಾಯವಾಗುತ್ತ ಬರುವುದು. ಅನಂತರ ಸೂಕ್ಷ್ಮ ಸೂಕ್ಷ್ಮವಾಗುತ್ತ ಬರುವುದು. ಅಂತೂ ನಾವು ಭೌತಿಕ ಪ್ರಪಂಚದಿಂದ ಭೌತಾತೀತ ಪ್ರಪಂಚಕ್ಕೆ ಹಾರಬೇಕಾದ ಸಮಯವು ಬರುವುದು. ಸ್ಥೂಲವು ಸೂಕ್ಷ್ಮದಲ್ಲಿ ಪರ್ಯವಸಾನವಾಗುವುದು. ಭೌತಶಾಸ್ತ್ರವು ತತ್ತ್ವಶಾಸ್ತ್ರದಲ್ಲಿ ಪರ್ಯವಸಾನವಾಗಬೇಕಾಗುವುದು. ಪ್ರತಿಯೊಂದು ಜ್ಞಾನಕ್ಷೇತ್ರದಲ್ಲಿಯೂ ಹೀಗೆಯೆ ಇದೆ.

ಇದರಂತೆಯೇ ನಮ್ಮಲ್ಲಿ ಎಲ್ಲವೂ – ನಮ್ಮ ಸಮಾಜ, ನಮ್ಮ ಪರಸ್ಪರ ಸಂಬಂಧ, ನಮ್ಮ ಧರ್ಮ ಮತ್ತು ನೀವು ಯಾವುದನ್ನು ನೀತಿ ಎಂದು ಕರೆಯುವಿರೋ ಇವುಗಳೆಲ್ಲ ಕೇವಲ ಪ್ರಯೋಜನ ದೃಷ್ಟಿಯ ಆಧಾರದ ಮೇಲೆಯೇ ನೀತಿಯನ್ನು ವಿವರಿಸುವ ಪ್ರಯತ್ನಗಳಾಗಿವೆ. ಆದರೆ ಯುಕ್ತಿಯುತವಾದ ಒಂದು ನೀತಿ ಸಿದ್ಧಾಂತವನ್ನು ಮಂಡಿಸಲು ನಿಮಗೆ ಸಾಧ್ಯವೇ ಎಂದು ಸವಾಲು ಹಾಕುತ್ತೇನೆ. ಇತರರಿಗೆ ಒಳ್ಳೆಯದನ್ನು ಮಾಡಿ; ಏತಕ್ಕೆಂದರೆ ಇದೇ ಸರ್ವ ಶ್ರೇಷ್ಠವಾದ ಪ್ರಯೋಜನ ಎನ್ನುವರು. ಬಹುಶಃ ಮತ್ತೊಬ್ಬ, “ನನಗೆ ಪ್ರಯೋಜನ ಬೇಕಾಗಿಲ್ಲ, ಇನ್ನೊಬ್ಬರ ತಲೆಯನ್ನು ಒಡೆದು ನಾನು ಶ‍್ರೀಮಂತ\-ನಾಗಬೇಕು'' ಎಂದು ಹೇಳಿದರೆ ನೀವು ಏನು ಹೇಳುತ್ತೀರಿ? ಇದು ಗುರುವಿಗೆ ತಿರುಮಂತ್ರ ಹೇಳಿದಂತೆ. ಈ ಪ್ರಪಂಚಕ್ಕೆ ಒಳ್ಳೆಯದನ್ನು ಮಾಡಿದರೆ ನನಗೆ ಬಂದ ಪ್ರಯೋಜನವೇನು? ಇತರರು ಸುಖವಾಗಿರಲಿ ಎಂದು ನಾನು ಕಷ್ಟಪಡುವುದಕ್ಕೆ ನಾನೇನು ಮೂರ್ಖನೆ? ಸಮಾಜಕ್ಕೆ ಅತೀತವಾಗಿರುವುದು ಮತ್ತಾವುದೂ ಇಲ್ಲದೇ ಇದ್ದರೆ, ಪಂಚೇಂದ್ರಿಯಗಳಿಗೆ ಅತೀತವಾದ ಯಾವ ಶಕ್ತಿಯೂ ಇಲ್ಲದೇ ಇದ್ದರೆ, ನಾನೇ ಏತಕ್ಕೆ ಸುಖಪಡಬಾರದು? ನನ್ನ ಸಹೋದರನ ಕತ್ತನ್ನು ಕತ್ತರಿಸಿ ಪೋಲಿಸಿನವರಿಗೆ ಸಿಕ್ಕದಂತೆ ನೋಡಿಕೊಂಡು ನಾನೇ ಏತಕ್ಕೆ ಸುಖವಾಗಿರಬಾರದು? ಇದಕ್ಕೆ ನೀವು ಏನು ಉತ್ತರ ಕೊಡುತ್ತೀರಿ? ನೀವು ಇರುವ ಸ್ಥಳದಿಂದ ನಿಮ್ಮನ್ನು ಕದಲಿಸಿದರೆ ನೀವು “ಸ್ನೇಹಿತನೇ, ಒಳ್ಳೆಯದಕ್ಕಾಗಿಯೇ ಒಳ್ಳೆಯವನಾಗಿರಬೇಕು" ಎಂದು ಹೇಳುತ್ತೀರಿ. ಒಳ್ಳೆಯದನ್ನು ಮಾಡುವುದು ಒಳ್ಳೆಯದು ಎಂದು ವ್ಯಕ್ತಿಯಲ್ಲಿರುವ ಯಾವುದು ಹೇಳುವುದು? ಇದು ನಮಗೆ ಆತ್ಮನ ಮಹಿಮೆಯನ್ನು ತೋರುವುದು. ಸೌಂದರ್ಯವನ್ನು, ಎಲ್ಲವನ್ನೂ ಆಕರ್ಷಿಸುವ ಒಳ್ಳೆಯದರ ಶಕ್ತಿಯನ್ನು, ಒಳ್ಳೆಯತನದಲ್ಲಿರುವ ಅನಂತವಾದ ಶಕ್ತಿಯನ್ನು, ಯಾವುದು ತೋರುವುದು? ನಾವು ದೇವರು ಎಂದು ಕರೆಯುವುದು ಅದನ್ನೇ ಅಲ್ಲವೇ?

ಎರಡನೆಯದಾಗಿ, ನಾನು ಮತ್ತೂ ಒಂದು ಸೂಕ್ಷ್ಮವಾದ ವಿಷಯವನ್ನು\break ಹೇಳಬೇಕಾಗಿದೆ. ನೀವು ಅದನ್ನು ಏಕಾಗ್ರತೆಯಿಂದ ಕೇಳಬೇಕು. ನಾನು ಹೇಳಿರುವುದರ ಮೇಲೆ ನೀವು ದೀರ್ಘವಾದ ಆಲೋಚನೆ ಮಾಡದೇ ಯಾವ ತೀರ್ಮಾನಕ್ಕೂ ಬರಬೇಡಿ. ಈ ಪ್ರಪಂಚಕ್ಕೆ ನಾವೇನೂ ಹೆಚ್ಚು ಒಳ್ಳೆಯದನ್ನು ಮಾಡಲಾರೆವು, ಪ್ರಪಂಚಕ್ಕೆ ಒಳ್ಳೆಯದನ್ನು ಮಾಡುವುದೇನೋ ಒಳ್ಳೆಯದು. ಆದರೆ ಪ್ರಪಂಚಕ್ಕೆ ನಾವು ಒಳ್ಳೆಯದನ್ನು ಮಾಡಬಲ್ಲೆವೇ? ನಾವು ನೂರಾರು ವರುಷಗಳಿಂದ ಅದಕ್ಕಾಗಿ ಹೋರಾಡಿ ಏನನ್ನಾದರೂ ಒಳ್ಳೆಯದನ್ನು ಮಾಡಿರುವೆವೆ? ಪ್ರಪಂಚದ ಸುಖದ ಮೊತ್ತವನ್ನು ನಾವೇನಾದರೂ ವೃದ್ಧಿ ಮಾಡಿರುವವೆ? ಪ್ರಪಂಚದ ಸುಖವನ್ನು ವೃದ್ಧಿಮಾಡಲು ನೂರಾರು ದಾರಿಗಳನ್ನು ಪ್ರತಿದಿನ ಹೊಸ ಹೊಸದಾಗಿ ಮಾಡಿರುವರು. ಇದು ನೂರಾರು ಸಾವಿರಾರು ವರುಷಗಳಿಂದ ಆಗುತ್ತಿದೆ. ಈಗ ಪ್ರಪಂಚದಲ್ಲಿ ಇರುವ ಸುಖದ ಮೊತ್ತ ಒಂದು ನೂರು ವರುಷಗಳ ಹಿಂದೆ ಇದ್ದುದಕ್ಕಿಂತ ಹೆಚ್ಚು ಆಗಿದೆಯೆ ಎಂದು ನಿಮ್ಮನ್ನು ಕೇಳುತ್ತೇನೆ. ಅದು ಆಗಿರಲಾರದು. ಸಾಗರದಲ್ಲಿ ಏಳುವ ಪ್ರತಿಯೊಂದು ಅಲೆಯ ಹಿಂದೆಯೂ ಒಂದು ಬೀಳು ಇದ್ದೇ ತೀರಬೇಕು. ಒಂದು ದೇಶವು ಬಲಾಢ್ಯವಾಗಿ ಶ‍್ರೀಮಂತವಾದರೆ ಮತ್ತೆಲ್ಲೋ ಇರುವ ಇನ್ನೊಂದು ದೇಶ ಅದಕ್ಕಾಗಿ ವ್ಯಥೆಪಟ್ಟಿರಬೇಕು. ಹೊಸದಾಗಿ ಕಂಡುಹಿಡಿದ ಪ್ರತಿಯೊಂದು ಯಂತ್ರವೂ ಇಪ್ಪತ್ತು ಜನರನ್ನು ಶ‍್ರೀಮಂತರನ್ನಾಗಿ ಮಾಡುವುದು, ಇಪ್ಪತ್ತು ಸಾವಿರ ಜನರನ್ನು ಬಡವರನ್ನಾಗಿ ಮಾಡುವುದು. ಯಾವಾಗಲೂ ಪೈಪೋಟಿಯ ನಿಯಮವೇ ಪ್ರಚಲಿತವಾಗಿರುವುದು. ಶಕ್ತಿಯ ಒಟ್ಟು ಮೊತ್ತ ಯಾವಾಗಲೂ ಒಂದೇ ಸಮನಾಗಿ ಇರುತ್ತದೆ. ದುಃಖವಿಲ್ಲದೆ ನಮಗೆ ಸುಖವು ಸಿಕ್ಕಬಲ್ಲದೆಂದು ಭಾವಿಸುವುದು ಅವಿವೇಕದ ಮಾತು. ಸುಖವನ್ನು ಹೆಚ್ಚಿಸುವ ಸಾಧನಗಳು ಹೆಚ್ಚಾದರೆ ಪ್ರಪಂಚದ ಬೇಡಿಕೆಗಳೂ ಜಾಸ್ತಿಯಾಗುವುವು. ಯಾವಾಗ ನಮ್ಮ ಬೇಡಿಕೆಗಳು ಹೆಚ್ಚಾಗುವುವೋ ಆಗ ಆಸೆಯೂ ವೃದ್ಧಿಯಾಗುವುದು. ಅದನ್ನು ಎಂದಿಗೂ ನಾವು ತೃಪ್ತಿಪಡಿಸಲಾರೆವು. ಈ ಬಯಕೆ, ಈ ಆಸೆಯನ್ನು ಯಾವುದು ತೃಪ್ತಿಪಡಿಸಬಲ್ಲದು? ಎಲ್ಲಿಯವರೆಗೆ ಈ ಬಯಕೆಗಳು ನಮ್ಮಲ್ಲಿ ಇರುವುವೋ ಅಲ್ಲಿಯವರೆಗೆ ದುಃಖ ತಪ್ಪಿದ್ದಲ್ಲ. ಸುಖವಾದ ಮೇಲೆ ದುಃಖ, ದುಃಖವಾದ ಮೇಲೆ ಸುಖ. ಇದೇ ಜೀವನದ ಸ್ವಭಾವವಾಗಿದೆ. ನಾವು ಒಳ್ಳೆಯದನ್ನು ಮಾಡಲೋಸುಗ ಈ ಪ್ರಪಂಚ ಇದೆ ಎಂದು ಭಾವಿಸಿದಿರಾ? ಈ ಪ್ರಪಂಚದಲ್ಲಿ ಬೇರೊಂದು ಶಕ್ತಿ ಕೆಲಸ ಮಾಡುತ್ತಿಲ್ಲ ಎಂದು ಭಾವಿಸಿದಿರಾ? ಪ್ರಪಂಚವನ್ನು ನಮಗೆ ನಿಮಗೆ ನೋಡಿಕೊಳ್ಳುವುದಕ್ಕೆ ಹೇಳಿ ದೇವರು ಸತ್ತುಹೋಗಿರುವನು ಎಂದು ಭಾವಿಸಿದಿರೇನು? ನಿತ್ಯನಾದ ದಯಾಮಯನಾದ, ಸರ್ವಶಕ್ತನಾದ, ಚಿರಜಾಗ್ರತನಾದ, ಪ್ರಪಂಚ ನಿದ್ರಿಸುತ್ತಿದ್ದರೂ ಯಾರು ಒಂದು ಕ್ಷಣವೂ ಕಣ್ಣನ್ನು ಮುಚ್ಚಿಕೊಳ್ಳುವುದಿಲ್ಲವೋ ಅಂಥವನಿಲ್ಲ ಎಂದು ಭಾವಿಸಿದಿರಾ? ಈ ಅನಂತ ಆಕಾಶವೇ ರೆಪ್ಪೆ ಹಾಕದೇ ನೋಡುತ್ತಿರುವ ಅವನ ಕಣ್ಣುಗಳಂತೆ ಇವೆ. ಅವನೇನು ಸತ್ತು ಹೋಗಿರುವನೇನು? ಅವನೇನು ಜಗತ್ತಿನಲ್ಲಿ ಕೆಲಸ ಮಾಡುತ್ತಿಲ್ಲವೇ? ಜಗತ್ತು ನಡೆಯುತ್ತಲೇ ಇಲ್ಲವೇ? ನೀವೇನೂ ವ್ಯಸ್ತರಾಗಬೇಕಾಗಿಲ್ಲ.

(ಸ್ವಾಮೀಜಿ ಅವರು ಇಲ್ಲಿ ಒಂದು ಕಥೆಯನ್ನು ಹೇಳಿದರು. ಒಬ್ಬನಿಗೆ ತಾನು ಹೇಳಿದ ಕೆಲಸವನ್ನೆಲ್ಲ ಮಾಡುವುದಕ್ಕೆ ಒಂದು ಭೂತ ಬೇಕಾಗಿತ್ತು. ಆದರೆ ಅವನಿಗೆ ಒಂದು ಭೂತ ಸಿಕ್ಕಿದ ಮೇಲೆ ಅದಕ್ಕೆ ಯಾವಾಗಲೂ ಕೆಲಸ ಕೊಡಲು ಆಗಲಿಲ್ಲ. ಕೊನೆಗೆ ಅದಕ್ಕೆ ಒಂದು ಡೊಂಕು ನಾಯಿ ಬಾಲವನ್ನು ಕೊಟ್ಟು ಅದನ್ನು ನೇರಮಾಡು ಎಂದು ಹೇಳಬೇಕಾಯಿತು.)

ಈ ಪ್ರಪಂಚಕ್ಕೆ ಒಳ್ಳೆಯದನ್ನು ಮಾಡಬೇಕೆಂದು ಇಚ್ಛಿಸುವ ನಮ್ಮ ಸ್ಥಿತಿಯೂ ಅಷ್ಟೇ. ಹಾಗೆಯೇ ಸಹೋದರರೇ, ಕಳೆದ ನೂರಾರು, ಸಾವಿರಾರು ವರುಷಗಳಿಂದ ನಾವು\break ನಾಯಿಯ ಬಾಲವನ್ನು ನೇರಮಾಡುವುದರಲ್ಲಿ ನಿರತರಾಗಿರುವೆವು. ಇದೊಂದು ವಾತ ರೋಗದಂತೆ. ಅದನ್ನು ನಾವು ಕಾಲಿನಿಂದ ಓಡಿಸಿದರೆ ಅದು ತಲೆಗೆ ಹೋಗುವುದು. ಅದನ್ನು ಅಲ್ಲಿಂದ ಕಳುಹಿಸಿದರೆ ಮತ್ತೆಲ್ಲಿಗೋ ಹೋಗುವುದು.

ನಿಮ್ಮಲ್ಲಿ ಹಲವರಿಗೆ ಇದು ಭಯಾನಕವಾದ, ಪ್ರಪಂಚದಲ್ಲಿ ಎಲ್ಲವೂ ದುಃಖಮಯ ಎಂಬ ಸಿದ್ಧಾಂತದಂತೆ ಕಾಣಬಹುದು. ಆದರೆ ಇದು ಹಾಗಿಲ್ಲ. ಈ ಜಗತ್ತೆಲ್ಲ ಸುಖಮಯ ಅಥವಾ ಎಲ್ಲ ದುಃಖಮಯ ಎಂಬ ಎರಡು ಅಭಿಪ್ರಾಯಗಳೂ ತಪ್ಪೆ. ಇವೆರಡೂ ಅತಿರೇಕದ ಅಭಿಪ್ರಾಯಗಳು. ಎಲ್ಲಿಯವರೆಗೆ ಒಬ್ಬನಿಗೆ ಬೇಕಾದಷ್ಟು ತಿನ್ನುವುದಕ್ಕೆ ಕುಡಿಯುವುದಕ್ಕೆ ಇದೆಯೋ, ಹಾಕಿಕೊಳ್ಳುವುದಕ್ಕೆ ಚೆನ್ನಾಗಿರುವ ಬಟ್ಟೆಗಳು ಇವೆಯೋ, ಅವನೊಬ್ಬ ದೊಡ್ಡ ಆಶಾವಾದಿ ಆಗುವನು. ಆದರೆ ಅದೇ ಮನುಷ್ಯ ಎಲ್ಲವನ್ನೂ ಕಳೆದುಕೊಂಡ ಮೇಲೆ ನಿರಾಶಾವಾದಿಯಾಗುವನು. ಒಬ್ಬ ತನ್ನ ಧನವನ್ನೆಲ್ಲ ಕಳೆದುಕೊಂಡು ನಿರ್ಗತಿಕನಾದ ಮೇಲೆ, ಮಾನವ ಸಹೋದರ ಎಂಬ ಭಾವನೆ ಬಹಳ ಉತ್ಕಟವೇಗದಲ್ಲಿ ಬರುವುದು. ಇದೇ ಜಗತ್ತು. ನಾನು ಬೇರೆ ಬೇರೆ ದೇಶಗಳಿಗೆ ಹೋಗಿ ಹೆಚ್ಚು ಅನುಭವವನ್ನು ಗಳಿಸಿದಂತೆಲ್ಲ, ವಯಸ್ಸಾದಂತೆಲ್ಲ, ಆಶಾವಾದ ಮತ್ತು ನಿರಾಶಾವಾದಗಳ ಅತಿಯಿಂದ ಪಾರಾಗಲು ಯತ್ನಿಸುತ್ತಿರುವೆನು. ಈ ಜಗತ್ತು ಒಳ್ಳೆಯದೂ ಅಲ್ಲ, ಕೆಟ್ಟದ್ದೂ ಅಲ್ಲ. ಇದು ಭಗವಂತನ ಜಗತ್ತು. ಇದು ಒಳ್ಳೆಯದು ಕೆಟ್ಟದ್ದು ಇವುಗಳಿಗೆ ಅತೀತವಾಗಿದೆ. ತನ್ನಲ್ಲಿ ತಾನು ಪರಿಪೂರ್ಣವೇ ಆಗಿದೆ. ಅವನ ಇಚ್ಛೆ ನೆರವೇರುತ್ತಿದೆ. ವೈವಿಧ್ಯದಿಂದ ತುಂಬಿರುವ ದೃಶ್ಯಗಳ ಹಿಂದೆಲ್ಲ ಅವನಿರುವನು. ಇದು ಆದಿ ಅಂತ್ಯವಿಲ್ಲದಂತೆ ಹೀಗೆಯೇ ಸಾಗುತ್ತಿರುವುದು. ಈ ಪ್ರಪಂಚ ಒಂದು ದೊಡ್ಡ ಗರಡಿಮನೆ. ಇಲ್ಲಿಗೆ ನಾವು ನೀವು ಮತ್ತು ಇನ್ನೂ ಲಕ್ಷಾಂತರ ಜನ ಅಂಗಸಾಧನೆಗಾಗಿ ಬರುತ್ತೇವೆ. ಇದರಿಂದ ನಾವು ಬಲಾಢ್ಯರಾಗಿ ಪರಿಪೂರ್ಣರಾಗುತ್ತೇವೆ. ಈ ಪ್ರಪಂಚ ಇರುವುದೇ ಅದಕ್ಕೆ. ದೇವರಿಗೆ ಪರಿಪೂರ್ಣವಾದ ಪ್ರಪಂಚವನ್ನು ಮಾಡುವುದಕ್ಕೆ ಆಗಲಿಲ್ಲ ಎಂದು ಅಲ್ಲ. ಈ ಪ್ರಪಂಚದ ದುಃಖವನ್ನು ತಗ್ಗಿಸುವುದಕ್ಕೆ ಅವನಿಗೆ ಸಾಧ್ಯವಿಲ್ಲದೇ ಇಲ್ಲ. ನಿಮಗೆ ದೂರದರ್ಶಕ ಯಂತ್ರದ ಮೂಲಕ ನೋಡುತ್ತಿದ್ದ ಒಬ್ಬ ಯುವತಿ ಮತ್ತು ಪಾದ್ರಿ ಇವರ ಕಥೆ ನೆನಪಿರಬಹುದು. ಅಲ್ಲಿ ಅವರಿಗೆ ಚಂದ್ರನಲ್ಲಿ ಮಚ್ಚೆಗಳು ಕಂಡವು. ಪಾದ್ರಿ ಅವುಗಳನ್ನು ನೋಡಿ, ಅವು ನಿಜವಾಗಿಯೂ ಚರ್ಚಿನ ಗೋಪುರಗಳು ಇರಬೇಕು ಎಂದನು. ಯುವತಿಯು “ಅವಿವೇಕದ ಮಾತು ಅದು. ಪ್ರಣಯಿ ಪ್ರಣಯಿನಿಯರು ಒಬ್ಬರು ಮತ್ತೊಬ್ಬರನ್ನು ಚುಂಬಿಸುತ್ತಿರುವ ದೃಶ್ಯವಿರಬೇಕು" ಎಂದಳು. ನಾವು ಈ ಪ್ರಪಂಚವನ್ನು ನೋಡುತ್ತಿರುವುದೂ ಅದರಂತೆಯೇ. ನಮ್ಮ ಒಳಗೆ ಇರುವುದೇ ಹೊರಗೂ ಇದೆ ಎಂದು ಭಾವಿಸುವೆವು. ನಾವು ಯಾವ ಮಟ್ಟದಲ್ಲಿರುವೆವೋ ಪ್ರಪಂಚವನ್ನು ಆ ದೃಷ್ಟಿಯಿಂದ ನೋಡುವೆವು. ಅಡಿಗೆ ಮನೆಯಲ್ಲಿರುವ ಬೆಂಕಿಯು ಒಳ್ಳೆಯದೂ ಅಲ್ಲ ಕೆಟ್ಟದ್ದೂ ಅಲ್ಲ. ಅದು ನಿಮ್ಮ ಅಡಿಗೆಗೆ ಸಹಾಯ ಮಾಡಿದರೆ, ನೀವು ಬೆಂಕಿಗೆ ಧನ್ಯವಾದವನ್ನು ಅರ್ಪಿಸಿ “ಇದೆಷ್ಟು ಒಳ್ಳೆಯದು'' ಎನ್ನುವಿರಿ. ಅದು ನಿಮ್ಮ ಕೈಯನ್ನು ಸುಟ್ಟರೆ “ಇದೆಂತಹ ಅನಿಷ್ಟ'' ಎನ್ನುವಿರಿ. ಇದರಂತೆಯೇ ಪ್ರಪಂಚವು ಒಳ್ಳೆಯದೂ ಅಲ್ಲ ಕೆಟ್ಟದ್ದೂ ಅಲ್ಲ ಎಂದು ಹೇಳಿದರೆ ಅದು ಯುಕ್ತಿಯುಕ್ತವಾದ ಮಾತು. ಈ ಪ್ರಪಂಚ ಒಳ್ಳೆಯದ್ದೂ ಅಲ್ಲ, ಕೆಟ್ಟದ್ದೂ ಅಲ್ಲ. ಈ ಪ್ರಪಂಚ ಪ್ರಪಂಚವೇ. ಇದು ಯಾವಾಗಲೂ ಹೀಗೆಯೇ ಇರುವುದು. ಪ್ರಪಂಚ ನಮಗೆ ಹಿತಕರವಾಗಿರುವ ರೀತಿಯಲ್ಲಿ ಇದ್ದರೆ ಅದನ್ನು ಒಳ್ಳೆಯದು ಎನ್ನುತ್ತೇವೆ. ಅದರಿಂದ ದುಃಖ ಬರುವ ರೀತಿಯಲ್ಲಿ ಇದ್ದರೆ ಅದನ್ನು ಕೆಟ್ಟದ್ದು ಎನ್ನುವೆವು. ಆದಕಾರಣವೇ ಈ ಪ್ರಪಂಚದಲ್ಲಿ ನಿರಪರಾಧಿಗಳಾದ ಎಳೆಹಸುಳೆಗಳು ಆನಂದದಲ್ಲಿ ಇರುವರು. ಅವರು ಮತ್ತಾರಿಗೂ ತೊಂದರೆಯನ್ನು ಕೊಡುವುದಿಲ್ಲ. ಅವರು ಆಶಾವಾದಿಗಳಾಗಿ ಇರುವರು. ಅವರು ಸ್ವರ್ಣ ಸ್ವಪ್ನಗಳನ್ನು ಕಾಣುತ್ತಿರುವರು. ಆದರೆ ಯಾವ ವೃದ್ಧರಲ್ಲಿ ಬೇಕಾದಷ್ಟು ಆಸೆಗಳಿವೆಯೋ, ಅವನ್ನು ಪೂರ್ಣಮಾಡಿಕೊಳ್ಳುವುದಕ್ಕೆ ಸಾಧ್ಯವಿಲ್ಲವೋ, ಅದರಲ್ಲಿಯೂ ಪ್ರಪಂಚದ ನೂಕು ನುಗ್ಗಲಿಗೆ ಸಿಕ್ಕಿ ಜರ್ಝರಿತರಾಗಿರುವರೋ, ಅವರು ನಿರಾಶಾವಾದಿಗಳಾಗಿರುವರು. ಧರ್ಮವು ಸತ್ಯವನ್ನು ತಿಳಿಯಲು ಯತ್ನಿಸುವುದು. ಅದು ಕಂಡುಹಿಡಿದ ಮೊದಲನೇ ವಿಷಯವೇ, ಈ ಸತ್ಯದ ಜ್ಞಾನವಿಲ್ಲದೇ ಇದ್ದರೆ ಈ ಪ್ರಪಂಚದಲ್ಲಿ ಬಾಳಿ ಪ್ರಯೋಜನವಿಲ್ಲ ಎಂಬುದು.

ಇಂದ್ರಿಯಾತೀತವಾದ ವಸ್ತುವನ್ನು ನಾವು ತಿಳಿಯದೇ ಇದ್ದರೆ, ಮಾನವ\break ಜೀವನವು ಒಂದು ಮರಳುಗಾಡಿನಂತೆ ಆಗುವುದು, ನಿಷ್ಪ್ರಯೋಜಕವಾಗುವುದು. ತತ್ಕಾಲದಲ್ಲಿ ಏನಿದೆಯೋ ಅದರಲ್ಲೇ ತೃಪ್ತನಾಗಿರು ಎಂದು ಹೇಳುವುದೇನೊ ಸರಿ. ದನಗಳು ನಾಯಿಗಳು ಹಾಗೆಯೇ ಇವೆ. ಮೃಗಗಳೆಲ್ಲ ಹಾಗೆಯೇ ಇವೆ. ಅವನ್ನು ಮೃಗಗಳನ್ನಾಗಿ ಮಾಡಿರುವುದೇ ಇದು. ಮನುಷ್ಯನೂ ಕೂಡ ತನಗೆ ಈಗ ಇರುವುದರಲ್ಲೇ ತೃಪ್ತನಾಗಿ, ಅತೀತವಾದ ಯಾವುದನ್ನೂ ತಿಳಿದುಕೊಳ್ಳಲು ಪ್ರಯತ್ನಿಸದೇ ಇದ್ದರೆ ಮಾನವಕೋಟಿ ಮತ್ತೊಮ್ಮೆ ಮೃಗದ ಸ್ಥಿತಿಗೆ ಹೋಗಬೇಕಾಗುವುದು. ಧರ್ಮವೆಂದರೆ ಇಂದ್ರಿಯಾತೀತವಾದುದನ್ನು ಅರಿಯಬೇಕೆಂಬ ಆಕಾಂಕ್ಷೆ. ಇದೇ ಮನುಷ್ಯನಿಗೂ ಮೃಗಕ್ಕೂ ಇರುವ ವ್ಯತ್ಯಾಸ. ಮನುಷ್ಯನೊಬ್ಬನೇ ಪ್ರಾಣಿಗಳಲ್ಲಿ ಸ್ವಭಾವತಃ ಮೇಲಕ್ಕೆ ನೋಡುವಂತಹನು ಎಂದು ಸರಿಯಾಗಿಯೇ ಹೇಳಲ್ಪಟ್ಟಿದೆ. ಇತರ ಎಲ್ಲ ಪ್ರಾಣಿಗಳೂ ಸ್ವಾಭಾವಿಕವಾಗಿ ಕೆಳಮುಖವಾಗಿ ನೋಡುತ್ತವೆ. ಹೀಗೆ ಮೇಲೆ ನೋಡುವುದು, ಮೇಲಕ್ಕೆ ಹೋಗುವುದು, ಪೂರ್ಣತೆಯನ್ನು ಹುಡುಕುವುದು – ಇವನ್ನೇ ಮುಕ್ತಿ ಎನ್ನುವುದು. ಅವನು ಎಷ್ಟು ಬೇಗ ಮೇಲಕ್ಕೆ ಹೋಗಲು ಪ್ರಯತ್ನಿಸುವನೋ ಅಷ್ಟು ಬೇಗ ಅವನು ಮುಕ್ತಿ ಎಂಬ ಸತ್ಯದ ಕಡೆಗೆ ಹೋಗುವನು. ನಿನ್ನ ಜೇಬಿನಲ್ಲಿ ಎಷ್ಟು ಹಣ ಇದೆ, ನೀನು ಎಂತಹ ಉಡುಪುಗಳನ್ನು ಧರಿಸುತ್ತೀಯೆ, ನೀನು ಎಂತಹ ಮನೆಯಲ್ಲಿ ವಾಸಮಾಡುತ್ತೀಯೆ, ಇವಲ್ಲ ಮುಖ್ಯ. ನಿನ್ನ ಮೆದುಳಿನಲ್ಲಿರುವ ಅಧ್ಯಾತ್ಮಶ‍್ರೀ ಒಂದೇ ಗಣನೀಯವಾಗಿರುವುದು. ಮಾನವ ಪ್ರಗತಿ ಎಂದರೆ ಇದೇ; ಇದೇ ಎಲ್ಲಾ ಭೌತಿಕ ಮತ್ತು ಬೌದ್ಧಿಕ ಪ್ರಗತಿಗೆ ಮೂಲ. ಮಾನವ ಜೀವನ ಮತ್ತು ಜನಾಂಗವನ್ನು ಮುಂದುಮುಂದಕ್ಕೆ ನೂಕಲು ಕ್ರಿಯೋತ್ತೇಜಕವಾದ ಶಕ್ತಿಯೇ, ಉತ್ಸಾಹವೇ ಇದು.

ಮಾನವನ ಗುರಿ ಎಂದರೆ ಮತ್ತೇನು? ಇದೇನು ಇಂದ್ರಿಯ ಸುಖವೇ, ಸಂತೋಷವೇ? ಜನರು ಸ್ವರ್ಗದಲ್ಲಿ ತುತ್ತೂರಿಯನ್ನು ಊದುತ್ತ ದೇವರ ಸಿಂಹಾಸನದ ಸುತ್ತಲೂ ಇರುವರು ಎಂದು ಹಿಂದಿನ ಕಾಲದಲ್ಲಿ ಹೇಳುತ್ತಿದ್ದರು. ಈಗಿನ ಕಾಲದಲ್ಲಿ ಆ ಆದರ್ಶವನ್ನು ದುರ್ಬಲವೆಂದು ಭಾವಿಸುವರು. ಅದನ್ನು ಬಲಪಡಿಸಿ ಸ್ವರ್ಗದಲ್ಲಿ ಮದುವೆ ಮುಂತಾದುವುಗಳೆಲ್ಲ ಇವೆ ಎನ್ನುವರು. ಈ ಆದರ್ಶಗಳಲ್ಲಿ ಏನಾದರೂ ಬೆಳವಣಿಗೆ ಇದೆ ಎಂದು ಭಾವಿಸಿದರೆ ಎರಡನೆಯದು ಮತ್ತೂ ಅಧೋಗತಿಗೆ ಇಳಿದಿದೆ ಎಂದು ಹೇಳಬೇಕಾಗುವುದು. ಹಲವು ಬಗೆಯ ಸ್ವರ್ಗದ ಆದರ್ಶಗಳಲ್ಲೆಲ್ಲ ಮನೋದೌರ್ಬಲ್ಯವು ಗೋಚರಿಸುತ್ತದೆ. ಆ ದೌರ್ಬಲ್ಯವೇ ಇದು: ಮೊದಲನೆಯದಾಗಿ ಇಂದ್ರಿಯ ಸುಖವೇ ಜೀವನದ ಗುರಿ ಎಂದು ಭಾವಿಸುವುದು; ಎರಡನೆಯದಾಗಿ ಪಂಚೇಂದ್ರಿಯಗಳಿಗೆ ಅತೀತವಾದ ಯಾವುದನ್ನೂ ಅವರು ಭಾವಿಸಲು ಅಸಮರ್ಥರಾಗಿರುವುದು. ಅವರು ಪ್ರಯೋಜನ ದೃಷ್ಟಿಯವರಷ್ಟೇ ಅವಿವೇಕಿಗಳು. ಆದರೂ ಅವರು ಆಧುನಿಕ ನಾಸ್ತಿಕ ಪ್ರಯೋಜನವಾದಿಗಳಿಗಿಂತ ಮೇಲು. ಕೊನೆಯದಾಗಿ ಈ ಪ್ರಯೋಜನ ದೃಷ್ಟಿ ಎಂಬುದು ಹುರುಳಿಲ್ಲದ ವಾದ. ಇದು ನನ್ನ ಪ್ರಮಾಣ, ಪ್ರಪಂಚವೆಲ್ಲ ಇದನ್ನೇ ಅನುಸರಿಸಬೇಕು ಎಂದು ಬಲಾತ್ಕರಿಸುವುದಕ್ಕೆ ನಿಮಗೆ ಏನು ಅಧಿಕಾರವಿದೆ? ಪ್ರತಿಯೊಂದು ಸತ್ಯವನ್ನೂ ನಿಮ್ಮ ಪ್ರಮಾಣದ ದೃಷ್ಟಿಯಿಂದಲೇ, ಅನ್ನ ಹಣ ಬಟ್ಟೆ ಇವೇ ದೇವರು ಎಂದು ಅಳೆಯಬೇಕು ಎನ್ನುವುದಕ್ಕೆ ನಿಮಗೇನು\break ಅಧಿಕಾರವಿದೆ?

\vskip 2pt

ಧರ್ಮವು ಅನ್ನದ ಮೇಲೆ ನಿಂತಿಲ್ಲ, ಅದು ಮನೆಯಲ್ಲಿ ಇಲ್ಲ. ಪದೇ ಪದೇ ಜನ ಮುಂದೆ ಹೇಳಿರುವ ಆಕ್ಷೇಪಣೆಯನ್ನು ಎತ್ತುವುದನ್ನು ನೀವು ನೋಡುವಿರಿ: ಧರ್ಮದಿಂದ ಏನು ಪ್ರಯೋಜನ? ಅದು ಜನರ ಬಡತನವನ್ನು ನೀಗಿಸಿ ಅವರಿಗೆ ಹೆಚ್ಚು ಬಟ್ಟೆಬರೆಗಳನ್ನು ಕೊಡಬಲ್ಲದೇ? ಒಂದು ವೇಳೆ ಅದು ಮಾಡಲಿಲ್ಲ ಎಂದು ಭಾವಿಸೋಣ. ಆದರೂ ಅದು ಧರ್ಮವನ್ನು ಸುಳ್ಳು ಎಂದಂತೆ ಆಯಿತೇ? ನೀವು ಖಗೋಳಶಾಸ್ತ್ರದ ವಿಷಯವನ್ನು ತೋರುತ್ತಿರುವಾಗ ಒಂದು ಮಗು ಎದ್ದು ನಿಂತು `ಅದರಿಂದ ನನಗೆ ಜಿಂಜರ್ ಬ್ರೆಡ್ ಸಿಕ್ಕುವುದೇ?' ಎಂದು ಪ್ರಶ್ನಿಸಿದರೆ ನೀವು “ಇಲ್ಲ, ಅದರಿಂದ ಜಿಂಜರ್ ಬ್ರೆಡ್ ಸಿಕ್ಕುವುದಿಲ್ಲ” ಎಂದರೆ ಮಗು, ಅದರಿಂದ ಏನೂ ಪ್ರಯೋಜನವಿಲ್ಲ ಎನ್ನುವುದು. ಮಕ್ಕಳು ಪ್ರಪಂಚವನ್ನೆಲ್ಲ ತಮ್ಮ ದೃಷ್ಟಿಯಿಂದಲೇ ನೋಡುತ್ತವೆ – ಜಿಂಜರ್ ಬ್ರೆಡ್ ಸಿಕ್ಕುವುದೇ, ಇಲ್ಲವೇ ಎಂದು. ಅದರಂತೆಯೇ ಪ್ರಪಂಚದಲ್ಲಿ ಬಾಲ ಬುದ್ದಿಯವರು ಕೂಡ.

\vskip 2pt

ಹತ್ತೊಂಬತ್ತನೇ ಶತಮಾನದ ಅಂತ್ಯದಲ್ಲಿ, ಮೇಲೆ ಹೇಳಿದಂತಹ ಜನರೇ ಪ್ರಪಂಚದಲ್ಲಿ ಮಹಾ ಬುದ್ದಿವಂತರೂ, ವಿಚಾರಪರರೂ, ನ್ಯಾಯಶೀಲರೂ ಆದವರು ಎಂದು ಭಾವಿಸುತ್ತಿರುವುದನ್ನು ನೋಡಿದರೆ ಬಹಳ ವ್ಯಸನವಾಗುವುದು.

\vskip 2pt

ನಮ್ಮ ಕೆಳಗಿನ ದೃಷ್ಟಿಯಿಂದ ಜೀವನದ ಉತ್ತಮ ಸತ್ಯಗಳನ್ನು ನಾವು ಅಳೆಯಬಾರದು. ಪ್ರತಿಯೊಂದನ್ನೂ ಅದರದರ ದೃಷ್ಟಿಯಿಂದಲೇ ಅಳೆಯಬೇಕು. ಅನಂತವನ್ನು ಅದರ ಪ್ರಮಾಣದಿಂದಲೇ ಅಳೆಯಬೇಕು. ಧಾರ್ಮಿಕ ಭಾವನೆಯು ಇಡೀ ಮಾನವ ಕೋಟಿಯಲ್ಲಿ ಓತಪ್ರೋತವಾಗಿದೆ. ಈಗಿನ ಕಾಲದಲ್ಲಿ ಮಾತ್ರವಲ್ಲ; ಹಿಂದೆ ಈಗ ಮತ್ತು ಮುಂದೆ ಎಂದೆಂದಿಗೂ ಅದು ವ್ಯಾಪಿಸಿಕೊಂಡಿದೆ. ಆದಕಾರಣ ಇದು ನಿತ್ಯ ಮಾನವನಿಗೂ ನಿತ್ಯ ಈಶ್ವರನಿಗೂ ಇರುವ ನಿತ್ಯ ಸಂಬಂಧ. ಮಾನವ ಜೀವನದಲ್ಲಿ ಐದು ನಿಮಿಷಗಳಲ್ಲಿ\break ಏನಾಗಬಲ್ಲುದು ಎಂಬ ದೃಷ್ಟಿಯಿಂದ ಇದನ್ನು ಪರೀಕ್ಷಿಸುವುದು ನ್ಯಾಯ ಸಮ್ಮತವೇ? ಎಂದಿಗೂ ಇಲ್ಲ. ಇವೆಲ್ಲ ಕೇವಲ ನಿಷೇಧಮಯವಾದ ವಾದಗಳು.

\vskip 2pt

ಧರ್ಮವು ಏನನ್ನಾದರೂ ಮಾಡಬಲ್ಲದೇ ಎಂಬ ಪ್ರಶ್ನೆ ಈಗ ಬರುವುದು. ಹೊಟ್ಟೆಗೆ ಬಟ್ಟೆಗೆ ಧರ್ಮ ಏನನ್ನಾದರೂ ಕೊಡಬಲ್ಲದೇ? ಅದು ಕೊಡಬಲ್ಲದು. ಅದು ಯಾವಾಗಲೂ ಕೊಡುತ್ತಿದೆ. ಅದು ಇದಕ್ಕಿಂತಲೂ ಹೆಚ್ಚು ಮಾಡಬಲ್ಲದು. ಅದು ಮಾನವನಿಗೆ ಅನಂತತ್ವವನ್ನು ತರಬಲ್ಲದು. ಈಗ ಮನುಷ್ಯನು ಇರುವ ಸ್ಥಿತಿಗೆ ಧರ್ಮವೇ ಕಾರಣ ಮತ್ತು ಧರ್ಮವು ಈ ಮಾನವಮೃಗವನ್ನು ದೇವನನ್ನಾಗಿ ಮಾಡಬಲ್ಲದು. ಧರ್ಮವು ಇದನ್ನು ಮಾಡಬಲ್ಲದು. ಮಾನವ ಸಮಾಜದಿಂದ ಧರ್ಮವನ್ನು ಕಳೆದು ಬಿಟ್ಟರೆ ಇನ್ನು ಉಳಿಯುವುದೇನು? ಮೃಗಸದೃಶ ಮಾನವ ಕೊಂಪೆಯಲ್ಲದೆ ಮತ್ತೇನೂ ಇಲ್ಲ. ನಾನು ನಿಮಗೆ ಈಗ ತಾನೇ ತೋರಿಸಲು ಯತ್ನಿಸಿದಂತೆ, ಇಂದ್ರಿಯ ಸುಖವೇ ಮಾನವನ ಪರಮ ಉದ್ದೇಶ ಎಂದು ಭಾವಿಸುವುದು ಅವಿವೇಕ. ಎಲ್ಲಾ ಜೀವನದ ಗುರಿಯೂ ಜ್ಞಾನಾರ್ಜನೆ ಎಂಬುದು ನಿರ್ವಿವಾದವಾಗಿ ಇತ್ಯರ್ಥವಾಗುವುದು. ಮಾನವಕೋಟಿಗೆ ಕಲ್ಯಾಣವನ್ನು ಉಂಟುಮಾಡಬೇಕೆಂದು, ನಾವು ಸಾವಿರಾರು ವರುಷಗಳಿಂದ ಮಾಡಿದ ಪ್ರಯತ್ನದಿಂದ, ನಾವು ಮುಂದುವರಿದಿರುವುದು ಬಹಳ ಅಲ್ಪ ಎನ್ನುವುದನ್ನು ನಿಮಗೆ ತೋರಲು ಪ್ರಯತ್ನಿಸಿದೆ. ಆದರೆ ಜ್ಞಾನಾರ್ಜನೆಯಲ್ಲಿ ಮಾನವಕೋಟಿ ಅದ್ಭುತವಾಗಿ ಮುಂದೆ ಸಾಗಿದೆ. ನಾವು ಮುಂದುವರಿದಿರುವುದರ ಶ್ರೇಷ್ಠ ಪ್ರಯೋಜನವನ್ನು, ಅದು ನಮಗೆ ಎಷ್ಟು ಲೌಕಿಕ ಸುಖವನ್ನು ಕೊಟ್ಟಿದೆ ಎಂಬ ದೃಷ್ಟಿಯಿಂದ ನೋಡದೆ, ಈ ಮೃಗಸದೃಶ ಮಾನವನನ್ನು ಎಷ್ಟರಮಟ್ಟಿಗೆ ದೇವತ್ವಕ್ಕೆ ಏರಿಸಿದೆ ಎಂಬ ದೃಷ್ಟಿಯಿಂದ ನೋಡಬೇಕು. ಜ್ಞಾನದೊಂದಿಗೆ ಆನಂದವೂ ಬರುತ್ತದೆ. ಮಕ್ಕಳು ತಮಗೆ ದೊರಕಬಹುದಾದ ಇಂದ್ರಿಯ ಸುಖವನ್ನೇ ಸರ್ವಶ್ರೇಷ್ಠ ಎಂದು ಭಾವಿಸುವುವು. ನಿಮ್ಮಲ್ಲಿ ಅನೇಕರಿಗೆ, ಮನುಷ್ಯನಿಗೆ ಇಂದ್ರಿಯಗಳಿಗಿಂತ ಬುದ್ದಿಯಲ್ಲಿ ಹೆಚ್ಚು ಆನಂದವಿದೆ ಎಂಬುದು ಗೊತ್ತಿದೆ. ನಿಮ್ಮಲ್ಲಿ ಯಾರೂ ತಿನ್ನುವುದರಲ್ಲಿ ನಾಯಿಯು ಎಷ್ಟು ಆನಂದ ಪಡುವುದೋ ಅಷ್ಟು ಸಂತೋಷವನ್ನು ಪಡಲಾರಿರಿ. ನೀವು ಇದನ್ನು ಗಮನಿಸಬಹುದು. ಮನುಷ್ಯನಿಗೆ ಸುಖ ಎಲ್ಲಿಂದ ಬರುವುದು? ನಾಯಿ ಅಥವಾ ಹಂದಿಯಂತೆ\break ತದೇಕಚಿತ್ತದಿಂದ ತಿನ್ನುವುದರಿಂದ ಅಲ್ಲ. ಹಂದಿ ಹೇಗೆ ತಿನ್ನುತ್ತದೆ ನೋಡಿ. ಅದು ತಿನ್ನುತ್ತಿರುವಾಗ ಪ್ರಪಂಚವನ್ನೇ ಮರೆಯುವುದು. ಅದು ತಿನ್ನುವುದರಲ್ಲಿ ತನ್ಮಯವಾಗಿರುವುದು. ಅದನ್ನು ಕೊಲ್ಲಬಹುದು. ಆದರೆ ಮುಂದೆ ಆಹಾರವಿರುವ ತನಕ ಯಾವುದನ್ನೂ ಅದು ಗಮನಿಸುವುದಿಲ್ಲ. ಆ ಹಂದಿಗೆ ಇರುವ ತೀವ್ರವಾದ ಭೋಗಾಸಕ್ತಿಯನ್ನು ನೋಡಿ! ಯಾವ ಮನುಷ್ಯನಿಗೂ ಇದು ಸಾಧ್ಯವಿಲ್ಲ. ಅದು ಎಲ್ಲಿ ಹೋಗಿದೆ? ಮಾನವನು ಅದನ್ನು ಬೌದ್ಧಿಕ ಸುಖವನ್ನಾಗಿ ಪರಿವರ್ತಿಸಿರುವನು. ಹಂದಿ ಧಾರ್ಮಿಕ ಪ್ರವಚನವನ್ನು ಕೇಳಿ ಸಂತೋಷ ಪಡಲಾರದು. ಬೌದ್ಧಿಕ ಸುಖಕ್ಕಿಂತ ಇದು ಇನ್ನೂ ಮೇಲಿನ ಮಟ್ಟದ್ದು. ಇದೇ ಆಧ್ಯಾತ್ಮಿಕ ಕ್ಷೇತ್ರ. ಭಗವಂತನಿಗೆ ಸಂಬಂಧಪಟ್ಟ ವಿಷಯಗಳನ್ನು ಆನಂದಿಸುವುದು, ಯುಕ್ತಿ ಮತ್ತು ತರ್ಕಕ್ಕೆ ಅತೀತವಾಗಿ ಹೋಗುವುದು ಇದೇ ಸರ್ವಶ್ರೇಷ್ಠವಾದ ಪ್ರಯೋಜನ. ಇದನ್ನು ಪಡೆಯುವುದಕ್ಕೆ ಇಂದ್ರಿಯ ಸುಖಗಳನ್ನೆಲ್ಲ ನಾವು ತ್ಯಜಿಸಬೇಕಾಗುತ್ತದೆ. ಪ್ರಯೋಜನ ಎಂದರೆ ನಾನು ಏನನ್ನು ಅನುಭವಿಸುತ್ತೇನೆಯೋ ಅದು, ಮಿಕ್ಕವರೆಲ್ಲ ಯಾವುದನ್ನು ಅನುಭವಿಸುತ್ತಾರೆಯೋ ಅದು. ಅದನ್ನು ಪಡೆಯುವುದಕ್ಕಾಗಿ ನಾವೆಲ್ಲ ಪ್ರಯತ್ನಿಸುತ್ತಿರುವೆವು.

\vskip 2pt

ಮೃಗವು ತನ್ನ ಇಂದ್ರಿಯಸುಖವನ್ನು ಅನುಭವಿಸುವುದಕ್ಕಿಂತ ಹೆಚ್ಚಾಗಿ ಮನುಷ್ಯನು ತನ್ನ ಬುದ್ಧಿಯ ಮೂಲಕ ಆನಂದಿಸುವನು. ಪ್ರಬುದ್ಧ ಮಾನವನು ಬುದ್ದಿಗಿಂತ ಹೆಚ್ಚಾಗಿ ತನ್ನ ಆಧ್ಯಾತ್ಮಿಕ ಸ್ವಭಾವದ ಮೂಲಕ ಆನಂದಿಸುವನು. ಆದಕಾರಣ ಶ್ರೇಷ್ಠವಾದ ಜ್ಞಾನವೇ ಅಧ್ಯಾತ್ಮ ವಿದ್ಯೆ. ಅಧ್ಯಾತ್ಮ ವಿದ್ಯೆಯಿಂದ ಆನಂದ ಪ್ರಾಪ್ತಿ. ಈ ಪ್ರಪಂಚದಲ್ಲಿ ಕಾಣುತ್ತಿರುವ ವಸ್ತುಗಳೆಲ್ಲ ನೆರಳಿನಂತೆ. ಇವುಗಳೆಲ್ಲ ನಿಜವಾದ ಜ್ಞಾನ ಮತ್ತು ಆನಂದಗಳು ಮೂರು ನಾಲ್ಕು ಭೂಮಿಕೆಗಳ ಕೆಳಗಿನ ಅಭಿವ್ಯಕ್ತಿ.

\vskip 2pt

ನಾವು ಮಾನವಕೋಟಿಯನ್ನು ಪ್ರೀತಿಸುವುದರ ಮೂಲಕ ನಮಗೆ ಈ ಆನಂದವು ದೊರಕುತ್ತದೆ. ಆಧ್ಯಾತ್ಮಿಕ ಆನಂದದ ನೆರಳೇ ಮಾನವ ಪ್ರೀತಿ ಎಂಬುದು. ಇದನ್ನೇ ನಿಜವಾದ ಆನಂದ ಎಂದು ಭ್ರಮೆಪಡಬೇಡಿ. ಇಲ್ಲೇ ನಾವು ದೊಡ್ಡ ಒಂದು ತಪ್ಪನ್ನು ಮಾಡುವುದು. ನಮ್ಮಲ್ಲಿರುವ ಕೇವಲ ಮೃಗೀಯ ಪ್ರೀತಿಯನ್ನು, ಜಡದೇಹದ ಮೇಲೆ ಇರುವ ವಿದ್ಯುತ್ ಆಕರ್ಷಣವನ್ನು ಆಧ್ಯಾತ್ಮಿಕ ಆನಂದ ಎಂದು ನಾವು ಯಾವಾಗಲೂ ಭ್ರಮಿಸುವೆವು. ಇದೇ ಶಾಶ್ವತ ಸ್ಥಿತಿ ಎಂದು ನಾವು ಭಾವಿಸುವೆವು. ಇದಲ್ಲ, ಇಂಗ್ಲಿಷಿನಲ್ಲಿ ಬೇರೆ ಯಾವ ಪದವೂ ಇಲ್ಲದೇ ಇರುವುದರಿಂದ ನಾನು ಇದನ್ನು \enginline{bliss} (ಆನಂದ) ಎನ್ನುತ್ತೇನೆ. ಇದೇ ಅನಂತ ಜ್ಞಾನದ ಸ್ಥಿತಿ. ಇದೇ ನಮ್ಮ ಗುರಿ. ಪ್ರಪಂಚದಲ್ಲಿ ಹಿಂದೆ ಎಲ್ಲಿಯಾದರೂ ಒಂದು ಧರ್ಮ ಇದ್ದಿರಲಿ, ಅಥವಾ ಮುಂದೆ ಇರಲಿ ಅವೆಲ್ಲ ಒಂದೇ ಮೂಲದಿಂದ ಬಂದಿವೆ. ಆ ಮೂಲವನ್ನು ಹಲವು ದೇಶಗಳು ಹಲವು ಹೆಸರುಗಳಿಂದ ಕರೆದಿವೆ. ಇದನ್ನೇ ಪಾಶ್ಚಾತ್ಯ ದೇಶಗಳಲ್ಲಿ \enginline{inspiration} (ಸ್ಫೂರ್ತಿ) ಎನ್ನುವುದು. ಈ ಸ್ಫೂರ್ತಿ ಎಂದರೆ ಏನು? ಆಧ್ಯಾತ್ಮಿಕ ಜ್ಞಾನವನ್ನು ಪಡೆಯಬೇಕಾದರೆ ಸ್ಫೂರ್ತಿಯೊಂದೇ ನಮಗೆ ಇರುವ ಮಾರ್ಗ. ಧರ್ಮವು ಮುಖ್ಯವಾಗಿ ಇಂದ್ರಿಯಾತೀತವಾದ ಪ್ರಪಂಚಕ್ಕೆ ಸೇರಿರುವುದು ಎಂಬುದನ್ನು ನಾವು ನೋಡಿರುವೆವು. “ಅದನ್ನು ಕಣ್ಣು ನೋಡಲಾರದು, ಕಿವಿ ಕೇಳಲಾರದು, ಮನಸ್ಸು ಚಿಂತಿಸಲಾರದು. ಅದನ್ನು ಮಾತು ವ್ಯಕ್ತಪಡಿಸಲಾರದು.”

\vskip 2pt

ಅದೇ ಧರ್ಮಕ್ಷೇತ್ರ, ಧರ್ಮದ ಗುರಿ. ಅಲ್ಲಿಂದಲೇ ನಾವು ಯಾವುದನ್ನು ಸ್ಪೂರ್ತಿ ಎನ್ನುವೆವೊ ಅದು ಬರುವುದು. ಆದಕಾರಣ ಇಂದ್ರಿಯಾತೀತರಾಗಿ ಹೋಗುವುದಕ್ಕೆ ಒಂದು ಮಾರ್ಗವಿರಬೇಕು ಎಂದಂತೆ ಆಯಿತು. ನಮ್ಮ ತರ್ಕ ಇಂದ್ರಿಯಕ್ಕೆ ಅತೀತವಾಗಿ ಹೋಗಿರಲಾರದು ಎಂಬುದು ನಿಜವಾಗಿಯೂ ಸತ್ಯ. ನಮ್ಮ ತರ್ಕ ಇಂದ್ರಿಯದ ವಲಯದೊಳಗೆ ಇದೆ. ಇಂದ್ರಿಯಕ್ಕೆ ನಿಲುಕುವ ಅನುಭವದ ಮೂಲಕ ಮಾತ್ರ ಕೆಲಸ ಮಾಡುವುದು. ಆದರೆ ಮನುಷ್ಯ ಇಂದ್ರಿಯಕ್ಕೆ ಅತೀತನಾಗಿ ಹೋಗಬಲ್ಲನೇ? ಮನುಷ್ಯ ಅಜ್ಞೇಯವಾದುದನ್ನು ಅರಿಯಬಲ್ಲನೆ? ಈ ಪ್ರಶ್ನೆಯ ಮೇಲೆ ಧರ್ಮದ ಸಮಸ್ಯೆ ಇತ್ಯರ್ಥವಾಗಬೇಕಾಗಿದೆ.\break ಹಾಗೆಯೇ ಇತ್ಯರ್ಥವಾಗಿದೆ. ಇಂದ್ರಿಯಗಳನ್ನು ತಡೆದು ನಿಲ್ಲಿಸುವ ಅಭೇದ್ಯವಾದ ಗೋಡೆ ಅನಂತಕಾಲದಿಂದಲೂ ಇರುವುದು. ಅಂದಿನಿಂದಲೂ ನೂರಾರು ಜನ, ಸಹಸ್ರಾರು ಜನ ಸ್ತ್ರೀ ಪುರುಷರು ಅದನ್ನು ಭೇದಿಸಿಕೊಂಡು ಹೋಗಲು ಅದಕ್ಕೆ ಡಿಕ್ಕಿ ಹೊಡೆದಿದ್ದಾರೆ. ಲಕ್ಷಾಂತರ ಜನ ಸೋತಿರುವರು. ಲಕ್ಷಾಂತರ ಜನ ಜಯಶೀಲರಾಗಿರುವರು. ಇದೇ ಜಗತ್ತಿನ ಇತಿಹಾಸ. ಹಾಗೆ ಯಾರಾದರೂ ಅದನ್ನು ದಾಟಿಕೊಂಡು ಹೋದರು ಎಂಬುದನ್ನು ಲಕ್ಷಾಂತರ ಜನ ನಂಬುತ್ತಲೇ ಇರಲಿಲ್ಲ. ಈಗಿನ ಕಾಲದಲ್ಲಿಯೂ ಹಾಗೆ ನಂಬದವರು ಇರುವರು. ಮನುಷ್ಯ ಪ್ರಯತ್ನಪಟ್ಟರೆ ಅದನ್ನು ಭೇದಿಸಿಕೊಂಡು ಹೋಗಬಹುದು. ಮನುಷ್ಯನಿಗೆ ವಿಚಾರ ಮಾತ್ರವಲ್ಲ ಇರುವುದು, ಇಂದ್ರಿಯಗಳು ಮಾತ್ರವಲ್ಲ ಇರುವುದು, ಇದಕ್ಕೆ ಅತೀತವಾಗಿರುವ ಒಂದು ಮಹಾವಸ್ತುವೂ ಅವನಲ್ಲಿದೆ. ನಾನು ಅದನ್ನು ಸ್ವಲ್ಪ ವಿವರಿಸುವುದಕ್ಕೆ ಪ್ರಯತ್ನಿಸುತ್ತೇನೆ. ಅದು ನಿಮ್ಮಲ್ಲಿಯೂ ಇದೆ ಎಂಬುದನ್ನು ಅನುಭವಿಸುತ್ತೀರಿ ಎಂದು ನಾನು ಭಾವಿಸುತ್ತೇನೆ.

\vskip 2pt

ನಾನು ನನ್ನ ಕೈಯನ್ನು ಚಲಿಸುತ್ತೇನೆ. ನಾನು ಕೈಯನ್ನು ಚಲಿಸುತ್ತಿರುವೆ ಎಂದು ಗೊತ್ತಿದೆ. ನನಗೆ ಅದರ ಅನುಭವವೂ ಇದೆ. ಇದನ್ನು ಪ್ರಜ್ಞೆಯೆಂದು ಕರೆಯುತ್ತೇನೆ. ನಾನು ಕೈಯನ್ನು ಚಲಿಸುತ್ತಿರುವೆ ಎಂಬುದು ನನಗೆ ಗೊತ್ತಿದೆ. ಆದರೆ ನನ್ನ ಹೃದಯ ಚಲಿಸುತ್ತಿದೆ. ಅದರ ಅರಿವು ನನಗಿಲ್ಲ. ಆದರೂ ನನ್ನ ಹೃದಯವನ್ನು ಯಾರು ಚಲಿಸುತ್ತಿರುವರು? ಇದನ್ನೂ ಕೂಡ ನಾನೇ ಮಾಡುತ್ತಿರಬೇಕು. ಆದಕಾರಣ ಯಾವನು ಕೈಯನ್ನು ಚಲಿಸುತ್ತಿರುವನೋ, ಮತ್ತು ಯಾರು ಮಾತನಾಡುತ್ತಿರುವನೋ ಅವನೇ, ಎಂದರೆ, ಯಾರು ಪ್ರಜ್ಞೆಯಿಂದ ಕೆಲಸ ಮಾಡುತ್ತಿರುವನೋ ಅವನೇ ಅಪ್ರಜ್ಞೆಯಿಂದಲೂ ಕೆಲಸ ಮಾಡುತ್ತಿರುವನು. ಈ ಮನುಷ್ಯ ಎರಡು ಭೂಮಿಕೆಯಲ್ಲಿ ಕೆಲಸ ಮಾಡುವಂತೆ ತೋರುವನು. ಒಂದು ಪ್ರಜ್ಞೆ, ಮತ್ತೊಂದು ಅದಕ್ಕಿಂತಲೂ ಕೆಳಗಿರುವ ಅಪ್ರಜ್ಞೆ. ಅಪ್ರಜ್ಞೆಯಿಂದ ಆಗುತ್ತಿರುವ ಕೆಲಸಗಳನ್ನು ಹುಟ್ಟುಗುಣ ಎನ್ನುವೆವು. ಕೆಲಸವು ಪ್ರಜ್ಞಾಪ್ರಪಂಚಕ್ಕೆ ಬಂದಾಗ ಅದನ್ನು ಯುಕ್ತಿ ಎನ್ನುವೆವು. ಆದರೆ ಮನುಷ್ಯನಲ್ಲಿ ಇದಕ್ಕಿಂತ ಮೇಲಿರುವ ಅತಿಪ್ರಜ್ಞೆ ಎಂಬ ಕ್ಷೇತ್ರವೂ ಇದೆ. ಇದು ತೋರಿಕೆಗೆ ಅಪ್ರಜ್ಞೆಯಂತೆಯೇ ಇರುವುದು. ಏಕೆಂದರೆ ಯುಕ್ತಿಗೆ ಇದು ನಿಲುಕಲಾರದು. ಆದರೆ ಇದು ಯುಕ್ತಿಗಿಂತ ಮೇಲೆ ಇದೆಯೇ ಹೊರತು ಯುಕ್ತಿಯ ಕೆಳಗೆ ಇಲ್ಲ. ಇದು ಹುಟ್ಟುಗುಣವಲ್ಲ,\break ಸ್ಫೂರ್ತಿ. ಇದಕ್ಕೆ ಪ್ರಮಾಣ ಇದೆ. ಈ ಪ್ರಪಂಚಕ್ಕೆ ಬಂದ ಮಹಾತ್ಮರು ಋಷಿಗಳು ಇವರನ್ನೆಲ್ಲ ಕುರಿತು ಯೋಚಿಸಿ. ಕೆಲವು ವೇಳೆ ತಮ್ಮ ಸುತ್ತಲೂ ಏನು ಇದೆ ಎಂಬ ಪರಿಜ್ಞಾನವೇ ಅವರಿಗೆ ಇರದ ಸ್ಥಿತಿ ಅವರ ಜೀವನದಲ್ಲಿ ಬರುತ್ತದೆ. ಅನಂತರ ಅವರಿಗೆ ಬಂದ ಜ್ಞಾನವನ್ನೆಲ್ಲ ಅವರು ಆ ಸ್ಥಿತಿಯಲ್ಲಿದ್ದ ಸಮಯದಲ್ಲಿ ಬಂದಿತು ಎಂದು ಹೇಳುವರು. ಒಂದುಸಲ ಸಾಕ್ರಟೀಸ್ ಸೈನ್ಯದೊಂದಿಗೆ ಹೋಗುತ್ತಿದ್ದಾಗ ರಮ್ಯವಾದ ಸೂರ್ಯೋದಯವನ್ನು ಕಂಡನು. ಇದನ್ನು ನೋಡಿದಾಗ ಸಾಕ್ರಟೀಸನ ಮನಸ್ಸಿನಲ್ಲಿ ಭಾವನೆಗಳು ಎದ್ದವು. ಅವನು ಅದೇ ಸ್ಥಳದಲ್ಲಿ ಎರಡು ದಿನಗಳು ಯಾವ ಬಾಹ್ಯ ಪ್ರಜ್ಞೆಯೂ ಇಲ್ಲದೆ ಸುಮ್ಮನೆ ನಿಂತಿದ್ದನು ಎಂದು ಹೇಳುವರು. ಈ ಪ್ರಪಂಚಕ್ಕೆ ಸಾಕ್ರಟೀಸನಿಗೆ ಬಂದಂತಹ ಜ್ಞಾನ ಬಂದದ್ದು ಅಂತಹ\break ಸ್ಥಿತಿಯಿಂದ. ಇದರಂತೆಯೇ ಎಲ್ಲ ಆಚಾರ್ಯರು, ಮಹಾತ್ಮರು ಇವರುಗಳ ಜೀವನದಲ್ಲಿ, ಅವರು ತಮ್ಮ ಪ್ರಜ್ಞೆಯ ಸ್ಥಿತಿಯಿಂದ ಅತೀತರಾಗಿ ಹೋದ ಕಾಲಗಳಿವೆ. ಅವರು ಪ್ರಜ್ಞಾಸ್ಥಿತಿಗೆ ಇಳಿಯುವಾಗ ಅವರು ಜ್ಞಾನದಿಂದ ಕೋರೈಸುತ್ತಿರುವರು. ಅವರು ಇಂದ್ರಿಯಾತೀತವಾದ ಜಗತ್ತಿನ ಸುದ್ದಿಗಳನ್ನು ತಂದಿರುವರು. ಇವರೇ ಜಗತ್ತಿನ ಸ್ಫೂರ್ತಿಗೊಂಡ ಮಹಾಮಹಿಮರು.

\vskip 2pt

ಆದರೆ ಇಲ್ಲಿ ಒಂದು ದೊಡ್ಡ ಅಪಾಯವಿದೆ. ಯಾರು ಬೇಕಾದರೂ ತಮಗೆ ಸ್ಫೂರ್ತಿ ಬಂದಿದೆ ಎಂದು ಹೇಳಿಕೊಳ್ಳಬಹುದು. ಹೀಗೆ ಎಷ್ಟೋ ಜನ ನಮಗೆ ಹೇಳಿರುವರು. ಇದಕ್ಕೆ ಪರೀಕ್ಷೆ ಎಲ್ಲಿದೆ? ನಿದ್ರಾಸಮಯದಲ್ಲಿ ನಾವು ಅಪ್ರಜ್ಞೆಯಲ್ಲಿ ಇರುವೆವು. ಒಬ್ಬ ಮೂಢ ನಿದ್ರೆಗೆ ಹೋಗುವನು. ಅವನು ಮೂರು ಗಂಟೆಗಳ ಕಾಲ ಚೆನ್ನಾಗಿ ನಿದ್ರಿಸುವನು. ಅವನು ನಂತರ ಆ ಸ್ಥಿತಿಯಿಂದ ಎದ್ದುಬಂದಮೇಲೆ ಹಿಂದಿನಂತೆಯೇ, ಅದಕ್ಕಿಂತ ಕಡಮೆ ಇಲ್ಲದೇ ಇದ್ದರೂ ಅದೇ ಮೂಢನಾಗಿರುವನು. ಜೀಸಸ್ಸನು ಅತಿಪ್ರಜ್ಞೆಯ ಸ್ಥಿತಿಗೆ ಹೋಗುವನು. ಅವನು ಅಲ್ಲಿಂದ ಬಂದಾಗ ಕ್ರೈಸ್ತನಾಗಿ ಪರಿವರ್ತಿತನಾಗಿ ಬರುವನು. ಇದೇ ಎರಡಕ್ಕೂ ಇರುವ ವ್ಯತ್ಯಾಸ. ಒಂದು ಸ್ಪೂರ್ತಿ, ಮತ್ತೊಂದು ಹುಟ್ಟುಗುಣ. ಒಂದು ಮಗುವಿನ ಸ್ಥಿತಿ, ಮತ್ತೊಂದು ಅನುಭವವನ್ನು ಪಡೆದ ವೃದ್ಧನ ಸ್ಥಿತಿ. ಈ ಸ್ಪೂರ್ತಿ ನಮ್ಮಲ್ಲಿ ಎಲ್ಲರಿಗೂ ಸಾಧ್ಯ. ಇದೇ ಎಲ್ಲಾ ಧರ್ಮದ ಮೂಲ. ಎಲ್ಲಾ ಮೇಲಿನ ಜ್ಞಾನದ ಸ್ಥಿತಿಯೂ ಇದೇ ಆಗಿರುವುದು. ಆದರೂ ಮಾರ್ಗದಲ್ಲಿ ಬಹಳ ಅಪಾಯಗಳಿವೆ. ಕೆಲವು ವೇಳೆ ಕಪಟಿಗಳು ತಾವು ಮಹಾತ್ಮರೆಂದು ಜನರೆದುರಿಗೆ ಸೋಗು ಹಾಕುವರು. ಈಗಿನ ಕಾಲದಲ್ಲಿ ಇದು ಹೆಚ್ಚು ಹೆಚ್ಚಾಗುತ್ತಿದೆ. ನನ್ನ ಸ್ನೇಹಿತನ ಹತ್ತಿರ ಒಂದು ಸುಂದರವಾದ ಚಿತ್ರವಿತ್ತು. ಮತ್ತೊಬ್ಬ ಸ್ವಲ್ಪ ಧಾರ್ಮಿಕ ಸ್ವಭಾವದವನೂ, ಶ‍್ರೀಮಂತನೂ ಆಗಿದ್ದನು. ಅವನಿಗೆ ಈ ಚಿತ್ರದ ಮೇಲೆ ಕಣ್ಣಿತ್ತು. ಆದರೆ ನನ್ನ ಸ್ನೇಹಿತ ಅವನಿಗೆ ಆ ಚಿತ್ರವನ್ನು ಕೊಡಲಿಲ್ಲ. ಮತ್ತೊಬ್ಬ ಮನುಷ್ಯ ಚಿತ್ರವನ್ನು ಇಟ್ಟುಕೊಂಡಿದ್ದ ಮನುಷ್ಯನ ಬಳಿಗೆ ಬಂದು, `ನನಗೆ ಒಂದು ಸ್ಫೂರ್ತಿ ಬಂದಿದೆ. ದೇವರಿಂದ ಒಂದು ಆದೇಶ ಬಂದಿದೆ' ಎಂದು ಹೇಳಿದನು. ಅದೇನು ನಿನಗೆ ಬಂದ ಆದೇಶ ಎಂದು ತನ್ನ ಸ್ನೇಹಿತನನ್ನು ಕೇಳಿದನು. ಆ ಚಿತ್ರವನ್ನು ನನಗೆ ಕೊಡಬೇಕು ಎನ್ನುವುದೇ ಆದೇಶ ಎಂದನು. ಆದರೆ ನನ್ನ ಸ್ನೇಹಿತ ಆ ಸಮಯಕ್ಕೆ ಸರಿಯಾಗಿ ಉತ್ತರವನ್ನು ಕೊಟ್ಟನು. “ಹೌದು, ಅದು\break ನಿಜವಾಗಿಯೂ ಸರಿ, ಇದೆಷ್ಟು ಚೆನ್ನಾಗಿದೆ! ನನಗೂ ಅದೇ ಆದೇಶ ಆಗಿತ್ತು – ನಿನಗೆ ಚಿತ್ರವನ್ನು ಕೊಡುವಂತೆ, ಆದರೆ ಚೆಕ್ ಎಲ್ಲಿ?” ಎಂದು ಕೇಳಿದನು. ಆಗ ಇವನು ದಿಗ್ಭ್ರಾಂತನಾದ. ಆಗ ನನ್ನ ಸ್ನೇಹಿತ, “ನಿನ್ನ ಸ್ಫೂರ್ತಿ ಸರಿಯಾದದ್ದು ಅಲ್ಲ ಎಂದು ತೋರುವುದು. ನನಗೆ ಆದ ಆದೇಶದಲ್ಲಿ ಒಂದು ಲಕ್ಷ ಡಾಲರನ್ನು ತರುವವನಿಗೆ ಚಿತ್ರವನ್ನು ಕೊಡು ಎಂದಿತು. ಆದಕಾರಣ ನೀನು ಮೊದಲು ಚೆಕ್ ತರಬೇಕು'' ಎಂದನು. ಮತ್ತೊಬ್ಬನಿಗೆ ತಾನು ಸಿಕ್ಕಿ ಹಾಕಿಕೊಂಡೆ ಎಂದು ಗೊತ್ತಾಯಿತು. ಸ್ಪೂರ್ತಿಯ ಸಿದ್ದಾಂತವನ್ನು ಅಲ್ಲಿಗೆ ಬಿಟ್ಟನು. ಇವೇ ಇರುವ ಅಪಾಯಗಳು. ಬಾಸ್ಟನ್‌ನಲ್ಲಿ ಒಬ್ಬ ನನ್ನ ಬಳಿಗೆ ಬಂದು, ತನಗೆ ಕೆಲವು ದರ್ಶನಗಳಾದುವೆಂದೂ ಅದರಲ್ಲಿ ಹಿಂದೂ ಭಾಷೆಯಲ್ಲಿ ಯಾರೊ ತನ್ನೊಡನೆ ಮಾತನಾಡಿದರು ಎಂದೂ ಹೇಳಿಕೊಂಡನು. ಅದಕ್ಕೆ ನಾನು, ಅವನು ಏನನ್ನು ಹೇಳುತ್ತಾನೆಯೋ ಅದನ್ನು ನಾನು ನೋಡಿದರೆ ಮಾತ್ರ ನಂಬಲು ಸಾಧ್ಯ ಎಂದು ಹೇಳಿದೆ. ಅವನು ಅರ್ಥವಾಗದ ಏನೇನನ್ನೊ ಬರೆದನು. ಅದನ್ನು ತಿಳಿದುಕೊಳ್ಳಲು ನಾನು ಎಷ್ಟೋ ಪ್ರಯತ್ನ ಪಟ್ಟೆ. ಆದರೆ ಸಾಧ್ಯವಾಗಲಿಲ್ಲ. ಅದಕ್ಕೆ ನಾನು, ನನಗೆ ತಿಳಿದಿರುವ ಮಟ್ಟಿಗೆ ಇಂತಹ ಭಾಷೆ ಎಂದಿಗೂ ಇರಲಿಲ್ಲ, ಇನ್ನು ಮುಂದೆಯೂ ಇರಲಾರದು ಎಂದು ಹೇಳಿದೆ. ಇಂತಹ ಭಾಷೆಯನ್ನು ಹೊಂದುವಷ್ಟು ನಾಗರಿಕರಾಗಿಲ್ಲ ಹಿಂದೂಗಳು ಎಂದೆ. ಅವನು ನನ್ನನ್ನು ನಂಬದವನೆಂದೂ, ಕಪಟಿಯೆಂದೂ ಭಾವಿಸಿ ಹೊರಟು ಹೋದನು. ಅನಂತರ ಅವನು ಯಾವುದೊ ಹುಚ್ಚರ ಆಸ್ಪತ್ರೆಯಲ್ಲಿರುವನು ಎಂಬುದನ್ನು ಕೇಳಿದರೆ ಆಶ್ಚರ್ಯವೇನೂ ಆಗುವುದಿಲ್ಲ. ಈ ಜಗತ್ತಿನಲ್ಲಿ ಯಾವಾಗಲೂ ಇರುವ ಎರಡು ಅಪಾಯಗಳು ಮೂಢರಿಂದ ಮತ್ತು ವಂಚಕರಿಂದ ಬರುವಂಥವು. ಆದರೆ ಇದು ನಮ್ಮನ್ನು ಕುಗ್ಗಿಸಬೇಕಾಗಿಲ್ಲ. ಏಕೆಂದರೆ ಈ ಪ್ರಪಂಚದಲ್ಲಿರುವ ಎಲ್ಲ ಶ್ರೇಷ್ಠ ವಿಷಯಗಳಿಗೂ ಈ ಅಪಾಯವಿದೆ. ನಾವು ಸ್ವಲ್ಪ ಜೋಪಾನವಾಗಿರಬೇಕು ಎಂಬುದನ್ನು ಇದು ತೋರುವುದು. ಕೆಲವು ವೇಳೆ ಜನರು ಮಾತನಾಡುವಾಗ ಅವರಲ್ಲಿ ಯಾವ ತರ್ಕಸರಣಿಯೂ ಇರುವುದಿಲ್ಲ. ಒಬ್ಬ ನನ್ನ ಬಳಿಗೆ ಬಂದು ಯಾವುದೋ ಒಂದು ದೇವತೆಯಿಂದ ನನಗೆ ಒಂದು ಸಂದೇಶ ಬಂದಿದೆ ಎಂದು ಹೇಳುತ್ತಾನೆ. ನೀವು ಇದನ್ನು ಒಪ್ಪುವುದಿಲ್ಲವೇ? ಇಂತಹ ದೇವರು ಇರುವನು, ಅವನು ಸಂದೇಶವನ್ನು ಕೊಡುವನು, ಎಂಬುದನ್ನು ನೀವು ನಂಬುವುದಿಲ್ಲವೇ ಎಂದು ಕೇಳುತ್ತಾನೆ. ಶೇಕಡಾ ತೊಂಬತ್ತು ಜನ ಮೂಢರು ಇದನ್ನು ನಂಬುತ್ತಾರೆ. ಇದಕ್ಕೆ ಸಾಕಾದಷ್ಟು ಪ್ರಮಾಣವಿದೆ ಎಂದು ನಂಬುತ್ತಾರೆ. ಆದರೆ ಈ ಪ್ರಪಂಚದಲ್ಲಿ ಏನು ಬೇಕಾದರೂ ಆಗಬಹುದು. ಇದು ಸಾಧ್ಯ ಎಂಬುದನ್ನು ನೀವು ತಿಳಿದಿರಬೇಕು. ಈ ಭೂಮಿ ಯಾವುದೋ ಒಂದು ನಕ್ಷತ್ರದ ಆಕರ್ಷಣೆಗಳಿಗೆ ಒಳಗಾಗಿ ಬರುವ ವರುಷ ನುಚ್ಚುನೂರಾಗಬಹುದು. ನಾನು ಏನಾದರೂ ಹಾಗೆ ಹೇಳಿದರೆ ಅದಕ್ಕೆ ತಕ್ಕ ಕಾರಣವನ್ನೂ ಕೊಡಿ ಎಂದು ನನ್ನನ್ನು ಕೇಳುವುದಕ್ಕೆ ನಿಮಗೆ ಅಧಿಕಾರವಿದೆ. ಯಾರು ಇದು ನಿಜ ಎಂದು ಸಾಧಿಸುತ್ತಾರೋ, ಅವರು ಅದನ್ನು ಸಪ್ರಮಾಣವಾಗಿ ಸಿದ್ದಾಂತ ಮಾಡಬೇಕಾಗಿದೆ. ವಕೀಲರ ವೃತ್ತಿಯಂತೆ ಅದನ್ನು ಪ್ರಮಾಣಪಡಿಸುವುದು ಅದನ್ನು ಸಾಧಿಸುವವನ ಮೇಲಿದೆ. ನಾನು ಯಾವುದೊ\break ದೇವತೆಯಿಂದ ಸ್ಪೂರ್ತಿ ಪಡೆದಿರುವುದನ್ನು ಪ್ರಮಾಣೀಕರಿಸುವುದು ನಿಮ್ಮ ಕರ್ತವ್ಯವಲ್ಲ, ಆದರೆ ಅದು ನನ್ನ ಕರ್ತವ್ಯ. ಏಕೆಂದರೆ ಆ ಹೇಳಿಕೆಯನ್ನು ನೀಡಿದವನು ನಾನು. ಅದನ್ನು ನಾನು ಸಪ್ರಮಾಣವಾಗಿ ಸಿದ್ದಾಂತಪಡಿಸಲು ಆಗದೆ ಹೋದರೆ ಸುಮ್ಮನೆ ಇರಬೇಕು. ಈ ಎರಡು ಅಪಾಯಗಳಿಂದಲೂ ಪಾರಾದರೆ ನೀವು ಯಾವುದನ್ನು ಬೇಕಾದರೂ ಪಡೆಯಬಹುದು. ನಮ್ಮಲ್ಲಿ ಅನೇಕರಿಗೆ ಎಷ್ಟೋ ಸಂದೇಶಗಳು ಬರುತ್ತವೆ ಅಥವಾ ಬಂದಿವೆ ಎಂದು ಭಾವಿಸುತ್ತೇನೆ. ಅವು ಎಲ್ಲಿಯವರೆಗೆ ಕೇವಲ ನಮ್ಮ ಜೀವನಕ್ಕೆ ಮಾತ್ರ ಅನ್ವಯಿಸುವುವೋ ಅಲ್ಲಿಯವರೆಗೆ ನಮಗೆ ತೋಚಿದ್ದನ್ನು ಮಾಡುತ್ತಿರಬಹುದು. ಆದರೆ ಅದು ಇತರರೊಡನೆ ನಮಗಿರುವ ಸಂಬಂಧಕ್ಕೆ ಅನ್ವಯಿಸುವುದಾಗಿದ್ದರೆ ಆಗ ಅದರಂತೆ ಮಾಡುವುದಕ್ಕೆ ಮುಂಚೆ ನೂರು ಬಾರಿ ಚೆನ್ನಾಗಿ ಆಲೋಚಿಸಬೇಕು – ಆಗ ಮಾತ್ರ ನಾವು ಸುರಕ್ಷಿತವಾಗಿರಬಹುದು.

\newpage

ಈ ಸ್ಪೂರ್ತಿಯೊಂದೇ ಧರ್ಮಕ್ಕೆ ಇರುವ ಏಕಮಾತ್ರ ಮಾರ್ಗ ಎಂಬುದನ್ನು ನಾವು ನೋಡುವೆವು. ಆದರೂ ಈ ಮಾರ್ಗದಲ್ಲಿ ಯಾವಾಗಲೂ ಅಪಾಯ ಬಹಳ. ಅಪಾಯಗಳಲ್ಲೆಲ್ಲ ಕೊನೆಯದೆ, ಬಹಳ ಉಗ್ರವಾಗಿರುವುದಾವುದೆಂದರೆ, ಅವುಗಳ ವಿಷಯದಲ್ಲಿ ವಿಶೇಷ ಹಕ್ಕನ್ನು ಪಡೆಯಲು ಬಯಸುವುದು. ಕೆಲವರು ಎದ್ದು ನಿಂತು ತಮಗೆ ದೇವರಿಂದ ಆದೇಶ ಬಂದಿದೆ ಎಂದು ಹೇಳುತ್ತಾರೆ, ಸರ್ವೇಶ್ವರನಾದ ಭಗವಂತ ತಮ್ಮ ಮೂಲಕ ಮಾತನಾಡುತ್ತಿರುವನು, ಇತರರಿಗೆ ಯಾವಾಗಲೂ ಈ ಹಕ್ಕು ಇರಲಿಲ್ಲ ಎಂದು ಹೇಳುವರು. ಇದನ್ನು ಕೇಳಿದೊಡನೆಯೇ ಅವಿವೇಕದ ಮಾತು ಎಂಬುದು ಸ್ವತಃ ಸಿದ್ಧವಾಗುತ್ತದೆ. ಈ ಪ್ರಪಂಚದಲ್ಲಿ ಏನಾದರೂ ಇದ್ದರೆ ಅದು ಸಾರ್ವತ್ರಿಕವಾಗಿರಬೇಕು. ಈ ಪ್ರಪಂಚದಲ್ಲಿ ಯಾವುದೂ ಒಂದಕ್ಕೆ ಮಾತ್ರ ಅನ್ವಯಿಸಿರುವುದಿಲ್ಲ. ಈ ಪ್ರಪಂಚವನ್ನೆಲ್ಲ ನಿಯಮ ಆಳುತ್ತಿದೆ. ಇದು ಎಲ್ಲಾ ಕ್ಷೇತ್ರಕ್ಕೂ ಅನ್ವಯಿಸುವುದು. ಎಲ್ಲಿಯೂ ಇದಕ್ಕೆ ವಿನಾಯಿತಿ ಇಲ್ಲ. ಆದಕಾರಣ ನಿಯಮ ಒಂದು ಕಡೆ ಕೆಲಸಮಾಡಿದರೆ ಎಲ್ಲಾ ಕಡೆಯೂ ಕೆಲಸ ಮಾಡಬೇಕು. ಪ್ರತಿಯೊಂದು ಕಣವೂ, ಅತಿ ದೊಡ್ಡ ಸೂರ್ಯ ಅಥವಾ ನಕ್ಷತ್ರ ಯಾವ ರೀತಿ ರಚಿತವಾಗಿದೆಯೊ ಅದೇ ರೀತಿ ರಚಿತವಾಗಿದೆ. ಎಂದಾದರೂ ಒಬ್ಬನಿಗೆ ಸ್ಪೂರ್ತಿ ಬಂದಿದ್ದರೆ ನಮ್ಮಲ್ಲಿ ಪ್ರತಿಯೊಬ್ಬರಿಗೂ ಅದು ಬರಲು ಸಾಧ್ಯ. ಅದೇ ಧರ್ಮ. ಈ ಭ್ರಾಂತಿ ಮತ್ತು ಮೋಹದಿಂದ ಪಾರಾಗಿ, ಕಾಪಟ್ಯ ಮತ್ತು ವಿಶೇಷ ಹಕ್ಕುಗಳಿಂದ ದೂರವಿರಿ. ಆಧ್ಯಾತ್ಮಿಕ ಸತ್ಯಗಳನ್ನು ಪ್ರತ್ಯಕ್ಷವಾಗಿ ಸಂದರ್ಶಿಸಿ. ಅಧ್ಯಾತ್ಮ ವಿಜ್ಞಾನದೊಂದಿಗೆ ಪ್ರತ್ಯಕ್ಷ ಸಂಬಂಧವನ್ನು ಕಲ್ಪಿಸಿಕೊಳ್ಳಿ. ಧರ್ಮವೆಂದರೆ ಬೇಕಾದಷ್ಟು ಸಿದ್ಧಾಂತಗಳನ್ನು, ಮೂಢಾಚಾರಗಳನ್ನು ನಂಬುವುದಲ್ಲ. ಚರ್ಚು ಅಥವಾ ದೇವಸ್ಥಾನಕ್ಕೆ ಹೋಗುವುದಲ್ಲ. ಯಾವುದೋ ಕೆಲವು ಶಾಸ್ತ್ರಗಳನ್ನು ಓದುವುದಲ್ಲ. ನೀನು ದೇವರನ್ನು ನೋಡಿರುವೆಯಾ? ನಿನಗೆ ಆತ್ಮನ ಪರಿಚಯವಾಗಿದೆಯೇ? ಇಲ್ಲದೇ ಇದ್ದರೆ ಅದನ್ನು ಪಡೆಯುವುದಕ್ಕೆ ನೀನು ಪ್ರಯತ್ನಿಸುತ್ತಿರುವೆಯಾ? ಧರ್ಮ ಎಂದರೆ ಈಗ ಇಲ್ಲಿ ಸಿಕ್ಕುವುದು. ಅದಕ್ಕಾಗಿ ಭವಿಷ್ಯದಲ್ಲಿ ಕಾಯಬೇಕಾಗಿಲ್ಲ. ಭವಿಷ್ಯ ಎಂದರೇನು? ಯಾವ ಮಿತಿಯೂ ಇಲ್ಲದ ವರ್ತಮಾನ ಕಾಲ ತಾನೆ? ಅನಂತ ಕಾಲ ಎಂದರೇನು? ಈಗಿರುವ ಕ್ಷಣಗಳನ್ನು ಅನಂತದಿಂದ ಗುಣಿಸುವುದು ಮಾತ್ರವಾಗಿದೆ. ಧರ್ಮ ಎಂದರೆ ಈಗ, ಇಲ್ಲಿ, ಈ ಜನ್ಮದಲ್ಲಿ ಪಡೆಯುವ ಅನುಭವವಾಗಿದೆ.

ಮತ್ತೊಂದು ಪ್ರಶ್ನೆ: ನಮ್ಮ ಗುರಿ ಏನು? ಈಗಿನ ಕಾಲದಲ್ಲಿ, ಮಾನವನು ಅನಂತವಾಗಿ ಮುಂದು ಮುಂದಕ್ಕೆ ಹೋಗುತ್ತಿರುವನು, ತಲುಪುವುದಕ್ಕೆ ಯಾವ ಪರಿಪೂರ್ಣವಾದ ಗುರಿಯೂ ಇಲ್ಲ ಎಂದು ಸಾಧಿಸುವನು. ಯಾವಾಗಲೂ ಮುಂದುವರಿಯುತ್ತಿರುವುದು, ಎಂದಿಗೂ ದೊರಕದೆ ಇರುವುದು. ಇದಕ್ಕೆ ಏನಾದರೂ ಅರ್ಥವಿರಲಿ, ಇದು ಎಷ್ಟೇ ಚಮತ್ಕಾರವಾಗಿ ಕಾಣಲಿ, ಈ ವಾದದಲ್ಲಿ ಹುರುಳಿಲ್ಲ. ಒಂದು ಸರಳ ರೇಖೆಯನ್ನು ದೀರ್ಘವಾಗಿ ಮುಂದುವರಿಸಿದರೆ ಅದೊಂದು ವೃತ್ತವಾಗುವುದು. ಅದು ಹೊರಟ ಸ್ಥಳಕ್ಕೆ ಬರುವುದು. ನೀನು ಎಲ್ಲಿ ಪ್ರಾರಂಭವಾದೆಯೊ ಅಲ್ಲಿ ಕೊನೆಗಾಣಬೇಕು. ನೀನು ದೇವರಿಂದ ಪ್ರಾರಂಭಿಸಿರುವುದರಿಂದ ಅಲ್ಲಿ ಕೊನೆಗಾಣಬೇಕು. ಅನಂತರ ಉಳಿಯುವುದೇನು? ವಿವರಗಳು – ಕೊನೆಯ ತನಕ ನಾವು ಇದನ್ನು ಮಾಡುತ್ತಿರಬೇಕು.

ಇನ್ನೊಂದು ಪ್ರಶ್ನೆಯೂ ಇದೆ: ನಾವು ಮುಂದುವರಿದಂತೆ ಹೊಸ ಹೊಸ ಸತ್ಯಗಳನ್ನು ಕಂಡು ಹಿಡಿಯುವೆವೆ? ಹೌದು ಮತ್ತು ಇಲ್ಲ. ಮೊದಲನೆಯದಾಗಿ ಧರ್ಮದ ವಿಷಯದಲ್ಲಿ ತಿಳಿಯಬೇಕಾದ ಹೊಸದೇನೂ ಇಲ್ಲ, ಅವೆಲ್ಲಾ ಹಿಂದೆಯೇ ಗೊತ್ತಾಗಿವೆ. ಈ ಪ್ರಪಂಚದಲ್ಲಿರುವ ಎಲ್ಲಾ ಧರ್ಮಗಳು, ನಮ್ಮಲ್ಲಿ ಒಂದು ಐಕ್ಯತೆ ಇದೆ ಎಂದು ಸಾರುವುವು. ನಾವು ಪರಮಾತ್ಮನೊಡನೆ ಒಂದಾದ ಮೇಲೆ ಮತ್ತಾವ ಪ್ರಗತಿಯೂ ಇಲ್ಲ. ಜ್ಞಾನ ಎಂದರೆ ವೈವಿಧ್ಯದಲ್ಲಿ ಐಕ್ಯತೆಯನ್ನು ಕಂಡುಹಿಡಿಯುವುದಾಗಿದೆ. ನಾನು ನಿಮ್ಮನ್ನು ಸ್ತ್ರೀ ಪುರುಷರಂತೆ ನೋಡುತ್ತೇನೆ. ಇದು ಭಿನ್ನತೆ. ನಾನು ನಿಮ್ಮನ್ನೆಲ್ಲಾ ಒಟ್ಟಿಗೇ ಸೇರಿಸಿ ಮಾನವರು ಎಂದಾಗ ಅದು ವೈಜ್ಞಾನಿಕವಾದ ಜ್ಞಾನವಾಗುವುದು. ರಸಾಯನಶಾಸ್ತ್ರವನ್ನು ಉದಾಹರಣೆಗೆ ತೆಗೆದುಕೊಳ್ಳಿ, ವಿಜ್ಞಾನಿಗಳು ತಮಗೆ ಗೊತ್ತಿರುವ ಎಲ್ಲಾ ವಸ್ತುಗಳ ಹಿಂದೆ ಇರುವ ಮೂಲಧಾತುಗಳನ್ನು ತಿಳಿದುಕೊಳ್ಳಲು ಯತ್ನಿಸುತ್ತಿರುವರು. ಸಾಧ್ಯವಾದರೆ ಯಾವ ಒಂದು ಧಾತುವಿನಿಂದ ಇವೆಲ್ಲಾ ಆಗಿದೆಯೋ ಅದನ್ನು ತಿಳಿದುಕೊಳ್ಳಲು ಯತ್ನಿಸುವರು. ಆ ಒಂದನ್ನು ತಿಳಿದುಕೊಳ್ಳುವ ಸಮಯ ಬರಬಹುದು. ಅದೇ ಎಲ್ಲಾ ವಸ್ತುಗಳಿಗೂ ಮೂಲ. ಅದನ್ನು ಪಡೆದ ಮೇಲೆ ಅದಕ್ಕಿಂತ ಮುಂದೆ ವಿಜ್ಞಾನಿಗಳು ಹೋಗಲಾರರು, ರಸಾಯನ ಶಾಸ್ತ್ರ ಅಲ್ಲಿಗೆ ಪೂರ್ಣಾವಸ್ಥೆಗೆ ಬಂದಂತೆ. ಇದರಂತೆಯೇ ಧಾರ್ಮಿಕ ವಿಜ್ಞಾನ ಕೂಡ. ನಾವು ಈ ಪರಿಪೂರ್ಣವಾದ ಐಕ್ಯತೆಯನ್ನು ಕಂಡುಹಿಡಿದರೆ, ಅದಕ್ಕಿಂತ ಮುಂದೆ ಇನ್ನು\break ಹೋಗಲಾಗುವುದಿಲ್ಲ.

ನಾನು ಮತ್ತು ನನ್ನ ತಂದೆ ಇಬ್ಬರೂ ಒಂದೇ ಎಂಬುದನ್ನು ಕಂಡುಹಿಡಿದಾಗ ಧರ್ಮವನ್ನು ಕುರಿತಂತೆ ಕೊನೆಯ ಮಾತನ್ನು ಆಡಿದ್ದು ಆಯಿತು. ಅನಂತರ ವಿವರಗಳು ಮಾತ್ರ ಉಳಿದಿವೆ. ನಿಜವಾದ ಧರ್ಮದಲ್ಲಿ ಅಂಧಶ್ರದ್ದೆ ಎನ್ನಬಹುದಾದ ನಂಬಿಕೆ ಇಲ್ಲ. ಯಾವ ಮಹಾ ಉಪದೇಶಕನೂ ಇದನ್ನು ಬೋಧಿಸಲಿಲ್ಲ. ಇದು ಅಧೋಗತಿಗೆ ಇಳಿದಾಗ ಮಾತ್ರ ಸಾಧ್ಯ. ಮೂಢರು ಯಾರಾದರೂ ಒಬ್ಬ ಮಹಾ ಪುರುಷನನ್ನು ಅನುಸರಿಸುತ್ತಿರುವೆವು ಎಂದು ನಟಿಸುವರು, ತಮಗೆ ಶಕ್ತಿ ಇಲ್ಲದೇ ಇದ್ದರೂ ಮಾನವಕೋಟಿಯನ್ನು ಸುಮ್ಮನೆ ನಂಬಿ ಎಂದು ಬೋಧಿಸುವರು. ಏನನ್ನು ನಂಬುವುದು? ಸುಮ್ಮನೆ ನಂಬುವುದು ಎಂದರೆ ಮಾನವನನ್ನು ಅಧೋಗತಿಗೆ ಒಯ್ದಂತೆ. ಸಾಧ್ಯವಾದರೆ ನಾಸ್ತಿಕರಾಗಿರುವುದಾದರೂ ಅಂಧಶ್ರದ್ದೆಗಿಂತ ಮೇಲು. ಮನುಷ್ಯನನ್ನು ಏತಕ್ಕೆ ಮೃಗದ ಅಧೋಗತಿಗೆ ಸೆಳೆಯುತ್ತೀರಿ? ಇದರಿಂದ ನಿಮಗೆ ಮಾತ್ರ ತೊಂದರೆ ಅಲ್ಲ. ನೀವು ಇಡೀ ಸಮಾಜಕ್ಕೆ ಅಪಾಯ ತರುವಿರಿ, ನಿಮ್ಮ ನಂತರ ಬರುವವರಿಗೆ ಅಪಾಯ ತರುವಿರಿ. ಎದ್ದು ನಿಂತು ವಿಚಾರ ಮಾಡಿ, ಯಾವುದನ್ನೂ ಹಾಗೆಯೇ ನಂಬಬೇಡಿ. ಧರ್ಮ ಎಂದರೆ ನಂಬುವುದಲ್ಲ, ಅದರಂತೆ ಇರುವುದು, ಅದರಂತೆ ಆಗುವುದು. ಇದೇ ಧರ್ಮ. ನಿಮಗೆ ಇದು ಲಭಿಸಿದ ಮೇಲೆ ನಿಮಗೆ ಧರ್ಮ ಸಿಕ್ಕಿದೆ. ಇದಕ್ಕೆ ಮುಂಚೆ ನೀವು ಮೃಗಗಳಿಗಿಂತ ಮೇಲಲ್ಲ. ಆ ಮಹಾಜ್ಞಾನಿ ಬುದ್ಧನು ಹೇಳುವುದು ಹೀಗೆ: “ನೀವು ಏನನ್ನು ಕೇಳಿರುವಿರೋ ಅದನ್ನು ನಂಬಬೇಡಿ; ಹಿಂದಿನಿಂದಲೂ ಬಂದಿದೆ ಎಂದು ನೀವು ಯಾವ ಸಿದ್ದಾಂತವನ್ನೂ ನಂಬಬೇಡಿ. ಯಾರೋ ಹಿಂದಿನ ಋಷಿಗಳು ಹೇಳುತ್ತಾರೆ ಎಂದು ನಂಬಬೇಡಿ. ಅಭ್ಯಾಸಬಲದಿಂದ ನೀವು ಯಾವುದಕ್ಕೆ ಅಂಟಿಕೊಂಡಿದ್ದೀರೋ ಅದನ್ನು\break ನಂಬಬೇಡಿ. ಹಿರಿಯರು, ಗುರುಗಳು ಹೇಳುತ್ತಾರೆ ಎಂದು ನಂಬಬೇಡಿ. ಎಲ್ಲವನ್ನೂ ವಿಮರ್ಶೆಮಾಡಿ, ಆಲೋಚನೆಮಾಡಿ; ಪ್ರತಿಫಲಗಳು ಯುಕ್ತಿಯುಕ್ತವಾಗಿದ್ದರೆ, ಅದರಿಂದ ಎಲ್ಲರಿಗೂ ಒಳ್ಳೆಯದಾದರೆ, ಅದನ್ನು ಸ್ವೀಕರಿಸಿ, ಅದರಂತೆ ಬಾಳಿ,''

