
\chapter[ಜೀವ, ಜಗತ್ ಮತ್ತು ಈಶ್ವರ]{ಜೀವ, ಜಗತ್ ಮತ್ತು ಈಶ್ವರ\protect\footnote{\engfoot{C. W. Vo1 II, P. 423}}}

ವೇದಾಂತ ತತ್ತ್ವದ ಪ್ರಕಾರ ಮನುಷ್ಯನಲ್ಲಿ ಮೂರು ವಸ್ತುಗಳಿವೆ. ಅತಿ ಹೊರಗಡೆ ಇರುವುದೇ ದೇಹ. ಮನುಷ್ಯನ ಸ್ಥೂಲವಸ್ತು ಇದು. ಅಲ್ಲೇ ಇಂದ್ರಿಯ ಕರಣಗಳು ಇರುವುದು. ಕಣ್ಣಲ್ಲ ನೋಡುವ ಇಂದ್ರಿಯ, ಅದು ಕೇವಲ ಉಪಕರಣ. ಅದರ ಹಿಂದೆ ನಿಜವಾದ ಇಂದ್ರಿಯವಿದೆ. ಇದರಂತೆಯೆ ಕಿವಿಗಳಲ್ಲ ಕೇಳುವುದು, ಅವು ಕೇವಲ ಉಪಕರಣ. ಅವುಗಳ ಹಿಂದೆ ಇಂದ್ರಿಯವಿದೆ. ಇದನ್ನೆ ಆಧುನಿಕ ಶಾರೀರಿಕ ಶಾಸ್ತ್ರದಲ್ಲಿ ಮಿದುಳು ಕೇಂದ್ರವೆನ್ನುವರು, ಕಣ್ಣುಗಳನ್ನು ಅಧೀನದಲ್ಲಿಟ್ಟಿರುವ ಕೇಂದ್ರ ನಾಶವಾದರೆ ಕಣ್ಣು ನೋಡಲಾರದು. ಇದರಂತೆಯೇ ಎಲ್ಲ ಕೇಂದ್ರಗಳು ಕೂಡ. ಇಂದ್ರಿಯಗಳು ಅವುಗಳ ಹಿಂದೆ ಮತ್ತೇನೋ ಇರುವತನಕ ಗ್ರಹಿಸುವುವು. ಆ ಮತ್ತೇನೂ ವಸ್ತುವೆ ಮನಸ್ಸಾಗಿದೆ. ಕೆಲವು ವೇಳೆ ನೀವು ಯಾವುದೋ ದೀರ್ಘ ಆಲೋಚನೆಯಲ್ಲಿ ತತ್ಪರರಾಗಿದ್ದಾಗ ಗಡಿಯಾರ ಹೊಡೆದರೂ ನಿಮಗೆ ಕೇಳಿಸುವುದಿಲ್ಲ. ಏತಕ್ಕೆ? ಕಿವಿ ಅಲ್ಲಿತ್ತು. ಶಬ್ದ ಸ್ಪಂದನ ಕಿವಿಗೆ ತಾಗಿ ಅದರ ಪ್ರತಿಕ್ರಿಯೆ ಮಿದುಳಿಗೆ ಒಯ್ಯಲ್ಪಟ್ಟಿತ್ತು. ಆದರೂ ನೀವು ಕೇಳಲಿಲ್ಲ. ಏಕೆಂದರೆ ಮನಸ್ಸು ಆ ಕೇಂದ್ರಕ್ಕೆ ಸೇರಿರಲಿಲ್ಲ. ಬಾಹ್ಯ ವಸ್ತುವಿನ ಸಂವೇದನೆಗಳು ಆಯಾ ಇಂದ್ರಿಯಕ್ಕೆ ಒಯ್ಯಲ್ಪಡುವುವು. ಮನಸ್ಸು ಅದರೊಂದಿಗೆ ಸೇರಿದಾಗ ಅದು ಸಂವೇದನೆಗಳನ್ನು ಸ್ವೀಕರಿಸಿ ಅದಕ್ಕೆ 'ನಾನು' ಎಂಬ ಅಹಂಕಾರವನ್ನು ಕೊಡುವುದು. ನಾನು ಯಾವುದಾದರೂ ಕೆಲಸದಲ್ಲಿ ನಿರತನಾಗಿದ್ದಾಗ ಸೊಳ್ಳೆಯೊಂದು ನನ್ನ ಬೆರಳನ್ನು ಕಚ್ಚಿತು ಎಂದು ಇಟ್ಟುಕೊಳ್ಳಿ. ಅದು ನನಗೆ ಗೊತ್ತಾಗುವುದಿಲ್ಲ. ಏಕೆಂದರೆ ಮನಸ್ಸು ಬೇರೆಲ್ಲೂ ಇದೆ. ಆನಂತರ ಹೊರಗಿನ ವಿಷಯ ತಾಗಿದ ಇಂದ್ರಿಯದೊಂದಿಗೆ ಸಂಬಂಧ ಬೆಳೆಸಿದಾಗ ಒಂದು ಪ್ರತಿಕ್ರಿಯೆಯಾಗುವುದು. ಈ ಪ್ರತಿಕ್ರಿಯೆಯಿಂದ ನನಗೆ ಸೊಳ್ಳೆಯ ಅರಿವಾಗುವುದು. ಬರೀ ಮನಸ್ಸು ಆಯಾ ಇಂದ್ರಿಯಗಳಿಗೆ ಸೇರುವುದು ಕೂಡ ಸಾಲದು. ಇಚ್ಛೆಯ ರೀತಿಯಲ್ಲಿ ಒಂದು ಪ್ರತಿಕ್ರಿಯೆ ಉಂಟಾಗಬೇಕು. ಈ ಪ್ರತಿಕ್ರಿಯೆಯುಂಟಾಗುವ ಸ್ವಭಾವವೆ ಬುದ್ದಿ. ಮೊದಲು ಇಂದ್ರಿಯಗಳೆಂಬ ಹೊರಗಿನ ಉಪಕರಣವಿರಬೇಕು. ಅನಂತರ ಆಯಾ ಇಂದ್ರಿಯಗಳಿಗೆ ಸಂಬಂಧ ಪಟ್ಟ ಕೇಂದ್ರ ಮೆದುಳಿನಲ್ಲಿರಬೇಕು. ಅನಂತರ ಮನಸ್ಸು ಆಯಾ ಕೇಂದ್ರದೊಂದಿಗೆ ಸೇರಬೇಕು. ಇವುಗಳೆಲ್ಲ ಪೂರ್ಣವಾದ ಮೇಲೆ "ನಾನು ಮತ್ತು ಹೊರಗಿನ ವಸ್ತು” ಎಂಬ ಭಾವನೆ ತಕ್ಷಣ ಹೊಳೆಯುವುದು. ಆಗ ಇಂದ್ರಿಯಗ್ರಹಣ ಅಥವಾ ಜ್ಞಾನ ಬರುವುದು. ಕೇವಲ ಉಪಕರಣವಾಗಿರುವ ಹೊರಗಿನ ಇಂದ್ರಿಯ ನಮ್ಮ ದೇಹದಲ್ಲಿದೆ. ಅದರ ಹಿಂದೆ ಸೂಕ್ಷ್ಮವಾಗಿರುವ ಇಂದ್ರಿಯ ಇದೆ. ಅನಂತರ ಮನಸ್ಸು, ಅನಂತರ ಬುದ್ದಿ, ಅನಂತರ 'ನಾನು ನೋಡುತ್ತೇನೆ, ನಾನು ಕೇಳುತ್ತೇನೆ' ಎಂಬ ಅಹಂಕಾರವಿದೆ. ಈ ಕಾರ್ಯಗಳೆಲ್ಲ ಒಂದು ಶಕ್ತಿಯಿಂದ ಆಗುವುವು. ಸಂಸ್ಕೃತದಲ್ಲಿ ಇದನ್ನು ಪ್ರಾಣವೆನ್ನುವರು. ಬಾಹ್ಯ ಇಂದ್ರಿಯಗಳಿರುವ ಸ್ಥೂಲ ದೇಹವನ್ನು ಸಂಸ್ಕೃತದಲ್ಲಿ ಸ್ಥೂಲಶರೀರವೆನ್ನುವರು. ಅದರ ಹಿಂದೆ ಆಯಾ\break ಕೇಂದ್ರಗಳು, ಮನಸ್ಸು, ಬುದ್ಧಿ, ಅಹಂಕಾರ ಇವು ಬರುವುವು. ಇವು ಮತ್ತು ಪ್ರಾಣಿಗಳು ಸೇರಿ ಸೂಕ್ಷ್ಮ ಶರೀರವಾಗುವುದು. ಇದು ಅತಿ ಸೂಕ್ಷ್ಮವಸ್ತುವಿನಿಂದ ಆಗಿದೆ. ಅದು ಎಷ್ಟು ಸೂಕ್ಷ್ಮ ಎಂದರೆ ಈ ದೇಹಕ್ಕೆ ಎಷ್ಟೇ ಆಘಾತವಾದರೂ ಅದು ನಾಶವಾಗುವುದಿಲ್ಲ. ನಾವು ನೋಡುವ ಸ್ಥೂಲ ದೇಹವು ಸ್ಥೂಲವಸ್ತುವಿನಿಂದ ಆದುದು. ಆದಕಾರಣ ಇದು ಯಾವಾಗಲೂ ನವೀಕರಿಸಲ್ಪಡುತ್ತಿರುತ್ತದೆ, ಬದಲಾಯಿಸುತ್ತಿರುತ್ತದೆ. ಆದರೆ ಇಂದ್ರಿಯಕೇಂದ್ರ, ಮನಸ್ಸು, ಬುದ್ಧಿ, ಅಹಂಕಾರಗಳು ಬಹಳ ಸೂಕ್ಷ್ಮವಸ್ತುವಿನಿಂದ ಆಗಿವೆ. ಅವು ಯುಗಯುಗಾಂತರದವರೆಗೆ ಬೇಕಾದರೂ ಇರಬಲ್ಲವು. ಯಾವುದೂ ಅವನ್ನು ತಡೆಯಲಾರದಷ್ಟು ಸೂಕ್ಷ್ಮವಾಗಿವೆ. ಅವು ಯಾವ ಆತಂಕವನ್ನಾದರೂ ತೂರಿಹೋಗಬಲ್ಲವು. ಸ್ಥೂಲ ದೇಹ ಜಡ. ಅದರಂತೆಯೆ ಸೂಕ್ಷ್ಮ ವಸ್ತುವಿನಿಂದ ಆದ ಸೂಕ್ಷ್ಮಶರೀರ ಕೂಡ. ಅದರಲ್ಲಿ ಒಂದು ಭಾಗವನ್ನು ಮನಸ್ಸು, ಇನ್ನೊಂದನ್ನು ಬುದ್ದಿ, ಮತ್ತೊಂದು ಅಹಂಕಾರ ಎಂದು ಕರೆದರೂ ಅವು ಯಾವುವನ್ನೂ ಜ್ಞಾತೃ ಎನ್ನಲಾಗುವುದಿಲ್ಲ. ಇವು ಯಾವುವೂ ಗ್ರಹಿಸಲಾರವು. ಯಾರಿಗಾಗಿ ಈ ಕ್ರಿಯೆಗಳೆಲ್ಲ ಆಗುತ್ತವೆಯೋ, ಯಾರು ಇದನ್ನೆಲ್ಲ ನೋಡುತ್ತಿರುವನೊ ಅಂತಹ ಸಾಕ್ಷಿ ಇವು ಆಗಲಾರವು. ಮನಸ್ಸು ಬುದ್ದಿ ಅಹಂಕಾರಗಳ ಚಲನವಲನವೆಲ್ಲ ಮತ್ತಾರಿಗೊ ಇರಬೇಕು. ಇವು ಸೂಕ್ಷ್ಮ ವಸ್ತುವಿನಿಂದ ಆಗಿರುವುದರಿಂದ ಸ್ವಯಂಪ್ರಭೆಯವುಗಳಲ್ಲ. ಅವುಗಳ ಕಾಂತಿ ಅವಕ್ಕೆ ಸೇರಿಲ್ಲ. ಮೇಜಿನ ಅಭಿವ್ಯಕ್ತಿ ಯಾವುದೋ ಒಂದು ವಸ್ತುವಿನಿಂದ ಆದುದಲ್ಲ. ಇದರ ಹಿಂದೆ ಮತ್ತಾರೋ ಇರಬೇಕು. ಅವನೇ ನಿಜವಾಗಿ ನೋಡುವವನು, ಜ್ಞಾತೃ, ಭೋಕ್ತೃ, ಸಂಸ್ಕೃತದಲ್ಲಿ ಇವನನ್ನೆ ನಿಜವಾದ ಆತ್ಮ ಎನ್ನುವುದು. ನಿಜವಾಗಿ ವಸ್ತುವನ್ನು ನೋಡುವವನು ಅವನು. ಹೊರಗಿನ ಇಂದ್ರಿಯ ಮತ್ತು ಕೇಂದ್ರಗಳು ಬಾಹ್ಯವಸ್ತುವನ್ನು ಸಂಗ್ರಹಿಸಿ, ಅವನ್ನು ಮನಸ್ಸಿಗೆ ಸಾಗಿಸುವುವು. ಮನಸ್ಸು ಬುದ್ದಿಗೆ ಒಯ್ಯುವುದು. ಬುದ್ದಿ ಒಂದು ಕನ್ನಡಿಯಂತೆ ಪ್ರತಿಬಿಂಬಿಸುವುದು. ಅದರ ಹಿಂದೆ ಆತ್ಮನಿರುವನು. ಅವನು ನೋಡುತ್ತಾ ಆಜ್ಞೆಮಾಡುವನು. ಈ ಉಪಕರಣಗಳನ್ನೆಲ್ಲ ಆಳುವವನೆ ಅವನು. ಅವನೆ ಮನೆಯ ಯಜಮಾನ, ದೇಹದ ಒಡೆಯ. ಅಹಂಕಾರ, ಬುದ್ದಿ, ಮನಸ್ಸು, ಕೇಂದ್ರ, ಇಂದ್ರಿಯ, ದೇಹವೆಲ್ಲ ಅವನು ಹೇಳಿದಂತೆ ಕೇಳುವುವು. ಅವನೇ ಇವುಗಳನ್ನೆಲ್ಲ ವ್ಯಕ್ತಗೊಳಿಸುವುದು. ಇದೇ ಆತ್ಮ. ಯಾವುದು ವಿಶ್ವದ ಒಂದು ಅಂಶದಲ್ಲಿದೆಯೋ ಅದು ವಿಶ್ವದಲ್ಲೆಲ್ಲ ಇರಬೇಕು. ಪ್ರಕೃತಿನಿಯಮ ಒಂದೇ ಸಮನಾಗಿದ್ದರೆ, ಇಡೀ ಜಗತ್ತು ಯಾವ ನಿಯಮದಿಂದ ಆಗಿದೆಯೊ ಅದೇ ನಿಯಮದ ಮೇಲೆ ಅದರ ಪ್ರತಿಯೊಂದು ಅಂಶವೂ ಆಗಿರಬೇಕು. ಸ್ಥೂಲ ಜಗತ್ತಿನ ಹಿಂದೆ ಸೂಕ್ಷ್ಮಜಗತ್ತು ಇರಬೇಕು. ನಾವು ಅದನ್ನೇ ಆಲೋಚನೆ ಎನ್ನುವೆವು. ಅದರ ಹಿಂದೆ ಒಂದು ಆತ್ಮನಿರಬೇಕು. ಅದರಿಂದಲೇ ಆಲೋಚನೆಯೆಲ್ಲ ಸಾಧ್ಯ. ಅದೇ ಆಜ್ಞೆ ಕೊಡುವುದು. ಅದೇ ವಿಶ್ವದ ಒಡೆಯ. ಪ್ರತಿಯೊಂದು ಮನಸ್ಸು, ದೇಹಗಳ ಹಿಂದೆ ಇರುವ ಆತ್ಮನನ್ನು ಪ್ರತ್ಯಗಾತ್ಮ ಎಂದು ಕರೆಯುತ್ತಾರೆ. ವಿಶ್ವದ ಹಿಂದೆ ಅದರ ಮಾರ್ಗದರ್ಶಕನಂತೆ ಅದರ ಸ್ವಾಮಿಯಂತೆ ಇರುವ ಆತ್ಮವೇ ಈಶ್ವರ.

\newpage

ಅನಂತರ ನಾವು ಪರ್ಯಾಲೋಚಿಸುವ ವಿಷಯ ಈ ವಸ್ತುಗಳೆಲ್ಲ ಎಲ್ಲಿಂದ ಬಂದುವು ಎಂಬುದು. ಬರುವುದೆಂದರೆ ಅರ್ಥವೇನು? ಶೂನ್ಯದಿಂದ ಏನೊ ಬರುವುದು ಎಂದರೆ ಅದು ಸಾಧ್ಯವಿಲ್ಲ. ಈ ಸೃಷ್ಟಿಯೆಲ್ಲ, ಈ ಅಭಿವ್ಯಕ್ತಿ ಶೂನ್ಯದಿಂದ ಬರಲಾರದು. ಕಾರಣವಿಲ್ಲದೆ ಏನೂ ಬರಲಾರದು. ಪರಿಣಾಮವೆಂದರೆ ಮಾರ್ಪಾಡಾದ ಕಾರಣ. ಇಲ್ಲೊಂದು ಗ್ಲಾಸಿದೆ. ನಾವು ಅದನ್ನು ಒಡೆದು ಚೂರು ಚೂರು ಮಾಡಿ ಪುಡಿ ಮಾಡಿ ರಾಸಾಯನಿಕ ದ್ರವ್ಯಗಳಿಂದ ಹೆಚ್ಚು ಕಡಮೆ ನಾಶ ಮಾಡಿದೆವು ಎಂದು ಇಟ್ಟು ಕೊಳ್ಳೋಣ. ಅದು ಶೂನ್ಯವಾಗುವುದೆ? ಎಂದಿಗೂ ಇಲ್ಲ. ಆಕಾರ ಒಡೆದು ಹೋಗುವುದು. ಆದರೆ ಅದು ಯಾವ ವಸ್ತುವಿನಿಂದ ಆಗಿದೆಯೊ ಅದು ಅಲ್ಲೇ ಇರುವುದು. ಅದು ನಮ್ಮ ಇಂದ್ರಿಯಕ್ಕೆ ಕಾಣದಂತೆ ಇರಬಹುದು. ಆದರೂ ಅದು ಇರುತ್ತದೆ. ಈ ವಸ್ತುವಿನಿಂದ ಮತ್ತೊಂದು ಗ್ಲಾಸು ಮುಂದೆ ಆದರೂ ಆಗಬಹುದು. ಇದು ಈ ಒಂದು ವಸ್ತುವಿನ ವಿಷಯದಲ್ಲಿ ನಿಜವಾದರೆ ಎಲ್ಲಾ ವಸ್ತುಗಳಿಗೂ ಅನ್ವಯಿಸುವುದು. ಶೂನ್ಯದಿಂದ ನಾವು ಏನನ್ನೂ ಸೃಷ್ಟಿಸಲಾರೆವು. ಅಥವಾ ಯಾವುದಾದರೂ ಒಂದು ವಸ್ತು ಶೂನ್ಯವಾಗುವುದೂ ಸಾಧ್ಯವಿಲ್ಲ. ಅದು ಸೂಕ್ಷ್ಮ, ಸೂಕ್ಷ್ಮತರವಾಗಬಹುದು, ಅನಂತರ ಸ್ಥೂಲ ಸ್ಥೂಲತರವಾಗಬಹುದು. ಮಳೆಯ ಹನಿ ಸಾಗರದಿಂದ ಆವಿಯಾಗಿ ಮೇಲೆದ್ದು ಬೆಟ್ಟಗಳ ಕಡೆಗೆ ಹೋಗುವುದು. ಅಲ್ಲಿ ಅದು ಪುನಃ ಮಳೆಯಾಗಿ ಭೂಮಿಗೆ ಬಿದ್ದು ಸಾವಿರಾರು ಮೈಲಿ ಹರಿದು ಸಾಗರವನ್ನು ಸೇರುವುದು. ಬೀಜ ಮರವನ್ನು ಸೃಷ್ಟಿಸುವುದು, ಮರ ಬೀಜ ಬಿಟ್ಟು ನಾಶವಾಗುವುದು. ಬೀಜ ಪುನಃ ಮತ್ತೊಂದು ಮರವಾಗುವುದು, ಅದು ಪುನಃ ಬೀಜವನ್ನು ಬಿಡುವುದು. ಹೀಗೆ ಆಗುತ್ತಿರುವುದು. ಹಕ್ಕಿಯನ್ನು ನೋಡಿ, ಅದು ಹೇಗೆ ಮೊಟ್ಟೆಯಿಂದ ಹೊರಗೆ ಬರುತ್ತದೆ, ಸುಂದರವಾದ ಹಕ್ಕಿಯಾಗುತ್ತದೆ, ಕೆಲವು ಕಾಲ ತನ್ನ ಬಾಳುವೆಯನ್ನು ಬಾಳಿ ಮೊಟ್ಟೆಯನ್ನು ಬಿಟ್ಟು ಸಾಯುವುದು. ಈ ಮೊಟ್ಟೆಯ ಭವಿಷ್ಯ ಹಕ್ಕಿಯ ಭ್ರೂಣಸ್ಥಿತಿ. ಇದರಂತೆಯೇ ಪ್ರಾಣಿಗಳು ಮತ್ತು ಮನುಷ್ಯರು ಕೂಡ. ಎಲ್ಲವೂ ಕೆಲವು ಮೂಲವಸ್ತುಗಳಿಂದ, ಸೂಕ್ಷ್ಮದಿಂದ, ಕೆಲವು ಬೀಜಗಳಿಂದ ಮೊದಲಾಗಿ, ಅದು ಬೆಳೆದಂತೆಲ್ಲ ಸ್ಥೂಲವಾಗುವುದು, ಪುನಃ ಕೊನೆಗೆ ಸೂಕ್ಷ್ಮಕ್ಕೆ ಹೋಗಿ ಅಲ್ಲಿ ನೆಲಸುವುದು. ವಿಶ್ವವೆಲ್ಲ ಹೀಗೆ ಸಾಗುತ್ತಿರುವುದು. ಒಂದು ಸಮಯ ಬರುವುದು, ಆಗ ಜಗತ್ತೆಲ್ಲ ಕರಗಿ ಹೋಗಿ ಸೂಕ್ಷ್ಮವಾಗಿ ಕೊನೆಗೆ ಸಂಪೂರ್ಣ ಮಾಯವಾಗುವುದು, ಅಂದರೆ ಸೂಕ್ಷ್ಮಸ್ಥಿತಿಯಲ್ಲಿ ಇರುವುದು. ಆಧುನಿಕ ಖಗೋಳಶಾಸ್ತ್ರ ಮುಂತಾದ ವಿಜ್ಞಾನ ಶಾಸ್ತ್ರಗಳ ಪ್ರಕಾರ ಪೃಥ್ವಿ ಕ್ರಮೇಣ ತಣ್ಣಗಾಗುತ್ತಿದೆ. ಕೊನೆಗೆ ಮುಂದೊಂದು ಕಾಲದಲ್ಲಿ ಅದು ಅತ್ಯಂತ ತಣ್ಣಗಾಗುತ್ತದೆ. ಅದು ಚೂರು ಚೂರಾಗಿ ಸೂಕ್ಷ್ಮ ಸೂಕ್ಷ್ಮವಾಗಿ ಪುನಃ ಆಕಾಶದಂತಾಗುವುದು. ಆದರೂ ಆ ಕಣಗಳೆಲ್ಲಾ ಇರುವುವು. ಇದರಿಂದಲೆ ಮತ್ತೊಂದು ಸಲ ಭೂಮಿ ಸೃಷ್ಟಿಯಾಗುವುದು, ಪುನಃ ಅದು ಮಾಯವಾಗಿ ಮತ್ತೊಂದು ಬರುವುದು. ಹೀಗೆ ಸೃಷ್ಟಿ ತನ್ನ ಮೂಲಕಾರಣಗಳಿಗೆ ಹೋಗುವುದು. ಪುನಃ ಆ ವಸ್ತುಗಳು ಹಿಂತಿರುಗಿ ಒಂದು ಆಕಾರವನ್ನು ಧರಿಸುವುವು. ಅಲೆ ಬಿದ್ದು ಪುನಃ ಎದ್ದು ಒಂದು ಆಕಾರವನ್ನು ಧರಿಸುವಂತೆ ಮೂಲ ಕಾರಣಕ್ಕೆ ಹಿಂತಿರುಗುವುದು, ಪುನಃ ಅಲ್ಲಿಂದ ಒಂದು ಆಕಾರವನ್ನು\break ಧರಿಸುವುದು. ಹೀಗೆ ಕಾರಣಗಳಿಗೆ ಹಿಂದಿರುಗುವುದು ಮತ್ತು ಹೊರಕ್ಕೆ ಬಂದು ರೂಪವನ್ನು ಪಡೆಯುವುದು. ಈ ಕ್ರಿಯೆಯನ್ನು ಸಂಸ್ಕೃತದಲ್ಲಿ ಸಂಕೋಚ, ವಿಕಾಸವೆನ್ನುವರು. ಇಡೀ ಬ್ರಹ್ಮಾಂಡ ಸಂಕೋಚವಾಗುವುದು. ಪುನಃ ವಿಕಾಸವಾಗುವುದು. ಆಧುನಿಕ ವಿಜ್ಞಾನಿಗಳ ಸ್ವೀಕಾರಕ್ಕೆ ಯೋಗ್ಯವಾಗುವಂತೆ ಸಂಕೋಚ ಮತ್ತು ವಿಕಾಸ ಎಂಬ ಪದಗಳನ್ನು ಬಳಸಬಹುದು. ವಿಕಾಸ ಎಂಬುದನ್ನು ಕೇಳಿದ್ದೀರಿ. ಅದರ ಪ್ರಕಾರ ಕೆಳಗಿನ ವರ್ಗದಿಂದ ಪ್ರತಿಯೊಂದೂ ಬೆಳೆದು ಮೇಲು ವರ್ಗಕ್ಕೆ ಹೋಗುತ್ತದೆ. ಇದು ಬಹಳ ನಿಜ. ಆದರೆ ಪ್ರತಿಯೊಂದು ವಿಕಾಸದ ಹಿಂದೆಯೂ ಒಂದು ಸಂಕೋಚವನ್ನು ಒಪ್ಪಲೇಬೇಕಾಗಿದೆ. ವಿಶ್ವದಲ್ಲಿ ವ್ಯಕ್ತವಾಗಿರುವ ಶಕ್ತಿಮೊತ್ತವೆಲ್ಲ ಎಲ್ಲ ಕಾಲದಲ್ಲಿಯೂ ಒಂದೇ ಸಮನಾಗಿದೆ. ವಸ್ತು ಅವಿನಾಶಿ ಎಂಬುದು ನಮಗೆ ವೇದ್ಯ. ನೀವು ಏನು ಮಾಡಿದರೂ ವಸ್ತುವಿನ ಒಂದು ಕಣವನ್ನೂ ಆಚೆಗೆ ತೆಗೆಯಲಾರಿರಿ. ನೀವು ಹೊಸತಾಗಿ ಒಂದು ಫುಟ್ ಪೌಂಡು ಶಕ್ತಿಯನ್ನು ಹಾಕಲಾರಿರಿ ಅಥವಾ ತೆಗೆಯಲಾರಿರಿ. ಯಾವಾಗಲೂ ಒಟ್ಟು ಮೊತ್ತ ಒಂದೇ ಸಮನಾಗಿರುವುದು. ಅಭಿವ್ಯಕ್ತಿ ಮಾತ್ರ ವ್ಯತ್ಯಾಸವಾಗುವುದು. ಕೆಲವು ವೇಳೆ ಸಂಕೋಚವಾಗುವುದು. ಕೆಲವು ವೇಳೆ ವಿಕಾಸವಾಗುವುದು. ಈಗಿನ ಸೃಷ್ಟಿ ಹಿಂದಿನ ಸೃಷ್ಟಿಯ ಸಂಕೋಚದಿಂದ ಬಂದಿದೆ. ಪುನಃ ಈಗಿನ ಸೃಷ್ಟಿ ಸೂಕ್ಷ್ಮ ಸೂಕ್ಷ್ಮವಾಗಿ ಸಂಕೋಚವಾಗುವುದು. ಇದರಿಂದ ಅನಂತರ ಪುನಃ ಒಂದು ಸೃಷ್ಟಿ ಬರುವುದು. ಇಡೀ ವಿಶ್ವವು ಹೀಗೆಯೇ ಸಾಗುತ್ತಿರುವುದು. ಶೂನ್ಯದಿಂದ ಹೊಸತಾಗಿ ಮತ್ತಾವುದೂ ಸೃಷ್ಟಿಯಾಯಿತೆಂಬುದಿಲ್ಲ. ಇನ್ನೂ ಉತ್ತಮವಾದ ಪದವನ್ನು ಬಳಸಬಹುದಾದರೆ ಅಭಿವ್ಯಕ್ತಿ ಎನ್ನಬಹುದು. ದೇವರೆ ಈ ಸೃಷ್ಟಿಯನ್ನು ಅಭಿವ್ಯಕ್ತಿಗೊಳಿಸುವನು. ಈ ವಿಶ್ವವು ಅವನ ನಿಃಶ್ವಾಸ. ಪುನಃ ಅವನಲ್ಲಿಗೆ ಹೋಗುವುದು. ಮತ್ತೆ ಅವನು ಅದನ್ನು ಹೊರಸೂಸುವನು. ವೇದದಲ್ಲಿ ಒಂದು ಸುಂದರವಾದ ಉಪಮಾನವನ್ನು ಕೊಡುವರು. ``ಆ ಸನಾತನ ವ್ಯಕ್ತಿ ಈ ಸೃಷ್ಟಿಯನ್ನು ಹೊರಗೆ ಉಸಿರಿನಂತೆ ಬಿಡುವನು, ಪುನಃ ಉಸಿರಿನಂತೆ ಒಳಗೆ ಸೆಳೆದುಕೊಳ್ಳುವನು.” ನಾವು ಒಂದು ಧೂಳಿನ ಕಣವನ್ನು ಉಸಿರಿನೊಡನೆ ಹೊರಗೆಸೆದು ಪುನಃ ಅದನ್ನು ಒಳಗೆ ಸೆಳೆದುಕೊಳ್ಳುವಂತೆ ಇದು. ಇದೇನೋ ಸರಿ, ಆದರೆ ಮೊದಲ ಸೃಷ್ಟಿ ಹೇಗಿತ್ತು ಎಂಬ ಪ್ರಶ್ನೆ ಹಾಕಬಹುದು. ಇದಕ್ಕೆ ಉತ್ತರವೆ, ಪ್ರಥಮ ಸೃಷ್ಟಿ ಎಂದರೆ ಅರ್ಥವೇನು? ಪ್ರಥಮಸೃಷ್ಟಿ ಎಂಬುದೇ ಇರಲಿಲ್ಲ. ಕಾಲಕ್ಕೆ ಒಂದು ಆದಿಯನ್ನು ನೀವು ಕೊಟ್ಟರೆ ಕಾಲದ ಭಾವನೆಯೆ ನಾಶವಾಗುವುದು. ಕಾಲ ಪ್ರಾರಂಭವಾದ ಸ್ಥಿತಿಯೊಂದನ್ನು ಆಲೋಚಿಸಲು ಯತ್ನಿಸಿ. ಈ ಸ್ಥಿತಿಯ ಆಚೆಗೂ ನಾವು ಕಾಲವನ್ನು ಕುರಿತು ಆಲೋಚಿಸಬೇಕಾಗಿದೆ. ದೇಶ ಪ್ರಾರಂಭವಾದ ಸ್ಥಿತಿಯೊಂದನ್ನು ಆಲೋಚಿಸಲು ಯತ್ನಿಸಿ, ಆ ಸ್ಥಿತಿಯ ಆಚೆಗೂ ದೇಶವನ್ನು ಕುರಿತು ಯೋಚಿಸಬೇಕಾಗಿದೆ. ಕಾಲ ದೇಶಗಳು ಅನಂತ, ಅವಕ್ಕೆ ಒಂದು ಆದಿ ಅಂತ್ಯವಿಲ್ಲ. ದೇವರು ಐದು ನಿಮಿಷದಲ್ಲಿ ಈ ವಿಶ್ವವನ್ನು ಸೃಷ್ಟಿಸಿ ಅನಂತರ ನಿದ್ದೆ ಮಾಡಲು ಹೋದನು. ಅಂದಿನಿಂದಲೂ ನಿದ್ದೆ ಮಾಡುತ್ತಿರುವನು ಎಂಬ ಭಾವನೆಗಿಂತ ಇದು ಮೇಲು. ನಿರಂತರ ಸೃಷ್ಟಿ ಶೀಲನಾದ ದೇವರ ಭಾವನೆಯನ್ನು ಈ ಸೃಷ್ಟಿವಾದವು ಕೊಡುವುದು. ಹಲವು ಅಲೆಗಳು ಬಿದ್ದು ಏಳುತ್ತಿವೆ. ದೇವರು ಈ ನಿರಂತರ\break ಕ್ರಮವನ್ನು ನಿರ್ದೆಶಿಸುತ್ತಿರುವನು. ವಿಶ್ವಕ್ಕೆ ಹೇಗೆ ಒಂದು ಆದಿ ಅಂತ್ಯವಿಲ್ಲವೊ\break ಅದರಂತೆಯೆ ದೇವರಿಗೂ ಕೂಡ. ಇದು ಅವಶ್ಯಕವಾಗಿ ಹೀಗೇ ಇರಬೇಕು. ಏಕೆಂದರೆ ಸ್ಥೂಲ ಸ್ಥಿತಿಯಲ್ಲಾಗಲಿ ಸೂಕ್ಷ್ಮಸ್ಥಿತಿಯಲ್ಲಾಗಲಿ ವಿಶ್ವವು ಇಲ್ಲದ ಒಂದು ಕಾಲವಿತ್ತು. ಎಂದರೆ, ಆಗ ದೇವರು ಇರಲಿಲ್ಲ ಎನ್ನಬೇಕಾಗುವುದು. ಏಕೆಂದರೆ ದೇವರು ವಿಶ್ವಸಾಕ್ಷಿ ಎಂದು ನಮಗೆ ಗೊತ್ತಿದೆ. ವಿಶ್ವವಿಲ್ಲದೆ ಇದ್ದರೆ ಅವನೂ ಇರಲಿಲ್ಲ. ಒಂದು ಭಾವನೆಯಿಂದ ಮತ್ತೊಂದು ಭಾವನೆ ಬರುವುದು. ಪರಿಣಾಮದ ಭಾವನೆಯಿಂದ ನಮಗೆ ಕಾರಣದ ಭಾವನೆ ಬರುವುದು. ಪರಿಣಾಮವಿಲ್ಲದೆ ಇದ್ದರೆ ಕಾರಣವೇ ಇರುತ್ತಿರಲಿಲ್ಲ. ವಿಶ್ವ ಹೇಗೆ ಆದಿ ಅಂತ್ಯ ರಹಿತವೊ ಹಾಗೆಯೆ ದೇವರೂ ಕೂಡ ಆದಿ ಅಂತ್ಯರಹಿತ ಎಂಬುದು ಸ್ವಾಭಾವಿಕವಾಗಿ ಸಿದ್ಧಿಸುವುದು.

ಜೀವ ಕೂಡ ಅನಂತವಾಗಿಯೇ ಇರಬೇಕು. ಏತಕ್ಕೆ? ಮೊದಲನೆಯದಾಗಿ ಆತ್ಮವು ಜಡವಸ್ತುವಲ್ಲವೆಂಬುದು ನಮಗೆ ಗೊತ್ತಿದೆ. ಅದು ಸ್ಥೂಲದೇಹವೂ ಅಲ್ಲ, ಮನಸ್ಸು ಅಥವಾ ಆಲೋಚನೆ ಎಂದು ನಾವು ಹೇಳುವ ಸೂಕ್ಷ್ಮದೇಹವೂ ಅಲ್ಲ. ಅದು ಭೌತದೇಹವೂ ಅಲ್ಲ, ಕ್ರೈಸ್ತರು ಹೇಳುವ ಆಧ್ಯಾತ್ಮಿಕ ತನುವೂ ಅಲ್ಲ. ಸ್ಥೂಲ ದೇಹ ಮತ್ತು ಆಧ್ಯಾತ್ಮಿಕ ತನು ಬದಲಾಯಿಸಬಹುದು. ಸ್ಥೂಲದೇಹ ಪ್ರತಿಕ್ಷಣವೂ ಬದಲಾಯಿಸುತ್ತಿರುವುದು. ಅದು ನಾಶವಾಗುವುದು, ಆದರೆ ಆಧ್ಯಾತ್ಮಿಕ ತನುವು ವ್ಯಕ್ತಿಯು ಮುಕ್ತನಾಗುವವರೆಗೂ ಇರುವುದು. ಅನಂತರ ಅದೂ ಬಿದ್ದು ಹೋಗುವುದು. ಮನುಷ್ಯ ಮುಕ್ತನಾದಾಗ ಅವನ ಆಧ್ಯಾತ್ಮಿಕ ತನು ಚದುರಿ ಹೋಗುತ್ತದೆ. ಪ್ರತಿಸಲ ಮನುಷ್ಯ ಸಾಯುವಾಗ ಸ್ಥೂಲ ದೇಹವು ನಾಶವಾಗುವುದು. ಆತ್ಮವು ಯಾವ ಕಣಗಳಿಂದಲೂ ಆಗದೆ ಇರುವುದರಿಂದ ಅವಿನಾಶಿಯಾಗಿರಬೇಕು. ನಾಶವೆಂದರೆ ಅರ್ಥವೇನು? ನಾಶವೆಂದರೆ ಯಾವ ವಸ್ತುವಿನಿಂದ ಅದು ಆಗಿದೆಯೋ ಆ ಮೂಲ ವಸ್ತುವಿಗೆ ಪುನಃ ಹಿಂತಿರುಗುವುದು ಎಂದು ಅರ್ಥ. ಈ ಗ್ಲಾಸನ್ನು ಒಡೆದುಹಾಕಿದರೆ, ಆ ವಸ್ತು ನಾಶವಾಗುವುದು, ಅಂದರೆ ಗ್ಲಾಸು ನಾಶವಾದಂತೆ. ನಾಶವೆಂದರೆ ಆ ವಸ್ತು ಕಣಗಳು ತಮ್ಮ ಮೂಲ ರೂಪಕ್ಕೆ ಹಿಂತಿರುಗುವುದು ಎಂದು. ಯಾವುದು ವಸ್ತುಕಣಗಳಿಂದ ಆಗಿಲ್ಲವೊ ಅದು ನಾಶವಾಗುವುದಿಲ್ಲ ಎಂಬುದು ಇದರಿಂದ ಸಿದ್ದಾಂತವಾದಹಾಗೆ ಆಯಿತು. ಆತ್ಮ ಯಾವ ವಸ್ತುಗಳ ಸಂಯೋಗದಿಂದಲೂ ಆಗಿಲ್ಲ. ಅದು ಅಖಂಡ, ಅವಿಭಾಜ್ಯ. ಆದಕಾರಣ ಇದು ಅವಿನಾಶಿಯಾಗಿರಬೇಕು. ಆದ ಕಾರಣವೇ ಇದಕ್ಕೆ ಯಾವ ಆದಿಯೂ ಇರಲಿಲ್ಲ. ಆದಕಾರಣ ಜೀವನಿಗೆ ಆದಿ ಅಂತ್ಯಗಳಿಲ್ಲ.

ನಮ್ಮಲ್ಲಿ ಮೂರು ವಸ್ತುಗಳಿವೆ. ಅನಂತವಾಗಿರುವ ಆದರೆ ಬದಲಾಗುತ್ತಿರುವ ಪ್ರಕೃತಿ ಇದೆ. ಒಟ್ಟು ಪ್ರಕೃತಿಗೆ ಆದಿಯೂ ಇಲ್ಲ, ಅಂತ್ಯವೂ ಇಲ್ಲ. ಆದರೆ ಅದರೊಳಗೆ ಬೇಕಾದಷ್ಟು ಬದಲಾವಣೆಗಳಿವೆ. ಇದು, ಸಾವಿರಾರು ವರ್ಷಗಳಿಂದ ಹರಿದು ಸಮುದ್ರಕ್ಕೆ ಸೇರುತ್ತಿರುವ ನದಿಯಂತೆ. ಯಾವಾಗಲೂ ಇದು ಒಂದೇ ನದಿಯಾಗಿರುವುದು. ಆದರೆ ಪ್ರತಿಕ್ಷಣವೂ ಬದಲಾಯಿಸುತ್ತಿದೆ. ನಿರಂತರವೂ ನೀರಿನ ಕಣ ತನ್ನ ಸ್ಥಾನವನ್ನು ಬದಲಾಯಿಸುತ್ತಿದೆ. ಅನಂತರ ಅಧಿಕಾರಿಯಾದ, ಎಲ್ಲವನ್ನೂ ಆಳುತ್ತಿರುವ ಈಶ್ವರನಿರುವನು. ಅನಂತರ ಆತ್ಮವಿದೆ. ದೇವರಂತೆ ಅದು ಅವಿಕಾರಿ ಅನಂತ; ಆದರೆ ಈಶ್ವರನ ಆಳ್ವಿಕೆಗೆ ಒಳಪಟ್ಟಿದೆ. ಒಬ್ಬ ಯಜಮಾನ, ಮತ್ತೊಬ್ಬ ಭೃತ್ಯ, ಮೂರನೆಯದೆ ಪ್ರಕೃತಿ.

ದೇವರು ವಿಶ್ವದ ಸೃಷ್ಟಿ, ಸ್ಥಿತಿ, ಪ್ರಳಯ ಕರ್ತೃವಾದುದರಿಂದ ಪರಿಣಾಮವನ್ನುಂಟುಮಾಡುವುದಕ್ಕೆ ಕಾರಣ ಇದ್ದೇ ತೀರಬೇಕು: ಇದು ಮಾತ್ರವಲ್ಲ, ಕಾರಣವೇ ಪರಿಣಾಮವಾಗುವುದು. ಗ್ಲಾಸು ಅದನ್ನು ತಯಾರುಮಾಡುವವನ ಕೈಯಲ್ಲಿರುವ ಕೆಲವು ವಸ್ತುಗಳಿಂದ ಮತ್ತು ಶಕ್ತಿಯಿಂದ ಆಗಿದೆ. ಗ್ಲಾಸಿನಲ್ಲಿ ಅವನ ಶಕ್ತಿ ಮತ್ತು ವಸ್ತುಗಳಿವೆ. ಅದರಲ್ಲಿ ಉಪಯೋಗಿಸಿರುವ ಶಕ್ತಿ ವಿವಿಧ ಕಣಗಳನ್ನು ಕೂಡಿಸುವ ಶಕ್ತಿಯಾಗಿದೆ. ಆ ಶಕ್ತಿ ಹೋದರೆ ಗ್ಲಾಸು ಒಡೆದು ಹೋಗುವುದು. ಅದಕ್ಕೆ ಉಪಯೋಗಿಸಿರುವ ವಸ್ತುಗಳು ಕೂಡ ನಿಸ್ಸಂಶಯವಾಗಿ ಗ್ಲಾಸಿನಲ್ಲಿದೆ. ಆದರೆ ಆಕಾರ ಮಾತ್ರ ಬದಲಾಯಿಸುತ್ತಿದೆ. ಕಾರಣವೆ ಪರಿಣಾಮವಾಗುವುದು. ನೀವು ಎಲ್ಲಿ ಒಂದು ಪರಿಣಾಮವನ್ನು ನೋಡಿದರೂ ಅದರ ಮೂಲಕಾರಣಕ್ಕೆ ವಿಶ್ಲೇಷಿಸಬಹುದು. ಕಾರಣವೇ ಪರಿಣಾಮವಾಗಿ ಕಾಣುವುದು. ದೇವರು ವಿಶ್ವಕ್ಕೆ ಕಾರಣನಾದರೆ, ವಿಶ್ವ ಅವನ ಪರಿಣಾಮವಾದರೆ, ದೇವರೆ ವಿಶ್ವವಾಗಿರುವನು ಎಂಬುದು ಸಿದ್ಧಿಸುವುದು. ಜೀವಿಗಳು ಪರಿಣಾಮವಾಗಿ, ದೇವರು ಕಾರಣವಾದರೆ, ದೇವರೆ ಜೀವಿಗಳಾಗಿರುವನು. ಆದಕಾರಣ ಪ್ರತಿಯೊಂದು ಜೀವಿಯೂ ಭಗವಂತನ ಅಂಶ, “ಬೆಂಕಿಯಿಂದ ಅನಂತ ಕಿಡಿಗಳೇಳುವಂತೆ ಆ ಬ್ರಹ್ಮನಿಂದ ಈ ಜೀವಿಗಳನ್ನು ಕೂಡಿರುವ ಜಗತ್ತು ಬರುವುದು.”

ಅನಂತ ಈಶ್ವರ, ಅನಂತ ಪ್ರಕೃತಿ ಇರುವುದನ್ನು ನೋಡಿದೆವು. ಅನಂತಾತ್ಮಗಳೂ ಇವೆ. ಇದೇ ಧರ್ಮದಲ್ಲಿ ಮೊದಲನೆ ಹಂತ. ಇದನ್ನು ದ್ವೈತ ಎನ್ನುವರು. ಈ ಹಂತದಲ್ಲಿ ವ್ಯಕ್ತಿಯು ತಾನು ಮತ್ತು ದೇವರು ಎಂದೆಂದಿಗೂ ಬೇರೆ ಎಂದು ಭಾವಿಸುವನು. ಈಶ್ವರ, ಜೀವ, ಜಗತ್ತು ಈ ಮೂರು ಬೇರೆ ಬೇರೆ. ಇದೇ ದ್ವೈತ. ಇದು ನೋಡುವವನು ಮತ್ತು ನೋಟ ಯಾವಾಗಲೂ ಬೇರೆ ಬೇರೆ ಎನ್ನುವುದು. ಮನುಷ್ಯ ಪ್ರಕೃತಿಯನ್ನು ನೋಡಿದಾಗ ಅವನು ದ್ರಕ್, ಪ್ರಕೃತಿ ದೃಶ. ಅವನು ದೃಗ್ ದೃಶ್ಯದ ಭೇದವನ್ನು ನೋಡುವನು. ಅವನು ದೇವರನ್ನು ನೋಡಿದಾಗ ಅದು ದೃಶ್ಯವಾಗುವುದು, ಇವನು ದೃಕ್ ಆಗುವನು. ಅವರಿಬ್ಬರೂ ಎಂದೆಂದಿಗೂ ಬೇರೆ. ದೇವರಿಗೂ ಮಾನವನಿಗೂ ಇರುವ ದ್ವೈತ ಭಾವನೆ ಇದು. ಸಾಧಾರಣವಾಗಿ ಧರ್ಮದಲ್ಲಿ ಇರುವ ಪ್ರಥಮ ಭಾವನೆ ಇದೇ.

ಅನಂತರ ನಾನು ಈಗತಾನೆ ನಿಮಗೆ ತೋರಿದ ಮತ್ತೊಂದು ಭಾವನೆ ಬರುವುದು. ದೇವರು ಸೃಷ್ಟಿಗೆ ಕಾರಣನಾದರೆ, ಸೃಷ್ಟಿ ಅವನ ಪರಿಣಾಮವಾದರೆ ದೇವರೇ ಜಗತ್ತು ಮತ್ತು ಜೀವ ಆಗಿರಬೇಕು. ಇವನು ಪೂರ್ಣವಾದ ದೇವರ ಒಂದು ಅಂಶವಾಗುವನು. ನಾವು ಸಣ್ಣ ವ್ಯಕ್ತಿಗಳು, ಆ ಅಗ್ನಿಯ ಕಿಡಿಗಳು; ಈ ವಿಶ್ವವೆಲ್ಲ ಆ ಭಗವಂತನ ಆವಿರ್ಭಾವ. ಇದೇ ಮುಂದಿನ ಮೆಟ್ಟಲು. ಇದನ್ನು ಸಂಸ್ಕೃತದಲ್ಲಿ ವಿಶಿಷ್ಟಾದ್ವೈತವೆನ್ನುವರು. ನನಗೆ ಒಂದು ದೇಹವಿದೆ, ಇದರಲ್ಲಿ ಒಂದು ಆತ್ಮವಿದೆ. ಆತ್ಮ ಈ ದೇಹದ ಮೂಲಕ ಇದೆ. ಇದರಂತೆಯೇ ಅನಂತ ಜೀವಿಗಳು ಮತ್ತು ಪ್ರಕೃತಿಯಿಂದ ಕೂಡಿದ ವಿಶ್ವವೆಲ್ಲ ದೇವರ ದೇಹ, ಪ್ರಳಯಸ್ಥಿತಿ ಬಂದಾಗ ವಿಶ್ವ ಸೂಕ್ಷ್ಮ, ಮತ್ತು ಸೂಕ್ಷ್ಮವಾಗುತ್ತದೆ. ಆದರೂ ದೇವರ ದೇಹವಾಗಿ ಉಳಿದಿರುವುದು. ಸ್ಥೂಲವಾಗಿ ಸೃಷ್ಟಿಯಾದ ಅನಂತರವೂ ಇದು ದೇವರ ದೇಹವಾಗಿಯೇ ಇರುವುದು. ಹೇಗೆ ಮಾನವನ ಆತ್ಮ ಅವನ ದೇಹ ಮತ್ತು ಮನಸ್ಸಿನ ಆತ್ಮವಾಗಿದೆಯೋ, ಹಾಗೆಯೆ ದೇವರು ಎಲ್ಲ ಆತ್ಮಗಳ ಆತ್ಮ. ಪ್ರತಿಯೊಂದು ಧರ್ಮದಲ್ಲಿಯೂ ಆತ್ಮದ ಆತ್ಮ ಎಂಬ ಭಾವನೆಯನ್ನು ನೀವು ಕೇಳಿರುವಿರಿ. ಹಾಗೆಂದರೆ ಇದೇ ಅರ್ಥ. ಅವನು ಅವುಗಳಲ್ಲಿರುವನು, ದಾರಿತೋರುವನು, ಅವನು ಆಳುವನು. ಮೊದಲನೆ ದ್ವೈತದ ದೃಷ್ಟಿಯಲ್ಲಿ ನಮ್ಮಲ್ಲಿ ಪ್ರತಿಯೊಬ್ಬರೂ ಒಂದು ಜೀವ. ಅದು ಈಶ್ವರ ಮತ್ತು ಜಗತ್ತಿನಿಂದ ಬೇರೆ. ಎರಡನೆಯದರಲ್ಲಿ ನಾವೇನೋ ಪ್ರತ್ಯೇಕ ಆತ್ಮ, ಆದರೆ ದೇವರಿಂದ ಬೇರೆ ಅಲ್ಲ. ನಾವು ಒಂದು ರಾಶಿಯಲ್ಲಿ ತೇಲುತ್ತಿರುವ ಕಣದಂತೆ; ಆ ರಾಶಿಯೇ ದೇವರು. ನಾವೆಲ್ಲ, ಬೇರೆ; ಆದರೆ ದೇವರಲ್ಲಿ ಒಂದು. ನಾವೆಲ್ಲ ಅವನಲ್ಲಿರುವೆವು, ಅವನ ಅಂಶ. ಅದಕ್ಕೇ ನಾವೆಲ್ಲ ಒಂದು ಆದರೂ ಒಂದು ಜೀವಕ್ಕೂ ಮತ್ತೊಂದು ಜೀವಕ್ಕೂ, ಜೀವಕ್ಕೂ ದೇವರಿಗೂ, ಪ್ರತ್ಯೇಕ ವ್ಯಕ್ತಿತ್ವವಿದೆ. ಅದು ಭಿನ್ನ ಮತ್ತು ಅಭಿನ್ನ.

ಅನಂತರ ಮತ್ತೂ ಸೂಕ್ಷ್ಮ ಪ್ರಶ್ನೆ ಏಳುವುದು. ಅದೇ, ಅನಂತದಲ್ಲಿ ಅಂಶಗಳು ಇರಬಲ್ಲವೆ ಎಂಬುದು. ಪೂರ್ಣದ ಭಾಗಗಳೆಂದರೆ ಅರ್ಥವೇನು? ಇದನ್ನು ನಾವು ತರ್ಕಿಸಿದರೆ, ಇದು ಅಸಾಧ್ಯ; ಅನಂತವನ್ನು ನಾವು ವಿಭಾಗಿಸಲು ಆಗುವುದಿಲ್ಲ. ಅದು ಯಾವಾಗಲೂ ಪೂರ್ಣವಾಗಿಯೇ ಇರುವುದು. ಅದನ್ನು ವಿಭಾಗಿಸಲು ಸಾಧ್ಯವಾದರೆ ಪ್ರತಿಯೊಂದು ಅಂಶವೂ ಅನಂತವಾಗುವುದು. ಎರಡು ಅನಂತಗಳು ಇರಲಾರವು. ಒಂದು ವೇಳೆ ಇದ್ದರೆ ಒಂದು ಮತ್ತೊಂದಕ್ಕೆ ಮಿತಿ ತರುವುದು. ಎರಡೂ ಸಾಂತವಾಗುವುವು. ಅನಂತ ಒಂದು ಮಾತ್ರ ಇರಬಲ್ಲುದು. ಅದು ಅವಿಭಾಜ್ಯ. ಅನಂತ ಏಕ, ಅನೇಕವಲ್ಲ. ಒಂದು ಅನಂತಾತ್ಮ ಸಹಸ್ರಾರು ಕನ್ನಡಿಗಳಲ್ಲಿ ಬೇರೆ ಬೇರೆ ಆತ್ಮದಂತೆ ಪ್ರತಿಬಿಂಬಿಸುತ್ತಿದೆ ಎಂಬುದು ನಿರ್ಣಯವಾದ ಹಾಗೆ ಆಯಿತು. ಈ ವಿಶ್ವದ ಹಿನ್ನೆಲೆಯಾಗಿರುವುದೆ ಅನಂತಾತ್ಮ, ಇದನ್ನೆ ನಾವು ಈಶ್ವರ ಎನ್ನುವುದು. ಈ ಮನುಷ್ಯನ ಹಿನ್ನೆಲೆಯಾಗಿರುವುದೂ ಅನಂತಾತ್ಮವೆ. ಇದನ್ನೇ ನಾವು ಜೀವ ಎನ್ನುವುದು.

