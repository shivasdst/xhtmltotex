
\chapter{ಜ್ಞಾನಯೋಗಕ್ಕೆ ಪೀಠಿಕೆ\protect\footnote{\enginline{* C.W, Vol. VI, P. 41}}}

ಇದು ಯೋಗದ ವೈಚಾರಿಕ ಮತ್ತು ತಾತ್ವಿಕ ವಿಭಾಗ. ಇದು ಬಹಳ ಕಷ್ಟ. ಇದರಲ್ಲಿ ಮೆಟ್ಟಲು ಮೆಟ್ಟಲಾಗಿ ನಿಮ್ಮನ್ನು ಕರೆದುಕೊಂಡು ಹೋಗುತ್ತೇನೆ.

ಯೋಗ ಎಂದರೆ ಮಾನವನನ್ನು ಮತ್ತು ದೇವರನ್ನು ಒಂದುಗೂಡಿಸುವ ಮಾರ್ಗ. ನಿಮಗೆ ಇದೊಂದು ಚೆನ್ನಾಗಿ ಗೊತ್ತಾದರೆ ದೇವರಿಗೆ ಮತ್ತು ಮಾನವನಿಗೆ ನೀವು ಯಾವ ವಿವರಣೆಯನ್ನು ಬೇಕಾದರೂ ಕೊಡಬಹುದು. ನೀವು ಕೊಡುವ ವಿವರಣೆಗಳಿಗೆಲ್ಲ ಯೋಗವು ಹೊಂದಿಕೊಳ್ಳುವುದು. ಬೇರೆ ಬೇರೆ ಪ್ರಕೃತಿಯವರಿಗೆ ಬೇರೆ ಬೇರೆ ಯೋಗಗಳಿವೆ ಎಂಬುದನ್ನು ನೀವು ಯಾವಾಗಲೂ ಜ್ಞಾಪಕದಲ್ಲಿಡಿ. ನಿಮಗೆ ಒಂದು ಸರಿಹೋಗದೆ ಇದ್ದರೆ ಮತ್ತೊಂದು ಸರಿಯಾಗಬಹುದು. ಧರ್ಮಗಳೆಲ್ಲ ತತ್ವ ಮತ್ತು ಅನುಷ್ಠಾನವೆಂದು ಎರಡು ಭಾಗಗಳಾಗಿವೆ. ಪಾಶ್ಚಾತ್ಯರು ಕೇವಲ ಸಿದ್ಧಾಂತಕ್ಕೆ ಮಾರುಹೋಗಿ, ಸತ್ಕಾರ್ಯವನ್ನು ಮಾಡುವುದೇ ಧರ್ಮದ ಆನುಷ್ಟಾನಿಕ ಅಂಶ ಎಂದು ತಿಳಿಯುತ್ತಾರೆ. ಯೋಗವು ಧರ್ಮದ ಅನುಷ್ಠಾನದ ಭಾಗ, ಧರ್ಮವೆಂದರೆ ಕೇವಲ ಸತ್ಕಾರ್ಯವಲ್ಲ, ಅದೊಂದು ಪ್ರಾಯೋಗಿಕ ಶಕ್ತಿ ಎಂಬುದನ್ನು ಇದು ತೋರಿಸುತ್ತದೆ.

ಹತ್ತೊಂಬತ್ತನೆಯ ಶತಮಾನದ ಪ್ರಾರಂಭದಲ್ಲಿ ಮಾನವ ಯುಕ್ತಿಯ ಮೂಲಕ ದೇವರನ್ನು ಕಂಡುಹಿಡಿಯಲು ಯತ್ನಿಸಿದ. ಇದರ ಪರಿಣಾಮವಾಗಿಯೇ ಶ್ರುತಿ ವಾಕ್ಕಿನಲ್ಲಿ ನಂಬಿಕೆ ಇಡದೆ ದೇವರಲ್ಲಿ ಆಸ್ತಿಕ್ಯ ಬುದ್ದಿಯನ್ನು ಬೆಳೆಸಿದರು. ಈ ವಿಚಾರ ಸರಣಿಯಲ್ಲಿ ಉಳಿದ ಅಲ್ಪ ಸ್ವಲ್ಪ ದೇವರ ಭಾವನೆ ಕೂಡ ಡಾರ್ವಿನ್ ಮತ್ತು ಮಿಲ್ ಇವರ ಪ್ರಭಾವದಿಂದ ಅಳಿಸಿಹೋಯಿತು. ಆಗ ಜನರು ಐತಿಹಾಸಿಕ ಧರ್ಮದ ಕಡೆ ಮತ್ತು ಹಲವು ಧರ್ಮಗಳನ್ನು ಹೋಲಿಸಿ ನೋಡುವ ಕಡೆ ತಿರುಗಿದರು. ಪ್ರಕೃತಿಯ ಆರಾಧನೆಯಿಂದ ಭಗವಂತನ ಭಾವನೆ ಉದಯಿಸಿತು ಎಂದು ಅವರು ಬಗೆದರು. (ಮ್ಯಾಕ್ಸ್ ಮುಲ್ಲರ್ ಬರೆದಿರುವ ಸೂರ್ಯಪೂಜೆ ಮುಂತಾದುವನ್ನು ನೋಡಿ, ಇತರರು ಧರ್ಮವು ಪಿತೃಪೂಜೆಯಿಂದ ಬಂದಿತು ಎಂದು ಹೇಳುತ್ತಾರೆ. ಹರ್ಬರ್ಟ್ ಸ್ಪೆನ್ಸರನ ಬರವಣಿಗೆಗಳನ್ನು ನೋಡಿ). ಆದರೆ ಒಟ್ಟಿನಲ್ಲಿ ಈ ಮಾರ್ಗಗಳಿಂದ ಪ್ರಯೋಜನವಾಗಲಿಲ್ಲ. ಮಾನವನು ಬಾಹ್ಯ ವಿಧಾನಗಳ ಮೂಲಕ ಸತ್ಯವನ್ನು ಪಡೆಯಲಾರ.

'ಒಂದು ಹಿಡಿ ಜೇಡಿಮಣ್ಣು ಯಾವುದರಿಂದ ಆಗಿದೆ ಎಂದು ನನಗೆ ಗೊತ್ತಾದರೆ, ಇಡಿ ರಾಶಿ ಯಾವುದರಿಂದ ಆಗಿದೆ ಎಂದು ಗೊತ್ತಾಗುವುದು.” ಇಡೀ ಬ್ರಹ್ಮಾಂಡ ಇದೇ ಯೋಜನೆಯ ಮೇಲೆ ರಚಿತವಾಗಿದೆ. ಒಂದು ಹಿಡಿ ಮಣ್ಣಿನಂತೆ ಮಾನವ ಕೂಡ ಇಡಿ ಬ್ರಹ್ಮಾಂಡದಲ್ಲಿ ಒಂದು ಅಂಶ. ಒಂದು ಕಣದಂತಿರುವ ಮಾನವನ ಆತ್ಮದ ಉಗಮವನ್ನೂ ಇತಿಹಾಸವನ್ನೂ ಅರಿತರೆ ಇಡೀ ಪ್ರಕೃತಿಯನ್ನೇ ಅರಿತಂತೆ. ಜನನ, ಬೆಳವಣಿಗೆ, ಅಭಿವೃದ್ಧಿ, ಕ್ಷಯ, ಮರಣ ಇದೇ ಪ್ರಪಂಚದಲ್ಲಿರುವ ವಸ್ತುಗಳೆಲ್ಲ ಅನುಸರಿಸುತ್ತಿರುವ ಕ್ರಮ. ಇದು ಸಸ್ಯಗಳಲ್ಲಿ ಹೇಗೋ ಮಾನವನಲ್ಲಿಯೂ ಹಾಗೆಯೇ. ವ್ಯತ್ಯಾಸವಿರುವುದು ಕಾಲದಲ್ಲಿ ಮಾತ್ರ. ಒಂದು, ಈ ವೃತ್ತವನ್ನೆಲ್ಲಾ ದಿನ ಒಂದರಲ್ಲಿ ಪೂರೈಸಬಹುದು. ಮತ್ತೊಂದು, ವೃತ್ತವನ್ನು ಪೂರೈಸುವುದಕ್ಕೆ ಎಪ್ಪತ್ತು ವರುಷಗಳನ್ನು ತೆಗೆದುಕೊಳ್ಳಬಹುದು. ಮಾರ್ಗಗಳು ಒಂದೇ. ವಿಶ್ವವು ಹೇಗೆ ಆಗಿದೆ ಎಂಬುದನ್ನು ತಿಳಿಯಬೇಕಾದರೆ ನಮ್ಮ ಮನಸ್ಸನ್ನು ವಿಶ್ಲೇಷಿಸುವುದರಿಂದ ಮಾತ್ರ ಸಾಧ್ಯ. ಧರ್ಮವನ್ನು ಅರ್ಥಮಾಡಿಕೊಳ್ಳಬೇಕಾದರೆ ಒಂದು ಸರಿಯಾದ ಮನಶ್ಶಾಸ್ತ್ರ ಆವಶ್ಯಕ. ಸತ್ಯವನ್ನು ಕೇವಲ ಯುಕ್ತಿಯ ಮೂಲಕ ಪಡೆಯಲು ಅಸಾಧ್ಯ. ಏಕೆಂದರೆ ಅಪೂರ್ಣವಾದ ಯುಕ್ತಿ ತನ್ನ ಮೂಲಸಿದ್ಧಾಂತಗಳನ್ನೇ ತಿಳಿಯಲಾರದು. ಆದಕಾರಣ ಮನಸ್ಸನ್ನು ತಿಳಿದುಕೊಳ್ಳಬೇಕಾದರೆ ಅದಕ್ಕೆ ಸಂಬಂಧಪಟ್ಟ ವಿಷಯಗಳನ್ನೆಲ್ಲಾ ಸಂಗ್ರಹಿಸಬೇಕು. ಅನಂತರ ಬುದ್ದಿ ಅದನ್ನೆಲ್ಲಾ ಜೋಡಿಸಿ ಹಿಂದೆ ಇರುವ ನಿಯಮವನ್ನು ಕಂಡು ಹಿಡಿಯುವುದು. ಬುದ್ಧಿಯು ಮನೆಯನ್ನು ಕಟ್ಟಬೇಕಾಗಿದೆ. ಆದರೆ ಇಟ್ಟಿಗೆ ಇಲ್ಲದೆ ಮನೆ ಆಗುವುದಿಲ್ಲ. ಅದು ಇಟ್ಟಿಗೆಯನ್ನು ತಯಾರುಮಾಡಲಾರದು. ವಾಸ್ತವಾಂಶಗಳನ್ನು ಕಂಡುಹಿಡಿಯಬೇಕಾದರೆ ಅದಕ್ಕೆ ಜ್ಞಾನಯೋಗವೇ ಸರಿಯಾದ ಮಾರ್ಗ.

ಮೊದಲನೆಯದಾಗಿ ಮನಸ್ಸಿಗೆ ಸಂಬಂಧಪಟ್ಟ ದೈಹಿಕಾಂಶವಿದೆ. ನಮಗೆ ಇಂದ್ರಿಯಗಳಿವೆ. ಅವು ಜ್ಞಾನೇಂದ್ರಿಯ ಕರ್ಮೇಂದ್ರಿಯಗಳೆಂದು ವಿಭಾಗವಾಗಿವೆ. ಇಂದ್ರಿಯಗಳೆಂದರೆ ಬಾಹ್ಯ ಇಂದ್ರಿಯಗಳನ್ನು ನಾನು ಉದ್ದೇಶಿಸುತ್ತಿಲ್ಲ. ನೋಟಕ್ಕೆ ಮುಖ್ಯ ಬಾಹ್ಯ ಕಣ್ಣು ಮಾತ್ರವಲ್ಲ, ಮಿದುಳಿನಲ್ಲಿ ಅದಕ್ಕೆ ಸಂಬಂಧಪಟ್ಟ ಕೇಂದ್ರ ಮುಖ್ಯ. ಇದರಂತೆಯೇ ಪ್ರತಿಯೊಂದು ಇಂದ್ರಿಯವೂ ಕೂಡ. ಇವುಗಳ ಕೇಂದ್ರಗಳೆಲ್ಲ ಮಿದುಳಿನಲ್ಲಿವೆ. ಕೇಂದ್ರವು ಪ್ರತಿಕ್ರಿಯೆಯನ್ನು ಉಂಟುಮಾಡಿದಾಗ ಮಾತ್ರ ಆ ವಸ್ತು ಗೋಚರಿಸುವುದು. ಜ್ಞಾನೇಂದ್ರಿಯ ಮತ್ತು ಕರ್ಮೇಂದ್ರಿಯಕ್ಕೆ ಸಂಬಂಧಪಟ್ಟ ನರಗಳೆರಡೂ ಒಂದು ವಸ್ತುವನ್ನು ಗ್ರಹಿಸುವುದಕ್ಕೆ ಆವಶ್ಯಕ.

ಅನಂತರ ಮನಸ್ಸು ಬರುವುದು. ಇದೊಂದು ಪ್ರಶಾಂತ ಸರೋವರದಂತೆ ಇದೆ. ಒಂದು ಕಲ್ಲನ್ನು ಅದಕ್ಕೆ ಎಸೆದರೆ ಅದು ಸ್ಪಂದಿಸುವುದು. ಈ ಸ್ಪಂದನಗಳೆಲ್ಲ ಒಟ್ಟಿಗೆ ಕಲೆತು ಕಲ್ಲಿನ ಮೇಲೆ ತಮ್ಮ ಪ್ರತಿಕ್ರಿಯೆಯನ್ನು ಬೀರುವುವು. ಇದು ಸರೋವರಕ್ಕೆಲ್ಲಾ ಹರಡಿ ಎಲ್ಲಾ ಕಡೆಗಳಲ್ಲಿಯೂ ಅದರ ಇರವನ್ನು ನೋಡಬಹುದು. ಮನಸ್ಸು ಒಂದು ಸರೋವರದಂತೆ, ಇದರಲ್ಲಿ ಯಾವಾಗಲೂ ಸ್ಪಂದನಗಳು ಏಳುತ್ತಿರುತ್ತವೆ. ಅವು ಮನಸ್ಸಿನ ಮೇಲೆ ತಮ್ಮ ಮುದ್ರೆಯನ್ನು ಒತ್ತುತ್ತಿರುತ್ತವೆ. ಇದರ ಪರಿಣಾಮವಾಗಿ ನಾನು ಎಂಬ ಅಹಂಕಾರ ಹುಟ್ಟುವುದು. ಈ ನಾನು ಎಂಬುದು ಶಕ್ತಿಯ ವೇಗವಾದ ಸಂವಹನ ಮಾತ್ರ. ಅದು ತನಗೆ ತಾನೆ ಸತ್ಯವಲ್ಲ.

ಮನಸ್ಸು ಅತಿ ಸೂಕ್ಷ್ಮವಾದ ಉಪಕರಣ. ಇದು ಪ್ರಾಣವನ್ನು ಕೊಂಡೊಯ್ಯುವುದಕ್ಕೆ ಸಹಕಾರಿ. ಮನುಷ್ಯ ಸತ್ತರೆ ಅವನ ದೇಹ ನಾಶವಾಗುವುದು. ಆದರೆ ಸ್ವಲ್ಪ ಮನಸ್ಸು, ಎಂದರೆ ಅದರ ಬೀಜಮಾತ್ರ ಹಿಂದೆ ಉಳಿಯುವುದು. ಇದೇ ಸಂತ ಪಾಲ್ ಹೇಳುವ ಆಧ್ಯಾತ್ಮಿಕ ತನು ಎಂಬುದಕ್ಕೆ ಬೀಜವಾಗಿರುವುದು. ಮನಸ್ಸು ಭೌತಿಕವಾದುದು ಎಂಬುದನ್ನು ಆಧುನಿಕ ಸಿದ್ಧಾಂತಗಳೆಲ್ಲ ಅನುಮೋದಿಸುತ್ತವೆ. ಮೂರ್ಖನಿಗೆ ಬುದ್ದಿಯಿಲ್ಲ. ಏಕೆಂದರೆ ಅವನ ಮನಸ್ಸಿಗೆ ಅಪಾಯ ತಗುಲಿದೆ. ಚೈತನ್ಯವು ಯಾವ ಭೌತವಸ್ತುವಿನಲ್ಲಿಯೂ ಇರಲಾರದು, ಅದು ಯಾವ ಭೌತ ವಸ್ತುವಿನ ಮಿಶ್ರಣದಿಂದಲೂ ಆಗಲಾರದು. ಹಾಗಾದರೆ ಚೈತನ್ಯ ಇರುವುದೆಲ್ಲಿ? ಅದು ಭೌತವಸ್ತುವಿನ ಹಿಂದೆ ಇದೆ. ಅದೇ ಜೀವ. ಅದು ಭೌತ ವಸ್ತುವೆಂಬ ಉಪಕರಣದ ಮೂಲಕ ಕೆಲಸ ಮಾಡುತ್ತಿದೆ. ಶಕ್ತಿಯನ್ನು ರವಾನಿಸಬೇಕಾದರೆ ಅದಕ್ಕೆ ಭೌತ ವಸ್ತು ಇರಬೇಕು. ಜೀವ ತಾನೇ ಸಂಚರಿಸಲು ಸಾಧ್ಯವಾಗದೆ ಇರುವುದರಿಂದ ಅದರ ಸಂಚಾರಕ್ಕೆ ಮಧ್ಯವರ್ತಿಯಂತೆ ಕೆಲಸಮಾಡಲು ಸ್ವಲ್ಪ ಮನಸ್ಸು ಹಿಂದೆ ಉಳಿದು ಮಿಕ್ಕಿರುವುದೆಲ್ಲ ಮೃತ್ಯುವಿನಿಂದ ನಾಶವಾಗುವುದು.

ವಿಷಯಗಳ ಗ್ರಹಣ (\enginline{perception}) ಹೇಗೆ ಆಗುವುದು? ಎದುರಿಗಿರುವ ಗೋಡೆ ಒಂದು ಸಂವೇದನೆಯನ್ನು ಕಳುಹಿಸುವುದು. ಆದರೆ ನನ್ನ ಮನಸ್ಸು ಒಂದು ಪ್ರತಿಕ್ರಿಯೆಯನ್ನು ಎಬ್ಬಿಸುವತನಕ ನಾನು ಗೋಡೆಯನ್ನು ನೋಡಲಾಗುವುದಿಲ್ಲ. ಎಂದರೆ, ಸುಮ್ಮನೆ ನೋಡಿದ ಮಾತ್ರಕ್ಕೆ ಮನಸ್ಸಿಗೆ ಗೋಡೆ ಎಂಬುದರ ಜ್ಞಾನ ಬರುವುದಿಲ್ಲ. ಮನಸ್ಸು ಎದುರಿಗಿರುವ ಗೋಡೆಯನ್ನು ಕಾಣುವಂತೆ ಮಾಡುವ ಪ್ರತಿಕ್ರಿಯೆ ಬೌದ್ಧಿಕ ಕ್ರಿಯೆ. ಹೀಗೆ ಇಡಿ ವಿಶ್ವ, ನಮ್ಮ ಕಣ್ಣು ಮನಸ್ಸಿನಿಂದ ಗೋಚರಿಸುವುದು. ಆದಕಾರಣ ಇದು ನಮ್ಮ ವೈಯಕ್ತಿಕ ಸ್ವಭಾವಗಳಿಗೆ ಅನುಗುಣವಾಗಿ ಕಾಣುವುದು. ನಿಜವಾದ ಗೋಡೆ ಅಥವಾ ವಿಶ್ವ ಇರುವುದು ಮನಸ್ಸಿನ ಹೊರಗಡೆಯೇ. ಅದು ಅಜ್ಞಾತ ಮತ್ತು ಅಜೇಯ. ಬಾಹ್ಯಪ್ರಪಂಚವನ್ನು \enginline{x} ಎಂದು ಕರೆಯಬಹುದು. ನಮಗೆ ಕಾಣುವ ಪ್ರಪಂಚ \enginline{x +} ಮನಸ್ಸು.

ಯಾವುದು ಬಾಹ್ಯಜಗತ್ತಿನ ವಿಷಯದಲ್ಲಿ ಸತ್ಯವಾಗಿರುವುದೋ ಅದೇ ಆಂತರಿಕ ಜಗತ್ತಿನ ವಿಷಯಕ್ಕೂ ಅನ್ವಯಿಸುತ್ತದೆ. ಮನಸ್ಸು ಕೂಡ ತನ್ನನ್ನು ತಾನು ತಿಳಿಯಲು ಬಯಸುತ್ತದೆ. ಆದರೆ ಆತ್ಮನನ್ನು ಮನಸ್ಸೆಂಬ ಮಧ್ಯವರ್ತಿಯ ಮೂಲಕ ಮಾತ್ರ ತಿಳಿಯುವುದು ಸಾಧ್ಯ. ಆದಕಾರಣ ಇದು ಒಂದು ಗೋಡೆಯಂತೆ ಅಜೇಯವಾಗಿರುವುದು. ಈ ಆತ್ಮನನ್ನು \enginline{y} ಎಂದು ಕರೆಯಬಹುದು. \enginline{y+} ಮನಸ್ಸು ಆತ್ಮವಾಗುವುದು. ಮನಸ್ಸನ್ನು ಕುರಿತಂತೆ ಈ ವಿಶ್ಲೇಷಣೆಯನ್ನು ಮೊತ್ತ ಮೊದಲಿಗೆ ಮಾಡಿದವನು ಕಾಂಟ್ ಎಂಬ ತತ್ವಜ್ಞಾನಿ. ಆದರೆ ಬಹಳ ಹಿಂದೆಯೇ ವೇದದಲ್ಲಿ ಇದನ್ನು ಹೇಳಿರುವರು. ಮನಸ್ಸು \enginline{x} ಮತ್ತು \enginline{y} ಇವುಗಳ ಮಧ್ಯದಲ್ಲಿದ್ದು ಎರಡರ ಮೇಲೂ ತನ್ನ ಪ್ರಭಾವವನ್ನು ಬೀರುತ್ತಿರುವುದು.

\enginline{x} ಅಜ್ಞಾತವಾದರೆ ಅದಕ್ಕೆ ಆರೋಪಿಸುವ ಗುಣಗಳೆಲ್ಲ ನಮ್ಮ ಮನಸ್ಸು ಕೊಟ್ಟದ್ದೇ. ಮನಸ್ಸು ಕಾಲ-ದೇಶ-ನಿಮಿತ್ತಗಳೆಂಬ ಮೂರು ಸ್ಥಿತಿಗಳ ಮೂಲಕ ಗ್ರಹಿಸುವುದು. ಆಲೋಚನೆಯ ಸಂಚಾರಕ್ಕೆ ಕಾಲ ಬೇಕು. ಸ್ಥೂಲವಾಗಿರುವ ದ್ರವ್ಯದ ಸ್ಪಂದನಕ್ಕೆ ದೇಶ ಬೇಕು. ಈ ಸ್ಪಂದನಗಳು ನಿಮಿತ್ತವನ್ನು ಅನುಸರಿಸಿ ಬರುವುವು. ಮನಸ್ಸು ಇವುಗಳ ಮೂಲಕ ಮಾತ್ರ ಗ್ರಹಿಸಬಲ್ಲದು. ಆದಕಾರಣ ಮನಸ್ಸಿಗೆ ಅತೀತವಾಗಿರುವುದೆಲ್ಲ ಕಾಲ-ದೇಶ-ನಿಮಿತ್ತಗಳಿಗೆ ಅತೀತವಾಗಿರಬೇಕು.

ಕುರುಡನು ಜಗತ್ತನ್ನು ಕೇವಲ ಸ್ಪರ್ಶ ಮತ್ತು ಶಬ್ದಗಳಿಂದ ಮಾತ್ರ ಗ್ರಹಿಸಬಲ್ಲನು. ಪಂಚೇಂದ್ರಿಯಗಳಿರುವ ನಮಗೆ ಅದು ಬೇರೊಂದು ಜಗತ್ತಾಗಿದೆ. ನಮ್ಮಲ್ಲಿ ಯಾರಿಗಾದರೂ ವಿದ್ಯುತ್ತನ್ನು ನೋಡುವ ಒಂದು ಇಂದ್ರಿಯ ಪ್ರಾಪ್ತವಾದರೆ ಜಗತ್ತು ಬೇರೆಯೇ ಆಗಿ ತೋರುತ್ತದೆ. ಆದರೆ ಜಗತ್ತು \enginline{x} ನಂತೆ ಎಲ್ಲರಿಗೂ ಒಂದೇ. ಪ್ರತಿಯೊಬ್ಬರೂ ತಮ್ಮ ಮನಸ್ಸಿಗೆ ಅನುಗುಣವಾಗಿ ನೋಡುವುದರಿಂದ ತಮ್ಮ ಜಗತ್ತನ್ನೇ ಕಾಣುತ್ತಾರೆ. ಅದೇ \enginline{x+} ಒಂದು ಇಂದ್ರಿಯ, \enginline{x+} ಎರಡು ಇಂದ್ರಿಯ, ಹೀಗೆ ಮಾನವ ನಮಗೆ ತಿಳಿದಿರುವಂತೆ ಐದು ಇಂದ್ರಿಯಗಳವರೆಗೆ ಹೋಗುವನು. ಪರಿಣಾಮ ಯಾವಾಗಲೂ ಬೇರೆ ಆಗುತ್ತಿದ್ದರೂ \enginline{x} ಮಾತ್ರ ಯಾವಾಗಲೂ ಒಂದೇ ಸಮನಾಗಿರುವುದು. \enginline{y} ಕೂಡ ನಮ್ಮ ಮನಸ್ಸಿಗೆ ಅತೀತವಾಗಿದೆ. ಇದು ಕೂಡ ಕಾಲ-ದೇಶ-ನಿಮಿತ್ತಗಳಿಗೆ ಹೊರತಾಗಿದೆ.

ಹಾಗಾದರೆ ಕಾಲ-ದೇಶ-ನಿಮಿತ್ತಗಳಾಚೆ \enginline{x} ಮತ್ತು \enginline{y} ಎಂಬುವು ಎರಡು ಇವೆ ಎಂದು ಹೇಗೆ ಹೇಳುತ್ತೀರಿ? ಇದು ನಿಜ. ಕಾಲವೇ ಈ ವ್ಯತ್ಯಾಸಕ್ಕೆಲ್ಲ ಕಾರಣ. ಇವೆರಡೂ ನಿಜವಾಗಿಯೂ ಕಾಲಕ್ಕೆ ಹೊರಗೆ ಇರುವುದರಿಂದ ಇವೆರಡೂ ಒಂದೇ ಇರಬೇಕು. ಮನಸ್ಸು ಈ ಒಂದನ್ನು ನೋಡಿದಾಗ ಬೇರೆ ಬೇರೆ ಎಂದು ಕರೆಯುವುದು. ಬಾಹ್ಯ ಪ್ರಪಂಚವನ್ನು \enginline{x} ಎನ್ನುವುದು, ಆಂತರಿಕ ಪ್ರಪಂಚವನ್ನು \enginline{y} ಎನ್ನುವುದು. ಈ ಜೋಡಿಯನ್ನು ಮನಸ್ಸು ಎಂಬ ಭೂತಕನ್ನಡಿಯ ಮೂಲಕ ನೋಡುತ್ತೇವೆ.

ನಮ್ಮೆಲ್ಲರಿಗೂ ಏಕಪ್ರಕಾರವಾಗಿ ಕಾಣುವ ಪೂರ್ಣಾತ್ಮನೆ ದೇವರು ಅಥವಾ ನಿರ್ಗುಣ ಬ್ರಹ್ಮ, ಆ ನಿರ್ಗುಣವೇ ಪರಿಪೂರ್ಣ ಸ್ಥಿತಿ. ಉಳಿದವುಗಳೆಲ್ಲ ಅದಕ್ಕಿಂತ ಕೆಳಗಿನವು, ಅಸ್ಥಿರವಾದುವು.

ಅಖಂಡವಾಗಿರುವುದು ಖಂಡವಾಗಿ ಕಾಣುವಂತೆ ಮಾಡಿರುವುದು ಯಾವುದು? ಇದು, ಪಾಪ ಮತ್ತು ಸ್ವತಂತ್ರ ಇಚ್ಛೆ ಹೇಗೆ ಮೊದಲಾದುವು ಎಂಬ ಪ್ರಶ್ನೆಯಂತೆಯೇ ಇದೆ. ಈ ಪ್ರಶ್ನೆಯೇ ವಿರೋಧಾಭಾಸ, ಅಸಾಧ್ಯ. ಪ್ರಶ್ನೆಯು ಕಾರ್ಯಕಾರಣಗಳು ಇವೆಯೆಂಬುದನ್ನು ಒಪ್ಪಿಕೊಳ್ಳುತ್ತದೆ. ಅವ್ಯಕ್ತದಲ್ಲಿ ಕಾರ್ಯಕಾರಣಗಳು ಇರುವುದಿಲ್ಲ. ವ್ಯಕ್ತದಂತೆಯೇ ಅವ್ಯಕ್ತವೂ ಇರುವುದೆಂದು ಊಹಿಸುವುದು ಹೇಗೆ, ಎಲ್ಲಿಂದ ಎನ್ನುವುದೆಲ್ಲ ಮನಸ್ಸಿನಲ್ಲಿದೆ. ಆತ್ಮವು ಕಾರ್ಯಕಾರಣಗಳಿಗೆ ಅತೀತವಾದುದು. ಅದೊಂದೇ ಸ್ವತಂತ್ರ. ಅದರ ಬೆಳಕೇ ಮನಸ್ಸಿನ ಎಲ್ಲ ರೂಪಗಳ ಮೂಲಕ ಪ್ರಕಾಶಿಸುತ್ತದೆ. ಪ್ರತಿಯೊಂದು ಕೆಲಸದಲ್ಲಿಯೂ ನಾನು ಸ್ವತಂತ್ರನು ಎಂದು ಸಾಧಿಸುತ್ತೇನೆ. ಆದರೂ ಪ್ರತಿಯೊಂದು ಕೆಲಸವೂ ನಾನು ಮೊದಲು ಬದ್ದನೆನ್ನುವುದನ್ನು ದೃಢಪಡಿಸುವುದು. ನಿಜವಾದ ಆತ್ಮ ಸ್ವತಂತ್ರವಾದುದು. ಆದರೂ ಅದು ಮನಸ್ಸಿನೊಂದಿಗೆ ಮತ್ತು ದೇಹದೊಂದಿಗೆ ಬೆರೆತಾಗ ಸ್ವತಂತ್ರವಲ್ಲ. ನಿಜವಾದ ಆತ್ಮನ ಪ್ರಥಮ ಆವಿರ್ಭಾವವೇ ಇಚ್ಛೆ, ಆದ್ದರಿಂದ ನಿಜವಾದ ಆತ್ಮನ ಮೊದಲ ಮಿತಿಯೇ ಇಚ್ಛೆ, ಇದು ಒಂದು ಮಿಶ್ರಣ. ಇದರಲ್ಲಿ ಆತ್ಮ ಮತ್ತು ಮನಸ್ಸು ಬೆರೆತಿವೆ. ಯಾವ ಒಂದು ಮಿಶ್ರಣವೂ ನಿತ್ಯವಾಗಿರಲಾರದು. ನಾವು ಬದುಕಿರಬೇಕೆಂದು ಇಚ್ಛಿಸಿದರೂ ಸಾಯಬೇಕಾಗುವುದು. ಮರಣವಿಲ್ಲದ ಜೀವನ ಒಂದು ವಿರೋಧಾಭಾಸ. ಏಕೆಂದರೆ ಜೀವನ ಒಂದು ಮಿಶ್ರಣವಾಗಿರುವುದರಿಂದ ಅದು ಅಮರವಾಗಲಾರದು. ನಿಜವಾದ ಆತ್ಮ ಅವ್ಯಕ್ತವಾದುದು, ಸನಾತನವಾದುದು. ಈ ಪೂರ್ಣಾತ್ಮ, ಇಚ್ಛೆ ಮನಸ್ಸು ಆಲೋಚನೆ ಮುಂತಾದ ಕುಂದುಕೊರತೆಗಳಿಂದ ಕೂಡಿದ ವಸ್ತುವಿನೊಂದಿಗೆ ಹೇಗೆ ಮಿಶ್ರವಾಗಿರಬಲ್ಲದು? ಅದೆಂದಿಗೂ ಮಿಶ್ರವಾಗಿಲ್ಲ. ನಿಜವಾಗಿಯೂ ನೀವೇ ಹಿಂದೆ ಹೇಳಿದ y ಆಗಿರುವಿರಿ. ನೀವೆಂದೂ ಇಚ್ಚೆಯಲ್ಲ, ನೀವು ಎಂದಿಗೂ ಬದಲಾಗಿಲ್ಲ. ನೀವು ಒಂದು ವ್ಯಕ್ತಿಯಂತೆ ಎಂದಿಗೂ ಇರಲಿಲ್ಲ. ಇದೊಂದು ಭ್ರಾಂತಿ. ಹಾಗಾದರೆ ಭ್ರಾಂತಿಗೆ ಆಧಾರ ಯಾವುದು? ಇದು ಕೆಟ್ಟ ಪ್ರಶ್ನೆ. ಭ್ರಾಂತಿಗೆ ಆಧಾರ ಭ್ರಾಂತಿಯೇ ಹೊರತು ಸತ್ಯವಲ್ಲ. ಪ್ರತಿಯೊಂದು ವಸ್ತುವೂ ಭ್ರಾಂತಿಗಿಂತ ಮುಂಚೆ ತಾನಿದ್ದ ಸ್ಥಿತಿಗೆ, ನಿಜವಾದ ಸ್ವಾತಂತ್ರ್ಯದ ಸ್ಥಿತಿಗೆ ಹೋಗಲು ತವಕಿಸುತ್ತಿದೆ. ಹಾಗಾದರೆ ಜಗತ್ತಿನಿಂದ ನಮಗೆ ಪ್ರಯೋಜನವೇನು? ಇದು ನಮಗೆ ಅನುಭವವನ್ನು ಕೊಡುವುದು. ಇದು ವಿಕಾಸವಾದವನ್ನು ಅಲ್ಲಗಳೆಯುವುದೆ? ಅದರ ಬದಲು ಇದು ಅದನ್ನು ವಿವರಿಸುತ್ತದೆ. ಅದು ಭೌತವಸ್ತುವನ್ನು ಸಂಸ್ಕಾರಗೊಳಿಸುವ ವಿಧಾನ. ಇದು ನಿಜವಾದ ಆತ್ಮವು ಆವಿರ್ಭವಿಸುವುದಕ್ಕೆ ಅವಕಾಶ ಕೊಡುತ್ತದೆ. ಇದು ನಮಗೂ ಒಂದು ವಸ್ತುವಿಗೂ ಮಧ್ಯೆ ಇರುವ ಒಂದು ತೆರೆಯಂತೆ ಇದೆ. ತೆರೆ ತಗ್ಗಿದಂತೆ ಅತ್ತಕಡೆ ಇರುವ ವಸ್ತು ಚೆನ್ನಾಗಿ ಕಾಣುವುದು. ಪರಮಾತ್ಮನ ಆವಿರ್ಭಾವವೇ ಇಲ್ಲಿ ಮುಖ್ಯ ವಿಷಯ.

