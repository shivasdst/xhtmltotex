
\chapter[ಹಕ್ಕು ಬಾಧ್ಯತೆಗಳು]{ಹಕ್ಕು ಬಾಧ್ಯತೆಗಳು\protect\footnote{\engfoot{* C.W, Vol. I, P. 430}}}

\begin{center}
(ಲಂಡನ್ನಿನ ಸೆಸೇಮ್ ಕ್ಲಬ್ಬಿನಲ್ಲಿ ನೀಡಿದ ಪ್ರವಚನ)
\end{center}

ಪ್ರಕೃತಿಯಲ್ಲಿ ಎರಡು ಶಕ್ತಿಗಳು ಕೆಲಸಮಾಡುತ್ತಿರುವಂತೆ ತೋರುತ್ತಿರುವುದು. ಒಂದು ವ್ಯತ್ಯಾಸವನ್ನು ಕಲ್ಪಿಸುವಂಥದು, ಮತ್ತೊಂದು ಏಕತೆಯನ್ನು ಉಂಟು ಮಾಡುವಂಥದು. ಒಂದು ವ್ಯಕ್ತಿಯ ವೈಶಿಷ್ಟ್ಯಕ್ಕೆ ಕಾರಣವಾಗಿದೆ, ಮತ್ತೊಂದು ಸಾಮ್ಯತೆಯ ಗುಣವನ್ನು ತರುತ್ತಿದೆ; ಎಲ್ಲ ವ್ಯತ್ಯಾಸಗಳ ನಡುವೆಯೂ ಸಾಮಾನ್ಯತೆಯನ್ನು ತರುತ್ತಿದೆ. ಈ ಎರಡು ಶಕ್ತಿಗಳು ಪ್ರಕೃತಿಯ ಮತ್ತು ಜೀವನದ ಎಲ್ಲಾ ಕಾರ್ಯಕ್ಷೇತ್ರಗಳಲ್ಲಿಯೂ ವ್ಯಾಪಿಸಿರುವಂತೆ ಕಾಣುವುವು. ಭೌತಿಕ ಜಗತ್ತಿನಲ್ಲಿ ಈ ಎರಡು ಶಕ್ತಿಗಳು ಕೆಲಸ ಮಾಡುವುದನ್ನು ಸ್ಪಷ್ಟವಾಗಿ ನೋಡುತ್ತೇವೆ. ಅವು ವಸ್ತುಗಳನ್ನು ವಿಭಾಗಮಾಡಿ, ಒಂದು ಮತ್ತೊಂದಕ್ಕಿಂತ ಬೇರೆಯಾಗುವಂತೆ ಮಾಡುತ್ತಿವೆ. ಪುನಃ ಆ ಭಿನ್ನವಾದ ವಸ್ತುಗಳನ್ನೆಲ್ಲ ಬೇರೆ ಬೇರೆ ಜಾತಿ ಪಂಗಡಗಳಿಗೆ ಸೇರಿಸಿ ಅವುಗಳಲ್ಲಿರುವ ಸಾಮಾನ್ಯ ಗುಣಗಳನ್ನು ನಿರೂಪಿಸುವುವು. ಸಾಮಾಜಿಕ ಜೀವನದಲ್ಲಿ ಮತ್ತು ಮನುಷ್ಯನ ವೈಯಕ್ತಿಕ ಜೀವನದಲ್ಲಿಯೂ ಹೀಗೆಯೆ. ಸಮಾಜವು ಪ್ರಾರಂಭವಾದಾಗಿನಿಂದಲೂ ಭಿನ್ನ ಮಾಡುವ ಮತ್ತು ಐಕ್ಯಗೊಳಿಸುವ ಶಕ್ತಿಗಳು ಕೆಲಸ ಮಾಡುವುದನ್ನು ನೋಡುತ್ತೇವೆ. ಬೇರೆ ಬೇರೆ ಕಾಲದೇಶಗಳಲ್ಲಿ ಈ ಕ್ರಿಯೆ ಹಲವು ವೇಷಗಳನ್ನು ಧರಿಸುವುದು, ಹಲವು ಹೆಸರುಗಳಿಂದ ಇದನ್ನು ಕರೆಯುವರು. ಆದರೆ ಎಲ್ಲದರಲ್ಲಿಯೂ ಇರುವ ಸಾರ ಒಂದೆ. ಒಂದು ವ್ಯತ್ಯಾಸಕ್ಕೆ ಕಾರಣವಾಗಿದೆ, ಮತ್ತೊಂದು ಸಾಮಾನ್ಯತೆಗೆ ಕಾರಣವಾಗಿದೆ. ಒಂದು ಜಾತಿಯನ್ನು ನಿರ್ಮಿಸುತ್ತಿರುವುದು. ಮತ್ತೊಂದು ಜಾತಿಯನ್ನು ನಿರ್ಮೂಲಮಾಡುತ್ತಿರುವುದು. ಒಂದು ವರ್ಣವನ್ನು ಮತ್ತು ಹಕ್ಕು ಬಾಧ್ಯತೆಯನ್ನು ತರುತ್ತಿರುವುದು, ಮತ್ತೊಂದು ಅವನ್ನು ನಾಶಮಾಡುತ್ತಿರುವುದು. ಇಡೀ ವಿಶ್ವವೇ ಈ ಎರಡು ಶಕ್ತಿಗಳ ಸಮರಭೂಮಿಯಾಗಿದೆ. ಒಂದು ಕಡೆ ಈ ಸಾಮಾನ್ಯತೆಗೆ ತರುವ ಶಕ್ತಿ ಇದ್ದರೂ ನಾವು ನಮ್ಮ ಶಕ್ತಿಯನ್ನೆಲ್ಲ ಪ್ರಯೋಗಿಸಿ ಅದನ್ನು ವಿರೋಧಿಸಬೇಕು. ಏಕೆಂದರೆ ಅದು ನಮ್ಮನ್ನು ಮೃತ್ಯುವಿನ ಸಮ್ಮುಖಕ್ಕೆ ಒಯ್ಯುವುದು. ಪೂರ್ಣ ಏಕತೆ ಎಂದರೆ ಸಂಪೂರ್ಣ ನಾಶವೆ. ಈ ಪ್ರಪಂಚದಲ್ಲಿ ಭಿನ್ನತೆಗೆ ಕಾರಣವಾದ ಶಕ್ತಿ ಎಂದು ನಿಲ್ಲುವುದೋ ಆಗ ಸೃಷ್ಟಿಯೇ ಕೊನೆಗೊಂಡಂತೆ ಎಂದು ವಾದಿಸುವರು. ನಮ್ಮ ಎದುರಿಗೆ ಇರುವ ದೃಶ್ಯಗಳಿಗೆ ಕಾರಣ ವೈವಿಧ್ಯತೆ. ಸಾಮಾನ್ಯತೆ, ಎಲ್ಲವನ್ನೂ ಒಂದೆ ಸಮನಾಗಿರುವ ನಿರ್ಜಿವ ವಸ್ತುವನ್ನಾಗಿ ಮಾಡುವುದು. ಮಾನವಕೋಟಿ ಇದನ್ನು ಸಾಧ್ಯವಾದಷ್ಟು ತಡೆಯಲು ಯತ್ನಿಸುವುದು. ನಮ್ಮ ಸುತ್ತಲೂ ಕಾಣುವ ಎಲ್ಲಕ್ಕೂ ಇದೇ ವಾದವನ್ನು ಅನ್ವಯಿಸುವರು. ಭೌತಿಕ ದೇಹ ಮತ್ತು ಸಮಾಜದ ಪಂಗಡಗಳಲ್ಲಿ ಕೂಡ ನಿರಪೇಕ್ಷವಾದ ಏಕತ್ವವು ವ್ಯಕ್ತಿಯ ಮತ್ತು ಸಮಾಜದ ನಾಶಕ್ಕೆ ಕಾರಣ ಎನ್ನುವರು. ಆಲೋಚನೆಗಳು ಮತ್ತು ಭಾವನೆಗಳು ಒಂದೇ ಸಮನಾಗಿದ್ದರೆ ಅದೇ ಮಾನಸಿಕ ಅವನತಿಗೆ ಕಾರಣ. ಆದಕಾರಣ ಸಮಾನತೆಯನ್ನು ತ್ಯಜಿಸಬೇಕು, ಎಂದು ಒಂದು ಪಕ್ಷದವರು ವಾದಿಸುವರು. ಪ್ರತಿಯೊಂದು ದೇಶದಲ್ಲಿಯೂ ಬೇರೆ ಬೇರೆ ಕಾಲಗಳಲ್ಲಿ ಇದೇ ವಾದವನ್ನು ವಿಧವಿಧವಾಗಿ ತರುವರು. ಭರತಖಂಡದ ಬ್ರಾಹ್ಮಣರು ಕೂಡ ವರ್ಣಾಶ್ರಮಗಳ ವಿಭಜನೆಯನ್ನು ಮತ್ತು ಕೆಲವು ವರ್ಣಗಳಿಗೆ ಇರುವ ವಿಶೇಷ ಹಕ್ಕು ಬಾಧ್ಯತೆಗಳನ್ನು ಸಮರ್ಥಿಸುವಾಗ ಇದೇ ವಾದವನ್ನು ತರುವರು. ವರ್ಣನಾಶವೇ ಸಮಾಜದ ನಾಶಕ್ಕೆ ಕಾರಣ ಎಂದೂ, ತಮ್ಮ ಸಮಾಜವೇ ಪೃಥ್ವಿಯಲ್ಲಿ ಅತಿ ದೀರ್ಘಕಾಲ ಇರುವ ಜನಾಂಗ ಎಂದೂ ಧೈರ್ಯವಾಗಿ ಚಾರಿತ್ರಿಕ ಪ್ರಮಾಣವನ್ನು ಉದಾಹರಿಸುವರು. ಅವರು ಇದೇ ವಾದವನ್ನು ಬಲವಾಗಿ ಪ್ರತಿಪಾದಿಸುವರು. ಅವರು ಅಧಿಕಾರ ವಾಣಿಯಿಂದ ಯಾವುದು ವ್ಯಕ್ತಿಯ ದೀರ್ಘಾಯಸ್ಸಿಗೆ ಕಾರಣವೊ ಅದು ಅಲ್ಪಾಯಸ್ಸಿಗೆ ಕಾರಣವಾಗುವುದಕ್ಕಿಂತ ಮೇಲು ಎಂದು ಹೇಳುವರು.

ಮತ್ತೊಂದು ಪಕ್ಷದಲ್ಲಿ ಸಮಾನತೆಯನ್ನು ಒತ್ತಿ ಹೇಳುವವರ ಅನುಯಾಯಿಗಳು ಎಲ್ಲಾ ಕಾಲಗಳಲ್ಲಿಯೂ ಇದ್ದರು. ಉಪನಿಷತ್ತಿನ ಕಾಲದಿಂದಲೂ, ಬುದ್ದ ಕ್ರಿಸ್ತ ಮುಂತಾದ ಶ್ರೇಷ್ಠ ಧಾರ್ಮಿಕ ಬೋಧಕರಿಂದ ಹಿಡಿದು ನಮ್ಮಲ್ಲಿಯವರೆಗೆ, ಹೊಸ ರಾಜಕೀಯ ಆಕಾಂಕ್ಷೆಗಳಲ್ಲಿ, ದೀನದಲಿತರು ಕೇಳುವ ಹಕ್ಕಿನಲ್ಲಿ, ಮತ್ತು ಯಾವ ಹಕ್ಕು ಬಾಧ್ಯತೆಗಳೂ ಇಲ್ಲದವರಲ್ಲಿ ಈ ಏಕತ್ವ ಮತ್ತು ಸಮತ್ವದಭಾವನೆ ವ್ಯಕ್ತವಾಗಿರುವುದು. ಆದರೆ ಮಾನವಸ್ವಭಾವ ಯಾವಾಗಲೂ ತನ್ನನ್ನು ಪ್ರತಿಪಾದಿಸುವುದು. ಯಾರಿಗೆ ಕೆಲವು ಅನುಕೂಲತೆಗಳಿವೆಯೋ ಅವರಿಗೆ ಅವನ್ನು ಕಳೆದುಕೊಳ್ಳಲು ಇಚ್ಚೆಯಿಲ್ಲ. ಅವರು ಇದನ್ನು ಪ್ರತಿಪಾದಿಸಲು ಎಷ್ಟೇ ಏಕಪಕ್ಷವಾದ ಮತ್ತು ಗ್ರಾಮ್ಯವಾದ ವಾದವನ್ನಾಗಲಿ ಆಶ್ರಯಿಸುವರು. ಇದು ಎರಡೂ ಪಕ್ಷದವರಿಗೂ ಅನ್ವಯಿಸುವುದು.

ತತ್ತ್ವರಂಗದಲ್ಲಿ ಈ ಪ್ರಶ್ನೆ ಬೇರೊಂದು ರೂಪವನ್ನು ತಾಳುವುದು. ಬೌದ್ಧರು, ವೈವಿಧ್ಯದಿಂದ ತುಂಬಿದ ಪ್ರಕೃತಿಯಲ್ಲಿ ಏಕತೆಯನ್ನು ನಾವು ಅರಸಬೇಕಾಗಿಲ್ಲ, ಈ ಪ್ರಕೃತಿಯಲ್ಲೇ ನಾವು ತೃಪ್ತರಾಗಬೇಕು ಎಂದು ಹೇಳುತ್ತಾರೆ. ವೈವಿಧ್ಯವೇ ಜೀವನದ ಸಾರ, ಇದು ಎಷ್ಟೇ ದುರ್ಬಲವಾಗಿರಲಿ ದುಃಖಪೂರ್ಣವಾಗಿರಲಿ ಇದಕ್ಕಿಂತ ಮಿಗಿಲಾಗಿರುವುದನ್ನು ನಾವು ನಿರೀಕ್ಷಿಸಲಾರೆವು. ವೇದಾಂತಿಗಳು “ಏಕತೆಯೊಂದೇ ಇರುವುದು, ವೈವಿಧ್ಯ ಕ್ಷಣಿಕ, ತಾತ್ಕಾಲಿಕ ಭ್ರಾಂತಿ' ಎನ್ನುವರು. ವೈವಿಧ್ಯವನ್ನು ನೋಡಬೇಡಿ, ಏಕತೆಗೆ ಹೋಗಿ' ಎನ್ನುವುದು ವೇದಾಂತ. “ಏಕತೆಯನ್ನು ಆಶಿಸಬೇಡಿ, ಇದೊಂದು ಭ್ರಾಂತಿ, ವೈವಿಧ್ಯಕ್ಕೆ ಹೋಗಿ' ಎನ್ನುವುದು ಬೌದ್ಧ ಧರ್ಮ. ಧರ್ಮದಲ್ಲಿ ಮತ್ತು ತತ್ತ್ವದಲ್ಲಿ ಇದೇ ಭಿನ್ನಾಭಿಪ್ರಾಯ ಈಗಿನವರೆಗೂ ಬಂದಿದೆ. ಏಕೆಂದರೆ ಜ್ಞಾನದ ಮೊತ್ತ ಬಹಳ ಅಲ್ಪ. ತತ್ತ್ವ ಮತ್ತು ತತ್ತ್ವಜ್ಞಾನ, ಧರ್ಮ ಮತ್ತು ಧರ್ಮಜ್ಞಾನ ಐದು ಸಾವಿರ ವರುಷಗಳ ಹಿಂದೆಯೇ ತಮ್ಮ ಪರಮಾವಧಿಯನ್ನು ಮುಟ್ಟಿದ್ದವು. ನಾವು ಭಿನ್ನ ಭಿನ್ನ ಭಾಷೆಗಳಲ್ಲಿ ಒಂದೇ ಸತ್ಯವನ್ನು ಪುನಃ ಪುನಃ ಹೇಳುತ್ತಿರುವೆವು. ಕೆಲವು ವೇಳೆ ಹೊಸ ಹೊಸ ಉದಾಹರಣೆಗಳನ್ನು ಮಾತ್ರ ಕೊಟ್ಟು, ನಮ್ಮ ಜ್ಞಾನ ನಿಧಿಯನ್ನು ವೃದ್ಧಿಗೊಳಿಸಿಕೊಳ್ಳುತ್ತಿರುವೆವು. ಆದಕಾರಣವೇ ಈಗಲೂ ಕೂಡ ಇದೇ ಹೋರಾಟವಿದೆ. ಒಂದು ಪಕ್ಷದವರು ಕೇವಲ\break ದೃಶ್ಯವನ್ನು ಮಾತ್ರ, ವೈವಿಧ್ಯವನ್ನು ಮಾತ್ರ ಗಮನಿಸಬೇಕೆಂದೂ, ವಾದಬಲದಿಂದ ಈ ವೈವಿಧ್ಯ ಇರಲೇಬೇಕೆಂದೂ, ಅದು ಹೋದರೆ ಸರ್ನನಾಶವೆಂದೂ ಸಾಧಿಸುವರು. ನಾವು ಯಾವುದನ್ನು ಜೀವನ ಎನ್ನುವೆವೋ ಅದು ವೈವಿಧ್ಯದಿಂದ ಉತ್ಪನ್ನವಾಯಿತು. ಮತ್ತೊಂದು ಪಕ್ಷದವರು ಧೀರರಾಗಿ ಏಕತೆಯನ್ನು ತೋರುವರು.

ನಾವು ನೀತಿಶಾಸ್ತ್ರಕ್ಕೆ ಬಂದರೆ ಇದು ಹೊಸದಾಗಿ ಬೇರೊಂದು ಮಾರ್ಗವನ್ನು ಅನುಸರಿಸುವುದು. ಈ ಹೋರಾಟದಿಂದ ಧೈರ್ಯವಾಗಿ ಹೊರಬರುವುದು ಈ ಶಾಸ್ತ್ರ ಒಂದೇ ಎಂದು ಹೇಳಬಹುದು. ಏಕೆಂದರೆ ನೀತಿಶಾಸ್ತ್ರವೇ ಏಕತೆಗೆ ಸಂಬಂಧಪಟ್ಟದ್ದು, ಇದರ ತಳಹದಿಯೇ ಪ್ರೀತಿ, ಅದು ಈ ವೈವಿಧ್ಯವನ್ನು ಗಮನಿಸುವುದಿಲ್ಲ. ನೀತಿಶಾಸ್ತ್ರದ ಗುರಿಯೇ ಏಕತೆ, ಸಮಾನತೆ, ಇದುವರೆಗೂ ಮಾನವಕೋಟಿ ಕಂಡುಹಿಡಿದ ಶ್ರೇಷ್ಠತಮನೀತಿಯಲ್ಲಿ ವೈವಿಧ್ಯಕ್ಕೆ ಎಡೆಯಿಲ್ಲ, ಅದನ್ನು ನೋಡುವುದಕ್ಕೆ ಅಲ್ಲಿ ಸಮಯವಿಲ್ಲ, ಏಕತೆಯೇ ಅವರ ಗುರಿಯಾಗಿರುವುದು. ಭಾರತೀಯನು (ಅದರಲ್ಲಿಯೂ ವೇದಾಂತಿ) ವಿಶ್ಲೇಷಣಾತ್ಮಕ ಬುದ್ದಿಯುಳ್ಳವನಾದುದರಿಂದ ಎಲ್ಲವನ್ನೂ ವಿಶ್ಲೇಷಣೆ ಮಾಡಿ ಆದಮೇಲೆ ಈ ಏಕತೆಯನ್ನು ಕಂಡುಹಿಡಿದನು. ಎಲ್ಲವನ್ನೂ ಈ ಏಕತೆಯ ತಳಹದಿಯ ಮೇಲೆ ಕಟ್ಟಲು ಯತ್ನಿಸಿದನು. ಆದರೆ ಇಡೀ ದೇಶದಲ್ಲಿ ನಾವು ನೋಡಿದಂತೆ ಇತರರಿಗೆ (ಬೌದ್ಧರಿಗೆ) ಈ ಏಕತೆ ಎಲ್ಲಿಯೂ ಕಾಣಲಿಲ್ಲ. ಅವರಿಗೆ ಈ ಭಿನ್ನತೆಯ ರಾಶಿಯೆ ಸತ್ಯ. ಅದರಲ್ಲಿ ಒಂದಕ್ಕೂ ಮತ್ತೊಂದಕ್ಕೂ ಸಂಬಂಧವಿರಲಿಲ್ಲ.

ಮಾಕ್ಸ್ ಮುಲ್ಲರ್ ತನ್ನ ಒಂದು ಪುಸ್ತಕದಲ್ಲಿ ಹೇಳಿದ ಕಥೆಯೊಂದು ಜ್ಞಾಪಕಕ್ಕೆ ಬರುವುದು. ಇದೊಂದು ಹಳೆಯ ಗ್ರೀಕರ ಕಥೆ. ಬ್ರಾಹ್ಮಣನೊಬ್ಬ ಸಾಕ್ರಟೀಸನನ್ನು ನೋಡಲು ಹೋದ. ಆತ ಸಾಕ್ರಟೀಸನನ್ನು ಯಾವುದು ಪರಮ ಜ್ಞಾನ ಎಂದು ಕೇಳಿದನು. ಅದಕ್ಕೆ ಸಾಕ್ರಟೀಸನು ಮನುಷ್ಯನನ್ನು ಅರಿಯುವುದೇ ಎಲ್ಲಾ ಜ್ಞಾನದ ಪರಮ ಗುರಿ ಎಂದನು. ಆಗ ಬ್ರಾಹ್ಮಣ, ನೀನು ದೇವರನ್ನು ತಿಳಿಯದೆ ಮಾನವನನ್ನು ಹೇಗೆ ಅರಿಯಬಲ್ಲೆ ಎಂದು ಕೇಳಿದನು. ಗ್ರೀಕ್ ಪಕ್ಷದವರು (ಆಧುನಿಕ ಐರೋಪ್ಯರೇ ಅವರ ಪ್ರತಿನಿಧಿಗಳು) ಮನುಷ್ಯನನ್ನು ತಿಳಿದುಕೊಳ್ಳಬೇಕೆಂದು ಬಯಸುವರು; ಭಾರತೀಯರು, (ಪುರಾತನ ಧರ್ಮಗಳೆಲ್ಲ) ದೇವರನ್ನು ಅರಿಯಬೇಕೆಂದು ಹೇಳುವರು. ಒಬ್ಬರು ಪ್ರಕೃತಿಯಲ್ಲಿ ದೇವರನ್ನು ನೋಡುವರು, ಮತ್ತೊಬ್ಬರು ದೇವರಲ್ಲಿ ಪ್ರಕೃತಿಯನ್ನು ನೋಡುವರು. ಸದ್ಯಕ್ಕೆ ನಾವು ಎರಡು ಪಕ್ಷಗಳನ್ನೂ ನಿಷ್ಪಕ್ಷಪಾತದಿಂದ ನೋಡುವ ಸ್ಥಿತಿಯಲ್ಲಿರುವೆವು. ವೈವಿಧ್ಯ ಎಂಬುದೇನೋ ಇದೆ, ಜೀವನ ಇರಬೇಕಾದರೆ ವೈವಿಧ್ಯ ಇದ್ದೇ ತೀರಬೇಕು. ಈ ವೈವಿಧ್ಯದಲ್ಲಿ ಮತ್ತು ಇದರ ಮೂಲಕ ಮಾತ್ರ ನಾವು ಏಕತೆಯನ್ನು ನೋಡಬೇಕಾಗಿದೆ. ದೇವರಲ್ಲಿ ಪ್ರಕೃತಿಯನ್ನು ನೋಡುವುದು ಕೂಡ ಸತ್ಯವೆ. ಮಾನವನನ್ನು ಅರಿಯುವುದೇ ಶ್ರೇಷ್ಠ ಜ್ಞಾನ, ಮಾನವನನ್ನು ಅರಿತರೆ ಮಾತ್ರ ದೇವರನ್ನು ಅರಿಯಬಹುದು ಎಂಬುದು ಸತ್ಯವೆ. ದೇವರ ಜ್ಞಾನವೇ ಶ್ರೇಷ್ಠ ಜ್ಞಾನ, ದೇವರನ್ನು ಅರಿತರೆ ಮಾತ್ರ ಮಾನವನನ್ನು ಅರಿಯಬಹುದು ಎಂಬುದೂ ಸತ್ಯವೆ. ಇವೆರಡೂ ಪರಸ್ಪರ ವಿರೋಧಾಭಾಸದಂತೆ ಕಂಡರೂ ಇವು ಮಾನವ ಸ್ವಭಾವಕ್ಕೆ ಆವಶ್ಯಕ. ಇಡೀ ಬ್ರಹ್ಮಾಂಡವೇ, ವೈವಿಧ್ಯದಲ್ಲಿ ಏಕತೆಯ ಮತ್ತು ಏಕತೆಯಲ್ಲಿ ವೈವಿಧ್ಯದ ಒಂದು ಲೀಲಾಭೂಮಿಯಾಗಿದೆ. ಇಡೀ ಬ್ರಹ್ಮಾಂಡ ವ್ಯತ್ಯಾಸದ ಮತ್ತು ಸಮತ್ವದ ಒಂದು ಲೀಲಾಭೂಮಿಯಾಗಿದೆ: ಸಾಂತ ಮತ್ತು ಅನಂತ ಇವುಗಳ ಒಂದು ಕ್ರೀಡಾಭೂಮಿಯಾಗಿದೆ. ಒಂದನ್ನು ಒಪ್ಪಿಕೊಳ್ಳದೆ ಮತ್ತೊಂದನ್ನು ಸ್ವೀಕರಿಸುವಂತಿಲ್ಲ. ಆದರೆ ಇವೆರಡನ್ನೂ ಒಂದೇ ನೋಟದ ಸತ್ಯಾಂಶಗಳೆಂದಾಗಲಿ, ಒಂದೇ ಅನುಭವದ ಸತ್ಯಾಂಶಗಳೆಂದಾಗಲಿ ಒಪ್ಪಿಕೊಳ್ಳುವಂತೆ ಇಲ್ಲ. ಆದರೂ ಇದು ಯಾವಾಗಲೂ ಹೀಗೆಯೆ ಇರುವುದು.

ನಾವೀಗ ಸ್ಪಷ್ಟವಾಗಿ ಧರ್ಮದ ವಿಷಯಕ್ಕೆ ಬರೋಣ. ನೀತಿಯಲ್ಲಿ ವೈವಿಧ್ಯವೆಲ್ಲಾ ಮಾಯವಾಗಿ ಮೃತ್ಯುಸಮವಾದ ಸಮತ್ವ ಇದೆ. ಇಂತಹ ಸ್ಥಿತಿ ಜೀವನ ಇರುವವರೆಗೆ ಸಾಧ್ಯವಿಲ್ಲ. ಇಂತಹ ಸ್ಥಿತಿಯನ್ನು ಇಚ್ಚಿಸುವುದೂ ಸೂಕ್ತವಲ್ಲ. ಅದೇ ಸಂದರ್ಭದಲ್ಲಿ ಏಕತೆ ಆಗಲೇ ಇದೆ ಎಂಬ ಸತ್ಯಾಂಶವನ್ನೂ ನಾವು ಒಪ್ಪಿಕೊಳ್ಳಬೇಕಾಗಿದೆ. ಈ ಏಕತೆಯನ್ನು ನಾವು ಸೃಷ್ಟಿಸಬೇಕಾಗಿಲ್ಲ, ಅದು ಆಗಲೇ ಇದೆ. ಈ ಏಕತೆಯಿಲ್ಲದೆ ನೀವು ವೈವಿಧ್ಯವನ್ನು ಕಾಣಲೇ ಆರಿರಿ. ದೇವರನ್ನು ನಾವು ಸೃಷ್ಟಿಸುವುದಿಲ್ಲ; ಆತನಾಗಲೇ ಅಲ್ಲಿ ಇರುವನು. ಎಲ್ಲಾ ಧರ್ಮಗಳೂ ಸಾರುವುದೇ ಇದನ್ನು. ಎಲ್ಲಿ ಒಬ್ಬ ಸಾಂತವನ್ನು ನೋಡಿರುವನೋ ಅಲ್ಲಿ ಅವನು ಅನಂತವನ್ನೂ ನೋಡಿರುವನು. ಕೆಲವರು ಸಾಂತಕ್ಕೆ ಹೆಚ್ಚು ಪ್ರಾಮುಖ್ಯತೆ ಕೊಟ್ಟರು, ಹೊರಗೆ ಇರುವ ಸಾಂತವನ್ನೇ ಪರಿಗಣಿಸಬೇಕು ಎಂದರು. ಮತ್ತೆ ಕೆಲವರು ಅನಂತಕ್ಕೆ ಹೆಚ್ಚು ಪ್ರಾಮುಖ್ಯತೆ ಕೊಟ್ಟರು, ಅದನ್ನು ಮಾತ್ರ ನೋಡಬೇಕು ಎಂದರು. ನ್ಯಾಯವಾಗಿ ಒಂದಿಲ್ಲದೆ ಮತ್ತೊಂದನ್ನು ನೋಡಲಾರೆವು ಎನ್ನುವುದನ್ನು ಒಪ್ಪಿಕೊಳ್ಳಬೇಕಾಗಿದೆ. ಆದಕಾರಣ ಈ ಸಮಾನತೆ, ಏಕತೆ, ಪೂರ್ಣತೆ ಇವನ್ನು ನಾವು ಹೊಸದಾಗಿ ಸೃಷ್ಟಿಸುವುದಿಲ್ಲ, ಅವು ಆಗಲೆ ಅಲ್ಲಿ ಇರುವುವು. ನಾವು ಅವನ್ನು ಒಪ್ಪಿಕೊಳ್ಳಬೇಕಾಗಿದೆ. ಅವು ನಮಗೆ ಗೊತ್ತಿರಲಿ, ನಾವು ಅವನ್ನು ಸ್ಪಷ್ಟ ಭಾಷೆಯಲ್ಲಿ ವ್ಯಕ್ತಪಡಿಸಲಿ ಬಿಡಲಿ, ಅವು ನಮಗೆ ಇಂದ್ರಿಯ ಗ್ರಾಹ್ಯವಸ್ತುವಿನಂತೆ ಸ್ಪಷ್ಟವಾಗಿ ಅತ್ಯಗತ್ಯವಾಗಿ ಕಾಣಲಿ ಬಿಡಲಿ, ಅವು ಅಲ್ಲಿವೆ. ಅವು ಅಲ್ಲಿವೆ ಎಂದು ನಾವು ನ್ಯಾಯವಾಗಿ ಒಪ್ಪಿಕೊಳ್ಳದೆ ವಿಧಿಯಿಲ್ಲ. ಇಲ್ಲದೇ ಇದ್ದರೆ ಸಾಂತ ಪ್ರಪಂಚವನ್ನು ನಾವು ನೋಡಲಾರೆವು. ದ್ರವ್ಯ ಮತ್ತು ಗುಣಗಳಿಗೆ ಸಂಬಂಧಿಸಿದ ಹಳೆಯ ಸಿದ್ದಾಂತವನ್ನು ಹೇಳುತ್ತಿಲ್ಲ, ಏಕತೆಯ ದೃಷ್ಟಿಯಿಂದ ಹೇಳುತ್ತಿರುವೆನು. ಇಷ್ಟೊಂದು ದೃಶ್ಯವೈವಿಧ್ಯ ಗಳಿದ್ದರೂ, ನಾನು ನೀವು ಬೇರೆ ಎಂಬ ಭಾವನೆಯೆ ನಾನು ನೀವು ಬೇರೆ ಅಲ್ಲ ಎಂಬ ಭಾವನೆಯನ್ನೂ ತರುವುದು. ಇಂತಹ ಏಕತೆ ಇಲ್ಲದೆ ಇದ್ದರೆ ಜ್ಞಾನವೇ ಸಾಧ್ಯವಿರುತ್ತಿರಲಿಲ್ಲ. ಸಮಾನತೆಯ ಭಾವನೆ ಇಲ್ಲದೆ ಇಂದ್ರಿಯ ಗ್ರಹಣವೂ ಇಲ್ಲ, ಜ್ಞಾನವೂ ಇಲ್ಲ. ಆದಕಾರಣ ಎರಡೂ ಒಟ್ಟಿಗೆ ಇರುವುವು.

ಆದಕಾರಣ ನಿರಪೇಕ್ಷ ಸಮಾನತೆ \enginline{(absolute sameness)} - ಇದೇ ನೀತಿಶಾಸ್ತ್ರದ ಗುರಿಯಾದರೆ, ಅದು ಸಾಧ್ಯವೇ ಇಲ್ಲ. ನಾವು ಎಷ್ಟೇ ಪ್ರಯತ್ನಪಟ್ಟರೂ ಎಲ್ಲರೂ ಒಂದೇ ಸಮನಾಗಿರುವುದಿಲ್ಲ. ಮನುಷ್ಯರಲ್ಲಿ ಹುಟ್ಟುವಾಗಲೇ ವ್ಯತ್ಯಾಸವಿರುವುದು. ಕೆಲವರಿಗೆ ಇತರರಿಗಿಂತ ಹೆಚ್ಚು ಶಕ್ತಿ ಇರುವುದು, ಇನ್ನು ಕೆಲವರಿಗೆ ಇತರರಿಗಿಂತ ಕೆಲವು ಕೆಲಸಗಳನ್ನು ಮಾಡುವುದು ಸುಲಭ. ಕೆಲವರಿಗೆ ಚೆನ್ನಾಗಿರುವ ದೇಹವಿರುವುದು; ಮತ್ತೆ ಕೆಲವರಿಗೆ ಇಲ್ಲ. ನಾವು ಇದನ್ನು ತಪ್ಪಿಸುವುದಕ್ಕೆ ಆಗುವುದಿಲ್ಲ. ಆದರೆ ಅದೇ ಕಾಲದಲ್ಲಿ ಹಲವು ಮಹಾಗುರುಗಳು ಹೇಳಿರುವ ಈ ಭಾವನೆ ಕೇಳಿಸುತ್ತಿರುತ್ತದೆ: “ಎಲ್ಲರಲ್ಲಿಯೂ ಒಂದೇ ದೇವರು ಸಮನಾಗಿರುವುದನ್ನು ಕಂಡು ಜ್ಞಾನಿ ಆತ್ಮನಿಂದ ಆತ್ಮನಿಗೆ ಹಾನಿ ಮಾಡುವುದಿಲ್ಲ. ಅವನು ಪರಮಗತಿಯನ್ನು ಸೇರುತ್ತಾನೆ. ಸಮತ್ವದಲ್ಲಿ ನೆಲಸಿರುವವರು ಈ ಜೀವನದಲ್ಲಿ ಸಂಸಾರದಿಂದ ಪಾರಾಗುವರು. ಏಕೆಂದರೆ ದೇವರು ಪರಿಶುದ್ಧ ಮತ್ತು ಎಲ್ಲರಿಗೂ ಸಮನಾಗಿರುವನು. ಇಂತಹವರು ದೇವರಲ್ಲಿ ನೆಲೆಸಿರುವರು." ಇದು ನಿಜ ಎಂಬುದನ್ನು ಅಲ್ಲಗಳೆಯಲಾಗುವುದಿಲ್ಲ. ಆದರೂ ಬಾಹ್ಯ ಆಕಾರದಲ್ಲಿ ಮತ್ತು ಅಂತಸ್ತಿನಲ್ಲಿ ಸಮಾನತೆ ಇರಲಾರದು ಎಂಬ ತೊಡಕು ಬರುವುದು.

ಆದರೆ ನಾವು ಹಕ್ಕು ಬಾಧ್ಯತೆಗಳನ್ನು ನಿರ್ಮೂಲ ಮಾಡಬಹುದು. ಪ್ರಪಂಚ ನಿಜವಾಗಿ ಮಾಡಬೇಕಾದ ಕೆಲಸ ಇದು. ಎಲ್ಲಾ ಸಮಾಜಗಳಲ್ಲಿ ಎಲ್ಲಾ ಜನಾಂಗಗಳಲ್ಲಿ ಮತ್ತು ದೇಶಗಳಲ್ಲಿ ಒಂದೇ ಹೋರಾಟವಿದೆ. ಒಬ್ಬರು ಮತ್ತೊಬ್ಬರಿಗಿಂತ ಹೆಚ್ಚು ವಿದ್ಯಾವಂತರಾಗಿರುವಾಗ, ಒಬ್ಬರಿಗೆ ಜಾಸ್ತಿ ವಿದ್ಯೆ ಇದೆ ಎಂದು ವಿದ್ಯೆ ಇಲ್ಲದವರ ಸುಖವನ್ನು ಅಪಹರಿಸಬೇಕೆ? ಈ ಹಕ್ಕನ್ನು ನಿರ್ಮೂಲ ಮಾಡುವುದೇ ಹೋರಾಟದ ಗುರಿಯಾಗಿದೆ. ಕೆಲವರು ಮತ್ತೆ ಕೆಲವರಿಗಿಂತ ಬಲಿಷ್ಠರಾಗಿರುವರು. ಅವರು ದುರ್ಬಲರಾಗಿರುವವರನ್ನು ಸೋಲಿಸುವುದು ಅಥವಾ ತಮ್ಮ ಆಳ್ವಿಕೆಯಲ್ಲಿ ಇಟ್ಟು ಕೊಳ್ಳುವುದು ಸ್ವತಃ ವೇದ್ಯವಾಗಿಯೇ ಇದೆ. ಆದರೆ ಅವರಿಗೆ ಹೆಚ್ಚು ಶಕ್ತಿ ಇದೆ ಎಂದು ಎಲ್ಲಾ ಸೌಖ್ಯವನ್ನೂ ತಾವೇ ದೋಚುವುದು ಧರ್ಮಕ್ಕೆ ವಿರೋಧ. ಇದಕ್ಕಾಗಿ ಹೊಡೆದಾಟ ನಡೆಯುತ್ತಿರುವುದು. ಕೆಲವರು ಸ್ವಾಭಾವಿಕವಾಗಿ ತಮ್ಮಲ್ಲಿರುವ ಬುದ್ದಿಗೆ ಅನುಸಾರವಾಗಿ ಇತರರಿಗಿಂತ ಹೆಚ್ಚು ಹಣವನ್ನು ಸಂಪಾದಿಸುವುದು ಸಹಜವಾಗಿಯೇ ಇದೆ. ಆದರೆ ತಮಗೆ ಐಶ್ವರ್ಯವನ್ನು ಪಡೆಯಲು ಸಾಧ್ಯ ಎಂದು ಯಾರಿಗೆ ಅದು ಸಾಧ್ಯವಿಲ್ಲವೋ ಅವರ ಮೇಲೆ ದಬ್ಬಾಳಿಕೆ ನಡೆಸುವುದು ಧರ್ಮಕ್ಕೆ ವಿರೋಧ. ಹೋರಾಟ ನಡೆಯುತ್ತಿರುವುದು ಇದಕ್ಕಾಗಿ. ಒಬ್ಬನು ಮತ್ತೊಬ್ಬನಿಗಿಂತ ಹೆಚ್ಚು ಸುಖವನ್ನು ಅನುಭವಿಸುವ ಅಧಿಕಾರವನ್ನು ಹಕ್ಕು ಎನ್ನುತ್ತಾರೆ. ನೀತಿ ಯಾವಾಗಲೂ ಇದನ್ನು ಧ್ವಂಸಮಾಡಲು ಯತ್ನಿಸುವುದು, ವೈವಿಧ್ಯವನ್ನು ನಾಶಮಾಡದೆ ಸಮಾನತೆಗೆ, ಏಕತೆಗೆ. ಪ್ರಯತ್ನಿಸುವುದು ಎಂದರೆ ಇದೇ.

ಈ ವೈವಿಧ್ಯ ಎಂದೆಂದಿಗೂ ಇರಲಿ, ಇದೇ ಜೀವನದ ಸಾರ. ಹೀಗೆಯೇ ನಾವು ಕೊನೆಯ ತನಕ ಆಟವಾಡೋಣ. ನೀವು ಶ‍್ರೀಮಂತವಾಗಿರುವಿರಿ, ನಾನು ದರಿದ್ರನಾಗಿರುವೆ. ನೀವು ಬಲಾಡ್ಯರಾಗಿರುವಿರಿ, ನಾನು ದುರ್ಬಲನಾಗಿರುವೆ. ಆದರೆ ಏನು? ನೀವು ನನಗಿಂತ ದೈಹಿಕವಾಗಿ ಅಥವಾ ಮಾನಸಿಕವಾಗಿ ಮೇಲೆಂದು ನನಗಿಂತ ನಿಮಗೆ ಹೆಚ್ಚು ಹಕ್ಕುಗಳೇತಕ್ಕೆ ಇರಬೇಕು? ನಿಮ್ಮ ಹತ್ತಿರ ಹೆಚ್ಚು ಹಣ ಇದೆ ಎಂದರೆ ನನಗಿಂತ ನೀವು ಮೇಲು ಎಂದು ಏತಕ್ಕೆ ಪರಿಗಣಿಸಬೇಕು? ವೈವಿಧ್ಯ ಇದ್ದರೂ ಸಮಾನತೆ ಎಲ್ಲರಲ್ಲಿಯೂ ಇದೆ.

ಭವಿಷ್ಯದಲ್ಲಿ ನೀತಿಯ ಕಾರ್ಯವು ವೈವಿಧ್ಯವನ್ನು ನಾಶಮಾಡಿ ಸಮಾನತೆಯನ್ನು ಸ್ಥಾಪಿಸುವುದಲ್ಲ. ಇದು ಸರ್ವನಾಶಕ್ಕೆ ಕಾರಣ. ಈ ವೈವಿಧ್ಯಗಳಿದ್ದರೂ ಏಕತೆಯನ್ನು\break ಒಪ್ಪಿಕೊಳ್ಳಬೇಕು; ಎಲ್ಲರಲ್ಲಿಯೂ ಒಂದೇ ದೇವರನ್ನು ಕಾಣಬೇಕಾಗಿದೆ. ಅದಕ್ಕೆ ವಿರೋಧವಾಗಿ ಏನಿದ್ದರೂ, ಎಲ್ಲರಲ್ಲಿಯೂ ಎಷ್ಟೇ ದುರ್ಬಲತೆ ಇದ್ದರೂ, ಅನಂತ ಶಕ್ತಿ ಸರ್ವರಿಗೂ ಸಾಮಾನ್ಯ ನಿಧಿಯಾಗಿದೆ ಎಂಬುದನ್ನು ಒಪ್ಪಿಕೊಳ್ಳಬೇಕಾಗಿದೆ. ಆತ್ಮ ಅನಂತವಾದುದು, ಸನಾತನವಾದುದು, ಪರಿಶುದ್ಧವಾದುದು ಎಂಬುದನ್ನು - ಅದಕ್ಕೆ ವಿರೋಧವಾಗಿ ಏನೇ ಹೊರಗೆ ಕಂಡರೂ - ಒಪ್ಪಿಕೊಳ್ಳಬೇಕಾಗಿದೆ. ನಾವು ಒಂದು ಪಕ್ಷವನ್ನು ಮಾತ್ರ ತೆಗೆದುಕೊಂಡರೆ, ಅರ್ಧವನ್ನು ಮಾತ್ರ ತೆಗೆದುಕೊಂಡರೆ, ಅಪಾಯಕರ; ಇದು ಮನಸ್ತಾಪಕ್ಕೆ ಕಾರಣವಾಗುವುದು. ನಾವು ಎಲ್ಲವನ್ನೂ ಒಟ್ಟಿಗೆ ತೆಗೆದುಕೊಳ್ಳಬೇಕು, ಅದರ ಆಧಾರದ ಮೇಲೆ ನಿಂತು, ವೈಯಕ್ತಿಕವಾಗಿಯೂ, ಸಮಾಜದ ಅವಿಭಾಜ್ಯ ಅಂಗಗಳಾಗಿಯೂ, ನಮ್ಮ ಜೀವನದ ಪ್ರತಿಯೊಂದು ಆಗುಹೋಗುಗಳಲ್ಲಿಯೂ ಇದನ್ನು ಕಾರ್ಯರೂಪಕ್ಕೆ ತರಬೇಕಾಗಿದೆ.

