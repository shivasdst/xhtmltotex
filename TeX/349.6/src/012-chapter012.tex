
\chapter[ಸೃಷ್ಟಿ ತತ್ತ್ವ]{ಸೃಷ್ಟಿ ತತ್ತ್ವ\protect\footnote{\engfoot{C.W, Vol. II, P. 432}}}

ಬ್ರಹ್ಮಾಂಡ ಮತ್ತು ಪಿಂಡಾಂಡ, ಬಾಹ್ಯ ಜಗತ್ತು ಮತ್ತು ಆಂತರಿಕ ಜಗತ್ತು ಎಂಬ ಎರಡು ಜಗತ್ತುಗಳಿವೆ. ಅನುಭವದಿಂದ ಈ ಎರಡು ಕ್ಷೇತ್ರಗಳಿಂದಲೂ ನಮಗೆ ಸತ್ಯ ದೊರಕುವುದು. ಆಂತರಿಕ ಅನುಭವದಿಂದ ಸಂಗ್ರಹಿಸಿರುವುದೆ ಮನಶ್ಶಾಸ್ತ್ರ, ತತ್ತ್ವಶಾಸ್ತ್ರ ಮತ್ತು ಧರ್ಮ. ಬಾಹ್ಯ ಜಗತ್ತಿನಿಂದ ನಮಗೆ ಭೌತವಿಜ್ಞಾನಗಳು ದೊರಕುವುವು. ಪೂರ್ಣವಾದ ಸತ್ಯವು ಈ ಎರಡು ಕ್ಷೇತ್ರಗಳಿಂದ ಬರುವ ಅನುಭವಗಳಿಗೆ ವಿರೋಧವಿಲ್ಲದೆ ಇರಬೇಕು. ಪಿಂಡಾಂಡವು ಬ್ರಹ್ಮಾಂಡಕ್ಕೆ ಪ್ರಮಾಣವಾಗಿರಬೇಕು. ಬ್ರಹ್ಮಾಂಡವು ಪಿಂಡಾಂಡಕ್ಕೆ ಪ್ರಮಾಣವಾಗಿರಬೇಕು. ಭೌತಿಕ ಸತ್ಯಗಳಿಗೆ ಸರಿಸಮನಾಗಿರುವುದು\break ಆಂತರಿಕ ಪ್ರಪಂಚದಲ್ಲಿರಬೇಕು. ಆಂತರಿಕ ಜಗತ್ತಿನಲ್ಲಿರುವ ಸತ್ಯವನ್ನು ಹೋಲಿಸುವುದಕ್ಕೆ ಬಾಹ್ಯ ಜಗತ್ತಿನಲ್ಲಿ ನಮಗೆ ವಸ್ತು ದೊರಕಬೇಕು. ಆದರೂ ಅನೇಕವೇಳೆ ಇವುಗಳಲ್ಲಿ ಪರಸ್ಪರ ವಿರೋಧವನ್ನು ನೋಡುವೆವು. ಇತಿಹಾಸದ ಒಂದು ಅವಧಿಯಲ್ಲಿ ಆಂತರಿಕ ಸತ್ಯಗಳು ಬಾಹ್ಯದೊಂದಿಗೆ ಪ್ರಬಲವಾಗಿ ಹೋರಾಡುವುವು. ಈಗ ಬಾಹ್ಯ ಭೌತಶಾಸ್ತ್ರಜ್ಞರು ಪ್ರಬಲರಾಗಿದ್ದಾರೆ. ಅವರು ಮನಶ್ಶಾಸ್ತ್ರ ಮತ್ತು ತತ್ವಶಾಸ್ತ್ರದ ಸಿದ್ಧಾಂತಗಳನ್ನು ನಿಕೃಷ್ಟ ದೃಷ್ಟಿಯಿಂದ ನೋಡುವರು. ನನ್ನ ಅನುಭವದ ದೃಷ್ಟಿಯಿಂದ ಮನಶ್ಶಾಸ್ತ್ರದ ಮುಖ್ಯ ಸಿದ್ದಾಂತಗಳಿಗೂ ಭೌತಶಾಸ್ತ್ರದ ಮುಖ್ಯ ಸಿದ್ಧಾಂತಗಳಿಗೂ ಯಾವ ವಿರೋಧವೂ ಇರುವಂತೆ ತೋರುವುದಿಲ್ಲ. ಪ್ರತಿಯೊಬ್ಬನೂ ಎರಡು ಕಾವ್ಯಕ್ಷೇತ್ರಗಳಲ್ಲಿಯೂ ಪ್ರಮುಖನಾಗಲಾರ. ಯಾವ ಒಂದು ಜನಾಂಗವೇ ಆಗಲಿ, ರಾಷ್ಟ್ರವೇ ಆಗಲಿ ಎಲ್ಲಾ ಕ್ಷೇತ್ರದಲ್ಲಿಯೂ ಬಲವಾಗಿರಲು ಅಸಾಧ್ಯ. ಆಧುನಿಕ ಐರೋಪ್ಯರಾಷ್ಟ್ರಗಳು ಬಾಹ್ಯ ಸಂಶೋಧನಾ ಕ್ಷೇತ್ರದಲ್ಲಿ ಬಲಾಢ್ಯವಾಗಿವೆ. ಆದರೆ ಮಾನವನ ಅಂತರಂಗದ ಸ್ವಭಾವಗಳ ಸಂಶೋಧನಾ ಕ್ಷೇತ್ರದಲ್ಲಿ ಅವು ಅಷ್ಟು ಬಲಾಢ್ಯವಾಗಿಲ್ಲ. ಪ್ರಾಚ್ಯ ದೇಶಗಳು ಬಾಹ್ಯ ಜಗತ್ತಿನ ಸಂಶೋಧನೆಯಲ್ಲಿ ಅಷ್ಟು ಬಲಾಢ್ಯವಾಗಿಲ್ಲ. ಆದರೆ ಆಂತರಿಕ ಜಗತ್ತಿನ ಸಂಶೋಧನೆಯಲ್ಲಿ ಪ್ರಬಲವಾಗಿವೆ. ಆದಕಾರಣವೆ ಪ್ರಾಚ್ಯ ಭೌತವಿಜ್ಞಾನ ಪಾಶ್ಚಾತ್ಯ ವಿಜ್ಞಾನಗಳಿಗೆ ಸರಿಸಮನಾಗಿಲ್ಲ. ಪಾಶ್ಚಾತ್ಯ ಮನಶ್ಶಾಸ್ತ್ರವೂ ಪ್ರಾಚ್ಯರ ಮನಶ್ಶಾಸ್ತ್ರಕ್ಕೆ ಹೊಂದಿಕೊಳ್ಳುವುದಿಲ್ಲ. ಪ್ರಾಚ್ಯ ಭೌತವಿಜ್ಞಾನಿಗಳು ಪಾಶ್ಚಾತ್ಯ ಭೌತವಿಜ್ಞಾನಿಗಳಿಂದ ಸೋಲಿಸಲ್ಪಟ್ಟಿರುವರು. ಆದರೂ ಪ್ರತಿಯೊಬ್ಬರೂ ತಾವು ಹೇಳುವುದು ಸತ್ಯವೆನ್ನುವರು. ನಾನು ಮುಂಚೆಯೆ ಹೇಳಿದಂತೆ ಸತ್ಯವು ಯಾವ ಕ್ಷೇತ್ರಕ್ಕೆ ಸೇರಿದ್ದರೂ ಪರಸ್ಪರ ವಿರೋಧಗಳಿರುವುದಿಲ್ಲ. ಆಂತರಿಕ ಸತ್ಯಕ್ಕೂ ಬಾಹ್ಯ ಸತ್ಯಕ್ಕೂ ಪರಸ್ಪರ ವಿರೋಧವಿಲ್ಲ.

ಆಧುನಿಕ ಖಗೋಳಶಾಸ್ತ್ರ ಮತ್ತು ಭೌತಶಾಸ್ತ್ರಗಳ ದೃಷ್ಟಿಯಿಂದ ಈ ವಿಶ್ವವು ಹೇಗೆ ಸೃಷ್ಟಿಯಾಗಿದೆ ಎಂಬುದು ನಮಗೆ ಗೊತ್ತಿದೆ. ಜೊತೆಗೆ ಅವು ಯುರೋಪಿನ ಧರ್ಮಕ್ಕೆ ಕುಠಾರಪ್ರಾಯವಾಗಿದೆ ಎಂಬುದೂ ಗೊತ್ತು. ಆಧುನಿಕ ವೈಜ್ಞಾನಿಕ ಸಂಶೋಧನೆಗಳು ಧರ್ಮಗಿರಿದುರ್ಗದ ಮೇಲೆ ಎಸೆದ ಬಾಂಬಿನಂತೆ ಇವೆ. ಧಾರ್ಮಿಕರು ವಿಜ್ಞಾನದ ಆವಿಷ್ಕಾರಗಳಿಗೆ ಯಾವಾಗಲೂ ಆತಂಕಪ್ರಾಯರಾಗಿದ್ದರು ಎಂಬುದು ನಮಗೆ ಗೊತ್ತು.

ಸೃಷ್ಟಿಯನ್ನು ಕುರಿತಂತೆ ಪ್ರಾಚ್ಯರು ಮನಶ್ಶಾಸ್ತ್ರೀಯವಾಗಿ ಏನನ್ನು ಆಲೋಚಿಸಿದ್ದರು ಎಂಬುದನ್ನು ನಾನೀಗ ಹೇಳುತ್ತೇನೆ. ಅವರ ಸಿದ್ಧಾಂತಗಳಿಗೂ ಅತ್ಯಂತ ಆಧುನಿಕ ವೈಜ್ಞಾನಿಕ ನವ ನವಾನ್ವೇಷಣೆಗಳಿಗೂ ಯಾವ ಘರ್ಷಣೆಯೂ ಇಲ್ಲವೆಂಬುದನ್ನು ತೋರುವೆನು. ಎಲ್ಲಿ ಹೊಂದಾಣಿಕೆ ಇಲ್ಲವೋ ಅಲ್ಲಿ ಕೊರತೆಯಿರುವುದು ಆಧುನಿಕ ವಿಜ್ಞಾನದಲ್ಲೇ ಹೊರತು ಅವರ ವಾದಗಳಲ್ಲಲ್ಲ ಎಂಬುದು ನಿಮಗೆ ತಿಳಿದುಬರುತ್ತದೆ. ನಾವೆಲ್ಲ ಪ್ರಕೃತಿ ಎಂಬ ಪದವನ್ನು ಬಳಸುತ್ತೇವೆ. ಹಳೆಯ ಸಾಂಖ್ಯಶಾಸ್ತ್ರಜ್ಞರು ಇದಕ್ಕೆ ಎರಡು ಪದವನ್ನು ಉಪಯೋಗಿಸಿರುವರು. ಒಂದು ಪ್ರಕೃತಿ, ನಿಮ್ಮ \enginline{Nature} ಎನ್ನುವ ಪದದಂತೆ, ಮತ್ತೊಂದು ಹೆಚ್ಚು ಸ್ಪಷ್ಟವಾದ ಪದವೇ ಅವ್ಯಕ್ತ. ಅಲ್ಲಿಂದಲೆ ಕಣ, ಶಕ್ತಿ, ಮನಸ್ಸು ಆಲೋಚನೆ ಬುದ್ದಿ ಎಲ್ಲವೂ ಬರುವುದು. ಭರತಖಂಡದ ದಾರ್ಶನಿಕರು ಹಲವು ಶತಮಾನಗಳ ಹಿಂದೆಯೆ ಮನಸ್ಸನ್ನು ಭೌತಿಕ ವಸ್ತುವೆಂದು ಸಾರಿರುವುದು ಆಶ್ಚರ್ಯದ ಸಂಗತಿಯೇ ಸರಿ. ಆಧುನಿಕ ಜಡವಾದಿಗಳು ಮನಸ್ಸು ನಮ್ಮ ದೇಹದಂತೆಯೇ ಪ್ರಕೃತಿಜನ್ಯವೆಂಬುದನ್ನು ತೋರುವುದಲ್ಲದೆ ಮತ್ತೇನು ಮಾಡುತ್ತಿರುವರು? ಇದರಂತೆಯೇ ಆಲೋಚನೆ ಕೂಡ. ಕ್ರಮೇಣ ಬುದ್ದಿಯೂ ಕೂಡ ಇದೆ ಎಂದು ಗೊತ್ತಾಗುವುದು. ಇವೆಲ್ಲ ಅವ್ಯಕ್ತದಿಂದ ಜನಿಸಿದವು. ಸಾಂಖ್ಯರು ಇದನ್ನು ಸತ್ಯ ರಜಸ್ಸು, ತಮಸ್ಸು ಎಂಬ ತ್ರಿಗುಣಗಳ ಸಮತ್ವವೆನ್ನುವರು. ಅತಿ ಕೆಳಗಿರುವುದೇ ತಮಸ್ಸು, ಅದೇ ಆಕರ್ಷಿಸುವುದು. ಅದಕ್ಕೆ ಮೇಲಿರುವುದೆ ರಜಸ್ಸು, ಅದು ದೂರ ತಳ್ಳುವುದು. ಶ್ರೇಷ್ಠವಾದುದೇ ಇವೆರಡರ ಸಮತ್ವ, ಅದೇ ಸತ್ಯ. ಆಕರ್ಷಿಸುವುದು ಮತ್ತು ದೂರ ಸರಿಯುವುದು ಈ ಎರಡು ಶಕ್ತಿಗಳನ್ನು ಎಂದು ಸತ್ಯವು ಸಮಸ್ಥಿತಿಯಲ್ಲಿ ಇಟ್ಟುಕೊಳ್ಳುವುದೋ ಆಗ ಸೃಷ್ಟಿಯಿಲ್ಲ, ಜಗತ್ತಿನಲ್ಲಿ ಯಾವ ಚಲನೆಯೂ ಇರುವುದಿಲ್ಲ. ಸಮತ್ವಕ್ಕೆ ಭಂಗ ಬಂದಾಗ, ಒಂದು ಶಕ್ತಿ ಮತ್ತೊಂದಕ್ಕಿಂತ ಮೇಲಾಗಿ, ಚಲನೆ ಪ್ರಾರಂಭವಾಗಿ ಸೃಷ್ಟಿಯಾಗುವುದು. ಈ ಘಟನೆ ಒಂದು ವೃತ್ತದಂತೆ ಪುನಃ ಪುನಃ ಆಗುತ್ತಿರುವುದು. ಒಮ್ಮೆ ಸಮತ್ವ ಭಂಗವಾಗುವುದು, ಶಕ್ತಿಗಳ ಸಂಯೋಗದಿಂದ ಹೊಸ ಸೃಷ್ಟಿಯಾಗುವುದು. ಪ್ರತಿಯೊಂದು ವಸ್ತುವಿಗೂ ಹಿಂದಿನ ಸಮತ್ವ ಸ್ಥಿತಿಗೆ ಹಿಂತಿರುಗಿ ಹೋಗುವ ಒಂದು ಸ್ವಭಾವವಿರುವುದು. ವ್ಯಕ್ತವಾಗಿರುವುದೆಲ್ಲ ನಾಶವಾಗುವ ಒಂದು ಸಮಯ ಬಂದೇ ಬರುವುದು. ಪುನಃ ಸಮತ್ವಕ್ಕೆ ಭಂಗ ಬರುವುದು, ಸೃಷ್ಟಿ ವ್ಯಕ್ತವಾಗುವುದು. ಹೀಗೆ ಅಲೆಯಂತೆ ಬಿದ್ದು ಏಳುತ್ತಿರುವುದು. ಚಲನೆಯಲ್ಲ, ಪ್ರಪಂಚದಲ್ಲಿರುವುದೆಲ್ಲ ಒಂದು ಅಲೆಯಂತಿದೆ, ಪುನಃಪುನಃ ಎದ್ದು ಬೀಳುತ್ತಿದೆ. ಸೃಷ್ಟಿಯೆಲ್ಲ ಕೆಲವು ಕಾಲ ಅವ್ಯಕ್ತದಲ್ಲಿರುವುದೆಂದು ಕೆಲವು ದಾರ್ಶನಿಕರು ಹೇಳುವರು. ಮತ್ತೆ ಕೆಲವರು, ಈ ಅವ್ಯಕ್ತಾವಸ್ಥೆಗೆ ಹೋಗುವುದು ಸೃಷ್ಟಿಯಲ್ಲಿರುವ ಒಂದೊಂದು ವ್ಯೂಹಕ್ಕೆ ಮಾತ್ರ ಅನ್ವಯಿಸುವುದೆಂದು ಹೇಳುವರು. ನಮ್ಮ ಸೌರವ್ಯೂಹ ಪ್ರಳಯವಾಗಿ ಅವ್ಯಕ್ತಸ್ಥಿತಿಗೆ ತೆರಳಿದರೆ, ಇತರ ಅನಂತ ವ್ಯೂಹಗಳು ವ್ಯಕ್ತವಾಗಿ ಸೃಷ್ಟಿಯಾಗುವುವು ಎಂದು ಹೇಳುವರು. ಎಲ್ಲ ಏಕಕಾಲದಲ್ಲಿ ಅವ್ಯಕ್ತಕ್ಕೆ ಹಿಂತಿರುಗಿ ಹೋಗುವುದಿಲ್ಲ. ಬೇರೆ ಬೇರೆ ಭಾಗದಲ್ಲಿ ಮಾತ್ರ ಹೀಗಾಗುತ್ತಿದೆ ಎಂಬ ಎರಡನೆಯ ಭಾವನೆಯನ್ನು ನಾನು ಅನುಮೋದಿಸುತ್ತೇನೆ. ಆದರೆ ಮೂಲಸಿದ್ಧಾಂತ ಒಂದೇ; ಅದೇ ನಮಗೆ ಕಾಣುವ ಪ್ರಕೃತಿಯೆಲ್ಲ ಅಲೆಯಂತೆ ಬಿದ್ದು ಎದ್ದು ಮುಂದುವರಿಯುತ್ತಿದೆ ಎಂಬುದು. ಕೆಳಗೆ ಬೀಳುವ, ಪೂರ್ಣ ಸಮತ್ವಕ್ಕೆ ಹಿಂತಿರುಗುವ ಸ್ಥಿತಿಯನ್ನೇ ಪ್ರಳಯ ಅಥವಾ ಎಲ್ಲ ಕಲ್ಪಗಳ ಅಂತ್ಯ ಎನ್ನುವರು. ಸೃಷ್ಟಿ ಪ್ರಳಯಗಳನ್ನು ನಮ್ಮ ದಾರ್ಶನಿಕರು ದೇವರ ಉಚ್ವಾಸ ನಿಶ್ವಾಸಕ್ಕೆ ಹೋಲಿಸುವರು. ದೇವರು ಸೃಷ್ಟಿಯನ್ನು ಹೊರಗೆ ಊದಿದಂತೆ, ಪುನಃ ಸೃಷ್ಟಿ ಅವನೆಡೆಗೆ ಬರುವುದು. ಅದು ಶಾಂತವಾದಾಗ ಜಗತ್ತು ಏನಾಗುವುದು? ಅದು ಸೂಕ್ಷ್ಮ ಸ್ಥಿತಿಯಲ್ಲಿರುವುದು. ಸಾಂಖ್ಯ ದರ್ಶನರೀತ್ಯಾ ಅದು ಕಾರಣಸ್ಥಿತಿಯಲ್ಲಿರುವುದು. ಕಾಲದೇಶಗಳು ನಿಮಿತ್ತ ಅಲ್ಲಿ ಮಾಯವಾಗುವುದಿಲ್ಲ. ಅವು ಅಲ್ಲಿವೆ. ಆದರೆ ಅತಿ ಸೂಕ್ಷ್ಮಸ್ಥಿತಿಯಲ್ಲಿವೆ, ಅಷ್ಟೆ. ನಾವು ಒಂದು ಸಣ್ಣ ಕಣವಾಗುವ ಸ್ಥಿತಿಗೆ ಸೃಷ್ಟಿ ಸಂಕುಚಿತವಾದರೆ ನಾವು ಹೀಗೆ ಬದಲಾಯಿಸಿರುವುದು ನಮಗೆ ಗೊತ್ತೇ ಆಗುವುದಿಲ್ಲ. ಏಕೆಂದರೆ ನಮಗೆ ಸಂಬಂಧಿಸಿರುವುದೆಲ್ಲ ಸಂಕುಚಿತವಾಗುತ್ತಿರುತ್ತದೆ. ಇದೆಲ್ಲ ಹಿಂತಿರುಗುವುದು, ಪುನಃ ಹೊರಬರುವುದು. ಕಾರಣ ಪರಿಣಾಮವಾಗುತ್ತಿರುವುದು. ಬದಲಾವಣೆ ಹೀಗೆ ಸಾಗುತ್ತಿರುವುದು.

ನಾವು ಈಗಿನ ಕಾಲದಲ್ಲಿ ಯಾವುದನ್ನು ದ್ರವ್ಯ ಎನ್ನುವೆವೊ ಅದನ್ನು ಹಿಂದಿನ ಮನಶ್ಶಾಸ್ತ್ರಜ್ಞರು ಪಂಚ ಭೂತಗಳೆಂದು ಕರೆದರು. ಅವರ ದೃಷ್ಟಿಯಲ್ಲಿ ಶಾಶ್ವತವಾಗಿರುವ ಒಂದು ಭೂತವಿದೆ. ಉಳಿದ ಭೂತಗಳೆಲ್ಲ ಅದರಿಂದಲೇ ಬರುತ್ತವೆ. ಅದೇ ಆಕಾಶ. ಅದು ಆಧುನಿಕ ಕಾಲದ \enginline{Ether} (ಆಕಾಶದ್ರವ್ಯ) ಎಂಬ ಭಾವನೆಯ ಸಮೀಪಕ್ಕೆ ಬರುವುದು. ಆದರೆ ಅದೇ ಅದಲ್ಲದಿರಬಹುದು. ಈ ದ್ರವ್ಯದೊಂದಿಗೆ ಮೂಲ ಶಕ್ತಿ ಒಂದಿದೆ. ಅದೇ ಪ್ರಾಣ. ಪ್ರಾಣ ಮತ್ತು ಆಕಾಶಗಳು ಹಲವು ಬಗೆಯಲ್ಲಿ ಸಂಯೋಗವಾಗಿ ಭೂತಗಳಾಗುವುವು. ಪ್ರತಿಯೊಂದು ಕಲ್ಪದ ಅಂತ್ಯದಲ್ಲಿ ಎಲ್ಲವೂ ಶಾಂತವಾಗಿ, ಆಕಾಶ ಪ್ರಾಣವೆಂಬ ಮೂಲ ಕಾರಣಕ್ಕೆ ಹಿಂತಿರುಗುವುವು. ಅತ್ಯಂತ ಪ್ರಾಚೀನ ಗ್ರಂಥವಾದ ಋಗ್ವೇದದಲ್ಲಿ ಸೃಷ್ಟಿಯನ್ನು ವಿವರಿಸುವ ಅತಿ ಸುಂದರವಾದ ಒಂದು ಮಂತ್ರವಿದೆ. ಅದು ಸುಂದರ ಕಾವ್ಯದಂತಿದೆ: "ಆದಿಯಲ್ಲಿ ವ್ಯಕ್ತವೂ ಇಲ್ಲ ಅವ್ಯಕ್ತವೂ ಇಲ್ಲದಾಗ, ತಮಸ್ಸು ತಮಸ್ಸಿನ ಮೇಲೆ ಹಾಯುತ್ತಿದ್ದಾಗ ಇದ್ದುದೇನು?” “ಅದು ಆಗ ಸ್ಪಂದಿಸದೆ ಇತ್ತು” ಎಂಬ ಉತ್ತರವನ್ನು ಕೊಡುವರು. ಪ್ರಾಣ ಆಗ ಇತ್ತು. ಆದರೆ ಅದಕ್ಕೆ ಚಲನೆ ಇರಲಿಲ್ಲ. “ಆನೀದವಾತಂ" ಎಂದರೆ ಇತ್ತು, ಆದರೆ ನಿಸ್ಪಂದವಾಗಿತ್ತು. ಬಹಳ ಕಾಲದ ಅನಂತರ ಕಲ್ಪ ಪ್ರಾರಂಭವಾದಾಗ ನಿಸ್ಪಂದವು ಸ್ಪಂದಿಸುವುದಕ್ಕೆ ಪ್ರಾರಂಭಿಸಿತು. ಪ್ರಾಣವು ಆಕಾಶಕ್ಕೆ ಹಲವು ಹೊಡೆತಗಳನ್ನು ಕೊಡುವುದು. ಕಣಗಳು ಘನೀಭೂತವಾಗುವುವು. ಅವು ಘನೀಭೂತವಾದಾಗ ಹಲವು ಭೂತಗಳು ಉತ್ಪತ್ತಿಯಾಗುವುವು. ಇದನ್ನು ಬಹಳ ವಿಚಿತ್ರವಾಗಿ ಭಾಷಾಂತರಿಸಿರುವುದನ್ನು ನಾವು ನೋಡುವೆವು, ಭಾಷಾಂತರಕಾರರು, ದಾರ್ಶನಿಕರ ಅಥವಾ ಭಾಷ್ಯಕಾರರ ಸಹಾಯವನ್ನು ಸ್ವೀಕರಿಸುವುದಿಲ್ಲ. ತಾವೇ ಅದನ್ನು ಅರ್ಥಮಾಡಿಕೊಳ್ಳುವುದಕ್ಕೂ ಬುದ್ಧಿಶಕ್ತಿಯಿಲ್ಲ. ಮೂಢರು ಕೆಲವು ಸಂಸ್ಕೃತ ಅಕ್ಷರಗಳನ್ನು ಓದಿ ಒಂದು ಇಡೀ ಪುಸ್ತಕವನ್ನು ಭಾಷಾಂತರಿಸುವರು. ಭೂತಗಳನ್ನು ಗಾಳಿ ಬೆಂಕಿ ಹೀಗೆ ಭಾಷಾಂತರಿಸುವರು. ಭಾಷ್ಯಕಾರರಲ್ಲಿ ನೋಡಿದರೆ ಅದಕ್ಕೆ ಗಾಳಿ ಮುಂತಾದ ಅರ್ಥವೇ ಇಲ್ಲ.

ಆಕಾಶವು ಪದೇ ಪದೇ ಪ್ರಾಣದ ಆಘಾತಕ್ಕೆ ಒಳಪಟ್ಟು ವಾಯು ಜನಿಸುತ್ತದೆ. ಈ ವಾಯುವು ಸ್ಪಂದಿಸುವುದು. ಸ್ಪಂದನ ಹೆಚ್ಚಾದಂತೆ ಅದರಿಂದ ಘರ್ಷಣೆ ಉಂಟಾಗಿ ಅಗ್ನಿ ಉದ್ಭವಿಸುವುದು. ಅನಂತರ ಅದು ದ್ರವೀಭೂತವಾಗಿ ನೀರು ಉದ್ಭವಿಸುವುದು. ಅನಂತರ ಆ ದ್ರವವು ಘನೀಭೂತವಾಗುವುದು. ಮೊದಲು ಆಕಾಶ, ಅನಂತರ ಚಲನೆ, ಅನಂತರ ಶಾಖ, ಆಮೇಲೆ ಅದರ ದ್ರವೀಕರಣ. ಅದು ಅನಂತರ ಘನೀಭೂತವಾಗಿ ಸ್ಥೂಲ ವಸ್ತುವಾಗುತ್ತದೆ. ಬಳಿಕ ಅದೇ ಕ್ರಮದಲ್ಲಿ ಹಿಂದಿರುಗಿ ಹೋಗುವುದು. ಘನವು ದ್ರವವಾಗುವುದು. ಅನಂತರ ಅದು ಶಾಖದ ರಾಶಿಯಾಗುವುದು. ಅದು ಅನಂತರ ಸ್ಪಂದನೆಯಾಗುವುದು. ಬಳಿಕ ಅದು ನಿಂತು ಈ ಕಲ್ಪ ಕೊನೆಗೊಳ್ಳುವುದು. ಅದು ಪುನಃ ಬಂದು ಪುನಃ ಆಕಾಶದಲ್ಲಿ ಲೀನವಾಗುವುದು. ಪ್ರಾಣವು ಆಕಾಶದ ಸಹಾಯವಿಲ್ಲದೆ ಕೆಲಸ ಮಾಡಲಾರದು. ಸ್ಪಂದನೆ, ಚಲನೆ, ಆಲೋಚನೆಯಂತೆ ನಮಗೆ ಗೊತ್ತಿರುವುದೆಲ್ಲ ಪ್ರಾಣದ ರೂಪಾಂತರ; ಆಕಾರ ಮತ್ತು ತಡೆಯಲ್ಪಡುವ ವಸ್ತುವಿನಂತೆ ಇರುವುದೆಲ್ಲ ಆಕಾಶದ ರೂಪಾಂತರ. ಪ್ರಾಣವು ಪ್ರತ್ಯೇಕವಾಗಿ ಇರಲಾರದು, ಅಥವಾ ಒಂದು ಮಧ್ಯವರ್ತಿ ಇಲ್ಲದೆ ಕೆಲಸ ಮಾಡಲಾರದು. ಅದು ಶುದ್ಧ ಪ್ರಾಣವಾಗಿರುವಾಗ ಆಕಾಶದಲ್ಲಿ ಇರುವುದು. ಅದು ಆಕರ್ಷಣ ಅಥವಾ ಕೇಂದ್ರಾಭಿಮುಖ ಶಕ್ತಿಯಂತೆ ರೂಪಾಂತರವಾಗಿರುವಾಗ ಅದಕ್ಕೆ ದ್ರವ್ಯವಿರಬೇಕು. ದ್ರವ್ಯವಿಲ್ಲದೆ ನೀವೆಂದೂ ಶಕ್ತಿಯನ್ನು ನೋಡಿರಲಾರಿರಿ. ಅಥವಾ ಶಕ್ತಿಯಿಲ್ಲದೆ ದ್ರವ್ಯವನ್ನು ನೋಡಿರಲಾರಿರಿ. ನಾವು ಯಾವುದನ್ನು ಸೂಕ್ಷ್ಮ ಸ್ಥಿತಿಯಲ್ಲಿ ಪ್ರಾಣ ಮತ್ತು ಆಕಾಶ ಎಂದು ಕರೆಯುತ್ತೇವೆಯೋ ಅವುಗಳ ಸ್ಥೂಲ ಅಭಿವ್ಯಕ್ತಿ, ಶಕ್ತಿ ಮತ್ತು ದ್ರವ್ಯ. ಆದರೆ ಪ್ರಾಣವನ್ನು ನೀವು ಇಂಗ್ಲಿಷಿನಲ್ಲಿ (\enginline{Life}) ಎನ್ನಬಹುದು. ಆದರೆ ಅದನ್ನು ಮಾನವ ಜೀವನಕ್ಕೆ ಅನ್ವಯಿಸಬಾರದು. ಅದೇ ಸಂದರ್ಭದಲ್ಲಿ ಅದು ಮಾನವನ ಅಂತರಾತ್ಮನೂ ಅಲ್ಲ. ಸೃಷ್ಟಿಗೆ ಒಂದು ಆದಿ ಅಂತ್ಯವಿಲ್ಲ. ಇದು ಎಂದೆಂದಿಗೂ ಆಗುತ್ತಿರುವುದು.

ಈ ಪುರಾತನ ಮನಶ್ಶಾಸ್ತ್ರಜ್ಞರ ಮತ್ತೊಂದು ಭಾವನೆಯನ್ನು ವಿವರಿಸುತ್ತೇನೆ. ಅದೇ ಎಲ್ಲಾ ಸ್ಥೂಲವಸ್ತುಗಳ ಸೂಕ್ಷ್ಮದ ಪರಿಣಾಮವೆನ್ನುವುದು. ಸ್ಥೂಲವಾಗಿರುವುದೆಲ್ಲ ಸೂಕ್ಷ್ಮದ ಆವಿರ್ಭಾವ. ಇದನ್ನೆ ತನ್ಮಾತ್ರ ಎಂದು ಕರೆಯುವರು. ನಾನೊಂದು ಹೂವನ್ನು ಮೂಸಿ ನೋಡುತ್ತೇನೆ. ಯಾವುದನ್ನಾದರೂ ನಾನು ಮೂಸಿ ನೋಡಬೇಕಾದರೆ ಅದು ನನ್ನ ಮೂಗಿನ ಸಂಪರ್ಕ ಹೊಂದಬೇಕು. ಹೂವು ಅಲ್ಲಿದೆ. ಅದು ನನ್ನ ಮೂಗಿನ ಹತ್ತಿರ ಬರುವುದು ಕಾಣುವುದಿಲ್ಲ. ಯಾವುದು ಹೂವಿನಿಂದ ಬರುವುದೊ ಮತ್ತು ನನ್ನ ಮೂಗಿನ ಸಂಪರ್ಕವನ್ನು ಪಡೆಯುವುದೊ ಅದೇ ತನ್ಮಾತ್ರ; ಹೂವಿನ ಸೂಕ್ಷ್ಮಕಣಗಳು. ಇದರಂತೆ ಶಾಖ ಬೆಳಕು ಎಲ್ಲವು ಕೂಡ. ಈ ತನ್ಮಾತ್ರಗಳನ್ನು ನಾವು ಪುನಃ ಕಣಗಳಂತೆ ಅಂತರ್ವಿಭಾಗ ಮಾಡಬಹುದು. ಹಲವು ತಜ್ಞರಲ್ಲಿ ಹಲವು ಸಿದ್ದಾಂತಗಳಿವೆ. ಇವು ಕೇವಲ ಸಿದ್ದಾಂತಗಳೆಂದು ನಮಗೆ ಗೊತ್ತಿದೆ. ಸ್ಥೂಲವಾಗಿರುವುದೆಲ್ಲಾ ಬಹು ಸೂಕ್ಷ್ಮದಿಂದ ಆಗಿದೆ ಎಂಬುದನ್ನು ಅರಿತರೆ ಸಾಕು. ಮೊದಲು ನಮಗೆ ಸ್ಥೂಲವಸ್ತುಗಳು ದೊರಕುವುವು. ಅವನ್ನು ಹೊರಗೆ ನೋಡುವೆವು. ಅನಂತರ ಮೂಗು, ಕಣ್ಣು ಮತ್ತು ಕಿವಿಗಳ ಸಂಪರ್ಕಕ್ಕೆ ಬರುವ ಸೂಕ್ಷ್ಮವಸ್ತುಗಳಿವೆ. ಈಥರ್‌ನ ಅಲೆಗಳು ನನ್ನ ಕಣ್ಣಿಗೆ ತಾಗಬೇಕು. ಅವನ್ನು ನಾನು ನೋಡಲಾರೆ. ಆದರೂ ನಾನು ಬೆಳಕನ್ನು ನೋಡಬೇಕಾದರೆ ಅವು ನನ್ನ ಕಣ್ಣಿಗೆ ತಾಗಬೇಕೆಂಬುದು ಗೊತ್ತಿದೆ.

ಇಲ್ಲಿ ಕಣ್ಣುಗಳಿವೆ. ಆದರೆ ಕಣ್ಣು ನೋಡಲಾರದು. ಮೆದುಳಿನಲ್ಲಿರುವ ಕೇಂದ್ರವನ್ನು ತೆಗೆದುಬಿಟ್ಟರೆ ಕಣ್ಣುಗಳಿದ್ದರೂ ರೆಟಿನಾ ಮೇಲೆ ಚಿತ್ರವೆಲ್ಲ ಬಿದ್ದರೂ ಕಣ್ಣು ನೋಡಲಾರದು. ಕಣ್ಣೇ ನೋಡುವ ಅವಯವವಲ್ಲ. ನೋಟಕ್ಕೆ ಅದು ಗೌಣ. ನೋಟಕ್ಕೆ ಮೆದುಳಿನಲ್ಲಿರುವ ನೋಟದ ನರಗಳ ಕೇಂದ್ರವೇ ಮುಖ್ಯ. ಇದರಂತೆಯೇ ಮೂಗು ಕೂಡ. ಇದರ ಹಿಂದೆ ಬೇರೊಂದು ಕೇಂದ್ರವಿರುವುದು. ಇಂದ್ರಿಯಗಳು ಕೇವಲ ಬಾಹ್ಯ ಅವಯವಗಳು ಅಷ್ಟೆ. ಇದರ ಹಿಂದೆ ಇರುವುವನ್ನೆ ಸಂಸ್ಕೃತದಲ್ಲಿ ಇಂದ್ರಿಯಗಳೆಂದು ಕರೆಯುವರು. ಇವೇ ನಿಜವಾಗಿಯೂ ಗ್ರಹಣ ಕೇಂದ್ರಗಳು.

ಮನಸ್ಸು ಇಂದ್ರಿಯಕ್ಕೆ ಸೇರಬೇಕು. ಆಗ ಮಾತ್ರ ನಮಗೆ ಅರಿವಾಗುತ್ತದೆ. ನಾವು ಅಧ್ಯಯನದಲ್ಲಿ ನಿರತರಾಗಿರುವಾಗ ಗಡಿಯಾರ ಹೊಡೆಯುವುದು ನಮಗೆ ಕೇಳಿಸುವುದಿಲ್ಲ. ಇದು ಸಾಮಾನ್ಯ ಅನುಭವವಾಗಿದೆ. ಏತಕ್ಕೆ? ಕಿವಿ ಇತ್ತು, ಧ್ವನಿ ಇದರ ಮೂಲಕ ಮೆದುಳಿಗೆ ಒಯ್ಯಲ್ಪಟ್ಟಿತು. ಆದರೂ ಕೇಳಿರಲಿಲ್ಲ. ಏತಕ್ಕೆಂದರೆ ಮನಸ್ಸು ಶ್ರವಣೇಂದ್ರಿಯಕ್ಕೆ ಸೇರಿರಲಿಲ್ಲ.

ಪ್ರತಿಯೊಂದು ಕರಣಗಳಿಗೂ ಬೇರೆ ಬೇರೆ ಇಂದ್ರಿಯಗಳಿವೆ. ಎಲ್ಲ ಕರಣಗಳಿಗೂ ಇಂದ್ರಿಯವು ಒಂದೇ ಆಗಿದ್ದರೆ ಮನಸ್ಸು ಅದರೊಡನೆ ಸೇರಿದಾಗ ಎಲ್ಲ ಕರಣಗಳೂ ಏಕಕಾಲದಲ್ಲಿ ಕೆಲಸ ಮಾಡುತ್ತಿದ್ದುವು. ಆದರೆ ಅದು ಹಾಗಿಲ್ಲ. ಗಡಿಯಾರದ ಉದಾಹರಣೆಯಲ್ಲಿ ನಾವು ಇದನ್ನು ನೋಡಿದೆವು. ಎಲ್ಲಾ ಕರಣಗಳಿಗೂ ಒಂದೇ ಇಂದ್ರಿಯವಿದ್ದಿದ್ದರೆ ಮನಸ್ಸು ಏಕಕಾಲದಲ್ಲಿ ನೋಡಲೂಬಹುದು, ಕೇಳಲೂ ಬಹುದು, ಮೂಸಿ ನೋಡಲೂಬಹುದು ಆಗಿತ್ತು. ಏಕಕಾಲದಲ್ಲಿ ಇವನ್ನೆಲ್ಲ ಮಾಡದೆ ಇರುವುದಕ್ಕೆ ಸಾಧ್ಯವೇ ಇರುತ್ತಿರಲಿಲ್ಲ. ಆದಕಾರಣ ಪ್ರತಿಯೊಂದು ಕರಣಕ್ಕೂ ಬೇರೆ ಬೇರೆ ಇಂದ್ರಿಯಗಳು ಇರಬೇಕೆಂಬುದು ನಿಶ್ಚಿತವಾಯಿತು. ಆಧುನಿಕ ಶರೀರ ಶಾಸ್ತ್ರ ಕೂಡ ಇದನ್ನು ಅನುಮೋದಿಸುವುದು. ಏಕಕಾಲದಲ್ಲಿ ನೋಡಲೂ ಕೇಳಲೂ ನಿಜವಾಗಿ ಸಾಧ್ಯ. ಏಕೆಂದರೆ ಮನಸ್ಸು ಎರಡು ಇಂದ್ರಿಯಗಳಲ್ಲಿಯೂ ಸ್ವಲ್ಪ ಸ್ವಲ್ಪವಿರುವುದು.

ಈ ಇಂದ್ರಿಯಗಳು ಯಾವುದರಿಂದ ಆಗಿವೆ? ಕಣ್ಣು ಕಿವಿ ಮುಂತಾದ ಬಾಹ್ಯ ಕರಣಗಳು ಸ್ಥೂಲ ವಸ್ತುಗಳಿಂದ ಆಗಿರುವುದನ್ನು ನಾವು ನೋಡುತ್ತೇವೆ. ಇಂದ್ರಿಯಗಳು ಕೂಡ ಈ ವಸ್ತುಗಳಿಂದಲೇ ಆಗಿವೆ. ದೇಹವು ಹೇಗೆ ಸ್ಥೂಲ ಭೂತಗಳಿಂದ ಆಗಿದ್ದು ಪ್ರಾಣವನ್ನು ಸ್ಥೂಲಶಕ್ತಿಗಳನ್ನಾಗಿ ಪರಿವರ್ತಿಸುತ್ತದೆಯೋ ಹಾಗೆಯೇ ಇಂದ್ರಿಯಗಳು ಆಕಾಶ, ವಾಯು, ತೇಜಸ್ಸು ಮುಂತಾದ ಸೂಕ್ಷ್ಮ ಭೂತಗಳಿಂದ ಆಗಿದ್ದು ಪ್ರಾಣವನ್ನು ಸೂಕ್ಷ್ಮ ಗ್ರಹಣ ಶಕ್ತಿಯಾಗಿ ಪರಿವರ್ತಿಸುತ್ತದೆ. ಪಂಚೇಂದ್ರಿಯಗಳು, ಪ್ರಾಣಕ್ರಿಯೆಗಳು, ಮನಸ್ಸು, ಬುದ್ದಿ ಇವೆಲ್ಲ ಸೇರಿ ಮನುಷ್ಯನ ಲಿಂಗಶರೀರ ಅಥವಾ ಸೂಕ್ಷ್ಮಶರೀರವಾಗುವುದು. ಲಿಂಗಶರೀರಕ್ಕೆ ನಿಜವಾದ ರೂಪವಿದೆ; ಏಕೆಂದರೆ ಭೌತಿಕವಾದ ಪ್ರತಿಯೊಂದಕ್ಕೂ ಒಂದು ರೂಪ ಬೇಕೇ ಬೇಕು.

ಇಂಗ್ಲಿಷಿನ \enginline{mind} ಎಂಬ ಪದವನ್ನು ಮನಸ್ಸು ಎಂದು ಕರೆಯಬಹುದು. ಮನಸ್ಸು ಎಂದರೆ ವೃತ್ತಿರೂಪದಲ್ಲಿರುವ ಚಿತ್ರ. ಸರೋವರಕ್ಕೆ ನೀವು ಒಂದು ಕಲ್ಲನ್ನು ಎಸೆದರೆ ಮೊದಲು ಅಲ್ಲಿ ಸ್ಪಂದನಗಳೇಳುವುವು. ಅನಂತರ ಪ್ರತಿರೋಧವಿರುತ್ತದೆ. ಮೊದಲು ನೀರು ಸ್ಪಂದಿಸುವುದು, ಅನಂತರ ಕಲ್ಲಿಗೆ ವಿರೋಧವಾಗಿ ಪ್ರತಿಕ್ರಿಯೆಯನ್ನು ವ್ಯಕ್ತಪಡಿಸುವುದು. ಇದರಂತೆಯೇ ಚಿತ್ತಕ್ಕೆ ಯಾವುದಾದರೊಂದು ಹೊಸ ವಿಷಯ ಬಂದರೆ ಮೊದಲು ಅದು ಸ್ಪಂದಿಸುವುದು. ಇದೇ ಮನಸ್ಸು. ಅನಂತರ ಮನಸ್ಸು ಈ ಸಂಸ್ಕಾರವನ್ನು ಇನ್ನೂ ಮುಂದೆ ಒಯ್ದು, ನಿಶ್ಚಯಾತ್ಮಕ ಶಕ್ತಿಯಾದ ಬುದ್ದಿಯ ಮುಂದೆ ಇಡುತ್ತದೆ. ಆಗ ಆ ಬುದ್ದಿಯು ಪ್ರತಿಕ್ರಿಯಿಸುತ್ತದೆ. ಬುದ್ದಿಯ ಹಿಂದೆ ಅಹಂಕಾರವಿದೆ; ಇದೇ 'ನಾನು'' ಎಂಬ ಅರಿವು. ಅಹಂಕಾರದ ಹಿಂದೆಯೇ ಮಹತ್ ಇರುವುದು. ಇದೇ ಪ್ರಕೃತಿಯ ಶ್ರೇಷ್ಠ ಆವಿರ್ಭಾವ. ಪ್ರತಿಯೊಂದೂ ಅದರ ಹಿಂದಿನ ಪರಿಣಾಮ. ಸರೋವರದ ಉದಾಹರಣೆಯಲ್ಲಿ ಅದಕ್ಕೆ ಬೀಳುವ ಪ್ರತಿಯೊಂದು ಆಘಾತವೂ ಹೊರಗಿನಿಂದ ಬರುವುದು: ಆದರೆ ಮನಸ್ಸಿಗೆ ಆಘಾತ ಹೊರಗಿನಿಂದ ಬೇಕಾದರೂ ಬರಬಹುದು, ಒಳಗಿನಿಂದ ಬೇಕಾದರೂ ಬರಬಹುದು. ಮಹತ್ತಿನ ಹಿಂದೆ ಪುರುಷನಿರುವನು, ಇವನೇ ಆತ್ಮ. ಅವನು ಪರಿಶುದ್ಧ, ಪರಿಪೂರ್ಣ, ಅವನೇ ನಿಜವಾಗಿಯೂ ದೃಕ್. ಈ ಬದಲಾವಣೆಯೆಲ್ಲ ಇರುವುದು ಅವನಿಗಾಗಿ,

ಮನುಷ್ಯನು ಈ ವಿಕಾರಗಳನ್ನೆಲ್ಲಾ ನೋಡುತ್ತಿರುವನು. ಅವನೆಂದಿಗೂ ಅಶುದ್ಧನಲ್ಲ. ಆದರೆ ಅಭ್ಯಾಸದಿಂದ ಅಶುದ್ದನಾದಂತೆ ತೋರುವನು. ಒಂದು ಸ್ಪಟಿಕಮಣಿಯ ಹಿಂದೆ ಕೆಂಪು ಅಥವಾ ನೀಲಿ ಹೂವನ್ನು ತಂದಂತೆ. ಸ್ಪಟಿಕದ ಮೂಲಕ ಅದರ ಬಣ್ಣ ಪ್ರತಿಬಿಂಬಿತವಾದರೂ ಸ್ಪಟಿಕವು ಶುದ್ಧವಾಗಿಯೇ ಇರುತ್ತದೆ. ದೋಷವೂ ಬರುವುದಿಲ್ಲ. ಹಲವು ಆತ್ಮಗಳಿವೆ ಎಂದು ಇಟ್ಟುಕೊಳ್ಳೋಣ. ಪ್ರತಿಯೊಂದು ಆತ್ಮವೂ ಪೂರ್ಣವಾಗಿದೆ, ಪರಿಶುದ್ಧವಾಗಿದೆ. ಹಲವು ಬಗೆಯ ಸ್ಥೂಲ ಸೂಕ್ಷ್ಮ ಮಾಲಿನ್ಯಗಳು ಅದರ ಮೇಲೆ ಕವಿದು ಅದನ್ನು ಹಲವು ಬಣ್ಣದಿಂದ ಕೂಡಿರುವಂತೆ ಮಾಡುವುದು. ಪ್ರಕೃತಿ ಇದನ್ನೆಲ್ಲ ಏತಕ್ಕೆ ಮಾಡುವುದು? ಆತ್ಮವಿಕಾಸಕ್ಕೆ ಪ್ರಕೃತಿ ಈ ಬದಲಾವಣೆಯನ್ನು ಹೊಂದುತ್ತಿದೆ. ಈ ಸೃಷ್ಟಿಯೆಲ್ಲ ಜೀವಿಯ ಉದ್ದಾರಕ್ಕೆ, ಜೀವಿ ಮುಕ್ತವಾಗಲಿ ಎಂಬುದಕ್ಕೆ. ವಿಶ್ವವೆಂಬ ಬೃಹದ್ಗ್ರಂಥವು ಮನುಷ್ಯನು ತನ್ನನ್ನು ಓದಲಿ ಎಂದು ಅವನೆದುರಿಗೆ ತೆರೆದಿದೆ. ಕ್ರಮೇಣ ಜೀವಿಯು ತಾನು ಸರ್ವಜ್ಞ, ಸರ್ವಶಕ್ತ ಎಂಬುದನ್ನು ಅರಿಯುವನು. ನಮ್ಮ ಕೆಲವು ಶ್ರೇಷ್ಠ ಮನಶ್ಶಾಸ್ತ್ರಜ್ಞರು ನೀವು ನಂಬುವಂತಹ ದೇವರನ್ನು ನಂಬುವುದಿಲ್ಲವೆಂಬುದನ್ನು ಇಲ್ಲಿ ಹೇಳಬೇಕಾಗಿದೆ. ನಮ್ಮ ಮನಶ್ಶಾಸ್ತ್ರದ ಪಿತಾಮಹನಂತಿರುವ ಕಪಿಲ, ದೇವರನ್ನು ಒಪ್ಪಿಕೊಳ್ಳುವುದೇ ಇಲ್ಲ. ಸಗುಣ ಈಶ್ವರ ಅನಾವಶ್ಯಕ, ಸೃಷ್ಟಿಯ ಕೆಲಸವನ್ನು ನೋಡಿಕೊಳ್ಳುವುದಕ್ಕೆ ಪ್ರಕೃತಿಯೇ ಸಾಕು ಎನ್ನುತ್ತಾನೆ. ಯಾವುದನ್ನು ರಚನಾವಾದ (\enginline{Design Theory}) ಎನ್ನುವರೋ ಅದನ್ನು ಇವನು ಅಲ್ಲಗಳೆಯುವನು. ಇದರಷ್ಟು ಬಾಲಬುದ್ದಿಯ ಸಿದ್ದಾಂತ ಎಂದೂ ಇರಲಿಲ್ಲ ಎನ್ನುವನು. ಆದರೆ ಅವನು ಒಂದು ಬಗೆಯ ವಿಚಿತ್ರ ದೇವರನ್ನು ಒಪ್ಪಿಕೊಳ್ಳುವನು. ಅವನ ಪ್ರಕಾರ ನಾವೆಲ್ಲ ಮುಕ್ತರಾಗುವುದಕ್ಕೆ ಪ್ರಯತ್ನಿಸುತ್ತಿರುವೆವು. ನಾವು ಮುಕ್ತರಾದಾಗ ಪ್ರಕೃತಿಲಯರಾಗುವೆವು ಮತ್ತು ಮತ್ತೊಂದು ಕಲ್ಪದ ಆದಿಯಲ್ಲಿ ಆ ಕಲ್ಪಕ್ಕೆ ಒಡೆಯರಾಗುವುದಕ್ಕೆ ಬರುವೆವು. ಆಗ ನಾವು ಸರ್ವಜ್ಞನೂ ಸರ್ವಶಕ್ತನೂ ಆದ ಜೀವಿಯಾಗಿ ಬರುವೆವು. ಈ ದೃಷ್ಟಿಯಲ್ಲಿ ನಮ್ಮನ್ನು ದೇವರು ಎಂದು ಬೇಕಾದರೆ ಹೇಳಿಕೊಳ್ಳಬಹುದು. ನಾನು ನೀವು ಮತ್ತು ಇನ್ನೂ ಕೆಳಗೆ ಇರುವವರು ಯಾವುದಾದರೂ ಕಲ್ಪದಲ್ಲಿ ದೇವರಾಗಬಹುದು. ಆದರೆ ಇವರೆಲ್ಲ ತಾತ್ಕಾಲಿಕ. ಆದರೆ ಸನಾತನವಾಗಿರುವ, ಸರ್ವದಾ ಸರ್ವಶಕ್ತನಾಗಿ ಸರ್ವೆಶ್ವರನಾಗಿರುವ ದೇವರು ಸಾಧ್ಯವಿಲ್ಲ ಎನ್ನುವರು. ಅಂತಹ ದೇವರಿದ್ದರೆ ಈ ತೊಂದರೆ ಬರುವುದು. ಅವನು ಬದ್ದನಾಗಿರಬೇಕು ಇಲ್ಲವೆ ಮುಕ್ತನಾಗಿರಬೇಕು. ನಿತ್ಯಮುಕ್ತನಾದ ದೇವರು ಸೃಷ್ಟಿಸಲಾರ. ಸೃಷ್ಟಿಯ ಆವಶ್ಯಕತೆಯೆ ಅವನಿಗಿರುವುದಿಲ್ಲ. ಅವನು ಬದ್ಧನಾಗಿದ್ದರೆ ಸೃಷ್ಟಿಸುತ್ತಿರಲಿಲ್ಲ. ಏಕೆಂದರೆ ಅವನಿಗೆ ಅದು ಆಗುತ್ತಿರಲಿಲ್ಲ. ಅದಕ್ಕೆ ಅವನಿಗೆ ಶಕ್ತಿ ಇರುತ್ತಿರಲಿಲ್ಲ. ಎರಡು ದೃಷ್ಟಿಯಿಂದಲೂ ಸರ್ವಜ್ಞನಾದ ಸರ್ವಶಕ್ತನಾದ ದೇವರು ಅಥವಾ ಈಶ್ವರ ಇರುವುದಕ್ಕೆ ಸಾಧ್ಯವಾಗುತ್ತಿರಲಿಲ್ಲ. ನಮ್ಮ ಶಾಸ್ತ್ರದಲ್ಲಿ ಎಲ್ಲಿ ದೇವರು ಎಂಬ ಪದವನ್ನು ಉಪಯೋಗಿಸುವರೋ ಅಲ್ಲೆಲ್ಲ ಈ ಪ್ರಕೃತಿಲಯರನ್ನು ಎಂದರೆ ಮುಕ್ತರಾಗಿರುವ ಮಾನವರನ್ನು ಉದ್ದೇಶಿಸುವರು ಎನ್ನುವರು.

ಕಪಿಲನು ಎಲ್ಲ ಆತ್ಮಗಳ ಒಂದು ಐಕ್ಯತೆಯನ್ನು ನಂಬುವುದಿಲ್ಲ. ಇಲ್ಲಿಯವರೆಗೆ ಅವನ ವಿಶ್ಲೇಷಣೆ ಅದ್ಭುತವಾಗಿದೆ. ಭಾರತದ ಆಲೋಚನಾಪರರ ಮೂಲ ಪುರುಷ ಇವನೆ. ಬೌದ್ಧ ಧರ್ಮ ಮತ್ತು ಇತರ ಸಿದ್ದಾಂತಗಳು ಇವನ ಭಾವನೆಯ ಪರಿಣಾಮ.

ಕಪಿಲನ ಮನಶ್ಶಾಸ್ತ್ರದ ಪ್ರಕಾರ ಪ್ರತಿಯೊಬ್ಬ ಜೀವರೂ ತಮ್ಮ ಆಜನ್ಮಸಿದ್ದ ಹಕ್ಕಾದ ಸರ್ವಶಕ್ತಿತ್ವ ಮತ್ತು ಸರ್ವಜ್ಞತೆಯನ್ನು ಪಡೆದು ಸ್ವತಂತ್ರರಾಗಬಹುದು. ಈ ಬಂಧನ ಹೇಗೆ ಬಂತು ಎಂಬ ಪ್ರಶ್ನೆ ಉದ್ಭವಿಸುವುದು. ಕಪಿಲನು ಇದನ್ನು ಅನಾದಿ ಎನ್ನುವನು. ಇದಕ್ಕೆ ಆದಿ ಇಲ್ಲದೆ ಇದ್ದರೆ ಅಂತ್ಯವೂ ಇಲ್ಲ. ನಾವೆಂದಿಗೂ ಮುಕ್ತರಾಗಲಾರವು. ಕಪಿಲನ ಪ್ರಕಾರ ಬಂಧನಕ್ಕೆ ಆದಿ ಇಲ್ಲದೆ ಇದ್ದರೂ, ಆತ್ಮನಿಗಿರುವಂತೆ ಅದಕ್ಕೆ ಏಕಪ್ರಕಾರವಾದ ಲಕ್ಷಣವಿಲ್ಲ. ಪ್ರಕೃತಿಗೆ (ಬಂಧನದ ಕಾರಣಕ್ಕೆ) ಆದಿ ಅಂತ್ಯಗಳಿಲ್ಲ ಎಂದರೆ ಆತ್ಮನ ಆದಿ ಅಂತ್ಯ ರಹಿತವಾದುದು ಎಂಬ ಅರ್ಥದಲ್ಲಲ್ಲ. ಏಕೆಂದರೆ ಪ್ರಕೃತಿಗೆ ಒಂದು ವ್ಯಕ್ತಿತ್ವವಿಲ್ಲ. ಪ್ರತಿಕ್ಷಣವೂ ಹೊಸ ನೀರು ಬರುತ್ತಿರುವ ನದಿಯಂತೆ ಅದು. ಆದರೆ ನದಿ ಎಂಬುದಕ್ಕೆ ಏಕಪ್ರಕಾರವಾದ ಒಂದು ಮೊತ್ತವಿಲ್ಲ. ಪ್ರಕೃತಿಯಲ್ಲಿ ಪ್ರತಿಯೊಂದು ಬದಲಾಗುತ್ತಿರುತ್ತದೆ. ಆತ್ಮವು ಎಂದೂ ಬದಲಾಗುವುದಿಲ್ಲ. ಪ್ರಕೃತಿ ಯಾವಾಗಲೂ ಬದಲಾಯಿಸುತ್ತಿರುವುದರಿಂದ ಆತ್ಮವು ಬಂಧನದಿಂದ ಪಾರಾಗಲು ಸಾಧ್ಯ.

ಇಡೀ ವಿಶ್ವವು, ಅದರ ಒಂದು ಅಂಶ ಹೇಗೆ ರಚನೆಯಾಗಿದೆಯೋ ಅದೇ ಕ್ರಮದ ಮೇಲೆ ಆಗಿದೆ. ನನಗೆ ಒಂದು ಮನಸ್ಸು ಇರುವಂತೆ ಒಂದು ವಿಶ್ವ ಮನಸ್ಸಿದೆ. ವ್ಯಷ್ಟಿಯಂತೆಯೆ ಸಮಷ್ಟಿ. ಒಂದು ವಿಶ್ವ ಸ್ಥೂಲದೇಹವಿದೆ, ಅದರ ಹಿಂದೆ ವಿಶ್ವ ಸೂಕ್ಷ್ಮದೇಹವಿದೆ, ಅದರ ಹಿಂದೆ ಒಂದು ವಿಶ್ವ ಅಹಂಕಾರ ಅಥವಾ ಚೇತನವಿದೆ, ಅದರ ಹಿಂದೆ ವಿಶ್ವ ಮಹತ್ತಿದೆ. ಇವೆಲ್ಲ ಪ್ರಕೃತಿಯಲ್ಲಿವೆ, ಅದರ ಆವಿರ್ಭಾವದಲ್ಲಿವೆ, ಅದರ ಹೊರಗೆ ಇಲ್ಲ.

ನಮಗೆ ತಂದೆತಾಯಿಗಳಿಂದ ಬಂದ ಸ್ಥೂಲದೇಹವಿದೆ. ಜೊತೆಗೆ ನಮ್ಮ ಅಹಂಕಾರವಿದೆ. ಖಂಡಿತವಾದ ಆನುವಂಶಿಕತೆಯು ನಮ್ಮ ದೇಹ ಮಾತಾಪಿತೃಗಳ ದೇಹದ ಅಂಶ, ನನ್ನ ಚಿತ್ತ ಮತ್ತು ಅಹಂಕಾರ ಕೂಡ ಅವರ ಚಿತ್ತ ಅಹಂಕಾರಗಳ ಅಂಶ ಎಂದು ಸಾರುವುದು. ನಮ್ಮ ತಾಯಿತಂದೆಗಳಿಂದ ಬಂದ ಭಾಗಕ್ಕೆ ವಿಶ್ವಪ್ರಜ್ಞೆಯಿಂದ ನಾವು ಹೀರಿದ ಭಾಗವನ್ನು ಸೇರಿಸಬಹುದು. ನಮಗೆ ಬೇಕಾದುದನ್ನು ಹೀರಲು ಅನಂತ ಪ್ರಜ್ಞೆಯ ಆಗರವಿದೆ, ಅದರಿಂದ ನಾವು ಪ್ರತಿಕ್ಷಣವೂ ನಮಗೆ ಬೇಕಾದುದನ್ನು ಸೆಳೆದುಕೊಳ್ಳುತ್ತಿರುವೆವು. ವಿಶ್ವದಲ್ಲಿ ಮಾನಸಿಕ ಶಕ್ತಿಯ ಅನಂತ ಆಗರವಿದೆ. ಅದರಿಂದ ನಾವು ನಮಗೆ ಬೇಕಾದುದನ್ನು ಪ್ರತಿಕ್ಷಣವೂ ಸೆಳೆದುಕೊಳ್ಳುತ್ತಿರುವೆವು. ಆದರೆ ಬೀಜ ಮಾತಾಪಿತೃಗಳಿಂದ ಬರಬೇಕು. ನಮ್ಮ ಸಿದ್ಧಾಂತದಲ್ಲಿ ಆನುವಂಶಿಕತೆ ಮತ್ತು ಪುನರ್ಜನ್ಮ ಎರಡೂ ಇವೆ. ಆನುವಂಶಿಕ ಸಿದ್ಧಾಂತದ ಪ್ರಕಾರ ಪುನರ್ಜನ್ಮ ಪಡೆಯುವ ಜೀವವು ತಂದೆತಾಯಿಗಳಿಂದ ತನಗೆ ದೇಹೋತ್ಪತ್ತಿಗೆ ಅಗತ್ಯವಾದ ಸಾಮಗ್ರಿಗಳನ್ನು ಪಡೆಯುತ್ತದೆ.

ಕೆಲವು ಐರೋಪ್ಯ ತತ್ತ್ವಶಾಸ್ತ್ರಜ್ಞರು ನಾನಿರುವುದರಿಂದ ಪ್ರಪಂಚವಿದೆ, ನಾನಿಲ್ಲದೆ ಇದ್ದರೆ ಪ್ರಪಂಚವಿಲ್ಲ ಎಂದು ಸಾರುವರು. ಕೆಲವು ವೇಳೆ ಹೀಗೆ ಹೇಳುವರು: ಪ್ರಪಂಚದ ಜನರೆಲ್ಲ ಸತ್ತುಹೋಗಿ, ವಿಷಯ ವಸ್ತುಗಳನ್ನು ಗ್ರಹಿಸುವ ಮತ್ತು ಆಲೋಚಿಸುವ ಪ್ರಾಣಿಗಳಿಲ್ಲದೆ ಇದ್ದರೆ, ಈ ಆವಿರ್ಭಾವವೆಲ್ಲ ಮಾಯವಾಗುವುದು. ಆದರೆ ಐರೋಪ್ಯ ದಾರ್ಶನಿಕರಿಗೆ ತತ್ವ ಗೊತ್ತಿದ್ದರೂ ಅದರ ಮನಶ್ಶಾಸ್ತ್ರ ಗೊತ್ತಿಲ್ಲ. ಆಧುನಿಕ ತತ್ತ್ವಶಾಸ್ತ್ರಕ್ಕೆ ಅದೆಲ್ಲೋ ಸ್ವಲ್ಪ ಹೊಳೆದಿದೆ. ಸಾಂಖ್ಯರ ದೃಷ್ಟಿಯಿಂದ ನೋಡಿದಾಗ ಇದನ್ನು ತಿಳಿದುಕೊಳ್ಳುವುದು ಸುಲಭ. ಸಾಂಖ್ಯದರ್ಶನದ ಪ್ರಕಾರ ನನ್ನ ಮನಸ್ಸಿನ ಅಂಶ ಇಲ್ಲದೆ ಇದ್ದರೆ ಯಾವ ವಸ್ತುವೂ ಇರಲಾರದು. ಮೇಜಿನ ನಿಜಸ್ಥಿತಿ ನನಗೆ ಗೊತ್ತಿಲ್ಲ. ಅದರ ಒಂದು ಭಾವನೆ ಕಣ್ಣಿಗೆ ಬೀಳುವುದು. ಅನಂತರ ಅದಕ್ಕೆ ಸಂಬಂಧಪಟ್ಟ ಇಂದ್ರಿಯ ಕೇಂದ್ರಕ್ಕೆ ಹೋಗುವುದು. ಅಲ್ಲಿಂದ ಮನಸ್ಸಿಗೆ ತಾಕುವುದು. ಅಲ್ಲಿ ಪ್ರತಿಕ್ರಿಯೆಯಾಗುವುದು. ಈ ಪ್ರತಿಕ್ರಿಯೆಯೇ ಮೇಜು. ಸರೋವರಕ್ಕೆ ಒಂದು ಕಲ್ಲನ್ನು ಎಸೆದಂತೆ ಇದು. ಸರೋವರ ಕಲ್ಲಿನ ಸುತ್ತಲೂ ಒಂದು ಅಲೆಯನ್ನು ಎಬ್ಬಿಸುವುದು. ನಮಗೆ ತಿಳಿದಿರುವುದು ಈ ಅಲೆ ಮಾತ್ರ. ಹೊರಗೆ ಏನು ಇದೆಯೊ ಅದು ಯಾರಿಗೂ ಗೊತ್ತಿಲ್ಲ. ನಾನು ಅದನ್ನು ತಿಳಿಯಲೆತ್ನಿಸಿದಾಗ ಅದು ನಾನು ಒದಗಿಸುವ ವಸ್ತುವೇ ಆಗುತ್ತದೆ. ನಾನು ನನ್ನ ಮನಸ್ಸಿನಿಂದ ಕಣ್ಣಿಗೆ ವಿಷಯವನ್ನು ಒದಗಿಸುವೆನು. ಹೊರಗೆ ಏನೋ ಒಂದು ವಸ್ತುವಿದೆ. ಅದು ಕೇವಲ ನಿಮಿತ್ತ ಮಾತ್ರ, ಸಲಹೆಮಾತ್ರ. ಆ ಸಲಹೆಯ ಮೇಲೆ ನನ್ನ ಮನಸ್ಸನ್ನು ಆರೋಪ ಮಾಡುವೆನು. ಅದು ನನಗೆ ಈಗ ಕಾಣುವ ರೂಪವನ್ನು ಪಡೆಯುವುದು. ನಾವೆಲ್ಲ ಹೇಗೆ ಒಂದೇ ವಸ್ತುವನ್ನು ನೋಡುವೆವು? ನಮಗೆಲ್ಲಾ ವಿಶ್ವಮನಸ್ಸಿನ ಸಮಾನ ಅಂಶಗಳಿವೆ. ಯಾರಿಗೆ ಒಂದೇ ಸಮನಾದ ಮನಸ್ಸು ಇದೆಯೋ ಅವರು ಒಂದೇ ಸಮನಾದ ವಸ್ತುಗಳನ್ನು ನೋಡುವರು. ಯಾರಿಗೆ ಇಲ್ಲವೊ ಅವರು ನೋಡುವುದಿಲ್ಲ.

