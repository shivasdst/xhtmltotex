
\chapter[ಏಕಾಗ್ರತೆ ಮತ್ತು ಪ್ರಾಣಾಯಾಮ]{ಏಕಾಗ್ರತೆ ಮತ್ತು ಪ್ರಾಣಾಯಾಮ\protect\footnote{\engfoot{C.W, Vol. VI, P. 37}}}

ಮನುಷ್ಯರಿಗೂ ಪ್ರಾಣಿಗಳಿಗೂ ಇರುವ ವ್ಯತ್ಯಾಸವೇ, ಅವರಲ್ಲಿ ಇರುವ ಏಕಾಗ್ರತೆಯ ಶಕ್ತಿಯಲ್ಲಿದೆ. ಜೀವನದ ಯಾವ ಕಾರ್ಯಕ್ಷೇತ್ರದಲ್ಲಾಗಲಿ, ಯಶಸ್ಸೆಲ್ಲ ಈ ಏಕಾಗ್ರತೆಯ ಪರಿಣಾಮ. ಎಲ್ಲರಿಗೂ ಏಕಾಗ್ರತೆಯ ವಿಷಯವಾಗಿ ಏನಾದರೂ ಗೊತ್ತೇ ಇರುವುದು. ನಾವು ಇದರ ಪರಿಣಾಮವನ್ನು ಪ್ರತಿ ದಿನ ನೋಡುತ್ತಿರುವೆವು. ಸಾಹಿತ್ಯ ಸಂಗೀತ ಮುಂತಾದ ಕಾರ್ಯಕ್ಷೇತ್ರಗಳಲ್ಲಿ ಮೇಲ್ಮಟ್ಟದ ಪರಿಣಾಮವೆಲ್ಲ ಮನಸ್ಸಿನ ಏಕಾಗ್ರತೆಯ ಪ್ರತಿಫಲ. ಮೃಗಗಳಲ್ಲಿ ಏಕಾಗ್ರತೆ ಬಹಳ ಕಡಮೆ. ಯಾರು ಪ್ರಾಣಿಗಳನ್ನು ತರಬೇತು ಮಾಡುತ್ತಿರುವರೊ ಅವರಿಗೆ ಇದು ವೇದ್ಯವಾಗುವುದು. ಏಕೆಂದರೆ ಅವು ಹೇಳಿಕೊಟ್ಟದ್ದನ್ನು ಪದೇ ಪದೇ ಮರೆಯುತ್ತಿರುತ್ತವೆ. ಅವು ಬಹಳ ಕಾಲ ಯಾವುದರ ಮೇಲೂ ಮನಸ್ಸನ್ನು ಏಕಾಗ್ರ ಮಾಡಲಾರವು. ಇಲ್ಲಿದೆ ಮನುಷ್ಯನಿಗೂ ಪ್ರಾಣಿಗೂ ವ್ಯತ್ಯಾಸ. ಮನುಷ್ಯನಲ್ಲಿ ಹೆಚ್ಚು ಏಕಾಗ್ರತೆ ಇದೆ. ಮನುಷ್ಯರಲ್ಲಿ ಪರಸ್ಪರ ವ್ಯತ್ಯಾಸಕ್ಕೆ ಕಾರಣ ಅವರ ಏಕಾಗ್ರತಾ ಶಕ್ತಿ. ಅತಿ ಕೀಳು ಮತ್ತು ಅತಿ ಶ್ರೇಷ್ಠ ಮಾನವನನ್ನು ಹೋಲಿಸಿ ನೋಡಿ. ವ್ಯತ್ಯಾಸಕ್ಕೆ ಕಾರಣ ಅವರಲ್ಲಿರುವ ಏಕಾಗ್ರತೆಯ ಹೆಚ್ಚು ಕಡಿಮೆಯಲ್ಲಿದೆ.

ಎಲ್ಲರ ಮನಸ್ಸೂ ಕೆಲವು ಸಮಯ ಏಕಾಗ್ರವಾಗುವುದು. ನಾವು ಪ್ರೀತಿಸುವ ವಸ್ತುವಿನ ಮೇಲೆ ಮನಸ್ಸನ್ನು ಏಕಾಗ್ರಗೊಳಿಸುವೆವು. ನಾವು ಯಾವ ವಸ್ತುವಿನ ಮೇಲೆ ಮನಸ್ಸನ್ನು ಏಕಾಗ್ರಗೊಳಿಸುವೆವೊ ಅದನ್ನು ಪ್ರೀತಿಸುತ್ತೇವೆ. ತನ್ನ ಮಗುವಿನ ಮುದ್ದು ಮುಖವನ್ನು ಪ್ರೀತಿಸದ ತಾಯಿ ಯಾರಿರುವಳು? ಆ ಮಗುವೆ ಪ್ರಪಂಚದಲ್ಲೆಲ್ಲಾ ತನಗೆ ಅತಿ ಸುಂದರವಾಗಿರುವುದು. ಅವಳು ಅದನ್ನು ಪ್ರೀತಿಸುತ್ತಾಳೆ, ಏಕೆಂದರೆ ಆಕೆ ಮಗುವಿನ ಮೇಲೆ ಮನಸ್ಸನ್ನು ಏಕಾಗ್ರಮಾಡಿದಳು. ಪ್ರತಿಯೊಬ್ಬರೂ ಆ ಮಗುವಿನ ಮೇಲೆ ಮನಸ್ಸನ್ನು ಏಕಾಗ್ರಮಾಡಿದರೆ ಪ್ರತಿಯೊಬ್ಬರೂ ಅದನ್ನು ಪ್ರೀತಿಸುವರು. ಪ್ರತಿಯೊಬ್ಬರಿಗೂ ಅದು ಅತಿ ಸುಂದರವಾದ ಮಗುವಾಗುವುದು. ನಾವೆಲ್ಲ ಪ್ರೀತಿಸುವ ವಸ್ತುವಿನ ಮೇಲೆ ಮನಸ್ಸನ್ನು ಏಕಾಗ್ರಗೊಳಿಸುವೆವು. ನಾವು ಸುಂದರವಾದ ಗಾನವನ್ನು ಕೇಳಿದರೆ, ಮನಸ್ಸು ಅದರ ಮೇಲೆ ನಾಟುವುದು, ಅಲ್ಲಿಂದ ಮನಸ್ಸನ್ನು ತೆಗೆಯಲು ಆಗುವುದಿಲ್ಲ. ಯಾರು ಶಾಸ್ತ್ರೀಯ ಸಂಗೀತದ ಮೇಲೆ ಏಕಾಗ್ರಮಾಡುವರೊ ಅವರಿಗೆ ಲಘುಸಂಗೀತ ಹಿಡಿಸುವುದಿಲ್ಲ. ಹಾಗೆಯೇ ಲಘುಸಂಗೀತ ಪ್ರಿಯರಿಗೆ ಶಾಸ್ತ್ರೀಯ ಸಂಗೀತ ಹಿಡಿಸುವುದಿಲ್ಲ. ಯಾವ ಸಂಗೀತದಲ್ಲಿ ಒಂದಾದ ಮೇಲೊಂದು ಸ್ವರಗಳು ಬರುತ್ತವೆಯೋ ಅದು ಮನಸ್ಸನ್ನು ಸೂರೆಗೊಳ್ಳುವುದು. ಮಗುವಿಗೆ ಉತ್ಸಾಹವನ್ನು ಹೆಚ್ಚಿಸುವ ಸಂಗೀತದ ಮೇಲೆ ಪ್ರೀತಿ. ಏಕೆಂದರೆ ಅಷ್ಟು ಬೇಗ ಸ್ವರಗಳು ಓಡುತ್ತಿದ್ದರೆ ಮನಸ್ಸು ಅಲೆಯಲು ಅವಕಾಶವೇ ಇರುವುದಿಲ್ಲ. ಲಘು ಸಂಗೀತ ಪ್ರಿಯರಿಗೆ ಶಾಸ್ತ್ರೀಯ ಸಂಗೀತ ಹಿಡಿಸುವುದಿಲ್ಲ. ಏಕೆಂದರೆ ಅದು ಜಟಿಲವಾದುದು. ಅದನ್ನು ಅನುಸರಿಸಬೇಕಾದರೆ ಹೆಚ್ಚು ಏಕಾಗ್ರತೆ ಬೇಕಾಗುವುದು.

ಇಂತಹ ಏಕಾಗ್ರತೆಯ ವಿಷಯದಲ್ಲಿರುವ ದೊಡ್ಡ ತೊಂದರೆ ಏನೆಂದರೆ, ನಾವು ಮನಸ್ಸನ್ನು ನಿಗ್ರಹಿಸುವುದಿಲ್ಲ, ಅದು ನಮ್ಮನ್ನು ನಿಗ್ರಹಿಸುತ್ತದೆ. ನಮ್ಮಿಂದ ಹೊರಗೆ ಇರುವ ಯಾವುದೋ ಒಂದು ಮನಸ್ಸನ್ನು ತನ್ನೆಡೆಗೆ ಸೆಳೆದು ತನ್ನ ಸ್ವಾಧೀನದಲ್ಲಿಟ್ಟುಕೊಳ್ಳುವುದು. ನಾವು ಮಧುರವಾದ ಹಾಡನ್ನು ಕೇಳುವೆವು, ಇಲ್ಲವೆ ಸುಂದರವಾದ ಚಿತ್ರವನ್ನು ನೋಡುವೆವು. ಮನಸ್ಸು ಅಲ್ಲೇ ಸ್ತಬ್ದವಾಗುವುದು, ಅದನ್ನು ಅಲ್ಲಿಂದ ಕದಲಿಸಲಾಗುವುದಿಲ್ಲ.

ನೀವು ಇಷ್ಟಪಡುವ ವಿಷಯದ ಮೇಲೆ ನಾನು ಚೆನ್ನಾಗಿ ಮಾತನಾಡಿದರೆ ನಾನು ಹೇಳುವುದರ ಮೇಲೆ ನಿಮ್ಮ ಮನಸ್ಸು ಏಕಾಗ್ರವಾಗುವುದು. ನಿಮ್ಮಿಂದ ಮನಸ್ಸನ್ನು ಸೆಳೆದುಕೊಂಡು ನಿಮ್ಮ ಇಚ್ಛೆ ಇಲ್ಲದೇ ಇದ್ದರೂ ವಿಷಯದ ಮೇಲೆ ಮನಸ್ಸು ನಾಟುವಂತೆ ಮಾತನಾಡುತ್ತೇನೆ. ಹೀಗೆ ನಮ್ಮ ಅರಿವಿಲ್ಲದೆ ಮನಸ್ಸು ಸ್ಥಿರವಾಗುವುದು, ಏಕಾಗ್ರವಾಗುವುದು. ಹೀಗೆ ಆಗದೆ ವಿಧಿಯಿಲ್ಲ.

ಈಗಿನ ಪ್ರಶ್ನೆಯೆ, ನಾವು ಈ ಏಕಾಗ್ರತೆಯನ್ನು ರೂಢಿಸಲು ಸಾಧ್ಯವೆ, ನಾವು ಅದನ್ನು ನಮ್ಮ ಸ್ವಾಧೀನಕ್ಕೆ ತರಬಲ್ಲೆವೆ ಎಂಬುದು. ಯೋಗಿಗಳು ಇದು ಸಾಧ್ಯ ಎನ್ನುವರು. ಮನಸ್ಸನ್ನು ಸಂಪೂರ್ಣವಾಗಿ ನಮ್ಮ ಸ್ವಾಧೀನಕ್ಕೆ ತರಲು ಸಾಧ್ಯ ಎನ್ನುವರು ಯೋಗಿಗಳು. ನೈತಿಕ ದೃಷ್ಟಿಯಿಂದ ಮನಸ್ಸಿನ ಏಕಾಗ್ರತೆಯನ್ನು ರೂಢಿಸುವುದರಿಂದ ಒಂದು ಅಪಾಯವಿದೆ. ಅದೇ ಒಂದು ವಸ್ತುವಿನ ಮೇಲೆ ಮನಸ್ಸನ್ನು ಏಕಾಗ್ರಗೊಳಿಸಿ ಅದರಲ್ಲಿ ಆಸಕ್ತವಾದರೆ ಅದನ್ನು ಪುನಃ ನಮ್ಮ ಇಚ್ಛೆಯಂತೆ ಬೇರೆ ಕಡೆಗೆ ಬಿಡಲು ಸಾಧ್ಯವಿಲ್ಲ. ಇಂತಹ ಸ್ಥಿತಿಯಿಂದ ಬಹಳ ವ್ಯಥೆಯಾಗುವುದು. ನಮ್ಮ ದುಃಖಕ್ಕೆಲ್ಲ ಕಾರಣ ಅನಾಸಕ್ತಿಯ ಅಭಾವವೆ. ನಾವು ಏಕಾಗ್ರತೆಯನ್ನು ರೂಢಿಸಿದಂತೆ ಅನಾಸಕ್ತಿಯನ್ನೂ ರೂಢಿಸಬೇಕು. ನಾವು ಎಲ್ಲ ವಸ್ತುಗಳನ್ನು ಬಿಟ್ಟು ಒಂದು ವಸ್ತುವಿನ ಮೇಲೆ ಮನಸ್ಸನ್ನು ಏಕಾಗ್ರ ಮಾಡುವುದನ್ನು ಕಲಿತುಕೊಳ್ಳುವಂತೆ, ಒಂದು ಕ್ಷಣದಲ್ಲಿ ಮನಸ್ಸನ್ನು ಅಲ್ಲಿಂದ ತೆಗೆದು ಬೇರೆ ಯಾವುದಾದರೂ ವಸ್ತುವಿನ ಮೇಲೆ ಇಡುವುದನ್ನು ಕಲಿಯಬೇಕು. ಅಪಾಯದಿಂದ ಪಾರಾಗುವುದಕ್ಕೆ ನಾವು ಇವೆರಡನ್ನೂ ಒಟ್ಟಿಗೆ ಅಭ್ಯಾಸ ಮಾಡಬೇಕು.

ಮನಸ್ಸನ್ನು ಒಂದು ಕ್ರಮದ ಪ್ರಕಾರ ರೂಢಿಸುವುದೆಂದರೆ ಇದೇ. ನನ್ನ ದೃಷ್ಟಿಯಲ್ಲಿ ವಿದ್ಯಾಭ್ಯಾಸದ ಸಾರವೆ ಮನಸ್ಸಿನ ಏಕಾಗ್ರತೆ. ಬರಿಯ ವಿಷಯ ಸಂಗ್ರಹವಲ್ಲ. ನಾನು ಪುನಃ ವಿದ್ಯಾಭ್ಯಾಸವನ್ನು ಪ್ರಾರಂಭಿಸಬೇಕಾದರೆ, ಈ ವಿಷಯದಲ್ಲಿ ನನಗೇನಾದರೂ ಸ್ವಲ್ಪ ಸ್ವಾತಂತ್ರ್ಯವಿದ್ದರೆ ನಾನು ವಿಷಯಗಳನ್ನು ಕಲಿತುಕೊಳ್ಳುವುದಿಲ್ಲ. ನಾನು ಮೊದಲು ಮನಸ್ಸನ್ನು ಹೇಗೆ ಏಕಾಗ್ರಗೊಳಿಸುವುದು ಮತ್ತು ಅದನ್ನು ಹೇಗೆ ಹಿಂದಕ್ಕೆ ಸೆಳೆಯುವುದು ಎಂಬುದನ್ನು ಕಲಿತುಕೊಂಡು ಅನಂತರ ಒಂದು ಸರಿಯಾದ ಉಪಕರಣದಿಂದ ನನ್ನ ಇಚ್ಚಾಪ್ರಕಾರ ವಿಷಯಗಳನ್ನು ಸಂಗ್ರಹಿಸುತ್ತೇನೆ. ಮಗುವಿಗೆ ಏಕಾಗ್ರತೆಯೊಡನೆ ಅನಾಸಕ್ತಿಯನ್ನೂ ಕಲಿಸಬೇಕು.

ನನ್ನ ಅಭಿವೃದ್ದಿ ಯಾವಾಗಲೂ ಒಮ್ಮುಖವಾಗಿತ್ತು. ನಾನು ಅನಾಸಕ್ತಿಯಿಲ್ಲದೆ ಏಕಾಗ್ರತೆಯನ್ನು ರೂಢಿಸಿದೆ. ನಾನು ಜೀವನದಲ್ಲಿ ಅನುಭವಿಸಿದ ಅತಿ ದಾರುಣ\break ವ್ಯಥೆಗೆಲ್ಲ ಇದು ಕಾರಣವಾಗಿದೆ. ಈಗ ನನ್ನಲ್ಲಿ ಅನಾಸಕ್ತಿ ಇದೆ. ಆದರೆ ಇದನ್ನು ಅನಂತರ ಜೀವನದಲ್ಲಿ ಅಭ್ಯಾಸ ಮಾಡಬೇಕಾಯಿತು.

ನಾವು ವಸ್ತುಗಳ ಮೇಲೆ ಮನಸ್ಸನ್ನು ಇಡಬೇಕು. ಅವು ನಮ್ಮ ಮನಸ್ಸನ್ನು ಸೆಳೆಯಕೂಡದು. ನಾವು ಅನೇಕ ವೇಳೆ ಬಲಾತ್ಕಾರದಿಂದ ಏಕಾಗ್ರ ಮಾಡುವೆವು. ನಮ್ಮ ಮನಸ್ಸು ಅನೇಕ ವೇಳೆ ಬಾಹ್ಯ ವಸ್ತುಗಳ ಮೇಲೆ ಹೋಗಿ ನೆಲಸುವುದು. ಇದರ ಪರಿಣಾಮವಾಗಿ ಬಾಹ್ಯ ಆಕರ್ಷಣೆಯನ್ನು ನಮ್ಮ ಮನಸ್ಸು ಎದುರಿಸಲಾರದಾಗಿದೆ. ಮನಸ್ಸನ್ನು ನಿಗ್ರಹಿಸಬೇಕಾದರೆ, ಅದನ್ನು ನಮ್ಮ ಇಷ್ಟ ಬಂದ ಕಡೆ ಇಡಬೇಕಾದರೆ, ಅದಕ್ಕೆ ಪ್ರತ್ಯೇಕವಾದ ಅಭ್ಯಾಸ ಬೇಕು. ಬೇರೆ ಯಾವ ರೀತಿಯಲ್ಲಿಯೂ ಅದನ್ನು ಮಾಡಲಾಗುವುದಿಲ್ಲ. ಧರ್ಮದ ಅಧ್ಯಯನದಲ್ಲಿ ಮನೋನಿಗ್ರಹ ಅತ್ಯಾವಶ್ಯಕ. ಇಲ್ಲಿ ಮನಸ್ಸನ್ನು ಅಂತರ್ಮುಖವಾಗಿ ಮಾಡಬೇಕು.

ಮನಸ್ಸನ್ನು ನಿಗ್ರಹಿಸಬೇಕಾದರೆ ಮೊದಲು ಪ್ರಾಣಾಯಾಮದಿಂದ ಪ್ರಾರಂಭಿಸಬೇಕು. ನೀವು ಸರಿಯಾಗಿ ಉಸಿರಾಡಿದರೆ ಅದು ದೇಹವನ್ನು ಒಂದು ಹೊಂದಿಕೆಯಾದ ಸ್ಥಿತಿಗೆ ತರುವುದು. ಆಗ ಮನಸ್ಸನ್ನು ತಲುಪುವುದು ಸುಲಭ. ಪ್ರಾಣಾಯಾಮಕ್ಕೆ ಪ್ರಥಮ ಸಾಧನೆಯೇ ಆಸನ. ಯಾವ ಸ್ಥಿತಿಯಲ್ಲಿ ಒಬ್ಬನು ಸುಖವಾಗಿ, ಸ್ಥಿರವಾಗಿ ಕುಳಿತುಕೊಳ್ಳಬಲ್ಲನೋ ಅದೇ ಆಸನ. ಬೆನ್ನುಮೂಳೆ ನೇರವಾಗಿ ಆತಂಕವಿಲ್ಲದೆ ಇರಬೇಕು. ದೇಹದ ಭಾರವೆಲ್ಲ ಪಕ್ಕೆಲುಬುಗಳ ಮೇಲೆ ಇರಲಿ, ಬಲಾತ್ಕಾರವಾಗಿ ಮನಸ್ಸನ್ನು ನಿಗ್ರಹಿಸಲು ಯತ್ನಿಸಬೇಡಿ. ಇಲ್ಲಿ ಸುಮ್ಮನೆ ಸರಳವಾಗಿ ಉಸಿರಾಡುವುದು ಮಾತ್ರ ಆವಶ್ಯಕ. ಮನಸ್ಸನ್ನು ನಿಗ್ರಹಿಸಬೇಕು. ಉಳಿದ ಕಸರತ್ತಿನಿಂದ ಏನೂ ಪ್ರಯೋಜನವಿಲ್ಲ. ಅದನ್ನು ಅಭ್ಯಾಸ ಮಾಡಬೇಡಿ.

ಮನಸ್ಸು ದೇಹದ ಮೇಲೆ ಕೆಲಸಮಾಡುವುದು, ಪುನಃ ದೇಹ ಮನಸ್ಸಿನ ಮೇಲೆ ಕೆಲಸಮಾಡುವುದು, ಅವು ಒಂದರ ಮೇಲೆ ಮತ್ತೊಂದು ಪ್ರತಿಕ್ರಿಯೆಯನ್ನು ಉಂಟುಮಾಡುತ್ತವೆ. ನಮ್ಮ ಮನಸ್ಸಿನ ಪ್ರತಿಯೊಂದು ಸ್ಥಿತಿಯೂ ದೇಹದಲ್ಲಿ ಯಾವುದೊ ಒಂದು ಬದಲಾವಣೆಯನ್ನು ಉಂಟುಮಾಡುವುದು. ದೇಹ ಮನಸ್ಸು ಎರಡೂ ಒಂದೇ ಎಂದು ಭಾವಿಸಿದರೂ ಒಂದೇ, ಬೇರೆ ಬೇರೆ ಎಂದು ಭಾವಿಸಿದರೂ ಒಂದೇ. ವ್ಯತ್ಯಾಸವೇನೂ ಇರುವುದಿಲ್ಲ. ದೇಹವೇ ಮನಸ್ಸಿನ ಸ್ಥೂಲಸ್ವರೂಪ. ಮನಸ್ಸೇ ದೇಹದ ಸೂಕ್ಷ್ಮಸ್ವರೂಪ. ಇವೆರಡೂ ಒಂದರಮೇಲೊಂದು ಪ್ರತಿಕ್ರಿಯೆಯನ್ನು ಬೀರುತ್ತಿರುತ್ತವೆ. ಮನಸ್ಸು ಪದೇಪದೇ ದೇಹವಾಗುತ್ತಿರುವುದು. ಮನಸ್ಸನ್ನು ನಿಗ್ರಹಿಸಬೇಕಾದರೆ ದೇಹದ ಮೂಲಕ ಹೋಗುವುದು ಸುಲಭ. ಮನಸ್ಸಿಗಿಂತ ದೇಹವನ್ನು ನಿಗ್ರಹಿಸುವುದು ಸುಲಭ.

ಉಪಕರಣ ಸೂಕ್ಷ್ಮವಾಗಿದ್ದಷ್ಟೂ ಶಕ್ತಿ ಹೆಚ್ಚು. ಮನಸ್ಸು ದೇಹಕ್ಕಿಂತ ಸೂಕ್ಷ್ಮ ಮತ್ತು ಬಲಾಢ್ಯವಾಗಿಯೂ ಇದೆ. ಆದಕಾರಣವೇ ದೇಹದ ಮೂಲಕ ಪ್ರಾರಂಭಿಸುವುದೇ ಸುಲಭವಾಗಿದೆ.

ಪ್ರಾಣಾಯಾಮವೆಂದರೆ ದೇಹದ ಮೂಲಕ ಮನಸ್ಸನ್ನು ನಿಗ್ರಹಿಸುವ ಶಾಸ್ತ್ರ. ಮೊದಲು ದೇಹದ ಮೇಲೆ ನಮಗೆ ಒಂದು ಸ್ವಾಧೀನತೆ ಬರುವುದು, ಅನಂತರ ದೇಹದ ಸೂಕ್ಷ್ಮ ಚಲನವಲನಗಳು ನಮಗೆ ವೇದ್ಯವಾಗುವುವು. ಸೂಕ್ಷ್ಮವಾದ ದೈಹಿಕ ಕ್ರಿಯೆಗಳು ನಮಗೆ ಅರಿವಾಗುವುವು. ಕೊನೆಗೆ ಮನಸ್ಸಿನ ಸಮೀಪಕ್ಕೆ ಹೋಗುವೆವು. ದೇಹದ ಸೂಕ್ಷ್ಮಚಲನೆಗಳು ನಮಗೆ ಅರಿವಾದಂತೆ ಅವು ನಮ್ಮ ನಿಗ್ರಹಕ್ಕೆ ಬರುವುವು. ಅನಂತರ ಮನಸ್ಸು ದೇಹದ ಮೇಲೆ ಹೇಗೆ ಕೆಲಸಮಾಡುತ್ತದೆ ಎಂಬುದು ಅರ್ಥವಾಗುವುದು. ಮನಸ್ಸಿನಲ್ಲಿ ಅದರ ಒಂದು ಭಾಗ ಇನ್ನೊಂದರೊಂದಿಗೆ ಹೇಗೆ ವ್ಯವಹರಿಸುತ್ತಿದೆ ಮತ್ತು ಮನಸ್ಸು ಹೇಗೆ ನರಗಳ ಕೇಂದ್ರವನ್ನು ತನ್ನ ಸ್ವಾಧೀನಕ್ಕೆ ತೆಗೆದುಕೊಳ್ಳುತ್ತಿದೆ ಎಂಬುದನ್ನು ಗಮನಿಸಬಹುದು. ಏಕೆಂದರೆ ಮನಸ್ಸು ನರಗಳ ಕೇಂದ್ರವನ್ನು ನಿಗ್ರಹಿಸಿ ಅದನ್ನು ಆಳುವುದು. ಬೇರೆ ಬೇರೆ ನರಗಳ ಸಮುದಾಯದ ಮೇಲೆ ಮನಸ್ಸು ಕೆಲಸ ಮಾಡುವುದು ಗೋಚರಿಸುವುದು.

ಪ್ರಾಣಾಯಾಮದಿಂದ ಮೊದಲು ದೇಹದ ಸ್ಥೂಲಭಾಗವನ್ನು, ಅನಂತರ ಸೂಕ್ಷ್ಮಭಾಗವನ್ನು ನಮ್ಮ ಸ್ವಾಧೀನಕ್ಕೆ ತಂದರೆ ಕ್ರಮೇಣ ಮನಸ್ಸು ನಮ್ಮ ನಿಗ್ರಹಕ್ಕೆ ಬರುವುದು.

ಪ್ರಾಣಾಯಾಮದ ಪ್ರಥಮ ಅಭ್ಯಾಸಗಳಿಂದ ಯಾವ ಅಪಾಯವೂ ಇಲ್ಲ,\break ಇದರಿಂದ ಪ್ರಯೋಜನವಾಗುವುದು. ಇದರಿಂದ ನಿಮ್ಮ ಆರೋಗ್ಯ ಉತ್ತಮವಾಗುವುದು, ದೇಹಸ್ಥಿತಿಯೂ ಉತ್ತಮಗೊಳ್ಳುವುದು. ಇತರ ಅಭ್ಯಾಸಗಳನ್ನು ಜಾಗರೂಕತೆಯಿಂದ, ನಿಧಾನವಾಗಿ ಮಾಡಬಹುದು.

