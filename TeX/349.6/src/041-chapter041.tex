
\chapter[ಧರ್ಮ ಎಂದರೇನು?]{ಧರ್ಮ ಎಂದರೇನು?\protect\footnote{\engfoot{C.W, Vol. I, P. 333}}}

ದೊಡ್ಡದೊಂದು ರೈಲುಗಾಡಿ ಕಂಬಿಯ ಮೇಲೆ ಉರುಳಿಕೊಂಡು ಹೋಗುವುದು. ರೈಲು ಕಂಬಿಯ ಮೇಲೆ ಸಂಚರಿಸುತ್ತಿದ್ದ ಕೀಟವೊಂದು ಸ್ವಲ್ಪ ಅತ್ತ ಚಲಿಸಿ ತನ್ನ ಪ್ರಾಣವನ್ನು ಉಳಿಸಿಕೊಳ್ಳುವುದು. ಕ್ಷಣಾರ್ಧದಲ್ಲಿ ನುಚ್ಚುನೂರಾಗುತ್ತಿದ್ದ ಈ ಕೀಟ ಕೆಲಸಕ್ಕೆ ಬಾರದುದಾದರೂ ಒಂದು ಜೀವಿ. ಉರುಳುತ್ತಿದ್ದ ಅದು ದೊಡ್ಡದಾಗಿದ್ದರೂ ಅದ್ಭುತವಾಗಿದ್ದರೂ ಅದು ಕೇವಲ ಒಂದು ಎಂಜಿನ್, ಒಂದು ಯಂತ್ರ. ಒಂದಕ್ಕೆ ಜೀವವಿದೆ. ಮತ್ತೊಂದಕ್ಕೆ ಬೇಕಾದಷ್ಟು ಶಕ್ತಿ ಸಾಮರ್ಥ್ಯ, ವೇಗಗಳಿದ್ದರೂ ಅದೊಂದು ನಿರ್ಜೀವವಾದ ಯಂತ್ರ, ಮಾನವ ತನ್ನ ಅನುಕೂಲಕ್ಕಾಗಿ ಸೃಷ್ಟಿಸಿದ ಯಂತ್ರವ್ಯೂಹ. ಕಂಬಿಯ ಮೇಲೆ ಚಲಿಸಿದ ಆ ಕ್ಷುದ್ರ ಕೀಟವು, ಆ ಹೊಗೆಬಂಡಿ ಸ್ವಲ್ಪ ಸೋಂಕಿದ್ದರೂ ಸಾಕು, ನಾಶವಾಗಿ ಹೋಗುತ್ತಿತ್ತು. ಆದರೂ ಅದರೊಂದಿಗೆ ಹೋಲಿಸಿದರೆ ಕೀಟ ಒಂದು ಅದ್ಭುತ ಜೀವಿ. ಅದು ಅನಂತದ ಒಂದು ಅಂಶ. ಆದಕಾರಣವೆ ಅದು ಎದುರಿಗಿರುವ ಪ್ರಚಂಡ ಯಂತ್ರಶಕ್ತಿಗಿಂತ ಅದ್ಭುತವಾದುದು. ಇದಕ್ಕೆ ಕಾರಣವೇನು? ಚೇತನಕ್ಕೂ ಜಡಕ್ಕೂ ಇರುವ ವ್ಯತ್ಯಾಸ ನಮಗೆ ಹೇಗೆ ಗೊತ್ತು? ಯಂತ್ರಗಳು, ಅವನ್ನು ಯಾವ ಉದ್ದೇಶದಿಂದ ಮಾನವ ರಚಿಸಿದನೊ ಅದರಂತೆ ಕೆಲಸ ಮಾಡುವುವು. ಅವು ಸ್ವಂತ ಶಕ್ತಿಯಿಂದ ಪ್ರೇರಿತವಾಗಿ ಕೆಲಸಮಾಡುವುದಿಲ್ಲ. ಹಾಗಾದರೆ ಚೇತನಕ್ಕೂ ಜಡಕ್ಕೂ ನಾವು ಹೇಗೆ ವ್ಯತಾಸವನ್ನು ಕಲ್ಪಿಸುವುದು? ಚೇತನದಲ್ಲಿ ಸ್ವಾತಂತ್ರ್ಯವಿದೆ, ಬುದ್ದಿ ಇದೆ. ಜಡದಲ್ಲಿ ಎಲ್ಲಾ ಬದ್ಧವಾಗಿದೆ; ಅಲ್ಲಿ ಸ್ವಾತಂತ್ರ್ಯಕ್ಕೆ ಎಡೆಯೇ ಇಲ್ಲ. ಏಕೆಂದರೆ ಅಲ್ಲಿ ಚೈತನ್ಯವೇ ಇಲ್ಲ. ಕೇವಲ ಯಂತ್ರಗಳಿಂದ ನಮ್ಮನ್ನು ಬೇರ್ಪಡಿಸುವ ಸ್ವಾತಂತ್ರ್ಯವನ್ನು ಪಡೆಯುವುದಕ್ಕಾಗಿಯೇ ನಾವೆಲ್ಲ ಹೋರಾಡುತ್ತಿರುವೆವು. ನಮ್ಮ ಹೋರಾಟದ ಗುರಿಯೆಲ್ಲ ಹೆಚ್ಚು ಸ್ವಾತಂತ್ರ್ಯವನ್ನು ಪಡೆಯುವುದಾಗಿದೆ. ಏಕೆಂದರೆ ಪೂರ್ಣವಾದ ಸ್ವಾತಂತ್ರ್ಯದಲ್ಲಿ ಮಾತ್ರ ನಮ್ಮ ಗುರಿಯ ಪರಿಪೂರ್ಣತೆ ಇರುವುದು. ನಮಗೆ ಗೊತ್ತಿರಲಿ ಬಿಡಲಿ, ಈ ಸ್ವಾತಂತ್ರ್ಯ ಸಾಧನೆಯೇ ಎಲ್ಲಾ ಪೂಜೆಯ ಮೂಲದಲ್ಲಿರುವುದು.

ಪ್ರಪಂಚದಲ್ಲೆಲ್ಲ ನಡೆಯುತ್ತಿರುವ ಹಲವು ವಿಧದ ಪೂಜೆಗಳನ್ನು ನಾವು ವಿಮರ್ಶಿಸಿ ನೋಡಿದರೆ, ಮಾನವರಲ್ಲಿ ಶುದ್ಧ ಅನಾಗರಿಕರಾದವರು ದೆವ್ವಗಳು, ಭೂತಗಳು, ಪ್ರೇತಗಳು, ತಮ್ಮ ಗತಿಸಿದ ಪಿತೃಗಳು ಇವರನ್ನು ಪೂಜಿಸುವುದನ್ನು ನಾವು ನೋಡುತ್ತೇವೆ. ಸರ್ಪಪೂಜೆ, ಆಯಾಯ ಕೋಮುಗಳ ದೇವರ ಪೂಜೆ, ಪಿತೃಪೂಜೆ ಇವೆಲ್ಲ ಏತಕ್ಕೆ? ಇವರೆಲ್ಲ ತಮಗೆ ಅರಿಯದ ರೀತಿಯಲ್ಲಿ ತಮಗಿಂತ ಉತ್ತಮರೆಂದೂ, ಶಕ್ತಿಶಾಲಿಗಳಾಗಿರುವರೆಂದೂ, ತಮ್ಮ ಸ್ವಾತಂತ್ರ್ಯಕ್ಕೆ ಮಿತಿಯನ್ನು ಉಂಟುಮಾಡುವರೆಂದೂ ಮಾನವರು ಭಾವಿಸುವರು. ತಮ್ಮನ್ನು ಹಿಂಸೆಗೆ ಗುರಿಮಾಡದಿರಲೆಂದು ಅವರನ್ನು ತೃಪ್ತಿ ಪಡಿಸಲೆತ್ನಿಸುವರು, ಅಂದರೆ ಅವರಿಂದ ಸ್ವಾತಂತ್ರ್ಯವನ್ನು ಪಡೆಯಲು ಇಚ್ಛಿಸುವರು. ಮಾನವ ಯಾವುದನ್ನು ತಾನು ಸ್ವತಃ ಸಂಪಾದಿಸಬೇಕೊ ಅದನ್ನು ತನಗಿಂತ ಹೆಚ್ಚು ಬಲಿಷ್ಠರಾದ ಇಂತಹವರ ಮೂಲಕ ವರವಾಗಿ ಪಡೆಯಲು ಯತ್ನಿಸುವನು.

ಒಟ್ಟಿನಲ್ಲಿ ಯಾವುದೋ ಒಂದು ಅದ್ಭುತವನ್ನು ಈ ಪ್ರಪಂಚ ನಿರೀಕ್ಷಿಸುತ್ತಿರುವಂತೆ ಕಾಣುವುದು. ಈ ನಿರೀಕ್ಷಣೆ ನಮ್ಮನ್ನು ಎಂದಿಗೂ ಬಿಟ್ಟು ಹೋಗುವುದಿಲ್ಲ. ನಾವು ಇದಕ್ಕೆ ವಿರೋಧವಾಗಿ ಎಷ್ಟೇ ಪ್ರಯತ್ನಿಸಿದರೂ, ಅದ್ಭುತವಾದುದನ್ನು, ವಿಚಿತ್ರವಾದುದನ್ನು ಅರಸುತ್ತಿರುವೆವು. ಮನಸ್ಸು ಎಂದರೇನು, ಜೀವನ ಎಂದರೇನು, ಅದರ ರಹಸ್ಯವೇನು, ಎಂದು ಬಿಡುವಿಲ್ಲದೆ ವಿಚಾರಿಸುವುದಲ್ಲದೆ ಮತ್ತೆ ಇನ್ನೇನು? ಅವಿದ್ಯಾವಂತರು ಮಾತ್ರ ಇದನ್ನು ಅರಸುತ್ತಿರುವರು ಎನ್ನಬಹುದು ನಾವು. ಆದರೆ ಅವರು ತಾನೆ ಏತಕ್ಕೆ ಅರಸಬೇಕು ಎಂಬ ಪ್ರಶ್ನೆಯೇನೋ ಉಳಿಯುವುದು. ಯೆಹೂದ್ಯರು ಪವಾಡವನ್ನು ನಿರೀಕ್ಷಿಸುತ್ತಿದ್ದರು. ಸಾವಿರಾರು ವರ್ಷಗಳಿಂದ ಪ್ರಪಂಚವೆಲ್ಲ ಅದನ್ನು ನಿರೀಕ್ಷಿಸುತ್ತಿದೆ. ಪ್ರಪಂಚದಲ್ಲೆಲ್ಲ ದೊಡ್ಡ ಅತೃಪ್ತಿ ತಲೆದೋರಿದೆ. ನಾವು ಒಂದು ಆದರ್ಶವನ್ನು ನಿರ್ಧರಿಸುವೆವು. ನಾವಿನ್ನೂ ಅದರೆಡೆಗೆ ಅರ್ಧ ಕೂಡ ಹೋಗುವುದಕ್ಕೆ ಮಾರ್ಗವಿಲ್ಲ. ಆಗಲೇ ಬೇರೆ ಒಂದು ಆದರ್ಶವನ್ನು ಆರಿಸಿಕೊಳ್ಳುವೆವು. ನಾವು ಯಾವುದೋ ಒಂದು ಗುರಿಯನ್ನು ಸೇರಬೇಕೆಂದು ಹೋರಾಡುವೆವು. ಅನಂತರ ಅದು ನಮಗೆ ಬೇಕಾಗಿಲ್ಲ ಎಂದು ನಿರ್ಧರಿಸುವೆವು, ಪದೇಪದೇ ಈ ಅತೃಪ್ತಿ ನಮ್ಮಲ್ಲಿ ಬರುತ್ತಿದೆ. ಬರಿಯ ಅತೃಪ್ತಿಯೇ ಇದ್ದರೆ ಮನಸ್ಸಿನ ಸ್ಥಿತಿ ಏನಾಗುವುದು? ಈ ಪ್ರಪಂಚದಲ್ಲೆಲ್ಲ ಇರುವ ಅತೃಪ್ತಿಯ ಅರ್ಥವೇನು? ಇದರ ಅರ್ಥವೆ, ಸ್ವಾತಂತ್ರ್ಯವೇ ಮಾನವನ ನಿತ್ಯಗುರಿಯಾಗಿದೆ ಎಂಬುದು. ಅವನು ಸದಾ ಅದನ್ನು ಅರಸುತ್ತಿರುವನು. ಅವನ ಇಡೀ ಬಾಳೆಲ್ಲ ಅದನ್ನು ಸಾಧಿಸುವುದಕ್ಕೆ ನಡೆಸುತ್ತಿರುವ ಒಂದು ಅವಿರತ ಹೋರಾಟ. ಮಗು ಹುಟ್ಟಿದೊಡನೆ ನಿಯಮವನ್ನು ವಿರೋಧಿಸುವುದು. ಅದರ ಮೊದಲ ಮಾತೇ ಅಳು. ತಾನು ಬಿದ್ದಿರುವ ಬಂಧನವನ್ನು ವಿರೋಧಿಸುವುದು. ಈ ಸ್ವಾತಂತ್ರ್ಯದ ಇಚ್ಛೆಯೇ ನಿತ್ಯಮುಕ್ತನಾದ ಒಂದು ವ್ಯಕ್ತಿಯ ಭಾವನೆಯನ್ನು ಸೃಷ್ಟಿಸುವುದು. ಭಗವಂತನ ಭಾವನೆ ಮಾನವನ ಸ್ವಭಾವದೊಂದಿಗೆ ಬೆಳೆದು ಬಂದ ಒಂದು ಮೂಲ ಭಾವನೆ. ವೇದಾಂತದಲ್ಲಿ ಮನಸ್ಸಿಗೆ ಸಾಧ್ಯವಾಗುವ ಅತಿಶ್ರೇಷ್ಠ ಭಗವಂತನ ಭಾವನೆಯೇ ಸತ್–ಚಿತ್–ಆನಂದ, ಇದೇ ಜ್ಞಾನದ ಸಾರ. ಸ್ವಭಾವತಃ ಇದೇ ಆನಂದದ ಸಾರವೂ ಆಗಿದೆ. ಬಹಳ ಕಾಲದಿಂದ ನಾವು ಪ್ರಕೃತಿ ನಿಯಮವನ್ನು ಅನುಸರಿಸುತ್ತ ಈ ಅಂತರ್ವಾಣಿಯು ಕೇಳಿಸದಂತೆ ಮಾಡಲು, ಮಾನವ ಸ್ವಭಾವವನ್ನು ತೆಪ್ಪಗಿರಿಸಲು ಪ್ರಯತ್ನಿಸುತ್ತಿರುವೆವು. ಆದರೆ ಮಾನವನಲ್ಲಿ ಪ್ರಕೃತಿಯ ನಿಯಮವನ್ನು ವಿರೋಧಿಸುವ ಒಂದು ಹುಟ್ಟುಗುಣವಿದೆ. ನಮಗೆ ಇದರ ಅರ್ಥವೇನೂ ಗೊತ್ತಾಗದಿರಬಹುದು. ಆದರೆ ಅಲ್ಲಿ ಮಾನವ ತನಗೆ ಅರಿಯದೇ ಆಧ್ಯಾತ್ಮಿಕದೊಂದಿಗೆ ಹೋರಾಡುತ್ತಿರುವನು. ಮನಸ್ಸಿನ ಕೀಳುಸ್ವಭಾವ ತನ್ನ ಉತ್ತಮ ಸ್ವಭಾವದೊಂದಿಗೆ ಹೋರಾಡುತ್ತಿರುವುದು. ಈ ಹೋರಾಟವೇ ಮಾನವನ ಪ್ರತ್ಯೇಕ ಜೀವನವನ್ನು ರಕ್ಷಿಸಲು ಯತ್ನಿಸುವುದು. ಇದನ್ನೇ ನಾವು ವ್ಯಕ್ತಿತ್ವ ಎನ್ನುವುದು.

ನರಕಲೋಕದ ಭಾವನೆ ಕೂಡ, ನಾವು ಹುಟ್ಟು ದಂಗೆಕೋರರು ಎಂಬ ಅದ್ಭುತ ಸತ್ಯವನ್ನು ಸಾರುವುದು. ಜೀವನೋದಯದಲ್ಲಿಯೇ ನಾವು ಪ್ರಕೃತಿಯನ್ನು ವಿರೋಧಿಸಿ, `ನಾನು ಯಾವ ನಿಯಮಕ್ಕೂ ದಾಸನಾಗುವುದಿಲ್ಲ' ಎಂದು ಅಳುತ್ತೇವೆ. ನಾವು ನಿಯಮವನ್ನು ಪಾಲಿಸುತ್ತಿರುವವರೆಗೆ ಒಂದು ಯಂತ್ರ ಮಾತ್ರದಂತೆ ಇರುವೆವು ಮತ್ತು ಪ್ರಪಂಚ\break ಎಂದಿನಂತೆ ಮುಂದೆ ಸಾಗುವುದು. ಅದನ್ನು ಧ್ವಂಸ ಮಾಡಲಾರೆವು. ನಿಯಮಗಳೇ ಮನುಷ್ಯನ ಸ್ವಭಾವವಾಗುವುವು. ಪ್ರಕೃತಿ ಬಂಧನದಿಂದ ಪಾರಾಗಿ ಮುಕ್ತರಾಗುವ\break ಹೋರಾಟದಲ್ಲಿ ಜೀವನದ ಮೇಲ್ಮಟ್ಟದ ಸ್ವಭಾವವನ್ನು ನೋಡುವೆವು. “ಮುಕ್ತಿ, ಓ ಮುಕ್ತಿ; ಮುಕ್ತಿ, ಓ ಮುಕ್ತಿ" ಎಂಬುದೇ ಆತ್ಮನ ಪಲ್ಲವಿಯಾಗಿದೆ. ಆದರೆ ಪಾಪ, ಬಂಧನವೆ – ಪ್ರಕೃತಿಯ ಪಂಜರದಲ್ಲಿ ಬದ್ಧವಾಗಿರುವುದೇ ಅದರ ದುರದೃಷ್ಟ!

\vskip 2pt

ಪವಾಡವನ್ನು ಮಾಡುವುದಕ್ಕಾಗಿ ಸರ್ಪ ಭೂತ ಪ್ರೇತಗಳ ಆರಾಧನೆ, ಹಲವು ಜಾತಿಮತಗಳು ಇವೆಲ್ಲ ಏತಕ್ಕೆ ಇರಬೇಕು? ಇದರಲ್ಲೆಲ್ಲ ಒಂದು ಜೀವವಿದೆ, ಒಂದು ವ್ಯಕ್ತಿತ್ವ ಇದೆ ಎಂದು ಏತಕ್ಕೆ ಹೇಳಬೇಕು? ಈ ಎಲ್ಲ ಅನ್ವೇಷಣೆಯಲ್ಲಿ ಜೀವನವನ್ನು ತಿಳಿದುಕೊಳ್ಳುವ ಈ ಪ್ರಯತ್ನದಲ್ಲಿ ಒಂದು ಅರ್ಥವಿರಬೇಕು. ಇದೆಲ್ಲ ನಿಷ್ಪ್ರಯೋಜಕವಲ್ಲ, ಇವಕ್ಕೆಲ್ಲ ಅರ್ಥವಿಲ್ಲದೆ ಇಲ್ಲ. ಇವೆಲ್ಲ ಮುಕ್ತನಾಗುವುದಕ್ಕೆ ಮಾನವ ಅವಿರತವಾಗಿ ಮಾಡುತ್ತಿರುವ ಹೋರಾಟ. ನಾವು ಈಗ ಯಾವುದನ್ನು ವಿಜ್ಞಾನ ಎನ್ನುವೆವೊ ಅದು ಸಾವಿರಾರು ವರುಷಗಳಿಂದ ಸ್ವಾತಂತ್ರ್ಯವನ್ನು ಗಳಿಸುವುದಕ್ಕೆ ಮಾಡುತ್ತಿರುವ ಪ್ರಯತ್ನ. ಜನರು ಸ್ವಾತಂತ್ರ್ಯವನ್ನು ಕೇಳುವರು. ಆದರೂ ಪ್ರಕೃತಿಯಲ್ಲಿ ಸ್ವಾತಂತ್ರ್ಯವಿಲ್ಲ. ಎಲ್ಲವೂ ನಿಯಮಗಳಿಂದ ಬದ್ದವಾಗಿವೆ. ಆದರೂ ಈ ಹೋರಾಟ ಮುಂದೆ ಸಾಗುವುದು. ಸೂರಿನಿಂದ ಒಂದು ಕಣದವರೆಗೆ ಎಲ್ಲವೂ ಒಂದು ಪ್ರಕೃತಿ ನಿಯಮದ ವಜ್ರಮುಷ್ಠಿಯಲ್ಲಿವೆ. ಮನುಷ್ಯನಿಗೂ ಸ್ವಾತಂತ್ರ್ಯವಿಲ್ಲ. ಆದರೆ ನಾವು ಅದನ್ನು ನಂಬುವುದಿಲ್ಲ. ನಾವು ಬಹಳ ಕಾಲದಿಂದ ಈ ಪ್ರಕೃತಿ ನಿಯಮಗಳನ್ನು ಪರೀಕ್ಷಿಸುತ್ತಿರುವೆವು. ಆದರೂ ನಾವು, ಮಾನವನು ನಿಯಮಕ್ಕೆ ಅಧೀನನಾಗಿರುವನು ಎಂಬುದನ್ನು ನಂಬಲಾರೆವು, ಇಲ್ಲ ನಂಬಲಿಚ್ಚಿಸುವುದಿಲ್ಲ. ಆತ್ಮ `ಮುಕ್ತಿ ಮುಕ್ತಿ' ಎಂದು ಸದಾಕಾಲವೂ ಒರಲುತ್ತಿರುವುದು. ಭಗವಂತ ನಿತ್ಯ ಮುಕ್ತ ಎಂಬ ಭಾವನೆ ಇರುವಾಗ ಮಾನವ ಎಂದೆಂದಿಗೂ ಬಂಧನದಲ್ಲಿ ಇರಲಾರ. ಅದೊಂದು ಕಡು ಕಷ್ಟಕರವಾದ ಹೋರಾಟ ಎಂದು ಭಾವಿಸದೆ ಇರಲಾರ. ಸದಾಕಾಲದಲ್ಲಿಯೂ ಅವನು ಮುಂದೆ ಮುಂದೆ ಹೋಗುತ್ತಿರಬೇಕು. ಮಾನವ, “ನಾನು ಹುಟ್ಟು ಗುಲಾಮ, ಬದ್ಧ. ಆದರೂ ಚಿಂತೆಯಿಲ್ಲ, ಪ್ರಕೃತಿಯ ಬಂಧನಕ್ಕೆ ಬೀಳದ ಒಬ್ಬನಿರುವನು. ಅವನು ಸ್ವತಂತ್ರ, ಪ್ರಕೃತಿಗೆ ಒಡೆಯನಾಗಿರುವನು'' ಎಂದು ಭಾವಿಸುವನು.

\vskip 2pt

ಬಂಧನದ ಭಾವನೆಯಷ್ಟೇ ಮುಖ್ಯವಾದುದು ಮತ್ತು ಮಾನವನಲ್ಲಿ ಮೂಲಭೂತವಾದುದು ದೇವರ ಭಾವನೆ. ಎರಡೂ ಸ್ವಾತಂತ್ರ್ಯದ ಭಾವನೆಯ ಪರಿಣಾಮಗಳು.\break ಸ್ವಾತಂತ್ರ್ಯದ ಭಾವನೆಯಿಲ್ಲದೆ ಇದ್ದರೆ ಒಂದು ಸಸಿಯಲ್ಲಿಯೂ ಜೀವವಿರಲಾರದು. ಒಂದು ಸಸಿಯಲ್ಲಿ ಅಥವಾ ಒಂದು ಕೀಟದಲ್ಲಿ ಜೀವನ ತಾನೊಂದು ವ್ಯಕ್ತಿ ಎನ್ನುವ ಹಂತಕ್ಕೆ ಏರಬೇಕಾಗಿದೆ. ಸಸಿಯು ಪ್ರಕೃತಿ ಹೇಳಿದಂತೆ ಕೇಳದೆ, ತನ್ನದೇ ಆದ ಆಕಾರ ಮತ್ತು ಸ್ವಭಾವಕ್ಕೆ ತಕ್ಕಂತೆ ಬೆಳೆಯುವುದರಲ್ಲಿ ಈ ಸ್ವಾತಂತ್ರ್ಯದ ಭಾವನೆ ಅಗೋಚರವಾಗಿ ಕೆಲಸಮಾಡುತ್ತಿದೆ. ಪ್ರತಿಯೊಂದು ಹಂತದಲ್ಲಿಯೂ ಅದನ್ನು ತನ್ನ ನಿಯಮದಲ್ಲಿ ಇಟ್ಟುಕೊಳ್ಳುವ ಪ್ರಕೃತಿಯ ಪ್ರಯತ್ನವು ಸ್ವಾತಂತ್ರ್ಯದ ಭಾವಕ್ಕೆ ವಿರೋಧವಾದುದು. ಬಾಹ್ಯ ಜಗತ್ತಿನ ಭಾವನೆ\break ಮುಂದುವರಿದಂತೆಲ್ಲ ಸ್ವಾತಂತ್ರ್ಯದ ಭಾವನೆಯೂ ಮುಂದುವರಿಯುವುದು. ಆದರೂ ಹೋರಾಟ ಮುಂದೆ ಸಾಗುವುದು. ಜಾತಿಮತಗಳ ಕಲಹಗಳನ್ನು ನಾವು ಕೇಳುತ್ತಿರುವೆವು. ಆದರೂ ಜಾತಿಮತಗಳು ಆವಶ್ಯಕ, ಅವು ಸರಿ, ಅವು ಇದ್ದೇ ತೀರಬೇಕು. ಬಂಧನದ ಸಂಕೋಲೆ ಹೆಚ್ಚುತ್ತಿದ್ದಂತೆ ಹೋರಾಟವೂ ಹೆಚ್ಚುವುದು. ನಾವೆಲ್ಲ ಒಂದೇ ಗುರಿಯನ್ನು ಸೇರುವುದಕ್ಕೆ ಶ್ರಮಿಸುತ್ತಿರುವೆವು ಎಂಬುದನ್ನು ಅರಿತರೆ ಯಾವ ವೈಮನಸ್ಯವೂ ಇರುವುದಿಲ್ಲ.

\vskip 2pt

ಸ್ವಾತಂತ್ರ್ಯ ಮೂರ್ತಿಯನ್ನೇ, ಪ್ರಕೃತಿಯ ಈಶ್ವರನನ್ನೇ ನಾವು ದೇವರೆಂದು ಕರೆಯುವುದು. ನೀವು ಅವನನ್ನು ಅಲ್ಲಗಳೆಯಲಾರಿರಿ. ಇಲ್ಲ, ಏಕೆಂದರೆ ಸ್ವಾತಂತ್ರ್ಯದ ಭಾವನೆಯಿಲ್ಲದೆ ನಾವು ಚಲಿಸಲಾರೆವು, ಬಾಳಲಾರೆವು. ನೀವು ಸ್ವತಂತ್ರರು ಎಂದು ಅರಿಯದಿದ್ದರೆ ಇಲ್ಲಿಗೆ ಬರುತ್ತಿದ್ದಿರಾ? ಮುಕ್ತನಾಗಬೇಕೆಂದು ನಡೆಸುತ್ತಿರುವ ಸತತ ಹೋರಾಟಕ್ಕೆ ಜೀವವಿಜ್ಞಾನಿಯು ಬೇರೆ ವಿಧವಾದ ವಿವರಣೆಯನ್ನು ಕೊಡಬಹುದು. ಆದರೆ ಸ್ವಾತಂತ್ರ್ಯದ ಭಾವನೆ ಇದ್ದೇ ಇದೆ. ನೀವು ಪ್ರಕೃತಿಗೆ ಬದ್ದರೆಂಬ ವಾಸ್ತವಾಂಶದಷ್ಟೇ ಸತ್ಯ ಇದು.

\vskip 2pt

ಬಂಧನ–ಮುಕ್ತಿ, ಕತ್ತಲೆ–ಬೆಳಕು, ಒಳ್ಳೆಯದು–ಕೆಟ್ಟದ್ದು ಯಾವಾಗಲೂ ಇರಲೇ ಬೇಕು. ಈ ಬಂಧನದ ಭಾವನೆಯೆ, ಅಲ್ಲಿ ಮುಕ್ತಿ ಗುಪ್ತವಾಗಿ ಅಡಗಿದೆ ಎಂಬುದನ್ನು ತೋರುವುದು. ಒಂದು ಸತ್ಯವಾದರೆ ಮತ್ತೊಂದು ಅದರಷ್ಟೇ ಸತ್ಯ. ಈ ಸ್ವಾತಂತ್ರ್ಯದ ಭಾವನೆ ಇದ್ದೇ ತೀರಬೇಕು. ಮೂಢರಲ್ಲಿ ಬಂಧನದ ಭಾವನೆ ಗೋಚರಿಸದೆ ಇದ್ದರೂ, ಅಲ್ಲಿ ಸ್ವಾತಂತ್ರ್ಯದ ಭಾವನೆ ಇದೆ. ಮೂಢನಾದ ಅನಾಗರಿಕನಲ್ಲಿ ಪಾಪದ ಮತ್ತು ದೋಷದ ಬಂಧನ ಅಷ್ಟು ಚೆನ್ನಾಗಿ ಕಾಣುವುದೇ ಇಲ್ಲ. ಏಕೆಂದರೆ ಅವನ ಸ್ವಭಾವ ಮೃಗಕ್ಕಿಂತ ಸ್ವಲ್ಪ ಮೇಲು ಅಷ್ಟೆ. ಅವನೀಗ ಹೋರಾಡುತ್ತಿರುವುದು ಬಾಹ್ಯ ಪ್ರಕೃತಿಯ ದಾಸ್ಯದಿಂದ ಬಿಡುಗಡೆಯಾಗುವುದಕ್ಕೆ ಮತ್ತು ಪಂಚೇಂದ್ರಿಯಗಳ ತೃಪ್ತಿಗಾಗಿ ಮಾತ್ರವಾಗಿದೆ. ಆದರೆ ಇಂತಹ ಕೆಳಗಿನ ಭಾವನೆಗಳಿಂದ ಮಾನಸಿಕ ಮತ್ತು ನೈತಿಕ ಬಂಧನದ ಅರಿವುಂಟಾಗಿ, ಅದು ಪ್ರಬುದ್ಧವಾಗಿ ಆತ್ಮಸ್ವರಾಜ್ಯವನ್ನು ಪಡೆಯಬೇಕೆಂಬ ಹಂಬಲ ಜಾಗೃತವಾಗುವುದು. ಇಲ್ಲಿ ಅಜ್ಞಾನದ ತೆರೆಯ ಹಿಂದೆ ಆತ್ಮಜ್ಯೋತಿಯು ಅಸ್ಪಷ್ಟವಾಗಿ ಬೆಳಗುತ್ತಿರುವುದನ್ನು ನೋಡುವೆವು. ಮೊದಲು ತೆರೆ ಗಾಢವಾಗಿರುವುದು, ಬೆಳಕು ಕಾಣುವಂತೆಯೇ ಇರುವುದಿಲ್ಲ. ಆದರೂ ಪೂರ್ಣತೆಯ ಮುಕ್ತಿಯ ದಿವ್ಯಜ್ಯೋತಿ ಸದಾಕಾಲದಲ್ಲಿಯೂ ಪರಿಶುದ್ಧವಾಗಿ ಅಲ್ಲೇ ಇರುವುದು. ಮನುಷ್ಯ ಅದಕ್ಕೆ ವ್ಯಕ್ತಿತ್ವವನ್ನು ಆರೋಪಿಸಿ `ಪರಮೇಶ್ವರ', `ಇವನು ಏಕಮಾತ್ರ ಮುಕ್ತ ಜೀವಿ,' ಎಂದು ಮುಂತಾಗಿ ಕರೆಯುವನು. ಈ ವಿಶ್ವವೆಲ್ಲ ಒಂದು, ವ್ಯತ್ಯಾಸ ಇರುವುದು ತರತಮದಲ್ಲಿ ಮಾತ್ರ ಎಂಬುದನ್ನು ಇನ್ನೂ ಅವನು ಅರಿಯನು.

\vskip 2pt

ವಿಶ್ವವೆಲ್ಲ ಭಗವಂತನನ್ನು ಆರಾಧಿಸುತ್ತಿದೆ. ಎಲ್ಲಿ ಜೀವನ ಇದೆಯೋ ಅಲ್ಲೆಲ್ಲ ಸ್ವಾತಂತ್ರ್ಯದ ಅನ್ವೇಷಣೆ ಇರುವುದು. ಆ ಸ್ವಾತಂತ್ರ್ಯವೇ ದೇವರು. ಈ ಸ್ವಾತಂತ್ರ್ಯ ಸ್ವಭಾವತಃ ನಮಗೆ ಪ್ರಕೃತಿಯ ಮೇಲೆ ಸ್ವಾಮಿತ್ವವನ್ನು ಕೊಡುವುದು. ಈ ಸ್ವಾಮಿತ್ವ ಜ್ಞಾನವಿಲ್ಲದೆ ಸಾಧ್ಯವಿಲ್ಲ. ಜ್ಞಾನ ನಮಗೆ ಹೆಚ್ಚಿದಷ್ಟೂ ಪ್ರಕೃತಿಯ ಮೇಲೆ ನಮ್ಮ ಅಧಿಕಾರವು ಹೆಚ್ಚುವುದು. ಈ ಅಧಿಕಾರವೇ ನಮ್ಮನ್ನು ಬಲಾಡ್ಯರನ್ನಾಗಿ ಮಾಡುವುದು. ಸಂಪೂರ್ಣ ಮುಕ್ತನಾದ, ಪ್ರಕೃತಿಗೆ ಒಡೆಯನಾದವನೊಬ್ಬನು ಇದ್ದರೆ ಅವನಲ್ಲಿ ಪ್ರಕೃತಿಯ ಪೂರ್ಣ ಜ್ಞಾನವಿರಬೇಕು. ಅವನು ಸರ್ವಜ್ಞನಾಗಿರಬೇಕು, ಸರ್ವವ್ಯಾಪಿಯಾಗಿರಬೇಕು. ಯಾವ ವ್ಯಕ್ತಿಗೆ ಇವೆಲ್ಲ ಇರುವುದೋ ಅವನೊಬ್ಬನೆ ಪ್ರಕೃತಿಗೆ ಅತೀತನು.

ಪೂರ್ಣಸ್ವಾತಂತ್ರ್ಯದಿಂದ ಹೊರಹೊಮ್ಮುವ ಧನ್ಯತೆ ಮತ್ತು ನಿತ್ಯಶಾಂತಿ – ಇದೇ ವೇದಾಂತದ ಭಗವದ್ಭಾವನೆಯ ಹಿಂದಿರುವ ಅತ್ಯುನ್ನತ ಧಾರ್ಮಿಕ ಕಲ್ಪನೆ. ಈ ಸ್ಥಿತಿಯೇ ಯಾವುದರಿಂದಲೂ ಬದ್ದವಾಗದ ಕೇವಲ ಸ್ವಾತಂತ್ರ್ಯ. ಇಲ್ಲಿ ಯಾವ ಬದಲಾವಣೆಯೂ ಇಲ್ಲ. ಈ ಸ್ವಾತಂತ್ರ್ಯ ಇರುವವನಲ್ಲಿ ವಿಕಾರವನ್ನು ತರುವಂತಹುದು ಯಾವುದೂ ಇಲ್ಲ. ಇದೇ ಸ್ವಾತಂತ್ರ್ಯ ನನ್ನಲ್ಲಿ ನಿಮ್ಮಲ್ಲಿ ಎಲ್ಲರಲ್ಲಿಯೂ ಇರುವುದು; ಇದೊಂದೇ ನಿಜವಾದ ಸ್ವಾತಂತ್ರ್ಯ.

ದೇವರು ಸ್ವಭಾವತಃ ಅವಿಕಾರಿಯಾದ ತನ್ನ ಮಹಿಮೆಯ ಮೇಲೆ ಪ್ರತಿಷ್ಠಿತ ನಾಗಿರುವನು. ನಾನು ನೀವು ಅವನಲ್ಲಿರಬೇಕೆಂದು ಯತ್ನಿಸುತ್ತಿರುವೆವು. ಆದರೆ ನಾವು ಪ್ರಕೃತಿ, ದೈನಂದಿನ ಕೆಲಸಕ್ಕೆ ಬಾರದ ವಸ್ತುಗಳು, ದ್ರವ್ಯ, ಕೀರ್ತಿ, ಮಾನವ ಪ್ರೀತಿ ಮುಂತಾದ ಬಂಧನಕ್ಕೆ ಹೇತುವಾದ ಅನಿತ್ಯ ವಸ್ತುಗಳು ಇವುಗಳನ್ನು ಆಶ್ರಯಿಸುತ್ತೇವೆ. ಪ್ರಕೃತಿ\break ಬೆಳಗಿದರೆ ಆ ಬೆಳಕಿಗೆ ಕಾರಣ ಯಾವುದು? ದೇವರು; ಅದಕ್ಕೆ ಕಾರಣ ಸೂರ್ಯ ಚಂದ್ರ ನಕ್ಷತ್ರಗಳಲ್ಲ. ಎಲ್ಲಿಯಾದರೂ ಒಂದು ಜ್ಯೋತಿ ಬೆಳಗುತ್ತಿದ್ದರೆ, ಅದು – ಅವನೆ. ಅವನು ಬೆಳಗಿದರೆ ಅನಂತರ ಉಳಿದವೆಲ್ಲ ಬೆಳಗುವುವು.

ಈ ದೇವರು ಸ್ವತಃಸಿದ್ದ, ಅವ್ಯಕ್ತ, ಸರ್ವಜ್ಞ, ಪ್ರಕೃತಿಯನ್ನು ಅರಿತವನು, ಅದರ ಒಡೆಯ, ಸರ್ವೇಶ್ವರ ಎಂಬುದನ್ನು ನೋಡಿದೆವು. ಅವನೇ ಎಲ್ಲಾ ಪೂಜೆಯ ಹಿಂದೆ ಇರುವುದು. ನಮಗೆ ಗೊತ್ತಿರಲಿ ಇಲ್ಲದಿರಲಿ ಇದೆಲ್ಲ ಅವನ ಇಚ್ಛೆಯಂತೆ ಆಗುತ್ತಿರುವುದು. ನಾನು ಮತ್ತೊಂದು ಹೆಜ್ಜೆ ಮುಂದೆ ಹೋಗುತ್ತೇನೆ. ಎಲ್ಲರೂ ಆಶ್ಚರ್ಯ ಪಡುವ, ನಾವು ಯಾವುದನ್ನು ಪಾಪ ಎನ್ನುವೆವೊ ಅದು ಕೂಡ ಅವನ ಆರಾಧನೆಯೆ, ಅದೂ ಕೂಡ ಸ್ವಾತಂತ್ರ್ಯದ ಒಂದು ಅಂಶವೇ. ಇಲ್ಲಿ ನಾನು ಇನ್ನೂ ಕಠೋರವಾದ ವಿಷಯವನ್ನು ಹೇಳುತ್ತೇನೆ: ನೀವು ಪಾಪವನ್ನು ಮಾಡುವಾಗಲೂ ಅದರ ಹಿಂದೆ ಇರುವುದು ಈ ಸ್ವಾತಂತ್ರ್ಯದ ಪ್ರೇರಣೆಯೆ. ಆ ಪ್ರೇರಣೆಯನ್ನು ನಾವು ಸರಿಯಾಗಿ ಉಪಯೋಗಿಸಿಕೊಳ್ಳದೆ ಇರಬಹುದು, ನಾವು ಅಡ್ಡದಾರಿಗೆ ಹೋಗಿರಬಹುದು, ಆದರೂ ಅದು ಅಲ್ಲಿದೆ. ಈ ಸ್ವಾತಂತ್ರ್ಯ ಅದರ ಹಿಂದೆ ಇಲ್ಲದೆ ಯಾವ ಚೇತನವೂ ಇರುತ್ತಿರಲಿಲ್ಲ. ಯಾವ ಪ್ರೇರಣೆಯೂ ಇರುತ್ತಿರಲಿಲ್ಲ. ನನ್ನ ಸಹೋದರರೇ, ಪ್ರಪಂಚದಲ್ಲೆಲ್ಲಾ ಸ್ವಾತಂತ್ರ್ಯ ಅನುರಣಿತವಾಗುತ್ತಿದೆ. ವಿಶ್ವದ ಅಂತರಾಳದಲ್ಲಿ ಏಕತೆ ಇಲ್ಲದೆ ಇದ್ದರೆ ನಾವು ವೈವಿಧ್ಯವನ್ನು ಅರ್ಥಮಾಡಿಕೊಳ್ಳಲು ಸಾಧ್ಯವಾಗುತ್ತಿರಲಿಲ್ಲ. ಉಪನಿಷತ್ತಿನಲ್ಲಿ ಬರುವ ಭಗವದ್ ಭಾವನೆಯೆ ಇದು. ಕೆಲವು ವೇಳೆ ಇದನ್ನು ಮೀರಿ ಹೋಗುವುದು – ನಾವು ಸ್ವಭಾವತಃ ದೇವರೆ ಎಂದು ಸಾರುವುದು. ಇದು ನಮ್ಮನ್ನು ಚಕಿತಗೊಳಿಸಬಹುದು. ಯಾರು ಹಾರುವ ಚಿಟ್ಟೆಯ ರೆಕ್ಕೆಯ ಮೇಲಿನ ಬಣ್ಣವಾಗಿರುವನೊ, ಯಾರು ಅರಳುವ ಗುಲಾಬಿಯ ಮೊಗ್ಗಿನಲ್ಲಿರುವನೊ, ಅದೇ ಶಕ್ತಿ ಗುಲಾಬಿ ಗಿಡದಲ್ಲಿ ಮತ್ತು ಹಾರುವ ಚಿಟ್ಟೆಯಲ್ಲಿದೆ. ನಮಗೆ ಜೀವನವನ್ನು ನೀಡುವವನೆ ನಮ್ಮ ಅಂತರಾಳದಲ್ಲಿರುವ ಶಕ್ತಿಯಾದ ಅವನು. ಆ ಜ್ಯೋತಿಯಿಂದ ಜೀವನ ಬರುವುದು, ಅತಿ ಕ್ರೂರವಾದ\break ಮೃತ್ಯು ಕೂಡ ಅವನ ಶಕ್ತಿಯೆ. ಯಾವುದರ ಛಾಯೆ ಮೃತ್ಯುವೊ, ಅದು ಅಮೃತವೂ ಆಗಿರುವುದು. ಮತ್ತೊಂದು ಮೆಟ್ಟಲು ಮೇಲೆ ಹೋಗಿ ನಿಲ್ಲಿ. ನೋಡಿ, ನಾವು ರೌದ್ರವಾಗಿರುವುದಕ್ಕೆ ಅಂಜಿ, ಬೇಟೆಗಾರರಿಂದ ಅಟ್ಟಿಸಿಕೊಂಡು ಬಂದ ಮೊಲದಂತೆ ಹೇಗೆ ಓಡಿಹೋಗುತ್ತಿರುವೆವು! ಅವುಗಳಂತೆ ನಮ್ಮ ತಲೆಯನ್ನು ಮುಚ್ಚಿಕೊಂಡು ನಾವು ಸುರಕ್ಷಿತವಾಗಿರುವೆವು ಎಂದು ಭಾವಿಸುವೆವು. ಇಡೀ ಪ್ರಪಂಚ ಭಯಾನಕವಾಗಿರುವುದರೆದುರಿಗೆ ಅಂಜಿ, ಹೇಗೆ ಓಡಿಹೋಗುತ್ತಿದೆ ಎಂಬುದನ್ನು ನೋಡಿ. ಒಂದು ಸಲ ನಾನು ಕಾಶಿಯಲ್ಲಿದ್ದಾಗ ಹೊರಗಡೆ ಓಡಾಡುತ್ತಿದ್ದೆ. ಅಲ್ಲಿ ಒಂದು ಕಡೆ ದೊಡ್ಡ ಕೆರೆ ಇತ್ತು. ಅದರ ಪಕ್ಕದಲ್ಲಿ ಒಂದು ದೊಡ್ಡ ಗೋಡೆ ಇತ್ತು, ಅಲ್ಲಿ ಬೇಕಾದಷ್ಟು ಕೋತಿಗಳು ಇದ್ದುವು. ಕಾಶಿಯ ಕೋತಿಗಳು ಭೀಮಾಕಾರದವುಗಳು. ಕೆಲವು ವೇಳೆ ಉಗ್ರ ಸ್ವಭಾವವನ್ನು ತಾಳುವುವು. ಅವು ನನ್ನನ್ನು ಆ ಮಾರ್ಗದಲ್ಲಿ ಹೋಗಬಿಡಕೂಡದೆಂಬ ಸಂಕಲ್ಪ ಮಾಡಿದವು. ನಾನು ಅಲ್ಲಿ ಹೋಗುತ್ತಿದ್ದಾಗ ಕಿರಿಚಾಡಿ ನನ್ನ ಕಾಲನ್ನು ಹಿಡಿದುಕೊಂಡವು. ಅವು ಹತ್ತಿರ ಬಂದಂತೆ ನಾನು ಓಡಲು ಯತ್ನಿಸಿದೆ. ಆದರೆ ನಾನು ವೇಗವಾಗಿ ಹೋದಂತೆ ಅವೂ ವೇಗವಾಗಿ ಓಡಿಬಂದು ನನ್ನನ್ನು ಕಚ್ಚಲು ಯತ್ನಿಸಿದವು. ಇದರಿಂದ ಪಾರಾಗುವಂತೆಯೇ ಕಾಣಲಿಲ್ಲ. ಆಗ ಒಬ್ಬ ಅಪರಿಚಿತನು ಗೋಚರಿಸಿದ, “ಮೂರ್ಖರನ್ನು ಎದುರಿಸು'' ಎಂದು ನನಗೆ ಗರ್ಜಿಸಿ ಹೇಳಿದ. ನಾನು ಹಿಂತಿರುಗಿ ಕಪಿಗಳನ್ನು ಎದುರಿಸಿದೆ. ಅವೆಲ್ಲ ಪಲಾಯನಗೈದವು. ಇದು ಜೀವನಕ್ಕೆ ಒಂದು ಪಾಠದಂತೆ ಇದೆ. ಭಯಾನಕವಾದುದನ್ನು ಎದುರಿಸಿ, ಧೈರ್ಯವಾಗಿ ಎದುರಿಸಿ. ನಾವು ಯಾವಾಗ ಅದನ್ನು ಎದುರಿ ಸುತ್ತೇವೆಯೋ ಆಗ ಕಪಿಗಳಂತೆ ಅವು ಪಲಾಯನಗೈಯುವುವು. ನಾವು ಎಂದಾದರೂ ಮುಕ್ತಿಯನ್ನು ಪಡೆಯಬೇಕಾದರೆ ಅದು ಪ್ರಕೃತಿಯನ್ನು ಗೆಲ್ಲುವುದರಿಂದ, ಅದರಿಂದ ಪಲಾಯನ ಮಾಡುವುದರಿಂದ ಅಲ್ಲ. ಹೇಡಿಗಳು ಎಂದಿಗೂ ಜಯವನ್ನು ಗಳಿಸಲಾರರು. ಅಂಜಿಕೆ, ಕಷ್ಟ, ಅಜ್ಞಾನ ಇವು ನಮ್ಮ ಮುಂದೆ ಪಲಾಯನವಾಗಬೇಕಾದರೆ ಅವುಗಳೊಂದಿಗೆ ಹೋರಾಡಬೇಕು.

ಸಾವೆಂದರೇನು? ಭಯವೆಂದರೇನು? ಭಗವಂತನ ಮೊಗದಾವರೆಯನ್ನು ನೀವು ಅದರಲ್ಲಿ ನೋಡುವುದಿಲ್ಲವೆ? ಪಾಪ ಭಯ ದುಃಖ ಇವುಗಳಿಂದ ಓಡಿ\break ಹೋಗಲೆತ್ನಿಸಿದಷ್ಟೂ ಅವು ನಿಮ್ಮನ್ನು ಹಿಂಬಾಲಿಸುವುವು. ಅವುಗಳನ್ನು ನೀವು ಎದುರಿಸಿದರೆ ಅವು ಓಡುವುವು. ಇಡೀ ಪ್ರಪಂಚವೆ, ಸುಖವನ್ನು ಮತ್ತು ಅನುಕೂಲವನ್ನು ಆರಾಧಿಸುತ್ತಿರುವುದು. ದುಃಖವಾಗಿರುವುದನ್ನು ಆರಾಧಿಸುವವರು ಅತಿ ವಿರಳರು. ಇವೆರಡನ್ನೂ ಮೀರಿ ಹೋಗಲೆತ್ನಿಸುವುದೇ ಸ್ವಾತಂತ್ರ್ಯದ ಭಾವನೆ. ನಾವು ಈ ಮಾರ್ಗವಾಗಿ ಹೋದಲ್ಲದೆ ಮುಕ್ತರಾಗೆವು. ನಾವು ಇವನ್ನೆಲ್ಲ ಎದುರಿಸಬೇಕಾಗಿದೆ. ನಾವು ಭಗವಂತನನ್ನು ಆರಾಧಿಸಲು ಇಚ್ಛಿಸುವೆವು. ಆದರೆ ನಮ್ಮಿಬ್ಬರ ಮಧ್ಯದಲ್ಲಿ ದೇಹ ತಡೆಯಾಗುವುದು; ನಮಗೂ ಅವನಿಗೂ ಮಧ್ಯೆ ಪ್ರಕೃತಿಯೆ ಬಂದು ನಮ್ಮ ದೃಷ್ಟಿಯನ್ನು ಮಸುಕುಗೊಳಿಸುವುದು. ಸಿಡಿಲಿನಲ್ಲಿ, ನಾಚಿಕೆಯಲ್ಲಿ, ದುಃಖದಲ್ಲಿ, ಪಾಪದಲ್ಲಿ ಅವನನ್ನು ಪೂಜಿಸುವುದನ್ನು, ಅವನನ್ನು ನೋಡುವುದನ್ನು ನಾವು ಅಭ್ಯಾಸಮಾಡಬೇಕು. ಪ್ರಪಂಚದಲ್ಲೆಲ್ಲಾ, ಎಲ್ಲರೂ ಸುಗುಣಗಳಿಗೆ\break ಒಡೆಯನಾದ ದೇವರನ್ನೇ ಇಂದಿನವರೆಗೂ ಬೋಧಿಸುತ್ತಿರುವುದು. ಪಾಪ ಪುಣ್ಯಗಳೆರೆಡೂ ಇರುವ ದೇವರನ್ನು ನಾನು ಬೋಧಿಸುತ್ತೇನೆ. ಧೈರ್ಯವಿದ್ದರೆ ಅವನನ್ನು ಸ್ವೀಕರಿಸಿ, ಮುಕ್ತಿಗೆ ಅದೊಂದೇ ಮಾರ್ಗ. ಆಗ ಮಾತ್ರ ಏಕತ್ವದಿಂದ ಉದ್ಭವಿಸುವ ಪರಮ ಸತ್ಯವು ನಮಗೆ ಪ್ರಾಪ್ತವಾಗುವುದು. ಆಗ ಒಬ್ಬ ಕೀಳು ಎಂಬ ಭಾವನೆ ಹೋಗುವುದು. ಸ್ವಾತಂತ್ರ್ಯದ ನಿಯಮವನ್ನು ಸಮೀಪಿಸಿದಷ್ಟೂ ಭಗವಂತನನ್ನು ಸಮೀಪಿಸುವೆವು, ನಮ್ಮ ಸಂಕಟ ಕೊನೆಗಾಣುವುದು. ಆಗ ಸ್ವರ್ಗನರಕಗಳ ದ್ವಾರದಲ್ಲಿ ನಾವು ವ್ಯತ್ಯಾಸವನ್ನು ಕಲ್ಪಿಸುವುದಿಲ್ಲ. ಮನುಷ್ಯರಲ್ಲಿ ಭೇದಭಾವನೆಯನ್ನು ತಂದು “ನಾನೇ ಪ್ರಪಂಚದಲ್ಲೆಲ್ಲಾ ಸರ್ವಶ್ರೇಷ್ಠ” ಎನ್ನುವುದಿಲ್ಲ. ಭಗವಂತನನ್ನಲ್ಲದೆ ನಾವು ಮತ್ತೇನನ್ನೊ ನೋಡಿದಾಗ ಈ ಲೋಪದೋಷಗಳೆಲ್ಲ ನಮ್ಮನ್ನು ಮುತ್ತುವುವು. ನಾವು ಈ ಭೇದಭಾವನೆಯನ್ನು ಕಲ್ಪಿಸಿಕೊಳ್ಳುವೆವು. ಭಗವಂತನಲ್ಲಿ ಮಾತ್ರ, ಪರಮಾತ್ಮನಲ್ಲಿ ಮಾತ್ರ, ನಾವು ಒಂದು. ನಾವು ಭಗವಂತನನ್ನು ಎಲ್ಲೆಲ್ಲೂ ನೋಡುವವರೆಗೆ ಏಕತೆಯಿಂದ ದೂರದಲ್ಲಿರುವೆವು.

ಸುಂದರವಾದ ಗರಿಗಳುಳ್ಳ ಎರಡು ಹಕ್ಕಿಗಳು, ಎಂದಿಗೂ ಬೇರೆಯಾಗದ ಗೆಳೆಯರಿಬ್ಬರು, ಒಂದೇ ಮರದ ಮೇಲೆ, ಒಂದು ಮೇಲೆ ಮತ್ತೊಂದು ಕೆಳಗೆ ಕುಳಿತುಕೊಂಡಿದ್ದವು. ಕೆಳಗಿರುವ ಆ ಸುಂದರವಾದ ಹಕ್ಕಿ ಸಿಹಿಕಹಿಯಾಗಿರುವ ಮರದ ಹಣ್ಣುಗಳನ್ನು ತಿನ್ನುತ್ತಿತ್ತು. ಒಂದು ಸಲ ಸಿಹಿ ಹಣ್ಣು ಸಿಕ್ಕುತ್ತಿತ್ತು, ಮತ್ತೊಂದು ಸಲ ಕಹಿ ಹಣ್ಣು ಸಿಕ್ಕುತ್ತಿತ್ತು. ಕಹಿ ಹಣ್ಣನ್ನು ತಿಂದೊಡನೆಯೆ ವ್ಯಥೆಪಡುತ್ತಿತ್ತು. ಸ್ವಲ್ಪ ಹೊತ್ತಾದ ಮೇಲೆ ಮತ್ತೊಂದು ಹಣ್ಣನ್ನು ತಿಂದಾಗ ಅದೂ ಕಹಿ ಆದಾಗ, ಸಿಹಿ ಕಹಿಯಾದ ಹಣ್ಣುಗಳನ್ನು ತಿನ್ನದೆ ಶಾಂತವಾಗಿಯೆ ಗಂಭೀರವಾಗಿ ಆತ್ಮಾನಂದದಲ್ಲಿ ಪ್ರತಿಷ್ಠಿತವಾಗಿರುವ ಮೇಲಿನ ಹಕ್ಕಿಯನ್ನು ನೋಡಿತು. ಪಾಪ, ಸ್ವಲ್ಪ ಹೊತ್ತಿನ ಮೇಲೆ ಕೆಳಗಿರುವ ಹಕ್ಕಿ ಇದನ್ನು ಮರೆತು ಪುನಃ ಸಿಹಿ ಕಹಿ ಹಣ್ಣುಗಳನ್ನು ತಿನ್ನಲು ಪ್ರಾರಂಭಿಸಿತು. ಪುನಃ ಒಂದು ಸಲ ಬಹಳ ಕಹಿಯಾದ ಹಣ್ಣನ್ನು ತಿಂದಾಗ, ಮುಂದೆ ತಿನ್ನುವುದನ್ನು ನಿಲ್ಲಿಸಿ, ಮೇಲಿರುವ ಆ ಗಂಭೀರವಾದ ಹಕ್ಕಿಯನ್ನು ನೋಡಿತು. ಆಗ ಅದು ಮೇಲಿನ ಹಕ್ಕಿಯ ಸಮೀಪಕ್ಕೆ ಬರತೊಡಗಿತು. ಅದು ಬಹಳ ಸಮೀಪಕ್ಕೆ ಬಂದಾಗ ಮೇಲಿನ ಹಕ್ಕಿಯ ಕಾಂತಿ ಇದನ್ನು ಆವರಿಸಿತು. ಅನಂತರ ತಾನೇ ಆ ಮೇಲಿನ ಹಕ್ಕಿಯಾಗಿ ಬದಲಾವಣೆ ಹೊಂದಿತು. ಆಗ ಅದು ಶಾಂತವಾಗಿ ಗಂಭೀರವಾಗಿ, ಸ್ವತಂತ್ರವಾಯಿತು. ಆಗ ಮರದ ಮೇಲೆ ಇದ್ದುದು ಒಂದೇ ಹಕ್ಕಿಯೆಂದು ತಿಳಿಯಿತು. ಕೆಳಗಿನ ಹಕ್ಕಿ ಮೇಲಿನ ಹಕ್ಕಿಯ ಪ್ರತಿಬಿಂಬ ಮಾತ್ರ. ನಾವು ನಿಜವಾಗಿಯೂ ಭಗವಂತನಲ್ಲಿ ಐಕ್ಯರಾಗಿರುವೆವು. ಆದರೆ ಪ್ರತಿಬಿಂಬದ ದೆಸೆಯಿಂದ ಹಲವನ್ನು ನೋಡುತ್ತಿರುವೆವು – ಒಂದೇ ಸೂರ್ಯ, ಕೋಟ್ಯಂತರ ಹಿಮ ಮಣಿಗಳಲ್ಲಿ ಪ್ರತಿಬಿಂಬಿತವಾಗಿ ಅಷ್ಟೊಂದು ಸೂರ್ಯರಿರುವಂತೆ ಕಾಣುವಂತೆ. ಪವಿತ್ರವಾದ ನಮ್ಮ ಸ್ವಭಾವದೊಂದಿಗೆ ಏಕತೆಯನ್ನು ತಾಳಬೇಕಾದರೆ ಈ ಪ್ರತಿಬಿಂಬ ಮಾಯವಾಗಬೇಕು. ಈ ಪ್ರಪಂಚ ನಮಗೆ ಎಂದಿಗೂ ತೃಪ್ತಿಯನ್ನು ನೀಡಲಾರದು. ಆದಕಾರಣವೇ ಜಿಪುಣ ಹೆಚ್ಚು ಹೊನ್ನನ್ನು ಮತ್ತು ಹಣವನ್ನು ಸಂಗ್ರಹಿಸುತ್ತಿರುವುದು, ಕಳ್ಳ ಕದಿಯುವುದು, ಪಾಪಿಗಳು ಪಾಪಮಾಡುವುದು. ಆದಕಾರಣವೇ ನೀವು ತತ್ತ್ವಶಾಸ್ತ್ರವನ್ನು ಕಲಿಯುವುದು. ಎಲ್ಲರಿಗೂ ಗುರಿ ಒಂದೇ. ಈ\break ಸ್ವಾತಂತ್ರ್ಯವನ್ನು ಸಾಧಿಸುವುದಲ್ಲದೆ ಬೇರಾವ ಗುರಿಯೂ ಜೀವನದಲ್ಲಿ ಇಲ್ಲ. ನಾವೆಲ್ಲ ಅರಿತೋ ಅರಿಯದೆಯೊ ಪೂರ್ಣತೆಗಾಗಿ ಹೋರಾಡುತ್ತಿರುವೆವು. ಪ್ರತಿಯೊಬ್ಬರೂ ಅದನ್ನು ಪಡೆಯಲೇಬೇಕು.

ಪಾಪದ ಮೂಲಕ, ದುಃಖದ ಮೂಲಕ ತಡಕಾಡುತ್ತಿರುವವನು, ನರಕದ ಮೂಲಕ ಹೋಗಲಿಚ್ಚಿಸುವವನು, ಎಲ್ಲರೂ ಗುರಿಯನ್ನು ಸೇರುವರು. ಆದರೆ ಕಾಲ ವಿಳಂಬವಾಗುವುದು. ನಾವು ಅವನಿಗೆ ಬುದ್ದಿ ಹೇಳಲಾರೆವು. ಕೆಲವು ಕಷ್ಟದ ಅನುಭವಗಳು ಭಗವಂತನ ಕಡೆಗೆ ತಿರುಗುವಂತೆ ಅವನನ್ನು ಪ್ರೇರಿಸುವುವು. ಕೊನೆಗೆ ಧರ್ಮದ ಮಾರ್ಗ, ಪಾವಿತ್ರ್ಯದ, ನಿಃಸ್ವಾರ್ಥದ ಆಧ್ಯಾತ್ಮಿಕ ಮಾರ್ಗ ಅವನಿಗೆ ಗೋಚರವಾಗುವುದು. ಯಾವುದನ್ನು ಎಲ್ಲರೂ ತಮ್ಮ ಅರಿವಿಲ್ಲದೆ ಮಾಡುತ್ತಿರುವರೋ ಅದನ್ನೇ ನಾವು ಅರಿತು ಮಾಡುವೆವು. ಸೈಂಟ್ ಪಾಲ್ ಈ ಭಾವನೆಯನ್ನು ಹೀಗೆ ವ್ಯಕ್ತಗೊಳಿಸುವನು: “ನೀವು ಯಾವ ದೇವರನ್ನು ಅಜ್ಞಾನದಿಂದ ಪೂಜಿಸುತ್ತಿರುವಿರೋ ಅದೇ ದೇವರನ್ನು ನಾನು ನಿಮಗೆ ಬೋಧಿಸುತ್ತೇನೆ.'' ಇಡೀ ಪ್ರಪಂಚ ಕಲಿಯಬೇಕಾದ ಪಾಠ ಇದು. ಈ ಏಕಮಾತ್ರ ಗುರಿಯನ್ನು ಸಾಧಿಸುವುದಕ್ಕೆ ಸಹಾಯಮಾಡುವುದಲ್ಲದೆ ಈ ತತ್ತ್ವ ಮತ್ತು ಸಿದ್ದಾಂತಗಳು ಮತ್ತೇನನ್ನು ಮಾಡುತ್ತಿವೆ? ಎಲ್ಲದರ ತಾದಾತ್ಮ್ಯ ಭಾವನೆಯ ಕೇಂದ್ರಕ್ಕೆ ಬರೋಣ. ಮಾನವನು ಎಲ್ಲದರಲ್ಲೂ ತನ್ನನ್ನು ನೋಡಲಿ, ಅಲ್ಪ ದೇವರ ಭಾವನೆಗಳನ್ನೊಳಗೊಂಡ ಮತಗಳನ್ನು ಇನ್ನು ಮೇಲೆ ಆಶ್ರಯಿಸದೆ ಇರೋಣ. ಜಗತ್ತಿನಲ್ಲೆಲ್ಲಾ ಅವನನ್ನು ನೋಡೋಣ. ನೀವು ಬ್ರಹ್ಮಜ್ಞಾನಿಗಳಾಗಿದ್ದರೆ, ನಿಮ್ಮ ಹೃದಯದಲ್ಲಿ ನೀವು ಯಾರ ಪೂಜೆಯನ್ನು ಮಾಡುತ್ತಿರುವಿರೋ ಅವನನ್ನೇ ಎಲ್ಲೆಲ್ಲಿಯೂ ನೋಡುವಿರಿ.

ಮೊದಲು ಈ ಅಲ್ಪಭಾವನೆಗಳಿಂದ ಪಾರಾಗಿ; ಪ್ರತಿಯೊಬ್ಬರಲ್ಲಿಯೂ ದೇವರನ್ನು ನೋಡಿ, ಅವನೇ ಎಲ್ಲ ಕೈಗಳ ಮೂಲಕ ಕೆಲಸ ಮಾಡುತ್ತಿರುವನು, ಎಲ್ಲಾ ಕಾಲುಗಳ ಮೂಲಕ ನಡೆಯುತ್ತಿರುವನು, ಎಲ್ಲಾ ಬಾಯಿಗಳ ಮೂಲಕ ತಿನ್ನುತ್ತಿರುವನು; ಅವನು ಎಲ್ಲರ ಹೃದಯದಲ್ಲೂ ಇರುವನು, ಎಲ್ಲರ ಮನಸ್ಸಿನ ಮೂಲಕ ಆಲೋಚಿಸುತ್ತಿರುವನು. ಅವನು ಸ್ವಯಂವೇದ್ಯ, ಅವನು ನಮಗಿಂತ ಹತ್ತಿರ ಇರುವನು. ಇದನ್ನು ಅರಿಯುವುದೇ ಧರ್ಮ; ಇದೇ ಶ್ರದ್ಧೆ, ಭಗವಂತ ನಮಗೆ ಇಂತಹ ಶ್ರದ್ಧೆಯನ್ನು ಅನುಗ್ರಹಿಸಲಿ. ನಾವು ಆ ಏಕತೆಯನ್ನು ಅನುಭವಿಸಿದಾಗ ಅಮರರಾಗುವೆವು. ನಾವು ಭೌತಿಕ ದೃಷ್ಟಿಯಿಂದಲೂ ಅಮರರು. ವಿಶ್ವದೊಡನೆ ಒಂದಾಗಿರುವೆವು. ಎಲ್ಲಿಯವರೆಗೆ ಪ್ರಪಂಚದಲ್ಲಿ ಒಬ್ಬನಾದರೂ ಜೀವಿಸುತ್ತಿರುವನೋ ಅಲ್ಲಿಯವರೆಗೆ ನಾನು ಅವನಲ್ಲಿರುವೆನು. ನಾನು ಉಪಾಧಿಗಳಿಗೆ ಒಳಗಾದ ಈ ಅಲ್ಪ ವ್ಯಕ್ತಿಯಲ್ಲ. ನಾನೇ ವಿಶ್ವಾತ್ಮ. ಹಿಂದೆ ಇದ್ದವರೆಲ್ಲರ ಚೇತನವೂ ನಾನೇ. ನಾನೇ ಬುದ್ಧನ, ಏಸುವಿನ, ಮಹಮ್ಮದನ ಆತ್ಮ. ನಾನೇ ಎಲ್ಲಾ ಗುರುಗಳ ಆತ್ಮ; ದರೋಡೆ ಮಾಡಿದ ಡಕಾಯಿತ ನಾನೇ, ಮರಣ ದಂಡನೆಗೆ ಗುರಿಯಾದ ಕೋಲೆಪಾತಕಿ ನಾನೇ, ನಾನೇ ವಿಶ್ವಾತ್ಮ. ಎದ್ದು ನಿಲ್ಲಿ, ಇದೇ ಪರಾಪೂಜೆ. ನೀವು ವಿಶ್ವಾತ್ಮರು. ಇದೇ ನಿಜವಾದ ನಮ್ರತೆ. ನೆಲದ ಮೇಲೆ ತೆವಳುತ್ತ ಪಾಪಿ ಎಂದು ಹಳಿದುಕೊಳ್ಳುವುದಲ್ಲ. ಭೇದಭಾವನೆಯ ತೆರೆ ಯಾವಾಗ ಜಾರಿ ಬೀಳುವುದೊ ಅದೇ ವಿಕಾಸದ ಪರಮಾವಧಿ. ಐಕ್ಯವೇ ಶ್ರೇಷ್ಠ ಧರ್ಮ. ನಾನು ಇಂತಹವನು ಅಂತಹವನು ಎಂದು ಹೇಳಿಕೊಳ್ಳುವುದು ಅಲ್ಪ ಭಾವನೆ. ಇದು ನಿಜವಾದ ನನ್ನ ಸತ್ಯವಲ್ಲ. ನಾನೇ ವಿಶ್ವಾತ್ಮ ಎಂದು ಈ ಭಾವನೆಯನ್ನು ಆಶ್ರಯಿಸಿ ಪರಮಾತ್ಮನ ಪರಾಪೂಜೆಯನ್ನು ನೀವು ಮಾಡಿ. ದೇವರು ಆತ್ಮ; ಅವನ ನೈಜ ಸ್ಥಿತಿಯನ್ನು ಆತ್ಮನ ಮೂಲಕ ಮಾತ್ರ ಪೂಜಿಸಬೇಕು. ಗೌಣೀ ಪೂಜೆಯ ಮೂಲಕ ಮಾನವನ ಪ್ರಾಪಂಚಿಕ ಚಿಂತನೆಗಳು ಆಧ್ಯಾತ್ಮಿಕ ಪೂಜೆಯಾಗಿ ಪರಿವರ್ತನೆಗೊಂಡು, ಕೊನೆಯಲ್ಲಿ ವಿಶ್ವವ್ಯಾಪಿಯಾದ ಪರಮಾತ್ಮನನ್ನು ಅಧ್ಯಾತ್ಮದ ಮೂಲಕ ಪೂಜಿಸುವನು. ಯಾವುದು ಮಿತಿಗೊಳಗಾಗಿರುವುದೋ ಅದು ಭೌತಿಕವಾದುದು. ಆತ್ಮ ಒಂದೇ ಅನಂತ, ದೇವರು, ಅಧ್ಯಾತ್ಮ, ವಿಭು. ಮಾನವ ಅಧ್ಯಾತ್ಮವಾದುದರಿಂದ ವಿಭು, ವಿಭುವೊಂದೇ ವಿಭುವನ್ನು ಪೂಜಿಸಬಲ್ಲದು. ನಾವು ವಿಭುವನ್ನು ಪೂಜಿಸುವೆವು. ಇದೇ ಶ್ರೇಷ್ಠವಾದ ಪರಾಪೂಜೆ. ಈ ಭವ್ಯ ಭಾವನೆಗಳನ್ನು ಸಾಕ್ಷಾತ್ಕಾರ ಮಾಡಿಕೊಳ್ಳುವುದು ಎಷ್ಟು ಕಷ್ಟವಾದುದು! ನಾನು ಸಿದ್ದಾಂತ ಮಾಡುತ್ತೇನೆ, ಮಾತನಾಡುತ್ತೇನೆ, ತತ್ತ್ವವನ್ನು ಪ್ರತಿಪಾದಿಸುತ್ತೇನೆ. ಆದರೆ ಮರುಕ್ಷಣದಲ್ಲಿ ಯಾವುದೋ ನನ್ನನ್ನು ವಿರೋಧಿಸುವುದು. ನನ್ನ ಅರಿವಿಲ್ಲದೆ ಕೋಪಗೊಳ್ಳುತ್ತೇನೆ. ಈ ಅಲ್ಪ ಅಹಂಕಾರವಲ್ಲದೆ ಪ್ರಪಂಚದಲ್ಲಿ ಮತ್ತೇನೂ ಇಲ್ಲ ಎಂದು ಭಾವಿಸುತ್ತೇನೆ. ನಾನು ಆತ್ಮ, ಈ ಕೆಲಸಕ್ಕೆ ಬರದ ವಸ್ತುವಿನಿಂದ ನನಗೇನು ಎನ್ನುವುದನ್ನು ಮರೆಯುವೆನು; ನಾನೇ ಆತ್ಮ, ನಾನೇ ಈ ಆಟವನ್ನೆಲ್ಲ ಆಡುತ್ತಿರುವುದು ಎಂಬುದನ್ನು ಮರೆಯುತ್ತೇನೆ. ನಾನು ದೇವರನ್ನು ಮರೆಯುತ್ತೇನೆ, ಸ್ವಾತಂತ್ರ್ಯವನ್ನು ಮರೆಯುತ್ತೇನೆ.

ಮುಕ್ತಿಗೆ ಮಾರ್ಗ ಕತ್ತಿಯ ಅಲಗಿನಷ್ಟು ಹರಿತವಾಗಿದೆ, ದೂರವಾಗಿದೆ ಮತ್ತು ಕಷ್ಟಮಯವಾಗಿದೆ. ಋಷಿಗಳು ಇದನ್ನು ಪುನಃ ಪುನಃ ಹೇಳಿರುವರು. ಆದರೂ ಮನೋದೌರ್ಬಲ್ಯ ಮತ್ತು ಸೋಲು ನಿಮ್ಮನ್ನು ಬಂಧಿಸದಿರಲಿ. ಉಪನಿಷತ್ತುಗಳು, “ಏಳಿ, ಜಾಗೃತರಾಗಿ, ಗುರಿ ಸೇರುವವರೆಗೆ ನಿಲ್ಲದಿರಿ'' ಎಂದು ಸಾರುವುವು. ದಾರಿ ಕತ್ತಿಯ ಅಲಗಿನಂತೆ ಹರಿತವಾಗಿದ್ದರೂ, ಬಹಳ ದೂರವಾಗಿದ್ದರೂ, ಕಷ್ಟಮಯವಾಗಿದ್ದರೂ, ನಾವು ನಿಜವಾಗಿ ದಾಟಿಯೇ ದಾಟುವೆವು. ಮಾನವನು ದೇವಾಸುರರಿಗೆ ಒಡೆಯನಾಗುವನು. ನಮ್ಮ ದುಃಖಕ್ಕೆ ನಾವಲ್ಲದೆ ಮತ್ತಾರೂ ಹೊಣೆಗಾರರಲ್ಲ. ಅಮೃತವನ್ನು ಅರಸಿಕೊಂಡು ಮಾನವ ಹೋದರೆ, ಅಲ್ಲಿ ಕಾರ್ಕೊಟಕ ವಿಷದ ಬಟ್ಟಲು ಮಾತ್ರ ಎಂದು ನೀವು ಭಾವಿಸುವಿರೇನು? ಅಮೃತ ಅಲ್ಲಿದೆ. ಹೋಗುವವರಿಗೆಲ್ಲ ಅದು ದೊರಕುವುದು. ದೇವರೇ ನಮಗೆ ಎಲ್ಲಾ ಧರ್ಮಗಳನ್ನು ತ್ಯಜಿಸಿ ನನ್ನಲ್ಲಿ ಶರಣಾಗು, ನಾನು ನಿನ್ನನ್ನು ಪಾರುಗಾಣಿಸುತ್ತೇನೆ, ಅಂಜಬೇಡ'' ಎಂಬ ಭರವಸೆ ನೀಡಿರುವನು. ಇಲ್ಲಿಯವರೆಗೆ ನಮಗೆ ಬಂದಿರುವ ಧರ್ಮಶಾಸ್ತ್ರಗಳೆಲ್ಲ ಇದನ್ನು ಸಾರುವುದನ್ನು ಕಾಣುತ್ತೇವೆ. ಇದೇ ವಾಣಿ ನಮಗೆ, “ಸ್ವರ್ಗದಲ್ಲಿರುವಂತೆ ಇಲ್ಲಿಯೂ ನಿನ್ನ ಅಣತಿಯನ್ನು ಪಾಲಿಸುತ್ತೇವೆ; ಏಕೆಂದರೆ, ಇದೆಲ್ಲ ನಿನ್ನ ರಾಜ್ಯ, ನಿನ್ನ ಶಕ್ತಿ ಮತ್ತು ನಿನ್ನ ಮಹಿಮೆ'' ಎಂದು ಸಾರುವುದು. ಇದೆಲ್ಲ ಕಷ್ಟ, ಬಹಳ ಕಷ್ಟ.

“ಹೇ ದೇವರೆ, ಈ ಕ್ಷಣವೇ ನಿನ್ನಲ್ಲಿ ಶರಣಾಗುವೆ. ನಿನ್ನ ಪ್ರೇಮಕ್ಕಾಗಿ ನಾನು ನನ್ನ ಸರ್ವವನ್ನೂ ತ್ಯಜಿಸುವೆ. ನಿನ್ನ ಪೀಠದ ಮೇಲೆ ಎಲ್ಲಾ ಒಳ್ಳೆಯದನ್ನೂ, ಎಲ್ಲಾ ಧರ್ಮಗಳನ್ನೂ ಅರ್ಪಿಸುವೆ. ನನ್ನ ಪಾಪ, ದುಃಖ, ಕರ್ಮ, ಒಳಿತು, ಕೆಡಕು ಎಲ್ಲವನ್ನೂ ನಿನಗೆ ಅರ್ಪಿಸುತ್ತೇನೆ. ನೀನು ಅವುಗಳನ್ನೆಲ್ಲ ಸ್ವೀಕರಿಸು. ನಿನ್ನನ್ನು ನಾನು ಎಂದಿಗೂ ಮರೆಯುವುದಿಲ್ಲ” ಎಂದು ಹೇಳುವೆ. ಒಂದು ಕ್ಷಣ ನಿನ್ನ ಇಚ್ಛೆಯಂತೆ ಆಗಲಿ ಎನ್ನುವೆ. ಮತ್ತೊಂದು ಕ್ಷಣ ಯಾವುದೋ ನನ್ನನ್ನು ಕೆಣಕಲು ಬಂದಾಗ ನಾನು ಕೋಪಗೊಂಡು ಭುಗಿಲ್ ಎಂದು ಏಳುವೆನು. ಎಲ್ಲಾ ಧರ್ಮಗಳ ಗುರಿ ಒಂದೇ. ಆದರೆ ಗುರುಗಳು ಹೇಳುವ ರೀತಿಯಲ್ಲಿ ವ್ಯತ್ಯಾಸವಾಗುವುದು. ಧರ್ಮ ಅನಿತ್ಯವಾದ ನಾನು ಎಂಬುದನ್ನು ನಾಶಮಾಡಲು ಯತ್ನಿಸುವುದು. ಆಗ ನಿಜವಾದ ಭಗವಂತನೇ ಆಗಿರುವ `ನಾನು' ಅಲ್ಲಿ ನೆಲಸಲು ಸಾಧ್ಯ. “ನಿನ್ನ ದೇವರಾದ ಸರ್ವೇಶ್ವರನಾದ ನಾನು ಬಹಳ ಅಸೂಯಾಪರನು. ನಾನಲ್ಲದೆ ಮತ್ತಾವ ದೇವರನ್ನೂ ನೀನು ಇಟ್ಟುಕೊಳ್ಳಕೂಡದು'' ಎಂದು ಹೀಬ್ರೂ ಶಾಸ್ತ್ರ ಸಾರುವುದು. ದೇವರೊಬ್ಬನೇ ಇರಬೇಕು. ಅಲ್ಲಿ “ನಾನಲ್ಲ, ನೀನು'' ಎಂದು ಹೇಳಬೇಕು. ದೇವರಲ್ಲದ ಎಲ್ಲವನ್ನೂ ತ್ಯಜಿಸಬೇಕು. ಅವನು, ಅವನೊಬ್ಬನೇ ಅಲ್ಲಿ ಆಳಬೇಕು. ನಾವು ಬಹಳ ಕಷ್ಟಪಟ್ಟು ಹೋರಾಡಬಹುದು. ಆದರೂ ಮರುಕ್ಷಣವೇ ಕಾಲು ಜಾರಬಹುದು. ಆಗ ಸಹಾಯಕ್ಕೆ ತಾಯಿಯನ್ನು ಕರೆಯುತ್ತೇವೆ. ನಾವೊಬ್ಬರೇ ಇರಲಾರೆವು ಎಂದು ತೋರುವುದು. ಜೀವನ ಅನಂತವಾಗಿರುವುದು. `ನಿನ್ನ ಆಣತಿಯಂತೆ ನಡೆಯುವೆನು' ಎಂಬುದು ಅದರಲ್ಲಿ ಒಂದು ಅಧ್ಯಾಯ. ನಾವು ಎಲ್ಲಾ ಅಧ್ಯಾಯಗಳನ್ನೂ ತಿಳಿದಲ್ಲದೆ ಇಡೀ ಗ್ರಂಥವನ್ನು ತಿಳಿಯಲಾಗುವುದಿಲ್ಲ. “ನಿನ್ನ ಇಚ್ಛೆಯಂತೆ ಆಗಲಿ" ಎನ್ನುವೆವು. ಆದರೆ ಪ್ರತಿಕ್ಷಣವೂ ಈ ದ್ರೋಹಿ ಅಹಂಕಾರ ತಲೆಯನ್ನು ಎತ್ತುವುದು. ಆದರೂ ನಾವು ಅಲ್ಪಾತ್ಮವನ್ನು ಗೆಲ್ಲಬೇಕಾದರೆ ಪದೇ ಪದೇ ಇದನ್ನು ಹೇಳುತ್ತಿರಬೇಕಾಗುವುದು. ನಾವು ದ್ರೋಹಿಯ ಸೇವೆಯನ್ನು ಮಾಡಿಯೂ ಉದ್ದಾರವಾಗಲಾರೆವು. ದ್ರೋಹಿಗಲ್ಲದೆ ಇತರರಿಗೆ ಮುಕ್ತಿಯಿದೆ. ದ್ರೋಹಿಗಳಂತೆ ನಾವು ಶಿಕ್ಷಾರ್ಹರು. ನಮ್ಮ ಪರಮಾತ್ಮನ ವಾಣಿಯಂತೆ ನಡೆಯದೆ ಇದ್ದರೆ ನಾವು ಆತ್ಮದ್ರೋಹಿಗಳಾಗುವೆವು. ಜಗನ್ಮಾತೆಯ ಗೌರವಕ್ಕೆ ದ್ರೋಹಿಗಳಾಗುವೆವು. ಏನಾದರೂ ಆಗಲಿ, ನಾವೆಲ್ಲ ಪರಮಾತ್ಮನಿಗೆ ನಮ್ಮ ತನುಮನಗಳನ್ನು ಅರ್ಪಿಸೋಣ. ಹಿಂದೂ ದಾರ್ಶನಿಕನೊಬ್ಬ “ನಿನ್ನ ಇಚ್ಛೆ ಯಂತಾಗಲಿ'' ಎಂದು ಎರಡು ವೇಳೆ ಹೇಳುವುದು ದ್ರೋಹಮಾಡಿದಂತೆ ಆಗುವುದು ಎನ್ನುವನು. “ನಿನ್ನ ಇಚ್ಛೆಯಂತಾಗಲಿ”, ಹೆಚ್ಚಿಗೆ ಏನು ಬೇಕು? ಎರಡು ಬಾರಿ ಏತಕ್ಕೆ ಹೇಳಬೇಕು. ಒಳ್ಳೆಯದು ಒಳ್ಳೆಯದೆ. ನಾವು ಅದನ್ನು ಪುನಃ ಹಿಂದಕ್ಕೆ ತೆಗೆದುಕೊಳ್ಳಬೇಕಾಗಿಲ್ಲ. “ಸ್ವರ್ಗದಂತೆ ಮರ್ತ್ಯಲೋಕದಲ್ಲಿಯೂ ನಿನ್ನ ಇಚ್ಛೆ ನೆರವೇರಲಿ. ಏಕೆಂದರೆ ಎಂದೆಂದಿಗೂ ಈ ರಾಜ್ಯ ನಿನ್ನದು, ಶಕ್ತಿ ನಿನ್ನದು, ಮಹಿಮೆ ನಿನ್ನದು.”

