
\chapter[ಗುರಿ]{ಗುರಿ\protect\footnote{\engfoot{C.W, Vol. II, P. 463}}}

\begin{center}
(೧೯೦೦ರ ಮಾರ್ಚಿ ೨೭ರಂದು ಸ್ಯಾನ್‌ಫ್ರಾನ್ಸಿಸ್ಕೋದಲ್ಲಿ ನೀಡಿದ ಪ್ರವಚನದ ಟಿಪ್ಪಣಿ)
\end{center}

ಮಾನವನು ತನಗಿಂತ ಶ್ರೇಷ್ಠವಾದ ಯಾವುದೋ ಒಂದರಿಂದ ಆವೃತನಾದಂತೆ ಕಾಣುವನು. ಅದನ್ನು ಅರಿಯಲು ಅವನು ಯತ್ನಿಸುತ್ತಿರುವನು. ಅವನು ಯಾವಾಗಲೂ ಅತ್ಯಂತ ಉನ್ನತವಾದ ಆದರ್ಶವನ್ನೇ ಅರಸುವನು. ಅದು ಇದೆ ಎಂದು ಅವನಿಗೆ ಗೊತ್ತಿದೆ. ಧರ್ಮ ಎಂದರೆ ಆ ಶ್ರೇಷ್ಠ ಆದರ್ಶದ ಅನ್ವೇಷಣೆ. ಮಾನವನ ಸಮಗ್ರ ಸ್ವಭಾವವು ಏನು ಎಂದು ತಾನು ಗ್ರಹಿಸಿದ್ದನೋ ಅದಕ್ಕೆ ಅನುಗುಣವಾಗಿ ಅವನು ಮೊದಲು ಸ್ವರ್ಗ ಮುಂತಾದ ಹೊರ ಜಗತ್ತಿನಲ್ಲಿ ಅನ್ವೇಷಣೆಗಳನ್ನು ನಡೆಸಿದನು.

ಅನಂತರ ಮಾನವನು ತನ್ನ ಕಡೆಗೇ ಸ್ವಲ್ಪ ಹತ್ತಿರದಿಂದ ನೋಡಲು ಯತ್ನಿಸಿದನು. ಆಗ ಸಾಧಾರಣವಾಗಿ ನಾವು ಯಾವುದನ್ನು `ನಾನು' ಎಂದು ಭಾವಿಸುವೆವೊ ಅದು ನಾನಲ್ಲ, ಎಂದು ಗೊತ್ತಾಗುವುದು. ಅದು ತನ್ನ ಇಂದ್ರಿಯಗಳಿಗೆ ಕಾಣುವಂತೆ ಇಲ್ಲ. ಅದನ್ನು ತನ್ನ ಒಳಗೆ ಹುಡುಕಲು ಯತ್ನಿಸಿದನು. ಕೊನೆಗೆ ತಾನು ಯಾವುದನ್ನು ಹೊರಗೆ ಹುಡುಕುತ್ತಿದನೋ ಅದು ತನ್ನಲ್ಲಿಯೇ ಸದಾ ಇದೆ ಎಂದು ಗೊತ್ತಾಗುವುದು. ತಾನು ಯಾವುದನ್ನು ಹೊರಗೆ ಪೂಜಿಸುತ್ತಿದ್ದನೊ, ಅದು ನಿಜವಾಗಿಯೂ ತನ್ನ ಆಂತರಿಕ ಸ್ವಾಭಾವವೇ. ಅದ್ವೈತಕ್ಕೂ, ದ್ವೈತಕ್ಕೂ ಇರುವ ವ್ಯತ್ಯಾಸ ಇಷ್ಟೇ – ಆದರ್ಶವನ್ನು ಹೊರಗೆ ಹುಡುಕುತ್ತಿದ್ದರೆ ದ್ವೈತ, ಅದನ್ನು ಒಳಗೆ ಹುಡುಕುತ್ತಿದ್ದರೆ ಅದ್ವೈತ.

ಈ ಜಗತ್ತು ಏತಕ್ಕೆ, ಎಲ್ಲಿಂದ ಬಂತು ಎಂಬ ಮೊದಲನೆ ಪ್ರಶ್ನೆ ಏಳುವುದು. ಮನುಷ್ಯ ಏತಕ್ಕೆ ಮಿತವಾಗುತ್ತಾನೆ? ಅನಂತ ಹೇಗೆ ಸಾಂತವಾಯಿತು? ಶುದ್ದವು ಅಶುದ್ಧ ಹೇಗೆ ಆಯಿತು? ಮೊದಲನೆಯದಾಗಿ ದ್ವೈತದ ಊಹೆಯ ಪ್ರಕಾರ ಈ ಪ್ರಶ್ನೆಗಳಿಗೆ ಉತ್ತರವನ್ನು ಹೇಳುವುದಕ್ಕೆ ಆಗುವುದಿಲ್ಲ ಎಂಬುದನ್ನು ನೆನಪಿನಲ್ಲಿಡಬೇಕು.

ದೇವರು ಏತಕ್ಕೆ ಇಂತಹ ಅಶುದ್ದವಾದ ಈ ಪ್ರಪಂಚವನ್ನು ಸೃಷ್ಟಿಸಿದ? ಪರಿಪೂರ್ಣನಾದ ದಯಾಮಯನಾದ ಅನಂತನಾದ ತಂದೆ, ಈ ಪ್ರಪಂಚವನ್ನು ಸೃಷ್ಟಿಸಿದ್ದರೆ ಮನುಷ್ಯ ಏತಕ್ಕೆ ಇಷ್ಟು ದುಃಖಿಯಾಗಿರುವನು? ಈ ಸ್ವರ್ಗ ನರಕಗಳೇಕೆ?ಇದನ್ನು ನೋಡಿಯೇ ನಾವು ನಿಯಮಾವಳಿಗಳನ್ನು ಊಹಿಸಿಕೊಳ್ಳುವೆವು. ತಾನು ನೋಡದೆ ಇರುವ ಯಾವುದನ್ನೂ ಯಾರೂ ಚಿತ್ರಿಸಿಕೊಳ್ಳಲಾರನು.

ಈ ಜೀವನದಲ್ಲಿ ನಾವು ಪಡುವ ಯಾತನೆಯನ್ನೆಲ್ಲ ಒಂದು ಸ್ಥಳದಲ್ಲಿಟ್ಟು, ಅದನ್ನು ನರಕ ಎಂದು ಕರೆಯುತ್ತೇವೆ.

ಅನಂತವಾದ ಪರಮೇಶ್ವರನು ಈ ಪ್ರಪಂಚವನ್ನು ಏತಕ್ಕೆ ಸೃಷ್ಟಿಸಿದನು? ದ್ವೈತಿಗಳು, ಕುಂಬಾರ ಮಡಕೆಕುಡಿಕೆಗಳನ್ನು ಮಾಡುವಂತೆ ಈ ಪ್ರಪಂಚವನ್ನು ದೇವರು ಮಾಡುತ್ತಾನೆ ಎನ್ನುವರು. ದೇವರು ಕುಂಬಾರ, ನಾವೇ ಮಡಕೆಗಳು... ತಾತ್ತ್ವಿಕ ರೀತಿ ಪ್ರಶ್ನೆ ಹೀಗೆ ಆಗುವುದು: ಮನುಷ್ಯನ ನಿಜವಾದ ಸ್ವಭಾವ ಪರಿಶುದ್ಧ, ಪರಿಪೂರ್ಣ ಮತ್ತು ಅನಂತ ಎಂದು ಕಲ್ಪಿಸಿಕೊಳ್ಳುವುದಕ್ಕೆ ಆಧಾರವೇನು? ಎಲ್ಲಾ ವಿಧವಾದ ಅದ್ವೈತ ಸಿದ್ಧಾಂತದಲ್ಲಿ ತೋರುವ ಒಂದು ತೊಡಕು ಇದು. ಉಳಿದವುಗಳೆಲ್ಲ ಸರಾಗವಾಗಿವೆ. ಈ ಪ್ರಶ್ನೆಗೆ ಉತ್ತರ ಕೊಡಲು ಸಾಧ್ಯವಿಲ್ಲ. ಅದ್ವೈತಿಗಳು, ಈ ಪ್ರಶ್ನೆಯೇ ವಿರೋಧಾಭಾಸ ಎನ್ನುವರು.

ದ್ವೈತಿಗಳ ದೃಷ್ಟಿಯನ್ನೇ ತೆಗೆದುಕೊಳ್ಳಿ. ದೇವರು ಈ ಪ್ರಪಂಚವನ್ನು ಏತಕ್ಕೆ ಸೃಷ್ಟಿಸಿದನು? ಇದು ವಿರೋಧೋಕ್ತಿ, ಏತಕ್ಕೆ? ಏಕೆಂದರೆ, ದೇವರ ಭಾವನೆ ಎಂದರೆ ಏನು? ಅವನು ತನ್ನಿಂದ ಹೊರಗೆ ಇರುವ ಯಾವುದರಿಂದಲೂ ಬಾಧಿತನಾಗದವನು.

ನಾನು, ನೀವು ಮುಕ್ತರಲ್ಲ. ನನಗೆ ಬಾಯಾರಿಕೆ ಎಂಬುದೊಂದು ಇದೆ. ಅದರ ಮೇಲೆ ನನಗೆ ಸ್ವಾಧೀನವಿಲ್ಲ. ಅದು ನನ್ನನ್ನು ನೀರು ಕುಡಿಯುವಂತೆ ಬಲಾತ್ಕರಿಸುತ್ತದೆ. ನನ್ನ ದೇಹ ಮತ್ತು ಮನಸ್ಸಿನಿಂದ ಆದ ಪ್ರತಿಯೊಂದು ಕ್ರಿಯೆಯೂ ಯಾರೋ ಬಲಾತ್ಕರಿಸಿದಂತೆ ಇದೆ. ನಾನು ಇದನ್ನು ಮಾಡಲೇಬೇಕಾಗಿದೆ. ಇದನ್ನು ಪಡೆಯಲೇ ಬೇಕಾಗಿದೆ. ಹೇಗೆ ಮತ್ತು ಏತರಿಂದ ಎಂಬ ಪ್ರಶ್ನೆಗೆ ಅರ್ಥವೇನು? ನೀವು ಏತಕ್ಕೆ ನೀರನ್ನು ಕುಡಿಯುವಿರಿ? ಏತಕ್ಕೆಂದರೆ ಬಾಯಾರಿಕೆಯು ನಿಮ್ಮನ್ನು ಬಲಾತ್ಕರಿಸುವುದು. ನೀವು ಗುಲಾಮರು, ಏಕೆಂದರೆ ನೀವು ಯಾವುದನ್ನೂ ಸ್ವಂತ ಇಚ್ಛೆಯಿಂದ ಮಾಡುವುದಿಲ್ಲ, ಬಲಾತ್ಕಾರದಿಂದ ಪ್ರೇರೇಪಿತರಾಗಿ ಎಲ್ಲವನ್ನೂ ಮಾಡುವಿರಿ. ನೀವು ಕೆಲಸ ಮಾಡುವುದಕ್ಕೆ ಕಾರಣ ಬಲಾತ್ಕಾರ.

ಭೂಮಿಯನ್ನು ಮತ್ಯಾವುದೋ ಚಲಿಸುವಂತೆ ಮಾಡದೆ ಇದ್ದರೆ ಅದು ಚಲಿಸುತ್ತಲೇ ಇರಲಿಲ್ಲ. ದೀಪ ಏತಕ್ಕೆ ಉರಿಯುವುದು? ಅದನ್ನು ಯಾರೂ ಕಡ್ಡಿಗೀರಿ ಹಚ್ಚದೆ ಇದ್ದರೆ ಅದು ಉರಿಯುತ್ತಿರಲಿಲ್ಲ. ಪ್ರಕೃತಿಯೊಂದಿಗೆ ಸಾಮರಸ್ಯದಿಂದ ಇರುವುದು ದಾಸ್ಯ. ಪ್ರಕೃತಿಯ ಗುಲಾಮನಾಗಿ ಒಂದು ಬಂಗಾರದ ಪಂಜರದಲ್ಲಿ ಇದ್ದರೆ ಏನು ಪ್ರಯೋಜನ? ಮಾನವ ಮುಖ್ಯವಾಗಿ ನಿತ್ಯಮುಕ್ತ, ಪವಿತ್ರ ಎಂಬ ಜ್ಞಾನದಲ್ಲೇ ಶ್ರೇಷ್ಠವಾದ ನಿಯಮ ಮತ್ತು ಕ್ರಮ ಇದೆ. ಹೇಗೆ, ಏತಕ್ಕೆ ಎಂಬ ಪ್ರಶ್ನೆಯನ್ನು ನಾವು ಅಜ್ಞಾನದಲ್ಲಿರುವಾಗ ಮಾತ್ರ ಕೇಳುವುದು ಎಂಬುದು ಗೊತ್ತಾಗುವುದು. ನಾನು ಮತ್ತಾವುದೋ ಬಲಾತ್ಕಾರಕ್ಕೆ ಸಿಕ್ಕಿಕೊಂಡು ಯಾವುದಾದರೂ ಒಂದನ್ನು ಮಾಡಬೇಕಾಗಿದೆ.

ದೇವರು ಸ್ವತಂತ್ರ ಎಂದು ನೀವು ಹೇಳುತ್ತೀರಿ, ಪುನಃ ದೇವರು ಏತಕ್ಕೆ ಪ್ರಪಂಚವನ್ನು ಸೃಷ್ಟಿಸಿದ ಎಂದು ಪ್ರಶ್ನಿಸುತ್ತೀರಿ. ನಿಮ್ಮ ಹೇಳಿಕೆಯನ್ನು ನೀವೇ ಪ್ರತಿಷೇಧಿಸುತ್ತಿರುವಿರಿ. ದೇವರೆಂದರೆ ಸರ್ವತಂತ್ರ ಸ್ವತಂತ್ರನೆಂದು ಅರ್ಥ. ತಾರ್ಕಿಕಭಾಷೆಯಲ್ಲಿ ಇದನ್ನು ಹೇಳಬೇಕಾದರೆ ಹೀಗಾಗುವುದು: ಯಾರನ್ನು ಯಾರೂ ಬಲಾತ್ಕರಿಸಲಾರರೊ, ಅವನನ್ನು ಈ ಪ್ರಪಂಚವನ್ನು ಸೃಷ್ಟಿಸುವಂತೆ ಯಾರು ಬಲಾತ್ಕರಿಸುವವರು? ನೀವು ಅದೇ ಪ್ರಶ್ನೆಯಲ್ಲಿ ಯಾರು ಅವನನ್ನು ಬಲಾತ್ಕರಿಸಿದರು ಎಂದು ಕೇಳುವಿರಿ? ಈ ಪ್ರಶ್ನೆಯೇ ಕುತರ್ಕ. ಅವನು ಸ್ವಭಾವತಃ ಅನಂತಾತ್ಮನಾಗಿರುವನು, ನಿತ್ಯಮುಕ್ತನಾಗಿರುವನು. ನೀವು ತಾರ್ಕಿಕವಾಗಿ ಪ್ರಶ್ನೆಯನ್ನು ಹಾಕಿದರೆ, ಆಗ ನಾವು ಅದಕ್ಕೆ ಉತ್ತರವನ್ನು ಹೇಳುತ್ತೇವೆ. ಈ ಪ್ರಪಂಚದಲ್ಲಿ ಒಂದೇ ಸತ್ಯ ಇರುವುದು, ಬೇರೆ ಏನೂ ಇಲ್ಲ ಎಂದು ಯುಕ್ತಿಯು\break ಹೇಳುವುದು. ಎಲ್ಲೆಲ್ಲಿ ದ್ವೈತ ತಲೆಯೆತ್ತುವುದೋ, ಅಲ್ಲೆಲ್ಲ ಅದ್ವೈತ ಅದರೊಂದಿಗೆ ಹೋರಾಡಿ ಅದನ್ನು ಆಚೆಗೆ ಅಟ್ಟಿದೆ.

ಇದನ್ನು ತಿಳಿದುಕೊಳ್ಳುವುದಕ್ಕೆ ಒಂದು ತೊಡಕಿದೆ. ಧರ್ಮ ಎಂದರೆ ವ್ಯವಹಾರಜ್ಞಾನ ಎಂಬುದು ಸರ್ವಸಾಮಾನ್ಯವಾಗಿರುವುದು. ಸಾಧಾರಣ ಮನುಷ್ಯನಿಗೆ ಧರ್ಮವನ್ನು ಅವನ ಭಾಷೆಯಲ್ಲಿ ಹೇಳಿದರೆ ಅವನಿಗೆ ಅರ್ಥವಾಗುವುದು. ಅದನ್ನು ಒಳ್ಳೆಯ ತತ್ತ್ವಜ್ಞಾನಿಯ ಭಾಷೆಯಲ್ಲಿ ಇಟ್ಟರೆ ಅರ್ಥವಾಗುವುದಿಲ್ಲ. ಮಾನವನ ಸ್ವಭಾವವೇ ಇತರರಲ್ಲಿ ತಮ್ಮನ್ನು ಆರೋಪಿಸಿಕೊಳ್ಳುವುದು. ಮಗುವಿನ ಮೇಲೆ ನಿಮಗೆ ಇರುವ ಭಾವವನ್ನು ಕುರಿತು ನೋಡಿ, ಮಗುವಿನ ಮೇಲೆ ನಿಮ್ಮನ್ನು ನೀವು ಆರೋಪಮಾಡಿಕೊಳ್ಳುತ್ತೀರಿ. ಆಮೇಲೆ ನಿಮಗೆ ಎರಡು ದೇಹಗಳು ಬರುತ್ತವೆ. ಅದರಂತೆಯೇ ಗಂಡನ ಮನಸ್ಸಿನ ಮೂಲಕ ನೀವು ಭಾವಿಸುವಿರಿ. ಇದಕ್ಕೆ ಒಂದು ಕೊನೆ ಎಲ್ಲಿ? ನೀವು ಅನಂತ ದೇಹಗಳಲ್ಲಿ ನಿಮ್ಮನ್ನು ಆರೋಪಿಸಿಕೊಳ್ಳಬಹುದು.

ಪ್ರತಿದಿನವೂ ಮನುಷ್ಯ ಪ್ರಕೃತಿಯನ್ನು ಗೆಲ್ಲುತ್ತಿರುವನು. ಮನುಷ್ಯ ಒಂದು ಜನಾಂಗದಂತೆ ತನ್ನ ಶಕ್ತಿಯನ್ನು ವ್ಯಕ್ತಪಡಿಸುತ್ತಿರುವನು. ಮನುಷ್ಯನ ಈ ಶಕ್ತಿಗೆ ಒಂದು ಮಿತಿ ಕಲ್ಪಿಸಲು ಸಾಧ್ಯವೇ, ವಿಚಾರಿಸಿ ನೋಡಿ, ಮನುಷ್ಯನಿಗೆ ಅವನ ಜನಾಂಗದ ದೃಷ್ಟಿಯಲ್ಲಿ ಅನಂತ ಶಕ್ತಿಗಳಿವೆ, ಅವನಿಗೆ ಅನಂತ ದೇಹಗಳು ಇವೆ ಎಂಬುದನ್ನು ಒಪ್ಪಿಕೊಳ್ಳುತ್ತೀರಿ. ನೀವು ಏನಾಗಿರುವಿರೋ ಅದೊಂದೇ ಪ್ರಶ್ನೆ. ನೀವು ಒಂದು ಜನಾಂಗವೇ ಅಥವಾ ಒಂದು ವ್ಯಕ್ತಿಯೆ? ನೀವು ಬೇರೆ ಆದೊಡನೆಯೇ ಪ್ರತಿಯೊಂದೂ ನಿಮಗೆ ವ್ಯಥೆಯನ್ನು ಕೊಡುವುದು. ನೀವು ವಿಕಾಸವಾಗಿ ಇತರರಿಗೆ ಅನುಕಂಪವನ್ನು ತೋರಿದೊಡನೆಯೇ ನಿಮಗೆ ಸಹಾಯ ಬರುವುದು. ಈ ಪ್ರಪಂಚದಲ್ಲಿ ಸ್ವಾರ್ಥಿಯೇ ಬಹಳ ದುಃಖಿ. ಯಾರು ಅತ್ಯಂತ ಸುಖಿಯೋ ಅವನಿಗೆ ಸ್ವಾರ್ಥವೆಂಬುದಿಲ್ಲ. ಅವನೇ ಸೃಷ್ಟಿಯೆಲ್ಲಾ ಆಗಿರುವನು. ಕ್ರೈಸ್ತರು, ಹಿಂದೂಗಳೂ ಮತ್ತು ಎಲ್ಲಾ ಧರ್ಮಗಳ ದ್ವೈತಿಗಳಲ್ಲಿಯೂ ನೀತಿ ಒಂದೇ. ಅದೇ ಸ್ವಾರ್ಥನಾಗಬೇಡ, ನಿಃಸ್ವಾರ್ಥಿಯಾಗಿರು, ಇತರರಿಗಾಗಿ ಕರ್ಮಮಾಡಿ ವಿಕಾಸವಾಗು ಎಂಬುದು.

ಇದನ್ನು ಅವಿದ್ಯಾವಂತರಿಗೆ ಅರ್ಥವಾಗುವಂತೆ ಸುಲಭವಾಗಿ ಹೇಳಬಹುದು.\break ವಿದ್ಯಾವಂತರಿಗೆ ಮತ್ತೂ ಸುಲಭವಾಗಿ ಅರ್ಥವಾಗುವಂತೆ ಹೇಳಬಹುದು. ಆದರೆ ಯಾರು ತೋರಿಕೆಯ ವಿದ್ಯಾವಂತರಂತೆ ಕಾಣುತ್ತಿರುವರೋ, ಅವರಿಗೆ ಗೊತ್ತಾಗುವಂತೆ ಹೇಳಲು ದೇವರಿಗೂ ಸಾಧ್ಯವಿಲ್ಲ. ಸತ್ಯಾಂಶವೇನೆಂದರೆ, ನೀವು ನಿಜವಾಗಿಯೂ ವಿಶ್ವದಿಂದ ಬೇರೆ ಅಲ್ಲ, ನಿಮ್ಮಿಂದ ನಿಮ್ಮ ಆತ್ಮ ಬೇರೆ ಹೇಗೆ ಅಲ್ಲವೋ ಹಾಗೆ. ಹಾಗೆ ಇಲ್ಲದೇ ಇದ್ದರೆ ನೀವು ಯಾವುದನ್ನೂ ನೋಡುವುದಕ್ಕೆ ಆಗುತ್ತಿರಲಿಲ್ಲ, ಅನುಭವಿಸುವುದಕ್ಕೆ ಆಗುತ್ತಿರಲಿಲ್ಲ. ಪಂಚಭೂತಗಳ ಮಹಾಸಾಗರದಲ್ಲಿ ನಮ್ಮ ದೇಹಗಳೆಂಬುವು ಸಣ್ಣ ಸುಳಿಗಳಂತೆ. ಜೀವನವು ತಿರುಗಿ ಬೇರೊಂದು ರೂಪವನ್ನು ಧರಿಸಿ ಸಾಗಿಹೋಗುತ್ತಿದೆ. ಸೂರ್ಯ ಚಂದ್ರ ತಾರೆ ನೀವು ನಾವುಗಳೆಲ್ಲ ಸಣ್ಣ ಸಣ್ಣ ಸುಳಿಗಳು, ಅಷ್ಟೆ. ನಾನು ಯಾವುದೋ ಒಂದು ಮನಸ್ಸನ್ನು ಏತಕ್ಕೆ ಆರಿಸಿಕೊಂಡೆ? ಮನಸ್ಸಿನ ಸರೋವರದಲ್ಲಿ ಇದೊಂದು ಮನೋಸುಳಿಯಷ್ಟೆ.

ಇಲ್ಲದೆ ಇದ್ದರೆ ನನ್ನ ಮನಸ್ಸಿನ ಸ್ಪಂದನ ನಿಮಗೆ ಹೇಗೆ ತಲುಪುತ್ತಿತ್ತು? ನೀವು ಸರೋವರಕ್ಕೆ ಒಂದು ಕಲ್ಲನ್ನು ಎಸೆದರೆ ಅದೊಂದು ಸ್ಪಂದನವನ್ನು ಉಂಟುಮಾಡುವುದು. ಅದು ನೀರನ್ನೆಲ್ಲ ಸ್ಪಂದಿಸುವಂತೆ ಮಾಡುವುದು. ನಾನು ಮನಸ್ಸನ್ನು ಒಂದು ಆನಂದದ ಸ್ಥಿತಿಗೆ ಒಯ್ಯುತ್ತೇನೆ. ಅದರಿಂದ ನಿಮ್ಮ ಮನಸ್ಸನ್ನು ಅದೇ ಆನಂದದ ಸ್ಥಿತಿಗೆ ಒಯ್ಯಬಹುದು. ನೀವು ಎಷ್ಟೋ ವೇಳೆ, ನಿಮ್ಮ ಮನಸ್ಸಿನಲ್ಲಿ ಅಥವಾ ಹೃದಯದಲ್ಲಿ ಯಾರಿಗೂ ಶಬ್ದದ ಮೂಲಕ ವಿವರಿಸದೆ, ಏನನ್ನೋ ಆಲೋಚಿಸಿರಬಹುದು. ಅದು ಮತ್ತೊಬ್ಬರಿಗೆ ಗೊತ್ತಾಗುತ್ತದೆ. ಎಲ್ಲಾ ಕಡೆಯಲ್ಲಿಯೂ ನಾವು ಒಂದೇ. ನಾವು ಇದನ್ನು ಅರ್ಥಮಾಡಿಕೊಳ್ಳುವುದೇ ಇಲ್ಲ. ಈ ಪ್ರಪಂಚವೆಲ್ಲ ದೇಶ ಕಾಲ ನಿಮಿತ್ತಗಳಿಂದ ಆಗಿದೆ. ದೇವರು ಈ ಪ್ರಪಂಚದಂತೆ ತೋರುತ್ತಾನೆ. ಪ್ರಕೃತಿ ಯಾವಾಗ ಪ್ರಾರಂಭವಾಯಿತು? ನಿಮ್ಮ ನೈಜ ಸ್ವಭಾವವನ್ನು ಮರೆತು, ಕಾಲ ದೇಶ ನಿಮಿತ್ತದೊಂದಿಗೆ ಬದ್ಧರಾದಾಗ!

ಅದೇ ಸುತ್ತುತ್ತಿರುವ ನಿಮ್ಮ ದೇಹಗಳ ವೃತ್ತ. ಆದರೂ ಅದು ನಿಮ್ಮ ಅನಂತ ಸ್ವಭಾವವಾಗಿದೆ. ಅದೇ ಪ್ರಕೃತ, ಕಾಲದೇಶನಿಮಿತ್ತಗಳು. ಪ್ರಕೃತಿ ಎಂದರೆ ಇಷ್ಟೇ. ನೀವು ಆಲೋಚಿಸುವುದಕ್ಕೆ ಪ್ರಾರಂಭಿಸಿದಾಗ ಕಾಲ ಪ್ರಾರಂಭವಾಯಿತು. ನಿಮಗೆ ದೇಹ ಬಂದಾಗ ದೇಶ ಪ್ರಾರಂಭವಾಯಿತು. ಇಲ್ಲದೇ ಇದ್ದರೆ ದೇಶವೆಂಬುದೇ ಇಲ್ಲ. ನೀವು ಯಾವಾಗ ಮಿತಿಗೊಳಪಟ್ಟಿರೊ ಆಗ ಕಾರಣ ಪ್ರಾರಂಭವಾಯಿತು. ನಮಗೆ ಯಾವುದಾದರೂ ಒಂದು ಬಗೆಯ ಉತ್ತರ ಬೇಕಾಗಿದೆ. ಉತ್ತರ ಅಲ್ಲಿಯೇ ಇದೆ. ನಾವು ಮಿತಿಗೆ ಒಳಗಾಗಿರುವೆವು ಎಂಬುದೇ ಒಂದು ಆಟ. ಇದು ಬರೀ ತಮಾಷೆ. ನಿಮ್ಮನ್ನು ಯಾವುದೂ ಬಂಧಿಸಲಾರದು. ಯಾವುದೂ ನಿಮ್ಮನ್ನು ಬಲಾತ್ಕರಿಸಲಾರದು. ನಾವೇ ಸೃಷ್ಟಿಸಿದ ನಾಟಕದಲ್ಲಿ ನಾವೆಲ್ಲ ನಮ್ಮ ಪಾತ್ರಗಳನ್ನು ಅಭಿನಯಿಸುತ್ತಿರುವೆವು.

ವ್ಯಕ್ತಿತ್ವಕ್ಕೆ ಸಂಬಂಧಪಟ್ಟ ಮತ್ತೊಂದು ಪ್ರಶ್ನೆಯನ್ನು ತರೋಣ. ಕೆಲವರಿಗೆ ತಮ್ಮ ವ್ಯಕ್ತಿತ್ವವನ್ನು ಕಳೆದುಕೊಳ್ಳುವ ಅಂಜಿಕೆ. ಒಂದು ಹಂದಿ ದೇವರಾದರೆ ತನ್ನ ಹಂದಿತನವನ್ನು ಕಳೆದುಕೊಳ್ಳುವುದು ಒಳ್ಳೆಯದೆ? ಒಳ್ಳೆಯದು. ಆದರೆ ಪಾಪ, ಸದ್ಯಕ್ಕೆ ಹಂದಿಯು ಹಾಗೆ ಭಾವಿಸುವುದಿಲ್ಲ. ಯಾವುದು ನಿಜವಾದ ನನ್ನ ವ್ಯಕ್ತಿತ್ವ? ನಾನು ಮಗುವಾಗಿ ನೆಲದ ಮೇಲೆ ಹೊರಳಾಡುತ್ತಿದ್ದಾಗ ಬೆರಳನ್ನು ಚೀಪುತ್ತಿದ್ದ ಕಾಲವೇ? ಆ ವ್ಯಕ್ತಿತ್ವವನ್ನು ಕಳೆದುಕೊಳ್ಳುವುದಕ್ಕೆ ನಾನು ವ್ಯಥೆಪಡಬೇಕೆ? ಐವತ್ತು ವರುಷಗಳಾದ ಮೇಲೆ ನನ್ನ ಈಗಿನ ಸ್ಥಿತಿಯನ್ನು ನೆನಸಿಕೊಂಡು ನಾನು ನಗುವೆನು, ಈಗ ನಾನು ನನ್ನ ಹಿಂದಿನ ಮನಸ್ಸಿನ ಸ್ಥಿತಿಯನ್ನು ನೋಡಿ ನಗುವಂತೆ. ನಾನು ಇವುಗಳಲ್ಲಿ ಯಾವ ವ್ಯಕ್ತಿತ್ವವನ್ನು ಕಾಪಾಡಿಕೊಳ್ಳಬೇಕಾಗಿದೆ?

ಈ ವ್ಯಕ್ತಿತ್ವ ಎಂದರೆ ಏನು ಎಂಬುದನ್ನು ನಾವು ತಿಳಿದುಕೊಳ್ಳಬೇಕಾಗಿದೆ. ನಮ್ಮಲ್ಲಿ ಎರಡು ಪರಸ್ಪರ ವಿರೋಧವಾದ ಸ್ವಭಾವಗಳಿವೆ. ಒಂದು ವ್ಯಕ್ತಿತ್ವವನ್ನು ರಕ್ಷಿಸುವುದಕ್ಕೆ, ಮತ್ತೊಂದು ವ್ಯಕ್ತಿತ್ವವನ್ನು ತ್ಯಜಿಸುವುದಕ್ಕೆ ಇರುವ ಉತ್ಕಟ ಆಕಾಂಕ್ಷೆ. ತಾಯಿ ತನ್ನ ಮಗುವಿನ ಅವಶ್ಯಕತೆಗೆ ತನ್ನ ಇಚ್ಛೆಯನ್ನೆಲ್ಲ ಅರ್ಪಣೆ ಮಾಡುವಳು. ಅವಳು ತನ್ನ ಕೈಗಳಲ್ಲಿ ಮಗುವನ್ನು ಎತ್ತಿಕೊಂಡಿರುವಾಗ ವ್ಯಕ್ತಿತ್ವದ ಕರೆಯನ್ನು, ತನ್ನನ್ನು ರಕ್ಷಿಸಿಕೊಳ್ಳಬೇಕು ಎಂಬ ಸ್ವಾರ್ಥ ಭಾವನೆಯನ್ನು ಕೇಳುವುದೇ ಇಲ್ಲ. ತಾನು ಕದಾನ್ನವನ್ನುಂಡು ಮೃಷ್ಟಾನ್ನವನ್ನು ಮಗುವಿಗೆ ಬಡಿಸುವಳು. ಇದರಂತೆಯೇ ನಾವು ಪ್ರೀತಿಸುವ ಜನರಿಗಾಗಿ ಸಾಯಲು ಸಿದ್ಧರಾಗಿರುವೆವು.

ಒಂದು ಕಡೆ ನಮ್ಮ ವ್ಯಕ್ತಿತ್ವವನ್ನು ಕಾಪಾಡಿಕೊಳ್ಳುವುದಕ್ಕಾಗಿ ಬಹಳ ಕಷ್ಟ ಪಡುತ್ತಿರುವೆವು. ಮತ್ತೊಂದು ಕಡೆ ಅದನ್ನು ನಾಶಮಾಡಲು ಯತ್ನಿಸುತ್ತಿರುವೆವು. ಇದರ ಪರಿಣಾಮವೇನು? ಟಾಮ್ ಬ್ರೌನ್ ಬಹಳ ಕಷ್ಟಪಡುತ್ತಿರಬಹುದು. ಅವನು ತನ್ನ ವ್ಯಕ್ತಿತ್ವಕ್ಕೆ ಹೋರಾಡುತ್ತಿರುವನು. ಅವನು ಸಾಯುವನು. ಪ್ರಪಂಚಕ್ಕೆ ಯಾವ ರೀತಿಯಲ್ಲಿಯೂ ನಷ್ಟ ಕಾಣಿಸುವುದಿಲ್ಲ. ಹತ್ತೊಂಭತ್ತು ನೂರು ವರುಷಗಳ ಹಿಂದೆ ಯಹೂದಿ ಒಬ್ಬ ಹುಟ್ಟಿದನು. ಅವನು ತನ್ನ ವ್ಯಕ್ತಿತ್ವವನ್ನು ರಕ್ಷಿಸಿಕೊಳ್ಳಲು ಒಂದು ಬೆರಳನ್ನೂ ಎತ್ತಲಿಲ್ಲ. ಇದನ್ನು ಕುರಿತು ಯೋಚಿಸಿ ನೋಡಿ. ಆ ಯಹೂದಿ ತನ್ನ ವ್ಯಕ್ತಿತ್ವದ ರಕ್ಷಣೆಗೆ ಸ್ವಲ್ಪವೂ ಪ್ರಯತ್ನಿಸಲಿಲ್ಲ. ಆದಕಾರಣವೇ ಆ ಮನುಷ್ಯ ಪ್ರಪಂಚದಲ್ಲಿ ಶ್ರೇಷ್ಠತಮನಾದ ವ್ಯಕ್ತಿಯಾದ, ಪ್ರಪಂಚಕ್ಕೆ ಗೊತ್ತಿಲ್ಲದೇ ಇರುವ ಸಂಗತಿಯೇ ಇದು.

ಕಾಲಕ್ರಮೇಣ ನಾವು ಒಂದು ವ್ಯಕ್ತಿಯಾಗಬೇಕಾಗಿದೆ. ಆದರೆ ಅದು ಯಾವ ದೃಷ್ಟಿಯಲ್ಲಿ? ಮನುಷ್ಯನ ವ್ಯಕ್ತಿತ್ವ ಏನು? ಅದು ಟಾಮ್ ಬ್ರೌನ್ ಅಲ್ಲ, ಅವನಲ್ಲಿರುವ ದೇವರು. ಅದೇ ನಿಜವಾದ ವ್ಯಕ್ತಿತ್ವ. ಮನುಷ್ಯ ಎಷ್ಟು ಮಟ್ಟಿಗೆ ಅದನ್ನು ಸಮೀಪಿಸಿರುವನೋ, ಅಷ್ಟೂ ತನ್ನ ತೋರಿಕೆಯ ವ್ಯಕ್ತಿತ್ವವನ್ನು ಅವನು ತ್ಯಜಿಸಿರುವನು. ಅವನು ತನಗಾಗಿ ಎಲ್ಲವನ್ನೂ ಸಂಗ್ರಹಿಸಿ ರಕ್ಷಿಸಿಕೊಳ್ಳುವುದಕ್ಕೆ ಹೆಚ್ಚು ಪ್ರಯತ್ನ ಪಟ್ಟಷ್ಟೂ, ಅವನಲ್ಲಿ ವ್ಯಕ್ತಿತ್ವ ಅಷ್ಟೂ ಕಡಿಮೆಯಾಗಿರುವುದು. ತನ್ನನ್ನು ಕುರಿತು ಎಷ್ಟು ಕಡಿಮೆ ಯೋಚಿಸುವನೋ, ಅವನು ಅಷ್ಟು ಹೆಚ್ಚಾಗಿ ತನ್ನ ವ್ಯಕ್ತಿತ್ವವನ್ನು ತನ್ನ ಜೀವನದಲ್ಲಿ ಧಾರೆಯೆರೆದಿರುವನು. ಈ ಒಂದು ರಹಸ್ಯವನ್ನು ಪ್ರಪಂಚವು ಅರ್ಥ ಮಾಡಿಕೊಂಡಿಲ್ಲ.

ಆ ವ್ಯಕ್ತಿತ್ವ ಎಂದರೆ ಏನು ಎಂಬುದನ್ನು ಮೊದಲು ನಾವು ಅರಿಯಬೇಕಾಗಿದೆ. ಅದು ಆದರ್ಶವನ್ನು ಪಡೆಯುವುದು. ಈಗ ನೀವು ಪುರುಷನೋ ಸ್ತ್ರೀಯೊ ಆಗಿರುವಿರಿ. ಆ ನೀವು ಯಾವಾಗಲೂ ಬದಲಾಯಿಸುತ್ತಿರುವಿರಿ. ನೀವು ಅದನ್ನು ನಿಲ್ಲಿಸಲು ಸಾಧ್ಯವೆ? ನಿಮ್ಮ ಮನಸ್ಸು ಈಗ ಹೇಗೆ ಇದೆಯೋ ಹಾಗೆಯೇ ಇಟ್ಟುಕೊಂಡಿರಲು ಸಾಧ್ಯವೇ? ಮನಸ್ಸಿನಲ್ಲಿರುವ ಕೋಪ ದ್ವೇಷ ಅಸೂಯೆ ಜಗಳ ಮುಂತಾದ ಸಾವಿರಾರು ಲೋಪದೋಷಗಳನ್ನು ಹಾಗೆಯೇ ಇಟ್ಟುಕೊಂಡಿರಬೇಕೆಂದು ಬಯಸುವಿರಾ?... ಪೂರ್ಣ ಜಯ ಸಿದ್ದಿಸುವವರೆಗೆ, ನೀವು ಪರಿಪೂರ್ಣರಾಗುವವರೆಗೆ ಮಧ್ಯದಲ್ಲಿ ಎಲ್ಲಿಯೂ ನಿಲ್ಲುವುದಕ್ಕೆ ಆಗುವುದಿಲ್ಲ.

ನೀವೇ ಎಲ್ಲಾ ಪ್ರೇಮ, ಆನಂದ ಅನಂತ ಅಸ್ತಿತ್ವ ಆದಮೇಲೆ ನಿಮ್ಮಲ್ಲಿ ಯಾವ ಕೋಪವೂ ಇರಲಾರದು. ನಿಮ್ಮ ಯಾವ ದೇಹವನ್ನು ನೀವು ಇಟ್ಟುಕೊಳ್ಳುವಿರಿ? ಎಂದಿಗೂ ನಾಶವಾಗದ ಜೀವನ ಮತ್ತು ಆನಂದ ಪ್ರಾಪ್ತವಾಗುವವರೆಗೆ ನೀವು ಮಧ್ಯದಲ್ಲಿ ಎಲ್ಲಿಯೂ ನಿಲ್ಲಲಾರಿರಿ. ಗುರಿ ಮುಟ್ಟಿದಾಗ ಮಾತ್ರ ನೀವು ನಿಲ್ಲುವಿರಿ. ಈಗ ನಿಮಗೆ ಸ್ವಲ್ಪ ಜ್ಞಾನವಿದೆ. ಯಾವಾಗಲೂ ಅದನ್ನು ವೃದ್ಧಿ ಮಾಡಿಕೊಳ್ಳಲು ಯತ್ನಿಸುತ್ತಿರುವಿರಿ. ನೀವು ನಿಲ್ಲುವುದೆಲ್ಲಿ? ಅನಂತ ಜೀವನವೇ ನೀವಾಗುವವರೆಗೆ ಮಧ್ಯದಲ್ಲಿ ನಿಲ್ಲಲಾರಿರಿ.

ಅನೇಕರು ಸುಖವನ್ನೇ ಗುರಿಯನ್ನಾಗಿ ಇಚ್ಛಿಸುವರು. ಅದಕ್ಕಾಗಿ ಅವರು ಇಂದ್ರಿಯ ಸುಖಗಳನ್ನು ಬಯಸುವರು. ಅದಕ್ಕಿಂತ ಮೇಲಿನ ಕ್ಷೇತ್ರದಲ್ಲಿ ಮತ್ತೂ ಉತ್ಕೃಷ್ಟವಾದ ಸುಖವನ್ನು ಅರಸಬೇಕಾಗಿದೆ. ಅದಾದ ಮೇಲೆ ಆಧ್ಯಾತ್ಮಿಕ ಕಾರ್ಯಕ್ಷೇತ್ರ. ಅನಂತರವೆ, ಅವನಲ್ಲಿರುವ ಭಗವಂತನಲ್ಲಿ. ಯಾರ ಸುಖ ಬಾಹ್ಯ ವಸ್ತುವಿನ ಆಸರೆಯ ಮೇಲಿದೆಯೋ ಆ ಸುಖ ಹೋದರೆ ಅವನು ವ್ಯಥೆಪಡುವನು. ಈ ಸುಖ (ಆತ್ಮಾನಂದ) ಬಾಹ್ಯ ವಸ್ತುಗಳ ಆಸರೆಯ ಮೇಲೆ ನಿಂತುದಲ್ಲ. ನನ್ನ ಸುಖವೆಲ್ಲ ನನ್ನಲ್ಲಿದ್ದರೆ, ಅಲ್ಲಿ ನನಗೆ ಸದಾಕಾಲದಲ್ಲಿಯೂ ಸುಖವಿರಬೇಕು. ಏಕೆಂದರೆ ಯಾವ ಸಮಯದಲ್ಲಿಯೂ ನನ್ನನ್ನು ನಾನು ಕಳೆದುಕೊಳ್ಳಲಾರೆ. ತಾಯಿ, ತಂದೆ, ಹೆಂಡತಿ, ಮಕ್ಕಳು, ದೇಹ, ಐಶ್ವರ್ಯ ಇವುಗಳನ್ನೆಲ್ಲಾ ನಾನು ಕಳೆದುಕೊಳ್ಳಬಹುದು, ಆದರೆ ನನ್ನನ್ನು ನಾನು ಕಳೆದುಕೊಳ್ಳಲಾರೆ, ನನ್ನಲ್ಲಿರುವ ಆನಂದವನ್ನು ಕಳೆದುಕೊಳ್ಳಲಾರೆ. ಎಲ್ಲಾ ಬಯಕೆಗಳೂ ಆತ್ಮವನ್ನು ಅವಲಂಬಿಸಿರುವುವು. ಎಂದಿಗೂ ಬದಲಾವಣೆ ಆಗದ ವ್ಯಕ್ತಿತ್ವವೇ ಇದು. ಇದೇ ಪರಿಪೂರ್ಣವಾಗಿರುವುದು.

ಇದನ್ನು ಪಡೆಯುವುದು ಹೇಗೆ? ಈ ಪ್ರಪಂಚದ ಮಹಾವ್ಯಕ್ತಿಗಳು, ಶ್ರೇಷ್ಠರಾದ ಪುರುಷರು ಮತ್ತು ಸ್ತ್ರೀಯರು, ಆತ್ಮ ಅನಾತ್ಮ ವಿವೇಕದ ಮೂಲಕ ಯಾವುದನ್ನು ಪಡೆದರೂ ಅದನ್ನು ಎಲ್ಲರೂ ಪಡೆಯುವರು. ಇಪ್ಪತ್ತು ಮೂವತ್ತು ದೇವರುಗಳನ್ನೊಳಗೊಂಡ ದ್ವೈತ ಸಿದ್ದಾಂತ ಏನಾಗಬೇಕು? ಚಿಂತೆ ಇಲ್ಲ. ಈ ತೋರಿಕೆಯ ವ್ಯಕ್ತಿತ್ವವು ಹೋಗಬೇಕು ಎಂಬ ಒಂದು ಸಾಮಾನ್ಯ ಸತ್ಯವು ಅವುಗಳಲ್ಲಿ ಇತ್ತು. ಈ ದೇಹಭಾವ ಎಷ್ಟು ಕಡಿಮೆ ಇದ್ದರೆ, ನನ್ನ ಪ್ರತ್ಯೇಕವಾದ ಅಹಂ ಎಷ್ಟು ಕಡಿಮೆ ಆದರೆ, ಅಷ್ಟೂ ನಾನು ನನ್ನ ನಿಜವಾದ ಅಸ್ತಿತ್ವಕ್ಕೆ – ವಿಶ್ವದೇಹಕ್ಕೆ – ಹತ್ತಿರವಾಗುತ್ತೇನೆ. ನನ್ನ ವೈಯಕ್ತಿಕ ಮನಸ್ಸನ್ನು ಕುರಿತು ಎಷ್ಟು ಕಡಮೆ ಚಿಂತಿಸಿದರೆ ಅಷ್ಟು ಹೆಚ್ಚಾಗಿ ನಾನು ವಿಶ್ವಮಾನಸಕ್ಕೆ ಹತ್ತಿರವಾಗುತ್ತೇನೆ. ನಾನು ನನ್ನದೇ ಆತ್ಮವನ್ನು ಕುರಿತು ಎಷ್ಟು ಕಡಿಮೆ ಆಲೋಚಿಸುತ್ತೇನೋ ಅಷ್ಟೂ ವಿಶ್ವಾತ್ಮಕ್ಕೆ ಸಮೀಪವಾಗುತ್ತೇನೆ.

ನಾವು ಒಂದು ದೇಹದಲ್ಲಿ ವಾಸಿಸುವೆವು. ನಮಗೆ ಕೆಲವು ಸುಖ ದುಃಖಗಳಿವೆ. ಈ ದೇಹದಲ್ಲಿರುವ ಅಲ್ಪ ಸುಖಕ್ಕಾಗಿ, ನಮ್ಮನ್ನು ಸಂರಕ್ಷಿಸಿಕೊಳ್ಳುವುದಕ್ಕಾಗಿ ಪ್ರಪಂಚದಲ್ಲಿ ಎಲ್ಲರನ್ನೂ ನಾಶಮಾಡಲು ಸಿದ್ಧರಾಗಿರುವೆವು. ನಮಗೆ ಎರಡು ದೇಹಗಳಿದ್ದರೆ ಅದು ಮತ್ತೂ ಉತ್ತಮವಾಗುತ್ತಿರಲಿಲ್ಲವೇ? ಹೀಗೆಯೆ ಆನಂದವನ್ನು ಪಡೆಯುವವರೆಗೆ ಮುಂದುವರಿಯಬೇಕು. ನಾನು ಎಲ್ಲ ದೇಹದಲ್ಲಿಯೂ ಇರುವೆನು. ನಾನು ಎಲ್ಲ ಕೈಗಳ ಮೂಲಕ ಕೆಲಸಮಾಡುತ್ತೇನೆ, ಎಲ್ಲ ಕಾಲುಗಳ ಮೂಲಕ ನಡೆಯುತ್ತೇನೆ. ನಾನು ಎಲ್ಲಾ ಬಾಯಿಗಳ ಮೂಲಕ ಮಾತನಾಡುತ್ತೇನೆ. ನನ್ನ ದೇಹಗಳು ಅನಂತ, ಮನಸ್ಸು ಅನಂತ. ನಾನು ನಜರಿತ್ತನ ಏಸುವಿನಲ್ಲಿದ್ದೆ, ಬುದ್ಧನಲ್ಲಿದ್ದೆ, ಮಹಮ್ಮದನಲ್ಲಿದ್ದೆ, ಎಲ್ಲಾ ಶ್ರೇಷ್ಠರಾದ ಹಿಂದಿನ ಮಹಾಮಹಿಮರಲ್ಲಿ ನಾನಿದ್ದೆನು. ಅನಂತರ ಬರುವವರಲ್ಲಿಯೂ ನಾನಿರುವೆನು. ಇದೇನು ಬರೀ ಸಿದ್ಧಾಂತವೇ? ಇಲ್ಲ ಇದು ಪ್ರತ್ಯಕ್ಷ ಸತ್ಯ.

ನೀವು ಇದನ್ನು ಸಾಕ್ಷಾತ್ಕಾರ ಮಾಡಿಕೊಂಡರೆ ಆಗ ಇನ್ನೂ ಅಷ್ಟು ಕೋಟಿಪಾಲು ಆನಂದವು ಜಾಸ್ತಿಯಾಗುವುದಿಲ್ಲವೆ? ಅದು ಎಂತಹ ಆನಂದ ಪರವಶತೆಯ ಸ್ಥಿತಿ! ಈ ಪ್ರಪಂಚದಲ್ಲಿ ಯಾವ ದೇಹವನ್ನು ಶ್ರೇಷ್ಠ ಎಂದು ಭಾವಿಸಿ ಅದನ್ನು ಇಟ್ಟಿಕೊಂಡಿರುವುದು? ಈ ಪ್ರಪಂಚದಲ್ಲಿರುವ ಎಲ್ಲಾ ದೇಹಗಳ ಮೂಲಕ ಅನುಭವಿಸಿ ಆದಮೇಲೆ ನಮ್ಮ ಸ್ಥಿತಿ ಏನಾಗುವುದು? ನಾವು ಅನಂತಾತ್ಮನಲ್ಲಿ ಒಂದಾಗುತ್ತೇವೆ. ಅದೇ ಗುರಿ. ಅದೊಂದೇ ದಾರಿ. ಒಬ್ಬ ತನಗೆ ಸತ್ಯ ಗೊತ್ತಾದರೆ ಬೆಣ್ಣೆಯಂತೆ ಕರಗಿ ಹೋಗುತ್ತೇನೆಂದು ಹೇಳುತ್ತಾನೆ. ಜನ ಹಾಗೆ ಆಗಲೆಂದು ನಾನು ಆಶಿಸುತ್ತೇನೆ. ಆದರೆ ಅವರು ಬೆಣ್ಣೆಯಂತೆ ಕರಗಿ ಹೋಗುವುದು ಅಷ್ಟು ಸುಲಭವಲ್ಲ!

ನಾವು ಮುಕ್ತರಾಗುವುದಕ್ಕೆ ಏನು ಮಾಡಬೇಕು? ನಾವು ಆಗಲೇ ಮುಕ್ತರು, ಮುಕ್ತ ಎಂದಾದರೂ ಬದ್ಧನಾಗಿರುವುದು ಹೇಗೆ? ಇದು ಸುಳ್ಳು. ನಾನು ಎಂದಿಗೂ ಬದ್ಧನಾಗಿರಲಿಲ್ಲ. ಅನಂತ ಯಾವುದರಿಂದಲಾದರೂ ಹೇಗೆ ಸಾಂತವಾದೀತು? ಅನಂತವನ್ನು ಅನಂತದಿಂದ ಭಾಗಿಸಿದರೆ, ಅನಂತಕ್ಕೆ ಅನಂತವನ್ನು ಸೇರಿಸಿದರೆ, ಅನಂತದಿಂದ ಗುಣಿಸಿದರೆ ಅನಂತವೇ ಆಗುವುದು. ನೀನು ಅನಂತ. ದೇವರು ಅನಂತ. ನೀವೆಲ್ಲ ಅನಂತ, ಎರಡು ಅಸ್ತಿತ್ವ ಇರಲಾರದು. ಒಂದೇ ಇರಬಲ್ಲದು. ಅನಂತವನ್ನು ಎಂದಿಗೂ ಸಾಂತಗೊಳಿಸಲಾಗುವುದಿಲ್ಲ. ನೀವು ಎಂದಿಗೂ ಬದ್ಧರಲ್ಲ. ನೀವಾಗಲೇ ನಿತ್ಯ ಮುಕ್ತರು, ಯಾವ ಗುರಿಯನ್ನು ಸೇರಬೇಕೋ ಅದನ್ನು ಆಗಲೇ ಸೇರಿರುವಿರಿ. ನಾನು ಗುರಿಯನ್ನು ಸೇರಿಲ್ಲ ಎಂದು ಮನಸ್ಸು ಚಿಂತಿಸುವುದಕ್ಕೆ ಎಂದಿಗೂ ಅವಕಾಶವನ್ನು ಕೊಡಬೇಡಿ.

ನಾವು ಭಾವಿಸಿದಂತೆ ಆಗುವೆವು. ನೀವು ಕೆಲಸಕ್ಕೆ ಬಾರದ ಪಾಪಿಗಳು ಎಂದು ಭಾವಿಸಿದರೆ ನಿಮ್ಮನ್ನು ನೀವೇ ಭ್ರಾಂತಿಗೆ ಒಳಪಡಿಸಿಕೊಳ್ಳುತ್ತೀರಿ. `ನೀವೊಂದು ಕೆಲಸಕ್ಕೆ ಬಾರದ ತೆವಳುತ್ತಿರುವ ಕೀಟ'ವಾಗುತ್ತೀರಿ. ಯಾರು ನರಕವನ್ನು ನಂಬುವರೋ ಅವರು ಸತ್ತಮೇಲೆ ನರಕಕ್ಕೆ ಹೋಗುವರು. ಯಾರು ತಾವು ಸ್ವರ್ಗಕ್ಕೆ ಹೋಗುವೆವು ಎಂದು ಭಾವಿಸುವರೋ ಅವರು ಸ್ವರ್ಗಕ್ಕೆ ಹೋಗುತ್ತಾರೆ.

ಇದೆಲ್ಲ ಒಂದು ಆಟ. ನಾವು ಏನನ್ನಾದರೂ ಮಾಡಬೇಕಾಗಿದೆ, ಅದಕ್ಕಾಗಿ ಒಳ್ಳೆಯದನ್ನು ಮಾಡೋಣ ಎಂದು ನೀವು ಹೇಳಿಕೊಳ್ಳಬಹುದು. ಆದರೆ ಒಳ್ಳೆಯದು ಕೆಟ್ಟದನ್ನು ಯಾರು ಗಮನಿಸುವರು? ಇದೆಲ್ಲ ಒಂದು ಆಟ. ಪರಮೇಶ್ವರನು ಆಡುತ್ತಿರುವನು ಅಷ್ಟೇ. ನೀವೇ ಆಡುತ್ತಿರುವ ಪರಮೇಶ್ವರ, ಆಟದಲ್ಲಿ ನೀವು ಭಾಗವಹಿಸಬೇಕೆಂದು ಭಿಕ್ಷುಕನ ಪಾತ್ರವನ್ನು ಸ್ವೀಕರಿಸಿದರೆ, ನೀವು ಅದಕ್ಕಾಗಿ ಮತ್ತೊಬ್ಬನನ್ನು ದೂರಲಾರಿರಿ. ಭಿಕ್ಷುಕನಂತೆಯೇ ಆನಂದಿಸಿರಿ. ನಿಮ್ಮ ನೈಜಸ್ವಭಾವ ದಿವ್ಯವಾದುದು ಎಂಬುದು ನಿಮಗೆ ಗೊತ್ತಿದೆ. ನೀವೇ ರಾಜರು; ಭಿಕ್ಷುಕನಂತೆ ಪಾತ್ರವಹಿಸಿರುವಿರಿ. ಇದೆಲ್ಲ ಒಂದು ತಮಾಷೆ. ಇದನ್ನು ಅರಿತು ಆಟವಾಡಿ. ನಾವು ತಿಳಿದುಕೊಳ್ಳಬೇಕಾದ ರಹಸ್ಯ ಇಷ್ಟೇ. ಅನಂತರ ಸಾಧನೆ ಮಾಡಿ. ಈ ಸೃಷ್ಟಿಯೇ ಒಂದು ಬೃಹತ್ ಲೀಲೆ. ಇದೆಲ್ಲ ಒಳ್ಳೆಯದೆ, ಏಕೆಂದರೆ ಇದೆಲ್ಲ ಒಂದು ತಮಾಷೆ. ಯಾವುದೋ ಒಂದು ನಕ್ಷತ್ರ ಈ ಧರೆಗೆ ಬಂದು ಸಿಡಿಯುವುದು. ಆಗ ನಾವೆಲ್ಲ ನಾಶವಾಗುವೆವು. ಇದೂ ಒಂದು ತಮಾಷೆ. ನಿಮಗೆ ತಮಾಷೆ ಎಂದರೆ ಪಂಚೇಂದ್ರಿಯಗಳಿಗೆ ಸುಖವನ್ನು ಕೊಡುವ ಅಲ್ಪ ವಸ್ತುಗಳು ಮಾತ್ರ ಆಗಿವೆ.

ಇಲ್ಲಿ ಒಬ್ಬ ಒಳ್ಳೆಯ ದೇವರು ಇರುವನು ಮತ್ತು ಅಲ್ಲಿ ಒಬ್ಬ ಕೆಟ್ಟ ದೇವರು ಇರುವನು, ಏನಾದರೂ ತಪ್ಪನ್ನು ಮಾಡಿದಾಗ ನನ್ನನ್ನು ಅವನ ಕಡೆಗೆ ಸೆಳೆದುಕೊಳ್ಳುವುದಕ್ಕೆ ಹೊಂಚು ಹಾಕುತ್ತಿರುವನು ಎಂದು ಯಾರೋ ಹೇಳುವರು. ನಾನು ಮಗುವಾಗಿದ್ದಾಗ ಯಾರೋ ದೇವರು ಎಲ್ಲವನ್ನೂ ನೋಡುತ್ತಿರುವನು ಎಂದು ನನಗೆ ಹೇಳಿದ್ದರು. ನಾನು ಹಾಸಿಗೆಯ ಮೇಲೆ ಮಲಗಿಕೊಂಡು ಯಾರೋ ಮೇಲಿನ ಛಾವಣಿಯನ್ನು ತೆರೆದು ಇಳಿದು ಬರುವರು ಎಂದು ಭಾವಿಸಿದ್ದೆ. ಏನೂ ಆಗಲಿಲ್ಲ. ನಾವಲ್ಲದೆ ಮತ್ತಾರೂ ನಮ್ಮನ್ನು ನೋಡುತ್ತಿಲ್ಲ. ನಮ್ಮಾತ್ಮನಲ್ಲದೆ ಬೇರೆ ದೇವರೇ ಇಲ್ಲ. ನಾವು ಏನನ್ನು ಅನುಭವಿಸುವೆವೊ ಅದಲ್ಲದೆ ಬೇರೆ ಪ್ರಕೃತಿ ಇಲ್ಲ. ಮನುಷ್ಯನ ಅಭ್ಯಾಸ ಎರಡನೇ ಗುಣ ಮಾತ್ರವಲ್ಲ, ಅದು ಅವನ ಮೊದಲನೇ ಗುಣವೂ ಕೂಡ ಆಗಿದೆ. ಪ್ರಕೃತಿಯಲ್ಲಿರುವುದೆಲ್ಲ ಇಷ್ಟೇ. ನಾನು ಯಾವುದನ್ನಾದರೂ ಎರಡು ಮೂರು ವೇಳೆ ಮಾಡುತ್ತೇನೆ. ಅನಂತರ ಅದೇ ನನ್ನ ಸ್ವಭಾವವಾಗುವುದು. ಅದಕ್ಕಾಗಿ ವ್ಯಥೆಪಡಬೇಡಿ, ಪಶ್ಚಾತ್ತಾಪ ಪಡಬೇಡಿ. ಆಗಿದ್ದು ಆಗಿಹೋಯಿತು. ನೀವು ಕೈಯನ್ನು ಸುಟ್ಟುಕೊಂಡರೆ ಪರಿಣಾಮವನ್ನು ಅನುಭವಿಸಬೇಕು.

ವಿವೇಕಿಯಾಗಿ ನಾವು ತಪ್ಪುಗಳನ್ನು ಮಾಡುತ್ತೇವೆ. ಆದರೇನು? ಅದೆಲ್ಲ ತಮಾಷೆ. ಅದಕ್ಕಾಗಿ ಅಳುತ್ತಾ ಕರೆಯುತ್ತ ಹುಚ್ಚರಾಗಿ ಹೋಗುತ್ತಾರೆ. ಅದನ್ನು ಪುನಃ ಮಾಡಬೇಡಿ. ಕೆಲಸವನ್ನು ಮಾಡಿ ಆದಮೇಲೆ ಅದನ್ನು ಕುರಿತು ಯೋಚಿಸಬೇಡಿ. ಮುಂದೆ ಹೋಗಿ, ನಿಲ್ಲಬೇಡಿ. ಹಿಂದೆ ನೋಡಬೇಡಿ. ಸುಮ್ಮನೆ ಹಿಂದಿನದನ್ನು ನೋಡಿದರೆ ಏನು ಲಾಭ? ನಿಮಗೇನೂ ಲಾಭವೂ ಇಲ್ಲ ನಷ್ಟವೂ ಇಲ್ಲ. ನೀವೇನು ಬೆಣ್ಣೆಯಂತೆ ಕರಗಿ ಹೋಗುವುದಿಲ್ಲ. ಸ್ವರ್ಗ, ನರಕ, ಅವತಾರ ಇವೆಲ್ಲ ಭ್ರಾಂತಿ,

ಹುಟ್ಟುವವರಾರು? ಸಾಯುವವರಾರು? ನೀವು ಪ್ರಪಂಚ ಮತ್ತು ಇತರ ವಸ್ತುಗಳೊಂದಿಗೆ ಆಡುತ್ತ ತಮಾಷೆಯಲ್ಲಿ ನಿರತರಾಗಿರುವಿರಿ. ನಿಮಗೆ ಇಚ್ಛೆಯಿರುವವರೆಗೆ ಈ ದೇಹವನ್ನು ನೀವು ಇಟ್ಟಿಕೊಂಡಿರುವಿರಿ. ನಿಮಗೆ ಇಚ್ಛೆ ಇಲ್ಲದೇ ಇದ್ದರೆ ಅದನ್ನು ನೀವು ಪಡೆಯಬೇಕಾಗಿಲ್ಲ. ಅನಂತವೇ ಸತ್ಯ, ಸಾಂತ ಒಂದು ಲೀಲೆ. ನೀವು ಏಕಕಾಲದಲ್ಲಿ ಅನಂತ ಮತ್ತು ಸಾಂತವಾಗಿರುವಿರಿ. ಇದನ್ನು ಅರಿಯಿರಿ. ಆದರೆ ಜ್ಞಾನದಿಂದ ಏನೂ ವ್ಯತ್ಯಾಸವಾಗುವುದಿಲ್ಲ, ಆಟ ಮುಂದೆ ಸಾಗುವುದು. ಆತ್ಮ ಮತ್ತು ದೇಹ ಎರಡೂ ಮಿಶ್ರವಾಗಿ ಹೋಗಿವೆ. ಇದರಿಂದಾಗಿ ಅಂಶಜ್ಞಾನ ಮಾತ್ರ ಬರುವುದು. ನೀವು ನಿತ್ಯಮುಕ್ತರೆಂಬುದನ್ನು ಅರಿಯಿರಿ. ಜ್ಞಾನಾಗ್ನಿ ಎಲ್ಲಾ ದೋಷಗಳನ್ನೂ ಮತ್ತು ಮಿತಿಗಳನ್ನೂ ಭಸ್ಮಮಾಡುವುದು. ನಾವೇ ಆ ಅನಂತ.

ನೀವು ಆದಿಯಲ್ಲಿ ಎಷ್ಟು ಸ್ವತಂತ್ರರಾಗಿದ್ದಿರೊ ಅಷ್ಟೇ ಸ್ವತಂತ್ರರು ಈಗಲೂ ಮತ್ತು ಮುಂದೆಯೂ. ಯಾರು ತಾನು ಮುಕ್ತನೆಂದು ಅರಿಯುವನೋ ಅವನು ಮುಕ್ತ. ಯಾರು ತಾನು ಬದ್ಧನೆಂದು ಭಾವಿಸುವನೊ ಅವನು ಬದ್ಧ.

ಆಗ ದೇವರು, ಪೂಜೆ ಮುಂತಾದವುಗಳೆಲ್ಲ ಏನಾಗುವುವು? ಅವುಗಳಿಗೆ ತಮ್ಮದೇ ಸ್ಥಾನಗಳಿವೆ. ನಾನು ಮತ್ತು ದೇವರು ಎಂದು ನಾನು ವಿಭಾಗವಾಗಿರುವೆನು. ನಾನೇ ಪೂಜಿಸಿಕೊಳ್ಳುವವನಾಗುತ್ತೇನೆ, ನನ್ನನ್ನು ನಾನೇ ಪೂಜಿಸುತ್ತೇನೆ. ಇದು ಏಕೆ ಆಗಬಾರದು? ದೇವರು ನಾನೇ ಆಗಿದ್ದೇನೆ. ನನ್ನ ಆತ್ಮವನ್ನು ಏತಕ್ಕೆ ಪೂಜಿಸಬಾರದು? ಪರಮೇಶ್ವರನೇ ನನ್ನ ಆತ್ಮವಾಗಿರುವನು. ಇದೆಲ್ಲ ಒಂದು ತಮಾಷೆ. ಇನ್ನು ಬೇರೆ ಯಾವ ಉದ್ದೇಶವೂ ಇಲ್ಲ.

ಜೀವನದ ಗುರಿ ಏನು? ಉದ್ದೇಶವೇನು? ಯಾವುದೂ ಇಲ್ಲ. ಏಕೆಂದರೆ ನಾನೇ ಅನಂತ ಎಂಬುದು ನನಗೆ ಗೊತ್ತಿದೆ. ನೀವು ಭಿಕ್ಷುಕರಾದರೆ ನಿಮಗೆ ಉದ್ದೇಶಗಳಿರಬಹುದು. ನನಗೆ ಯಾವ ಗುರಿಯೂ ಇಲ್ಲ, ಅಗತ್ಯಗಳೂ ಇಲ್ಲ, ಉದ್ದೇಶಗಳೂ ಇಲ್ಲ. ನಾನು ಕೇವಲ ತಮಾಷೆಗಾಗಿ ನಿಮ್ಮ ದೇಶಕ್ಕೆ ಬಂದು ಉಪನ್ಯಾಸ ಮಾಡುತ್ತೇನೆ, ಬೇರೇನೂ ಅರ್ಥವಿಲ್ಲ. ಇನ್ನು ಬೇರೆ ಅರ್ಥ ಏನು ಇರಬಲ್ಲದು? ಗುಲಾಮರು ಮಾತ್ರ ಇತರರಿಗಾಗಿ ಕರ್ಮವನ್ನು ಮಾಡುವರು. ನೀವು ಮತ್ತಾರಿಗೂ ಕರ್ಮವನ್ನು ಮಾಡುವುದಿಲ್ಲ. ನಿಮಗೆ ಅದು ಪ್ರಯೋಜನವೆಂದು ತೋರಿದರೆ, ನೀವು ಪೂಜಿಸುವಿರಿ. ನೀವು ಕ್ರೈಸ್ತರು, ಮಹಮ್ಮದೀಯರು, ಚೈನೀಯರು ಮತ್ತು ಜಪಾನೀಯರು ಪೂಜಿಸುತ್ತಿರುವಾಗ ಅದರಲ್ಲಿ ಭಾಗಿಗಳಾಗಬಹುದು. ಹಿಂದೆ ಇದ್ದ ಮತ್ತು ಮುಂದೆ ಬರುವ ಎಲ್ಲಾ ದೇವರುಗಳನ್ನೂ ನೀವು ಪೂಜಿಸಬಹುದು.

ನಾನೇ ಸೂರ್ಯ, ಚಂದ್ರ, ನಕ್ಷತ್ರಗಳು. ನಾನು ದೇವರೊಂದಿಗೆ ಇರುವೆನು. ಎಲ್ಲಾ ದೇವತೆಗಳಲ್ಲಿಯೂ ಇರುವೆನು. ನಾನು ನನ್ನನ್ನೇ ಪೂಜಿಸುತ್ತೇನೆ.

ಇದಕ್ಕೆ ಬೇರೊಂದು ದೃಷ್ಟಿ ಇದೆ. ನಾನು ಅದನ್ನು ಹೇಳುವುದಕ್ಕೆ ಕಾದಿರಿಸಿರುವೆನು. ನೇಣಿಗೆ ಹಾಕಲ್ಪಡುವನೂ ನಾನೆ, ನಾನೇ ಎಲ್ಲಾ ದುರ್ಜನರು. ನರಕದಲ್ಲಿ ಯಮಕೋಟಲೆಗಳಿಗೆ ತುತ್ತಾಗುವವನೂ ನಾನೇ. ಅದೂ ಒಂದು ತಮಾಷೆಯೆ. ನಾನೇ ಅನಂತ ಎಂದು ಅರಿಯುವುದೇ ತತ್ವಶಾಸ್ತ್ರದ ಆದರ್ಶ. ಉದ್ದೇಶ, ಗುರಿ, ಕಾರಣ, ಕರ್ತವ್ಯ ಇವುಗಳೆಲ್ಲ ಹಿಂದೆ ನಿಲ್ಲುತ್ತವೆ.

ಸತ್ಯವನ್ನು ಮೊದಲು ಕೇಳಬೇಕು. ಅನಂತರ ಅದನ್ನು ಆಲೋಚಿಸಬೇಕು. ಅದನ್ನು ಕುರಿತು ಚೆನ್ನಾಗಿ ವಿಚಾರಮಾಡಬೇಕು. ಜ್ಞಾನಿಗಳಿಗೆ ಅದಕ್ಕಿಂತ ಹೆಚ್ಚು ಏನೂ ಗೊತ್ತಿಲ್ಲ. ನೀವೆ ಸರ್ವರಲ್ಲಿಯೂ ಇರುವಿರಿ ಎಂಬುದನ್ನು ನಿಜವಾಗಿ ನಂಬಿ, ಆದಕಾರಣವೇ ನೀವು ಯಾರನ್ನೂ ಹಿಂಸಿಸಕೂಡದು. ಅವರನ್ನು ಹಿಂಸಿಸಿದರೆ ನಿಮ್ಮನ್ನೇ ಹಿಂಸಿಸಿಕೊಂಡಂತೆ. ಕೊನೆಗೆ ಅದನ್ನು ಕುರಿತು ಧ್ಯಾನಿಸಬೇಕು. ಇದನ್ನು ಕುರಿತು ಆಲೋಚಿಸಿ ನೋಡಿ – ಈ ಪ್ರಪಂಚದಲ್ಲಿ ಎಲ್ಲವೂ ಸರ್ವನಾಶವಾಗುವ ಕಾಲವೊಂದು ಬರುವುದು, ನೀವೊಬ್ಬರೇ ಉಳಿಯುವುದು ಎಂಬುದನ್ನು ಸಾಕ್ಷಾತ್ಕಾರ ಮಾಡಿಕೊಳ್ಳಬಲ್ಲಿರಾ? ಆ ಆನಂದದ ಉದ್ರೇಕ ಎಂದಿಗೂ ನಿಮ್ಮನ್ನು ಅಗಲುವುದಿಲ್ಲ. ನಿಮಗೆ ದೇಹವೇ ಇಲ್ಲ ಎಂದು ಆಗ ಸತ್ಯವಾಗಿ ವೇದ್ಯವಾಗುವುದು. ನಿಮಗೆ ಎಂದಿಗೂ ದೇಹವಿರಲಿಲ್ಲ.

ನಾನು ಅನಂತ ಕಾಲದಲ್ಲೆಲ್ಲ ಏಕಾಕಿ. ನಾನು ಅಂಜುವುದು ಯಾರಿಗೆ? ಇದೆಲ್ಲ ನನ್ನ ಆತ್ಮವೇ, ಇದನ್ನು ಸದಾ ಕಾಲದಲ್ಲಿಯೂ ಧ್ಯಾನಿಸಬೇಕು. ಇದರಿಂದ ಸಾಕ್ಷಾತ್ಕಾರ ಲಭಿಸುವುದು. ಸಾಕ್ಷಾತ್ಕಾರ ಲಭಿಸಿದ ಮೇಲೆ ನೀವು ಇತರರ ಪಾಲಿಗೆ ಒಂದು ಧನ್ಯತೆಯೇ ಆಗುತ್ತೀರಿ.

“ನಿನ್ನ ಮುಖ ಬ್ರಹ್ಮನನ್ನು ಸಾಕ್ಷಾತ್ಕಾರ ಮಾಡಿಕೊಂಡವನಂತೆ ಕಾಣುತ್ತಿದೆ"\break (ಛಾಂದೋಗ್ಯ \enginline{IV, 9.2).} ಅದೇ ಗುರಿ. ನಾನೀಗ ಮಾಡುತ್ತಿರುವಂತೆ ಅದನ್ನು ಕುರಿತು ಉಪನ್ಯಾಸ ಮಾಡುವುದಲ್ಲ. “ನಾನು ಮರದ ಕೆಳಗೆ ಒಬ್ಬ ಗುರುವನ್ನು ನೋಡಿದೆನು. ಅವನು ಹದಿನಾರು ವರುಷದ ಯುವಕ. ಶಿಷ್ಯನಾದರೊ ಎಂಬತ್ತು ವರುಷದ ವೃದ್ದ. ಗುರು ಮೌನವಾಗಿ ವ್ಯಾಖ್ಯಾನ ಮಾಡುತ್ತಿದ್ದನು. ಶಿಷ್ಯನ ಸಂಶಯಗಳು ದೂರವಾದವು" (ದಕ್ಷಿಣಾಮೂರ್ತಿ ಸ್ತೋತ್ರ \enginline{12).} ಯಾರು ಮಾತನಾಡುವವರು? ಸೂರ್ಯನನ್ನು ನೋಡುವುದಕ್ಕೆ ಯಾರು ಒಂದು ದೀಪವನ್ನು ಹತ್ತಿಸುವರು? ಸತ್ಯ ಉದಿಸಿದರೆ ಅದಕ್ಕೆ ಸಾಕ್ಷಿ ಯಾರೂ ಬೇಕಾಗಿಲ್ಲ. ಅದು ನಿಮಗೆ ಗೊತ್ತಿದೆ. ನೀವು ಇದನ್ನೇ ಮಾಡಬೇಕಾಗಿರುವುದು. ಇದನ್ನು ಸಾಕ್ಷಾತ್ಕಾರ ಮಾಡಿಕೊಳ್ಳಿ. ಮೊದಲು ಇದನ್ನು ಆಲೋಚಿಸಿ, ವಿಚಾರ ಮಾಡಿ, ನಿಮ್ಮ ಕುತೂಹಲವನ್ನು ತೃಪ್ತಿಪಡಿಸಿಕೊಳ್ಳಿ. ಅನಂತರ ಮತ್ತಾವುದನ್ನೂ ಕುರಿತು ಚಿಂತಿಸಬೇಡಿ, ನಾವು ಇನ್ನು ಬೇರೆ ಏನನ್ನೂ ಓದದೇ ಇದ್ದರೆ ಚೆನ್ನಾಗಿರುತ್ತಿತ್ತು ಎನ್ನಿಸುತ್ತದೆ. ದೇವರು ನಮಗೆಲ್ಲ ಸಹಾಯಮಾಡಲಿ. ಒಬ್ಬ ಪಂಡಿತ ಏನಾಗುತ್ತಾನೆ ಎಂಬುದನ್ನು ನೋಡಿ!

“ಇದನ್ನು ಹೇಳಿ ಆಯಿತು, ಅದನ್ನು ಹೇಳಿ ಆಯಿತು.”

ಪ್ರಶ್ನೆ: “ಸ್ನೇಹಿತನೇ, ನೀನು ಏನನ್ನು ಹೇಳುತ್ತೀಯೆ?”

ಉತ್ತರ: “ನಾನು ಏನನ್ನೂ ಹೇಳುವುದಿಲ್ಲ. ಇತರರ ಆಲೋಚನೆಯನ್ನು ಮಾತ್ರ ಉದಾಹರಿಸುವೆನು.” ಆದರೆ ಅವನೇ ಏನನ್ನೂ ಕುರಿತು ಚಿಂತಿಸುವುದಿಲ್ಲ. ಇದು\break ವಿದ್ಯಾಭ್ಯಾಸವಾದರೆ ಹುಚ್ಚು ಎಂದರೆ ಇನ್ನೇನು? ಪುಸ್ತಕವನ್ನು ಬರೆದ ಎಲ್ಲರನ್ನೂ ನೋಡಿ, ಆಧುನಿಕ ಕಾಲದ ಗ್ರಂಥಕರ್ತರಲ್ಲಿ ತಮ್ಮ ಸ್ವಂತದ್ದು ಎರಡು ವಾಕ್ಯಗಳೂ ಇಲ್ಲ. ಎಲ್ಲಾ ಇನ್ನೊಬ್ಬರು ಹೇಳಿದ್ದು.

ಆ ಪುಸ್ತಕದಿಂದ ಅಷ್ಟು ಪ್ರಯೋಜನವಿಲ್ಲ. ಇನ್ನೊಬ್ಬರ ಧರ್ಮದಿಂದಲೂ ಅಷ್ಟು ಪ್ರಯೋಜನವಿಲ್ಲ. ಇದು ಊಟ ಮಾಡುವಂತೆ ನಿಮ್ಮ ಧರ್ಮ ನನಗೆ ತೃಪ್ತಿಯನ್ನು ಕೊಡಲಾರದು. ಏಸು ದೇವರನ್ನು ನೋಡಿದನು. ಬುದ್ದ ದೇವರನ್ನು ನೋಡಿದನು. ಆದರೆ ನೀವು ದೇವರನ್ನು ನೋಡದೆ ಇದ್ದರೆ ನಾಸ್ತಿಕರಿಗಿಂತ ಮೇಲಲ್ಲ. ನಾಸ್ತಿಕ ತೆಪ್ಪಗೆ ಇರುವನು. ನೀವಾದರೋ, ಮಾತನಾಡಿ ವಿಶ್ವಶಾಂತಿಗೆ ಭಂಗತರುವಿರಿ. ಪುಸ್ತಕಗಳು, ಬೈಬಲ್ಲುಗಳು, ಶಾಸ್ತ್ರಗಳು, ಇವುಗಳಿಂದ ಪ್ರಯೋಜನವಿಲ್ಲ. ನಾನು ಹುಡುಗನಾಗಿದ್ದಾಗ ಒಬ್ಬ ವೃದ್ಧರನ್ನು ಕಂಡೆ. ಅವರು ಯಾವ ಶಾಸ್ತ್ರವನ್ನೂ ಓದಿದವರಲ್ಲ. ಅವರು ಕೇವಲ ಸ್ಪರ್ಶದಿಂದ ಭಗವಂತನ ಸತ್ಯವನ್ನು ಇನ್ನೊಬ್ಬರ ಅನುಭವಕ್ಕೆ ತರುತ್ತಿದ್ದರು.

ಪ್ರಪಂಚದ ಬೋಧಕರೇ, ನೀವೆಲ್ಲ ಸುಮ್ಮನಿರಿ. ಪಂಡಿತರೇ, ಸುಮ್ಮನಿರಿ. ದೇವರೆ, ನೀನೊಬ್ಬನೇ ಮಾತನಾಡು, ನಿನ್ನ ಸೇವಕ ಕೇಳುವನು. ಸತ್ಯವಿಲ್ಲದೇ ಇದ್ದರೆ ಈ ಜೀವನದಿಂದ ಏನು ಪ್ರಯೋಜನ? ನಾವೆಲ್ಲ ಅದನ್ನು ಗ್ರಹಿಸುತ್ತೇವೆ ಎಂದು ಆಶಿಸುತ್ತೇವೆ. ಆದರೆ ನಮಗೆ ಅದು ಸಾಧ್ಯವಿಲ್ಲ. ನಮ್ಮಲ್ಲಿ ಅನೇಕರಿಗೆ ಬರೀ ಧೂಳು ಸಿಕ್ಕುವುದು. ದೇವರು ಇಲ್ಲವೇ ಇಲ್ಲ. ದೇವರಿಲ್ಲದೆ ಇದ್ದರೆ ಜೀವನದಿಂದ ಏನು ಪ್ರಯೋಜನ? ಈ ಪ್ರಪಂಚದಲ್ಲಿ ಎಲ್ಲಿಯಾದರೂ ಆಶ್ರಯ ಸ್ಥಾನವಿದೆಯೆ? ನಾವು ಅದನ್ನು ಹುಡುಕಬೇಕಾಗಿದೆ. ನಾವು ಅದಕ್ಕಾಗಿ ತೀವ್ರವಾಗಿ ಹುಡುಕುತ್ತಿಲ್ಲ ಅಷ್ಟೇ. ಪ್ರವಾಹದಲ್ಲಿ ಕೊಚ್ಚಿಕೊಂಡು ಹೋಗುತ್ತಿರುವ\break ಹುಲ್ಲಿನೆಸಳಿನಂತೆ ನಾವು.

ಸತ್ಯವಿದ್ದರೆ, ದೇವರಿದ್ದರೆ ಅದು ನಮ್ಮ ಒಳಗೆ ಇರಬೇಕು. ನಾನು ಅವನನ್ನು ನನ್ನ ಸ್ವಂತ ಕಣ್ಣಿನಿಂದ ನೋಡಿರುವೆನು ಎಂದು ಹೇಳುವಂತಾಗಬೇಕು. ಇಲ್ಲದೇ ಇದ್ದರೆ ನನ್ನ ಪಾಲಿಗೆ ಧರ್ಮವಿಲ್ಲ. ನಂಬಿಕೆ, ಸಿದ್ದಾಂತ, ಉಪದೇಶಗಳು ಧರ್ಮವಾಗಲಾರವು. ಸಾಕ್ಷಾತ್ಕಾರ, ಭಗವದ್ದರ್ಶನ – ಇದೊಂದೇ ಧರ್ಮ. ಈ ಪ್ರಪಂಚವೆಲ್ಲ ಆರಾಧಿಸುವ ಈ ವ್ಯಕ್ತಿಗಳ ಮಾಹಾತ್ಮೆಯಲ್ಲಿರುವುದು? ಅವರಿಗೆ ದೇವರು ಒಂದು ಸಿದ್ದಾಂತವಾಗಿರಲಿಲ್ಲ. ತಮ್ಮ ಅಜ್ಜಂದಿರು ನಂಬುತ್ತಾರೆ ಎಂದು ಅವರು ನಂಬಿದರೇನು? ಇಲ್ಲ. ತಮ್ಮ ದೇಹಕ್ಕಿಂತ, ಮನಸ್ಸಿಗಿಂತ, ಎಲ್ಲಕ್ಕಿಂತಲೂ ಅತೀತವಾದ ಅನಂತವನ್ನು ಅವರು ಸಾಕ್ಷಾತ್ಕಾರ ಮಾಡಿಕೊಂಡರು. ಆ ಭಗವಂತನ ಒಂದು ಕಣ ಈ ಪ್ರಪಂಚದಲ್ಲಿ ಪ್ರತಿಬಿಂಬಿಸುವುದರಿಂದ ಈ ಪ್ರಪಂಚ ಸತ್ಯ. ನಾವು ಒಳ್ಳೆಯ ಮನುಷ್ಯನನ್ನು ಪ್ರೀತಿಸುವುದು, ಅವನ ಮುಖವು ಭಗವಜ್ಯೋತಿಯನ್ನು ಹೆಚ್ಚು ಪ್ರತಿಬಿಂಬಿಸುವುದರಿಂದ. ನಾವೇ ಅದನ್ನು ಪಡೆಯಬೇಕಾಗಿದೆ. ಬೇರೆ ಮಾರ್ಗವೇ ಇಲ್ಲ.

ಅದೇ ಗುರಿ. ಅದಕ್ಕಾಗಿ ಹೋರಾಡಿ, ನಿಮ್ಮದೇ ಬೈಬಲ್ಲನ್ನು ಪಡೆಯಿರಿ, ನಿಮ್ಮದೇ ಆದ ಕ್ರಿಸ್ತನನ್ನು ಇಟ್ಟುಕೊಳ್ಳಿ, ಇಲ್ಲದೆ ಇದ್ದರೆ ನೀವು ಧಾರ್ಮಿಕರಾಗಲಾರಿರಿ. ಧರ್ಮದ ವಿಷಯವನ್ನು ಮಾತನಾಡಬೇಡಿ, ಜನ ಸುಮ್ಮನೇ ಮಾತನಾಡುತ್ತಾರೆ. “ಅಜ್ಞಾನದ ಅಂಧಕಾರದಲ್ಲಿ ತಾವಿದ್ದರೂ ತಮಗೆ ಜ್ಞಾನ ಗೊತ್ತಿದೆ ಎಂದು ಕೆಲವರು ಅಹಂಕಾರ ಪಡುವರು. ಅದು ಮಾತ್ರವಲ್ಲ ಇತರರನ್ನು ಗುರಿಯ ಎಡೆಗೆ ಕರೆದೊಯ್ಯುತ್ತೇವೆಂದು ಹೇಳಿ ಇಬ್ಬರೂ ಹಳ್ಳಕ್ಕೆ ಬೀಳುವರು.”

ಯಾವ ಚರ್ಚೂ ಜನರನ್ನು ಉದ್ದಾರ ಮಾಡಲಿಲ್ಲ. ಒಂದು ದೇವಸ್ಥಾನದಲ್ಲಿ ಹುಟ್ಟುವುದು ಒಳ್ಳೆಯದು. ಆದರೆ ಯಾರು ದೇವಸ್ಥಾನ ಅಥವಾ ಚರ್ಚಿನಲ್ಲೇ ಸಾಯುತ್ತಾರೋ ಅವರನ್ನು ದೇವರೇ ರಕ್ಷಿಸಬೇಕು. ಅದರಿಂದ ಪಾರಾಗಿ, ಅದು ಒಳ್ಳೆಯ ಪ್ರಾರಂಭವೇನೋ ನಿಜ, ಆದರೆ ಅದನ್ನು ತ್ಯಜಿಸಿ. ಅದು ಬಾಲ್ಯಾವಸ್ಥೆ, ಆದರೆ ಅದು ಬೇಕಾದರೆ ಇದ್ದುಕೊಳ್ಳಲಿ, ನೀವು ನೇರವಾಗಿ ದೇವರೆಡೆಗೆ ಹೋಗಿ. ಯಾವ ಊಹೆಗಳೂ ಬೇಕಾಗಿಲ್ಲ, ಸಿದ್ದಾಂತಗಳೂ ಬೇಕಾಗಿಲ್ಲ. ಆಗ ಮಾತ್ರವೇ ಎಲ್ಲಾ ಸಂಶಯಗಳೂ ಮಾಯವಾಗುವುವು. ಆಗ ಮಾತ್ರ ಹೃದಯದ ವಕ್ರತೆಯೆಲ್ಲ ನೇರವಾಗುವುದು.

ಅನೇಕದ ಮಧ್ಯದಲ್ಲಿ ಯಾರು ಏಕವನ್ನು ನೋಡುತ್ತಾರೆಯೋ, ಯಾರು ಅನಂತ ಮೃತ್ಯುವಿನ ಮಧ್ಯದಲ್ಲಿ ಆ ಒಂದು ಅಮೃತಾತ್ಮನನ್ನು ನೋಡುತ್ತಾರೆಯೋ, ಹಲವು ವಸ್ತುಗಳ ಮಧ್ಯದಲ್ಲಿ ಬದಲಾಯಿಸದೇ ಇರುವ ಒಂದನ್ನು ಯಾರು ತಮ್ಮ ಹೃದಯದಲ್ಲಿ ನೋಡುವರೋ ಅವರಿಗೆ ಮಾತ್ರ ಶಾಶ್ವತ ಶಾಂತಿ.

