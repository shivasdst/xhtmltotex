
\chapter[ಏಕತೆಯೇ ಧರ್ಮದ ಗುರಿ]{ಏಕತೆಯೇ ಧರ್ಮದ ಗುರಿ\protect\footnote{\engfoot{C.W, Vol. III, P.1}}}

\begin{center}
(೧೮೯೬ರಲ್ಲಿ ನ್ಯೂಯಾರ್ಕಿನಲ್ಲಿ ನೀಡಿದ ಉಪನ್ಯಾಸ)
\end{center}

\vskip -1pt

ಈ ನಮ್ಮ ಪ್ರಪಂಚ, ಪಂಚೇಂದ್ರಿಯಗಳ ಮೂಲಕ ಅರಿತ ಪ್ರಪಂಚ, ಯುಕ್ತಿ ಪೂರಿತವಾದುದು, ಬುದ್ಧಿಯಿಂದ ಅರಿತುದು. ಇದು ಎರಡು ಕಡೆಗಳಲ್ಲಿಯೂ ಅಮಿತವಾದ, ಅಜೇಯವಾದ ಎಂದೆಂದಿಗೂ ಅಜೇಯವಾಗಿಯೇ ಉಳಿಯುವ ಯಾವುದೋ ಒಂದರಿಂದ ಪರಿವೃತವಾಗಿರುವುದು. ನಮ್ಮ ಅನ್ವೇಷಣೆ ಇರುವುದು ಈ ಅನಂತದಲ್ಲಿ ನಾವು ಕೇಳುವ ಪ್ರಶ್ನೆ ಇದಕ್ಕೆ ಸಂಬಂಧಪಟ್ಟದ್ದು. ಇದಕ್ಕೆ ಬೇಕಾದ ಉತ್ತರವೂ ಈ ಕ್ಷೇತ್ರದಲ್ಲಿದೆ. ಪ್ರಪಂಚದಲ್ಲಿ ಯಾವುದನ್ನು ಧರ್ಮ ಎನ್ನುವರೋ ಆ ಬೆಳಕು ಇಲ್ಲಿಂದ ಬಂದಿದೆ. ಹೇಗಾದರಾಗಲಿ, ಮುಖ್ಯವಾಗಿ ಧರ್ಮ ಅತೀಂದ್ರಿಯಕ್ಕೆ ಸೇರಿದ್ದು, ಇಂದ್ರಿಯಜನ್ಯ ಪ್ರಪಂಚಕ್ಕೆ ಸೇರಿದ್ದು ಅಲ್ಲ. ಇದು ಯುಕ್ತಿಗೆ ಅತೀತವಾದುದ್ದು, ಬುದ್ದಿಯ ಕಾರ್ಯಕ್ಷೇತ್ರಕ್ಕೆ ನಿಲುಕದುದು. ಇದೊಂದು ದರ್ಶನ, ಇದೊಂದು ಸ್ಫೂರ್ತಿ. ಇದು ಇದುವರೆಗೆ ಅರಿಯದ, ಎಂದೆಂದಿಗೂ ಅರಿಯಲಾಗದ ಲೋಕಕ್ಕೆ ಧುಮುಕುವ ಸಾಹಸ, ಅಜ್ಞೇಯವಾದ ಇದು ಜ್ಞೇಯಕ್ಕಿಂತ ಶ್ರೇಷ್ಠವಾದುದು. ನಾವು ಇದನ್ನು ಸಾಧಾರಣ ವಸ್ತುವನ್ನು ತಿಳಿದುಕೊಳ್ಳುವಂತೆ ತಿಳಿದುಕೊಳ್ಳ ಲಾಗುವುದಿಲ್ಲ. ನಾನು ತಿಳಿದಿರುವಂತೆ, ಈ ಅನ್ವೇಷಣೆಯು ಮಾನವ ಕೋಟಿಯ ಆದಿಕಾಲದಿಂದಲೂ ನಡೆಯುತ್ತಿರುವುದು. ಪ್ರಪಂಚದಲ್ಲಿ ಯಾವ ಇತಿಹಾಸ ಕಾಲದಲ್ಲಿಯೂ ಇದರ ಅನ್ವೇಷಣೆ ಮತ್ತು ಇದರ ಸಾಧನೆಯಿಲ್ಲದೆ, ಮನುಷ್ಯನ ಬುದ್ದಿ ಸುಮ್ಮನೆ ಇರಲಿಲ್ಲ. ಮಾನವನ ಮನಸ್ಸು ಎಂಬ ಸಣ್ಣ ಪ್ರಪಂಚದಲ್ಲಿ ಒಂದು ಆಲೋಚನೆ ಏಳುವುದನ್ನು ನೋಡುತ್ತೇವೆ. ಅದು ಎಲ್ಲಿಂದ ಬಂತು, ಯಾವಾಗ ಹೋಗುವುದು, ಎಲ್ಲಿಗೆ ಹೋಗುವುದು, ಇದಾವುದೂ ನಮಗೆ ಗೊತ್ತಾಗುವುದಿಲ್ಲ. ಬ್ರಹ್ಮಾಂಡ ಪಿಂಡಾಂಡಗಳೆರಡೂ ಒಂದೇ ರೀತಿಯಲ್ಲಿದ್ದು, ಒಂದೇ ಕಂಪನ ತರಂಗದಲ್ಲಿ ಇದ್ದು, ಒಂದೇ ಹಂತಗಳಲ್ಲಿ ಸಾಗಿ ಹೋಗುತ್ತಿರುವಂತೆ ಕಾಣುವುವು.

\vskip -1pt

ಧರ್ಮ ಹೊರಗಿನಿಂದ ಬರುವುದಿಲ್ಲ, ಒಳಗಿನಿಂದಲೇ ಬರುವುದು ಎಂಬ ಹಿಂದೂ ಸಿದ್ದಾಂತವನ್ನು ನಿಮಗೆ ವಿವರಿಸಲು ಯತ್ನಿಸುತ್ತೇನೆ. ಧಾರ್ಮಿಕ ಭಾವನೆ ಮನುಷ್ಯನ ಸ್ವಭಾವದಲ್ಲಿ ಅಂತರ್ಗತವಾಗಿದೆ ಎಂಬುದು ನನ್ನ ಭಾವನೆ. ಅವನು ತನ್ನ ಮನಸ್ಸು, ದೇಹ, ಆಲೋಚನೆ, ಮತ್ತು ಜೀವನ ಇವನ್ನು ತೊರೆಯುವವರೆಗೂ ಧರ್ಮವನ್ನು ತೊರೆಯಲಾರ. ಮಾನವನು ಆಲೋಚನಾ ಜೀವಿಯಾಗಿರುವವರೆಗೆ ಈ ಹೋರಾಟ ಮುಂದೆ ಸಾಗಬೇಕಾಗಿದೆ; ಅಲ್ಲಿಯವರೆಗೆ ಯಾವುದಾದರೂ ಒಂದು ಬಗೆಯ ಧರ್ಮ ಇರಬೇಕು. ಆದಕಾರಣ ಪ್ರಪಂಚದಲ್ಲಿ ನಾವು ಹಲವು ಬಗೆಯ ಧರ್ಮಗಳನ್ನು ನೋಡುತ್ತೇವೆ. ಇದೊಂದು ದಿಗ್ಭ್ರಮೆಯನ್ನು ತರುವಂತಹ ಅಧ್ಯಯನ. ಆದರೆ ನಮ್ಮಲ್ಲಿ ಹಲವರು ಭಾವಿಸುವಂತೆ ಇದೊಂದು ವ್ಯರ್ಥ ಊಹೆಯಲ್ಲ. ಈ ಗೊಂದಲದ ಹಿಂದೆ ಒಂದು ಸಾಮರಸ್ಯ ಇದೆ. ಈ ಅಪಶ್ರುತಿಗಳ ನಡುವೆ ಒಂದು ಶ್ರುತಿ ಇದೆ. ಯಾರು ಇದನ್ನು ಆಲಿಸಲು ಸಿದ್ಧರಾಗಿರುವರೋ ಅವರಿಗೆ ಇದು ಕೇಳಿಸುವುದು.

\vskip -1pt

ಈಗಿನ ಕಾಲದಲ್ಲಿ ಪ್ರಶ್ನೆಗಳಲ್ಲೆಲ್ಲಾ ದೊಡ್ಡ ಪ್ರಶ್ನೆಯೇ ಇದು: ತಿಳಿದಿರುವುದು ಮತ್ತು ಮುಂದೆ ತಿಳಿಯಬಲ್ಲದು – ಇವು ಎರಡೂ ಕಡೆಗಳಿಂದಲೂ ಅನಂತವಾದ ಅಜ್ಞೇಯದಿಂದ ಸುತ್ತುವರಿಯಲ್ಪಟ್ಟಿವೆ ಎಂದಿಟ್ಟುಕೊಳ್ಳೋಣ; ಆದರೆ ಆ ಚಿರಂತನ ಅಜ್ಞೇಯವಾದುದನ್ನು ಅರಿಯುವುದಕ್ಕೆ ನಾವೇಕೆ ಪ್ರಯತ್ನಿಸಬೇಕು? ನಮಗೆ ತಿಳಿದಿರುವುದರಲ್ಲಿ ನಾವೇಕೆ ತೃಪ್ತರಾಗಬಾರದು? ತಿನ್ನುವುದು, ಕುಡಿಯುವುದು, ಸಮಾಜಕ್ಕೆ ಸ್ವಲ್ಪ ಒಳ್ಳೆಯದನ್ನು ಮಾಡುವುದು ಇಷ್ಟರಲ್ಲೇ ಏಕೆ ನಾವು ತೃಪ್ತರಾಗಿರಬಾರದು? ಸದ್ಯದಲ್ಲಿ ಘನ ವಿದ್ವಾಂಸನಿಂದ ಹಿಡಿದು ಈಗ ತಾನೆ ತೊದಲು ಮಾತನಾಡುತ್ತಿರುವ ಶಿಶುವಿನವರೆಗೆ ಎಲ್ಲರೂ ಹೇಳುವುದೇ, “ಪ್ರಪಂಚಕ್ಕೆ ಒಳ್ಳೆಯದನ್ನು ಮಾಡಿ; ಅದೇ ಧರ್ಮದ ಸರ್ವಸ್ವ; ಅದಿಲ್ಲದೆ ಇಂದ್ರಿಯಾತೀತ ಪ್ರಪಂಚಕ್ಕೆ ಸೇರಿದ ಪ್ರಶ್ನೆಗಳನ್ನು ಕೇಳುವುದು ವೃಥಾ ಕಾಲಹರಣ'' ಎಂಬುದು. ಇದು ಈಗ ಎಷ್ಟು ಮಟ್ಟಿಗೆ ಪ್ರಬಲವಾಗಿದೆ ಎಂದರೆ ಇದೇ ಸ್ವತಃಸಿದ್ದ ಸತ್ಯದಂತೆ ಕಾಣುತ್ತಿದೆ.

\vskip -1pt

ಆದರೆ ಅದೃಷ್ಟವಶಾತ್ ನಾವು ಇಂದ್ರಿಯಾತೀತ ಪ್ರಪಂಚಕ್ಕೆ ಸೇರಿದ ಪ್ರಶ್ನೆಗಳನ್ನು ಹಾಕಬೇಕಾಗಿದೆ. ಈಗಿರುವುದು – ವ್ಯಕ್ತವಾಗಿರುವುದು, ಅವ್ಯಕ್ತದ ಯಾವುದೋ ಒಂದು ಅಂಶ ಮಾತ್ರ. ನಮಗೆ ಗೋಚರಿಸುವ ಇಂದ್ರಿಯ ಜಗತ್ತು ಅನಂತ ಆಧ್ಯಾತ್ಮಿಕ ಜಗತ್ತಿನ ಯಾವುದೊ ಒಂದು ಕಣಮಾತ್ರ, ಇಂದ್ರಿಯಾನುಭವಕ್ಕೆ ಒಳಪಡಿಸಿದ ಆಧ್ಯಾತ್ಮಿಕ ಜಗತ್ತಿನ ಒಂದು ಕಣಮಾತ್ರವಾಗಿದೆ. ಇದರ ಆಚೆ ಏನಿದೆ ಎಂಬುದನ್ನು ಅರಿಯದೆ ಇದನ್ನು ವಿವರಿಸುವುದು ಹೇಗೆ? ಇದನ್ನು ಅರ್ಥ ಮಾಡಿಕೊಳ್ಳುವುದು ಹೇಗೆ? ಸಾಕ್ರಟೀಸ್ ಒಂದು ದಿನ ಅಥೆನ್ಸ್ ನಗರದಲ್ಲಿ ಸಂಚರಿಸುತ್ತಿದ್ದಾಗ ಗ್ರೀಸಿಗೆ ಬಂದ ಬ್ರಾಹ್ಮಣನೊಬ್ಬನನ್ನು ಸಂಧಿಸಿದ ಪ್ರಸಂಗವಿದೆ. ಸಾಕ್ರಟೀಸ್ ಬ್ರಾಹ್ಮಣನಿಗೆ ಮಾನವ ಪ್ರಪಂಚದಲ್ಲಿ ಕಲಿಯಬೇಕಾದ ದೊಡ್ಡ ವಿಷಯವೆ ಮನುಷ್ಯನನ್ನು ಕುರಿತದ್ದು ಎಂದನು. ಆಗ ಬ್ರಾಹ್ಮಣ ತಕ್ಷಣ ಉತ್ತರ ಕೊಟ್ಟ: “ದೇವರನ್ನು ಅರಿಯುವವರೆಗೆ ನೀನು ಮಾನವನನ್ನು ಹೇಗೆ ಅರಿಯಬಲ್ಲೆ?” ಈ ದೇವರು ಎಂದೆಂದಿಗೂ ಆಜ್ಞೇಯನು ಅಥವಾ ನಿರಪೇಕ್ಷನು ಅಥವಾ ಅನಂತನು ಅಥವಾ ನಾಮರಹಿತನು. ಇವನನ್ನು ನೀವು ಯಾವ ನಾಮದಿಂದ ಬೇಕಾದರೂ ಕರೆಯಬಹುದು. ಈ ಭಗವತತ್ತ್ವವೇ ಯುಕ್ತಿಪೂರ್ವಕ ವಿವರಣೆ, ಇದೇ ಏಕಮಾತ್ರ ವಿವರಣೆ; ಈ ಜೀವನದಲ್ಲಿ, ಈಗ ತಿಳಿದಿರುವ ಮತ್ತು ತಿಳಿಯಬಲ್ಲ ವಸ್ತುವಿನ ಸಾರ. ನಿಮ್ಮ ಎದುರಿಗೆ ಇರುವ ಯಾವುದನ್ನಾದರೂ ತೆಗೆದುಕೊಳ್ಳಿ. ಅತ್ಯಂತ ಭೌತಿಕವಾದ ವಸ್ತುವನ್ನಾದರೂ ತೆಗೆದುಕೊಳ್ಳಿ, ರಸಾಯನ ಶಾಸ್ತ್ರ, ಭೌತಶಾಸ್ತ್ರ, ಖಗೋಳಶಾಸ್ತ್ರ ಮತ್ತು ಜೀವ–ವಿಜ್ಞಾನಶಾಸ್ತ್ರ, ಇವುಗಳಲ್ಲಿ ಯಾವುದನ್ನಾದರೂ ತೆಗೆದುಕೊಳ್ಳಿ. ಅದನ್ನು ಅಧ್ಯಯನ ಮಾಡಿ, ಮುಂದೆ ಮುಂದೆ ಹೋಗಿ, ಸ್ಥೂಲ ಆಕಾರಗಳು ಮಾಯವಾಗಿ ಸೂಕ್ಷ್ಮ ಸೂಕ್ಷ್ಮವಾಗುತ್ತ ಬರುವುವು. ಕೊನೆಗೆ ನೀವೊಂದು ಹಂತಕ್ಕೆ ಬರುತ್ತೀರಿ. ಅಲ್ಲಿ ಭೌತವಸ್ತುವಿನಿಂದ ಅತಿಭೌತ ವಸ್ತುವಿನ ಕಡೆ ನೆಗೆಯಬೇಕಾಗುವುದು.\break ಪ್ರತಿಯೊಂದು ಜ್ಞಾನ ಕ್ಷೇತ್ರದಲ್ಲಿಯೂ ಸ್ಥೂಲ ಸೂಕ್ಷ್ಮವಾಗುವುದು, ಭೌತವಿದ್ಯೆ ಆಧ್ಯಾತ್ಮಿಕ ವಿದ್ಯೆಯಲ್ಲಿ ಪರ್ಯವಸಾನವಾಗುವುದು.

\vskip -1pt

ಹೀಗೆ ಮಾನವನು ಅತೀಂದ್ರಿಯಕ್ಕೆ ಸಂಬಂಧಪಟ್ಟ ವಿಷಯವನ್ನು ಅಧ್ಯಯನ\break ಮಾಡಲೇಬೇಕಾದ ಪರಿಸ್ಥಿತಿಗೆ ಬರುತ್ತಾನೆ. ಈ ಅತೀಂದ್ರಿಯವನ್ನು ನಾವು ಅರಿಯದಿದ್ದರೆ ಜೀವನ ಒಂದು ಮರಳುಕಾಡಾಗುವದು, ಮಾನವ ಜೀವನ ವ್ಯರ್ಥವಾಗುವುದು.\break “ಈಗಿರುವ ಸ್ಥಿತಿಯಲ್ಲೆ ತೃಪ್ತನಾಗಿರು'' ಎಂದು ಬೇಕಾದರೆ ಹೇಳಬಹುದು. ದನಗಳು, ನಾಯಿಗಳು ಹಾಗೆಯೇ ಇವೆ; ಅದರಂತೆಯೆ ಎಲ್ಲಾ ಪ್ರಾಣಿಗಳೂ ಇವೆ. ಅದರಿಂದಲೇ ಅವು ಪ್ರಾಣಿಗಳಾಗಿರುವುವು. ಈಗಿರುವ ಸ್ಥಿತಿಯಲ್ಲೇ ತೃಪ್ತನಾಗಿ, ಅತೀತದ ಅನ್ವೇಷಣೆಯನ್ನೆಲ್ಲಾ ತ್ಯಜಿಸಿದರೆ ಮಾನವ ಕೋಟಿ ಪ್ರಾಣಿಗಳ ಸ್ಥಿತಿಗೆ ಪುನಃ ಹೋಗಬೇಕಾಗುವುದು. ಧರ್ಮವೇ, ಅಂದರೆ ಅತೀತದ ಅನ್ವೇಷಣೆಯೇ ಮಾನವನನ್ನು ಪ್ರಾಣಿಗಳಿಂದ ಪ್ರತ್ಯೇಕಿಸಿರುವುದು. ಮಾನವನೊಬ್ಬ ಮಾತ್ರ ಮೇಲೆ ನೋಡುವ ಪ್ರಾಣಿ, ಮಿಕ್ಕವುಗಳೆಲ್ಲ ಕೆಳಗೆ ನೋಡುತ್ತಿವೆ ಎಂದು ಚೆನ್ನಾಗಿಯೇ ಹೇಳಿರುವರು. ಹಾಗೆ ಮೇಲೆ ನೋಡುವುದು, ಮೇಲೆ ಹೋಗುವುದು, ಪೂರ್ಣತೆಯನ್ನು ಅರಸುವುದು ಇದೇ ಮೋಕ್ಷ. ಅವನು ಮೇಲೆ ಬೇಗ ಹೋದಷ್ಟೂ, ಸತ್ಯವೇ ಮೋಕ್ಷ ಎಂಬ ಭಾವನೆಯನ್ನು ಅವನು ಬೇಗ ಪಡೆಯುವನು. ಅದನ್ನು ಪಡೆಯುವುದಕ್ಕೆ ನಿಮ್ಮ ಜೇಬಿನಲ್ಲಿರುವ ಹಣ ಮುಖ್ಯವಲ್ಲ, ನೀವು ವಾಸಿಸುವ ಮನೆ ಮುಖ್ಯವಲ್ಲ; ಇದು ನಿಮ್ಮಲ್ಲಿರುವ ಆಧ್ಯಾತ್ಮಿಕ ಭಾವನಾ ಸಂಪತ್ತಿನ ಮೇಲೆ ಇದೆ. ಇದೇ ಮಾನವನ ಶ್ರೇಯಸ್ಸಿಗೆ ಮಾರ್ಗ. ಇದೇ ಎಲ್ಲಾ ವಿಧವಾದ ಭೌತಿಕ ಮತ್ತು ಲೌಕಿಕದ ಉನ್ನತಿಗೆ ಮೂಲ; ಮಾನವ ಕೋಟಿಯನ್ನು ಮುಂದೊಯ್ಯುವ ಕ್ರಿಯೋತ್ತೇಜಕ ಶಕ್ತಿ ಇದು. ಉತ್ಸಾಹ ಶ್ರದ್ದೆಗಳನ್ನು ನೀಡುವುದೇ ಇದು.

\vskip -1pt

ಧರ್ಮವು ಆಹಾರವನ್ನು ಅವಲಂಬಿಸಿಲ್ಲ; ಅದು ಮನೆಯಲ್ಲಿ ಬಾಳುತ್ತಿಲ್ಲ. ಪದೇ ಪದೇ ಜನರು ಈ ಆಕ್ಷೇಪಣೆಯನ್ನು ತರುತ್ತಿರುವರು: `ಧರ್ಮದಿಂದ ಏನು ಪ್ರಯೋಜನ? ದರಿದ್ರನ ಬಡತನವನ್ನು ಅದು ಹೋಗಲಾಡಿಸಬಲ್ಲದೇ?' ಎನ್ನುವರು. ಬಹುಶಃ ಅದರಿಂದ ಸಾಧ್ಯವಿಲ್ಲದೆ ಇದ್ದರೆ ಧರ್ಮ ಸುಳ್ಳು ಎಂದು ತೋರುವುದೆ? ಖಗೋಳಶಾಸ್ತ್ರಕ್ಕೆ ಸಂಬಂಧಪಟ್ಟ ಒಂದು ಲೆಕ್ಕವನ್ನು ನೀವು ವಿವರಿಸುತ್ತಿದ್ದಾಗ, ಒಂದು ಮಗು ನಿಮ್ಮ ಹತ್ತಿರ ನಿಂತುಕೊಂಡು, `ಇದರಿಂದ ನನಗೆ ಮಿಠಾಯಿ ದೊರಕುವುದೇ?' ಎಂದು ಪ್ರಶ್ನಿಸಿದರೆ, ಇಲ್ಲ ಎಂದು ನೀವು ಉತ್ತರವೀಯುವಿರಿ. “ಹಾಗಾದರೆ ಇದರಿಂದ ಏನೂ ಪ್ರಯೋಜನವಿಲ್ಲ" ಎಂದು ಮಗು ಹೇಳುವುದು. ಮಕ್ಕಳು ಕೇವಲ ತಮ್ಮ ದೃಷ್ಟಿಯಿಂದ ಪ್ರಪಂಚವನ್ನು ಅಳೆಯುತ್ತವೆ. ಅದರಂತೆಯೇ ಪ್ರಪಂಚದ ಮಕ್ಕಳೂ (ಬಾಲ ಬುದ್ದಿಯ ದೊಡ್ಡವರು). ನಾವು ಪರವಸ್ತುವನ್ನು ಲೌಕಿಕ ದೃಷ್ಟಿಯಿಂದ ಅಳೆಯಬಾರದು. ಪ್ರತಿಯೊಂದನ್ನು ಅದರದರ ಪ್ರಮಾಣದಿಂದಲೇ ಅಳೆಯಬೇಕು. ಧರ್ಮ ಮಾನವನ ಬಾಳಿನಲ್ಲಿ ಓತಪ್ರೋತವಾಗಿರುವುದು. ಈಗಿನ ಜೀವನ ಮಾತ್ರವಲ್ಲ, ಭೂತ ಭವಿಷ್ಯತ್ ವರ್ತಮಾನ ಕಾಲಗಳೆಲ್ಲ ಇದರಲ್ಲಿ ಹಾಸು ಹೊಕ್ಕಾಗಿವೆ. ಆದಕಾರಣ ಧರ್ಮವೆಂದರೆ ಆದಿ ಅಂತ್ಯಗಳಿಲ್ಲದ ಜೀವನಿಗೂ, ಆದಿ ಅಂತ್ಯಗಳಿಲ್ಲದ ದೇವರಿಗೂ ಇರುವ ಶಾಶ್ವತ ಸಂಬಂಧ. ಮಾನವನ ಐದು ನಿಮಿಷದ ಬಾಳುವೆಯ ದೃಷ್ಟಿಯಿಂದ ಅದರ ಪ್ರಭಾವವನ್ನು ಅಳೆಯುವುದು ನ್ಯಾಯವೆ? ಎಂದಿಗೂ ಇಲ್ಲ. ಇದು ಕೇವಲ ನಿಷೇಧಾತ್ಮಕವಾದ ವಾದಸರಣಿ.

ಧರ್ಮ ಏನನ್ನಾದರೂ ಸಾಧಿಸಬಲ್ಲದೆ ಎಂಬುದೆ ಈಗ ಬರುವ ಪ್ರಶ್ನೆ. ಸಾಧಿಸಬಲ್ಲದು; ಇದು ಮಾನವನಿಗೆ ಚಿರಂತನ ಬಾಳುವೆಯನ್ನು ನೀಡುವುದು. ಮನುಷ್ಯನು ಈಗಿರುವ ಸ್ಥಿತಿಗೆ ಕಾರಣ ಧರ್ಮವೇ. ಅದೇ ಈ ಮೃಗಸದೃಶ ಮಾನವನನ್ನು ದೇವನನ್ನಾಗಿ ಮಾಡಬಲ್ಲದು. ಇದು ಧರ್ಮಕ್ಕೆ ಸಾಧ್ಯ. ಮಾನವ ಸಮಾಜದಿಂದ ಧರ್ಮವನ್ನು ತೆಗೆದುಬಿಟ್ಟರೆ ಉಳಿಯುವುದೇನು? ಆಸುರೀ ಪ್ರವೃತ್ತಿಗಳ ಸಮೂಹವೇ ಉಳಿಯುವುದು. ಇಂದ್ರಿಯಸುಖ ಮಾನವನ ಗುರಿಯಲ್ಲ. ಜ್ಞಾನವೇ ಎಲ್ಲಾ ಜೀವನದ ಗುರಿ. ಪ್ರಾಣಿಯು\break ವಿಷಯೇಂದ್ರಿಯಗಳನ್ನು ಅನುಭವಿಸುವುದಕ್ಕಿಂತ ಹೆಚ್ಚಾಗಿ ಮಾನವನು ಬೌದ್ಧಿಕ ವಿಷಯವನ್ನು ಆನಂದಿಸುವನು; ಮತ್ತು ಬೌದ್ಧಿಕ ವಿಷಯಕ್ಕಿಂತ ಹೆಚ್ಚಾಗಿ ಆಧ್ಯಾತ್ಮಿಕ ವಿಷಯಗಳನ್ನು ಆನಂದಿಸಬಲ್ಲ. ಆದ ಕಾರಣ ಶ್ರೇಷ್ಠ ಜ್ಞಾನವೆ ಆಧ್ಯಾತ್ಮಜ್ಞಾನವಾಗಬೇಕು. ಈ ಜ್ಞಾನದಿಂದ ಆನಂದ ಪ್ರಾಪ್ತವಾಗುವುದು. ಈ ಪ್ರಪಂಚದ ವಸ್ತುಗಳಲ್ಲಿ ಕೇವಲ ಛಾಯೆಗಳು, ನಿಜವಾದ ಸಚ್ಚಿದಾನಂದದಿಂದ ದೂರದಲ್ಲಿರುವ ಮೂರನೆಯದೊ ನಾಲ್ಕನೆಯದೊ ಕೋಶದ ಆಚೆಗಿನ ಅಭಿವ್ಯಕ್ತಿಗಳು.

ಮತ್ತೊಂದು ಪ್ರಶ್ನೆ. ನಿಜವಾದ ಗುರಿ ಏನು? ಈಗಿನ ಕಾಲದಲ್ಲಿ ಮಾನವ ಅಂತ್ಯವಿಲ್ಲದೆ ಮುಂದುವರಿಯುತ್ತಿರುವನು, ಅವನು ಸೇರಬೇಕಾದ ಗುರಿ ಏನೂ ಇಲ್ಲ ಎಂದು ಹೇಳುವುದನ್ನು ಕೇಳುತ್ತಿರುವೆವು. ಯಾವಾಗಲೂ ಮುಂದೆ ಮುಂದೆ ಹೋಗುತ್ತಿರುವುದು, ಎಂದೆಂದಿಗೂ ಗುರಿ ಸೇರದಿರುವುದು! ಇದೇನಾದರೂ ಆಗಲಿ, ಇದು ಎಷ್ಟೇ ಅದ್ಭುತವಾಗಲಿ, ಈ ವಾದ ಸರಣಿಗೆ ಅರ್ಥವಿಲ್ಲ. ನೇರವಾಗಿ ಯಾವುದಾದರೂ ಚಲಿಸುವುದೆ? ಒಂದು ಸರಳ ರೇಖೆಯನ್ನು ಎಲ್ಲಿಯೂ ನಿಲ್ಲಿಸದೆ ಮುಂದುವರಿಸಿದರೆ, ಅದು ವೃತ್ತವಾಗುವುದು, ಹೊರಟ ಸ್ಥಳಕ್ಕೆ ಬರುವುದು. ನೀವು ಪ್ರಾರಂಭಿಸಿದ ಸ್ಥಳದಲ್ಲೇ ಕೊನೆಗಾಣಬೇಕಾಗಿದೆ. ನೀವು ದೇವರಿಂದ ಪ್ರಾರಂಭಿಸಿದುದರಿಂದ ಅವನಲ್ಲೇ ಕೊನೆಗಾಣಬೇಕಾಗಿದೆ. ಹಾಗಾದರೆ ಉಳಿಯುವುದೇನು? ವಿಸ್ತಾರವಾದ ಸಾಧನೆಗಳು. ಅನಂತಕಾಲದವರೆಗೆ ನೀವು ಸಾಧನೆಯನ್ನು ಮಾಡುತ್ತಿರಬೇಕು.

ಮತ್ತೊಂದು ಪ್ರಶ್ನೆ ಇರುವುದು. ನಾವು ಮುಂದುವರಿದಂತೆ ಹೊಸ ಹೊಸ ಆಧ್ಯಾತ್ಮಿಕ ಸತ್ಯಗಳನ್ನು ಕಂಡುಹಿಡಿಯುವೆವೆ? ಹೌದು ಮತ್ತು ಇಲ್ಲ. ಮೊದಲನೆಯದಾಗಿ ಧರ್ಮದ ವಿಷಯದಲ್ಲಿ ತಿಳಿದುಕೊಳ್ಳುವಂಥದ್ದು ಏನೂ ಇಲ್ಲ. ಇದೆಲ್ಲ ಆಗಲೆ ಗೊತ್ತಿದೆ. ಪ್ರಪಂಚದ ಧರ್ಮಗಳೆಲ್ಲ, ಅಂತರಾಳದಲ್ಲಿ ಏಕತೆ ಆಗಲೆ ಇದೆ ಎಂದು ಸಾರುವುದನ್ನು ನೀವು ನೋಡುತ್ತೀರಿ. ನಾವು ದೇವರಲ್ಲಿ ಐಕ್ಯವಾದ ಮೇಲೆ ಮತ್ತೆ ಮುಂದುವರಿಯುವುದಿಲ್ಲ. ಜ್ಞಾನವೆಂದರೆ ಈ ಏಕತೆಯನ್ನು ಕಂಡುಹಿಡಿಯುವುದು. ನಾನು ನಿಮ್ಮನ್ನು ಪುರುಷ ಮತ್ತು ಸ್ತ್ರೀಯರನ್ನಾಗಿ ನೋಡುತ್ತಿರುವೆನು. ಇದು ವೈವಿಧ್ಯತೆ. ನಿಮ್ಮನ್ನೆಲ್ಲ ಒಟ್ಟು ಸೇರಿಸಿ ಮಾನವರು ಎಂದು ಕರೆದಾಗ ಅದು ವೈಜ್ಞಾನಿಕ ಜ್ಞಾನವಾಗುತ್ತದೆ. ಉದಾಹರಣೆಗೆ ರಸಾಯನಶಾಸ್ತ್ರವನ್ನು ತೆಗೆದುಕೊಳ್ಳಿ. ರಸಾಯನಶಾಸ್ತ್ರಜ್ಞರು, ತಿಳಿದ ಎಲ್ಲ ವಸ್ತುಗಳನ್ನೂ ಅವುಗಳ ಮೂಲ ದ್ರವ್ಯಗಳಿಗೆ ಇಳಿಸಲು ಪ್ರಯತ್ನಿಸುವರು; ಸಾಧ್ಯವಾದರೆ, ಒಂದೇ ದ್ರವ್ಯದಿಂದ ಇವುಗಳೆಲ್ಲ ಬಂದಿವೆ, ಎಂಬುದನ್ನು ತೋರಿಸಲು ಪ್ರಯತ್ನಿಸುವರು. ಎಲ್ಲ ದ್ರವ್ಯಗಳಿಗೂ ಮೂಲವಾದ ಒಂದು ದ್ರವ್ಯವನ್ನು ಕಂಡುಹಿಡಿಯುವ ಸಮಯವೊಂದು ಬರಬಹುದು. ಆ ಹಂತವನ್ನು ಮುಟ್ಟಿದಮೇಲೆ ಮತ್ತೆ ಅವರು ಮುಂದುವರಿಯಲಾರರು; ರಸಾಯನಶಾಸ್ತ್ರವು ಅಲ್ಲಿ ಪರಿಪೂರ್ಣತೆಯನ್ನು ಮುಟ್ಟುತ್ತದೆ. ಹಾಗೆಯೇ ಧರ್ಮ ವಿಜ್ಞಾನವೂ ಕೂಡ. ಆ ಪೂರ್ಣ ಏಕತೆಯನ್ನು ಪಡೆದ ಮೇಲೆ ಮತ್ತೆ ಪ್ರಗತಿ ಇರುವುದಿಲ್ಲ.

ಅನಂತರದ ಪ್ರಶ್ನೆಯೇ ನಾವು ಇಂತಹ ಏಕತೆಯನ್ನು ಕಾಣಬಹುದೆ ಎಂಬುದು. ಭರತಖಂಡದಲ್ಲಿ ಬಹಳ ಪುರಾತನ ಕಾಲದಿಂದಲೂ ಧರ್ಮಕ್ಕೆ ಮತ್ತು ತತ್ತ್ವಕ್ಕೆ ಸೇರಿದ ಒಂದು ವಿಜ್ಞಾನವನ್ನು ಕಂಡುಹಿಡಿಯಲು ಪ್ರಯತ್ನಿಸಿರುವರು. ಪಾಶ್ಚಾತ್ಯರಲ್ಲಿ ಧರ್ಮ ಮತ್ತು ತತ್ತ್ವ ಬೇರೆ ಆಗಿರುವಂತೆ ಹಿಂದೂಗಳು ಅವುಗಳನ್ನು ಬೇರ್ಪಡಿಸುವುದಿಲ್ಲ. ನಾವು ಧರ್ಮ ಮತ್ತು ತತ್ತ್ವಗಳೆರಡೂ ಒಂದೇ ವಸ್ತುವಿನ ಎರಡು ಮುಖಗಳೆಂದು ನೋಡುತ್ತೇವೆ. ಇವೆರಡೂ ಯುಕ್ತಿಯಲ್ಲಿ ಮತ್ತು ವೈಜ್ಞಾನಿಕ ಸತ್ಯದಲ್ಲಿ ಪ್ರತಿಷ್ಠಿತವಾಗಿರಬೇಕು.

ಸಾಂಖ್ಯಸಿದ್ದಾಂತ ಭರತಖಂಡದಲ್ಲೆಲ್ಲಾ, ಅಷ್ಟೇ ಏಕೆ, ಇಡೀ ಪ್ರಪಂಚದಲ್ಲೇ ಅತಿ ಪುರಾತನವಾದುದು. ಅದರ ಕರ್ತೃಗಳಾದ ಕಪಿಲ ಮಹರ್ಷಿಗಳು ಹಿಂದೂ ಮನಃಶಾಸ್ತ್ರಕ್ಕೆಲ್ಲ ಪಿತರಂತೆ ಇರುವರು. ಅವರು ಬೋಧಿಸಿದ ಹಳೆಯ ಸಿದ್ದಾಂತವೇ ಸರ್ವಮಾನ್ಯವಾದ ಷಡ್ದರ್ಶನಗಳಿಗೆ ತಳಹದಿಯಾಗಿದೆ. ಇವುಗಳೆಲ್ಲ (ಇನ್ನು ಬೇರೆ ವಿಷಯಗಳಲ್ಲಿ ವ್ಯತ್ಯಾಸವಿದ್ದರೂ) ಕಪಿಲರ ಮನಃಶಾಸ್ತ್ರವನ್ನು ಒಪ್ಪಿಕೊಳ್ಳುವುವು.

ಸಾಂಖ್ಯಸಿದ್ಧಾಂತದ ಯುಕ್ತಿಪೂರಿತ ಪರಿಣಾಮವೆ ವೇದಾಂತ. ಇದು ನಿರ್ಣಯಗಳನ್ನು ಮತ್ತೂ ಮುಂದಕ್ಕೆ ಒಯ್ಯುವುದು. ವೇದಾಂತದ ಸೃಷ್ಟಿ–ಸಿದ್ಧಾಂತ \enginline{(cosmology)} ಕಪಿಲರ ಸಾಂಖ್ಯದಂತೆ ಇದ್ದರೂ ವೇದಾಂತವು ದ್ವೈತದಲ್ಲಿ ಕೊನೆಗಾಣಲು ಇಚ್ಚಿಸುವುದಿಲ್ಲ. ಧರ್ಮದ ಮತ್ತು ವಿಜ್ಞಾನದ ಗುರಿಯಂತೆ, ಇದೂ ಕೂಡ ಪರಮ ಏಕತೆಯ ಕಡೆಗೆ\break ಮುಂದುವರಿಯುವುದು.

