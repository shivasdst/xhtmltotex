
\chapter[ಶ್ವಾಸ–ಪ್ರಶ್ವಾಸ ಕ್ರಿಯೆ]{ಶ್ವಾಸ–ಪ್ರಶ್ವಾಸ ಕ್ರಿಯೆ\protect\footnote{\engfoot{C.W. Vol. I, P. 503}}}

\begin{center}
(೨೮ನೇ ಮಾರ್ಚ್ ೧೯೦೦ ರಂದು ಸ್ಯಾನ್‌ಫ್ರಾನ್ಸಿಸ್ಕೋನಲ್ಲಿ ಕೊಟ್ಟ ಉಪನ್ಯಾಸ)
\end{center}

\vskip 2pt

ಅತ್ಯಂತ ಪ್ರಾಚೀನ ಕಾಲದಿಂದಲೂ ಭಾರತದಲ್ಲಿ ಶ್ವಾಸ–ಪ್ರಶ್ವಾಸದ ನಿಯಂತ್ರಣದ ಅಭ್ಯಾಸಗಳು ಬಹಳ ಜನಪ್ರಿಯವಾಗಿವೆ. ಹೇಗೆ ನಿಮ್ಮಲ್ಲಿ ಚರ್ಚಿಗೆ ಹೋಗುವುದು, ಕೆಲವು ಸ್ತೋತ್ರಾದಿಗಳನ್ನು ಪಠಿಸುವುದು ಮುಂತಾದುವು ನಿಮ್ಮ ಧಾರ್ಮಿಕ ಜೀವನದ ಅಂಗವಾಗಿವೆಯೋ ಹಾಗೆಯೇ ಈ ಪ್ರಾಣಾಯಾಮದ ಅಭ್ಯಾಸ ಸಹ ನಮ್ಮ ಜನರ ಧಾರ್ಮಿಕ ಜೀವನದ ಅಂಗವಾಗಿದೆ. ಅದರ ಕೆಲವು ವಿಚಾರಗಳನ್ನು ನಿಮ್ಮ ಮುಂದಿಡಲು ಪ್ರಯತ್ನಿಸುತ್ತೇನೆ.

\vskip 2pt

ಭಾರತೀಯ ದಾರ್ಶನಿಕರು ಹೇಗೆ ಈ ಸಮಸ್ತ ಬ್ರಹ್ಮಾಂಡವನ್ನು ಪ್ರಾಣ ಮತ್ತು ಆಕಾಶಗಳೆಂಬ ಎರಡು ವಸ್ತುಗಳಿಗೇ ಸೀಮಿತಗೊಳಿಸಿರುತ್ತಾರೆ ಎಂಬುದನ್ನು ಈ ಮೊದಲೇ ನಾನು ಹೇಳಿದ್ದೇನೆ.

ಪ್ರಾಣ ಎಂದರೆ ಶಕ್ತಿ; ಯಾವುದೇ ಚಲನೆಯಲ್ಲಿ ಅಥವಾ ಸಂಭವಿಸಬಹುದಾದ ಚಲನೆಯಲ್ಲಿ, ಒತ್ತಡ, ಅಥವಾ ಆಕರ್ಷಣೆಯಲ್ಲಿ – ಅಭಿವ್ಯಕ್ತವಾಗುತ್ತಿರುವ ಶಕ್ತಿ ಎಲ್ಲವೂ ಪ್ರಾಣವೇ. ವಿದ್ಯುತ್, ಅಯಸ್ಕಾಂತ ಶಕ್ತಿ, ಶರೀರದಲ್ಲಿನ ಎಲ್ಲ ರೀತಿಯ ಚಲನೆಗಳು, ಮನಸ್ಸಿನ ಎಲ್ಲ ರೀತಿಯ ಗತಿಗಳು – ಈ ಎಲ್ಲ ಶಕ್ತಿ ಸಮೂಹಗಳೂ ಪ್ರಾಣವೆಂದು ಕರೆಯಲ್ಪಡುವ ಒಂದೇ ವಸ್ತುವಿನ ವಿವಿಧ ಅಭಿವ್ಯಕ್ತಿಗಳಷ್ಟೇ. ಆದರೂ ಸಹ, ಅರಿವಿನ ಬೆಳಕಾಗಿ ಪ್ರಕಾಶಗೊಳ್ಳುತ್ತಿರುವ ಪ್ರಾಣದ ಸರ್ವಶ್ರೇಷ್ಠವಾದ ಅಭಿವ್ಯಕ್ತಿ ಮೆದುಳಿನಲ್ಲಿದೆ. ಈ ಅರಿವು ವಿಚಾರ ಶಕ್ತಿಯಿಂದ ನಿರ್ದಿಷ್ಟವಾಗಿರುತ್ತದೆ.

\vskip 2pt

ಶರೀರದಲ್ಲಿ ಉಪಯೋಗಿಸಲ್ಪಡುತ್ತಿರುವ ಪ್ರಾಣದ ಪ್ರತಿಯೊಂದು ತುಣುಕೂ ಮನಸ್ಸಿನ ನಿಯಂತ್ರಣದಲ್ಲಿರಬೇಕು. ಸಮಸ್ತ ಶರೀರವೂ ಮನಸ್ಸಿನ ಅಧೀನದಲ್ಲಿರಬೇಕು. ಆದರೆ ಎಲ್ಲರ ವಿಷಯದಲ್ಲಿಯೂ ಆ ರೀತಿ ಇಲ್ಲ. ನಮ್ಮಲ್ಲಿನ ಬಹುಭಾಗ ಜನರ ವಿಷಯದಲ್ಲಿ ಇದು ವ್ಯತಿರಿಕ್ತವಾಗಿದೆ. (ಶರೀರದಲ್ಲಿ ಪ್ರತಿಯೊಂದು ಅಂಗವನ್ನೂ ಕೇವಲ ಸಂಕಲ್ಪ ಮಾತ್ರದಿಂದಲೇ ಮನಸ್ಸು ಹತೋಟಿಯಲ್ಲಿಡುವಂತಾಗಬೇಕು. (ನಿಜವಾದ) ವೈಚಾರಿಕತೆ, ತಾತ್ತ್ವಿಕತೆ ಎಂದರೆ ಅದು. ಆದರೆ ವಸ್ತುಸ್ಥಿತಿ ಹಾಗಿಲ್ಲ. ನಿಮ್ಮಲ್ಲಾದರೋ ಇದಕ್ಕೆ ವ್ಯತಿರಿಕ್ತವಾಗಿ – ಗಾಡಿ ಮುಂದೆ ಕುದುರೆ ಅದರ ಹಿಂದೆ (ಇರುವಂತೆ) – ಶರೀರವೇ ಮನಸ್ಸನ್ನು ವಶಪಡಿಸಿಕೊಂಡಿದೆ. ನನ್ನ ಬೆರಳೇನಾದರೂ ಗಾಯಗೊಂಡರೆ ನನಗೆ ದುಃಖವಾಗುತ್ತದೆ. ಈ ರೀತಿ ಶರೀರವು ಮನಸ್ಸಿನ ಮೇಲೆ ಕರ್ತೃತ್ವ ನಡೆಸುತ್ತಿದೆ. ದೇಹದ ಪ್ರಭಾವ ಮನಸ್ಸಿನ ಮೇಲೆ ನಡೆಯುತ್ತಲೇ ಇದೆ. ನನಗೆ ಇಷ್ಟವಿಲ್ಲದ್ದು ಏನಾದರೂ ಸಂಭವಿಸಿದರೆ ಕೂಡಲೇ ನನಗೆ ಚಿಂತೆ ಹತ್ತಿಕೊಳ್ಳುತ್ತದೆ; ನನ್ನ ಮನಸ್ಸಿನ ಸಮತೋಲನ ತಪ್ಪುತ್ತದೆ. ಈ ಅವಸ್ಥೆಯಲ್ಲಿ ದೇಹವೇ ಮನಸ್ಸಿನ ನಿಯಂತೃವಾಗಿದೆ. ನಾವೆಲ್ಲ ಶರೀರಮಾತ್ರವೇ ಆಗಿರುತ್ತೇವೆ. ಈಗಂತೂ ನಾವು ದೇಹವಲ್ಲದೆ, ದೇಹಕ್ಕೆ ಅಕೀಕವಾಗಿ ಬೇರೇನೂ ಆಗಿಲ್ಲ.

\vskip 2pt

ಇಲ್ಲಿಯೇ ದಾರ್ಶನಿಕ ನಮಗೆ ದಾರಿ ತೋರಲು, ಹಾಗೂ ನಾವು ನಿಜವಾಗಿಯೂ ಏನಾಗಿದ್ದೇವೆ ಎಂಬುದನ್ನು ತಿಳಿಸಲು ಬರುವನು. ನೀವು ಈ ವಿಷಯದಲ್ಲಿ ತರ್ಕ ಮಾಡಬಹುದು ಹಾಗೂ ಬುದ್ಧಿಯ ದ್ವಾರಾ ಇದನ್ನು ಗ್ರಹಿಸಬಹುದು; ಆದರೆ ಇದನ್ನು ಬುದ್ದಿಯ ಮೂಲಕ ಗ್ರಹಿಸುವುದಕ್ಕೂ ಮತ್ತು ವಾಸ್ತವಿಕವಾಗಿ ಅನುಭವ ಪಡೆಯುವುದಕ್ಕೂ ಬಹಳ ಅಂತರವಿದೆ. ಕಟ್ಟಡದ ನಕ್ಷೆಗೂ ನಿರ್ಮಾಣವಾದ ಕಟ್ಟಡಕ್ಕೂ ಬಹಳ ದೊಡ್ಡ ಅಂತರವಿದೆ. ಆದ್ದರಿಂದ ಧರ್ಮದ ಗುರಿಯಾದ ಅನುಭೂತಿಯನ್ನು ತಲುಪಲು ವಿವಿಧ ಮಾರ್ಗಗಳಿರಲೇಬೇಕು. ಹಿಂದಿನ ಉಪನ್ಯಾಸದಲ್ಲಿ ತತ್ತ್ವಜ್ಞಾನದ ವಿಧಾನವನ್ನು ಅಧ್ಯಯನ ಮಾಡುತ್ತಿದ್ದೆವು. ಅದೇ ಆತ್ಮದ (ನೈಜ) ಸ್ವಾತಂತ್ರ್ಯವನ್ನು ಸಾಧಿಸಿ, ಎಲ್ಲವನ್ನೂ ನಿಯಂತ್ರಣಕ್ಕೆ ತರಲು ಪ್ರಯತ್ನಿಸುವುದು... “ಇದು ಬಹಳ ಕಠಿಣ. ಈ ಸಾಧನೆ ಸರ್ವಸುಲಭವಲ್ಲ. ದೇಹಬದ್ಧವಾದ ಮನಸ್ಸು ಬಹಳ ಕಷ್ಟದಿಂದ ಇದನ್ನು ಪ್ರಯತ್ನಿಸುತ್ತದೆ.” (ಗೀತೆ: \enginline{12.5).}

ಶಾರೀರಿಕ ಸಹಾಯ ಸ್ವಲ್ಪ ದೊರಕಿದರೆ ಮನಸ್ಸು ಶಾಂತವಾಗುತ್ತದೆ. ಮನಸ್ಸಿನಿಂದಲೇ ಈ ಸತ್ಯದ ಅನುಭೂತಿಯನ್ನು ಸಾಧಿಸುವಂತಾದರೆ ಅದಕ್ಕಿಂತ ಯುಕ್ತಾಯುಕ್ತವಾದುದು, ತರ್ಕಸಮ್ಮತವಾದುದು ಯಾವುದಿದೆ? ಆದರೆ ಇದು ಹಾಗಾಗಲಾರದು. ನಮ್ಮಲ್ಲಿನ ಬಹಳ ಮಂದಿಗೆ ಶಾರೀರಿಕ ಸಹಾಯ ಬೇಕೇ ಬೇಕು. ಈ ಶಾರೀರಿಕ ಸಹಾಯಗಳನ್ನು ಉಪಯೋಗಿಸುವುದು, ದೇಹದಲ್ಲಿರುವ ಶಕ್ತಿ ಮತ್ತು ಸಾಮರ್ಥ್ಯಗಳನ್ನು ಬಳಸಿ ಕೆಲವು (ಉನ್ನತ) ಮಾನಸಿಕ ಸ್ಥಿತಿಗಳನ್ನುಂಟುಮಾಡುವುದು, ಮನಸ್ಸು ತನ್ನಿಂದ ಲುಪ್ತವಾದ ಸಾಮ್ರಾಜ್ಯವನ್ನು ಮರಳಿ ಪಡೆಯುವವರೆಗೂ ಮನಸ್ಸನ್ನು ಶಕ್ತಿಶಾಲಿಯನ್ನಾಗಿ ಮಾಡುವುದು – ಇವೇ ರಾಜಯೋಗದ ವಿಧಾನವಾಗಿದೆ. ಕೇವಲ ತಮ್ಮ ಇಚ್ಛಾಶಕ್ತಿಯಿಂದಲೇ ಯಾರಾದರೂ ಈ ಸ್ಥಿತಿಯನ್ನು ಸಾಧಿಸಿದರೆ – ಅದು ಒಳ್ಳೆಯದೇ. ಆದರೆ ನಮ್ಮಲ್ಲಿನ ಬಹುಮಂದಿಗೆ ಇದು ಕೈಲಾಗದ ಕೆಲಸ. ಆದ್ದರಿಂದಲೇ ನಾವು ಶಾರೀರಿಕ ಸಹಾಯವನ್ನು ಬಳಸಿ ಇಚ್ಛಾಶಕ್ತಿಯ ದಾರಿಯನ್ನು ಸುಗಮ ಮಾಡುತ್ತೇವೆ.

ಈ ಸಮಸ್ತ ವಿಶ್ವವೂ ವೈವಿಧ್ಯತೆಯಲ್ಲಿ ಏಕತೆಗೆ ಒಂದು ವಿರಾಟ್ ನಿದರ್ಶನ. ಒಂದೇ ವಿಶ್ವಮನಸ್ಸು ಇರುವುದು – ಆ ಮನಸ್ಸಿನ ವಿವಿಧ ಅವಸ್ಥೆಗಳಿಗೆ ವಿಭಿನ್ನವಾದ ಹೆಸರುಗಳಿವೆ. (ಅವೆಲ್ಲಾ) ಈ ಮಾನಸ ಸಾಗರದಲ್ಲಿ ವಿಭಿನ್ನವಾದ ಚಿಕ್ಕ ಚಿಕ್ಕ ಸುಳಿಗಳಂತೆ ಅಥವಾ ಆವರ್ತಗಳಂತೆ ಇವೆ. ನಾವು ಏಕಕಾಲದಲ್ಲಿ ವ್ಯಷ್ಟಿಯೂ – ಸಮಷ್ಟಿಯೂ ಆಗಿದ್ದೇವೆ. ಈ ರೀತಿಯಾಗಿ ಲೀಲೆ ನಡೆಯುತ್ತಿದೆ. ವಾಸ್ತವದಲ್ಲಿ ಈ ಏಕತ್ವ ಎಂದಿಗೂ ಭಿನ್ನವಾಗುವುದಿಲ್ಲ. (ಜಡ–ದ್ರವ್ಯ, ಮನಸ್ಸು, ಮತ್ತು ಆತ್ಮ – ಇವು ಮೂರೂ ಒಂದೇ.)

ಇವೆಲ್ಲವೂ ಕೇವಲ ವಿವಿಧ ನಾಮಗಳಷ್ಟೇ. ಈ ಬ್ರಹ್ಮಾಂಡದಲ್ಲಿರುವುದು ಒಂದೇ ಸತ್ಯ; ನಾವು ಅದನ್ನೇ ವಿವಿಧ ದೃಷ್ಟಿಕೋನಗಳಿಂದ ನೋಡುತ್ತೇವೆ. ಒಂದು ದೃಷ್ಟಿಕೋನದಿಂದ ನೋಡಿದಾಗ ಇದೇ ಸತ್ಯ ಜಡತತ್ತ್ವವಾಗಿರುತ್ತದೆ; ಇನ್ನೊಂದು ದೃಷ್ಟಿಕೋನದಿಂದ ಅದೇ ಮನಸ್ಸಾಗಿದೆ. ಇವೆರಡು ವಿಭಿನ್ನ ವಸ್ತುಗಳೇನೂ ಅಲ್ಲ. ಹಗ್ಗವನ್ನು ಹಾವೆಂದು ಎಣಿಸಿದಾಗ (ಆ ಮನುಷ್ಯನಿಗೆ) ಭಯವುಂಟಾಗಿ ಆ ಹಾವನ್ನು ಕೊಲ್ಲಲು ಬೇರೆಯವರನ್ನು ಕೂಗುವಂತೆ ಮಾಡಿತು. ಅವನ ನರಮಂಡಲ ಅದುರಲು ಪ್ರಾರಂಭವಾಯಿತು. ಹೃದಯದ ಬಡಿತ ಹೆಚ್ಚಾಯಿತು. ಈ ಎಲ್ಲ ಸ್ಥಿತಿಗಳೂ ಭಯದಿಂದ ಬಂದಿದ್ದುವು ಮತ್ತು ಅದನ್ನು ಹಗ್ಗವೆಂದು ಕಂಡುಹಿಡಿದಾಗ ಅವೆಲ್ಲವೂ ನಾಶವಾದುವು. ನಾವು ವಾಸ್ತವದಲ್ಲಿ ನೋಡುತ್ತಿರುವುದು ಇದನ್ನೇ. ನಮ್ಮ ಇಂದ್ರಿಯಗಳಿಗೆ ಗೋಚರವಾಗುತ್ತಿರುವ ಏನನ್ನು ನಾವು ಜಡ ಎಂದು ಕರೆಯುತ್ತೇವೆಯೋ ಅದೂ ಸಹ ಸತ್ಯವೇ – ಅದನ್ನು ನಾವು ನೋಡುತ್ತಿರುವ ರೂಪದಲ್ಲಿ ಅದು ಇಲ್ಲ ಅಷ್ಟೆ! ಹಗ್ಗವನ್ನು ನೋಡಿ ಹಾವೆಂದು ಎಣಿಸಿದ ಮನಸ್ಸಿಗೆ ಭ್ರಮೆಯಾಗಿರಲಿಲ್ಲ. ಹಾಗೇನಾದರೂ ಮನಸ್ಸು ಭ್ರಮೆಯಲ್ಲಿದ್ದಿದ್ದರೆ ಅದು ಏನನ್ನೂ ನೋಡುತ್ತಿರಲಿಲ್ಲ. ಒಂದು ವಸ್ತುವನ್ನು – ಮತ್ತೊಂದು ವಸ್ತುವಾಗಿ ಎಣಿಸಲಾಯಿತಷ್ಟೇ ವಿನಃ ಯಾವುದು ಅಸ್ತಿತ್ವದಲ್ಲಿಯೇ ಇಲ್ಲವೋ ಅಂತಹದು ಯಾವುದನ್ನೂ ನೋಡಿದ್ದಲ್ಲ. ನಾವು ಇಲ್ಲಿ ಶರೀರವನ್ನು ನೋಡುತ್ತಿದ್ದೇವೆ ಹಾಗೂ ಅನಂತವನ್ನು ಜಡ ವಸ್ತುವೆಂದು ತಿಳಿದಿದ್ದೇವೆ – (ಜಡವಸ್ತುವಿನ ರೂಪದಲ್ಲಿ ಗ್ರಹಿಸುತ್ತಿದ್ದೇವೆ.) ಹಾಗೆಯೇ ನಾವು ಅರಸುತ್ತಿರುವುದಾದರೋ ಪರಮಸತ್ಯವನ್ನೇ ನಾವೆಂದೂ ಭ್ರಾಂತಿಗೊಳಗಾಗಿಲ್ಲ. ನಾವು ಸದಾ ತಿಳಿಯುವುದು ಸತ್ಯವನ್ನೇ – ನಾವು ಯಾವುದನ್ನು ಸತ್ಯವೆಂದು ಎಣಿಸುತ್ತೇವೆಯೋ ಅದು ಕೆಲವೊಮ್ಮೆ ತಪ್ಪಾಗಿರುತ್ತದೆ ಅಷ್ಟೆ. ನೀವು ಒಂದು ಬಾರಿಗೆ ಒಂದು ವಸ್ತುವನ್ನಷ್ಟೇ ನೋಡಬಲ್ಲಿರಿ. ನಾನು ಯಾವಾಗ ಹಾವನ್ನು ನೋಡುತ್ತೇನೆಯೋ ಆಗ ಹಗ್ಗ ಸಂಪೂರ್ಣವಾಗಿ ಕಣ್ಮರೆಯಾಗಿರುತ್ತದೆ ಮತ್ತೆ ನಾನು ಯಾವಾಗ ಹಗ್ಗವನ್ನು ಕಾಣುತೇನೆಯೋ ಆಗ ಹಾವು ಮಾಯವಾಗಿರುತ್ತದೆ. ಒಂದು ವಸ್ತುವಷ್ಟೇ ಆಗಿರಬೇಕು.

ಯಾವಾಗ ನಾವು ಪ್ರಪಂಚವನ್ನು ನೋಡುತ್ತೇವೆಯೋ ಆಗ ಹೇಗೆ ದೇವರನ್ನು ಕಾಣಬಲ್ಲೆವು! ಆಲೋಚಿಸಿ ನೋಡಿ. ನಾವು ಪ್ರಪಂಚ ಎಂದು ಯಾವುದನ್ನು ಕರೆಯುತ್ತೇವೆಯೋ ಅದು ಪಂಚೇಂದ್ರಿಯಗಳಿಂದ ಗ್ರಹಿಸಲ್ಪಡುತ್ತಿರುವ ಸಮಷ್ಟಿ ಪದಾರ್ಥದ ರೂಪದಲ್ಲಿರುವ ದೇವರಲ್ಲದೆ ಮತ್ತೇನು? ಇಲ್ಲಿ ನೀವು ಹಾವನ್ನು ನೋಡುತ್ತೀರಿ; ಹಗ್ಗವಿಲ್ಲ – ನೀವು ಆತ್ಮನನ್ನು ತಿಳಿದಾಗ ಮಿಕ್ಕೆಲ್ಲವೂ ಅಂತರ್ಧಾನವಾಗುತ್ತದೆ. ನೀವು ಆತ್ಮನನ್ನೇ ನೋಡುವಾಗ ಇನ್ನಾವುದೇ ದ್ರವ್ಯವನ್ನು ಕಾಣಲಾರಿರಿ. ಕಾರಣ, ನೀವು ಯಾವುದನ್ನು ದ್ರವ್ಯ ಎಂದು ಕರೆಯುತ್ತಿದ್ದಿರೋ ಅದೇ ಆತ್ಮವಾಗಿದೆ. ಈ ವೈವಿಧ್ಯತೆಗಳೆಲ್ಲ ಇಂದ್ರಿಯಗಳಿಂದಾದ ಉಪಾಧಿಗಳಷ್ಟೇ. ಒಬ್ಬನೇ ಸೂರ್ಯ ಸಾವಿರ ಚಿಕ್ಕ ಅಲೆಗಳಲ್ಲಿ ಪ್ರತಿಫಲಿತನಾದಾಗ ಸಾವಿರಾರು ಚಿಕ್ಕ ಚಿಕ್ಕ ಸೂರ್ಯರಾಗಿ ತೋರುತ್ತಾನೆ. ನಾನು ಈ ವಿಶ್ವವನ್ನು ನನ್ನ ಇಂದ್ರಿಯಗಳಿಂದ ನೋಡುತ್ತಿದ್ದರೆ ಅದನ್ನು ದ್ರವ್ಯ ಮತ್ತು ಶಕ್ತಿ ಎಂದು ಅರ್ಥೈಸುತ್ತೇನೆ. ಇದು ಒಂದೇ ಕಾಲದಲ್ಲಿ – ಏಕವೂ ಅನೇಕವೂ ಆಗಿದೆ. ವೈವಿಧ್ಯತೆ ಏಕತೆಯನ್ನು ನಾಶ ಮಾಡುವುದಿಲ್ಲ. ಕೋಟ್ಯಾಂತರ ಅಲೆಗಳು ಸಮುದ್ರದ ಏಕತೆಯನ್ನು ನಾಶಮಾಡುವುದಿಲ್ಲ. ಅದು ಒಂದೇ ಸಮುದ್ರವಾಗಿಯೇ ಉಳಿಯುತ್ತದೆ. ನೀವು ವಿಶ್ವವನ್ನು ನೋಡಿದಾಗ ಅದನ್ನು ದ್ರವ್ಯ ಮತ್ತು ಶಕ್ತಿಗೆ ಸೀಮಿತಗೊಳಿಸ\break ಬಹುದೆಂಬುದನ್ನು ನೆನಪಿನಲ್ಲಿಡಿ. (ಯಾವುದೇ ವಸ್ತುವಿನ) ವೇಗವನ್ನು ಹೆಚ್ಚಿಸಿದಾಗ ಅದರ ದ್ರವ್ಯ ಪರಿಮಾಣ ಕಡಿಮೆಯಾಗುತ್ತದೆ... ಆದರೆ ಇದಕ್ಕೆ ವ್ಯತಿರಿಕ್ತವಾಗಿ ದ್ರವ್ಯದ ಪರಿಮಾಣವನ್ನು ಹೆಚ್ಚಿಸಿ ವೇಗವನ್ನು ಕಡಿಮೆಮಾಡಬಹುದು... ದ್ರವ್ಯದ ಪರಿಮಾಣ ಸಂಪೂರ್ಣವಾಗಿ ಅಂತರ್ಧಾನವಾಗುವ ಒಂದು ಹಂತಕ್ಕೆ ನಾವು ಹೆಚ್ಚು ಕಡಿಮೆ ತಲುಪಬಹುದು.

ಜಡದ್ರವ್ಯವು ಶಕ್ತಿಗೆ ಕಾರಣವೆಂದಾಗಲೀ ಅಥವಾ ಶಕ್ತಿಯು ಜಡದ್ರವ್ಯದ ಕಾರಣವೆಂದಾಗಲೀ ಕರೆಯಲಾಗುವುದಿಲ್ಲ. ಅವೆರಡೂ ಯಾವ ರೀತಿ ಒಂದಕ್ಕೊಂದು ಸಂಬಂಧಿಸಿವೆಯೆಂದರೆ ಒಂದು ಮತ್ತೊಂದರಲ್ಲಿ ವಿಲೀನವಾಗಬಹುದು. (ಆದ್ದರಿಂದ ಇವೆರಡಕ್ಕೂ ಕಾರಣವಾಗಿರುವ) ಮತ್ತೊಂದು ಮೂರನೆಯ ತತ್ತ್ವ ಇರಲೇಬೇಕು. ಆ ಯಾವುದೊ ಮೂರನೆಯ ತತ್ತ್ವವೇ ಮನಸ್ಸು. ವಿಶ್ವವು ಜಡದ್ರವ್ಯದಿಂದ ಉತ್ಪನ್ನವಾಗಲು ಸಾಧ್ಯವಿಲ್ಲ. ಶಕ್ತಿಯಿಂದಲೂ ಸಹ ವಿಶ್ವವು ಉತ್ಪನ್ನವಾಗಲಾರದು. ಮನಸ್ಸು ಶಕ್ತಿಯೂ ಅಲ್ಲದ, ಜಡದ್ರವ್ಯವೂ ಅಲ್ಲದ ಬೇರೊಂದು ವಸ್ತುವೇ ಆಗಿದೆ; ಆದರೂ ಶಕ್ತಿ ಮತ್ತು ಜಡದ್ರವ್ಯಗಳು ಹುಟ್ಟಿರುವುದು ಮನಸ್ಸಿನಿಂದಲೇ. ಕಟ್ಟಕಡೆಯಲ್ಲಿ ಮನಸ್ಸೇ ಎಲ್ಲ ಶಕ್ತಿಯ ಮೂಲವಾಗಿದೆ. ಆ ಮನಸ್ಸೇ ವಿಶ್ವಮನಸ್ಸು. ಇದು ಎಲ್ಲ (ವ್ಯಷ್ಟಿ) ಸಂಘಾತ. ಪ್ರತಿಯೊಬ್ಬರ ಮನಸ್ಸಿನ ಮೂಲಕವೂ ಸೃಜನಕಾರ್ಯ ನಡೆಯುತ್ತಿದೆ ಮತ್ತು ಈ ಎಲ್ಲ ಸೃಷ್ಟಿಗಳ ಸಮಷ್ಟಿಯೇ ಈ ವಿಶ್ವ, ವೈವಿಧ್ಯತೆಯಲ್ಲಿ ಏಕತೆ. ಇದು ಏಕವೂ ಆಗಿದೆ, ಅದೇ ಸಮಯದಲ್ಲಿ ಬಹುವೂ ಆಗಿದೆ.

ಸಗುಣ ಬ್ರಹ್ಮನೆಂದರೆ ಎಲ್ಲ ಜೀವರ ಸಮಷ್ಟಿ, ಆದರೂ ಮತ್ತೆ ಅದೇ ಒಂದು ಪ್ರತ್ಯೇಕ ವ್ಯಕ್ತಿತ್ವವುಳ್ಳದ್ದಾಗಿದೆ. ಹೇಗೆ ನೀವು ಪ್ರತ್ಯೇಕ ಅಸ್ತಿತ್ವವಿರುವ ಜೀವಕೋಶಗಳಿಂದ ಕೂಡಿದ ಪ್ರತ್ಯೇಕ ಶರೀರವಾಗಿದ್ದೀರೋ ಅದೇ ರೀತಿ ಸಗುಣ ಬ್ರಹ್ಮನೂ ಕೂಡ.

ಚಲನೆಯಿರುವ ಪ್ರತಿಯೊಂದೂ ಪ್ರಾಣದಲ್ಲಿ ಅಥವಾ ಶಕ್ತಿಯಲ್ಲಿ ನಿಹಿತವಾಗಿದೆ. ಈ ಪ್ರಾಣವೇ ನಕ್ಷತ್ರ, ಸೂರ್ಯ ಚಂದ್ರರನ್ನು ಚಲಿಸುತ್ತಿರುವುದು. ಪ್ರಾಣವೇ\break ಗುರುತ್ವಾಕರ್ಷಣೆ.

ಆದ್ದರಿಂದ ಪ್ರಾಕೃತಿಕ ಶಕ್ತಿಗಳೆಲ್ಲವೂ ಅವಶ್ಯವಾಗಿ ವಿಶ್ವ ಮನಸ್ಸಿನಿಂದಲೇ ಉದ್ಭೂತವಾಗಿರಬೇಕು. ಮನಸ್ಸಿನ ಸಣ್ಣ ತುಣುಕುಗಳಾದ ನಾವು ಪ್ರಕೃತಿಯಿಂದ ಆ ಪ್ರಾಣವನ್ನು ತೆಗೆದುಕೊಂಡು, ಅದನ್ನು ಮತ್ತೆ ನಮ್ಮ ವ್ಯಷ್ಟಿ ಪ್ರಕೃತಿಯಲ್ಲಿ ಬಳಸಿ ಶಾರೀರಿಕ ಕ್ರಿಯೆಗಳನ್ನು ನಡೆಸುತ್ತಾ, ಆಲೋಚನೆಗಳನ್ನು ಉತ್ಪಾದಿಸುತ್ತಾ ಹೋಗುತ್ತೇವೆ. ಆಲೋಚನೆಗಳನ್ನು ಉತ್ಪಾದಿಸಲಾಗುವುದಿಲ್ಲವೆಂದು ನೀವೆಂದುಕೊಂಡಿದ್ದರೆ, ಇಪ್ಪತ್ತು ದಿನ ಏನೂ ತಿನ್ನದೆ ಇದ್ದು ಹೇಗೆನ್ನಿಸುತ್ತದೋ ನೋಡಿ. ಇಂದೇ ಆರಂಭಿಸಿ – ಲೆಕ್ಕ ಹಾಕುತ್ತಾ ಹೋಗಿ, ಆಲೋಚನೆಯೂ ಸಹ ಆಹಾರದಿಂದಲೇ ಉತ್ಪನ್ನವಾಗುತ್ತದೆ. ಅದರಲ್ಲಿ ಸಂದೇಹವೇ ಇಲ್ಲ.

ಎಲ್ಲೆಲ್ಲೂ, ಪ್ರತಿಯೊಂದರಲ್ಲೂ ಕ್ರಿಯಾಶೀಲವಾಗಿರುವ ಹಾಗೂ ನಮ್ಮ ದೇಹದಲ್ಲಿರುವ ಈ ಪ್ರಾಣದ ನಿಯಂತ್ರಣವೇ ಪ್ರಾಣಯಾಮ ಎಂದು ಕರೆಯಲ್ಪಡುವುದು. ಪ್ರತಿಯೊಂದನ್ನೂ ನಡೆಸುತ್ತಿರುವುದೇ ಶ್ವಾಸ ಎಂಬುದನ್ನು ನಮಗಿರುವ ಸಾಮಾನ್ಯ ಜ್ಞಾನದಿಂದ ನಾವು ನೋಡಬಹುದಾಗಿದೆ. ನಾನು ಉಸಿರಾಡುವುದನ್ನು ನಿಲ್ಲಿಸಿದರೆ (ಶರೀರದ ಎಲ್ಲ ಕ್ರಿಯೆಗಳೂ) ನಿಲ್ಲುತ್ತವೆ. ಉಸಿರಾಡಲು ಆರಂಭಿಸಿದರೆ (ಶರೀರ) ಗತಿಶೀಲವಾಗುತ್ತದೆ. ನಾವು ಅವಶ್ಯವಾಗಿ ಪಡೆಯಬೇಕಾದದ್ದು ಶ್ವಾಸನಿರೋಧವನ್ನಲ್ಲ. ಶ್ವಾಸದ ಹಿಂದೆ ಇರುವ, ಶ್ವಾಸಕ್ಕಿಂತಲೂ ಸೂಕ್ಷ್ಮತರವಾದುದನ್ನು ವಶಗೊಳಿಸಿಕೊಳ್ಳುವುದೇ ನಮ್ಮ ಲಕ್ಷ್ಯ.

ಪ್ರಖ್ಯಾತ ದೊರೆಯೊಬ್ಬನಿಗೆ ಮಂತ್ರಿಯೊಬ್ಬನಿದ್ದ. ಒಮ್ಮೆ ದೊರೆ ಮಂತ್ರಿಯ ಮೇಲೆ ಅಸಮಾಧಾನಗೊಂಡು ಅವನನ್ನು ಗೋಪುರದ ಮೇಲೆ ಬಂಧಿಸಿಡಲು ಆಜ್ಞಾಪಿಸಿದನು. (ರಾಜಾಧಿಕಾರಿಗಳು) ಹಾಗೆಯೇ ಮಾಡಿದ ನಂತರ ಆ ಮಂತ್ರಿಯನ್ನು ಅಲ್ಲಿಯೇ ಸಾಯಲು ಬಿಡಲಾಯಿತು. ಮಂತ್ರಿಯ ಹೆಂಡತಿ ರಾತ್ರಿವೇಳೆ ಗೋಪುರದ ಹತ್ತಿರ ಬಂದು ತನ್ನ ಗಂಡನನ್ನು ಕೂಗಿದಳು. ಮಂತ್ರಿ ಅವಳಿಗೆ, “ಅಳುವುದರಿಂದ ಏನೂ ಪ್ರಯೋಜನವಿಲ್ಲ" ಎಂದು ಸಂತೈಸಿ, ಸ್ವಲ್ಪ ಜೇನುತುಪ್ಪ, ಜೀರುದುಂಬಿ, ಸಣ್ಣದಾದ ರೇಷ್ಮೆದಾರದ ಉಂಡೆ, ಗಟ್ಟಿದಾರದ ಉಂಡೆ ಹಾಗೂ ಒಂದು ಹಗ್ಗವನ್ನು ತರಲು ಹೇಳಿದನು. ರೇಷ್ಮೆದಾರದ ಒಂದು ತುದಿಯನ್ನು ಜೀರುದುಂಬಿಯ ಒಂದು ಕಾಲಿಗೆ ಕಟ್ಟಿ ಅದರ ಮುಖದ ಮೇಲ್ಬಾಗದಲ್ಲಿ ಜೇನುತುಪ್ಪವನ್ನು ಸವರಿ (ಮೇಲ್ಮುಖವಾಗಿ, ಗೋಪುರದ ಗೋಡೆಯ ಮೇಲೆ) ಬಿಡಲು ಹೇಳಿದನು. (ಜೀರುದುಂಬಿ ಜೇನು ತುಪ್ಪವನ್ನು ತಲುಪುವ ಆಸೆಯಿಂದ ಗೋಪುರದ ತುದಿಯನ್ನು ತಲುಪುವವರೆಗೂ) ಮೆಲ್ಲಗೆ ಮೇಲಕ್ಕೆ ತೆವಳಿಕೊಂಡು ಹೋಯಿತು. ಆಗ ಮಂತ್ರಿಯು ಜೀರುದುಂಬಿಯನ್ನು ಹಿಡಿದುಕೊಂಡು ಅದಕ್ಕೆ ಕಟ್ಟಿದ ರೇಷ್ಮೆ ಎಳೆಯನ್ನು ತೆಗೆದುಕೊಂಡನು. ನಂತರ (ಅದರಿಂದ) ಗಟ್ಟಿ ದಾರವನ್ನು, ನಂತರ ದಪ್ಪದಾರವನ್ನು ಕೊನೆಯಲ್ಲಿ ಹಗ್ಗವನ್ನು ಪಡೆದನು. ಮಂತ್ರಿಯು ಹಗ್ಗದ ಮೂಲಕ ಗೋಪುರದಿಂದ ಇಳಿದು ತಪ್ಪಿಸಿಕೊಂಡು ಹೋದನು. ಈ ನಮ್ಮ ಶರೀರದಲ್ಲಿ ಉಸಿರಾಟದ ಚಲನೆಯೇ “ರೇಷ್ಮೆಯ ನೂಲು”. ಇದನ್ನು ಸ್ವಾಧೀನಕ್ಕೆ ತೆಗೆದುಕೊಂಡು ನಾವು ನಾಡೀಪ್ರವಾಹವೆಂಬ ದಾರದ ಉಂಡೆಯನ್ನು ವಶಪಡಿಸಿಕೊಂಡು, ಅದರ ನಂತರ ಆಲೋಚನೆಗಳೆಂಬ ದಪ್ಪ ದಾರವನ್ನೂ ಕೊನೆಯದಾಗಿ ಪ್ರಾಣವೆಂಬ ಹಗ್ಗವನ್ನೂ ಪಡೆದು ಅದನ್ನು ನಿಗ್ರಹಿಸಿದರೆ – ನಾವು ಮುಕ್ತರಾಗುತ್ತೇವೆ.

ಭೌತಿಕ ಸ್ತರದಲ್ಲಿರುವ ವಸ್ತುಗಳ ಸಹಾಯದಿಂದ, ಸೂಕ್ಷ್ಮಾತಿಸೂಕ್ಷ್ಮವಾದುವುಗಳನ್ನು ಗ್ರಹಿಸುವ ಹಂತಕ್ಕೆ ನಾವು ಬರಬೇಕು. ವಿಶ್ವ ಒಂದೇ; ಅದರ ಯಾವುದೇ ಒಂದು ಬಿಂದುವನ್ನು ನೀವು ಸ್ಪರ್ಶಿಸಬಹುದು. ಆದರೆ ವಿಶ್ವದ ಎಲ್ಲ ಬಿಂದುಗಳೂ ಸಹ ಆ ಒಂದೇ ಬಿಂದುವಿನ ರೂಪಾಂತರವಷ್ಟೆ. ವಿಶ್ವದ ನೆಲೆಯಲ್ಲಿರುವ ಏಕತ್ವ ವಿಶ್ವದಾದ್ಯಂತ ಎಷ್ಟು\break ವ್ಯಾಪಕವೆಂದರೆ... ಸ್ಥೂಲವಾಗಿರುವಂತಹ ಉಸಿರಿನಿಂದಲೂ (ಅಥವಾ ಶ್ವಾಸ – ಪ್ರಶ್ವಾಸ ಕ್ರಿಯೆಯಿಂದಲೂ) ನಾನು ಆತ್ಮವನ್ನೇ ಪಡೆಯಬಹುದು.

ಪ್ರಾಣಯಾಮದ ಅಭ್ಯಾಸದಿಂದ ಈಗ ನಮ್ಮ ಅನುಭವಕ್ಕೆ ಬಾರದ ಶರೀರದ\break ಸೂಕ್ಷ್ಮವಾದ ಸ್ಪಂದನಗಳೆಲ್ಲವನ್ನೂ ನಾವು ಅನುಭವಿಸಲು ಪ್ರಾರಂಭಿಸುತ್ತೇವೆ. ನಮ್ಮ ಅನುಭವಕ್ಕೆ ಬಂದೊಡನೆಯೇ ನಾವು ಅವುಗಳನ್ನು ನಮ್ಮ ನಿಯಂತ್ರಣದಲ್ಲಿಡಬಹುದು. ಆಲೋಚನೆಗಳು ತಮ್ಮ ಅಂಕುರಾವಸ್ಥೆಯಲ್ಲಿಯೇ ನಮಗೆ ಸ್ಫುಟಿತವಾಗುತ್ತವೆ. ಹೀಗಾಗಿ ನಾವು ಅವುಗಳನ್ನು ಸ್ವಾಧೀನದಲ್ಲಿಡಲು ಸಮರ್ಥರಾಗುತ್ತೇವೆ. ನಿಸ್ಸಂದೇಹವಾಗಿ ಇಂತಹ (ಅವಸ್ಥೆಯ) ಸಾಧನೆಗಾಗಿ ನಮ್ಮಲ್ಲಿ ಎಲ್ಲರಿಗೂ ಅವಕಾಶವಾಗಲೀ, ಅಥವಾ ಇಚ್ಛಾ ಶಕ್ತಿಯಾಗಲೀ, ತಾಳ್ಮೆಯಾಗಲೀ, ಶ್ರದ್ದೆಯಾಗಲೀ ಇಲ್ಲ. ಆದರೆ ಇದರ ಸಂಬಂಧವಾದ ಸಾಮಾನ್ಯ ಜ್ಞಾನ ಪ್ರತಿಯೊಬ್ಬರಿಗೂ ಸ್ವಲ್ಪವಾದರೂ ಪ್ರಯೋಜನವಾಗಿಯೇ ಆಗುತ್ತದೆ.

ಮೊದಲನೆ ಪ್ರಯೋಜನವೇ ಸ್ವಾಸ್ಥ್ಯ, ಆರೋಗ್ಯ. ನಮ್ಮಲ್ಲಿನ ಶೇ.\enginline{99} ಮಂದಿ ಸರಿಯಾದ ರೀತಿಯಲ್ಲಿ ಉಸಿರಾಡುವುದೇ ಇಲ್ಲ. ನಾವು ನಮ್ಮ ಶ್ವಾಸಕೋಶಗಳನ್ನು ಪೂರ್ತಿ (ಉಸಿರಿನಿಂದ) ತುಂಬಿಸುವುದೇ ಇಲ್ಲ. ನಿಯಮಿತವಾದ ಉಸಿರಾಟ ದೇಹವನ್ನು ಶುದ್ಧಗೊಳಿಸುತ್ತದೆ... ಅದು ಮನಸ್ಸನ್ನು ಶಾಂತಗೊಳಿಸುತ್ತದೆ. ನೀವು ಶಾಂತರಾಗಿದ್ದಾಗ ನಿಮ್ಮ ಉಸಿರಾಟ ಶಾಂತವಾಗಿ ನಡೆಯುತ್ತಿದ್ದು ಲಯಬದ್ದವಾಗಿರುತ್ತದೆ. ನಿಮ್ಮ ಉಸಿರಾಟ ಲಯಬದ್ಧವಾಗಿದ್ದರೆ, ನೀವು ಶಾಂತವಾಗಿರಲೇಬೇಕು. ಮನಸ್ಸು ವಿಚಲಿತವಾದಾಗ ಉಸಿರಾಟದ ಲಯವೂ ತಪ್ಪುತ್ತದೆ. ನೀವೇನಾದರೂ ಸಾಧನೆಯ ಮೂಲಕ ಉಸಿರಾಟವನ್ನು ಬಲಪೂರ್ವಕವಾಗಿ ಲಯಬದ್ಧ ಗೊಳಿಸಿದರೆ ನೀವು ಶಾಂತರಾಗಿರಬಾರದೇಕೆ? ನೀವು ವಿಚಲಿತರಾದಾಗ ಕೋಣೆಯಲ್ಲಿ ಹೋಗಿ ಬಾಗಿಲನ್ನು ಮುಚ್ಚಿ ಮನಸ್ಸನ್ನು ನಿಯಂತ್ರಿಸಲು ಪ್ರಯತ್ನಿಸಬೇಡಿ; (ಬದಲಾಗಿ) ಕೇವಲ ಹತ್ತು ನಿಮಿಷಗಳು ಲಯಬದ್ಧವಾಗಿ ಉಸಿರಾಡುತ್ತಾ ಹೋಗಿ, ಅಂತರಂಗ ಶಾಂತವಾಗುತ್ತದೆ. ಇವೆಲ್ಲವೂ ಸಹ ಪ್ರತಿಯೊಬ್ಬರ ಸಾಮಾನ್ಯ ಜ್ಞಾನಕ್ಕೆ ಬರುವ ಉಪಯೋಗಗಳು. ಮಿಕ್ಕವು ಯೋಗಿಗೆ ಸಾಧ್ಯವಾಗುವಂತಹವು.

ದೀರ್ಘ ಶ್ವಾಸ – ಪ್ರಶ್ವಾಸದ ಅಭ್ಯಾಸಗಳು ಕೇವಲ ಮೊದಲನೆಯ ಹಂತ. ಭಿನ್ನ\break ಭಿನ್ನವಾದ ಅಭ್ಯಾಸಗಳಿಗೋಸ್ಕರ ಸುಮಾರು \enginline{84} ಆಸನಗಳಿವೆ. ಕೆಲವರು ಈ ಶ್ವಾಸ – ಪ್ರಶ್ವಾಸ ಪ್ರಕ್ರಿಯೆಗಳನ್ನೇ ತಮ್ಮ ಜೀವನದ ಪೂರ್ಣ ಸಾಧನೆಯನ್ನಾಗಿ ತೆಗೆದುಕೊಂಡಿರುತ್ತಾರೆ. ಶ್ವಾಸದ ಸ್ವರವನ್ನು ಪರಾಮರ್ಶಿಸದೆ ಅವರು ಏನನ್ನೂ ಮಾಡುವುದಿಲ್ಲ. ಮೂಗಿನ ಯಾವ ಹೊಳ್ಳೆಯಲ್ಲಿ ಹೆಚ್ಚು ಉಸಿರಾಟವಾಗುತ್ತಿದೆ ಎಂಬುದನ್ನೇ ಅವರು ಸದಾ ಗಮನಿಸುತ್ತಿರುತ್ತಾರೆ. ಮೂಗಿನ ಬಲಗಡೆಯ ಹೊಳ್ಳೆಯಿಂದ ಹೆಚ್ಚಿನ ಉಸಿರಾಟವಿದ್ದಾಗ ಕೆಲವು ವಿಶಿಷ್ಟವಾದ ಕೆಲಸಗಳನ್ನು ಮಾಡುತ್ತಾರೆ, ಮತ್ತೆ ಎಡಗಡೆಯ ಹೊಳ್ಳೆಯಲ್ಲಿದ್ದಾಗ ಇನ್ನು ಕೆಲವನ್ನು ಮಾಡುತ್ತಾರೆ. ಮೂಗಿನ ಎರಡೂ ಹೊಳ್ಳೆಗಳಲ್ಲಿ ಉಚ್ಛ್ವಾಸ – ನಿಃಶ್ವಾಸಗಳು ಸಮವಾಗಿದ್ದಾಗ ಅವರು ಉಪಾಸನೆಯನ್ನು ಮಾಡುತ್ತಾರೆ.

ಮೂಗಿನ ಎರಡೂ ಹೊಳ್ಳೆಗಳಿಂದ ಲಯಸಹಿತವಾಗಿ ಉಸಿರಾಟ ನಡೆಯುತ್ತಿರುವಾಗಲೇ ನಿಮ್ಮ ಮನಸ್ಸನ್ನು ನಿಯಂತ್ರಿಸಲು ಸರಿಯಾದ ಸಮಯ. ಶ್ವಾಸದ ಸಹಾಯದಿಂದ ನೀವು ಇಚ್ಚಾಮಾತ್ರದಿಂದ ಶರೀರದ ಯಾವ ಭಾಗದ ಮೂಲಕವಾದರೂ ಸ್ನಾಯು ಪ್ರವಾಹವನ್ನು ಅಥವಾ ಶಕ್ತಿ ತರಂಗಗಳನ್ನು ಹಾಯಿಸಬಹುದು. ಶರೀರದ ಯಾವುದಾದರೂ ಅವಯವಕ್ಕೆ ಖಾಯಿಲೆಯಾದರೆ ಆ ಭಾಗಕ್ಕೆ ಶ್ವಾಸದ ಸಹಾಯದಿಂದಲೇ ಪ್ರಾಣವನ್ನು ಕಳುಹಿಸಿ (ಖಾಯಿಲೆಯನ್ನು ಉಪಶಮನಗೊಳಿಸಬಹುದು.)

ಇನ್ನೂ ಅನೇಕ ನಾನಾ ವಿಧವಾದ ಯೌಗಿಕ ಕ್ರಿಯೆಗಳು ಪ್ರಚಲಿತವಾಗಿವೆ. ಉಸಿರಾಟವನ್ನೇ ನಿಲ್ಲಿಸಲು ಪ್ರಯತ್ನಿಸುವ ಅನೇಕ ಸಂಪ್ರದಾಯಗಳಿವೆ. ಗಟ್ಟಿಯಾಗಿ ಉಸಿರಾಡುವಂತೆ ಮಾಡುವ ಯಾವ ಕೆಲಸವನ್ನೂ ಅವರು ಮಾಡುವುದಿಲ್ಲ. ಅವರು (ಎಂತಹ) ಒಂದು ರೀತಿಯ ಸಮಾಧಿಸ್ಥಿತಿಗೆ ತಲುಪಿರುತ್ತಾರೆಂದರೆ, ಅವರ ಶರೀರದ ಯಾವ ಅವಯವವೂ ಕ್ರಿಯಾಶೀಲವಾಗಿರುವುದಿಲ್ಲ. ಹೃದಯದ ಬಡಿತ ಹೆಚ್ಚು ಕಡಿಮೆ ನಿಂತೇಹೋಗುತ್ತದೆ. ಇವುಗಳಲ್ಲಿ ಅಧಿಕಾಂಶ ಅಭ್ಯಾಸಗಳು ಬಹಳ ಅಪಾಯಕಾರಿ. ಉಚ್ಛಶ್ರೇಣಿಯ ವಿಧಾನಗಳು ಉನ್ನತ ಸಿದ್ದಿಗಳನ್ನು ಪಡೆಯಲು ಇವೆ. ಉಸಿರಾಟವನ್ನು ಸಂಪೂರ್ಣವಾಗಿ ನಿಲ್ಲಿಸಿ ಇಡೀ ಶರೀರವನ್ನು ಹಗುರವಾಗಿಸಲು ಪ್ರಯತ್ನಿಸಿ ಗಾಳಿಯಲ್ಲಿ ಮೇಲೆ ಹೋಗುವ ಅನೇಕ ಸಂಪ್ರದಾಯಗಳೂ ಇವೆ. ಗಾಳಿಯಲ್ಲಿಯೇ ತೇಲಾಡುವ ಯಾರನ್ನೂ ಕೂಡ ನಾನೆಂದೂ ನೋಡಿಲ್ಲ... ಆದರೆ ಗ್ರಂಥಗಳಲ್ಲಿ ಇದರ ಉಲ್ಲೇಖ ಸಿಗುತ್ತದೆ. ನನಗೆಲ್ಲವೂ ತಿಳಿದಿದೆ ಎಂದು ನಾನು ಸೋಗು ಹಾಕುವುದಿಲ್ಲ. ನಾನು ಸದಾಕಾಲವು ಆಶ್ಚರ್ಯಜನಕವಾದುವುಗಳನ್ನೇ ನೋಡುತ್ತಿರುತ್ತೇನೆ. ಒಮ್ಮೆ ನಾನು ಶೂನ್ಯದಿಂದ ಹಣ್ಣು – ಹೂವು ಮುಂತಾದುವುಗಳನ್ನು ಹೊರತರುವ ಸಾಧುವೊಬ್ಬನನ್ನು ನೋಡಿದೆ.

ಯೋಗಿಯು ಪೂರ್ಣತೆಯನ್ನು ಪಡೆದಾಗ ತನ್ನ ಶರೀರವನ್ನು ಎಷ್ಟು ಸೂಕ್ಷ್ಮವಾದ ಅಣುವಾಗಿಸಬಲ್ಲನೆಂದರೆ ಅದು ಈ ಗೋಡೆಯ ಮೂಲಕ ಹಾದು ಹೋಗುತ್ತದೆ. ಹೌದು, ಈ ಶರೀರವನ್ನೇ (ಅಷ್ಟು ಚಿಕ್ಕದಾಗಿಸಬಹುದು.) ಮತ್ತೆ ಅವನು ಈ ಶರೀರವನ್ನು ಎಷ್ಟು ಭಾರವಾಗಿಸುಬಹುದೆಂದರೆ \enginline{200} ಜನ ಸೇರಿದರೂ ಅವನನ್ನು ಎತ್ತಲಾಗುವುದಿಲ್ಲ. ಅವನು ಇಷ್ಟಪಟ್ಟಲ್ಲಿ ಗಾಳಿಯಲ್ಲಿ ಹಾರಿಕೊಂಡು ಹೋಗಲು ಸಮರ್ಥನಾಗುತ್ತಾನೆ. ಆದರೆ, ಯಾರೂ ಸಹ ದೇವರಷ್ಟು ಶಕ್ತಿಶಾಲಿಯಾಗಲಾರರು. ಹಾಗೇನಾದರೂ ಆಗಿದ್ದಿದ್ದರೆ ಒಬ್ಬ (ಯೋಗಿ) ಸೃಷ್ಟಿಸಿದ್ದನ್ನು ಮತ್ತೊಬ್ಬ (ಯೋಗಿ) ಧ್ವಂಸಗೊಳಿಸಬಹುದಿತ್ತು...

ಗ್ರಂಥಗಳಲ್ಲಿ ಇವುಗಳ ವರ್ಣನೆ ಇದೆ. ನಾನು ಅವುಗಳಲ್ಲಿ ಎಲ್ಲವನ್ನೂ ನಂಬಲಾರೆ; ಹಾಗೆಂದು ಅವುಗಳಲ್ಲಿ ಅಪನಂಬಿಕೆಯೂ ಇಲ್ಲ. ನಾನು ಏನನ್ನು ನೋಡಿದ್ದೇನೋ ಅದಷ್ಟನ್ನು ಮಾತ್ರ ತೆಗೆದುಕೊಳ್ಳುತ್ತೇನೆ.

ಈ ಪ್ರಪಂಚದ ವಿದ್ಯಮಾನಗಳ ಉನ್ನತಿ ಎಂದಾದರೂ ಸಾಧ್ಯವಾದರೆ ಅದು\break ಸ್ಪರ್ಧೆಯಿಂದಲ್ಲ; ಆದರೆ ಮನಸ್ಸನ್ನು ನಿಯಮಿತಗೊಳಿಸುವುದರಿಂದ. ಪಾಶ್ಚಾತ್ಯರು\break ಹೇಳುತ್ತಾರೆ: ಅದು ನಮ್ಮ ಸ್ವಭಾವ, ನಾವೇನೂ ಮಾಡಲಾಗುವುದಿಲ್ಲ ಎಂದು. ನಿಮ್ಮ ಸಮಾಜದ ಸಮಸ್ಯೆಗಳನ್ನು ಅಧ್ಯಯನ ಮಾಡಿದ ನಂತರ, ನನ್ನ ನಿರ್ಣಯವೇನೆಂದರೆ ನೀವು (ಸ್ಪರ್ಧೆಯಿಂದ) ಅವುಗಳನ್ನು ಪರಿಹರಿಸಲು ಸಾಧ್ಯವಿಲ್ಲ. ಕೆಲವು ವಿಷಯಗಳಲ್ಲಂತೂ ನೀವು ನಮಗಿಂತಲೂ ಹೊಲಸು... ಈ ಪ್ರಕಾರವಾದ ಸ್ಪರ್ಧಾ ಮನೋಭಾವ ಪ್ರಪಂಚವನ್ನು ಎಲ್ಲಿಗೂ ಮುಟ್ಟಿಸಲಾರವು.

ಬಲಶಾಲಿಯಾದವನು ಎಲ್ಲವನ್ನೂ ದೋಚುತ್ತಾನೆ. ದುರ್ಬಲರಾದವರು ಮೂಲೆಗುಂಪಾಗುತ್ತಾರೆ. ಸಮರ್ಥನಾದ ಮನುಷ್ಯ ಎಲ್ಲವನ್ನೂ ಬಾಚಿಕೊಳ್ಳುತ್ತಾನೆ. ವಂಚಿತರು, ಬಡವರು ಅವನನ್ನು ದ್ವೇಷಿಸುತ್ತಾರೆ. ಏಕೆ? ಅವರು ತಮ್ಮ ಸರದಿಗಾಗಿ ಕಾಯುತ್ತಿದ್ದಾರೆ ಅಷ್ಟೆ. ಅವರು (ಪಾಶ್ಚಾತ್ಯರು) ಆವಿಷ್ಕರಿಸುವ ಎಲ್ಲಾ ವ್ಯವಸ್ಥೆಗಳೂ ಇದನ್ನೇ ಬೋಧಿಸುತ್ತವೆ. ಈ ಸಮಸ್ಯೆಯನ್ನು ಮನುಷ್ಯನ ಮನಸ್ಸಿನಲ್ಲಿ ಮಾತ್ರ ಬಗೆಹರಿಸಬಹುದು... ವ್ಯಕ್ತಿ ತಾನು ಒಲ್ಲೆ ಎಂಬ ಕೆಲಸವನ್ನು ಯಾವ ಕಾನೂನೂ ಅವನಿಂದ ಮಾಡಿಸಲಾಗುವುದಿಲ್ಲ. ಆಂತರಿಕ ಪ್ರೇರಣೆಯಿಂದ ತಾನು ಒಳ್ಳೆಯವನಾಗಬೇಕು ಎಂದು ಬಲವಾಗಿ ಇಚ್ಚಿಸಿದಾಗ ಮಾತ್ರ ಮನುಷ್ಯ ಒಳ್ಳೆಯವನಾಗುತ್ತಾನೆ. ವ್ಯಕ್ತಿ ತಾನು ಮನಸ್ಸು ಮಾಡದಿದ್ದಲ್ಲಿ ಎಲ್ಲ ಕಾನೂನುಗಳೂ, ನ್ಯಾಯ ಪಂಡಿತರೂ ಅವನನ್ನು ಒಳ್ಳೆಯವನನ್ನಾಗಿ ಮಾಡಲಾಗುವುದಿಲ್ಲ. ಸರ್ವಶಕ್ತಿ ಸಂಪನ್ನನಾದ ಮನುಷ್ಯ ಹೇಳುತ್ತಾನೆ: “ನಾನು ಯಾರನ್ನೂ ಲೆಕ್ಕಿಸುವುದಿಲ್ಲ” ಎಂದು. ಈ ಸಮಸ್ಯೆಗೆ ಏಕೈಕ ಪರಿಹಾರವೆಂದರೆ ನಾವೆಲ್ಲರೂ ಒಳ್ಳೆಯವರಾಗಬೇಕೆಂದು ಬಯಸಬೇಕು. ಅದನ್ನು ಹೇಗೆ ಮಾಡಲು ಸಾಧ್ಯ?

ಸಮಸ್ತ ಜ್ಞಾನರಾಶಿಯೂ ಮನಸ್ಸಿನಲ್ಲಿಯೇ ಇದೆ. ಜ್ಞಾನವು ಕಲ್ಲಿನಲ್ಲಿರುವುದನ್ನು ಅಥವಾ ಖಗೋಳ ಶಾಸ್ತ್ರವು ನಕ್ಷತ್ರಗಳಲ್ಲಿರುವುದನ್ನು ನೋಡಿದವರು ಯಾರು?\break ಅದೆಲ್ಲವೂ ಮಾನವನಲ್ಲಿಯೇ ಇದೆ.

ನಾವು ಅನಂತಶಕ್ತರೆಂಬುದನ್ನು ಸಾಕ್ಷಾತ್ಕಾರ ಮಾಡಿಕೊಳ್ಳೋಣ. ಮನಸ್ಸಿನ ಶಕ್ತಿಗೆ ಮಿತಿಯಿಟ್ಟವರು ಯಾರು? ನಾವೆಲ್ಲರೂ ಮನಸ್ಸೆಂಬುದನ್ನು ಸಾಕ್ಷಾತ್ಕಾರ ಮಾಡಿಕೊಳ್ಳೋಣ. ಸಮುದ್ರದ ಪ್ರತಿಯೊಂದು ಬಿಂದುವೂ ತನ್ನಲ್ಲಿ ಸಮಸ್ತ ಸಮುದ್ರವನ್ನೇ ಒಳಗೊಂಡಿದೆ. ಅದೇ ರೀತಿ ಮನುಷ್ಯನ ಮನಸ್ಸೂ ಸಹ, ಭಾರತೀಯ ಮನೀಷಿಗಳು ಮನಸ್ಸಿನ ಈ ಶಕ್ತಿಗಳ ಮತ್ತು ಸಾಧ್ಯತೆಗಳ ಮೇಲೆ ಚಿಂತನ ಮನನಗಳಲ್ಲಿ ತೊಡಗಿ ಅವುಗಳೆಲ್ಲವನ್ನೂ ಹೊರತರಲು ಬಯಸುತ್ತಾರೆ. ಕೆಚ್ಚಿನಿಂದ ಸಾಧನೆಯಲ್ಲಿ ತೊಡಗುತ್ತಾ ತಮಗೇನಾಗುತ್ತದೆ ಎಂಬುದನ್ನು ಅವರು ಲೆಕ್ಕಿಸುವುದೇ ಇಲ್ಲ. ಪೂರ್ಣತೆಯನ್ನು ತಲುಪಲು ಬಹಳ ದೀರ್ಘಕಾಲ ಬೇಕಾಗುತ್ತದೆ. ಐವತ್ತು ಸಾವಿರ ವರ್ಷಗಳೇ ಆಗಲಿ, ಮುಳುಗಿ ಹೋದದ್ದೇನು?

ಸಮಾಜದ ತಳಪಾಯವೇ, ಅದರ ರಚನೆಯೇ ದೋಷಯುಕ್ತವಾಗಿದೆ. ಮನುಷ್ಯ ಸ್ವಯಂಪ್ರೇರಿತನಾಗಿ ತನ್ನ ಮನಸ್ಸಿನ ಪರಿವರ್ತನೆಗಾಗಿ ಪ್ರಯತ್ನಿಸಿ – ಅವನ ಅಂತರಂಗ ಬದಲಾದಾಗ ಮಾತ್ರ – ಪೂರ್ಣತೆ ಸಾಧ್ಯ. ಇದರಲ್ಲಿ ದೊಡ್ಡ ಕಷ್ಟವೆಂದರೆ ಅವನು ತನ್ನ ಮನಸ್ಸನ್ನು ಬಲಾತ್ಕಾರಗೊಳಿಸಲೂ ಸಹ ಸಾಧ್ಯವಿಲ್ಲ.

ರಾಜಯೋಗದಲ್ಲಿ ಘೋಷಿಸಲಾಗಿರುವ ಎಲ್ಲವನ್ನೂ ನೀವು ನಂಬದಿರಬಹುದು. ಆದರೆ ಅವಶ್ಯವಾಗಿ ಪ್ರತಿಯೊಬ್ಬ ವ್ಯಕ್ತಿಯೂ ದೈವತ್ವವನ್ನು ಸಾಧಿಸಬಲ್ಲ. ಪ್ರತಿಯೊಬ್ಬ ವ್ಯಕ್ತಿಯೂ ತನ್ನ ಆಲೋಚನೆಗಳ ಮೇಲೆ ಸಂಪೂರ್ಣ ಹಿಡಿತವನ್ನು ಸಾಧಿಸಿದಾಗ ಮಾತ್ರ ಈ ದೈವತ್ವದ ವಿಕಾಸ ಸಾಧ್ಯವಾಗುತ್ತದೆ. (ಆಲೋಚನೆಗಳು ಇಂದ್ರಿಯಗಳು) ಎಲ್ಲವೂ ನನ್ನ ಗುಲಾಮರಾಗಿರಬೇಕು, ನನ್ನ ಒಡೆಯರಲ್ಲ. ಹಾಗಾದಾಗ ಮಾತ್ರವೇ ಎಲ್ಲ ಅನಿಷ್ಟಗಳೂ ನಾಶವಾಗಲು ಸಾಧ್ಯ.

ಮನಸ್ಸಿಗೆ ವಿಷಯರಾಶಿಗಳನ್ನು ತುರುಕುವುದು ಶಿಕ್ಷಣವಲ್ಲ. ಜ್ಞಾನದ ಉಪಕರಣವನ್ನು ಪರಿಪೂರ್ಣಗೊಳಿಸುವುದು ಹಾಗೂ ತನ್ನ ಮನಸ್ಸಿನ ಮೇಲೆ ಸಂಪೂರ್ಣ ಪ್ರಭುತ್ವವನ್ನು ಪಡೆಯುವುದು (ಶಿಕ್ಷಣದ ಆದರ್ಶವಾಗಿದೆ). ನಾನೇನಾದರೂ ಯಾವುದಾದರೂ (ಅಂಶದ) ಮೇಲೆ ಮನಸ್ಸನ್ನು ಏಕಾಗ್ರಗೊಳಿಸಬಯಸಿದರೆ, ಅದು ಅಲ್ಲಿಗೆ ಹೋಗುತ್ತದೆ, ಹಾಗೂ ನಾನು ಅದನ್ನು ಕರೆದಾಕ್ಷಣ ಅದು ಮತ್ತೆ ಮುಕ್ತವಾಗುತ್ತದೆ.

ದೊಡ್ಡ ತೊಂದರೆಯೆಂದರೆ ಇದೇ: ಬಹಳ ಹೋರಾಟಮಾಡಿ ಕಷ್ಟಪಟ್ಟು ಕೆಲವು ವಿಷಯಗಳಲ್ಲಿ ಮನಸ್ಸನ್ನು ಏಕಾಗ್ರಗೊಳಿಸುವ ಶಕ್ತಿಯನ್ನು ಪಡೆಯುತ್ತೇವೆ. ಆ ವಿಷಯಗಳಲ್ಲಿ ಮನಸ್ಸು ತೀವ್ರವಾಗಿ ಪ್ರವೃತ್ತವಾಗುವಂತೆ ಮಾಡಬಹುದು. ಆದರೆ ಆ ಮನಸ್ಸನ್ನು ಹಿಂತೆಗೆದುಕೊಳ್ಳಲು ಬೇಕಾದ ನಿವೃತ್ತಿ ಶಕ್ತಿ ಅಲ್ಲಿ ಇಲ್ಲ. ಆ ವಸ್ತುವಿನಿಂದ ಮನಸ್ಸನ್ನು ಹಿಂತೆಗೆದುಕೊಳ್ಳಲು ನನ್ನ ಬಾಳಿನ ಅರ್ಧಭಾಗವನ್ನೇ ಸವೆಸುತ್ತೇನೆ. ಆದರೂ ಅದು ಸಾಧ್ಯವಾಗುವುದಿಲ್ಲ. ಮನಸ್ಸನ್ನು ಏಕಾಗ್ರಗೊಳಿಸುವ ಪ್ರವೃತ್ತಿ ಶಕ್ತಿಯ ಜೊತೆ ಜೊತೆಗೇ ಅದನ್ನು ಹಿಂತೆಗೆದುಕೊಳ್ಳುವ ನಿವೃತ್ತಿ ಶಕ್ತಿಯನ್ನೂ ಬೆಳೆಸಿಕೊಳ್ಳಬೇಕು. ವ್ಯಕ್ತಿಯು ಪ್ರವೃತ್ತಿ ಮತ್ತು ನಿವೃತ್ತಿ ಶಕ್ತಿಗಳನ್ನು ಸಮಸಮವಾಗಿ ಹೊಂದಿರುವವನಾಗಿದ್ದರೆ – ಆಗ ಆ ವ್ಯಕ್ತಿ ಮನುಷ್ಯತ್ವವನ್ನು ಸಾಧಿಸಿದವನಾಗಿರುತ್ತಾನೆ. ಇಡೀ ವಿಶ್ವವೇ ತಲ್ಲಣಿಸಿದರೂ ಯಾವುದೂ ಅವನನ್ನು ದುಃಖಿಯಾಗುವಂತೆ ಮಾಡುವುದಿಲ್ಲ. ಈ ರೀತಿಯ ಸ್ಥಿರತೆಯನ್ನು ನಿಮಗೆ ಯಾವ ಪುಸ್ತಕಗಳು ಬೋಧಿಸಬಲ್ಲವು? ನೀವೆಷ್ಟು ಪುಸ್ತಕಗಳನ್ನು ಬೇಕಾದರೂ ಓದಬಹುದು, ಕ್ಷಣಕ್ಕೆ \enginline{50} ಸಾವಿರ ಪದಗಳನ್ನು ನೀವು ಮಗುವಿನ ಮನಸ್ಸಿನಲ್ಲಿ ತುರುಕಬಹುದು. ಅದಕ್ಕೆ ಎಲ್ಲ ಸಿದ್ಧಾಂತಗಳನ್ನೂ, ದರ್ಶನಗಳನ್ನೂ ಬೋಧಿಸಬಹುದು... (ಆದರೆ) ಅವನಿಗೆ ಯಾಥಾರ್ಥ ಸತ್ಯಗಳನ್ನು ಬೋಧಿಸುವ ಶಾಸ್ತ್ರ ಕೇವಲ ಒಂದೇ ಒಂದು: ಅದೇ ಮನಶ್ಶಾಸ್ತ್ರ... ಮತ್ತು ಉಸಿರಿನ ನಿಯಂತ್ರಣದೊಂದಿಗೇ ಈ ಕೆಲಸ ಆರಂಭವಾಗುವುದು.

ಮನಸ್ಸಿನ ಗಹ್ವರಗಳಲ್ಲಿ ನಿಧಾನವಾಗಿ ಮತ್ತು ಕ್ರಮವಾಗಿ ಪ್ರವೇಶಿಸಿ, ಕ್ರಮೇಣ ನಿಮ್ಮ ಮನಸ್ಸನ್ನು ಹತೋಟಿಗೆ ತೆಗೆದುಕೊಳ್ಳಿ. ಇದು ಬಹಳ ದೀರ್ಘವಾದ, ಕಷ್ಟಕರವಾದ ಹೋರಾಟ. ಇದನ್ನು ಏನೋ ಒಂದು ಕುತೂಹಲಕ್ಕಾಗಿಯಷ್ಟೇ ತೆಗೆದುಕೊಳ್ಳಬಾರದು. ಯಾರಾದರೂ ಏನನ್ನಾದರೂ ಅನುಷ್ಠಾನ ಮಾಡಬೇಕೆಂದು ಬಯಸಿದಾಗ ಅವನಿಗೊಂದು ಯೋಜನೆಯಿರುತ್ತದೆ. (ರಾಜಯೋಗವು) ಯಾವುದೇ ನಂಬಿಕೆಯನ್ನಾಗಲೀ,\break ಮತವನ್ನಾಗಲೀ, ದೇವರನ್ನಾಗಲೀ ನಿಮ್ಮ ಮುಂದಿಡುವುದಿಲ್ಲ. ನೀವು ಎರಡು ಸಾವಿರ ದೇವತೆಗಳನ್ನು ಬೇಕಾದರೆ ನಂಬಿ... ಈ ಕ್ಷೇತ್ರದಲ್ಲೇ ನಿಮ್ಮ ಪ್ರಯತ್ನವನ್ನು ಮಾಡುತ್ತಾ ಹೋಗಿ, ಏಕಾಗಬಾರದು? ಆದರೆ ರಾಜಯೋಗದಲ್ಲಿ ನಿರ್ಗುಣ ತತ್ತ್ವವೇ ಮುಖ್ಯ.

ಅತ್ಯಂತ ಕಠಿಣವಾದದ್ದು ಯಾವುದು? ನಾವು ಮಾತನಾಡುತ್ತೇವೆ, ಸಿದ್ದಾಂತಗಳನ್ನು ಪ್ರತಿಪಾದಿಸುತ್ತೇವೆ. ಮಾನವ ಸಮಾಜದ ಅಧಿಕಾಂಶ ವ್ಯಕ್ತಿಗಳು ಮೂರ್ತವಾದ ಪದಾರ್ಥಗಳೊಂದಿಗೆ ವ್ಯವಹರಿಸಬೇಕು. ಕಾರಣ ಮಂದಬುದ್ದಿಯುಳ್ಳ ಜನರಿಗೆ ಉಚ್ಚತಮವಾದ ದರ್ಶನವನ್ನು ಗ್ರಹಿಸಲಾಗುವುದಿಲ್ಲ. ಅಷ್ಟರಲ್ಲೇ ಅದು ಮುಕ್ತಾಯವಾಗುತ್ತದೆ. ಪ್ರಪಂಚದ ಸಮಸ್ತ ವಿಜ್ಞಾನಗಳಲ್ಲೂ ನೀವು ಸ್ನಾತಕರಾಗಿರಬಹುದು... ಆದರೆ ನೀವು ಸಾಕ್ಷಾತ್ಕಾರವನ್ನು ಪಡೆಯದಿದ್ದಲ್ಲಿ, ನೀವು ಮತ್ತೆ ಶಿಶುವಾಗಬೇಕು ಮತ್ತು ಕಲಿಯಬೇಕು.

\newpage

...ನೀವು ಅವರ ಮುಂದೆ ಅನಂತವಾದದ್ದನ್ನು, ಅಮೂರ್ತವಾದದ್ದನ್ನು ಇಟ್ಟರೆ ಅವರು ತಬ್ಬಿಬ್ಬಾಗುತ್ತಾರೆ. ಅವರಿಗೆ ಒಂದು ಬಾರಿಗೆ ಅನುಷ್ಠಾನ ಮಾಡಲು ಸ್ವಲ್ಪ ಮಾತ್ರವನ್ನೇ ಕೊಡಿ. “ನೀವು ಇಷ್ಟು ಬಾರಿ ಉಸಿರನ್ನು ತೆಗೆದುಕೊಳ್ಳಿ. ನೀವು ಇದನ್ನು ಮಾಡಿ'' ಎಂದಷ್ಟೇ ಅವರಿಗೆ ಹೇಳಿ. ಅವರು ಅದನ್ನು ಮಾಡುತ್ತಾ ಹೋದಂತೆ ಅವರಿಗೆ ಅರ್ಥವಾಗುತ್ತದೆ; ಅವರು ಅದರಲ್ಲಿ ಆನಂದವನ್ನು ಪಡೆಯುತ್ತಾರೆ. ಇವು ಧರ್ಮದ ಶಿಶುವಿಹಾರಗಳು. ಆದ್ದರಿಂದಲೇ ಈ ಉಸಿರಾಟದ ಅಭ್ಯಾಸಗಳು, ಪ್ರಾಣಾಯಾಮಗಳು ಅಷ್ಟೊಂದು ಪ್ರಯೋಜನಕಾರಿ. ನೀವು ಬರೀ ಮೇಲೆ ಮೇಲೆ ಕುತೂಹಲಿಗಳಷ್ಟೇ ಆಗಬೇಡಿರೆಂದು ನಾನು ಬಿನ್ನವಿಸಿಕೊಳ್ಳುತ್ತೇನೆ. ಕೆಲವು ದಿನಗಳು ಅನುಷ್ಠಾನ ಮಾಡಿ, ಮತ್ತು ನೀವು ಯಾವುದೇ ಪ್ರಯೋಜನವನ್ನು ಕಾಣದಿದ್ದಲ್ಲಿ ಆಗ ಬಂದು ನನ್ನನ್ನು ಶಪಿಸಿ...

ಇಡೀ ವಿಶ್ವವೇ ಒಂದು ಶಕ್ತಿಪುಂಜವಾಗಿದೆ. ಮತ್ತು ಆ ಶಕ್ತಿ (ಸರ್ವತ್ರ) ಪ್ರತಿಯೊಂದು ಬಿಂದುವಿನಲ್ಲೂ ವಿದ್ಯಮಾನವಾಗಿದೆ. ಅಲ್ಲಿರುವುದನ್ನು ಹೇಗೆ ಪಡೆಯಬೇಕೆಂಬುದು ನಮಗೆ ಗೊತ್ತಾದರೆ, ನಮಗೆಲ್ಲ ಅದರ ಒಂದು ಕಣದಷ್ಟು ಯಥೇಚ್ಛ.

ಕರ್ಮದಿಂದ ಬದ್ದರಾಗಿ “ಇದನ್ನು ಮಾಡಲೇಬೇಕು” ಎಂಬ ಗುಲಾಮ ಬುದ್ದಿ ನಮ್ಮನ್ನು ಸಾಯಿಸುತ್ತಿರುವ ವಿಷ... ಗುಲಾಮರಿಗೆ ಯಾವುದು ಖುಷಿ ತರುತ್ತದೋ ಅದೇ ಕರ್ತವ್ಯವೆಂದು ಕರೆಯಲ್ಪಡುವ ಕರ್ಮಬದ್ದತೆ. (ಆದರೆ) ನಾನು ಮುಕ್ತನಾಗಿದ್ದೇನೆ. ನಾನು ಚಿರಮುಕ್ತ, ನಾನೇನು ಕರ್ಮ ಮಾಡುತ್ತೇನೋ ಅದೆಲ್ಲ ನನ್ನ ಕ್ರೀಡೆ. (ನಾನೇನೂ ಗುಲಾಮನಲ್ಲ). ಇವೆಲ್ಲದರಿಂದ ನನಗೆ ಸ್ವಲ್ಪ ವಿನೋದ ಸಿಗುತ್ತಿದೆ ಅಷ್ಟೆ...

ಪ್ರೇತಾತ್ಮರೆಲ್ಲ ದುರ್ಬಲರು. ನಮ್ಮಿಂದ ಜೀವಸತ್ತ್ವವನ್ನು ಪಡೆಯಲು ಪ್ರಯಾಸಪಡುತ್ತಿವೆ.

ಆಧ್ಯಾತ್ಮಿಕ ಬಲವನ್ನು ಒಂದು ಮನಸ್ಸಿನಿಂದ ಮತ್ತೊಂದು ಮನಸ್ಸಿಗೆ ಧಾರೆ ಎರೆಯಬಹುದು. ಇದನ್ನು ಕೊಡುವಾತ ಗುರು; ಯಾರು ಇದನ್ನು ಪಡೆಯುತ್ತಾನೋ ಅವನೇ ಶಿಷ್ಯ. ಆಧ್ಯಾತ್ಮಿಕ ಸತ್ಯಗಳನ್ನು ಈ ಪ್ರಪಂಚದಲ್ಲಿ ಹರಡುವ ಮಾರ್ಗ ಇದೊಂದೇ.

(ಮೃತ್ಯುಕಾಲದಲ್ಲಿ) ಎಲ್ಲ ಇಂದ್ರಿಯಗಳೂ ಮನಸ್ಸಿನಲ್ಲಿ ಲೀನವಾಗುತ್ತದೆ ಮತ್ತು ಮನಸ್ಸು ಪ್ರಾಣದಲ್ಲಿ ಲೀನವಾಗುತ್ತದೆ. ಆತ್ಮ, ದೇಹದಿಂದ ಹೊರಬಿದ್ದು ಮನಸ್ಸಿನ ಅಲ್ಪಾಂಶವನ್ನು ತನ್ನೊಡನೆ ಒಯ್ಯುತ್ತದೆ. ಅದು ಪ್ರಾಣದ ಅಲ್ಪಾಂಶವನ್ನೂ ಹಾಗೂ ಜಡತತ್ತ್ವದ ಸೂಕ್ಷ್ಮತಮವಾದ ಭಾಗವನ್ನೂ ಸಹ ಲಿಂಗ ಶರೀರದ ಬೀಜರೂಪವಾಗಿ ತನ್ನ ಜೊತೆ ಕೊಂಡೊಯ್ಯುತ್ತದೆ. ಪ್ರಾಣವು ಯಾವುದೇ ರೀತಿಯ (ವಾಹಕವಿಲ್ಲದೆ) ನಿರಾಧಾರವಾಗಿ ಎಂದಿಗೂ ಇರಲಾರದು... ಅದು ಆಲೋಚನೆಗಳಲ್ಲಿ ಹುದುಗಿದ್ದು ಮತ್ತೆ ಹೊರಗೆ\break ವ್ಯಕ್ತವಾಗುತ್ತದೆ. ಈ ಹೊಸ ಶರೀರವನ್ನು ಹೊಸ ಮೆದುಳನ್ನು ನೀವು ನಿರ್ಮಾಣ\break ಮಾಡುತ್ತಿರುತ್ತೀರಿ. ಅದರ ಮೂಲಕ ಪ್ರಾಣವು ಅಭಿವ್ಯಕ್ತವಾಗುತ್ತದೆ.

ಪ್ರೇತಗಳು ಶರೀರ ನಿರ್ಮಾಣ ಮಾಡಲು ಅಸಮರ್ಥರು. ಅವುಗಳಲ್ಲಿ ಯಾರು ತೀರಾ ದುರ್ಬಲರೋ ಅವರಿಗೆ ತಾವು ಸತ್ತು ಹೋಗಿದ್ದೇವೆ ಎಂಬುದು ಕೂಡ ನೆನಪಿರುವುದಿಲ್ಲ. ಅವರು ಅನ್ಯ ಶರೀರಗಳಲ್ಲಿ ಪ್ರವೇಶಮಾಡಿ ಈ ಪ್ರೇತ ಜೀವನದಿಂದಲೇ ಹೆಚ್ಚು ಹೆಚ್ಚು ಸುಖವನ್ನು ಗಳಿಸಲು ಪ್ರಯತ್ನಿಸುತ್ತಾರೆ. ಯಾವುದೇ ವ್ಯಕ್ತಿ ತನ್ನ ಶರೀರವನ್ನು ಆ ಪ್ರೇತಗಳಿಗೆ ತೆರೆದುಕೊಳ್ಳುತ್ತಾನೋ ಅವನು ಭಯಾನಕವಾದ ಅಪಾಯದಲ್ಲಿರುತ್ತಾನೆ. ಕಾರಣ ಆ ಪ್ರೇತಾತ್ಮಗಳು ಅವನ ಜೀವನಸತ್ತ್ವವನ್ನು ಶೋಷಣೆ ಮಾಡುತ್ತವೆ.

ಈ ಸಂಸಾರದಲ್ಲಿ ಭಗವಂತನೊಬ್ಬನನ್ನು ಬಿಟ್ಟರೆ ಬೇರಾವುದೂ ಶಾಶ್ವತವಲ್ಲ. ವಿಮೋಚನೆ ಎಂದರೆ ಸತ್ಯವನ್ನು ಅರಿಯುವುದು. (ಸತ್ಯವನ್ನು ಅರಿತು) ನಾವು ಏನೋ ಆಗಿಬಿಡುವುದಿಲ್ಲ. ನಮ್ಮ ನಿಜಸ್ವರೂಪದಲ್ಲಿ ನಾವು ಏನಾಗಿರುವೆವೋ ಹಾಗೆ ಇರುತ್ತೇವೆ. ವಿಮೋಚನೆ ಬರುವುದು ಶ್ರದ್ಧೆಯಿಂದ; ಕರ್ಮಕ್ಕಂಟಿಕೊಂಡಿರುವುದರಿಂದಲ್ಲ. ಇದು ಜ್ಞಾನದ ಪ್ರಶ್ನೆ! ನೀವು ಏನಾಗಿದ್ದೀರೋ ನಿಮಗದರ ಬೋಧೆಯಾಗಬೇಕು. ಅಲ್ಲಿಗೆ ಕೆಲಸ ಮುಕ್ತಾಯವಾಗುವುದು. ಆಗ ಸ್ವಪ್ನವು ಕಳೆಯುವುದು. ಈಗ ನೀವು (ಮತ್ತು ಇತರರು) ಸ್ವಪ್ನದಲ್ಲಿದ್ದೀರಿ. ಯಾವಾಗ ಅವರು ಕಾಲವಾಗುತ್ತಾರೋ ಅವರು ತಮ್ಮ (ಸ್ವಪ್ನದ) ಸ್ವರ್ಗಕ್ಕೆ ಹೋಗುತ್ತಾರೆ. ಆ ಸ್ವಪ್ನದಲ್ಲಿಯೇ ವಾಸಿಸುತ್ತಿರುತ್ತಾರೆ. ಆ ಸ್ವಪ್ನ ಮುಕ್ತಾಯವಾದಾಗ ಮತ್ತೊಂದು ಸುಂದರವಾದ ಶರೀರವನ್ನು ಪಡೆದು ಇಲ್ಲಿ ಹುಟ್ಟುತ್ತಾರೆ. ಅವರೇ ಒಳ್ಳೆಯ ವ್ಯಕ್ತಿಗಳು.

ಜ್ಞಾನಿ ಹೇಳುತ್ತಾನೆ: “ಈ ಎಲ್ಲ ಆಸೆಗಳೂ ನನ್ನಿಂದ ಅಳಿಸಿಹೋಗಿವೆ. ಈ ಬಾರಿ ಈ ಗದ್ದಲಗಳ ಗೊಡವೆಯೇ ಬೇಡ'' ಎಂದು. ಅವನು ಜ್ಞಾನ ಪ್ರಾಪ್ತಿಗಾಗಿ ಪ್ರಯತ್ನಿಸಿ ಕಷ್ಟಪಟ್ಟು ಹೋರಾಡುತ್ತಾನೆ. ಆಗ ಮನಗಾಣುತ್ತಾನೆ – ಇದೆಂತಹ ಕನಸು, ಇದೆಂತಹ ಭ್ರಾಂತಿ – ಈ (ಕನಸು ಕಾಣುವುದು) ಸ್ವರ್ಗ, ಸಂಸಾರ, ಇನ್ನೂ ಅನಿಷ್ಟವಾದುವುಗಳನ್ನು ಕಟ್ಟುವುದು – ಇವುಗಳೆಲ್ಲವನ್ನೂ ನೋಡಿ ನಗುತ್ತಾನೆ.

