
\chapter{ಅನುಷ್ಠಾನ ಧರ್ಮ: ಪ್ರಾಣಾಯಾಮ ಮತ್ತು ಧ್ಯಾನ\protect\footnote{\enginline{* C.W. Vol. I, P. 513}}\\(1900ರ 5ನೇ ಏಪ್ರಿಲ್‌ರಂದು ಸ್ಯಾನ್‌ಫ್ರಾನ್ಸಿಸ್ಕೋದಲ್ಲಿ ಕೊಟ್ಟ ಉಪನ್ಯಾಸ)}

ಅನುಷ್ಠಾನ ಧರ್ಮದ ಬಗೆಗಿನ ಪ್ರತಿಯೊಬ್ಬ ವ್ಯಕ್ತಿಯ ವಿಚಾರವೂ, ಅನುಷ್ಠಾನದ ಸಂಬಂಧವಾಗಿ ಆ ವ್ಯಕ್ತಿ ಹೊಂದಿರುವ ಸಿದ್ದಾಂತಕ್ಕೂ ಹಾಗೂ ಆತ ಕಾರ್ಯಾರಂಭ ಮಾಡುವಾಗ ತಳೆದಿರುವ ದೃಷ್ಟಿಕೋನಕ್ಕೂ ಅನುರೂಪವಾಗಿರುತ್ತದೆ. (ಅನುಷ್ಠಾನ ಯೋಗ್ಯವಾದ) ಕರ್ಮವಿದೆ, ಭಕ್ತಿ - ಉಪಾಸನೆಗಳಿವೆ, ಜ್ಞಾನವಿದೆ.

ದಾರ್ಶನಿಕನ ಅಭಿಪ್ರಾಯದಲ್ಲಿ ಜ್ಞಾನ ಮತ್ತು ಅಜ್ಞಾನವೇ ಮುಕ್ತಿ ಮತ್ತು ಬಂಧನಗಳಿಗಿರುವ ವ್ಯತ್ಯಾಸಕ್ಕೆ ಕಾರಣವಾಗಿದೆ. ಆತನಿಗೆ ಜ್ಞಾನವೇ ಗುರಿ ಮತ್ತು ಅನುಷ್ಠಾನ ಎಂದರೆ ಆ ಜ್ಞಾನವನ್ನು ಪಡೆಯುವುದೇ ಆಗಿದೆ... ಭಕ್ತ ಅಥವಾ ಉಪಾಸಕನ ಅನುಷ್ಠಾನಧರ್ಮವಾದರೂ, ಪ್ರೇಮ ಮತ್ತು ಭಕ್ತಿಯ ಅಮಿತವಾದ ಶಕ್ತಿಯಲ್ಲಡಗಿದೆ... ಹಾಗೆಯೇ ಕರ್ಮಮಾರ್ಗಿಯು ಸತ್ಕರ್ಮಗಳನ್ನು ಹಾಗೂ ಒಳಿತನ್ನು ಮಾಡುವುದನ್ನೇ ತನ್ನ ಅನುಷ್ಠಾನಧರ್ಮವನ್ನಾಗಿ ಮಾಡಿಕೊಂಡಿರುತ್ತಾನೆ. ಹೀಗೆ ನಾವು ಅನ್ಯತ್ರ ಎಲ್ಲದರಲ್ಲಿಯೂ ನೋಡುವಂತೆ, ಮತ್ತೊಬ್ಬರ ಆದರ್ಶ ಮತ್ತು ಮಾನದಂಡವನ್ನು ಅಲಕ್ಷಿಸಿ ನಮ್ಮ ಮಾನದಂಡದಿಂದಷ್ಟೇ ಪ್ರಪಂಚವನ್ನು ಅಳೆದು ಅದನ್ನು ಕಟ್ಟಿಹಾಕಲು ಯಾವಾಗಲೂ ಪ್ರಯತ್ನಿಸುತ್ತಿರುತ್ತೇವೆ. ಪ್ರೇಮಭರಿತನಾದ, ಹೃದಯವಂತನಾದ ವ್ಯಕ್ತಿಗೆ ತನ್ನ ಸಹಜೀವಿಗಳಿಗೆ ಒಳಿತನ್ನು ಮಾಡುವುದೇ ಅವನ ಅನುಷ್ಠಾನ ಧರ್ಮ. ಮನುಷ್ಯರು ಆಸ್ಪತ್ರೆಗಳನ್ನು ಕಟ್ಟಲು ಸಹಾಯ ಮಾಡದಿದ್ದರೆ ಅವರಿಗೆ ಯಾವುದೇ ಧರ್ಮವಾಗಲೀ ಇಲ್ಲ ಎಂದು ಭಾವಿಸುತ್ತಾನೆ. ಆದರೆ ಪ್ರತಿಯೊಬ್ಬರೂ ಅದನ್ನೇ ಮಾಡಬೇಕೆಂಬುದರಲ್ಲಿ ಯಾವ ಯುಕ್ತಿಯೂ ಇಲ್ಲ. ಹಾಗೆಯೇ ದಾರ್ಶನಿಕರೂ ಅವರಂತೆಯೇ... ಜ್ಞಾನವನ್ನು ಸಾಧಿಸಿರದ ಪ್ರತಿಯೊಬ್ಬನನ್ನೂ ಅವಹೇಳನ ಮಾಡಬಹುದು. ಜನರು 20 ಸಾವಿರ ಆಸ್ಪತ್ರೆಗಳನ್ನು ಕಟ್ಟಬಹುದು. ದಾರ್ಶನಿಕನು ಅವರನ್ನು ದೇವತೆಗಳು ಸಾಕಿದ ಭಾರವಾಹಕ ಪಶುಗಳಷ್ಟೇ ಎಂದು ತೀರ್ಮಾನಿಸುತ್ತಾನೆ. ಉಪಾಸಕನೂ ತನ್ನದೇ ಆದ ವಿಚಾರವನ್ನು ಹಾಗೂ ಮಾನದಂಡವನ್ನು (ಈ ರೀತಿ) ಹೊಂದಿರುತ್ತಾನೆ: ಯಾರು ಭಗವಂತನನ್ನು ಪ್ರೀತಿಸಲಾರರೋ ಅವರು ಯಾವ ಕೆಲಸ ಮಾಡಿದರೇನಂತೆ, ಒಟ್ಟಿನಲ್ಲಿ ಯಾವ ಪ್ರಯೋಜನಕ್ಕೂ ಬಾರದವರೆಂದು. ಯೋಗಿ ಚಿತ್ತವೃತ್ತಿ ನಿರೋಧದಲ್ಲಿ ಹಾಗೂ ಆಂತರಿಕ ಪ್ರಕೃತಿಯ ಮೇಲೆ ಸಂಪೂರ್ಣ ಜಯ ಸಾಧಿಸುವುದರಲ್ಲಿ ಬಲವಾಗಿ ನಂಬುತ್ತಾನೆ. “ಆ ದಿಕ್ಕಿನಲ್ಲಿ ಎಷ್ಟನ್ನು ಸಾಧಿಸಿದ್ದೀರಿ? ನಿಮ್ಮ ಇಂದ್ರಿಯ ಮತ್ತು ಶರೀರಗಳನ್ನು ಎಷ್ಟು ಹತೋಟಿಯಲ್ಲಿಟ್ಟಿದ್ದೀರಿ' - ಎಂದಷ್ಟೇ ಯೋಗಿ ಕೇಳುವುದು. ನಾವು ಆಗಲೇ ಹೇಳಿದಂತೆ ಪ್ರತಿಯೊಬ್ಬನೂ ಇತರರನ್ನು ತನ್ನ ಮಾನದಂಡದಿಂದ ಅಳೆಯುತ್ತಾನೆ.

ಭಾರತದಲ್ಲಿ ಕೆಲವರು ಮಾಡಿರುವಂತೆ, ಮನುಷ್ಯನು ಇಲಿಗಳ ಬೆಕ್ಕುಗಳ ಹೊಟ್ಟೆ ತುಂಬಿಸುವುದಕ್ಕಾಗಿ ಲಕ್ಷಾಂತರ ಡಾಲರುಗಳನ್ನು ದಾನ ಮಾಡಿರಬಹುದು. ಅವರೆನ್ನುವುದೇನೆಂದರೆ, ಮನುಷ್ಯರು ತಮ್ಮ ಬಗ್ಗೆ ತಾವೇ ಕಾಳಜಿ ವಹಿಸಬಲ್ಲರು, ಆದರೆ ಪಾಪ! ಪ್ರಾಣಿಗಳಿಗೆ ಅದಾಗದೆಂದು. ಅದು ಅವರ ವಿಚಾರ. ಆದರೆ ಯೋಗಿಗಾದರೋ (ಆಂತರಿಕ) ಪ್ರಕೃತಿಯ ಮೇಲಿನ ಸ್ವಾಮ್ಯತೆಯೇ ಗುರಿ... ಅವನು ಮನುಷ್ಯನನ್ನು ಆ ಮಾನದಂಡದಿಂದಲೇ ಅಳೆಯುವುದು.

ನಾವು ಯಾವಾಗಲೂ ಅನುಷ್ಠಾನ ಧರ್ಮವನ್ನು ಕುರಿತು ಮಾತನಾಡುತ್ತೇವೆ. ಆದರೆ ನಮ್ಮ ದೃಷ್ಟಿಯಲ್ಲಿ ಅದು ಅನುಷ್ಠಾನಯೋಗ್ಯವಾಗಿರಲೇಬೇಕು. ವಿಶೇಷತಃ ಪಾಶ್ಚಿಮಾತ್ಯ ದೇಶಗಳಲ್ಲಂತೂ ಈ ಮಾತಿನ ಅರ್ಥ ಹೀಗೆಯೇ. ಪ್ರಾಟೆಸ್ಟಂಟರ ಆದರ್ಶ ಸತ್ಕರ್ಮಗಳನ್ನು ಮಾಡುವುದು. ಅವರು ಭಕ್ತಿ, ದರ್ಶನ ಇವುಗಳ ಬಗ್ಗೆ ಅಷ್ಟು ಲಕ್ಷ್ಯ ಕೊಡುವುದಿಲ್ಲ. ಅವುಗಳಲ್ಲಿ ಏನೂ ಉಪಯುಕ್ತವಾದುದು ಇಲ್ಲವೆಂದೇ ಭಾವಿಸುತ್ತಾರೆ. ಅವರೆನ್ನುತ್ತಾರೆ, ನಿಮ್ಮ ಜ್ಞಾನದಿಂದ ಆಗುವುದೇನು? ಮನುಷ್ಯ ಏನನ್ನಾದರೂ (ಸತ್ಕರ್ಮಗಳನ್ನು ) ಮಾಡಬೇಕು, ಒಂದಿಷ್ಟು ಲೋಕಕಲ್ಯಾಣ ಅಷ್ಟೇ! ಚರ್ಚುಗಳಾದರೋ ಹಗಲೂ ರಾತ್ರಿ ಕಲ್ಲೆದೆಯ ಅಜೇಯತಾವಾದದ ವಿರುದ್ದ ಕೂಗೆಬ್ಬಿಸುತ್ತಿರುತ್ತವೆ. ಆದರೆ (ವಿಪರ್ಯಾಸವೆಂದರೆ) ತಾವೇ ಆ (ಅಜೇಯತಾ ವಾದಿಗಳ) ದಿಕ್ಕಿನಲ್ಲಿ ತೀವ್ರಗತಿಯಲ್ಲಿ ವಾಲುತ್ತಿರುವ ಹಾಗಿದೆ. ಕಲ್ಲೆದೆಯ ಗುಲಾಮರು! ಇದು ಪ್ರಯೋಜನ ದೃಷ್ಟಿಯ ಧರ್ಮ! ಈಗ ಸಧ್ಯದಲ್ಲಿರುವ (ಧರ್ಮವೆಂಬುದರ ) ಆಶಯ ಅಷ್ಟೆ. ಆದ್ದರಿಂದಲೇ ಕೆಲವು ಬೌದ್ಧರು ಈ ಪಾಶ್ಚಾತ್ಯ ದೇಶಗಳಲ್ಲಿ ಅಷ್ಟೊಂದು ಜನಪ್ರಿಯರಾಗಿರುವುದು. ಜನರಿಗೆ ದೇವರಿದ್ದಾನೋ ಇಲ್ಲವೋ ಅಥವಾ ಆತ್ಮವೆಂಬುದು ಇದೆಯೋ ಇಲ್ಲವೋ ಗೊತ್ತಿಲ್ಲ. ಈ ಪ್ರಪಂಚವೆಲ್ಲ ದುಃಖದಿಂದ ಕೂಡಿದೆ. ಈ ಪ್ರಪಂಚಕ್ಕೆ ಸಹಾಯಮಾಡಲು ಪ್ರಯತ್ನಿಸಿ - (ಎಂಬುದು ಅವರ ನಿಲುವು. )

(ಆದರೆ) ನಾವಿಂದು ಉಪನ್ಯಾಸಕ್ಕಾಗಿ ಇಟ್ಟುಕೊಂಡಿರುವ ಯೋಗಸಿದ್ದಾಂತ ಆ ನಿಲುವಿನಿಂದ ಕೂಡಿದ್ದಲ್ಲ. (ಅದರ ಬೋಧನೆಯೇ) ಆತ್ಮವೆಂಬುದಿದೆ ಮತ್ತು ಈ ಆತ್ಮನಲ್ಲಿಯೇ ಎಲ್ಲ ಶಕ್ತಿಯೂ ಇದೆ ಎಂಬುದು. ಈ ಶಕ್ತಿ ಆಗಲೇ ಅದರಲ್ಲಿ ಅಂತರ್ನಿಹಿತವಾಗಿದೆ. ನಾವು ಈ ಶರೀರದ ಮೇಲೆ ಸ್ವಾಮ್ಯತೆ ಪಡೆಯಬಲ್ಲೆವಾದರೆ ಆ ಎಲ್ಲ ಶಕ್ತಿಯೂ ಹೊರಗೆ ಪ್ರಕಾಶಗೊಳ್ಳುತ್ತದೆ. ಎಲ್ಲ ಜ್ಞಾನವೂ ಆತ್ಮನಲ್ಲಿಯೇ ಇದೆ. ಜನ ಇಷ್ಟೊಂದು ಪರದಾಡುತ್ತಿರುವುದು ಏತಕ್ಕೆ? ದುಃಖವನ್ನು ಕಡಮೆ ಮಾಡುವುದಕ್ಕಾಗಿ... ನಮ್ಮ ಶರೀರದ ಮೇಲೆ ನಾವು ಸ್ವಾಮ್ಯತೆಯನ್ನು ಪಡೆಯದಿರುವುದೇ ಎಲ್ಲ ಅತೃಪ್ತಿ ಅಶಾಂತಿಗೂ ಕಾರಣ... ನಾವೆಲ್ಲ ಕುದುರೆಯನ್ನು ಗಾಡಿಯ ಹಿಂಭಾಗದಲ್ಲಿಟ್ಟು, ಹಿಂದು ಮುಂದೆ ಮಾಡಿ ಕ್ರಮವಿಪರ್ಯಯ ಮಾಡುತ್ತಿದ್ದೇವೆ... ಉದಾಹರಣೆಗೆ ಕರ್ಮಮಾರ್ಗವನ್ನೇ ತೆಗೆದುಕೊಳ್ಳಿ. ನಾವು ಒಳ್ಳೆಯದನ್ನು ಮಾಡಲು ಪ್ರಯತ್ನಿಸುತ್ತಿದ್ದೇವೆ - ಬಡವರನ್ನು ಸಂತೈಸುವುದರ ಮೂಲಕ. ಆದರೆ ದುಃಖವನ್ನುಂಟುಮಾಡಿದ ಮೂಲಕಾರಣದ ಬಗ್ಗೆ ನಮಗೇನೂ ಗೊತ್ತಿಲ್ಲ. ಇದು ಹೇಗೆಂದರೆ ಸಮುದ್ರದಲ್ಲಿರುವ ನೀರನ್ನೆಲ್ಲಾ ಒಂದು ಬಕೆಟ್‌ನಿಂದ ಖಾಲಿಮಾಡಲು ತೊಡಗಿದಂತೆ. ಪ್ರತಿಬಾರಿಯೂ (ನೀರು) ಮತ್ತೂ ತುಂಬಿಕೊಳ್ಳುತ್ತಿರುತ್ತದಷ್ಟೆ. ಯೋಗಿ ಇದರಲ್ಲಿನ ಅಸಂಬದ್ಧತೆ ಮತ್ತು ಮೂರ್ಖತೆಯ ವೈಪರೀತ್ಯವನ್ನು ಮನಗಾಣುತ್ತಾನೆ. (ಅವನು ಹೇಳುವುದೇನೆಂದರೆ) ದುಃಖದಿಂದ ಪಾರಾಗಬೇಕಾದರೆ ಮೊದಲು ದುಃಖದ ಕಾರಣವನ್ನು ಅರಿಯಬೇಕು. ನಾವು ಯಥಾಶಕ್ತಿ ಒಳ್ಳೆಯದನ್ನು ಮಾಡಲು ಪ್ರಯತ್ನಿಸುತ್ತೇವೆ. ಇದು ಏತಕ್ಕಾಗಿ? ವಾಸಿಯಾಗಲಾರದ ಯಾವುದಾದರೂ ವ್ಯಾಧಿ ಇದ್ದರೆ ನಾವೇಕೆ ನಮ್ಮ ಆರೈಕೆಗಾಗಿ ಅಷ್ಟೊಂದು ಹೋರಾಡಬೇಕು? ''ಪ್ರಯೋಜನ ದೇವರ ಬಗ್ಗೆ, ಆತ್ಮದ ಬಗ್ಗೆ ಅಷ್ಟೊಂದು ತಲೆ ಕೆಡಿಸಿಕೊಳ್ಳಬೇಡಿ'' ಎಂದು ಹೇಳಿದರೆ, ಅದರಿಂದ ಯೋಗಿಗಾಗಲೀ ಪ್ರಪಂಚಕ್ಕಾಗಲೀ ಆಗಬೇಕಾದ್ದೇನು? ಅಂತಹ ಮನೋಭಾವದಿಂದ ಪ್ರಪಂಚಕ್ಕೆ ಯಾವುದೇ ಒಳಿತಾಗಲೀ ಉಂಟಾಗುವುದಿಲ್ಲ. ಎಂದೆಂದೂ ಅಧಿಕಾಧಿಕ ದುಃಖವಷ್ಟೇ ಇರುತ್ತದೆ...

ಯೋಗಿ ಹೇಳುವುದೇನೆಂದರೆ ನೀವು ಇದೆಲ್ಲದರ ಮೂಲಕ್ಕೆ ಹೋಗಬೇಕು. ಈ ಸಂಸಾರದಲ್ಲಿ ಏಕೆ ದುಃಖವಿದೆ? ಇದಕ್ಕೆ ಯೋಗಿಯ ಉತ್ತರ: 'ಇದೆಲ್ಲ ನಮ್ಮ ಮೂರ್ಖತೆಯಿಂದುಂಟಾದುದು; ನಮ್ಮ ದೇಹದ ಮೇಲೆ ಸರಿಯಾದ ಸ್ವಾಮ್ಯತೆ. ಇಲ್ಲದಿರುವುದೇ ಕಾರಣ. ಅಷ್ಟಲ್ಲದೆ (ಬೇರೇನೂ ಅಲ್ಲ).” ಈ ದುಃಖವನ್ನು (ಜಯಿಸಲು) ಸಾಧ್ಯವಾದ ಮಾರ್ಗವನ್ನು ನಮಗೆ ಉಪದೇಶಿಸುತ್ತಾನೆ. ನೀವೇನಾದರೂ ಈ ರೀತಿ ನಿಮ್ಮ ಶರೀರದ ಮೇಲೆ ನಿಯಂತ್ರಣ ಪಡೆದರೆ ಪ್ರಪಂಚದಲ್ಲಿರುವ ದುಃಖವೆಲ್ಲ ನಾಶವಾಗುವುದು. ಪ್ರತಿಯೊಂದು ಆಸ್ಪತ್ರೆಯೂ ಹೆಚ್ಚು ಹೆಚ್ಚು ಅಸ್ವಸ್ಥರು ಅಲ್ಲಿ ಬರಲೆಂದು ಬೇಡುತ್ತಿರುತ್ತದೆ. ನೀವು ದಾನಮಾಡಬೇಕೆಂದು ಯೋಚಿಸಿದ ಪ್ರತಿಬಾರಿಯೂ, ನಿಮ್ಮ ದಾನವನ್ನು ಸ್ವೀಕಾರಮಾಡಲು ಯಾವ ಭಿಕಾರಿಯಿದ್ದಾನೆ ಎಂದುಕೊಳ್ಳುತ್ತೀರಿ. “ಪ್ರಭುವೇ, ಈ ಪ್ರಪಂಚವೆಲ್ಲ ದಾನಿಗಳಿಂದ ತುಂಬಿರಲಿ' - ಎಂದು ನೀವು ಹೇಳಿದಾಗ ನಿಮ್ಮ ಇಂಗಿತ ಇಷ್ಟೆ - ಈ ಪ್ರಪಂಚವು ಭಿಕಾರಿಗಳಿಂದ ತುಂಬಿರಲಿ ಎಂದು. ಈ ಪ್ರಪಂಚವೆಲ್ಲ (ದಾನವೇ ಮುಂತಾದ) ಒಳ್ಳೆಯ ಕರ್ಮಗಳಿಂದ ತುಂಬಿರಲಿ (ಎಂದಾಗ) ಈ ಪ್ರಪಂಚ ದುಃಖದಿಂದ ತುಂಬಿರಲಿ (ಎಂದೇ). ಇದು ನಖಶಿಖಾಂತ ಗುಲಾಮಗಿರಿಯಷ್ಟೇ!

ಯೋಗಿ ಹೇಳುತ್ತಾನೆ: ನೀವು ದುಃಖ ಏತಕ್ಕಿದೆ ಎಂಬುದನ್ನು ಮೊದಲು ಅರಿತರೆ ಮಾತ್ರ ಧರ್ಮ ಅನುಷ್ಠಾನಪೂರ್ಣವಾಗಿರುತ್ತದೆ. ಪ್ರಪಂಚದ ಎಲ್ಲ ದುಃಖವೂ ಇಂದ್ರಿಯಗಳಲ್ಲಿದೆ. ಸೂರ್ಯ, ಚಂದ್ರ, ನಕ್ಷತ್ರಗಳಲ್ಲಿ ಯಾವುದಾದರೂ ರೋಗದ ನರಳಾಟವಿದೆಯೇನು? ನಿಮ್ಮ ಊಟ ತಯಾರುಮಾಡಲು ಬೇಕಾದ ಬೆಂಕಿಯೇ ಮಗುವಿನ ಬೆರಳನ್ನು ಸುಡುತ್ತದೆ. ಅದು ಬೆಂಕಿಯ ದೋಷವೇ? ಬೆಂಕಿ ಧನ್ಯವಾದುದು! ವಿದ್ಯುಚ್ಛಕ್ತಿಯೂ ಧನ್ಯವಾದುದು!ಕಾರಣ ಅದು ಪ್ರಕಾಶವನ್ನು ಕೊಡುತ್ತದೆ... ನೀವು ಯಾರಮೇಲೆ ತಪ್ಪನ್ನು ಹೊರಿಸುತ್ತೀರಿ? ಪಂಚಭೂತಗಳ ಮೇಲಂತೂ ಆಗದು. ಈ ಪ್ರಪಂಚ ಒಳ್ಳೆಯದೂ ಅಲ್ಲ, ಕೆಟ್ಟದ್ದೂ ಅಲ್ಲ. ಪ್ರಪಂಚ ಪ್ರಪಂಚವೇ! ಬೆಂಕಿ ಬೆಂಕಿಯೇ! ನಿಮ್ಮ ಬೆರಳನ್ನು ಅದರಲ್ಲಿ ಸುಟ್ಟುಕೊಂಡರೆ ನೀವೊಬ್ಬ ಮೂರ್ಖರು. ನೀವು (ಅದರಿಂದ ಊಟ ತಯಾರುಮಾಡಿ ನಿಮ್ಮ ಹಸಿವನ್ನು ಹಿಂಗಿಸಿಕೊಂಡರೆ) ನೀವೊಬ್ಬ ವಿವೇಕಿಗಳು. ಅಷ್ಟೇ ವ್ಯತ್ಯಾಸ. ಪರಿಸ್ಥಿತಿ ಎಂದೂ ಒಳ್ಳೆಯದಾಗಿ ಆಗಲಿ, ಕೆಟ್ಟದಾಗಿ ಆಗಲಿ ಇರಲಾರದು. ಈ ಪ್ರಪಂಚ ಒಳ್ಳೆಯದು ಅಥವಾ ಕೆಟ್ಟದ್ದು ಎಂಬುದರ ಅರ್ಥವಾದರೂ ಏನು? ದುಃಖ ಮತ್ತು ಸುಖಗಳು ಇಂದ್ರಿಯಲೋಲುಪನಾದ ವೈಯಕ್ತಿಕ ಜೀವಿಗೆ ಮಾತ್ರ ಸೇರಿದ್ದಾಗಿವೆ. ಯೋಗಿಗಳ ಪ್ರಕಾರ ಪ್ರಕೃತಿಯು ಭೋಗ್ಯ ಮತ್ತು ಆತ್ಮವು ಭೋಕ್ತಾ (ಅನುಭವಿಸುವವನು). ಎಲ್ಲ ದುಃಖ ಮತ್ತು ಸುಖಗಳು ಎಲ್ಲಿವೆ? ಈ ಇಂದ್ರಿಯಗಳಲ್ಲಿ, ಇಂದ್ರಿಯಗಳ (ಒಡನೆ ನಮಗಿರುವ) ಸಂಸ್ಪರ್ಶವೇ ಶರೀರದ ಸುಖಕ್ಕೂ ಮತ್ತು ನೋವಿಗೂ, ಶೀತೋಷ್ಣಗಳಿಗೂ ಕಾರಣ. ಈಗ ಇಂದ್ರಿಯಗಳು ನಮ್ಮನ್ನು (ತಮ್ಮ ವಶದಲ್ಲಿಟ್ಟುಕೊಂಡು) ಅವುಗಳ ಆಜ್ಞಾಧಾರಕರಾಗಿರುವಂತೆ ಮಾಡಿವೆ. ಹಾಗಾಗಲು ಬಿಡದೆ ನಾವು ಇಂದ್ರಿಯಗಳನ್ನು ನಿಯಂತ್ರಿಸಬಲ್ಲೆವಾದರೆ ಮತ್ತು ಅವುಗಳು ಏನನ್ನು ಸಂವೇದಿಸಬೇಕು ಎಂಬುದನ್ನು ಆದೇಶಿಸಬಲ್ಲೆವಾದರೆ - ಅವು ನಮ್ಮ ಆದೇಶವನ್ನು ಪಾಲಿಸಿ, ನಮ್ಮ ಗುಲಾಮರಂತಾದರೆ, ನಮ್ಮ ಸಮಸ್ಯೆ ಒಮ್ಮೆಲೇ ಪರಿಹಾರವಾದಂತೆಯೇ. ನಾವು ಇಂದ್ರಿಯಗಳಿಂದ ಬದ್ಧರಾಗಿದ್ದೇವೆ. ಅವು ನಮ್ಮನ್ನು ಹೇಗೆಂದರೆ ಹಾಗೆ ಕುಣಿಸಿ ನಮ್ಮನ್ನು ಮೂರ್ಖರನ್ನಾಗಿಸಿವೆ.

(ಉದಾಹರಣೆಗೆ ) ಇಲ್ಲೊಂದು ದುರ್ಗಂಧವಿದೆ. ಅದು ನಮ್ಮ ಮೂಗನ್ನು ಪ್ರವೇಶಿಸಿದೊಡನೆಯೇ ನನಗೆ ಕಸಿವಿಸಿಯಾಗುತ್ತದೆ. ನಾನು ನನ್ನ ಮೂಗಿನ ಗುಲಾಮ. ನಾನು (ಆ ರೀತಿ) ಗುಲಾಮನಾಗಿರದಿದ್ದರೆ, ನಾನದನ್ನು ಲಕ್ಷಿಸುತ್ತಲೇ ಇರಲಿಲ್ಲ. ಒಬ್ಬನು ನನ್ನನ್ನು ಬೈಯುತ್ತಾನೆ. ಅವನ ಬೈಗಳು ನನ್ನ ಕಿವಿಯನ್ನು ಪ್ರವೇಶಿಸಿ ನನ್ನ ಮನಸ್ಸಿನಲ್ಲಿ ಮತ್ತು ಶರೀರದಲ್ಲಿ ಉಳಿಯುತ್ತವೆ. ನಾನೇನಾದರೂ (ಇಂದ್ರಿಯಗಳ ) ಒಡೆಯನಾಗಿದ್ದರೆ - ನಾನು ಹೇಳುತ್ತೇನೆ: "ಅವುಗಳ ಪಾಡಿಗೆ ಅವುಗಳಿರಲಿ, ಅವು ನನಗೇನೂ ಅಲ್ಲ. ನಾನು ದುಃಖಿಯೂ ಅಲ್ಲ. ನಾನು (ಅವುಗಳ ಬಗ್ಗೆ ತಲೆಕೆಡಿಸಿಕೊಳ್ಳುವುದೂ ಇಲ್ಲ.” ಇದು ನೇರವಾದ, ಸರಳವಾದ, ಮತ್ತು ಸ್ಪಷ್ಟವಾದ ಸತ್ಯ.

ಬಗೆಹರಿಸಬೇಕಾದ ಮತ್ತೊಂದು ಸಮಸ್ಯೆಯೆಂದರೆ – ಇದು ಅನುಷ್ಠಾನಸಾಧ್ಯವೇ ಎಂಬುದು. ಮನುಷ್ಯ ತನ್ನ ದೇಹದ ಮೇಲೆ ಸಂಪೂರ್ಣ ನಿಯಂತ್ರಣ ಶಕ್ತಿಯನ್ನು ಪಡೆಯಲು ಸಾಧ್ಯವೇ? ಯೋಗ ಹೇಳುತ್ತದೆ: ಇದು ಅನುಷ್ಠಾನಸಾಧ್ಯ ಎಂದು. ಒಂದು ವೇಳೆ ಹಾಗಲ್ಲ ಎಂದುಕೊಳ್ಳೋಣ. ನಿಮ್ಮ ಮನಸ್ಸಿನಲ್ಲಿ ಅನೇಕ ಸಂದೇಹಗಳಿರಬಹುದು. (ಆದರೆ) ನೀವು ಆ ದಿಕ್ಕಿನಲ್ಲಿ ಪ್ರಯತ್ನವನ್ನಂತೂ ಮಾಡಲೇಬೇಕಾಗಿದೆ. (ಕಾರಣ ಇದನ್ನು ಬಿಟ್ಟು ) ಬೇರೆ ದಾರಿಯೇ ಇಲ್ಲ.

ನೀವು ಸದಾ ಲೋಕಹಿತವನ್ನೇ ಮಾಡಬಹುದು. ಆದರೂ ನೀವು ಇಂದ್ರಿಯಗಳ ಗುಲಾಮರಾಗಿದ್ದರೆ ನೀವು ದುಃಖಿಗಳೂ, ನೆಮ್ಮದಿಯಿಲ್ಲದವರೂ ಆಗಿರುತ್ತೀರಿ. ನೀವು ಪ್ರತಿಯೊಂದು ಧರ್ಮದ ದರ್ಶನಶಾಸ್ತ್ರಗಳನ್ನು ಅಧ್ಯಯನ ಮಾಡಬಹುದು. ಈ ದೇಶದಲ್ಲಂತೂ ಜನ ಪುಸ್ತಕಗಳ ಹೊರೆಯನ್ನೇ ತಮ್ಮ ಬೆನ್ನಿನಲ್ಲಿ ಹೊತ್ತಿರುತ್ತಾರೆ. ಆದರೆ ಅವರು ಬರೆ ಪಂಡಿತರು, ಇಂದ್ರಿಯಗಳ ಗುಲಾಮರು. ಆದ್ದರಿಂದಲೇ ಅವರು ಸುಖ - ದುಃಖಗಳ (ಹೊಡೆತಕ್ಕೆ ಸಿಲುಕಿರುವರು). ಅವರು ಎರಡು ಸಾವಿರ ಗ್ರಂಥಗಳನ್ನು ಓದುವರು - ಅದು ಸರಿ, ಆದರೆ ಏನಾದರೂ ಸ್ವಲ್ಪ ದುಃಖ ಬಂದಿತೆಂದರೆ ಚಿಂತಾಕ್ರಾಂತರಾಗುತ್ತಾರೆ, ಕಂಗಾಲಾಗುತ್ತಾರೆ, ಉದ್ವಿಗ್ನರಾಗುತ್ತಾರೆ! ನಿಮ್ಮನ್ನು ನೀವು ಮನುಷ್ಯರೆಂದುಕೊಳ್ಳುತ್ತೀರಿ! ನೀವು ಉದ್ಯುಕ್ತರಾಗಿ... ಆಸ್ಪತ್ರೆಗಳನ್ನು ಕಟ್ಟುತ್ತೀರಿ. (ಆದರೆ) ನೀವು ಮೂರ್ಖರಷ್ಟೇ!

ಮನುಷ್ಯನಿಗೂ ಮೃಗಗಳಿಗೂ ಇರುವ ವ್ಯತ್ಯಾಸವೇನು? "ಆಹಾರ (ನಿದ್ರೆ), ಸಂತಾನೋತ್ಪತ್ತಿ ಮತ್ತು ಭಯ ಮನುಷ್ಯ ಮತ್ತು ಪಶುಗಳೆರಡರಲ್ಲಿಯೂ ಸಮಾನ ರೂಪದಲ್ಲಿರುವಂಥದ್ದೆ. ಆದರೆ ಒಂದು ವ್ಯತ್ಯಾಸವಿದೆ. ಮನುಷ್ಯ ಇವುಗಳನ್ನೆಲ್ಲ ನಿಯಂತ್ರಿಸಿ ದೇವನಾಗಬಲ್ಲ, ಅವನು ಈಶ್ವರನಾಗಬಲ್ಲ, ಪ್ರಭುವಾಗಬಲ್ಲ.'' ಪಶುಗಳು ಇದನ್ನು ಮಾಡಲಾರವು. ಪಶುಗಳೂ ಸಹ ಪರೋಪಕಾರವನ್ನು ಮಾಡಬಲ್ಲವು. ಇರುವೆಗಳು ಮಾಡುತ್ತವೆ, ನಾಯಿಗಳು ಮಾಡುತ್ತವೆ. ಹಾಗಾದರೆ ವ್ಯತ್ಯಾಸವಿರುವುದು ಎಲ್ಲಿ? ಮನುಷ್ಯರು ತಮ್ಮನ್ನೇ (ಜಯಿಸಿದ) ಪ್ರಭುವಾಗಬಲ್ಲರು. ಮನುಷ್ಯರು (ವಾಸನಾ ಜನ್ಯವಾದುದಕ್ಕೆ) ಪ್ರತಿಕ್ರಿಯೆಯನ್ನು ತಡೆಯಬಲ್ಲರು... ಪಶುವಾದರೋ ಯಾವುದನ್ನೂ ತಡೆಯಲಾರದು. ಅದು ಎಲ್ಲ ಕಡೆಯಿಂದಲೂ ಪ್ರಕೃತಿಯ ಶೃಂಖಲೆಗಳಿಂದ ಹಿಡಿದಿಡಲ್ಪಟ್ಟಿದೆ. ಅದಷ್ಟೇ ವ್ಯತ್ಯಾಸ. ಒಬ್ಬನು ಪ್ರಕೃತಿಯ (ವಶದಲ್ಲಿಟ್ಟುಕೊಂಡಿರುವ) ಪ್ರಭು, ಮತ್ತೊಂದು ಪ್ರಕೃತಿಯ (ವಶದಲ್ಲಿರುವ) ಗುಲಾಮ. ಪ್ರಕೃತಿಯೆಂದರೆ ಯಾವುದು? ನಮ್ಮ ಪಂಚೇಂದ್ರಿಯಗಳೇ..

ಯೋಗಶಾಸ್ತ್ರದ ಪ್ರಕಾರ (ಆಂತರಿಕ ಪ್ರಕೃತಿಯ, ಚಿತ್ತವೃತ್ತಿಗಳ ಮೇಲಿನ ವಿಜಯವೇ ) ಏಕಮಾತ್ರ ಉಪಾಯ. ದೇವರಿಗೋಸ್ಕರ (ನಮ್ಮಲ್ಲಿನ) ದಾಹವೇ ಧರ್ಮ. ಲೋಕೋಪಕಾರಿ ಸತ್ಕರ್ಮಗಳು ಮನಸ್ಸನ್ನು ಸ್ವಲ್ಪಮಟ್ಟಿಗೆ ಶಾಂತ ಗೊಳಿಸುತ್ತವೆ, ಅಷ್ಟೇ. ಪರಿಪೂರ್ಣರಾಗಲು ಮಾಡುವ ಈ ಯೋಗ ಸಾಧನೆಗಳೆಲ್ಲವೂ ನಮ್ಮ ಪೂರ್ವ (ಸಂಸ್ಕಾರಗಳ ) ಮೇಲೆ ನಿರ್ಭರವಾಗಿವೆ. ನನ್ನ ಇಡೀ ಜೀವನ ನಾನು (ಯೋಗದ) ಅಧ್ಯಯನದಲ್ಲಿ ತೊಡಗಿದ್ದೇನೆ ಮತ್ತು ಈವರೆಗೂ ನಾನು ಸಾಧಿಸಿರುವ ಪ್ರಗತಿ ಬಹಳ ಅಲ್ಪವಾದುದು. ಆದರೆ ಇದೊಂದೇ ನಿಜವಾದ ಮಾರ್ಗ ಎಂದು ನಂಬಲು ಸಾಕಷ್ಟು ಫಲಿತಾಂಶ ನನಗೆ ದೊರಕಿದೆ. ನಾನೇ ನನ್ನ ಪ್ರಭುವಾಗುವ ಒಂದು ದಿನ ಬರುತ್ತದೆ. ಈ ಜನ್ಮದಲ್ಲಿ ಅಲ್ಲದಿದ್ದರೆ ಮತ್ತೊಂದು ಜನ್ಮದಲ್ಲಾಗಬಹುದು. ಆದರೆ ನನ್ನ ಹೋರಾಟವಂತೂ ಮುಂದುವರಿಯುತ್ತದೆ, ಅದನ್ನೆಂದೂ ನಾನು ಬಿಡುವುದಿಲ್ಲ. (ಇಲ್ಲಿ ) ಯಾವುದೂ ನಷ್ಟವಾಗುವುದಿಲ್ಲ. ಈ ಕ್ಷಣದಲ್ಲಿ ನಾನು ಸತ್ತುಹೋದರೂ (ನನ್ನ ಮುಂದಿನ ಜನ್ಮದಲ್ಲಿ) ಗತಕಾಲದ ಹೋರಾಟಗಳೆಲ್ಲ (ನನ್ನ ಸಹಾಯಕ್ಕೆ ಬರುತ್ತವೆ). ಒಬ್ಬನ ಮತ್ತು ಮತ್ತೊಬ್ಬನ ನಡುವಿನ ವ್ಯತ್ಯಾಸವನ್ನುಂಟು ಮಾಡುವುದು ಯಾವುದು ಎಂಬುದು ನಿಮಗೆ ಕಾಣುವುದಿಲ್ಲವೇನು? ಅದೇ ಅವರ ಪ್ರಾರಬ್ಧ - (ಪೂರ್ವಜನ್ಮ ಸಂಸ್ಕಾರಗಳು). ಪೂರ್ವ ಸಂಸ್ಕಾರಗಳು ಒಬ್ಬನನ್ನು ಪ್ರತಿಭಾಶಾಲಿಯನ್ನಾಗಿಯೂ ಮತ್ತೊಬ್ಬನನ್ನು ಮೂರ್ಖನನ್ನಾಗಿಯೂ ಮಾಡುತ್ತವೆ. ಪೂರ್ವಸಂಸ್ಕಾರಗಳ ಬಲದಿಂದ ನೀವು ಐದೇ ನಿಮಿಷದಲ್ಲಿ ಯಶಸ್ವಿಯಾಗಬಲ್ಲಿರಿ. ಮುಂದಿನ ಕ್ಷಣದಲ್ಲಿ ಏನಾಗುತ್ತದೆ ಎಂಬುದನ್ನು ಯಾರೂ ಹೇಳಲಾರರು. ಒಂದಲ್ಲ ಒಂದು ದಿನ ನಾವೆಲ್ಲರೂ (ಪೂರ್ಣತ್ವವನ್ನು, ಸಿದ್ದಿಯನ್ನು) ಪಡೆಯಲೇಬೇಕು.

ಯೋಗಿಯು ನಮಗೆ ಕೊಡುವ ಅನುಷ್ಠಾನಿಕ ಪಾಠಗಳಲ್ಲಿ ಅಧಿಕಾಂಶ ನಮ್ಮ ಮನಸ್ಸಿನಲ್ಲಿರುವ ಏಕಾಗ್ರತೆಯ ಶಕ್ತಿ ಮತ್ತು ಧ್ಯಾನಕ್ಕೆ (ಸಂಬಂಧಿಸಿದ್ದಾಗಿದೆ). ನಾವೆಷ್ಟು ಭೌತವಾದಿಗಳಾಗಿದ್ದೇವೆಂದರೆ, ನಮ್ಮ ಬಗ್ಗೆ ನಾವು ಯೋಚಿಸಿದಾಗಲೆಲ್ಲ ದೇಹವನ್ನು ಮಾತ್ರ ಕಾಣಬಲ್ಲೆವು. ದೇಹವೇ ನಮ್ಮ ಆದರ್ಶವಾಗಿದೆ - ಮತ್ತಿನ್ಯಾವುದೂ ಅಲ್ಲ. ಆದ್ದರಿಂದ ಸ್ವಲ್ಪ ಶಾರೀರಿಕ ಸಹಾಯದ ಅಗತ್ಯವಿದೆ.

ಮೊದಲನೆಯದಾಗಿ, ಬಹಳ ಹೊತ್ತಿನವರೆಗೆ ಕದಲದೆ ಕುಳಿತುಕೊಳ್ಳಬಹುದಾದ ಆಸನದಲ್ಲಿ ಸ್ಥಿರವಾದ ಭಂಗಿಯಲ್ಲಿ ಕುಳಿತುಕೊಳ್ಳುವುದು. ಸಕ್ರಿಯವಾಗಿರುವ ಎಲ್ಲ ನಾಡೀ ಪ್ರವಾಹಗಳೂ ಮೇರುದಂಡದ ಮೂಲಕವೇ ಹಾದುಹೋಗುತ್ತವೆ. ಶರೀರದ ಸ್ಥೂಲವಾದ ಭಾರವನ್ನು ಹೊರುವುದಕ್ಕಲ್ಲ ಮೇರುದಂಡವಿರುವುದು. ಆದ್ದರಿಂದ ನಾವು ಕುಳಿತುಕೊಳ್ಳುವ ಆಸನ ಅಥವಾ ಭಂಗಿ ಯಾವರೀತಿ ಇರಬೇಕೆಂದರೆ ಶರೀರದ ಭಾರ ಮೇರುದಂಡದ ಮೇಲೆ ಬೀಳದಂತಿರಬೇಕು. ಅದು ಎಲ್ಲ ಪ್ರಕಾರವಾದ ಭಾರ, ಒತ್ತಡಗಳಿಂದ ಮುಕ್ತವಾಗಿರಲಿ.

ಮತ್ತೂ ಕೆಲವು ಪೂರ್ವಭಾವಿಯಾದ ವಿಷಯಗಳಿವೆ. ಊಟ, ತಿಂಡಿ, ವ್ಯಾಯಾಮಗಳಿಗೆ ಸಂಬಂಧಿಸಿದಂತೆ ಮುಖ್ಯವಾದ ಪ್ರಶ್ನೆಯಿದೆ...

ಊಟ, ತಿಂಡಿ ಬಹಳ ಸರಳವಾಗಿರಬೇಕು. ಒಮ್ಮೆಯೋ ಅಥವಾ ಎರಡು ಬಾರಿಯೋ (ಹೆಚ್ಚು) ತೆಗೆದುಕೊಳ್ಳುವುದಕ್ಕಿಂತಲೂ ಕೆಲವೊಂದು ಬಾರಿ (ಸ್ವಲ್ಪ ಸ್ವಲ್ಪ) ತೆಗೆದುಕೊಳ್ಳಬೇಕು. ಯಾವಾಗಲೇ ಆಗಲೀ ಬಹಳ ಹಸಿವಿನಿಂದ ಇರಬಾರದು. ಯಾವನು ಹೆಚ್ಚು ತಿನ್ನುತ್ತಾನೋ ಅವನು ಯೋಗಿಯಾಗಲಾರ. ಯಾವನು ಹೆಚ್ಚು ಉಪವಾಸವಿರುತ್ತಾನೋ ಅವನೂ ಯೋಗಿಯಾಗಲಾರ. ಯಾವನು ಹೆಚ್ಚು ನಿದ್ದೆ ಮಾಡುತ್ತಾನೋ ಅವನಾಗಲೀ ಅಥವಾ ಯಾವನು ಹೆಚ್ಚು ಜಾಗರಣೆ ಮಾಡುತ್ತಾನೋ ಅವನಾಗಲೀ ಯೋಗಿಯಾಗಲಾರ. ಯಾರು ಏನೂ ಕೆಲಸವನ್ನೇ ಮಾಡುವುದಿಲ್ಲವೋ ಮತ್ತೆ ಯಾರು ಮಿತಿಮಿಾರಿ (ದೈಹಿಕ) ಶ್ರಮದ ಕೆಲಸಗಳಲ್ಲಿ ತೊಡಗಿರುತ್ತಾನೋ ಅವನೂ ಯಶಸ್ವಿಯಾಗಲಾರ. ಯೋಗ್ಯವಾದ ಆಹಾರ, ಯೋಗ್ಯವಾದ ವ್ಯಾಯಾಮ, ಯೋಗ್ಯವಾದ ನಿದ್ದೆ, ಯೋಗ್ಯವಾದ ಎಚ್ಚರ - ಇವಿಷ್ಟೂ ಯಾವುದೇ ಯಶಸ್ಸಿಗೂ ಬಹಳ ಅಗತ್ಯ.

ಯಾವುದು ಯೋಗ್ಯವಾದ ಆಹಾರ, ಅದು ಯಾವ ರೀತಿಯದ್ದಾಗಿರಬೇಕು ಎಂಬುದನ್ನು ನಾವೇ ನಿರ್ಧರಿಸಬೇಕು. ಬೇರೆ ಯಾರೂ ಅದನ್ನು (ನಮಗೋಸ್ಕರ) ನಿರ್ಧರಿಸಲಾರರು. ಸಾಧಾರಣವಾದ ರೂಢಿಯೆಂದರೆ ನಾವು ಉತ್ತೇಜಕವಾದ ಆಹಾರವನ್ನು ತ್ಯಜಿಸಬೇಕು. ನಮ್ಮ ಉದ್ಯೋಗಗಳಿಗೆ ತಕ್ಕಂತೆ ನಮ್ಮ ಆಹಾರವನ್ನು ಹೇಗೆ ಬದಲಾಯಿಸಿಕೊಳ್ಳಬೇಕೆಂಬುದು ನಮಗೆ ಗೊತ್ತಿಲ್ಲ. ಆಹಾರವೇ ನಮ್ಮಲ್ಲಿರುವ ಪ್ರತಿಯೊಂದನ್ನೂ ಉತ್ಪತ್ತಿ ಮಾಡುವುದು ಎಂಬುದನ್ನು ನಾವು ಯಾವಾಗಲೂ ಮರೆಯುತ್ತೇವೆ. ಆದ್ದರಿಂದ ನಮಗೆ ಬೇಕಾಗಿರುವ ಶಕ್ತಿಯ ಪ್ರಮಾಣ ಮತ್ತು ರೀತಿಯನ್ನು ಆಹಾರವೇ ನಿರ್ಧರಿಸಬೇಕು.

ಉಗ್ರವಾದ ವ್ಯಾಯಾಮಗಳ ಅಗತ್ಯವೇನೂ ಇಲ್ಲ... ಮಾಂಸಖಂಡಗಳನ್ನು ಪುಷ್ಟಿಗೊಳಿಸುವುದಷ್ಟೇ ನಿಮಗೆ ಬೇಕಾಗಿದ್ದರೆ ಯೋಗ ನಿಮಗಲ್ಲ. ಈಗ ನಿಮ್ಮಲ್ಲಿರುವ ಶರೀರಕ್ಕಿಂತಲೂ ಸೂಕ್ಷ್ಮವಾದ ಶರೀರವನ್ನು ನೀವು ಉತ್ಪತ್ತಿ ಮಾಡಬೇಕು. ಬಲಾತ್ಕಾರವಾದ (ಶರೀರವನ್ನು ಪೀಡಿಸುವ) ಕಸರತ್ತುಗಳು, ನಿಸ್ಸಂದೇಹವಾಗಿ ಹಾನಿಕರ... ಯಾರು ಮಿತಿಮಿಾರಿ ಕಸರತ್ತುಗಳಲ್ಲಿ ತೊಡಗುವುದಿಲ್ಲವೋ ಅವರೊಡನೆ ವಾಸಿಸಿ. ನೀವು ಉಗ್ರವಾದ ಕಸರತ್ತುಗಳಲ್ಲಿ ತೊಡಗದಿದ್ದಲ್ಲಿ ಹೆಚ್ಚು ಕಾಲ ಬದುಕುತ್ತೀರಿ. ನಿಮ್ಮ (ಜೀವನ) ದೀಪ ಬರೀ ಮಾಂಸಖಂಡಗಳನ್ನು ಪುಷ್ಟಿಗೊಳಿಸುವುದರಲ್ಲೇ ಉರಿದು ಹೋಗುವುದು ಬೇಕಿಲ್ಲ, ಅಲ್ಲವೇ! ಮೆದುಳಿನೊಂದಿಗೆ ಕೆಲಸ ಮಾಡುವ ಜನರೇ ದೀರ್ಘಕಾಲ ಬದುಕುವವರು. (ಜೀವನ) ದೀಪವನ್ನು ಬೇಗ ಬೇಗ ಉರಿಸಿಬಿಡಬೇಡಿ. ಅದು ನಿಧಾನವಾಗಿ, ಮೆಲ್ಲಗೆ ಶಾಂತವಾಗಿ ಉರಿಯಲಿ. ಪ್ರತಿಯೊಂದು ಉದ್ವೇಗ, ಕಾತರತೆ, ಉಗ್ರ ವ್ಯಾಯಾಮ ಇವುಗಳ ಮೂಲಕ ಜೀವನದೀಪವು ಬೇಗ ಉರಿದುಹೋಗುತ್ತದೆ.

ಯೋಗ್ಯವಾದ ಆಹಾರ ಎಂದರೆ, ಸಾಧಾರಣವಾಗಿ, ಹೆಚ್ಚು ಖಾರ, ಮಸಾಲೆ ಇರುವ ಪದಾರ್ಥಗಳನ್ನು ಸೇವಿಸಬೇಡಿ. ಯೋಗಿಯ ಪ್ರಕಾರ ಪ್ರಕೃತಿಯ ಗುಣಗಳಿಗನುಗುಣವಾಗಿ ಮನಸ್ಸು ಮೂರು ವಿಧವಾಗಿದೆ. ಒಂದು ತಮಸ್ಸು, ಮಂಕಾಗಿ ಆಲಸ್ಯದಿಂದಿರುವುದು - ಇದು ಆತ್ಮದ ತೇಜಸ್ಸನ್ನು ಹೆಚ್ಚು ಮುಸುಕುವುದು. ನಂತರ ಜನರನ್ನು ಕ್ರಿಯಾಶೀಲರನ್ನಾಗಿ ಮಾಡುವ ರಾಜಸಿಕ ಮನಸ್ಸು ಮತ್ತು ಕೊನೆಯದಾಗಿ ಅವರನ್ನು ಸೌಮ್ಯವಾಗಿ, ಶಾಂತರನ್ನಾಗಿ ಮಾಡುವ ಸಾತ್ವಿಕ ಮನಸ್ಸು.

ಕೆಲವರಿರುತ್ತಾರೆ, ಅವರು ಯಾವಾಗಲೂ ನಿದ್ದೆಯ ಮಂಪರಿನಲ್ಲಿರುವ ಪ್ರವಣತೆಯನ್ನು ಹುಟ್ಟಿನಿಂದಲೇ ಹೊಂದಿರುತ್ತಾರೆ. ಅವರ ರುಚಿ ಯಾವಾಗಲೂ ಹಳಸಿಹೋದ ಆಹಾರದ ಕಡೆಗೇ. ಅವರು ಕೊಳೆತು ಹರಿದುಹೋಗುವ ಚೀಸಿನಂಥ ಆಹಾರ ಪದಾರ್ಥಗಳನ್ನು ತಿನ್ನುತ್ತಾರೆ. ಅದು ಅವರ ಮಟ್ಟಿಗೆ ಸಹಜ ಪ್ರವೃತ್ತಿ. ನಂತರ ರಾಜಸಿಕ ಜನ - ಅವರ ರುಚಿ ಯಾವಾಗಲೂ ಖಾರ, ಘಾಟು ಮತ್ತು ನಾತಹೊಡೆಯುತ್ತಿರುವ ಮದ್ಯದಲ್ಲಿಯೇ.

ಸಾತ್ವಿಕ ಜನರು ಯೋಚನಾಪರರೂ, ಶಾಂತರೂ, ತಾಳ್ಮೆಯುಳ್ಳವರೂ ಆಗಿರುತ್ತಾರೆ. ಅವರು ಆಹಾರವನ್ನು ಸಣ್ಣ ಪ್ರಮಾಣದಲ್ಲಿ ತೆಗೆದುಕೊಳ್ಳುತ್ತಾರೆ. ಕೆಟ್ಟುಹೋದದ್ದನ್ನು ಅವರೆಂದಿಗೂ ಸೇವಿಸುವುದಿಲ್ಲ.

ನನ್ನನ್ನು ಯಾವಾಗಲೂ ಕೇಳುವ ಪ್ರಶ್ನೆಯೆಂದರೆ, "ನಾನು ಮಾಂಸಾಹಾರವನ್ನು ತ್ಯಜಿಸಬೇಕೆ?'' ಎಂದು. ನನ್ನ ಗುರುದೇವರು ಹೇಳುತ್ತಿದ್ದರು: 'ನೀನೇಕೆ ಯಾವುದನ್ನಾದರೂ ತ್ಯಜಿಸಬೇಕು? ಅದೇ ನಿನ್ನನ್ನು ತ್ಯಜಿಸುತ್ತದೆ. ಪ್ರಕೃತಿಯಲ್ಲಿರುವ ಯಾವುದೇ ಪದಾರ್ಥವನ್ನೂ ತ್ಯಜಿಸಬೇಡಿ. ಆದರೆ ನಿಮ್ಮನ್ನೇ ನೀವು ಎಷ್ಟು ತೀಕ್ಷ್ಣಗೊಳಿಸಬೇಕೆಂದರೆ ಪ್ರಕೃತಿಯೇ ಸ್ವಯಂ ನಿಮ್ಮನ್ನು ಬಿಟ್ಟು ಬಿಡಬೇಕು. ಒಂದು ಸಮಯ ಬರುತ್ತದೆ. ಆಗ ನಿಮ್ಮಿಂದ ಮಾಂಸಾಹಾರ ಸೇವನೆಯೇ ಸಾಧ್ಯವಾಗುವುದಿಲ್ಲ. ಅದನ್ನು ಕಂಡಾಕ್ಷಣ ನಿಮಗೆ ವಾಕರಿಕೆ ಬರುವಂತಾಗುತ್ತದೆ. ಮತ್ತೂ ಒಂದು ಸಮಯ ಬರುತ್ತದೆ. ನೀವು ಇಂದು ಯಾವ ಯಾವ ವಸ್ತುಗಳನ್ನು ಬಿಟ್ಟುಬಿಡಲು ಇಷ್ಟೊಂದು ಹೋರಾಟ ಮಾಡುತ್ತಿದ್ದೀರೋ ಅವೆಲ್ಲವೂ ಹೇಸಿಗೆಯಾಗಿ, ಅಸಹ್ಯವಾಗಿ ಬಿಡುತ್ತವೆ.

ನಂತರ ಪ್ರಾಣಾಯಾಮದ ಅನೇಕ ವಿಧಾನಗಳಿವೆ. ಮೂರು ಭಾಗಗಳಿಂದ ಕೂಡಿದ ಒಂದು ವಿಧಾನವಿದೆ: (ಮೊದಲು) ಉಸಿರನ್ನು ತೆಗೆದುಕೊಳ್ಳುವುದು (ಪೂರಕ); (ನಂತರ) ಉಸಿರನ್ನು ಹಿಡಿದಿಟ್ಟಿರುವುದು ಅಂದರೆ ಉಸಿರಾಟವನ್ನೇ ನಿಲ್ಲಿಸಿ ಸ್ತಬ್ದರಾಗಿರುವುದು (ಕುಂಭಕ ) ಮತ್ತು ಉಸಿರನ್ನು ಹೊರಬಿಡುವುದು (ರೇಚಕ). ಕೆಲವು ಪ್ರಾಣಾಯಾಮದ ಅಭ್ಯಾಸಗಳು ಕಠಿಣವೇ ಸರಿ; ಮತ್ತು ಜಟಿಲವಾದ ಇನ್ನೂ ಕೆಲವು ವಿಧಾನಗಳಂತೂ ಯೋಗ್ಯವಾದ ಆಹಾರವನ್ನು ಸೇವಿಸದಿದ್ದರೆ ಬಹಳ ಅಪಾಯಗಳಿಂದ ಕೂಡಿರುತ್ತವೆ. ಅತ್ಯಂತ ಸರಳವಾದ (ಕೆಲವು ಪ್ರಾಣಾಯಾಮಗಳನ್ನು ) ಬಿಟ್ಟು (ಜಟಿಲವಾದ) ಇನ್ನಾವುದರ ಸಾಧನೆಯಲ್ಲಿ ತೊಡಗಲೂ ನಾನು ಖಂಡಿತ ಸಲಹೆ ಮಾಡುವುದಿಲ್ಲ.

ದೀರ್ಘವಾಗಿ ಉಸಿರೆಳೆದುಕೊಂಡು ನಿಮ್ಮ ಪುಪ್ಪುಸಗಳನ್ನು ತುಂಬಿಸಿ, ನಿಧಾನವಾಗಿ ಉಸಿರನ್ನು ಹೊರಹಾಕಿ. ಮೂಗಿನ ಒಂದು ಹೊಳ್ಳೆಯಿಂದ ಉಸಿರನ್ನು ತೆಗೆದುಕೊಂಡು ಪುಪ್ಪುಸಗಳನ್ನು ಭರ್ತಿಮಾಡಿ ಮತ್ತೊಂದು ಹೊಳ್ಳೆಯಿಂದ ಅದನ್ನು ಹೊರಹಾಕಿ. ನಮ್ಮಲ್ಲಿ ಕೆಲವರಂತೂ ಸಾಕಷ್ಟು ದೀರ್ಘವಾಗಿ ಉಸಿರಾಡುವುದೇ ಇಲ್ಲ. ಮತ್ತೆ ಕೆಲವರು ತಮ್ಮ ಶ್ವಾಸಕೋಶಗಳನ್ನು ಸಾಕಷ್ಟು ಪ್ರಮಾಣದಲ್ಲಿ (ಉಸಿರಿನಿಂದ) ತುಂಬಿಸಲಾರರು. ಈ ಶ್ವಾಸಪ್ರಶ್ವಾಸ ವಿಧಾನಗಳು ಆ (ನ್ಯೂನತೆಗಳನ್ನೆಲ್ಲ ) ಬಹಳಮಟ್ಟಿಗೆ ಸರಿಪಡಿಸುತ್ತವೆ. ಪ್ರಾತಃಕಾಲದಲ್ಲಿ ಅರ್ಧಘಂಟೆ ಸಾಯಂಕಾಲದಲ್ಲಿ ಅರ್ಧಘಂಟೆ, ಈ ರೀತಿಯ ಶ್ವಾಸ - ಪ್ರಶ್ವಾಸದ ಪ್ರಕ್ರಿಯೆಗಳು ನಿಮ್ಮನ್ನು ಬೇರೆಯ ವ್ಯಕ್ತಿಯನ್ನಾಗಿ ಮಾಡಿಬಿಡುತ್ತವೆ. ಈ ವಿಧವಾದ ಶ್ವಾಸ - ಪ್ರಶ್ವಾಸ ಪ್ರಕ್ರಿಯೆಗಳು ಎಂದಿಗೂ ಹಾನಿಕಾರಕವಲ್ಲ. ಬೇರೆ ಪ್ರಾಣಾಯಾಮದ ಸಾಧನೆಗಳನ್ನು ಬಹಳ ನಿಧಾನವಾಗಿ ಮಂದಗತಿಯಲ್ಲಿ ಮಾಡಬೇಕು. ಮೊದಲು ನಿಮ್ಮ ಶಕ್ತಿಯನ್ನು ತೂಗಿನೋಡಿ ಪರೀಕ್ಷಿಸಿ. ಹತ್ತು ನಿಮಿಷ ಹೆಚ್ಚೆನಿಸಿದರೆ ಐದೇ ನಿಮಿಷ ಸಾಕು.

ಯೋಗಿ ತನ್ನ ಶರೀರವನ್ನು ಸ್ವಸ್ಥವಾಗಿಟ್ಟಿರಬೇಕಾಗುತ್ತದೆ. ಪ್ರಾಣಾಯಾಮದ ವಿವಿಧ ವಿಧಾನಗಳು ಶರೀರದ ಅಂಗಾಂಗಗಳ (ಕ್ರಿಯೆ)ಗಳನ್ನು ನಿಯತಗೊಳಿಸಲು ಬಹಳ ಸಹಾಯಕಾರಿಯಾಗಿವೆ. ಶರೀರದ ವಿಭಿನ್ನ ಅಂಗಾಂಗಗಳು ಪ್ರಾಣ ವಾಯುವಿನಿಂದ (ಆವೃತ)ವಾಗಿವೆ. ಶ್ವಾಸದ ಮೂಲಕವೇ ಅವುಗಳೆಲ್ಲದರ ಮೇಲೂ ನಾವು ಹತೋಟಿ ಪಡೆಯಬಹುದು. ಶರೀರದ ಬೇರೆ ಬೇರೆ ಭಾಗಗಳಲ್ಲಿ ಉಂಟಾಗುವ ಏರು-ಪೇರುಗಳು ನಾಡೀ ಪ್ರವಾಹಗಳನ್ನು, ಹೆಚ್ಚು ಪ್ರಾಣವಾಯುಗಳ ತರಂಗಗಳನ್ನು, ಅವುಗಳೆಡೆಗೆ ಹರಿಯಬಿಡುವುದರಿಂದ ನಿಯಂತ್ರಣಕ್ಕೆ ಬರುತ್ತವೆ. ಶರೀರದಲ್ಲಿ ಎಲ್ಲಾದರೂ ನೋವಿದ್ದರೆ ಅದು ಪ್ರಾಣದ ಕೊರತೆಯಿಂದಾಗಲೀ ಅಥವಾ ಆಧಿಕ್ಯದಿಂದಾಗಲೀ ಉಂಟಾಗಿದೆಯೆಂದು ಹೇಳಲು ಯೋಗಿ ಸಮರ್ಥನಾಗಿರಬೇಕು, ಅದನ್ನು ಸಂತುಲನಗೊಳಿಸಬೇಕು.

ಯೋಗ ಸಾಧನೆಯಲ್ಲಿ (ಯಶಸ್ಸನ್ನು ಸಾಧಿಸಲು) ಮತ್ತೊಂದು ಕಟ್ಟುನಿಟ್ಟಾದ ನಿಯಮವೇ ಬ್ರಹ್ಮಚರ್ಯ. ಅದು ಎಲ್ಲ ಸಾಧನೆಗೂ ಆಧಾರಶೀಲವಾಗಿದೆ. ವಿವಾಹಿತರಾಗಿರಲಿ, ಅವಿವಾಹಿತರಾಗಿರಲಿ - ಕಟ್ಟುನಿಟ್ಟಾದ ಸಂಪೂರ್ಣ ಬ್ರಹ್ಮಚರ್ಯ ಪಾಲನೆಯಿರಬೇಕು. ಇದು ನಿಸ್ಸಂದೇಹವಾಗಿ ದೀರ್ಘವಾದ ವಿಷಯ. ಆದರೆ ನಿಮಗೆ ನಾನು ಹೇಳಬಯಸುವುದೇನೆಂದರೆ, ಈ ವಿಷಯವನ್ನು ಸಾರ್ವಜನಿಕವಾಗಿ ಚರ್ಚಿಸುವುದು ಈ ದೇಶದ ಜನರಿಗೆ ರುಚಿಸುವುದಿಲ್ಲ. ಈ ಪಾಶ್ಚಿಮಾತ್ಯ ರಾಷ್ಟ್ರಗಳು ಬ್ರಹ್ಮಚರ್ಯದ ಪಾಲನೆ ಸ್ತ್ರೀ-ಪುರುಷರಿಗೆ ಹಾನಿಕಾರಕವೆಂದು ಉಪದೇಶಿಸುವ ಬೋಧಕರ ರೂಪದಲ್ಲಿರುವ ಅಧಮಾಧಮ ನೀಚರಿಂದ ತುಂಬಿಹೋಗಿದೆ. ಈ (ವಿಚಿತ್ರವಾದ) ವಿಚಾರಗಳನ್ನು ಅವರು ಎಲ್ಲಿಂದ ಪಡೆದರು?... ಪ್ರತಿವರ್ಷವೂ ಸಾವಿರಾರು ಮಂದಿ ಈ ಒಂದು ಪ್ರಶ್ನೆಯೊಂದಿಗೆ ನನ್ನಲ್ಲಿಗೆ ಬರುತ್ತಾರೆ. ಯಾರೋ ಅವರಿಗೆ ಹೇಳಿದ್ದಾರಂತೆ, ಬ್ರಹ್ಮಚರ್ಯವನ್ನು ಪಾಲಿಸಿದರೆ ಅವರ ಶಾರೀರಿಕ ಸ್ವಾಸ್ಥ್ಯಕ್ಕೆ ಹಾನಿಯಾಗುತ್ತದಂತೆ. ಈ ಬೋಧಕರು ಇದನ್ನು ತಿಳಿದದ್ದಾದರೂ ಹೇಗೆ? ಅವರೆಂದಾದರೂ ಬ್ರಹ್ಮಚರ್ಯ ಪಾಲಿಸಿದ್ದರೆ? ಈ ಲಂಪಟರೂ, ತಿಳಿಗೇಡಿ ವ್ಯಭಿಚಾರಿಗಳೂ, ಕಾಮುಕ ಜೀವಿಗಳೂ ಇಡೀ ಪ್ರಪಂಚವನ್ನೇ ತಾವಿರುವ ಅಧೋಗತಿಗೆ ಎಳೆಯಬೇಕೆಂದಿದ್ದಾರೆ.

ತ್ಯಾಗವಿಲ್ಲದೆ ಏನನ್ನೂ ಸಾಧಿಸಲಾಗುವುದಿಲ್ಲ. ಮನುಷ್ಯ -ಚೇತನದ ಪವಿತ್ರತಮವಾದ ಸರ್ವೋತ್ಕೃಷ್ಟವಾದ ಸಾಧ್ಯತೆ, ಕ್ರಿಯಾಶಕ್ತಿ (ಎಂದರೆ ತ್ಯಾಗ. ) ಅದನ್ನು ಅಪವಿತ್ರಗೊಳಿಸಬೇಡಿ! ಮೃಗಗಳ ಮಟ್ಟಕ್ಕೆ ಅದನ್ನು ಎಳೆಯಬೇಡಿ! ಸಭ್ಯತೆಯುಳ್ಳ ಮನುಜರಾಗಿ! ಶೀಲವಂತರಾಗಿ ಮತ್ತು ಪರಿಶುದ್ದರಾಗಿ!... (ಇದನ್ನು ಬಿಟ್ಟು ) ಬೇರೆ ದಾರಿಯೇ ಇಲ್ಲ. ಏಸುಕ್ರಿಸ್ತ ಬೇರೆ ಯಾವ ಮಾರ್ಗವನ್ನೇನಾದರೂ ಕಂಡುಹಿಡಿದಿದ್ದನೆ?... ನೀವು ನಿಮ್ಮ ಶಕ್ತಿಯನ್ನು ಸಂಚಯಿಸಿ, ಆ ಶಕ್ತಿಯನ್ನು ಯಥೋಚಿತವಾಗಿ ಉಪಯೋಗಿಸಿದರೆ ಅದು ನಿಮ್ಮನ್ನು ದೇವರೆಡೆಗೆ ಒಯ್ಯುತ್ತದೆ. (ಇಲ್ಲ ಒಂದು ವೇಳೆ ) ಅದನ್ನು ಅಧೋಮುಖವಾಗಿಸಿದರೆ ಅದು ನರಕವೇ ಸರಿ.

ಬಾಹ್ಯಕ್ಷೇತ್ರದಲ್ಲಿ ಏನನ್ನಾದರೂ ಸಾಧಿಸಿ ತೋರಿಸುವುದು ಸುಲಭ. ಆದರೆ ಪ್ರಪಂಚದಲ್ಲೇ ಅತ್ಯಂತ ದೊಡ್ಡ ಜಯಶಾಲಿಯು ತನ್ನ ಮನಸ್ಸನ್ನೇ ನಿಯಂತ್ರಿಸಲು ಪ್ರಯತ್ನಿಸಿದಾಗ, ತಾನು ಚಿಕ್ಕ ಮಗುವಷ್ಟೇ ಎಂದು ಅವನಿಗನಿಸುತ್ತದೆ. ಅವನು (ನಿಜವಾಗಿ) ಗೆಲ್ಲಬೇಕಾದುದು (ಈ ಬಾಹ್ಯ ಪ್ರಪಂಚಕ್ಕಿಂತಲೂ) ದೊಡ್ಡದಾದ ಹಾಗೂ ಅದಕ್ಕಿಂತಲೂ ಕಠಿಣವಾದ ಈ (ಆಂತರಿಕ) ಪ್ರಪಂಚವನ್ನು. ಆದರೆ ಎದೆಗುಂದಬೇಡಿ. ಉತ್ತಿಷ್ಠತ! ಜಾಗ್ರತ! ಪ್ರಾಪ್ಯವರಾನ್ನಿ ಬೋಧತ!

