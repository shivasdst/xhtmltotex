
\chapter[ಧ್ಯಾನ]{ಧ್ಯಾನ\protect\footnote{\engfoot{C.W, Vol. IV, P. 227}}}

\begin{center}
(೧೯೦೦ರ ಏಪ್ರಿಲ್ ೩ರಂದು ಸ್ಯಾನ್‌ಫ್ರಾನ್ಸಿಸ್ಕೋದ ವಾಷಿಂಗ್ಟನ್ ಹಾಲಿನಲ್ಲಿ ನೀಡಿದ ಪ್ರವಚನ.)
\end{center}

ಎಲ್ಲಾ ಧರ್ಮಗಳೂ ಧ್ಯಾನಕ್ಕೆ ಪ್ರಾಮುಖ್ಯವನ್ನು ಕೊಡುತ್ತವೆ. ಧ್ಯಾನಾವಸ್ಥೆಯಲ್ಲಿರುವ ಮನಸ್ಸನ್ನೇ ಯೋಗಿಗಳು ಶ್ರೇಷ್ಠವಾದ ಅವಸ್ಥೆ ಎನ್ನುವರು. ಮನಸ್ಸು ಬಾಹ್ಯ ವಿಷಯಗಳನ್ನು ವಿಚಾರಿಸುತ್ತಿರುವಾಗ ಅದರಲ್ಲಿಯೇ ತನ್ಮಯವಾಗುವುದು. ಪುರಾತನ ಭಾರತೀಯ ದಾರ್ಶನಿಕರ ಉಪಮಾನವನ್ನು ಬಳಸುವುದಾದರೆ, ಮನುಷ್ಯನ ಮನಸ್ಸು ಒಂದು ಸ್ಫಟಿಕಶಿಲೆಯಂತೆ ಇದೆ. ಅದರ ಹತ್ತಿರ ಯಾವುದು ಇದೆಯೊ ಅದರ ಬಣ್ಣವನ್ನು ಅದು ತೆಗೆದುಕೊಳ್ಳುವುದು. ಮನಸ್ಸು ಯಾವುದನ್ನು ಸ್ಪರ್ಶಿಸುವುದೊ ಅದರ ಬಣ್ಣವನ್ನು ತಾಳಬೇಕಾಗಿದೆ. ಇದೇ ಬಂದಿರುವ ಕಷ್ಟ; ಇದೇ ಬಂಧನ. ಬಣ್ಣ ಎಷ್ಟು ಬಲವಾಗಿದೆ ಎಂದರೆ ಸ್ಫಟಿಕಶಿಲೆ ತನ್ನನ್ನು ಮರೆತು ಆ ಬಣ್ಣವನ್ನು ಹೋಲುವುದು. ಆ ಸ್ಫಟಿಕಶಿಲೆಯ ಹತ್ತಿರ ಒಂದು ಕೆಂಪು ಹೂವನ್ನು ಇಟ್ಟರೆ, ಸ್ಫಟಿಕ ಆ ಬಣ್ಣವನ್ನೇ ತೆಗೆದುಕೊಂಡು ತನ್ನನ್ನು ಮರೆತು ತಾನು ಕೆಂಪು ಎಂದು ಭಾವಿಸುವುದು. ನಾವು ದೇಹದ ಬಣ್ಣವನ್ನು ತೆಗೆದುಕೊಂಡು ನಮ್ಮ ನೈಜಸ್ವಭಾವವನ್ನು ಮರೆತಿರುವೆವು. ಅನಂತರ ಬರುವ ಕಷ್ಟಕ್ಕೆಲ್ಲ ಕಾರಣ ಈ ಅನಿಷ್ಟ ದೇಹವನ್ನು ನನ್ನದು ಎಂದು ಭಾವಿಸುವುದು. ನಮ್ಮ ಅ೦ಜಿಕೆ, ಚಿಂತೆ, ವ್ಯಾಕುಲ, ಕಷ್ಟ, ತಪ್ಪು, ದೌರ್ಬಲ್ಯ, ಪಾಪ ಮುಂತಾದುವುಗಳೆಲ್ಲಾ ನಾವು ದೇಹ ಎಂದು ಭಾವಿಸಿದ ತಪ್ಪಿನಿಂದ ಬರುವುದು. ಇದೇ ಸಾಧಾರಣ ಮನುಷ್ಯನ ಪಾಡು. ತನ್ನ ಹತ್ತಿರ ಇರುವ ಹೂವಿನ ಬಣ್ಣವನ್ನು ತಾಳುವ ಮನುಷ್ಯನ ಸ್ಥಿತಿ ಇದು. ಹೇಗೆ ಆ ಸ್ಫಟಿಕ ಕೆಂಪು ಹೂವು ಅಲ್ಲವೋ ಅದರಂತೆಯೇ ನಾವು ದೇಹವಲ್ಲ.

ಧ್ಯಾನವನ್ನು ಮುಂದುವರಿಸಬೇಕು. ಸ್ಫಟಿಕಕ್ಕೆ ಗೊತ್ತಿದೆ ತಾನು ಏನು ಎಂಬುದು. ಅದು ತನ್ನ ಬಣ್ಣವನ್ನು ತಾಳುವುದು. ಉಳಿದವುಗಳೆಲ್ಲಕ್ಕಿಂತ ಧ್ಯಾನವೊಂದೇ ನಮ್ಮನ್ನು ಸತ್ಯದ ಸಮೀಪಕ್ಕೆ ಒಯ್ಯುವುದು.

ಭಾರತ ದೇಶದಲ್ಲಿ ಇಬ್ಬರು ಸಂಧಿಸುವರು. ಇಂಗ್ಲಿಷ್‌ನಲ್ಲಿ "ನೀನು ಹೇಗಿರುವೆ?” ಎಂದು ಕೇಳುವರು. ದೇಶ ಭಾಷೆಯಲ್ಲಾದರೋ "ನೀನು ಸ್ವಸ್ಥನಾಗಿರುವೆಯಾ?” ಎನ್ನುವರು. ಜೀವನದಲ್ಲಿ ಯಾವಾಗ ನೀವು ಅನ್ಯವಸ್ತುವನ್ನು ಆಶ್ರಯಿಸಿರುವಿರೋ ಆಗ ದುಃಖ ಸಂಭವ ಬಹಳ. ನಾನು ಧ್ಯಾನವೆನ್ನುವುದು ಇದನ್ನೇ. ಆತ್ಮವು ತನ್ನಲ್ಲಿ ತಾನು ನೆಲೆ ನಿಲ್ಲಲು ಪ್ರಯತ್ನಿಸುವುದು. ಇದೇ ಆತ್ಮನ ಅತ್ಯಂತ ಆರೋಗ್ಯಕರ ಸ್ಥಿತಿ ಇರಬೇಕು. ಆಗ ಅದು ತನ್ನ ಮಹಿಮೆಯಲ್ಲಿ ತಾನು ನೆಲಸಿರುವುದು. ನಮ್ಮಲ್ಲಿರುವ ಇತರ ಮಾರ್ಗಗಳ – ಭಾವೋದ್ದೀಪನೆ, ಪ್ರಾರ್ಥನೆ, ಮುಂತಾದುವುಗಳ ಗುರಿಯೆಲ್ಲಾ ಇದೊಂದೇ. ಭಕ್ತಿಯು ಉತ್ಕಟವಾದಾಗ ಆತ್ಮವು ತನ್ನಲ್ಲಿ ತಾನು ನೆಲಸಲು ಪ್ರಯತ್ನಿಸುವುದು. ಬಾಹ್ಯವಸ್ತುಗಳಿಂದ ಭಾವವು ಉದ್ದೀಪನೆಯಾದರೂ ಮನಸ್ಸಿಗೆ ಏಕಾಗ್ರತೆ ಇರುವುದು.

ಧ್ಯಾನದಲ್ಲಿ ಮೂರು ಸ್ಥಿತಿಗಳಿವೆ. ಮೊದಲನೆಯದೆ ಧಾರಣ. ಮನಸ್ಸನ್ನು ಯಾವುದಾದರೂ ಒಂದು ವಸ್ತುವಿನ ಮೇಲೆ ನೆಲಸುವಂತೆ ಮಾಡುವುದು. ನಾನು ಈ ಒಂದು ಗ್ಲಾಸಿನ ಮೇಲೆ, ಉಳಿದ ವಸ್ತುಗಳನ್ನೆಲ್ಲಾ ಬಿಟ್ಟು, ಇದರ ಮೇಲೆ ಮಾತ್ರ ಮನಸನ್ನು, ಏಕಾಗ್ರಗೊಳಿಸುತ್ತೇನೆ. ಮನಸ್ಸು ಸಾಧಾರಣವಾಗಿ ಚಂಚಲವಾಗುತ್ತದೆ. ಮನಸ್ಸು ಏಕಾಗ್ರವಾಗಿ ಯಾವಾಗ ಹೆಚ್ಚು ಚಂಚಲವಾಗುವುದಿಲ್ಲವೋ ಆಗ ಅದು ಧ್ಯಾನ. ಇದಕ್ಕಿಂತಲೂ ಮೇಲಿನ ಸ್ಥಿತಿ ಇದೆ. ಅಲ್ಲಿ ನನಗೂ ಗ್ಲಾಸಿಗೂ ಇರುವ ವ್ಯತ್ಯಾಸ ಮಾಯವಾಗುವುದು. ಅದೇ ಸಮಾಧಿ. ಮನಸ್ಸು ಮತ್ತು ಗ್ಲಾಸು ಒಂದೇ ಆಗುವುವು. ನನಗೆ ಯಾವ ವ್ಯತ್ಯಾಸವೂ ಕಾಣುವುದಿಲ್ಲ. ಇಂದ್ರಿಯಗಳ ಕೆಲಸವೆಲ್ಲ ನಿಲ್ಲುವುವು. ಇತರ ಇಂದ್ರಿಯಗಳ ಮೂಲಕ ಹರಿಯುತ್ತಿದ್ದ ಶಕ್ತಿಯೆಲ್ಲಾ ಒಂದೆಡೆ ಕೇಂದ್ರೀಕೃತವಾಗುವುದು. ಆಗ ಗ್ಲಾಸು ಸಂಪೂರ್ಣವಾಗಿ ಮನಸ್ಸಿನ ಅಧೀನದಲ್ಲಿರುವುದು. ನಾವು ಇದನ್ನು ಪಡೆಯಬೇಕಾಗಿದೆ. ಯೋಗಿಗಳು ಆಡುವ ಒಂದು ಅದ್ಭುತವಾದ ಆಟ ಇದು. ಅವರು ಬಾಹ್ಯದಲ್ಲಿ ಒಂದು ಜಗತ್ತು ಇದೆ ಎಂದು ಭಾವಿಸುವರು. ಆದರೆ ನಿಜವಾಗಿ ನಮ್ಮಿಂದ ಹೊರಗೆ ಇರುವುದು ನಾವು ಏನನ್ನು ಕಾಣುತ್ತೇವೆಯೋ ಅದಲ್ಲ. ನಾನು ನೋಡುವ ಗ್ಲಾಸು ನಿಜವಾಗಿಯೂ ಬಾಹ್ಯ ವಸ್ತುವಲ್ಲ. ಹೊರಗೆ ಯಾವುದು ಗ್ಲಾಸು ಆಗಿದೆಯೋ ಅದು ನನಗೆ ಗೊತ್ತಿಲ್ಲ. ಮತ್ತು ಗೊತ್ತಾಗುವಂತೆಯೂ ಇಲ್ಲ.

ಯಾವುದೋ ಒಂದು ವಸ್ತು ನನ್ನ ಮನಸ್ಸಿನ ಮೇಲೆ ಒಂದು ಮುದ್ರೆಯನ್ನು ಒತ್ತುತ್ತದೆ. ನಾನು ತಕ್ಷಣ ಆ ಕಡೆಗೆ ಗಮನವನ್ನು ಕೊಡುತ್ತೇನೆ. ಆ ಪ್ರತಿಕ್ರಿಯೆ ಮತ್ತು ಹೊರಗೆ ಇರುವ ವಸ್ತು ಇವೆರಡೂ ಸೇರಿ ಗ್ಲಾಸು ಆಗಿದೆ. ಹೊರಗಿನ ಕ್ರಿಯೆ \enginline{x}. ಒಳಗಿನ ಕ್ರಿಯೆ \enginline{y}. ಗ್ಲಾಸು \enginline{xy}. ನೀವು \enginline{x} ಅನ್ನು ನೋಡಿದಾಗ ಬಾಹ್ಯ ಪ್ರಪಂಚ ಎನ್ನುವಿರಿ. \enginline{y} ಅನ್ನು ಆಂತರಿಕ ಪ್ರಪಂಚ ಎನ್ನುತ್ತೀರಿ. ನಿಮ್ಮ ಮನಸ್ಸು ಯಾವುದು, ಬಾಹ್ಯ ಪ್ರಪಂಚ ಯಾವುದು ಎಂದು ಪ್ರತ್ಯೇಕಪಡಿಸಲು ಯತ್ನಿಸಿದರೆ ಅದು ಸಾಧ್ಯವಾಗುವುದಿಲ್ಲ. ಪ್ರಪಂಚ ಆಗಿರುವುದು ನೀವು ಮತ್ತು ಮತ್ತಾವುದೋ ಒಂದು ಮಿಶ್ರದಿಂದ.

ನಾವು ಮತ್ತೊಂದು ಉದಾಹರಣೆಯನ್ನು ತೆಗೆದುಕೊಳ್ಳೋಣ. ನೀವು ಪ್ರಶಾಂತವಾದ ಒಂದು ಸರೋವರದ ಮೇಲೆ ಕಲ್ಲುಗಳನ್ನು ಎಸೆಯುತ್ತಿರುವಿರಿ ಎಂದು ಭಾವಿಸೋಣ. ನೀವು ಎಸೆದ ಪ್ರತಿಯೊಂದು ಕಲ್ಲಿನಿಂದಲೂ ಒಂದು ಪ್ರತಿಕ್ರಿಯೆ ಉಂಟಾಗುವುದು. ಎಸೆದ ಕಲ್ಲನ್ನು ಅಲೆಗಳು ಮುತ್ತುವುವು. ಇದರಂತೆಯೇ ಬಾಹ್ಯ ಘಟನೆಗಳು ಮನಸ್ಸೆಂಬ ಸರೋವರಕ್ಕೆ ಎಸೆದ ಕಲ್ಲುಗಳಂತೆ. ನಾವು ನಿಜವಾಗಿಯೂ ಬಾಹ್ಯ ಪ್ರಪಂಚವನ್ನು ನೋಡುವುದಿಲ್ಲ. ನಾವು ಕೇವಲ ಅಲೆಗಳನ್ನು ಮಾತ್ರ ನೋಡುತ್ತೇವೆ.

ಮನಸ್ಸಿನಲ್ಲಿ ಏಳುವ ಅಲೆಗಳು ಎಷ್ಟೋ ವಸ್ತುಗಳನ್ನು ಸೃಷ್ಟಿ ಮಾಡಿವೆ. ನಾವಿಲ್ಲಿ ಭಾವಸತ್ತಾವಾದ, ವಸ್ತುಸತ್ತಾವಾದ ಇವನ್ನು ವಿಮರ್ಶಿಸುತ್ತಿಲ್ಲ. ವಸ್ತುಗಳು ಬಾಹ್ಯದಲ್ಲಿವೆ ಎಂದು ನಾವು ಭಾವಿಸುತ್ತೇವೆ. ಆದರೆ ನಾವು ನೋಡುವುದು ಬೇರೆ, ನಿಜವಾಗಿ ಇರುವ ವಸ್ತುಗಳು ಬೇರೆ. ಏಕೆಂದರೆ ನಮಗೆ ಬಾಹ್ಯ ಪ್ರಪಂಚದಂತೆ ಕಾಣುವುದರಲ್ಲೆಲ್ಲಾ ನಾವು ಮಿಶ್ರವಾಗಿ ಹೋಗಿರುತ್ತೇವೆ.

ನಾನು ಗ್ಲಾಸಿಗೆ ಕೊಟ್ಟಿರುವುದನ್ನು ತೆಗೆದುಕೊಂಡು ಬಿಟ್ಟೆ ಎಂದು ಭಾವಿಸೋಣ. ಅನಂತರ ಉಳಿಯುವುದೇನು? ಏನೂ ಇಲ್ಲ. ನಾನು ಮೇಜಿಗೆ ಏನನ್ನು ಕೊಟ್ಟಿದ್ದೇನೆಯೋ ಅದನ್ನು ತೆಗೆದುಕೊಂಡು ಬಿಟ್ಟರೆ ಉಳಿಯುವುದೇನು? ಏನೂ ಇಲ್ಲ. ಏಕೆಂದರೆ ಅದು ಹೊರಗಿನ ವಸ್ತು ಮತ್ತು ನಾನು ಕೊಟ್ಟದ್ದು ಇವೆರಡರ ಮಿಶ್ರ. ಬಡ ಸರೋವರವು ಅಲೆಗಳನ್ನು ಕಲ್ಲಿನ ಕಡೆ, ಅದು ಬಿದ್ದಾಗಲೆಲ್ಲ ಎಸೆಯಲೇ ಬೇಕು. ಮನಸ್ಸು ಯಾವುದೇ ಸಂವೇದನೆ ಬಂದಾಗಲೂ ಪ್ರತಿಕ್ರಿಯೆಯನ್ನು ತೋರಿಸಿ ಅಲೆಗಳನ್ನು ಸೃಷ್ಟಿಸಲೇಬೇಕು. ನಾವು ಮನಸ್ಸನ್ನು ನಿಗ್ರಹಿಸುತ್ತೇವೆ ಎಂದು ಭಾವಿಸೋಣ. ಆಗ ನಾವೇ ಒಡೆಯರಾಗುತ್ತೇವೆ. ಈ ಎಲ್ಲ ಘಟನೆಗಳಿಗೂ ಆಗ ನಮ್ಮ ಪಾಲನ್ನು ಕೊಡಲು ನಿರಾಕರಿಸುತ್ತೇವೆ. ನಾನು ನನ್ನ ಪಾಲನ್ನು ಕೊಡದೆ ಹೋದರೆ ಘಟನೆಗಳೇ ಇರುವುದಿಲ್ಲ.

ಈ ಬಂಧನವನ್ನು ನೀವು ಸದಾ ಸೃಷ್ಟಿಸುತ್ತಿದ್ದೀರಿ. ಹೇಗೆ? ನಿಮ್ಮ ಪಾಲನ್ನು ನೀಡುವುದರ ಮೂಲಕ. ನಮ್ಮ ಹಾಸುಗೆಗಳನ್ನು ನಾವೇ ತಯಾರಿಸುತ್ತಿದ್ದೇವೆ. ನಮ್ಮ ಸಂಕೋಲೆಗಳನ್ನೂ ಅಷ್ಟೆ. ಬಾಹ್ಯವಸ್ತುವಿಗೂ ನಮಗೂ ಇರುವ ತಾದಾತ್ಮ್ಯವು ನಿಂತಾಗ ನಮ್ಮ ಪಾಲಿನ ಕೊಡುಗೆಯನ್ನು ನಿಲ್ಲಿಸುವುದು ಸಾಧ್ಯವಾಗುತ್ತದೆ. ಆಗ ವಸ್ತುವೂ ಮಾಯವಾಗುತ್ತದೆ. ಆಗ ನಾನು ಗ್ಲಾಸು ಇಲ್ಲಿದೆ ಎಂದು ಹೇಳಿ ನನ್ನ ಮನಸ್ಸನ್ನು ಅದರಿಂದ ಹಿಂತೆಗೆದುಕೊಳ್ಳುತ್ತೇನೆ. ಆಗ ಗ್ಲಾಸು ಮಾಯವಾಗುತ್ತದೆ. ನಿಮ್ಮ ಪಾಲನ್ನು ನೀವು ಹಿಂತೆಗೆದುಕೊಳ್ಳಬಲ್ಲಿರಾದರೆ ನೀವು ನೀರಿನ ಮೇಲೆ ನಡೆಯಬಹುದು. ಅದು ಹೇಗೆ ನಿಮ್ಮನ್ನು ಮುಳುಗಿಸಬಲ್ಲುದು? ವಿಷದ ವಿಷಯ ಇನ್ನೇನು? ಅದು ತಾನೆ ಏನು ಮಾಡಬಲ್ಲದು? ಇನ್ನೇನೂ ಕಷ್ಟಗಳಿಲ್ಲ. ಪ್ರತಿಯೊಂದು ಘಟನೆಯಲ್ಲೂ ನಿಮ್ಮ ಪಾಲು ಅರ್ಧವಾದರೆ ಪ್ರಕೃತಿಯ ಪಾಲು ಇನ್ನರ್ಧ ನಿಮ್ಮ ಪಾಲನ್ನು ತೆಗೆದುಕೊಂಡರೆ ಘಟನೆ ನಿಲ್ಲಬೇಕು.

ಪ್ರತಿಯೊಂದು ಕ್ರಿಯೆಗೂ ಸಮಸಮನಾದ ಪ್ರತಿಕ್ರಿಯೆ ಇದೆ. ಯಾರಾದರೂ ನನ್ನನ್ನು ಹೊಡೆದು ಗಾಯಗೊಳಿಸಿದರೆ ಅದು ಆತನ ಕ್ರಿಯೆ ಮತ್ತು ನನ್ನ ದೇಹದ ಪ್ರತಿಕ್ರಿಯೆಯ ಫಲ. ತಾನೇ ತಾನಾಗಿ ನಡೆಯುವ ಆ ಕ್ರಿಯೆಯನ್ನು ನಿಗ್ರಹಿಸುವಷ್ಟು ನನ್ನ ದೇಹದ ಮೇಲೆ ನನಗೆ ನಿಯಂತ್ರಣವಿದೆ ಎಂದಿಟ್ಟುಕೊಳ್ಳಿ. ಇಂತಹ ಶಕ್ತಿಯನ್ನು ನಾವು ಪಡೆಯಲು ಸಾಧ್ಯವೆ? ಶಾಸ್ತ್ರಗಳು ಇವು ಸಾಧ್ಯ ಎನ್ನುತ್ತವೆ. ಅವುಗಳನ್ನು ನೀವು ದಿಢೀರನೆ ಪಡೆದರೆ ಅವು ಪವಾಡಗಳು, ಅವನ್ನು ವಿಧಿವತ್ತಾಗಿ ಕಲಿತುಕೊಂಡರೆ ಅದು ಯೋಗ.

ಮನಸ್ಸಿನ ಶಕ್ತಿಯಿಂದ ಖಾಯಿಲೆ ವಾಸಿಮಾಡುವವರನ್ನು ನಾನು ನೋಡಿರುವೆನು. ಇಂತಹ ಮಂತ್ರವಾದಿ ಇರುವನು. ಅವನು ಯಾರಿಗೊ ಪ್ರಾರ್ಥನೆ ಮಾಡುತ್ತಾನೆ ಎನ್ನುತ್ತೇವೆ. ರೋಗಿ ಗುಣವಾಗುತ್ತಾನೆ. ಮತ್ತೊಬ್ಬ ಅದು ಅವನ ಪ್ರಾರ್ಥನೆಯಿಂದ ಆದುದಲ್ಲ, ಅದು ಅವನ ಮನೋಶಕ್ತಿಯಿಂದ ಆದುದು ಎನ್ನುತ್ತಾನೆ. ಅವನು ವಿಜ್ಞಾನಿ. ತಾನು ಏನು ಮಾಡುತ್ತಿರುವೆ ಎಂಬುದು ಅವನಿಗೆ ಗೊತ್ತಿದೆ.

\newpage

ಧ್ಯಾನಶಕ್ತಿ ನಮಗೆ ಏನನ್ನು ಬೇಕಾದರೂ ಕೊಡಬಲ್ಲದು. ನಿಮಗೆ ಪ್ರಕೃತಿಯ ಮೇಲೆ ಸ್ವಾಧೀನ ಬೇಕಾದರೆ ಧ್ಯಾನದಿಂದ ಇದನ್ನು ಪಡೆಯಬಹುದು. ಧ್ಯಾನ ಶಕ್ತಿಯಿಂದಲೇ ಇಂದಿನ ವೈಜ್ಞಾನಿಕ ವಿಷಯಗಳನ್ನೆಲ್ಲಾ ಕಂಡುಹಿಡಿದಿರುವುದು. ಅವರು ವಿಷಯವನ್ನು ಅಧ್ಯಯನ ಮಾಡುವರು. ಬೇರೆ ಎಲ್ಲವನ್ನೂ, ತಮ್ಮ ದೇಹವನ್ನೂ ಮರೆಯುತ್ತಾರೆ. ಅನಂತರ ಸತ್ಯ ಮಿಂಚಿನಂತೆ ಹೊಳೆಯುತ್ತದೆ. ಕೆಲವರು ಅದನ್ನು ಸ್ಫೂರ್ತಿ ಎನ್ನುತ್ತಾರೆ. ಅದು ಬರೀ ಸ್ಫೂರ್ತಿ ಮಾತ್ರವಲ್ಲ, ಅದರ ಹಿಂದೆ ಶ್ರಮವೂ ಇತ್ತು. ಈ ಪ್ರಪಂಚದಲ್ಲಿ ಬೆಲೆಯಿಲ್ಲದೆ ಯಾವುದೂ ಎಂದಿಗೂ ದೊರಕಿಲ್ಲ.

ಏಸುವಿನದೇ ಅತ್ಯಂತ ಶ್ರೇಷ್ಠವಾದ ಸ್ಫೂರ್ತಿ. ಅವನು ಹಿಂದಿನ ಜನ್ಮಗಳಲ್ಲಿ ಬೇಕಾದಷ್ಟು ಪ್ರಯತ್ನ ಪಟ್ಟಿದ್ದನು. ಅವನು ಕೊನೆಯ ಜನ್ಮದಲ್ಲಿ ಸಾಧಿಸಿದ್ದು ಹಿಂದಿನ ಜನ್ಮಗಳ ಪರಿಪಾಕ, ಕಷ್ಟವಾದ ಸಾಧನೆಯ ಫಲ. ಸುಮ್ಮನೆ ಸ್ಫೂರ್ತಿ ಎಂದು ಹೇಳಿದರೆ ಅದಕ್ಕೆ ಅರ್ಥವಿಲ್ಲ. ಅದೇನಾದರೂ ಹಾಗೆ ಇದ್ದಿದ್ದರೆ, ಅದು ಮಳೆಯಂತೆ ಬೀಳುತ್ತಿತ್ತು. ಜೀವನದ ಯಾವ ಕಾರ್ಯಕ್ಷೇತ್ರದಲ್ಲಾದರೂ ಆಗಲಿ, ಸ್ಫೂರ್ತಿಯನ್ನು ಪಡೆದ ವ್ಯಕ್ತಿಗಳು, ಯಾವ ಜನಾಂಗದಲ್ಲಿ ವಿದ್ಯೆ ಮತ್ತು ಸಂಸ್ಕೃತಿ ಇವೆಯೊ ಅಂತಹ ಕಡೆಗಳಿಂದ ಬರುತ್ತಾರೆ. ಸ್ಫೂರ್ತಿ ಎಂಬುದಿಲ್ಲ. ಯಾವುದನ್ನು ಸ್ಫೂರ್ತಿ ಎನ್ನುತ್ತೇವೆಯೋ ಅದು ಈಗಾಗಲೇ ಮನಸ್ಸಿನಲ್ಲಿರುವ ಕಾರಣಗಳ ಫಲಿತಾಂಶ. ಅದು ಮಿಂಚಿನಂತೆ ಹೊಳೆಯುವುದು. ಅದಕ್ಕೆ ಕಾರಣ ಹಿಂದಿನ ಸಾಧನೆ.

ಅಲ್ಲಿಯೂ ಕೂಡ ಧ್ಯಾನದ ಶಕ್ತಿಯನ್ನೇ ನೋಡುತ್ತೀರಿ. ಅದೇ ಭಾವನೆಯ ತೀವ್ರತೆ. ಅವರು ತಮ್ಮ ಅಂತರಂಗವನ್ನೆಲ್ಲ ಮಥಿಸುವರು. ಆಗ ಮಹಾಸತ್ಯಗಳು ಮೇಲಕ್ಕೆ ತೇಲಿ ವ್ಯಕ್ತವಾಗುವುವು. ಧ್ಯಾನಾಭ್ಯಾಸವೇ ಜ್ಞಾನವನ್ನು ಪಡೆಯಲು ಇರುವ ಶ್ರೇಷ್ಠ ಸಾಧನೆ. ಧ್ಯಾನದ ಶಕ್ತಿಯಿಲ್ಲದೆ ಯಾವ ಜ್ಞಾನವೂ ಇಲ್ಲ. ಅಜ್ಞಾನ ಮತ್ತು ಮೂಢನಂಬಿಕೆಯ ಫಲವಾಗಿ ಧ್ಯಾನದಿಂದ ನಮಗೆ ವಾಸಿಯಾಗಬಹುದು. ಆದರೆ ಅದು ತಾತ್ಕಾಲಿಕ. ಯಾರೋ ಒಬ್ಬರು ನನಗೆ ಇಂತಹ ವಿಷವನ್ನು ಕುಡಿದರೆ ನೀನು ಸಾಯುವೆ ಎಂದು ಹೇಳಿರುವರು. ಮತ್ತೊಬ್ಬ ರಾತ್ರಿ ಬಂದು "ಹೋಗು, ಕುಡಿ ವಿಷವನ್ನು!'' ಎಂದು ಹೇಳುತ್ತಾನೆ. ಕುಡಿದರೂ ನಾನು ಸಾಯುವುದಿಲ್ಲ. ಆಗ ಏನಾಗುವುದು! ನನ್ನ ಮನಸ್ಸು ಧ್ಯಾನದ ಸಹಾಯದಿಂದ ತತ್ಕಾಲಕ್ಕೆ ನನಗೂ ವಿಷಕ್ಕೂ ಇರುವ ಸಂಬಂಧವನ್ನು ಛೇದಿಸಿತು. ನಾನು ಬೇರೆ ವೇಳೆ ಹಾಗೆ ವಿಷವನ್ನು ಕುಡಿದರೆ ಸಾಯುವೆ.

ನನಗೆ ಕಾರಣ ಗೊತ್ತಿದ್ದರೆ, ನಾನು ಸರಿಯಾದ ಮಾರ್ಗದ ಮೂಲಕ ಧ್ಯಾನಾವಸ್ಥೆಗೆ ಏರಿದ್ದರೆ, ನಾನು ಯಾರನ್ನು ಬೇಕಾದರೂ ರಕ್ಷಿಸಬಹುದು. ಶಾಸ್ತ್ರಗಳು ಇವನ್ನು ಹೇಳುವುವು. ಆದರೆ ಅವು ಎಷ್ಟು ಮಟ್ಟಿಗೆ ನಿಜವೋ ಅದನ್ನು ನೀವು ಪರೀಕ್ಷೆ ಮಾಡಬೇಕಾಗಿದೆ.

ಜನ ನನ್ನನ್ನು ಹೀಗೆ ಪ್ರಶ್ನಿಸುತ್ತಾರೆ: 'ಭಾರತ ದೇಶದವರು ಏತಕ್ಕೆ ಇವನ್ನು ಗೆಲ್ಲಬಾರದು? ನೀವು ಯಾವಾಗಲೂ ಎಲ್ಲರಿಗಿಂತ ಮೇಲು ಎಂದು ಹೆಮ್ಮೆ ಕೊಚ್ಚಿಕೊಳ್ಳುತ್ತಿರುವಿರಿ. ನೀವು ಯೋಗಾಭ್ಯಾಸವನ್ನು ಮಾಡಿ ಇತರರಿಗಿಂತ ಬೇಗ ಇವುಗಳನ್ನು ಮಾಡಬಹುದು. ನೀವು ಇದಕ್ಕೆ ಯೋಗ್ಯರು. ಇದನ್ನು ನೀವು ಮಾಡಿ. ನೀವು ಶ್ರೇಷ್ಠ\break ಜನಾಂಗವಾದರೆ ನಿಮಗೆ ಶ್ರೇಷ್ಠ ಸಿದ್ದಾಂತ ಗೊತ್ತಿರಬೇಕು. ನೀವು ಎಲ್ಲಾ ದೇವರುಗಳನ್ನೂ ಬಿಡಬೇಕು. ನೀವು ದೊಡ್ಡ ದೊಡ್ಡ ತತ್ತ್ವಗಳನ್ನು ತೆಗೆದುಕೊಳ್ಳಿ. ದೇವರು ಬೇಕಾದರೆ ನಿದ್ರೆಗೆ ಹೋಗಲಿ. ಆದರೆ ನೀವೂ ಕೂಡ ಇತರರಂತೆಯೇ ಮೂಢರು. ಕೇವಲ ಶಿಶುಗಳೋಪಾದಿಯಲ್ಲಿ ಇರುವಿರಿ. ನೀವು ಏನನ್ನು ಹೆಮ್ಮೆ ಕೊಚ್ಚಿಕೊಳ್ಳುವಿರೋ ಅವೆಲ್ಲಾ ನಿಷ್ಪ್ರಯೋಜನ, ನೀವು ಹೇಳುವುದು ನಿಜವಾದರೆ ಮೇಲೆದ್ದು ನಿಲ್ಲಿ, ಧೀರರಾಗಿ. ಈ ಪ್ರಪಂಚದಲ್ಲಿ ಇರಬಹುದಾದ ಸ್ವರ್ಗಗಳೆಲ್ಲ ನಿಮ್ಮವು. ನಾಭಿಯಲ್ಲಿ ಗಂಧವಿರುವ ಕಸ್ತೂರಿ ಮೃಗವಿದೆ. ಅದಕ್ಕೆ ಎಲ್ಲಿಂದ ವಾಸನೆ ಬರುವುದು ಎಂಬುದು ಗೊತ್ತಿಲ್ಲ. ಹಲವು ದಿನಗಳಾದ ಮೇಲೆ ಅದಕ್ಕೆ ಇದು ಗೊತ್ತಾಗುವುದು. ದೇವದಾನವರೆಲ್ಲ ಅವರಲ್ಲಿ ಇರುವರು. ವಿಚಾರ, ಶಿಕ್ಷಣ ಮತ್ತು ಅಭ್ಯಾಸದಿಂದ ನೀವು ಇದನ್ನು ಕಂಡುಹಿಡಿಯಬೇಕಾಗಿದೆ. ಇನ್ನು ಮೇಲೆ ಯಾವ ದೇವರುಗಳಾಗಲಿ ಅಥವಾ ಮೌಡ್ಯವಾಗಲಿ ಇಲ್ಲ. ನೀವು ಯುಕ್ತಿವಂತರಾಗಲಿಚ್ಚಿಸುವಿರಿ. ಯೋಗಿಗಳಾಗಬೇಕು, ನಿಜವಾಗಿ ಆಧ್ಯಾತ್ಮಿಕ ವ್ಯಕ್ತಿಗಳಾಗಬೇಕು ಎಂಬುದನ್ನು ನೀವು ಆಶಿಸುವಿರಿ.”

ಅದಕ್ಕೆ ನನ್ನ ಉತ್ತರವೇ ಇದು: ನಿಮ್ಮಲ್ಲಿ ಕೂಡ ಎಲ್ಲವೂ ಭೌತಿಕವಾದುದೇ. ಸಿಂಹಾಸನದ ಮೇಲೆ ಕುಳಿತಿರುವ ದೇವರಿಗಿಂತ ಹೆಚ್ಚು ಯಾವುದು ಭೌತಿಕವಾಗಿದೆ? ಯಾರೊ ವಿಗ್ರಹಾರಾಧನೆ ಮಾಡುತ್ತಿದ್ದರೆ ಅವನೇನೋ ಹೀನಾಯವಾದುದನ್ನು ಮಾಡುತ್ತಿರುವನೆಂದು ನಿಕೃಷ್ಟ ದೃಷ್ಟಿಯಿಂದ ನೋಡುತ್ತೀರಿ. ನೀವೇನು ಅದಕ್ಕಿಂತ ಮೇಲಲ್ಲ. ನೀವು ಸ್ವರ್ಣಾರಾಧಕರಲ್ಲದೆ ಮತ್ತೇನು? ವಿಗ್ರಹಾರಾಧಕನು ದೇವತೆಯನ್ನು, ತಾನು ನೋಡುವ ಯಾವುದೋ ಒಂದನ್ನು ಪೂಜಿಸುತ್ತಾನೆ. ನೀವು ಅದನ್ನು ಕೂಡ ಮಾಡುವುದಿಲ್ಲ. ನೀವು ಅಧ್ಯಾತ್ಮವನ್ನೂ ಪೂಜಿಸುವುದಿಲ್ಲ, ಅಥವಾ ನೀವು ಯಾವುದನ್ನು ಅರ್ಥಮಾಡಿಕೊಳ್ಳಬಲ್ಲಿರೊ ಅದನ್ನೂ ಪೂಜಿಸುವುದಿಲ್ಲ. ನೀವು ಬರೀ ಮಾತಿನ ಪೂಜಕರು. 'ದೇವರು ಚೈತನ್ಯ', ನಾವು ಅವನನ್ನು ಹಾಗೇಯೇ ಪೂಜಿಸಬೇಕು. ಆತ್ಮ ಎಲ್ಲಿರುವುದು? ಮರದ ಮೇಲೆಯೇ? ಮುಗಿಲ ಮೇಲೆಯೇ? ದೇವರು ನಮ್ಮವನು ಎಂದರೆ ಏನು? ನೀವು ಆತ್ಮ, ನೀವು ಈ ಮುಖ್ಯ ಭಾವನೆಯನ್ನು ಎಂದಿಗೂ ತ್ಯಜಿಸಕೂಡದು. ನಾನು ಆಧ್ಯಾತ್ಮಿಕ ವ್ಯಕ್ತಿ. ಅದು ಅಲ್ಲಿದೆ. ಯೋಗ ಮತ್ತು ಧ್ಯಾನ ಇವುಗಳೆಲ್ಲ ಅವನನ್ನು ಅಲ್ಲಿ ನೋಡುವುದಕ್ಕಾಗಿ.

ನಾನು ಇವುಗಳನ್ನೆಲ್ಲಾ ಏತಕ್ಕೆ ಇಲ್ಲಿ ಹೇಳುತ್ತಿರುವುದು? ದೇವರನ್ನು ಒಂದು ನಿರ್ದಿಷ್ಟ ಸ್ಥಾನದಲ್ಲಿ ಸ್ಥಾಪಿಸುವವರೆಗೆ ನೀವು ಮಾತನಾಡಲಾರಿರಿ. ಅವನು ಸ್ವರ್ಗದಲ್ಲಿರುವನು, ಮತ್ತು ಇನ್ನು ಎಲ್ಲೆಲ್ಲೋ ಇರುವನು ಎಂದು ಹೇಳುತ್ತೀರಿ. ಆದರೆ ಅವನು ನಿಜವಾಗಿ ಎಲ್ಲಿರುವನೋ ಅದೊಂದನ್ನು ಮಾತ್ರ ಹೇಳುವುದಿಲ್ಲ. ನಾನು ಆತ್ಮ, ಎಲ್ಲರ ಆತ್ಮನ ಆತ್ಮನಾಗಿರುವವನು ನನ್ನಲ್ಲಿರಬೇಕು. ಯಾರು ಅದನ್ನು ಬೇರೆ ಕಡೆ ಭಾವಿಸುವರೋ ಅವರು ಅಜ್ಞರು. ಆದಕಾರಣ ಅದನ್ನು ಇಲ್ಲಿ ಅಂತರಂಗದ ಸ್ವರ್ಗದಲ್ಲಿ ಕಂಡುಹಿಡಿಯಬೇಕಾಗಿದೆ. ಎಲ್ಲಾ ಸ್ವರ್ಗಗಳೂ ನಿಮ್ಮಲ್ಲಿವೆ. ಇದನ್ನು ಅರಿತ ಕೆಲವು ಋಷಿಗಳು ತಮ್ಮ ಕಣ್ಣನ್ನು ಅಂತರ್ಮುಖಮಾಡಿ, ತಮ್ಮಲ್ಲಿ ಪರಮಾತ್ಮ ವಸ್ತುವನ್ನು ಕಾಣುವರು. ಇದೇ ಧ್ಯಾನದ ಮಾರ್ಗ. ದೇವರು ಮತ್ತು ನಿಮ್ಮ ಆತ್ಮನ ಸತ್ಯವನ್ನು ಕಂಡು ಹಿಡಿಯಿರಿ. ಇದರಿಂದ ಮುಕ್ತಿ ಲಭಿಸುವುದು.

ನೀವೆಲ್ಲಾ ಜೀವನವನ್ನು ಅರಸಿಕೊಂಡು ಹೋಗುತ್ತಿರುವಿರಿ. ಇದು ಮೌಢ್ಯ\break ಎಂಬುದು ಅನಂತರ ಗೊತ್ತಾಗುವುದು. ಜೀವನಕ್ಕಿಂತ ಶ್ರೇಷ್ಠವಾಗಿರುವುದು ಮತ್ತೊಂದು ಇರುವುದು. ಈ ಭೌತಿಕ ಜೀವನ ಕೀಳುದರ್ಜೆಗೆ ಸೇರಿದ್ದು. ನಾನು ಏತಕ್ಕೆ ಜೀವಿಸಿರಬೇಕು? ನಾನು ಜೀವನಕ್ಕಿಂತ ಮೇಲಾದವನು, ಬಾಳುವುದು ಯಾವಾಗಲೂ ಒಂದು ದಾಸ್ಯ, ನಾವು ಯಾವಾಗಲೂ ಬೆರೆತುಹೋಗುವೆವು. ಪ್ರತಿಯೊಂದೂ ಅಂತ್ಯವಿಲ್ಲದ ದಾಸ್ಯ.

ನಿಮಗೆ ಏನೋ ದೊರಕುವುದು–ಯಾರೂ ಮತ್ತೊಬ್ಬರಿಗೆ ಕಲಿಸಲಾರರು. ಅನುಭವದ ಮೂಲಕ ನಾವು ಕಲಿತುಕೊಳ್ಳುತ್ತೇವೆ. ಯುವಕನಿಗೆ ಜೀವನದಲ್ಲಿ ಕಷ್ಟವಿದೆ ಎಂದು ಒಪ್ಪಿಕೊಳ್ಳುವಂತೆ ಮಾಡಲಾಗುವುದಿಲ್ಲ. ವೃದ್ಧನಿಗೆ ಜೀವನವೆಲ್ಲಾ ಸುಖಮಯ ಎಂದು ಹೇಳಲಾಗುವುದಿಲ್ಲ. ಅವನಿಗೆ ಎಷ್ಟೋ ಅನುಭವಗಳಾಗಿವೆ. ಅದೇ ವ್ಯತ್ಯಾಸ.

ಧ್ಯಾನದ ಮೂಲಕ ಕ್ರಮೇಣ ಇವುಗಳನ್ನೆಲ್ಲಾ ನಾವು ನಿಗ್ರಹಿಸಬೇಕಾಗಿದೆ. ತತ್ತವಶಃ ಮನಸ್ಸು ಪಂಚಭೂತಗಳು ಇವುಗಳ ವ್ಯತ್ಯಾಸಗಳು ನಿಜವಾಗಿ ಇಲ್ಲ. ಇರುವುದೆಲ್ಲಾ ಒಂದೇ ಎಂಬುದು ನಮಗೆ ಗೊತ್ತಿದೆ. ಹಲವು ಇರಲಾರವು. ಜ್ಞಾನ ಮತ್ತು ವಿಜ್ಞಾನ ಎಂದರೆ ಇದೇ. ಅಜ್ಞಾನ ಹಲವನ್ನು ನೋಡುವುದು, ಜ್ಞಾನ ಒಂದನ್ನು ಸಾಕ್ಷಾತ್ಕಾರ ಮಾಡಿಕೊಳ್ಳುವುದು. ಹಲವನ್ನು ಏಕಕ್ಕೆ ತರುವುದೇ ವಿಜ್ಞಾನ, ಇಡೀ ಜಗತ್ತು ಒಂದು ಎಂದು ತೋರಿಸಲಾಗಿದೆ. ಅದನ್ನೇ ವೇದಾಂತ ಶಾಸ್ತ್ರ ಎನ್ನುವುದು. ಈ ಪ್ರಪಂಚವೆಲ್ಲಾ ಒಂದು. ವೈವಿಧ್ಯಗಳ ಮಧ್ಯದಲ್ಲೆಲ್ಲಾ ಇರುವುದು ಒಂದು.

ಈಗ ಈ ವೈವಿಧ್ಯಗಳೆಲ್ಲಾ ಇವೆ. ನಾವು ಈಗ ಅದನ್ನು ನೋಡುತ್ತಿರುವೆವು. ಅದೇ ಪಂಚಭೂತಗಳು–ಪೃಥ್ವಿ, ಅಪ್, ಮರುತ್, ತೇಜಸ್ಸು, ಮತ್ತು ವ್ಯೋಮ. ಇದಾದ ಮೇಲೆ ಮನೋಕ್ಷೇತ್ರ, ಅದಾದ ಮೇಲೆ ಆಧ್ಯಾತ್ಮಿಕ ಕ್ಷೇತ್ರ, ಆಧ್ಯಾತ್ಮವೇ ಬೇರೆ, ಮನಸ್ಸೇ ಬೇರೆ ಎಂದಲ್ಲ. ಒಂದೇ ಇವುಗಳ ಹಿಂದೆಲ್ಲಾ ಇರುವುದು. ಘನವು ದ್ರವವಾಗಬೇಕು. ಪಂಚಭೂತಗಳು ಯಾವ ಕ್ರಮದಲ್ಲಿ ವಿಕಾಸಹೊಂದಿದವೊ ಅದೇ ಕ್ರಮದಲ್ಲಿ ಅವು ಹಿಂತಿರುಗಬೇಕು. ಘನ ದ್ರವವಾಗಿ ಕೊನೆಯಲ್ಲಿ ಆಕಾಶವಾಗಬೇಕಾಗಿದೆ. ಇದೇ ಬ್ರಹ್ಮಾಂಡ. ಇಲ್ಲಿ ಬಾಹ್ಯ ಬ್ರಹ್ಮಾಂಡವಿದೆ, ವಿಶ್ವೇಶ್ವರನು ಇರುವನು, ಮನಸ್ಸು ಮತ್ತು ಪಂಚಭೂತಗಳು ಇವೆ.

ಇದರಂತೆಯೇ ಜೀವನೂ ಕೂಡ. ಪಿಂಡಾಂಡದ ದೃಷ್ಟಿಯಿಂದ ನಾನು ಹಾಗೆಯೇ ಇರುವೆನು. ನಾನು ಆತ್ಮ ಮನಸ್ಸು ಪಂಚಭೂತಗಳಾಗಿದ್ದೇನೆ. ನಾನೀಗ ನನ್ನ ಅಧ್ಯಾತ್ಮ ಸ್ಥಿತಿಗೆ ಹಿಂತಿರುಗಬೇಕಾಗಿದೆ. ಜೀವ ತನ್ನ ಒಂದು ಬಾಳುವೆಯಲ್ಲಿ ವಿಶ್ವವನ್ನೆಲ್ಲಾ ಬಾಳುವುದು. ಹೀಗೆ ಮಾನವ ಈ ಜೀವನದಲ್ಲೇ ಮುಕ್ತನಾಗಬಲ್ಲ. ಅವನು ತನ್ನ ಒಂದು ಅಲ್ಪ ಜೀವನದಲ್ಲಿ ಇಡೀ ವಿಶ್ವದ ಜೀವನವನ್ನೇ ಬಾಳಬಲ್ಲ.

\newpage

ನಾವೆಲ್ಲಾ ಹೋರಾಡುತ್ತಿರುವೆವು. ನಾವು ಪರಮ ಸತ್ಯವನ್ನು ಸೇರದೇ ಇದ್ದರೂ ಸ್ವಲ್ಪ ಮೇಲಕ್ಕೆ ಆದರೂ ಹೋಗುವೆವು. ಅದು ನಾವು ಈಗ ಇರುವ ಸ್ಥಿತಿಗಿಂತ ಮೇಲು.

ಧ್ಯಾನ ಎಂದರೆ ಪ್ರತಿಯೊಂದನ್ನೂ ಅದರ ಮೂಲ ಸತ್ಯವಾದ ಆತ್ಮನಿಗೆ ಒಯ್ಯುವ ಅಭ್ಯಾಸ. ಘನ ದ್ರವವಾಗುವುದು, ಅದು ಅನಿಲವಾಗುವುದು, ಅನಿಲ ಆಕಾಶವಾಗುವುದು, ಅನಂತರ ಮನಸ್ಸು. ಕೊನೆಗೆ ಮನಸ್ಸು ಮಾಯವಾಗುವುದು. ಆತ್ಮವೊಂದೇ ಉಳಿಯುವುದು.

ಕೆಲವು ಯೋಗಿಗಳು ಈ ದೇಹವೇ ದ್ರವ ಮುಂತಾದುವುಗಳು ಆಗುತ್ತವೆ ಎಂದು ಹೇಳುತ್ತಾರೆ. ನೀವು ಈ ದೇಹವನ್ನು ಏನು ಬೇಕಾದರೂ ಮಾಡಬಹುದು. ಸಣ್ಣದಾಗಿ ಮಾಡಬಹುದು, ಗಾಳಿಯಂತೆ ಮಾಡಬಹುದು, ಈ ಗೋಡೆಯ ಮೂಲಕ ಬೇಕಾದರೆ ಹೋಗುವಂತೆ ಮಾಡಬಹುದು ಎಂದು ಅವರು (ಯೋಗಿಗಳು) ಹೇಳುತ್ತಾರೆ. ನಮಗೆ ಅದು ಗೊತ್ತಿಲ್ಲ. ಹೀಗೆ ಮಾಡುವ ಯಾರನ್ನೂ ನಾನು ನೋಡಿಲ್ಲ. ಆದರೆ ಇವೆಲ್ಲಾ ಶಾಸ್ತ್ರದಲ್ಲಿ ಇವೆ. ನಾವು ಶಾಸ್ತ್ರವನ್ನು ನಂಬದೆ ಇರುವುದಕ್ಕೆ ಆಧಾರವಿಲ್ಲ.

ಬಹುಶಃ ನಮ್ಮಲ್ಲಿ ಕೆಲವರು ಅದನ್ನು ಈ ಜೀವನದಲ್ಲಿ ಮಾಡಬಹುದು. ನಮ್ಮ ಹಿಂದಿನ ಕರ್ಮಗಳ ಫಲದಿಂದ ಮಿಂಚಿನಂತೆ ಅದು ಬರಬಹುದು. ಇಲ್ಲಿ ಇರುವವರೆಲ್ಲ ಕೆಲವರು ಹಿಂದಿನ ಜನ್ಮದಲ್ಲಿ ದೊಡ್ಡ ಯೋಗಿಗಳಾಗಿದ್ದಿರಬಹುದು. ಅದನ್ನು ಪೂರೈಸುವುದಕ್ಕೆ ಇನ್ನು ಸ್ವಲ್ಪ ಎಲ್ಲೊ ಮಾಡಬೇಕಾಗಿರಬಹುದು. ಅಭ್ಯಾಸ ಮಾಡಿ.

ಧ್ಯಾನ ಕಲ್ಪನೆಯಿಂದ ಸಿದ್ಧಿಸುವುದು. ಮುಂಚೆ ಭೂತ ಶುದ್ದಿ ಮಾಡಿಕೊಳ್ಳಬೇಕು. ಒಂದು ಮತ್ತೊಂದರಲ್ಲಿ ಲಯವಾಗುವಂತೆ ಮಾಡಬೇಕು. ಅದು ಅದಕ್ಕಿಂತ ಉತ್ತಮವಾದುದರಲ್ಲಿ ಲಯವಾಗಬೇಕು. ಅನಂತರ ಮನಸ್ಸು ಆತ್ಮನಲ್ಲಿ ಲಯವಾಗಬೇಕು. ಕೊನೆಗೆ ನೀನೇ ಆ ಆತ್ಮ.

ಆತ್ಮ ಯಾವಾಗಲೂ ಮುಕ್ತ, ಸರ್ವಶಕ್ತ, ಸರ್ವಜ್ಞ. ಆದರೆ ಇದು ಈಶ್ವರನ ಅಧೀನದಲ್ಲೇನೋ ಇದೆ. ಹಲವು ದೇವರುಗಳು ಇರುವುದು ಸಾಧ್ಯವಿಲ್ಲ. ಮುಕ್ತಜೀವರಿಗೆ ಅಸಾಧ್ಯವಾದ ಶಕ್ತಿಗಳಿವೆ. ಸರ್ವಶಕ್ತಿಯ ಸಮೀಪಕ್ಕೆ ಅವರು ಬರುವರು. ಆದರೆ ಯಾರೂ ದೇವರಷ್ಟು ಶಕ್ತಿಯನ್ನು ಪಡೆಯಲಾರರು. ಒಬ್ಬ ಮುಕ್ತಾತ್ಮ ನಾನು ಈ ಗ್ರಹವನ್ನು ಹೀಗೆ ಹೋಗುವಂತೆ ಮಾಡುತ್ತೇನೆ ಎಂದರೆ ಮತ್ತೊಬ್ಬ ಮುಕ್ತಾತ್ಮ ಅದಕ್ಕೆ ವಿರೋಧವಾಗಿ ಹೋಗುವಂತೆ ಮಾಡುತ್ತೇನೆ ಎಂದರೆ ಈ ಪ್ರಪಂಚ ದೊಡ್ಡ ಗೊಂದಲಕ್ಕೆ ಈಡಾಗುವುದು.

ನಾನು ಇಂಗ್ಲಿಷಿನಲ್ಲಿ 'ನಾನು ದೇವರು' ಎಂದರೆ ತಪ್ಪು ತಿಳಿದುಕೊಳ್ಳಬೇಡಿ. ನಾನು ಇಂಗ್ಲಿಷಿನಲ್ಲಿ ದೇವರು ಎಂಬ ಪದವನ್ನು ಉಪಯೋಗಿಸುವುದಕ್ಕೆ ಕಾರಣ ಮತ್ತೆ ಬೇರೆ ಯಾವ ಪದವೂ ಇಲ್ಲದೆ ಇರುವುದರಿಂದ. ಸಂಸ್ಕೃತದಲ್ಲಿ ದೇವರು ಎಂದರೆ ಸಚ್ಚಿದಾನಂದ. ಅದು ಅನಂತವಾದ ಚಿತ್. ಯಾವ ವ್ಯಕ್ತಿಗೂ ಅನ್ವಯಿಸುವುದಿಲ್ಲ. ವ್ಯಕ್ತ್ಯಾತೀತವಾದುದು ಅದು.

ನಾನು ಎಂದಿಗೂ ಈಶ್ವರನಾಗುವುದಿಲ್ಲ. ಸರ್ವವ್ಯಾಪಿಯಾದ, ಅವ್ಯಕ್ತವಾದ ಬ್ರಹ್ಮನಲ್ಲಿ ಬೇಕಾದರೆ ಒಂದಾಗಬಹುದು. ಇಲ್ಲೊಂದು ದೊಡ್ಡ ಮಣ್ಣಿನ ರಾಶಿ ಇದೆ. ನೀವೊಂದು ಆನೆಯನ್ನು ಮಾಡುತ್ತೀರಿ. ಇದರ ಮೂಲಕ ನಾನು ಒಂದು ಚಿಕ್ಕ ಇಲಿಯನ್ನು ಮಾಡುತ್ತೇನೆ. ಎರಡೂ ಜೇಡಿಯ ಮಣ್ಣೆ, ಎರಡೂ ಕರಗಿ ಹೋಗುವುವು. ಅವೆರಡೂ ಮೂಲತಃ ಒಂದೇ. ನಾನು ನನ್ನ ತಂದೆ ಇಬ್ಬರೂ ಒಂದೇ. ಆದರೆ ಮಣ್ಣಿನ ಇಲಿ ಎಂದಿಗೂ ಮಣ್ಣಿನ ಆನೆಯಾಗಲಾರದು.

ನನಗೆ ಸ್ವಲ್ಪ ಜ್ಞಾನವಿದೆ. ನಾನು ಎಲ್ಲಿಯೋ ನಿಲ್ಲುತ್ತೇನೆ. ನಿಮಗೆ ಇನ್ನೂ ಸ್ವಲ್ಪ ಹೆಚ್ಚು ಜ್ಞಾನವಿದೆ. ನೀವು ಸ್ವಲ್ಪ ಮುಂದೆ ನಿಲ್ಲುವಿರಿ. ಎಲ್ಲರಿಗಿಂತ ಶ್ರೇಷ್ಠನಾದ ಒಬ್ಬ ಆತ್ಮನಿರುವನು. ಇವನೇ ಈಶ್ವರ, ಯೋಗೇಶ್ವರ, ಸೃಷ್ಟಿಸುವ ಸಗುಣ ಬ್ರಹ್ಮ. ಅವನಿಗೊಂದು ವ್ಯಕ್ತಿತ್ವವಿದೆ. ಅವನು ಸರ್ವಶಕ್ತನು. ಅವನು ಸರ್ವಾಂತರ್ಯಾಮಿ. ಅವನಿಗೊಂದು ದೇಹವಿಲ್ಲ. ಅವನಿಗೆ ದೇಹದ ಆವಶ್ಯಕತೆಯಿಲ್ಲ. ಧ್ಯಾನದಿಂದ ನಿಮಗೆ ಏನೇನು ಸಿಕ್ಕುವುದೋ ಅದೆಲ್ಲಾ ಯೋಗೇಶ್ವರನಾದ ಈಶ್ವರನ ಚಿಂತನೆಯಿಂದ ಬರುವುದು.

ಮುಕ್ತಾತ್ಮನ ಮೇಲೆ ಧ್ಯಾನಿಸುವುದರಿಂದ ಅಥವಾ ಜೀವನದ ಸಾಮರಸ್ಯದ ಮೇಲೆ ಧ್ಯಾನಿಸುವುದರಿಂದ ಇದೇ ಪ್ರತಿಫಲವನ್ನು ಪಡೆಯಬಹುದು. ಇವುಗಳನ್ನೇ ವಸ್ತುಧ್ಯಾನ ಎನ್ನುವುದು. ನೀವು ಕೆಲವು ಬಾಹ್ಯ ವಸ್ತುಗಳ ಮೇಲೆ ಒಳಗೋ ಅಥವಾ ಹೊರಗೋ ಧ್ಯಾನಿಸುವುದನ್ನು ಪ್ರಾರಂಭಿಸುತ್ತೀರಿ. ನೀವು ಒಂದು ದೊಡ್ಡ ವಾಕ್ಯವನ್ನು ತೆಗೆದುಕೊಂಡರೆ ಅದು ಧ್ಯಾನವಲ್ಲ. ಅದು ಕೇವಲ ಮನಸ್ಸನ್ನು ಕೇಂದ್ರೀಕರಿಸಲು ಸಹಾಯ ಮಾಡಬಹುದು. ಧ್ಯಾನ ಎಂದರೆ ಮನಸ್ಸು ಕೇವಲ ತನ್ನ ಮೇಲೆಯೇ ಕುರಿತು ಚಿಂತಿಸುವುದು. ಚಿತ್ತ ಯಾವಾಗ ವೃತ್ತಿಗಳನ್ನು ನಿಗ್ರಹಿಸುವುದೋ, ಆಗ ಪ್ರಪಂಚವೂ ನಿಂತಂತೆ. ನಿಮ್ಮ ಚೈತನ್ಯ ವಿಕಾಸವಾಗುವುದು. ನೀವು ಧ್ಯಾನಿಸಿದಷ್ಟೂ ಮುಂದುವರಿಯುತ್ತಾ ಹೋಗುವಿರಿ. ಕಷ್ಟಪಟ್ಟು ಅಭ್ಯಾಸವನ್ನು ಮಾಡುತ್ತಾ ಹೋಗಿ. ಧ್ಯಾನ ಸಿದ್ಧಿಸುವುದು. ನಿಮಗೆ ದೇಹ ಅಥವಾ ಮತ್ತೆ ಯಾವ ಭಾವನೆಯೂ ಇರುವುದಿಲ್ಲ. ನೀವು ಒಂದು ಗಂಟೆ ಧ್ಯಾನಮಾಡಿ ಆದ ಮೇಲೆ ನಿಮಗೆ ನಿಮ್ಮ ಜೀವನದಲ್ಲೇ ಇದುವರೆಗೆ ಸಿಕ್ಕದ ವಿಶ್ರಾಂತಿಯು ಸಿಕ್ಕುವುದು. ನಿಮ್ಮ ದೇಹಕ್ಕೆ ಕೊಡುವ ಏಕಮಾತ್ರ ವಿಶ್ರಾಂತಿಯೇ ಇದು. ಮನಸ್ಸು ಗಾಢವಾದ ನಿದ್ರಾವಸ್ಥೆಯಲ್ಲೂ ಚಂಚಲವಾಗಿರುವುದು. ಧ್ಯಾನದ ಕೆಲವು ನಿಮಿಷಗಳಲ್ಲಿ ಅದು ಸ್ತಬ್ದವಾಗುವುದು. ಎಲ್ಲೋ ಸ್ವಲ್ಪ ಮಾತ್ರ ಸ್ಪಂದನೆ ಇರುವುದು. ಆಗ ನಿಮಗೆ ದೇಹಭಾವ ಮಾಯವಾಗುವುದು. ನಿಮ್ಮನ್ನು ಚೂರು ಚೂರು ಮಾಡಿದರೂ ಅದು ನಿಮಗೆ ಆಗ ಗೊತ್ತಾಗುವುದಿಲ್ಲ. ನಿಮಗೆ ಧ್ಯಾನದಲ್ಲಿ ಅಂತಹ ಆನಂದ ದೊರಕುವುದು. ನೀವಾಗ ಅಷ್ಟು ಹಗುರವಾಗುವಿರಿ. ಇಂತಹ ಪರಿಪೂರ್ಣವಾದ ವಿಶ್ರಾಂತಿಯು ಧ್ಯಾನದಿಂದ ಲಭಿಸುವುದು.

ಅನಂತರ ಹಲವು ವಸ್ತುಗಳ ಮೇಲೆ ಧ್ಯಾನಮಾಡುವುದಿದೆ. ಹಲವು ಚಕ್ರಗಳ ಮೇಲೆ ಧ್ಯಾನಮಾಡುವುದು ಬೇರೆ ಇದೆ. ಯೋಗಿಗಳ ಪ್ರಕಾರ ಸುಷುಮ್ನಾ ಕಾಲುವೆಯಲ್ಲಿ ಇಡ ಮತ್ತು ಪಿಂಗಳ ಎಂಬ ಎರಡು ನರಗಳ ಪ್ರವಾಹವಿದೆ. ಇವುಗಳ ಮೂಲಕವೇ ಜ್ಞಾನೇಂದ್ರಿಯ ಮತ್ತು ಕರ್ಮೇಂದ್ರಿಯದ ಸಂವೇದನೆಗಳು ಚಲಿಸುವುವು. ಸುಷುಮ್ನಾ ಕಾಲುವೆ ಬೆನ್ನೆಲುಬಿನ ಮಧ್ಯದಲ್ಲಿದೆ. ಈ ಸುಷುಮ್ನಾ ಕಾಲುವೆ ಸಾಮಾನ್ಯವಾಗಿ\break ಮುಚ್ಚಿಹೋಗಿದೆ ಎಂದು ಯೋಗಿಗಳು ಹೇಳುತ್ತಾರೆ. ಆದರೆ ಧ್ಯಾನದಿಂದ ಇದು ತೆರೆಯುವುದು. ಶಕ್ತಿಯನ್ನು ಸುಷುಮ್ನೆಯ ಮೂಲಕ್ಕೆ ಕಳುಹಿಸಬೇಕು. ಆಗ ಕುಂಡಲಿನಿ ಜಾಗ್ರತವಾಗುವುದು. ಆಗ ಜಗತ್ತೇ ಬದಲಾಯಿಸುವುದು.

ಸಹಸ್ರಾರು ದೇವ ದೇವತೆಗಳು ನಿಮ್ಮ ಸುತ್ತಮುತ್ತ ನಿಂತಿರಬಹುದು. ನೀವು ಅವರನ್ನು ನೋಡಲಾರಿರಿ. ಏಕೆಂದರೆ ನೀವು ನಿಮ್ಮ ಸ್ಥೂಲ ಇಂದ್ರಿಯದ ಮೂಲಕ ಮಾತ್ರ ಗ್ರಹಿಸಲು ಸಾಧ್ಯ. ನಾವು ಕೇವಲ ಬಾಹ್ಯವನ್ನು ಮಾತ್ರ ನೋಡಬಹುದು. ನಾವು ಅದನ್ನು \enginline{x} ಎಂದು ಇಟ್ಟುಕೊಳ್ಳೋಣ. \enginline{x} ಅನ್ನು ನಮ್ಮ ಮನಸ್ಸಿನ ಸ್ಥಿತಿಗೆ ಅನುಗುಣವಾಗಿ ನೋಡುತ್ತೇವೆ. ಹೊರಗೆ ಒಂದು ಮೋಟು ಮರ ಇದೆ ಎಂದುಕೊಳ್ಳೋಣ. ಒಬ್ಬ ಕಳ್ಳ ಬಂದರೆ ಅಲ್ಲಿ ಅವನಿಗೆ ಏನು ಕಂಡಿತು ಎಂದು ಭಾವಿಸಿದಿರಿ? ಒಬ್ಬ ಪೋಲೀಸಿನವನು. ಮಗು ದೊಡ್ಡದೊಂದು ಭೂತವನ್ನೇ ನೋಡಿತು. ತನ್ನ ಪ್ರಿಯತಮಳಿಗಾಗಿ ಕಾಯುತ್ತಿದ್ದ ಯುವಕನಿಗೆ ಏನು ಕಂಡಿತು ಎಂದು ಭಾವಿಸಿದಿರಿ? ಅವನ ಪ್ರಿಯತಮಳು. ಆದರೆ ಆ ಮೋಟುಮರ ಮಾತ್ರ ಬದಲಾಯಿಸಲಿಲ್ಲ. ಅದು ಯಾವಾಗಲೂ ಒಂದೇ ಸಮನಾಗಿರುವುದು. ಇದು ದೇವರೇ ಆಗಿರುವುದು. ನಾವು ಅದನ್ನು ನಮ್ಮ ಮೌಢ್ಯದಿಂದ ಮನುಷ್ಯನಂತೆ, ಮಣ್ಣಿನಂತೆ, ಮೂಢನಂತೆ, ದುಃಖಿಯಂತೆ ನೋಡುತ್ತಿರುವೆವು.

ಯಾರ ಮನಸ್ಸು ಒಂದೇ ರೀತಿ ಇದೆಯೋ ಅವರೆಲ್ಲ ಈ ಪ್ರಪಂಚದಲ್ಲಿ ಸ್ವಾಭಾವಿಕವಾಗಿ ಒಂದೇ ಕಡೆ ನೆಲಸಿರುವರು. ಬೇರೆ ವಿಧವಾಗಿ ಹೇಳುವುದಾದರೆ, ನೀವೆಲ್ಲಾ ಒಂದೇ ಪ್ರಪಂಚದಲ್ಲಿರುವಿರಿ. ಎಲ್ಲಾ ಸ್ವರ್ಗ ನರಕಗಳೂ ಇಲ್ಲಿವೆ. ವಿವಿಧ ಸ್ವರಗಳನ್ನು ಒಂದನ್ನೊಂದು ಸಂಧಿಸುವ ದೊಡ್ಡ ವೃತ್ತಗಳಿಗೆ ಹೋಲಿಸಬಹುದು. ನಾವು ಒಂದು ವೃತ್ತದಲ್ಲಿದ್ದರೆ ಬೇರೆ ವೃತ್ತದ ಬಿಂದುಗಳನ್ನೂ ನಾವು ಮುಟ್ಟಬಹುದು. ಮನಸ್ಸು ಕೇಂದ್ರಕ್ಕೆ ಹೋದರೆ, ಆಗ ಎಲ್ಲಾ ಸ್ತರಗಳ ಅರಿವೂ ಆಗುವುದು. ಕೆಲವು ವೇಳೆ ಧ್ಯಾನದಲ್ಲಿ ನೀವು ಬೇರೊಂದು ಕ್ಷೇತ್ರವನ್ನು ಮುಟ್ಟುತ್ತೀರಿ. ಆಗ ನಿಮಗೆ ಬೇರೆ ಬೇರೆ ಜೀವಿಗಳು ಗೋಚರಿಸುವುವು. ಭೂತಾದಿಗಳು ಗೋಚರಿಸುವುವು. ಧ್ಯಾನದ ಶಕ್ತಿಯಿಂದ ನೀವು ಅಲ್ಲಿಗೆ ಹೋದಿರಿ. ಈ ಶಕ್ತಿ ನಮ್ಮ ಇಂದ್ರಿಯಗಳನ್ನು ಬದಲಾಯಿಸುವುದು. ಅವುಗಳನ್ನು ಪರಿಶುದ್ಧ ಮಾಡುವುದು. ನೀವು ಐದು ದಿನ ಧ್ಯಾನವನ್ನು ಅಭ್ಯಾಸ ಮಾಡಿದರೆ ನಿಮಗೆ ಪ್ರಜ್ಞಾ ಕೇಂದ್ರದ ಮೂಲಕ ಸಂವೇದನೆ ಕಾಣಿಸಿಕೊಳ್ಳುವುದು. ಕೇಳುವ ಶಕ್ತಿ ತುಂಬಾ ಸೂಕ್ಷ್ಮವಾಗುವುದು. ಆದಕಾರಣವೇ ಹಿಂದೂ ದೇವ ದೇವತೆಗಳಿಗೆಲ್ಲ ಮೂರು ಕಣ್ಣುಗಳಿವೆ. ಮೂರನೆಯದೇ ಮನೋಚಕ್ಷಸ್ಸು; ಅದು ಆಧ್ಯಾತ್ಮಿಕ ವಸ್ತುಗಳನ್ನು ತೋರುವುದು.

ಕುಂಡಲಿನಿ ಶಕ್ತಿ ಚಕ್ರದಿಂದ ಚಕ್ರಕ್ಕೆ ಏಳುತ್ತ ಹೋದಂತೆಲ್ಲ ಇಂದ್ರಿಯಗಳನ್ನು ಬದಲಾಯಿಸುವುದು. ನೀವು ಈ ಪ್ರಪಂಚವನ್ನೇ ಬೇರೆ ರೀತಿಯಿಂದ ನೋಡುವಿರಿ. ಇದೇ ಸ್ವರ್ಗವಾಗುವುದು. ನೀವು ಆಗ ಮಾತನಾಡಲಾರಿರಿ. ಮಾತನಾಡಿದರೆ ಕುಂಡಲಿನಿ ಕೆಳಗಿನ ಮೆಟ್ಟಲಿಗೆ ಹೋಗುವುದು. ಕುಂಡಲಿನಿ ಎಲ್ಲಾ ಚಕ್ರಗಳನ್ನೂ ದಾಟಿ ಮೆದುಳಿನಲ್ಲಿ ಸಹಸ್ರಾರವನ್ನು ಸೇರಿದಾಗ ದೃಶ್ಯ ಪ್ರಪಂಚವೆಲ್ಲಾ ಮಾಯವಾಗಿ ನೀವು ನಿಜವಾಗಿ ಏನಾಗಿರುವಿರೋ ಅದಾಗುವಿರಿ. ಆಗ ಏಕಮೇವ ಅದ್ವಿತೀಯವಲ್ಲದೆ ಬೇರಿಲ್ಲ. ನೀವೇ ದೇವರು. ಎಲ್ಲಾ ಸ್ವರ್ಗಗಳನ್ನೂ ನೀವು ಅದರಿಂದ ಮಾಡಿದ್ದು, ಎಲ್ಲಾ ಲೋಕಗಳನ್ನೂ ಅದರಿಂದ ಮಾಡಿದ್ದು, ಇರುವುದು ಅದೊಂದೇ. ಬೇರೊಂದು ಇಲ್ಲ.

