
\chapter[ಜ್ಞಾನಮಾರ್ಗ]{ಜ್ಞಾನಮಾರ್ಗ\protect\footnote{\engfoot{C.W. Vol. VIII, P.3}}}

\begin{center}
೧
\end{center}

ಓಂ ತತ್ ಸತ್! `ಓಂ' ಎನ್ನುವುದನ್ನು ತಿಳಿಯುವುದು ವಿಶ್ವ ರಹಸ್ಯವನ್ನು ತಿಳಿದಂತೆ. ಜ್ಞಾನಯೋಗದ ಗುರಿ ರಾಜಯೋಗ, ಭಕ್ತಿಯೋಗಗಳಂತೆಯೇ. ಆದರೆ ಮಾರ್ಗ ಮಾತ್ರ ಬೇರೆ. ಇದು ಬಲಾಢ್ಯರಿಗೆ ಮಾತ್ರ, ಯೋಗಿಗಳಿಗೂ ಅಲ್ಲ, ಭಕ್ತರಿಗೂ ಅಲ್ಲ, ವಿಚಾರಪರರಿಗೆ ಮಾತ್ರ. ಭಕ್ತಿಯೋಗಿ ಹೇಗೆ ಪರಮಾತ್ಮನೊಂದಿಗೆ ಏಕತೆಯನ್ನು ಪಡೆಯಲು ಪ್ರೀತಿಯ ಮತ್ತು ಭಕ್ತಿಯ ಮೂಲಕ ಯತ್ನಿಸುತ್ತಾನೆಯೋ ಹಾಗೆಯೇ ಜ್ಞಾನಿಯಾದವನು ಬ್ರಹ್ಮಸಾಕ್ಷಾತ್ಕಾರಕ್ಕೆ ಕೇವಲ ವಿಚಾರದ ಮೂಲಕ ಮಾರ್ಗವನ್ನು ಭೇದಿಸಿಕೊಂಡು ಹೋಗಲು ಯತ್ನಿಸುವನು. ಜ್ಞಾನಿ ತನ್ನ ಹಳೆಯ ವಿಗ್ರಹಗಳನ್ನೆಲ್ಲಾ, ಮೂಢನಂಬಿಕೆಗಳನ್ನೆಲ್ಲಾ, ಆಚಾರಗಳನ್ನೆಲ್ಲಾ, ಇಹಪರಲೋಕಗಳ ಸುಖಗಳನ್ನೆಲ್ಲಾ ತ್ಯಜಿಸಲು ಸಿದ್ಧನಾಗಿ ಕೇವಲ ಮುಕ್ತಿಯನ್ನು ಮಾತ್ರ ಪಡೆಯಲು ಹಟತೊಟ್ಟಿರಬೇಕು. ಜ್ಞಾನವಿಲ್ಲದೆ ನಮಗೆ ಮುಕ್ತಿ ಲಭಿಸಲಾರದು. ಜ್ಞಾನವೆಂದರೆ ನಮ್ಮ ನೈಜಸ್ವಭಾವವನ್ನು ಅರಿಯುವುದು. ನಾವು ಜನನಮರಣಗಳಾಚೆ ಇರುವವರು ಮತ್ತು ಎಲ್ಲಾ ಅಂಜಿಕೆಗಳನ್ನು ಮೀರಿದವರು ಎಂಬುದನ್ನು ಅರಿಯುವುದು. ಆತ್ಮ ಸಾಕ್ಷಾತ್ಕಾರವೇ ಪರಮ ಶ್ರೇಯಸ್ಸು. ಅದು ಇಂದ್ರಿಯಾತೀತ, ಮನೋತೀತ. ನಿಜವಾದ `ನಾನು' ಎಂಬುದನ್ನು ಗ್ರಹಿಸಲಾಗುವುದಿಲ್ಲ. ಅದು ನಿತ್ಯ ಜ್ಞಾತೃ, ಅದೆಂದಿಗೂ ಜ್ಞೇಯವಾಗಲಾರದು. ಏಕೆಂದರೆ ಜ್ಞಾನ ಯಾವಾಗಲೂ ಸಾಪೇಕ್ಷವಾದುದು. ಅದು ನಿರಪೇಕ್ಷವಾದುದಕ್ಕೆ ಅನ್ವಯಿಸುವುದಿಲ್ಲ. ಎಲ್ಲಾ ಇಂದ್ರಿಯಜ್ಞಾನವೂ ಮಿತವಾಗಿರುವುದಕ್ಕೆ ಅನ್ವಯಿಸುವುದು. ಅದೊಂದು ಕಾರ್ಯಕಾರಣಗಳ ತುದಿಮೊದಲಿಲ್ಲದ ವೃತ್ತದಂತೆ. ಈ ಪ್ರಪಂಚ ಸಾಪೇಕ್ಷವಾದುದು, ಸತ್ಯದ ಒಂದು ಛಾಯೆ. ಆದರೂ ಇದು ಸುಖದುಃಖಗಳು ಸಮವಾಗಿರುವ ಕ್ಷೇತ್ರವಾದುದರಿಂದ ಇಲ್ಲಿ ಮಾತ್ರ ಮಾನವ ತನ್ನ ನೈಜಸ್ಥಿತಿಯನ್ನು ತಿಳಿದು ತಾನೇ ಬ್ರಹ್ಮವೆಂದು ಅರಿಯಲು ಸಾಧ್ಯ.

ಈ ಜಗತ್ತು “ಪ್ರಕೃತಿಯ ವಿಕಾಸ ಮತ್ತು ಬ್ರಹ್ಮನ ಆವಿರ್ಭಾವ,'' ಮಾಯೆಯ ತೆರೆಯ ಮೂಲಕ ಕಾಣುವ ಬ್ರಹ್ಮನಿಗೆ ನಾವು ಕೊಡುವ ವಿವರಣೆ ಇದು. ಪ್ರಪಂಚ ಶೂನ್ಯವಲ್ಲ. ಅದಕ್ಕೆ ಒಂದು ವಿಧವಾದ ಅಸ್ತಿತ್ವವಿದೆ. ಬ್ರಹ್ಮನಿರುವುದರಿಂದ ಈ ಜಗತ್ತು ಕಾಣಿಸುತ್ತಿದೆ.

ನಾವು ಜ್ಞಾತೃವನ್ನು ಅರಿಯುವುದು ಹೇಗೆ? “ನಾವು ಅದೇ ಆಗಿರುವೆವು. ನಾವು ಅದನ್ನು ಎಂದಿಗೂ ಅರಿಯುವುದಕ್ಕೆ ಆಗುವುದಿಲ್ಲ. ಅದೆಂದಿಗೂ ಜ್ಞೇಯವಾಗಲಾರದು" ಎನ್ನುವುದು ವೇದಾಂತ, ಆಧುನಿಕ ವಿಜ್ಞಾನ ಕೂಡ ಇದನ್ನೇ ಹೇಳುವುದು. ಅದನ್ನು ನಾವು ತಿಳಿಯಲಾರೆವು. ಆದರೆ ಕೆಲವು ವೇಳೆ ಅದರ ಹೊಳಹುಗಳು ಕಾಣಬಹುದು. ಈ ಪ್ರಪಂಚದ ಭ್ರಾಂತಿ ಒಮ್ಮೆ ನಾಶವಾಯಿತೆಂದರೆ ಅದು ಪುನಃ ನಮ್ಮೆದುರಿಗೆ ಬರುವುದು. ಆದರೆ ಪುನಃ ಅದು ಸತ್ಯದಂತೆ ಕಾಣಿಸಲಾರದು. ಅದೊಂದು ಮರೀಚಿಕೆಯೆಂದು ಗೊತ್ತಾಗುವುದು. ಅದನ್ನು ದಾಟಿ ಹೋಗುವುದೇ ಎಲ್ಲ ಧರ್ಮಗಳ ಗುರಿ. ಜೀವ ಮತ್ತು ಬ್ರಹ್ಮ ಒಂದೇ ಎಂದು ವೇದಗಳು ಪದೇಪದೇ ಸಾರುವುವು. ಆದರೆ ಎಲ್ಲೋ ಕೆಲವರು ಮಾತ್ರ ಈ ಮಾಯೆಯ ತೆರೆಯ ಹಿಂದೆ ಹೋಗಿ ಈ ಸತ್ಯವನ್ನು ಸಾಕ್ಷಾತ್ಕಾರ ಮಾಡಿಕೊಂಡಿರುವರು.

ಜ್ಞಾನಿಯಾಗಲಿಚ್ಚಿಸುವವನು ಮೊದಲು ಪಾರಾಗಬೇಕಾಗಿರುವುದು ಅಂಜಿಕೆಯಿಂದ. ಭಯವು ನಮ್ಮ ಪರಮ ಶತ್ರುಗಳಲ್ಲಿ ಒಂದು. ಎರಡನೆಯದೆ ನಿಮಗೆ ತಿಳಿಯುವವರೆಗೆ ಯಾವುದನ್ನೂ ನಂಬದೆ ಇರುವುದು. ಸದಾಕಾಲದಲ್ಲಿಯೂ “ನಾನು ದೇಹವಲ್ಲ, ಮನಸ್ಸಲ್ಲ, ಆಲೋಚನೆಯಲ್ಲ, ಅಂತಃಕರಣವೂ ಅಲ್ಲ. ನಾನು ಆತ್ಮ” ಎಂದು ಯೋಚಿಸಿ. ನೀವು ಎಲ್ಲವನ್ನೂ ಆಚೆಗೆ ಕಿತ್ತೊಗೆದರೆ ನಿಜವಾದ ಆತ್ಮವೊಂದೇ ಉಳಿಯುವುದು. ಜ್ಞಾನಿಗಳ ಧ್ಯಾನ ಎರಡು ಬಗೆಯದು. ಯಾವುದೆಲ್ಲ ನಮ್ಮ ನಿಜಸ್ವರೂಪವಲ್ಲವೋ ಅದನ್ನೆಲ್ಲ ಅಲ್ಲಗಳೆಯುವುದು ಮತ್ತು ನಮ್ಮ ನಿಜಸ್ವರೂಪವಾಗಿರುವ ಸಚ್ಚಿದಾನಂದ ಆತ್ಮವನ್ನು ಒತ್ತಿಹೇಳುವುದು. ನಿಜವಾದ ಜಿಜ್ಞಾಸು ಯಾವಾಗಲೂ ಮುಂದೆ ನುಗ್ಗಬೇಕು, ಧೈರ್ಯದಿಂದ ವಿಚಾರದ ಪರಮಾವಧಿಯನ್ನು ಮುಟ್ಟಬೇಕು. ದಾರಿಯ ಮಧ್ಯದಲ್ಲಿ ಎಲ್ಲಿಯೂ ನಿಲ್ಲುವುದಕ್ಕೆ ಆಗುವುದಿಲ್ಲ. ನಾವು ನೇರಮಾರ್ಗವನ್ನು ಅನುಸರಿಸಿದರೆ ಎಲ್ಲವನ್ನೂ ತ್ಯಜಿಸಬೇಕು. ಕೊನೆಗೆ ನಾವು ತ್ಯಜಿಸಲಾಗದ, ಇಲ್ಲವೆನ್ನಲಾಗದ ನಿಜವಾದ `ನಾನು' ಎಂಬುದಕ್ಕೆ ಬರುತ್ತೇವೆ. ಆ `ನಾನೇ' ಜಗತ್ತಿನ ಸಾಕ್ಷಿ, ಪ್ರಭು, ತುದಿಮೊದಲಿಲ್ಲದುದು, ಅನಂತ, ಆದರೆ ಈಗ ಅದು ನಮ್ಮ ಕಣ್ಣಿಗೆ ಕಾಣದಂತೆ ಹಲವು ಅಜ್ಞಾನದ ಆವರಣಗಳು ಅದನ್ನು ಮುತ್ತಿವೆ. ಆದರೂ ಅದು ಎಂದಿನಂತೆಯೇ ಇದೆ.

ಎರಡು ಹಕ್ಕಿಗಳು ಒಂದು ಮರದ ಮೇಲಿದ್ದವು. ಮೇಲಿರುವ ಹಕ್ಕಿ ಶಾಂತವಾಗಿತ್ತು, ಗಂಭೀರವಾಗಿತ್ತು, ಸುಂದರವಾಗಿತ್ತು, ಪೂರ್ಣವಾಗಿತ್ತು. ಕೆಳಗೆ ಇರುವ ಹಕ್ಕಿ ಯಾವಾಗಲೂ ರೆಂಬೆಯಿಂದ ರೆಂಬೆಗೆ ನೆಗೆಯುತ್ತ, ಒಂದು ಸಲ ಸಿಹಿಹಣ್ಣನ್ನು ತಿಂದು ಸಂತೋಷದಿಂದ ನಲಿಯುತ್ತ, ಇನ್ನೊಂದು ಸಲ ಕಹಿಹಣ್ಣನ್ನು ತಿಂದು ದುಃಖದಿಂದ ಸೊರಗುತ್ತ ಇತ್ತು. ಒಂದು ದಿನ ಎಂದಿಗಿಂತ ಹೆಚ್ಚು ಕಹಿಯಾದ ಹಣ್ಣನ್ನು ತಿಂದು ವ್ಯಥೆಪಡುತ್ತಿದ್ದಾಗ, ಮೇಲಿರುವ, ಶಾಂತವಾಗಿರುವ, ಗಂಭೀರವಾಗಿರುವ ಹಕ್ಕಿಯನ್ನು ನೋಡಿ “ನಾನು ಅದರಂತೆ ಇದ್ದಿದ್ದರೆ!?" ಎಂದು ಕೊಂಚ ಅದರ ಹತ್ತಿರ ನೆಗೆಯಿತು. ಆದರೆ ಸ್ವಲ್ಪ ಹೊತ್ತಿನಲ್ಲೇ ತಾನು ಅದರಂತೆ ಇರಬೇಕು ಎಂಬುದನ್ನು ಮರೆತು ಪುನಃ ಸಿಹಿ–ಕಹಿ ಹಣ್ಣುಗಳನ್ನು ತಿಂದು ಸುಖದುಃಖಗಳಲ್ಲಿ ಸಿಕ್ಕಿ ತೊಳಲಹತ್ತಿತು. ಪುನಃ ಮೇಲೆ ನೋಡಿತು. ಶಾಂತಭಾವದಲ್ಲಿ ಗಂಭೀರವಾಗಿರುವ ಹಕ್ಕಿಯ ಸಮೀಪಕ್ಕೆ ತೆರಳಿತು. ಅನೇಕ ಸಲ ಹೀಗೆಯೇ ಆಗಿ ಕೊನೆಗೆ ಅದು ಮೇಲಿನ ಹಕ್ಕಿಯ ಅತ್ಯಂತ ಸಮೀಪಕ್ಕೆ ಹೋಯಿತು. ಆ ಹಕ್ಕಿಯ ಪುಕ್ಕದ ಕಾಂತಿ ಕಣ್ಣು ಕೋರೈಸಿತು. ಅದರಲ್ಲೇ ಮಗ್ನವಾಯಿತು. ಕೊನೆಗೆ ಅದಕ್ಕೆ ಆಶ್ಚರ್ಯವಾಯಿತು. ಇರುವುದು ಒಂದೇ ಒಂದು ಹಕ್ಕಿ ಎಂದು ಅದಕ್ಕೆ ತಿಳಿಯಿತು. ಇದುವರೆಗೂ ತಾನೇ ಮೇಲಿನ ಹಕ್ಕಿ ಆಗಿದ್ದೆ ಎಂದೂ, ತನಗೆ ಈಗ ಮಾತ್ರ ಅದು ಗೊತ್ತಾಯಿತು ಎಂದೂ ಅರಿಯಿತು. ಮಾನವನು ಆ ಕೆಳಗಿನ ಹಕ್ಕಿಯಂತೆ. ಅವನು ಪರಮಾದರ್ಶದ ಕಡೆಗೆ ಹೋಗಲು ಸತತ ಯತ್ನಿಸಿದರೆ, ಕೊನೆಗೆ ತಾನೇ ಆ ಪರಮಾತ್ಮನಾಗಿರುವವನು, ಉಳಿದುದೆಲ್ಲ ಭ್ರಾಂತಿ ಎಂದು ಅವನಿಗೆ ಗೊತ್ತಾಗುವುದು. ಎಲ್ಲ ವಿಧದ ಪ್ರಕೃತಿಯ ಪಾಶಗಳಿಂದ ಮತ್ತು ನಂಬಿಕೆಗಳಿಂದ ಪಾರಾಗುವುದೇ ನಿಜವಾದ ಜ್ಞಾನ. ಜ್ಞಾನಿ ಯಾವಾಗಲೂ `ಓಂ ತತ್ಸತ್' ಅದೊಂದೇ ಸತ್ಯ ಎಂಬುದನ್ನು ಗಮನದಲ್ಲಿಡಬೇಕು. ಕೇವಲ ಏಕತೆಯೇ ಜ್ಞಾನಯೋಗದ ತಳಹದಿ. ಇದನ್ನೇ ಅದ್ವೈತ ಎನ್ನುವುದು. ಇದೇ ವೇದಾಂತದ ಮೂಲತತ್ತ್ವಯ, ಸಾರಸರ್ವಸ್ವ. ಬ್ರಹ್ಮ ಒಂದೇ ಸತ್ಯ ಉಳಿದುದೆಲ್ಲ ಮಿಥ್ಯ. `ನಾನೇ ಬ್ರಹ್ಮ'. ಇದನ್ನು ಪದೇಪದೇ ನಮ್ಮ ಸಹಜ ಸ್ವಭಾವವಾಗುವವರೆಗೆ ಮನನ ಮಾಡಿದರೆ ಮಾತ್ರ ಎಲ್ಲ ವಿಧವಾದ ದ್ವಂದ್ವಗಳಿಂದ, ಪಾಪ ಪುಣ್ಯಗಳಿಂದ, ಸುಖದುಃಖಗಳಿಂದ, ಅಳುನಗೆಗಳಿಂದ ಪಾರಾದ, ಏಕಮೇವಾದ್ವಿತೀಯವಾದ, ಸನಾತನವಾದ, ಅವಿಕಾರಿಯಾದ ಅನಂತವು ನಾವೇ ಎಂಬುದು ಗೊತ್ತಾಗುವುದು.

ಜ್ಞಾನಯೋಗಿಗೆ ಅತಿ ಸಂಕುಚಿತ ಮತಭ್ರಾಂತನಲ್ಲಿರುವ ತೀವ್ರತೆ ಇರಬೇಕು, ಆದರೂ ಮನಸ್ಸು ಆಗಸದಷ್ಟು ವಿಶಾಲವಾಗಿರಬೇಕು. ಅವನು ತನ್ನ ಮನಸ್ಸನ್ನು ಸಂಪೂರ್ಣವಾಗಿ ನಿಗ್ರಹಿಸಬೇಕು. ಬೌದ್ಧ ಅಥವಾ ಕ್ರೈಸ್ತನಾಗಲು ಸಾಧ್ಯವಾಗಬೇಕು. ಭಿನ್ನ ಭಿನ್ನ ಭಾವನೆಗಳಿಗೆ ಇಚ್ಛಾನುಸಾರ ಮಾರ್ಪಡಲು ಸಾಧ್ಯವಾಗಬೇಕು. ಆದರೂ ನಿತ್ಯಸಾಮರಸ್ಯವನ್ನು ತೊರೆಯದಂತೆ ಇರಬೇಕು. ನಿರಂತರ ಸಾಧನೆಯಿಂದ ಮಾತ್ರ ಇಂತಹ ನಿಗ್ರಹವನ್ನು ಸಂಪಾದಿಸಲು ಸಾಧ್ಯ. ಭಿನ್ನತೆಗಳೆಲ್ಲಾ ಇರುವುದು ಏಕದಲ್ಲಿ. ಆದರೆ ನಾವು ಮಾಡುವ ಕ್ರಿಯೆಯೊಂದಿಗೆ ತಾದಾತ್ಮ್ಯಭಾವವನ್ನು ಹೊಂದಕೂಡದು. ಆತ್ಮನನ್ನಲ್ಲದೆ ಮತ್ತಾವುದನ್ನೂ ಕುರಿತು ಕೇಳಕೂಡದು, ನೋಡಕೂಡದು, ಮಾತನಾಡಕೂಡದು. ನಮ್ಮ ಮನಸ್ಸನ್ನೆಲ್ಲಾ ತೀವ್ರವಾಗಿ ಅದರಲ್ಲಿ ನೆಟ್ಟಿರಬೇಕು. ಹಗಲು ರಾತ್ರಿ, ನಿಮಗೆ ನೀವೇ `` ಸೋಽಹಂ, ಸೋಽಹಂ'' ಎಂದು ಹೇಳಿಕೊಳ್ಳಿ.

\begin{center}
೨
\end{center}

ವೇದಾಂತ ತತ್ವದ ಆಚಾರ್ಯಶ್ರೇಷ್ಠರೇ ಶ‍್ರೀ ಶಂಕರಾಚಾರ್ಯರು. ಅವರು ಶುದ್ದ ಯುಕ್ತಿಯ ಸಹಾಯದಿಂದ ವೇದಗಳಿಂದ ವೇದಾಂತ ತತ್ತ್ವವನ್ನು ಕಡೆದು, ಅದರ ಆಧಾರದ ಮೇಲೆ ಅದ್ಭುತವಾದ ಜ್ಞಾನಸೌಧವನ್ನು ಕಟ್ಟಿ, ಅದನ್ನು ತಮ್ಮ ಭಾಷ್ಯದಲ್ಲಿ ಬೋಧಿಸಿರುವರು. ಬ್ರಹ್ಮವನ್ನು ಕುರಿತಾಗಿ ಇರುವ ಹಲವು ಪರಸ್ಪರ ವಿರೋಧವಾದ ಅಭಿಪ್ರಾಯಗಳನ್ನು ಒಂದುಗೂಡಿಸಿ, ಇರುವುದೊಂದೇ ಅನಂತ ಸತ್ಯವೆಂಬುದನ್ನು ತೋರಿರುವರು. ವ್ಯಕ್ತಿಯು ತನ್ನ ಊರ್ಧ್ವಮುಖ ಗಮನದಲ್ಲಿ ನಿಧಾನವಾಗಿ ಮುಂದುವರಿಯಬೇಕಾಗಿರುವುದರಿಂದ, ಹಲವು ರೀತಿಯ ವ್ಯಾಖ್ಯಾನಗಳೆಲ್ಲ ವಿವಿಧ ವ್ಯಕ್ತಿಗಳ ಸಾಮರ್ಥ್ಯಕ್ಕೆ ಆವಶ್ಯಕವೆಂಬುದನ್ನು ತೋರಿದರು. ಏಸುವಿನ ಬೋಧನೆಯಲ್ಲಿ ಕೂಡ ನಾವು ಇದನ್ನೇ ನೋಡುವೆವು, ಶ್ರೋತೃಗಳ ಸಾಮರ್ಥ್ಯಕ್ಕೆ ತಕ್ಕಂತೆ ತನ್ನ ಬೋಧನೆಯನ್ನು ಅವನು ಹೊಂದಿಸಿಕೊಂಡನು. ಮೊದಲು ಸ್ವರ್ಗದಲ್ಲಿರುವ ತಂದೆಯ ವಿಷಯವಾಗಿ ಹೇಳಿ, ಅವನನ್ನು ಪ್ರಾರ್ಥಿಸಿ ಎಂದು ಬೋಧಿಸಿದನು. ಅನಂತರ ಒಂದು ಮೆಟ್ಟಲು ಮೇಲೆ ಹೋಗಿ “ನಾನು ದ್ರಾಕ್ಷಿಯ ಬಳ್ಳಿ, ನೀವು ಅದರ ಟೊಂಗೆಗಳು'' ಎಂದನು. ಕೊನೆಗೆ ಅವರಿಗೆ “ನಾನು ಮತ್ತು ನನ್ನ ತಂದೆ ಒಂದೇ”, “ಸ್ವರ್ಗಲೋಕ ನಿಮ್ಮಲ್ಲಿಯೇ ಇರುವುದು'' ಎಂಬ ಶ್ರೇಷ್ಠ ಸತ್ಯವನ್ನು ಬೋಧಿಸಿದನು. ಶಂಕರಾಚಾರ್ಯರು ಮನುಷ್ಯತ್ವ, ಮುಮುಕ್ಷುತ್ವ ಮತ್ತು ಸತ್ಯವನ್ನು ತೋರಬಲ್ಲ ಮಹಾಗುರುವಿನ ಆಶೀರ್ವಾದ ಈ ಮೂರೂ ಭಗವತ್ ಕೃಪೆಯಿಂದ ಲಭಿಸುವುವು ಎಂದು ಹೇಳುವರು. ಈ ಮೂರು ಸೌಲಭ್ಯಗಳೂ ನಮಗೆ ಒದಗಿದರೆ ನಮ್ಮ ಮುಕ್ತಿ ಕರಗತವಾದಂತೆಯೇ. ಜ್ಞಾನ ಒಂದೇ ನಮ್ಮನ್ನು ಬಂಧನದಿಂದ ಪಾರುಮಾಡಬಲ್ಲದು. ಆದರೆ ಜ್ಞಾನದೊಂದಿಗೆ ಶೀಲವೂ ಇರಬೇಕು.

ಇರುವುದೊಂದೇ, ಪ್ರತಿಯೊಂದು ಜೀವಿಯೂ ಪೂರ್ಣ, ಪ್ರತಿಯೊಂದೂ ಅದೇ ಆಗಿರುವುದು, ಅದರ ಅಂಶವಲ್ಲ ಎಂಬುದೇ ವೇದಾಂತದ ಸಾರ. ಪ್ರತಿಯೊಂದು ಹಿಮಮಣಿಯಲ್ಲಿಯ ಸೂರ್ಯ ಪ್ರತಿಬಿಂಬಿಸುತ್ತಿರುವನು. ಅದೇ ಕಾಲ ದೇಶನಿಮಿತ್ತಗಳ ಮೂಲಕ ತೋರುವಾಗ ನಮಗೆ ಜೀವದಂತೆ ಕಾಣಿಸುವುದು. ಆದರೆ ಈ ತೋರಿಕೆಯ ಹಿಂದೆಲ್ಲ ಇರುವುದೊಂದೇ ಸತ್ಯ, ನಿಃಸ್ವಾರ್ಥತೆ ಎಂದರೆ ನಮ್ಮ ಅಲ್ಪತ್ವವನ್ನು ಅಥವಾ ತೋರಿಕೆಯ ಸ್ವಭಾವವನ್ನು ನಿರಾಕರಿಸುವುದು ಎಂದರ್ಥ. ನಾವು ದೇಹ ಎಂದು ಭಾವಿಸಿರುವ ದುಃಖಕರವಾದ ಸ್ವಪ್ನದಿಂದ ಪಾರಾಗಬೇಕಾಗಿದೆ. “ಸೋಹಂ '' ಎಂಬ ಸತ್ಯವನ್ನು ನಾವು ಅರಿಯಬೇಕು. ಸಿಂಧುವಿನಲ್ಲಿ ಬಿದ್ದು ಮಾಯವಾಗಿ ಹೋಗುವ ಬಿಂದುಗಳಲ್ಲ ನಾವು. ಪ್ರತಿಯೊಬ್ಬರೂ ಪೂರ್ಣವಾದ ಆನಂದಸಾಗರ. ನಾವು ಭ್ರಾಂತಿಬಂಧನದಿಂದ ಪಾರಾದ ಮೇಲೆ ಈ ಸತ್ಯವನ್ನು ಅರಿಯುತ್ತೇವೆ. ಅನಂತವನ್ನು ನಾವು ಭಾವಿಸಲು ಆಗುವುದಿಲ್ಲ. ಏಕಮೇವಾದ್ವಿತೀಯಕ್ಕೆ ಎರಡನೆಯದು ಯಾವುದೂ ಇಲ್ಲ. ಇರುವುದೆಲ್ಲಾ ಒಂದೇ, ಈ ಜ್ಞಾನ ಎಲ್ಲರಿಗೂ ಬರುವುದು. ಆದರೆ ಈಗ ಅದನ್ನು ಪಡೆಯುವುದಕ್ಕಾಗಿ ಹೋರಾಡಬೇಕು. ಏಕೆಂದರೆ ಅದು ನಮಗೆ ಬರುವವರೆಗೆ ಮಾನವತೆಗೆ ಶ್ರೇಷ್ಠ ಸಹಾಯವನ್ನು ನೀಡಲಾರೆವು. ಜೀವನ್ಮುಕ್ತನು ಮಾತ್ರ ನಿಜವಾದ ಪ್ರೀತಿಯನ್ನೂ, ದಾನವನ್ನೂ, ಸತ್ಯವನ್ನೂ ಕೊಡಬಲ್ಲ. ಸತ್ಯವೊಂದೇ ನಮ್ಮನ್ನು ಮುಕ್ತರನ್ನಾಗಿ ಮಾಡಬಲ್ಲುದು. ಆಸೆ ನಮ್ಮನ್ನು ಗುಲಾಮರನ್ನಾಗಿ ಮಾಡುವುದು, ಅದು ಎಂದಿಗೂ ತೃಪ್ತಿಗೊಳ್ಳದ ಕ್ರೂರಿ. ಅದು ತನ್ನ ಬಳಿಗೆ ಬಂದವರಿಗೆ ಎಳ್ಳಷ್ಟೂ ವಿರಾಮ ಕೊಡುವುದಿಲ್ಲ. ಆದರೆ ಜೀವನ್ಮುಕ್ತನು, ತಾನೊಬ್ಬನೇ ಇರುವುದು, ತಾನಿಚ್ಚಿಸುವುದು ಯಾವುದೂ ಇಲ್ಲ ಎಂಬ ಜ್ಞಾನದಿಂದ ಎಲ್ಲಾ ಆಸೆಗಳಿಂದಲೂ ಪಾರಾಗಿರುವನು.

\newpage

ಮನಸ್ಸೇ ದೇಹ, ಲಿಂಗ, ಜಾತಿ, ಮತ, ಬಂಧನ ಮುಂತಾದ ಭ್ರಾಂತಿಗಳನ್ನು ತಂದೊಡ್ಡುವುದು. ಆದಕಾರಣವೇ ಮನಸ್ಸಿಗೆ ಪದೇ ಪದೇ ಈ ಸತ್ಯವನ್ನು ಅದಕ್ಕೆ ಚೆನ್ನಾಗಿ ಮನವರಿಕೆಯಾಗುವ ತನಕ ಒತ್ತಿ ಒತ್ತಿ ಹೇಳುತ್ತಿರಬೇಕು. ನಮ್ಮ ನೈಜ ಸ್ವಭಾವವೇ ಆನಂದ. ನಮಗೆ ತಿಳಿದಿರುವ ಸಂತೋಷವೆಲ್ಲ ಅದರ ಒಂದು ಪ್ರತಿಬಿಂಬ ಮಾತ್ರ, ನಮ್ಮ ನೈಜಸ್ವಭಾವದ ಸ್ಪರ್ಶದಿಂದ ಬರುವ ಆನಂದದ ಪರಮಾಣು ಮಾತ್ರ. ಅದು ಸುಖ ದುಃಖಗಳಾಚೆ ಇದೆ. ಇದೇ ವಿಶ್ವದ ನಿತ್ಯಸಾಕ್ಷಿ; ಬದಲಾಗದೆ ಕುಳಿತಿರುವ ಓದುಗ, ಅವನ ಮುಂದೆ ಜೀವನದ ಪುಸ್ತಕದ ಹಾಳೆಗಳು ತೆರೆಯುತ್ತಾ ಹೋಗುವುವು.

ಸಾಧನದಿಂದ ಯೋಗಪ್ರಾಪ್ತಿ, ಯೋಗದಿಂದ ಜ್ಞಾನ, ಜ್ಞಾನದಿಂದ ಪ್ರೇಮ, ಪ್ರೇಮದಿಂದ ಆನಂದ.

“ನಾನು, ನನ್ನದು” ಎಂಬುದೇ ಮೌಢ್ಯ. ನಾವು ಅದರಲ್ಲಿ ಬಹಳ ಕಾಲವಿದ್ದು ಅದರಿಂದ ಪಾರಾಗದಷ್ಟು ದುಃಸ್ಥಿತಿಗೆ ಬಂದಿರುವೆವು. ಆದರೂ ನಾವು ಅತ್ಯುನ್ನತವಾದುದನ್ನು ಪಡೆಯಬೇಕಾದರೆ ಅದರಿಂದ ಪಾರಾಗಲೇಬೇಕು. ನಾವು ಹಸನ್ಮುಖಿಗಳಾಗಿರಬೇಕು, ಮಂದಹಾಸದಿಂದ ಕೂಡಿರಬೇಕು. ಅಳುಮೋರೆಯಿಂದ ನಾವು ಆಧ್ಯಾತ್ಮಿಕ ಜೀವಿಗಳಾಗಲಾರೆವು. ಅಧ್ಯಾತ್ಮ ಪರಮಾನಂದಭರಿತವಾಗಿರಬೇಕು. ಏಕೆಂದರೆ ಅದು ಶ್ರೇಷ್ಠ, ಕೇವಲ ದೇಹದಂಡನೆ ನಮ್ಮನ್ನು ಪವಿತ್ರರನ್ನಾಗಿ ಮಾಡಲಾರದು. ದೇವರನ್ನು ಪ್ರೀತಿಸುವ ಭಕ್ತ, ಪರಿಶುದ್ಧನಾದ ಭಕ್ತ, ಏತಕ್ಕೆ ಅಳುಮೋರೆಯಿಂದ ಕೂಡಿರಬೇಕು? ಅವನು ಸಂತೋಷದಿಂದ ನಲಿಯುತ್ತಿರುವ ಮಗುವಿನಂತಿರಬೇಕು ನಿಜವಾಗಿ ದೇವರ ಮಗುವಾಗಿರಬೇಕು. ಧರ್ಮದ ಸಾರವೇ ಹೃದಯವನ್ನು ಪರಿಶುದ್ಧಗೊಳಿಸುವುದು. ಪರಂಧಾಮ ನಮ್ಮ ಹೃದಯದಲ್ಲೇ ಇರುವುದು. ಪರಿಶುದ್ಧಾತ್ಮರು ಮಾತ್ರ ಭಗವಂತನನ್ನು ಅಲ್ಲಿ ನೋಡಬಹುದು. ನಾವು ಪ್ರಪಂಚವನ್ನು ಕುರಿತು ಯೋಚಿಸುವಾಗ ಅದು ನಮಗೆ ಕೇವಲ ಪ್ರಪಂಚದಂತೆ ಕಾಣಿಸುವುದು. ಆದರೆ ಇದು ದೇವರು ಎಂಬ ಭಾವನೆಯಿಂದ ಅದನ್ನು ನೋಡಿದರೆ ನಮಗೆ ದೇವರು ದೊರಕುವನು. ಎಲ್ಲಾ ವಸ್ತುಗಳನ್ನೂ, ಎಲ್ಲಾ ವ್ಯಕ್ತಿಗಳನ್ನೂ, ತಾಯಿ ತಂದೆ ಮಕ್ಕಳು ಗಂಡ ಹೆಂಡತಿ ಸ್ನೇಹಿತರು ಶತ್ರುಗಳು ಎಲ್ಲರನ್ನೂ ನಾವು ಈ ದೃಷ್ಟಿಯಿಂದ ನೋಡಬೇಕು. ನಾವು ಎಲ್ಲವನ್ನೂ ಹೃತ್ತೂರ್ವಕವಾಗಿ ಭಗವಂತನಿಂದ ತುಂಬಿದರೆ ಆಗ ಅದು ಹೇಗೆ ಬದಲಾಗುವುದೆಂಬುದನ್ನು ಕುರಿತು ಯೋಚಿಸಿ. ದೇವರಲ್ಲದೆ ಮತ್ತೇನನ್ನೂ ನೋಡಬೇಡಿ. ಆಗ ದುಃಖ ಹೋರಾಟ ವ್ಯಥೆಗಳು – ಇವೆಲ್ಲ ಎಂದೆಂದಿಗೂ ಕೊನೆಗೊಂಡಂತೆ.

ಜ್ಞಾನವು ಪಂಥಾತೀತವಾದುದು. ಆದರೆ ಅದು ಯಾವುದೇ ಪಂಥವನ್ನೂ ನಿರಾಕರಿಸುವುದಿಲ್ಲ. ಅಂದರೆ ಅದು ಪಂಥಾತೀತ ಅವಸ್ಥೆಯನ್ನು ಪಡೆದಿದೆ ಎಂದು ಅರ್ಥ. ಜ್ಞಾನಿ ಧ್ವಂಸಮಾಡಲು ಇಚ್ಛಿಸುವುದಿಲ್ಲ. ಎಲ್ಲರಿಗೂ ನೆರವು ನೀಡುವನು. ನದಿಗಳೆಲ್ಲ ಸಾಗರಕ್ಕೆ ಸೇರಿ ಒಂದಾಗುವಂತೆ ಪಂಥಗಳೆಲ್ಲ ಜ್ಞಾನದಲ್ಲಿ ಐಕ್ಯವಾಗಿ ಒಂದಾಗಬೇಕು.

\newpage

ಎಲ್ಲದರ ಅಸ್ತಿತ್ವ ಬ್ರಹ್ಮನ ಅಸ್ತಿತ್ವದ ಮೇಲೆ ನಿಂತಿದೆ. ಹೀಗೆ ನಾವು ನೋಡಿದರೆ ಮಾತ್ರ ನಮಗೆ ಸತ್ಯ ಸಿಕ್ಕಬಹುದು. ವೈವಿಧ್ಯವನ್ನು ಎಂದು ನಾವು ನೋಡುವುದಿಲ್ಲವೋ ಆಗ, ನಾನು ಮತ್ತು ನನ್ನ ತಂದೆ ಇಬ್ಬರೂ ಒಂದೇ ಎನ್ನುವುದು ಗೊತ್ತಾಗುವುದು.

ಕೃಷ್ಣನು ಗೀತೆಯಲ್ಲಿ ಅತ್ಯಂತ ಸ್ಪಷ್ಟವಾಗಿ ಜ್ಞಾನವನ್ನು ಬೋಧಿಸಿರುವನು. ಈ ಮಹಾಕಾವ್ಯವನ್ನು ಸಾಹಿತ್ಯಭಾರತ ಚೂಡಾಮಣಿ ಎನ್ನುವರು. ಇದೊಂದು ಬಗೆಯ ವೇದಭಾಷ್ಯ, ಆಧ್ಯಾತ್ಮಿಕ ಸಂಗ್ರಾಮವನ್ನು ನಾವು ಈ ಜೀವನದಲ್ಲಿಯೇ ಎದುರಿಸಬೇಕು ಎಂಬುದನ್ನು ಅದು ತೋರಿಸುತ್ತದೆ. ಇಲ್ಲಿ ಬೆನ್ನು ತೋರಲಾಗದು. ಅದರಲ್ಲಿರುವುದನ್ನೆಲ್ಲಾ ನಮಗೆ ಕೊಡುವಂತೆ ಬಲಾತ್ಕರಿಸಬೇಕು ಎನ್ನುವುದು ಗೀತೆ, ಜೀವನದಲ್ಲಿ ಮಹೋನ್ನತ ಆದರ್ಶಸಾಧನೆಗೆ ನಡೆಯುವ ಸಂಗ್ರಾಮದ ಪ್ರತೀಕದಂತೆ ಇದೆ ಗೀತೆ. ಆದುದರಿಂದ ಅದನ್ನು ಸಮರಾಂಗಣದ ಹಿನ್ನೆಲೆಯಲ್ಲಿ ಚಿತ್ರಿಸಿರುವುದು ಕಾವ್ಯಪೂರ್ಣವಾಗಿಯೇ ಇದೆ. ಒಂದಕ್ಕೊಂದು ವಿರೋಧಿಸುವ ಎರಡು ಪಕ್ಷಗಳಲ್ಲಿ ಒಂದು ಪಕ್ಷದ ನಾಯಕನಾದ ಅರ್ಜುನನಿಗೆ ಸಾರಥಿಯಂತೆ ಇದ್ದ ಕೃಷ್ಣ. ಅವನು ಅರ್ಜುನನಿಗೆ “ವಿಷಾದಪಡಬೇಡ, ಸಾವಿಗೆ ಅಂಜಬೇಡ, ಏಕೆಂದರೆ ನಿಜವಾಗಿಯೂ ನೀನು ಅಮೃತಸ್ವರೂಪನು, ವಿಕಾರವಾಗುವ ಯಾವುದೂ ಮನುಷ್ಯನ ಸಹಜ ಸ್ಥಿತಿಯಲ್ಲ'' ಎಂದು ಬೋಧಿಸುವನು. ಕೃಷ್ಣನು ಪ್ರತಿಯೊಂದು ಅಧ್ಯಾಯದಲ್ಲಿಯೂ ತತ್ತ್ವದ ಮತ್ತು\break ಧರ್ಮದ ಶ್ರೇಷ್ಠ ಸಿದ್ದಾಂತಗಳನ್ನು ವಿವರಿಸುತ್ತಾನೆ. ಈ ಬೋಧನೆಗಳಿಂದಲೇ ಗೀತೆ ಅಷ್ಟು ಅದ್ಭುತವಾಗಿರುವುದು, ವೇದಾಂತದ ಬಹುಪಾಲು ತತ್ತ್ವವೆಲ್ಲ ಇದರಲ್ಲಿ ಹುದುಗಿದೆ. ಆತ್ಮ ಅನಂತವಾದುದು; ದೇಹನಾಶದಿಂದ ಅದು ಯಾವ ವಿಕಾರಕ್ಕೂ ಒಳಗಾಗುವುದಿಲ್ಲವೆಂದು ವೇದಗಳು ಬೋಧಿಸುತ್ತವೆ. ಆತ್ಮ ಒಂದು ವೃತ್ತದಂತೆ. ಅದರ ಪರಿಧಿ ಎಲ್ಲೂ ಇಲ್ಲ. ಆದರೆ ಅದರ ಕೇಂದ್ರ ಒಂದು ದೇಹದಲ್ಲಿದೆ. ಮೃತ್ಯುವೆಂದರೆ ಕೇಂದ್ರದ ಬದಲಾವಣೆ ಮಾತ್ರ, ದೇವರು, ಎಲ್ಲೂ ಪರಿಧಿ ಇಲ್ಲದ, ಎಲ್ಲೆಲ್ಲೂ ಕೇಂದ್ರಗಳಿರುವ ಒಂದು ವೃತ್ತದಂತೆ. ನಾವು ದೇಹವೆಂಬ ಸಂಕುಚಿತ ಕೇಂದ್ರದಿಂದ ಪಾರಾದರೆ ನಮ್ಮ ನಿಜಸ್ವರೂಪವಾದ ಬ್ರಹ್ಮವನ್ನು ಕಾಣುತ್ತೇವೆ.

ವರ್ತಮಾನವೆಂಬುದು ಭೂತ ಭವಿಷ್ಯತ್ತುಗಳನ್ನು ಬೇರ್ಪಡಿಸುವ ಒಂದು ರೇಖೆ. ಆದಕಾರಣ ನಾವು ವರ್ತಮಾನವನ್ನು ಮಾತ್ರ ಲೆಕ್ಕಿಸುತ್ತೇವೆ ಎಂದು ಯುಕ್ತಿ ಪೂರ್ವಕವಾಗಿ ಹೇಳಲಾಗುವುದಿಲ್ಲ. ಏಕೆಂದರೆ ಭೂತ ಭವಿಷ್ಯತ್ತುಗಳಿಲ್ಲದೆ ವರ್ತಮಾನ ಮಾತ್ರ ನಿಲ್ಲಲಾರದು. ಇದೆಲ್ಲ ಒಂದು ಅಖಂಡವಾದುದು. ಕಾಲದ ಕಲ್ಪನೆ ನಮ್ಮ ಗ್ರಹಣಶಕ್ತಿಯು ನಮ್ಮ ಮೇಲೆ ಹೇರಿರುವ ಒಂದು ನಿಯಮ.

\begin{center}
೩
\end{center}

ಜಗತ್ತನ್ನು ತ್ಯಜಿಸಬೇಕೆಂದು ಜ್ಞಾನವು ಬೋಧಿಸುವುದು; ಆದರೆ ಆ ಕಾರಣಕ್ಕೆ ಜಗತ್ತನ್ನು ನಾವು ಧಿಕ್ಕರಿಸಬೇಕೆಂದು ಅದು ಬೋಧಿಸುವುದಿಲ್ಲ. ಜಗತ್ತಿನಲ್ಲಿದ್ದು ಅದಕ್ಕೆ ಸೇರದಿರುವುದೇ ಸಂನ್ಯಾಸಿಗೆ ನಿಜವಾದ ಪರೀಕ್ಷೆ. ಎಲ್ಲಾ ಧರ್ಮಗಳಲ್ಲಿಯೂ ಒಂದಲ್ಲ ಒಂದು ರೂಪದಲ್ಲಿ ಇಂತಹ ಒಂದು ತ್ಯಾಗವನ್ನು ನಾವು ಸಾಧಾರಣವಾಗಿ ನೋಡುತ್ತೇವೆ. ಎಲ್ಲರನ್ನೂ ಒಂದೇ ಸಮನಾಗಿ ಕಾಣುವಂತೆ, ಸಮತ್ವದರ್ಶನ ಮಾಡುವಂತೆ ಜ್ಞಾನ ಬೋಧಿಸುವುದು. ನಿಂದೆ–ಸ್ತುತಿ, ಶುಭ–ಅಶುಭ, ಶೀತೋಷ್ಣ ಇವುಗಳನ್ನು ಕೂಡ ನಾವು ಒಂದೇ ದೃಷ್ಟಿಯಿಂದ ಸ್ವೀಕರಿಸಲು ಸಿದ್ಧರಾಗಿರಬೇಕು. ಭರತಖಂಡದಲ್ಲಿ ಎಷ್ಟೋ ಜನ ಮಹಾತ್ಮರಿರುವರು; ಅವರ ಜೀವನದಲ್ಲಿ ಈ ಸತ್ಯ ಅಕ್ಷರಶಃ ನಿಜವಾಗಿರುವುದು. ಅವರು ಹಿಮಾಲಯದ ಮಂಜಿನ ಮೇಲೆ ಅಥವಾ ದಹಿಸುತ್ತಿರುವ ಮರಳುಕಾಡಿನ ಮೇಲೆ ಬಟ್ಟೆಬರೆಗಳಿಲ್ಲದೆ, ಹೊರಗಿನ ಶೀತೋಷ್ಣಗಳನ್ನು ಗಮನಕ್ಕೆ ತರದೆ ಸಂಚರಿಸುವರು.

ಮೊದಲು, ನಾವು ದೇಹ ಎಂಬ ಮೂಢನಂಬಿಕೆಯನ್ನು ತ್ಯಜಿಸಬೇಕು. ನಾವು ದೇಹವಲ್ಲ. ಅನಂತರ, ನಾವು ಮನಸ್ಸು ಎಂಬ ಮತ್ತೊಂದು ಮೂಢನಂಬಿಕೆ ಹೋಗಬೇಕು. ನಾವು ಮನಸ್ಸಲ್ಲ, ಅದು ಸೂಕ್ಷ್ಮಶರೀರ, ಆತ್ಮನ ಭಾಗವಲ್ಲ. ಎಲ್ಲ ವಸ್ತುಗಳನ್ನೂ `ದೇಹ' ಎಂಬ ಶಬ್ದದಿಂದ ಕರೆಯುವುದರಿಂದ, ಎಲ್ಲ ವಸ್ತುಗಳಲ್ಲಿಯೂ ಸಾಮಾನ್ಯವಾದ ಒಂದು ತತ್ತ್ವವಿದೆ ಎಂದರ್ಥವಾಗುತ್ತದೆ. ಇದೇ ಅಸ್ತಿತ್ವ. ನಮ್ಮ ದೇಹ, ಅದರ ಹಿಂದೆ ಇರುವ ಚಿಂತನೆಯ ಸಂಕೇತ. ಆ ಚಿಂತನೆಗಳು ಕೂಡ, ಅವುಗಳ ಹಿಂದೆ ಇರುವ ಮತ್ತೊಂದರ ಸಂಕೇತ. ಅದೇ ಏಕಮಾತ್ರ ಅಸ್ತಿತ್ವ, ಆತ್ಮದ ಆತ್ಮ, ವಿಶ್ವಾತ್ಮ, ನಮ್ಮ ಪ್ರಾಣದ ಪ್ರಾಣ, ನಮ್ಮ ನೈಜಸ್ಥಿತಿ. ನಾವು ಎಲ್ಲಿಯವರೆಗೂ ದೇವರಿಂದ ಸ್ವಲ್ಪವಾದರೂ ಬೇರೆ ಎಂದು ಭಾವಿಸುವೆವೋ ಅಲ್ಲಿಯವರೆಗೂ ಭಯ ನಮ್ಮಲ್ಲಿರುವುದು. ಆದರೆ ನಾವೇ ಆ ಒಂದು ಎಂದು ಅರಿತರೆ ಅಂಜಿಕೆ ತೊಲಗುವುದು. ನಾವು ಅಂಜುವುದೇತಕ್ಕೆ? ಕೇವಲ ತನ್ನ ಇಚ್ಛಾಶಕ್ತಿಯಿಂದ ಜ್ಞಾನಿಯು ಜಗತ್ತನ್ನು ಶೂನ್ಯಗೊಳಿಸಿ, ತನ್ನ ದೇಹ–ಮನಸ್ಸುಗಳನ್ನು ಅತಿಕ್ರಮಿಸಿ ಹೋಗುವನು. ಅವನು ಹೀಗೆ ಅವಿದ್ಯೆಯನ್ನು ನಾಶಮಾಡಿ ತನ್ನ ನೈಜಸ್ಥಿತಿಯನ್ನು ಅರಿಯುವನು. ಸುಖದುಃಖಗಳು ಕೇವಲ ನಮ್ಮ ಇಂದ್ರಿಯಗಳಲ್ಲಿವೆ. ಅವು ನಮ್ಮ ಆತ್ಮವನ್ನು ಸೋಂಕಲಾರವು. ಆತ್ಮ ದೇಶ–ಕಾಲ–ನಿಮಿತ್ತಗಳಾಚೆ ಇರುವುದು; ಅದು ಅನಂತ, ವಿಭುವಾಗಿರುವುದು.

ಜ್ಞಾನಿಯು ಎಲ್ಲಾ ರೂಪಗಳಿಂದ, ಎಲ್ಲಾ ನಿಯಮಗಳಿಂದ, ಶಾಸ್ತ್ರಗಳಿಂದ ಪಾರಾಗಬೇಕು. ಅವನು ತಾನೇ ಸ್ವತಃ ತನ್ನ ಶಾಸ್ತ್ರವಾಗಿರಬೇಕು. ರೂಪದಿಂದ ಬಂಧಿತರಾಗಿ, ಅದರಂತಾಗಿ ನಾವು ಸಾಯುವೆವು. ಆದರೂ ಜ್ಞಾನಿಯಾದವನು, ಯಾರು ರೂಪವನ್ನು ಮೀರಿ ಹೋಗಲಾರರೋ ಅವರನ್ನು ದೂರಕೂಡದು. ಇನ್ನು ಮೇಲೆ ಅವನು “ನಿನಗಿಂತ ನಾನು ಹೆಚ್ಚು ಪವಿತ್ರನಾದವನು" ಎಂದು ಭಾವಿಸಬಾರದು.

ನಿಜವಾದ ಜ್ಞಾನಿಯ ಲಕ್ಷಣಗಳು ಇವು: (೧) ಜ್ಞಾನವಲ್ಲದೆ ಮತ್ತೇನನ್ನೂ ಅವನು ಇಚ್ಚಿಸುವುದಿಲ್ಲ. (೨) ಅವನ ಇಂದ್ರಿಯಗಳೆಲ್ಲ ಸಂಪೂರ್ಣ ನಿಗ್ರಹದಲ್ಲಿವೆ. ಅವನು ಗೊಣಗಾಡದೆ ಎಲ್ಲವನ್ನೂ ಅನುಭವಿಸುವನು. ಅವನಿಗೆ ನೆಲವೇ ಹಾಸಿಗೆಯಾಗಿ\break ಆಕಾಶವೇ ಹೊದಿಕೆಯಾದರೂ ಅವನು ಚಿಂತಿಸುವುದಿಲ್ಲ. ಅವನು ರಾಜನ ಅರಮನೆಯಲ್ಲಿದ್ದರೂ ಚಿಂತಿಸುವುದಿಲ್ಲ. ಅವನು ಯಾವ ವ್ಯಥೆಯನ್ನೂ ನಿರಾಕರಿಸುವುದಿಲ್ಲ.\break ಅವನು ಅದನ್ನೆಲ್ಲಾ ಸಹಿಸುವನು. ಅವನು ಆತ್ಮೇತರ ವಸ್ತುಗಳನ್ನೆಲ್ಲಾ ತ್ಯಜಿಸಿರುವನು. (೩) ಅವನಿಗೆ ಆತ್ಮವನ್ನು ಬಿಟ್ಟು ಉಳಿದುವೆಲ್ಲ ಭ್ರಾಂತಿ ಎಂದು ಗೊತ್ತಿದೆ. (೪) ಮುಕ್ತನಾಗಬೇಕೆಂಬ ಉತ್ಕಟೇಚ್ಛೆ ಅವನಲ್ಲಿರುವುದು. ಪ್ರಚಂಡ ಇಚ್ಛಾಶಕ್ತಿಯಿಂದ ಮನಸ್ಸನ್ನು ಶ್ರೇಷ್ಠ ವಸ್ತುಗಳ ಕಡೆ ತಿರುಗಿಸಿ ಶಾಂತಿಯನ್ನು ಪಡೆಯುವನು. ನಮಗೆ ಶಾಂತಿ ಇಲ್ಲದಿದ್ದರೆ ನಾವು ಮೃಗಗಳಿಗಿಂತ ಯಾವ ರೀತಿಯಲ್ಲಿ ಮೇಲು? ಜ್ಞಾನಿ ಸರ್ವವನ್ನೂ ಇತರರಿಗಾಗಿ ಮಾಡುವನು, ದೇವರಿಗಾಗಿ ಮಾಡುವನು, ಕರ್ಮಫಲಗಳನ್ನು ತ್ಯಜಿಸುವನು. ಇಹದಲ್ಲಿ ಮತ್ತು ಪರದಲ್ಲಿ ಯಾವ ಪ್ರತಿಫಲವನ್ನೂ ಆಶಿಸುವುದಿಲ್ಲ. ಆತ್ಮನಿಗಿಂತ ಹೆಚ್ಚಾಗಿ ಈ ಪ್ರಪಂಚ ನಮಗೆ ಏನನ್ನು ಕೊಡಬಲ್ಲದು? ಅದೊಂದಿದ್ದರೆ ನಮಗೆ ಎಲ್ಲವೂ ಇರುವುದು. ಆತ್ಮವು ಅವಿಭಕ್ತವೆಂದು ವೇದಗಳು ಸಾರುವುವು. ಅದು ಮನಸ್ಸು, ನೆನಪು, ಆಲೋಚನೆ ಅಥವಾ ನಮಗೆ ತಿಳಿದ ಪ್ರಜ್ಞೆಗೆ ಅತೀತವಾಗಿರುವುದು. ಅದರಿಂದ ಎಲ್ಲವೂ ಉದಯಿಸುವುದು. ಯಾವುದರಿಂದ ನಾವು ನೋಡುವೆವೋ, ಕೇಳುವೆವೋ, ಅನುಭವಿಸುವೆವೋ, ಆಲೋಚಿಸುವೆವೋ ಅದೇ ಆತ್ಮ. ವಿಶ್ವದ ಗಮ್ಯಸ್ಥಾನವೇ ಓಂಕಾರ ಅಥವಾ ಏಕ ಅಸ್ತಿತ್ವದೊಂದಿಗೆ ಏಕತೆಯನ್ನು ಪಡೆಯುವುದು. ಜ್ಞಾನಿ, ಎಲ್ಲ ಬಾಹ್ಯ ರೂಪಗಳನ್ನೂ ಮೀರಿರಬೇಕು. ಅವನು ಹಿಂದುವೂ ಅಲ್ಲ, ಬೌದ್ಧನೂ ಅಲ್ಲ, ಕ್ರಿಸ್ತನೂ ಅಲ್ಲ, ಇವೆಲ್ಲವೂ ಆಗಿರುವನು. ಅವನು ಕರವನ್ನೆಲ್ಲಾ ತ್ಯಜಿಸಿರುವನು, ಭಗವಂತನಿಗೆ ಅರ್ಪಿಸಿರುವನು. ಆಗ ಯಾವ ಕರ್ಮವೂ ಅವನನ್ನು ಬಂಧಿಸಲಾರದು. ಜ್ಞಾನಿ ಪ್ರಚಂಡ ವಿಚಾರವಾದಿ, ಅವನು ಎಲ್ಲವನ್ನೂ ಅಲ್ಲಗಳೆಯುವನು. ಅವನು ಹಗಲು ರಾತ್ರಿ "ಮಂತ್ರವಿಲ್ಲ, ನಂಬಿಕೆಯಿಲ್ಲ, ಸ್ವರ್ಗನರಕಗಳಿಲ್ಲ, ದೇವಾಲಯಗಳಿಲ್ಲ, ಜಾತಿಕುಲಗಳಿಲ್ಲ, ಆತ್ಮವೊಂದೇ ಇರುವುದು" ಎಂದು ಮನನ ಮಾಡುವನು. ಎಲ್ಲವನ್ನೂ ತ್ಯಜಿಸಿದ ಮೇಲೆ, ತ್ಯಜಿಸಲು ಆಗದ ಯಾವ ಒಂದನ್ನು ಸೇರುತ್ತೇವೋ, ಅದೇ ಆತ್ಮ, ಜ್ಞಾನಿ ಯಾವುದನ್ನೂ ಸುಮ್ಮಸುಮ್ಮನೆ ಸ್ವೀಕರಿಸುವುದಿಲ್ಲ. ಶುದ್ದ ಯುಕ್ತಿಯಿಂದ ಮತ್ತು ಪ್ರಚಂಡ ಇಚ್ಛಾಶಕ್ತಿಯಿಂದ ಪ್ರತಿಯೊಂದನ್ನೂ, ನಿರ್ವಾಣವನ್ನು ಪಡೆಯುವತನಕ, ವಿಶ್ಲೇಷಣೆ ಮಾಡುವನು. ನಿರ್ವಾಣವೆಂದರೆ ಎಲ್ಲ ಸಾಪೇಕ್ಷಗಳ ಅಳಿವು. ಈ ಸ್ಥಿತಿಯನ್ನು ವಿವರಿಸುವುದಕ್ಕೆ ಆಗುವುದಿಲ್ಲ, ಊಹಿಸುವುದಕ್ಕೂ ಆಗುವುದಿಲ್ಲ. ಜ್ಞಾನವನ್ನು ಪ್ರಾಪಂಚಿಕ ಪ್ರತಿಫಲಗಳ ದೃಷ್ಟಿಯಿಂದ ಎಂದೂ ಅಳೆಯಬಾರದು. ಕಣ್ಣಿಗೆ ಕಾಣಿಸದಷ್ಟು ಮೇಲೆ ಹಾರಿಹೋದರೂ ಕೊಳೆತು ನಾರುವ ಮಾಂಸದ ಚೂರೊಂದಕ್ಕಾಗಿ ತಕ್ಷಣ ನೆಲಕ್ಕೆ ಇಳಿದುಬರುವ ರಣಹದ್ದಿನಂತೆ ಇರಬೇಡಿ. ದೀರ್ಘಾಯುಷ್ಯ, ಐಶ್ವರ್ಯ, ವ್ಯಾಧಿಗಳನ್ನು ಗುಣ ಮಾಡುವುದು ಇತ್ಯಾದಿಗಳನ್ನು ಕೇಳಬೇಡಿ. ಕೇವಲ ಮುಕ್ತಿಯನ್ನು ಮಾತ್ರ ಬೇಡಿ.

ನಾವು ಸಚ್ಚಿದಾನಂದ ಸ್ವರೂಪರು. ವಿಶ್ವದ ಕೊನೆಯ ಸಾಮಾನೀಕರಣವೇ ಸತ್. ಅದಕ್ಕೇ ನಾವಿರುವುದು. ಅದು ನಮಗೆ ಗೊತ್ತಿದೆ. ಶುದ್ಧ ಸತ್ಯದ ನೈಜಸ್ಥಿತಿಯೇ ಆನಂದ. ಆಗಾಗ ಪರಮಾನಂದದ ಕ್ಷಣಿಕ ಅನುಭವ ನಮಗೆ ಆಗುವುದು. ಆಗ ನಾವು ಏನನ್ನೂ ಕೇಳುವುದಿಲ್ಲ, ಏನನ್ನೂ ಕೊಡುವುದಿಲ್ಲ, ಪರಮಾನಂದವಲ್ಲದೆ ನಮಗೇನೂ ಗೊತ್ತಿರುವುದಿಲ್ಲ. ಅದು ಪುನಃ ಮಾಯವಾಗುವುದು. ಪುನಃ ಈ ಸೃಷ್ಟಿಯ ಲೀಲೆ ನಮ್ಮ ಕಣ್ಣ ಮುಂದೆ ಸುಳಿಯುವುದು. ಆಗ ಇವೆಲ್ಲ ದೇವರೆಂಬ ಏಕಮಾತ್ರ ಸತ್ಯದ ತಳಹದಿಯ ಮೇಲೆ ಮೂಡಿಸಿದ ಚಿತ್ರಕಲೆಯಂತೆ ತೋರುವುದು. ನಾವು ಪುನಃ ಈ ಜಗತ್ತಿಗೆ ಬಂದು, ನಿರಪೇಕ್ಷವನ್ನು ಸಾಪೇಕ್ಷದಂತೆ ನೋಡಿದಾಗ, ಸಚ್ಚಿದಾನಂದವು ಮೂರು ಕವಲೊಡೆದಂತೆ ಕಾಣುವುದು. ಅದೇ ತಂದೆ, ಮಗ ಮತ್ತು “ಹೋಲಿ ಗೋಸ್ಟ್ (ಪವಿತ್ರಾತ್ಮ)'', ಸತ್ ಅಂದರೆ ಸೃಷ್ಟಿ ತತ್ತ್ವ, ಚಿತ್ ಅಂದರೆ ಮಾರ್ಗದರ್ಶಕ ತತ್ತ್ವ, ಆನಂದ ಎಂದರೆ ಸಾಕ್ಷಾತ್ಕಾರ ತತ್ತ್ವ, ಎಂದರೆ ಏಕದೊಡನೆ ಐಕ್ಯವಾಗುವುದು. ಯಾರಿಗೂ ಸತ್ ಎಂಬುದು ಚಿತ್ತಿನ ಸಹಾಯವಿಲ್ಲದೆ ಗೋಚರಿಸುವುದಿಲ್ಲ. ಆದಕಾರಣವೇ ಏಸುವು “ಮಗನ ಮೂಲಕ ಅಲ್ಲದೆ ಯಾರೂ ತಂದೆಯನ್ನು ನೋಡಲಾರರು" ಎಂದು ಹೇಳುವುದು. ಈಗ, ಇಲ್ಲಿ, ನಿರ್ವಾಣವನ್ನು ಪಡೆಯಬಹುದು; ಅದಕ್ಕಾಗಿ ಸಾಯುವವರೆಗೂ ಕಾಯಬೇಕಾಗಿಲ್ಲ ಎನ್ನುವುದು, ವೇದಾಂತ. ನಿರ್ವಾಣವೇ ಆತ್ಮಸಾಕ್ಷಾತ್ಕಾರ. ಒಂದು ಕ್ಷಣವಾದರೂ ಅದನ್ನು ಸಾಕ್ಷಾತ್ಕರಿಸಿಕೊಂಡರೆ ಈ ಭ್ರಾಂತಿಯಿಂದ ಹುಟ್ಟಿದ ವ್ಯಕ್ತಿತ್ವದಿಂದ ವ್ಯಕ್ತಿ ಪುನಃ ಬದ್ದನಾಗಲಾರನು. ನಮಗೆ ಕಣ್ಣಿರುವುದರಿಂದ ನಾವು ತೋರಿಕೆಯ ಜಗತ್ತನ್ನು ನೋಡಬೇಕಾಗಿದೆ. ಆದರೆ ನೋಡುವಾಗಲೆಲ್ಲಾ ಅದರ ನೈಜಸ್ಥಿತಿ ಅರಿವಾಗುತ್ತದೆ. ಅದರ ಸತ್ತ್ವರೂಪ ನಮಗೆ ಗೊತ್ತಾಗಿದೆ. ಅದು ವಿಕಾರವಾಗದೆ ಇರುವ ಆತ್ಮನನ್ನು ಮರೆಮಾಡಿರುವ ತೆರೆ, ತೆರೆ ಜಾರಿದಾಗ ಅದರ ಹಿಂದೆ ಇರುವ ಆತ್ಮ ಗೋಚರಿಸುವುದು, ಬದಲಾವಣೆಯೆಲ್ಲಾ ತೆರೆಯಲ್ಲಿದೆ, ಪುಣ್ಯಾತ್ಮನಲ್ಲಿ ತೆರೆಯು ತುಂಬ ತೆಳ್ಳಗಿರುವುದು. ಸತ್ಯವು ಹೆಚ್ಚು ಆತಂಕವಿಲ್ಲದೆ ಅವನ ಮೂಲಕ ವ್ಯಕ್ತವಾಗುವುದು. ಪಾಪಾತ್ಮನಲ್ಲಿ ಪುಣ್ಯಾತ್ಮನಲ್ಲಿರುವಂತೆಯೇ ಆತ್ಮನಿರುವುದು. ಆದರೆ ತೆರೆ ದಟ್ಟವಾಗಿರುವುದರಿಂದ ಅದು ನಮಗೆ ಗೋಚರವಾಗುವುದಿಲ್ಲ.

ಏಕತೆ ಕಂಡಾಗ ಮಾತ್ರ ಎಲ್ಲ ವಿಚಾರಕ್ರಿಯೆ ನಿಲ್ಲುವುದು. ಆದಕಾರಣವೇ ಮೊದಲು ನಾವು ವಿಶ್ಲೇಷಣೆಮಾಡಿ ಅನಂತರ ಸಮುಚ್ಚಯಗೊಳಿಸಬೇಕು. ವಿಜ್ಞಾನಪ್ರಪಂಚದಲ್ಲಿ ಸರ್ವವ್ಯಾಪಿಯಾದ ಒಂದು ಶಕ್ತಿಯನ್ನು ಅನ್ವೇಷಣೆ ಮಾಡುವಾಗ ಭಿನ್ನಭಿನ್ನ ಶಕ್ತಿಗಳಿಗೆ ಇರುವ ವೈವಿಧ್ಯ ಕ್ರಮೇಣ ಕಡಿಮೆಯಾಗುತ್ತ ಬರುವುದು. ಭೌತಿಕಶಾಸ್ತ್ರವು ಅಂತಿಮ ಐಕ್ಯವನ್ನು ಅರಿತಾಗ ಅದು ತನ್ನ ಗುರಿಯನ್ನು ಮುಟ್ಟಿದಂತೆ. ಆ ಏಕತೆಯಲ್ಲೇ ಶಾಂತಿ ನೆಲಸಿರುವುದು. ಜ್ಞಾನವೇ ಪರಮಗುರಿ.

ವಿಜ್ಞಾನಗಳಲ್ಲೆಲ್ಲಾ ಅತಿ ಉತ್ಕೃಷ್ಟವಾದುದು ಧರ್ಮ. ಅದು ಬಹಳ ಹಿಂದೆಯೇ ಅಂತಿಮ ಏಕತೆಯನ್ನು ಕಂಡುಹಿಡಿಯಿತು. ಅದನ್ನು ಸೇರುವುದೇ ಜ್ಞಾನಯೋಗದ ಗುರಿ. ವಿಶ್ವದಲ್ಲಿರುವುದು ಒಂದು ಆತ್ಮ, ಇತರ ಕೆಳಗಿನ ಜೀವಗಳೆಲ್ಲ ಅದರ ಆವಿರ್ಭಾವಗಳು. ಆತ್ಮವು ಈ ಆವಿರ್ಭಾವಗಳೆಲ್ಲವನ್ನೂ ಮೀರಿ ಅನಂತವಾಗಿರುವುದು. ಸರ್ವವೂ ಬ್ರಹ್ಮಮಯ – ಪಾಪಿ, ಯತಿ, ಕುರಿಮರಿ, ವ್ಯಾಘ್ರ, ಕೊಲೆಪಾತಕಿ ಕೂಡ. ಅವರಲ್ಲಿ ಏನಾದರೂ ಸತ್ಯವಿದ್ದರೆ, ಬ್ರಹ್ಮನಲ್ಲದೇ ಬೇರೆಯಲ್ಲ. ಏಕೆಂದರೆ ಅದಲ್ಲದೆ ಬೇರಾವುದೂ ಇಲ್ಲವೇ ಇಲ್ಲ. “ಏಕಂ ಸತ್ ವಿಪ್ರಾ ಬಹುಧಾ ವದನ್ತಿ.” ಈ ಜ್ಞಾನಕ್ಕಿಂತ ಮಿಗಿಲಾದುದು\break ಮತ್ತಾವುದೂ ಇಲ್ಲ. ಯಾರು ಯೋಗದಿಂದ ಪರಿಶುದ್ಧರಾಗಿರುವರೋ ಅವರಿಗೆ ಮಿಂಚಿನಂತೆ ಇದು ಹೊಳೆಯುವುದು. ಯಾರು ಯೋಗದಿಂದ ಮತ್ತು ಧ್ಯಾನದಿಂದ ಪರಿಶುದ್ಧರಾಗಿ ಅಣಿಯಾಗಿರುವರೋ ಅವರ ಸಾಕ್ಷಾತ್ಕಾರ ಹೆಚ್ಚು ಹೆಚ್ಚು ಸ್ಪಷ್ಟವಾಗುತ್ತಾ ಬರುವುದು. ನಾಲ್ಕು ಸಾವಿರ ವರ್ಷಗಳ ಹಿಂದೆ ಇದನ್ನು ಕಂಡುಹಿಡಿದರು. ಆದರೆ ಇದು ಇನ್ನೂ ಇಡೀ ಜನಾಂಗದ ಸ್ವತ್ತಾಗಿಲ್ಲ; ಎಲ್ಲೋ ಕೆಲವೇ ವ್ಯಕ್ತಿಗಳ ಸ್ವಂತ ಅನುಭವವಾಗಿದೆ.

\begin{center}
೪
\end{center}

ಮಾನವರೆಂದು ಕರೆಸಿಕೊಳ್ಳುವವರೆಲ್ಲಾ ಇನ್ನೂ ನಿಜವಾಗಿ ಮಾನವರಾಗಿಲ್ಲ. ಪ್ರತಿಯೊಬ್ಬರೂ ತಮ್ಮ ಮನಸ್ಸಿನ ಮೂಲಕ ಈ ಪ್ರಪಂಚವನ್ನು ಅಳೆಯಬೇಕಾಗಿದೆ. ಉತ್ತಮ ತಿಳಿವಳಿಕೆ ಪಡೆಯುವುದು ಬಹಳ ಕಷ್ಟ. ಅನೇಕರಿಗೆ ಸೂಕ್ಷ್ಮಕ್ಕಿಂತ ಸ್ಥೂಲವೇ ಹೆಚ್ಚು ಪ್ರಧಾನ. ಇದನ್ನು ಉದಾಹರಿಸುವ ಒಂದು ಕಥೆಯಿದೆ. ಬೊಂಬಾಯಿಯಲ್ಲಿ ಇಬ್ಬರು ಇದ್ದರು. ಅವರಲ್ಲಿ ಒಬ್ಬ ಹಿಂದೂ ಮತ್ತೊಬ್ಬ ಜೈನ. ಅವರು ಒಬ್ಬ ಶ‍್ರೀಮಂತ ವ್ಯಾಪಾರಿಯ ಮನೆಯಲ್ಲಿ ಚದುರಂಗ ಆಡುತ್ತಿದ್ದರು. ಮನೆ ಸಮುದ್ರದ ಪಕ್ಕದಲ್ಲಿ ಇತ್ತು. ಆಟಕ್ಕೆ ಬಹಳ ಹೊತ್ತುಹಿಡಿಯಿತು. ಅವರು ಬಿಸಿಲು ಮಚ್ಚಿನ ಮೇಲೆ ಇದ್ದಾಗ ಸಮುದ್ರದ ಉಬ್ಬರ ಇಳಿತ ಅವರನ್ನು ಆಕರ್ಷಿಸಿತು. ಒಬ್ಬ ಇದನ್ನು ಒಂದು ಕಥೆಯ ಮೂಲಕ ವಿವರಿಸತೊಡಗಿದ. ದೇವತೆಗಳು ಆಟಕ್ಕಾಗಿ ಸಮುದ್ರದ ನೀರನ್ನು ದೊಡ್ಡದೊಂದು ಹಳ್ಳಕ್ಕೆ ಹಾಕಿ ಪುನಃ ಅದನ್ನೇ ತೆಗೆಯುತ್ತಿರುವರು ಎಂದ. ಮತ್ತೊಬ್ಬ “ಇಲ್ಲ, ದೇವತೆಗಳು ನೀರನ್ನು ಒಂದು ಬೆಟ್ಟದ ಮೇಲಕ್ಕೆ ತಮ್ಮ ಉಪಯೋಗಕ್ಕೆ ಒಯ್ಯುವರು; ಕೆಲಸ ಆದಮೇಲೆ ನೀರನ್ನು ಪುನಃ ಸಮುದ್ರಕ್ಕೆ ಹಾಕುವರು” ಎಂದನು. ಅಲ್ಲಿದ್ದ ಒಬ್ಬ ತರುಣ ವಿದ್ಯಾರ್ಥಿ ಇದನ್ನು ನೋಡಿ ನಕ್ಕ. “ಆ ಉಬ್ಬರವಿಳಿತಕ್ಕೆ ಕಾರಣ ಚಂದ್ರನ ಆಕರ್ಷಣೆ ಎಂಬುದು ನಿಮಗೆ ಗೊತ್ತಿಲ್ಲವೇ'' ಎಂದ. ಇದನ್ನು ಕೇಳಿದ ಇಬ್ಬರೂ ಕೋಪದಿಂದ ಆ ಹುಡುಗನ ಕಡೆಗೆ ತಿರುಗಿ, “ಏನು, ನಮ್ಮನ್ನು ಮೂರ್ಖರೆಂದು ಭಾವಿಸುವೆಯಾ? ಚಂದ್ರನ ಬಳಿ ಅಲೆಗಳನ್ನು ಎಳೆಯುವುದಕ್ಕೆ ಬೇಕಾದಷ್ಟು ಉದ್ದದ ಹಗ್ಗಗಳು ಇವೆ ಎಂದು ನಂಬುವಷ್ಟು ನಾವು ಮೂರ್ಖರೊ?” ಎಂದು ಕೇಳಿದರು. ಇಂತಹ ಅವಿವೇಕದ ವಿವರಣೆಯನ್ನು ಒಪ್ಪಿಕೊಳ್ಳುವುದಕ್ಕೆ ಅವರು ಎಳ್ಳಷ್ಟೂ ಇಷ್ಟಪಡಲಿಲ್ಲ. ಅಷ್ಟು ಹೊತ್ತಿಗೆ ಮನೆಯ ಯಜಮಾನ ಬಂದ. ಇಬ್ಬರೂ ಅವನನ್ನು ತೀರ್ಪುಕೊಡು ಎಂದರು. ಅವನು ವಿದ್ಯಾವಂತ, ಅವನಿಗೆ ನಿಜಾಂಶ ಗೊತ್ತಿತ್ತು. ಆದರೆ ಚದುರಂಗ ಆಡುವವರಿಗೆ ಇದು ಅರ್ಥವಾಗುವಂತೆ ಇಲ್ಲ ಎಂಬುದನ್ನು ತಿಳಿದು, ವಿದ್ಯಾರ್ಥಿಗೆ ಸುಮ್ಮನಿರುವಂತೆ ಸನ್ನೆ ಮಾಡಿ, ಇವರಿಗೆ ಉಬ್ಬರವಿಳಿತ ಹೇಗೆ ಆಗುತ್ತದೆ ಎಂಬುದನ್ನು ವಿವರಿಸತೊಡಗಿದನು. ಅದನ್ನು ಕೇಳಿದಾಗ ಚದುರಂಗ ಆಡುತ್ತಿದ್ದವರಿಗೆ ತುಂಬಾ ತೃಪ್ತಿಯಾಯಿತು. ಅವನ ವಿವರಣೆ ಹೀಗಿತ್ತು: “ದೂರದಲ್ಲಿ ಸಮುದ್ರದ ಒಳಗೆ ದೊಡ್ಡದೊಂದು ಸ್ಪಂಜಿನ ಬೆಟ್ಟವಿದೆ. ನೀವಿಬ್ಬರೂ ಸ್ಪಂಜನ್ನು ನೋಡಿರಬಹುದು. ನಾನು ಹೇಳುವುದು ನಿಮಗೆ ಗೊತ್ತಿದೆ ತಾನೆ? ಈ ಸ್ಪಂಜಿನ ಬೆಟ್ಟ ಬೇಕಾದಷ್ಟು ನೀರನ್ನು ಹೀರಿಕೊಂಡಾಗ ಸಮುದ್ರ ಇಳಿಯುವುದು. ಆಮೇಲೆ ದೇವತೆಗಳು ಬಂದು ಆ ಬೆಟ್ಟದ ಮೇಲೆ ಕುಣಿದಾಗ, ಅವರ ಭಾರದಿಂದ ಬೆಟ್ಟ ಕೆಳಗೆ ಕುಗ್ಗಿ ಹೀರಿಕೊಂಡ ನೀರೆಲ್ಲಾ ಹೊರಗೆ ಬಂದಾಗ ಸಮುದ್ರ ಉಬ್ಬುವುದು, ಉಬ್ಬರವಿಳಿತಕ್ಕೆ ಇದೇ ಕಾರಣ, ಗೊತ್ತಾಯಿತೆ? ಇದು ಎಷ್ಟು ವಿಚಾರಯುಕ್ತವಾಗಿದೆ ಮತ್ತು ಸುಲಭವಾಗಿದೆ! ಇದು ನಿಮಗೆ ಚೆನ್ನಾಗಿ ಅರ್ಥವಾಗುವುದು, '' ಚಂದ್ರನ ಆಕರ್ಷಣೆಯಿಂದ ಸಮುದ್ರ ಮೇಲೇಳುವುದೆಂಬುದನ್ನು ಅಲ್ಲಗಳೆದಿದ್ದವರಿಗೆ ಸ್ಪಂಜಿನ ಬೆಟ್ಟ ಮತ್ತು ಅದರ ಮೇಲೆ ಕುಣಿದಾಡುವ ದೇವತೆಗಳು ಇದರಲ್ಲಿ ನಂಬಲು ಅಸಾಧ್ಯವಾಗಿರುವುದು ಏನೂ ಇರಲಿಲ್ಲ. ದೇವತೆಗಳು ಅವರಿಗೆ ನಿಜವಾಗಿದ್ದರು, ಅವರು ನಿಜವಾಗಿ ಸ್ಪಂಜನ್ನು ನೋಡಿದ್ದರು. ಇವೆರಡರಿಂದ ಸಮುದ್ರದಲ್ಲಿ ಆಗುವ ಬದಲಾವಣೆಯನ್ನು ಗ್ರಹಿಸುವುದು ಸುಲಭವಾಗಿತ್ತು.

ಸತ್ಯದ ಪರೀಕ್ಷೆ ಇರುವುದು ನೆಮ್ಮದಿಯಲ್ಲಿ ಅಲ್ಲ, ಅದಕ್ಕೆ ವಿರೋಧವಾಗಿ. ಸತ್ಯ ಅನೇಕ ವೇಳೆ ನೆಮ್ಮದಿಗೆ ಭಂಗ ತರುವುದು. ಒಬ್ಬ ನಿಜವಾಗಿ ಸತ್ಯವನ್ನು ಪಡೆಯಬೇಕೆಂದು ಇದ್ದರೆ ಅವನು ನೆಮ್ಮದಿಯನ್ನು ನೆಚ್ಚಕೂಡದು. ಎಲ್ಲವನ್ನೂ ತ್ಯಜಿಸುವುದು ಬಹಳ ಕಷ್ಟ. ಆದರೆ ಜ್ಞಾನಿ ಇದನ್ನು ಮಾಡಬೇಕು. ಅವನು ಪರಿಶುದ್ದನಾಗಬೇಕು, ಆಸೆಯನ್ನೆಲ್ಲಾ ನಿರ್ಮೂಲಮಾಡಬೇಕು, ದೇಹದ ಮೇಲಿನ ಅಭಿಮಾನವನ್ನೆಲ್ಲಾ ತ್ಯಜಿಸಬೇಕು. ಆಗ ಮಾತ್ರ ಪರಮಸತ್ಯದ ಪರಂಜ್ಯೋತಿ ಅವನಲ್ಲಿ ಬೆಳಗುವುದು. ತ್ಯಾಗ ಆವಶ್ಯಕ. ಅಲ್ಪಾತ್ಮನ ಬಲಿಯೇ ಎಲ್ಲ ಧರ್ಮಗಳ ಮೂಲ ಸತ್ಯ. ಆದಕಾರಣವೇ ತ್ಯಾಗ ಎಲ್ಲಾ ಧರ್ಮಗಳಲ್ಲಿಯೂ ಒಂದು ಅತಿ ಮುಖ್ಯವಾದ ಅಂಶವಾಗಿದೆ. ದೇವತೆಗಳ ಪ್ರೀತ್ಯರ್ಥವಾಗಿ ಮಾಡಿದ ದಾನಗಳ ಹಿಂದೆ ಏಕಮಾತ್ರ ನಿಜವಾದ, ಯಾವುದನ್ನು ನಾವು ತ್ಯಾಗ ಎಂದು ಅಸ್ಪಷ್ಟವಾಗಿ ತಿಳಿದುಕೊಂಡಿದ್ದೇವೆಯೋ ಅದರ ಭಾವನೆ ಇದೆ. ಆ ನಿಜವಾದ ತ್ಯಾಗವೇ ತೋರಿಕೆಯ ವ್ಯಕ್ತಿತ್ವವನ್ನು ವಿಸರ್ಜಿಸುವುದು, ಅದರ ಮೂಲಕ ಮಾತ್ರ ನಾವು ಪರಮಾತ್ಮನನ್ನು ಸಾಕ್ಷಾತ್ಕಾರ ಮಾಡಿಕೊಳ್ಳಬಹುದು. ಜ್ಞಾನಿ ದೇಹವನ್ನು ರಕ್ಷಿಸಲು ಪ್ರಯತ್ನಿಸಕೂಡದು. ಅದನ್ನು ಹಾಗೆ ಮಾಡಲು ಇಚ್ಚಿಸಲೂ ಕೂಡದು. ಅವನು ಧೀರನಾಗಿ ಸತ್ಯವನ್ನು ಅನುಸರಿಸಬೇಕು, ಪ್ರಪಂಚ ನಾಶವಾದರೂ ಚಿಂತೆಯಿಲ್ಲ. ಯಾರು ಕೇವಲ ಮೂಢಾಚಾರಗಳನ್ನು ಅನುಸರಿಸುವರೋ ಅವರು ಎಂದಿಗೂ ಹೀಗೆ ಮಾಡಲಾರರು. ಇದು ನಾವು ಒಂದು ಜೀವನದಲ್ಲಿ ಮಾಡಬೇಕಾದ ಕೆಲಸ ಅಲ್ಲ, ನೂರಾರು ಜನ್ಮಗಳ ಕೆಲಸ. ಎಲ್ಲೋ ಕೆಲವು ಧೀರಮತಿಗಳಿಗೆ ಮಾತ್ರ ಅಂತರಾತ್ಮನನ್ನು ಸಂದರ್ಶಿಸುವ, ಸ್ವರ್ಗ, ಈಶ್ವರ ಇವುಗಳನ್ನೂ ಮತ್ತು ಎಲ್ಲಾ ಫಲಾಪೇಕ್ಷೆಯನ್ನೂ ತುಚ್ಛವಾಗಿ ನೋಡುವ ಧೈರ್ಯವಿದೆ. ಇದನ್ನು ಮಾಡುವುದಕ್ಕೆ ವಜ್ರಸಂಕಲ್ಪ ಇರಬೇಕು. ಅನುಮಾನಿಸುವುದೂ ಕೂಡ ಅತಿ ಮನೋದೌರ್ಬಲ್ಯದ ಕುರುಹು. ಮಾನವನು ಯಾವಾಗಲೂ ಪೂರ್ಣನು, ಇಲ್ಲದೆ ಇದ್ದರೆ ಅವನೆಂದಿಗೂ ಪೂರ್ಣನಾಗುತ್ತಿರಲಿಲ್ಲ. ಆದರೆ ಅವನು ಅದನ್ನು ತಿಳಿಯಬೇಕಾಗಿದೆ. ಮಾನವನು ಬಾಹ್ಯ ಕಾರಣಗಳಿಂದ ಬಂಧಿತನಾಗಿದ್ದರೆ ಅವನು ಕೇವಲ ಮರ್ತ್ಯನಾಗುತ್ತಿದ್ದನು. ನಿರುಪಾಧಿಕವಾದುದು ಮಾತ್ರ ಅಮೃತವಾಗಬಲ್ಲದು. ಯಾವುದೂ ಆತ್ಮನ ಮೇಲೆ ಪ್ರಭಾವವನ್ನು ಬೀರಲಾರದು, ಇದು ಕೇವಲ ಭ್ರಾಂತಿ. ಆದರೆ ಮನುಷ್ಯ ತಾನು ಆತ್ಮನೆಂದು ಅರಿಯಬೇಕು. ತಾನು ದೇಹ ಅಥವಾ ಮನಸ್ಸು ಎಂದು ಭಾವಿಸಬಾರದು. ಅವನು ತಾನು ಕೇವಲ ಸಾಕ್ಷಿ ಎಂದು ತಿಳಿಯಲಿ. ಆಗ ತನ್ನೆದುರಿಗಿರುವ ಅದ್ಭುತ ದೃಶ್ಯದ ಸೊಬಗನ್ನು ಆನಂದಿಸಬಲ್ಲನು. ಅವನು “ನಾನೇ ವಿಶ್ವ, ನಾನೇ ಬ್ರಹ್ಮ'' ಎಂದು ಹೇಳಿಕೊಳ್ಳಲಿ. ಮನುಷ್ಯ ಎಂದು ತಾನೇ ಆ ಏಕವೆಂದು ಭಾವಿಸುವನೋ, ಆಗ ಅವನಿಗೆ ಎಲ್ಲವೂ ಸಾಧ್ಯವಾಗುವುದು. ಆಗ ಪಂಚಭೂತಗಳೆಲ್ಲಾ ಅವನ ಆಜ್ಞಾಧಾರಕವಾಗುವುವು. ಶ‍್ರೀರಾಮಕೃಷ್ಣರು ಹೇಳುತ್ತಿದ್ದಂತೆ, ಬೆಣ್ಣೆಯನ್ನು ತೆಗೆದಾದ ಮೇಲೆ ಅದನ್ನು ನೀರಿನಲ್ಲಿ ಬೇಕಾದರೂ ಇಡಬಹುದು, ಮಜ್ಜಿಗೆಯಲ್ಲಿ ಬೇಕಾದರೂ ಇಡಬಹುದು, ಅದು ಕರಗಿಹೋಗುವುದಿಲ್ಲ. ಅದರಂತೆಯೆ ಮಾನವ ಒಮ್ಮೆ ಆತ್ಮಸಾಕ್ಷಾತ್ಕಾರವನ್ನು ಪಡೆದನೆಂದರೆ ಅವನು ಮತ್ತೊಮ್ಮೆ ಪ್ರಪಂಚದ ಭ್ರಾಂತಿಗೆ ಬೀಳುವುದಿಲ್ಲ.

ವಿಮಾನದ ಮೇಲಿನಿಂದ ನೋಡಿದರೆ ಕೆಳಗಿರುವ ಹಳ್ಳದಿಣ್ಣೆಗಳು ಕಾಣುವುದಿಲ್ಲ. ಅದರಂತೆಯೇ ಮನುಷ್ಯ ಉತ್ತಮನಾದರೆ ಅವನಿಗೆ ಒಳ್ಳೆಯ ಜನರು ಮತ್ತು ಕೆಟ್ಟ ಜನರು ಎಂದು ಭೇದ ಕಾಣುವುದಿಲ್ಲ. ಮಡಕೆಯನ್ನು ಒಂದು ಸಲ ಸುಟ್ಟಮೇಲೆ ಪುನಃ ಅದರ ಆಕಾರವನ್ನು ಬದಲಾಯಿಸಲಾಗುವುದಿಲ್ಲ. ಅದರಂತೆಯೇ ಒಮ್ಮೆ ದೇವರ ಸ್ಪರ್ಶವನ್ನು ಪಡೆದು ಜ್ಞಾನವೆಂಬ ಬೆಂಕಿಯಲ್ಲಿ ಬೆಂದ ಮೇಲೆ ಜೀವಿಯು ಪುನಃ ಬದಲಾಗುವುದಿಲ್ಲ. ತತ್ವ ಎಂದರೆ ಸಂಸ್ಕೃತದಲ್ಲಿ ಸ್ಪಷ್ಟ ದೃಷ್ಟಿ ಎಂದು ಮತ್ತು ಆನುಷ್ಠಾನಿಕ ತತ್ವವೇ ಧರ್ಮ. ಕೇವಲ ಊಹೆಯ ಸಿದ್ದಾಂತಗಳನ್ನು ಭರತಖಂಡದಲ್ಲಿ ಅಷ್ಟು ಗೌರವಿಸುವುದಿಲ್ಲ. ಅಲ್ಲಿ ಯಾವ ಚರ್ಚೂ ಇಲ್ಲ, ಮತವಿಲ್ಲ, ಮೂಢನಂಬಿಕೆಗಳಿಲ್ಲ. ಇರುವ ಎರಡು ದೊಡ್ಡ ಪಂಗಡಗಳೇ ದ್ವೈತ ಮತ್ತು ಅದ್ವೈತ. ದ್ವೈತಿಗಳು “ಭಗವಂತನ ಕೃಪೆಯೇ ಮುಕ್ತಿಗೆ ಮಾರ್ಗ. ಒಂದು ಸಲ ಜಾರಿಗೆ ಬಂದ ಕರ್ಮನಿಯಮವನ್ನು ಯಾರೂ ನಿಲ್ಲಿಸಲಾರರು. ಕರ್ಮಕ್ಕೆ ಬದ್ದನಾಗದ ದೇವರು ಮಾತ್ರ ತನ್ನ ಕೃಪೆಯಿಂದ ಕರ್ಮಬಂಧನವನ್ನು ಮುರಿಯಲು ನಮಗೆ ಸಹಾಯಮಾಡುವನು” ಎನ್ನುವರು. ಅದ್ವೈತಿಗಳು ಹೀಗೆ ಹೇಳುವರು: “ಈ ಪ್ರಕೃತಿಯ ಹಿಂದೆ ಸ್ವತಂತ್ರವಾಗಿರುವುದು ಯಾವುದೋ ಒಂದು ಇದೆ. ಈ ನಿಯಮಗಳಿಗೆ ಒಳಗಾಗದ ಅದನ್ನು ಕಂಡ ಮೇಲೆಯೇ ನಮಗೆ ಸ್ವಾತಂತ್ರ್ಯ, ಸ್ವಾತಂತ್ರ್ಯವೇ ಮುಕ್ತಿ.” ದ್ವೈತವು ದಾರಿಯಲ್ಲಿರುವ ಕೇವಲ ಒಂದು ಘಟ್ಟ. ಅದ್ವೈತ ದಾರಿಯ ತುದಿ, ಪರಿಶುದ್ದನಾಗುವುದೇ ಮುಕ್ತನಾಗುವುದಕ್ಕೆ ನೇರ ಹಾದಿ. ನಾವು ಯಾವುದನ್ನು ಸಂಪಾದಿಸುತ್ತೇವೆಯೋ ಅದು ಮಾತ್ರ ನಮ್ಮದು. ಯಾವ ಅಧಿಕಾರವೂ, ನಂಬಿಕೆಯೂ ನಮ್ಮನ್ನು ಪಾರುಮಾಡಲಾರದು. ದೇವರೊಬ್ಬನಿದ್ದರೆ ಎಲ್ಲರೂ ಅವನನ್ನು ನೋಡಬೇಕು. ಈಗ ತುಂಬಾ ಸೆಕೆ ಇದೆ ಎಂದು ಯಾರೂ ಹೇಳಬೇಕಾಗಿಲ್ಲ. ಎಲ್ಲರೂ ತಮಗೆ ತಾವೇ ಅದನ್ನು ತಿಳಿಯಬಹುದು. ಇದರಂತೆಯೇ ದೇವರೂ ಕೂಡ ಇರಬೇಕು. ಎಲ್ಲರ ಪ್ರಜ್ಞೆಯಲ್ಲಿಯೂ ಅವನೊಂದು ಅನುಭವವಾಗಿರಬೇಕು. ಪಾಶ್ಚಾತ್ಯರು ಯಾವುದನ್ನು ಪಾಪ ಎನ್ನುತ್ತಾರೆಯೋ ಅದನ್ನು ಹಿಂದೂಗಳು ಒಪ್ಪಿಕೊಳ್ಳುವುದಿಲ್ಲ.\break ದುಷ್ಕೃತ್ಯಗಳು ಪಾಪವಲ್ಲ. ಇವನ್ನು ಮಾಡುವುದರಿಂದ ನಾವು ದೇವರಿಗೆ ಕೋಪವನ್ನುಂಟುಮಾಡುವುದಿಲ್ಲ. ಇದರಿಂದ ನಮಗೆ ಮಾತ್ರ ಹಾನಿ, ನಾವು ಅದಕ್ಕೆ ತಕ್ಕ ಶಿಕ್ಷೆಯನ್ನು ಅನುಭವಿಸಬೇಕು. ಬೆಂಕಿಯೊಳಗೆ ಬೆರಳಿಡುವುದೊಂದು ಪಾಪವಲ್ಲ. ಯಾರು ಅದರಲ್ಲಿ ಬೆರಳಿಡುವರೋ ಅವರ ಕೈಗಳು ಸುಡುವುದರಲ್ಲಿ ಸಂದೇಹವಿಲ್ಲ. ಎಲ್ಲ ಕರ್ಮಗಳಿಂದ ಕೆಲವು ಪರಿಣಾಮಗಳು ಉಂಟಾಗುವುವು. ಪ್ರತಿಯೊಂದು ಕರ್ಮವೂ ಮಾಡಿದವನಿಗೆ ಹಿಂತಿರುಗುವುದು. ಟ್ರೆನಿಟೇರಿಯನಿಸಮ್ (ದೇವರು, ಮಗ ಮತ್ತು ಹೋಲಿಗೋಸ್ಟ್ – ಇವು ಒಟ್ಟಿಗೆ ಇರುವುದು) ಎಂಬುದು ಯೂನಿಟೇರಿಯಾನಿಸಮ್ (ದೇವರು ಮತ್ತು ಮನುಷ್ಯ ಎಂದೆಂದಿಗೂ ಬೇರೆ ಎಂದು ಹೇಳುವುದು) ಎಂಬುದಕ್ಕಿಂತ ಉತ್ತಮ ಸಿದ್ಧಾಂತ. ನಾವು ಭಗವಂತನ ಮಕ್ಕಳೆಂದು ಭಾವಿಸುವುದು, ಮೇಲೆ ಹೋಗಲು ಮೊದಲ ಮೆಟ್ಟಲು. ನಾವೆಲ್ಲ ಒಂದು, ನಾವೆಲ್ಲ ಆತ್ಮ ಎಂದು ಭಾವಿಸುವುದೇ ಕೊನೆಯ ಮೆಟ್ಟಲು.

\begin{center}
೫
\end{center}

ಶಾಶ್ವತವಾದ ದೇಹ ಏತಕ್ಕೆ ಇರಬಾರದು ಎಂದು ಕೇಳುವುದೇ ಕುತರ್ಕ. ಕೆಲವು ಬದಲಾವಣೆಯಾಗುವ ಧಾತುಗಳ ಸಂಯೋಗಕ್ಕೆ ದೇಹ ಎಂದು ಹೆಸರು. ದೇಹ ಬದಲಾಗುವುದು; ಸ್ವಭಾವತಃ ಅದು ಅಶಾಶ್ವತ. ನಾವು ಬದಲಾಗದೆ ಇದ್ದರೆ ನಮಗೆ ದೇಹವೇ ಇರುತ್ತಿರಲಿಲ್ಲ. ಕಾಲ–ದೇಶ–ನಿಮಿತ್ತಗಳ ಆಚೆ ಇರುವ ವಸ್ತು ವಸ್ತುವೇ ಅಲ್ಲ, ಕಾಲ ದೇಶಗಳು ನಮ್ಮಲ್ಲಿವೆ. ನಾವು ಮಾತ್ರ ಶಾಶ್ವತ. ರೂಪಗಳೆಲ್ಲ ಕ್ಷಣಿಕ. ಆದಕಾರಣವೇ ಧರ್ಮಗಳೆಲ್ಲ ದೇವರಿಗೆ ಆಕಾರವಿಲ್ಲವೆನ್ನುವುವು. ಮಿನಾಂಡರ್ ಎಂಬುವನು ಗ್ರೀಕೊಬ್ಯಾಕ್ಟ್ರಿಯದ ರಾಜ. ಕ್ರಿ.ಪೂ. ೧೫೦ರಲ್ಲಿ ಅವನನ್ನು ಒಬ್ಬ ಬೌದ್ಧ ಭಿಕ್ಷು ಬೌದ್ಧ ಧರ್ಮಕ್ಕೆ ಸೇರಿಸಿ ಅವನನ್ನು ಮಿಲಿಂದ ಎಂದು ಕರೆಯತೊಡಗಿದನು. ಅವನು ತನ್ನ ಗುರುವಾದ ತರುಣ ಭಿಕ್ಷುವನ್ನು, ಬುದ್ದನಂತಹ ಧರ್ಮಾತ್ಮರು ತಪ್ಪುಮಾಡಬಲ್ಲರೆ, ಅಥವಾ ತಪ್ಪು ತಿಳಿದುಕೊಳ್ಳಬಲ್ಲರೆ ಎಂದು ಕೇಳಿದ. ಅದಕ್ಕೆ ಅವನು ಕೊಟ್ಟ ಉತ್ತರ: ಸಿದ್ದನಿಗೆ ಗೌಣ ವಿಷಯಗಳು ತಿಳಿಯದೆ ಇರಬಹುದು. ಆದರೆ ಅವನ ಅನುಭವ ಎಂದಿಗೂ ತಪ್ಪಾಗಿರುವುದಿಲ್ಲ, ಅವನ ಅಂತರ್ಜ್ಯೋತಿ ಎಂದಿಗೂ ತಪ್ಪು ತಿಳಿದುಕೊಳ್ಳಲಾರದು. ಅವನು ಇಲ್ಲಿ ಈಗಲೇ ಸಿದ್ಧನಾಗಿರುವನು. ವಿಶ್ವದ ಸಾರ, ವಿಶ್ವದ ರಹಸ್ಯವೆಲ್ಲಾ ಅವನಿಗೆ ಗೊತ್ತಿದೆ. ಆ ಸಾರವಸ್ತು ಕಾಲದೇಶಗಳ ಪ್ರಪಂಚದಲ್ಲಿ ಹೇಗೆ ವ್ಯಕ್ತವಾಗಿದೆ ಎಂದು ಬಿಡಿಬಿಡಿಯಾಗಿ ಅವನಿಗೆ ಗೊತ್ತಿಲ್ಲದೆ ಇರಬಹುದು. ಅವನಿಗೆ ಜೇಡಿಮಣ್ಣು ಚೆನ್ನಾಗಿ ಗೊತ್ತಿದೆ. ಆದರೆ ಅದರಿಂದ ಯಾವ ಯಾವ ಪದಾರ್ಥಗಳನ್ನು ಮಾಡಬಹುದೆಂಬುದು ಅವನಿಗೆ ಗೊತ್ತಿಲ್ಲದೆ ಇರಬಹುದು. ಸಿದ್ದನಿಗೆ ಆತ್ಮದ ಪರಿಚಯವಾಗಿದೆ. ಆದರೆ ಅದರ ಆವಿರ್ಭಾವದ ಪ್ರತಿಯೊಂದು ರೂಪರೇಖೆಗಳೂ ಅವನಿಗೆ ಗೊತ್ತಿಲ್ಲದೆ ಇರಬಹುದು. ನಮ್ಮಂತೆಯೇ ಅವನೂ ಕೂಡ ಸಾಪೇಕ್ಷಜ್ಞಾನವನ್ನು ಪಡೆಯಬೇಕಾಗಿದೆ. ಆದರೆ ಅವನು ತನ್ನಲ್ಲಿರುವ ಅದ್ಭುತಶಕ್ತಿಯಿಂದ ಬಹಳ ಬೇಗ ಕಲಿತುಕೊಳ್ಳಬಹುದು.

ಸಂಪೂರ್ಣ ನಿಗ್ರಹಕ್ಕೆ ಬಂದ ಮನಸ್ಸಿನ ಅದ್ಭುತಶಕ್ತಿಯನ್ನು ಯಾವ ವಸ್ತುವಿನ ಮೇಲೆ ಬೀರಿದರೂ ಅದರ ರಹಸ್ಯ ನಮ್ಮ ವಶವಾಗುವುದು. ಇದನ್ನು ತಿಳಿದಿರುವುದು ಬಹು ಮುಖ್ಯ. ಏಕೆಂದರೆ ಇದರಿಂದ ಸಾಪೇಕ್ಷ ಜ್ಞಾನದ ಸಂಬಂಧದಲ್ಲಿ, (ನಮಗೆ ತಿಳಿದಿರುವಂತೆ,) ಬುದ್ಧ ಅಥವಾ ಏಸುವಿನ ತಪ್ಪು ತಿಳಿವಳಿಕೆಯನ್ನು ವಿವರಿಸಲು ಸುಲಭವಾಗುವುದು. ಶಿಷ್ಯರು ಗುರುಗಳ ಸಂದೇಶವನ್ನು ತಪ್ಪು ತಿಳಿದುಕೊಂಡು ಬರೆದಿಟ್ಟರು ಎಂದು ನಾವು ಅವರನ್ನು ಹಳಿಯಬೇಕಾಗಿಲ್ಲ. ಅವರು ಹೇಳಿದ್ದರಲ್ಲಿ ಒಂದು ಸತ್ಯ, ಮತ್ತೊಂದು ಸತ್ಯವಲ್ಲ ಎನ್ನುವುದು ಆತ್ಮ ವಂಚನೆ, ಅವನ್ನೆಲ್ಲ ಒಟ್ಟಿಗೆ ಒಪ್ಪಿಕೊಳ್ಳಿ, ಇಲ್ಲವೆ ತಿರಸ್ಕರಿಸಿ. ಅವರು ಹೇಳಿದುದರಲ್ಲಿ ಸರಿ ಯಾವುದು, ತಪ್ಪು ಯಾವುದು ಎಂದು ಹೇಗೆ ಕಂಡುಹಿಡಿಯುವುದು?

ಯಾವುದಾದರೂ ಒಂದು ಒಮ್ಮೆ ಸಂಭವಿಸಿದರೆ ಅದು ಮತ್ತೊಮ್ಮೆಯೂ ಸಂಭವಿಸಬಹುದು. ಯಾರಾದರೂ ಒಬ್ಬರು ಪೂರ್ಣತೆಯನ್ನು ಪಡೆದರೆ ನಾವೂ ಅದನ್ನು ಪಡೆಯಬಹುದು. ಈಗ, ಮತ್ತು ಇಲ್ಲಿ ನಾವು ಪೂರ್ಣರಾಗದೆ ಇದ್ದರೆ ನಾವು ಊಹಿಸುವ ಯಾವ ಸ್ಥಿತಿಯಲ್ಲಿಯೂ, ಸ್ವರ್ಗ ಅಥವಾ ನಾವು ಊಹಿಸಿಕೊಳ್ಳ ಬಹುದಾದ ಬೇರಾವ ಸ್ಥಿತಿಯಲ್ಲಿಯೂ ಪೂರ್ಣರಾಗಲಾರೆವು. ಏಸುಕ್ರಿಸ್ತ ಪೂರ್ಣಾತ್ಮನಲ್ಲದೆ ಇದ್ದರೆ ಅವನ ಹೆಸರನ್ನು ಹೊತ್ತ ಧರ್ಮ ಕುಸಿದುಬೀಳುವುದು. ಅವನು ಪೂರ್ಣಾತ್ಮನಾಗಿದ್ದರೆ ನಾವೂ ಆಗಬಹುದು. ಪೂರ್ಣಾತ್ಮನು ಏನನ್ನಾದರೂ ನಾವು ಹೇಗೆ ತಿಳಿಯುತ್ತೇವೆಯೋ ಹಾಗೆ ತಿಳಿಯುವುದಿಲ್ಲ. ಏಕೆಂದರೆ ನಮ್ಮ ಜ್ಞಾನವೆಲ್ಲ ಪರಸ್ಪರ ಹೋಲಿಕೆಯಲ್ಲಿದೆ. ನಿರಪೇಕ್ಷದಲ್ಲಿ ಯಾವ ಹೋಲಿಕೆಯೂ ಇಲ್ಲ, ಅದನ್ನು ವರ್ಗಿಕರಿಸುವುದಕ್ಕೆ ಆಗುವುದಿಲ್ಲ. ಹುಟ್ಟುಗುಣವು ಯುಕ್ತಿಯಷ್ಟು ತಪ್ಪು ಮಾಡುವುದಿಲ್ಲ. ಆದರೆ ಯುಕ್ತಿ ಹುಟ್ಟುಗುಣಕ್ಕಿಂತ ಮೇಲು. ಅದು ನಮ್ಮನ್ನು ಒಳ ಅರಿವಿಗೆ ಒಯ್ಯುವುದು. ಆ ಒಳ ಅರಿವು ಯುಕ್ತಿಗಿಂತಲೂ ಮೇಲಿರುವುದು. ಜ್ಞಾನವೇ ಒಳ ಅರಿವಿನ ಮೂಲ. ಅದು ಹುಟ್ಟುಗುಣದಂತೆ ಎಂದಿಗೂ ತಪ್ಪುವುದಿಲ್ಲ. ಆದರೆ ಅದಕ್ಕಿಂತ ಮೇಲಿನ ಕ್ಷೇತ್ರದಲ್ಲಿ ಕೆಲಸಮಾಡುವುದು. ಜೀವಿಗಳ ವಿಕಾಸದಲ್ಲಿ ಮೂರು ಹಂತಗಳಿವೆ: (\enginline{Subconscious}) ಅಪ್ರಜ್ಞೆ: ಇದು ಯಾಂತ್ರಿಕವಾಗಿರುವುದು, ಇದು ತಪ್ಪು ಮಾಡುವುದಿಲ್ಲ: (\enginline{Conscious}) ಪ್ರಜ್ಞೆ, ಇದು ತಿಳಿದುಕೊಳ್ಳುತ್ತದೆ, ತಪ್ಪು ಮಾಡುತ್ತದೆ. (\enginline{Superconscious}) ಅತಿಪ್ರಜ್ಞೆ, ಸ್ಫೂರ್ತಿ, ಇದು ಎಂದಿಗೂ ತಪ್ಪು ಮಾಡುವುದಿಲ್ಲ. ಇವಕ್ಕೆ ಕ್ರಮವಾಗಿ ಪ್ರಾಣಿ, ಮಾನವ ಮತ್ತು ದೇವಮಾನವ – ಇವು ಉದಾಹರಣೆಗಳು. ಪೂರ್ಣಾತ್ಮನಿಗೆ ಉಳಿದಿರುವುದು ಯಾವುದೂ ಇಲ್ಲ, ಅವನು ತನ್ನ ತಿಳಿವಳಿಕೆಯನ್ನು ಪ್ರಯೋಗಿಸಬೇಕು ಅಷ್ಟೆ. ಅವನು ಯಾವ ಫಲಾಪೇಕ್ಷೆಯೂ ಇಲ್ಲದೆ ಜಗತ್ತಿಗೆ ಸಹಾಯ ಮಾಡಲು ಮಾತ್ರ ಬದುಕುತ್ತಾನೆ. ನಮಗೂ ಅವನಿಗೂ ಇರುವ ವ್ಯತ್ಯಾಸ ನಕಾರಾತ್ಮಕವಾದದ್ದು. ಅಸ್ತ್ಯಾತ್ಮಕವಾದ ಭಾವನೆ ವಿಶಾಲವಾಗುತ್ತ ಬರುವುದು. ಸರ್ವಸಾಮಾನ್ಯಭಾವನೆಯು ಅತ್ಯಂತ ವಿಶಾಲವಾಗಿರುವುದು. ಅದೇ ನಿಜವಾದ ಅಸ್ತಿತ್ವ.

\newpage

“ನಿಯಮವೆನ್ನುವುದು ಬಾಹ್ಯ ಘಟನೆಗಳನ್ನು ವಿವರಿಸುವ ಒಂದು ಮಾನಸಿಕ ಶೀಘ್ರಲಿಪಿ.” ಆದರೆ ನಿಯಮವೆನ್ನುವುದೇ ಬೇರೆ ಇಲ್ಲ. ಒಂದು ವಿಧವಾದ ಘಟನೆಗಳು ಅನುಕ್ರಮವಾಗಿ ನಡೆಯುವುದನ್ನು ನಿರೂಪಿಸುವುದನ್ನೇ ನಾವು ನಿಯಮ ಎಂದು ಕರೆಯುತ್ತೇವೆ. ನಿಯಮವನ್ನು ನಾವು ಅನಿವಾರ್ಯವಾಗಿ ಶರಣಾಗತರಾಗಲೇಬೇಕಾದ ಒಂದು ಮೂಢನಂಬಿಕೆಯನ್ನಾಗಿ ಮಾಡಬಾರದು. ಯುಕ್ತಿಯಲ್ಲಿ ದೋಷ ಇರಲೇಬೇಕಾಗಿದೆ. ಆದರೆ ಆ ತಪ್ಪನ್ನು ಜಯಿಸುವ ಹೋರಾಟವೇ ನಮ್ಮನ್ನು ದೇವತೆಗಳನ್ನಾಗಿ ಮಾಡುವುದು. ರೋಗವೆಂದರೆ ನಮ್ಮ ದೇಹದಲ್ಲಿರುವ ಯಾವುದೋ ಒಂದು ದೋಷವನ್ನು ಹೊರದೂಡುವುದಕ್ಕೆ ಪ್ರಕೃತಿಯು ಮಾಡುವ ಹೋರಾಟ. ಹಾಗೆಯೇ ಪಾಪವೆಂದರೆ ನಮ್ಮಲ್ಲಿರುವ ದೈವತ್ವವು ಮೃಗೀಯತೆಯನ್ನು ಹೊರದೂಡಲು ಮಾಡುವ ಪ್ರಯತ್ನ. ನಾವು ದೈವತ್ವಕ್ಕೆ ಏರಬೇಕಾದರೆ ತಪ್ಪನ್ನು ಮಾಡಲೇಬೇಕಾಗಿದೆ.

ಯಾರಿಗಾಗಿಯೂ ಮರುಗಬೇಡಿ. ಎಲ್ಲರೂ ನಿಮಗೆ ಸರಿಸಮಾನರೆಂದು ಭಾವಿಸಿ. ಮೇಲುಕೀಳೆಂಬ ಮೂಲಪಾಪದಿಂದ ಪಾರಾಗಲು ಯತ್ನಿಸಿ. ನಾವೆಲ್ಲ ಸರಿಸಮಾನರು. ನಾನು ಮೇಲು, ನೀನು ಕೀಳು, ನಾನು ನಿನ್ನನ್ನು ಉದ್ಧಾರಮಾಡಲು ಯತ್ನಿಸುತ್ತೇನೆ ಎಂದು ಭಾವಿಸಬಾರದು. ಸಮಾನತೆಯೇ ಮುಕ್ತನ ಚಿಹ್ನೆ. ಏಸುವು ಜನಸಾಮಾನ್ಯರಲ್ಲಿ ಅವತಾರವೆತ್ತಿ ಅವರೊಡನೆ ಬಾಳಿದ. ಅವನೆಂದಿಗೂ ಅವರಿಗಿಂತ ತಾನು ಮೇಲೆಂದು ಭಾವಿಸಲಿಲ್ಲ. ಪಾಪಿಗಳು ಮಾತ್ರ ಪಾಪವನ್ನು ನೋಡಬಲ್ಲರು. ಮನುಷ್ಯನನ್ನು ನೋಡಬೇಡಿ, ದೇವರನ್ನು ಮಾತ್ರ ನೋಡಿ. ನಾವೇ ನಮ್ಮ ಸ್ವರ್ಗವನ್ನು ಸೃಷ್ಟಿಸುವವರು. ನಾವು ನರಕದಲ್ಲಿ ಕೂಡ ಸ್ವರ್ಗವನ್ನು ನಿರ್ಮಿಸಬಲ್ಲೆವು. ಪಾಪಿಗಳು ನರಕದಲ್ಲಿ ಮಾತ್ರ ಇರುವುದು. ಸುತ್ತಲೂ ನಾವು ಪಾಪಿಗಳನ್ನು ನೋಡಿದರೆ ನಾವು ಕೂಡ ನರಕದಲ್ಲೇ ಇರುವೆವು. ಆತ್ಮವು ಕಾಲದಲ್ಲಿಯೂ ಇಲ್ಲ, ದೇಶದಲ್ಲಿಯೂ ಇಲ್ಲ. ನಾನೇ ಸಚ್ಚಿದಾನಂದಬ್ರಹ್ಮಸ್ವರೂಪ, 'ಸೋಽಹಂ' ಎಂಬುದನ್ನು ಸಾಕ್ಷಾತ್ಕಾರ ಮಾಡಿಕೊಳ್ಳಿ. ಹುಟ್ಟುವಾಗ ಸಂತೋಷಪಡಿ. ಸಾಯುವಾಗಲೂ ಸಂತೋಷಪಡಿ. ಯಾವಾಗಲೂ ಭಗವತ್ಪ್ರೇಮದಲ್ಲಿ ಆನಂದಪಡಿ, ದೇಹಬಂಧನದಿಂದ ಪಾರಾಗಿ, ನಾವು ಅದಕ್ಕೆ ಗುಲಾಮರಾಗಿ, ನಮ್ಮನ್ನು ಬಿಗಿದ ಸರಪಳಿಯನ್ನೇ ಅಪ್ಪಿ, ನಮ್ಮ ಗುಲಾಮಗಿರಿಯನ್ನೇ ಪ್ರೀತಿಸುತ್ತಿರುವೆವು. ಮುಂದೆಯೂ ಹಾಗೆಯೇ ಇರಬೇಕೆಂದು ಎಂದೆಂದಿಗೂ `ದೇಹ' `ದೇಹ' ಎಂದು ಆಶಿಸುತ್ತಿರುವೆವು. ಈ ದೇಹವೆಂಬ ಭಾವನೆಯಲ್ಲಿ ಆಸಕ್ತರಾಗಬೇಡಿ. ಈಗಿನಂತೆಯೇ ಮುಂದೆಯೂ ಇರಬೇಕೆಂದು ಆಶಿಸಬೇಡಿ. ನಮಗೆ ಪ್ರಿಯರಾಗಿರುವವರ ದೇಹವನ್ನಾದರೂ ಸರಿಯೆ ಪ್ರೀತಿಸಲೂ ಬೇಡಿ, ಬಯಸಲೂ ಬೇಡಿ. ಈ ಜೀವನವೇ ನಮ್ಮ ಗುರು. ಸಾವೆಂದರೆ ಪುನಃ ಜನಿಸುವುದಕ್ಕೆ ಮತ್ತೊಂದು ಅವಕಾಶ ದೊರಕಿದಂತೆ. ದೇಹವೇ ನಮ್ಮ ಉಪಾಧ್ಯಾಯ. ಆದರೆ ಆತ್ಮಹತ್ಯೆ ತಪ್ಪು. ಅದು ಉಪಾಧ್ಯಾಯನನ್ನು ಕೊಂದಂತೆ. ಹಾಗೆ ಮಾಡಿದರೆ ಮತ್ತೊಬ್ಬ ಉಪಾಧ್ಯಾಯ ಹಾಜರಾಗುತ್ತಾನೆ. ಆದ್ದರಿಂದ ದೇಹವನ್ನು ಮೀರಿ ಹೋಗುವುದನ್ನು ಕಲಿಯುವವರೆಗೆ ನಮಗೆ ದೇಹದ ಅಗತ್ಯವಿದೆ. ಒಂದು ಹೋದರೆ ಮತ್ತೊಂದು ಬರುವುದು. ಆದರೂ ನಾವೇ ದೇಹವೆಂದು ಭ್ರಮಿಸಕೂಡದು. ದೇಹ ಪೂರ್ಣತೆಯನ್ನು ಪಡೆಯುವುದಕ್ಕೆ ಇರುವ ಒಂದು ಕರಣ ಎಂದು ನೋಡಬೇಕು. ರಾಮಚಂದ್ರನ ಭಕ್ತ ಹನುಮಂತ ಈ ವಾಕ್ಯಗಳ ತತ್ತ್ವವನ್ನು ವ್ಯಕ್ತಪಡಿಸುವನು: “ಹೇ ರಾಮ, ನಾನು ದೇಹವೆಂದು ಭಾವಿಸಿದಾಗ ನಿನ್ನಿಂದ ಎಂದೆಂದಿಗೂ ಬೇರೆ ಇರುವ ಸೇವಕ. ನಾನು ಜೀವನೆಂದು ಭಾವಿಸಿದಾಗ ನಿನ್ನ ಪರಂಜ್ಯೋತಿಯ ಅಂಶ. ನಾನು ಆತ್ಮವೆಂದು ಭಾವಿಸಿದಾಗ ನಾನು ನೀನೂ ಒಂದೇ.'' ಆದಕಾರಣವೇ ಜ್ಞಾನಿಯು ಆತ್ಮನನ್ನು ಪಡೆಯಲು ಇಚ್ಛಿಸುವನು, ಮತ್ತೇನನ್ನೂ ಅಲ್ಲ.

\begin{center}
೬
\end{center}

ಆಲೋಚನೆ ಬಹಳ ಮುಖ್ಯ. ಏಕೆಂದರೆ ನಾವು ಆಲೋಚಿಸಿದಂತೆ ಆಗುತ್ತೇವೆ. ಹಿಂದೆ ಒಬ್ಬ ಸಂನ್ಯಾಸಿ ಇದ್ದ. ಸಾಧುಮನುಷ್ಯ ಆತ. ಒಂದು ಮರದ ಕೆಳಗೆ ಕುಳಿತುಕೊಂಡು ಇತರರಿಗೆ ಬೋಧಿಸುತ್ತಿದ್ದ. ಅವನು ಹಾಲು–ಹಣ್ಣುಗಳಿಂದಲೇ ಹೊಟ್ಟೆ ಹೊರೆಯುತ್ತ ಬೇಕಾದಷ್ಟು ಪ್ರಾಣಾಯಾಮ ಮಾಡುತ್ತ ತಾನು ತುಂಬಾ ಪವಿತ್ರಾತ್ಮ ಎಂದು ಭಾವಿಸಿದ್ದ. ಅದೇ ಹಳ್ಳಿಯಲ್ಲಿ ಕೆಟ್ಟ ಹೆಂಗಸೊಬ್ಬಳು ಇದ್ದಳು. ಪ್ರತಿದಿನ ಸಂನ್ಯಾಸಿ ಅವಳಿಗೆ ಆಕೆಯ ದುಷ್ಟತನದಿಂದ ನರಕ ಸಂಭವಿಸುವುದೆಂದು ಎಚ್ಚರಿಸುತ್ತಿದ್ದನು. ಆದರೆ ತನ್ನ ಜೀವನೋಪಾಯಕ್ಕೆ ಏಕಮಾತ್ರ ಮಾರ್ಗವಾದ ಕಸುಬನ್ನು ತ್ಯಜಿಸಲು ಆಕೆಗೆ ಆಗಲಿಲ್ಲ. ಆದರೂ ಆ ಸಂನ್ಯಾಸಿ ಅವಳ ಭವಿಷ್ಯವನ್ನು ಕುರಿತು ಹೇಳುವುದನ್ನು ಕೇಳಿ ಭೀತಳಾದಳು. ಅಳುತ್ತ ಭಗವಂತನನ್ನು ನನ್ನನ್ನು ಕ್ಷಮಿಸು, ನನಗೆ ಬೇರೆ ದಾರಿಯೇ ಇಲ್ಲ ಎಂದು ಪ್ರಾರ್ಥಿಸುತ್ತಿದ್ದಳು. ಕೆಲವು ದಿನಗಳಾದ ಮೇಲೆ ಆ ಸಾಧು ಮತ್ತು ದುಷ್ಟ ಹೆಂಗಸು ಇಬ್ಬರೂ ಕಾಲವಾದರು. ದೇವತೆಗಳು ಬಂದು ಆ ಸ್ತ್ರೀಯನ್ನು ದೇವಲೋಕಕ್ಕೆ ಒಯ್ದರು. ಆದರೆ ಯಮದೂತರು ಸಾಧುವನ್ನು ನರಕಕ್ಕೆ ಒಯ್ದರು. ಆಗ ಸಾಧು, “ನನ್ನನ್ನು ಏತಕ್ಕೆ ನರಕಕ್ಕೆ ಒಯ್ಯುವಿರಿ? ನಾನು ಅತ್ಯಂತ ಪರಿಶುದ್ಧವಾದ ಜೀವನ ನಡೆಸಲಿಲ್ಲವೆ, ಒಳ್ಳೆಯದನ್ನು ಇತರರಿಗೆ ಬೋಧಿಸಲಿಲ್ಲವೆ? ನನ್ನನ್ನು ನರಕಕ್ಕೆ ಒಯ್ಯುವುದು ಏತಕ್ಕೆ? ಇವಳನ್ನು ಸ್ವರ್ಗಕ್ಕೆ ಒಯ್ಯುವುದು ಏತಕ್ಕೆ?' ಎಂದ. ಆಗ ಯಮದೂತರು ಹೇಳಿದರು: “ಅವಳು ಬಲಾತ್ಕಾರಕ್ಕೆ ಸಿಕ್ಕಿ ಪಾಪಕೃತ್ಯಗಳನ್ನು ಮಾಡಿದ್ದರೂ ಅವಳ ಮನಸ್ಸು ಯಾವಾಗಲೂ ದೇವರ ಮೇಲಿತ್ತು. ಅವಳು ಬಿಡುಗಡೆಯನ್ನು ಬೇಡುತ್ತಿದ್ದಳು. ಅದೀಗ ಬಂದಿದೆ. ಆದರೆ ಅದಕ್ಕೆ ವಿರುದ್ದವಾಗಿ ನೀನು ಯಾವಾಗಲೂ ಒಳ್ಳೆಯ ಕೆಲಸವನ್ನು ಮಾಡುತ್ತಿದ್ದರೂ ನಿನ್ನ ಮನಸ್ಸು ಮಾತ್ರ ಇತರರ ದೋಷಗಳ ಮೇಲೆ ನೆಲೆಸಿತ್ತು. ನೀನು ಬರಿಯ ಪಾಪವನ್ನು ನೋಡಿದೆ, ಪಾಪವನ್ನು ಆಲೋಚನೆ ಮಾಡಿದೆ. ಈಗ ಪಾಪವು ಮಾತ್ರ ಇರುವ ಸ್ಥಳಕ್ಕೆ ನೀನು ಹೋಗಬೇಕಾಗಿದೆ.'' ಈ ಕಥೆಯ ನೀತಿ ಸ್ಪಷ್ಟವಾಗಿಯೇ ಇದೆ. ಬಾಹ್ಯವೇಷದಿಂದ ಏನೂ ಪ್ರಯೋಜನವಿಲ್ಲ. ಹೃದಯ ಪರಿಶುದ್ಧವಾಗಿರಬೇಕು. ಮಾನವಕೋಟಿಯ ರಕ್ಷಕರಾಗಲು ಎಂದಿಗೂ ಯತ್ನಿಸಬಾರದು; ಪಾಪಿಗಳನ್ನು ಉದ್ಧಾರಮಾಡುವ ಮಹಾ ಸಾಧುಪುರುಷರಂತೆ ಮೇಲೆ ನಿಲ್ಲಲು ಯತ್ನಿಸಬಾರದು. ಅದರ ಬದಲು ನಮ್ಮನ್ನು ನಾವು ಶುದ್ದೀಕರಿಸಿಕೊಳ್ಳೋಣ. ಹಾಗೆ ಮಾಡಿ ಇತರರಿಗೆ ನೆರವಾಗೋಣ.

ಭೌತಶಾಸ್ತ್ರವನ್ನು ಅತಿಭೌತಿಕಶಾಸ್ತ್ರ ಆವರಿಸಿದೆ. ಇದರಂತೆಯೇ ಯುಕ್ತಿ ಕೂಡ. ಇದು ಅಯುಕ್ತಿಯಲ್ಲಿ ಮೊದಲಾಗಿ ಪುನಃ ಅಯುಕ್ತಿಯಲ್ಲಿ ಕೊನೆಗಾಣುವುದು.\break ಇಂದ್ರಿಯಗ್ರಾಹ್ಯ ಪ್ರಪಂಚದಲ್ಲಿ ಯುಕ್ತಿಯನ್ನು ಮಿತಿಮೀರಿ ಅನುಸರಿಸಿದರೆ ಅದು ಇಂದ್ರಿಯಕ್ಕೆ ಅತೀತವಾದ ಸ್ಥಳಕ್ಕೆ ಒಯ್ಯುವುದು. ಯುಕ್ತಿ ಎಂದರೆ ಸಂಗ್ರಹಿಸಿದ ಮತ್ತು ವರ್ಗಿಕರಿಸಿದ ಇಂದ್ರಿಯಗ್ರಹಣ, ಇದು ನೆನಪಿನ ಆಧಾರದ ಮೇಲೆ ನಿಂತಿದೆ. ನಮ್ಮ ಇಂದ್ರಿಯಗ್ರಹಣಕ್ಕೆ ಅತೀತವಾದುದನ್ನು ನಾವು ಊಹಿಸಲೂ ಆರೆವು, ತರ್ಕಿಸಲೂ ಆರೆವು. ಯುಕ್ತಿಗೆ ಅತೀತವಾಗಿರುವುದು ಯಾವುದೂ ವಿಷಯಜ್ಞಾನಕ್ಕೆ ವಸ್ತುವಾಗಲಾರದು. ಯುಕ್ತಿಯ ಪರಿಮಿತಿ ನಮಗೆ ಭಾಸವಾಗುವುದು. ಆದರೂ ಅಲ್ಲಿ ಯುಕ್ತಿಗೆ ಅತೀತವಾಗಿರುವುದರ ಒಂದು ಕ್ಷಣಿಕ ನೋಟ ದೊರಕುವುದು. ಯುಕ್ತಿಯನ್ನು ಮಾರಿದ ಒಂದು ಸಾಧನ ಮನುಷ್ಯನಿಗೆ ಇದೆಯೇ ಎಂಬ ಪ್ರಶ್ನೆ ಏಳುವುದು. ಯುಕ್ತಿಗೆ ಅತೀತವಾಗಿ ಹೋಗುವ ಶಕ್ತಿ ಮನುಷ್ಯನಲ್ಲಿ ಬಹುಶಃ ಇರಬಹುದು. ಹಿಂದಿನಿಂದಲೂ ಸಾಧು ಸಂತರು ಇಂತಹ ಶಕ್ತಿ ತಮ್ಮಲ್ಲಿದೆ ಎಂದು ಸಾರುತ್ತಿರುವರು. ಸ್ವಭಾವತಃ ಆಧ್ಯಾತ್ಮಿಕ ಭಾವನೆಗಳನ್ನು ಯುಕ್ತಿಗೆ ನಿಲುಕುವ ಭಾಷೆಯಲ್ಲಿ ವ್ಯಕ್ತಪಡಿಸಲಿಕ್ಕೆ ಆಗುವುದೇ ಇಲ್ಲ. ಪ್ರತಿಯೊಬ್ಬ ಸಾಧುವೂ ತನ್ನ ಆಧ್ಯಾತ್ಮಿಕ ಅನುಭವಗಳನ್ನು ವ್ಯಕ್ತಪಡಿಸಲು ಅಸಾಧ್ಯವೆಂದು ಹೇಳಿರುವನು. ಭಾಷೆ ಇದನ್ನು ವ್ಯಕ್ತಪಡಿಸಲು ಪದವನ್ನು ಕೊಡಲಾರದು. ಇದು ನಿಜವಾದ ಅನುಭವ; ಯಾರು ಬೇಕಾದರೂ ಅದನ್ನು ಪಡೆಯಬಹುದೆಂದು ಒತ್ತಿ ಸಾರಬಹುದು. ಹೀಗೆ ಮಾತ್ರ ಅದು ಗೊತ್ತಾಗುವುದು. ಆದರೆ ಸರಿಯಾಗಿ ಅದನ್ನು ವಿವರಿಸುವುದಕ್ಕೆ ಆಗುವುದಿಲ್ಲ. ಧರ್ಮ ಎಂಬುದು ಪ್ರಕೃತಿಯಲ್ಲಿ ಅತೀತವಾಗಿ ಇರುವುದನ್ನು ಮಾನವನಲ್ಲಿ ಅತೀತವಾಗಿ ಇರುವುದರ ಮೂಲಕ ತಿಳಿಯುವುದು. ನಮಗೆ ಮಾನವನ ವಿಷಯದಲ್ಲಿ ತಿಳಿದಿರುವುದು ಬಹಳ ಅಲ್ಪ. ಆದಕಾರಣ ಜಗತ್ತಿನ ವಿಷಯದಲ್ಲಿ ತಿಳಿದಿರುವುದೂ ಕೂಡ ಅಲ್ಪವೇ. ನಮಗೆ ಮಾನವನ ವಿಷಯ ಹೆಚ್ಚು ಗೊತ್ತಾದರೆ ವಿಶ್ವದ ವಿಷಯವೂ ಹೆಚ್ಚು ಗೊತ್ತಾಗುವುದು. ಮಾನವನೆ ಜಗತ್ತಿನಲ್ಲಿರುವ ಎಲ್ಲ ವಸ್ತುಗಳ ಸಾರಸ್ವರೂಪ, ಎಲ್ಲ ಜ್ಞಾನವೂ ಅವನಲ್ಲಿದೆ. ವಿಶ್ವದಲ್ಲಿ ಇಂದ್ರಿಯಗ್ರಹಣದ ಎಲ್ಲೆಯೊಳಗೆ ಬರುವ ಯಾವುದೋ ಕನಿಷ್ಟಾಂಶಕ್ಕೆ ಮಾತ್ರ ನಾವು ಕಾರಣವನ್ನು ಕೊಡಬಲ್ಲೆವು. ಯಾವ ಮೂಲ ನಿಯಮಕ್ಕೂ ನಾವು ಕಾರಣವನ್ನು ಕೊಡಲಾರೆವು. ಒಂದು ವಸ್ತುವಿಗೆ ಕಾರಣವನ್ನು ಕೊಡುವುದು ಎಂದರೆ ಅದನ್ನು ವರ್ಗೀಕರಿಸಿ ಮನಸ್ಸಿನ ಒಂದು ಮೂಲೆಯಲ್ಲಿಡುವುದು ಎಂದು ಅರ್ಥ. ನಾವು ಹೊಸ ವಸ್ತು ಒಂದನ್ನು ಕಂಡರೆ ಅದನ್ನು ಆಗಲೇ ಇರುವ ಯಾವುದಾದರೊಂದು ವರ್ಗದ ಕೆಳಗೆ ತರಲು ಇಚ್ಛಿಸುತ್ತೇವೆ. ಹೀಗೆ ಮಾಡುವ ಪ್ರಯತ್ನವೇ ಯುಕ್ತಿ. ಆ ವಸ್ತುವನ್ನು ಒಂದು ವರ್ಗದ ಕೆಳಗೆ ತಂದ ಮೇಲೆ ಒಂದು ಬಗೆಯ ತೃಪ್ತಿ ಉಂಟಾಗುವುದು. ಆದರೆ ಇಂತಹ ವರ್ಗಿಕರಣದಿಂದ ನಾವು ಸ್ಥೂಲ ಪ್ರಪಂಚವನ್ನು ಮೀರಿ ಹೋಗಲಾರೆವು. ಹಿಂದಿನ ಕಾಲದ ಮಹಾತ್ಮರೆಲ್ಲ ಮನುಷ್ಯನು ಇಂದ್ರಿಯಗಳನ್ನು ಮೀರಿ ಹೋಗಬಲ್ಲ ಎಂದು ಸಾರಿ ಹೇಳುವ ಪ್ರಮಾಣಗಳಾಗಿರುವರು. ಐದು ಸಾವಿರ ವರುಷಗಳ ಹಿಂದೆಯೇ ಉಪನಿಷತ್ತುಗಳು ಭಗವತ್ಸಾಕ್ಷಾತ್ಕಾರವು ಇಂದ್ರಿಯಗಳ ಮೂಲಕ ಸಾಧ್ಯವಿಲ್ಲವೆಂದು ಸಾರಿದವು. ಆಧುನಿಕ ಅಜೇಯತಾ ವಾದಿಗಳೂ ಕೂಡ ಇದನ್ನು ಒಪ್ಪುವರು. ಆದರೆ ವೇದಗಳು ಈ ನಿಷೇಧವನ್ನು ಮೀರಿಹೋಗಿ, ಇಂದ್ರಿಯಗಳಿಂದ ಬಂಧಿತನಾಗಿ ಘನೀಭೂತವಾಗಿರುವ ಜಗತ್ತನ್ನು ಮಾನವ ಮೀರಿ ಹೊಗಬಲ್ಲ ಎಂದು ಸಾರುವುವು. ಅವನು ತೇಲುವ ನೀರ್ಗಲ್ಲಿನಲ್ಲಿ ಎಲ್ಲೋ ಒಂದು ರಂಧ್ರದ ಮೂಲಕ ತೂರಿಹೋಗಿ ಜೀವನ–ಸಮುದ್ರವನ್ನು ಸೇರುವಂತಿದೆ. ಇಂದ್ರಿಯಪ್ರಪಂಚವನ್ನು ಹೀಗೆ ಅತಿಕ್ರಮಿಸಿ ಹೋಗುವುದರಿಂದ ಮಾತ್ರ ಅವನು ತನ್ನ ಆತ್ಮವನ್ನು ಅರಿಯಬಲ್ಲ, ತನ್ನ ನೈಜಸ್ಥಿತಿಯನ್ನು ಅರಿಯಬಲ್ಲ.

ಜ್ಞಾನವೆಂದರೆ ಇಂದ್ರಿಯಗ್ರಹಣದ ಜ್ಞಾನವಲ್ಲ. ನಾವು ಬ್ರಹ್ಮನನ್ನು ತಿಳಿಯಲಾರೆವು; ಆದರೆ ನಾವೇ ಬ್ರಹ್ಮ, ನಾವು ಪೂರ್ಣಬ್ರಹ್ಮರು, ಅಂಶಬ್ರಹ್ಮರಲ್ಲ. ಅವ್ಯಕ್ತವನ್ನು ನಾವು ಎಂದಿಗೂ ವಿಭಾಗಮಾಡಲಾಗುವುದಿಲ್ಲ. ತೋರಿಕೆಯ ವೈವಿಧ್ಯ ನಮಗೆ ಕಾಲದೇಶಗಳ ಮೂಲಕ ಕಾಣುವ ಪ್ರತಿಬಿಂಬ. ಸೂರ್ಯ ಒಂದೇ ಆದರೂ ಕೋಟ್ಯಂತರ ಹಿಮಮಣಿಗಳಲ್ಲಿ ಪ್ರತಿಬಿಂಬಿಸುತ್ತಿರುವಂತೆ, ಜ್ಞಾನದಲ್ಲಿ ನಾವು ವೈವಿಧ್ಯವನ್ನು ಮರೆತು ಏಕತೆಯನ್ನು ಮಾತ್ರ ನೋಡಬೇಕಾಗಿದೆ. ಇಲ್ಲಿ ನೋಡುವವನಿಲ್ಲ, ನೋಡುವ ವಸ್ತುವಿಲ್ಲ, ನೋಡುವ ಕ್ರಿಯೆಯೂ ಇಲ್ಲ. ಇಲ್ಲಿ ನೀನು ಅವನು ನಾನು ಎಂಬುದೇ ಇಲ್ಲ. ಇರುವುದೆಲ್ಲಾ ಒಂದೇ, ಅವಿಭಕ್ತವಾದುದು. ನಾವು ಸದಾ ಇದೇ ಆಗಿರುವೆವು. ಒಮ್ಮೆ ಮುಕ್ತನಾದರೆ ಎಂದೆಂದಿಗೂ ಮುಕ್ತನಾದಂತೆ. ಮನುಷ್ಯ ಕಾರ್ಯಕಾರಣನಿಯಮಗಳಿಂದ ಬಂಧಿತನಲ್ಲ. ದುಃಖವ್ಯಥೆಗಳು ಮನುಷ್ಯನಲ್ಲಿ ಇಲ್ಲ. ಅವು ಚಲಿಸುವ ಮೋಡ ಸೂರ್ಯನನ್ನು ಮುಚ್ಚುವಂತೆ. ಆದರೆ ಮೋಡ ಚಲಿಸುವುದು, ಸೂರ್ಯ ಯಾವ ಕಾಲಕ್ಕೂ ಮೋಡಕ್ಕೆ ಒಳಗಾಗುವುದಿಲ್ಲ. ಇದರಂತೆಯೇ ಮನುಷ್ಯನೂ ಕೂಡ. ಅವನು ಹುಟ್ಟಿಯೂ ಇಲ್ಲ, ಸಾಯುವುದೂ ಇಲ್ಲ, ಅವನು ಕಾಲದೇಶಗಳಲ್ಲಿ ಇಲ್ಲ. ಇವೆಲ್ಲ ಕೇವಲ ಮನಸ್ಸಿನ ಕಲ್ಪನೆ. ಇವೇ ಸತ್ಯವೆಂದು ಭಾವಿಸಿ ಇದರ ಹಿಂದೆ ಇರುವ ಭವ್ಯವಾದ ಸತ್ಯಗಳನ್ನೇ ಮರೆಯುವೆವು, ಕಾಲವೆಂಬುದು ನಾವು ಆಲೋಚಿಸುವ ರೀತಿ. ಆದರೆ ನಾವು ನಿರಂತರ ವರ್ತಮಾನ ಕಾಲದಲ್ಲಿ ಇರುವೆವು. ಒಳ್ಳೆಯದು ಕೆಟ್ಟದ್ದು ಕೇವಲ ನಮ್ಮ ಸಂಬಂಧದಿಂದ ಮಾತ್ರ ಇವೆ. ಒಂದಿಲ್ಲದೆ ಮತ್ತೊಂದು ದೊರಕದು. ಏಕೆಂದರೆ ಒಂದನ್ನು ಬಿಟ್ಟರೆ ಮತ್ತೊಂದಕ್ಕೆ ಅರ್ಥವೇ ಇಲ್ಲ. ನಾವು ದ್ವೈತ ಭಾವನೆಯನ್ನು ಎಲ್ಲಿಯವರೆಗೂ\break ಒಪ್ಪಿಕೊಳ್ಳುವೆವೋ, ಅಥವಾ ವ್ಯಕ್ತಿಯೂ ಬ್ರಹ್ಮವೂ ಬೇರೆ ಬೇರೆ ಎಂದು ಹೇಳುವೆವೋ, ಅಲ್ಲಿಯವರೆಗೆ ಒಳ್ಳೆಯದನ್ನು ಮತ್ತು ಕೆಟ್ಟದ್ದನ್ನು ಒಪ್ಪಿಕೊಳ್ಳಲೇಬೇಕು. ನಾವು ಕೇಂದ್ರಕ್ಕೆ ಹೋಗುವುದರಿಂದ ಮಾತ್ರ, ದೇವರಲ್ಲಿ ನೆಲಸುವುದರಿಂದ ಮಾತ್ರ, ಇಂದ್ರಿಯದ ಭ್ರಾಂತಿಯಿಂದ ಪಾರಾಗಬಲ್ಲೆವು. ಕೊನೆಮೊದಲಿಲ್ಲದ ಆಸೆಯ ಜ್ವರದಿಂದ ಪಾರಾದಾಗ, ನಮಗೆ ಸ್ವಲ್ಪವೂ ವಿರಾಮ ಕೊಡದ ಇಚ್ಛೆಯಿಂದ ಪಾರಾದಾಗ, ನಾವು ತೃಷ್ಣೆಯನ್ನು ಕೊನೆಗೆ ನಿಲ್ಲಿಸಿದಾಗ ಮಾತ್ರ ನಾವು ಒಳ್ಳೆಯದರಿಂದ ಮತ್ತು ಕೆಟ್ಟದ್ದರಿಂದ ಪಾರಾಗಬಲ್ಲೆವು. ಏಕೆಂದರೆ ಆಗ ಎರಡನ್ನೂ ಅತಿಕ್ರಮಿಸಿ ಹೋಗುವೆವು, ತೃಷ್ಣೆಯನ್ನು\break ತೃಪ್ತಿಪಡಿಸುವುದರಿಂದ ಅದು ಮತ್ತಷ್ಟು ವೃದ್ಧಿಯಾಗುವುದು. ಬೆಂಕಿಗೆ ಎಣ್ಣೆಯನ್ನು ಹಾಕಿದರೆ, ಅದು ಮತ್ತೂ ಜೋರಾಗಿ ಉರಿಯುವಂತೆ, ಅದು. ಚಕ್ರವು ಕೇಂದ್ರದಿಂದ ದೂರ ಹೋದಷ್ಟೂ ಅದರ ವೇಗ ಹೆಚ್ಚುತ್ತದೆ, ಅದಕ್ಕೆ ವಿರಾಮ ಇಲ್ಲ. ಕೇಂದ್ರದ ಸಮೀಪಕ್ಕೆ ಬನ್ನಿ, ಆಸೆಯನ್ನು ನಿಗ್ರಹಿಸಿ, ಅದನ್ನು ಹೊರದೂಡಿ, ಭ್ರಾಂತಿಯ ಅಹಂ ಹೋಗಲಿ, ಆಗ ನಮ್ಮ ದೃಷ್ಟಿ ಸ್ಪಷ್ಟವಾಗಿ, ನಾವು ದೇವರನ್ನು ನೋಡುವೆವು. ಇಹ ಮತ್ತು ಪರಲೋಕಗಳನ್ನು ತ್ಯಜಿಸಿದಾಗ ಮಾತ್ರವೇ ನಿಜವಾದ ಆತ್ಮನ ಮೇಲೆ ಅಚಲವಾಗಿ ನಿಲ್ಲುವ ಸ್ಥಿತಿಗೆ ಬರುತ್ತೇವೆ. ಎಲ್ಲಿಯವರೆಗೆ ನಾವು ಯಾವುದನ್ನಾದರೂ ಆಸೆಯಿಂದ ನಿರೀಕ್ಷಿಸುತ್ತೇವೋ ಅಲ್ಲಿಯವರೆಗೆ ಆಸೆ ನಮ್ಮನ್ನು ಆಳುತ್ತಿರುವುದು. ಒಂದು ಕ್ಷಣವಾದರೂ ಆಸೆ ಅಳಿದು ಹೋದರೆ ಆಗ ಮೋಡಜಾರುವುದು.ಒಬ್ಬ ತಾನೇ ಸರ್ವವೂ ಆಗಿರುವಾಗ ಇಚ್ಚಿಸುವುದಾವುದನ್ನು? ಎಲ್ಲವನ್ನು ತ್ಯಜಿಸಿ ಆತ್ಮತೃಪ್ತರಾಗುವುದೇ ಜ್ಞಾನ. ಇಲ್ಲ ಎಂದರೆ ನೀವು ಇಲ್ಲವಾಗುವಿರಿ, ಇದೆ ಎಂದರೆ ನೀವು ಇರುವಿರಿ. ಆಂತರ್ಯದಲ್ಲಿರುವ ಆತ್ಮನನ್ನು ಪೂಜಿಸಿ. ಅದಲ್ಲದೆ ಮತ್ತಾವುದೂ ಇಲ್ಲ. ನಮ್ಮನ್ನು ಮರೆಗೊಳಿಸುವುದಲ್ಲ ಮಾಯೆ, ಭ್ರಾಂತಿ.

\begin{center}
೭
\end{center}

ಪ್ರಪಂಚದಲ್ಲಿರುವುದರ ನೈಜಸ್ಥಿತಿಯೇ ಆತ್ಮ. ಆದರೆ ಅದೆಂದಿಗೂ ಪರಿಮಿತಿಗೆ ಸಿಲುಕದು. ನಾವೇ ಅದು ಎಂದು ಅರಿತೊಡನೆಯೆ, ನಾವು ಮುಕ್ತರಾಗುವೆವು. ನಾವು ಮರ್ತ್ಯರಾಗಿರುವವರೆಗೆ ಇಲ್ಲಿ ಎಂದೆಂದಿಗೂ ಮುಕ್ತರಾಗಲಾರೆವು. ನಿತ್ಯ ಮರ್ತ್ಯರೆನ್ನುವುದು ವಿರೋಧಾಭಾಸ. ಏಕೆಂದರೆ ಮರ್ತ್ಯರೆಂದರೆ ವಿಕಾರಕ್ಕೆ ಒಳಗಾಗುವವರು ಎಂದು ಅರ್ಥ. ಆದರೆ ಅವಿಕಾರಿಯಾದುದು ಮಾತ್ರ ಮುಕ್ತವಾಗಬಲ್ಲದು. ಆತ್ಮ ಒಂದೇ ನಿತ್ಯ, ಮುಕ್ತ, ಅದೇ ನಮ್ಮ ಸ್ವಭಾವ. ನಮಗೆ ಈ ಆಂತರಿಕ ಸ್ವಾತಂತ್ರ್ಯದ ಅನುಭವವಿದೆ. ಎಷ್ಟೇ ಸಿದ್ದಾಂತಗಳು, ನಂಬಿಕೆಗಳು ಅದನ್ನು ವಿರೋಧಿಸಲಿ, ನಮಗೆ ಅದು ಗೊತ್ತಿದೆ. ನಮ್ಮ ಪ್ರತಿಯೊಂದು ಕ್ರಿಯೆಯೂ ಅದನ್ನು ದೃಢಪಡಿಸುವುದು. ಇಚ್ಛೆ ಸ್ವತಂತ್ರವಲ್ಲ, ಅದು ತೋರಿಕೆಯ ಸ್ವಾತಂತ್ರ್ಯ, ನಿಜವಾದ ಸ್ವಾತಂತ್ರ್ಯದ ಒಂದು ಛಾಯೆ ಮಾತ್ರ. ಜಗತ್ತು ಎಂಬುದು ಕಾರ್ಯಕಾರಣಗಳ ಕೊನೆಮೊದಲಿಲ್ಲದ ಒಂದು ಪರಿಣಾಮವಾದರೆ ಅದಕ್ಕೆ ಸಹಾಯವನ್ನು ಮಾಡುವುದೆಲ್ಲಿಂದ? ಅದನ್ನು ಉದ್ಧರಿಸುವವರು ಬೇರೆ ನಿಂತುಕೊಳ್ಳಲು ಒಂದು ಸ್ಥಳ ಇರಬೇಕು. ಇಲ್ಲದೆ ಇದ್ದರೆ ಸಂಸಾರದಲ್ಲಿ ಮುಳುಗಿ ಹೋಗುತ್ತಿರುವವರನ್ನು ಆಚೆಗೆ ಎಳೆಯುವುದು ಹೇಗೆ? ನಾನೊಂದು ಕೀಟವೆಂದು ಭಾವಿಸುವ ಮತಭ್ರಾಂತನು ಕೂಡ ತಾನು ಪುಣ್ಯಾತ್ಮನಾಗುವ ಮಾರ್ಗದಲ್ಲಿರುವೆ ಎಂದು ಭಾವಿಸುವನು. ಅವನು ಒಂದು ಕೀಟದಲ್ಲಿಯೂ ಪುಣ್ಯಾತ್ಮನನ್ನು ನೋಡುವನು.

ಮಾನವ ಜೀವನಕ್ಕೆ ಎರಡು ಗುರಿಗಳಿವೆ. ಅವು ನಿಜವಾದ ಅರಿವು (ವಿಜ್ಞಾನ) ಮತ್ತು ಆನಂದ. ಸ್ವಾತಂತ್ರ್ಯವಿಲ್ಲದೆ ಇವೆರಡೂ ಅಸಾಧ್ಯ. ಇವೇ ಎಲ್ಲಾ ಜೀವನದ ಒರೆಗಲ್ಲು. ನಾವು ನಿತ್ಯ ಏಕತೆಯನ್ನು ಎಷ್ಟು ಅನುಭವಿಸಬೇಕು ಎಂದರೆ ನಾವು ಎಲ್ಲ ಪಾಪಿಗಳಿಗಾಗಿ ಅಳಬೇಕು, ಏಕೆಂದರೆ ನಾವೇ ಅಲ್ಲಿ ಪಾಪ ಮಾಡುತ್ತಿರುವವರು. ಆತ್ಮತ್ಯಾಗವೆ ಸನಾತನ ನಿಯಮ; ಆತ್ಮಪ್ರತಿಷ್ಠೆಯಲ್ಲ. ಎಲ್ಲ ಒಂದೇ ಆಗಿರುವಾಗ ಯಾವ ಆತ್ಮನ ಪ್ರತಿಷ್ಠೆ? ಹಕ್ಕುಗಳೆಂಬುವು ಇಲ್ಲ; ಎಲ್ಲವೂ ಪ್ರೇಮ. ಏಸುವು ಬೋಧಿಸಿದ ಅಮರ ಸಂದೇಶಗಳನ್ನು ಯಾರೂ ಎಂದಿಗೂ ಅನುಷ್ಠಾನಕ್ಕೆ ತರಲಿಲ್ಲ. ಅವನ ಮಾರ್ಗವನ್ನು ಅನುಸರಿಸಿ ಪ್ರಪಂಚವನ್ನು ರಕ್ಷಿಸಲು ಸಾಧ್ಯವೆ ಎಂದು ನೋಡೋಣ. ಅವನ ಬೋಧನೆಗೆ ವಿರೋಧವಾದ ಮಾರ್ಗ ಜಗತ್ತನ್ನು ಹಾಳುಮಾಡಿದೆ. ನಿರ್ಮಮತ್ವವಲ್ಲದೆ ಮಮತ್ವ ಎಂದಿಗೂ ಈ ಪ್ರಶ್ನೆಯನ್ನು ಬಗೆಹರಿಸಲಾರದು. ಹಕ್ಕು ಎಂಬ ಭಾವನೆಗೆ ಒಂದು ಪರಿಮಿತಿ ಇದೆ. ನಿಜವಾಗಿ ನನ್ನದು ನಿನ್ನದು ಎಂಬುದಿಲ್ಲ. ಏಕೆಂದರೆ ನಾನೇ ನೀನು, ನೀನೇ ನಾನು. ನಮಗೆ ಜವಾಬ್ದಾರಿಗಳು ಇವೆ, ಹಕ್ಕುಗಳಿಲ್ಲ. ಈ ಮಿತಿಗಳೇ ಭ್ರಾಂತಿ. ನಮ್ಮನ್ನು ಬಂಧನದಲ್ಲಿ ಇಡುವುವು ಇವೇ. ನಾನು ಜಾನ್ ಎಂದೊಡನೆಯೇ ನನಗೆ ಪ್ರತ್ಯೇಕವಾಗಿ ಕೆಲವು ವಸ್ತುಗಳು ಬೇಕಾಗುತ್ತವೆ, ತಕ್ಷಣ ನಾನು ನನ್ನದು ಎನ್ನುತ್ತೇನೆ. ಹೀಗೆ ಮಾಡುವುದರಿಂದ ಪ್ರತ್ಯೇಕ ಭಾವನೆಗೆ ಅವಕಾಶ ಕೊಡುತ್ತೇನೆ. ಪ್ರತ್ಯೇಕ ಭಾವನೆ ಹೆಚ್ಚುತ್ತಾ ಹೋದಷ್ಟೂ ನಮ್ಮ ಬಂಧನ ಹೆಚ್ಚುತ್ತಾ ಹೋಗುವುದು. ಅವಿಭಕ್ತವಾದುದು ಒಂದೆ. ನಾವೇ ಅದು. ಏಕತೆಯೊಂದೇ ಪ್ರೇಮ ಮತ್ತು ನಿರ್ಭಯತೆ. ಪ್ರತ್ಯೇಕ ಭಾವನೆ ದ್ವೇಷಕ್ಕೆ, ಅಂಜಿಕೆಗೆ ಎಡೆಕೊಡುವುದು, ಏಕತೆ ನಿಯಮವನ್ನು ಸಾರ್ಥಕಗೊಳಿಸುವುದು. ಇಲ್ಲಿ ಪ್ರಪಂಚದಲ್ಲಿ ಸಣ್ಣ ಸಣ್ಣ ಸ್ಥಳಗಳ ಸುತ್ತಲೂ ಬೇಲಿಗಳನ್ನು ಕಟ್ಟಿ ಉಳಿದವರನ್ನು ಬೇರ್ಪಡಿಸಲು ಯತ್ನಿಸುವೆವು. ಆದರೆ ಆಕಾಶದಲ್ಲಿ ನಾವು ಹಾಗೆ ಮಾಡಲಾರೆವು. ಕೇವಲ ಒಂದು ಕೋಮಿನ ಭಾವನೆಯುಳ್ಳ ಧರ್ಮ, “ಈ ಮಾರ್ಗ ಒಂದೇ ಮುಕ್ತಿಗೆ ದಾರಿ, ಉಳಿದವೆಲ್ಲ ತಪ್ಪು'' ಎಂದು ಸಾರಿದಾಗ ಹಾಗೆಯೇ ಮಾಡುವುದು. ಈ ಸಣ್ಣ ಬೇಲಿಗಳನ್ನು ಕಿತ್ತೊಗೆಯುವುದೇ ನಮ್ಮ ಗುರಿಯಾಗಬೇಕು. ಅದನ್ನು ವಿಸ್ತರಿಸಬೇಕು, ಅದರ ಸುಳಿವೇ ಇಲ್ಲದಂತೆ ಹೋಗಿ, ಎಲ್ಲ ಧರ್ಮಗಳೂ ದೇವರೆಡೆಗೆ ಒಯ್ಯುವುವು ಎಂಬುದನ್ನೇ ಅರಿಯಬೇಕು. ಈ ಅಲ್ಪ ಕ್ಷುದ್ರ ಅಹಂಕಾರವನ್ನು ಬಲಿಕೊಡಬೇಕು. ನವಬಾಳಿನ ದೀಕ್ಷೆ, ಹಳೆಯ ವ್ಯಕ್ತಿಯ ನಾಶ, ಹೊಸ ವ್ಯಕ್ತಿಯ ಉದಯ, ತೋರಿಕೆಯ ಅಹಂಕಾರದ ನಾಶ, ವಿಶ್ವ ವ್ಯಾಪಿಯಾಗಿರುವ ಆತ್ಮನ ಸಾಕ್ಷಾತ್ಕಾರ – ಇವೆಲ್ಲ ಇದೇ, ಅದೇ ಜ್ಞಾನಸ್ನಾನ.

ವೇದಗಳಲ್ಲಿ ಕರ್ಮಕಾಂಡ ಮತ್ತು ಜ್ಞಾನಕಾಂಡ ಎಂಬ ಎರಡು ಭಾಗಗಳಿವೆ. ಧಾರ್ಮಿಕ ಭಾವನೆ ಹೇಗೆ ಬೆಳೆಯಿತು ಎಂಬುದನ್ನೆಲ್ಲ ನಾವು ವೇದಗಳಲ್ಲಿ ನೋಡಬಹುದು. ಇದು ಸಾಧ್ಯವಾದುದು ಹೇಗೆಂದರೆ ಮೇಲಿನ ಸತ್ಯವನ್ನು ಸೇರಿ ಆದಮೇಲೆ ಕೆಳಗಿರುವುದನ್ನು ಹಾಗೆಯೇ ಇಟ್ಟರು. ಏಕೆಂದರೆ ಋಷಿಗಳಿಗೆ ಈ ಸೃಷ್ಟಿ ಅನಾದಿಯಾಗಿರುವುದರಿಂದ ಜ್ಞಾನಕ್ಕೆ ಮೊದಲಿನ ಹಂತದ ಆವಶ್ಯಕವಿರುವವರು ಯಾವಾಗಲೂ ಇದ್ದೇ ಇರುವರು ಎಂಬುದು ಗೊತ್ತಿತ್ತು. ಶ್ರೇಷ್ಠ ಸತ್ಯ ಎಲ್ಲರೆದುರು ಇದ್ದರೂ ಅದು ಅರ್ಥವಾಗುವುದು ಬಹಳ ಕಡಮೆ ಜನರಿಗೆ ಮಾತ್ರ. ಇತರ ಧರ್ಮಗಳಲ್ಲೆಲ್ಲ ಬಹುಪಾಲು ಕೊನೆಯ ಅಥವಾ ಅತಿ ಶ್ರೇಷ್ಠ ತತ್ತ್ವವನ್ನು ಮಾತ್ರ ಇಟ್ಟಿರುವರು. ಇದರ ಪರಿಣಾಮವಾಗಿ ಹಳೆಯ ಭಾವನೆಗಳೆಲ್ಲ ನಾಶವಾಗಿಹೋದವು. ಹೊಸ ಭಾವನೆಯನ್ನು ಎಲ್ಲೋ ಕೆಲವು ಜನರು ಮಾತ್ರ ತಿಳಿದುಕೊಂಡರು. ಬಹುಪಾಲು ಜನರಿಗೆ ಕ್ರಮೇಣ ಇದರ ಅರ್ಥವೇ ಗೊತ್ತಾಗಲಿಲ್ಲ. ಹಳೆಯ ಅಧಿಕಾರ–ಆಚಾರಗಳ ವಿರುದ್ಧ ದಂಗೆ ಏಳುವುದರಲ್ಲಿ ನಾವು ಇದನ್ನು ಸ್ಪಷ್ಟವಾಗಿ ಕಾಣುವೆವು, ಆಧುನಿಕರು ಅವನ್ನು ಸುಮ್ಮನೆ ಒಪ್ಪಿಕೊಳ್ಳುವ ಬದಲು ಅದು ಹೇಗೆ ಸರಿ, ಅದಕ್ಕೆ ಕಾರಣ ಕೊಡಿ, ಎನ್ನುವರು. ಯಾವುದನ್ನು ಒಪ್ಪಿಕೊಳ್ಳಬೇಕೋ ಅದನ್ನು ಸಕಾರಣವಾಗಿ ವಿವರಿಸಿ ಎನ್ನುವರು ಈಗಿನವರು. ಈಗ ಕ್ರೈಸ್ತ ಧರ್ಮದಲ್ಲಿರುವ ಮುಕ್ಕಾಲು ಪಾಲು ವಿಷಯಗಳು ಹಳೆಯ ಪೇಗನ್ ನಂಬಿಕೆಗೆ ಮತ್ತು ಆಚಾರಗಳಿಗೆ ಕೊಡುವ ಹೊಸ ಹೆಸರುಗಳು ಮತ್ತು ಅರ್ಥಗಳಾಗಿವೆ. ಪುರಾತನ ಮೂಲವನ್ನು ಹಾಗೆಯೇ ಇಟ್ಟು ಅದನ್ನು ಬದಲಾಯಿಸುವುದಕ್ಕೆ ಕಾರಣವನ್ನು ಕೊಟ್ಟಿದ್ದರೆ ಎಷ್ಟೋ ವಿಷಯಗಳು ಮತ್ತೂ ಸ್ಪಷ್ಟವಾಗುತ್ತಿದ್ದುವು. ವೇದಗಳು ಹಳೆಯ ಭಾವನೆಗಳನ್ನು ಹಾಗೆಯೇ ಇಟ್ಟವು. ಅದನ್ನು ಏತಕ್ಕೆ ಇಟ್ಟರು ಎಂದು ವಿವರಿಸುವುದಕ್ಕೆ ದೊಡ್ಡ ಭಾಷ್ಯವೇ ಬೇಕಾಯಿತು. ಇದರಿಂದ ಹಲವು ಮೂಢನಂಬಿಕೆಗಳೂ ಹುಟ್ಟಿದವು. ಹಳೆಯ ಆಚಾರದ ಅರ್ಥವೆಲ್ಲಾ ಮಾಯವಾಗಿ ಹೋದರೂ ಅದೇ ಆಚಾರವನ್ನು ನಂಬಿ ಕುಳಿತರು. ಈಗ ಆಚರಣೆಯಲ್ಲಿರುವ ಎಷ್ಟೋ ಧಾರ್ಮಿಕ ಕರ್ಮಗಳಲ್ಲಿ ಅನೇಕ ಪದಗಳನ್ನು ಉಪಯೋಗಿಸುವರು; ಅವು ಈಗ ಮರೆತು ಹೋಗಿರುವ ಭಾಷೆಗೆ ಸೇರಿದವು; ಅವಕ್ಕೆ ಯಾವ ನಿಜವಾದ ಅರ್ಥವನ್ನೂ ಕಲ್ಪಿಸುವುದಕ್ಕೆ ಆಗುವುದಿಲ್ಲ. ಕ್ರಿಸ್ತಶಕಕ್ಕಿಂತ ಬಹು ಹಿಂದೆಯೇ ವಿಕಾಸದ ಭಾವನೆ ವೇದಗಳಲ್ಲಿ ಬರುವುದು. ಆದರೆ ಡಾರ್ವಿನ್ ಇದು ಸತ್ಯವೆಂದು ಹೇಳುವವರೆಗೆ ಇದನ್ನು ಕೇವಲ ಹಿಂದೂಗಳ ಮೂಢನಂಬಿಕೆ ಎಂದು ತಿಳಿದಿದ್ದರು.

ಕರ್ಮಕಾಂಡದಲ್ಲಿ ಬಾಹ್ಯಪೂಜೆ ಪ್ರಾರ್ಥನೆಗಳು ಇವೆಲ್ಲ ಬರುವುವು. ಇವು ಕೇವಲ ನಿರ್ಜಿವ ಆಚಾರವಾಗಿರದೆ, ನಿಃಸ್ವಾರ್ಥ ದೃಷ್ಟಿಯಿಂದ ಭಾವಪೂರ್ವಕವಾಗಿ ಮಾಡಿದರೆ ಒಳ್ಳೆಯವು. ಇವು ಚಿತ್ತಶುದ್ದಿಯನ್ನುಂಟುಮಾಡುತ್ತವೆ. ಕರ್ಮಯೋಗಿ ತನಗಿಂತ ಮುಂಚೆ ಇತರರನ್ನು ಉದ್ದರಿಸಲು ಯತ್ನಿಸುವನು. ಇತರರ ಮುಕ್ತಿಗೆ ಸಹಾಯಮಾಡುವುದೇ ಅವನ ಮುಕ್ತಿಯಾಗಿ ಪರಿಣಮಿಸುವುದು. ಭಗವಂತನ ಭಕ್ತರ ಸೇವೆಯೇ ಪರಮ ಪೂಜೆಯಾಗುವುದು. ಒಬ್ಬ ಮಹಾಜ್ಞಾನಿ, “ನಾನು ಪ್ರಪಂಚದ ಪಾಪವನ್ನೆಲ್ಲಾ ಹೊತ್ತು ನರಕಕ್ಕೆ ಹೋದರೂ ಚಿಂತೆಯಿಲ್ಲ, ಪ್ರಪಂಚ ಉದ್ದಾರವಾಗಲಿ" ಎಂದಿರುವನು. ಇಂತಹ ನಿಜವಾದ ಪೂಜೆಯೇ ಪ್ರಚಂಡ ಆತ್ಮತ್ಯಾಗಕ್ಕೆ ಒಯ್ಯುವುದು. ಒಬ್ಬ ಸಾಧುಪುರುಷ, ಒಂದು ನಾಯಿ ಸ್ವರ್ಗಕ್ಕೆ ಹೋಗಲಿ ಎಂದು ತನ್ನ ಪುಣ್ಯವನ್ನೆಲ್ಲಾ ಅದಕ್ಕೆ ಧಾರೆ ಎರೆದುಕೊಟ್ಟನಂತೆ. ಏಕೆಂದರೆ ಆ ನಾಯಿ ಬಹಳಕಾಲ ಅವನೊಂದಿಗೆ ಇತ್ತು. ಆತನೆ ಅದರ ಬದಲು ನರಕಕ್ಕೆ ಹೋಗಲು ಸಿದ್ಧನಾಗಿದ್ದನು.

ಜ್ಞಾನವೊಂದೇ ಮನುಷ್ಯನನ್ನು ಉದ್ಧಾರಮಾಡಬಲ್ಲುದೆಂದು ಜ್ಞಾನಕಾಂಡ ಸಾರುವುದು. ಅಂದರೆ ಅವನು ಮುಕ್ತಿ ಪಡೆಯುವಷ್ಟು ಜ್ಞಾನಿಯಾಗಬೇಕು. ಜ್ಞಾನವೇ ಪರಮಗುರಿ ಆಗಬೇಕು. ಅಂದರೆ ಜ್ಞಾತೃವೇ ತನ್ನನ್ನು ತಾನು ತಿಳಿದುಕೊಳ್ಳಬೇಕು. ಏಕಮಾತ್ರ ವಿಷಯವಾಗಿರುವ ಆತ್ಮವು ತನ್ನನ್ನು ತಾನೇ ತಿಳಿದುಕೊಳ್ಳುವುದಕ್ಕೆ ವ್ಯಕ್ತವಾಗಿದೆ. ಕನ್ನಡಿ ಸ್ಪಷ್ಟವಾಗಿದ್ದಷ್ಟೂ ಅದರಲ್ಲಿ ಬೀಳುವ ಪ್ರತಿಬಿಂಬ ಸ್ಪಷ್ಟವಾಗುವುದು. ಮನುಷ್ಯನೆ\break ಕನ್ನಡಿಗಳಲ್ಲೆಲ್ಲ ಶ್ರೇಷ್ಠನು. ಮನುಷ್ಯ ಎಷ್ಟು ಪರಿಶುದ್ದನಾದರೆ ಅಷ್ಟು ಸ್ಪಷ್ಟವಾಗಿ ಭಗವಂತನನ್ನು ಪ್ರತಿಬಿಂಬಿಸುವನು. ಮಾನವನು ದೇವರಿಂದ ಬೇರೆಯಾಗಿ ದೇಹವೇ ತಾನೆಂದು ಭಾವಿಸುವ ತಪ್ಪು ಮಾಡುವನು. ಈ ತಪ್ಪು ಮಾಯೆಯಿಂದ ಜನಿತವಾಗುವುದು. ಇದು ಕೇವಲ ಭ್ರಾಂತಿ ಎನ್ನಲಾಗುವುದಿಲ್ಲ; ಸತ್ಯವನ್ನು ಅದಿರುವಂತಲ್ಲದೆ ಬೇರೆಯಾಗಿ ಕಾಣುವುದು. ದೇಹವೇ ತಾನೆಂಬ ಭ್ರಾಂತಿಯೇ ಅಸಮತೆಗೆ ಕಾರಣ. ಇದೇ ಎಲ್ಲ ಹೋರಾಟಕ್ಕೆ ಮತ್ತು ಅಸೂಯೆಗೆ ಕಾರಣ. ಎಲ್ಲಿಯವರೆಗೆ ನಾವು ಅಸಮತೆಯನ್ನು ನೋಡುವೆವೋ ಅಲ್ಲಿಯವರೆಗೆ ನಮಗೆ ನಿಜವಾದ ಆನಂದದ ಅನುಭವವಾಗಿಲ್ಲ ಎಂದು ಅರ್ಥ. ಅಜ್ಞಾನ ಮತ್ತು ಅಸಮಾನತೆ ಇವೆರಡೇ ದುಃಖಕ್ಕೆಲ್ಲಾ ಮೂಲ ಎನ್ನುವುದು ಜ್ಞಾನ.

ಮಾನವನು ಪ್ರಪಂಚದ ಆಘಾತಕ್ಕೆ ಸಿಕ್ಕಿ ಜರ್ಝರಿತನಾದ ಮೇಲೆ ಮುಕ್ತನಾಗಬೇಕೆಂಬ ಆಸೆ ಅವನಲ್ಲಿ ಏಳುವುದು, ಜನನಮರಣಗಳ ಕೋಟಲೆಯಿಂದ ಪಾರಾಗಲು ಮಾರ್ಗವನ್ನು ಹುಡುಕಿ ತನ್ನ ನೈಜಸ್ಥಿತಿಯನ್ನು ಅರಿತು ಮುಕ್ತನಾಗುವನು. ಅನಂತರ ಸೃಷ್ಟಿಯನ್ನು ಒಂದು ದೊಡ್ಡ ಯಂತ್ರ ಎಂಬಂತೆ ನೋಡುವನು. ಪುನಃ ಆ ಯಂತ್ರದ ಚಕ್ರಕ್ಕೆ ತನ್ನ ಬೆರಳು ಸಿಕ್ಕಿಕೊಳ್ಳದಂತೆ ಜೋಪಾನವಾಗಿರುವನು. ಮುಕ್ತನಿಗೆ ಇನ್ನು ಯಾವ ಕರ್ತವ್ಯವೂ ಇಲ್ಲ. ಮುಕ್ತನನ್ನು ಯಾವ ಶಕ್ತಿ ಹಿಡಿದು ನಿಲ್ಲಿಸಬಲ್ಲದು? ಅವನು ಒಳ್ಳೆಯದನ್ನು ಮಾಡುವನು. ಏಕೆಂದರೆ ಅದೇ ಅವನ ಸ್ವಭಾವ, ಯಾವುದೇ ಕಾಲ್ಪನಿಕ ಕರ್ತವ್ಯ ಇವನನ್ನು ನಿರ್ಬಂಧಿಸುತ್ತಿದೆ ಎಂದಲ್ಲ. ವಿಷಯೇಂದ್ರಿಯಗಳ ಬಂಧನದಲ್ಲಿರುವವರಿಗೆ ಇದು ಅನ್ವಯಿಸುವುದಿಲ್ಲ. ಯಾವನು ತನ್ನ ಕೆಳಗಿನ ಸ್ವಭಾವವನ್ನು ಮಾರಿ ಹೋಗಿರುವನೋ ಅವನಿಗೆ ಮಾತ್ರ ಈ ಸ್ವಾತಂತ್ರ್ಯ, ಅವನು ತನ್ನ ಆತ್ಮನ ಆಧಾರದ ಮೇಲೆ ನಿಲ್ಲುವನು. ಅವನು ಯಾವ ನಿಯಮಕ್ಕೂ ಅಡಿಯಾಳು ಅಲ್ಲ, ಅವನೇ ಮುಕ್ತ ಮತ್ತು ಪೂರ್ಣ. ಅವನು ಹಳೆಯ ಮೂಢ ನಂಬಿಕೆಗಳಿಂದ ಪಾರಾಗಿರುವನು. ಅವನು ಸಂಸಾರಚಕ್ರದಿಂದ ಮುಕ್ತನಾಗಿರುವನು. ಪ್ರಕೃತಿ ಕೇವಲ ನಮ್ಮ ಆತ್ಮದ ಪ್ರತಿಬಿಂಬ. ಮಾನವನಿಗೆ ಕೆಲಸಮಾಡುವುದಕ್ಕೆ ಒಂದು ಮಿತಿ ಇದೆ. ಆದರೆ ಅವನ ಆಸೆಗೆ ಒಂದು ಮಿತಿ ಇಲ್ಲ. ಆದಕಾರಣವೇ ನಾವು ಕೆಲಸಮಾಡುವುದನ್ನು ಬಿಟ್ಟು ಮತ್ತೊಬ್ಬರ ಶ್ರಮವನ್ನೇ ಸ್ವಾಧೀನಪಡಿಸಿಕೊಂಡು ಅದರ ಮೂಲಕ ಅನುಭವಿಸಲು ಯತ್ನಿಸುವೆವು. ನಮ್ಮ ತೃಪ್ತಿಗಾಗಿ ಯಂತ್ರವನ್ನು ಕಂಡುಹಿಡಿದರೆ ಅದರಿಂದ ನಮಗೆ ಎಂದಿಗೂ ಒಳ್ಳೆಯದಾಗುವುದಿಲ್ಲ. ಏಕೆಂದರೆ ಒಂದು ಆಸೆಯನ್ನು ತೃಪ್ತಿ ಮಾಡಿದರೆ ಅದಕ್ಕೆ ಕೊನೆಯಿಲ್ಲದೆ ಇನ್ನೂ ಹೆಚ್ಚು ಆಸೆಗಳು ಬೆಳೆಯುತ್ತಾ ಹೋಗುವುವು. ಸಾಯುವಾಗಲೂ ಆಸೆ ಇನ್ನೂ ತೃಪ್ತಿಯಾಗದೆ ಇರುವುದರಿಂದ ಅವುಗಳ ತೃಪ್ತಿಗಾಗಿ ಪುನಃ ಪುನಃ ಹುಟ್ಟಬೇಕಾಗಿದೆ. ನಾವು ಮನುಷ್ಯಜನ್ಮಕ್ಕೆ ಬರುವುದಕ್ಕೆ ಮುಂಚೆ ಎಂಟು ಕೋಟಿ ಜನ್ಮಗಳನ್ನು ತಾಳಿದ್ದೆವು ಎನ್ನುವನು ಹಿಂದೂ. ಆಸೆಯನ್ನು ನಾಶಮಾಡಿ ಬಂಧನದಿಂದ ಪಾರಾಗಿ ಎನ್ನುವುದು ಜ್ಞಾನ. ಇದೊಂದೇ ಮಾರ್ಗ. ಜನನಕ್ಕೆ ಮೂಲವಾದ ಕಾರಣಗಳನ್ನೆಲ್ಲ ನಾಶಮಾಡಿ ಮುಕ್ತರಾಗಿ. ಸ್ವಾತಂತ್ರ್ಯ ಮಾತ್ರ ನಿಜವಾದ ನೀತಿಗೆ ದಾರಿ. ಕಾರ್ಯಕಾರಣಗಳ ತುದಿಮೊದಲಿಲ್ಲದ ಸರಪಳಿ ಮಾತ್ರ ಇದ್ದಿದ್ದರೆ ನಿರ್ವಾಣವನ್ನು ಪಡೆಯಲು ಆಗುತ್ತಿರಲಿಲ್ಲ. ಬಂಧನದಲ್ಲಿ ಬಿದ್ದಿರುವ ಈ ತೋರಿಕೆಯ ಜೀವಾತ್ಮನ ನಾಶವಾಗಬೇಕು. ಮುಕ್ತಿಯೆಂದರೆ ಇದೇ; ಕಾರ್ಯಕಾರಣಗಳ ಉಪಟಳದಿಂದ ಪಾರಾಗುವುದು ಎಂದು ಅರ್ಥ.

ನಮ್ಮ ನೈಜಸ್ವಭಾವ ಒಳ್ಳೆಯದು, ಮುಕ್ತವಾಗಿರುವುದು, ಪರಿಶುದ್ದ ವಾಗಿರುವುದು. ಅದೆಂದಿಗೂ ಕೆಟ್ಟದ್ದನ್ನು ಮಾಡಲಾರದು, ಕೆಟ್ಟದ್ದು ಆಗಲಾರದು. ದೇವರನ್ನು ನಮ್ಮ ಮನಸ್ಸಿನ ಮತ್ತು ಕಣ್ಣಿನ ಮೂಲಕ ನೋಡಿದಾಗ ಅವನನ್ನು ಅದು ಇದು ಎನ್ನುವೆವು. ಆದರೆ ಸತ್ಯವಾಗಿರುವುದು ಒಂದೇ. ವೈವಿಧ್ಯವೆಲ್ಲ ಆ ಏಕದ ಬೇರೆಬೇರೆ ವಿವರಣೆಗಳು ಮಾತ್ರ. ನಾವು ಇನ್ನೇನೂ ಆಗುವುದಿಲ್ಲ, ನಮ್ಮ ಸಹಜಸ್ಥಿತಿಗೆ ಬರುತ್ತೇವೆ, ಅಷ್ಟೆ. ದುಃಖದ ಮೂಲವೆಲ್ಲ ಅಜ್ಞಾನ ಮತ್ತು ಜಾತಿ ಎಂಬ ಬುದ್ಧನ ನಿರ್ಧಾರವನ್ನು ವೇದಾಂತ ಸ್ವೀಕರಿಸಿದೆ. ಏಕೆಂದರೆ ಇದು ಶ್ರೇಷ್ಠ ವಿವರಣೆ, ಮಾನವಶ್ರೇಷ್ಠನ ಅಂತರ್‌ದೃಷ್ಟಿಯನ್ನು ಇದು ವ್ಯಕ್ತಪಡಿಸುವುದು. ಹಾಗಾದರೆ ನಾವು ಧೀರರಾಗೋಣ. ನಿಸ್ಪೃಹಿಗಳಾಗೋಣ. ನಾವು ಯಾವ ಮಾರ್ಗವನ್ನಾಗಲಿ ಶ್ರದ್ದೆಯಿಂದ ಅನುಸರಿಸಿದರೆ ಅದು ನಮ್ಮನ್ನು ಮುಕ್ತಿಗೆ ಒಯ್ಯಲೇಬೇಕು. ನೀವು ಸರಪಳಿಯ ಒಂದು ಕೊಂಡಿಯನ್ನು ಹಿಡಿದುಕೊಂಡರೆ ಕ್ರಮೇಣ ಇಡೀ ಸರಪಳಿಯು ನಿಮ್ಮ ಕೈಗೆ ಬರುವುದು. ನೀವು ಮರದ ಬೇರಿಗೆ ನೀರು ಹಾಕಿದರೆ ಇಡೀ ಮರಕ್ಕೆ ನೀರು ಹಾಕಿದಂತೆ. ಪ್ರತಿಯೊಂದು ಎಲೆಗೂ ನೀರು ಹಾಕುವುದು ವೃಥಾ ಕಾಲಹರಣ, ಬೇರೆ ಮಾತಿನಲ್ಲಿ ಹೇಳುವುದಾದರೆ, ಮೊದಲು ದೇವರನ್ನು ಅರಸಿ; ಅವನು ದೊರೆತರೆ ಎಲ್ಲವೂ ದೊರೆತಂತೆ. ಚರ್ಚು, ಸಿದ್ದಾಂತ, ಆಚಾರ ಇವೆಲ್ಲಾ ಧರ್ಮವೆಂಬ ಸಸಿಯ ರಕ್ಷಣೆಗೆ ಸುತ್ತಲೂ ಹಾಕಿರುವ ಬೇಲಿ. ಆದರೆ ಆ ಸಸಿ ಮರವಾಗಬೇಕಾದರೆ ಅನಂತರ ಬೇಲಿಯನ್ನೆಲ್ಲಾ ತೆಗೆಯಬೇಕು. ಅದರಂತೆಯೇ ಧರ್ಮ. ಬೈಬಲ್, ವೇದಶಾಸ್ತ್ರ ಇವುಗಳೆಲ್ಲ ಸಸಿಗೆ ಇರುವ ಒಂದು ಕುಂಡದಂತೆ. ಆದರೆ ಅದು ಕುಂಡದಿಂದ ಪಾರಾಗಿ ಭೂಮಿಯ ಮೇಲೆ ಬೆಳೆಯಬೇಕಾಗಿದೆ.

ನಾವು ಇಲ್ಲಿರುವಂತೆಯೇ ಸೂರ್ಯ ನಕ್ಷತ್ರಗಳಲ್ಲಿಯೂ ಇರುವೆವು ಎಂದು ಭಾವಿಸಬೇಕು. ಆತ್ಮವು ಕಾಲದೇಶಾತೀತ. ನೋಡುವ ಪ್ರತಿಯೊಂದು ಕಣ್ಣೂ ನನ್ನದೇ, ಭಗವಂತನನ್ನು ಕೊಂಡಾಡುವ ಪ್ರತಿಯೊಂದು ಬಾಯಿಯೂ ನನ್ನದೇ. ಪ್ರತಿಯೊಬ್ಬ ಪಾಪಿಯೂ ನಾನೇ, ನಾವೆಲ್ಲೂ ಬಂಧಿತರಲ್ಲ, ನಾವು ದೇಹವಲ್ಲ, ವಿಶ್ವವೇ ನಮ್ಮ ದೇಹ, ಎಲ್ಲವನ್ನೂ ಪ್ರತಿಬಿಂಬಿಸುತ್ತಿರುವ ಸ್ಪಟಿಕ ನಾವು. ಆದರೆ ಸ್ಪಟಿಕ ಮಾತ್ರ ಯಾವ ಬದಲಾವಣೆಗೂ ಒಳಗಾಗುವುದಿಲ್ಲ. ನಮ್ಮ ಇಚ್ಛೆಯ ಪ್ರಕಾರ ಮಾಯಾದಂಡದಿಂದ ದೃಶ್ಯಗಳನ್ನು ನಿರ್ಮಿಸುತ್ತಿರುವ ಮಂತ್ರವಾದಿಗಳು ನಾವು. ಆದರೆ ನಾವು ಈ ತೋರಿಕೆಯ ಹಿಂದೆ ಹೋಗಿ ಆತನನ್ನು ಕಾಣಬೇಕಾಗಿದೆ. ಒಂದು ಪಾತ್ರೆಯಲ್ಲಿ ಕುದಿಯಲಿಕ್ಕೆ ಇಟ್ಟಿರುವ ನೀರಿನಂತೆ ಈ ಪ್ರಪಂಚ. ಮೊದಲು ಒಂದು ಗುಳ್ಳೆ ಏಳುವುದು, ಅನಂತರ ಹಲವು, ಕೊನೆಗೆ ನೀರೆಲ್ಲಾ ಕುದಿದು ಗುಳ್ಳೆಗಳಾಗಿ ಆವಿಯಂತೆ ಹೊರಟು ಹೋಗುವುವು. ಅದರಂತೆ ಕೊನೆಗೆ ಪ್ರತಿಯೊಂದು ಜೀವಿಯೂ ಒಂದು ಗುಳ್ಳೆಯಾಗಿ ಪಾರಾಗಿ ಹೋಗಬೇಕಾಗಿದೆ. ಚಿರನೂತನವಾದ ಸೃಷ್ಟಿಯು ಹೊಸ ನೀರನ್ನು ತರುವುದು. ಯಥಾಪ್ರಕಾರ ಹಿಂದಿನಂತೆ ಜಗತ್ತು ಪ್ರಾರಂಭವಾಗುವುದು. ಜಗತ್ತಿನ ಮಹಾಗುರುಗಳು ಪ್ರಾರಂಭದಲ್ಲಿ\break ಗುಳ್ಳೆಗಳು, ಇಲ್ಲೊಂದು ಅಲ್ಲೊಂದು. ಬುದ್ಧ ಮತ್ತು ಕ್ರಿಸ್ತ ಇವರು ಪ್ರಪಂಚದಲ್ಲೆಲ್ಲಾ ಕಾಣಿಸಿಕೊಂಡ ಶ್ರೇಷ್ಠತಮ ಗುಳ್ಳೆಗಳು. ಮಹಾ ವ್ಯಕ್ತಿಗಳವರು; ತಾವು ಮುಕ್ತರಾಗಿ ಇತರರಿಗೂ ಬಂಧನದಿಂದ ಪಾರಾಗುವುದಕ್ಕೆ ಸಹಾಯಮಾಡಿದವರು. ಇಬ್ಬರೂ ಎಲ್ಲಾ ದೃಷ್ಟಿಗಳಿಂದಲೂ ಪೂರ್ಣರಾಗಿರಲಿಲ್ಲ. ಆದರೆ ಅವರನ್ನು ಅವರ ಲೋಪದಿಂದ ಅಳೆಯುವುದಲ್ಲ, ಅವರ ಗುಣದಿಂದ ಅಳೆಯಬೇಕು. ಜೀಸಸ್‌ನಲ್ಲೂ ದೋಷಗಳಿದ್ದುವು. ಏಕೆಂದರೆ ತನ್ನ ಶ್ರೇಷ್ಠ ಆದರ್ಶಕ್ಕೆ ತಕ್ಕಂತೆ ಅವನು ಬಾಳಲಿಲ್ಲ. ಎಲ್ಲಕ್ಕಿಂತ ಹೆಚ್ಚಾಗಿ ಅವನು ಸ್ತ್ರೀಯರಿಗೆ ಪುರುಷರಿಗೆ ಸಮನಾದ ಸ್ಥಾನವನ್ನು ಕೊಡಲಿಲ್ಲ. ಸ್ತ್ರೀಯರೆ ಅವನಿಗೆ ಎಲ್ಲವನ್ನೂ ಮಾಡಿದರು. ಆದರೂ ಅವರಲ್ಲಿ ಯಾರೂ ಆಚಾರ್ಯರಾಗಲಿಲ್ಲ. ಇದಕ್ಕೆ ನಿಸ್ಸಂದೇಹವಾದ ಕಾರಣ ಅವನು ಸೆಮಿಟಿಕ್ ಗುಂಪಿಗೆ ಸೇರಿರುವುದೇ ಆಗಿದೆ. ಆದರೆ ಮಹಿಮರಾದ ಆರ್ಯರು ಮತ್ತು ಅವರಂತೆ ಬುದ್ದನೂ ಕೂಡ ಪುರುಷರಿಗೆ ಸರಿಸಮನಾದ ಸ್ಥಾನವನ್ನು ಸ್ತ್ರೀಯರಿಗೆ ಕೊಟ್ಟಿರುವರು. ಆರ್ಯರಿಗೆ ಧರ್ಮದಲ್ಲಿ ಲಿಂಗಭೇದವಿರಲಿಲ್ಲ. ವೇದ–ಉಪನಿಷತ್ತುಗಳಲ್ಲಿ ಸ್ತ್ರೀಯರು ಶ್ರೇಷ್ಠ ತತ್ತ್ವಗಳನ್ನು ಇತರರಿಗೆ ಬೋಧಿಸಿದ್ದರು, ಪುರುಷರಂತೆಯೇ ಗೌರವವನ್ನು ಪಡೆಯುತ್ತಿದ್ದರು.

\begin{center}
೮
\end{center}

ಸುಖದುಃಖಗಳೆರಡೂ ಸರಪಳಿಗಳೇ. ಒಂದು ಚಿನ್ನದ್ದು, ಮತ್ತೊಂದು ಕಬ್ಬಿಣದ್ದು. ಎರಡೂ ಬಲವಾದವುಗಳೇ. ಅವು ಆತ್ಮಸಾಕ್ಷಾತ್ಕಾರಕ್ಕೆ ಅಡ್ಡಲಾಗಿ ಬಂದು ನಮ್ಮನ್ನು ಬಂಧಿಸುವುವು. ಆತ್ಮನಿಗೆ ಸುಖದುಃಖಗಳಿಲ್ಲ. ಇವೆರಡೂ ಅವಸ್ಥೆಗಳು, ಅವಸ್ಥೆಯಾದುದರಿಂದ ಅನವರತ ಬದಲಾಗಬೇಕಾಗಿದೆ. ಆದರೆ ಆತ್ಮನ ಸ್ವಭಾವವಾದರೋ ಆನಂದ ಮತ್ತು ನಿರ್ವಿಕಾರವಾದ ಶಾಂತಿ. ನಾವು ಅದನ್ನು ಹೊಸದಾಗಿ ಪಡೆಯಬೇಕಾಗಿಲ್ಲ. ಅದು ಆಗಲೇ ನಮ್ಮಲ್ಲಿದೆ. ನಮ್ಮ ಚಿತ್ರದ ಮಲಿನತೆಯನ್ನು ತೊಳೆದರೆ ಅದು ಅಲ್ಲೇ ಕಾಣುವುದು. ನಾವು ಸದಾ ಆತ್ಮನಲ್ಲಿ ಪ್ರತಿಷ್ಠಿತರಾಗಿ ಸೃಷ್ಟಿವೈಚಿತ್ರ್ಯವನ್ನು ಪೂರ್ಣ ಶಾಂತಿಯಿಂದ ನೋಡಬೇಕು. ಇದೊಂದು ಮಕ್ಕಳಾಟ, ಇದೆಂದಿಗೂ ನಮ್ಮ ಮನಸ್ಸಿನಲ್ಲಿ ಕ್ಷೋಭೆಯನ್ನುಂಟುಮಾಡಲಾರದು. ಹೊಗಳಿಕೆಯಿಂದ ಮನಸ್ಸಿಗೆ ಸಂತೋಷವಾದರೆ ತೆಗಳಿಕೆಯಿಂದ ಮನಸ್ಸಿಗೆ ದುಃಖವಾಗುವುದು. ಇಂದ್ರಿಯಗಳ ಸುಖ ಮಾತ್ರವಲ್ಲ ಮನಸ್ಸಿನ ಸುಖ ಕೂಡ ಕ್ಷಣಿಕ. ಆದರೆ ನಿರಪೇಕ್ಷವಾದ ಆನಂದ ನಮ್ಮ ಹೃದಯದಲ್ಲಿಯೇ ಇದೆ. ಅದು ಯಾವ ಬಾಹ್ಯ ವಸ್ತುವನ್ನೂ ಅನುಸರಿಸಿಕೊಂಡಿಲ್ಲ. “ಆತ್ಮಾನಂದವನ್ನೇ ಪ್ರಪಂಚ ಧರ್ಮವೆಂದು ಕರೆಯುವುದು." ನಮ್ಮ ಆನಂದ ಆತ್ಮನಲ್ಲಿ ಹೆಚ್ಚು ನೆಲಸಿದಷ್ಟೂ ಹೆಚ್ಚು ಆಧ್ಯಾತ್ಮಿಕರು ನಾವು. ನಮ್ಮ ಸುಖಕ್ಕೆ ಜಗತ್ತನ್ನು ನಾವು ನೆಚ್ಚಬೇಕಾಗಿಲ್ಲ.

ಕೆಲವು ಬೆಸ್ತ ಹೆಂಗಸರು ದೊಡ್ಡದೊಂದು ಬಿರುಗಾಳಿಯಿಂದ ಕೂಡಿದ ಮಳೆ ಬಂದಾಗ ಒಬ್ಬ ಶ‍್ರೀಮಂತನ ಮನೆಯ ತೋಟಕ್ಕೆ ಹೋದರು. ಮನೆಯ ಯಜಮಾನ ಅವರನ್ನು ವಿಶ್ವಾಸದಿಂದ ಬರಮಾಡಿಕೊಂಡು ಅವರಿಗೆ ತಂಗಲು ಹೇಳಿದನು. ಸ್ವರ್ಗದಂತೆ ಇದ್ದ, ಪರಿಮಳದಿಂದ ಕೂಡಿದ ಸ್ಥಳದಲ್ಲಿ ಬೆಸ್ತ ಹೆಂಗಸರು ಮಲಗಿಕೊಂಡರು. ಆದರೆ ಅವರಿಗೆ ನಿದ್ರೆ ಬರಲಿಲ್ಲ. ತಮ್ಮ ಜೀವನದಲ್ಲಿ ನಿತ್ಯವೂ ಇದ್ದ ಏನೋ ಒಂದು ಬೇಕಾಗಿತ್ತು. ಅದಿಲ್ಲದೆ ಇದ್ದರೆ ಅವರಿಗೆ ನೆಮ್ಮದಿಯೇ ಇಲ್ಲ. ಕೊನೆಗೆ ಒಬ್ಬ ಹೆಂಗಸು. ಎದ್ದು ತನ್ನ ಮೀನಿನ ಬುಟ್ಟಿಯನ್ನು ಹೊರಗಿನಿಂದ ತಂದು ಹತ್ತಿರ ಇಟ್ಟುಕೊಂಡಳು. ನಿತ್ಯವೂ ಅನುಭವಿಸುತ್ತಿದ್ದ ಆ ವಾಸನೆಯಿಂದ ಸಂತುಷ್ಟರಾಗಿ ಅವರೆಲ್ಲ ಚೆನ್ನಾಗಿ ನಿದ್ದೆ ಮಾಡಿದರು.

ಈ ಪ್ರಪಂಚವೆನ್ನುವುದು ನಮ್ಮ ಸುಖಕ್ಕೆ ಆಶ್ರಯವಾದ ಒಂದು ಮೀನಿನ ಬುಟ್ಟಿಯಾಗದಿರಲಿ. ಇದು ತಾಮಸಿಕ ಸ್ವಭಾವ; ತ್ರಿಗುಣದಲ್ಲಿ ಅತ್ಯಂತ ಕನಿಷ್ಟವಾದ ಗುಣದಿಂದ ಬಂಧಿತವಾಗಿರುವ ಸ್ಥಿತಿ. ಅನಂತರವೇ ಅಹಂಕಾರದಿಂದ ಪ್ರೇರೇಪಿತರಾಗಿರುವವರು ಬರುವರು. ಅವರು ಯಾವಾಗಲೂ “ನಾನು'' “ನಾನು'' ಎನ್ನುತ್ತಿರುವರು. ಕೆಲವು ವೇಳೆ ಅವರು ಸತ್ಕರ್ಮವನ್ನು ಮಾಡಿ ಧಾರ್ಮಿಕರಾಗಬಹುದು. ಇವರೆಲ್ಲ ರಾಜಸಿಕ ಪ್ರವೃತ್ತಿಯವರು. ಶ್ರೇಷ್ಠರೆಂದರೆ ಅಂತರ್ಮುಖಿಗಳಾದ ಸಾತ್ವಿಕರು. ಅವರು ಯಾವಾಗಲೂ ಆತ್ಮಪ್ರತಿಷ್ಠಿತರು. ಈ ಮೂರು ಗುಣಗಳೂ ಬೇರೆಬೇರೆ ಪ್ರಮಾಣಗಳಲ್ಲಿ ಪ್ರತಿಯೊಬ್ಬರಲ್ಲಿಯೂ ಇವೆ. ಬೇರೆಬೇರೆ ಕಾಲಗಳಲ್ಲಿ ಬೇರೆಬೇರೆ ಗುಣಗಳು ಮೇಲೆದ್ದು ಕಾಣುವುವು. ನಾವು ರಜಸ್ಸಿನಿಂದ ತಮಸ್ಸನ್ನು ಗೆಲ್ಲಬೇಕು; ಅನಂತರ ಎರಡನ್ನೂ ಸತ್ಯದಲ್ಲಿ ಅಡಗಿಸಬೇಕು.

ಸೃಷ್ಟಿ ಎಂದರೆ ಹೊಸದಾಗಿ ಏನನ್ನೋ ತಯಾರುಮಾಡುವುದಲ್ಲ. ಕಳೆದುಕೊಂಡ ಸಮತ್ವವನ್ನು ಪುನಃ ಪಡೆಯುವುದಕ್ಕೆ ನಡೆಸುವ ಹೋರಾಟ ಅಷ್ಟೆ. ನಾವು ಕಾರ್ಕುಗಳನ್ನು ನೀರಿನ ಅಡಿಗೆ ಎಸೆದರೆ ಅವು ಒಂದೊಂದಾಗಿ ಮತ್ತು ಗುಂಪುಗುಂಪಾಗಿ ಮೇಲೆದ್ದು ಬರುವುವು. ಅವೆಲ್ಲ ಮೇಲಕ್ಕೆ ಬಂದ ಮೇಲೆ ಸಮತ್ವ ಸಿದ್ದಿಸುವುದು. ಆಗ ಚಲನೆ ನಿಲ್ಲುವುದು. ಇದರಂತೆಯೇ ಸೃಷ್ಟಿ ಕೂಡ. ಸಮತ್ವ ಬಂದರೆ ಬದಲಾವಣೆಯೆಲ್ಲ ನಿಲ್ಲುವುದು, ಜೀವನ ಕೂಡ ಕೊನೆಗಾಣುವುದು. ಜೀವನದ ಜೊತೆಯಲ್ಲಿ ಪಾಪ ಇದ್ದೇ ತೀರುವುದು. ಏಕೆಂದರೆ ಸಮತ್ವದಲ್ಲಿ ಸೃಷ್ಟಿ ಕೊನೆಗಾಣುವುದು. ಏಕೆಂದರೆ ಸಮತ್ವ ಮತ್ತು ಪ್ರಳಯ ಎರಡೂ ಒಂದೆ. ದುಃಖವಿಲ್ಲದೆ ಸುಖ, ಕೆಟ್ಟದ್ದಿಲ್ಲದೆ ಒಳ್ಳೆಯದು, ಇವನ್ನು ಎಂದಿಗೂ ಹೊಂದಲು ಸಾಧ್ಯವಿಲ್ಲ. ಏಕೆಂದರೆ ಬದುಕುವುದು ಎಂದರೇನೆ ಕಳೆದುಕೊಂಡ ಸಮತ್ವ. ನಮಗೆ ಬೇಕಾಗಿರುವುದು ಮುಕ್ತಿ; ಜೀವನವೂ ಅಲ್ಲ, ಸುಖವೂ ಅಲ್ಲ, ದುಃಖವೂ ಅಲ್ಲ. ಸೃಷ್ಟಿ ಅನಂತವಾದುದು, ಆದಿಅಂತ್ಯವಿಲ್ಲದುದು; ಅದು ಅನಂತ ಸರೋವರದಲ್ಲಿ ಚಿರಕಾಲ ಚಲಿಸುತ್ತಿರುವ ಅಲೆಯಂತೆ. ಇದರಲ್ಲಿ ಗೊತ್ತಾಗದ ಆಳವೆಷ್ಟೊ ಇರಬಹುದು. ಶಾಂತಿಯನ್ನು ಪಡೆದವರು ಎಷ್ಟೋ ಇರುವರು. ಆದರೆ ಅಲೆ ಮಾತ್ರ ಎಂದೆಂದಿಗೂ ಇರುವುದು. ಸಮತ್ವವನ್ನು ಪಡೆಯುವ ಹೋರಾಟವೂ ಅನಂತವಾದುದು. ಒಂದೇ ವಸ್ತುವಿಗೆ ಜನನಮರಣಗಳೆಂಬುವು ಬೇರೆಬೇರೆ ಹೆಸರುಗಳು. ಅವು ನಾಣ್ಯದ ಎರಡು ಕಡೆಗಳಂತೆ, ಎರಡೂ ಮಾಯೆಯೇ. ಒಂದು ಕಡೆ ಬದುಕಲು ಹೋರಾಟ, ಮತ್ತೊಂದು ಕಡೆ ಸಾಯಲು ಹೋರಾಟ. ಈ ಹೋರಾಟವನ್ನು ನಾವು ವಿವರಿಸಲಾರೆವು. ಆದರೆ ಇದರಾಚೆ ನಿಜಸ್ಥಿತಿ ಇರುವುದು, ಆತ್ಮವಿರುವುದು. ನಾವು ಸೃಷ್ಟಿಯನ್ನು ಪ್ರವೇಶಿಸುವೆವು. ಅನಂತರ ಅದು ನಮಗೆ ಜೀವಂತವಾಗುವುದು. ವಸ್ತುಗಳು ಸ್ವಭಾವತಃ ನಿರ್ಜಿವವಾಗಿವೆ. ನಾವೇ ಅವಕ್ಕೆ ಜೀವವನ್ನು ಕೊಡುವವರು. ಅನಂತರ ಮೂರ್ಖರಂತೆ ಅವನ್ನು ನೋಡಿ ಅಂಜುವೆವು. ಇಲ್ಲವೆ ಅವನ್ನು ನೋಡಿ ಆನಂದಿಸುವೆವು. ಪ್ರಪಂಚ ಸುಳ್ಳೂ ಅಲ್ಲ, ನಿಜವೂ ಅಲ್ಲ, ಸತ್ಯದ ಛಾಯೆ ಅದು.

`ಕಲ್ಪನೆಯೇ ಸತ್ಯದ ಹೊಂಬೆಳಕಿನ ಛಾಯೆ' ಎನ್ನುವನು ಕವಿ. ಒಳಗಿರುವ ಆಂತರಿಕ ಪ್ರಪಂಚ ಸತ್ಯವಾಗಿರುವುದು; ಅದು ಬಾಹ್ಯಕ್ಕಿಂತ ಬಹುಪಾಲು ದೊಡ್ಡದು. ಬಾಹ್ಯವೆಂಬುದು ಸತ್ಯದ ಛಾಯೆಯ ಆವಿರ್ಭಾವ ಮಾತ್ರ. ನಾವು ಹಗ್ಗವನ್ನು ನೋಡಿದಾಗ ಹಾವನ್ನು ನೋಡುವುದಿಲ್ಲ, ಹಾವನ್ನು ನೋಡಿದಾಗ ಹಗ್ಗವನ್ನು ನೋಡುವುದಿಲ್ಲ. ಎರಡೂ ಏಕಕಾಲದಲ್ಲಿ ಇರಲಾರವು. ಇದರಂತೆಯೇ ನಾವು ಸೃಷ್ಟಿಯನ್ನು ನೋಡಿದಾಗ ಆತ್ಮನನ್ನು ನೋಡಲಾರೆವು. ಅದು ಕೇವಲ ನಮ್ಮ ಯುಕ್ತಿಯ ಕಲ್ಪನೆ. ಬ್ರಹ್ಮಸಾಕ್ಷಾತ್ಕಾರದಲ್ಲಿ `ನಾನು' ಆಗಲಿ ಪ್ರಪಂಚದ ಭಾವನೆಯಾಗಲಿ ಯಾವುದೂ ಇರುವುದಿಲ್ಲ. ಬೆಳಕಿಗೆ ಕತ್ತಲೆ ಗೊತ್ತಾಗುವುದಿಲ್ಲ. ಏಕೆಂದರೆ ಅದು ಬೆಳಕಿನಲ್ಲಿರಲಾರದು. ಹಾಗೆಯೇ ಬ್ರಹ್ಮನೇ ಸರ್ವ. ನಾವು ಒಬ್ಬ ದೇವರನ್ನು ಒಪ್ಪಿಕೊಳ್ಳುವುದು ಅಂದರೆ, ನಮ್ಮಿಂದ ಬೇರೆಯಾದಂತೆ ಇರುವ ಆತ್ಮನನ್ನು ಬಾಹ್ಯದಲ್ಲಿ ದೇವರೆಂದು ಪೂಜಿಸುವುದು. ಆದರೆ ಅದು ಯಾವಾಗಲೂ ನಮ್ಮ ನಿಜವಾದ ಆತ್ಮವೆ, ಅದೇ ಏಕಮಾತ್ರ ದೇವರು. ಪ್ರಾಣಿಯ ಸ್ವಭಾವ ಇದ್ದಂತೆಯೇ ಇರುವುದು; ಮನುಷ್ಯನ ಸ್ವಭಾವ ಒಳ್ಳೆಯದನ್ನು ಆಶಿಸುವುದು ಮತ್ತು ಕೆಟ್ಟದ್ದನ್ನು ತ್ಯಜಿಸುವುದು; ದೇವನ ಸ್ವಭಾವ ಯಾವುದನ್ನೂ ಇಚ್ಚಿಸುವುದೂ ಇಲ್ಲ, ತ್ಯಜಿಸುವುದೂ ಇಲ್ಲ, ನಿರಂತರ ಆನಂದದಲ್ಲಿರುವುದು. ನಾವು ದೇವರಾಗೋಣ. ಹೃದಯವನ್ನು ಸಾಗರದಷ್ಟು ವಿಶಾಲ ಮಾಡೋಣ. ಪ್ರಪಂಚದ ಕೆಲಸಕ್ಕೆ ಬಾರದ ವಸ್ತುಗಳನ್ನೆಲ್ಲ ತ್ಯಜಿಸಿ ಅದನ್ನು ಒಂದು ಚಿತ್ರದಂತೆ ನೋಡೋಣ. ಅನಂತರ ಅದರಿಂದ ಯಾವ ವಿಧದಲ್ಲಿಯೂ ವ್ಯಥಿತರಾಗದೆ, ಅದನ್ನು ನೋಡಿ ಆನಂದಿಸಬಹುದು. ಪ್ರಪಂಚದಲ್ಲಿ ಒಳ್ಳೆಯದನ್ನೇಕೆ ಹುಡುಕುವಿರಿ? ನಮಗೆ ಅಲ್ಲಿ ಕಾಣುವುದೇನು? ಮಣ್ಣಿನಲ್ಲಿ ಆಟವಾಡುವ ಮಕ್ಕಳಿಗೆ ಸಿಕ್ಕಿದ ಕೆಲವು ಗಾಜಿನ ಚೂರುಗಳಂತಿದೆ ಪ್ರಪಂಚದಲ್ಲಿ ಸಿಕ್ಕುವ ಶ್ರೇಷ್ಠ ಆನಂದವೆಂಬುದು. ಮಕ್ಕಳು ಪುನಃ ಆ ಗಾಜಿನ ಚೂರುಗಳನ್ನು ಕಳೆದುಕೊಂಡು ಹುಡುಕಾಡುವುವು. ಅನಂತಶಕ್ತಿಯೇ ಧರ್ಮ ಮತ್ತು ದೇವರು. ನಾವು ಮುಕ್ತರಾದಾಗ ಮಾತ್ರ ಅಮರರು, ಮುಕ್ತರಾದಾಗ ಮಾತ್ರ ದೇವರು.

ಅಹಂಕಾರದಿಂದ ಜನಿತವಾದ ಪ್ರಪಂಚವನ್ನು ತ್ಯಜಿಸುವವರೆಗೆ ನಮಗೆ ಮುಕ್ತಿ ದೊರಕಲಾರದು. ಅದು ಯಾರಿಗೂ ಸಿಕ್ಕುವಂತೆ ಇಲ್ಲ. ಪ್ರಪಂಚವನ್ನು ತ್ಯಜಿಸುವುದೆಂದರೆ ಸಂಪೂರ್ಣ ಅಹಂಕಾರವನ್ನು ಮರೆಯುವುದು ಎಂದು ಅರ್ಥ. ಅದರ ಇರವನ್ನೇ ಮರೆಯುವುದು. ದೇಹದಲ್ಲಿ ಇರುವುದು, ಆದರೆ ಅದರ ದಾಸನಾಗುವುದಲ್ಲ. ಈ ದುಷ್ಟ “ಅಹಂ''ಅನ್ನು ನಾಶಗೊಳಿಸಬೇಕು. ಮೌನಿಗಳಾದವರು ಬಾಳಿ ಪ್ರೀತಿಸಿ ತಮ್ಮ ವ್ಯಕ್ತಿತ್ವವನ್ನೆಲ್ಲ ಹಿಂದಕ್ಕೆ ಸೆಳೆದುಕೊಳ್ಳುವರು. ಅವರೆಂದಿಗೂ ನಾನು ನನ್ನದು ಎಂದು ಹೇಳಿಕೊಳ್ಳುವುದಿಲ್ಲ. ಇತರರ ಸಹಾಯಕ್ಕೆ ತಾವೊಂದು ಯಂತ್ರವಾಗುವುದಾದರೆ ಧನ್ಯರು ಎಂದು ಭಾವಿಸುವರು. ಅವರು ದೇವರಲ್ಲಿ ತನ್ಮಯರಾಗಿರುವರು. ಏನನ್ನೂ ಕೇಳುವುದಿಲ್ಲ; ಮಾಡಬೇಕಲ್ಲ ಎಂದು ಏನನ್ನೂ ಮಾಡುವುದಿಲ್ಲ. ಅವರೇ ನಿಜವಾದ ಜೀವನ್ಮುಕ್ತರು, ಸ್ವಾರ್ಥ ಲವಲೇಶವಿಲ್ಲದವರು. ಅವರ ಅಹಂಕಾರವೆಲ್ಲ ಸಂಪೂರ್ಣ ತೊಡೆದುಹೋಗಿದೆ. ಅವರಲ್ಲಿ ಯಾವ ಆಸೆಯೂ ಇಲ್ಲ. ಅವರೆಲ್ಲ ವ್ಯಕ್ತಿತ್ವವಿಲ್ಲದ ಒಂದು ತತ್ತ್ವ. ನಮ್ಮ ಅಹಂಕಾರವನ್ನು ಎಷ್ಟು ಮರೆಮಾಡಿದರೆ ಅಷ್ಟು ಹೆಚ್ಚು ದೇವರು ಅಲ್ಲಿ ವ್ಯಕ್ತವಾಗುವನು. ನಾವು ಅಲ್ಪಾತ್ಮನಿಂದ ಪಾರಾಗೋಣ. ಭಗವಂತನೊಬ್ಬನೇ ನಮ್ಮಲ್ಲಿ ಇರಲಿ. ನಮ್ಮಲ್ಲಿ ಅಹಂಕಾರದ ಸುಳಿವಿಲ್ಲದಾಗ ಮಾತ್ರ ನಾವು ಶ್ರೇಷ್ಠ ಕೆಲಸವನ್ನು ಮಾಡಬಲ್ಲೆವು, ಅದ್ಭುತವಾದ ಪ್ರಭಾವವನ್ನು ಬೀರಬಲ್ಲೆವು. ಆಶಾಹೀನರು ಮಾತ್ರ ಫಲಗಳನ್ನು ಉಂಟುಮಾಡಬಲ್ಲರು. ಜನರು ನಿಮ್ಮನ್ನು ಹೀಯಾಳಿಸಿದಾಗ ಅವರನ್ನು ಆಶೀರ್ವದಿಸಿ. ನಿಮ್ಮಲ್ಲಿರುವ ಭ್ರಾಂತಿಯ ಅಹಂಕಾರವನ್ನು ನಾಶಗೊಳಿಸುವುದಕ್ಕೆ ಅವರು ಎಷ್ಟು ಉಪಕಾರ ಮಾಡುತ್ತಿರುವರು ಎಂಬುದನ್ನು ನೋಡಿ. ನಿಜವಾದ ಆತ್ಮನಲ್ಲಿ ಪ್ರತಿಷ್ಠಿತನಾಗಿ, ಪರಿಶುದ್ದ ಭಾವನೆಗಳನ್ನು ಮಾತ್ರ ಆಲೋಚಿಸಿ. ಆಗ ಬರಿಯ ಮಾತಾಳಿಗಳ ಒಂದು ಸೇನೆ ಸಾಧಿಸುವುದಕ್ಕಿಂತ ಹೆಚ್ಚು ಸಾಧಿಸುತ್ತೀರಿ. ಪರಿಶುದ್ಧ ಹೃದಯದಿಂದ, ಮೌನದ ಆಳದಿಂದ, ಪರಮಶಕ್ತಿಯುತವಾದ ಮಾತುಗಳು ಹೊರಹೊಮ್ಮುವುವು.

\begin{center}
೯
\end{center}

ವ್ಯಕ್ತಗೊಳಿಸುವುದು ಎಂದರೆ ನಿಜವಾಗಿಯೂ ಅಧೋಗತಿಗೆ ಇಳಿದಂತೆಯೇ.\break ಏಕೆಂದರೆ “ಅಧ್ಯಾತ್ಮ"ವನ್ನು ಅಕ್ಷರದ ಮೂಲಕ ಮಾತ್ರ ವಿವರಿಸಬಹುದು ಮತ್ತು ಸೇಂಟ್‌ಪಾಲ್ ಹೇಳುವಂತೆ “ಅಕ್ಷರವು ಭಾವವನ್ನು ಕೊಲ್ಲುವುದು.'' ಜೀವನ ಎಂಬುದು ಕೇವಲ ಛಾಯೆಯಾದ ಅಕ್ಷರದಲ್ಲಿ ಇರಲಾರದು. ಆದರೂ ತತ್ತ್ವವನ್ನು ತಿಳಿಯಬೇಕಾದರೆ ಅದಕ್ಕೆ ಒಂದು ಭೌತಿಕ ರೂಪವನ್ನು ಕೊಡಬೇಕು. ಆ ಆವರಣದಲ್ಲಿ ನಮಗೆ ಸತ್ಯ ಮರೆಯಾಗುವುದು. ಅದೊಂದು ಸಂಕೇತ ಎಂದು ಭಾವಿಸದೆ ಅದೇ ಸತ್ಯ ಎಂದು ಬಗೆಯುವೆವು. ಇದು ಸರ್ವಸಾಮಾನ್ಯವಾಗಿ ಆಗುವ ತಪ್ಪು. ಪ್ರತಿಯೊಬ್ಬ ಮಹಾ ಆಚಾರ್ಯನಿಗೂ ಇದು ಗೊತ್ತಿದೆ. ಈ ತಪ್ಪಿಗೆ ಬಲಿಬೀಳದಂತೆ ಅವರು ಜಾಗ್ರತರಾಗಿರುವರು. ಆದರೆ ಸಾಧಾರಣ ಮಾನವರು ಅವ್ಯಕ್ತಕ್ಕಿಂತ ವ್ಯಕ್ತವನ್ನೇ ಆರಾಧಿಸುವರು. ಆದಕಾರಣವೇ ಪ್ರವಾದಿಗಳ ಪರಂಪರೆಯೇ ಜಗತ್ತಿಗೆ ಬಂದಿರುವುದು. ಅವರು ವ್ಯಕ್ತಿತ್ವದ ಹಿಂದೆ ಇರುವ ತತ್ತ್ವವನ್ನು ಮತ್ತೆ ಮತ್ತೆ ತೋರಿಸುತ್ತಾರೆ ಮತ್ತು ಕಾಲಕ್ಕೆ ಅನುಗುಣವಾಗಿ ತತ್ವಕ್ಕೆ ಹೊಸ ಆವರಣವನ್ನು ಸೃಷ್ಟಿಸಿಕೊಡುವರು. ಸತ್ಯ ಯಾವಾಗಲೂ ಬದಲಾಗುವುದಿಲ್ಲ. ಆದರೆ ಅದಕ್ಕೆ ಒಂದು ಆಕಾರವನ್ನು ಕೊಡಬಹುದು. ಮಾನವ ಪ್ರಗತಿಯ ಹೊಸ ಹೊಸ ರೂಪಗಳನ್ನು ಸ್ವೀಕರಿಸಲು ಯೋಗ್ಯವಾಗುತ್ತ ಬಂದಂತೆಲ್ಲ, ಸತ್ಯಕ್ಕೆ ಹೊಸ ಹೊಸ ರೂಪಗಳನ್ನು ಕೊಡುವರು. ನಾವು ನಾಮರೂಪಗಳಿಂದ ಪಾರಾದಾಗ, ನಮಗೆ ಯಾವ ವಿಧವಾದ ಒಳ್ಳೆಯ ಅಥವಾ ಕೆಟ್ಟ, ಸೂಕ್ಷ್ಮವಾದ ಅಥವಾ ಸ್ಥೂಲವಾದ ಯಾವ ದೇಹವೂ ಬೇಕಿಲ್ಲದಾಗ ಮಾತ್ರ ನಾವು ಬಂಧನದಿಂದ ಪಾರಾಗುವೆವು. ನಿರಂತರ ಅಭಿವೃದ್ಧಿ ಎಂಬುದು ನಿರಂತರ ದಾಸ್ಯ. ನಾವು ಎಲ್ಲಾ ವಿಧವಾದ ವೈವಿಧ್ಯಗಳಿಂದ ಪಾರಾಗಿ ಸನಾತನವಾಗಿರುವ ಸಮತ್ವವನ್ನು, ಸಮರೂಪವನ್ನು ಅಥವಾ ಬ್ರಹ್ಮನನ್ನು ನೋಡಬೇಕು. ಆತ್ಮನೇ ವ್ಯಕ್ತಿಗಳ ಏಕತೆ. ಅದು ಅವಿನಾಶಿಯಾದುದು, ಏಕಮೇವ ಅದ್ವಿತೀಯವಾದುದು. ಅದು ಜೀವನವಲ್ಲ; ಅದು ಜೀವನದ ರೂಪವನ್ನು ಧರಿಸುತ್ತದೆ. ಅದು ಜನನಮರಣಗಳಾಚೆ, ಶುಭಾಶುಭಗಳಾಚೆ ಇದೆ. ಅದು ನಿರಪೇಕ್ಷವಾದ ಏಕತೆ, ನರಕದ ಮೂಲಕವಾಗಿಯಾದರೂ ಹೋಗಿ ಸತ್ಯವನ್ನು ಅರಸಲು ಧೈರ್ಯಮಾಡಿ. ಸಾಪೇಕ್ಷ ಪ್ರಪಂಚದ ನಾಮರೂಪಗಳಲ್ಲಿ ಸ್ವಾತಂತ್ರ್ಯವೆಂಬುದಿಲ್ಲ. ಯಾವ ಆಕಾರವೂ ಆಕಾರದ ರೂಪದಲ್ಲಿ ಮುಕ್ತವಾಗಲಾರದು. ಆಕಾರದ ಭಾವನೆಯೆಲ್ಲ ಸಂಪೂರ್ಣ ನಾಶವಾಗುವವರೆಗೆ ಸ್ವಾತಂತ್ರ್ಯ ಬರಲಾರದು. ನಮ್ಮ ಸ್ವಾತಂತ್ರ್ಯ ಇತರರನ್ನು ನೋಯಿಸಿದರೆ ಅದು ಸ್ವಾತಂತ್ರ್ಯವಾಗಲಾರದು. ನಾವು ಇತರರನ್ನು ನೋಯಿಸಬಾರದು. ನಿಜವಾದ ಜ್ಞಾನ ಒಂದೇ ಆದರೂ ಸಾಪೇಕ್ಷಜ್ಞಾನ ಹಲವು ಇರಬೇಕು. ಎಲ್ಲಾ ಜ್ಞಾನದ ಮೂಲ, ಇರುವೆಯಿಂದ ಹಿಡಿದು ದೇವಾದಿದೇವನವರೆಗೆ, ಆಗಲೇ ಎಲ್ಲರಲ್ಲಿಯೂ ಇದೆ. ನಿಜವಾದ ಧರ್ಮ ಒಂದೇ; ಭಿನ್ನತೆಯೆಲ್ಲ ಬಾಹ್ಯ ಆಕಾರ, ಚಿಹ್ನೆ, ಉದಾಹರಣೆಗಳಲ್ಲಿ ಮಾತ್ರ. ರಾಮರಾಜ್ಯ, ಯಾರು ಅದನ್ನು ಹುಡುಕುವರೋ, ಅವರಿಗೆ ಆಗಲೇ ಅಲ್ಲಿ ಇದೆ. ಸದ್ಯಕ್ಕೆ ನಾವು ಹಾಳಾಗಿರುವೆವು. ಅದಕ್ಕೆ ಪ್ರಪಂಚವೇ ಹಾಳಾಗಿರುವುದೆಂದು ಭಾವಿಸುವೆವು. “ಮೂರ್ಖ, ನಿನಗೆ ಕೇಳುವುದಿಲ್ಲವೇ? ನಿನ್ನ ಹೃದಯದೊಳಗೇ ಹಗಲೂ ರಾತ್ರಿ ಸಚ್ಚಿದಾನಂದ ಶಿವೋಹಂ ಶಿವೋಹಂ ಎಂಬ ಚಿರಂತನ ಗಾನವಾಗುತ್ತಿದೆ.''

ಯಾವ ಕಲ್ಪನೆಯ ಆಧಾರವೂ ಇಲ್ಲದೆ ಆಲೋಚಿಸಲು ಯತ್ನಿಸುವುದು ಅಸಾಧ್ಯವನ್ನು ಸಾಧ್ಯಮಾಡಲು ಯತ್ನಿಸುವಂತೆ. ಪ್ರತಿಯೊಂದು ಆಲೋಚನೆಗೂ ಎರಡು ಭಾಗಗಳಿವೆ: ಆಲೋಚನೆ ಮತ್ತು ಪದ. ನಮಗೆ ಎರಡೂ ಬೇಕು. ಭಾವ ಸತ್ಯವಾದಿಗಳಾಗಲಿ, ಪ್ರಪಂಚ–ಸತ್ಯವಾದಿಗಳಾಗಲಿ ಯಾರೂ ಪ್ರಪಂಚವನ್ನು ವಿವರಿಸಲಾರರು. ಅದನ್ನು ವಿವರಿಸಬೇಕಾದರೆ ಭಾವನೆ ಮತ್ತು ಅದರ ಅಭಿವ್ಯಕ್ತಿ ಎರಡೂ ಬೇಕಾಗಿವೆ. ಎಲ್ಲ ಜ್ಞಾನವೂ ಪ್ರತಿಬಿಂಬವನ್ನು ಮಾತ್ರ ಕುರಿತದ್ದು, ನಾವು ನಮ್ಮ ಮುಖವನ್ನು ಕನ್ನಡಿಯ ಪ್ರತಿಬಿಂಬದಲ್ಲಿ ಮಾತ್ರ ಕಾಣುವಂತೆ. ಯಾರೂ ತನ್ನಾತ್ಮನನ್ನಾಗಲಿ, ಬ್ರಹ್ಮನನ್ನಾಗಲಿ ತಿಳಿಯಲಾರರು. ಆದರೆ ಪ್ರತಿಯೊಬ್ಬರೂ ಆ ಆತ್ಮ. ಅದು ಜ್ಞಾನದ ಒಂದು ವಿಷಯವಾಗಬೇಕಾದರೆ ಅದರ ಪ್ರತಿಬಿಂಬವನ್ನು ಮಾತ್ರ ನೋಡಬೇಕಾಗುವುದು. ಅವ್ಯಕ್ತ ತತ್ವದ ಪ್ರತಿಬಿಂಬವನ್ನು ನೋಡುವುದೇ ವಿಗ್ರಹಾರಾಧನೆಗೆ ಕಾರಣ. ವಿಗ್ರಹಗಳ ವ್ಯಾಪ್ತಿಯು ನಾವು ಊಹಿಸಿರುವುದಕ್ಕಿಂತ ವಿಶಾಲವಾದುದು. ಅವು ಮರದಲ್ಲಿ ಮತ್ತು ಕಲ್ಲಿನಲ್ಲಿ ಮಾಡಿದ ವಿಗ್ರಹಗಳಿಂದ ಹಿಡಿದು ಪ್ರಖ್ಯಾತನಾಮರಾದ ಬುದ್ದ ಕ್ರಿಸ್ತರವರೆಗೆ ಇರುವುವು. ಬುದ್ಧನು ಸತತವೂ ಸಾಕಾರದೇವರನ್ನು ಅಲ್ಲಗಳೆಯುತ್ತಿದ್ದುದೇ ವಿಗ್ರಹಗಳು ಭರತಖಂಡದಲ್ಲಿ ಜಾರಿಗೆ ಬರುವುದಕ್ಕೆ ಕಾರಣ. ವೇದಗಳಲ್ಲಿ ಇವು ಇಲ್ಲ. ಒಬ್ಬ ಸೃಷ್ಟಿಕರ್ತನಾದ, ನಮ್ಮ ಗೆಳೆಯನಾದ ಈಶ್ವರನನ್ನು ತೆಗೆದುಹಾಕಿದ ಮೇಲೆ ಮಹಾಗುರುಗಳನ್ನೇ ವಿಗ್ರಹಗಳನ್ನಾಗಿ ಮಾಡಿದರು. ಬುದ್ದನೇ ಹಾಗೆ ಒಂದು ವಿಗ್ರಹವಾದ. ಲಕ್ಷಾಂತರ ಜನರು ಈಗ ಅವನನ್ನು ಪೂಜಿಸುತ್ತಿರುವರು. ಉಗ್ರ ಸುಧಾರಣಾ ಪ್ರಯತ್ನ ಯಾವಾಗಲೂ ನಿಜವಾಗಿ ಆಗುವ ಸುಧಾರಣೆಯನ್ನು ತಗ್ಗಿಸುವುದು. ಪ್ರತಿಯೊಬ್ಬನಲ್ಲಿಯೂ ಆರಾಧಿಸಬೇಕೆಂಬ ಭಾವನೆ ಸ್ವಭಾವಸಿದ್ಧವಾಗಿದೆ. ಶ್ರೇಷ್ಠ ತತ್ತ್ವಜ್ಞರು ಮಾತ್ರ ಶುದ್ಧ ನಿರಾಕಾರವನ್ನು ಯೋಚಿಸಬಲ್ಲರು. ಆದಕಾರಣ ಮಾನವನು ದೇವರನ್ನು ಪೂಜಿಸುವುದಕ್ಕಾಗಿ ಅವನಿಗೆ ಯಾವಾಗಲೂ ಒಂದು ಆಕಾರವನ್ನು ಕೊಡುವನು. ಇದೊಂದು ಪ್ರತೀಕವಾಗಿರುವವರೆಗೆ ಅದು ಒಳ್ಳೆಯದು. ಪ್ರತೀಕ ಏನಾದರೂ ಆಗಿರಲಿ, ಕೇವಲ ಪ್ರತೀಕವೆ ದೇವರೆಂದು ಭಾವಿಸದೆ, ಅದೊಂದು ಭಗವಂತನ ಚಿಹ್ನೆ ಎಂದು ಅರಿತು ಅದರ ಮೂಲಕ ಆರಾಧನೆ ಮಾಡಿದರೆ ಮೇಲು. ಎಲ್ಲಕ್ಕಿಂತ ಹೆಚ್ಚಾಗಿ “ಇದು ಶಾಸ್ತ್ರದಲ್ಲಿ ಹೇಳಿದೆ. ಆದುದರಿಂದ ಇದನ್ನು ನಂಬಬೇಕು'' ಎಂಬ ಮೂಢನಂಬಿಕೆಯಿಂದ ಪಾರಾಗಬೇಕು.

ಎಲ್ಲವನ್ನೂ ಎಂದರೆ, ವಿಜ್ಞಾನ, ಧರ್ಮ, ತತ್ತ್ವ – ಇವನ್ನೆಲ್ಲ ಯಾವುದೋ ಒಂದು ಶಾಸ್ತ್ರಕ್ಕೆ ಹೊಂದಿಕೊಂಡು ಹೋಗುವಂತೆ ಮಾಡಲು ಯತ್ನಿಸುವುದು ಒಂದು ಮಹಾ ದೌರ್ಜನ್ಯ. ಗ್ರಂಥ –ಆರಾಧನೆಯೆ ವಿಗ್ರಹಾರಾಧನೆಯಲ್ಲೆಲ್ಲ ಅತಿ ಭಯಂಕರವಾಗಿರುವುದು. ಹಿಂದೆ ಒಂದು ಜಿಂಕೆ ಇತ್ತು. ಕಾಡಿನಲ್ಲಿ ಅದು ಸ್ವತಂತ್ರವಾಗಿತ್ತು. ತನ್ನ ಸಮಾನ ಯಾರೂ ಇಲ್ಲ ಎಂದು ತಿಳಿದುಕೊಂಡಿತ್ತು. ಅದು ತನ್ನ ಮರಿಗೆ ಬಹಳ ಅಹಂಕಾರದಿಂದ “ನನ್ನನ್ನು ನೋಡು, ನನ್ನ ಕೊಂಬುಗಳು ಎಷ್ಟು ಬಲವಾಗಿವೆ. ಇದರಿಂದ ಒಂದು ಸಲ ತಿವಿದರೆ ಮನುಷ್ಯ ಸಾಯುವನು. ಜಿಂಕೆಯಾಗಿರುವುದು ಒಂದು ಅದೃಷ್ಟ" ಎಂದು ಜಂಭ ಕೊಚ್ಚಿಕೊಂಡಿತು. ಆಗತಾನೇ ಬೇಟೆಗಾರರ ಕಹಳೆ ದೂರದಲ್ಲಿ ಕೇಳಿತು. ಜಿಂಕೆ ತಕ್ಷಣ ಓಡಿತು. ಆಶ್ಚರ್ಯಪಡುತ್ತಿದ್ದ ಮರಿ ಕೂಡ ಇದರೊಡನೆ ಓಡಿಹೋಯಿತು. ಸುರಕ್ಷಿತವಾದ ಒಂದು ಸ್ಥಳವನ್ನು ಸೇರಿದ ಮೇಲೆ, “ಅಪ್ಪ, ನೀನಿಷ್ಟು ಬಲಶಾಲಿಯಾಗಿದ್ದರೆ ಮನುಷ್ಯರನ್ನು ಕಂಡರೆ ಏತಕ್ಕೆ ಓಡುವುದು?'' ಎಂದು ಮರಿ ಕೇಳಿತು. ಅದಕ್ಕೆ ಜಿಂಕೆ “ಮಗು, ನಾನು ತುಂಬಾ ಬಲಶಾಲಿ ಎನ್ನುವುದು ಗೊತ್ತಿದೆ. ಆದರೆ ಆ ಕಹಳೆ ಕೇಳಿದಾಗ ಯಾವುದೋ ಒಂದು ಅಂಜಿಕೆ ನನ್ನನ್ನು ಮೆಟ್ಟಿಕೊಂಡು, ನಾನು ಇಚ್ಚಿಸಲಿ, ಬಿಡಲಿ, ಓಡಿಹೋಗುವಂತೆ ಮಾಡುವುದು" ಎಂದಿತು. ಇದರಂತೆಯೇ ಶಾಸ್ತ್ರದ ನಿಯಮಗಳೆಂಬ ಕಹಳೆಯ ಧ್ವನಿಯನ್ನು ಕೇಳಿದೊಡನೆಯೇ ಹಳೆಯ ಮೂಢ ನಂಬಿಕೆ ಆಚಾರಗಳು ನಮ್ಮನ್ನು ಮೆಟ್ಟಿಕೊಳ್ಳುವುವು. ನಮಗೆ ಅದು ಗೊತ್ತಾಗುವುದಕ್ಕೆ ಮುಂಚೆಯೇ ಸ್ವಾತಂತ್ರ್ಯವೆಂಬ ನಮ್ಮ ನೈಜಸ್ವಭಾವವನ್ನು ಮರೆತು ಹಳೆಯ ಆಚಾರಕ್ಕೆ ಮತ್ತು ಮೂಢನಂಬಿಕೆ\-ಗಳಿಗೆ ಒಳಗಾಗುವೆವು.

ಜ್ಞಾನ ಸನಾತನವಾಗಿರುವುದು. ಒಬ್ಬ ಒಂದು ಆಧ್ಯಾತ್ಮಿಕ ನಿಯಮವನ್ನು ಕಂಡುಹಿಡಿದಾಗ ಅವನನ್ನು ಸ್ಫೂರ್ತಿಗೊಂಡವನು, ಎನ್ನುವೆವು. ಅವನು ಪ್ರಪಂಚಕ್ಕೆ ತರುವುದೆಲ್ಲ ಅಪೌರುಷೇಯವಾದುದು. ಆದರೆ ಈ ಅಪೌರುಷೇಯವಾಣಿಯನ್ನು, ಅದೇ ಕೊನೆ ಎಂದು ಅಂಧಶ್ರದ್ಧೆಯಿಂದ ಅನುಸರಿಸುವುದು ಸರಿಯಲ್ಲ. ದಿವ್ಯದರ್ಶನವು, ಯಾರು ಅದನ್ನು ಸ್ವೀಕರಿಸಲು ಯೋಗ್ಯರಾಗಿರುವರೋ ಅವರಿಗೆಲ್ಲ ಲಭಿಸುವುದು.\break ಪೂರ್ಣಪರಿಶುದ್ಧತೆಯೆ ಇದಕ್ಕೆ ಅತ್ಯಾವಶ್ಯಕವಾಗಿ ಬೇಕಾಗಿರುವುದು. ಪರಿಶುದ್ಧಾತ್ಮರು ಮಾತ್ರ ದೇವರನ್ನು ನೋಡಬಲ್ಲರು. ಇರುವುದರಲ್ಲೆಲ್ಲ ಶ್ರೇಷ್ಠ ಜೀವಿಯೇ ಮಾನವ, ಶ್ರೇಷ್ಠ ಜಗತ್ತೇ ಈ ಪೃಥ್ವಿ. ಏಕೆಂದರೆ ಮಾನವ ಇಲ್ಲಿ ಮುಕ್ತಿಯನ್ನು ಪಡೆಯಬಹುದು. ದೇವರ ವಿಷಯವಾಗಿ ನಮಗೆ ದೊರಕುವ ಶ್ರೇಷ್ಠ ಭಾವನೆಯೇ ಮಾನವ. ನಾವು ದೇವರಿಗೆ ಆರೋಪಿಸುವ ಪ್ರತಿಯೊಂದು ವಿಶೇಷಣವೂ ಮಾನವನಿಗೇ\break ಅನ್ವಯವಾಗುತ್ತದೆ; ಆದರೆ ಅಲ್ಪ ಪ್ರಮಾಣದಲ್ಲಿ ನಾವು ಉತ್ತಮರಾಗಿ ದೇವರ ಭಾವನೆಯನ್ನು ಮಾರಿಹೋಗಬೇಕಾದರೆ, ನಾವು ಈ ದೇಹ, ಮನಸ್ಸು, ಕಲ್ಪನೆ, ಇವನ್ನೆಲ್ಲ ಮೀರಿಹೋಗಬೇಕು. ಈ ಜಗತ್ತನ್ನು ದಾಟಿ ಹೋಗಬೇಕು. ನಾವು ನಿರಪೇಕ್ಷ ಸ್ಥಿತಿಗೆ ಹೋದರೆ ಆಗ ನಾವು ಈ ಪ್ರಪಂಚದಲ್ಲಿ ಇರುವುದಿಲ್ಲ. ಅಲ್ಲಿ ದೃಶ್ಯವೇ ಇಲ್ಲ. ಇರುವುದೆಲ್ಲ ದೃಕ್, ನಾವು ಅರಿತುಕೊಳ್ಳಬಹುದಾದ ಏಕೈಕ ಜಗತ್ತಿನ ಅತ್ಯುಚ್ಚ ಶಿಖರವೆಂದರೆ ಮಾನವನೇ. ಯಾರು ಸಮತ್ವವನ್ನು ಅಥವಾ ಪೂರ್ಣತೆಯನ್ನು ಪಡೆದಿರುವರೋ ಅವರು ದೇವರಲ್ಲಿ ಬಾಳುತ್ತಿರುವರು. ದ್ವೇಷವೆಲ್ಲ ತನ್ನಿಂದ ತನಗೇ ಹಾನಿಯನ್ನು ಉಂಟುಮಾಡಿಕೊಳ್ಳುವುದಾಗಿದೆ. ಆದಕಾರಣ ಪ್ರೇಮವೇ ಜೀವನದ ಏಕಮಾತ್ರ ನಿಯಮ. ನಾವು ಈ ಸ್ಥಿತಿಗೆ ಏರಿದರೆ ಪೂರ್ಣರಾಗುವೆವು. ಆದರೆ ನಾವು ಹೆಚ್ಚು ಪೂರ್ಣರಾದಷ್ಟೂ ಕೆಲಸಮಾಡುವುದು ಕಡಮೆ. ಈ ಪ್ರಪಂಚವೆಲ್ಲ ಮಕ್ಕಳಾಟವೆಂದು ಬಗೆದು ಜ್ಞಾನಿ ಅದನ್ನು ಗಣನೆಗೆ ತೆಗೆದುಕೊಳ್ಳುವುದಿಲ್ಲ. ಎರಡು ನಾಯಿಮರಿಗಳು ಜಗಳಕಾಯುತ್ತಾ ಒಂದನ್ನೊಂದು ಕಚ್ಚುತ್ತಿದ್ದರೆ ನಾವು ಅದಕ್ಕೆ ಅಷ್ಟೊಂದು ಗಮನವನ್ನು ಕೊಡುವುದಿಲ್ಲ. ನಾವು ಅದನ್ನು ಅಷ್ಟು ಗಂಭೀರವಾದ ವಿಷಯ ಎಂದು ಭಾವಿಸುವುದಿಲ್ಲ. ಪೂರ್ಣಜ್ಞಾನಿಗೆ ಪ್ರಪಂಚವೆಲ್ಲ ಮಾಯೆ ಎಂದು ಗೊತ್ತಾಗಿದೆ. ಜೀವನವನ್ನು ಸಂಸಾರ ಎನ್ನುವರು – ಹಲವು ವಿರುದ್ಧ ಶಕ್ತಿಗಳು ನಮ್ಮ ಮೇಲೆ ಪ್ರಭಾವ ಬೀರುತ್ತಿರುವುದರ ಪರಿಣಾಮ ಇದು. ಜಡವಾದವು ಸ್ವಾತಂತ್ರ್ಯವೆಂಬುದು ಒಂದು ಭ್ರಾಂತಿ ಎನ್ನುವುದು. ಭಾವಸತ್ತಾವಾದವು, ಬಂಧನವನ್ನು ಕುರಿತ ಮಾತುಗಳು ಕೇವಲ ಕನಸು ಎನ್ನುತ್ತದೆ. ವೇದಾಂತವು “ನಾವು ಏಕಕಾಲದಲ್ಲಿ ಮುಕ್ತರು ಮತ್ತು ಬದ್ದರು ಆಗಿರುವೆವು'' ಎನ್ನುವುದು. ಆದರೆ ನಾವು ಭೌತಿಕ ಪ್ರಪಂಚದಲ್ಲಿ ಎಂದಿಗೂ ಸ್ವತಂತ್ರರಲ್ಲ; ಆದರೆ ಆಧ್ಯಾತ್ಮಿಕ ಪ್ರಪಂಚದಲ್ಲಿ ಸದಾ ಮುಕ್ತರು. ಬಂಧನಮುಕ್ತಿಗಳಾಚೆ ಆತ್ಮನಿರುವುದು. ನಾವೇ ಬ್ರಹ್ಮ. ಅಮೃತಸ್ವರೂಪವಾದ ಜ್ಞಾನ ನಾವೇ. ನಾವು ಇಂದ್ರಿಯಗಳಿಗೆ ಅತೀತರಾಗಿರುವೆವು. ನಾವು ಆನಂದಮಯರು.

