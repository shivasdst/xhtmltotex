
\chapter[ಯುಕ್ತಿ ಮತ್ತು ಧರ್ಮ]{ಯುಕ್ತಿ ಮತ್ತು ಧರ್ಮ\protect\footnote{\engfoot{C.W, Vol.I, P. 366}}}

\begin{center}
(ಇಂಗ್ಲೆಂಡಿನಲ್ಲಿ ನೀಡಿದ ಉಪನ್ಯಾಸ)
\end{center}

ನಾರದ ಋಷಿಗಳು ಪರಮ ಸತ್ಯವನ್ನು ತಿಳಿದುಕೊಳ್ಳಬೇಕೆಂದು ಸನತ್ಕುಮಾರ ಋಷಿಗಳ ಬಳಿಗೆ ಹೋದರು. ಸನತ್ಕುಮಾರರು ನಾರದರನ್ನು ಇದುವರೆಗೆ ನೀವು ಏನೇನು ಕಲಿತಿರುವಿರಿ ಎಂದು ಪ್ರಶ್ನಿಸಿದರು. ಅದಕ್ಕೆ ನಾರದರು, ತಾವು ಇದುವರೆಗೆ ವೇದ, ಜ್ಯೋತಿಷ್ಯ ಇವನ್ನು ಮತ್ತು ಇನ್ನೂ ಹಲವು ವೇದಾಂಗಗಳನ್ನು ಓದಿದ್ದರೂ ಮನಸ್ಸಿಗೆ ಶಾಂತಿಯಿಲ್ಲ ಎಂದರು. ಅನಂತರ ಅವರಿಬ್ಬರಿಗೂ ಸಂಭಾಷಣೆ ನಡೆಯಿತು. ಆಗ ಸನತ್ಕುಮಾರರು ವೇದ, ಜ್ಯೋತಿಷ್ಯ ಮತ್ತು ಇತರ ತತ್ತ್ವಶಾಸ್ತ್ರಗಳೆಲ್ಲ ಗೌಣ, ವಿಜ್ಞಾನಶಾಸ್ತ್ರಗಳೆಲ್ಲವೂ ಗೌಣ, ಯಾವುದು ಬ್ರಹ್ಮಸಾಕ್ಷಾತ್ಕಾರಕ್ಕೆ ನಮಗೆ ಸಹಾಯಮಾಡುವುದೊ ಅದು ಮಾತ್ರ ಶ್ರೇಷ್ಠ ಮತ್ತು ಪರಜ್ಞಾನ ಎಂದರು. ನಮಗೆ ಈ ಭಾವನೆ ಎಲ್ಲಾ ಧರ್ಮಗಳಲ್ಲಿಯೂ ದೊರಕುವುದು. ಆದಕಾರಣವೆ ಧರ್ಮವನ್ನು ಪರಜ್ಞಾನವೆಂದು ಯಾವಾಗಲೂ ಕರೆಯುತ್ತಿದ್ದರು. ವಿಜ್ಞಾನ ಶಾಸ್ತ್ರದ ಮೂಲಕ ಬರುವ ಜ್ಞಾನ ನಮ್ಮ ಜೀವನದ ಯಾವುದೋ ಅಲ್ಪ ಭಾಗಕ್ಕೆ ಅನ್ವಯಿಸುವುದು. ಆದರೆ ಧರ್ಮ ನಮಗೆ ನೀಡುವ ಜ್ಞಾನ ಶಾಶ್ವತವಾದುದು, ಅದು ಬೋಧಿಸುವ ಸತ್ಯದಷ್ಟೇ ಅನಂತವಾದದು. ತಮ್ಮದೇ ಶ್ರೇಷ್ಠ ಜ್ಞಾವೆಂದು ಭಾವಿಸಿ ಧರ್ಮಗಳು ಅನೇಕವೇಳೆ ಲೌಕಿಕ ವಿದ್ಯೆಯನ್ನು ನಿಕೃಷ್ಟ ದೃಷ್ಟಿಯಿಂದ ನೋಡಿವೆ. ಇದು ಮಾತ್ರವಲ್ಲ, ಲೌಕಿಕ ಜ್ಞಾನದಿಂದ ಪ್ರಯೋಜನ ಪಡೆಯಲು ನಿರಾಕರಿಸಿವೆ. ಇದರ ಪರಿಣಾಮವಾಗಿ ಪ್ರಪಂಚದಲ್ಲೆಲ್ಲ ಲೌಕಿಕ ವಿದ್ಯೆಗೂ ಪಾರಮಾರ್ಥಿಕ ವಿದ್ಯೆಗೂ ಘರ್ಷಣೆಗಳಾಗಿವೆ. ಧರ್ಮವು, ನಿರ್ದೋಷವಾದ ಪ್ರಮಾಣವೇ ತನಗೆ ಮಾರ್ಗದರ್ಶಕ ಎಂದು, ಲೌಕಿಕ ವಿದ್ಯೆ ತನ್ನ ವಿಷಯದಲ್ಲಿ ಹೇಳುವುದನ್ನೆಲ್ಲಾ ನಿರಾಕರಿಸಿದೆ. ರಾರಾಜಿಸುತ್ತಿರುವ ಯುಕ್ತಿಯೆಂಬ ಮಹಾಶಸ್ತ್ರದಿಂದ ಸನ್ನದ್ಧವಾಗಿರುವ ವಿಜ್ಞಾನವಾದರೊ, ಧರ್ಮ ಹೇಳುವುದನ್ನೆಲ್ಲ ಖಂಡಿಸುವುದರಲ್ಲಿ ನಿರತವಾಗಿದೆ. ಪ್ರತಿಯೊಂದು ದೇಶದಲ್ಲಿಯೂ ಈ ಹೋರಾಟ ನಡೆದಿದೆ ಮತ್ತು ಈಗಲೂ ನಡೆಯುತ್ತಿದೆ. ಧರ್ಮ ಪದೇ ಪದೇ ಸೋಲಿಗೆ ಗುರಿಯಾಗಿ ನಿರ್ನಾಮವಾಗುವ ಸ್ಥಿತಿಗೆ ಬಂತು. ಫ್ರೆಂಚ್ ಕ್ರಾಂತಿಯ ಸಮಯದಲ್ಲಿ ನಡೆದ ಯುಕ್ತಿದೇವತೆಯ ಉಪಾಸನೆ ಮಾನವ ಇತಿಹಾಸದಲ್ಲಿಯೇ ಪ್ರಥಮಬಾರಿಯಲ್ಲ. ಅದು ಪುರಾತನ ಕಾಲದಲ್ಲಿ ಆದ ಘಟನೆಗಳ ಪುನರಾವೃತ್ತಿ ಅಷ್ಟೆ. ಆದರೆ ಆಧುನಿಕ ಕಾಲದಲ್ಲಿ ಇದು ಬೃಹದಾಕಾರವನ್ನು ತಾಳಿತು. ವಿಜ್ಞಾನಶಾಸ್ತ್ರ, ಹಿಂದಿಗಿಂತ ಇಂದು ಹೆಚ್ಚು ಸಜ್ಜಿತವಾಗಿದೆ, ಧರ್ಮವಾದರೊ ದುರ್ಬಲವಾಗುತ್ತಿದೆ. ಧರ್ಮದ ತಳಪಾಯವೆಲ್ಲ ಕುಸಿದು ಹೋಗುತ್ತಿದೆ. ಈಗಿನ ಜನರು ಹೊರಗಡೆ ಏನು ಹೇಳಿದರೂ ಅಂತರಾಳದಲ್ಲಿ ಅವರ ನಂಬಿಕೆಯೆಲ್ಲ ಜಾರಿಹೋಗಿದೆ. ಕೆಲವು ಸಂಘಬದ್ಧರಾದ ಪಾದ್ರಿಗಳು ಏನು ಹೇಳುತ್ತಾರೆಂದು ನಂಬುವುದು, ಕೆಲವು ಶಾಸ್ತ್ರಗಳಲ್ಲಿ ಹೇಳಿದೆ ಎಂದು ನಂಬುವುದು, ತನ್ನ ಜನರು ಮೆಚ್ಚುತ್ತಾರೆಂದು ನಂಬುವುದು ಈಗಿನ ಜನರಿಗೆ ಸಾಧ್ಯವೇ ಇಲ್ಲ. ಸಾಮಾನ್ಯ ನಂಬಿಕೆಗೆ ಒಪ್ಪಿಗೆ ಕೊಡುವ ಹಲವು ಜನರೇನೊ ಇರುವರು. ಆದರೆ ಅವರು ವಿಮರ್ಶೆ ಮಾಡುವುದೇ ಇಲ್ಲವೆಂದು ನಮಗೆ ಚೆನ್ನಾಗಿ ಗೊತ್ತಿದೆ. ಅವರ ನಂಬಿಕೆಯನ್ನು `ವಿವೇಚನರಹಿತ ಅಸಡ್ಡೆ' ಎಂದು ಹೇಳಬೇಕಾಗುವುದು. ಈ ಹೋರಾಟ ಮುಂದೆ ಸಾಗಿದರೆ ಧರ್ಮಸೌಧವೆಲ್ಲ ಚೂರು ಚೂರಾಗಿ ಕುಸಿದು ಬೀಳುವುದು.

ಈಗಿನ ಸಮಸ್ಯೆ ಏನೆಂದರೆ ಇದರಿಂದ ಪಾರಾಗುವುದಕ್ಕೆ ಮಾರ್ಗವಿದೆಯೇ\break ಎಂಬುದು. ಇದನ್ನು ಮತ್ತೂ ಸ್ಪಷ್ಟವಾಗಿ ಹೇಳಬೇಕಾದರೆ, ಇತರ ವಿಜ್ಞಾನಶಾಸ್ತ್ರಗಳು ಯುಕ್ತಿಯ ನವಾನ್ವೇಷಣೆಯ ಮೂಲಕ ಸಮರ್ಥಿಸಲ್ಪಡುವಂತೆ ಧರ್ಮವೂ ಅದರ ಮೂಲಕ ಸಮರ್ಥಿಸಲ್ಪಡಬೇಕೆ? ವಿಜ್ಞಾನದಲ್ಲಿ ಮತ್ತು ಇತರ ಬಾಹ್ಯ ಶಾಸ್ತ್ರದಲ್ಲಿ ಉಪಯೋಗಿಸುವ ಪರಿಶೋಧನಾ ಮಾರ್ಗವನ್ನೇ ಧರ್ಮದಲ್ಲಿಯೂ ಉಪಯೋಗಿಸಬೇಕೆ? ನನ್ನ ದೃಷ್ಟಿಯಲ್ಲಿ ಇದು ಅಗತ್ಯ ಎನ್ನುತ್ತೇನೆ ಮತ್ತು ಇದನ್ನು ಎಷ್ಟು ಬೇಗ ಮಾಡಿದರೆ ಅಷ್ಟು ಒಳ್ಳೆಯದು. ಇಂತಹ ಪರಿಶೋಧನೆಯಿಂದ ಧರ್ಮ ನಾಶವಾಗುವಂತೆ ಇದ್ದರೆ, ಅದು ಇದುವರೆಗೂ ನಿಷ್ಪ್ರಯೋಜಕವಾಗಿತ್ತು; ಕೆಲಸಕ್ಕೆ ಬಾರದ ಒಂದು ಮೂಢನಂಬಿಕೆಯಾಗಿತ್ತು. ಇಂತಹ ಧರ್ಮ ಎಷ್ಟು ಬೇಗ ನಾಶವಾದರೆ ಅಷ್ಟು ಮೇಲು. ಇಂತಹ ಧರ್ಮ ನಾಶವಾಗುವುದೇ ಅತ್ಯುತ್ತಮವೆಂದು ನಾನು ನಿಸ್ಸಂದೇಹವಾಗಿ ಭಾವಿಸುತ್ತೇನೆ. ಇಂತಹ ಒಂದು ಶೋಧನೆಯಿಂದ ಅಯೋಗ್ಯವಾದುದೆಲ್ಲ ನಿಸ್ಸಂದೇಹವಾಗಿ ನಾಶವಾಗುವುದು ನಿಜ. ಆದರೆ ನಿಜವಾದ ಧರ್ಮದ ಸಾರ ಇಂತಹ ಅಗ್ನಿಪರೀಕ್ಷೆಯಲ್ಲಿ ಉತ್ತೀರ್ಣವಾಗಿ ಅಪ್ರತಿಹತವಾಗಿ ನಿಲ್ಲುವುದು. ಧರ್ಮವು ವೈಜ್ಞಾನಿಕವಾಗುವುದು ಮಾತ್ರವಲ್ಲ, ಭೌತಶಾಸ್ತ್ರ, ರಸಾಯನಶಾಸ್ತ್ರಗಳ ನಿರ್ಣಯಗಳಷ್ಟು ವೈಜ್ಞಾನಿಕವಾಗುವುದು; ಆದರೆ ಅವಕ್ಕಿಂತ ಹೆಚ್ಚು ಶಕ್ತಿಯುತವಾಗಿರುವುದು. ಏಕೆಂದರೆ ರಸಾಯನಶಾಸ್ತ್ರಭೌತಶಾಸ್ತ್ರಗಳಿಗೆ ಧರ್ಮಕ್ಕೆ ಇರುವಂತೆ, ಇದು ಸತ್ಯ ಎಂದು ಸಾರುವುದಕ್ಕೆ ಯಾವ ವಿಧವಾದ ಆಂತರಿಕ ಶಾಸನವೂ ಇಲ್ಲ.

ಧರ್ಮವನ್ನು ಕುರಿತಂತೆ ವೈಚಾರಿಕ ಶೋಧನೆಯ ಸಫಲತೆಯನ್ನು ಒಪ್ಪದವರು ನನ್ನ ದೃಷ್ಟಿಯಲ್ಲಿ ಸ್ವ-ವಿರೋಧಿಗಳಂತೆ ಕಾಣುತ್ತಾರೆ. ಉದಾಹರಣೆಗೆ ಕ್ರೈಸ್ತರು ತಮ್ಮದೊಂದೇ ನಿಜವಾದ ಧರ್ಮ, ಏಕೆಂದರೆ ಇಂಥವನಿಗೆ ಅದು ಪ್ರತ್ಯಕ್ಷವಾಯಿತು ಎನ್ನುವರು. ಮಹಮ್ಮದೀಯರೂ ಕೂಡ ತಮ್ಮ ಧರ್ಮಕ್ಕೆ ಈ ಕಾರಣವನ್ನೇ ಕೊಡುವರು. ಆದರೆ ಕ್ರೈಸ್ತರು ಮಹಮ್ಮದೀಯರಿಗೆ ಹೇಳುವರು: “ನಿಮ್ಮ ಕೆಲವು ನೀತಿಗಳು ಸರಿಯಾಗಿ ಕಾಣುವುದಿಲ್ಲ. ಅನ್ಯಧರ್ಮದವರನ್ನು ಬಲಾತ್ಕಾರವಾಗಿ ಬೇಕಾದರೆ ಮಹಮ್ಮದೀಯರನ್ನಾಗಿ ಮಾಡಬಹುದು, ಅವನು ಹಾಗೆ ಮಹಮ್ಮದೀಯನಾಗಲು ಒಪ್ಪದೆ ಇದ್ದರೆ ಅವನನ್ನು ಕೊಲ್ಲಬಹುದು. ಯಾವ ಮಹಮದೀಯ ಅವನನ್ನು ಕೊಲ್ಲುವನೋ ಅವನು ಎಂತಹ ತಪ್ಪನ್ನು ಅಥವಾ ಪಾಪವನ್ನು ಮಾಡಿದ್ದರೂ ಚಿಂತೆಯಿಲ್ಲ, ಅವನು ಸ್ವರ್ಗಕ್ಕೆ ಹೋಗುವನು ಎಂದು ನಿಮ್ಮ ಧರ್ಮ ಸಾರುವುದು." ಅದಕ್ಕೆ ಮಹಮ್ಮದೀಯನು “ಹಾಗೆ ಮಾಡುವುದು ನನ್ನ ಕರ್ತವ್ಯ, ಏಕೆಂದರೆ ನನ್ನ ಗ್ರಂಥ ಹಾಗೆ ಸಾರುವುದು. ನಾನು ಹಾಗೆ ಮಾಡದೆ ಇರುವುದೇ ತಪ್ಪು" ಎನ್ನುವನು. ಕ್ರಿಸ್ತನು “ಆದರೆ ನನ್ನ ಗ್ರಂಥ ಹಾಗೆ ಹೇಳುವುದಿಲ್ಲವಲ್ಲ" ಎನ್ನುವನು. ಆದರೆ ಮಹಮ್ಮದೀಯ “ನನಗೆ ಇದು ಗೊತ್ತಿಲ್ಲ, ನಿನ್ನ ಗ್ರಂಥದ ಪ್ರಮಾಣ ನನಗೆ ಆಧಾರವಾಗಲಾರದು, ನನ್ನ ಶಾಸ್ತ್ರ ನಮ್ಮ ಮತಕ್ಕೆ ಸೇರದವರನ್ನೆಲ್ಲ ಕೊಲ್ಲು ಎಂದು ವಿಧಿಸುವುದು'' ಎನ್ನುವನು. ಈಗ ಯಾವುದು ಸರಿ, ಯಾವುದು ತಪ್ಪು ಎಂದು ಹೇಗೆ ಕಂಡುಹಿಡಿಯುವಿರಿ? “ನಿಜವಾಗಿ ನನ್ನ ಗ್ರಂಥದಲ್ಲಿ ಏನು ಹೇಳಿದೆಯೋ ಅದೇ ಸರಿ, ನಿನ್ನ ಗ್ರಂಥದಲ್ಲಿ ಕೊಲ್ಲಬೇಡ ಎಂದು ಹೇಳಿರುವುದು ತಪ್ಪು. ನನ್ನ ಕ್ರೈಸ್ತ ಸಹೋದರನೇ, ನೀನು ಕೂಡ ಅದನ್ನೇ ಹೇಳುವೆ; ಯಾವುದನ್ನು ಯಹೋವ ಯಹೂದ್ಯರಿಗೆ ಸರಿ ಎಂದನೋ ಅದೇ ನಿಮಗೆ ಸರಿ, ಯಾವುದನ್ನು ನಿಷೇಧಿಸಿದನೋ ಅದು ನಿಮಗೆ ತಪ್ಪು. ಅದರಂತೆಯೇ ನನ್ನ ಶಾಸ್ತ್ರದಲ್ಲಿ ಕೆಲವನ್ನು ಮಾಡಿ ಎಂದೂ, ಕೆಲವನ್ನು ಮಾಡಬೇಡಿ ಎಂದೂ ಅಲ್ಲಾಹನು ಸಾರಿರುವನು. ಇದೇ ಸರಿ ಮತ್ತು ತಪ್ಪು ಎನ್ನುವುದಕ್ಕೆ ಪ್ರಮಾಣ.” ಎಷ್ಟು ಹೇಳಿದರೂ ಕ್ರೈಸ್ತನಿಗೂ ತೃಪ್ತಿಯಾಗುವುದಿಲ್ಲ. ಅವನು ಗುಡ್ಡದ ಮೇಲೆ ಮಾಡಿದ ಬೋಧನೆಯಲ್ಲಿರುವ ನೀತಿಯನ್ನೂ ಕೊರಾನಿನಲ್ಲಿ ಬರುವ ನೀತಿಯನ್ನೂ ಹೋಲಿಸಿ ನೋಡುವನು. ಇದನ್ನು ಹೇಗೆ ನಿರ್ಧರಿಸುವುದು? ಇದನ್ನು ಶಾಸ್ತ್ರಗಳ ಸಹಾಯದಿಂದ ನಿರ್ಧರಿಸಲು ಆಗುವುದಿಲ್ಲ. ಶಾಸ್ತ್ರಗಳು ತಮ್ಮಲ್ಲೇ ಹೋರಾಡುತ್ತಿರುವಾಗ ತಾವೇ ನ್ಯಾಯಾಧಿಪತಿಗಳಾಗಲಾರವು. ಈ ಶಾಸ್ತ್ರಗಳಿಗಿಂತ ಮತ್ಯಾವುದೋ ಸರ್ವವ್ಯಾಪಿಯಾಗಿರಬೇಕು. ಈ ಪ್ರಪಂಚದಲ್ಲಿ ಜಾರಿಯಲ್ಲಿರುವ ನೀತಿಸಂಹಿತೆಗಳಿಗಿಂತ ಮತ್ಯಾವುದೋ ಮೇಲಿರಬೇಕು; ಭಿನ್ನ ಭಿನ್ನ ಜನಾಂಗಗಳ ಸ್ಪೂರ್ತಿ-ಶಕ್ತಿಗಳನ್ನು ನಿಷ್ಕರ್ಷಿಸುವ ಮತ್ಯಾವುದೋ ಇದೆ, ಎಂದು ನಾವು ನಿರ್ವಿವಾದವಾಗಿ ಒಪ್ಪಲೇಬೇಕಾಗಿದೆ. ನಾವು ಅದನ್ನು ಧೈರ್ಯವಾಗಿ ಹೇಳಲಿ ಅಥವಾ ಬಿಡಲಿ, ಇಲ್ಲಿ ನಾವು ಯುಕ್ತಿಗೆ ಮೊರೆಯಿಡುತ್ತೇವೆ ಎಂಬುದು ನಿರ್ವಿವಾದವಾಯಿತು.

ಈಗ ಮತ್ತೊಂದು ಪ್ರಶ್ನೆ ಏಳುವುದು: ಯಾರಿಗೆ ದೊರೆತಿರುವ ಸ್ಫೂರ್ತಿ ಸರಿ ಎಂಬುದನ್ನು ಈ ಯುಕ್ತಿಯು ವಿಮರ್ಶಿಸಬಲ್ಲದೆ? ಒಬ್ಬ ದೇವದೂತನಿಗೂ ಮತ್ತೊಬ್ಬ ದೇವದೂತನಿಗೂ ನಡುವೆ ಭಿನ್ನಾಭಿಪ್ರಾಯಗಳು ಬಂದಾಗ ಈ ಯುಕ್ತಿ ಯಾವುದು ಸರಿ ಎಂಬುದನ್ನು ನಿಷ್ಕರ್ಷಿಸಬಲ್ಲದೆ? ಧರ್ಮಕ್ಕೆ ಸಂಬಂಧಪಟ್ಟ ವಿಷಯಗಳನ್ನು ಯುಕ್ತಿ ಅರ್ಥ ಮಾಡಿಕೊಳ್ಳಬಲ್ಲದೆ? ಇದಕ್ಕೆ ಸಾಧ್ಯವಿಲ್ಲದೆ ಇದ್ದರೆ ಭಿನ್ನ ಭಿನ್ನ ಶಾಸ್ತ್ರಗಳಿಗೂ ಮತ್ತು ಭಿನ್ನ ಭಿನ್ನ ಮಹಾತ್ಮರಿಗೂ ಆಗುತ್ತಿರುವ ಬಹಳ ಕಾಲದ ಹೋರಾಟವನ್ನು ಮತ್ತಾವುದೂ ನಿಷ್ಕರ್ಷಿಸುವಂತಿಲ್ಲ. ಧರ್ಮಗಳೆಲ್ಲ ಬರಿಯ ಸುಳ್ಳುಗಳ ಕಂತೆ, ಅವುಗಳಲ್ಲಿ ಯಾವ ಸರಿಯಾದ ನೀತಿಯೂ ಇಲ್ಲದೆ, ಒಂದು ನಿರ್ಣಯಕ್ಕೆ ಬಾರದ ವಿರೋಧಾಭಾಸಗಳ ಕಂತೆಯಾಗುವುದು. ಮನುಷ್ಯನ ಸ್ವಭಾವದ ಸಹಜತೆಯ ಮೇಲೆ ಧರ್ಮದ ಪ್ರಮಾಣ ನಿಂತಿರುವುದು. ಅದು ಯಾವ ಶಾಸ್ತ್ರದ ಆಧಾರದ ಮೇಲೂ ನಿಂತಿಲ್ಲ. ಈ ಗ್ರಂಥಗಳು ಮಾನವನ ಸ್ವಭಾವದಿಂದ ನಿರ್ಮಿತವಾದವು, ಅವನಿಂದ ಹೊರಬಂದವು, ಅವನು ಇವನ್ನೆಲ್ಲಾ ಸೃಷ್ಟಿಸಿದನು. ಮನುಷ್ಯನನ್ನು ನಿರ್ಮಿಸಿದ ಶಾಸ್ತ್ರವನ್ನು ನಾವಿನ್ನೂ ಕಂಡಿಲ್ಲ. ಯುಕ್ತಿ ಕೂಡ ಮಾನವನ ಸ್ವಭಾವ ಎಂಬ ಅದೇ ಕಾರಣದಿಂದಲೇ ಬಂದಿರುವುದು. ಯುಕ್ತಿಗೆ ಮಾತ್ರ ನಾವು ಮೊರೆ ಇಡಬಹುದು. ಮಾನವನ ಸ್ವಭಾವದೊಂದಿಗೆ ಯುಕ್ತಿ ಮಾತ್ರ ನೇರವಾಗಿ ಸಂಬಂಧಪಟ್ಟಿರುವುದರಿಂದ, ಎಲ್ಲಿಯವರೆಗೆ ಅದು ನಮ್ಮನ್ನು ಒಯ್ಯುವುದೋ, ಅಲ್ಲಿಯವರೆಗೆ ನಾವು ಅದನ್ನು ವಿಧೇಯರಾಗಿ ಅನುಸರಿಸಬೇಕು. ನನ್ನ ಪ್ರಕಾರ ವಿಚಾರ ಎಂದರೆ ಏನು? - ಅದು ಯಾವುದೆಂದರೆ, ಆಧುನಿಕ ಕಾಲದಲ್ಲಿ ಪ್ರತಿಯೊಬ್ಬ ವಿದ್ಯಾವಂತರಾದ ಸ್ತ್ರೀ ಪುರುಷರೂ ಏನನ್ನು ಮಾಡಲು ಬಯಸುವರೋ ಅದೇ ಲೌಕಿಕ ವಿದ್ಯೆಯ ಆವಿಷ್ಕಾರಗಳನ್ನು ಧರ್ಮ ಪ್ರಪಂಚಕ್ಕೂ ಅನ್ವಯಿಸುವುದು. ಯುಕ್ತಿಯ ಪ್ರಥಮ ನಿಯಮವೆ ವಿಶಿಷ್ಟವಾದುದನ್ನು \enginline{(particular)} ಸಾಮಾನ್ಯ ದೃಷ್ಟಿಯಿಂದ \enginline{(general)} ವಿವರಿಸುವುದು. ಸರ್ವ ಸಾಮಾನ್ಯವಾದುದಕ್ಕೆ ಬರುವವರೆಗೆ ಸಾಮಾನ್ಯವಾಗಿರುವುದನ್ನು ಅದಕ್ಕೂ ಹೆಚ್ಚು ಸಾಮಾನ್ಯವಾಗಿರುವುದರಿಂದ ವಿವರಿಸುತ್ತ ಹೋಗುವುದು. ಉದಾಹರಣೆಗೆ ನಮಗೆ ನಿಯಮ ಎಂಬ ಭಾವನೆ ಇದೆ. ಯಾವುದಾದರೂ ಘಟನೆ ನಡೆದರೆ ಅದು ಒಂದು ನಿಯಮದಿಂದ ಆಯಿತು ಎಂದು ವಿವರಿಸಿದರೆ ನಮಗೊಂದು ತೃಪ್ತಿ. ನಮಗೆ ಅದೊಂದು ವಿವರಣೆ ಆಗುವುದು. `ವಿವರಣೆ' ಎಂದರೆ, ಈಗ ನಮಗೆ ಅತೃಪ್ತಿಯನ್ನುಂಟುಮಾಡಿದ ಘಟನೆಯು, ನಿಯಮವೆಂದು ಕರೆಯುವ ಹಲವು ಘಟನಾವಳಿಗಳ ಒಳಗೆ ಒಂದು ಎಂದು ವಿವರಿಸುವುದು. ಒಂದು ಸೇಬು, ಮರದಿಂದ ಬಿದ್ದಾಗ ನ್ಯೂಟನ್ನನಿಗೆ ಅತೃಪ್ತಿಯಾಯಿತು. ಎಲ್ಲಾ ಸೇಬುಗಳೂ ಬೀಳುವುದನ್ನು ನೋಡಿದಾಗ ಅದು ಆಕರ್ಷಣೆಯಿಂದ ಎಂದು ತಿಳಿದಮೇಲೆ ಅವನು ತೃಪ್ತನಾದನು. ಇದು ಮಾನವ ಜ್ಞಾನದ ಒಂದು ನಿಯಮ. ನಾನು ದಾರಿಯಲ್ಲಿ ಒಬ್ಬ ವ್ಯಕ್ತಿಯನ್ನು ನೋಡುತ್ತೇನೆ. ನಾನು ಅವನನ್ನು ಮನುಷ್ಯ ಎಂಬ ಮತ್ತೂ ವಿಸ್ತಾರವಾದ ಭಾವನೆಗೆ ಹೋಲಿಸಿ ತೃಪ್ತನಾಗುತ್ತೇನೆ. ಅವನನ್ನು ಮತ್ತೂ ಸಾಧಾರಣವಾಗಿರುವ ಭಾವನೆಯೊಂದಿಗೆ ಹೋಲಿಸಿದಾಗ ಅವನು ಮನುಷ್ಯ ಎಂಬುದು ಗೊತ್ತಾಗುವುದು. ವಿಶಿಷ್ಟವಾದುದ್ದನ್ನು ಸಾಮಾನ್ಯವಾದುದಕ್ಕೆ ಹೋಲಿಸಬೇಕು, ಸಾಮಾನ್ಯವಾದುದನ್ನೊ ಅದಕ್ಕಿಂತ ಮತ್ತೂ ಸಾಮಾನ್ಯವಾದುದಕ್ಕೆ ಹೋಲಿಸಬೇಕು. ಪ್ರತಿಯೊಂದನ್ನು ಕೊನೆಗೆ ಸರ್ವಸಾಮಾನ್ಯವಾದುದಕ್ಕೆ ಹೋಲಿಸಬೇಕು. ನಮ್ಮಲ್ಲಿರುವ ಕೊನೆಯ ಭಾವನೆಯೇ \enginline{(concept),} ಸರ್ವವ್ಯಾಪ್ತಿಯಾದದ್ದೆ ಅಸ್ತಿತ್ವ \enginline{(existence).} ಅಸ್ತಿತ್ವವೇ ಸರ್ವ ಸಾಮಾನ್ಯವಾದುದರಲ್ಲಿ ಚರಮ ಭಾವನೆ.

ನಾವೆಲ್ಲರೂ ಮಾನವರು. ಅಂದರೆ ನಮ್ಮಲ್ಲಿ ಪ್ರತಿಯೊಬ್ಬರೂ ಮಾನವತೆ ಎಂಬ ಸಾಮಾನ್ಯ ಭಾವನೆಯ ಒಂದು ಅಂಶ. ಮನುಷ್ಯ ಬೆಕ್ಕು ನಾಯಿ ಎಲ್ಲಾ ಪ್ರಾಣಿಗಳು. ಮನುಷ್ಯ ಬೆಕ್ಕು ನಾಯಿ ಎಂಬ ವಿಶಿಷ್ಟ ಉದಾಹರಣೆಗಳು ಪ್ರಾಣಿ ಎಂಬ ಮತ್ತೂ ವಿಸ್ತಾರವಾಗಿರುವ ಭಾವನೆಯ ಅಂಶ. ಮನುಷ್ಯ ಬೆಕ್ಕು ನಾಯಿ ಗಿಡಮರ ಇವುಗಳೆಲ್ಲ ಮೊದಲನೆಯದಕ್ಕಿಂತಲೂ ದೊಡ್ಡದಾದ ಸರ್ವಸಾಮಾನ್ಯವಾದ ಜೀವ ಎಂಬ ಭಾವನೆಯ ಕೆಳಗೆ ಬರುವುವು. ಅನಂತರ ಎಲ್ಲಾ ಪ್ರಾಣಿಗಳು ಮತ್ತು ಎಲ್ಲಾ ವಸ್ತುಗಳೂ ಅಸ್ತಿತ್ವ ಎಂಬ ಏಕಮಾತ್ರ ಭಾವನೆಯ ಕೆಳಗೆ ಬರುವುವು. ನಾವೆಲ್ಲ ಅದರಲ್ಲಿ ಇರುವೆವು. ವಿವರಣೆ ಎಂದರೆ ವಿಶಿಷ್ಟವನ್ನು ಸಾಮಾನ್ಯಕ್ಕೆ ಹೋಲಿಸಿ, ಮತ್ತೂ ಹೆಚ್ಚು ಅಂತಹ ವಿಶಿಷ್ಟ\break ವಸ್ತುಗಳನ್ನು ಕಂಡುಹಿಡಿಯುವುದು ಎಂದು ಅರ್ಥ. ಮನಸ್ಸಿನಲ್ಲಿ ಹಲವು ಕವಾಟಗಳಿರುವಂತೆಯೂ ಅಲ್ಲೆಲ್ಲಾ ಈ ನಿಯಮಗಳು ಸಂಗ್ರಹವಾಗಿರುವಂತೆಯೂ ಕಾಣುವುದು. ಮನಸ್ಸು ಯಾವಾಗಲಾದರೂ ಒಂದು ಹೊಸ ವಿಷಯವನ್ನು ಕಂಡೊಡನೆಯೆ ಅಂತಹ ಮತ್ತಾವುದಾದರೂ ತನ್ನ ಸಂಗ್ರಹದಲ್ಲಿ ಇದೆಯೆ ಎಂದು ತಕ್ಷಣ ನೋಡುವುದು. ಅದು ಇದ್ದರೆ ಹೊಸದನ್ನು ಅಲ್ಲಿ ಸೇರಿಸಿ ತೃಪ್ತರಾಗುವೆವು. ಆಗ ಅದು ನಮಗೆ ಅರ್ಥವಾದಂತೆ. ಜ್ಞಾನ ಎಂದರೆ ಇಷ್ಟೆ, ಅದಕ್ಕಿಂತ ಹೆಚ್ಚೇನೂ ಅಲ್ಲ. ಇಂತಹುದು ಯಾವುದೂ ಇಲ್ಲದೇ ಇದ್ದರೆ ನಮಗೆ ಅತೃಪ್ತಿ. ಮನಸ್ಸಿನಲ್ಲಿ ಅದನ್ನು ಅಳವಡಿಸಿಕೊಳ್ಳುವ ಮತ್ತೊಂದು ವರ್ಗೀಕರಣ ಅಣಿಯಾಗುವವರೆಗೆ ಕಾಯಬೇಕು. ನಾನು ಆಗಲೇ ನಿಮಗೆ ತೋರಿಸಿದಂತೆ, ಜ್ಞಾನ ಎಂದರೆ ಸ್ವಲ್ಪ ಹೆಚ್ಚು ಕಡಿಮೆ ವರ್ಗಿಕರಣ ಎಂದಂತೆ ಆಯಿತು. ಮತ್ತೊಂದು ಇದೆ. ಜ್ಞಾನದ ಎರಡನೆಯ ನಿಯಮವೆ, ವಸ್ತುವಿನ ವಿವರಣೆ ಅದರೊಳಗಿನಿಂದಲೇ ಬರಬೇಕು, ಹೊರಗಡೆಯಿಂದ ಬರಕೂಡದು ಎಂಬುದು. ಮನುಷ್ಯ ಕಲ್ಲನ್ನು ಮೇಲಕ್ಕೆ ಎಸೆದಾಗ ಅದು ಕೆಳಗೆ ಬಿದ್ದರೆ ಅದನ್ನು ಯಾವುದೋ ರಾಕ್ಷಸನು ಕೆಳಗೆ ಎಸೆಯುತ್ತಿರುವನು ಎಂಬ ನಂಬಿಕೆ ಇತ್ತು. ಸ್ವಾಭಾವಿಕವಾದ ಹಲವು ಘಟನೆಗಳನ್ನು ಜನರು ಯಾವುದೋ ಅತಿಭೌತಿಕ ವ್ಯಕ್ತಿಗಳು ಮಾಡುತ್ತಿದ್ದರು ಎಂದು ಹೇಳುತ್ತಿದ್ದರು. ಕಲ್ಲನ್ನು ಯಾವುದೋ ಒಂದು ಭೂತ ಕೆಳಗೆ ಎಳೆಯಿತು ಎಂಬುದು ವಸ್ತುವಿನಿಂದಲೇ ಬಂದ ವಿವರಣೆಯಲ್ಲ. ಅದು ಹೊರಗಿನಿಂದ ಬಂದ ವಿವರಣೆ. ಆದರೆ ಆಕರ್ಷಣೆ ಎಂಬ ಎರಡನೆಯ ವಿವರಣೆ ಕಲ್ಲಿನ ಪ್ರಭಾವಕ್ಕೆ ಅನ್ವಯಿಸುವುದರಿಂದ ವಿವರಣೆ ಆಂತರ್ಯದಿಂದಲೇ ಬಂದಂತೆ ಆಯಿತು. ಆಧುನಿಕ ಭಾವನೆಯಲ್ಲೆಲ್ಲ ನೀವು ಈ ಸ್ವಭಾವವನ್ನು ನೋಡುವಿರಿ. ಒಂದು ಮಾತಿನಲ್ಲಿ ಹೇಳುವುದಾದರೆ, ವಿಜ್ಞಾನವೆಂದರೆ ವಸ್ತುವಿನ ವಿವರಣೆಯು ಅದರ ಸ್ವಭಾವದಲ್ಲೇ ಇರುವುದು, ಅದನ್ನು ವಿವರಿಸುವುದಕ್ಕೆ ಮತ್ತಾವ ಹೊರಗಿನ ವಸ್ತುವಾಗಲಿ ವ್ಯಕ್ತಿಯಾಗಲಿ ಅವಶ್ಯಕವಿಲ್ಲ - ಎಂಬುದು. ರಸಾಯನಶಾಸ್ತ್ರಜ್ಞನಿಗೆ ರಸಾಯನದ್ರವ್ಯಗಳನ್ನು ವಿವರಿಸಬೇಕಾದರೆ ಯಾವ ಭೂತ ಪ್ರೇತ ಪಿಶಾಚಿಗಳೂ ಬೇಕಿಲ್ಲ. ಹಾಗೆಯೇ ಇತರ ಶಾಸ್ತ್ರಗಳೂ ಕೂಡ. ವಿಜ್ಞಾನದ ಈ ದೃಷ್ಟಿಯನ್ನೇ ನಾನು ಧರ್ಮಕ್ಕೆ ಅನ್ವಯಿಸುತ್ತೇನೆ. ಧರ್ಮದಲ್ಲಿ ಈ ದೃಷ್ಟಿಯಿಲ್ಲ. ಅದಕ್ಕೇ ಅದು ಕುಸಿದು ಬೀಳುತ್ತಿರುವುದು. ಪ್ರತಿಯೊಂದು ವಿಜ್ಞಾನಕ್ಕೂ ವಿವರಣೆ ವಸ್ತುವಿನ ಸ್ವಭಾವದಲ್ಲೇ ದೊರಕಬೇಕು. ಧರ್ಮ ಇದನ್ನು ಒದಗಿಸಲಾರದು. ಈ ಪ್ರಪಂಚದಿಂದ ಸಂಪೂರ್ಣ ಬೇರೆಯಾದ ಭಗವಂತನಿರುವನು ಎಂಬ ಪುರಾತನ ಭಾವನೆ ಇದೆ. ಬಹಳ ಪುರಾತನ ಕಾಲದಿಂದಲೂ ಇದನ್ನು ನಂಬುತ್ತಿರುವರು. ಇದನ್ನು ಸಮರ್ಥಿಸುವ ವಾದಸರಣಿಯನ್ನು ಪುನಃ ಪುನಃ ತರುವರು. ವಿಶ್ವಕ್ಕಿಂತ ಬೇರೆಯಾದ ಒಬ್ಬ ದೇವರು ಅತ್ಯಾವಶ್ಯಕ; ಅವನು ವಿಶ್ವವನ್ನು ತನ್ನ ಕೇವಲ ಇಚ್ಚಾಮಾತ್ರದಿಂದ ಸೃಷ್ಟಿಸಿರುವನು, ಅವನೇ ವಿಶ್ವಕ್ಕೆ ಅಧಿಪತಿಯೆಂದು ಧರ್ಮಗಳು ಸಾರುವುವು. ಅವನನ್ನು ಕರುಣಾಮಯನೆಂದು ಕೊಂಡಾಡುತ್ತಾರೆ; ಆದರೂ ಪ್ರಪಂಚದಲ್ಲಿ ಬೇಕಾದಷ್ಟು ಪಕ್ಷಪಾತವನ್ನು ನೋಡುತ್ತೇವೆ. ತಾತ್ತಿಕನು ಇದನ್ನು ಗಮನಕ್ಕೆ ತೆಗೆದುಕೊಳ್ಳುವುದಿಲ್ಲ. ವಿವರಣೆಯ ಅಂತರಾಳದಲ್ಲಿ ಲೋಪದೋಷವಿದೆ, ಈ ವಿವರಣೆ ಹೊರಗಿನಿಂದ ಬಂದುದು, ವಸ್ತುವಿನ ಅಂತರಾಳದಿಂದ ಬಂದುದಲ್ಲ ಎಂದು ಅವನು ಹೇಳುತ್ತಾನೆ. ಈ ವಿಶ್ವಕ್ಕೆ ಕಾರಣ ಯಾವುದು? ಅದಕ್ಕಿಂತ ಹೊರಗಿರುವುದೇ? ಈ ಪ್ರಪಂಚವನ್ನು ಚಲಿಸುತ್ತಿರುವ ಯಾರೋ ಒಬ್ಬರೆ? ಬೀಳುವ ಕಲ್ಲಿನ ವಿವರಣೆ ಹೇಗೆ ಅಸಮರ್ಪಕವಾಗಿ ಕಂಡಿತೋ ಹಾಗೆಯೇ ಧರ್ಮದ ವಿವರಣೆ ಅಸಮರ್ಪಕವಾಗಿದೆ. ಧರ್ಮಗಳು ಅಳಿಯುತ್ತಿರುವುದಕ್ಕೆ ಕಾರಣ ಅವು ಸಮರ್ಪಕವಾದ ವಿವರಣೆಯನ್ನು ಕೊಡದಿರುವುದೇ ಆಗಿದೆ.

ವಸ್ತುವಿನ ವಿವರಣೆ ಅದರ ಅಂತರಾಳದಿಂದಲೇ ಬರಬೇಕು ಎಂಬ ನಿಯಮದೊಂದಿಗೆ ಬರುವ, ಅದೇ ನಿಯಮದ ಮತ್ತೊಂದು ಅಭಿವ್ಯಕ್ತಿಯೇ ಆಧುನಿಕ ವಿಕಾಸವಾದ. ವಿಕಾಸದ ಅರ್ಥವೆಲ್ಲ ಇಷ್ಟು ಮಾತ್ರ ಆಗಿದೆ: ವಸ್ತುವಿನ ಸ್ವಭಾವದ ಪುನರಾವೃತ್ತಿಯಾಗುವುದು, ಪರಿಣಾಮವು ಕಾರಣದ ಮತ್ತೊಂದು ರೂಪ ಮಾತ್ರ, ಕಾರ್ಯದ ಸಾಧ್ಯತೆಗಳೆಲ್ಲ ಆಗಲೇ ಕಾರಣದಲ್ಲಿ ಸುಪ್ತವಾಗಿವೆ. ಬ್ರಹ್ಮಾಂಡವೆಲ್ಲ ಒಂದು ವಿಕಸನವೆ ವಿನಃ ನವ ಸೃಷ್ಟಿಯಲ್ಲ. ಅಂದರೆ, ಪ್ರತಿಯೊಂದು ಕಾರ್ಯವೂ ಅದರ ಹಿಂದಿನ ಕಾರಣದ ಪುನರಾವೃತ್ತಿ; ಹೊಸ ವಾತಾವರಣದಿಂದ ಬೇರೆ ರೂಪವನ್ನು ಧರಿಸಿದೆ ಅಷ್ಟೆ. ವಿಶ್ವದಲ್ಲೆಲ್ಲಾ ಹೀಗೆಯೇ ಆಗುತ್ತಿರುವುದು. ಈ ಬದಲಾವಣೆಗೆ ಕಾರಣವನ್ನು ಹುಡುಕಲು ನಾವು ಹೊರಗೆ ಹೋಗಬೇಕಾಗಿಲ್ಲ. ಅವು ವಸ್ತುವಿನಲ್ಲೇ ಅಂತರ್ಗತವಾಗಿವೆ. ಹೊರಗೆ ನಾವು ಕಾರಣವನ್ನು ಹುಡುಕುವ ಅವಶ್ಯಕತೆಯೇ ಇರುವುದಿಲ್ಲ. ಇದೂ ಕೂಡ ಧರ್ಮವನ್ನು ಧ್ವಂಸಮಾಡುತ್ತಿದೆ. ಧರ್ಮವನ್ನು ಧ್ವಂಸಮಾಡುತ್ತಿದೆ ಎಂದರೆ, ವಿಶ್ವದ ಹೊರಗೆ ಒಬ್ಬ ಪ್ರತ್ಯೇಕವಾದ ದೇವರಿರುವನು, ಇವನೊಬ್ಬ ದೊಡ್ಡ ಮನುಷ್ಯ, ಮತ್ತೇನೂ ಅಲ್ಲ, ಎಂದು ಸಾರುವ ಧರ್ಮಗಳು ತಮ್ಮ ಕಾಲ ಮೇಲೆ ಇನ್ನು ಮುಂದೆ ನಿಲ್ಲಲಾರವು. ಅವನ್ನು ಕೆಳಗುರುಳಿಸಿದಂತೆ ಆಗಿದೆ.

ಈ ಎರಡು ನಿಯಮಗಳನ್ನೂ ತೃಪ್ತಿಪಡಿಸುವಂತಹ ಧರ್ಮ ಸಾಧ್ಯವೆ? ಇದು ಸಾಧ್ಯ ಎಂದು ನನಗೆ ತೋರುತ್ತದೆ. ಮೊದಲು ಸಾಮಾನ್ಯೀಕರಣದ ನಿಯಮವನ್ನು ತೃಪ್ತಿಪಡಿಸಬೇಕಾಗಿದೆ. ನಾವು ಸಾಮಾನ್ಯೀಕರಣದ ಪರಾಕಾಷ್ಠೆಯನ್ನು ಮುಟ್ಟಬೇಕಾಗಿದೆ. ಅದು ಎಲ್ಲಾ ವರ್ಗಿಕರಣಕ್ಕಿಂತಲೂ ಸರ್ವ ಸಾಮಾನ್ಯವಾಗಿರುವುದು ಮಾತ್ರವಲ್ಲ, ಅದರಿಂದಲೇ ಪ್ರತಿಯೊಂದೂ ಬರಬೇಕು. ಅತಿ ಕ್ಷುದ್ರ ಕಾರ್ಯಗಳು ಕೂಡಾ ಈ ವರ್ಗಕ್ಕೆ ಸೇರಿರಬೇಕು. ಕಾರಣ - ಬಹಳ ಮೇಲಿರುವ ಕಾರಣ, ಕೊನೆಯ ಕಾರಣ, ಮೂಲ ಕಾರಣ - ಅದರ ಅತಿ ಕ್ಷುದ್ರವಾದ, ಇವು ದೂರವಾದ ಕಾರ್ಯ ಆಗಿರಬೇಕು. ಇವು ಒಂದೇ ವಿಕಸನ - ಸರಪಳಿಯ ಹಲವು ಕೊಂಡಿಗಳಂತೆ ಇವೆ. ವೇದಾಂತದ ಬ್ರಹ್ಮ ಈ ನಿಯಮವನ್ನು ತೃಪ್ತಿಪಡಿಸುವುದು. ಏಕೆಂದರೆ ನಾವು ಪಡೆಯಬಹುದಾದ ಕೊನೆಯ ಸಾಮಾನ್ಯ ವರ್ಗಿಕರಣವೇ ಬ್ರಹ್ಮ. ಅದಕ್ಕೆ ಯಾವ ಗುಣಗಳೂ ಇಲ್ಲ. ಅದು ಕೇವಲ ಸಚ್ಚಿದಾನಂದ ಸ್ವರೂಪ, ಮಾನವನ ಮನಸ್ಸು ಪಡೆಯುವ ಕೊನೆಯ ವರ್ಗಿಕರಣವೇ ಸತ್ (ಅಂದರೆ ಅಸ್ತಿತ್ವ) ಎಂಬುದನ್ನು ನೋಡಿದೆವು. ಚಿತ್ ಎಂದರೆ ನಮ್ಮಲ್ಲಿರುವ ಜ್ಞಾನವಲ್ಲ; ಆದರೆ ಅದರ ಸಾರ; ಅನಂತ ವಿಕಾಸ ಪಥದಲ್ಲಿ ಮಾನವನಲ್ಲಿ ಮತ್ತು\break ಪ್ರಾಣಿಗಳಲ್ಲಿ ವ್ಯಕ್ತವಾಗುತ್ತಿರುವ ಜ್ಞಾನದ ಸಾರ. ಜ್ಞಾನದ ಸಾರವೆಂದರೆ ಅದಕ್ಕೂ ಅತೀತವಾದ ಸ್ಥಿತಿ. ನಾನು ಹಾಗೆ ಹೇಳಬಹುದಾದರೆ, ಅದು ಪ್ರಜ್ಞೆಗೂ ಅತೀತವಾದ ಸ್ಥಿತಿ. ಇದನ್ನೇ ನಾವು ಚಿತ್ ಎನ್ನುವುದು. ವಿಶ್ವದಲ್ಲಿ ನಾವು ಕಾಣುವ ಮೂಲಭೂತ ಏಕತೆಯೇ ಇದು. ಆಧುನಿಕ ವಿಜ್ಞಾನ ಪದೇ ಪದೇ ಏನನ್ನಾದರೂ ಸಮರ್ಥಿಸುತ್ತಿದ್ದರೆ ಅದೇ ದೈಹಿಕವಾಗಿ, ಮಾನಸಿಕವಾಗಿ, ಮತ್ತು ಆಧ್ಯಾತ್ಮಿಕವಾಗಿ ನಾವೆಲ್ಲ ಒಂದು ಎಂಬುದು. ದೈಹಿಕವಾಗಿ ಆದರೂ ನಾವು ಬೇರೆ ಬೇರೆ ಎಂದು ಭಾವಿಸುವುದು ತಪ್ಪು. ವಾದದೃಷ್ಟಿಯಿಂದ ನಾವು ಜಡವಾದಿಗಳು ಎಂದು ಭಾವಿಸೋಣ. ಆಗ ಈ ವಿಶ್ವವೆಲ್ಲ ಒಂದು ಭೌತಿಕ ಸಾಗರವಾಗುವುದು. ಅಲ್ಲಿ ನಾವು ನೀವು ಎಂಬುದು ಸಣ್ಣ ಸುಳಿಗಳಿದ್ದಂತೆ. ಪ್ರತಿಯೊಂದು ಸುಳಿಗೂ ಭೌತಿಕವಸ್ತುಗಳು ಬಂದು ಸುಳಿಯ ಆಕಾರವನ್ನು ತಾಳಿ ಪುನಃ ಭೌತಿಕ ವಸ್ತುಗಳಂತೆ ಹೊರಗೆ ಬರುತ್ತವೆ. ಈಗ ನನ್ನ ದೇಹದಲ್ಲಿರುವ ವಸ್ತು ಕೆಲವು ಕಾಲದ ಹಿಂದೆ ನಿಮ್ಮ ದೇಹದಲ್ಲಿದ್ದಿರಬಹುದು, ಇಲ್ಲವೆ ಸೂರ್ಯನಲ್ಲಿದ್ದಿರಬಹುದು ಅಥವಾ ಯಾವುದೋ ಒಂದು ಸಸ್ಯದಲ್ಲಿ ಇದ್ದಿರಬಹುದು. ಅದು ಒಂದು ನಿರಂತರ ಬದಲಾವಣೆಯ ಸ್ಥಿತಿಯಲ್ಲಿರಬಹುದು. ನನ್ನ ದೇಹ, ನಿಮ್ಮ ದೇಹ ಎಂದರೆ ಅರ್ಥವೇನು? ಇದು ದೇಹದ ಏಕತೆ. ಇದರಂತೆ ಆಲೋಚನೆ ಕೂಡ. ಒಂದು ದೊಡ್ಡ ಆಲೋಚನಾ ಸಾಗರವಿದೆ, ಅನಂತವಾದುದು ಅದು; ಅಲ್ಲಿ ನನ್ನ ನಿಮ್ಮ ಮನಸ್ಸೆಂಬುವು ಸಣ್ಣ ಸುಳಿಗಳಂತೆ. ಅದರ ಪರಿಣಾಮವನ್ನು ನೀವು ಈಗ ನೋಡುತ್ತಿಲ್ಲವೆ? ನನ್ನ ಆಲೋಚನೆ ಹೇಗೆ ನಿಮಗೆ ಬರುತ್ತಿದೆ, ನಿಮ್ಮ ಆಲೋಚನೆ ನನಗೆ ಹೇಗೆ ಬರುತ್ತಿದೆ ಎಂಬುದು ಕಾಣುವುದಿಲ್ಲವೆ? ನಮ್ಮೆಲ್ಲರ ಜೀವನಗಳೂ ಒಂದು; ಆಲೋಚನೆಯಲ್ಲಿಯೂ ನಾವೆಲ್ಲರೂ ಒಂದು. ಇದಕ್ಕೂ ಮುಂದಿನ ಒಂದು ವರ್ಗಿಕರಣಕ್ಕೆ ಹೋದರೆ ಭೌತಿಕ ವಸ್ತುಗಳಲ್ಲಿ ಮತ್ತು ಆಲೋಚನೆಗಳಲ್ಲಿ ಆತ್ಮ ಸುಪ್ತವಾಗಿರುವುದು ಕಾಣುವುದು. ಈ ಮೂಲದಿಂದ ಎಲ್ಲವೂ ಬಂದಿದೆ. ಅದು ಮುಖ್ಯವಾಗಿ ಒಂದೇ ಆಗಿರಬೇಕು. ನಾವು ತತ್ತ್ವತಃ ಒಂದೇ ಆಗಿದ್ದೇವೆ. ನಾವು ದೈಹಿಕವಾಗಿಯೂ ಒಂದೇ ಆಗಿದ್ದೇವೆ. ಮಾನಸಿಕವಾಗಿಯೂ ಒಂದೇ ಆಗಿದ್ದೇವೆ. ನಾವು ಆತ್ಮವನ್ನು ನಂಬುವುದಾದರೆ, ಆಧ್ಯಾತ್ಮಿಕವಾಗಿಯೂ ಒಂದೇ ಎಂದು ಬೇರೆಯಾಗಿ ಹೇಳಬೇಕಾಗಿಲ್ಲ. ಆಧುನಿಕ ವಿಜ್ಞಾನ ಪ್ರತಿದಿನ ಸಮರ್ಥಿಸುತ್ತಿರುವುದು ಈ ಒಂದನ್ನೇ. ಅಹಂಕಾರದಿಂದ ಮೆರೆಯುತ್ತಿರುವ ಮನುಷ್ಯನಿಗೆ, ಅಲ್ಲಿರುವ ಕೀಟಕ್ಕೂ ತನಗೂ ಏನೊ ತುಂಬಾ ವ್ಯತ್ಯಾಸವಿದೆ ಎಂದು ಅನ್ನಿಸುವುದು. ನೀವೆಲ್ಲ ಒಂದೇ ಎಂದು ವಿಜ್ಞಾನ ಸಾರುವುದು. ನಿಮ್ಮ ಹಿಂದಿನ ಜನ್ಮಗಳಲ್ಲಿ ನೀವು ಅದರಂತೆ ಇದ್ದಿರಿ. ನೀವು ಈಗ ಅಹಂಕಾರಪಡುವ ಮನುಷ್ಯತ್ವಕ್ಕೆ ಆ ಕೀಟ ಕ್ರಮೇಣ ಸಾಗಿಬಂದಿದೆ. ಅಸ್ತಿತ್ವದ ಈ ತತ್ತ್ವವೆ, ಪ್ರಪಂಚದಲ್ಲಿರುವ ಎಲ್ಲ ವಸ್ತುಗಳೊಂದಿಗೆ ನಮ್ಮ ಏಕತೆಯನ್ನು ಸಾರುವ ಈ ಘನ ಸಂದೇಶವೆ, ನಾವು ಕಲಿಯಬೇಕಾದ ದೊಡ್ಡ ನೀತಿ. ನಮ್ಮಲ್ಲಿ ಹಲವರಿಗೆ ನಮಗಿಂತ ಉತ್ತಮವಾದವರೊಡನೆ ಒಂದಾಗಲು ಆಸೆ; ತಮಗಿಂತ ಕೀಳಾದವರೊಡನೆ ಒಂದಾಗಲು ಯಾರಿಗೂ ಆಸೆಯಿಲ್ಲ. ಇದೇ ಮಾನವನ ಅಜ್ಞಾನ. ಸಮಾಜವು ಗೌರವದಿಂದ ಕಾಣುವ ಯಾರಾದರೂ ಪೂರ್ವಿಕರಿದ್ದರೆ, ಆ ಪೂರ್ವಿಕರು ಕ್ರೂರಿಗಳಾಗಿರಲಿ, ದರೋಡೆಗಾರರಾಗಿರಲಿ, ಅಥವಾ ದರೋಡೆಕಾರರ ನಾಯಕನಾದರೂ ಚಿಂತೆಯಿಲ್ಲ, ನಾವೆಲ್ಲ ನಮ್ಮ ಪೂರ್ವಿಕರು ಅವರೇ ಎಂದು ಸಮರ್ಥಿಸುವೆವು. ಆದರೆ ನಮ್ಮ ಪೂರ್ವಿಕರಲ್ಲಿ ದರಿದ್ರರೂ ಸತ್ಯವಂತರೂ ಗುಣಾಢ್ಯರೂ ಯಾರಾದರು ಇದ್ದರೆ, ಯಾರೂ ಅವರು ತಮ್ಮ ಮೂಲ ಪುರುಷ ಎಂದು ಅವರಿಗೆ ಗೌರವ ಕೊಡಲು ಇಚ್ಚಿಸುವುದಿಲ್ಲ. ಆದರೆ ಅಜ್ಞಾನದ ತೆರೆ ನಮ್ಮ ಕಣ್ಣಿಂದ ಜಾರುತ್ತಿದೆ. ಸತ್ಯ ಹೆಚ್ಚು ಹೆಚ್ಚು ಪ್ರಕಾಶಕ್ಕೆ ಬರುತ್ತಿದೆ. ಇದರಿಂದ ಧರ್ಮಕ್ಕೆ ಎಷ್ಟೋ ಅನುಕೂಲ. ನಾನು ನಿಮಗೆ ಹೇಳುತ್ತಿರುವ ಅದ್ವೈತ ಬೋಧನೆಯೆ ಅದು. ಆತ್ಮವೇ ಈ ವಿಶ್ವದ ಸಾರ, ಎಲ್ಲಾ ಜೀವಿಗಳ ಸಾರ, ಅಷ್ಟು ಮಾತ್ರವಲ್ಲ ನೀನೇ ಅದಾಗಿರುವೆ. ನೀನು ವಿಶ್ವದೊಂದಿಗೆ ಒಂದು. ಯಾರು ತನಗೂ ಮತ್ತೊಬ್ಬನಿಗೂ ಒಂದು ಕೂದಲಿನ ಎಳೆಯಷ್ಟಾದರೂ ವ್ಯತ್ಯಾಸವಿದೆ ಎಂದು ಸಾರುವನೋ ಅವನು ತಕ್ಷಣವೇ ವಧೆಗೆ ಒಳಗಾಗುವನು. ಯಾರಿಗೆ ಈ ಏಕತೆ ಗೊತ್ತಿದೆಯೋ, ಯಾರು ತಾನು ವಿಶ್ವದೊಂದಿಗೆ ಒಂದು ಎಂದು ಅರಿತಿರುವನೋ ಅವನಿಗೆ ಮಾತ್ರ ಆನಂದ.

ಹೀಗೆ ವೇದಾಂತಧರ್ಮವು ವೈಜ್ಞಾನಿಕ ಜಗತ್ತಿನ ಕೋರಿಕೆಯನ್ನು, ವರ್ಗಿಕರಣದ ಪರಾಕಾಷ್ಠೆ ಮತ್ತು ವಿಕಾಸವಾದ ಇವುಗಳ ಮೂಲಕ, ತೃಪ್ತಿ ಪಡಿಸಬಲ್ಲದು ಎಂಬುದನ್ನು ನೋಡುತ್ತೇವೆ. ವಸ್ತುವಿನ ವಿವರಣೆ ಅದರಿಂದಲೇ ಬರುವುದು ಎಂಬ ಮತ್ತೊಂದು ನಿಯಮ ವೇದಾಂತದಿಂದ ಮತ್ತೂ ಚೆನ್ನಾಗಿ ಸಮರ್ಥಿಸಲ್ಪಡುವುದು. ವೇದಾಂತದ ದೇವರಾದ ಬ್ರಹ್ಮನ ಹೊರಗೆ ಏನೂ ಇಲ್ಲ, ಇಲ್ಲವೇ ಇಲ್ಲ. ಇದೆಲ್ಲ ಅವನೆ, ಅವನು ವಿಶ್ವದಲ್ಲಿರುವನು, ಅವನೇ ವಿಶ್ವವಾಗಿರುವನು. “ನೀನೇ ಪುರುಷ, ನೀನೇ ಸ್ತ್ರೀ, ಯೌವನದ ಅಹಂಕಾರದಲ್ಲಿ ಮೆರೆಯುತ್ತಿರುವ ಯುವಕ ನೀನೆ, ಊರುಗೋಲಿನ ಮೇಲೆ ತತ್ತರಿಸಿಕೊಂಡು ನಡೆಯುತ್ತಿರುವ ಮುದುಕ ನೀನೆ.” ಅವನು ಇಲ್ಲಿರುವನು, ನಾವು ನೋಡುವುದು ಅವನನ್ನು, ಅನುಭವಿಸುವುದು ಅವನನ್ನು, ನಾವಿರುವುದು ಅವನಲ್ಲಿ, ನಮ್ಮ ವ್ಯಕ್ತಿತ್ವ ಇರುವುದು ಅವನಲ್ಲಿ. ಹೊಸ ಟೆಸ್ಟಮೆಂಟಿನಲ್ಲಿ ನಿಮಗೆ ಈ ಭಾವನೆ ಬರುವುದು. ಅವನೇ ವಿಶ್ವದಲ್ಲೆಲ್ಲ ಓತಪ್ರೋತನಾಗಿರುವ ದೇವರು, ಅವನು ಜೀವನದ ಸಾರ, ಕೇಂದ್ರ, ಪ್ರಪಂಚದ ಮೂಲ. ಅವನು ಈ ವಿಶ್ವದಲ್ಲಿ ಇದ್ದಾನೆ. ಇದೆಲ್ಲವೂ ಅವನ ಆವಿರ್ಭಾವವಿದ್ದಂತೆ. ಆ ಸಚ್ಚಿದಾನಂದ ಸಾಗರದಲ್ಲಿ ನೀವು ನಾವೆಲ್ಲ ಒಂದು ಅಂಶ, ಒಂದು ಸುಳಿ, ಹರಿಯುವ ಒಂದು ಸಣ್ಣ ಕಾಲುವೆ, ಅದರ ಒಂದು ಅಭಿವ್ಯಕ್ತಿ. ಮನುಷ್ಯ ಮನುಷ್ಯರಿಗೆ, ಮನುಷ್ಯರಿಗೆ ಮತ್ತು ದೇವತೆಗಳಿಗೆ, ಮನುಷ್ಯರಿಗೆ ಮತ್ತು ಪ್ರಾಣಿಗಳಿಗೆ, ಪ್ರಾಣಿಗಳಿಗೆ ಮತ್ತು ಸಸ್ಯಗಳಿಗೆ, ಸಸ್ಯಗಳಿಗೂ ಮತ್ತು ಕಲ್ಲಿಗೂ ಇರುವ ವ್ಯತ್ಯಾಸ ವಸ್ತುವಿನ ಭಿನ್ನತೆಯಲ್ಲಿ ಅಲ್ಲ; ಏಕೆಂದರೆ ವಿಶ್ವದಲ್ಲಿ ಒಂದು ಕಣದವರೆಗೆ ಎಲ್ಲವೂ ಆ ಅನಂತ ಬ್ರಹ್ಮನ ಅಭಿವ್ಯಕ್ತಿ. ಅವುಗಳಲ್ಲಿ ಇರುವ ವ್ಯತ್ಯಾಸ ಅಭಿವ್ಯಕ್ತಿಯ ತರತಮದಲ್ಲಿ ಮಾತ್ರ. ನಾನು ಕೀಳು ಅಭಿವ್ಯಕ್ತಿ, ನೀವು ಮೇಲು ಅಭಿವ್ಯಕ್ತಿ ಇರಬಹುದು. ಆದರೆ ಎರಡರಲ್ಲಿ ಇರುವ ವಸ್ತು ಒಂದೇ. ನಾವಿಬ್ಬರೂ ಭಗವಂತನೆಂಬ ಸರೋವರಕ್ಕೆ ಇರುವ ಕಾಲುವೆಗಳಂತೆ. ಆದಕಾರಣ ನಿಮ್ಮ ಸ್ವಭಾವ ದೇವರಾಯಿತು. ಅದರಂತೆಯೇ ನನ್ನದೂ ಕೂಡ. ನೀವು ಆಜನ್ಮಸಿದ್ಧ ಹಕ್ಕಿನಿಂದಲೇ ಭಗವಂತನ ಸ್ವರೂಪವಾಗಿರುವಿರಿ. ಅದರಂತೆಯೇ ನಾನು ಕೂಡ. ನೀವೇನೊ ಪುಣ್ಯಪುರುಷನಾದ ದೇವರಾಗಿರಬಹುದು, ನಾನು ಅತಿ ಘೋರ ಪಾಪಿಯಾಗಿರಬಹುದು. ಆದರೂ ಕೂಡ ಸಚ್ಚಿದಾನಂದ ಸ್ವರೂಪ ನನ್ನ ಆಜನ್ಮಸಿದ್ಧ ಹಕ್ಕು. ಅದರಂತೆಯೇ ನಿಮ್ಮದು ಕೂಡ. ನೀವು ಇಂದು ಹೆಚ್ಚು ಅದನ್ನು ವ್ಯಕ್ತಗೊಳಿಸಿರಬಹುದು. ಸ್ವಲ್ಪ ತಾಳಿ, ನಾನು ಕೂಡ ಹೆಚ್ಚು ವ್ಯಕ್ತಗೊಳಿಸುವೆ. ಏಕೆಂದರೆ ಅದೆಲ್ಲವೂ ನನ್ನೊಳಗೇ ಇದೆ. ಯಾವ ಹೊರಗಿನ ವಿವರಣೆಯನ್ನೂ ಇಲ್ಲಿ ಹುಡುಕುತ್ತಿಲ್ಲ, ಕೇಳುತ್ತಿಲ್ಲ. ಈ ವಿಶ್ವದ ಸಮಷ್ಟಿಯು ದೇವರೇ ಆಗಿರುವನು. ಹಾಗಾದರೆ ದೇವರು ಜಡವಸ್ತುವೆ? ಅಲ್ಲ, ನಿಜವಾಗಿಯೂ ಅಲ್ಲ. ಪಂಚೇಂದ್ರಿಯಗಳ ಮೂಲಕ ನೋಡಿದಾಗ ಜಡವಸ್ತುವೇ ದೇವರು; ಬುದ್ದಿಯ ಮೂಲಕ ನೋಡಿದಾಗ ಅದೇ ಮನಸ್ಸು. ಅದನ್ನು ಆತ್ಮನ ಮೂಲಕ ನೋಡಿದಾಗ ಅದು ಆತ್ಮ, ಅವನು ಜಡವಸ್ತುವಲ್ಲ, ಆದರೆ ಜಡವಸ್ತುವಿನಲ್ಲಿ ಯಾವುದು ಸತ್ಯವಾಗಿರುವುದೋ ಅದೇ ಅವನು, ಈ ಕುರ್ಚಿಯಲ್ಲಿ ಯಾವುದು ಸತ್ಯವಾಗಿರುವುದೋ ಅದೇ ಅವನು. ಏಕೆಂದರೆ ಕುರ್ಚಿಯನ್ನು ತಯಾರುಮಾಡಲು ಎರಡು ವಸ್ತುಗಳು ಬೇಕು. ಯಾವುದೊ ಒಂದು ಹೊರಗೆ ಇತ್ತು; ಅದನ್ನು ನನ್ನ ಇಂದ್ರಿಯಗಳು ಒಳಗೆ ತಂದವು. ಅದಕ್ಕೆ ನನ್ನ ಮನಸ್ಸು ಮತ್ತೇನನ್ನೂ ಸೇರಿಸಿತು. ಈ ಮಿಶ್ರಣವೇ ಕುರ್ಚಿ. ಯಾವುದು ಸನಾತನವಾಗಿತ್ತೋ, ಇಂದ್ರಿಯಗಳ ಮತ್ತು ಬುದ್ಧಿಯ ಆಶ್ರಯವಿಲ್ಲದೆ ಇತ್ತೊ, ಅದೇ ಸ್ವಯಂ ಭಗವಂತ. ಅವನ ಮೇಲೆ ಇಂದ್ರಿಯಗಳು ಕುರ್ಚಿ, ಮೇಜು, ಕೋಣೆ, ಮನೆ, ಜಗತ್ತು, ಸೂರ್ಯ, ಚಂದ್ರ, ನಕ್ಷತ್ರ ಇವುಗಳನ್ನೆಲ್ಲ ಚಿತ್ರಿಸುತ್ತಿವೆ. ಹಾಗಾದರೆ ನಾವೆಲ್ಲ ಒಂದೇ ಬಗೆಯ ಕುರ್ಚಿಯನ್ನು ಹೇಗೆ ನೋಡುತ್ತೇವೆ? ಒಂದೇ ಬಗೆಯ ಚಿತ್ರವನ್ನು ಆ ಸಚ್ಚಿದಾನಂದದ ಮೇಲೆ ಹೇಗೆ ಚಿತ್ರಿಸುತ್ತೇವೆ? ಎಲ್ಲರೂ ಒಂದೇ ರೀತಿ ಚಿತ್ರಿಸುತ್ತಾರೆಂದು ಅಲ್ಲ. ಯಾರು ಒಂದೇ ರೀತಿ ಚಿತ್ರಿಸುತ್ತಾರೋ ಅವರೆಲ್ಲ ಒಂದೇ ಕ್ಷೇತ್ರದಲ್ಲಿರುವುದರಿಂದ ಒಬ್ಬರು ಮತ್ತೊಬ್ಬರ ಚಿತ್ರವನ್ನು ನೋಡುತ್ತಾರೆ. ಮತ್ತು ಒಬ್ಬರು ಮತ್ತೊಬ್ಬರನ್ನು ನೋಡುತ್ತಾರೆ. ನಮ್ಮ ನಿಮ್ಮ ಮಧ್ಯದಲ್ಲಿ ಕೋಟ್ಯಂತರ ಜೀವಿಗಳು ಇರಬಹುದು. ಅವರು ನಮ್ಮಂತೆ ಚಿತ್ರಿಸಲ್ಪಡದೆ ಇರುವುದರಿಂದ ನಾವು ಅವರ ಚಿತ್ರವನ್ನು ನೋಡುತ್ತಿಲ್ಲ.

ನಿಮಗೆಲ್ಲ ತಿಳಿದಿರುವಂತೆ ಆಧುನಿಕ ಭೌತಿಕ ಅನ್ವೇಷಣೆಗಳು, ಯಾವುದು ಸತ್ಯವೊ ಅದು ಸೂಕ್ಷ್ಮವಾದುದು, ಸ್ಕೂಲವಾಗಿರುವುದು ಕೇವಲ ತೋರಿಕೆ ಅಷ್ಟೆ, ಎಂಬುದನ್ನು ಹೆಚ್ಚು ಹೆಚ್ಚು ತೋರಲು ಪ್ರಯತ್ನಿಸುತ್ತಿವೆ. ಅದು ಹೇಗಾದರೂ ಇರಲಿ, ಆಧುನಿಕ ವಿಚಾರ ಸರಣಿಗೆ ಸರಿಯಾಗಿ ನಿಲ್ಲಬಲ್ಲ ಧರ್ಮವೆಂದರೆ ಅದು ಅದೈತ. ಏಕೆಂದರೆ ಅದು ಎರಡು ಷರತ್ತುಗಳನ್ನು ಪೂರೈಸುವುದು. ಅದು ವ್ಯಕ್ತಿತ್ವವನ್ನು ಮಾರಿ ನಿಂತಿರುವ ಸಾಮಾನ್ಯವರ್ಗಿಕರಣ, ಅದು ಎಲ್ಲರಿಗೂ ಸಾಮಾನ್ಯವಾದುದು. ಸಾಕಾರ ಈಶ್ವರನಲ್ಲಿ ಕೊನೆಗಾಣುವ ಸಾಮಾನೀಕರಣ ಎಂದಿಗೂ ಸರ್ವಸಾಮಾನ್ಯವಾಗಲಾರದು. ನಾನು ಪ್ರಥಮತಃ ಒಬ್ಬ ಸಾಕಾರ ದೇವರನ್ನು ಕಲ್ಪಿಸಿಕೊಳ್ಳಬೇಕಾದರೆ ಅವನನ್ನು ದಯಾಮಯ, ಸರ್ವಗುಣಸಂಪನ್ನ ಎಂದು ಕರೆಯಬೇಕಾಗುವುದು. ಆದರೆ ಈ ಪ್ರಪಂಚ ಒಂದು ಮಿಶ್ರಣ. ಇದರಲ್ಲಿ ಒಳ್ಳೆಯದು ಮತ್ತು ಕೆಟ್ಟದ್ದು ಎಂಬ ಎರಡೂ ಇವೆ. ನಮಗೆ ಬೇಕಾದುದನ್ನು ಮಾತ್ರ ಬೇರೆ ತೆಗೆದುಕೊಂಡು ಅದನ್ನು ಸಾಕಾರ ದೇವರು ಎನ್ನುವೆವು, ನೀವು ಹೇಗೆ ಸಾಕಾರ ದೇವರನ್ನು ಅವನು ಅದು, ಅವನು ಇದು, ಎಂದು ಹೇಳುತ್ತೀರೋ ಹಾಗೆಯೇ ಅವನು ಅದಲ್ಲ, ಇದಲ್ಲ, ಎಂದು ಹೇಳಬೇಕಾಗುವುದು. ಯಾವಾಗಲೂ ಈ ಸಾಕಾರ ದೇವರೊಡನೆ ಒಬ್ಬ ಸೈತಾನನೂ ಇರಬೇಕಾಗುವುದು. ಆದಕಾರಣವೆ ನಾವು ಸ್ಪಷ್ಟವಾಗಿ ಸಾಕಾರ ದೇವರು ಸರಿಯಾದ ಸಾಮಾನ್ಯೀಕರಣ ಎಂದು ಹೇಳಲಾರೆವು. ನಾವು ಇದನ್ನು ಅತಿಕ್ರಮಿಸಿ ನಿರಾಕಾರಕ್ಕೆ ಹೋಗಬೇಕು. ವಿಶ್ವವು ಸುಖದುಃಖಗಳೊಡನೆ ಅದರಲ್ಲಿದೆ. ಏಕೆಂದರೆ ಇಲ್ಲಿರುವುದೆಲ್ಲ ಆ ನಿರಾಕಾರದಿಂದ ಬಂದಿದೆ. ಪಾಪ ಮುಂತಾದವನ್ನೆಲ್ಲ ಆರೋಪ ಮಾಡಿದರೆ ಆ ದೇವರು ಎಂತಹವನು? ಒಳ್ಳೆಯದು ಮತ್ತು ಕೆಟ್ಟದ್ದು ಬೇರೆ ಬೇರೆ ಮುಖಗಳಷ್ಟೆ, ಎರಡೂ ಒಂದೇ ವಸ್ತುವಿನ ಅಭಿವ್ಯಕ್ತಿಗಳು ಅಷ್ಟೆ. ಎರಡಿದ್ದವು ಎಂದು ಹೇಳುವುದು ಮೊದಲಿನಿಂದಲೇ ತಪ್ಪು ಮಾಡಿದಂತೆ. ನಮ್ಮ ಜಗತ್ತಿನಲ್ಲಿ ಎಷ್ಟೋ ಪಾಲು ದುಃಖಕ್ಕೆಲ್ಲ, ಒಳ್ಳೆಯದು ಕೆಟ್ಟದ್ದು ಎಂಬುವು ಎರಡು ಸಂಪೂರ್ಣ ಬೇರೆ ಬೇರೆ, ಅವು ಸ್ವತಂತ್ರವಾದುವು, ಪಾಪ ಪುಣ್ಯಗಳು ಬೇರೆ ಬೇರೆ, ಅವನ್ನು ಪ್ರತ್ಯೇಕಿಸಬಹುದು ಎಂದು ಭಾವಿಸಿದುದೇ ಕಾರಣ. ಸದಾಕಾಲದಲ್ಲಿಯೂ ಕೆಟ್ಟದ್ದಾಗಿರುವುದನ್ನು ಅಥವಾ ಸದಾಕಾಲದಲ್ಲಿಯೂ ಒಳ್ಳೆಯದಾಗಿರುವುದನ್ನು ತೋರಬಲ್ಲವನನ್ನು ನಾನು ನೋಡಲು ಆಶಿಸುವೆನು. ಒಬ್ಬನು ಧೈರ್ಯವಾಗಿ ನಮ್ಮ ಜೀವನದಲ್ಲಿ ಇದು ಬರಿಯ ಒಳ್ಳೆಯದೆಂದೂ, ಮತ್ತೊಂದು ಬರಿಯ ಕೆಟ್ಟದೆಂದೂ ಹೇಳುವುದಕ್ಕೆ ಸಾಧ್ಯವೆನ್ನುವಂತೆ ಮಾತನಾಡುವನು. ಯಾವುದು ಇಂದು ಒಳ್ಳೆಯದೊ ಅದು ನಾಳೆ ಕೆಟ್ಟದ್ದಾಗಬಹುದು. ಯಾವುದು ಇಂದು ಕೆಟ್ಟದ್ದೊ ಅದು ನಾಳೆ ಒಳ್ಳೆಯದಾಗಬಹುದು. ಯಾವುದು ನನಗೆ ಒಳ್ಳೆಯದು ಅದು ನಿಮಗೆ ಕೆಟ್ಟದ್ದಾಗಬಹುದು. ನಿರ್ಣಯವೇನಾಯಿತೆಂದರೆ ಇತರ ವಸ್ತುಗಳಂತೆಯೇ ಒಳ್ಳೆಯದು ಕೆಟ್ಟದ್ದು ಕೂಡ ವಿಕಾಸವಾಗುತ್ತಿರುವುವು. ವಿಕಾಸವಾಗುತ್ತಿರುವಾಗ ಒಂದು ಹಂತದಲ್ಲಿ ಅದನ್ನು ಕೆಟ್ಟದ್ದು ಎನ್ನುವೆವು, ಮತ್ತೊಂದು ಹಂತದಲ್ಲಿ ಒಳ್ಳೆಯದು ಎನ್ನುವೆವು. ನನ್ನ ಸ್ನೇಹಿತನನ್ನು ಕೊಲ್ಲುವ ಬಿರುಗಾಳಿಯನ್ನು ಕೆಟ್ಟದ್ದು ಎನ್ನುತ್ತೇನೆ. ಆದರೆ ಅದೇ ಬಿರುಗಾಳಿ ಗಾಳಿಯಲ್ಲಿರುವ ವಿಷಕ್ರಿಮಿಗಳನ್ನು ನಾಶಮಾಡಿ ಸಾವಿರಾರು ಜೀವಿಗಳ ಪ್ರಾಣವನ್ನು ರಕ್ಷಿಸಿರಬಹುದು. ಅವರು ಅದನ್ನು ಒಳ್ಳೆಯದು ಎನ್ನುತ್ತಾರೆ, ನಾನು ಅದನ್ನು ಕೆಟ್ಟದ್ದು ಎನ್ನುತ್ತೇನೆ. ಒಳ್ಳೆಯದು ಕೆಟ್ಟದ್ದು ಎರಡೂ ಕೂಡ ಸಾಪೇಕ್ಷ ಜಗತ್ತಿಗೆ ಸೇರಿದವುಗಳು. ನಾವು ಸೂಚಿಸುವ ನಿರ್ಗುಣ ದೇವರು ಸಾಪೇಕ್ಷ ದೇವರಲ್ಲ. ಆದಕಾರಣ ನಾವು ಅವನನ್ನು ಒಳ್ಳೆಯವನು ಅಥವಾ ಕೆಟ್ಟವನು ಎಂದು ಹೇಳಲಾರೆವು. ನಿರ್ಗುಣ ದೇವರು ಇವನ್ನು ಮೀರಿದ್ದಾನೆ. ಏಕೆಂದರೆ ಅವನು ಒಳ್ಳೆಯವನೂ ಅಲ್ಲ, ಕೆಟ್ಟವನೂ ಅಲ್ಲ, ಆದರೆ ಒಳ್ಳೆಯದಾದರೊ ಕೆಟ್ಟದಕ್ಕಿಂತ ಆ ದೇವರಿಗೆ ಹತ್ತಿರವಾದ ಅಭಿವ್ಯಕ್ತಿ.

ಅಂತಹ ನಿರ್ಗುಣ ದೇವರನ್ನು ಒಪ್ಪಿಕೊಳ್ಳುವುದರ ಪರಿಣಾಮ ಏನಾಯಿತು? ಇದರಿಂದ ನಮಗೇನು ಲಾಭ? ಮಾನವಜೀವನದಲ್ಲಿ ಧರ್ಮಕ್ಕೆ ಏನಾದರೂ ಬೆಲೆ ಇದೆಯೆ? ಅದು ನಮ್ಮನ್ನು ಸಮಾಧಾನ ಮಾಡಬಲ್ಲುದೆ? ಅದು ನಮ್ಮನ್ನು ರಕ್ಷಿಸಬಲ್ಲುದೆ? ಯಾರಿಗಾದರೂ ಸಹಾಯಕ್ಕಾಗಿ ಪ್ರಾರ್ಥಿಸುವ ಮಾನವನ ಸ್ವಭಾವ ಏನಾಗುವುದು? ಅದೆಲ್ಲಾ ಇರುವುದು. ಸಗುಣ ದೇವರು ಇರುವನು. ಆದರೆ ಅವನು ಮತ್ತೂ ಭದ್ರವಾದ ತಳಹದಿಯ ಮೇಲೆ ನೆಲಸುವನು. ಅವನು ನಿರ್ಗುಣದ ಸಹಾಯದಿಂದ ಮತ್ತೂ ಬಲಶಾಲಿಯಾಗುವನು. ನಿರ್ಗುಣವಿಲ್ಲದೆ ಸಗುಣ ಇರಲಾರದು ಎಂಬುದನ್ನು ನಾವು ನೋಡಿದೆವು. ಈ ವಿಶ್ವದಿಂದ ಸಂಪೂರ್ಣ ಬೇರೆಯಾದ ದೇವರು, ಕೇವಲ ತನ್ನ ಇಚ್ಚೆಯಿಂದ, ಶೂನ್ಯದಿಂದ ಜಗತ್ತನ್ನು ಸೃಷ್ಟಿಸಿದ ಎಂದರೆ ಅದನ್ನು ಸಮರ್ಥಿಸಲು ಆಗುವುದಿಲ್ಲ. ಇಂತಹ ಸ್ಥಿತಿ ಇರಲಾರದು. ನಿರ್ಗುಣವನ್ನು ನಾವು ಚೆನ್ನಾಗಿ ತಿಳಿದುಕೊಂಡರೆ, ಸಗುಣದ ಭಾವನೆಯೂ ಇರುತ್ತದೆ. ವೈವಿಧ್ಯದಿಂದ ಕೂಡಿರುವ ವಿಶ್ವವೆಲ್ಲ ಆ ನಿರ್ಗುಣದ ಹಲವು ಅಭಿವ್ಯಕ್ತಿಗಳಷ್ಟೆ. ನಾವು ಅದನ್ನು ಪಂಚೇಂದ್ರಿಯಗಳ ಮೂಲಕ ನೋಡಿದಾಗ ಜಡವಸ್ತು ಎನ್ನುತ್ತೇವೆ. ಯಾರಿಗಾದರೂ ಪಂಚೇಂದ್ರಿಯಗಳಿಗಿಂತ ಬೇರೊಂದು ಇಂದ್ರಿಯವಿದ್ದರೆ ಅವನು ಆ ದೃಷ್ಟಿಯಿಂದ ಜಗತ್ತನ್ನು ನೋಡುವನು. ಯಾರಿಗಾದರೂ ವಿದ್ಯುತ್ ಶಕ್ತಿಯನ್ನು ಅರಿಯುವ ಇಂದ್ರಿಯವಿದ್ದರೆ ಅವನು ಜಗತ್ತನ್ನು ಆ ದೃಷ್ಟಿಯಿಂದ ನೋಡುವನು. ಆ ಒಂದರ ಹಲವು ಅಭಿವ್ಯಕ್ತಿಗಳು ಇವೆಲ್ಲ. ಈ ಪ್ರಪಂಚದ ಹಲವು ಭಾವನೆಗಳೆಲ್ಲ ಅದರ ಭಿನ್ನ ಭಿನ್ನ ಅಭಿವ್ಯಕ್ತಿಗಳು. ಮಾನವನ ಬುದ್ದಿ ಗ್ರಹಿಸಬಲ್ಲ ಶ್ರೇಷ್ಠ ನಿರ್ಗುಣಬ್ರಹ್ಮನ ಭಾವನೆಯೇ ಸಗುಣ. ಆದಕಾರಣ ಸಗುಣ ದೇವರು ಈ ಮೇಜಿನಷ್ಟೇ ಈ ಪ್ರಪಂಚದಷ್ಟೇ ಸತ್ಯ, ಅದಕ್ಕಿಂತ ಹೆಚ್ಚೇನೂ ಅಲ್ಲ. ಇದು ನಿರಪೇಕ್ಷ ಸತ್ಯವಲ್ಲ. ಅಂದರೆ ಸಗುಣ ದೇವರೇ ನಿರ್ಗುಣದಿಂದ ಬಂದುದಾದುದರಿಂದ ಅದು ಸತ್ಯವಾಗಬೇಕಾಯಿತು. ನಾನು ಒಬ್ಬ ಮಾನವ ಎನ್ನುವುದು ಒಂದು ದೃಷ್ಟಿಯಿಂದ ಸತ್ಯವೂ ಮತ್ತೊಂದು ದೃಷ್ಟಿಯಿಂದ ಅಸತ್ಯವೂ ಆಗಿರುವ ಹಾಗೆ. ನನ್ನನ್ನು ನೀವು ಹೇಗೆ ತಿಳಿದಿರುವಿರೊ ಹಾಗೆ ನಾನಿಲ್ಲ. ಈ ಒಂದು ಉದಾಹರಣೆಯಿಂದ ನೀವು ಬಗೆಹರಿಸಿಕೊಳ್ಳಬಹುದು. ನಾನು ಯಾವ ವ್ಯಕ್ತಿ ಎಂದು ನೀವು ಭಾವಿಸಿರುವಿರೊ ಆ ವ್ಯಕ್ತಿ ನಾನಲ್ಲ. ಇದನ್ನು ಯುಕ್ತಿ ಪೂರ್ವಕ ಬೇಕಾದರೆ ನೀವು ತರ್ಕಿಸಿ ಸಮಾಧಾನ ಪಡಬಹುದು. ಏಕೆಂದರೆ ಬೆಳಕು ಮತ್ತು ನನ್ನಲ್ಲಿ ಆಗುತ್ತಿರುವ ಹಲವು ಬಗೆಯ ಸ್ಪಂದನಗಳು ಅಥವಾ ಬಾಹ್ಯಪ್ರಕೃತಿಯ ವಾತಾವರಣ ಮತ್ತು ಬದಲಾವಣೆ ಇವುಗಳೆಲ್ಲ ನಾನು ಈಗ ಹೇಗೆ ಕಾಣುವೆನೂ ಹಾಗೆ ಇರುವಂತೆ ಮಾಡಿವೆ. ಇವುಗಳಲ್ಲಿ ಯಾವುದಾದರೊಂದು ಬದಲಾದರೂ ನಾನು ಬೇರೆ ವ್ಯಕ್ತಿಯಾಗುವೆನು. ನೀವು ಒಬ್ಬ ವ್ಯಕ್ತಿಯ ಫೋಟೋವನ್ನು ಬೇರೆ ಬೇರೆ ಬೆಳಕಿನಲ್ಲಿ ತೆಗೆದುಕೊಂಡು ಅದನ್ನು ಹೋಲಿಸಿ ನೋಡಬಹುದು. ನಿಮ್ಮ ಪಂಚೇಂದ್ರಿಯಗಳ ದೃಷ್ಟಿಯಿಂದ ನೋಡಿದಾಗ ನಾನು ಈ ರೀತಿ ಕಾಣುತ್ತಿರುವೆನು. ಆದರೂ ಇವುಗಳೆಲ್ಲ ಯಾವುದೊ ಒಂದು ಬದಲಾಗದ ವಸ್ತುವಿನ ಬೇರೆ ಬೇರೆ ಅಭಿವ್ಯಕ್ತಿಗಳಂತೆ ತೋರುತ್ತಿವೆ. ಅವುಗಳ ಹಿಂದೆ ಅವ್ಯಕ್ತವಾದ ನಾನೆಂಬುದಿದೆ. ಅದರ ಸಾವಿರಾರು ರೂಪಗಳು ಈ ತೋರಿಕೆಯ ನಾನೆಂಬುದು. ನಾನು ಹಿಂದೆ ಮಗುವಾಗಿದ್ದೆ,\break ಯುವಕನಾಗಿದ್ದೆ, ಈಗ ವೃದ್ಧನಾಗುತ್ತಿರುವೆನು. ನನ್ನ ಜೀವನದಲ್ಲಿ ಪ್ರತಿಯೊಂದು ದಿನವೂ ನನ್ನ ದೇಹ ಮತ್ತು ಆಲೋಚನೆಗಳು ಬದಲಾಗುತ್ತಿವೆ. ಇಷ್ಟೊಂದು ಬದಲಾವಣೆಯಾದರೂ ಇವುಗಳ ಒಟ್ಟು ಮೊತ್ತ ಯಾವಾಗಲೂ ಒಂದೇ ಸಮನಾಗಿರುವುದು. ಅದೇ ನನ್ನ ಅವ್ಯಕ್ತ ಸ್ವಭಾವ. ಅದರಲ್ಲಿ ಈ ವಿವಿಧ ಅಭಿವ್ಯಕ್ತಿಗಳೆಲ್ಲ ಅಂಶಗಳು ಮಾತ್ರ.

ಇದರಂತೆಯೇ ಈ ವಿಶ್ವದ ಮೊತ್ತ ಸ್ಥಾಣು ಎಂಬುದು ನಮಗೆ ಗೊತ್ತಿದೆ. ಆದರೆ ವಿಶ್ವದಲ್ಲಿರುವುದೆಲ್ಲ ಚಲಿಸುತ್ತಿರುವುದು. ಪ್ರತಿಯೊಂದೂ ನಿರಂತರ ಚಲಿಸುತ್ತಿದೆ. ಪ್ರತಿಯೊಂದೂ ವಿಕಾರವಾಗುತ್ತಿದೆ, ಚಲಿಸುತ್ತಿದೆ. ಆದರೂ ವಿಶ್ವವು ಸಮಷ್ಟಿ ದೃಷ್ಟಿಯಿಂದ ಚಲಿಸಲಾರದು ಎಂದು ನಮಗೆ ತೋರುವುದು. ಏಕೆಂದರೆ ಚಲನೆ ಸಾಪೇಕ್ಷವಾದುದು \enginline{(relative);} ನಾನು ಚಲಿಸುವುದು ಚಲಿಸದೆ ಇರುವ ಕುರ್ಚಿಯೊಡನೆ ಹೋಲಿಸಿದಾಗ. ಒಂದು ಚಲಿಸ ಬೇಕಾದರೆ ಎರಡು ವಸ್ತುಗಳಾದರೂ ಇರಬೇಕು. ಈ ವಿಶ್ವವನ್ನೇ ಒಂದು ಎಂದು ಭಾವಿಸಿದರೆ, ಅದು ತನಗೆ ಹೊರಗೆ ಇರುವ ಯಾವುದರ ಸಂಬಂಧದಿಂದ ಚಲಿಸಬಲ್ಲದು? ಆದಕಾರಣ ನಿರಪೇಕ್ಷವು ಅವಿಕಾರಿಯಾದುದು, ಅಚಲವಾದುದು. ಮತ್ತು ಚಲನವಲನಗಳು ಬದಲಾವಣೆಗಳೆಲ್ಲ ಸಾಪೇಕ್ಷತೆಯಿಂದ ಮತ್ತು ಮಿತಿಯಿಂದ ಕೂಡಿದ ಜಗತ್ತಿಗೆ ಅನ್ವಯಿಸುವುವು. ಆ ಸಮಷ್ಟಿ ನಿರ್ಗುಣವಾದುದು. ಅದರಲ್ಲಿ ಕ್ಷುದ್ರತಮ ಕೀಟದಿಂದ ಹಿಡಿದು, ನಾವು ಬಾಗಿ ಪ್ರಾರ್ಥಿಸುವ ಈ ವಿಶ್ವದ ಸೃಷ್ಟಿಕರ್ತನೂ ಒಡೆಯನೂ ಆದ ಸಗುಣ ಈಶ್ವರನವರೆಗೂ ಎಲ್ಲರೂ ಸೇರಿದ್ದಾರೆ. ಇಂತಹ ಈಶ್ವರನನ್ನು ಯುಕ್ತಿಪೂರ್ವಕ ಬೇಕಾದರೆ ನಾವು ಸಮರ್ಥಿಸ ಬಹುದು. ನಿರಾಕಾರದ ಶ್ರೇಷ್ಠ ಅಭಿವ್ಯಕ್ತಿಯೇ ಈಶ್ವರ ಎಂದು ನಾವು ವಿವರಿಸಬಹುದು. ನಾವು ನೀವುಗಳೆಲ್ಲ ಅತಿ ಅಲ್ಪ ಅಭಿವ್ಯಕ್ತಿಗಳು; ಸಗುಣ ಈಶ್ವರನೇ ನಾವು ಊಹಿಸಬಹುದಾದ ಶ್ರೇಷ್ಠ ಕಲ್ಪನೆ. ನಾನೋ ನೀವೋ ಆ ದೇವರಾಗಲಾರೆವು. ವೇದಾಂತ, ನಾನು ನೀವು ಎಲ್ಲರೂ ಬ್ರಹ್ಮ ಎಂದು ಹೇಳಿದರೆ ಅದು ಸಗುಣದೇವರಿಗೆ ಸಂಬಂಧಿಸಿದುದಲ್ಲ. ಉದಾಹರಣೆಗಾಗಿ ಒಂದು ಜೇಡಿ ಮಣ್ಣಿನ ರಾಶಿಯಿಂದ ದೊಡ್ಡದೊಂದು ಆನೆಯನ್ನು ಮಾಡಿರುವರು, ಅದೇ ಮಣ್ಣಿನ ರಾಶಿಯಿಂದ ಸಣ್ಣದೊಂದು ಇಲಿಯನ್ನು ಮಾಡಿರುವರು. ಮಣ್ಣಿನ ಇಲಿ ಎಂದಾದರೂ ಮಣ್ಣಿನ ಆನೆಯಾಗಬಲ್ಲದೆ? ಆದರೆ ಅವೆರಡನ್ನೂ ನೀರಿಗೆ ಹಾಕಿದರೆ ಅವೆರಡೂ ಜೇಡಿಮಣ್ಣಾಗುವುವು. ಮಣ್ಣಿನ ದೃಷ್ಟಿಯಿಂದ ಅವೆರಡೂ ಒಂದು, ಆದರೆ ಇಲಿಯ ಮತ್ತು ಆನೆಯ ದೃಷ್ಟಿಯಿಂದ ಎಂದೆಂದಿಗೂ ವ್ಯತ್ಯಾಸ ಇದ್ದೇ ತೀರುವುದು. ಅನಂತವಾದ ನಿರ್ಗುಣವೇ ಜೇಡಿಮಣ್ಣಿನಂತೆ. ನಾವು ಪರಮೇಶ್ವರ ಇಬ್ಬರೂ ಒಂದೇ. ಆದರೆ ಅಭಿವ್ಯಕ್ತಿಯ ದೃಷ್ಟಿಯಿಂದ ಮಾನವರಾದ ನಾವು ಅವನ ಭಕ್ತರು, ಅವನ ನಿತ್ಯ ಭೃತ್ಯರು. ನೋಡಿ, ಸಾಕಾರ ಈಶ್ವರ ಹಾಗೆಯೇ ನಿಲ್ಲುವನು. ಸಾಪೇಕ್ಷ ಪ್ರಪಂಚದಲ್ಲಿ ಎಲ್ಲವೂ ಹಾಗೆಯೇ ಉಳಿಯುವುದು; ಧರ್ಮ ಮತ್ತೂ ಭದ್ರವಾದ ತಳಹದಿಯ ಮೇಲೆ ಪ್ರತಿಷ್ಠಿತವಾಗುವುದು. ಆದಕಾರಣ ನಾವು ಸಗುಣವನ್ನು ಅರಿಯಬೇಕಾದರೆ ಮೊದಲು ನಿರ್ಗುಣವನ್ನು ಅರಿಯಬೇಕು.

ಒಂದನ್ನು \enginline{(particular)} ಅದರ ಸಾಮಾನ್ಯದ \enginline{(general)} ಮೂಲದಿಂದ ಮಾತ್ರ ನಾವು ಅರಿಯಬಹುದು ಎಂಬುದನ್ನು ತರ್ಕದ ನಿಯಮವು ತೋರಿಸುತ್ತದೆ ಎಂಬುದು ಸ್ಪಷ್ಟವಾಯಿತು. ದೇವರಿಂದ ಮಾನವನವರೆಗೆ ಇರುವ ವ್ಯಷ್ಟಿ ಭಾವನೆಗಳೆಲ್ಲ ಅತಿ ಶ್ರೇಷ್ಠ ಸಾಮಾನ್ಯ ವರ್ಗಿಕರಣವಾದ ನಿರ್ಗುಣ ಬ್ರಹ್ಮನ ಮೂಲಕ ಮಾತ್ರ ನಮಗೆ ಅರಿವಾಗುವುದು. ಪ್ರಾರ್ಥನೆಗಳು ಉಳಿಯುವುವು, ಆದರೆ ಅವಕ್ಕೆ ಮತ್ತೂ ಒಳ್ಳೆಯ ಅರ್ಥ ಬರುವುದು. ನಿಷ್ಪ್ರಯೋಜಕವಾದ ಆಸೆಗಳಿಗೆ ಸಂಬಂಧಿಸಿದ, ಕೆಲಸಕ್ಕೆ ಬಾರದ ಪ್ರಾರ್ಥನೆಗಳೆಲ್ಲ ಬಹುಶಃ ಹೋಗಬೇಕಾಗುವುದು. ಯುಕ್ತಿಗರ್ಭಿತವಾದ ಯಾವ ಧರ್ಮವೂ ಭಗವಂತನಿಗೆ ನಮ್ಮ ಆಸೆಯ ಕೋರಿಕೆಗಳನ್ನು ಅರ್ಪಿಸಲು ಅವಕಾಶವೀಯುವುದಿಲ್ಲ. ಉಪದೇವತೆಗಳಿಗೆ ಬೇಕಾದರೆ ನಾವು ಇದಕ್ಕಾಗಿ ಪ್ರಾರ್ಥಿಸಬಹುದು. ಇದು ಸ್ವಭಾವಸಹಜವಾಗಿಯೇ ಇದೆ. ರೋಮನ್ ಕ್ಯಾಥೋಲಿಕರು ಸಾಧುಸಂತರನ್ನು ಇದಕ್ಕಾಗಿ ಪ್ರಾರ್ಥಿಸುವರು. ಇದು ಒಳ್ಳೆಯದು. ಆದರೆ ದೇವರಲ್ಲಿ ಇದನ್ನು ಪ್ರಾರ್ಥಿಸುವುದು ಮೌಡ್ಯ, ದೇವರನ್ನು ಸ್ವಲ್ಪ ಗಾಳಿ ಕೊಡು, ಮಳೆ ಕೊಡು, ತೋಟದಲ್ಲಿ ಹಣ್ಣು ಬೆಳೆಯುವಂತೆ ಮಾಡು ಎಂದು ಮುಂತಾಗಿ ಪ್ರಾರ್ಥಿಸುವುದು ಅಸ್ವಾಭಾವಿಕ. ನಮ್ಮಂತೆಯೇ ಅಲ್ಪಾತ್ಮರಾಗಿದ್ದ ಸಾಧುಸಂತರು ಬೇಕಾದರೆ ಇದಕ್ಕೆ ಸಹಾಯಮಾಡಬಹುದು. ಆದರೆ ವಿಶ್ವೇಶ್ವರನಲ್ಲಿ ನಮ್ಮ ಅಲ್ಪ ಬಯಕೆಗಳನ್ನು ಕೋರುವುದಕ್ಕೆ, “ನನ್ನ ತಲೆನೋವು ಗುಣವಾಗಲಿ'' ಎಂದು ಮುಂತಾದ ಕೋರಿಕೆಗಳನ್ನು ಸಲ್ಲಿಸುವುದಕ್ಕೆ, ಅರ್ಥವಿಲ್ಲ. ಪ್ರಪಂಚದಲ್ಲಿ ಕೋಟ್ಯಂತರ ಜೀವಿಗಳು ಕಾಲವಾಗಿರುವರು, ಅವರೆಲ್ಲ ಇಲ್ಲಿರುವರು. ಅವರೇ ದೇವದೂತರು, ಉಪದೇವತೆಗಳು ಆಗಿರುವರು. ಅವರು ಬೇಕಾದರೆ ನಿಮ್ಮ ಸಹಾಯಕ್ಕೆ ಬರಲಿ. ಭಗವಂತನನ್ನು ಇದಕ್ಕಾಗಿ ಬೇಡಕೂಡದು. ಅವನ ಸಮೀಪಕ್ಕೆ ನಾವು ಮತ್ತೂ ಶ್ರೇಷ್ಠವಾದ ವಸ್ತುಗಳನ್ನು ಪಡೆಯಲು ಹೋಗಬೇಕು. ಗಂಗಾನದಿಯ ತೀರದಲ್ಲಿ ಕುಳಿತು ನೀರಿಗಾಗಿ ಒಂದು ಸಣ್ಣ ಬಾವಿಯನ್ನು ತೋಡುವವನು ಮೂಢ, ವಜ್ರದ ಗಣಿಯ ಸಮೀಪದಲ್ಲಿ ಕುಳಿತು ಗಾಜಿನ ಚೂರಿಗಾಗಿ ಅಲೆಯುವವನು ಮೂಢ.

ಈ ಎಲ್ಲಾ ದಯೆಗೆ ಮತ್ತು ಪ್ರೀತಿಗೆ ಪಿತನಾದ ಭಗವಂತನ ಸಮೀಪದಲ್ಲಿ ಕೆಲಸಕ್ಕೆ ಬಾರದ ಪ್ರಾಪಂಚಿಕ ವಸ್ತುಗಳನ್ನು ಬೇಡಿದರೆ ನಾವು ಮೂರ್ಖರಾಗುವೆವು. ಆದಕಾರಣ ನಾವು ಅವನ ಬಳಿಗೆ ಜ್ಞಾನ ಭಕ್ತಿ ಪ್ರೀತಿ ಇವುಗಳನ್ನು ಬೇಡುವುದಕ್ಕಾಗಿ ಹೊಗೋಣ. ಆದರೆ ಎಲ್ಲಿಯವರೆಗೆ ನಮ್ಮಲ್ಲಿ ದೌರ್ಬಲ್ಯ ಇದೆಯೊ, ಮತ್ತೊಬ್ಬರ ಆಶ್ರಯದಲ್ಲಿ ಬೆಳೆಯಬೇಕೆಂಬ ದಾಸ್ಯಸ್ವಭಾವವಿದೆಯೊ, ಅಲ್ಲಿಯವರೆಗೆ ಇಂತಹ ಅಲ್ಪ ಪ್ರಾರ್ಥನೆಗಳು, ಸಗುಣ ದೇವರ ಪೂಜೆ ಇವು ಇರುವುವು. ಆದರೆ ಯಾರು ಬಹಳ ಮುಂದುವರಿದು ಹೋಗಿರುವರೋ, ಅವರು ಇಂತಹ ಅಲ್ಪ ಸಹಾಯವನ್ನು ಲೆಕ್ಕಿಸುವುದಿಲ್ಲ. ತಮಗಾಗಿ ಆಶಿಸುವ, ತಮಗಾಗಿ ಬೇಡುವ ಸ್ಥಿತಿಯನ್ನು ಅವರು ಮರತೇ ಹೋಗುವರು; ಅವನ್ನು ಮೀರಿ ಹೋಗುವರು. ಅವರಲ್ಲಿ ಬಹಳ ಮುಖ್ಯವಾದ ಭಾವನೆಯೆ, “ನಾನಲ್ಲ ನೀನು, ಸಹೋದರನೆ'' ಎಂಬುದು. ಅವರೇ ನಿರ್ಗುಣವನ್ನು ಉಪಾಸಿಸಲು ಯೋಗ್ಯರು. ನಿರ್ಗುಣ ಆರಾಧನೆ ಎಂದರೇನು? “ದೇವರೆ, ನಾನು ನಿರ್ಗತಿಕ, ನನ್ನ ಮೇಲೆ ದಯೆಯನ್ನು ಇಡು'' ಎಂಬ ದಾಸ್ಯಭಾವವಲ್ಲ. ಇಂಗ್ಲಿಷಿಗೆ ಅನುವಾದವಾದ ಪಾರ್ಸಿಭಾಷೆಯ ಕವನ ನಿಮಗೆ ಗೊತ್ತು: “ನಾನು ನನ್ನ ಪ್ರಿಯತಮೆಯನ್ನು ನೋಡಲು ಬಂದೆ. ಬಾಗಿಲು ಹಾಕಿತ್ತು. ನಾನು ಬಾಗಿಲನ್ನು ತಟ್ಟಿದೆ. ಒಳಗಿನಿಂದ `ನೀನಾರು' ಎಂಬ ಧ್ವನಿ ಬಂದಿತು. ನಾನು ಇಂತಹವನು ಎಂದೆ. ಬಾಗಿಲು ತೆರೆಯಲಿಲ್ಲ. ನಾನು ಎರಡನೆಯ ವೇಳೆ ಬಂದು ತಟ್ಟಿದೆ. ನನ್ನನ್ನು ಅದೇ ಪ್ರಶ್ನೆಯನ್ನು ಕೇಳಿದರು, ನಾನು ಹಿಂದಿನಂತೆಯೇ ಉತ್ತರವಿತ್ತೆ. ಆಗಲೂ ಬಾಗಿಲು ತೆರೆಯಲಿಲ್ಲ. ನಾನು ಮೂರನೆಯ ವೇಳೆ ಬಂದಾಗಲೂ ಅದೇ ಪ್ರಶ್ನೆಯನ್ನು ಹಾಕಿದರು; `ಪ್ರಿಯಳೇ, ನಾನು ನೀನೇ' ಎಂದೆ. ಆಗ ಬಾಗಿಲು ತೆರೆಯಿತು." ಸತ್ಯದ ಮೂಲಕ ನಿರ್ಗುಣ ಭಗವಂತನನ್ನು ಆರಾಧಿಸಬೇಕು. ಸತ್ಯವೆಂದರೇನು? ನಾನೆ ಅವನು ಎಂಬುದು. ನಾನು ನೀನಲ್ಲ ಎಂಬುದು ಸುಳ್ಳು. ನಾನು ನಿನ್ನಿಂದ ಬೇರೆ ಎಂದು ಹೇಳುವುದು ಸುಳ್ಳು, ಭಯಂಕರ ಸುಳ್ಳು. ನಾನು ಪ್ರಪಂಚದೊಡನೆ ಒಂದು, ಜನ್ಮತಃ ಒಂದು. ನಾನು ಪ್ರಪಂಚದೊಡನೆ ಒಂದು ಎನ್ನುವುದು ಸ್ವತಃಸಿದ್ಧವಾದುದು. ನನ್ನ ಸುತ್ತಲಿರುವ ಗಾಳಿಯೊಡನೆ ನಾನು ಒಂದು, ಶಾಖದೊಡನೆ ನಾನು ಒಂದು, ಬೆಂಕಿಯೊಡನೆ ನಾನು ಒಂದು, ಈ ಪ್ರಪಂಚವೆಂದು ತಪ್ಪಾಗಿ ತಿಳಿದಿರುವ ವಿಶ್ವಾತ್ಮನೊಂದಿಗೆ ನಾನು ಒಂದು. ಹೃದಯದಲ್ಲಿ ನಿತ್ಯ ಸಾಕ್ಷಿಯಾಗಿರುವವನು ಅವನೆ, ಮತ್ತೊಂದು ಅಲ್ಲ. ಅವನೇ ಪ್ರತಿಯೊಂದು ಹೃದಯದಲ್ಲಿಯೂ `ನಾನು' ಎನ್ನುತ್ತಿರುವನು. ಅವನೇ, ಮರಣಾತೀತ, ನಿದ್ರಾತೀತ, ಚಿರಜಾಗೃತ, ಅಮೃತಸ್ವರೂಪ; ಅವನ ಮಹಿಮೆ ಎಂದಿಗೂ ಕುಗ್ಗುವುದಿಲ್ಲ; ನಾನು ಅಂತಹವನೊಂದಿಗೆ ಒಂದು.

ನಿರ್ಗುಣದ ಉಪಾಸನೆ ಎಂದರೆ ಇದೇ. ಇದರ ಪರಿಣಾಮವೇನು? ಇದರಿಂದ ಇಡೀ ಮಾನವನ ಬಾಳೇ ಪರಿವರ್ತನೆ ಹೊಂದುವುದು. ಈ ಜೀವನದಲ್ಲಿ ಅತ್ಯಾವಶ್ಯಕವಾಗಿ ಬೇಕಾಗಿರುವುದು ಶಕ್ತಿ. ಯಾವುದನ್ನು ಪಾಪ, ದುಃಖ ಎನ್ನುವೆವೊ ಅದಕ್ಕೆ ಒಂದು ಕಾರಣವಿದೆ. ಅದೇ ನಮ್ಮ ದೌರ್ಬಲ್ಯ. ದೌರ್ಬಲ್ಯದಿಂದ ಅಜ್ಞಾನ, ಅಜ್ಞಾನದಿಂದ ದುಃಖ. ನಿರ್ಗುಣದ ಉಪಾಸನೆಯಿಂದ ನಾವು ಶಕ್ತಿವಂತರಾಗುವೆವು. ಆಗ ದುಃಖವನ್ನು ನೋಡಿ ನಗುವೆವು. ಆಗ ದುಷ್ಟನ ಕಾರ್ಯವನ್ನು ನೋಡಿ ನಗುವೆವು. ಆ ಕ್ರೂರ ವ್ಯಾಘ್ರ ಕೂಡ ತನ್ನ ಸ್ವಭಾವದ ಹಿಂದೆ ನನ್ನಾತ್ಮನನ್ನು ವ್ಯಕ್ತಪಡಿಸುವುದು. ಇದೇ ಅದರ ಪರಿಣಾಮ, ಯಾವ ಆತ್ಮ ಭಗವಂತನೊಂದಿಗೆ ಒಂದಾಗಿರುವುದೊ ಅದು ಮಹಾ ಬಲಶಾಲಿ, ಮತ್ತಾರೂ ಬಲಶಾಲಿಗಳಲ್ಲ. ನಿಮ್ಮ ಬೈಬಲ್ಲಿನಲ್ಲಿಯೇ ಬರುವ, ನಜರತ್ತಿನ ಜೀಸಸ್ಸಿನಲ್ಲಿದ್ದ ಶಕ್ತಿಗೆ ಯಾವುದು ಕಾರಣವೆಂದು ಊಹಿಸುತ್ತೀರಿ? ಅವನು ಕೃತಘ್ನರನ್ನು ನೋಡಿ ನಕ್ಕ, ಅವನನ್ನು ಕೊಲೆ ಮಾಡಲು ಸನ್ನದ್ದರಾಗಿ ಬಂದವರನ್ನು ಆಶೀರ್ವದಿಸಿದ. ಆ ಅಗಾಧವಾದ ಅನಂತ ಶಕ್ತಿ ಎಂತಹುದು ಎಂದು ಭಾವಿಸಿದಿರಿ? “ನಾನು ಮತ್ತು ನನ್ನ ತಂದೆ ಒಂದೆ" ಎಂಬುದೇ ಆ ಶಕ್ತಿ. “ತಂದೆ, ಹೇಗೆ ನಾನು ನಿನ್ನಲ್ಲಿ ಒಂದೊ, ಹಾಗೆಯೆ ಇವರನ್ನು ನಿನ್ನಲ್ಲಿ ಒಂದಾಗಿ ಮಾಡು'' ಎಂಬುದೇ ಆ ಪ್ರಾರ್ಥನೆ. ಇದೇ ನಿರ್ಗುಣದ ಉಪಾಸನೆ. ಪ್ರಪಂಚದಲ್ಲಿ ಒಂದಾಗಿ, ದೇವರಲ್ಲಿ ಒಂದಾಗಿ. ಈ ನಿರ್ಗುಣ ಬ್ರಹ್ಮನನ್ನು ಯಾರೂ ತೋರಬೇಕಾಗಿಲ್ಲ, ಇದಕ್ಕೆ ಯಾವ ಪ್ರಮಾಣಗಳೂ ಬೇಕಿಲ್ಲ. ಅವನು ನಮ್ಮ ಇಂದ್ರಿಯಗಳಿಗಿಂತ ಸಮೀಪನು. ಅವನು ನಮ್ಮ ಭಾವನೆಗಳಿಗಿಂತಲೂ ಸಮೀಪನು. ಅವನಿಂದ ಅವನ ಮೂಲಕ ಮಾತ್ರ ನಾವು ನೋಡುವುದು, ಆಲೋಚಿಸುವುದು.\break ಮತ್ಯಾವುದನ್ನು ನೋಡಬೇಕಾದರೂ ನಾನು ಮೊದಲು ಅವನನ್ನು ನೋಡಬೇಕು. ನಾನು ಈ ಗೋಡೆಯನ್ನು ನೋಡಬೇಕಾದರೆ ಮೊದಲು ಅವನನ್ನು ನೋಡಬೇಕು, ಅನಂತರ ಗೋಡೆಯನ್ನು ನೋಡಬೇಕು. ಏಕೆಂದರೆ ಅವನೇ ನಿತ್ಯ ಸಾಕ್ಷಿ. ಯಾರು ಯಾರನ್ನು ನೋಡುತ್ತಿರುವರು? ಅವನು ಇಲ್ಲಿ ನಮ್ಮ ಹೃದಯದ ಅಂತರಾಳದಲ್ಲಿ ಇರುವನು. ದೇಹ-ಮನಸ್ಸುಗಳು ಬದಲಾಗುವುವು; ಸುಖ ದುಃಖ, ಒಳ್ಳೆಯದು ಕೆಟ್ಟದ್ದು ಬಂದು ಹೋಗುವುವು; ದಿನ-ವರುಷಗಳು ಸಾಗಿ ಹೋಗುವುವು; ಜೀವನ ಬಂದುಹೋಗುವುದು. ಆದರೆ ಅವನು ನಾಶವಾಗುವುದಿಲ್ಲ. “ನಾನು, ನಾನು" ಎಂಬ ಧ್ವನಿಯೊಂದೇ ಸನಾತನ, ಎಂದೆಂದಿಗೂ ಬದಲಾಗದೆ ಇರುವುದು. ಅವನಲ್ಲಿ ಮತ್ತು ಅವನ ಮೂಲಕ ಮಾತ್ರ ನಾವು ಎಲ್ಲವನ್ನೂ ಅರಿಯುವೆವು. ಅವನಲ್ಲಿ ಅವನ ಮೂಲಕ ಮಾತ್ರ ನಾವು ಗ್ರಹಿಸುವೆವು, ಆಲೋಚಿಸುವೆವು, ಬದುಕಿರುವೆವು, ನಾವೇನಾಗಿರುವೆವೋ ಅದಾಗಿರುವೆವು. ಆ ವಿಭುವಾದ `ನಾನು' ಎಂಬುದನ್ನು ಅಲ್ಪವಾದ “ನಾನು'' ಎಂದು ತಪ್ಪಾಗಿ ತಿಳಿದಿರುವೆವು. ಅದು ಕೇವಲ ನನ್ನ “ನಾನು” ಮಾತ್ರವಲ್ಲ, ಅದು ನಿನ್ನ ಮತ್ತು ಎಲ್ಲರ “ನಾನು.” ಅದು, ದೇವತೆಗಳ, ಪ್ರಾಣಿಗಳ, ಅತಿ ಕ್ಷುದ್ರ ಕ್ರಿಮಿಕೀಟಗಳ `ನಾನು' ಆಗಿರುವುದು. ಆ `ನಾನೆ' ಕೊಲೆಪಾತಕಿಯಲ್ಲಿರುವುದು ಮತ್ತು ಸಾಧುವಿನಲ್ಲಿರುವುದು; ಶ‍್ರೀಮಂತನಲ್ಲಿರುವುದು ಮತ್ತು ಬಡವನಲ್ಲಿರುವುದು. ಅದೇ ಸ್ತ್ರೀಪುರುಷರಲ್ಲಿರುವುದು, ಅದೇ ಮನುಷ್ಯರಲ್ಲಿ ಮತ್ತು ಪ್ರಾಣಿಗಳಲ್ಲಿ ಇರುವುದು. ಕ್ಷುದ್ರತಮ ಕೀಟದಿಂದ ಹಿಡಿದು ಪರಿಪೂರ್ಣನಾದ ದೇವನವರೆಗೆ ಅದೇ ಪ್ರತಿಯೊಂದು ಆತ್ಮನಲ್ಲಿಯೂ ವಾಸಿಸಿರುವುದು, `ಸೋಽಹಂ, ಸೋಽಹಂ' ಎಂದು ನಿರಂತರ ಸಾರುತ್ತಿರುವುದು. ಚಿರಕಾಲವೂ ಅಲ್ಲಿರುವ ಆ ಧ್ವನಿಯನ್ನು ನಾವು ಅರ್ಥ ಮಾಡಿಕೊಂಡರೆ, ನಾವು ಈ ಪಾಠವನ್ನು ಕಲಿತರೆ, ಈ ಇಡೀ ವಿಶ್ವವು ತನ್ನ ರಹಸ್ಯವನ್ನು ನಮಗೆ ಕೊಟ್ಟಂತೆ ಆಯಿತು, ಪ್ರಕೃತಿ ತನ್ನ ಗುಪ್ತ ವಿಷಯಗಳನ್ನು ನಮಗೆ ಕೊಟ್ಟಂತೆ ಆಯಿತು. ನಾವು ಇನ್ನು ಮೇಲೆ ತಿಳಿದುಕೊಳ್ಳ ಬೇಕಾಗಿರುವುದು ಮತ್ತೇನೂ ಉಳಿದಿರುವುದಿಲ್ಲ. ಎಲ್ಲಾ ಧರ್ಮಗಳೂ ಅರಸುತ್ತಿರುವ ಸತ್ಯವನ್ನು ನಾವು ಕಾಣುವುದು ಹೀಗೆ. ಈ ಭೌತಿಕ ವಿಜ್ಞಾನಶಾಸ್ತ್ರಗಳೆಲ್ಲ ಬ್ರಹ್ಮಜ್ಞಾನದೊಂದಿಗೆ ಹೋಲಿಸಿದರೆ ಗೌಣ. ಈ ಪ್ರಪಂಚದಲ್ಲಿರುವ ವಿಶ್ವಾತ್ಮನೊಂದಿಗೆ ಯಾವುದು ನಮ್ಮನ್ನು ಒಂದುಗೂಡಿಸುವುದೊ ಅದೇ ನಿಜವಾದ ಜ್ಞಾನ.

