
\chapter[ಬಾಹ್ಯಪೂಜೆ]{ಬಾಹ್ಯಪೂಜೆ\protect\footnote{\engfoot{C.W, Vol. VI, P. 59}}}

\begin{center}
(೧೯೦೦ರ ಏಪ್ರಿಲ್ ೧೦ರಂದು ಸ್ಯಾ ನ್‌ಫ್ರಾನ್ಸಿಸ್ಕೋ ಪ್ರದೇಶದಲ್ಲಿ ನೀಡಿದ ಪ್ರವಚನ)
\end{center}

ಯಾರು ಬೈಬಲನ್ನು ಅಧ್ಯಯನ ಮಾಡಿರುವರೊ ಅವರಿಗೆ, ಯೊಹೂದ್ಯರ ಚರಿತ್ರೆಯಲ್ಲಿ ಯೊಹೂದ್ಯರ ಭಾವನೆಗಳೆಲ್ಲ ಎರಡು ಬಗೆಯ ಗುರುಗಳಿಂದ ಬಂದಿವೆ ಎಂಬುದು ಗೊತ್ತಾಗುವುದು. ಅವರೇ ಪೂಜಾರಿಗಳು ಮತ್ತು ಪ್ರವಾದಿಗಳು. ಪೂಜಾರಿಗಳು\break ಸಂಪ್ರದಾಯದ ಪ್ರತಿನಿಧಿಗಳು; ಪ್ರವಾದಿಗಳು ಪ್ರಗತಿಪರ ಶಕ್ತಿಯ ಪ್ರತಿನಿಧಿಗಳು, ಹೇಗೋ ಸಂಪ್ರದಾಯಬದ್ಧವಾದ ಬಾಹ್ಯಾಚಾರ ನುಸುಳಿಕೊಂಡು ಬರುವುದು, ಎಲ್ಲವನ್ನೂ ಕೇವಲ ಬಾಹ್ಯಾಚಾರ ಆಕ್ರಮಿಸುವುದು. ಇದು ಪ್ರತಿಯೊಂದು ದೇಶದಲ್ಲಿಯೂ ಮತ್ತು ಧರ್ಮದಲ್ಲಿಯೂ ನಿಜ. ಅನಂತರ ಕೆಲವರು ಹೊಸ ಮಹಾತ್ಮರು, ಹೊಸ ಆದರ್ಶಗಳೊಡನೆ ಬರುವರು. ಅವರು ಸಮಾಜಕ್ಕೆ ಹೊಸ ಭಾವನೆಗಳನ್ನು, ಆದರ್ಶಗಳನ್ನು ಬೋಧಿಸಿ, ಅವನ್ನು ಮುಂದಕ್ಕೆ ತರುವರು. ಕೆಲವು ತಲೆಮಾರುಗಳ ಒಳಗೆ ಅವರ ಅನುಯಾಯಿಗಳು ತಮ್ಮ ಗುರುಗಳಲ್ಲಿ ಎಷ್ಟು ಶ್ರದ್ಧೆಯುಳ್ಳವರಾಗುವರು ಎಂದರೆ, ಬೇರೇನನ್ನೂ ಅವರು ಗಮನಿಸುವುದೇ ಇಲ್ಲ. ಈಗಿನ ಕಾಲದ ಅತಿ ಪ್ರಗತಿಪರರಾದ ಬೋಧಕರೆ ಕೆಲವು ವರ್ಷಗಳ ನಂತರ ಕೇವಲ ಸಂಪ್ರದಾಯ ಬದ್ದರಾದ ಪುರೋಹಿತರಾಗುವರು. ಪ್ರಗತಿಗಾಮಿಗಳಾದ ವಿಚಾರಶೀಲರೆ ಕೆಲವು ಕಾಲದ ಮೇಲೆ, ತಮಗಿಂತ ಮುಂದುವರಿದು ಹೋಗುವವನನ್ನು ತಡೆಗಟ್ಟುವರು: ತಾವು ಎಷ್ಟು ದೂರ ಹೋಗಿರುವರೊ ಅದನ್ನು ಮೀರಿ ಯಾರೂ ಹೋಗದಂತೆ ಅವರು ನೋಡಿಕೊಳ್ಳುವರು. ಈಗಿರುವ ಪರಿಸ್ಥಿತಿಯಲ್ಲಿ ಅವರು ತೃಪ್ತರಾಗುವರು.

\vskip 2pt

ಪ್ರತಿಯೊಂದು ದೇಶದಲ್ಲಿಯೂ, ಪ್ರತಿಯೊಂದು ಧರ್ಮದಲ್ಲಿಯೂ ಪ್ರಗತಿಗಾಮಿಗಳಾದ ತತ್ತ್ವಗಳ ಮೂಲಕ ಕೆಲಸಮಾಡುವ ಶಕ್ತಿ ಧರ್ಮದ ಮೂಲಕ ವ್ಯಕ್ತವಾಗುವುದು. ಸಿದ್ದಾಂತಗಳು, ಶಾಸ್ತ್ರಗಳು, ಕೆಲವು ನಿಯಮಾವಳಿಗಳು, ಕೆಲವು ಚಲನವಲನಗಳು, ಕುಳಿತುಕೊಳ್ಳುವುದು, ಏಳುವುದು ಇವುಗಳೆಲ್ಲ ಪೂಜೆಯ ಪ್ರಪಂಚಕ್ಕೆ ಸೇರಿದವುಗಳೆ. ಜನಸಾಧಾರಣರಿಗೆ ಗ್ರಹಿಸಲು ಸುಲಭಸಾಧ್ಯವಾಗುವುದಕ್ಕಾಗಿ ಆಧ್ಯಾತ್ಮಿಕ ಪೂಜೆ ಸ್ಥೂಲರೂಪವನ್ನು ತಾಳುವುದು. ಯಾವ ದೇಶದಲ್ಲಿಯಾದರೂ ಆಗಲಿ ಬಹುಪಾಲು ಜನರು ಆತ್ಮವನ್ನು ಆತ್ಮದಂತೆ ಪೂಜಿಸಲಾರರು. ಇದು ಇನ್ನೂ ಸಾಧ್ಯವಾಗಿಲ್ಲ. ಎಲ್ಲರಿಗೂ ಸಾಧ್ಯವಾಗುವಂತಹ ಸಮಯ ಎಂದಾದರೂ ಬರುವುದೇನೊ ನನಗೆ ಗೊತ್ತಿಲ್ಲ. ಈ ನಗರದಲ್ಲಿ ಎಷ್ಟು ಸಾವಿರ ಜನ ದೇವರನ್ನು ಆಧ್ಯಾತ್ಮಿಕ ದೃಷ್ಟಿಯಿಂದ ಪೂಜಿಸಬಲ್ಲರು? ಬಹಳ ವಿರಳ ಅಂತಹವರ ಸಂಖ್ಯೆ. ಇದು ಅವರಿಗೆ ಸಾಧ್ಯವಿಲ್ಲ. ಅವರು ಇಂದ್ರಿಯ ಪ್ರಪಂಚದಲ್ಲಿ ಜೀವಿಸುತ್ತಿರುವರು. ಅವರಿಗೆ ನಿರ್ದಿಷ್ಟವಾದ ಸ್ಥೂಲವಾದ ಭಾವನೆಗಳನ್ನು ಕೊಡಬೇಕಾಗುವುದು. ದೈಹಿಕವಾಗಿ ಅವರಿಗೆ ಏನನ್ನಾದರೂ ಮಾಡಲು ಹೇಳಬೇಕು. ಇಪ್ಪತ್ತು ಸಲ ಏಳುವಂತೆ ಹೇಳಿ, ಇಪ್ಪತ್ತು ಸಲ ಕುಳಿತುಕೊಳ್ಳುವಂತೆ ಹೇಳಿ. ಒಂದು ಮೂಗಿನ ಮೂಲಕ ಉಸಿರನ್ನು ತೆಗೆದುಕೊಂಡು, ಮತ್ತೊಂದು ಮೂಗಿನ ಮೂಲಕ ಉಸಿರನ್ನು ಬಿಡಲು ಹೇಳಿ, ಅವರು ಅದನ್ನು ಅರ್ಥಮಾಡಿಕೊಳ್ಳಬಲ್ಲರು. ಆಧ್ಯಾತ್ಮಿಕ ಆದರ್ಶಗಳನ್ನು ಅವರು ಗ್ರಹಿಸಲೇ ಆರರು. ಇದು ಅವರ ತಪ್ಪಲ್ಲ. ನಿಮಗೆ ದೇವರನ್ನು ಆಧ್ಯಾತ್ಮಿಕ ದೃಷ್ಟಿಯಿಂದ ಪೂಜಿಸಲು ಸಾಧ್ಯವಾದರೆ ಒಳ್ಳೆಯದು! ಆದರೆ ಇದು ಸಾಧ್ಯವಾಗದ ಒಂದು ಕಾಲ ನಿಮಗೆ ಇತ್ತು. ಜನರು ಅನಾಗರಿಕರಾಗಿದ್ದರೆ, ಅವರ ಧಾರ್ಮಿಕಭಾವನೆ ಅನಾಗರಿಕವಾಗಿರುತ್ತದೆ; ಅವರ ಆಚಾರ ವ್ಯವಹಾರಗಳು ಕೂಡ ಸ್ಥೂಲವಾಗಿರುತ್ತವೆ. ಜನರು ಸುಸಂಸ್ಕೃತರಾಗಿದ್ದರೆ, ನಾಜೂಕಾಗಿದ್ದರೆ, ಅವರ ಆದರ್ಶಗಳು ಕೂಡ ಸೂಕ್ಷ್ಮವಾಗಿರುತ್ತವೆ. ಬಾಹ್ಯ ಆಚಾರಗಳು ಇರಲೇಬೇಕಾಗುವುದು. ಆದರೆ ಕಾಲಕ್ಕೆ ತಕ್ಕಂತೆ ಅವು ಬದಲಾಗುತ್ತವೆ.

\vskip 2pt

ಬಾಹ್ಯಸಾಧನಗಳನ್ನು, ಪೂಜಿಸುವುದನ್ನು ಮಹಮ್ಮದೀಯರಷ್ಟು ವಿರೋಧಿಸಿದವರು ಮತ್ತೊಬ್ಬರಿರಲಿಲ್ಲ. ಇದೊಂದು ಅತಿ ವಿಚಿತ್ರವಾದ ಸಂಗತಿ ಆಗಿದೆ. ಮಹಮ್ಮದೀಯರು ಧರ್ಮದಲ್ಲಿ ಚಿತ್ರಗಳನ್ನಾಗಲಿ, ವಿಗ್ರಹಗಳನ್ನಾಗಲಿ, ಸಂಗೀತವನ್ನಾಗಲಿ – ಎಲ್ಲವನ್ನೂ\break ಬಹಿಷ್ಕರಿಸಿದರು. ಇವೆಲ್ಲಾ ಬಾಹ್ಯಾಚಾರಕ್ಕೆ ಒಯ್ಯುವುವು ಎಂದು ಭಾವಿಸಿದರು.\break ಪುರೋಹಿತನು ಸಭಿಕರ ಎದುರಿಗೆ ನಿಲ್ಲುವುದಿಲ್ಲ. ಅವನೇನಾದರೂ ಹಾಗೆ ಮಾಡಿದರೆ, ಅವನು ತನ್ನನ್ನು ತಾನು ಪ್ರತ್ಯೇಕಿಸಿಕೊಂಡಂತೆ ಆಗುವುದು. ಅವನು ಅವರನ್ನು ಎದುರಿಗೆ ನೋಡದೆ ಇದ್ದರೆ ಯಾವ ಅಪಾಯವೂ ಇಲ್ಲ. ಆದರೂ ಮಹಮ್ಮದ್ ಕಾಲವಾಗಿ ಎರಡು ಶತಮಾನಗಳು ಆಗುವುದಕ್ಕೆ ಮುಂಚೆಯೇ ಮಹಾತ್ಮರ ಪೂಜೆ ತಲೆದೋರಿತು. ಇಲ್ಲಿ ಮಹಾತ್ಮನ ಬೆರಳಿದೆ! ಅಲ್ಲಿ ಅವನ ಚರ್ಮವಿದೆ! ಹೀಗೆಯೇ ಇವುಗಳೆಲ್ಲ ಪ್ರಾರಂಭವಾದುವು. ಬಾಹ್ಯಪೂಜೆ, ನಾವು ಸಾಗಿ ಹೋಗಬೇಕಾದ ಒಂದು ಹಂತ. ಆದಕಾರಣ ಅದಕ್ಕೆ ವಿರೋಧವಾಗಿ ನಾವು ಹೋರಾಡದೆ, ಪೂಜೆಯಲ್ಲಿ ಶ್ರೇಷ್ಠವಾದುದನ್ನು ತೆಗೆದುಕೊಂಡು ಅದರ ಹಿಂದೆ ಇರುವ ತತ್ತ್ವವನ್ನು ಪರಿಶೀಲನೆ ಮಾಡೋಣ.

\vskip 2pt

ಅತ್ಯಂತ ಕೆಳಮಟ್ಟದ ಪೂಜೆಯೇ ಮರಗಳ ಮತ್ತು ಕಲ್ಲಿನ ಪೂಜೆ, ಅಸಂಸ್ಕೃತರಾದ ಒಡ್ಡು ಜನರು, ಯಾವುದನ್ನಾದರೂ ತೆಗೆದುಕೊಂಡು ಅದಕ್ಕೆ ತಮ್ಮ ಕೆಲವು ಭಾವನೆಗಳನ್ನು ಬೆರಸುತ್ತಾರೆ. ಇದು ಅವರಿಗೆ ಸಹಾಯಮಾಡುವುದು. ಅವರು ಒಂದು ಮೂಳೆಯ ಚೂರನ್ನು ಅಥವಾ ಕಲ್ಲನ್ನು ಪೂಜಿಸಬಹುದು. ಯಾವ ಅನಾಗರಿಕ ಪೂಜಾವಿಧಾನದಲ್ಲಿಯೂ ಮನುಷ್ಯ, ಕಲ್ಲನ್ನು ಕಲ್ಲಿನಂತೆ ಅಥವಾ ಮರವನ್ನು ಮರದಂತೆ ಪೂಜಿಸಿಲ್ಲ. ನಿಮ್ಮ ವ್ಯವಹಾರ ಜ್ಞಾನದಿಂದ ನಿಮಗೆ ಇದು ಗೊತ್ತಾಗುವುದು: ವಿದ್ವಾಂಸರು ಕೆಲವು ವೇಳೆ, “ಜನರು ಕಲ್ಲನ್ನು ಪೂಜಿಸುವರು, ಮರವನ್ನು ಪೂಜಿಸುವರು” ಎಂದು ಹೇಳಬಹುದು. ಇವೆಲ್ಲ ಅಜ್ಞಾನ, ಮರಗಳ ಪೂಜೆ ಮಾನವಕೋಟಿ ಸಾಗಿಹೋದ ಒಂದು ಪೂಜಾಸ್ಥಿತಿ. ಎಂದೂ ಮನುಷ್ಯ ಅಧ್ಯಾತ್ಮವನ್ನಲ್ಲದೆ ಬೇರೆ ಯಾವುದನ್ನೂ ಪೂಜಿಸಲಿಲ್ಲ. ಅವನು ಆತ್ಮ. ಅಧ್ಯಾತ್ಮವನ್ನಲ್ಲದೆ ಬೇರೆ ಯಾವುದನ್ನೂ ಅವನು ಅರಿಯಲಾರ. ಅಧ್ಯಾತ್ಮವನ್ನು ಜಡವಸ್ತುವಿನಂತೆ ಪೂಜಿಸುವ ತಪ್ಪನ್ನು ದಿವ್ಯಾತ್ಮನಾದ ಮಾನವ ಎಂದೂ ಮಾಡಿಲ್ಲ. ಇಲ್ಲಿ ಮನುಷ್ಯ ಕಲ್ಲನ್ನು ಆತ್ಮನಂತೆ ಗ್ರಹಿಸಿದನು, ಮರವನ್ನು ಆತ್ಮನಂತೆ ಗ್ರಹಿಸಿದನು. ದೇವರ ಯಾವುದೊ ಅಂಶ ಕಲ್ಲಿನಲ್ಲಿ ಮತ್ತು ಮರದಲ್ಲಿ ಇದೆ ಎಂದು ಅವನು ಭಾವಿಸಿದನು. ಆ ಕಲ್ಲಿಗೆ, ಮರಕ್ಕೆ ಒಂದು ಆತ್ಮವಿದೆ ಎಂದು ಭಾವಿಸಿದನು.

\vskip 2pt

ಮರದ ಪೂಜೆ ಮತ್ತು ಹಾವಿನ ಪೂಜೆ ಯಾವಾಗಲೂ ಒಟ್ಟಿಗೆ ಹೋಗುವುದು. ಜ್ಞಾನದ ಮರವೊಂದಿದೆ. ಅದು ಯಾವಾಗಲೂ ಇರಲೇಬೇಕಾಗುವುದು. ಆ ಮರಕ್ಕೂ ಹಾವಿಗೂ ಹೇಗೊ ಸಂಬಂಧ ಬಂದು ಹೋಗಿದೆ. ಇದು ಅತ್ಯಂತ ಪುರಾತನವಾದ ಪೂಜೆ. ಅಲ್ಲಿ ಕೂಡ ಯಾವುದೋ ಒಂದು ಮರವನ್ನು, ಯಾವುದೋ ಒಂದು ಬಗೆಯ ಕಲ್ಲನ್ನು ಪೂಜಿಸುವರೆ ವಿನಃ ಎಲ್ಲಾ ಮರಗಳನ್ನು ಮತ್ತು ಕಲ್ಲುಗಳನ್ನು ಪೂಜಿಸುವುದಿಲ್ಲ.

\vskip 2pt

ಬಾಹ್ಯಪೂಜೆಯಲ್ಲಿ ಸ್ವಲ್ಪ ಉತ್ತಮಮಟ್ಟದ ಪೂಜೆಯೇ, ಪೂರ್ವಿಕರ ಮತ್ತು\break ದೇವತೆಗಳ ಮೂರ್ತಿಗಳ ಪೂಜೆ. ತೀರಿಹೋದ ಮನುಷ್ಯರ ವಿಗ್ರಹಗಳನ್ನು ಮತ್ತು ಕಾಲ್ಪನಿಕ ದೇವರುಗಳ ವಿಗ್ರಹಗಳನ್ನು ಮಾಡಿ ಜನರು ಅವುಗಳನ್ನು ಪೂಜಿಸುವರು. ಅದಕ್ಕಿಂತ ಉತ್ತಮವಾದುದೇ ಭಗವದ್ ಭಕ್ತರಾದ ತೀರಿಹೋದ ಉತ್ತಮ ಸ್ತ್ರೀ ಪುರುಷರ ವಿಗ್ರಹಗಳು. ಜನರು ಅವರ ಅವಶೇಷಗಳನ್ನು ಪೂಜಿಸುವರು. ಮಹಾತ್ಮರ ಮಹಿಮೆ ಅವರ ಅವಶೇಷಗಳಲ್ಲಿರುವುದೆಂದೂ, ಅದು ಅವರಿಗೆ ಸಹಾಯ ಮಾಡುವುದೆಂದೂ ಅವರು ಭಾವಿಸುವರು. ಮಹಾತ್ಮರ ಮೂಳೆಯನ್ನು ಮುಟ್ಟಿದರೆ ಅವರ ಕಾಯಿಲೆಯು ಗುಣವಾಗುವುದೆಂದು ಭಾವಿಸುವರು. ಮೂಳೆ ಹಾಗೆ ಗುಣ ಮಾಡುವುದೆಂದು ಅಲ್ಲ, ಅಲ್ಲಿ ವಾಸಿಸುವ ಮಹಾತ್ಮನ ಮಹಿಮೆ ಗುಣಮಾಡುವುದೆಂದು ಭಾವಿಸುವರು.

ಇವುಗಳೆಲ್ಲಾ ಕೆಳಮಟ್ಟದ ಪೂಜೆಗಳಾದರೂ ಪೂಜೆಗಳೇ. ನಾವುಗಳೆಲ್ಲ ಇವುಗಳ\break ಮೂಲಕ ಸಾಗಿಹೋಗಬೇಕಾಗಿದೆ. ಯುಕ್ತಿಯ ದೃಷ್ಟಿಯಿಂದ ಮಾತ್ರ ಅವು ಅಷ್ಟು ಶ್ರೇಷ್ಠವಲ್ಲ. ನಾವು ನಮ್ಮ ಮನಸ್ಸಿನಲ್ಲಿ ಅವುಗಳಿಂದ ಪಾರಾಗಲಾರೆವು, ನೀವು ಒಬ್ಬ ಮನುಷ್ಯನಿಂದ, ಅವನ ಮಹಾತ್ಮರನ್ನು ಮತ್ತು ವಿಗ್ರಹಗಳನ್ನೆಲ್ಲ ಕಸಿದುಕೊಂಡು ಅವನನ್ನು\break ದೇವಸ್ಥಾನಕ್ಕೆ ಹೋಗಲು ಬಿಡದೇ ಇದ್ದರೂ ಅವನು ಎಲ್ಲಾ ದೇವತೆಗಳನ್ನು ಕಲ್ಪಿಸಿಕೊಳ್ಳುವನು. ಅದನ್ನು ಅವನು ಮಾಡದೇ ವಿಧಿಯಿಲ್ಲ. ಎಂಭತ್ತು ವರ್ಷದ ಒಬ್ಬ ವೃದ್ದ, ತಾನು ದೇವರನ್ನು, ದೊಡ್ಡ ಗಡ್ಡದಿಂದ ಕೂಡಿದ, ಮುಗಿಲಿನ ಮೇಲೆ ಕುಳಿತಿರುವ ಒಬ್ಬ ವೃದ್ದನಂತೆ ಅಲ್ಲದೆ ಬೇರೆ ರೀತಿ ಕಲ್ಪಿಸಿಕೊಳ್ಳಲು ಸಾಧ್ಯವೇ ಇಲ್ಲವೆಂದು ಹೇಳಿದನು. ಇದು ಏನನ್ನು ತೋರುವುದು? ಅವನಿನ್ನೂ ಪೂರ್ಣ ವಿದ್ಯಾವಂತನಾಗಿರಲಿಲ್ಲ. ಅವನು ಮಾನವ ಆಕಾರದ ಮೂಲಕ ಅಲ್ಲದೆ ಬೇರೆ ಯಾವ ವಿಧವಾಗಿಯೂ ಏನನ್ನೂ ಕಲ್ಪಿಸಿಕೊಳ್ಳಲಾರದವನಾಗಿದ್ದನು.

ಇದಕ್ಕಿಂತ ಮೇಲಿನ ಮಟ್ಟದ ಮತ್ತೊಂದು ವಿಧದ ಪೂಜೆ ಇದೆ. ಅದೇ ಪ್ರತೀಕದ ಪೂಜೆ, ಅಲ್ಲಿ ಇನ್ನೂ ಆಕಾರಗಳು ಇವೆ. ಆದರೆ ಅವು ಮರಗಿಡಗಳಲ್ಲ, ಕಲ್ಲು ಅಲ್ಲ, ವಿಗ್ರಹವಲ್ಲ, ಯಾವ ಸಾಧುವಿನ ಅವಶೇಷಗಳೂ ಅಲ್ಲ. ಅವು ಕೇವಲ ಪ್ರತೀಕಗಳು. ಪ್ರಪಂಚದಲ್ಲೆಲ್ಲ ಎಷ್ಟೋ ಬಗೆಯ ಪ್ರತೀಕಗಳಿವೆ. ವೃತ್ತವು ಅನಂತಕ್ಕೆ ಒಂದು ಮಹಾ ಪ್ರತೀಕವಾಗಿದೆ. ಚೌಕದ ಪ್ರತೀಕವಿದೆ. ಪ್ರಖ್ಯಾತವಾದ ಶಿಲುಬೆಯ ಚಿಹ್ನೆಯಿದೆ. \enginline{S} ಮತ್ತು \enginline{Z} ಅಕ್ಷರಗಳನ್ನು ಹೋಲುವ ಒಂದನ್ನೊಂದು ಸಂಧಿಸುತ್ತಿರುವ ಸ್ವಸ್ತಿಕದ ಪ್ರತೀಕಗಳಿವೆ.

ಕೆಲವರು ಪ್ರತೀಕಗಳಲ್ಲಿ ಏನನ್ನೂ ನೋಡಬಾರದೆಂದು ಶಪಥ ತೊಡುವರು. ಮತ್ತೆ ಕೆಲವರು ಎಲ್ಲಾ ವಿಧವಾದ, ಅರ್ಥವಿಲ್ಲದ, ಏನೇನನ್ನೋ ಅಲ್ಲಿ ನೋಡಲು ಇಚ್ಛಿಸುವರು. ನೀವು ಅವರಿಗೆ ಸರಳವಾದ ಸತ್ಯಗಳನ್ನು ಹೇಳಿದರೆ ಅವರು ಅದನ್ನು ಒಪ್ಪಿಕೊಳ್ಳುವುದಿಲ್ಲ. ಮನುಷ್ಯನ ಸ್ವಭಾವವೇ ಇದು. ನೀವು ಅವರಿಗೆ ಎಷ್ಟು ಕಡಮೆ ಅರ್ಥವಾದರೆ, ನೀವು ಅವರ ಪಾಲಿಗೆ ಅಷ್ಟು ಶ್ರೇಷ್ಠ ವ್ಯಕ್ತಿಗಳಾಗುತ್ತೀರಿ. ಪ್ರತಿಯೊಂದು ದೇಶದಲ್ಲಿಯೂ, ಪ್ರತಿಯೊಂದು ಕಾಲದಲ್ಲಿಯೂ, ಇಂತಹ ಉಪಾಸಕರು ಕೆಲವು ರೇಖಾಚಿತ್ರಗಳು, ಆಕಾರಗಳು – ಇವುಗಳಿಗೆ ಮನಸೋಲುವರು. ರೇಖಾಗಣಿತ ದೊಡ್ಡ ಶಾಸ್ತ್ರವಾಗುವುದು. ಜನರಲ್ಲಿ ಬಹುಮಂದಿಗೆ ಇದರ ಗಂಧವೇ ಇರುವುದಿಲ್ಲ. ರೇಖಾಶಾಸ್ತ್ರಜ್ಞನು ಒಂದು ಚೌಕಟ್ಟನ್ನು ಬರೆದು ನಾಲ್ಕು ಮೂಲೆಗಳಲ್ಲಿ ಏನೇನನ್ನೋ ಉಚ್ಚಾರ ಮಾಡಿದರೆ, ಈ ಪ್ರಪಂಚವೇ ಬದಲಾಗುವುದು, ಸ್ವರ್ಗದ ಬಾಗಿಲೇ ತೆರೆಯುವುದು, ದೇವರೇ ಇಲ್ಲಿಗೆ ಇಳಿದುಬಂದು ನಾವು ಹೇಳಿದಂತೆ ಕೇಳುವನು ಎಂದು ಭಾವಿಸಿದ್ದರು. ಹಗಲು ರಾತ್ರಿ ಅವುಗಳನ್ನು ಅಧ್ಯಯನ ಮಾಡುತ್ತಿರುವ ಎಷ್ಟೋ ಮಂದಿ ಹುಚ್ಚರ ತಂಡ ಈಗಲೂ ಇದೆ. ಇವುಗಳೆಲ್ಲ ಒಂದು ಬಗೆಯ ರೋಗ. ಇವುಗಳೆಲ್ಲ ತತ್ವಜ್ಞಾನಿಗೆ ಅಲ್ಲ, ಕೇವಲ ವೈದ್ಯನಿಗೆ.

ನಾನು ಹಾಸ್ಯ ಮಾಡುತ್ತಿರುವೆನು. ಆದರೆ ನನಗೆ ದುಃಖವಾಗಿದೆ. ಭರತ ಖಂಡದಲ್ಲಿ ಈ ಜಾಡ್ಯ ತುಂಬಾ ದಾರುಣವಾಗಿದೆ. ಇವುಗಳೆಲ್ಲ ಕ್ಷಯಿಸುತ್ತಿರುವ, ಬಂಧನದಲ್ಲಿರುವ ಜನಾಂಗದ ಚಿಹ್ನೆ, ಪತನಸೂಚಕ, ಅಂಜಿಕೆಯ ಲಕ್ಷಣ. ಜೀವನದ ಲಕ್ಷಣ, ಓಜಸ್ಸಿನ ಲಕ್ಷಣ, ಭರವಸೆಯ ಲಕ್ಷಣ, ಆರೋಗ್ಯದ ಲಕ್ಷಣ, ಪ್ರತಿಯೊಂದು ಒಳ್ಳೆಯದರ ಲಕ್ಷಣ ಶಕ್ತಿ. ಬದುಕಿರುವವರೆಗೆ ದೇಹದಲ್ಲಿ ಶಕ್ತಿ ಇರಬೇಕು, ಮನಸ್ಸಿನಲ್ಲಿ ಶಕ್ತಿ ಇರಬೇಕು, ಕೈಕಾಲುಗಳಲ್ಲಿ ಶಕ್ತಿ ಇರಬೇಕು. ಕೆಲಸಕ್ಕೆ ಬಾರದ ಗೊಣಗಾಟಗಳಿಂದ ಆಧ್ಯಾತ್ಮಿಕ ಶಕ್ತಿಯನ್ನು ಗಳಿಸಲೆತ್ನಿಸುವುದರಲ್ಲಿ, ಭಯ ಇದೆ. ನಾನು ಇಂತಹ ಸಂಕೇತಗಳನ್ನಲ್ಲ ಹೇಳುತ್ತಿರುವುದು. ಆದರೆ ಸಂಕೇತಗಳಲ್ಲಿ ಸ್ವಲ್ಪ ಸತ್ಯವಿದೆ. ಸ್ವಲ್ಪ ಸತ್ಯಾಂಶವಿಲ್ಲದ ಯಾವ ಸುಳ್ಳೂ ಇರಲಾರದು. ಮೂಲವಸ್ತುವಿಲ್ಲದೆ ಅನುಕರಣ ಸಾಧ್ಯವಿಲ್ಲ.

ಬೇರೆ ಬೇರೆ ಧರ್ಮಗಳಲ್ಲಿ ಸಂಕೇತಗಳ ಪೂಜೆಗಳು ಜಾರಿಯಲ್ಲಿವೆ. ಇವೆಲ್ಲ ಹೊಸದು, ಶಕ್ತಿಯಿಂದ ಕೂಡಿವೆ, ಕಾವ್ಯಮಯವಾಗಿವೆ, ಮತ್ತು ಆರೋಗ್ಯದಿಂದ ಕೂಡಿವೆ. ಲಕ್ಷಾಂತರ ಜನರ ಮನಸ್ಸಿನಲ್ಲಿ ಶಿಲುಬೆಗೆ ಇರುವ ಅದ್ಭುತವಾದ ಶಕ್ತಿಯಲ್ಲಿರುವ ನಂಬಿಕೆಯನ್ನು ಕುರಿತು ಆಲೋಚಿಸಿ ನೋಡಿ. ಈ ಒಂದು ಸಂಕೇತದ ಆಕರ್ಷಣೆಯನ್ನು ಕುರಿತು ಚಿಂತಿಸಿ ನೋಡಿ. ಜಗತ್ತಿನಲ್ಲಿ ಎಲ್ಲಾ ಕಡೆಗಳಲ್ಲಿಯೂ ಒಳ್ಳೆಯ ಮಹತ್ತಾದ ಸಂಕೇತಗಳು ಇವೆ. ಅವು ಆತ್ಮಶಕ್ತಿಯನ್ನು ವಿವರಿಸುತ್ತವೆ. ಮನಸ್ಸನ್ನು ಒಂದು ಸ್ಥಿಮಿತಕ್ಕೆ ತರುತ್ತವೆ. ಸಾಧಾರಣವಾಗಿ ಇವು ಅದ್ಭುತವಾದ ಶ್ರದ್ಧೆಯನ್ನು ಮತ್ತು ಭಕ್ತಿಯನ್ನು ಹುಟ್ಟಿಸುವುವು.

\newpage

ಪ್ರಾಟೆಸ್ಟಂಟರನ್ನು ಕ್ಯಾಥೋಲಿಕ್ ಪಂಗಡದೊಂದಿಗೆ ಹೋಲಿಸಿ ನೋಡಿ. ಕಳೆದ ನಾಲ್ಕುನೂರು ವರುಷಗಳಿಂದ ಇವೆರಡು ಪಂಗಡಗಳಲ್ಲಿ ಯಾವುದು ಹೆಚ್ಚು ಧರ್ಮವೀರರನ್ನು ಮತ್ತು ಮಹಾತ್ಮರನ್ನು ಸೃಷ್ಟಿಸಿರುವುದು? ಕ್ಯಾಥೋಲಿಕ್ ಪಂಗಡದಲ್ಲಿ ಜಾರಿಯಲ್ಲಿರುವ ಹಲವಾರು ಆಚಾರಗಳಿಂದ ಜನರು ಅದ್ಭುತವಾಗಿ ಆಕರ್ಷಿತರಾಗುವರು. ಆ ಬೆಳಕು, ಧೂಪ, ಕ್ಯಾಂಡಲ್, ಪಾದ್ರಿಗಳ ನಿಲುವಂಗಿ ಇವು ಅದ್ಭುತವಾಗಿ ಪರಿಣಾಮಕಾರಿಯಾಗಿವೆ. ಪ್ರಾಟೆಸ್ಟೆಂಟರಲ್ಲಿ ಯಾವ ಕಾವ್ಯವೂ ಇಲ್ಲ, ಅದು ನಿಸ್ಸಾರವಾಗಿದೆ. ಪ್ರಾಟೆಸ್ಟೆಂಟರು ಎಷ್ಟೋ ಹೊಸ ವಿಷಯಗಳನ್ನು ಸಾಧಿಸಿರುವರು. ಕ್ಯಾಥೋಲಿಕ್ ಜನರಿಗಿಂತ ಹೆಚ್ಚಾಗಿ ಕೆಲವು ವಿಷಯಗಳಲ್ಲಿ ಹೆಚ್ಚು ಸ್ವಾತಂತ್ರ್ಯವನ್ನು ಕೊಟ್ಟಿರುವರು. ಆದುದರಿಂದ ಅವರಲ್ಲಿ ಭಾವನೆ ಸ್ಪಷ್ಟವಾಗಿ ತಿಳಿಯಾಗಿರಬಹುದು. ಇದೇನೋ ಬಹಳ ಒಳ್ಳೆಯದು. ಆದರೆ ಅವರು ಎಷ್ಟೋ ವಿಷಯಗಳನ್ನು ಕಳೆದುಕೊಂಡಿರುವರು. ಚರ್ಚಿನಲ್ಲಿರುವ ಚಿತ್ರಗಳನ್ನು ತೆಗೆದುಕೊಳ್ಳಿ. ಇವು ಕಾವ್ಯಮಯವಾಗಿ ಚಿತ್ರಿಸಲು ಮಾಡಿದ ಪ್ರಯತ್ನ, ಕಾವ್ಯಕ್ಕೆ ನಮ್ಮ ಮನಸ್ಸು ಕಾತರಿಸಿದರೆ ಏತಕ್ಕೆ ಅದನ್ನು ಪಡೆಯಬಾರದು? ಜೀವನಕ್ಕೆ ಏನು ಬೇಕೋ ಅದನ್ನು ಏತಕ್ಕೆ ಕೊಡಬಾರದು? ನಮಗೆ ಸಂಗೀತ ಬೇಕಾಗಿದೆ. ಪ್ರೆಸ್ಬಿಟೇರಿಯನ್ನರು ಸಂಗೀತಕ್ಕೂ ವಿರೋಧವಾಗಿದ್ದರು. ಅವರು ಮಹಮ್ಮದೀಯರಂತೆ. ಎಲ್ಲಾ ಕಾವ್ಯಗಳನ್ನೂ ನಾಶಮಾಡಿ, ಎಲ್ಲಾ ಬಾಹ್ಯಾಚಾರಗಳನ್ನೂ ನಾಶಮಾಡಿ ಅನಂತರ ಅವರು ಸಂಗೀತವನ್ನು ತಯಾರುಮಾಡುವರು. ಅವು ಕೇವಲ ಪಂಚೇಂದ್ರಿಯಗಳಿಗೆ ಪ್ರಿಯವಾಗಿರುತ್ತವೆ ಅಷ್ಟೆ. ಅವರೆಲ್ಲ ಒಟ್ಟು ಸೇರಿ ವೇದಿಕೆಯ ಮೇಲೆ ಒಂದು ಬೆಳಕಿನ ಕಿರಣಕ್ಕಾಗಿ ಎಷ್ಟು ವ್ಯಥೆಪಡುತ್ತಾರೆ ಎಂಬುದನ್ನು ನಾನು\break ನೋಡಿರುವೆನು.

ಜೀವಿಯು ಬಾಹ್ಯಪ್ರಪಂಚದಲ್ಲಿ ಬೇಕಾದಷ್ಟು ಕಾವ್ಯವನ್ನು ಮತ್ತು ಧರ್ಮವನ್ನು ಪಡೆಯಲಿ. ಏತಕ್ಕೆ ಪಡೆಯಬಾರದು? ನೀವು ಬಾಹ್ಯಪೂಜೆಗೆ ವಿರೋಧವಾಗಿ ಹೋರಾಡಲಾರಿರಿ. ಅದು ಪುನಃ ನಮ್ಮನ್ನು ಗೆಲ್ಲುವುದು. ಕ್ಯಾಥೋಲಿಕರು ಏನು ಮಾಡುತ್ತಾರೊ ಅದನ್ನು ನೀವು ಮೆಚ್ಚದೇ ಇದ್ದರೆ ಅದಕ್ಕಿಂತ ಉತ್ತಮವಾದುದನ್ನು ಮಾಡಿ. ನಾವು ಹಾಗೆ ಉತ್ತಮವಾದುದನ್ನು ಮಾಡುವುದಿಲ್ಲ. ಅಥವಾ ಆಗಲೇ ಇರುವ ಕಾವ್ಯಭಾಗವನ್ನು ಸ್ವೀಕರಿಸುವುದೂ ಇಲ್ಲ. ಇದೊಂದು ಅತಿ ಶೋಚನೀಯ ಸ್ಥಿತಿ. ಕಾವ್ಯ ನಮ್ಮ ಜೀವನಕ್ಕೆ ಅತ್ಯಾವಶ್ಯಕ. ನೀವು ಪ್ರಪಂಚದಲ್ಲಿ ಶ್ರೇಷ್ಠ ತತ್ತ್ವಜ್ಞಾನಿಗಳಾಗಿರಬಹುದು. ಆದರೆ ತತ್ತ್ವವೇ ಶ್ರೇಷ್ಠವಾದ ಕಾವ್ಯ. ಅದು ನೀರಸವಲ್ಲ. ಅದು ವಸ್ತುಗಳ ಸಾರ. ಸತ್ಯವು ದ್ವೈತ ಭಾವನೆಗಳೆಲ್ಲಕ್ಕಿಂತಲೂ ಹೆಚ್ಚು ಕಾವ್ಯಮಯವಾಗಿದೆ.

ಧರ್ಮಕ್ಷೇತ್ರದಲ್ಲಿ ಪಾಂಡಿತ್ಯಕ್ಕೆ ಅವಕಾಶವಿಲ್ಲ. ಹಲವರಿಗೆ ಪಾಂಡಿತ್ಯವು ದಾರಿಯಲ್ಲಿನ ಒಂದು ಆತಂಕ. ಒಬ್ಬನು ಪ್ರಪಂಚದಲ್ಲಿರುವ ಪುಸ್ತಕ ಭಂಡಾರದ ಪುಸ್ತಕಗಳನ್ನೆಲ್ಲ ಓದಿರಬಹುದು. ಆದರೂ ಅವನು ಸ್ವಲ್ಪವೂ ಆಧ್ಯಾತ್ಮಿಕ ವ್ಯಕ್ತಿಯಾಗದೆ ಇರಬಹುದು. ಆದರೆ ಮತ್ತೊಬ್ಬನಿಗೆ ತನ್ನ ಸಹಿಯನ್ನು ಕೂಡ ಮಾಡಲು ಬರದೆ ಇರಬಹುದು. ಆದರೆ ಅವನು ಆಧ್ಯಾತ್ಮಿಕ ವಿಷಯಗಳನ್ನು ಗ್ರಹಿಸಬಲ್ಲ ಮತ್ತು ಅವನ್ನು ಸಾಕ್ಷಾತ್ಕಾರಮಾಡಿಕೊಳ್ಳಬಲ್ಲ. ನಮ್ಮ ಆಂತರಿಕ ಅನುಭವವೇ ಧರ್ಮದ ಸರ್ವಸ್ವ. ನಾನು ಪುರುಷಸಿಂಹರನ್ನು ಮಾಡುವ\break ಧರ್ಮವನ್ನು ಕುರಿತು ಹೇಳುತ್ತಿರುವೆನು, ಗ್ರಂಥಗಳು ಮೂಢನಂಬಿಕೆಗಳು ಮತ್ತು ಸಿದ್ಧಾಂತಗಳು – ಇವನ್ನು ಹೇಳುತ್ತಿಲ್ಲ. ಯಾರು ತಮ್ಮಲ್ಲಿಯೇ ಅನಂತಾತ್ಮನನ್ನು ಸಾಕ್ಷಾತ್ಕಾರ ಮಾಡಿಕೊಂಡಿರುವರೋ ಅವರನ್ನು ಕುರಿತು ಹೇಳುತ್ತಿರುವೆನು.

ನಾನು ಇಡಿಯ ಜೀವನ ಯಾರ ಪದತಳದಲ್ಲಿ ಕುಳಿತೆನೊ, ಅವರ ಕೆಲವು ಭಾವನೆಗಳನ್ನು ಮಾತ್ರ ಬೋಧಿಸಲು ಯತ್ನಿಸುತ್ತಿರುವೆನು. ಅವರಿಗೆ ತಮ್ಮ ಹೆಸರನ್ನು ಕೂಡ ಸರಿಯಾಗಿ ಬರೆಯಲು ಬರುತ್ತಿರಲಿಲ್ಲ. ನನ್ನ ಜೀವನದಲ್ಲಿ ಇಂತಹ ಮತ್ತೊಬ್ಬರನ್ನು ನಾನು ಕಂಡಿಲ್ಲ. ನಾನು ಪ್ರಪಂಚವನ್ನೆಲ್ಲ ಸಂಚಾರ ಮಾಡಿರುವೆನು. ಅವರನ್ನು ಕುರಿತು ಚಿಂತಿಸಿದಾಗ ನಾನು ಮೂರ್ಖನಂತೆ ತೋರುವೆನು. ಏಕೆಂದರೆ ನಾನು ಪುಸ್ತಕಗಳನ್ನು ಓದಬೇಕೆಂದಿರುವೆನು. ಅವರು ಯಾವ ಪುಸ್ತಕವನ್ನೂ ಓದಿದವರಲ್ಲ. ಅವರು ಇತರರು ಊಟಮಾಡಿದ ತಟ್ಟೆಯನ್ನು ನೆಕ್ಕಲು ಇಚ್ಚಿಸಲಿಲ್ಲ. ಆದಕಾರಣವೆ ಅವರ ಜೀವನವೇ ಒಂದು ಸ್ವತಂತ್ರವಾದ ಪುಸ್ತಕವಾಗಿತ್ತು. ನಾನು ಜೀವನವೆಲ್ಲ ಜಾಕ್ ಏನು ಹೇಳಿದ, ಐದು ವರ್ಷಗಳ ಹಿಂದೆ ಜಾಕ್ ಏನು ಹೇಳಿದ ಎಂದು ಹೇಳುತ್ತಿರುವೆನು. ನಾನೇ ಏನನ್ನೂ ಹೇಳುವುದಿಲ್ಲ. ಇಪ್ಪತ್ತೈದು ವರ್ಷಗಳ ಹಿಂದೆ ಜಾನ್ ಏನು ಹೇಳಿದ, ಜಾನ್ ಏನು ಹೇಳಿದ ಎಂಬುದನ್ನು ಈಗ ಹೇಳಿದರೆ ಅದರಲ್ಲಿ ದೊಡ್ಡಸ್ತಿಕೆ ಏನು ಇದೆ? ನೀನು ಏನು ಹೇಳಬೇಕೆಂದಿರುವೆಯೊ ಅದನ್ನು ಹೇಳು.

ಪಾಂಡಿತ್ಯದಿಂದ ಏನೂ ಪ್ರಯೋಜನವಿಲ್ಲ ಎಂಬುದನ್ನು ತಿಳಿಯಿರಿ. ಪಾಂಡಿತ್ಯದಿಂದ ನೀವು ಭ್ರಾಂತರಾಗಿರುವಿರಿ. ಇದು ನಮ್ಮ ಮನಸ್ಸನ್ನು ಪುಷ್ಟಿಗೊಳಿಸುವುದು, ಅದಕ್ಕೆ ಒಂದು ಶಿಸ್ತನ್ನು ಕೊಡುವುದು. ಇವಿಷ್ಟೇ ಪಾಂಡಿತ್ಯದಿಂದಾಗುವ ಪ್ರಯೋಜನ. ನಾವು ಯಾವಾಗಲೂ ಇನ್ನೊಬ್ಬರು ಹೇಳಿರುವುದನ್ನು ಸ್ವೀಕರಿಸುವುದರಿಂದ ನಮಗೆ ಅಜೀರ್ಣವಾಗದೇ ಇದ್ದರೆ ಅದು ಹೆಚ್ಚು. ಪಾಂಡಿತ್ಯದ ಸಂಪಾದನೆಯನ್ನು ನಿಲ್ಲಿಸೋಣ. ಗ್ರಂಥಗಳನ್ನೆಲ್ಲ ಬೆಂಕಿಗೆ ಹಾಕಿ, ನಮ್ಮ ವ್ಯಕ್ತಿತ್ವವನ್ನು ಕುರಿತು ವಿಚಾರ ಮಾಡೋಣ. ನೀವೆಲ್ಲ ಮಾತನಾಡಿ, ನಿಮ್ಮ ವ್ಯಕ್ತಿತ್ವಕ್ಕೆ ಭಂಗ ತಂದುಕೊಳ್ಳುತ್ತಿರುವಿರಿ. ಇತರರು ಹೇಳುವುದನ್ನು ಸುಮ್ಮನೆ ಸ್ವೀಕರಿಸಿ ವ್ಯಕ್ತಿತ್ವವನ್ನು ಪ್ರತಿಕ್ಷಣವೂ ಕಳೆದುಕೊಳ್ಳುತ್ತಿರುವಿರಿ. ನಾನು ಬೋಧಿಸುವುದನ್ನು ಯಾರಾದರೂ ನಂಬಿದರೆ ನನಗೆ ವ್ಯಥೆಯಾಗುವುದು. ನೀವೇ ಸ್ವತಂತ್ರವಾಗಿ ಯೋಚಿಸುವ ಸ್ವಭಾವವನ್ನು ಕೆರಳಿಸಬೇಕೆಂಬುದು ನನ್ನ ಇಚ್ಛೆ. ನನ್ನ ಇಚ್ಚೆಯೇ ಸ್ತ್ರೀ ಪುರುಷರಿಗೆ ಹೇಳಬೇಕೆಂಬುದು, ಒಂದು ಕುರಿಯ ಮಂದೆಗಲ್ಲ. ಸ್ತ್ರೀ ಪುರುಷರು ಎಂದರೆ ಒಂದು ವ್ಯಕ್ತಿಯಾಗಿರುವವರು, ದಾರಿಯಲ್ಲಿ ಸಿಕ್ಕುವ ಚಿಂದಿಬಟ್ಟೆಗಳನ್ನೆಲ್ಲ ಮನೆಗೆ ತಂದು ಅದರಿಂದ ಒಂದು ಬೊಂಬೆಯನ್ನು ಮಾಡುವುದಕ್ಕೆ ನೀವೇನು ಆಡುವ ಮಕ್ಕಳಲ್ಲ.

ಇದೇನೊ ಕಲಿಯುವುದಕ್ಕೆ ದೊಡ್ಡ ಸ್ಥಳ! ಮನುಷ್ಯನನ್ನು ವಿಶ್ವವಿದ್ಯಾನಿಲಯಕ್ಕೆ ಸೇರಿಸುವರು. ಅವನಿಗೆ ಮಿಸ್ಟರ್ ಬ್ಲಾಂಕ್ ಏನು ಹೇಳಿರುವನೋ ಅದೆಲ್ಲ ಗೊತ್ತಿದೆ. ಆದರೆ ಮಿಸ್ಟರ್ ಬ್ಲಾಂಕ್ ಏನನ್ನೂ ಹೇಳಲಿಲ್ಲ. ನನಗೆ ಏನಾದರೂ ಅವಕಾಶ ಇದ್ದರೆ ಪ್ರಾಧ್ಯಾಪಕನಿಗೆ “ನೀನು ಹೊರಡು, ನಿನಗೆ ಏನೂ ಗೊತ್ತಿಲ್ಲ" ಎನ್ನುತ್ತಿದ್ದೆ. ಏನಾದರೂ ಆಗಲಿ ವ್ಯಕ್ತಿತ್ವವನ್ನು ರಕ್ಷಿಸಿಕೊಳ್ಳಬೇಕು ಎಂಬುದನ್ನು ಗಮನದಲ್ಲಿಡಿ. ನಿಮಗೆ ಇಚ್ಛೆ ಇದ್ದರೆ ತಪ್ಪನ್ನಾದರೂ ಕುರಿತು ಯೋಚಿಸಿ, ಸತ್ಯ ನಿಮಗೆ ದೊರಕುವುದೊ ಇಲ್ಲವೊ, ಅದನ್ನು ಲೆಕ್ಕಿಸಬೇಕಾಗಿಲ್ಲ. ಮುಖ್ಯ ಉದ್ದೇಶವೇ ಮನಸ್ಸಿಗೆ ತರಬೇತಿಯನ್ನು ಕೊಡುವುದು. ಯಾವ ಸತ್ಯವನ್ನು ನೀವು ಮತ್ತೊಬ್ಬರಿಂದ ಸ್ವೀಕರಿಸುವಿರೋ ಅದೆಂದಿಗೂ ನಿಮ್ಮದಾಗಲಾರದು. ನೀವು ನನ್ನ ಬಾಯಿಯ ಮೂಲಕ ಸತ್ಯವನ್ನು ಅರಿಯಲಾರಿರಿ. ಯಾರೂ ಮತ್ತೊಬ್ಬರಿಗೆ ಕಲಿಸಲಾರರು. ನೀವೇ ಸತ್ಯವನ್ನು ಸಾಕ್ಷಾತ್ಕಾರಮಾಡಿಕೊಂಡು ನಿಮ್ಮ ಸ್ವಭಾವಕ್ಕೆ ಅನುಗುಣವಾಗಿ ಅದನ್ನು ವ್ಯಕ್ತಪಡಿಸಬೇಕಾಗಿದೆ. ಪ್ರತಿಯೊಬ್ಬರೂ ಒಂದು ವ್ಯಕ್ತಿಯಾಗಲು ಪ್ರಯತ್ನಿಸಬೇಕು. ಬಲಿಷ್ಠರಾಗಿ, ನಿಮ್ಮ ಕಾಲಿನ ಮೇಲೆ ನೀವೇ ನಿಂತುಕೊಂಡು, ನಿಮ್ಮ ಭಾವನೆಗಳನ್ನು ನೀವೇ ಯೋಚಿಸುವಂತೆ ಆಗಬೇಕು. ನಿಮ್ಮ ಆತ್ಮವನ್ನು ನೀವೇ ಸಾಕ್ಷಾತ್ಕಾರ ಮಾಡಿಕೊಳ್ಳಬೇಕು. ಇತರರ ಸಿದ್ದಾಂತವನ್ನು ಸ್ವೀಕರಿಸಿ ಪ್ರಯೋಜನವಿಲ್ಲ. ಜೈಲಿನಲ್ಲಿರುವ ಸಿಪಾಯಿಗಳಂತೆ ಒಟ್ಟಿಗೆ ಎದ್ದು ನಿಲ್ಲುವುದು, ಒಟ್ಟಿಗೆ ಕುಳಿತುಕೊಳ್ಳುವುದು, ಎಲ್ಲರೂ ಒಂದೇ ಬಗೆಯ ಆಹಾರವನ್ನು ತೆಗೆದುಕೊಳ್ಳುವುದು, ಎಲ್ಲರೂ ಏಕಕಾಲದಲ್ಲೇ ತಲೆಯನ್ನು ಅಲ್ಲಾಡಿಸುವುದು, ಇವುಗಳಿಂದ ಪ್ರಯೋಜನವಿಲ್ಲ. ವೈವಿಧ್ಯವೇ ಜೀವನದ ಚಿಹ್ನೆ. ಒಂದೇ ಸಮನಾಗಿರುವುದು ಮೃತ್ಯುವಿನ ಚಿಹ್ನೆ.

ನಾನು ಒಮ್ಮೆ ಇಂಡಿಯಾದೇಶದಲ್ಲಿ ಒಂದು ನಗರದಲ್ಲಿದ್ದೆ. ಒಬ್ಬ ವೃದ್ದ ನನ್ನ ಬಳಿಗೆ ಬಂದನು. ಆತನು “ಸ್ವಾಮಿ, ನನಗೆ ಮಾರ್ಗವನ್ನು ತೋರಿ” ಎಂದು ಕೇಳಿಕೊಂಡ. ಆತನು ಎದುರಿಗೆ ಇರುವ ಮೇಜಿನಷ್ಟು ನಿರ್ಜಿವವಾಗಿದ್ದುದನ್ನು ಕಂಡೆ. ಆಧ್ಯಾತ್ಮಿಕವಾಗಿ ಮತ್ತು ಮಾನಸಿಕವಾಗಿ ಆ ಮನುಷ್ಯ ನಿಜವಾಗಿಯೂ ನಿರ್ಜಿವವಾಗಿದ್ದ. "ನಾನು ಹೇಳುವುದನ್ನು ನೀನು ಮಾಡಬಲ್ಲೆಯಾ? ನೀನು ಕದಿಯಬಲ್ಲೆಯಾ? ನೀನು ಮದ್ಯಪಾನ ಮಾಡಬಲ್ಲೆಯಾ? ಮಾಂಸವನ್ನು ತಿನ್ನಬಲ್ಲೆಯಾ?" ಎಂದು ಕೇಳಿದೆ. ಆ ಮನುಷ್ಯ "ಏನು ಸ್ವಾಮಿ ನೀವು ಬೋಧಿಸುತ್ತಿರುವುದು?” ಎಂದು ಆಶ್ಚರ್ಯಪಟ್ಟ. ಅದಕ್ಕೆ ನಾನು ಅವನಿಗೆ “ಈ ಗೋಡೆ ಎಂದಾದರೂ ಕದಿಯಿತೆ? ಈ ಗೋಡೆ ಎಂದಾದರೂ ಮದ್ಯವನ್ನು ಕುಡಿಯಿತೆ?'' ಎಂದು ಪ್ರಶ್ನಿಸಿದೆ. “ಇಲ್ಲ”, ಎಂದ ಅವನು. ಮನುಷ್ಯ ಕದಿಯುವನು, ಮದ್ಯವನ್ನು ಕುಡಿಯುವನು, ಆದರೆ ಅವನು ದೇವರೂ ಆಗಬಲ್ಲ. "ನನ್ನ ಸಖನೆ ನೀನು ಗೋಡೆಯಲ್ಲ ಎಂಬುದು ನನಗೆ ಗೊತ್ತಿದೆ. ಏನನ್ನಾದರೂ ಮಾಡು, ಏನನ್ನಾದರೂ ಮಾಡು'' ಎಂದೆ. ಆ ಮನುಷ್ಯ ಏನನ್ನಾದರೂ ಕದಿಯಲು ಪ್ರಾರಂಭಿಸಿದರೆ ಮುಕ್ತಿಯ ಕಡೆ ಹೋಗಬಲ್ಲ ಎಂಬುದನ್ನು ನಾನು ಕಂಡೆ.

ಎಲ್ಲರೂ ಒಂದೇ ಮಾತನ್ನು ಆಡುತ್ತಿರುವುದು, ಎಲ್ಲರೂ ಒಟ್ಟಿಗೆ ಏಳುವುದು, ಕುಳಿತುಕೊಳ್ಳುವುದು, ಇವುಗಳನ್ನು ಮಾಡುತ್ತಿರುವಾಗ ನಿಮ್ಮನ್ನು ಒಂದು ವ್ಯಕ್ತಿ ಎಂದು ಹೇಗೆ ಕರೆಯಬಹುದು? ಇದು ಸಾವಿನ ದಾರಿ, ನಿಮ್ಮ ಆತ್ಮೋದ್ಧಾರಕ್ಕಾಗಿ ಏನನ್ನಾದರೂ ಮಾಡಿ. ನಿಮಗೆ ಮನಸ್ಸು ಬಂದರೆ ತಪ್ಪನ್ನಾದರೂ ಮಾಡಿ, ಅಂತೂ ಏನನ್ನಾದರೂ ಮಾಡಿ. ನಿಮಗೆ ಸದ್ಯಕ್ಕೆ ನನ್ನನ್ನು ತಿಳಿದುಕೊಳ್ಳಲು ಆಗದೇ ಇದ್ದರೂ ಕಾಲ ಕ್ರಮೇಣ ತಿಳಿದುಕೊಳ್ಳುವಿರಿ. ಆತ್ಮವನ್ನು ವೃದ್ದಾಪ್ಯ ಆವರಿಸಿಕೊಂಡಂತೆ ಇದೆ. ಅದರ ಮೇಲೆ ಧೂಳು ಕವಿದಿದೆ. ಧೂಳನ್ನು ಮುಂಚೆ ಕೊಡವಬೇಕಾಗಿದೆ. ಅನಂತರ ನಾವು ಮುಂದುವರಿಯಬಹುದು. ಪ್ರಪಂಚದಲ್ಲಿ ಏತಕ್ಕೆ ಪಾಪವಿದೆ ಎಂಬುದು ಈಗ ನಿಮಗೆ ಅರ್ಥವಾಗುವುದು. ಮನೆಗೆ ಹೋಗಿ ಮುಸುಕಿರುವ ಧೂಳನ್ನು ಹೋಗಲಾಡಿಸುವುದರ ಬಗ್ಗೆ ಯೋಚಿಸಿ.

ನಾವು ಲೌಕಿಕ ವಸ್ತುಗಳಿಗಾಗಿ ಪ್ರಾರ್ಥಿಸುವೆವು. ನಾವು ಯಾವುದನ್ನೋ ಪಡೆಯುವುದಕ್ಕಾಗಿ ವ್ಯಾಪಾರಿಯಂತೆ ದೇವರನ್ನು ಪೂಜಿಸುವೆವು, ಅನ್ನ ಬಟ್ಟೆಗಳಿಗಾಗಿ ಪ್ರಾರ್ಥಿಸುವೆವು. ಪೂಜೆಯೇನೊ ಒಳ್ಳೆಯದು. ಏನನ್ನಾದರೂ ಮಾಡುವುದು, ಏನನ್ನೂ ಮಾಡದೆ ಇರುವುದಕ್ಕಿಂತ ಮೇಲು. “ಯಾವ ಸೋದರಮಾವನೂ ಇಲ್ಲದಿರುವುದಕ್ಕಿಂತ ಕುರುಡು ಸೋದರಮಾವ ಇರುವುದು ಮೇಲು.” ಒಬ್ಬ ಶ‍್ರೀಮಂತನಾದ ಯುವಕ ರೋಗಗ್ರಸ್ತನಾಗುವನು. ಅನಂತರ ಅವನು ರೋಗದಿಂದ ಪಾರಾಗುವುದಕ್ಕಾಗಿ ಬಡವರಿಗೆ ದಾನ ಮಾಡುವನು. ಅದೇನೊ ಒಳ್ಳೆಯದು, ಆದರೆ ಅದು ಇನ್ನೂ ಧರ್ಮವಾಗಿಲ್ಲ, ಆಧ್ಯಾತ್ಮಿಕ ಧರ್ಮವಾಗಿಲ್ಲ. ಇದೆಲ್ಲ ಲೌಕಿಕ ಕ್ಷೇತ್ರಕ್ಕೆ ಸಂಬಂಧಪಟ್ಟದ್ದು. ಯಾವುದು ಲೌಕಿಕ ಮತ್ತು ಯಾವುದು ಲೌಕಿಕವಲ್ಲ? ಪ್ರಪಂಚವೇ ಗುರಿಯಾಗಿ, ದೇವರು ಅದಕ್ಕೆ ಸಹಾಯವಾದರೆ ಅದು ಲೌಕಿಕ. ದೇವರೇ ಗುರಿಯಾಗಿ, ಈ ಪ್ರಪಂಚಕ್ಕೆ ಆ ಗುರಿಯನ್ನು ಸಾಧಿಸಲು ಸಾಧ್ಯವಾದರೆ ಆಗ ಆಧ್ಯಾತ್ಮಿಕ ಜೀವನ ಪ್ರಾರಂಭವಾಗಿದೆ.

ಯಾರಿಗೆ ಇನ್ನೂ ಈ ಲೋಕವೇ ಬೇಕಾಗಿದೆಯೋ, ಅವರ ಸ್ವರ್ಗವೆಲ್ಲ ಈಗಿರುವ ಸ್ಥಿತಿಯ ಮುಂದುವರಿಯುವಿಕೆ, ಗತಿಸಿ ಹೋದವರನ್ನೆಲ್ಲ ಅಲ್ಲಿ ಅವನು ನೋಡಲು ಇಚ್ಛಿಸುವನು. ಪುನಃ ಅವನು ಅಲ್ಲಿ ಸುಖವಾಗಿ ಕಾಲವನ್ನು ಕಳೆಯಲು ಇಚ್ಛಿಸುವನು.

ಗತಿಸಿ ಹೋದವರನ್ನು ಪುನಃ ನಮ್ಮ ಕಣ್ಣೆದುರಿಗೆ ತರಬಲ್ಲ ಮಾಧ್ಯಮವಾಗುವ ವ್ಯಕ್ತಿಯಾದ (\enginline{medium}) ಹೆಂಗಸೊಬ್ಬಳಿದ್ದಳು. ಅವಳು ತುಂಬಾ ಸ್ಥೂಲ ಕಾಯದವಳು. ಆದರೂ ಅವಳನ್ನು `ಮಾಧ್ಯಮಾ' ಎಂದು ಕರೆಯುತ್ತಿದ್ದರು. ಇದು ಬಹಳ ಒಳ್ಳೆಯದು! ಆ ಹೆಂಗಸಿಗೆ ನನ್ನ ಮೇಲೆ ವಿಶ್ವಾಸವಿತ್ತು. ಅವಳು ಒಂದು ದಿನ ನನ್ನನ್ನು ಆಹ್ವಾನಿಸಿದಳು. ಪ್ರೇತಗಳೆಲ್ಲ ನನ್ನನ್ನು ಆದರದಿಂದ ಕಂಡವು. ನನಗೆ ಒಂದು ವಿಚಿತ್ರವಾದ ಅನುಭವ ಆಯಿತು. ಪ್ರೇತಗಳು ಬರುವ ಅರ್ಧರಾತ್ರಿಯ ಸಮಯ, ಅದು. ಮಾಧ್ಯಮವಾಗಿದ್ದ ವ್ಯಕ್ತಿ ಹೇಳಿತು: “ಇಲ್ಲಿ ಒಂದು ಪ್ರೇತ ನಿಂತಿರುವುದನ್ನು ನಾನು ನೋಡುತ್ತಿರುವೆ. ಆ ಪ್ರೇತ ನನಗೆ ಆ ಬೆಂಚಿನ ಮೇಲೆ ಒಬ್ಬ ಹಿಂದೂ ಭದ್ರ ಮನುಷ್ಯನು ಕುಳಿತಿರುವನು ಎಂದು ಹೇಳುತ್ತದೆ." ನಾನು ಎದ್ದು ನಿಂತು, “ಇದನ್ನು ಹೇಳುವುದಕ್ಕೆ ಯಾವ ಪ್ರೇತವೂ ಬೇಕಾಗಿಲ್ಲ!” ಎಂದೆ.

ಒಬ್ಬ ಯುವಕ ಅಲ್ಲಿದ್ದ. ಅವನು ವಿದ್ಯಾವಂತ, ಬುದ್ದಿವಂತ, ಅವನಿಗೆ ಮದುವೆ ಆಗಿತ್ತು. ಅವನು ತನ್ನ ಗತಿಸಿಹೋದ ತಾಯಿಯನ್ನು ನೋಡಬೇಕೆಂದು ಬಂದಿದ್ದ. ಮಾಧ್ಯಮವಾಗಿದ್ದ ವ್ಯಕ್ತಿ ಇಂತಹವನ ತಾಯಿ ಇಲ್ಲಿರುವಳು ಎಂದು ಹೇಳಿತು. ಆ ಯುವಕ ನನಗೆ ತನ್ನ ತಾಯಿಯ ವಿಷಯವನ್ನು ಹೇಳುತ್ತಿದ್ದ. ಅವಳು ಸಾಯುವ ಸಮಯದಲ್ಲಿ ಬಹಳ ತೆಳ್ಳಗಿದ್ದಳು. ಆದರೆ ತೆರೆಯ ಹಿಂದಿನಿಂದ ಬಂದ ತಾಯಿಯ ಪ್ರೇತವಾದರೋ! ನೀವು ಅದನ್ನು ನೋಡಬೇಕಿತ್ತು! ಆ ಯುವಕ ಏನು ಮಾಡುತ್ತಾನೋ ನಾನು ನೋಡಬೇಕೆಂದು ಇದ್ದೆ. ನನಗೆ ಆಶ್ಚರ್ಯವಾಯಿತು. ಆ ಮನುಷ್ಯ ನೆಗೆದು ಆ ಪ್ರೇತದ ಹತ್ತಿರ ಹೋಗಿ ಅದನ್ನು ಆಲಿಂಗಿಸಿಕೊಂಡನು. “ತಾಯಿ, ಪ್ರೇತಲೋಕದಲ್ಲಿ ನೀನೆಷ್ಟು ಮೈತುಂಬಿಕೊಂಡಿರುವಿ!” ಎಂದನು. ನಾನು ಇಲ್ಲಿ ಇದ್ದುದು ಸಾರ್ಥಕವಾಯಿತು, ಮನುಷ್ಯನ ಸ್ವಭಾವವನ್ನು ತಿಳಿದುಕೊಳ್ಳಲು ಇದು ಸಹಾಯಮಾಡುವುದು ಎಂದುಕೊಂಡೆ.

ನಮ್ಮ ಬಾಹ್ಯಪೂಜೆಗೆ ಈಗ ಮರಳೋಣ. ಈ ಜೀವನಕ್ಕಾಗಿ, ಈ ಪ್ರಪಂಚಕ್ಕಾಗಿ ದೇವರನ್ನು ಪೂಜಿಸುವುದು ಬಹಳ ಕೀಳು ದರ್ಜೆಯ ಪೂಜೆ. ಮಾನವರಲ್ಲಿ ಬಹುಮಂದಿಗೆ ಈ ಮಾಂಸದ ಮುದ್ದೆಗಿಂತ, ಇಂದ್ರಿಯಗಳ ಸುಖಕ್ಕಿಂತ ಮೇಲಾಗಿರುವ ಯಾವುದನ್ನೂ ಗ್ರಹಿಸಲು ಅಸಾಧ್ಯ. ಈ ಜೀವನದಲ್ಲಿಯೂ ಈ ಬಡ ಪ್ರಾಣಿಗಳ ಸುಖವೆಲ್ಲ ಇರುವುದು ಪ್ರಾಣಿಗಳಂತೆ ಇಂದ್ರಿಯಗಳಲ್ಲಿ. ಅವರು ಪ್ರಾಣಿಗಳನ್ನು ತಿನ್ನುವರು. ಅವರು ತಮ್ಮ ಮಕ್ಕಳನ್ನು ಪ್ರೀತಿಸುತ್ತಾರೆ. ಇದಿಷ್ಟೇ ಮನುಷ್ಯನ ಮಹಿಮೆ! ನಾವು ಸರ್ವಶಕ್ತನಾದ ಪರಮಾತ್ಮನನ್ನು ಪೂಜಿಸುತ್ತೇವೆ. ಇದು ಏತಕ್ಕೆ? ನಮಗೆ ಪ್ರಾಪಂಚಿಕ ವಸ್ತುಗಳನ್ನು ಕೊಡುವುದಕ್ಕೆ ಮತ್ತು ನಮ್ಮನ್ನು ರಕ್ಷಿಸುವುದಕ್ಕೆ ಒಬ್ಬನು ಬೇಕಾಗಿರುವುದರಿಂದ, ಅಂದರೆ ನಾವು ಪಶುಪಕ್ಷಿಗಳಿಗಿಂತ ಮೇಲೆ ಹೋಗಿಲ್ಲ ಎಂದಾಯಿತು. ನಾವು ಅವುಗಳಿಗಿಂತ ಮೇಲಲ್ಲ. ಅವುಗಳಿಗಿಂತ ಉತ್ತಮವಾಗಿರುವುದೇನೂ ನಮಗೆ ಗೊತ್ತಿಲ್ಲ. ಇದು ತುಂಬ ಶೋಚನೀಯ, ನಮಗೆ ಇನ್ನೂ ಹೆಚ್ಚು ತಿಳಿವಳಿಕೆ ಬೇಕಾಗಿದೆ! ಪ್ರಾಣಿಗಳಿಗೆ ನಮ್ಮಂತೆ ಒಬ್ಬ ದೇವರಿಲ್ಲ. ಅದೇ ನಮಗೂ ಅವುಗಳಿಗೂ ಇರುವ ವ್ಯತ್ಯಾಸ, ಪ್ರಾಣಿಗಳಿಗೆ ಇರುವಂತೆ ನಮಗೂ ಪಂಚೇಂದ್ರಿಯಗಳಿವೆ. ಆದರೆ ಅವುಗಳ ಇಂದ್ರಿಯಗಳು ನಮ್ಮ ಇಂದ್ರಿಯಗಳಿಗಿಂತ\break ಚುರುಕಾಗಿವೆ. ನಾಯಿಯು ಮೂಳೆಯನ್ನು ಯಾವ ಆನಂದದಿಂದ ಕಡಿಯುವುದೋ ಅದೇ ಆನಂದದಿಂದ ನಾವು ಊಟ ಮಾಡಲಾರೆವು. ನಮಗಿಂತ ಅವಕ್ಕೆ ಜೀವನದಲ್ಲಿ ಹೆಚ್ಚು ಸಂತೋಷವಿದೆ. ಆ ದೃಷ್ಟಿಯಿಂದ ನಾವು ಪ್ರಾಣಿಗಳಿಗಿಂತಲೂ ಸ್ವಲ್ಪ ಕಡಮೆ.

ಪ್ರಕೃತಿಯ ಶಕ್ತಿ ನಿಮ್ಮ ಮೇಲೆ ಚೆನ್ನಾಗಿ ಪರಿಣಾಮಕಾರಿಯಾಗುವುದಕ್ಕಾಗಿ ನೀವು ಏನು ಬೇಕಾದರೂ ಆಗಲು ಬಯಸುವಿರಿ. ನೀವು ಕೂಲಂಕಷವಾಗಿ ಪರ್ಯಾಲೋಚಿಸಬೇಕಾದ ಮಹಾ ಪ್ರಶ್ನೆ ಇದು. ನಿಮಗೆ ಈ ಜೀವನ, ಈ ಪಂಚೇಂದ್ರಿಯಗಳು, ಈ ದೇಹ – ಇವು ಬೇಕೆ? ಅಥವಾ ಇವಕ್ಕಿಂತ ಬಹಳ ಮೇಲಿರುವ ಶ್ರೇಷ್ಠವಾದ ಸ್ಥಿತಿ, ನಾವು ಪುನಃ ಅಲ್ಲಿಂದ ಜಾರಲಾರದ, ಬದಲಾಯಿಸಲಾರದ ಸ್ಥಿತಿ ಬೇಕೆ?

ಹಾಗೆಂದರೆ ಏನು ಅರ್ಥ? ನಾವು ದೇವರಿಗೆ, “ದೇವರೆ ನನಗೆ ಅನ್ನ ಕೊಡು, ಹಣ ಕೊಡು, ನನ್ನ ರೋಗವನ್ನು ಗುಣಮಾಡು, ಮತ್ತು ಇದನ್ನು ಮಾಡು, ಅದನ್ನು ಮಾಡು” ಎಂದು ಹೇಳಿಕೊಳ್ಳುವೆವು. ನೀವು ಹೀಗೆ ಪ್ರತಿಯೊಂದು ವೇಳೆ ಹೇಳಿದಾಗಲೂ, `ನಾನು ದೇಹ, ಈ ದೇಹವೇ ನನ್ನ ಗುರಿ' ಎಂಬ ಭಾವಕ್ಕೆ ದಾಸರಾಗುತ್ತಿರುವಿರಿ. ನೀವು ನಿಮ್ಮ ದೇಹದ ಆಸೆಯನ್ನು ಪೂರೈಸಿಕೊಳ್ಳುವಾಗಲೆಲ್ಲ, `ನಾನು ದೇಹ, ಆತ್ಮವಲ್ಲ' ಎಂಬುದನ್ನು ನಿಮಗೆ ನೀವೇ ಹೇಳಿಕೊಳ್ಳುವಿರಿ.

\newpage

ದೇವರಿಗೆ ಧನ್ಯವಾದ, ಸದ್ಯಕ್ಕೆ ಇದೊಂದು ಕನಸು! ದೇವರಿಗೆ ಧನ್ಯವಾದ, ಏಕೆಂದರೆ ಈ ಕನಸು ಮಾಯವಾಗುವುದು. ಕೊನೆಗೆ ಮೃತ್ಯು ಇರುವುದು. ಮಹಿಮಾಮಯವಾದ ಮೃತ್ಯು ಇರುವುದರಿಂದ ದೇವರಿಗೆ ಧನ್ಯವಾದ. ಏಕೆಂದರೆ ಇದು ನಮ್ಮ ಭ್ರಮೆಯನ್ನೆಲ್ಲ, ಕನಸನ್ನೆಲ್ಲ, ದೇಹ ಭಾವನೆಯನ್ನೆಲ್ಲ ನೋವನ್ನೆಲ್ಲ ಕೊನೆಗಾಣಿಸುವುದು. ಯಾವ ಕನಸೂ ಎಂದೆಂದಿಗೂ ಇರಲಾರದು. ವೇಗವಾಗಿಯೊ, ನಿಧಾನವಾಗಿಯೊ ಅದು ಕೊನೆಗಾಣಲೇ ಬೇಕು. ಯಾರೂ ತಮ್ಮ ಕನಸನ್ನು ಎಂದೆಂದಿಗೂ ಇಟ್ಟುಕೊಂಡಿರಲಾರರು. ಇದು\break ಹೀಗಿರುವುದರಿಂದ ದೇವರಿಗೆ ಧನ್ಯವಾದ. ಆದರೂ ಇಂತಹ ಪೂಜೆಯೂ ಕೂಡ ಸರಿಯೆ. ಹೀಗೆಯೆ ಮಾಡಿಕೊಂಡು ಹೋಗಿ, ಯಾವುದಕ್ಕಾದರೂ ಪ್ರಾರ್ಥಿಸುವುದು ಪ್ರಾರ್ಥನೆಯನ್ನೇ ಮಾಡದೆ ಇರುವುದಕ್ಕಿಂತ ಮೇಲು, ಇವುಗಳೆಲ್ಲ ನಾವು ಸಾಗಿಹೋಗಬೇಕಾದ ಮೆಟ್ಟಲುಗಳು. ಇವುಗಳೆಲ್ಲ ಪ್ರಾಥಮಿಕ ಪಾಠಗಳು. ಕ್ರಮೇಣ ಮನಸ್ಸು ಇಂದ್ರಿಯಾತೀತವಾದುದನ್ನು, ದೇಹಾತೀತವಾದುದನ್ನು, ಈ ಪ್ರಪಂಚದ ಭೋಗಕ್ಕೆ ಅತೀತವಾದುದನ್ನು ಕುರಿತು ಯೋಚಿಸುವುದು.

ಮನುಷ್ಯ ಇದನ್ನು ಹೇಗೆ ಮಾಡುವನು? ಮೊದಲು ಅವನು ಆಲೋಚನಾಪರನಾಗುವನು. ನೀವು ಒಂದು ಸಮಸ್ಯೆಯನ್ನು ಕುರಿತು ಯೋಚಿಸತೊಡಗಿದರೆ, ಅಲ್ಲಿ ಇಂದ್ರಿಯ ಸುಖವಿರುವುದಿಲ್ಲ, ಅಲ್ಲಿ ಆಲೋಚನೆಯ ಮಹದಾನಂದವಿದೆ. ಇದೇ ಮನುಷ್ಯನನ್ನು ನಿರ್ಮಿಸುವುದು. ಒಂದು ಮಹಾಭಾವನೆಯನ್ನು ತೆಗೆದುಕೊಳ್ಳಿ. ಅದು ಆಳ ಆಳಕ್ಕೆ ಹೋಗುತ್ತ ಬರುವುದು. ಏಕಾಗ್ರತೆ ಬರುವುದು. ಆಗ ನಿಮಗೆ ದೇಹದ ಪರಿವೆಯೇ ಇರುವುದಿಲ್ಲ. ನಿಮ್ಮ ಇಂದ್ರಿಯಗಳು ಕೆಳಗೆ ನಿಲ್ಲುವುವು. ನೀವು ಎಲ್ಲಾ ವಿಧವಾದ ದೈಹಿಕಭಾವನೆಗಳಿಂದಲೂ ಮೇಲೆ ಇರುವಿರಿ. ಇಂದ್ರಿಯಗಳ ಮೂಲಕ ಯಾವುದು ವ್ಯಕ್ತವಾಗುವುದೊ ಅದೆಲ್ಲ ಯಾವುದೊ ಒಂದು ಭಾವನೆಯ ಮೇಲೆ ಕೇಂದ್ರೀಕೃತವಾಗುವುದು. ಆ ಕಾಲದಲ್ಲಿ ನೀವು ಪ್ರಾಣಿಗಳಿಗಿಂತ ಮೇಲೆ ಇರುವಿರಿ. ನಿಮ್ಮಿಂದ ಯಾರೂ ಅಪಹರಿಸಲಾರದಂತಹ ಅನುಭವವೊಂದು ನಿಮಗೆ ಬರುವುದು. ದೇಹಕ್ಕಿಂತ ಅತೀತವಾದುದರ ಅನುಭವ ಆಗುವುದು. ಅಲ್ಲಿಯೇ ಮನಸ್ಸಿನ ಗುರಿ ಇರುವುದು, ಇಂದ್ರಿಯ ಪ್ರಪಂಚದಲ್ಲಿ ಅಲ್ಲ.

ಇಂದ್ರಿಯದ ಕ್ಷೇತ್ರದ ಮೂಲಕ ಕೆಲಸಮಾಡುತ್ತಿದ್ದರೆ, ನಿಮಗೆ ಬೇರೆ ಬೇರೆ ಕ್ಷೇತ್ರಗಳಿಗೂ ಹೋಗಲು ಸಾಧ್ಯವಾಗುವುದು. ಅನಂತರ ಈ ಪ್ರಪಂಚ ನಿಮ್ಮಿಂದ ಮಾಯವಾಗುವುದು. ಆಗ ಅನಂತಾತ್ಮನ ಕ್ಷಣಿಕ ದರ್ಶನವೊಂದು ನಿಮಗೆ ಸಿಕ್ಕುವುದು. ಅನಂತರ ನಿಮ್ಮ ಇಂದ್ರಿಯಗಳು, ಇಂದ್ರಿಯ ಸುಖ, ನಿಮ್ಮಲ್ಲಿರುವ ದೇಹಾಸಕ್ತಿ ಇವುಗಳೆಲ್ಲ ನಿಮ್ಮಿಂದ ಮಾಯವಾಗುವುವು. ಆಧ್ಯಾತ್ಮಿಕ ಪ್ರಪಂಚದಿಂದ ದೃಶ್ಯಗಳಾದ ಮೇಲೆ ದೃಶ್ಯಗಳು ಕಾಣತೊಡಗುವುವು. ಆಗ ನೀವು ಯೋಗವನ್ನು ಹೊಂದುವಿರಿ. ಅಧ್ಯಾತ್ಮ ಅಧ್ಯಾತ್ಮದಂತೆ ನಿಮ್ಮ ಕಣ್ಣಮುಂದೆ ನಿಲ್ಲುವುದು. ಆಗ ನೀವು ದೇವರನ್ನು ಅಧ್ಯಾತ್ಮದಂತೆ ಆರಾಧಿಸಬಲ್ಲಿರಿ. ಆಗ ಪೂಜೆ ಎಂದರೆ ಯಾವುದೋ ಲೌಕಿಕವಾದುದನ್ನು ಪಡೆಯುವುದಲ್ಲ ಎಂಬುದು ಅರ್ಥವಾಗುವುದು. ಯಥಾರ್ಥವಾಗಿ ನಮ್ಮ ಪೂಜೆ ಅನಂತಸಾಂತದ ಪೂಜೆಯಾಗಿತ್ತು. ಭಗವಂತನ ಪಾದಾರವಿಂದಗಳಲ್ಲಿ ಜೀವಾತ್ಮ ತನ್ನ ಪ್ರೇಮವನ್ನು ನಿತ್ಯ ನಿವೇದಿಸುತ್ತಿರುವುದು. “ನೀನು, ನಾನಲ್ಲ. ನಾನೆಂಬುದು ನಾಶವಾಯಿತು. ನೀನೇ ಇರುವುದು ನಾನಲ್ಲ. ನನಗೆ ಐಶ್ವರ್ಯ ಬೇಡ. ಸೌಂದರ್ಯ ಬೇಡ, ಪಾಂಡಿತ್ಯವೂ ಬೇಡ, ನನಗೆ ಮುಕ್ತಿಯೂ ಬೇಡ. ನಿನ್ನ ಇಚ್ಚೆಯಾದರೆ ಲಕ್ಷಾಂತರ ನರಕಗಳಿಗೆ ಬೇಕಾದರೂ ಹೋಗುತ್ತೇನೆ. ನನಗೆ ಬೇಕಾಗಿರುವುದೊಂದೇ. ಅದೇ, ನೀನೇ ಎಂದೆಂದಿಗೂ ನನ್ನ ಪ್ರೇಮೇಶ್ವರನಾಗಿರುವುದು.”

