
\chapter[ಸಾಕ್ಷಾತ್ಕಾರಕ್ಕೆ ಮೆಟ್ಟಿಲುಗಳು]{ಸಾಕ್ಷಾತ್ಕಾರಕ್ಕೆ ಮೆಟ್ಟಿಲುಗಳು\protect\footnote{\engfoot{C.W. Vol.1, P. 405}}}

\begin{center}
(ಅಮೆರಿಕದಲ್ಲಿ ಒಂದು ತರಗತಿಯಲ್ಲಿ ನೀಡಿದ ಪ್ರವಚನ)
\end{center}

ಜ್ಞಾನವನ್ನು ಪಡೆಯಬೇಕೆಂಬ ಸಾಧಕನಿಗೆ ಮೊದಲು ಆವಶ್ಯಕವಾಗಿ ಬೇಕಾಗಿರುವುದು ಶ್ರಮ ಮತ್ತು ದಮ. ಇವೆರಡನ್ನೂ ಒಟ್ಟಿಗೆ ಪರಿಗಣಿಸಬಹುದು. ಶಮ ದಮಗಳೆಂದರೆ ಇಂದ್ರಿಯಗಳನ್ನು ಹೊರಗೆ ಹೋಗದಂತೆ ಅವುಗಳ ಕೇಂದ್ರದಲ್ಲಿ ನಿಲ್ಲಿಸುವುದು ಎಂದು ಅರ್ಥ. ಮೊದಲು ಇಂದ್ರಿಯಗಳೆಂದರೆ ಏನೆಂಬುದನ್ನು ವಿವರಿಸುತ್ತೇನೆ. ಇಲ್ಲಿ ಕಣ್ಣು ಇದೆ. ಕಣ್ಣು ನೋಡುವ ಇಂದ್ರಿಯವಲ್ಲ. ಇದು ನೋಡುವುದಕ್ಕೆ ಇರುವ ಒಂದು ಉಪಕರಣ ಮಾತ್ರ. ನನಗೆ ಕಣ್ಣು ಇದ್ದರೂ ಅದಕ್ಕೆ ಸಂಬಂಧಪಟ್ಟ ಇಂದ್ರಿಯ ಇಲ್ಲದೇ ಇದ್ದರೆ ನೋಡುವುದಕ್ಕೆ ಆಗುವುದಿಲ್ಲ. ಇಂದ್ರಿಯ ಮತ್ತು ಉಪಕರಣ ಎರಡೂ ಇದ್ದರೂ ಇದರ ಹಿಂದೆ ಮನಸ್ಸು ಇಲ್ಲದೇ ಇದ್ದರೆ ನನಗೆ ನೋಡುವುದಕ್ಕೆ ಆಗುವುದಿಲ್ಲ. ಪ್ರತಿಯೊಂದು ಇಂದ್ರಿಯ-ಗ್ರಹಣದ ಕ್ರಿಯೆಯಲ್ಲಿಯೂ ಇವು ಮೂರೂ ಆವಶ್ಯಕ - ಮೊದಲು ಹೊರಗಿನ ಉಪಕರಣ, ಅನಂತರ ಒಳಗಿನ ಇಂದ್ರಿಯ, ಕೊನೆಗೆ ಮನಸ್ಸು. ಇವುಗಳಲ್ಲಿ ಯಾವುದಾದರೊಂದು ಇಲ್ಲದೇ ಇದ್ದರೂ ಇಂದ್ರಿಯ ಗ್ರಹಣದ ಕ್ರಿಯೆ ನಡೆಯಲಾರದು. ಮನಸ್ಸು ಹೀಗೆ ಒಂದು ಬಾಹ್ಯ ಮತ್ತೊಂದು ಆಂತರಿಕ ಸಹಾಯಗಳಿಂದ ಕೆಲಸ ಮಾಡುವುದು. ನಾನು ನೋಡುವಾಗ ಮನಸ್ಸು ಹೊರಗೆ ಹೋಗುವುದು, ಬಾಹ್ಯವಾಗುವುದು. ಆದರೆ ನಾನು ಕಣ್ಣುಗಳನ್ನು ಮುಚ್ಚಿಕೊಂಡು ಆಲೋಚಿಸ ತೊಡಗಿದರೆ ಮನಸ್ಸು ಹೊರಗೆ ಹೋಗುವುದಿಲ್ಲ; ಒಳಗೆ ಸಂಚರಿಸುತ್ತಿರುವುದು. ಆದರೂ ಎರಡು ಪ್ರಸಂಗಗಳಲ್ಲಿಯೂ ಇಂದ್ರಿಯ ಕೆಲಸ ಮಾಡುತ್ತಿರುವುದು. ನಾನು ನಿಮ್ಮನ್ನು ನೋಡುತ್ತ ನಿಮ್ಮೊಂದಿಗೆ ಮಾತನಾಡುತ್ತಿದ್ದರೆ ಇಂದ್ರಿಯ ಮತ್ತು ಉಪಕರಣ ಎರಡೂ ಕೆಲಸಮಾಡುತ್ತವೆ. ನಾನು ಕಣ್ಣುಗಳನ್ನು ಮುಚ್ಚಿಕೊಂಡು ಆಲೋಚಿಸುತ್ತಿದ್ದರೆ, ಕೆಲಸ ಮಾಡುವುದು ಇಂದ್ರಿಯ, ಉಪಕರಣವಲ್ಲ. ಇಂದ್ರಿಯಗಳ ಕ್ರಿಯೆ ಇಲ್ಲದೇ ಇದ್ದರೆ ಆಲೋಚನೆ ಸಾಧ್ಯವಿಲ್ಲ. ಯಾವ ವಿಧವಾದ ಸಂಕೇತವೂ ಇಲ್ಲದೇ ಇದ್ದರೆ ಆಲೋಚಿಸುವುದಕ್ಕೆ ನಿಮಗೆ ಅಸಾಧ್ಯವೆಂದು ತೋರುವುದು. ಕುರುಡನೂ ಕೂಡ ಯಾವುದಾದರೊಂದು ಆಕಾರದ ಮೂಲಕ ಚಿಂತಿಸಬೇಕಾಗಿದೆ.ನೋಡುವ ಮತ್ತು ಕೇಳುವ ಇಂದ್ರಿಯಗಳು ಯಾವಾಗಲೂ ಚಟುವಟಿಕೆಯುಳ್ಳವುಗಳಾಗಿರುತ್ತವೆ. ಇಂದ್ರಿಯ ಎಂದರೆ ಮಿದುಳಿನಲ್ಲಿರುವ ನರಗಳ ಕೇಂದ್ರವೆಂಬುದನ್ನು ನೀವು ಜ್ಞಾಪಕದಲ್ಲಿಡಬೇಕು. ಕಣ್ಣು ಮತ್ತು ಕಿವಿ ಕೇವಲ ನೋಡುವ ಮತ್ತು ಕೇಳುವ ಉಪಕರಣಗಳು ಮಾತ್ರ. ಇಂದ್ರಿಯಗಳು ಒಳಗೆ ಇರುವುವು. ಒಳಗಡೆ ಇರುವ ಇಂದ್ರಿಯಗಳು ನಾಶವಾದರೆ, ನಮಗೆ ಕಣ್ಣು ಮತ್ತು ಕಿವಿ ಇದ್ದರೂ ನೋಡಲು ಅಥವಾ ಕೇಳಲು ಆಗುವುದಿಲ್ಲ. ಆದ್ದರಿಂದ ನಾವು ಮನಸ್ಸನ್ನು ನಿಗ್ರಹಿಸಬೇಕಾದರೆ, ಮೊದಲು ಈ ಆಂತರಿಕ -ಇಂದ್ರಿಯ ನಿಗ್ರಹ ಸಾಧ್ಯವಾಗಬೇಕು. ಮನಸ್ಸು ಒಳಗೆ ಅಥವಾ ಹೊರಗೆ ಸಂಚರಿಸದಂತೆ ಮಾಡಿ, ಆಯಾಯಾ ಇಂದ್ರಿಯಗಳು ಅವುಗಳ ಕೇಂದ್ರದಲ್ಲಿ ಇರುವಂತೆ ಮಾಡುವುದೇ ಶಮ ಮತ್ತು ದಮ ಎನ್ನುವುದು. ಶಮ ಎಂದರೆ ಮನಸ್ಸನ್ನು ಹೊರಗೆ ಹೋಗದಂತೆ ತಡೆಯುವುದು, ದಮವೆಂದರೆ ಹೊರಗಿನ ಉಪಕರಣವನ್ನು ನಿಗ್ರಹಿಸುವುದು.

ಉಪರತಿ ಅನಂತರ ಬರುವುದು. ಉಪರತಿ ಎಂದರೆ ಇಂದ್ರಿಯಕ್ಕೆ ಸಂಬಂಧಪಟ್ಟ ವಿಷಯಗಳನ್ನು ಆಲೋಚಿಸದೆ ಇರುವುದು. ನಮ್ಮ ಕಾಲದ ಬಹು ಭಾಗವನ್ನು ಇಂದ್ರಿಯಗಳಿಗೆ ಸಂಬಂಧಪಟ್ಟ ವಿಷಯಗಳನ್ನು ಕುರಿತು ಆಲೋಚಿಸುವುದರಲ್ಲಿ ಕಳೆಯುವೆವು. ನಾವು ಹಿಂದೆ ನೋಡಿದ ಅಥವಾ ಕೇಳಿದ, ಮುಂದೆ ನೋಡಬಹುದಾದ ಅಥವಾ ಕೇಳಬಹುದಾದ ವಿಷಯಗಳು, ನಾವು ಹಿಂದೆ ತಿಂದ, ಈಗ ತಿನ್ನುತ್ತಿರುವ, ಅಥವಾ ಮುಂದೆ ತಿನ್ನಬಹುದಾದ ತಿಂಡಿಗಳು, ನಾವು ಇದ್ದ ಸ್ಥಳ ಮುಂತಾದುವನ್ನೆಲ್ಲ ಕುರಿತು ಚಿಂತಿಸುತ್ತಿರುವೆವು. ಬಹಳಕಾಲ ಅವುಗಳ ವಿಚಾರವಾಗಿ ಆಲೋಚಿಸುವೆವು ಅಥವಾ ಮಾತನಾಡುವೆವು. ಯಾರು ವೇದಾಂತಿಗಳಾಗಬೇಕೆಂದು ಬಯಸುವರೋ ಅವರು ಈ ಸ್ವಭಾವದಿಂದ ಪಾರಾಗಬೇಕು.

ತಿತಿಕ್ಷೆಯೇ ಅನಂತರದ ಸಿದ್ಧತೆ. ಎಲ್ಲಕ್ಕಿಂತ ಕಷ್ಟ ಇದು. ತತ್ತ್ವಜ್ಞಾನಿಯಾಗುವುದು ಬಹಳ ಕಷ್ಟ! ಇದು ಸಹಿಷ್ಣುತೆ; ಪಾಪವನ್ನು ಎದುರಿಸದೆ ಇರುವುದು ಎಂಬ ಭಾವಕ್ಕಿಂತ ಇದು ಕಡಮೆ ಏನೂ ಅಲ್ಲ. ಇದನ್ನು ತಿಳಿದುಕೊಳ್ಳಬೇಕಾದರೆ ಸ್ವಲ್ಪ ವಿವರಣೆ ಆವಶ್ಯಕ. ನಾವು ಪಾಪವನ್ನು ವಿರೋಧಿಸದೆ ಇರಬಹುದು. ಆದರೆ ಅದೇ ಸಮಯದಲ್ಲಿ ತುಂಬ ದುಃಖಿಗಳಾಗಬಹುದು. ಒಬ್ಬ ನನ್ನನ್ನು ಅತಿ ಹೀನವಾಗಿ ನಿಂದಿಸುತ್ತಾನೆ ಎಂದು ಭಾವಿಸೋಣ. ಇದಕ್ಕಾಗಿ ನಾನು ಹೊರಗೆ ಅವನ ಮೇಲೆ ಕೋಪ ತೋರದೆ ಇರಬಹುದು, ಮಾರುತ್ತರವನ್ನು ಕೊಡದೆ ದ್ವೇಷವನ್ನು ತಾಳದೆ ಇರಬಹುದು. ಆದರೂ ನನ್ನ ಮನಸ್ಸಿನಲ್ಲಿ ಕೋಪದೋಷಗಳು ಅವನ ಮೇಲೆ ಇರಬಹುದು; ಅವನನ್ನು ಕಂಡರೆ ನನಗೆ ಆಗದೆ ಇರಬಹುದು. ಇದು ತಿತಿಕ್ಷೆಯಲ್ಲ. ನನ್ನ ಮನಸ್ಸಿನಲ್ಲಿ ಯಾವ ದ್ವೇಷದ ಅಥವಾ ಕೋಪದ ಭಾವನೆಯಾಗಲಿ, ಸೇಡನ್ನು ತೀರಿಸಿಕೊಳ್ಳುವ ಭಾವನೆಯಾಗಲಿ ಇರಕೂಡದು. ಏನೂ ಆಗಿಯೇ ಇಲ್ಲವೋ ಎಂಬಂತೆ ಮನಸ್ಸು ಶಾಂತವಾಗಿರಬೇಕು. ನಾನು ಆ ಸ್ಥಿತಿಗೆ ಬಂದಾಗ ಮಾತ್ರ ನನಗೆ ತಿತಿಕ್ಷೆಯ ಸಿದ್ದಿಗಿದೆ, ಅದಕ್ಕೆ ಮುಂಚೆ ಇಲ್ಲ. ಬರುವ ದುಃಖವನ್ನೆಲ್ಲ ಸಹಿಸಬೇಕು, ಅದನ್ನು ವಿರೋಧಿಸಿ ಆಚೆಗೆ ಓಡಿಸಲು ಕೂಡ ಪ್ರಯತ್ನಪಡದೆ ಇರಬೇಕು. ಮನಸ್ಸಿನ ಒಳಗೆ ಯಾವ ನೋವಿನ ಭಾವನೆಯಾಗಲಿ, ಪರಿತಾಪವಾಗಲಿ, ಇರಕೂಡದು. ಇದು ತಿತಿಕ್ಷೆ. ನಾನು ಎದುರಿಸುವುದಿಲ್ಲ ಎಂದು ಭಾವಿಸೋಣ. ಇದರಿಂದ ದಾರುಣ ದುಃಖ ಪ್ರಾಪ್ತವಾಗಬಹುದು. ಆಗ ಅದನ್ನು ಎದುರಿಸದೆ ಇದ್ದುದು ತಪ್ಪಾಯಿತು ಎಂಬ ಯಾವ ಭಾವನೆಯೂ ಬರಕೂಡದು. ಮನಸ್ಸು ಈ ಸ್ಥಿತಿಗೆ ಬಂದಾಗ ಮಾತ್ರ ಅದು ತಿತಿಕ್ಷೆಯಲ್ಲಿ ನೆಲಸಿರುವುದು. ಈ ತಿತಿಕ್ಷೆಯನ್ನು ಸಾಧಿಸುವುದಕ್ಕಾಗಿ ಇಂಡಿಯಾದೇಶದಲ್ಲಿ ಜನರು ಅದ್ಭುತ ಕೆಲಸಗಳನ್ನು ಮಾಡುವರು; ಮಿತಿಮೀರಿದ ಶೀತೋಷ್ಣಗಳನ್ನು\break ಸಹಿಸುವರು; ಹಿಮವನ್ನು ಕೂಡ ಅವರು ಲೆಕ್ಕಿಸುವುದಿಲ್ಲ. ದೇಹದ ಭಾವನೆಯೇ ಅವರಿಗಿಲ್ಲ. ಅದೊಂದು ಬಾಹ್ಯವಸ್ತುವಿನಂತೆ ಅದರ ಪಾಡಿಗೆ ಅದನ್ನು ಬಿಡುವರು.

ಶ್ರದ್ದೆಯೇ ಅನಂತರ ಬೇಕಾದ ಗುಣ. ಒಬ್ಬನಿಗೆ ದೇವರಲ್ಲಿ ಮತ್ತು ಧರ್ಮದಲ್ಲಿ ಅದ್ಭುತವಾದ ನಂಬಿಕೆ ಇರಬೇಕು. ಈ ನಂಬಿಕೆ ಹುಟ್ಟುವವರೆಗೆ ಅವನು ಜ್ಞಾನಿಯಾಗಲಾರ. ಒಬ್ಬರು ದೊಡ್ಡ ಯೋಗಿಗಳು ನನಗೆ ಒಮ್ಮೆ, ಇಪ್ಪತ್ತು ಕೋಟಿಯಲ್ಲಿ ಒಬ್ಬನು ಕೂಡ ದೇವರನ್ನು ನಂಬುವುದಿಲ್ಲ ಎಂದರು. ಏತಕ್ಕೆ ಎಂದು ನಾನು ಕೇಳಿದೆ. ಅದಕ್ಕೆ ಅವರು ಹೀಗೆ ಹೇಳಿದರು: “ಈ ಕೋಣೆಯಲ್ಲಿ ಒಬ್ಬ ಕಳ್ಳ ಇರುವನು ಎಂದು ಭಾವಿಸೋಣ. ಪಕ್ಕದ ಕೋಣೆಯಲ್ಲಿ ಅವನಿಗೆ ಹೊನ್ನಿನ ರಾಶಿ ಇದೆ ಎಂದು ಗೊತ್ತಿದೆ. ಆ ಕೋಣೆಗೂ ಈ ಕೋಣೆಗೂ ಮಧ್ಯೆ ಒಂದು ಸಣ್ಣ ಗೋಡೆ ಮಾತ್ರ ಇದೆ. ಇಂತಹ ಸ್ಥಿತಿಯಲ್ಲಿ ಅವನ ಮನಸ್ಸು ಹೇಗಿರುವುದು ಗೊತ್ತೆ??” ಅದಕ್ಕೆ ನಾನು, “ಅವನಿಗೆ ನಿದ್ದೆ ಮಾಡುವುದಕ್ಕೆ ಆಗುವುದಿಲ್ಲ. ಹೇಗೆ ಹೊನ್ನನ್ನು ಪಡೆಯಬಹುದೆಂಬುದನ್ನು ಕುರಿತು ಅವನು ಆಲೋಚಿಸುತ್ತಿರುವನು. ಮತ್ತಾವ ವಿಷಯವೂ ಅವನ ಮನಸ್ಸಿಗೆ ಬರುವುದಿಲ್ಲ” ಎಂದೆ. ಅನಂತರ ಅವರು ಹೀಗೆ ಹೇಳಿದರು: “ಮನುಷ್ಯ ದೇವರನ್ನು ನಂಬಿ ಅವನನ್ನು ಪಡೆಯಲು ಹುಚ್ಚನಾಗಿ ಹೋಗುವುದಿಲ್ಲವೆಂದು ಭಾವಿಸುವಿಯೇನು? ಒಬ್ಬನಲ್ಲಿ ನಿಜವಾಗಿ ಅಪಾರವಾದ ಅನಂತಾನಂದದ ರಾಶಿ ಇದೆ; ಅದನ್ನು ಬೇಕಾದರೆ ಪಡೆಯಬಹುದು ಎಂದು ಗೊತ್ತಾದರೆ, ಅದನ್ನು ಹೊಂದಲು ಅವನು ಹುಚ್ಚನಾಗಿ ಹೋಗುವುದಿಲ್ಲವೆ?'' ದೇವರಲ್ಲಿ ದೃಢನಂಬಿಕೆ, ಅವನನ್ನು ಸೇರಲು ತೀವ್ರ ತವಕ - ಇವನ್ನೇ ಶ್ರದ್ದೆ ಎನ್ನುವುದು.

ಅನಂತರ ಸಮಾಧಾನ ಅಥವಾ ಅನವರತವೂ ಚಿತ್ತವು ಭಗವಂತನಲ್ಲಿ ನೆಲಸುವಂತೆ ಮಾಡುವ ಅಭ್ಯಾಸ. ಯಾವುದನ್ನೂ ಒಂದು ದಿನದಲ್ಲಿ ಸಾಧಿಸುವುದಕ್ಕೆ ಆಗುವುದಿಲ್ಲ. ಆಧ್ಯಾತ್ಮಿಕತೆಯನ್ನು ಗುಳಿಗೆಯಂತೆ ನುಂಗಲು ಆಗುವುದಿಲ್ಲ. ಅದಕ್ಕೆ ಕಡುಕಷ್ಟದ ಸತತ ಪ್ರಯತ್ನ ಆವಶ್ಯಕ. ನಿಧಾನವಾದ ಮತ್ತು ದೃಢವಾದ ಅಭ್ಯಾಸದಿಂದ ಮಾತ್ರ ಮನಸ್ಸನ್ನು ಗೆಲ್ಲುವುದು ಸಾಧ್ಯ.

ಮುಮುಕ್ಷುತ್ವ ಅಥವಾ ಮುಕ್ತಿಯನ್ನು ಪಡೆಯಲು ಉತ್ಸಾಹ ಅನಂತರ ಬರುವುದು. ನಿಮ್ಮಲ್ಲಿ ಯಾರು ಎಡ್ವಿನ್ ಆರ್ನಾಲ್ಡ್ ನ “ಏಷ್ಯಾದ ಜ್ಯೋತಿ”ಯನ್ನು ಓದಿರುವಿರೋ, ಅವರು ಅದರಲ್ಲಿ ಬರುವ ಬುದ್ಧನ ಮೊದಲನೆಯ ಬೋಧನೆಯನ್ನು ನೆನಪಿನಲ್ಲಿಟ್ಟಿರಬಹುದು. “ನಿಮ್ಮಿಂದಲೇ ನೀವು ದುಃಖಕ್ಕೆ ಬೀಳುವಿರಿ. ನಿಮ್ಮನ್ನು ಯಾರೂ ಬಲಾತ್ಕರಿಸುವುದಿಲ್ಲ. ನಿಮ್ಮನ್ನು ಯಾರೂ ಎತ್ತಿ ಹಿಡಿಯುವುದಿಲ್ಲ. ನೀವೇ ಹುಟ್ಟುವಿರಿ, ನೀವೇ ಸಂಸಾರಚಕ್ರದಲ್ಲಿ ಉರುಳುತ್ತಿರುವಿರಿ. ದುಃಖವೆಂಬ ಚಕ್ರದ ಅರೆಗಳನ್ನು ನೀವೇ ಹಿಡಿದುಕೊಂಡು ಅದರ ಗಾಲಿಯನ್ನು ಮತ್ತು ಶೂನ್ಯವೆಂಬ ಅದರ ಕೇಂದ್ರವನ್ನು ಚುಂಬಿಸುತ್ತಿರುವಿರಿ.”

ನಮ್ಮ ದುಃಖಗಳನ್ನೆಲ್ಲ ನಾವೇ ಆರಿಸಿಕೊಂಡದ್ದು. ನಮ್ಮ ಸ್ವಭಾವವೇ ಅಂತಹದು. ಅರವತ್ತು ವರುಷಗಳವರೆಗೆ ಸೆರೆಮನೆಯಲ್ಲಿ ಕೊಳೆಯುತ್ತಿದ್ದ ಒಬ್ಬ ಚೈನಾದೇಶದ ಬಂದಿಯನ್ನು, ಹೊಸ ಚಕ್ರವರ್ತಿಗೆ ಪಟ್ಟಾಭಿಷೇಕವಾದಾಗ ಬಿಡುಗಡೆ ಮಾಡಿದರು. ಆತ ಹೊರಗೆ ಬಂದಮೇಲೆ ಅಲ್ಲಿ ತಾನು ಬಾಳಲಾರೆನೆಂದು ಅರಚಿಕೊಂಡನು. ಇಲಿ\break ಹೆಗ್ಗಣಗಳಿಂದ ತುಂಬಿದ ಸೆರೆಮನೆಗೆ ತಾನು ಪುನಃ ಹೋಗಬೇಕು, ಹೊರಗಿನ ಪ್ರಕಾಶವನ್ನು ಸಹಿಸಲು ತನಗೆ ಅಸಾಧ್ಯವೆಂದನು. ದಯವಿಟ್ಟು ತನ್ನನ್ನು ಬೇಕಾದರೆ ಕೊಲ್ಲಬಹುದು, ಇಲ್ಲವೆ ಸೆರೆಮನೆಗೆ ಕಳುಹಿಸಬೇಕೆಂದು ಬೇಡಿಕೊಂಡನು. ಅವನನ್ನು ಪುನಃ ಸೆರೆಮನೆಗೆ ಕಳುಹಿಸಿದರು. ಎಲ್ಲಾ ಜನರ ಸ್ಥಿತಿಯೂ ಹೀಗೆ. ನಾವು ಸ್ವಲ್ಪವೂ ಯೋಚಿಸದೆ ದುಃಖವನ್ನು ಹಿಂಬಾಲಿಸುವೆವು. ಅದರಿಂದ ಪಾರಾಗುವ ಇಚ್ಛೆಯೇ ನಮಗೆ ಇಲ್ಲ. ಪ್ರತಿದಿನವೂ ನಾವು ಸುಖವನ್ನು ಬೆನ್ನಟ್ಟುವೆವು. ನಾವು ಅದನ್ನು ಪಡೆಯುವುದಕ್ಕೆ ಮುಂಚೆಯೇ ಅದು ನಮ್ಮಿಂದ ಮಾಯವಾಗಿರುವುದು. ನಮ್ಮ ಬೆರಳುಗಳ ಸಂದಿನಿಂದ ಅದು ನುಸುಳಿಕೊಂಡು ಹೋಗಿರುವುದು. ಆದರೂ ಅದನ್ನು ಅನುಸರಿಸುವ ಭ್ರಾಂತಿ ನಮ್ಮನ್ನು ಬಿಡದು. ಮುಂದೆ ಮುಂದೆ, ಕಡುಮೂರ್ಖರಾದ ಕುರುಡರಂತೆ ನಾವು ಅದನ್ನು ಹಿಂಬಾಲಿಸುವೆವು.

ಭರತಖಂಡದಲ್ಲಿ ಎಣ್ಣೆಯಾಡಿಸಲು ಗಾಣಕ್ಕೆ ಎತ್ತನ್ನು ಕಟ್ಟುವರು. ಎತ್ತಿನ ಕೊರಳಿನ ಮೇಲೆ ನೊಗವಿರುವುದು. ಆ ನೊಗದ ಮುಂದೆ ಚಾಚಿರುವ ಕೋಲಿಗೆ ಸ್ವಲ್ಪ ಹುಲ್ಲನ್ನು ಕಟ್ಟುವರು. ಎತ್ತಿನ ಕಣ್ಣು ಮುಂದೆ ಮಾತ್ರ ಸ್ವಲ್ಪ ಕಾಣುವಂತೆ ಮಾಡಿರುವರು. ಅದು ಮುಂದಿರುವ ಹುಲ್ಲನ್ನು ತಿನ್ನಲು ಕತ್ತನ್ನು ಚಾಚಿ ಮುಂದೆ ಹೋಗುವುದು. ಆಗ ನೊಗದ ಮುಂದಿರುವ ಹುಲ್ಲೂ ಮುಂದೆ ಸಾಗುವುದು. ಹೀಗೆ ಅದು ಹುಲ್ಲನ್ನು ಹಿಡಿಯಲು ಎಷ್ಟು ಪ್ರಯತ್ನಪಟ್ಟರೂ ಅದು ಮುಂದೆ ಮುಂದೆ ಹೋಗುತ್ತಿರುವುದು. ಅದಕ್ಕೆ ಹುಲ್ಲು ಸಿಕ್ಕುವುದೇ ಇಲ್ಲ. ಇದನ್ನು ಪಡೆಯುವ ಆಸೆಯಿಂದ ಸುತ್ತಲೂ ಹೋಗುವುದು. ಹೀಗೆ ಮಾಡಿ ಮಾಡಿ ಎಣ್ಣೆ ಬರುವುದಕ್ಕೆ ಗಾಣವನ್ನು ಮಾತ್ರ ಆಡಿಸುವುದು. ಇದರಂತೆಯೇ ನಾವು ಪ್ರಕೃತಿ, ಐಶ್ವರ್ಯ, ಹೆಂಡತಿ, ಮಕ್ಕಳು ಇವರಿಗೆ ದಾಸರು. ಕೇವಲ ಭ್ರಾಂತಿಯ ಹುಲ್ಲನ್ನು ಬೆನ್ನಟ್ಟಿ ಹೋಗುತ್ತಿರುವೆವು. ನಮ್ಮ ಕೈಗಳಿಗೆ ಅದು ಸಿಕ್ಕದೆ ಹಲವಾರು ಜನ್ಮಗಳವರೆಗೆ ಸುತ್ತುತ್ತಿರುವೆವು. ನಮ್ಮ ಒಂದು ದೊಡ್ಡ ಕನಸೇ ಪ್ರೀತಿ. ನಾವು ಒಬ್ಬರನ್ನು ಪ್ರೀತಿಸುತ್ತೇವೆ. ಅವರಿಂದ ಪ್ರೀತಿಯನ್ನು ನಿರೀಕ್ಷಿಸುತ್ತೇವೆ. ನಮಗೆಲ್ಲ ಸುಖ ಬೇಕು, ದುಃಖ ಬೇಡ. ಅಂದರೆ ಸುಖವನ್ನು ನಾವು ಸಮೀಪಿಸಿದಷ್ಟೂ ಅದು ನಮ್ಮಿಂದ ದೂರ ಸರಿಯುವುದು. ಪ್ರಪಂಚ ಹೀಗೆಯೇ ನಡೆಯುವುದು, ಸಮಾಜ ಹೀಗೆಯೇ ಸಾಗುವುದು. ನಾವು ಕಣ್ಣಿಲ್ಲದ ಗುಲಾಮರು, ವಸ್ತುಗಳ ಸ್ಥಿತಿಯನ್ನು ಅರಿಯದೆ ದುಃಖಪಡಬೇಕಾಗಿದೆ. ನಿಮ್ಮ ಜೀವನವನ್ನು ಪರೀಕ್ಷಿಸಿ ನೋಡಿ. ಅದರಲ್ಲಿ ಸುಖ ಎಷ್ಟು ಕಡಮೆ ಇದೆ! ಪ್ರಪಂಚವೆಂಬ ಮಾಯಾ ಮೃಗದ ಬೇಟೆಯಿಂದ ನಮಗೆ ಎಷ್ಟು ಸುಖ ಸಿಕ್ಕಿದೆ? ಇದನ್ನು ವಿಮರ್ಶಿಸಿ ನೋಡಿ.

ನಿಮಗೆ ಸೋಲಾನ್ ಮತ್ತು ಕ್ರೊಯಿಸಸ್ ಇವರ ಕಥೆ ಜ್ಞಾಪಕವಿದೆಯೆ?\break ಕ್ರೊಯಿಸಸ್ಸಿನ ದೊರೆ ಸೋಲಾನ್ ಎಂಬ ಜ್ಞಾನಿಗೆ ಏಷ್ಯಮೈನರ್ ಅತಿ ಸುಖಕರವಾದ ಸ್ಥಳವೆಂದನು. ಜ್ಞಾನಿ ಅವನಿಗೆ “ಪ್ರಪಂಚದಲ್ಲಿ ಯಾರು ಅತ್ಯಂತ ಸುಖಿ? ನಾನು ಇದುವರೆಗೆ ಯಾರನ್ನೂ ನೋಡಿಲ್ಲ” ಎಂದನು. ರಾಜ “ಎಂತಹ ಅವಿವೇಕಿ ನೀನು! ಪ್ರಪಂಚದಲ್ಲಿ ನಾನು ಅತ್ಯಂತ ಸುಖಿ'' ಎಂದನು. ಜ್ಞಾನಿ, “ನಿಮ್ಮ ಜೀವನದ ಕೊನೆಯವರೆವಿಗೂ ಸ್ವಲ್ಪ ತಾಳಿ ಸ್ವಾಮಿ; ಅವಸರಪಡಬೇಡಿ” ಎಂದು ಹೊರಟು ಹೋದನು. ಕೆಲವು ವರ್ಷಗಳಾದಮೇಲೆ ಪರ್ಷಿಯನ್ನರು ರಾಜನನ್ನು ಗೆದ್ದರು. ಜೀವಸಹಿತ ಅವನನ್ನು ಸುಡುವಂತೆ ಆಜ್ಞಾಪಿಸಿದರು. ಚಿತೆ ಅಣಿಯಾಯಿತು. ಅದನ್ನು ಕ್ರೋಯಿಸಸ್ಸನು ಕಂಡಾಗ “ಸೋಲಾನ್! ಸೋಲಾನ್!?” ಎಂದು ದುಃಖದಿಂದ ಅರಚಿಕೊಂಡನು. ಯಾರನ್ನು ನೀನು ಕೂಗುತ್ತಿರುವೆ ಎಂದು ಪ್ರಶ್ನೆ ಮಾಡಿದಾಗ, ಅವನು ತನ್ನ ಹಿಂದಿನ ಕಥೆಯನ್ನು ಹೇಳಿಕೊಂಡನು. ಪರ್ಷಿಯ ಚಕ್ರವರ್ತಿಯ ಹೃದಯ ಕರಗಿ ಅವನ ಪ್ರಾಣವನ್ನು ಉಳಿಸಿದನು.

ಇದೇ ನಮ್ಮೆಲ್ಲರ ಜೀವನದ ಕಥೆ, ಪ್ರಕೃತಿಗೆ ನಮ್ಮ ಮೇಲಿರುವ ಅದ್ಭುತ ಶಕ್ತಿಯೇ ಇದು. ಪ್ರಕೃತಿ ಮರಳಿ ಮರಳಿ ನಮ್ಮನ್ನು ಒದೆಯುವುದು. ಆದರೂ ಅತಿ ಅವಾಂತರದಿಂದ ಅದನ್ನು ಹಿಂಬಾಲಿಸುವೆವು. ನಾವು ಯಾವಾಗಲೂ ನಿರಾಶೆಯಲ್ಲಿ ಆಸೆಯನ್ನು ಮೆಲ್ಲುತ್ತಿರುವೆವು. ಈ ನಿರೀಕ್ಷಣೆ ನಮ್ಮನ್ನು ಭ್ರಾಂತರನ್ನಾಗಿ ಮಾಡಿರುವುದು. ನಾವೆಲ್ಲ ಸುಖದ ಭರವಸೆಯನ್ನು ಆಶಿಸುತ್ತಿರುವೆವು.

ಪುರಾತನ ಭರತಖಂಡದಲ್ಲಿ ಒಬ್ಬ ರಾಜನಿದ್ದ. ಅವನಿಗೆ ನಾಲ್ಕು ಪ್ರಶ್ನೆಗಳನ್ನು ಹಾಕಿದರು. ಅದರಲ್ಲಿ ಒಂದು: “ಪ್ರಪಂಚದಲ್ಲಿ ಯಾವುದು ಅತ್ಯಂತ ವಿಚಿತ್ರವಾದುದು?" ಎಂಬುದು. ಅದಕ್ಕೆ ಉತ್ತರ “ಭರವಸೆ”. ಇದು ಅತ್ಯದ್ಭುತವಾದ ಉತ್ತರ. ಹಗಲು ರಾತ್ರಿ ನಮ್ಮ ಸುತ್ತಲೂ ಜನರು ಸಾಯುವುದನ್ನು ನೋಡುವೆವು. ಆದರೂ ನಾವು ಸಾಯುವುದಿಲ್ಲವೆಂದು ಭಾವಿಸುವೆವು. ನಾವು ಸಾಯುತ್ತೇವೆ, ಕಷ್ಟಕ್ಕೆ ಈಡಾಗುತ್ತೇವೆ ಎಂದು ತಿಳಿಯುವುದೇ ಇಲ್ಲ. ನಿರಾಶೆ ಸುತ್ತಲೂ ಕವಿದಿದ್ದರೂ, ಎಲ್ಲವೂ ನಮಗೆ ವಿರೋಧವಾಗಿದ್ದರೂ, ನಮ್ಮ ಗಣನೆಗೆ ವಿರೋಧವಾಗಿದ್ದರೂ, ಪ್ರತಿಯೊಬ್ಬನೂ ಜಯ ತನಗೆ ಲಭಿಸುವುದೆಂದು ಭಾವಿಸುವನು. ಯಾರೂ ಇಲ್ಲಿ ನಿಜವಾಗಿಯೂ ಸುಖಿಯಾಗಿಲ್ಲ. ಒಬ್ಬ ಮನುಷ್ಯ ಶ‍್ರೀಮಂತನಾಗಿ ತಿನ್ನಲು ಬೇಕಾದಷ್ಟು ಇದ್ದರೆ, ಅವನ ಜೀರ್ಣಶಕ್ತಿ ಚೆನ್ನಾಗಿಲ್ಲ. ಅವನು ಏನನ್ನೂ ತಿನ್ನಲಾರ. ಒಬ್ಬನ ಜೀರ್ಣಶಕ್ತಿ ಚೆನ್ನಾಗಿದ್ದು, ಅವನಿಗೆ ಹೊಟ್ಟೆಬಾಕನಂತೆ ಏನನ್ನು ಬೇಕಾದರೂ ತಿಂದು ಅರಗಿಸಿಕೊಳ್ಳಬಲ್ಲ ಶಕ್ತಿ ಇದ್ದರೆ, ಅವನಿಗೆ ತಿನ್ನಲು ಏನೂ ಇಲ್ಲ. ಐಶ್ವರ್ಯವಂತನಾದರೆ ಆತನಿಗೆ ಮಕ್ಕಳಿಲ್ಲ. ಉಪವಾಸದಿಂದ ನರಳುವ ಬಡವನಾದರೆ ಆತನಿಗೆ ದೊಡ್ಡದೊಂದು ಮಕ್ಕಳ ಸೇನೆ. ಅವರನ್ನು ಏನುಮಾಡಬೇಕೋ ಗೊತ್ತಿಲ್ಲ. ಇದು ಏತಕ್ಕೆ? ಸುಖ ಮತ್ತು ದುಃಖ ಒಂದೇ ನಾಣ್ಯದ ಎರಡು ಮುಖಗಳು. ಯಾರು ಸುಖವನ್ನು ಬಯಸುವರೋ ಅವರು ದುಃಖವನ್ನು ಸ್ವೀಕರಿಸಬೇಕು. ನಮಗೆಲ್ಲ ದುಃಖವಿಲ್ಲದೆ ಸುಖ ಸಿಕ್ಕುವುದೆಂಬ ಭ್ರಾಂತಿ ಹಿಡಿದಿರುವುದು. ಈ ಭ್ರಾಂತಿ ನಮ್ಮನ್ನು ಸಂಪೂರ್ಣ ವಶಮಾಡಿಕೊಂಡಿರುವುದು. ನಮಗೆ ಇಂದ್ರಿಯದ ಮೇಲೆ ನಿಗ್ರಹವೇ ಇಲ್ಲ.

ನಾನು ಬಾಸ್ಟನ್ನಿನಲ್ಲಿದ್ದಾಗ ಒಬ್ಬ ಯುವಕ ನನ್ನ ಹತ್ತಿರ ಬಂದು ತನ್ನ ಹೆಸರು ಮತ್ತು ವಿಳಾಸ ಬರೆದ ಕಾಗದ ಕೊಟ್ಟನು. ಅದರಲ್ಲಿ ಹೀಗೆ ಬರೆದಿತ್ತು: “ಪ್ರಪಂಚದ ಐಶ್ವರ್ಯವೆಲ್ಲವನ್ನು ಮತ್ತು ಸುಖವೆಲ್ಲವನ್ನು ಹೇಗೆ ಪಡೆಯಬೇಕೆಂಬುದು ಗೊತ್ತಿದ್ದರೆ ಅವು\break ನಮ್ಮದಾಗುವುವು. ನೀವು ನನ್ನ ಹತ್ತಿರ ಬಂದರೆ ಅವನ್ನು ಹೇಗೆ ಪಡೆದುಕೊಳ್ಳಬಹುದು\break ಎಂಬುದನ್ನು ಹೇಳಿಕೊಡುತ್ತೇನೆ. ಚಾರ್ಜು ಐದು ಡಾಲರು.” ಇದನ್ನು ನನಗೆ ಕೊಟ್ಟು, ಹೇಗಿದೆ ಎಂದು ಕೇಳಿದನು. ಅದಕ್ಕೆ ನಾನು “ಅಯ್ಯಾ ಯುವಕನೆ, ಇದನ್ನು ಮುದ್ರಿಸುವುದಕ್ಕೆ ನೀನು ಏತಕ್ಕೆ ಮೊದಲು ಹಣ ಪಡೆಯಬಾರದು. ನಿನಗೆ ಇದನ್ನು ಛಾಪಿಸುವುದಕ್ಕೇ ಹಣವಿಲ್ಲ!?” ಎಂದೆ..

ಅವನಿಗೆ ಇದು ತಿಳಿಯಲಿಲ್ಲ. ಅವನು ಯಾವ ತೊಂದರೆಯೂ ಇಲ್ಲದೆ ಬೇಕಾದಷ್ಟು ಐಶ್ವರ್ಯವನ್ನು ಮತ್ತು ಸೌಖ್ಯವನ್ನು ಪಡೆಯಬಹುದು ಎಂಬ ಭಾವನೆಯಿಂದ ಹುಚ್ಚನಾಗಿ ಹೋಗಿದ್ದನು. ಮನುಷ್ಯರು ಎರಡು ಅತಿರೇಕಗಳ ಕಡೆಗೆ ನುಗ್ಗುವರು. ಒಂದು ಅತಿ ಆಶಾಭಾವನೆ, ಅಂದರೆ ಎಲ್ಲವೂ ಸುಂದರವಾಗಿರುವುದು ಎಂದು ಭಾವಿಸುವುದು; ಮತ್ತೊಂದು ಅತಿ ದುಃಖದ ಭಾವನೆ, ಎಲ್ಲಾ ತಮಗೆ ವಿರೋಧವಾಗಿರುವಂತೆ ಭಾವಿಸುವುದು. ಹೆಚ್ಚಿನ ಜನರ ಮಿದುಳು ಇನ್ನೂ ಅಷ್ಟು ವಿಕಾಸವಾಗಿಲ್ಲ. ಎಲ್ಲೋ ಕೋಟಿಗೆ ಒಬ್ಬನ ಮಿದುಳು ಚೆನ್ನಾಗಿ ಅಭಿವೃದ್ಧಿಯಾಗಿದೆ. ಮಿಕ್ಕವರಿಗೆ ಯಾವುದಾದರೊಂದು ವಿಚಿತ್ರ ವ್ಯಕ್ತಿವೈಶಿಷ್ಟ್ಯ ಅಥವಾ ಭ್ರಾಂತಿ ಇರುವುದು.

ಸ್ವಾಭಾವಿಕವಾಗಿ ನಾವು ಒಂದು ಅತಿರೇಕಕ್ಕೆ ಹೋಗುವೆವು. ನಾವು ಆರೋಗ್ಯವಾಗಿ ಯುವಕರಾಗಿರುವಾಗ ಪ್ರಪಂಚದ ಐಶ್ವರ್ಯವೆಲ್ಲಾ ನಮಗೆ ದೊರಕುವುದೆಂದು ಭಾವಿಸುವೆವು, ಅನಂತರ ಸ್ವಲ್ಪ ವಯಸ್ಸಾದ ಮೇಲೆ, ಕಾಲ್ಚೆಂಡಿನಂತೆ ಸಮಾಜದಿಂದ ಒದೆಸಿಕೊಂಡು ಒಂದು ಮೂಲೆಯಲ್ಲಿ ಕುಳಿತು ಅಳುವೆವು. ಇತರರ ಸ್ಫೂರ್ತಿಯನ್ನು ಭಂಗಿಸುವೆವು, ಸುಖದೊಂದಿಗೆ ದುಃಖವಿದೆ, ದುಃಖದೊಂದಿಗೆ ಸುಖವೂ ಇದೆ ಎಂದು ಎಲ್ಲೋ ಕೆಲವರಿಗೆ ಮಾತ್ರ ಗೊತ್ತು. ದುಃಖದಿಂದ ಹೇಗೆ ಜಿಗುಪ್ಸೆ ಹುಟ್ಟುವುದೋ, ಹಾಗೆಯೇ ಅದರ ಅವಳಿಜವಳಿಯಾದ ಸುಖದಿಂದಲೂ ಜಿಗುಪ್ಸೆ ಹುಟ್ಟುವುದು. ದುಃಖವನ್ನು ಅರಸುವುದು ಕೂಡ ಅಷ್ಟೇ ನಾಚಿಕೆಗೇಡು. ಯಾರು ಯುಕ್ತಾಯುಕ್ತ ಪರಿಜ್ಞಾನ ಉಳ್ಳವರೋ ಅವರು ಎರಡನ್ನೂ ನಿಕೃಷ್ಟ ದೃಷ್ಟಿಯಿಂದ ನೋಡಬೇಕು. ಸುಖದುಃಖಗಳ ಕೈಯಲ್ಲಿ ಸಿಕ್ಕಿಕೊಂಡು ನರಳುವುದಕ್ಕಿಂತ ವ್ಯಕ್ತಿಯು ಸ್ವಾತಂತ್ರ್ಯವನ್ನು ಏತಕ್ಕೆ ಇಚ್ಚಿಸಬಾರದು? ಈ ಕ್ಷಣ ಚಾವಟಿ ಏಟು ಬೀಳುವುದು. ನಾವು ಅಳುವುದಕ್ಕೆ ಮೊದಲುಮಾಡಿದಾಗ, ಪ್ರಕೃತಿ ಒಂದು ಡಾಲರನ್ನು ಕೊಡುವುದು. ಆಗ ನಗುವೆವು. ಪುನಃ ಚಾವಟಿಯ ಪೆಟ್ಟು ಬೀಳುವುದು. ಆಗ ಅಳುವೆವು. ಪ್ರಕೃತಿ ಪುನಃ ಒಂದು ಚೂರು ಸಿಹಿರೊಟ್ಟಿಯನ್ನು ಕೊಡುವುದು. ಆಗ ಪುನಃ ನಗಲು ಪ್ರಾರಂಭಿಸುವೆವು.

ಆ ಸಂತನಿಗೆ ಸ್ವಾತಂತ್ರ್ಯ ಬೇಕಾಗಿದೆ. ಇಂದ್ರಿಯ ವಿಷಯಗಳು ಬರಿಯ ಭ್ರಾಂತಿ, ಸುಖದುಃಖಗಳಿಗೆ ಅಂತ್ಯವಿಲ್ಲವೆಂಬುದು ಗೊತ್ತಾಗಿದೆ. ಎಷ್ಟು ಜನ ಶ‍್ರೀಮಂತರು ಹೊಸ ಹೊಸ ಸುಖವನ್ನು ಹುಡುಕುವುದರಲ್ಲಿ ಕಾತರರಾಗಿರುವರು! ಈಗಿರುವ ಸುಖ ಹಳೆಯದಾಗಿದೆ. ಅವರಿಗೆ ಹೊಸತಾದ ಸುಖ ಬೇಕು. ಪ್ರತಿದಿನವೂ ಎಷ್ಟು ಕೆಲಸಕ್ಕೆ ಬಾರದ ವಿಷಯಗಳನ್ನು ಕಂಡುಹಿಡಿಯುತ್ತಿರುವರು ಎಂಬುದು ಕಾಣಿಸುವುದಿಲ್ಲವೆ? ಇದು ಕೇವಲ ಕೆಲವು ಕ್ಷಣ ನಮ್ಮ ನರಗಳಿಗೆ ಸುಖವನ್ನು ಕೊಡುವುದಕ್ಕೆ ಮಾತ್ರ. ಅದಾದ ಮೇಲೆ ದುಃಖದ ಪ್ರತಿಫಲ, ಜನಸಾಮಾನ್ಯರು ಒಂದು ಕುರಿಯ ಮಂದೆಯಂತೆ.\break ಮುಂದೆ ಇರುವ ಕುರಿ ಹಳ್ಳಕ್ಕೆ ಬಿದ್ದರೆ ಉಳಿದವೆಲ್ಲ ಹಳ್ಳಕ್ಕೆ ಬಿದ್ದು ತಮ್ಮ ಕತ್ತನ್ನು ಮುರಿದುಕೊಳ್ಳುವುವು. ಇದರಂತೆಯೇ ಸಮಾಜದ ಒಬ್ಬ ಮುಂದಾಳು ಏನು ಮಾಡುತ್ತಾನೆಯೋ ಅದನ್ನು ಉಳಿದವರೆಲ್ಲರೂ, ತಾವು ಏನು ಮಾಡುತ್ತಿರುವೆವು ಎಂಬುದನ್ನು ಕೂಡ ಆಲೋಚಿಸದೆ, ಅನುಕರಿಸುವರು. ಮನುಷ್ಯನು ಪ್ರಾಪಂಚಿಕ ವಸ್ತುಗಳ ನಶ್ವರತೆ ಗೊತ್ತಾದಾಗ ಪ್ರಕೃತಿಯ ಆಟಕ್ಕೆ ತಾನು ತುತ್ತಾಗಬಾರದೆಂದು, ಅದರ ಪ್ರಭಾವಕ್ಕೆ ತಾನು ಬೀಳಬಾರದೆಂದು ಯೋಚಿಸುವನು. ಇದು ಗುಲಾಮಗಿರಿ, ಯಾರಾದರೂ ತನಗೆ ಒಳ್ಳೆಯ ಮಾತನ್ನು ಹೇಳಿದರೆ ನಗುವನು. ಸ್ವಲ್ಪ ಕಟುಮಾತನ್ನು ಕೇಳಿದರೆ ಅಳುವನು. ಒಂದು ಚೂರು ರೊಟ್ಟಿಗೆ, ಸ್ವಲ್ಪ ಗಾಳಿಗೆ, ಅವನು ದಾಸ, ಬಟ್ಟೆಬರೆಗಳಿಗೆ ದಾಸ, ದೇಶಕ್ಕೆ ದೇಶಪ್ರೇಮಕ್ಕೆ, ಹೆಸರಿಗೆ, ಕೀರ್ತಿಗೆ ಅವನು ದಾಸ, ದಾಸ್ಯದ ಮಧ್ಯದಲ್ಲಿ ಅವನು ಇರುವನು. ನಿಜವಾದ ಮಾನವನು ಬಂಧನದಿಂದ ಮುಚ್ಚಿಹೋಗುವನು. ನೀವು ಯಾರನ್ನು ಮಾನವನೆನ್ನುವಿರೋ ಆತ ದಾಸ, ಈ ದಾಸ್ಯ ಎಂದು ಒಬ್ಬನಿಗೆ ಮನದಟ್ಟಾಗುವುದೋ ಆಗ ಮುಕ್ತನಾಗಬೇಕೆಂಬ ಆಸೆ ಉದಯಿಸುವುದು, ತೀವ್ರ ಆಕಾಂಕ್ಷೆ ಏಳುವುದು. ಒಬ್ಬನ ತಲೆಯ ಮೇಲೆ ಕೆಂಗೆಂಡವನ್ನು ಇಟ್ಟರೆ ಅವನು ಅದನ್ನು ಆಚೆಗೆಸೆಯಲು ಎಷ್ಟು ಕಾತುರನಾಗುತ್ತಾನೆ, ನೋಡಿ. ಯಾರು ತಾವು ಪ್ರಕೃತಿಯ ಗುಲಾಮರೆಂದು ನಿಜವಾಗಿ ತಿಳಿಯುವರೋ, ಅವರು ತಾವು ಮುಕ್ತರಾಗಬೇಕೆಂದು ಮಾಡುವ ಹೋರಾಟವೂ ಹೀಗೆಯೇ ಇರುವುದು.

ಮುಮುಕ್ಷುತ್ವ ಅಥವಾ ಮುಕ್ತನಾಗಲು ಇಚ್ಛೆ ಎಂದರೆ ಏನೆಂಬುದನ್ನು ಈಗ ನೋಡಿದೆವು. ಮುಂದೆ ಬರುವ ಸಾಧನೆ ಕೂಡ ಅಷ್ಟೇ ಕಷ್ಟವಾಗಿರುವುದು. ನಿತ್ಯ-ಅನಿತ್ಯ-\break ವಸ್ತು-ವಿವೇಕ ಎಂದರೆ, ಯಾವುದು ಸತ್ಯ ಮತ್ತು ಯಾವುದು ಅಸತ್ಯ, ಯಾವುದು ನಿತ್ಯ ಮತ್ತು ಯಾವುದು ಅನಿತ್ಯ ಎಂಬುದನ್ನು ವಿಮರ್ಶಿಸುವುದು. ದೇವರೊಬ್ಬನೇ ನಿತ್ಯ, ಉಳಿದುದೆಲ್ಲ ಅನಿತ್ಯ. ಎಲ್ಲರೂ ನಾಶಹೊಂದುವರು. ದೇವತೆಗಳು ಸಾಯುವರು, ಮನುಷ್ಯರು ಸಾಯುವರು, ಪ್ರಾಣಿಗಳು ಸಾಯುವುವು. ಭೂಮಿ, ಸೂರ್ಯ, ಚಂದ್ರ, ತಾರೆ, ಎಲ್ಲಾ ನಾಶವಾಗುವುವು. ಪ್ರತಿಯೊಂದೂ ಅನವರತ ಬದಲಾಗುತ್ತಿದೆ. ಇಂದಿನ ಬೆಟ್ಟ ಹಿಂದಿನ ಸಾಗರವಾಗಿತ್ತು. ಮುಂದೆಯೂ ಸಾಗರವಾಗುವುದು. ಪ್ರತಿಯೊಂದೂ ಓಡುತ್ತಿದೆ. ಈ ಪ್ರಪಂಚವೇ ಒಂದು ಬದಲಾವಣೆಯ ತಿರುಗುಚಕ್ರ. ಬದಲಾಗದೆ ಇರುವವನು ಒಬ್ಬ. ಅವನೇ ದೇವರು. ನಾವು ಅವನ ಹತ್ತಿರಕ್ಕೆ ಹೋದಂತೆಲ್ಲ ಬದಲಾವಣೆ ನಮ್ಮಲ್ಲಿ ಕಡಮೆಯಾಗುವುದು. ನಮ್ಮ ಮೇಲೆ ಪ್ರಕೃತಿಯ ಪ್ರಭಾವ\break ಕಡಮೆಯಾಗುವುದು. ನಾವು ಅವನನ್ನು ಸೇರಿ, ಅವನ ಹತ್ತಿರ ನಿಂತಾಗ ಪ್ರಕೃತಿಯನ್ನು ಗೆಲ್ಲುವೆವು. ನಾವು ಈ ಸೃಷ್ಟಿ ನಾಟಕದ ಒಡೆಯರಾಗುವೆವು. ನಮ್ಮ ಮೇಲೆ ಯಾವ ಪರಿಣಾಮವನ್ನೂ ಅದು ಉಂಟುಮಾಡಲಾರದು.

ಮೇಲಿನ ಶಿಕ್ಷಣವನ್ನು ನಾವು ಪಡೆದಿದ್ದರೆ ಪ್ರಪಂಚದಲ್ಲಿ ನಮಗೆ ಮತ್ತೇನೂ ಬೇಕಾಗಿಲ್ಲ. ಜ್ಞಾನವೆಲ್ಲ ನಮ್ಮಲ್ಲಿರುವುದು. ಪೂರ್ಣತೆ ಆಗಲೆ ಆತ್ಮನಲ್ಲಿರುವುದು. ಆದರೆ ಪ್ರಕೃತಿ ಈ ಪರಿಪೂರ್ಣತೆಯನ್ನು ಮುಚ್ಚಿರುವುದು. ಆತ್ಮನ ಪವಿತ್ರತೆಯನ್ನು ಮಾಯೆಯ ಮುಸುಕುಗಳು ಒಂದಾದಮೇಲೊಂದು ಕವಿದಿರುವುವು. ನಾವು ಮಾಡುವುದೇನು? ನಾವು ನಿಜವಾಗಿಯೂ ನಮ್ಮ ಆತ್ಮನನ್ನು ಅಭಿವೃದ್ಧಿಗೊಳಿಸುವುದಿಲ್ಲ. ಪೂರ್ಣತೆಯನ್ನು ಯಾವುದು ವೃದ್ಧಿಗೊಳಿಸಬಲ್ಲದು? ನಾವು ಮುಸುಕನ್ನು ತೆಗೆಯುವೆವು. ಆತ್ಮ ತನ್ನ ಸ್ವಭಾವಸಿದ್ಧವಾದ ಸ್ವಾತಂತ್ರ್ಯವನ್ನು ವ್ಯಕ್ತಗೊಳಿಸುವುದು.

ಸಾಧನೆ ಏತಕ್ಕೆ ಇಷ್ಟು ಆವಶ್ಯಕ ಎಂಬ ವಿಚಾರ ಈಗ ಮೊದಲಾಗುವುದು. ಆಧ್ಯಾತ್ಮವನ್ನು ನಾವು ಕಿವಿಯ ಮೂಲಕವಾಗಿ ಆಗಲಿ ಅಥವಾ ಮಿದುಳಿನ ಮೂಲಕವಾಗಲಿ ತಿಳಿದುಕೊಳ್ಳಲಾರೆವು. ಯಾವ ಶಾಸ್ತ್ರವೂ ನಮ್ಮನ್ನು ಧಾರ್ಮಿಕರನ್ನಾಗಿ ಮಾಡದು. ಪ್ರಪಂಚದಲ್ಲಿರುವ ಗ್ರಂಥಗಳನ್ನೆಲ್ಲಾ ನಾವು ಓದಬಹುದು. ಆದರೂ ದೇವರ ಮತ್ತು ಧರ್ಮದ ಒಂದು ಪದವೂ ಅರ್ಥವಾಗದೆ ಇರಬಹುದು. ನಾವು ಇಡೀ ಜೀವನವೆಲ್ಲ ಅದನ್ನು ಕುರಿತು ಮಾತನಾಡಬಹುದು. ಆದರೂ ಅದರಿಂದ ಏನೂ ಪ್ರಯೋಜನವಾಗದೆ ಇರಬಹುದು. ಇಡೀ ಪ್ರಪಂಚವೇ ಇದುವರೆಗೂ ಕಾಣದ ವಿದ್ವಾಂಸರು ನಾವಾಗಿರಬಹುದು. ಆದರೂ ನಾವು ದೇವರ ಸಮೀಪಕ್ಕೆ ಬಾರದೆ ಇರಬಹುದು. ಘನ ಪಂಡಿತರ ವರ್ಗದಿಂದ ಎಂತಹ ಅಧಮರು ಬಂದಿರುವರು ಎಂಬುದು ನಿಮಗೆ ಗೊತ್ತಿಲ್ಲವೆ? ಪಾಶ್ಚಾತ್ಯ ಸಂಸ್ಕೃತಿಯ ದೊಡ್ಡ ಕೊರತೆ ಇದು. ನಿಮಗೆ ಕೇವಲ ಪಾಂಡಿತ್ಯ ಬೇಕು. ನೀವು ಹೃದಯದ ಕಡೆ ಗಮನವನ್ನೇ ಕೊಡುವುದಿಲ್ಲ. ಇದು ಮಾನವರನ್ನು ಹತ್ತುಪಾಲು ಹೆಚ್ಚು ಸ್ವಾರ್ಥಿಗಳನ್ನಾಗಿ ಮಾಡಿರುವುದು. ಇದೇ ನಿಮ್ಮ ನಾಶಕ್ಕೆ ಕಾರಣವಾಗುವುದು. ಹೃದಯಕ್ಕೆ ಮತ್ತು ಮಿದುಳಿಗೆ ಘರ್ಷಣೆಯಾದಾಗ ಹೃದಯವನ್ನು ಅನುಸರಿಸಿ. ಏಕೆಂದರೆ ಬುದ್ಧಿಗೆ ಇರುವುದು ಒಂದೇ ಸ್ಥಿತಿ, ಅದು ವಿಚಾರ. ಆ ಎಲ್ಲೆಯಲ್ಲಿ ಮಾತ್ರ ಬುದ್ಧಿ ಕೆಲಸಮಾಡಬಲ್ಲದು. ಅದನ್ನು ಅದು ಅತಿಕ್ರಮಿಸಲಾರದು. ಹೃದಯ ಒಬ್ಬನನ್ನು ಅತ್ಯುನ್ನತ ಶಿಖರಕ್ಕೆ ಕರೆದೊಯ್ಯಬಲ್ಲದು. ಬುದ್ಧಿ ಅಲ್ಲಿಗೆ ತಲುಪಲಾರದು. ಹೃದಯ, ಬುದ್ಧಿಯನ್ನು ಮೀರಿದ ಸ್ಫೂರ್ತಿಯನ್ನು ಮುಟ್ಟಬಲ್ಲದು. ಬುದ್ಧಿ ಎಂದಿಗೂ ಸ್ಫೂರ್ತಿಯನ್ನು ಪಡೆಯಲಾರದು. ಹೃದಯ ಮಾತ್ರ - ಬೆಳಕನ್ನು ಕಂಡಾಗ ಸ್ಫೂರ್ತಿಯನ್ನು ಹೊಂದುವುದು, ಹೃದಯಶೂನ್ಯನಾದ ಕೇವಲ ಬುದ್ಧಿವಂತಿಕೆಯ ಮನುಷ್ಯನು ಎಂದಿಗೂ ಸ್ಫೂರ್ತಿವಂತನಾಗಲಾರ. ಪ್ರೇಮಪೂರ್ಣ ವ್ಯಕ್ತಿಯಲ್ಲಿ ಯಾವಾಗಲೂ ಮಾತನಾಡುವುದು ಹೃದಯ, ಬುದ್ಧಿ ಕೊಡುವ ಉಪಕರಣಕ್ಕಿಂತ ಉತ್ತಮವಾದ ಸ್ಫೂರ್ತಿಯನ್ನು ಹೃದಯ ನಮಗೆ ಕೊಡುವುದು, ಬುದ್ಧಿ ಹೇಗೆ ಜ್ಞಾನದ ಉಪಕರಣವೋ ಹಾಗೆಯೇ ಹೃದಯ ಸ್ಫೂರ್ತಿಯ ಉಪಕರಣ. ಕೆಳಗಿನ ಹಂತದಲ್ಲಿ ಅದು ಬುದ್ಧಿಗಿಂತ ಅತಿ ದುರ್ಬಲವಾಗಿರುವುದು. ಅಜ್ಞಾನಿಗೆ ಏನೂ ತಿಳಿಯದು. ಸ್ವಭಾವತಃ ಅವನು ಭಾವಜೀವಿ. ಅವನನ್ನು ಒಬ್ಬ ದೊಡ್ಡ ವಿದ್ವಾಂಸನೊಂದಿಗೆ ಹೋಲಿಸಿ ನೋಡಿ. ವಿದ್ವಾಂಸನಿಗೆ ಎಷ್ಟೊಂದು ಪಾಂಡಿತ್ಯವಿರುವುದು! ಆದರೆ ವಿದ್ವಾಂಸ ತನ್ನ ಬುದ್ಧಿಯಲ್ಲಿ ಬದ್ಧನಾಗಿರುವನು. ಅವನು ಏಕಕಾಲದಲ್ಲಿ ಪಂಡಿತನೂ ಮತ್ತು ನೀಚನೂ ಆಗಿರಬಹುದು. ಆದರೆ ಹೃದಯವಿರುವವನು ಎಂದಿಗೂ ನೀಚನಾಗಲಾರ. ಭಾವಜೀವಿಗಳಾರೂ ನೀಚರಾಗಿರಲಿಲ್ಲ. ಸರಿಯಾಗಿ ಅಭ್ಯಾಸಮಾಡಿದರೆ ಹೃದಯವನ್ನು ಪರಿವರ್ತನೆಗೊಳಿಸಬಹುದು. ಅದು ಬುದ್ಧಿಯನ್ನು ಮೀರಿ ಹೋಗುವುದು, ಸ್ಫೂರ್ತಿಯಾಗುವುದು. ಮನುಷ್ಯ ಕೊನೆಗೆ ಬುದ್ಧಿಯನ್ನು ಮೀರಿ ಹೋಗಬೇಕಾಗಿದೆ. ಮನುಷ್ಯನ ಗ್ರಹಣಶಕ್ತಿ, ಅನುಭವ, ಜ್ಞಾನ, ಹೃದಯ ಇವೆಲ್ಲ ಪ್ರಪಂಚವೆಂಬ ಹಾಲನ್ನು ಕಡೆಯುವುದರಲ್ಲಿ ನಿರತವಾಗಿರುವುವು. ಚೆನ್ನಾಗಿ ಕಡೆದ ಮೇಲೆ ಬೆಣ್ಣೆ ಬರುವುದು. ಈ ಬೆಣ್ಣೆಯೇ ದೇವರು, ಹೃದಯವಿದ್ದವರಿಗೆ ಬೆಣ್ಣೆ ಸಿಕ್ಕುವುದು, ಬುದ್ಧಿವಂತರಿಗೆ ಮಜ್ಜಿಗೆ ಸಿಕ್ಕುವುದು.

ಹೃದಯಕ್ಕೆ ಸಂಬಂಧಪಟ್ಟ ಪ್ರೇಮಕ್ಕೆ, ತೀವ್ರ ಸಹಾನುಭೂತಿಗೆ ಇವೆಲ್ಲ ಸಿದ್ಧತೆಗಳು. ದೇವರನ್ನು ಪಡೆಯಬೇಕಾದರೆ ವಿದ್ಯಾಬುದ್ಧಿಗಳು ಆವಶ್ಯಕವಲ್ಲ. ಒಬ್ಬ ಋಷಿ ನನಗೆ ಒಮ್ಮೆ ಹೇಳಿದನು: “ಇತರರನ್ನು ಕೊಲ್ಲಬೇಕಾದರೆ ಒಬ್ಬನಿಗೆ ಕತ್ತಿ ಗುರಾಣಿಗಳು ಬೇಕು. ಆದರೆ ಆತ್ಮಹತ್ಯೆ ಮಾಡಿಕೊಳ್ಳಬೇಕಾದರೆ ಒಂದು ಸಣ್ಣ ಸೂಜಿ ಸಾಕು” ಎಂದು. ಹಾಗೆಯೇ ಮತ್ತೊಬ್ಬನಿಗೆ ಕಲಿಸಬೇಕಾದರೆ ಬೇಕಾದಷ್ಟು ಪಾಂಡಿತ್ಯ, ಬುದ್ಧಿ ಇವು ಬೇಕು. ನಿಮ್ಮ ಆತ್ಮಸಾಕ್ಷಾತ್ಕಾರಕ್ಕೆ ಅವು ಬೇಕಿಲ್ಲ. ನೀವು ಪರಿಶುದ್ದರೆ? ನೀವು ಪರಿಶುದ್ಧರಾದರೆ ದೇವರನ್ನು ಸೇರುವಿರಿ, “ಪರಿಶುದ್ಧರೇ ಅದೃಷ್ಟಶಾಲಿಗಳು. ಅವರು ದೇವರನ್ನು ನೋಡುವರು.” ನೀವು ಪರಿಶುದ್ದರಲ್ಲದೇ ಇದ್ದರೆ ಜಗತ್ತಿನಲ್ಲಿರುವ ಶಾಸ್ತ್ರಗಳನ್ನೆಲ್ಲ ತಿಳಿದಿದ್ದರೂ ಪ್ರಯೋಜನವಿಲ್ಲ. ನೀವು ಓದುವ ಪುಸ್ತಕಗಳ ರಾಶಿಯು ನಿಮ್ಮನ್ನು ಹೂತುಬಿಡಬಹುದು. ಆದರೆ ಅದರಿಂದ ಹೆಚ್ಚು ಪ್ರಯೋಜನವಿಲ್ಲ. ಗುರಿಯನ್ನು ಸೇರುವುದು ಹೃದಯ. ಅದನ್ನು ಅನುಸರಿಸಿ, ಪರಿಶುದ್ದ ಹೃದಯ, ಬುದ್ಧಿಯನ್ನು ಮೀರಿ ನೋಡುವುದು; ಸ್ಫೂರ್ತಿಯನ್ನು ಪಡೆಯುವುದು. ಬುದ್ಧಿಗೆ ನಿಲುಕದುದನ್ನು ಅದು ತಿಳಿದುಕೊಳ್ಳಬಲ್ಲದು. ಶುದ್ದ ಹೃದಯಕ್ಕೂ ಬುದ್ಧಿಗೂ ಎಲ್ಲಿ ಘರ್ಷಣೆ ಇರುವುದೋ ಅಲ್ಲಿ, ಹೃದಯ ಮಾಡುತ್ತಿರುವುದು ಅಯುಕ್ತವೆಂದು ತಿಳಿದರೂ, ಹೃದಯವನ್ನು ಅನುಸರಿಸಿ, ಮತ್ತೊಬ್ಬರಿಗೆ ಉಪಕಾರಮಾಡುವುದು ಒಳ್ಳೆಯದೆಂದು ತೋರಿದಾಗ, ಬುದ್ಧಿ ಅದು ಸರಿಯಲ್ಲವೆಂದು ಹೇಳಬಹುದು. ಆದರೂ ನಿಮ್ಮ ಹೃದಯವನ್ನು ಅನುಸರಿಸಿ. ಆಗ ಬುದ್ಧಿಯನ್ನು ಅನುಸರಿಸುವಾಗ ಮಾಡುವ ತಪ್ಪಿಗಿಂತ ಕಡಿಮೆ ತಪ್ಪನ್ನು ಮಾಡುವೆವು. ಸತ್ಯವನ್ನು ಪ್ರತಿಬಿಂಬಿಸುವುದಕ್ಕೆ ಪರಿಶುದ್ದ ಹೃದಯವೇ ಅತಿ ಶ್ರೇಷ್ಠ ಕನ್ನಡಿ. ಈ ಗುರಿ ಸಾಧನೆಯ ಹೃದಯವನ್ನು ಪರಿಶುದ್ಧ ಮಾಡುವುದು. ಅದು ಪರಿಶುದ್ಧವಾದೊಡನೆಯೇ ಸತ್ಯ ಕ್ಷಣದಲ್ಲಿ ಹೊಳೆಯುವುದು. ನೀವು ತಕ್ಕಷ್ಟು ಪರಿಶುದ್ಧರಾಗಿದ್ದರೆ ಪ್ರಪಂಚದ ಸತ್ಯವೆಲ್ಲ ನಿಮ್ಮಲ್ಲಿ ವ್ಯಕ್ತವಾಗುವುದು.

ಕಣಗಳ, ಸೂಕ್ಷ್ಮವಸ್ತುಗಳ ಮತ್ತು ಮನುಷ್ಯನ ಸೂಕ್ಷ್ಮಗ್ರಹಣ ಶಕ್ತಿಯ ಸತ್ಯವನ್ನು ಸಾವಿರಾರು ವರುಷಗಳ ಹಿಂದೆ, ದೂರದರ್ಶಕ ಯಂತ್ರ, ಸೂಕ್ಷ್ಮದರ್ಶಕ ಯಂತ್ರ, ವೈಜ್ಞಾನಿಕ ಪ್ರಯೋಗಶಾಲೆ ಇವುಗಳಲ್ಲಿ ಯಾವ ಒಂದನ್ನೂ ನೋಡದವರು ಕಂಡುಹಿಡಿದರು. ಇವೆಲ್ಲ ಅವರಿಗೆ ಹೇಗೆ ತಿಳಿಯಿತು? ಹೃದಯದ ಮೂಲಕ. ಅವರು ಹೃದಯವನ್ನು ಪರಿಶುದ್ಧಗೊಳಿಸಿದರು. ಇಂದಿಗೂ ನಾವು ಅದನ್ನು ಮಾಡಬಹುದು. ನಿಜವಾಗಿಯೂ ಹೃದಯದ ಪರಿವರ್ತನೆ, ಬುದ್ಧಿಯ ಪರಿವರ್ತನೆಯಲ್ಲ, ಪ್ರಪಂಚದ ದುಃಖವನ್ನು ತಗ್ಗಿಸುವುದು.

ಬುದ್ಧಿಯನ್ನು ರೂಢಿಸಿದುದರ ಪರಿಣಾಮವಾಗಿ ನೂರಾರು ವಿಜ್ಞಾನ ಶಾಸ್ತ್ರಗಳನ್ನು ಕಂಡುಹಿಡಿದಿರುವರು. ಇದರ ಫಲವಾಗಿ ಸ್ವಲ್ಪ ಜನರು ಹಲವರನ್ನು ದಾಸರನ್ನಾಗಿ ಮಾಡಿಕೊಂಡಿರುವರು. ಇದೇ ಇದರಿಂದ ಆಗಿರುವ ಪ್ರಯೋಜನ, ಮಾನವ ಕೃತಕ ಬಯಕೆಗಳನ್ನು ಸೃಷ್ಟಿಸಿರುವನು. ಪ್ರತಿಯೊಬ್ಬ ಬಡವನೂ, ಅವನಿಗೆ ದುಡ್ಡು ಇರಲಿ ಇಲ್ಲದೆ ಇರಲಿ, ಆ ಬಯಕೆಗಳನ್ನು ಈಡೇರಿಸಿಕೊಳ್ಳಲು ಬಯಸುವನು. ಅದು ಸಾಧ್ಯವಿಲ್ಲದೇ ಇರುವಾಗ ಅದಕ್ಕಾಗಿ ಹೋರಾಡುವನು; ಆ ಹೋರಾಟದಲ್ಲಿ ಮಡಿಯುವನು. ಇದೇ ಅದರ ಫಲ. ದುಃಖನಿವಾರಣೆಗೆ ದಾರಿ ಬುದ್ಧಿಯಲ್ಲ, ಹೃದಯ. ಇಷ್ಟೊಂದು ಪ್ರಯತ್ನವನ್ನು ನಮ್ಮ ಹೃದಯ ಪರಿಶುದ್ಧವಾಗುವುದಕ್ಕೆ, ಮೃದುವಾಗುವುದಕ್ಕೆ, ಹೆಚ್ಚು ಸಹಿಷ್ಣುತಾಶಕ್ತಿಯನ್ನು ಪಡೆಯುವುದಕ್ಕೆ ಉಪಯೋಗಿಸಿದ್ದರೆ, ಪ್ರಪಂಚದಲ್ಲಿ ಈಗ ಇರುವುದಕ್ಕಿಂತ ಒಂದುಸಾವಿರ ಪಾಲು ಹೆಚ್ಚು ಸುಖವಿರುತ್ತಿತ್ತು. ಯಾವಾಗಲೂ ಹೃದಯವನ್ನು ಪೋಷಿಸಿ, ಹೃದಯದ ಮೂಲಕ ದೇವರು ಮಾತನಾಡುವನು. ಬುದ್ಧಿಯ ಮೂಲಕ ನೀವು ಮಾತ್ರ ಮಾತನಾಡುತ್ತೀರಿ.

ಹಳೆಯ ಟೆಸ್ಟಮೆಂಟಿನಲ್ಲಿ ಮೊಸಸ್ಸಿಗೆ ಹೇಳಿದ ಈ ನುಡಿ ಜ್ಞಾಪಕವಿರಬಹುದು: “ಕಾಲಿನಿಂದ ನಿನ್ನ ಪಾದರಕ್ಷೆಯನ್ನು ತೆಗೆ. ನೀನು ನಿಂತಿರುವ ಸ್ಥಳ ಪವಿತ್ರವಾದುದು.'' ನಾವು ಯಾವಾಗಲೂ ಆಧ್ಯಾತ್ಮಿಕ ವಿಷಯಗಳನ್ನು ಇಂತಹ ಗೌರವದ ದೃಷ್ಟಿಯಿಂದ ನೋಡಬೇಕು. ಯಾರು ಪರಿಶುದ್ಧ ಹೃದಯದಿಂದ, ಪೂಜ್ಯದೃಷ್ಟಿಯಿಂದ ಬರುವರೋ, ಅವರ ಹೃದಯ ವಿಕಸಿತವಾಗುವುದು. ಅವರಿಗೆ ಬಾಗಿಲು ತೆರೆಯುವುದು. ಅವರು ಸತ್ಯವನ್ನು ಸಂದರ್ಶಿಸುವರು.

ನೀವು ಕೇವಲ ಬುದ್ಧಿಯಿಂದ ಪ್ರೇರಿತರಾಗಿ ಬಂದರೆ, ನಿಮಗೆ ಸ್ವಲ್ಪ ಬುದ್ಧಿಯ ಕಸರತ್ತಾಗಬಹುದು, ಸಿದ್ದಾಂತಗಳು ಸಿಕ್ಕಬಹುದು, ಆದರೆ ಸತ್ಯ ದೊರೆಯದು. ಸತ್ಯಕ್ಕೆ ಎಂತಹ ಕಾಂತಿ ಇರುವುದೆಂದರೆ ಯಾರು ಅದನ್ನು ನೋಡುವರೋ ಅವರಿಗೆ ಇದು ಸತ್ಯವೆಂದು ನಿಸ್ಸಂಶಯವಾಗಿ ಗೊತ್ತಾಗುವುದು. ಸೂರ್ಯನನ್ನು ನೋಡಬೇಕಾದರೆ ಯಾವ ದೀವಟಿಗೆಯೂ ಬೇಕಿಲ್ಲ. ಸೂರ್ಯ ಸ್ವಯಂಪ್ರಕಾಶಮಾನನು. ಸತ್ಯಕ್ಕೆ ಒಂದು ಪ್ರಮಾಣ ಬೇಕಾದರೆ ಆ ಪ್ರಮಾಣಕ್ಕೆ ಪ್ರಮಾಣ ಯಾವುದು? ಸತ್ಯಕ್ಕೆ ಒಂದು ಸಾಕ್ಷಿ ಬೇಕಾದರೆ ಆ ಸಾಕ್ಷಿಗೆ ಸಾಕ್ಷಿ ಯಾವುದು? ನಾವು ಆಧ್ಯಾತ್ಮಿಕತೆಯನ್ನು ಗೌರವದಿಂದ, ವಿಶ್ವಾಸದಿಂದ ಕಾಣಬೇಕು. ನಮ್ಮ ಹೃದಯ ಜಾಗ್ರತವಾಗಿ ಇದು ಸತ್ಯ, ಇದು ಅಸತ್ಯ ಎಂದು ತೋರುವುದು.

ಆಧ್ಯಾತ್ಮಿಕ ಕ್ಷೇತ್ರ ನಮ್ಮ ಇಂದ್ರಿಯಗಳಾಚೆ ಇದೆ; ನಮ್ಮ ಪ್ರಜ್ಞೆಯನ್ನೂ ಮೀರಿದೆ. ನಾವು ದೇವರನ್ನು ನೋಡುವುದಕ್ಕೆ ಆಗುವುದಿಲ್ಲ. ಯಾರೂ ತಮ್ಮ ಜಡ ಕಣ್ಣಿನಿಂದ ದೇವರನ್ನು ನೋಡಿಲ್ಲ; ನೋಡುವಂತೆಯೂ ಇಲ್ಲ. ದೇವರು ಯಾರ ಪ್ರಜ್ಞೆಗೂ\break ಒಳಪಟ್ಟಿಲ್ಲ. ನನಗೆ ದೇವರ ಪ್ರಜ್ಞೆ ಇಲ್ಲ. ನಿಮಗೆ ಇಲ್ಲ. ಮತ್ತಾರಿಗೂ ಇಲ್ಲ. ದೇವರು ಎಲ್ಲಿರುವನು? ಧರ್ಮದ ಕ್ಷೇತ್ರವಾವುದು? ಅದು ಇಂದ್ರಿಯದ ಆಚೆ ಇದೆ; ಪ್ರಜ್ಞೆಯ ಆಚೆ ಇದೆ. ಪ್ರಜ್ಞೆ ನಾನು ಕೆಲಸ ಮಾಡುವ ವಿವಿಧ ಸ್ತರಗಳಲ್ಲಿ ಒಂದು ಮಾತ್ರ. ನಾವು ಇಂದ್ರಿಯಾತೀತರಾಗಬೇಕಾದರೆ ಆತ್ಮದ ಕೇಂದ್ರದ ಹತ್ತಿರ ಹತ್ತಿರ ಬರಬೇಕಾದರೆ, ಪ್ರಜ್ಞೆಯ ಕ್ಷೇತ್ರವನ್ನು ದಾಟಿ ಹೋಗಬೇಕು. ಎಷ್ಟು ಹೆಚ್ಚು ಹತ್ತಿರ ಬಂದರೆ ಅಷ್ಟೂ ದೇವರ ಹತ್ತಿರ ಬರುತ್ತೀರಿ. ದೇವರಿಗೆ ಪ್ರಮಾಣವಾವುದು? ಪ್ರತ್ಯಕ್ಷಾನುಭವ. ಈ ಗೋಡೆ ಇರುವುದಕ್ಕೆ ಪ್ರಮಾಣ ನಾನು ಅದನ್ನು ನೋಡುವುದು. ಸಾವಿರಾರು ಜನರು ಹೀಗೆ ದೇವರನ್ನು ಅನುಭವಿಸಿರುವರು. ಯಾರು ಅವನನ್ನನುಭವಿಸಬೇಕೆಂದು ಈಗ ಪ್ರಯತ್ನ ಪಡುವರೋ ಅವರೆಲ್ಲರಿಗೂ ಅವನು ದೊರಕುವನು. ಈ ಗ್ರಹಣಶಕ್ತಿ ಇಂದ್ರಿಯ ಗ್ರಹಣವಲ್ಲ. ಇದು ಅತೀಂದ್ರಿಯ ತುರೀಯಾವಸ್ಥೆ. ಈ ಸಾಧನೆ ನಮ್ಮನ್ನು ಇಂದ್ರಿಯಾತೀತರನ್ನಾಗಿ ಮಾಡುವುದಕ್ಕೆ ಆವಶ್ಯಕ. ಹಿಂದಿನ ಕರ್ಮಗಳೆಲ್ಲ ಮತ್ತು ಬಂಧನಗಳೆಲ್ಲ ನಮ್ಮನ್ನು ಅಧೋಗತಿಗೆ ತಂದಿವೆ. ಈ ಸಾಧನೆ ನಮ್ಮನ್ನು ಪರಿಶುದ್ಧರಾಗಿ, ಹಗುರವಾಗಿ ಮಾಡುವುದು, ಬಂಧನಗಳು ತಮಗೆ ತಾವೆ ಕಳಚಿ ಬೀಳುವುವು. ನಾವು ಬದ್ಧರಾಗಿರುವ ಸಾಧಾರಣ ಇಂದ್ರಿಯಗ್ರಹಣದ ಸ್ಥಿತಿಯಿಂದ ಪಾರಾಗಿ ಮೇಲೆ ತೇಲುವೆವು. ಆಗ ನಾವು ಜಾಗೃತ್ ಸ್ವಪ್ನ ಸುಷುಪ್ತಿಗಳಲ್ಲಿ ಕಾಣದ ಕೇಳದ ಭಾವಿಸದ ವಸ್ತುಗಳನ್ನು ಕಾಣುವೆವು, ಕೇಳುವೆವು, ಭಾವಿಸುವೆವು. ಆಗ ನಾವು ಬೇರೆ ಒಂದು ಭಾಷೆಯಲ್ಲಿ ಮಾತನಾಡುವಂತೆ ತೋರುವುದು. ಪ್ರಪಂಚದ ಜನರಿಗೆ ಅದು ಗೊತ್ತಾಗುವುದಿಲ್ಲ. ಏಕೆಂದರೆ ಅವರಿಗೆ ಇಂದ್ರಿಯಕ್ಕೆ ಸಂಬಂಧಪಟ್ಟ ವಿಷಯವಲ್ಲದೆ ಬೇರಾವುದೂ ತಿಳಿಯದು. ನಿಜವಾದ ಆಧ್ಯಾತ್ಮಿಕತೆ ಅತೀಂದ್ರಿಯವಾದುದು. ಪ್ರಪಂಚದಲ್ಲಿರುವ ಪ್ರತಿಯೊಬ್ಬರಲ್ಲಿಯೂ ಇಂದ್ರಿಯವನ್ನು ಮೀರಿಹೋಗುವ ಶಕ್ತಿ ಸುಪ್ತವಾಗಿದೆ. ಒಂದು ಸಣ್ಣ ಕೀಟ ಕೂಡ ಒಂದಲ್ಲ ಒಂದು ದಿನ, ಇಂದ್ರಿಯವನ್ನು ಮೀರಿ ದೇವರನ್ನು ಸೇರುವುದು. ಯಾವ ಜೀವನವೂ ನಿರರ್ಥಕವಾಗುವುದಿಲ್ಲ. ಪ್ರಪಂಚದಲ್ಲಿ ನಿರರ್ಥಕ ಎನ್ನುವುದು ಇಲ್ಲ. ಮನುಷ್ಯ ನೂರುವೇಳೆ ಹಿಂಸೆಗೆ ಒಳಗಾಗಬಹುದು. ಸಾವಿರವೇಳೆ ಬೀಳಬಹುದು. ಆದರೂ ಕೊನೆಗೆ ತಾನು ದೇವರೆಂಬುದನ್ನು ತಿಳಿದುಕೊಳ್ಳುವನು. ಪ್ರಗತಿಯು ಸರಳರೇಖೆಯಂತೆ ಮುಂದುವರಿಯುವುದಿಲ್ಲವೆಂಬುದು ನಿಮಗೆ ಗೊತ್ತಿದೆ. ಪ್ರತಿಯೊಂದು ಜೀವವೂ ಒಂದು ವೃತ್ತದೊಳಗೆ ಸಂಚರಿಸುವಂತೆ ತೋರುವುದು. ಅದನ್ನು ಎಲ್ಲರೂ ಪೂರ್ಣಮಾಡಬೇಕು. ಎಂದಾದರೊಂದು ದಿನ ಮೇಲಕ್ಕೆ ಏಳದೆ ಇರುವ ಅಧಃಪಾತಾಳಕ್ಕೆ ಯಾರೂ ಬೀಳಲಾರರು. ಯಾರೂ ನಾಶವಾಗುವುದಿಲ್ಲ. ನಾವೆಲ್ಲ ಒಂದು ಸಾಮಾನ್ಯ ಕೇಂದ್ರವಾದ ದೇವರಿಂದ ಬಂದಿರುವೆವು. ದೇವರಿಂದ ಬಂದಿರುವ ಅತಿ ಕ್ಷುದ್ರ ಅಥವಾ ಕೀಳು ಪ್ರಾಣಿಯೂ ಎಲ್ಲ ಜೀವಿಗಳ ಪಿತನೆಡೆಗೆ ಕೊನೆಗೆ ಬರಲೇಬೇಕು. “ಯಾರಿಂದ ಎಲ್ಲವೂ ಬಂದಿದೆಯೋ, ಯಾರಲ್ಲಿ ಎಲ್ಲವೂ ಜೀವಿಸಿದೆಯೋ, ಎಲ್ಲಿಗೆ ಕೊನೆಗೆ ಎಲ್ಲವೂ ತೆರಳುವುದೋ ಅದೇ ದೇವರು."

