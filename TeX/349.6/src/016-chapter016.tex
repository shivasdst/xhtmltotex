
\chapter[ಆಧ್ಯಾತ್ಮಿಕ ಸಾಧನೆಗೆ ಸಲಹೆಗಳು]{ಆಧ್ಯಾತ್ಮಿಕ ಸಾಧನೆಗೆ ಸಲಹೆಗಳು\protect\footnote{\engfoot{C.W, Vol. II, P. 24}}}

\begin{center}
(ಲಾಸ್‌ಎಂಜಲೀಸ್‌ನ ಹೋಂ ಆಫ್ ಟ್ರೂತ್‌ನಲ್ಲಿ ನೀಡಿದ ಪ್ರವಚನ)
\end{center}

ಈ ದಿನ ಪ್ರಾಣಾಯಾಮ ಮುಂತಾದ ವಿಷಯಗಳನ್ನು ನಿಮಗೆ ತಿಳಿಸಲು\break ಪ್ರಯತ್ನಿಸುವೆನು. ಇದುವರೆಗೆ ಸಿದ್ಧಾಂತಗಳನ್ನು ಚರ್ಚಿಸುತ್ತಿದ್ದೆವು. ಈಗ ಅನುಷ್ಠಾನಕ್ಕೆ ಸಂಬಂಧಪಟ್ಟ ಕೆಲವು ವಿಷಯಗಳನ್ನು ತಿಳಿದುಕೊಳ್ಳುವುದು ಒಳ್ಳೆಯದು. ಭರತ ಖಂಡದಲ್ಲಿ ಇದಕ್ಕೆ ಸಂಬಂಧಪಟ್ಟ ಹಲವು ಗ್ರಂಥಗಳಿವೆ. ಹೇಗೆ ನಿಮ್ಮ ಜನ ಇತರ ವ್ಯವಹಾರಗಳಲ್ಲಿ ಅನುಷ್ಠಾನಪರರೊ ಹಾಗೆಯೆ ನಮ್ಮ ಜನ ಆಧ್ಯಾತ್ಮಿಕ ವಿಷಯದಲ್ಲಿ ಅನುಷ್ಟಾನಪರರು. ಈ ದೇಶದಲ್ಲಿ (ಅಮೆರಿಕಾದಲ್ಲಿ) ಐದು ಜನ ಒಟ್ಟಿಗೆ ಕಲೆತು ನಾವೊಂದು ಜಾಯಿಂಟ್ ಸ್ಟಾಕ್ ಕಂಪೆನಿ ಸ್ಥಾಪಿಸೋಣ ಎಂದು ಮಾತನಾಡಿಕೊಳ್ಳುವರು. ಅದು ಐದು ಗಂಟೆಗಳಲ್ಲಿ ಆಗುವುದು. ಭರತಖಂಡದಲ್ಲಿ ಐವತ್ತು ವರುಷಗಳಾದರೂ ಇದನ್ನು ಮಾಡಲಾರರು. ಇಂತಹ ವಿಷಯಗಳಲ್ಲಿ ನಮ್ಮ ಜನ ಕಾರ್ಯಚತುರರಲ್ಲ. ಆದರೆ ಯಾವುದಾದರೊಂದು ತಾತ್ತ್ವಿಕ ಸಿದ್ದಾಂತವನ್ನು ಜಾರಿಗೆ ತಂದರೆ ಅದೆಷ್ಟು ವಿಚಿತ್ರವಾಗಿರಲಿ ಅದಕ್ಕೆ ಅನುಯಾಯಿಗಳು ದೊರಕುವರು. ಉದಾಹರಣೆಗೆ ಒಬ್ಬ ಹನ್ನೆರಡು ವರುಷ ಹಗಲು ರಾತ್ರಿ ಒಂಟಿ ಕಾಲಮೇಲೆ ನಿಂತರೆ ಅವನಿಗೆ ಮುಕ್ತಿ ದೊರಕುವುದೆಂಬ ಸಿದ್ದಾಂತವನ್ನು ಜಾರಿಗೆ ತಂದನೆಂದು ಇಟ್ಟುಕೊಳ್ಳಿ. ಅದನ್ನು ಅನುಸರಿಸುವ ನೂರಾರು ಜನ ಅವನಿಗೆ ದೊರಕುವರು. ಎಲ್ಲಾ ಕಷ್ಟಗಳನ್ನೂ ಸಮಾಧಾನದಿಂದ ಸಹಿಸುವರು. ಪುಣ್ಯ ಸಂಪಾದಿಸುವುದಕ್ಕೆ ಕೈಯನ್ನು ಹಲವು ವರುಷ ಎತ್ತಿ ಹಿಡಿದವರು ದೊರಕುವರು. ಇಂತಹ ನೂರಾರು ಜನರನ್ನು ನಾನು ನೋಡಿರುವೆನು. ಅವರೆಲ್ಲ ಮೂರ್ಖರೆಂದು ಭಾವಿಸಬೇಡಿ. ತಮ್ಮ ಬುದ್ಧಿಶಕ್ತಿಯ ವಿಸ್ತಾರ ಮತ್ತು ಆಳದಿಂದ ನಿಮ್ಮನ್ನು ಆಶ್ಚರ್ಯಚಕಿತರನ್ನಾಗಿ ಮಾಡುವರು. ಆದಕಾರಣ ಯಾವುದನ್ನು ಅನುಷ್ಠಾನ ಎನ್ನುವೆವೊ ಅದು ಕೇವಲ ಸಾಪೇಕ್ಷ ಎಂದು ಗೊತ್ತಾಗುವುದು.

ಇತರರನ್ನು ನಾವು ಅಳೆಯುವಾಗ ಯಾವಾಗಲೂ ಈ ಒಂದು ತಪ್ಪನ್ನು ಮಾಡುವೆವು. ಅದು ನಮ್ಮ ಕಿರು ಮಾನಸಿಕ ಜಗತ್ತೇ ಸರ್ವಸ್ವ ಎಂದು ಭಾವಿಸುವುದು. ನಮ್ಮ ನೀತಿ ನಡವಳಿಕೆ, ನಮ್ಮ ಕರ್ತವ್ಯಭಾವನೆ, ನಮಗೆ ಪ್ರಯೋಜನವಾಗಿರುವುದು - ಇಷ್ಟು ಮಾತ್ರ ಪ್ರಪಂಚದಲ್ಲಿ ಇರುವುದಕ್ಕೆ ಯೋಗ್ಯ ಎಂದು ಭಾವಿಸುವೆವು. ನಾನು ಕೆಲವು ದಿನದ ಹಿಂದೆ ಯೂರೋಪಿಗೆ ಹೊರಟಾಗ ಮಾರ್ಸೆಲ್ ಪಟ್ಟಣದ ಮೂಲಕ ಹೋಗುತ್ತಿದ್ದೆ. ಅಲ್ಲಿ ಒಂದು ಗೂಳಿಗಳ ಕಾಳಗ ಆಗ ನಡೆಯುತ್ತಿತ್ತು. ಹಡಗಿನಲ್ಲಿದ್ದ ಇಂಗ್ಲಿಷಿನವರೆಲ್ಲ ಉದ್ವೇಗದಿಂದ ಉನ್ಮತ್ತರಾಗಿ ಕಾಳಗವನ್ನು ಕ್ರೂರ ಎಂದು ಟೀಕಿಸುತ್ತಾ ಬೈಯುತ್ತಿದ್ದರು. ನಾನು ಇಂಗ್ಲೆಂಡಿನಲ್ಲಿ ಇದ್ದಾಗ ಬಹುಮಾನಕ್ಕಾಗಿ ಮುಷ್ಟಿ ಕಾಳಗ ಮಾಡುವ ಕೆಲವರು ಅಲ್ಲಿಂದ ಪ್ಯಾರಿಸ್ಸಿಗೆ ಹೋಗಿದ್ದಾಗ, ಅವರನ್ನು ಫ್ರೆಂಚರು ದಯೆದಾಕ್ಷಿಣ್ಯಗಳಿಲ್ಲದೆ ಒದ್ದು ಕಳುಹಿಸಿದರು ಎಂದು ಕೇಳಿದೆ. ಏಕೆಂದರೆ ಬಹುಮಾನಕ್ಕಾಗಿ ಕಾದಾಡುವುದು ಫ್ರೆಂಚರ ದೃಷ್ಟಿಯಲ್ಲಿ ಪಾಶವಿಕ. `ಹಲವು ದೇಶಗಳಲ್ಲಿ ನಾನು ಇದನ್ನು ಕೇಳಿದಾಗ ಇತರರ ಮೇಲೆ ತೀರ್ಪನ್ನು ನೀಡಲು ಹೋಗಬೇಡ, ಹಾಗೆ ಮಾಡಿದರೆ ಇತರರು ನಿನ್ನ ಮೇಲೆ ತೀರ್ಪನ್ನು ನೀಡುತ್ತಾರೆ' ಎಂಬ ಕ್ರಿಸ್ತನ ಅದ್ಭುತ ವಾಣಿಯನ್ನು ಅರ್ಥಮಾಡಿಕೊಂಡೆನು. ನಮಗೆ ಹೆಚ್ಚು ತಿಳಿದಷ್ಟೂ ನಮ್ಮ ಅಜ್ಞಾನದ ಅರಿವಾಗುವುದು. ಮನುಷ್ಯನ ಮನಸ್ಸಿಗೆ ಎಷ್ಟು ಮುಖಗಳಿವೆ, ಎಷ್ಟು ವೈವಿಧ್ಯದಿಂದ ತುಂಬಿ ತುಳುಕಾಡುತ್ತಿದೆ ಎಂಬುದು ಗೊತ್ತಾಗುವುದು. ನಾನು ಹುಡುಗನಾಗಿದ್ದಾಗ ನಮ್ಮ ದೇಶದ ದೇಹದಂಡನಾತ್ಮಕ ತಪಸ್ಸುಗಳನ್ನು ಟೀಕಿಸುತ್ತಿದ್ದೆ. ನಮ್ಮ ದೇಶದ ಪ್ರಖ್ಯಾತ ಬೋಧಕರೆ ಅವನ್ನು ಟೀಕಿಸಿರುವರು. ಪೃಥ್ವಿಯಲ್ಲಿ ಜನ್ಮತಾಳಿದ ಶ್ರೇಷ್ಠತಮ ಮಾನವನಾದ ಬುದ್ಧನೇ ಅವನ್ನು ಟೀಕಿಸಿರುವನು. ಆದರೂ ನನಗೆ ವಯಸ್ಸಾದಂತೆಲ್ಲ ಅವರನ್ನು ಹಳಿಯುವುದಕ್ಕೆ ನನಗೆ ಅಧಿಕಾರವಿಲ್ಲವೆಂದು ಭಾವಿಸುವೆನು. ಅವರಲ್ಲಿ ಎಷ್ಟೇ ತಪ್ಪು ಇದ್ದರೂ ಅವರಂತೆ ಮಾಡಿ ಅನುಭವಿಸಬಲ್ಲ ಶಕ್ತಿಯ ಒಂದು ಅಂಶ ನನ್ನಲ್ಲಿದ್ದರೆ ಸಾಕು ಎಂದು ಅನೇಕ ವೇಳೆ ನಾನು ಯೋಚಿಸುವೆನು. ಅನೇಕ ವೇಳೆ ನನ್ನ ಟೀಕೆ ಅವರು ಮಾಡುವ ದೇಹ ದಂಡನೆಯ ವಿಷಯದಲ್ಲಿ ಇರುವ ತಿರಸ್ಕಾರದಿಂದಲ್ಲ, ಆದರೆ ಹಾಗೆ ನಾನು ಮಾಡಲಾರೆನಲ್ಲ, ಅದನ್ನು ಮಾಡುವುದಕ್ಕೆ ಧೈರ್ಯ ಸಾಲದಲ್ಲ ಎಂಬ ಕಾರಣಕ್ಕಾಗಿ.

ಬಲ, ಧೈರ್ಯ, ಶಕ್ತಿ ಎನ್ನುವುದೆಲ್ಲ ವಿಚಿತ್ರವಾಗಿ ತೋರುವುದು. ಸಾಧಾರಣವಾಗಿ ಧೈರ್ಯಶಾಲಿ, ಬಲಶಾಲಿ, ಸಾಹಸಿ ಎನ್ನುವೆವು. ಆದರೆ ಆ ಧೈರ್ಯ ಬಲ ಸಾಹಸ ಅಥವಾ ಇನ್ನು ಯಾವುದಾದರೂ ಗುಣವೇ ಆಗಲಿ, ಆ ಮನುಷ್ಯನ ಮುಖ್ಯ ಲಕ್ಷಣವಾಗಿರುವುದಿಲ್ಲ. ಗುಂಡಿನ ಬಾಯಿಗೆ ಧೈರ್ಯದಿಂದ ಧಾವಿಸುವ ಮನುಷ್ಯ ಶಸ್ತ್ರಚಿಕಿತ್ಸಕನ ಕೈಯಲ್ಲಿರುವ ಚಾಕುವಿಗೆ ಅಂಜುವನು. ಮತ್ತೊಬ್ಬ ಎಂದಿಗೂ ಕೋವಿಯ ಬಾಯಿಗೆ ಎದೆಯೊಡ್ಡಲು ಧೈರ್ಯವಿಲ್ಲದವನು ಆವಶ್ಯಕವಾದರೆ ತೀವ್ರ ಶಸ್ತ್ರಚಿಕಿತ್ಸೆಯನ್ನು ಧೈರ್ಯವಾಗಿ ಸಹಿಸುವನು. ಇತರರನ್ನು ನೀವು ಅಳೆಯುವಾಗ ಧೈರ್ಯ, ಮಾಹಾತ್ಮ್ಯ ಎಂಬುದಕ್ಕೆ ನೀವು ಕೊಡುವ ಅರ್ಥವೇನು ಎಂಬುದನ್ನು ಹೇಳಬೇಕು. ಯಾರನ್ನು ಒಳ್ಳೆಯವನಲ್ಲವೆಂದು ನಾನು ಟೀಕಿಸುತ್ತಿರುವೆನೋ ಅವನಲ್ಲಿ, ನನ್ನಲ್ಲಿ ಇಲ್ಲದ ಹಲವು ಒಳ್ಳೆಯ ಗುಣಗಳು ಇರಬಹುದು.

ಮತ್ತೊಂದು ಉದಾಹರಣೆ ತೆಗೆದುಕೊಳ್ಳಿ, ಜನರು ಹೆಂಗಸರು ಏನು ಮಾಡಬಲ್ಲರು, ಗಂಡಸರು ಏನು ಮಾಡಬಲ್ಲರು ಎಂಬುದನ್ನು ಚರ್ಚಿಸುವಾಗ ಅವರು ಯಾವಾಗಲೂ ಈ ತಪ್ಪನ್ನು ಮಾಡುವರು. ಮನುಷ್ಯ ಕಾದಾಡಬಲ್ಲ, ಬೇಕಾದಷ್ಟು ಶ್ರಮವನ್ನು ಸಹಿಸಬಲ್ಲ, ಕಾದಾಡದ, ಇವನಷ್ಟು ಶ್ರಮವನ್ನು ಸಹಿಸಲಾಗದ ಹೆಂಗಸನ್ನು ಗಂಡಸಿನೊಂದಿಗೆ ಹೋಲಿಸಿ ಅಲ್ಲಗಳೆಯುವರು. ಇದು ಅನ್ಯಾಯ. ಸ್ತ್ರೀ ಪುರುಷನಷ್ಟೇ ಧೈರ್ಯಶಾಲಿ, ಪ್ರತಿಯೊಬ್ಬರೂ ತಮ್ಮ ತಮ್ಮ ರೀತಿಯಲ್ಲಿ ಧೈರ್ಯಶಾಲಿಗಳು. ಹೆಂಗಸಿನಷ್ಟು ತಾಳ್ಮೆ, ಸಹಿಷ್ಣುತೆ, ಪ್ರೀತಿಗಳಿಂದ ಯಾವ ಗಂಡಸು ಒಂದು ಶಿಶುವನ್ನು ಪೋಷಿಸಬಲ್ಲ? ಒಬ್ಬರು ಕಾರ್ಯಮಾಡುವ ಧೈರ್ಯ ಪಡೆದಿದ್ದರೆ ಮತ್ತೊಬ್ಬರು\break ಕಷ್ಟವನ್ನು ಸಹಿಸುವ ಧೈರ್ಯ ಪಡೆದಿದ್ದಾರೆ. ಸ್ತ್ರೀಗೆ ಸಾಹಸ ಮಾಡುವುದಕ್ಕೆ ಆಗದಿದ್ದರೆ ಪುರುಷನಿಗೆ ಸ್ತ್ರೀಯಷ್ಟು ಕಷ್ಟವನ್ನು ಸಹಿಸುವುದಕ್ಕೂ ಆಗುವುದಿಲ್ಲ. ವಿಶ್ವದಲ್ಲಿ ನಾವು ಒಂದು ಅದ್ಭುತ ಸಮತೋಲನವನ್ನು ನೋಡುವೆವು. ನನಗೆ ಈಗ ಗೊತ್ತಿಲ್ಲ, ಆದರೆ ಮುಂದೆ ಕೆಲವು ಕಾಲದ ಮೇಲೆ ಕ್ರಿಮಿ ಕೂಡ ಮಾನವನ ಬೆಳವಣಿಗೆಗೆ ಸಹಾಯಕವಾಗಿದೆ ಎಂಬುದನ್ನು ಕಂಡುಹಿಡಿಯಬಹುದು. ಅತಿ ದುಷ್ಟರಲ್ಲಿಯೂ ನನ್ನಲ್ಲಿ ಇಲ್ಲದ ಕೆಲವು ಒಳ್ಳೆ ಗುಣಗಳು ಇರಬಹುದು. ನಾನು ಇದನ್ನು ಜೀವನದಲ್ಲಿ ಪ್ರತಿನಿತ್ಯವೂ ನೋಡುತ್ತಿರುವೆನು. ಕಾಡುಮನುಷ್ಯನನ್ನು ನೋಡಿ, ಎಷ್ಟು ಗಟ್ಟಿಮುಟ್ಟಾದ ದೇಹವಿದೆ. ಇದೇ ದೇಹ ನನಗೆ ಇದ್ದರೆ ಎಷ್ಟು ಚೆನ್ನಾಗಿರುತ್ತಿತ್ತು ಎಂದು ಆಶಿಸುತ್ತೇನೆ. ಅವನು ತನ್ನ ಮನದಣಿಯೆ ಊಟಮಾಡುತ್ತಾನೆ, ಕುಡಿಯುತ್ತಾನೆ; ಅವನಿಗೆ ಕಾಯಿಲೆ ಎಂಬುದೇ ಗೊತ್ತಿಲ್ಲ. ಆದರೆ ನಾನಾದರೊ ಪ್ರತಿದಿನ ರೋಗರುಜಿನಗಳಲ್ಲಿ ನರಳುತ್ತಿರುವೆ. ನನ್ನ ಬುದ್ದಿವಂತಿಕೆ ಕೊಟ್ಟು ಅವನ ದೇಹದಾರ್ಢ್ಯವನ್ನು ತೆಗೆದುಕೊಳ್ಳುವಹಾಗಿದ್ದರೆ ಎಷ್ಟೋ ಚೆನ್ನಾಗಿತ್ತು ಎಂದು ಎಷ್ಟೋ ಸಲ ಅನ್ನಿಸಿದೆ. ಪ್ರಪಂಚವೆಲ್ಲ ಒಂದು ಅಲೆಯ ಏರಿಳಿತದಂತೆ. ಇಳಿತವಿಲ್ಲದೆ ಏರಿಲ್ಲ. ಎಲ್ಲಾ ಕಡೆಯಲ್ಲಿ ಒಂದಕ್ಕೆ ಪ್ರತಿಯಾದದ್ದು ಇದ್ದೇ ಇರುತ್ತದೆ. ನಿಮ್ಮಲ್ಲಿ ಒಂದು ಒಳ್ಳೆಯ ಗುಣವಿದೆ. ನಿಮ್ಮ ನೆರೆಯವನಲ್ಲಿ ಮತ್ತೊಂದು ಒಳ್ಳೆಯ ಗುಣವಿದೆ. ನೀವು ಸ್ತ್ರೀಪುರುಷರನ್ನು ಅಳೆಯುವಾಗ ಅವರವರ ಶ್ರೇಷ್ಠತೆಯ ದೃಷ್ಟಿಯಿಂದ ಅಳೆಯಿರಿ. ಒಬ್ಬ ಮತ್ತೊಬ್ಬನಂತೆ ಇರಲಾರ. ಒಬ್ಬನಿಗೆ ಮತ್ತೊಬ್ಬ ದುಷ್ಟ ಎಂದು ಹೇಳಲು ಅಧಿಕಾರವಿಲ್ಲ. “ಓ! ಇದನ್ನು ಮಾಡಿಬಿಟ್ಟರೆ ಸರ್ವನಾಶವಾಗುವುದು'' ಎಂಬ ಹಳೆಯ ಮೌಢ್ಯವೇ ಇದು. ಆದರೂ ಪ್ರಪಂಚದಲ್ಲಿ ಸರ್ವನಾಶವಾಗಿಲ್ಲ. ಈ ದೇಶದಲ್ಲಿ (ಅಮೆರಿಕಾದಲ್ಲಿ) ನೀಗ್ರೋ ಜನರನ್ನು ಗುಲಾಮಗಿರಿಯಿಂದ ಬಿಡುಗಡೆ ಮಾಡಿದರೆ ದೇಶ ನಾಶವಾಗುವುದೆಂದು ಹೇಳುತ್ತಿದ್ದರು. ಆದರೆ ಹಾಗೆ ಆಯಿತೆ? ಜನಸಾಧಾರಣರಿಗೆ ವಿದ್ಯೆ ಕಲಿಸಿದರೆ ಪ್ರಪಂಚ ಹಾಳಾಗುವುದೆಂದು ಹೇಳುತ್ತಿದ್ದರು. ಆದರೆ ಇದರಿಂದ ಪ್ರಪಂಚ ಮೇಲಾಯಿತು. ಕೆಲವು ವರ್ಷಗಳ ಹಿಂದೆ ಇಂಗ್ಲೆಂಡಿಗೆ ಒದಗಬಹುದಾದ ದುಃಸ್ಥಿತಿಯನ್ನು ವಿವರಿಸುವ ಒಂದು ಗ್ರಂಥ ಬಂತು. ಕೂಲಿಕಾರರ ವೇತನ ಹೆಚ್ಚಿದಂತೆ ಇಂಗ್ಲೆಂಡಿನ ವ್ಯಾಪಾರ ಇಳಿಯುತ್ತಿದೆ ಎಂದು ಲೇಖಕರು ಬರೆದಿದ್ದರು. ಇಂಗ್ಲೆಂಡಿನ ಕೂಲಿಯವರು ಹೆಚ್ಚು ಸಂಬಳಕ್ಕಾಗಿ ಕಾದಾಡುತ್ತಿರುವರೆಂದೂ ಜರ್ಮನ್ನರು ಕಡಮೆ ಸಂಬಳಕ್ಕೆ ಮಾಡುತ್ತಿರುವರೆಂದೂ ಹೇಳಿದ್ದರು. ಇದನ್ನು ತಿಳಿಯುವುದಕ್ಕೆ ಜರ್ಮನಿಗೆ ಒಂದು ಸಮಿತಿಯನ್ನು ಕಳುಹಿಸಿದರು. ಅವರು ವರದಿಯಲ್ಲಿ ಜರ್ಮನಿ ಕೂಲಿಗಾರರಿಗೆ ಹೆಚ್ಚು ಕೂಲಿ ಬರುತ್ತಿದೆ ಎಂದು ನಮೂದಿಸಿದರು. ಇದೇತಕ್ಕೆ? ಸಾಮಾನ್ಯ ಜನರು ಹೆಚ್ಚು ವಿದ್ಯೆ ಪಡೆದಿದ್ದುದರಿಂದ, ಜನಸಾಧಾರಣರು ವಿದ್ಯಾವಂತರಾದರೆ ಪ್ರಪಂಚವೇ ಸರ್ವ ನಾಶವಾಗುವುದು ಎಂಬುದೇನಾಯಿತು? ಭರತಖಂಡದಲ್ಲಿ ಪೂರ್ವಾಚಾರ ಪರಾಯಣರನ್ನು ದೇಶಾದ್ಯಂತವೂ ನೋಡುತ್ತೇವೆ. ಅವರು ಜನಸಾಮಾನ್ಯರಿಂದ ಎಲ್ಲವನ್ನೂ ರಹಸ್ಯವಾಗಿಡಲು ಯತ್ನಿಸುವರು. ಅವರು, ತಾವೇ ಪ್ರಪಂಚದ ಶ್ರೇಷ್ಠತಮ ವ್ಯಕ್ತಿಗಳೆಂದು ಭಾವಿಸುವರು. ಇಂತಹ ಅಪಾಯಕರವಾದ ಪ್ರಯೋಗಗಳಿಂದ ತಮಗೆ ಅಪಾಯವಿಲ್ಲ, ಕೇವಲ ಜನಸಾಮಾನ್ಯರಿಗೆ ಮಾತ್ರ ಅಪಾಯ ಎಂದು ಭಾವಿಸುವರು.

ಈಗ ಅನುಷ್ಠಾನ ಪ್ರಸಂಗಕ್ಕೆ ಬರೋಣ. ಬಹಳ ಹಿಂದಿನಕಾಲದಿಂದಲೂ ಮನಶ್ಶಾಸ್ತ್ರವನ್ನು ಅನುಷ್ಠಾನದಲ್ಲಿ ತರುವುದಕ್ಕೆ ಪ್ರಯತ್ನಿಸಿರುವರು. ಕ್ರಿಸ್ತ ಹುಟ್ಟುವುದಕ್ಕೆ ಸಾವಿರದ ನಾನೂರು ವರುಷಗಳ ಹಿಂದೆ ಭರತಖಂಡದಲ್ಲಿ ಪತಂಜಲಿ ಎಂಬ ಪ್ರಖ್ಯಾತ ದಾರ್ಶನಿಕನಿದ್ದನು. ಮಾನಸಿಕ ಶಾಸ್ತ್ರದಲ್ಲಿ ದೊರಕುವ ವಿಷಯಗಳನ್ನೆಲ್ಲಾ ಆತ ಸಂಗ್ರಹಿಸಿದ. ಹಿಂದಿನಿಂದ ಬಂದ ಅನುಭವಗಳ ಸಹಾಯವನ್ನೂ ಪಡೆದ. ಈ ಪ್ರಪಂಚ ಬಹಳ ಹಳೆಯದೆಂಬುದನ್ನು ಜ್ಞಾಪಕದಲ್ಲಿಡಿ. ಇದೆಲ್ಲೂ ಎರಡು ಮೂರು ಸಾವಿರ ವರುಷಗಳ ಹಿಂದೆ ಸೃಷ್ಟಿಯಾದದ್ದಲ್ಲ. ಪಾಶ್ಚಾತ್ಯ ದೇಶಗಳಲ್ಲಿ ಸಮಾಜ ನ್ಯೂ ಟೆಸ್ಟಮೆಂಟಿನೊಂದಿಗೆ ಕ್ರಿ.ಪೂ.೧೮೦೦ ವರುಷಗಳ ಹಿಂದೆ ಜಾರಿಗೆ ಬಂತೆಂದು ಹೇಳುವರು. ಇದಕ್ಕೆ ಮುಂಚೆ ಸಮಾಜವೇ ಇರಲಿಲ್ಲವಂತೆ. ಇದು ಪಾಶ್ಚಾತ್ಯ ದೇಶಗಳಿಗೆ ಸತ್ಯವಾಗಿರಬಹುದು. ಆದರೆ ಇಡೀ ಸೃಷ್ಟಿಗೆ ಇದು ಅನ್ವಯಿಸುವುದಿಲ್ಲ. ನಾನು ಲಂಡನ್ನಿನಲ್ಲಿ ಉಪನ್ಯಾಸ\break ಮಾಡುತ್ತಿದ್ದಾಗ ವಿದ್ಯಾವಂತನಾದ ಬುದ್ಧಿವಂತ ಸ್ನೇಹಿತನೊಬ್ಬ ನನ್ನೊಂದಿಗೆ ಇದರ ವಿಷಯವಾಗಿ ವಾದಿಸುತ್ತಿದ್ದ. ಒಂದು ದಿನ ತನ್ನ ವಾದವನ್ನೆಲ್ಲ ವ್ಯರ್ಥಮಾಡಿದ ಮೇಲೆ ತಕ್ಷಣ “ನಿಮ್ಮ ಋಷಿಗಳೇಕೆ ನಮಗೆ ಬೋಧಿಸುವುದಕ್ಕೆ ಇಂಗ್ಲೆಂಡಿಗೆ ಬರಲಿಲ್ಲ?” ಎಂದು ಕೇಳಿದನು. ಅದಕ್ಕೆ ನಾನು “ಬರುವುದಕ್ಕೆ ಆಗ ಇಂಗ್ಲೆಂಡೇ ಇರಲಿಲ್ಲ, ಅವರು ಕಾಡಿಗೆ ಬೋಧಿಸಬೇಕಾಗಿತ್ತೆ!?” ಎಂದೆ. ಇಂಗರ್‌ಸಾಲ್ ನನಗೆ, “ಐವತ್ತು ವರುಷದ ಹಿಂದೆ ನೀನು ಈ ದೇಶಕ್ಕೆ ಪ್ರಚಾರಕನಾಗಿ ಬಂದಿದ್ದರೆ ಜನ ನಿನ್ನನ್ನು ಫಾಸಿಗೆ ಹಾಕುತ್ತಿದ್ದರು. ನಿನ್ನನ್ನು ಜೀವಸಹಿತ ಸುಡುತ್ತಿದ್ದರು. ಇಲ್ಲವೆ ಕಲ್ಲೆಸೆದು ಇಲ್ಲಿಂದ ಅಟ್ಟುತ್ತಿದ್ದರು" ಎಂದನು.

ಆದಕಾರಣ ಕ್ರಿ.ಪೂ. ೧೪೦೦ ವರುಷಕ್ಕೆ ಹಿಂದೆಯೇ ನಾಗರಿಕತೆ ಇತ್ತು ಎಂದು ಊಹಿಸುವುದರಲ್ಲಿ ಏನೂ ದೋಷವಿಲ್ಲ. ಸಂಸ್ಕೃತಿ ಕೆಳಗಿನಿಂದ ಮೇಲಕ್ಕೆ ಬಂದಿದೆಯೆ ಎಂಬುದನ್ನು ಇನ್ನೂ ಇತ್ಯರ್ಥ ಮಾಡಿಲ್ಲ. ಈ ಊಹೆಯನ್ನು ಸಾಧಿಸುವುದಕ್ಕೆ ತರುವ ವಾದಗಳನ್ನೆ ಕಾಡುಮನುಷ್ಯನು ಹೀನಸ್ಥಿತಿಗೆ ಇಳಿದ ನಾಗರಿಕ ಎಂದು ಸಾಧಿಸುವುದಕ್ಕೂ ಉಪಯೋಗಿಸಬಹುದು. ಚೈನಾ ದೇಶದವರು ನಾಗರಿಕತೆ ಅನಾಗರಿಕ ಜನಾಂಗದಿಂದ ವಿಕಾಸವಾಗಿ ಬಂತು ಎಂಬುದನ್ನು ನಂಬುವುದೇ ಇಲ್ಲ. ಏಕೆಂದರೆ ಅವರ ಅನುಭವ ಇದಕ್ಕೆ ವಿರೋಧವಾಗಿದೆ. ಆದರೆ ನೀವು ಅಮೆರಿಕಾ ಸಂಸ್ಕೃತಿಯನ್ನು ಕುರಿತು ಮಾತನಾಡುತ್ತಿದ್ದರೆ ಅದು ಕೇವಲ ನಿಮ್ಮ ಜನಾಂಗದ ಅಭಿವೃದ್ಧಿ ಮತ್ತು ಅದೇ ಜನಾಂಗ ಮುಂದುವರಿಯಬೇಕು ಎನ್ನುವ ದೃಷ್ಟಿಯಲ್ಲಿ ಹೇಳುವಿರಿ.

ಕಳೆದ ಏಳು ಶತಮಾನಗಳಿಂದ ಅವನತಿ ಹೊಂದುತ್ತಿರುವ ಹಿಂದೂಗಳು ಹಿಂದೆ ಸುಸಂಸ್ಕೃತರಾಗಿದ್ದರೆಂದು ನಂಬುವುದು ಸುಲಭ. ಇದು ಹಾಗಲ್ಲ ಎಂದು ನಾವು ತೋರಿಸಲಾರೆವು.

ಯಾವ ಒಂದು ನಾಗರಿಕತೆಯೂ ತನ್ನಷ್ಟಕ್ಕೆ ತಾನೆ ಅಭಿವೃದ್ದಿಯಾಗಲಿಲ್ಲ. ಮತ್ತೊಂದು ನಾಗರಿಕ ಜನಾಂಗ ಬಂದು ಒಂದು ನಾಗರಿಕ ಜನಾಂಗದೊಂದಿಗೆ ಸೇರಿ ಕಸಿಯಾದಾಗ ಮಾತ್ರ ಆ ಜನಾಂಗ ಹೆಚ್ಚು ನಾಗರಿಕವಾಗಿದೆ. ನಾಗರಿಕತೆಯ ಮೂಲ ಯಾವುದೊ ಒಂದೋ ಎರಡೋ ಜನಾಂಗದಲ್ಲಿರಬೇಕು. ಆ ಜನಾಂಗವು ಹೊರಗೆ ಹೋಗಿ ಬೇರೆ ಜನಾಂಗಗಳೊಡನೆ ಬೆರೆತು ತನ್ನ ಭಾವನೆಗಳನ್ನು ಹರಡುವುದರ ಮೂಲಕ ನಾಗರಿಕತೆಯ ಬೆಳವಣಿಗೆಗೆ ಕಾರಣವಾಗುತ್ತದೆ.

ವ್ಯವಹಾರದ ದೃಷ್ಟಿಯಿಂದ ಆಧುನಿಕ ವೈಜ್ಞಾನಿಕ ಭಾಷೆಯಲ್ಲಿ ಮಾತನಾಡೋಣ. ಆದರೆ ಹೇಗೆ ಧರ್ಮದಲ್ಲಿ ಅಂಧಶ್ರದ್ಧೆ ಇದೆಯೋ ಹಾಗೆಯೆ ವಿಜ್ಞಾನದಲ್ಲಿಯೂ ಅಂಧಶ್ರದ್ಧೆ ಇದೆ ಎಂಬುದನ್ನು ನೀವು ನೆನಪಿನಲ್ಲಿಡಬೇಕು. ಕೆಲವು ಪುರೋಹಿತರು ಧಾರ್ಮಿಕ ವಿಷಯದಲ್ಲಿ ವೈಶಿಷ್ಟ್ಯ ಪಡೆದಿರುವರು. ಮತ್ತೆ ಕೆಲವರು ಅದರಂತೆಯೆ ಭೌತಿಕ ನಿಯಮಗಳಲ್ಲಿ ಪುರೋಹಿತರಾದ ವಿಜ್ಞಾನಿಗಳಿರುವರು. ಡಾರ್ವಿನ್, ಹಕ್ಸ್ಲೆ ಮುಂತಾದ ಪ್ರಖ್ಯಾತ ವೈಜ್ಞಾನಿಗಳ ಹೆಸರನ್ನು ಕೇಳಿದ ತಕ್ಷಣ ಅವರು ಹೇಳಿದ್ದನ್ನು ವಿವೇಚನೆ ಇಲ್ಲದೆ ಒಪ್ಪಿಕೊಳ್ಳುವೆವು. ಈಗಿನ ಕಾಲದ ಒಂದು ಫ್ಯಾಶನ್ ಇದು. ನಾವು ಯಾವುದನ್ನು ವೈಜ್ಞಾನಿಕ ಜ್ಞಾನ ಎನ್ನುವವೊ ಅದರಲ್ಲಿ ೧೦೦ಕ್ಕೆ ೯೯ ಭಾಗ ಕೇವಲ ಊಹಾ ಪ್ರತಿಜ್ಞೆಗಳು ಮಾತ್ರ. ಇವು ಹಿಂದಿನ ಕಾಲದಲ್ಲಿ ರೂಢಿಯಲ್ಲಿದ್ದ ಅನೇಕ ತಲೆಕಾಲುಗಳಿಂದ ಕೂಡಿದ ಭೂತಪ್ರೇತಗಳ ಮೂಢನಂಬಿಕೆಗಿಂತ ಮೇಲಲ್ಲ. ಆಗಿನವರು ಮನುಷ್ಯನನ್ನು ಕಲ್ಲು ಮರಕ್ಕಿಂತ ಬೇರೆ ಎನ್ನುತ್ತಿದ್ದರು. ಅಷ್ಟೆ ವ್ಯತ್ಯಾಸ. ನಿಜವಾದ ವಿಜ್ಞಾನ ಎಚ್ಚರಿಕೆಯಿಂದಿರಿ ಎನ್ನುವುದು. ನಾವು ಹೇಗೆ ಧಾರ್ಮಿಕ ಪುರೋಹಿತರ ಹತ್ತಿರ ಎಚ್ಚರಿಕೆಯಿಂದಿರಬೇಕೊ ಹಾಗೆಯೇ ವೈಜ್ಞಾನಿಕ ಪುರೋಹಿತರ ಹತ್ತಿರವೂ ಎಚ್ಚರಿಕೆಯಿಂದಿರಬೇಕು. ಅಪನಂಬಿಕೆಯಿಂದ ಹೊರಡಿ. ಪ್ರತಿಯೊಂದನ್ನೂ ವಿಶ್ಲೇಷಣೆಮಾಡಿ, ಪರೀಕ್ಷೆ ಮಾಡಿ; ಅನಂತರ ಸ್ವೀಕರಿಸಿ, ಹೆಚ್ಚು ಬಳಕೆಯಲ್ಲಿರುವ ಆಧುನಿಕ ಕಾಲದ ಹಲವು ವೈಜ್ಞಾನಿಕ ಭಾವನೆಗಳು ಇನ್ನೂ ಪ್ರಮಾಣೀಕೃತವಾಗಿಲ್ಲ. ಗಣಿತವೆಂಬ ವಿಜ್ಞಾನಶಾಸ್ತ್ರದಲ್ಲಿ ಕೂಡ ಅದರ ಮುಕ್ಕಾಲುಪಾಲು ಸಿದ್ದಾಂತಗಳು ಕೇವಲ ಊಹಾಪ್ರತಿಜ್ಞೆಗಳಾಗಿ (\enginline{Working hypotheses}) ಪ್ರಯೋಜನಕಾರಿಯಾಗಿವೆಯೇ ಹೊರತು ಅವನ್ನು ಪೂರ್ಣ ಸತ್ಯ ಎಂದು ಇನ್ನೂ ಸಾಧಿಸಿಲ್ಲ. ಜ್ಞಾನ ಹೆಚ್ಚು ವಿಕಾಸವಾದಂತೆ ಅವನ್ನು ಆಚೆಗೆ ಎಸೆಯುವರು.

ಕ್ರಿ. ಪೂ. ೧೪೦೦ರಲ್ಲಿ ಒಬ್ಬ ಮಹಾಋಷಿ ಮಾನಸಿಕ ವಿಷಯಗಳನ್ನು ವಿಶ್ಲೇಷಣೆ ಮಾಡಿ ಜೋಡಿಸಿ ಅವನ್ನು ಒಂದು ಸಾಮಾನ್ಯ ವಿಧಾನಕ್ಕೆ ತರಲು ಪ್ರಯತ್ನಿಸಿದನು. ಹಲವರು ಅವನನ್ನು ಅನುಸರಿಸಿ ಈ ಮಾನಸಿಕ ವಸ್ತುವನ್ನೇ ಪ್ರತ್ಯೇಕವಾಗಿ ಅಧ್ಯಯನಮಾಡಿದರು. ಪುರಾತನ ಜನಾಂಗಗಳಲ್ಲೆಲ್ಲಾ ಹಿಂದೂಗಳು ಮಾತ್ರ ಈ ಜ್ಞಾನವನ್ನು ಉತ್ಸಾಹದಿಂದ ಅಧ್ಯಯನ ಮಾಡಲು ಉಪಕ್ರಮಿಸಿದರು. ನಾನು ಈಗ ಅದನ್ನೆ ನಿಮಗೆ ಬೋಧಿಸುತ್ತಿರುವೆ. ಆದರೆ ಎಷ್ಟು ಜನ ಅದನ್ನು ಅಭ್ಯಾಸ ಮಾಡುವಿರಿ? ಎಷ್ಟು ದಿನ ಅಥವಾ ತಿಂಗಳಲ್ಲಿ ನೀವು ಇದನ್ನು ತೊರೆಯುವಿರಿ? ಈ ವಿಷಯದಲ್ಲಿ ನೀವು ವ್ಯವಹಾರಪರರಲ್ಲ. ಭಾರತ ದೇಶದಲ್ಲಿಯಾದರೆ ಶತ ಶತಮಾನಗಳು ಅದನ್ನು ಬಿಡದೆ ಅನುಷ್ಠಾನ ಮಾಡುವರು. ಭಾರತ ದೇಶದಲ್ಲಿ ಇಲ್ಲಿರುವಂತೆ ಚರ್ಚುಗಳು, ಸಾಮಾನ್ಯ ಪ್ರಾರ್ಥನೆ ಮುಂತಾದುವು ಯಾವುದೂ ಇಲ್ಲ. ಆದರೆ ಪ್ರತಿದಿನವೂ ಪ್ರಾಣಾಯಾಮ ಮಾಡಿ ಜನರು ಧ್ಯಾನಮಾಡಲು ಪ್ರಯತ್ನಿಸುವರು. ಇದೇ ಅವರ ಸಾಧನೆಯ ಮುಖ್ಯ ಭಾಗ. ಇವೇ ಮುಖ್ಯವಾದ ವಿಷಯಗಳು. ಪ್ರತಿಯೊಬ್ಬ ಹಿಂದುವೂ ಇವನ್ನು ಮಾಡಬೇಕು. ಇದು ದೇಶದ ಧರ್ಮ, ಪ್ರತಿಯೊಬ್ಬನಿಗೂ ಬೇರೊಂದು ಮಾರ್ಗವಿರಬಹುದು. ಬೇರೆ ಬೇರೆ ಪ್ರಾಣಾಯಾಮ, ಬೇರೆ ಬೇರೆ ಇಷ್ಟದೇವತೆಗಳಿರಬಹುದು. ಯಾವುದು ಒಬ್ಬನ ವಿಶಿಷ್ಟ ಪಥವೋ ಅದು ಅವನ ಹೆಂಡತಿಗೆ ಕೂಡ ಗೊತ್ತಿಲ್ಲದೆ ಇರಬಹುದು. ತಂದೆಗೆ ಮಗನ ಸಾಧನೆಯ ಮಾರ್ಗ ತಿಳಿಯದೆ ಇರಬಹುದು. ಆದರೆ ಅವರೆಲ್ಲ ಇದನ್ನು ಮಾಡಲೇಬೇಕಾಗಿದೆ. ಇವುಗಳ ವಿಷಯವಾಗಿ ರಹಸ್ಯವೇನೂ ಇಲ್ಲ. ರಹಸ್ಯಕ್ಕೂ ಇದಕ್ಕೂ ಏನೂ ಸಂಬಂಧವಿಲ್ಲ. ಗಂಗಾನದಿಯ ತೀರದಲ್ಲಿ ಪ್ರತಿದಿನ ಸಹಸ್ರಾರು ಜನ ಪ್ರಾಣಾಯಾಮ ಮಾಡುತ್ತ ಯಾವುದೋ ದೇವರನ್ನು ಕುರಿತು ಕಣ್ಣು ಮುಚ್ಚಿಕೊಂಡು ಧ್ಯಾನ ಮಾಡುವುದನ್ನು ನೋಡಬಹುದು. ಜನಸಾಧಾರಣರಿಗೆ ಕೆಲವು ಸಾಧನ ಮಾರ್ಗಗಳನ್ನು ಅನುಸರಿಸಲು ಆಗದೆ ಇರುವುದಕ್ಕೆ ಎರಡು ಕಾರಣಗಳಿರಬಹುದು. ಒಂದನೆಯದೆ ಬೋಧಕರು ಸಾಧಾರಣ ಜನ ಇದಕ್ಕೆ ಅರ್ಹರಲ್ಲ ಎಂದು ಭಾವಿಸಿರುವುದು. ಇದರಲ್ಲಿ ಸ್ವಲ್ಪ ಸತ್ಯವಿರಬಹುದು. ಆದರೆ ಇದನ್ನು ಹೇಳುವುದಕ್ಕೆ ಕಾರಣ ತಮ್ಮ ಸಮಾನ ಯಾರೂ ಇಲ್ಲವೆಂಬ ಅಹಂಕಾರ. ಎರಡನೆಯದೆ ಎಲ್ಲಿ ಮತ್ತೊಬ್ಬರಿಂದ ತೊಂದರೆಗೆ ಒಳಗಾಗುವೆನೋ ಎಂಬ ಭಯ. ಅಮೆರಿಕಾ ದೇಶದಲ್ಲಿ ಬಹಿರಂಗವಾಗಿ ಯಾರೂ ಪ್ರಾಣಾಯಾಮವನ್ನು ಮಾಡಲಿಚ್ಚಿಸುವುದಿಲ್ಲ. ಏಕೆಂದರೆ ಇದೊಂದು ವಿಚಿತ್ರ ಎಂದು ಜನ ಭಾವಿಸುವರು. ಇಲ್ಲಿ ಇದು ಫ್ಯಾಶನ್ ಅಲ್ಲ. ಆದರೆ ಭಾರತ ದೇಶದಲ್ಲಿ `ದೇವರೆ ಇಂದಿನ ಊಟವನ್ನು ಇಂದು ಒದಗಿಸು' ಎಂದು ಪ್ರಾರ್ಥಿಸಿದರೆ ಜನ ಅದನ್ನು ನೋಡಿ ನಗುವರು. ಭಾರತ ದೇಶದಲ್ಲಿ “ಸ್ವರ್ಗದಲ್ಲಿರುವ ನಮ್ಮ ತಂದೆ'' ಎಂದು ಹೇಳುವುದಕ್ಕಿಂತ ಹೆಚ್ಚಿನ ಮೂರ್ಖತನ ಯಾವುದೂ ಇಲ್ಲ. ಹಿಂದೂಗಳು ಪೂಜೆ ಮಾಡುವಾಗ ದೇವರು ತಮ್ಮ ಅಂತರಾಳದಲ್ಲಿರುವನು ಎಂದು ಭಾವಿಸುವರು.

ಯೋಗಶಾಸ್ತ್ರದ ಪ್ರಕಾರ ಮೂರು ನರಗಳಿವೆ. ಒಂದನ್ನು ಇಡ, ಮತ್ತೊಂದನ್ನು ಪಿಂಗಳ ಎನ್ನುವರು. ಮಧ್ಯದಲ್ಲಿರುವುದನ್ನು ಸುಷುಮ್ನ ಎನ್ನುವರು. ಇವೆಲ್ಲ ಬೆನ್ನುಮೂಳೆ ಒಳಗೆ ಇವೆ. ಎಡಗಡೆ ಬಲಗಡೆ ಇರುವ ಇಡಾ ಮತ್ತು ಪಿಂಗಳ ಎಂಬುವು ನರಗಳ ಕಂತೆ. ಸುಷುಮ್ನ ಎಂಬುದು ಮಧ್ಯದಲ್ಲಿ ಟೊಳ್ಳಾಗಿರುವ ಭಾಗ, ನರವಲ್ಲ. ಸುಷುಮ್ನ ಮುಚ್ಚಿದೆ. ಸಾಧಾರಣ ಮನುಷ್ಯನಿಗೆ ಇದರಿಂದ ಏನೂ ಪ್ರಯೋಜನವಿಲ್ಲ. ಏಕೆಂದರೆ ಅವನು ಇಡಾ ಮತ್ತು ಪಿಂಗಳದ ಮೂಲಕ ಮಾತ್ರ ಕೆಲಸ ಮಾಡುತ್ತಿರುವನು. ಇವುಗಳ ಮೂಲಕ ಶಕ್ತಿ ಯಾವಾಗಲೂ ಬರುತ್ತ ಹೋಗುತ್ತ ಇರುವುದು. ದೇಹಕ್ಕೆಲ್ಲ ಅಪ್ಪಣೆಯನ್ನು ಇತರ ನರಗಳ ಮೂಲಕ ಇಡ ಮತ್ತು ಪಿಂಗಳ ಒಯ್ಯುತ್ತಿರುವುವು.

ಪ್ರಾಣಾಯಾಮದ ಮುಖ್ಯ ಉದ್ದೇಶವೆ ಇಡಾಪಿಂಗಳಗಳನ್ನು ಲಯಬದ್ದಗೊಳಿಸುವುದು. ಆದರೆ ಬರೇ ಈ ಕ್ರಿಯೆಯಿಂದಲೇ ಏನೂ ಆಗಿಬಿಡುವುದಿಲ್ಲ. ಇದರಿಂದ ಶ್ವಾಸಕೋಶಗಳಿಗೆ ಸ್ವಲ್ಪ ಗಾಳಿಯನ್ನು ತುಂಬಿದಂತಾಗುತ್ತದೆ. ಇದರಿಂದ ಸ್ವಲ್ಪ ರಕ್ತ\break ಶುದ್ದಿಯಾಗುವುದು ಅಷ್ಟೆ. ಅದಕ್ಕಿಂತ ಹೆಚ್ಚು ಏನೂ ಇಲ್ಲ. ರಕ್ತಶುದ್ದಿಗಾಗಿ ನಾವು ಉಸಿರನ್ನು ಸೆಳೆದುಕೊಳ್ಳುವುದರಲ್ಲಿ ಯಾವ ರಹಸ್ಯವೂ ಇಲ್ಲ. ಇದೊಂದು ಚಲನೆ ಮಾತ್ರ. ಈ ಚಲನೆಯನ್ನು ನಾವು ಪ್ರಾಣಾಯಾಮ ಎನ್ನುವೆವು. ಎಲ್ಲಾ ಕಡೆಯಲ್ಲಿಯೂ ಇರುವ ಚಲನೆಗಳೆಲ್ಲ ಈ ಪ್ರಾಣದ ವಿವಿಧ ಆವಿರ್ಭಾವಗಳು. ಈ ಪ್ರಾಣವೇ ವಿದ್ಯುತ್ ಮತ್ತು ಅಯಸ್ಕಾಂತ ಶಕ್ತಿ. ಮೆದುಳು ಇದನ್ನೆ ಆಲೋಚನೆಯಂತೆ ಹೊರಗೆಡಹುವುದು. ಎಲ್ಲವೂ ಪ್ರಾಣಮಯ. ಸೂರ್ಯ ಚಂದ್ರ ತಾರಾವಳಿಗಳು ಚಲಿಸುವಂತೆ ಮಾಡುತ್ತಿರುವುದೇ ಇದು. ವಿಶ್ವದಲ್ಲಿರುವುದೆಲ್ಲ ಪ್ರಾಣಸ್ಪಂದನದಿಂದ ಆವಿರ್ಭವಿಸಿದೆ. ಈ ಸ್ಪಂದನದ ಶ್ರೇಷ್ಠ ಪ್ರತಿಫಲವೇ ಆಲೋಚನೆ. ಯಾವುದಾದರೂ ಇದಕ್ಕೂ ಮೇಲಿದ್ದರೆ ಅದನ್ನು ನಾವು ಗ್ರಹಿಸಲಾರೆವು. ಇಡ ಪಿಂಗಳ ನರಗಳು ಪ್ರಾಣದಿಂದ ಚಲಿಸುವುವು. ಪ್ರಾಣವೇ ದೇಹದಲ್ಲಿ ಹಲವು ಬಗೆಯ ಶಕ್ತಿಯಾಗಿ ಎಲ್ಲವನ್ನೂ ಚಲಿಸುವಂತೆ ಮಾಡುತ್ತಿರುವುದು. ದೇವರು ಎಲ್ಲೋ ಕುಳಿತುಕೊಂಡು ನ್ಯಾಯ ತೀರ್ಪನ್ನು ಕೊಡುತ್ತಿರುವನು, ಕರ್ಮಕ್ಕೆ ತಕ್ಕ ಪ್ರತಿಫಲವನ್ನು ನೀಡುತ್ತಿರುವನು ಎಂಬ ಹಳೆಯ ಭಾವನೆಯನ್ನು ತ್ಯಜಿಸಿ. ಕೆಲಸ ಮಾಡುವಾಗ ನಮಗೆ ಸಾಕಾಗುವುದು, ಏಕೆಂದರೆ ನಾವು ಅಷ್ಟೊಂದು ಪ್ರಾಣವನ್ನು ಉಪಯೋಗಿಸುವೆವು.

ಪ್ರಾಣಾಯಾಮ ನಮ್ಮ ಉಚ್ಛ್ವಾಸನಿಶ್ವಾಸಗಳನ್ನು ಕ್ರಮಪಡಿಸುವುದು. ಪ್ರಾಣ ಸರಿಯಾಗಿ ಕೆಲಸ ಮಾಡುತ್ತಿರುವಾಗ ಇತರ ಕೆಲಸಗಳೆಲ್ಲ ಸರಿಯಾಗಿ ನಡೆಯುವುವು. ಯೋಗಿಗಳು ತಮ್ಮ ದೇಹದ ಮೇಲೆ ನಿಯಂತ್ರಣ ಸಾಧಿಸಿದಾಗ ದೇಹದ ಯಾವುದಾದರೊಂದು ಕಡೆ ರೋಗರುಜಿನಗಳಿದ್ದರೆ ಅಲ್ಲಿ ಪ್ರಾಣ ಸರಿಯಾಗಿ ಕೆಲಸ ಮಾಡುತ್ತಿಲ್ಲ ಎಂದು ತಿಳಿದು ಪ್ರಾಣವನ್ನು ಅಲ್ಲಿ ಕ್ರಮಗೊಳಿಸುವರು. ಆಗ ಅದು ಗುಣವಾಗುವುದು.

ಹೇಗೆ ನಿಮ್ಮ ದೇಹದ ಪ್ರಾಣವನ್ನು ನೀವು ನಿಗ್ರಹಿಸಬಲ್ಲಿರೊ ಹಾಗೆಯೆ ನೀವು ಪ್ರಬಲರಾಗಿದ್ದರೆ, ಇಲ್ಲಿಯೇ ಇದ್ದುಕೊಂಡು ಭಾರತ ದೇಶದಲ್ಲಿನ ಮತ್ತೊಬ್ಬನ ಪ್ರಾಣವನ್ನು ನಿಗ್ರಹಿಸಬಹುದು. ಇದೆಲ್ಲ ಒಂದು. ಇಲ್ಲಿ ಯಾವ ಅಂತರವೂ ಇಲ್ಲ. ಐಕ್ಯವಾಗಿರುವುದೆ ನಿಯಮ. ದೈಹಿಕ, ಮಾನಸಿಕ, ನೈತಿಕ, ಭೌತಾತೀತವಾಗಿ ಇದೆಲ್ಲ ಒಂದು. ಜೀವನ ಒಂದು ಸ್ಪಂದನ ಮಾತ್ರ. ಯಾವುದು ಆಕಾಶವು ಸ್ಪಂದಿಸುವಂತೆ ಮಾಡುವುದೋ ಅದೇ ನಿಮ್ಮ ಸ್ಪಂದನೆಗೂ ಕಾರಣ. ಹೇಗೆ ಸರೋವರದಲ್ಲಿ ಘನ ಘನತರವಾದ ಹಿಮರಾಶಿಗಳು ರೂಪಗೊಳ್ಳುತ್ತಿವೆಯೊ, ಅಥವಾ ಆವಿಯ ಸಾಗರದಲ್ಲಿ ಹೇಗೆ ಸಾಂದ್ರತೆ ಹೆಚ್ಚು ಕಡಮೆಯಾಗಿದೆಯೊ ಇದರಂತೆಯೆ ವಿಶ್ವವು ಕೂಡ ಭೌತವಸ್ತುವಿನ ಒಂದು ಸಾಗರ. ಇದೊಂದು ಆಕಾಶದ ಸಾಗರ. ಇದರಲ್ಲಿ ಸೂರ್ಯ ಚಂದ್ರ ತಾರೆಗಳು ಮತ್ತು ನಾವು ಇರುವೆವು. ಇವುಗಳಲ್ಲಿ ಬೇರೆ ಬೇರೆ ಸಾಂದ್ರತೆ ಇವೆ. ಆದರೆ ನಿರಂತರತೆಯು ಭಂಗವಾಗಿಲ್ಲ. ಎಲ್ಲಾ ಕಡೆಯಲ್ಲಿಯೂ ಅಖಂಡವಾಗಿರುವುದು.

ನಾವು ತತ್ತ್ವಶಾಸ್ತ್ರವನ್ನು ಓದಿದರೆ ಜಗತ್ತೆಲ್ಲ ಒಂದು ಎಂದು ನಮಗೆ ಗೊತ್ತಾಗುವುದು. ಆಧ್ಯಾತ್ಮಿಕ, ಮಾನಸಿಕ, ಮತ್ತು ದೈಹಿಕ ಶಕ್ತಿಗಳು ಮತ್ತು ಹೊರಗೆ ಇರುವ ಪ್ರಪಂಚ ಬೇರೆ ಬೇರೆ ಎಂದಲ್ಲ. ಇವೆಲ್ಲ ಒಂದೆ. ಆದರೆ ಬೇರೆಬೇರೆ ದೃಷ್ಟಿಯಿಂದ ನೋಡಿದುದು. ನೀವು ಒಂದು ದೇಹವೆಂದು ಭಾವಿಸಿದರೆ ನಿಮಗೆ ಬರಿಯ ದೇಹವಿರುವುದು; ಮನಸ್ಸು ಗೋಚರವೇ ಆಗುವುದಿಲ್ಲ. ನೀವು ಒಂದು ಮನಸ್ಸು ಎಂದು ಭಾವಿಸಿದರೆ ನಿಮಗೆ ದೇಹ ಇರುವುದೇ ಗೊತ್ತಾಗುವುದಿಲ್ಲ. ನೀವು ಒಂದೇ ಆಗಿದ್ದೀರಿ. ನೀವು ಅದನ್ನು ದ್ರವ್ಯ ಅಥವಾ ದೇಹದಂತೆ ಬೇಕಾದರೆ ನೋಡಬಹುದು, ಮನಸ್ಸಿನಂತೆ ಬೇಕಾದರೆ ನೋಡಬಹುದು, ಆತ್ಮದಂತೆ ಬೇಕಾದರೆ ನೋಡಬಹುದು. ಜನನ ಮರಣಗಳೆಂಬುದು ಹಿಂದಿನಿಂದ ಬಂದ ಮೂಢನಂಬಿಕೆ. ಯಾರೂ ಎಂದೂ ಹುಟ್ಟಿರಲಿಲ್ಲ. ಯಾರೂ ಎಂದೂ ಸಾಯುವುದೂ ಇಲ್ಲ. ಒಬ್ಬ ತನ್ನ ಸ್ಥಿತಿಯನ್ನು ಮಾತ್ರ ಬದಲಾಯಿಸುವನು ಅಷ್ಟೆ. ಪಾಶ್ಚಾತ್ಯರು ಸಾವಿಗೆ ಅಷ್ಟೊಂದು ಅಂಜುವರು. ಹೇಗೊ ಸ್ವಲ್ಪ ಕಾಲ ಬದುಕಬೇಕೆಂಬುದನ್ನು ನೋಡಿದಾಗ ನನಗೆ ವ್ಯಥೆಯಾಗುವುದು. “ನಮಗೆ ಮರಣಾನಂತರ ಜೀವನ ಕೊಡಿ'' ಎಂದು ಪ್ರಾರ್ಥಿಸುವರು. ಮರಣಾನಂತರ ನೀವು ಬದುಕುತ್ತೀರಿ ಎಂದು ಯಾರಾದರೂ ಹೇಳಿದರೆ ಅವರಿಗೆ ಪರಮ ಸಂತೋಷ. ಇದನ್ನು ನಾನು ಹೇಗೆ ಅನುಮಾನಿಸುವುದಕ್ಕೆ ಸಾಧ್ಯ? ನಾನು ಸತ್ತೆ ಎಂದು ಊಹಿಸುವುದು ತಾನೆ ಹೇಗೆ ಸಾಧ್ಯ? ನೀವು ಸರಿ ಎಂದು ಅಲೋಚಿಸಿ ನೋಡಿ. ನಿಮ್ಮ ಹೆಣವನ್ನು ನೀವೇ ನೋಡುತ್ತಿರುವಿರಿ. ಜೀವನ ಎಂಬುದು ಅದ್ಭುತ ಸತ್ಯ. ನೀವು ಅದನ್ನು ಒಂದು ಕ್ಷಣವೂ ಮರೆಯಲಾರಿರಿ. ನೀವಿರುವುದೇ ಸಂದೇಹವೆಂದು ಆಲೋಚಿಸಬಹುದು. `ನಾನಿರುವೆ' ಎಂಬುದೇ ಪ್ರಜ್ಞೆಯ ಪ್ರಥಮ ಚಿಹ್ನೆ. ಶೂನ್ಯ ಸ್ಥಿತಿಯನ್ನು ಯಾರು ಆಲೋಚಿಸಲು ಸಾಧ್ಯ? ಇದು ಸ್ವತಃ ಪ್ರಮಾಣವಾದುದು. ಅಮೃತತ್ವವೆಂಬ ಭಾವನೆ ಮನುಷ್ಯನಲ್ಲಿ ಅಂತಸ್ಥವಾಗಿದೆ. ಕಲ್ಪನೆಗೆ ಅಸಾಧ್ಯವಾದುದನ್ನು ಚರ್ಚಿಸುವುದು ಹೇಗೆ? ಸ್ವತಃ ಪ್ರಮಾಣವಾಗಿರುವುದರ ಪಕ್ಷ ಪ್ರತಿಪಕ್ಷಗಳ ಕುರಿತು ಚರ್ಚಿಸುವುದೇಕೆ?

ನೀವು ಯಾವ ದೃಷ್ಟಿಯಿಂದ ನೋಡಿದರೂ ವಿಶ್ವವೆಲ್ಲ ಒಂದು ಅಖಂಡ. ಈಗ ನಮಗೆ ಈ ಪ್ರಪಂಚ, ಪ್ರಾಣ ಮತ್ತು ಆಕಾಶ, ಅಥವಾ ದ್ರವ್ಯ ಮತ್ತು ಶಕ್ತಿಗಳಿಂದ ಕೂಡಿರುವಂತೆ ಕಾಣುತ್ತಿದೆ. ಎಲ್ಲ ಮೂಲಸಿದ್ದಾಂತಗಳಂತೆ ಇದೂ ಕೂಡ ಸ್ವವಿರೋಧವಾಗಿರುವುದು. ಶಕ್ತಿ ಎಂದರೆ ಏನು? ಯಾವುದು ದ್ರವ್ಯವನ್ನು ಚಲಿಸುವಂತೆ ಮಾಡುವುದೋ ಅದು. ದ್ರವ್ಯವೆಂದರೇನು? ಶಕ್ತಿಯಿಂದ ಯಾವುದು ಚಲಿಸುವುದೋ ಅದು. ಇದು ಅನ್ಯೋನ್ಯ ಆಶ್ರಯ. ನಾವು ನಮ್ಮ ವಿಜ್ಞಾನದ ಮತ್ತು ಪಾಂಡಿತ್ಯದ ಬಗ್ಗೆ ಎಷ್ಟು ಹೆಮ್ಮೆ ಕೊಚ್ಚಿಕೊಂಡರೂ ನಮ್ಮ ಕೆಲವು ಯುಕ್ತಿಯ ಮೂಲಸಿದ್ಧಾಂತಗಳು ವಿಚಿತ್ರವಾಗಿವೆ. ಸಂಸ್ಕೃತ ಗಾದೆ ಹೇಳುವಂತೆ “ತಲೆ ಇಲ್ಲದೆ ತಲೆ ನೋವು”. ಇದನ್ನು ಮಾಯೆ ಎನ್ನುವುದು. ಇಲ್ಲಿ ಅಸ್ತಿತ್ವವೂ ಇಲ್ಲ. ನಾಸ್ತಿತ್ವವೂ ಇಲ್ಲ. ನೀವು ಇದನ್ನು ಇದೆ ಎನ್ನಲಾಗುವುದಿಲ್ಲ. ಏಕೆಂದರೆ ಕಾಲದೇಶಗಳಾಚೆ ಯಾವುದು ಸ್ವತಃ ಸಿದ್ಧವಾಗಿ ಇರುವುದೊ ಅದು ಮಾತ್ರ ಇರುವುದು. ಆದರೂ ನಮ್ಮ ಅಸ್ತಿತ್ವದ ಭಾವನೆಯನ್ನು ಸ್ವಲ್ಪ ತೃಪ್ತಿ ಪಡಿಸುವುದರಿಂದ ಇದು ಇದೆ. ಆದಕಾರಣ ಇದು ತೋರಿಕೆಗೆ ಇದೆ.

ಪ್ರತಿಯೊಂದು ವಸ್ತುವಿನ ಅಂತರಾಳದಲ್ಲಿಯೂ ನಿಜವಾದ ಅಸ್ತಿತ್ವವಿದೆ. ಆ ಸತ್ಯವು ದೇಶಕಾಲ ನಿಮಿತ್ತದ ಜಾಲದಲ್ಲಿ ಸಿಕ್ಕಿರುವಂತೆ ತೋರುತ್ತಿದೆ. ಅನಾದ್ಯಾನಂತನಾದ ವಿಶ್ವವ್ಯಾಪಿಯಾದ ನಿತ್ಯಮುಕ್ತನಾದ ಧನ್ಯಾತ್ಮನಾದ ನಿಜವಾದ ಮಾನವನಿರುವನು. ಅವನು ದೇಶ ಕಾಲ ನಿಮಿತ್ತಗಳ ಜಾಲದಲ್ಲಿ ಸಿಲುಕಿರುವನು. ಇದರಂತೆಯೇ ಪ್ರಪಂಚದಲ್ಲಿ ಪ್ರತಿಯೊಂದು ವಸ್ತುವೂ ಕೂಡ. ಪ್ರತಿಯೊಂದು ವಸ್ತುವಿನ ಸತ್ಯವೇ ಅಖಂಡವಾಗಿರುವ ಅನಂತ. ಇದು ಕೇವಲ ಭಾವಸತ್ತಾವಾದವಲ್ಲ (\enginline{Idealism}). ಪ್ರಪಂಚ ಇಲ್ಲವೆಂದಲ್ಲ. ಇದು ಸಾಪೇಕ್ಷಕವಾಗಿದೆ. ಅದಕ್ಕೆ ಬೇಕಾದ ವಿಶೇಷಣಗಳನ್ನೆಲ್ಲ ಇದು ತೃಪ್ತಿಪಡಿಸುವುದು. ಆದರೆ ಇದು ತನಗೆ ತಾನೇ ಇರಲಾರದು. ಕಾಲ ದೇಶ ನಿಮಿತ್ತಗಳ ಆಚೆ ಇರುವ ಸತ್ಯವಿರುವುದರಿಂದ ಇದು ಇದೆ.

ನಾನು ಮುಖ್ಯ ವಿಷಯದಿಂದ ಬಹಳ ಸರಿದು ಹೋಗಿರುವೆ. ಈಗ ಅದನ್ನು ತೆಗೆದುಕೊಳ್ಳೋಣ.

ಪ್ರಾಣವೇ ಐಚ್ಛಿಕ ಮತ್ತು ಅನೈಚ್ಛಿಕ ಕ್ರಿಯೆಗಳನ್ನು ನರಗಳ ಮೂಲಕ ಮಾಡಿಸುತ್ತಿದೆ. ಆದಕಾರಣ ನಮ್ಮ ಅನೈಚ್ಛಿಕ ಕ್ರಿಯೆಗಳ ಮೇಲೆ ಹತೋಟಿಯನ್ನು ಪಡೆಯುವುದು. ಒಳ್ಳೆಯದಾಗಿ ಕಾಣುತ್ತದೆ.

ಹಿಂದೆ ಒಂದು ಸಲ ನಿಮಗೆ ಮನುಷ್ಯ ಮತ್ತು ದೇವರು ಎಂಬುದಕ್ಕೆ ವಿವರಣೆ ಕೊಟ್ಟೆ. ಮನುಷ್ಯ ಅನಂತ ವೃತ್ತದಂತೆ. ಅದರ ಪರಿಧಿ ಎಲ್ಲಿಯೂ ಇಲ್ಲ. ಕೇಂದ್ರ ಮಾತ್ರ ಒಂದು ಕಡೆ ಇದೆ. ದೇವರು ಅನಂತ ವೃತ್ತದಂತೆ, ಅವನ ಪರಿಧಿ ಎಲ್ಲಿಯೂ ಇಲ್ಲ, ಕೇಂದ್ರ ಎಲ್ಲಾ ಕಡೆಯಲ್ಲಿಯೂ ಇದೆ. ಅವನು ಎಲ್ಲಾ ಕೈಗಳ ಮೂಲಕ ಕೆಲಸ ಮಾಡುವನು. ಎಲ್ಲಾ ಕಣ್ಣುಗಳ ಮೂಲಕ ನೋಡುವನು. ಎಲ್ಲರ ಕಾಲುಗಳ ಮೂಲಕವೂ ನಡೆಯುವನು. ಎಲ್ಲ ದೇಹಗಳ ಮೂಲಕ ಉಸಿರಾಡುವನು. ಎಲ್ಲಾ ಜೀವಿಗಳಲ್ಲಿಯೂ ನೆಲಸಿರುವನು. ಎಲ್ಲ ಬಾಯಿಗಳ ಮೂಲಕ ಮಾತನಾಡುವನು. ಪ್ರತಿಯೊಂದು ಮೆದುಳಿನ ಮೂಲಕ ಆಲೋಚಿಸುವನು. ಮನುಷ್ಯ ತನ್ನ ಪ್ರಜ್ಞೆಯ ಕೇಂದ್ರವನ್ನು ಅನಂತವಾಗಿ ವಿಕಾಸಗೊಳಿಸಿದರೆ ಅವನು ದೇವನಂತೆ ಆಗಬಲ್ಲ. ಎಲ್ಲವನ್ನೂ ತನ್ನ ಸ್ವಾಧೀನಕ್ಕೆ ತರಬಲ್ಲ. ನಾವು ತಿಳಿದುಕೊಳ್ಳಬೇಕಾದ ಮುಖ್ಯ ವಿಷಯವೇ ಪ್ರಜ್ಞೆ. ಗಾಢಾಂಧಕಾರದಲ್ಲಿ ಒಂದು ಅನಂತವಾದ ಸರಳರೇಖೆ ಇದೆ ಎಂದು ಭಾವಿಸೋಣ; ನಮಗೆ ರೇಖೆ ಕಾಣುವುದಿಲ್ಲ. ಆದರೆ ಅದರ ಮೇಲೆ ಜ್ಯೋತಿರ್ಮಯ ವಸ್ತುವೊಂದು ಚಲಿಸುತ್ತಿರುವುದು. ಅದು ರೇಖೆಯ ಮೇಲೆ ಹೋಗುತ್ತಿರುವಾಗ ದಾರಿಯಲ್ಲಿ ಸಿಕ್ಕುವ ಸ್ಥಳವನ್ನೆಲ್ಲಾ ಬೆಳಕಿನಲ್ಲಿ ತೋರುವುದು. ಹಿಂದೆ ಉಳಿದುದೆಲ್ಲ ಪುನಃ ಗಾಢಾಂಧಕಾರದಲ್ಲಿ ಮರೆಯಾಗುವುದು. ನಮ್ಮ ಪ್ರಜ್ಞೆಯನ್ನು ಈ ಜ್ಯೋತಿಯ ವಸ್ತುವಿಗೆ ಹೋಲಿಸಬಹುದು. ಅದರ ಹಿಂದಿನ ಅನುಭವ ಈಗಿನ ಅನುಭವಕ್ಕೆ ಎಡೆಮಾಡಿದೆ ಅಥವಾ ಅಪ್ರಜ್ಞಾವಸ್ಥೆಗೆ ಹೋಗಿದೆ. ಅವು ನಮ್ಮಲ್ಲಿರುವುದರ ಅರಿವು ನಮಗಿಲ್ಲ. ಆದರೆ ಅವು ಅಲ್ಲಿರುವುವು. ನಮ್ಮ ಅಪ್ರಜ್ಞಾಪೂರ್ವಕವಾಗಿ ದೇಹ ಮನಸ್ಸುಗಳ ಮೇಲೆ ತನ್ನ ಪ್ರಭಾವವನ್ನು ಬೀರುತ್ತಿರುವುವು. ಈಗ ಅರಿವಿಲ್ಲದೆ ಮಾಡಿದ ಪ್ರತಿಯೊಂದು ಚಲನೆಯೂ ಹಿಂದೆ ಅರಿವಿನಿಂದ ಕೂಡಿತ್ತು. ಅದು ತನಗೆ ತಾನೇ ಕೆಲಸಮಾಡಿಕೊಳ್ಳುವಷ್ಟು ಶಕ್ತಿಯನ್ನು ಪಡೆದಿದೆ.

\newpage

ವಿನಾಯಿತಿ ಇಲ್ಲದೆ ಎಲ್ಲಾ ಧಾರ್ಮಿಕ ಸಿದ್ದಾಂತಗಳಲ್ಲಿಯೂ ಇರುವ ಒಂದು ಕೊರತೆಯೆ, ಮನುಷ್ಯ ಹೇಗೆ ತಪ್ಪನ್ನು ಮಾಡದೆ ಇರುವುದು ಎಂಬುದನ್ನು ಬೋಧಿಸದೆ ಇರುವುದು. ಎಲ್ಲಾ ಧಾರ್ಮಿಕ ಸಿದ್ದಾಂತಗಳೂ “ಕದಿಯ ಬೇಡಿ" ಎಂದು ಹೇಳುವುವು. ಅದೇನೋ ಒಳ್ಳೆಯದೆ. ಆದರೆ ಮನುಷ್ಯ ಏತಕ್ಕೆ ಕದಿಯುವನು? ಎಲ್ಲಾ ಕಳ್ಳತನ, ದರೋಡೆ ಮತ್ತು ದುಷ್ಕೃತ್ಯಗಳು ಯಾಂತ್ರಿಕವಾಗಿವೆ, ನಮ್ಮ ಇಚ್ಛೆ ಇಲ್ಲದೆ ನಡೆಯುತ್ತಿವೆ. ದರೋಡೆಕಾರ, ಚೋರ, ಸುಳ್ಳು ಹೇಳುವವನು, ಅಧರ್ಮಿಗಳಾದ ಸ್ತ್ರೀ ಪುರುಷರೆಲ್ಲ ತಮ್ಮ ಇಚ್ಛೆ ಇಲ್ಲದೆ ಹಾಗೆ ಇರುವರು. ನಿಜವಾಗಿ ಇದೊಂದು ಅದ್ಭುತ ಮಾನಸಿಕ ಸಮಸ್ಯೆ. ಮಾನವನನ್ನು ಸಾಧ್ಯವಾದಷ್ಟು ಉದಾರ ದೃಷ್ಟಿಯಿಂದ ನೋಡಬೇಕು. ಒಳ್ಳೆಯವರಾಗಿರುವುದು ಅಷ್ಟು ಸುಲಭವಲ್ಲ. ನೀವು ಮುಕ್ತರಾಗುವ ತನಕ ಯಂತ್ರವಲ್ಲದೆ ಮತ್ತೇನು? ನೀವು ಒಳ್ಳೆಯವರೆಂದು ಅಹಂಕಾರ ಪಡಬೇಕೆ? ಎಂದಿಗೂ ಇಲ್ಲ. ನೀವು ಒಳ್ಳೆಯವನಾಗಿರುವಿರಿ, ಏಕೆಂದರೆ ಅನ್ಯಥಾ ಇರಲು ವಿಧಿಯೇ ಇಲ್ಲ. ಮತ್ತೊಬ್ಬ ಕೆಟ್ಟವನಾಗಿರುವನು; ಏಕೆಂದರೆ ಅವನಿಗೆ ಬೇರೆ ರೀತಿ ಇರಲು ಸಾಧ್ಯವೇ ಇಲ್ಲ. ನೀವು ಅವನ ಸ್ಥಿತಿಯಲ್ಲಿದ್ದರೆ ನೀವು ಹೇಗೆ ಇರುತ್ತಿದ್ದಿರೊ ಯಾರಿಗೆ ಗೊತ್ತು? ನಮ್ಮ ಸಮಾಜದಲ್ಲಿರುವ ಜಾರೆ, ಸೆರೆಮನೆಯಲ್ಲಿ ಕೊಳೆಯುವ ಚೋರ, ನೀವು ಒಳ್ಳೆಯವರಾಗಲಿ ಎಂದು ತ್ಯಾಗಮಾಡಿದ ಕ್ರಿಸ್ತನಂತಹ ಮಹಾ ಪುರುಷರು. ಅಧಮರಿದ್ದರೆ ಉತ್ತಮರೂ ಇರುವರು. ಇದೇ ಪ್ರಕೃತಿ ನಿಯಮ, ಚೋರರು ಕೊಲೆಪಾತಕಿಗಳು ಅಧರ್ಮಿಗಳು ದುರ್ಬಲರು ದುರಾತ್ಮರು ಪಾಪಾತ್ಮರು ಇವರೆಲ್ಲ ನನ್ನ ಕ್ರಿಸ್ತರು. ಸಾಧು ಕ್ರಿಸ್ತನಿಗೆ ಪೂಜೆ ಮಾಡುವಂತೆ ಚೋರ ಕ್ರಿಸ್ತನಿಗೂ ಪೂಜೆ ಮಾಡಬೇಕಾಗಿದೆ. ಇದೇ ನನ್ನ ಸಿದ್ದಾಂತ, ಬೇರೆ ವಿಧಿಯೇ ಇಲ್ಲ. ಸಾಧು ಸತ್ಪುರುಷರಿಗೆ ನನ್ನ ಪ್ರಣಾಮಗಳು. ಅವರಂತೆಯೇ ದುರಾತ್ಮರಿಗೂ ಪಾಪಿಗಳಿಗೂ ಪ್ರಣಾಮಗಳು. ಇವರೆಲ್ಲ ನನ್ನ ಗುರುಗಳು, ನನ್ನ ಆಧ್ಯಾತ್ಮಿಕ ಪಿತರು, ಇವರೆಲ್ಲರೂ ನನ್ನ ಉದ್ದಾರಕರು. ನಾನು ಒಬ್ಬನನ್ನು ದೂರಬಹುದು, ಆದರೂ ಅವನ ತಪ್ಪಿನಿಂದ ನೀತಿಯನ್ನು ಕಲಿಯಬಹುದು. ನಾನು ಮತ್ತೊಬ್ಬನನ್ನು ಆಶೀರ್ವದಿಸಿ ಅವನಿಂದ ನೀತಿಯನ್ನು ಕಲಿಯಬಹುದು. ಇದು ನಾನು ಈಗ ಇಲ್ಲಿ ನಿಂತಿರುವಷ್ಟೇ ಸತ್ಯ. ಜಾರೆಯನ್ನು ನೋಡಿ ನಾನು ಟೀಕಿಸಬೇಕಾಗಿದೆ. ಏಕೆಂದರೆ ಸಮಾಜ ಹೀಗೆ ಮಾಡುವಂತೆ ಬಲಾತ್ಕರಿಸುವುದು. ಆದರೆ ಆಕೆ ನನ್ನ ಉದ್ದಾರಕಳು, ಆಕೆಯ ಜಾರತನ ಇತರ ನಾರಿಯರ ಪಾತಿವ್ರತ್ಯಕ್ಕೆ ಕಾರಣವಾಗಿದೆ. ಇದನ್ನು ಆಲೋಚಿಸಿ ನೋಡಿ! ಸ್ತ್ರೀ ಪುರುಷರೆ, ಈ ಸಮಸ್ಯೆಯನ್ನು ಆಲೋಚಿಸಿ ನೋಡಿ. ಇದು ಸತ್ಯ. ಉಪಾಧಿರಹಿತ ಧೈರ್ಯವಾದ ಸತ್ಯ. ನನಗೆ ಹೆಚ್ಚು ಪ್ರಾಪಂಚಿಕ ಅನುಭವ ಆದಂತೆಲ್ಲ, ಹೆಚ್ಚು ಸ್ತ್ರೀಪುರುಷರನ್ನು ನೋಡಿದಂತೆಲ್ಲ, ಈ ನಂಬಿಕೆ ನನ್ನಲ್ಲಿ ಹೆಚ್ಚು ದೃಢವಾಗುತ್ತಿದೆ. ನಾನು ಯಾರನ್ನು ನಿಂದಿಸುವುದು, ಯಾರನ್ನು ಹೊಗಳುವುದು? ಸಮಸ್ಯೆಯ ಎರಡು ಪಕ್ಷವನ್ನೂ ನಾನು ತಿಳಿದುಕೊಳ್ಳಬೇಕಾಗಿದೆ.

ನಮ್ಮ ಎದುರಿಗೆ ಇರುವ ಕಾರ್ಯ ಮಹತ್ತರವಾದುದು. ನಮ್ಮ ಪ್ರಥಮ ಮುಖ್ಯ ಕರ್ತವ್ಯವೇ ನಮ್ಮಲ್ಲಿ ಅಪ್ರಜ್ಞಾಪೂರ್ವಕವಾಗಿರುವ ಆಲೋಚನಾ ರಾಶಿಯನ್ನು ನಾವು ನಿಗ್ರಹಿಸುವುದು. ಪಾಪಕೃತ್ಯವೇನೊ ನಿಸ್ಸಂಶಯವಾಗಿ ನಮ್ಮ ಪ್ರಜ್ಞೆಯ ಕ್ಷೇತ್ರದಲ್ಲಿದೆ. ಆದರೆ ಈ ಪಾಪಕೃತ್ಯವನ್ನು ಮಾಡುವಂತೆ ಪ್ರೇರೇಪಿಸಿದ ಕಾರಣ, ನಮಗೆ ಅತೀತವಾದ ನಮಗೆ ಗೋಚರವಿಲ್ಲದ ಅಪ್ರಜ್ಞೆಯ ಕ್ಷೇತ್ರದಲ್ಲಿದೆ, ಅದಕ್ಕೇ ಅಷ್ಟು ಪ್ರಬಲವಾಗಿದೆ.

ಅನುಷ್ಠಾನ ಮನಶ್ಶಾಸ್ತ್ರವು ಮನಸ್ಸಿನ ಅಪ್ರಜ್ಞೆಯ ಸ್ವರವನ್ನು ನಿಗ್ರಹಿಸಲು ತನ್ನ ಶಕ್ತಿಯನ್ನೆಲ್ಲ ಬಳಸುವುದು. ಇದನ್ನು ನಾವು ಮಾಡಬಹುದೆಂದು ನಮಗೆ ಗೊತ್ತಿದೆ. ಏತಕ್ಕೆ? ಏಕೆಂದರೆ ಅಪ್ರಜ್ಞಾವಸ್ಥೆಗೆ ಪ್ರಜ್ಞಾವಸ್ಥೆಯೇ ಕಾರಣ ಎಂಬುದು ಗೊತ್ತಿದೆ. ಅಪ್ರಜ್ಞಾವಸ್ಥೆಯ ನಮ್ಮ ಚಿಂತನೆಗಳು ಎಂದರೆ ಆ ಅವಸ್ಥೆಯಲ್ಲಿ ಮುಳುಗಿರುವ ನಿಶ್ಚೇಷ್ಟಿತಗೊಂಡ ನಮ್ಮ ಲಕ್ಷಾಂತರ ಪುರಾತನ ಪ್ರಜ್ಞಾಪೂರ್ವಕ ಕ್ರಿಯೆಗಳು ಮತ್ತು ಚಿಂತನೆಗಳು. ನಾವು ಅವುಗಳ ಕಡೆ ನೋಡುವುದಿಲ್ಲ, ನಮಗೆ ಅವು ಗೊತ್ತಿಲ್ಲ. ನಾವು ಅವನ್ನು ಮರೆತುಬಿಟ್ಟಿರುವೆವು. ಆದರೆ ಇದನ್ನು ನೆನಪಿನಲ್ಲಿಡಿ. ಅಪ್ರಜ್ಞಾವಸ್ಥೆಯಲ್ಲಿ ಪಾಪ ಶಕ್ತಿಯು ಅಡಗಿರುವಂತೆಯೇ ಪುಣ್ಯಶಕ್ತಿಯೂ ಅಡಗಿರುವುದು. ನಾವು ನಮ್ಮ ಜೇಬಿನಲ್ಲಿ ಇಟ್ಟಿರುವಂತೆ ಅನೇಕ ವಿಷಯಗಳನ್ನು ಶೇಖರಿಸಿಟ್ಟುಕೊಂಡಿದ್ದೇವೆ. ನಮಗೆ ಅವು ಮರೆತುಹೋಗಿವೆ. ಅವನ್ನು ಕುರಿತು ಆಲೋಚಿಸುವುದು ಕೂಡ ಇಲ್ಲ. ಅವು ಕೊಳೆತು, ತುಂಬ ಅಪಾಯಕರವಾಗುವುವು. ಅವೇ ಮಾನವ ಕೋಟಿಯನ್ನು ಕೊಲ್ಲಲು ಧಾವಿಸುತ್ತಿರುವ ಅಪ್ರಜ್ಞಾತ್ಮಕ ಕ್ರಿಯೆಗಳು. ಆದಕಾರಣ ನಿಜವಾದ ಮನಶ್ಶಾಸ್ತ್ರವು ನಮ್ಮ ಪ್ರಜ್ಞೆಯ ಅಧೀನಕ್ಕೆ ಅಪ್ರಜ್ಞಾತ್ಮಕವಾದ ಆಲೋಚನೆಗಳನ್ನು ತರಲು ಯತ್ನಿಸುವುದು. ಮಾನವ ತನಗೆ ತಾನೆ ಸ್ವಾಮಿಯಾಗಬೇಕಾದರೆ ಪೂರ್ಣ ಮಾನವನನ್ನು ಉದ್ಧಾರಮಾಡಬೇಕಾಗಿದೆ. ಇದೇ ಆದ್ಯ ಕರ್ತವ್ಯ. ನಮ್ಮ ದೇಹದಲ್ಲಿ ಅನೈಚ್ಛಿಕವಾಗಿ ನಡೆಯುತ್ತಿರುವ ಯಕೃತ್ ಮುಂತಾದುವುಗಳ ಕ್ರಿಯೆಯನ್ನು ಕೂಡ ನಮ್ಮ ಸ್ವಾಧೀನಕ್ಕೆ ತರಬಹುದು.

ಅಪ್ರಜ್ಞಾವಸ್ಥೆಯನ್ನು ನಿಗ್ರಹಿಸುವುದೆ ನಮ್ಮ ಅಧ್ಯಯನದ ಪ್ರಥಮ ಭಾಗ. ಎರಡನೆಯದು ಪ್ರಜ್ಞಾವಸ್ಥೆಯನ್ನು ಮೀರಿಹೋಗುವುದು. ಅಪ್ರಜ್ಞಾವಸ್ಥೆಯು ಹೇಗೆ ನಮ್ಮ ಪ್ರಜ್ಞಾವಸ್ಥೆಯ ಕೆಳಗಿನ ಪದರದಲ್ಲಿದೆಯೋ ಹಾಗೆಯೇ ಪ್ರಜ್ಞಾವಸ್ಥೆಯ ಮೇಲೆ\break ಮತ್ತೊಂದು ಇದೆ. ಅದೇ ಪ್ರಜ್ಞೆಗೆ ಅತೀತವಾದುದು. ಮಾನವ ಈ ಪ್ರಜ್ಞಾತೀತ ಅವಸ್ಥೆಯನ್ನು ಸೇರಿದಾಗ ಮುಕ್ತನಾಗಿ ಪವಿತ್ರಾತ್ಮನಾಗುವನು. ಮರ್ತ್ಯ ಅಮೃತವಾಗುವುದು, ದುರ್ಬಲತೆ ಅನಂತ ಶಕ್ತಿಯಾಗರವಾಗುವುದು. ದಾಸ್ಯಶೃಂಖಲೆ ಕಳಚಿ ಮುಕ್ತವಾಗುವನು. ಇದೇ ಗುರಿ, ಇದೇ ಪ್ರಜ್ಞಾತೀತದ ಅಸೀಮ ಕ್ಷೇತ್ರ.

ಆದಕಾರಣ ಈಗ ಎರಡು ಬಗೆಯ ಸಾಧನೆಗಳು ಇರಬೇಕೆಂಬುದನ್ನು ನೋಡುವೆವು. ಮೊದಲನೆಯದೆ ಸ್ವಾಭಾವಿಕವಾಗಿರುವ ಇಡ ಪಿಂಗಳಗಳ ಶಕ್ತಿಯನ್ನು ಒಂದು ಕ್ರಮಕ್ಕೆ ತಂದು ಅದರಿಂದ ನಮ್ಮ ಅಪ್ರಜ್ಞಾಸ್ಥಿತಿಯ ಕ್ರಿಯೆಗಳನ್ನು ನಿಗ್ರಹಿಸುವುದು. ಎರಡನೆಯದೆ ಪ್ರಜ್ಞಾತೀತರಾಗುವುದು.

ದೀರ್ಘ ಏಕಾಗ್ರತೆಯ ಸಾಧನದಿಂದ ಯಾರು ಈ ಸತ್ಯವನ್ನು ಅರಿತಿರುವರೋ ಅವರೇ ಯೋಗಿಗಳೆಂದು ಶಾಸ್ತ್ರಗಳು ಹೇಳುವುವು. ಸುಷುಮ್ನ ಕಾಲುವೆ ಈಗ ತೆರೆಯುವುದು. ಇದುವರೆವಿಗೂ ಈ ಕಾಲುವೆಯೊಳಗೆ ಪ್ರವೇಶಿಸದ ಹೊಸ ಶಕ್ತಿ ಈಗ ಪ್ರವಹಿಸುವುದು. ಕ್ರಮೇಣ ಮೇಲಮೇಲಕ್ಕೆ ಏರುತ್ತ ಹಲವು ಚಕ್ರಗಳನ್ನು ದಾಟಿ ಮೆದುಳಿನಲ್ಲಿರುವ ಸಹಸ್ರಾರಕ್ಕೆ ಹೋಗುವುದು. ಆಗ ಯೋಗಿಗೆ ತನ್ನ ನೈಜಸ್ವಭಾವದ\break ಅರಿವಾಗುವುದು. ಅವನು ದೇವರೇ ಆಗಿರುವನು. ವಿನಾಯಿತಿ ಇಲ್ಲದೆ ನಮ್ಮಲ್ಲಿ ಪ್ರತಿಯೊಬ್ಬರೂ ಯೋಗದ ಈ ಗುರಿಯನ್ನು ಮುಟ್ಟಬಹುದು. ಆದರೆ ಇದು ಅದ್ಭುತವಾದ ಸಾಹಸ. ಒಬ್ಬ ಈ ಗುರಿಯನ್ನು ಮುಟ್ಟಬೇಕಾದರೆ ಉಪನ್ಯಾಸ ಕೇಳಿದರೆ ಮಾತ್ರ ಸಾಲದು. ಕೆಲವು ಸಲ ಪ್ರಾಣಾಯಾಮ ಮಾಡಿದರೆ ಮಾತ್ರ ಸಾಲದು. ಎಲ್ಲಾ ಸಾಧನೆಯಲ್ಲಿದೆ. ದೀಪವನ್ನು ಹಚ್ಚುವುದಕ್ಕೆ ಎಷ್ಟು ಹೊತ್ತು ಹಿಡಿಯುವುದು? ಒಂದು ಕ್ಷಣ. ಆದರೆ ಒಂದು ಮೇಣದ ಬತ್ತಿಯನ್ನು ತಯಾರಿಸುವುದಕ್ಕೆ ಎಷ್ಟೊಂದು ಹೊತ್ತು ಹಿಡಿಯುವುದು! ಊಟ ಮಾಡುವುದಕ್ಕೆ ಎಷ್ಟು ಹೊತ್ತು ಬೇಕು? ಅರ್ಧ ಗಂಟೆ ಹಿಡಿಯಬಹುದು. ಆದರೆ ಅದನ್ನು ತಯಾರು ಮಾಡಬೇಕಾದರೆ ಗಂಟೆಗಳು ಹಿಡಿಯುವುದು. ಕ್ಷಣದಲ್ಲಿ ನಮಗೆ ದೀಪವನ್ನು ಹಚ್ಚಲು ಆಸೆ. ಆದರೆ ಮೊದಲು ದೀಪವನ್ನು ಅಣಿ ಮಾಡುವುದು ಮುಖ್ಯ ಎಂಬುದನ್ನು ಮರೆಯುವೆವು.

ನಾವು ಗುರಿಯನ್ನು ಸೇರುವುದು ಇಷ್ಟು ಕಷ್ಟವಾದರೂ ನಮ್ಮ ಯಾವ ಅಲ್ಪ ಪ್ರಯತ್ನಗಳೂ ನಿಷ್ಪಲವಾಗುವುದಿಲ್ಲ. ಯಾವುದೂ ವ್ಯರ್ಥವಾಗುವುದಿಲ್ಲ ಎಂಬುದು ನಮಗೆ ಗೊತ್ತಿದೆ. ಗೀತೆಯಲ್ಲಿ ಅರ್ಜುನ ಶ‍್ರೀಕೃಷ್ಣನನ್ನು “ಯಾರು ಈ ಜನ್ಮದಲ್ಲಿ ಮುಕ್ತಿಯನ್ನು ಪಡೆಯಲಾರರೊ ಅಂತಹ ಯೋಗಿಗಳು ಬೇಸಗೆಯ ಮೋಡದಂತೆ ನಾಶವಾಗುವರೆ?'' ಎಂದು ಪ್ರಶ್ನಿಸುವನು. ಶ‍್ರೀಕೃಷ್ಣ “ಮಿತ್ರನೆ, ಪ್ರಪಂಚದಲ್ಲಿ ಯಾವುದೂ ನಾಶವಾಗುವುದಿಲ್ಲ. ಅವನ ಪ್ರಯತ್ನದ ಫಲವೆಲ್ಲ ಅವನಿಗೆ ಉಳಿಯುವುದು. ಈ ಜನ್ಮದಲ್ಲಿ ಅವನಿಗೆ ಮುಕ್ತಿ ದೊರಕದೆ ಇದ್ದರೆ ಇನ್ನೊಂದು ಜನ್ಮದಲ್ಲಿ ಮುಂದುವರಿಯುವನು” ಎನ್ನುವನು. ಇಲ್ಲದೆ ಇದ್ದರೆ ಬುದ್ಧ, ಕ್ರಿಸ್ತ, ಶಂಕರರ ಅದ್ಭುತ ಬಾಲ್ಯವನ್ನು ಹೇಗೆ ವಿವರಿಸುತ್ತೀರಿ?

ಆಸನ ಪ್ರಾಣಾಯಾಮ ಮುಂತಾದುವು ಯೋಗ ಜೀವನಕ್ಕೆ ಸಹಕಾರಿಗಳೇನೊ ನಿಜ. ಆದರೆ ಇವೆಲ್ಲ ಕೇವಲ ಶಾರೀರಿಕ ಅಷ್ಟೆ. ಮುಖ್ಯವಾದ ಸಾಧನೆಯೆಲ್ಲ ಮಾನಸಿಕವಾದುದು. ಮೊದಲು ಆವಶ್ಯಕವಾಗಿ ಬೇಕಾಗಿರುವುದು ಶಾಂತವಾದ ನಿಶ್ಚಿಂತ ಜೀವನ. ನೀವು ಯೋಗಿಯಾಗಬೇಕಾದರೆ ಮೊದಲು ಸ್ವತಂತ್ರರಾಗಿ, ಯಾವ ಕಾತರತೆಯೂ ಇಲ್ಲದೆ ನೀವೊಬ್ಬರೇ ಇರುವ ವಾತಾವರಣದಲ್ಲಿರಬೇಕು. ಯಾರು ಸುಖಭೋಗಗಳ ಇಚ್ಛೆಯ ಜೊತೆಗೆ ಆತ್ಮಸಾಕ್ಷಾತ್ಕಾರವನ್ನೂ ಪಡೆಯಬೇಕೆಂದು ಬಯಸುವರೊ ಅವರು ಒಂದು ನದಿಯನ್ನು ದಾಟಲು, ಮರದ ತುಂಡೆಂದು ಭ್ರಮಿಸಿ ಮೊಸಳೆಯನ್ನು ಹಿಡಿದುಕೊಂಡು ಹೋಗುವವರಂತೆ ಹುಚ್ಚರು. “ಮೊದಲು ಭಗವಂತನ ಸಾಮ್ರಾಜ್ಯವನ್ನು ಅರಸಿ, ಅನಂತರ ಉಳಿದುವೆಲ್ಲ ನಿಮಗೆ ಸೇರಿ ಬರುವುದು.'' ಇದೇ ನಮ್ಮ ಮುಖ್ಯ ಕರ್ತವ್ಯ. ಇದನ್ನೇ ತ್ಯಾಗವೆನ್ನುವುದು. ಒಂದು ಆದರ್ಶಕ್ಕಾಗಿ ಬಾಳಿ, ಉಳಿದಾವುದಕ್ಕೂ ಅಲ್ಲಿ ಅವಕಾಶವನ್ನು ನೀಡಬೇಡಿ. ಯಾವಾಗಲೂ ನಮ್ಮ ಸಹಾಯಕ್ಕೆ ಬರುವ ಆಧ್ಯಾತ್ಮಿಕ ಪೂರ್ಣತೆಯನ್ನು ಪಡೆಯಲು ನಮ್ಮಲ್ಲಿರುವ ಶಕ್ತಿಯನ್ನೆಲ್ಲ ಉಪಯೋಗಿಸೋಣ. ಆತ್ಮಸಾಕ್ಷಾತ್ಕಾರಕ್ಕೆ ನಿಜವಾಗಿ ನಾವು ಹಂಬಲಿಸಿದರೆ ನಾವು ಹೋರಾಡಬೇಕು. ಹೋರಾಟದಿಂದ ನಮ್ಮ ಜೀವನ ವಿಕಾಸವಾಗುವುದು. ನಾವು ತಪ್ಪನ್ನು ಮಾಡಬಹುದು. ಆದರೆ ಅದರಿಂದ ದೊಡ್ಡ ನೀತಿಯನ್ನು ಕಲಿಯಬಹುದು.

ಆಧ್ಯಾತ್ಮಿಕ ಜೀವನಕ್ಕೆ ಮುಖ್ಯ ಸಹಾಯವೆ ಧ್ಯಾನ. ಧ್ಯಾನದಲ್ಲಿ ನಮ್ಮ ಭೌತಿಕ ಉಪಾಧಿಗಳನ್ನೆಲ್ಲಾ ಮರೆತು ನಾವು ಪವಿತ್ರಾತ್ಮರೆಂದು ಭಾವಿಸುವೆವು. ಧ್ಯಾನಮಾಡುವಾಗ ನಾವು ಯಾವ ಬಾಹ್ಯ ಸಹಾಯದ ಮೇಲೂ ನಿಲ್ಲುವುದಿಲ್ಲ. ಆತ್ಮ ಸ್ಪರ್ಶವಾದರೆ ಎಂತಹ ಹೀನ ವಿಷಯಕ್ಕೂ ಅದು ಒಳ್ಳೆಯ ಮೆರಗು ಕೊಡುವುದು. ದುರಾತ್ಮನು ಪುಣ್ಯಾತ್ಮನಾಗುವನು. ಎಲ್ಲಾ ದ್ವೇಷ, ಸ್ವಾರ್ಥ ಮಾಯವಾಗುವುದು. ನಾವು ಎಷ್ಟು ಕಡಮೆ ಈ ದೇಹವನ್ನು ಕುರಿತು ಚಿಂತಿಸಿದರೆ ಅಷ್ಟು ಮೇಲು. ಏಕೆಂದರೆ ನಮ್ಮ ದೇಹಾಸಕ್ತಿಯೇ ನಮ್ಮನ್ನು ಅಧೋಗತಿಗೆ ಎಳೆಯುವುದು. ನಾವು ದೇಹ ಎಂಬ\break ಭಾವನೆಯೇ ಎಲ್ಲಾ ದುಃಖಕ್ಕೂ ಮೂಲ. ಇದೇ ರಹಸ್ಯ. ನಾನು ಆತ್ಮ, ದೇಹವಲ್ಲ ಎಂದು ಭಾವಿಸಬೇಕು. ಒಳ್ಳೆಯದು ಕೆಟ್ಟದ್ದು ಮುಂತಾದ ಹಲವು ಸಂಬಂಧಗಳಿಂದ ಕೂಡಿದ ವಿಶ್ವವೆಲ್ಲ ಒಂದು ದೊಡ್ಡ ಪಟದ ಮೇಲೆ ಬರೆದ ಚಿತ್ರದಂತೆ ಇದೆ. ಅದರ ಸಾಕ್ಷಿಯೇ ನಾನು ಎಂದು ಯೋಚಿಸಬೇಕು.

