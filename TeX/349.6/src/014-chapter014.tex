
\chapter[ಸಾಂಖ್ಯ ಮತ್ತು ವೇದಾಂತ]{ಸಾಂಖ್ಯ ಮತ್ತು ವೇದಾಂತ\protect\footnote{\engfoot{C.W, Vol. II, P. 454}}}

ಇದುವರೆಗೂ ನಾನು ಹೇಳಿದ ಸಾಂಖ್ಯಸಿದ್ದಾಂತದ ತಾತ್ಪರ್ಯವನ್ನು ಕೊಡುತ್ತೇನೆ. ಸಾಂಖ್ಯತತ್ತ್ವದ ಲೋಪದೋಷಗಳೇನು? ಎಲ್ಲಿ ವೇದಾಂತವು ಸಾಂಖ್ಯದೊಂದಿಗೆ\break ಹೋಲಿಸಿದಾಗ ಅದಕ್ಕೆ ಪೂರಕವೆಂದು ತೋರುತ್ತದೆ ಎಂಬುದನ್ನು ಈ ಉಪನ್ಯಾಸದಲ್ಲಿ ತಿಳಿದುಕೊಳ್ಳಲು ಯತ್ನಿಸೋಣ. ಸಾಂಖ್ಯ ತತ್ತ್ವದ ಪ್ರಕಾರ ಪ್ರಕೃತಿಯೇ ಆಲೋಚನೆ ಬುದ್ದಿ ಯುಕ್ತಿ ಪ್ರೀತಿ ದ್ವೇಷ ಸ್ಪರ್ಶ ರುಚಿ ದ್ರವ್ಯ ಮುಂತಾದ ಅಭಿವ್ಯಕ್ತಿಗಳಿಗೆಲ್ಲಾ ಕಾರಣ. ಪ್ರತಿಯೊಂದೂ ಪ್ರಕೃತಿಯಿಂದಲೇ ಆಗಿದೆ. ಈ ಪ್ರಕೃತಿ ಸತ್ತ್ವ ರಜಸ್ಸು ತಮಸ್ಸು ಎಂಬ ಮೂರು ದ್ರವ್ಯಗಳಿಂದ ಆಗಿದೆ. ಇವು ಗುಣಗಳ ದ್ರವ್ಯಗಳು. ಇದರಿಂದಲೇ ಇಡೀ ಸೃಷ್ಟಿ ಉದ್ಭವಿಸಿರುವುದು. ಕಲ್ಪದ ಆದಿಯಲ್ಲಿ ಇವು ಸಮತ್ವದಲ್ಲಿರುತ್ತವೆ; ಸೃಷ್ಟಿ ಪ್ರಾರಂಭವಾದಾಗ ಇವು ಹಲವು ಬಗೆಯಲ್ಲಿ ಮಿಶ್ರವಾಗಿ ವಿಶ್ವದಂತೆ ಕಾಣುವುವು. ಸಾಂಖ್ಯರ ಪ್ರಕಾರ ಪ್ರಥಮ ಅಭಿವ್ಯಕ್ತಿಯೆ ಮಹತ್. ಅದರಿಂದ ಪ್ರಜ್ಞೆ ಹುಟ್ಟುವುದು. ಸಾಂಖ್ಯರ ದೃಷ್ಟಿಯಲ್ಲಿ ಇದೊಂದು ತತ್ತ್ವ. ಇದರಿಂದ ಮನಸ್ಸು ಪಂಚೇಂದ್ರಿಯಗಳು ತನ್ಮಾತ್ರಗಳು ಎಲ್ಲಿ ಹುಟ್ಟುವುವು. ಎಲ್ಲ ಸೂಕ್ಷ್ಮಕಣಗಳು ಪ್ರಜ್ಞೆಯಿಂದ ಬರುವುವು. ಈ ಸೂಕ್ಷ್ಮದಿಂದ ಸ್ಥೂಲ ವಸ್ತುಗಳಾಗುವುವು. ನಾವು ತನ್ಮಾತ್ರಗಳನ್ನು ನೋಡಲಾರೆವು. ಸ್ಥೂಲ ವಸ್ತುಗಳಾದಾಗ ಅವು ನಮ್ಮ ಅನುಭವಕ್ಕೆ ವೇದ್ಯವಾಗುವುವು.

ಚಿತ್ತವು, ಬುದ್ಧಿ ಪ್ರಜ್ಞೆ ಮನಸ್ಸು ಈ ಮೂರು ಕೆಲಸಗಳನ್ನು ಮಾಡುವಾಗ ಪ್ರಾಣವೆಂಬ ಶಕ್ತಿಯನ್ನು ಸೃಷ್ಟಿಸುವುದು. ಪ್ರಾಣವೆಂದರೆ ಕೇವಲ ಉಸಿರಾಡುವುದು ಎಂದು ಮಾತ್ರ ಅರ್ಥವಲ್ಲ. ಪ್ರಾಣದ ಒಂದು ಕ್ರಿಯೆಯೆ ಉಸಿರಾಡುವುದು. ಪ್ರಾಣವೆಂದರೆ ಇಡೀ ದೇಹದ ರಕ್ಷಣೆ ಮತ್ತು ಚಲನೆಗೆ ಕಾರಣವಾದ ನರಗಳ ಶಕ್ತಿ. ಅದು ಆಲೋಚನೆಯಂತೆಯೂ ವ್ಯಕ್ತವಾಗುವುದು. ಪ್ರಾಣದ ಬಹು ಮುಖ್ಯವಾದ, ಎಲ್ಲರಿಗೂ ಕಾಣುವ ಕೆಲಸವೆ ಉಸಿರಾಡುವುದು. ಪ್ರಾಣ ಗಾಳಿಯ ಮೇಲೆ ತನ್ನ ಪ್ರಭಾವವನ್ನು ಬೀರುವುದೇ ಹೊರತು, ಗಾಳಿ ಪ್ರಾಣದ ಮೇಲೆ ತನ್ನ ಪ್ರಭಾವವನ್ನು ಬೀರುವುದಿಲ್ಲ. ಉಸಿರಾಟವನ್ನು ನಿಗ್ರಹಿಸುವುದೆ ಪ್ರಾಣಾಯಾಮ. ಪ್ರಾಣದ ಚಲನೆಯನ್ನು ನಮ್ಮ ಸ್ವಾಧೀನಕ್ಕೆ ತರುವುದಕ್ಕಾಗಿಯೇ ಪ್ರಾಣಾಯಾಮವನ್ನು ಅಭ್ಯಾಸ ಮಾಡುವುದು. ಗುರಿ ಕೇವಲ ಉಸಿರಾಟವನ್ನು ನಿಗ್ರಹಿಸುವುದಾಗಲಿ, ಅಥವಾ ಶ್ವಾಸಕೋಶವನ್ನು ಬಲವಾಗಿ ಮಾಡುವುದಾಗಲಿ ಅಲ್ಲ. ಇದು ಡೆಲ್ ಸಾರ್ಟೆಯ ವ್ಯಾಯಾಮವೇ ಹೊರತು ಪ್ರಾಣಾಯಾಮವಲ್ಲ. ಪ್ರಾಣಗಳು ಇಡೀ ದೇಹವನ್ನು ನಿಯಂತ್ರಿಸುವ ಶಕ್ತಿ. ಪುನಃ ಪ್ರಾಣಗಳು ದೇಹದಲ್ಲಿರುವ ಮನಸ್ಸು ಮತ್ತು ಅಂತಃಕರಣಗಳ ಪ್ರಭಾವಕ್ಕೆ ಒಳಪಟ್ಟಿವೆ. ಇಲ್ಲಿಯವರೆಗೂ ಸರಿಯಾಯಿತು. ಮನಶ್ಶಾಸ್ತ್ರವು ಸ್ಪಷ್ಟವಾಗಿದೆ, ಕರಾರುವಾಕ್ಕಾಗಿದೆ. ಆದರೂ ಇದು ಜಗತ್ತಿನ ಶ್ರೇಷ್ಠತಮ ಅತ್ಯಂತ ಪುರಾತನ ಯುಕ್ತಿಬದ್ದ ಆಲೋಚನೆ. ಎಲ್ಲಿಯಾದರೂ ಸ್ವಲ್ಪ ತತ್ತ್ವಶಾಸ್ತ್ರವಿದ್ದರೆ, ನಿಯಮಬದ್ದ ಆಲೋಚನೆ ಇದ್ದರೆ ಅವೆಲ್ಲ ಸ್ವಲ್ಪಮಟ್ಟಿಗಾದರೂ ಕಪಿಲನಿಗೆ ಋಣಿಗಳು. ಪೈಥಾಗರಸ್ ಇದನ್ನು ಭರತಖಂಡದಲ್ಲಿ ಕಲಿತು ಗ್ರೀಸ್‌ನಲ್ಲಿ ಬೋಧಿಸಿದನು. ಅನಂತರ ಪ್ಲೇಟೋಗೆ ಈ ಭಾವನೆ ಸ್ವಲ್ಪ ಹೊಳೆಯಿತು. ಇನ್ನೂ ಸ್ವಲ್ಪ ಕಾಲಾನಂತರ ನೋಸ್ಟಿಕ್ ಪಂಗಡದವರು ಇದನ್ನು ಅಲೆಗ್ಸಾಂಡ್ರಿಯಾಕ್ಕೆ ಒಯ್ದರು; ಅಲ್ಲಿಂದ ಇದು ಯೂರೋಪಿಗೆ ಬಂದಿತು. ಎಲ್ಲಿಯಾದರೂ ಮನಶ್ಶಾಸ್ತ್ರದ ಅಥವಾ ತತ್ತ್ವಶಾಸ್ತ್ರದ ವಿಕಾಸಕ್ಕೆ ಸ್ವಲ್ಪ ಪ್ರಯತ್ನ ನಡೆದಿದ್ದರೆ ಅಲ್ಲೆಲ್ಲ ಆ ಪ್ರಯತ್ನಕ್ಕೆ ಮೂಲಪಿತಾಮಹನೆ ಕಪಿಲಋಷಿ. ಇಲ್ಲಿಯವರೆಗೆ ಅವನ ಮನಶ್ಶಾಸ್ತ್ರ ಅದ್ಭುತವಾಗಿದೆ ಎಂಬುದನ್ನು ನೋಡಿದೆವು. ಆದರೆ ನಾವು ಮುಂದುವರಿದಂತೆ ಅವನ ಕೆಲವು ಅಭಿಪ್ರಾಯ\-ಗಳ ವಿಷಯದಲ್ಲಿ ಭಿನ್ನಾಭಿಪ್ರಾಯವನ್ನು ತಾಳಬೇಕಾಗಿದೆ. ಕಪಿಲನ ಸಿದ್ಧಾಂತಕ್ಕೆ\break ತಳಹದಿಯೆ ವಿಕಾಸವಾದವೆನ್ನುವುದು ನಮಗೆ ಗೋಚರಿಸುವುದು. ಒಂದು ವಸ್ತುವಿನಿಂದ ಮತ್ತೊಂದು ವಸ್ತು ವಿಕಾಸವಾಗುತ್ತದೆ ಎಂದು ಅವನು ಭಾವಿಸುವನು. ಏಕೆಂದರೆ ಪರಿಣಾಮವೆನ್ನುವುದಕ್ಕೆ ಅವನ ವಿವರಣೆಯೆ “ಮತ್ತೊಂದು ರೂಪದಲ್ಲಿ ವ್ಯಕ್ತವಾಗಿರುವ ಕಾರಣ,'' ಏಕೆಂದರೆ ಇಡೀ ವಿಶ್ವವು ನಮಗೆ ಕಾಣುವ ಮಟ್ಟಿಗೆ ವಿಕಾಸವಾಗುತ್ತಿದೆ, ಮುಂದುವರಿಯುತ್ತಿದೆ. ನಾವು ಜೇಡಿ ಮಣ್ಣನ್ನು ನೋಡುತ್ತೇವೆ. ಅದರ ಬೇರೊಂದು ಆಕಾರ ಮಡಕೆ ಎನ್ನುವೆವು. ಜೇಡಿ ಮಣ್ಣು ಕಾರಣ ಮಡಿಕೆ ಅದರ ಕಾರ್ಯ ಇದರಾಚೆ ನಮಗೆ ಕಾರ್ಯ ಕಾರಣದ ಭಾವನೆಯೆ ದೊರಕುವುದಿಲ್ಲ. ಈ ದೃಷ್ಟಿಯಿಂದ ಇಡೀ ಬ್ರಹ್ಮಾಂಡ ಪ್ರಕೃತಿಯಿಂದ ಉದ್ಭವಿಸಿದೆ. ಆದಕಾರಣ ವಿಶ್ವವು ತನ್ನ ಮೂಲಕಾರಣಕ್ಕಿಂತ ಮೂಲತಃ ಬೇರೆ ಆಗಲಾರದು. ಕಪಿಲನ ಪ್ರಕಾರ ಅವ್ಯಕ್ತ ಪ್ರಕೃತಿಯಿಂದ ಹಿಡಿದು ಬುದ್ದಿಯವರೆಗೆ ಯಾವುದೂ ಭೋಕ್ತೃವಲ್ಲ, ಜ್ಞಾತೃವಲ್ಲ. ಒಂದು ಜೇಡಿಮಣ್ಣಿನ ಮುದ್ದೆಯಂತೆ ಮನಸ್ಸು. ಮನಸ್ಸಿಗೆ ಜ್ಞಾನವಿಲ್ಲ. ಆದರೂ ಅದು ಆಲೋಚಿಸುವಂತೆ ತೋರುವುದು. ಆದಕಾರಣ ಅದರ ಹಿಂದೆ ಯಾವುದೊ ವಸ್ತು ಇರಬೇಕು, ಆ ವಸ್ತುವೇ ಮಹತ್ ಮತ್ತು ಅದರಿಂದಾದ ಹಲವು ವಸ್ತುಗಳ ಮೂಲಕ ಪ್ರಕಾಶಿಸುತ್ತಿದೆ. ಇದನ್ನೇ ಕಪಿಲ ಪುರುಷ ಎನ್ನುವುದು, ವೇದಾಂತಿಗಳು ಆತ್ಮವೆನ್ನುವುದು. ಕಪಿಲನ ಅಭಿಪ್ರಾಯದಲ್ಲಿ ಪುರುಷ ಕೇವಲ, ಮಿಶ್ರವಲ್ಲ. ಅವನು ಅದ್ರವ್ಯ, ಅವನೊಬ್ಬನೇ ಅದ್ರವ್ಯ, ಪ್ರಪಂಚದಲ್ಲಿರುವ ವಿಭಿನ್ನ ಆವಿರ್ಭಾವಗಳೆಲ್ಲಾ ದ್ರವ್ಯಗಳು. ನಾನೊಂದು ಕಪ್ಪು ಹಲಗೆಯನ್ನು ನೋಡುತ್ತೇನೆ. ಮೊದಲು ಬಾಹ್ಯ ಪಂಚೇಂದ್ರಿಯಗಳು ಆ ಸುದ್ದಿಯನ್ನು ಮೆದುಳಿನ ಆಯಾ ಕೇಂದ್ರಕ್ಕೆ ತರುವುವು. ಈ ಕೇಂದ್ರದಿಂದ ಮನಸ್ಸಿಗೆ ಹೋಗಿ ಅಲ್ಲಿ ಒಂದು ಪರಿಣಾಮವನ್ನು ಉಂಟುಮಾಡುವುವು, ಮನಸ್ಸು ಬುದ್ದಿಗೆ ಒಯ್ಯುವುದು. ಬುದ್ದಿ ಕೆಲಸ ಮಾಡಲಾರದು, ಕೆಲಸವು ಪುರುಷನಿಂದ ಬರಬೇಕು. ಉಳಿದವುಗಳೆಲ್ಲ ಅವನ ಪರಿಚಾರಕರಂತೆ, ಅವು ಸುದ್ದಿಯನ್ನು ತರುತ್ತವೆ. ಇವನು ಅಪ್ಪಣೆ ಮಾಡುವನು. ಪುರುಷನೆ ನಿಜವಾದ ಭೋಕ್ತೃ, ಜ್ಞಾತೃ. ಅವನೇ ಸತ್ಯ. ಸಿಂಹಾಸನದ ಮೇಲೆ ಮಂಡಿಸಿರುವವನೆ ಅವನು. ಅವನೇ ಆತ್ಮ, ಚೇತನಸ್ವರೂಪ. ಅವನು ಚೇತನಸ್ವರೂಪನಾಗಿರುವುದರಿಂದ ಅನಂತನಾಗಿರಬೇಕೆಂಬುದು ಸ್ವಭಾವತಃ ಸಿದ್ಧಿಸುವುದು. ಅವನಿಗೆ ಯಾವ ಮಿತಿಯೂ ಇರಲಾರದು. ಪ್ರತಿಯೊಬ್ಬ ಪುರುಷನೂ ಸರ್ವವ್ಯಾಪಿ. ನಮ್ಮಲ್ಲಿ ಪ್ರತಿಯೊಬ್ಬರೂ ಸರ್ವವ್ಯಾಪಿಗಳೆ. ಆದರೆ ನಾವು ಲಿಂಗ ಶರೀರದ ಮೂಲಕ ಎಂದರೆ ಸೂಕ್ಷ್ಮ ಶರೀರದ ಮೂಲಕ ಕೆಲಸ ಮಾಡಬಲ್ಲೆವು. ಈ ಲಿಂಗಶರೀರವು ಮನಸ್ಸು ಬುದ್ದಿ ಇಂದ್ರಿಯ ಮತ್ತು ಪ್ರಾಣಗಳಿಂದ ಆಗಿದೆ. ಇದನ್ನೆ ಕ್ರೈಸ್ತತಾತ್ವಿಕರು ಮಾನವನ ಆಧ್ಯಾತ್ಮಿಕ ತನು ಎನ್ನುವರು. ಈ ದೇಹವೇ ಮೋಕ್ಷ ಶಿಕ್ಷೆಗಳಿಗೆ ಅಥವಾ ಸ್ವರ್ಗ, ಜನ್ಮ ಪುನರ್ಜನ್ಮಗಳಿಗೆ ಗುರಿಯಾಗುವುದು. ಏಕೆಂದರೆ ಪ್ರಾರಂಭದಿಂದಲೂ ನಿಜವಾದ ಪುರುಷ ಬಂದು ಹೋಗಲಾರನು ಎಂಬುದನ್ನು ನೋಡಿದೆವು. ಚಲನೆ ಎಂದರೆ ಹೊಗುವುದು ಅಥವಾ ಬರುವುದು. ಯಾವುದು ಒಂದು ಸ್ಥಳದಿಂದ ಮತ್ತೊಂದು ಸ್ಥಳಕ್ಕೆ ಹೋಗುವುದೋ ಬರುವುದೋ ಅದು ಸರ್ವವ್ಯಾಪಿಯಾಗಿರಲಾರದು. ಆತ್ಮ ಅನಂತವಾದುದು, ಇದೊಂದೆ ಪ್ರಕೃತಿಯಿಂದ ಆಗದೆ ಇರುವುದು ಎಂಬುದನ್ನು ಕಪಿಲನ ಮನಶ್ಶಾಸ್ತ್ರದ ಸಿದ್ಧಾಂತದ ಮೂಲಕ ಅರಿತೆವು. ಪುರುಷನೊಬ್ಬನೆ ಪ್ರಕೃತಿಯ ಹೊರಗೆ ಇರುವವನು. ಆದರೆ ಅವನು ಪ್ರಕೃತಿಯಿಂದ ಬದ್ದನಾದಂತೆ ತೋರುತ್ತಿರುವನು. ಪ್ರಕೃತಿ ಅವನ ಸುತ್ತಲೂ ಇದೆ. ತಾನೇ ಪ್ರಕೃತಿಯೆಂದು ಆರೋಪಿಸಿಕೊಂಡಿರುವನು. ತಾನೇ ಲಿಂಗಶರೀರ, ತಾನೇ ಸ್ಥೂಲಶರೀರ ಎಂದು ಭಾವಿಸುವನು. ಆದಕಾರಣ ಅವನು ಸುಖದುಃಖಗಳನ್ನು ಅನುಭವಿಸುವನು. ಆದರೆ ನಿಜವಾಗಿ ಅವು ಅವನಿಗೆ ಸೇರಿಲ್ಲ. ಅವು ಕೇವಲ ಲಿಂಗಶರೀರಕ್ಕೆ ಮಾತ್ರ ಸೇರಿದವು.

ಯೋಗಿಗಳು ಧ್ಯಾನಾವಸ್ಥೆಯನ್ನು ಅತ್ಯಂತ ಶ್ರೇಷ್ಠ ಅವಸ್ಥೆ ಎನ್ನುವರು. ಆಗ ಮನಸ್ಸು ಸ್ತಬ್ಧವಾಗಿಯೂ ಇರುವುದಿಲ್ಲ, ಚಲಿಸುತ್ತಲೂ ಇರುವುದಿಲ್ಲ. ಆ ಸ್ಥಿತಿಯಲ್ಲಿ ನಾವು ಪುರುಷನಿಗೆ ಅತಿ ಸಮೀಪದಲ್ಲಿರುವೆವು, ಆತ್ಮನಿಗೆ ಸುಖ ದುಃಖಗಳಿಲ್ಲ; ಅದು ಎಲ್ಲದರ ಸಾಕ್ಷಿ; ಕರ್ಮಫಲಗಳನ್ನು ಸ್ವೀಕರಿಸುವುದಿಲ್ಲ. ಸೂರ್ಯನೇ ಪ್ರತಿಯೊಬ್ಬರ ಕಣ್ಣಿಗೂ ಹೊರಗಿನದು ಕಾಣುವಂತೆ ಮಾಡುವನು. ಆದರೂ ಕಣ್ಣಿನ ದೋಷದಿಂದ ಅವನು ಬಾಧಿತನಾಗುವುದಿಲ್ಲ. ಒಂದು ಸ್ಪಟಿಕದ ಹಿಂದೆ ಕೆಂಪು ಅಥವಾ ನೀಲಿ ಹೂವನ್ನು ಇಟ್ಟರೆ ಸ್ಪಟಿಕಮಣಿ ಆ ಬಣ್ಣದಂತೆ ಕಾಣುತ್ತಿದ್ದರೂ ಅದರಿಂದ ಬಾಧಿತವಾಗುವುದಿಲ್ಲ. ಇದರಂತೆಯೆ ಆತ್ಮ ಅಚಲವೂ ಅಲ್ಲ ಚಲವೂ ಅಲ್ಲ. ಅದು ಎರಡನ್ನೂ ಮೀರಿರುವುದು. ಆತ್ಮದ ಈ ಸ್ಥಿತಿಯನ್ನು ವಿವರಿಸುವ ಅತಿ ಹತ್ತಿರದ ಭಾವನೆಯೆ ಧ್ಯಾನಸ್ಥಿತಿ. ಇದೇ ಸಾಂಖ್ಯ ತತ್ತ್ವ, ಸಾಂಖ್ಯರು ಪ್ರಕೃತಿ ಆವಿರ್ಭವಿಸುವುದು ಆತ್ಮನಿಗಾಗಿ ಎನ್ನುವರು. ಎಲ್ಲಾ ಮಿಶ್ರಣಗಳೂ ಇತರರಿಗಾಗಿ, ನೀವು ಪ್ರಕೃತಿ ಎಂದು ಕರೆಯುವ ಈ ಮಿಶ್ರಣವೆಲ್ಲ, ಈ ನಿರಂತರ ಬದಲಾವಣೆಯೆಲ್ಲ ಆತ್ಮನ ಅನುಭವಕ್ಕಾಗಿ ಆಗುತ್ತಿದೆ. ಕ್ಷುದ್ರತಮ ಜೀವದಿಂದ ಹಿಡಿದು ಶ್ರೇಷ್ಠತಮ ಜೀವದವರೆಗೆ ಎಲ್ಲರು ಅನುಭವಪಡೆದು ಮುಕ್ತರಾಗಲಿ ಎಂಬುದಕ್ಕಾಗಿ ಇದು ಆಗುತ್ತಿದೆ. ಇದನ್ನು ಪಡೆದ ಮೇಲೆ ಆತ್ಮವು ತಾನೆಂದಿಗೂ ಪ್ರಕೃತಿಯಲ್ಲಿ ಇರಲಿಲ್ಲ, ಬೇರೆ ಆಗಿದ್ದೆ, ತಾನು ಅವಿನಾಶಿ, ತಾನು ಬಂದು ಹೋಗಲಾರದಂಥದು ಎಂಬುದನ್ನು ಅದು ಅರಿಯುವುದು. ಸ್ವರ್ಗಕ್ಕೆ ತೆರಳುವುದು. ಪುನರ್ಜನ್ಮಧಾರಣೆ ಮಾಡುವುದು ಇವೆಲ್ಲ ಪ್ರಕೃತಿಯಲ್ಲಿತ್ತು, ತನ್ನಲ್ಲಿ ಇರಲಿಲ್ಲ ಎಂಬುದನ್ನು ಅರಿಯುವುದು. ಆತ್ಮ ಹೀಗೆ ಮುಕ್ತವಾಗುವುದು. ಪ್ರಕೃತಿಯೆಲ್ಲ ಪುರುಷನ ಅನುಭವಕ್ಕಾಗಿ ಕೆಲಸ ಮಾಡುತ್ತಿದೆ. ಅವನು ಗುರಿಯನ್ನು ಸೇರುವುದಕ್ಕಾಗಿ ಈ ಅನುಭವಗಳನ್ನು ಪಡೆಯುತ್ತಿರುವನು, ಆ ಗುರಿಯೆ ಸ್ವಾತಂತ್ರ್ಯ. ಸಾಂಖ್ಯರ ದೃಷ್ಟಿಯಲ್ಲಿ ಹಲವು ಆತ್ಮಗಳಿವೆ. ಕಪಿಲರ ಮತ್ತೊಂದು ಸಿದ್ದಾಂತದ ಪ್ರಕಾರ ಜಗತ್ತನ್ನು ವಿವರಿಸುವುದಕ್ಕೆ ಪ್ರಕೃತಿಯೊಂದೆ ಸಾಕು, ದೇವರು ಅನಾವಶ್ಯಕ.

ಆತ್ಮವು ಸ್ವಭಾವತಃ ಸಚ್ಚಿದಾನಂದ ಸ್ವರೂಪ ಎಂದು ವೇದಾಂತಿಗಳು ಹೇಳುವರು. ಆದರೆ ಈ ಸತ್-ಚಿತ್-ಆನಂದಗಳು ಆತ್ಮನ ಗುಣಗಳಲ್ಲ. ಇವೆಲ್ಲಾ ಒಂದು, ಮೂರಲ್ಲ. ಇದೇ ಆತ್ಮನ ಸಾರ. ಬುದ್ದಿ ಪ್ರಕೃತಿಗೆ ಸೇರಿದ್ದು. ಏಕೆಂದರೆ ಅದು ಪ್ರಕೃತಿಯಿಂದಲೇ ಬಂದದ್ದು. ಇದು ಅದರಿಂದಲೇ ಬಂದದ್ದು - ಎಂಬ ಸಾಂಖ್ಯರ ವಾದವನ್ನು ವೇದಾಂತಿಗಳೂ ಒಪ್ಪಿಕೊಳ್ಳುವರು. ಯಾವುದನ್ನು ನಾವು ಬುದ್ದಿ ಎನ್ನುವೆವೊ ಅದೊಂದು ಮಿಶ್ರವೆಂದು ವೇದಾಂತಿಗಳೂ ಹೇಳುತ್ತಾರೆ. ಉದಾಹರಣೆಗೆ ನಮ್ಮ ಇಂದ್ರಿಯಗ್ರಹಣಗಳನ್ನು ಪರೀಕ್ಷೆ ಮಾಡೋಣ. ನಾನೊಂದು ಕಪ್ಪು ಹಲಗೆಯನ್ನು ನೋಡುತ್ತೇನೆ. ಈ ಭಾವನೆ ಹೇಗೆ ಬರುವುದು? ಜರ್ಮನ್ ತತ್ವಶಾಸ್ತ್ರಜ್ಞರು ಕಪ್ಪು ಹಲಗೆಯ ನಿಜವಾದ ಸ್ಥಿತಿ (\enginline{thing in itself}) ನಮಗೆ ಗೊತ್ತಿಲ್ಲ, ಅದು ಎಂದಿಗೂ ನಮಗೆ ಗೊತ್ತಾಗುವಂತೆ ಇಲ್ಲ ಎನ್ನುವರು. ಅದನ್ನು ಒಂದು `ಎ' ಎಂದು ಕರೆಯೋಣ. ಈ `ಎ' ನನ್ನ ಮನಸ್ಸಿನ ಮೇಲೆ ತನ್ನ ಪರಿಣಾಮವನ್ನು ಬಿಡುವುದು. ಆಗ ಮನಸ್ಸು ಪ್ರತಿಕ್ರಿಯೆಯನ್ನು ಉಂಟುಮಾಡುವುದು. ಮನಸ್ಸು ಒಂದು ಸರೋವರದಂತೆ. ನೀವೊಂದು ಕಲ್ಲನ್ನು ನೀರಿಗೆ ಎಸೆದರೆ ಪ್ರತಿಕ್ರಿಯಾತರಂಗವೊಂದು ಎದ್ದು ಇದರೆಡೆಗೆ ಬರುವುದು. ಈ ಅಲೆ ಕಲ್ಲಿನಂತೆ ಇಲ್ಲ. ಇದೊಂದು ಅಲೆ ಮಾತ್ರ. ಕಪ್ಪು ಹಲಗೆ ಮನಸ್ಸಿಗೆ ತಾಕುವಂತಹ ಒಂದು ಕಲ್ಲಿನಂತೆ, ಮನಸ್ಸು ಅದರ ಕಡೆಗೆ ಒಂದು ಅಲೆಯನ್ನು ನೂಕುತ್ತದೆ. ಈ ಅಲೆಯನ್ನು ನಾವು ಕಪ್ಪು ಹಲಗೆ ಎನ್ನುತ್ತೇವೆ. ನಾನು ನಿಮ್ಮನ್ನು ನೋಡುತ್ತೇನೆ. ನಿಮ್ಮ ನೈಜಸ್ಥಿತಿ ನನಗೆ ಗೊತ್ತಿಲ್ಲ, ಗೊತ್ತಾಗುವಂತೆಯೂ ಇಲ್ಲ. ನೀವು `ಎ', ನನ್ನ ಮನಸ್ಸಿನ ಮೇಲೆ ಒಂದು ಪರಿಣಾಮವನ್ನು ಬೀರುತ್ತೀರಿ. ಎಲ್ಲಿಂದ ಈ ಸುದ್ದಿ ಬರುತ್ತದೆಯೋ ಆ ಕಡೆ ಮನಸ್ಸು ಒಂದು ಪ್ರತಿಕ್ರಿಯೆಯ ಅಲೆಯನ್ನು ಕಳುಹಿಸುವುದು. ಅದನ್ನೆ ನಾನು ಇಂತಹ ಶ‍್ರೀಮಾನ್ ಅಥವಾ ಶ‍್ರೀಮತಿ ಎನ್ನುವುದು. ಗ್ರಹಣದಲ್ಲಿ ಎರಡು ಕ್ರಿಯೆಗಳಿವೆ. ಒಂದು ಹೊರಗಿನಿಂದ ಬರುವುದು, ಮತ್ತೊಂದು ಒಳಗಿನಿಂದ ಬರುವುದು. ಇವೆರಡೂ ಎಂದರೆ `ಎ' ಮತ್ತು ಮನಸ್ಸು ಇವೇ ನಮ್ಮ ಬಾಹ್ಯ ಜಗತ್ತು. ಜ್ಞಾನವೆಲ್ಲ ಪ್ರತಿಕ್ರಿಯೆಯಿಂದ ಜನಿಸುವುದು. ತಿಮಿಂಗಿಲದ ಬಾಲವನ್ನು ಹೊಡೆದ ಎಷ್ಟೋ ಕ್ಷಣದ ಮೇಲೆ ಅದಕ್ಕೆ ಆ ಅರಿವಾಗಿ ನೋವಾಗುವುದೆಂಬುದನ್ನು ಲೆಕ್ಕಾಚಾರ ಮಾಡಿ ನೋಡಿರುವರು. ಇದರಂತೆಯೆ ಆಂತರಿಕ ಗ್ರಹಣ ಕೂಡ. ನನ್ನಲ್ಲಿರುವ ನಿಜವಾದ ಆತ್ಮ ಕೂಡ ನನಗೆ ಗೊತ್ತಿಲ್ಲ. ಗೊತ್ತಾಗುವಂತೆಯೂ ಇಲ್ಲ. ಅದನ್ನು “ಬಿ” ಎಂದು ಕರೆಯೋಣ. ನಾನು ಇಂತಹವನು ಎಂದರೆ ಬಿ+ಮನಸ್ಸು. ಈ `ಬಿ' ಮನಸ್ಸಿಗೆ ತಾಕುವುದು. ಆದಕಾರಣ ನಮ್ಮ ಬಾಹ್ಯ ಜಗತ್ತು ಎ+ಮನಸ್ಸು, ಆಂತರಿಕ ಜಗತ್ತು ಬಿ+ಮನಸ್ಸು, ಎ ಮತ್ತು ಬಿ ಗಳು ಕ್ರಮವಾಗಿ ಹೊರಗಿನ ಮತ್ತು ಒಳಗಿನ ಜಗತ್ತಿನ ಹಿಂದೆ ಇರುವ ವಸ್ತುವಿನ ನೈಜಸ್ಥಿತಿಯ ಚಿಹ್ನೆಗಳು.

ವೇದಾಂತದ ಪ್ರಕಾರ ಮೂಲಭೂತವಾದ ಮೂರು ಭಾವನೆಗಳೆಂದರೆ ನಾನಿರುವೆ, ನನಗೆ ಗೊತ್ತಿದೆ, ನಾನು ಧನ್ಯ ಎಂಬುದು. ನನಗೆ ಯಾವ ಬಯಕೆಗಳೂ ಇಲ್ಲ. ನಾನು ಶಾಂತನು, ಅಚಲನು. ತಾತ್ಕಾಲಿಕವಾಗಿ ಬರುವ ಯಾವುದೂ ನನ್ನ ಸಮತ್ವಕ್ಕೆ ಭಂಗತರಲಾರದು ಎನ್ನುವುದೆ ನಮ್ಮ ಜೀವನದ ಮುಖ್ಯ ಭಾವನೆ, ನಮ್ಮ ಅಸ್ತಿತ್ವದ ತಳಹದಿ. ಅದು ಮಿತವಾದಾಗ ಅದು ವ್ಯಾವಹಾರಿಕ ಜಗತ್ತಿನಂತೆ, ವ್ಯಾವಹಾರಿಕ ಜ್ಞಾನದಂತೆ, ಪ್ರೇಮದಂತೆ, ತೋರುವುದು. ಪ್ರತಿಯೊಬ್ಬನೂ ಇರುವನು, ಪ್ರತಿಯೊಬ್ಬನೂ ತಿಳಿದುಕೊಳ್ಳಬೇಕಾಗಿದೆ, ಪ್ರತಿಯೊಬ್ಬನೂ ಪ್ರೀತಿಯನ್ನು ತೀವ್ರವಾಗಿ ಬಯಸುವನು. ಅವನಿಗೆ ಪ್ರೀತಿಸದೆ ವಿಧಿಯಿಲ್ಲ, ಅತಿ ಅಲ್ಪಾತ್ಮನಿಂದ ಪುಣ್ಯಾತ್ಮನವರೆಗೆ ಎಲ್ಲ ರೂ ಪ್ರೀತಿಸಬೇಕಾಗಿದೆ. “ಬಿ'' ಎನ್ನುವ ಅಂತರಂಗದಲ್ಲಿರುವ ನೈಜವಸ್ತು ಮನಸ್ಸಿನೊಂದಿಗೆ ಸೇರಿ ಇರವು, ಜ್ಞಾನ, ಪ್ರೀತಿಗಳನ್ನು ಸೃಷ್ಟಿಸುವುದು. ಇದನ್ನೇ ವೇದಾಂತಿಗಳು ಸಚ್ಚಿದಾನಂದ ಅದ್ವೈತ ಎನ್ನುವರು. ಆ ನಿಜವಾದ ಅಸ್ತಿತ್ವಕ್ಕೆ ಮಿತಿಯಿಲ್ಲ. ಅಮಿಶ್ರ, ಅವಿಕಾರಿ, ಮುಕ್ತಾತ್ಮ, ಮನಸ್ಸಿನೊಂದಿಗೆ ಸೇರಿಹೋದಾಗ ಅದು ಜೀವಾತ್ಮವಾಗುವುದು. ಸರ್ವವ್ಯಾಪಿಯಾದ ಆಕಾಶವು ಮಡಕೆ ಕುಡಿಕೆಗಳ ಮೂಲಕ ಭಿನ್ನವಾದಂತೆ ತೋರುವ ಹಾಗೆ ಆತ್ಮವು, ಸಸ್ಯ, ಪ್ರಾಣಿ, ಮಾನವ ಮುಂತಾದುವು ಆದಂತೆ ತೋರುತ್ತಿದೆ. ನಿಜವಾದ ಜ್ಞಾನವು ನಮಗೆ ತಿಳಿದಿರುವುದಲ್ಲ, ಅದು ಪ್ರತಿಭೆಯಲ್ಲ, ಯುಕ್ತಿಯಲ್ಲ, ಹುಟ್ಟುಗುಣವಲ್ಲ. ಅದು ಅಧೋಗತಿಗೆ ಇಳಿದು ಭ್ರಾಂತವಾದಾಗ ಅದನ್ನು ಪ್ರತಿಭೆ ಎಂತಲೂ, ಅದಕ್ಕಿಂತಲೂ ಕೆಳಗೆ ಬಂದರೆ ಯುಕ್ತಿಯೆಂತಲೂ ಅದಕ್ಕೂ ಕೆಳಗೆ ಬಂದರೆ, ಅದನ್ನು ಹುಟ್ಟುಗುಣವೆಂದೂ ಕರೆಯುತ್ತೇವೆ. ಯಾವುದು ನಿಜವಾದ ಜ್ಞಾನವಾಗಿದೆಯೊ ಅದೇ ವಿಜ್ಞಾನ, ಅದು ಪ್ರತಿಭೆಯೂ ಅಲ್ಲ, ಯುಕ್ತಿಯೂ ಅಲ್ಲ, ಹುಟ್ಟುಗುಣವೂ ಅಲ್ಲ. ಸರ್ವಜ್ಞತೆ ಎಂಬುದೊಂದೆ ಅದರ ಸಮೀಪಕ್ಕೆ ಬರುವ ವಿವರಣೆ. ಅದಕ್ಕೆ ಮಿತಿ ಇಲ್ಲ, ಅದು ಮಿಶ್ರವಾಗುವುದಿಲ್ಲ. ಆ ಆನಂದವು ಆವೃತವಾದಾಗ ಅದನ್ನು ಪ್ರೀತಿ, ಸ್ಥೂಲ ಅಥವಾ ಸೂಕ್ಷ್ಮ ದೇಹಗಳ ಅಥವಾ ಭಾವನೆಗಳ ಕಡೆಗಿರುವ ಆಕರ್ಷಣೆ ಎನ್ನುವೆವು. ಆ ಧನ್ಯತಾಸ್ಥಿತಿಯ ಒಂದು ವಿಕೃತಿರೂಪ ಮಾತ್ರ ಇದು. ಸಚ್ಚಿದಾನಂದವು ಆತ್ಮನ ಗುಣವಲ್ಲ, ಇದೇ ಅದರ ಸಾರ. ಇದಕ್ಕೂ ಆತ್ಮಕ್ಕೂ ಯಾವ ವ್ಯತ್ಯಾಸವೂ ಇಲ್ಲ, ಈ ಮೂರೂ ಒಂದೇ, ಒಂದನ್ನೇ ಮೂರು ದೃಷ್ಟಿಯಿಂದ ನೋಡುವೆವು. ಇದು ಎಲ್ಲ ಸಾಪೇಕ್ಷಜ್ಞಾನದಾಚೆ ಇದೆ. ಆತ್ಮನ ಅನಂತ ಜ್ಞಾನ ಮಾನವನ ಮೆದುಳಿಗೆ ಇಳಿದುಬಂದಾಗ ಪ್ರತಿಭೆ, ಯುಕ್ತಿ ಮುಂತಾದವುಗಳಂತೆ ವ್ಯಕ್ತವಾಗುವುದು. ತಾನು ವ್ಯಕ್ತವಾಗುವ ಮಧ್ಯವರ್ತಿಗೆ ತಕ್ಕಂತೆ ಅದರ ಆವಿರ್ಭಾವ ವ್ಯತ್ಯಾಸವಾಗುವುದು. ಆತ್ಮನ ದೃಷ್ಟಿಯಿಂದ ಮನುಷ್ಯನಿಗೂ ಅತಿ ಕ್ಷುದ್ರ ಪ್ರಾಣಿಗೂ ವ್ಯತ್ಯಾಸವಿಲ್ಲ. ಪ್ರಾಣಿಯ ಮೆದುಳು ಮಾತ್ರ ಅಷ್ಟು ಪ್ರವರ್ಧಮಾನಕ್ಕೆ ಬಂದಿಲ್ಲ. ಅದರ ಮೂಲಕ ವ್ಯಕ್ತವಾಗುವ ಹುಟ್ಟುಗುಣ ಅಷ್ಟು ಚುರುಕಾಗಿಲ್ಲ. ಮನುಷ್ಯನಲ್ಲಿ ಮೆದುಳು ಮತ್ತೂ ಸೂಕ್ಷ್ಮವಾಗಿದೆ. ಅದಕ್ಕೇ ಆತ್ಮನ ಆವಿರ್ಭಾವ ಸ್ಪಷ್ಟವಾಗಿದೆ. ಶ್ರೇಷ್ಠ ಮಾನವನಲ್ಲಿ ಅದು ಪೂರ್ಣ ಸ್ಪಷ್ಟವಾಗುವುದು. ಇದರಂತೆಯೆ ಅಸ್ತಿತ್ವ ಕೂಡ. ನಮಗೆ ತಿಳಿದಿರುವ ಅಲ್ಪ ಅಸ್ತಿತ್ವವು ಆತ್ಮನ ನೈಜಸ್ವಭಾವದ ಒಂದು ಪ್ರತಿಬಿಂಬವಷ್ಟೆ. ಇದರಂತೆ ಆನಂದ ಕೂಡ. ನಾವು ಯಾವುದನ್ನು ಪ್ರೀತಿ, ಆಕರ್ಷಣೆ ಎನ್ನುವೆವೊ ಅದು ಆತ್ಮನ ಅನಂತ ಧನ್ಯತೆಯ ಒಂದು ಪ್ರತಿಬಿಂಬ ಮಾತ್ರ. ಆವಿರ್ಭಾವದಿಂದ ಮಿತವಾಗುವುದು. ಆದರೆ ಅವ್ಯಕಕ್ತ್ಕೆ, ಆತ್ಮನ ನೈಜಸ್ವಭಾವಕ್ಕೆ ಮಿತಿ ಇಲ್ಲ. ಆ ಧನ್ಯತೆಗೆ ಯಾವ ಮೇರೆಯೂ ಇಲ್ಲ. ಆದರೆ ಪ್ರೀತಿಯಲ್ಲಿ ಒಂದು ಮಿತಿ ಇದೆ. ನಾನು ಒಂದು ದಿನ ನಿನ್ನನ್ನು ಪ್ರೀತಿಸುವೆನು. ಮಾರನೆಯ ದಿನ ದ್ವೇಷಿಸುವೆನು. ನನ್ನ ಪ್ರೀತಿ ಒಂದು ದಿನ ಹಿಗ್ಗುವುದು, ಮತ್ತೊಂದು ದಿನ ಕುಗ್ಗುವುದು. ಏಕೆಂದರೆ ಇದು ಕೇವಲ ಒಂದು ಆವಿರ್ಭಾವ ಮಾತ್ರ.

ನಾವು ವಿರೋಧಿಸುವ ಕಪಿಲರ ಪ್ರಥಮ ಭಾವನೆಯೇ ಅವರ ದೇವರ ವಿಷಯ. ವ್ಯಷ್ಟಿ ಮಹತ್ತಿನಿಂದ ಹಿಡಿದು ವ್ಯಷ್ಟಿ ದೇಹದವರೆಗಿನ ಈ ವಿಕಾರ ಸರಣಿಯನ್ನು ಆಳುವ ಒಬ್ಬ ಪುರುಷ ಹೇಗೆ ಆವಶ್ಯಕವೋ, ಹಾಗೆಯೆ ವಿಶ್ವದಲ್ಲಿ, ವಿಶ್ವಬುದ್ದಿ, ವಿಶ್ವ ಅಹಂಕಾರ, ವಿಶ್ವಮನಸ್ಸು, ಸೂಕ್ಷ್ಮವಿಶ್ವ, ಸ್ಥೂಲವಿಶ್ವ ಇವುಗಳನ್ನು ಆಳುವುದಕ್ಕೆ ಒಬ್ಬ ಈಶ್ವರನಿರಬೇಕಾಗುವುದು. ವಿಶ್ವ ಸೃಷ್ಟಿಯ ಹಿಂದೆ ಸರ್ವವ್ಯಾಪಿಯಾದ, ಎಲ್ಲವನ್ನೂ ನಿಯಂತ್ರಿಸುವ ಈಶ್ವರನೊಬ್ಬನಿಲ್ಲದೆ ಇದ್ದರೆ ಸೃಷ್ಟಿ ಹೇಗೆ ಪೂರ್ಣವಾಗುವುದು? ಸಮಷ್ಟಿಯ ಹಿಂದೆ ವಿಶ್ವ ಪುರುಷನನ್ನು ನಿರಾಕರಿಸಿದರೆ ವ್ಯಷ್ಟಿಯ ಹಿಂದೆಯೂ ಪುರುಷನನ್ನು ನಿರಾಕರಿಸಬೇಕಾಗುವುದು. ವ್ಯಷ್ಟಿಯ ತರತಮ ಆವಿರ್ಭಾವಗಳ ಹಿಂದೆ ಈ ಆವಿರ್ಭಾವಗಳನ್ನೆಲ್ಲಾ ಮಾರಿದ ಅದ್ರವ್ಯನಾದ ಒಬ್ಬ ಪುರುಷನಿರುವನು ಎಂದು ನೀವು ಊಹಿಸಿದರೆ, ಇದೇ ಯುಕ್ತಿಯೇ ಸಮಷ್ಟಿಗೂ ಅನ್ವಯಿಸುವುದು. ಪ್ರಕೃತಿಯ ಆವಿರ್ಭಾವಕ್ಕೆ ಅತೀತವಾದ ವಿಶ್ವಾತ್ಮನೇ ಪರಮೇಶ್ವರ.

ಈಗ ಮುಖ್ಯವಾದ ವ್ಯತ್ಯಾಸ ಬರುವುದು. ಒಬ್ಬರಿಗಿಂತ ಹೆಚ್ಚು ಪುರುಷರು ಇರಬಹುದೆ? ನಾವು ನೋಡಿದಂತೆ ಪುರುಷ ಅನಂತವಾಗಿರುವನು, ಸರ್ವವ್ಯಾಪಿಯಾಗಿರುವನು. ಸರ್ವವ್ಯಾಪಿಯಾಗಿರುವುದು, ಅನಂತವಾಗಿರುವುದು ಎರಡಿರಲಾರದು. ಎ ಮತ್ತು ಬಿ ಎಂಬ ಎರಡು ಅನಂತವಿದ್ದರೆ ಅನಂತವಾದ ಎ ಮತ್ತು ಅನಂತವಾದ ಬಿ ಪರಸ್ಪರ ಮಿತಿಗಳಿಗೆ ಕಾರಣವಾಗುವುದು. ಏಕೆಂದರೆ ಅನಂತ ಎ ಅನಂತ ಬಿ ಅಲ್ಲ; ಮತ್ತು ಅನಂತ ಬಿ ಅನಂತ ಎ ಅಲ್ಲ. ವ್ಯತ್ಯಾಸ ಎಂದರೆ ಪ್ರತ್ಯೇಕತೆ, ಪ್ರತ್ಯೇಕತೆ ಎಂದರೆ ಮಿತಿ, ಎ ಮತ್ತು ಬಿ ಇವುಗಳು ಒಂದು ಮತ್ತೊಂದಕ್ಕೆ ಮಿತಿ ತರುವುದರಿಂದ ಅನಂತ ವಾಗಿರಲಾರವು. ಆದಕಾರಣ ಒಂದು ಅನಂತ ಮಾತ್ರ ಇರಲು ಸಾಧ್ಯ. ಅದೇ ಏಕಪುರುಷ.

ಈಗ ಅಂತರ್ಜಗತ್ತನ್ನು (ಬಿ) ಮತ್ತು ಬಾಹ್ಯ ಜಗತ್ತನ್ನು (ಎ) ತೆಗೆದುಕೊಂಡು ಅವೆರಡೂ ಒಂದೇ ಎಂಬುದನ್ನು ತೋರಿಸುತ್ತೇನೆ. ಬಾಹ್ಯಜಗತ್ತು ಎಂದರೆ ಎ ಮತ್ತು ಮನಸ್ಸು. ಅಂತರ್‌ ಜಗತ್ತು ಬಿ ಮತ್ತು ಮನಸ್ಸು, ಎ ಮತ್ತು ಬಿ ಎರಡೂ ಅಜ್ಞಾತವೂ ಎಂದರೆ ತಿಳಿದಿಲ್ಲದೇ ಇರುವುದೂ ಮತ್ತು ಅಜ್ಞೇಯವು ಎಂದರೆ ತಿಳಿಯಲಾರದ್ದೂ ಆದ ವಸ್ತುಗಳು. ವ್ಯತ್ಯಾಸಕ್ಕೆಲ್ಲ ಕಾರಣ ಕಾಲ ದೇಶ ನಿಮಿತ್ತಗಳು. ಇವೆಲ್ಲ ಮನಸ್ಸಿಗೆ ಸೇರಿದುವು. ಇವಿಲ್ಲದೆ ಯಾವ ಮಾನಸಿಕ ಕ್ರಿಯೆಯೂ ಸಾಧ್ಯವಿಲ್ಲ. ಕಾಲದ ಭಾವನೆ ಇಲ್ಲದೆ ನೀವು ಆಲೋಚನೆ ಮಾಡಲಾರಿರಿ. ದೇಶದ ಭಾವನೆಯಿಲ್ಲದೆ ನೀವು ಕಲ್ಪಿಸಿ ಕೊಳ್ಳಲಾರಿರಿ. ಕಾರ್ಯಕಾರಣ ಸಂಬಂಧವಿಲ್ಲದೆ ಏನೂ ಆಗಲಾರದು. ಇವು ಮನಸ್ಸಿನ ರೀತಿ. ಇವನ್ನು ತೆಗೆದುಬಿಟ್ಟರೆ ಮನಸ್ಸೇ ಇರುವುದಿಲ್ಲ. ವ್ಯತ್ಯಾಸಕ್ಕೆಲ್ಲ ಕಾರಣ ಮನಸ್ಸು. ವೇದಾಂತದ ದೃಷ್ಟಿಯಲ್ಲಿ ಮನಸ್ಸಿನ ಸ್ವಭಾವವೇ ಅಂತರ ಜಗತ್ತು ಮತ್ತು ಬಾಹ್ಯಜಗತ್ತೆಂಬ ತೋರಿಕೆಯ ಭೇದವನ್ನು ತಂದೊಡ್ಡಿರುವುದು. ಬಾಹ್ಯಜಗತ್ತು ಮತ್ತು ಅಂತರ್‌ಜಗತ್ತು ಎರಡೂ ಮನಸ್ಸಿಗೆ ಅತೀತವಾಗಿರುವುದರಿಂದ, ಅವುಗಳಲ್ಲಿ ವ್ಯತ್ಯಾಸವಿಲ್ಲ. ಅವೆರಡೂ ಒಂದು. ಅದಕ್ಕೆ ನಾವು ಯಾವ ಗುಣವನ್ನೂ ಆರೋಪಿಸುವುದಕ್ಕೆ ಆಗುವುದಿಲ್ಲ. ಏಕೆಂದರೆ ಗುಣಗಳು ಹುಟ್ಟುವುದು ಮನಸ್ಸಿನಲ್ಲಿ, ನಿರ್ಗುಣ ವಾಗಿರುವುದು ಏಕವಾಗಿರಬೇಕು. ಬಾಹ್ಯಜಗತ್ತಿಗೆ ಯಾವ ಗುಣವೂ ಇಲ್ಲ. ಅದು ನಮ್ಮ ಮನಸ್ಸಿನ ಗುಣವನ್ನು ತೆಗೆದುಕೊಳ್ಳುವುದು. ಅದರಂತೆಯೇ ಅಂತರ್ಜಗತ್ತು ಕೂಡ. ಆದಕಾರಣ ಬಾಹ್ಯಜಗತ್ತು ಮತ್ತು ಅಂತರ್ಜಗತ್ತುಗಳೆರಡೂ ಒಂದು. ಇಡೀ ವಿಶ್ವವೆಲ್ಲ ಒಂದು. ಪ್ರಪಂಚದಲ್ಲೆಲ್ಲಾ ಒಂದೇ ಒಂದು ಆತ್ಮವಿರುವುದು, ಒಂದೇ ಒಂದು ಅಸ್ತಿತ್ವವಿರುವುದು. ಆ ಒಂದು ದೇಶಕಾಲ ನಿಮಿತ್ತದ ಮೂಲಕ ಪ್ರವೇಶಿಸಿದಾಗ ಹಲವು ಹೆಸರುಗಳಿಂದ ಕರೆಸಿಕೊಳ್ಳಲ್ಪಡುವುದು - ಬುದ್ದಿ, ಸ್ಥೂಲ ಸೂಕ್ಷ್ಮವಸ್ತುಗಳು ಇತ್ಯಾದಿ. ಒಂದೇ ಎಲ್ಲಾ ವಿವಿಧ ಮನೋರೂಪಗಳಂತೆ ಕಾಣುತ್ತಿದೆ. ಅದರ ಒಂದು ಅಂಶ ಕಾಲದೇಶನಿಮಿತ್ತ ಜಾಲಕ್ಕೆ ಸಿಕ್ಕಿದಂತೆ ತೋರಿದಾಗ ಬೇರೆ ಆಕಾರವನ್ನು ತಾಳುವುದು. ಆ ಜಾಲವನ್ನು ತೆಗೆದರೆ ಅದೆಲ್ಲ ಒಂದೆ. ಆದಕಾರಣವೆ ಅದ್ವೈತ ತತ್ವದಲ್ಲಿ ಇಡೀ ಬ್ರಹ್ಮಾಂಡವೆಲ್ಲ ಆತ್ಮೈಕ್ಯವಾಗಿದೆ. ಅದನ್ನೇ ಬ್ರಹ್ಮ ಎಂದು ಕರೆಯುವರು. ವಿಶ್ವದ ಹಿಂದೆ ನಾವು ಅದನ್ನು ನೋಡಿದಾಗ ಅದನ್ನು ಈಶ್ವರ ಎನ್ನುವೆವು. ಅದನ್ನು ಈ ವ್ಯಷ್ಟಿಯ ಹಿಂದೆ ಎಂದರೆ ದೇಹದ ಹಿಂದೆ ನೋಡಿದಾಗ ಜೀವ ಎನ್ನುವೆವು. ಈ ಜೀವವೇ ಮಾನವನಲ್ಲ ಆತ್ಮ. ಇರುವುದು ಒಂದೇ ಪುರುಷ. ಅದೇ ವೇದಾಂತದ ಬ್ರಹ್ಮ. ಈಶ್ವರ ಮತ್ತು ಜೀವಿಯನ್ನು ವಿಶ್ಲೇಷಣೆಮಾಡಿದಾಗ ಅವೆರಡೂ ಬ್ರಹ್ಮವೆ. ನೀನೇ ವಿಶ್ವ, ಅಖಂಡ ಸರ್ವವ್ಯಾಪಿ. “ಎಲ್ಲಾ ಕೈಗಳ ಮೂಲಕ ನೀನು ಕೆಲಸ ಮಾಡುವೆ, ಎಲ್ಲಾ ಬಾಯಿ ಮೂಲಕ ನೀನು ಊಟಮಾಡುವೆ, - ಎಲ್ಲಾ ಮೂಗಿನ ಮೂಲಕ ನೀನು ಉಸಿರಾಡುವೆ, ಎಲ್ಲಾ ಮನಸ್ಸಿನ ಮೂಲಕ ನೀನು ಆಲೋಚಿಸುತ್ತಿರುವೆ.'' ಇಡೀ ವಿಶ್ವವೇ ನೀನು, ವಿಶ್ವವೇ ನಿನ್ನ ದೇಹ, ನೀನೇ ವಿಶ್ವ. ವ್ಯಕ್ತ ಅವ್ಯಕ್ತವೆರಡೂ ನೀನೇ, ನೀನೇ ವಿಶ್ವದ ಆತ್ಮ ಮತ್ತು ದೇಹ ಕೂಡ. ನೀನೇ ದೇವರು, ದೇವತೆಗಳು, ಮನುಷ್ಯ, ಪ್ರಾಣಿ, ಸಸ್ಯ, ಲೋಹ; ನೀನೇ ಸರ್ವವೂ, ಎಲ್ಲಾ ಅಭಿವ್ಯಕ್ತಿಗಳೂ ನಿನ್ನವೆ. ಇರುವುದೆಲ್ಲ ನೀನೆ, ನೀನೇ ಅನಂತಾತ್ಮ, ಅನಂತ ಅವಿಭಾಜ್ಯ. ಅದರಲ್ಲಿ ಅಂಶಗಳಿರಲಾರವು. ಇದ್ದರೆ ಪ್ರತಿ ಅಂಶವೂ ಅನಂತವಾದುದು, ಪ್ರತಿ ಅಂಶವೂ ಪೂರ್ಣಕ್ಕೆ ಸಮನಾಗುವುದು. ನೀನು ಇಂತಹ ಎ ಬಿ ಎಂಬುದು ಸತ್ಯವಾಗಲಾರದು. ಇದೊಂದು ಹಗಲುಗನಸು, ಇದನ್ನರಿತು ಮುಕ್ತನಾಗು. ಇದೇ ಅದ್ವೈತ ನಿರ್ಣಯ. “ನಾನು ದೇಹವಲ್ಲ, ಇಂದ್ರಿಯವಲ್ಲ, ಮನಸ್ಸಲ್ಲ, ನಾನೆ ಸಚ್ಚಿದಾನಂದ ಶಿವೋಹಂ.'' ಇದೇ ನಿಜವಾದ ಜ್ಞಾನ. ಉಳಿದ ಬುದ್ದಿವಂತಿಕೆ, ಯುಕ್ತಿ ಮುಂತಾದುವೆಲ್ಲ ಅಜ್ಞಾನ, ನನಗೆ ಜ್ಞಾನವೆಲ್ಲಿದೆ? ಏಕೆಂದರೆ ನಾನೇ ಜ್ಞಾನಸ್ವರೂಪನಾಗಿರುವೆನು. ನನಗೆ ಜೀವ ವೆಲ್ಲಿದೆ? ಏಕೆಂದರೆ ನಾನೇ ಜೀವಸ್ವರೂಪನಾಗಿರುವೆನು. ನಾನು ಬದುಕಿರುವೆನೆಂಬುದು ನಿಜ. ಏಕೆಂದರೆ ನಾನೇ ಜೀವ. ನನ್ನ ಮೂಲಕ, ನನ್ನಲಿ, ನನ್ನಂತೆ ಅಲ್ಲದೆ ಪ್ರಪಂಚದಲ್ಲಿ ಬೇರೆ ಯಾವುದೂ ಇರಲಾರದು. ನಾನು ಪಂಚಭೂತಗಳ ಮೂಲಕ ವ್ಯಕ್ತವಾಗುತ್ತಿರುವೆನು. ಆದರೆ ನಾನು ಮುಕ್ತಾತ್ಮ, ಸ್ವಾತಂತ್ರ್ಯವನ್ನು ಅರಸುವವರಾರು? ಯಾರೂ ಇಲ್ಲ. ಬದ್ಧನೆಂದು ಭಾವಿಸಿದರೆ ನೀನು ಬದ್ಧನಾಗಿರುವೆ. ಆ ಬಂಧನವನ್ನು ನೀನೇ ಸೃಷ್ಟಿಸಿಕೊಳ್ಳುವೆ. ನೀನು ಮುಕ್ತನೆಂದು ಭಾವಿಸಿದರೆ ಈ ಕ್ಷಣ ಮುಕ್ತ. ಇದೇ ಜ್ಞಾನ, ಮುಕ್ತಿ ನೀಡುವ ಜ್ಞಾನ. ಸಮಸ್ತ ಪ್ರಕೃತಿಯ ಗುರಿಯೆ ಮುಕ್ತಿ.

