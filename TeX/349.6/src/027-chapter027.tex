
\chapter[ಬಹಿರಂಗ ರಹಸ್ಯ]{ಬಹಿರಂಗ ರಹಸ್ಯ\protect\footnote{\engfoot{* C.W, Vol. II, P. 397}}}

\begin{center}
(೧೯೦೦ರ ಜನವರಿ ೫ ರಂದು ಲಾಸ್‌ಏಂಜಲೀಸ್‌ನಲ್ಲಿ ನೀಡಿದ ಪ್ರವಚನ.)
\end{center}

ವಸ್ತುಗಳ ನೈಜಸ್ಥಿತಿಯನ್ನು ಅರಿಯಲು ನಾವು ಯಾವ ಕಡೆ ಪ್ರಯತ್ನಪಟ್ಟರೂ, ವಿಶ್ಲೇಷಣೆಯನ್ನು ಹೆಚ್ಚು ಮಾಡುತ್ತಾ ಹೋದಷ್ಟೂ ಕೊನೆಗೆ ಒಂದು ವಿರೋಧಾಭಾಸದ ವಿಚಿತ್ರ ಸ್ಥಿತಿಗೆ ಬರುವುದನ್ನು ಕಾಣುವೆವು: ನಮ್ಮ ಯುಕ್ತಿಗೆ ಅದನ್ನು ಅರಿಯಲು ಅಸಾಧ್ಯ, ಆದರೂ ಅದು ಸತ್ಯ. ನಾವು ಯಾವುದನ್ನಾದರೂ ತೆಗೆದುಕೊಳ್ಳುತ್ತೇವೆ. ಅದು ಮಿತವಾದುದೆಂದು ನಮಗೆ ಗೊತ್ತಿದೆ. ನಾವು ಅದನ್ನು ವಿಶ್ಲೇಷಣೆ ಮಾಡಲು ಪ್ರಯತ್ನ ಪಟ್ಟೊಡನೆಯೆ ಅದು ನಮ್ಮನ್ನು ಯುಕ್ತಿಯ ಆಚೆಗೆ ಒಯ್ಯುತ್ತದೆ. ಅದರ ಗುಣ, ಸಾಧ್ಯತೆ, ಶಕ್ತಿ ಸಂಬಂಧಗಳಿಗೆ ಒಂದು ಮಿತಿ ಇರುವಂತೆ ಕಾಣುವುದಿಲ್ಲ. ಅದು ಅನಂತವಾಗಿದೆ. ಮಿತವಾದ ಒಂದು ಹೂವನ್ನೆ ತೆಗೆದುಕೊಳ್ಳಿ. ಆದರೆ ಆ ಹೂವಿನ ವಿಚಾರವೆಲ್ಲ ತನಗೆ ಗೊತ್ತು ಎಂದು ಯಾರು ಧೈರ್ಯವಾಗಿ ಹೇಳಬಲ್ಲರು? ಆ ಹೂವಿನ ವಿಷಯವನ್ನೆಲ್ಲ ತಿಳಿಯುವುದಕ್ಕೆ ಸಾಧ್ಯವೇ ಇಲ್ಲದಂತೆ ಕಾಣುವುದು. ಹೂವು ಅನಂತವಾಯಿತು; ಪ್ರಾರಂಭದಲ್ಲಿ ಅದು ಸಾಂತವಾಗಿತ್ತು. ಒಂದು ಮರಳುಕಣವನ್ನು ತೆಗೆದುಕೊಳ್ಳಿ. ಅದನ್ನು ವಿಶ್ಲೇಷಣೆ ಮಾಡಿ, ಅದು ಸಾಂತವೆಂದು ನಾವು ಮೊದಲು ಹೊರಡುವೆವು. ಕೊನೆಗೆ ಅದು ಸಾಂತವಲ್ಲ ಅನಂತವೆಂದು ಅರಿಯುವೆವು. ಆದರೂ ನಾವು ಸಾಂತವೆಂದು ತಿಳಿದಿದ್ದೆವು. ಅದರಂತೆಯೆ ಹೂವನ್ನು ಒಂದು ಸಾಂತವಸ್ತುವೆಂದು ಪರಿಗಣಿಸುವೆವು.

ಹೀಗೆಯೆ ಭೌತಿಕ ಮತ್ತು ಮಾನಸಿಕ ಅನುಭವ, ಆಲೋಚನೆಗಳೆಲ್ಲ ಕೂಡ. ಅವು ಸಣ್ಣ ವಸ್ತುಗಳು, ಅವನ್ನು ಕನಿಷ್ಟಪಕ್ಷದಲ್ಲಿ ತಿಳಿದುಕೊಳ್ಳೋಣವೆಂದು ಪ್ರಾರಂಭಿಸುವೆವು. ಆದರೆ ಅವು ಬೇಗ ನಮ್ಮ ಯುಕ್ತಿಗೆ ಅತೀತವಾಗುವುವು, ಅನಂತಸಾಗರದಲ್ಲಿ ಮಾಯವಾಗುವುವು. ನಮಗೆ ಮೊದಲು ಕಾಣುವ ಮಹಾ ವಸ್ತುವೆ ನಾವು. ನಾವು ಕೂಡ ಅಸ್ತಿತ್ವದ ದೃಷ್ಟಿಯಲ್ಲಿ ಇದೇ ಸಂದೇಹದಲ್ಲಿರುವೆವು. ಸಾಂತವೆಂದು ನಮಗೆ ಕಾಣುತ್ತಿದೆ. ನಾವು ಹುಟ್ಟಿದೆವು, ಸಾಯುವೆವು. ನಮ್ಮ ದೃಷ್ಟಿವಲಯ ಬಹಳ ಕಿರಿದು. ಸುತ್ತಲೂ ನಾವು ಮಿತಿಗೆ ಒಳಪಟ್ಟಿರುವೆವು. ಪ್ರಪಂಚ ನಮ್ಮ ಸುತ್ತಲೂ ಆವರಿಸಿದೆ. ಒಂದು ಕ್ಷಣದಲ್ಲಿ ಪ್ರಕೃತಿ ನಮ್ಮನ್ನು ನಿರ್ನಾಮ ಮಾಡಬಲ್ಲದು. ನಮ್ಮ ದೇಹ ಹೇಗೊ ಒಟ್ಟಿಗೆ ಇದೆ. ಕ್ಷಣವೊಂದರಲ್ಲಿ ಅದು ಚೂರುಚೂರಾಗಿ ಹೋಗಬಲ್ಲದು. ಇದು ನಮಗೆ ಗೊತ್ತಿದೆ. ಕರ್ಮ ಪ್ರಪಂಚದಲ್ಲಿ ನಾವೆಷ್ಟು ಬಲಹೀನರು! ಹೆಜ್ಜೆ ಹೆಜ್ಜೆಗೆ ನಮ್ಮ ಇಚ್ಛಾಶಕ್ತಿಯು ಕುಂಠಿತವಾಗುವುದು. ನಮಗೆ ಎಷ್ಟೋ ಮಾಡಬೇಕೆಂದು ಆಸೆ. ಆದರೆ ನಾವು ಸಾಧಿಸುವುದು ಎಷ್ಟು ಅಲ್ಪ! ನಾವು ಏನೇನೊ ಇಚ್ಚಿಸಬಹುದು. ನಮಗೆ ಎಲ್ಲಾ ಬೇಕು. ದೂರದ ನಕ್ಷತ್ರಕ್ಕೆ ಹೋಗಬೇಕೆಂದು ಆಸೆ, ಆದರೆ ಹೋಗಲಾರೆವು. ಆದರೆ ನಮ್ಮ ಆಸೆಯಲ್ಲಿ ಸಾಧಿಸುವುದು ಎಷ್ಟು ಅಲ್ಪ! ದೇಹ ಅದಕ್ಕೆ ಆಸ್ಪದ ಕೊಡುವುದಿಲ್ಲ. ಸರಿ, ಪ್ರಕೃತಿ ಕೂಡ ನಮ್ಮ ಇಚ್ಛೆ ಈಡೇರುವುದಕ್ಕೆ ಆತಂಕವಾಗಿರುವುದು. ನಾವು ದುರ್ಬಲರು. ಯಾವುದು ಹೂವಿನಲ್ಲಿ, ಒಂದು ಮರಳು ಕಣದಲ್ಲಿ, ಪ್ರಕೃತಿಯಲ್ಲಿ ಮತ್ತು ಪ್ರತಿಯೊಂದು ಆಲೋಚನೆಯಲ್ಲಿಯೂ ಸತ್ಯವೊ ಅದು ನಮ್ಮ ವಿಷಯದಲ್ಲಿ ಅದಕ್ಕಿಂತ ನೂರರಷ್ಟು ಸತ್ಯ. ನಾವೂ ಕೂಡ ಈ ಇಕ್ಕಟ್ಟಿನ ಮಧ್ಯೆ ಸಿಕ್ಕಿಕೊಂಡಿರುವೆವು. ಏಕಕಾಲದಲ್ಲಿ ನಾವು ಸಾಂತ ಮತ್ತು ಅನಂತ, ಅಲೆಯೆ ಸಾಗರ, ಆದರೂ ಅಲ್ಲ. ಅಲೆಯ ಯಾವ ಭಾಗವನ್ನೂ ಇದು ಸಮುದ್ರ ಅಲ್ಲ ಎಂದು ಹೇಳಲಾಗುವುದಿಲ್ಲ. ಸಾಗರವೆಂದರೆ ಅಲೆ ಮತ್ತು ಇತರ ವಸ್ತುಗಳಿಗೆಲ್ಲಾ ಅನ್ವಯಿಸುವುದು. ಆದರೂ ಅಲೆ ಸಾಗರದಿಂದ ಬೇರೆ. ಅನಂತ ಅಸ್ತಿತ್ವ ಸಾಗರದಲ್ಲಿ ನಾವು ಕಿರಿ ಅಲೆಗಳಂತೆ. ಆದರೂ ನಮ್ಮನ್ನು ನಾವು ಅರಿಯಬೇಕೆಂದು ಯತ್ನಿಸಿದರೆ ಅದು ಅಸಾಧ್ಯ - ನಾವು ಅನಂತವಾಗಿರುವೆವು.

ನಾವು ಕನಸಿನಲ್ಲಿ ಸಂಚರಿಸುತ್ತಿರುವಂತೆ ತೋರುವುದು. ಕನಸು ಅದನ್ನು ಕಾಣುತ್ತಿರುವವನಿಗೆ ಸತ್ಯ. ಆದರೆ ಅದನ್ನು ಅರಿಯಬೇಕೆಂದು ಯತ್ನಿಸಿದೊಡನೆಯೆ ಅದು ಮಾಯವಾಗುವುದು. ಏತಕ್ಕೆ? ಅದು ಅಸತ್ಯವೆಂದಲ್ಲ. ಅದನ್ನು ಯುಕ್ತಿಯಿಂದ, ಬುದ್ದಿಯಿಂದ ತಿಳಿದುಕೊಳ್ಳುವುದಕ್ಕೆ ಆಗುವುದಿಲ್ಲ. ಈ ಜೀವನದಲ್ಲಿ ಎಲ್ಲವೂ ಅಷ್ಟು ಅಗಾಧವಾಗಿದೆ. ಬುದ್ದಿ ಇದರೊಂದಿಗೆ ಸರಿತೂಗಲಾರದು. ಯುಕ್ತಿನಿಯಮಕ್ಕೆ ಅದು ಮಣಿಯುವುದಿಲ್ಲ. ಬುದ್ದಿ ತನ್ನನ್ನು ಹಿಡಿಯಲು ಬೀಸುವ ಬಲೆಯನ್ನು ನೋಡಿ ಅದು ನಗುವುದು. ಇದು ಆತ್ಮನಿಗೆ ಸಾವಿರ ಪಾಲು ಹೆಚ್ಚು. ನಾವೇ ವಿಶ್ವದ ಅತಿ ದೊಡ್ಡ ರಹಸ್ಯ.

ಇದೆಲ್ಲ ಎಷ್ಟು ಅದ್ಭುತವಾಗಿದೆ! ಮನುಷ್ಯರ ಕಣ್ಣನ್ನು ನೋಡಿ, ನಾವು ಅದನ್ನು ಎಷ್ಟು ಸುಲಭವಾಗಿ ನಾಶಮಾಡಬಹುದು. ಆದರೂ ಬೃಹತ್ ಸೂರ್ಯನಿರುವುದು ನೀವು ನೋಡುವುದರಿಂದ. ಜಗತ್ತು ಇದೆ, ಕಾರಣ ನಿಮ್ಮ ಕಣ್ಣು ಅದಕ್ಕೆ ಪ್ರಮಾಣವಾಗಿದೆ. ಈ ರಹಸ್ಯವನ್ನು ಕುರಿತು ಆಲೋಚಿಸಿ ನೋಡಿ. ಈ ಅಲ್ಪ ಕಿರಿಯ ಚಕ್ಷುಗಳು! ಪ್ರಬಲ ಕಾಂತಿಯಾಗಲಿ ಒಂದು ಗುಂಡು ಸೂಜಿಯಾಗಲಿ ಅದನ್ನು ನಾಶಮಾಡಬಲ್ಲದು. ಆದರೂ ಪ್ರಚಂಡ ಧ್ವಂಸಕಾರಕಯಂತ್ರ, ಪ್ರಕೃತಿಯ ವಿನಾಶಹೇತುಗಳು, ಕೋಟ್ಯಂತರ ಸೂರ್ಯ, ಚಂದ್ರ, ತಾರೆ, ಪೃಥ್ವಿಯ ಅದ್ಭುತಗಳೆಲ್ಲ ಇರುವುದಕ್ಕೆ ಕಾರಣ ಈ ಎರಡು ಕಣ್ಣುಗಳು. ಇವುಗಳೆಲ್ಲ ಇವೆ ಎಂದು ಈ ಎರಡು ಪುಟ್ಟಕಣ್ಣುಗಳು ದಾಖಲೆ ಕೊಡಬೇಕು. ಪ್ರಕೃತಿ ನೀನಿರುವೆ ಎಂದು ಕಣ್ಣು ಹೇಳುವುದು. ಪ್ರಕೃತಿ ಇದೆ ಎಂದು ನಾವು ನಂಬುವೆವು. ಹೀಗೆ ನಮ್ಮ ಇಂದ್ರಿಯಗಳೆಲ್ಲ.

ಇದೇನು? ದುರ್ಬಲತೆ ಇರುವುದೆಲ್ಲಿ? ಯಾರು ಬಲಶಾಲಿಗಳು? ಯಾವುದು ದೊಡ್ಡದು, ಯಾವುದು ಸಣ್ಣದು? ಒಂದು ಕ್ಷುದ್ರಕಣವು ಕೂಡ ಇಡೀ ಬ್ರಹ್ಮಾಂಡದ ಇರುವಿಕೆಗೆ ಅವಶ್ಯವಾಗಿರುವಂತಹ ಅದ್ಭುತ ಸಾಪೇಕ್ಷ ಜಗತ್ತಿನಲ್ಲಿ, ಯಾರು ಮೇಲು ಯಾರು ಕೀಳು? ಇದನ್ನು ನಾವು ನಿಷ್ಕರ್ಷಿಸಲಾರೆವು. ಏತಕ್ಕೆ? ಏಕೆಂದರೆ ಯಾವುದೂ ಮೇಲಲ್ಲ ಕೀಳಲ್ಲ. ಈ ಅನಂತ ಅಸ್ತಿತ್ವದ ಕಡಲಿನಲ್ಲಿ ಎಲ್ಲವೂ ಒಂದಕ್ಕೊಂದು ಅಧೀನ. ಇವುಗಳ ಸತ್ಯವೇ ಅನಂತ. ಮೇಲಿರುವುದೆಲ್ಲ ಅನಂತವೆ. ಮರ ಅನಂತ, ಅದರಂತೆಯೆ ನೀವು ನೋಡುವ, ಅನುಭವಿಸುವ ಪ್ರತಿಯೊಂದು ಮರಳು ಕಣ, ಯೋಚನೆ, ಜೀವ, ಇರುವುದೆಲ್ಲ ಅನಂತ. ಅನಂತವೇ ಸಾಂತ, ಸಾಂತವೇ ಅನಂತ. ಇದೇ ನಮ್ಮ ಅಸ್ತಿತ್ವ.

ಇದೆಲ್ಲ ನಿಜವಿರಬಹುದು. ಆದರೆ ಈಗ ಯಾವುದು ಅನಂತವನ್ನು ಬಯಸುತ್ತಿದೆಯೋ ಅದು ಅಪ್ರಜ್ಞಾಪೂರ್ವಕವಾಗಿದೆ. ನಮ್ಮ ಅನಂತ ಸ್ವಭಾವವನ್ನು ನಾವು ಮರೆತಿರುವೆವೆಂದಲ್ಲ. ಯಾರೂ ಅದನ್ನು ಮರೆಯಲಾರರು. ತಾನು ನಾಶವಾಗುತ್ತೇನೆ ಎಂದು ಯಾರು ಆಲೋಚಿಸಬಲ್ಲರು? ನಾನು ಸಾಯುತ್ತೇನೆ ಎಂದು ಯಾರಿಗೆ ಆಲೋಚಿಸುವುದಕ್ಕೆ ಸಾಧ್ಯ? ಯಾರಿಗೂ ಸಾಧ್ಯವಿಲ್ಲ. ಅನಂತಕ್ಕೂ ನಮಗೆ ಇರುವ ಸಂಬಂಧವೆಲ್ಲ ಅಪ್ರಜ್ಞಾಪೂರ್ವಕವಾಗಿ ಕೆಲಸ ಮಾಡುತ್ತಿದೆ. ಒಂದು ರೀತಿಯಲ್ಲಿ, ನಮ್ಮ ನಿಜವಾದ ಸ್ಥಿತಿಯನ್ನು ಮರೆಯುವೆವು. ಅದಕ್ಕೇ ಈ ದುಃಖವೆಲ್ಲ.

ನಮ್ಮ ದೈನಂದಿನ ವ್ಯವಹಾರದಲ್ಲಿ ಸಣ್ಣ ಪುಟ್ಟ ವಸ್ತುಗಳ ವಿಷಯದಲ್ಲಿ ವ್ಯಸ್ತರಾಗುವೆವು, ಕುದ್ರ ವ್ಯಕ್ತಿಗಳಿಗೆ ದಾಸರಾಗುವೆವು. ನಾವು ಸಾಂತರು, ಕುದ್ರವ್ಯಕ್ತಿಗಳು ಎಂದು ಭಾವಿಸುವೆವು. ಅದರಿಂದಲೇ ದುಃಖ. ಆದರೂ ನಾವು ಅನಂತರೆಂದು ಭಾವಿಸುವುದು ಎಷ್ಟು ಕಷ್ಟ! ಈ ದುಃಖಸಂಕಟಗಳ ಮಧ್ಯದಲ್ಲಿ ಒಂದು ಸಣ್ಣ ಘಟನೆ ನಮ್ಮ ಸಮತ್ವಕ್ಕೆ ಭಂಗ ತರುವಾಗ ನಾವು ಅನಂತ ಎಂದು ಭಾವಿಸಬೇಕಾಗಿದೆ. ವಾಸ್ತವಿಕವಾಗಿ ನಾವು ಅದೇ ಆಗಿರುವೆವು. ತಿಳಿದೊ ತಿಳಿಯದೆಯೊ ಆ ಅನಂತವನ್ನು ನಾವೆಲ್ಲ ಅರಸುತ್ತಿರುವೆವು; ನಿತ್ಯಮುಕ್ತವಾದುದನ್ನು ನಾವೆಲ್ಲ ಅರಸುತ್ತಿರುವೆವು.

ಒಂದು ಧರ್ಮವಿಲ್ಲದ, ಯಾವ ದೇವದೇವತೆಗಳನ್ನೂ ಆರಾಧಿಸದ, ಯಾವ ಜನಾಂಗವೂ ಪ್ರಪಂಚದಲ್ಲಿ ಇಲ್ಲ. ದೇವರಿರುವನೆ ಇಲ್ಲವೆ ಎಂಬುದಲ್ಲ ಪ್ರಶ್ನೆ. ಆದರೆ ಈ ಭಾವನೆಯ ಹಿಂದಿರುವ ಮಾನಸಿಕ ಸ್ಥಿತಿಯ ಸ್ವರೂಪವೇನು? ಪ್ರಪಂಚವೆಲ್ಲ ಏತಕ್ಕೆ ದೇವರನ್ನು ಹುಡುಕುವುದಕ್ಕೆ ಪ್ರಯತ್ನಿಸುತ್ತಿದೆ? ಇದಕ್ಕೆ ಕಾರಣವೇನು? ನಮಗೆ ಎಷ್ಟೇ ಬಂಧನವಿದ್ದರೂ, ನಮಗೆ ಒಂದು ಕ್ಷಣ ಪುರಸತ್ತು ಕೊಡದೆ ಅದ್ಭುತ ಶಕ್ತಿಯ ನಿಯಮ ತುಳಿಯುತ್ತಿದ್ದರೂ, ನಾವೆಲ್ಲಿಗೆ ಹೋದರೂ, ಏನು ಮಾಡಲೆತ್ನಿಸಿದರೂ, ಸರ್ವ ವ್ಯಾಪಿಯಾದ ಈ ನಿಯಮ ನಮಗೆ ತೊಂದರೆ ಕೊಡುತ್ತಿದ್ದರೂ ಮಾನವ ತನ್ನ ಸ್ವಾತಂತ್ರ್ಯವನ್ನು ಮರೆಯಲಾರ, ಅದನ್ನು ಅರಸುತ್ತಿರುವನು. ಸ್ವಾತಂತ್ರ್ಯದ ಅನ್ವೇಷಣೆಯೇ ಎಲ್ಲಾ ಧರ್ಮಗಳ ಗುರಿ. ಅದು ಅವುಗಳಿಗೆ ಗೊತ್ತಿರಲಿ ಇಲ್ಲದಿರಲಿ, ಅದನ್ನು ಚೆನ್ನಾಗಿ ವಿವರಿಸಲಿ, ಅಥವಾ ಒರಟಾಗಿ ವಿವರಿಸಲಿ ಈ ಭಾವನೆ ಮಾತ್ರ ಅಲ್ಲಿದೆ. ಅತಿ ಕೆಳಗಿರುವ ಅಜ್ಞಾನಿ ಕೂಡ ಪ್ರಕೃತಿ ನಿಯಮವನ್ನು ತನ್ನ ಸ್ವಾಧೀನದಲ್ಲಿಟ್ಟುಕೊಂಡಿರುವ ಯಾವುದಾದರೊಂದನ್ನು ನೋಡಲಿಚ್ಛಿಸುವನು. ಪ್ರಕೃತಿಯನ್ನು ಸೋಲಿಸುವ, ಅದರ ಸರ್ವಾಧಿಕಾರವನ್ನು ಅಲ್ಲಗಳೆಯಬಲ್ಲ, ಯಾವ ನಿಯಮದ ಅಂಕೆಯಲ್ಲಿಯೂ ಇಲ್ಲದ ಒಂದು ಭೂತವೊ ಪ್ರೇತವೊ ದೇವರನ್ನೊ ನೋಡಲಿಚ್ಚಿಸುವನು. “ಈ ನಿಯಮವನ್ನು ಅಲ್ಲಗಳೆಯಬಲ್ಲ ಯಾರಾದರೊಬ್ಬರು” ಎಂಬ ಧ್ವನಿಯೇ ಮಾನವನ ಹೃದಯದಿಂದ ಬರುತ್ತಿದೆ. ನಿಯಮವನ್ನು ತುಳಿಯುವ ಯಾರಾದರೊಬ್ಬರನ್ನು ನಾವು ಅರಸುತ್ತಿರುವೆವು. ರೈಲು, ಕಂಬಿಯ ಮೇಲೆ ವೇಗದಿಂದ ಓಡುತ್ತಿದೆ. ಒಂದು ಸಣ್ಣ ಕೀಟ ರೈಲಿನ ದಾರಿಯಿಂದ ತಪ್ಪಿಸಿಕೊಳ್ಳುವುದು. ನಾವು ತಕ್ಷಣ ''ರೈಲು ಜಡ, ಒಂದು ಯಂತ್ರ, ಕೀಟ ಚೇತನ, ಬದುಕಿದೆ' ಎನ್ನುವೆವು. ಏಕೆಂದರೆ ಅದು ನಿಯಮವನ್ನು ಉಲ್ಲಂಘಿಸಲು ಯತ್ನಿಸಿತು. ರೈಲ್ವೆ ಯಂತ್ರಕ್ಕೆ ಎಷ್ಟೇ ಶಕ್ತಿಯಿರಲಿ, ಬಲವಿರಲಿ ನಿಯಮವನ್ನು ಉಲ್ಲಂಘಿಸಲಾರದು. ಮನುಷ್ಯ ಇಚ್ಚಿಸಿದ ಕಡೆ ಅದು ಹೋಗಬೇಕು, ಬೇರೆ ವಿಧಿಯೆ ಇಲ್ಲ. ಕೀಟಸಣ್ಣದಾದರೂ ಕ್ಷುದ್ರವಾದರೂ ನಿಯಮವನ್ನು ಉಲ್ಲಂಘಿಸಿ ಅಪಾಯದಿಂದ ಪಾರಾಗಲು ಯತ್ನಿಸಿತು. ನಿಯಮವನ್ನು ಮೀರುವುದಕ್ಕೆ ಸ್ವಾತಂತ್ರ್ಯವನ್ನು ಚಲಾಯಿಸುವುದಕ್ಕೆ ಯತ್ನಿಸಿತು. ಆ ಕೀಟದಲ್ಲಿ ಅದು ಭವಿಷ್ಯದಲ್ಲಿ ದೇವರಾಗುವ ಚಿಹ್ನೆ ಇತ್ತು.

ಎಲ್ಲ ಕಡೆಯಲ್ಲಿಯೂ ಸ್ವಾತಂತ್ರ್ಯವನ್ನು, ಆತ್ಮನ ಸ್ವಾತಂತ್ರ್ಯವನ್ನು ವ್ಯಕ್ತಪಡಿಸುವುದನ್ನು ನೋಡುವೆವು. ಪ್ರತಿಯೊಂದು ಧರ್ಮದ ದೇವತೆಗಳ ರೂಪದಲ್ಲಿ ಅದು ಪ್ರತಿಬಿಂಬಿತವಾಗಿದೆ. ಆದರೆ ಇದಿನ್ನೂ ಕೇವಲ ಬಾಹ್ಯವಾಗಿದೆ, ದೇವರನ್ನು ಹೊರಗಡೆ ನೋಡುವವರಿಗೆ ಮಾತ್ರ ಆಗಿದೆ. ಮಾನವ ತಾನು ಏನೂ ಇಲ್ಲ ಎಂದು ನಿರ್ಧರಿಸಿದ; ನಾನು ಮುಕ್ತನಾಗುವಂತೆಯೇ ಇಲ್ಲ ಎಂದು ಅಂಜಿದ, ಅದಕ್ಕೆ ಪ್ರಕೃತಿಯಾಚೆ ಮುಕ್ತನಾದ ಒಬ್ಬನನ್ನು ಹುಡುಕಿಕೊಂಡು ಹೋದ. ಕ್ರಮೇಣ ಹಲವರು ಇಂತಹ ಮುಕ್ತಜೀವಿಗಳಿರುವರು ಎಂಬುದನ್ನು ನೋಡಿದನು. ಕ್ರಮೇಣ ಅವರನ್ನೆಲ್ಲ ಸರ್ವೇಶ್ವರ ದೇವದೇವ ಎಂಬ ಒಂದು ವ್ಯಕ್ತಿಯಲ್ಲಿ ಹುದುಗಿಸಿದ. ಇದರಿಂದಲೂ ಅವನಿಗೆ ತೃಪ್ತಿ ಬರಲಿಲ್ಲ. ಸತ್ಯಕ್ಕೆ ಸ್ವಲ್ಪ ಸಮೀಪ ಬಂದ. ತಾನೇನಾಗಿದ್ದರೂ, ಒಂದು ರೀತಿಯಲ್ಲಿ ತನಗೂ ಆ ದೇವದೇವ ಸರ್ವೆಶ್ವರನಿಗೂ ಒಂದು ಸಂಬಂಧವಿದೆ ಎಂಬುದನ್ನು ಕಂಡುಹಿಡಿದನು. ಅವನು ತಾನು ಬದ್ಧ ದುರ್ಬಲ ಕೀಳು ಎಂದು ಎಷ್ಟೇ ಭಾವಿಸಿದ್ದರೂ, ಹೇಗೋ ಆ ಸರ್ವೇಶ್ವರನೊಂದಿಗೆ ಒಂದು ಸಂಬಂಧವಿದೆ ಎಂದು ಭಾವಿಸಿದ. ಆಗ ಅವನಿಗೆ ಹಲವು ಬಗೆಯ ದರ್ಶನಗಳಾದವು, ಆಲೋಚನೆ ವಿಕಾಸವಾಯಿತು, ಜ್ಞಾನ ವೃದ್ಧಿಯಾಯಿತು. ಕ್ರಮೇಣ ಅವನು ದೇವರನ್ನು ಸಮೀಪಿಸಿದನು. ದೇವತೆಗಳೆಲ್ಲ, ಸರ್ವಶಕ್ತನಾದ ಒಬ್ಬ ಮುಕ್ತಾತ್ಮನನ್ನು ಹುಡುಕುವ ಮಾನಸಿಕ ಸಾಹಸವೆಲ್ಲ ಕೇವಲ ತನ್ನ ಸಂಬಂಧದ ಭಾವನೆಯ ಪ್ರತಿಬಿಂಬ ಎಂಬುದನ್ನು ಅರಿತನು. ಕೊನೆಗೆ ಅವನು 'ದೇವರು ತನ್ನಂತೆ ಮಾನವನನ್ನು ಸೃಷ್ಟಿಸಿರುವುದು ಮಾತ್ರವಲ್ಲ 'ಮಾನವ ತನ್ನಂತೆ ದೇವರನ್ನು ಸೃಷ್ಟಿಸಿರುವನು” ಎಂಬುದೂ ಸತ್ಯವೆಂದು ಅರಿತನು. ದಿವ್ಯ ಸ್ವಾತಂತ್ರ್ಯ ಭಾವನೆಯನ್ನು ಇದು ವ್ಯಕ್ತಪಡಿಸಿತು. ಆ ದಿವ್ಯಾತ್ಮ ಯಾವಾಗಲೂ ತಮ್ಮ ಆಂತರ್ಯದಲ್ಲಿದ್ದನು, ಅಂತರತಮನಾಗಿದ್ದನು. ಅವನನ್ನು ನಾವು ಹೊರಗೆ ಅರಸುತ್ತಿದ್ದೆವು. ಕೊನೆಗೆ ನಮ್ಮ ಹೃದಯಾಂತರಾಳದಲ್ಲೇ ಅವನಿರುವನೆಂಬುದನ್ನು ಅರಿತೆವು. ಒಬ್ಬ ಮನುಷ್ಯ ತನ್ನ ಎದೆಯ ಬಡಿತವನ್ನೇ ಯಾರೋ ಹೊರಗೆ ಬಾಗಿಲನ್ನು ಶಬ್ದ ಮಾಡುತ್ತಿರುವರೆಂದು ಭ್ರಮಿಸಿ, ಬಾಗಿಲು ತೆರೆದು ಯಾರೂ ಕಾಣದೆ ಹಿಂತಿರುಗಿದ ಕಥೆ ನಿಮಗೆ ಗೊತ್ತಿರಬಹುದು. ಪುನಃ ಬಾಗಿಲ ಸದ್ದನ್ನು ಕೇಳಿದನು. ಆಗಲೂ ಯಾರೂ ಇರಲಿಲ್ಲ. ಅನಂತರ ಇದು ತನ್ನದೇ ಎದೆಯ ಬಡಿತ, ಅದನ್ನು ಹೊರಗಿನ ಬಾಗಿಲ ಸದ್ದು ಎಂದು ತಪ್ಪು ತಿಳಿದಿದ್ದೆ ಎಂದು ಅರಿತನು. ಇದರಂತೆಯೇ ಹೊರಗೆ ಅನ್ವೇಷಣೆ ಮುಗಿದ ಮೇಲೆ, ಮೊದಲು ಹೊರಗಿನ ಪ್ರಕೃತಿಯಲ್ಲಿ ಇದೆ ಎಂದು ಕಲ್ಪಿಸಿಕೊಂಡಿದ್ದ ಅನಂತ ಸ್ವಾತಂತ್ರ್ಯ ನಿಜವಾಗಿಯೂ ತನ್ನ ಅಂತರಂಗ ಸಾಕ್ಷಿ, ಆತ್ಮ, ಸತ್ಯ, ಅದು ತಾನೆ ಆಗಿರುವೆನು ಎಂಬುದನ್ನು ಅರಿಯುವನು.

ಕೊನೆಗೆ ಮಾನವ ಅಸ್ತಿತ್ವದ ಈ ಅದ್ಭುತ ದ್ವಿಮುಖವನ್ನು, ಅಂದರೆ ಆತ್ಮ ಸಾಂತ ಮತ್ತು ಅನಂತವಾಗಿದೆ, ಆ ಸರ್ವವ್ಯಾಪಿಯೆ ಸಾಂತವಾಗಿರುವ ಆತ್ಮ ಎಂಬುದನ್ನು ತಿಳಿಯುವನು. ಅನಂತವು ಬುದ್ದಿಜಾಲದಲ್ಲಿ ಸಿಕ್ಕಿ ಸಾಂತವಾದಂತೆ ತೋರುತ್ತಿರುವುದು, ಆದರೆ ಸತ್ಯ ಎಂದಿಗೂ ಬದಲಾಯಿಸುವುದಿಲ್ಲ.

ಇದೇ ನಿಜವಾದ ಜ್ಞಾನ: ನಮ್ಮ ಆತ್ಮನ ಆತ್ಮ, ನಮ್ಮಲ್ಲಿರುವ ಸತ್ಯ. ಅವಿಕಾರಿಯೂ, ಸನಾತನವೂ, ಶುದ್ದವೂ ಸ್ವತಂತ್ರವೂ ಆದುದು. ನಾವು ನಿಲ್ಲಲು ಇರುವ ಗಟ್ಟಿಯಾದ ನೆಲವೇ ಅದು.

ಇದೇ ಎಲ್ಲ ಮರಣದ ಅಂತ್ಯ, ಎಲ್ಲ ಅಮೃತತ್ವದ ಉದಯ, ಎಲ್ಲ ದುಃಖದ ಕೊನೆ. ಯಾರು ಅನೇಕದಲ್ಲಿ ಆ ಏಕವನ್ನು ನೋಡುವನೊ, ಬದಲಾಯಿಸುತ್ತಿರುವುದರಲ್ಲಿ ಬದಲಾಯಿಸದೆ ಇರುವವನನ್ನು ನೋಡುವನೊ, ಅವನನ್ನು ತನ್ನಾತ್ಮನ ಆತ್ಮವೆಂದು ಅರಿಯುವನೋ ಅವನಿಗೆ ಮಾತ್ರ ಶಾಶ್ವತ ಶಾಂತಿ, ಇತರರಿಗಲ್ಲ.

ಎಷ್ಟೇ ಅವನತಿಗೆ ಇಳಿದಿದ್ದರೂ, ದುಃಖ ಕೂಪದಲ್ಲಿ ಬಿದ್ದಿದ್ದರೂ ಆತ್ಮ ತನ್ನ ಜ್ಯೋತಿಕಿರಣ ಒಂದನ್ನು ಕಳಿಸುವುದು. ಮಾನವನು ಜಾಗ್ರತನಾಗಿ ಯಾವುದು ನಿಜವಾಗಿ ತನ್ನದೊ ಅದನ್ನು ಎಂದಿಗೂ ಕಳೆದುಕೊಳ್ಳಲಾರನೆಂದು ಅರಿಯುವನು. ನಿಜವಾಗಿ ಯಾವುದು ನಮ್ಮದೊ ಅದನ್ನು ನಾವು ಎಂದಿಗೂ ಕಳೆದುಕೊಳ್ಳುವುದಿಲ್ಲ. ಯಾರು ತಮ್ಮ ವ್ಯಕ್ತಿತ್ವವನ್ನು ಕಳೆದುಕೊಳ್ಳಲಿಚ್ಛಿಸುವರು? ತಮ್ಮ ಜೀವನವನ್ನು ಕಳೆದುಕೊಳ್ಳಲಿಚ್ಛಿಸುವರು? ನಾನು ಒಳ್ಳೆಯವನಾಗಿದ್ದರೆ, ಮೊದಲು ನನ್ನ ಅಸ್ತಿತ್ವ, ಅನಂತರ ಅದು ಒಳ್ಳೆಯ ಗುಣದ ಬಣ್ಣವನ್ನು ಪಡೆಯುವುದು. ನಾನು ಕೆಟ್ಟವನಾಗಿದ್ದರೆ, ಮೊದಲು ಅಸ್ತಿತ್ವ, ಅನಂತರ ಅದು ಕೆಟ್ಟ ಗುಣದ ಬಣ್ಣವನ್ನು ಪಡೆಯುವುದು. ಮೊದಲು ಇರವು, ಕೊನೆಗೆ ಇರವು, ಯಾವಾಗಲೂ ಇರವು. ಅದೆಂದಿಗೂ ನಾಶವಾಗುವುದಿಲ್ಲ, ಅದು ಯಾವಾಗಲೂ ಇರುವುದು.

ಆದಕಾರಣ ಎಲ್ಲರಿಗೂ ಭರವಸೆ ಇದೆ. ಯಾರೂ ನಾಶವಾಗುವುದಿಲ್ಲ. ಯಾರೂ ಎಂದೆಂದಿಗೂ ಪತಿತರಾಗಿಯೆ ಉಳಿಯುವುದಿಲ್ಲ. ಜೀವನವೆಂಬುದು ಒಂದು ಕ್ರೀಡಾಭೂಮಿ, ಆಟದಲ್ಲಿ ಎಷ್ಟೇ ಪೆಟ್ಟು ಬೀಳಲಿ, ಎಷ್ಟೇ ಜರ್ಝರಿತರಾಗಲಿ, ಆತ್ಮ\break ಯಾವಾಗಲೂ ಇರುವುದು. ಅದಕ್ಕೆ ಘಾಸಿಯಿಲ್ಲ, ಆ ಅನಂತವೆ ನಾವು.

ವೇದಾಂತಿಯೊಬ್ಬ ಹೀಗೆ ಹಾಡಿದ: 'ನನಗೆ ಅಂಜಿಕೆಯಾಗಲಿ, ಸಂಶಯವಾಗಲಿ ಎಂದಿಗೂ ಇರಲಿಲ್ಲ. ಮೃತ್ಯು ನನಗೆ ಬರಲಿಲ್ಲ. ನನಗೆ ತಂದೆತಾಯಿಗಳಿರಲಿಲ್ಲ. ನಾನು ಎಂದಿಗೂ ಹುಟ್ಟಿಯೇ ಇರಲಿಲ್ಲ. ನನಗೆ ವೈರಿಗಳೆಲ್ಲಿ? ಎಲ್ಲಾ ನಾನೇ, ಸಚ್ಚಿದಾನಂದ ಶಿವೋಽಹಂ ಶಿವೋಽಹಂ. ಕೋಪ, ಕಾಮ, ಅಸೂಯೆ, ಹೀನಭಾವನೆ ಇವಾವುದೂ ನನಗೆ ಇರಲಿಲ್ಲ, ನಾನೇ ಸಚ್ಚಿದಾನಂದಸ್ವರೂಪ, ಶಿವೋಽಹಂ ಶಿವೋಽಹಂ.''

ಇದೇ ಎಲ್ಲ ಭವರೋಗಕ್ಕೆ ಚಿಕಿತ್ಸೆ: ಮೃತ್ಯುವನ್ನು ಗುಣಪಡಿಸುವ ಅಮೃತತ್ವ. ನಾವಿಲ್ಲಿ ಪ್ರಪಂಚದಲ್ಲಿ ಇರುವೆವು. ನಮ್ಮ ಪ್ರಕೃತಿ ಇದನ್ನು ವಿರೋಧಿಸುವುದು. ಆದರೆ ಹೀಗೆ ಹೇಳೋಣ: 'ಓಂ ತತ್ಸತ್, ನನಗೆ ಅಂಜಿಕೆ ಇಲ್ಲ, ಅನುಮಾನವಿಲ್ಲ, ನನಗೆ ಲಿಂಗ ಜಾತಿ ಬಣ್ಣವಿಲ್ಲ. ನನಗೆ ಯಾವ ಜಾತಿ ಇರಬಹುದು? ಯಾವ ಪಂಗಡ ನನ್ನನ್ನು ಅಳವಡಿಸಿಕೊಳ್ಳಬಹುದು? ನಾನು ಪ್ರತಿಯೊಂದರಲ್ಲಿಯೂ ಇರುವೆನು.”

ದೇಹ ಎಷ್ಟು ದಂಗೆ ಎದ್ದರೂ, ಮನಸ್ಸು ಎಷ್ಟು ವಿರೋಧಿಸಿದರೂ, ಅಜ್ಞಾನದ ಕೂಪದಲ್ಲಿ, ಯಮಯಾತನೆಯ ದಾಡೆಯಲ್ಲಿ ನಿರಾಶೆಯ ಆಳದಲ್ಲಿರುವಾಗಲೂ ಇದನ್ನು ಒಂದು ಎರಡು ಮೂರು ಸಲ, ಇಲ್ಲ ಅನೇಕವೇಳೆ ಉಚ್ಚರಿಸಿ. ಜ್ಞಾನಜ್ಯೋತಿ ನಿಧಾನವಾಗಿ ಮೃದುವಾಗಿ ಬರುವುದು. ನಿಸ್ಸಂದೇಹವಾಗಿ ಬರುವುದು.

ಅನೇಕ ವೇಳೆ ನಾನು ಮೃತ್ಯುವಿನ ದಾಡೆಯಲ್ಲಿದ್ದೆ, ಉಪವಾಸವಿದ್ದೆ, ಕಾಲೆಲ್ಲಾ ಗಾಯವಾಗಿತ್ತು, ಸಾಕಾಗಿತ್ತು. ಹಲವು ದಿನಗಳಿಂದ ಅನ್ನಾಹಾರಗಳಿರಲಿಲ್ಲ. ಮುಂದೆ ಒಂದು ಹೆಜ್ಜೆ ಇಡುವುದಕ್ಕೂ ಆಗುತ್ತಿರಲಿಲ್ಲ. ಒಂದು ಮರದ ಕೆಳಗೆ ಸಾಕಾಗಿ ಕುಳಿತುಕೊಳ್ಳುತ್ತಿದ್ದೆ. ಪ್ರಾಣವಾಯು ಹಾರಿಹೋಗುವಂತೆ ಇತ್ತು. ಮಾತನಾಡಲಾರದಾದೆ,\break ಆಲೋಚಿಸಲಾರದ ಸ್ಥಿತಿಯಲ್ಲಿದ್ದೆ. ಆಗಲೂ ಮನಸ್ಸು ಪುನಃ ಈ ಭಾವವನ್ನು ಮೆಲ್ಲುತ್ತಿತ್ತು: “ನನಗೆ ಅಂಜಿಕೆ, ಮರಣಗಳಿಲ್ಲ, ನನಗೆ ಹಸಿವು ಬಾಯಾರಿಕೆ ಇಲ್ಲ. ನಾನೆ ಅದು! ನಾನೆ ಅದು! ಇಡೀ ಪ್ರಕೃತಿ ನನ್ನನ್ನು ನಾಶಮಾಡಲಾರದು. ಇದು ನನ್ನ ಸೇವಕ, ಹೇ, ದೇವದೇವನೆ, ಸರ್ವೇಶ್ವರನೆ ನಿನ್ನ ಶಕ್ತಿಯನ್ನು ವ್ಯಕ್ತಗೊಳಿಸು, ಕಳೆದುಕೊಂಡ ಸ್ವರಾಜ್ಯವನ್ನು ಪುನಃ ಪಡೆ, ಜಾಗ್ರತನಾಗು, ನಡೆ, ನಿಲ್ಲಬೇಡ!?” ಆಗ ಪುನಃ ಜೀವ ಬಂದಂತೆ ಏಳುತ್ತಿದ್ದೆ. ನೋಡಿ ಇಂದಿಗೂ ನಾನು ಜೀವಿಸಿರುವೆನು. ಯಾವಾಗಲಾದರೂ ಅಜ್ಞಾನ ಕವಿದರೆ ಸತ್ಯವನ್ನು ವ್ಯಕ್ತಗೊಳಿಸಿ, ಆಗ ನಿಮಗೆ ವಿರೋಧವಾಗಿರುವುದೆಲ್ಲ ಮಾಯವಾಗಲೇಬೇಕು. ಏಕೆಂದರೆ ಎಷ್ಟಾದರೂ ಇದೆಲ್ಲ ಒಂದು ಕನಸು. ಕಷ್ಟಗಳು ಮೇರುಪ್ರಾಯವಾಗಿದ್ದರೂ, ಎಲ್ಲಾ ನಿರಾಶಾಸೂಚಕವಾಗಿ ಭಯಂಕರವಾಗಿದ್ದರೂ\break ಇವೆಲ್ಲ ಮಾಯೆ. ಅಂಜಬೇಡಿ. ಅದು ತೋಲಗುವುದು. ಅದನ್ನು ಹಿಸುಕಿ, ಅದು ನಾಶವಾಗುವುದು. ತುಳಿಯಿರಿ ಅದು ಧ್ವಂಸವಾಗುವುದು. ಎಷ್ಟು ವೇಳೆ ನೀವು ಸೋಲುವಿರಿ ಎಂದು ನೋಡಬೇಡಿ, ಚಿಂತೆ ಇಲ್ಲ. ಕಾಲ ಅನಂತವಾದುದು. ಮುಂದೆ ಹೋಗಿ, ಪುನಃ ಪುನಃ ನಿಮ್ಮ ಶಕ್ತಿಯನ್ನು ವ್ಯಕ್ತಗೊಳಿಸಿ, ಜ್ಯೋತಿ ಬಂದೇ ಬರುವುದು. ಹುಟ್ಟಿದವರನ್ನೆಲ್ಲ ನೀವು ಪ್ರಾರ್ಥಿಸಬಹುದು, ಆದರೆ ಯಾರು ನಿಮ್ಮ ಸಹಾಯಕ್ಕೆ ಬರುವರು? ಯಾರೂ ತಪ್ಪಿಸಿಕೊಳ್ಳಲಾರದ ಮೃತ್ಯುವಿನಿಂದ ಹೇಗೆ ಪಾರಾಗುವಿರಿ? ನಿಮ್ಮಿಂದ ನೀವೇ ಉದ್ದಾರವಾಗಬೇಕು. ಸ್ನೇಹಿತನೆ, ಯಾರೂ ನಿನಗೆ ಸಹಾಯಮಾಡಲಾರರು. ನಿನಗೆ ನೀನೇ ದೊಡ್ಡ ಶತ್ರು, ನಿನಗೆ ನೀನೇ ದೊಡ್ಡ ಮಿತ್ರ. ಹಾಗಾದರೆ ಆತ್ಮನನ್ನು ನೀನು ದೃಢವಾಗಿ ಹಿಡಿದುಕೊ. ಎದ್ದು ನಿಲ್ಲು. ಅಂಜಬೇಕಾಗಿಲ್ಲ. ಎಲ್ಲ ದುಃಖ ದುರ್ಬಲತೆಗಳಲ್ಲಿಯೂ ನಿನ್ನ ಆತ್ಮ ವ್ಯಕ್ತವಾಗಲಿ. ಮೊದಲು ಅಸ್ಪಷ್ಟವಾಗಿ ಅಗೋಚರವಾಗಿದ್ದರೂ ಚಿಂತೆಯಿಲ್ಲ. ಕ್ರಮೇಣ ನಿನಗೆ ಧೈರ್ಯ ಬರುವುದು. ಸಿಂಹದಂತೆ "ಹರಿಃ ಓಂ ತತ್ ಸತ್'' ಎಂದು ಗರ್ಜಿಸುವೆ. “ನಾನು ಪುರುಷನೂ ಅಲ್ಲ ಸ್ತ್ರೀಯೂ ಅಲ್ಲ, ದೇವನೂ ಅಲ್ಲ, ಅಸುರನೂ ಅಲ್ಲ. ನಾನು ಯಾವ ಬಗೆಯ ತರು ಲತೆ ಪ್ರಾಣಿಯೂ ಅಲ್ಲ. ನಾನು ಶ‍್ರೀಮಂತನೂ ಅಲ್ಲ, ಬಡವನೂ ಅಲ್ಲ. ನಾನು ಬುದ್ದಿವಂತನೂ ಅಲ್ಲ, ದಡ್ಡನೂ ಅಲ್ಲ. ಇವೆಲ್ಲ ಬಹಳ ಅಲ್ಪ, ನನ್ನ ನೈಜಸ್ಥಿತಿಗೆ ಹೋಲಿಸಿ ನೋಡಿದರೆ, ನಾನೇ ಅದು, ನಾನೇ ಅದು. ಸೂರ್ಯಚಂದ್ರತಾರೆಗಳನ್ನು ನೋಡಿ, ಅವುಗಳಲ್ಲಿ ಪ್ರಕಾಶಿಸುತ್ತಿರುವ ಕಾಂತಿಯೇ ನಾನು, ನಾನೇ ಬೆಂಕಿಯ ಸೌಂದರ್ಯ, ಪ್ರಪಂಚದ ಶಕ್ತಿ. ನಾನೇ ಅವನು, ನಾನೇ ಅವನು!”

ಹೀಗೆ ನಮ್ಮ ಋಷಿಗಳೊಬ್ಬರು ಹೇಳಿರುವರು: “ಯಾರು ನಾನು ಅಲ್ಪ ಎಂದು ಭಾವಿಸುವರೊ ಅವರು ಒಂದು ತಪ್ಪನ್ನು ಮಾಡುವರು. ಏಕೆಂದರೆ ಇರುವುದೊಂದೆ ಆತ್ಮ. ನಾನು ಸೂರ್ಯ ಇದೆ ಎನ್ನುವುದರಿಂದ ಅದು ಇದೆ; ನಾನು ಪ್ರಪಂಚವಿದೆ ಎನ್ನುವುದರಿಂದ ಅದು ಇದೆ. ನಾನಿಲ್ಲದೆ ಅವು ಇರಲಾರವು. ನಾನೆ ಸಚ್ಚಿದಾನಂದ ಸ್ವರೂಪ, ನಾನಾವಾಗಲೂ ಶಾಂತ ಶಿವ ಸುಂದರ. ನೋಡಿ, ಸೂರ್ಯನೇ ನಮ್ಮ ದೃಷ್ಟಿಗೆ ಕಾರಣ. ಆದರೂ ಯಾರ ಕಣ್ಣಿನ ದೋಷವೂ ಅದಕ್ಕೆ ಸೋಂಕುವುದಿಲ್ಲ, ಇದರಂತೆಯೆ ನಾನು. ನಾನು ಎಲ್ಲ ಅವಯವಗಳ ಮೂಲಕ, ಎಲ್ಲದರ ಮೂಲಕ ಕೆಲಸಮಾಡುತ್ತಿರುವೆನು. ಆದರೂ ಕರ್ಮಗಳ ಗುಣದೋಷಗಳು ನನಗೆ ಸೋಂಕುವುದಿಲ್ಲ. ನನಗೆ ನಿಯಮವಿಲ್ಲ, ಕರ್ಮವಿಲ್ಲ. ಕರ್ಮನಿಯಮ ನನ್ನ ಸ್ವಾಧೀನದಲ್ಲಿದೆ. ನಾನು ಹಿಂದೆಯೂ ಇದ್ದೆ, ಮುಂದೆಯೂ ಇರುವೆನು.

“ನನ್ನ ನಿಜವಾದ ಸುಖ ಎಂದಿಗೂ ಪ್ರಾಪಂಚಿಕ ವಸ್ತುಗಳಲ್ಲಿ ಇರಲಿಲ್ಲ. ಗಂಡ ಹೆಂಡತಿ, ಮಕ್ಕಳಲ್ಲಿ ಇರಲಿಲ್ಲ. ನಾನು ಅನಂತಾಕಾಶದಂತೆ. ಹಲವು ಬಣ್ಣದ ಮುಗಿಲುಗಳು ಹಾದು ಹೋಗುವುವು, ಕ್ಷಣಕಾಲ ಆಡಿ ಅನಂತರ ಚಲಿಸುವುವು. ಅದರ ಹಿಂದೆ ಅವಿಕಾರಿಯಾದ ನೀಲಿಯಾಗಸ ಒಂದೆ ಇರುವುದು. ಕ್ಷಣಕಾಲ ಸುಖದುಃಖ ಒಳ್ಳೆಯದು ಕೆಟ್ಟದ್ದು ಆತ್ಮವನ್ನು ಕವಿಯಬಹುದು. ಆದರೆ ನಾನು ಯಾವಾಗಲೂ ಇದ್ದೇನೆ. ಅವು ಸವಿಕಾರವಾದುದರಿಂದ ಹೊರಟುಹೋಗುತ್ತವೆ. ನಾನು ಅವಿಕಾರಿ; ಅದಕ್ಕೆ ಪ್ರಕಾಶಿಸುತ್ತಿರುವುದು. ದುಃಖ ಬಂದರೆ ಅದು ಸಾಂತವೆಂದು ನನಗೆ ಗೊತ್ತಿದೆ. ಆದಕಾರಣ ಅದು ನಾಶವಾಗಬೇಕು. ದೋಷ ಬಂದರೆ ಅದು ಸಾಂತವೆಂದು ಗೊತ್ತಿದೆ. ಆದಕಾರಣ ಅದು ಕೊನೆಗಾಣಬೇಕು. ನಾನೊಬ್ಬನೇ ಅನಂತ, ಅಸ್ಪರ್ಶ, ನಾನೇ ಸರ್ವವ್ಯಾಪಿ, ಆದ್ಯಂತರಹಿತ ಅವಿಕಾರಿಯಾದ ಆತ್ಮ.''

ಎದುರಿಗಿರುವ ಈ ಅಮೃತವನ್ನು ಹೀರೋಣ. ಅಮೃತತ್ವವೆಲ್ಲ ಇದರಲ್ಲಿ ದೊರಕುವುದು, ಅವಿಕಾರಿಯಾದುದೆಲ್ಲ ಇದರಲ್ಲಿ ದೊರಕುವುದು. ಅಂಜಬೇಡಿ! ನಾವು ಪಾಪಿಗಳು, ಸಾಂತರು, ನಾವು ಸಾಯುವೆವು ಎಂಬುದನ್ನು ನಂಬಬೇಡಿ. ಇದು ಸತ್ಯವಲ್ಲ.

"ಇದನ್ನು ಮೊದಲು ಕೇಳಬೇಕು, ಮನನ ಮಾಡಬೇಕು, ಧ್ಯಾನ ಮಾಡಬೇಕು.'' ಕೈಗಳು ಕೆಲಸಮಾಡುತ್ತಿರುವಾಗ "ನಾನೇ ಅವನು ನಾನೇ ಅವನು" ಎಂದು ಉಚ್ಚರಿಸುತ್ತಿರಬೇಕು. ಈ ಭಾವನೆ ನಿಮ್ಮ ರಕ್ತಗತವಾಗಿ ಅಸ್ಥಿಗತವಾಗಿ ಮಾಂಸಗತವಾಗುವವರೆಗೆ, ಅಲ್ಪತೆ, ದುರ್ಬಲತೆ, ದುಃಖದೋಷಗಳೆಂಬ ಭಯಾನಕ ಸ್ವಪ್ನ ವಿಶೇಷವಾಗಿ ಮಾಯವಾಗುವವರೆಗೆ ಇದನ್ನು ಕುರಿತು ಆಲೋಚಿಸಿ, ಇದನ್ನೇ ಕನಸುಕಟ್ಟಿ. ಆಗ ಒಂದು ಕ್ಷಣವಾದರೂ ಸತ್ಯ ನಿಮ್ಮಿಂದ ಕಣ್ಮರೆಯಾಗುವುದಿಲ್ಲ.

