
\chapter[ಧರ್ಮದ ಅನುಷ್ಠಾನ]{ಧರ್ಮದ ಅನುಷ್ಠಾನ\protect\footnote{\engfoot{C.W, Vol. IV, P. 238}}}

\begin{center}
(೧೯೦೦ರ ಏಪ್ರಿಲ್‌ನಲ್ಲಿ ಕ್ಯಾಲಿಫೋರ್ನಿಯಾದ ಅಲಮೇಡಾದಲ್ಲಿ ನೀಡಿದ ಉಪನ್ಯಾಸ)
\end{center}

ನಾವು ಹಲವು ಶಾಸ್ತ್ರಗಳನ್ನು ಓದುತ್ತೇವೆ, ಹಲವು ಗ್ರಂಥಗಳನ್ನು ಓದುತ್ತೇವೆ, ಬಾಲ್ಯದಿಂದಲೂ ಹಲವು ಭಾವನೆಗಳನ್ನು ಪಡೆಯುತ್ತಿರುತ್ತೇವೆ, ಮತ್ತು ಅವುಗಳನ್ನು ಪದೇ ಪದೇ ಬದಲಾಯಿಸುತ್ತಿರುತ್ತೇವೆ. ಧರ್ಮದ ಸಿದ್ದಾಂತವನ್ನು ನಾವು ಅರ್ಥ ಮಾಡಿಕೊಳ್ಳುತ್ತೇವೆ. ಧರ್ಮದ ಅನುಷ್ಠಾನವನ್ನೂ ಅರ್ಥಮಾಡಿಕೊಳ್ಳಬಲ್ಲೆವು ಎಂದು ಭಾವಿಸುತ್ತೇವೆ. ಈಗ ನಾನು ನನ್ನ ದೃಷ್ಟಿಯಲ್ಲಿ ಅನುಷ್ಠಾನ ಧರ್ಮವೆಂದರೆ ಏನು ಎಂಬುದನ್ನು ವಿವರಿಸುತ್ತೇನೆ.

ನಾವು ಸುತ್ತಮುತ್ತಲೂ ಅನುಷ್ಠಾನ ಧರ್ಮದ ವಿಷಯವನ್ನು ಕೇಳುತ್ತಿರುವೆವು. ಅವುಗಳನ್ನೆಲ್ಲಾ ವಿಶ್ಲೇಷಿಸಿದರೆ ಅದರಲ್ಲಿ ಈ ಒಂದು ಭಾವನೆ ದೊರಕುವುದು: ನೆರೆಯವರಿಗೆ ದಯೆತೋರಿ ಎಂಬುದು. ಧರ್ಮ ಎಂದರೆ ಇಷ್ಟೇಯೇ? ನಾವು ಪ್ರತಿದಿನ ಈ ದೇಶದಲ್ಲಿ ಅನುಷ್ಠಾನಿಕ ಕ್ರೈಸ್ತ ಧರ್ಮದ ವಿಷಯವನ್ನು ಕೇಳುತ್ತಿರುವೆವು – ಯಾರೊ ಇತರರಿಗೆ ಒಳ್ಳೆಯದನ್ನು ಮಾಡಿರುವರು. ಧರ್ಮವೆಂದರೆ ಇಷ್ಟೇ ಏನು?

ಜೀವನದ ಗುರಿ ಏನು? ಈ ಪ್ರಪಂಚವೇ ನಮ್ಮ ಜೀವನದ ಗುರಿಯೆ, ಇದಕ್ಕಿಂತ ಹೆಚ್ಚು ಏನೂ ಇಲ್ಲವೆ? ನಾವು ಈಗಿರುವ ಸ್ಥಿತಿಯಲ್ಲೇ ಇರಬೇಕೆ? ಇದಕ್ಕಿಂತ ಉತ್ತಮರಾಗಕೂಡದೆ? ಮಾನವನು ಸ್ವಲ್ಪವೂ ಘರ್ಷಣೆ ಇಲ್ಲದೆ ಹೋಗುವಂತಹ ಒಂದು ಯಂತ್ರ ಮಾತ್ರವೆ? ಅವನು ಇಂದು ಅನುಭವಿಸುವ ದುಃಖಗಳಷ್ಟೇಯೆ ಅವನ ಪಾಲಿಗೆ? ಅವನಿಗೆ ಇನ್ನೇನೂ ಬೇಡವೆ?

ಬಹುಪಾಲು ಧರ್ಮಗಳ ಪರಮಾದರ್ಶವೇ ಈ ಜಗತ್ತು. ಹಲವು ಜನರು, ಮುಂದೆ ಒಂದು ಕಾಲ ಬರುವುದು, ಆಗ ರೋಗರುಜಿನ, ದಾರಿ ಮತ್ತು ಯಾವ ವಿಧವಾದ ದುಃಖವೂ ಇರುವುದಿಲ್ಲ ಎಂದು ಯೋಚಿಸುತ್ತಿರುವರು. ಆಗ ಎಲ್ಲರೂ ಎಲ್ಲಾ ವಿಧದಲ್ಲೂ ಸುಖವಾಗಿರಬಹುದು. ಅನುಷ್ಠಾನ ಧರ್ಮ ಎಂದರೆ ಸುಮ್ಮನೆ “ರಸ್ತೆಯನ್ನು ಗುಡಿಸಿ! ಅದನ್ನು ಶುಚಿಮಾಡಿ!?” ಎಂದಂತೆ ಆಗುವುದು. ಎಲ್ಲರೂ ಇದನ್ನು ಆಶಿಸುತ್ತಿರುವುದನ್ನು ನಾವು ನೋಡುತ್ತಿರುವೆವು.

ಸುಖವನ್ನು ಅನುಭವಿಸುವುದೇ ಜೀವನದ ಪರಮಗುರಿಯೆ? ಹಾಗಿದ್ದರೆ ಮಾನವನಾಗಿ ಹುಟ್ಟಿ ಬಂದಿದ್ದು ಒಂದು ದೊಡ್ಡ ದೊಷವಾಗುವುದು. ಯಾವ ಮಾನವನು ಒಂದು ನಾಯಿ ಅಥವಾ ಬೆಕ್ಕಿಗಿಂತ ಚೆನ್ನಾಗಿ ಊಟ ಮಾಡಬಲ್ಲ? ಒಂದು ಮೃಗಾಲಯಕ್ಕೆ ಹೋಗಿ ಮೃಗಗಳು ಮೂಳೆಯಿಂದ ಮಾಂಸವನ್ನು ಹೇಗೆ ಕೀಳುತ್ತವೆ ಎಂಬುದನ್ನು ನೋಡಿ. ನೀವು ವಿಕಾಸದ ಏಣಿಯಲ್ಲಿ ಹಿಂದೆ ಹೋಗಿ ಪುನಃ ಒಂದು ಹಕ್ಕಿ ಆಗಿ, ಮನುಷ್ಯನಾಗಿ ಹುಟ್ಟಿದ್ದು ಆಗ ಎಂಥ ದೋಷವಾಗುವುದು! ಇಷ್ಟು ವರುಷಗಳು ಬದುಕಿ ವ್ಯರ್ಥವಾಯಿತು. ನೂರಾರು ವರುಷಗಳು ಹೋರಾಡಿದ್ದು ಕೇವಲ ಇಂದ್ರಿಯ ಸುಖಗಳನ್ನು ಅನುಭವಿಸುವುದಕ್ಕೆ ಮಾತ್ರ ಆಯಿತು.

ಸಾಧಾರಣವಾಗಿ ಅನುಷ್ಠಾನ ಧರ್ಮ ಎಂದರೆ ಏನು ಎಂಬುದನ್ನು ಗಮನಿಸಿ. ಇದರಿಂದ ಏನು ಪ್ರಯೋಜನವಾಗುವುದು? ದಾನವೇನೊ ದೊಡ್ಡದು. ಆದರೆ ಇದೇ ಪರಮಾವಧಿ ಎಂದರೆ ನೀವು ಜಡವಾದಿಗಳಾಗುವ ಸಂಭವವಿದೆ. ಇದು ಧರ್ಮವಲ್ಲ. ಇದು ಚಾರ್ವಾಕ ಮತಕ್ಕಿಂತ ಉತ್ತಮವಾದುದೇನೂ ಅಲ್ಲ. ಅದಕ್ಕಿಂತ ಸ್ವಲ್ಪ ಕಡಮೆ ಅಷ್ಟೆ. ಕ್ರೈಸ್ತರೆ, ನಿಮ್ಮ ಬೈಬಲ್ಲಿನಲ್ಲಿ ನೆರೆಯವರಿಗೆ ಉಪಕಾರ ಮಾಡುವುದಕ್ಕಾಗಿ ಆಸ್ಪತ್ರೆಗಳನ್ನು ಕಟ್ಟುವುದು ಮುಂತಾದುವುಗಳಲ್ಲದೆ ಬೇರೇನೂ ಅಲ್ಲಿ ನಿಮಗೆ ಕಾಣುವುದಿಲ್ಲವೆ? ಒಬ್ಬ ವರ್ತಕ ಜೀಸಸ್ ಹೇಗೆ ಅಂಗಡಿಯನ್ನು ನೋಡಿಕೊಳ್ಳುತ್ತಿದ್ದ ಎಂದು ಆಲೋಚಿಸುತ್ತಾನೆ. ಜೀಸಸ್ ಯಾವ ಅಂಗಡಿಯನ್ನೂ ಇಡುತ್ತಿರಲಿಲ್ಲ, ಸೆಲೂನ್ ಇಡುತ್ತಿರಲಿಲ್ಲ, ಒಂದು ವೃತ್ತಪತ್ರಿಕೆಯನ್ನೂ ಸಂಪಾದಿಸುತ್ತಿರಲಿಲ್ಲ. ಇಂತಹ ಅನುಷ್ಠಾನ ಧರ್ಮವೇನೊ ಒಳ್ಳೆಯದು; ಕೆಟ್ಟದ್ದಲ್ಲ. ಆದರೆ ಇದು ಕೇವಲ ಶಿಶು ವಿಹಾರದ ಧರ್ಮ. ಇದು ನಮ್ಮನ್ನು ಮುಂದಕ್ಕೆ ಒಯ್ಯಲಾರದು. ನೀವು ದೇವರನ್ನು ನಂಬಿದರೆ, ನೀವು ಕ್ರೈಸ್ತರಾಗಿದ್ದು ಪ್ರತಿದಿನ "ನಿನ್ನ ಇಚ್ಛೆಯಂತಾಗಲಿ” ಎಂದು ಹೇಳಿದರೆ, ಅದರ ಅರ್ಥವೇನು ಎಂಬುದನ್ನು ಕುರಿತು ಯೋಚಿಸಿ ನೋಡಿ. ನೀವು ಪ್ರತಿಕ್ಷಣವೂ ನಿನ್ನ ಇಚ್ಛೆಯಂತಾಗಲಿ ಎಂದು ಹೇಳುವಿರಿ. ಆದರೆ ನಿಮ್ಮ ಮನಸ್ಸಿನಲ್ಲಿ 'ದೇವರೇ, ನೀನು ನನ್ನ ಇಚ್ಛೆಯಂತೆ ಮಾಡು'', ಎಂದುಕೊಳ್ಳುವಿರಿ. ಅನಂತಾತ್ಮನಾದ ಭಗವಂತನು ತನ್ನ ಇಚ್ಛೆಯಂತೆ ಮಾಡುತ್ತಿರುವನು. ನಾನು ನೀವು ಅದನ್ನು ಸರಿಪಡಿಸುವವರೆ? ವಿಶ್ವನಿರ್ಮಾಪಕನು ಕೇವಲ ಬಡಗಿಗಳಿಂದ ಕಲಿಯಬೇಕೆ? ಅವನು ಈ ಪ್ರಪಂಚವನ್ನು ಒಂದು ಕೊಳಕು ಕೊಂಪೆಯನ್ನಾಗಿ ಮಾಡಿದ್ದನು, ನೀವು ಅದನ್ನು ಒಂದು ಸುಂದರ ಸ್ಥಳವನ್ನಾಗಿ ಮಾಡಬೇಕಲ್ಲವೆ?

ಇದೆಲ್ಲದರ ಗುರಿ ಏನು? ಇಂದ್ರಿಯಗಳೇ ಗುರಿಯಾಗಬಲ್ಲವೆ? ಕೇವಲ ಸುಖವನ್ನು ಅನುಭವಿಸುವುದೇ ಜೀವನದ ಗುರಿಯಾಗಬಲ್ಲದೆ? ಈ ಜೀವನವೆ ಆತ್ಮನ ಗುರಿಯಾಗಬಲ್ಲದೆ? ಅದೇನಾದರೂ ಹಾಗೆ ಆದರೆ ಈ ಕ್ಷಣ ಸಾಯುವುದು ಮೇಲು. ಈ ಜೀವನ ಬೇಕಾಗಿಲ್ಲ. ಇದು ಮನುಷ್ಯನ ಹಣೆಯಬರಹವಾದರೆ, ಅವನು ಕೇವಲ ಒಂದು ಪರಿಪೂರ್ಣವಾದ ಯಂತ್ರ ಆಗುವ ಹಾಗೆ ಇದ್ದರೆ, ನಾವು ಕಲ್ಲುಗಿಡಮರಗಳ ಗುಂಪಿಗೆ ಹೋದಂತೆ ಆಗುವುದು. ನೀವು ಎಂದಾದರೂ ದನ ಸುಳ್ಳು ಹೇಳುವುದನ್ನು ಕೇಳಿರುವಿರಾ? ಅವು ಪರಿಪೂರ್ಣವಾದ ಯಂತ್ರಗಳು, ಯಾವ ತಪ್ಪನ್ನೂ ಮಾಡುವುದಿಲ್ಲ. ಎಲ್ಲಾ ಸಿದ್ಧವಾಗಿರುವ ಪ್ರಪಂಚದಲ್ಲಿ ಅವು ಇವೆ.

ಇವೆಲ್ಲ ಅನುಷ್ಠಾನ ಧರ್ಮವಾಗದೇ ಇದ್ದರೆ ಧರ್ಮದ ಆದರ್ಶವೇನು? ಈ ಜಗತ್ತೇ ನಮ್ಮ ಆದರ್ಶವಾಗಲಾರದು. ನಾವು ಇಲ್ಲಿರುವುದು ಏತಕ್ಕೆ? ನಾವು ಇಲ್ಲಿರುವುದು ಸ್ವಾತಂತ್ರ್ಯ ಮತ್ತು ಜ್ಞಾನವನ್ನು ಪಡೆಯುವುದಕ್ಕಾಗಿ. ಮುಕ್ತರಾಗುವುದಕ್ಕಾಗಿ ನಮಗೆ ಜ್ಞಾನ ಬೇಕಾಗಿದೆ. ಅದೇ ನಮ್ಮ ಜೀವನ. ಮುಕ್ತಿಗಾಗಿ ವಿಶ್ವವೇ ಹಾತೊರೆಯುತ್ತಿದೆ. ಬೀಜದಿಂದ ಒಂದು ಸಸಿ ಹುಟ್ಟುವುದು. ನೆಲದಿಂದ ಅದು ಮೇಲೆದ್ದು ಆಗಸದ ಕಡೆಗೆ ಏರುತ್ತಿರುವುದು. ಇದರ ಕಾರಣವೇನು? ಸೂರ್ಯ ಭೂಮಿಗೆ ಏನನ್ನು ಕೊಡುತ್ತಿರುವನು? ನಿಮ್ಮ ಜೀವನವೇನು? ಅದೇ ಮುಕ್ತರಾಗಲು ನಾವು ನಡೆಸುವ ಹೋರಾಟ. ಪ್ರಕೃತಿ ನಮ್ಮ ಸುತ್ತಲೂ ನಮ್ಮನ್ನು ತುಳಿಯುವುದಕ್ಕೆ ಪ್ರಯತ್ನಿಸುತ್ತಿದೆ. ಆತ್ಮನಾದರೂ ತಾನು ಮುಕ್ತನಾಗಬೇಕೆಂದು ಪ್ರಯತ್ನಿಸುತ್ತಿರುವನು. ಪ್ರಕೃತಿಯೊಂದಿಗೆ ಹೋರಾಟದಲ್ಲಿ ಹಲವು ವಸ್ತುಗಳು ನುಚ್ಚುನೂರಾಗಿ ಹೋಗುವುವು. ಅದೇ ನಿಮ್ಮ ನಿಜವಾದ ದುಃಖಕ್ಕೆ ಕಾರಣ. ಜೀವನದ ಸಮರಕ್ಷೇತ್ರದಲ್ಲಿ ಬೇಕಾದಷ್ಟು ಧೂಳು ಕೊಳೆ ಏಳಬೇಕಾಗುವುದು. ಪ್ರಕೃತಿ ನಾನು ನಿನ್ನನ್ನು ಗೆಲ್ಲುತ್ತೇನೆ ಎನ್ನುವುದು. ಜೀವನಾದರೆ ನಾನು ಜಯಶಾಲಿಯಾಗಬೇಕು ಎನ್ನುವುದು. ಪ್ರಕೃತಿ ಸ್ವಲ್ಪ ತಾಳು, ನಿಮ್ಮನ್ನು ಸುಮ್ಮನಿರಿಸಲು ನಿಮಗೆ ಸ್ವಲ್ಪ ಭೋಗವಸ್ತುವನ್ನು ಕೊಡುವೆನು ಎನ್ನುವುದು. ಜೀವ ಸ್ವಲ್ಪ ಸುಖವನ್ನು ಅನುಭವಿಸಿ ಭ್ರಾಂತಿಯಲ್ಲಿದ್ದು ಪುನಃ ಸ್ವಾತಂತ್ರ್ಯಕ್ಕಾಗಿ ಗೋಳಿಡುವುದು. ಪ್ರತಿಯೊಂದು ಜೀವಿಯ ಹೃದಯಾಂತರಾಳದಿಂದಲೂ ಎಲ್ಲಾ ಕಾಲದಲ್ಲಿಯೂ ಏಳುತ್ತಿರುವ ಕ್ರಂದನವನ್ನು ಗಮನಿಸಿರುವಿರಾ? ದಾರಿದ್ರ್ಯದಿಂದ ನಾವು ಮೋಸಹೋಗುತ್ತೇವೆ. ನಾವು ಶ‍್ರೀಮಂತರಾಗಿ ಐಶ್ವರ್ಯದಿಂದ ಮೋಸ ಹೊಗುತ್ತೇವೆ. ನಾವು ಅಜ್ಞರು. ನಾವು ಸ್ವಲ್ಪ ಬರೆಯುವುದು ಓದುವುದನ್ನು ಕಲಿತು ಜ್ಞಾನದಿಂದ ಮೋಸಹೋಗುತ್ತೇವೆ. ಯಾರೂ ತೃಪ್ತರಾಗಿಲ್ಲ. ಇದೇ ನಮ್ಮ ದುಃಖಕ್ಕೆಲ್ಲಾ ಕಾರಣ. ಇದೇ ನಮ್ಮ ಧನ್ಯತೆಗೂ ಕಾರಣ. ಇದೇ ಸರಿಯಾದ ಶುಭಸೂಚನೆ. ಈ ಪ್ರಪಂಚದಿಂದ ನೀವು ಹೇಗೆ ತೃಪ್ತರಾಗಬಲ್ಲಿರಿ? ನಾಳೆ ಈ ಪ್ರಪಂಚವೇ ಸ್ವರ್ಗವಾದರೆ "ಇದನ್ನು ತೆಗೆದುಕೊಂಡು ಬೇರೆ ಏನನ್ನಾದರೂ ಕೊಡಿ'' ಎಂದು ಕೇಳುತ್ತೇವೆ.

ಮಾನವನ ಅನಂತಾತ್ಮವು, ಅನಂತವಲ್ಲದೆ ಬೇರೆ ಯಾವುದರಿಂದಲೂ ತೃಪ್ತವಾಗಲಾರದು. ಅನಂತ ಆಸೆಗಳು ಅನಂತ ಜ್ಞಾನದಿಂದ ಮಾತ್ರ ತೃಪ್ತವಾಗಬಲ್ಲವು. ಬೇರೆ ಯಾವುದರಿಂದಲೂ ಅಲ್ಲ. ಜಗತ್ತುಗಳು ಬಂದು ಹೋಗುತ್ತವೆ. ಆದರಿಂದೇನು? ಆತ್ಮ ಇರುವುದು, ಅದು ಯಾವಾಗಲೂ ವಿಕಾಸವಾಗುತ್ತಿರುವುದು. ಜಗತ್ತುಗಳು ಆತ್ಮನಲ್ಲಿಗೆ ಬರಬೇಕಾಗಿವೆ. ಹನಿಗಳು ಸಾಗರದಲ್ಲಿ ಹೇಗೆ ಮಾಯವಾಗಿ ಹೋಗುವುವೋ ಹಾಗೆ ಜಗತ್ತುಗಳು ಆತ್ಮನಲ್ಲಿ ಮಾಯವಾಗಿ ಹೋಗುವುವು. ಈ ಜಗತ್ತೇ ಹೇಗೆ ಆತ್ಮನಿಗೆ ಗುರಿಯಾಗಬಲ್ಲದು? ನಮಗೆ ವ್ಯವಹಾರ ಜ್ಞಾನವಿದ್ದರೆ ಈ ಜಗತ್ತಿನಲ್ಲಿ ನಾವು ತೃಪ್ತರಾಗಲಾರೆವು. ಎಲ್ಲಾ ಕಾಲದ ಎಲ್ಲಾ ದೇಶದ ಕವಿಗಳೂ, ಇದೇ ಸರ್ವಸ್ವ, ಇದರಲ್ಲೇ ತೃಪ್ತರಾಗಿ ಎಂದು ಸಾರಿದ್ದರೂ ನಾವು ತೃಪ್ತರಾಗಲಾರೆವು. ಇದುವರೆಗೆ ಯಾರೂ ತೃಪ್ತರಾಗಿಲ್ಲ. ಲಕ್ಷಾಂತರ ದೇವದೂತರು 'ನೀವು ಈಗಿರುವ ಸ್ಥಿತಿಯಲ್ಲಿ ತೃಪ್ತರಾಗಿ” ಎಂದು ಹೇಳಿದ್ದರೂ, ಕವಿಗಳು ಈ ವಿಷಯವಾಗಿ ಹಾಡಿದ್ದರೂ, ನಾವಂತೂ ತೃಪ್ತರಾಗುವುದಿಲ್ಲ. ತೆಪ್ಪಗಿರು, ಇರುವುದರಲ್ಲಿ ತೃಪ್ತನಾಗು ಎಂದು ನಮಗೆ ನಾವೇ ಎಷ್ಟೋ ವೇಳೆ ಹೇಳಿಕೊಂಡಿರುವೆವು. ಆದರೂ ನಾವು ತೃಪ್ತರಾಗಿಲ್ಲ. ಇದು ವಿಧಾತನ ಇಚ್ಛೆ. ನನ್ನ ಆತ್ಮವನ್ನು ತೃಪ್ತಿಪಡಿಸುವಂತಹುದು ಈ ಲೋಕದಲ್ಲಿ ಇಲ್ಲ, ಮೇಲಿರುವ ಸ್ವರ್ಗದಲ್ಲಿಯೂ ಇಲ್ಲ, ಕೆಳಗಿನ ಪಾತಾಳ ಲೋಕದಲ್ಲಿಯೂ ಇಲ್ಲ. ನನ್ನ ಜೀವನ ಆಸೆಯ ಎದುರಿಗೆ ತಾರೆ ನಿಹಾರಿಕೆಗಳು ಜಗತ್ತುಗಳು, ಸ್ವರ್ಗಮರ್ತ್ಯ ಲೋಕಗಳು, ಇಡೀ ವಿಶ್ವವೇ ಒಂದು ಭಯಂಕರವಾದ ರೋಗದಂತೆ. ಅದಕ್ಕಿಂತ ಹೆಚ್ಚೇನೂ ಇಲ್ಲ. ಇದೇ ಅರ್ಥ. ಈ ಜಗತ್ತಿನ ಎಲ್ಲವೂ ದುಃಖಮಯ, ನೀವು ಅದರ ಅಂತರಾರ್ಥವನ್ನು ತಿಳಿದುಕೊಂಡಾಗ ಮಾತ್ರ, ಅದರ ಉದ್ದೇಶವನ್ನು ಅರಿತಾಗ ಮಾತ್ರ ಅದು ಹಾಗಲ್ಲ. ಜಗತ್ತಿನಲ್ಲಿರುವ ಪ್ರತಿಯೊಂದು ಕಣವೂ ಪ್ರಕೃತಿಯಲ್ಲಿ ಸ್ವಾತಂತ್ರ್ಯಕ್ಕಾಗಿ ಕ್ರಂದಿಸುತ್ತಿದೆ.

ಹಾಗಾದರೆ ಅನುಷ್ಠಾನ ಧರ್ಮ ಯಾವುದು? ಅದೇ, ಸ್ವಾತಂತ್ರ್ಯದ ಸ್ಥಿತಿಗೆ ಹೋಗುವುದು, ಮುಕ್ತರಾಗುವುದು. ಈ ಜಗತ್ತು ಆ ಗುರಿಯೆಡೆಗೆ ಹೋಗಲು ನಮಗೆ ಸಹಾಯಮಾಡಿದರೆ ಒಳ್ಳೆಯದು. ಅದಲ್ಲದೆ, ಆಗಲೆ ನಮಗೆ ಇರುವ ಸಾವಿರಾರು ಬಂಧನಗಳಿಗೆ ಮತ್ತೊಂದನ್ನು ಸೇರಿಸಿದರೆ ಅದು ಪಾಪಮಯವಾಗುವುದು. ಆಸ್ತಿ, ಪಾಂಡಿತ್ಯ, ಸೌಂದರ್ಯ ಮತ್ತೆ ಉಳಿದವುಗಳೆಲ್ಲ ಆ ಗುರಿಯೆಡೆಗೆ ಹೋಗಲು ನಮಗೆ ಸಹಾಯಮಾಡಿದರೆ ಒಳ್ಳೆಯದು. ಆಗ ಅವಕ್ಕೆ ವ್ಯಾವಹಾರಿಕ ಬೆಲೆ ಇದೆ. ನಮ್ಮನ್ನು ಆ ಸ್ವಾತಂತ್ರ್ಯದೆಡೆಗೆ ಯಾವಾಗ ಕರೆದೊಯುವುದಿಲ್ಲವೋ ಅವು ನಿಜವಾಗಿಯೂ ಅಪಾಯಕಾರಿ. ಹಾಗಾದರೆ ನಿಜವಾದ ವ್ಯಾವಹಾರಿಕ ಧರ್ಮ ಯಾವುದು? ಈ ಜಗತ್ತು ಮತ್ತು ಮುಂದೆ ಬರುವ ಜಗತ್ತನ್ನು ಮುಕ್ತಿಯನ್ನು ಪಡೆಯುವುದಕ್ಕಾಗಿ ಉಪಯೋಗಿಸಿಕೊಳ್ಳುವುದು. ಈ ಜಗತ್ತಿನಲ್ಲಿ ಪ್ರತಿಯೊಂದು ಭೋಗವನ್ನೂ, ಒಂದು ಹನಿ ಸುಖವನ್ನೂ, ನಮ್ಮ ಅನಂತ ಹೃದಯ ಮತ್ತು ಮನಸ್ಸನ್ನು ವ್ರಯಮಾಡಿ ಪಡೆದುಕೊಂಡಿರುವೆವು.

ಈ ಜಗತ್ತಿನಲ್ಲಿರುವ ಒಟ್ಟು ಸುಖದುಃಖಗಳ ಮೊತ್ತವನ್ನು ನೋಡಿ, ಇದು\break ಬದಲಾಯಿಸಿದೆಯೆ? ಸಾವಿರಾರು ವರುಷಗಳವರೆಗೆ ಈ ಜಗತ್ತನ್ನು ತಮ್ಮ ಗುರಿಯನ್ನಾಗಿ ಮಾಡಿಕೊಂಡ ಧರ್ಮಗಳು ಬೇಕಾದಷ್ಟು ಇದಕ್ಕಾಗಿ ದುಡಿದಿವೆ. ಈ ಸಲ ಈ ಸಮಸ್ಯೆ ಪರಿಹಾರವಾಗುವುದೆಂದು ಜಗತ್ತು ಪ್ರತಿ ಸಲವೂ ಭಾವಿಸಿತ್ತು. ಸಮಸ್ಯೆ ಯಾವಾಗಲೂ ಒಂದೇ ಹೆಚ್ಚು ಎಂದರೆ ಅದು ತನ್ನ ರೂಪವನ್ನು ಬದಲಾಯಿಸುವುದು. ಹತ್ತಾರು ಸಹಸ್ರ ಅಂಗಡಿಗಳನ್ನು ಪಡೆಯಲು ಶ್ರಮಿಸಿ ನರಗಳ ದೌರ್ಬಲ್ಯವನ್ನೋ ಕ್ಷಯರೋಗವನ್ನೋ ಪ್ರತಿಫಲವಾಗಿ ಪಡೆಯುತ್ತಿದೆ. ಇದು ಹಳೆಯ ವಾತರೋಗದಂತೆ. ಅದನ್ನು ಒಂದು ಸ್ಥಳದಿಂದ ಓಡಿಸಿದರೆ ಅದು ಮತ್ತೊಂದು ಸ್ಥಳಕ್ಕೆ ಓಡುವುದು. ನೂರು ವರುಷಗಳ ಹಿಂದೆ ಮನುಷ್ಯ ಕಾಲು ನಡಿಗೆಯಲ್ಲಿ ಹೋಗುತ್ತಿದ್ದ. ಅಥವಾ ಕುದುರೆಗಳನ್ನು ಕೊಳ್ಳುತ್ತಿದ್ದ. ಈಗ ಅವನು ರೈಲಿನಲ್ಲಿ ಹೋಗುತ್ತಿರುವೆ ಎಂದು ಆನಂದಿಸುವನು. ಆದರೆ ಅವನು ದುಃಖಿ. ಏಕೆಂದರೆ ಅವನು ಹೆಚ್ಚು ಕೆಲಸ ಮಾಡಿ ಹೆಚ್ಚು ಹಣವನ್ನು ಸಂಪಾದಿಸಬೇಕಾಗಿದೆ. ಶ್ರಮವನ್ನು ಕಡಿಮೆ ಮಾಡುವ ಪ್ರತಿಯೊಂದು ಯಂತ್ರವೂ ಶ್ರಮಜೀವಿಗಳ ಮೇಲೆ ಹೆಚ್ಚು ಕಷ್ಟವನ್ನು ಹೊರಿಸುವುದು.

ಈ ಪ್ರಪಂಚವನ್ನು ಪ್ರಕೃತಿ ಎಂದು ಕರೆಯಿರಿ. ಅದಕ್ಕೆ ಒಂದು ಮಿತಿ ಇರಬೇಕು. ಅದು ಅನಂತವಾಗಲಾರದು. ನಿರಪೇಕ್ಷವು ಪ್ರಕೃತಿಯಾಗಬೇಕಾದರೆ ದೇಶ ಕಾಲ ನಿಮಿತ್ತಕ್ಕೆ ಒಳಗಾಗಬೇಕು. ನಮ್ಮಲ್ಲಿರುವ ಶಕ್ತಿ ಮಿತವಾಗಿರುವುದು. ನೀವು ಅದನ್ನು ಒಂದು ಕಡೆ ಖರ್ಚುಮಾಡಿದರೆ, ಇನ್ನೊಂದು ಕಡೆ ಖರ್ಚುಮಾಡುವುದಕ್ಕೆ\break ಉಳಿಯುವುದಿಲ್ಲ. ಒಟ್ಟು ಮೊತ್ತವು ಯಾವಾಗಲೂ ಒಂದೇ ಸಮನಾಗಿರುವುದು. ಎಲ್ಲಿ ಒಂದು ಅಲೆ ಇರುವುದೋ ಅದರ ಹಿಂದೆ ಒಂದು ಬೀಳು ಇರಬೇಕಾಗುವುದು. ಒಂದು ದೇಶ ಶ‍್ರೀಮಂತವಾದರೆ ಮತ್ತೊಂದು ದರಿದ್ರವಾಗಬೇಕು. ಪಾಪಪುಣ್ಯಗಳು ಸಮಾನವಾಗಿವೆ. ಯಾರು ಅಲೆಯ ಮೇಲೆ ಇರುವನೋ ಅವನು ಎಲ್ಲಾ ಚೆನ್ನಾಗಿರುವುದು ಎಂದು ಭಾವಿಸುವನು. ಕೆಳಗೆ ಇರುವವನು ಪ್ರಪಂಚವೆಲ್ಲಾ ಪಾಪಮಯ ಎಂದು ಭಾವಿಸುವನು. ಆದರೆ ಯಾರು ಸಾಕ್ಷಿಯಂತೆ ನಿಂತು ನೋಡುತ್ತಿರುವನೊ ಅವನು ಭಗವಂತನ ಲೀಲೆ ಸಾಗುತ್ತಿರುವುದನ್ನು ನೋಡುವನು. ಕೆಲವರು ಅಳುವರು, ಕೆಲವರು ನಗುವರು. ಈಗ ಅಳುವವರು ನಂತರ ನಗುವರು. ನಾವು ಇದಕ್ಕೆ ಏನು ಮಾಡಬೇಕು? ಅದಕ್ಕೆ ಏನನ್ನೂ ಮಾಡುವುದಕ್ಕೆ ಆಗುವುದಿಲ್ಲ ಎಂಬುದು ನಮಗೆ ಗೊತ್ತಿದೆ.

ನಾವು ಒಳ್ಳೆಯದನ್ನು ಮಾಡಬೇಕೆಂದು ನಮ್ಮಲ್ಲಿ ಯಾರು ಅದನ್ನು ಮಾಡುತ್ತಿರುವರು? ಹೀಗೆ ಮಾಡುವವರು ಎಷ್ಟು ವಿರಳ! ಅವರನ್ನು ಕೈಬೆರಳುಗಳಿಂದ ಎಣಿಸಬಹುದು. ಉಳಿದವರು ಒಳ್ಳೆಯದನ್ನು ಮಾಡುವರು. ಅವರು ಬಲಾತ್ಕಾರಕ್ಕೆ ಒಳಗಾಗಿ ಹಾಗೆ ಮಾಡುವರು.... ನಾವು ಹಾಗೆ ಮಾಡದೆ ವಿಧಿಯೇ ಇಲ್ಲ. ಅಲ್ಲಿ ಇಲ್ಲಿ ಬಿದ್ದು ಒದ್ದಾಡಿ ನಾವು ಮುಂದೆ ಸಾಗುತ್ತೇವೆ. ನಾವಿನ್ನು ಏನು ಮಾಡಬಲ್ಲೆವು? ಈ ವಿಶ್ವ ಹೀಗೆಯೇ ಇರುವುದು. ನಮ್ಮ ಭೂಮಿ ಹೀಗೆಯೇ ಇರುವುದು. ಅದು ನೀಲಿಯಿಂದ ಕಂದುಬಣ್ಣಕ್ಕೆ ಬರುತ್ತದೆ, ಕಂದು ಬಣ್ಣದಿಂದ ನೀಲಿ ಬಣ್ಣಕ್ಕೆ ಹೋಗುತ್ತದೆ. ಒಂದು ಭಾಷೆಯನ್ನು ಮತ್ತೊಂದು ಭಾಷೆಗೆ ಪರಿವರ್ತಿಸಿದಂತೆ ಇದೆ. ಒಂದು ಬಗೆಯ ನ್ಯೂನತೆಗಳು ಬೇರೊಂದು ಬಗೆಯ ನ್ಯೂನತೆಗಳಾಗುವುವು. ಈ ಪ್ರಪಂಚದಲ್ಲಿ ಆಗುತ್ತಿರುವುದೇ ಹೀಗೆ. ಒಂದು ಆರು, ಮತ್ತೊಂದು ಅರ್ಧ ಡಜನ್. ಅಮೆರಿಕದ ಇಂಡಿಯನ್ನನು ನಿಮ್ಮಂತೆ ತತ್ತ್ವಶಾಸ್ತ್ರದ ಮೇಲೆ ಉಪನ್ಯಾಸವನ್ನು ಕೇಳಲಾರನು. ಆದರೆ ಅವನು ತಾನು ತಿಂದದ್ದನ್ನು ಚೆನ್ನಾಗಿ ಅರಗಿಸಿಕೊಳ್ಳುವನು. ನೀವು ಅವನಿಗೆ ಎಷ್ಟು ಗಾಯ ಮಾಡಿದರೂ ಬಹಳ ಬೇಗ ಚೇತರಿಸಿಕೊಳ್ಳುತ್ತಾನೆ. ನಮಗೆ ನಿಮಗೆ ಸ್ವಲ್ಪ ಗಾಯವಾದರೂ ನಾವು ಆರು ತಿಂಗಳು ಆಸ್ಪತ್ರೆಯಲ್ಲಿರಬೇಕಾಗುವುದು.

ಜೀವವು ಕೆಳಮಟ್ಟದಲ್ಲಿ ಇದ್ದಷ್ಟೂ ಇಂದ್ರಿಯಗಳಲ್ಲಿ ಅದರ ಸುಖ ಜಾಸ್ತಿ. ತುಂಬಾ ಕ್ಷುದ್ರಪ್ರಾಣಿಯ ಸ್ಪರ್ಶೇಂದ್ರಿಯವನ್ನು ನೋಡಿ, ಅದು ಎಲ್ಲವನ್ನೂ ಸ್ಪರ್ಶದಿಂದಲೇ ಅರಿಯುವುದು. ನೀವು ಮನುಷ್ಯನ ಕ್ಷೇತ್ರಕ್ಕೆ ಬಂದರೆ, ಅವನು ವಿಕಾಸದ ಏಣಿಯಲ್ಲಿ ಎಷ್ಟು ಕೆಳಗಿರುವನೋ ಅಷ್ಟೂ ಇಂದ್ರಿಯಗಳ ಅಧೀನಕ್ಕೆ ಹೆಚ್ಚು ಬದ್ದನು. ಮೇಲಮೇಲಕ್ಕೆ ಹೋದಂತೆ ಅವನಿಗೆ ಇಂದ್ರಿಯಗಳ ಮೇಲಿನ ಲಾಲಸೆ ಕಡಿಮೆಯಾಗುತ್ತಾ ಬರುವುದು. ನಾಯಿ ಊಟಮಾಡಬಲ್ಲದು. ಆದರೆ ತತ್ತ್ವಶಾಸ್ತ್ರವನ್ನು ಕುರಿತು ಆಲೋಚಿಸಿ ಆನಂದವನ್ನು ಪಡೆಯಲಾರದು. ನೀವು ಬುದ್ಧಿಯ ಮೂಲಕ ಯಾವ ಆನಂದವನ್ನು ಪಡೆಯುವಿರೋ ಅದನ್ನು ನಾಯಿ ಪಡೆಯಲಾರದು. ಇಂದ್ರಿಯಗಳ ಸುಖ ಹೆಚ್ಚು. ಅದಕ್ಕಿಂತ ಹೆಚ್ಚು ಬುದ್ಧಿಯ ಮೂಲಕ ಬರುವುದು. ನೀವು ಪ್ಯಾರಿಸ್ಸಿನಲ್ಲಿ ಐವತ್ತು ಬಗೆಯ ಸೊಗಸಾದ ಊಟವನ್ನು ಮಾಡಿದರೆ ಅದೊಂದು ನಿಜವಾದ ಸುಖವೇ. ಆದರೆ\break ದೂರದರ್ಶಕ ಯಂತ್ರದ ಮೂಲಕ ನಕ್ಷತ್ರಗಳನ್ನು ನೋಡುತ್ತಿರುವಾಗ, ಹೇಗೆ ಪ್ರಪಂಚ ಸೃಷ್ಟಿಯಾಗುತ್ತಿದೆ ಎಂಬುದನ್ನು ನೋಡಿದಾಗ, ಅದರಿಂದ ಆಗುವ ಸುಖವನ್ನು ಊಹಿಸಿನೋಡಿ. ಅದು ಊಟಕ್ಕಿಂತ ಮೇಲಿರಬೇಕು. ಏಕೆಂದರೆ ಅದನ್ನು ನೋಡುತ್ತಿರುವಾಗ ಊಟವನ್ನೂ ಮರೆಯುವಿರಿ. ಅದು ಇಂದ್ರಿಯ ಸುಖಕ್ಕಿಂತ ಮೇಲಿನ ಮಟ್ಟದಲ್ಲಿರಬೇಕು. ಆಗ ಹೆಂಡತಿ ಗಂಡ ಮಕ್ಕಳು ಎಲ್ಲವನ್ನೂ ಮರೆಯುವಿರಿ. ಇಂದ್ರಿಯ ಪ್ರಪಂಚಕ್ಕೆ ಸೇರಿರುವುದನ್ನೆಲ್ಲಾ ಮರೆಯುವಿರಿ. ಇದೇ ಬುದ್ದಿಯಿಂದ ಬರುವ ಆನಂದ. ಇದು ಇಂದ್ರಿಯ ಸುಖಕ್ಕಿಂತ ಮೇಲಿನದು ಎಂದು ಊಹಿಸುವುದು ಸರ್ವ ಸಾಮಾನ್ಯವಾಗಿದೆ. ನೀವು ಯಾವಾಗಲೂ ಉತ್ತಮ ಸುಖಕ್ಕಾಗಿ ಕಡಿಮೆಯ ಸುಖವನ್ನು ಮರೆಯುವಿರಿ. ಇದೇ ಅನುಷ್ಠಾನ ಧರ್ಮ – ತ್ಯಾಗಮಾಡುವುದು ಸ್ವಾತಂತ್ರ್ಯವನ್ನು ಪಡೆಯುವುದು ಮತ್ತು ಬಂಧನವನ್ನು ತ್ಯಜಿಸುವುದು.

ಕೆಳಗಿನದನ್ನು ತ್ಯಜಿಸಿದರೆ ಮೇಲಿನದು ದೊರಕುವುದು. ಸಮಾಜದ ತಳಹದಿ ಯಾವುದು? ನೀತಿ, ನಿಯಮ, ಶಾಸನಗಳು. ಇದು, ಮತ್ತೊಬ್ಬರ ಆಸ್ತಿಯನ್ನು ದೋಚುವುದನ್ನು ತ್ಯಜಿಸಿ, ಮತ್ತೊಬ್ಬರನ್ನು ಹಿಂಸಿಸುವುದನ್ನು ತ್ಯಜಿಸಿ, ದುರ್ಬಲರನ್ನು ಪೀಡಿಸುವುದನ್ನು ತ್ಯಜಿಸಿ, ಸುಳ್ಳು ಹೇಳಿ ಇತರರನ್ನು ಮೋಸಗೊಳಿಸುವುದರಿಂದ ಬರುವ ಆನಂದವನ್ನು ತ್ಯಜಿಸಿರಿ ಎಂದು ಹೇಳುವುದು. ಸಮಾಜದ ತಳಹದಿ ನೀತಿಯಲ್ಲವೆ? ಮದುವೆ ಎಂದರೆ ಏನು? ವ್ಯಭಿಚಾರವನ್ನು ತ್ಯಜಿಸುವುದು ತಾನೇ. ಕಾಡು ಜನರು ಮದುವೆಯಾಗುವುದಿಲ್ಲ. ಮನುಷ್ಯ ತ್ಯಾಗ ಮಾಡುವುದರಿಂದ ಮದುವೆಯಾಗುವನು. ತ್ಯಾಗ ಮಾಡು, ತ್ಯಾಗ ಮಾಡು! ತ್ಯಾಗ, ಬರೀ ಬಿಡುವುದು, ಶೂನ್ಯಕ್ಕಾಗಿ ಅಲ್ಲ. ಏನನ್ನೂ ಪಡೆಯದೆ ಬಿಡುವುದಲ್ಲ. ಮೇಲಿನದನ್ನು ಪಡೆಯವುದಕ್ಕೆ ಕೆಳಗಿನದನ್ನು ಬಿಡುವುದು. ಆದರೆ ಇದನ್ನು ಯಾರು ಮಾಡಬಲ್ಲರು? ಮೇಲಿನದು ಸಿಕ್ಕುವ ತನಕ ನೀವು ಇದನ್ನು ಬಿಡಲಾರಿರಿ. ನೀವು ಮಾತನಾಡಬಹುದು, ಸಾಧನೆಮಾಡಬಹುದು. ನೀವು ಮತ್ತೆ ಏನನ್ನೂ ಮಾಡಲು ಯತ್ನಿಸಬಹುದು. ಮೇಲಿನದು ಬಂದರೆ ತ್ಯಾಗ ಸ್ವಾಭಾವಿಕವಾಗಿ ಬರುವುದು. ಆಗ ಕೆಳಗಿನದು ತನಗೆ ತಾನೆ ಬಿಟ್ಟು ಹೋಗುವುದು.

ಇದೇ ಅನುಷ್ಠಾನ ಧರ್ಮ. ಅಲ್ಲದೆ ಬೇರೇನು? ರಸ್ತೆ ಗುಡಿಸುವುದು, ಆಸ್ಪತ್ರೆಗಳನ್ನು ಕಟ್ಟಿಸುವುದೆ? ಇವೆಲ್ಲಾ ತ್ಯಾಗಕ್ಕೆ ನಮಗೆ ಸಹಾಯಮಾಡಿದರೆ ಮಾತ್ರ ಒಳ್ಳೆಯದು. ತ್ಯಾಗಕ್ಕೆ ಒಂದು ಕೊನೆಯಿಲ್ಲ. ಅದಕ್ಕೆ ಒಂದು ಮಿತಿಯನ್ನು ಕಲ್ಪಿಸಲು ಯತ್ನಿಸುವವರು ಇಷ್ಟು ದೂರ ಹೋಗಿ–ಇನ್ನು ಮುಂದೆ ಹೋಗಬೇಡಿ ಎನ್ನುವರು. ಇದೇ ಬಂದಿರುವ ತೊಂದರೆ. ಆದರೆ ಈ ತ್ಯಾಗಕ್ಕೆ ಒಂದು ಪರಿಮಿತಿ ಇಲ್ಲ.

ದೇವರಿರುವ ಎಡೆ ಬೇರೇನೂ ಇಲ್ಲ. ಎಲ್ಲಿ ಪ್ರಪಂಚ ಇರುವುದೊ ಅಲ್ಲಿ ದೇವರಿಲ್ಲ. ಇವೆರಡೂ ಎಂದಿಗೂ ಒಟ್ಟಿಗೆ ಸೇರಲಾರವು, ಕತ್ತಲೆ ಬೆಳಕಿನಂತೆ. ಕೈಸ್ತರ ಧರ್ಮ ಮತ್ತು ಕ್ರಿಸ್ತನ ಜೀವನದಿಂದ ನಾನು ಕಲಿತಿರುವುದು ಇದು. ಇದು ಬೌದ್ಧ ಧರ್ಮವಲ್ಲವೆ? ಇದು ಹಿಂದೂಧರ್ಮವಲ್ಲವೆ? ಇದು ಮಹಮ್ಮದೀಯ ಧರ್ಮವಲ್ಲವೆ? ಎಲ್ಲಾ ಗುರುಗಳೂ ಮಹಾತ್ಮರೂ ಮಾಡಿದ ಬೋಧನೆ ಇದೇ ತಾನೆ? ನಾವು ತ್ಯಜಿಸಬೇಕಾಗಿರುವ ಜಗತ್ತು ಯಾವುದು? ಅದು ಇಲ್ಲಿದೆ. ಅದನ್ನು ಯಾವಾಗಲೂ ನಾನು ಹೊತ್ತುಕೊಂಡು ಹೋಗುತ್ತಿರುವೆನು. ಅದೇ ನನ್ನ ದೇಹ. ಈ ದೇಹಕ್ಕಾಗಿ ನಾನು ಮತ್ತೊಬ್ಬನಿಗೆ ತೊಂದರೆ ಕೊಡುವೆನು; ಈ ದೇಹವನ್ನು ಚೆನ್ನಾಗಿಟ್ಟಿರಲು, ಇದಕ್ಕೆ ಸ್ವಲ್ಪ ಸುಖಕೊಡಲು, ಇತರರನ್ನು ಪೀಡಿಸುವೆನು. ಈ ದೇಹಕ್ಕಾಗಿಯೇ ನಾನು ಇತರರನ್ನು ಹಿಂಸಿಸುವುದು, ತಪ್ಪುಗಳನ್ನು ಮಾಡುವುದು.

ಮಹಾತ್ಮರು ಕಾಲವಾಗಿರುವರು, ದುರ್ಬಲರೂ ಕಾಲವಾಗಿರುವರು, ದೇವತೆಗಳೂ ಕಾಲವಾಗಿರುವರು. ಸಾವು, ಸಾವು, ಎಲ್ಲೆಲ್ಲಿಯೂ ಸಾವಿನ ಲೀಲೆಯೆ. ಈ ಭೂಮಿ ಆದಿಯಿಲ್ಲದ ಭೂತಕಾಲದಿಂದ ಬಂದುದನ್ನೆಲ್ಲಾ ಸಮಾಧಿಮಾಡಿರುವ ಸ್ಮಶಾನ. ಆದರೂ ನಾವು ಈ ದೇಹಕ್ಕೆ ಅಂಟಿಕೊಂಡಿರುವೆವು. "ನಾನು ಎಂದಿಗೂ ಸಾಯುವುದಿಲ್ಲ” ಎಂದು ಭಾವಿಸುವೆನು. ಈ ದೇಹ ಒಂದಲ್ಲ ಒಂದು ದಿನ ಸತ್ತು ಹೋಗುವುದು ಎಂದು ಅರಿತಿದ್ದರೂ ನಾವು ಅದರಲ್ಲಿ ಆಸಕ್ತರು. ಇದರಲ್ಲಿಯೂ ಒಂದು ಅರ್ಥವಿದೆ. ಒಂದು ದೃಷ್ಟಿಯಿಂದ ನಾವು ಸಾಯುವುದಿಲ್ಲ. ನಾವು ಮಾಡುವ ತಪ್ಪೆ ಅಮೃತವಾದ ಆತ್ಮವನ್ನು ಅಪ್ಪುವ ಬದಲು ದೇಹವನ್ನು ಅಪ್ಪಿಕೊಂಡಿರುವುದು.

ನೀವೆಲ್ಲಾ ಜಡವಾದಿಗಳು, ಏಕೆಂದರೆ ನೀವು ಬರೀ ದೇಹ ಎಂದು ನಂಬುವಿರಿ. ಯಾರಾದರೂ ನನ್ನನ್ನು ಬಲವಾಗಿ ಗುದ್ದಿದರೆ ಅವನು ನನ್ನನ್ನು ಗುದ್ದಿದ ಎನ್ನುತ್ತೇನೆ. ಯಾರಾದರೂ ನನಗೆ ಹೊಡೆದರೆ ಅವನು ನನಗೆ ಹೊಡೆದ ಎನ್ನುತ್ತೇನೆ. ನಾನು ದೇಹ ಅಲ್ಲದೇ ಇದ್ದರೆ ಹೀಗೆ ಏತಕ್ಕೆ ಹೇಳುತ್ತಿದ್ದೆ? ನಾನು ಒಂದು ಆತ್ಮ ಎಂದರೆ ಏನೂ ತಪ್ಪಾಗುತ್ತಿರಲಿಲ್ಲ. ನಾನೀಗ ದೇಹವಾಗಿರುವೆನು, ಭೌತಿಕವಾಗಿರುವೆನು. ಆದಕಾರಣವೇ ನನ್ನ ನಿಜವಾದ ಆತ್ಮನ ಸ್ಥಿತಿಗೆ ಹೋಗಬೇಕಾದರೆ ದೇಹ ಭಾವವನ್ನು ತ್ಯಜಿಸಬೇಕಾಗಿದೆ. ನಾನು ಆತ್ಮ, ಯಾವ ಆಯುಧವೂ ಅದನ್ನು ಭೇದಿಸಲಾರದು, ಯಾವ ಖಡ್ಗವೂ ಛೇದಿಸಲಾರದು. ಯಾವ ಬೆಂಕಿಯೂ ದಹಿಸಲಾರದು, ಯಾವ ಗಾಳಿಯೂ ಒಣಗಿಸಲಾರದು. ನಾನು ಅಜ, ಎಂದಿಗೂ ಸೃಷ್ಟಿಯಾದವನಲ್ಲ. ಅನಾದಿ ಅನಂತನು ನಾನು. ಜನನ ಮರಣಾತೀತನು ನಾನು. ಸರ್ವವ್ಯಾಪಿ ನಾನು. ಮಣ್ಣಿನ ಮುದ್ದೆಯಂತಿರುವ ಈ ದೇಹವೇ ನಾನು ಎಂಬುದರಿಂದಲೇ ಈ ದುಃಖವೆಲ್ಲ ಬಂದಿರುವುದು. ನಾನು ಪಂಚಭೂತಗಳೊಂದಿಗೆ ತಾದಾತ್ಮ್ಯಭಾವವನ್ನು ಮಾಡಿಕೊಂಡು ಅದರ ಪರಿಣಾಮದಿಂದ ಪರಿತಪಿಸುತ್ತಿರುವೆನು.

ಅನುಷ್ಠಾನ ಧರ್ಮ ಎಂದರೆ, ನಿಜವಾದ ನನ್ನ ಆತ್ಮದೊಂದಿಗೆ ನಾನು ಸೇರುವುದು. ಅನ್ಯಥಾ ಭಾವನೆಯನ್ನು ತ್ಯಜಿಸಿ. ನೀವು ಈ ಮಾರ್ಗದಲ್ಲಿ ಎಷ್ಟು ಮುಂದುವರಿದಿರುವಿರಿ? ನೀವು ಎರಡು ಸಾವಿರ ಆಸ್ಪತ್ರೆಗಳನ್ನು ಕಟ್ಟಿಸಿರಬಹುದು; ಐವತ್ತು ಸಾವಿರ ಮೈಲಿ ರಸ್ತೆಗಳನ್ನು ಮಾಡಿಸಿರಬಹುದು. ಆದರೂ ನೀವು ಆತ್ಮ ಎಂದು ಅರಿಯದೆ ಇದ್ದರೆ ಇದರಿಂದೆಲ್ಲಾ ಏನು ಪ್ರಯೋಜನ? ನೀವು ನಾಯಿಗಳಂತೆ ಸಾಯುತ್ತೀರಿ. ನಾಯಿಗಳು ಹೇಗಿವೆಯೋ ಹಾಗೆಯೇ ಇರುತ್ತೀರಿ. ನಾಯಿ ಬೊಗಳುತ್ತದೆ, ಒರಲುತ್ತದೆ. ಏಕೆಂದರೆ ಅದು ತಾನೊಂದು ದೇಹ ಎಂದು ಭಾವಿಸಿದೆ. ಅದು ನಾಶವಾಗುತ್ತದೆ.

ಸಾವಿದೆ, ಶತಃಸಿದ್ಧವಾದ ಸಾವು ಎಲ್ಲೆಲ್ಲಿಯೂ ಇದೆ. ನೀರಿನಲ್ಲಿ, ಗಾಳಿಯಲ್ಲಿ, ಅರಮನೆಯಲ್ಲಿ, ಸೆರೆಮನೆಯಲ್ಲಿ, ಮೃತ್ಯು ಸರ್ವವ್ಯಾಪಿಯಾಗಿದೆ. ನಿಮ್ಮನ್ನು\break ಯಾವುದು ನಿರ್ಭಯರನ್ನಾಗಿ ಮಾಡುವುದು? ನೀವು ಏನಾಗಿರುವಿರೋ ಅದನ್ನು ಅರಿತಾಗ – ಅನಂತಾತ್ಮವು, ಜನನ ಮರಣಾತೀತ, ಅದನ್ನು ಯಾವ ಬೆಂಕಿಯೂ ಸುಡಲಾರದು, ಯಾವ ಶಸ್ತ್ರವೂ ಛೇದಿಸಲಾರದು, ಯಾವ ವಿಷವೂ ಕೊಲ್ಲಲಾರದು – ಎಂದು ಅರಿತಾಗ ನೀವು ನಿರ್ಭಯರಾಗುತ್ತೀರಿ. ಬರೀ ಸಿದ್ದಾಂತಗಳಲ್ಲ, ಬರೀ ಪುಸ್ತಕಗಳನ್ನು ಓದುವುದಲ್ಲ, ಅರಗಿಳಿಯಂತೆ ಕಂಠಪಾಠ ಮಾಡುವುದಲ್ಲ. ನನ್ನ ಗುರುಗಳು, ಗಿಳಿಗೆ ರಾಮ ರಾಮ ಎಂದು ಕಲಿಸುವುದು ಒಳ್ಳೆಯದು, ಆದರೆ ಬೆಕ್ಕು ಅದರ ಕತ್ತನ್ನು ಹಿಡಿದಾಗ ತಾನು ಕಲಿತದ್ದನ್ನೆಲ್ಲಾ ಮರೆಯುವುದು, ಎಂದು ಹೇಳುತ್ತಿದ್ದರು. ನೀವು ಯಾವಾಗಲೂ ಪ್ರಾರ್ಥನೆ ಮಾಡುತ್ತಿರಬಹುದು, ಪ್ರಪಂಚದಲ್ಲಿರುವ ಶಾಸ್ತ್ರಗಳನ್ನೆಲ್ಲಾ ಓದಬಹುದು. ಇರುವ ದೇವರನ್ನೆಲ್ಲಾ ಪೂಜಿಸಬಹುದು. ಆದರೆ ಆತ್ಮನನ್ನು ಅರಿಯುವ ತನಕ ಮುಕ್ತಿ ಎಂಬುದಿಲ್ಲ. ಮಾತಲ್ಲ, ಸಿದ್ದಾಂತವಲ್ಲ, ವಿವಾದವಲ್ಲ, ಬೇಕಾಗಿರುವುದು ಸಾಕ್ಷಾತ್ಕಾರ. ನಾನು ಅದನ್ನೆ ಅನುಷ್ಠಾನ ಧರ್ಮ ಎನ್ನುವುದು.

ಈ ಆತ್ಮನ ವಿಷಯವಾದ ಈ ಸತ್ಯವನ್ನು ನಾವು ಮೊದಲು ಕೇಳಬೇಕು. ಅದನ್ನು ಕೇಳಿದ ಮೇಲೆ ಅದನ್ನು ಮನನ ಮಾಡಬೇಕು. ಮನನ ಮಾಡಿದ ಮೇಲೆ ಅದನ್ನು ಧ್ಯಾನಿಸಬೇಕು. ಇನ್ನು ಮೇಲೆ ವಾದವಿವಾದಗಳನ್ನೆಲ್ಲಾ ಮರೆಯಬೇಕು. ನೀವು\break ಅನಂತಾತ್ಮರು ಎಂಬುದನ್ನು ಒಂದು ಸಲ ಸಾಕ್ಷಾತ್ಕಾರ ಮಾಡಿಕೊಳ್ಳಿ. ಅದು ನಿಜವಾದರೆ ನೀವು ದೇಹ ಎನ್ನುವುದಕ್ಕೆ ಅರ್ಥವಿಲ್ಲ. ನೀವು ಆತ್ಮ, ಅದನ್ನು ನೀವು ಸಾಕ್ಷಾತ್ಕಾರ ಮಾಡಿಕೊಳ್ಳಬೇಕಾಗಿದೆ. ಆತ್ಮ ಆತ್ಮನಂತೆ ತನ್ನನ್ನು ನೋಡಿಕೊಳ್ಳಬೇಕು. ಈಗ ಆತ್ಮ ತಾನು ದೇಹ ಎಂದು ಭಾವಿಸುತ್ತಿದೆ. ಅದು ನಿಲ್ಲಬೇಕು. ನೀವು ಅದನ್ನು ಸಾಕ್ಷಾತ್ಕಾರಮಾಡಿಕೊಂಡ ಒಡನೆಯೇ ನೀವು ಮುಕ್ತರು.

ನೀವು ಈ ಲೋಟವನ್ನು ನೋಡುತ್ತೀರಿ. ಇದೊಂದು ಭ್ರಾಂತಿ ಎಂಬುದು ನಿಮಗೆ ಗೊತ್ತಿದೆ. ಕೆಲವು ವೈಜ್ಞಾನಿಕರು ಇವೆಲ್ಲಾ ಕೇವಲ ಬೆಳಕು ಮತ್ತು ಸ್ಪಂದನ ಎಂದು ಹೇಳುತ್ತಾರೆ. ಆತ್ಮನನ್ನು ನೋಡುವುದು ಇದಕ್ಕಿಂತ ಹೆಚ್ಚು ಸತ್ಯವಾಗಿರಬೇಕು. ಅದೇ ನಿಜವಾದ ಅವಸ್ಥೆ ಆಗಬೇಕು. ಅದೊಂದೇ ನಿಜವಾದ ಗ್ರಹಿಕೆ ಆಗಬೇಕು, ಅದೊಂದೇ ನಿಜವಾದ ದರ್ಶನ ಆಗಬೇಕು. ನೀವು ನೋಡುವ ದೃಶ್ಯವೆಲ್ಲಾ ಬರೀ ಕನಸು. ನಿಮಗೆ ಅದು ಈಗ ಗೊತ್ತಿದೆ. ಹಿಂದಿನ ಕಾಲದ ಭಾವಸತ್ತಾವಾದಿಗಳು ಮಾತ್ರವಲ್ಲ, ಈಗಿನ ಕಾಲದ ವೈಜ್ಞಾನಿಕರು ಕೂಡ ಅಲ್ಲಿರುವುದು ಬೆಳಕು ಎಂದು ಹೇಳುತ್ತಾರೆ. ಸ್ಪಂದನದ ತಾರತಮ್ಯವೇ ವ್ಯತ್ಯಾಸಕ್ಕೆಲ್ಲ ಕಾರಣ ಎಂದು ಅವರು ಹೇಳುವರು.

ನೀವು ದೇವರನ್ನು ನೋಡಬೇಕು. ಆತ್ಮನನ್ನು ಸಾಕ್ಷಾತ್ಕಾರ ಮಾಡಿಕೊಳ್ಳಬೇಕು. ಅದೇ ನಿಜವಾದ ಅನುಷ್ಠಾನ ಧರ್ಮ. ಕ್ರಿಸ್ತ ಏನನ್ನು ಬೋಧಿಸಿದನೊ ''ದೀನರು ಧನ್ಯರು. ಏಕೆಂದರೆ ಅವರೆ ಭಗವಂತನ ರಾಜ್ಯಕ್ಕೆ ಹೋಗುವವರು'' ಎಂಬುದನ್ನೆ ಅಲ್ಲವೆ ನೀವು ಅನುಷ್ಠಾನ ಧರ್ಮ ಎನ್ನುವುದು? ಇದೊಂದು ತಮಾಷೆಯೆ? ನೀವು\break ಆಲೋಚಿಸುತ್ತಿರುವ ಅನುಷ್ಠಾನ ಧರ್ಮ ಯಾವುದು? ದೇವರು ನಮ್ಮನ್ನು ರಕ್ಷಿಸಬೇಕು! 'ಪರಿಶುದ್ಧಾತ್ಮರೇ ಧನ್ಯರು. ಅವರೇ ದೇವರನ್ನು ನೋಡಬಲ್ಲರು!” ಹೀಗೆಂದರೆ ರಸ್ತೆ ಗುಡಿಸುವುದು, ಆಸ್ಪತ್ರೆ ಕಟ್ಟಿಸುವುದು ಇವುಗಳೇನು? ಒಳ್ಳೆಯ ಕೆಲಸಗಳನ್ನು ನೀವು ಪರಿಶುದ್ಧವಾದ ಮನಸ್ಸಿನಿಂದ ಮಾಡಬೇಕು. ನೀವು ಒಬ್ಬ ಮನುಷ್ಯನಿಗೆ ಇಪ್ಪತ್ತು ಡಾಲರುಗಳನ್ನು ದಾನ ಕೊಟ್ಟು ಅನಂತರ ನಿಮ್ಮ ಹೆಸರು ಬರುವ ಎಲ್ಲಾ ಪತ್ರಿಕೆಗಳನ್ನೂ ಕೊಂಡುಕೊಳ್ಳುವುದಲ್ಲ. ಯಾರೂ ನಿಮಗೆ ಸಹಾಯ ಮಾಡಲಾರರು ಎಂಬುದನ್ನು ನಿಮ್ಮ ಶಾಸ್ತ್ರದಲ್ಲಿಯೇ ನೀವು ಓದುವುದಿಲ್ಲವೆ? ಬಡವರಲ್ಲಿ, ದುಃಖಿಗಳಲ್ಲಿ, ದುರ್ಬಲರಲ್ಲಿ ಸಾಕ್ಷಾತ್ ಭಗವಂತನೇ ಇರುವನು ಎಂದು ಅವನನ್ನು ಪೂಜಿಸಿ. ಅದನ್ನು ಮಾಡಿದ ಮೇಲೆ ಅದರಿಂದ ದೊರಕುವ ಪ್ರತಿಫಲ ಗೌಣ. ಯಾವುದೇ ಪ್ರತಿಫಲಾಪೇಕ್ಷೆಯಿಲ್ಲದೆ ಮಾಡಿದ ಕೆಲಸದಿಂದ ಜೀವನಿಗೆ ಶ್ರೇಯಸ್ಸಾಗುವುದು. ಸ್ವರ್ಗ ಸಿಕ್ಕುವುದು ಇಂತಹವರಿಗೆ.

ಭಗವಂತನ ಸಾಮ್ರಾಜ್ಯ ನಮ್ಮಲ್ಲಿಯೇ ಇದೆ. ಅವನು ಇಲ್ಲಿಯೇ ಇರುವನು. ಅವನೇ ಎಲ್ಲರ ಆತ್ಮ. ಅವನನ್ನು ನಿಮ್ಮ ಆತ್ಮನಲ್ಲಿಯೂ ನೋಡಿ. ಅದೇ ವ್ಯಾವಹಾರಿಕ ಧರ್ಮ, ಅದೇ ಮುಕ್ತಿ. ನಾವು ಎಷ್ಟರ ಮಟ್ಟಿಗೆ ಆ ಗುರಿಯ ಕಡೆಗೆ ಮುಂದುವರಿದಿರುವೆವು ಎಂಬುದನ್ನು ನಾವೇ ಕೇಳಿಕೊಳ್ಳೋಣ. ನಾವು ಎಷ್ಟು ಮಟ್ಟಿಗೆ ದೇಹಾರಾಧಕರು ಅಥವಾ ದೇವರನ್ನು ನಂಬುವವರಾಗಿದ್ದೇವೆ, ನಾವು ಎಷ್ಟರ ಮಟ್ಟಿಗೆ ಆತ್ಮನನ್ನು ನಂಬುತ್ತೇವೆ ಎಂಬುದನ್ನು ಕೇಳಿಕೊಳ್ಳೋಣ. ಅದೇ ನಿಜವಾದ ಸ್ವಾರ್ಥ ಶೂನ್ಯತೆ, ಮುಕ್ತಿ. ಅದೇ ನಿಜವಾದ ಪೂಜೆ. ಆತ್ಮ ಸಾಕ್ಷಾತ್ಕಾರವನ್ನು ಮಾಡಿಕೊಳ್ಳಿ. ನೀವು ಮಾಡಬೇಕಾದುದು ಇಷ್ಟೇ. ನೀವು ಅನಂತಾತ್ಮ ಎಂದು ನಿಮ್ಮನ್ನು ಅರಿತುಕೊಳ್ಳಿ. ಅದೇ ಅನುಷ್ಠಾನ ಧರ್ಮ. ಉಳಿದವುಗಳೆಲ್ಲ ಅವ್ಯವಹಾರಿಕ. ಏಕೆಂದರೆ ಉಳಿದವುಗಳೆಲ್ಲ ಮಾಯವಾಗುವುವು. ಆತ್ಮನೊಂದೇ ಎಂದಿಗೂ ಮಾಯವಾಗುವುದಿಲ್ಲ. ಅದು ನಿತ್ಯವಾದುದು. ಆಸ್ಪತ್ರೆಗಳು ಬಿದ್ದು ಹೋಗುವುವು. ರೈಲ್ವೆ ಕಂಬಿಗಳನ್ನು ಹಾಕಿದವರೆಲ್ಲ ಸತ್ತು ಹೋಗುವರು. ಈ ಜಗತ್ತು ಚೂರು ಚೂರಾಗಿ ಹೋಗುವುದು. ಸೂರ್ಯ ಕಣ್ಮರೆಯಾಗುವನು. ಆದರೆ ಆತ್ಮ ಎಂದೆಂದಿಗೂ ಇರುವುದು.

ಯಾವುದು ಮೇಲು? ನಾಶವಾಗುತ್ತಿರುವುದನ್ನು ಅರಸುವುದೋ, ನಾಶವಾಗದೆ ಇರುವುದನ್ನು ಅರಸುವುದೋ? ನಿಮ್ಮ ಶಕ್ತಿಯನ್ನೆಲ್ಲಾ ಖರ್ಚುಮಾಡಿ ಯಾವುದೋ ಪ್ರಾಪಂಚಿಕ ವಸ್ತುವನ್ನು ಪಡೆದುಕೊಳ್ಳುವಿರಿ. ಅದನ್ನು ಅನುಭವಿಸುವ ಹೊತ್ತಿಗೆ ಮೃತ್ಯು ಬರುವುದು. ನೀವು ಎಲ್ಲವನ್ನೂ ತ್ಯಜಿಸಬೇಕಾಗುವುದು. ಒಬ್ಬ ಚಕ್ರವರ್ತಿ ಎಲ್ಲರನ್ನೂ ಜಯಿಸಿದ್ದ. ಸಾಯುವ ಸಮಯ ಬಂದಾಗ ಕೊಪ್ಪರಿಗೆಯಲ್ಲಿಟ್ಟಿದ್ದ ಐಶ್ವರ್ಯಗಳನ್ನೆಲ್ಲಾ ತನ್ನ ಮುಂದೆ ತರುವಂತೆ ಹೇಳಿದ. ತಂದು ಅವನ್ನೆಲ್ಲಾ ಅವನ ಮುಂದೆ ಸುರಿದರು. ಆ ದೊಡ್ಡ ವಜ್ರವನ್ನು ನನ್ನ ಕೈಗೆ ಕೊಡಿ ಎಂದ. ಅದನ್ನು ಎದೆಯ ಮೇಲೆ ಇಟ್ಟುಕೊಂಡು ಅಳತೊಡಗಿದನು. ಹೀಗೆ ಅಳುತ್ತಾ ಅಳುತ್ತಾ ನಾಯಿ ಸಾಯುವಂತೆಯೇ ಅವನೂ ಸತ್ತ.

ನಾನು ಬದುಕಿರುವೆ ಎಂದು ಮನುಷ್ಯ ಹೇಳುತ್ತಾನೆ. ಮೃತ್ಯುವಿನ ಅಂಜಿಕೆಯಿಂದ ಅವನು ದಾಸ್ಯಭಾವದಿಂದ ಜೀವನಕ್ಕೆ ಅಂಟಿಕೊಂಡಿರುವನು ಎಂಬುದು ಅವನಿಗೆ ಗೊತ್ತಿಲ್ಲ. ನಾನು ಅನುಭವಿಸುತ್ತಿರುವೆನು ಎಂದು ಅವನು ಹೇಳುತ್ತಾನೆ. ಪ್ರಕೃತಿ ಅವನನ್ನು ತನ್ನ ಬಲೆಗೆ ಹಾಕಿಕೊಂಡಿದೆ ಎಂಬುದು ಅವನಿಗೆ ಗೊತ್ತೇ ಇಲ್ಲ.

ಪ್ರಕೃತಿ ನಮ್ಮನ್ನೆಲ್ಲಾ ಅರೆಯುತ್ತಿದೆ. ನಿಮಗೆ ಸಿಕ್ಕುವ ಆಸೆ ಎಷ್ಟು ಕಡಿಮೆ ಎಂಬುದನ್ನು ಗಮನಿಸಿ. ಪ್ರಕೃತಿ ನಿಮ್ಮ ಮೂಲಕ ತನ್ನ ಕೆಲಸವನ್ನು ಸಾಧಿಸಿಕೊಂಡಿತು. ನೀವು ಸತ್ತರೆ ಇತರ ಗಿಡಗಳಿಗೆ ಅದು ಗೊಬ್ಬರವಾಗುವುದು. ಆದರೆ ನಾವು ಯಾವಾಗಲೂ ಅದರಿಂದ ಸುಖವನ್ನು ಪಡೆಯುವೆವು ಎಂದು ಹೇಳಿಕೊಳ್ಳುತ್ತೇವೆ. ಚಕ್ರ ಹೀಗೆ ಉರುಳುತ್ತಿದೆ.

ಆತ್ಮನನ್ನು ಆತ್ಮನಂತೆ ಅರಿಯುವುದೇ ಅನುಷ್ಠಾನ ಧರ್ಮ. ಈ ಆದರ್ಶದೆಡೆಗೆ ಹೋಗುವುದಕ್ಕೆ ನಮಗೆ ಸಹಾಯ ಮಾಡುವುದೆಲ್ಲ ಒಳ್ಳೆಯದೆ. ಈ ಸಾಕ್ಷಾತ್ಕಾರವನ್ನು ತ್ಯಾಗದಿಂದ ಪಡೆಯಬೇಕಾಗಿದೆ, ಧ್ಯಾನದಿಂದ ಪಡೆಯಬೇಕಾಗಿದೆ. ಇಂದ್ರಿಯಗಳನ್ನೆಲ್ಲಾ ತ್ಯಜಿಸಬೇಕು, ನಮ್ಮನ್ನು ಭೌತವಸ್ತುಗಳಿಗೆ ಬಿಗಿದಿರುವ ಶೃಂಖಲೆಗಳನ್ನೆಲ್ಲಾ ತುಂಡರಿಸಬೇಕು. ನಮಗೆ ಈ ಪ್ರಪಂಚದ ಸುಖಬೇಕಾಗಿಲ್ಲ, ಇಂದ್ರಿಯ ಸುಖ ಬೇಕಾಗಿಲ್ಲ, ಇದಕ್ಕಿಂತ ಮೇಲಿರುವುದು ಬೇಕು. ಅದೇ ತ್ಯಾಗ. ಅನಂತರ ಧ್ಯಾನದ ಶಕ್ತಿಯ ಸಹಾಯದಿಂದ ಆಗಲೆ ಇರುವ ಅಜ್ಞಾನದ ಕರ್ಮಫಲದಿಂದ ಪಾರಾಗಿ.

ಪ್ರಕೃತಿ ಹೇಳಿದಂತೆ ನಾವು ಕೇಳುತ್ತಿರುವೆವು. ಹೊರಗೆ ಏನಾದರೂ ಶಬ್ದವಾದರೆ ನಾನು ಅದನ್ನು ಕೇಳಬೇಕು. ಹೊರಗೆ ಏನಾದರೂ ಆಗುತ್ತಿದ್ದರೆ ನಾನು ಅದನ್ನು ನೋಡಬೇಕು. ಕಪಿಗಳಂತೆ ಇರುವೆವು ನಾವು. ನಮ್ಮಲ್ಲಿ ಪ್ರತಿಯೊಬ್ಬರೂ ಎರಡು ಸಾವಿರ ಕೋತಿಗಳು ಏಕಸ್ಥಳದಲ್ಲಿರುವಂತೆ ಇರುವೆವು. ಕೋತಿಗಳು ಅತಿ ವಿಚಿತ್ರ ಪ್ರಾಣಿಗಳು. ಆದಕಾರಣ ನಮಗೆ ಬೇರೆ ಮಾರ್ಗವೇ ಇಲ್ಲ. ಸುಖವನ್ನು ಅನುಭವಿಸುತ್ತಿರುವೆ ಎನ್ನುವೆವು. ಈ ಭಾಷೆ ಅತಿ ವಿಚಿತ್ರವಾದುದು. ನಾವು ಪ್ರಪಂಚವನ್ನು ಅನುಭವಿಸುತ್ತಿರುವೆವು! ಅದನ್ನು ಅನುಭವಿಸದೆ ಬೇರೆ ದಾರಿಯೇ ಇಲ್ಲ. ಪ್ರಕೃತಿ ನಾವು ಹಾಗೆ ಮಾಡುವಂತೆ ಆಜ್ಞಾಪಿಸುವುದು. ಯಾವುದೋ ಒಂದು ಮಂಜುಳವಾದ ಧ್ವನಿ ಬರುವುದು. ನಾನು ಅದನ್ನು ಕೇಳುತ್ತಿರುವೆನು. ಬೇಕಾದರೆ ಅದನ್ನು ಕೇಳುವುದಕ್ಕೆ, ಬೇಡವಾದರೆ ಅದನ್ನು ಬಿಡುವುದಕ್ಕೆ ನನಗೆ ಸ್ವಾತಂತ್ರ್ಯವಿರುವುದೇನು? ಪ್ರಕೃತಿ ದುಃಖದ ಆಳಕ್ಕೆ ಹೋಗು ಎನ್ನುವುದು. ನಾನು ತಕ್ಷಣ ದುಃಖಕ್ಕೆ ಒಳಗಾಗುವೆನು. ನಾವು ಇಂದ್ರಿಯಸುಖ ಮತ್ತು ಐಶ್ವರ್ಯಗಳನ್ನು ಕುರಿತು ಮಾತನಾಡುವೆವು. ಒಬ್ಬ ನಾನು ತುಂಬಾ ಬುದ್ದಿವಂತನೆಂದು ಭಾವಿಸುತ್ತಾನೆ. ಮತ್ತೊಬ್ಬನನ್ನು ಮೂರ್ಖ ಎನ್ನುವನು. ಏನೂ ಗೊತ್ತಿಲ್ಲ. ಇದೇ ಅಧೋಗತಿ, ಇದೇ ಗುಲಾಮಗಿರಿ. ಒಂದು ಕತ್ತಲೆಯ ಕೋಣೆಯಲ್ಲಿ ಯಾರಿಗೂ ಏನೂ ಗೊತ್ತಿಲ್ಲದೆ ಒಬ್ಬರ ತಲೆಯನ್ನು ಇನ್ನೊಬ್ಬರ ತಲೆಗೆ ಬಡಿಯುತ್ತಿದ್ದೇವೆ.

ಧ್ಯಾನ ಎಂದರೆ ಏನು? ಧ್ಯಾನ ಎಂದರೆ ಮೇಲೆ ಹೇಳಿದವುಗಳನ್ನೆಲ್ಲಾ ಎದುರಿಸುವ ಶಕ್ತಿ. ಪ್ರಕೃತಿ, ನೋಡು ಅಲ್ಲೊಂದು ಚೆನ್ನಾದ ವಸ್ತುವಿದೆ ಎಂದು ಹೇಳಬಹುದು. ನಾನು ನೋಡುವುದಿಲ್ಲ. ಇಲ್ಲಿ ಚೆನ್ನಾದ ಗಂಧವಿದೆ ಅದನ್ನು ಮೂಸಿನೋಡು ಎನ್ನಬಹುದು. ನಾನು ನನ್ನ ಮೂಗಿಗೆ ಅದನ್ನು ಮೂಸಿ ನೋಡಬೇಡ ಎಂದರೆ ಅದನ್ನು ನೋಡುವುದಿಲ್ಲ. ಕಣ್ಣುಗಳು ನೋಡುವುದಿಲ್ಲ. ಪ್ರಕೃತಿ ಮತ್ತೊಂದು ಭಯಾನಕವಾದ\break ಪ್ರಸಂಗವನ್ನು ತರುವುದು. ನನ್ನ ಮಕ್ಕಳಲ್ಲಿ ಒಂದನ್ನು ಕೊಲ್ಲುವುದು. ಹೇ ಮೂರ್ಖ, ಈಗ ಕುಳಿತುಕೊಂಡು ಅಳು, ದುಃಖದ ಆಳಕ್ಕೆ ಹೋಗು ಎನ್ನುವುದು. ಇಲ್ಲ ನಾನು ಹಾಗೆ ಮಾಡುವುದಿಲ್ಲ ಎನ್ನುತ್ತೇನೆ. ನಾನು ಅಲ್ಲಿಂದ ನೆಗೆಯುತ್ತೇನೆ. ನಾನು ಸ್ವತಂತ್ರನಾಗಬೇಕು. ನೀವು ಕೆಲವು ವೇಳೆ ಇದನ್ನು ಅಭ್ಯಾಸಮಾಡಿ. ಧ್ಯಾನದಲ್ಲಿ ನಿಮ್ಮ ಸ್ವಭಾವವನ್ನು ಬದಲಾಯಿಸಬಹುದು. ನಿಮಗೆ ಅಂತಹ ಶಕ್ತಿ ಇದ್ದರೆ ಅದೇ ಸ್ವಾತಂತ್ರ್ಯವಲ್ಲವೆ? ಸ್ವರ್ಗವಲ್ಲವೆ? ಇದೇ ಧ್ಯಾನದ ಶಕ್ತಿ.

ನಾವು ಇದನ್ನು ಹೇಗೆ ಪಡೆಯಬಹುದು? ಹಲವು ರೀತಿ ನಾವು ಇದನ್ನು ಪಡೆಯಬಹುದು. ಪ್ರತಿಯೊಂದು ಸ್ವಭಾವದವರಿಗೂ ಒಂದೊಂದು ಮಾರ್ಗವಿದೆ. ಆದರೆ ಅವುಗಳಲ್ಲೆಲ್ಲಾ ಸಾಮಾನ್ಯವಾದ ಒಂದು ವಿಷಯ ಇದೆ. ಅದೇ ಮನೋನಿಗ್ರಹ, ಮನಸ್ಸು ಒಂದು ಸರೋವರದಂತೆ ಇದೆ. ಅದರಲ್ಲಿ ಹಾಕಿದ ಪ್ರತಿಯೊಂದು ಕಲ್ಲಿನಿಂದಲೂ ಅಲೆಗಳು ಏಳುವುವು. ಈ ಅಲೆಗಳು ನಮ್ಮ ಸ್ವಭಾವವನ್ನು ನೋಡುವುದಕ್ಕೆ ಬಿಡುವುದಿಲ್ಲ. ಸರೋವರದ ಮೇಲೆ ಹುಣ್ಣಿಮೆಯ ಚಂದ್ರ ಪ್ರಕಾಶಿಸುತ್ತಿರುವನು. ಆದರೆ ಸರೋವರದ ಮೇಲಿನ ಭಾಗ ಪ್ರಕ್ಷುಬ್ದವಾಗಿರುವುದರಿಂದ ಸರಿಯಾದ ಪ್ರತಿಬಿಂಬ ಬೀಳುವುದಿಲ್ಲ. ಅದು ಪ್ರಶಾಂತವಾಗಲಿ. ಪ್ರಕೃತಿ ಅಲೆಯನ್ನು ಎಬ್ಬಿಸದಂತೆ ನೋಡಿಕೊಳ್ಳಿ. ನೀವು ಸುಮ್ಮನೆ ಇರಿ. ಸ್ವಲ್ಪಕಾಲವಾದ ಮೇಲೆ ಅದು ನಿಮ್ಮನ್ನು ಬಿಟ್ಟುಬಿಡುವುದು. ಆಗ ನೀವು ಯಾರು ಎಂಬುದು ಗೊತ್ತಾಗುವುದು. ದೇವರು ಆಗಲೆ ಇಲ್ಲಿರುವನು, ಆದರೆ ಮನಸ್ಸು ಚಂಚಲವಾಗಿದೆ. ಅದು ಯಾವಾಗಲೂ ಇಂದ್ರಿಯಗಳ ಕಡೆ ಧಾವಿಸುತ್ತಿದೆ. ನೀವು ಇಂದ್ರಿಯಗಳ ಬಾಗಿಲನ್ನು ಮುಚ್ಚಿದರೂ ಮನಸ್ಸು ತೆಪ್ಪಗಿರುವುದಿಲ್ಲ. ಈ ಕ್ಷಣ ನಾನು ಚೆನ್ನಾಗಿರುವೆನು, ದೇವರನ್ನು ಕುರಿತು ಚಿಂತಿಸುವೆನು ಎಂದು ಆಲೋಚನೆ ಮಾಡುತ್ತೇನೆ. ಒಂದು ಕ್ಷಣದಲ್ಲಿ ನನ್ನ ಮನಸ್ಸು ಲಂಡನ್ನಿಗೆ ಹಾರುವುದು. ಅಲ್ಲಿಂದ ಅದನ್ನು ಸೆಳೆದುಕೊಂಡು ಬಂದರೆ ನ್ಯೂಯಾರ್ಕಿಗೆ ಅದು ಹಾರುವುದು. ನಾನು ಹಿಂದೆ ಅಲ್ಲಿ ಏನು ಮಾಡಿದ್ದೆನೋ ಅದನ್ನು ಆಲೋಚಿಸುವುದು. ಧ್ಯಾನದ ಶಕ್ತಿಯಿಂದ ನಾವು ಚಿತ್ತವೃತ್ತಿಯನ್ನು ನಿಲ್ಲಿಸಬೇಕು.

ನಿಧಾನವಾಗಿ ಕ್ರಮೇಣ ನಾವು ಸಾಧನೆಯನ್ನು ಮಾಡಬೇಕಾಗಿದೆ. ಅದೇನು ತಮಾಷೆಯಲ್ಲ. ಒಂದು ದಿನ ಅಥವಾ ಕೆಲವು ವರುಷಗಳ ವಿಷಯವೂ ಅಲ್ಲ. ಕೆಲವು ಜನ್ಮಗಳನ್ನೆ ಅದಕ್ಕಾಗಿ ಮುಡಿಪಾಗಿ ಇಡಬೇಕಾಗುವುದು. ಚಿಂತೆಯಿಲ್ಲ. ಸಾಧನೆಯನ್ನು ಬಿಡಬಾರದು. ನಾವು ತಿಳಿದುಕೊಂಡು ಹೃತ್ಪೂರ್ವಕ ಸಾಧನೆಯನ್ನು ಮಾಡುತ್ತಿರಬೇಕು. ಅಂಗುಲ ಅಂಗುಲ ನಾವು ಮುಂದುವರಿಯುತ್ತೇವೆ. ನಾವು ಕ್ರಮೇಣ ಆ ಸ್ಥಿತಿಯನ್ನು ಪಡೆಯುತ್ತೇವೆ. ಯಾವ ಐಶ್ವರ್ಯವನ್ನು ಯಾರೂ ತೆಗೆದುಕೊಳ್ಳಲಾರರೊ, ಯಾವ ಐಶ್ವರ್ಯವನ್ನು ಯಾರೂ ನಾಶಮಾಡಲಾರರೊ, ಯಾವ ಆನಂದವನ್ನು ಪ್ರಪಂಚದ ಯಾವ ದುಃಖವೂ ನಾಶಮಾಡಲಾರದೊ ಅದು ನಮ್ಮದಾಗುವುದು.

ಇಷ್ಟು ವರುಷಗಳೂ ನಾವು ಇತರರನ್ನು ನೆಚ್ಚಿಕೊಂಡಿದ್ದೆವು. ಯಾವುದಾದರೂ ವ್ಯಕ್ತಿಯ ದೆಸೆಯಿಂದ ನನಗೆ ಸ್ವಲ್ಪ ಸಂತೋಷ ಆಗಿದ್ದರೆ ಆ ವ್ಯಕ್ತಿ ಹೋದಮೇಲೆ ಆ ಸಂತೋಷವೂ ಹೋಗುವುದು. ಮನುಷ್ಯನ ಹುಚ್ಚುತನವನ್ನು ನೋಡಿ, ಅವನು ತನ್ನ ಸುಖಕ್ಕೆ ಇತರರನ್ನು ಆಶ್ರಯಿಸಿರುವನು. ಎಲ್ಲಾ ವಿಯೋಗವೂ ದುಃಖವೇ. ಸ್ವಾಭಾವಿಕವಾಗಿಯೇ ಇದೆ ಇದು. ನಿಮ್ಮ ಸುಖಕ್ಕೆ ಐಶ್ವರ್ಯವನ್ನು ನೆಚ್ಚಿಕೊಂಡಿರುವಿರೇನು? ಐಶ್ವರ್ಯಕ್ಕೆ ಏಳು ಬೀಳುಗಳಿವೆ. ಅವಿಕಾರಿಯಾದ ಆತ್ಮನನ್ನು ಅಲ್ಲದೆ ಆರೋಗ್ಯ ಮುಂತಾದ ಯಾವುದನ್ನೂ ನೆಚ್ಚಿದ್ದರೂ ಇಂದೋ ನಾಳೆಯೋ ದುಃಖ ನಮಗೆ ತಪ್ಪಿದ್ದಲ್ಲ.

ಅನಂತಾತ್ಮನಲ್ಲದೆ ಪ್ರಪಂಚದಲ್ಲಿ ಪ್ರತಿಯೊಂದು ವಸ್ತುವೂ ಬದಲಾಯಿಸುತ್ತಿದೆ. ಬದಲಾವಣೆಯ ಸುಂಟರಗಾಳಿ ಬೀಸುತ್ತಿದೆ. ನಿಮ್ಮಲ್ಲಿ ಅಲ್ಲದೆ ಬೇರೆ ಎಲ್ಲಿಯೂ ಶಾಶ್ವತವಾದುದು ಇಲ್ಲ. ಇಲ್ಲಿ ಮಾತ್ರ ಬದಲಾಯಿಸದ ಅನಂತ ಆನಂದವಿದೆ. ಧ್ಯಾನ ಅದನ್ನು ತೆರೆಯುವ ಬಾಗಿಲು. ಪ್ರಾರ್ಥನೆ, ಆಚಾರಗಳು ಮತ್ತು ಎಲ್ಲಾ ವಿಧವಾದ ಬಾಹ್ಯಪೂಜೆಗಳೂ ಧ್ಯಾನಕ್ಕೆ ನಮ್ಮನ್ನು ಒಯ್ಯುವ ಸಾಧನಗಳಾಗಿವೆ. ನೀವು ಪ್ರಾರ್ಥಿಸುವಿರಿ. ನೀವು ಏನನ್ನೊ ಕೊಡುತ್ತೀರಿ. ಪ್ರತಿಯೊಂದೂ ನಮ್ಮ ಆತ್ಮಶಕ್ತಿಯನ್ನು ಉದ್ದೀಪನಗೊಳಿಸುವುದು ಎಂಬ ಒಂದು ಸಿದ್ದಾಂತವಿತ್ತು. ಕೆಲವು ಮಂತ್ರಗಳನ್ನು ಉಚ್ಚರಿಸುವುದು, ಹೂವು, ವಿಗ್ರಹ, ದೇವಸ್ಥಾನ, ದೇವರಿಗೆ ಮಂಗಳಾರತಿ ಮಾಡುವುದು ಇವುಗಳೆಲ್ಲ ನಮ್ಮನ್ನು ಧ್ಯಾನಾವಸ್ಥೆಗೆ ತರುವುವು. ಆದರೆ ಆ ಸ್ಥಿತಿ ಯಾವಾಗಲೂ ನಮ್ಮಲ್ಲಿಯೇ ಇದೆ. ಹೊರಗೆ ಇಲ್ಲ. ಜನರೆಲ್ಲ ಇದನ್ನು ಮಾಡುತ್ತಿರುವರು. ಯಾವುದನ್ನು ಅವರು ತಿಳಿಯದೆ ಮಾಡುತ್ತಿರುವರೋ ಅದನ್ನು ನೀವು ತಿಳಿದು ಮಾಡಿ. ಇದೇ ಧ್ಯಾನದ ಶಕಿ. ನಿಮ್ಮಲ್ಲಿರುವ ಜ್ಞಾನವೆಲ್ಲ ಹೇಗೆ ಬಂದಿತು? ಧ್ಯಾನಶಕ್ತಿಯಿಂದ, ಜೀವ ತನ್ನ ಅಂತರಾಳದಿಂದಲೇ ಅದನ್ನು ಮಥಿಸಿ ಮೇಲೆ ತೆಗೆಯಿತು. ಯಾವ ಜ್ಞಾನವೂ ಮನಸ್ಸಿನ ಹೊರಗೆ ಇರಲಿಲ್ಲ? ಧ್ಯಾನಶಕ್ತಿ ನಮ್ಮನ್ನು ಕ್ರಮೇಣ ದೇಹಭಾವದಿಂದ ಪಾರುಮಾಡುವುದು. ಆಗ ಆತ್ಮ ತನ್ನ ನೈಜಸ್ಥಿತಿಯನ್ನು, ಅಜನಾದ ಜನನ ಮರಣಗಳಿಲ್ಲದವನೇ ನಾನು ಎಂಬುದನ್ನು ಅರಿಯುವುದು. ಅನಂತರ ಈ ಲೋಕದಲ್ಲಿ ಇನ್ನು ಮೇಲೆ ದುಃಖವಿರುವುದಿಲ್ಲ, ಜನನವಿಲ್ಲ, ವಿಕಾಸವೂ ಇಲ್ಲ. ತಾನು ಯಾವಾಗಲೂ ಪರಿಪೂರ್ಣ ಮತ್ತು ನಿತ್ಯಮುಕ್ತ ಎಂಬುದನ್ನು ಆತ್ಮ ಅರಿಯುವುದು.

