
\chapter{ಯಾಜ್ಞವಲ್ಕ್ಯ ಮತ್ತು ಮೈತ್ರೇಯೀ\protect\footnote{\enginline{* C.W. Vol. II, P. 416}}}

“ಮೇಘಾಚ್ಛಾದಿತ ದಿನವಲ್ಲ ದುರ್ದಿನ, ಭಗವಂತನ ಹೆಸರನ್ನು ಕೇಳದ ದಿನ ದುರ್ದಿನ'' ಎಂದು ಹೇಳುತ್ತೇವೆ. ಯಾಜ್ಞವಲ್ಕ್ಯ ಮಹಾಜ್ಞಾನಿ. ಭರತಖಂಡದಲ್ಲಿ ವ್ಯಕ್ತಿಯು ವಯಸ್ಸಾದ ಮೇಲೆ ಪ್ರಪಂಚವನ್ನು ತ್ಯಜಿಸಬೇಕೆಂದು ಶಾಸ್ತ್ರಗಳು ಸಾರುತ್ತವೆ. ಯಾಜ್ಞವಲ್ಕ್ಯ ತನ್ನ ಸತಿಗೆ ಒಂದು ದಿನ 'ಪ್ರಿಯಳೆ, ನನ್ನ ದ್ರವ್ಯ, ಆಸ್ತಿಗಳನ್ನೆಲ್ಲ ನಿನಗೆ ಬಿಟ್ಟು ಹೋಗುವೆ'' ಎಂದ. ಅದಕ್ಕೆ ಅವಳು “ಸ್ವಾಮಿ, ನನಗೆ ಜಗತ್ತಿನ ಐಶ್ವರ್ಯವೆಲ್ಲ ದೊರೆತರೂ ಅದರಿಂದ ನನಗೆ ಮುಕ್ತಿ ದೊರಕುವುದೆ?'' ಎಂದು ಕೇಳಿದಳು. ''ಇಲ್ಲ. ಅದರಿಂದ ನೀನು ಶ‍್ರೀಮಂತಳಾಗುವೆ ಅಷ್ಟೆ. ಐಶ್ವರ್ಯ ನಮಗೆ ಮುಕ್ತಿ ದೊರಕಿಸಲಾರದು'' ಎಂದ. ಆಕೆ 'ನಾನು ಯಾವುದರಿಂದ ಮುಕ್ತಿಯನ್ನು ಪಡೆಯಬಹುದು? ಗೊತ್ತಿದ್ದರೆ ತಿಳಿಸಿ” ಎಂದಳು. ಯಾಜ್ಞವಲ್ಕ್ಯ ಹೀಗೆಂದನು: “ನೀನಾವಾಗಲೂ ನನಗೆ ಪ್ರಿಯಳಾಗಿದ್ದೆ. ಈ ಪ್ರಶ್ನೆಯಿಂದ ಮತ್ತೂ ಪ್ರಿಯಳಾದೆ. ಬಾ ಕುಳಿತುಕೊ, ಹೇಳುತ್ತೇನೆ. ಅದನ್ನು ಕೇಳಿ ಆದಮೇಲೆ ಮನನ ಮಾಡು. ಗಂಡನಿಗಾಗಿ ಅಲ್ಲ ಹೆಂಡತಿ ಗಂಡನನ್ನು ಪ್ರೀತಿಸುವುದು, ಆತ್ಮನಿಗಾಗಿ ಅವಳು ಗಂಡನನ್ನು ಪ್ರೀತಿಸುವಳು. ಯಾವ ಗಂಡನೂ ಹೆಂಡತಿಗಾಗಿ ಹೆಂಡತಿಯನ್ನು ಪ್ರೀತಿಸುವುದಿಲ್ಲ. ಆತ್ಮನನ್ನು ಪ್ರೀತಿಸುವುದರಿಂದ ಅದಕ್ಕಾಗಿ ಅವಳನ್ನು ಪ್ರೀತಿಸುತ್ತಾನೆ. ಮಕ್ಕಳಿಗಾಗಿ ಅಲ್ಲ ಮಕ್ಕಳನ್ನು ಪ್ರೀತಿಸುವರು. ಐಶ್ವರ್ಯಕ್ಕಾಗಿ ಯಾರೂ ಐಶ್ವರ್ಯವನ್ನು ಪ್ರೀತಿಸುವುದಿಲ್ಲ, ಆತ್ಮನಿಗಾಗಿ ಐಶ್ವರ್ಯವನ್ನು ಪ್ರೀತಿಸುವರು. ಬ್ರಾಹ್ಮಣನಿಗಾಗಿ ಬ್ರಾಹ್ಮಣನನ್ನು ಯಾರೂ ಪ್ರೀತಿಸುವುದಿಲ್ಲ, ಆತ್ಮನಿಗಾಗಿ ಬ್ರಾಹ್ಮಣನನ್ನು ಪ್ರೀತಿಸುವರು. ಕ್ಷತ್ರಿಯನಿಗಾಗಿ ಯಾರೂ ಕ್ಷತ್ರಿಯನನ್ನು ಪ್ರೀತಿಸುವುದಿಲ್ಲ, ಆತ್ಮನಿಗಾಗಿ ಕ್ಷತ್ರಿಯನನ್ನು ಪ್ರೀತಿಸುವರು. ಪ್ರಪಂಚಕ್ಕಾಗಿ ಯಾರೂ ಪ್ರಪಂಚವನ್ನು ಪ್ರೀತಿಸುವುದಿಲ್ಲ, ಆತ್ಮನಿಗಾಗಿ ಪ್ರಪಂಚವನ್ನು ಪ್ರೀತಿಸುವರು. ಇದರಂತೆಯೇ ದೇವತೆಗಳಿಗಾಗಿ ದೇವತೆಗಳನ್ನು ಯಾರೂ ಪ್ರೀತಿಸುವುದಿಲ್ಲ, ಆತ್ಮನಿಗಾಗಿ ಅವರನ್ನು ಪ್ರೀತಿಸುವರು. ಯಾರೂ ವಸ್ತುವಿಗೋಸ್ಕರ ಅದನ್ನು ಪ್ರೀತಿಸುವುದಿಲ್ಲ. ಆತ್ಮಪ್ರೀತಿಗೋಸುಗ ಅದನ್ನು ಪ್ರೀತಿಸುವರು. ಈ ಆತ್ಮನ ವಿಷಯವಾಗಿ ಶ್ರವಣ ಮಾಡಬೇಕು, ಮನನ ಮಾಡಬೇಕು, ಧ್ಯಾನ ಮಾಡಬೇಕು. ಓ ಮೈತ್ರೇಯೀ, ಆತ್ಮನ ವಿಷಯವನ್ನು ಕೇಳಿದ ಮೇಲೆ ಅದನ್ನು ನೋಡಿದ ಮೇಲೆ ಸಾಕ್ಷಾತ್ಕಾರ ಮಾಡಿಕೊಂಡಮೇಲೆ ಇದನ್ನೆಲ್ಲ ತಿಳಿದುಕೊಳ್ಳಬಹುದು". ನಮಗೆ ಇದರಿಂದೇನು ದೊರಕುವುದು? ನಮ್ಮೆದುರಿಗೆ ಈ ವಿಚಿತ್ರ ಸಿದ್ದಾಂತವಿದೆ. ಅದೇ ಪ್ರತಿಯೊಂದು ಪ್ರೀತಿಯೂ ಅತಿ ಸ್ವಾರ್ಥವಾದುದು, ನಾನು ನನ್ನನ್ನು ಪ್ರೀತಿಸುವುದರಿಂದ ಮತ್ತೊಬ್ಬರನ್ನು ಪ್ರೀತಿಸುತ್ತೇನೆ ಎಂಬುದು. ಇದು ಹಾಗಲ್ಲ. ಆಧುನಿಕ ಕಾಲದಲ್ಲಿ ಕೆಲವು ತಜ್ಞರಿರುವರು. ಅವರು ಪ್ರಪಂಚದಲ್ಲಿ ಸ್ವಾರ್ಥವೊಂದೇ ಕ್ರಿಯೋತ್ತೇಜನ ಶಕ್ತಿ ಎಂದು ಹೇಳುವರು. ಇದು ನಿಜ, ಆದರೂ ತಪ್ಪು. ಈ ಸ್ವಾರ್ಥದ ಹಿಂದೆ ಇರುವುದು ನಿಜವಾದ ಆತ್ಮನ ಛಾಯೆ ಮಾತ್ರ. ಇದು ಅಲ್ಪವಾಗಿರುವುದರಿಂದ, ತಪ್ಪಾಗಿ ಕಾಣುವುದು, ದೋಷವಾಗಿ ಕಾಣುವುದು. ವಿಶ್ವವೇ ಆಗಿರುವ ಆತ್ಮನ ವಿಷಯಕ್ಕಿರುವ ಅನಂತ ಪ್ರೇಮ ದೋಷದಂತೆ ಕಾಣುವುದು. ಅಲ್ಪದಂತೆ ಕಾಣುವುದು, ಏಕೆಂದರೆ ಇದು ಅಲ್ಪದ ಮೂಲಕ ವ್ಯಕ್ತವಾಗುತ್ತಿರುವುದು. ಸತಿಯು ಪತಿಯನ್ನು ಪ್ರೀತಿಸುವಾಗಲೂ ಅವಳಿಗೆ ಗೊತ್ತಿರಲಿ ಇಲ್ಲದೆ ಇರಲಿ, ಆತ್ಮಾರ್ಥವಾಗಿ ಅವಳು ಪತಿಯನ್ನು ಪ್ರೀತಿಸುವಳು. ನಮಗೆ ಈಗ ಪ್ರಪಂಚದಲ್ಲಿ ವ್ಯಕ್ತವಾಗುತ್ತಿರುವುದು ಸ್ವಾರ್ಥ. ಆದರೆ ಆ ಸ್ವಾರ್ಥವು ಅನಂತತ್ವದ ಸ್ವಲ್ಪ ಭಾಗ ಮಾತ್ರ. ಎಂದಾದರೂ ಒಬ್ಬನು ಪ್ರೀತಿಸಿದರೆ ಅವನು ಆತನ ಮೂಲಕ ಮಾತ್ರ ಪ್ರೀತಿಸಲು ಸಾಧ್ಯ. ಈ ಆತ್ಮನನ್ನು ನಾವು ಅರಿಯಬೇಕು. ಏನು ವ್ಯತ್ಯಾಸ? ಆತ್ಮನ ಅರಿವಿಲ್ಲದೆ ಯಾರು ಪ್ರೀತಿಸುವರೊ ಅದು ಸ್ವಾರ್ಥಪ್ರೀತಿ. ಆತ್ಮನನ್ನು ಅರಿತು ಯಾರು ಪ್ರೀತಿಸುವರೊ ಅವರದು ನಿಃಸ್ವಾರ್ಥಪ್ರೇಮ. ಅವರೇ ಜ್ಞಾನಿಗಳು, “ಯಾರು ಆತ್ಮನ ವಿನಃ ಬೇರೆಡೆ ಬ್ರಾಹ್ಮಣನನ್ನು ನೋಡುವರೋ ಅವರನ್ನು ಬ್ರಾಹ್ಮಣ ತ್ಯಜಿಸುವನು. ಯಾರು ಆತ್ಮನ ವಿನಃ ಬೇರೆ ಕ್ಷತ್ರಿಯನನ್ನು ನೋಡುವರೊ ಅವರನ್ನು ಕ್ಷತ್ರಿಯರು ತ್ಯಜಿಸುವರು. ಆತ್ಮನಲ್ಲಿ ಅಲ್ಲದೆ ಬೇರೆ ಕಡೆ ಪ್ರಪಂಚವನ್ನು ನೋಡುವವನನ್ನು ಪ್ರಪಂಚ ತ್ಯಜಿಸುವುದು. ಆತ್ಮನಲ್ಲಿ ಅಲ್ಲದೆ ಬೇರೆ ಕಡೆ ದೇವತೆಗಳನ್ನು ಪ್ರೀತಿಸುವವನನ್ನು ದೇವತೆಗಳು ತ್ಯಜಿಸುವರು. ಯಾರು ಎಲ್ಲವನ್ನೂ ಆತ್ಮನಿಂದ ಬೇರೆ ಎಂದು ಭಾವಿಸುವರೋ ಅವರನ್ನು ಎಲ್ಲವೂ ತ್ಯಜಿಸುವುದು. ಬ್ರಾಹ್ಮಣ, ಕ್ಷತ್ರಿಯ, ಪ್ರಪಂಚ ದೇವತೆಗಳು ಮತ್ತು ಪ್ರಪಂಚದಲ್ಲಿ ಇರುವುದೆಲ್ಲ ಆತ್ಮ” ಪ್ರೀತಿ ಎಂದರೆ ಏನೆಂಬುದನ್ನು ಹೀಗೆ ಯಾಜ್ಞವಲ್ಕ್ಯನು ವಿವರಿಸಿದ್ದಾನೆ.

ನಾವು ಒಂದು ವಸ್ತುವನ್ನು ಪ್ರತ್ಯೇಕವಾಗಿ ನಿರ್ದೇಶಿಸುವಾಗಲೆಲ್ಲ ಅದನ್ನು ಆತ್ಮನಿಂದ ಬೇರ್ಪಡಿಸುವೆವು. ನಾನು ಒಬ್ಬ ಸ್ತ್ರೀಯನ್ನು ಪ್ರೀತಿಸಲು ಯತ್ನಿಸುತ್ತಿರುವೆನು. ನಾನು ಅವಳನ್ನು ಪ್ರತ್ಯೇಕಗೊಳಿಸಿದೊಡನೆಯೆ ಅವಳು ಆತ್ಮನಿಂದ ಬೇರೆ ಆಗುವಳು. ಅವಳ ಮೇಲಿರುವ ನನ್ನ ಪ್ರೀತಿ ಅನಂತ ವಾಗಲಾರದು. ಅದು ದುಃಖದಲ್ಲಿ ಪರ್ಯವಸಾನವಾಗುವುದು. ನಾನು ಅವಳನ್ನು ಆತ್ಮನ ದೃಷ್ಟಿಯಿಂದ ನೋಡಿದಾಗ ಆ ಪ್ರೀತಿ ಪೂರ್ಣವಾಗುವುದು, ಎಂದಿಗೂ ವ್ಯಥೆಗೆ ಸಿಕ್ಕುವುದಿಲ್ಲ. ಇದರಂತೆಯೆ ಪ್ರತಿಯೊಂದು ವಸ್ತುವೂ ಕೂಡ. ಪ್ರಪಂಚದಲ್ಲಿ ನೀವು ಯಾವುದಾದರೂ ಒಂದು ವಸ್ತುವಿನಲ್ಲಿ ಆಸಕ್ತರಾಗಿ ಅದನ್ನು ಪೂರ್ಣದಿಂದ, ಆತ್ಮದಿಂದ ಬೇರ್ಪಡಿಸಿದರೆ ಅದರಿಂದ ಪ್ರತಿಕ್ರಿಯೆ ಉಂಟಾಗುವುದು. ಆತ್ಮನ ಮೂಲಕವಾಗಿ ಅಲ್ಲದೆ ನಾವು ಯಾವುದನ್ನು ಪ್ರೀತಿಸಿದರೂ ದುಃಖ ಸಂಕಟಗಳೇ ಪ್ರತಿಫಲ. ನಾವು ಎಲ್ಲವನ್ನೂ ಆತ್ಮನ ಮೂಲಕ ಅನುಭವಿಸಿದರೆ, ಆತ್ಮನಂತೆ ಅನುಭವಿಸಿದರೆ ಯಾವ ದುಃಖವಾಗಲಿ ಪ್ರತಿಕ್ರಿಯೆಯಾಗಲಿ ಬರುವುದಿಲ್ಲ. ಇದೇ ಪೂರ್ಣಾನಂದ. ಈ ಆದರ್ಶವನ್ನು ತಲುಪುವುದು ಹೇಗೆ, ಇದನ್ನು ಹೇಗೆ ಪಡೆಯಬೇಕು ಎಂಬುದನ್ನು ಯಾಜ್ಞವಲ್ಕ್ಯ ಹೇಳುವನು. ಈ ವಿಶ್ವ ಅನಂತವಾಗಿದೆ. ನಾವು ಇದರಲ್ಲಿ ಯಾವುದಾದರೊಂದು ಪ್ರತ್ಯೇಕ ವಸ್ತುವನ್ನು ತೆಗೆದುಕೊಂಡು ಆತ್ಮನ ಅರಿವಿಲ್ಲದೆ ಅದನ್ನು ಆತ್ಮನೆಂದು ಹೇಗೆ ನೋಡುವುದು? 'ನಾವು ಒಂದು ಮೃದಂಗದಿಂದ ದೂರವಿರುವವರೆಗೆ ನಮಗೆ ಧ್ವನಿ ಕೇಳಿಸುವುದಿಲ್ಲ. ಧ್ವನಿ ನಮ್ಮ ಸ್ವಾಧೀನವಲ್ಲ. ನಾವು ಅದರ ಸಮೀಪಕ್ಕೆ ಬಂದು ಕೈಯನ್ನು ಅದರ ಮೇಲೆ ಇಟ್ಟೊಡನೆಯೇ ಧ್ವನಿಯನ್ನು ನಾವು ಗೆಲ್ಲುವೆವು. ಶಂಖವನ್ನು ಊದಿದಾಗ ಧ್ವನಿಯನ್ನು ಹಿಡಿಯಲಾರೆವು, ಅದನ್ನು ಸ್ವಾಧೀನಮಾಡಿಕೊಳ್ಳಲಾರೆವು. ನಾವು ಸಮೀಪಕ್ಕೆ ಬಂದು ಶಂಖವನ್ನು ಸ್ವಾಧೀನಪಡಿಸಿಕೊಂಡಾಗ ಅದನ್ನು ಗೆಲ್ಲುವೆವು. ವೀಣೆಯನ್ನು ನುಡಿಸುತ್ತಿರುವಾಗ, ನಾವು ವೀಣೆಯನ್ನು ಸಮೀಪಿಸಿದರೆ ಆಗ ಧ್ವನಿ ಬರುತ್ತಿರುವ ಕೇಂದ್ರಕ್ಕೆ ಬರುವೆವು. ಒಬ್ಬ ಹಸಿ ಸೌದೆಯನ್ನು ಉರಿಸುತ್ತಿದ್ದರೆ ಹಲವು ಬಗೆ ಹೊಗೆಗಳು, ಕಿಡಿ ಬರುವುವು. ಇದರಂತೆಯೆ ಆ ಮಹಾಸತ್ಯದಿಂದ ಜ್ಞಾನ ವ್ಯಕ್ತವಾಗಿದೆ, ಪ್ರತಿಯೊಂದೂ ಅದರಿಂದ ಬಂದಿದೆ. ಎಲ್ಲ ಜ್ಞಾನವನ್ನೂ ಅವನೇ ಹೊರಗೆಡವಿದಂತಿದೆ. ನೀರಿಗೆಲ್ಲ ಸಾಗರ ಗುರಿಯಾದಂತೆ, ಎಲ್ಲ ಸ್ಪರ್ಶಕ್ಕೆ ಚರ್ಮ ಗುರಿಯಾದಂತೆ, ವಾಸನೆಗೆಲ್ಲ ಮೂಗು ಗುರಿಯಾದಂತೆ, ರುಚಿಗೆಲ್ಲ ನಾಲಗೆ ಗುರಿಯಾದಂತೆ, ಆಕಾರಕ್ಕೆಲ್ಲ ಕಣ್ಣು ಗುರಿಯಾದಂತೆ, ಶಬ್ದಕ್ಕೆಲ್ಲ ಕಿವಿ ಗುರಿಯಾದಂತೆ, ಆಲೋಚನೆಗೆಲ್ಲ ಮನಸ್ಸು ಗುರಿಯಾದಂತೆ, ಜ್ಞಾನಕ್ಕೆಲ್ಲ ಹೃದಯ ಗುರಿಯಾದಂತೆ, ಕರ್ಮಕ್ಕೆಲ್ಲ ಕೈಗಳು ಗುರಿಯಾದಂತೆ, ಸಮುದ್ರಕ್ಕೆ ಹಾಕಿದ ಉಪ್ಪು ಕರಗಿಹೋಗಿ ನಮಗೆ ದೊರಕದಂತೆ, ಇದರಂತೆಯೆ ಮೈತ್ರೇಯಿ, ವಿಶ್ವಾತ್ಮನು ನಿತ್ಯನಾಗಿರುವನು. ಜ್ಞಾನವೆಲ್ಲ ಅವನಲ್ಲಿದೆ. ವಿಶ್ವವೆಲ್ಲ ಅವನಿಂದ ಏಳುವುದು, ಪುನಃ ಅವನಲ್ಲಿಗೆ ಹೋಗುವುದು. ಆಗ ಅಜ್ಞಾನ, ಜನನ ಮರಣಗಳಾವುವೂ ಇಲ್ಲ". ನಾವೆಲ್ಲಾ ಪುರುಷನಿಂದ ಸಿಡಿದ ಕಿಡಿಗಳಂತೆ ತೋರುವೆವು, ನೀವು ಅವನನ್ನು ಅರಿತರೆ ಪುನಃ ಅಲ್ಲಿಗೆ ಹೋಗಿ ಐಕ್ಯರಾಗುವಿರಿ. ನಾವೇ ವಿಶ್ವಾತ್ಮ.

ಎಲ್ಲ ಕಡೆಯಲ್ಲಿ ಜನ ಅಂಜುವಂತೆ ಮೈತ್ರೇಯಿ ಅಂಜಿದಳು. ಹೀಗೆಂದಳು: 'ಸ್ವಾಮಿ, ಇಲ್ಲಿ ನನಗೆ ದೊಡ್ಡದೊಂದು ಭ್ರಾಂತಿಯನ್ನು ತಂದೊಡ್ಡಿದಿರಿ. ಇನ್ನು ದೇವತೆಗಳಿಲ್ಲ, ವ್ಯಕ್ತಿತ್ವವಿಲ್ಲ ಎಂದು ಅಂಜಿಸಿರುವಿರಿ. ಗುರುತು ಕಂಡುಹಿಡಿಯುವುದಕ್ಕೆ ಯಾರೂ ಇರುವುದಿಲ್ಲ. ಪ್ರೀತಿಸುವುದಕ್ಕೆ ಇರುವುದಿಲ್ಲ. ದ್ವೇಷಿಸುವುದಕ್ಕೆ ಇರುವುದಿಲ್ಲ. ನಮ್ಮ ಗತಿ ಏನಾಗುವುದು?'' ಅದಕ್ಕೆ ಯಾಜ್ಞವಲ್ಕ್ಯನೆಂದನು: “ಮೈತ್ರೇಯಿ, ನಿನಗೆ ಭ್ರಾಂತಿಯೊಡ್ಡಲು ಇಚ್ಛೆಯಿಲ್ಲ. ಇಲ್ಲಿಗೇ ಸಾಕು. ಅದು ನಿನಗೆ ಅಂಜಿಕೆಯಾಗಿರಬಹುದು. ಎಲ್ಲಿ ಇಬ್ಬರಿರುವರೊ ಅಲ್ಲಿ ಒಬ್ಬ ಮತ್ತೊಬ್ಬನನ್ನು ನೋಡುವನು, ಮತ್ತೊಬ್ಬನನ್ನು ಸ್ವಾಗತಿಸುವನು, ಮತ್ತೊಬ್ಬನನ್ನು ಯಾಚಿಸುವನು, ಮತ್ತೊಬ್ಬನನ್ನು ಯೋಚಿಸುವನು. ಆದರೆ ಎಲ್ಲಾ ಆತ್ಮವಾದಾಗ ಯಾರು ಯಾರನ್ನು ನೋಡುವರು? ಯಾರು ಯಾರನ್ನು ಕೇಳುವರು? ಯಾರು ಯಾರನ್ನು ಸ್ವಾಗತಿಸುವರು? ಯಾರು ಯಾರನ್ನು ತಿಳಿಯುವರು?” ಶೋಫನೇಯರ್ ಈ ಒಂದು ಭಾವನೆಯನ್ನು ಸ್ವೀಕರಿಸಿ ತನ್ನ ಸಿದ್ದಾಂತವನ್ನು ರೂಪಿಸಿದನು. ಯಾರ ಮೂಲಕ ನಮಗೆ ಈ ಪ್ರಪಂಚ ತಿಳಿಯುವುದು? ಯಾವುದರ ಮೂಲಕ ಅವನನ್ನು ತಿಳಿಯುವುದು? ತಿಳಿಯುವವನನ್ನು ತಿಳಿಯುವುದು ಹೇಗೆ? ಯಾವುದರಿಂದ ಅರಿಯುವವನ್ನು ಅರಿಯುವುದು? ಇದು ಹೇಗೆ ಸಾಧ್ಯ? ಏಕೆಂದರೆ ಅದರಿಂದ ಅದರ ಮೂಲಕ ಮಾತ್ರ ನಮಗೆಲ್ಲ ಅರಿವಾಗುತ್ತಿದೆ. ಯಾವುದರ ಮೂಲಕ ಅವನನ್ನು ತಿಳಿಯುವುದು? ಮತ್ಯಾವುದರಿಂದಲೂ ಅಲ್ಲ - ಅವನೇ ಮಾರ್ಗ.

ಇಲ್ಲಿಯವರೆಗೂ ತಿಳಿದಿದ್ದೆಂದರೆ ಇರುವುದು ಒಂದೇ ಅನಂತ ಸತ್ಯ ಎಂಬುದು. ಅದೇ ನಿಜವಾದ ವ್ಯಕ್ತಿತ್ವ, ಅಲ್ಲಿ ಭಾಗಗಳಿಲ್ಲ ಅಂಶಗಳಿಲ್ಲ. ಈ ಅಲ್ಪಭಾವನೆಗಳು ತುಂಬಾ ಕ್ಷುದ್ರವಾದುವು, ಭ್ರಾಮಿಕವಾದುವು. ಆದರೂ ಪ್ರತಿಯೊಂದು ವ್ಯಕ್ತಿತ್ವದ ಕಿಡಿಯ ಹಿಂದೆಯೂ ಆ ಅನಂತವು ಪ್ರಕಾಶಿಸುತ್ತಿರುವುದು, ಪ್ರತಿಯೊಂದೂ ಆತ್ಮನ ಅಭಿವ್ಯಕ್ತಿ. ಅದನ್ನು ಪಡೆಯುವುದು ಹೇಗೆ? ಯಾಜ್ಞವಲ್ಕ್ಯನು ಹೇಳಿದಂತೆ ಮೊದಲು ಆತ್ಮನ ವಿಷಯವನ್ನು ಕೇಳಬೇಕು. ಯಾಜ್ಞವಲ್ಕ್ಯನು ಮೊದಲು ವಿಷಯವನ್ನು ನಿರೂಪಿಸಿದನು. ಅನಂತರ ಅದನ್ನು ಚರ್ಚಿಸತೊಡಗಿದನು. ಯಾವುದರಿಂದ ಎಲ್ಲ ಜ್ಞಾನವೂ ಸಾಧ್ಯವೋ ಅದನ್ನು ತಿಳಿಯುವುದು ಹೇಗೆ ಎಂಬುದನ್ನು ಕೊನೆಗೆ ತೋರಿದನು. ಅನಂತರವೆ ಅದನ್ನು ಕುರಿತು ಧ್ಯಾನ ಮಾಡಬೇಕು. ಬ್ರಹ್ಮಾಂಡ ಪಿಂಡಾಂಡಗಳ ವ್ಯತ್ಯಾಸವನ್ನು ತೆಗೆದುಕೊಂಡು ಅವು ಹೇಗೆ ಒಂದು ಮಾರ್ಗದಲ್ಲಿ ಹೋಗುತ್ತಿವೆ, ಅವು ಎಷ್ಟು ಸುಂದರವಾಗಿವೆ ಎಂಬುದನ್ನು ತೋರುವನು. “ಈ ಪೃಥ್ವಿ ಎಲ್ಲರಿಗೂ ಆನಂದದಾಯಕವಾಗಿದೆ, ಸಹಾಯಕವಾಗಿದೆ. ಪ್ರತಿಯೊಬ್ಬರೂ ಈ ಭೂಮಿಗೆ ಸಹಾಯಕರಾಗಿರುವರು. ಇವುಗಳೆಲ್ಲ ಸ್ವಯಂ ಪ್ರಕಾಶಮಾನನಾದವನ ಅಭಿವ್ಯಕ್ತಿ.' ಇವೆಲ್ಲವೂ ಆನಂದ, ಅತಿಕೀಳು ಆನಂದ ಕೂಡ ಅವನ ಪ್ರತಿಬಿಂಬವೆ. ಒಳ್ಳೆಯದೆಲ್ಲ ಅವನ ಪ್ರತಿಬಿಂಬವೆ. ಆ ಪ್ರತಿಬಿಂಬ ಒಂದು ನೆರಳಾಗಿರುವಾಗ ಅದನ್ನು ಪಾಪವೆನ್ನುವರು. ಇಬ್ಬರು ದೇವರಿಲ್ಲ. ಅವನು ಕಡಿಮೆ ಪ್ರಕಾಶಿಸುತ್ತಿರುವಾಗ ಎಲ್ಲವೂ ಕತ್ತಲು, ಪಾಪಮಯವಾಗಿ ಕಾಣುವುದು; ಅವನು ಹೆಚ್ಚು ಪ್ರಕಾಶಿಸುತ್ತಿರುವಾಗ ಎಲ್ಲ ಪ್ರಕಾಶಮಯವಾಗಿ ಕಾಣುವುದು, ಅಷ್ಟೇ. ಒಳ್ಳೆಯದು ಕೆಟ್ಟದ್ದು ಎಂಬುದು ಕೇವಲ ತರತಮ ವಿಷಯ ಅಷ್ಟೆ - ಹೆಚ್ಚು ಅಥವಾ ಕಡಿಮೆ ವ್ಯಕ್ತವಾಗಿರುವುದು. ನಮ್ಮ ಜೀವನದ ಉದಾಹರಣೆಯನ್ನು ತೆಗೆದುಕೊಳ್ಳಿ. ಕೆಟ್ಟದ್ದಾದರೂ ಒಳ್ಳೆಯದೆಂದು ಭಾವಿಸುವ ಎಷ್ಟು ವಿಷಯಗಳನ್ನು ನಾವು ಬಾಲ್ಯದಲ್ಲಿ ನೋಡುವೆವು! ಎಷ್ಟು ಒಳ್ಳೆಯವು ಕೆಟ್ಟದ್ದರಂತೆ ಕಾಣುವುವು! ನಮ್ಮ ಭಾವನೆಗಳು ಹೇಗೆ ಬದಲಾಯಿಸುತ್ತವೆ! ಹೇಗೆ ಒಂದು ಭಾವನೆ ಬೆಳೆಯುತ್ತಾ ಬರುವುದು! ಒಂದು ಕಾಲದಲ್ಲಿ ನಾವು ಯಾವುದನ್ನು ಬಹಳ ಒಳ್ಳೆಯದೆಂದು ಭಾವಿಸಿದ್ದೆವೊ ಅದನ್ನು ಈಗ ಹಾಗೆ ಭಾವಿಸುವುದಿಲ್ಲ. ಒಳ್ಳೆಯದು ಕೆಟ್ಟದ್ದು ಎಂಬುದು ಕೇವಲ ನಮ್ಮ ಮೌಢ್ಯ, ಅವು ನಿಜವಾಗಿ ಇಲ್ಲ. ವ್ಯತ್ಯಾಸ ಹೆಚ್ಚು ಕಡಿಮೆಯಲ್ಲಿ ಮಾತ್ರ. ಇವೆಲ್ಲ ಆತ್ಮನ ಆವಿರ್ಭಾವ, ಅವನು ಎಲ್ಲಾ ಕಡೆಯಲ್ಲಿಯೂ ವ್ಯಕ್ತವಾಗಿರುವನು. ಆವಿರ್ಭಾವಕ್ಕೆ ಆತಂಕ ಹೆಚ್ಚಾದಾಗ ಪಾಪವೆಂದೂ, ಕಡಿಮೆಯಾದಾಗ ಪುಣ್ಯವೆಂದೂ ಕರೆಯುವೆವು. ಆವರಣಗಳೆಲ್ಲ ಕಳಚಿ ಬಿದ್ದಾಗಲೆ ಅದು ಶ್ರೇಷ್ಠ. ಮೊದಲು ಪ್ರಪಂಚದಲ್ಲಿರುವ ಪ್ರತಿಯೊಂದನ್ನೂ ಈ ದೃಷ್ಟಿಯಿಂದ ಧ್ಯಾನಿಸಬೇಕು. ನಾವು ಎಲ್ಲವನ್ನೂ ಒಳ್ಳೆಯದೆಂದು ಭಾವಿಸಬೇಕು. ಏಕೆಂದರೆ ಅದೇ ಶ್ರೇಷ್ಠ. ಒಳ್ಳೆಯದಿದೆ, ಕೆಟ್ಟದ್ದಿದೆ, ಅದರ ಕೇಂದ್ರದಲ್ಲಿ ಸತ್ಯವಿದೆ. ಅವನು ಒಳ್ಳೆಯವನೂ ಅಲ್ಲ, ಕೆಟ್ಟವನೂ ಅಲ್ಲ. ಅವನೇ ಪರಮ ಶ್ರೇಷ್ಠ. ಪರಮಶ್ರೇಷ್ಠವಾದುದು ಒಂದು ಮಾತ್ರ ಇರಲು ಸಾಧ್ಯ; ಒಳ್ಳೆಯದು ಕೆಟ್ಟದ್ದು ಹಲವು ಇರಬಹುದು. ಒಳ್ಳೆಯದು ಮತ್ತು ಕೆಟ್ಟದ್ದರಲ್ಲಿ ಅಂತರಗಳಿರಬಹುದು. ಆದರೆ ಪರಮ ಶ್ರೇಷ್ಠ ಒಂದೇ ಒಂದು. ಅದನ್ನು ಬಹಳ ತೆಳ್ಳಗಿರುವ ಆವರಣದ ಮೂಲಕ ನೋಡಿದಾಗ ಹಲವು ಬಗೆಯ ಒಳ್ಳೆಯದಿವೆ ಎನ್ನುವೆವು. ಆವರಣ ಮಂದವಾದಾಗ ಅದನ್ನು ಪಾಪ ಎನ್ನುವೆವು. ಒಳ್ಳೆಯದು ಕೆಟ್ಟದ್ದು ಮೌಡ್ಯದ ಬೇರೆಬೇರೆ ರೂಪಗಳಷ್ಟೆ. ಅವು ಪ್ರಪಂಚದಲ್ಲಿ ಹಲವು ಬಗೆಯ ದ್ವೈತಭ್ರಾಂತಿಯಾಗಿ, ಹಲವು ಭಾವನೆಗಳಾಗಿವೆ. ಈ ಭಾವನೆಗಳು ಮಾನವನ ಹೃದಯವನ್ನು ಮೆಟ್ಟಿ ಕ್ರೂರ ರಕ್ಕಸರಂತೆ ಕಾಡುತ್ತಿವೆ. ಅವು ನಮ್ಮನ್ನು ವ್ಯಾಘ್ರಗಳನ್ನಾಗಿ ಮಾಡುವುವು. ನಾವು ಮತ್ತೊಬ್ಬರನ್ನು ದ್ವೇಷಿಸುವ ಭಾವನೆಯೆಲ್ಲ ನಾವು ಬಾಲ್ಯದಿಂದಲೂ ಮೆದ್ದ ಒಳ್ಳೆಯದು ಕೆಟ್ಟದ್ದು ಎಂಬ ಭಾವನೆಯಿಂದಾದುದು. ನಾವು ಮಾನವರನ್ನು ತಪ್ಪಾಗಿ ಅಳೆಯುತ್ತೇವೆ; ಈ ಸುಂದರ ತಿರೆಯನ್ನು ನರಕಕೂಪವನ್ನಾಗಿ ಮಾಡುವೆವು. ನಾವು ಒಳ್ಳೆಯದು ಕೆಟ್ಟದ್ದು ಎಂಬ ಭಾವನೆಯನ್ನು ತ್ಯಜಿಸಿದೊಡನೆಯೆ ಅದು ಸ್ವರ್ಗವಾಗುವುದು.

“ತಿರೆ ಎಲ್ಲರಿಗೂ ಆನಂದದಾಯಕವಾಗಿದೆ. ಎಲ್ಲರೂ ತಿರೆಗೆ ಆನಂದದಾಯಕರಾಗಿರುವರು. ಎಲ್ಲರೂ ಒಬ್ಬರು ಪ್ರತಿಯೊಬ್ಬರಿಗೆ ಸಹಾಯ ಮಾಡುವರು. ಎಲ್ಲ ಆನಂದವೂ ಆತ್ಮನದು. ಅವನು ಅಮೃತನು, ಪ್ರಕಾಶಮಾನನು. ಈ ತಿರೆಯೊಳಗೆ ಇರುವನು.” ಈ ಮಾಧುರ್ಯ ಯಾರದು? ಅವನಿಲ್ಲದೆ ಬೇರಾವ ಮಾಧುರ್ಯ ಇರಬಲ್ಲದು? ಆ ಒಂದು ಮಾಧುರ್ಯವೇ ಹಲವು ವಿಧದಲ್ಲಿ ವ್ಯಕ್ತವಾಗುತ್ತಿದೆ. ಯಾವ ಮಾನವನ ಹೃದಯದಲ್ಲಾದರೂ, ಅವನು ಪಾಪಿಯಾಗಲಿ, ಯತಿಯಾಗಲಿ, ಕೊಲೆಪಾತಕನಾಗಲಿ, ಸಾಧುಸ್ವಭಾವದವನಾಗಲಿ, ದೇಹ ಮನಸ್ಸು ಇಂದ್ರಿಯಗಳಲ್ಲಿ ಎಲ್ಲಿಯಾದರೂ ಸ್ವಲ್ಪ ಪ್ರೀತಿಯಿದ್ದರೆ, ಆನಂದವಿದ್ದರೆ ಅದೆಲ್ಲ ಅವನೆ. ಅವನೇ ದೈಹಿಕ ಆನಂದ, ಮಾನಸಿಕ ಆನಂದ, ಆಧ್ಯಾತ್ಮಿಕ ಆನಂದ. ಅವನಲ್ಲದೆ ಮತ್ತೇನು ಇರಬಲ್ಲದು? ಇಪ್ಪತ್ತು ಸಹಸ್ರ ದೇವದಾನವರು ಹೇಗೆ ಕಾದಾಡುತ್ತಿರಬಲ್ಲರು? ಇವೆಲ್ಲ ಕೆಲಸಕ್ಕೆ ಬಾರದ ಕನಸು. ಯಾವುದು ಅತಿ ಹೀನ ದೈಹಿಕ ಆನಂದವೊ ಅದೂ ಅವನೆ, ಅತಿಶ್ರೇಷ್ಠ ಅಧ್ಯಾತ್ಮ ಆನಂದವೊ ಅವನೆ, ಅವನಿಲ್ಲದೆ ಆನಂದವೇ ಇಲ್ಲ. ಯಾಜ್ಞವಲ್ಕ್ಯ ಹೀಗೆ ಹೇಳುವನು. ಈ ಸ್ಥಿತಿಗೆ ಬಂದು ಪ್ರತಿಯೊಂದನ್ನೂ ಈ ದೃಷ್ಟಿಯಿಂದ ನೋಡಿದರೆ, ಕುಡುಕನ ಆನಂದದಲ್ಲಿಯೂ ಈ ಆನಂದವನ್ನು ನೋಡಿದಾಗ ಮಾತ್ರ ನಿಮಗೆ ಸತ್ಯ ದೊರೆತಿದೆ. ಆಗ ಮಾತ್ರ ನಿಮಗೆ ನಿಜವಾದ ಆನಂದದ ಅರಿವಾಗುವುದು, ಶಾಂತಿಯ ಅರಿವಾಗುವುದು, ಪ್ರೀತಿಯ ಅರಿವಾಗುವುದು. ಎಲ್ಲಿಯವರೆಗೂ ನೀವು ಕೆಲಸಕ್ಕೆ ಬಾರದ ವ್ಯತ್ಯಾಸಗಳನ್ನು ಕಲ್ಪಿಸಿಕೊಳ್ಳುತ್ತೀರೋ, ಎಲ್ಲಿಯವರೆಗೆ ನೀವು ಮೂರ್ಖ, ವ್ಯರ್ಥ ನಿರರ್ಥಕ ಮೌಢ್ಯವನ್ನು ಕಲ್ಪಿಸಿಕೊಳ್ಳುತ್ತೀರೋ ಅಲ್ಲಿಯವರೆಗೆ ಎಲ್ಲಾ ಬಗೆಯ ದುಃಖವೂ ನಿಮ್ಮನ್ನು ಕಾಡುವುದು. ಆ ಅಮೃತಾತ್ಮನು, ಸ್ವಯಂಪ್ರಕಾಶನು, ಈ ತಿರೆಯಲ್ಲಿರುವನು. ಅವನೇ ದೇಹದಲ್ಲಿರುವನು. ಇದೆಲ್ಲ ಅವನ ಮಾಧುರ್ಯ. ಈ ದೇಹ ತಿರೆಯಂತಿದೆ. ಇದರೊಳಗಿರುವ ಎಲ್ಲಾ ದೈಹಿಕ ಶಕ್ತಿಯಲ್ಲಿಯೂ, ಎಲ್ಲಾ ಆನಂದದಲ್ಲಿಯೂ ಅವನೇ ಇರುವನು. ಕಣ್ಣು ನೋಡುವುದು, ಚರ್ಮ ಸ್ಪರ್ಶಿಸುವುದು - ಈ ಸುಖವೆಲ್ಲ ಏನು? ಆ ಸ್ವಯಂ ಪ್ರಕಾಶಮಾನನು ದೇಹದಲ್ಲಿರುವನು. ಅವನೇ ಆತ್ಮ. ಎಲ್ಲರಿಗೂ ಪ್ರಪಂಚ ಇಷ್ಟು ಆನಂದದಾಯಕವಾಗಿರುವುದು ಮತ್ತು ಪ್ರತಿಯೊಬ್ಬರೂ ಪ್ರಪಂಚಕ್ಕೆ ಇಷ್ಟು ಆನಂದದಾಯಕವಾಗಿರುವುದಕ್ಕೆ ಕಾರಣ ಆ ಸ್ವಯಂಪ್ರಕಾಶಮಾನವಾದ ಆತ್ಮ. ಅಮೃತವೇ ಈ ಪ್ರಪಂಚದಲ್ಲಿರುವ ಆನಂದ. ನಮ್ಮಲ್ಲಿರುವ ಆ ಆನಂದವೂ ಅವನೇ. ಅವನೇ ಬ್ರಹ್ಮ, 'ಈ ಗಾಳಿ ಎಲ್ಲರಿಗೂ ಇಷ್ಟು ಆನಂದದಾಯಕವಾಗಿದೆ, ಎಲ್ಲರೂ ಗಾಳಿಗೆ ಅಷ್ಟು ಆನಂದದಾಯಕರಾಗಿರುವರು. ಯಾರು ಗಾಳಿಯಲ್ಲಿ ಸ್ವಯಂ ಜ್ಯೋತಿ ಅಮೃತಾತ್ಮನಾಗಿರುವನೊ ಅವನು ಈ ದೇಹದಲ್ಲಿಯೂ ಇರುವನು. ಅವನೇ ಎಲ್ಲ ಜೀವಿಗಳ ಪ್ರಾಣದಂತೆ ಕಾಣಿಸುತ್ತಿರುವನು. ಸೂರ್ಯ ಎಲ್ಲಾ ಪ್ರಾಣಿಗಳಿಗೂ ಇಷ್ಟು ಪ್ರಿಯನಾಗಿರುವನು. ಎಲ್ಲರೂ ಸೂರ್ಯನಿಗೆ ಅಷ್ಟು ಪ್ರಿಯರಾಗಿರುವರು. ಯಾರು ಸೂರ್ಯನಲ್ಲಿ ಸ್ವಯಂ ಪ್ರಕಾಶಮಾನನಾಗಿರುವನೊ, ಅವನನ್ನು ಒಂದು ಕಿರಿಯ ಜ್ಯೋತಿಯಂತೆ ನಾವು ಪ್ರತಿಬಿಂಬಿಸುವೆವು. ಅವನ ಪ್ರತಿಬಿಂಬವಲ್ಲದೆ ಮತ್ತೇನು ಇರಬಲ್ಲದು? ಅವನು ನಮ್ಮ ದೇಹದಲ್ಲಿರುವನು. ಅವನ ಪ್ರತಿಬಿಂಬವೇ ನಾವು ಬೆಳಕನ್ನು ನೋಡುವಂತೆ ಮಾಡುವುದು. ಚಂದ್ರ ಎಲ್ಲರಿಗೂ ಹರ್ಷಪ್ರದನು, ಎಲ್ಲರೂ ಚಂದ್ರನಿಗೆ ಹರ್ಷಪ್ರದರು, ಸ್ವಯಂಜ್ಯೋತಿ ಅಮೃತಾತ್ಮನೆ ಚಂದ್ರನಲ್ಲಿರುವುದು. ಅವನು ಮನಸ್ಸಿನಂತೆ ನಮ್ಮಲ್ಲಿ ವ್ಯಕ್ತವಾಗುತ್ತಿರುವನು. ಈ ಮಿಂಚು ಇಷ್ಟು ಸುಂದರವಾಗಿದೆ, ಎಲ್ಲರೂ ಅದಕ್ಕೆ ಪ್ರಿಯರು, ಎಲ್ಲರಿಗೂ ಅದು ಪ್ರಿಯ. ಆ ಸ್ವಯಂಜ್ಯೋತಿ ಅಮೃತಾತ್ಮನೆ ಆ ಮಿಂಚಿನ ಕಾಂತಿ. ಅವನು ನಮ್ಮಲ್ಲಿಯೂ ಇರುವನು. ಏಕೆಂದರೆ ಸರ್ವವೂ ಬ್ರಹ್ಮಮಯ. ಈ ಆತ್ಮನೇ ಸರ್ವ ಜೀವಿಗಳ ರಾಜ.” ಇಂತಹ ಭಾವನೆಗಳು ಮಾನವನಿಗೆ ಅತಿ ಪ್ರಯೋಜನಕಾರಿಗಳು, ಧ್ಯಾನಯೋಗ್ಯ ಭಾವನೆಗಳು ಇವು. ಉದಾಹರಣೆಗೆ ಪೃಥ್ವಿಯ ಮೇಲೆ ಧ್ಯಾನ ಮಾಡಿ ಪೃಥ್ವಿಯನ್ನು ಕುರಿತು ಯೋಚಿಸಿ. ಯಾವುದು ಪೃಥ್ವಿಯಲ್ಲಿದೆಯೊ ಅದು ನಮ್ಮಲ್ಲಿಯೂ ಇದೆ. ಅವೆರಡೂ ಒಂದೆ ಎಂದು ಭಾವಿಸಿ, ದೇಹವನ್ನು ಪೃಥ್ವಿಯೊಂದಿಗೆ ಏಕೀಭವಿಸಿ, ಆತ್ಮವನ್ನು ಪೃಥ್ವಿಯ ಹಿಂದೆ ಇರುವ ಆತ್ಮನಲ್ಲಿ ಏಕೀಭವಿಸಿ. ನನ್ನಲ್ಲಿ ಮತ್ತು ಹೊರಗಡೆ ಗಾಳಿಯಲ್ಲಿರುವ ಆತ್ಮ ಒಂದೆ ಎಂದು ಭಾವಿಸಿ. ಅವೆಲ್ಲ ಒಂದೇ, ಹಲವು ಆಕಾರಗಳಾಗಿ ಮಾರ್ಪಟ್ಟಿರುವುವು. ಈ ಐಕ್ಯತೆಯನ್ನು ಸಾಕ್ಷಾತ್ಕರಿಸಿಕೊಳ್ಳುವುದೇ ಧ್ಯಾನದ ಗುರಿ. ಇದನ್ನೇ ಯಾಜ್ಞವಲ್ಕ್ಯನು ಮೈತ್ರೇಯಿಗೆ ವಿವರಿಸಲು ಪ್ರಯತ್ನಿಸುತ್ತಿದ್ದುದು.

