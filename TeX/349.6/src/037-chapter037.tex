
\chapter[ನಾಗರಿಕತೆಯಲ್ಲಿ ವೇದಾಂತದ ಪಾತ್ರ]{ನಾಗರಿಕತೆಯಲ್ಲಿ ವೇದಾಂತದ ಪಾತ್ರ\protect\footnote{\engfoot{C.W, Vol. I, P. 383}}}

\begin{center}
(ಇಂಗ್ಲೆಂಡಿನ ರಿಡ್ಜ್ ವೇ ಗಾರ್ಡನಿನಲ್ಲಿ ಎರ್ಲಿ ಲಾಡ್ಜ್ ನಲ್ಲಿ ನೀಡಿದ ಉಪನ್ಯಾಸದಿಂದ ಆಯ್ದ ಭಾಗಗಳು)
\end{center}

ಸ್ಥೂಲವಾದ ಬಾಹ್ಯ ವಿಷಯಗಳನ್ನು ಮಾತ್ರ ನೋಡಬಲ್ಲ ಜನರಿಗೆ ಭರತಖಂಡದಲ್ಲಿ ತಮ್ಮ ಸ್ವಾತಂತ್ರ್ಯವನ್ನು ಕಳೆದುಕೊಂಡು ನರಳುತ್ತಿರುವ ಜನಾಂಗ ಮಾತ್ರ ಕಾಣುವುದು. ಭಾರತವು ಕೇವಲ ಕನಸುಣಿಗಳಿಂದ ಮತ್ತು ತಾತ್ತ್ವಿಕರಿಂದ ತುಂಬಿರುವಂತೆ ತೋರುವುದು. ಆಧ್ಯಾತ್ಮಿಕ ಕ್ಷೇತ್ರದಲ್ಲಿ ಭರತಖಂಡ ಪ್ರಪಂಚವನ್ನು ಗೆಲ್ಲಬಲ್ಲದು ಎಂಬ ಭಾವನೆಯನ್ನು ಗ್ರಹಿಸಲು ಅವರಿಗೆ ಅಸಾಧ್ಯವಾಗಿರುವಂತೆ ತೋರುವುದು. ಅತಿ ಕರ್ಮಪಟುಗಳಾದ ಪಾಶ್ಚಾತ್ಯರಿಗೆ ಪ್ರಾಚ್ಯರ ಅಂತರ್ಮುಖ ಜೀವನ ಮತ್ತು ಅವರ ಧ್ಯಾನಪರತೆ ಹೆಚ್ಚು ಪ್ರಯೋಜನಕಾರಿಯಾಗಿರುವಂತೆಯೇ ಪ್ರಾಚ್ಯರಿಗೆ ಪಾಶ್ಚಾತ್ಯರ ಕರ್ಮಶೀಲತೆ ಮತ್ತು ಉತ್ಸಾಹ ಪ್ರಯೋಜನಕಾರಿ ಎಂಬುದರಲ್ಲಿ ಸಂದೇಹವಿಲ್ಲ. ಆದರೂ ನಾವು ಈ ಪ್ರಶ್ನೆಯನ್ನು ಕೇಳಬೇಕಾಗಿದೆ: ಪಾರತಂತ್ರದಲ್ಲಿ ಮತ್ತು ಕಷ್ಟದಲ್ಲಿ ನರಳುತ್ತಿರುವ ಹಿಂದೂ ಮತ್ತು ಯೆಹೂದ್ಯ ಜನಾಂಗಗಳು ಉಳಿದ ಜನಾಂಗಗಳೆಲ್ಲ ನಾಶವಾದರೂ ಇನ್ನೂ ಬದುಕಿ ಇರುವುದಕ್ಕೆ ಕಾರಣಗಳೇನು? (ಈ ಎರಡು ಜನಾಂಗಗಳಿಂದಲೇ ಪ್ರಪಂಚದ ಮುಖ್ಯ ಧರ್ಮಗಳೆಲ್ಲ ಜನಿಸಿದುವು) ಇದಕ್ಕೆ ಉತ್ತರವೇ ಅವರ ಆಧ್ಯಾತ್ಮಿಕ ಶಕ್ತಿ. ಹಿಂದೂಗಳು ಈಗ ಮೌನವಾಗಿದ್ದರೂ ಜೀವಂತರಾಗಿರುವರು; ಯೆಹೂದ್ಯರು ಹಿಂದೆ ಅವರು ಪ್ಯಾಲೆಸ್ತೈನಿನಲ್ಲಿದ್ದಾಗ ಇದ್ದುದಕ್ಕಿಂತ ಹೆಚ್ಚಿನ ಸಂಖ್ಯೆಯಲ್ಲಿ ಇರುವರು. ಭಾರತೀಯ ತತ್ತ್ವಜ್ಞಾನ ನಾಗರಿಕ ಪ್ರಪಂಚದಲ್ಲೆಲ್ಲ ತನ್ನ ಪ್ರಭಾವವನ್ನು ಬೀರುತ್ತಾಹೋಗಿದೆ. ಇದರಂತೆಯೇ ಪೂರ್ವಕಾಲದಲ್ಲಿ, ಯೂರೋಪಿನ ಹೆಸರು ಪ್ರಪಂಚಕ್ಕೆ ಗೊತ್ತಾಗುವುದಕ್ಕೆ ಮುಂಚೆ, ಭಾರತೀಯರು ಆಫ್ರಿಕಾ ದೇಶದೊಂದಿಗೆ ವ್ಯಾಪಾರದ ಸಂಬಂಧವನ್ನು ಏರ್ಪಡಿಸಿಕೊಂಡಿದ್ದರು. ಭಾರತೀಯರು ತಮ್ಮ ದೇಶ ಬಿಟ್ಟು ಹೊರಗೆ ಹೋಗಲಿಲ್ಲ ಎಂಬ ಅಭಿಪ್ರಾಯವನ್ನು ಸುಳ್ಳಾಗಿಸಿರುವರು.

ಪರಕೀಯರು ಭರತಖಂಡವನ್ನು ಆಕ್ರಮಿಸಿದಾಗಲೆಲ್ಲ ಪರಕೀಯರ ಜೀವನಗತಿಯಲ್ಲಿ ಹೊಸದೊಂದು ಅಧ್ಯಾಯ ಪ್ರಾರಂಭವಾಗಿರುವುದೊಂದು ಸೋಜಿಗ. ಭರತಖಂಡ ತನ್ನನ್ನು ಆಕ್ರಮಿಸಿದವರಿಗೆ ಐಶ್ವರ್ಯ, ಪ್ರಗತಿ, ಸಾರ್ವಭೌಮತ್ವ, ಆಧ್ಯಾತ್ಮಿಕತೆ ಇವುಗಳನ್ನು ಕುರಿತಾದ ಭಾವನೆಗಳನ್ನು ಕೊಟ್ಟಿದೆ. ಪಾಶ್ಚಾತ್ಯನು ತಾನು ಎಷ್ಟೊಂದನ್ನು ಪಡೆಯಬಹುದು, ಅನುಭವಿಸಬಹುದು ಎಂದು ಆಲೋಚಿಸುತ್ತಿದ್ದರೆ, ಪ್ರಾಚ್ಯನು ತನಗೆ ಎಷ್ಟು ಕಡಿಮೆ ಪ್ರಾಪಂಚಿಕ ವಸ್ತುಗಳು ಸಾಕು ಎಂದು ಆಲೋಚಿಸುತ್ತಿರುವನು. ವೇದಗಳಲ್ಲಿ ಆ ಪುರಾತನ ಜನಾಂಗವು ದೇವರನ್ನು ಹುಡುಕಲು ಮಾಡಿದ ಪ್ರಯತ್ನವನ್ನು ನೋಡುವೆವು. ಅವನನ್ನು ಹುಡುಕುವಾಗ ಅವರು ಹಲವು ಮೆಟ್ಟಿಲುಗಳನ್ನು ದಾಟಿ ಹೋದರು. ಪಿತೃಪೂಜೆಗಳಿಂದ ಪ್ರಾರಂಭವಾಗಿ ಅಗ್ನಿ ಇಂದ್ರ ವರುಣ ಮುಂತಾದ ದೇವರುಗಳ ಪೂಜೆಯವರೆಗೆ ಅವರು ಮುಂದುವರಿದರು. ಹಲವು ದೇವರುಗಳಿಂದ ಏಕದೇವ ಭಾವನೆ ಬರುವುದನ್ನು ನಾವು ಎಲ್ಲಾ ಧರ್ಮಗಳಲ್ಲಿಯೂ ಕಾಣುವೆವು. ಅದರ ನಿಜವಾದ ಅರ್ಥ, ಇರುವ ಹಲವು ಬುಡಕಟ್ಟು ದೇವರುಗಳಿಗೆಲ್ಲ ಅವನೇ ದೇವದೇವ, ಅವನೇ ಸೃಷ್ಟಿಕರ್ತ, ಅವನೇ ಪ್ರಪಂಚವನ್ನು ಆಳುವವನು, ಅವನೇ ಎಲ್ಲರ ಹೃದಯಾಂತರ್ಯಾಮಿ ಎಂಬುದು. ಹಲವು ದೇವರುಗಳಿಂದ ಪ್ರಾರಂಭವಾಗಿ ಮುಂದುವರಿದಂತೆ ಅದು ಏಕೇಶ್ವರಭಾವನೆಯಲ್ಲಿ ಕೊನೆಗಾಣುವುದು. ಈ ಮಾನವ ಸ್ವಭಾವದ ದೇವರಿಂದ ಹಿಂದೂಗಳಿಗೆ ತೃಪ್ತಿ ಬರಲಿಲ್ಲ. ದಿವ್ಯತೆಯನ್ನು ಹುಡುಕುತ್ತಿದ್ದ ಅವರಿಗೆ ಈ ದೇವರ ಭಾವನೆ ತೀರ ಮಾನವೀಯವಾಗಿ ಕಂಡಿತು. ಆದಕಾರಣ ಪಂಚೇಂದ್ರಿಯಗಳ ಮೂಲಕ ಗೋಚರವಾಗುವ ಬಾಹ್ಯಪ್ರಪಂಚದಲ್ಲಿ ದೇವರನ್ನು ಹುಡುಕುವ ಪ್ರಯತ್ನವನ್ನು ತ್ಯಜಿಸಿ ಅಂತರ್ಮುಖಿಗಳಾದರು. ಅಂತರಂಗದಲ್ಲಿ ಒಂದು ಪ್ರಪಂಚವಿದೆಯೇ? ಅದು ಏನು? ಅದು ಆತ್ಮ. ವ್ಯಕ್ತಿಗೆ ಸತ್ಯಸ್ಯ ಸತ್ಯವಾಗಿರುವುದು ಇದೊಂದೆ. ಅವನು ತನ್ನನ್ನು ಅರಿತರೆ ಮಾತ್ರ ಪ್ರಪಂಚವನ್ನು ಅರಿಯಬಲ್ಲ, ಇಲ್ಲದೆ ಇದ್ದರೆ ಇಲ್ಲ. ಋಗ್ವೇದದಲ್ಲಿ ಕೂಡ ಆದಿಯಲ್ಲಿ ಇದೇ ಪ್ರಶ್ನೆಯನ್ನು ಹಾಕುವರು: “ಆದಿಯಲ್ಲಿ ಇದ್ದುದೇನು?" ವೇದಾಂತ ತತ್ತ್ವ ಈ ಪ್ರಶ್ನೆಯನ್ನು ಕ್ರಮೇಣ ಬಗೆಹರಿಸಿತು. ಆದಿಯಲ್ಲಿದ್ದುದೇ ಆತ್ಮ. ಅಂದರೆ ನಾವು ಯಾವುದನ್ನು ನಿರಪೇಕ್ಷ, ವಿಶ್ವಾತ್ಮ, ಪರಮಾತ್ಮ ಎನ್ನುವೆವೋ ಅದರಿಂದಲೇ ಹಿಂದಿನಿಂದಲೂ ಎಲ್ಲಾ ವಸ್ತುಗಳೂ ಅಭಿವ್ಯಕ್ತ ವಾಗುತ್ತಿವೆ, ಈಗಲೂ ಹಾಗೆಯೇ ಬರುತ್ತಿವೆ, ಮುಂದೆಯೂ ಹಾಗೆಯೇ ಬರಬೇಕು.

ವೇದಾಂತ ತತ್ತ್ವವು ಈ ಪ್ರಶ್ನೆಯನ್ನು ಬಗೆಹರಿಸಿದ ಮೇಲೆ ನೀತಿಯ ತಳಹದಿಯನ್ನು ಕಂಡುಹಿಡಿಯಿತು. ಎಲ್ಲಾ ಧರ್ಮಗಳು, “ಕೊಲ್ಲಬೇಡಿ, ಹಿಂಸಿಸಬೇಡಿ, ನಿಮ್ಮಂತೆ ನೆರೆಯವರನ್ನು ಪ್ರೀತಿಸಿ,'' ಎಂದು ಬೋಧಿಸಿದರೂ ಅದಕ್ಕೆ ಕಾರಣವನ್ನು ಕೊಟ್ಟಿಲ್ಲ. ನೆರೆಯವರನ್ನು ನಾನೇಕೆ ಹಿಂಸಿಸಕೂಡದು? ಈ ಪ್ರಶ್ನೆಗೆ ಸಮರ್ಪಕವಾದ ಉತ್ತರ ಬರಲಿಲ್ಲ. ಕೇವಲ ಮತ ತತ್ತ್ವಗಳಿಂದ ತೃಪ್ತಿ ಪಡದ ಹಿಂದೂಗಳ ವೇದಾಂತತತ್ತ್ವವು ಮಾತ್ರ ಇದಕ್ಕೆ ಒಂದು ವಿಚಾರಪೂರ್ಣವಾದ ವಿವರಣೆಯನ್ನು ಕೊಟ್ಟಿತು. ಈ ಆತ್ಮ ನಿರಪೇಕ್ಷವಾದುದು, ಸರ್ವಾಂತರ್ಯಾಮಿಯಾದುದು; ಆದಕಾರಣ ಇದು ಅನಂತವಾದುದು. ಎರಡು ಅನಂತಗಳು ಇರಲಾರವು. ಏಕೆಂದರೆ ಒಂದು ಮತ್ತೊಂದರ ಅನಂತತೆಗೆ ಭಂಗ ತಂದು ಎರಡೂ ಸಾಂತವಾಗುವುವು. ಪ್ರತಿಯೊಂದು ಆತ್ಮವೂ ಅನಂತವಾದ ವಿಶ್ವಾತ್ಮನಲ್ಲಿ ಒಂದಾಗಿದೆ. ಆದಕಾರಣ ಒಬ್ಬನು ನೆರೆಯವರನ್ನು ಹಿಂಸಿಸಿದರೆ ತನ್ನನ್ನೇ ಹಿಂಸಿಸಿಕೊಂಡಂತೆ. ಎಲ್ಲಾ ನೀತಿಯ ಹಿಂದೆ ಇರುವ ಮೂಲತತ್ತ್ವ ಇದು. ಮಾನವ ಪೂರ್ಣತೆಯೆಡೆಗೆ ಹೋಗುವಾಗ ತಪ್ಪಿನಿಂದ ಸತ್ಯದ ಕಡೆ ಹೋಗುತ್ತಿರುವನು ಎಂದು ಅನೇಕ ವೇಳೆ ನಂಬುವರು. ಒಂದು ಆಲೋಚನೆಯಿಂದ ಮತ್ತೊಂದಕ್ಕೆ ಹೋದರೆ ಅವನು ಅನಿವಾರ್ಯವಾಗಿ ಮೊದಲನೆಯದನ್ನು ತ್ಯಜಿಸಬೇಕಾಗಿದೆ. ಆದರೆ ಯಾವ ತಪ್ಪೂ ನಮ್ಮನ್ನು ಸತ್ಯಕ್ಕೆ ಒಯ್ಯಲಾರದು. ಹಲವು ಸ್ಥಿತಿಗಳ ಮೂಲಕ ಜೀವವು\break ವಿಕಾಸವಾಗುತ್ತಿರುವಾಗ ಸತ್ಯದಿಂದ ಸತ್ಯಕ್ಕೆ ಹೋಗುತ್ತಿರುವುದು. ಪ್ರತಿಯೊಂದು\break ಸ್ಥಿತಿಯೂ ಸತ್ಯವೇ. ಜೀವವು ಕೆಳಗಿನ ಸತ್ಯದಿಂದ ಮೇಲಿನ ಸತ್ಯಕ್ಕೆ ಹೋಗುತ್ತದೆ. ಈ ವಿಷಯವನ್ನು ಮುಂದಿನ ಉದಾಹರಣೆಯ ಮೂಲಕ ವಿವರಿಸಬಹುದು. ಒಬ್ಬನು ಸೂರ್ಯನ ಕಡೆಗೆ ಹೋಗುತ್ತಿರುವನು. ಅವನು ಪ್ರತಿಯೊಂದು ಹಂತದಲ್ಲಿಯೂ ಸೂರ್ಯನ ಛಾಯಾಚಿತ್ರವನ್ನು ತೆಗೆದುಕೊಳ್ಳುವನು. ಅವನು ನಿಜವಾಗಿ ಸೂರ್ಯನನ್ನು ಸೇರಿದ ಮೇಲೆ ಮೊದಲನೆಯ ಎರಡನೆಯ ಚಿತ್ರಗಳು ಸೂರ್ಯನಿಗಿಂತ ಎಷ್ಟು ಭಿನ್ನವಾಗಿರುವುವು! ಇವುಗಳಲ್ಲಿ ಎಷ್ಟು ವ್ಯತ್ಯಾಸವಿದ್ದರೂ ಎಲ್ಲವೂ ನಿಜ. ಈ ವ್ಯತ್ಯಾಸಕ್ಕೆಲ್ಲ ಕಾರಣ ಕಾಲ ಮತ್ತು ದೇಶ. ಈ ದೃಷ್ಟಿಯಿಂದ ಪ್ರೇರಿತವಾಗಿರುವುದರಿಂದ ಹಿಂದೂವಿಗೆ, ಅತಿ ಕ್ಷುದ್ರ ಧರ್ಮದಿಂದ ಹಿಡಿದು ಶ್ರೇಷ್ಠತಮ ಧರ್ಮದವರೆಗಿನ ಹಿಂದೆ ಇರುವ ಸಾಮಾನ್ಯವಾದ ಸತ್ಯವನ್ನು ನೋಡಲು ಸಾಧ್ಯವಾಗಿದೆ. ಧರ್ಮದ ಹೆಸರಿನಲ್ಲಿ ಹಿಂಸೆಮಾಡದ ಏಕಮಾತ್ರ ಜನಾಂಗವೇ ಹಿಂದೂ ಜನಾಂಗ. ಒಬ್ಬ ಮಹಮ್ಮದೀಯ ಮಹಾತ್ಮನ ಗೋರಿಯನ್ನು ಈಗ ಮಹಮ್ಮದೀಯರೇ ಮರೆತಿರುವರು, ಅಂತಹವನ್ನು ಹಿಂದೂಗಳು ಪೂಜಿಸುವರು. ಈ ಸೌಹಾರ್ದವನ್ನು ವಿವರಿಸಲು ಹಲವು ಉದಾಹರಣೆಗಳನ್ನು ಕೊಡಬಹುದು.

ಮಾನವರೆಲ್ಲ ಅರಸುವ ಏಕತೆ ಎಂಬ ಗುರಿಯನ್ನು ಪ್ರಾಚ್ಯ ಕಂಡು ಹಿಡಿಯುವವರೆಗೆ ಮನಸ್ಸಿಗೆ ತೃಪ್ತಿಯಿರಲಿಲ್ಲ. ಪಾಶ್ಚಾತ್ಯ ವಿಜ್ಞಾನಿ ಅಣುವಿನಲ್ಲೂ ಕಣದಲ್ಲೂ ಏಕತೆಯನ್ನು ಕಂಡುಹಿಡಿಯಲು ಯತ್ನಿಸುತ್ತಿರುವನು. ಅವನು ಇದನ್ನು ಕಂಡುಹಿಡಿದ ಮೇಲೆ ಅವನು ಇನ್ನೂ ಮುಂದೆ ಹೋಗಲಾರ. ಅದರಂತೆಯೇ ಆತ್ಮನ ಏಕತೆಯನ್ನು ಕಂಡುಹಿಡಿದ ಮೇಲೆ ನಾವು ಇನ್ನೂ ಮುಂದೆ ಹೋಗುವಂತೆ ಇಲ್ಲ. ಈ ಇಂದ್ರಿಯಜನ್ಯ ಪ್ರಪಂಚದಲ್ಲಿರುವುದೆಲ್ಲ ಆ ಒಂದು ವಸ್ತುವಿನ ಅಭಿವ್ಯಕ್ತಿ ಎಂದು ಅನಂತರ ಸ್ಪಷ್ಟವಾಗುವುದು. ವಿಜ್ಞಾನಿ ಕೂಡ ಈ ತಾತ್ತ್ವಿಕಾಂಶವನ್ನು ಒಪ್ಪಿಕೊಳ್ಳುವ ಸ್ಥಿತಿಗೆ ಬರಬೇಕಾಗಿದೆ. ಏಕೆಂದರೆ ಉದ್ದವಿಲ್ಲದ ಅಗಲವಿಲ್ಲದ ಅಣುಗಳು ಒಟ್ಟು ಕಲೆತಾಗ ಅವಕ್ಕೆ ಉದ್ದ -ಎತ್ತರ-ಅಗಲಗಳು ಬರುವಂತೆ ಕಾಣುವುದು. ಒಂದು ಅಣು ಮತ್ತೊಂದು ಅಣುವಿನ ಮೇಲೆ ತನ್ನ ಪ್ರಭಾವವನ್ನು ಬೀರಬೇಕಾದರೆ ಒಂದು ಮಧ್ಯವರ್ತಿ ಆವಶ್ಯಕ. ಈ ಮಧ್ಯವರ್ತಿ ಯಾವುದು? ಅದು ಮೂರನೆಯ ಅಣು. ಹಾಗಾದರೂ ಪ್ರಶ್ನೆ ಬಗೆಹರಿಯಲಿಲ್ಲ. ಈ ಎರಡೂ ಆ ಮೂರನೆಯದರ ಮೇಲೆ ತಮ್ಮ ಪ್ರಭಾವವನ್ನು ಹೇಗೆ ಬೀರುವುವು? ಇದು ಅನವಸ್ಥಾ ದೋಷವಾಗುವುದು. ಭೌತ ವಿಜ್ಞಾನಗಳಿಗೆ ಅವಶ್ಯಕವಾದ ಊಹಾಪ್ರತಿಜ್ಞೆಗಳಲ್ಲೆಲ್ಲಾ ಈ ದೋಷವಿದೆ: ಬಿಂದುವಿಗೆ ಗಾತ್ರವಿಲ್ಲ, ಅಂಶವಿಲ್ಲ ಎನ್ನುವರು; ಸರಳರೇಖೆಗೆ ಅಗಲವಿಲ್ಲ, ಉದ್ದ ಇದೆ ಎನ್ನುವರು. ಇದು ಕಾಣುವುದೂ ಇಲ್ಲ, ಇದನ್ನು ಊಹಿಸುವುದಕ್ಕೂ ಅಸಾಧ್ಯ. ಏತಕ್ಕೆ? ಏತಕ್ಕೆಂದರೆ ಇವು ನಮ್ಮ ಗ್ರಹಣೇಂದ್ರಿಯಗಳ ವ್ಯಾಪ್ತಿಗೆ ಬರುವುದಿಲ್ಲ. ಇವೆಲ್ಲಾ ಕೇವಲ ತಾತ್ತ್ವಿಕ ಭಾವನೆಗಳು. ಕೊನೆಗೆ ಮನಸ್ಸೇ ಎಲ್ಲಾ ಭಾವನೆಗಳಿಗೂ ಒಂದು ಆಕಾರವನ್ನು ಕೊಡುವುದು ಎಂದಂತೆ ಆಯಿತು. ನಾನು ಒಂದು ಕುರ್ಚಿಯನ್ನು ನೋಡಿದರೆ, ಹೊರಗಿರುವ ನಿಜವಾದ ಕುರ್ಚಿಯನ್ನಲ್ಲ ನಾನು ನೋಡುವುದು, ಹೊರಗೆ ಏನೋ ಒಂದು ಇದೆ; ಅದಕ್ಕೆ ನನ್ನ ಮನಸ್ಸಿನ ಭಾವನೆಯನ್ನು ಬೆರಸಿ ನೋಡಿದಾಗ ಅದು ಕುರ್ಚಿಯಂತೆ ಕಾಣುತ್ತದೆ. ಕೊನೆಗೆ ಭೌತಿಕವಾದಿಯೂ ತಾತ್ತ್ವಿಕನಾಗಬೇಕಾಗಿದೆ.

