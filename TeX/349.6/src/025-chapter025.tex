
\chapter[ಆತ್ಮ: ಅದರ ಸ್ವರೂಪ ಮತ್ತು ಗುರಿ]{ಆತ್ಮ: ಅದರ ಸ್ವರೂಪ ಮತ್ತು ಗುರಿ\protect\footnote{\engfoot{* C.W, Vol. VI, P. 18}}}

ಅತಿ ಪುರಾತನವಾದ ಭಾವನೆಯೆ, ಮಾನವನು ಕಾಲವಾದರೆ ಅವನು ಶೂನ್ಯವಾಗಿ ಹೋಗುವುದಿಲ್ಲ ಎಂಬುದು. ಅವನು ಕಾಲವಾದ ಮೇಲೂ ಅವನ ಯಾವುದೋ ಒಂದು ಭಾಗ ಜೀವಿಸಿರುವುದು, ಬಹಳ ಕಾಲ ಜೀವಿಸುತ್ತ ಹೋಗುವುದು. ಅತಿ ಪ್ರಾಚೀನವಾದ ಈಜಿಪ್ಟ್, ಬ್ಯಾಬಿಲೋನಿಯ, ಹಿಂದೂ ಈ ಮೂರು ದೇಶಗಳವರು ಈ ವಿಷಯದಲ್ಲಿ ಹೇಳಿರುವ ಭಾವನೆಗಳನ್ನು ಹೋಲಿಸುವುದು ಒಳ್ಳೆಯದು. ಅನಂತರ ಇತರರ ಭಾವನೆಗಳನ್ನು ಸ್ವೀಕರಿಸುವುದು ಒಳ್ಳೆಯದು. ಈಜಿಪ್ಟಿನ ಮತ್ತು ಬ್ಯಾಬಿಲೋನಿಯಾದ ಜನರಲ್ಲಿ ಒಂದು ವಿಧವಾದ ಜೀವದ ಭಾವನೆ ಇರುವುದು - ಅದೇ ಒಂದು ಪ್ರತಿರೂಪದ ಭಾವನೆ. ಈ ದೇಹದಲ್ಲಿ ಕೆಲಸ ಮಾಡುವ ಮತ್ತೊಂದು ದೇಹ ಇದೆ ಎನ್ನುವರು. ಹೊರಗಿನ ದೇಹವು ಸತ್ತಾಗ ಅದರ ಪ್ರತಿರೂಪವು ಹೊರಬಂದು ಒಂದು ನಿರ್ದಿಷ್ಟ ಕಾಲದವರೆಗೆ ಬದುಕುತ್ತದೆ. ಆದರೆ ಆ ಎರಡನೆಯ ದೇಹ ಜೀವಿಸಬೇಕಾದರೆ ಮೊದಲನೆಯ ದೇಹವನ್ನು ಸುರಕ್ಷಿತವಾಗಿಡಬೇಕು. ಈ ದೇಹದ ಯಾವ ಭಾಗಕ್ಕೆ ಊನವಾದರೂ ಪ್ರತಿರೂಪಕ್ಕೂ ಅದೇ ಊನವಾಗುವುದು. ಆದಕಾರಣವೆ ಪುರಾತನ ಈಜಿಪ್ಟಿನವರು ಗತಿಸಿಹೋದವರ ದೇಹವನ್ನು ರಕ್ಷಿಸಲು, ಅದನ್ನು ಬಹಳ ಕಾಲದವರೆಗೆ ಇಡಲು ಪಿರಮಿಡ್‌ಗಳನ್ನು ಕಟ್ಟುತ್ತಿದ್ದರು ಮತ್ತು ಹೆಣವನ್ನು ಹಿಂದಿನ ಆಕಾರದಲ್ಲಿ ಇಡುವುದಕ್ಕೆ ತಕ್ಕ ಕ್ರಮಗಳನ್ನು ತೆಗೆದುಕೊಳ್ಳುತ್ತಿದ್ದರು. ಬ್ಯಾಬಿಲೋನ್ ಮತ್ತು ಪುರಾತನ ಈಜಿಪ್ಟಿನವರ ಪ್ರಕಾರ ಪ್ರತಿರೂಪಕ್ಕೆ ಶಾಶ್ವತ ಅಸ್ತಿತ್ವವಿಲ್ಲ. ಹೆಚ್ಚು ಎಂದರೆ ಮೊದಲನೆಯ ದೇಹವನ್ನು ಸರಿಯಾಗಿ ಇಟ್ಟಿರುವವರೆಗೆ ಮಾತ್ರ ಅದು ಇರುತ್ತಿತ್ತು.

ಮತ್ತೊಂದು ವಿಚಿತ್ರವೆ, ಈ ಪ್ರತಿರೂಪಕ್ಕೆ ಜನರು ಅಂಜುತ್ತಿದ್ದರು. ಅದು\break ಯಾವಾಗಲೂ ಅಸುಖ, ದುಃಖದಿಂದ ನರಳುತ್ತಿತ್ತು. ಅತಿ ದಾರುಣ ಯಾತನೆಯಲ್ಲಿ ಅದು ಇರುತ್ತಿತ್ತು. ಅದು ಬದುಕಿರುವವರ ಬಳಿಗೆ ಪುನಃ ಪುನಃ ಬಂದು ಈಗ ತನಗೆ ಸಿಕ್ಕದಿರುವ ಆಹಾರ ಪಾನೀಯಗಳನ್ನು ಮತ್ತು ಇತರ ಭೋಗವಸ್ತುಗಳನ್ನು ಕೇಳುತ್ತಿತ್ತು. ಈಗ ತಾನು ಕುಡಿಯಲಾಗದ ನೈಲ್ ನದಿಯ ಹೊಸ ನೀರನ್ನು ಕುಡಿಯಬಯಸುವುದು. ಬದುಕಿರುವಾಗ ಏನು ಆಹಾರವನ್ನು ತಿನ್ನುತ್ತಿತ್ತೋ, ಅದೇ ಮತ್ತೆ ಬೇಕೆನ್ನುವುದು. ಯಾವಾಗ ಅದಕ್ಕೆ ಅದು ಸಿಕ್ಕಲಿಲ್ಲವೋ ಆಗ ಬದುಕಿರುವವರನ್ನು ಸಾವಿಗೆ ಅಥವಾ ಯಾವುದಾದರೂ ಸಂಕಟಕ್ಕೆ ಗುರಿ ಮಾಡುವೆನೆಂದು ಹೆದರಿಸುತ್ತಿತ್ತು.

ಆರ್ಯರ ಭಾವನೆಗೆ ಬಂದೊಡನೆಯೆ ಅಲ್ಲಿ ಎಷ್ಟೋ ವ್ಯತ್ಯಾಸ ಕಾಣುವುದು. ಅಲ್ಲಿಯೂ ಒಂದು ಪ್ರತಿರೂಪದ ಭಾವನೆ ಇರುವುದು. ಆದರೆ ಎರಡನೆಯದು ಒಂದು ಆಧ್ಯಾತ್ಮಿಕ ತನುವಾಗಿರುವುದು. ಅದರಲ್ಲಿ ಒಂದು ದೊಡ್ಡ ವ್ಯತ್ಯಾಸವೆ ಈ ಆಧ್ಯಾತ್ಮಿಕ ತನುವಿಗೂ, ಅದನ್ನು ಜೀವ ಅಥವಾ ಮತ್ತಾವ ಹೆಸರಿನಿಂದ ಬೇಕಾದರೂ ಕರೆಯಬಹುದು, ಕಾಲವಾದ ಸ್ಥೂಲದೇಹಕ್ಕೂ ಯಾವ ವಿಧವಾದ ಸಂಬಂಧವೂ ಇಲ್ಲ. ಅದರ ಬದಲು ಆಧ್ಯಾತ್ಮಿಕ ತನು ದೇಹದಿಂದ ಪಾರಾಯಿತು ಎನ್ನುವರು. ಆದಕಾರಣವೇ ಕಾಲವಾದವರನ್ನು ಸುಟ್ಟುಬಿಡುವ ಒಂದು ವಿಚಿತ್ರ ಅಭ್ಯಾಸ ಆರ್ಯರಲ್ಲಿ ಇರುವುದು. ಆರ್ಯರು ಜೀವಿ ತ್ಯಜಿಸಿಹೋದ ದೇಹವನ್ನು ಒಪ್ಪಮಾಡುವರು. ಆದರೆ ಈಜಿಪ್ಟಿನವರಾದರೋ ಅದನ್ನು ಹೂಳುವುದು, ಹೆಣದ ಸ್ಥಿತಿಯಲ್ಲೇ ಇಟ್ಟಿರುವುದು ಮತ್ತು ಅದಕ್ಕೆ ಒಂದು ದೊಡ್ಡ ಪಿರಮಿಡ್ಡನ್ನು ಕಟ್ಟುವುದು ಇವನ್ನೆಲ್ಲ ಮಾಡುವರು. ನಾಗರಿಕತೆಯಲ್ಲಿ ಸ್ವಲ್ಪ ಮುಂದುವರಿದ ಜನಾಂಗದಲ್ಲಿ ಅವರು ಗತಿಸಿದವರನ್ನು ಹೇಗೆ ವಿಲೇವಾರಿ ಮಾಡುವರು ಎಂಬುದರಲ್ಲಿ ಅವರ ಜೀವದ ಭಾವನೆ ಸ್ವಲ್ಪ ಮಟ್ಟಿಗೆ ವ್ಯಕ್ತವಾಗುವುದು. ಎಲ್ಲಿ ಬಿಟ್ಟುಹೋದ ಜೀವಕ್ಕೂ ಅದರ ಹಿಂದಿನ ದೇಹಕ್ಕೂ ಒಂದು ಸಂಬಂಧವಿದೆ ಎಂಬ ಭಾವನೆ ಇದೆಯೋ ಅಲ್ಲೆಲ್ಲ ದೇಹವನ್ನು ಸುರಕ್ಷಿತವಾಗಿಡುವ ಪ್ರಯತ್ನ ಮಾಡುವರು. ಅಲ್ಲೆಲ್ಲ ಹೆಣವನ್ನು ಯಾವುದಾದರೊಂದು ಬಗೆಯಲ್ಲಿ ಹೂಳುವರು. ಆದರೆ ಎಲ್ಲಿ ಜೀವ ದೇಹಕ್ಕಿಂತ ಸಂಪೂರ್ಣ ಬೇರೆ, ದೇಹಕ್ಕೆ ತೊಂದರೆಯಾದರೂ ಜೀವಕ್ಕೆ ತೊಂದರೆಯಾಗುವುದಿಲ್ಲ ಎಂಬ ಭಾವವಿದೆಯೋ ಅಲ್ಲೆಲ್ಲ ಹೆಣವನ್ನು ಸುಡುವರು. ಪುರಾತನ ಆರ್ಯಜನಾಂಗದಲ್ಲೆಲ್ಲ ಹೆಣ ಸುಡುವುದನ್ನು ನೋಡುವೆವು. ಆದರೆ ಪಾರ್ಸಿಯವರು ಮಾತ್ರ ಮೃತ್ಯು ಶಿಖರದಲ್ಲಿ ಅದನ್ನು (ಹದ್ದುಗಳಿಗೆ) ಇಡುತ್ತಿದ್ದರು. ಆದರೆ ಆ ಶಿಖರದ ಹೆಸರೆ (ದಖಮ) ಸುಡುವ ಸ್ಥಳ ಎಂದು. ಅಂದರೆ ಹಿಂದಿನ ಕಾಲದಲ್ಲಿ ಅವರಲ್ಲಿ ಕೂಡ ಹೆಣವನ್ನು ಸುಡುವ ಅಭ್ಯಾಸವಿತ್ತು. ಮತ್ತೊಂದು ವಿಶೇಷವೆ ಆರ್ಯರಲ್ಲಿ ಪ್ರತಿರೂಪದ ವಿಷಯದಲ್ಲಿ ಯಾವ ಅಂಜಿಕೆಯ ಭಾವನೆಯೂ ಇರಲಿಲ್ಲ. ಅವು ಊಟ ಉಪಚಾರಗಳಿಗಾಗಿ ಬರುತ್ತಿರಲಿಲ್ಲ. ಅವು ಸಿಕ್ಕದೆ ಇದ್ದರೆ ಕೋಪಗೊಳ್ಳುತ್ತಿರಲಿಲ್ಲ, ಬದುಕಿರುವವರನ್ನು ನಾಶಮಾಡಲು ಯತ್ನಿಸುತ್ತಿರಲಿಲ್ಲ. ಅದರ ಬದಲು ಅವರು ದೇಹದಿಂದ ಮುಕ್ತರಾದುದಕ್ಕೆ ಸಂತೋಷ ಪಡುತ್ತಿದ್ದರು. ಚಿತಾಗ್ನಿಯು ದೇಹವು ಮೂಲವಸ್ತುಗಳಿಗೆ ಹಿಂತಿರುಗುವುದರ ಸಂಕೇತ. ಆ ಅಗ್ನಿಯನ್ನು ಕುರಿತು, ಪ್ರಾಣತೊರೆದ ಜೀವಿಯನ್ನು ಮೃದುವಾಗಿ ಪಿತೃಗಳು ಇರುವ ಲೋಕಕ್ಕೆ ಕರೆದೊಯ್ಯಿ, ಎಲ್ಲಿ ದುಃಖವಿಲ್ಲವೋ, ಎಲ್ಲಿ ಎಂದೆಂದಿಗೂ ಸಂತೋಷವಿರುವುದೋ ಅಲ್ಲಿಗೆ ಒಯ್ಯಿ ಎನ್ನುವರು.

ಈ ಎರಡು ಭಾವನೆಗಳಲ್ಲಿ ಒಂದು ದುಃಖಸೂಚಕವಾಗಿದೆ, ಮತ್ತೊಂದು ಸುಖಸೂಚಕವಾಗಿದೆ. ಸುಖಭಾವನೆ ದುಃಖಭಾವನೆಯ ಅನಂತರ ಬಂದಿತು ಎನ್ನಬಹುದು. ಆರ್ಯರಲ್ಲಿಯೂ ಪುರಾತನಕಾಲದಲ್ಲಿ ಈಜಿಪ್ಟಿನವರಿಗಿದ್ದ ಭಾವನೆಗಳೇ ಬಹುಶಃ ಇದ್ದಿರಬಹುದು. ಇವರ ಅತಿ ಪುರಾತನ ಬರಹಗಳನ್ನು ಓದಿದರೆ ನಮಗೆ ಇದು ಹಾಗೆಯೇ ಇದ್ದಿರಬೇಕೆಂದು ತೋರುವುದು. ಆದರೆ ಆರ್ಯರ ಪ್ರಕಾರ ಪ್ರತಿರೂಪವು ಅತ್ಯಂತ ಕಾಂತಿಯುತವಾದುದು. ಮನುಷ್ಯನು ಕಾಲವಾದರೆ ಅವನ ಜೀವವು ಪಿತೃಲೋಕಕ್ಕೆ ಹೋಗಿ ಅಲ್ಲಿ ಆನಂದವನ್ನು ಅನುಭವಿಸುವುದು. ಪಿತೃಗಳು ಬಹಳ ದಯೆಯಿಂದ ಅದನ್ನು ಬರಮಾಡಿಕೊಳ್ಳುವರು. ಭರತಖಂಡದಲ್ಲಿ ಜೀವದ ವಿಷಯವಾಗಿ ಬರುವ ಪುರಾತನ ಭಾವನೆ ಇದು. ಬರಬರುತ್ತ ಇದು ಉತ್ತಮವಾಗುತ್ತ ಬರುವುದು. ಯಾವುದನ್ನು ಮುಂಚೆ ಜೀವ ಎಂದಿದ್ದರೆ ಅದು ಜೀವವಲ್ಲ ಎಂಬುದನ್ನು ಅನಂತರ\break ಕಂಡುಹಿಡಿದರು. ಈ ಕಾಂತಿಯ ದೇಹ, ಸೂಕ್ಷ ದೇಹ, ಎಷ್ಟೇ ಸೂಕ್ಷ್ಮವಾಗಿದ್ದರೂ, ಎಷ್ಟೆಂದರೂ ಒಂದು ದೇಹವಾಗಿದೆ. ಎಲ್ಲಾ ದೇಹಗಳೂ ಸ್ಥೂಲ ಅಥವಾ ಸೂಕ್ಷ್ಮ ವಸ್ತುಗಳ ಸಂಯೋಗದಿಂದ ಆಗಿರಬೇಕು. ಯಾವುದಕ್ಕೆ ಒಂದು ಆಕಾರವಿದೆಯೊ ಅದೊಂದು ಮಿತಿಗೆ ಒಳಪಡಬೇಕು. ಅದು ನಿತ್ಯವಾಗಲಾರದು. ಪ್ರತಿಯೊಂದು ಆಕಾರವೂ ಬದಲಾಗಬೇಕು. ಯಾವುದು ಬದಲಾಗುವುದೊ ಅದು ಹೇಗೆ ನಿತ್ಯವಾಗಬಲ್ಲುದು?\break ಈ ಜೋತಿರ್ಮಯ ದೇಹದ ಹಿಂದೆ ಮನುಷ್ಯನ ಜೀವವನ್ನು ಕಂಡುಹಿಡಿದರು. ಅದನ್ನೇ ಆತ್ಮ ಎಂದರು. ಆನಂತರ ಈ ಆತ್ಮನ ಭಾವನೆ ಪ್ರಾರಂಭವಾಯಿತು. ಅದೂ ಎಷ್ಟೋ ಸಲ ಬದಲಾಗಬೇಕಾಯಿತು. ಕೆಲವರು ಅದನ್ನು ಅನಂತವಾದುದು ಎಂದರು. ಅದು ದೇಹದ ಯಾವುದೋ ಒಂದು ಭಾಗದಲ್ಲಿದ್ದು, ಮಾನವನು ಕಾಲವಾದ ಮೇಲೆ ಸೂಕ್ಷ್ಮ ದೇಹವನ್ನು ತೆಗೆದುಕೊಂಡು ಹೋಗುವುದು ಎಂದು ಭಾವಿಸಿದರು. ಕೆಲವರು ಆತ್ಮವೆಂಬುದು ಒಂದು ಕಣ ಎಂಬುದನ್ನು ನಿರಾಕರಿಸಿದರು. ಹೇಗೆ ಸೂಕ್ಷ್ಮದೇಹವೇ ಆತ್ಮನಲ್ಲ ಎಂದು ನಿರಾಕರಿಸಿದರೊ, ಅದೇ ವಾದದ ಆಧಾರದಿಂದ ಇದನ್ನೂ ನಿರಾಕರಿಸಿದರು.

ಇಂತಹ ಹಲವು ಅಭಿಪ್ರಾಯಗಳ ನಡುವೆ ಸಾಂಖ್ಯ ತತ್ತ್ವ ಹೊರ ಹೊಮ್ಮಿತು. ಅಲ್ಲಿ ತಕ್ಷಣವೇ ಭಾರಿ ವ್ಯತ್ಯಾಸಗಳನ್ನು ಕಾಣುತ್ತೇವೆ. ಮನುಷ್ಯನಿಗೆ ಒಂದು ಸ್ಫೂಲದೇಹವಿದೆ ಎನ್ನುವರು. ಅದರ ಹಿಂದೆ ಒಂದು ಸೂಕ್ಷ್ಮ ಶರೀರವಿದೆ. ಅದು ಮನಸ್ಸಿನ ಒಂದು ಯಂತ್ರದಂತೆ ಇದೆ. ಅದರ ಹಿಂದೆ ಆತ್ಮನಿರುವನು. ಅವನೇ ಸಾಕ್ಷಿ. ಅವನು ಮನಸ್ಸನ್ನು ನೋಡುತ್ತಿರುವನು. ಅವನು ಸರ್ವವ್ಯಾಪಿ. ಅವನೇ ನಿಮ್ಮ ನಮ್ಮ ಎಲ್ಲರ ಆತ್ಮ ಕೂಡ. ಅವನು ಏಕಕಾಲದಲ್ಲಿ ಎಲ್ಲೆಡೆಯೂ ಇರುವನು. ಅವನು ನಿರಾಕಾರನಾದರೆ ಒಂದು ದೇಶದಲ್ಲಿ ಹೇಗೆ ಇರಬಲ್ಲನು? ದೇಶದಲ್ಲಿ ಇರುವುದಕ್ಕೆಲ್ಲ ಆಕಾರವಿರುವುದು. ನಿರಾಕಾರ ಮಾತ್ರ ಸರ್ವವ್ಯಾಪಿಯಾಗಬಲ್ಲುದು. ಆದಕಾರಣ ಪ್ರತಿಯೊಂದು ಆತ್ಮವೂ ಎಲ್ಲಾ ಕಡೆಗಳಲ್ಲಿಯೂ ಇರುವುದು. ಅವರು ಪ್ರತಿಪಾದಿಸುವ ಎರಡನೆಯ ಸಿದ್ದಾಂತ ಮತ್ತೂ ವಿಸ್ಮಯಕಾರಿಯಾಗಿರುವುದು. ಹಿಂದಿನ ಕಾಲದಲ್ಲಿ ಮಾನವರೆಲ್ಲ, ಅವರಲ್ಲಿ ಬಹುಪಾಲು ಜನರಾದರೂ ಪ್ರಗತಿಪರರು ಎಂಬುದನ್ನು ಕಂಡರು. ಪಾವಿತ್ರ್ಯ, ಶಕ್ತಿ, ಜ್ಞಾನ, ಇವುಗಳಲ್ಲಿ ಅವರು ಬೆಳೆಯುತ್ತ ಹೋಗುವುದನ್ನು ಕಂಡರು. ಮಾನವನಲ್ಲಿ ವ್ಯಕ್ತವಾಗುವ ಈ ಪಾವಿತ್ರ್ಯ ಶಕ್ತಿ, ಜ್ಞಾನ, -ಇವೆಲ್ಲ ಎಲ್ಲಿಂದ ಬಂದುವು? ಇಲ್ಲೊಂದು ಮಗುವಿದೆ. ಅದಕ್ಕೆ ಯಾವ ಜ್ಞಾನವೂ ಇಲ್ಲ. ಈ ಮಗು ಬೆಳೆದು ದೃಢಕಾಯವಾಗಿ ಬಲಾಢ್ಯವಾಗಿ ಬುದ್ದಿವಂತವಾಗುತ್ತದೆ. ಈ ಮಗು ಇಷ್ಟೊಂದು ಜ್ಞಾನವನ್ನು ಮತ್ತು ಶಕ್ತಿಯನ್ನು ಎಲ್ಲಿಂದ ಸಂಗ್ರಹಿಸಿತು? ಇದಕ್ಕೆ ಉತ್ತರವೇ ಇವೆಲ್ಲ ಆಗಲೇ ಆತ್ಮನಲ್ಲಿದ್ದವು ಎಂಬುದು. ಈ ಮಗುವಿನಲ್ಲಿ ಹುಟ್ಟಿದಂದಿನಿಂದಲೇ ಶಕ್ತಿ ಜ್ಞಾನಗಳಿದ್ದವು. ಶಕ್ತಿ, ಪಾವಿತ್ರ್ಯ, ಬಲಗಳೆಲ್ಲ ಆಗಲೇ ಆತ್ಮನಲ್ಲಿದ್ದವು. ಆದರೆ ಅವು ಸುಪ್ತಾವಸ್ಥೆಯಲ್ಲಿದ್ದವು. ಈಗ ಅವು ವ್ಯಕ್ತವಾಗಿವೆ. ಈ ಸುಪ್ತ ಮತ್ತು ವ್ಯಕ್ತ ಎಂದರೆ ಅರ್ಥವೇನು? ಸಾಂಖ್ಯರು ಹೇಳುವಂತೆ ಪ್ರತಿಯೊಂದು ಆತ್ಮವೂ ಪರಿಪೂರ್ಣವಾದುದು, ಪರಿಶುದ್ಧವಾದುದು. ಅದು ಸರ್ವಶಕ್ತಿಮಾನ್, ಸರ್ವಜ್ಞ. ಆದರೆ ಅದು ತನಗೆ ದೊರೆತ ಮನಸ್ಸಿಗೆ ತಕ್ಕಂತೆ ಅಭಿವ್ಯಕ್ತವಾಗುವುದು.\break ಮನಸ್ಸು ಆತ್ಮವನ್ನು ಪ್ರತಿಬಿಂಬಿಸುವ ಕನ್ನಡಿಯಂತಿದೆ ಎಂದು ಹೇಳಬಹುದು. ನನ್ನ ಮನಸ್ಸು ನನ್ನ ಆತ್ಮದ ಶಕ್ತಿಗಳನ್ನು ಸ್ವಲ್ಪ ಮಟ್ಟಿಗೆ ಪ್ರತಿಫಲಿಸುತ್ತದೆ. ಹಾಗೆಯೇ ನಿಮ್ಮ ಮತ್ತು ಇತರರ ಮನಸ್ಸುಗಳು ಕೂಡ. ಯಾವ ಕನ್ನಡಿಯು ಶುದ್ದವಾಗಿದೆಯೋ ಅದು ಆತ್ಮವನ್ನು ಚೆನ್ನಾಗಿ ಪ್ರತಿಬಿಂಬಿಸುತ್ತದೆ. ಆದ್ದರಿಂದ ಅಭಿವ್ಯಕ್ತಿಯು ಮನಸ್ಸಿನ ಸ್ಥಿತಿಯನ್ನು ಅವಲಂಬಿಸಿರುತ್ತದೆ. ಆದರೆ ಆತ್ಮ ಸ್ವಭಾವತಃ ಪರಿಶುದ್ಧ ಮತ್ತು ಪರಿಪೂರ್ಣವಾದುದು.

ಇದು ಹೀಗಿರಲು ಸಾಧ್ಯವಿಲ್ಲ ಎಂದು ಭಾವಿಸಿದ ಮತ್ತೊಂದು ಸಿದ್ಧಾಂತ ಇತ್ತು. ಆತ್ಮವು ಸ್ವಭಾವತಃ ಪರಿಪೂರ್ಣ ಮತ್ತು ಪರಿಶುದ್ಧವಾದರೂ ಅದು ಕೆಲವು ವೇಳೆ ಸಂಕುಚಿತವಾಗುವುದು, ಮತ್ತೆ ಕೆಲವು ವೇಳೆ ವಿಕಾಸವಾಗುವುದು ಎಂದು ಆ ಸಿದ್ದಾಂತದವರು ಹೇಳಿದರು. ಕೆಲವು ಆಲೋಚನೆಗಳು ಮತ್ತು ಕೆಲವು ಕೆಲಸಗಳು ಆತ್ಮದ ಸ್ವರೂಪವನ್ನು ಸಂಕೋಚಗೊಳಿಸುವುವು. ಇತರ ಕೆಲವು ಕರ್ಮಗಳು ಮತ್ತು ಆಲೋಚನೆಗಳು ಆತ್ಮದ ಸ್ವರೂಪವನ್ನು ಪ್ರಕಟಗೊಳಿಸುತ್ತವೆ. ಇದನ್ನು ಅವರು ವಿವರಿಸುವರು. ಯಾವುವು ಆತ್ಮದ ಶಕ್ತಿಯನ್ನು ಮತ್ತು ಪಾವಿತ್ರ್ಯವನ್ನು ಸಂಕೋಚಗೊಳ್ಳುವಂತೆ ಮಾಡುವುವೋ ಅವೆಲ್ಲ ಪಾಪಕರ್ಮಗಳು ಮತ್ತು ಪಾಪಚಿಂತನೆಗಳು; ಯಾವ ಕಾರ್ಯಗಳು ಮತ್ತು ಆಲೋಚನೆಗಳು ಆತ್ಮನ ಸ್ವರೂಪವನ್ನು ವಿಕಾಸಗೊಳಿಸುವುವೊ, ಅದರ ಶಕ್ತಿ ಹೊರಗೆ ಬರುವಂತೆ ಮಾಡುವುವೊ, ಅವೆಲ್ಲ ಒಳ್ಳೆಯವು, ಪುಣ್ಯ ಕರ್ಮಗಳು. ಈ ಎರಡು ಸಿದ್ಧಾಂತಗಳಿಗೆ ವ್ಯತ್ಯಾಸ ಇರುವುದು ಬಹಳ ಕಡಮೆ - ಅದು ಸಂಕೋಚ-ವಿಕಾಸಗಳಿಗೆ ಕೊಡುವ ಅರ್ಥದ ಮೇಲಿದೆ. ಹೆಚ್ಚು ಕಡಮೆಯಾಗುವುದು ಕೇವಲ ಆತ್ಮನ ಯಂತ್ರವಾದ ಮನಸ್ಸು ಮಾತ್ರ ಎಂಬ ವಿವರಣೆ ಉತ್ತಮವಾದುದು. ಸಂಕೋಚ-ವಿಕಾಸಗಳು ಆತ್ಮನಿಗೆ ಸೇರಿದುವು ಎಂದು ನಂಬುವವರನ್ನು, ಆತ್ಮನ ಸಂಕೋಚ-ವಿಕಾಸ ಎಂದರೆ ಅರ್ಥವೇನು ಎಂಬುದನ್ನು ಕೇಳಬೇಕು. ಆತ್ಮವು ಚೈತನ್ಯ ಸ್ವರೂಪವಾದುದು. ಭೌತಿಕವಾದುದಕ್ಕೆ, ಸ್ಕೂಲವಾದ ದೇಹಕ್ಕಾಗಲಿ ಅಥವಾ ಸೂಕ್ಷ್ಮವಾದ ಮನಸ್ಸಿಗೆ ಆಗಲಿ ಸಂಕೋಚ-ವಿಕಾಸಗಳನ್ನು ಹೇಳಬಹುದು. ಆದರೆ ಇವುಗಳಾಚೆ ಇರುವ ಕಾಲದೇಶಗಳಿಂದ ಬಾಧಿತವಲ್ಲದ ಭೌತಿಕವಲ್ಲದ ವಸ್ತುವಿನ ವಿಷಯದಲ್ಲಿ ಸಂಕೋಚ-ವಿಕಾಸ\break ಎಂಬುದಕ್ಕೆ ಅರ್ಥವೇನು? ಆದಕಾರಣ ಆತ್ಮ ಸದಾ ಪೂರ್ಣ ಪರಿಶುದ್ದ. ಆದರೆ ಕೆಲವು ಮನಸ್ಸುಗಳಲ್ಲಿ ಇದು ಹೆಚ್ಚು ವ್ಯಕ್ತವಾಗುವುದು, ಮತ್ತೆ ಕೆಲವುದರಲ್ಲಿ ಇದು ಕಡಮೆ ವ್ಯಕ್ತವಾಗುವುದು ಎಂಬುದು ಉತ್ತಮ ವಿವರಣೆಯಂತೆ ತೋರುವುದು. ಮನಸ್ಸು ಬೆಳೆದಂತೆ, ಅದರ ಸ್ವಭಾವ ವಿಕಾಸವಾದಂತೆ ಹೆಚ್ಚು ಸ್ಪಷ್ಟವಾದ ಆತ್ಮನ ಪ್ರತಿಬಿಂಬಗಳನ್ನು ಕೊಡುವುದು. ಇದು ಹೀಗೆ ವೃದ್ಧಿಯಾಗಿ ಮನಸ್ಸು ಸಂಪೂರ್ಣ ಪರಿಶುದ್ಧವಾದ ಮೇಲೆ ಆತ್ಮನ ಸ್ವಭಾವವನ್ನು ಪೂರ್ಣವಾಗಿ ಪ್ರತಿಬಿಂಬಿಸಬಲ್ಲದು. ಆಗ ಆತ್ಮ ಮುಕ್ತವಾಗುವುದು.

ಇದೇ ಆತ್ಮನ ಸ್ವರೂಪ, ಅದರ ಗುರಿಯೇನು? ಭರತಖಂಡದಲ್ಲಿರುವ ವಿಭಿನ್ನ ಪಂಗಡದವರಲ್ಲೆಲ್ಲಾ ಗುರಿ ಒಂದೇ ಆಗಿ ಕಾಣುವುದು. ಅವರಲ್ಲೆಲ್ಲ ಒಮ್ಮತವಿದೆ. ಅದೇ ಮುಕ್ತಿ. ಮಾನವ ಅನಂತ. ಈಗ ಸಾಂತ ಸ್ಥಿತಿಯಲ್ಲಿರುವುದು ಅವನ ಸ್ವಭಾವವಲ್ಲ. ಈ ಆತಂಕಗಳ ಮೂಲಕ ಅವನು ತನ್ನ ಸ್ವಭಾವವಾದ, ಆಜನ್ಮಸಿದ್ಧ ಹಕ್ಕಾದ, ಅನಂತತೆಯ ಕಡೆಗೆ, ಅಸೀಮದ ಕಡೆಗೆ ಮುಂದೆ ಮುಂದೆ ಹೋಗಲು ಹೋರಾಡುತ್ತಿರುವನು. ನಮ್ಮ ಸುತ್ತಲೂ ಕಾಣುವ ಹಲವು ಬಗೆಯ ಸಂಯೋಜನೆಗಳು ಮತ್ತು ಅಭಿವ್ಯಕ್ತಿಗಳೇ ಗುರಿಯಲ್ಲ. ಅವು ಕೇವಲ ತಾತ್ಕಾಲಿಕ ಮಾತ್ರ. ಪೃಥ್ವಿ, ಸೂರ್ಯ, ಚಂದ್ರ, ತಾರೆಗಳಂತೆ, ಒಳ್ಳೆಯದು ಕೆಟ್ಟದರಂತೆ, ಪಾಪ ಪುಣ್ಯಗಳಂತೆ, ಅಳು ನಗುಗಳಂತೆ, ಸುಖ ದುಃಖಗಳಂತೆ ತೋರಿಕೊಳ್ಳುವ ಸಂಯೋಜನೆಗಳೆಲ್ಲ ನಾವು ಅನುಭವಗಳನ್ನು ಪಡೆಯುವುದಕ್ಕಾಗಿ. ಅನುಭವದ ಮೂಲಕ ಆತ್ಮ ತನ್ನ ಉಪಾಧಿಗಳನ್ನು ಒಗೆದು ತನ್ನ ಸ್ವಭಾವವನ್ನು ಪ್ರಕಾಶಗೊಳಿಸುವುದು. ಅದು ಅನಂತರ ಬಾಹ್ಯ ಮತ್ತು ಆಂತರಿಕವಾದ ಯಾವ ನಿಯಮಕ್ಕೂ ಬಂಧಿಯಲ್ಲ. ಅದು ಎಲ್ಲಾ ನಿಯಮಗಳನ್ನೂ, ಮಿತಿಯನ್ನೂ ಸ್ವಭಾವವನ್ನೂ ಮೀರಿಹೋಗುವುದು. ಪ್ರಕೃತಿ ಆತ್ಮನ ಅಧೀನಕ್ಕೆ ಬರುವುದು. ಈಗ ಭಾವಿಸುವಂತೆ ಪ್ರಕೃತಿಯ ಅಧೀನದಲ್ಲಿ ಆತ್ಮನಿಲ್ಲ. ಆತ್ಮನ ಏಕಮಾತ್ರ ಗುರಿಯೇ ಇದು. ಮುಕ್ತಿಯೆಂಬ ತನ್ನ ಗುರಿಯನ್ನು ಸೇರುವುದಕ್ಕಾಗಿ, ಅದು ಹಲವು ಸ್ಥಿತಿಗಳ ಮೂಲಕ, ಹಲವು ಅನುಭವಗಳ ಮೂಲಕ, ಬರುತ್ತಿರುವುದನ್ನು ಜನ್ಮ ಎನ್ನುವರು. ಆತ್ಮ ಕೆಳಗಿನ ದೇಹವನ್ನು ತೆಗೆದುಕೊಂಡು ಅದರ ಮೂಲಕ ತಾನು ವ್ಯಕ್ತವಾಗಲು ಯತ್ನಿಸುವುದು. ಅದು ಸಾಲದೆಂದು ಆಚೆಗೆ ಎಸೆದು ಬೇರೊಂದು ಉತ್ತಮ ಜನ್ಮವನ್ನು ಧರಿಸುವುದು, ಅದರ ಮೂಲಕ ವ್ಯಕ್ತವಾಗಲು ಯತ್ನಿಸುವುದು. ಅದೂ ಸಾಲದೆಂದು ಅದನ್ನು ತ್ಯಜಿಸಿ ಬೇರೊಂದು ಜನ್ಮವನ್ನು ಧರಿಸುವುದು. ಹಾಗೆ ತನ್ನ ಶ್ರೇಷ್ಠವಾದುದನ್ನು ವ್ಯಕ್ತಗೊಳಿಸುವುದಕ್ಕೆ ಯೋಗ್ಯವಾದ ದೇಹ ಸಿಕ್ಕುವವರೆಗೆ ಯತ್ನಿಸುತ್ತ ಹೋಗಿ ಕೊನೆಗೆ ಮುಕ್ತವಾಗುವುದು.

ಆತ್ಮವು ಅನಂತವಾಗಿ ಸರ್ವವ್ಯಾಪಿಯಾಗಿದ್ದರೆ, ಅದೊಂದು ಅಧ್ಯಾತ್ಮ ವಸ್ತುವಾಗಿದ್ದರೆ, ದೇಹ ಧರಿಸುವುದು, ಒಂದಾದಮೇಲೊಂದು ದೇಹವನ್ನು ಪಡೆಯುವುದು ಎಂಬುದರ ಅರ್ಥವೇನು? ಆತ್ಮ ಬರುವುದೂ ಇಲ್ಲ, ಹೋಗುವುದೂ ಇಲ್ಲ, ಅದು ಹುಟ್ಟುವುದೂ ಇಲ್ಲ, ಸಾಯುವುದೂ ಇಲ್ಲ. ಸರ್ವವ್ಯಾಪಿಯಾಗಿರುವುದು ಹೇಗೆ ಹುಟ್ಟಬಲ್ಲುದು? ಆತ್ಮ ಒಂದು ದೇಹದಲ್ಲಿ ಇರುವುದು ಎಂಬುದಕ್ಕೆ ಅರ್ಥವಿಲ್ಲ. ಯಾವ ಮಿತಿಯೂ ಇಲ್ಲದುದು ಒಂದು ಮಿತವಾದ ದೇಹದಲ್ಲಿದೆ ಎಂದು ಹೇಗೆ ಹೇಳುವುದು? ಅದು ಒಬ್ಬ ಪುಸ್ತಕವೊಂದನ್ನು ಕೈಯಲ್ಲಿ ಹಿಡಿದುಕೊಂಡು ಒಂದಾದಮೇಲೆ ಒಂದು ಪುಟವನ್ನು ಓದಿ ತಿರುವಿ ಹಾಕಿದಂತೆ, ಹಾಳೆಗಳು ಮಾತ್ರ ಬದಲಾಗುತ್ತಿವೆ. ಓದುವವನು ಇದ್ದ ಕಡೆಯೇ ಇರುವನು. ಇದರಂತೆಯೇ ಆತ್ಮ. ಇಡಿಯ ಪ್ರಕೃತಿಯೇ ಓದುತ್ತಿರುವ ಗ್ರಂಥ. ಪ್ರತಿಯೊಂದು ಜನ್ಮವೂ ಅಂತಹ ಗ್ರಂಥದಲ್ಲಿ ಒಂದು ಪುಟದಂತೆ. ಅದನ್ನು ಓದಿ ಆದ ಮೇಲೆ ಹಾಳೆಯನ್ನು ಮಗುಚಿ ಬೇರೊಂದಕ್ಕೆ ಬರುವುದು. ಇದರಂತೆಯೇ ಪುಸ್ತಕವನ್ನೆಲ್ಲ ಪೂರೈಸಿ ಅನುಭವವನ್ನು ಗಳಿಸಿ ಮುಕ್ತರಾಗುವವರೆಗೆ ಆಗುತ್ತಿರುವುದು. ಆದರೆ ಆತ್ಮನೇ ಎಂದಿಗೂ ಚಲಿಸಲಿಲ್ಲ, ಬರಲೂ ಇಲ್ಲ, ಹೋಗಲೂ ಇಲ್ಲ, ಅದು ಕೇವಲ ಪ್ರಕೃತಿಯ ಅನುಭವವನ್ನು ಸಂಗ್ರಹಿಸುತ್ತಿತ್ತು ಅಷ್ಟೆ. ಆದರೆ ನಾವು ಚಲಿಸುತ್ತಿರುವಂತೆ ಕಾಣುವುದು. ಭೂಮಿ ಸುತ್ತುತ್ತಿದೆ. ಆದರೆ ಸೂರ್ಯನೇ ಚಲಿಸುತ್ತಿರುವಂತೆ ಕಾಣುತ್ತಿದೆ. ಅದರಂತೆಯೇ ನಮ್ಮ ಹುಟ್ಟು ಸಾವು, ಜನನ ಮರಣಗಳೆಂಬ ಭ್ರಮೆ ಕೂಡ. ಆತ್ಮ ಎಲ್ಲಿಗೆ ಹೋಗಬೇಕು? ಅದಕ್ಕೆ ಹೋಗುವುದಕ್ಕೆ ಸ್ಥಳವಿಲ್ಲ. ಅದು ಆಗಲೆ ಇಲ್ಲದೆ ಇರುವ ಸ್ಥಳವೆಲ್ಲಿ?

ಪ್ರಕೃತಿಯ ವಿಕಸನ, ಆತ್ಮನ ಆವಿರ್ಭಾವ ಎಂಬ ಸಿದ್ದಾಂತ ಹೀಗೆ ಬರುವುದು. ವಿಕಸನ, ಉತ್ತಮೋತ್ತಮ ದೇಹಗಳನ್ನು ಪಡೆಯುವುದು ಇವೆಲ್ಲ ಆತ್ಮನಲ್ಲಿ ಇಲ್ಲ. ಆತ್ಮ ಏನಾಗಿರುವುದೋ ಅದು ಆಗಲೇ ಆಗಿರುವುದು. ಈ ಬದಲಾವಣೆ ಪ್ರಕೃತಿಯಲ್ಲಿ ಮಾತ್ರ. ಪ್ರಕೃತಿ ಉತ್ತಮೋತ್ತಮ ದೇಹಗಳ ಮೂಲಕ ವಿಕಾಸವಾಗುತ್ತಿರುವಾಗ ಆತ್ಮನ ಮಹಿಮೆ ಹೆಚ್ಚು ಹೆಚ್ಚಾಗಿ ವ್ಯಕ್ತವಾಗುತ್ತ ಬರುತ್ತದೆ. ಎದುರಿಗೆ ತೆರೆ ಇದೆ. ಅದರ ಹಿಂದೆ ಒಂದು ಅದ್ಭುತವಾದ ದೃಶ್ಯವಿದೆ ಎಂದು ಭಾವಿಸಿ. ಆ ತೆರೆಯಲ್ಲಿ ಸಣ್ಣದೊಂದು ರಂಧ್ರವಿದೆ. ಅದರ ಮೂಲಕ ನಾವು ಹಿಂದೆ ಇರುವ ದೃಶ್ಯವನ್ನು ನೋಡುತ್ತೇವೆ. ಆ ರಂಧ್ರ ದೊಡ್ಡದಾಯಿತು ಎನ್ನಿ, ಆಗ ದೃಶ್ಯ ಹೆಚ್ಚು ಹೆಚ್ಚು ಕಾಣುತ್ತಾ ಬರುವುದು. ತೆರೆಯೆಲ್ಲ ಮಾಯವಾದ ಮೇಲೆ ದೃಶ್ಯಕ್ಕೂ ನಿಮಗೂ ಮಧ್ಯದಲ್ಲಿ ಏನೂ ಇರುವುದಿಲ್ಲ. ಈ ತೆರೆಯೆ ಮನುಷ್ಯನ ಮನಸ್ಸು. ಇದರ ಹಿಂದೆಯೇ ಆತ್ಮನ ಗಾಂಭೀರ್ಯ, ಪಾವಿತ್ರ್ಯ, ಶಕ್ತಿಗಳೆಲ್ಲ ಇವೆ. ಮನಸ್ಸು ತಿಳಿಯಾದಂತೆ, ಪರಿಶುದ್ಧವಾದಂತೆ ಆತ್ಮನ ಮಹಿಮೆ ಹೆಚ್ಚು ಹೆಚ್ಚು ವ್ಯಕ್ತವಾಗುತ್ತಾ ಹೋಗುವುದು. ಆತ್ಮವಲ್ಲ ಬದಲಾಗುವುದು. ಬದಲಾವಣೆ ತೆರೆಯಲ್ಲಿದೆ. ಆತ್ಮ ಅವಿಕಾರಿ, ಅಮೃತ, ಪವಿತ್ರ, ನಿತ್ಯ ಧನ್ಯವಾದುದು.

ಕೊನೆಗೆ ಸಿದ್ದಾಂತ ಹೀಗೆ ಆಗುವುದು. ಶ್ರೇಷ್ಠತಮ ವ್ಯಕ್ತಿಯಿಂದ ಅತಿ ನೀಚನವರೆಗೆ, ಮಾನವೋತ್ತಮನಿಂದ ಹಿಡಿದು ನಮ್ಮ ಕಾಲಕೆಳಗೆ ತೆವಳುತ್ತಿರುವ ಕೀಟದವರೆಗೆ ಎಲ್ಲರಲ್ಲಿಯೂ ಆತ್ಮ ಪವಿತ್ರವಾಗಿ ಪರಿಶುದ್ಧವಾಗಿದೆ, ಅನಂತವಾಗಿದೆ, ನಿತ್ಯಮುಕ್ತವಾಗಿದೆ. ಕೀಟ ಆತ್ಮದ ಎಲ್ಲೋ ಒಂದು ಅತ್ಯಲ್ಪ ಅಂಶವನ್ನು ಮಾತ್ರ ವ್ಯಕ್ತಗೊಳಿಸುತ್ತಿದೆ, ಅತಿ ಶ್ರೇಷ್ಠ ಮಾನವನಲ್ಲಿ ಅದು ಬಹುಮಟ್ಟಿಗೆ ವ್ಯಕ್ತವಾಗುತ್ತಿದೆ. ವ್ಯತ್ಯಾಸವಿರುವುದು ಅಭಿವ್ಯಕ್ತಿಯ ತರತಮದಲ್ಲಿ ಮಾತ್ರವೆ ಹೊರತು ವಸ್ತುವಿನ ಸಾರದಲ್ಲಿ ಅಲ್ಲ. ಎಲ್ಲರಲ್ಲಿಯೂ ಒಂದೇ ಪರಿಶುದ್ಧವಾದ ಪರಿಪೂರ್ಣವಾದ ಆತ್ಮ ಇದೆ.

ಸ್ವರ್ಗ ಮುಂತಾದ ಲೋಕಗಳು ಕೂಡ ಇವೆ. ಆದರೆ ಇವೆಲ್ಲ ಗೌಣ. ಸ್ವರ್ಗದ ಭಾವನೆಯನ್ನು ಅತಿ ಅಲ್ಪ ಎಂದು ಭಾವಿಸುವರು. ಭೋಗೇಚ್ಛೆಯಿಂದ ಇದು ಉದಯಿಸುವುದು. ಪ್ರಪಂಚವೆಲ್ಲ ನಮ್ಮ ಈಗಿನ ಅನುಭವ ಅಷ್ಟೆ ಎಂದು ನಾವು ಮಿತಿಗೊಳಿಸಲು ಯತ್ನಿಸುವೆವು. ಮಕ್ಕಳು, ಪ್ರಪಂಚದಲ್ಲೆಲ್ಲಾ ಬರಿಯ ಮಕ್ಕಳೇ ಇರುವರು ಎಂದು ಭಾವಿಸುವುವು, ಹುಚ್ಚರು ಪ್ರಪಂಚವೇ ದೊಡ್ಡದೊಂದು ಹುಚ್ಚರ ಆಸ್ಪತ್ರೆ ಎಂದು ಭಾವಿಸುವರು, ಇತ್ಯಾದಿ. ಯಾರಿಗೆ ಪ್ರಪಂಚ ಕೇವಲ ಭೋಗ ಭೂಮಿಯೋ, ಯಾರಿಗೆ ಪ್ರಪಂಚವೆಂದರೆ ಕೇವಲ ಊಟ ಉಪಚಾರಗಳಲ್ಲಿ ಕಳೆಯುವುದಾಗಿದೆಯೊ, ಅವರಿಗೂ ಪ್ರಾಣಿಗಳಿಗೂ ವ್ಯತ್ಯಾಸವೇನೂ ಇಲ್ಲ. ಅಂತಹ ಜನರೇ ಸ್ವಾಭಾವಿಕವಾಗಿ ಒಂದು ಸ್ವರ್ಗ ಲೋಕವನ್ನು ಕಲ್ಪಿಸಿಕೊಳ್ಳುವರು. ಏಕೆಂದರೆ ಅವರ ಭೋಗಕ್ಕೆ ಈ ಜೀವನ ಸಾಕಾಗುವುದಿಲ್ಲ. ಅವರ ಭೋಗಾಸಕ್ತಿಗೆ ಒಂದು ಮೇರೆ ಇಲ್ಲ. ಆದಕಾರಣವೇ ಯಾವ ತಡೆಯೂ ಇಲ್ಲದೆ ಇಂದ್ರಿಯ ಸುಖವನ್ನು ಪಡೆಯಬಹುದಾದ ಒಂದು ಲೋಕವನ್ನು ಅವರು ಚಿತ್ರಿಸಿಕೊಳ್ಳುವರು. ಯಾರು ಇಂತಹ ಸ್ಥಳಕ್ಕೆ ಹೋಗಬೇಕೆಂದು ಬಯಸುವರೋ ಅವರು ಅಲ್ಲಿಗೆ ಹೋಗಲೇಬೇಕು. ಈ ಲೋಕದ ಕನಸು ಆದಮೇಲೆ ಸ್ವರ್ಗಲೋಕದ ಕನಸಿಗೆ ಹೋಗುವರು. ಅಲ್ಲಿ ಅವರಿಗೆ ಬೇಕಾದಷ್ಟು ಭೋಗ್ಯವಸ್ತುಗಳಿವೆ. ಈ ಕನಸು ಭಗ್ನವಾದರೆ ಅವರು ಮತ್ತೊಂದು ಕನಸನ್ನು ಕಲ್ಪಿಸಿಕೊಳ್ಳಬೇಕಾಗುವುದು. ಹೀಗೆ ಅವರು ಕನಸಿನಿಂದ ಕನಸಿಗೆ ಚಲಿಸುತ್ತಿರುವರು.

ಅನಂತರ ಕೊನೆಯ ಸಿದ್ಧಾಂತಕ್ಕೆ ಬರುತ್ತೇವೆ. ಅದೇ ಆತ್ಮನ ಮತ್ತೊಂದು ಭಾವನೆ. ಆತ್ಮ ಸ್ವಭಾವತಃ ಪರಿಪೂರ್ಣವಾಗಿ ಪರಿಶುದ್ಧವಾಗಿದ್ದರೆ, ಪ್ರತಿಯೊಂದು ಆತ್ಮವೂ ಅನಂತವಾಗಿ ಸರ್ವವ್ಯಾಪಿಯಾಗಿದ್ದರೆ, ಇನ್ನೊಂದು ಆತ್ಮಗಳು ಹೇಗೆ ಇರುವುದಕ್ಕೆ ಸಾಧ್ಯ? ಹಲವು ಅನಂತ ವಸ್ತುಗಳು ಇರಲಾರವು. ಹಲವು ಮಾತ್ರವಲ್ಲ ಎರಡು ಕೂಡ ಇರಲಾರವು. ಎರಡು ಅನಂತ ವಸ್ತುಗಳಿದ್ದರೆ ಒಂದು ಇನ್ನೊಂದರ ಅನಂತತೆಗೆ ಭಂಗ ತರುವುದು. ಆಗ ಎರಡೂ ಸಾಂತವಾಗುವುವು. ಅನಂತ ಏಕಮಾತ್ರ ಆಗಿರಬಲ್ಲದು. ಅವರು ಧೈರ್ಯವಾಗಿ ಕೊನೆಯ ನಿರ್ಣಯಕ್ಕೆ ಬರುತ್ತಾರೆ. ಅದೇ ಆತ್ಮ ಒಂದು, ಎರಡಲ್ಲ ಎಂಬುದು.

ಎರಡು ಹಕ್ಕಿಗಳು ಒಂದೇ ಮರದ ಮೇಲೆ ಕುಳಿತಿವೆ. ಒಂದು ಮೇಲಿನ ರೆಂಬೆಯಲ್ಲಿದೆ, ಮತ್ತೊಂದು ಕೆಳಗಿನ ರಂಬೆಯಲ್ಲಿದೆ. ಎರಡಕ್ಕೂ ಸುಂದರವಾದ ಗರಿಗಳಿವೆ. ಒಂದು (ಕೆಳಗಿರುವುದು) ಹಣ್ಣನ್ನು ತಿನ್ನುವುದು, ಮತ್ತೊಂದು ತನ್ನ ಮಹಿಮೆಯಲ್ಲಿ ತಾನು ಪ್ರತಿಷ್ಠಿತವಾಗಿ ಶಾಂತಿಯಿಂದ ಗಂಭೀರವಾಗಿ ಕುಳಿತಿದೆ. ಕೆಳಗಿನ ಹಕ್ಕಿ ಒಳ್ಳೆಯ ಮತ್ತು ಕೆಟ್ಟ ಹಣ್ಣುಗಳನ್ನು ತಿನ್ನುತ್ತಿದೆ. ವಿಷಯ ವಸ್ತುಗಳ ಕಡೆ ಹೋಗುತ್ತಿದೆ. ಯಾವಾಗಲಾದರೂ ಒಮ್ಮೆ ಕಹಿ ಹಣ್ಣನ್ನು ತಿಂದಾಗ ಸ್ವಲ್ಪ ಮುಂದೆ ಹೋಗಿ ಮೇಲೆ ಕಹಿ-ಸಿಹಿಯಾದ ಯಾವ ಹಣ್ಣನ್ನೂ ತಿನ್ನದೆ ಆತ್ಮತೃಪ್ತನಾಗಿ ಕುಳಿತಿರುವ ಹಕ್ಕಿಯನ್ನು ನೋಡುವುದು. ಅದು ಆತ್ಮತೃಪ್ತ. ಹೊರಗಿನದೇನನ್ನೂ ಆಶಿಸದು. ಕೆಳಗಿನ ಹಕ್ಕಿ ಮೇಲಿನ ಹಕ್ಕಿಯನ್ನು ನೋಡಿ ಅದರ ಸಮೀಪಕ್ಕೆ ಹೋಗಲು ಯತ್ನಿಸುವುದು. ಅದು ಸ್ವಲ್ಪ ಮೇಲೆ ಹೋಗುವುದು. ಆದರೆ ಹಿಂದಿನ ಸಂಸ್ಕಾರ ಬಿಟ್ಟಿಲ್ಲ. ಪುನಃ ಅದೇ ಹಣ್ಣನ್ನು ತಿನ್ನುವುದು. ಪುನಃ ಯಾವುದೋ ಒಂದು ಅತಿ ಕಹಿಯಾದ ಹಣ್ಣನ್ನು ತಿಂದಾಗ ಅತಿ ವ್ಯಥೆಯಾಗಿ ಮೇಲೆ ನೋಡುವುದು. ಅಲ್ಲಿ ಅದೇ ಶಾಂತಿಯಿಂದ ಗಂಭೀರವಾಗಿರುವ ಹಕ್ಕಿಯನ್ನು ನೋಡುವುದು. ಅದು ಸ್ವಲ್ಪ ಹತ್ತಿರ ಹೋಗುವುದು, ಆದರೆ ಪುನಃ ತನ್ನ ಸಂಸಾರದ ಆವೇಗಕ್ಕೆ ಸಿಕ್ಕಿ ಅತ್ತ ಇತ್ತ ಚಲಿಸುವುದು. ಪುನಃ ಮತ್ತೊಂದು ಕಹಿ ಹಣ್ಣನ್ನು ತಿಂದಾಗ ಮೇಲೆ ನೋಡಿ ಮೇಲಿನ ಹಕ್ಕಿಯ ಹತ್ತಿರ ಹೋಗುವುದು. ಮೇಲಿನ ಹಕ್ಕಿಯ ಗರಿಗಳ ಕಾಂತಿ ತನ್ನ ಮೇಲೆ ಬೀಳುವುದು, ತನ್ನ ಪುಕ್ಕವೆಲ್ಲ ಮಾಯವಾಗತೊಡಗಿತು. ಅದು ಮೇಲಿನ ಹಕ್ಕಿಯ ಅತಿ ಸಮೀಪಕ್ಕೆ ಬಂದಾಗ ದೃಶ್ಯ ಬದಲಾಗುವುದು. ಕೆಳಗಿನ ಹಕ್ಕಿ ಇರಲೇ ಇಲ್ಲ. ಇದ್ದುದು ಮೇಲಿನ ಹಕ್ಕಿ ಮಾತ್ರ. ಯಾವುದನ್ನು ಕೆಳಗಿನ ಹಕ್ಕಿಯೆಂದು ಭಾವಿಸಿತ್ತೊ ಅದೊಂದು ಅಲ್ಪ ಪ್ರತಿಬಿಂಬ ಮಾತ್ರ.

\newpage

ಆತ್ಮನ ಸ್ವಭಾವ ಇಂತಹದು. ಮಾನವನ ಜೀವವು ಭೋಗವನ್ನು ಆರಿಸಿಕೊಂಡು ಹೋಗುವುದು, ಕ್ಷಣಿಕ ವಸ್ತುಗಳನ್ನು ಅರಸಿಕೊಂಡು ಹೋಗುವುದು. ಪ್ರಾಣಿಗಳಂತೆ ಜನರು ಇಂದ್ರಿಯದಲ್ಲಿ ನೆಲಸಿರುವರು, ದೇಹದ ಕ್ಷಣಿಕ ಸುಖದಲ್ಲಿ ತೃಪ್ತರಾಗಿರುವರು. ಪೆಟ್ಟೊಂದು ಬಿದ್ದಾಗ, ತಲೆ ಸುತ್ತುವುದು, ಎಲ್ಲಾ ಮಾಯವಾದಂತೆ ಕಾಣುವುದು. ಜಗತ್ತು ತಾನು ಭಾವಿಸಿದಂತೆ ಇಲ್ಲ, ಜೀವನ ಅಷ್ಟು ಸುಖಮಯವಲ್ಲ ಎಂದು ತಿಳಿಯುವುದು. ಕತ್ತೆತ್ತಿ ಕ್ಷಣಕಾಲ ಭಗವಂತನನ್ನು ಜೀವ ನೋಡುವುದು, ಗಾಂಭೀರ್ಯದಲ್ಲಿ ಪ್ರತಿಷ್ಠಿತನಾದ ಭಗವಂತನ ಕ್ಷಣಿಕ ದರ್ಶನ ಪಡೆಯುವುದು, ದೇವರ ಹತ್ತಿರ ಬರುವುದು. ಆದರೆ ಪುನಃ ಅದನ್ನು ಹೀನ ಸಂಸ್ಕಾರಗಳು ದೂರ ಸೆಳೆಯುವುವು. ಮತ್ತೊಂದು ಪೆಟ್ಟು ಬಿದ್ದು ದೇವರೆಡೆಗೆ ಪುನಃ ಕಳುಹಿಸುವುದು. ಭಗವಂತನ ದರ್ಶನವನ್ನು ಕ್ಷಣಕಾಲ ಪಡೆದು ಅವನ ಸಮೀಪಕ್ಕೆ ಬರುವುದು. ಅದು ಭಗವಂತನ ಸಮೀಪಕ್ಕೆ ಬಂದಂತೆಲ್ಲ ಪ್ರಕೃತಿ ಮಾಯವಾಗುವುದು. ಅದು ಸಾಧ್ಯವಾದಷ್ಟೂ ದೇವರ ಸಮೀಪಕ್ಕೆ ಬಂದ ಮೇಲೆ ದೃಶ್ಯವೆಲ್ಲ ಬದಲಾಗುವುದು. ಯಾವ ಅನಂತವಾದುದನ್ನು ತನ್ನ ಹೊರಗಡೆ ಇದೆ ಎಂದು ಭಾವಿಸಿತ್ತೋ ಅದು ತಾನೆ ಆಗಿರುವುದು. ಎಲ್ಲಿ ಮಹಿಮೆಯನ್ನು ಮತ್ತು ಗಾಂಭೀರ್ಯವನ್ನು ಅದು ಹೊರಗೆ ನೋಡಿತೋ ಅದು ತನ್ನದೇ ಆಗಿರುವುದು. ಅದೇ ತನ್ನ ನೈಜಸ್ವಭಾವ ಎಂದು ಗೊತ್ತಾಗುವುದು. ಎಲ್ಲದರಲ್ಲಿಯೂ ಯಾವುದು ಸತ್ಯವಾಗಿದೆಯೋ ಅದನ್ನು ಆತ್ಮ ಎಂದು ಆಗ ಅರಿಯುವುದು. ಯಾವುದು ಪ್ರತಿಯೊಂದು ಕಣದ ಅಂತರಾಳದಲ್ಲಿ ಸರ್ವವ್ಯಾಪಿಯಾಗಿರುವುದೊ, ಎಲ್ಲ ವಸ್ತುಗಳ ಸಾರವಾಗಿದೆಯೋ, ಯಾರು ವಿಶ್ವೇಶ್ವರವೊ, ಅವನೇ ನಾನು ಎಂದು ಅರಿ, ಮುಕ್ತ ನೀನೆಂದು ಅರಿ.

