
\chapter[ಜೀವಾತ್ಮ ಮತ್ತು ಪರಮಾತ್ಮ]{ಜೀವಾತ್ಮ ಮತ್ತು ಪರಮಾತ್ಮ\protect\footnote{\engfoot{C.W, Vol. I, P. 489}}}

\begin{center}
(೧೯೦೦ ಮಾರ್ಚ್ ೨೩ ರಂದು ಸ್ಯಾನ್‌ಫ್ರಾನ್ಸಿಸ್ಕೋನಲ್ಲಿ ನೀಡಿದ ಉಪನ್ಯಾಸ)
\end{center}

ತನಗಿಂತಲೂ ಮಿಗಿಲಾದ ಶಕ್ತಿಗಳನ್ನು ಕುರಿತು ಆಲೋಚಿಸುವಂತೆ ಮನುಷ್ಯನಿಗೆ ಮೊದಲು ಪ್ರೇರೇಪಣೆ ಒದಗಿಸಿದ್ದು ಭಯವೋ ಅಥವಾ ಕೇವಲ ಕುತೂಹಲವೋ ಎಂಬುದರ ಬಗ್ಗೆ ಚರ್ಚೆಮಾಡುವುದು ನಮಗೆ ಅಗತ್ಯವಿಲ್ಲ... ಇವು ಅವನ ಮನಸ್ಸಿನಲ್ಲಿ ವಿಚಿತ್ರವಾದ ಪೂಜಾ ಪ್ರವೃತ್ತಿಗಳೇ ಮುಂತಾದುವನ್ನು ಜಾಗೃತಗೊಳಿಸಿದುವು. (ಮಾನವ ಜಾತಿಯ ಇತಿಹಾಸದಲ್ಲಿ ಯಾವುದಾದರೂ ಆದರ್ಶದ) ಪೂಜೆಯೇ ಇರದಿದ್ದ ಕಾಲ ಎಂದೂ ಇರಲಿಲ್ಲ. ಹೀಗೇಕೆ? ಮೋಹಕವಾದ ಮುಂಜಾನೆಯೇ ಆಗಿರಬಹುದು ಅಥವಾ ಭೂತಪ್ರೇತಗಳ ಭಯವೇ ಆಗಿರಬಹುದು – ಆದರೆ ನಾವು ನೋಡುವುದರ ಆಚೆಗಿರುವ ಅತೀತಕ್ಕಾಗಿ ನಮ್ಮನ್ನು ಹೋರಾಡುವಂತೆ ಮಾಡುವ ರಹಸ್ಯವಾವುದು?... (ಇದನ್ನರಿಯಲು) ನಾವು ಇತಿಹಾಸ ಪೂರ್ವದ ಯುಗಕ್ಕೆ ಹಿಂತಿರುಗುವ ಆವಶ್ಯಕತೆಯೇನೂ ಇಲ್ಲ. ಕಾರಣ, ಎರಡು ಸಾವಿರ ವರ್ಷಗಳ ಹಿಂದೆ ವಾಸ್ತವವಾಗಿದ್ದ ತಥ್ಯವೇ ಇಂದೂ ಇದೆ. ಬದುಕಿನಲ್ಲಿ ನಮ್ಮ ಅಂತಸ್ತು ಯಾವುದೇ ಇರಲಿ – ನಾವು ಶ‍್ರೀಮಂತರು ಹಾಗೂ ಶಕ್ತಿಶಾಲಿಗಳೂ ಆಗಿದ್ದಾಗ್ಯೂ ಸಹ – ನಾವು ಸಂತೃಪ್ತರಂತೂ ಅಲ್ಲ.

ಆಸೆಗಳು ಅಪಾರ. ಅದರ ಪೂರೈಕೆ ಮಾತ್ರ ಬಹಳ ಮಿತಿಯುಳ್ಳದ್ದು. ನಮ್ಮ ಆಸೆಗಳಿಗೆ ಕೊನೆಯೆಂಬುದೇ ಇಲ್ಲ. ಆದರೆ ಅವನ್ನು ಈಡೇರಿಸಿಕೊಳ್ಳಲು ಹೊರಟಾಗ ನಮಗೆ ಕಷ್ಟಗಳು ಎದುರಾಗುತ್ತವೆ. ಆದಿಮಾನವನ ಮನಸ್ಸಿನಲ್ಲಿ ಕೆಲವೇ ಕಾಮನೆಗಳಿದ್ದಾಗ್ಯೂ – ಆಗಲೂ ಹೀಗೆಯೇ ಇತ್ತು; ಆ ಆಸೆಗಳನ್ನೂ ಸಹ ಅವನು ಪೂರೈಸಿಕೊಳ್ಳಲಾಗುತ್ತಿರಲಿಲ್ಲ. ಈಗ ಕಲೆ – ವಿಜ್ಞಾನಗಳು ದ್ವಿಗುಣಗೊಂಡು ವರ್ಧಿಸುತ್ತಿದ್ದರೂ ನಮ್ಮ ಆಸೆಗಳನ್ನು ಪೂರೈಸಿಕೊಳ್ಳಲಾಗುತ್ತಿಲ್ಲ. ಬದಲಾಗಿ, ನಮ್ಮ ಆಸೆಗಳನ್ನು ಪೂರೈಸುವ ಸಾಧನಗಳನ್ನು ನಾವು ಸಿದ್ಧಪಡಿಸಲು ಹವಣಿಸಿದಷ್ಟೂ – ನಮ್ಮ ಆಸೆಗಳು ಹೆಚ್ಚುತ್ತಲೇ ಇವೆ.

ಆದಿ ಮಾನವನು ತನ್ನ ಕೈಯಿಂದ ಸಾಧಿಸಲಾಗದ್ದರ ವಿಷಯಗಳಲ್ಲಿ ಸಹಜವಾಗಿ ಬಾಹ್ಯ (ಅಲೌಕಿಕ) ಶಕ್ತಿಗಳ ಸಹಾಯವನ್ನು ಕೋರುತ್ತಿದ್ದನು. ಅವನು ಏನನ್ನಾದರೂ ಬಯಸಿ, ಅದನ್ನು ಪಡೆಯಲಾಗದಿದ್ದಲ್ಲಿ ಅವನು ಇತರೆ (ಅಲೌಕಿಕ) ಶಕ್ತಿಗಳ ಸಹಾಯವನ್ನು ಕೋರುತ್ತಿದ್ದನು, ಅಪೇಕ್ಷಿಸುತ್ತಿದ್ದನು. ತಮ್ಮ ಯಾವುದೇ ಬಯಕೆಯನ್ನು ಪೂರೈಸೆಂದು ಕೇಳಿಕೊಂಡು ದೇವರಲ್ಲಿ ಮೊರೆಹೋಗುವುದರಲ್ಲಿ – ಯಾವುದೇ ತಿಳುವಳಿಕೆ ಇಲ್ಲದ ಆದಿಮಾನವ ಮತ್ತು ಇಂದಿನ ಅತ್ಯಂತ ಸುಸಂಸ್ಕೃತ – ಇವರಿಬ್ಬರೂ ಒಂದೇ. ಇವರ ನಡುವೆ ಯಾವುದೇ ವ್ಯತ್ಯಾಸವೂ ಇಲ್ಲ. (ಇವರಿಬ್ಬರ ನಡುವೆ) ಏನು ವ್ಯತ್ಯಾಸ? (ಕೆಲವರು ಇವರಿಬ್ಬರ ನಡುವೆ) ದೊಡ್ಡ ವ್ಯತ್ಯಾಸವನ್ನು ಕಾಣುತ್ತಾರೆ. ನಾವು ಯಾವ ವಿಷಯಗಳ ನಡುವೆ ಏನೂ ವ್ಯತ್ಯಾಸವೇ ಇಲ್ಲವೋ ಅಲ್ಲಿ ಬಹು ದೊಡ್ಡ ವ್ಯತ್ಯಾಸವನ್ನು ಕಾಣುತ್ತೇವೆ. ಇಬ್ಬರೂ ಸಹ (ಆದಿ ಮಾನವ ಮತ್ತು ಸುಸಂಸ್ಕೃತ ಮನುಷ್ಯ) ಒಂದೇ (ಶಕ್ತಿಗೆ) ಮೊರೆಹೋಗುತ್ತಾರೆ. ನೀವು ದೇವರು ಎನ್ನಿ, ಅಲ್ಲಾ ಎನ್ನಿ ಅಥವಾ ಜಿಹೋವಾ ಎನ್ನಿ (ಎಲ್ಲಾ ಒಂದೇ) ಮನುಷ್ಯರು ಏನನ್ನಾದರೂ ಬಯಸುತ್ತಾರೆ ಮತ್ತು ತಮ್ಮ ಶಕ್ತಿಯಿಂದಲೇ ಅದನ್ನು ಪಡೆಯಲಾಗದಿದ್ದಾಗ, ಅವರಿಗೆ ಸಹಾಯ ಮಾಡುವ ಯಾರಾದರೂ ಒಬ್ಬರ ಹುಡುಕಾಟದಲ್ಲಿರುತ್ತಾರೆ. ಇದು ಆದಿಮಾನವನ ವೈಶಿಷ್ಟ್ಯತೆ, ಅದು ಇನ್ನೂ ನಮ್ಮಲ್ಲಿಯೂ ಉಳಿದಿದೆ... ನಾವೆಲ್ಲ ಹುಟ್ಟಿನಿಂದ ಕಾಡುಮನುಷ್ಯರು, ಆದರೆ ಕ್ರಮಶಃ ಸಭ್ಯತೆಯುಳ್ಳವರಾಗುತ್ತೇವೆ. ಇಲ್ಲಿರುವ ನಾವೆಲ್ಲರೂ ನಮ್ಮನ್ನು ಶೋಧಿಸಿಕೊಂಡಾಗ ಈ ತಥ್ಯವನ್ನು ಮನಗಾಣುತ್ತೇವೆ. ಈ (ಅಲೌಕಿಕ ಶಕ್ತಿಯ) ಭಯ ಈಗಲೂ ನಮ್ಮನ್ನು ಬಿಟ್ಟಿಲ್ಲ. ನಾವು ದೊಡ್ಡ ದೊಡ್ಡ ಮಾತುಗಳನ್ನಾಡಬಹುದು, ದಾರ್ಶನಿಕರೇ ಮೊದಲಾಗಿ ಏನೆಲ್ಲಾ ಆಗಿರಬಹುದು. ಆದರೆ ಆಘಾತಗಳು ಬಂದಾಗ, ನಾವು ಸಹಾಯ ಯಾಚಿಸುವುದನ್ನು ಕಾಣಬಹುದು. ಆಗ ಅಸ್ತಿತ್ವದಲ್ಲಿರುವ ಎಲ್ಲ ಮೂಢಾಚರಣೆಗಳನ್ನೂ ನಂಬಿ ಅವುಗಳಿಗೆ ಶರಣುಹೋಗುತ್ತೇವೆ. (ಆದರೆ ಸತ್ಯದ ಆಧಾರವನ್ನು ಸ್ವಲ್ಪವಾದರೂ ಹೊಂದಿರದ) ಯಾವ ಮೂಢಾಚರಣೆಯೂ ಈ ಪ್ರಪಂಚದಲಿಲ್ಲ. ನಾನು ನನ್ನ ಮುಖವನ್ನೆಲ್ಲ ಮುಚ್ಚಿಕೊಂಡು, ನನ್ನ ಮೂಗಿನ ಎಲ್ಲೋ ಸ್ವಲ್ಪ ಭಾಗ ಕಾಣಿಸಿಕೊಳ್ಳುತ್ತಿದ್ದರೆ, ನನ್ನ ಮುಖವೇ ಸ್ವಲ್ಪ ಮಟ್ಟಿಗೆ ಕಾಣಿಸಿಕೊಂಡ ಹಾಗೆ. ಹಾಗೆಯೇ ಮೂಢಾಚರಣೆಗಳೂ ಕೂಡ... ಅವುಗಳ ಯಾವುದೋ ಸ್ವಲ್ಪ ಭಾಗ ನಿಜವಿರುತ್ತದೆ.

ಸತ್ತವರ ಅಂತ್ಯಕ್ರಿಯೆಯಲ್ಲಿ, ಧರ್ಮದ ಅತ್ಯಂತ ಕ್ಷುದ್ರವಾದ ರೂಪವನ್ನು ನೀವು ಕಾಣುತ್ತೀರಿ. ಮೃತರಾದವರನ್ನು ಶವದ ಪೆಟ್ಟಿಗೆಯಲ್ಲಿಟ್ಟು ಗೋರಿಗಳಲ್ಲಿಡುತ್ತಿದ್ದರು ಮತ್ತು ಮೃತಾತ್ಮಗಳು (ರಾತ್ರಿಯ ವೇಳೆ ಆ ಗೋರಿಗಳಲ್ಲಿ) ಬಂದಿರುತ್ತಿದ್ದವು... ನಂತರ (ಆ ಶವಗಳನ್ನು) ಹೂಳಿ ಮಣ್ಣು ಮಾಡಲು ಪ್ರಾರಂಭಿಸಿದರು... ಬಾಗಿಲಿನಲ್ಲಿ ಸಾವಿರ ದಂಷ್ಟ್ರಗಳುಳ್ಳ ಭೀಕರವಾದ ದೇವತೆಯಿರುತ್ತಿದ್ದಳು... ನಂತರ ಹೆಣವನ್ನು ಸುಡುವ ಪದ್ದತಿ (ಬಂದಿತು) ಮತ್ತು ಆ ಜ್ವಾಲೆಗಳು ಆತ್ಮವನ್ನು ತನ್ನೊಡನೆ ಮೇಲಕ್ಕೆ\break ಕೊಂಡೊಯ್ಯುತ್ತಿರುವಂತೆ (ಎಣಿಸಲಾಯಿತು). ಈಜಿಪ್ಷಿಯನ್ನರು ಮೃತಾತ್ಮರಿಗೋಸ್ಕರ ಅನ್ನಾಹಾರಗಳ ವ್ಯವಸ್ಥೆಯನ್ನೂ ಮಾಡುತ್ತಿದ್ದರು.

ಮತ್ತೊಂದು ಮಹತ್ವಪೂರ್ಣವಾದ ವಿಚಾರವೇ ಬುಡಕಟ್ಟಿನವರ ದೇವತೆಗಳದು. ಒಂದು ಬುಡಕಟ್ಟಿನವರಿಗೆ ಯಾವುದೋ ಒಂದು ದೇವತೆಯಿದ್ದರೆ ಮತ್ತೊಂದು ಬುಡಕಟ್ಟಿನವರಿಗೆ ಬೇರೊಂದು ದೇವತೆಯಿರುತ್ತಿತ್ತು. ಯಹೂದಿಗಳ ದೇವರು ಜಿಹೋವಾ – ಇದು ಆ ಬುಡಕಟ್ಟಿನವರದೇ ದೇವತೆಯಾಗಿದ್ದು – ಮತ್ತು ಇತರೆ ಎಲ್ಲ ದೇವತೆಗಳೊಂದಿಗೂ ಬುಡಕಟ್ಟಿನವರೊಂದಿಗೂ ಕಾದಾಡುತ್ತಿದ್ದರು. ಆ ದೇವತೆ ತನ್ನ ಜನರನ್ನು ಸಂತುಷ್ಟಗೊಳಿಸಲು ಏನನ್ನು ಬೇಕಾದರೂ ಮಾಡುತ್ತಿತ್ತು. ತನ್ನ ರಕ್ಷಣೆಗೆ ಒಳಗಾಗದ ಬೇರೊಂದು ಬುಡುಕಟ್ಟಿನ ಇಡೀ ಜನಾಂಗವನ್ನೇ ಕೊಂದರೂ ಅದು ಸರಿಯಾಗಿರುತ್ತಿತ್ತು ಮತ್ತು ಒಳ್ಳೆಯದೆಂದು (ಎಣಿಸಲಾಗುತ್ತಿತ್ತು). ಆ ದೇವತೆಯಿಂದ ಸ್ವಲ್ಪ ಪ್ರೀತಿ ಸಿಗುತ್ತಿತ್ತು; ಆದರೆ ಆ ಪ್ರೀತಿ ಒಂದು ಸಣ್ಣ ಗುಂಪಿಗಷ್ಟೇ ಸೀಮಿತವಾಗಿತ್ತು.

ಕ್ರಮಶಃ ಉಚ್ಚತರ ಆದರ್ಶಗಳು ಬಂದವು. ವಿಜಯಿಗಳಾದ ಬುಡಕಟ್ಟಿನ\break ಮುಖಂಡನೇ ಎಲ್ಲ ಮುಖಂಡರ ಮುಖಂಡನಾಗಿದ್ದ, ದೇವತೆಗಳ ದೇವರಾಗಿದ್ದ... ಪರ್ಷಿಯನ್ನರು ಈಜಿಪ್ಟಿಗೆ ಮುತ್ತಿಗೆ ಹಾಕಿ ಗೆದ್ದಾಗ ಈ ರೀತಿಯೇ ಆಯಿತು.\break ಪರ್ಷಿಯನ್ನರ ಚಕ್ರವರ್ತಿಯೇ ದೇವದೇವನಾದ. ಚಕ್ರವರ್ತಿಯ ಮುಂದೆ ಯಾರೂ ನಿಲ್ಲಲಾಗುತ್ತಿರಲಿಲ್ಲ. ಆ ಪರ್ಷಿಯನ್ ಚಕ್ರವರ್ತಿಯನ್ನು ಯಾರಾದರು ಕಣ್ಣೆತ್ತಿ ನೋಡಿದರೂ ಸಾಕು – ಅವನಿಗೆ ಮರಣದಂಡನೆಯಾಗುತ್ತಿತ್ತು.

ನಂತರವೇ ಈ ಜಗತ್ತನ್ನು ಆಳುವ ಸರ್ವೆಶ್ವರನಾದ, ಸರ್ವಶಕ್ತನಾದ, ಸರ್ವವ್ಯಾಪಿಯಾದ, ಸರ್ವಾತ್ಮನಾದ, ದೇವರ ಆದರ್ಶ ಮೂಡಿತು. ಆತ ಸ್ವರ್ಗದಲ್ಲಿರುತ್ತಾನೆ; ಮತ್ತು ಮನುಷ್ಯನಿಗೋಸ್ಕರ ಪ್ರತಿಯೊಂದನ್ನೂ ಸೃಷ್ಟಿಸಿರುವ ಪರಮಪ್ರಿಯನಾದ ಆತನಿಗೆ ಮನುಷ್ಯನು ವಿಶೇಷ ಗೌರವಾದರಗಳನ್ನು ತೋರುತ್ತಾನೆ. ಈ ಇಡೀ ಪ್ರಪಂಚವೇ ಮನುಷ್ಯನಿಗಾಗಿ ಇರುತ್ತದೆ. ಸೂರ್ಯ, ಚಂದ್ರ, ನಕ್ಷತ್ರಗಳೆಲ್ಲವೂ\break (ಅವನಿಗೋಸ್ಕರವೇ)... ಈ ರೀತಿಯ ವಿಚಾರಗಳಿರುವ ಎಲ್ಲರೂ ಆದಿ ಮಾನವರಷ್ಟೇ; ಸಭ್ಯ ಹಾಗೂ ಸುಸಂಸ್ಕೃತರಂತೂ ಅಲ್ಲವೇ ಅಲ್ಲ. ಎಲ್ಲ ಶ್ರೇಷ್ಠವಾದ ಧರ್ಮಗಳೂ ಗಂಗಾ ಮತ್ತು ಯೂಫ್ರೆಟೀಸ್‌ಗಳ ನಡುವೆ ತಮ್ಮ ಪ್ರವರ್ಧಮಾನವನ್ನು ಕಂಡವು. ಭರತವರ್ಷದ ಹೊರಗೆ ನಾವು (ಸ್ವರ್ಗದಲ್ಲಿರುವ ಈಶ್ವರನ ಕಲ್ಪನೆಗಿಂತ ಆಚೆಗಿನ ಧರ್ಮದ) ಬೆಳವಣಿಗೆಯನ್ನು ಕಾಣುವುದೇ ಇಲ್ಲ. ಭರತವರ್ಷದ ಹೊರಗೆ ಬೇರೆಕಡೆ (ದೇವರ) ಸಂಬಂಧವಾಗಿ ಉಪಲಬ್ಧವಿರುವ ಅತ್ಯಂತ ಶ್ರೇಷ್ಠವಾದ, ಸರ್ವೊಚ್ಛವಾದ (ಜ್ಞಾನ)ವೆಂದರೆ (ಸ್ವರ್ಗದಲ್ಲಿರುವ ದೇವರ) ಭಾವನೆಯಷ್ಟೇ. ಅವನು ತನ್ನ ಆವಾಸ ಸ್ಥಾನವಾದ ಸ್ವರ್ಗದಲ್ಲಿರುತ್ತಾನೆ ಮತ್ತು ಶ್ರದ್ಧಾಳುಗಳು ಸತ್ತನಂತರ (ಅಲ್ಲಿಗೆ) ಹೋಗುತ್ತಾರೆ... ನನಗೆ ಕಂಡಂತೆ (ಈ ಭಾವನೆಯನ್ನು), ಬಲಿಯದ ಅತ್ಯಂತ ಬಾಲಿಶವಾದದ್ದೆಂದೇ ಕರೆಯಬೇಕು. ಆಫ್ರಿಕಾದ ಬುಡಕಟ್ಟಿನವರು ಪೂಜಿಸುತ್ತಿದ್ದರೆಂದು ಹೇಳಲಾದ ಮೂಢಾರಾಧನೆಯ ವಿಕಾರ ವಿಗ್ರಹವಾದ ಮುಂಬೋಜುಂಬೋ ಮತ್ತು ಸ್ವರ್ಗದಲ್ಲಿರುವ ದೇವರ ಭಾವನೆ ಎರಡೂ ಒಂದೇ. ಜಗತ್ತನ್ನು ನಡೆಸುತ್ತಿರುವವನು ಆ ದೇವರೇ. ಅವನ ಇಚ್ಛಾನುಸಾರವಾಗಿಯೇ ಎಲ್ಲವೂ ಆಗುತ್ತಿದೆ.

ಪ್ರಾಚೀನ ಯಹೂದಿಗಳು ಯಾವುದೇ ಸ್ವರ್ಗವನ್ನೂ ಲೆಕ್ಕಿಸುತ್ತಿರಲಿಲ್ಲ. ಅವರು ನಜರೆತ್‌ನ ಜೀಸಸ್‌ನನ್ನು (ವಿರೋಧಿಸಲು) ಅದು ಒಂದು ಕಾರಣವಾಗಿತ್ತು – ಏಕೆಂದರೆ ಸತ್ತನಂತರವೂ ಇರುವ ನಿತ್ಯ (ಸ್ವರ್ಗ ಅಥವ ನರಕ) ಜೀವನದ ಬಗ್ಗೆ ಅವನು ಬೋಧಿಸುತ್ತಿದ್ದನು. ಪ್ಯಾರಾಡೈಸ್ ಅಥವಾ ಸ್ವರ್ಗಕ್ಕೆ ಸಂಸ್ಕೃತದಲ್ಲಿ ಅರ್ಥ, ಈ ಜೀವನದ ಆಚೆ ಇರುವ ಲೋಕ ಅಥವಾ ಪರಲೋಕವೆಂದು. ಆದ್ದರಿಂದ ಸ್ವರ್ಗ ಈ ಜೀವನದ ಎಲ್ಲ ಅನಿಷ್ಟ, ಪಾಪಗಳ ನಿವಾರಣೆ ಮಾಡುವಂತಿರಬೇಕು. ಆದಿಮಾನವ ಅನಿಷ್ಟವನ್ನು ಗಣನೆಗೆ ತೆಗೆದುಕೊಳ್ಳಲಿಲ್ಲ, ಅದು ಏಕೆ ಇದೆ ಎಂದು ಅವನೆಂದೂ ಪ್ರಶ್ನಿಸಿದವನಲ್ಲ.

ಡೆವಿಲ್ ಎಂಬುದು ಪರ್ಷಿಯನ್ ಪದ. ತಮ್ಮ ಧಾರ್ಮಿಕ ಚಿಂತನೆಯ ನೆಲೆಯಾಗಿ ಹಿಂದೂಗಳ ಮತ್ತು ಪರ್ಷಿಯನ್ನರ ಪಾಲಿಗೆ (ಆರ್ಯರೇ ಪೂರ್ವಜರಾಗಿದ್ದರು). ಮತ್ತು... ಅವರಿಬ್ಬರೂ ಒಂದೇ ರೀತಿಯ ಭಾಷೆಯನ್ನು ಮಾತನಾಡುತ್ತಿದ್ದರು; (ಆದರೆ) ಒಬ್ಬರು ಶುಭವಾಚಕವಾಗಿ ಉಪಯೋಗಿಸಿದ ಶಬ್ದವನ್ನು ಮತ್ತೊಬ್ಬರು ಅಶುಭವಾಚಕವಾಗಿ ಉಪಯೋಗಿಸುತ್ತಿದ್ದರು ಅಷ್ಟೇ. ಹಳೆಯ ಸಂಸ್ಕೃತದಲ್ಲಿ 'ದೇವ' ಎಂಬುದು ದೇವರನ್ನು (ಸೂಚಿಸುವುದಕ್ಕಾಗಿ) ಇತ್ತು – ಆರ್ಯಭಾಷೆಗಳಲೆಲ್ಲ ದೇವರಿಗೆ\break ಪರ್ಯಾಯಪದ ಅದೇ ಆಗಿದೆ. ಆದರೆ ಇಲ್ಲಿ (ಪರ್ಷಿಯನ್ ಭಾಷೆಯಲ್ಲಿ) ಈ ಪದ ಸೈತಾನನ \enginline{(Devil}ನ) ಸೂಚಕವಾಗಿದೆ.

ನಂತರ ಯಾವಾಗ ಮನುಷ್ಯ (ತನ್ನ ಆಂತರಿಕ ಜೀವನವನ್ನು) ಬೆಳೆಸಿಕೊಂಡನೋ ಆಗ ಅವನು ಜಿಜ್ಞಾಸೆ ಮಾಡಲು ತೊಡಗಿದನು ಹಾಗೂ ದೇವರನ್ನು ಒಳ್ಳೆಯವ, ಶಿವ, ಎನ್ನಲು ಪ್ರಾರಂಭಿಸಿದನು. ಪರ್ಷಿಯನ್ನರು ಇಬ್ಬರು ದೇವರುಗಳು ಇದ್ದಾರೆಂದು ಹೇಳಿದರು – ಒಬ್ಬನು ಕೆಟ್ಟವ (ಅರ್ಹಿಮನ್‌) ಇನ್ನೊಬ್ಬ ಒಳ್ಳೆಯವ (ಅಹರ್ಮಜ್‌ದ್). (ಅವರ ವಿಚಾರ ಹೇಗಿತ್ತೆಂದರೆ) ಈ ಜೀವನದಲ್ಲಿರುವುದೆಲ್ಲಾ ಒಳ್ಳೆಯದೇ ಆಗಿತ್ತು: ಈ ರಮಣೀಯ ದೇಶ ಇಡೀ ವರ್ಷವೆಲ್ಲಾ ವಸಂತಗಳಿಂದ ಭರಿತವಾಗಿತ್ತು. ಯಾರೂ ಸಾಯುತ್ತಲೇ ಇರಲಿಲ್ಲ, ರೋಗವೆಂಬುದೇ ಇರಲಿಲ್ಲ – ಎಲ್ಲವೂ ಕೂಡ ಅಚ್ಚುಕಟ್ಟಾಗಿ ಸುಂದರವಾಗಿತ್ತು. ನಂತರ ಈ ಅನಿಷ್ಟ ಸೈತಾನ ಈ ಭೂಮಿಯಲ್ಲಿ ಕಾಲಿಟ್ಟಿದ್ದೇ ತಡ ಇಲ್ಲಿ ಸಾವು, ನೋವು, ರೋಗಗಳು, ಸೊಳ್ಳೆಗಳು, ಹುಲಿಗಳು, ಸಿಂಹಗಳು – ಎಲ್ಲ ದುಷ್ಟಜಂತುಗಳು ಬಂದವು. ನಂತರ ಆರ್ಯರು ತಮ್ಮ ಪಿತೃಭೂಮಿಯನ್ನು ಬಿಟ್ಟು ದಕ್ಷಿಣದ ಕಡೆ ವಲಸೆಹೋದರು. ಪುರಾತನ ಆರ್ಯರು ಉತ್ತರದ ದಾರಿ ಹಿಡಿದೇ ಸಾಗಿರಬೇಕು. ಯಹೂದ್ಯರು ಪರ್ಷಿಯನ್ನರಿಂದ (ಈ ಸೈತಾನನ ಭಾವನೆಯನ್ನು) ಕಲಿತರು. ಈ ದುರಾತ್ಮನ (ಸೈತಾನನ) ಸಂಹಾರ ಆಗುವ ಒಂದು ದಿನ ಬರುತ್ತದೆಂದೂ ಸಹ ಪರ್ಷಿಯನ್ನರು ಬೋಧಿಸಿದ್ದರು; ಮತ್ತು ನಮ್ಮ ಕರ್ತವ್ಯವೇನೆಂದರೆ ನಾವು ಈ ಒಳ್ಳೆಯ ದೇವರೊಡನೆ ಇದ್ದು, ಅವನ ಮತ್ತು ಈ ದುರಾತ್ಮನ (ಸೈತಾನನ) ನಡುವೆ ಜರುಗುತ್ತಿರುವ ಚಿರಂತನವಾದ ಸಂಗ್ರಾಮದಲ್ಲಿ ನಮ್ಮ ಶಕ್ತಿಯನ್ನು ಅವನ ಶಕ್ತಿಯೊಡನೆ ಸೇರಿಸಿ ಅದನ್ನು ಬಲಪಡಿಸುವುದು... ಇಡೀ ಪ್ರಪಂಚವೇ ಭಸ್ಮೀಭೂತವಾಗುವುದು ಮತ್ತು ಪ್ರತಿಯೊಬ್ಬರೂ ಹೊಸ ಶರೀರವನ್ನು ಪಡೆಯುವರು.

ಪರ್ಷಿಯನ್ನರ ವಿಚಾರಧಾರೆ – ದುರಾತ್ಮನೂ ಸಹ ಪರಿಶುದ್ದನಾಗುತ್ತಾನೆ ಮತ್ತು ಅವನಿನ್ನೆಂದೂ ಕೆಟ್ಟವನಾಗುವುದಿಲ್ಲ – ಎಂಬುದಾಗಿತ್ತು. ಆರರು ಸ್ವಭಾವತಃ ಕಾವ್ಯ ಮತ್ತು ಒಲವಿನ ಪ್ರವೃತ್ತಿಯುಳ್ಳವರಾಗಿದ್ದರು. ಶಾಶ್ವತವಾಗಿ ತಾವು ಭಸ್ಮವಾಗಿ ಹೋಗುವುದನ್ನು ಅವರಿಗೆ ಕಲ್ಪಿಸಿಕೊಳ್ಳಲು ಸಾಧ್ಯವಾಗುತ್ತಿರಲಿಲ್ಲ. (ಆದ್ದರಿಂದಲೇ) ಅವರು ಹೊಸ\break ಶರೀರವನ್ನು ಪಡೆಯುವರು. ನಂತರ ಸಾವೆಂಬುದೇ ಇರುವುದಿಲ್ಲ. ಭಾರತದ ಹೊರಗಡೆ, ಧಾರ್ಮಿಕ ಚಿಂತನೆಗಳ ಬಗ್ಗೆ ಸರ್ವಶ್ರೇಷ್ಠವಾದ (ಭಾವನೆಯೆಂದರೆ) ಇದೇ.

ಅದರೊಂದಿಗೆ ನೈತಿಕ ಧೋರಣೆ ಯ ಇತ್ತು. ಮನುಷ್ಯ ಮೂರು ವಿಷಯಗಳ ಬಗ್ಗೆಯ ಎಚ್ಚರ ವಹಿಸಬೇಕಿತ್ತು – ಒಳ್ಳೆಯ ಆಲೋಚನೆ, ಒಳ್ಳೆಯ ಮಾತು ಮತ್ತು ಒಳ್ಳೆಯ ಕರ್ಮ – ಇವಿಷ್ಟೇ. ಇದೊಂದು ಅನುಷ್ಟಾನಿಕವಾದ, ವಿವೇಕಸಮ್ಮತವಾದ ಧರ್ಮ. ಇದರಲ್ಲಿ ಆಗಲೇ ಸ್ವಲ್ಪಮಟ್ಟಿಗೆ ಕಾವ್ಯದ ಲೇಪನ ಬಂದಿತ್ತು. ಆದರೆ ಇದಕ್ಕಿಂತಲೂ ಹೆಚ್ಚಿನದಾದ ಕಾವ್ಯ ಮತ್ತು ಹೆಚ್ಚಿನ ಚಿಂತನೆಗಳು ಇವೆ.

ಭಾರತದಲ್ಲಿ ವೇದಗಳ ಅತ್ಯಂತ ಪ್ರಾಚೀನತಮವಾದ ಭಾಗದಲ್ಲಿ ಈ ಸೈತಾನನನ್ನು ಕಾಣುತ್ತೇವೆ. ಅವನು ಇದ್ದಕ್ಕಿದ್ದಂತೆ ಕಾಣಿಸಿಕೊಂಡು ತಕ್ಷಣವೇ ಅದೃಶ್ಯನಾಗುತ್ತಾನೆ... ವೇದಗಳಲ್ಲಿ ಈ ದುರಾತ್ಮ (ಕೆಟ್ಟ ದೈತ್ಯ, ಸೈತಾನ) ವಜ್ರಾಘಾತದಿಂದ ಪಲಾಯನ ಮಾಡುತ್ತಾನೆ. ಓಡಿಹೋದ ಅವನನ್ನು ಪರ್ಷಿಯನ್ನರು ಕರೆದುಕೊಂಡರು. ನಾವು ಅವನನ್ನು ಈ ಪ್ರಪಂಚವನ್ನೇ ಬಿಟ್ಟು ಹೋಗುವಂತೆ ಮಾಡಲು ಪ್ರಯತ್ನಿಸುತ್ತೇವೆ. ಪರ್ಷಿಯನ್ನರ ಭಾವನೆಯನ್ನು ತೆಗೆದುಕೊಂಡು, ಅವನನ್ನು (ಸೈತಾನನನ್ನು) ಒಂದು ಸಂಭಾವಿತ ವ್ಯಕ್ತಿಯನ್ನಾಗಿ ಮಾಡಲು ಹೋಗುತ್ತೇವೆ; ಅವನಿಗೊಂದು ಹೊಸ ಶರೀರ (ಆಕಾರ) ಕೊಡುತ್ತೇವೆ. ಭಾರತದಲ್ಲಿ ಸೈತಾನನ ವಿಚಾ (ಅಲ್ಲಿಗೆ ವೇದಗಳ ಪ್ರಾಚೀನತಮ ಭಾಗಕ್ಕೆ) ಮುಗಿಯಿತು.

ಆದರೆ ದೇವರ (ಈಶ್ವರನ) ಬಗೆಗಿನ ಚಿಂತನೆ ಸಾಗುತ್ತಲೇ ಇತ್ತು. ಇಲ್ಲಿ ಮತ್ತೊಂದು ಅಂಶವನ್ನು ಲಕ್ಷ್ಯದಲ್ಲಿಡಿ. ನೀವು (ಚಿಂತನೆಯ ಮೂಲವನ್ನು) ಶೋಧಿಸುತ್ತಾ ಹೋದಂತೆ, ಪರ್ಷಿಯನ್ ಚಕ್ರವರ್ತಿಯವರೆವಿಗೂ ದೇವರ ಬಗೆಗಿನ ಚಿಂತನೆಯು ಭೋಗ\break ಪರಾಯಣತೆಯ ವಿಚಾರದೊಂದಿಗೇ ಜೊತೆ ಜೊತೆಯಾಗಿ ಬೆಳೆಯಿತು. ಆದರೆ\break (ಇನ್ನೊಂದೆಡೆ) ಅದರೊಡನೆಯೇ ತತ್ತ್ವಜ್ಞಾನ ಅಥವಾ ದಾರ್ಶನಿಕತೆಯೂ ಬರುತ್ತದೆ. ಅಲ್ಲಿ ಮತ್ತೊಂದು ವಿಚಾರಧಾರೆಯಿದೆ – ಅದೇ ತನ್ನದೇ ಆತ್ಮಭಾವನೆ – (ಮನುಷ್ಯನ, ಅದ್ವಿತೀಯವಾದ ಆತ್ಮ). ಅದೂ ಸಹ ಬೆಳೆಯುತ್ತದೆ. ಆದ್ದರಿಂದ ಭಾರತದ ಹೊರಗಡೆ ದೇವರ ಸಂಬಂಧವಾದ ಕಲ್ಪನೆ ಈ ಸ್ಥೂಲರೂಪದಲ್ಲಿಯೇ ಇರಬೇಕಾಯಿತು – ಭಾರತದಿಂದ ಈ ದಿಶೆಯಲ್ಲಿ ಸ್ವಲ್ಪ ಪ್ರೇರಣೆ ಸಿಗುವವರೆಗೂ ಅದು ಬೆಳೆಯಲು ಸಾಧ್ಯವಾಗಲಿಲ್ಲ. ಬೇರೆ ದೇಶಗಳು ದೇವರ ಬಗೆಗಿನ ಚಿಂತನೆಯನ್ನು ಆ ಹಳೆಯ ಸ್ಥೂಲ ಭಾವನೆಯೊಂದಿಗೇ ನಿಲ್ಲಿಸಿದುವು. ಈಗಲೂ ಈ (ಅಮೆರಿಕಾ) ದೇಶದಲ್ಲಿ ದೇವರಿಗೂ ದೇಹವಿದೆ ಎಂದು ನಂಬುವ ಲಕ್ಷಾಂತರ ಜನರಿದ್ದಾರೆ. ಇಡೀ ಮತ – ಪಂಗಡಗಳ ಮಂದೆಯೇ ಅವನ್ನು ಘೋಷಿಸುತ್ತವೆ. ಅವನು ಈ ಪ್ರಪಂಚವನ್ನು ಆಳುತ್ತಾನೆಂದೂ ಅವನಿಗೆ ದೇಹವಿರುವ ಒಂದು ಸ್ಥಾನವಿದೆಯೆಂದೂ ಅವರು ನಂಬುತ್ತಾರೆ, ಅವನು ಸಿಂಹಾಸನದ ಮೇಲೆ ಕುಳಿತಿರುತ್ತಾನೆ. ನಮ್ಮ ದೇವಸ್ಥಾನಗಳಲ್ಲಿ ಮಾಡುವ ಹಾಗೆ (ಅವನ ಮುಂದೆ) ಮೇಣದ ದೀಪಗಳನ್ನು ಹಚ್ಚಿಟ್ಟು ಹಾಡುಗಳನ್ನು ಹಾಡುತ್ತಾರೆ.

ಆದರೆ ಭಾರತೀಯರು (ಈ ರೀತಿ ತಮ್ಮ ದೇವರನ್ನು ಎಂದಿಗೂ ಒಂದು ಭೌತಿಕ ವ್ಯಕ್ತಿಯಾಗಿ) ಮಾಡದಿರುವಷ್ಟು ಜಾಣ್ಮೆಯುಳ್ಳವರಾಗಿದ್ದರು. ಬ್ರಹ್ಮನಿಗೋಸ್ಕರ ಭಾರತದಲ್ಲಿ ಯಾವುದೇ ದೇವಸ್ಥಾನವಿರುವುದನ್ನು ನೀವು ಎಂದಿಗೂ ನೋಡಲಾರಿರಿ. ಏಕೆ? (ದೇಹಾತೀತವಾದ) ಆತ್ಮಭಾವನೆ ಅಲ್ಲಿ ಯಾವತ್ತೂ ಜೀವಂತವಾಗಿತ್ತು. ಯಹೂದಿ ಜನಾಂಗ ಆತ್ಮದ (ಅಸ್ತಿತ್ವವನ್ನು) ಕುರಿತು ಎಂದಿಗೂ ಜಿಜ್ಞಾಸೆ ಮಾಡಲಿಲ್ಲ. ಹಳೆಯ ಒಡಂಬಡಿಕೆಯಲ್ಲಂತೂ ಆತ್ಮ ಭಾವನೆಯ ಉಲ್ಲೇಖವೇ ಇಲ್ಲ. ಮೊಟ್ಟ ಮೊದಲ (ಉಲ್ಲೇಖ) ಹೊಸ ಒಡಂಬಡಿಕೆಯಲ್ಲಿದೆ. ಪರ್ಷಿಯನ್ನರು, ಅವರೆಷ್ಟು ವ್ಯವಹಾರ ನಿಪುಣರೆಂದರೆ – ಯಾರನ್ನಾದರೂ ಬೆರಗುಗೊಳಿಸುವ ವ್ಯವಹಾರಚತುರತೆಯುಳ್ಳ ಅವರು – ಹೋರಾಟಗಳನ್ನು ನಡೆಸಿ ಇತರರನ್ನು ಗೆಲ್ಲುವ ಜನಾಂಗದವರಾಗಿದ್ದರು. ಅವರು ಹಳೆಯ ಯುಗದ ಬ್ರಿಟಿಷರೇ ಸರಿ. ಯಾವತ್ತೂ ಅವರ ನೆರೆಹೊರೆಯವರೊಡನೆ ಕಾದಾಡಿ ಅವರನ್ನು ನಾಶಗೊಳಿಸುವುದರಲ್ಲೇ ನಿರತರಾಗಿದ್ದ ಅವರಿಗೆ ಆತ್ಮಭಾವನೆ ಕುರಿತು ಚಿಂತಿಸುವ ವ್ಯವಧಾನವೇ ಇರಲಿಲ್ಲ.

ಆತ್ಮದ ಬಗೆಗಿನ ಅತ್ಯಂತ ಪ್ರಾಚೀನವಾದ ಭಾವನೆಯೆಂದರೆ – ಅದು ಈ ಸ್ಥೂಲ ಶರೀರದೊಳಗಿನ ಒಂದು ಸೂಕ್ಷ್ಮ ಶರೀರವೇ ಎಂಬುದಾಗಿತ್ತು. ಸ್ಥೂಲವಾದದ್ದು ಕಣ್ಮರೆಯಾದ ನಂತರ ಸೂಕ್ಷ್ಮವಾದದ್ದು ಗೋಚರವಾಗುತ್ತದೆ. ಈಜಿಪ್ಟ್ ದೇಶದಲ್ಲಿ ಈ ಸೂಕ್ಷ್ಮವಾದದ್ದೂ ಸಾಯುತ್ತದೆ ಮತ್ತು ಈ ಸ್ಥೂಲ ಶರೀರ ಶಿಥಿಲವಾದೊಡನೆಯೇ ಸೂಕ್ಷ್ಮವಾದದ್ದೂ ಶಿಥಿಲಗೊಳ್ಳುತ್ತದೆ. ಈ ಕಾರಣದಿಂದಲೇ ಅವರು ಪಿರಮಿಡ್ಡುಗಳನ್ನು\break ನಿರ್ಮಾಣಮಾಡುತ್ತಿದ್ದುದು. (ಮತ್ತು ಅದರೊಳಗಿನ ತಮ್ಮ ಪೂರ್ವಜರ ಮೃತ ಶರೀರಕ್ಕೆ ಸುಗಂಧ ದ್ರವ್ಯಗಳನ್ನು ಹಾಕಿ ಅವು ಕೆಡದಂತೆ ಕಾಪಾಡಿ ತನ್ಮೂಲಕ ಮೃತಾತ್ಮರಿಗೆ ಅಮರತ್ವವನ್ನು ಪಡೆಯಲು ಸಹಾಯ ಮಾಡುತ್ತಿದ್ದರು.

ಭಾರತೀಯರಿಗೆ ಮೃತಶರೀರದ ಮೇಲೆ ಎಳ್ಳಷ್ಟೂ ವಾಂಛಲ್ಯವಿಲ್ಲ. (ಅವರ ಮನೋಭಾವ) “ಇದನ್ನು ತೆಗೆದುಕೊಂಡು ಹೋಗಿ ಸುಟ್ಟುಬಿಡೋಣ'' ಎಂಬುದು ಆಗಿದೆ. ತಂದೆಯ ಮೃತಶರೀರಕ್ಕೆ ಮಗನೇ ಅಗ್ನಿಸ್ಪರ್ಶಮಾಡಬೇಕು.

ಎರಡು ಬಗೆಯ (ವಿಪರೀತ) ಜಾತಿಗಳಿವೆ – ಒಂದು ದೈವೀಸಂಪತ್ತುಳ್ಳದ್ದು ಮತ್ತೊಂದು ಆಸುರೀ ಸಂಪತ್ತುಳ್ಳದ್ದು. ದೈವೀಸಂಪದರು ತಾವು ಆತ್ಮ ಹಾಗೂ ಚೈತನ್ಯ (ಸ್ವರೂಪ)ರೆಂದು ಭಾವಿಸುತ್ತಾರೆ. ಆಸುರೀ ಸಂಪದರು ತಾವು ದೇಹ ಮಾತ್ರವೆಂದು ಭಾವಿಸುತ್ತಾರೆ. ಭಾರತದ ಪ್ರಾಚೀನ ದಾರ್ಶನಿಕರು ದೇಹ ನಶ್ವರವೆಂದು ಒತ್ತಿ ಒತ್ತಿ ಹೇಳಿದರು. `ಹೇಗೆ ವ್ಯಕ್ತಿ ಹಳೆಯ ಬಟ್ಟೆಯನ್ನು ಬಿಟ್ಟು ಹೊಸದನ್ನು ತೊಟ್ಟುಕೊಳ್ಳುವನೋ ಅದೇರೀತಿ ಹಳೇ ಶರೀರವನ್ನು ಬಿಟ್ಟು ಬೇರೆ ಹೊಸ ಶರೀರವನ್ನು ಧರಿಸುತ್ತಾನೆ' (ಗೀತಾ, \enginline{II, 22),} ನನ್ನ ಮಟ್ಟಿಗಂತೂ (ನನಗೆ ದೊರಕಿದ ) ಎಲ್ಲ ವಾತಾವರಣ ಮತ್ತು ಶಿಕ್ಷಣ ವಿಪರೀತವಾಗಿಯೇ ಇತ್ತು. ಅದು ನನ್ನನ್ನು ಬೇರೆ ರೀತಿಯಲ್ಲೇ ರೂಪಿಸಲು ಪ್ರಯತ್ನಿಸುತ್ತಿತ್ತು. ಯಾರಿಗೆ ದೇಹದ ಬಗ್ಗೆಯೇ ಹೆಚ್ಚು ಕಾಳಜಿಯಿತ್ತೋ ಅಂತಹ ಮುಸಲ್ಮಾನ ಮತ್ತು ಕ್ರಿಶ್ಚಿಯನ್ನರ ಸಂಪರ್ಕದಲ್ಲಿಯೇ ನಾನು ಯಾವಾಗಲೂ ಇರುತ್ತಿದ್ದೆ.

ದೇಹದಿಂದ ಒಂದು ಹೆಜ್ಜೆ ಮುಂದಕ್ಕೆ ಆತ್ಮಭಾವನೆ ಬರುತ್ತದೆ. ಭಾರತದಲ್ಲಿ ಈ ಆತ್ಮದ ಆದರ್ಶದ ಮೇಲೆ ಬಹಳ ಒತ್ತನ್ನು ಕೊಟ್ಟಿದ್ದರು. ನಮ್ಮವರಿಗೆ ಈ ಆದರ್ಶ ದೇವರ ಬಗೆಗಿನ ಕಲ್ಪನೆಯ ಪರ್ಯಾಯವೇ ಆಗಿತ್ತು... (ವ್ಯಕ್ತಿಯ) ಆತ್ಮದ ಬಗೆಗಿನ\break ಭಾವನೆ ವಿಕಾಸವಾದಂತೆ (ಅವನು ಇದು ನಾಮ – ರೂಪಗಳಿಗೆ ಅತೀತವಾದದ್ದು ಎಂಬ ತೀರ್ಮಾನಕ್ಕೆ ಬರಲೇಬೇಕು.) ಭಾರತೀಯ ಚಿಂತನೆಯಲ್ಲಿ ಆತ್ಮ ರೂಪರಹಿತವಾದದ್ದು. ಸಾಕಾರವಾದುದೆಲ್ಲ ಒಂದಲ್ಲ ಒಂದು ದಿನ ಶಿಥಿಲವಾಗಲೇ ಬೇಕು. ಜಡದ್ರವ್ಯ ಮತ್ತು ಶಕ್ತಿಯ ಸಂಘಾತವಿಲ್ಲದ ಯಾವುದಕ್ಕೂ ಆಕಾರವಿರಲಾರದು; ಎಲ್ಲ ಸಂಘಾತ, ಸಂಯೋಜನೆಗಳೂ ವಿಭಜನೆಯಾಗಲೇಬೇಕು. ಹೀಗಿರುವಾಗ ನಿಮ್ಮ ಆತ್ಮವೇನಾದರೂ ನಾಮರೂಪಾತ್ಮಕವಾಗಿದ್ದರೆ, ಅದು ವಿಭಜನೆಯಾಗಿ ಶಿಥಿಲಗೊಳ್ಳುತ್ತದೆ ಮತ್ತು ನೀವು ಸಾವನ್ನಪ್ಪಿದಾಗ ಶಾಶ್ವತವಾಗಿರುವುದೇ ಇಲ್ಲ. ಆತ್ಮವೇನಾದರೂ (ಸ್ಥೂಲ – ಸೂಕ್ಷ್ಮ) ಎರಡನ್ನೂ ಕೂಡಿದ್ದಾಗಿದ್ದು, ಅದು ಪ್ರಕೃತಿಗೆ ಸೇರಿದ್ದರೆ, ಅದು ಜನನಮರಣವೆಂಬ ಪ್ರಕೃತಿಯ ನಿಯಮಕ್ಕೆ ಅಧೀನವಾಗಿರುತ್ತದೆ... ಭಾರತೀಯ ಋಷಿಗಳು, ಆತ್ಮವು ಮನಸ್ಸಲ್ಲ ಎಂಬುದನ್ನು ತಿಳಿದುಕೊಂಡರು.

ಆಲೋಚನೆಗಳನ್ನು ನಿಯಂತ್ರಿಸಬಹುದು ಹಾಗೂ ನಿರ್ದೆಶಿಸಬಹುದು.\break (ಭಾರತೀಯ ಯೋಗಿಗಳು) ಆಲೋಚನೆಗಳನ್ನು ಎಷ್ಟರಮಟ್ಟಿಗೆ ನಿಯಂತ್ರಿಸಬಹುದು ಹಾಗೂ ನಿರ್ದೆಶಿಸಬಹುದು ಎಂಬುದನ್ನು ಸಾಧನೆಯಿಂದ ತಿಳಿದರು. ಕಠೋರವಾದ ತಪಸ್ಸಿನ ಬಲದಿಂದ ಆಲೋಚನಾಗತಿಯನ್ನು ಸಂಪೂರ್ಣವಾಗಿ ನಿಲ್ಲಿಸಬಹುದು.\break ಮನಸ್ಸು ಅಥವಾ ಆಲೋಚನೆಗಳೇ ಮನಷ್ಯನ ನೈಜಸ್ವರೂಪವಾಗಿದ್ದಲ್ಲಿ ಅವೆಲ್ಲ ನಿಂತೊಡನೆಯೇ ಅವನು ಸಾಯಬೇಕಾಗಿತ್ತು. ಧ್ಯಾನದಲ್ಲಿ ಎಲ್ಲ ಆಲೋಚನೆಯೂ ಲುಪ್ತವಾಗುತ್ತದೆ, ಮನಸ್ಸಿನ ಉಪಾದಾನಗಳೂ ಸಹ ಎಲ್ಲ ರೀತಿಯಲ್ಲೂ ಸ್ಥಿರವಾಗುತ್ತವೆ; ರಕ್ತಸಂಚಾರ ನಿಲ್ಲುತ್ತದೆ. ಶ್ವಾಸ – ಪ್ರಶ್ವಾಸ ಕ್ರಿಯೆಯೂ ಸ್ತಬ್ದವಾಗಿ ಹೋಗುತ್ತದೆ. ಆದರೂ ಅವನು ಸತ್ತಿರುವುದಿಲ್ಲ. ಮನುಷ್ಯನ ಜೀವ ಬರೀ ಚಿಂತಾ ಸಮೂಹದ ಮೇಲೆಯೇ ನಿಂತಿದ್ದರೆ ಇಂತಹ ಅವಸ್ಥೆಯಲ್ಲಿ ದೇಹ – ಮನಸ್ಸುಗಳ ಸಂಘಾತವೆಲ್ಲವೂ ನಾಶವಾಗಬೇಕಾದ್ದು ಸ್ವಾಭಾವಿಕ. ಆದರೆ ಭಾರತೀಯ ಋಷಿಗಳು ಕಂಡುಕೊಂಡಂತೆ ಅದು ಹಾಗಾಗುವುದಿಲ್ಲ. ಇದು ಪ್ರಯೋಗದಿಂದ (ಪ್ರಮಾಣ) ಸಿದ್ಧವಾದದ್ದು. ಆದ್ದರಿಂದ ಅವರು ಮನಸ್ಸಾಗಲೀ ಅಥವಾ ಮನಸ್ಸಿನ ಚಿಂತಾರಾಶಿಯಾಗಲೀ ಮನುಷ್ಯನ ನೈಜಸ್ವರೂಪವಲ್ಲ ಎಂಬ ಸಿದ್ಧಾಂತಕ್ಕೆ ಬಂದರು. ನಂತರ ವಿಚಾರವೂ ಮನಸ್ಸು ಎಂದಿಗೂ ಮನುಷ್ಯನ ಆತ್ಮವಾಗಲಾರದು ಎಂಬುದನ್ನು ಪುಷ್ಟಿಗೊಳಿಸಿತು.

ನಾನು ಇಲ್ಲಿ ಬಂದೆ, ಯೋಚಿಸಿದೆ, ಮಾತನಾಡಿದೆ ಈ ಎಲ್ಲ ಕ್ರಿಯೆಗಳಲ್ಲೂ (ಆತ್ಮದ) ಏಕತೆಯ ಒಂದು ಸೂತ್ರ ಅಡಗಿದೆ. ನನ್ನ ವಿಚಾರಗಳು ಹಾಗೂ ಕರ್ಮರಾಶಿ ವಿಚಿತ್ರವೂ ವಿವಿಧವೂ ಆಗಿವೆ. ಆದರೆ ಅವುಗಳಲ್ಲಿ, ಅವುಗಳ ಮೂಲಕ, ಅಪರಿವರ್ತನೀಯವಾದ, ಅಖಂಡವಾದ ಸತ್ ಪ್ರಕಾಶವಾಗುತ್ತಿದೆ. ಈ ಸತ್ ಎಂದಿಗೂ ಶರೀರವಾಗಿರಲಾರದು. ಶರೀರವಾದರೋ ಪ್ರತಿನಿಮಿಷವೂ ಬದಲಾಗುತ್ತಿದೆ. ಅದು ಮನಸ್ಸೂ ಸಹ ಆಗಿರಲಾರದು. ಏಕೆಂದರೆ ಅದರಲ್ಲಿ ಸರ್ವದಾ ಹೊಸ ಹೊಸ ಯೋಚನೆಗಳು ಬರುತ್ತಲೇ ಇರುತ್ತವೆ. ಆ ಏಕತ್ವದ ಭಿತ್ತಿ ಶರೀರವೂ ಅಲ್ಲ, ಮನಸ್ಸೂ ಅಲ್ಲ. ಶರೀರ ಮತ್ತು ಮನಸ್ಸುಗಳ ಸಂಘಾತವೆಂದೂ ಹೇಳಲಾಗುವುದಿಲ್ಲ. ಶರೀರ ಮತ್ತು ಮನಸ್ಸುಗಳು ಪ್ರಕೃತಿಗೆ ಸೇರಿದವು ಮತ್ತು ಅವು ಪ್ರಕೃತಿನಿಯಮಗಳಿಗೆ ಅಧೀನವಾಗಿರಲೇಬೇಕು. ಮನಸ್ಸೇನಾದರೂ ಮುಕ್ತವಾದರೆ... ಮುಕ್ತವಾದ ಮನಸ್ಸು (ಎಂದೂ ನಿಯಮಾಧೀನವಾಗಿರಲಾರದು.)

ಆದ್ದರಿಂದ ಮನುಷ್ಯ ಅವನ ನಿಜಸ್ವರೂಪದಲ್ಲಿ ಪ್ರಕೃತಿಗೆ ಸೇರಿದವನಂತೂ ಅಲ್ಲ. ಅವನು ಅಪರಿವರ್ತನೀಯವಾದ ಚೈತನ್ಯಮಯ ಪುರುಷ. ಆದರೆ ಅವನ ದೇಹ ಮತ್ತು ಮನಸ್ಸುಗಳು ಅವಶ್ಯವಾಗಿ ಪ್ರಕೃತಿಗೆ ಅಧೀನವಾಗಿವೆ. ನಾವು ಅಷ್ಟರಮಟ್ಟಿಗೆ ಪ್ರಕೃತಿಯೊಂದಿಗೆ ವ್ಯವಹರಿಸುತ್ತಿದ್ದೇವೆ. ಹೇಗೆ ನೀವು ಪೆನ್ನು, ಶಾಯಿ, ಕುರ್ಚಿ ಮುಂತಾದುವನ್ನು ಉಪಯೋಗಿಸುತ್ತೀರೋ ಅದೇರೀತಿ ಆ ಪುರುಷನೂ ಸ್ಥೂಲ ಮತ್ತು ಸೂಕ್ಷ್ಮರೂಪದಲ್ಲಿ ಪ್ರಕೃತಿಯನ್ನು ಉಪಯೋಗಿಸುತ್ತಾನೆ. ಸ್ಥೂಲರೂಪದ ದೇಹವನ್ನೂ ಸೂಕ್ಷ್ಮರೂಪದ ಮನಸ್ಸನ್ನೂ ಅವನು ಉಪಯೋಗಿಸುತ್ತಾನೆ. ಪುರುಷನು ನಿರವಯವ, ಸರಳನಾಗಿದ್ದಲ್ಲಿ, ಅವನು ಎಲ್ಲ ಪ್ರಕಾರಗಳಿಂದಲೂ ಆಕೃತಿಹೀನ, ನಿರಾಕಾರನಾಗಿರಬೇಕು. ಆಕಾರ ಸಮೂಹವೆಲ್ಲ ಪ್ರಕೃತಿಯಲ್ಲಿ ಮಾತ್ರವೇ. ಯಾವುದು ಪ್ರಕೃತಿಯಲ್ಲಿಲ್ಲವೊ ಪ್ರಕೃತಿಗೆ ಪಾರವೊ ಅದಕ್ಕೆ ಸ್ಥೂಲವಾಗಲೀ, ಸೂಕ್ಷವಾಗಲೀ ಯಾವುದೇ ಆಕಾರವೂ ಇರಲಾಗುವುದಿಲ್ಲ. ಅದು ನಿಶ್ಚಯವಾಗಿಯೂ ರೂಪರಹಿತವಾಗಿರಬೇಕು. ಅದು ಸರ್ವವ್ಯಾಪಿಯೂ ಆಗಿರಬೇಕು. ಇದನ್ನು ನಾವು ತಿಳಿದುಕೊಳ್ಳಬೇಕು. ಈ ಮೇಜಿನ ಮೇಲಿನ ಗ್ಲಾಸನ್ನೇ ತೆಗೆದುಕೊಂಡರೆ, ಅದಕ್ಕೆ ಒಂದು ಆಕಾರವಿದೆ, ಮೇಜಿಗೂ ಒಂದು ಆಕಾರವಿದೆ. ಅವುಗಳು ಒಡೆದು ಮುರಿದು ಹೋದಾಗ, ಅಷ್ಟರಮಟ್ಟಿಗೆ ತಮ್ಮತನವನ್ನು ಕಳೆದುಕೊಳ್ಳುತ್ತವೆ.

ಆತ್ಮಕ್ಕೆ ಯಾವ ರೂಪವೂ ಇಲ್ಲದಿರುವುದರಿಂದ ಅದು ನಾಮರಹಿತವಾಗಿದೆ. ಅದು ಹೇಗೆ ಈ ಗ್ಲಾಸಿನೊಳಕ್ಕೆ ಹೋಗಲಾರದೋ ಅದರಂತೆಯೇ ಅದು ಸ್ವರ್ಗಕ್ಕೂ ಹೋಗಲಾರದು; ನರಕಕ್ಕೂ ಹೋಗಲಾರದು. ಯಾವುದರಲ್ಲಿ ಅದು ವಿದ್ಯಮಾನವೋ ಅದರ ರೂಪವನ್ನು ಅದು ಧರಿಸುತ್ತದೆ. ಆತ್ಮವು, ದೇಶದಲ್ಲಿಲ್ಲದಿದ್ದರೆ ಈ ಎರಡರಲ್ಲಿ ಒಂದು ಮಾತ್ರ ಸಾಧ್ಯ: ಒಂದು ಆತ್ಮವು ದೇಶವನ್ನು ವ್ಯಾಪಿಸುತ್ತದೆ ಅಥವಾ ದೇಶವೇ ಅದರಲ್ಲಿ ಅಂತಸ್ಥವಾಗಿದೆ. ನಿಮ್ಮ ದೇಹ ದೇಶದಲ್ಲಿರುವುದರಿಂದ ನಿಮಗೊಂದು ಆಕಾರವಿರಲೇಬೇಕು. ದೇಶ, ನಮ್ಮನ್ನು ಸೀಮಿತಗೊಳಿಸುತ್ತದೆ, ನಮ್ಮನ್ನು ಬಂಧಿಸುತ್ತದೆ. ನಮ್ಮನ್ನು ಒಂದು ಆಕಾರದಲ್ಲಿರುವಂತೆ ಮಾಡುತ್ತದೆ. ನೀವು ದೇಶದಲ್ಲಿಲ್ಲದಿದ್ದರೆ, ದೇಶವೇ ನಿಮ್ಮಲ್ಲಿರುತ್ತದೆ. ಸ್ವರ್ಗಲೋಕ, ಇಹಲೋಕ ಎಲ್ಲವೂ ಸಚೇತನವಾದ ಪುರುಷನಲ್ಲಿಯೇ ಇದೆ.

ದೇವರ ಸಂಬಂಧವಾಗಿಯೂ ಹೀಗೆಯೇ ಇರಬೇಕಾದ್ದು ಸರಿ. ದೇವರು\break ಸರ್ವವ್ಯಾಪಿ, “ಹಸ್ತಗಳಿಲ್ಲದಿದ್ದರೂ ಅವನು ಸಮಸ್ತವನ್ನೂ ಹಿಡಿದಿಟ್ಟಿರುತ್ತಾನೆ.\break ಪಾದಗಳಿಲ್ಲದಿದ್ದರೂ ಅವನು ಸರ್ವತ್ರ ಸಂಚರಿಸಬಲ್ಲನು...'' ಅವನು ನಿರಾಕಾರ, ಮೃತ್ಯುಹೀನ, ಶಾಶ್ವತ, ಕಾಲಕ್ರಮದಲ್ಲಿ ದೇವರ ಸಂಬಂಧವಾಗಿ ಇಂತಹ ವಿಚಾರಗಳು ಮೊಳಕೆಯೊಡೆದುವು. ನನ್ನ ಆತ್ಮವು ನನ್ನ ಈ ದೇಹದ ಅಧಿಪತಿಯಾಗಿರುವಂತೆ ಅವನು ಎಲ್ಲ ಆತ್ಮಗಳ ಅಧೀಶ್ವರ. ನನ್ನಾತ್ಮವೇನಾದರೂ ಈ ದೇಹವನ್ನು ಬಿಟ್ಟು ಹೋದದ್ದೇ ಆದರೆ ಈ ದೇಹ ಒಂದು ಕ್ಷಣವೂ ಬದುಕಿರಲಾರದು. ಅದೇ ರೀತಿ ಪರಮಾತ್ಮನೇನಾದರೂ ನನ್ನಾತ್ಮದಿಂದ ಬೇರ್ಪಟ್ಟಿದ್ದೇ ಆದರೆ ಆತ್ಮದ ಅಸ್ತಿತ್ವವೇ ಸಾಧ್ಯವಿಲ್ಲ. ಅವನು ವಿಶ್ವ ಭುವನದ ಸ್ರಷ್ಟಾ; ಮತ್ತೆ ನಾಶವಾಗುವ ಪ್ರತಿಯೊಂದರ ಸಂಹಾರಕನೂ ಅವನೇ. ಜೀವನವೂ ಅವನ ನೆರಳು; ಮೃತ್ಯುವೂ ಅವನ ನೆರಳೇ.

ಪ್ರಾಚೀನ ಭಾರತದ ದಾರ್ಶನಿಕರ ಭಾವನೆಯಲ್ಲಿ, ಈ ಕಲುಷಿತವಾದ, ಜಿಗುಪ್ಸೆ ಹುಟ್ಟಿಸುವ ಸಂಸಾರ ಎಂದಿಗೂ ಮನುಷ್ಯನ ಅವಗಾಹನೆಗೆ ಯೋಗ್ಯವಾದುದಲ್ಲ. ಈ ವಿಶ್ವದಲ್ಲಿ – ಒಳ್ಳೆಯದೇ ಆಗಲಿ ಕೆಟ್ಟದ್ದೇ ಆಗಲಿ – ಯಾವುದೂ ಚಿರ ಸ್ಥಾಯಿಯಲ್ಲ, ಶಾಶ್ವತವಲ್ಲ.

ನಾನು ಹಿಂದೆಯೇ ಹೇಳಿದಂತೆ ಭಾರತದಲ್ಲಿ ಸೈತಾನನಿಗೆ ಹೆಚ್ಚು ಅವಕಾಶ ಸಿಗಲಿಲ್ಲ. ಕಾರಣವೇನು? ಏಕೆಂದರೆ ಧರ್ಮದ ಚಿಂತನೆಯಲ್ಲಿ ಭಾರತೀಯ ಋಷಿಗಳು ಬಹಳ ಸಾಹಸಿಗಳಾಗಿದ್ದರು. ಅವರು ಧರ್ಮದ ಕ್ಷೇತ್ರದಲ್ಲಿ (ಬೇಜವಾಬ್ದಾರಿ) ಶಿಶುಗಳಂತೆ ನಡೆದುಕೊಳ್ಳುತ್ತಿರಲಿಲ್ಲ. ಮಕ್ಕಳ ವಿಶೇಷವಾದ ಲಕ್ಷಣವನ್ನು ನೀವು ಗಮನಿಸಿದ್ದೀರೇನು? ಅವರು ಯಾವಾಗಲೂ ದೋಷವನ್ನು ಇನ್ನೊಬ್ಬರ ಮೇಲೆ ಹಾಕಲು ಪ್ರಯತ್ನಿಸುತ್ತಿರುತ್ತಾರೆ. ಬೆಳವಣಿಗೆಯಾಗದ ಮನಸ್ಸಿನವರೂ ತಾವು ತಪ್ಪನ್ನು ಮಾಡಿದಾಗ ಮತ್ತೊಬ್ಬರ ಮೇಲೆ ಆ ತಪ್ಪನ್ನು ಹೊರೆಸುವ ಪ್ರಯತ್ನದಲ್ಲಿರುತ್ತಾರೆ. ಒಂದುಕಡೆ ದೇವರನ್ನು ಈ ರೀತಿ ಬೇಡುತ್ತಿರುತ್ತೇವೆ – ನನಗಿದನ್ನು ಕೊಡು, ನನಗದನ್ನು ಪಾಲಿಸು ಎಂದು. ಮತ್ತೊಂದು ಕಡೆ ಹೇಳುತ್ತಿರುತ್ತೇವೆ: “ನಾನಿದನ್ನು ಮಾಡಲಿಲ್ಲ, ಸೈತಾನ ನನ್ನನ್ನು ಪ್ರಲೋಭನೆ ಮಾಡುತ್ತಿದ್ದ. ಅವನೇ ಇದೆಲ್ಲವನ್ನೂ ಮಾಡಿದವನು" ಎಂದು. ಇದೇ ಮನುಷ್ಯನ ಇತಿಹಾಸ – ದುರ್ಬಲ ಮಾನವ ಜಾತಿಯ ಇತಿವೃತ್ತಾಂತ.

ಪಾಪ, ಅನಿಷ್ಟ ಏತಕ್ಕಿದೆ? ಈ ಸಂಸಾರ ಏಕೆ ಒಂದು ಗಬ್ಬುನಾತದ ಹಲಸಿನ ಹೊಂಡದಂತೆ ಇದೆ? ನಾವೇ ಅದನ್ನು ಮಾಡಿದವರು. ಬೇರಾರನ್ನೂ ಇದಕ್ಕಾಗಿ ದೂರಲಾಗುವುದಿಲ್ಲ. ನಾವೇ ಬೆಂಕಿಯಲ್ಲಿ ಕೈಯನ್ನಿಟ್ಟವರು. ದೇವರು ನಮ್ಮನ್ನು ಆಶೀರ್ವದಿಸಲಿ. (ಮನುಷ್ಯ) ಅವನವನ ಯೋಗ್ಯತೆಗೆ ತಕ್ಕದಾದುದನ್ನೇ ಪಡೆಯುತ್ತಾನೆ. ದೇವರಂತೂ ಅನುಕಂಪ, ಕರುಣೆಗಳ ಸಾಕಾರಮೂರ್ತಿ. ಅವನಲ್ಲಿ ನಾವು ಪ್ರಾರ್ಥಿಸಿದರೆ ನಮ್ಮ ಪ್ರಾರ್ಥನೆಯನ್ನು ಕೇಳಿ ಸಹಾಯ ಮಾಡುತ್ತಾನೆ. – ಅವನು ತನ್ನನ್ನೇ ನಮಗಾಗಿ ಅರ್ಪಿಸಿಕೊಳ್ಳುತ್ತಾನೆ.

ಇದು ಭಾರತೀಯರ ದೃಷ್ಟಿಕೋನ. ಅವರು ಸ್ವಭಾವತಃ ಕಾವ್ಯದ – ಸೌಂದರ್ಯದ ಉಪಾಸಕರು. ಕಾವ್ಯವೆಂದರೆ ಅವರಿಗೆ ಎಲ್ಲಿಲ್ಲದ ಹುಚ್ಚು. ಅವರ ದರ್ಶನವೂ ಕಾವ್ಯವೇ. (ನಾನು ಹೇಳುತ್ತಿರುವ) ಈ ದರ್ಶನ ಒಂದು ದೊಡ್ಡ ಕವಿತೆಯೇ, ಸಂಸ್ಕೃತ ಭಾಷೆಯಲ್ಲಿರುವ ಸಮಸ್ತ (ಉನ್ನತ ವಿಚಾರಗಳೂ) ಕವನಗಳಲ್ಲಿಯೇ ಬರೆಯಲ್ಪಟ್ಟಿವೆ – ತತ್ತ್ವವಿದ್ಯೆ, ಜೋತಿರ್ವಿದ್ಯೆ – ಎಲ್ಲವೂ ಛಂದೋಬದ್ಧವಾದ ಕವನಗಳೇ.

ನಮ್ಮ ಈ ಅವಸ್ಥೆಗೆ ನಾವೇ ಹೊಣೆಗಾರರು. ನಾವು ಇಂತಹ ದುರವಸ್ಥೆಗೆ ಬಂದದ್ದಾದರೂ ಹೇಗೆ? ನೀವು ಹೇಳಬಹುದು: “ನಾನು ಹುಟ್ಟಿನಿಂದಲೇ ಬಡವನಾಗಿ, ದುಃಖಿಯಾಗಿದ್ದೇನೆ. ನನ್ನ ಜೀವನದುದ್ದಕ್ಕೂ ನಾನು ಪಟ್ಟ ಪಾಡು ಅಷ್ಟಿಷ್ಟಲ್ಲ. ಅವೆಲ್ಲವೂ ನನ್ನ ಮನಸ್ಸಿನಲ್ಲಿ ಅಚ್ಚಳಿಯದೇ ಇದೆ'' ಎಂದು. ದಾರ್ಶನಿಕರು ಇದಕ್ಕೆ “(ಈ ದುಃಖ ಭೋಗಕ್ಕೆ ಹೊಣೆಗಾರ), ದೋಷಿ ನೀನೇ'' ಎನ್ನುವರು. ನೀವು ಇವೆಲ್ಲವೂ ಏನೂ ಕಾರಣವಿಲ್ಲದೆಯೇ ಅಕಾರಣವಾಗಿ ಎಲ್ಲಿಂದಲೋ ಹುಟ್ಟಿಬಂದಿತೆಂದು ಹೇಳುತ್ತಿಲ್ಲವಷ್ಟೆ? ನೀವಾದರೋ ವಿಚಾರವಾದಿಗಳು, ನಿಮ್ಮ ಜೀವನ ಕಾರಣವಿಲ್ಲದೆ ಘಟಿಸಿದುದಲ್ಲ ಮತ್ತು ಆ ಕಾರಣ ನೀವೇ! ಸರ್ವದಾ ನಿಮ್ಮ ಬದುಕನ್ನು ನೀವೇ ನಿರ್ಮಾಣ ಮಾಡುತ್ತಿದ್ದೀರಿ. ನೀವೇ ನಿಮ್ಮ ಬದುಕಿನ ಮೂಸೆಯನ್ನು ಮಾಡಿ ನಿಮ್ಮನ್ನು ರೂಪಿಸಿಕೊಳ್ಳುತ್ತಿದ್ದೀರಿ. ನಿಮಗೆ ನೀವೇ ಹೊಣೆ, ಬೇರೆ ಯಾರ ಮೇಲೂ, ಯಾವುದೇ ಸೈತಾನನ ಮೇಲೂ ದೋಷವನ್ನು ಹೊರಿಸಬೇಡಿ. ಅದರಿಂದ ನೀವೇ ಇನ್ನೂ ಹೆಚ್ಚು ದುಃಖವನ್ನು ಅನುಭವಿಸಬೇಕಾಗುತ್ತದೆ ಅಷ್ಟೆ.

ಒಬ್ಬ ವ್ಯಕ್ತಿಯ (ಮರಣಾನಂತರ) ಅವನನ್ನು ದೇವರ ಮುಂದೆ ವಿಚಾರಣೆಗಾಗಿ ಕರೆತರಲಾಯಿತು. ವಿಚಾರಣೆಯ ನಂತರ “ನಿನಗೆ \enginline{31} ಬೆತ್ತದ ಹೊಡೆತಗಳು'' ಎಂದು ದೇವರು ಅವನಿಗೆ ಶಿಕ್ಷೆ ವಿಧಿಸಿದನು. ಮತ್ತೊಬ್ಬನು ಬಂದಾಗ ದೇವರು ನುಡಿದನು: \enginline{“30} ಬೆತ್ತದ ಏಟುಗಳು – \enginline{15} ಅವನ ಸ್ವಂತ ಪಾಪಕ್ಕಾಗಿ ಮತ್ತು ಇನ್ನುಳಿದ \enginline{15} ಅವನಿಗೆ ಉಪದೇಶವನ್ನು ಕೊಟ್ಟ ತಿಳಿಗೇಡಿ ಗುರುವಿಗೆ." ಧರ್ಮೋಪದೇಶ ಕೊಡುವುದರಲ್ಲಿನ ದೊಡ್ಡ ಅನಾಹುತವೇ ಇದು. ನನಗಾಗಿ ಏನು ಕಾದಿದೆಯೋ ನಾ ಕಾಣೆ. ಕಾರಣ, ನಾನಂತೂ ಜಗತ್ತಿನಲ್ಲೆಲ್ಲೆಡೆಯಲ್ಲಿಯೂ ಧರ್ಮೋಪದೇಶ ಮಾಡುತ್ತಲೇ ಸುತ್ತುತ್ತಿರುತ್ತೇನೆ. ನನ್ನಿಂದ ಉಪದೇಶಿಸಲ್ಪಟ್ಟ ಪ್ರತಿಯೊಬ್ಬನಿಂದಾಗಿಯೂ ನನಗೆ \enginline{15} ಬೆತ್ತದ ಏಟುಗಳಾದರೆ – ನನಗೆ ದೇವರೇ ಗತಿ!

ನಮ್ಮ ವಿಚಾರಪ್ರಕ್ರಿಯೆ ಈ ಹಂತವನ್ನು ತಲುಪಬೇಕಾಗುತ್ತದೆ: “ಇದು ನನ್ನ (ಈಶ್ವರನ) ಮಾಯೆ – ದೈವೀ ಮಾಯೆ.'' ಇದು ಅವನ ಕ್ರಿಯಾಶಕ್ತಿ, ದೈವಿಕತೆ. (ಶ‍್ರೀಕೃಷ್ಣ ಗೀತೆಯಲ್ಲಿ ಹೇಳಿದಂತೆ) “ಈ ನನ್ನ ಮಾಯೆ ದುರತಿಕ್ರಮವಾದುದು.” ಆದರೆ ಯಾರು ನನ್ನಲ್ಲಿ ಶರಣಾಗುತ್ತಾರೋ ಅವರು ಈ ಮಾಯೆಯನ್ನು ದಾಟುವರು. ತಮ್ಮ ಸ್ವಂತ ಪ್ರಯತ್ನದಿಂದಲೇ ಯಾರೂ ಈ ಮಾಯೆಯ (ಸಂಸಾರ) ಸಾಗರವನ್ನು ದಾಟುವುದು ಅತ್ಯಂತ ಕಠಿಣ ಎಂಬುದನ್ನು ಮನಗಾಣಬಹುದು. ಒಂದು ವಿಧದಲ್ಲಿ ಅದು ನಿಮ್ಮಿಂದ ಅಸಾಧ್ಯ. (ಕಾರಣ) ಇದು ಅದೇ ಹಳೆಯ ಪ್ರಶ್ನೆಯಂತೆ: ಕೋಳಿ ಅಥವಾ ಮೊಟ್ಟೆ ಯಾವುದು ಮೊದಲೆಂದು? ನೀವ್ಯಾವುದೇ ಕರ್ಮವೆಸಗಿದರೂ ಅದರ ಫಲ ಉಂಟಾಗುತ್ತದೆ. ಕರ್ಮವು ಕಾರಣವಾಯಿತು; ಫಲವು ಪರಿಣಾಮವಾಯಿತು. ಆದರೆ ಈ ಫಲ ಮತ್ತೆ ನಿಮ್ಮನ್ನು ಹೊಸ ಕರ್ಮಕ್ಕೆ ಪ್ರವೃತ್ತರಾಗುವಂತೆ ಮಾಡುತ್ತದೆ. ಆಗ ಪರಿಣಾಮ ಕಾರಣವಾಗುತ್ತದೆ. ಮತ್ತು ಹೊಸ ಕರ್ಮ ಪರಿಣಾಮವಾಗುತ್ತದೆ. ಈ ರೀತಿಯಾಗಿ ಕಾರ್ಯಕಾರಣ ಪರಂಪರೆ ಮುಂದುವರಿಯುತ್ತಲೇ ಇರುತ್ತದೆ. ಒಂದು ಬಾರಿ ಚಲಿಸಲು ಬಿಟ್ಟರೆ ಈ ಕರ್ಮಚಕ್ರ ಸುತ್ತುತ್ತಲೇ ಇರುತ್ತದೆ. ಅದಿನ್ನು ನಿಲ್ಲುವುದೇ ಇಲ್ಲ. ಯಾವುದೇ ಕೆಲಸ – ಒಳ್ಳೆಯದನ್ನು ಅಥವಾ ಕೆಟ್ಟದ್ದನ್ನು ಮಾಡಿದೆವೆಂದರೆ (ಅದು ಪ್ರತಿಕ್ರಿಯೆಗಳ ಸರಣಿಯನ್ನೇ ಉಂಟುಮಾಡುತ್ತದೆ.) ಇನ್ನು ಅದನ್ನು ನನ್ನಿಂದ ನಿಲ್ಲಿಸಲಾಗುವುದಿಲ್ಲ.”

(ನಮ್ಮ ಪ್ರಯತ್ನದಿಂದಲೇ) ಈ ಕರ್ಮಬಂಧನದಿಂದ ಹೊರಬರುವುದಂತೂ\break ಅಸಾಧ್ಯವಾದ ಮಾತು. ಆದರೆ ಈ ಕಾರ್ಯ–ಕಾರಣ ನಿಯಮಕ್ಕಿಂತಲೂ ಹೆಚ್ಚು ಶಕ್ತಿಶಾಲಿಯಾದ ಯಾರಾದರೂ ಇದ್ದು, ಆತ ನಮ್ಮ ಮೇಲೆ ದಯೆತೋರಿ ನಮ್ಮನ್ನು ಈ ಕರ್ಮಪಾಶದಿಂದ ಬಿಡಿಸಿ ಹೊರಗೆ ತಂದರೆ ಮಾತ್ರ ಇದು ಸಾಧ್ಯವಾಗುತ್ತದೆ.

ಮತ್ತು ಅಂಥವನೊಬ್ಬನಿದ್ದಾನೆಂದೇ ನಾವು ಘೋಷಿಸುತ್ತೇವೆ. ಅವನೇ – ಈಶ್ವರ ಅಥವಾ ದೇವರು, ಪರಮದಯಾಳುವಾದ ಅಂತಹದೊಂದು ದೇವರು, ಇದ್ದೇ ಇದ್ದಾನೆ. ಅಂತಹ ದೇವರೊಬ್ಬನಿದ್ದಾನೆಂದೇ ನಮ್ಮ ರಕ್ಷಣೆ ಸಾಧ್ಯವಾಗುವುದು. ನಿಮ್ಮ ಇಚ್ಛೆಯಿಂದಲೇ ನಿಮ್ಮ ಬಲದಿಂದಲೇ ನಿಮ್ಮ ರಕ್ಷಣೆ ಹೇಗೆ ಸಾಧ್ಯ? (ಈಶ್ವರನ) ಕೃಪೆಯಿಂದ ಮಾತ್ರವೇ ವಿಮೋಚನೆ. ಮುಕ್ತಿ ಎಂಬ ಮತವಾದದಲ್ಲಿನ ಅಂತರ್ನಿಹಿತವಾದ ದರ್ಶನ ನಿಮಗೆ ಸ್ಪಷ್ಟವಾಗುತ್ತಿದೆಯೆಷ್ಟೆ? ನೀವು, ಪಾಶ್ಚಾತ್ಯ ದಾರ್ಶನಿಕರು, ಬಹಳ ಚಮತ್ಕಾರ ಬುದ್ಧಿಯುಳ್ಳವರು. ಆದರೆ ನೀವು ತತ್ತ್ವವಿಚಾರವನ್ನು ವ್ಯಾಖ್ಯೆಮಾಡಲು ತೊಡಗಿದಾಗ ನೀವದನ್ನು ಅಷ್ಟೇ ಕಲಸುಮೇಲೋಗರ ಮಾಡುತ್ತೀರಿ. ವಿಮೋಚನೆ ಅಥವಾ ಮುಕ್ತಿ ಎಂಬುದರ ಅರ್ಥ ನಿಮ್ಮನ್ನು ಈ ಸಮಸ್ತ ಪ್ರಕೃತಿಯಿಂದ ಹೊರತರುವುದು ಎಂದೇ? ಆದರೆ ಬರೀ ಕರ್ಮನಿಷ್ಠೆಯ ಮೂಲಕ ನಿಮ್ಮನ್ನು ನೀವು ಹೇಗೆ ಮುಕ್ತರಾಗಿಸಿಕೊಳ್ಳಬಲ್ಲಿರಿ? ಮುಕ್ತಿ ಎಂದರೆ (ಪ್ರಕೃತಿಗೆ ಅತೀತನಾದ) ದೇವರನ್ನು ಆಶ್ರಯಿಸಿ ದೇವರಲ್ಲೇ ಅವಸ್ಥಾನರಾಗಿರುವುದು ಎಂದಷ್ಟೇ. ನೀವೇನಾದರೂ ಮುಕ್ತಿ ಎಂಬ ಭಾವನೆಯನ್ನು ಸರಿಯಾಗಿ ಗ್ರಹಿಸಿದ್ದೇ ಆದರೆ ಆಗ ನೀವು ಪ್ರತ್ಯಗಾತ್ಮನೇ ಆಗುತ್ತೀರಿ, ನೀವು ಪ್ರಕೃತಿಸ್ವರೂಪರಾಗಿರುವುದಿಲ್ಲ. ಆಗ, ಇತರೇ ಬದ್ಧ ಜೀವಾತ್ಮರೂ ದೇವತೆಗಳು ಮತ್ತು ಪ್ರಕೃತಿ – ಇವೆಲ್ಲದರ ಹೊರಗೆ ನೀವಷ್ಟೇ ಇರುತ್ತೀರಿ. ಇವೆಲ್ಲಕ್ಕೂ ಕೇವಲ ಬಾಹ್ಯ ಸತ್ತೆ ಇದೆ ಅಷ್ಟೆ. ಪ್ರಕೃತಿ ಮತ್ತು ಜೀವ ಇವೆರಡರಲ್ಲಿಯೂ ದೇವರು ಓತಪ್ರೋತನಾಗಿರುವನು.

ಆದ್ದರಿಂದ ಹೇಗೆ ಈ ವ್ಯಷ್ಟಿ ಆತ್ಮ ಈ ಶರೀರದೊಡನೆ ಸಂಬಂಧಿಸಿದೆಯೋ ಅಂತೆಯೇ ನಾವು ಪರಮಾತ್ಮನ ಶರೀರಗಳೇ ಆಗಿದ್ದೇವೆ. ದೇವರು – ಜೀವಾತ್ಮ – ಪ್ರಕೃತಿ ಇವು ಮೂರೂ ಸೇರಿ ಒಂದೇ ಆಗಿದೆ. ಅದು ಒಂದೇ ಎಂದು ಹೇಳುವುದೇಕೆಂದರೆ, ದೇಹ, ದೇಹಿಯಾದ (ಆತ್ಮ) ಮತ್ತು ಮನಸ್ಸು – ಎಲ್ಲವೂ ಒಂದೇ. ಆದರೆ ನಾವು ನೋಡಿದಂತೆ\break ಕಾರ್ಯಕಾರಣ ನಿಯಮವು ಪ್ರಕೃತಿಯ ಕಣಕಣವನ್ನೂ ಆವರಿಸಿದೆ. ಒಂದುಬಾರಿ ಅದರಲ್ಲಿ ನೀವು ಸಿಕ್ಕಿಹಾಕಿಕೊಂಡುಬಿಟ್ಟರೆ ಅದರಿಂದ ನೀವು ಹೊರಬರಲಾಗುವುದಿಲ್ಲ. ಹಾಗೇನಾದರೂ ಒಮ್ಮೆ ಈ ಕಾರ್ಯಕಾರಣ ನಿಯಮದ ಬಲೆಯಲ್ಲಿ ಸಿಕ್ಕಿಬಿದ್ದಿರೆಂದರೆ, ಅದರಿಂದ ಪಾರಾಗುವ ಮಾರ್ಗ ನೀವು ಮಾಡುವ ಸತ್ಕರ್ಮಗಳಲ್ಲ. ಈ ಪ್ರಪಂಚದಲ್ಲಿರುವ ಕ್ರಿಮಿ – ಕೀಟಾದಿಗಳಿಗೂ ನೀವು ಆಸ್ಪತ್ರೆಗಳನ್ನು ಕಟ್ಟಬಹುದು... ಇಷ್ಟೆಲ್ಲ ಮಾಡಿದರೂ, ಇವುಗಳಿಂದ ಎಂದಿಗೂ ನಿಮಗೆ ಮೋಕ್ಷಪ್ರಾಪ್ತಿಯಾಗುವುದಿಲ್ಲ. (ಆಸ್ಪತ್ರೆಗಳೇನು) ಒಂದರ ನಂತರ ಒಂದು ತಲೆ ಎತ್ತುತ್ತವೆ. ಹಾಗೆಯೇ ಅಷ್ಟೇ ಬೇಗ ಉರುಳಿ ಹೋಗುತ್ತವೆ. ವಿಮೋಚನೆ ನಮಗೆಂದಾದರೂ ಪ್ರಾಪ್ತವಾಗುವುದಿದ್ದರೆ, ಅದೇನಿದ್ದರೂ ಪ್ರಕೃತಿಯಿಂದ ಎಂದಿಗೂ ಬದ್ದನಾಗದೆ ಪ್ರಕೃತಿಯ ಅಧೀಶ್ವರನಾಗಿರುವಂತಹ\break ಯಾರಾದರೊಬ್ಬರಿಂದ ಮಾತ್ರ ಸಾಧ್ಯ. ಅವನು ಪ್ರಕೃತಿಯನ್ನು ನಿಯಂತ್ರಿಸುತ್ತಾನೆಯೇ ವಿನಃ ಅವನೆಂದಿಗೂ ಪ್ರಕೃತಿಯಿಂದ ನಿಯಂತ್ರಿಸಲ್ಪಡುವುದಿಲ್ಲ. ನಿಯಮಗಳು ಅವನನ್ನು ನಡೆಸುವುದಿಲ್ಲ. ಬದಲಾಗಿ ಅವನ ಇಚ್ಛೆಯಿಂದ ನಿಯಮಗಳು ಚಾಲ್ತಿಯಲ್ಲಿರುತ್ತವೆ. ಅವನು ಯಾವತ್ತೂ ಅಸ್ತಿತ್ವದಲ್ಲಿರುತ್ತಾನೆ. ಹಾಗೂ ಅವನು ಪರಮ ದಯಾಳುವೂ ಆಗಿದ್ದಾನೆ. ಅವನನ್ನು ನೀವು ಕರೆದಾಕ್ಷಣ (ನಿಮ್ಮನ್ನವನು) ಕಾಪಾಡುತ್ತಾನೆ.

ಹಾಗಾದರೆ ಅವನೇಕೆ ನಮ್ಮನ್ನು ಈ ಜಂಜಾಟದಿಂದ ವಿಮೋಚನೆಗೊಳಿಸುತ್ತಿಲ್ಲ? ನಿಜವೆಂದರೆ ನಿಮಗವನು ಬೇಕಿಲ್ಲ. ಅವನನ್ನು ಬಿಟ್ಟು ನಿಮಗೆ ಮಿಕ್ಕದ್ದೆಲ್ಲವೂ ಬೇಕು. ಯಾವಾಗ ನೀವು ಅವನಿಗಾಗಿ ಕಾತರರಾಗುತ್ತೀರೋ ಆ ಕ್ಷಣವೇ ನೀವು ಅವನನ್ನು ಪಡೆಯಬಹುದು. ಆದರೆ ನಾವು ಅವನನ್ನೆಂದಿಗೂ ಬಯಸುವುದಿಲ್ಲ. ನಾವು ದೇವರಿಗೆ ಹೇಳುತ್ತೇವೆ: “ಪ್ರಭು, ನನಗೊಂದು ಸುಂದರವಾದ ಬಂಗಲೆಯನ್ನು ಕೊಡು” ಎಂದು. ನಮಗೆ ಬೇಕಾಗಿರುವುದು ಬಂಗಲೆ, ಅವನಲ್ಲ. “ಪ್ರಭು, ನನಗೆ ಆರೋಗ್ಯಾದಿಗಳನ್ನು ಕೊಟ್ಟು ನನ್ನನ್ನು ಕಷ್ಟದಿಂದ ಪಾರುಮಾಡು!?” ಇತ್ಯಾದಿಯಾಗಿ ದೇವರಲ್ಲಿ ಯಾಚಿಸುತ್ತೇವೆ. ಆದರೆ ಯಾವಾಗ ವ್ಯಕ್ತಿ ದೇವರನ್ನಲ್ಲದೆ ಮತ್ತೇನನ್ನೂ ಬಯಸುವುದಿಲ್ಲವೋ ಆಗ ಅವನು ದೇವರನ್ನು ಪಡೆಯುತ್ತಾನೆ. “ಹೇ ಭಗವನ್, ಧನಿಕರು ಚಿನ್ನ, ಬೆಳ್ಳಿ ಮತ್ತು ಸಂಪತ್ತನ್ನು ಅದೆಷ್ಟು ಪ್ರೀತಿಸುತ್ತಾರೋ, ನಿನ್ನ ಮೇಲೆ ನನಗೆ ಅಂತಹ ಪ್ರೀತಿಯುಂಟಾಗಲಿ. ನನಗೆ ಇಹಲೋಕವೂ ಬೇಡ, ಸ್ವರ್ಗವೂ ಬೇಡ. ಅಥವಾ ಸೌಂದರ್ಯವಾಗಲೀ, ಪಾಂಡಿತ್ಯವಾಗಲೀ ಯಾವುದೂ ಬೇಡ. ಅಷ್ಟೇಕೆ, ನನಗೆ ಮುಕ್ತಿಯೂ ಬೇಡ. ಮತ್ತೆ ಮತ್ತೆ ನಾನು ನರಕಕ್ಕೆ ಹೋಗುವಂತಾದರೆ ಹಾಗೆಯೇ ಆಗಲಿ. ಆದರೆ ನಾನು ಅಪೇಕ್ಷಿಸುವುದೊಂದನ್ನೇ – ಅದೇ ನಿನ್ನಲ್ಲಿ ಪ್ರೇಮಭಕ್ತಿಯನ್ನು; ಕೇವಲ ನಿನ್ನಲ್ಲಿನ ಪ್ರೀತಿಗೋಸ್ಕರವಾಗಿಯೇ ನಿನ್ನನ್ನು ಪ್ರೀತಿಸುವುದನ್ನು, ಸ್ವರ್ಗಕ್ಕಾಗಿಯೂ ಅಲ್ಲ.''

ಮನುಷ್ಯ ತಾನು ಬಯಸಿದ್ದನ್ನು ಪಡೆಯುವನು. ನೀವು ಸದಾ ಶರೀರದ\break ಲಾಲಸೆಯಲ್ಲೇ ಇದ್ದರೆ ಮತ್ತೊಂದು ಶರೀರವನ್ನು ಪಡೆಯುತ್ತೀರಿ. ಈ ಶರೀರ ಬಿಟ್ಟುಹೋದಾಗ ಮತ್ತೊಂದನ್ನು ಬಯಸುತ್ತಾ ಒಂದು ಶರೀರದ ನಂತರ ಮತ್ತೊಂದನ್ನು ಪಡೆಯುತ್ತಿರುತ್ತೀರಿ. ನೀವು ಜಡವಸ್ತುಗಳಲ್ಲಿ ಅನುರಕ್ತರಾದರೆ, ನೀವು\break ಜಡವಸ್ತುಗಳಂತೆಯೇ ಆಗುತ್ತೀರಿ – ಮೊದಲು ನೀವು ಪಶುಜನ್ಮ ಪಡೆಯುತ್ತೀರಿ. ನಾಯಿ ಮೂಳೆಯೊಂದನ್ನು ಕಡಿಯುತ್ತಿರುವುದನ್ನು ನೋಡಿದಾಗ “ದೇವರೇ ನಮ್ಮನ್ನು ಕಾಪಾಡು'' ಎಂದೆನ್ನುತ್ತೇನೆ. ಅತ್ಯಂತ ದೇಹಾಸಕ್ತಿಯುಳ್ಳವರಾದರೆ ನೀವು ನಾಯಿ–ಬೆಕ್ಕುಗಳಾಗಿ ಹುಟ್ಟುತ್ತೀರಿ! ಮತ್ತೂ ಅವನತಿಗೊಂಡರೆ, ಯಾವುದು ಜಡಪಿಂಡವಲ್ಲದೆ ಮತ್ತೇನೂ ಅಲ್ಲವೋ – ಅಂತಹ ಖನಿಜ ಪದಾರ್ಥಗಳಾಗಿ ಇರಬೇಕಷ್ಟೆ.

ಮತ್ತೆ ಕೆಲವರಿದ್ದಾರೆ; ಅವರು ಯಾವುದರೊಂದಿಗೂ ರಾಜಿಮಾಡಿಕೊಳ್ಳುವುದಿಲ್ಲ. ಮೋಕ್ಷಕ್ಕಿರುವ ಮಾರ್ಗ ಸತ್ಯದ ಮೂಲಕವೇ. ಇದೇ ಇನ್ನೊಂದು ಮೂಲಮಂತ್ರ.

ಮನುಷ್ಯ ಯಾವಾಗ ಸೈತಾನನನ್ನು ಒದ್ದು ಹೊರಗೆ ಓಡಿಸಿದನೋ ಆಗ ಅವನು ಆಧ್ಯಾತ್ಮಿಕ ಉನ್ನತಿಯನ್ನು ಸಾಧಿಸಿದನು. ಅವನು ಎದ್ದು ನಿಂತು ಇಡೀ ಸಂಸಾರದಲ್ಲಿರುವ ದುಃಖ–ಕಷ್ಟಗಳ ಹೊಣೆಯನ್ನು ತನ್ನ ಹೆಗಲಮೇಲೆ ಹೇರಿಕೊಂಡನು. ಆದರೆ ಯಾವಾಗ ಅವನು ಭೂತ ಮತ್ತು ಭವಿಷ್ಯತ್ತುಗಳ ಕಡೆಗೆ ದೃಷ್ಟಿ ಹಾಯಿಸಿದನೋ ಹಾಗೂ ಅವನಲ್ಲಿ ಕಾರ್ಯ–ಕಾರಣ ಸಂಬಂಧವಾಗಿ ವಿಚಾರ ಮೂಡಿತೋ ಆಗ ಅವನು ದೇವರಿಗೆ ತಲೆಬಾಗಿ ಪ್ರಾರ್ಥಿಸತೊಡಗಿದನು: “ದೇವಾಧಿದೇವಾ, ನನ್ನನ್ನು ಕಾಪಾಡು. ನೀನೇ ನಮ್ಮ ಸ್ರಷ್ಟಾ, ನಮ್ಮ ತಂದೆ, ಹಾಗೂ ನಮ್ಮ ಪ್ರಿಯತಮನಾದ ಸಖನಾಗಿದ್ದೀಯೆ!?” ಇದು ಕಾವ್ಯ ಹೌದು; ಆದರೆ ನನಗನ್ನಿಸುತ್ತದೆ, ಇದು ಅಂತಹ ಉತ್ಕೃಷ್ಟ ಕಾವ್ಯದ ಮಾದರಿಯೇನಲ್ಲ. ಕಾರಣವೇನು? ಇದು ಆ ಅನಂತದ ವರ್ಣನೆಯೇನೋ ಸರಿ. ಪ್ರತಿಯೊಂದು ಭಾಷೆಯಲ್ಲೂ ಈ ರೀತಿಯ ಅನಂತದ ವರ್ಣನೆ ನಿಮಗೆ ದೊರಕುತ್ತದೆ. ಆದರೆ ಇದು ಇಂದ್ರಿಯ ಗೋಚರವಾದ, ಸ್ನಾಯು ಸಂಬಂಧವಾದ ಅನಂತ...

“ಅವನನ್ನು ಸೂರ್ಯನು ಪ್ರಕಾಶಗೊಳಿಸಲಾಗುವುದಿಲ್ಲ. ಚಂದ್ರನಾಗಲೀ, ತಾರಾಗಣಗಳಾಗಲೀ ಅಥವಾ ವಿದ್ಯುತ್ತಾಗಲೀ ಅವನನ್ನು ಬೆಳಗಲಾಗುವುದಿಲ್ಲ." ಇದು ಅನಂತದ ಮತ್ತೊಂದು ವರ್ಣನೆ. ಆದರೆ ಈ ವರ್ಣನೆಯ ಭಾಷೆ ನಿಷೇಧಾತ್ಮಕವಾಗಿದೆ.

ಉಪನಿಷತ್ತುಗಳಲ್ಲಿನ ಆಧ್ಯಾತ್ಮಿಕ ಚಿಂತನೆಗಳಲ್ಲಿ ಅನಂತವನ್ನು ಈ ರೀತಿ ಉತ್ಕೃಷ್ಟವಾದ, ಅದರ ಚರಮಭಾವದಲ್ಲಿ ವರ್ಣಿಸಲಾಗಿದೆ. ಈ ಜಗತ್ತಿನಲ್ಲಿ ವೇದಾಂತವು ಅತ್ಯಂತ ಶ್ರೇಷ್ಠದರ್ಶನ ಮಾತ್ರವೇ ಅಲ್ಲ; ಅದ್ಭುತವಾದ ಅತ್ಯಂತ ಶ್ರೇಷ್ಠ ಕಾವ್ಯವೂ ಹೌದು.

ನೀವಿಲ್ಲಿ ಗಮನಿಸಬೇಕಾದುದು, ವೇದಗಳ ಮೊದಲ ಮತ್ತು ಎರಡನೆಯ ಭಾಗಗಳಲ್ಲಿರುವ ಈ ವ್ಯತ್ಯಾಸವನ್ನು. ಮೊದಲ ಭಾಗದಲ್ಲಿನ ಮಂತ್ರಗಳೆಲ್ಲ ಇಂದ್ರಿಯವೇದ್ಯವಾಗಿರುವ ಜಗತ್ತನ್ನೇ ಕುರಿತದ್ದು. ಎಲ್ಲ ಧರ್ಮಗಳೂ ಬಾಹ್ಯಜಗತ್ತಿನ ಅನಂತತೆಯ ಬಗ್ಗೆ – ಅಂದರೆ ಪ್ರಕೃತಿ ಮತ್ತು ಪ್ರಕೃತಿಯ ಅಧಿಕರ್ತಾ – ಇವಿಷ್ಟಕ್ಕೆ ಮಾತ್ರ ಸೀಮಿತವಾಗಿದೆಯಷ್ಟೇ. ಆದರೆ ಎರಡನೆಯ ಭಾಗವಾದ ವೇದಾಂತ ಆ ರೀತಿಯದಲ್ಲ. ಇದರಲ್ಲೇ ಮೊಟ್ಟ ಮೊದಲಬಾರಿಗೆ ಮಾನವ ಚೇತನದ ಸಮುಜ್ವಲವಾದ ಬೆಳಕು ಕಾಣಿಸಿಕೊಳ್ಳುತ್ತದೆ. ದೇಶಗತವಾದ, ದಿಗಂತ ವ್ಯಾಪ್ತಿಯಾದ ಅನಂತತೆಯಲ್ಲಿ ಯಾವುದೇ ತೃಪ್ತಿಯೂ ಉಂಟಾಗಲಿಲ್ಲ – “ಸ್ವಯಂಭುವಾದ ಪರಮೇಶ್ವರನು ಮನುಷ್ಯನ ಸಮಸ್ತ ಇಂದ್ರಿಯಗಳನ್ನು ಬಹಿರ್ಮುಖವಾಗಿ ಹರಿದುಹೋಗುವಂತೆ ಮಾಡಿದನು. ಆದ್ದರಿಂದ ಯಾರು ಬಾಹ್ಯನಿರತರೋ ಅವರು ಬಾಹ್ಯವಸ್ತುಗಳನ್ನಲ್ಲದೆ ಅಂತರಾತ್ಮನನ್ನೆಂದಿಗೂ ಕಾಣಲಾರರು. ಆದರೆ ಯಾರೋ ಕೆಲವು ಭಾಗ್ಯಶಾಲಿಗಳು ಸತ್ಯವನ್ನು ತಿಳಿಯಲು ಇಚ್ಛಿಸಿ ತಮ್ಮ ದೃಷ್ಟಿಯನ್ನು ಅಂತರ್ಮುಖಗೊಳಿಸಿ, ತಮ್ಮ ಆಂತರದಲ್ಲಿರುವ ಪ್ರತ್ಯಗಾತ್ಮನ ಮಹಿಮೆಯನ್ನು ಕಾಣುತ್ತಾರೆ.”

ಈ ಆತ್ಮದ ಅನಂತತೆ ದೇಶಗತವಾದ ಅನಂತವಲ್ಲ. ಆದರೆ ಇದು ಯಥಾರ್ಥವಾದ, ದೇಶ–ಕಾಲಾತೀತವಾದ, ಅನಂತ. ಪಾಶ್ಚಾತ್ಯರು ಈ (ಅನಂತ) ಜಗತ್ತಿನ ತತ್ತ್ವಬೋಧೆಯಿಂದ ವಂಚಿತರು. ಅವರ ಮನಸ್ಸೆಲ್ಲ ಬಾಹ್ಯಪ್ರಕೃತಿ ಮತ್ತು ಅದರ ಅಧೀಶ್ವರನೆಡೆಗೆ ತಿರುಗಿಸಲ್ಪಟ್ಟಿದೆ. ಅಂತರ್ಮುಖರಾಗಿ, ವಿಸ್ಮೃತವಾದ ಸತ್ಯವನ್ನು ಅಲ್ಲಿ ಸಂದರ್ಶಿಸಿ, ದೇವತೆಗಳ ಸಹಾಯವಿಲ್ಲದೆ ಈ ಸಂಸಾರ–ಸ್ವಪ್ನದ ಬಲೆಯಿಂದ ಮನಸ್ಸು ಹೊರಬರಲು ಸಾಧ್ಯವೇ? ಒಮ್ಮೆ ಕರ್ಮದಲ್ಲಿ ಪ್ರವೃತ್ತವಾಗಿ ಹೋದಮೇಲೆ ಅದರ ಶೃಂಖಲೆಯಿಂದ ಮುಕ್ತನಾಗಬೇಕಾದರೆ ಕೃಪಾಮಯನಾದ ಪರಮಪಿತನು ದಯೆತೋರಿದಲ್ಲದೆ ಬೇರೆ ದಾರಿಯೇ ಇಲ್ಲ.

ಆದರೆ ಭಗವಂತನ ಆಶ್ರಯದಲ್ಲಿರುವುದೂ ಚರಮ ಮುಕ್ತಿಯಲ್ಲ. ದಾಸತ್ವ\break ದಾಸತ್ವವೇ. ಚಿನ್ನದ ಸರಪಳಿಯೂ ಕಬ್ಬಿಣದ ಸರಪಳಿಯಷ್ಟೇ ಬಾಧಕವಾದುದು. ಇದರಿಂದ ಹೊರಬರಲು ದಾರಿಯಿದೆಯೆ?

ಹಾಗೆ ನೋಡಿದರೆ ವಾಸ್ತವವಾಗಿ, ನೀವು ಬದ್ಧರಾಗಿಲ್ಲ. ಯಾರೂ ಯಾವತ್ತೂ ಬದ್ದರಾಗಿರಲಿಲ್ಲ. ಆತ್ಮವು ಈ ಸಂಸಾರದಿಂದ ಅತೀತವಾದುದರಿಂದ ಬಂಧನರಹಿತವಾದುದು. ಅದೇ ಸರ್ವಸ್ವವೂ ಆಗಿದೆ. ನೀವು ಭಿನ್ನರಲ್ಲ – ಅಖಂಡರಾಗಿದ್ದೀರಿ. ಆ ಏಕವೇ ನೀವು. ದ್ವೈತವೆಂಬುದೇ ಇಲ್ಲ. ದೇವರು ಅಥವಾ ಈಶ್ವರನಾದರೋ ಮಾಯೆಯ ಪರದೆಯ ಮೇಲೆ ಬಿದ್ದಿರುವ ನಿಮ್ಮ ಪ್ರತಿಬಿಂಬವೇ ಆಗಿದೆ. ನಿಜವಾದ ದೇವರೆಂದರೆ ಈ ಆತ್ಮವೇ. ಮನುಷ್ಯನು ತಿಳಿಯದೆ ಅಜ್ಞಾನವಶನಾಗಿ ಯಾರನ್ನು ಪೂಜಿಸುತ್ತಾನೋ ಅದು ಆತ್ಮದ ಪ್ರತಿಬಿಂಬವೇ ಆಗಿದೆ. ಸ್ವರ್ಗದಲ್ಲಿರುವ ಪರಮಪಿತನೇ ದೇವರೆಂದು ಕೆಲವರು ಹೇಳುತ್ತಾರೆ. ಆತ ದೇವರಾಗಿರಲು ಕಾರಣವೇನು? ಆತ ನಿಮ್ಮದೇ ಪ್ರತಿಬಿಂಬವಾಗಿರುವುದರಿಂದ ದೇವರಾಗಿದ್ದಾನೆ. ಆದ್ದರಿಂದ ಸರ್ವದಾ ನೀವು ದೇವರನ್ನು ಹೇಗೆ ನೋಡುತ್ತಿರುವಿರೆಂಬುದು ನಿಮಗೆ ಸ್ಪಷ್ಟವಾಗಿರಬೇಕು. ನಿಮ್ಮನ್ನಾವರಿಸಿರುವ ಮುಸುಕು ಎಷ್ಟು ತೆಗೆಯಲ್ಪಡುವುದೋ ನಿಮ್ಮ ಪ್ರತಿಬಿಂಬ ಅಷ್ಟೂ ಸ್ಪಷ್ಟವಾಗುತ್ತದೆ.

“ಒಂದೇ ಮರದ ಮೇಲೆ ಎರಡು ಸುಂದರವಾದ ಪಕ್ಷಿಗಳಿವೆ. ಮೇಲಿನ ಪಕ್ಷಿ ಸ್ಥಿರ, ಶಾಂತ ಹಾಗೂ ರಾಜಗಂಭೀರವಾಗಿದೆ. ಕೆಳಗಿರುವ ಪಕ್ಷಿ (ಜೀವಾತ್ಮ) ಸಿಹಿ–ಕಹಿ ಹಣ್ಣುಗಳನ್ನು ತಿನ್ನುತ್ತಿದ್ದು ಸುಖ–ದುಃಖದಿಂದ ಕೂಡಿರುತ್ತದೆ. (ಆದರೆ ಯಾವಾಗ ಜೀವಾತ್ಮರೂಪಿಯಾದ ಕೆಳಗಿನ ಪಕ್ಷಿ ಆ ಪರಮಾತ್ಮ ಸ್ವರೂಪಿಯಾದ ಮೇಲಿರುವ ಪಕ್ಷಿಯನ್ನು ತನ್ನಾತ್ಮನೆಂದು ಅರಿಯುತ್ತದೋ ಆಗ ಅದಿನ್ನೆಂದೂ ಶೋಕಿಸುವುದಿಲ್ಲ.”

...“ದೇವರು, ಈಶ್ವರ” ಎಂದು ಹೇಳಬೇಡಿ, “ನೀನು'' ಎಂದು ಸಹ ಹೇಳಬೇಡಿ. “ನಾನು” ಎಂದು ಹೇಳಿ. ದ್ವೈತವಾದದ ಭಾಷೆಯಲ್ಲಿ, “ದೇವರೇ! ನೀನು ನನ್ನ ತಂದೆ” ಎಂದು ಹೇಳಲಾಗುತ್ತದೆ. ಆದರೆ ಅದ್ವೈತದ ಭಾಷೆಯಲ್ಲಿ ಅದು ಆತ್ಮ “ನನಗೆ ನಾನೆಷ್ಟು ಪ್ರಿಯನೋ ಅದಕ್ಕಿಂತಲೂ ಅಧಿಕವಾಗಿ (ನೀನು) ನನಗೆ ಪ್ರಿಯನಾಗಿರುವೆ. ನಿನಗೆ ನಾನು ಯಾವ ಹೆಸರನ್ನೂ ಇಡುವುದಿಲ್ಲ. ನಿನ್ನೊಡನೆ ಅತ್ಯಂತ ನಿಕಟವಾದ ಸಂಬಂಧವನ್ನು ಸೂಚಿಸಲು, ನನ್ನಿಂದ ವ್ಯವಹರಿಸಲ್ಪಡುವ ಪದವೇ 'ನಾನು' ಎಂಬುದು...”

“ಭಗವಂತನೊಬ್ಬನೇ ಸತ್ಯ. ಜಗತ್ತು ಸ್ವಪ್ನಮಾತ್ರವಷ್ಟೆ. ನಾನು ಯಾವತ್ತೂ ಬಂಧನರಹಿತನಾಗಿ ಮುಕ್ತನಾಗಿದ್ದೆ. ಹಾಗೂ ಚಿರಕಾಲವೂ ಆರೀತಿ ಮುಕ್ತನಾಗಿರುವೆನು. ನಾನು ಧನ್ಯ! ನಾನು ಪೂಜಿಸುತ್ತಿರುವುದು ನನ್ನನ್ನೇ. ಪ್ರಕೃತಿಗಾಗಲೀ, ಭ್ರಮೆಗಾಗಲೀ, ನನ್ನ ಮೇಲೆ ಯಾವ ಹಿಡಿತವೂ ಯಾವತ್ತೂ ಇಲ್ಲ. ಪ್ರಕೃತಿ ನನ್ನಿಂದ ದೂರವಾಗಲಿ; ಪೂಜೆ, ಕಂದಾಚಾರಗಳು ನನ್ನಿಂದ ಮಾಯವಾಗಲಿ. ಕಾರಣ, ನನ್ನನ್ನು ನಾನರಿತಿದ್ದೇನೆ, ನಾನು ಅನಂತವೇ ಆಗಿದ್ದೇನೆ. ಈ ಎಲ್ಲ ವ್ಯಕ್ತಿವಿಶೇಷಗಳೂ, ಜವಾಬ್ದಾರಿಗಳೂ, ಸುಖ – ದುಃಖ ಎಲ್ಲದರ ಬೋಧೆಯೂ ಲಯವಾಗಿದೆ. ನಾನೇ ಆ ಭೂಮಾ! ನನಗೆ ಜನನವಾಗಲೀ, ಮರಣವಾಗಲೀ ಇರಲು ಹೇಗೆ ಸಾಧ್ಯ? ನಾನೇ ಆ ಒಂದಾಗಿರುವಾಗ ನನಗೆ ಭಯವೆಲ್ಲಿಯದು? ನನ್ನನ್ನೇ ಕಂಡು ನಾನು ಅಂಜಲೇ? ಯಾರು ಯಾರಿಗೆ ಭಯಪಡಬೇಕಾಗಿದೆ? ಅಖಂಡವಾಗಿರುವ ಒಂದೇ ಸತ್ತೆ – ಅದು ನಾನೇ ಆಗಿದ್ದೇನೆ. ಬೇರಿನ್ನಾವುದೂ ಇಲ್ಲ. ಅಸ್ತಿತ್ವದಲ್ಲಿರುವ ಎಲ್ಲವೂ ನಾನೇ, ಪ್ರತಿಯೊಂದೂ ನಾನೇ ಆಗಿದ್ದೇನೆ?”

ವಿಮೋಚನೆ ಅಥವಾ ಮುಕ್ತಿ ಕರ್ಮಸಾಧನೆಯಿಂದಲ. ಇದೇನಿದ್ದರೂ ಚಿರಮುಕ್ತವಾಗಿರುವ ನಿಮ್ಮ ನಿಜಸ್ವರೂಪದ ಪ್ರತಿಬೋಧವಷ್ಟೆ, ಸ್ಮರಣೆ ಮಾತ್ರವಷ್ಟೇ. ನೀವು ಮುಕ್ತಿಯನ್ನು ಪಡೆಯುವುದೆಂತು? ನೀವಾಗಲೇ ಮುಕ್ತರಾಗಿದ್ದೀರಿ.

“ನಾನು ಮುಕ್ತನಾಗಿದ್ದೇನೆ” ಎಂದು ಹೇಳುತ್ತಾ ಹೋಗಿ. ಒಂದು ವೇಳೆ ಇನ್ನೊಂದರೆಕ್ಷಣದಲ್ಲೇ "ನಾನು ಬದ್ಧ' ಎಂಬ ಭ್ರಾಂತಿ ಬಂದರೂ ಚಿಂತಿಸದಿರಿ. ಸಮ್ಮೋಹನಕ್ಕೊಳಗಾದ ಸಮಸ್ತವನ್ನೂ ಭ್ರಮನಿರಸನಗೊಳಿಸಿ.

ಈ ಪರಮ ಸತ್ಯವನ್ನು ಮೊಟ್ಟಮೊದಲನೆಯದಾಗಿ ಆಲಿಸಬೇಕು. ಆದ್ದರಿಂದ ಇದನ್ನು ಮೊದಲು ಕಿವಿಗೊಟ್ಟು ಆಲಿಸಿ, ಹಗಲು – ರಾತ್ರಿ ಅದರ ಮನನವನ್ನು ಮಾಡಿ. ಅಹೋರಾತ್ರಿ ಸದಾ ನಿಮ್ಮ ಮನಸ್ಸನ್ನು – “ನಾನು ಆ ಪರಮ ಸತ್ಯವೇ ಆಗಿದ್ದೇನೆ, ನಾನು ವಿಶ್ವದ ವಿಭುವಾಗಿದ್ದೇನೆ, ಯಾವ ಭ್ರಮೆಯೂ ನನಗಿಲ್ಲ" – ಈ ಮುಂತಾದ ಭಾವನೆಗಳಿಂದ ತುಂಬಿ. ಎಲ್ಲಿಯವರೆಗೂ ಈ ಗೋಡೆಗಳು, ಮನೆಗಳು, ಪ್ರತಿಯೊಂದೂ ವಿಲೀನವಾಗಿ, ಶರೀರವೇ ಮೊದಲಾಗಿ ಎಲ್ಲವೂ ಲಯವಾಗುತ್ತದೋ ಅಲ್ಲಿಯವರೆವಿಗೂ ಸರ್ವಪ್ರಯತ್ನದಿಂದಲೂ ಮನಸ್ಸಿನ ಶಕ್ತಿಗಳನ್ನೆಲ್ಲ ಒಟ್ಟುಗೂಡಿಸಿ ಈ ಪರಮಸತ್ಯದ ಮೇಲೆ ಧ್ಯಾನ ಮಾಡಿ, “ನಾನು ಏಕಾಕಿಯಾಗಿಯೇ ಇರುತ್ತೇನೆ. ಆ ಏಕಾಮೇವಾದ್ವಿತೀಯವೇ ನಾನು" ಎನ್ನುತ್ತಾ ಸಾಧನೆ ಮಾಡುತ್ತಾ ಹೋಗಿ, “ನಮಗೆ ಬೇಕಾಗಿರುವುದು ಮುಕ್ತಿ. ಇನ್ನಾವುದರ ಗೊಡವೆ ನಮಗೇತಕ್ಕೆ? ನಮಗೆ ಯಾವ ಸಿದ್ದಿಗಳೂ ಬೇಡ. ಇಹಲೋಕ, ಸ್ವರ್ಗ, ನರಕ ಮುಂತಾದ ಎಲ್ಲ ಲೋಕಗಳನ್ನೂ ನಾವು ತ್ಯಜಿಸಿದ್ದೇವೆ. ಈ ಎಲ್ಲ ಅಲೌಕಿಕ ಸಿದ್ದಿಗಳು ಅದೂ, ಇದೂ, ಇವುಗಳನ್ನೆಲ್ಲ ಕಟ್ಟಿಕೊಂಡು ನಮಗೇನಾಗಬೇಕಾಗಿದೆ? ಮನಸ್ಸು ವಶೀಭೂತವಾಗಿದೆಯೋ ಇಲ್ಲವೋ ಎಂಬುದರ ಉಸಾಬರಿ ನನಗೇತಕ್ಕೆ? ಅದರ ಗತಿಯಲ್ಲಿ ಅದಿರಲಿ, ಅದು ಹೇಗಿದ್ದರೇನು? ನಾನು ಮನಸ್ಸೇ ಅಲ್ಲ. ಅದು ಹೇಗಾದರೂ ಅಲೆಯುತ್ತಿರಲಿ!”

ಸಾಧು – ಅಸಾಧು ಎಲ್ಲದರ ಮೇಲೂ ಸೂರ್ಯನು ಸಮವಾಗಿಯೇ ತನ್ನ ಬೆಳಕನ್ನು ಚೆಲ್ಲುತ್ತಾನೆ. ಹಾಗೆಂದು ಯಾರದೇ ಕಳಂಕಗಳಿಂದ ಅವನು ಕಲುಷಿತನಾಗುತ್ತಾನೆಯೇ? ಅದೇ ರೀತಿ – “ಸೋಹಂ. ಮನದ ವ್ಯಾಪಾರಗಳಿಂದ ನಾನು ನಿರ್ಲಿಪ್ತನಾಗಿದ್ದೇನೆ. ಮಲಿನವಾದ ಸ್ಥಳಗಳನ್ನು ಬೆಳಗುವುದರಿಂದ ಸೂರ್ಯನೆಂದೂ ಮಲಿನವಾಗಿಲ್ಲ. ಓಂ ತತ್ ಸತ್”.

ಇದೇ ಅದ್ವೈತ ದರ್ಶನದ ಧರ್ಮ. ಇದು ಬಹಳ ಕಠಿಣ. ಆದರೆ ಸಾಧನೆ ಮಾಡುತ್ತಾ ಹೋಗಿ. ಕಂದಾಚಾರಗಳನ್ನೆಲ್ಲ ಕಿತ್ತೊಗೆಯಿರಿ. ಗುರುಗಳಾಗಲೀ, ಗ್ರಂಥಗಳಾಗಲೀ, ದೇವತೆಗಳಾಗಲೀ ಇವು ಯಾವುವೂ ಇಲ್ಲ. ದೇವಸ್ಥಾನಗಳು, ಪೂಜಾರಿಗಳು, ದೇವತೆಗಳು, ಅವತಾರಗಳು – ಅಷ್ಟೇಕೆ ಸ್ವಯಂ ಭಗವಂತನಿಗೇ ವಿದಾಯ ಹೇಳೋಣ. ಇರುವ ಎಲ್ಲ ದೇವರು ನಾನೇ ಆಗಿದ್ದೇನೆ. ಆದ್ದರಿಂದ ಸತ್ಯಾನ್ವೇಷಿಗಳಾದ ದಾರ್ಶನಿಕರೇ ಎದ್ದು ನಿಲ್ಲಿ! ಉತ್ತಿಷ್ಠತ! ನಿರ್ಭೀತರಾಗಿ! ಇನ್ನೆಂದೂ ದೇವರು, ಜಗತ್ತಿನ ಭ್ರಾಂತಿ – ಇವುಗಳ ಮಾತನ್ನೆತ್ತಬೇಡಿ. ಸತ್ಯಮೇವ ಜಯತೇ! ಎಂಬುದು ದಿಟ. ಇದರಲ್ಲಿ ಸಂದೇಹವೇ ಇಲ್ಲ: ನಾನು ಅನಂತವೇ ಆಗಿದ್ದೇನೆ.

ಈ ಎಲ್ಲ ಮತಧರ್ಮಗಳ ಕಂದಾಚಾರಗಳು ಕೆಲಸಕ್ಕೆ ಬಾರದ ಮನೋವಿಕಲ್ಪಗಳಷ್ಟೇ. ಈ ಸಭೆ, ನನ್ನ ಮುಂದಿರುವ ನಿಮ್ಮನ್ನು ನಾನು ನೋಡುತ್ತಿರುವುದು, ನಾನು ಮಾತನಾಡುತ್ತಿರುವುದು – ಇವೆಲ್ಲ ಭ್ರಮೆಯಷ್ಟೇ. ಇವೆಲ್ಲವನ್ನೂ ನಾವು ಅವಶ್ಯವಾಗಿ ತೊರೆಯಬೇಕು. ತತ್ತ್ವಜ್ಞಾನಿಯಾಗ ಬೇಕಾದರೆ ಎಂತಹ ಕೆಚ್ಚೆದೆಯುಳ್ಳವರಾಗಿರಬೇಕು ಎಂಬುದನ್ನು ಸ್ವಲ್ಪ ಪರಾಂಬರಿಸಿ ನೋಡಿ. ಈ ಸಾಧನೆಯು ಜ್ಞಾನಮಾರ್ಗದ ಸಾಧನೆ. ಅಂದರೆ – ಇದು ಜ್ಞಾನದ, ವಿಚಾರದ ಮೂಲಕವೇ ಮೋಕ್ಷವನ್ನು ಪಡೆಯುವ ವಿಧಾನವಾಗಿದೆ. ಅನ್ಯಮಾರ್ಗಗಳು ಸುಲಭ ಹಾಗೂ ನಿಧಾನ! ಆದರೆ ಈ (ಜ್ಞಾನ) ಮಾರ್ಗವಾದರೋ ಅಪೂರ್ವವಾದ ಮನೋಬಲವುಳ್ಳವರು ಮಾತ್ರ ತುಳಿಯಲು ಸಾಧ್ಯ. ದುರ್ಬಲರಾದ ಯಾರಿಗೂ ಎಟುಕುವ ಮಾರ್ಗ ಇದಲ್ಲ. ನೀವು ದೃಢ ಸ್ವರದಿಂದ ಈ ರೀತಿ ಹೇಳಬೇಕಾಗುತ್ತದೆ: “ನಾನು ಆತ್ಮವೇ ಆಗಿದ್ದೇನೆ. ನಾನು ನಿತ್ಯಮುಕ್ತ, ನಾನೆಂದೂ ಬದ್ಧನಾಗಿರಲಿಲ್ಲ. ಕಾಲ ನನ್ನಲ್ಲಿದೆ, ನಾನು ಕಾಲದಲ್ಲಿಲ್ಲ; ಕಾಲದ\break ಅಧೀನದಲ್ಲಿಲ್ಲ. ದೇವರು ಹುಟ್ಟಿದ್ದೇ ನನ್ನ ಮನಸ್ಸಿನಲ್ಲಿ. ಯಾರು ಪರಮಪಿತನಾದ ದೇವರೋ, ವಿಶ್ವಶ್ರೇಷ್ಠನೋ – ಅವನು ನನ್ನ ಮನಸ್ಸಿನ ಸೃಷ್ಟಿಯೇ ಆಗಿದ್ದಾನೆ''

ನೀವು ನಿಮ್ಮನ್ನು ತತ್ತ್ವವೇತ್ತರು, ದಾರ್ಶನಿಕರೆಂದು ಕರೆದುಕೊಳ್ಳುತ್ತೀರಾದರೆ, ಅದನ್ನು ಪ್ರಮಾಣೀಕರಿಸಿ! ಈ ಪರಮಸತ್ಯವನ್ನು ಅನುಧ್ಯಾನಮಾಡಿ, ಇದರ ಮೇಲೆ ವಿಚಾರವಿನಿಮಯ ಮಾಡಿ ಮತ್ತು ಸಾಧನಾಪಥದಲ್ಲಿ ಒಬ್ಬರಿಗೊಬ್ಬರು ಸಹಾಯವನ್ನು ಮಾಡಿ. ಎಲ್ಲ ಮೂಢನಂಬಿಕೆಗಳನ್ನೂ ತೊರೆಯಿರಿ!

