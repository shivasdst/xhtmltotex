
\chapter[ಏಕಾಗ್ರತೆ]{ಏಕಾಗ್ರತೆ\protect\footnote{\enginline{* C.W, Vol. IV, P. 218}}}

\begin{center}
(೧೯೦೦ರ ಮಾರ್ಚ್ ೧೬ರಂದು ಸ್ಯಾನ್‌ಫ್ರಾನ್ಸಿಸ್ಕೋದ ವಾಷಿಂಗ್ಟನ್ ಹಾಲಿನಲ್ಲಿ ನೀಡಿದ ಉಪನ್ಯಾಸ)
\end{center}

[ಇದು ಮತ್ತು ಮುಂದಿನ ಎರಡು ಉಪನ್ಯಾಸಗಳು ದಕ್ಷಿಣ ಕ್ಯಾಲಿಫೋರ್ನಿಯದ ವೇದಾಂತ ಸೊಸೈಟಿಯಿಂದ ಪ್ರಕಟವಾಗುತ್ತಿದ್ದ 'ವೇದಾಂತ ಅಂಡ್ ದಿ ವೆಸ್ಟ್' ಎಂಬ ಪತ್ರಿಕೆಯಿಂದ ತೆಗೆದುಕೊಂಡವುಗಳು. ಇಡಾ ಆಂಸೆಲ್ ಎಂಬ ಭಕ್ತಮಹಿಳೆ ಇವನ್ನು ಶೀಘ್ರಲಿಪಿಯಲ್ಲಿ ಬರೆದುಕೊಂಡಳು.]

ಬಾಹ್ಯ ಪ್ರಪಂಚದ ಜ್ಞಾನವಾಗಲೀ, ಆಂತರಿಕ ಪ್ರಪಂಚದ ಜ್ಞಾನವಾಗಲೀ ಎಲ್ಲವೂ ಬರುವುದು ಒಂದೇ ವಿಧಾನದ ಮೂಲಕ. ಅದೇ ಮನಸ್ಸಿನ ಏಕಾಗ್ರತೆಯಿಂದ. ನಾವು ಯಾವ ವಸ್ತುವಿನ ವಿಷಯವನ್ನು ತಿಳಿದುಕೊಳ್ಳಬೇಕೆಂದಿರುವೆವೋ ಅದರ ಮೇಲೆ ಮನಸ್ಸನ್ನು ಏಕಾಗ್ರ ಮಾಡಿದಲ್ಲದೆ ಯಾವ ಜ್ಞಾನವೂ ಬರುವುದಿಲ್ಲ. ಖಗೋಳ ಶಾಸ್ತ್ರಜ್ಞನು ಟೆಲಿಸ್ಕೋಪಿನ ಮೂಲಕ ತನ್ನ ಮನಸ್ಸನ್ನು ಏಕಾಗ್ರಗೊಳಿಸುವನು. ನೀವು ನಿಮ್ಮ ಮನಸ್ಸಿನ ವಿಷಯವನ್ನು ತಿಳಿದುಕೊಳ್ಳಬೇಕಾದರೂ ಇದೇ ವಿಧಾನವನ್ನು ಅನುಸರಿಸಬೇಕು. ನೀವು ಮನಸ್ಸನ್ನು ಏಕಾಗ್ರಮಾಡಿ ಅದರ ಮೇಲೆಯೇ ಮನಸ್ಸನ್ನು ಬಿಡಬೇಕು. ಒಬ್ಬರಿಗೂ ಮತ್ತೊಬ್ಬರಿಗೂ ಇರುವ ವ್ಯತ್ಯಾಸ ಏಕಾಗ್ರತೆಯ ತರತಮದಲ್ಲಿದೆ. ಯಾರ ಮನಸ್ಸು ಮತ್ತೊಬ್ಬರ ಮನಸ್ಸಿಗಿಂತ ಹೆಚ್ಚು ಏಕಾಗ್ರವಾಗಿದೆಯೋ ಅವರಿಗೆ ಹೆಚ್ಚು ಜ್ಞಾನ ಪ್ರಾಪ್ತವಾಗುವುದು.

ಹಿಂದೆ ಮತ್ತು ಈಗಿನ ಕಾಲದ ಮಹಾಮಹಿಮರ ಜೀವನದಲ್ಲಿ ಇಂತಹ ಅದ್ಭುತವಾದ ಮನೋ ಏಕಾಗ್ರತೆ ಇರುವುದು ಕಾಣುತ್ತದೆ. ಅವರು ಮಹಾ ಪ್ರತಿಭಾವಂತರು ಎಂದು ನೀವು ಹೇಳಬಹುದು. ಆದರೆ ಯೋಗಶಾಸ್ತ್ರವು, ಸಾಕಷ್ಟು ಪ್ರಯತ್ನಪಟ್ಟರೆ ನಾವೆಲ್ಲರೂ ಪ್ರತಿಭಾವಂತರಾಗಬಲ್ಲೆವು - ಎಂದು ಸಾರುತ್ತದೆ. ಕೆಲವರು ಈ ಜೀವನಕ್ಕೆ ಬರುವಾಗಲೇ ಹೆಚ್ಚು ಸಿದ್ದರಾಗಿ ಬಂದಿರುವುದರಿಂದ ಅವರು ಇತರರಿಗಿಂತ ಬೇಗ ಇದನ್ನು ಸಾಧಿಸುವರು. ನಾವೆಲ್ಲರೂ ಇದನ್ನು ಸಾಧಿಸಬಹುದು. ಇದೇ ಶಕ್ತಿಯು ಎಲ್ಲರಲ್ಲಿಯೂ ಇದೆ. ಈ ಉಪನ್ಯಾಸದ ವಿಷಯವೇ, ಮನಸ್ಸನ್ನು ಹೇಗೆ ಏಕಾಗ್ರಗೊಳಿಸುವುದು ಎಂಬುದು. ಯೋಗಿಗಳು ಕೆಲವು ನಿಯಮಗಳನ್ನು ಹೇಳುತ್ತಾರೆ. ನಾನು ಈ ರಾತ್ರಿ ಆ ನಿಯಮದ ಕೆಲವು ವಿಷಯಗಳನ್ನು ಹೇಳುತ್ತೇನೆ.

ಏಕಾಗ್ರತೆ ನಮಗೆ ಹಲವು ಮೂಲಗಳಿಂದ ಲಭಿಸುವುದು. ಇಂದ್ರಿಯಗಳ ಮೂಲಕವಾಗಿಯೂ ನಮಗೆ ಏಕಾಗ್ರತೆ ದೊರಕಬಲ್ಲದು. ಕೆಲವರಿಗೆ ಒಳ್ಳೆಯ ಸಂಗೀತವನ್ನು ಕೇಳಿದಾಗ ಮನಸ್ಸು ಏಕಾಗ್ರವಾಗುವುದು. ಇತರರಿಗೆ ರಮ್ಯವಾದ ದೃಶ್ಯವನ್ನು ನೋಡಿದಾಗ ಮನಸ್ಸು ಏಕಾಗ್ರವಾಗುವುದು. ಕೆಲವರು ಕಬ್ಬಿಣದ ಮೊಳೆಗಳ ಮೇಲೆ ಮಲಗಿಕೊಂಡಾಗ ಅವರ ಮನಸ್ಸು ಏಕಾಗ್ರವಾಗುವುದು. ಮತ್ತೆ ಕೆಲವರಿಗೆ ಗಟ್ಟಿಯಾದ ಜಲ್ಲಿ ಕಲ್ಲಿನ ಮೇಲೆ ಕುಳಿತುಕೊಂಡಾಗ ಮನಸ್ಸು ಏಕಾಗ್ರವಾಗುವುದು. ಇದೆಲ್ಲಾ ವಿಜ್ಞಾನಕ್ಕೆ ವಿರೋಧವಾದ ಅಸ್ವಾಭಾವಿಕವಾದ ಮಾರ್ಗವನ್ನು ಅನುಸರಿಸುವುದಾಯಿತು. ಮನಸ್ಸನ್ನು ಸ್ವಾಭಾವಿಕವಾಗಿ ಕ್ರಮೇಣ ಏಕಾಗ್ರಗೊಳಿಸುವುದೇ ವೈಜ್ಞಾನಿಕ ರೀತಿ.

ಕೆಲವರಿಗೆ ತಮ್ಮ ಕೈಯನ್ನು ಎತ್ತಿಹಿಡಿದರೆ ಏಕಾಗ್ರತೆ ಸಿಕ್ಕುವುದು. ಯಾತನೆಯು ಅವನಿಗೆ ಬೇಕಾದ ಏಕಾಗ್ರತೆಯನ್ನು ಕೊಡುವುದು. ಆದರೆ ಯಾತನೆ ಅಸ್ವಾಭಾವಿಕವಾದ ಮಾರ್ಗ.

ಬೇರೆ ಬೇರೆ ತತ್ವಜ್ಞಾನಿಗಳು ಎಲ್ಲರಿಗೂ ಅನ್ವಯಿಸುವ ಮಾರ್ಗಗಳನ್ನು ವ್ಯವಸ್ಥೆಗೊಳಿಸಿರುವರು. ಕೆಲವರು ಮನಸ್ಸಿನ ಪ್ರಜ್ಞಾತೀತ ಸ್ಥಿತಿಯನ್ನು ಮುಟ್ಟಬೇಕೆಂದು ಬಯಸುವರು. ದೇಹವು ನಮಗೆ ವಿಧಿಸಿರುವ ಮಿತಿಯನ್ನು ದಾಟಲು ಅವರು ಇಚ್ಚಿಸುವರು. ಯೋಗಿಗೆ ನೀತಿ ಆವಶ್ಯಕ. ಇದು ಅವನ ಮನಸ್ಸನ್ನು ಶುದ್ಧಗೊಳಿಸುವುದು. ಮನಸ್ಸು ಶುದ್ಧವಾದಷ್ಟೂ ಅದನ್ನು ನಿಗ್ರಹಿಸುವುದು ಸುಲಭ. ಮನಸ್ಸು ಪ್ರತಿಯೊಂದು ಆಲೋಚನೆಯನ್ನೂ ತೆಗೆದುಕೊಂಡು ಅದನ್ನು ಬಳಸಿಕೊಳ್ಳುತ್ತದೆ. ಮನಸ್ಸು ಹೆಚ್ಚು ಸ್ಥೂಲವಾದಷ್ಟೂ ಅದನ್ನು ನಿಗ್ರಹಿಸುವುದು ಕಷ್ಟ. ಅನೈತಿಕ ವ್ಯಕ್ತಿಯು ಎಂದಿಗೂ ತನ್ನ ಮನಸ್ಸನ್ನು ಕೇಂದ್ರೀಕರಿಸಿ ಮನಶ್ಶಾಸ್ತ್ರವನ್ನು ಅಧ್ಯಯನ ಮಾಡಲಾರನು. ಅವನಿಗೆ ಪ್ರಾರಂಭದಲ್ಲಿ ಸ್ವಲ್ಪ ನಿಗ್ರಹ ಸಿಕ್ಕಬಹುದು. ಕೇಳುವುದು ಮುಂತಾದ ಶಕ್ತಿಗಳು ಬರಬಹುದು. ಈ ಶಕ್ತಿಗಳು ಕೂಡ ಅವನಿಂದ ಹೊರಟು ಹೋಗುವುವು. ನೀವು ಅವುಗಳನ್ನು ಸರಿಯಾಗಿ ಪರೀಕ್ಷಿಸಿದರೆ, ಅದ್ಭುತಶಕ್ತಿಯು ವೈಜ್ಞಾನಿಕ ತರಬೇತಿಯಿಂದ ಬರಲಿಲ್ಲವೆನ್ನುವುದು ಗೊತ್ತಾಗುವುದು. ಯಾರು ಕೇವಲ ಮಾಯಾ ಮಂತ್ರಗಳಿಂದ ಸರ್ಪಗಳನ್ನು ನಿಗ್ರಹಿಸುವರೋ ಅವರು ಸರ್ಪದ ಕೈಯಿಂದ ಕಚ್ಚಿಸಿಕೊಂಡೇ ಸಾಯುವರು. ಯಾರಿಗೆ ಯಾವುದಾದರೂ ಅತೀಂದ್ರಿಯ ಶಕ್ತಿ ಬರುವುದೋ ಅವರು ಕೊನೆಗೆ ಅದಕ್ಕೇ ಬಲಿಯಾಗುವರು. ಹಲವು ರೀತಿ ಶಕ್ತಿಯನ್ನು ಗಳಿಸುವ ಲಕ್ಷಾಂತರ ಜನರು ಭಾರತ ದೇಶದಲ್ಲಿ ಇರುವರು. ಅವರಲ್ಲಿ ಬಹುಮಂದಿ ಹುಚ್ಚರಾಗಿ ಅರಚಿಕೊಂಡೇ ಸಾಯುವರು. ಅನೇಕ ಜನ ಬುದ್ಧಿ ಸ್ಥಿಮಿತವಿಲ್ಲದೆ ಇರುವುದರಿಂದ ಆತ್ಮಹತ್ಯೆ ಮಾಡಿಕೊಳ್ಳುವರು.

ಅಪಾಯವಿಲ್ಲದಂತೆ ಅಧ್ಯಯನವನ್ನು ಮಾಡಬೇಕು. ಅದು ವೈಜ್ಞಾನಿಕವಾಗಿರಬೇಕು, ಶಾಂತವಾಗಿರಬೇಕು. ಚಾರಿತ್ರ್ಯ ಶುದ್ಧವಾಗಿರಬೇಕು. ಇದೇ ಮೊದಲನೆ ನಿಯಮ. ಇಂತಹ ಮನುಷ್ಯ ದೇವತೆಗಳು ತನಗೆ ಗೋಚರವಾಗಬೇಕೆಂದು ಕೇಳಿಕೊಂಡರೆ ಅವರು ಗೋಚರವಾಗುವರು. ನಮ್ಮ ಮನಶ್ಶಾಸ್ತ್ರ ಮತ್ತು ತತ್ತ್ವಶಾಸ್ತ್ರದ ಅಧ್ಯಯನಕ್ಕೆ ಅವಶ್ಯವಾದುದೇ ಚಾರಿತ್ರ್ಯ ಶುದ್ದಿ. ಇದೇ ಅತ್ಯಂತ ಮುಖ್ಯವಾದುದು. ಹೀಗೆಂದರೆ ಏನು ಎಂಬುದನ್ನು ಕುರಿತು ಯೋಚಿಸಿ ನೋಡಿ. ಯಾರಿಗೂ ಹಿಂಸೆಯನ್ನು ಕೊಡಕೂಡದು; ಪರಿಶುದ್ಧವಾಗಿರುವುದು ಮತ್ತು ತಪಸ್ಸು ಇವು ಅತ್ಯಂತ ಆವಶ್ಯಕ. ಒಬ್ಬ ಇವನ್ನೆಲ್ಲ ಪೂರ್ಣವಾಗಿ ಪಡೆದುಕೊಂಡ ಎಂದರೆ ಅವನಿಗೆ ಇನ್ನೇನು ಬೇಕು; ಅವನಿಗೆ ಯಾರ ಮೇಲೆಯೂ ದ್ವೇಷಭಾವನೆ ಇಲ್ಲದೇ ಇದ್ದರೆ ಎಲ್ಲಾ ಪ್ರಾಣಿಗಳೂ ಅವನೆದುರಿಗೆ ದ್ವೇಷಭಾವನೆಯನ್ನು ತ್ಯಜಿಸುವುವು. ಯೋಗಿಗಳು ಬಹಳ ಕಠಿಣವಾದ ನಿಯಮವನ್ನು ಹಾಕಿಕೊಳ್ಳುವರು. ಯಾರೂ ಉದಾರಿಗಳಾಗಿರದೆ ತಾವು ದೊಡ್ಡ ಉದಾರಿಗಳೆಂದು ನಟಿಸಿಕೊಂಡು ಹೋಗುವುದಕ್ಕೆ ಆಗುವುದಿಲ್ಲ.

ನೀವೇನಾದರೂ ನನ್ನನ್ನು ನಂಬುವ ಹಾಗೆ ಇದ್ದರೆ ನಾನೊಂದು ಸಂಗತಿಯನ್ನು ಹೇಳುತ್ತೇನೆ. ಹಿಂದೆ ನಾನೊಂದು ಗುಹೆಯಲ್ಲಿ ವಾಸಮಾಡುತ್ತಿದ್ದ ಮನುಷ್ಯನನ್ನು ಕಂಡಿದ್ದೆ. ಅವನೊಡನೆ ಕಪ್ಪೆಗಳು ಮತ್ತು ಹಾವುಗಳು ಒಟ್ಟಿಗೆ ಇದ್ದವು. ಅವನು ಕೆಲವು ವೇಳೆ ಹಲವು ದಿನಗಳು, ಹಲವು ತಿಂಗಳು ಉಪವಾಸಮಾಡಿ ಹೊರಗೆ ಬರುತ್ತಿದ್ದನು. ಅವನು ಯಾವಾಗಲೂ ಮೌನವಾಗಿರುತ್ತಿದ್ದನು.

ನನ್ನ ಹಿರಿಯ ಗುರುಗಳು ಹೇಳುತ್ತಿದ್ದರು: ಹೃದಯಕಮಲ ಅರಳಿದರೆ ದುಂಬಿಗಳು ತಮಗೆ ತಾವೇ ಬರುತ್ತವೆ ಎಂದು. ಇಂತಹ ಜನರು ಇಂದಿಗೂ ಇರುವರು. ಅವರು ಮಾತನಾಡಬೇಕಾಗಿಲ್ಲ. ಯಾರಲ್ಲಿ ಒಂದು ದ್ವೇಷದ ಭಾವನೆಯೂ ಇಲ್ಲದೆ ಪರಿಪೂರ್ಣನಾಗಿರುವನೋ, ಅವನ ಸಮ್ಮುಖದಲ್ಲಿ ಪ್ರಾಣಿಗಳೆಲ್ಲ ತಮ್ಮ ವೈರದ ಸ್ವಭಾವವನ್ನು ತ್ಯಜಿಸುವುವು. ಇದರಂತೆಯೇ ಪಾವಿತ್ರ್ಯ. ನಾವು ಇತರರೊಂದಿಗೆ ವ್ಯವಹರಿಸುವಾಗ ಇದು ಆವಶ್ಯಕ. ನಾವು ಎಲ್ಲರನ್ನೂ ಪ್ರೀತಿಸಬೇಕು, ನಾವು ಇತರರ ದೋಷದ ಕಡೆಗೆ ಗಮನವನ್ನು ಕೊಡಕೂಡದು. ಇದರಿಂದ ನಮಗೆ ಏನೂ ಪ್ರಯೋಜನವಾಗುವುದಿಲ್ಲ. ನಾವು ಅವುಗಳನ್ನು ಕುರಿತು ಚಿಂತಿಸಲೂ ಕೂಡದು. ನಾವು ಒಳ್ಳೆಯದರ ಕಡೆಗೆ ಮಾತ್ರ ಗಮನವನ್ನು ಕೊಡಬೇಕು. ನಾವು ಇರುವುದು ಮತ್ತೊಬ್ಬರ ತಪ್ಪನ್ನು ನೋಡುವುದಕ್ಕೆ ಅಲ್ಲ. ಒಳ್ಳೆಯವರಾಗಬೇಕು, ಅದೇ ನಮ್ಮ ಉದ್ದೇಶ.

ಇಲ್ಲಿ ಒಬ್ಬಳು ಮಿಸ್...ಬರುವಳು. ತಾನು ಯೋಗಿಯಾಗಬೇಕೆಂದು ಹೇಳುವಳು. ಅವಳು ಈ ವಿಷಯವನ್ನು ಇಪ್ಪತ್ತು ಸಲ ಹೇಳುವಳು. ಐವತ್ತು ದಿನಗಳು ಧ್ಯಾನಮಾಡುವಳು. ಅನಂತರ ಈ ಧರ್ಮದಲ್ಲಿ ಏನೂ ಇಲ್ಲ, ಇದನ್ನು ಪರೀಕ್ಷಿಸಿದೆ, ಆದರೂ ಏನೂ ಪ್ರಯೋಜನವಾಗಲಿಲ್ಲ ಎನ್ನುವಳು.

ಆಧ್ಯಾತ್ಮಿಕ ಜೀವನದ ತಳಹದಿಯೇ ಅಲ್ಲಿ ಇಲ್ಲ. ಪರಿಶುದ್ಧವಾದ ಚಾರಿತ್ರ್ಯವೇ ತಳಹದಿ. ಇದೇ ಬಂದಿರುವ ದೊಡ್ಡ ಕಷ್ಟ.

ನಮ್ಮ ದೇಶದಲ್ಲಿ ಕೇವಲ ಸಸ್ಯಾಹಾರಿಗಳ ಗುಂಪು ಇದೆ. ಅವರು ಬೆಳಿಗ್ಗೆ ಹೊತ್ತಿಗೆ ಮುಂಚೆಯೆ ಎದ್ದು ಪೌಂಡುಗಟ್ಟಲೆ ಸಕ್ಕರೆಯನ್ನು ಕೈಯಲ್ಲಿ ತೆಗೆದುಕೊಂಡು ಹೋಗಿ ಅದನ್ನು ಇರುವೆ ಗೂಡಿನ ಹತ್ತಿರ ಇಡುವರು. ಇವರಲ್ಲಿ ಒಬ್ಬನು ಹಾಗೆ ಸಕ್ಕರೆ ಇಡುತ್ತಿದ್ದಾಗ, ಯಾರೋ ತನ್ನ ಕಾಲನ್ನು ಇರುವೆಯ ಮೇಲೆ ಇಟ್ಟನು. ಸಕ್ಕರೆ ಹಾಕುವವನು, 'ಪಾಪಿ, ನೀನು ಇರುವೆಯನ್ನು ಕೊಂದುಹಾಕಿದೆ' ಎಂದು ಅವನಿಗೆ ಬಲವಾಗಿ ಹೊಡೆದನು. ಇದರಿಂದ ಪೆಟ್ಟು ತಿಂದವನು ಸತ್ತು ಹೋದನು!

ಬಾಹ್ಯಶುದ್ದಿ ಬಹಳ ಸುಲಭ. ಪ್ರಪಂಚವೆಲ್ಲ ಅದನ್ನು ಪಡೆಯಲು ಧಾವಿಸುವುದು. ಯಾವುದಾದರೂ ಒಂದು ವಿಧವಾದ ಉಡುಪನ್ನು ಹಾಕಿಕೊಳ್ಳುವುದು ಧರ್ಮವಾದರೆ, ಇದನ್ನು ಯಾವ ಮೂಢನಾದರೂ ಮಾಡಬಹುದು. ಆದರೆ ನಾವು ಮನಸ್ಸನ್ನು ಗೆಲ್ಲವುದೇ ಕಷ್ಟವಾದ ಕೆಲಸ.

ಯಾರು ಬಾಹ್ಯಾಚಾರಗಳನ್ನು ಅನುಸರಿಸುವರೋ ಅವರು ದೊಡ್ಡ ಮಹಾತ್ಮರಂತೆ ಕಾಣುವರು!ನಾನು ಹುಡುಗನಾಗಿದ್ದಾಗ ಏಸುಕ್ರಿಸ್ತನ ಮೇಲೆ ನನಗೆ ಬಹಳ ಗೌರವವಿತ್ತು. ಬೈಬಲ್ಲಿನಲ್ಲಿ ಬರುವ ಮದುವೆಯ ಊಟದ ಬಗ್ಗೆ ಓದಿದೆ. ಅವನು ಅಲ್ಲಿ ಮಾಂಸ ತಿಂದಿದ್ದು, ಮದ್ಯವನ್ನು ಕುಡಿದಿದ್ದು ಬರುವುದು. ಅವನೊಬ್ಬ ಒಳ್ಳೆಯ ಮನುಷ್ಯನಾಗಿರಲಾರ ಎಂದು ಪುಸ್ತಕವನ್ನು ಮುಚ್ಚಿದೆ.

ನಾವು ಯಾವಾಗಲೂ ಮುಖ್ಯವಾದ ಅಂಶಗಳನ್ನು ಮರೆಯುವೆವು. ತಿನ್ನುವುದು, ಉಡುವುದು ಇವೇ ಮುಖ್ಯ ನಮಗೆ! ಪ್ರತಿಯೊಬ್ಬ ಮೂಢನಿಗೂ ಇದು ಸಾಧ್ಯ. ಯಾವುದು ಅತೀತವಾಗಿರುವುದೋ ಅದನ್ನು ನೋಡುವವರು ಯಾರು? ನಮಗೆ ಬೇಕಾಗಿರುವುದು ಮನಸ್ಸನ್ನು ಸುಸಂಸ್ಕೃತಗೊಳಿಸುವುದು. ಭಾರತ ದೇಶದಲ್ಲಿ ಕೆಲವರು ದಿನಕ್ಕೆ ಇಪ್ಪತ್ತು ವೇಳೆ ಸ್ನಾನಮಾಡುವುದನ್ನು ನೋಡುತ್ತೇವೆ. ತಮ್ಮನ್ನು ಬಾಹ್ಯವಾಗಿ ಬಹಳ ಚೊಕ್ಕಟವಾಗಿಟ್ಟುಕೊಂಡಿರುವರು. ಅವರು ಯಾರನ್ನೂ ಮುಟ್ಟುವುದಿಲ್ಲ. ಇದು ಕೇವಲ ಸ್ಥೂಲವಾದ ಬಾಹ್ಯ ವಿಷಯ ಮಾತ್ರ. ಸ್ನಾನಮಾಡುವುದರಿಂದಲೇ ಒಬ್ಬನು ಪರಿಶುದ್ದನಾಗಿರುವ ಹಾಗಿದ್ದರೆ ಮೀನುಗಳೆ ಪ್ರಪಂಚದಲ್ಲಿ ಅತ್ಯಂತ ಪರಿಶುದ್ದ ಪ್ರಾಣಿಗಳು.

ಸ್ನಾನ ಬಟ್ಟೆ ಆಹಾರ ಇವುಗಳೆಲ್ಲಾ ಆಧ್ಯಾತ್ಮಿಕ ಜೀವನಕ್ಕೆ ಪೋಷಕವಾದಾಗ ಇವುಗಳಿಗೆಲ್ಲಾ ಒಂದು ಸ್ಥಾನವಿದೆ. ಆಧ್ಯಾತ್ಮಿಕ ಜೀವನ ಮುಖ್ಯ. ಉಳಿದುವೆಲ್ಲಾ ಅದಕ್ಕೆ ಸಹಾಯಕವಾಗಿರುವುವು. ಅದಲ್ಲದೇ ಸುಮ್ಮನೇ ಸಸ್ಯಾಹಾರದಿಂದಲೇ ಪ್ರಯೋಜನವಿಲ್ಲ. ಸರಿಯಾಗಿ ಅದನ್ನು ತಿಳಿದುಕೊಂಡರೆ ಆಹಾರ ನಮಗೆ ಸಹಾಯಮಾಡುವುದು, ಸರಿಯಾಗಿ ತಿಳಿದುಕೊಳ್ಳದೇ ಇದ್ದರೆ ಆಧ್ಯಾತ್ಮಿಕ ಜೀವನಕ್ಕೆ ಹಾನಿಯಾಗುವುದು.

ಆದಕಾರಣವೇ ನಾನು ನಿಮಗೆ ಈ ವಿಷಯಗಳನ್ನು ವಿವರಿಸುತ್ತಿರುವುದು. ಮೊದಲನೆಯದಾಗಿ ಮೂಢರು ಯಾವುದನ್ನಾದರೂ ಆಚರಿಸಿದರೆ ಕ್ರಮೇಣ ಅದು ಅಧೋಗತಿಗೆ ಬರುವುದು. ಸೀಸೆಯಲ್ಲಿರುವ ಕರ್ಪೂರ ಆವಿಯಾಗಿ ಹೋಗಿದೆ. ನಾವು ಈಗ ಬರಿ ಸೀಸೆಗೆ ಹೋರಾಡುತ್ತಿರುವೆವು.

ಮತ್ತೊಂದು ವಿಷಯ. ಯಾವಾಗ ಇದು ಸರಿ ಅದು ತಪ್ಪು ಎನ್ನುವೆವೋ ಆಗ ಅಧ್ಯಾತ್ಮ ಮಾಯವಾಗುವುದು. ಜಗಳವೆಲ್ಲಾ ಬಾಹ್ಯಾಚಾರದಲ್ಲಿ ಮಾತ್ರ. ಒಳಗಿನ ತಿರುಳಿಗಾಗಲ್ಲ. ಬೌದ್ಧರು ಹಲವು ಕಾಲದವರೆಗೆ ಬಹಳ ಸುಂದರವಾದ ಬೋಧನೆಯನ್ನು ಕೊಟ್ಟರು. ಕ್ರಮೇಣ ಅವರಲ್ಲಿದ್ದ ಅಧ್ಯಾತ್ಮ ಮಾಯವಾಯಿತು. ಇದರಂತೆಯೇ ಕ್ರೈಸ್ತಧರ್ಮ ಕೂಡ. ಅನಂತರ ಮೂರು ದೇವರಲ್ಲಿ ಒಂದು ದೇವರು ನಿಜವೇ ಅಥವಾ ಒಂದು ದೇವರು ಮೂರು ಭಾಗವಾಗಿರುವನೇ ಎಂಬ ವಿಷಯದಲ್ಲಿ ಹೋರಾಟ ಪ್ರಾರಂಭವಾಯಿತು. ಯಾರಿಗೂ ದೇವರ ಬಳಿಗೆ ಹೋಗಿ ಇದನ್ನು ತಿಳಿದುಕೊಳ್ಳಲು ಮನಸ್ಸಿರಲಿಲ್ಲ.

ಈ ವಿವರಣೆಯಾದ ಮೇಲೆ ಆಸನ ಬರುವುದು. ಮನಸ್ಸನ್ನು ಏಕಾಗ್ರ ಮಾಡಬೇಕಾದರೆ ಯಾವುದಾದರೊಂದು ಆಸನ ಅತ್ಯಾವಶ್ಯಕ. ಯಾವ ಆಸನದಲ್ಲಿ ಅವನು ಸುಖವಾಗಿ ಕುಳಿತುಕೊಳ್ಳುವನೋ ಅದು ಅವನಿಗೆ ಒಳ್ಳೆಯ ಆಸನ. ಒಂದು ನಿಯಮವೇನೆಂದರೆ ಬೆನ್ನುಮೂಳೆ ಸರಾಗವಾಗಿ ನಿಂತಿರಬೇಕು. ಅದರ ಮೇಲೆ ದೇಹದ ಭಾರವನ್ನೆಲ್ಲಾ ಹೇರಬಾರದು. ನಾವು ಹೇಗೆ ಕುಳಿತುಕೊಳ್ಳುತ್ತೇವೆ ಎಂಬುದನ್ನು ಗಮನದಲ್ಲಿಡಬೇಕು. ಬೆನ್ನು ಮೂಳೆಯ ಮೇಲೆ ಭಾರವಿಲ್ಲದೆ ಇರುವ ಯಾವ ಆಸನದಲ್ಲಿ ಬೇಕಾದರೂ ಕುಳಿತುಕೊಳ್ಳಿ.

ಅನಂತರವೇ ಪ್ರಾಣಾಯಾಮ. ಉಸಿರಾಡುವುದಕ್ಕೆ ಬೇಕಾದಷ್ಟು ಪ್ರಾಮುಖ್ಯವನ್ನು ಕೊಟ್ಟಿರುವರು. ನಾನು ನಿಮಗೆ ಏನನ್ನು ಹೇಳುವೆನೊ ಅದು ಯಾವುದೋ ಒಂದು ಪಂಥದವರ ಅಭಿಪ್ರಾಯವಲ್ಲ. ಇದು ಎಲ್ಲರಿಗೂ ಅನ್ವಯಿಸುವಂತಹುದು. ನೀವು ಈ ದೇಶದಲ್ಲಿ ಮಕ್ಕಳಿಗೆ ಹೇಗೆ ಕೆಲವು ಪ್ರಾರ್ಥನೆಗಳನ್ನು ಕಲಿಸುತ್ತೀರೊ ಹಾಗೆ ಭಾರತ ದೇಶದಲ್ಲಿ ಮಕ್ಕಳಿಗೆ ಕೆಲವು ವಿಷಯಗಳನ್ನು ಹೇಳುವರು.

ಭರತಖಂಡದಲ್ಲಿ ಮಕ್ಕಳಿಗೆ ಕೆಲವು ಪ್ರಾರ್ಥನೆಗಳು ವಿನಃ ಬೇರೆ ಯಾವ ಧರ್ಮವನ್ನೂ ಕಲಿಸುವುದಿಲ್ಲ. ಅವರು ಆಧ್ಯಾತ್ಮಿಕ ಜೀವನದ ವಿಷಯಗಳನ್ನು ಕಲಿಯುವುದಕ್ಕೆ ಯಾರಾದರೂ ಒಬ್ಬ ಗುರುಗಳನ್ನು ಹುಡುಕಿಕೊಂಡು ಹೋಗುವರು. ಅವರು ಅನೇಕರ ಹತ್ತಿರ ಹೋಗಿ ಅವರಲ್ಲಿ ಯಾರೋ ಒಬ್ಬರನ್ನು ಈತ ತಮಗೆ ಸಹಾಯಮಾಡಬಲ್ಲನು ಎಂದು ಆರಿಸಿಕೊಳ್ಳುವರು. ಅವನಿಂದ ಉಪದೇಶ ಪಡೆಯುವರು. ನನಗೇನಾದರೂ ಮದುವೆಯಾಗಿದ್ದರೆ ನನ್ನ ಹೆಂಡತಿ ಒಬ್ಬ ಗುರುವನ್ನು ಆರಿಸಿಕೊಳ್ಳುವಳು, ಮಗ ಒಬ್ಬ ಗುರುವನ್ನು ಆರಿಸಿಕೊಳ್ಳುವನು. ಇದು ಯಾವಾಗಲೂ ಗುರುಶಿಷ್ಯರ ಮಧ್ಯೆ ಇರುವ ರಹಸ್ಯ. ಹೆಂಡತಿಯ ಇಷ್ಟದೇವತೆಯನ್ನು ಗಂಡ ತಿಳಿದುಕೊಳ್ಳಬೇಕಾಗಿಲ್ಲ. ಅವಳ ಇಷ್ಟದೇವತೆ ಯಾವುದು ಎಂದು ಅವನು ಕೂಡ ಕೇಳುವುದಕ್ಕೆ ಇಚ್ಛೆಪಡುವುದಿಲ್ಲ. ಇದನ್ನು ಯಾರಿಗೂ ಹೇಳಕೂಡದು. ಇದು ಶಿಷ್ಯನಿಗೆ ಮತ್ತು ಆತನ ಗುರುವಿಗೆ ಮಾತ್ರ ಗೊತ್ತು. ಕೆಲವುವೇಳೆ ಯಾವುದು ಒಬ್ಬನಿಗೆ ಹಾಸ್ಯಾಸ್ಪದವಾಗಿರುವುದೋ ಅದೇ ಮತ್ತೊಬ್ಬನಿಗೆ ಯೋಗ್ಯವಾದ ಬೋಧನೆಯಾಗಿರುತ್ತದೆ. ಪ್ರತಿಯೊಬ್ಬನೂ ತನ್ನದೇ ಹೊರೆಯನ್ನು ಹೊರುತ್ತಿರುವನು. ಅವರವರಿಗೆ ಯೋಗ್ಯವಾದ ಸಲಹೆಯನ್ನು ನಾವು ಯಾವಾಗಲೂ ಕೊಡಬೇಕು. ಇದು ಗುರು ಶಿಷ್ಯ ಮತ್ತು ದೇವರಿಗೆ ಸಂಬಂಧಪಟ್ಟ ವ್ಯವಹಾರ. ಆದರೆ ಗುರುಗಳೆಲ್ಲ ಬೋಧಿಸುವ ಕೆಲವು ಸಾಮಾನ್ಯ ವಿಷಯಗಳಿವೆ. ಪ್ರಾಣಾಯಾಮ ಮತ್ತು ಧ್ಯಾನ ಇವು ಸಾಮಾನ್ಯವಾಗಿರುವುವು. ಭರತಖಂಡದಲ್ಲಿ ಪೂಜೆ ಎಂದರೆ ಇದೇ.

ಗಂಗಾನದಿಯ ತೀರದಲ್ಲಿ ಗಂಡಸರು ಹೆಂಗಸರು ಮಕ್ಕಳು ಎಲ್ಲರೂ ಕುಳಿತುಕೊಂಡು ಪ್ರಾಣಾಯಾಮ ಮಾಡುತ್ತ ಧ್ಯಾನವನ್ನು ಮಾಡುತ್ತಿರುವುದು ಕಾಣುವುದು. ಅವರಿಗೆ ಮಾಡುವುದಕ್ಕೆ ಇನ್ನೂ ಬೇರೆ ಕೆಲಸಗಳೂ ಇರುತ್ತವೆ. ಇದಕ್ಕಾಗಿಯೇ ಹೆಚ್ಚುಕಾಲವನ್ನು ವಿನಿಯೋಗಿಸುವುದಕ್ಕೆ ಆಗುವುದಿಲ್ಲ. ಆದರೆ ಯಾರು ಜೀವನದಲ್ಲಿ ಇದನ್ನು ಮಾತ್ರ ತಮ್ಮ ಗುರಿಯನ್ನಾಗಿ ಮಾಡಿಕೊಂಡಿರುವರೋ ಅವರು ಮತ್ತೆ ಬೇರೆ ಸಾಧನೆಗಳನ್ನೂ ಮಾಡುವರು. ಎಂಭತ್ತ ನಾಲ್ಕು, ಬೇರೆ ಬೇರೆ ಆಸನಗಳಿವೆ. ಯಾರಾದರೂ ಒಬ್ಬ ಗುರುವಿನ ನೇತೃತ್ವದಲ್ಲಿ ಸಾಧನೆಗಳನ್ನು ಮಾಡುವವರು ಪ್ರಾಣ ತಮ್ಮ ದೇಹದಲ್ಲೆಲ್ಲಾ ಸಂಚಾರಮಾಡುವುದನ್ನು ಅನುಭವಿಸುವರು.

ಅನಂತರ ಧಾರಣ ಬರುವುದು. ಧಾರಣ ಎಂದರೆ ಯಾವುದಾದರೂ ಕೆಲವು ಕೇಂದ್ರಗಳಲ್ಲಿ ಮನಸ್ಸನ್ನು ಏಕಾಗ್ರಗೊಳಿಸುವುದು.

ಹಿಂದೂ ಹುಡುಗ ಹುಡುಗಿಯರು ಉಪದೇಶವನ್ನು ಪಡೆಯುವರು. ಅವರು ಗುರುವಿನಿಂದ ಒಂದು ಮಂತ್ರವನ್ನು ಪಡೆಯುವರು. ಇದನ್ನು ಬೀಜಮಂತ್ರ ಎಂದು ಕರೆಯುವರು. ಗುರುವಿಗೆ ಅವನ ಗುರು ಇದನ್ನು ಕೊಡುತ್ತಿದ್ದ. ಈಗಿನ ಗುರು ಅದನ್ನು ತನ್ನ ಶಿಷ್ಯನಿಗೆ ಕೊಡುವನು. ಇಂತಹ ಒಂದು ಬೀಜಮಂತ್ರವೇ 'ಓಂ' ಎಂಬುದು. ಇವಕ್ಕೆಲ್ಲಾ ಬೇಕಾದಷ್ಟು ಅರ್ಥವಿರುತ್ತವೆ. ಅವರು ಅದನ್ನು ಬಹಳ ರಹಸ್ಯ ಎಂದು ಭಾವಿಸುವರು. ಅದನ್ನು ಎಂದಿಗೂ ಬರೆಯುವುದಿಲ್ಲ. ಅವರು ಅದನ್ನು ಮತ್ತೊಬ್ಬನಿಂದ ಕೇಳಿ ತಿಳಿದುಕೊಳ್ಳಬೇಕೆ ವಿನಃ ಬರವಣಿಗೆಯಿಂದಲ್ಲ. ಅನಂತರ ಆ ಮಂತ್ರವನ್ನು ದೇವರೆಂದೇ ಭಾವಿಸಬೇಕು. ಅವರು ಅನಂತರ ಅದರ ಮೇಲೆ ಧ್ಯಾನಮಾಡುವರು.

ನಾನು ಒಂದು ಸಲ ಪ್ರಾರ್ಥಿಸುತ್ತಿದ್ದೆ. ನಾಲ್ಕು ತಿಂಗಳು ಮಳೆಗಾಲವೆಲ್ಲಾ ಹಾಗೆ ಮಾಡಿದೆ. ನಾನು ಬೆಳಿಗ್ಗೆ ಅಷ್ಟು ಹೊತ್ತಿಗೇ ಎದ್ದು ನದಿಯಲ್ಲಿ ಸ್ನಾನಮಾಡಿ ಒದ್ದೆಯ ಬಟ್ಟೆಯೊಂದಿಗೆ ಸೂರ್ಯ ಮುಳುಗುವವರೆಗೆ ಮಂತ್ರ ಜಪ ಮಾಡುತ್ತಿದ್ದೆ. ಅನಂತರ ಏನಾದರೂ ಸ್ವಲ್ಪ ಆಹಾರವನ್ನು ತೆಗೆದುಕೊಳ್ಳುತ್ತಿದ್ದೆ. ಮಳೆಗಾಲದಲ್ಲಿ ಹೀಗೆ ನಾಲ್ಕು ತಿಂಗಳು ಮಾಡಿದೆ!

ನಾವು ಪಡೆಯುವುದಕ್ಕೆ ಆಗದೇ ಇರುವುದು ಈ ಪ್ರಪಂಚದಲ್ಲಿ ಯಾವುದೂ ಇಲ್ಲ ಎಂದು ಹಿಂದೂ ನಂಬುತ್ತಾನೆ. ಈ ದೇಶದಲ್ಲಿ ಜನರಿಗೆ ಹಣಬೇಕಾದರೆ ಒಂದು ಕೆಲಸಕ್ಕೆ ಸೇರಿ ಹಣವನ್ನು ಸಂಪಾದಿಸುತ್ತಾನೆ. ಅಲ್ಲಿ (ಭಾರತ ದೇಶದಲ್ಲಿ) ಯಾರಿಂದಲಾದರೂ ಒಂದು ಮಂತ್ರವನ್ನು ಪಡೆದು ಒಂದು ಮರದ ಕೆಳಗೆ ಕುಳಿತುಕೊಂಡು ಅದನ್ನು ಉಚ್ಚಾರ ಮಾಡುತ್ತಾನೆ. ಇದರಿಂದ ಹಣ ಬಂದೇ ತೀರಬೇಕು ಎಂದು ಭಾವಿಸುತ್ತಾನೆ - ತನ್ನ ಮನಸ್ಸಿನ ಶಕ್ತಿ ಈ ಪ್ರಪಂಚದಲ್ಲಿ ಎಲ್ಲವನ್ನೂ ಬರುವಂತೆ ಮಾಡುವುದು. ನೀವು ಇಲ್ಲಿ ಹಣ ಮಾಡುತ್ತೀರಿ. ನೀವು ನಿಮ್ಮ ಮನಸ್ಸಿನ ಶಕ್ತಿಯನ್ನೆಲ್ಲಾ ಹಣಮಾಡುವುದಕ್ಕೆ ಖರ್ಚು ಮಾಡುವಿರಿ.

ಹಠಯೋಗಿಗಳೆಂಬ ಒಂದು ಪಂಗಡದವರು ಇರುವರು. ದೇಹವನ್ನು ಸಾಯದಂತೆ ನೋಡಿಕೊಳ್ಳುವುದೇ ಈ ಪ್ರಪಂಚದಲ್ಲಿ ಸರ್ವಶ್ರೇಷ್ಠ ಎಂದು ಅವರು ಭಾವಿಸುವರು. ಅವರು ಮೊದಲಿನಿಂದ ದೇಹದಲ್ಲಿ ಆಸಕ್ತರು. ಅದಕ್ಕಾಗಿ ಹನ್ನೆರಡು ವರುಷ ಸಾಧನೆ ಮಾಡುವರು! ಅವರು ತಾವು ಸಣ್ಣ ಮಗುವಾಗಿದ್ದಾಗಿನಿಂದಲೇ ಪ್ರಾರಂಭ ಮಾಡುವರು. ಇಲ್ಲದೇ ಇದ್ದರೆ ಅದು ಸಾಧ್ಯವೇ ಇಲ್ಲ. ಹಠಯೋಗಿಗಳಲ್ಲಿ ಒಂದು ಬಹಳ ವಿಚಿತ್ರ ರೂಢಿಯಿದೆ. ಯೋಗಿಯು ಮೊದಲು ಶಿಷ್ಯನಾದ ಮೇಲೆ ಕಾಡಿಗೆ ಹೋಗಿ ಏಕಾಂಗಿಯಾಗಿ ಸರಿಯಾಗಿ ನಲವತ್ತು ದಿನಗಳು ಇರುವನು. ಅವನಿಗೆ ಗೊತ್ತಿರುವುದೆಲ್ಲ ಅವನು ಆ ನಲವತ್ತು ದಿನಗಳಲ್ಲಿ ಕಲಿತದ್ದು.

ಕಲ್ಕತ್ತೆಯಲ್ಲಿ ಒಬ್ಬ ತನಗೆ ಐನೂರು ವರುಷಗಳಾಗಿವೆ ಎಂದು ಹೇಳುತ್ತಾನೆ. ಅಲ್ಲಿಯ ಜನ, ತಮ್ಮ ತಾತಂದಿರು ಆತನನ್ನು ಕಂಡಿದ್ದರು ಎಂದು ಹೇಳುತ್ತಿದ್ದರು. ಆತ ಪ್ರತಿದಿನ ಇಪ್ಪತ್ತು ಮೈಲಿಗಳು ನಡೆಯುತ್ತಾನೆ. ಅದನ್ನು ಕಾಲುನಡಿಗೆ ಎನ್ನಲಾಗುವುದಿಲ್ಲ, ಓಡುವನು! ನದಿ ತೀರಕ್ಕೆ ಹೋಗಿ ತಲೆಯಿಂದ ಕಾಲಿನವರೆಗೆ ಮಣ್ಣಿನಲ್ಲಿ ಮುಚ್ಚಿಕೊಳ್ಳುವನು. ಇದಾದಮೇಲೆ ಅವನು ನದಿಗೆ ಬೀಳುವನು. ಅನಂತರ ಮಣ್ಣಿನಲ್ಲಿ ಮುಚ್ಚಿಕೊಳ್ಳುವನು. ಇದರಿಂದ ಯಾವ ಪ್ರಯೋಜನವೂ ನನಗೆ ಕಾಣುವುದಿಲ್ಲ. ಹಾವುಗಳು ಇನ್ನೂರು ವರುಷಗಳು ಬದುಕುತ್ತವೆ ಎನ್ನುತ್ತಾನೆ. ಆತನಿಗೆ ಈಗ ಬಹಳ ವಯಸ್ಸಾಗಿರಬೇಕು. ಏಕೆಂದರೆ ನಾನು ಕಳೆದ ಹದಿನಾಲ್ಕು ವರುಷಗಳಿಂದ ಭರತಖಂಡದಲ್ಲಿ ಅಲೆಯುತ್ತಿರುವೆನು. ನಾನು ಹೋದ ಕಡೆಯಲ್ಲೆಲ್ಲಾ ಜನರಿಗೆ ಅವನ ಪರಿಚಯವಿದೆ. ಅವನು ಯಾವಾಗಲೂ ಸಂಚರಿಸುತ್ತಿರುವನು. ಹಠಯೋಗಿ ಎಂಟು ಅಂಗುಲ ಉದ್ದವಿರುವ ರಬ್ಬರನ್ನು ನುಂಗಿ ಅದನ್ನು ಪುನಃ ಹೊರಗೆ ತೆಗೆಯುವನು. ದಿನಕ್ಕೆ ನಾಲ್ಕು ವೇಳೆ ಅವನು ದೇಹದ ಒಳಗೆ ಮತ್ತು ಹೊರಗೆ ಎಲ್ಲಾ ಭಾಗಗಳನ್ನೂ ತೊಳೆಯಬೇಕಾಗುವುದು.

ಗೋಡೆಗಳು ಬೇಕಾದರೆ ಸಾವಿರಾರು ವರುಷಗಳು ಇರುತ್ತವೆ. ಇದರಿಂದ ಏನು ಪ್ರಯೋಜನ? ನನಗೆ ಇಷ್ಟು ದೀರ್ಘಕಾಲ ಬದುಕಲು ಇಚ್ಛೆ ಇಲ್ಲ. "ನಮಗೆ ಇಲ್ಲಿಯವರೆಗೆ ಬಂದ ಕಷ್ಟನಷ್ಟಗಳೇ ಸಾಲವೆ?” ಎಲ್ಲಾ ದುಃಖ ಮತ್ತು ವ್ಯಾಮೋಹಗಳಿಂದ ಒಳಗೊಂಡ ಒಂದು ದೇಹವೇ ಸಾಕಾಗಿದೆ.

ಇನ್ನೂ ಬೇರೆ ಪಂಗಡಗಳಿವೆ. ನಿಮಗೆ ಅವರು ಒಂದು ಅಮೃತ ಬಿಂದುವನ್ನು ಕೊಡುವರು. ನೀವು ಅನಂತರ ಎಂದೆಂದಿಗೂ ಯುವಕರಾಗಿಯೇ ಇರುವಿರಿ! ಇಂತಹ ಹಲವು ಪಂಗಡಗಳ ವಿಷಯವನ್ನೆಲ್ಲಾ ಹೇಳಬೇಕಾದರೆ ತಿಂಗಳುಗಳು ಹಿಡಿಯುವುದು. ಅವರೆಲ್ಲಾ ದೇಹದ ಕಡೆಗೇ ದೃಷ್ಟಿಯನ್ನೆಲ್ಲಾ ಹರಿಸಿರುವರು. ಪ್ರತಿದಿನವೂ ಒಂದೊಂದು ಹೊಸ ಪಂಗಡ ಹುಟ್ಟುತ್ತಿರುತ್ತದೆ.

ಈ ಪಂಗಡಗಳ ಶಕ್ತಿಯೆಲ್ಲಾ ಮನಸ್ಸಿನಲ್ಲಿದೆ: ಅವರು ಮನಸ್ಸನ್ನು ಬಿಗಿ ಹಿಡಿಯುವರು. ಮೊದಲು ಅದನ್ನು ಏಕಾಗ್ರಮಾಡಿ ಯಾವುದಾದರೂ ವಸ್ತುವಿನ ಮೇಲೆ ಕೇಂದ್ರೀಕರಿಸಬೇಕು. ಬೆನ್ನೆಲುಬಿನ ಮೇಲೆಯೋ ಅಥವಾ ಯಾವುದಾದರೂ ನರಗಳ ಮೇಲೆಯೋ ಏಕಾಗ್ರಮಾಡಿ ಎಂದು ಹೇಳುವರು. ನರಗಳ ಕೇಂದ್ರದ ಮೇಲೆ ಮನಸ್ಸನ್ನು ಏಕಾಗ್ರಮಾಡಿದರೆ ಅವರಿಗೆ ದೇಹದ ಮೇಲೆ ಶಕ್ತಿ ಬರುವುದು. ಅವನ ಶಾಂತಿಗೆ ದೇಹ ಒಂದು ದೊಡ್ಡ ಕಂಟಕ. ಅವನ ಆದರ್ಶಕ್ಕೆ ವಿರೋಧವಾಗಿದೆ. ಆದ ಕಾರಣವೇ ಅವನು ದೇಹವನ್ನು ನಿಗ್ರಹಿಸುವುದು, ಸೇವಕನಂತೆ ದೇಹವನ್ನು ಇಟ್ಟುಕೊಂಡಿರುವುದು.

ಅನಂತರ ಧ್ಯಾನ ಬರುವುದು. ಅದೇ ಶ್ರೇಷ್ಠವಾದ ಅವಸ್ಥೆ. ಮನಸ್ಸು ಸಂದೇಹ ಸ್ಥಿತಿಯಲ್ಲಿದ್ದರೆ ಅದು ಅದರ ಶ್ರೇಷ್ಠಾವಸ್ಥೆಯಲ್ಲ. ಧ್ಯಾನವೇ ಅದರ ಶ್ರೇಷ್ಠ ಅವಸ್ಥೆ. ಆಗ ಮನಸ್ಸು ಕೇವಲ ವಸ್ತುಗಳನ್ನು ನೋಡುವುದು, ಅದರೊಂದಿಗೆ ಬೆರೆಯುವುದಿಲ್ಲ. ಎಲ್ಲಿಯವರೆಗೆ ನಾನು ನೋವನ್ನು ಅನುಭವಿಸುತ್ತೇನೆಯೋ ಅಲ್ಲಿಯವರೆಗೆ ನಾನು ದೇಹದೊಂದಿಗೆ ಸಂಬಂಧವನ್ನು ಇಟ್ಟುಕೊಂಡಿರುವೆನು. ಸುಖವಾದರೂ ಆಗಲಿ, ಸಂತೋಷವಾದರೂ ಆಗಲಿ, ಆಗಲೂ ದೇಹದೊಂದಿಗೆ ಸಂಬಂಧವನ್ನು ಇಟ್ಟುಕೊಂಡಿರುವೆನು. ಶ್ರೇಷ್ಠ ಅವಸ್ಥೆಯಲ್ಲಿ ಸುಖದುಃಖಗಳನ್ನು ಒಂದೇ ಸಮನಾಗಿ ನೋಡುವೆನು. ಪ್ರತಿಯೊಂದು ಧ್ಯಾನವೂ ನೇರವಾದ ಒಂದು ಪ್ರಜ್ಞಾತೀತ ಸ್ಥಿತಿ. ಪರಿಪೂರ್ಣವಾದ ಏಕಾಗ್ರತೆಯಲ್ಲಿ ಆತ್ಮವು ಸ್ಥೂಲದೇಹದ ಬಂಧನದಿಂದ ಪಾರಾಗಿ ತನ್ನ ನೈಜಸ್ಥಿತಿಯಲ್ಲಿ ಇರುವುದು. ಕೋರಿದುದು ಅವನಿಗೆ ಆಗ ಪ್ರಾಪ್ತವಾಗುವುದು. ಶಕ್ತಿ ಮತ್ತು ಜ್ಞಾನ ಆಗಲೇ ಅಲ್ಲಿವೆ. ಆತ್ಮವು ಶಕ್ತಿಹೀನವಾದ ದೇಹದೊಂದಿಗೆ ತಾದಾತ್ಮ್ಯ ಭಾವವನ್ನು ಬೆಳೆಸಿ ವ್ಯಥೆಪಡುವುದು. ಮರ್ತ್ಯವಾದ ಕೆಲವು ಆಕಾರಗಳೊಡನೆ ಅದು ತನ್ನನ್ನು ತಾನು ಒಂದಾಗಿಸಿಕೊಳ್ಳುವುದು. ಆದರೆ ಮುಕ್ತಾತ್ಮ ಯಾವುದಾದರೂ ಶಕ್ತಿಯನ್ನು ಗಳಿಸಬೇಕಾದರೆ ಅದು ಅದಕ್ಕೆ ದೊರೆಯುವುದು. ಅದಕ್ಕೆ ಇಚ್ಛೆ ಇಲ್ಲದೇ ಇದ್ದರೆ ಅದು ಬರುವುದಿಲ್ಲ. ದೇವರನ್ನು ಅರಿತವನು ದೇವರಾಗುವನು. ಮುಕ್ತಾತ್ಮನಿಗೆ ಅಸಾಧ್ಯವೆಂಬುದಿಲ್ಲ. ಇನ್ನು ಮೇಲೆ ಅವನಿಗೆ ಜನನ ಮರಣಗಳಿಲ್ಲ. ಅವನು ಎಂದೆಂದಿಗೂ ಮುಕ್ತ.

