
\chapter[ಭಕ್ತ ಮತ್ತು ಭಗವಂತ]{ಭಕ್ತ ಮತ್ತು ಭಗವಂತ\protect\footnote{\engfoot{C.W, Vol. VI, P. 49}}}

\begin{center}
(೧೯೦೦ರ ಏಪ್ರಿಲ್ ೯ರಂದು ಸ್ಯಾನ್‌ಫ್ರಾನ್ಸಿಸ್ಕೋ ಪ್ರಾಂತದಲ್ಲಿ ನೀಡಿದ ಉಪನ್ಯಾಸ)
\end{center}

ಇದುವರೆಗೆ ನಾವು ಮಾನವ ಸ್ವಭಾವದಲ್ಲಿ ಹೆಚ್ಚು ವಿಶ್ಲೇಷಣಾತ್ಮಕ ಅಂಶವನ್ನು ಕುರಿತು ಪರ್ಯಾಲೋಚಿಸುತ್ತಿದ್ದೆವು. ಈ ಪ್ರವಚನದಲ್ಲಿ ಮಾನವನ ಭಾವನಾತ್ಮಕ ಮುಖವನ್ನು ಕುರಿತು ಪರ್ಯಾಲೋಚಿಸುವೆವು. ಮೊದಲನೆಯದು ಮನುಷ್ಯನನ್ನು ಅಖಂಡವಾಗಿ, ಒಂದು ತತ್ವದಂತೆ ಪರಿಶೀಲಿಸುವುದು. ಎರಡನೆಯದು ಮನುಷ್ಯನನ್ನು ಒಂದು ಸಾಂತವಾದ ವ್ಯಕ್ತಿಯಂತೆ ಪರಿಶೀಲಿಸುವುದು. ಮೊದಲನೆಯದಕ್ಕೆ ಕಂಬನಿಯ ಬಿಂದುಗಳಿಗಾಗಲಿ, ಭಾವಾತಿರೇಕಗಳಿಗಾಗಲಿ ಗಮನಕೊಡಲು ಸಮಯವಿಲ್ಲ. ಎರಡನೆಯದಾದರೂ ಕಣ್ಣೀರನ್ನು ಒರಸದೆ, ದುಃಖವನ್ನು ಶಮನ ಮಾಡದೆ ಮುಂದೆ ಹೆಜ್ಜೆಯನ್ನು ಇಡುವಂತೆಯೇ ಇಲ್ಲ. ಮೊದಲನೆಯದು ಮಹತ್ತರವಾದುದು. ಅದು ಎಷ್ಟು ಮಹತ್ತರವಾದುದು ಮತ್ತು ಭವ್ಯವಾದುದು ಎಂದರೆ ಕೆಲವು ವೇಳೆ ಅದರ ಅಸೀಮತೆಯನ್ನು ನೋಡಿಯೇ ನಾವು ಸೊರಗಿ ಹೋಗುವೆವು. ಎರಡನೆಯದಾದರೋ ಸರ್ವಸಾಮಾನ್ಯವಾದುದು. ಆದರೂ ಅದು ನಮಗೆ ಅತ್ಯಂತ ಸುಂದರವಾಗಿರುವುದು, ಪ್ರಿಯವಾಗಿರುವುದು. ಮೊದಲನೆಯದು ನಮ್ಮನ್ನು ಆಕ್ರಮಿಸಿ, ನಮ್ಮ ಶ್ವಾಸಕೋಶಗಳು ನುಚ್ಚು ನೂರಾಗುವಂತಹ ಎತ್ತರಕ್ಕೆ ನಮ್ಮನ್ನು ಒಯ್ಯುತ್ತದೆ. ನಾವು ಅಲ್ಲಿ ಉಸಿರಾಡಲಾರೆವು. ಎರಡನೆಯದಾದರೆ ನಾವು ಎಲ್ಲಿರುವೆವೋ ಅಲ್ಲಿಯೇ ನಮ್ಮನ್ನು ಬಿಡುವುದು; ಜೀವನದಲ್ಲಿರುವ ವಸ್ತುಗಳನ್ನು ನೋಡಲು ಅವಕಾಶ ಕೊಡುವುದು. ಮೊದಲನೆಯದಾದರೋ ವಿಚಾರ-ಮುದ್ರೆಯಿಲ್ಲದ ಯಾವುದನ್ನೂ ಪುರಸ್ಕರಿಸುವುದಿಲ್ಲ. ಎರಡನೆಯದಕ್ಕೆ ನಂಬಿಕೆ ಇದೆ. ಯಾವುದನ್ನು ಅದು ನೋಡಲಾರದೊ ಅದನ್ನು ನಂಬುವುದು. ನಮಗೆ ಎರಡೂ ಆವಶ್ಯಕ. ಹಕ್ಕಿ ತನ್ನ ಒಂದೇ ರೆಕ್ಕೆಯಿಂದ ಹಾರಾಡಲಾರದು.

ಹೃದಯವಂತಿಕೆ, ವಿಚಾರಶಕ್ತಿ ಮತ್ತು ಕಾರ್ಯತತ್ಪರತೆ ಈ ಮೂರೂ ಒಂದೇ ಸಮವಾಗಿ ಯಾರಲ್ಲಿ ಪಕ್ವತೆಯನ್ನು ಹೊಂದಿದೆಯೋ ಅಂತಹ ವ್ಯಕ್ತಿಯನ್ನು ನೋಡಲು ನಾವು ಬಯಸುತ್ತೇವೆ. ಈ ಪ್ರಪಂಚದ ಕಷ್ಟ ದುಃಖಗಳನ್ನು ಹೃತ್ಪೂರ್ವಕವಾಗಿ ಅನುಭವಿಸಬಲ್ಲಂತಹ ವ್ಯಕ್ತಿ ನಮಗೆ ಬೇಕು. ಕೇವಲ ಇವುಗಳನ್ನು ಅನುಭವಿಸಬಲ್ಲಂತಹ ವ್ಯಕ್ತಿ ಮಾತ್ರವಲ್ಲ, ವಸ್ತುವಿನ ಅಂತರಾಳಕ್ಕೆ ಪ್ರವೇಶಿಸಿ, ಅದನ್ನು ಚೆನ್ನಾಗಿ ಅರ್ಥಮಾಡಿಕೊಳ್ಳುವಂಥವನೂ ನಮಗೆ ಬೇಕು. ಅವನು ಇಲ್ಲಿಯೂ ನಿಲ್ಲಕೂಡದು. ತಾನು ಏನನ್ನು ಅನುಭವಿಸಿರುವನೋ, ಏನನ್ನು ತಿಳಿದುಕೊಂಡಿರುವನೊ, ಅದನ್ನು ಕಾರ್ಯಗತಗೊಳಿಸಬಲ್ಲ ವ್ಯಕ್ತಿಯೂ ನಮಗೆ ಬೇಕು. ಇಂತಹ ಭಾವ, ಯುಕ್ತಿ ಮತ್ತು ಕ್ರಿಯಾಶೀಲತೆಯ ಸಾಮರಸ್ಯ ನಮಗೆ ಬೇಕು. ಈ ಪ್ರಪಂಚದಲ್ಲಿ ಎಷ್ಟೋ ಜನ ಗುರುಗಳು ಇರುವರು. ಅವರಲ್ಲಿ ಬಹುಮಂದಿ ಏಕಮುಖದವರು. ಒಬ್ಬನು ಮುಕ್ತಿಯ ಜಾಜ್ವಲ್ಯಮಾನವಾದ ಮಧ್ಯಾಹ್ನದ ಸೂರ್ಯನನ್ನು ನೋಡುವನು; ಮತ್ತೆ ಯಾವುದನ್ನೂ ನೋಡುವುದಿಲ್ಲ. ಮತ್ತೊಬ್ಬನು ಪ್ರೇಮದ ಹೃದಯಸ್ಪರ್ಶಿಯಾದ ಗಾನವನ್ನು ಕೇಳುವನು; ಮತ್ತೆ ಯಾವುದನ್ನೂ ಕೇಳುವುದಿಲ್ಲ. ಮತ್ತೊಬ್ಬನು ಕರ್ಮದ ಚಟುವಟಿಕೆಯಲ್ಲಿ ತಲ್ಲೀನನಾಗಿ ಹೋಗಿರುವನು. ಅವನಿಗೆ ಭಾವಕ್ಕಾಗಲಿ, ವಿಚಾರಕ್ಕಾಗಲಿ ಸಮಯವೇ ಇಲ್ಲ. ಕ್ರಿಯಾಶೀಲನೂ, ಅಷ್ಟೇ ವಿಚಾರವೇತ್ತನೂ ಮತ್ತು ಭಕ್ತನೂ ಆಗಿರುವ ಮೇರುಸದೃಶ ವ್ಯಕ್ತಿಯನ್ನು ನಾವು ಏತಕ್ಕೆ ಪಡೆಯಬಾರದು? ಇದೇನು ಅಸಾಧ್ಯವೆ? ನಿಜವಾಗಿಯೂ ಅಲ್ಲ. ಅವನೇ ಭವಿಷ್ಯದ ಮಾನವ, ಸದ್ಯಕ್ಕೆ ವಿರಳರಾಗಿರುವ ಅಂತಹ ವ್ಯಕ್ತಿಗಳಿಂದ ಜಗತ್ತು ತುಂಬಿ ತುಳುಕಾಡುವಷ್ಟು ಅವರು ಮುಂದೆ ವೃದ್ಧಿಯಾಗುವರು.

ನಾನು ನಿಮ್ಮೊಡನೆ ಇಷ್ಟು ಕಾಲವೂ ಬುದ್ದಿಯ ಮತ್ತು ವಿಚಾರದ ವಿಷಯವಾಗಿ ಮಾತನಾಡುತ್ತಿದ್ದೆ. ನಾವು ವೇದಾಂತವನ್ನೆಲ್ಲ ಕೇಳಿರುವೆವು. ಮಾಯಾ ತೆರೆ ಸರಿಯುವುದು. ಚಳಿಗಾಲದ ಮೋಡಗಳು ಮಾಯವಾಗಿ ಸೂರ್ಯನ ಹೊಂಬಿಸಿಲು ನಮ್ಮ ಮೇಲೆ ಬೀಳುವುದು. ಮುಗಿಲನ್ನು ತೂರಿಹೋಗುತ್ತಿರುವ ಶಿಖರಗಳನ್ನುಳ್ಳ ಹಿಮಾಲಯ ಪರ್ವತಗಳನ್ನು ನಾನು ಹತ್ತಲು ಯತ್ನಿಸುತ್ತಿರುವೆನು. ನಾನು ಈಗ ನಿಮ್ಮೊಂದಿಗೆ\break ಬೇರೊಂದು ವಿಷಯವನ್ನು ಅಧ್ಯಯನ ಮಾಡಬೇಕೆಂದು ಬಯಸುವೆನು. ಅವೇ ಅತಿ ಸುಂದರವಾದ ಕಣಿವೆಗಳು, ಪ್ರಕೃತಿಯ ಮನೋಹರವಾದ ದೃಶ್ಯಗಳು. ಪ್ರಪಂಚದಲ್ಲಿ ಎಷ್ಟೇ ಕಷ್ಟವಿದ್ದರೂ ನಮ್ಮನ್ನು ಇಲ್ಲಿರುವಂತೆ ಮಾಡಿರುವುದೆ ಪ್ರೀತಿ. ಮನುಷ್ಯ ತಾನಾಗಿ ಪ್ರೇರಿತನಾಗಿ ದುಃಖದ ಅನಂತ ಸರಪಳಿಯಿಂದ ಬದ್ದನಾಗುವಂತೆ ಮಾಡಿದ ಆ ಪ್ರೀತಿಯ ವಿಷಯವನ್ನು ಇಲ್ಲಿ ವಿಚಾರಿಸೋಣ. ಯಾವುದಕ್ಕಾಗಿ ಇವನು ದುಃಖವನ್ನು ಅನುಭವಿಸುತ್ತಿರುವನೊ, ಆ ಅನಂತ ಪ್ರೀತಿಯ ವಿಷಯವನ್ನು ಇಲ್ಲಿ ವಿಚಾರಮಾಡೋಣ, ನಾವು ಅದರ ಮತ್ತೊಂದು ಭಾಗವನ್ನು ಮರೆಯಲು ಯತ್ನಿಸುವುದಿಲ್ಲ. ಹಿಮಾಲಯದ ನೀರ್ಗಲ್ಲಿನ ಮಹಾಪ್ರವಾಹಗಳು ಕಾಶ್ಮೀರದ ಕಣಿವೆಯಲ್ಲಿರುವ ಭತ್ತದ ಗದ್ದೆಗಳನ್ನು ಸಂಧಿಸಬೇಕಾಗಿದೆ. ಸಿಡಿಲಿನ ಭೀಮ ಘರ್ಜನೆ ಹಕ್ಕಿಗಳ ಇಂಚರದೊಂದಿಗೆ ಹೊಂದಿಕೊಳ್ಳಬೇಕಾಗಿದೆ.

ಮನೋಹರವಾದ, ಸುಂದರವಾದ, ಪ್ರತಿಯೊಂದು ವಿಷಯವನ್ನೂ ಈ ಪ್ರವಚನ ಗಮನಿಸಬೇಕಾಗಿದೆ. ಪೂಜೆ ಎಲ್ಲಾ ಕಡೆಗಳಲ್ಲಿಯೂ ಇದೆ. ಪ್ರತಿಯೊಂದು ಜೀವಿಯ ಹೃದಯಾಂತರಾಳದಲ್ಲಿಯೂ ಆ ಭಾವವಿದೆ. ಪ್ರತಿಯೊಬ್ಬನೂ ಈಶ್ವರನನ್ನು ಪೂಜಿಸುತ್ತಿರುವನು. ಪೂಜೆಯ ಆದಿ, ಸುಂದರವಾದ ಕಮಲದ ಆದಿಯಂತೆ, ಜೀವನದ ಆದಿಯಂತೆ ಪ್ರಪಂಚದ ಕೆಸರಿನಲ್ಲಿರುವುದು. ಅಲ್ಲಿ ಅಂಜಿಕೆಯ ಸ್ವಭಾವವಿರುವುದು. ಪ್ರಾಪಂಚಿಕ ವಸ್ತುಗಳನ್ನು ಪಡೆಯಬೇಕೆಂಬ ಆಕಾಂಕ್ಷೆ ಇರುವುದು. ಇದೇ ಭಿಕ್ಷುಕನ ಪೂಜೆ. ಪೂಜಾಪ್ರಪಂಚದ ಆದಿಯೆ ಇದು. ಇದು ಭಗವಂತನ ಪ್ರೀತಿಯಲ್ಲಿ, ಮಾನವನ ಮೂಲಕ ಭಗವಂತನ ಪೂಜೆಯಲ್ಲಿ ಪರ್ಯವಸಾನವಾಗುವುದು.

ದೇವರೇನಾದರೂ ಇರುವನೆ? ಪ್ರೀತಿಸುವುದಕ್ಕೆ ಯಾರಾದರೂ ಇರುವರೆ? ಪ್ರೀತಿಸುವುದಕ್ಕೆ ಯೋಗ್ಯರಾದಂತಹವರು ಯಾರಾದರೂ ಇರುವರೆ? ಸುಮ್ಮನೆ ಒಂದು ಕಲ್ಲನ್ನು ಪ್ರೀತಿಸಿದರೆ ಅದರಿಂದ ಯಾವ ಪ್ರಯೋಜನವೂ ಇಲ್ಲ. ಯಾರು ಪ್ರೀತಿಯನ್ನು ಅರ್ಥಮಾಡಿಕೊಳ್ಳಬಲ್ಲರೋ, ಯಾರು ನಮ್ಮ ಪ್ರೀತಿಯನ್ನು ಸ್ವೀಕರಿಸಬಲ್ಲರೋ, ಅವರನ್ನು ಮಾತ್ರ ನಾವು ಪ್ರೀತಿಸಬಲ್ಲೆವು. ಇದರಂತೆಯೇ ಪೂಜೆಯು ಕೂಡ. ಈ ಪ್ರಪಂಚದಲ್ಲಿ ಕಲ್ಲನ್ನು ಕಲ್ಲಿನ ಚೂರು ಎಂದು ಭಾವಿಸಿ ಪ್ರೀತಿಸಿದವರು ಯಾರೂ ಇಲ್ಲ. ಅವರು ಯಾವಾಗಲೂ ಕಲ್ಲಿನಲ್ಲಿರುವ ಸರ್ವಾಂತರ್ಯಾಮಿಯನ್ನು ಪೂಜಿಸಿದರು. ಆ ಸರ್ವಾಂತರ್ಯಾಮಿಯು ನಮ್ಮಲ್ಲಿಯೂ ಇರುವನೆಂಬುದನ್ನು ನಾವು ಕಂಡುಹಿಡಿಯುತ್ತೇವೆ. ಅವನು ನಮ್ಮಿಂದ ಬೇರೆ ಇಲ್ಲದೆ ಇದ್ದರೆ ಅವನನ್ನು ಪೂಜಿಸಲು ಸಾಧ್ಯವೆ? ನಾನು `ನಿನ್ನನ್ನು' ಮಾತ್ರ ಪೂಜಿಸಬಲ್ಲೆ, ನನ್ನನ್ನು ಪೂಜಿಸಲಾರೆ. ನಾನು `ನಿನಗೆ' ಮಾತ್ರ ಪ್ರಾರ್ಥನೆ ಸಲ್ಲಿಸಬಲ್ಲೆ; ನನಗೆ ನಾನೇ ಪ್ರಾರ್ಥನೆ ಸಲ್ಲಿಸಲಾರೆ. ನೀನೆಂಬುದು ಯಾವುದಾದರೂ ಇದೆಯೆ?

ಒಂದು ಹಲವು ಆಗುತ್ತದೆ. ನಾವು ಒಂದನ್ನು ನೋಡಿದಾಗ, ಮಾಯೆಯ ಮೂಲಕವಾಗಿ ತೋರಿದ ಹಲವುಗಳೆಲ್ಲ ಹೋಗಬಹುದು. ಆದರೆ ಆ ಹಲವು ನಿಷ್ಟ್ರಯೋಜಕವೆಂದಲ್ಲ. ನಾವು ಹಲವುಗಳ ಮೂಲಕ ಒಂದನ್ನು ಸೇರಬೇಕಾಗಿದೆ.

ನಾವು ಆಲೋಚಿಸುವಂತಹ, ತಿಳಿದುಕೊಳ್ಳುವಂತಹ, ನಮಗೆ ದಾರಿಯನ್ನು ತೋರಬಲ್ಲಂತಹ ಯಾವನಾದರೂ ಸಗುಣಬ್ರಹ್ಮನಿರುವನೆ? ಇರುವನು. ನಿರ್ಗುಣದೇವರಿಗೆ ಮೇಲಿನ ಗುಣಗಳು ಯಾವುವೂ ಇರಲಾರವು. ನಿಮ್ಮಲ್ಲಿ ಪ್ರತಿಯೊಬ್ಬರೂ ಒಂದು ವ್ಯಕ್ತಿ. ನೀವು ಆಲೋಚಿಸುತ್ತೀರಿ, ಪ್ರೀತಿಸುತ್ತೀರಿ, ದ್ವೇಷಿಸುತ್ತೀರಿ, ಕೋಪಗೊಳ್ಳುತ್ತೀರಿ, ವ್ಯಥೆಪಡುತ್ತೀರಿ. ಆದರೂ ನೀವು ನಿರ್ಗುಣರು, ಅನಂತಾತ್ಮರು. ನಿಮಗೆ ನಿರ್ಗುಣದ ಮತ್ತು ಸಗುಣದ ಸ್ವಭಾವಗಳೆರಡೂ ಇರುತ್ತವೆ. ಆ ನಿರ್ಗುಣ ಕೋಪಗೊಳ್ಳಲಾರದು, ವ್ಯಥೆಪಡಲಾರದು, ದುಃಖಪಡಲಾರದು, ಅದು ಅವುಗಳನ್ನು ಆಲೋಚಿಸಲೂ ಆರದು. ಅದು ವಿಚಾರ ಮಾಡಲಾರದು ಅರಿಯಲಾರದು. ಅದು ಜ್ಞಾನವೇ ಆಗಿದೆ. ಆದರೆ ಸಗುಣಕ್ಕೆ ಜ್ಞಾನವಿದೆ. ಅದು ಆಲೋಚಿಸುತ್ತದೆ, ಅದು ಸಾಯುತ್ತದೆ, ಇವುಗಳೆಲ್ಲ ಅದಕ್ಕೆ ಇವೆ. ಇದರಂತೆಯೇ ಅಖಂಡವಾದ ನಿರ್ಗುಣಕ್ಕೆ ಎರಡು ಸ್ವಭಾವಗಳಿರಬೇಕು. ಮೊದಲನೆಯದರಲ್ಲಿ ಇರುವುದೊಂದೇ ಅಖಂಡ ಸತ್ಯ. ಎರಡನೆಯದರಲ್ಲಿ ಅವನಿಗೆ ನಮ್ಮ ಆತ್ಮನ ಆತ್ಮನಾದ ಪರಮೇಶ್ವರನ ಸ್ವಭಾವವಿದೆ. ಅವನೇ ಈ ಪ್ರಪಂಚವನ್ನು ಸೃಷ್ಟಿಸಿರುವನು. ಅವನ ನೇತೃತ್ವದಲ್ಲಿ ಈ ಪ್ರಪಂಚ ಇದೆ. ಅವನು ಅನಂತಾತ್ಮನು, ನಿತ್ಯಶುದ್ದನು, ನಿತ್ಯಮುಕ್ತನು. ಅವನು ನ್ಯಾಯಾಧಿಪತಿಯಲ್ಲ. ದೇವರೆಂದಿಗೂ ನ್ಯಾಯಾಧಿಪತಿಯಾಗಲಾರನು. ಅವನೊಂದು ಸಿಂಹಾಸನದ ಮೇಲೆ ಕುಳಿತುಕೊಂಡು ಪಾಪಪುಣ್ಯಗಳನ್ನು ಪರೀಕ್ಷಿಸುವುದಿಲ್ಲ. ಅವನೇನು ಮ್ಯಾಜಿಸ್ಟ್ರೇಟ್ ಅಲ್ಲ, ದಂಡನಾಯಕನಲ್ಲ, ಅಥವಾ ಸ್ವಾಮಿಯೂ ಅಲ್ಲ. ಸಗುಣಬ್ರಹ್ಮನು ಅನಂತ ಕೃಪಾ ಸಿಂಧು; ಅನಂತ ಪ್ರೀತಿಯ ಮಹೋದಧಿ.

ನೀವು ಮತ್ತೊಂದು ಉದಾಹರಣೆಯನ್ನು ತೆಗೆದುಕೊಳ್ಳಿ - ನಿಮ್ಮ ದೇಹದಲ್ಲಿರುವ ಪ್ರತಿಯೊಂದು ಜೀವಾಣುವಿಗೂ, ಜೀವಾಣುವಿನ ಚೈತನ್ಯವಿದೆ. ಅವುಗಳೆಲ್ಲವೂ\break ಪ್ರತ್ಯೇಕವಾಗಿವೆ. ಅವಕ್ಕೆ ತಮ್ಮದೇ ಒಂದು ಇಚ್ಛಾಶಕ್ತಿಯಿದೆ, ತಮ್ಮದೇ ಒಂದು ಕ್ರಿಯಾಪರಿಧಿಯಿದೆ. ಈ ಜೀವಾಣುಗಳೆಲ್ಲಾ ಸೇರಿ ಒಂದು ವ್ಯಕ್ತಿಯಾಗುವುದು. ಇದರಂತೆಯೇ ಪ್ರಪಂಚದ ಪರಮೇಶ್ವರನು ಜೀವಿಗಳ ಮೊತ್ತದಿಂದ ಆಗುವನು.

ಇದನ್ನು ಮತ್ತೊಂದು ದೃಷ್ಟಿಯಿಂದ ನೋಡೋಣ. ನನ್ನ ಕಣ್ಣಿಗೆ ಕಾಣುವ\break ನೀವಾದರೋ, ನನ್ನ ದೃಷ್ಟಿಯಿಂದ ಪರಿಮಿತಿಗೊಂಡ ಅನಂತಾತ್ಮರು. ನನ್ನ ಕಣ್ಣಿನ ಇಂದ್ರಿಯದ ಮೂಲಕ ನಿಮ್ಮನ್ನು ನೋಡುವುದಕ್ಕಾಗಿ ನಿಮ್ಮನ್ನು ಪರಿಮಿತಿ ಗೊಳಿಸಿರುವೆನು. ನನ್ನ ಕಣ್ಣು ನಿಮ್ಮ ವಿಷಯದಲ್ಲಿ ಎಷ್ಟನ್ನು ನೋಡಬಲ್ಲದೋ ಅಷ್ಟನ್ನು ಮಾತ್ರ ನಾನು ನೋಡಬಲ್ಲೆ. ನನ್ನ ಮನಸ್ಸು ನಿಮ್ಮ ವಿಷಯದಲ್ಲಿ ಎಷ್ಟನ್ನು ಅರ್ಥಮಾಡಿಕೊಳ್ಳಬಲ್ಲದೊ, ಅಷ್ಟೇ ನನಗೆ ಗೊತ್ತಿರುವುದು. ಅದಕ್ಕಿಂತ ಹೆಚ್ಚೇನೂ ಇಲ್ಲ. ಇದರಂತೆಯೇ ನಾನು ಅಖಂಡವನ್ನು, ನಿರ್ಗುಣವನ್ನು ಸಗುಣದಂತೆ ನೋಡುತ್ತಿರುವೆನು. ಎಲ್ಲಿಯವರೆಗೆ ನಮಗೆ ದೇಹ ಮತ್ತು ಮನಸ್ಸುಗಳು ಇವೆಯೊ ಅಲ್ಲಿಯವರೆಗೆ ನಾವು ಈಶ್ವರ, ಪ್ರಕೃತಿ ಮತ್ತು ಜೀವಗಳನ್ನು ನೋಡುತ್ತೇವೆ. ಯಾವಾಗಲೂ ಈ ಮೂರು ವಿಭಾಗವಾಗದ ಒಂದರಲ್ಲಿಯೇ ಇರಬೇಕು. ಪ್ರಕೃತಿ ಇದೆ. ಜೀವಿಗಳು ಇರುವರು. ಪ್ರಕೃತಿ ಮತ್ತು ಜೀವಗಳು ಇರುವಂತಹ ಈಶ್ವರನೂ ಇರಬೇಕು.

ವಿಶ್ವಾತ್ಮನು ಒಂದು ಆಕಾರವನ್ನು ಧರಿಸಿರುವನು. ನನ್ನ ಆತ್ಮನೇ ಭಗವಂತನ ಒಂದು ಅಂಶ, ಅವನೇ ನಮ್ಮ ಕಣ್ಣಿನ ಕಣ್ಣು, ಜೀವದ ಜೀವ, ಮನಸ್ಸಿನ ಮನಸ್ಸು, ಆತ್ಮನ ಆತ್ಮ. ನಮಗೆ ದೊರಕಬಲ್ಲ ಶ್ರೇಷ್ಠತಮವಾದ ಈಶ್ವರನ ಭಾವನೆಯೇ ಇದು.

ನೀವು ದ್ವೈತಿಗಳಲ್ಲದೆ ಏಕತತ್ತ್ವವಾದಿಗಳಾದರೂ, ಒಬ್ಬ ಸಗುಣಬ್ರಹ್ಮನು ಇರಲು ಸಾಧ್ಯವಾಗುವುದು. ಏಕಮೇವ ಅದ್ವಿತೀಯನೊಬ್ಬನಿರುವನು. ಅವನು ತನ್ನನ್ನು ತಾನೇ ಪ್ರೀತಿಸಲೆತ್ನಿಸಿದನು. ಆದಕಾರಣ ಆ ಒಂದರಿಂದ ಹಲವನ್ನು ಉಂಟುಮಾಡಿದನು. ನನ್ನ ದೊಡ್ಡದನ್ನು, ನನ್ನ ಸತ್ಯವನ್ನು ಈ ಸಣ್ಣ ನಾನು ಎಂಬುದು ಪೂಜಿಸುತ್ತಿರುವುದು. ಆದಕಾರಣ ಎಲ್ಲಾ ಸಿದ್ಧಾಂತಗಳಲ್ಲಿಯೂ ಒಬ್ಬ ಸಗುಣಬ್ರಹ್ಮನಿರುವುದು ಸಾಧ್ಯವಾಗುವುದು.

ಕೆಲವರು ಇತರರಿಗಿಂತ ಹೆಚ್ಚು ಸುಖವಾಗಿರುವ ಒಂದು ಸನ್ನಿವೇಶದಲ್ಲಿ ಜನ್ಮವೆತ್ತುವರು. ಒಬ್ಬ ನ್ಯಾಯಮೂರ್ತಿಯಾದ ಭಗವಂತನ ಆಳ್ವಿಕೆಯಲ್ಲಿ ಇದು ಹೇಗೆ. ಸಾಧ್ಯ? ಈ ಪ್ರಪಂಚದಲ್ಲಿ ಮರಣವಿದೆ. ಇವೇ ಇರುವ ಕೆಲವು ತೊಂದರೆಗಳು. ಈ ಸಮಸ್ಯೆಗಳನ್ನು ಯಾರೂ ಇದುವರೆಗೆ ಬಗೆಹರಿಸಿಲ್ಲ. ದ್ವೈತಸಿದ್ಧಾಂತದ ದೃಷ್ಟಿಯಿಂದ ಇವುಗಳಿಗೆ ನಾವು ಉತ್ತರವನ್ನು ಹೇಳಲಾರೆವು. ನಾವು ಇವುಗಳನ್ನು ಬಗೆಹರಿಸಬೇಕಾದರೆ ತತ್ವಕ್ಕೆ ಹೋಗಬೇಕಾಗಿದೆ. ನಮ್ಮ ಕರ್ಮದಿಂದ ನಾವು ವ್ಯಥೆಪಡುತ್ತಿರುವೆವು, ಅದು ದೇವರ ತಪ್ಪಲ್ಲ ಎಂಬುದು. ಇದು ನಮ್ಮ ತಪ್ಪೇ ಹೊರತು ಅನ್ಯರದಲ್ಲ. ಇದಕ್ಕಾಗಿ ದೇವರನ್ನು ಏತಕ್ಕೆ ದೂಷಿಸಬೇಕು?

ಪ್ರಪಂಚದಲ್ಲಿ ಪಾಪ ಏತಕ್ಕೆ ಇರುವುದು? ಈ ಸಮಸ್ಯೆಯನ್ನು ನೀವು ಪರಿಹರಿಸಬೇಕಾದರೆ ಇರುವುದು ಒಂದೇ ಮಾರ್ಗ. ಅದು ದೇವರೇ ಪಾಪ ಪುಣ್ಯಗಳೆರಡಕ್ಕೂ ಕಾರಣ ಎನ್ನುವುದು. ಸಗುಣಬ್ರಹ್ಮನ ಸಿದ್ದಾಂತದಲ್ಲಿ ಬರುವ ಒಂದು ತೊಡಕು, ನೀವು ಅವನನ್ನು ಕೇವಲ ಒಳ್ಳೆಯವನು, ಕೆಟ್ಟವನಲ್ಲ ಎಂದರೆ, ನಿಮ್ಮ ವಾದಸರಣಿಯಲ್ಲಿ ನೀವೇ ಸಿಕ್ಕಿಕೊಳ್ಳುವಿರಿ. ದೇವರಿರುವನು ಎಂಬುದು ನಿಮಗೆ ಹೇಗೆ ಗೊತ್ತು? ನೀವು ಅವನನ್ನು ಪ್ರಪಂಚದ ಕರ್ತ ಎನ್ನುವಿರಿ, ಅವನು ಒಳ್ಳೆಯವನು ಎನ್ನುವಿರಿ. ಈ ಪ್ರಪಂಚದಲ್ಲಿ ಕೆಟ್ಟದ್ದೂ ಇರುವುದರಿಂದ ದೇವರು ಕೆಟ್ಟವನೂ ಆಗಬೇಕಾಗಿದೆ. ಅಂತೂ ಕೊನೆಗೆ ಅದೇ ತೊಂದರೆ ಬರುವುದು!

ಒಳ್ಳೆಯದೂ ಇಲ್ಲ, ಕೆಟ್ಟದ್ದೂ ಇಲ್ಲ. ಇರುವುದೆಲ್ಲಾ ಅವನೇ. ಯಾವುದು ಒಳ್ಳೆಯದು ಎಂಬುದು ನಿಮಗೆ ಹೇಗೆ ಗೊತ್ತು? ಇದು ಒಳ್ಳೆಯದು ಎಂದು ನಿಮಗೆ ಅನ್ನಿಸುವುದು. ಇದು ಕೆಟ್ಟದ್ದು ಎಂಬುದು ನಿಮಗೆ ಹೇಗೆ ಗೊತ್ತು? ಇದು ಕೆಟ್ಟದ್ದು ಎಂದು ನಿಮಗೆ ಅನ್ನಿಸುವುದು. ನಾವು ಒಳ್ಳೆಯದನ್ನು ಮತ್ತು ಕೆಟ್ಟದ್ದನ್ನು ನಮ್ಮ ಭಾವನೆಯ ಮೇಲೆ ನಿರ್ಧರಿಸುವೆವು. ಕೇವಲ ಒಳ್ಳೆಯ, ಸಂತೋಷಕರವಾದ ಭಾವನೆಗಳನ್ನು ಮಾತ್ರ\break ಅನುಭವಿಸುವವನು ಯಾರೂ ಇಲ್ಲ.

ನಮ್ಮ ಸುಖ ಮತ್ತು ದುಃಖಕ್ಕೆಲ್ಲಾ ಕಾರಣ ಬಯಕೆ ಮತ್ತು ಕಳವಳ. ನಮ್ಮ ಬಯಕೆಗಳು ಹೆಚ್ಚುತ್ತಿವೆಯೊ, ಕಡಿಮೆಯಾಗುತ್ತಿವೆಯೊ? ಜೀವನ ಸರಳವಾಗುತ್ತಿದೆಯೊ, ಜಟಿಲವಾಗುತ್ತಿದೆಯೋ? ನಿಜವಾಗಿಯೂ ಜೀವನ ಜಟಿಲವಾಗುತ್ತಿದೆ. ನಮ್ಮ ಬಯಕೆಗಳು ವೃದ್ಧಿಯಾಗುತ್ತಿವೆ. ನಿಮ್ಮ ಮುತ್ತಜ್ಜನಿಗೆ ನಿಮ್ಮ ಉಡುಪಾಗಲಿ, ನಿಮಗೆ ಬೇಕಾದಷ್ಟು ಹಣವಾಗಲಿ ಬೇಕಾಗಿರಲಿಲ್ಲ. ಅವನಿಗೆ ವಿದ್ಯುತ್ ವಾಹನಗಳಿರಲಿಲ್ಲ. ರೈಲುಬಂಡಿಗಳು ಮುಂತಾದುವು ಇರಲಿಲ್ಲ. ಆದಕಾರಣ ಅವನು ಕಡಿಮೆ ಕೆಲಸಮಾಡಬೇಕಾಗಿತ್ತು. ಇವುಗಳು ಬಂದೊಡನೆಯೇ ನಮ್ಮ ಬಯಕೆಗಳು ವೃದ್ಧಿಯಾಗುತ್ತವೆ. ಅದಕ್ಕೆ ನಾವು ಹೆಚ್ಚು ಹೆಚ್ಚು ಕೆಲಸವನ್ನು ಮಾಡಬೇಕಾಗುವುದು. ಇದರಿಂದ ಹೆಚ್ಚು ಹೆಚ್ಚು ಕಳವಳ ಮತ್ತು ಹೆಚ್ಚು ಹೆಚ್ಚು ಸ್ಪರ್ಧೆ ಹುಟ್ಟುವುವು.

ಹಣವನ್ನು ಸಂಪಾದಿಸುವುದು ಬಹಳ ಕಷ್ಟದ ಕೆಲಸ. ಅದನ್ನು ಸಂರಕ್ಷಿಸುವುದು ಅದಕ್ಕಿಂತ ದೊಡ್ಡ ಕಷ್ಟದ ಕೆಲಸ. ನೀವು ಸ್ವಲ್ಪ ಹಣವನ್ನು ಸಂಪಾದಿಸಲು ಪ್ರಪಂಚದಲ್ಲಿ ಹೋರಾಡುವಿರಿ. ಅದನ್ನು ಅನಂತರ ರಕ್ಷಿಸಲು ಇಡೀ ಜೀವನ ಹೋರಾಡಬೇಕಾಗುವುದು. ಆದಕಾರಣವೇ ಬಡವನಿಗಿಂತ ಭಾಗ್ಯವಂತನಿಗೆ ಕಳವಳ ಹೆಚ್ಚು. ಪ್ರಪಂಚದ ಸ್ವಭಾವವೇ ಇದು.

ಈ ಪ್ರಪಂಚದಲ್ಲಿ ಎಲ್ಲಾ ಕಡೆಗಳಲ್ಲಿಯೂ ಒಳ್ಳೆಯದು ಕೆಟ್ಟದ್ದು ಇದ್ದೇ ಇರುತ್ತವೆ. ಕೆಲವು ವೇಳೆ ಕೆಟ್ಟದ್ದೇ ಒಳ್ಳೆಯದಾಗುತ್ತದೆ ಎಂಬುದು ನಿಜ. ಮತ್ತೆ ಕೆಲವು ವೇಳೆ ಒಳ್ಳೆಯದೇ ಕೆಟ್ಟದ್ದು ಆಗುತ್ತದೆ. ನಮ್ಮ ಇಂದ್ರಿಯಗಳೆಲ್ಲ ಯಾವಾಗಲಾದರೂ ಒಂದು ಸಲ ಕೆಟ್ಟದ್ದನ್ನು ಮಾಡಿಯೇ ಮಾಡುತ್ತವೆ. ಮನುಷ್ಯ ಮದ್ಯಪಾನ ಮಾಡಲಿ. ಮೊದಲು ಅದರಿಂದ ಯಾವ ಬಾಧಕವೂ ಇರುವುದಿಲ್ಲ. ಆದರೆ ಅವನು ಅದನ್ನು ಕುಡಿಯುತ್ತಾ ಹೋದರೆ ಅದು ಅವನಿಗೆ ಕೇಡನ್ನು ಉಂಟುಮಾಡುವುದು. ಒಬ್ಬನು ಶ‍್ರೀಮಂತ ತಂದೆತಾಯಿಗಳಿಗೆ ಜನಿಸುತ್ತಾನೆ ಎಂದು ಭಾವಿಸೋಣ. ಅದೇನೋ ಒಳ್ಳೆಯದೆ. ಅವನು ತನ್ನ ದೇಹವನ್ನು ಮತ್ತು ಮನಸ್ಸನ್ನು ಉಪಯೋಗಿಸದೆ ಮೂರ್ಖನಾಗುತ್ತಾನೆ. ಇಲ್ಲಿ ಒಳ್ಳೆಯದು ಕೆಟ್ಟದ್ದನ್ನು ಉಂಟುಮಾಡುತ್ತದೆ. ಜೀವನದ ಮೇಲೆ ನಮಗಿರುವ ಆಸಕ್ತಿಯನ್ನು ಕುರಿತು ಯೋಚಿಸಿ, ನಾವು ಕೆಲವು ಕ್ಷಣಗಳು ಕುಣಿದಾಡಿ ಬಾಳುತ್ತೇವೆ. ಕಷ್ಟಪಟ್ಟು ಕೆಲಸಮಾಡುತ್ತೇವೆ. ಯಾವ ಕೆಲಸವನ್ನೂ ಮಾಡಲಾಗದ ಎಳೆ ಹಸುಳೆಗಳಂತೆ ನಾವು ಹುಟ್ಟುವೆವು. ನಾವು ವಿಷಯಗಳನ್ನು ತಿಳಿದುಕೊಳ್ಳಬೇಕಾದರೆ ಹಲವು ವರ್ಷಗಳು ಬೇಕು. ಅರವತ್ತೋ ಎಪ್ಪತ್ತೋ ವರ್ಷಗಳ ಹೊತ್ತಿಗೆ ನಮಗೆ ಸ್ವಲ್ಪ ಜ್ಞಾನೋದಯವಾಗುವುದು. ಅಷ್ಟು ಹೊತ್ತಿಗೆ ಪ್ರಪಂಚವನ್ನು ಬಿಟ್ಟು ಹೊರಡು ಎಂಬ ಕರೆ ಬರುವುದು. ಇದೇ ನಮ್ಮ ಪರಿಸ್ಥಿತಿ.

ಒಳ್ಳೆಯದು ಮತ್ತು ಕೆಟ್ಟದ್ದು ಎಂಬ ಎರಡು ಪದಗಳು ಸಾಪೇಕ್ಷ ಎಂಬುದನ್ನು ನಾವು ನೋಡಿದೆವು. ಯಾವುದು ನನಗೆ ಒಳ್ಳೆಯದೊ ಅದು ನಿಮಗೆ ಕೆಟ್ಟದು. ನಾನು ಮಾಡುವ ಊಟವನ್ನು ನೀವು ಮಾಡಿದರೆ ನೀವು ಅಳುತ್ತೀರಿ. ನಾನು ನಗುವೆ. ನಾವಿಬ್ಬರೂ ಕುಣಿಯಬಹುದು. ನಾನು ಸಂತೋಷದಿಂದ ಕುಣಿಯುವೆ, ನೀವು ನೋವಿನಿಂದ ಕುಣಿಯುತ್ತೀರಿ, ನಮ್ಮ ಜೀವನದ ಒಂದು ಕಡೆ ಕೆಟ್ಟದ್ದು ಮತ್ತೊಂದು ಕಡೆ ಅದೇ ಒಳ್ಳೆಯದು. ಒಳ್ಳೆಯದು ಕೆಟ್ಟದ್ದು ಎರಡೂ ಪ್ರತ್ಯೇಕವಾಗಿವೆ. ಇದೆಲ್ಲ ಒಳ್ಳೆಯದು, ಅದೆಲ್ಲ ಕೆಟ್ಟದ್ದು ಎಂದು ಹೇಗೆ ಹೇಳಬಲ್ಲಿರಿ?

ದೇವರು ಯಾವಾಗಲೂ ಒಳ್ಳೆಯವನೇ ಆದರೆ, ಈ ಕೆಟ್ಟದ್ದು ಮತ್ತು ಒಳ್ಳೆಯದಕ್ಕೆಲ್ಲಾ ಯಾರು ಜವಾಬ್ದಾರರು? ಕ್ರೈಸ್ತರು ಮತ್ತು ಮಹಮ್ಮದೀಯರು, ಸೈತಾನನೆಂಬ ಸಭ್ಯನೊಬ್ಬನು ಇರುವನು ಎನ್ನುತ್ತಾರೆ. ಇಬ್ಬರು ಸಭ್ಯರು ಕೆಲಸ ಮಾಡುತ್ತಿರುವರು ಎಂದು ನೀವು ಹೇಗೆ ಹೇಳುತ್ತೀರಿ? ಒಬ್ಬನೇ ಇರಬೇಕಾಗುವುದು. ಯಾವ ಬೆಂಕಿ ಮಗುವನ್ನು ಸುಡುವುದೋ ಅದೇ ಬೆಂಕಿ ಅನ್ನವನ್ನು ಬೇಯಿಸುವುದು. ನೀವು ಬೆಂಕಿಯನ್ನು ಒಳ್ಳೆಯದು ಅಥವಾ ಕೆಟ್ಟದ್ದು ಎಂದು ಹೇಗೆ ಹೇಳುತ್ತೀರಿ? ಈ ಕೆಟ್ಟದ್ದನ್ನೆಲ್ಲಾ ಯಾರು ಸೃಷ್ಟಿಸುವರು? ದೇವರು. ಬೇರೆ ದಾರಿಯೇ ಇಲ್ಲ. ಅವನೇ ಜನನ ಮರಣಗಳನ್ನು ಕಳುಹಿಸುವವನು. ಪ್ಲೇಗನ್ನು ಮತ್ತು ಇತರ ಸಾಂಕ್ರಾಮಿಕ ಜಾಡ್ಯಗಳನ್ನು ಕಳುಹಿಸುವವನೂ ಅವನೇ. ದೇವರು ಇಂತಹವನಾದರೆ, ಅವನೇ ಒಳ್ಳೆಯವನು ಮತ್ತು ಕೆಟ್ಟವನು. ಅವನೇ ಸುಂದರನಾಗಿರುವವನು ಮತ್ತು ಭಯಾನಕನಾಗಿರುವವನು. ಅವನೆ, ಜನನ ಮತ್ತು ಮರಣ.

ಇಂತಹ ದೇವರನ್ನು ನಾವು ಪೂಜಿಸುವುದು ಹೇಗೆ? ಜೀವನು, ನಿಜವಾಗಿ ಭಯಾನಕವಾಗಿರುವುದನ್ನು ಪೂಜಿಸುವುದು ಹೇಗೆ ಎಂಬುದನ್ನು ಅರಿಯುವನು. ಆಗ ಮಾತ್ರ ಅದಕ್ಕೆ ಶಾಂತಿ. ನಿಮಗೆ ಶಾಂತಿ ಇದೆಯೆ? ಕಳವಳಗಳಿಂದ ನೀವು ಪಾರಾಗಬಲ್ಲಿರಾ? ಮೊದಲು ಸುತ್ತಲೂ ತಿರುಗಿ ನೋಡಿ, ಭಯಾನಕವಾದುದನ್ನು ಎದುರಿಸಿ, ಮೇಲೆ\break ಮುಸುಕಿರುವ ತೆರೆಯನ್ನು ಅತ್ತ ಸರಿಸಿ, ಒಬ್ಬನೇ ದೇವರನ್ನು ಎಲ್ಲಾ\break ಕಡೆಗಳಲ್ಲಿಯೂ ನೋಡಿ, ಅವನೇ ಸಗುಣಬ್ರಹ್ಮ, ತಾತ್ಕಾಲಿಕವಾಗಿ ಒಳ್ಳೆಯದರಂತೆ ಮತ್ತು ಕೆಟ್ಟದರಂತೆ ಇರುವವನೆಲ್ಲ ಅವನೇ. ಅವನಲ್ಲದೆ ಬೇರೆ ವಸ್ತುವೇ ಇಲ್ಲ. ಇಬ್ಬರು ದೇವರು ಇದ್ದರೆ ಪ್ರಕೃತಿ ಒಂದು ಕ್ಷಣವೂ ಇರುತ್ತಿರಲಿಲ್ಲ. ಪ್ರಕೃತಿಯಲ್ಲಿ ಎರಡನೆಯವನೇ ಇಲ್ಲ. ಎಲ್ಲಾ ಕಡೆಗಳಲ್ಲಿಯೂ ಸಾಮರಸ್ಯವಿದೆ. ದೇವರು ಒಂದು ಕಡೆ, ಸೈತಾನನು ಒಂದು ಕಡೆ ಸೇರಿಕೊಂಡರೆ ಪ್ರಪಂಚವೆಲ್ಲ ಅನಾಯಕವಾಗುವುದು. ಪ್ರಕೃತಿಯ ನಿಯಮವನ್ನು ಯಾರು ಉಲ್ಲಂಘಿಸಬಲ್ಲರು? ನಾನು ಈ ಗ್ಲಾಸನ್ನು ಒಡೆದರೆ ಅದು ಚೂರು ಚೂರಾಗಿ ಕೆಳಗೆ ಬೀಳುವುದು. ಯಾರಿಗಾದರೂ ಒಂದು ಕಣವನ್ನು ಅದರ ಸ್ಥಳದಿಂದ ಕದಲಿಸಲು ಸಾಧ್ಯವಾದರೆ, ಆಗ ಇತರ ಕಣಗಳೆಲ್ಲ ಚೆಲ್ಲಾಪಿಲ್ಲಿಯಾಗುವುವು. ನಿಯಮವನ್ನು ಯಾರೂ ಉಲ್ಲಂಘಿಸಲಾರರು. ಪ್ರತಿಯೊಂದು ಕಣವೂ ಆಯಾ ಸ್ಥಳದಲ್ಲಿ ನಿಲ್ಲಿಸಲ್ಪಟ್ಟಿದೆ. ಪ್ರತಿಯೊಂದನ್ನೂ ತೂಕಮಾಡಿದೆ, ಅಳತೆಮಾಡಿದೆ. ಅದು ತನ್ನ ಸ್ಥಳದಲ್ಲಿ ಯಾವ ಕೆಲಸವನ್ನು ಮಾಡಬೇಕೋ ಅದನ್ನು ಮಾಡುವುದು. ಅವನ ಆಜ್ಞಾನುಸಾರ ಗಾಳಿ ಬೀಸುವುದು, ಸೂರ್ಯ ಬೆಳಗುವನು. ಅವನ ಆಜ್ಞೆಗೆ ತಲೆಬಾಗಿ ಪ್ರಪಂಚಗಳು ತಮ್ಮ ತಮ್ಮ ಸ್ಥಳಗಳಲ್ಲಿವೆ. ಅವನ ಆಜ್ಞೆಗೆ ಅನುಸಾರವಾಗಿ ಮೃತ್ಯು ಪ್ರಪಂಚದಲ್ಲಿ ನರ್ತನ ಮಾಡುತ್ತಿದೆ. ಈ ಪ್ರಪಂಚದಲ್ಲಿ ಇಬ್ಬರು ಮೂವರು ದೇವರುಗಳು ಪರಸ್ಪರ ಕುಸ್ತಿ ಮಾಡುವುದನ್ನು ಕುರಿತು ಯೋಚಿಸಿ ನೋಡಿ! ಇದು ಸಾಧ್ಯವೇ ಇಲ್ಲ.

ಈ ಪ್ರಪಂಚವನ್ನು ಸೃಷ್ಟಿಸಿರುವ, ದಯಾಮಯನಾದ ಮತ್ತು ಕ್ರೂರಿಯೂ ಆದ ಸಗುಣಬ್ರಹ್ಮನನ್ನು ನಾವು ಹೊಂದಲು ಸಾಧ್ಯ ಎಂಬುದನ್ನು ನೋಡುತ್ತೇವೆ. ಅವನು ಒಳ್ಳೆಯವನು ಮತ್ತು ಕೆಟ್ಟವನು. ಅವನು ನಗುವನು ಮತ್ತು ಕೋಪಗೊಳ್ಳುವನು. ಯಾರೂ ಅವನ ಆಜ್ಞೆಗೆ ವಿರೋಧವಾಗಿ ಹೋಗಲಾರರು. ಅವನೇ ಈ ಪ್ರಪಂಚದ ಸೃಷ್ಟಿಕರ್ತ.

ಸೃಷ್ಟಿ ಎಂದರೇನು? ಶೂನ್ಯದಿಂದ ಏನಾದರೂ ಬರುವುದೆ? ಆರುಸಾವಿರ ವರ್ಷಗಳ ಹಿಂದೆ ದೇವರು ತನ್ನ ನಿದ್ರೆಯಿಂದ ಎದ್ದು ಈ ಪ್ರಪಂಚವನ್ನು ಸೃಷ್ಟಿಸಿದನು. ಅದಕ್ಕೆ ಮುಂಚೆ ಪ್ರಪಂಚವೇ ಇರಲಿಲ್ಲವೆ? ಆಗ ದೇವರು ಏನು ಮಾಡುತ್ತಿದ್ದ? ಅವನೇನು ಚೆನ್ನಾಗಿ ನಿದ್ರೆ ಹೊಡೆಯುತ್ತಿದ್ದನೆ? ದೇವರು ಈ ಪ್ರಪಂಚಕ್ಕೆ ಕಾರಣ. ಪರಿಣಾಮದ ಮೂಲಕ ನಾವು ಕಾರಣವನ್ನು ಊಹಿಸಬಹುದು. ಪರಿಣಾಮ ಇಲ್ಲದೇ ಇದ್ದರೆ, ಕಾರಣ ಕಾರಣವೇ ಅಲ್ಲ. ಪರಿಣಾಮದಿಂದ ಮತ್ತು ಪರಿಣಾಮದ ಮೂಲಕ ಮಾತ್ರ ಕಾರಣವನ್ನು ಅರಿಯಬಹುದು. ಸೃಷ್ಟಿ ಅನಾದಿ ಮತ್ತು ಅನಂತವಾದುದು. ನೀವು ಕಾಲದಲ್ಲೇ ಆಗಲಿ ದೇಶದಲ್ಲೇ ಆಗಲಿ ಆದಿಯನ್ನು ಕುರಿತು ಆಲೋಚಿಸಲಾರಿರಿ.

ಅವನು ಏತಕ್ಕೆ ಸೃಷ್ಟಿಸುವನು? ಏಕೆಂದರೆ ಇದು ಅವನಿಗೆ ಇಷ್ಟ. ಏಕೆಂದರೆ ಅವನು ಸ್ವತಂತ್ರನು. ನಾವು ನೀವುಗಳು ನಿಯಮದಿಂದ ಬದ್ದರು. ನಾವು ಯಾವುದೋ ಒಂದು ರೀತಿ ಮಾತ್ರ ಕೆಲಸ ಮಾಡಬಲ್ಲೆವು, ಬೇರೆ ರೀತಿ ಮಾಡಲಾರೆವು. “ಕೈಗಳಿಲ್ಲದೆ ಅವನು ಎಲ್ಲವನ್ನೂ ಹಿಡಿದುಕೊಳ್ಳಬಲ್ಲ; ಕಾಲುಗಳಿಲ್ಲದೆ ಅವನು ವೇಗವಾಗಿ ಚಲಿಸಬಲ್ಲ.'' ದೇಹವಿಲ್ಲದೇ ಇದ್ದರೂ ಅವನು ಸರ್ವಶಕ್ತನು. “ಯಾರನ್ನು ಯಾವ ಕಣ್ಣುಗಳೂ ನೋಡಲಾರವೊ, ಆದರೆ ಪ್ರತಿಯೊಂದು ಕಣ್ಣು ನೋಡುವ ದೃಶ್ಯಕ್ಕೂ ಯಾವನು ಕಾರಣನೋ ಅವನೇ ಪರಮಾತ್ಮನು ಎಂದು ಅರಿ.” ನೀವು ಬೇರೆ ಯಾವುದನ್ನೂ ಪೂಜಿಸಲಾರಿರಿ. ದೇವರ ಅನಂತ ಶಕ್ತಿಯು ಈ ಪ್ರಪಂಚವನ್ನು ರಕ್ಷಿಸುತ್ತಿರುವುದು. ನಾವು ಯಾವುದನ್ನು ನಿಯಮ ಎನ್ನುತ್ತೇವೆಯೋ ಅದು ಅವನ ಇಚ್ಚೆಯ ಅಭಿವ್ಯಕ್ತಿ. ಅವನು ವಿಶ್ವವನ್ನು ತನ್ನ ನಿಯಮಗಳ ಮೂಲಕ ಆಳುತ್ತಾನೆ.

ಇದುವರೆಗೆ ನಾವು ಈಶ್ವರನನ್ನು ಮತ್ತು ಪ್ರಕೃತಿಯನ್ನು, ಅನಾದಿಯಾದ ಈಶ್ವರನನ್ನು ಮತ್ತು ಅನಾದಿಯಾದ ಪ್ರಕೃತಿಯನ್ನು ಕುರಿತು ಚರ್ಚಿಸುತ್ತಿದ್ದೆವು. ಜೀವಿಗಳ ವಿಷಯವೇನು? ಅವರು ಕೂಡ ಅನಾದಿಯೆ. ಯಾವ ಜೀವರೂ ಹೊಸದಾಗಿ ಸೃಷ್ಟಿಸಲ್ಪಡಲಿಲ್ಲ. ಅಥವಾ ಯಾವ ಜೀವವೂ ಎಂದಿಗೂ ನಾಶವಾಗಲಾರದು. ಯಾರೂ ತಮ್ಮ ಸಾವನ್ನು ತಾವೇ ಕಲ್ಪಿಸಿಕೊಳ್ಳಲೂ ಆರರು. ಜೀವ ಅನಾದಿ ಮತ್ತು ಅನಂತ. ಅದು ಹೇಗೆ ನಾಶವಾಗಬಲ್ಲದು? ಅದು ಕೇವಲ ದೇಹವನ್ನು ಮಾತ್ರ ಬದಲಾಯಿಸಿಕೊಳ್ಳುವುದು. ಒಬ್ಬ ತನ್ನ ಹಳೆಯ, ಹರಿದು ಹೋದ ಬಟ್ಟೆಯನ್ನು ಬಿಸುಟು ಹೊಸ ಬಟ್ಟೆಯನ್ನು ಧರಿಸುವಂತೆ, ಜೀರ್ಣವಾದ ಹಳೆಯ ದೇಹವನ್ನು ತ್ಯಜಿಸಿ ಹೊಸ ದೇಹವನ್ನು ಧರಿಸುವನು.

ಜೀವರ ಸ್ವಭಾವ ಏನು? ಜೀವರು ಕೂಡ ಸರ್ವಶಕ್ತರು, ಸರ್ವವ್ಯಾಪಕರು. ಆತ್ಮ ನಿಗೆ ಉದ್ದ ಅಗಲ ದಪ್ಪ ಇರಲಾರದು. ಅದು ಅಲ್ಲಿ ಇಲ್ಲಿ ಇದೆ ಎಂದು ಹೇಗೆ ಹೇಳುವುದು? ಈ ದೇಹ ನಾಶವಾಗುವುದು. ಜೀವ ಮತ್ತೊಂದು ದೇಹದ ಮೂಲಕ ಕೆಲಸ ಮಾಡುವುದು. ಜೀವ ಒಂದು ವೃತ್ತದಂತೆ. ಅದಕ್ಕೆ ಪರಿಧಿ ಎಲ್ಲಿಯೂ ಇಲ್ಲ. ಆದರೆ ಅದರ ಕೇಂದ್ರ ದೇಹದಲ್ಲಿದೆ. ದೇವರು ಕೂಡ ಒಂದು ವೃತ್ತದಂತೆ. ಅವನ ಪರಿಧಿ ಎಲ್ಲಿಯೂ ಇಲ್ಲ. ಆದರೆ ಅವನ ಕೇಂದ್ರ ಎಲ್ಲಾ ಕಡೆಗಳಲ್ಲಿಯೂ ಇರುವುದು. ಜೀವ ಸ್ವಭಾವತಃ ಧನ್ಯವಾದುದು. ಪರಿಶುದ್ಧವಾದುದು ಮತ್ತು ಪರಿಪೂರ್ಣವಾದುದು. ಅದರ ಸ್ವಭಾವ ಪರಿಶುದ್ಧವಲ್ಲದೇ ಇದ್ದರೆ ಅದು ಎಂದಿಗೂ ಪರಿಶುದ್ಧವಾಗಿ ಇರುತ್ತಿರಲಿಲ್ಲ. ಜೀವದ ಸ್ವಭಾವ ಪಾವಿತ್ರ್ಯ. ಅದಕ್ಕಾಗಿ ಅದು ಪವಿತ್ರವಾಗಬಲ್ಲದು. ಇದು ಸ್ವಭಾವತಃ ಧನ್ಯವಾದುದು. ಅದಕ್ಕೇ ಇದು ಧನ್ಯವಾಗಬಲ್ಲದು. ಇದು ಸ್ವಭಾವತಃ ಪ್ರಶಾಂತವಾದುದು. ಅದಕ್ಕೇ ಇದು ಪ್ರಶಾಂತವಾಗಬಲ್ಲದು.

ಈ ಸ್ತರದಲ್ಲಿರುವ ನಾವೆಲ್ಲರೂ ಒಂದು ದೇಹದಲ್ಲಿ ಆಸಕ್ತರಾಗಿ ಜೀವನೋಪಾಯಕ್ಕಾಗಿ, ಅಸೂಯೆ, ಹೋರಾಟ, ಕಷ್ಟಕಾರ್ಪಣ್ಯಗಳೊಂದಿಗೆ ಹೋರಾಡಿ ಒಂದು ದಿನ ಸಾಯುತ್ತೇವೆ. ನಾವು ನಿಜವಾಗಿ ಏನಾಗಿರುವೆವೊ ಅದು ಆಗಿಲ್ಲ ಎಂಬುದನ್ನು ಇದು ತೋರುವುದು. ನಾವು ಮುಕ್ತರಲ್ಲ, ಪರಿಶುದ್ದರಲ್ಲ. ಜೀವ ಅಧೋಗತಿಗೆ ಬಂದಂತೆ ಇದೆ. ಈಗ ಅದಕ್ಕೆ ಬೇಕಾಗಿರುವುದು ವಿಕಾಸ.

ನೀವು ಇದನ್ನು ಹೇಗೆ ಮಾಡಬಲ್ಲಿರಿ? ನೀವೇ ಇದನ್ನು ಮಾಡಬಲ್ಲಿರಾ? ಇಲ್ಲ. ಒಬ್ಬನ ಮುಖ ಕೊಳೆಯಾಗಿದ್ದರೆ, ಕೊಳೆಯಿಂದಲೇ ಅದನ್ನು ಹೋಗಲಾಡಿಸಲು\break ಸಾಧ್ಯವೆ? ನಾನು ನೆಲದಲ್ಲಿ ಒಂದು ಬೀಜವನ್ನು ಹಾಕಿದರೆ, ಅದೊಂದು ಮರವಾಗುವುದು. ಆ ಮರ ಬೀಜವನ್ನು ಕೊಡುವುದು. ಆ ಬೀಜ ಒಂದು ಮರವನ್ನು ಕೊಡುವುದು. ಕೋಳಿಯಿಂದ ಮೊಟ್ಟೆ, ಮೊಟ್ಟೆಯಿಂದ ಕೋಳಿ. ನೀವು ಏನನ್ನಾದರೂ ಒಳ್ಳೆಯದನ್ನು ಮಾಡಿದರೆ ಅದರ ಪರಿಣಾಮವನ್ನು ಅನುಭವಿಸಬೇಕಾಗುವುದು. ನೀವು ಪುನಃ ಜನ್ಮತಾಳಿ ವ್ಯಥೆಪಡಬೇಕಾಗುವುದು. ಈ ಅನಂತ ಸರಪಳಿ ಒಮ್ಮೆ ಪ್ರಾರಂಭವಾದರೆ ಅದನ್ನು ನಿಲ್ಲಿಸುವಂತೆ ಇಲ್ಲ. ನೀವು ಮೇಲೆ ಕೆಳಗೆ, ಸ್ವರ್ಗ ನರಕಗಳ ಮೂಲಕ ಹಲವು ದೇಹಗಳ ಮೂಲಕ ಹೋಗಿ ಬರುತ್ತ ಇರಬೇಕಾಗುವುದು. ಇದರಿಂದ ತಪ್ಪಿಸಿಕೊಳ್ಳಲು ಸಾಧ್ಯವೇ ಇಲ್ಲ.

ಹಾಗಾದರೆ ನೀವು ಇದರಿಂದ ಪಾರಾಗಿಹೋಗುವುದು ಹೇಗೆ? ನೀವು ಇಲ್ಲಿ ಇರುವುದಾದರೂ ಏತಕ್ಕೆ? ಒಂದು ಉದ್ದೇಶವೆಂದರೆ ದುಃಖದಿಂದ ಪಾರಾಗುವುದು. ನಾವು ಇದನ್ನು ಕರ್ಮದ ಮೂಲಕ ಮಾಡಲಾರೆವು. ಕರ್ಮ ಮತ್ತಷ್ಟು ಹೆಚ್ಚು ಕರ್ಮವನ್ನು ಉತ್ಪತ್ತಿ ಮಾಡುವುದು. ಯಾವನು ತಾನೇ ಮುಕ್ತನಾಗಿರುವನೋ, ಅವನು ನಮಗೆ ಸಹಾಯಮಾಡಿದರೆ ಮಾತ್ರ ಅದು ಸಾಧ್ಯ. “ಅಮೃತಪುತ್ರರೇ, ಯಾರು ಈ ಲೋಕದಲ್ಲಿ ಇರುವರೋ ಮತ್ತು ಯಾರು ಸ್ವರ್ಗದಲ್ಲಿ ಕೂಡ ಇರುವರೋ ಅವರೆಲ್ಲ ಕೇಳಲಿ: ನನಗೆ ಅದರ ರಹಸ್ಯ ಗೊತ್ತಾಗಿದೆ'' ಎಂದು ಮಹಾಋಷಿಗಳು ಹೇಳುವರು. “ಯಾವನು ಅಜ್ಞಾನಕ್ಕೆ ಅತೀತನಾಗಿರುವನೋ ಅವನನ್ನು ನಾನು ಕಂಡಿರುವೆನು. ಅವನ ಕೃಪೆಯಿಂದ ಮಾತ್ರ ನಾವು ಈ ಸಂಸಾರದಿಂದ ಪಾರಾಗಬಲ್ಲೆವು.”

ಭರತಖಂಡದಲ್ಲಿ ಗುರಿ ಎಂದರೆ ಇದು. ಸ್ವರ್ಗಗಳಿವೆ, ನರಕಗಳಿವೆ, ಪೃಥ್ವಿ ಮತ್ತು ಹಲವು ಲೋಕಗಳಿವೆ. ಆದರೆ ಇವುಗಳಾವುವೂ ನಿತ್ಯವಲ್ಲ. ನನ್ನನ್ನು ನರಕಕ್ಕೆ ಕಳುಹಿಸಿದರೆ ಅದು ಎಂದೆಂದಿಗೂ ಅಲ್ಲಿ ಇರುವುದಕ್ಕಲ್ಲ. ನಾನು ಎಲ್ಲಿದ್ದರೂ ಇದೇ ಹೋರಾಟ ಮುಂದೆ ಸಾಗುವುದು. ನಾವು ಈ ಹೋರಾಟಗಳಿಂದ ಪಾರಾಗುವುದು ಹೇಗೆ ಎಂಬುದೇ ಸಮಸ್ಯೆ. ನಾನೇನಾದರೂ ಸ್ವರ್ಗಕ್ಕೆ ಹೋದರೆ, ಬಹುಶಃ ಅಲ್ಲಿ ನನಗೆ ಸ್ವಲ್ಪ ವಿಶ್ರಾಂತಿ ಸಿಕ್ಕಬಹುದು. ಅಥವಾ ನನ್ನ ತಪ್ಪಿಗೆ ಯಾರನ್ನಾದರೂ ಶಿಕ್ಷಿಸಿದರೆ ಅದು ಎಂದೆಂದಿಗೂ ಸರಿಯಾಗಲಾರದು. ಸ್ವರ್ಗಲೋಕಕ್ಕೆ ಹೋಗುವುದಲ್ಲ ಭಾರತೀಯನ ಆದರ್ಶ, ಈ ಭೂಲೋಕದಿಂದ ಪಾರಾಗಬೇಕು, ಸ್ವರ್ಗಲೋಕದಿಂದ ಪಾರಾಗಬೇಕು, ನರಕಲೋಕದಿಂದಲೂ ಪಾರಾಗಲೇಬೇಕು! ಹಾಗಾದರೆ ಗುರಿ ಏನು? ಅದೇ ಮುಕ್ತಿ, ನೀವೆಲ್ಲ ಮುಕ್ತರಾಗಬೇಕು. ಆತ್ಮನ ಮಹಿಮೆ ಯಾವುದರಿಂದಲೋ ಮುಚ್ಚಿಹೋಗಿದೆ. ನಾವು ಅದನ್ನು ಪುನಃ ಪ್ರಕಾಶಕ್ಕೆ ತರಬೇಕಾಗಿದೆ. ಆತ್ಮ ಇರುವುದು. ಅದು ಎಲ್ಲೆಲ್ಲಿಯೂ ಇರುವುದು. ಅದು ಎಲ್ಲಿಗೆ ಹೋಗುವುದು? ಅದು ಎಲ್ಲಿಗೆ ಹೋಗಬೇಕು? ಅದು ಎಲ್ಲಿ ಇಲ್ಲವೋ ಅಲ್ಲಿಗೆ ಮಾತ್ರ ಹೋಗಬಲ್ಲದು. ಅದು ಎಲ್ಲಿಯೂ ಇರುವುದು ಎಂಬುದನ್ನು ನೀವು ಅರಿತರೆ, ಇನ್ನು ಮೇಲೆ ನೀವು ಎಂದೆಂದಿಗೂ ಸುಖಿಗಳು, ಇನ್ನು ಮೇಲೆ ನಿಮಗೆ ಜನನ ಮರಣಗಳಿಲ್ಲ, ಇನ್ನು ಮೇಲೆ ಯಾವ ರೋಗ ರುಜಿನವೂ ಇಲ್ಲ, ಇನ್ನು ಮೇಲೆ ಯಾವ ದೇಹವೂ ಇಲ್ಲ. ದೇಹವೇ ಎಲ್ಲಕ್ಕಿಂತ ದೊಡ್ಡ ಕಾಯಿಲೆ.

ಆತ್ಮವು ಆತ್ಮನಂತೆ ನಿಲ್ಲುವುದು, ಚೈತನ್ಯವು ಚೈತನ್ಯದಂತೆ ನಿಲ್ಲುವುದು. ಇದನ್ನು ಮಾಡುವುದು ಹೇಗೆ? ಆತ್ಮನಲ್ಲಿರುವ ಪರಮಾತ್ಮನನ್ನು ಪೂಜಿಸುವ ಮೂಲಕ. ಸ್ವಭಾವತಃ ಅವನು ಎಲ್ಲರಲ್ಲಿಯೂ ಸದಾಕಾಲದಲ್ಲಿಯೂ ಪರಿಶುದ್ದನಾಗಿ, ಪರಿಪೂರ್ಣನಾಗಿ ಇರುವನು. ಈ ಪ್ರಪಂಚದಲ್ಲಿ ಇಬ್ಬರು ಸರ್ವಶಕ್ತರಾದ ದೇವರು ಇರಲಾರರು. ಇಬ್ಬರು ಮೂವರು ದೇವರುಗಳು ಇರುವುದನ್ನು ಯೋಚಿಸಿ ನೋಡಿ! ಒಬ್ಬ ಒಂದು\break ಪ್ರಪಂಚವನ್ನು ಸೃಷ್ಟಿಸುವನು, ಮತ್ತೊಬ್ಬ ಅದನ್ನು ನಾಶಮಾಡುತ್ತೇನೆ ಎನ್ನುವನು. ಇದು ಎಂದಿಗೂ ಹೀಗೆ ಆಗಲಾರದು. ದೇವರು ಒಬ್ಬನೇ ಇರಬೇಕಾಗುವುದು. ಆತ್ಮ ಪರಿಪೂರ್ಣತೆಯನ್ನು ಪಡೆಯುವುದು. ಅದು ಸ್ವಲ್ಪ ಕಡಿಮೆ ಈಶ್ವರನಂತೆಯೇ ಸರ್ವಶಕ್ತವಾಗುವುದು, ಸರ್ವವ್ಯಾಪಿಯಾಗುವುದು. ಇವನೇ ಪೂಜಿಸುವವನು. ಹಾಗಾದರೆ ಪೂಜೆ ಮಾಡಿಸಿಕೊಳ್ಳುವವನು ಯಾರು? ಅವನೇ ಸ್ವಯಂ ಭಗವಂತ, ಸರ್ವವ್ಯಾಪಿಯಾದ, ಸರ್ವಜ್ಞನಾದ ಪ್ರಭು. ಎಲ್ಲಕ್ಕಿಂತ ಹೆಚ್ಚಾಗಿ ಅವನು ಪ್ರೇಮಸ್ವರೂಪಿ. ಈ ಪರಿಪೂರ್ಣತೆಯನ್ನು ಆತ್ಮವು ಹೇಗೆ ಪಡೆಯುವುದು? ಪೂಜೆಯಿಂದ.

