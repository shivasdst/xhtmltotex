
\chapter{ಪಾಂಬನ್ನಿನ ಬಿನ್ನವತ್ತಳೆಗೆ ಉತ್ತರ}

ಸ್ವಾಮಿ ವಿವೇಕಾನಂದರು ಪಾಂಬನ್ನನ್ನು ತಲುಪಿದ ಕೂಡಲೆ ರಾಮನಾಡಿನ\break ಮಹಾರಾಜರು ಅವರನ್ನು ಸ್ವಾಗತಿಸಿದರು. ಸ್ವಾಮೀಜಿ ಹಡಗನ್ನು ಇಳಿಯುವ ಸ್ಥಳದಲ್ಲಿ ಔಪಚಾರಿಕವಾದ ಸ್ವಾಗತಕ್ಕೆ ತಕ್ಕ ಸಿದ್ಧತೆಗಳು ನಡೆದಿದ್ದವು. ಅಲ್ಲಿ ಅತ್ಯಂತ ಸುಂದರವೂ ಸುಸಂಸ್ಕೃತವೂ ಆದ ರೀತಿಯಲ್ಲಿ ಅಲಂಕೃತವಾಗಿದ್ದ ಚಪ್ಪರದಲ್ಲಿ ಸ್ವಾಮೀಜಿಯವರಿಗೆ ಈ ಮುಂದಿನ ಬಿನ್ನವತ್ತಳೆಯನ್ನು ಅರ್ಪಿಸಲಾಯಿತು:

\textbf{ಪರಮ ಪೂಜ್ಯ ಯತಿವರ್ಯರೇ,}

ಅತ್ಯಂತ ಕೃತಜ್ಞತೆಯಿಂದಲೂ ಅತಿ ಹೆಚ್ಚಿನ ಗೌರವದಿಂದಲೂ ತಮ್ಮನ್ನು ಸ್ವಾಗತಿಸಲು ನಮ್ಮೆಲ್ಲರಿಗೂ ಅತೀವವಾದ ಆನಂದವಾಗುತ್ತಿದೆ. ನಮ್ಮ ಕೃತಜ್ಞತೆಗೆ ಕಾರಣ, ತಮಗೆ ಬಿಡುವಿಲ್ಲದಷ್ಟು ಕಾರ್ಯಕ್ರಮಗಳಿದ್ದರೂ ನಮ್ಮ ಕರೆಗೆ ಓಗೊಟ್ಟು ಸ್ವಲ್ಪ ಸಮಯದ ಮಟ್ಟಿಗಾದರೂ ತಾವು ನಮ್ಮಲ್ಲಿಗೆ ಬರಲು ಒಪ್ಪಿದುದು. ಇನ್ನು, ಗೌರವಕ್ಕೆ ಕಾರಣ ತಮ್ಮಲ್ಲಿರುವ ಅತ್ಯಂತ ಉದಾತ್ತವೂ ಶ್ರೇಷ್ಠವೂ ಆದ ಗುಣಗಳು, ಮತ್ತು ಜಗತ್ತಿಗೆ ಕಾಣುವಂತಹ ದಕ್ಷತೆ, ಉತ್ಸಾಹ ಮತ್ತು ಪ್ರಾಮಾಣಿಕತೆಗಳಿಂದ ತಾವು ಅತ್ಯಂತ ಮಹತ್ತರವಾದ ಕಾರ್ಯಗಳನ್ನು ಕೈಗೊಂಡಿರುವಿರಿ.

ಪಶ್ಚಿಮ ರಾಷ್ಟ್ರಗಳ ಸುಸಂಸ್ಕೃತ ಹೃದಯಗಳಲ್ಲಿ ತಾವು ಬಿತ್ತಿದ ಹಿಂದೂ ದರ್ಶನದ ಬೀಜಗಳು ಬಹು ಬೇಗನೆ ಫಲವನ್ನು ನೀಡುತ್ತಿರುವುದನ್ನು ಕಂಡು ನಮಗೆ ನಿಜವಾಗಿಯೂ ಸಂತೋಷವಾಗಿದೆ. ನಮ್ಮ ಸುತ್ತ ಯಶಸ್ಸಿನ ಬೆಳಕು ಹರಿಯುತ್ತಿರುವುದನ್ನು ನಾವು ಈಗಾಗಲೇ ಕಾಣುತ್ತಿದ್ದೇವೆ. ಈ ಆರ್ಯಾವರ್ತದಲ್ಲಿ ಕೂಡ ನಮ್ಮ ತಾಯಿನಾಡಿನ ಸಹೋದರರು ದೀರ್ಘಕಾಲದ ನಿದ್ರೆಯಿಂದ ಎಚ್ಚೆತ್ತು, ಬಹುಕಾಲ ಅವರು ಮರೆತಿದ್ದ ಸತ್ಯದ ಸಂದೇಶವನ್ನು ಮತ್ತೆ ನೆನಪಿಗೆ ತಂದುಕೊಳ್ಳುವಂತೆ ಮಾಡಲು ತಾವು ಪಶ್ಚಿಮದಲ್ಲಿ ಪಟ್ಟ ಶ್ರಮಕ್ಕಿಂತಲೂ ಹೆಚ್ಚಿನ ಶ್ರಮವನ್ನು ವಹಿಸಬೇಕೆಂದು ನಾವು ತಮ್ಮನ್ನು ಅತ್ಯಂತ ವಿನಯದಿಂದ ಬೇಡಿಕೊಳ್ಳುತ್ತೇವೆ.

ಪೂಜ್ಯರಾದ ತಮ್ಮ ಬಗ್ಗೆ ಅತ್ಯಂತ ಆತ್ಮೀಯತೆಯೂ ಭಕ್ತಿಯೂ ಮೆಚ್ಚುಗೆಯೂ ನಮ್ಮ ಹೃದಯಗಳಲ್ಲಿ ತುಂಬಿ ತುಳುಕುತ್ತಿದೆ. ತಾವು ನಮ್ಮ ಮಹಾನ್​ ಆಧ್ಯಾತ್ಮಿಕ ನಾಯಕರು. ನಮ್ಮ ಭಾವನೆಗಳನ್ನು ಸಮರ್ಪಕವಾಗಿ ವ್ಯಕ್ತಗೊಳಿಸುವುದು ಅಸಾಧ್ಯವೆಂಬುದನ್ನು ನಾವು ಕಂಡುಕೊಂಡಿದ್ದೇವೆ. ಆದುದರಿಂದ ಅತ್ಯಂತ ದಯಾಮಯನಾದ ಭಗವಂತನು ತಮಗೆ ದೀರ್ಘಾಯುಷ್ಯವನ್ನು ನೀಡಲಿ ಎಂದು ಪ್ರಾರ್ಥಿಸುತ್ತೇವೆ. ವಿಶ್ವಭ್ರಾತೃತ್ವವನ್ನು ಸಾಧಿಸಲು ತಾವು ನಡೆಸುತ್ತಿರುವ ಪ್ರಯತ್ನಗಳಿಗೆ ಅಗತ್ಯವಾದ ಶಕ್ತಿಯನ್ನು ಭಗವಂತನು ನೀಡಲಿ ಎಂದೂ ನಾವು ಪ್ರಾರ್ಥಿಸುತ್ತೇವೆ.

ಮೇಲಿನ ಬಿನ್ನವತ್ತಳೆಯ ಮಾತುಗಳಿಗೆ ಮಹಾರಾಜರು ತಮ್ಮ ವೈಯಕ್ತಿಕ ಸ್ವಾಗತದಲ್ಲಿ ಅತ್ಯಂತ ಆತ್ಮೀಯತೆಯಿಂದ ಕೂಡಿದ ಮಾತುಗಳನ್ನು ಸೇರಿಸಿದರು.

ಅದರ ನಂತರ ಸ್ವಾಮೀಜಿಯವರು ಈ ರೀತಿ ಉತ್ತರಿಸಿದರು:

ನಮ್ಮ ಪವಿತ್ರ ಭಾರತವರ್ಷ ಧರ್ಮಗಳ ಮತ್ತು ದರ್ಶನಗಳ ಪುಣ್ಯಭೂಮಿ; ವಿಖ್ಯಾತನಾಮರಾದ ಮಹಾತ್ಮರು, ಋಷಿಗಳು ಜನ್ಮವೆತ್ತ ನಾಡು, ತ್ಯಾಗಭೂಮಿ. ಕೇವಲ ಇಲ್ಲಿ ಮಾತ್ರವೇ ಪುರಾತನ ಕಾಲದಿಂದಲೂ ಇಂದಿನ ಅತ್ಯಾಧುನಿಕ\break ಕಾಲದವರೆಗೂ ಮಾನವನ ಎದುರಿಗೆ ಸರ್ವೋಚ್ಚ ಆದರ್ಶ ಇರುವುದು.

ನಾನು ಪಾಶ್ಚಾತ್ಯ ದೇಶಗಳಲ್ಲಿ ಸಂಚರಿಸಿರುವೆನು, ಹಲವು ದೇಶಗಳನ್ನು, ಜನಾಂಗಗಳನ್ನು ನೋಡಿರುವೆನು. ಪ್ರತಿಯೊಂದು ದೇಶ, ಜನಾಂಗಕ್ಕೂ ವಿಶೇಷವಾದ ಒಂದು ಆದರ್ಶವಿರುವುದನ್ನು ನಾನು ಕಂಡಿದ್ದೇನೆ. ಆ ಆದರ್ಶವು ರಾಷ್ಟ್ರೀಯ ಜೀವನವನ್ನೆಲ್ಲಾ ವ್ಯಾಪಿಸಿರುತ್ತದೆ. ಈ ಆದರ್ಶವೇ ರಾಷ್ಟ್ರದ ಜೀವಾಳ. ಭರತ ಖಂಡದ ಜೀವಾಳ ರಾಜನೀತಿಯಲ್ಲ, ಸೈನ್ಯಶಕ್ತಿಯಲ್ಲ, ಆರ್ಥಿಕ ಅಥವಾ ಯಾಂತ್ರಿಕ ಪರಮಾಧಿಕಾರವೂ ಅಲ್ಲ. ಧರ್ಮ ಒಂದೇ ನಮ್ಮ ಜೀವಾಳ, ಅದನ್ನೇ ನಾವು ಇಟ್ಟುಕೊಳ್ಳುವೆವು. ಅಧ್ಯಾತ್ಮವೇ ಸದಾಕಾಲದಲ್ಲಿಯೂ ಭಾರತದಲ್ಲಿದ್ದುದು ಕಂಡು ಬರುತ್ತದೆ.

ಶಾರೀರಿಕ ಶಕ್ತಿಯ ಮೂಲಕ ಅದ್ಭುತ ಕಾರ್ಯಗಳಾಗಿವೆ. ವೈಜ್ಞಾನಿಕ ನಿಯಮಾ ವಳಿಗಳ ಮೂಲಕ ಯಂತ್ರಗಳನ್ನು ಕಂಡುಹಿಡಿದಿರುವ ಧೀಶಕ್ತಿ ಪ್ರಚಂಡವಾದುದು. ಆದರೆ ಅಧ್ಯಾತ್ಮವು ತನ್ನ ಪ್ರಭಾವವನ್ನು ಜಗತ್ತಿನ ಮೇಲೆ ಬೀರಿದಷ್ಟು ಉಳಿದವು ಬೀರಲು ಸಾಧ್ಯವಾಗಿಲ್ಲ.

ಭಾರತವರ್ಷ ಈ ಕ್ಷೇತ್ರದಲ್ಲಿ ಸದಾ ಜಾಗೃತವಾಗಿತ್ತು ಎಂಬುದನ್ನು ನಮ್ಮ ಜನಾಂಗದ ಚರಿತ್ರೆಯು ತೋರಿಸುತ್ತದೆ. ಹೆಚ್ಚೇನು ತಿಳಿವಳಿಕೆ ಇಲ್ಲದ ಜನರು ಹಿಂದೂಗಳು ಮೃದು ಸ್ವಭಾವದವರು, ನಿಶ್ಚೇಷ್ಟರು ಎಂದು ಹೇಳುತ್ತಿರುವರು. ಇತರ ದೇಶದವರು ಕೂಡ ಇದನ್ನು ಒಂದು ನಾಣ್ಣುಡಿಯಂತೆ ಹೇಳುತ್ತಿರುವರು. ಹಿಂದೂಗಳು ನಿಶ್ಚೇಷ್ಟರು ಎಂಬುದನ್ನು ನಾನು ವಿರೋಧಿಸುತ್ತೇನೆ. ನಮ್ಮ ಪುಣ್ಯ ಭೂಮಿಯಲ್ಲಿರುವಷ್ಟು ಚಟುವಟಿಕೆ ಮತ್ತಾವ ದೇಶದಲ್ಲಿಯೂ ಇರಲಿಲ್ಲ. ಅತಿ ಪುರಾತನ ಉದಾರಹೃದಯದ ಈ ಜನಾಂಗವು ಇನ್ನೂ ಜೀವಂತವಾಗಿರುವುದೇ ಇದಕ್ಕೆ ಸಾಕ್ಷಿ. ನಮ್ಮ ಜನಾಂಗ ತನ್ನ ಉಜ್ವಲ ಜೀವನಕಾಲದಲ್ಲಿ ಕಾಲಕಾಲಕ್ಕೆ ಪುನಃ ಪುನಃ ಚಿರಯೌವನವನ್ನು ಧರಿಸಿರುವುದು ನಮಗೆ ತೋರುವುದು. ರಾಷ್ಟ್ರದ ಚಟುವಟಿಕೆಯು ಧಾರ್ಮಿಕ ಜೀವನದಲ್ಲಿ ಪ್ರಕಟವಾಗುತ್ತದೆ. ಆದರೆ ಮಾನವರ ಒಂದು ವಿಚಿತ್ರ ಸ್ವಭಾವವೇನೆಂದರೆ, ಪ್ರತಿಯೊಬ್ಬರೂ ತಮ್ಮ ಚಟುವಟಿಕೆಗಳ ಮಾನದಂಡದಿಂದ ಮತ್ತೊಬ್ಬರನ್ನು ಅಳೆಯುವುದು. ಒಬ್ಬ ಮೋಚಿಯನ್ನು ತೆಗೆದುಕೊಳ್ಳಿ. ಅವನಿಗೆ ಜೋಡು ಹೊಲಿಯುವುದೊಂದೇ ಅರ್ಥವಾಗುತ್ತದೆ; ಪ್ರಪಂಚದಲ್ಲಿ ಅದಲ್ಲದೆ ಬೇರೆ ಕೆಲಸವೇ ಇಲ್ಲ ಎಂದು ಅವನು ಭಾವಿಸುವನು. ಇಟ್ಟಿಗೆ ಸುಡುವವನಿಗೆ ಆ ಕೆಲಸವಲ್ಲದೆ ಬೇರೆ ಯಾವುದೂ ಗೊತ್ತಾಗುವುದೇ ಇಲ್ಲ. ಅವನ ದೈನಂದಿನ ಜೀವನವು ಅದನ್ನೇ ಪ್ರತಿಪಾದಿಸುತ್ತದೆ. ಇದನ್ನು ವಿವರಿಸಲು ಮತ್ತೊಂದು ಕಾರಣವಿದೆ. ಜ್ಯೋತಿಯ ಸ್ಪಂದನ ಹೆಚ್ಚು ತೀವ್ರವಾದಾಗ ನಾವು ಅದನ್ನು ಕಾಣಲಾರೆವು. ಏಕೆಂದರೆ ನಾವು ನಮ್ಮ ದೃಷ್ಟಿಯ ಮಟ್ಟಕ್ಕೆ ಅತೀತವಾದುದನ್ನು ಕಾಣಲಾರೆವು. ಸಾಮಾನ್ಯ ಜನರು ನೋಡುತ್ತಿರುವ ಜಡವಾದದ ತೆರೆಯನ್ನು ಸೀಳಿ ನೋಡಬಲ್ಲುದು ಯೋಗಿಯ ಅಂತಮುರ್ಖವಾದ ಅಧ್ಯಾತ್ಮ ಚಕ್ಷುಸ್ಸು.

ಆಧ್ಯಾತ್ಮಿಕ ಆಹಾರಕ್ಕಾಗಿ ಇಡಿಯ ಜಗತ್ತು ಭರತಖಂಡದ ಕಡೆಗೆ ನೋಡುತ್ತಿದೆ. ಭರತಖಂಡವು ಎಲ್ಲಾ ಜನಾಂಗಗಳಿಗೂ ಇದನ್ನು ಒದಗಿಸಬೇಕಾಗಿದೆ. ಮಾನವಕೋಟಿಗೆ ಅತಿಶ್ರೇಷ್ಠ ಆದರ್ಶ ಇಲ್ಲಿ ಮಾತ್ರ ಇದೆ. ಸಂಸ್ಕೃತ ಸಾಹಿತ್ಯದಲ್ಲಿ ಮತ್ತು ದರ್ಶನಗಳಲ್ಲಿ ಹುದುಗಿರುವ, ನಮ್ಮ ಭರತಭೂಮಿಯ ಒಂದು ಸ್ವಭಾವವಾಗಿರುವ, ಈ ಆದರ್ಶವನ್ನು ಪಾಶ್ಚಾತ್ಯ ವಿದ್ವಾಂಸರು ಈಗ ತಿಳಿದುಕೊಳ್ಳಲು ಪ್ರಯತ್ನಿಸುತ್ತಿರುವರು.

ಇತಿಹಾಸದ ಪ್ರಾರಂಭದಿಂದಲೂ ಭರತಖಂಡದ ಹೊರಗೆ ಹಿಂದೂ ಧರ್ಮವನ್ನು ಪ್ರಚಾರ ಮಾಡಲು ಯಾರೂ ಹೋಗಲಿಲ್ಲ. ಆದರೆ ಈಗ ಒಂದು ಅದ್ಭುತ ಬದಲಾವಣೆ ಆಗುತ್ತಿದೆ. ಭಗವಾನ್​ ಶ‍್ರೀಕೃಷ್ಣ “ಎಂದು ಧರ್ಮ ನಷ್ಟವಾಗುವುದೋ, ಅಧರ್ಮ ತಲೆಯೆತ್ತುವುದೋ ಆಗ ಮಾನವನ ಉದ್ಧಾರಕ್ಕೆ ಪುನಃ ಪುನಃ ಜನ್ಮವೆತ್ತುವೆನು” ಎಂದು ಸಾರಿದ್ದಾನಷ್ಟೆ. ಧಾರ್ಮಿಕ ಕ್ಷೇತ್ರದಲ್ಲಿ ನಡೆದಿರುವ ಸಂಶೋಧನೆಗಳಿಂದ ನಮಗೆ ಗೊತ್ತಾಗುವುದೇನೆಂದರೆ, ಉದಾತ್ತ ನೈತಿಕ ನಿಯಮಗಳನ್ನು ಹೊಂದಿರುವ ಯಾವುದೇ ದೇಶವಾದರೂ ಅವುಗಳ ಸ್ವಲ್ಪಾಂಶಗಳನ್ನಾದರೂ ನಮ್ಮಿಂದಲೇ ತೆಗೆದುಕೊಂಡಿವೆ ಎಂಬುದು. ಆತ್ಮದ ಅಮರತ್ವವನ್ನು ಸಾರುವ ಯಾವ ಅನ್ಯಧರ್ಮವಾಗಲಿ, ಪ್ರತ್ಯಕ್ಷವಾಗಿ ಅಥವಾ ಪರೋಕ್ಷವಾಗಿ, ಆ ಭಾವನೆಗಾಗಿ ನಮಗೆ ಚಿರಋಣಿ.

ಪ್ರಪಂಚದ ಇತಿಹಾಸದಲ್ಲಿ ಹತ್ತೊಂಬತ್ತನೆಯ ಶತಮಾನದ ಅಂತ್ಯದಲ್ಲಿರುವಷ್ಟು ದರೋಡೆ, ಕ್ರೌರ್ಯ, ಬಲಾಢ್ಯರು ಬಲಹೀನರ ಮೇಲೆ ಪ್ರಯೋಗಿಸುವ ದಬ್ಬಾಳಿಕೆ ಎಂದಿಗೂ ಇರಲಿಲ್ಲ. ಆಸೆಯನ್ನು ಗೆಲ್ಲದೆ ಯಾರಿಗೂ ಮುಕ್ತಿ ಲಭಿಸದು ಎಂಬುದನ್ನು ಎಲ್ಲರೂ ಅರಿಯಬೇಕು. ಯಾರು ಜಡ ವಸ್ತುಗಳ ಅಧೀನದಲ್ಲಿರುವರೋ ಅವರು ಸ್ವತಂತ್ರರಲ್ಲ. ಈ ಮಹಾ ಸತ್ಯವನ್ನು ಎಲ್ಲಾ ದೇಶಗಳೂ ಕ್ರಮೇಣ ಅರಿಯುತ್ತಿವೆ, ಶ್ಲಾಘಿಸುತ್ತಿವೆ. ಸತ್ಯವನ್ನು ತಿಳಿಯುವ ಸ್ಥಿತಿಗೆ ಶಿಷ್ಯನು ಬಂದರೆ ಗುರುವಿನ ಬೋಧನೆ ಅವನ ನೆರವಿಗೆ ಬರುತ್ತದೆ. ಭಗವಂತನು ಅಪಾರ ಕರುಣೆ\-ಯಿಂದ ತನ್ನ ಮಕ್ಕಳಿಗೆ ಸಹಾಯವನ್ನು ಕರುಣಿಸುವನು. ಈ ದಯೆ ಯಾವ ಧರ್ಮದಲ್ಲಿಯೂ ಬತ್ತುವುದಿಲ್ಲ. ಯಾವಾಗಲೂ ಪೂರ್ಣವಾಗಿ ಹರಿಯುತ್ತಿರುವುದು. ನಮ್ಮ ದೇವರೇ ಎಲ್ಲಾ ಧರ್ಮಗಳ ದೇವರು. ಈ ಉದಾರ ಭಾವನೆ ಭರತಖಂಡಕ್ಕೆ ಮಾತ್ರ ಸೇರಿದ್ದು. ಪ್ರಪಂಚದ ಇತರ ಯಾವುದೇ ಧರ್ಮಶಾಸ್ತ್ರದಲ್ಲಿ ಇದು ಇರುವುದಾದರೆ ಇದನ್ನು ತೋರಿಸಿ ನೋಡೋಣ.

ದೈವಾನುಗ್ರಹದಿಂದ ಹಿಂದೂಗಳು ಇಂದು ಒಂದು ಸಂದಿಗ್ಧ ಸ್ಥಿತಿಯಲ್ಲಿ ಇರುವರು; ಒಂದು ಜವಾಬ್ದಾರಿಯನ್ನು ಹೊತ್ತಿರುವರು. ಆಧ್ಯಾತ್ಮಿಕ ಸಹಾಯಕ್ಕಾಗಿ ಪಾಶ್ಚಾತ್ಯರು ನಮ್ಮ ಬಳಿಗೆ ಬರುತ್ತಿರುವರು. ಮಾನವ ಅಸ್ತಿತ್ವದ ಸಮಸ್ಯೆಗಳನ್ನು ಬಗೆಹರಿಸಲು ಸಿದ್ಧರಾಗಬೇಕಾದಂತಹ ಒಂದು ಗುರುತರ ಕರ್ತವ್ಯ ಭರತಖಂಡದ ಮಕ್ಕಳ ಮೇಲೆ ಬಿದ್ದಿದೆ. ಅನ್ಯದೇಶಗಳ ಪ್ರಖ್ಯಾತ ಪುರುಷರು, ಗುಡ್ಡಗಾಡುಗಳಲ್ಲಿ ಇದ್ದು ಪ್ರಯಾಣಿಕರನ್ನು ಸುಲಿದು ದರೋಡೆಮಾಡುತ್ತಿದ್ದ ಪಾಳೆಗಾರರ ವಂಶಕ್ಕೆ ತಾವು ಸೇರಿದವರೆಂದು ಹೇಳಿಕೊಳ್ಳಲು ಹೆಮ್ಮೆಪಡುವರು. ಆದರೆ ಹಿಂದೂಗಳು, ಗಿರಿಗುಹೆಗಳಲ್ಲಿ ವಾಸಿಸುತ್ತ ಗೆಡ್ಡೆಗೆಣಸುಗಳನ್ನು ತಿಂದು ಬ್ರಹ್ಮಚಿಂತನೆಯಲ್ಲಿ ಕಾಲಕಳೆಯುತ್ತಿದ್ದ ಋಷಿಗಳ ಕುಲಕ್ಕೆ ಸೇರಿದವರು ತಾವೆಂದು ಹೇಳಿಕೊಳ್ಳಲು ಹೆಮ್ಮೆ ತಾಳುವರು. ನಾವು ಈಗ ಅಧೋಗತಿಗೆ ಇಳಿದಿರಬಹುದು, ಅಧಮರಾಗಿರಬಹುದು, ಆದರೆ ನಮ್ಮ ಧರ್ಮಕ್ಕಾಗಿ ನಾವು ಶ್ರದ್ಧೆಯಿಂದ ಕೆಲಸ ಮಾಡಿದರೆ ಪುನಃ ಪ್ರಖ್ಯಾತರಾಗಬಲ್ಲೆವು.

ನೀವು ನನಗೆ ನೀಡಿರುವ ಆತ್ಮೀಯವೂ, ಶ್ರದ್ಧಾಪೂರ್ವಕವೂ ಆದ ಸ್ವಾಗತಕ್ಕೆ ನನ್ನ ಹೃತ್ಪೂರ್ವಕ ಧನ್ಯವಾದವನ್ನು ಸ್ವೀಕರಿಸಿ. ರಾಮನಾಡಿನ ರಾಜರು ನನಗೆ ತೋರಿದ ಪ್ರೀತಿಗೆ ಕೃತಜ್ಞತೆಯನ್ನು ತೋರಲು ಅಸಾಧ್ಯ. ನನ್ನಿಂದ ಮತ್ತು ನನ್ನ ಮೂಲಕ ಏನಾದರೂ ಒಳ್ಳೆಯ ಕೆಲಸವಾಗಿದ್ದರೆ ಭರತಖಂಡವು ಈ ರಾಜರಿಗೆ ಹೆಚ್ಚು ಋಣಿ. ನಾನು ಚಿಕಾಗೋ ನಗರಕ್ಕೆ ಹೋಗಬೇಕೆಂಬ ಕಲ್ಪನೆ ಮೂಡಿದ್ದು ಇವರ ಮನಸ್ಸಿನಲ್ಲಿಯೇ. ಇವರು ಆ ಭಾವನೆಯನ್ನು ನನ್ನ ತಲೆಯೊಳಗೆ ಹಾಕಿ ಅದನ್ನು ನೆರವೇರಿಸಬೇಕೆಂದು ಪುನಃ ಪುನಃ ಒತ್ತಾಯ ಮಾಡಿದವರು. ಹಿಂದಿನ ಶ್ರದ್ಧೆ, ಉತ್ಸಾಹಗಳಿಂದಲೇ ಇಂದು ನನ್ನ ಪಕ್ಕದಲ್ಲಿ ನಿಂತು ನಾನು ಇನ್ನೂ ಹೆಚ್ಚಿನ ಕಾರ್ಯವನ್ನು ಮಾಡಬೇಕೆಂದು ಇವರು ಪ್ರೋತ್ಸಾಹಿಸುತ್ತಿರುವರು. ನಮ್ಮ ಮಾತೃಭೂಮಿಯ ಕಲ್ಯಾಣಕ್ಕಾಗಿ ಆಧ್ಯಾತ್ಮಿಕ ಕ್ಷೇತ್ರದಲ್ಲಿ ಶ್ರದ್ಧೆವಹಿಸಿ ದುಡಿಯಬಲ್ಲ ಇಂತಹ ಹತ್ತಾರು ಮಂದಿ ರಾಜರಿರಲಿ ಎಂಬುದೇ ನನ್ನ ಆಶಯ.

