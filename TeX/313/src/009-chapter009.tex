
\chapter{ವೇದಾಂತದ ಧ್ಯೇಯ}

\begin{center}
(ಕುಂಭಕೋಣದ ಬಿನ್ನವತ್ತಳೆಗೆ ಉತ್ತರ)
\end{center}

ಸ್ವಾಮಿ ವಿವೇಕಾನಂದರು ಕುಂಭಕೋಣಕ್ಕೆ ಭೇಟಿ ನೀಡಿದ ಸಂದರ್ಭದಲ್ಲಿ ಅಲ್ಲಿಯ ಹಿಂದೂಗಳು ಈ ಕೆಳಗಿನ ಬಿನ್ನವತ್ತಳೆಯನ್ನು ಅರ್ಪಿಸಿದರು:

\textbf{ಪರಮಪೂಜ್ಯ ಸ್ವಾಮೀಜಿಯವರೇ–}

ಧಾರ್ಮಿಕ ಪ್ರಾಮುಖ್ಯವನ್ನು ಪಡೆದಿರುವ ಪುರಾತನ ನಗರವಾದ ಈ ಕುಂಭ\-ಕೋಣದ ಹಿಂದೂಗಳ ಪರವಾಗಿ, ಶ್ರೇಷ್ಠ ದೇವಾಲಯಗಳ ಮತ್ತು ಪ್ರಸಿದ್ಧ ಸಂತ ಮಹಾಪುರುಷರ ನಮ್ಮ ಈ ನಾಡಿಗೆ, ಪಶ್ಚಿಮ ದೇಶಗಳಿಂದ ಹಿಂದಿರುಗಿರುವ ತಮ್ಮನ್ನು ಹೃತ್ಪೂರ್ವಕವಾಗಿ ಸ್ವಾಗತಿಸುತ್ತೇವೆ. ಅಮೆರಿಕ ಮತ್ತು ಯೂರೋಪುಗಳಲ್ಲಿ ತಮ್ಮ ಧಾರ್ಮಿಕ ಕಾರ್ಯಕ್ಕೆ ದೊರೆತ ಯಶಸ್ಸಿಗಾಗಿ ಭಗವಂತನಿಗೆ ನಾವು ಅತ್ಯಂತ ಕೃತಜ್ಞ\-ರಾಗಿದ್ದೇವೆ. ಹಿಂದೂಧರ್ಮ ಮತ್ತು ದರ್ಶನಗಳು ಎಲ್ಲ ಭಗವದ್ಭಾವನೆ ಗಳನ್ನೂ ಮಾನವಕೋಟಿಯ ಎಲ್ಲ ಆಧ್ಯಾತ್ಮಿಕ ಚಿಂತನೆಗಳನ್ನೂ ಉದಾತ್ತೀಕರಿಸಿ ಸಮನ್ವಯ\-ಗೊಳಿಸುವಷ್ಟು ವಿಶಾಲವೂ ವೈಚಾರಿಕವೂ ಆಗಿವೆ ಎಂಬುದನ್ನು ಚಿಕಾಗೊದಲ್ಲಿ ನೆರೆದಿದ್ದ ಜಗತ್ತಿನ ಶ್ರೇಷ್ಠ ಧರ್ಮಗಳ ಪರಮ ಪ್ರಮುಖ ಪ್ರತಿನಿಧಿಗಳು ಅರ್ಥಮಾಡಿಕೊಳ್ಳುವಂತೆ ತಾವು ಮಾಡಿದ್ದೀರಿ. 

ಜಗತ್ತಿನ ಆತ್ಮವಾದ ಭಗವಂತನು ಸತ್ಯವನ್ನು ಯಾವಾಗಲೂ ಎತ್ತಿಹಿಡಿಯುತ್ತಾನೆ ಎಂಬ ದೃಢವಿಶ್ವಾಸವನ್ನು ಸಾವಿರಾರು ವರ್ಷಗಳಿಂದಲೂ ನಾವು ಉಳಿಸಿಕೊಂಡು ಬಂದಿದ್ದೇವೆ. ಇಂದು ಕ್ರೈಸ್ತ ದೇಶದಲ್ಲಿ ತಾವು ಸಾಧಿಸಿದ ಕಾರ್ಯಕ್ಕಾಗಿ ನಾವು ಆನಂದಿಸುತ್ತೇವೆ. ಏಕೆಂದರೆ ತಮ್ಮ ಈ ಸಾಧನೆಯಿಂದಾಗಿ ಪ್ರಮುಖವಾಗಿ ಧಾರ್ಮಿಕ ನಾಡಾದ ನಮ್ಮ ದೇಶದ ಅಮೂಲ್ಯ ಆಧ್ಯಾತ್ಮಿಕ ಪರಂಪರೆಯ ಮೌಲ್ಯವನ್ನು ಭಾರತದ ಒಳಗಿನ ಮತ್ತು ಹೊರಗಿನ ಜನರು ಗುರುತಿಸುವಂತಾಗಿದೆ. ತಮ್ಮ ಕಾರ್ಯವು ಸಾಧಿಸಿದ ಯಶಸ್ಸಿನಿಂದಾಗಿ ಈಗಾಗಲೇ ಖ್ಯಾತಿಯನ್ನು ಪಡೆದಿರುವ ತಮ್ಮ ಶ್ರೇಷ್ಠ ಗುರುದೇವರ ಹೆಸರು ಇನ್ನೂ ಹೆಚ್ಚು ಬೆಳಗುವಂತೆ ಆಯಿತು. ನಾಗರಿಕ ಜಗತ್ತಿನಲ್ಲಿ ನಮ್ಮ ಹಿರಿಮೆಯೂ ಹೆಚ್ಚುವಂತೆ ಆಯಿತು, ಮತ್ತು ಎಲ್ಲಕ್ಕಿಂತ ಹೆಚ್ಚಾಗಿ, ನಾವೂ ನಮ್ಮ ಹಿಂದಿನ ಸಾಧನೆಯ ಬಗ್ಗೆ ಹೆಮ್ಮೆಪಡುವುದಕ್ಕೆ ಕಾರಣವಿದೆ ಮತ್ತು ನಮ್ಮ ನಾಗರಿಕತೆಯಲ್ಲಿರುವ ಆಕ್ರಮಣ ಪ್ರವೃತ್ತಿಯ ಕೊರತೆಯು ಸತ್ತ್ವ ಹೀನತೆಯ ಚಿಹ್ನೆಯಲ್ಲ ಎಂಬುದು ನಮಗೆ ವೇದ್ಯವಾಗುವಂತಾಯಿತು. ತಮ್ಮಂತಹ ಸ್ಪಷ್ಟದೃಷ್ಟಿಯುಳ್ಳ, ನಿಷ್ಠಾವಂತ ನಿಃಸ್ವಾರ್ಥ ಕಾರ್ಯಶೀಲರು ನಮ್ಮೊಂದಿಗಿರುವಾಗ ಹಿಂದೂ ರಾಷ್ಟ್ರದ ಭವಿಷ್ಯವು ಆಶಾದಾಯಕವಾಗಿದೆ ಎಂಬುದರಲ್ಲಿ ಸಂದೇಹವಿಲ್ಲ. ಜಗದೀಶ್ವರನಾದ ಭಗವಂತನು ತಮಗೆ ಆರೋಗ್ಯವನ್ನು ದೀರ್ಘಾಯುಷ್ಯವನ್ನು ನೀಡಲಿ ಮತ್ತು ಹಿಂದೂಧರ್ಮದ ಸಮರ್ಥ ಬೋಧಕರಾಗಿ ಉದಾತ್ತ ಮತ್ತು ಶ್ರೇಷ್ಠ ಕಾರ್ಯಗಳನ್ನು ಎಸಗಲು ಶಕ್ತಿಯನ್ನು ನೀಡಲಿ ಎಂದು ಪ್ರಾರ್ಥಿಸುತ್ತೇವೆ. 

ಅನಂತರ ನಗರದ ಹಿಂದೂ ವಿದ್ಯಾರ್ಥಿಗಳು ಇನ್ನೊಂದು ಬಿನ್ನವತ್ತಳೆಯನ್ನು ಅರ್ಪಿಸಿದರು. ಅನಂತರ ಸ್ವಾಮೀಜಿಯವರು ಈ ಕೆಳಗಿನ ಉಪನ್ಯಾಸವನ್ನು ನೀಡಿದರು. 

“ಸ್ವಲ್ಪ ಪ್ರಮಾಣದ ಧರ್ಮಾಚರಣೆಯಿಂದಲೂ ಮಹತ್​ ಪರಿಣಾಮವುಂಟಾ ಗುತ್ತದೆ,” ಎಂಬ ಗೀತೆಯ ಸಂದೇಶಕ್ಕೆ ನಿದರ್ಶನ ಬೇಕಾದರೆ ಅದನ್ನು ನನ್ನ ಅಲ್ಪ ಜೀವನದಲ್ಲೇ ಪ್ರತಿದಿನ ನೋಡುತ್ತಿರುವೆನು. ನಾನು ಮಾಡಿದ ಕೆಲಸ ಅತ್ಯಲ್ಪ. ಆದರೆ ಕೊಲಂಬೊ ನಗರದಿಂದ ಇಲ್ಲಿಯವರೆಗೆ ನನ್ನ ಪ್ರಯಾಣದ ಪ್ರತಿಯೊಂದು ಹಂತದಲ್ಲಿಯೂ ಸಿಕ್ಕಿದ ಹಾರ್ದಿಕ ದಯಾಪೂರಿತ ಸ್ವಾಗತವು ನನ್ನ ಎಲ್ಲಾ ನಿರೀಕ್ಷಣೆಯನ್ನು ಮೀರಿದ್ದಾಗಿದೆ. ಇದು ಹಿಂದೂ ಸಂಪ್ರದಾಯಕ್ಕೆ ಯೋಗ್ಯವಾಗಿಯೇ ಇದೆ. ನಮ್ಮ ಜನಾಂಗಕ್ಕೆ ತಕ್ಕ ಗೌರವವನ್ನು ತರತಕ್ಕದ್ದಾಗಿದೆ. ಏಕೆಂದರೆ ಹಿಂದೂಗಳಾದ ನಮ್ಮ ಜನಾಂಗದ ಶಕ್ತಿ, ಜೀವಾಳ ಮತ್ತು ಆತ್ಮವೇ ಧರ್ಮ ಮೂಲದಲ್ಲಿವೆ. ನಾನು ಒಂದಿಷ್ಟು ಪ್ರಪಂಚವನ್ನು ನೋಡಿರುವೆನು. ಪೂರ್ವ ಮತ್ತು ಪಾಶ್ಚಾತ್ಯ ದೇಶಗಳಲ್ಲಿ ಸಂಚರಿಸಿರುವೆನು. ಪ್ರತಿಯೊಂದು ಜನಾಂಗದಲ್ಲಿಯೂ ಯಾವುದಾದರೊಂದು ಆದರ್ಶ ಅದರ ಬೆನ್ನೆಲುಬಿನಂತಿರುವುದನ್ನು ನಾನು ಕಂಡಿರುವೆನು. ಕೆಲವರಲ್ಲಿ ಅದು ರಾಜಕೀಯವಾಗಿದ್ದರೆ, ಕೆಲವರಲ್ಲಿ ಸಾಮಾಜಿಕ ಸಂಸ್ಕೃತಿಯಾಗಿರುತ್ತದೆ, ಮತ್ತೆ ಕೆಲವರಲ್ಲಿ ಅದು ಬೌದ್ಧಿಕ ಸಂಸ್ಕೃತಿಯಾಗಿರಬಹುದು. ಆದರೆ ನಮ್ಮ ಮಾತೃಭೂಮಿಯಾದ ಭರತಖಂಡದಲ್ಲಿ ಆ ಮೂಲ ತಳಹದಿಯೇ ಧರ್ಮ. ಧರ್ಮವೇ ಅದರ ಬೆನ್ನೆಲುಬು; ಧರ್ಮದ ಆಧಾರದ ಮೇಲೆಯೇ ಇಡೀ ರಾಷ್ಟ್ರ ಜೀವನದ ಸೌಧವು ಕಟ್ಟಲ್ಪಟ್ಟಿದೆ. ಅಮೆರಿಕಾದಿಂದ ಮದ್ರಾಸಿನ ಬಿನ್ನವತ್ತಳೆಗೆ ನಾನು ಬರೆದು ಕಳುಹಿದ ಉತ್ತರದಲ್ಲಿ, ಭಾರತ ದೇಶದ ಸಾಮಾನ್ಯ ಬೇಸಾಯಗಾರನಿಗೂ ಅನೇಕ ಪಾಶ್ಚಾತ್ಯ ಸಭ್ಯರಿ\-ಗಿಂತ ಧರ್ಮದ ವಿಷಯ ಹೆಚ್ಚು ಗೊತ್ತಿದೆ ಎಂದು ನಾನು ಹೇಳಿರುವುದು ನಿಮ್ಮಲ್ಲಿ ಕೆಲವರಿಗೆ ನೆನಪಿರಬಹುದು. ಇಂದು ನಿಸ್ಸಂಶಯವಾಗಿ ನಾನು ಅದನ್ನು ನೋಡುತ್ತಿರುವೆನು. ಭಾರತದ ಜನಸಾಮಾನ್ಯರಲ್ಲಿ ಕಂಡುಬರುವ ಸಾಮಾನ್ಯ ವಿಷಯಗಳ ಬಗ್ಗೆ \break ಮಾಹಿತಿಯ ಕೊರತೆ ಮತ್ತು ಮಾಹಿತಿ ಸಂಗ್ರಹದಲ್ಲಿ ಅವರಿಗಿರುವ ನಿರುತ್ಸಾಹ ಇವುಗಳನ್ನು ನೋಡಿ ನಾನು ಅಸಂತುಷ್ಟ ನಾಗಿದ್ದ ಕಾಲವೊಂದಿತ್ತು. ಆದರೆ ನನಗೆ ಇದು ಈಗ ತಿಳಿಯಿತು. ಎಲ್ಲಿ ಅವರಿಗೆ ಆಸಕ್ತಿಯಿರುವುದೊ ಆ ವಿಷಯದಲ್ಲಿ ಅವರು ನಾನು ನೋಡಿದ ಇತರ ಎಲ್ಲ ದೇಶದವರಿಗಿಂತ ಹೆಚ್ಚು ಕುತೂಹಲಿಗಳು. ಪಾಶ್ಚಾತ್ಯ ದೇಶಗಳಲ್ಲಿ ಆಗುತ್ತಿರುವ ಅದ್ಭುತ ರಾಜಕೀಯ ಬದಲಾವಣೆಗಳ ವಿಚಾರವಾಗಿ ನಮ್ಮವರನ್ನು ಕೇಳಿ, ಆ ವಿಷಯ ಅವರಿಗೆ ಏನೂ ಗೊತ್ತಿರುವುದಿಲ್ಲ, ಮತ್ತು ತಿಳಿಯುವುದಕ್ಕೆ ಇಷ್ಟವನ್ನೂ ಪಡುವುದಿಲ್ಲ. ಆದರೆ ಭಾರತದಿಂದ ಪ್ರತ್ಯೇಕವಾಗಿರುವ ಮತ್ತು ಭಾರತದ ವಿಷಯದಲ್ಲಿ ಆಸಕ್ತಿಯನ್ನು ತೋರಿಸದ ಸಿಲೋನಿನ ರೈತರಿಗೂ ಕೂಡ ಅಮೆರಿಕಾದಲ್ಲಿ ಒಂದು ವಿಶ್ವಧರ್ಮ ಸಮ್ಮೇಳನ ನಡೆಯಿತೆಂದೂ, ಅಲ್ಲಿಗೆ ಒಬ್ಬ ಹಿಂದೂ ಸಂನ್ಯಾಸಿ ಹೋಗಿ ಅಲ್ಲಿ ಆತನಿಗೆ ಸ್ವಲ್ಪ ಜಯ ಲಭಿಸಿತೆಂದೂ ತಿಳಿದಿದೆ. 

ಅವರ ಉತ್ಸಾಹ ಎಲ್ಲಿದೆಯೊ ಅಲ್ಲಿ ಅವರು ಇತರರಷ್ಟೇ ವಿಷಯ ಕುತೂಹಲಿಗಳಾಗಿರುವರು. ಧರ್ಮವೊಂದೇ ಭಾರತೀಯರ ಏಕಮಾತ್ರ ಆಸಕ್ತಿಯ ವಿಷಯ. ನಮ್ಮ ರಾಷ್ಟ್ರಜೀವಾಳ ಧರ್ಮದಲ್ಲಿರುವುದು ಒಳ್ಳೆಯದೆ, ರಾಜಕೀಯದಲ್ಲಿರುವುದು ಒಳ್ಳೆಯದೆ ಎಂಬ ವಿಚಾರವಾಗಿ ಈಗ ನಾನು ಮಾತನಾಡುತ್ತಿಲ್ಲ. ಒಳ್ಳೆಯದಕ್ಕೋ, ಕೆಟ್ಟದಕ್ಕೋ, ಅಂತೂ ನಮ್ಮ ಜನಾಂಗದ ಜೀವವು ಧರ್ಮದಲ್ಲಿದೆ. ಅದನ್ನು ಈಗ ಬದಲಾಯಿಸುವುದಕ್ಕೆ ಆಗುವುದಿಲ್ಲ. ಅದನ್ನು ನಾಶಮಾಡಿ ಬೇರೊಂದನ್ನು ಅಲ್ಲಿ ಪ್ರತಿಷ್ಠಾಪನೆ ಮಾಡಲಾರಿರಿ. ಒಂದು ದೊಡ್ಡ ಮರವನ್ನು ಒಂದು ಕಡೆಯಿಂದ ಕಿತ್ತು ಮತ್ತೊಂದು ಕಡೆ ನೆಟ್ಟು ತಕ್ಷಣ ಅದು ಅಲ್ಲಿ ಬೇರುಬಿಡುವಂತೆ ಮಾಡಲಾರಿರಿ. ಒಳ್ಳೆಯ\-ದಕ್ಕೊ, ಕೆಟ್ಟದಕ್ಕೊ, ಧರ್ಮದ ಆದರ್ಶ ಹಲವು ಸಹಸ್ರ ವರುಷಗಳಿಂದ ಭಾರತಕ್ಕೆ ಹರಿದು ಬರುತ್ತಿದೆ. ಒಳ್ಳೆಯದಕ್ಕೊ, ಕೆಟ್ಟದಕ್ಕೊ, ನೂರಾರು ವರ್ಷಗಳಿಂದ ಭಾರತದ ವಾತಾವರಣವು ಧಾರ್ಮಿಕ ಆದರ್ಶದಿಂದ ತುಂಬಿಹೋಗಿದೆ. ಒಳ್ಳೆಯದಕ್ಕೊ, ಕೆಟ್ಟದಕ್ಕೊ, ನಾವು ಅದೇ ವಾತಾವರಣದಲ್ಲಿ ಹುಟ್ಟಿ ಬೆಳೆದಿರುವೆವು. ಅದು ನಮ್ಮ ರಕ್ತದಲ್ಲಿ ಮಿಶ್ರವಾಗಿ, ನಮ್ಮ ನರನಾಡಿಗಳಲ್ಲಿ ಹರಿಯುತ್ತಾ ನಮ್ಮ ದೇಹದ ಅಂಗೋಪಾಂಗಗಳೊಡನೆ ಒಂದಾಗಿ ನಮ್ಮ ಬಾಳಿನ ಉಸಿರಾಗಿರುವುದು. ನೀವು ಅಂತಹ ಧರ್ಮವನ್ನು ಅಷ್ಟೇ ಶಕ್ತಿಪೂರ್ಣವಾದ ಪ್ರತಿಕ್ರಿಯೆಯನ್ನು ಉದ್ರೇಕಿಸದೆ ತ್ಯಜಿಸಬಲ್ಲಿರಾ? ಸಹಸ್ರಾರು ವರುಷಗಳಿಂದ ಹರಿದು ಬಂದಿರುವ ಆ ಮಹಾನದಿಯ ಪಾತ್ರವನ್ನು ಬದಲಾಯಿಸಬಲ್ಲಿರಾ? ಗಂಗಾನದಿಯು ಪುನಃ ಹಿಮಾಲಯಕ್ಕೆ ಹಿಂತಿರುಗಿ ಹೋಗಿ ಬೇರೆ ಪಾತ್ರದಲ್ಲಿ ಹರಿಯಲು ಸಾಧ್ಯವೇ? ಅದು ಒಂದು ವೇಳೆ ಸಾಧ್ಯವಾದರೂ ಆ ದೇಶವು ತನ್ನ ಪ್ರಧಾನ ಲಕ್ಷಣವಾದ ಧರ್ಮ ಜೀವನವನ್ನು ತೊರೆದು ರಾಜಕೀಯವನ್ನೊ ಅಥವಾ ಮತ್ತಾವುದನ್ನೊ ತನ್ನ ಹೊಸ ಜೀವನ ವಿಧಾನವನ್ನಾಗಿ ಮಾಡಿಕೊಳ್ಳುವುದು ಅಸಾಧ್ಯ. ಎಲ್ಲಿ ವಿರೋಧ ಕಡಿಮೆ ಇದೆಯೋ ಅಲ್ಲಿ ಮಾತ್ರ ಕೆಲಸ ಮಾಡಬಹುದು. ಭರತಖಂಡದಲ್ಲಿ ಧರ್ಮಕ್ಕೆ ಎಲ್ಲಕ್ಕಿಂತ ಕಡಿಮೆ ವಿರೋಧ\-ವಿದೆ. ಆದ್ದರಿಂದ ಇಲ್ಲಿ ಧರ್ಮವನ್ನು ಅನುಸರಿಸುವುದೇ ಜೀವನದ ಮಾರ್ಗ, ಏಳ್ಗೆಯ ಮಾರ್ಗ, ಶ್ರೇಯಸ್ಸಿನ ಮಾರ್ಗ. 

ಉಳಿದ ದೇಶಗಳಲ್ಲಿ ಧರ್ಮ ಜೀವನದ ಹಲವು ಆವಶ್ಯಕ ವಸ್ತುಗಳಲ್ಲಿ ಒಂದು. ಒಂದು ಸಾಮಾನ್ಯ ದೃಷ್ಟಾಂತವನ್ನು ನೀಡಬೇಕಾದರೆ, ಈಗಿನ ಕಾಲದಲ್ಲಿ ಕೋಣೆಯಲ್ಲಿ ಇತರ ವಸ್ತುಗಳ ಜತೆ ಒಂದು ಜಪಾನಿ ಹೂವಿನ ಜಾಡಿ ಇಡುವುದೊಂದು ಷೋಕಿ. ಅದನ್ನು ಗೃಹಿಣಿ ಪಡೆಯಲೇಬೇಕು. ಅದಿಲ್ಲದ್ದಿದ್ದರೆ ಚೆನ್ನಾಗಿರುವುದಿಲ್ಲ. ಹಾಗೆಯೆ ಈ ಸಭ್ಯ ವ್ಯಕ್ತಿಗಳಿಗೆ ಜೀವನದಲ್ಲಿ ಎಷ್ಟೋ ಕೆಲಸ ಕಾರ್ಯಗಳಿರುತ್ತವೆ. ಅವು ಪೂರ್ಣವಾಗಬೇಕಾದರೆ ಸ್ವಲ್ಪ ಧರ್ಮವೂ ಬೇಕು. ಅದಕ್ಕೇ ಅವರಲ್ಲಿ ಸ್ವಲ್ಪ ಧರ್ಮವಿದೆ. ರಾಜಕೀಯ ಮತ್ತು ಸಾಮಾಜಿಕ ಪ್ರಗತಿ, ಒಂದೇ ಮಾತಿನಲ್ಲಿ ಹೇಳುವುದಾದರೆ, ಈ ವಿಷಯ ಪ್ರಪಂಚವೇ ಪಾಶ್ಚಾತ್ಯರ ಜೀವನದ ಗುರಿ. ಈ ಗುರಿಯ ಸಾಧನೆಗೆ ಸಹಾಯಮಾಡಲು ದೇವರು ಮತ್ತು ಧರ್ಮ ಮೆಲ್ಲಗೆ ಪ್ರವೇಶಿಸುವುವು. ಈ ಜಗತ್ತಿನಲ್ಲಿ ಅವರಿಗೆ ಬೇಕಾದುದನ್ನು ಒದಗಿಸಿ, ಜಗತ್ತನ್ನು ಶುಚಿಯಾಗಿಡುವವನೇ ಅವರ ದೇವರು. ಅವರ ಪಾಲಿಗೆ ದೇವರ ಮೌಲ್ಯ ಇಷ್ಟೇ. ಕಳೆದ ಒಂದೆರಡು ನೂರು ವರ್ಷಗಳಿಂದ ಹೆಚ್ಚು ಅನುಭವವಿಲ್ಲದವರು ಅಥವಾ ತಮಗೆ ಹೆಚ್ಚು ಗೊತ್ತಿದೆ ಎಂದು ಭಾವಿಸುವವರು, ಭರತಖಂಡದ ಧರ್ಮವನ್ನು ಹೀಗೆ ಟೀಕಿಸುವುದನ್ನುಪದೇ ಪದೇ ಕೇಳಲಾಗಿದೆ: ನಮ್ಮ ಧರ್ಮವು ಇಹಲೋಕ ಸುಖಕ್ಕೆ ಸಹಾಯ ಮಾಡುವುದಿಲ್ಲ, ಇದು ನಮಗೆ ಐಶ್ವರ್ಯವನ್ನು ಕೊಡುವುದಿಲ್ಲ, ಇತರ ರಾಷ್ಟ್ರಗಳ ಐಶ್ವರ್ಯವನ್ನು ದೋಚುವಂತೆ ಮಾಡಲಾರದು, ದುರ್ಬಲರ ಮೇಲೆ ಬಲಿಷ್ಠರು ನಿಂತು ಅವರ ರಕ್ತವನ್ನು ಹೀರಿ ತಾವು ಜೀವಿಸುವಂತೆ ಮಾಡಲಾರದು. ಹೌದು ನಿಜ, ನಮ್ಮ ಧರ್ಮ ಇದನ್ನು ಮಾಡುವುದಿಲ್ಲ; ಅದು ಇತರ ದೇಶಗಳನ್ನು ನಾಶಮಾಡಿ ಸುಲಿಗೆ ಮಾಡುವುದಕ್ಕೆ ಸೇನೆಯನ್ನು ಕಳುಹಿಸಲಾರದು. ಅದಕ್ಕಾಗಿಯೇ ಅವರು, ಈ ಧರ್ಮದಲ್ಲಿ ಏನಿದೆ, ಅದು ಹಣಸಂಪಾದನೆಗೆ ಸಹಾಯ ಮಾಡುವುದಿಲ್ಲ, ಮಾಂಸ ಖಂಡಗಳಿಗೆ ಶಕ್ತಿಯನ್ನು ನೀಡುವುದಿಲ್ಲ, ಅಂತಹ ಧರ್ಮದಲ್ಲಿ ಏನಿದೆ–ಎಂದು ಕೇಳುತ್ತಾರೆ. 

ಈ ವಾದದ ಮೂಲಕವೇ ನಾವು ನಮ್ಮ ಧರ್ಮವನ್ನು ಸಮರ್ಥಿಸುತ್ತೇವೆ ಎಂಬುದು ಅವರಿಗೆ ಗೊತ್ತೇ ಇಲ್ಲ. ಏಕೆಂದರೆ ಧರ್ಮ ಇಹಲೋಕದ ಭೋಗಕ್ಕಲ್ಲ. ನಮ್ಮದೊಂದೇ ನಿಜವಾದ ಧರ್ಮ. ಏಕೆಂದರೆ ನಮ್ಮ ಧರ್ಮದ ಪ್ರಕಾರ ಅಲ್ಪ ವಿಸ್ತಾರದ, ಮೂರು ದಿನ ಬಾಳುವ ಈ ಕ್ಷುದ್ರ ಇಂದ್ರಿಯ ಜಗತ್ತನ್ನು ನಮ್ಮ ಜೀವನದ ಪರಮ ಗುರಿಯಾಗಿ ಮಾಡಲಾಗದು. ಭೂಕ್ಷೇತ್ರವೇ ನಮ್ಮ ಧರ್ಮ ದೃಷ್ಟಿಯ ಪರಮಾವಧಿಯಲ್ಲ. ನಮ್ಮದು ಇದನ್ನೆಲ್ಲಾ ಅತಿಕ್ರಮಿಸುವುದು. ಆಚೆ, ಇಂದ್ರಿಯ ಪರಿಮಿತಿಯಾಚೆ, ಕಾಲ ದೇಶಗಳಾಚೆ, ಈ ಜಗತ್ತಿನ ಲವಲೇಶವೂ ಉಳಿಯದಿರುವ ಅತೀತದ ಕಡೆ ನಮ್ಮ ಧರ್ಮದ ದೃಷ್ಟಿ ಇರುವುದು. ಅಲ್ಲಿ ಆತ್ಮಜ್ಯೋತಿಯ ಮಹಾಸಾಗರದಲ್ಲಿ, ಈ ಬ್ರಹ್ಮಾಂಡವೇ ಒಂದು ಬಿಂದುವಿನಂತೆ ತೋರುತ್ತದೆ, ನಮ್ಮದೇ ನಿಜವಾದ ಧರ್ಮ, ಏಕೆಂದರೆ ದೇವರೊಬ್ಬನೇ ಸತ್ಯ, ಈ ಸಂಸಾರ ಕ್ಷಣಿಕ, ನಿಮ್ಮ ಐಶ್ವರ್ಯವೆಲ್ಲಾ ಮಣ್ಣಿನ ಸಮಾನ, ನಿಮ್ಮ ಅಧಿಕಾರ ಸಾಂತವಾದುದು ಮತ್ತು ಈ ಜೀವನವೇ ಬಹುಪಾಲು ದೋಷಮಯ ಎಂದು ಸಾರುವುದು ನಮ್ಮ ಧರ್ಮ. ಅದಕ್ಕಾಗಿಯೇ ನಮ್ಮ ಧರ್ಮವೇ ನಿಜವಾದ ಧರ್ಮ ವೆನ್ನಿಸುವುದು. ನಮ್ಮದೇ ನಿಜವಾದ ಧರ್ಮ, ಏಕೆಂದರೆ ಎಲ್ಲಕ್ಕಿಂತ ಹೆಚ್ಚಾಗಿ ಇದು ತ್ಯಾಗವನ್ನು ಬೋಧಿಸುತ್ತದೆ. ಭರತಖಂಡದ ಅತಿ ಪುರಾತನ ಮಹರ್ಷಿಗಳು ಕಂಡುಹಿಡಿದ ಅನರ್ಘ್ಯ ಧರ್ಮ ಭಂಡಾರಕ್ಕೆ ಅಧಿಕಾರಿಗಳಾದ ಹಿಂದೂಗಳೊಂದಿಗೆ ಇತರರನ್ನು ಹೋಲಿಸಿ ನೋಡಿದರೆ ಅವರು ನಿನ್ನೆಯ ಮಕ್ಕಳಂತೆ ತೋರುವರು. ನಮ್ಮ ಧರ್ಮ ಅವರಿಗೆ ಮುಕ್ತ ಕಂಠದಿಂದ ಬಹಿರಂಗವಾಗಿ: “ಮಕ್ಕಳೇ, ನೀವು ವಿಷಯಸುಖಕ್ಕೆ ಗುಲಾಮರು. ಇಂದ್ರಿಯಸುಖ ಪರಿಮಿತಿಯುಳ್ಳದ್ದು. ಈ ಮಾರ್ಗದಲ್ಲಿ ಹೋದರೆ ನಿಮಗೆ ನಾಶ ಸಿದ್ಧ. ಮೂರು ದಿನಗಳ ಸುಖದಿಂದ ಕೊನೆಗೆ ಸರ್ವ ನಾಶ ಪ್ರಾಪ್ತಿ. ಇದನ್ನೆಲ್ಲಾ ತ್ಯಜಿಸಿ, ಪ್ರಪಂಚವನ್ನು ತ್ಯಜಿಸಿ. ಅದೇ ಧರ್ಮ ಮಾರ್ಗ” ಎಂದು ಸಾರುತ್ತದೆ. ತ್ಯಾಗದಿಂದ ಮಾತ್ರ ಅಮೃತತ್ವ ಪ್ರಾಪ್ತಿ, ಭೋಗದಿಂದ ಅಲ್ಲ. ಅದಕ್ಕೆ ನಮ್ಮದೊಂದೇ ನಿಜವಾದ ಧರ್ಮ. 

ಇದೊಂದು ಸೋಜಿಗದ ಸಂಗತಿ. ಜಗತ್ತಿನ ರಂಗಭೂಮಿಯ ಮೇಲೆ ರಾಷ್ಟ್ರಗಳಾದ ಮೇಲೆ ರಾಷ್ಟ್ರಗಳು ಬಂದು, ಕೆಲವು ಕ್ಷಣಗಳವರೆಗೆ ಬಿಡುವಿಲ್ಲದೆ ಮೆರೆದಾಡಿ, ಕೊನೆಗೆ ಯಾವುದೊಂದು ಪರಿಣಾಮವನ್ನೂ ಪ್ರಪಂಚದ ಮೇಲೆ ಬೀರದೆಮಾಯವಾಗಿ ಹೋಗಿವೆ. ಆದರೆ ನಾವು ಇನ್ನೂ ಶಾಶ್ವತವಾಗಿ ಬದುಕಿರುವೆವು. ಯೋಗ್ಯತೆಯುಳ್ಳದ್ದೆ ಉಳಿಯುವುದು ಎಂಬ ಹೊಸ ಸಿದ್ಧಾಂತವನ್ನು ಪಾಶ್ಚಾತ್ಯರಿಂದು ಸಾರುತ್ತಿರುವರು. ಕೇವಲ ದೈಹಿಕ ಶಕ್ತಿಯೇ ಆ ಯೋಗ್ಯತೆಯ ಲಕ್ಷಣವೆಂದು ಅವರು ತಿಳಿಯುತ್ತಾರೆ. ಇದು ನಿಜವಾಗಿದ್ದರೆ, ಹಿಂದಿನ ಕಾಲದ ಪ್ರಖ್ಯಾತ ಬಲಾಢ್ಯ ಜನಾಂಗಗಳು ಇಂದೂ ವೈಭವದಿಂದ ಬಾಳುತ್ತಿರಬೇಕಿತ್ತು. ಅನ್ಯರನ್ನು ಎಂದಿಗೂ ಗೆಲ್ಲದ ದೀನ ಹಿಂದೂ ಎಂದೋ ನಾಶವಾಗಿ ಹೋಗಬೇಕಾಗಿತ್ತು. ಆದರೆ ಮೂವತ್ತು ಕೋಟಿ ಸಂಖ್ಯೆ ಬಲಶಾಲಿಗಳಾದ ನಾವು ಈಗಲೂ ಇರುವೆವು. (ಒಬ್ಬ ಆಂಗ್ಲೇಯ ತರುಣಿ ಒಮ್ಮೆ “ಹಿಂದೂಗಳು ಏನು ಮಾಡಿರುವರು? ಯಾವ ರಾಷ್ಟ್ರವನ್ನೂ ಅವರು ಗೆದ್ದಿಲ್ಲ!” ಎಂದು ಪ್ರಶ್ನಿಸಿದ್ದಳು. ) ಭರತಖಂಡದ ಶಕ್ತಿಯೆಲ್ಲಾ ವ್ಯಯವಾಗಿ ಈಗ ಅದಕ್ಕೆ ಮುದಿತನ ಬಂದಿದೆ ಎಂಬುದು ಸುಳ್ಳು. ನಮ್ಮಲ್ಲಿ ಬೇಕಾದಷ್ಟು ಶಕ್ತಿ ಇದೆ. ಅದು ಸಮಯ ಬಂದಾಗ, ಆವಶ್ಯಕತೆ ಇರುವಾಗ, ಪ್ರವಾಹದಂತೆ ಹರಿದು ಪ್ರಪಂಚವನ್ನೇ ಮುಳುಗಿಸಬಲ್ಲುದು. 

ಅತ್ಯಂತ ಪುರಾತನ ಕಾಲದಿಂದಲೂ ನಾವು ಪ್ರಪಂಚಕ್ಕೆ ಒಂದು ಸವಾಲನ್ನು ಹಾಕಿರುವೆವು. ಮಾನವನು ಎಷ್ಟು ಹೆಚ್ಚು ವಸ್ತುಗಳನ್ನು ಹೊಂದಬಹುದೆಂಬುದನ್ನು ನಿಶ್ಚಯಿಸಲು ಪಾಶ್ಚಾತ್ಯರು ಕಾತರರಾಗಿರುವರು. ಭಾರತೀಯರು ಮಾನವನ ಜೀವನಕ್ಕೆ ಎಷ್ಟು ಅಲ್ಪ ಸಾಕಾಗಬಹುದು ಎಂಬುದನ್ನು ಕಂಡುಹಿಡಿಯಲು ಯತ್ನಿಸುತ್ತಿರುವರು. ಈ ಹೋರಾಟ, ಈ ವ್ಯತ್ಯಾಸ ಕೆಲವು ಶತಮಾನಗಳವರೆಗೆ ಮುಂದುವರಿಯುತ್ತದೆ. ಆದರೆ ಚರಿತ್ರೆಯಲ್ಲಿ ಏನಾದರೂ ಸತ್ಯವಿರುವುದಾದರೆ, ಮುನ್ಸೂಚನೆಯು ಏನಾದರೂ ಪ್ರಮಾಣೀಕರಿಸುವುದಾದರೆ, ಯಾರು ಅತಿ ಕಡಿಮೆ ವಸ್ತುಗಳ ಮೇಲೆ ಜೀವಿಸುವರೋ, ಯಾರು ಸಂಯಮಿಗಳೋ, ಅವರೇ ಕೊನೆಗೆ ಜಯಶಾಲಿಗಳಾಗುವರು ಎಂಬುದು ಕಂಡುಬರುತ್ತದೆ. ಯಾರು ಸುಖಭೋಗಗಳನ್ನು ಬೆನ್ನಟ್ಟಿ ಹೋಗುವರೋ ಅವರು ತಾತ್ಕಾಲಿಕವಾಗಿ ಎಷ್ಟೇ ಬಲಾಢ್ಯರಾಗಿ ಕಂಡರೂ, ಕೊನೆಗೆ ನಾಶವಾಗಲೇಬೇಕು. ವ್ಯಕ್ತಿ ಜೀವನದಲ್ಲಿ ಮತ್ತು ಜನಾಂಗದ ಜೀವನದಲ್ಲಿಯೂ ಕೂಡ ಕೆಲವು ವೇಳೆ ಪ್ರಪಂಚದ ಮೇಲಿನ ಜುಗುಪ್ಸೆ ತೀವ್ರತೆಯನ್ನು ಮುಟ್ಟುತ್ತದೆ. ಪಾಶ್ಚಾತ್ಯ ದೇಶಗಳಲ್ಲಿ ಈಗ ಅಂತಹ ಕಾಲ ಪ್ರಾಪ್ತವಾಗಿದೆ ಎಂದು ತೋರುತ್ತದೆ. ಅವರಲ್ಲಿಯೂ ಮೇಧಾವಿಗಳಿರುವರು. ಐಶ್ವರ್ಯವನ್ನೂ, ಅಧಿಕಾರವನ್ನೂ ಬೇಟೆಯಾಡುವುದು ನಿರರ್ಥಕವೆಂದು ಅವರು ತಿಳಿಯುತ್ತಿರುವರು. ಅಲ್ಲಿರುವ ಸುಸಂಸ್ಕೃತ ಸ್ತ್ರೀ ಪುರುಷರನೇಕರಿಗೆ ಆಗಲೇ ಈ ಹೋರಾಟ, ಸ್ಪರ್ಧೆ ವಾಣಿಜ್ಯ ನಾಗರಿಕತೆಯ ಕ್ರೌರ್ಯ ಸಾಕಾಗಿ ಹೋಗಿದೆ. ಅವರು ಅದಕ್ಕಿಂತ ಉತ್ತಮವಾದುದನ್ನು ನಿರೀಕ್ಷಿಸುತ್ತಿರುವರು. ಯೂರೋಪಿನ ಸಮಸ್ಯೆಗಳಿಗೆಲ್ಲಾ ರಾಜಕೀಯ ಮತ್ತು ಸಾಮಾಜಿಕ ಸುಧಾರಣೆಯೊಂದೇ ಪರಮೌಷಧಿ ಎಂದು ನಂಬುವ ಕೆಲವರು ಇನ್ನೂ ಇರುವರು. ಆದರೆ ಅಲ್ಲಿರುವ ಶ್ರೇಷ್ಠ ಆಲೋಚನಾಶೀಲರು ಬೇರೆ ಆದರ್ಶಗಳನ್ನು ಕುರಿತು ಚಿಂತಿಸುತ್ತಿರುವರು. ಅವರು ಕೇವಲ ರಾಜಕೀಯ ಅಥವಾ ಸಾಮಾಜಿಕ ಬದಲಾವಣೆಗಳಿಂದಲೇ ಜೀವನದ ದೋಷಗಳನ್ನು ಪರಿಹರಿಸುವುದು ಅಸಾಧ್ಯ ಎಂದು ಕಂಡುಕೊಂಡಿರುವರು. ಆತ್ಮ ಪರಿವರ್ತನೆಯಿಂದ ಮಾತ್ರ ಜೀವನದ ದೋಷಗಳನ್ನು ಪರಿಹರಿಸಬಹುದು. ಯಾವುದೇ ಬಲಾತ್ಕಾರವಾಗಲಿ, ಸರ್ಕಾರವಾಗಲಿ, ನಿರ್ದಯ ಕಾನೂನುಗಳಾಗಲಿ ಜನಾಂಗದ ಪರಿಸ್ಥಿತಿಗಳನ್ನು ಬದಲಾಯಿಸಲಾರದು. ಜನಾಂಗದಲ್ಲಿರುವ ದೋಷಗಳನ್ನು, ಆಧ್ಯಾತ್ಮಿಕ ಮತ್ತು ನೈತಿಕ ಆದರ್ಶಗಳು ಮಾತ್ರ ನಿರ್ಮೂಲ ಮಾಡಬಲ್ಲವು. ಪಾಶ್ಚಾತ್ಯ ಜನಾಂಗಗಳು ಹೊಸ ಆದರ್ಶಕ್ಕಾಗಿ, ಹೊಸ ತತ್ತ್ವಕ್ಕಾಗಿ ಕಾಯುತ್ತಿರುವುವು. ಅವರ ಕ್ರೈಸ್ತ ಧರ್ಮವು ಹಲವು ರೀತಿಗಳಿಂದ ಎಷ್ಟೇ ಮೇಲಾಗಿದ್ದರೂ, ಅದನ್ನು ಅವರು ತಪ್ಪಾಗಿ ತಿಳಿದುಕೊಂಡಿರುವರು. ಈಗ ಅದನ್ನು ಅರ್ಥಮಾಡಿಕೊಂಡಿರುವಂತೆ ಅದು ಅಸಮರ್ಪಕವಾಗಿ ಕಾಣುತ್ತದೆ. ಪಾಶ್ಚಾತ್ಯ ವಿಚಾರಶೀಲರು ನಮ್ಮ ಪುರಾತನ ದರ್ಶನಗಳಲ್ಲಿ, ಅದರಲ್ಲೂ ವಿಶೇಷವಾಗಿ ವೇದಾಂತದಲ್ಲಿ ಅವರು ಅರಸುವ ಹೊಸ ಚಿಂತನ ಪ್ರೇರಣೆಯನ್ನೂ, ಅವರು ಹಾತೊರೆಯುತ್ತಿದ್ದ ಆಧ್ಯಾತ್ಮಿಕ ಆಹಾರವನ್ನೂ ಕಂಡುಕೊಂಡಿರುವರು. ಇದರಲ್ಲಿ ಏನೂ ಆಶ್ಚರ್ಯವಿಲ್ಲ. 

ಪ್ರತಿಯೊಬ್ಬರೂ ತಮ್ಮ ತಮ್ಮ ಧರ್ಮದ ಶ್ರೇಷ್ಠತೆಯ ಬಗ್ಗೆ ಮಾತನಾಡುವುದನ್ನು ನಾನು ಕೇಳುತ್ತಿರುವೆನು. ನಮ್ಮ ಸ್ನೇಹಿತರಾದ ಡಾಕ್ಟರ್​ ಬರೋಸ್​ರವರು ಕ್ರೈಸ್ತ ಧರ್ಮವೊಂದೇ ವಿಶ್ವಕ್ಕೆಲ್ಲಾ ಅನ್ವಯಿಸುವ ಧರ್ಮವೆಂದು ಹೇಳಿದುದನ್ನು ನೀವು ಕೇಳಿರುವಿರಿ. ಆದರೆ ಆ ಪಾತ್ರವನ್ನು ವಹಿಸಲು ವೇದಾಂತ ತತ್ತ್ವಕ್ಕೆ ಮಾತ್ರ ಅಧಿಕಾರ\-ವಿದೆ. ಉಳಿದ ಧರ್ಮಗಳು ಆ ಪಾತ್ರಕ್ಕೆ ಅರ್ಹವಾಗಿಲ್ಲ ಎಂಬ ವಿಷಯವನ್ನು ಕುರಿತು ಈಗ ಸ್ವಲ್ಪ ಚರ್ಚಿಸುತ್ತೇನೆ. ನಮ್ಮ ಧರ್ಮದ ವಿನಃ ಉಳಿದ ಧರ್ಮಗಳೆಲ್ಲಾ ಆಯಾಯ ಧರ್ಮದ ಒಬ್ಬರು ಅಥವಾ ಹಲವು ಜನ ಮತ ಸ್ಥಾಪಕರ ಜೀವನವನ್ನು ಅವಲಂಬಿಸಿರುತ್ತದೆ. ಆ ಧರ್ಮಗಳ ಸಿದ್ಧಾಂತಗಳು, ಬೋಧನೆಗಳು, ಮತತತ್ತ್ವಗಳು ಮತ್ತು ನೈತಿಕ ನಿಯಮಾವಳಿಗಳು ಇವೆಲ್ಲವೂ ಒಬ್ಬ ಮತಸ್ಥಾಪಕನ ಜೀವನದಲ್ಲಿ ಕೇಂದ್ರೀಕೃತವಾಗಿರುತ್ತವೆ. ಅವನಿಂದಲೇ ಆ ಧರ್ಮದ ಅನುಯಾಯಿಗಳಿಗೆ ಪ್ರಮಾಣ, ಅಧಿಕಾರ ಮತ್ತು ಶಕ್ತಿ ದೊರೆಯುವುವು. ಇನ್ನೂ ವಿಶೇಷವೇನೆಂದರೆ, ಆ ಧರ್ಮದ ಸೌಧವೇ ಧರ್ಮಸ್ಥಾಪಕನ ಜೀವನದ ಐತಿಹ್ಯದ ಮೇಲೆ ನಿಂತಿರುತ್ತದೆ. ಅಂತಹ ಐತಿಹ್ಯಕ್ಕೆ ಒಂದು ಪೆಟ್ಟು ಬಿದ್ದರೆ, (ಆಧುನಿಕ ಕಾಲದಲ್ಲಿ ಮುಕ್ಕಾಲು ಪಾಲು ಧರ್ಮಸಂಸ್ಥಾಪಕ ವ್ಯಕ್ತಿಗಳಿಗೆಲ್ಲಾ ಈ ಆಘಾತ ತಗಲುತ್ತಿದೆ. ಅವರ ಜೀವನದ ಅರ್ಧಭಾಗವನ್ನು ನಂಬುವುದಕ್ಕೆ ಆಗುವುದಿಲ್ಲ. ಉಳಿದ ಅರ್ಧಭಾಗವನ್ನು ಕುರಿತು ಸಂದೇಹಪಡುವರು. ) ಅವರು ಭಾವಿಸಿಕೊಂಡಿರುವ ಆ ಐತಿಹ್ಯ ಎಂಬ ವಜ್ರ ತಳಹದಿ ಕುಸಿದುಬಿಟ್ಟರೆ ಇಡೀ ಕಟ್ಟಡ ಚೂರುಚೂರಾಗಿ ನೆಲಕ್ಕೆ ಬೀಳುವುದು. ಅದು ಪುನಃ ಚೇತರಿಸಿಕೊಳ್ಳಲಾರದು. 

ನಮ್ಮ ಧರ್ಮದ ವಿನಃ ಪ್ರಪಂಚದ ಉಳಿದ ಪ್ರಮುಖ ಧರ್ಮಗಳೆಲ್ಲಾ ಅಂತಹ ಐತಿಹಾಸಿಕ ವ್ಯಕ್ತಿಗಳ ಮೇಲೆ ನಿಂತಿವೆ. ನಮ್ಮ ಧರ್ಮ ನಿಂತಿರುವುದು ತತ್ತ್ವಗಳ ಮೇಲೆ. ಯಾವ ಸ್ತ್ರೀಪುರುಷರಾಗಲಿ ತಾವು ವೇದಗಳನ್ನು ಸೃಷ್ಟಿಸಿದವರೆಂದು ಹೇಳಿಕೊಳ್ಳಲಾರರು. ಅಲ್ಲಿರುವುದು ಸನಾತನ ತತ್ತ್ವ. ಋಷಿಗಳು ಅವನ್ನು ಕಂಡು\-ಹಿಡಿದರು. ಮಧ್ಯೆ ಮಧ್ಯೆ ಈ ಋಷಿಗಳು ಹೆಸರುಗಳೂ ಬರುತ್ತದೆ. ಬರಿಯ ಹೆಸರುಗಳು; ಅವರು ಯಾರು, ಎಲ್ಲಿದ್ದರು, ಎಂಬುದು ಯಾರಿಗೂ ತಿಳಿಯದು. ಅವರ ತಂದೆ ಯಾರು ಎಂಬುದೇ ಅನೇಕವೇಳೆ ಗೊತ್ತಾಗುವುದಿಲ್ಲ. ಅವರು ಹುಟ್ಟಿದ್ದು ಎಲ್ಲಿ, ಯಾವಾಗ, ಎಂಬುದು ಎಲ್ಲಿಯೂ ನಮಗೆ ಕಾಣುವುದಿಲ್ಲ. ಆ ಋಷಿಗಳು ಕೀರ್ತಿಯನ್ನು ಲೆಕ್ಕಿಸಲಿಲ್ಲ. ಅವರು ತತ್ತ್ವಬೋಧಕರು, ಸಾಧ್ಯವಾದಷ್ಟು ತಮ್ಮ ಬೋಧನೆಗೆ ದೃಷ್ಟಾಂತಪ್ರಾಯವಾದ ಜೀವನವನ್ನು ನಡೆಸಿದರು. ಜೊತೆಗೆ ನಮ್ಮ ದೇವರು ನಿರ್ಗುಣನಾಗಿದ್ದರೂ ಸಗುಣನಾಗಿರುವಂತೆ, ನಮ್ಮ ಧರ್ಮವೂ ಕೂಡ ತತ್ತ್ವಗಳ ಮೇಲೆ ಮಾತ್ರ ನಿಂತಿರುವುದರಿಂದ ಅವೈಯಕ್ತಿಕವಾಗಿದ್ದರೂ ಅಲ್ಲಿ ವ್ಯಕ್ತಿಗಳ ಪಾತ್ರಕ್ಕೆ ಬೇಕಾದಷ್ಟು ಅವಕಾಶವಿದೆ. ಏಕೆಂದರೆ ಬೇರೆ ಯಾವ ಧರ್ಮ ಇಷ್ಟೊಂದು ಅವತಾರಗಳನ್ನೂ, ಮಹಾತ್ಮರನ್ನೂ, ದೇವದೂತರನ್ನೂ ಒದಗಿಸಿಕೊಟ್ಟು ಮುಂದೆ ಬರುವ ಇಂತಹ ಅಸಂಖ್ಯ ಮಹಾಪುರುಷರಿಗಾಗಿ ಕಾದು ಕುಳಿತಿರುವುದು? ಅವತಾರಗಳು ಅಸಂಖ್ಯವಾಗಿವೆ ಎನ್ನುವುದು ಭಾಗವತಪುರಾಣ. ನಿಮ್ಮ ಇಚ್ಛೆ ಇದ್ದಷ್ಟೂ ಅವತಾರಗಳಿಗೆ ಅಲ್ಲಿ ಅವಕಾಶವಿದೆ. ಆದಕಾರಣ ನಮ್ಮ ಧಾರ್ಮಿಕ ಇತಿಹಾಸದಲ್ಲಿರುವ ಇಂತಹ ಕೆಲವು ಅವತಾರಗಳು ಅಥವಾ ದೇವದೂತರು ಐತಿಹಾಸಿಕ ವ್ಯಕ್ತಿಗಳಲ್ಲವೆಂದು ಪ್ರಮಾಣಿತವಾದರೆ ನಮ್ಮ ಧರ್ಮಕ್ಕೆ ಇದರಿಂದ ನಷ್ಟವೇನೂ ಇಲ್ಲ. ಆಗಲೂ ಅದು ಎಂದಿನಂತೆ ಅಚಲವಾಗಿಯೇ ನಿಂತಿರುತ್ತದೆ. ಏಕೆಂದರೆ ಅದು ತತ್ತ್ವಗಳ ಮೇಲೆ ನಿಂತಿದೆಯೇ ಹೊರತು ವ್ಯಕ್ತಿಗಳ ಮೇಲೆ ಅಲ್ಲ. ಯಾವುದೇ ಒಂದು ವ್ಯಕ್ತಿಯ ಅಧೀನಕ್ಕೆ ಜಗತ್ತಿನ ಜನರೆಲ್ಲರನ್ನೂ ತರಲೆತ್ನಿಸುವುದು ವ್ಯರ್ಥ. ಸನಾತನವಾದ, ಸಾರ್ವತ್ರಿಕವಾದ ತತ್ತ್ವಗಳ ಅಡಿಯಲ್ಲಿಯೂ ನಾವು ಅವರನ್ನು ತರುವುದು ಕಷ್ಟ. ಧರ್ಮಕ್ಷೇತ್ರದಲ್ಲಿ ಪ್ರಪಂಚದ ಬಹುಮಂದಿ ಜನರನ್ನು ಒಂದೇ ರೀತಿ ಆಲೋಚಿಸುವಂತೆ ಮಾಡುವುದು ಒಂದು ವೇಳೆ ಸಾಧ್ಯವಾದರೆ ಅದು ತತ್ತ್ವದ ಮೂಲಕ ಮಾತ್ರ ಸಾಧ್ಯ. ವ್ಯಕ್ತಿ ಮೂಲಕ ಅಲ್ಲ. ಆದರೂ ನಾನು ಮೊದಲೇ ಹೇಳಿರುವಂತೆ ನಮ್ಮ ಧರ್ಮದಲ್ಲಿ ವ್ಯಕ್ತಿಯ ಪ್ರಭಾವಕ್ಕೆ ಮತ್ತು ಅಧಿಕಾರಕ್ಕೆ ಬೇಕಾದಷ್ಟು ಅವಕಾಶವಿದೆ. ನಮ್ಮ ಧರ್ಮದಲ್ಲಿ ಅತ್ಯಪೂರ್ವವಾದ ಇಷ್ಟ ಸಿದ್ಧಾಂತವಿದೆ. ಇದರ ಪ್ರಕಾರ ನೀವು ಈ ಶ್ರೇಷ್ಠ ಧಾರ್ಮಿಕ ವ್ಯಕ್ತಿಗಳಲ್ಲಿ ಯಾರನ್ನು ಬೇಕಾದರೂ ಸ್ವೀಕರಿಸಬಹುದು. ನೀವು ಇವರಲ್ಲಿ ಯಾರಾದರೂ ಒಬ್ಬ ದೇವದೂತನನ್ನೋ, ಆಚಾರ್ಯರನ್ನೋ ಗುರುಗಳಂತೆ ಸ್ವೀಕರಿಸಿ ವಿಶೇಷವಾಗಿ ಪೂಜಿಸಬಹುದು. ಯಾರನ್ನು ನೀವು ಸ್ವೀಕರಿಸುವಿರೋ, ಅವರನ್ನು ನೀವು ಅವತಾರ ಶ್ರೇಷ್ಠ ಅತ್ಯುತ್ತಮ ದೇವದೂತನೆಂದು ಬೇಕಾದರೆ ತಿಳಿಯಬಹುದು, ಅದರಲ್ಲಿ ಏನೂ ತೊಂದರೆ ಇಲ್ಲ. ಆದರೆ ಸನಾತನ ತತ್ತ್ವಗಳ ಹಿನ್ನೆಲೆಯನ್ನು ಮಾತ್ರ ಭದ್ರವಾಗಿ ಹಿಡಿದುಕೊಂಡಿರಬೇಕು. ನಮ್ಮಲ್ಲಿರುವ ಒಂದು ವಿಶೇಷವೇನೆಂದರೆ ನಮ್ಮ ಅವತಾರಪುರುಷರು ಎಲ್ಲಿಯವರೆವಿಗೂ ವೈದಿಕ ತತ್ತ್ವಗಳನ್ನು ಪ್ರತಿನಿಧಿಸುತ್ತಾರೆಯೋ, ಅಲ್ಲಿಯವರೆಗೆ ಮಾತ್ರ ಅವರಿಗೆ ನಮ್ಮ ಮೇಲೆ ಅಧಿಕಾರ. ಶ‍್ರೀಕೃಷ್ಣನ ಮಹಿಮೆ ಇರುವುದು ಅವನು ನಮ್ಮ ಪುರಾತನ ತತ್ತ್ವಗಳ ಅತಿ ಶ್ರೇಷ್ಠ ಬೋಧಕ ಮತ್ತು ವೇದಾಂತದ ಸರ್ವೋತ್ಕೃಷ್ಟ ವ್ಯಾಖ್ಯಾನಕಾರ ಎನ್ನುವುದರ ಮೇಲೆ. 

ಜಗತ್ತಿನ ದೃಷ್ಟಿಯನ್ನು ವೇದಾಂತವು ತನ್ನ ಕಡೆಗೆ ಆಕರ್ಷಿಸಿದುದಕ್ಕೆ ಎರಡನೆಯ ಕಾರಣವೆಂದರೆ, ಜಗತ್ತಿನ ಶಾಸ್ತ್ರಗಳಲ್ಲೆಲ್ಲಾ ವೇದಾಂತದ ಬೋಧನೆಗಳು ಮಾತ್ರವೇ ಬಾಹ್ಯಪ್ರಕೃತಿಗೆ ಸಂಬಂಧಿಸಿದಂತೆ ಆಧುನಿಕ ವೈಜ್ಞಾನಿಕ ಅನ್ವೇಷಣೆಗಳೊಂದಿಗೆ ಸಂಪೂರ್ಣವಾಗಿ ಹೊಂದಿಕೆಯಾಗಿರುವುದು. ಪ್ರಾಚೀನ ಕಾಲದಲ್ಲಿ ಸಮಾನ ಆಕಾರ ಪ್ರಕಾರ, ಸಹಾನುಭೂತಿಗಳಿಂದ ಕೂಡಿದ ಎರಡು ಜನಾಂಗಗಳು ಬೇರೆ ಬೇರೆ ವಾತಾವರಣಗಳಲ್ಲಿ ಬೆಳೆದುದರಿಂದ ಬೇರೆ ಬೇರೆ ಮಾರ್ಗಗಳಲ್ಲಿ ಮುಂದುವರಿದುವು. ಒಂದು ಪುರಾತನ ಹಿಂದೂ, ಮತ್ತೊಂದು ಪುರಾತನ ಯವನ. ಹಿಂದೂಗಳು ಆಂತರಿಕ ಜಗತ್ತಿನ ಮಾರ್ಗದಲ್ಲಿ ಹೊರಟರು. ಯವನರು ಬಾಹ್ಯ\-ಪ್ರಪಂಚದ ವಿಶ್ಲೇಷಣೆಯ ಮೂಲಕ ಗುರಿ ಸೇರಲು ಹೊರಟರು. ಎರಡೂ ಜನಾಂಗಗಳ ಇತಿಹಾಸದಲ್ಲಿ ಅನೇಕ ಏಳುಬೀಳುಗಳು ನಡೆದಿದ್ದರೂ ಈ ಎರಡೂ ಚಿಂತನಾ ಸ್ಪಂದನಗಳೂ ಒಂದೇ ಅತೀತದ ಗುರಿಯ ಸಮಾನ ಪ್ರತಿಧ್ವನಿಗಳಾಗಿರುವುದನ್ನು ನಾವು ಸುಲಭವಾಗಿ ಗುರುತಿಸಬಹುದು. ಆಧುನಿಕ ವೈಜ್ಞಾನಿಕ ಸಿದ್ಧಾಂತಗಳನ್ನು ಹಿಂದೂಗಳು ಅಥವಾ ವೇದಾಂತಿಗಳು ಮಾತ್ರ ವಿರೋಧವಿಲ್ಲದೇ ತಮ್ಮ ಧರ್ಮಕ್ಕೆ ಹೊಂದಿಕೆಯಾಗುವಂತೆ ಸ್ವೀಕರಿಸಬಲ್ಲರು. ಆಧುನಿಕ ವಿಜ್ಞಾನವು ಕೂಡ ತನ್ನ ಸಿದ್ಧಾಂತಗಳನ್ನು ಬಿಟ್ಟುಕೊಡದೆ ವೇದಾಂತ ತತ್ತ್ವಗಳನ್ನು ಸ್ವೀಕರಿಸುವುದರ ಮೂಲಕ ಅಧ್ಯಾತ್ಮದ ಕಡೆ ಮುಂದು ವರಿಯಬಹುದು ಎಂಬುದು ಸ್ಪಷ್ಟವಾದಂತೆ ಕಾಣುತ್ತದೆ. ನಮಗೂ, ಎಲ್ಲಾ ವಿಚಾರಮತಿಗಳಿಗೂ, ಹಲವು ಸಹಸ್ರವರ್ಷಗಳ ಹಿಂದೆ ವೇದಾಂತವು ಯಾವ ಸಿದ್ಧಾಂತವನ್ನು ಸಾರಿತೋ ಅದನ್ನು ಈಗ ಆಧುನಿಕ ವಿಜ್ಞಾನವು ಸಾರುತ್ತಿದೆ ಎಂಬುದು ಕಂಡು ಬರುತ್ತದೆ. ಕೇವಲ ಆಧುನಿಕ ವಿಜ್ಞಾನವು ಸಾರುತ್ತಿದೆ ಎಂಬುದು ಕಂಡು ಬರುತ್ತದೆ. ಕೇವಲ ಆಧುನಿಕ ವಿಜ್ಞಾನವು ವೈಜ್ಞಾನಿಕ ಪರಿಭಾಷೆಯ ಮೂಲಕ ಆ ಸಿದ್ಧಾಂತಗಳನ್ನು ವಿವರಿಸುತ್ತದೆ ಅಷ್ಟೆ. 

ಆದ್ದರಿಂದ ವರ್ತಮಾನದ ಪಾಶ್ಚಾತ್ಯ ಜನಾಂಗಕ್ಕೆ ಮೆಚ್ಚುಗೆಯಾಗುವಂತಹ ಮತ್ತೊಂದು ವೇದಾಂತದ ಅಪೂರ್ವ ವಿಶೇಷವೇನೆಂದರೆ, ಇದು ಅದ್ಭುತವಾದ ರೀತಿಯಲ್ಲಿ ಯುಕ್ತಿಯುಕ್ತವಾಗಿದೆ ಎಂಬುದು. ವೇದಾಂತದ ನಿರ್ಣಯಗಳು ಎಷ್ಟು ಅದ್ಭುತವಾಗಿ ಯುಕ್ತಿಯುಕ್ತವಾಗಿವೆ ಎಂಬುದನ್ನು ಕೆಲವು ಪಾಶ್ಚಾತ್ಯ ಅತಿಶ್ರೇಷ್ಠ ವಿಜ್ಞಾನಿಗಳೇ ನನಗೆ ಹೇಳಿರುವರು. ಅವರಲ್ಲಿ ಒಬ್ಬರ ವೈಯಕ್ತಿಕ ಪರಿಚಯ ನನಗೆ ಇದೆ. ಅವರಿಗೆ ಊಟಮಾಡುವುದಕ್ಕೂ ತಮ್ಮ ಪ್ರಯೋಗ ಶಾಲೆಯಿಂದ ಹೊರಗೆ ಹೋಗುವುದಕ್ಕೂ ಸಾಕಷ್ಟು ಸಮಯವಿರಲಿಲ್ಲ. ಅಂತಹವರು ನನ್ನ ವೇದಾಂತ ಉಪನ್ಯಾಸಗಳನ್ನು ಹಲವು ಗಂಟೆಗಳವರೆಗೆ ನಿಂತು ಕೇಳುತ್ತಿದ್ದರು. ಕಾರಣ, ಅವರೇ ಹೇಳಿದಂತೆ ಅದು ಅಷ್ಟೊಂದು ವೈಜ್ಞಾನಿಕವಾಗಿದೆ. ಆಧುನಿಕ ಯುಗದ ಆಶೋತ್ತರಗಳೊಂದಿಗೆ ಮತ್ತು ಆಧುನಿಕ ವಿಜ್ಞಾನದ ಇಂದಿನ ನಿರ್ಣಯಗಳೊಂದಿಗೆ ಅದು ಸರಿಯಾದ ರೀತಿಯಲ್ಲಿ ಹೊಂದಿಕೊಳ್ಳುತ್ತದೆ. 

ವಿಭಿನ್ನ ಧರ್ಮಗಳ ತುಲನಾತ್ಮಕ ಅಧ್ಯಯನದಿಂದ ದೊರೆತ ಎರಡು ವೈಜ್ಞಾನಿಕ ನಿರ್ಣಯಗಳನ್ನು ನಾನು ಇಂದು ನಿಮ್ಮ ಮುಂದೆ ಇಡುತ್ತೇನೆ: ಒಂದು, ಎಲ್ಲಾ ಧರ್ಮಗಳೂ ಸಾರ್ವತ್ರಿಕ ಎಂಬ ಭಾವನೆ; ಮತ್ತೊಂದು, ಜಗತ್ತಿನಲ್ಲಿ ಇರುವುದು ಒಂದೇ ವಸ್ತು ಎನ್ನುವುದು. ಬ್ಯಾಬಿಲೋನಿಯಾ ಮತ್ತು ಯಹೂದ್ಯ ಇತಿಹಾಸದಲ್ಲಿ ಒಂದು ಧಾರ್ಮಿಕ ವೈಶಿಷ್ಟ್ಯವನ್ನು ಕಾಣುತ್ತೇವೆ. ಈ ಎರಡು ಜನಾಂಗಗಳೂ ಹಲವು ಪಂಗಡಗಳಾಗಿ ವಿಭಾಗಗೊಂಡಿದ್ದವು. ಪ್ರತಿಯೊಂದು ಪಂಗಡಕ್ಕೂ ತನ್ನದೇ ಆದ ದೇವತೆಯಿತ್ತು; ಮತ್ತು ಈ ಎಲ್ಲಾ ದೇವತೆಗಳಿಗೂ ಒಂದು ಸಾಮಾನ್ಯ ಹೆಸರು ರೂಢಿಯಲ್ಲಿತ್ತು. ಬ್ಯಾಬಿಲೋನಿಯಾ ದೇವತೆಗಳಿಗೆಲ್ಲಾ ಬಾಲ್​ ಎಂದು ಹೆಸರು. ಅವರಲ್ಲಿ ಬಾಲ್​ ಮಾರಡೆಕ್​ ಎಂಬುದು ಮುಖ್ಯ ದೇವತೆ. ಕ್ರಮೇಣ ಒಂದು ಪಂಗಡವು ಉಳಿದ ಪಂಗಡಗಳನ್ನು ಗೆಲ್ಲುತ್ತಿತ್ತು ಮತ್ತು ಉಳಿದ ಪಂಗಡಗಳನ್ನು ತನ್ನ ಪಂಗಡಕ್ಕೇ ಸೇರಿಸಿಕೊಳ್ಳುತ್ತಿತ್ತು. ಆಗ ಗೆದ್ದ ಜನಾಂಗದ ದೇವರು ಉಳಿದ ಎಲ್ಲಾ ದೇವತೆಗಳಿಗಿಂತಲೂ ಶ್ರೇಷ್ಠನಾಗುತ್ತಿದ್ದನು. ಸೆಮಿಟಿಕ್​ ಜನರು ಜಂಭ ಕೊಚ್ಚಿಕೊಳ್ಳುತ್ತಿರುವ ಏಕದೇವತಾವಾದವು ಅಸ್ತಿತ್ವಕ್ಕೆ ಬಂದದ್ದು ಹೀಗೆ. ಯಹೂದ್ಯರಲ್ಲಿ ದೇವರಿಗೆ ಮೊಲಾಕ್​ ಎಂದು ಹೆಸರು. ಇಸ್ರೇಲ್​ ಪಂಗಡಕ್ಕೆ ಸೇರಿದ ಮೊಲಾಕಿಗೆ “ಮೊಲಾಕ್​ ಯಾವಾ” ಎಂದು ಹೆಸರು. ಕ್ರಮೇಣ ಯಹೂದ್ಯ ಪಂಗಡ ಇತರ ಪಂಗಡಗಳನ್ನೆಲ್ಲಾ ಸೋಲಿಸಿ ತಮ್ಮ ಮೊಲಾಕನೇ ಇತರ ಮೊಲಾಕರಿಗಿಂತ ಶ್ರೇಷ್ಠ ಎಂದು ಸಾರಿತು. ಈ ಧಾರ್ಮಿಕ ಜಯದ ಹಿಂದೆ ಇದ್ದ ರಕ್ತಪಾತ, ದೌರ್ಜನ್ಯ, ಹಿಂಸೆ ನಿಮಗೆಲ್ಲಾ ಗೊತ್ತಿರಬಹುದು. ಅನಂತರ ಬ್ಯಾಬಿಲೋನಿಯನರು “ಮೊಲಾಕ್​ ಯಾವಾ”ನ ಸರ್ವಾಧಿಕಾರವನ್ನು ಕಸಿದುಕೊಳ್ಳಲು ಯತ್ನಿಸಿ ವಿಫಲರಾದರು. 

ಧಾರ್ಮಿಕ ವಿಷಯಗಳಲ್ಲಿ ಹಲವು ಪಂಗಡಗಳ ಘರ್ಷಣೆ ಭರತಖಂಡದ ಸರಹದ್ದಿ\-ನಲ್ಲಿಯೂ, ಮತ್ತು ದೇಶದ ಒಳಗೂ ನಡೆದಿರಬಹುದೆಂದು ನನಗೆ ತೋರುವುದು. ಇಲ್ಲಿಯ ಆರ್ಯರ ಬೇರೆ ಬೇರೆ ಪಂಗಡದವರು ತಮ್ಮ ತಮ್ಮ ಪಂಥೀಯ ದೇವತೆಗಳ ಸಾರ್ವಭೌಮತ್ವವನ್ನು ಸ್ಥಾಪಿಸಲು ಹೋರಾಟ ನಡೆಸಿರಬಹುದು. ಆದರೆ ಭರತ ಖಂಡದ ಇತಿಹಾಸ ಬೇರೆ ರೂಪವನ್ನು ತಾಳಬೇಕಿತ್ತು, ಯಹೂದ್ಯರ ಇತಿಹಾಸಕ್ಕಿಂತ ಭಿನ್ನವಾಗಿರಬೇಕಾಗಿತ್ತು. ಜಗತ್ತಿನಲ್ಲೆಲ್ಲಾ ಭಾರತವೇ ಸಹಿಷ್ಣುತೆ ಮತ್ತು ಆಧ್ಯಾತ್ಮಿಕತೆಗಳಿಗೆ ತೌರುಮನೆಯಾಗಬೇಕಾಗಿತ್ತು. ಆದ್ದರಿಂದ ಬಹಳ ಕಾಲ ಇಲ್ಲಿ ಆ ಘರ್ಷಣೆ ನಡೆಯಲಿಲ್ಲ. ಭಾರತಖಂಡದಲ್ಲಿ ಜನಿಸಿದ ಋಷಿಶ್ರೇಷ್ಠರು ಇತಿಹಾಸದ ಬಗೆಗೆ ನಿಲುಕದ ಸಂಪ್ರದಾಯವೂ ಕೂಡ ಇಣುಕಿ ನೋಡಲಾಗದ ಅತ್ಯಂತ ಪ್ರಾಚೀನ ಕಾಲದಲ್ಲಿ “ಏಕಂ ಸದ್ವಿಪ್ರಾ ಬಹುಧಾ ವದಂತಿ”–ಇರುವವನೊಬ್ಬನೇ, ಋಷಿಗಳು ಹಲವು ವಿಧಗಳಲ್ಲಿ ಅವನನ್ನು ಕರೆಯುವರು, ಎಂದು ಸಾರಿದರು. ಇಂತಹ ಚಿರಸ್ಮರಣೀಯ ಪವಿತ್ರವಾಣಿಯನ್ನು ಪ್ರಪಂಚದಲ್ಲಿ ಅದುವರೆಗೆ ಯಾರೂ ಉಚ್ಚರಿಸಿರಲಿಲ್ಲ. ಇಂತಹ ಮಹಾಸತ್ಯವನ್ನು ಅದುವರೆಗೆ ಯಾರೂ ಆವಿಷ್ಕಾರ ಮಾಡಿರಲಿಲ್ಲ. ಈ ಮಹಾಸತ್ಯ ಹಿಂದೂಗಳಾದ ನಮ್ಮ ಜನಜೀವನದ ಉಸಿರಾಗಿದೆ. ಈ ಸತ್ಯವು ಹಿಂದೂಗಳಾದ ನಮ್ಮ ರಾಷ್ಟ್ರಜೀವನದ ಬೆನ್ನೆಲುಬಾಗಿದೆ. ನಮ್ಮ ರಾಷ್ಟ್ರ ಜೀವನದ ಶತಮಾನಗಳ ಪರಿಧಿಯಲ್ಲಿ ಈ ಸತ್ಯವು, ಕ್ರಮ ಕ್ರಮವಾಗಿ, ಹೆಚ್ಚು ಹೆಚ್ಚಾಗಿ ಉದ್ಘೋಷವಾಗುತ್ತಿದೆ. ಈ ಸತ್ಯ ನಮ್ಮಲ್ಲಿ ಪ್ರಚಾರವಾಗುತ್ತ ಆಗುತ್ತ, ನಮ್ಮ ರಾಷ್ಟ್ರ ಜೀವನದಲ್ಲಿ ಓತಪ್ರೋತವಾಗಿದೆ. ನಮ್ಮ ರಕ್ತದ ಪ್ರತಿಯೊಂದು ಬಿಂದುವಿನಲ್ಲೂ ಅನುರಣಿತವಾಗಿದೆ. ಈ ಮಹಾಸತ್ಯವೇ ನಮ್ಮ ಜೀವನದ ಉಸಿರಾಗಿದೆ. ನಮ್ಮದು ಧರ್ಮ ಸಹಿಷ್ಣುತೆಯ ವೈಭವೋಪೇತ ರಾಷ್ಟ್ರವಾಗಿದೆ. ನಮ್ಮ ಧರ್ಮವನ್ನು ಅಲ್ಲಗಳೆಯುವುದಕ್ಕೆ ಬಂದ ಜನರಿಗೆ ಚರ್ಚುಗಳನ್ನು ಕಟ್ಟಿಕೊಟ್ಟಿರುವುದು ಈ ದೇಶದಲ್ಲಿ ಮಾತ್ರ. ಜಗತ್ತು ಈ ಒಂದು ಮಹಾತತ್ತ್ವವನ್ನು ನಮ್ಮಿಂದ ಕಲಿಯುವುದಕ್ಕೆ ಕಾಯುತ್ತಿದೆ. ಹೊರಗೆ ಎಷ್ಟರ ಮಟ್ಟಿನ ಧಾರ್ಮಿಕ ಅಸಹಿಷ್ಣುತೆ ಇದೆ ಎಂಬುದು ನಿಮಗೆ ತಿಳಿಯದು. ಇಷ್ಟರಮಟ್ಟಿನ ಧಾರ್ಮಿಕ ಅಸಹಿಷ್ಣುತೆಯ ಪರಿಣಾಮವಾಗಿ ನಾನು ಅನ್ಯದೇಶಗಳಲ್ಲಿ ಮರಣಕ್ಕೆ ತುತ್ತಾಗುವೆನೆಂದು ಭಾವಿಸಿದ್ದೆ. ಧರ್ಮಕ್ಕಾಗಿ ಮತ್ತೊಬ್ಬರನ್ನು ಕೊಲ್ಲಲು ಅವರು ಸ್ವಲ್ಪವೂ ಹೇಸುವುದಿಲ್ಲ. ಇಂದು ಹಾಗೆ ಮಾಡದೇ ಇದ್ದರೆ ನಾಳೆ ಪಾಶ್ಚಾತ್ಯ ಸಭ್ಯತಾ ಕೇಂದ್ರದಲ್ಲಿಯೇ ಅದನ್ನು ಮಾಡಬಹುದು. ಪಶ್ಚಿಮದಲ್ಲಿ ತಮ್ಮ ದೇಶದ ಧರ್ಮಕ್ಕೆ ವಿರೋಧವಾಗಿ ಯಾರಾದರೂ ಮಾತನಾಡಿದರೆ ಆತನಿಗೆ ಅತಿ ಭಯಾನಕ ಬಹಿಷ್ಕಾರ ಅನೇಕ ವೇಳೆ ಕಾದು ಕುಳಿತಿದೆ. ಆದರೂ ಅವರು ಇಲ್ಲಿರುವ ನಮ್ಮ ವರ್ಣಾಶ್ರಮಗಳನ್ನು ಕುರಿತು ತಮಗೆ ತೋರಿದಂತೆ ಮಾತನಾಡುವರು. ನನ್ನಂತೆ ನೀವು ಪಾಶ್ಚಾತ್ಯ ದೇಶಗಳಿಗೆ ಹೋಗಿ ಅಲ್ಲಿ ಇದ್ದರೆ, ಪ್ರಖ್ಯಾತ ವಿದ್ವಾಂಸರೇ ಜನರಿಗೆ ಅಂಜಿ ಹೇಡಿಗಳಂತೆ ತಾವು ನಂಬುವ ಧಾರ್ಮಿಕ ಸತ್ಯದ ನೂರನೆ ಒಂದು ಭಾಗವನ್ನೂ ವ್ಯಕ್ತಪಡಿಸಲಾರರೆಂಬುದು ಗೊತ್ತಾಗುವುದು. 

ಅನ್ಯಧರ್ಮ ಸಹಿಷ್ಣುತೆ ಎಂಬ ಮಹಾಸತ್ಯಕ್ಕಾಗಿ ಜಗತ್ತು ಕಾಯುತ್ತಿದೆ. ಅದು ಲಭ್ಯವಾದಲ್ಲಿ ನಾಗರಿಕತೆಗೆ ಅದೊಂದು ದೊಡ್ಡ ಲಾಭವಾಗುತ್ತದೆ. ಅಷ್ಟೇ ಅಲ್ಲ, ಅದಿಲ್ಲದೆ ಯಾವ ನಾಗರಿಕತೆಯೂ ಬಹುಕಾಲ ಬಾಳಲಾರದು. ಯಾವ ನಾಗರಿಕತೆಯೂ ಮತಭ್ರಾಂತಿ, ರಕ್ತಪಾತ, ಹಿಂಸೆ ಇವು ನಿಂತಲ್ಲದೆ ಅಭಿವೃದ್ಧಿಯಾಗಲಾರದು. ಒಬ್ಬರು ಮತ್ತೊಬ್ಬರನ್ನು ಉದಾರ ಭಾವದಿಂದ ನೋಡುವವರೆಗೆ ಯಾವ ನಾಗರಿಕತೆಯೂ ಗೌರವದಿಂದ ತಲೆ ಎತ್ತಲಾರದು. ಅತ್ಯಂತ ಅಗತ್ಯವಾದ ಆ ಉದಾರ ಗುಣವನ್ನು ಸಂಪಾದಿಸಲು ಇಡಬೇಕಾದ ಮೊದಲ ಹೆಜ್ಜೆಯೇ ಅನ್ಯ ಧರ್ಮಗಳ ನಂಬಿಕೆ\-ಗಳನ್ನು ಔದಾರ್ಯದಿಂದ ಕಾಣುವುದು. ನಾವು ಅದನ್ನು ಅರ್ಥಮಾಡಿಕೊಳ್ಳಲು ಸಹಾನುಭೂತಿಯನ್ನು ತೋರಿಸುವುದು ಮಾತ್ರವಲ್ಲ; ನಮ್ಮ ಧರ್ಮದ ನಂಬಿಕೆ ಸಿದ್ಧಾಂತಗಳು ಅವರ ಧರ್ಮಕ್ಕಿಂತ ಎಷ್ಟೋ ಭಿನ್ನವಾಗಿದ್ದರೂ ಒಬ್ಬರು ಇನ್ನೊಬ್ಬರಿಗೆ ಸಹಾಯ ಮಾಡಬೇಕು. ನಾನು ನಿಮಗೆ ಈಗತಾನೆ ಹೇಳಿದಂತೆ ಭರತಖಂಡದಲ್ಲಿ ನಾವು ಮಾಡುತ್ತಿರುವುದೇ ಇದನ್ನು. ಭರತಖಂಡದಲ್ಲಿ ಹಿಂದೂಗಳು ಕ್ರೈಸ್ತರಿಗೆ ಚರ್ಚನ್ನೂ ಮತ್ತು ಮಹಮ್ಮದೀಯರಿಗೆ ಮಸೀದಿಯನ್ನೂ ಕಟ್ಟಿಕೊಟ್ಟಿರುವರು. ಮಾಡಬೇಕಾದದ್ದು ಅದೇ. ಅವರು ನಮ್ಮನ್ನು ಅತಿ ಹೀನವಾಗಿ ನಿಂದಿಸುತ್ತಿರಲಿ, ಹಿಂಸಿಸಲಿ, ಮೃಗಗಳಂತೆ ವರ್ತಿಸಲಿ, ಚಿಂತೆಯಿಲ್ಲ. ಪ್ರೀತಿಯಿಂದ ಅವರನ್ನು ಗೆಲ್ಲುವವರೆಗೆ, ಪ್ರೀತಿಯೊಂದೇ ಬಾಳಲು ಯೋಗ್ಯ, ದ್ವೇಷವಲ್ಲ. ಸಾಧು ಸ್ವಭಾವದಲ್ಲಿ ಮಾತ್ರ ನಾವು ಸೌಹಾರ್ದದಿಂದ ಬಾಳಬಹುದು; ಕೇವಲ ಮೃಗೀಯ ಬಲ ಪ್ರದರ್ಶನದಿಂದ ಅಲ್ಲ ಎಂಬುದನ್ನು ಪ್ರದರ್ಶಿಸುವುದಕ್ಕಾಗಿ, ಕ್ರೈಸ್ತರಿಗೆ ಚರ್ಚನ್ನು ಕಟ್ಟುತ್ತಾ ಹೋಗೋಣ, ಮಹಮ್ಮದೀಯರಿಗೆ ಮಸೀದಿಯನ್ನು ಕಟ್ಟುತ್ತಾ ಹೋಗೋಣ. 

ಯೂರೋಪಿನಲ್ಲಿ ಮಾತ್ರವಲ್ಲ, ಇಡೀ ವಿಶ್ವದ ವಿಚಾರಶೀಲ ಜನರು ಮತ್ತೊಂದು ಮಹಾಸತ್ಯವನ್ನು ನಮ್ಮಿಂದ ನಿರೀಕ್ಷಿಸುತ್ತಿರುವರು. ಅದೇನೆಂದರೆ, ಇಡಿಯ ಸೃಷ್ಟಿಯ ಆಧ್ಯಾತ್ಮಿಕ ಏಕತ್ವ ಎಂಬ ಸನಾತನ ಮಹಾತತ್ತ್ವ. ಇದು ಇಂದು ಸುಸಂಸ್ಕೃತರಿಗಿಂತ ಹೆಚ್ಚಾಗಿ ಜನಸಾಮಾನ್ಯರಿಗೆ, ವಿದ್ಯಾವಂತರಿಗಿಂತ ಹೆಚ್ಚಾಗಿ ಅಜ್ಞಾನಿಗಳಿಗೆ, ಉಚ್ಚ ವರ್ಗದಲ್ಲಿರುವವರಿಗಿಂತ ಹೆಚ್ಚಾಗಿ ನೀಚರಿಗೆ ಬೇಕಾಗಿದೆ. ಬಲಾಢ್ಯರಿಗಿಂತ ಹೆಚ್ಚಾಗಿ ಬಲಹೀನರಿಗೆ ಬೇಕಾಗಿದೆ. ಮದ್ರಾಸಿನ ವಿಶ್ವವಿದ್ಯಾ ನಿಲಯದ ಪದವೀಧರರೇ, ಪಾಶ್ಚಾತ್ಯ ವೈಜ್ಞಾನಿಕ ಸಂಶೋಧನೆಗಳು ಭೌತಿಕ ಸಾಧನಗಳ ಮೂಲಕ ವಿಶ್ವವೆಲ್ಲಾ ಒಂದೇ ಅಖಂಡಸತ್ಯ ಎಂಬುದನ್ನು ಹೇಗೆ ಪ್ರದರ್ಶಿಸುತ್ತಿವೆ ಎಂದು ನಿಮಗೆ ಹೇಳಬೇಕಾಗಿಲ್ಲ; ಅನಂತ ವಸ್ತುಸಾಗರದಲ್ಲಿ, ನಾನು ನೀವು ಎಲ್ಲರೂ, ಸೂರ್ಯ ಚಂದ್ರ ನಕ್ಷತ್ರಗಳೂ, ಬಾಹ್ಯ ಸ್ಥೂಲಭಾಷೆಯಲ್ಲಿ ಹೇಳಬೇಕಾದರೆ, ಬರಿಯ ಅಲೆಗಳೆಂಬುದನ್ನು ವಿಜ್ಞಾನ ಸಾರಿದೆ. ಇದರಂತೆಯೇ ಹಿಂದೂ ಮನಃಶ್ಶಾಸ್ತ್ರವು ಹಲವು ಶತಮಾನಗಳ ಹಿಂದೆಯೇ, ಸಮಷ್ಟಿ ಎಂಬ ವಸ್ತು ಮಹಾಸಾಗರದಲ್ಲಿ ದೇಹ ಮತ್ತು ಮನಸ್ಸುಗಳು ಕೇವಲ ಹೆಸರುಗಳು ಅಥವಾ ಕಿರು ಅಲೆಗಳೆಂದು ಸಾರಿತು. ವೇದಾಂತವು ಮತ್ತೊಂದು ಹೆಜ್ಜೆ ಮುಂದೆ ಹೋಗಿ ಪ್ರಕೃತಿಯ ಏಕತ್ವದ ಭಾವನೆಯ ಹಿಂದೆ ಇರುವುದೊಂದೇ ಆತ್ಮವೆಂದು ಸಾರಿದೆ. ಸೃಷ್ಟಿಯಲ್ಲೆಲ್ಲಾ ಇರುವುದೊಂದೇ ಆತ್ಮ. ಎಲ್ಲಾ ಒಂದೇ ಅಸ್ತಿತ್ವ. ಪ್ರಪಂಚದ ಮೂಲದಲ್ಲಿ ಒಂದು ಏಕತ್ವವಿದೆ ಎಂಬ ಮಹಾಭಾವನೆ ಈ ದೇಶದಲ್ಲಿಯೂ ಹಲವು ಭಯಕ್ಕೆ ಕಾರಣವಾಗಿದೆ. ಈಗಲೂ ಇದರ ಅನುಯಾಯಿಗಳಿಗಿಂತ ಇದನ್ನು ವಿರೋಧಿಸುವವರೇ ಹೆಚ್ಚು. ಆದರೆ ಇಂದು ಜಗತ್ತು ಈ ಜೀವನ ಸಂಜೀವಿನಿಯನ್ನು ನಮ್ಮಿಂದ ನಿರೀಕ್ಷಿಸುತ್ತಿದೆ ಎಂದು ಹೇಳುವೆನು. ಇದು ಭರತಖಂಡದ ಮೂಕ ಜನಕೋಟಿಯ ಉದ್ಧಾರಕ್ಕೆ ಬೇಕಾಗಿದೆ. ಏಕತ್ವದ ಮಹಾಭಾವನೆಯನ್ನು ಶ್ರೇಯಸ್ಕರವಾಗಿ ಅನುಷ್ಠಾನಕ್ಕೆ ತರುವ ಪರ್ಯಂತರ ನಾವು ಈ ನಮ್ಮ ಮಾತೃಭೂಮಿಯನ್ನು ಮೇಲೆತ್ತಲಾರೆವು. 

ವಿಚಾರಪರ ಪಶ್ಚಿಮವು ನೀತಿ ಮತ್ತು ದರ್ಶನಗಳಲ್ಲಿ ವೈಚಾರಿಕತೆಯನ್ನು ಇಂದು ಅರಸುತ್ತಿದೆ. ಒಂದು ವ್ಯಕ್ತಿ ಎಷ್ಟೇ ದೊಡ್ಡವನಾಗಿರಲಿ, ದೈವಿಕನಾಗಿರಲೀ, ಅವನ ನುಡಿಯೊಂದೇ ನೀತಿಯ ತಳಹದಿಗೆ ಪರಮ ಪ್ರಮಾಣವಾಗಲಾರದು. ಜಗತ್ತಿನ ಪ್ರಖ್ಯಾತ ಮೇಧಾವಿಗಳಿಗೆ, ನೀತಿಯ ನೆಲೆಗೆ ಇಂತಹ ಪ್ರಮಾಣ ಹಿಡಿಸುವುದಿಲ್ಲ. ನೀತಿ ನಿಯಮಗಳ ಅನುಕರಣೆಗೆ ಮಾನವ ಒಪ್ಪಿಗೆಗಿಂತ ಹೆಚ್ಚಾದ ಪ್ರಮಾಣವೊಂದು ಅವರಿಗೆ ಬೇಕಾಗಿದೆ. ನೀತಿಗೆ ಪ್ರಮಾಣವಾಗಿ ಅವರಿಗೆ ಶಾಶ್ವತ ತತ್ತ್ವವೊಂದರ ಆವಶ್ಯಕತೆಯಿದೆ. ನಿಮ್ಮಲ್ಲಿ, ನನ್ನಲ್ಲಿ, ಜಗತ್ತಿನ ಪ್ರತಿಯೊಂದರಲ್ಲಿಯೂ ಇರುವ ಏಕಾತ್ಮನಲ್ಲಿ ಅಲ್ಲದೆ ಆ ಸನಾತನ ಪ್ರಮಾಣ ನಮಗೆ ಮತ್ತೆಲ್ಲಿ ದೊರಕುವುದು? ಆತ್ಮ ಸರ್ವವ್ಯಾಪಿ ಮತ್ತು ಏಕ ಎಂಬುದೇ ಎಲ್ಲಾ ನೀತಿಗೆ ಪ್ರಾಮಾಣ್ಯವನ್ನು ನೀಡುವಂಥದ್ದು. ನಾನು, ನೀವು ಕೇವಲ ಸಹೋದರರು ಮಾತ್ರವಲ್ಲ್ಧ್ಧ್ಧ್ಧ ಸ್ವಾತಂತ್ರ್ಯಕ್ಕೆ ಹೋರಾಡಿದ ಮಾನವಜನಾಂಗದ ಸಾಹಿತ್ಯವೆಲ್ಲಾ ಇದನ್ನು ಸಾರಿರುವುದು. ನಾನು ನೀವು ನಿಜವಾಗಿ ಒಬ್ಬನೇ, ಇದೇ ಭಾರತೀಯ ಸಿದ್ಧಾಂತ. ಏಕತ್ವವೇ ಎಲ್ಲಾ ಅಧ್ಯಾತ್ಮಕ್ಕೆ ಮತ್ತು ನೀತಿಗೆ ಯುಕ್ತಿಯ ತಳಹದಿಯನ್ನು ಕೊಡಬಲ್ಲದು. ಶೋಷಿತರಾದ ನಮ್ಮ ಜನಸಾಮಾನ್ಯರಿಗೆ ಇದು ಇಂದು ಎಷ್ಟು ಅಗತ್ಯವೋ ಯೂರೋಪಿಗೂ ಇದು ಅಷ್ಟೇ ಅಗತ್ಯವಾಗಿದೆ. ಈಗಲೂ ಅವರಿಗೆ ತಿಳಿಯದಂತೆ ಈ ಮಹಾಸಿದ್ಧಾಂತವೇ ಇಂಗ್ಲೆಂಡ್​, ಜರ್ಮನಿ, ಫ್ರಾನ್​್ಸ, ಅಮೆರಿಕಾ ದೇಶಗಳ ಆಧುನಿಕ ರಾಜಕೀಯ ಮತ್ತು ಸಾಮಾಜಿಕ ಚಳುವಳಿಗಳ ಮೂಲ ತತ್ತ್ವವಾಗಿದೆ. ಮಾನವನು ಸ್ವಾತಂತ್ರ್ಯಕ್ಕಾಗಿ ನಡೆಸಿದ ಹೋರಾಟಗಳನ್ನು ನಿರೂಪಿಸುವ ಸಾಹಿತ್ಯದ ಹಿಂದೆಲ್ಲ, ಪುನಃ ಪುನಃ ವೇದಾಂತದ ಆದರ್ಶವೇ ಪ್ರಮುಖವಾಗಿ ಕಾಣುತ್ತಿರುವುದು. ಸ್ನೇಹಿತರೇ! ಇದನ್ನು ಗಮನಿಸಿ! ಕೆಲವು ಸಂದರ್ಭಗಳಲ್ಲಿ ಈ ಭಾವನೆಯನ್ನು ವ್ಯಕ್ತಪಡಿಸಿರುವ ಗ್ರಂಥಕರ್ತರಿಗೆ ಇದರ ಮೂಲ ತಿಳಿಯದು. ಕೆಲವು ವೇಳೆ ಇದು ಅವರ ಸ್ವತಂತ್ರ ಭಾವನೆಯಂತೆ ತೋರುವುದು. ಮತ್ತೆ ಕೆಲವು ವೇಳೆ ಧೀರ ಮಹಾವ್ಯಕ್ತಿಗಳಿರುವರು, ಅವರು ಈ ಭಾವನೆಯ ಮೂಲಕ್ಕೆ ತಮ್ಮ ಚಿರಕೃತಜ್ಞತೆಯನ್ನು ವ್ಯಕ್ತಪಡಿಸಿರುವರು. 

ನಾನು ಅಮೆರಿಕಾದಲ್ಲಿ ಇದ್ದಾಗ ಅದ್ವೈತವನ್ನೇ ಹೆಚ್ಚಾಗಿ ಬೋಧಿಸುತ್ತಿದ್ದೆ, ಬಹಳ ಅಲ್ಪ ದ್ವೈತವನ್ನು ಬೋಧಿಸಿದೆ ಎಂಬ ಅಕ್ಷೇಪಣೆಯನ್ನು ಕೇಳಿದೆ. ದ್ವೈತ ಸಿದ್ಧಾಂತದ ಮಹತ್ತು, ಅದರ ಉಪಾಸನಾ ಕ್ರಮದಲ್ಲಿರುವ ಪ್ರೇಮದ ವೈಶಾಲ್ಯ, ಅನಂತ ಭಾವಪೂರ್ಣ ಆನಂದ, ಈ ಎಲ್ಲವನ್ನೂ ನಾನು ಬಲ್ಲೆ. ಆದರೆ ಆನಂದಾಶ್ರು ಗಳನ್ನು ಸುರಿಸುವ ಸಮಯವು ಇದಲ್ಲ. ನಾವು ಸಾಕಷ್ಟು ಅತ್ತಿದ್ದೇವೆ. ನಾವು ಮೃದುವಾಗಿರತಕ್ಕ ಕಾಲ ಇದಲ್ಲ. ನಾವೊಂದು ಹತ್ತಿಯ ಮುದ್ದೆಯಾಗಿ ಸಾಯುವ ಸ್ಥಿತಿಗೆ ಬರುವವರೆಗೂ ಮೃದುವಾಗಿರುವೆವು. ನಮ್ಮ ದೇಶಕ್ಕೆ ಇಂದು ಬೇಕಾಗಿರುವುದು ಕಬ್ಬಿಣದಂತಹ ಮಾಂಸಖಂಡಗಳು, ಉಕ್ಕಿನಂತಹ ನರಗಳು, ಯಾವುದನ್ನೂ ಲೆಕ್ಕಿಸದೆ ವಿಶ್ವದ ರಹಸ್ಯತಮ ಸತ್ಯಗಳನ್ನು ಭೇದಿಸಿ ಸಾಧ್ಯವಾದರೆ ಕಡಲಿನ ಆಳಕ್ಕಾ ದರೂ ಹೋಗಿ ಮೃತ್ಯುವಿನೊಂದಿಗೆ ಹೋರಾಡಿ ಗುರಿಯನ್ನು ಸಾಧಿಸಬಲ್ಲ ಅದಮ್ಯ, ಪ್ರಚಂಡ ಇಚ್ಛಾಶಕ್ತಿ. ನಮಗೆ ಇಂದು ಬೇಕಾಗಿರುವುದು ಅದು. ಎಲ್ಲರೂ ಒಂದು ಎಂಬ ಅದ್ವೈತ ಸಿದ್ಧಾಂತವನ್ನು ತಿಳಿದುಕೊಳ್ಳುವುದರಿಂದ ಮಾತ್ರ ಆ ಪ್ರಚಂಡ ಇಚ್ಛಾಶಕ್ತಿಯನ್ನು ಸೃಷ್ಟಿಸಬಹುದು, ಅದಕ್ಕೆ ಪ್ರಾಣ ಪ್ರತಿಷ್ಠೆ ಮಾಡಿ ಶಕ್ತಿಯನ್ನುತುಂಬ ಬಹುದು. ಶ್ರದ್ಧೆ! ಶ್ರದ್ಧೆ! ಆತ್ಮಶ್ರದ್ಧೆ!, ಶ್ರದ್ಧೆ! ಶ್ರದ್ಧೆ! ಈಶ್ವರನಲ್ಲಿ ಶ್ರದ್ಧೆ!–ಇದೇ ಮಹಾತ್ಮ್ಯೆಯ ಮೂಲ. ನೀವು ನಿಮ್ಮ ಮೂವತ್ತಮೂರು ಕೋಟಿ ದೇವತೆ\-ಗಳನ್ನು ನಂಬಿ, ಜೊತೆಗೆ ಪಾಶ್ಚಾತ್ಯರಿಂದ ಆಮದು ಮಾಡಿಕೊಂಡಿರುವ ದೇವತೆಗಳನ್ನು ನಂಬಿದರೂ ನಿಮ್ಮಲ್ಲಿ ಆತ್ಮಶ್ರದ್ಧೆ ಇಲ್ಲದೇ ಇದ್ದರೆ ನಿಮಗೆ ಉದ್ಧಾರವಿಲ್ಲ. ಮೊದಲು ಆತ್ಮಶ್ರದ್ಧೆ ಇರಲಿ. ಆ ಶ್ರದ್ಧೆಯ ಮೇಲೆ ನಿಂತು ವೀರರಾಗಿ. ನಮಗೆ ಇಂದು ಬೇಕಾಗಿರುವುದು ಅದು. ಕಳೆದ ಒಂದು ಸಾವಿರ ವರುಷಗಳಿಂದ ಎಲ್ಲೋ ಕೆಲವು ಮಂದಿ ಹೊರಗಿನವರು, ಸತ್ವಹೀನರಾದ ಮೂವತ್ತಮೂರು ಕೋಟಿ ಜನರನ್ನು ತುಳಿದು ಆಳುವುದಕ್ಕೆ ಹೇಗೆ ಸಾಧ್ಯವಾಯಿತು? ಏಕೆಂದರೆ ಅವರಲ್ಲಿ ಆತ್ಮಶ್ರದ್ಧೆ ಇತ್ತು, ನಮ್ಮಲ್ಲಿ ಇರಲಿಲ್ಲ. ಪಾಶ್ಚಾತ್ಯ ದೇಶಗಳಲ್ಲಿ ನಾನು ಕಲಿತಿದ್ದೇನು? ಮಾನವನು ಪತಿತನಾದ, ಗತಿಯೇ ಇಲ್ಲದ ಪಾಪಿ ಎಂದು ಸಾರುವ ಕ್ರೈಸ್ತಧರ್ಮದ ಒಣಹರಟೆಯ ಹಿಂದೆ ಇರುವುದೇನು? ಯೂರೋಪು ಮತ್ತು ಅಮೆರಿಕಾ ದೇಶಗಳ ಅಂತರಾಳದಲ್ಲಿ ಪ್ರಚಂಡ ಆತ್ಮಶ್ರದ್ಧೆಯ ಶಕ್ತಿ ಉರಿಯುತ್ತಿರುವುದನ್ನು ಕಂಡೆ. ಒಬ್ಬ ಇಂಗ್ಲಿಷ್​ ಹುಡುಗ “ನಾನು ಇಂಗ್ಲೀಷಿನವನು, ಏನನ್ನು ಬೇಕಾದರೂ ಸಾಧಿಸಬಲ್ಲೆ” ಎನ್ನುತ್ತಾನೆ. ಅಮೆರಿಕಾ ದೇಶದ ಹುಡುಗನೂ ಹೀಗೆಯೇ ಹೇಳುವನು. ಯುರೋಪಿನ ಹುಡುಗನೂ ಅಷ್ಟೇ. ನಮ್ಮ ಹುಡುಗರು ಹೀಗೆ ಹೇಳಬಲ್ಲರೇ? ಇಲ್ಲ, ಹುಡುಗನ ಅಪ್ಪ ಕೂಡ ಹೀಗೆ ಹೇಳಲಾರ. ನಾವು ಆತ್ಮಶ್ರದ್ಧೆಯನ್ನು ಕಳೆದುಕೊಂಡಿರುವೆವು. ಮನುಷ್ಯರನ್ನು ಜಾಗೃತಗೊಳಿಸುವುದಕ್ಕೆ ಅವರ ಆತ್ಮದ ಮಹಿಮೆಯನ್ನು ಪ್ರಕಾಶಮಾಡುವುದಕ್ಕೆ, ವೇದಾಂತದ ಅದ್ವೈತ ಸಿದ್ಧಾಂತವನ್ನು ಬೋಧಿಸಬೇಕಾಗಿದೆ. ಆದ್ದರಿಂದಲೇ ನಾನು ಅದ್ವೈತವನ್ನು ಬೋಧಿಸುವುದು. ನಾನು ಅದನ್ನು ಸಣ್ಣ ಕೋಮಿನ ದೃಷ್ಟಿಯಿಂದ ವಿವರಿಸುವುದಿಲ್ಲ. ಎಲ್ಲರ ಒಪ್ಪಿಗೆಗೂ ಪಾತ್ರವಾದ ಸರ್ವಸಾಮಾನ್ಯ ತತ್ತ್ವಗಳ ಆಧಾರದ ಮೇಲೆ ಬೋಧಿಸುತ್ತೇನೆ. 

ದ್ವೈತ ಮತ್ತು ವಿಶಿಷ್ಟಾದ್ವೈತಗಳ ಅನುಯಾಯಿಗಳನ್ನು ನೋಯಿಸದ ಒಂದು ರಾಜಿಯ ಮಾರ್ಗವನ್ನು ಕಂಡುಹಿಡಿಯುವುದು ಸುಲಭ. ದೇವರು ಅಂತರ್ಯಾಮಿ, ಪವಿತ್ರತೆ ಎಲ್ಲದರಲ್ಲಿಯೂ ಇದೆ ಎಂಬುದನ್ನು ಒಪ್ಪದ ಒಂದು ಪಂಥವೂ ಭರತಖಂಡದಲ್ಲಿ ಇಲ್ಲ. ಪ್ರತಿಯೊಂದು ವೇದಾಂತ ದರ್ಶನವೂ, ಪಾವಿತ್ರ್ಯ, ಪರಿಪೂರ್ಣತೆ, ಶಕ್ತಿ ಎಲ್ಲವೂ ಆಗಲೇ ಆತ್ಮನಲ್ಲಿ ಇವೆ ಎಂದು ಒಪ್ಪಿಕೊಳ್ಳುವುವು. ಕೆಲವರ ದೃಷ್ಟಿಯಲ್ಲಿ ಈ ಪೂರ್ಣತೆ ಕೆಲವು ವೇಳೆ ಸಂಕುಚಿತವಾಗುವುದು ಮತ್ತೆ ಕೆಲವು ಸಲ ಅದು ವಿಕಾಸಗೊಳ್ಳುತ್ತದೆ. ಆದರೂ ಅದು ಅಲ್ಲೇ ಇದೆ. ಅದ್ವೈತಿಗಳ ದೃಷ್ಟಿಯಿಂದ ಅದು ಸಂಕೋಚಗೊಳ್ಳುವುದೂ ಇಲ್ಲ, ವಿಕಾಸಗೊಳ್ಳುವುದೂ ಇಲ್ಲ. ಆದರೆ ಕೆಲವು ವೇಳೆ ಸುಪ್ತವಾಗುವುದು, ಮತ್ತೆ ಕೆಲವು ವೇಳೆ ವ್ಯಕ್ತವಾಗುವುದು. ಪರಿಣಾಮದಲ್ಲಿ ಎರಡೂ ಹೆಚ್ಚು ಕಡಿಮೆ ಒಂದೇ. ಒಂದು ಮತ್ತೊಂದಕ್ಕಿಂತ ಯುಕ್ತಿಯುಕ್ತವಾಗಿರಬಹುದು. ಆದರೆ ಪರಿಣಾಮದ ದೃಷ್ಟಿಯಿಂದ, ಅನುಷ್ಠಾನದ ದೃಷ್ಟಿಯಿಂದ ಎರಡೂ ಒಂದೇ. ಜಗತ್ತಿಗೆ ಇಂದು ಅತ್ಯಾವಶ್ಯಕವಾಗಿ ಬೇಕಾಗಿರುವ ಮುಖ್ಯ ಭಾವನೆ ಇದು. ಈ ಭಾವನೆಯು ಎಲ್ಲ ಕಡೆಗಳಿಗಿಂತ ಹೆಚ್ಚಾಗಿ ಈ ದೇಶಕ್ಕೆ ಅಗತ್ಯವಾಗಿದೆ. 

ನನ್ನ ಸ್ನೇಹಿತರೇ, ನಾನು ನಿಮಗೆ ಕೆಲವು ಕಟುವಾದ ಸತ್ಯಗಳನ್ನು ಹೇಳಬೇಕಾಗಿದೆ. ನಮ್ಮ ದೇಶಬಾಂಧವನೊಬ್ಬನನ್ನು ಆಂಗ್ಲೇಯರು ಕೊಲೆಮಾಡಿದಾಗ ಅಥವಾ ಅವ ಹೇಳನ ಮಾಡಿದಾಗ ಇಂಗ್ಲಿಷರಿಗೆ ಶಾಪ ಕೊಡುವ ಕೂಗಿನಿಂದ ಭರತಖಂಡವೆಲ್ಲಾ ತುಂಬುವುದು ಎಂಬುದನ್ನು ನಾನು ಪತ್ರಿಕೆಗಳಲ್ಲಿ ಓದಿದ್ದೇನೆ. ನಾನು ಅದನ್ನು ಓದಿ ಕಣ್ಣೀರು ಸುರಿಸಿರುವೆನು. ಮರುಕ್ಷಣವೇ ಇದಕ್ಕೆ ಕಾರಣ ಯಾರು ಎಂಬ ಪ್ರಶ್ನೆ ಏಳುವುದು. ವೇದಾಂತಿಯಾಗಿರುವುದರಿಂದ ಆ ಪ್ರಶ್ನೆಯನ್ನು ನನಗೆ ನಾನೇ ಹಾಕಿಕೊಳ್ಳದೆ ವಿಧಿಯಿಲ್ಲ. ಹಿಂದೂವು ಅಂತರ್ಮುಖಿ; ಅವನು ಪ್ರತಿಯೊಂದನ್ನು ತನ್ನ ಮೂಲಕ ನೋಡಲಿಚ್ಛಿಸುವನು. ಇದಕ್ಕೆ ಜವಾಬ್ದಾರರು ಯಾರು ಎಂದು ಹಾಕುವ ಪ್ರಶ್ನೆಗೆ ಪ್ರತಿಸಲವೂ ಇಂಗ್ಲಿಷರಲ್ಲವೆಂಬ ಉತ್ತರ ಬರುವುದು. ನಮ್ಮ ದುಃಖಕ್ಕೆ, ನಮ್ಮ ಅವನತಿಗೆ ನಾವೇ ಕಾರಣಕರ್ತರು. ನಮ್ಮ ಅದೃಷ್ಟಕ್ಕೆ ನಾವೇ ಹೊಣೆ. ನಮ್ಮ ಶ‍್ರೀಮಂತ ಕುಲೀನರು ಜನಸಾಮಾನ್ಯರನ್ನು ತುಳಿಯುತ್ತಾ ಹೋದರು. ಅವರು ನಿರಾಶರಾಗಿ ಈ ಕೋಟಲೆಯಲ್ಲಿ ತಾವು ಮನುಷ್ಯ\-ರೆಂಬುದನ್ನೂ ಮರೆತರು. ಹಲವು ಶತಮಾನಗಳಿಂದ ಅವರು ಕಟ್ಟಿಗೆ ಒಡೆಯುವ, ನೀರೆಳೆಯುವ ಶ್ರಮಜೀವಿ ಗಳಾಗಿರುವರು. ಈಗ ಅವರು ಹುಟ್ಟು ಗುಲಾಮರು. ತಾವು ಹುಟ್ಟಿರುವುದೇ ನೀರೆಳೆಯುವುದಕ್ಕೆ ಮತ್ತು ಕಟ್ಟಿಗೆ ಒಡೆಯುವುದಕ್ಕೆ ಎಂಬ ದೃಢನಂಬಿಕೆ ಇಂದು ಅವರಲ್ಲಿ ಬಂದುಹೋಗಿದೆ. ನಾವು ಎಷ್ಟರ ಮಟ್ಟಿಗೆ ಆಧುನಿಕ ವಿದ್ಯೆಯನ್ನು ಪಡೆದವರಾಗಿದ್ದರೂ ಅವರ ಬಗ್ಗೆ ಒಂದಾದರೂ ಒಳ್ಳೆಯ ಮಾತನ್ನು ಆಡುತ್ತೇವೆಯೇ? ದಲಿತರಾದ, ದರಿದ್ರರಾದ ಆ ಜನರನ್ನು ಮೇಲೆತ್ತುವ ಕೆಲಸ ಬಂದಾಗ ನಮ್ಮ ಜನರು ಕೂಡಲೇ ಹಿಂಜರಿಯುವುದನ್ನು ನಾನು ನೋಡಿದ್ದೇನೆ. ಇದು ಮಾತ್ರವಲ್ಲ, ಪಾಶ್ಚಾತ್ಯರಿಂದ ಬಂದ ಆನುವಂಶೀಯತೆ ಮುಂತಾದ ನಿರರ್ಥಕ ಸಿದ್ಧಾಂತಗಳ ಸಹಾಯದಿಂದ ರಾಕ್ಷಸೀಯ ವಾದಗಳನ್ನು ತಯಾರುಮಾಡಿ ದೀನರನ್ನು ಮತ್ತೂ ಹೆಚ್ಚಾಗಿ ಪೀಡಿಸುವುದಕ್ಕೆ ಅವನ್ನು ಬಳಸುತ್ತಿದ್ದೇವೆ. ಅಮೆರಿಕಾ ದೇಶದ ವಿಶ್ವಧರ್ಮ ಸಮ್ಮೇಳನದಲ್ಲಿ ಆಫ್ರಿಕಾ ದೇಶದಲ್ಲಿ ಹುಟ್ಟಿದ ಒಬ್ಬ ನೀಗ್ರೋ ಬಂದು ಮಾತನಾಡಿದನು. ಅವನು ಸೊಗಸಾದ ಭಾಷಣ ಮಾಡಿದನು. ನಾನು ಅವನಲ್ಲಿ ಆಸಕ್ತನಾಗಿ ಆಗಾಗ ಅವನೊಂದಿಗೆ ಮಾತನಾಡಿದೆ. ಆದರೆ ಆತ ತನ್ನ ವಿಷಯ ಏನನ್ನೂ ಹೇಳಲಿಲ್ಲ. ನಾನು ಇಂಗ್ಲೆಂಡಿನಲ್ಲಿ ಒಂದು ದಿನ ಕೆಲವು ಅಮೆರಿಕನರನ್ನು ಕಂಡೆ. ಅವರು ಈ ಹುಡುಗನ ವಿಚಾರವಾಗಿ ಹೀಗೆ ಹೇಳಿದರು: ಈತ ನೀಗ್ರೋ ಜನಾಂಗದ ಒಬ್ಬ ಮುಖಂಡನ ಮಗ. ಮತ್ತೊಂದು ಪಂಗಡದ ಮುಖಂಡ ಇವನ ತಂದೆಯ ಮೇಲೆ ಕೋಪಗೊಂಡು ಈತನ ತಂದೆ ತಾಯಿಗಳನ್ನು ಕೊಂದು ಅವರ ಮಾಂಸವನ್ನು ಬೇಯಿಸಿ ತಿಂದನು. ಹುಡುಗನನ್ನೂ ಕೊಂದು ತಿನ್ನುವಂತೆ ತನ್ನವರಿಗೆ ಆಜ್ಞಾಪಿಸಿದನು. ಆದರೆ ಹುಡುಗ ಹೇಗೊ ತಪ್ಪಿಸಿಕೊಂಡನು. ಬಹಳ ಕಷ್ಟನಷ್ಟಗಳನ್ನು ಅನುಭವಿಸಿ ನೂರಾರು ಮೈಲಿಗಳನ್ನು ನಡೆದು ಸಮುದ್ರದ ಕರೆಯನ್ನು ಸೇರಿದನು. ಅಲ್ಲಿ ಒಂದು ಹಡಗಿನಲ್ಲಿ ಅಮೆರಿಕಾ ದೇಶ ಸೇರಿದನು. ಆ ಭಾಷಣವನ್ನು ಮಾಡಿದವನೇ ಆ ಹುಡುಗ ಎಂದರು. ಇದನ್ನು ನೋಡಿದ ಮೇಲೆ ನಿಮ್ಮ ಆನುವಂಶಿಕ ಸಿದ್ಧಾಂತ ಏನಾಯಿತು ನೋಡಿ!

ಹೇ ಬ್ರಾಹ್ಮಣರೆ! ಆನುವಂಶಿಕವಾಗಿ ಬಂದ ಗುಣಗಳ ಪ್ರಕಾರ ಬ್ರಾಹ್ಮಣನಿಗೆ ಹೊಲೆಯನಿಗಿಂತ ವಿದ್ಯಾಭ್ಯಾಸಕ್ಕೆ ಹೆಚ್ಚು ಯೋಗ್ಯತೆ ಇದ್ದರೆ ಪರೆಯನ ವಿದ್ಯಾಭ್ಯಾಸಕ್ಕೆ ಹೆಚ್ಚು ಖರ್ಚುಮಾಡಿ, ಬ್ರಾಹ್ಮಣನಿಗಾಗಿ ಖರ್ಚುಮಾಡಬೇಡಿ. ದುರ್ಬಲರಿಗೆ ಕೊಡಿ. ಅಲ್ಲಿ ನಿಮ್ಮ ದಾನವೆಲ್ಲ ಬೇಕಾಗಿದೆ. ಬ್ರಾಹ್ಮಣನು ಹುಟ್ಟಿನಿಂದ ಹೆಚ್ಚು ಬುದ್ಧಿವಂತನಾಗಿದ್ದರೆ ಯಾವ ಸಹಾಯವೂ ಇಲ್ಲದೆ ಬುದ್ಧಿವಂತನಾಗಬಲ್ಲ. ಉಳಿದವರು ಬುದ್ಧಿವಂತರಾಗಿಲ್ಲದೇ ಇದ್ದರೆ ಅವರಿಗೆ ಶಿಕ್ಷಣ ಮತ್ತು ಗುರುಗಳು ಮೀಸ\-ಲಾಗಿರಲಿ. ನನಗೆ ಗೋಚರಿಸುವ ಮಟ್ಟಿಗೆ ಇದು ನ್ಯಾಯ ಮತ್ತು ಯುಕ್ತಿ ಯುಕ್ತವಾದುದು. ಭರತಖಂಡದಲ್ಲಿ ದಬ್ಬಾಳಿಕೆಗೆ ತುತ್ತಾದ ಈ ನಿರ್ಭಾಗ್ಯರು ತಮ್ಮ ನಿಜಸ್ವರೂಪವನ್ನು ತಿಳಿದುಕೊಳ್ಳಬೇಕಾಗಿದೆ. ಪ್ರತಿಯೊಬ್ಬ ಸ್ತ್ರೀಪುರುಷ ಮತ್ತು ಮಗು ಜಾತಿ–ಕುಲ–ಗೋತ್ರಗಳ ಭೇದವಿಲ್ಲದೆ, ದುರ್ಬಲರು–ಬಲಾಢ್ಯರು ಎಂಬ ಭೇದ ಭಾವವಿಲ್ಲದೆ, ಸಬಲ–ದುರ್ಬಲರ ಹಿಂದೆ, ಉಚ್ಚ–ನೀಚರ ಹಿಂದೆ, ಪ್ರತಿಯೊಬ್ಬರಲ್ಲಿಯೂ ಅನಂತಾತ್ಮನಿರುವನು ಎಂಬುದನ್ನು ಅರಿತುಕೊಳ್ಳಲಿ. ಎಲ್ಲರೂ ಸತ್ಪುರುಷರಾಗುವುದಕ್ಕೆ ಅನಂತ ಅವಕಾಶವಿದೆ. ಎಲ್ಲರಲ್ಲಿಯೂ ಅನೇಕ ಶಕ್ತಿ ಇದೆ. ಪ್ರತಿಯೊಬ್ಬರಿಗೂ “ಉತ್ತಿಷ್ಠತ ಜಾಗ್ರತ, ಪ್ರಾಪ್ಯವರಾನ್ನಿಬೋಧತ” “ಏಳಿ, ಜಾಗೃತರಾಗಿ, ಗುರಿ ಸೇರುವವರೆಗೂ ನಿಲ್ಲಬೇಡಿ” ಎಂದು ಬೋಧಿಸೋಣ. ಏಳಿ, ದುರ್ಬಲತೆಯ ಪರವಶತೆಯಿಂದ ಜಾಗೃತರಾಗಿ. ಯಾರೂ ದುರ್ಬಲರಲ್ಲ. ಆತ್ಮ ಸರ್ವಶಕ್ತ, ಸರ್ವಜ್ಞ. ಏಳಿ, ಯಾರಿಗೂ ಮಣಿಯಬೇಡಿ. ನಿಮ್ಮಲ್ಲಿರುವ ಭಗವಂತನನ್ನು ವ್ಯಕ್ತಗೊಳಿಸಿ. ಅದನ್ನು ಅಲ್ಲಗಳೆಯಬೇಡಿ. ಹಿಂದೆ ಬೇಕಾದಷ್ಟು ನಿಷ್ಕ್ರಿಯತೆ ದೌರ್ಬಲ್ಯ, ನಮ್ಮ ಜನಾಂಗದ ಮೇಲೆ ಇತ್ತು. ಈಗಲೂ ಇದೆ. ಆಧುನಿಕ ಹಿಂದೂಗಳೇ, ಆ ಪರವಶತೆಯಿಂದ ಪಾರಾಗಿ. ಪಾರಾಗುವುದಕ್ಕೆ ಮಾರ್ಗ ನಿಮ್ಮ ಶಾಸ್ತ್ರಗಳಲ್ಲಿಯೇ ಇದೆ. ನಿಜವಾದ ಆತ್ಮನ ವಿಷಯವನ್ನು ನೀವು ಕೇಳಿ, ಇತರರಿಗೆ ಬೋಧಿಸಿ. ನಿದ್ರಿಸುತ್ತಿರುವ ಜೀವವನ್ನು ಎಬ್ಬಿಸಿ, ಹೇಗೆ ಅದು ಜಾಗೃತವಾಗುತ್ತದೆ ಎಂಬುದನ್ನು ನೋಡಿ. ಆಗ ಶಕ್ತಿ ಬರುವುದು, ಮಹಿಮೆ ಬರುವುದು, ಶ್ರೇಯಸ್ಸು ಬರುವುದು, ಪಾವಿತ್ರ್ಯ ಬರುವುದು. ಈಗ ನಿದ್ರಿಸುತ್ತಿರುವ ಜೀವ ಜಾಗೃತವಾಗಿ ಕಾರ್ಯೋ ನ್ಮುಖವಾದಾಗ ಅವನಲ್ಲಿ ಎಲ್ಲಾ ಕಲ್ಯಾಣ ಗುಣಗಳೂ ಬರುವುವು. ಗೀತೆಯಲ್ಲಿ ನಾನು ಯಾವುದನ್ನಾದರೂ ಹೆಚ್ಚಾಗಿ ಮೆಚ್ಚುವುದಿದ್ದರೆ, ಅದೇ ಶ‍್ರೀಕೃಷ್ಣನ ಬೋಧನೆಯ ಸಾರದಂತಿರುವ ಈ ಎರಡು ಶ್ಲೋಕಗಳು:

\begin{verse}
\textbf{“ಸಮಂ ಸರ್ವೇಷು ಭೂತೇಷು ತಿಷ್ಠಂತಂ ಪರಮೇಶ್ವರಂ}\\\textbf{ವಿನಶ್ಯತ್ಸ್ವವಿನಶ್ಯಂತಂ ಯಃ ಪಶ್ಯತಿ ಸ ಪಶ್ಯತಿ~॥} 
\end{verse}

\begin{verse}
\textbf{ಸಮಂ ಪಶ್ಯನ್​ ಹಿ ಸರ್ವತ್ರ ಸಮವಸ್ಥಿತಮೀಶ್ವರಂ}\\\textbf{ನ ಹಿನಸ್ತ್ಯಾತ್ಮನಾತ್ಮಾನಂ ತತೋ ಯಾತಿ ಪರಾಂ ಗತಿಮ್​~॥}
\end{verse}

“ಯಾರು ಎಲ್ಲ ಭೂತಗಳಲ್ಲೂ ಸಮನಾಗಿ ನೆಲೆಸಿರುವ ಅವಿನಾಶಿಯಾಗಿರು ವಂತಹ ಪರಮೇಶ್ವರನನ್ನು ವಿನಾಶಿಗಳಾಗಿರುವ ವಸ್ತುಗಳಲ್ಲಿ ನೋಡುವರೋ ಅವರೇ ನಿಜವಾಗಿ ನೋಡುವರು.”

“ಯಾರು ದೇವರನ್ನು ಎಲ್ಲಾ ಕಡೆಗಳಲ್ಲೂ ಒಂದೇ ಸಮನಾಗಿ ನೋಡುತ್ತಿರುವರೋ ಅವರು ಆತ್ಮನಿಂದ ಆತ್ಮನನ್ನು ನಾಶಮಾಡದೆ ಪರಮಗತಿಯನ್ನು ಹೊಂದುವರು.”

ವೇದಾಂತದ ತತ್ತ್ವಪ್ರಚಾರದಿಂದ ಭರತಖಂಡದಲ್ಲಿ ಮತ್ತು ಹೊರಗೆ ಪರೋಪಕಾರವನ್ನು ಮಾಡುವುದಕ್ಕೆ ಒಳ್ಳೆಯ ಅವಕಾಶ ದೊರೆತಿದೆ. ಇಲ್ಲಿ ಮತ್ತು ಹೊರಗೆ ಮಾನವಕೋಟಿಯ ದುಃಖವನ್ನು ದೂರಮಾಡಿ ಮಾನವರು ಶ್ರೇಯಸ್ಸಿನ ಮಾರ್ಗದ ಕಡೆ ಹೋಗುವಂತೆ ಮಾಡುವುದಕ್ಕಾಗಿ, ಪರಮಾತ್ಮನ ಸರ್ವವ್ಯಾಪಿತ್ವ ಮತ್ತು ಅವನು ಸರ್ವರಲ್ಲಿ ಒಂದೇ ಸಮನಾಗಿರುವನು ಎಂಬ ಮಹಾಸಿದ್ಧಾಂತವನ್ನು ಪ್ರಚಾರ ಮಾಡಬೇಕು. ಎಲ್ಲಿ ಪಾಪವಿದೆಯೋ, ಅಜ್ಞಾನವಿದೆಯೋ ಅದಕ್ಕೆಲ್ಲಾ, ನಮ್ಮ ಶಾಸ್ತ್ರ\-ಪ್ರಕಾರ, ಕಾರಣ ಭಿನ್ನತೆ ಎಂದು ಒತ್ತಿ ಹೇಳಬೇಕು. ಸರ್ವಸಮಾನತೆಯಲ್ಲಿ ಶ್ರದ್ಧೆ ಮತ್ತು ನಂಬಿಕೆ ಒಳ್ಳೆಯದನ್ನು ನಮಗೆ ನೀಡುತ್ತದೆ. ಇದೇ ವೇದಾಂತದ ಪ್ರಖ್ಯಾತ ಸಿದ್ಧಾಂತ. ಇದನ್ನು ಆದರ್ಶವಾಗಿಟ್ಟುಕೊಂಡಿರುವುದೊಂದು, ಅದನ್ನು ಅನುಷ್ಠಾನಕ್ಕೆ ತರುವುದು ಮತ್ತೊಂದು. ಒಂದು ಆದರ್ಶವನ್ನು ತೋರುವುದು. ಇದು ಸುಲಭ. ಅದನ್ನು ಸಾಧಿಸುವುದಕ್ಕೆ ಮಾರ್ಗ ಯಾವುದು?

ನಮ್ಮ ದೇಶದ ಜನರ ಮನಸ್ಸನ್ನು ಹಲವು ಶತಮಾನಗಳಿಂದ ಬಹಳ ಕಾಡುತ್ತಿರುವ ಜಾತಿಪದ್ಧತಿಗೆ, ಸಮಾಜಸುಧಾರಣೆಗೆ ಸಂಬಂಧಪಟ್ಟ ಕೆಲವು ಜಟಿಲ ಸಮಸ್ಯೆಗಳು ಉದ್ಭವವಾಗುತ್ತವೆ. ನಾನು ಸಮಾಜಸುಧಾರಕನೂ ಅಲ್ಲ, ಜಾತಿ ಪದ್ಧತಿಯನ್ನು ನಾಶಮಾಡುವವನೂ ಅಲ್ಲ ಎಂದು ಸ್ಪಷ್ಟವಾಗಿ ಹೇಳಬೇಕಾಗಿದೆ. ನಿಮ್ಮ ಜಾತಿ ಪದ್ಧತಿಗೂ ಮತ್ತು ಸಮಾಜಸುಧಾರಣೆಗೂ, ನನಗೂ ನೇರವಾದ ಯಾವ ಸಂಬಂಧವೂ ಇಲ್ಲ. ನೀವು ಯಾವ ಜಾತಿಯಲ್ಲಾದರೂ ಇರಿ. ಆದರೆ ಬೇರೆ ಜಾತಿಯವರನ್ನು ಅಥವಾ ಅನ್ಯರನ್ನು ನೀವು ಏತಕ್ಕೆ ದ್ವೇಷಿಸಬೇಕು? ಪ್ರೀತಿ, ಪ್ರೀತಿಯೊಂದನ್ನೇ ನಾನು ಬೋಧಿಸುವುದು. ಸರ್ವವ್ಯಾಪಿತ್ವ ಮತ್ತು ಸಮಾನತೆಯ ವೇದಾಂತದ ತಳ ಹದಿಯೇ ನನ್ನ ಬೋಧನೆಗೆ ಮೂಲ. ಕಳೆದ ನೂರು ವರ್ಷಗಳಿಂದ ಸಮಾಜಸುಧಾರಕರು ಮತ್ತು ಸಮಾಜಸುಧಾರಣೆಗಳು ದೇಶದಲ್ಲೆಲ್ಲಾ ತುಂಬಿಕೊಂಡಿವೆ. ವೈಯಕ್ತಿಕವಾಗಿ ಈ ಸುಧಾರಕರಲ್ಲಿ ಯಾರಲ್ಲಿಯೂ ನಾನು ದೋಷವನ್ನು ಕಾಣುತ್ತಿಲ್ಲ. ಮುಕ್ಕಾಲುಪಾಲು ಅವರೆಲ್ಲಾ ಒಳ್ಳೆಯವರು. ಅವರ ಉದ್ದೇಶಗಳು ಒಳ್ಳೆಯವು. ಕೆಲವು ಅಂಶಗಳಲ್ಲಿ ಅವರ ಗುರಿಯೂ ಮೆಚ್ಚತಕ್ಕದ್ದೇ. ಆದರೆ ಒಂದು ನೂರು ವರ್ಷದ ಸುಧಾರಣೆಯಿಂದ ಸ್ಪಷ್ಟವಾಗಿ, ಶಾಶ್ವತವಾದ, ಸಾರ್ಥಕವಾದ ಯಾವ ಪ್ರತಿಫಲವೂ ದೊರೆತಿಲ್ಲ. ವೇದಿಕೆಗಳ ಮೇಲೆ ನಿಂತು ಸಾವಿರಾರು ಉಪನ್ಯಾಸಗಳನ್ನು ನೀಡಿದ್ದಾರೆ, ಹಿಂದೂ ಜನಾಂಗ ಮತ್ತು ನಾಗರಿಕತೆಯ ಮೇಲೆ ಹೊರೆ ಹೊರೆ ಆಪಾದನೆಗಳನ್ನು ಹೊರೆಸಿದ್ದಾರೆ. ಆದರೂ ಸ್ಪಷ್ಟವಾಗಿ ಯಾವ ಹಿತ ಸಾಧನೆಯೂ ಆಗಿಲ್ಲ. ಇದಕ್ಕೆ ಕಾರಣವೇನು? ಇದನ್ನು ಹುಡುಕುವುದು ಅಷ್ಟು ಕಷ್ಟವಲ್ಲ – ಇದು ಅವರು ಮಾಡಿದ ನಿಂದನೆಗಳೇ. ನಾನು ಮೊದಲಿನಿಂದಲೂ ಹೇಳುತ್ತಿರುವಂತೆ, ಒಂದು ರಾಷ್ಟ್ರವಾಗಿ, ಐತಿಹಾಸಿಕವಾಗಿ ನಾವು ಪಡೆದಿರುವ ಜನಾಂಗೀಯ ಲಕ್ಷಣಗಳನ್ನು ಕಾಪಾಡಿಕೊಳ್ಳಲು ಪ್ರಯತ್ನಿಸಬೇಕು. ನಾವು ಅನ್ಯರಿಂದ ಎಷ್ಟೋ ವಿಷಯಗಳನ್ನು ಕಲಿತುಕೊಳ್ಳಬೇಕಾಗಿದೆ ಎಂಬುದನ್ನು ಒಪ್ಪಿಕೊಳ್ಳುತ್ತೇನೆ. ಆದರೆ ಮುಕ್ಕಾಲುಪಾಲು ನಮ್ಮ ಆಧುನಿಕ ಸುಧಾರಕರು ವಿವೇಚನೆಯಿಲ್ಲದೆ ಪಾಶ್ಚಾತ್ಯ ಮಾರ್ಗವನ್ನು ಅನು ಸರಿಸುತ್ತಿರುವುದನ್ನು ನೋಡಿದರೆ ದುಃಖವಾಗುತ್ತದೆ. ಇದು ಭರತಖಂಡಕ್ಕೆ ಹಿತಕಾರಿಯಲ್ಲ. ಆದಕಾರಣವೇ ನಮ್ಮ ಇತ್ತೀಚಿನ ಸುಧಾರಣಾ ಕಾರ್ಯಗಳು ವಿಫಲವಾಗಿರುವುದು. 

ಎರಡನೆಯದಾಗಿ, ಕೇವಲ ಅಲ್ಲಗಳೆಯುವುದಲ್ಲ, ಒಳ್ಳೆಯದಕ್ಕೆ ದಾರಿ. ನಮ್ಮ ಸಮಾಜದಲ್ಲಿ ಹಲವು ಕುಂದುಕೊರತೆಗಳಿವೆ ಎಂಬುದು ಮಗುವಿಗೂ ಗೊತ್ತು. ಅವು ಯಾವ ಸಮಾಜದಲ್ಲಿ ಇಲ್ಲ? ನನ್ನ ದೇಶಬಾಂಧವರೇ, ನಾನು ಇದನ್ನು ಹೇಳುತ್ತೇನೆ, ಗಮನಿಸಿ. ನಮ್ಮ ದೇಶವನ್ನು ಇತರ ದೇಶಗಳೊಂದಿಗೆ ಮತ್ತು ಜನಾಂಗಗಳೊಂದಿಗೆ ಹೋಲಿಸಿದರೆ ನಮ್ಮ ಜನರು ಹೆಚ್ಚು ಧಾರ್ಮಿಕರು, ದೈವಭಕ್ತರು. ನಮ್ಮ ಸಮಾಜ ವ್ಯವಸ್ಥೆ ಮಾನವಕೋಟಿಯ ಕಲ್ಯಾಣಕ್ಕೆ ಅನುಕೂಲಕರವಾಗಿದೆ. ನಮಗೆ ಇಂದು ಯಾವ ಸುಧಾರಣೆಯೂ ಬೇಕಾಗಿಲ್ಲ. ನಮ್ಮ ದೇಶದ ಸಂಸ್ಕೃತಿಗೆ ತಕ್ಕಂತೆ ನಾವು ಬೆಳೆಯಬೇಕು, ಅಭಿವೃದ್ಧಿಯಾಗಬೇಕು. ಇದೇ ನನ್ನ ಆದರ್ಶ. ನಮ್ಮ ದೇಶದ ಇತಿಹಾಸವನ್ನು ನೋಡಿದರೆ ಜಗತ್ತಿನ ಮತ್ತಾವ ದೇಶವೂ ಜನರ ಮನಸ್ಸಿನ ಬೆಳವಣಿಗೆಗಾಗಿ ಇಷ್ಟೊಂದು ಪ್ರಯತ್ನಪಟ್ಟಿಲ್ಲ ಎಂಬುದು ಗೊತ್ತಾಗುವುದು. ಆದಕಾರಣ ನಾನು ನನ್ನ ದೇಶವನ್ನು ನಿಂದಿಸುವುದಿಲ್ಲ. “ನೀವು ಚೆನ್ನಾಗಿ ಮಾಡಿರುವಿರಿ, ಇನ್ನೂ ಉತ್ತಮರಾಗಿ” ಎನ್ನುವೆನು. ಈ ದೇಶದಲ್ಲಿ ಹಿಂದೆ ಅದ್ಭುತ ಕಾರ್ಯಗಳು ನೆರವೇರಿವೆ. ಅವನ್ನು ಮೀರಿದ ಕಾರ್ಯಸಾಧನೆಗೆ ಅವಕಾಶವಿದೆ, ಕಾಲವಿದೆ. ನಾವು ಒಂದೇ ಕಡೆ ನಿಲ್ಲಲಾರೆವು ಎಂಬುದು ನಿಮಗೆ ಗೊತ್ತಿರಬಹುದೆಂದು ಊಹಿಸುತ್ತೇನೆ. ನಾವು ನಿಂತ ಕಡೆಯೇ ನಿಂತರೆ ಸಾಯಬೇಕಾಗುವುದು. ನಾವು ಮುಂದೆ ಹೋಗಬೇಕು. ಇಲ್ಲವೇ ಹಿಂದೆ ಸರಿಯಬೇಕು. ನಾವು ಮುಂದುವರಿಯಬೇಕು, ಇಲ್ಲವೇ ಹಿಂದೆ ಬೀಳಬೇಕು. ನಮ್ಮ ಪೂರ್ವಿಕರು ಹಿಂದೆ ಮಹತ್ಕಾರ್ಯಗಳನ್ನು ಸಾಧಿಸಿದರು. ನಾವು ಇನ್ನೂ ಪೂರ್ಣವಾದ ವಿಕಾಸವನ್ನು ಹೊಂದಬೇಕಾಗಿದೆ. ಅವರು ಸಾಧಿಸಿದುದಕ್ಕಿಂತಲೂ ಹೆಚ್ಚಿನದನ್ನು ಸಾಧಿಸಬೇಕಾಗಿದೆ. ನಾವು ಹೇಗೆ ತಾನೆ ಹಿಂದೆ ಸರಿದು ಅವನತಿಗೆ ಬೀಳೋಣ? ಅದು ಸಾಧ್ಯವಿಲ್ಲ. ಅದು ಆಗಕೂಡದು. ಹಿಂದೆ ಸರಿದರೆ ಜನಾಂಗದ ಅವನತಿ ಮತ್ತು ಮರಣ ಸನ್ನಿಹಿತವಾಗುವುದು. ಮುಂದೆ ಮುಂದೆ ಹೋಗಿ ಮತ್ತೂ ಮಹತ್ಕಾರ್ಯಗಳನ್ನು ಸಾಧಿಸೋಣ ಎಂಬುದನ್ನೇ ನಾನು ನಿಮಗೆ ಹೇಳಬೇಕಾಗಿದೆ. 

ನಾನು ಯಾವ ಕ್ಷಣಿಕ ಸಮಾಜಸುಧಾರಣೆಯನ್ನೂ ಬೋಧಿಸುವವನಲ್ಲ. ನಾನು ದೋಷಗಳನ್ನು ನಿವಾರಿಸಲು ಕೈಹಾಕುವುದಿಲ್ಲ. ನಮ್ಮ ಪೂರ್ವಿಕರು ಮಾನವ ಕಲ್ಯಾಣಕ್ಕೆ ತೋರಿದ ಅತಿಶ್ರೇಷ್ಠ ಮಾರ್ಗದಲ್ಲಿ ಸಾಗಿ, ಆದರ್ಶವನ್ನು ಅನುಷ್ಠಾನದಲ್ಲಿ ವ್ಯಕ್ತಪಡಿಸಿ ಎನ್ನುವೆನು. ಮಾನವನ ಏಕತ್ವ, ಮತ್ತು ಅವನು ಸ್ವಭಾವತಃ ಪರಿಶುದ್ಧನೆಂಬ ವೇದಾಂತ ಸಿದ್ಧಾಂತಗಳನ್ನು ಹೆಚ್ಚು ಹೆಚ್ಚು ಅನುಷ್ಠಾನಕ್ಕೆ ತನ್ನಿ ಎನ್ನುವೆನು. ನನಗೆ ಇಂದು ಸಮಯವಿದ್ದಿದ್ದರೆ, ನಾವು ಇಂದು ಮಾಡಬೇಕಾದುದನ್ನೆಲ್ಲಾ ನಮ್ಮ ಹಿಂದಿನ ಸ್ಮೃತಿಕಾರರು ಬಹಳ ಹಿಂದೆಯೇ ಹೇಳಿದ್ದರು, ಸಮಾಜದಲ್ಲಿ ಆಗಿರುವ ಮತ್ತು ಆಗುವ ಬದಲಾವಣೆಗಳನ್ನೆಲ್ಲಾ ಅವರು ನಿರೀಕ್ಷಿಸಿದ್ದರು ಎಂಬುದನ್ನು ತೋರಿಸುತ್ತಿದ್ದೆ. ಅವರೂ ಜಾತಿಪದ್ಧತಿಯನ್ನು ವಿರೋಧಿಸಿದರು. ಆದರೆ ಅವರು ಇಂದಿನ ಆಧುನಿಕರಂತೆ ಅಲ್ಲ. ಜಾತಿಪದ್ಧತಿಯನ್ನು ರದ್ದು ಮಾಡುವುದೆಂದರೆ, ಊರಿನ ಜನರೆಲ್ಲಾ ಒಟ್ಟಿಗೆ ಕುಳಿತು ದನದ ಮಾಂಸವನ್ನು ತಿಂದು ಶರಾಬು ಕುಡಿಯಬೇಕೆಂದು ಅಲ್ಲ. ದೇಶದ ಮೂರ್ಖರು, ಮುಠ್ಠಾಳರು, ಹುಚ್ಚರೆಲ್ಲಾ ಎಲ್ಲಿ, ಯಾವಾಗ ಯಾರನ್ನು ಬೇಕಾದರೂ ಮದುವೆ ಮಾಡಿಕೊಂಡು ಇಡಿಯ ದೇಶವನ್ನೇ ಒಂದು ಹುಚ್ಚರ ಆಸ್ಪತ್ರೆಯನ್ನಾಗಿ ಮಾಡಬೇಕೆಂದು ಅಲ್ಲ. ಅಥವಾ ಒಂದು ದೇಶದ ವಿಧವೆಯರಿಗೆ ಸಿಕ್ಕುವ ಗಂಡಂದಿರ ಸಂಖ್ಯೆಯ ಮೇಲೆ ಒಂದು ದೇಶದ ಅಭ್ಯುದಯ ನಿಂತಿದೆ ಎಂದೂ ಅವರು ತಿಳಿಯಲಿಲ್ಲ. ಇಂಥವನ್ನೆಲ್ಲ ಮಾಡಿ ಅಭ್ಯುದಯವನ್ನು ಸಾಧಿಸಿದ ಒಂದು ದೇಶವನ್ನು ನಾನಿನ್ನೂ ನೋಡಬೇಕಾಗಿದೆ. 

ನಮ್ಮ ಪೂರ್ವಿಕರ ಆದರ್ಶ ವ್ಯಕ್ತಿ ಬ್ರಾಹ್ಮಣ. ನಮ್ಮ ಶಾಸ್ತ್ರಗಳಲ್ಲೆಲ್ಲಾ ಈ ಆದರ್ಶವೇ ಸ್ಪಷ್ಟವಾಗಿ ಚಿತ್ರಿತವಾಗಿದೆ. ಯೂರೋಪಿನಲ್ಲಿ ಪಾದ್ರಿಗಳು ಇರುವರು. ತಮ್ಮ ಪೂರ್ವಿಕರು ಯಾವ ಗೌರವಸ್ಥ ವಂಶಕ್ಕೆ ಸೇರಿದವರೆಂಬುದನ್ನು ಸ್ಥಾಪಿಸುವುದಕ್ಕೆ ಅವರು ಸಹಸ್ರಾರು ಪೌಂಡುಗಳನ್ನು ವೆಚ್ಚಮಾಡಿ ಹೆಣಗಾಡುತ್ತಿರುವರು. ಆ ಪೂರ್ವಿಕರೋ ಯಾವುದೋ ಗುಡ್ಡದ ಮೇಲೆ ಇದ್ದ ದುಷ್ಟ ಪಾಳೆಯಗಾರರು. ಅವರು ಜನರು ಹೋಗುವುದನ್ನು ಕಾಯುತ್ತಾ, ಅವಕಾಶ ಸಿಕ್ಕಿದಾಗ ಅವರನ್ನು ಸುಲಿಯುತ್ತಿದ್ದರು, ಕೊಳ್ಳೆಹೊಡೆಯುತ್ತಿದ್ದರು. ಅಂತಹವರ ವಂಶಕ್ಕೆ ತಾವು ಸೇರಿದವರೆಂಬುದನ್ನು ಪ್ರಮಾಣಸಹಿತ ಸ್ಥಾಪಿಸುವವರೆಗೆ ಆ ಪಾದ್ರಿಗಳಿಗೆ ತೃಪ್ತಿಯೇ ಇಲ್ಲ. ಅವರಿಗೆ ಗೌರವಸ್ಥಾನವನ್ನು ದಾನಮಾಡುವ ಪೂರ್ವಿಕರ ಕೆಲಸ ಅಂಥದು. ಆದರೆ ಭರತಖಂಡದಲ್ಲಿ ಪ್ರಖ್ಯಾತ ಚಕ್ರವರ್ತಿ ಕೂಡ ತನ್ನ ಪೂರ್ವಿಕರು ಕಾಡಿನಲ್ಲಿ ವಾಸಮಾಡುತ್ತಾ, ಗೆಡ್ಡೆಗೆಣಸು ತಿಂದು, ನಾರುಬಟ್ಟೆಯನ್ನುಟ್ಟು ವೇದಾಧ್ಯಯನ ಮಾಡುತ್ತಿದ್ದ ಋಷಿಗಳು ಎಂದು ತೋರುವುದಕ್ಕೆ ಪ್ರಯತ್ನಪಡುವನು. ಭರತಖಂಡದ ರಾಜ ತನ್ನ ಪೂರ್ವಿಕರ ಮೂಲವನ್ನು ಕಂಡುಹಿಡಿಯುವುದು ಹೀಗೆ. ನೀವೊಬ್ಬ ಋಷಿಯ ಕುಲಕ್ಕೆ ಸೇರಿದವರೆಂದರೆ ಮಾತ್ರ ಕುಲೀನ ವಂಶಸ್ಥರು. ಇಲ್ಲದೆ ಇದ್ದರೆ ಇಲ್ಲ. 

ನಮ್ಮ ದೇಶದ ಕುಲೀನ ಜಾತಿಯ ಆದರ್ಶ ಇತರ ದೇಶಗಳಂತೆ ಅಲ್ಲ. ತ್ಯಾಗಿ\-ಯಾದ ಮತ್ತು ಅಧ್ಯಾತ್ಮಸಂಪನ್ನನಾದ ಬ್ರಾಹ್ಮಣ ನಮ್ಮ ಆದರ್ಶ. ಬ್ರಾಹ್ಮಣ ಆದರ್ಶವೆಂದರೆ ಏನು ಅರ್ಥ? ಆದರ್ಶ ಬ್ರಾಹ್ಮಣನಲ್ಲಿ ಸಾಂಸಾರಿಕತೆ ಸಂಪೂರ್ಣ ಇಲ್ಲವಾಗಿ, ಜ್ಞಾನವು ತುಂಬಿ ತುಳುಕಾಡುತ್ತಿರುವುದು. ಇದೇ ಹಿಂದೂ ಜನಾಂಗದ ಆದರ್ಶ. ಬ್ರಾಹ್ಮಣನು ಯಾವ ನಿಯಮಕ್ಕೂ ಒಳಪಟ್ಟಿಲ್ಲ. ಅವನಿಗೆ ಯಾವ ನಿಯಮವೂ ಇಲ್ಲ, ಅವನು ಯಾವ ರಾಜರ ಅಧೀನದಲ್ಲೂ ಇಲ್ಲ. ಅವನಿಗೆ ತೊಂದರೆ ಕೊಡಕೂಡದು ಎಂಬುದನ್ನು ನಮ್ಮ ಶಾಸ್ತ್ರ ಸಾರುವುದು. ಇದು ಅಕ್ಷರಶಃ ಸತ್ಯ. ಏನೂ ತಿಳಿಯದ, ದುರುದ್ದೇಶದಿಂದ ಪ್ರೇರಿತರಾದ ಜನರ ಅಭಿಪ್ರಾಯವನ್ನು ಈ ವಿಷಯದಲ್ಲಿ ಗಮನಕ್ಕೆ ತೆಗೆದುಕೊಳ್ಳಬೇಡಿ. ಅದನ್ನು ಮೂಲ ವೇದಾಂತದ ಬೆಳಕಿನಲ್ಲಿ ನಿಶ್ಚಯಿಸಿ. ತಮ್ಮ ಸ್ವಾರ್ಥವನ್ನೆಲ್ಲಾ ನಾಶಮಾಡಿ, ಜ್ಞಾನವನ್ನೂ ಪ್ರೇಮ ಶಕ್ತಿಯನ್ನೂ ಪಡೆಯುವುದಕ್ಕೆ, ಅವನ್ನು ಪ್ರಚಾರಮಾಡುವುದಕ್ಕೆ ಯಾರು ತಮ್ಮ ಜೀವನವನ್ನು ಮುಡುಪಾಗಿಡುತ್ತಾರೋ ಅವರೇ ಬ್ರಾಹ್ಮಣರು. ಇಂತಹ ಆಧ್ಯಾತ್ಮಿಕ ಮತ್ತು ನೈತಿಕ ಶಕ್ತಿಯ ಸ್ತ್ರೀ ಪುರುಷರಿಂದ ತುಂಬಿರುವ ದೇಶವು ಎಲ್ಲ ನಿಯಮಗಳನ್ನು ಮೀರಿ ಬೆಳೆದಿದ್ದರೆ ಅದರಲ್ಲಿ ಆಶ್ಚರ್ಯವೇನಿದೆ? ಅಂಥ ಪ್ರಜೆಗಳನ್ನು ಆಳುವುದಕ್ಕೆ ಪೋಲೀಸ್​ ಏತಕ್ಕೆ? ಸಿಪಾಯಿ ಪಡೆ ಏತಕ್ಕೆ? ಅವರನ್ನು ಏತಕ್ಕೆ ಯಾರಾದರೂ ಆಳ ಬೇಕು? ಅವರು ಒಂದು ಸರ್ಕಾರದ ಅಧೀನದಲ್ಲಿ ತಾನೆ ಏತಕ್ಕೆ ಇರಬೇಕು? ಅವರು ಸತ್ಪುರುಷರು, ಮಹಾವ್ಯಕ್ತಿಗಳು, ಭಗವಂತನ ಮಕ್ಕಳು, ಅವರೇ ನಮ್ಮ ಆದರ್ಶ ಬ್ರಾಹ್ಮಣರು. ಸತ್ಯಯುಗದಲ್ಲಿ ಒಂದೇ ಜಾತಿ ಇತ್ತು, ಅದೇ ಬ್ರಾಹ್ಮಣರದು ಎಂದು ಓದುತ್ತೇವೆ. ಮಹಾಭಾರತದಲ್ಲಿ ಆದಿಯಲ್ಲಿ ಬ್ರಾಹ್ಮಣರೊಬ್ಬರೇ ಇದ್ದರು, ಅವರು ಅಧೋಗತಿಗೆ ಬರುತ್ತಾ ಹಲವು ವರ್ಣಗಳಾದರೆಂದೂ ಪುನಃ ಕೊನೆಯಲ್ಲಿ ಅವರೆಲ್ಲಾ ಬ್ರಾಹ್ಮಣರಾಗುವರೆಂದೂ ಹೇಳಿದೆ. ಈಗ ಅದು ಹಾಗೆಯೇ ಆಗುತ್ತಿದೆ, ಅದನ್ನು ಗಮನಿಸಿ. ವರ್ಣಸಮಸ್ಯೆಗೆ ಪರಿಹಾರ ಆಗಲೇ ಮೇಲೆ ಇರುವವರನ್ನು ಕೆಳಗೆ ಎಳೆಯುವುದಲ್ಲ, ಕೈಗೆ ಸಿಕ್ಕಿದ್ದುದನ್ನು ತಿಂದು ಕುಡಿಯುವುದಲ್ಲ, ಮಿತಿಮೀರಿ ಸುಖ ಭೋಗವನ್ನು ಅರಸುವುದಲ್ಲ. ವೇದಾಂತ ತತ್ತ್ವದ ಸಿದ್ಧಾಂತಗಳನ್ನು ಪರಿಪಾಲಿಸು ವುದರಿಂದ ಮಾತ್ರ ಅದು ಸಾಧ್ಯ. ಆದರ್ಶ ಬ್ರಾಹ್ಮಣರಾಗಿ ಅಧ್ಯಾತ್ಮ ಸಂಪನ್ನರಾದರೆ ಮಾತ್ರ ಸಾಧ್ಯ. ನಮ್ಮ ಪೂರ್ವಿಕರು ಈ ಮಾತೃಭೂಮಿಯಲ್ಲಿರುವ ಎಲ್ಲರಿಗೂ, ಅವರು ಆರ್ಯರಾಗಲೀ ಅನಾರ್ಯರಾಗಲೀ, ಋಷಿಗಳಾಗಲೀ, ಬ್ರಾಹ್ಮಣರಾಗಲೀ, ಅಥವಾ ನೀಚ ಚಂಡಾಲರಾಗಲೀ, ಪ್ರತಿಯೊಬ್ಬರೂ ಒಂದು ನಿಯಮವನ್ನು ಪರಿ ಪಾಲಿಸಬೇಕೆಂದು ಹೇಳಿದರು. ಅವರ ಆಜ್ಞೆ ಎಲ್ಲರಿಗೂ ಅನ್ವಯಿಸುವುದು. ಅದು ಯಾವುವೆಂದರೆ ನೀವು ನಿಲ್ಲದೆ ಯಾವಾಗಲೂ ಮುಂದುವರಿಯುತ್ತಿರಬೇಕು; ಪರಮೋತ್ಕೃಷ್ಟನಿಂದ ಹಿಡಿದು ಅತಿ ನೀಚ ಚಂಡಾಲನವರೆಗೆ ಪ್ರತಿಯೊಬ್ಬರೂ ಆದರ್ಶ ಬ್ರಾಹ್ಮಣರಾಗಲು ಯತ್ನಿಸಬೇಕು, ಎಂಬುದು. ಈ ವೇದಾಂತ ನಿಯಮವನ್ನು ಇಲ್ಲಿ ಮಾತ್ರವಲ್ಲ, ಜಗತ್ತಿನಲ್ಲೆಲ್ಲಾ ಅನುಷ್ಠಾನಕ್ಕೆ ತರಲು ಸಾಧ್ಯ. ಅವರ ಉದ್ದೇಶ ಇದು: ನಿಧಾನವಾಗಿ ಎಲ್ಲ ಮಾನವ ಸಮಾಜ ಆದರ್ಶ ಧಾರ್ಮಿಕವಾಗಬೇಕು; ಧೃತಿ, ಕ್ಷಮೆ, ಶೌಚ, ಶಾಂತಿ, ಮತ್ತು ಧ್ಯಾನ ಇವುಗಳಲ್ಲಿ ನಿರತರಾಗಬೇಕು. ಈ ಸ್ಥಿತಿಯಲ್ಲಿ ಭಗವಂತನ ಸಾಯುಜ್ಯಪದವಿ ಇರುವುದು. 

ಇದನ್ನು ಹೇಗೆ ಸಾಧಿಸಬೇಕು? ಶಾಪ ಮತ್ತು ನಿಂದೆ, ಇವುಗಳಿಂದ ನಾವು ಏನನ್ನೂ ಸಾಧಿಸಲಾರೆವು. ಇದನ್ನು ನೀವು ಲಕ್ಷ್ಯದಲ್ಲಿಡಿ. ಹಲವು ವರ್ಷಗಳಿಂದ ಹೀಗೆ ಮಾಡಿರುವರು. ಅದರಿಂದ ಯಾವ ಪ್ರಯೋಜನವೂ ಆಗಲಿಲ್ಲ. ಸಹಾನು ಭೂತಿಯಿಂದ ಮತ್ತು ಪ್ರೀತಿಯಿಂದ ಮಾತ್ರ ಒಳ್ಳೆಯ ಪರಿಣಾಮ ದೊರಕುವುದು. ಇದು ದೊಡ್ಡ ವಿಷಯ. ನನ್ನ ಮನಸ್ಸಿನಲ್ಲಿರುವ ಹಂಚಿಕೆ, ಇದಕ್ಕೆ ಸಂಬಂಧಪಟ್ಟಂತೆ ನನ್ನ ಮನಸ್ಸಿನಲ್ಲಿ ಪ್ರತಿದಿನವೂ ಬರುತ್ತಿರುವ ಭಾವನೆಗಳು ಇವನ್ನು ವಿವರಿಸಬೇಕಾದರೆ ಹಲವು ಉಪನ್ಯಾಸಗಳನ್ನು ಕೊಡಬೇಕಾಗುತ್ತದೆ. ನನ್ನ ಭಾತೃಗಳೇ, ನಮ್ಮ ಮಾತೃಭೂಮಿ ಎಂಬ ಹಡಗು ಸಹಸ್ರಾರು ವರ್ಷಗಳಿಂದಲೂ ಸಂಚರಿಸುತ್ತಿದೆ, ಇಂದು ಅದು ಎಲ್ಲೋ ಸ್ವಲ್ಪ ಸೋರುತ್ತಿರಬಹುದು; ಅದರ ಯಾವುದೋ ಭಾಗ ಸ್ವಲ್ಪ ಸವೆದು ಹೋಗಿರಬಹುದು. ಇಂತಹ ಪರಿಸ್ಥಿತಿಯಲ್ಲಿ ಸಾಧ್ಯವಾದ ಮಟ್ಟಿಗೆ ಆ ರಂಧ್ರವನ್ನು ಮುಚ್ಚುವುದು ನಿಮ್ಮ ಮತ್ತು ನನ್ನ ಪವಿತ್ರ ಕರ್ತವ್ಯ. ಈ ಅಪಾಯವನ್ನು ನಮ್ಮ ದೇಶದ ಜನರ ಗಮನಕ್ಕೆ ತರೋಣ; ಅವರು ಜಾಗೃತರಾಗಿ ನಮ್ಮ ಸಹಾಯಕ್ಕೆ ಬರಲಿ. ಜನರಿಗೆ ಅಪಾಯವನ್ನು ಮನಗಾಣಿಸಲು, ಕರ್ತವ್ಯವನ್ನು ಜ್ಞಾಪಿಸಲು, ದೇಶದ ಒಂದು ಮೂಲೆಯಿಂದ ಮತ್ತೊಂದು ಮೂಲೆಗೆ, ತಾರಸ್ವರ ದಲ್ಲಿ ಕೂಗುತ್ತಾ ಹೋಗುವೆನು. ಅವರು ನನ್ನನ್ನು ಕೇಳಲಿಲ್ಲ ಎಂದು ಭಾವಿಸೋಣ. ಆದರೂ ಅವರನ್ನು ದೂರುವುದಿಲ್ಲ, ಶಪಿಸುವುದಿಲ್ಲ. ಪೂರ್ವದಲ್ಲಿ ನಮ್ಮ ದೇಶ ಮಹತ್ಕಾರ್ಯಗಳನ್ನು ಸಾಧಿಸಿತ್ತು. ನಾವು ಅದಕ್ಕೂ ಮಿಗಿಲಾದ ಮಹತ್ಕಾರ್ಯಗಳನ್ನು ಮುಂದೆ ಮಾಡದೇ ಇದ್ದರೆ ನಾವೆಲ್ಲಾ ನೀರಿನಲ್ಲಿ ಒಟ್ಟಿಗೆ ಮುಳುಗಿ ಶಾಂತಿಯಿಂದ ಸಾಯಬಲ್ಲೆವೆಂಬ ಸಮಾಧಾನವಾದರೂ ಇರಲಿ. ದೇಶಭಕ್ತರಾಗಿ; ನಮಗಾಗಿ ಹಿಂದೆ ಇಷ್ಟೊಂದು ಉಪಕಾರವನ್ನು ಮಾಡಿದ ಜನಾಂಗವನ್ನು ಪ್ರೀತಿಸಿ. ನನ್ನ ದೇಶಬಾಂಧವರೆ, ನಾನು ಇತರ ದೇಶದವರೊಂದಿಗೆ ನಿಮ್ಮನ್ನು ಹೆಚ್ಚು ಹೋಲಿಸಿದಷ್ಟೂ ನಿಮ್ಮನ್ನು ಹೆಚ್ಚು ಪ್ರೀತಿಸುತ್ತೇನೆ. ನೀವು ಒಳ್ಳೆಯವರು, ಶುದ್ಧರು, ಸಾಧುಸ್ವಭಾವದವರು. ಈ ಮಾಯಾ ಪ್ರಪಂಚದ ಒಂದು ವಿಪರ್ಯಾಸವೆಂದರೆ ನಾವು ಸದಾ ತುಳಿತಕ್ಕೆ ಒಳಗಾಗಿರುವುದು. ಅದನ್ನು ಗಮನಿಸಬೇಕಾಗಿಲ್ಲ. ಕೊನೆಗೆ ಆತ್ಮವು ಜಯಿಸಲೇಬೇಕು. ಅಲ್ಲಿಯವರೆವಿಗೂ ಕಾರ್ಯೋನ್ಮುಖರಾಗೋಣ. ನಮ್ಮ ದೇಶವನ್ನು ನಿಂದಿಸದೇ ಇರೋಣ. ಹಲವು ಶತಮಾನಗಳ ಕಾಲ ಪೆಟ್ಟು ತಿಂದು ಜರ್ಝರಿತವಾದ ಜನಾಂಗದ ಮೇಲೆ ನಿಂದೆಯ ಮಳೆಯನ್ನು ಕರೆಯಬೇಡಿ. ನಮ್ಮ ಸಮಾಜದಲ್ಲಿರುವ, ಅತಿ ಮೂಢವಾಗಿರುವ, ವಿಚಾರಹೀನವಾಗಿರುವ ಯಾವ ಆಚಾರ ವ್ಯವಹಾರಗಳನ್ನೂ ದೂರಬೇಡಿ. ಅವು ಕೂಡ ಹಿಂದೆ ಉಪಕಾರ ಮಾಡಿದ್ದುವು. ಮತ್ತಾವ ದೇಶದಲ್ಲಿ ಇರುವ ವ್ಯವಸ್ಥೆಯ ಆದರ್ಶವೂ ಇಲ್ಲಿಗಿಂತ ಉತ್ತಮವಾಗಿಲ್ಲವೆಂಬುದನ್ನು ಗಮನದಲ್ಲಿಡಿ. ನಾನು ಎಲ್ಲಾ ದೇಶದಲ್ಲಿಯೂ ಜಾತಿಗಳನ್ನು ನೋಡಿರುವೆನು. ಆದರೆ ಎಲ್ಲಿಯೂ ಇಲ್ಲಿರುವಂತಹ ಭವ್ಯವಾದ ಯೋಜನೆಯಾಗಲಿ, ಆದರ್ಶವಾಗಲಿ ಇಲ್ಲ. ಜಾತಿ ಆವಶ್ಯಕವಾದರೆ ಪಾವಿತ್ರ್ಯ. ಸಂಸ್ಕೃತಿ, ತ್ಯಾಗ ಇವುಗಳ ಮೇಲೆ ನಿಂತ ಜಾತಿ ವ್ಯವಸ್ಥೆಯು ಕೇವಲ ಹಣದ ಮೇಲೆ ನಿಂತ ಜಾತಿ ವ್ಯವಸ್ಥೆಗಿಂತ ಮೇಲು. ಯಾರನ್ನೂ ದೂರಬೇಡಿ, ನಿಮ್ಮ ಬಾಯಿ ಸುಮ್ಮನಿರಲಿ, ಹೃದಯ ತೆರೆದಿರಲಿ. ನಿಮ್ಮ ದೇಶದ ಮತ್ತು ಇಡಿಯ ಜಗತ್ತಿನ ಕಲ್ಯಾಣ ನಿಮ್ಮ ಮೇಲೆ ನಿಂತಿದೆ ಎಂದು ಭಾವಿಸಿ ಕಾರ್ಯೋನ್ಮುಖರಾಗಿ. ವೇದಾಂತದ ಸಂದೇಶವನ್ನು ಪ್ರತಿ ಮನೆಗೂ ಸಾರಿ. ಪ್ರತಿಯೊಂದು ಆತ್ಮದಲ್ಲಿರುವ ಸುಪ್ತ ದಿವ್ಯತೆಯನ್ನು ಜಾಗೃತಗೊಳಿಸಿ. ಅನಂತರ ನೀವು ಎಷ್ಟೆ ಅಲ್ಪವನ್ನು ಸಾಧಿಸಿದ್ದರೂ ನೀವು ಒಂದು ಮಹಾ ಆದರ್ಶಕ್ಕಾಗಿ ಬಾಳಿ ದುಡಿದು ಮಡಿದಿರಿ ಎಂಬ ತೃಪ್ತಿಯಾದರೂ ಇರುತ್ತದೆ. ಈ ಮಹತ್ಕಾರ್ಯವು ಯಾವ ರೀತಿಯಿಂದಲಾದರೂ ಸಿದ್ಧಿಸಿದರೂ ಸರಿಯೇ; ಅದರ ಮೇಲೆ ಮಾನವ ಕೋಟಿಯ ಇಂದಿನ ಮತ್ತು ಮುಂದಿನ ಕಲ್ಯಾಣ ನಿಂತಿದೆ. 

