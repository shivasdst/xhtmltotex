
\chapter{ಮದ್ರಾಸಿನ ಬಿನ್ನವತ್ತಳೆಗೆ ಉತ್ತರ}

ಸ್ವಾಮಿ ವಿವೇಕಾನಂದರು ಮದ್ರಾಸಿಗೆ ಆಗಮಿಸಿದಾಗ ಅಲ್ಲಿನ ಸ್ವಾಗತ ಸಮಿತಿಯು ಅವರಿಗೆ ಈ ಕೆಳಗಿನ ಬಿನ್ನವತ್ತಳೆಯನ್ನು ಸಮರ್ಪಿಸಿತು.

\textbf{ಪೂಜ್ಯ ಸ್ವಾಮೀಜಿ,}

ಪಶ್ಚಿಮದಲ್ಲಿ ಧಾರ್ಮಿಕ ಕಾರ್ಯವನ್ನು ಮುಗಿಸಿ ತಾವು ಹಿಂದಿರುಗಿರುವ ಸಂದರ್ಭಲ್ಲಿ ಮದರಾಸಿನ ತಮ್ಮ ಸಹಧರ್ಮೀಯರ ಪರವಾಗಿ ತಮಗೆ ಅತ್ಯಂತ ಹಾರ್ದಿಕವಾದ ಸ್ವಾಗತವನ್ನು ಕೋರುತ್ತೇವೆ. ಈ ಬಿನ್ನವತ್ತಳೆಯನ್ನು ತಮಗೆ ಅರ್ಪಿಸುತ್ತಿರುವುದು ಕೇವಲ ಔಪಚಾರಿಕವಾದ ಅಥವಾ ಸಾಂಪ್ರದಾಯಿಕವಾದ ಕಾರ್ಯವಲ್ಲ. ಭಗವಂತನ ಕೃಪೆಯಿಂದ ಭಾರತದ ಉದಾತ್ತ ಧಾರ್ಮಿಕ ಆದರ್ಶಗಳನ್ನು ಪ್ರತಿಪಾದಿಸುವುದರ ಮೂಲಕ, ತಾವು ಸತ್ಯಕ್ಕಾಗಿ ಸಲ್ಲಿಸಿದ ಸೇವೆಯನ್ನು ಗಮನಿಸಿ, ತಮಗೆ ನಮ್ಮ ಹೃದಯದ ಪ್ರೀತಿ ಕೃತಜ್ಞತೆಗಳನ್ನು ಸಲ್ಲಿಸುವುದು ನಮ್ಮ ಉದ್ದೇಶವಾಗಿದೆ.

ಚಿಕಾಗೋದಲ್ಲಿ ಸರ್ವಧರ್ಮ ಸಮ್ಮೇಳನವು ಸಮಾವೇಶಗೊಳ್ಳಲು ವ್ಯವಸ್ಥೆ ಯಾಯಿತಷ್ಟೆ. ಆಗ ನಮ್ಮಲ್ಲಿ ಕೆಲವರು ಹೀಗೆ ಭಾವಿಸಿದೆವು: ಉದಾತ್ತವೂ ಪ್ರಾಚೀನವೂ ಆದ ನಮ್ಮ ಧರ್ಮವನ್ನು ಯಾರಾದರೂ ಸರಿಯಾದ ರೀತಿಯಲ್ಲಿ ಪ್ರತಿನಿಧಿಸಿ ಅಮೆರಿಕಾ ದೇಶದವರಿಗೂ ತನ್ಮೂಲಕ ಪಶ್ಚಿಮ ಜಗತ್ತಿಗೂ ಅದನ್ನು ಸಮರ್ಥವಾದ ರೀತಿಯಲ್ಲಿ ವ್ಯಾಖ್ಯಾನಿಸಬೇಕು. ಆ ಸಂದರ್ಭದಲ್ಲಿ ನಮಗೆ ತಮ್ಮ ದರ್ಶನ, ಪರಿಚಯಗಳ ಲಾಭ ದೊರಕಿತು. ಆವಶ್ಯಕತೆ ಬಂದಾಗ ಸತ್ಯದ ಧ್ವಜವನ್ನು ಎತ್ತಿಹಿಡಿಯುವ ವ್ಯಕ್ತಿ ಬಂದೇ ತೀರುತ್ತಾನೆ ಎಂಬುದು ಜಗತ್ತಿನ ರಾಷ್ಟ್ರಗಳ ಇತಿಹಾಸದಲ್ಲಿ ಪದೇ ಪದೇ ಸಿದ್ಧಾಂತಗೊಂಡಿರುವುದು ಈ ಸಂದರ್ಭದಲ್ಲಿಯೂ ನಮ್ಮ ಅನುಭವಕ್ಕೆ ಬಂದಿತು. ಸರ್ವಧರ್ಮ ಸಮ್ಮೇಳನದಲ್ಲಿ ಹಿಂದೂಧರ್ಮವನ್ನು ಪ್ರತಿನಿಧಿಸಲು ತಾವು ನಿರ್ಣಯಿಸಿದಾಗ, ತಮ್ಮ ಪ್ರತಿಭೆಯ ಪರಿಚಯವನ್ನು ಪಡೆದು ಕೊಂಡಿದ್ದ ನಮ್ಮಲ್ಲಿನ ಅನೇಕರಿಗೆ ಆ ಚಿರಸ್ಮರಣೀಯ ಧಾರ್ಮಿಕ ಸಭೆಯಲ್ಲಿ ತಾವು ಹಿಂದೂಧರ್ಮವನ್ನು ಸಮರ್ಥವಾಗಿ ಪ್ರತಿನಿಧಿಸುವ ವಿಷಯದಲ್ಲಿ ಯಾವ ಸಂದೇಹವೂ ಉಳಿಯಲಿಲ್ಲ. ತಾವು ಹಿಂದೂಧರ್ಮದ ಸಿದ್ಧಾಂತಗಳನ್ನು ಸ್ಪಷ್ಟವಾಗಿ, ಅತ್ಯಂತ ನಿಷ್ಕೃಷ್ಟವಾಗಿ,\break ಅಧಿಕಾರಯುತವಾದ ವಾಣಿಯಲ್ಲಿ, ಆ ಮಹಾಸಭೆಗೆ ನಿವೇದಿಸಿದಿರಿ. ಅಷ್ಟು ಮಾತ್ರವಲ್ಲ, ಭಾರತದ ಆಧ್ಯಾತ್ಮಿಕ ಚಿಲುಮೆಯಲ್ಲಿ ವಿದೇಶದ ಸ್ತ್ರೀಪುರುಷರು ಅಮರವಾದ ಜೀವನವನ್ನು ಪಡೆಯಲು ಚೈತನ್ಯದಾಯಕವಾದ ಪೋಷಣೆಯನ್ನು ಪಡೆಯಬಹುದು ಎಂಬುದನ್ನೂ ತೋರಿಸಿಕೊಟ್ಟಿರಿ. ತನ್ಮೂಲಕ ನಮ್ಮ ಈ ಭೂಮಿಯು ಇದುವರೆಗೆ ಕಂಡಿರುವುದಕ್ಕಿಂತ ಹೆಚ್ಚು ವಿಶಾಲವಾದ ಪೂರ್ಣವಾದ ಪವಿತ್ರವಾದ ವಿಕಾಸವನ್ನು ಮಾನವತೆಯು ಪಡೆಯುವುದು ಸಾಧ್ಯ ಎಂಬುದನ್ನೂ ತೋರಿಸಿಕೊಟ್ಟಿರಿ. ಜಗತ್ತಿನ ಮಹಾನ್​ ಧರ್ಮಗಳ ಪ್ರತಿನಿಧಿಗಳ ಗಮನವನ್ನು ಹಿಂದೂಧರ್ಮವು ಪ್ರತಿಪಾದಿಸಿದ ಧರ್ಮಸಾಮರಸ್ಯ ಮತ್ತು ಭ್ರಾತೃತ್ವದ ಸಿದ್ಧಾಂತದ ಕಡೆಗೆ ತಾವು ಸೆಳೆದುದಕ್ಕಾಗಿ ನಾವು ತಮಗೆ ವಿಶೇಷವಾಗಿ ಕೃತಜ್ಞರಾಗಿದ್ದೇವೆ. ಸತ್ಯ ಮತ್ತು ಧಾರ್ಮಿಕತೆಗಳು ಯಾವುದೇ ನಿರ್ದಿಷ್ಟ ಪ್ರದೇಶದ ಅಥವಾ ಜನಾಂಗದ ಅಥವಾ ಸಿದ್ಧಾಂತದ ಗುತ್ತಿಗೆ ಎಂಬುದನ್ನಾಗಲಿ, ಯಾವುದೇ ಧರ್ಮವಾಗಲಿ ಸಿದ್ಧಾಂತವಾಗಲಿ ಉಳಿದವುಗಳನ್ನು ನಾಶಮಾಡಿ ತಾನು ಮಾತ್ರ ಬದುಕಬಲ್ಲದು ಎಂಬುದನ್ನಾಗಲೀ ಯಾವ ವಿಚಾರಶೀಲ ಪ್ರಾಮಾಣಿಕ ವ್ಯಕ್ತಿಗಳೂ ಒಪ್ಪುವುದು ಸಾಧ್ಯವಾಗುವುದಿಲ್ಲ. ತಮ್ಮದೇ ಮಧುರವಾದ ಭಾಷೆಯಲ್ಲಿ ಭಗವದ್ಗೀತೆಯ ಸಾಮರಸ್ಯದ ಸಾರವನ್ನು ಸಾರುವುದಾದರೆ: “ಜಗತ್ತಿನ ಧರ್ಮಗಳೆಲ್ಲ ಒಂದು ಯಾತ್ರೆ ಮಾತ್ರ, ವಿಭಿನ್ನ ಸ್ಥಿತಿಗತಿಗಳ ಮೂಲಕ ಭಿನ್ನ ಬಿನ್ನ ಸ್ತ್ರೀಪುರುಷರು ಒಂದೇ ಗುರಿಯೆಡೆಗೆ ಕೈಗೊಳ್ಳುವ ಯಾತ್ರೆ ಮಾತ್ರ.”

ತಮಗೆ ಒಪ್ಪಿಸಿದ ಉದಾತ್ತವೂ ಪವಿತ್ರವೂ ಆದ ಕರ್ತವ್ಯವನ್ನು ಮುಗಿಸಿ ಅಷ್ಟರಲ್ಲೇ ತಾವು ತೃಪ್ತಿಪಟ್ಟುಕೊಂಡಿದ್ದರೂ ಹಿಂದೂ ಸಹಧರ್ಮೀಯರಾದ ನಾವು ತಾವು ನೆರವೇರಿಸಿದ ಕಾರ್ಯದ ಅಮೂಲ್ಯ ಸ್ವರೂಪವನ್ನು ಅರಿತು, ಸಂತೋಷಿಸಿ ತಮಗೆ ಕೃತಜ್ಞರಾಗಿರುತ್ತಿದ್ದೆವು. ಆದರೆ ಪಾಶ್ಚಾತ್ಯ ಜಗತ್ತಿನ ಹೃದಯವನ್ನು ಪ್ರವೇಶಿಸುವ ಸಂದರ್ಭದಲ್ಲಿ ತಾವು ಭಾರತದ ಸನಾತನ ಧರ್ಮದ ಬೋಧನೆಗಳನ್ನು ಆಧರಿಸಿದ, ಇಡೀ ಮಾನವತೆಗೆ ಅಗತ್ಯವಾದ ಜ್ಯೋತಿ ಶಾಂತಿಗಳ ಸಂದೇಶವಾಹಕರೂ ಆಗಿದ್ದೀರಿ. ವೇದಾಂತ ಧರ್ಮದ ಅದ್ಭುತ ವೈಚಾರಿಕತೆಯನ್ನು ಜಗತ್ತಿಗೆ ತೋರಿಸಿದುದಕ್ಕಾಗಿ ತಮಗೆ ನಾವು ಕೃತಜ್ಞರಾಗಿದ್ದೇವೆ. ತಾವು ನಮ್ಮ ಧರ್ಮ ಮತ್ತು ದರ್ಶನಗಳನ್ನು ಪ್ರಚಾರ ಮಾಡುವುದಕ್ಕಾಗಿ ಶಾಶ್ವತವಾದ ಕೇಂದ್ರಗಳನ್ನು ಸ್ಥಾಪಿಸಬೇಕೆಂದಿರುವ ಉದ್ದೇಶವನ್ನೂ ತಿಳಿದು ನಮಗೆ ಅತಿ ಹೆಚ್ಚಿನ ಸಂತೋಷವಾಗಿದೆ. ಯಾವ ಮಹೋದ್ದೇಶಕ್ಕಾಗಿ ತಮ್ಮ ಶಕ್ತಿ ಸಮಸ್ತವನ್ನೂ ಸಮರ್ಪಿಸಲು ತಾವು ಸಿದ್ಧರಾಗಿದ್ದೀರೊ ಅದು ತಾವು ಪ್ರತಿನಿಧಿಸುತ್ತಿರುವ ಪವಿತ್ರ ಸಂಪ್ರದಾಯಗಳಿಗೆ ಅನುಗುಣವಾಗಿಯೇ ಇದೆ, ಮಾತ್ರವಲ್ಲ, ತಮ್ಮ ಜೀವನ ಮತ್ತು ಧ್ಯೇಯಗಳಿಗೆ ಸ್ಫೂರ್ತಿ ನೀಡಿದ ಮಹಾ ಗುರುವಿನ ಮನೋಧರ್ಮಕ್ಕೂ ಅನುಗುಣವಾಗಿಯೇ ಇದೆ. ಈ ಮಹಾಕಾರ್ಯದಲ್ಲಿ ಭಾಗಿಗಳಾಗುವ ಅವಕಾಶವು ನಮಗೂ ದೊರಕುತ್ತದೆ ಎಂಬ ನಂಬಿಕೆ ನಮಗಿದೆ. ಸರ್ವಜ್ಞನೂ ದಯಾಮಯನೂ ಆದ ವಿಶ್ವೇಶ್ವರನು ತಮಗೆ ದೀರ್ಘಾಯುಷ್ಯವನ್ನೂ ಸಂಪೂರ್ಣ ಶಕ್ತಿಯನ್ನೂ ನೀಡಲಿ ಎಂದೂ ಅಮರವಾದ ಸತ್ಯಕ್ಕೆ ಕಿರೀಟಪ್ರಾಯವಾದ ಯಶಸ್ಸು ತಮಗೆ ದೊರಕಲಿ ಎಂದೂ ನಾವು ಪ್ರಾರ್ಥಿಸುತ್ತೇವೆ.

ಅನಂತರ ಖೇತ್ರಿಯ ಮಹಾರಾಜರು ಕಳುಹಿಸಿದ್ದ ಬಿನ್ನವತ್ತಳೆಯನ್ನು\break ಓದಲಾಯಿತು.

\textbf{ಪೂಜ್ಯಪಾದರೆ,}

ತಾವು ಕ್ಷೇಮವಾಗಿ ಭಾರತಕ್ಕೆ ಹಿಂದಿರುಗಿದುದಕ್ಕಾಗಿ ನನಗೆ ಮಹದಾನಂದವಾಗಿದೆ. ತಮ್ಮ ಆಗಮನದ ಸಂದರ್ಭದಲ್ಲಿ ಮದರಾಸಿನ ಜನತೆ ನೀಡುತ್ತಿರುವ ಸ್ವಾಗತದ ಸಂದರ್ಭದಲ್ಲೇ ನನ್ನ ಸಂತೋಷವನ್ನು ವ್ಯಕ್ತಪಡಿಸುವುದಕ್ಕೆ ಆತುರನಾಗಿದ್ದೇನೆ. ಪಶ್ಚಿಮ ದೇಶಗಳಲ್ಲಿ ತಮ್ಮ ನಿಃಸ್ವಾರ್ಥ ಪ್ರಯತ್ನಗಳಿಗೆ ದೊರೆತ ಮಹಾನ್​ ಯಶಸ್ಸಿಗಾಗಿ ನನ್ನ ಹೃತ್ಪೂರ್ವಕ ಅಭಿವಂದನೆಗಳನ್ನು ಅರ್ಪಿಸಲು ಬಯಸುತ್ತೇನೆ. ವಾಸ್ತವವಾಗಿ ವಿಜ್ಞಾನವು ನಿಜವಾದ ಧರ್ಮವನ್ನು ವಿರೋಧಿಸುವುದಿಲ್ಲವಾದರೂ, “ವಿಜ್ಞಾನವು ಒಮ್ಮೆ ಜಯಿಸಿ ಆಕ್ರಮಿಸಿಕೊಂಡ ಒಂದು ಅಂಗುಲ ನೆಲವನ್ನೂ ಧರ್ಮವು ಎಂದೂ ಮತ್ತೆ ಜಯಿಸಿ ಪಡೆದುಕೊಂಡಿಲ್ಲ” ಎಂಬ ಹೆಮ್ಮೆ ಪಶ್ಚಿಮದ ಮಹಾ ಮೇಧಾವಿಗಳದ್ದು. ಅಂಥಲ್ಲಿ ತಾವು ಈ ಪ್ರಮಾಣದ ಸಾಧನೆಯನ್ನು ಗಳಿಸಿದ್ದೀರಿ. “ಈ ಪುಣ್ಯಭೂಮಿಯಾದ ಆರ್ಯಾವರ್ತವು ತನ್ನನ್ನು ಚಿಕಾಗೋ ಸರ್ವಧರ್ಮ ಸಮ್ಮೇಳನದಲ್ಲಿ ಪ್ರತಿನಿಧಿಸುವುದಕ್ಕೆ ತಮ್ಮಂತಹ ಋಷಿಗಳನ್ನು ಪಡೆಯುವಷ್ಟು ಪುಣ್ಯಶಾಲಿಯಾಗಿದೆ, ಇಂದಿಗೂ ಭರತಖಂಡವು ಅಪಾರವಾದ ಅಧ್ಯಾತ್ಮದ ಭಂಡಾರವಾಗಿದೆ” ಎಂಬುದನ್ನು ತಾವು ತಮ್ಮ ಬುದ್ಧಿಶಕ್ತಿಯಿಂದಲೂ ಸಾಹಸದಿಂದಲೂ ಉತ್ಸಾಹದಿಂದಲೂ ಪಾಶ್ಚಾತ್ಯ ಜಗತ್ತಿಗೆ ಅರ್ಥಮಾಡಿಸಿದ್ದೀರಿ. ವೇದಾಂತದ ವಿಶ್ವವ್ಯಾಪಿಯಾದ ಬೆಳಕಿನಲ್ಲಿ ಜಗತ್ತಿನ ಅಸಂಖ್ಯಾತ ಪಂಥಗಳ ವಿರೋಧಗಳೆಲ್ಲ ತಮ್ಮ ಶ್ರಮದಿಂದ ಪರಿಹಾರಗೊಂಡಿವೆ ಎಂಬುದರಲ್ಲಿ ಯಾವ ಸಂದೇಹವೂ ಇಲ್ಲ. ವಿಶ್ವದ ವಿಕಾಸ ಪಥದಲ್ಲಿ “ವೈವಿಧ್ಯದಲ್ಲಿ ಏಕತೆ” ಎಂಬುದೇ ಪ್ರಕೃತಿಯ ಯೋಜನೆ ಎಂಬ ಮಹಾಸತ್ಯವನ್ನು ಜಗತ್ತಿನ ಎಲ್ಲ ಜನರೂ ಅರ್ಥಮಾಡಿಕೊಳ್ಳಬೇಕಾದ ಆವಶ್ಯಕತೆಯಿದೆ ಮತ್ತು ವಿವಿಧ ಧರ್ಮಗಳಲ್ಲಿ ಸಾಮರಸ್ಯ, ಭ್ರಾತೃತ್ವ ಮತ್ತು ಸಹಿಷ್ಣುತೆಗಳಿಂದ ಮಾತ್ರವೇ ಮಾನವತೆ ಅಂತಿಮಗುರಿಯ ಕಡೆಗೆ ತನ್ನ ಯಾತ್ರೆಯನ್ನು ಕೈಗೊಳ್ಳುವುದು ಸಾಧ್ಯ ಎಂಬುದನ್ನು ಜನತೆ ಅರ್ಥಮಾಡಿಕೊಳ್ಳಬೇಕಾಗಿದೆ. ತಮ್ಮ ಪವಿತ್ರ ಮಾರ್ಗ ದರ್ಶನದಲ್ಲಿ, ತಮ್ಮ ಉದಾತ್ತ ಬೋಧನೆಗಳಿಂದ ಈ ತಲೆಮಾರಿನವರಾದ ನಮಗೆ ಜಗತ್ತಿನ ಇತಿಹಾಸದಲ್ಲಿ ಒಂದು ಹೊಸಯುಗದ ಶುಭೋದಯವನ್ನು ಕಾಣುವ ಸೌಭಾಗ್ಯವು ಒದಗಿದೆ. ಆ ನವಯುಗದಲ್ಲಿ ಮತಾಂಧತೆ, ದ್ವೇಷ, ಘರ್ಷಣೆಗಳು ಕೊನೆಗೊಂಡು ಶಾಂತಿ, ಸಹಾನುಭೂತಿ, ಪ್ರೀತಿಗಳು ಜನರನ್ನು ಆಳುತ್ತವೆ ಎಂದು ಆಶಿಸಬಹುದಾಗಿದೆ. ತಮ್ಮನ್ನೂ ತಮ್ಮ ಪ್ರಯತ್ನಗಳನ್ನು ಭಗವಂತನು ಆಶೀರ್ವದಿಸಲಿ ಎಂದು ನಾನೂ ನನ್ನ ಪ್ರಜೆಗಳೂ ಪ್ರಾರ್ಥಿಸುತ್ತೇವೆ.

ಬಿನ್ನವತ್ತಳೆಗಳ ವಾಚನವು ಮುಗಿದಮೇಲೆ ಸ್ವಾಮೀಜಿ ಸಭಾಂಗಣದಿಂದ ಹೊರಕ್ಕೆ ಬಂದು ಅಲ್ಲಿದ್ದ ಗಾಡಿಯೊಂದನ್ನು ಹತ್ತಿದರು. ಅಲ್ಲಿ ನೆರೆದಿದ್ದ ಅಪಾರವಾದ ಜನಸ್ತೋಮದ ಉತ್ಸಾಹವು ಉಕ್ಕಿ ಹರಿಯುತ್ತಿದ್ದುದರಿಂದ ಸ್ವಾಮೀಜಿ ತಾವು ನೀಡ ಬೇಕಾಗಿದ್ದ ಉತ್ತರವನ್ನು ಮುಂದಕ್ಕೆ ಹಾಕಿ ಹ್ರಸ್ವವಾದ ಭಾಷಣವನ್ನು ಮಾಡಿದರು:

ಮನುಷ್ಯನ ಇಚ್ಛೆ ಒಂದು, ದೈವೇಚ್ಛೆ ಬೇರೊಂದು. ಆಂಗ್ಲೇಯ ರೀತಿಯಲ್ಲಿ ಬಿನ್ನವತ್ತಳೆಯನ್ನು ಅರ್ಪಿಸುವುದು ಮತ್ತು ಉತ್ತರ ಕೊಡುವುದು ಎಂದು ನಿರ್ಧಾರವಾಗಿತ್ತು. ಆದರೆ ದೈವೇಚ್ಛೆ ಬೇರೆಯಾಗಿತ್ತು. ನಾನು ಗೀತಾ ವಿಧಾನದಲ್ಲಿ ರಥದ ಮೇಲೆ ನಿಂತು, ವಿಶಾಲವಾದ ಜನಸ್ತೋಮವನ್ನು ಉದ್ದೇಶಿಸಿ ಮಾತನಾಡುತ್ತಿರುವೆನು. ಅದು ಹೀಗೆ ಆದುದು ಒಂದು ಸಂತೋಷವೇ ಸರಿ. ಇದು ಭಾಷಣಕ್ಕೆ ಒಂದು ಸ್ಫೂರ್ತಿಯನ್ನು ಕೊಡುವುದು, ನಾನು ಏನೇನು ಹೇಳಬೇಕೆಂದು ಇರುವೆನೊ ಅದಕ್ಕೆ ಒಂದು ಶಕ್ತಿ ಕೊಡುವುದು. ನನ್ನ ಧ್ವನಿ ನಿಮಗೆಲ್ಲಾ ಕೇಳುವುದೊ ಇಲ್ಲವೊ ಗೊತ್ತಿಲ್ಲ. ಆದರೆ ನಾನು ಸಾಧ್ಯವಾದ ಮಟ್ಟಿಗೆ ಪ್ರಯತ್ನಪಡುವೆನು. ನಾನು ಇದುವರೆಗೆ ಇಂತಹ ಭಾರಿ ಬಹಿರಂಗ ಸಭೆಯಲ್ಲಿ ಮಾತನಾಡಿಲ್ಲ.

ಕೊಲಂಬೊ ಇಂದ ಮದ್ರಾಸಿನವರೆಗೆ ಜನರು ನನ್ನನ್ನು ಅದ್ಭುತ ಪ್ರೇಮ, ಉತ್ಸಾಹ, ಸಂತೋಷಗಳಿಂದ ಸ್ವಾಗತಿಸಿರುವರು. ಭರತಖಂಡದಲ್ಲೆಲ್ಲಾ\break ಹೀಗೆಯೇ ಆಗುವ ಸೂಚನೆಗಳು ಕಾಣುತ್ತಿರುವುವು. ಇದನ್ನು ನಾನು ಕನಸಿನಲ್ಲಿಯೂ ನಿರೀಕ್ಷಿಸಿರಲಿಲ್ಲ. ನಾನು ಹಿಂದೆ ಒತ್ತಿ ಹೇಳಿದ ಒಂದು ವಿಷಯವನ್ನು ಇದು ದೃಢಪಡಿಸುವುದು. ಅದೇನೆಂದರೆ ಪ್ರತಿಯೊಂದು ದೇಶಕ್ಕೂ ಒಂದು ಆದರ್ಶ ಕೇಂದ್ರವಿದೆ ಎಂಬುದು, ಪ್ರತಿಯೊಂದು ದೇಶವೂ ಒಂದು ನಿರ್ದಿಷ್ಟ ಮಾರ್ಗದಲ್ಲಿ ಹೋಗಬೇಕಾಗಿದೆ ಎಂಬುದು. ಹಾಗೆಯೇ ಭರತಖಂಡದ ವಿಶೇಷ ಆದರ್ಶ ಧರ್ಮ. ಪ್ರಪಂಚದ ಇತರ ಭಾಗಗಳಲ್ಲಿ ಧರ್ಮ ಎಂಬುದು ಹಲವು ವಿಷಯಗಳಲ್ಲಿ ಒಂದು. ಅದು ನಿಜವಾಗಿ ಗೌಣಸ್ಥಾನದಲ್ಲಿಯೇ ಇದೆ. ಉದಾಹರಣೆಗೆ ಇಂಗ್ಲೆಂಡಿನಲ್ಲಿ ಧರ್ಮವು ರಾಜಕೀಯಕ್ಕೆ ಅಧೀನವಾಗಿದೆ. ಇಂಗ್ಲೆಂಡಿನ ಚರ್ಚು ಆಳರಸರ ಅಧೀನ. ಆದಕಾರಣ ಅದರಲ್ಲಿ ನಂಬಿಕೆಯಿರಲಿ, ಬಿಡಲಿ, ಅದು ತಮ್ಮ ಚರ್ಚು ಎಂದು ಎಲ್ಲರೂ ಅದಕ್ಕೆ ಬೆಂಬಲ ಕೊಡುವರು. ಪ್ರತಿಯೊಬ್ಬ ಮಹಿಳೆ ಮತ್ತು ಮನುಷ್ಯ ಆ ಚರ್ಚಿಗೆ ಸೇರಿರಬೇಕು. ಇದು ಗೌರವದ ಚಿಹ್ನೆ. ಇದರಂತೆಯೇ ಇತರ ದೇಶಗಳಲ್ಲಿ ಕೂಡ ಒಂದು ಮಹತ್ತಾದ ರಾಷ್ಟ್ರೀಯ ಶಕ್ತಿಯಿದೆ. ಅದು ರಾಜಕಾರಣವಾಗಿರಬಹುದು ಅಥವಾ ಬೌದ್ಧಿಕ ಚಟುವಟಿಕೆಯಾಗಿರಬಹುದು. ಅದು ಸೈನ್ಯಶಕ್ತಿಯಾಗಿರಬಹುದು ಅಥವಾ ವಾಣಿಜ್ಯ ಶಕ್ತಿಯಾಗಿರಬಹುದು. ಅದರಲ್ಲಿ ಜನಾಂಗದ ಭಾವನಾಡಿ ಮಿಡಿಯುತ್ತಿರುವುದು. ಒಂದು ಜನಾಂಗದಲ್ಲಿರುವ ಹಲವು ವಸ್ತುಗಳಲ್ಲಿ ಧರ್ಮವೂ ಒಂದು ಅಲಂಕಾರದ ವಸ್ತು.

ಆದರೆ ಭರತಖಂಡದಲ್ಲಿ ಧರ್ಮವೇ ಜನರ ಜೀವಾಳ. ಇದೇ ಮೂಲ, ಅಸ್ಥಿಭಾರ. ಇದರ ಮೇಲೆ ರಾಷ್ಟ್ರಸೌಧ ನಿಂತಿರುವುದು. ಇಲ್ಲಿ ರಾಜಕೀಯ, ಅಧಿಕಾರ, ಪಾಂಡಿತ್ಯ ಕೂಡ ಗೌಣ. ಭರತಖಂಡದಲ್ಲಿ ಧರ್ಮವೊಂದನ್ನೇ ಲಕ್ಷ್ಯದಲ್ಲಿಟ್ಟಿರುವರು. ಭಾರತದ ಜನಸಾಮಾನ್ಯರಲ್ಲಿ ತಿಳುವಳಿಕೆಯ ಅಭಾವ ಇದೆ ಎಂದು ಸಾವಿರಾರು ವೇಳೆ ನಾನು ಹಿಂದೆ ಕೇಳಿದ್ದೆ. ಅದು ನಿಜ. ಕೊಲಂಬೊಗೆ ಬಂದ ಮೇಲೆ ನನಗೆ ತಿಳಿಯಿತು, ಅಲ್ಲಿನ ಜನರಿಗೆ,\break ಯೂರೋಪಿನಲ್ಲಿ ಆಗುತ್ತಿರುವ ರಾಜಕೀಯ ಕ್ರಾಂತಿಯ ವಿಷಯವಾಗಿ ಒಂದು ಗೊತ್ತಿಲ್ಲ, ಮಂತ್ರಿಮಂಡಲಗಳ ಸತತ ಬದಲಾವಣೆ ಮುಂತಾದವು ಯಾವುದೂ ಗೊತ್ತಿಲ್ಲ, ಯೂರೋಪಿನಲ್ಲಿ ಆಗುತ್ತಿರುವ ಸೋಷಿಯಲಿಸಮ್​, ಅನಾರ್ಕಿಸಮ್​ ಮುಂತಾದವು ಯಾವುದೂ ಅವರಿಗೆ ತಿಳಿದಿಲ್ಲ. ಆದರೆ ವಿಶ್ವಧರ್ಮ ಸಮ್ಮೇಳನಕ್ಕೆ ಭರತಖಂಡದಿಂದ ಸಂನ್ಯಾಸಿಯೊಬ್ಬನು ಹೋಗಿದ್ದನು, ಅವನು ಯಶಸ್ಸನ್ನು ಗಳಿಸಿದನು ಎಂಬುದು ಪ್ರತಿಯೊಬ್ಬ ಸ್ತ್ರೀ – ಪುರುಷ–ಶಿಶುಗಳಿಗೂ ಗೊತ್ತಿದೆ. ಇದರಿಂದ ಜನರಿಗೆ ವಿಷಯ ಗೊತ್ತಿಲ್ಲವೆಂದು ಅಲ್ಲ ಅಥವಾ ತಿಳಿದುಕೊಳ್ಳಬೇಕೆಂಬ ಕುತೂಹಲವಿಲ್ಲವೆಂತಲೂ ಅಲ್ಲ. ಅವರ ಜೀವಕ್ಕೆ ಆವಶ್ಯಕವಾದ ವಿಷಯಗಳನ್ನು ಕುರಿತದ್ದಾದರೆ, ಅವರ ಸ್ವಭಾವಕ್ಕೆ ಅನುಗುಣವಾದ ವಿಷಯವೆಂದರೆ, ಕುತೂಹಲ ವ್ಯಕ್ತಪಡಿಸುವರು. ರಾಜಕೀಯ ಮುಂತಾದವು ಎಂದಿಗೂ ನಮ್ಮ ರಾಷ್ಟ್ರದ ಮುಖ್ಯ ವಸ್ತುಗಳಾಗಿರಲಿಲ್ಲ. ಅದು ಜೀವಿಸಿದ್ದು ಧರ್ಮದ ಮೇಲೆ ಮತ್ತು ಅಧ್ಯಾತ್ಮದ ಮೇಲೆ. ಮುಂದೆಯೂ ಇದರ ಆಧಾರದ ಮೇಲೆಯೇ ಜೀವಿಸಬೇಕಾಗಿದೆ.

ಎರಡು ಮುಖ್ಯ ಸಮಸ್ಯೆಗಳನ್ನು ಜಗತ್ತಿನ ರಾಷ್ಟ್ರಗಳು ನಿರ್ಣಯಿಸುತ್ತಿವೆ. ಅದರಲ್ಲಿ ಭರತಖಂಡ ಒಂದು ಕಡೆ, ಇತರ ರಾಷ್ಟ್ರಗಳು ಇನ್ನೊಂದು ಕಡೆ ಇವೆ. ಸಮಸ್ಯೆ ಇದು: ಯಾರು ಉಳಿಯುವರು ಎಂಬುದು. ಒಂದು ರಾಷ್ಟ್ರ ಏಕೆ ಉಳಿಯುವುದು, ಮತ್ತೊಂದು ಏಕೆ ಅಳಿಯುವುದು? ಜೀವನದ ಹೋರಾಟದಲ್ಲಿ ದ್ವೇಷ ಉಳಿಯಬೇಕೆ ಅಥವಾ ಪ್ರೀತಿ ಉಳಿಯಬೇಕೆ? ತ್ಯಾಗ ಉಳಿಯಬೇಕೆ ಅಥವಾ ಭೋಗ ಉಳಿಯಬೇಕೆ? ಜಡವಸ್ತು ಉಳಿಯಬೇಕೆ ಅಥವಾ ಚೈತನ್ಯ ಉಳಿಯಬೇಕೆ? ಇತಿಹಾಸ ಪೂರ್ವದಲ್ಲಿ ನಮ್ಮ ಪೂರ್ವಿಕರು ಹೇಗೆ ಆಲೋಚನೆ ಮಾಡುತ್ತಿದ್ದರೊ ನಾವೂ ಹಾಗೆ ಮಾಡುವೆವು. ಎಲ್ಲಿ ಸಂಪ್ರದಾಯ ಕೂಡ ಪ್ರವೇಶಿಸಲು ಅಸಾಧ್ಯವೊ, ಅಷ್ಟು ಹಿಂದೆಯೇ, ನಮ್ಮ ಪ್ರಖ್ಯಾತ ಪೂರ್ವಿಕರು ಸಮಸ್ಯೆಯ ತಮ್ಮ ಭಾಗವನ್ನು ಆರಿಸಿಕೊಂಡು ಜಗತ್ತಿಗೇ ಒಂದು ಸವಾಲನ್ನು ಎಸೆದಿರುವರು. ನಾವು ಕಂಡುಕೊಂಡಿರುವ ಪರಿಹಾರ ತ್ಯಾಗ, ನಿರ್ಭಯತೆ, ಪ್ರೇಮ ಇವೇ ಉಳಿಯಲು ಯೋಗ್ಯವಾದವು ಗಳು, ಎಂಬುದು. ಇಂದ್ರಿಯ ಸುಖದ ತ್ಯಾಗ ಒಂದು ರಾಷ್ಟ್ರ ಉಳಿಯುವಂತೆ ಮಾಡುತ್ತದೆ. ಇದನ್ನು ನಿದರ್ಶಿಸುವುದಕ್ಕೆ ಇತಿಹಾಸ ಇದೆ. ಹೇಗೆ ಹಲವು ದೇಶಗಳು ನಾಯಿ ಅಣಬೆಗಳಂತೆ ಪ್ರತಿ ಶತಮಾನದಲ್ಲಿಯೂ ಶೂನ್ಯದಿಂದ ಎದ್ದು, ಕೆಲವು ಕಾಲ ಅಟ್ಟಹಾಸದಿಂದ ಬಾಳಿ, ಕೊನೆಗೆ ನಿರ್ನಾಮವಾಗಿ ಹೋದವು ಎಂಬುದನ್ನು ಅದು ತೋರುತ್ತದೆ. ಜಗತ್ತಿನಲ್ಲಿ ಇತರರಿಗೆ ಎದುರಿಸಲು ಅಸಾಧ್ಯವಾದ ದುರ್ಗತಿ, ಅಪಾಯ, ಏರಿಳಿತಗಳನ್ನು ಎದುರಿಸಿಯೂ ಈ ಮಹಾ ಜನಾಂಗವು ಇನ್ನೂ ಜೀವಂತವಾಗಿದೆ. ಅದಕ್ಕೆ ಕಾರಣ ಇದು ತ್ಯಾಗವನ್ನು ಆರಿಸಿಕೊಂಡದ್ದು. ತ್ಯಾಗವಿಲ್ಲದೆ ಧರ್ಮ ಹೇಗೆ ಇರಬಲ್ಲದು? ಯೂರೋಪು ಬೇರೆ ದೃಷ್ಟಿಯಿಂದ ಈ ಸಮಸ್ಯೆಯನ್ನು ಪರಿಹರಿಸಲು ಪ್ರಯತ್ನಿಸುತ್ತಿದೆ. ಅದೇನೆಂದರೆ, ಮಾನವನು ಎಷ್ಟೊಂದನ್ನು ಹೊಂದಬಹುದು, ನ್ಯಾಯವಾಗಿಯೋ ಅನ್ಯಾಯವಾಗಿಯೋ ಎಷ್ಟು ಹೆಚ್ಚು ಅಧಿಕಾರವನ್ನು ಹೇಗೆ ಸಂಪಾದಿಸಬಹುದು ಎಂಬುದು. ಪಾಶ್ಚಾತ್ಯರ ನಿಯಮ ನಿರ್ದಯವಾದ ಕ್ರೂರ ಸ್ಪರ್ಧೆ. ನಮ್ಮ ನಿಯಮ ವರ್ಣವ್ಯವಸ್ಥೆ–ಎಂದರೆ, ಸ್ಪರ್ಧೆಯನ್ನು ವಿರೋಧಿಸು\-ವುದು, ಅದರ ಶಕ್ತಿಯನ್ನು ಕುಗ್ಗಿಸುವುದು, ಇದರಿಂದ ಬರುವ ಅನ್ಯಾಯಗಳನ್ನು ಕಡಮೆ ಮಾಡುವುದು ಮತ್ತು ಈ ಜೀವನ ರಹಸ್ಯವನ್ನು ದಾಟಿಹೋಗಲು ಪ್ರಯತ್ನಿಸುತ್ತಿರುವ ಮಾನವಜೀವಿಯ ಮಾರ್ಗವನ್ನು ಸುಗಮ ಮಾಡುವುದು.

(ಈ ಸಮಯದಲ್ಲಿ ಪ್ರೇಕ್ಷಕರ ಸಂದಣಿ ಹೆಚ್ಚಿ, ಸ್ವಾಮೀಜಿಯವರ ಧ್ವನಿಯನ್ನು ಕೇಳಲು ಆಗಲಿಲ್ಲ. ಕೊನೆಗೆ ಕೆಲವು ಮಾತುಗಳಿಂದ ಸ್ವಾಮೀಜಿ ಭಾಷಣವನ್ನು ಮುಕ್ತಾಯಗೊಳಿಸಿದರು.)

ಸ್ನೇಹಿತರೇ, ನಿಮ್ಮ ಉತ್ಸಾಹವನ್ನು ನೋಡಿ ನನಗೆ ಪರಮ ಸಂತೋಷವಾಗಿದೆ. ಇದು ಅತ್ಯದ್ಭುತವಾಗಿದೆ. ನನಗೆ ನಿಮ್ಮ ಮೇಲೆ ಅತೃಪ್ತಿ ಎಂದು ಎಣಿಸಬೇಡಿ. ನಿಮ್ಮ ಉತ್ಸಾಹ ನೋಡಿ ನನಗೆ ಅತ್ಯಾನಂದವಾಗಿದೆ. ಅದ್ಭುತ ಉತ್ಸಾಹ ನಮಗೆ ಇಂದು ಬೇಕಾಗಿರುವುದು. ಆದರೆ ಅದನ್ನು ಚಿರಸ್ಥಾಯಿಯಾಗಿ ಮಾಡಿ. ಕುಗ್ಗಿ ಹೋಗದಂತೆ ನೋಡಿಕೊಳ್ಳಿ. ಉತ್ಸಾಹ ಆರದೆ ಇರಲಿ. ಭರತಖಂಡದಲ್ಲಿ ನಾವು ಮಹತ್ಕಾರ್ಯಗಳನ್ನು ಸಾಧಿಸಬೇಕಾಗಿದೆ, ಅದಕ್ಕೆ ನಿಮ್ಮ ಸಹಾಯ ಅತ್ಯಾವಶ್ಯಕ; ಸ್ಫೂರ್ತಿ ಅಗತ್ಯ. ಈ ಸಭೆಯನ್ನು ಮುಂದುವರಿಸುವುದಕ್ಕೆ ಅಸಾಧ್ಯ. ನಿಮ್ಮ ಹಾರ್ದಿಕ ಉತ್ಸಾಹಪೂರಿತ ಸ್ವಾಗತಕ್ಕೆ ನಿಮಗೆ ಧನ್ಯವಾದ. ಪ್ರಶಾಂತ ಸಮಯದಲ್ಲಿ ಒಳ್ಳೆಯ ಆಲೋಚನೆಗಳನ್ನು ಮತ್ತು ಭಾವನೆಗಳನ್ನು ವ್ಯಕ್ತಪಡಿಸಬಹುದು. ಸ್ನೇಹಿತರೆ, ಸದ್ಯಕ್ಕೆ ಧನ್ಯವಾದಗಳು.

\newpage

ಎಲ್ಲಾ ಕಡೆಗಳಲ್ಲಿಯೂ ಮಾತನಾಡಲು ಆಗುವುದಿಲ್ಲ. ಇಂದು ನೀವು ನನ್ನ ದರ್ಶನದಿಂದಲೇ ತೃಪ್ತರಾಗಿ ಹೋಗಬೇಕಾಗಿದೆ. ಬೇರೊಂದು ಸಮಯದಲ್ಲಿ ಭಾಷಣ ಮಾಡುವೆನು. ನಿಮ್ಮ ಉತ್ಸಾಹಪೂರಿತ ಸ್ವಾಗತಕ್ಕೆ ಅನಂತ ಧನ್ಯವಾದಗಳು.

