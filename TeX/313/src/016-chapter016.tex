
\chapter{ದಾನ}

\vskip 5pt

ಸ್ವಾಮಿ ವಿವೇಕಾನಂದರು ಮದ್ರಾಸಿನಲ್ಲಿದ್ದಾಗ ಚನ್ನಪುರಿ ಅನ್ನದಾನ ಸಮಾಜ ಎಂಬ ಧಾರ್ಮಿಕ ಸಂಸ್ಥೆಯ ವಾರ್ಷಿಕ ಸಮ್ಮೇಳನದ ಅಧ್ಯಕ್ಷತೆ ವಹಿಸಿದ್ದರು. ಹಿಂದೆ ಮಾತನಾಡಿದವರೊಬ್ಬರು, ಇತರರಿಗಿಂತ ಬ್ರಾಹ್ಮಣರು ದಾನಕ್ಕೆ ಹೆಚ್ಚು ಪಾತ್ರರು ಎಂಬ ಭಾವನೆಯನ್ನು ಟೀಕಿಸಿದರು. ಸ್ವಾಮೀಜಿಯವರು ಇದರಲ್ಲಿ ಒಳ್ಳೆಯ ಮತ್ತು ಕೆಟ್ಟ ಅಂಶಗಳೆರಡೂ ಇವೆ ಎಂಬುದನ್ನು ತೋರಿಸುತ್ತಾ ಹೀಗೆ ಮಾತನಾಡಿದರು:

\vskip 3pt

“ದೇಶದ ಸಂಸ್ಕೃತಿಯೆಲ್ಲಾ ಬ್ರಾಹ್ಮಣರಲ್ಲಿತ್ತು. ಅವರು ಈ ರಾಷ್ಟ್ರದ\break ಚಿಂತಕರೂ ಆಗಿದ್ದರು. ಆಲೋಚನಾಪರರಾಗುವುದಕ್ಕೆ ಬೇಕಾದ ಜೀವನ ಸೌಲಭ್ಯಗಳನ್ನು ಅವರಿಗೆ ಕೊಡದೇ ಹೋದರೆ ಇದರಿಂದ ದೇಶಕ್ಕೆ ನಷ್ಟ. ಭರತಖಂಡದಲ್ಲಿ ಪ್ರಚಲಿತವಾಗಿರುವ, ವಿಮರ್ಶೆಮಾಡದೆ ದಾನಮಾಡುವ ಪದ್ಧತಿಯನ್ನು ಪಾಶ್ಚಾತ್ಯದೇಶದಲ್ಲಿರುವ ಕಾನೂನು ರೀತಿಯ ದಾನ ಪದ್ಧತಿಯೊಂದಿಗೆ ಹೋಲಿಸಿನೋಡಿ. ಭರತಖಂಡದಲ್ಲಿ ಭಿಕ್ಷುಕನು ಕೊಟ್ಟಷ್ಟರಲ್ಲಿ ಸಂತುಷ್ಟನಾಗಿ ಬಾಳುತ್ತಿದ್ದನು. ಪಾಶ್ಚಾತ್ಯ ದೇಶದ ಭಿಕಾರಿಗಳು ಬಡವರಿಗಾಗಿ ಕಟ್ಟಿಸಿರುವ ವಸತಿಗೃಹಗಳಿಗೆ ಹೋಗಲು ಇಚ್ಛಿಸುವುದಿಲ್ಲ. ಏಕೆಂದರೆ ಅವರಿಗೆ ಊಟಕ್ಕಿಂತ ಸ್ವಾತಂತ್ರ್ಯವೇ ಹೆಚ್ಚು ಪ್ರಿಯ. ಆದ್ದರಿಂದ ಅವರು ಸಮಾಜ ಕಂಟಕರಾದರು, ದರೋಡೆಕೋರರಾದರು. ಇದರಿಂದ ಪೋಲೀಸು, ನ್ಯಾಯಾಲಯ, ಜೈಲು ಇವುಗಳೆಲ್ಲಾ ಆವಶ್ಯಕವಾದುವು. ನಾಗರಿಕತೆ ಎಂಬ ಜಾಡ್ಯವಿರುವ ತನಕ ದಾರಿದ್ರ್ಯವಿದ್ದೇ ತೀರಬೇಕು. ಮತ್ತೆ ಅದಕ್ಕೆ ಪರಿಹಾರೋಪಾಯವೂ ಆವಶ್ಯಕ. ಈ ಎರಡು ಪದ್ಧತಿಗಳಲ್ಲಿ ಒಂದನ್ನು ಆರಿಸಿಕೊಳ್ಳಬೇಕಾಗಿದೆ. ಭಾರತದಲ್ಲಿರುವ, ಪಾತ್ರಾಪಾತ್ರಗಳ ವಿಚಾರವಿಲ್ಲದೆ ದಾನ ಮಾಡುವ ಪದ್ಧತಿಯಿಂದಾಗಿ, ಸಂನ್ಯಾಸಿಗಳು, ಅವರು ಅಷ್ಟೇನೂ ಪ್ರಾಮಾಣಿಕರಲ್ಲದೇ ಇದ್ದರೂ, ಆಹಾರವನ್ನು ಪಡೆಯುವುದಕ್ಕಾಗಿ ಸ್ವಲ್ಪವಾದರೂ ಶಾಸ್ತ್ರಾಧ್ಯಯನ ಮಾಡಬೇಕಾಗಿತ್ತು. ಪಾಶ್ಚಾತ್ಯರ ವಿವೇಚನಾಪೂರ್ವಕ ದಾನಪದ್ಧತಿಗೆ ಶಾಸನಬದ್ಧ ಸಂಸ್ಧೆಗಳು ಅವಶ್ಯಕವಾದುವು. ಅವಕ್ಕೆ\break ಹೆಚ್ಚಿನ ಹಣ ಬೇಕಾಗಿದೆ. ಆದರೂ ಕೊನೆಗೆ ಅದು ಭಿಕ್ಷುಕರನ್ನು ದರೋಡೆ\-ಕಾರರನ್ನಾಗಿ ಮಾಡುತ್ತದೆ, ಅಷ್ಟೆ.”

