
\chapter{ನಿಜವಾದ ಪೂಜೆ}

\begin{center}
\textbf{(ರಾಮೇಶ್ವರ ದೇವಸ್ಥಾನದಲ್ಲಿ ನೀಡಿದ ಉಪನ್ಯಾಸ)}
\end{center}

ಧರ್ಮ ಇರುವುದು ಪ್ರೀತಿಯಲ್ಲಿ, ಬಾಹ್ಯಾಚಾರದಲ್ಲಿ ಅಲ್ಲ, ಧರ್ಮಕ್ಕೆ ಶುದ್ಧ ನಿಷ್ಕಪಟ ಹೃದಯವೇ ಮುಖ್ಯ. ಒಬ್ಬನ ದೇಹ ಮತ್ತು ಮನಸ್ಸು ಪರಿಶುದ್ಧವಿಲ್ಲದೇ ಇದ್ದರೆ ಅವನು ದೇವಸ್ಥಾನಕ್ಕೆ ಬಂದು ಪ್ರಾರ್ಥಿಸಿದರೆ ಪ್ರಯೋಜನವಿಲ್ಲ. ಪರಿಶುದ್ಧ ದೇಹದಿಂದ ಮತ್ತು ಮನಸ್ಸಿನಿಂದ ಪ್ರಾರ್ಥಿಸಿದರೆ ಶಿವ ಅದನ್ನು ನೆರವೇರಿಸುವನು. ಯಾರು ಅಶುದ್ಧರಾದರೂ ಧಾರ್ಮಿಕ ವಿಷಯವನ್ನು ಬೋಧಿಸಲಿಚ್ಛಿಸುವರೋ, ಅವರ ಬೋಧನೆಯು ಕೊನೆಗೆ ನಿಷ್ಪ್ರಯೋಜಕವಾಗುವುದು. ಬಾಹ್ಯ ಪೂಜೆ ಆಂತರಿಕ ಪೂಜೆಯ ಸಂಕೇತ ಮಾತ್ರ. ಆಂತರಿಕ ಪೂಜೆಗೆ ಅಂತರಂಗ ಶುದ್ಧಿಯೇ ಮುಖ್ಯ. ಇದಿಲ್ಲದೆ ಬರಿಯ ಬಾಹ್ಯ ಪೂಜೆಯಿಂದ ಪ್ರಯೋಜನವಿಲ್ಲ. ಆದಕಾರಣ ನೀವು ಇದನ್ನು ಜ್ಞಾಪಕದಲ್ಲಿಡಬೇಕು. 

ಕಲಿಯುಗದಲ್ಲಿ ಜನರು ಬಹಳ ಕುಲಗೆಟ್ಟು ಹೋಗಿರುವರು. ಎಂತಹ ಮಹಾಪಾತಕ ಮಾಡಿದರೂ ಪುಣ್ಯಸ್ಥಳಕ್ಕೆ ಹೋದರೆ ದೇವರು ತಮ್ಮ ಪಾತಕವನ್ನು ಕ್ಷಮಿಸುವನೆಂದು ಭಾವಿಸುವರು. ಅಶುದ್ಧ ಮನಸ್ಸಿನಿಂದ ದೇವಸ್ಥಾನಕ್ಕೆ ಹೋದರೆ, ತನ್ನಲ್ಲಿರುವ ಪಾಪದ ಮೊತ್ತಕ್ಕೆ ಮತ್ತಷ್ಟನ್ನು ಸೇರಿಸಿಕೊಳ್ಳುವನು. ಮನೆಗೆ ಹೋಗುವಾಗ ಹಿಂದೆ ಇದ್ದುದಕ್ಕಿಂತ ಹೆಚ್ಚು ಹೀನನಾಗಿರುತ್ತಾನೆ. ತೀರ್ಥವೆಂದರೆ ಪವಿತ್ರ ವಸ್ತುಗಳು ಇರುವ ಸ್ಥಳ, ಮಹಾತ್ಮರು ಇರುವ ಸ್ಥಳ. ಮಹಾತ್ಮರು ಇದ್ದು, ಅಲ್ಲಿ ದೇವಸ್ಥಾನವಿಲ್ಲದೇ ಇದ್ದರೂ, ಆ ಸ್ಥಾನ ತೀರ್ಥವಾಗುತ್ತದೆ. ಅಪವಿತ್ರರು ಇರುವಲ್ಲಿ ನೂರು ದೇವಸ್ಥಾನಗಳಿದ್ದರೂ ತೀರ್ಥವು ಮಾಯವಾಗಿ ಹೋಗುತ್ತದೆ. ತೀರ್ಥಕ್ಷೇತ್ರದಲ್ಲಿ ವಾಸಿಸುವುದು ಅತಿ ಕಷ್ಟ. ಸಾಧಾರಣ ಸ್ಥಳದಲ್ಲಿ ಮಾಡಿದ ಪಾಪವನ್ನು ಸುಲಭವಾಗಿ ಕಳೆದುಕೊಳ್ಳುಬಹುದು. ತೀರ್ಥಕ್ಷೇತ್ರದಲ್ಲಿ ಮಾಡಿದ ಪಾಪದಿಂದ ಪಾರಾಗಲಾಗುವುದಿಲ್ಲ. ನಾವು ಸ್ವತಃ ಪರಿಶುದ್ಧರಾಗಿರುವುದು, ಇತರರಿಗೆ ಒಳ್ಳೆಯದನ್ನು ಮಾಡುವುದು, ಇದು ಎಲ್ಲ ಪೂಜೆಯ ಸಾರ. ಯಾರು ದೀನರಲ್ಲಿ, ದುರ್ಬಲರಲ್ಲಿ, ರೋಗಿಗಳಲ್ಲಿ ಶಿವನನ್ನು ನೋಡುವರೋ ಅವರೇ ನಿಜವಾಗಿ ಶಿವನನ್ನು ಪೂಜಿಸುವವರು. ಒಬ್ಬನು ಕೇವಲ ವಿಗ್ರಹದಲ್ಲಿ ಮಾತ್ರ ಶಿವನನ್ನು ನೋಡಿದರೆ ಅವನ ಪೂಜೆ ಬರಿಯ ಗೌಣ. ಜಾತಿ–ಕುಲ–ಗೋತ್ರಗಳನ್ನು ಲೆಕ್ಕಿಸದೆ ಯಾರು ಒಬ್ಬ ಬಡವನಲ್ಲಿ ಶಿವನನ್ನು ನೋಡಿ ಅವನಿಗೆ ಸೇವೆ ಸಲ್ಲಿಸುವರೋ, ಅವನಿಗೆ ಸಹಾಯ ಮಾಡುವರೋ, ಅವರ ಮೇಲೆ ಶಿವನಿಗೆ ಕೇವಲ ತನ್ನನ್ನು ವಿಗ್ರಹದಲ್ಲಿ ನೋಡುವವರಿಗಿಂತ ಹೆಚ್ಚು ಪ್ರೀತಿ. 

ಒಬ್ಬ ಜಮೀನ್​ದಾರನಿಗೆ ಒಂದು ತೋಟವಿತ್ತು. ಅಲ್ಲಿ ಇಬ್ಬರು ಆಳುಗಳಿದ್ದರು. ಅವರಲ್ಲಿ ಒಬ್ಬ ಸೋಮಾರಿ ಅವನು ಕೆಲಸ ಮಾಡುತ್ತಿರಲಿಲ್ಲ. ಆದರೆ ತೋಟದ ಯಜಮಾನ ಬಂದಾಗ ಕೈ ಮುಗಿದು, “ನಮ್ಮ ಯಜಮಾನನ ವದನಾರವಿಂದ ಎಷ್ಟು ಸೊಗಸಾಗಿದೆ!” ಎಂದು ಹೊಗಳುತ್ತಾ ಕುಣಿಯುತ್ತಿದ್ದನು. ಮತ್ತೊಬ್ಬ ಹೆಚ್ಚು ಮಾತನಾಡುತ್ತಿರಲಿಲ್ಲ. ಶ್ರಮಪಟ್ಟು ಕೆಲಸಮಾಡಿ ಹಲವು ಬಗೆಯ ತರಕಾರಿ ಹಣ್ಣು ಮುಂತಾದುವನ್ನು ಬೆಳಸಿ, ದೂರದಲ್ಲಿ ವಾಸಿಸುತ್ತಿದ್ದ ಯಜಮಾನನಿಗೆ ಅವನ್ನು ತಲೆಯ ಮೇಲೆ ಹೊತ್ತುಕೊಂಡು ಹೋಗಿ ಕೊಡುತ್ತಿದ್ದನು. ಯಜಮಾನನಿಗೆ ಈ ಇಬ್ಬರಲ್ಲಿ ಯಾರ ಮೇಲೆ ಪ್ರೀತಿ? ಶಿವನೇ ಯಜಮಾನ, ಈ ಪ್ರಪಂಚವೇ ಅವನ ತೋಟ. ಎರಡು ಬಗೆಯ ಸೇವಕರಿರುವರು. ಒಬ್ಬ ಶುದ್ಧ ಸೋಮಾರಿ, ಕಪಟಿ, ಏನೂ ಕೆಲಸಮಾಡುವುದಿಲ್ಲ. ಅವನು ನಮ್ಮ ಶಿವನ ಕಣ್ಣು ಮೂಗುಗಳನ್ನು ಹೊಗಳುವನು. ಮತ್ತೊಬ್ಬನು ಶಿವನ ಸಂಸಾರದಲ್ಲಿರುವ ದೀನರು, ದುರ್ಬಲರು, ಪಶುಪಕ್ಷಿಗಳು ಎಲ್ಲರಿಗೂ ಸೇವೆ ಸಲ್ಲಿಸುತ್ತಿರುವನು. ಶಿವನಿಗೆ ಯಾರ ಮೇಲೆ ಇಷ್ಟ? ನಿಜವಾಗಿ ಯಾರು ತಂದೆಗೆ ಸೇವೆಯನ್ನು ಸಲ್ಲಿಸಲು ಇಚ್ಛಿಸುವರೋ ಅವರು ಮೊದಲು ಅವನ ಮಕ್ಕಳಿಗೆ ಸೇವೆ ಸಲ್ಲಿಸಬೇಕು. ಶಿವನನ್ನು ಪೂಜಿಸಲಿಚ್ಛಿಸುವವನು ಮೊದಲು ಅವನ ಮಕ್ಕಳಿಗೆ ಸೇವೆ ಸಲ್ಲಿಸಬೇಕು. ಈ ಸೃಷ್ಟಿಯಲ್ಲಿರುವ ಎಲ್ಲಕ್ಕೂ ಅವನು ಸೇವೆ ಸಲ್ಲಿಸಬೇಕು. ಯಾರು ಭಗವಂತನ ಭೃತ್ಯರಿಗೆ ಸೇವೆ ಸಲ್ಲಿಸುವರೋ ಅವರೇ ಭಗವಂತನ ಶ್ರೇಷ್ಠ ಸೇವಕರು ಎಂದು ಶಾಸ್ತ್ರಗಳು ಸಾರುತ್ತವೆ. 

ಇದನ್ನು ನೀವು ಗಮನಿಸಿ. ಮೊದಲು ನೀವು ಪರಿಶುದ್ಧರಾಗಿ. ಯಾರು ನಿಮ್ಮ ಬಳಿಗೆ ಬರುವರೋ ಅವರಿಗೆ ಸಾಧ್ಯವಾದಷ್ಟು ಸಹಾಯಮಾಡಿ. ಇದೇ ಸತ್ಕರ್ಮ. ಇದರಿಂದ ಚಿತ್ತಶುದ್ಧಿಯಾಗುವುದು. ಸರ್ವಜೀವಿಗಳಲ್ಲಿಯೂ ನೆಲೆಸಿರುವ ಶಿವನ ದರ್ಶನ ಆಗ ನಿಮಗೆ ಆಗುವುದು. ಅವನು ಆಗಲೇ ಪ್ರತಿಯೊಬ್ಬರ ಹೃದಯದಲ್ಲಿಯೂ ಇರುವನು. ಕನ್ನಡಿಯ ಮೇಲೆ ಧೂಳು ಇದ್ದರೆ ಅದರಲ್ಲಿ ನಮ್ಮ ಪ್ರತಿ ಬಿಂಬ ಕಾಣುವುದಿಲ್ಲ. ಅಜ್ಞಾನ ಮತ್ತು ದೌರ್ಜನ್ಯವೇ ನಮ್ಮ ಹೃದಯ ಕನ್ನಡಿಯ ಮೇಲೆ ಕುಳಿತಿರುವ ಧೂಳು. ನಮ್ಮ ಹಿತವನ್ನೇ ಕುರಿತು ಆಲೋಚಿಸುವ ಸ್ವಾರ್ಥವೇ ಮಹಾಪಾಪ. ಯಾರು “ನಾನು ಮೊದಲು ಊಟ ಮಾಡುತ್ತೇನೆ! ಇತರರಿಗಿಂತ ಹೆಚ್ಚು ಸಂಪತ್ತನ್ನು ನಾನು ಇಟ್ಟುಕೊಳ್ಳುತ್ತೇನೆ! ನಾನೇ ಎಲ್ಲವನ್ನು ತೆಗೆದುಕೊಳ್ಳುತ್ತೇನೆ! ಇತರರಿಗಿಂತ ಮುಂಚೆ ನಾನು ಸ್ವರ್ಗಕ್ಕೆ ಹೋಗುತ್ತೇನೆ! ನಾನು ಮೊದಲು ಮುಕ್ತಿ ಗಳಿಸುತ್ತೇನೆ” ಎಂದು ಯೋಚಿಸುವರೋ ಅವರೇ ಸ್ವಾರ್ಥಿಗಳು. ನಿಃಸ್ವಾರ್ಥಿಗಳು: “ನಾನೇ ಕೊನೆಯಲ್ಲಿರುವೆನು, ನನಗೆ ಸ್ವರ್ಗದ ಇಚ್ಛೆಯಿಲ್ಲ. ನರಕಕ್ಕೆ ಹೋಗುವುದರಿಂದ ಇತರರಿಗೆ ನಾನು ಸಹಾಯಮಾಡುವಂತೆ ಇದ್ದರೆ ನಾನು ಅದಕ್ಕೆ ಸಿದ್ಧನಾಗಿರುವೆನು” ಎನ್ನುವರು. ಈ ನಿಃಸ್ವಾರ್ಥವೇ ಧರ್ಮದ ಪರೀಕ್ಷೆ. ಯಾರಲ್ಲಿ ಈ ನಿಃಸ್ವಾರ್ಥ ಹೆಚ್ಚು ಇದೆಯೋ ಅವನು ಹೆಚ್ಚು ಧಾರ್ಮಿಕ, ಅವನು ಶಿವನ ಸಮೀಪದಲ್ಲಿರುವನು. ಅವನು ಪಂಡಿತನಾಗಲಿ ಪಾಮರನಾಗಲಿ, ಅವನಿಗೆ ಗೊತ್ತಿರಲಿ, ಗೊತ್ತಿಲ್ಲದೆ ಇರಲಿ, ಅವನು ಇತರರಿಗಿಂತ ಹೆಚ್ಚು ದೇವರ ಸಮೀಪದಲ್ಲಿರುವನು. ಅವನು ಸ್ವಾರ್ಥಿಯಾಗಿದ್ದರೆ, ಎಲ್ಲಾ ದೇವಸ್ಥಾನಗಳನ್ನು ಸಂದರ್ಶಿಸಿದ್ದರೂ ಎಲ್ಲಾ ತೀರ್ಥಯಾತ್ರೆಯನ್ನು ಮಾಡಿದ್ದರೂ, ಮೈಮೇಲೆ ಚಿರತೆಯಂತೆ ಪಟ್ಟೆಗಳನ್ನು ಎಳೆದುಕೊಂಡಿದ್ದರೂ ಶಿವನಿಗೆ ಬಹಳ ದೂರದಲ್ಲಿರುವನು. 

