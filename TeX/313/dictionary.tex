\sethyphenation{kannada}{
ಅ
ಅಂಕುರ-ದಲ್ಲೇ
ಅಂಕುರಾರ್ಪಣ-ವಾಗಿದೆ
ಅಂಕುರಾರ್ಪಣೆ-ಗೊಂಡ
ಅಂಕುರಾವಸ್ಥೆ-ಯ-ಲ್ಲಿವೆ
ಅಂಕೆ
ಅಂಗ-ಗಳಾದ
ಅಂಗಡಿ-ಗಳಲ್ಲಿ
ಅಂಗಡಿ-ಯಲ್ಲಿ
ಅಂಗಲಾಚಿ
ಅಂಗ-ವಾಗಿ
ಅಂಗ-ವಾಗಿ-ದ್ದಾನೆ
ಅಂಗ-ವಾಗುವ-ವರೆಗೂ
ಅಂಗುಲ
ಅಂಗುಲ-ವನ್ನೂ
ಅಂಗೈ
ಅಂಗೋಪಾಂಗ-ಗಳೊಡನೆ
ಅಂಜದೆ
ಅಂಜದೇ
ಅಂಜ-ಬೇಕಾಗಿಲ್ಲ
ಅಂಜ-ಬೇಡಿ
ಅಂಜಿ
ಅಂಜಿಕೆ
ಅಂಜಿ-ಕೆ-ಗಳಿಂದ
ಅಂಜಿ-ಕೆಗೆ
ಅಂಜಿ-ಕೆಯ
ಅಂಜಿ-ಕೆ-ಯಿಲ್ಲ
ಅಂಜಿ-ಕೆಯೇ
ಅಂಜಿ-ಕೊಳ್ಳು-ವು-ದಕ್ಕೆ
ಅಂಜಿ-ದೊ-ಡ-ನೆಯೇ
ಅಂಜಿ-ಸುವ
ಅಂಜಿ-ಸು-ವು-ದಕ್ಕೆ
ಅಂಜಿ-ಸು-ವುದು
ಅಂಜುಕುಳಿ-ಗಳಾಗಿ
ಅಂಜುತ್ತಾರೆ
ಅಂಜು-ವುದು
ಅಂಟಿ-ಕೊಂಡಿ-ದ್ದರು
ಅಂಟಿ-ಯೋ-ಕ್
ಅಂತ
ಅಂತಃ
ಅಂತಃ-ಕರಣ
ಅಂತಃ-ಕರ-ಣ-ವನ್ನು
ಅಂತಃ-ಕಲಹ
ಅಂತಃ-ಕಲಹ-ಗಳೂ
ಅಂತಃ-ಸತ್ವ
ಅಂತಃ-ಸತ್ವದ
ಅಂತಃ-ಸತ್ವ-ವನ್ನೇ
ಅಂತಃ-ಸಾರ-ವೆಂದು
ಅಂತಃ-ಸ್ವ-ಭಾವ-ದಿಂದ
ಅಂತ-ಮುರ್ಖ-ವಾದ
ಅಂತ-ರಂಗ
ಅಂತರಂಗಕ್ಕೆ
ಅಂತ-ರಂಗದ
ಅಂತ-ರಂಗ-ದಲ್ಲಿ
ಅಂತ-ರಂಗ-ದಲ್ಲಿ-ರುವ
ಅಂತ-ರಂಗ-ವನ್ನು
ಅಂತ-ರಂಗ-ಶುದ್ಧಿ
ಅಂತ-ರ-ತಮ
ಅಂತ-ರವಿರು-ವುದು
ಅಂತ-ರಾತ್ಮ-ನಿಂದ
ಅಂತ-ರಾಳಕ್ಕೆ
ಅಂತ-ರಾಳ-ದಲ್ಲಿ
ಅಂತ-ರಾಳ-ದಲ್ಲಿ-ರುವ
ಅಂತ-ರಾಳ-ದಿಂದ
ಅಂತ-ರಾಳ-ವನ್ನು
ಅಂತ-ರಾಷ್ಟ್ರೀಯ
ಅಂತ-ರಿಕ
ಅಂತರ್ಗತ-ವಾಗಿದೆ
ಅಂತರ್ಗತ-ವಾಗಿ-ರುವ
ಅಂತರ್ಜಗತ್ತನ್ನು
ಅಂತರ್ಜಗ-ತ್ತಿನ
ಅಂತರ್ಜ್ಯೋತಿ
ಅಂತರ್ದೃಷ್ಟಿ
ಅಂತರ್ದೃಷ್ಟಿ-ಪರಾ-ಯಣತೆ
ಅಂತರ್ದೃಷ್ಟಿಯ
ಅಂತರ್ದೃಷ್ಟಿ-ಯನ್ನು
ಅಂತರ್ದೃಷ್ಟಿ-ಯಿಂದ
ಅಂತರ್ಧಾನ
ಅಂತರ್ನಿ-ಹಿತ
ಅಂತರ್ನಿ-ಹಿ-ತ-ವಾಗಿವೆ
ಅಂತರ್ಮುಖ
ಅಂತರ್ಮುಖ-ವಾಗಿ
ಅಂತರ್ಮುಖ-ವಾಗು-ವಂತೆ
ಅಂತರ್ಮುಖಿ
ಅಂತರ್ಮುಖಿ-ಯಾ-ದನು
ಅಂತರ್ಯಾಮಿ
ಅಂತರ್ಯಾ-ಮಿ-ಯಾಗಿ
ಅಂತರ್ಯಾ-ಮಿ-ಯಾಗಿ-ದ್ದಾನೆ
ಅಂತರ್ಯಾ-ಮಿ-ಯಾಗಿ-ರ-ಬೇಕು
ಅಂತರ್ಯಾ-ಮಿ-ಯೆಂದೂ
ಅಂತ-ಸ್ತು-ಗಳಿವೆ
ಅಂತಹ
ಅಂತ-ಹ-ಮಹಾ
ಅಂತ-ಹ-ವ-ನಿಗೆ
ಅಂತ-ಹ-ವರ
ಅಂತ-ಹ-ವ-ರಿಗೆ
ಅಂತ-ಹ-ವರು
ಅಂತಾಗು-ತ್ತೀರಿ
ಅಂತಿಮ-ಗುರಿಯ
ಅಂತಿಮ-ವಾಗಿ
ಅಂತೂ
ಅಂತೆಯೇ
ಅಂತ್ಯ-ಗಳ
ಅಂತ್ಯ-ಗಳಿಲ್ಲ
ಅಂತ್ಯ-ಗಳಿ-ಲ್ಲದ
ಅಂತ್ಯ-ಗಳಿ-ಲ್ಲ-ವೆಂದು
ಅಂತ್ಯ-ಗಳಿ-ಲ್ಲವೋ
ಅಂತ್ಯ-ಗಳೆಂದು
ಅಂತ್ಯ-ಜ-ನಿಂದಲೂ
ಅಂತ್ಯ-ಜ-ರಿಂದಲೂ
ಅಂತ್ಯ-ಜರಿ-ಗಾಗಿ
ಅಂತ್ಯದ
ಅಂತ್ಯ-ದಲ್ಲಿ
ಅಂತ್ಯ-ದಲ್ಲಿ-ರು-ವಷ್ಟು
ಅಂತ್ಯ-ಭಾಗ-ದಲ್ಲಿ
ಅಂತ್ಯ-ರ-ಹಿ-ತ-ನೆಂದೂ
ಅಂತ್ಯ-ವನ್ನು
ಅಂತ್ಯ-ವಿಲ್ಲದ
ಅಂತ್ಯಾ-ದಪಿ
ಅಂಥ
ಅಂಥದು
ಅಂಥಲ್ಲಿ
ಅಂಥ-ವನೇ
ಅಂಥ-ವರ
ಅಂಥ-ವ-ರಿಗೂ
ಅಂಥ-ವ-ರಿಗೆ
ಅಂದರೆ
ಅಂದ-ವಾಗಿ
ಅಂದಿನ
ಅಂದಿನ-ವ-ರೆಗೆ
ಅಂದಿ-ನಿಂದ
ಅಂದು
ಅಂದು-ಕೊಂಡರೆ
ಅಂಧ
ಅಂಧ-ಕಾರ
ಅಂಧ-ಕಾರ-ದಲ್ಲಿ
ಅಂಧ-ಕಾರ-ದಿಂದ
ಅಂಧ-ಕಾರ-ವನ್ನು
ಅಂಧ-ಕಾರಾ-ವೃತ
ಅಂಧನು
ಅಂಧ-ರಾಗಿ
ಅಂಧರು
ಅಂಧಾನು-ಕ-ರಣೆ
ಅಂಧೇನೈವ
ಅಂಶ
ಅಂಶ-ಗಳ
ಅಂಶ-ಗಳನ್ನು
ಅಂಶ-ಗಳನ್ನೂ
ಅಂಶ-ಗಳಲ್ಲಿ
ಅಂಶ-ಗಳಿಂದಲೂ
ಅಂಶ-ಗಳಿಗೆ
ಅಂಶ-ಗಳಿವೆ
ಅಂಶ-ಗಳು
ಅಂಶ-ಗಳೂ
ಅಂಶ-ಗಳೆಂದು
ಅಂಶ-ಗಳೆಂದೂ
ಅಂಶ-ಗಳೆ-ರಡೂ
ಅಂಶದ
ಅಂಶ-ವನ್ನಾ-ದರೂ
ಅಂಶ-ವನ್ನು
ಅಂಶ-ವನ್ನೂ
ಅಂಶ-ವಷ್ಟೆ
ಅಂಶ-ವಾಗಲಿ
ಅಂಶ-ವಿದೆ
ಅಂಶವು
ಅಂಶ-ವೆಂದರೆ
ಅಂಶವೇ
ಅಕಾಮ-ಹತ-ಕಾಮ-ದಿಂದ
ಅಕಾ-ಮಹತೋ
ಅಕ್ಬ-ರನ
ಅಕ್ಬರ್
ಅಕ್ಷ-ರವು
ಅಕ್ಷ-ರವೂ
ಅಕ್ಷರಶಃ
ಅಕ್ಷೇಪಣೆ-ಯನ್ನು
ಅಖಂಡ
ಅಖಂಡತ್ವ
ಅಖಂಡ-ತ್ವ-ವನ್ನು
ಅಖಂಡ-ವಾಗಿ
ಅಖಂಡ-ವಾಗಿಯೂ
ಅಖಂಡ-ವಾದ
ಅಖಂಡವೂ
ಅಖಂಡ-ಸತ್ಯ
ಅಖಿಲ
ಅಖಿಲಂ
ಅಖಿಲ-ಭಾರತ
ಅಗತ್ಯ
ಅಗ-ತ್ಯಕ್ಕೆ
ಅಗತ್ಯ-ಗಳನ್ನು
ಅಗತ್ಯ-ವನ್ನೂ
ಅಗತ್ಯ-ವಾಗಿ
ಅಗತ್ಯ-ವಾಗಿದೆ
ಅಗತ್ಯ-ವಾದ
ಅಗತ್ಯ-ವಿಲ್ಲ
ಅಗತ್ಯ-ವೆಂದು
ಅಗತ್ಯವೋ
ಅಗಲ-ದಂತೆ
ಅಗಾಧ
ಅಗಾಧ-ವಾದ
ಅಗಿದೆ
ಅಗಿರು-ತ್ತದೆ
ಅಗಿ-ರು-ವಂತೆಯೇ
ಅಗು-ವನು
ಅಗುವು-ದ-ರಲ್ಲಿ
ಅಗೋಚರ-ವಾಗಿ
ಅಗ್ನಿಯ
ಅಗ್ರ-ಸ್ಥಾನ
ಅಚಂಚಲ-ವಾಗಿ-ರುವ
ಅಚಂಚಲ-ವಾದ
ಅಚಲ
ಅಚಲ-ವಾಗಿ
ಅಚಲ-ವಾಗಿಯೇ
ಅಚಲ-ವಾದುದು
ಅಚಲವೂ
ಅಚಾರ್ಯರು
ಅಚೇ-ತನ
ಅಚ್ಚಳಿ-ಯದೆ
ಅಚ್ಯುತನೂ
ಅಚ್ಯುತಾನಂತ-ವಾಗಿ
ಅಜ
ಅಜನ್ಮ
ಅಜಾಗರೂ-ಕತೆ-ಯಿಂದಲೇ
ಅಜೀರ್ಣ-ವಾಗುವ-ಷ್ಟಾ-ಗಿದೆ
ಅಜೇಯ-ವಾಗಿ
ಅಜೇ-ಯವೂ
ಅಜ್ಞ-ನ-ಲ್ಲಿಯೂ
ಅಜ್ಞ-ರಾಗಿ-ರ-ಬಹುದು
ಅಜ್ಞಾತ
ಅಜ್ಞಾತ-ನಾಗಿ
ಅಜ್ಞಾತ-ವಾಗಿ
ಅಜ್ಞಾತ-ವಾಗಿಯೂ
ಅಜ್ಞಾತ-ವಾದು-ದೆಂದೂ
ಅಜ್ಞಾತ-ವಾ-ಸ-ವನ್ನು
ಅಜ್ಞಾಧಾ-ರಕ-ರಾದರೆ
ಅಜ್ಞಾನ
ಅಜ್ಞಾನ-ಜನ್ಯ-ವೆಂದೂ
ಅಜ್ಞಾನದ
ಅಜ್ಞಾನ-ದಿಂದ
ಅಜ್ಞಾನ-ದಿಂದಾಗಿ
ಅಜ್ಞಾನ-ವಿದೆಯೋ
ಅಜ್ಞಾನವು
ಅಜ್ಞಾನ-ವೆಂದು
ಅಜ್ಞಾನಾಂಧ-ಕಾರ-ದಲ್ಲಿ
ಅಜ್ಞಾನಿ-ಗಳ
ಅಜ್ಞಾನಿ-ಗಳನ್ನು
ಅಜ್ಞಾನಿ-ಗಳಿಗೂ
ಅಜ್ಞಾನಿ-ಗಳಿಗೆ
ಅಜ್ಞಾನಿ-ಗಳು
ಅಜ್ಞಾನಿಯು
ಅಜ್ಞೇಯತಾ-ವಾದಿಯ
ಅಜ್ಞೇಯ-ವಾಗಿಯೂ
ಅಜ್ಞೇಯ-ವಾದುದು
ಅಟ್ಟ-ಹಾ-ಸ-ದಿಂದ
ಅಟ್ಟ-ಹಾ-ಸವಿ-ರ-ಲಿಲ್ಲ
ಅಟ್ಟಿ-ದರೆ
ಅಟ್ಲಾಂಟಿ-ಕ್
ಅಡಕ
ಅಡಕ-ವಾಗಿದೆ
ಅಡಕ-ವಾಗಿ-ರುವ
ಅಡಕ-ವಾಗಿವೆ
ಅಡಕ-ವಾದ
ಅಡಗಿ
ಅಡಗಿ-ರುವ
ಅಡ-ಗಿವೆ
ಅಡಗಿ-ಸ-ಬೇಕು
ಅಡಚಣೆ-ಗಳನ್ನು
ಅಡಚಣೆ-ಯನ್ನು
ಅಡಚಣೆಯೇ
ಅಡಿ
ಅಡಿಗೆ
ಅಡಿ-ಗೆ-ಮ-ನೆಯ
ಅಡಿ-ದಾ-ವರೆ-ಯಲ್ಲಿ
ಅಡಿ-ದಾ-ವರೆ-ಯಲ್ಲಿ-ಡು-ವು-ದಕ್ಕೆ
ಅಡಿ-ಪಾ-ಯವೇ
ಅಡಿ-ಯ-ಲ್ಲಿಯೂ
ಅಡುಗೆ
ಅಡ್ಡ
ಅಡ್ಡ-ಲಾಗಿ
ಅಡ್ಡಿ
ಅಡ್ಡಿ-ಯಾ-ದರೆ
ಅಣಕಿಸುತ್ತಿ-ರು-ವರು
ಅಣಕಿಸುತ್ತೇವೆ
ಅಣಕಿ-ಸು-ವನು
ಅಣಕಿಸು-ವು-ದೊಂದು
ಅಣಬೆ-ಗಳಂತೆ
ಅಣಿಮಾ
ಅಣಿ-ಮಾಡಿ
ಅಣಿ-ಮಾದಿ
ಅಣಿ-ಯಾಗಿ-ರು-ವುವು
ಅಣಿ-ಯಾ-ದಾಗ
ಅಣು
ಅಣು-ವಿ-ನಂತೆ
ಅಣು-ವೆಂದು
ಅಣ್ಣತಮ್ಮಂದಿ-ರಂತಿರ-ಬಹುದೋ
ಅತಂಹ
ಅತಿ
ಅತಿ-ಕ್ರಮಿಸು-ವುದು
ಅತಿ-ಗಳಿಂದಲೂ
ಅತಿಥಿ
ಅತಿ-ಥಿ-ಯಾಗಿ
ಅತಿ-ಪ್ರ-ಬಲ-ವಾಗಿ
ಅತಿ-ಪ್ರಾ-ಕೃತ
ಅತಿ-ಮಾ-ನ-ವನು
ಅತಿ-ಮುಖ್ಯ-ವಾ-ಯಿತು
ಅತಿ-ಯಾಗಿ
ಅತಿ-ಯಾಗಿ-ತ್ತು
ಅತಿ-ಯಾದ
ಅತಿ-ರೇಕ
ಅತಿ-ರೇಕಕ್ಕೆ
ಅತಿ-ರೇಕದ
ಅತಿ-ಶಯ-ವಾಗಲು
ಅತಿ-ಶಯ-ವಾದ
ಅತಿ-ಶ-ಯೋ-ಕ್ತಿ-ಯಾಗ-ಲಾ-ರದು
ಅತಿ-ಶ್ರೇಷ್ಠ
ಅತಿ-ಸೂಕ್ಷ್ಮ
ಅತೀಂದ್ರಿಯ
ಅತೀತ
ಅತೀತದ
ಅತೀತನು
ಅತೀತ-ರಾಗ-ಬೇಕೆಂಬ
ಅತೀತ-ವಾಗಿ
ಅತೀತ-ವಾಗಿದೆ
ಅತೀತ-ವಾದು-ದನ್ನು
ಅತೀವ-ವಾದ
ಅತೃಪ್ತಿ
ಅತ್ತ
ಅತ್ತಿ-ದ್ದೇವೆ
ಅತ್ತಿರು-ವೆವು
ಅತ್ತು
ಅತ್ಯಂತ
ಅತ್ಯಂತ-ವಾಗಿ
ಅತ್ಯ-ದ್ಭುತ
ಅತ್ಯ-ದ್ಭುತ-ವಾಗಿದೆ
ಅತ್ಯ-ದ್ಭುತ-ವಾದ
ಅತ್ಯಧಿಕ
ಅತ್ಯ-ಪೂರ್ವ-ವಾದ
ಅತ್ಯಮೂಲ್ಯ
ಅತ್ಯಮೂಲ್ಯ-ವಾದ
ಅತ್ಯಮೋಘ
ಅತ್ಯಲ್ಪ
ಅತ್ಯಾ-ಚಾರ
ಅತ್ಯಾ-ಧುನಿಕ
ಅತ್ಯಾ-ನಂದ-ದಲ್ಲಿ
ಅತ್ಯಾ-ನಂದ-ವಾಗಿದೆ
ಅತ್ಯಾ-ನಂದ-ವಾಗು-ತ್ತದೆ
ಅತ್ಯಾ-ವ-ಶ್ಯಕ
ಅತ್ಯಾ-ವ-ಶ್ಯಕ-ವಾಗಿ
ಅತ್ಯಾ-ವ-ಶ್ಯಕ-ವಾಗಿ-ರು-ವುದು
ಅತ್ಯಾ-ವ-ಶ್ಯಕ-ವಾದ
ಅತ್ಯಾ-ವ-ಶ್ಯಕ-ವಾದುದೆ
ಅತ್ಯುಚ್ಚ-ನಾಗಿ
ಅತ್ಯು-ತ್ಕೃಷ್ಟ-ವಾದ
ಅತ್ಯು-ತ್ಕೃಷ್ಟ-ವಾದು-ದೆಂದು
ಅತ್ಯು-ತ್ತಮ
ಅತ್ಯು-ತ್ತಮ-ವಾದ
ಅತ್ಯು-ತ್ತಮ-ವಾದುದೇ
ಅತ್ಯು-ತ್ಸಾಹ-ದಿಂದ
ಅತ್ಯುದಾತ್ತ-ವಾದ
ಅತ್ಯು-ನ್ನತ
ಅತ್ಯು-ನ್ನತ-ವಾದ
ಅಥವಾ
ಅಥೆನ್ಸ್
ಅದಂತಿ-ರಲಿ
ಅದ-ಕ್ಕಾಗಿ
ಅದ-ಕ್ಕಾಗಿಯೆ
ಅದ-ಕ್ಕಾಗಿಯೇ
ಅದ-ಕ್ಕಿಂತ
ಅದ-ಕ್ಕಿಂತಲೂ
ಅದಕ್ಕೂ
ಅದಕ್ಕೆ
ಅದ-ಕ್ಕೆಯೇ
ಅದ-ಕ್ಕೆಲ್ಲಾ
ಅದಕ್ಕೇ
ಅದನ್ನರಿತ
ಅದನ್ನರಿಯ-ಬೇ-ಕಾದರೆ
ಅದ-ನ್ನಾರು
ಅದನ್ನು
ಅದನ್ನೂ
ಅದನ್ನೆಲ್ಲಾ
ಅದನ್ನೇ
ಅದನ್ನೇ-ಎಂದರೆ
ಅದಮ್ಯ
ಅದಮ್ಯ-ನಾಗಿ-ರ-ಬೇಕು
ಅದರ
ಅದ-ರಂತೆ
ಅದರಂತೆಯೆ
ಅದ-ರಂತೆಯೇ
ಅದರ-ಗತಿ-ಯನ್ನು
ಅದರದೇ
ಅದ-ರಲ್ಲಿ
ಅದ-ರಲ್ಲಿ-ತ್ತು
ಅದ-ರ-ಲ್ಲಿಯೂ
ಅದ-ರಲ್ಲಿ-ರುವ
ಅದ-ರಲ್ಲಿ-ರು-ವುದು
ಅದ-ರಲ್ಲೂ
ಅದರ-ಲ್ಲೆಲ್ಲಾ
ಅದ-ರಲ್ಲೇ
ಅದರ-ಲ್ಲೊಬ್ಬನು
ಅದರಾಚೆ
ಅದ-ರಿಂದ
ಅದರೆ
ಅದ-ರೊಡನೆ
ಅದರೊಳಗೆ
ಅದಲ್ಲದೆ
ಅದಾ-ಗಲೇ
ಅದಾಗಿ-ದ್ದೇವೆ
ಅದಾದ
ಅದಾದರೆ
ಅದಾ-ಯಿತು
ಅದಾ-ವು-ದೆಂದರೆ
ಅದಿದೆ
ಅದಿ-ಲ್ಲದೆ
ಅದಿಲ್ಲ-ದ್ದಿ-ದ್ದರೆ
ಅದು
ಅದು-ಎಂದು
ಅದು-ದ-ರಿಂದ
ಅದು-ವ-ರೆಗೆ
ಅದೂ
ಅದೃಶ್ಯ-ವಾಗಿದ್ದ
ಅದೃಷ್ಟ
ಅದೃ-ಷ್ಟಕ್ಕೆ
ಅದೃಷ್ಟ-ವನ್ನು
ಅದೃಷ್ಟ-ವಶಾ-ತ್
ಅದೃಷ್ಟ-ವಾದ
ಅದೃಷ್ಟ-ವಿದ್ದರೆ
ಅದೆಂದರೆ
ಅದೆಂದೂ
ಅದೆಲ್ಲ
ಅದೆಲ್ಲ-ವನ್ನೂ
ಅದೆ-ಲ್ಲವೂ
ಅದೆಲ್ಲಾ
ಅದೇ
ಅದೇನು
ಅದೇನೂ
ಅದೇ-ನೆಂದರೆ
ಅದೇನೊ
ಅದೇನೋ
ಅದೊಂದು
ಅದೊಂದೇ
ಅದೋ
ಅದ್ಧಕ್ಕೆ
ಅದ್ಧನ್ನು
ಅದ್ಬುತ
ಅದ್ಬು-ತವೂ
ಅದ್ಭುತ
ಅದ್ಭುತ-ಗಳನ್ನು
ಅದ್ಭುತ-ಗಳಿ-ಗಾಗಿ
ಅದ್ಭುತ-ಪ್ರ-ವಾಹ
ಅದ್ಭುತವಾ
ಅದ್ಭುತ-ವಾಗಿ
ಅದ್ಭುತ-ವಾಗಿವೆ
ಅದ್ಭುತ-ವಾದ
ಅದ್ಭುತ-ವಾದುದು
ಅದ್ಭುತ-ವಿಸ್ಮಯ
ಅದ್ಭುತವೂ
ಅದ್ಭುತ-ಶಕ್ತಿಯ
ಅದ್ವಿ-ತೀಯ
ಅದ್ವಿ-ತೀಯ-ವಾಗಿವೆ
ಅದ್ವಿ-ತೀಯವೂ
ಅದ್ವೈತ
ಅದ್ವೈ-ತಕ್ಕೆ
ಅದ್ವೈ-ತ-ಗಳನ್ನು
ಅದ್ವೈ-ತ-ಗಳೆ-ರಡೂ
ಅದ್ವೈ-ತದ
ಅದ್ವೈ-ತ-ದಲ್ಲಿ
ಅದ್ವೈ-ತ-ದಿಂದ
ಅದ್ವೈ-ತ-ಪರ
ಅದ್ವೈ-ತ-ಭಾಗ-ಗಳಿಗೆ
ಅದ್ವೈ-ತ-ವನ್ನಾ-ಗಲೀ
ಅದ್ವೈ-ತ-ವನ್ನು
ಅದ್ವೈ-ತ-ವನ್ನೇ
ಅದ್ವೈ-ತ-ವಾದ
ಅದ್ವೈ-ತ-ವಾದಿ-ಗಳಿಗೂ
ಅದ್ವೈ-ತವು
ಅದ್ವೈ-ತವೂ
ಅದ್ವೈ-ತ-ವೆಂದು
ಅದ್ವೈ-ತ-ವೆಂಬ
ಅದ್ವೈ-ತ-ವೆ-ನ್ನು-ವುದು
ಅದ್ವೈ-ತವೇ
ಅದ್ವೈ-ತ-ವೊಂದೇ
ಅದ್ವೈ-ತಾ-ಚಾರ್ಯ-ರಾದ
ಅದ್ವೈತಿ
ಅದ್ವೈ-ತಿ-ಗಳ
ಅದ್ವೈ-ತಿ-ಗಳಂತೆಯೇ
ಅದ್ವೈ-ತಿ-ಗಳಾಗಲಿ
ಅದ್ವೈ-ತಿ-ಗಳಾಗ-ಲಿ-ಎ-ಲ್ಲರೂ
ಅದ್ವೈ-ತಿ-ಗಳಾಗಲೀ
ಅದ್ವೈ-ತಿ-ಗಳಾಗಿ
ಅದ್ವೈ-ತಿ-ಗಳಾಗಿ-ರಲೀ
ಅದ್ವೈ-ತಿ-ಗಳಾದರೆ
ಅದ್ವೈ-ತಿ-ಗಳಾದರೋ
ಅದ್ವೈ-ತಿ-ಗಳಿಗೆ
ಅದ್ವೈ-ತಿ-ಗಳು
ಅದ್ವೈ-ತಿ-ಗಳೂ
ಅದ್ವೈ-ತಿ-ಗಳೆ-ಲ್ಲರೂ
ಅದ್ವೈ-ತಿ-ಗಳೆಲ್ಲಾ
ಅದ್ವೈ-ತಿ-ಗಳೋ
ಅದ್ವೈ-ತಿಯ
ಅದ್ವೈ-ತಿ-ಯಷ್ಟೇ
ಅದ್ವೈ-ತಿ-ಯಾಗಲೀ
ಅದ್ವೈ-ತಿ-ಯಾಗಿ-ದ್ದರೆ
ಅದ್ವೈ-ತಿ-ಯಾಗಿ-ರ-ಬಹುದು
ಅದ್ವೈ-ತಿ-ಯಾ-ದರೂ
ಅದ್ವೈ-ತಿಯು
ಅದ್ವೈ-ತಿಯೂ
ಅದ್ವೈವ
ಅಧಃ-ಪ-ತ-ನಕ್ಕೆ
ಅಧಃಪಾತಾಳಕ್ಕೆ
ಅಧಮ-ದಿಂದ
ಅಧಮ-ರಾಗಿ-ರ-ಬಹುದು
ಅಧಮ-ವಾದು-ದೆಂದರೆ
ಅಧ-ಮಾ-ಧಮ
ಅಧರ್ಮ
ಅಧರ್ಮದ
ಅಧರ್ಮ-ವನ್ನು
ಅಧಿಕ
ಅಧಿಕ-ತರ-ವಾಗಿ
ಅಧಿಕ-ವಾದ
ಅಧಿಕಾಂಶ-ದಲ್ಲಿ
ಅಧಿ-ಕಾರ
ಅಧಿಕಾ-ರಕ್ಕೆ
ಅಧಿ-ಕಾರ-ಗಳೆಲ್ಲ
ಅಧಿ-ಕಾರದ
ಅಧಿ-ಕಾರ-ದಿಂದ
ಅಧಿ-ಕಾರ-ಯುತ-ವಾದ
ಅಧಿ-ಕಾರ-ವನ್ನು
ಅಧಿ-ಕಾರ-ವನ್ನೂ
ಅಧಿ-ಕಾರ-ವಾಗಲಿ
ಅಧಿ-ಕಾರ-ವಿದೆ
ಅಧಿ-ಕಾರ-ವಿದೆ-ಯೆಂದು
ಅಧಿ-ಕಾರ-ವಿಲ್ಲ
ಅಧಿ-ಕಾರ-ವಿಲ್ಲ-ವೆಂದು
ಅಧಿ-ಕಾರ-ವೆಲ್ಲ
ಅಧಿ-ಕಾರ-ವೆಲ್ಲಿ
ಅಧಿ-ಕಾರಿ-ಗಳಾದ
ಅಧಿ-ಕಾರಿ-ಯಾಗು-ವಿರಿ
ಅಧಿ-ಕಾರಿ-ಯುತ-ವಾದ
ಅಧಿ-ಕಾರಿಯೋ
ಅಧಿ-ಕೃತ
ಅಧಿವೇಶ
ಅಧೀನ
ಅಧೀ-ನಕ್ಕೆ
ಅಧೀನ-ದಲ್ಲಿ
ಅಧೀನ-ದಲ್ಲಿದೆ
ಅಧೀನ-ದಲ್ಲಿ-ದ್ದರು
ಅಧೀನ-ದಲ್ಲಿ-ರ-ಬೇಕು
ಅಧೀನ-ದಲ್ಲಿ-ರು-ವರೋ
ಅಧೀನ-ದಲ್ಲೂ
ಅಧೀನ-ರಾ-ದಂದಿ-ನಿಂದ
ಅಧೀನ-ವಾಗ-ಬೇಕು
ಅಧೀನ-ವಾಗಿದೆ
ಅಧೀನ-ವಾಗಿ-ರ-ಬೇಕು
ಅಧೀನ-ವಾಗಿ-ರು-ವುವೋ
ಅಧೀನ-ವಾಗಿ-ವೆಯೋ
ಅಧೀರ
ಅಧೈರ್ಯ
ಅಧೋ
ಅಧೋ-ಗತಿ
ಅಧೋ-ಗತಿ-ಗಿಳಿಯುತ್ತಿರು-ವುದೂ
ಅಧೋ-ಗ-ತಿಗೆ
ಅಧೋ-ಗ-ತಿಯ
ಅಧೋ-ಗತಿ-ಯನ್ನು
ಅಧ್ಯಕ್ಷ
ಅಧ್ಯಕ್ಷತೆ
ಅಧ್ಯಕ್ಷ-ತೆ-ಯಲ್ಲಿ
ಅಧ್ಯಕ್ಷ-ನೊಡನೆ
ಅಧ್ಯಕ್ಷ-ರನ್ನು
ಅಧ್ಯಕ್ಷರು
ಅಧ್ಯ-ಯನ
ಅಧ್ಯಯ-ನಕ್ಕೆ
ಅಧ್ಯಯ-ನದ
ಅಧ್ಯ-ಯನ-ದಷ್ಟು
ಅಧ್ಯ-ಯನ-ದಿಂದ
ಅಧ್ಯ-ಯನ-ಮಾಡ-ಬೇಕು
ಅಧ್ಯ-ಯನ-ಮಾಡಿ
ಅಧ್ಯ-ಯನ-ವನ್ನು
ಅಧ್ಯಾತ್ಮ
ಅಧ್ಯಾತ್ಮಕ್ಕೆ
ಅಧ್ಯಾತ್ಮ-ಗೀತೆ-ಯನ್ನು
ಅಧ್ಯಾತ್ಮ-ಜ್ಞಾನದ
ಅಧ್ಯಾತ್ಮ-ಜ್ಞಾನ-ದಾನ
ಅಧ್ಯಾತ್ಮ-ತತ್ತ್ವ
ಅಧ್ಯಾತ್ಮದ
ಅಧ್ಯಾತ್ಮ-ದಲ್ಲಿ
ಅಧ್ಯಾತ್ಮ-ದಲ್ಲಿಯೂ
ಅಧ್ಯಾತ್ಮ-ದಾನ
ಅಧ್ಯಾತ್ಮ-ದಾನ-ಕ್ಕಿಂತ
ಅಧ್ಯಾತ್ಮ-ದಾನ-ಗಳನ್ನು
ಅಧ್ಯಾತ್ಮ-ದಿಂದ
ಅಧ್ಯಾತ್ಮ-ದೊಂದಿಗೆ
ಅಧ್ಯಾತ್ಮ-ವನ್ನು
ಅಧ್ಯಾತ್ಮ-ವಿತ್ತು
ಅಧ್ಯಾತ್ಮ-ವಿದ್ಯೆ
ಅಧ್ಯಾತ್ಮ-ವಿಲ್ಲ
ಅಧ್ಯಾತ್ಮವು
ಅಧ್ಯಾತ್ಮ-ವೆಂಬು-ದನ್ನು
ಅಧ್ಯಾತ್ಮವೇ
ಅಧ್ಯಾತ್ಮ-ಶಕ್ತಿಯ
ಅಧ್ಯಾತ್ಮ-ಸಂಪನ್ನ-ನಾದ
ಅಧ್ಯಾಪಕ-ತನ
ಅಧ್ಯಾಪನ
ಅಧ್ಯಾಯ
ಅಧ್ಯಾಯ-ವಾಗಿದೆ
ಅಧ್ಯಾಯವೂ
ಅಧ್ಯಾರೋಪ
ಅನ
ಅನಂತ
ಅನಂತ-ಕಾಲ-ದ-ವ-ರೆಗೆ
ಅನಂತ-ಕಾಲ-ದ-ವರೆ-ವಿಗೂ
ಅನಂತ-ಗಳು
ಅನಂತ-ಜೀವ
ಅನಂತ-ಜ್ಞಾನ
ಅನಂತತೆ
ಅನಂತ-ತೆಯ
ಅನಂತ-ತೆ-ಯನ್ನು
ಅನಂತ-ತೆಯೇ
ಅನಂತ-ತ್ವ-ಇವು
ಅನಂತದ
ಅನಂತ-ದಿಂದ
ಅನಂತ-ದೊಡನೆ
ಅನಂತ-ನಾಮ-ಗಳಿಂದ
ಅನಂತನು
ಅನಂತ-ಪ್ರೇಮ-ದಿಂದ
ಅನಂತರ
ಅನಂತ-ರದ
ಅನಂತ-ರವೂ
ಅನಂತ-ರವೇ
ಅನಂತ-ವನ್ನು
ಅನಂತ-ವನ್ನೂ
ಅನಂತ-ವಲ್ಲ
ಅನಂತ-ವಾಗಿ
ಅನಂತ-ವಾಗಿದೆ
ಅನಂತ-ವಾಗಿ-ರುವ
ಅನಂತ-ವಾಗಿ-ರು-ವಂತೆ
ಅನಂತ-ವಾಗು
ಅನಂತ-ವಾದ
ಅನಂತ-ವಾದುದು
ಅನಂತ-ವೆಂದು
ಅನಂತ-ವೆಂಬ
ಅನಂತ-ಶಕ್ತಿ
ಅನಂತ-ಶಕ್ತಿಯ
ಅನಂತ-ಸ್ವ-ರೂಪ
ಅನಂತಾ-ಕಾಶ-ವನ್ನು
ಅನಂತಾತ್ಮನಿ-ರು-ವನು
ಅನಂತಾ-ತ್ಮವು
ಅನರ್ಘ್ಯ
ಅನರ್ಘ್ಯ-ರತ್ನ-ಗಳನ್ನು
ಅನರ್ಘ್ಯ-ವಾದ
ಅನರ್ಹ
ಅನರ್ಹ-ವಾಗಿವೆ
ಅನವ-ರತ
ಅನವ-ರ-ತವೂ
ಅನವ-ಶ್ಯಕ
ಅನವ-ಶ್ಯಕ-ವಾಗಿ
ಅನಾಗ-ರಿಕ
ಅನಾಗ-ರಿಕನೂ
ಅನಾಗ-ರಿಕರು
ಅನಾಗರೀ-ಕರು
ಅನಾ-ಚಾರ-ಗಳ
ಅನಾಥ
ಅನಾಥರ
ಅನಾದಿ
ಅನಾದಿ-ಕಾಲ-ದಲ್ಲಿ
ಅನಾದಿ-ಕಾಲ-ದಿಂದಲೂ
ಅನಾದಿ-ಯಾಗಿ-ರುವ
ಅನಾದಿ-ಯಿಂದಲೂ
ಅನಾಯಸ-ವಾಗಿ
ಅನಾ-ಯಾ-ಸ-ವಾಗಿಯೂ
ಅನಾರ್ಕಿ-ಸಮ್
ಅನಾರ್ಯ
ಅನಾರ್ಯ-ರಾಗಲೀ
ಅನಾರ್ಯರು
ಅನಾರ್ಯರೇ
ಅನಾ-ವರಣ
ಅನಾವ-ಶ್ಯಕ
ಅನಾವ-ಶ್ಯಕ-ವಾಗಿ-ರು-ವುದು
ಅನಾವ-ಶ್ಯಕ-ವೆಂದೂ
ಅನಾಸಕ್ತಿಯ
ಅನಾಸಕ್ತಿ-ಯಿಂದ
ಅನಾಹು-ತಕ್ಕೆ
ಅನಿತ್ಯ
ಅನಿತ್ಯ-ತೆ-ಯನ್ನು
ಅನಿಬೆ-ಸಂಟ್
ಅನಿಬೆಸೆಂಟ-ರಿಗೂ
ಅನಿರೀಕ್ಷಿ-ತ-ವಾದು-ದ್ದನ್ನು
ಅನಿಶ್ಚಯಕ್ಕೆ
ಅನಿಷ್ಟ
ಅನಿ-ಸುತ್ತದೆ
ಅನೀತಿ-ಯು-ತರ-ನ್ನಾಗಿ
ಅನು
ಅನು-ಕಂಪ
ಅನು-ಕಂಪ-ದಿಂದ
ಅನು-ಕ-ರಣೆ
ಅನು-ಕ-ರಣೆಗೆ
ಅನು-ಕ-ರಣೆಯೇ
ಅನು-ಕರಿಸ
ಅನು-ಕರಿಸ-ಬೇಕೆಂದಾ-ಗಲೀ
ಅನು-ಕರಿಸ-ಬೇಡಿ
ಅನು-ಕ-ರಿ-ಸಲು
ಅನು-ಕ-ರಿಸಿ
ಅನು-ಕರಿ-ಸುತ್ತ
ಅನು-ಕರಿ-ಸುವ-ವರ-ನ್ನಲ್ಲ
ಅನು-ಕರಿ-ಸು-ವು-ದಕ್ಕೆ
ಅನು-ಕರಿಸುವುದರಿಂದೇನೂ
ಅನು-ಕೂಲ
ಅನು-ಕೂಲ-ಕರ-ವಾಗಿದೆ
ಅನು-ಕೂಲ-ಕ್ಕಾಗಿ
ಅನು-ಕೂಲ-ವಾಗಿ-ರ-ಲಿಲ್ಲ
ಅನು-ಗಾಲವೂ
ಅನು-ಗುಣ-ವಾಗಿ
ಅನು-ಗುಣ-ವಾಗಿಯೇ
ಅನು-ಗುಣ-ವಾದ
ಅನು-ಗ್ರ-ಹಕ್ಕೆ
ಅನು-ಗ್ರ-ಹಿ-ಸಲಿ
ಅನು-ಗ್ರ-ಹಿ-ಸಿ-ದ್ದೀರಿ
ಅನು-ದಿ-ನವೂ
ಅನು-ಪಮ-ವಾಗಿ
ಅನು-ಭವ
ಅನು-ಭವಕ್ಕೆ
ಅನು-ಭವ-ಗಳ
ಅನು-ಭವ-ಗಳನ್ನೂ
ಅನು-ಭವ-ಗಳಿಗೆ
ಅನು-ಭವದ
ಅನು-ಭವ-ದಲ್ಲಿ
ಅನು-ಭವ-ದಿಂದ
ಅನು-ಭವ-ಬೇಕಾಗಿದೆ
ಅನು-ಭವ-ವನ್ನು
ಅನು-ಭವ-ವಾಗಿದೆ
ಅನು-ಭವ-ವಾಗು-ವುದು
ಅನು-ಭವ-ವಿದೆ
ಅನು-ಭವ-ವಿಲ್ಲ-ದ-ವರು
ಅನು-ಭ-ವವು
ಅನು-ಭವವೂ
ಅನು-ಭವವೇ
ಅನು-ಭವಿಸ-ಬಹುದು
ಅನು-ಭವಿಸ-ಬೇಕು
ಅನು-ಭವಿ-ಸಲಿ
ಅನು-ಭವಿ-ಸಲು
ಅನು-ಭವಿಸಿ
ಅನು-ಭವಿಸಿದ
ಅನು-ಭವಿಸಿ-ದಳು
ಅನು-ಭವಿಸಿ-ರು-ವುದು
ಅನು-ಭವಿಸಿ-ರು-ವೆವು
ಅನು-ಭವಿ-ಸುತ್ತದೆ
ಅನು-ಭವಿಸುತ್ತಿದ್ದುದ-ರಿಂದಲೇ
ಅನು-ಭವಿಸುತ್ತಿರು-ತ್ತೇನೆ
ಅನು-ಭವಿಸುತ್ತಿರು-ವಿರಿ
ಅನು-ಭವಿಸುತ್ತೀರೋ
ಅನು-ಭವಿಸುತ್ತೇ-ವೆಯೋ
ಅನು-ಭೂತಿ
ಅನು-ಮತಿ
ಅನು-ಮಾ-ನಕ್ಕೆ
ಅನು-ಮಾ-ನ-ವಿ-ಲ್ಲದೆ
ಅನು-ಯಾಯಿ
ಅನು-ಯಾ-ಯಿ-ಗಳನ್ನು
ಅನು-ಯಾ-ಯಿ-ಗಳಲ್ಲಿ
ಅನು-ಯಾ-ಯಿ-ಗಳಾಗಿ-ರು-ವರು
ಅನು-ಯಾ-ಯಿ-ಗಳಿ-ಗಿಂತ
ಅನು-ಯಾ-ಯಿ-ಗಳಿಗೂ
ಅನು-ಯಾ-ಯಿ-ಗಳಿಗೆ
ಅನು-ಯಾ-ಯಿ-ಗಳು
ಅನು-ರಣಿತ-ವಾಗಿದೆ
ಅನು-ರಣಿತ-ವಾಗು-ತ್ತಿದೆಯೋ
ಅನು-ರಣಿತ-ವಾಗು-ತ್ತಿ-ರಲಿ
ಅನು-ರಣಿತ-ವಾಗುವ-ವರೆಗೂ
ಅನು-ರಣಿತ-ವಾಗುವ-ವ-ರೆಗೆ
ಅನು-ರೂಪತೆ
ಅನು-ವಾದ
ಅನು-ವಾದ-ವಾಗಿದ್ದ
ಅನು-ವಾದವು
ಅನು-ವಾದಿ-ಸಿದನು
ಅನು-ಷ್ಟುಪ್
ಅನು-ಷ್ಠಾನ
ಅನು-ಷ್ಠಾನಕ್ಕೆ
ಅನು-ಷ್ಠಾನ-ಗಳನ್ನು
ಅನು-ಷ್ಠಾನ-ಗಳಿಂದ
ಅನು-ಷ್ಠಾನ-ಗಳು
ಅನು-ಷ್ಠಾನ-ಗೊಳಿಸು-ವುದು
ಅನು-ಷ್ಠಾನದ
ಅನು-ಷ್ಠಾನ-ದಲ್ಲಿ
ಅನು-ಷ್ಠಾನ-ದಲ್ಲಿ-ದ್ದುವು
ಅನು-ಷ್ಠಾನ-ದಲ್ಲಿ-ರು-ವುದು
ಅನು-ಷ್ಠಾನ-ದಿಂದಲೂ
ಅನು-ಷ್ಠಾನ-ಯೋಗ್ಯ
ಅನು-ಷ್ಠಾನ-ಯೋಗ್ಯ-ವಾಗಿ
ಅನು-ಷ್ಠಾನ-ವನ್ನೂ
ಅನು-ಷ್ಠಾನವೂ
ಅನು-ಸರಿ-ಸ-ದರೆ
ಅನು-ಸರಿ-ಸದೆ
ಅನು-ಸರಿ-ಸ-ಬಹುದು
ಅನು-ಸರಿ-ಸ-ಬೇ-ಕಾದ
ಅನು-ಸರಿ-ಸ-ಬೇಕು
ಅನು-ಸರಿ-ಸ-ಬೇಕೆಂದು
ಅನು-ಸರಿ-ಸಬೇಕೆಂದೂ
ಅನು-ಸ-ರಿಸಿ
ಅನು-ಸರಿ-ಸಿದ
ಅನು-ಸರಿ-ಸಿ-ದರು
ಅನು-ಸರಿ-ಸಿ-ದರೂ
ಅನು-ಸರಿ-ಸಿ-ದರೆ
ಅನು-ಸರಿ-ಸಿ-ರು-ವರು
ಅನು-ಸರಿ-ಸಿ-ರು-ವು-ದನ್ನು
ಅನು-ಸರಿ-ಸುತ್ತದೆ
ಅನು-ಸರಿ-ಸು-ತ್ತಿದೆ
ಅನು-ಸರಿ-ಸು-ತ್ತಿದ್ದರು
ಅನು-ಸರಿ-ಸು-ತ್ತಿದ್ದ-ರು-ಎಂಬ
ಅನು-ಸರಿ-ಸುತ್ತಿ-ರುವ
ಅನು-ಸರಿ-ಸುತ್ತಿ-ರು-ವರು
ಅನು-ಸರಿ-ಸು-ತ್ತಿ-ರುವೆ
ಅನು-ಸರಿ-ಸು-ತ್ತಿಲ್ಲ
ಅನು-ಸರಿ-ಸುತ್ತೇವೆ
ಅನು-ಸರಿ-ಸುವ
ಅನು-ಸರಿ-ಸು-ವಂತೆ
ಅನು-ಸರಿ-ಸು-ವರು
ಅನು-ಸರಿ-ಸುವ-ವ-ರಲ್ಲಿ
ಅನು-ಸರಿ-ಸುವ-ವರಾ-ಗಿರಿ
ಅನು-ಸರಿ-ಸು-ವು-ದಿಲ್ಲ
ಅನು-ಸರಿ-ಸು-ವು-ದಿಲ್ಲವೋ
ಅನು-ಸರಿ-ಸು-ವುದು
ಅನು-ಸರಿ-ಸು-ವುದೆ
ಅನು-ಸರಿ-ಸು-ವುದೇ
ಅನು-ಸರಿ-ಸು-ವುವು
ಅನು-ಸರಿ-ಸು-ವೆನು
ಅನು-ಸಾರ-ವಾಗಿ
ಅನೃತ-ವಲ್ಲ
ಅನೇಕ
ಅನೇ-ಕ-ತೆ-ಯಲ್ಲಿ
ಅನೇ-ಕ-ದಲ್ಲಿ
ಅನೇ-ಕರ
ಅನೇ-ಕ-ರಿಗೆ
ಅನೇ-ಕರು
ಅನೇ-ಕ-ವಾಗಿ
ಅನೇ-ಕ-ವೇಳೆ
ಅನೇ-ಕಾತ್ಮ-ಗಳ
ಅನೈತಿಕ
ಅನೈತಿಕ-ವಾದದು
ಅನೈತಿ-ಹಾಸ
ಅನ್ತ್ಯಾ-ದಪಿ
ಅನ್ನ
ಅನ್ನ-ದಾನ
ಅನ್ನ-ದಾನ-ವನ್ನು
ಅನ್ನ-ವನ್ನು
ಅನ್ನ-ವಿ-ಲ್ಲದೆ
ಅನ್ಯ
ಅನ್ಯ-ಜನಾಂಗ-ಗಳಲ್ಲಿ
ಅನ್ಯ-ಜನಾಂಗದ
ಅನ್ಯಥಾ
ಅನ್ಯ-ದೇಶ-ಗಳ
ಅನ್ಯ-ದೇಶ-ಗಳಲ್ಲಿ
ಅನ್ಯ-ದೇಶ-ದಲ್ಲಾ-ಗಲಿ
ಅನ್ಯ-ದೇಶಿಯರೂ
ಅನ್ಯ-ದೇಶೀ-ಯರ
ಅನ್ಯ-ಧರ್ಮ
ಅನ್ಯ-ಧರ್ಮ-ಗಳಿ
ಅನ್ಯ-ಧರ್ಮ-ವಾಗಲಿ
ಅನ್ಯ-ಧರ್ಮ-ಸ-ಹಿ-ಷ್ಣು-ತೆಯ
ಅನ್ಯ-ಪದಾರ್ಥ
ಅನ್ಯ-ಮತ
ಅನ್ಯ-ಮತ-ಸ-ಹಿಷ್ಣು
ಅನ್ಯರ
ಅನ್ಯ-ರನ್ನು
ಅನ್ಯ-ರಾಗ-ಬೇಡಿ
ಅನ್ಯ-ರಾಷ್ಟ್ರ-ಗಳನ್ನು
ಅನ್ಯ-ರಾಷ್ಟ್ರ-ಗಳು
ಅನ್ಯ-ರಿಂದ
ಅನ್ಯ-ರಿಗೆ
ಅನ್ಯರು
ಅನ್ಯರೋ
ಅನ್ಯಾ
ಅನ್ಯಾ-ಯ-ಗಳನ್ನು
ಅನ್ಯಾ-ಯ-ವಾಗಿಯೋ
ಅನ್ಯೋನ್ಯ
ಅನ್ವಯ-ವಾಗು-ತ್ತದೆ
ಅನ್ವಯಿಸ-ಬಹು-ದಾದ
ಅನ್ವಯಿಸ-ಬೇಕು
ಅನ್ವಯಿ-ಸುತ್ತದೆ
ಅನ್ವಯಿ-ಸುತ್ತವೆ
ಅನ್ವಯಿ-ಸುವ
ಅನ್ವಯಿಸು-ವಂತ-ಹದು
ಅನ್ವಯಿ-ಸು-ವಂತೆ
ಅನ್ವಯಿ-ಸು-ವು-ದಿಲ್ಲ
ಅನ್ವಯಿಸು-ವುದು
ಅನ್ವಯಿಸು-ವುದೋ
ಅನ್ವಯಿಸು-ವುವು
ಅನ್ವಯಿಸು-ವುವೋ
ಅನ್ವೇಷಣೆ
ಅನ್ವೇಷಣೆ-ಗಳೊಂದಿಗೆ
ಅನ್ವೇಷಣೆಯ
ಅನ್ವೇಷಣೆ-ಯನ್ನು
ಅನ್ವೇಷಣೆ-ಯಲ್ಲಿ
ಅನ್ವೇಷಣೆ-ಯಿಂದ
ಅನ್ವೇಷಣೆಯು
ಅನ್ವೇಷಣೆ-ಯೆಲ್ಲ
ಅಪ-ಕೀರ್ತಿ
ಅಪ-ಕೀರ್ತಿ-ಗಳ
ಅಪಕ್ವ
ಅಪ-ನಂಬಿ-ಕೆ-ಯು-ಳ್ಳ-ವ-ರಿಗೂ
ಅಪ-ಮಾ-ನ-ಕರ-ವ-ಲ್ಲವೆ
ಅಪರಾ-ಧ-ವನ್ನು
ಅಪರಾ-ಧ-ವಲ್ಲ
ಅಪರಾ-ಧಿ-ಗಳ
ಅಪರಾ-ಧಿ-ಗಳನ್ನು
ಅಪರಿ-ಚಿತ
ಅಪರಿ-ಪಕ್ವ-ವಾದ
ಅಪರಿ-ವರ್ತನ
ಅಪರಿ-ವರ್ತನ-ಶೀಲ-ವಾದ
ಅಪ-ರೂಪ
ಅಪ-ರೂಪಕ್ಕೆ
ಅಪ-ವಾದ-ಗಳೇ
ಅಪವಿತ್ರ
ಅಪವಿತ್ರತೆ
ಅಪವಿತ್ರ-ನಾ-ಗಿದ್ದರೆ
ಅಪವಿತ್ರರು
ಅಪವಿತ್ರ-ವೆಂದು
ಅಪ-ಹಾಸ್ಯ
ಅಪಾಯ
ಅಪಾಯ-ಕರ-ವಿರ-ಬಹುದು
ಅಪಾಯ-ಕಾರಿ
ಅಪಾಯ-ಕಾರಿ-ಯಾಗಿ-ರ-ಬಹುದು
ಅಪಾಯ-ಗಳಿಂದ
ಅಪಾಯ-ಗಳಿವೆ
ಅಪಾಯ-ದಲ್ಲಿ-ತ್ತು
ಅಪಾಯ-ದಲ್ಲಿದೆ
ಅಪಾಯ-ವನ್ನಾ-ದರೂ
ಅಪಾಯ-ವನ್ನು
ಅಪಾಯ-ವಿದೆ
ಅಪಾಯ-ವಿಲ್ಲ
ಅಪಾ-ಯವೂ
ಅಪಾ-ಯವೇ
ಅಪಾಯ-ವೇ-ನೆಂದರೆ
ಅಪಾಯ-ವೊದ-ಗಿರ-ಬಹುದು
ಅಪಾರ
ಅಪಾರ-ವಾಗಿದೆ
ಅಪಾರ-ವಾದ
ಅಪಾರ-ವಾದುದು
ಅಪೂರ್ಣ
ಅಪೂರ್ಣ-ತೆ-ಯ-ಲ್ಲ-ವೆಂದೂ
ಅಪೂರ್ವ
ಅಪೂರ್ವ-ವಾದ
ಅಪೂರ್ವ-ವಾದುದು
ಅಪೇಕ್ಷಿ-ಸು-ವಷ್ಟು
ಅಪೌರುಷೇಯ
ಅಪೌರುಷೇಯತ್ವ
ಅಪೌರುಷೇಯ-ವಾಗಿದೆ
ಅಪ್ಪ
ಅಪ್ಪಣೆ
ಅಪ್ಪ-ಣೆ-ಯನ್ನು
ಅಪ್ರತಿ-ಹತ-ವಾಗಿ
ಅಪ್ರಮೇ-ಯವೂ
ಅಪ್ರ-ಯೋ-ಜ-ಕನೂ
ಅಪ್ರಾ-ಕೃತ
ಅಪ್ರಾಪ್ಯ
ಅಪ್ರಾ-ಮಾ-ಣಿ-ಕತೆ
ಅಬಲ-ರನ್ನು
ಅಭಾವ
ಅಭಾವ-ವಿ-ರ-ಲಿಲ್ಲ
ಅಭಾವ-ವಿಲ್ಲ
ಅಭಿಚಾಕಶೀತಿ
ಅಭಿ-ನಂದನಾ
ಅಭಿ-ನಂದ-ನೆಗೆ
ಅಭಿ-ನಂದ-ನೆಯ
ಅಭಿನಯಿ-ಸಿದರೆ
ಅಭಿಪ್ರಾಯ
ಅಭಿಪ್ರಾಯಕ್ಕೆ
ಅಭಿಪ್ರಾಯ-ಗಳ
ಅಭಿಪ್ರಾಯ-ಗಳನ್ನು
ಅಭಿಪ್ರಾಯ-ಗಳಿಗೆ
ಅಭಿಪ್ರಾಯ-ಗಳೂ
ಅಭಿಪ್ರಾಯದ
ಅಭಿಪ್ರಾಯ-ದಂತೆ
ಅಭಿಪ್ರಾಯ-ದಲ್ಲಿ
ಅಭಿಪ್ರಾಯ-ಪಟ್ಟರು
ಅಭಿಪ್ರಾಯ-ವನ್ನು
ಅಭಿಪ್ರಾಯ-ವನ್ನೇ
ಅಭಿಪ್ರಾಯ-ವಲ್ಲ
ಅಭಿಪ್ರಾಯ-ವಿದು
ಅಭಿಪ್ರಾಯ-ವಿದ್ದರೆ
ಅಭಿಪ್ರಾಯ-ವುಳ್ಳ-ರಾಗಿ-ರ-ಬಹುದು
ಅಭಿ-ಮಾನ
ಅಭಿ-ಮಾ-ನ-ಗಳಲ್ಲಿ
ಅಭಿ-ಮಾ-ನಿ-ಗಳು
ಅಭಿ-ವಂದ-ನೆ-ಗಳನ್ನು
ಅಭಿವೃದ್ದಿ
ಅಭಿವೃದ್ಧಿ
ಅಭಿವೃದ್ಧಿಯ
ಅಭಿವೃದ್ಧಿ-ಯಾಗ-ಬೇಕು
ಅಭಿವೃದ್ಧಿ-ಯಾಗ-ಲಾ-ರದು
ಅಭಿವೃದ್ಧಿ-ಯಾಗು-ತ್ತಿ-ರಲಿಲ್ಲ
ಅಭಿವೃದ್ಧಿ-ಯಾಗು-ವುದು
ಅಭಿ-ವ್ಯಕ್ತ-ಗೊಂಡ-ದ್ದೆಂದರೆ
ಅಭಿ-ವ್ಯಕ್ತ-ವಾಗ-ದಂತೆ
ಅಭಿ-ವ್ಯಕ್ತ-ವಾಗಿದೆ
ಅಭಿ-ವ್ಯಕ್ತ-ವಾಗುತ್ತವೆ
ಅಭಿ-ವ್ಯಕ್ತ-ವಾಗು-ವುದು
ಅಭಿ-ವ್ಯಕ್ತಿ
ಅಭಿ-ವ್ಯಕ್ತಿ-ಗಳು
ಅಭಿ-ವ್ಯಕ್ತಿಗೆ
ಅಭಿ-ವ್ಯಕ್ತಿ-ಗೊಂಡಿತು
ಅಭಿ-ವ್ಯಕ್ತಿಯ
ಅಭಿ-ವ್ಯಕ್ತಿ-ಯಲ್ಲಿ
ಅಭಿ-ವ್ಯಕ್ತಿಯೂ
ಅಭಿ-ಶಾಪವು
ಅಭೀಃ
ಅಭೀಪ್ಸೆಯ
ಅಭೂತ-ಪೂರ್ವ-ವಾದ
ಅಭೇದ್ಯ-ವಾದ
ಅಭ್ಯಂತರ-ವಿದೆ
ಅಭ್ಯಸಿ-ಸು-ತ್ತಿದ್ದರು
ಅಭ್ಯಾಸ
ಅಭ್ಯಾ-ಸಕ್ಕೆ
ಅಭ್ಯಾ-ಸ-ಗಳನ್ನು
ಅಭ್ಯಾ-ಸ-ಗಳ-ನ್ನೆಲ್ಲಾ
ಅಭ್ಯಾ-ಸದ
ಅಭ್ಯಾ-ಸ-ಮಾಡ-ಬೇಕು
ಅಭ್ಯಾ-ಸ-ವನ್ನು
ಅಭ್ಯಾ-ಸ-ವಾಗಿದೆ
ಅಭ್ಯುತ್ಥಾನಮ-ಧರ್ಮಸ್ಯ
ಅಭ್ಯುದಯ
ಅಭ್ಯುದ-ಯದ
ಅಭ್ಯುದ-ಯ-ವನ್ನು
ಅಭ್ಯುದ-ಯವೂ
ಅಮರ
ಅಮ-ರತ್ವ
ಅಮರ-ತ್ವ-ವನ್ನು
ಅಮರ-ವಾಗಿ
ಅಮರ-ವಾ-ಣಿ-ಯನ್ನು
ಅಮರ-ವಾದ
ಅಮ-ರವೂ
ಅಮಲೇರಿದ
ಅಮಿತ
ಅಮೀಬ
ಅಮೀಬ-ವಾದ್ಧರೆ
ಅಮೀಬವು
ಅಮುಖ್ಯ
ಅಮೂಲ್ಯ
ಅಮೃತ
ಅಮೃತತ್ವ
ಅಮೃತ-ತ್ವದ
ಅಮೃತ-ತ್ವ-ಮಾ-ನಶುಃ
ಅಮೃ-ತದ
ಅಮೃತ-ಪಾ-ನಕ್ಕೆ
ಅಮೃತ-ಪ್ರ-ವಾಹ
ಅಮೃತ-ರಾಗು-ತ್ತೀರಿ
ಅಮೃತ-ವಾದುದು
ಅಮೃ-ತವೂ
ಅಮೆ-ರಿಕ
ಅಮೆ-ರಿಕ-ಗಳ
ಅಮೆ-ರಿಕ-ಗಳು
ಅಮೆ-ರಿಕ-ನ-ರನ್ನು
ಅಮೆ-ರಿಕ-ನ್ನರ
ಅಮೆ-ರಿಕ-ನ್ನರು
ಅಮೆ-ರಿಕ-ವನ್ನು
ಅಮೆರಿಕಾ
ಅಮೆರಿಕಾ-ಗಳಲ್ಲಿ
ಅಮೆರಿಕಾ-ಗಳಲ್ಲಿಯೂ
ಅಮೆರಿ-ಕಾದ
ಅಮೆರಿಕಾ-ದಲ್ಲಿ
ಅಮೆರಿಕಾ-ದ-ವನು
ಅಮೆರಿಕಾ-ದ-ವರು
ಅಮೆರಿಕಾ-ದಿಂದ
ಅಮೆರಿಕಾ-ದೇಶಕ್ಕೆ
ಅಮೆರಿಕಾ-ವನ್ನು
ಅಮೇ-ರಿಕಕ್ಕೆ
ಅಮೇ-ರಿಕ-ನ್ನರು
ಅಮೇರಿಕಾ
ಅಮೇಲೆ
ಅಮೋಘ
ಅಮೋಘ-ವಾದ
ಅಯಸ್ಕಾಂತ
ಅಯಸ್ಕಾಂತ-ಶಕ್ತಿ
ಅಯುಕ್ತ-ವೆಂದೂ
ಅಯೋಗ್ಯ-ನಿಗೆ
ಅಯೋಗ್ಯ-ವಾದ
ಅಯ್ಯರ್
ಅಯ್ಯೊ
ಅಯ್ಯೋ
ಅರಗಿಳಿ-ಗಳಾಗಿ-ರು-ವರು
ಅರಗಿಳಿ-ಗಳೂ
ಅರ-ಗಿಸಿ-ಕೊಳ್ಳು-ತ್ತೀರಿ
ಅರಗಿ-ಸಿ-ಕೊಳ್ಳು-ವು-ದಕ್ಕೆ
ಅರಚಿ-ಕೊಂಡನು
ಅರಣ್ಯ
ಅರಣ್ಯ-ಗಳಲ್ಲಿ
ಅರಣ್ಯ-ದಲ್ಲಿ
ಅರಣ್ಯ-ವಾ-ಸಿ-ಯಾ-ದನು
ಅರಣ್ಯಾ-ರಣ್ಯ
ಅರಬ್ಬರ
ಅರಬ್ಬಿ-ಯರು
ಅರ-ಮ-ನೆ-ಗಳಲ್ಲಿಯೂ
ಅರ-ಮ-ನೆ-ಗಳಲ್ಲೂ
ಅರ-ಮ-ನೆ-ಯಲ್ಲಿ
ಅರಳಿ
ಅರಳಿ-ಸು-ವಂತೆ
ಅರಳಿ-ಸು-ವಂತೆ-ಭ-ರತ-ಖಂಡದ
ಅರ-ಸರ
ಅರಸ-ರಿಗೆ
ಅರ-ಸಿ-ಕೊಂಡು
ಅರ-ಸಿದೆ
ಅರ-ಸು-ತ್ತಿದೆ
ಅರ-ಸುವ
ಅರ-ಸುವ-ವರು
ಅರಸು-ವಾಗ
ಅರಸು-ವು-ದಲ್ಲ
ಅರ-ಸು-ವು-ದಾದರೆ
ಅರಿತ
ಅರಿ-ತರು
ಅರಿತ-ವ-ನಿಗೆ
ಅರಿತಿ-ದ್ದೇವೆ
ಅರಿತಿ-ರುವ-ವ-ರಂತೆ
ಅರಿತು
ಅರಿ-ತು-ಕೊಂಡರು
ಅರಿ-ತು-ಕೊಳ್ಳಲಿ
ಅರಿ-ಯದೆ
ಅರಿಯ-ಬಲ್ಲರು
ಅರಿಯ-ಬಲ್ಲೆವು
ಅರಿಯ-ಬೇಕು
ಅರಿಯ-ಬೇಕೆಂದು
ಅರಿ-ಯಲಿ
ಅರಿ-ಯಲು
ಅರಿಯಳು
ಅರಿಯು-ತ್ತಿವೆ
ಅರಿಯುತ್ತೀರೋ
ಅರಿ-ಯು-ವು-ದಕ್ಕೆ
ಅರಿಯು-ವುದು
ಅರಿಯು-ವು-ದೆಂದರೆ
ಅರಿ-ಯು-ವುದೇ
ಅರಿ-ವನ್ನು
ಅರಿ-ವಾಗು-ತ್ತದೆ
ಅರಿ-ವಾಗು-ತ್ತಿದೆ
ಅರಿ-ವಾಗು-ವುದು
ಅರಿ-ವಿಗೂ
ಅರಿ-ವಿಗೆ
ಅರಿ-ವಿ-ನಿಂದ
ಅರಿ-ವಿ-ಲ್ಲದೆ
ಅರಿವು
ಅರಿ-ವುಂಟಾ-ಗು-ತ್ತದೆ
ಅರಿವೇ
ಅರುಂಧತಿ
ಅರೆ
ಅರೆ-ಬತ್ತಲೆ
ಅರೆ-ಹುಚ್ಚ-ರೆಂದು
ಅರ್ಚಕ-ರಿಗೆ
ಅರ್ಜುನ
ಅರ್ಥ
ಅರ್ಥ-ಗರ್ಭಿತ-ವಾಗಿದೆ
ಅರ್ಥ-ಗಳನ್ನು
ಅರ್ಥದ
ಅರ್ಥ-ದಲ್ಲಿ
ಅರ್ಥ-ದಲ್ಲಿಯೇ
ಅರ್ಥ-ದಲ್ಲೂ
ಅರ್ಥ-ಪೂರ್ಣ-ವಾಗಿ-ತ್ತು
ಅರ್ಥ-ಪೂರ್ಣ-ವಾದ
ಅರ್ಥ-ಪೂರ್ಣವೂ
ಅರ್ಥ-ಬರು-ವಂತೆ
ಅರ್ಥ-ಮಾಡಿ-ಕೊಂಡಿ-ರು-ವಂತೆ
ಅರ್ಥ-ಮಾಡಿ-ಕೊಳ್ಳ-ಬಲ್ಲರು
ಅರ್ಥ-ಮಾಡಿ-ಕೊಳ್ಳ-ಬೇಕಾಗಿದೆ
ಅರ್ಥ-ಮಾಡಿ-ಕೊಳ್ಳ-ಬೇ-ಕಾದ
ಅರ್ಥ-ಮಾಡಿ-ಕೊಳ್ಳಲಿಲ್ಲ
ಅರ್ಥ-ಮಾಡಿ-ಕೊಳ್ಳಲು
ಅರ್ಥ-ಮಾಡಿ-ಕೊಳ್ಳು-ವಂತೆ
ಅರ್ಥ-ಮಾಡಿ-ಕೊಳ್ಳು-ವು-ದಕ್ಕೆ
ಅರ್ಥ-ಮಾಡಿ-ಸಿ-ದ್ದೀರಿ
ಅರ್ಥ-ವನ್ನು
ಅರ್ಥ-ವನ್ನು-ಕಳೆ-ದು-ಕೊಂಡಿದೆ
ಅರ್ಥ-ವನ್ನೆಲ್ಲಾ
ಅರ್ಥ-ವಲ್ಲ
ಅರ್ಥ-ವಾಗು-ತ್ತದೆ
ಅರ್ಥ-ವಾಗುತ್ತಿ-ದ್ದುದು
ಅರ್ಥ-ವಾಗು-ವು-ದಿಲ್ಲ
ಅರ್ಥ-ವಾಗು-ವುದು
ಅರ್ಥ-ವಿತ್ತು
ಅರ್ಥ-ವಿದೆ
ಅರ್ಥ-ವಿಲ್ಲ
ಅರ್ಥ-ವಿಲ್ಲದ್ದು
ಅರ್ಥವೂ
ಅರ್ಥ-ವೆಲ್ಲ
ಅರ್ಥ-ವೆಲ್ಲಾ
ಅರ್ಥವೇ
ಅರ್ಥ-ವೇನು
ಅರ್ಧ
ಅರ್ಧ-ಗಂಟೆ
ಅರ್ಧ-ಭಾಗ
ಅರ್ಧ-ಭಾಗ-ವನ್ನು
ಅರ್ಧ-ರಾತ್ರಿ
ಅರ್ಧ-ವನ್ನು
ಅರ್ಪಿ-ಸ-ಬೇಕು
ಅರ್ಪಿ-ಸ-ಲಾ-ಯಿತು
ಅರ್ಪಿ-ಸಲು
ಅರ್ಪಿ-ಸಲೂ
ಅರ್ಪಿ-ಸಿತು
ಅರ್ಪಿ-ಸಿದರು
ಅರ್ಪಿ-ಸಿ-ರು-ವರು
ಅರ್ಪಿ-ಸಿರು-ವೆವು
ಅರ್ಪಿ-ಸುತ್ತಿ-ರು-ವರು
ಅರ್ಪಿ-ಸುತ್ತಿರು-ವುದು
ಅರ್ಪಿ-ಸು-ತ್ತೇನೆ
ಅರ್ಪಿ-ಸು-ವುದ-ರಲ್ಲೇ
ಅರ್ಪಿ-ಸು-ವುದು
ಅರ್ಪಿ-ಸು-ವುದೇ
ಅರ್ಪಿ-ಸೋಣ
ಅರ್ಹ-ನಲ್ಲ
ಅರ್ಹ-ವಾಗಿದೆ
ಅರ್ಹ-ವಾಗಿ-ದ್ದರೆ
ಅರ್ಹ-ವಾಗಿಲ್ಲ
ಅಲಂಕ-ರಿಸಿದೆ
ಅಲಂಕಾರ
ಅಲಂಕಾರ-ಗಳಂತೆ
ಅಲಂಕಾರ-ಗೊಂಡ
ಅಲಂಕಾರದ
ಅಲಂಕಾರಿಕ
ಅಲಂಕೃತ-ವಾಗಿದ್ದ
ಅಲಗಿ-ನಂತೆ
ಅಲ-ರನ್ನು
ಅಲು-ಗಿ-ಸದೆ
ಅಲೆ
ಅಲೆ-ಗಳಂತೆ
ಅಲೆ-ಗಳ-ನ್ನೆಲ್ಲ
ಅಲೆ-ಗಳಾಗಿ
ಅಲೆ-ಗಳಿ-ರು-ತ್ತವೆ
ಅಲೆ-ಗಳು
ಅಲೆ-ಗಳೆಂದು
ಅಲೆ-ಗಳೆಂಬು-ದನ್ನು
ಅಲೆ-ಗಳೆದ್ದು
ಅಲೆ-ಗಳೇ-ಳು-ತ್ತವೆ
ಅಲೆಗೆ
ಅಲೆ-ಗ್ಸಾಂಡ-ರನ
ಅಲೆ-ಗ್ಸಾಂಡರ-ನೊಂದಿಗೆ
ಅಲೆ-ಗ್ಸಾಂಡರ್
ಅಲೆ-ಗ್ಸಾಂಡ್ರಿಯಾ
ಅಲೆ-ದಾಡಿದ್ದರೂ
ಅಲೆ-ದಾಡಿ-ದ್ದಾರೆ
ಅಲೆಯ
ಅಲೆ-ಯಂತೆ
ಅಲೆ-ಯಾಗಿಯೂ
ಅಲೆ-ಯಾಗಿ-ರ-ಬಹುದು
ಅಲೆಯು
ಅಲೆ-ಯುಂಟಾ-ಗು-ತ್ತದೆ
ಅಲೆ-ಯು-ತ್ತಿದ್ದ-ವನು
ಅಲೆ-ಯು-ತ್ತಿದ್ದೆ
ಅಲೋಚನಾ-ಪರ-ರಿ-ರು-ವರು
ಅಲೌಕಿಕ
ಅಲ್ಪ
ಅಲ್ಪ-ಜ್ಞಾನ
ಅಲ್ಪ-ಜ್ಞಾನಿಯು
ಅಲ್ಪ-ಭಾಗ
ಅಲ್ಪ-ಮತಿ-ಗಳಿಗೂ
ಅಲ್ಪ-ಮ-ತಿಗೆ
ಅಲ್ಪ-ವನ್ನು
ಅಲ್ಪ-ವಾಗಿ-ರು-ವುದೋ
ಅಲ್ಪ-ಸಂಖ್ಯಾತ
ಅಲ್ಪ-ಸಂಖ್ಯಾ-ತರ
ಅಲ್ಪಾಂಶವು
ಅಲ್ಪಾ-ವಧಿ-ಯಲ್ಲಿ
ಅಲ್ಲ
ಅಲ್ಲ-ಗಳೆದು
ಅಲ್ಲ-ಗಳೆ-ದು-ದ-ರಿಂದಲೇ
ಅಲ್ಲ-ಗಳೆ-ಯ-ಕೂಡದು
ಅಲ್ಲ-ಗಳೆ-ಯ-ಬಹುದು
ಅಲ್ಲ-ಗಳೆ-ಯಬೇಕೆಂಬುದು
ಅಲ್ಲ-ಗಳೆ-ಯ-ಬೇಡಿ
ಅಲ್ಲ-ಗಳೆ-ಯ-ಲಿಲ್ಲ
ಅಲ್ಲ-ಗಳೆ-ಯು-ವು-ದಕ್ಕೆ
ಅಲ್ಲ-ಗಳೆ-ಯು-ವು-ದಲ್ಲ
ಅಲ್ಲ-ಗಳೆಯು-ವುದಾ-ಗಲೀ
ಅಲ್ಲ-ಗಳೆ-ಯು-ವು-ದಿಲ್ಲ
ಅಲ್ಲ-ಗಳೆಯು-ವುದು
ಅಲ್ಲದೆ
ಅಲ್ಲನೋ
ಅಲ್ಲಲ್ಲಿ
ಅಲ್ಲವೆ
ಅಲ್ಲ-ವೆಂದು
ಅಲ್ಲ-ವೆಂಬು-ದನ್ನು
ಅಲ್ಲವೇ
ಅಲ್ಲವೋ
ಅಲ್ಲಾ
ಅಲ್ಲಾ-ಡುವ
ಅಲ್ಲಾ-ನನ್ನು
ಅಲ್ಲಿ
ಅಲ್ಲಿಂದ
ಅಲ್ಲಿಗೂ
ಅಲ್ಲಿಗೆ
ಅಲ್ಲಿದೆ
ಅಲ್ಲಿದ್ದ
ಅಲ್ಲಿನ
ಅಲ್ಲಿಯ
ಅಲ್ಲಿ-ಯ-ವರೆಗೂ
ಅಲ್ಲಿ-ಯ-ವ-ರೆಗೆ
ಅಲ್ಲಿ-ಯ-ವರೆ-ವಿಗೂ
ಅಲ್ಲಿಯೂ
ಅಲ್ಲಿ-ರುವ
ಅಲ್ಲಿ-ರು-ವುದು
ಅಲ್ಲಿಲ್ಲ
ಅಲ್ಲಿವೆ
ಅಲ್ಲೂ
ಅಲ್ಲೆಲ್ಲ
ಅಲ್ಲೆಲ್ಲಾ
ಅಲ್ಲೇ
ಅಲ್ಲೊಂದು
ಅಲ್ಲೋಪ-ನಿಷತ್ತು
ಅಳವಡಿಸಿ
ಅಳವಡಿಸಿ-ಕೊಂಡಿ-ರು-ವರು
ಅಳವಡಿಸಿ-ಕೊಳ್ಳು-ವ-ವ-ರೆಗೆ
ಅಳವಡಿಸಿ-ಕೊಳ್ಳು-ವಷ್ಟು
ಅಳಿ-ದರೂ
ಅಳಿದು-ಳಿದ
ಅಳಿದು-ಹೋ-ಗು-ತ್ತದೆ
ಅಳಿ-ಯುವ
ಅಳಿಯು-ವುದು
ಅಳಿಲಿ-ನಂತೆ
ಅಳಿಸಿ
ಅಳು
ಅಳು-ತ್ತಿರು-ವೆವು
ಅಳು-ಮೋರೆ-ಯಿಲ್ಲ
ಅಳು-ವಲ್ಲ
ಅಳೆಯ-ಬೇಡಿ
ಅಳೆಯು-ವುದು
ಅವ
ಅವ-ಕಾಶ
ಅವ-ಕಾಶ-ಕೊ-ಡದೆ
ಅವ-ಕಾಶ-ಕೊಡ-ಬೇಕು
ಅವ-ಕಾಶ-ಕೊಡಿ
ಅವ-ಕಾಶ-ಕ್ಕಾಗಿ
ಅವ-ಕಾಶ-ಗಳ
ಅವ-ಕಾಶ-ಮಾಡಿ-ದೊಡ-ನೆಯೆ
ಅವ-ಕಾಶ-ವನ್ನು
ಅವ-ಕಾಶ-ವನ್ನೂ
ಅವ-ಕಾಶ-ವನ್ನೇಕೆ
ಅವ-ಕಾಶ-ವಿದೆ
ಅವ-ಕಾಶ-ವಿರ-ಬೇಕು
ಅವ-ಕಾಶ-ವಿ-ರ-ಲಿಲ್ಲ
ಅವ-ಕಾಶ-ವಿಲ್ಲ
ಅವ-ಕಾಶವು
ಅವ-ಕಾಶವೇ
ಅವ-ಕ್ಕಾಗಿ
ಅವಕ್ಕೂ
ಅವಕ್ಕೆ
ಅವ-ಗಾಹ-ನೆಗೆ
ಅವ-ತ-ರಿಸಿ
ಅವ-ತರಿಸು-ತ್ತೇನೆ
ಅವ-ತಾರ
ಅವ-ತಾರ-ಗಳನ್ನು
ಅವ-ತಾರ-ಗಳನ್ನೂ
ಅವ-ತಾರ-ಗಳಾಗಲೀ
ಅವ-ತಾರ-ಗಳಿಗೆ
ಅವ-ತಾರ-ಗಳು
ಅವ-ತಾರದ
ಅವ-ತಾರ-ನಾದ
ಅವ-ತಾರ-ಪುರು-ಷರು
ಅವ-ತಾರ-ವಾದದ
ಅವ-ತಾರ-ವೆ-ತ್ತಿದನು
ಅವನ
ಅವ-ನಂತೆ
ಅವ-ನಂತೆಯೇ
ಅವ-ನತಿ
ಅವ-ನ-ತಿ-ಗಳ
ಅವ-ನ-ತಿ-ಗಿಂತ
ಅವ-ನ-ತಿಗೆ
ಅವ-ನ-ತಿ-ಗೆಲ್ಲಾ
ಅವ-ನ-ತಿಯ
ಅವ-ನ-ತಿ-ಯನ್ನು
ಅವ-ನ-ತಿ-ಯಾಗು-ವುದೋ
ಅವ-ನ-ತಿ-ಯಿಂದ
ಅವ-ನದೇ
ಅವ-ನನ್ನು
ಅವ-ನ-ನ್ನೇ
ಅವ-ನ-ನ್ನೇಕೆ
ಅವ-ನಲ್ಲಿ
ಅವ-ನ-ಲ್ಲಿಯೇ
ಅವ-ನಲ್ಲಿ-ರ-ಕೂಡದು
ಅವ-ನ-ವ-ನಿಗೇ
ಅವ-ನ-ವನು
ಅವ-ನಿಂದ
ಅವ-ನಿಂದಲೇ
ಅವ-ನಿ-ಗಿಂತ
ಅವ-ನಿಗೂ
ಅವ-ನಿಗೆ
ಅವ-ನಿಗೊಂದು
ಅವನು
ಅವನೂ
ಅವ-ನೆ-ಲ್ಲಿ-ಯಾ-ದರೂ
ಅವ-ನೆ-ಲ್ಲಿ-ರು-ತ್ತಿದ್ದನು
ಅವ-ನೆಲ್ಲೋ
ಅವನೇ
ಅವ-ನೊಂದಿಗೆ
ಅವ-ನೊಂದು
ಅವ-ನೊಬ್ಬ
ಅವನೊ-ಬ್ಬನೇ
ಅವನ್ನು
ಅವನ್ನೆಲ್ಲ
ಅವನ್ನೇ
ಅವ-ಮಾನ
ಅವ-ಮಾ-ನ-ವಾ-ಯಿ-ತೆಂದು
ಅವರ
ಅವ-ರಂತೆ
ಅವ-ರಂತೆಯೇ
ಅವ-ರ-ದಲ್ಲ
ಅವ-ರದು
ಅವ-ರ-ನ್ನಾಳುವ
ಅವ-ರನ್ನು
ಅವ-ರ-ನ್ನೆಲ್ಲ
ಅವ-ರಲ್ಲಿ
ಅವ-ರ-ಲ್ಲಿದೆ
ಅವ-ರ-ಲ್ಲಿಯೂ
ಅವ-ರಲ್ಲಿ-ರ-ಬೇಕು
ಅವ-ರಲ್ಲಿ-ರುವ
ಅವ-ರ-ವರ
ಅವ-ರಾತ-ಗಳಂತೆ
ಅವ-ರಾರು
ಅವ-ರಿಂದ
ಅವ-ರಿಂದಲೇ
ಅವ-ರಿ-ಗಾಗಿ
ಅವ-ರಿಗಾ-ದರೂ
ಅವ-ರಿ-ಗಿಂತ
ಅವ-ರಿಗಿ-ರುವ
ಅವ-ರಿಗೂ
ಅವ-ರಿಗೆ
ಅವ-ರಿ-ಗೆಲ್ಲಾ
ಅವ-ರಿಗೊಂದು
ಅವ-ರಿನ್ನೂ
ಅವರು
ಅವರು-ಧರ್ಮ-ವನ್ನು
ಅವರು-ನಿಜ-ವಾದ
ಅವರೂ
ಅವ-ರೆದು-ರಿಗೆ
ಅವ-ರೆಲ್ಲ
ಅವ-ರೆಲ್ಲ-ರನ್ನೂ
ಅವ-ರೆಲ್ಲರೂ
ಅವ-ರೆಲ್ಲಾ
ಅವರೇ
ಅವ-ರೇಕೆ
ಅವ-ರೇ-ನಾ-ದರೂ
ಅವ-ರೇನು
ಅವ-ರೇ-ಶ್ರೀ-ರಾಮ-ಕೃಷ್ಣ
ಅವ-ರೊಂದಿಗೆ
ಅವ-ರೊಡನೆ
ಅವ-ಲಂಬಿಸ-ಬೇಕು
ಅವ-ಲಂಬಿಸಿತ್ತೋ
ಅವ-ಲಂಬಿಸಿ-ದಂಥವು
ಅವ-ಲಂಬಿಸಿ-ದರೆ
ಅವ-ಲಂಬಿ-ಸಿದೆ
ಅವ-ಲಂಬಿಸಿ-ರು-ತ್ತದೆ
ಅವ-ಲಂಬಿಸಿ-ರುವ
ಅವ-ಲಂಬಿಸು-ವುದು
ಅವಳ
ಅವ-ಳನ್ನು
ಅವ-ಳಿಗೆ
ಅವ-ಳಿ-ರುವ
ಅವಳು
ಅವಳೇ
ಅವ-ಶೇಷ
ಅವ-ಶೇಷ-ಗಳನ್ನು
ಅವ-ಶೇಷ-ದಿಂದ
ಅವ-ಶೇಷ-ವಷ್ಟೆ
ಅವ-ಶ್ಯಕ
ಅವ-ಶ್ಯಕತೆ
ಅವ-ಶ್ಯಕ-ತೆ-ಗಳ-ನ್ನೆಲ್ಲಾ
ಅವ-ಶ್ಯಕ-ತೆ-ಯನ್ನು
ಅವ-ಶ್ಯಕ-ತೆ-ಯಿದೆ
ಅವ-ಶ್ಯಕ-ತೆ-ಯೇನು
ಅವ-ಶ್ಯಕ-ವಾಗಿ-ತ್ತು
ಅವ-ಶ್ಯಕ-ವಾದ
ಅವ-ಶ್ಯಕ-ವಾದುವು
ಅವ-ಶ್ಯಕವೋ
ಅವ-ಶ್ಯ-ವಲ್ಲ
ಅವ-ಶ್ಯ-ವಾಗಿದೆ
ಅವ-ಶ್ಯ-ವಾಗಿ-ದ್ದವು
ಅವ-ಶ್ಯವಿ-ರುವ
ಅವ-ಸರ-ದಲ್ಲಿ
ಅವ-ಸರ-ದಿಂದ
ಅವ-ಸರ-ಪಡ-ಬೇಕಾಗಿಲ್ಲ
ಅವ-ಸರ-ಪಡ-ಬೇಡಿ
ಅವ-ಸಾನ
ಅವ-ಸ್ಥೆ-ಗಳ-ನ್ನೊಳ-ಗೊಂಡ
ಅವ-ಸ್ಥೆ-ಗಳಲ್ಲಿಯೂ
ಅವ-ಸ್ಥೆಯ
ಅವ-ಸ್ಥೆಯನ್ನು
ಅವ-ಸ್ಥೆಯಲ್ಲಿ
ಅವ-ಸ್ಥೆಯು
ಅವಿ-ಕಾರಿ-ಯಾ-ದು-ದನ್ನು
ಅವಿ-ಕಾರಿ-ಯಾ-ದುದು
ಅವಿ-ಕಾರಿಯೂ
ಅವಿಚ್ಛೇದ
ಅವಿದ್ಯಾ
ಅವಿದ್ಯಾ-ಯಾ-ಮಂತರೇ
ಅವಿದ್ಯಾ-ವಂತ
ಅವಿದ್ಯೆ
ಅವಿದ್ಯೆಯ
ಅವಿ-ಧೇಯ-ರಿಗೆ
ಅವಿನಾಶಿ
ಅವಿನಾ-ಶಿ-ಯಾಗಿರು
ಅವಿನಾ-ಶಿ-ಯಾದ
ಅವಿನಾ-ಶಿಯೂ
ಅವಿನ್ನೂ
ಅವಿಭಾಜ್ಯ
ಅವಿ-ರಾಮ-ವಾಗಿ
ಅವಿರ್ಭವಿಸು-ವುದು
ಅವು
ಅವು-ಗಳ
ಅವು-ಗಳಂತೆಯೇ
ಅವು-ಗಳದೇ
ಅವು-ಗಳನ್ನು
ಅವು-ಗಳ-ನ್ನೆಲ್ಲ
ಅವು-ಗಳ-ನ್ನೆಲ್ಲಾ
ಅವು-ಗಳ-ಲ್ಲ-ವೆಂದೇ
ಅವು-ಗಳಲ್ಲಿ
ಅವು-ಗಳ-ಲ್ಲಿ-ರುವ
ಅವು-ಗಳ-ಲ್ಲೆಲ್ಲಾ
ಅವು-ಗಳಾ-ಗುವ
ಅವು-ಗಳಿಂದ
ಅವು-ಗಳಿ-ಗಾಗಿ
ಅವು-ಗಳಿಗೆ
ಅವು-ಗಳಿ-ರು-ವುದೇ
ಅವು-ಗಳಿ-ಸಿವೆ
ಅವು-ಗಳು
ಅವು-ಗಳೂ
ಅವು-ಗಳೆಲ್ಲ
ಅವು-ಗಳೆ-ಲ್ಲ-ದ-ರ-ಲ್ಲಿಯೂ
ಅವು-ಗಳೆ-ಲ್ಲವೂ
ಅವು-ಗಳೆಲ್ಲಾ
ಅವು-ಗಳೊಡನೆ
ಅವೂ
ಅವೃಜಿನ-ಪಾಪ
ಅವೃಜಿನೋ
ಅವೆಂದಿಗೂ
ಅವೆ-ರಡೂ
ಅವೆಲ್ಲ
ಅವೆ-ಲ್ಲಕ್ಕೂ
ಅವೆ-ಲ್ಲ-ದರ
ಅವೆ-ಲ್ಲವೂ
ಅವೆಲ್ಲಾ
ಅವೇ
ಅವೈ-ಯಕ್ತಿಕ-ವಾಗಿ-ದ್ದರೂ
ಅವ್ಯಕ್ತ
ಅವ್ಯಕ್ತ-ಕ್ಕಿಂತ
ಅವ್ಯಕ್ತ-ಬ್ರಹ್ಮ-ನೆಂದು
ಅವ್ಯಕ್ತ-ವಾಗಿ-ರಲೇ-ಬೇಕು
ಅವ್ಯಕ್ತ-ವಾಗು-ವುದು
ಅವ್ಯಕ್ತ-ವಾದ
ಅವ್ಯಕ್ತವೂ
ಅವ್ಯವ-ಸ್ಥಿತ-ವಾಗಿದೆ
ಅವ್ಯವ-ಸ್ಥೆಯೇ
ಅಶಕ್ತ-ರಾಗಿ-ದ್ದೇವೆ
ಅಶಕ್ತಿ-ಗಳಿಗೆ
ಅಶಾಂತ-ರಾಗಿ-ದ್ದೀರಾ
ಅಶಾಂತ-ವಾಗಿ
ಅಶಿಕ್ಷಿತ
ಅಶುದ್ದಿ-ಗೊಳಿಸು-ವುದು
ಅಶುದ್ಧ
ಅಶುದ್ಧತೆ
ಅಶುದ್ಧ-ನಾ-ಗಿಯೇ
ಅಶುದ್ಧ-ರಾಗು-ವಿರಿ
ಅಶುದ್ಧ-ರಾ-ದರೂ
ಅಶುದ್ಧ-ರೆಂದು
ಅಶುದ್ಧ-ವಾಗಿ-ದೆಯೋ
ಅಶುದ್ಧ-ವಾಗು-ವುದು
ಅಶುದ್ಧ-ವಾದರೆ
ಅಶುದ್ಧಾತ್ಮ-ರಿ-ರು-ವರು
ಅಶುಭಕ್ಕೊ
ಅಶುಭ-ದಿಂದಲೂ
ಅಶುಭ-ವನ್ನು
ಅಶೋಕ
ಅಶ್ರದ್ಧೆ
ಅಶ್ಲೀಲ
ಅಷ್ಟಕ್ಕೆ
ಅಷ್ಟನ್ನು
ಅಷ್ಟರ-ಮಟ್ಟಿಗೆ
ಅಷ್ಟ-ರಲ್ಲಿ
ಅಷ್ಟ-ರಲ್ಲೇ
ಅಷ್ಟ-ರಿಂದಲೇ
ಅಷ್ಟಸಿದ್ಧಿ-ಗಳಿವೆ
ಅಷ್ಟು
ಅಷ್ಟೂ
ಅಷ್ಟೆ
ಅಷ್ಟೇ
ಅಷ್ಟೇಕೆ
ಅಷ್ಟೇನೂ
ಅಷ್ಟೊಂದು
ಅಸಂಖ್ಯ
ಅಸಂಖ್ಯತೆ
ಅಸಂಖ್ಯ-ಲೋಕ-ಗಳಿಂದಲೂ
ಅಸಂಖ್ಯ-ವಾಗಿವೆ
ಅಸಂಖ್ಯವೂ
ಅಸಂಖ್ಯಾತ
ಅಸಂತುಷ್ಟ
ಅಸಂದಿಗ್ಧ-ವಾಗಿ
ಅಸಂಪೂರ್ಣನು
ಅಸಂಬದ್ಧ
ಅಸಂಭವ
ಅಸಂಸ್ಕೃತ
ಅಸಂಸ್ಕೃತರು
ಅಸತ್ಯ
ಅಸತ್ಯ-ವನ್ನೂ
ಅಸತ್ಯ-ವಾಗಿ-ರು-ವುದೋ
ಅಸತ್ಯ-ವಾದ
ಅಸ-ತ್ಯವು
ಅಸತ್ಯವೆ
ಅಸತ್ಯ-ವೆಂದು
ಅಸತ್ಯ-ವೆಂದೂ
ಅಸದೃಶ
ಅಸದೃಶ-ವಾದ
ಅಸಭ್ಯ
ಅಸಮಂಜಸ
ಅಸಮರ್ಥನು
ಅಸಮರ್ಥ-ರಾಗಿ-ದ್ದೇವೆ
ಅಸಮರ್ಪಕ
ಅಸಮರ್ಪಕ-ವಾಗಿ
ಅಸ-ಮಾ-ಧಾನ-ಪಡು-ವಿರಿ
ಅಸ-ಹಿ-ಷ್ಣುತೆ
ಅಸ-ಹಿ-ಷ್ಣು-ತೆಯ
ಅಸಹ್ಯ-ಕರ
ಅಸಾ-ಧಾರಣ
ಅಸಾ-ಧಾರಣ-ವೆಂದರೆ
ಅಸಾಧು
ಅಸಾಧುವೂ
ಅಸಾಧ್ಯ
ಅಸಾಧ್ಯ-ವನ್ನು
ಅಸಾಧ್ಯ-ವಾಗಿ
ಅಸಾಧ್ಯ-ವಾಗು-ತ್ತದೆ
ಅಸಾಧ್ಯ-ವಾದ
ಅಸಾಧ್ಯ-ವಾ-ಯಿತು
ಅಸಾಧ್ಯ-ವೆಂದು
ಅಸಾಧ್ಯ-ವೆಂಬ
ಅಸಾಧ್ಯ-ವೆಂಬು-ದನ್ನು
ಅಸಾಧ್ಯ-ವೆಂಬುದು
ಅಸಾಧ್ಯವೊ
ಅಸೀಮ
ಅಸೀಮ-ವಾದ
ಅಸೂಯಾ
ಅಸೂ-ಯಾ-ಜ್ವಾಲೆ
ಅಸೂಯೆ
ಅಸೂಯೆ-ಪಡು-ವುದು
ಅಸೂಯೆ-ಯನ್ನು
ಅಸ್ತಿತ್ವ
ಅಸ್ತಿ-ತ್ವಕ್ಕೆ
ಅಸ್ತಿತ್ವ-ಗಳನ್ನು
ಅಸ್ತಿ-ತ್ವದ
ಅಸ್ತಿತ್ವ-ದಲ್ಲಿದ್ದ
ಅಸ್ತಿತ್ವ-ದಲ್ಲಿ-ರುವ
ಅಸ್ತಿತ್ವ-ದಷ್ಟೇ
ಅಸ್ತಿತ್ವ-ವನ್ನು
ಅಸ್ತಿತ್ವ-ವನ್ನೇ
ಅಸ್ತಿತ್ವ-ವಿದೆ
ಅಸ್ತಿ-ತ್ವವೂ
ಅಸ್ತಿ-ತ್ವವೇ
ಅಸ್ತಿ-ಭಾರ
ಅಸ್ತಿ-ಭಾರದ
ಅಸ್ತಿರ-ಭಾರ-ವಾಗಿದೆ
ಅಸ್ತೀತ್ಯೈಕೇ
ಅಸ್ಥಿ
ಅಸ್ಪದ-ವಾಗಿದೆ
ಅಸ್ಪಷ್ಟ-ತೆಯೂ
ಅಸ್ಪಷ್ಟ-ವಾಗಿ-ರಲಿ
ಅಸ್ಪಷ್ಟ-ವಾದ
ಅಸ್ಫುಟ
ಅಸ್ವಾ-ಭಾವಿಕ-ವಾದರೂ
ಅಹಂ
ಅಹಂಕಾರ
ಅಹಂಕಾರ-ಗಳೆಂದ-ರೇನು
ಅಹಂಕಾರದ
ಅಹಂಕಾರ-ದಲ್ಲಿದೆ
ಅಹಂಕಾರ-ದಿಂದ
ಅಹಂಕಾರ-ವನ್ನು
ಅಹಿಂಸಾ-ತತ್ತ್ವ-ವನ್ನು
ಅಹಿ-ತವೆ
ಅಹೇತು
ಅಹೇ-ತುಕ
ಅಹೋ
ಆ
ಆಂಗ್ಲ
ಆಂಗ್ಲ-ಭಾಷೆಯ
ಆಂಗ್ಲ-ಭಾಷೆ-ಯನ್ನು
ಆಂಗ್ಲ-ಶಕ್ತಿ
ಆಂಗ್ಲೇಯ
ಆಂಗ್ಲೇ-ಯನ
ಆಂಗ್ಲೇಯ-ನಾ-ಗು-ತ್ತಿ-ರಲಿಲ್ಲ
ಆಂಗ್ಲೇಯ-ನಾದು-ದ-ರಿಂದ
ಆಂಗ್ಲೇಯ-ನಿಗೆ
ಆಂಗ್ಲೇಯನು
ಆಂಗ್ಲೇ-ಯರ
ಆಂಗ್ಲೇಯ-ರನ್ನು
ಆಂಗ್ಲೇಯ-ರಲ್ಲಿ
ಆಂಗ್ಲೇಯ-ರಿಂದ
ಆಂಗ್ಲೇಯ-ರಿಗೂ
ಆಂಗ್ಲೇ-ಯರು
ಆಂಗ್ಲೋ
ಆಂಗ್ಲೋ-ಸಾಕ್ಸನ್ರ
ಆಂಗ್ಲೋ-ಸ್ಯಾ-ಕ್ಸನ್
ಆಂತ-ರಿಕ
ಆಂತ-ರಿಕ-ವಾದುದು
ಆಂತ-ರಿಕ-ವೆಂಬುದೇನೂ
ಆಂತರ್ಮುಖಿ
ಆಂತರ್ಯದ
ಆಂತರ್ಯದಲ್ಲಿರು-ವುದು
ಆಂದೋ-ಲನ-ಗಳನ್ನು
ಆಕರ್ಷಣ
ಆಕರ್ಷಣೀಯ
ಆಕರ್ಷ-ಣೆಗೆ
ಆಕರ್ಷಣೆ-ಯಾಗಿ
ಆಕರ್ಷಿತ
ಆಕರ್ಷಿತ-ರಾಗಿ
ಆಕರ್ಷಿತ-ರಾಗುತ್ತಾರೆ
ಆಕರ್ಷಿತ-ರಾಗು-ತ್ತಿದ್ದರೋ
ಆಕರ್ಷಿಸಿ
ಆಕರ್ಷಿಸಿ-ದು-ದಕ್ಕೆ
ಆಕರ್ಷಿ-ಸುವ
ಆಕಸ್ಮಿಕ
ಆಕಸ್ಮಿಕ-ವಾದು-ದಲ್ಲ
ಆಕಾಂಕ್ಷೆ
ಆಕಾಂಕ್ಷೆಯ
ಆಕಾಂಕ್ಷೆಯೂ
ಆಕಾರ
ಆಕಾರ-ಕ್ಕಾಗಿ
ಆಕಾರ-ಗಳಿವೆ
ಆಕಾರ-ಗಳು
ಆಕಾ-ರವೇ
ಆಕಾಶ
ಆಕಾಶ-ಗಳನ್ನು
ಆಕಾಶ-ಗಳೆ-ರಡೂ
ಆಕಾಶದ
ಆಕಾಶ-ದಂತೆ
ಆಕಾಶ-ದಲ್ಲಿ
ಆಕಾಶ-ದಿಂದ
ಆಕಾಶ-ದಿಂದೇನೂ
ಆಕಾಶ-ವನ್ನು
ಆಕಾಶ-ವನ್ನೆಲ್ಲಾ
ಆಕಾಶ-ವಿರು-ತ್ತದೆ
ಆಕಾಶವು
ಆಕಾಶ-ವೆಂದು
ಆಕಾಶ-ವೆಲ್ಲಾ
ಆಕಾಶವೇ
ಆಕಾಶ-ಸಂಬಂಧದ
ಆಕೆ
ಆಕೆ-ಯನ್ನು
ಆಕ್ರಮಣ
ಆಕ್ರಮ-ಣ-ಕಾರರು
ಆಕ್ರಮ-ಣ-ಕಾರಿ
ಆಕ್ರಮ-ಣ-ಕಾರಿ-ಯಲ್ಲ-ದಿ-ರು-ವುದೇ
ಆಕ್ರಮ-ಣ-ಗಳನ್ನು
ಆಕ್ರಮ-ಣದ
ಆಕ್ರಮ-ಣವು
ಆಕ್ರಮ-ಣವೂ
ಆಕ್ರ-ಮಿಸಿ
ಆಕ್ರ-ಮಿಸಿ-ಕೊಂಡ
ಆಕ್ರಮಿ-ಸಿದೆ
ಆಕ್ರ-ಮಿಸಿ-ಬಿಟ್ಟಿದೆ
ಆಕ್ಷೇಪಣೆಯನ್ನೇ
ಆಗ
ಆಗ-ಕೂಡದು
ಆಗ-ದಿ-ರಲಿ
ಆಗದೇ
ಆಗ-ಬೇಕಾಗಿದೆ
ಆಗ-ಬೇ-ಕಾದ
ಆಗ-ಬೇ-ಕಾದರೆ
ಆಗ-ಬೇಕು
ಆಗ-ಬೇಕೆಂದು
ಆಗ-ಮನದ
ಆಗ-ಮನ-ದಿಂದ
ಆಗ-ಮ-ನವು
ಆಗ-ಮಿಸಿ
ಆಗ-ಮಿಸಿ-ದಾಗ
ಆಗ-ಮಿಸಿ-ದ್ದ-ರಿಂದ
ಆಗರ
ಆಗ-ಲಾ-ರದು
ಆಗ-ಲಾರೆವು
ಆಗಲಿ
ಆಗ-ಲಿಲ್ಲ
ಆಗಲೀ
ಆಗಲೂ
ಆಗಲೆ
ಆಗಲೇ
ಆಗ-ಸ-ದಷ್ಟು
ಆಗಾಗ
ಆಗಿ
ಆಗಿ-ತ್ತು
ಆಗಿದೆ
ಆಗಿ-ದೆ-ಅದೇ
ಆಗಿ-ದೆಯೋ
ಆಗಿದ್ದ
ಆಗಿ-ದ್ದರು
ಆಗಿ-ದ್ದರೂ
ಆಗಿ-ದ್ದಾ-ರೆಂಬುದು
ಆಗಿ-ದ್ದಿತು
ಆಗಿ-ದ್ದೀರಿ
ಆಗಿದ್ದು
ಆಗಿ-ದ್ದೆವು
ಆಗಿ-ದ್ದೇವೆ
ಆಗಿನ
ಆಗಿಯೂ
ಆಗಿ-ರ-ಬಹುದು
ಆಗಿ-ರ-ಬಾ-ರದು
ಆಗಿ-ರ-ಬೇಕು
ಆಗಿ-ರ-ಬೇಕೆಂದು
ಆಗಿ-ರಲಿ
ಆಗಿ-ರ-ಲಿಲ್ಲ
ಆಗಿ-ರು-ತ್ತದೆ
ಆಗಿ-ರುತ್ತಾನೆ
ಆಗಿ-ರುವ
ಆಗಿ-ರು-ವಂತೆ
ಆಗಿ-ರು-ವರು
ಆಗಿ-ರು-ವಾಗ
ಆಗಿ-ರು-ವು-ದಿಲ್ಲ
ಆಗಿ-ರು-ವುದು
ಆಗಿ-ರು-ವುವು
ಆಗಿ-ರು-ವೆವು
ಆಗಿಲ್ಲ
ಆಗಿ-ಲ್ಲದೆ
ಆಗಿವೆ
ಆಗಿ-ಹೋ-ಗಿ-ದ್ದೀರಿ
ಆಗಿ-ಹೋ-ಯಿತು
ಆಗುತ್ತ
ಆಗು-ತ್ತದೆ
ಆಗು-ತ್ತದೆಯೇ
ಆಗುತ್ತಾನೆ
ಆಗು-ತ್ತಿದೆ
ಆಗು-ತ್ತಿ-ರಲಿಲ್ಲ
ಆಗುತ್ತಿ-ರುವ
ಆಗುತ್ತಿರು-ವಿರಿ
ಆಗುತ್ತಿರು-ವುದು
ಆಗು-ತ್ತೀರಿ
ಆಗುವ
ಆಗು-ವಂತಿಲ್ಲ
ಆಗು-ವಂತೆ
ಆಗು-ವಂತೆಯೂ
ಆಗು-ವಿರಿ
ಆಗುವು-ದನ್ನೇ
ಆಗು-ವು-ದಿಲ್ಲ
ಆಗು-ವು-ದಿಲ್ಲವೆ
ಆಗು-ವು-ದಿಲ್ಲ-ವೆಂದು
ಆಗು-ವುದು
ಆಗು-ವುದೇ
ಆಗು-ವುವು
ಆಗು-ವೆವು
ಆಘಾತ
ಆಘಾತ-ಗಳಿಂದ
ಆಘಾ-ತದ
ಆಘಾತ-ದಿಂದ
ಆಘಾತ-ವನ್ನು
ಆಚ-ರಣೆ
ಆಚ-ರಣೆ-ಗಳಿಗೆ
ಆಚ-ರಣೆಗೆ
ಆಚ-ರಣೆಯ
ಆಚ-ರಣೆ-ಯ-ಲ್ಲಿ-ರುವ
ಆಚ-ರಣೆ-ಯ-ಲ್ಲಿರು-ವುದು
ಆಚ-ರಿ-ಸಲು
ಆಚರಿ-ಸಿ-ದ್ದಾರೆ
ಆಚರಿ-ಸುವ
ಆಚರಿ-ಸು-ವರು
ಆಚರಿಸುವಿ-ರೇನು
ಆಚರಿಸು-ವು-ದನ್ನು
ಆಚಾರ
ಆಚಾ-ರಕ್ಕೆ
ಆಚಾರ-ಗಳ
ಆಚಾರ-ಗಳನ್ನು
ಆಚಾರ-ಗಳನ್ನೂ
ಆಚಾರ-ಗಳ-ನ್ನೆಲ್ಲಾ
ಆಚಾರ-ಗಳಿಗೆ
ಆಚಾರ-ಗಳಿವೆ
ಆಚಾರ-ಗಳು
ಆಚಾರ-ಗಳೆಲ್ಲ
ಆಚಾರದ
ಆಚಾರ-ವಂತ-ನಲ್ಲ
ಆಚಾರ-ವನ್ನು
ಆಚಾ-ರವೂ
ಆಚಾರ-ವೆಂದರೆ
ಆಚಾರ-ವ್ಯವ-ಹಾ-ರ-ಗಳನ್ನು
ಆಚಾರ-ವ್ಯವ-ಹಾ-ರ-ಗಳು
ಆಚಾರ-ಶೀಲ-ತೆಯ
ಆಚಾರ-ಶೀಲ-ರ-ನ್ನಾಗಿ
ಆಚಾರ-ಶೀಲ-ರಾ-ದಷ್ಟೂ
ಆಚಾರ-ಶೀಲರು
ಆಚಾರ್ಯ
ಆಚಾರ್ಯರ
ಆಚಾರ್ಯ-ರನ್ನು
ಆಚಾರ್ಯ-ರ-ನ್ನೋ
ಆಚಾರ್ಯ-ರಲ್ಲಿ
ಆಚಾರ್ಯ-ರಾದ
ಆಚಾರ್ಯರು
ಆಚಾರ್ಯ-ರು-ಗಳು
ಆಚಾರ್ಯ-ರೆಲ್ಲ
ಆಚಾರ್ಯರೇ
ಆಚೆ
ಆಚೆ-ಕಡೆ
ಆಚೆಗೆ
ಆಚೆ-ಗೆ-ಸೆಯ-ಬೇಕು
ಆಚೆಯ
ಆಜನ್ಮ-ಸಿದ್ಧ
ಆಜ್ಞಾತ
ಆಜ್ಞಾಧಾ-ರಕ-ರಾಗಲು
ಆಜ್ಞಾ-ಪಿಸಿ
ಆಜ್ಞಾ-ಪಿ-ಸಿ-ದನು
ಆಜ್ಞಾ-ಪಿ-ಸಿ-ದರೂ
ಆಜ್ಞಾ-ಪಿ-ಸುವ-ವರೆ
ಆಜ್ಞೆ
ಆಜ್ಞೇಯತಾ-ವಾದ
ಆಜ್ಞೋಲ್ಲಂಘ-ನೆಗೆ
ಆಟ
ಆಟದ
ಆಟ-ದಿಂದ
ಆಟ-ವಾ-ಡುವ
ಆಡಂಬರ
ಆಡಂಬರ-ವೇನೂ
ಆಡಗಿ-ಕೊಂಡಿ-ರು-ವುವು
ಆಡತೊಡಗಿ-ದರು
ಆಡದೆ
ಆಡಲಿ
ಆಡಳಿತ-ದಲ್ಲಿ
ಆಡಿದ
ಆಡು-ಗಳನ್ನು
ಆಡುತ್ತಿರು-ವಿರಿ
ಆಡುತ್ತೇ-ವೆಯೇ
ಆಡುವ-ವರೋ
ಆಣ್ಣತಮ್ಮಂದಿ-ರೆಂಬ
ಆತ
ಆತಂಕ
ಆತಂಕ-ಗಳನ್ನು
ಆತಂಕ-ಗಳಿವೆ
ಆತಂಕ-ಗಳು
ಆತಂಕ-ಗಳೆಲ್ಲ
ಆತಂಕ-ವನ್ನು
ಆತಂಕ-ವಾಗು-ವುದು
ಆತಂಕ-ವಿ-ಲ್ಲದೆ
ಆತಂಕವೂ
ಆತಂಕ-ವೆಂದು
ಆತನ
ಆತ-ನನ್ನು
ಆತ-ನ-ನ್ನೆ
ಆತ-ನಲ್ಲಿ
ಆತ-ನಲ್ಲಿ-ರುವ
ಆತ-ನಿಂದ
ಆತ-ನಿ-ಗಿಂತ
ಆತ-ನಿಗೆ
ಆತನು
ಆತನೇ
ಆತ-ನೊಬ್ಬನೇ
ಆತು-ರ-ದಿಂದ
ಆತು-ರ-ನಾಗಿ
ಆತ್ಮ
ಆತ್ಮಕ್ಕೂ
ಆತ್ಮಕ್ಕೆ
ಆತ್ಮ-ಗಳ
ಆತ್ಮ-ಗಳಂತೆ
ಆತ್ಮ-ಗಳನ್ನೂ
ಆತ್ಮ-ಗಳಲ್ಲಿ
ಆತ್ಮ-ಗಳ-ಲ್ಲಿಯೂ
ಆತ್ಮ-ಗಳಿಗೆ
ಆತ್ಮ-ಗಳಿಗೆಯೇ
ಆತ್ಮ-ಗಳಿವೆ
ಆತ್ಮ-ಗಳು
ಆತ್ಮ-ಗಳೂ
ಆತ್ಮ-ಗಳೆ-ರಡನ್ನೂ
ಆತ್ಮ-ಚೈತನ್ಯ-ವೆಂಬ
ಆತ್ಮ-ಜ್ಯೋತಿ
ಆತ್ಮ-ಜ್ಯೋ-ತಿಯ
ಆತ್ಮ-ತತ್ತ್ವದ
ಆತ್ಮ-ತೃಪ್ತ-ನಾಗಿ-ರು-ವನು
ಆತ್ಮ-ತೃಪ್ತ-ರಾಗಿ-ರು-ವರೋ
ಆತ್ಮ-ತೃಪ್ತ-ವಾಗಿ
ಆತ್ಮ-ತ್ಯಾಗಕ್ಕೆ
ಆತ್ಮದ
ಆತ್ಮ-ದಲ್ಲಿ
ಆತ್ಮ-ದಲ್ಲಿಯೂ
ಆತ್ಮ-ದಲ್ಲಿ-ರುವ
ಆತ್ಮ-ದಿಂದ
ಆತ್ಮ-ದಿಂದಲೇ
ಆತ್ಮ-ದೊಂದಿಗೆ
ಆತ್ಮನ
ಆತ್ಮ-ನಂತೆ
ಆತ್ಮ-ನನ್ನು
ಆತ್ಮ-ನನ್ನೂ
ಆತ್ಮ-ನಲ್ಲಿ
ಆತ್ಮ-ನ-ಲ್ಲಿದೆ
ಆತ್ಮ-ನ-ಲ್ಲಿ-ರು-ವುವು
ಆತ್ಮ-ನ-ಲ್ಲಿವೆ
ಆತ್ಮ-ನಲ್ಲೂ
ಆತ್ಮ-ನಲ್ಲೇ
ಆತ್ಮ-ನಿಂದ
ಆತ್ಮ-ನಿಗೆ
ಆತ್ಮ-ನಿ-ರು-ವನು
ಆತ್ಮ-ನಿ-ರುವನೇ
ಆತ್ಮ-ನಿರುವಿಕೆ-ಯಲ್ಲಿ
ಆತ್ಮ-ನಿರು-ವುದು
ಆತ್ಮನು
ಆತ್ಮ-ನೆಂದು
ಆತ್ಮನೇ
ಆತ್ಮ-ನ್
ಆತ್ಮನ್ಯೇವ
ಆತ್ಮ-ಪರೀಕ್ಷಾ
ಆತ್ಮ-ಭಾ-ವನೆ
ಆತ್ಮ-ಭಾವ-ನೆ-ಯಲ್ಲಿ
ಆತ್ಮ-ಭಾವ-ವಿದ್ದರೆ
ಆತ್ಮ-ಮ-ಹಿ-ಮೆಯ
ಆತ್ಮ-ವಂಚ-ನೆಯ
ಆತ್ಮ-ವಂಚ-ನೆ-ಯಿಂದ
ಆತ್ಮ-ವನ್ನು
ಆತ್ಮ-ವಲ್ಲ
ಆತ್ಮ-ವಾಗಿಯೂ
ಆತ್ಮ-ವಾಗಿ-ರ-ಬೇಕು
ಆತ್ಮ-ವಾದ
ಆತ್ಮ-ವಿಕಾಸಕ್ಕೆ
ಆತ್ಮ-ವಿ-ಜ್ಞಾನ
ಆತ್ಮ-ವಿದೆ
ಆತ್ಮ-ವಿದೆ-ಯೆಂದೂ
ಆತ್ಮ-ವಿದ್ಯೆ
ಆತ್ಮ-ವಿರು-ವುದು
ಆತ್ಮ-ವಿಲ್ಲ
ಆತ್ಮ-ವಿಶ್ವಾ-ಸ-ವಿದೆ
ಆತ್ಮ-ವಿಶ್ವಾ-ಸ-ವಿ-ರಲಿ
ಆತ್ಮವು
ಆತ್ಮವೂ
ಆತ್ಮ-ವೆಂದ-ರೇನು
ಆತ್ಮ-ವೆಂದರೇ-ನೆಂಬುದು
ಆತ್ಮ-ವೆಂದು
ಆತ್ಮವೇ
ಆತ್ಮ-ಶಕ್ತಿಯ
ಆತ್ಮ-ಶಕ್ತಿ-ಯನ್ನು
ಆತ್ಮ-ಶ್ರದ್ಧೆ
ಆತ್ಮ-ಶ್ರದ್ಧೆಯ
ಆತ್ಮ-ಶ್ರದ್ಧೆ-ಯನ್ನು
ಆತ್ಮ-ಶ್ರದ್ಧೆ-ಯಿತ್ತು
ಆತ್ಮ-ಶ್ರದ್ಧೆಯು
ಆತ್ಮ-ಶ್ರದ್ಧೆ-ಯುಳ್ಳ
ಆತ್ಮ-ಶ್ರದ್ಧೆಯೇ
ಆತ್ಮಶ್ರೀ
ಆತ್ಮ-ಸಾಕ್ಷಾತ್ಕಾರ
ಆತ್ಮ-ಸಾಕ್ಷಾತ್ಕಾರಕ್ಕೆ
ಆತ್ಮ-ಸಾಕ್ಷಾತ್ಕಾರ-ದಿಂದಲೂ
ಆತ್ಮ-ಸ್ವ-ರೂಪವು
ಆತ್ಮ-ಹತ್ಯೆ
ಆತ್ಮ-ಹತ್ಯೆಗೆ
ಆತ್ಮ-ಹತ್ಯೆ-ಯನ್ನು
ಆತ್ಮಾನಂ
ಆತ್ಮಾ-ನು-ಭೂತಿ
ಆತ್ಮೀಯತೆ-ಯಿಂದ
ಆತ್ಮೀಯ-ತೆಯೂ
ಆತ್ಮೀಯ-ವಾಗಿ
ಆತ್ಮೀ-ಯವೂ
ಆತ್ಯಂತಿಕ
ಆತ್ಯ-ಹತ್ಯೆ
ಆದ
ಆದಂತಹ-ಯುವ-ಕರು
ಆದಂತಾಗಿ
ಆದಂತೆಯೇ
ಆದ-ಕಾರಣ
ಆದ-ಕಾರ-ಣವೆ
ಆದ-ಕಾರ-ಣವೇ
ಆದ-ದೀತ
ಆದ-ದ್ದಲ್ಲ
ಆದ-ದ್ದಾಗಿ-ರ-ಲಾ-ರದು
ಆದದ್ದು
ಆದ-ಮೇಲೆ
ಆದರ
ಆದ-ರ-ಕ್ಕೆಲ್ಲ
ಆದ-ರ-ಣೆಗೆ
ಆದ-ರದ
ಆದ-ರ-ದಿಂದ
ಆದ-ರ-ವಿದೆ
ಆದ-ರಿಂದ
ಆದರೂ
ಆದರೆ
ಆದ-ರೇನು
ಆದರ್ಶ
ಆದರ್ಶ-ಕ್ಕಾಗಿ
ಆದರ್ಶ-ಕ್ಕಿಂತ
ಆದರ್ಶಕ್ಕೆ
ಆದರ್ಶ-ಗಳನ್ನು
ಆದರ್ಶ-ಗಳ-ನ್ನೆಲ್ಲಾ
ಆದರ್ಶ-ಗಳಾದ
ಆದರ್ಶ-ಗಳಿಂದಲೇ
ಆದರ್ಶ-ಗಳಿವೆ
ಆದರ್ಶ-ಗಳು
ಆದರ್ಶ-ಗಳೆ-ಲ್ಲವೂ
ಆದರ್ಶ-ಜಗ-ತ್ತಿನ
ಆದರ್ಶದ
ಆದರ್ಶ-ದಿಂದ
ಆದರ್ಶ-ದಿಂದಲೂ
ಆದರ್ಶ-ದೆ-ಡೆಗೆ
ಆದರ್ಶ-ಪತ್ನಿ
ಆದರ್ಶ-ಪ್ರಾಯ-ವಾದ
ಆದರ್ಶ-ಮಯ
ಆದರ್ಶ-ಮೂರ್ತಿ
ಆದರ್ಶ-ಮೂರ್ತಿಯೂ
ಆದರ್ಶ-ರೂಪ
ಆದರ್ಶ-ವನ್ನು
ಆದರ್ಶ-ವಾಗಲಿ
ಆದರ್ಶ-ವಾಗಿ-ಟ್ಟು-ಕೊಂಡಿ-ರು-ವು-ದೊಂದು
ಆದರ್ಶ-ವಾಗಿದೆ
ಆದರ್ಶ-ವಾಗಿ-ರು-ವನು
ಆದರ್ಶ-ವಿದು
ಆದರ್ಶ-ವಿದೆ
ಆದರ್ಶ-ವಿದ್ದರೆ
ಆದರ್ಶ-ವಿರು-ವು-ದನ್ನು
ಆದರ್ಶವು
ಆದರ್ಶವೂ
ಆದರ್ಶ-ವೆಂದರೆ
ಆದರ್ಶ-ವೆಂದು
ಆದರ್ಶ-ವೆ-ನ್ನು-ವುದು
ಆದರ್ಶವೇ
ಆದರ್ಶ-ವೇನು
ಆದರ್ಶ-ವೇ-ನೆಂದರೆ
ಆದರ್ಶ-ವೇನೋ
ಆದ-ವರು
ಆದಾಗ
ಆದಾ-ಗಲೇ
ಆದಾ-ರದ
ಆದಿ
ಆದಿ-ಅಂತ್ಯ
ಆದಿ-ಪ್ರಾಣ-ದೊಂದಿಗೆ
ಆದಿ-ಯಲ್ಲಿ
ಆದಿ-ಯ-ಲ್ಲಿಯೇ
ಆದಿ-ಯಿದೆ
ಆದಿ-ರೂಪ
ಆದಿ-ಸ್ಥಿತಿಗೆ
ಆದಿ-ಸ್ಥಿತಿ-ಯಲ್ಲಿ
ಆದು
ಆದು-ದ-ರಿಂದ
ಆದು-ದ-ರಿಂದಲೇ
ಆದುದು
ಆದುವು
ಆದು-ವೆಂದು
ಆದು-ವೆಂದೂ
ಆದೇಶ-ವನ್ನು
ಆದ್ದ-ರಿಂದ
ಆದ್ದ-ರಿಂದಲೆ
ಆದ್ದ-ರಿಂದಲೇ
ಆದ್ಯ
ಆದ್ಯಂತ
ಆದ್ಯಂತ-ಗಳಿಲ್ಲ
ಆದ್ಯಂತ-ರ-ಹಿ-ತ-ವಾದ
ಆದ್ಯಂತ-ರ-ಹಿ-ತ-ವಾದು
ಆದ್ಯಂತವು
ಆದ್ವೈ-ತದ
ಆದ್ವೈ-ತ-ವೆಂದರೆ
ಆಧರಿ
ಆಧರಿ-ಸಿದೆ
ಆಧರಿ-ಸಿದ್ದು
ಆಧಾರ
ಆಧಾರ-ಗಳನ್ನು
ಆಧಾ-ರದ
ಆಧಾರ-ದಿಂದಲೇ
ಆಧಾರ-ಭೂತ-ವಾಗಿ-ರ-ಬೇಕು
ಆಧಾರ-ವಾಗಿ
ಆಧಾರ-ವಾಗಿದೆ
ಆಧಾರ-ವಾಗಿದ್ದ
ಆಧಾರ-ವಾಗಿ-ರುವ
ಆಧಾರ-ವಾದ
ಆಧಾರ-ವಾ-ವುದೆಂಬು-ದನ್ನೂ
ಆಧಾರ-ವಿದೆ
ಆಧಾರ-ವಿ-ಲ್ಲದೆ
ಆಧಾ-ರವೂ
ಆಧಾರ-ವೆಂದರೆ
ಆಧಾರ-ವೆಂಬು-ದನ್ನು
ಆಧಾ-ರವೇ
ಆಧಿಪತ್ಯ-ದಲ್ಲಿ
ಆಧುನಿಕ
ಆಧುನಿಕ-ರಂತೆ
ಆಧುನಿಕ-ರ-ನ್ನಾಗಿ
ಆಧುನಿಕರು
ಆಧುನಿಕ-ವಾದವು
ಆಧ್ಯಾತ್ಮ
ಆಧ್ಯಾ-ತ್ಮ-ಗಗನ-ದಿಂದ
ಆಧ್ಯಾ-ತ್ಮ-ಜೀವ-ನವೇ
ಆಧ್ಯಾ-ತ್ಮ-ವನ್ನು
ಆಧ್ಯಾ-ತ್ಮಿಕ
ಆಧ್ಯಾ-ತ್ಮಿಕಕ್ಕೆ
ಆಧ್ಯಾ-ತ್ಮಿಕತೆ
ಆಧ್ಯಾ-ತ್ಮಿಕ-ತೆ-ಗಳಿಗೆ
ಆಧ್ಯಾ-ತ್ಮಿಕ-ತೆ-ಗಿಂತಲೂ
ಆಧ್ಯಾ-ತ್ಮಿಕ-ತೆಗೆ
ಆಧ್ಯಾ-ತ್ಮಿಕ-ತೆಯ
ಆಧ್ಯಾ-ತ್ಮಿಕ-ತೆ-ಯನ್ನು
ಆಧ್ಯಾ-ತ್ಮಿಕ-ತೆಯು
ಆಧ್ಯಾ-ತ್ಮಿಕ-ವಾಗಲೀ
ಆಧ್ಯಾ-ತ್ಮಿಕ-ವಾಗಿ
ಆಧ್ಯಾ-ತ್ಮಿಕ-ವಾದವು
ಆಧ್ಯಾ-ರೋ-ಪ-ವಾಗಿವೆ
ಆನಂತ-ಬ್ರಹ್ಮವು
ಆನಂದ
ಆನಂದಕ್ಕೆ
ಆನಂದ-ಗಳ
ಆನಂದದ
ಆನಂದ-ದಿಂದಲೂ
ಆನಂದ-ಪ-ಡುವ
ಆನಂದ-ವನ್ನು
ಆನಂದ-ವಾಗಿದೆ
ಆನಂದ-ವಾಗು-ತ್ತಿದೆ
ಆನಂದ-ವಾಗು-ವುದು
ಆನಂದ-ವಿದೆಯೊ
ಆನಂದವು
ಆನಂದ-ವುಂಟಾ-ಯಿತು
ಆನಂದವೇ
ಆನಂದಾಶ್ರು
ಆನಂದಿಸ-ಬಲ್ಲ-ವ-ನಾಗಿ-ದ್ದೇನೆ
ಆನಂದಿಸುತ್ತೇವೆ
ಆನಂದಿ-ಸು-ವನು
ಆನಂದಿ-ಸು-ವುದೇ
ಆನಿಬೆಸೆಂಟ-ರಿಗೆ
ಆನೀದ-ವಾತಂ
ಆನುವಂಶಿಕ
ಆನುವಂಶಿಕ-ವಾಗಿ
ಆನುವಂಶೀಯತೆ
ಆಪಾದ-ನೆ-ಗಳನ್ನು
ಆಪೂರಣ
ಆಪೂರಣ-ದಿಂದ
ಆಪೂರ-ಣವೇ
ಆಪ್ತ
ಆಪ್ತೇ-ಷ್ಟರ
ಆಫ್ಘಾನಿ-ಸ್ಥಾನ
ಆಫ್ರಿಕಾ
ಆಫ್ರಿಕಾ-ದಲ್ಲಿ-ರುವ
ಆಫ್ರಿಕಾ-ದೇಶದ
ಆಬಾ-ಲವೃದ್ದ
ಆಬಾ-ಲ-ವೃದ್ಧ
ಆಬಾ-ಲ-ವೃದ್ಧರ
ಆಮ-ದಾದ
ಆಮದು
ಆಮೂಲಾಗ್ರ
ಆಮೂಲಾಗ್ರ-ವಾಗಿ
ಆಮೃ-ತತ್ವ-ವನ್ನು
ಆಮೇಲೆ
ಆಮೋದ
ಆಯಸ್ಸು
ಆಯಾ
ಆಯಾಯ
ಆಯಾ-ಯ-ಧರ್ಮಾ-ನು-ಯಾಯಿ
ಆಯಾಯಾ
ಆಯಿತು
ಆಯುಷ್ಯ-ವನ್ನೆಲ್ಲ
ಆಯುಷ್ಯವು
ಆಯುಸ್ಸು
ಆಯ್ಕೆ
ಆಯ್ಕೆ-ಯಲ್ಲ
ಆಯ್ಕೆಯೆ
ಆರಂಭ-ದಲ್ಲಿ
ಆರಂಭ-ಮಾಡಿ-ರು-ವರು
ಆರಂಭ-ವನ್ನು
ಆರಂಭ-ವಾಗಿ-ದ್ದರೆ
ಆರಂಭ-ವಾಗಿಲ್ಲ
ಆರಂಭ-ವಾ-ಯಿತು
ಆರಂಭ-ವಾ-ಯಿತೆಂದೂ
ಆರಣ್ಯಕ
ಆರದು
ಆರದೆ
ಆರಾಧನೆ
ಆರಾಧ-ನೆಗೆ
ಆರಾಧ-ನೆಯ
ಆರಾಧ-ನೆ-ಯನ್ನು
ಆರಾಧ-ನೆ-ಯಿಂದ
ಆರಾಧಿ-ಸಲಿ
ಆರಾಧಿ-ಸಲು
ಆರಾಧಿಸಿ-ದಂತೆ
ಆರಾಧಿಸಿ-ದರೂ
ಆರಾಧಿ-ಸಿದರೆ
ಆರಾಧಿ-ಸುವ
ಆರಾಧಿ-ಸು-ವನು
ಆರಾಧ್ಯನು
ಆರಿಸಿ-ಕೊಂಡದ್ದು
ಆರಿಸಿ-ಕೊಂಡರು
ಆರಿಸಿ-ಕೊಂಡಿ-ದ್ದಾರೆ
ಆರಿಸಿ-ಕೊಂಡಿ-ರುವ
ಆರಿ-ಸಿ-ಕೊಂಡು
ಆರಿಸಿ-ಕೊಂಡೆವು
ಆರಿಸಿ-ಕೊಳ್ಳ-ಬೇಕಾಗಿದೆ
ಆರಿಸಿ-ಕೊಳ್ಳ-ಬೇಕು
ಆರಿಸಿ-ಕೊಳ್ಳಿ
ಆರಿಸಿ-ಕೊಳ್ಳು-ವರು
ಆರಿಸಿ-ಕೊಳ್ಳೋಣ
ಆರು
ಆರುಂಧ-ತಿಯ
ಆರುಂಧತೀ
ಆರೂಢರಾಗುತ್ತಾರೆ
ಆರೇ
ಆರೈ-ಸಿದ-ವ-ರಾರು
ಆರೋಗ್ಯ
ಆರೋಗ್ಯ-ಕರ-ವಾಗಿ-ತ್ತು
ಆರೋಗ್ಯ-ಕರ-ವಾದ
ಆರೋಗ್ಯ-ವಂತನು
ಆರೋಗ್ಯ-ವನ್ನು
ಆರೋಗ್ಯ-ವನ್ನೂ
ಆರೋಗ್ಯ-ವಾಗಿ-ರ-ಬೇಕು
ಆರೋಗ್ಯ-ವಾಗಿ-ರು-ವುದು
ಆರೋಗ್ಯ-ಶಾಲಿ-ಯಾಗಿ
ಆರೋಪ-ಣೆಯ
ಆರೋ-ಪಿ-ಸದೆ
ಆರೋ-ಪಿ-ಸ-ಬೇಕು
ಆರೋ-ಪಿ-ಸಲು
ಆರೋ-ಪಿ-ಸಿ-ಕೊಳ್ಳು-ತ್ತೇವೆ
ಆರೋ-ಪಿ-ಸಿ-ದ್ದಾರೆ
ಆರೋ-ಪಿ-ಸುತ್ತಾರೆ
ಆರೋ-ಪಿ-ಸುವ
ಆರೋ-ಪಿ-ಸು-ವರು
ಆರೋ-ಪಿ-ಸು-ವು-ದಕ್ಕೆ
ಆರೋ-ಪಿ-ಸುವು-ದ-ರಲ್ಲಿ
ಆರೋ-ಪಿ-ಸು-ವುದು
ಆರ್ಕ್
ಆರ್ತ-ನಾದ
ಆರ್ಥಿಕ
ಆರ್ದಶ-ವೆಂಬುದು
ಆರ್ದ್ರ-ವಾಗಿ
ಆರ್ಭಟಿ-ಸಿದರು
ಆರ್ಮೇನಿಯಾ
ಆರ್ಯ
ಆರ್ಯ-ಕುಲದ
ಆರ್ಯ-ಧರ್ಮ
ಆರ್ಯ-ಧರ್ಮದ
ಆರ್ಯನ
ಆರ್ಯ-ನಲ್ಲ
ಆರ್ಯನು
ಆರ್ಯ-ಪುತ್ರರೇ
ಆರ್ಯ-ಭೂಮಿ-ಯೊಂದ-ರಲ್ಲಿ
ಆರ್ಯ-ಮಾ-ತೆಯ
ಆರ್ಯರ
ಆರ್ಯ-ರಂತೆ
ಆರ್ಯ-ರನ್ನು
ಆರ್ಯ-ರಾಗಲೀ
ಆರ್ಯ-ರಿ-ಗಿಂತ
ಆರ್ಯ-ರಿಗೆ
ಆರ್ಯರು
ಆರ್ಯರೂ
ಆರ್ಯ-ರೆಂದೂ
ಆರ್ಯ-ರೆಲ್ಲ
ಆರ್ಯಾ-ವ-ತಕ್ಕೆ
ಆರ್ಯಾ-ವರ್ತ-ದಲ್ಲಿ
ಆರ್ಯಾ-ವರ್ತ-ವಾಗು-ವುದು
ಆರ್ಯಾ-ವರ್ತವು
ಆರ್ಯೇ-ತರ-ರಿಗೆ
ಆರ್ಹರು
ಆಲಂಗಿ-ಸಿದನು
ಆಲಂಗಿಸು-ವುದು
ಆಲಿಂಗಿ-ಸಲು
ಆಲಿಂಗಿಸಿ
ಆಲಿಸಿ
ಆಲಿಸಿ-ದರೆ
ಆಲಿಸಿ-ದಿರ-ಲ್ಲವೆ
ಆಲಿಸಿರಿ
ಆಲಿಸಿ-ರು-ವೆವು
ಆಲಿಸು-ವೆವು
ಆಲೆದಾಡುತ್ತಿ-ರು-ವರು
ಆಲೋಚನಾ
ಆಲೋಚನಾ-ಪರ-ರಾ-ಗು-ವು-ದಕ್ಕೆ
ಆಲೋಚನಾ-ಶೀಲರು
ಆಲೋಚನೆ
ಆಲೋಚ-ನೆ-ಗಳನ್ನು
ಆಲೋಚ-ನೆ-ಗಳಲ್ಲಿಲ್ಲ
ಆಲೋಚ-ನೆ-ಗಳು
ಆಲೋಚ-ನೆಗೆ
ಆಲೋಚ-ನೆಯ
ಆಲೋಚ-ನೆ-ಯನ್ನು
ಆಲೋಚ-ನೆ-ಯ-ನ್ನೆಲ್ಲ
ಆಲೋಚ-ನೆ-ಯಿಂದ
ಆಲೋಚ-ನೆಯು
ಆಲೋಚ-ನೆಯೂ
ಆಲೋಚಿಸ-ಬೇಕಾಗಿದೆ
ಆಲೋಚಿಸ-ಬೇಕು
ಆಲೋಚಿಸ-ಬೇಡಿ
ಆಲೋಚಿಸ-ಲಾರದೋ
ಆಲೋಚಿ-ಸ-ಲಾರೆವು
ಆಲೋಚಿಸಿ
ಆಲೋಚಿಸಿ-ದಂತೆ
ಆಲೋಚಿಸಿ-ದರು
ಆಲೋಚಿಸಿ-ದರೆ
ಆಲೋಚಿಸಿದೆ
ಆಲೋಚಿಸಿದ್ದೆ
ಆಲೋಚಿ-ಸು-ತ್ತಿದ್ದ
ಆಲೋಚಿಸುತ್ತಿ-ರುವನೋ
ಆಲೋಚಿಸುತ್ತೀರೋ
ಆಲೋಚಿಸುತ್ತೇವೆ
ಆಲೋಚಿ-ಸುವ
ಆಲೋಚಿ-ಸು-ವಂತೆ
ಆಲೋಚಿ-ಸುವನು
ಆಲೋಚಿ-ಸುವರು
ಆಲೋಚಿಸು-ವಾಗ
ಆಲೋಚಿಸು-ವಿರಿ
ಆಲೋಚಿ-ಸು-ವು-ದಕ್ಕೆ
ಆಲೋಚಿಸು-ವುದು
ಆಲೋಚಿ-ಸು-ವುದೂ
ಆಲೋಚಿ-ಸು-ವುದೇ
ಆಲೋಚಿಸು-ವೆವು
ಆಲ್ಮೋರಕೆ
ಆಲ್ಮೋ-ರಕ್ಕೆ
ಆಲ್ಮೋ-ರದ
ಆಲ್ಮೋರ-ದಲ್ಲಿ
ಆಲ್ಲ-ಗಳೆ-ದಿ-ರು-ವರು
ಆಳ
ಆಳಕ್ಕಾ
ಆಳಕ್ಕೆ
ಆಳ-ಬಲ್ಲನೇ
ಆಳ-ಬಹುದು
ಆಳ-ರ-ಸರ
ಆಳಲು
ಆಳ-ವಾಗಿ
ಆಳ-ವಾದ
ಆಳವೂ
ಆಳಿ
ಆಳಿ-ತೆಂಬುದು
ಆಳಿ-ದೆಡೆ-ಯಲ್ಲಿ
ಆಳು
ಆಳು-ಗಳಿ-ದ್ದರು
ಆಳು-ತ್ತವೆ
ಆಳು-ತ್ತಿದ್ದರು
ಆಳು-ತ್ತಿದ್ದ-ರೆಂಬುದು
ಆಳು-ತ್ತಿದ್ದಾಗ್ಯೂ
ಆಳು-ತ್ತಿ-ರು-ವನು
ಆಳು-ತ್ತಿರು-ವುದು
ಆಳು-ತ್ತಿವೆ
ಆಳುವ
ಆಳು-ವನು
ಆಳು-ವರು
ಆಳು-ವ-ವರು
ಆಳು-ವು-ದಕ್ಕೆ
ಆಳು-ವು-ದನ್ನೂ
ಆಳು-ವು-ದ-ರಿಂದ
ಆಳು-ವುದು
ಆಳ್ವಿ-ಕೆಗೂ
ಆಳ್ವಿ-ಕೆಯ
ಆಳ್ವಿಕೆ-ಯಲ್ಲಿ
ಆಳ್ವಿಕೆಯಿಂದಾದ
ಆಳ್ವಿಕೆ-ಯೆಲ್ಲಾ
ಆವನ
ಆವ-ನನ್ನು
ಆವನು
ಆವರಣ
ಆವರ-ಣ-ದಿಂದುಂಟಾದು-ದೆಂದೂ
ಆವರ-ಣವೇ
ಆವ-ರಿಗೆ
ಆವರಿಸ-ಬಲ್ಲ
ಆವರಿಸ-ಬೇಕು
ಆವ-ರಿಸಿ-ಕೊಂಡಿದೆ
ಆವರಿ-ಸಿದ
ಆವ-ರಿಸಿ-ದಂತಿತ್ತು
ಆವ-ರಿಸಿ-ರು-ವುದು
ಆವ-ರಿಸಿ-ರು-ವುವು
ಆವರಿ-ಸಿವೆ
ಆವರಿ-ಸುತ್ತದೆ
ಆವ-ಶ್ಯಕ
ಆವ-ಶ್ಯಕತೆ
ಆವ-ಶ್ಯಕ-ತೆ-ಗಳನ್ನು
ಆವ-ಶ್ಯಕ-ತೆ-ಗಾಗಿ
ಆವ-ಶ್ಯಕ-ತೆಗೆ
ಆವ-ಶ್ಯಕ-ತೆ-ಯನ್ನು
ಆವ-ಶ್ಯಕ-ತೆ-ಯಿದೆ
ಆವ-ಶ್ಯಕ-ತೆ-ಯಿ-ಲ್ಲದೆ
ಆವ-ಶ್ಯಕ-ತೆಯೂ
ಆವ-ಶ್ಯಕ-ವಾಗಿ
ಆವ-ಶ್ಯಕ-ವಾಗಿ-ತ್ತು
ಆವ-ಶ್ಯಕ-ವಾಗಿದೆ
ಆವ-ಶ್ಯಕ-ವಾಗಿ-ದ್ದರೆ
ಆವ-ಶ್ಯಕ-ವಾಗಿವೆ
ಆವ-ಶ್ಯಕ-ವಾಗು-ವು-ದಿಲ್ಲವೋ
ಆವ-ಶ್ಯಕ-ವಾದ
ಆವ-ಶ್ಯಕ-ವಾದರೆ
ಆವ-ಶ್ಯಕ-ವಾದಲ್ಲಿ
ಆವ-ಶ್ಯಕ-ವಾದಾ-ಗೆಲ್ಲಾ
ಆವ-ಶ್ಯಕ-ವಾದುವು
ಆವ-ಶ್ಯಕ-ವಿದೆ
ಆವ-ಶ್ಯಕ-ವೆಂದು
ಆವ-ಶ್ಯಕ-ವೆಂಬು-ದನ್ನು
ಆವ-ಶ್ಯಕವೋ
ಆವಾಗಾ-ವಾಗ
ಆವಿರ್ಭವಿಸಿ
ಆವಿರ್ಭವಿ-ಸುತ್ತದೆ
ಆವಿರ್ಭಾವ
ಆವಿರ್ಭೂತ-ವಾ-ಯಿತೊ
ಆವಿ-ಷ್ಕಾರ
ಆವಿ-ಷ್ಕಾರ-ಗಳ
ಆವಿ-ಷ್ಕಾರ-ಗಳಿಂದಾಗಿ
ಆವಿ-ಷ್ಕಾರ-ಗಳೆಲ್ಲ
ಆವಿ-ಷ್ಕಾರ-ವಾದರೆ
ಆವು-ಗಳನ್ನು
ಆವೃತ-ವಾಗಿದೆ
ಆವೃತ-ವಾಗು-ವುದು
ಆವೃತ-ವಾದಂತೆ
ಆವೃತ್ತಿ
ಆವೇಗ-ವನ್ನೆಲ್ಲಾ
ಆಶಯ
ಆಶಾಕಿರ-ಣವು
ಆಶಾ-ಜ-ನಕ
ಆಶಾ-ಜನ-ಕ-ವಾಗಿದೆ
ಆಶಾ-ಜನ-ಕ-ವಾಗಿಲ್ಲ
ಆಶಾ-ಜನ-ಕ-ವಾಗಿ-ಲ್ಲದೆ
ಆಶಾ-ದಾಯ-ಕ-ವಾಗಿದೆ
ಆಶಾ-ಪೂರ್ಣ
ಆಶಾ-ಪ್ರ-ವಾ-ಹ-ಗಳು
ಆಶಾ-ವಾದಿ
ಆಶಿಷ್ಠ
ಆಶಿ-ಷ್ಠರೂ
ಆಶಿ-ಸದೆ
ಆಶಿಸ-ಬಹು-ದಾ-ಗಿದೆ
ಆಶಿ-ಸಿದರು
ಆಶಿ-ಸಿದ್ದರು
ಆಶಿಸು-ತ್ತೇನೆ
ಆಶಿಸುತ್ತೇವೆ
ಆಶಿ-ಸು-ವು-ದಿಲ್ಲ
ಆಶಿಸು-ವುದು
ಆಶಿಸು-ವೆನು
ಆಶೀರ್ವದಿ-ಸಲಿ
ಆಶೀರ್ವದಿಸಿ
ಆಶೀರ್ವದಿಸಿ-ದರು
ಆಶೀರ್ವದಿಸುತ್ತಾನೆ
ಆಶೀರ್ವದಿಸುತ್ತಾರೆ
ಆಶೀರ್ವಾದ
ಆಶೀರ್ವಾದ-ಗಳು
ಆಶೀರ್ವಾದದ
ಆಶೀರ್ವಾದ-ವನ್ನು
ಆಶುದ್ಧ್ದ
ಆಶೆಯು
ಆಶೋತ್ತರ-ಗಳಿಗೂ
ಆಶೋತ್ತರ-ಗಳು
ಆಶೋತ್ತರ-ಗಳೊಂದಿಗೆ
ಆಶ್ಚರ್ಯ
ಆಶ್ಚರ್ಯ-ಕರ-ವಾಗಿದೆ
ಆಶ್ಚರ್ಯ-ಕರ-ವಾದ
ಆಶ್ಚರ್ಯದ
ಆಶ್ಚರ್ಯ-ಪಡುತ್ತೇವೆ
ಆಶ್ಚರ್ಯ-ವಾಗ-ಬಹುದು
ಆಶ್ಚರ್ಯ-ವಾಗುತ್ತಿತ್ತು
ಆಶ್ಚರ್ಯ-ವಾಗುವ-ಷ್ಟರ-ಮಟ್ಟಿನ
ಆಶ್ಚರ್ಯ-ವಾಗು-ವು-ದಿಲ್ಲ
ಆಶ್ಚರ್ಯ-ವಾಗು-ವುದು
ಆಶ್ಚರ್ಯ-ವಾದುದೇ-ನೆಂದರೆ
ಆಶ್ಚರ್ಯ-ವಿಲ್ಲ
ಆಶ್ಚರ್ಯವೂ
ಆಶ್ಚರ್ಯ-ವೇ-ನಿದೆ
ಆಶ್ಚರ್ಯ-ವೇ-ನಿಲ್ಲ
ಆಶ್ರಮ
ಆಶ್ರಮ-ವನ್ನು
ಆಶ್ರಯ
ಆಶ್ರ-ಯಕ್ಕೆ
ಆಶ್ರಯ-ದೋಷ
ಆಶ್ರಯ-ವಾಗಿದೆ
ಆಶ್ರಯಿಸ-ಬೇಕು
ಆಶ್ರಯಿಸಿ
ಆಶ್ರಯಿಸಿ-ದ್ದರು
ಆಶ್ರಯಿಸಿ-ರು-ವುದು
ಆಶ್ರಯಿಸುತ್ತಾ-ನೆಯೋ
ಆಶ್ರಯಿ-ಸುವ
ಆಶ್ವಾ-ಸ-ನೆ-ಯನ್ನೂ
ಆಷಾಢ-ಭೂತಿ-ಗಳು
ಆಸಕ್ತ-ನಾಗಿ
ಆಸಕ್ತಿ-ಪೂರ್ಣ
ಆಸಕ್ತಿಯ
ಆಸಕ್ತಿ-ಯನ್ನು
ಆಸಕ್ತಿ-ಯಿಂದ
ಆಸಕ್ತಿ-ಯಿಂದಾಗಿ
ಆಸಕ್ತಿಯಿ-ರು-ವುದೊ
ಆಸಕ್ತಿ-ಯುಳ್ಳ-ವ-ರಾಗಿ-ದ್ದಾರೆ
ಆಸಕ್ತಿಯೇ
ಆಸರೆ
ಆಸರೆಯ
ಆಸರೆಯೂ
ಆಸೀ-ತ್
ಆಸುರೀ
ಆಸೆ
ಆಸೆಗೂ
ಆಸೆ-ಪ-ಡುವ
ಆಸೆಯ
ಆಸೆ-ಯನ್ನು
ಆಸೆ-ಯಾ-ಯಿತು
ಆಸೆಯು
ಆಸೆಯೂ
ಆಸೆ-ಯೇನೋ
ಆಸ್ಟ್ರಿಯ-ದಲ್ಲಿದ್ದ
ಆಸ್ತಿ
ಆಸ್ತಿಕ
ಆಸ್ತಿ-ಕರೂ
ಆಸ್ತಿ-ಗಳಲ್ಲಿ
ಆಸ್ತಿ-ಗಾಗಿ
ಆಸ್ತಿ-ಯಾಗದೆ
ಆಸ್ತಿ-ಯಾದ
ಆಸ್ತಿ-ಯೆಂದರೆ
ಆಸ್ತಿಯೇ
ಆಸ್ಪತ್ರೆ
ಆಸ್ಪತ್ರೆಗೆ
ಆಸ್ಪತ್ರೆ-ಯ-ನ್ನಾಗಿ
ಆಸ್ಪತ್ರೆ-ಯನ್ನು
ಆಸ್ಪತ್ರೆ-ಯಲ್ಲಿ
ಆಹಾರ
ಆಹಾ-ರ-ಕ್ಕಾಗಿ
ಆಹಾ-ರಕ್ಕೂ
ಆಹಾ-ರಕ್ಕೆ
ಆಹಾ-ರದ
ಆಹಾ-ರ-ದಿಂದ
ಆಹಾ-ರ-ದಿಂದಲೇ
ಆಹಾ-ರ-ವನ್ನು
ಆಹಾ-ರ-ವನ್ನೂ
ಆಹಾ-ರ-ವಾಗಲಿ
ಆಹಾ-ರವು
ಆಹಾ-ರ-ವೆಂದರೆ
ಆಹಾ-ರ-ಶುದ್ಧೌ
ಆಹ್ವಾನ
ಆಹ್ವಾ-ನ-ಗಳನ್ನು
ಇಂಗ್ಲಿಷ-ರಲ್ಲ-ವೆಂಬ
ಇಂಗ್ಲಿಷ-ರಿಗೆ
ಇಂಗ್ಲಿಷರು
ಇಂಗ್ಲಿಷ-ರೊಂದಿಗೆ
ಇಂಗ್ಲಿಷಿನ
ಇಂಗ್ಲಿಷಿ-ನಲ್ಲಿ
ಇಂಗ್ಲಿಷಿನ-ವ-ರಿಗೂ
ಇಂಗ್ಲಿಷ್
ಇಂಗ್ಲಿಷ್ನಲ್ಲಿ
ಇಂಗ್ಲೀಷಿಗೆ
ಇಂಗ್ಲೀಷಿ-ನ-ವನು
ಇಂಗ್ಲೀಷ್
ಇಂಗ್ಲೀಷ್ನಲ್ಲಿ
ಇಂಗ್ಲೆಂಡನ್ನು
ಇಂಗ್ಲೆಂಡಿನ
ಇಂಗ್ಲೆಂಡಿನಂಥ
ಇಂಗ್ಲೆಂಡಿ-ನಲ್ಲಿ
ಇಂಗ್ಲೆಂಡು
ಇಂಗ್ಲೆಂಡು-ಗಳಲ್ಲಿ
ಇಂಗ್ಲೆಂಡ್
ಇಂಡಿ-ಯನ
ಇಂಡಿಯಾ
ಇಂತಹ
ಇಂತಹ-ವನು
ಇಂತಹ-ವನೆ
ಇಂತಹ-ವರು
ಇಂತ-ಹವು
ಇಂತಹು-ದನ್ನು
ಇಂತಿ-ರಲು
ಇಂಥ
ಇಂಥ-ವನ್ನೆಲ್ಲ
ಇಂದ
ಇಂದಿಗೂ
ಇಂದಿಗೆ
ಇಂದಿನ
ಇಂದಿನ-ವ-ರಾಗಲೀ
ಇಂದಿನ-ವರೆಗೂ
ಇಂದಿನ-ವ-ರೆಗೆ
ಇಂದು
ಇಂದೂ
ಇಂದೇ
ಇಂದೊ
ಇಂದೋ
ಇಂದ್ರ
ಇಂದ್ರ-ಜಾಲಿ-ಕನ
ಇಂದ್ರ-ನಾಗು-ವನು
ಇಂದ್ರ-ನಾದ
ಇಂದ್ರ-ನಿಗಿ-ರು-ವಂತೆ
ಇಂದ್ರನು
ಇಂದ್ರ-ರಿ-ರು-ವರು
ಇಂದ್ರಾದಿ
ಇಂದ್ರಾದಿ-ಗಳಿಂದ
ಇಂದ್ರಿಯ
ಇಂದ್ರಿಯ-ಗಳ
ಇಂದ್ರಿಯ-ಗಳನ್ನು
ಇಂದ್ರಿಯ-ಗಳಲ್ಲ
ಇಂದ್ರಿಯ-ಗಳಲ್ಲಿ
ಇಂದ್ರಿಯ-ಗಳಾಚೆ
ಇಂದ್ರಿಯ-ಗಳಿಂದ
ಇಂದ್ರಿಯ-ಗಳಿಗೆ
ಇಂದ್ರಿಯ-ಗಳಿವೆ
ಇಂದ್ರಿಯ-ಗಳು
ಇಂದ್ರಿಯ-ಗಳೇ
ಇಂದ್ರಿಯ-ಗ್ರಾಹ್ಯ
ಇಂದ್ರಿಯ-ದಿಂದ
ಇಂದ್ರಿಯ-ಲೋಲುಪತೆ-ಯಲ್ಲಿ
ಇಂದ್ರಿಯ-ವನ್ನು
ಇಂದ್ರಿ-ಯವೇ
ಇಂದ್ರಿಯ-ಸುಖ
ಇಂದ್ರಿ-ಯಾ-ತೀತ
ಇಚ್ಚಿಸು-ತ್ತೇನೆ
ಇಚ್ಛಾ-ಶಕ್ತಿ
ಇಚ್ಛಾ-ಶಕ್ತಿಯ
ಇಚ್ಛಾ-ಶಕ್ತಿ-ಯನ್ನು
ಇಚ್ಛಾ-ಶಕ್ತಿಯು
ಇಚ್ಛಿಸ-ಬೇಕು
ಇಚ್ಛಿ-ಸಲಿ
ಇಚ್ಛಿ-ಸ-ಲಿಲ್ಲ
ಇಚ್ಛಿ-ಸಿದ
ಇಚ್ಛಿ-ಸಿದನು
ಇಚ್ಛಿ-ಸಿದರು
ಇಚ್ಛಿಸು-ತ್ತೇನೆ
ಇಚ್ಛಿಸುತ್ತೇವೆ
ಇಚ್ಛಿ-ಸು-ವನು
ಇಚ್ಛಿ-ಸು-ವರೋ
ಇಚ್ಛಿ-ಸು-ವು-ದಿಲ್ಲ
ಇಚ್ಛಿ-ಸು-ವುದೂ
ಇಚ್ಛಿಸು-ವು-ದೆಲ್ಲಾ
ಇಚ್ಛಿಸು-ವೆವು
ಇಚ್ಛೆ
ಇಚ್ಛೆಗೆ
ಇಚ್ಛೆ-ಪಟ್ಟ
ಇಚ್ಛೆ-ಪಟ್ಟರೆ
ಇಚ್ಛೆ-ಯಂತೆ
ಇಚ್ಛೆ-ಯಾಗಲಿ
ಇಚ್ಛೆ-ಯಿಂದ
ಇಚ್ಛೆ-ಯಿಲ್ಲ
ಇಚ್ಛೆಯೂ
ಇಚ್ಛೆ-ಯೆಲ್ಲ
ಇಚ್ಛೆ-ಯೊಂದು
ಇಟ್ಟ
ಇಟ್ಟಿಗೆ
ಇಟ್ಟಿರುತ್ತಾರೆ
ಇಟ್ಟಿರು-ವಂತಹ
ಇಟ್ಟಿ-ರು-ವು-ದಕ್ಕೆ
ಇಟ್ಟು
ಇಟ್ಟು-ಕೊಂಡ
ಇಟ್ಟು-ಕೊಂಡರೂ
ಇಟ್ಟು-ಕೊಂಡರೆ
ಇಟ್ಟು-ಕೊಂಡಿ-ದ್ದೇವೆ
ಇಟ್ಟು-ಕೊಂಡಿ-ರ-ಬಹುದು
ಇಟ್ಟು-ಕೊಂಡಿ-ರ-ಲಾರಿರಿ
ಇಟ್ಟು-ಕೊಂಡಿ-ರುವ
ಇಟ್ಟು-ಕೊಂಡಿ-ರು-ವುವು
ಇಟ್ಟು-ಕೊಳ್ಳ-ಬಾ-ರದು
ಇಟ್ಟು-ಕೊಳ್ಳ-ಬೇಕಾಗಿದೆ
ಇಟ್ಟು-ಕೊಳ್ಳ-ಬೇಕು
ಇಟ್ಟು-ಕೊಳ್ಳ-ಲಾರಿರಿ
ಇಟ್ಟು-ಕೊಳ್ಳಲಿ
ಇಟ್ಟು-ಕೊಳ್ಳಿ
ಇಟ್ಟು-ಕೊಳ್ಳು-ತ್ತೇನೆ
ಇಟ್ಟು-ಕೊಳ್ಳು-ವೆವು
ಇಟ್ಟು-ಕೊಳ್ಳೋಣ
ಇಡ-ಬೇ-ಕಾದ
ಇಡ-ಬೇಕು
ಇಡ-ಬೇಕೆ
ಇಡಲು
ಇಡಿ
ಇಡಿಯ
ಇಡೀ
ಇಡುತ್ತಾನೆ
ಇಡುತ್ತಾರೆ
ಇಡು-ತ್ತಿದೆ
ಇಡುತ್ತಿ-ರುವ
ಇಡು-ತ್ತೇನೆ
ಇಡುವ
ಇಡುವನು
ಇಡು-ವು-ದಕ್ಕೂ
ಇಡು-ವು-ದಕ್ಕೆ
ಇಡು-ವು-ದಿಲ್ಲ
ಇಡು-ವು-ದೊಂದು
ಇಡೋಣ
ಇಣುಕಿ
ಇತರ
ಇತ-ರರ
ಇತರ-ರನ್ನು
ಇತರ-ರಲ್ಲ
ಇತರ-ರ-ಲ್ಲಿ-ರುವ
ಇತರ-ರಷ್ಟೇ
ಇತರ-ರಾಗ-ವಿಸ್ಮಾ-ರಣಂ
ಇತರ-ರಿಂದ
ಇತರ-ರಿ-ಗಾಗಿ
ಇತರ-ರಿ-ಗಿಂತ
ಇತರ-ರಿಗೂ
ಇತರ-ರಿಗೆ
ಇತ-ರರು
ಇತರರೂ
ಇತರ-ರೆಲ್ಲಾ
ಇತರ-ರೊಂದಿಗೆ
ಇತರೆ
ಇತಿ
ಇತಿ-ಹಾಸ
ಇತಿ-ಹಾ-ಸ-ಕಾರ-ರನ್ನು
ಇತಿ-ಹಾ-ಸ-ಕ್ಕಿಂತ
ಇತಿ-ಹಾ-ಸಕ್ಕೆ
ಇತಿ-ಹಾ-ಸ-ಜ್ಞರೊಬ್ಬರು
ಇತಿ-ಹಾ-ಸದ
ಇತಿ-ಹಾ-ಸ-ದಲ್ಲಿ
ಇತಿ-ಹಾ-ಸ-ದಲ್ಲಿಯೇ
ಇತಿ-ಹಾ-ಸ-ದಲ್ಲಿ-ರುವ
ಇತಿ-ಹಾ-ಸ-ದಲ್ಲೆಲ್ಲಾ
ಇತಿ-ಹಾ-ಸ-ದಲ್ಲೇ
ಇತಿ-ಹಾ-ಸ-ದಿಂದ
ಇತಿ-ಹಾ-ಸ-ವನ್ನು
ಇತಿ-ಹಾ-ಸವು
ಇತಿ-ಹಾ-ಸ-ವೆಲ್ಲಾ
ಇತಿ-ಹಾ-ಸವೇ
ಇತ್ತ
ಇತ್ತರು
ಇತ್ತ-ವ-ರಾರು
ಇತ್ತೀಚಿಗಿ-ನವ-ರಾದ
ಇತ್ತೀಚಿನ
ಇತ್ತೀಚಿನದು
ಇತ್ತೀಚಿನ-ದೆಂದು
ಇತ್ತೀಚಿ-ನವು
ಇತ್ತೀಚೆ
ಇತ್ತೀಚೆಗೆ
ಇತ್ತು
ಇತ್ತೇ
ಇತ್ತೋ
ಇತ್ಯಾ-ತ್ಮಕ
ಇತ್ಯಾದಿ
ಇತ್ಯಾ-ದಿ-ಗಳನ್ನು
ಇತ್ಯಾ-ದಿ-ಗಳಿಗೆ
ಇತ್ಯಾ-ದಿ-ಗಳು
ಇತ್ಯಾ-ದಿ-ಗಳೆಲ್ಲ
ಇತ್ಯಾ-ದಿ-ಯಾಗಿ
ಇತ್ಯ್ಧರ್ಥ
ಇದ-ಕ್ಕಾಗಿ
ಇದ-ಕ್ಕಾಗಿಯೇ
ಇದ-ಕ್ಕಿಂತ
ಇದ-ಕ್ಕಿಂತಲೂ
ಇದಕ್ಕೂ
ಇದಕ್ಕೆ
ಇದನ್ನು
ಇದನ್ನೂ
ಇದನ್ನೆಲ್ಲ
ಇದನ್ನೆಲ್ಲಾ
ಇದನ್ನೇ
ಇದರ
ಇದ-ರಂತೆಯೇ
ಇದರ-ಪ್ರ-ಮಾಣ
ಇದ-ರಲ್ಲಿ
ಇದ-ರ-ಲ್ಲಿದೆ
ಇದ-ರ-ಲ್ಲಿಯೆ
ಇದ-ರಲ್ಲೇ
ಇದರ-ಲ್ಲೊಂದು
ಇದರಾಚೆ
ಇದ-ರಿಂದ
ಇದ-ರಿಂದಲೂ
ಇದ-ರಿಂದಲೇ
ಇದ-ರೊಂದಿಗೆ
ಇದಲ್ಲ
ಇದಲ್ಲದೆ
ಇದಾ-ಗಲೇ
ಇದಾದ
ಇದಾವು-ದನ್ನೂ
ಇದಾ-ವುದೂ
ಇದಿರಿ-ಸಿ-ದಾಗ
ಇದಿ-ಲ್ಲದೆ
ಇದು
ಇದು-ವರೆಗೂ
ಇದು-ವ-ರೆಗೆ
ಇದು-ವರೆ-ವಿಗೂ-ಯಾರೂ
ಇದೂ
ಇದೆ
ಇದೆಂದಿಗೂ
ಇದೆಯೆ
ಇದೆ-ಯೆಂದರೆ
ಇದೆಯೇ
ಇದೆ-ಯೇನು
ಇದೆಯೋ
ಇದೆಲ್ಲ
ಇದೆ-ಲ್ಲಕ್ಕೂ
ಇದೆ-ಲ್ಲವೂ
ಇದೆಲ್ಲಾ
ಇದೇ
ಇದೇನು
ಇದೇನೂ
ಇದೇನೋ
ಇದೊಂದು
ಇದೊಂದೆ
ಇದೊಂದೇ
ಇದ್ದ
ಇದ್ದಂತೆ
ಇದ್ದಂತೆಯೇ
ಇದ್ದದ್ದು
ಇದ್ದನೇ
ಇದ್ದರು
ಇದ್ದರೂ
ಇದ್ದರೆ
ಇದ್ದ-ರೆಂದು
ಇದ್ದ-ರೆಂದೂ
ಇದ್ದರೇ
ಇದ್ದ-ರೇನು
ಇದ್ದ-ವ-ರನ್ನು
ಇದ್ದ-ವರು
ಇದ್ದವು
ಇದ್ದಷ್ಟೂ
ಇದ್ದಾಗ
ಇದ್ದಾಗ್ಯೂ
ಇದ್ದಾನೆ
ಇದ್ದಿ-ತೆಂದು
ಇದ್ದಿತೆಂದೂ
ಇದ್ದಿ-ದ್ದರೆ
ಇದ್ದಿರ-ಬಹುದು
ಇದ್ದಿರಲೇ-ಬೇಕು
ಇದ್ದಿರಿ
ಇದ್ದೀರಿ
ಇದ್ದು
ಇದ್ದು-ದ-ಕ್ಕಿಂತ
ಇದ್ದು-ದ-ರಿಂದ
ಇದ್ದುದು
ಇದ್ದುದೇ
ಇದ್ದುವು
ಇದ್ದು-ವೆಂದು
ಇದ್ದೇ
ಇನ್ನಷ್ಟು
ಇನ್ನಾಕೆ
ಇನ್ನಾರನ್ನಾ-ದರೂ
ಇನ್ನಾವ
ಇನ್ನಿಲ್ಲ
ಇನ್ನು
ಇನ್ನು-ಮೇಲೆ
ಇನ್ನು-ಳಿದ
ಇನ್ನೂ
ಇನ್ನೂರು
ಇನ್ನೆ-ರಡು
ಇನ್ನೆಲ್ಲಿ
ಇನ್ನೆ-ಲ್ಲಿಯೂ
ಇನ್ನೆ-ಷ್ಟರ
ಇನ್ನೆಷ್ಟು
ಇನ್ನೆಷ್ಟೋ
ಇನ್ನೊಂದ-ಕ್ಕಿಂತ
ಇನ್ನೊಂದನ್ನು
ಇನ್ನೊಂದರ್ಥ-ದಲ್ಲಿ
ಇನ್ನೊಂದಿರು-ವುದು
ಇನ್ನೊಂದು
ಇನ್ನೊ-ಬ್ಬ-ನನ್ನು
ಇನ್ನೊಬ್ಬ-ನೊಡನೆ
ಇನ್ನೊಬ್ಬ-ರನ್ನು
ಇನ್ನೊಬ್ಬ-ರಿಗೆ
ಇನ್ನೊಬ್ಬರು
ಇನ್ನೊಮ್ಮೆ
ಇನ್ಯಾರೂ
ಇಪ್ಪತ್ತರಷ್ಟು
ಇಪ್ಪತ್ತಾರನೆ
ಇಪ್ಪತ್ತಾರ-ನೆಯ
ಇಪ್ಪತ್ತಾರ-ರಲ್ಲಿ
ಇಪ್ಪತ್ತು
ಇಪ್ಪತ್ತೈದು
ಇಪ್ಪತ್ತ್ತು
ಇಬ್ಬರ
ಇಬ್ಬ-ರಲ್ಲಿ
ಇಬ್ಬ-ರಿಗೂ
ಇಬ್ಬರು
ಇಬ್ಬರೂ
ಇರ
ಇರ-ಕೂಡದು
ಇರ-ದಿ-ದ್ದಾಗ
ಇರ-ಬಯ-ಸುವ
ಇರ-ಬಲ್ಲದು
ಇರ-ಬಲ್ಲಿರಾ
ಇರ-ಬಹುದು
ಇರ-ಬಾ-ರದು
ಇರ-ಬಾ-ರ-ದೆಂದರೆ
ಇರ-ಬೇ-ಕಂತೆ
ಇರ-ಬೇಕಾಗಿ-ತ್ತು
ಇರ-ಬೇಕಾ-ಗು-ತ್ತದೆ
ಇರ-ಬೇಕಾಗು-ವುದು
ಇರ-ಬೇ-ಕಾದ
ಇರ-ಬೇ-ಕಾದರೆ
ಇರ-ಬೇ-ಕಾ-ದುದು
ಇರ-ಬೇಕು
ಇರ-ಬೇಕೆಂದು
ಇರ-ಲಾ-ರದು
ಇರಲಾರ-ದೆಂದು
ಇರಲಾ-ರವು
ಇರ-ಲಾರೆವು
ಇರಲಿ
ಇರ-ಲಿಲ್ಲ
ಇರಲು
ಇರಲೇ
ಇರ-ಲೇ-ಬೇಕು
ಇರ-ಲೇ-ಬೇಕೆಂದು
ಇರಳೂ
ಇರಾನೀಯ-ರಲ್ಲಿ
ಇರಿ
ಇರಿ-ಸಿ-ಕೊಂಡು
ಇರಿ-ಸುವ
ಇರಿ-ಸು-ವರು
ಇರು
ಇರು-ತ್ತದೆ
ಇರು-ತ್ತದೆಯೋ
ಇರು-ತ್ತವೆ
ಇರು-ತ್ತಾ-ನೆಂದು
ಇರು-ತ್ತಾರೆ
ಇರು-ತ್ತಾಳೆ
ಇರು-ತ್ತಿತ್ತು
ಇರು-ತ್ತಿ-ರಲಿಲ್ಲ
ಇರು-ಬಹುದು
ಇರುವ
ಇರು-ವಂತಹ
ಇರು-ವಂತ-ಹ-ವು-ಗಳು
ಇರು-ವಂತೆ
ಇರು-ವನು
ಇರು-ವ-ನೆಂದು
ಇರು-ವರು
ಇರು-ವರೋ
ಇರು-ವಲ್ಲಿ
ಇರು-ವಳು
ಇರು-ವ-ವನೇ
ಇರು-ವ-ವ-ರನ್ನು
ಇರು-ವ-ವರ-ನ್ನೆಲ್ಲಾ
ಇರು-ವ-ವ-ರಿಗೂ
ಇರು-ವ-ವ-ರಿ-ಗೆಲ್ಲ
ಇರು-ವ-ವರು
ಇರು-ವ-ವ-ರೆಗೆ
ಇರು-ವ-ವ-ರೊಂದಿಗೆ
ಇರು-ವಷ್ಟು
ಇರು-ವಷ್ಟೇ
ಇರು-ವಾಗ
ಇರು-ವಿ-ಕೆಗೆ
ಇರು-ವಿಕೆ-ಯನ್ನು
ಇರು-ವಿರಿ
ಇರು-ವಿರೋ
ಇರು-ವು-ದ-ಕ್ಕಾಗಿ
ಇರು-ವು-ದ-ಕ್ಕಿಂತ
ಇರು-ವು-ದಕ್ಕೆ
ಇರು-ವು-ದನ್ನು
ಇರು-ವು-ದ-ರಿಂದ
ಇರು-ವು-ದಲ್ಲ
ಇರು-ವು-ದಾದ
ಇರು-ವು-ದಾದರೆ
ಇರು-ವು-ದಿಲ್ಲ
ಇರು-ವು-ದಿಲ್ಲವೋ
ಇರು-ವುದು
ಇರು-ವು-ದೆಂದು
ಇರು-ವುದೆಲ್ಲ
ಇರು-ವು-ದೆಲ್ಲಾ
ಇರು-ವುದೆಲ್ಲಿ
ಇರು-ವುದೇ
ಇರು-ವುದೇನು
ಇರು-ವು-ದೊಂದೇ
ಇರು-ವುದೋ
ಇರು-ವುವು
ಇರು-ವುವೋ
ಇರು-ವೆನು
ಇರು-ವೆ-ನೆಂಬ
ಇರು-ವೆನೊ
ಇರು-ವೆಯ
ಇರು-ವೆಯು
ಇರು-ವೆವು
ಇರೋಣ
ಇಲ್ಲ
ಇಲ್ಲದ
ಇಲ್ಲ-ದಂತೆ
ಇಲ್ಲ-ದ-ವ-ರಿಗೆ
ಇಲ್ಲ-ದಾಗ
ಇಲ್ಲ-ದಿ-ದ್ದರೆ
ಇಲ್ಲ-ದಿ-ರಲಿ
ಇಲ್ಲ-ದಿರು-ವು-ದನ್ನು
ಇಲ್ಲ-ದಿ-ರು-ವುದೇ
ಇಲ್ಲದೆ
ಇಲ್ಲದೇ
ಇಲ್ಲ-ರಿಗೂ
ಇಲ್ಲ-ವಾಗಿ
ಇಲ್ಲ-ವಾಗಿದೆ
ಇಲ್ಲ-ವಾಗುವ
ಇಲ್ಲವೆ
ಇಲ್ಲ-ವೆಂದಾದರೆ
ಇಲ್ಲ-ವೆಂದಿತು
ಇಲ್ಲ-ವೆಂದು
ಇಲ್ಲ-ವೆಂದೂ
ಇಲ್ಲ-ವೆಂಬ
ಇಲ್ಲ-ವೆಂಬಂತೆ
ಇಲ್ಲ-ವೆಂಬು-ದನ್ನು
ಇಲ್ಲ-ವೆ-ನ್ನುತ್ತಾರೆ
ಇಲ್ಲವೇ
ಇಲ್ಲವೊ
ಇಲ್ಲವೋ
ಇಲ್ಲಾ
ಇಲ್ಲಿ
ಇಲ್ಲಿಂದ
ಇಲ್ಲಿ-ಗಿಂತ
ಇಲ್ಲಿಗೆ
ಇಲ್ಲಿಗೇ
ಇಲ್ಲಿದೆ
ಇಲ್ಲಿನ
ಇಲ್ಲಿಯ
ಇಲ್ಲಿ-ಯ-ವರೆಗೂ
ಇಲ್ಲಿ-ಯ-ವ-ರೆಗೆ
ಇಲ್ಲಿಯೂ
ಇಲ್ಲಿ-ರು-ತ್ತಿದ್ದರು
ಇಲ್ಲಿ-ರುವ
ಇಲ್ಲಿ-ರು-ವಂತಹ
ಇಲ್ಲಿ-ರು-ವರು
ಇಲ್ಲಿವೆ
ಇಲ್ಲೇ
ಇಳಿತ
ಇಳಿತ-ವಿರಲೇ-ಬೇಕು
ಇಳಿದರು
ಇಳಿ-ದಿದ್ದ
ಇಳಿದಿರ-ಬಹುದು
ಇಳಿದು
ಇಳಿದು-ಬರು-ವು-ದಿಲ್ಲ
ಇಳಿ-ದುವು
ಇಳಿದೂ
ಇಳಿಯಲ-ಸಾಧ್ಯ-ವಾದ
ಇಳಿ-ಯ-ಲಿಲ್ಲ
ಇಳಿ-ಯಿತು
ಇಳಿ-ಯು-ತ್ತಿದ್ದರು
ಇಳಿಯುತ್ತಿ-ರುವ
ಇಳಿ-ಯುವ
ಇಳಿಯು-ವರೆ
ಇಳಿ-ಯುವ-ವರ
ಇಳಿಯುವು-ದ-ರಲ್ಲಿ
ಇಳಿಯು-ವುದು
ಇಳಿಯು-ವುವು
ಇಳಿಯು-ವೆವು
ಇಳಿ-ಸಿದರು
ಇವ
ಇವಕ್ಕೂ
ಇವನ
ಇವ-ನನ್ನಲ್ಲಿ
ಇವ-ನನ್ನು
ಇವ-ನಿಗೇಕೆ
ಇವನೇ
ಇವ-ನೊಂದಿಗೆ
ಇವನ್ನರ-ಸುತ್ತಾ
ಇವನ್ನು
ಇವನ್ನೇನೋ
ಇವರ
ಇವ-ರಂತೆಯೇ
ಇವ-ರನ್ನು
ಇವ-ರ-ನ್ನೇ
ಇವ-ರಲ್ಲಿ
ಇವ-ರಿಂದಲೇ
ಇವ-ರಿ-ಗಿಂತ
ಇವ-ರಿಗೆ
ಇವ-ರಿಬ್ಬ-ರಲ್ಲಿ
ಇವರು
ಇವ-ರೆಲ್ಲ
ಇವ-ರೆಲ್ಲ-ರನ್ನು
ಇವ-ರೆಲ್ಲ-ರಿಗೂ
ಇವ-ರೆಲ್ಲರೂ
ಇವ-ರೆಲ್ಲಾ
ಇವರೇ
ಇವ-ಲ್ಲದೆ
ಇವು
ಇವು-ಗಳ
ಇವು-ಗಳನ್ನು
ಇವು-ಗಳ-ನ್ನೆಲ್ಲ
ಇವು-ಗಳ-ಲ್ಲ-ನೇ-ಕವು
ಇವು-ಗಳಲ್ಲಿ
ಇವು-ಗಳಿಂದ
ಇವು-ಗಳಿಂದಾಗಿದೆ
ಇವು-ಗಳಿ-ಗಾಗಿ
ಇವು-ಗಳಿ-ಗಿಂತಲೂ
ಇವು-ಗಳಿ-ಗಿಲ್ಲ
ಇವು-ಗಳಿಗೂ
ಇವು-ಗಳಿಗೆ
ಇವು-ಗಳು
ಇವು-ಗಳೆಲ್ಲ
ಇವು-ಗಳೆ-ಲ್ಲ-ದ-ರ-ಲ್ಲಿಯೂ
ಇವು-ಗಳೆ-ಲ್ಲ-ವನ್ನೂ
ಇವು-ಗಳೆ-ಲ್ಲವೂ
ಇವು-ಗಳೆಲ್ಲಾ
ಇವೂ
ಇವೆ
ಇವೆಯೋ
ಇವೆ-ರಡಕ್ಕೂ
ಇವೆ-ರಡನ್ನು
ಇವೆ-ರಡನ್ನೂ
ಇವೆ-ರಡರ
ಇವೆ-ರಡ-ರಲ್ಲಿ
ಇವೆ-ರಡ-ರಿಂದ
ಇವೆ-ರಡೇ
ಇವೆಲ್ಲ
ಇವೆ-ಲ್ಲ-ಕ್ಕಿಂತಲೂ
ಇವೆ-ಲ್ಲ-ವನ್ನು
ಇವೆ-ಲ್ಲ-ವನ್ನೂ
ಇವೆ-ಲ್ಲವೂ
ಇವೆಲ್ಲಾ
ಇವೇ
ಇಷ್ಟ
ಇಷ್ಟಕ್ಕೆ
ಇಷ್ಟದ
ಇಷ್ಟ-ದಂತೆ
ಇಷ್ಟ-ದೇವತಾ
ಇಷ್ಟ-ದೇ-ವತೆ
ಇಷ್ಟ-ದೇವ-ತೆಯ
ಇಷ್ಟ-ದೇವ-ತೆ-ಯಾಗಿ-ರು-ವನು
ಇಷ್ಟ-ದೈವ-ವಾಗ-ಬೇಕು
ಇಷ್ಟನ್ನು
ಇಷ್ಟ-ಪಟ್ಟರೂ
ಇಷ್ಟ-ಪಟ್ಟಿ-ದ್ದರೆ
ಇಷ್ಟ-ಪಡು-ತ್ತೇನೆ
ಇಷ್ಟ-ಪ-ಡು-ವು-ದಿಲ್ಲ-ವೆಂದು
ಇಷ್ಟ-ಬಂದ
ಇಷ್ಟ-ರ-ಮಟ್ಟಿನ
ಇಷ್ಟ-ವನ್ನೂ
ಇಷ್ಟ-ವಾಗಿ-ರಲಿ
ಇಷ್ಟ-ವಾದ
ಇಷ್ಟ-ವಾದಂತೆ
ಇಷ್ಟ-ವಿಲ್ಲ
ಇಷ್ಟ-ವೆಂದು
ಇಷ್ಟು
ಇಷ್ಟು-ಕಾಲ
ಇಷ್ಟೆ
ಇಷ್ಟೆಲ್ಲಾ
ಇಷ್ಟೇ
ಇಷ್ಟೊಂದು
ಇಸವಿ
ಇಸವಿಯ
ಇಸ್ರೇ-ಲ್
ಇಸ್ಲಾಂಧರ್ಮ
ಇಹ
ಇಹ-ಪರ-ಗಳ
ಇಹ-ಲೋಕ
ಇಹ-ಲೋಕದ
ಇಹ-ಲೋಕ-ದಲ್ಲಿ
ಇಹ-ಲೋಕ-ವನ್ನು
ಇಹೈವ
ಈ
ಈಗ
ಈಗ-ಕಾರ್ಯ-ರೂಪಕ್ಕೆ
ಈಗ-ತಾನೆ
ಈಗ-ಲಾದರೋ
ಈಗಲೂ
ಈಗಲೇ
ಈಗಾ-ಗಲೇ
ಈಗಾಗಿ-ರು-ವು-ದ-ಕ್ಕಿಂತಲೂ
ಈಗಿನ
ಈಗಿ-ರುವ
ಈಗಿ-ರು-ವಂತೆ
ಈಗಿ-ರುವ-ವರು
ಈಗಿ-ರುವ-ವ-ರೆಲ್ಲ
ಈಗಿಲ್ಲ
ಈಗೀಗ
ಈಗೊಂದು
ಈಚಿನ
ಈಚೀಚೆಗೆ
ಈಚೆಗೆ
ಈಜಿಪ್ಟಿ-ಯ-ನರು
ಈಜಿಪ್ಪಿನ
ಈಡೇರು-ವು-ದೆಂದು
ಈತ
ಈತನ
ಈರುಳ್ಳಿ
ಈರ್ವರ
ಈವಾದ-ವನ್ನು
ಈಶ್ವರ
ಈಶ್ವರಃ
ಈಶ್ವ-ರನ
ಈಶ್ವರ-ನನ್ನು
ಈಶ್ವರ-ನ-ನ್ನೇ
ಈಶ್ವರ-ನಲ್ಲಿ
ಈಶ್ವರ-ನಿ-ಗಿಂತಲೂ
ಈಶ್ವರ-ನಿಗೆ
ಈಶ್ವರ-ನಿ-ದ್ದಾನೆ
ಈಶ್ವರ-ನಿಲ್ಲ
ಈಶ್ವ-ರನು
ಈಶ್ವರ-ನೆಂದು
ಈಶ್ವರ-ನೆ-ಡೆಗೆ
ಈಶ್ವ-ರನೇ
ಈಶ್ವರ-ನೊಬ್ಬನೇ
ಈಶ್ವರ-ಪ್ರೇರಿತ
ಈಶ್ವರ-ಭಾವ-ದಲ್ಲಿ
ಈಶ್ವರ-ಭಾ-ವನೆ
ಈಶ್ವರಾ-ನು-ಭೂತಿ
ಈಶ್ವರಾವ-ತಾರ-ವೆಂದು
ಈಶ್ವರೀ-ಶಕ್ತಿ
ಈಶ್ವರೇ-ಚ್ಛೆ-ಯಿಂದ
ಉಂಟಾ-ಗಿದೆ
ಉಂಟಾಗಿ-ರುವ
ಉಂಟಾ-ಗು-ತ್ತದೆ
ಉಂಟಾ-ಗುವ
ಉಂಟಾಗು-ವು-ದೆಂಬು-ದನ್ನು
ಉಂಟಾ-ದಾಗ
ಉಂಟಾ-ದುದೆಲ್ಲ
ಉಂಟು
ಉಂಟು-ಮಾಡಿ
ಉಂಟು-ಮಾಡಿ-ದಾಗ
ಉಂಟು-ಮಾಡಿದೆ
ಉಂಟು-ಮಾಡು-ತ್ತದೆ
ಉಂಟು-ಮಾಡು-ತ್ತಿದೆ
ಉಂಟು-ಮಾಡುವ
ಉಂಟು-ಮಾಡು-ವಂತೆ
ಉಂಟು-ಮಾಡು-ವುದು
ಉಂಟು-ಮಾಡು-ವುದೇ
ಉಂಡು
ಉಕ್ಕಿ
ಉಕ್ಕಿ-ನಂತಹ
ಉಕ್ತ-ವಾಗಿದೆ
ಉಕ್ತ-ವಾಗಿವೆ
ಉಕ್ತ-ವಾಗು-ತ್ತದೆ
ಉಗಮದ
ಉಗ-ಮಿಸಿ
ಉಗಮಿ-ಸಿದೆ
ಉಗಿಯ-ಶಕ್ತಿ
ಉಗುಳ-ಬೇಡಿ
ಉಗ್ರತೆ-ಯನ್ನು
ಉಗ್ರಾಣ-ವನ್ನಾಗಿ
ಉಚಿತ-ವಲ್ಲ
ಉಚ್ಚ
ಉಚ್ಚ-ಕಂಠ-ದಿಂದ
ಉಚ್ಚ-ತಮ
ಉಚ್ಚ-ತರ
ಉಚ್ಚ-ನೀಚರ
ಉಚ್ಚ-ಭಾವ-ನೆ-ಯ-ನ್ನೇ
ಉಚ್ಚ-ರಿ-ಸದೆ
ಉಚ್ಚ-ರಿಸ-ಬಲ್ಲಿರಾ
ಉಚ್ಚ-ರಿ-ಸ-ಲಿಲ್ಲ
ಉಚ್ಚ-ರಿ-ಸಲು
ಉಚ್ಚ-ರಿಸಿ
ಉಚ್ಚ-ರಿಸಿ-ದರೆ
ಉಚ್ಚ-ರಿಸಿ-ರ-ಲಿಲ್ಲ
ಉಚ್ಚ-ರಿ-ಸುತ್ತ
ಉಚ್ಚ-ರಿ-ಸು-ತ್ತಿದ್ದರು
ಉಚ್ಚ-ರಿಸು-ತ್ತೇನೆ
ಉಚ್ಚ-ವರ್ಗದ
ಉಚ್ಚ-ವರ್ಗದ-ವ-ರನ್ನು
ಉಚ್ಛಾಟಿ-ಸಲಿ
ಉಚ್ಛ್ರಾಯ
ಉಜ್ಜೀವ-ನದ
ಉಜ್ವಲ
ಉಜ್ವಲ-ತರ-ಮಾಡಿ
ಉಜ್ವಲ-ವಾಗಿದೆ
ಉಜ್ವಲ-ವಾಗಿಯೂ
ಉಜ್ವಲ-ವಾಗಿಲ್ಲ
ಉಡುಗೆ
ಉಡುಗೆ-ಗಳನ್ನು
ಉಡುಗೆ-ಯ-ನ್ನುಟ್ಟು
ಉಡುಪನ್ನು
ಉಡುಪು
ಉತ
ಉತ್ಕಟ
ಉತ್ಕಟ-ವಾಗಿ
ಉತ್ಕೃಷ್ಟ
ಉತ್ತಮ
ಉತ್ತಮ-ಕ್ಕೆಂದು
ಉತ್ತಮ-ಗೊಳಿಸ-ಬೇ-ಕಾದರೆ
ಉತ್ತಮ-ಗೊಳಿಸ-ಬೇಕು
ಉತ್ತಮ-ಗೊಳಿಸುವ
ಉತ್ತಮ-ತಮ
ಉತ್ತಮ-ತರ
ಉತ್ತಮ-ದಿಂದ
ಉತ್ತಮ-ನಾಗ-ಬಹು-ದೆಂಬು-ದನ್ನು
ಉತ್ತಮ-ನಾ-ಗಲು
ಉತ್ತಮನೂ
ಉತ್ತಮ-ಪಡಿ-ಸು-ವುದು
ಉತ್ತಮರ
ಉತ್ತಮ-ರಾಗಲಿ
ಉತ್ತಮ-ರಾಗಿ
ಉತ್ತಮ-ರಾ-ಗು-ವು-ದಕ್ಕೆ
ಉತ್ತಮ-ವಾಗಿ
ಉತ್ತಮ-ವಾಗಿ-ತ್ತು
ಉತ್ತಮ-ವಾಗಿ-ರ-ಬಹುದು
ಉತ್ತಮ-ವಾಗಿ-ಲ್ಲ-ವೆಂಬು-ದನ್ನು
ಉತ್ತಮ-ವಾಗು-ವು-ದಿಲ್ಲ
ಉತ್ತಮ-ವಾದ
ಉತ್ತಮ-ವಾದು-ದನ್ನು
ಉತ್ತಮ-ವಾದು-ದನ್ನೂ
ಉತ್ತಮ-ವಾದು-ದರ
ಉತ್ತಮ-ವಾದುದು
ಉತ್ತಮ-ಸ್ಥಾನಕ್ಕೆ
ಉತ್ತಮಾಂಶ-ಗಳನ್ನು
ಉತ್ತ-ಮಾ-ಣು-ಗಳನ್ನು
ಉತ್ತರ
ಉತ್ತ-ರಕ್ಕೆ
ಉತ್ತರ-ಗಳನ್ನು
ಉತ್ತರ-ಗಳು
ಉತ್ತ-ರದ
ಉತ್ತರ-ದಲ್ಲಿ
ಉತ್ತರ-ದಲ್ಲಿ-ರುವ
ಉತ್ತರ-ದಿಂದ
ಉತ್ತರ-ಧ್ರುವದ
ಉತ್ತರ-ರೂಪ-ವಾಗಿ
ಉತ್ತರ-ವನ್ನು
ಉತ್ತರ-ವನ್ನೇನೂ
ಉತ್ತರ-ವಾಗಿ
ಉತ್ತ-ರವೂ
ಉತ್ತ-ರವೇ
ಉತ್ತರ-ಸಿದರು
ಉತ್ತರಾಧಿ-ಕಾರಿ-ಗಳು
ಉತ್ತರಿ-ಸಿದ
ಉತ್ತರಿ-ಸಿದರು
ಉತ್ತಿಲ್ಲವೋ
ಉತ್ತಿಷ್ಠತ
ಉತ್ತೇ-ಜನ
ಉತ್ತೇ-ಜನ-ವನ್ನು
ಉತ್ಥಾ-ನದ
ಉತ್ಪತ್ತಿ-ಯಾಗು-ತ್ತದೆ
ಉತ್ಪತ್ತಿ-ಯಾಗು-ವುದು
ಉತ್ಪನ್ನ-ವಾಗಿವೆ
ಉತ್ಪನ್ನ-ವಾಗು-ತ್ತದೆ
ಉತ್ಪನ್ನ-ವಾಗು-ವುದು
ಉತ್ಪನ್ನ-ವಾದ
ಉತ್ಪನ್ನ-ವಾ-ಯಿತು
ಉತ್ಪ್ರೇಕ್ಷೆ
ಉತ್ಸಾಹ
ಉತ್ಸಾಹಕ್ಕೂ
ಉತ್ಸಾಹಕ್ಕೆ
ಉತ್ಸಾಹ-ಗಳಿಂದ
ಉತ್ಸಾಹ-ಗಳಿಂದಲೇ
ಉತ್ಸಾಹ-ದಲ್ಲಿ
ಉತ್ಸಾಹ-ದಾಯ-ಕ-ವಾಗಿ-ತ್ತು
ಉತ್ಸಾಹ-ದಿಂದ
ಉತ್ಸಾಹ-ದಿಂದಲೂ
ಉತ್ಸಾಹ-ಪೂರಿತ
ಉತ್ಸಾಹ-ಪೂರ್ಣ
ಉತ್ಸಾಹ-ಪೂರ್ಣ-ವಾದ
ಉತ್ಸಾಹ-ವನ್ನು
ಉತ್ಸಾಹವು
ಉತ್ಸಾ-ಹಿ-ಗಳಾಗಿ-ರು-ವಂತೆಯೇ
ಉತ್ಸಾ-ಹಿ-ಗಳಾಗಿ-ರು-ವರು
ಉತ್ಸು-ಕತೆ-ಗಳು
ಉದ-ಯಕ್ಕೆ
ಉದಯಿಸ-ಬೇಕಾಗಿದೆ
ಉದಯಿಸಿ
ಉದಯಿ-ಸಿತು
ಉದಯಿ-ಸಿದ
ಉದಯಿಸಿ-ದರೆ
ಉದಯಿ-ಸಿದ್ದು
ಉದಯಿ-ಸಿವೆ
ಉದಯಿ-ಸು-ವರು
ಉದಯಿಸು-ವುದು
ಉದಾತ್ತ
ಉದಾತ್ತ-ವಾದ
ಉದಾತ್ತವೂ
ಉದಾತ್ತೀ-ಕ-ರಿಸಿ
ಉದಾರ
ಉದಾರ-ಭಾ-ವನೆ
ಉದಾರ-ವಾಗಿ
ಉದಾರ-ವಾಗಿ-ತ್ತು
ಉದಾರ-ಹೃ-ದಯದ
ಉದಾರಿ
ಉದಾಹ-ರಣೆ
ಉದಾಹ-ರಣೆ-ಗಳನ್ನು
ಉದಾಹ-ರಣೆ-ಗಳಿವೆ
ಉದಾಹ-ರಣೆ-ಗಳು
ಉದಾಹ-ರಣೆ-ಗಾಗಿ
ಉದಾ-ಹರ-ಣೆಗೆ
ಉದಾಹ-ರಣೆಯ
ಉದಾಹ-ರಣೆ-ಯನ್ನು
ಉದಾಹ-ರಣೆ-ಯಾಗಿ
ಉದಾಹ-ರಣೆ-ಯಾಗಿದೆ
ಉದಾಹ-ರಣೆ-ಯಾಗಿ-ರ-ಬೇಕು
ಉದಾಹ-ರಣೆ-ಯಾ-ದರೆ
ಉದಾ-ಹ-ರಿಸಿ
ಉದಾ-ಹ-ರಿಸಿದ
ಉದಾ-ಹ-ರಿಸಿ-ದರೆ
ಉದಾಹರಿಸುತ್ತಾರೆ
ಉದಾಹರಿಸು-ವುದು
ಉದಿಸ-ಬೇಕಾಗಿದೆ
ಉದಿ-ಸಿತು
ಉದಿ-ಸಿದ
ಉದಿ-ಸಿದರು
ಉದಿ-ಸಿದವು
ಉದಿ-ಸಿದೆ
ಉದ್ಘೋಷ-ವಾಗು-ತ್ತಿದೆ
ಉದ್ಘೋಷಿ-ಸುತ್ತಾ
ಉದ್ಘೋಷಿಸುತ್ತಿ-ರುವ
ಉದ್ದರಿ-ಸುವ
ಉದ್ದೇಶ
ಉದ್ದೇಶ-ಗಳಿಂದಲೂ
ಉದ್ದೇಶ-ಗಳು
ಉದ್ದೇಶದ
ಉದ್ದೇಶ-ದಿಂದ
ಉದ್ದೇಶ-ವನ್ನು
ಉದ್ದೇಶ-ವನ್ನೂ
ಉದ್ದೇಶ-ವಾಗಿದೆ
ಉದ್ದೇಶ-ವಾಗಿ-ರು-ತ್ತದೆ
ಉದ್ದೇಶವು
ಉದ್ದೇಶವೂ
ಉದ್ದೇಶವೇ
ಉದ್ದೇಶಿಸಿ
ಉದ್ಧರಿ-ಸು-ತ್ತಾರೆ
ಉದ್ಧರಿ-ಸು-ತ್ತಿ-ದ್ದೇವೆ
ಉದ್ಧರಿ-ಸುವ
ಉದ್ಧರಿಸು-ವುದು
ಉದ್ಧರಿಸು-ವು-ದೆಂದು
ಉದ್ಧಾರ
ಉದ್ಧಾರ-ಕ್ಕಾಗಿ
ಉದ್ಧಾ-ರಕ್ಕೆ
ಉದ್ಧಾ-ರದ
ಉದ್ಧಾರ-ಮಾಡುವ
ಉದ್ಧಾರ-ವಾಗ-ಬೇ-ಕಾದರೆ
ಉದ್ಧಾರ-ವಾಗು-ವುದು
ಉದ್ಧಾರ-ವಿಲ್ಲ
ಉದ್ಧಾರೋನ್ಮುಖವೂ
ಉದ್ಧೇಶ
ಉದ್ಧೇಶ-ವೊಂದಿರು-ವುದು
ಉದ್ಧೇಶಿಸಿ
ಉದ್ಧ್ದಾರ-ಕ್ಕಾಗಿ
ಉದ್ಭವ-ವಾಗು-ತ್ತದೆ
ಉದ್ಭವ-ವಾಗುತ್ತವೆ
ಉದ್ಭವ-ವಾಗುವ
ಉದ್ಭವಿಸಿತು
ಉದ್ಭವಿಸಿತೋ
ಉದ್ಭವಿಸಿ-ರು-ವುದೋ
ಉದ್ಭವಿ-ಸುವ
ಉದ್ಭವಿಸು-ವುದು
ಉದ್ಭವಿಸು-ವುದೋ
ಉದ್ಯಮ-ದಲ್ಲಿ
ಉದ್ರೇಕಿ-ಸದೆ
ಉದ್ರೇಕಿ-ಸು-ತ್ತಿದೆ
ಉದ್ವಿಗ್ನತೆ-ಯನ್ನು
ಉದ್ವೇಗ-ದೈತ್ಯ
ಉದ್ವೇಗ-ದೊಂದಿಗೆ
ಉದ್ವೇಗ-ವನ್ನು
ಉನ್ನತ
ಉನ್ನತ-ತರ-ವಾದ
ಉನ್ನತ-ವಾದ
ಉನ್ನತಿ
ಉನ್ನತಿ-ಅವ-ನ-ತಿ-ಗಳ
ಉನ್ನತಿ-ಗಳು
ಉನ್ನತಿ-ಗಾಗಿ
ಉನ್ನ-ತಿಗೆ
ಉನ್ನತಿ-ಗೇ-ರು-ವು-ದಕ್ಕೆ
ಉನ್ನ-ತಿಯ
ಉನ್ನತಿ-ಯನ್ನು
ಉನ್ನತಿ-ಯಾಗು-ವುದೋ
ಉನ್ನತ್ತತೆ-ಯನ್ನು
ಉನ್ಮತ್ತತೆ
ಉನ್ಮತ್ತ-ನಾಗಿ-ರುವನೋ
ಉನ್ಮತ್ತರ-ನ್ನಾಗಿ
ಉನ್ಮತ್ತ-ರಾಗಿ-ರ-ಬಹುದು
ಉನ್ಮತ್ತ-ವಾಗಿ
ಉಪ
ಉಪ-ಕಥೆ-ಗಳನ್ನು
ಉಪ-ಕರ-ಣ-ಮಾತ್ರ
ಉಪ-ಕರ-ಣ-ವನ್ನಾಗಿ
ಉಪ-ಕರ-ಣ-ವೆಂದರೆ
ಉಪ-ಕಾರ
ಉಪ-ಕಾರ-ವನ್ನು
ಉಪ-ಚರಿ-ಸು-ವರು
ಉಪ-ಚರಿಸು-ವುದು
ಉಪ-ಚಾರ-ಗಳಲ್ಲಿ
ಉಪ-ಚುನಾವಣೆ
ಉಪ-ಜಾತಿ-ಗಳು
ಉಪ-ಜಾತಿ-ಯವ
ಉಪದೇಶ
ಉಪದೇ-ಶಕ
ಉಪ-ದೇಶ-ಕ್ಕಿಂತ
ಉಪ-ದೇಶಕ್ಕೆ
ಉಪದೇ-ಶ-ಗಳಿವೆ
ಉಪ-ದೇಶ-ಗಳು
ಉಪದೇ-ಶದ
ಉಪ-ದೇಶ-ದಲ್ಲಿ
ಉಪ-ದೇಶ-ವನ್ನು
ಉಪದೇ-ಶಿ-ಸಿದ
ಉಪದೇ-ಶಿ-ಸಿ-ದ್ದೀರಿ
ಉಪದೇ-ಶಿಸಿ-ರು-ವನು
ಉಪದೇ-ಶಿ-ಸುತ್ತವೆ
ಉಪದೇ-ಶಿಸುತ್ತಾರೆ
ಉಪನಿ
ಉಪನಿ-ಷತ್
ಉಪನಿ-ಷತ್ತನ್ನು
ಉಪನಿ-ಷತ್ತನ್ನೇ
ಉಪನಿ-ಷತ್ತಿ-ಗಿಂತ
ಉಪನಿ-ಷ-ತ್ತಿಗೆ
ಉಪನಿ-ಷ-ತ್ತಿನ
ಉಪನಿ-ಷ-ತ್ತಿನಲ್ಲಿ
ಉಪನಿ-ಷ-ತ್ತಿನ-ಲ್ಲಿ-ರು-ವುದು
ಉಪನಿ-ಷತ್ತಿ-ನಿಂದ
ಉಪನಿ-ಷತ್ತು
ಉಪನಿ-ಷತ್ತು-ಗಳ
ಉಪನಿ-ಷತ್ತು-ಗಳನ್ನು
ಉಪನಿ-ಷತ್ತು-ಗಳ-ನ್ನೇ
ಉಪನಿ-ಷತ್ತು-ಗಳಲ್ಲಿ
ಉಪನಿ-ಷತ್ತು-ಗಳ-ಲ್ಲಿ-ರು-ವುದು
ಉಪನಿ-ಷತ್ತು-ಗಳ-ಲ್ಲಿವೆ
ಉಪನಿ-ಷತ್ತು-ಗಳ-ಲ್ಲೆಲ್ಲಾ
ಉಪನಿ-ಷತ್ತು-ಗಳಿಂದ
ಉಪನಿ-ಷತ್ತು-ಗಳಿಗೆ
ಉಪನಿ-ಷತ್ತು-ಗಳಿಗೆ-ಪ್ರ-ಮಾಣ
ಉಪನಿ-ಷತ್ತು-ಗಳಿವೆ
ಉಪನಿ-ಷತ್ತು-ಗಳು
ಉಪನಿ-ಷತ್ತು-ಗಳೂ
ಉಪನಿ-ಷತ್ತು-ಗಳೇ
ಉಪನಿ-ಷತ್ತೇ
ಉಪನಿ-ಷತ್ಶಾಸ್ತ್ರ
ಉಪ-ನ್ಯಾಸ
ಉಪ-ನ್ಯಾಸ-ಕರು
ಉಪ-ನ್ಯಾಸಕ್ಕೆ
ಉಪ-ನ್ಯಾಸ-ಗಳನ್ನು
ಉಪ-ನ್ಯಾಸ-ಗಳಲ್ಲಿ
ಉಪ-ನ್ಯಾಸ-ಗಳಿಂದ
ಉಪ-ನ್ಯಾಸ-ಗಳಿಗೆ
ಉಪ-ನ್ಯಾಸ-ಗಳು
ಉಪ-ನ್ಯಾಸದ
ಉಪ-ನ್ಯಾಸ-ದಲ್ಲಿ
ಉಪ-ನ್ಯಾಸ-ದಿಂದ
ಉಪ-ನ್ಯಾಸ-ವನ್ನು
ಉಪ-ನ್ಯಾಸ-ವನ್ನೇ
ಉಪ-ಪಂಗಡ-ಗಳಿಗೆ
ಉಪ-ಮಾನ
ಉಪ-ಮಾ-ನ-ದಲ್ಲಿ
ಉಪ-ಮಾ-ನ-ವನ್ನು
ಉಪ-ಮೆ-ಯಲ್ಲಿ-ರುವ
ಉಪ-ಯುಕ್ತ-ತೆ-ಗಳಿಂದ
ಉಪ-ಯೋಗ
ಉಪ-ಯೋಗ-ಕರ-ವಾಗು-ವುದೋ
ಉಪ-ಯೋಗದ
ಉಪ-ಯೋಗ-ವನ್ನು
ಉಪ-ಯೋಗ-ವಾಗು-ವಂತಹ
ಉಪ-ಯೋಗ-ವುಳ್ಳ-ವನಾಗ-ಬೇಕು
ಉಪ-ಯೋಗಿ-ಸದೆ
ಉಪ-ಯೋಗಿಸ-ಬಹುದು
ಉಪ-ಯೋಗಿಸ-ಬೇಕು
ಉಪ-ಯೋಗಿಸ-ಬೇಕೆಂದು
ಉಪ-ಯೋಗಿ-ಸ-ಲಾ-ಗಿದೆ
ಉಪ-ಯೋಗಿ-ಸಲು
ಉಪ-ಯೋಗಿಸಿ
ಉಪ-ಯೋಗಿಸಿ-ಕೊಂಡರು
ಉಪ-ಯೋಗಿಸಿ-ಕೊಳ್ಳ-ಬಹುದು
ಉಪ-ಯೋಗಿಸಿ-ಕೊಳ್ಳಲು
ಉಪ-ಯೋಗಿಸಿ-ಕೊಳ್ಳು-ವು-ದರ
ಉಪ-ಯೋಗಿಸಿ-ಕೊಳ್ಳು-ವುದು
ಉಪ-ಯೋಗಿಸಿ-ದರೆ
ಉಪ-ಯೋಗಿ-ಸಿಲ್ಲ
ಉಪ-ಯೋಗಿ-ಸುತ್ತದೆ
ಉಪ-ಯೋಗಿಸುತ್ತಾರೆ
ಉಪ-ಯೋಗಿಸುತ್ತಿ-ದ್ದೇನೆ
ಉಪ-ಯೋಗಿ-ಸುತ್ತಿರು-ವೆವು
ಉಪ-ಯೋಗಿಸು-ತ್ತೇನೆ
ಉಪ-ಯೋಗಿ-ಸುವ
ಉಪ-ಯೋಗಿಸು-ವಾಗ
ಉಪ-ಯೋಗಿಸು-ವು-ದ-ರಿಂದ
ಉಪ-ಯೋಗಿ-ಸು-ವು-ದಿಲ್ಲ
ಉಪ-ಯೋಗಿಸು-ವುದು
ಉಪ-ವಾ-ಸ-ದಿಂದ
ಉಪ-ವಾ-ಸ-ವನ್ನು
ಉಪ-ವಾ-ಸ-ವಿರ-ಬೇಕಾಗಿಲ್ಲ
ಉಪ-ವಾ-ಸ-ವಿ-ರು-ವಂತೆ
ಉಪಾ-ದಾನ
ಉಪಾ-ದಾನ-ಕಾರಣ
ಉಪಾಧಿ-ಯಿಂದ
ಉಪಾಧ್ಯಾ-ಯ-ನಲ್ಲ
ಉಪಾಧ್ಯಾ-ಯನಾಗ-ಬೇಕಾಗಿಲ್ಲ
ಉಪಾ-ಪಿ-ಸು-ವರು
ಉಪಾಯ
ಉಪಾಸ-ಕರು
ಉಪಾಸನಾ
ಉಪಾಸನೆ
ಉಪಾಸ-ನೆ-ಗಳು
ಉಪಾಸ-ನೆಗೆ
ಉಪಾಸ-ನೆಯ
ಉಪಾಸ-ನೆ-ಯಿಂದ
ಉಬ್ಬರ
ಉಭಯ
ಉಭಯ-ಪಕ್ಷದ-ವ-ರಿಗೂ
ಉರಿ-ಯಲಿ
ಉರಿಯುತ್ತಿರು-ವು-ದನ್ನು
ಉರಿಸುವುದಕ್ಕಷ್ಟು
ಉರುಳಲೇ
ಉರುಳಿ-ದಂತೆ
ಉಲ್ಲೇಖ-ಗೊಂಡಿದೆ
ಉಲ್ಲೇಖಿ-ಸದೆ
ಉಳಲು
ಉಳಲ್ಪಟ್ಟಿದೆಯೋ
ಉಳಿ-ಗಾಲ-ವಿಲ್ಲ
ಉಳಿದ
ಉಳಿ-ದದ್ದು
ಉಳಿದರು
ಉಳಿದರೆ
ಉಳಿದರ್ಧ
ಉಳಿದ-ವರ
ಉಳಿದ-ವ-ರದು
ಉಳಿದ-ವರು
ಉಳಿದ-ವ-ರೆಲ್ಲ
ಉಳಿದ-ವ-ರೆಲ್ಲ-ರಿ-ಗಿಂತಲೂ
ಉಳಿದ-ವ-ರೆಲ್ಲರೂ
ಉಳಿದವು
ಉಳಿದ-ವು-ಗಳನ್ನು
ಉಳಿದ-ವು-ಗಳಿ-ಗಿಂತ
ಉಳಿದ-ವೆಲ್ಲಾ
ಉಳಿ-ದಿದೆ
ಉಳಿದಿ-ದ್ದುವು
ಉಳಿದಿ-ರುವ
ಉಳಿದಿ-ರುವ-ಷ್ಟನ್ನು
ಉಳಿದಿ-ರು-ವು-ದಕ್ಕೆ
ಉಳಿದಿರು-ವುದು
ಉಳಿದಿರು-ವು-ದೊಂದೇ
ಉಳಿದು-ಕೊಂಡಿವೆ
ಉಳಿದು-ಕೊಂಡು
ಉಳಿ-ದು-ದನ್ನು
ಉಳಿದು-ದ-ರಲ್ಲಿ
ಉಳಿ-ದುದೆಲ್ಲ
ಉಳಿದು-ದೆಲ್ಲಾ
ಉಳಿದು-ಬಿಟ್ಟಿದೆ
ಉಳಿದು-ವೆಲ್ಲ
ಉಳಿದು-ವೆ-ಲ್ಲವೂ
ಉಳಿದೆಲ್ಲ-ರನ್ನೂ
ಉಳಿ-ದೆಲ್ಲ-ವನ್ನೂ
ಉಳಿದೆ-ಲ್ಲವೂ
ಉಳಿ-ಯದಿ-ರುವ
ಉಳಿಯ-ಬೇ-ಕಾದರೆ
ಉಳಿಯ-ಬೇಕೆ
ಉಳಿಯಲಾ-ಗದ
ಉಳಿ-ಯ-ಲಿಲ್ಲ
ಉಳಿ-ಯಲು
ಉಳಿಯಲೇ
ಉಳಿ-ಯಿತು
ಉಳಿಯು-ತ್ತಿ-ರಲಿಲ್ಲ
ಉಳಿ-ಯು-ವಂತೆ
ಉಳಿಯು-ವರು
ಉಳಿಯು-ವುದು
ಉಳಿ-ಯು-ವುದೇ
ಉಳಿಯು-ವುದೇನು
ಉಳಿ-ಸಿ-ಕೊಂಡು
ಉಳಿಸಿಕೊಳ್ಳ-ಬೇ-ಕಾದರೆ
ಉಳ್ಳ
ಉಳ್ಳ-ವನು
ಉಳ್ಳ-ವ-ರಾಗಿ
ಉಳ್ಳ-ವ-ರಿ-ಗೆಲ್ಲ
ಉಷ್ಣ
ಉಷ್ಣ-ವ-ಲಯ-ದಲ್ಲಿ
ಉಸಿ-ರನ್ನು
ಉಸಿ-ರಾ-ಗಿದೆ
ಉಸಿ-ರಾಗಿ-ರು-ವುದು
ಉಸಿರಾಡು-ವೆವು
ಉಸಿರಿ-ನಂತೆ
ಉಸಿರು
ಉಸಿರೆಳೆದು
ಉಸುರಿದ-ರೆಂದು
ಊಟ
ಊಟ-ಉಪ-ಚಾ-ರಕ್ಕೆ
ಊಟ-ಕ್ಕಿಂತ
ಊಟಕ್ಕೆ
ಊಟ-ಮಾಡ-ಬಾ-ರದು
ಊಟ-ಮಾಡಿ-ದರೆ
ಊಟ-ಮಾಡು-ತ್ತಿರು-ವಿರಿ
ಊಟ-ಮಾಡು-ವು-ದಕ್ಕೂ
ಊಟ-ಮಾಡು-ವುದು
ಊರಿಗೆ
ಊರಿನ
ಊರಿ-ನ-ಲ್ಲಿಯೂ
ಊರಿ-ನಲ್ಲೇ
ಊರು-ಗಳಲ್ಲಿ
ಊರ್ಧ್ವಗಾಮಿ-ಯಾಗಲು
ಊರ್ಧ್ವ-ಮೂಲದ
ಊಹಿ-ಸ-ಬಹುದು
ಊಹಿ-ಸ-ಬೇಕಾಗಿದೆ
ಊಹಿ-ಸಲೂ
ಊಹಿಸಿ
ಊಹಿ-ಸಿದ್ದೆ
ಊಹಿ-ಸಿಯೇ
ಊಹಿ-ಸು-ತ್ತೀರಿ
ಊಹಿ-ಸು-ತ್ತೇನೆ
ಊಹಿ-ಸು-ವರು
ಊಹಿ-ಸು-ವು-ದಕ್ಕೂ
ಊಹಿ-ಸು-ವು-ದ-ರಿಂದ
ಊಹಿ-ಸು-ವುದು
ಊಹಿ-ಸು-ವುದೂ
ಊಹಿ-ಸೋಣ
ಊಹೆ
ಊಹೆ-ಗಳು
ಊಹೆ-ಗಿಂತ
ಊಹೆಯ
ಊಹೆ-ಯಲ್ಲ
ಋಕ್
ಋಗ್ವೇದ
ಋಗ್ವೇದದ
ಋಗ್ವೇದ-ದಲ್ಲಿ
ಋಣ
ಋಣ-ದಲ್ಲಿ
ಋಣ-ವನ್ನು
ಋಣಿ
ಋಣಿ-ಗಳಾಗಿ-ರುತ್ತಾರೆ
ಋಣಿ-ಗಳು
ಋಣಿ-ಯಾಗಿ-ತ್ತೆಂದು
ಋಣಿ-ಯಾಗಿದೆ
ಋತು
ಋತು-ವಿ-ನಲ್ಲಿ
ಋಷಿ
ಋಷಿ-ಗಳ
ಋಷಿ-ಗಳನ್ನು
ಋಷಿ-ಗಳಾಗ-ಬಹುದು
ಋಷಿ-ಗಳಾಗ-ಬೇಕು
ಋಷಿ-ಗಳಾಗಲೀ
ಋಷಿ-ಗಳಾಗಿ
ಋಷಿ-ಗಳಾಗಿ-ದ್ದರು
ಋಷಿ-ಗಳಾಗಿ-ದ್ದರೂ
ಋಷಿ-ಗಳಾಗಿ-ರು-ವಾಗ
ಋಷಿ-ಗಳಾಗು-ತ್ತಿದ್ದವು
ಋಷಿ-ಗಳಾ-ಗು-ವಂತೆ
ಋಷಿ-ಗಳಾ-ಗುವ-ವ-ರೆಗೆ
ಋಷಿ-ಗಳಾಗು-ವೆವು
ಋಷಿ-ಗಳಾದರೆ
ಋಷಿ-ಗಳಿಂದ
ಋಷಿ-ಗಳಿಗೆ
ಋಷಿ-ಗಳಿ-ದ್ದರು
ಋಷಿ-ಗಳು
ಋಷಿ-ಗಳು-ಅ-ವರು
ಋಷಿ-ಗಳೆಂದು
ಋಷಿ-ಗಳೆ-ನ್ನು-ವರು
ಋಷಿ-ಗಳೆಲ್ಲಾ
ಋಷಿ-ಗಳೇ
ಋಷಿಗೆ
ಋಷಿತ್ವ
ಋಷಿ-ತ್ವ-ವನ್ನು
ಋಷಿ-ತ್ವವು
ಋಷಿ-ದರ್ಶನ-ವನ್ನು
ಋಷಿ-ಮುನಿ-ಗಳ
ಋಷಿ-ಮುನಿ-ಗಳಾಗಿ
ಋಷಿಯ
ಋಷಿ-ಯ-ನ್ನಾ-ದರೂ
ಋಷಿ-ಯಾಗು-ವು-ದ-ರಿಂದ
ಋಷಿ-ಯಿಂದ
ಋಷಿಯು
ಋಷಿಯೂ
ಋಷಿ-ವರ್ಯರು
ಋಷಿ-ವಾಣಿ
ಋಷಿ-ಶ್ರೇಷ್ಠರು
ಋಷಿ-ಸಂತಾ-ನರು
ಎ
ಎಂತಲೂ
ಎಂತಹ
ಎಂತಹುದು
ಎಂತಿವೆ
ಎಂತೆಂತಹದೋ
ಎಂಥ
ಎಂಥದು
ಎಂದ
ಎಂದನು
ಎಂದರು
ಎಂದರೆ
ಎಂದ-ರೆ-ಕೇವಲ
ಎಂದ-ರೇನು
ಎಂದರ್ಥ
ಎಂದಳು
ಎಂದಾಗ
ಎಂದಾ-ಗಲಿ
ಎಂದಾ-ಗಲೀ
ಎಂದಾ-ದರೂ
ಎಂದಾ-ಯಿತು
ಎಂದಿಗಾ-ದರೂ
ಎಂದಿಗೂ
ಎಂದಿತು
ಎಂದಿದೆ
ಎಂದಿದ್ದಾನಷ್ಟೆ
ಎಂದಿ-ನಂತೆ
ಎಂದಿನ-ವ-ರೆಗೆ
ಎಂದಿ-ರು-ವನು
ಎಂದು
ಎಂದು-ತೋರು-ತ್ತದೆ
ಎಂದು-ದ-ರಿಂದ
ಎಂದೂ
ಎಂದೆ
ಎಂದೆಂದಿಗೂ
ಎಂದೆಂದಿಗೂ-ಹೀಗೆಯೇ
ಎಂದೆಂದೂ
ಎಂದೆನು
ಎಂದೆಲ್ಲ
ಎಂದೇ
ಎಂದೋ
ಎಂಬ
ಎಂಬಂತಹ
ಎಂಬಂತೆ
ಎಂಬರ್ಥ-ದಲ್ಲಿಯೂ
ಎಂಬ-ವಿಷ-ಯದ
ಎಂಬೀ
ಎಂಬು-ದಕ್ಕೂ
ಎಂಬು-ದಕ್ಕೆ
ಎಂಬುದ-ನ್ನಾ-ಗಲಿ
ಎಂಬುದನ್ನಾ-ಗಲೀ
ಎಂಬು-ದನ್ನು
ಎಂಬು-ದನ್ನೂ
ಎಂಬುದನ್ನೇ
ಎಂಬು-ದರ
ಎಂಬು-ದ-ರಲ್ಲಿ
ಎಂಬು-ದ-ರಿಂದ
ಎಂಬು-ದಲ್ಲ
ಎಂಬು-ದಾಗಿ
ಎಂಬು-ದಾಗಿ-ತ್ತು
ಎಂಬು-ದಿಲ್ಲ
ಎಂಬುದು
ಎಂಬುದೂ
ಎಂಬುದೆ-ಲ್ಲವೂ
ಎಂಬುದೇ
ಎಂಬು-ದೇನು
ಎಂಬು-ವನು
ಎಂಬ್ಧುದ್ಧಕ್ಕೆ
ಎಂಬ್ಧುದ್ಧನ್ನು
ಎಚ್ಚರಗೊಳ್ಳಿ
ಎಚ್ಚರಿಕೆ
ಎಚ್ಚರಿಕೆ-ಯನ್ನು
ಎಚ್ಚರಿಕೆ-ಯಿಂದ
ಎಚ್ಚರಿಕೆ-ಯಿಂದಿರ-ಬೇಕು
ಎಚ್ಚರಿಕೆ-ಯಿಂದಿರಿ
ಎಚ್ಚೆತ್ತು
ಎಡಗೈ-ಯಿಂದ
ಎಡೆ
ಎಡೆಗೆ
ಎಡೆ-ಬಿ-ಡದೆ
ಎಡೆ-ಯಲ್ಲಿ
ಎಣಿಸ-ಬೇಡಿ
ಎಣಿಸಿ-ದರೂ
ಎಣಿ-ಸಿದ್ದೇ
ಎತ್ತ
ಎತ್ತನ್ನು
ಎತ್ತ-ರವಿರ-ಬಹುದು
ಎತ್ತ-ಲಾ-ರದು
ಎತ್ತ-ಲಿಲ್ಲ
ಎತ್ತಲು
ಎತ್ತಿ
ಎತ್ತಿ-ತೋರಿ-ಸುವ
ಎತ್ತಿ-ದನು
ಎತ್ತಿ-ರುವ
ಎತ್ತಿ-ಹಿ-ಡಿಯುತ್ತಾನೆ
ಎತ್ತಿ-ಹಿ-ಡಿ-ಯುವ
ಎತ್ತು-ಗಳನ್ನು
ಎತ್ತು-ವರು
ಎದು-ರಲ್ಲಿ
ಎದುರಾ-ಗು-ತ್ತದೆ
ಎದುರಾಳಿ-ಯನ್ನೂ
ಎದುರಿಗಿ-ರುವ
ಎದು-ರಿಗೆ
ಎದುರಿಸಬೇಕಾ-ದಾಗ
ಎದುರಿ-ಸಲಾ-ಗದ
ಎದುರಿಸಲಾ-ರವು
ಎದು-ರಿ-ಸಲು
ಎದು-ರಿಸಿ
ಎದು-ರಿಸಿ-ದರೂ
ಎದು-ರಿಸಿ-ದರೆ
ಎದು-ರಿಸಿ-ದ-ವ-ನನ್ನು
ಎದು-ರಿಸಿಯೂ
ಎದುರಿ-ಸುವ
ಎದುರಿ-ಸೋಣ
ಎದುರು
ಎದೆಗಪ್ಪಿ-ಕೊಂಡಿ-ರುವ
ಎದೆಗಾರಿಕೆ
ಎದೆ-ಗೆಟ್ಟು
ಎದೆ-ಯನ್ನು
ಎದ್ದ
ಎದ್ದು
ಎದ್ದು-ನಿಂತು
ಎದ್ದೇಳಿ
ಎದ್ಧೇಳಿ
ಎನಿತೊಮ್ಮೆ
ಎನಿ-ಸುತ್ತದೆ
ಎನ್ನ-ಬಹುದು
ಎನ್ನ-ಬೇಕಾಗಿದೆ
ಎನ್ನ-ಬೇಡಿ
ಎನ್ನ-ಲಾ-ಗು-ವು-ದಿಲ್ಲ
ಎನ್ನ-ಲಾರೆವು
ಎನ್ನ-ಲಿಲ್ಲ
ಎನ್ನ-ವು-ದಲ್ಲ
ಎನ್ನಿ
ಎನ್ನಿ-ಸಿ-ಕೊಳ್ಳು-ತ್ತದೆ
ಎನ್ನಿ-ಸಿತು
ಎನ್ನಿ-ಸು-ವಂತಹ
ಎನ್ನು-ತ್ತದೆ
ಎನ್ನು-ತ್ತವೆ
ಎನ್ನುತ್ತಾನೆ
ಎನ್ನುತ್ತಾರೆ
ಎನ್ನು-ತ್ತಿದ್ದರು
ಎನ್ನು-ತ್ತೀರಿ
ಎನ್ನು-ತ್ತೇನೆ
ಎನ್ನುವ
ಎನ್ನು-ವಂತೆ
ಎನ್ನು-ವನು
ಎನ್ನು-ವರು
ಎನ್ನುವ-ವರ
ಎನ್ನುವ-ವ-ರಾಗಿ-ದ್ದಾರೆ
ಎನ್ನುವ-ವರು
ಎನ್ನು-ವು-ದಕ್ಕೆ
ಎನ್ನು-ವು-ದನ್ನು
ಎನ್ನುವುದನ್ನೇ
ಎನ್ನು-ವು-ದರ
ಎನ್ನು-ವು-ದ-ರಲ್ಲಿ
ಎನ್ನುವುದಾ-ಗಲೀ
ಎನ್ನು-ವುದು
ಎನ್ನು-ವುದೆಲ್ಲ
ಎನ್ನು-ವುದೇ
ಎನ್ನು-ವು-ದೊಂದು
ಎನ್ನು-ವೆನು
ಎನ್ನು-ವೆವು
ಎನ್ನು-ವೆವೋ
ಎಪ್ಪತ್ತು
ಎಬ್ಬಿಸಿ
ಎಬ್ಬಿ-ಸುತ್ತಿರು-ವೆವು
ಎಮ್ಮೆ-ಯೊಂದನ್ನು
ಎರಕ-ಹೊಯ್ದಂತೆ
ಎರಗು
ಎರಚುತ್ತಾ
ಎರಡಕ್ಕೂ
ಎರಡ-ನೆಯ
ಎರಡ-ನೆ-ಯ-ದನ್ನು
ಎರಡ-ನೆ-ಯ-ದಾಗಿ
ಎರಡ-ನೆ-ಯದು
ಎರಡ-ನೆ-ಯದೇ
ಎರಡನೇ
ಎರಡರ
ಎರಡ-ರಲ್ಲಿ
ಎರಡು
ಎರಡೂ
ಎರಡೇ
ಎರವ-ಲಾಗಿ
ಎರೆ-ದಿದೆ
ಎರೆದಿರು-ವುದು
ಎರೆದು
ಎರೆ-ಯದೆ
ಎರೆ-ಯುವ
ಎರೆಯು-ವುದು
ಎರೆ-ಯೋಣ
ಎರೆಹುಳು-ಗಳಂತೆ
ಎಲೆ-ಗಳ
ಎಲೈ
ಎಲ್ಲ
ಎಲ್ಲ-ಕ್ಕಿಂತ
ಎಲ್ಲ-ಕ್ಕಿಂತಲೂ
ಎಲ್ಲಕ್ಕೂ
ಎಲ್ಲ-ದ-ಕ್ಕಿಂತಲೂ
ಎಲ್ಲ-ದಕ್ಕೂ
ಎಲ್ಲ-ದರ
ಎಲ್ಲ-ದರ-ಲ್ಲಿಯೂ
ಎಲ್ಲರ
ಎಲ್ಲ-ರನ್ನು
ಎಲ್ಲ-ರನ್ನೂ
ಎಲ್ಲ-ರ-ಲ್ಲಿಯೂ
ಎಲ್ಲ-ರಿ-ಗಾ-ಗಿಯೂ
ಎಲ್ಲ-ರಿಗೂ
ಎಲ್ಲ-ರಿಗೆ
ಎಲ್ಲರೂ
ಎಲ್ಲ-ವನ್ನು
ಎಲ್ಲ-ವನ್ನೂ
ಎಲ್ಲ-ವು-ಗಳನ್ನು
ಎಲ್ಲ-ವು-ಗಳಿ-ಗಿಂತ
ಎಲ್ಲವೂ
ಎಲ್ಲಾ
ಎಲ್ಲಿ
ಎಲ್ಲಿಂದ
ಎಲ್ಲಿಗೆ
ಎಲ್ಲಿದೆ
ಎಲ್ಲಿ-ದೆಯೊ
ಎಲ್ಲಿ-ದ್ದರು
ಎಲ್ಲಿ-ದ್ದಾರೆ
ಎಲ್ಲಿಯ
ಎಲ್ಲಿ-ಯ-ವರೆಗೂ
ಎಲ್ಲಿ-ಯ-ವ-ರೆಗೆ
ಎಲ್ಲಿ-ಯ-ವರೆ-ವಿಗೂ
ಎಲ್ಲಿ-ಯಾ-ದರೂ
ಎಲ್ಲಿಯೂ
ಎಲ್ಲಿಯೇ
ಎಲ್ಲಿಯೋ
ಎಲ್ಲಿ-ರು-ತ್ತಿದ್ದೆ
ಎಲ್ಲಿ-ರುವಳು
ಎಲ್ಲಿ-ಲ್ಲವೊ
ಎಲ್ಲಿ-ವರೆ-ಗಿರು-ವುದೋ
ಎಲ್ಲೂ
ಎಲ್ಲೆಡೆ-ಯ-ಲ್ಲಿಯೂ
ಎಲ್ಲೆಯ
ಎಲ್ಲೆ-ಯೊಳಗೆ
ಎಲ್ಲೆಲ್ಲಿ
ಎಲ್ಲೆ-ಲ್ಲಿಯೂ
ಎಲ್ಲೆಲ್ಲೂ
ಎಲ್ಲೊ
ಎಲ್ಲೋ
ಎಳೆ
ಎಳೆ-ದತ್ತ
ಎಳೆದು
ಎಳೆ-ದು-ಕೊಂಡಿ-ದ್ದರೂ
ಎಳೆ-ಯ-ಬೇಡಿ
ಎಳೆ-ಯ-ಲಾರೆವು
ಎಳೆ-ಯಲು
ಎಳೆ-ಯಲೂ
ಎಳೆ-ಯು-ವು-ದಲ್ಲ
ಎಳ್ಳಿ-ನಷ್ಟು
ಎಳ್ಳಿನಿತೂ
ಎಷ್ಟನ್ನು
ಎಷ್ಟರ
ಎಷ್ಟರ-ಮಟ್ಟಿಗೆ
ಎಷ್ಟಿದ್ದರೂ
ಎಷ್ಟಿರ-ಬಹುದು
ಎಷ್ಟು
ಎಷ್ಟು-ಕ್ಲಿಷ್ಟ-ವಾದುದು
ಎಷ್ಟು-ಗೌರವ
ಎಷ್ಟು-ಮಟ್ಟಿಗೆ
ಎಷ್ಟೆ
ಎಷ್ಟೇ
ಎಷ್ಟೇ-ಪ್ರ-ಚಾರ
ಎಷ್ಟೊಂದನ್ನು
ಎಷ್ಟೊಂದು
ಎಷ್ಟೋ
ಎಸ-ಗಲು
ಎಸೆದ
ಎಸೆದ-ರೆಂದು
ಎಸೆದಿ-ರುವ
ಎಸೆದಿ-ರು-ವರು
ಎಸೆಯಲ್ಪಡು-ವರು
ಎಸೆ-ಯಿತು
ಎಸೆ-ಯಿ-ತೆಂದು
ಎಸೆ-ಯಿರಿ
ಎಸೆಯುತ್ತಾರೆ
ಎಸೆಯು-ವುದು
ಏಕ
ಏಕಂ
ಏಕ-ಕಂಠ-ರಾಗಿ
ಏಕ-ಕಾಲ-ದಲ್ಲಿ
ಏಕತೆ
ಏಕ-ತೆಗೆ
ಏಕ-ತೆ-ಯನ್ನು
ಏಕ-ತೆ-ಯಿಂದ
ಏಕ-ತೆ-ಯಿಲ್ಲ
ಏಕ-ತೆಯು
ಏಕತ್ರ
ಏಕತ್ವ
ಏಕ-ತ್ವದ
ಏಕ-ತ್ವ-ದಲ್ಲಿ
ಏಕ-ತ್ವ-ವನ್ನು
ಏಕ-ತ್ವ-ವಿದೆ
ಏಕ-ತ್ವ-ವೆಂಬ
ಏಕ-ತ್ವವೇ
ಏಕ-ದೃಷ್ಟಿ
ಏಕ-ದೇವ-ತಾ-ವಾದವು
ಏಕ-ಪ್ರ-ಕಾರ-ವಾಗಿ
ಏಕ-ಭಾವ
ಏಕ-ಭಾವ-ವಿಲ್ಲ
ಏಕ-ಮತ-ದ-ವ-ರಾಗಿ-ರು-ವು-ದ-ರಿಂದಲೇ
ಏಕ-ಮತ-ವನ್ನು
ಏಕ-ಮತ-ವಿದೆ
ಏಕ-ಮ-ತೀಯರು
ಏಕ-ಮ-ಹೇ-ಶ್ವ-ರನ
ಏಕ-ಮಾತ್ರ
ಏಕ-ಮುಖ-ವಾಗಿ
ಏಕ-ಮೇ-ವಾದ್ವಿ-ತೀಯ-ವನ್ನು
ಏಕ-ರೀತಿ-ಯಾಗಿದೆ
ಏಕ-ವನ್ನು
ಏಕ-ವನ್ನೇ
ಏಕ-ವಾಗಿಯೂ
ಏಕ-ವಾಗಿವೆ
ಏಕ-ವಾಣಿ
ಏಕ-ವಾದ
ಏಕ-ವಾದುದನ್ನೇ
ಏಕವೂ
ಏಕಾತ್ಮ-ನಲ್ಲಿ
ಏಕಾತ್ಮನು
ಏಕಾ-ಭಿಪ್ರಾಯ-ವಿರು-ವುದು
ಏಕಾ-ಭಿಪ್ರಾಯ-ವುಳ್ಳ-ರಾಗಿ-ರು-ವೆವು
ಏಕಾ-ಭಿಪ್ರಾಯ-ವುಳ್ಳ-ವ-ರಾಗಿ-ದ್ದೇವೆ
ಏಕೀ-ಕ-ರಿಸಿ
ಏಕೀ-ಭೂತ
ಏಕೆ
ಏಕೆಂದರೆ
ಏಕೆ-ತೆ-ಯಿದೆ-ಯೆಂಬು-ದನ್ನು
ಏಜಕಂಪಿ-ಸು-ವುದು
ಏಜತಿ
ಏತಕ್ಕೆ
ಏತಕ್ಕೆಂದರೆ
ಏತೇ
ಏನಂತೆ
ಏನ-ಕೇನ
ಏನನ್ನಾ-ದರೂ
ಏನನ್ನು
ಏನನ್ನು-ಬೇ-ಕಾದರೂ
ಏನನ್ನೂ
ಏನನ್ನೇ
ಏನನ್ನೋ
ಏನಾಗ-ಬಹುದು
ಏನಾಗಿ
ಏನಾಗಿ-ರು-ವಿರಿ
ಏನಾ-ಗು-ತ್ತದೆ
ಏನಾ-ಗುತ್ತವೆ
ಏನಾಗು-ತ್ತಾ-ನೆ-ಎಂಬ
ಏನಾ-ಗು-ತ್ತಿದೆ
ಏನಾಗು-ವು-ದೆಂಬು-ದನ್ನು
ಏನಾಗು-ವುದೆಂಬುದು
ಏನಾ-ದರೂ
ಏನಾ-ಯಿತು
ಏನಿದೆ
ಏನಿದೆ-ಎಂದು
ಏನಿದೆಯೋ
ಏನು
ಏನೂ
ಏನೆಂದರೆ
ಏನೆಂದು
ಏನೆಂಬು-ದನ್ನು
ಏನೆಂಬುದು
ಏನೇ
ಏನೇ-ನಿದೆಯೋ
ಏನೇನು
ಏನೇನೂ
ಏನೋ
ಏರಿ
ಏರಿ-ದವು
ಏರಿ-ಳಿತ-ಗಳನ್ನು
ಏರಿ-ಸು-ವುದು
ಏರಿ-ಸು-ವುದೇ
ಏರುತ್ತಾರೆ
ಏರು-ವಿರಿ
ಏರ್ಪಡಿ-ಸ-ಬಹುದು
ಏರ್ಪಡಿ-ಸಿದ್ದ
ಏರ್ಪಡಿ-ಸಿದ್ದರು
ಏರ್ಪಡಿ-ಸು-ತ್ತಿವೆ
ಏಳಿ
ಏಳಿ-ಗೆ-ಗಾಗಿ
ಏಳಿ-ಗೆಗೆ
ಏಳಿ-ಗೆ-ಯನ್ನು
ಏಳು
ಏಳು-ತ್ತದೆ
ಏಳು-ತ್ತಿ-ವೆಯೋ
ಏಳು-ಬೀಳು-ಗಳು
ಏಳು-ವು-ದಿಲ್ಲ
ಏಳು-ವುದು
ಏಳು-ವುದೇ
ಏಳು-ವುವು
ಏಳ್ಗೆಯ
ಏಷ್ಯಾ-ಖಂಡದ
ಏಷ್ಯಾ-ಖಂಡ-ದಲ್ಲಿ
ಏಷ್ಯಾ-ದಿಂದ
ಏಷ್ಯಾ-ಮೈನರ್
ಏಸು-ಕ್ರಿಸ್ತನ
ಏಸು-ಕ್ರಿಸ್ತ-ನಿಗೆ
ಐಕ್ಯಗೊಳ್ಳು-ವುವು
ಐಕ್ಯ-ತತ್ತ್ವ-ವೆ-ಲ್ಲಿದೆ
ಐಕ್ಯತೆ
ಐಕ್ಯತೆ-ಯನ್ನು
ಐಕ್ಯ-ಭಾವ-ವನ್ನು
ಐಕ್ಯ-ಭಾವ-ವಿದೆ
ಐಕ್ಯ-ಭಾವ-ವಿನ್ನೂ
ಐಕ್ಯ-ಭಾವವು
ಐಕ್ಯ-ಮಾಡಿ-ಕೊಂಡು
ಐಕ್ಯ-ವಾಗ-ಬೇಕು
ಐಕ್ಯ-ವಾಗಿ-ತ್ತು
ಐಕ್ಯ-ವಾಗಿ-ರುವ
ಐಕ್ಯ-ವಾಗಿ-ರು-ವರು
ಐಕ್ಯ-ವಾಗು-ತ್ತದೆ
ಐಕ್ಯ-ವಾಗುತ್ತವೆ
ಐತಿ-ಹಾ-ಸಿಕ
ಐತಿ-ಹಾ-ಸಿಕ-ವ-ಲ್ಲದೆ
ಐತಿ-ಹಾ-ಸಿಕ-ವಾಗಿ
ಐತಿ-ಹಾ-ಸಿಕವೂ
ಐತಿಹ್ಯ
ಐತಿಹ್ಯಕ್ಕೆ
ಐತಿಹ್ಯದ
ಐದ-ನೆಯ
ಐದು
ಐರೋಪ್ಯ
ಐರೋಪ್ಯ-ರನ್ನು
ಐವತ್ತು
ಐಶ್ವರ್ಯ
ಐಶ್ವರ್ಯ-ಕ್ಕಾಗಿ
ಐಶ್ವರ್ಯದ
ಐಶ್ವರ್ಯ-ದಿಂದ-ಮತ್ತು
ಐಶ್ವರ್ಯ-ಭೋಗ-ಗಳ
ಐಶ್ವರ್ಯ-ಯುಕ್ತ
ಐಶ್ವರ್ಯ-ವನ್ನು
ಐಶ್ವರ್ಯ-ವನ್ನೂ
ಐಶ್ವರ್ಯ-ವೆಲ್ಲಾ
ಐಹಿಕ
ಒಂದ
ಒಂದಕ್ಕೆ
ಒಂದ-ಕ್ಕೊಂದು
ಒಂದ-ನೆಯ
ಒಂದನ್ನು
ಒಂದ-ನ್ನೇ
ಒಂದ-ನ್ನೊಂದು
ಒಂದರ
ಒಂದ-ರಂತೆ
ಒಂದ-ರ-ಮೇಲೊಂದು
ಒಂದ-ರಲ್ಲಿ
ಒಂದ-ರಲ್ಲೇ
ಒಂದ-ರಿಂದೊಂದು
ಒಂದ-ರೆ-ನಿಮಿ-ಷವೂ
ಒಂದಲ್ಲ
ಒಂದಲ್ಲಾ
ಒಂದಾಗ-ಬೇಕು
ಒಂದಾಗಿ
ಒಂದಾಗು-ವುದು
ಒಂದಾಗು-ವುದೋ
ಒಂದಾದ
ಒಂದಾದ-ಮೇಲೊಂದು
ಒಂದಾದರೂ
ಒಂದಾದಾ-ಗಲೇ
ಒಂದಾ-ನೊಂದು
ಒಂದಾ-ಯಿತು
ಒಂದಿತ್ತು
ಒಂದಿದೆ
ಒಂದಿನಿತೂ
ಒಂದಿಷ್ಟು
ಒಂದು
ಒಂದು-ಗೂ-ಡಿ-ದಾಗ
ಒಂದು-ಗೂ-ಡಿ-ದುದು
ಒಂದು-ಗೂಡಿಸಿ-ದಾಗ
ಒಂದು-ಗೂಡಿಸಿ-ದಾಗ-ಲೆಲ್ಲ
ಒಂದು-ಗೂಡಿಸಿದೆ
ಒಂದು-ಗೂಡಿಸು-ವುದು
ಒಂದು-ಪೃಥ್ವಿಯ-ಲ್ಲಿ-ನಮ್ಮ
ಒಂದು-ವರೆ
ಒಂದೆ
ಒಂದೆಡೆ
ಒಂದೆ-ರಡನ್ನು
ಒಂದೆ-ರಡು
ಒಂದೇ
ಒಂದೇ-ದೇವರ
ಒಂದೊಂದ-ನ್ನಾಗಿ
ಒಂದೊಂದು
ಒಗೆಯ-ಬೇಕು
ಒಗ್ಗಟ್ಟಿಲ್ಲ
ಒಗ್ಗೂ-ಡುವಿಕೆಯ
ಒಟ್ಟಾಗಿ
ಒಟ್ಟಾರೆ
ಒಟ್ಟಿಗೆ
ಒಟ್ಟಿ-ನಲ್ಲಿ
ಒಟ್ಟು
ಒಟ್ಟು-ಗೂಡ-ಬೇಕಾ-ಗು-ತ್ತದೆ
ಒಟ್ಟು-ಗೂಡಿ-ಸ-ಲಾ-ಗಿದೆ
ಒಟ್ಟು-ಗೂಡಿ-ಸಲಿ
ಒಟ್ಟು-ಗೂಡಿಸಿ
ಒಟ್ಟು-ಗೂಡಿಸಿ-ರು-ವುದು
ಒಟ್ಟು-ಗೂಡಿ-ಸು-ವರು
ಒಡಂಬಡಿ-ಕೆಯ
ಒಡ-ನೆಯೇ
ಒಡೆದು
ಒಡೆಯ
ಒಡೆ-ಯುವ
ಒಡೆ-ಯು-ವು-ದಕ್ಕೆ
ಒಡ್ಡಿದ
ಒಡ್ಡಿವೆ
ಒಣಗಿ-ಸದು
ಒಣಗಿಸ-ಲಾ-ರದು
ಒಣ-ಹರ-ಟೆಯ
ಒತ್ತಾಯ
ಒತ್ತಿ
ಒದಗಿತ್ತು
ಒದಗಿ-ದರೆ
ಒದಗಿ-ದವು
ಒದ-ಗಿದೆ
ಒದಗಿ-ಬರ-ಲಿಲ್ಲ
ಒದಗಿಸ-ಬಹುದು
ಒದಗಿಸ-ಬೇಕಾಗಿದೆ
ಒದ-ಗಿಸಿ
ಒದ-ಗಿಸಿ-ಕೊಟ್ಟರೆ
ಒದ-ಗಿಸಿ-ಕೊಟ್ಟು
ಒದ-ಗಿಸಿ-ಕೊಡ-ಬೇಕಾಗಿ-ರುವ
ಒದ-ಗಿಸಿ-ರುವ
ಒದಗಿ-ಸಿವೆ
ಒದಗಿ-ಸು-ವರು
ಒದ-ಗಿ-ಸುವ-ವರು
ಒದ್ದರೆ
ಒಪ್ಪತ್ತು
ಒಪ್ಪದ
ಒಪ್ಪದ-ವರಾ-ರಿಗೂ
ಒಪ್ಪ-ದಿ-ದ್ದರೆ
ಒಪ್ಪದೆ
ಒಪ್ಪದೇ
ಒಪ್ಪ-ಬಹು-ದಾದ
ಒಪ್ಪಬೇಕೆಂದೂ
ಒಪ್ಪ-ಲಾ-ರದು
ಒಪ್ಪಲೇ-ಬೇಕಾ-ಗು-ತ್ತದೆ
ಒಪ್ಪಲೇ-ಬೇಕು
ಒಪ್ಪಿ
ಒಪ್ಪಿ-ಕೊಂಡರು
ಒಪ್ಪಿ-ಕೊಂಡರೆ
ಒಪ್ಪಿ-ಕೊಂಡಿ-ದ್ದಾರೆ
ಒಪ್ಪಿ-ಕೊಂಡಿ-ರು-ವರು
ಒಪ್ಪಿ-ಕೊಂಡು
ಒಪ್ಪಿ-ಕೊಳ್ಳದೆ
ಒಪ್ಪಿ-ಕೊಳ್ಳದೇ
ಒಪ್ಪಿ-ಕೊಳ್ಳ-ಬಹುದು
ಒಪ್ಪಿ-ಕೊಳ್ಳ-ಬೇಕು
ಒಪ್ಪಿ-ಕೊಳ್ಳಲೇ-ಬೇ-ಕಾದ
ಒಪ್ಪಿ-ಕೊಳ್ಳಿ
ಒಪ್ಪಿ-ಕೊಳ್ಳು-ತ್ತದೆ
ಒಪ್ಪಿ-ಕೊಳ್ಳು-ತ್ತಾರೆ
ಒಪ್ಪಿ-ಕೊಳ್ಳು-ತ್ತಿದ್ದಿರಿ
ಒಪ್ಪಿ-ಕೊಳ್ಳು-ತ್ತೇನೆ
ಒಪ್ಪಿ-ಕೊಳ್ಳು-ತ್ತೇವೆ
ಒಪ್ಪಿ-ಕೊಳ್ಳುವ
ಒಪ್ಪಿ-ಕೊಳ್ಳು-ವಂತಹ
ಒಪ್ಪಿ-ಕೊಳ್ಳು-ವಂತೆ
ಒಪ್ಪಿ-ಕೊಳ್ಳು-ವರು
ಒಪ್ಪಿ-ಕೊಳ್ಳು-ವಷ್ಟೇ
ಒಪ್ಪಿ-ಕೊಳ್ಳು-ವು-ದ-ರಲ್ಲಿ
ಒಪ್ಪಿ-ಕೊಳ್ಳು-ವು-ದಲ್ಲದೆ
ಒಪ್ಪಿ-ಕೊಳ್ಳು-ವು-ದಿಲ್ಲ
ಒಪ್ಪಿ-ಕೊಳ್ಳು-ವು-ದಿಲ್ಲವೋ
ಒಪ್ಪಿ-ಕೊಳ್ಳು-ವುದು
ಒಪ್ಪಿ-ಕೊಳ್ಳು-ವುವು
ಒಪ್ಪಿ-ಕೊಳ್ಳು-ವುವೋ
ಒಪ್ಪಿ-ಕೊಳ್ಳು-ವೆನು
ಒಪ್ಪಿಗೆ
ಒಪ್ಪಿ-ಗೆ-ಗಿಂತ
ಒಪ್ಪಿ-ಗೆಗೂ
ಒಪ್ಪಿ-ಗೆ-ಯಿತ್ತು
ಒಪ್ಪಿ-ದರೂ
ಒಪ್ಪಿ-ದರೆ
ಒಪ್ಪಿ-ದು-ದ-ಕ್ಕಾಗಿ
ಒಪ್ಪಿ-ದು-ದನ್ನು
ಒಪ್ಪಿ-ದುದು
ಒಪ್ಪಿ-ದ್ದವೋ
ಒಪ್ಪಿ-ದ್ದೇವೆ
ಒಪ್ಪಿ-ರುವ
ಒಪ್ಪಿ-ರು-ವರು
ಒಪ್ಪಿ-ಸಿದ
ಒಪ್ಪುತ್ತಾರೆ
ಒಪ್ಪು-ತ್ತೇನೆ
ಒಪ್ಪುತ್ತೇವೆ
ಒಪ್ಪುವ
ಒಪ್ಪು-ವಂತಹ
ಒಪ್ಪು-ವರು
ಒಪ್ಪು-ವು-ದಾದರೆ
ಒಪ್ಪು-ವು-ದಿಲ್ಲ
ಒಪ್ಪು-ವುದು
ಒಪ್ಪು-ವುವೋ
ಒಬ್ಬ
ಒಬ್ಬನ
ಒಬ್ಬ-ನಂತೆ
ಒಬ್ಬ-ನನ್ನು
ಒಬ್ಬ-ನ-ನ್ನೇ
ಒಬ್ಬ-ನಿಗೂ
ಒಬ್ಬ-ನಿಗೆ
ಒಬ್ಬನು
ಒಬ್ಬ-ನೆ-ನ್ನ-ಬಹು-ದಾದ
ಒಬ್ಬನೇ
ಒಬ್ಬನೋ
ಒಬ್ಬರ
ಒಬ್ಬ-ರದು
ಒಬ್ಬ-ರನ್ನು
ಒಬ್ಬ-ರ-ನ್ನೊಬ್ಬರು
ಒಬ್ಬ-ರಾದ
ಒಬ್ಬ-ರಿಗೊಬ್ಬರು
ಒಬ್ಬರು
ಒಬ್ಬರೇ
ಒಬ್ಬ-ರೊಂದಿಗೆ
ಒಬ್ಬೊಬ್ಬರ
ಒಬ್ಬೊಬ್ಬ-ರನ್ನೂ
ಒಬ್ಬೊಬ್ಬರು
ಒಬ್ಬೊಬ್ಬರೂ
ಒಮ್ಮತ-ದ-ವರಾದು-ದ-ರಿಂದ
ಒಮ್ಮತ-ವಿ-ರುವ
ಒಮ್ಮ-ತವೇ
ಒಮ್ಮೆ
ಒಯ್ದಿದೆ
ಒಯ್ದಿ-ದ್ದೀರಿ
ಒಯ್ಯ-ಬೇಕು
ಒಯ್ಯು-ತ್ತದೆ
ಒಯ್ಯು-ತ್ತಿದ್ದರು
ಒಯ್ಯುವ
ಒಯ್ಯುವಂಥದು
ಒಯ್ಯು-ವುದು
ಒಯ್ಯು-ವುವು
ಒಯ್ಯು-ವುವೋ
ಒರಟಾದ
ಒರಟಾ-ದುದು
ಒರೆ-ಸಿದರು
ಒಲಿ-ಯಲಿ
ಒಲಿಯುತ್ತಿ-ರು-ವರು
ಒಲಿಸಿ-ಕೊಳ್ಳ-ಲಾರಿರಿ
ಒಳಕ್ಕೆ
ಒಳ-ಗಡೆಯೇ
ಒಳಗಣ್ಣನ್ನು
ಒಳ-ಗಾ-ಗಲೀ
ಒಳ-ಗಾಗಿ-ದ್ದರೂ
ಒಳ-ಗಾಗಿ-ದ್ದವು
ಒಳ-ಗಾಗಿ-ರ-ವೆವೋ
ಒಳ-ಗಾಗಿ-ರುವ
ಒಳ-ಗಾಗಿ-ರು-ವರು
ಒಳ-ಗಾಗಿ-ರು-ವುದು
ಒಳಗಾಗು-ವುದು
ಒಳಗಾ-ದರು
ಒಳಗಾ-ದರೆ
ಒಳ-ಗಿನ
ಒಳ-ಗಿನದು
ಒಳ-ಗಿ-ನಿಂದ
ಒಳ-ಗಿ-ನಿಂದಲೂ
ಒಳಗೂ
ಒಳಗೆ
ಒಳಗೇ
ಒಳ-ಗೊಂಡಿದೆ
ಒಳಗೊಂಡಿ-ರುವ
ಒಳಗೊಂಡಿ-ರು-ವುದೇ
ಒಳ-ಗೊಂಡು
ಒಳಗೊಳ್ಳು-ತ್ತದೆ
ಒಳ-ಗೊಳ್ಳು-ತ್ತವೆ
ಒಳ-ಗೊಳ್ಳುವ
ಒಳ-ಜಗಳ-ಗಳೂ
ಒಳ-ತುಂಬು
ಒಳ-ಪಟ್ಟಿಲ್ಲ
ಒಳ-ಪಡಿಸಿ
ಒಳ-ಪಡಿ-ಸಿದರೆ
ಒಳ-ಮುಖವೇ
ಒಳಹೊಕ್ಕರು
ಒಳ್ಳೆ
ಒಳ್ಳೆಯ
ಒಳ್ಳೆ-ಯ-ದಕ್ಕೂ
ಒಳ್ಳೆ-ಯ-ದಕ್ಕೆ
ಒಳ್ಳೆ-ಯ-ದಕ್ಕೊ
ಒಳ್ಳೆ-ಯ-ದಕ್ಕೋ
ಒಳ್ಳೆ-ಯ-ದನ್ನಾ-ದರೂ
ಒಳ್ಳೆ-ಯ-ದನ್ನು
ಒಳ್ಳೆ-ಯ-ದಾಗುತ್ತಿತ್ತು
ಒಳ್ಳೆ-ಯ-ದಾದು-ದನ್ನು
ಒಳ್ಳೆ-ಯದು
ಒಳ್ಳೆ-ಯದೂ
ಒಳ್ಳೆ-ಯದೆ
ಒಳ್ಳೆ-ಯ-ದೆಂದು
ಒಳ್ಳೆ-ಯ-ದೆಂಬ
ಒಳ್ಳೆ-ಯದೇ
ಒಳ್ಳೆ-ಯದೋ
ಒಳ್ಳೆ-ಯ-ವ-ರಾಗಿ
ಒಳ್ಳೆ-ಯ-ವರು
ಒಳ್ಳೆ-ಯವು
ಓ
ಓಂ
ಓಂಕಾರದ
ಓಗೊಟ್ಟು
ಓಗೊಳ್ಳಿ
ಓಜ-ಸ್ಸನ್ನು
ಓಜ-ಸ್ಸಿ-ನಿಂದ
ಓಟು
ಓಡ-ಬೇಡಿ
ಓಡಾಡಿ
ಓಡಿಸ-ಬಲ್ಲ
ಓಡಿ-ಸಬೇಕೆಂದೂ
ಓಡಿ-ಸಲು
ಓಡಿಸಲೇ
ಓಡಿಸಿ
ಓಡಿಸಿ-ದರೆ
ಓಡಿ-ಸು-ವು-ದಕ್ಕೆ
ಓಡು-ವರು
ಓಡು-ವುದು
ಓತಪ್ರೋತ-ನಾಗಿ-ರುವ
ಓತಪ್ರೋತ-ನಾಗಿ-ರುವ-ನೆಂದೂ
ಓತಪ್ರೋತ-ವಾಗಿ
ಓತಪ್ರೋತ-ವಾಗಿದೆ
ಓದ-ದಂತೆ
ಓದದೇ
ಓದ-ಬಹುದು
ಓದ-ಬೇಕೆಂದು
ಓದ-ಲಾ-ಯಿತು
ಓದಲಿ
ಓದಲು
ಓದಿ
ಓದಿದ
ಓದಿ-ದಂತೆಲ್ಲಾ
ಓದಿ-ದರೆ
ಓದಿ-ದಾಗ
ಓದಿ-ದೆನು
ಓದಿ-ದ್ದೇನೆ
ಓದಿ-ಯಾದ
ಓದಿ-ರದ
ಓದಿ-ರು-ವರೋ
ಓದಿ-ರುವಿರಾ
ಓದಿ-ರು-ವೆನು
ಓದಿ-ರು-ವೆವೋ
ಓದಿಲ್ಲ
ಓದಿ-ಲ್ಲವೇ
ಓದುಗರ
ಓದುತ್ತಾ
ಓದು-ತ್ತಿದ್ದರೆ
ಓದು-ತ್ತಿದ್ದೆ
ಓದುತ್ತೇವೆ
ಓದುವ
ಓದು-ವಂತೆ
ಓದುವ-ವ-ರಿಗೆ
ಓದು-ವಾಗ
ಓದು-ವು-ದ-ಕ್ಕಿಂತ
ಓದು-ವು-ದಕ್ಕೆ
ಓದು-ವು-ದ-ರಿಂದಲೂ
ಓದು-ವೆವು
ಓಮಿತ್ಯೇಕಾಕ್ಷರಂ
ಔತಣ-ಕ್ಕಾಗಿ
ಔತ್ತರೇಯ
ಔತ್ತರೇ-ಯರು
ಔದಾರ್ಯ
ಔದಾರ್ಯ-ದಿಂದ
ಔದಾರ್ಯ-ವಿದೆ
ಔನ್ನತ್ಯ-ವನ್ನು
ಔನ್ಯತ್ಯ-ದಲ್ಲಿ-ಹೆಪ್ಪುಗಟ್ಟಿದ್ದ
ಔಪಚಾ-ರಿಕ-ವಾದ
ಔಷಧ
ಔಷಧಿ
ಕಂಕಣ
ಕಂಗೆಟ್ಟರು
ಕಂಗೊಳಿಸು-ತ್ತಿ-ರು-ವನು
ಕಂಟಕ-ರಾದರು
ಕಂಠ-ದಿಂದ
ಕಂಠಪಾಠ
ಕಂಡ
ಕಂಡದ್ದು
ಕಂಡನು
ಕಂಡರು
ಕಂಡರೂ
ಕಂಡರೆ
ಕಂಡ-ವನು
ಕಂಡ-ವ-ರಿಲ್ಲ
ಕಂಡ-ವರು
ಕಂಡವೋ
ಕಂಡಾಗ
ಕಂಡಾ-ಗಲೇ
ಕಂಡಿತು
ಕಂಡಿ-ದ್ದೀರಾ
ಕಂಡಿ-ದ್ದೇನೆ
ಕಂಡಿ-ರ-ಲಿಲ್ಲ
ಕಂಡಿ-ರು-ವರು
ಕಂಡಿ-ರು-ವರೋ
ಕಂಡಿ-ರುವಿರಾ
ಕಂಡಿ-ರು-ವು-ದ-ಕ್ಕಿಂತ
ಕಂಡಿರು-ವೆನು
ಕಂಡಿಲ್ಲ
ಕಂಡಿಲ್ಲ-ವಾದರೆ
ಕಂಡು
ಕಂಡು-ಕೊಂಡ
ಕಂಡು-ಕೊಂಡರು
ಕಂಡು-ಕೊಂಡಾಗ
ಕಂಡು-ಕೊಂಡಿ-ದ್ದೇವೆ
ಕಂಡು-ಕೊಂಡಿ-ರುವ
ಕಂಡು-ಕೊಂಡಿ-ರು-ವರು
ಕಂಡು-ಕೊಳ್ಳಲು
ಕಂಡು-ಕೊಳ್ಳುವ
ಕಂಡು-ಬಂದರೆ
ಕಂಡು-ಬರು-ತ್ತದೆ
ಕಂಡು-ಬರು-ತ್ತವೆ
ಕಂಡು-ಬರು-ತ್ತಿ-ವೆಯೋ
ಕಂಡು-ಬರುವ
ಕಂಡು-ಬರು-ವು-ದಿಲ್ಲ
ಕಂಡು-ಹಿ-ಡಿದ
ಕಂಡು-ಹಿ-ಡಿದನು
ಕಂಡು-ಹಿ-ಡಿದರು
ಕಂಡು-ಹಿ-ಡಿದರೆ
ಕಂಡು-ಹಿ-ಡಿದ-ರೆಂದು
ಕಂಡು-ಹಿ-ಡಿದ-ವರು
ಕಂಡು-ಹಿ-ಡಿ-ದಾಗ
ಕಂಡು-ಹಿ-ಡಿ-ದಿದೆ
ಕಂಡು-ಹಿ-ಡಿದಿ-ರುವ
ಕಂಡು-ಹಿ-ಡಿದಿ-ರುವಿರಾ
ಕಂಡು-ಹಿ-ಡಿದಿರು-ವೆನು
ಕಂಡು-ಹಿ-ಡಿ-ದಿಲ್ಲ
ಕಂಡು-ಹಿ-ಡಿದು-ಕೊಂಡವು
ಕಂಡು-ಹಿ-ಡಿ-ದುದು
ಕಂಡು-ಹಿ-ಡಿದೆ
ಕಂಡು-ಹಿ-ಡಿದೆವೊ
ಕಂಡು-ಹಿ-ಡಿ-ಯಲು
ಕಂಡು-ಹಿ-ಡಿಯಲ್ಪಟ್ಟ
ಕಂಡು-ಹಿ-ಡಿಯುತ್ತಿ-ರುವ
ಕಂಡು-ಹಿ-ಡಿಯು-ತ್ತಿ-ರು-ವಂತೆ
ಕಂಡು-ಹಿ-ಡಿ-ಯುವ
ಕಂಡು-ಹಿ-ಡಿಯು-ವುದು
ಕಂಡು-ಹಿ-ಡಿ-ಯು-ವುದೇ
ಕಂಡೆ
ಕಂಡೇ
ಕಂತೆ
ಕಂತೆ-ಯಾಗು-ವನು
ಕಂದರ
ಕಂದರ-ಗಳಲ್ಲಿ
ಕಂಪಿ-ಸತೊಡ-ಗಿದೆ
ಕಂಬನಿಗರೆ-ದರೂ
ಕಂಬನಿ-ಗಳನ್ನು
ಕಂಬನಿ-ಗಳಿಂದ
ಕಂಬವನ್ನೋ
ಕಕ್ಕಸನ್ನು
ಕಚ್ಚಾಟ-ವೇಕೆ
ಕಚ್ಚಿದವ-ನಿಂದ
ಕಚ್ಚಿ-ದೆಯೋ
ಕಟು
ಕಟು-ಟೀಕೆ
ಕಟು-ವಾಗಿ
ಕಟು-ವಾಗಿದೆ
ಕಟು-ವಾದ
ಕಟ್ಟ-ಕ-ಡೆಗೆ
ಕಟ್ಟ-ಕಡೆಯ
ಕಟ್ಟಡ
ಕಟ್ಟಡ-ಗಳ-ನ್ನೆಲ್ಲ
ಕಟ್ಟ-ಡದ
ಕಟ್ಟಡ-ವನ್ನು
ಕಟ್ಟಡ-ವನ್ನೇ
ಕಟ್ಟ-ಬಹುದು
ಕಟ್ಟ-ಬೇಡಿ
ಕಟ್ಟ-ಲಾ-ಯಿತು
ಕಟ್ಟಲು
ಕಟ್ಟಲ್ಪಟ್ಟಿದೆ
ಕಟ್ಟಾ
ಕಟ್ಟಿ
ಕಟ್ಟಿ-ಕೊಟ್ಟಿ-ರು-ವರು
ಕಟ್ಟಿ-ಕೊಟ್ಟಿರು-ವುದು
ಕಟ್ಟಿ-ಕೊಡು-ವರು
ಕಟ್ಟಿಗೆ
ಕಟ್ಟಿದ
ಕಟ್ಟಿ-ದರೆ
ಕಟ್ಟಿ-ರ-ಬೇಕು
ಕಟ್ಟಿ-ರುವ
ಕಟ್ಟಿ-ರು-ವೆನೋ
ಕಟ್ಟಿ-ಸ-ಬೇಕು
ಕಟ್ಟಿ-ಸಿ-ಕೊಡಿ
ಕಟ್ಟಿ-ಸಿ-ರುವ
ಕಟ್ಟು-ಕಥೆಯ-ನ್ನಾ-ಗಲೀ
ಕಟ್ಟುತ್ತಾ
ಕಟ್ಟು-ತ್ತಾರೆ
ಕಟ್ಟುತ್ತಿರು-ವುದು
ಕಟ್ಟುತ್ತೇವೆ
ಕಟ್ಟುನಿಟ್ಟಾದ
ಕಟ್ಟುನಿಟ್ಟಿನ
ಕಟ್ಟು-ಪಾಡು-ಗಳನ್ನು
ಕಟ್ಟು-ಬಿದ್ದು
ಕಟ್ಟುವ
ಕಟ್ಟು-ವನು
ಕಟ್ಟು-ವಾಗ
ಕಠಿಣ
ಕಠಿಣ-ವಾಗಿದೆ
ಕಠಿಣ-ವಾದ
ಕಠೋಪ-ನಿ-ಷ-ತ್
ಕಠೋಪನಿ-ಷತ್ತನ್ನು
ಕಠೋಪ-ನಿ-ಷ-ತ್ತಿನಲ್ಲಿ
ಕಠೋರ-ನಾಗಿ-ರ-ಬೇಕು
ಕಡಮೆ
ಕಡಮೆ-ಯಾಗಿವೆ
ಕಡಮೆ-ಯಾಗು-ತ್ತಿದೆ
ಕಡಮೆ-ಯಾಗು-ವುದು
ಕಡಮೆ-ಯಾಗು-ವುವು
ಕಡಮೆ-ಯಿಲ್ಲ
ಕಡಮೆಯೋ
ಕಡಲ
ಕಡ-ಲನ್ನು
ಕಡಲಿನ
ಕಡಲಿ-ನಾಚೆ
ಕಡಿಮೆ
ಕಡಿಮೆ-ಯಾಗಿ
ಕಡಿಮೆ-ಯಾಗು-ವುದು
ಕಡಿಯ-ಲಾ-ರದು
ಕಡಿಸಿ-ಕೊಂಡ-ವನು
ಕಡು-ಬಡ-ವ-ರಾದ
ಕಡೆ
ಕಡೆ-ಗಣಿ-ಸ-ಬಾ-ರದು
ಕಡೆ-ಗಣಿ-ಸ-ಲಾ-ಗಿದೆ
ಕಡೆ-ಗಣಿ-ಸ-ಲಾ-ಗಿದೆ-ಇ-ದನ್ನು
ಕಡೆ-ಗಳಲ್ಲಿ
ಕಡೆ-ಗಳಲ್ಲಿಯೂ
ಕಡೆ-ಗಳಲ್ಲೂ
ಕಡೆ-ಗಳ-ಲ್ಲೆಲ್ಲಾ
ಕಡೆ-ಗಳಿ-ಗಿಂತ
ಕಡೆ-ಗಿಂತ
ಕಡೆ-ಗಿಂದು
ಕಡೆಗೂ
ಕಡೆಗೆ
ಕಡೆ-ಯಂತೆ
ಕಡೆ-ಯದು
ಕಡೆ-ಯಲ್ಲಿ
ಕಡೆ-ಯ-ಲ್ಲಿಯೂ
ಕಡೆ-ಯಿಂದ
ಕಡೆಯೂ
ಕಡೆಯೇ
ಕಣ
ಕಣ-ವನ್ನು
ಕಣವು
ಕಣವೇ
ಕಣ್ಣನ್ನು
ಕಣ್ಣಲ್ಲ
ಕಣ್ಣಾರೆ
ಕಣ್ಣಿಗೆ
ಕಣ್ಣಿಟ್ಟು
ಕಣ್ಣಿ-ದ್ದರೆ
ಕಣ್ಣಿನ
ಕಣ್ಣೀ-ರಿನ
ಕಣ್ಣೀರು
ಕಣ್ಣು
ಕಣ್ಣು-ಗಳನ್ನು
ಕಣ್ಣು-ಗಳು
ಕಣ್ಣು-ಮುಚ್ಚಿ
ಕಣ್ಣು-ಮುಚ್ಚಿ-ಕೊಂಡು
ಕಣ್ಣೆದು-ರಿಗೆ
ಕಣ್ತೆರೆದು
ಕಣ್ಮರೆ
ಕಣ್ಮರೆ-ಯಾಗಿ
ಕಣ್ಮರೆ-ಯಾಗಿದೆ
ಕಣ್ಮರೆ-ಯಾಗುವು-ದ-ರೊಂದಿಗೆ
ಕತೆ-ಯೆಲ್ಲ
ಕತ್ತರಿ-ಸದು
ಕತ್ತರಿಸ-ಬೇಕು
ಕತ್ತರಿಸ-ಲಾ-ರದು
ಕತ್ತ-ರಿಸಿ
ಕತ್ತ-ಲನ್ನು
ಕತ್ತ-ಲಲ್ಲಿ
ಕತ್ತಲು
ಕತ್ತಲೆ
ಕತ್ತಲೆ-ಯನ್ನು
ಕತ್ತಲೆ-ಯ-ಲ್ಲಿಯೂ
ಕತ್ತಿ
ಕತ್ತಿ-ಗಳನ್ನು
ಕತ್ತಿಯ
ಕತ್ತಿ-ಯನ್ನು
ಕತ್ತಿ-ಯನ್ನೂ
ಕತ್ತಿಯೂ
ಕತ್ತಿಯೇ
ಕತ್ತೆ
ಕತ್ತೆಗೆ
ಕತ್ತೆತ್ತಿ
ಕತ್ಯ-ಗಳನ್ನೂ
ಕಥೆ-ಗಳ
ಕಥೆ-ಗಳನ್ನು
ಕಥೆ-ಗಳು
ಕಥೆಯ
ಕಥೆ-ಯನ್ನು
ಕದನ-ಕ್ಕಾಗಿ
ಕದ್ದರೆ
ಕದ್ದು
ಕನ-ಸನ್ನು
ಕನಸಿ-ನಲ್ಲಾ-ದರೂ
ಕನಸಿ-ನ-ಲ್ಲಿಯೂ
ಕನಸು
ಕನ-ಸು-ಗಳನ್ನು
ಕನಿ-ಕರ
ಕನಿಷ್ಠ
ಕನಿಷ್ಠ-ನಿಂದಲೂ
ಕನಿಷ್ಠ-ವಾದು
ಕನ್ನಡ
ಕನ್ನಡಿ-ಗಳಂತೆ
ಕನ್ನಡಿಯ
ಕನ್ಯಾ-ಕು-ಮಾರಿ
ಕನ್ಸರ್ವೇಟಿವ್
ಕಪಟಿ
ಕಪಟಿ-ಗಳಂತೆ
ಕಪಟಿ-ಗಳು
ಕಪಿಗೆ
ಕಪಿಲ
ಕಪಿ-ಲನ
ಕಪಿ-ಲ-ನಿಗೆ
ಕಪಿ-ಲನು
ಕಪ್ಪು
ಕಬೀರ
ಕಬೀರ್
ಕಬ್ಬಿಣ
ಕಬ್ಬಿಣ-ದಂತಹ
ಕಮಂಡಲು
ಕಮಲಲೋಚನ-ನಾದ
ಕರ-ಕಮಲ-ಸಂಜಾತ
ಕರ-ಗಳಿಂದ
ಕರ-ಗಿಸ-ಲಾ-ರದು
ಕರ-ಗಿಸಿ
ಕರ-ಗಿಸು-ವಂತಹ
ಕರ-ಣಿಕ
ಕರ-ವನ್ನು
ಕರ-ವಾ-ವಹೈ
ಕರು-ಣಿಸಿ-ದು-ದಕ್ಕೆ
ಕರು-ಣಿಸಿ-ರುವ
ಕರು-ಣಿಸು
ಕರು-ಣಿಸು-ವನು
ಕರುಣೆ
ಕರು-ಣೆ-ದೋರಿ
ಕರು-ಣೆ-ಯಿಂದ
ಕರೆ
ಕರೆಗೆ
ಕರೆದ
ಕರೆ-ದರು
ಕರೆ-ದರೂ
ಕರೆ-ದರೆ
ಕರೆ-ದಿ-ದ್ದಾರೆ
ಕರೆ-ದಿ-ದ್ದೇವೆ
ಕರೆ-ದಿ-ರು-ವರು
ಕರೆದು
ಕರೆ-ದು-ಕೊಂಡು
ಕರೆ-ದು-ಕೊಳ್ಳ-ಬೇಕು
ಕರೆ-ದು-ಕೊಳ್ಳಲಿ
ಕರೆ-ದು-ಕೊಳ್ಳು-ತ್ತಿ-ರುವ
ಕರೆ-ದು-ಕೊಳ್ಳು-ವುದು
ಕರೆ-ದು-ದ-ರಿಂದ
ಕರೆ-ದೊಯ್ದರು
ಕರೆ-ದೊಯ್ಯಿ
ಕರೆ-ದೊಯ್ಯುವ
ಕರೆ-ದೊಯ್ಯು-ವುದು
ಕರೆ-ನೀಡು-ವು-ದಕ್ಕೆ
ಕರೆ-ಯದೆ
ಕರೆ-ಯನ್ನು
ಕರೆ-ಯ-ಬಹುದು
ಕರೆ-ಯ-ಬೇಕು
ಕರೆ-ಯ-ಬೇಡಿ
ಕರೆ-ಯ-ಲಾ-ಗು-ವು-ದಿಲ್ಲ
ಕರೆ-ಯ-ಲಾರೆ
ಕರೆ-ಯಲಿ
ಕರೆ-ಯಲಿ-ಚ್ಛಿಸು-ತ್ತೇನೆ
ಕರೆ-ಯಲ್ಪಡು-ತ್ತಿದೆ
ಕರೆ-ಯ-ಲ್ಪ-ಡುವ
ಕರೆ-ಯಿರಿ
ಕರೆ-ಯಿ-ಸಿ-ಕೊಳ್ಳಲು
ಕರೆ-ಯು-ತ್ತದೆ
ಕರೆ-ಯುತ್ತಾನೆ
ಕರೆ-ಯುತ್ತಾರೆ
ಕರೆ-ಯುತ್ತಾರೋ
ಕರೆ-ಯು-ತ್ತಿದೆ
ಕರೆ-ಯು-ತ್ತಿದ್ದರು
ಕರೆ-ಯುತ್ತಿ-ವೆಯೋ
ಕರೆ-ಯು-ತ್ತೀರಿ
ಕರೆ-ಯುತ್ತೇವೆ
ಕರೆ-ಯುತ್ತೇ-ವೆಯೋ
ಕರೆ-ಯುವ
ಕರೆ-ಯು-ವಂತೆ
ಕರೆ-ಯು-ವನು
ಕರೆ-ಯು-ವರು
ಕರೆ-ಯುವುದರಿಂದಾ-ಗುವ
ಕರೆ-ಯು-ವು-ದಾದರೆ
ಕರೆ-ಯು-ವು-ದಿಲ್ಲ
ಕರೆ-ಯು-ವುದು
ಕರೆ-ಯು-ವುದೂ
ಕರೆ-ಯು-ವುದೋ
ಕರೆ-ಯು-ವೆವು
ಕರೆ-ಯೋಣ
ಕರೆ-ಸಿ-ಕೊಂಡ
ಕರೆ-ಸಿ-ಕೊಳ್ಳಲು
ಕರೆ-ಸಿ-ಕೊಳ್ಳುವ
ಕರ್ತ-ರು-ಅ-ವರು
ಕರ್ತವ್ಯ
ಕರ್ತ-ವ್ಯ-ಕ್ಕಾಗಿ
ಕರ್ತ-ವ್ಯ-ಕ್ಕಾಗಿಯೇ
ಕರ್ತ-ವ್ಯ-ಕ್ಷೇತ್ರ
ಕರ್ತ-ವ್ಯ-ಗಳಲ್ಲಿ
ಕರ್ತ-ವ್ಯ-ಗಳು
ಕರ್ತ-ವ್ಯದ
ಕರ್ತ-ವ್ಯ-ನಿಷ್ಠ
ಕರ್ತ-ವ್ಯ-ವನ್ನು
ಕರ್ತ-ವ್ಯ-ವಿದು
ಕರ್ತ-ವ್ಯವೂ
ಕರ್ತ-ವ್ಯವೇ
ಕರ್ತೃ
ಕರ್ತೃ-ವಾದ
ಕರ್ಮ
ಕರ್ಮ-ಕಾಂಡ
ಕರ್ಮ-ಕಾಂಡಕ್ಕೆ
ಕರ್ಮ-ಕಾಂಡದ
ಕರ್ಮ-ಕಾಂಡ-ದಲ್ಲಿ
ಕರ್ಮ-ಕಾಂಡ-ದಲ್ಲಿಯೇ
ಕರ್ಮ-ಕಾಂಡ-ವನ್ನು
ಕರ್ಮ-ಕಾಂಡ-ವೆಂದು
ಕರ್ಮ-ಕ್ಕಾಗಿ
ಕರ್ಮಕ್ಕೆ
ಕರ್ಮ-ಕ್ಕೋಸ್ಕರ
ಕರ್ಮ-ಗಳ
ಕರ್ಮ-ಗಳನ್ನು
ಕರ್ಮ-ಗಳ-ನ್ನೆಲ್ಲಾ
ಕರ್ಮ-ಗಳಲ್ಲಿ
ಕರ್ಮ-ಗಳ-ಲ್ಲಿಯೂ
ಕರ್ಮ-ಗಳು
ಕರ್ಮ-ದಲ್ಲಿ
ಕರ್ಮ-ದಿಂದ
ಕರ್ಮ-ನಿಯಮ
ಕರ್ಮ-ನಿಯಮ-ಗಳು
ಕರ್ಮ-ಪಟು-ಗಳಾಗಿ-ರು-ವು-ದನ್ನು
ಕರ್ಮ-ಫಲ
ಕರ್ಮ-ಫಲ-ಗಳ
ಕರ್ಮ-ಭೂಮಿ
ಕರ್ಮ-ಭೂಮಿಗೆ
ಕರ್ಮ-ಮಾಡ-ಬೇಕು
ಕರ್ಮ-ಮಾಡು-ವುದು
ಕರ್ಮ-ಯೋಗಿ-ಗಳಲ್ಲಿ
ಕರ್ಮ-ವನ್ನು
ಕರ್ಮ-ವನ್ನೇ
ಕರ್ಮ-ವಿದೆ
ಕರ್ಮ-ವಿ-ಭಾಗ
ಕರ್ಮ-ವಿ-ಲ್ಲದೆ
ಕರ್ಮವು
ಕರ್ಮವೇ
ಕರ್ಮವೋ
ಕರ್ಮ-ಸಿದ್ಧಾಂತ
ಕರ್ಮ-ಸಿದ್ಧಾಂತದ
ಕರ್ಮಾ-ನು-ಸಾರ
ಕರ್ಮಾ-ನು-ಸಾರ-ವಾಗಿ
ಕಲ-ಕತ್ತೆಗೆ
ಕಲ-ಕತ್ತೆಯ
ಕಲಕುವ
ಕಲಕು-ವಂತಹ
ಕಲಸು
ಕಲಸು-ಮೇಲೊಗರ
ಕಲಹ
ಕಲಹ-ಗಳ-ನ್ನೆ-ಬ್ಬಿ-ಸುವ-ವ-ರನ್ನು
ಕಲಹ-ಗಳಿವೆ
ಕಲಹ-ಗಳು
ಕಲಹ-ಗಳೂ
ಕಲಹ-ವಿಲ್ಲ-ದಿ-ರಲಿ
ಕಲಹವು
ಕಲಾ
ಕಲಾ-ಕೌಶಲ್ಯ
ಕಲಾ-ಜೀವನ
ಕಲಿ
ಕಲಿತ
ಕಲಿ-ತರೆ
ಕಲಿ-ತ-ವರ
ಕಲಿ-ತಿದ್ದೇನು
ಕಲಿ-ತಿರ-ಬಾ-ರದು
ಕಲಿ-ತಿರುವಿರೊ
ಕಲಿ-ತಿರು-ವುದು
ಕಲಿ-ತಿರು-ವುದೇನು
ಕಲಿ-ತಿ-ರುವೆ
ಕಲಿ-ತಿರು-ವೆನು
ಕಲಿತು
ಕಲಿ-ತು-ಕೊಳ್ಳ-ಬಹುದು
ಕಲಿ-ತು-ಕೊಳ್ಳ-ಬೇಕಾಗಿದೆ
ಕಲಿ-ತು-ಕೊಳ್ಳ-ಬೇಕಾ-ಯಿತು
ಕಲಿ-ತು-ಕೊಳ್ಳ-ಬೇಕು
ಕಲಿ-ತು-ಕೊಳ್ಳ-ಬೇಕೆಂದು
ಕಲಿ-ತು-ಕೊಳ್ಳಬೇಕೆಂದೂ
ಕಲಿ-ತು-ಕೊಳ್ಳಿ
ಕಲಿ-ತು-ಕೊಳ್ಳು-ತ್ತೇನೆ
ಕಲಿ-ತು-ಕೊಳ್ಳುವ
ಕಲಿ-ತು-ಕೊಳ್ಳು-ವನು
ಕಲಿ-ತು-ಕೊಳ್ಳು-ವಾಗ
ಕಲಿ-ತು-ಕೊಳ್ಳು-ವುದು
ಕಲಿ-ತು-ದೇನು
ಕಲಿತೆ
ಕಲಿ-ಯ-ಬಹುದು
ಕಲಿ-ಯ-ಬೇಕಾಗಿದೆ
ಕಲಿ-ಯ-ಬೇಕಾಗಿ-ರುವ
ಕಲಿ-ಯ-ಬೇ-ಕಾದ
ಕಲಿ-ಯ-ಬೇ-ಕಾದರೆ
ಕಲಿ-ಯ-ಬೇಕು
ಕಲಿ-ಯ-ಬೇಕೆ
ಕಲಿ-ಯಲಿ
ಕಲಿ-ಯಲು
ಕಲಿ-ಯಲೇ-ಬೇಕು
ಕಲಿ-ಯಲೊ-ಲ್ಲದ-ವನು
ಕಲಿ-ಯಿರಿ
ಕಲಿ-ಯು-ಗಕ್ಕೆ
ಕಲಿ-ಯು-ಗದ
ಕಲಿ-ಯುಗ-ದಲ್ಲಿ
ಕಲಿ-ಯುಗ-ದಲ್ಲೂ
ಕಲಿ-ಯುಗ-ವೆಂದು
ಕಲಿ-ಯು-ತ್ತಿದ್ದರೆ
ಕಲಿ-ಯುವ
ಕಲಿ-ಯು-ವು-ದಕ್ಕೆ
ಕಲಿ-ಯುವು-ದಿದೆಯೆ
ಕಲಿ-ಯು-ವುದು
ಕಲಿ-ಯು-ವುದೇ
ಕಲಿ-ಯೋಣ
ಕಲಿ-ಸ-ಬಹುದು
ಕಲಿ-ಸ-ಬೇಕು
ಕಲಿ-ಸಲು
ಕಲಿಸಿ
ಕಲಿ-ಸಿ-ಕೊಟ್ಟು
ಕಲಿ-ಸಿ-ತು-ಅತಿ
ಕಲಿ-ಸಿ-ರುವ
ಕಲಿ-ಸಿ-ರು-ವುದು
ಕಲಿ-ಸುತ್ತ
ಕಲಿ-ಸು-ತ್ತೀರಿ
ಕಲಿ-ಸು-ವರು
ಕಲಿ-ಸು-ವಷ್ಟು
ಕಲಿ-ಸು-ವಾಗ
ಕಲೆ
ಕಲೆತು
ಕಲೆ-ಯ-ಲಾರೆವು
ಕಲೆ-ಯಲ್ಲಿ
ಕಲೆ-ಯು-ವು-ದಿಲ್ಲ
ಕಲೊಸ-ಸ್ಸಿನ
ಕಲ್ಕತ್ತ
ಕಲ್ಕತ್ತಾ
ಕಲ್ಕತ್ತೆಯ
ಕಲ್ಕತ್ತೆ-ಯಲ್ಲಿ
ಕಲ್ಪದ
ಕಲ್ಪ-ದಂತೆಯೇ
ಕಲ್ಪನಾ
ಕಲ್ಪನಾಂಶ-ದಲ್ಲಿ
ಕಲ್ಪನೆ
ಕಲ್ಪ-ನೆ-ಇವು
ಕಲ್ಪ-ನೆ-ಗಳನ್ನು
ಕಲ್ಪ-ನೆ-ಗಳಿ-ರ-ಬಹು-ದೆಂದು
ಕಲ್ಪ-ನೆ-ಗಳಿವೆ
ಕಲ್ಪ-ನೆ-ಗಳು
ಕಲ್ಪ-ನೆಗೂ
ಕಲ್ಪ-ನೆಯ
ಕಲ್ಪ-ನೆ-ಯನ್ನು
ಕಲ್ಪ-ನೆ-ಯಿಂದ
ಕಲ್ಪ-ನೆಯೇ
ಕಲ್ಪಿ-ಸದೇ
ಕಲ್ಪಿ-ಸಬಲ್ಲುದು
ಕಲ್ಪಿಸಿ
ಕಲ್ಪಿ-ಸಿ-ಕೊಂಡರೆ
ಕಲ್ಪಿ-ಸಿ-ಕೊಂಡಿ-ರ-ಲಿಲ್ಲ
ಕಲ್ಪಿ-ಸಿ-ಕೊಟ್ಟಂತಾಗು-ವುದು
ಕಲ್ಪಿ-ಸಿ-ಕೊಟ್ಟರು
ಕಲ್ಪಿ-ಸಿ-ಕೊಳ್ಳ-ದಂತಹ
ಕಲ್ಪಿ-ಸಿ-ಕೊಳ್ಳು-ತ್ತಿ-ರು-ವರು
ಕಲ್ಪಿ-ಸಿ-ಕೊಳ್ಳುವ
ಕಲ್ಪಿ-ಸಿ-ಕೊಳ್ಳು-ವಿರಿ
ಕಲ್ಪಿ-ಸಿತು
ಕಲ್ಪಿ-ಸಿ-ದಂತೆ
ಕಲ್ಪಿ-ಸಿಯೇ
ಕಲ್ಪಿ-ಸಿವೆ
ಕಲ್ಪಿ-ಸುತ್ತದೆ
ಕಲ್ಪಿ-ಸುತ್ತಾನೆ
ಕಲ್ಪಿ-ಸು-ವುದು
ಕಲ್ಯಾಣ
ಕಲ್ಯಾ-ಣ-ಕಾರಿ
ಕಲ್ಯಾ-ಣ-ಕ್ಕಾಗಿ
ಕಲ್ಯಾ-ಣ-ಕ್ಕಾಗಿಯೇ
ಕಲ್ಯಾ-ಣಕ್ಕೆ
ಕಲ್ಯಾ-ಣ-ವನ್ನುಂಟು-ಮಾಡಲಿ
ಕಲ್ಯಾ-ಣ-ವಾಗಲಿ
ಕಲ್ಯಾ-ಣೀ-ಮಾ-ವ-ದಾನಿ
ಕಲ್ಲನ್ನು
ಕಲ್ಲಿಗೆ
ಕಲ್ಲು
ಕಲ್ಲು-ಗಳು
ಕಲ್ಲು-ಬಂಡೆ
ಕಳಂಕ-ರ-ಹಿತ
ಕಳಂಕ-ವನ್ನು
ಕಳಕಳಿ
ಕಳಕಳಿಯ
ಕಳಚಿಕೊಳ್ಳ-ಲಾ-ರದು
ಕಳತ್ರರ
ಕಳು-ಹಿದ
ಕಳು-ಹಿ-ಸ-ಬಹು-ದಾ-ಗಿತ್ತು
ಕಳು-ಹಿ-ಸ-ಬೇಕು
ಕಳು-ಹಿ-ಸ-ಬೇಕೆಂಬುದೇ
ಕಳು-ಹಿ-ಸ-ಲಾ-ಯಿತು
ಕಳು-ಹಿ-ಸ-ಲಾ-ರದು
ಕಳು-ಹಿಸಿ
ಕಳು-ಹಿ-ಸಿ-ಕೊಟ್ಟರು
ಕಳು-ಹಿ-ಸಿ-ದನು
ಕಳು-ಹಿ-ಸಿ-ದರು
ಕಳು-ಹಿ-ಸಿ-ದಳು
ಕಳು-ಹಿ-ಸಿ-ದಾಗ
ಕಳು-ಹಿ-ಸಿದೆ
ಕಳು-ಹಿ-ಸಿದ್ದ
ಕಳು-ಹಿ-ಸಿ-ದ್ದರು
ಕಳು-ಹಿ-ಸುತ್ತದೆ
ಕಳು-ಹಿ-ಸು-ವನು
ಕಳು-ಹಿ-ಸು-ವರು
ಕಳೆದ
ಕಳೆ-ದಂತೆ
ಕಳೆ-ದಂತೆಲ್ಲಾ
ಕಳೆ-ದರೆ
ಕಳೆದು
ಕಳೆ-ದು-ಕೊಂಡ-ದ್ದ-ರಿಂದ
ಕಳೆ-ದು-ಕೊಂಡರು
ಕಳೆ-ದು-ಕೊಂಡರೆ
ಕಳೆ-ದು-ಕೊಂಡಿತು
ಕಳೆ-ದು-ಕೊಂಡಿದೆ
ಕಳೆ-ದು-ಕೊಂಡಿ-ದೆಯೋ
ಕಳೆ-ದು-ಕೊಂಡಿ-ರುವ
ಕಳೆ-ದು-ಕೊಂಡಿ-ರು-ವೆವು
ಕಳೆ-ದು-ಕೊಂಡು
ಕಳೆ-ದು-ಕೊಂಡೊ
ಕಳೆ-ದು-ಕೊಳ್ಳದಿ-ರುವ
ಕಳೆ-ದು-ಕೊಳ್ಳ-ಬೇಕು
ಕಳೆ-ದು-ಕೊಳ್ಳು-ತ್ತದೆ
ಕಳೆ-ದು-ಕೊಳ್ಳು-ಬಹುದು
ಕಳೆ-ದು-ಕೊಳ್ಳು-ವಿರಿ
ಕಳೆ-ದು-ಕೊಳ್ಳು-ವೆನು
ಕಳೆ-ದು-ಹೋ-ಗಿವೆ
ಕಳೆ-ಯನ್ನು
ಕಳೆ-ಯ-ಬೇಕೆಂದು
ಕಳೆ-ಯಿಂದ
ಕಳೆ-ಯು-ತ್ತಿದೆ
ಕಳೆಯು-ತ್ತೇನೆ
ಕಳೆಯುವ
ಕಳೆಯು-ವುದರೊಳಗೆ
ಕಳೆಯು-ವುದು
ಕಳ್ಳ-ತನ
ಕಳ್ಳರು
ಕವನ
ಕವನ-ದಲ್ಲಿ
ಕವಲೊಡೆ-ದರೂ
ಕವಿ
ಕವಿ-ಗಳನ್ನು
ಕವಿ-ಗಳು
ಕವಿತಾಂ
ಕವಿತೆ
ಕವಿತ್ವ
ಕವಿ-ತ್ವ-ವನ್ನು
ಕವಿ-ದಿರ-ಬಹುದು
ಕವಿ-ದು-ಕೊಂಡಿ-ರು-ವಂತೆ
ಕವಿ-ಯು-ವುವು
ಕವಿ-ಯೊಬ್ಬ-ನಿದ್ದ
ಕಶ್ಮಲ-ವನ್ನು
ಕಶ್ಮಲವನ್ನೆಲ್ಲಾ
ಕಷ್ಟ
ಕಷ್ಟ-ಕ್ಕೆಲ್ಲಾ
ಕಷ್ಟ-ಗಳ-ನ್ನನು-ಭವಿಸಿ-ದರೂ
ಕಷ್ಟ-ಗಳನ್ನು
ಕಷ್ಟ-ಗಳ-ನ್ನೆಲ್ಲ
ಕಷ್ಟ-ಗಳಿಗೆ
ಕಷ್ಟ-ಗಳಿಗೆಲ್ಲ
ಕಷ್ಟ-ತರ-ವಾಗು-ತ್ತದೆ
ಕಷ್ಟದ
ಕಷ್ಟ-ದಲ್ಲಿದ್ದ-ವ-ರಿಗೆ
ಕಷ್ಟ-ದಿಂದ
ಕಷ್ಟ-ನಷ್ಟ-ಗಳನ್ನು
ಕಷ್ಟನ್ನು
ಕಷ್ಟ-ಪಟ್ಟು
ಕಷ್ಟ-ವನ್ನು
ಕಷ್ಟ-ವಲ್ಲ
ಕಷ್ಟ-ವಾಗಿ
ಕಷ್ಟ-ವಾಗಿದೆ
ಕಷ್ಟ-ವಾಗಿ-ರ-ಬೇಕು
ಕಷ್ಟ-ವಾಗು-ವಷ್ಟು
ಕಷ್ಟ-ವಾದ
ಕಷ್ಟ-ವಾದರೂ
ಕಷ್ಟ-ವಿದೆ
ಕಷ್ಟ-ವಿಲ್ಲ
ಕಷ್ಟ-ವೆಂದು
ಕಷ್ಟ-ವೆ-ನ್ನು-ವುದು
ಕಷ್ಟ-ಸಂಕಟ-ಗಳು
ಕಷ್ಟ-ಸಾಧ್ಯ-ವೆಂದು
ಕಷ್ಮಲ
ಕಸದ
ಕಸರ-ತ್ತು
ಕಸವನ್ನೆಲ್ಲ
ಕಸಿದು-ಕೊಳ್ಳಲು
ಕಸಿಯಲು
ಕಸುಬು-ಗಳಲ್ಲಿ
ಕಸ್ಮಿನ್ನು
ಕಹ-ಳೆಯ
ಕಹಿ
ಕಹಿ-ಸಿ-ಹಿ-ಗಳನ್ನು
ಕಾಂಚನ
ಕಾಂಚ-ನವೇ
ಕಾಂಟನ
ಕಾಂಟ-ನಲ್ಲಿ
ಕಾಂತಿ
ಕಾಂತಿ-ಯನ್ನು
ಕಾಂತಿ-ಯಿಂದ
ಕಾಂತಿ-ಯುಕ್ತ-ವಾಗಿ-ದ್ದವು
ಕಾಂತಿ-ಯುಕ್ತ-ವಾದ
ಕಾಂತಿ-ಹೀನ-ವಾಗಿ-ಲ್ಲದೆ
ಕಾಡಿ-ನಲ್ಲಿ
ಕಾಡು
ಕಾಡು-ಗಳಲ್ಲಿ
ಕಾಡು-ಜನ-ರಂತೆ
ಕಾಡು-ತ್ತಿ-ರುವ
ಕಾಡು-ಪಾ-ಲಾಗಿ
ಕಾಡು-ಮನುಷ್ಯ
ಕಾಡು-ಮನು-ಷ್ಯ-ರಂತೆ
ಕಾಡು-ಹರ-ಟೆ-ಯಲ್ಲಿ
ಕಾಣದ
ಕಾಣ-ದಂತೆ
ಕಾಣದ-ವನು
ಕಾಣದಿ-ರುವ
ಕಾಣದು
ಕಾಣದೆ
ಕಾಣದೇ
ಕಾಣ-ಬಹುದು
ಕಾಣ-ಬೇ-ಕಾದರೆ
ಕಾಣ-ಬೇಕು
ಕಾಣರು
ಕಾಣಲಾರದ್ದ-ರಿಂದಲ್ಲವೇ
ಕಾಣ-ಲಾ-ರರು
ಕಾಣ-ಲಾರಿರಿ
ಕಾಣ-ಲಾರೆ
ಕಾಣ-ಲಾರೆವು
ಕಾಣಲು
ಕಾಣಲೊಲ್ಲರು
ಕಾಣಿಕೆ
ಕಾಣಿಕೆ-ಗಳನ್ನು
ಕಾಣಿಕೆ-ಯನ್ನು
ಕಾಣಿಕೆ-ಯಾಗಿ
ಕಾಣಿಸಿ-ಕೊಂಡಿವೆ
ಕಾಣಿಸಿ-ಕೊಳ್ಳದೆ
ಕಾಣಿ-ಸಿ-ಕೊಳ್ಳಲು
ಕಾಣಿಸಿ-ಕೊಳ್ಳು-ತ್ತಿದೆ
ಕಾಣಿಸಿ-ಕೊಳ್ಳು-ವು-ದಿಲ್ಲವೋ
ಕಾಣಿ-ಸುತ್ತದೆ
ಕಾಣಿಸು-ತ್ತಿವೆ
ಕಾಣಿಸು-ವಾಗ
ಕಾಣಿ-ಸು-ವು-ದಿಲ್ಲ
ಕಾಣಿಸು-ವುದು
ಕಾಣು-ತ್ತದೆ
ಕಾಣು-ತ್ತವೆ
ಕಾಣು-ತ್ತಾನೆ
ಕಾಣು-ತ್ತಾ-ನೆಯೋ
ಕಾಣು-ತ್ತಾರೆ
ಕಾಣು-ತ್ತಿದೆಯೋ
ಕಾಣು-ತ್ತಿದ್ದ
ಕಾಣು-ತ್ತಿ-ದ್ದೀರಿ
ಕಾಣು-ತ್ತಿ-ದ್ದೇನೆ
ಕಾಣು-ತ್ತಿ-ದ್ದೇವೆ
ಕಾಣು-ತ್ತಿ-ರುವ
ಕಾಣು-ತ್ತಿರು-ವಾಗ
ಕಾಣು-ತ್ತಿರು-ವಿರಿ
ಕಾಣು-ತ್ತಿರು-ವುದು
ಕಾಣು-ತ್ತಿರು-ವುವು
ಕಾಣು-ತ್ತಿಲ್ಲ
ಕಾಣು-ತ್ತಿವೆ
ಕಾಣು-ತ್ತೇವೆ
ಕಾಣುವ
ಕಾಣು-ವಂತಹ
ಕಾಣು-ವಂತೆ
ಕಾಣು-ವಂತೆಯೇ
ಕಾಣು-ವನು
ಕಾಣು-ವರು
ಕಾಣು-ವಾಗ
ಕಾಣು-ವು-ದ-ರಿಂದ
ಕಾಣು-ವು-ದಿಲ್ಲ
ಕಾಣು-ವು-ದಿ-ಲ್ಲವೆ
ಕಾಣು-ವು-ದಿಲ್ಲವೋ
ಕಾಣು-ವುದು
ಕಾಣು-ವುದೇ
ಕಾಣು-ವುದೇನು
ಕಾಣು-ವುವು
ಕಾಣು-ವೆವು
ಕಾಣೆ
ಕಾತರ-ತೆ-ಯಿಂದ
ಕಾತರ-ರಾಗಿ-ರು-ವರು
ಕಾತರಿಸುತ್ತಿ-ರು-ವರು
ಕಾತರಿಸು-ವುದು
ಕಾದ
ಕಾದಾಟ
ಕಾದಾಟದ
ಕಾದಾಡ-ಬೇ-ಕಾದ
ಕಾದಾಡಿ
ಕಾದಾಡಿ-ದನೋ
ಕಾದಾಡಿ-ದ-ರೆಂದರೆ
ಕಾದಾಡುತ್ತಾನೆ
ಕಾದಾ-ಡುವ
ಕಾದಾಡು-ವಂತೆ
ಕಾದಾ-ಡು-ವು-ದಕ್ಕೆ
ಕಾದಾ-ಡು-ವು-ದಿಲ್ಲ
ಕಾದಾಡು-ವುವು
ಕಾದಿದೆ
ಕಾದಿರು-ವಾಗ
ಕಾದಿರು-ವುದು
ಕಾದಿವೆ
ಕಾದು
ಕಾನನ-ಗಳಿಂದ
ಕಾನನಾಂತರ-ಗಳಲ್ಲಿ
ಕಾನೂನಿ-ನಿಂದ
ಕಾನೂನು
ಕಾನೂನು-ಗಳಾಗಲಿ
ಕಾನೂನೂ
ಕಾಪಾಡ-ಬಹುದು
ಕಾಪಾಡ-ಬೇಕು
ಕಾಪಾಡಿ-ಕೊಳ್ಳಲು
ಕಾಪಾಡಿ-ಕೊಳ್ಳೋಣ
ಕಾಪಾಡು-ವು-ದಕ್ಕೆ
ಕಾಪಿಲ
ಕಾಬಾ
ಕಾಬಾದ
ಕಾಮ
ಕಾಮ-ಕಾಂಚನ-ಗಳನ್ನು
ಕಾಮನ
ಕಾಮ-ನಿ-ರುವ
ಕಾಮ-ನಿಲ್ಲ
ಕಾಮ-ನೆಯ
ಕಾಮಯೇ
ಕಾಮ-ವಿ-ಲ್ಲದೆ
ಕಾಮ-ಹೀನನೂ
ಕಾಮೋದ್ರೇಕ-ವಾಗಿ
ಕಾಯಕ್ಕೆ
ಕಾಯ-ಬೇಕು
ಕಾಯಿಸಿ
ಕಾಯುತ್ತಾ
ಕಾಯು-ತ್ತಿದೆ
ಕಾಯು-ತ್ತಿದ್ದ
ಕಾಯು-ತ್ತಿದ್ದರು
ಕಾಯು-ತ್ತಿ-ರಲಿಲ್ಲ
ಕಾಯುತ್ತಿ-ರುವ
ಕಾಯುತ್ತಿ-ರು-ವರು
ಕಾಯುತ್ತಿ-ರುವಳು
ಕಾಯುತ್ತಿರು-ವುವು
ಕಾಯುತ್ತೇವೆ
ಕಾಯು-ವರು
ಕಾಯು-ವಿರಿ
ಕಾಯುವಿ-ರೇನು
ಕಾಯು-ವು-ದ-ಕ್ಕಿಂತ
ಕಾಯು-ವು-ದಕ್ಕೆ
ಕಾಯು-ವುದು
ಕಾಯು-ವೆವು
ಕಾರಣ
ಕಾರ-ಣ-ಕರ್ತರು
ಕಾರ-ಣ-ಕರ್ತರೋ
ಕಾರ-ಣ-ಕ್ಕಾಗಿ
ಕಾರ-ಣ-ಗಳನ್ನು
ಕಾರ-ಣ-ಗಳಲ್ಲ
ಕಾರ-ಣ-ಗಳಿಂದಾಗಿ
ಕಾರ-ಣ-ಗಳಿ-ಗಾ-ಗಿಯೋ
ಕಾರ-ಣ-ಗಳಿ-ರ-ಬೇಕು
ಕಾರ-ಣ-ಗಳು
ಕಾರ-ಣ-ಗಳೆಂದರೆ
ಕಾರ-ಣ-ದಿಂದ
ಕಾರ-ಣ-ದಿಂದಲೂ
ಕಾರ-ಣ-ದಿಂದಲೋ
ಕಾರ-ಣ-ನಲ್ಲ
ಕಾರ-ಣ-ನಾದೆ-ನೆಂದು
ಕಾರ-ಣ-ನೆಂದು
ಕಾರ-ಣ-ರಲ್ಲ
ಕಾರ-ಣ-ರಾಗಿ-ದ್ದೀರಿ
ಕಾರ-ಣ-ರಾದ-ರೆಂದು
ಕಾರ-ಣರು
ಕಾರ-ಣ-ರೆಂಬುದು
ಕಾರ-ಣ-ವನ್ನು
ಕಾರ-ಣ-ವಾಗ-ದಿ-ರಲಿ
ಕಾರ-ಣ-ವಾಗ-ಲಾ-ರದು
ಕಾರ-ಣ-ವಾಗಿದೆ
ಕಾರ-ಣ-ವಾಗಿ-ರ-ಬಹುದು
ಕಾರ-ಣ-ವಾಗು-ತ್ತದೆ
ಕಾರ-ಣ-ವಾಗುತ್ತವೆ
ಕಾರ-ಣ-ವಾಗು-ವುದು
ಕಾರ-ಣ-ವಾಗು-ವುವು
ಕಾರ-ಣ-ವಾದ
ಕಾರ-ಣ-ವಾ-ಯಿ-ತೆಂದು
ಕಾರ-ಣ-ವಿದೆ
ಕಾರ-ಣ-ವಿದ್ದೇ
ಕಾರ-ಣವು
ಕಾರ-ಣವೂ
ಕಾರ-ಣ-ವೆಂದರೆ
ಕಾರ-ಣ-ವೆಂದು
ಕಾರ-ಣ-ವೆಂದೂ
ಕಾರ-ಣ-ವೆಂಬುದು
ಕಾರ-ಣವೇ
ಕಾರ-ಣ-ವೇನು
ಕಾರ-ಣ-ವೇ-ನೆಂದರೆ
ಕಾರ-ತತ್ತ್ವ-ವನ್ನು
ಕಾರಾಗೃಹ-ವಾಸ
ಕಾರಿ-ಗಳು
ಕಾರ್ಮೋಡ-ದಂತೆ
ಕಾರ್ಯ
ಕಾರ್ಯಂ
ಕಾರ್ಯ-ಕಲಾ-ಪ-ಗಳಲ್ಲ
ಕಾರ್ಯ-ಕಲಾ-ಪ-ಗಳು
ಕಾರ್ಯ-ಕಾರಿ-ಯಲ್ಲ-ವೆಂದು
ಕಾರ್ಯ-ಕಾರಿ-ಯಾಗ-ಬೇಕು
ಕಾರ್ಯ-ಕಾರಿ-ಯಾಗಿ
ಕಾರ್ಯ-ಕಾರಿ-ಯಾಗಿ-ತ್ತು
ಕಾರ್ಯ-ಕಾರಿ-ಯಾದ
ಕಾರ್ಯ-ಕ್ಕಾಗಿ
ಕಾರ್ಯಕ್ಕೂ
ಕಾರ್ಯಕ್ಕೆ
ಕಾರ್ಯಕ್ಕೇ
ಕಾರ್ಯ-ಕ್ರಮ-ಗಳಿ-ದ್ದರೂ
ಕಾರ್ಯ-ಕ್ರಮ-ವನ್ನು
ಕಾರ್ಯ-ಕ್ರಮ-ವಿ-ಲ್ಲದೇ
ಕಾರ್ಯ-ಕ್ಷೇತ್ರ-ಗಳಲ್ಲಿ-ರುವ
ಕಾರ್ಯ-ಕ್ಷೇತ್ರ-ಗಳಿಗೆ
ಕಾರ್ಯ-ಕ್ಷೇತ್ರ-ದಲ್ಲಿ
ಕಾರ್ಯ-ಕ್ಷೇತ್ರ-ದಲ್ಲಿಯೂ
ಕಾರ್ಯ-ಕ್ಷೇತ್ರ-ದಲ್ಲಿ-ರಲಿ
ಕಾರ್ಯ-ಕ್ಷೇತ್ರವೂ
ಕಾರ್ಯ-ಗತ
ಕಾರ್ಯ-ಗತ-ಗೊಳಿಸ-ಬೇಕೆಂದು
ಕಾರ್ಯ-ಗಳ
ಕಾರ್ಯ-ಗಳನ್ನು
ಕಾರ್ಯ-ಗಳನ್ನೂ
ಕಾರ್ಯ-ಗಳಲ್ಲಿ
ಕಾರ್ಯ-ಗಳಾಗಿವೆ
ಕಾರ್ಯ-ಗಳಿಂದಲೂ
ಕಾರ್ಯ-ಗಳಿಗೂ
ಕಾರ್ಯ-ಗಳಿ-ರು-ತ್ತವೆ
ಕಾರ್ಯ-ಗಳು
ಕಾರ್ಯ-ಗಳೂ
ಕಾರ್ಯ-ಚಟು-ವಟಿಕೆ-ಗಳಿಗೆ
ಕಾರ್ಯತಃ
ಕಾರ್ಯ-ತತ್ಪರತೆ
ಕಾರ್ಯದ
ಕಾರ್ಯ-ದರ್ಶಿ
ಕಾರ್ಯ-ದಲ್ಲಿ
ಕಾರ್ಯ-ದಲ್ಲಿಯೂ
ಕಾರ್ಯ-ನಿರ್ವ-ಹಣೆ-ಯಲ್ಲಿ
ಕಾರ್ಯ-ರೂಪ-ಕ್ಕಿಳಿಯ-ಬೇಕು
ಕಾರ್ಯ-ರೂಪಕ್ಕೆ
ಕಾರ್ಯ-ವನ್ನು
ಕಾರ್ಯ-ವಲ್ಲ
ಕಾರ್ಯ-ವಾಗಿ-ರ-ಬಹುದು
ಕಾರ್ಯ-ವಾಗು-ತ್ತದೆ
ಕಾರ್ಯವು
ಕಾರ್ಯವೂ
ಕಾರ್ಯ-ವೆ-ಲ್ಲಾ-ಬೌದ್ಧ-ಧರ್ಮ-ದಿಂದ
ಕಾರ್ಯ-ವೇನೋ
ಕಾರ್ಯ-ಶೀಲ-ರಾಗ
ಕಾರ್ಯ-ಶೀಲರು
ಕಾರ್ಯ-ಸಾಧನೆ
ಕಾರ್ಯ-ಸಾಧ-ನೆಗೆ
ಕಾರ್ಯ-ಸಾಧ-ನೆ-ಯಲ್ಲಿ
ಕಾರ್ಯ-ಸಾಧ-ನೆ-ಯಾಗು-ವುದು
ಕಾರ್ಯ-ಹೀಗೇ
ಕಾರ್ಯಾ-ರಂಭ
ಕಾರ್ಯೋ
ಕಾರ್ಯೋ-ತ್ಸಾಹ
ಕಾರ್ಯೋ-ನ್ಮುಖ-ನ-ನ್ನಾಗಿ
ಕಾರ್ಯೋ-ನ್ಮುಖ-ನಾಗ-ಬೇಕು
ಕಾರ್ಯೋ-ನ್ಮುಖ-ರ-ನ್ನಾಗಿ
ಕಾರ್ಯೋ-ನ್ಮುಖ-ರಾಗಿ
ಕಾರ್ಯೋ-ನ್ಮುಖ-ರಾಗೋಣ
ಕಾಲ
ಕಾಲ-ಕಳೆ-ದಂತೆ
ಕಾಲ-ಕ-ಳೆಯು-ತ್ತಿದ್ದ
ಕಾಲ-ಕಾಲಕ್ಕೆ
ಕಾಲಕ್ಕೂ
ಕಾಲಕ್ಕೆ
ಕಾಲ-ಕ್ರಮೇಣ
ಕಾಲ-ಗರ್ಭ-ದಲ್ಲಿ
ಕಾಲ-ಗರ್ಭ-ದಿಂದಲೂ
ಕಾಲ-ಗಳಲ್ಲಿ
ಕಾಲ-ಗಳಿಂದಲೂ
ಕಾಲ-ಚಿಹ್ನೆ-ಯನ್ನು
ಕಾಲ-ಡಿ-ಯಲ್ಲಿ
ಕಾಲದ
ಕಾಲ-ದಲ್ಲಿ
ಕಾಲ-ದಲ್ಲಿಯೂ
ಕಾಲ-ದಲ್ಲಿ-ಯೇ-ರಾಜ-ಕೀಯ
ಕಾಲ-ದಲ್ಲೇ
ಕಾಲ-ದ-ವರಂತಿಲ್ಲ
ಕಾಲ-ದ-ವರೆಗೂ
ಕಾಲ-ದ-ವ-ರೆಗೆ
ಕಾಲ-ದಿಂದ
ಕಾಲ-ದಿಂದಲೂ
ಕಾಲ-ದಿಂದಲೇ
ಕಾಲ-ದೇಶ
ಕಾಲ-ದೇಶ-ಕಾರ-ಣ-ಗಳೆಲ್ಲಾ
ಕಾಲ-ದೇಶ-ಗಳನ್ನು
ಕಾಲ-ದೇಶ-ಗಳಿಗೆ
ಕಾಲ-ದೇಶ-ನಿಮಿತ್ತ-ಗಳು
ಕಾಲ-ದೇಶ-ಪಂಗಡ-ಗಳಿಗೆ
ಕಾಲರ
ಕಾಲ-ವನ್ನು
ಕಾಲ-ವನ್ನೂ
ಕಾಲ-ವನ್ನೆಲ್ಲಾ
ಕಾಲ-ವಾಗ-ಬಹು-ದಾದ
ಕಾಲ-ವಾಗಿ
ಕಾಲ-ವಾಗು-ವಿರಿ
ಕಾಲ-ವಾಗು-ವು-ದಕ್ಕೆ
ಕಾಲ-ವಾದ
ಕಾಲ-ವಾದರೂ
ಕಾಲ-ವಾದರೆ
ಕಾಲ-ವಿತ್ತು
ಕಾಲ-ವಿದೆ
ಕಾಲ-ವಿದ್ದು
ಕಾಲ-ವಿರು-ತ್ತದೆ
ಕಾಲ-ವಿರು-ವುದು
ಕಾಲ-ವೀಗ
ಕಾಲವು
ಕಾಲವೂ
ಕಾಲವೇ
ಕಾಲ-ವೊಂದಿತ್ತು
ಕಾಲಾ-ತೀತ-ರಾಗಿ
ಕಾಲಾ-ನಂತರ
ಕಾಲಾ-ನಂತರ-ದಲ್ಲಿ
ಕಾಲಾವ-ಕಾಶ
ಕಾಲಾವ-ಕಾಶ-ವಿಲ್ಲ
ಕಾಲಿಟ್ಟ
ಕಾಲಿಟ್ಟಿದೆ
ಕಾಲಿಡುತ್ತಿರು
ಕಾಲಿನ
ಕಾಲಿನ-ಮೇಲೆ
ಕಾಲಿ-ನಿಂದ
ಕಾಲಿಲ್ಲದೇ
ಕಾಲು
ಕಾಲು-ಗಳ
ಕಾಲು-ಗಳೇ
ಕಾಲು-ಪಾಲು
ಕಾಲು-ವೆಯ
ಕಾಲೇಜಿನ
ಕಾಲ್ಪನಿಕ
ಕಾಲ್ಪನಿಕ-ವಾದದ್ದು
ಕಾಳಿ-ದಾ-ಸನ
ಕಾವ್ಯ
ಕಾವ್ಯ-ಗಳಲ್ಲೂ
ಕಾವ್ಯ-ಗಳು
ಕಾವ್ಯದ
ಕಾವ್ಯ-ಮಯ-ವಾಗಿ
ಕಾವ್ಯ-ರಸ-ವನ್ನು
ಕಾವ್ಯ-ವನ್ನು
ಕಾವ್ಯ-ವಿದು
ಕಾವ್ಯವು
ಕಾಶಿಗೆ
ಕಾಶಿ-ಯನ್ನು
ಕಾಶ್ಮೀರ
ಕಾಶ್ಮೀರ-ದಲ್ಲಿ
ಕಾಸು
ಕಿಂಚ
ಕಿಂವದಂತಿ
ಕಿಕ್ಕಿರಿದು
ಕಿಡಿ-ಗಳೂ
ಕಿಡಿ-ಯೆಂದೂ
ಕಿತ್ತು
ಕಿತ್ತು-ಕೊಳ್ಳದೆ
ಕಿತ್ತೊಗೆದಿರು-ವು-ದನ್ನು
ಕಿತ್ತೊಗೆದು
ಕಿತ್ತೊಗೆಯ-ಬಲ್ಲಿರಾ
ಕಿತ್ತೊಗೆ-ಯಲು
ಕಿತ್ತೊಗೆ-ಯಿ-ತೆಂದರೆ
ಕಿತ್ತೊಗೆ-ಯಿರಿ
ಕಿರಣ-ಗಳು
ಕಿರಾ-ತರ
ಕಿರಾ-ತ-ರನ್ನು
ಕಿರಿ-ದಾದ
ಕಿರಿ-ದಾದು-ದ-ರಿಂದಲೂ
ಕಿರಿಯ
ಕಿರೀಟಪ್ರಾಯ-ವಾದ
ಕಿರೀಟ-ವನ್ನು
ಕಿರು
ಕಿರು-ದೆರೆ-ಯಂತೆ
ಕಿವಿ
ಕಿವಿ-ಗಳು-ಹೀಗೆ
ಕಿವಿ-ಗಳೇ
ಕಿವಿಗೆ
ಕಿವಿ-ಗೊಡಿ
ಕಿವಿ-ಗೊಡುವಿ-ರೇನು
ಕಿವಿ-ಯಲ್ಲ
ಕೀಟಕ್ಕೂ
ಕೀಟದ
ಕೀಟ-ದಲ್ಲಿಯೂ
ಕೀಟ-ದಿಂದ
ಕೀಟ-ವಾ-ವುದು
ಕೀರ್ತಿ
ಕೀರ್ತಿ-ಗಳಾಗಲಿ
ಕೀರ್ತಿಯ
ಕೀರ್ತಿ-ಯನ್ನು
ಕೀರ್ತಿ-ಯಾ-ಸೆಗೆ
ಕೀರ್ತಿ-ಶೇಷ-ರಾದ
ಕೀಳಲು
ಕೀಳು
ಕೀಳ್ಮೆ-ಯಿಂದ
ಕುಂಠಿತ-ವಾ-ಯಿತು
ಕುಂದು
ಕುಂದು-ಕೊರತೆ-ಗಳಿ-ಗಾಗಿ
ಕುಂದು-ಕೊರತೆ-ಗಳಿ-ರು-ವುದೂ
ಕುಂದು-ಕೊರತೆ-ಗಳಿವೆ
ಕುಂದು-ಕೊರತೆ-ಗಳು
ಕುಂದು-ತ್ತಿದೆ
ಕುಂಬಾ-ರನು
ಕುಂಬಾ-ರನೂ
ಕುಂಭಕೋ-ಣಕ್ಕೆ
ಕುಂಭಕೋ-ಣದ
ಕುಕರ್ಮಿ
ಕುಗ್ಗಿ
ಕುಗ್ಗಿದ
ಕುಗ್ಗಿ-ಸು-ವುದು
ಕುಟೀರ-ದಲ್ಲಿ
ಕುಠಾರ
ಕುಡಿ-ಯ-ಬೇಕೆಂದು
ಕುಡಿ-ಯ-ಬೇಕೇ
ಕುಡಿ-ಯಲು
ಕುಡಿ-ಯುತ್ತಾನೆ
ಕುಡಿ-ಯುದೇ
ಕುಡಿ-ಯುವ
ಕುಡಿ-ಯು-ವು-ದಲ್ಲ
ಕುಡಿ-ಯು-ವುದು
ಕುಡಿ-ಯು-ವುದೇ
ಕುಡು-ಕರು
ಕುಣಿಯು-ತ್ತಿದ್ದನು
ಕುತರ್ಕ
ಕುತುಮಿ
ಕುತೂಹಲ
ಕುತೂ-ಹಲ-ವನ್ನು
ಕುತೂಹಲ-ವಿಲ್ಲ
ಕುತೂಹಲ-ವಿಲ್ಲ-ವೆಂತಲೂ
ಕುತೂಹಲ-ವೆಲ್ಲ
ಕುತೂಹಲಿ-ಗಳಾಗಿ-ರು-ವರು
ಕುತೂಹಲಿ-ಗಳು
ಕುತೋ-ಽಯಮಗ್ನಿಃ
ಕುತ್ಯೋ-ಽಯಮಗ್ನಿಃ
ಕುತ್ಸಿತ
ಕುದಿಯು-ತ್ತಿದ್ದ
ಕುದುರೆ
ಕುದುರೆ-ಗಳನ್ನು
ಕುದುರೆ-ಗಳಿಗೂ
ಕುದುರೆಯ
ಕುಬೇ-ರನ
ಕುಬ್ಜ-ವಾಗು-ವುದು
ಕುಮಾ-ನಿನ
ಕುಮಾರ
ಕುಮಾ-ರರು
ಕುಮಾ-ರಿಲ
ಕುಮಾ-ರಿ-ಲನು
ಕುಮಾರೀ
ಕುರಿಗಳಂತಿದ್ದ
ಕುರಿತ
ಕುರಿತಂತೆ
ಕುರಿತ-ದ್ದಾದರೆ
ಕುರಿತಾದ
ಕುರಿತು
ಕುರಿ-ತು-ಚಿಂತಿಸು-ವುದು
ಕುರಿ-ತು-ಮಾತ
ಕುರುಡ
ಕುರುಡ-ನಾ-ದರೂ
ಕುರುಡ-ನಿಗೆ
ಕುರುಡನು
ಕುರುಡ-ರಂತೆ
ಕುರುಡ-ರಾಗಿ-ರುವ-ವ-ರಿಗೂ
ಕುರುಡ-ರಿಗೆ
ಕುರುಡರು
ಕುರುಹು-ಗಳು
ಕುರುಹೂ
ಕುಲ
ಕುಲಕ್ಕೆ
ಕುಲ-ಗುರು
ಕುಲ-ಗುರು-ಗಳೆಂದು
ಕುಲ-ಗೆಟ್ಟ
ಕುಲ-ಗೆಟ್ಟು
ಕುಲ-ಗೋತ್ರಕ್ಕೇ
ಕುಲ-ದಿಂದ
ಕುಲ-ವೀರ
ಕುಲವೇ
ಕುಲೀನ
ಕುಲೀ-ನರು
ಕುಲುಕಿ-ಕೊಂಡು
ಕುಲೋದ್ಭ-ವನು
ಕುಳಿ-ತನು
ಕುಳಿ-ತರೆ
ಕುಳಿ-ತಿದೆ
ಕುಳಿ-ತಿದೆ-ಯೆಂದೂ
ಕುಳಿತಿ-ದ್ದನು
ಕುಳಿತಿದ್ದರೂ
ಕುಳಿತಿ-ರುವ
ಕುಳಿತಿರು-ವುದು
ಕುಳಿತು
ಕುಳಿ-ತು-ಕೊಳ್ಳ-ದಂತೆ
ಕುಳಿ-ತು-ಕೊಳ್ಳು-ತ್ತದೆ
ಕುಳಿ-ತು-ಕೊಳ್ಳುವ
ಕುಳಿ-ತು-ಕೊಳ್ಳು-ವು-ದಿಲ್ಲ
ಕುಳ್ಳಿರಿಸ-ಬಹುದು
ಕುವೆಂಪು
ಕುಶಲತೆ-ಯನ್ನು
ಕುಶಲ-ಪ್ರಶ್ನೆ-ಯನ್ನು
ಕುಶಲಿ-ಯಾದ
ಕುಷ್ಠ-ನಂತೆ
ಕುಸಂಸ್ಕಾರ-ಗಳ
ಕುಸಿದು
ಕುಸಿದು-ಬಿಟ್ಟರೆ
ಕುಸಿದು-ಬಿದ್ದಂತೆ
ಕುಸಿದು-ಹೋ-ಗು-ವುದು
ಕುಸಿಯುತ್ತಲೂ
ಕುಸಿ-ಯು-ತ್ತಿದೆ
ಕುಸಿಯುತ್ತಿ-ರುವ
ಕುಸ್ತಿ
ಕೂಗ-ಬೇಕಾಗಿದೆ
ಕೂಗಿ
ಕೂಗಿ-ನಿಂದ
ಕೂಗು
ಕೂಗುತ್ತಾ
ಕೂಗು-ತ್ತಿ-ರು-ವರು
ಕೂಗು-ತ್ತೇವೆ
ಕೂಗು-ವಂತಾ-ಗು-ತ್ತದೆ
ಕೂಗೆಬ್ಬಿಸುತ್ತಾನೆ
ಕೂಗೇ
ಕೂಡ
ಕೂಡದು
ಕೂಡಲೆ
ಕೂಡಲೇ
ಕೂಡಾ
ಕೂಡಿ
ಕೂಡಿಟ್ಟು
ಕೂಡಿದ
ಕೂಡಿದೆ
ಕೂಡಿ-ರುವ
ಕೂಡಿ-ರು-ವು-ದ-ರಿಂದ
ಕೂಡಿ-ರು-ವುದು
ಕೂಡಿ-ರು-ವುದೋ
ಕೂದಲಿ-ನವ-ನಾದರೆ
ಕೂದಲಿನ-ವರು
ಕೂದಲಿ-ನವ-ರೆಂದೂ
ಕೂದಲು
ಕೂಪ
ಕೂಪ-ದಲ್ಲಿ-ರು-ವರು
ಕೂಪ-ಮಂಡೂಕ
ಕೂರಿಸಿ
ಕೂಲಂಕಷ-ವಾಗಿ
ಕೂಲಿ-ಗಾರ-ರಿಗೆ
ಕೂಲಿಗೂ
ಕೂಲಿ-ಯನ್ನು
ಕೂಲಿಯೂ
ಕೃತಕ
ಕೃತ-ಘ್ನ-ನಾಗು-ತ್ತೇನೆ
ಕೃತ-ಘ್ನ-ರನ್ನು
ಕೃತ-ಜ್ಞತಾ-ಪೂರ್ವ-ಕ-ವಾಗಿ
ಕೃತ-ಜ್ಞತಾ-ಪೂರ್ವ-ಕ-ವಾಗಿಯೂ
ಕೃತ-ಜ್ಞತಾ-ಭಾವ-ಗಳೆ-ರಡೂ
ಕೃತ-ಜ್ಞತೆ
ಕೃತ-ಜ್ಞತೆ-ಗಳನ್ನು
ಕೃತ-ಜ್ಞ-ತೆಗೆ
ಕೃತ-ಜ್ಞ-ತೆಯ
ಕೃತ-ಜ್ಞತೆ-ಯನ್ನು
ಕೃತ-ಜ್ಞತೆ-ಯಿಂದ
ಕೃತ-ಜ್ಞತೆ-ಯಿಂದಲೂ
ಕೃತ-ಜ್ಞನು
ಕೃತ-ಜ್ಞ-ರಾಗಿ-ದ್ದರೂ
ಕೃತ-ಜ್ಞ-ರಾಗಿ-ದ್ದೇವೆ
ಕೃತ-ಜ್ಞ-ರಾಗಿ-ರ-ಬೇಕು
ಕೃತ-ಜ್ಞ-ರಾಗಿ-ರು-ತ್ತಿದ್ದೆವು
ಕೃತ-ಜ್ಞ-ವಾಗು-ತ್ತದೆ
ಕೃತ-ವಿದ್ಯ-ರಾದ
ಕೃತ-ವಿದ್ಯರು
ಕೃತಿ
ಕೃತ್ಯ
ಕೃತ್ಯ-ವನ್ನು
ಕೃತ್ರಿಮತೆ
ಕೃಪಾ-ಸಾಗರ-ನಾಗಿ-ರ-ಬೇಕು
ಕೃಪೆ
ಕೃಪೆ-ಗಾಗಿ
ಕೃಪೆ-ದೋರಿ-ದ್ದೀರಿ
ಕೃಪೆಯ
ಕೃಪೆ-ಯಿಂದ
ಕೃಪೆ-ಯಿಟ್ಟು
ಕೃಶ-ಳಾದ
ಕೃಷಿ-ಪ್ರ-ಧಾನ
ಕೃಷ್ಣ
ಕೃಷ್ಣನ
ಕೃಷ್ಣ-ನ-ಪರಿ-ಚಯ
ಕೃಷ್ಣ-ನಲ್ಲ
ಕೃಷ್ಣ-ನಾದರೋ
ಕೃಷ್ಣ-ನಿಗೆ
ಕೃಷ್ಣನು
ಕೃಷ್ಣ-ನೆಂದು
ಕೃಷ್ಣನೇ
ಕೃಷ್ಣ-ನೊಬ್ಬನ
ಕೃಷ್ಣ-ನೊಬ್ಬ-ನ-ನ್ನೇ
ಕೃಷ್ಣ-ಸ್ತು
ಕೃಷ್ಣಾವ-ತಾರದ
ಕೆಂಪು
ಕೆಚ್ಚನ್ನು
ಕೆಚ್ಚಿನ
ಕೆಟ್ಟ
ಕೆಟ್ಟ-ದಕ್ಕೂ
ಕೆಟ್ಟ-ದಕ್ಕೆ
ಕೆಟ್ಟ-ದಕ್ಕೊ
ಕೆಟ್ಟ-ದಕ್ಕೋ
ಕೆಟ್ಟ-ದರ
ಕೆಟ್ಟ-ದೆಂದು
ಕೆಟ್ಟ-ದೇನೂ
ಕೆಟ್ಟ-ದ್ದನ್ನು
ಕೆಟ್ಟ-ದ್ದಲ್ಲ
ಕೆಟ್ಟ-ದ್ದಲ್ಲ-ವೆಂಬ
ಕೆಟ್ಟದ್ದು
ಕೆಟ್ಟದ್ದೂ
ಕೆಟ್ಟದ್ದೆ
ಕೆಟ್ಟದ್ದೊ
ಕೆಡಹಿ
ಕೆಡಿಸಿ
ಕೆಡು-ವುದು
ಕೆತ್ತ-ನೆ-ಗಳನ್ನು
ಕೆತ್ತಿದ
ಕೆರಳಿ
ಕೆರಳಿ-ಸು-ವಂತಹ
ಕೆರಳಿ-ಸು-ವು-ದಿಲ್ಲ
ಕೆರೆಗೂ
ಕೆರೆ-ದರೆ
ಕೆರೆಯ
ಕೆರೆ-ಯಲ್ಲಿ
ಕೆರೆ-ಯಿಂದ
ಕೆಲ-ಕಾಲ-ವಿದ್ದು
ಕೆಲ-ವನ್ನು
ಕೆಲ-ವರ
ಕೆಲ-ವ-ರನ್ನು
ಕೆಲವ-ರಲ್ಲಿ
ಕೆಲ-ವರ-ಲ್ಲಿದೆ
ಕೆಲವರಿಗಂತೂ
ಕೆಲವ-ರಿಗೆ
ಕೆಲ-ವರು
ಕೆಲ-ವರು-ಮಧ್ಯ
ಕೆಲವು
ಕೆಲವು-ಕಾಲ
ಕೆಲವೇ
ಕೆಲವೊಮ್ಮೆ
ಕೆಲಸ
ಕೆಲಸ-ಕಾರ್ಯ-ಗಳನ್ನು
ಕೆಲಸ-ಕ್ಕಾಗಿ
ಕೆಲಸ-ಕ್ಕಿಂತ
ಕೆಲಸಕ್ಕೂ
ಕೆಲ-ಸಕ್ಕೆ
ಕೆಲಸ-ಗಳ-ನ್ನೆಲ್ಲಾ
ಕೆಲಸ-ದಲ್ಲಿ
ಕೆಲಸ-ಬೇಕೊ
ಕೆಲಸ-ಮಾಡದ
ಕೆಲಸ-ಮಾಡ-ಬೇಕು
ಕೆಲಸ-ಮಾಡಲು
ಕೆಲಸ-ಮಾಡಿ
ಕೆಲಸ-ಮಾಡಿರಿ
ಕೆಲಸ-ಮಾಡು-ತ್ತಿದೆ
ಕೆಲಸ-ಮಾಡು-ತ್ತಿ-ರುವ-ನೆಂದೂ
ಕೆಲಸ-ಮಾಡು-ತ್ತಿರು-ವು-ದನ್ನು
ಕೆಲಸ-ಮಾಡು-ತ್ತಿವೆ
ಕೆಲಸ-ಮಾಡುವ
ಕೆಲಸ-ಮಾಡು-ವು-ದಕ್ಕೆ
ಕೆಲಸ-ಮಾಡು-ವು-ದನ್ನು
ಕೆಲಸ-ಮಾಡು-ವು-ದಿಲ್ಲ
ಕೆಲಸ-ಮಾಡು-ವುದು
ಕೆಲಸ-ವನ್ನಾ-ದರೂ
ಕೆಲಸ-ವನ್ನು
ಕೆಲಸ-ವನ್ನೇನೋ
ಕೆಲಸ-ವಲ್ಲ
ಕೆಲಸ-ವ-ಲ್ಲದೆ
ಕೆಲಸ-ವಾಗಿ-ದ್ದರೆ
ಕೆಲಸ-ವಾ-ವುದು
ಕೆಲಸ-ವಿತ್ತು
ಕೆಲಸ-ವಿದು
ಕೆಲಸ-ವೆಲ್ಲ
ಕೆಲ-ಸವೇ
ಕೆಲಸ-ವೊಂದಿರು-ವುದು
ಕೆಲಸವೋ
ಕೆಳ-ಕಂಡ
ಕೆಳ-ಗಿನ
ಕೆಳಗಿ-ನಂತೆ
ಕೆಳ-ಗಿನದು
ಕೆಳ-ಗಿನ-ವ-ರನ್ನು
ಕೆಳಗಿ-ರುವ
ಕೆಳಗಿಳಿಸಿ
ಕೆಳ-ಗುರು-ಳು-ವನು
ಕೆಳಗೆ
ಕೆಳಮಟ್ಟದ
ಕೆಳಮಟ್ಟದ್ದು
ಕೆಸ-ರನ್ನು
ಕೆಸ-ರಿನ
ಕೇಂದ್ರ
ಕೇಂದ್ರ-ಗಳ
ಕೇಂದ್ರ-ಗಳನ್ನು
ಕೇಂದ್ರ-ಗಳು
ಕೇಂದ್ರ-ದಲ್ಲಿ
ಕೇಂದ್ರ-ದಲ್ಲಿಯೇ
ಕೇಂದ್ರ-ದಿಂದ
ಕೇಂದ್ರ-ವನ್ನು
ಕೇಂದ್ರ-ವಾಗಿ
ಕೇಂದ್ರ-ವಾಗಿದೆ
ಕೇಂದ್ರ-ವಾದ
ಕೇಂದ್ರ-ವಿದೆ
ಕೇಂದ್ರವು
ಕೇಂದ್ರಿಕ-ರಿಸು-ವುದು
ಕೇಂದ್ರಿಕ-ರೀಸು-ವು-ದ-ರಿಂದ
ಕೇಂದ್ರೀ-ಕ-ರಿಸಿ
ಕೇಂದ್ರೀ-ಕೃತ-ವಾಗಿದೆ
ಕೇಂದ್ರೀ-ಕೃತ-ವಾಗಿ-ರು-ತ್ತವೆ
ಕೇಂದ್ರೀ-ಕೃತ-ವಾಗು-ವುವು
ಕೇಡನ್ನು
ಕೇಡಾ-ಗಲೀ
ಕೇಡು
ಕೇಡು-ಗಳನ್ನು
ಕೇಡು-ಗಾಲ
ಕೇತುವೇ
ಕೇನ
ಕೇಳದ-ವ-ರನ್ನು
ಕೇಳ-ದಿ-ದ್ದರೆ
ಕೇಳದೆ
ಕೇಳ-ಬಹುದು
ಕೇಳ-ಬೇಕು
ಕೇಳ-ಲಾ-ಗಿದೆ
ಕೇಳಲಿ
ಕೇಳ-ಲಿಲ್ಲ
ಕೇಳಲು
ಕೇಳಲೂ
ಕೇಳಿ
ಕೇಳಿ-ಕೊಳ್ಳ-ಬೇಕು
ಕೇಳಿ-ಕೊಳ್ಳು-ತ್ತೇನೆ
ಕೇಳಿತು
ಕೇಳಿದ
ಕೇಳಿ-ದನು
ಕೇಳಿ-ದರು
ಕೇಳಿ-ದರೂ
ಕೇಳಿ-ದರೆ
ಕೇಳಿ-ದ-ವ-ರಿಲ್ಲ
ಕೇಳಿ-ದಾಗ
ಕೇಳಿ-ದಾ-ಗಲೂ
ಕೇಳಿ-ದಾ-ಗಿ-ನಿಂದಲೂ
ಕೇಳಿದೆ
ಕೇಳಿ-ದೊ-ಡ-ನೆಯೇ
ಕೇಳಿ-ದ್ದೀರಾ
ಕೇಳಿ-ದ್ದೀರಿ
ಕೇಳಿದ್ದೆ
ಕೇಳಿ-ದ್ದೇನೆ
ಕೇಳಿ-ದ್ದೇವೆ
ಕೇಳಿ-ಬರು-ತ್ತದೆ
ಕೇಳಿ-ಬರು-ತ್ತಿದೆ
ಕೇಳಿ-ರ-ಬಹುದು
ಕೇಳಿ-ರುವಿರಾ
ಕೇಳಿ-ರು-ವಿರಿ
ಕೇಳಿ-ರು-ವೆವೋ
ಕೇಳಿಲ್ಲ
ಕೇಳಿ-ಸದೇ
ಕೇಳಿ-ಸಿತು
ಕೇಳಿ-ಸು-ವುದು
ಕೇಳು
ಕೇಳುತ್ತ
ಕೇಳು-ತ್ತವೆ
ಕೇಳು-ತ್ತಾನೆ
ಕೇಳು-ತ್ತಾರೆ
ಕೇಳು-ತ್ತಿದೆ
ಕೇಳು-ತ್ತಿದ್ದರು
ಕೇಳು-ತ್ತಿದ್ದೆ
ಕೇಳು-ತ್ತಿ-ರಲ್ಲ
ಕೇಳು-ತ್ತಿ-ರು-ವರು
ಕೇಳು-ತ್ತಿರು-ವೆನು
ಕೇಳು-ತ್ತೀ-ರಲ್ಲ
ಕೇಳು-ತ್ತೇನೆ
ಕೇಳುವ
ಕೇಳು-ವಂತೆ
ಕೇಳು-ವರು
ಕೇಳು-ವು-ದಕ್ಕೆ
ಕೇಳು-ವುದಕ್ಕೇ
ಕೇಳು-ವುದು
ಕೇಳು-ವುದೊ
ಕೇಳು-ವೆನು
ಕೇಳು-ವೆವು
ಕೇವಲ
ಕೇವಲ-ವಾಗಿ
ಕೇಶ-ರಾಶಿ-ಯಿಂದ
ಕೇಸರಿ
ಕೇಸರಿಯ
ಕೇಸು-ಗಳು
ಕೈ
ಕೈಕಟ್ಟಿ-ಕೊಂಡು
ಕೈಕಾಲು-ಗಳು
ಕೈಕುಲ-ಕಿಸು-ವು-ದನ್ನು
ಕೈಕೆಳಗೆ
ಕೈಗಳನ್ನು
ಕೈಗಳಿಂದಲೂ
ಕೈಗಳೂ
ಕೈಗಳೇ
ಕೈಗೂ-ಡುತ್ತವೆ
ಕೈಗೂಡು-ವುದು
ಕೈಗೆ
ಕೈಗೊಂಡ
ಕೈಗೊಂಡಿ-ದ್ದೀರಿ
ಕೈಗೊಂಡಿ-ರುವ
ಕೈಗೊಂಡಿರು-ವಿರಿ
ಕೈಗೊಂಡು
ಕೈಗೊಳ್ಳ-ದಿದ್ದ
ಕೈಗೊಳ್ಳುವ
ಕೈಗೊಳ್ಳು-ವುದು
ಕೈದೀ-ವಿಗೆ-ಯಾಗುವು
ಕೈನೀಡಿ
ಕೈಬಿಡ-ಬೇಕೆಂದು
ಕೈಮೇಲೆ
ಕೈಯನ್ನು
ಕೈಯಲ್ಲಿ
ಕೈಯಲ್ಲಿ-ರು-ವುದು
ಕೈಯಿಂದ
ಕೈಯಿ-ಲ್ಲದೆ
ಕೈಲಾ-ಗದ-ವನು
ಕೈಲಾ-ಗದ-ವ-ರಿಗೆ
ಕೈಲಾ-ಗು-ವು-ದಿಲ್ಲ
ಕೈಲಾದ
ಕೈಲಾದುದ್ದನ್ನೆಲ್ಲಾ
ಕೈಹಾ-ಕ-ಬೇಕಾ-ಯಿತು
ಕೈಹಾ-ಕಲು
ಕೈಹಾಕಿ
ಕೈಹಾ-ಕಿ-ದರೆ
ಕೈಹಾ-ಕಿ-ರುವ
ಕೈಹಾ-ಕು-ವು-ದಿಲ್ಲ
ಕೈಹಾ-ಕು-ವುದು
ಕೈಹಿ-ಡಿದು
ಕೈಹಿ-ಡಿದೆ-ತ್ತುವ
ಕೊಂಚ
ಕೊಂಚವೂ
ಕೊಂಡ
ಕೊಂಡಾ-ಡಲು
ಕೊಂಡಾ-ಡಿವೆ
ಕೊಂಡಾಡಿಸಿ
ಕೊಂಡಿ
ಕೊಂಡಿದ್ದ
ಕೊಂಡಿ-ಯನ್ನು
ಕೊಂಡಿರಿ
ಕೊಂಡಿ-ಲ್ಲ-ವೆಂದು
ಕೊಂಡೊ-ಯ್ದಿದೆ
ಕೊಂಡೊ-ಯ್ಯಲೂ-ಬಹುದು
ಕೊಂಡೊ-ಯ್ಯವ
ಕೊಂಡೊ-ಯ್ಯಿರಿ
ಕೊಂಡೊ-ಯ್ಯು-ತ್ತದೆ
ಕೊಂಡೊ-ಯ್ಯುವ
ಕೊಂದರೆ
ಕೊಂದಲ್ಲದೆ
ಕೊಂದ-ವ-ರನ್ನು
ಕೊಂದು
ಕೊಚ್ಚಿ-ಕೊಂಡು
ಕೊಚ್ಚಿ-ಕೊಂಡು-ಹೋ-ಗಿದೆ
ಕೊಚ್ಚಿ-ಕೊಳ್ಳಲಿ
ಕೊಚ್ಚಿ-ಕೊಳ್ಳು-ತ್ತಿ-ರುವ
ಕೊಚ್ಚಿ-ಕೊಳ್ಳುವ
ಕೊಚ್ಚಿ-ಕೊಳ್ಳು-ವುದು
ಕೊಚ್ಚಿತು
ಕೊಟ್ಟ
ಕೊಟ್ಟರು
ಕೊಟ್ಟರೂ
ಕೊಟ್ಟರೆ
ಕೊಟ್ಟ-ವರು
ಕೊಟ್ಟ-ಷ್ಟ-ರಲ್ಲಿ
ಕೊಟ್ಟಿಗೆ-ಯಲ್ಲಿ
ಕೊಟ್ಟಿತು
ಕೊಟ್ಟಿದೆ
ಕೊಟ್ಟಿದ್ದ-ಕ್ಕಾಗಿ
ಕೊಟ್ಟಿ-ದ್ದಾರೆ
ಕೊಟ್ಟಿ-ರುವ
ಕೊಟ್ಟಿ-ರು-ವನು
ಕೊಟ್ಟಿ-ರು-ವರು
ಕೊಟ್ಟಿರು-ವುದು
ಕೊಟ್ಟಿರು-ವೆವು
ಕೊಟ್ಟು
ಕೊಟ್ಟು-ದೆಲ್ಲ-ವನ್ನೂ
ಕೊಟ್ಟೆ
ಕೊಟ್ಟೆವೋ
ಕೊಡ-ಕೂಡದು
ಕೊಡದ
ಕೊಡದೆ
ಕೊಡದೇ
ಕೊಡ-ಬಲ್ಲದು
ಕೊಡ-ಬಲ್ಲವು
ಕೊಡ-ಬಹುದು
ಕೊಡ-ಬಹು-ದೆಂದು
ಕೊಡ-ಬಾ-ರದು
ಕೊಡ-ಬೇಕಾಗಿ-ತ್ತು
ಕೊಡ-ಬೇಕಾಗಿದೆ
ಕೊಡ-ಬೇಕಾಗಿ-ದೆಯೋ
ಕೊಡ-ಬೇಕಾ-ಗು-ತ್ತದೆ
ಕೊಡ-ಬೇ-ಕಾದ
ಕೊಡ-ಬೇ-ಕಾದರೆ
ಕೊಡ-ಬೇಕಾ-ಯಿತು
ಕೊಡ-ಬೇಕು
ಕೊಡ-ಬೇಡಿ
ಕೊಡ-ಲಾ-ರದು
ಕೊಡ-ಲಾ-ರರು
ಕೊಡ-ಲಿಲ್ಲ
ಕೊಡಲು
ಕೊಡಲೇ-ಬೇಕು
ಕೊಡಹಿ
ಕೊಡಿ
ಕೊಡಿ-ಅದು
ಕೊಡಿ-ಎಂದರು
ಕೊಡು
ಕೊಡುಗೆ
ಕೊಡು-ಗೆ-ಯಾಗಿ
ಕೊಡು-ತ್ತದೆ
ಕೊಡು-ತ್ತದೆಯೋ
ಕೊಡು-ತ್ತವೆ
ಕೊಡು-ತ್ತಿದ್ದನು
ಕೊಡು-ತ್ತಿದ್ದೆವು
ಕೊಡು-ತ್ತಿ-ರು-ವರು
ಕೊಡು-ತ್ತಿರು-ವೆವು
ಕೊಡು-ತ್ತೇನೆ
ಕೊಡು-ತ್ತೇವೆ
ಕೊಡುವ
ಕೊಡು-ವಂತೆ
ಕೊಡು-ವನು
ಕೊಡು-ವರು
ಕೊಡು-ವ-ವನು
ಕೊಡು-ವು-ದಕ್ಕೆ
ಕೊಡು-ವು-ದಕ್ಕೋ
ಕೊಡು-ವು-ದನ್ನು
ಕೊಡು-ವು-ದಿಲ್ಲ
ಕೊಡು-ವುದು
ಕೊಡು-ವುದೂ
ಕೊಡು-ವುದೆಂದರ್ಥ
ಕೊಡು-ವುದೋ
ಕೊಡೋಣ
ಕೊತ್ತಲ-ಗಳಲ್ಲಿ
ಕೊನೆ
ಕೊನೆ-ಗಾಣ-ಕೂಡದು
ಕೊನೆ-ಗಾಣ-ಬೇಕು
ಕೊನೆ-ಗಾಣಲಿ
ಕೊನೆ-ಗಾಣಿಸಿ
ಕೊನೆ-ಗಾಣು-ತ್ತದೆ
ಕೊನೆ-ಗಾಣು-ತ್ತದೆಯೋ
ಕೊನೆ-ಗಾಣುತ್ತ-ವೆಯೋ
ಕೊನೆ-ಗಾಣು-ತ್ತಿದೆ
ಕೊನೆ-ಗಾಣು-ವಂತೆ
ಕೊನೆ-ಗಾಣು-ವು-ದಕ್ಕೆ
ಕೊನೆ-ಗಾಣು-ವುದು
ಕೊನೆ-ಗಾಣು-ವುವು
ಕೊನೆ-ಗಾಲ
ಕೊನೆ-ಗಾಲ-ದಲ್ಲಿ
ಕೊನೆ-ಗಾಲ-ವನ್ನು
ಕೊನೆ-ಗಾಲವು
ಕೊನೆಗೆ
ಕೊನೆ-ಗೊಂಡಿದೆಯೋ
ಕೊನೆ-ಗೊಂಡು
ಕೊನೆ-ಗೊಳ್ಳುತ್ತಿತ್ತು
ಕೊನೆಯ
ಕೊನೆ-ಯಂತೆ
ಕೊನೆ-ಯ-ದನ್ನು
ಕೊನೆ-ಯ-ದಾಗಿ
ಕೊನೆ-ಯದು
ಕೊನೆ-ಯದೇ
ಕೊನೆ-ಯಲ್ಲಿ
ಕೊನೆ-ಯ-ಲ್ಲಿ-ರು-ವೆನು
ಕೊನೆ-ಯ-ವ-ನಲ್ಲ
ಕೊನೆ-ಯ-ವರೆಗೂ
ಕೊನೆ-ಯ-ವ-ರೆಗೆ
ಕೊರತೆ
ಕೊರತೆ-ಗಳಿ-ದ್ದರೂ
ಕೊರತೆ-ಗಳಿ-ದ್ದರೆ
ಕೊರತೆ-ಗಳೆಲ್ಲಾ
ಕೊರ-ತೆಯ
ಕೊರತೆ-ಯನ್ನು
ಕೊರತೆ-ಯಿ-ದ್ದರೆ
ಕೊರತೆ-ಯಿ-ರ-ಲಿಲ್ಲ
ಕೊರ-ತೆಯು
ಕೊಲಂಬೊ
ಕೊಲಂಬೊಗೆ
ಕೊಲಂಬೊದ
ಕೊಲಂಬೋದ
ಕೊಲಂಬೋ-ದಲ್ಲಿ
ಕೊಲಂಬೋ-ನಗ-ರದ
ಕೊಲೆ
ಕೊಲೆ-ಪಾತ-ಕರು
ಕೊಲೆ-ಮಾಡಿ-ದಾಗ
ಕೊಲ್ಲದೆ
ಕೊಲ್ಲ-ಬಲ್ಲರು
ಕೊಲ್ಲಲಾ-ರವು
ಕೊಲ್ಲಲು
ಕೊಲ್ಲು-ತ್ತದೆ
ಕೊಲ್ಲುತ್ತಾನೆ
ಕೊಲ್ಲು-ತ್ತಿದ್ದರು
ಕೊಲ್ಲು-ತ್ತಿ-ರಲಿಲ್ಲ
ಕೊಲ್ಲು-ತ್ತೇನೆ
ಕೊಲ್ಲುವ
ಕೊಲ್ಲು-ವರು
ಕೊಲ್ಲು-ವು-ದಿಲ್ಲ
ಕೊಲ್ಲು-ವುದು
ಕೊಲ್ಲು-ವೆಯಾ
ಕೊಳೆ
ಕೊಳೆತು
ಕೊಳೆ-ಯನ್ನು
ಕೊಳೆ-ಯಬೇಕಾ-ಗು-ವು-ದೆಂದು
ಕೊಳೆ-ಯಿಂದ
ಕೊಳೆ-ಯು-ವುದು
ಕೊಳ್ಳಲೇ
ಕೊಳ್ಳಿ
ಕೊಳ್ಳು-ವು-ದ-ಕ್ಕಿಂತ
ಕೊಳ್ಳು-ವುದನ್ನೇ
ಕೊಳ್ಳು-ವು-ದಾದರೆ
ಕೊಳ್ಳೆಹೊಡೆ-ಯು-ತ್ತಿದ್ದರು
ಕೋಟಲೆ-ಗಳನ್ನೂ
ಕೋಟಲೆ-ಯಲ್ಲಿ
ಕೋಟಿ
ಕೋಟಿಯ
ಕೋಟಿ-ಯನ್ನು
ಕೋಟಿಯು
ಕೋಟೆ
ಕೋಟ್ಯಂತರ
ಕೋಟ್ಯಧೀಶ್ವರ-ನಿಗೂ
ಕೋಟ್ಯಧೀಶ್ವ-ರರ
ಕೋಡಿ
ಕೋಣೆಯ
ಕೋಣೆ-ಯಲ್ಲಿ
ಕೋನವೇ
ಕೋಪ
ಕೋಪ-ಗೊಂಡು
ಕೋಪ-ನಿ-ಷ-ತ್
ಕೋಮಲ
ಕೋಮಲತೆ
ಕೋಮಿನ
ಕೋಮು-ಗಳ
ಕೋರಿ-ದನು
ಕೋರುತ್ತೇವೆ
ಕೋರು-ವರು
ಕೋರುವಿರಾ-ದರೆ
ಕೋರೈಸಿ
ಕೋರೈಸುತ್ತಿತ್ತು
ಕೋರೈ-ಸು-ತ್ತಿದೆ
ಕೋರೈಸುತ್ತಿರು-ವಾಗಲೂ
ಕೋರೈಸು-ವುದು
ಕೋಲಾಹಲದ
ಕೋಲು
ಕೋವಿ
ಕೋವಿಯ
ಕೋವಿ-ಯನ್ನು
ಕೋಶ-ಗಳಲ್ಲಿ
ಕೌಟುಂಬಿಕ
ಕೌಶಲ
ಕೌಶಲವೂ
ಕ್ಯಾ-ಥೊಲಿ-ಕ್
ಕ್ರಮ
ಕ್ರಮ-ಕ್ರಮ-ವಾಗಿ
ಕ್ರಮ-ದಲ್ಲಿ-ರುವ
ಕ್ರಮ-ದಿಂದ
ಕ್ರಮ-ಪಡಿ-ಸು-ವುದೇ
ಕ್ರಮ-ಬದ್ಧ-ಗೊಳಿಸುವ
ಕ್ರಮ-ವಾಗಿ
ಕ್ರಮ-ವಿಕಾಸ-ವಾದ
ಕ್ರಮ-ವೆಂತು
ಕ್ರಮಶಃ
ಕ್ರಮೇಣ
ಕ್ರಾಂತಿಯ
ಕ್ರಾಂತಿ-ಯನ್ನು
ಕ್ರಿಮಿ
ಕ್ರಿಮಿ-ಕೀಟ
ಕ್ರಿಮಿ-ಕೀಟ-ಗಳಂತೆ
ಕ್ರಿಮಿ-ಗಳು
ಕ್ರಿಮಿ-ಯಿಂದ
ಕ್ರಿಮಿಯು
ಕ್ರಿಯಾ-ಶೀಲ-ರಾಗಿ-ರು-ವೆವು
ಕ್ರಿಯಾ-ಶೀಲ-ವಾಗಿ-ರು-ವು-ದನ್ನು
ಕ್ರಿಯೆ
ಕ್ರಿಯೆ-ಗಳನ್ನು
ಕ್ರಿಯೆ-ಗಳಿಂದ
ಕ್ರಿಯೆ-ಗಳೆಲ್ಲ
ಕ್ರಿಯೆ-ಗಳೇ
ಕ್ರಿಯೆ-ಯಿಂದ
ಕ್ರಿಯೆಯೂ
ಕ್ರಿಯೋ-ತ್ತೇ-ಜಕ
ಕ್ರಿಸ್ತ
ಕ್ರಿಸ್ತನ
ಕ್ರಿಸ್ತನು
ಕ್ರಿಸ್ತನೋ
ಕ್ರೀಡಾಂಗಣ-ವಾಗು-ತ್ತದೆ
ಕ್ರೀಡೆ-ಯಾ-ಡುತ್ತಿ-ದ್ದನೋ
ಕ್ರೂರ
ಕ್ರೂರ-ವಾಗಿ-ರಲಿ
ಕ್ರೂರಿಗಳಾಗುತ್ತಾರೆ
ಕ್ರೂರಿ-ಗಳೇ
ಕ್ರೈಸ್ತ
ಕ್ರೈಸ್ತ-ಧರ್ಮ
ಕ್ರೈಸ್ತ-ಧರ್ಮದ
ಕ್ರೈಸ್ತ-ಧರ್ಮ-ವನ್ನು
ಕ್ರೈಸ್ತ-ಪಾದ್ರಿ
ಕ್ರೈಸ್ತರ
ಕ್ರೈಸ್ತ-ರಲ್ಲಿ
ಕ್ರೈಸ್ತ-ರಾಗು-ವರು
ಕ್ರೈಸ್ತ-ರಿ-ಗಾಗಿ
ಕ್ರೈಸ್ತ-ರಿಗೆ
ಕ್ರೈಸ್ತರು
ಕ್ರೈಸ್ತ-ರೊಡನೆ
ಕ್ರೈಸ್ತ-ಶಕ್ತಿ
ಕ್ರೋಢೀ-ಕೃತ-ವಾಗಿ-ರುವ
ಕ್ರೌರ್ಯ
ಕ್ರೌರ್ಯ-ಕ್ಕಿಂತ
ಕ್ರೌರ್ಯಕ್ಕೆ
ಕ್ರೌರ್ಯವೂ
ಕ್ಲುಪ್ತ-ವಾಗಿ
ಕ್ಲೇದಯಂತ್ಯಾಪೋ
ಕ್ಲೇಶ
ಕ್ಲೇಶ-ಪಟ್ಟುದೇ
ಕ್ಲೈಬ್ಯ-ವನ್ನು
ಕ್ಲೈಬ್ಯ-ವನ್ನೂ
ಕ್ಲೈಬ್ಯವೇ
ಕ್ಲೈವ್
ಕ್ಷಣ
ಕ್ಷಣ-ಕಾಲ
ಕ್ಷಣ-ಕಾಲದ
ಕ್ಷಣ-ಕ್ಷಣವೂ
ಕ್ಷಣ-ಗಳಲ್ಲಿ
ಕ್ಷಣ-ಗಳ-ವ-ರೆಗೆ
ಕ್ಷಣ-ದಲ್ಲಿ
ಕ್ಷಣ-ದಲ್ಲಿಯೂ
ಕ್ಷಣ-ಭಂಗುರ
ಕ್ಷಣವೇ
ಕ್ಷಣ-ವೊಂದ-ರಲ್ಲಿ
ಕ್ಷಣಿಕ
ಕ್ಷಣಿಕ-ವಾದಿ-ಗಳ
ಕ್ಷಣಿಕ-ವಾದಿ-ಗಳಾಗಿ
ಕ್ಷಣಿಕ-ವಾದಿ-ಗಳು
ಕ್ಷತ್ರಿಯ
ಕ್ಷತ್ರಿಯ-ರ-ನ್ನಾಗಿ
ಕ್ಷತ್ರಿಯ-ರಲ್ಲಿ
ಕ್ಷತ್ರಿಯ-ರಿಲ್ಲದೆ
ಕ್ಷತ್ರಿ-ಯರು
ಕ್ಷಮಿಸ-ಬೇಕು
ಕ್ಷಮಿಸಿ
ಕ್ಷಮಿಸು
ಕ್ಷಮಿಸು-ತ್ತೀ-ರೆಂದು
ಕ್ಷಮಿಸು-ವ-ನೆಂದು
ಕ್ಷಮೆ
ಕ್ಷಯಿ-ಸು-ವು-ದಕ್ಕೆ
ಕ್ಷಾತ್ರ
ಕ್ಷಿಪ್ರ
ಕ್ಷಿಪ್ರ-ವಾಗಿ
ಕ್ಷೀಣಿ-ಸದ
ಕ್ಷೀಣಿಸು-ವುದು
ಕ್ಷುದ್ರ
ಕ್ಷುದ್ರ-ಕೀಟ-ದಲ್ಲಿ
ಕ್ಷುದ್ರ-ಜೀವಿ
ಕ್ಷುದ್ರ-ಪ್ರಾಣಿ-ಯ-ಲ್ಲಿಯೂ
ಕ್ಷುದ್ರ-ಬುದ್ಧಿ-ಯನ್ನು
ಕ್ಷುದ್ರ-ಬುದ್ಧಿ-ಯಿಂದ
ಕ್ಷುದ್ರ-ಮಾ-ನ-ವ-ರೆಂದು
ಕ್ಷುದ್ರ-ವಾದ
ಕ್ಷೇತ್ರಕ್ಕೆ
ಕ್ಷೇತ್ರ-ಗಳನ್ನು
ಕ್ಷೇತ್ರ-ಗಳಲ್ಲಿ
ಕ್ಷೇತ್ರ-ಗಳಲ್ಲಿಯೂ
ಕ್ಷೇತ್ರದ
ಕ್ಷೇತ್ರ-ದಲ್ಲಿ
ಕ್ಷೇತ್ರ-ದಲ್ಲಿಯೂ
ಕ್ಷೇತ್ರ-ದಲ್ಲಿ-ರ-ಬೇಕು
ಕ್ಷೇತ್ರ-ದಲ್ಲಿ-ರಲಿ
ಕ್ಷೇತ್ರ-ದಲ್ಲಿ-ರುವ
ಕ್ಷೇತ್ರ-ವನ್ನು
ಕ್ಷೇತ್ರ-ವಲ್ಲ
ಕ್ಷೇತ್ರ-ವಾಗ-ಬೇಕು
ಕ್ಷೇತ್ರ-ವಾಗಿ
ಕ್ಷೇತ್ರ-ವಾಗಿ-ತ್ತು
ಕ್ಷೇತ್ರ-ವಾಗಿ-ರದೆ
ಕ್ಷೇತ್ರ-ವಾದ
ಕ್ಷೇತ್ರಿಕ
ಕ್ಷೇತ್ರಿಕ-ವ-ತ್
ಕ್ಷೇಮ-ಕ್ಕಾಗಿ
ಕ್ಷೇಮ-ವಾಗಿ
ಖಂಡಕ್ಕೆ
ಖಂಡ-ಗಳಲ್ಲಿನ
ಖಂಡ-ಗಳಿಗೆ
ಖಂಡದ
ಖಂಡ-ದಲ್ಲಿ
ಖಂಡ-ದಲ್ಲಿ-ರುವ
ಖಂಡ-ನಾ-ತ್ಮಕ-ವಾದು-ದಲ್ಲ
ಖಂಡ-ನೆಯ
ಖಂಡ-ವನ್ನು
ಖಂಡ-ವಾದ
ಖಂಡಿತ
ಖಂಡಿತ-ವಾಗಿ
ಖಂಡಿತ-ವಾಗಿಯೂ
ಖಂಡಿಸ-ಬೇಕಾ-ಯಿತು
ಖಂಡಿ-ಸಲು
ಖಂಡಿಸಿ
ಖಂಡಿಸಿ-ದರು
ಖಂಡಿಸಿ-ದ-ರೆಂಬ
ಖಂಡಿಸಿ-ದ್ದರು
ಖಂಡಿಸಿ-ಬಿಡುತ್ತೇ-ನೆಂದೂ
ಖಂಡಿ-ಸಿಯೇ
ಖಂಡಿ-ಸುತ್ತ
ಖಂಡಿ-ಸುವ
ಖಂಡಿ-ಸು-ವನು
ಖಂಡಿ-ಸು-ವರು
ಖಂಡಿ-ಸುವ-ವ-ರೆಲ್ಲರೂ
ಖಂಡಿ-ಸು-ವು-ದಕ್ಕೆ
ಖಂಡಿ-ಸು-ವು-ದಿಲ್ಲ
ಖಗೋಳ
ಖಗೋಳ-ಶಾಸ್ತ್ರ-ವನ್ನು
ಖಡ್ಗ-ವನ್ನು
ಖನಿ
ಖರಶ್ಚಂದನ-ಭಾರ-ವಾಹೀ
ಖರ್ಚು-ಮಾಡ-ಬಾ-ರದು
ಖರ್ಚು-ಮಾಡ-ಬೇಡಿ
ಖರ್ಚು-ಮಾಡಿ
ಖಿನ್ನ-ಮನಸ್ಕ
ಖಿನ್ನ-ಮನ-ಸ್ಕ-ನಾಗು
ಖಿಲ-ವಾಗುತ್ತವೆ
ಖುರಾನು
ಖೇತ್ರಿಯ
ಖೇತ್ರಿ-ಯಲ್ಲಿ
ಖೇದದ
ಖ್ಯಾ-ತಿ-ಗೊಂಡಿದೆಯೋ
ಖ್ಯಾ-ತಿ-ಯನ್ನು
ಗಂಗಾ-ನ-ದಿಯ
ಗಂಗಾ-ನದಿ-ಯಲ್ಲಿ
ಗಂಗಾ-ನ-ದಿಯು
ಗಂಟು
ಗಂಟು-ಗಳನ್ನು
ಗಂಟೆ
ಗಂಟೆ-ಗಳ
ಗಂಟೆ-ಗಳ-ವ-ರೆಗೆ
ಗಂಟೆಗೂ
ಗಂಡಂದಿರ
ಗಂಡ-ನನ್ನು
ಗಂಡ-ನಲ್ಲಿ-ರುವ
ಗಂಡನಾ-ರೆಂದು
ಗಂಡ-ಸರಿ-ಗಿಂತ
ಗಂಡಸರೇ
ಗಂಡಸಾ-ಗಲಿ
ಗಂಡ-ಸಿಗೂ
ಗಂಡಸು
ಗಂಡು
ಗಂಧದ
ಗಂಭೀರ
ಗಂಭೀರ-ತಮ-ವಾದುದು
ಗಂಭೀರ-ತೆಯ-ಪರಿ-ಚಯ
ಗಂಭೀರ-ವಾಗಿದೆ
ಗಂಭೀರ-ವಾಗಿದ್ದ
ಗಂಭೀರ-ವಾಣಿ
ಗಗನ
ಗಗ-ನಕ್ಕೆ
ಗಚ್ಛತಿ
ಗಚ್ಛತು
ಗಟ್ಟಿಮುಟ್ಟಾದ
ಗಟ್ಟಿ-ಯಾಗಿಲ್ಲ
ಗಣ-ನೆಗೆ
ಗಣ-ಪತಿ-ಗಳೆಂಬ
ಗಣಿ
ಗಣಿ-ಗಳು
ಗಣಿ-ತ-ಶಾಸ್ತ್ರ
ಗತ
ಗತ-ಕಾಲದ
ಗತ-ಕಾಲ-ದಲ್ಲಿ
ಗತ-ಕಾಲ-ದಿಂದಲೂ
ಗತ-ವೈ-ಭವದ
ಗತಾನು-ಶೋಚ-ನೆ-ಯಿಂದ
ಗತಿ
ಗತಿ-ಮ್
ಗತಿಯ
ಗತಿಯೇ
ಗತಿ-ಯೇನು
ಗತಿಸಿ
ಗತಿ-ಸಿ-ಹೋದ
ಗದಾ
ಗದಾ-ಪ್ರ-ಹಾರ
ಗದ್ದಲ
ಗದ್ದೆಗೂ
ಗದ್ದೆಗೆ
ಗಮನ
ಗಮನ-ಕೊ-ಡದೆ
ಗಮನ-ಕೊಡ-ಬೇಕೆಂದು
ಗಮ-ನಕ್ಕೆ
ಗಮನ-ದಲ್ಲಿ-ಡ-ಬೇ-ಕಾದ
ಗಮನ-ದಲ್ಲಿ-ಡ-ಬೇಕು
ಗಮನ-ದಲ್ಲಿಡಿ
ಗಮನ-ವನ್ನು
ಗಮನ-ವಿಟ್ಟು
ಗಮನಾರ್ಹವೂ
ಗಮನಿಸ-ದಿ-ದ್ದರೆ
ಗಮನಿಸ-ದಿರಿ
ಗಮನಿ-ಸದೆ
ಗಮನಿ-ಸದೇ
ಗಮನಿಸ-ಬಹುದು
ಗಮನಿಸ-ಬಾ-ರದು
ಗಮನಿಸ-ಬೇಕಾಗಿದೆ
ಗಮನಿಸ-ಬೇಕಾಗಿಲ್ಲ
ಗಮನಿಸ-ಬೇ-ಕಾದ
ಗಮನಿಸ-ಬೇಕು
ಗಮನಿ-ಸ-ಲಿಲ್ಲ
ಗಮ-ನಿಸಿ
ಗಮ-ನಿಸಿ-ದರೇ
ಗಮ-ನಿಸಿ-ದೆನು
ಗಮನಿ-ಸುತ್ತಾ
ಗಮನಿ-ಸು-ತ್ತಿ-ದ್ದೇವೆ
ಗಮನಿಸು-ತ್ತಿ-ರು-ವಂತೆ
ಗಮನಿ-ಸು-ವು-ದಿಲ್ಲ
ಗಮ್ಯ
ಗಮ್ಯ-ಸ್ತ್ವ-ಮಸಿ
ಗರಿ-ಗಳ
ಗರಿಗೆ-ದರಿ
ಗರೀಬ
ಗರೀಯಸೀ
ಗರ್ಜನೆ
ಗರ್ಭ-ದಿಂದ
ಗಲಭೆ-ಗಳನ್ನು
ಗಲ್ಲಿಗೇಕೆ
ಗಳನ್ನು
ಗಳನ್ನೂ
ಗಳಾಗಿ
ಗಳಾಗಿ-ದ್ದ-ರೆಂದು
ಗಳಾಗಿ-ರು-ವರು
ಗಳಾ-ಗುತ್ತ-ದೆಂದು
ಗಳಾಗು-ವಿರಿ
ಗಳಾದ
ಗಳಾದರು
ಗಳಿಂದ
ಗಳಿಂದಲೂ
ಗಳಿಗೆ
ಗಳಿಗೆಗೆ
ಗಳಿಗೆಯೂ
ಗಳಿ-ಲ್ಲವೋ
ಗಳಿ-ಸಿ-ಕೊಂಡ
ಗಳಿ-ಸಿ-ಕೊಂಡಿ-ರು-ವು-ದಕ್ಕೆ
ಗಳಿ-ಸಿದನು
ಗಳಿ-ಸಿದರು
ಗಳಿ-ಸಿ-ದ್ದೀರಿ
ಗಳಿ-ಸಿವೆ
ಗಳಿ-ಸು-ತ್ತೇನೆ
ಗಳಿ-ಸು-ವು-ದಕ್ಕೆ
ಗಳಿ-ಸು-ವು-ದರ
ಗಳಿ-ಸು-ವು-ದರ-ಲ್ಲಿಯೂ
ಗಳಿ-ಸು-ವುದು
ಗಳು
ಗಳೆ-ಲ್ಲ-ವನ್ನೂ
ಗಳೊಳಗೆ
ಗಷ್ಟೇ
ಗಹನ
ಗಹ್ವರ-ಗಳಲ್ಲಿ
ಗಹ್ವರ-ದಲ್ಲಿ
ಗಾಂಧೀಜಿ
ಗಾಂಭೀರ್ಯ-ಗಳಿಂದ
ಗಾಗಿ
ಗಾಗಿಯೂ
ಗಾಜಿನ
ಗಾಜಿ-ನಂತೆ
ಗಾಜು-ಗಳ
ಗಾಡಿ
ಗಾಡಿ-ಗಟ್ಟಲೆ
ಗಾಡಿ-ಯನ್ನು
ಗಾಡಿ-ಯೊಂದನ್ನು
ಗಾಢ-ವಾಗು-ತ್ತಿದೆ
ಗಾಢವೂ
ಗಾಣಪತ್ಯ
ಗಾಣಪತ್ಯ-ರಾಗಲಿ
ಗಾಣಿ-ಸಲು
ಗಾತ್ರ
ಗಾದೆಯಂತಿದೆ
ಗಾದೆ-ಯನ್ನು
ಗಾಮಿ-ಗಳಾಗಿ-ರೋಣ
ಗಾಯ
ಗಾಯ-ಕ್ಕಿಂತ
ಗಾಯ-ಗೊಂಡ
ಗಾಯ-ವಾದರೆ
ಗಾರ-ನಿಗೂ
ಗಾರ್ಗಿ
ಗಾರ್ಹಸ್ಥ್ಯ
ಗಾಳಿ
ಗಾಳಿ-ಯಂತೆ
ಗಾಳಿಯೂ
ಗಿಂತ
ಗಿಡ
ಗಿಡಕ್ಕೆ
ಗಿಡ-ದಂತೆ
ಗಿಡ-ಮರ-ಗಳಂತೆ
ಗಿಡ-ಮರ-ಗಳಿಂದ
ಗಿಡ-ವಾಗಿ
ಗಿಡ-ವಾಗಿ-ರು-ತ್ತದೆ
ಗಿಡವು
ಗಿರಿ
ಗಿರಿ-ಗುಹೆ-ಗಳಲ್ಲಿ
ಗಿರಿ-ರಾಜನ
ಗಿರಿ-ಶಿಖರ-ಗಳ
ಗಿರಿ-ಶಿಖರ-ಗಳಲ್ಲಿ
ಗಿಳಿಯ
ಗಿಳಿ-ಯಂತೆ
ಗಿಳಿಯುತ್ತಿ-ರುವ
ಗೀಚು-ವು-ದಕ್ಕೆ
ಗೀತಾ
ಗೀತಾ-ಚಾರ್ಯ
ಗೀತಾ-ಧ್ಯ-ಯನ-ಕ್ಕಿಂತ
ಗೀತಾ-ಭಾಷ್ಯದ
ಗೀತಾ-ವಾ-ಕ್ಯವು
ಗೀತೆ
ಗೀತೆಯ
ಗೀತೆ-ಯನ್ನು
ಗೀತೆ-ಯಲ್ಲಿ
ಗೀತೆ-ಯ-ಲ್ಲಿ-ರು-ವುದು
ಗೀತೆಯು
ಗೀತೆಯೂ
ಗೀತೋ-ಪದೇ-ಶಕ-ನಾದ
ಗೀರ್ವಾ-ಣ-ಶಿಖರ-ಗಳ
ಗುಂಡಿಕ್ಕಿ
ಗುಂಡಿ-ನಿಂದ
ಗುಂಡಿ-ಯಂತೆ
ಗುಂಡಿ-ಯಲ್ಲಿ
ಗುಂಪಿಗೆ
ಗುಂಪಿಗೋ
ಗುಂಪಿ-ನ-ವರು
ಗುಂಪಿ-ರು-ವುದು
ಗುಂಪು
ಗುಂಪೊಂದು
ಗುಜರಾತಿನ
ಗುಟ್ಟನ್ನು
ಗುಟ್ಟು
ಗುಡಿ-ಗಳ
ಗುಡಿ-ಗಳನ್ನು
ಗುಡಿ-ಯ-ನ್ನಾಗಿ
ಗುಡಿ-ಯನ್ನು
ಗುಡಿ-ಯಲ್ಲಿ
ಗುಡಿಸ-ಬಯ-ಸಿದ
ಗುಡಿ-ಸಲಿ-ನ-ಲ್ಲಿಯೂ
ಗುಡಿ-ಸಲು
ಗುಡಿ-ಸಲು-ಗಳಲ್ಲೂ
ಗುಡಿಸಿ
ಗುಡಿಸುತ್ತೇವೆ
ಗುಡಿ-ಸು-ವು-ದಕ್ಕೆ
ಗುಡ್ಡಗಾಡು-ಗಳಲ್ಲಿ
ಗುಡ್ಡದ
ಗುಣ
ಗುಣ-ಗಳ
ಗುಣ-ಗಳನ್ನು
ಗುಣ-ಗಳನ್ನೂ
ಗುಣ-ಗಳ-ನ್ನೆಲ್ಲ
ಗುಣ-ಗಳಲ್ಲ
ಗುಣ-ಗಳಲ್ಲಿ
ಗುಣ-ಗಳಿಂದ
ಗುಣ-ಗಳಿಂದಲೇ
ಗುಣ-ಗಳಿಂದಾಗಿ
ಗುಣ-ಗಳಿಗೆ
ಗುಣ-ಗಳಿ-ದ್ದರೆ
ಗುಣ-ಗಳಿ-ರುತ್ತ-ವೆಯೋ
ಗುಣ-ಗಳಿ-ರುವ
ಗುಣ-ಗಳಿಲ್ಲ
ಗುಣ-ಗಳಿ-ಲ್ಲ-ದಿ-ದ್ದರೆ
ಗುಣ-ಗಳಿವೆ
ಗುಣ-ಗಳು
ಗುಣ-ಗಳೂ
ಗುಣ-ಗಳೇ
ಗುಣ-ಗುಣಿ-ಗಳ
ಗುಣದ
ಗುಣ-ದಲ್ಲಿ
ಗುಣ-ಪಡಿ-ಸ-ಬ-ಲ್ಲದೆ
ಗುಣ-ಪಡಿ-ಸು-ವುದೆ
ಗುಣ-ರ-ಹಿ-ತ-ವಾದು-ದಾದರೆ
ಗುಣ-ವಂತ
ಗುಣ-ವನ್ನು
ಗುಣ-ವನ್ನೇ
ಗುಣ-ವಲ್ಲ
ಗುಣ-ವಾಗಿ
ಗುಣ-ವಾಗಿದೆ
ಗುಣ-ವಾ-ಚಕ-ಗಳನ್ನು
ಗುಣ-ವಾ-ಚಕ-ವನ್ನು
ಗುಣ-ವಾ-ಚ-ಕವೂ
ಗುಣ-ವಾದ
ಗುಣವು
ಗುಣ-ವೆಂದರೆ
ಗುಣ-ಸಂಪನ್ನನು
ಗುಣಿ
ಗುಣಿ-ಗಳನ್ನು
ಗುಣಿ-ಗಳಿಗೆ
ಗುಣಿ-ಗಳು
ಗುಣಿ-ಯನ್ನು
ಗುತ್ತದೆ
ಗುತ್ತಿಗೆ
ಗುಪ್ತ-ನಿಧಿ
ಗುಪ್ತಯೇ
ಗುಪ್ತ-ವಾಗಿ
ಗುಪ್ತವೂ
ಗುರಿ
ಗುರಿಯ
ಗುರಿ-ಯಂತೇ
ಗುರಿ-ಯ-ನ್ನಾಗಿ
ಗುರಿ-ಯನ್ನು
ಗುರಿ-ಯನ್ನೂ
ಗುರಿ-ಯ-ನ್ನೇ
ಗುರಿ-ಯಲ್ಲ
ಗುರಿ-ಯಾಗಿ
ಗುರಿ-ಯಾಗು-ತ್ತಿ-ದ್ದೇವೆ
ಗುರಿ-ಯಾ-ದರೆ
ಗುರಿ-ಯಾ-ದು-ದ-ರಿಂದ
ಗುರಿಯು
ಗುರಿಯೂ
ಗುರಿ-ಯೆ-ಡೆಗೆ
ಗುರಿಯೇ
ಗುರು
ಗುರು-ಗಳ
ಗುರು-ಗಳಂತೆ
ಗುರು-ಗಳಂತೆ-ಎನ್ನುತ್ತೇವೆ
ಗುರು-ಗಳಾಗ
ಗುರು-ಗಳಾಗಿ
ಗುರು-ಗಳಾದ
ಗುರು-ಗಳು
ಗುರು-ಗಳೂ
ಗುರು-ಗಳೆಲ್ಲ
ಗುರು-ಗಳೇ
ಗುರು-ಗಿರಿಯ
ಗುರು-ತರ
ಗುರು-ತರ-ವಾಗಿವೆ
ಗುರು-ತರ-ವಾದುದು
ಗುರು-ತಿನಂತಾಗು-ವುದು
ಗುರು-ತಿಸ-ಬಹುದು
ಗುರು-ತಿಸಿ
ಗುರು-ತಿಸಿ-ದರು
ಗುರು-ತಿಸಿ-ದೆನು
ಗುರು-ತಿಸುವಂತಾ-ಗಿದೆ
ಗುರುತು
ಗುರುತ್ವ
ಗುರು-ದೇವ
ಗುರು-ದೇವರ
ಗುರು-ದೇ-ವರೂ
ಗುರು-ವನ್ನಾಗಿ
ಗುರು-ವನ್ನು
ಗುರು-ವರ್ಯರ
ಗುರು-ವಲ್ಲ
ಗುರು-ವಾಗ-ಬಲ್ಲನು
ಗುರು-ವಾಗಿ
ಗುರು-ವಾಗಿ-ದ್ದರು
ಗುರು-ವಾಗು-ವೆನು
ಗುರು-ವಿನ
ಗುರು-ವಿ-ನಂತೆ
ಗುರು-ವಿನ-ಲ್ಲಾ-ದರೂ
ಗುರು-ವಿ-ನಲ್ಲಿ
ಗುರು-ವಿ-ನಿಂದ
ಗುರು-ವಿ-ನೊಂದಿಗೆ
ಗುರುವು
ಗುರು-ಶಿಷ್ಯರ
ಗುರು-ಶಿಷ್ಯ-ರಲ್ಲಿ
ಗುಲಾಬಿಯ
ಗುಲಾಬಿ-ಯನ್ನು
ಗುಲಾಮ
ಗುಲಾಮ-ಗಿರಿ
ಗುಲಾಮ-ಗಿರಿಗೆ
ಗುಲಾಮ-ಗಿರಿಯ
ಗುಲಾಮ-ಗಿರಿ-ಯನ್ನು
ಗುಲಾಮ-ಗಿರಿ-ಯಲ್ಲೇ
ಗುಲಾಮ-ಗಿರಿ-ಯಿಂದ
ಗುಲಾಮ-ನಲ್ಲ
ಗುಲಾಮ-ನಾ-ಗಿದ್ದ-ವನು
ಗುಲಾಮ-ನೆಂದು
ಗುಲಾಮ-ರಾ-ಗಿಯೇ
ಗುಲಾಮ-ರಾಗಿ-ರು-ವೆವು
ಗುಲಾಮರು
ಗುಲಾಮ-ಳಾಗ-ಬೇಕು
ಗುಳ್ಳೆ
ಗುಳ್ಳೆ-ಯಂತೆ
ಗುಳ್ಳೆ-ಯಾಗಿ-ರ-ಬಹುದು
ಗುಹವನಾಂತರ-ಗಳಲ್ಲಿ-ರುವ
ಗುಹೆ-ಗಳಲ್ಲಿ
ಗುಹೆ-ಗಳಲ್ಲಿಯೂ
ಗುಹೆ-ಯನ್ನು
ಗುಹೆ-ಯಲ್ಲಿ
ಗೂಡಿ-ಸಲು
ಗೂಡಿ-ಸುತ್ತದೆ
ಗೂಢ
ಗೂಢಂ
ಗೂಢ-ಮಗ್ರೇ
ಗೂಢ-ರಹಸ್ಯವೇ
ಗೂಳಿ-ಯುದ್ಧ
ಗೃಹ-ಗಳನ್ನು
ಗೃಹ-ವನ್ನು
ಗೃಹಸ್ಥ
ಗೃಹ-ಸ್ಥರೂ
ಗೃಹಿಣಿ
ಗೃಹ್ಯ
ಗೆಡ್ಡೆಗೆ-ಣಸು
ಗೆಡ್ಡೆಗೆ-ಣಸು-ಗಳನ್ನು
ಗೆದ್ದ
ಗೆದ್ದ-ವರಾಗುತ್ತಾರೆ
ಗೆದ್ದಿತು
ಗೆದ್ದಿಲ್ಲ
ಗೆದ್ದು
ಗೆರೆ
ಗೆಲ್ಲದ
ಗೆಲ್ಲ-ದಿದ್ದದ್ದು
ಗೆಲ್ಲದೆ
ಗೆಲ್ಲ-ಬೇಕಾಗಿದೆ
ಗೆಲ್ಲ-ಬೇ-ಕಾದ
ಗೆಲ್ಲ-ಬೇಕು
ಗೆಲ್ಲ-ಬೇಕೆಂಬುದೇ
ಗೆಲ್ಲ-ಲಾ-ರದು
ಗೆಲ್ಲ-ಲಿಲ್ಲ
ಗೆಲ್ಲಲು
ಗೆಲ್ಲು
ಗೆಲ್ಲು-ತ್ತಿತ್ತು
ಗೆಲ್ಲುವ
ಗೆಲ್ಲು-ವ-ರೆಂಬುದು
ಗೆಲ್ಲು-ವ-ವ-ರೆಗೆ
ಗೆಲ್ಲು-ವು-ದಕ್ಕೆ
ಗೆಲ್ಲು-ವುದು
ಗೆಲ್ಲು-ವು-ದೆಂದರೆ
ಗೆಲ್ಲು-ವುದೇ
ಗೆಳೆಯನ
ಗೆಳೆಯ-ರಾಗಿ-ರುವ
ಗೊಂತಿ-ನಲ್ಲಿ-ರುವ
ಗೊಂದಲ-ಗಳಲ್ಲಿ
ಗೊಂದಲದ
ಗೊಂದಲದಲ್ಲಿಯೂ
ಗೊಂದಲ-ಮಯ-ವಾಗಿ
ಗೊಡ-ವೆಯೇ
ಗೊಡು-ವು-ದಿಲ್ಲ
ಗೊಣ-ಗದೆ
ಗೊಣಗಾಡ-ಬೇಕು
ಗೊಣ-ಗಾಡಿ-ದರೆ
ಗೊಣಗುಟ್ಟದೆ
ಗೊತ್ತಾಗದೆ
ಗೊತ್ತಾಗ-ಬೇ-ಕಾದರೆ
ಗೊತ್ತಾಗಲಿ
ಗೊತ್ತಾಗಿ
ಗೊತ್ತಾ-ಗಿದೆ
ಗೊತ್ತಾ-ಗಿವೆ
ಗೊತ್ತಾ-ಗು-ತ್ತದೆ
ಗೊತ್ತಾ-ಗು-ತ್ತಿದೆಯೆ
ಗೊತ್ತಾ-ಗು-ತ್ತಿ-ರಲಿಲ್ಲ
ಗೊತ್ತಾ-ಗುವ-ವ-ರೆಗೆ
ಗೊತ್ತಾ-ಗು-ವು-ದಿಲ್ಲ
ಗೊತ್ತಾಗು-ವುದು
ಗೊತ್ತಾಗು-ವುದೆ
ಗೊತ್ತಾ-ಗು-ವುದೇ
ಗೊತ್ತಾ-ಗು-ವುದೇ-ನೆಂದರೆ
ಗೊತ್ತಾದ
ಗೊತ್ತಾದರೆ
ಗೊತ್ತಾ-ಯಿತು
ಗೊತ್ತಿತ್ತು
ಗೊತ್ತಿದೆ
ಗೊತ್ತಿದೆಯೇ
ಗೊತ್ತಿದೆಯೊ
ಗೊತ್ತಿದ್ದರೂ
ಗೊತ್ತಿದ್ದರೆ
ಗೊತ್ತಿ-ದ್ದುವು
ಗೊತ್ತಿರ-ಬಹುದು
ಗೊತ್ತಿರ-ಬಹು-ದೆಂದು
ಗೊತ್ತಿರ-ಬೇಕಾಗಿ-ತ್ತು
ಗೊತ್ತಿರ-ಬೇಕು
ಗೊತ್ತಿ-ರಲಿ
ಗೊತ್ತಿ-ರಲಿಲ್ಲ
ಗೊತ್ತಿ-ರು-ವಂತೆ
ಗೊತ್ತಿ-ರು-ವು-ದಿಲ್ಲ
ಗೊತ್ತಿ-ರು-ವುದೇ
ಗೊತ್ತಿಲ್ಲ
ಗೊತ್ತಿಲ್ಲ-ದ-ವ-ನನ್ನು
ಗೊತ್ತಿಲ್ಲದೆ
ಗೊತ್ತಿಲ್ಲದೇ
ಗೊತ್ತಿಲ್ಲವೆ
ಗೊತ್ತಿಲ್ಲ-ವೆಂದು
ಗೊತ್ತಿಲ್ಲ-ವೆಂದೂ
ಗೊತ್ತಿವೆ
ಗೊತ್ತು
ಗೊತ್ತು-ಗುರಿ-ಗಳಿ-ಲ್ಲದೆ
ಗೊತ್ತೆ
ಗೊತ್ತೇ
ಗೊಪೀ
ಗೊಬ್ಬರ-ಗಳನ್ನು
ಗೊಳಿ-ಸ-ಲಾ-ಗ-ಲಿಲ್ಲ
ಗೊಳಿಸಿ-ರು-ವರು
ಗೊಳಿಸು-ತ್ತೀರಿ
ಗೊಳಿಸು-ವೆನು
ಗೊಳ್ಳು-ತ್ತಿದ್ದಾಳೆ
ಗೊಳ್ಳುವ
ಗೋಚರ-ವಾಗದೆ
ಗೋಚರ-ವಾಗಿರ
ಗೋಚರ-ವಾಗು-ತ್ತದೆ
ಗೋಚರ-ವಾಗು-ತ್ತಿದೆ
ಗೋಚರ-ವಾಗುವ
ಗೋಚರ-ವಾಗು-ವುದು
ಗೋಚರ-ವಾದಾಗ
ಗೋಚರ-ವಾದುದು
ಗೋಚರ-ವಾ-ಯಿತು
ಗೋಚ-ರಿಸಿ-ದಂತಾ
ಗೋಚರಿ-ಸಿದನು
ಗೋಚರಿ-ಸು-ತ್ತಿದೆ
ಗೋಚರಿ-ಸುವ
ಗೋಚರಿ-ಸು-ವು-ದಿಲ್ಲ
ಗೋಚರಿಸು-ವುದು
ಗೋಜಿಗೆ
ಗೋಡೆ
ಗೋಡೆ-ಗಳ
ಗೋಡೆ-ಗಳ-ನ್ನೊಡೆದು
ಗೋಪಾಲ-ತಾ-ಪಿನಿ
ಗೋಪಾಲ-ನಾ-ಗಿಯೇ
ಗೋಪಿ
ಗೋಪಿ-ಯರ
ಗೋಪಿ-ಯ-ರಿಗೆ
ಗೋಪಿ-ಯರು
ಗೋಪಿ-ಯ-ರೊಂದಿಗೆ
ಗೋಪೀ
ಗೋಪೀ-ಜ-ನವ-ಲ್ಲಭ
ಗೋಪೀ-ಜ-ನವ-ಲ್ಲಭ-ನಾಗಿ-ರುವ
ಗೋಪ್ಯ-ವಾಗಿ
ಗೋಪ್ಯ-ವಾಗಿ-ಡು-ವರು
ಗೋಮಾಂಸ
ಗೋರಿ
ಗೋಳು
ಗೋವಿಂದ
ಗೋವಿಂದ-ಸಿಂಗನಾಗ-ಬೇಕು
ಗೋವಿಂದ-ಸಿಂಗನು
ಗೋವಿಂದ-ಸಿಂಗ-ರಂತೆ
ಗೋವಿನ
ಗೋಷ್ಪ
ಗೋಹತ್ಯೆ
ಗೌಣ
ಗೌಣ-ವಾದದ್ದು
ಗೌಣ-ಸ್ಥಾನ-ದಲ್ಲಿಯೇ
ಗೌತಮ
ಗೌರವ
ಗೌರವಕ್ಕೆ
ಗೌರವ-ಗಳನ್ನೂ
ಗೌರವ-ಗಳಿಗೆ
ಗೌರವ-ಗಳು
ಗೌರವದ
ಗೌರವ-ದಿಂದ
ಗೌರವ-ದಿಂದಲೂ
ಗೌರವ-ಪೂರ್ವ-ಕ-ವಾಗಿಯೂ
ಗೌರವ-ಪೂರ್ವ-ಕ-ವಾದ
ಗೌರವ-ಯುಕ್ತ
ಗೌರವ-ವನ್ನು
ಗೌರವ-ವನ್ನೂ
ಗೌರವ-ವ-ವನ್ನು
ಗೌರವ-ವಿತ್ತು
ಗೌರವ-ವಿದೆ
ಗೌರವ-ಶೀಲ
ಗೌರವಸ್ಥ
ಗೌರವ-ಸ್ಥಾನ-ವನ್ನು
ಗೌರವಿಸ-ಕೂಡದು
ಗೌರವಿಸ-ಬೇಕೆ
ಗೌರವಿಸಿ
ಗೌರವಿಸುತ್ತಿ-ರುವ
ಗೌರವಿಸು-ತ್ತೇನೆ
ಗ್ಧ್ಧಳ್ಧಿವೆ
ಗ್ರಂಥ
ಗ್ರಂಥ-ಕರ್ತ
ಗ್ರಂಥ-ಕರ್ತರ
ಗ್ರಂಥ-ಕರ್ತ-ರಿಗೆ
ಗ್ರಂಥ-ಕರ್ತೃ-ಗಳು
ಗ್ರಂಥ-ಗಳ-ನ್ನಾಗಿ
ಗ್ರಂಥ-ಗಳನ್ನು
ಗ್ರಂಥ-ಗಳ-ನ್ನೆಲ್ಲ
ಗ್ರಂಥ-ಗಳಲ್ಲಿ
ಗ್ರಂಥ-ಗಳಲ್ಲಿ-ರುವ
ಗ್ರಂಥ-ಗಳಾಗಿ-ದ್ದರೂ
ಗ್ರಂಥ-ಗಳಿಂದ
ಗ್ರಂಥ-ಗಳಿಗೆ
ಗ್ರಂಥ-ಗಳು
ಗ್ರಂಥ-ಗಳೆಂದು
ಗ್ರಂಥ-ಗಳೆಂಬು-ದನ್ನು
ಗ್ರಂಥ-ಗಳೆಲ್ಲ
ಗ್ರಂಥ-ಗಳೆ-ಲ್ಲವೂ
ಗ್ರಂಥ-ಗಳೇ
ಗ್ರಂಥದ
ಗ್ರಂಥ-ದಲ್ಲಿ
ಗ್ರಂಥ-ಪೀಡನೆ
ಗ್ರಂಥ-ಪೀಡ-ನೆಯ
ಗ್ರಂಥ-ಭಾಗ-ಗಳನ್ನು
ಗ್ರಂಥ-ರಚ-ನೆ-ಯನ್ನೆಲ್ಲಾ
ಗ್ರಂಥ-ರಾಶಿ
ಗ್ರಂಥ-ವನ್ನು
ಗ್ರಂಥ-ವಾ-ಯಿತು
ಗ್ರಂಥವು
ಗ್ರಂಥವೂ
ಗ್ರಂಥ-ವೆಂಬ
ಗ್ರಹ-ಣದ
ಗ್ರಹಣ-ಮಾಡಿ-ಕೊಳ್ಳು-ವರು
ಗ್ರಹವ್ಯೂಹ-ಗಳೆಂದು
ಗ್ರಹಿ-ಸದೆ
ಗ್ರಹಿ-ಸ-ಬಲ್ಲರೋ
ಗ್ರಹಿ-ಸ-ಬಲ್ಲಿರಿ
ಗ್ರಹಿ-ಸ-ಬಹುದು
ಗ್ರಹಿ-ಸ-ಬಾ-ರ-ದೆಂದು
ಗ್ರಹಿ-ಸ-ಬೇಕಾಗಿಲ್ಲ
ಗ್ರಹಿ-ಸ-ಬೇಕು
ಗ್ರಹಿ-ಸಲಾ-ಗದ
ಗ್ರಹಿ-ಸಲಾಗ-ದು-ದನ್ನು
ಗ್ರಹಿ-ಸ-ಲಾ-ರದು
ಗ್ರಹಿ-ಸ-ಲಾ-ರರು
ಗ್ರಹಿ-ಸಲು
ಗ್ರಹಿ-ಸಲ್ಪಟ್ಟದ್ದೂ
ಗ್ರಹಿಸಿ
ಗ್ರಹಿ-ಸಿ-ದರೆ
ಗ್ರಹಿ-ಸಿ-ದಷ್ಟೂ
ಗ್ರಹಿ-ಸಿದ್ದೇ
ಗ್ರಹಿ-ಸಿ-ರು-ವೆವು
ಗ್ರಹಿ-ಸುತ್ತೇ-ವೆಯೋ
ಗ್ರಹಿ-ಸು-ವಂತ-ಹ-ವಿದ್ಯಾ-ಭ್ಯಾಸ
ಗ್ರಹಿ-ಸು-ವುದು
ಗ್ರಹಿ-ಸು-ವುದೇ
ಗ್ರಹಿ-ಸು-ವೆವು
ಗ್ರಾಮ
ಗ್ರಾಮ-ಗಳಲ್ಲಿ
ಗ್ರಾಮ-ಗಳಿಂದ
ಗ್ರಾಮದ
ಗ್ರಾಮ-ದಲ್ಲಿ-ರುವ
ಗ್ರಾಮ-ದೇವತಾ
ಗ್ರಾಮ್ಯ
ಗ್ರಾಮ್ಯ-ವಾಗಿ-ರಲಿ
ಗ್ರೀಕನ
ಗ್ರೀಕನ್ನು
ಗ್ರೀಕರ
ಗ್ರೀಕ-ರಂತೆ
ಗ್ರೀಕ-ರಿಗೆ
ಗ್ರೀಕರು
ಗ್ರೀಕ್
ಗ್ರೀಸಿಗೆ
ಗ್ರೀಸಿ-ನಲ್ಲಿ
ಗ್ರೀಸಿ-ನವ-ಲ್ಲದೆ
ಗ್ರೀಸ್
ಗ್ಲಾನಿ
ಗ್ಲಾನಿರ್ಭವತಿ
ಘಂಟಾ
ಘಂಟಾ-ಘೋಷ-ವಾಗಿ
ಘಟನೆ
ಘಟ-ನೆ-ಗಳ
ಘಟ-ನೆ-ಗಳ-ನ್ನೆಲ್ಲಾ
ಘಟ-ನೆ-ಗಳಿಂದ
ಘಟ-ನೆ-ಗಳಿಗೆ
ಘಟ-ನೆಯ
ಘಟ-ನೆ-ಯನ್ನು
ಘಟ-ವನ್ನು
ಘನ
ಘನ-ಗರ್ಜ-ನೆಯ
ಘನ-ವಾದ
ಘನೀ-ಭೂತ-ವಾಗು-ವುದು
ಘನೀ-ಭೂತ-ವಾದ
ಘರ್ಷ-ಣಕ್ಕೆ
ಘರ್ಷಣ-ದಿಂದಲೇ
ಘರ್ಷಣೆ
ಘರ್ಷಣೆ-ಗಳು
ಘರ್ಷಣೆ-ಯನ್ನು
ಘಾಟಿ-ನಲ್ಲಿ
ಘೋರ
ಘೋರ-ಪಾಪ
ಘೋಷ-ಕ್ಕಾಗಿ
ಘೋಷಣೆ
ಘೋಷಣೆ-ಯನ್ನು
ಘೋಷ-ವಾಗಿ
ಘೋಷಿಸಿ-ಕೊಂಡ
ಘೋಷಿಸು-ವುವು
ಚ
ಚಂಚಲ-ವಲ್ಲದ
ಚಂಚಲ-ವಾದುದು
ಚಂಡಾಲ
ಚಂಡಾ-ಲನ
ಚಂಡಾಲ-ನನ್ನು
ಚಂಡಾಲ-ನನ್ನೂ
ಚಂಡಾಲ-ನ-ವ-ರೆಗೆ
ಚಂಡಾಲ-ನಾ-ದರೂ
ಚಂಡಾಲ-ನಿಗೆ
ಚಂಡಾಲ-ರನ್ನು
ಚಂಡಾಲ-ರಾಗಲೀ
ಚಂಡಿ
ಚಂದನ-ಸೈ-ಗಂಧದ
ಚಂದ್ರ
ಚಂದ್ರ-ತಾರಕಂ
ಚಂದ್ರ-ತಾರ-ಕ-ಮ್
ಚಂದ್ರನು
ಚಂದ್ರನೂ
ಚಂದ್ರ-ರೆಲ್ಲ
ಚಂದ್ರ-ಲೋಕಕ್ಕೆ
ಚಂದ್ರಾದಿ-ಗಳು
ಚಕ್ರಕ್ಕೆ
ಚಕ್ರ-ದಲ್ಲಿ
ಚಕ್ರ-ಬಡ್ಡಿ
ಚಕ್ರ-ವರ್ತಿ
ಚಕ್ರ-ವರ್ತಿ-ಗಳು
ಚಕ್ರ-ವರ್ತಿ-ಯಾದ
ಚಕ್ರ-ವರ್ತಿಯೂ
ಚಕ್ರಾಧಿಪತ್ಯ
ಚಕ್ರಾಧಿಪ-ತ್ಯದ
ಚಕ್ಷುರ್ಗ-ಚ್ಛತಿ
ಚಕ್ಷುರ್ಗ-ಚ್ಛತಿನ
ಚಕ್ಷುಸ್ಸು
ಚಟರ್ಜಿ
ಚಟು
ಚಟು-ವಟಿಕೆ
ಚಟು-ವಟಿಕೆ-ಗಳ
ಚಟು-ವಟಿಕೆ-ಗಳಲ್ಲಿ
ಚಟು-ವಟಿಕೆ-ಗಳಿಗೆ
ಚಟು-ವಟಿಕೆ-ಯನ್ನು
ಚಟು-ವಟಿಕೆ-ಯಾಗಿ-ರ-ಬಹುದು
ಚಟು-ವಟಿಕೆಯು
ಚಟ್ನಿ-ಮಾಡಿ
ಚತುರ್ಮುಖ
ಚದುರಿ-ಹೋ-ಗಿ-ರುವ
ಚನ್ನಪುರಿ
ಚಪ್ಪರ-ದಲ್ಲಿ
ಚಮತ್ಕಾರ
ಚಮತ್ಕಾರ-ವಾದ
ಚರಂಡಿಯ
ಚರಣ
ಚರಣೆ
ಚರಮ
ಚರಮ-ಗುರಿ-ಯನ್ನು
ಚರಮ-ಗುರಿ-ಯಾಗಿದೆ
ಚರ-ಮಾ-ವಸ್ಥೆ-ಯನ್ನು
ಚರಿತ-ಳಾದ
ಚರಿತ್ರೆ
ಚರಿತ್ರೆಯ
ಚರಿತ್ರೆ-ಯಂತೆ
ಚರಿತ್ರೆ-ಯನ್ನು
ಚರಿತ್ರೆ-ಯನ್ನೂ
ಚರಿತ್ರೆ-ಯಲ್ಲ
ಚರಿತ್ರೆ-ಯಲ್ಲಿ
ಚರಿತ್ರೆಯು
ಚರ್ಚನ್ನು
ಚರ್ಚನ್ನೂ
ಚರ್ಚಿಗೆ
ಚರ್ಚಿ-ನಲ್ಲಿ
ಚರ್ಚಿಸ-ಬಲ್ಲ
ಚರ್ಚಿಸ-ಬೇಕು
ಚರ್ಚಿಸಲ್ಪಟ್ಟಿ-ರುವ
ಚರ್ಚಿಸಿ
ಚರ್ಚಿ-ಸುತ್ತ
ಚರ್ಚಿ-ಸುತ್ತಾ
ಚರ್ಚಿ-ಸು-ತ್ತಿದ್ದರೆ
ಚರ್ಚಿಸುತ್ತಿ-ದ್ದು-ದನ್ನು
ಚರ್ಚಿಸು-ತ್ತೇನೆ
ಚರ್ಚಿ-ಸುವ-ವ-ರಿಗೆ
ಚರ್ಚಿಸು-ವಾಗ
ಚರ್ಚಿಸು-ವು-ದಕ್ಕೂ
ಚರ್ಚಿಸು-ವು-ದನ್ನು
ಚರ್ಚು
ಚರ್ಚು-ಗಳನ್ನು
ಚರ್ಚೆ
ಚರ್ಚೆ-ಮಾಡ-ಬಹುದು
ಚರ್ಚೆ-ಯನ್ನು
ಚರ್ಮ-ವನ್ನು
ಚಲನದ
ಚಲನ-ಶೀಲ-ವಾದು-ದೆಂದು
ಚಲನೆ
ಚಲ-ನೆಯ
ಚಲ-ನೆಯು
ಚಲ-ನೆಯೇ
ಚಲಾಯಿ
ಚಲಾವಣೆ-ಗೊಳಿಸು
ಚಲಿಸಬಲ್ಲುದೆ
ಚಲಿಸ-ಬೇಕಾಗಿಲ್ಲ
ಚಲಿಸ-ಬೇಕು
ಚಲಿಸ-ಲಾ-ರದು
ಚಲಿ-ಸುತ್ತದೆ
ಚಲಿಸುತ್ತಿತ್ತು
ಚಲಿ-ಸು-ತ್ತಿದೆ
ಚಲಿ-ಸುತ್ತಿ-ದ್ದೀರಿ
ಚಲಿಸುತ್ತಿರು-ತ್ತದೆ
ಚಲಿ-ಸುವ
ಚಲಿ-ಸು-ವಂತೆ
ಚಲಿಸು-ವಾಗಲೂ
ಚಲಿಸು-ವುದಕ್ಕೇ
ಚಲಿಸು-ವುದು
ಚಳಿ-ಗಾಲ
ಚಳಿ-ದೇಶ-ದಲ್ಲಿ
ಚಳುವಳಿ
ಚಳುವಳಿ-ಗಳ
ಚಳುವಳಿ-ಗಳು
ಚಳುವಳಿ-ಯನ್ನು
ಚಾಂಶಕಲಾಃ
ಚಾಚಿ
ಚಾರ-ಗಳ
ಚಾರ-ಣಾಯ
ಚಾರ-ತಂತ್ರ-ಗಳನ್ನು
ಚಾರಿತ್ರಿಕ
ಚಾರಿತ್ರಿಕ-ತೆಯ
ಚಾರಿತ್ರಿಕ-ವಲ್ಲ
ಚಾರಿತ್ರಿಕ-ವಾಗಿದೆ
ಚಾರಿತ್ರ್ಯ
ಚಾರಿ-ತ್ರ್ಯದ
ಚಾರ್ಯ
ಚಾರ್ಯಾಯ
ಚಾಲಕ
ಚಾಲ-ನೆ-ಗೊಳಿಸು
ಚಾಲಿತ-ವಾದ
ಚಾಳಿ-ಯಾಗಿ
ಚಾವಟಿಯ
ಚಾವಟಿ-ಯಿಂದ
ಚಿಂತ-ಕರೂ
ಚಿಂತನ
ಚಿಂತನಾ
ಚಿಂತನಾ-ಶಕ್ತಿ-ಯ-ನ್ನೇ
ಚಿಂತನೆ
ಚಿಂತ-ನೆ-ಗಳ
ಚಿಂತನೆ-ಗಳನ್ನು
ಚಿಂತ-ನೆ-ಗಳನ್ನೂ
ಚಿಂತ-ನೆ-ಗಳಿಂದ
ಚಿಂತ-ನೆ-ಗಳಿಗೂ
ಚಿಂತ-ನೆ-ಗಳು
ಚಿಂತ-ನೆಗೆ
ಚಿಂತ-ನೆಯ
ಚಿಂತ-ನೆಯು
ಚಿಂತಾ-ಜ-ನಕ
ಚಿಂತಾ-ತೀತ
ಚಿಂತಾ-ಮಣಿ-ಯಾಗಿ-ದೆಯೋ
ಚಿಂತಿ-ಸತೊಡ-ಗಿದೆ
ಚಿಂತಿಸ-ಬೇಕು
ಚಿಂತಿಸಿತೊ
ಚಿಂತಿಸುತ್ತಿ-ರು-ವರು
ಚಿಂತೆ
ಚಿಂತೆ-ಯಿಲ್ಲ
ಚಿಂದಿ-ಬಟ್ಟೆ-ಯ-ನ್ನುಟ್ಟು
ಚಿಕಾಗೊ
ಚಿಕಾಗೊ-ದಲ್ಲಿ
ಚಿಕಾಗೋ
ಚಿಕಾಗೋ-ದಲ್ಲಿ
ಚಿಕಿತ್ಸೆ
ಚಿಕ್ಕ
ಚಿಕ್ಕ-ಮ-ನೆ-ಯಲ್ಲಿ
ಚಿಗುರು
ಚಿಗುರು-ತ್ತಿದೆ
ಚಿಟ್ಟೆ-ಗಳಂತೆ
ಚಿತೆಗೆ
ಚಿತ್
ಚಿತ್ತ
ಚಿತ್ತಕ್ಕೂ
ಚಿತ್ತಕ್ಕೆ
ಚಿತ್ತದ
ಚಿತ್ತ-ದಲ್ಲಿ
ಚಿತ್ತ-ವನ್ನು
ಚಿತ್ತವು
ಚಿತ್ತ-ವೃತ್ತಿ-ಗಳ
ಚಿತ್ತವೇ
ಚಿತ್ತ-ಶುದ್ಧಿ
ಚಿತ್ತ-ಶುದ್ಧಿ-ಯಾಗು-ವುದು
ಚಿತ್ತಾಕರ್ಷಕ-ವಾದ
ಚಿತ್ರ
ಚಿತ್ರ-ಗುಪ್ತಾಯ
ಚಿತ್ರ-ವನ್ನು
ಚಿತ್ರವು
ಚಿತ್ರವೇ
ಚಿತ್ರ-ಹಿಂಸೆ
ಚಿತ್ರಿತ-ವಾಗಿದೆ
ಚಿತ್ರಿತ-ವಾಗಿ-ರು-ವುದು
ಚಿತ್ರಿ-ಸಲು
ಚಿತ್ರಿಸಿ
ಚಿತ್ರಿಸಿ-ರುವ
ಚಿತ್ರಿಸಿ-ರು-ವನು
ಚಿತ್ರಿಸಿ-ರು-ವರು
ಚಿತ್ರಿ-ಸುವ
ಚಿನ್ನ
ಚಿನ್ನದ
ಚಿನ್ನ-ವನ್ನು
ಚಿನ್ಮಯ
ಚಿನ್ಮಯದ
ಚಿನ್ಮಯ-ವಾಗಿ-ರ-ಬೇಕು
ಚಿಪ್ಪಿ-ನಲ್ಲಿ
ಚಿಪ್ಪಿ-ನಿಂದ
ಚಿಪ್ಪಿ-ನೊಳಗೆ
ಚಿಮ್ಮಿ
ಚಿಮ್ಮಿ-ದವು
ಚಿರ-ಋಣಿ
ಚಿರ-ಋಣಿ-ಗಳಾಗಿ-ರ-ಬೇಕು
ಚಿರ-ಋಣಿ-ಗಳು
ಚಿರ-ಋಣಿ-ಯಾಗಿ-ರು-ವನು
ಚಿರ-ಕಾಲ
ಚಿರ-ಕಾಲವೂ
ಚಿರ-ಕೃತ-ಜ್ಞ-ತೆಗೆ
ಚಿರ-ಕೃತ-ಜ್ಞತೆ-ಯನ್ನು
ಚಿರ-ಜಾಗೃತ-ವಾಗಿ-ರು-ವುವು
ಚಿರತೆ-ಯಂತೆ
ಚಿರ-ಪರಿ-ಚಿತ
ಚಿರಯೌ-ವನ-ವನ್ನು
ಚಿರ-ವಾಗಿ
ಚಿರಸ್ಥಾಯಿ-ಯಾಗಿ
ಚಿರಸ್ಮರಣೀಯ
ಚಿಲುಮೆ
ಚಿಲುಮೆ-ಯಲ್ಲಿ
ಚಿಲುಮೆ-ಯಾಗಿ-ರು-ವುದು
ಚಿಲುಮೆ-ಯಿಂದ
ಚಿಹ್ನೆ
ಚಿಹ್ನೆ-ಗಳನ್ನು
ಚಿಹ್ನೆ-ಗಳು
ಚಿಹ್ನೆ-ಗಳೆಲ್ಲ
ಚಿಹ್ನೆ-ಯನ್ನು
ಚಿಹ್ನೆ-ಯಲ್ಲ
ಚಿಹ್ನೆ-ಯಲ್ಲಿ
ಚಿಹ್ನೆ-ಯಾಗಿದೆ
ಚಿಹ್ನೆ-ಯಾಗು-ವು-ದಿಲ್ಲ
ಚಿಹ್ನೆ-ಯಾಗು-ವುದು
ಚಿಹ್ನೆಯೂ
ಚೀನಾ
ಚೀನಾ-ಜಪಾ-ನಿನ
ಚುಂಬನ
ಚುಂಬನ-ವನ್ನು
ಚುಂಬಿತಂ
ಚುಕ್ಷುರ್ಗ-ಚ್ಛತಿ
ಚುನಾವಣೆ
ಚುನಾವಣೆ-ಗಳ
ಚೂಡಾ
ಚೂರಾಗಿ
ಚೂರಾ-ಗಿದೆ
ಚೂರಾ-ಗು-ತ್ತಿವೆ
ಚೂರಾಗು-ವುದು
ಚೂರು
ಚೂರು-ಚೂ-ರಾಗ-ಬಹುದು
ಚೂರು-ಚೂ-ರಾಗಿ
ಚೂರು-ಪಾರು-ಗಳನ್ನು
ಚೆದುರಿ-ಹೋ-ಗಿ-ರುವ
ಚೆನ್ನಾಗಿ
ಚೆನ್ನಾಗಿ-ತ್ತು
ಚೆನ್ನಾಗಿದೆ
ಚೆನ್ನಾಗಿ-ರು-ವು-ದಿಲ್ಲ
ಚೆನ್ನಾಗಿ-ಲ್ಲದ
ಚೆಲ್ಲಾ-ಪಿ-ಲ್ಲಿ-ಯಾಗಿ
ಚೆಲ್ಲಾ-ಪಿ-ಲ್ಲಿ-ಯಾಗಿ-ರುವ
ಚೆಲ್ಲಾ-ಪಿ-ಲ್ಲಿ-ಯಾಗಿ-ರು-ವು-ದ-ರಿಂದ
ಚೆಲ್ಲಿ
ಚೆಲ್ಲಿ-ದ್ದರು
ಚೆಲ್ಲಿ-ದ್ದೀರಿ
ಚೇತನ-ಗಳಲ್ಲಿ
ಚೇತ-ನದ
ಚೇತನ-ದೊಂದಿಗೆ
ಚೇತ-ನವು
ಚೇತ-ನವೇ
ಚೇತನಾ-ತ್ಮಕ-ವಾದ-ದ್ದಾಗಿ-ರ-ಬೇಕು
ಚೇತ-ರಿಸಿ-ಕೊಳ್ಳ-ಲಾ-ರದು
ಚೈಕೇ
ಚೈತನ್ಯ
ಚೈತನ್ಯದ
ಚೈತನ್ಯ-ದಾಯ-ಕ-ವಾದ
ಚೈತನ್ಯ-ದೇ-ವನೇ
ಚೈತ-ನ್ಯರ
ಚೈತ-ನ್ಯರು
ಚೈತನ್ಯ-ವಾದು-ದ-ರಿಂದ
ಚೈತನ್ಯ-ವಿರ-ಲಾಗಿ
ಚೈತನ್ಯ-ವಿ-ರುವ
ಚೈತನ್ಯವು
ಚೈತನ್ಯ-ಹೀನ
ಚೈನ
ಚೈನಂ
ಚೈನಾ
ಚೈನೀಯನು
ಚ್ಯುತ-ರಾಗಿ-ದ್ಧರೂ
ಚ್ಯುತಿ
ಚ್ಯುತಿ-ಯಿಲ್ಲದ
ಛಂದಸ್ಸು-ಗಳಲ್ಲಿ
ಛಲವಿತ್ತು
ಛಾಯಾ-ಮೂರ್ತಿ-ಗಳಂತೆ
ಛಾಯೆ-ಯಂತೆ
ಛಾಯೆ-ಯಲ್ಲಿ
ಛಾಯೆ-ಯಿದೆ
ಛಿಂದಂತಿ
ಛೇದಿಸಿ
ಜಂಭ
ಜಂಭ-ಕೊಚ್ಚಿ-ಕೊಂಡರೂ
ಜಂಭ-ಕೊಚ್ಚಿ-ಕೊಳ್ಳುತ್ತ
ಜಗಚ್ಚಾಲಕ
ಜಗಜ್ಜನ್ಮಾ-ದಿ-ಕಾರ-ಣನು
ಜಗತೀ
ಜಗತ್
ಜಗತ್ಈ
ಜಗತ್ತನ್ನು
ಜಗತ್ತನ್ನೂ
ಜಗತ್ತನ್ನೆಲ್ಲ
ಜಗತ್ತನ್ನೆಲ್ಲಾ
ಜಗತ್ತನ್ನೇ
ಜಗತ್ತಾಗಿ
ಜಗತ್ತಿಗೂ
ಜಗತ್ತಿಗೆ
ಜಗತ್ತಿಗೆ-ನೀಡ-ಬೇಕು
ಜಗತ್ತಿಗೆಲ್ಲ
ಜಗತ್ತಿಗೆಲ್ಲಾ
ಜಗತ್ತಿಗೇ
ಜಗತ್ತಿದೆ
ಜಗತ್ತಿನ
ಜಗತ್ತಿನ-ಲ್ಲಾ-ಗಲಿ
ಜಗತ್ತಿನ-ಲ್ಲಾ-ಗಲೀ
ಜಗತ್ತಿನಲ್ಲಿ
ಜಗತ್ತಿನ-ಲ್ಲಿಯೇ
ಜಗತ್ತಿನ-ಲ್ಲಿ-ರುವ
ಜಗತ್ತಿನ-ಲ್ಲಿ-ರುವ-ವ-ರೆಲ್ಲ
ಜಗತ್ತಿನ-ಲ್ಲಿಲ್ಲ
ಜಗತ್ತಿನ-ಲ್ಲೆಲ್ಲ
ಜಗತ್ತಿನ-ಲ್ಲೆಲ್ಲಾ
ಜಗತ್ತಿನಲ್ಲೇ
ಜಗತ್ತಿನಾಚೆಗೆ
ಜಗತ್ತಿನಾದ್ಯಂತ
ಜಗತ್ತಿ-ನಾದ್ಯಂತವೂ
ಜಗ-ತ್ತಿಲ್ಲವೋ
ಜಗತ್ತು
ಜಗತ್ತು-ಗಳನ್ನು
ಜಗತ್ತು-ಗಳಲ್ಲಿ
ಜಗತ್ತು-ಗಳಲ್ಲಿಯೂ
ಜಗತ್ತೆಲ್ಲ
ಜಗತ್ತೆಲ್ಲ-ವನ್ನು
ಜಗತ್ತೆಲ್ಲಿ
ಜಗತ್ತೇ
ಜಗತ್ತೊಂದಿದೆ
ಜಗತ್ತೊಂದು
ಜಗತ್ಪ್ರ-ಸಿದ್ಧ-ವಾದ
ಜಗತ್ಸರ್ವವೂ
ಜಗದ
ಜಗದೀಶ
ಜಗದೀಶ್ವರ-ನಾದ
ಜಗದೀಶ್ವರ-ನಿಗೆಯೇ
ಜಗದೀ-ಶ್ವ-ರನು
ಜಗದೊಡೆಯ
ಜಗದ್ವಿಖ್ಯಾ-ತ-ರಾಗಿ
ಜಗದ್ವ್ಯಾಪಿ
ಜಗನ್ನಾಥ
ಜಗನ್ನಾಥ-ದಲ್ಲಿ
ಜಗನ್ಮಾತೆ
ಜಗಳ
ಜಗಳ-ಗಳು
ಜಗಳ-ಗಳೂ
ಜಗಳ-ವಾಡಿ
ಜಗಳ-ವಾ-ಡು-ವು-ದ-ರಿಂದಲೂ
ಜಗಳ-ವಿರು-ವಾಗ
ಜಟಿಲ
ಜಟಿಲ-ತೆಯೂ
ಜಟಿಲ-ವಾಗಿ-ರು-ವು-ದ-ರಿಂದ
ಜಟಿಲ-ವಾದ
ಜಟಿಲ-ವಾದುದು
ಜಡ
ಜಡ-ಚೇ-ತನ-ಗಳೆಂಬ
ಜಡ-ತೆಯು
ಜಡದ
ಜಡ-ದಿಂದ
ಜಡ-ನಾಗ-ರಿಕ-ತೆಯ
ಜಡ-ಪದಾರ್ಥ-ದಿಂದ
ಜಡ-ಪ್ರ-ಕೃತಿಯ
ಜಡ-ಪ್ರ-ಕೃತಿ-ಯ-ನ್ನೆಲ್ಲಾ
ಜಡ-ಪ್ರ-ಕೃತಿಯು
ಜಡ-ಪ್ರೀತಿ
ಜಡರೂ
ಜಡ-ವಸ್ತು
ಜಡ-ವಸ್ತು-ವನ್ನು
ಜಡ-ವಸ್ತು-ವಿನ
ಜಡ-ವಾಗಿದೆ
ಜಡ-ವಾಗಿ-ರದೆ
ಜಡ-ವಾದ
ಜಡ-ವಾದದ
ಜಡ-ವಾದ-ವನ್ನೂ
ಜಡ-ವಾದವು
ಜಡ-ವಾದಿ-ಗಳೆಂದು
ಜಡ-ವಾದಿಯ
ಜಡ-ವಾದುದು
ಜಡ-ವಿ-ಜ್ಞಾನ-ವನ್ನು
ಜಡ-ವೆಂದು
ಜಡವೇ
ಜಡ್ಜ
ಜಡ್ಜರ
ಜತೆ
ಜತೆಗೆ
ಜನ
ಜನಂ
ಜನಕ
ಜನ-ಕ-ತ್ವ-ವನ್ನು
ಜನ-ಕ-ನ-ಲ್ಲಿದ್ದ
ಜನ-ಕರು
ಜನ-ಕ-ರೆಂದು
ಜನ-ಕೋಟಿಯ
ಜನಕ್ಕೂ
ಜನಕ್ಕೆ
ಜನ-ಗಳ
ಜನ-ಗಳು
ಜನ-ಜಾಗೃತಿಯ
ಜನ-ಜೀವ-ನಕ್ಕೆ
ಜನ-ಜೀವ-ನದ
ಜನ-ಜೀವ-ನ-ದಲ್ಲಿ
ಜನತೆ
ಜನ-ತೆಗೆ
ಜನ-ತೆಯ
ಜನನ
ಜನ-ನಕ್ಕೆ
ಜನ-ನದ
ಜನ-ನ-ಮರಣ-ಗಳಿಂದ
ಜನ-ನ-ಮರಣಾ-ತೀತ
ಜನ-ನ-ವಿಲ್ಲ
ಜನ-ನ-ವಿಲ್ಲದ
ಜನನೀ
ಜನ-ಮನಸ್ಸು
ಜನರ
ಜನ-ರನ್ನು
ಜನ-ರಲ್ಲಿ
ಜನ-ರಲ್ಲಿ-ರುವ
ಜನ-ರಾ-ಡುವ
ಜನ-ರಾದ
ಜನ-ರಿಂದ
ಜನ-ರಿಗಾ-ಗಲಿ
ಜನ-ರಿ-ಗಾಗಿ
ಜನ-ರಿಗೆ
ಜನ-ರಿರ-ಬಹುದು
ಜನ-ರಿ-ರುವ
ಜನ-ರಿ-ರು-ವರು
ಜನ-ರಿ-ರುವ-ರು-ಅ-ವರು
ಜನರು
ಜನ-ರು-ವಾ-ಸ್ತವಿಕ-ವಾಗಿ
ಜನರೂ
ಜನ-ರೆಲ್ಲ
ಜನ-ರೆಲ್ಲ-ರನ್ನೂ
ಜನ-ರೆಲ್ಲಾ
ಜನ-ರೆಲ್ಲಿ-ದ್ದಾರೆ
ಜನ-ರೆಷ್ಟು
ಜನರೇ
ಜನ-ರೇನೂ
ಜನರೋ
ಜನ-ವರಿ
ಜನ-ಶಕ್ತಿ
ಜನ-ಸಂಖ್ಯೆ-ಯಲ್ಲಿ
ಜನ-ಸಂದಣಿಯ
ಜನ-ಸಂದಣಿ-ಯನ್ನು
ಜನ-ಸಂದಣಿ-ಯೆಲ್ಲಾ
ಜನ-ಸಮೂಹದ
ಜನ-ಸಾ-ಧಾರಣ-ರನ್ನು
ಜನ-ಸಾ-ಧಾರಣ-ರಲ್ಲಿ
ಜನ-ಸಾ-ಮಾ-ನ್ಯರ
ಜನ-ಸಾ-ಮಾ-ನ್ಯ-ರನ್ನು
ಜನ-ಸಾ-ಮಾ-ನ್ಯ-ರಲ್ಲಿ
ಜನ-ಸಾ-ಮಾ-ನ್ಯ-ರಿಗೆ
ಜನ-ಸಾ-ಮಾ-ನ್ಯರು
ಜನ-ಸ್ತೋಮ-ವನ್ನು
ಜನ-ಸ್ತೋಮವು
ಜನಾಂಗ
ಜನಾಂಗ-ಕ್ಕಾ-ಗಲಿ
ಜನಾಂಗಕ್ಕೂ
ಜನಾಂಗಕ್ಕೆ
ಜನಾಂಗ-ಗಳ
ಜನಾಂಗ-ಗಳನ್ನು
ಜನಾಂಗ-ಗಳಲ್ಲಿ
ಜನಾಂಗ-ಗಳ-ಲ್ಲಿಯೂ
ಜನಾಂಗ-ಗಳಾಗಿ
ಜನಾಂಗ-ಗಳಾದ
ಜನಾಂಗ-ಗಳಿ-ಗಿಂತಲೂ
ಜನಾಂಗ-ಗಳಿಗೂ
ಜನಾಂಗ-ಗಳಿಗೆ
ಜನಾಂಗ-ಗಳಿವೆ
ಜನಾಂಗ-ಗಳು
ಜನಾಂಗ-ಗಳೂ
ಜನಾಂಗ-ಗಳೆ-ಲ್ಲ-ದರ
ಜನಾಂಗ-ಗಳೇ
ಜನಾಂಗ-ಗಳೊಂದಿಗೆ
ಜನಾಂಗ-ಜೀವ-ನದ
ಜನಾಂಗದ
ಜನಾಂಗ-ದಲ್ಲಿ
ಜನಾಂಗ-ದಲ್ಲಿಯೂ
ಜನಾಂಗ-ದಲ್ಲಿ-ರುವ
ಜನಾಂಗ-ದ-ವ-ರಂತೆ
ಜನಾಂಗ-ದ-ವರು
ಜನಾಂಗ-ದಿಂದ
ಜನಾಂಗ-ದೊಂದಿಗೆ
ಜನಾಂಗ-ವನ್ನು
ಜನಾಂಗ-ವಾದರೂ
ಜನಾಂಗವು
ಜನಾಂಗವೂ
ಜನಾಂಗ-ವೆಂಬ
ಜನಾಂಗವೇ
ಜನಾಂಗೀಯ
ಜನಾದ-ರಣೀಯ
ಜನಾದ-ರಣೀಯ-ನಾ-ಗಿದ್ದು
ಜನಾ-ನ್
ಜನಿಸಿ
ಜನಿಸಿದ
ಜನಿಸಿ-ದಂತೆ
ಜನಿಸಿ-ದರು
ಜನಿಸಿ-ದ-ವನು
ಜನಿಸಿ-ದವು
ಜನಿ-ಸಿದ್ದ
ಜನಿಸಿ-ದ್ದರೂ
ಜನಿಸಿ-ರು-ವರು
ಜನಿ-ಸುತ್ತದೆ
ಜನಿಸು-ವುದು
ಜನೇ-ಭ್ಯಃ
ಜನ್ಮ
ಜನ್ಮಕ್ಕೆ
ಜನ್ಮ-ಗಳನ್ನು
ಜನ್ಮ-ಗಳಿಂದ
ಜನ್ಮ-ಗಳು
ಜನ್ಮತಃ
ಜನ್ಮ-ದಲ್ಲಿ
ಜನ್ಮ-ದಲ್ಲಿಯೇ
ಜನ್ಮ-ಧಾರಣ
ಜನ್ಮ-ಧಾ-ರಣೆ
ಜನ್ಮನಿ
ಜನ್ಮ-ನೀಶ್ವರೇ
ಜನ್ಮ-ಭೂಮಿಶ್ಚ
ಜನ್ಮ-ವನ್ನು
ಜನ್ಮ-ವೆತ್ತ
ಜನ್ಮ-ವೆತ್ತಿ
ಜನ್ಮ-ವೆ-ತ್ತಿದ
ಜನ್ಮ-ವೆ-ತ್ತಿ-ದನು
ಜನ್ಮ-ವೆ-ತ್ತಿ-ದರು
ಜನ್ಮ-ವೆ-ತ್ತು-ವರು
ಜನ್ಮ-ವೆ-ತ್ತು-ವೆನು
ಜನ್ಮ-ಸ್ಥಾನ
ಜನ್ಮಾಂತರ-ಗಳಲ್ಲಿಯೂ
ಜನ್ಮಾಂತರ-ಗಳ-ವರೆಗೂ
ಜನ್ಮಾಂಧರು
ಜಪ
ಜಪಾನಿ
ಜಪಾ-ನಿನ
ಜಪಾನೀ
ಜಪಾನ್
ಜಪಿ-ಸಲಿ
ಜಮೀನು-ದಾ-ರರು
ಜಮೀನ್ದಾರ-ನಿಗೆ
ಜಯ
ಜಯಕ್ಕೆ
ಜಯ-ಗಳನ್ನು
ಜಯತೇ
ಜಯದ
ಜಯ-ವನ್ನು
ಜಯ-ವಾಗಲಿ
ಜಯ-ಶಾಲಿ-ಗಳಾಗಿ
ಜಯ-ಶಾಲಿ-ಗಳಾಗು-ವರು
ಜಯ-ಶೀಲ-ರಾದರೆ
ಜಯಿಸ-ಬೇಕು
ಜಯಿಸಲೇ-ಬೇಕು
ಜಯಿಸಿ
ಜಯಿ-ಸುತ್ತದೆ
ಜಯಿಸು-ವೆನೋ
ಜರಿ-ದರೆ
ಜರ್ಝರಿ-ತ-ರಾಗಿ
ಜರ್ಝರಿ-ತ-ವಾಗಿ
ಜರ್ಝರಿ-ತ-ವಾದ
ಜರ್ಮನಿ
ಜರ್ಮನಿಈ
ಜರ್ಮನಿ-ಗಳಿಗೂ
ಜರ್ಮನ್
ಜಲದ
ಜಲ-ದಲ್ಲಿ
ಜಲ-ರಾಶಿ-ಯಂತೆ
ಜವಾ-ಬ್ದಾ-ರರು
ಜವಾ-ಬ್ದಾರಿ
ಜವಾ-ಬ್ದಾರಿ-ಯನ್ನು
ಜವಾ-ಬ್ದಾರಿಯೂ
ಜಾಗ-ದಲ್ಲಿ
ಜಾಗರೂಕ-ರಾಗಿ-ರ-ಬೇಕು
ಜಾಗವಿರು-ತ್ತದೆಯೆ
ಜಾಗೃತ
ಜಾಗೃತ-ಗೊಳಿಸ-ಬಲ್ಲ
ಜಾಗೃತ-ಗೊಳಿಸಿ
ಜಾಗೃತ-ಗೊಳಿ-ಸು-ವು-ದಕ್ಕೆ
ಜಾಗೃತ-ರಾಗಿ
ಜಾಗೃತ-ರಾ-ಗಿಲ್ಲ
ಜಾಗೃತ-ವಾಗಿ
ಜಾಗೃತ-ವಾಗಿ-ತ್ತು
ಜಾಗೃತ-ವಾಗು
ಜಾಗೃತ-ವಾಗು-ತ್ತದೆ
ಜಾಗೃತ-ವಾಗು-ತ್ತಿದೆ
ಜಾಗೃತ-ವಾಗು-ತ್ತಿರು-ವು-ದನ್ನು
ಜಾಗೃತಿ
ಜಾಗೃತಿಗೆ
ಜಾಗೃತಿ-ಯನ್ನು
ಜಾಗೃತಿ-ಯಾಗಿದೆ
ಜಾಗೃತಿ-ಯಿಂದ
ಜಾಗ್ರತ
ಜಾಗ್ರತ-ಗೊಂಡಿತು
ಜಾಗ್ರತ-ಗೊಳಿಸು-ವಷ್ಟು
ಜಾಗ್ರತ-ಗೊಳಿಸು-ವಿರಿ
ಜಾಗ್ರತ-ಗೊಳಿ-ಸು-ವು-ದಕ್ಕೆ
ಜಾಗ್ರತ-ಗೊಳಿಸು-ವು-ದಲ್ಲ
ಜಾಗ್ರತ-ನಾಗಿ-ರುವ
ಜಾಗ್ರತ-ರಾಗಿ
ಜಾಗ್ರತ-ವಾಗಲು
ಜಾಗ್ರತ-ವಾಗಿ
ಜಾಗ್ರತ-ವಾಗಿ-ರುವ
ಜಾಗ್ರತ-ವಾಗು-ತ್ತಿಲ್ಲ
ಜಾಗ್ರತ-ವಾಗು-ವು-ದನ್ನು
ಜಾಗ್ರತ-ವಾಗು-ವು-ದಿಲ್ಲ
ಜಾಗ್ರತ-ವಾದ
ಜಾಗ್ರತಿ
ಜಾಗ್ರ-ತಿಗೆ
ಜಾಟರು
ಜಾಡಿ
ಜಾಡ್ಯವಿ-ರುವ
ಜಾಣ್ಮೆ-ಯಿಂದ
ಜಾತ-ಕ-ಗಳನ್ನು
ಜಾತಿ
ಜಾತಿ-ಕುಲ-ಗಳ
ಜಾತಿ-ಕುಲ-ಗೋತ್ರ-ಗಳ
ಜಾತಿ-ಕುಲ-ಗೋತ್ರ-ಗಳನ್ನು
ಜಾತಿ-ಗಳನ್ನು
ಜಾತಿ-ಗಳಾಗಿ
ಜಾತಿ-ಗಳಿ-ಗಿಂತ
ಜಾತಿ-ಗಳಿಗೂ
ಜಾತಿ-ಗಳಿವೆ
ಜಾತಿ-ಗಳಿ-ವೆಯೆ
ಜಾತಿ-ಗಳು
ಜಾತಿ-ಗಳೂ
ಜಾತಿ-ಗಳೆಲ್ಲಾ
ಜಾತಿ-ಗಳೊಂದಿಗೆ
ಜಾತಿ-ಗಿಂತ
ಜಾತಿಗೆ
ಜಾತಿ-ಚ್ಯುತ
ಜಾತಿ-ಪದ್ಧತಿ
ಜಾತಿ-ಪದ್ಧತಿಗೆ
ಜಾತಿ-ಪದ್ಧತಿ-ಯನ್ನು
ಜಾತಿ-ಭೇದ-ಗಳಿ-ಲ್ಲದೆ
ಜಾತಿಯ
ಜಾತಿ-ಯ-ಲ್ಲಾ-ದರೂ
ಜಾತಿ-ಯಲ್ಲಿ
ಜಾತಿ-ಯಲ್ಲೇ
ಜಾತಿ-ಯ-ವರ
ಜಾತಿ-ಯ-ವ-ರನ್ನು
ಜಾತಿ-ಯ-ವ-ರಿಗೆ
ಜಾತಿ-ಯ-ವರು
ಜಾತಿ-ಯ-ವರೂ
ಜಾತಿ-ಯಾಗು-ತ್ತದೆ
ಜಾತಿಯೂ
ಜಾತಿಯೆ
ಜಾತಿಯೇ
ಜಾತಿ-ವರ್ಗದ-ವರು
ಜಾತಿ-ವಿಷಯ-ದಲ್ಲಿ
ಜಾತಿ-ಸಂಕರ-ದಿಂದ
ಜಾತೋ
ಜಾತ್ಯಂತರ
ಜಾನತಾ-ಮ್
ಜಾನಥ
ಜಾಫ್ನದ
ಜಾಫ್ನಾದ
ಜಾಯ-ಮಾನೋ
ಜಾರ-ಬೇಡಿ
ಜಾರಿ
ಜಾರಿಗೆ
ಜಾರಿ-ಗೆ-ತಂದ-ವ-ರೆಂದು
ಜಾರಿ-ದಾಗ
ಜಾರಿ-ಯಲ್ಲಿ
ಜಾರಿ-ಯ-ಲ್ಲಿದೆ
ಜಾರಿ-ಯ-ಲ್ಲಿವೆ
ಜಾರುತ್ತಿ-ರುವ
ಜಾಲ-ದಲ್ಲಿ
ಜಾಲ-ದಿಂದ
ಜಾಲ-ವನ್ನು
ಜಾವಾ
ಜಾಸ್ತಿ
ಜಿಂಕೆ
ಜಿಂಕೆ-ಗಳು
ಜಿಗುಪ್ಸೆ
ಜಿಜ್ಞಾಸೆ
ಜಿಜ್ಞಾಸೆ-ಯಲ್ಲಿ
ಜಿನನ
ಜಿನನೋ
ಜಿಲ್ಲಾ
ಜಿಹೋ-ವನು
ಜೀರ್ಣ-ವಾಗದೆ
ಜೀರ್ಣಿಸಿ-ಕೊಂಡಿಲ್ಲ
ಜೀರ್ಣಿ-ಸಿ-ಕೊಂಡು
ಜೀರ್ಣಿಸಿ-ಕೊಳ್ಳಿ
ಜೀರ್ಣೋದಂಡೇನ
ಜೀರ್ಣೋದ್ಧಾರ-ಗೊಳಿಸುವ
ಜೀವ
ಜೀವಂತ
ಜೀವಂತ-ನಾಗಿ-ರು-ವನು
ಜೀವಂತ-ವಾಗಿದೆ
ಜೀವಂತ-ವಾಗಿ-ರುವ
ಜೀವಂತ-ವಾಗಿ-ರು-ವುದೇ
ಜೀವಂತ-ವಾಗಿ-ರು-ವೆವು
ಜೀವಂತಿಕೆಯ
ಜೀವ-ಕಳೆ
ಜೀವ-ಕ-ಳೆಯು
ಜೀವ-ಕೋಟಿಗೆ
ಜೀವಕ್ಕೆ
ಜೀವ-ಗಳಲ್ಲಿ
ಜೀವ-ಗಳು
ಜೀವ-ದಾನ
ಜೀವನ
ಜೀವ-ನ-ಕಾಲ-ದಲ್ಲಿ
ಜೀವ-ನಕ್ಕೆ
ಜೀವ-ನ-ಕ್ರಮ-ವನ್ನೂ
ಜೀವ-ನ-ಕ್ರಮವೇ
ಜೀವ-ನ-ಚರಿತ್ರೆ-ಯನ್ನು
ಜೀವ-ನದ
ಜೀವ-ನ-ದಲ್ಲಿ
ಜೀವ-ನ-ದಲ್ಲಿಯೂ
ಜೀವ-ನ-ದಲ್ಲಿಯೇ
ಜೀವ-ನ-ದಲ್ಲೂ
ಜೀವ-ನ-ದಲ್ಲೇ
ಜೀವ-ನ-ದಿಂದ
ಜೀವ-ನ-ದಿಂದಲೂ
ಜೀವ-ನ-ದೆದುರು
ಜೀವ-ನ-ಪ್ರದ
ಜೀವ-ನ-ಪ್ರದ-ವಾಗಿ-ತ್ತು
ಜೀವ-ನ-ಪ್ರದವೋ
ಜೀವ-ನ-ಯಾ-ತ್ರೆ-ಯನ್ನು
ಜೀವ-ನ-ವನ್ನು
ಜೀವ-ನ-ವನ್ನೆಲ್ಲಾ
ಜೀವ-ನ-ವಾಗಿ-ದ್ದ-ರಿಂದ
ಜೀವ-ನ-ವಿಲ್ಲ
ಜೀವ-ನವು
ಜೀವ-ನವೂ
ಜೀವ-ನ-ವೆಲ್ಲ
ಜೀವ-ನ-ವೆಲ್ಲಾ
ಜೀವ-ನವೇ
ಜೀವ-ನಾದರ್ಶ
ಜೀವ-ನಾದ್ಯಂತವೂ
ಜೀವ-ನಿಗೆ
ಜೀವ-ನಿಗೇ
ಜೀವನು
ಜೀವ-ನೋ-ತ್ಸಾಹ-ವನ್ನು
ಜೀವ-ನೋ-ದ್ದೇಶ-ಗಳು
ಜೀವ-ನೋ-ದ್ದೇಶ-ವಿದೆ
ಜೀವ-ನೋ-ದ್ಧಾ-ರದ
ಜೀವ-ನ್ಮುಕ್ತ-ರಾದ
ಜೀವ-ಪೋಷಕ
ಜೀವ-ಬ್ರಹ್ಮ-ರಿಗೆ
ಜೀವ-ಮಾ-ನ-ವೆಲ್ಲ
ಜೀವ-ಮಾ-ನ-ವೆಲ್ಲಾ
ಜೀವ-ರಷ್ಟೇ
ಜೀವ-ವನ್ನು
ಜೀವ-ವನ್ನೇ
ಜೀವ-ವಿಲ್ಲ
ಜೀವವು
ಜೀವವೂ
ಜೀವ-ಶಕ್ತಿ
ಜೀವ-ಶಕ್ತಿ-ಯನ್ನು
ಜೀವ-ಸ-ಹಿತ
ಜೀವ-ಸಿದ್ಧಾಂತ-ದಲ್ಲಿ
ಜೀವಾತ್ಮ
ಜೀವಾ-ತ್ಮ-ಗಳು
ಜೀವಾ-ತ್ಮದ
ಜೀವಾ-ತ್ಮನ
ಜೀವಾ-ತ್ಮ-ನಿಗೂ
ಜೀವಾ-ತ್ಮ-ನಿಗೆ
ಜೀವಾ-ತ್ಮನು
ಜೀವಾ-ತ್ಮ-ನೆಂದೂ
ಜೀವಾ-ತ್ಮಲ್ಲಿ
ಜೀವಾ-ತ್ಮವು
ಜೀವಾಳ
ಜೀವಾ-ಳ-ವನ್ನು
ಜೀವಾ-ಳ-ವಾದ
ಜೀವಿ
ಜೀವಿ-ಗಳ
ಜೀವಿ-ಗಳಂತೆ
ಜೀವಿ-ಗಳನ್ನು
ಜೀವಿ-ಗಳಲ್ಲಿ
ಜೀವಿ-ಗಳಿ-ಗಿಂತ
ಜೀವಿ-ಗಳು
ಜೀವಿಯ
ಜೀವಿ-ಯ-ನ್ನಾಗಿ
ಜೀವಿ-ಯನ್ನು
ಜೀವಿ-ಯಲ್ಲಿ
ಜೀವಿ-ಯ-ಲ್ಲಿಯೂ
ಜೀವಿಯೂ
ಜೀವಿ-ಸ-ಬಲ್ಲ
ಜೀವಿ-ಸ-ಬೇಕಾಗಿದೆ
ಜೀವಿ-ಸ-ಲಾ-ರರು
ಜೀವಿ-ಸಿದ್ದು
ಜೀವಿ-ಸಿರ-ಬೇ-ಕಾದರೆ
ಜೀವಿ-ಸಿರು-ವೆವು
ಜೀವಿ-ಸು-ತ್ತಿದೆ
ಜೀವಿ-ಸು-ತ್ತಿ-ರಲಿಲ್ಲ
ಜೀವಿ-ಸುತ್ತಿ-ರುವ
ಜೀವಿ-ಸುತ್ತಿರು-ವುದೋ
ಜೀವಿ-ಸುತ್ತೇವೆ
ಜೀವಿ-ಸು-ವಂತೆ
ಜೀವಿ-ಸು-ವರೋ
ಜೀವಿ-ಸು-ವೆವು
ಜೀಸ-ಸ್
ಜುಗುಪ್ಸೆ
ಜುಗುಪ್ಸೆ-ಯಾಗು-ವುದು
ಜುಗುಪ್ಸೆ-ಯಿಂದ
ಜುಗುಪ್ಸೆ-ಯುಂಟಾಗು-ವುದು
ಜುಷ್ಟಂ
ಜೂಜನ್ನು
ಜೂನ್
ಜೇಡ
ಜೇಬಿ-ನಲ್ಲಿ
ಜೈನ-ಧರ್ಮ
ಜೈನ-ಧರ್ಮ-ಗಳು
ಜೈನ-ಧರ್ಮದ
ಜೈನರ
ಜೈನ-ರಾಗಲೀ
ಜೈನರು
ಜೈನರೂ
ಜೈಲು
ಜೊತೆಗೆ
ಜೊತೆಗೇ
ಜೊತೆ-ಯಲ್ಲಿ
ಜೊತೆ-ಯಲ್ಲೇ
ಜೊತೆ-ಯಾಗಿಯೇ
ಜೊಳ್ಳು
ಜೋಡು
ಜೋಪಡಿ-ಗಳನ್ನು
ಜೋಪಾನ-ದಿಂದಿರ-ಬೇಕು
ಜೋಪಾನ-ವಾಗಿ-ರ-ಬೇಕು
ಜೋಪಾನ-ವಾಗಿರಿ
ಜೋಪಾ-ನ-ವಾಗಿ-ರು-ವುದು
ಜೋಪಾನ-ವಾಗಿ-ಲ್ಲದೇ
ಜೋರಾ-ಗಿದೆ
ಜ್ಞಾತೃ
ಜ್ಞಾತ್ವಾ
ಜ್ಞಾನ
ಜ್ಞಾನ-ಕಾಂಡ
ಜ್ಞಾನ-ಕಾಂಡಕ್ಕೆ
ಜ್ಞಾನ-ಕಾಂಡ-ಗಳ
ಜ್ಞಾನ-ಕಾಂಡ-ದಲ್ಲಿ
ಜ್ಞಾನ-ಕಾಂಡವು
ಜ್ಞಾನ-ಕಾಂಡ-ವೆಂದು
ಜ್ಞಾನ-ಕೂಡ
ಜ್ಞಾನ-ಕ್ಕಿಂತ
ಜ್ಞಾನ-ಕ್ಕಿಂತಲೂ
ಜ್ಞಾನಕ್ಕೂ
ಜ್ಞಾನಕ್ಕೆ
ಜ್ಞಾನ-ಜೋತ್ಯಿ-ಯನ್ನು
ಜ್ಞಾನದ
ಜ್ಞಾನ-ದಲ್ಲಿ
ಜ್ಞಾನ-ದಾನ
ಜ್ಞಾನ-ದಾನ-ವೆಂಬ
ಜ್ಞಾನ-ದಿಂದ
ಜ್ಞಾನ-ಬೋಧೆ-ಗಳಿಂದ
ಜ್ಞಾನ-ಮಯ
ಜ್ಞಾನ-ರಾಶಿಗೆ
ಜ್ಞಾನ-ರಾಶಿ-ಯನ್ನು
ಜ್ಞಾನ-ರಾಶಿ-ಯನ್ನೇ
ಜ್ಞಾನ-ರಾಶಿ-ಯಾದ
ಜ್ಞಾನ-ವನ್ನು
ಜ್ಞಾನ-ವನ್ನೂ
ಜ್ಞಾನ-ವಿಲ್ಲ
ಜ್ಞಾನ-ವಿಲ್ಲದ
ಜ್ಞಾನವು
ಜ್ಞಾನವೂ
ಜ್ಞಾನ-ವೆಂದ-ರೇನು
ಜ್ಞಾನ-ವೆಂಬು-ದನ್ನು
ಜ್ಞಾನ-ವೆಲ್ಲ
ಜ್ಞಾನ-ವೆಲ್ಲಾ
ಜ್ಞಾನವೇ
ಜ್ಞಾನ-ಸಂಪತ್ತನ್ನು
ಜ್ಞಾನ-ಸಮ-ಷ್ಟಿಗೆ
ಜ್ಞಾನ-ಸಾಗರ
ಜ್ಞಾನ-ಸಾರ-ವನ್ನೇ
ಜ್ಞಾನ-ಸುರೆ-ಯನ್ನು
ಜ್ಞಾನಾ-ಗಾರದ
ಜ್ಞಾನಿ
ಜ್ಞಾನಿ-ಗಳನ್ನು
ಜ್ಞಾನಿ-ಗಳು
ಜ್ಞಾನಿಯು
ಜ್ಞಾನಿಯೂ
ಜ್ಞಾನು-ವುಂಟಾ-ದಾಗ
ಜ್ಞಾಪಕಕ್ಕೆ
ಜ್ಞಾಪಕ-ದಲ್ಲಿ
ಜ್ಞಾಪಕ-ದಲ್ಲಿ-ಟ್ಟಿರ-ಬೇಕು
ಜ್ಞಾಪಕ-ದಲ್ಲಿ-ಡ-ಬೇ-ಕಾದ
ಜ್ಞಾಪಕ-ದಲ್ಲಿ-ಡಬೇ-ಕಾ-ದುದೇ
ಜ್ಞಾಪಕ-ದಲ್ಲಿ-ಡ-ಬೇಕು
ಜ್ಞಾಪಕ-ದಲ್ಲಿ-ಡ-ಬೇಕೆಂದು
ಜ್ಞಾಪಕ-ದಲ್ಲಿ-ರ-ಬಹುದು
ಜ್ಞಾಪಿ-ಸಲು
ಜ್ಞಾಪಿಸಿ
ಜ್ಞೇಯ-ಕ್ಕಿಂತ
ಜ್ಯೋತಿ
ಜ್ಯೋ-ತಿಯ
ಜ್ಯೋ-ತಿ-ಯನ್ನು
ಜ್ಯೋ-ತಿ-ಯಾಗು-ವಂತೆ
ಜ್ಯೋ-ತಿಯು
ಜ್ಯೋ-ತಿರ್
ಜ್ಯೋ-ತಿರ್ಮಯ
ಜ್ಯೋ-ತಿಷ್ಯ
ಜ್ವರ
ಜ್ವರ-ವನ್ನು
ಜ್ವಲಂತ
ಜ್ವಾ-ಲಾಮುಖಿಯ
ಜ್ವಾ-ಲೆಯ
ಜ್ವಾ-ಲೆ-ಯಿಂದ
ಝರಿ
ಝರಿ-ಗಳು
ಟಂಕಸಾಲೆ
ಟಾರ್ಟರ್
ಟಿಬೆಟ್ಟಿ-ನಲ್ಲಿ-ರುವ
ಟಿಬೆಟ್ಟಿ-ನಿಂದ
ಟೀಕಾ-ಚಾರ್ಯರು
ಟೀಕಿಸ-ಬಹುದು
ಟೀಕಿ-ಸಲು
ಟೀಕಿ-ಸಿದರು
ಟೀಕಿಸಿ-ದರೂ
ಟೀಕಿಸಿರಲೂ
ಟೀಕಿಸಿರು
ಟೀಕಿಸುತ್ತಾರೆ
ಟೀಕಿ-ಸು-ತ್ತಿದ್ದರು
ಟೀಕಿ-ಸುತ್ತಿ-ದ್ದಾರೆ
ಟೀಕಿ-ಸು-ವರು
ಟೀಕಿಸು-ವು-ದನ್ನು-ಪದೇ
ಟೀಕಿಸು-ವುದು
ಟೀಕೆ-ಗಳೂ
ಟ್ರೆಪ್ಲಿ-ಕೇನ್
ಠೀವಿ-ಯಿಂದ
ಡಂಬರ-ಗಳಾಗಿವೆ
ಡಚ್ಚರು
ಡನೆಯೇ
ಡಾಂಟೆ
ಡಾಕ್ಟರ್
ಡಾಯ್ಸನ್ರಂತಹ
ಡಾಲರು
ಡಿಂಡಿಮ-ದಿಂದ
ಡಿಗ್ರಿ
ಡಿಸೆಂಬರ್
ಡೆಮೊಕ್ರಾ-ಟ್
ಢಾಕಾ-ದಲ್ಲಿ
ತಂಗಾಳಿ-ಯಂತೆ
ತಂಗಿದ
ತಂಗಿದ್ದರು
ತಂಡ-ದ-ವರು
ತಂಡ-ವನ್ನು
ತಂಡೋಪತಂಡ-ವಾಗಿ
ತಂತಮ್ಮ
ತಂತಿ-ಯನ್ನು
ತಂತಿಸ-ಮಾ-ಚಾರ-ವನ್ನು
ತಂತ್ರ
ತಂತ್ರಕ್ಕೆ
ತಂತ್ರ-ಗಳಲ್ಲಿ
ತಂತ್ರ-ಗಳಿಂದ
ತಂತ್ರ-ಗಳಿವೆ
ತಂತ್ರ-ಗಳು
ತಂತ್ರ-ವನ್ನು
ತಂತ್ರವು
ತಂತ್ರ-ಶಾಸ್ತ್ರ
ತಂತ್ರ-ಶಾಸ್ತ್ರ-ದಲ್ಲಿ
ತಂತ್ರ-ಶಾಸ್ತ್ರ-ದಲ್ಲಿಯೂ
ತಂತ್ರ-ಶಾಸ್ತ್ರ-ದಲ್ಲಿ-ರು-ವಂತೆಯೇ
ತಂತ್ರ-ಶಾಸ್ತ್ರ-ವನ್ನು
ತಂತ್ರ-ಶಾಸ್ತ್ರ-ವೆಂಬುದೂ
ತಂದ
ತಂದರು
ತಂದರೂ
ತಂದರೆ
ತಂದ-ವರು
ತಂದಿದೆ
ತಂದಿ-ದ್ದರು
ತಂದಿ-ದ್ದರೆ
ತಂದಿ-ದ್ದೀರಿ
ತಂದು
ತಂದು-ಕೊಂಡು
ತಂದು-ಕೊ-ಡುತ್ತಾ
ತಂದು-ಕೊಳ್ಳಿ
ತಂದು-ಕೊಳ್ಳಿಓಂ
ತಂದು-ಕೊಳ್ಳು-ವಂತೆ
ತಂದೆ
ತಂದೆ-ಗಳ
ತಂದೆಗೆ
ತಂದೆಯ
ತಂದೊ-ಡ-ನೆಯೇ
ತಂದೊಡ್ಡಲು
ತಂದೊಡ್ಡಿ
ತಕ್ಕ
ತಕ್ಕಂತೆ
ತಕ್ಕಡಿ
ತಕ್ಕ-ಭಾಷೆ
ತಕ್ಷಣ
ತಕ್ಷಣದ
ತಕ್ಷಣವೆ
ತಕ್ಷಣವೇ
ತಗಡನ್ನು
ತಗಲು-ತ್ತಿದೆ
ತಗೊ
ತಗ್ಗಿ-ಸಿತು
ತಗ್ಗಿಸಿ-ರು-ವರು
ತಟ್ಟಿ-ದೆನು
ತಟ್ಟು-ವಾಗ
ತಡ
ತಡ-ವಾಗು-ತ್ತಿದೆ
ತಡ-ವಾದರೂ
ತಡೆಗಟ್ಟಿ
ತಡೆಗಟ್ಟು-ತ್ತವೆ
ತಡೆದು
ತಡೆಯ-ಬಲ್ಲ-ವ-ರಿಲ್ಲ
ತಡೆಯ-ಬೇಕಾಗಿದೆ
ತಡೆಯ-ಬೇಕು
ತಡೆ-ಯ-ಲಾ-ಗು-ವು-ದಿಲ್ಲ
ತಡೆಯ-ಲಾ-ರದು
ತಡೆಯ-ಲಾ-ರರು
ತಡೆ-ಯಲು
ತಡೆ-ಯು-ವಂತೆ
ತಡೆ-ಯು-ವು-ದಕ್ಕೆ
ತಡೆಯು-ವುದು
ತಣ್ಣಗಾ-ಯಿತು
ತತಃ
ತತೋ
ತತ್
ತತ್ಕಾಲಕ್ಕೆ
ತತ್ಕ್ಷಣವೇ
ತತ್ತ
ತತ್ತ-ರಿಸಿ
ತತ್ತ-ರಿಸುತ್ತಿ-ರು-ವರು
ತತ್ತ್ವ
ತತ್ತ್ವ-ಕೂಡ
ತತ್ತ್ವ-ಕ್ಕಾಗಿ
ತತ್ತ್ವಕ್ಕೂ
ತತ್ತ್ವಕ್ಕೆ
ತತ್ತ್ವ-ಗಳ
ತತ್ತ್ವ-ಗಳನ್ನು
ತತ್ತ್ವ-ಗಳಲ್ಲಿ
ತತ್ತ್ವ-ಗಳ-ಲ್ಲೊಂದಾ-ಗಿದೆ
ತತ್ತ್ವ-ಗಳಾವುವೂ
ತತ್ತ್ವ-ಗಳಿಗೆ
ತತ್ತ್ವ-ಗಳು
ತತ್ತ್ವ-ಗಳೂ
ತತ್ತ್ವ-ಗಳೆಲ್ಲ
ತತ್ತ್ವ-ಜ್ಞರ
ತತ್ತ್ವ-ಜ್ಞರು
ತತ್ತ್ವ-ಜ್ಞಾನ-ದಿಂದ
ತತ್ತ್ವ-ಜ್ಞಾನಿ
ತತ್ತ್ವ-ಜ್ಞಾನಿ-ಗಳನ್ನು
ತತ್ತ್ವ-ಜ್ಞಾನಿ-ಯಂತೆ-ಹುಡುಗ
ತತ್ತ್ವತಃ
ತತ್ತ್ವದ
ತತ್ತ್ವ-ದರ್ಶಿ
ತತ್ತ್ವ-ದರ್ಶಿ-ಗಳು
ತತ್ತ್ವ-ದಲ್ಲೆಲ್ಲ
ತತ್ತ್ವ-ದಿಂದ
ತತ್ತ್ವ-ನಿಷ್ಠ
ತತ್ತ್ವ-ಪ್ರ-ಚಾರ-ದಿಂದ
ತತ್ತ್ವ-ಬೋಧ-ಕರು
ತತ್ತ್ವ-ಮಸಿ
ತತ್ತ್ವ-ಮ-ಸಿಯೇ
ತತ್ತ್ವ-ರತ್ನ-ಗಳು
ತತ್ತ್ವ-ವನ್ನಾ-ಗಲೀ
ತತ್ತ್ವ-ವನ್ನು
ತತ್ತ್ವ-ವನ್ನೇ
ತತ್ತ್ವ-ವಾಗಲೀ
ತತ್ತ್ವ-ವಾಗಿದೆ
ತತ್ತ್ವ-ವಿದೆ
ತತ್ತ್ವ-ವಿದ್ದರೆ
ತತ್ತ್ವ-ವಿರ-ಬೇಕು
ತತ್ತ್ವವು
ತತ್ತ್ವವೂ
ತತ್ತ್ವ-ವೊಂದರ
ತತ್ತ್ವ-ಶಾಸ್ತ್ರ-ಗಳ
ತತ್ತ್ವ-ಶಾಸ್ತ್ರ-ಗಳಲ್ಲಿ
ತತ್ತ್ವ-ಶಾಸ್ತ್ರ-ಗಳಿಗೂ
ತತ್ತ್ವ-ಶಾಸ್ತ್ರ-ಗಳು
ತತ್ತ್ವ-ಶಾಸ್ತ್ರ-ಜ್ಞ-ನಾದ
ತತ್ತ್ವ-ಶಾಸ್ತ್ರದ
ತತ್ತ್ವ-ಶಾಸ್ತ್ರ-ದಲ್ಲಿ
ತತ್ತ್ವ-ಶಾಸ್ತ್ರ-ದಲ್ಲಿಯೂ
ತತ್ತ್ವ-ಶಾಸ್ತ್ರವು
ತತ್ರ
ತತ್ವ
ತತ್ವ-ಗಳ
ತತ್ವ-ಗಳನ್ನು
ತತ್ವ-ಗಳು
ತತ್ವ-ಗಳೆಲ್ಲ
ತತ್ವ-ಜ್ಞಾನಿ
ತತ್ವ-ಜ್ಞಾನಿಗೆ
ತತ್ವದ
ತತ್ವ-ದರ್ಶಿ
ತತ್ವ-ಬೋಧೆ-ಯಲ್ಲಿ
ತತ್ವ-ಮಸಿ
ತತ್ವ-ವನ್ನಾ-ದರೂ
ತತ್ವ-ವನ್ನು
ತತ್ವ-ವನ್ನೇ
ತತ್ವ-ವಿದೆ
ತತ್ವವು
ತತ್ವ-ಶಾಸ್ತ್ರ
ತತ್ವ-ಶಾಸ್ತ್ರ-ಗಳ-ಲ್ಲಿಯೂ
ತತ್ವ-ಶಾಸ್ತ್ರ-ಜ್ಞರು
ತತ್ವ-ಶಾಸ್ತ್ರದ
ತತ್ವ-ಶಾಸ್ತ್ರ-ಪ್ರಾ-ರಂಭ-ವಾಗು-ತ್ತದೆ
ತದಾತ್ಮಾನಂ
ತದ್ವದ್ಭುಕ್ತಯೇ
ತನ
ತನಕ
ತನ-ಗಿಂತಲೂ
ತನಗೂ
ತನಗೆ
ತನಗೇ
ತನಗೋ
ತನ-ದಿಂದ
ತನಯ
ತನ್ನ
ತನ್ನ-ದೀರ್ಘ
ತನ್ನ-ದೆಂದು
ತನ್ನದೇ
ತನ್ನ-ದೊಂದು
ತನ್ನನ್ನು
ತನ್ನನ್ನೂ
ತನ್ನಲ್ಲಿ
ತನ್ನ-ಲ್ಲಿಯೂ
ತನ್ನ-ಲ್ಲಿಯೇ
ತನ್ನಲ್ಲಿ-ರುವ
ತನ್ನ-ವ-ರಿಗೆ
ತನ್ನ-ಷ್ಟಕ್ಕೆ
ತನ್ನಷ್ಟೇ
ತನ್ನಿ
ತನ್ನಿಂದ
ತನ್ನಿಂದಲೇ
ತನ್ನಿ-ಷ್ಟ-ಮಿತ್ರ
ತನ್ನೆಲ್ಲ
ತನ್ನೊಂದಿಗೆ
ತನ್ಮ-ತಾನು-ಸಾರೇಣ
ತನ್ಮಯ-ರಾಗಿ-ರು-ವಾಗ
ತನ್ಮಯ-ವಾಗಿ-ತ್ತು
ತನ್ಮಾ-ತ್ರ-ಗಳು
ತನ್ಮಾ-ತ್ರ-ದಿಂದ
ತನ್ಮೂಲಕ
ತಪಶ್ಯಕ್ತಿ-ಯಿಂದ
ತಪ-ಸ್
ತಪಸ್ವಿ-ಗಳ
ತಪಸ್ವಿ-ಗಳಾದ-ವರ
ತಪಸ್ವಿ-ಗಳು
ತಪ-ಸ್ಸನ್ನು
ತಪ-ಸ್ಸಿನ
ತಪ-ಸ್ಸಿ-ನಿಂದ
ತಪಸ್ಸು
ತಪೋ-ಭೂಮಿ-ಯಾಗಿದೆ
ತಪ್ಪದು
ತಪ್ಪನ್ನು
ತಪ್ಪಲ್ಲ
ತಪ್ಪಾಗ-ಲಾ-ರದು
ತಪ್ಪಾಗಿ
ತಪ್ಪಾಗಿ-ದ್ದರೂ
ತಪ್ಪಾಗು-ವುದು
ತಪ್ಪಿ
ತಪ್ಪಿ-ತೆಂದರೆ
ತಪ್ಪಿದ
ತಪ್ಪಿ-ದ್ದರೂ
ತಪ್ಪಿ-ದ್ದಲ್ಲ
ತಪ್ಪಿ-ನಿಂದ
ತಪ್ಪಿ-ನಿಂದಲೋ
ತಪ್ಪಿ-ರು-ವು-ದಿಲ್ಲ
ತಪ್ಪಿ-ಸಿ-ಕೊಂಡನು
ತಪ್ಪಿ-ಸಿಕೊಳ್ಳ-ಲೋಸುಗವೇ
ತಪ್ಪಿ-ಸಿ-ಕೊಳ್ಳು-ವು-ದಕ್ಕೆ
ತಪ್ಪಿ-ಸಿ-ಕೊಳ್ಳು-ವುದು
ತಪ್ಪಿ-ಸಿದ
ತಪ್ಪಿ-ಸಿದರೆ
ತಪ್ಪು
ತಪ್ಪು-ಗಳನ್ನು
ತಪ್ಪು-ಗಳನ್ನೂ
ತಪ್ಪು-ಗಳಿವೆ
ತಪ್ಪು-ಗಳು
ತಪ್ಪು-ತ್ತದೆ
ತಪ್ಪೆ
ತಪ್ಪೆಂದು
ತಬ್ಬಿ-ರುವ
ತಮ
ತಮ-ಗಿಂತ
ತಮ-ಗಿ-ರುವ
ತಮಗೂ
ತಮಗೆ
ತಮ-ಗೆಯೇ
ತಮ-ಗೆಲ್ಲ
ತಮಗೇ
ತಮಸಾ
ತಮ-ಸ್ಸನ್ನು
ತಮ-ಸ್ಸಿ-ನಲ್ಲಿ
ತಮಸ್ಸು
ತಮಾ-ಷೆಯೇ
ತಮೇವ
ತಮೇ-ವೈಕಂ
ತಮೋ
ತಮೋ-ಗು-ಣವು
ತಮ್ಮ
ತಮ್ಮಂತಹ
ತಮ್ಮಂತೆಯೇ
ತಮ್ಮದು
ತಮ್ಮದೇ
ತಮ್ಮನ್ನು
ತಮ್ಮನ್ನೂ
ತಮ್ಮ-ನ್ನೇ
ತಮ್ಮಲ್ಲಿ
ತಮ್ಮ-ಲ್ಲಿಗೆ
ತಮ್ಮ-ಲ್ಲಿದ್ದ
ತಮ್ಮ-ಲ್ಲಿ-ರುವ
ತಮ್ಮ-ಲ್ಲಿ-ರು-ವು-ದನ್ನು
ತಮ್ಮಲ್ಲೂ
ತಮ್ಮಲ್ಲೇ
ತಮ್ಮ-ವ-ರನ್ನೂ
ತಮ್ಮೊಡನೆ
ತಮ್ಮೊಳಗೇ
ತಯಾ-ರಾಗುತ್ತಿ-ರುವ
ತಯಾ-ರಾ-ಗು-ತ್ತಿವೆ
ತಯಾ-ರಾ-ಗುವ
ತಯಾ-ರಿಸಿ-ರು-ವುದು
ತಯಾರು
ತಯಾ-ರು-ಮಾಡಿ
ತಯಾ-ರು-ಮಾಡಿ-ಕೊಂಡಿ-ದ್ದೀರಿ
ತಯಾ-ರು-ಮಾಡಿ-ದಾಗ
ತಯಾ-ರು-ಮಾಡುವ
ತಯಾ-ರು-ಮಾಡು-ವಂತೆ
ತಯೋ-ರನ್ಯಃ
ತರ
ತರಂಗ
ತರಂಗ-ಗಳನ್ನು
ತರಂಗ-ಗಳು
ತರಂಗ-ಗಳೇ-ಳು-ತ್ತವೆ
ತರಂಗ-ದೊಂದಿಗೇ
ತರಂಗ-ರೂಪದ
ತರಂಗ-ವಾಗಿ-ರ-ಬಹುದು
ತರಂಗ-ವಿದೆ-ಯೆಂದು
ತರಂಗವು
ತರಂಗಿತ
ತರ-ಕಾರಿ
ತರ-ಕಾರಿ-ಯನ್ನು
ತರ-ಕ್ಕೆಂದು
ತರ-ಗತಿ-ಗಳ
ತರ-ಗಳಲ್ಲಿ
ತರ-ತಕ್ಕ-ದ್ದಾ-ಗಿದೆ
ತರ-ತಮ-ದಲ್ಲಿ
ತರ-ತ-ರದ
ತರದ
ತರ-ಬಲ್ಲರು
ತರ-ಬಹುದು
ತರ-ಬೇಕಾಗಿಲ್ಲ
ತರ-ಬೇ-ಕಾದ
ತರ-ಬೇಕು
ತರ-ಬೇಕೆಂದಿರು
ತರ-ಬೇಕೆಂದು
ತರ-ಬೇತಿ-ಯಾಗಲೀ
ತರ-ಬೇತು
ತರ-ಲಾ-ರದು
ತರ-ಲಾ-ರರು
ತರಲಿ
ತರಲು
ತರ-ಲೆತ್ನಿಸು-ವುದು
ತರವೆ
ತರ-ಹದ
ತರುಣ
ತರು-ಣ-ನೊಬ್ಬ
ತರು-ಣ-ರಿಗೆ
ತರು-ಣ-ರಿರಾ
ತರು-ಣರೇ
ತರುಣಿ
ತರು-ಣಿ-ಯೊಬ್ಬ-ಳನ್ನು
ತರು-ತ್ತಿದೆ
ತರು-ತ್ತೇನೆ
ತರುವ
ತರು-ವರು
ತರು-ವಾಗ-ಬಹುದು
ತರು-ವಾಯ
ತರುವು
ತರುವು-ದ-ಕ್ಕಾಗಿ
ತರುವು-ದಕ್ಕೆ
ತರುವು-ದರ
ತರುವು-ದಿಲ್ಲ
ತರು-ವುದು
ತರುವು-ದು-ಕ್ಕೋಸ್ಕರ
ತರು-ವುದೇ
ತರು-ವೆನು
ತರು-ವೆವು
ತರೋಣ
ತರ್ಕ
ತರ್ಕಕ್ಕೆ
ತರ್ಕ-ದಲ್ಲಿ
ತರ್ಕ-ದಲ್ಲಿಯೂ
ತರ್ಕ-ದಲ್ಲೇ
ತರ್ಕ-ದೋಷ
ತರ್ಕ-ಬದ್ಧ
ತರ್ಕ-ಬದ್ಧ-ವಾದುದು
ತರ್ಕ-ವನ್ನೆಲ್ಲಾ
ತರ್ಕ-ವಲ್ಲ
ತರ್ಕ-ಶಕ್ತಿ-ಯಲ್ಲಿ
ತರ್ಜುಮೆ
ತಲು-ಪಿದ
ತಲೆ
ತಲೆ-ಕೆಳಗೆ
ತಲೆಗೆ
ತಲೆ-ತಿರು-ಗು-ವಂತೆ
ತಲೆ-ದೋರು-ತ್ತದೆ
ತಲೆ-ದೋರು-ತ್ತಿದೆ
ತಲೆ-ನೋವು
ತಲೆ-ಬಾಗಿ
ತಲೆ-ಮಾ-ರಿನ-ವ-ರಾದ
ತಲೆ-ಮಾರು
ತಲೆಯ
ತಲೆ-ಯ-ನ್ನಾ-ದರೂ
ತಲೆ-ಯನ್ನು
ತಲೆ-ಯಲ್ಲಿ
ತಲೆಯೂ
ತಲೆ-ಯೆ-ತ್ತಲು
ತಲೆ-ಯೆತ್ತು-ತ್ತಿತ್ತು
ತಲೆ-ಯೆತ್ತುವ
ತಲೆ-ಯೆತ್ತು-ವುದೋ
ತಲೆಯೇ
ತಲೆ-ಯೊಳಗೆ
ತಲ್ಲಣಿಸುತ್ತಿತ್ತು
ತಲ್ಲೀನ-ವಾಗಿ-ರುವ
ತಲ್ಲೀ-ನ-ವಾಗಿ-ರು-ವುದು
ತಳ
ತಳ-ಪಾ-ಯದ
ತಳ-ಹದಿ
ತಳ-ಹದಿ-ಗಳನ್ನು
ತಳ-ಹದಿ-ಗಳೆಲ್ಲಾ
ತಳ-ಹದಿಗೆ
ತಳ-ಹದಿಯ
ತಳ-ಹದಿ-ಯನ್ನು
ತಳ-ಹದಿಯೇ
ತಳಿರು
ತಳಿರು-ಗಳಾಗಲೇ
ತಳೆ-ದಿದ್ದ
ತಳ್ಳಿ-ಹಾ-ಕು-ವುದು
ತಳ್ಳುವ
ತಳ್ಳು-ವರು
ತವಕಪಡುತ್ತಿ-ರು-ವರು
ತವಕಪಡು-ವುದು
ತವರು
ತವರೂರು
ತವ್ಯಃ
ತಸ್ಮಾ-ತ್
ತಸ್ಯ
ತಸ್ಯ-ತ-ತ್
ತಾಗಿ
ತಾಗಿ-ತೆನ್ನ-ದಿರಿ
ತಾತ
ತಾತ್ಕಾಲಿಕ
ತಾತ್ಕಾಲಿಕ-ವಾಗಿ
ತಾತ್ತ್ವಿಕ
ತಾತ್ತ್ವಿಕ-ವಾಗಿ
ತಾತ್ತ್ವಿಕ-ವಾದ
ತಾತ್ವಿಕ
ತಾತ್ವಿ-ಕರ
ತಾತ್ವಿಕ-ವಾಗಿ
ತಾತ್ಸಾರ-ದಿಂದ
ತಾತ್ಸಾರ-ವುಂಟಾ-ಗಿದೆ
ತಾದಾತ್ಮ್ಯ-ಭಾವ
ತಾನಾಗಿ
ತಾನಾ-ಗಿಯೇ
ತಾನು
ತಾನು-ರಿಪಬ್ಲಿ-ಕ್
ತಾನೂ
ತಾನೆ
ತಾನೇ
ತಾನೊಬ್ಬ
ತಾಯಿ
ತಾಯಿ-ಗಳನ್ನು
ತಾಯಿ-ನಾಡಿನ
ತಾಯಿಯ
ತಾಯ್ನಾಡಿಗೆ
ತಾಯ್ನಾಡಿ-ನಲ್ಲಿ
ತಾರಕಂ
ತಾರ-ತಮ್ಯ
ತಾರ-ತಮ್ಯ-ವನ್ನು
ತಾರ-ಯನ್ತಃ
ತಾರ-ಸ್ವರ
ತಾರಾ
ತಾರಾ-ವಳಿ-ಗಳನ್ನು
ತಾರಾ-ವಳಿ-ಗಳೂ
ತಾರುಣ್ಯ
ತಾರುಣ್ಯದ
ತಾರೆ
ತಾರೆ-ಗಳನ್ನು
ತಾರೆ-ಗಳು
ತಾರೆಯೂ
ತಾರ್ಕಿಕ
ತಾರ್ಕಿ-ಕರ
ತಾಳ-ಬಲ್ಲ-ವರು
ತಾಳಬೇ-ಕಿತ್ತು
ತಾಳ-ಬೇಕು
ತಾಳಿ
ತಾಳಿ-ಕೊಂಡಿ-ರಲು
ತಾಳಿತು
ತಾಳಿದ
ತಾಳಿದೆ
ತಾಳಿ-ರುವ
ತಾಳಿ-ರು-ವುವು
ತಾಳು-ತ್ತದೆ
ತಾಳು-ತ್ತದೆಯೋ
ತಾಳು-ತ್ತಿದೆ
ತಾಳು-ತ್ತಿವೆ
ತಾಳು-ವಂತೆ
ತಾಳು-ವರು
ತಾಳು-ವುದು
ತಾಳ್ಮೆ
ತಾಳ್ಮೆ-ಯಿಂದ
ತಾಳ್ಮೆ-ಯಿ-ರಲಿ
ತಾವಾಗಿಯೇ
ತಾವಾದರೋ
ತಾವಿಲ್ಲ-ದಂತಾ-ಗಿದೆ
ತಾವು
ತಾವು-ಲಂಡನ್ನನ್ನು
ತಾವೂ
ತಾವೆಂದು
ತಾವೇ
ತಿಂಗಳಲ್ಲಿ
ತಿಂಗಳು-ಗಳ
ತಿಂಗಳು-ಗಳು
ತಿಂದಂತೆಲ್ಲಾ
ತಿಂದ-ದ್ದನ್ನು
ತಿಂದನು
ತಿಂದಿರ-ಬಹು-ದೆಂದು
ತಿಂದು
ತಿತಿಕ್ಷಾ-ಜೀವನ
ತಿತಿಕ್ಷೆ
ತಿದ್ದಲೆಳ-ಸು-ವರು
ತಿದ್ದಿ-ಕೊಳ್ಳು-ವು-ದಕ್ಕೂ
ತಿದ್ದಿ-ಕೊಳ್ಳು-ವುದು
ತಿದ್ದುವ
ತಿನ್ನದ-ವ-ರನ್ನು
ತಿನ್ನದೆ
ತಿನ್ನಲು
ತಿನ್ನಲೂ
ತಿನ್ನುತ್ತಾ
ತಿನ್ನು-ತ್ತಿದೆ
ತಿನ್ನುತ್ತಿ-ರು-ವನು
ತಿನ್ನು-ತ್ತಿಲ್ಲ
ತಿನ್ನು-ವಂತೆ
ತಿನ್ನು-ವು-ದನ್ನು
ತಿನ್ನು-ವುದು
ತಿರಸ್ಕರಿಸ-ಬೇಕು
ತಿರಸ್ಕರಿ-ಸಿದರೆ
ತಿರಸ್ಕ-ರಿಸಿಯೋ
ತಿರಸ್ಕ-ರಿಸಿ-ರು-ವರು
ತಿರಸ್ಕಾರ-ಭಾವ
ತಿರು-ಕನೂ
ತಿರುಗ-ಬೇಕು
ತಿರುಗಿ
ತಿರುಗಿತು
ತಿರುಗಿ-ದರು
ತಿರುಗಿ-ರು-ವರೋ
ತಿರುಗಿ-ಸಲು
ತಿರುಗಿ-ಸಿದರು
ತಿರುಗಿ-ಸುತ್ತಾರೆ
ತಿರುಗಿ-ಸು-ವನು
ತಿರುಗಿ-ಸು-ವರು
ತಿರು-ಗುವ
ತಿರುಚಿ
ತಿರುಚು-ವುದು
ತಿರುಪೆ
ತಿರೆ
ತಿರೆಯ
ತಿರೋ-ಭಾವ-ಗೊಳ್ಳು-ವುದು
ತಿಲತರ್ಪಣ
ತಿಳಿ
ತಿಳಿ-ಗೇಡಿ-ತನ-ವೆಂದು
ತಿಳಿದ
ತಿಳಿ-ದಂತೆ
ತಿಳಿ-ದ-ಕೂಡಲೇ
ತಿಳಿ-ದರೆ
ತಿಳಿ-ದ-ವ-ನಂತೆ
ತಿಳಿ-ದ-ವನು
ತಿಳಿ-ದಿತ್ತು
ತಿಳಿ-ದಿದೆ
ತಿಳಿ-ದಿದ್ದ
ತಿಳಿ-ದಿ-ದ್ದೇವೆ
ತಿಳಿ-ದಿ-ರಲಿ
ತಿಳಿ-ದಿ-ರ-ಲಿಲ್ಲ
ತಿಳಿ-ದಿರು-ತ್ತೀರಿ
ತಿಳಿ-ದಿ-ರುವ
ತಿಳಿ-ದಿ-ರು-ವಂತೆ
ತಿಳಿ-ದಿ-ರುವ-ವನು
ತಿಳಿ-ದಿರು-ವು-ದನ್ನು
ತಿಳಿ-ದಿರು-ವುದು
ತಿಳಿ-ದಿರು-ವು-ದೆಂದು
ತಿಳಿ-ದಿರು-ವೆವೋ
ತಿಳಿ-ದಿಲ್ಲ
ತಿಳಿ-ದಿ-ಲ್ಲವೆ
ತಿಳಿದು
ತಿಳಿ-ದು-ಕೊಂಡರು
ತಿಳಿ-ದು-ಕೊಂಡರೆ
ತಿಳಿ-ದು-ಕೊಂಡಿ-ದ್ದರು
ತಿಳಿ-ದು-ಕೊಂಡಿ-ದ್ದಾರೆ
ತಿಳಿ-ದು-ಕೊಂಡಿ-ದ್ದೇನೆ
ತಿಳಿ-ದು-ಕೊಂಡಿ-ರ-ಬೇಕು
ತಿಳಿ-ದು-ಕೊಂಡಿ-ರುವ
ತಿಳಿ-ದು-ಕೊಂಡಿ-ರು-ವಂತಹ
ತಿಳಿ-ದು-ಕೊಂಡಿ-ರು-ವಂತೆ
ತಿಳಿ-ದು-ಕೊಂಡಿ-ರು-ವನು
ತಿಳಿ-ದು-ಕೊಂಡಿ-ರು-ವರು
ತಿಳಿ-ದು-ಕೊಂಡಿ-ರು-ವಷ್ಟು
ತಿಳಿ-ದು-ಕೊಂಡಿ-ರು-ವಿರೋ
ತಿಳಿ-ದು-ಕೊಂಡಿಲ್ಲ
ತಿಳಿ-ದು-ಕೊಂಡು
ತಿಳಿ-ದು-ಕೊಂಡೆನು
ತಿಳಿ-ದು-ಕೊಳ್ಳದೆ
ತಿಳಿ-ದು-ಕೊಳ್ಳ-ಬಲ್ಲ
ತಿಳಿ-ದು-ಕೊಳ್ಳ-ಬಲ್ಲರು
ತಿಳಿ-ದು-ಕೊಳ್ಳ-ಬಲ್ಲರೋ
ತಿಳಿ-ದು-ಕೊಳ್ಳ-ಬಲ್ಲ-ವರು
ತಿಳಿ-ದು-ಕೊಳ್ಳ-ಬಲ್ಲಿರಿ
ತಿಳಿ-ದು-ಕೊಳ್ಳ-ಬಹುದು
ತಿಳಿ-ದು-ಕೊಳ್ಳ-ಬೇಕಾಗಿದೆ
ತಿಳಿ-ದು-ಕೊಳ್ಳ-ಬೇಕಾಗಿಯೂ
ತಿಳಿ-ದು-ಕೊಳ್ಳ-ಬೇಕಾಗಿ-ರುವ
ತಿಳಿ-ದು-ಕೊಳ್ಳ-ಬೇ-ಕಾದ
ತಿಳಿ-ದು-ಕೊಳ್ಳ-ಬೇ-ಕಾದರೆ
ತಿಳಿ-ದು-ಕೊಳ್ಳ-ಬೇಕು
ತಿಳಿ-ದು-ಕೊಳ್ಳ-ಬೇಕೆಂದು
ತಿಳಿ-ದು-ಕೊಳ್ಳ-ಬೇಕೆಂಬ
ತಿಳಿ-ದು-ಕೊಳ್ಳ-ಬೇಕೋ
ತಿಳಿ-ದು-ಕೊಳ್ಳಲಾರ
ತಿಳಿ-ದು-ಕೊಳ್ಳಲಾರದ
ತಿಳಿ-ದು-ಕೊಳ್ಳಲಾರರು
ತಿಳಿ-ದು-ಕೊಳ್ಳ-ಲಾರೆವು
ತಿಳಿ-ದು-ಕೊಳ್ಳ-ಲಾರೆವೋ
ತಿಳಿ-ದು-ಕೊಳ್ಳಲಿ
ತಿಳಿ-ದು-ಕೊಳ್ಳಲು
ತಿಳಿ-ದು-ಕೊಳ್ಳಲೆತ್ನಿ-ಸಲು
ತಿಳಿ-ದು-ಕೊಳ್ಳ-ವರು
ತಿಳಿ-ದು-ಕೊಳ್ಳಿ
ತಿಳಿ-ದು-ಕೊಳ್ಳು-ವನು
ತಿಳಿ-ದು-ಕೊಳ್ಳು-ವಷ್ಟು
ತಿಳಿ-ದು-ಕೊಳ್ಳು-ವು-ದಕ್ಕೆ
ತಿಳಿ-ದು-ಕೊಳ್ಳು-ವು-ದ-ರಿಂದ
ತಿಳಿ-ದು-ಕೊಳ್ಳು-ವುದು
ತಿಳಿ-ದು-ಕೊಳ್ಳು-ವೆವು
ತಿಳಿ-ದು-ಕೊಳ್ಳೋಣ
ತಿಳಿ-ದುದೇ
ತಿಳಿ-ದು-ಬಂದಿದೆ
ತಿಳಿ-ದು-ಬರು-ತ್ತದೆ
ತಿಳಿದೊ
ತಿಳಿದೋ
ತಿಳಿ-ನೀ-ರಿನ
ತಿಳಿ-ಯದ
ತಿಳಿ-ಯ-ದಂತೆ
ತಿಳಿ-ಯದ-ವರು
ತಿಳಿ-ಯದಿ-ರಲಿ
ತಿಳಿ-ಯದು
ತಿಳಿ-ಯದೆ
ತಿಳಿ-ಯ-ದೆಯೊ
ತಿಳಿ-ಯ-ದೆಯೋ
ತಿಳಿ-ಯ-ಬಲ್ಲೆ
ತಿಳಿ-ಯ-ಬಹುದು
ತಿಳಿ-ಯ-ಬೇಕಾಗಿ-ರುವ
ತಿಳಿ-ಯ-ಬೇಕಾಗಿಲ್ಲ
ತಿಳಿ-ಯ-ಬೇ-ಕಾದ
ತಿಳಿ-ಯ-ಬೇ-ಕಾದರೆ
ತಿಳಿ-ಯ-ಬೇಕು
ತಿಳಿ-ಯ-ಬೇಡಿ
ತಿಳಿ-ಯ-ಲಾ-ರದೆ
ತಿಳಿ-ಯ-ಲಾರೆವು
ತಿಳಿ-ಯಲಿ
ತಿಳಿ-ಯ-ಲಿಲ್ಲ
ತಿಳಿ-ಯಲು
ತಿಳಿ-ಯಲ್ಪಟ್ಟರೆ
ತಿಳಿ-ಯಲ್ಪಡ
ತಿಳಿ-ಯಲ್ಪಡು-ತ್ತದೆ
ತಿಳಿ-ಯಿತು
ತಿಳಿ-ಯಿರಿ
ತಿಳಿ-ಯು-ತ್ತದೆ
ತಿಳಿ-ಯುತ್ತಾನೆ
ತಿಳಿ-ಯುತ್ತಾರೆ
ತಿಳಿ-ಯುತ್ತಿತ್ತು
ತಿಳಿ-ಯುತ್ತಿ-ದ್ದುದು
ತಿಳಿ-ಯುತ್ತಿ-ರು-ವರು
ತಿಳಿ-ಯು-ತ್ತೀರಿ
ತಿಳಿ-ಯು-ತ್ತೇನೆ
ತಿಳಿ-ಯುತ್ತೇವೆ
ತಿಳಿ-ಯುದ
ತಿಳಿ-ಯುವ
ತಿಳಿ-ಯು-ವನು
ತಿಳಿ-ಯು-ವರು
ತಿಳಿ-ಯುವ-ವನ-ನ್ನೆ
ತಿಳಿ-ಯುವ-ವನ-ನ್ನೇ
ತಿಳಿ-ಯುವ-ವನಾ-ಗು-ವು-ದಿಲ್ಲ
ತಿಳಿ-ಯುವ-ವನು
ತಿಳಿ-ಯುವ-ವನೇ
ತಿಳಿ-ಯುವಿರಾ
ತಿಳಿ-ಯು-ವು-ದಕ್ಕೆ
ತಿಳಿ-ಯು-ವು-ದ-ರಿಂದ
ತಿಳಿ-ಯುವು-ದಾ-ಗಿದೆ
ತಿಳಿ-ಯು-ವು-ದಿಲ್ಲ
ತಿಳಿ-ಯು-ವು-ದಿಲ್ಲವೋ
ತಿಳಿ-ಯು-ವುದು
ತಿಳಿ-ಯುವುದೆಂತು
ತಿಳಿ-ಯು-ವುದೇ
ತಿಳಿ-ವ-ಳಿಕೆ
ತಿಳಿ-ವ-ಳಿಕೆ-ಯಿಲ್ಲದ
ತಿಳಿ-ವಿನ
ತಿಳಿ-ವಿಲ್ಲ
ತಿಳಿ-ಸ-ಬಲ್ಲರು
ತಿಳಿ-ಸ-ಬೇಕಾಗಿದೆ
ತಿಳಿಸ-ಬೇಕೆಂದಿ-ರುವ
ತಿಳಿ-ಸಲಿ-ಚ್ಛಿಸು-ತ್ತೇನೆ
ತಿಳಿ-ಸಲು
ತಿಳಿಸಿ
ತಿಳಿ-ಸಿ-ದ್ದೀರಿ
ತಿಳಿ-ಸಿ-ದ್ದೇನೆ
ತಿಳಿ-ಸಿ-ರು-ತ್ತೇನೆ
ತಿಳಿ-ಸಿ-ರು-ವುದು
ತಿಳಿ-ಸುತ್ತದೆ
ತಿಳಿ-ಸು-ತ್ತೇನೆ
ತಿಳಿ-ಸೋಣ
ತಿಳು-ವ-ಳಿಕೆಗೆ
ತಿಳು-ವ-ಳಿಕೆಯ
ತಿಳು-ವ-ಳಿಕೆ-ಯನ್ನು
ತಿಳು-ವ-ಳಿಕೆ-ಯಿಂದ
ತಿವಿಯ-ಲಾ-ರದು
ತಿಷ್ಠಂತಂ
ತಿಷ್ಠತಿ
ತೀಕ್ಷ್ಣ
ತೀಕ್ಷ್ಣ-ಬುದ್ಧಿ-ಶಕ್ತಿಯ
ತೀಕ್ಷ್ಣ-ಮತಿ-ಯೆಂದೂ
ತೀರದ
ತೀರ-ದಲ್ಲಿ
ತೀರ-ದಲ್ಲಿದ್ದು
ತೀರ-ಬೇಕು
ತೀರು-ತ್ತದೆ
ತೀರುತ್ತಾನೆ
ತೀರುತ್ತೇವೆ
ತೀರ್ಣಾಃ
ತೀರ್ಥ
ತೀರ್ಥ-ಕ್ಷೇತ್ರ-ದಲ್ಲಿ
ತೀರ್ಥ-ಕ್ಷೇತ್ರ-ವಾಗಿದೆ
ತೀರ್ಥ-ಯಾ-ತ್ರೆ-ಯನ್ನು
ತೀರ್ಥ-ವಾಗು-ತ್ತದೆ
ತೀರ್ಥವು
ತೀರ್ಥ-ವೆಂದರೆ
ತೀರ್ಪನ್ನು
ತೀರ್ಪಿಗೆ
ತೀರ್ಮಾ-ನಿಸಿ-ದಂತೆ
ತೀವ್ರ
ತೀವ್ರ-ತೆ-ಯನ್ನು
ತೀವ್ರ-ವಾಗಿ
ತೀವ್ರ-ವಾಗಿ-ರುತ್ತಿತ್ತು
ತೀವ್ರ-ವಾಗುತ್ತಾ
ತೀವ್ರ-ವಾದಾಗ
ತು
ತುಂಟ
ತುಂಬ
ತುಂಬಲು
ತುಂಬ-ಲೇ-ಬೇಕು
ತುಂಬಾ
ತುಂಬಿ
ತುಂಬಿ-ಕೊಂಡಿವೆ
ತುಂಬಿ-ತ್ತು
ತುಂಬಿದ
ತುಂಬಿ-ದರೆ
ತುಂಬಿ-ದಿರಿ
ತುಂಬಿದೆ
ತುಂಬಿ-ಬಂದಾಗ
ತುಂಬಿ-ರುವ
ತುಂಬಿವೆ
ತುಂಬಿ-ಹೋ-ಗಿದೆ
ತುಂಬು
ತುಂಬುವ
ತುಂಬು-ವಿರಿ
ತುಂಬು-ವು-ದ-ಕ್ಕಿಂತ
ತುಂಬು-ವುದು
ತುಚ್ಛ-ವಾಗಿ
ತುತ್ತ-ತುದಿ
ತುತ್ತನ್ನೂ
ತುತ್ತಾಗಿ-ದ್ದರೂ
ತುತ್ತಾಗಿ-ರ-ಬಾ-ರದು
ತುತ್ತಾಗು-ವೆ-ನೆಂದು
ತುತ್ತಾದ
ತುತ್ತಾದ-ವ-ರಿಗೆ
ತುದಿ-ಮೊದ-ಲಿಲ್ಲದ
ತುದಿ-ಯಲ್ಲಿ
ತುದಿ-ಯಾ-ದರೆ
ತುದಿಯೋ
ತುಮುಲ-ದಿಂದ
ತುಮುಲ-ವನ್ನು
ತುರಿಕೆ-ಯಾಗಲು
ತುರುಕುತ್ತಿದ್ಧೇವೆ
ತುರ್ಕಿ
ತುಲನಾ-ತ್ಮಕ
ತುಲ-ನೆ-ಮಾಡಿ
ತುಲಸೀ
ತುಲಾನಾ-ತ್ಮಕ-ವಾಗಿ
ತುಳಿ-ತಕ್ಕೆ
ತುಳಿತಕ್ಕೊಳಗಾದ
ತುಳಿದು
ತುಳಿಯ-ಬೇಕು
ತುಳಿ-ಯಲು
ತುಳಿ-ಯುತ್ತಾ
ತುಳುಕಾಡಿ
ತುಳು-ಕಾಡು-ತ್ತಿರು-ವುದು
ತುಳು-ಕಾ-ಡುವ
ತುಳು-ಕಾಡು-ವಂತೆ
ತುಳುಕು-ತ್ತದೆ
ತುಳುಕು-ತ್ತಿದೆ
ತುಳುಕು-ತ್ತಿ-ದ್ದುವು
ತುಷಾರಭೂಷಣ
ತುಷ್ಟಿ
ತೂಗು-ವಾಗ
ತೂಬನ್ನು
ತೂಬು-ಗಳು
ತೂರಿ
ತೂರಿ-ಹೋ-ಗ-ಲಾರದೋ
ತೂರ್ಯ-ವಾಣಿ
ತೂರ್ಯ-ವಾ-ಣಿ-ಯೊಂದು
ತೃಣಪ್ರಾಯ
ತೃಪ್ತ
ತೃಪ್ತ-ನಾ-ಗದ-ವನು
ತೃಪ್ತ-ನಾ-ಗ-ಲಿಲ್ಲ
ತೃಪ್ತ-ನಾಗು-ತ್ತೇನೆ
ತೃಪ್ತ-ರಾಗ-ಲಿಲ್ಲ
ತೃಪ್ತ-ರಾಗಿ
ತೃಪ್ತ-ರಾ-ಗುವ-ವ-ರೆಗೆ
ತೃಪ್ತ-ರಾಗು-ವುದು
ತೃಪ್ತಿ
ತೃಪ್ತಿ-ಕರ-ವಾಗಿದೆ
ತೃಪ್ತಿ-ಗಾಗಿ
ತೃಪ್ತಿ-ದಾಯಕ
ತೃಪ್ತಿ-ಪಟ್ಟು-ಕೊಂಡಿ-ದ್ದರೂ
ತೃಪ್ತಿ-ಪಡಿ-ಸಲು
ತೃಪ್ತಿ-ಪಡಿ-ಸಿ-ಕೊಳ್ಳ-ಬಯ-ಸಿದರೆ
ತೃಪ್ತಿ-ಪಡು-ತ್ತಿದ್ದನು
ತೃಪ್ತಿ-ಯನ್ನೀ-ಯ-ಲಿಲ್ಲ
ತೃಪ್ತಿ-ಯನ್ನು
ತೃಪ್ತಿ-ಯಾಗಿಲ್ಲ
ತೃಪ್ತಿ-ಯಾ-ದರೂ
ತೃಪ್ತಿಯೇ
ತೃಷ್ಣೆ-ಗಳ
ತೃಷ್ಣೆ-ಗಳನ್ನು
ತೆಗಳಲಿ
ತೆಗೆ-ದ-ನೆಂದರೆ
ತೆಗೆದು
ತೆಗೆ-ದು-ಕೊಂಡಂತೆ
ತೆಗೆ-ದು-ಕೊಂಡರು
ತೆಗೆ-ದು-ಕೊಂಡರೂ
ತೆಗೆ-ದು-ಕೊಂಡರೆ
ತೆಗೆ-ದು-ಕೊಂಡಿ-ರ-ಬೇಕು
ತೆಗೆ-ದು-ಕೊಂಡಿ-ರಲಿ-ಕ್ಕಿಲ್ಲ
ತೆಗೆ-ದು-ಕೊಂಡಿವೆ
ತೆಗೆ-ದು-ಕೊಂಡು
ತೆಗೆ-ದು-ಕೊಂಡೆವು
ತೆಗೆ-ದು-ಕೊಳ್ಳ-ಬಹುದು
ತೆಗೆ-ದು-ಕೊಳ್ಳ-ಬೇಡಿ
ತೆಗೆ-ದು-ಕೊಳ್ಳಲಿ
ತೆಗೆ-ದು-ಕೊಳ್ಳಿ
ತೆಗೆ-ದು-ಕೊಳ್ಳು-ತ್ತೇನೆ
ತೆಗೆ-ದು-ಕೊಳ್ಳು-ತ್ತೇವೆ
ತೆಗೆ-ದು-ಕೊಳ್ಳುವ
ತೆಗೆ-ದು-ಕೊಳ್ಳು-ವು-ದನ್ನು
ತೆಗೆ-ದು-ಕೊಳ್ಳು-ವು-ದ-ರ-ಲ್ಲಿದೆ
ತೆಗೆ-ದು-ಕೊಳ್ಳು-ವು-ದಿಲ್ಲ
ತೆಗೆ-ದು-ಕೊಳ್ಳು-ವುದು
ತೆಗೆ-ದು-ಕೊಳ್ಳೋಣ
ತೆಗೆ-ದು-ಬಿಟ್ಟರೆ
ತೆಗೆ-ದು-ಹಾಕಿ
ತೆಗೆ-ದೊ-ಡ-ನೆಯೇ
ತೆಗೆ-ಯು-ತ್ತಿದ್ದರು
ತೆಗೆ-ಯು-ವು-ದ-ರಿಂದ
ತೆತ್ತಾ-ದರೂ
ತೆರ-ನಾದ
ತೆರುತ್ತಿ-ರುವ
ತೆರೆ
ತೆರೆ-ಗಳಂತೆ
ತೆರೆದ
ತೆರೆ-ದರು
ತೆರೆ-ದಿಟ್ಟಿದ್ದು
ತೆರೆ-ದಿದೆ
ತೆರೆ-ದಿ-ದ್ದರೆ
ತೆರೆ-ದಿ-ದ್ದೀರಿ
ತೆರೆ-ದಿ-ರಲಿ
ತೆರೆ-ದಿರು-ವುದು
ತೆರೆದು
ತೆರೆಯ
ತೆರೆ-ಯದೆ
ತೆರೆ-ಯನ್ನು
ತೆರೆ-ಯ-ಬೇಕು
ತೆರೆ-ಯ-ಬೇಕೆಂಬುದೇ
ತೆರೆ-ಯ-ಲಿಲ್ಲ
ತೆರೆ-ಯಲ್ಲಿ
ತೆರೆ-ಯಷ್ಟೇ
ತೆರೆ-ಯಿಂದ
ತೆರೆ-ಯಿತು
ತೆರೆ-ಯಿರಿ
ತೆರೆಯು
ತೆರೆ-ಯು-ತ್ತಿ-ರು-ವರು
ತೆರೆ-ಯು-ವಂತೆ
ತೆರೆ-ಯು-ವುದು
ತೆರೆ-ಸಿ-ದ್ದೀರಿ
ತೆಳು
ತೆಳು-ವಾಗುತ್ತಾ
ತೆಳ್ಳ-ಗಿದೆ-ಯೆಂದರೆ
ತೆವಳಿ-ಕೊಂಡು
ತೆವಳುತ್ತಾ
ತೆವಳು-ತ್ತಿ-ರುವ
ತೇ
ತೇಜಸ್ವಿ
ತೇಜಸ್ವಿ-ಗಳು
ತೇಜಸ್ವಿ-ನಾವಧೀ-ತಮ-ಸ್ತು
ತೇಜಸ್ವಿ-ಯಾದ
ತೇಜಸ್ಸು
ತೇಜೋಂಶ-ಸಂಭವ-ಮ್
ತೇಜೋ-ಮ-ಯವೂ
ತೇದಿ
ತೇದಿಗೆ
ತೇಲಿ
ತೇಲು-ತ್ತಿದ್ದರೆ
ತೈರ್ಜಿತಃ
ತೊಂದರೆ
ತೊಂದರೆ-ಗಳ-ನ್ನೆಲ್ಲಾ
ತೊಂದರೆಯೂ
ತೊಂಬತ್ತರಷ್ಟು
ತೊಂಬತ್ತು
ತೊಂಬತ್ತೊಂಬತ್ತು
ತೊಂಭತ್ತೆಂಟು
ತೊಟ್ಟಿ-ಲಲ್ಲಿ
ತೊಟ್ಟು-ಕೊಂಡ
ತೊಡಕಿದೆ
ತೊಡಕು-ಗಳು
ತೊಡಗಿ-ದ್ದಾರೆ
ತೊಡ-ಗು-ತ್ತದೆ
ತೊಡಿ-ಸಿದ
ತೊಡುಗೆ-ಗಳು
ತೊಡೆದು
ತೊಯ್ದಿರು-ವುದು
ತೊರೆ-ದನು
ತೊರೆದಿ-ರು-ವರು
ತೊರೆದಿ-ರು-ವರೋ
ತೊರೆದು
ತೊರೆ-ಯ-ಲಾರೆವು
ತೊರೆ-ಯಲು
ತೊರೆ-ಯಿರಿ
ತೊರೆ-ಯು-ವನು
ತೊರೆಯು-ವುದು
ತೊಲಗ-ಬೇಕು
ತೊಲಗು
ತೊಳಲುತ್ತಾ
ತೊಳಲುವ
ತೊಳೆಯ-ಬಲ್ಲೆಯಾ
ತೊಳೆಯ-ಬೇಕೆ
ತೋಟ
ತೋಟದ
ತೋಟ-ವಿತ್ತು
ತೋಡುವ-ವನು
ತೋಯಿ-ಸದು
ತೋಯಿಸ-ಬೇಕಾ-ಯಿತು
ತೋಯಿಸಿ-ರು-ವರು
ತೋರದೆ
ತೋರ-ಬಲ್ಲರು
ತೋರ-ಬಲ್ಲರೋ
ತೋರ-ಬಲ್ಲಿರಾ
ತೋರ-ಬಲ್ಲೆ
ತೋರ-ಬಹುದು
ತೋರ-ಬಾ-ರದು
ತೋರ-ಬೇಕು
ತೋರ-ಬೇಕೆಂದು
ತೋರಬೇಕೆಂಬುದು
ತೋರಲು
ತೋರಲೆ
ತೋರಿ
ತೋರಿ-ಇದೇ
ತೋರಿ-ಕೆಗೆ
ತೋರಿ-ಕೆಯ
ತೋರಿತು
ತೋರಿದ
ತೋರಿ-ದಂತೆ
ತೋರಿ-ದರು
ತೋರಿ-ದರೂ
ತೋರಿ-ದರೆ
ತೋರಿ-ದು-ದ-ಕ್ಕಿಂತಲೂ
ತೋರಿ-ದು-ದನ್ನು
ತೋರಿ-ದುದು
ತೋರಿ-ದ್ದರೆ
ತೋರಿ-ರು-ವರು
ತೋರಿ-ರು-ವೆನು
ತೋರಿ-ಸದ
ತೋರಿ-ಸ-ಬೇ-ಕಾದರೆ
ತೋರಿ-ಸ-ಬೇಕು
ತೋರಿ-ಸಲಿ
ತೋರಿ-ಸಲು
ತೋರಿಸಿ
ತೋರಿ-ಸಿ-ಕೊಟ್ಟಿ-ದ್ದೀರಿ
ತೋರಿ-ಸಿ-ಕೊಟ್ಟಿರಿ
ತೋರಿ-ಸಿ-ಕೊಡ-ಬೇಕೆಂಬ
ತೋರಿ-ಸಿ-ಕೊಡು-ತ್ತದೆ
ತೋರಿ-ಸಿ-ದನು
ತೋರಿ-ಸಿ-ದರು
ತೋರಿ-ಸಿ-ದರೆ
ತೋರಿ-ಸಿದು
ತೋರಿ-ಸಿದೆ
ತೋರಿ-ಸಿದ್ದ
ತೋರಿ-ಸಿ-ದ್ದೀರಿ
ತೋರಿ-ಸಿ-ರುವ
ತೋರಿ-ಸಿ-ರು-ವನು
ತೋರಿ-ಸುತ್ತದೆ
ತೋರಿ-ಸುತ್ತಾ
ತೋರಿ-ಸು-ತ್ತಾರೆ
ತೋರಿ-ಸು-ತ್ತಿದೆ
ತೋರಿ-ಸು-ತ್ತಿದ್ದ
ತೋರಿ-ಸು-ತ್ತಿದ್ದೆ
ತೋರಿ-ಸು-ತ್ತೀರಿ
ತೋರಿ-ಸುವ
ತೋರಿ-ಸು-ವಂತೆ
ತೋರಿ-ಸು-ವರು
ತೋರಿ-ಸು-ವು-ದ-ರಿಂದಲೂ
ತೋರಿ-ಸು-ವುದು
ತೋರು-ತ್ತದೆ
ತೋರು-ತ್ತದೆಯೇ
ತೋರು-ತ್ತವೆ
ತೋರುತ್ತಾ
ತೋರು-ತ್ತಿದೆ
ತೋರುತ್ತಿ-ರುವ
ತೋರು-ತ್ತಿಲ್ಲ
ತೋರು-ತ್ತಿವೆ
ತೋರುವ
ತೋರು-ವಂತೆ
ತೋರು-ವರು
ತೋರುವು-ದ-ಕ್ಕಾಗಿ
ತೋರು-ವು-ದಕ್ಕೆ
ತೋರುವು-ದ-ರ-ಲ್ಲಿಯೂ
ತೋರು-ವು-ದಿಲ್ಲ
ತೋರು-ವು-ದಿಲ್ಲವೋ
ತೋರು-ವುದು
ತೋರು-ವುದು-ಅಲೆ-ಗಳಂತೆ
ತೋರು-ವುದು-ಯಾವ
ತೋರು-ವುದೇ
ತೋರು-ವುದೋ
ತೋರು-ವುವು
ತೋರ್ಧುತ್ತ್ಧದೆ
ತೋಲನ-ವಿದೆ
ತೌರು-ಮ-ನೆ-ಯಾಗ-ಬೇಕಾಗಿ-ತ್ತು
ತ್ಕಾರ್ಯ-ಗಳನ್ನು
ತ್ತದೆ
ತ್ತಮ
ತ್ತೀರಿ
ತ್ತೇನೆ
ತ್ನಿಸು-ವರು
ತ್ಯಜಿಸ-ಬಲ್ಲಿರಾ
ತ್ಯಜಿಸ-ಬಹುದು
ತ್ಯಜಿಸ-ಬೇಕು
ತ್ಯಜಿಸ-ಬೇಕೆಂದು
ತ್ಯಜಿಸ-ಬೇಡಿ
ತ್ಯಜಿಸ-ಲಾರೆ
ತ್ಯಜಿ-ಸಲಿ
ತ್ಯಜಿಸ-ಲ್ಪಟ್ಟಿವೆ
ತ್ಯಜಿಸಿ
ತ್ಯಜಿಸಿದ
ತ್ಯಜಿಸಿ-ದನು
ತ್ಯಜಿಸಿ-ದರು
ತ್ಯಜಿಸಿ-ದರೆ
ತ್ಯಜಿಸಿ-ಬಿಡಿ
ತ್ಯಜಿಸಿ-ಯೇ-ಬಿಟ್ಟರು
ತ್ಯಜಿಸುತ್ತಾನೆ
ತ್ಯಜಿ-ಸು-ವು-ದಕ್ಕೆ
ತ್ಯಜಿಸು-ವುದು
ತ್ಯಾಗ
ತ್ಯಾಗಕ್ಕೆ
ತ್ಯಾಗ-ಗಳಿ-ಲ್ಲದೆ
ತ್ಯಾಗ-ಜೀವ-ನದ
ತ್ಯಾಗದ
ತ್ಯಾಗ-ದಿಂದ
ತ್ಯಾಗ-ಭಾ-ವನೆ
ತ್ಯಾಗ-ಭೂಮಿ
ತ್ಯಾಗ-ಭೂಮಿ-ಯಾದ
ತ್ಯಾಗ-ಮಾಡ-ಬಲ್ಲ
ತ್ಯಾಗ-ಮಾಡ-ಬಲ್ಲಿರಿ
ತ್ಯಾಗ-ಮಾಡಿ-ದರೆ
ತ್ಯಾಗ-ಮಾಡಿ-ದುದ-ರಿಂದಲೇ
ತ್ಯಾಗ-ಮಾಡು
ತ್ಯಾಗ-ಮಾಡು-ವುದು
ತ್ಯಾಗ-ವನ್ನು
ತ್ಯಾಗ-ವಿ-ಲ್ಲದೆ
ತ್ಯಾಗವು
ತ್ಯಾಗವೂ
ತ್ಯಾಗವೇ
ತ್ಯಾಗ-ವೊಂದೇ
ತ್ಯಾಗ-ಶಕ್ತಿ
ತ್ಯಾಗ-ಶೀಲ-ವಾಗಿ-ತ್ತು
ತ್ಯಾಗಿ-ಗಳಾಗಿ-ದ್ದರು
ತ್ಯಾಗಿ-ಗಳಾರು
ತ್ಯಾಗಿ-ಯಾದ
ತ್ಯಾಗೇನೈಕೇ
ತ್ರಯಮೇ-ವೈ-ತ-ತ್
ತ್ರಯೀ
ತ್ರಾಯತೇ
ತ್ರಿಕಾಲ
ತ್ರಿಪಿ-ಟಕ-ಗಳಿಗೂ
ತ್ರಿಪಿ-ಟಕ-ಗಳು
ತ್ರಿಪಿ-ಟಿಕ-ಗಳು
ತ್ರಿಭು-ಜಕ್ಕೆ
ತ್ರಿವರ್ಣದ-ವ-ರಿಗೂ
ತ್ರಿವಿ-ಕ್ರಮ
ತ್ರಿಷ್ಟುಪ್
ತ್ರೇತಾ-ಯುಗಕ್ಕೆ
ತ್ವಂ
ತ್ವಯಿ
ಥರ್ಮಾ-ಮೀಟರ್
ಥಿಯಾಸ
ಥಿಯಾ-ಸಫಿ
ಥಿಯಾ-ಸ-ಫಿ-ಕ-ಲ್
ಥಿಯಾ-ಸ-ಫಿಯ
ಥಿಯಾ-ಸ-ಫಿ-ಸ್ಟರ
ಥಿಯಾ-ಸ-ಫಿ-ಸ್ಟ-ರನ್ನು
ಥಿಯಾ-ಸ-ಫಿ-ಸ್ಟ-ರಿಗೆ
ಥಿಯಾ-ಸ-ಫಿ-ಸ್ಟರು
ಥಿಯೇಟರಿ-ನಲ್ಲಿ
ದ
ದಂಡ
ದಂಡ-ದಿಂದ
ದಂಡೆತ್ತಿ
ದಂತಿದೆ
ದಂತೆ
ದಂದ್ರಮ್ಯ-ಮಾಣಾಃ
ದಕ್ಕಾಗಿ
ದಕ್ಷತೆ
ದಕ್ಷಿಣ
ದಕ್ಷಿ-ಣಕ್ಕೆ
ದಕ್ಷಿಣ-ಗಳೆ-ನ್ನದೆ
ದಕ್ಷಿಣ-ದಲ್ಲಿ
ದಕ್ಷಿಣ-ದ-ವ-ರೆಗೆ
ದಕ್ಷಿಣ-ದೇಶಕ್ಕೆ
ದಕ್ಷಿಣ-ದೇಶ-ದಲ್ಲಿ-ರುವ
ದಕ್ಷಿಣೇಶ್ವರ
ದಡವು
ದಡ್ಡಿರಿ-ಗಾಗಿ
ದಣಿವು
ದನ
ದನ-ಗಳ
ದನ-ಗಳು
ದನದ
ದನ-ಲ್ಲದೆ
ದನೆಂದಾ-ಗಲಿ
ದನ್ನೂ
ದನ್ನೆಲ್ಲಾ
ದಬ್ಬಾ-ಳಿಕೆ
ದಬ್ಬಾ-ಳಿ-ಕೆಗೆ
ದಬ್ಬಾ-ಳಿ-ಕೆ-ಯಲ್ಲಿ
ದಬ್ಬಾ-ಳಿ-ಕೆಯು
ದಮನ-ಕ್ಕಾಗಿ
ದಯಗೆ
ದಯ-ದಿಂದ
ದಯ-ಪಾಲಿಸ-ಬೇಕೆಂದು
ದಯ-ಪಾಲಿ-ಸಲಿ
ದಯ-ಪಾಲಿಸಿ-ರುವ
ದಯ-ಪಾಲಿಸು
ದಯ-ವಿಟ್ಟು
ದಯಾ
ದಯಾ-ದಾಕ್ಷಿಣ್ಯ-ವಿ-ಲ್ಲದೆ
ದಯಾ-ನಂದ
ದಯಾ-ಪೂರಿತ
ದಯಾ-ಪೂರ್ಣವೂ
ದಯಾ-ಮಯ-ನಾದ
ದಯಾ-ಮ-ಯನೂ
ದಯಾ-ಮೂರ್ತಿ-ಯಾದ
ದಯೆ
ದಯೆಗೆ
ದಯೆ-ಯನ್ನೂ
ದಯೆ-ಯಿಂದ
ದಯೆ-ಯಿಂದ-ನೋ-ಡಿರು
ದರಲ್ಲಿ
ದರಿ
ದರಿಂದ
ದರಿದ್ರ
ದರಿ-ದ್ರರ
ದರಿ-ದ್ರ-ರಾಗಿ
ದರಿ-ದ್ರ-ರಾದ
ದರಿ-ದ್ರರೂ
ದರೂ
ದರೋಡೆ
ದರೋ-ಡೆ-ಕಾರ-ರ-ನ್ನಾಗಿ-ಮಾಡು-ತ್ತದೆ
ದರೋ-ಡೆ-ಕೋರ-ರಾದರು
ದರೋ-ಡೆ-ಮಾಡು-ತ್ತಿದ್ದ
ದರ್ಜೆಗೆ
ದರ್ಜೆಯ
ದರ್ಪ-ದಿಂದ
ದರ್ಶಕ-ರಾಗ-ಬೇಕು
ದರ್ಶಕವೂ
ದರ್ಶನ
ದರ್ಶನಕ್ಕೆ
ದರ್ಶನ-ಗಳ
ದರ್ಶನ-ಗಳನ್ನು
ದರ್ಶನ-ಗಳಲ್ಲಿ
ದರ್ಶನ-ಗಳು
ದರ್ಶನ-ಗಳೆಲ್ಲ
ದರ್ಶನ-ಗಳೆ-ಲ್ಲವೂ
ದರ್ಶನ-ಗಳೆಲ್ಲಾ
ದರ್ಶನದ
ದರ್ಶನ-ದಲ್ಲಿ
ದರ್ಶನ-ದಿಂದಲೇ
ದರ್ಶನ-ಬದ್ಧ-ವಾಗುವ
ದರ್ಶನ-ವನ್ನಾ-ದರೂ
ದರ್ಶನ-ವನ್ನು
ದರ್ಶನ-ವನ್ನೂ
ದರ್ಶನ-ವಾಗಲೀ
ದರ್ಶನ-ವಾಗು-ತ್ತದೆ
ದರ್ಶನ-ವಾದ
ದರ್ಶನವು
ದರ್ಶನವೂ
ದರ್ಶನ-ವೆಂಬ
ದರ್ಶನ-ಶಾಸ್ತ್ರವೂ
ದರ್ಶನ-ಶಾಸ್ತ್ರ-ಶಿರೋ-ಮಣಿ-ಯಂತೆ
ದರ್ಶನಾಕಾಂಕ್ಷಿ-ಗಳಾಗಿ-ದ್ದೇವೆ
ದರ್ಶಿ-ಯಾದ
ದರ್ಶಿ-ಸುತ್ತಾ-ನೆಯೋ
ದಲಿತ-ರಾದ
ದಲಿತ-ರಿಗೆ
ದಲ್ಲಿ
ದಲ್ಲಿದ್ದೆ
ದಲ್ಲಿ-ರು-ವ-ವ-ರೆಲ್ಲಾ
ದಲ್ಲಿ-ರು-ವೆನು
ದಲ್ಲಿಲ್ಲ
ದಳ
ದಳ್ಳುರಿಯ
ದವರು
ದಶ-ದಿಕ್ಕು-ಗಳಿಂದಲೂ
ದಸ್ಯು-ಗಳನ್ನು
ದಸ್ಯು-ಗಳು
ದಹಿ-ಸದು
ದಹಿ-ಸ-ಲಾ-ರದು
ದಹಿ-ಸಿದರೆ
ದಹಿ-ಸು-ತ್ತಿದ್ದ
ದಹಿ-ಸುತ್ತಿ-ರುವ
ದಹಿ-ಸು-ತ್ತೇನೆ
ದಹ್ಧತಿ
ದಾಕ್ಷಿಣಾತ್ಯರು
ದಾಕ್ಷಿಣ್ಯ-ವಿ-ಲ್ಲದೆ
ದಾಖ-ಲಾ-ಗಿದೆ
ದಾಖಲೆಗೆ
ದಾಖಲೆ-ಯನ್ನು
ದಾಟ-ಬಲ್ಲ
ದಾಟಲು
ದಾಟಿ
ದಾಟಿ-ರುವನೋ
ದಾಟಿ-ಹೋ-ಗಲು
ದಾಟಿ-ಹೋ-ಗಿ-ದ್ದೀರಿ
ದಾದು
ದಾನ
ದಾನ-ಕೊ-ಡುವ
ದಾನ-ಕ್ಕಿಂತ
ದಾನಕ್ಕೆ
ದಾನ-ಗಳಿ-ಗಿಂತ
ದಾನ-ಗಳು
ದಾನ-ಗಳೆಲ್ಲಾ
ದಾನ-ದಲ್ಲಿ
ದಾನ-ಪದ್ಧತಿಗೆ
ದಾನ-ಮಾಡ-ಬೇಕು
ದಾನ-ಮಾಡುವ
ದಾನ-ಮಾಡು-ವುದು
ದಾನ-ವನ್ನು
ದಾನ-ವಲ್ಲ
ದಾನ-ವೆಲ್ಲ
ದಾನವೇ
ದಾನ-ವೊಂದೇ
ದಾನ-ಶೀಲ
ದಾನಿ
ದಾನಿ-ಗಳಲ್ಲ
ದಾಯತೇ
ದಾಯ-ದ-ವರು
ದಾರ-ದಲ್ಲಿ
ದಾರಾಶುಕೂ
ದಾರಿ
ದಾರಿ-ಗಳು
ದಾರಿ-ಗಾಣ-ದಂತೆ
ದಾರಿ-ಗಾಣದೆ
ದಾರಿಗೆ
ದಾರಿ-ತಪ್ಪಿ
ದಾರಿದ್ರ್ಯ
ದಾರಿ-ದ್ರ್ಯದ
ದಾರಿ-ದ್ರ್ಯ-ದಲ್ಲಿಯೇ
ದಾರಿ-ದ್ರ್ಯ-ವನ್ನೂ
ದಾರಿ-ದ್ರ್ಯ-ವಿದ್ದೇ
ದಾರಿ-ದ್ರ್ಯವೇ
ದಾರಿಯ
ದಾರಿ-ಯನ್ನು
ದಾರಿ-ಯಲ್ಲಿ
ದಾರಿ-ಯಲ್ಲೇ
ದಾರಿ-ಯಿಲ್ಲ
ದಾರಿ-ಯೊಂದೇ
ದಾರುಣ
ದಾರುಣ-ವಾಗಿ
ದಾರುಣ-ವಾದಾಗ
ದಾರ್ಶನಿಕ
ದಾರ್ಶನಿಕ-ರಲ್ಲಿ
ದಾರ್ಶನಿಕ-ರಾ-ದರೂ
ದಾರ್ಶನಿ-ಕರು
ದಾರ್ಶನಿಕ-ರೆಲ್ಲರೂ
ದಾರ್ಶನಿಕ-ರೆಲ್ಲಾ
ದಾಳಿ
ದಾಳಿ-ಯಿಡ-ಬೇಕು
ದಾಳಿ-ಯಿಡ-ಲಾ-ಗು-ವು-ದಿಲ್ಲ
ದಾಸರ
ದಾಸ-ರಂತೆ
ದಾಸ-ರಾಗಿ-ರುತ್ತಾರೆ
ದಾಸರೋ
ದಾಸಾನುದಾಸನಾಗ
ದಾಸಿಯ
ದಾಸ್ಯ-ಗಳ
ದಿಂದ
ದಿಂದಲೂ
ದಿಂದಲೋ
ದಿಕ್ಕಿಗೆ
ದಿಕ್ಕಿನ
ದಿಕ್ಕಿ-ನಲ್ಲಿ
ದಿಕ್ಕು
ದಿಕ್ಕು-ಗಳಿಂದ
ದಿಕ್ಕು-ಗಳಿಗೂ
ದಿಗ್ಭಿತ್ತಿ
ದಿಗ್ಭ್ರಮೆ
ದಿಗ್ಭ್ರಮೆಯೂ
ದಿಗ್ಭ್ರಾಂತಿ-ಯಾಗಿದೆ
ದಿಗ್ಭ್ರೆಮೆ
ದಿಗ್ವಜಯೀ
ದಿಗ್ವಿ-ಜಯ-ಕೋಲಾಹಲ
ದಿಗ್ವಿ-ಜ-ಯದ
ದಿಗ್ವಿ-ಜಯ-ದಿಂದಲೋ
ದಿಗ್ವಿಜಯಿ-ಗಳು
ದಿಗ್ವಿಜಯಿ-ಯಾಗಿ
ದಿಟ್ಟಿ
ದಿನ
ದಿನಂಪ್ರತಿ
ದಿನ-ಗಳ
ದಿನ-ಗಳಲ್ಲಿ
ದಿನ-ಗಳ-ವ-ರೆಗೆ
ದಿನ-ಗಳಿದ್ದು
ದಿನ-ಗಳು
ದಿನ-ಚರಿ-ಯಲ್ಲಿ
ದಿನದ
ದಿನ-ದಿಂದ
ದಿನ-ದಿಂದಲೇ
ದಿನ-ದಿನದ
ದಿನ-ದಿ-ನವೂ
ದಿನದ್ದು
ದಿನ-ವ-ಲ್ಲವೆ
ದಿನ-ವಾಗಿ
ದಿನೇ
ದಿವ್ಯ
ದಿವ್ಯ-ಜ್ಞಾನ
ದಿವ್ಯ-ಜ್ಯೋತಿ
ದಿವ್ಯ-ತೆಯ
ದಿವ್ಯ-ತೆ-ಯನ್ನು
ದಿವ್ಯ-ದರ್ಶನ-ಗಳಲ್ಲಿ
ದಿವ್ಯ-ದರ್ಶನ-ಶಾಸ್ತ್ರ-ಗಳು
ದಿವ್ಯ-ದಿನ
ದಿವ್ಯ-ದಿ-ನಕ್ಕೆ
ದಿವ್ಯ-ದೃಷ್ಟಿ
ದಿವ್ಯ-ವಾ-ಣಿ-ಯ-ಲ್ಲಿಯೇ
ದಿವ್ಯ-ವಾದ
ದಿವ್ಯ-ಶಕ್ತಿ-ಯನ್ನೂ
ದಿವ್ಯ-ಸ್ಪರ್ಶ-ವನ್ನೂ
ದೀಕ್ಷಾ-ದಿನ
ದೀಕ್ಷೆಯನ್ನಿತ್ತ
ದೀನ
ದೀನ-ದರಿದ್ರ
ದೀನ-ದಲಿ-ತರ
ದೀನರ
ದೀನ-ರಕ್ಷಕ
ದೀನ-ರನ್ನು
ದೀನ-ರಲ್ಲಿ
ದೀನ-ರಾ-ದರೂ
ದೀನ-ರಿಗೆ
ದೀನರು
ದೀನರೇ
ದೀಪ-ದಂತೆ
ದೀಪ-ವನ್ನು
ದೀರ್ಘ
ದೀರ್ಘ-ಕಾಲ
ದೀರ್ಘ-ಕಾಲದ
ದೀರ್ಘಾಯುಷ್ಯ-ವನ್ನು
ದೀರ್ಘಾಯುಷ್ಯ-ವನ್ನೂ
ದುಃಖ
ದುಃಖಕ್ಕೆ
ದುಃಖ-ಗಳ
ದುಃಖ-ಗಳನ್ನು
ದುಃಖ-ಗಳೆಂಬ
ದುಃಖದ
ದುಃಖ-ದಲ್ಲಿ
ದುಃಖ-ದಲ್ಲಿ-ರು-ವಾಗ
ದುಃಖ-ದಿಂದ
ದುಃಖ-ಪ-ಡುವನು
ದುಃಖ-ಮಯ
ದುಃಖ-ಮಯ-ವಾದ
ದುಃಖ-ವನ್ನು
ದುಃಖ-ವನ್ನೂ
ದುಃಖ-ವಾಗು-ತ್ತದೆ
ದುಃಖ-ವಾಗು-ವುದು
ದುಃಖ-ವಿಲ್ಲದ
ದುಃಖ-ವಿ-ಲ್ಲದೆ
ದುಃಖವೂ
ದುಃಖ-ವೆಲ್ಲ
ದುಃಖ-ವೊಂದಿಲ್ಲದ
ದುಃಖಾ-ತೀತ
ದುಃಖಿ
ದುಃಖಿ-ಗಳ
ದುಃಖಿ-ಗಳಿಗೆ
ದುಃಖಿ-ಗಳು
ದುಃಖಿ-ಯಾಗಿ-ದ್ದರೆ
ದುಃಸ್ಥಿತಿ-ಗಿಳಿದು
ದುಃಸ್ಥಿತಿಯು
ದುಡಿತ-ದಿಂದ
ದುಡಿ-ದಾಗ
ದುಡಿದು
ದುಡಿಯ-ಬಲ್ಲ
ದುಡಿ-ಯಲು
ದುಡಿಯುತ್ತಿ-ರು-ವರು
ದುಡಿ-ಯುವ
ದುಡಿ-ಯು-ವಂತೆ
ದುಡಿಯು-ವುದು
ದುಡಿ-ಯೋಣ
ದುಡುಕ-ಬೇಡಿ
ದುಡ್ಡಿ-ಲ್ಲದ
ದುಡ್ಡು
ದುಮುದುಮುಕಿ
ದುರಂತಕ್ಕೆ
ದುರಂತ-ದಿಂದ
ದುರದೃಷ್ಟ
ದುರದೃಷ್ಟ-ಕರ
ದುರದೃಷ್ಟದ
ದುರದೃಷ್ಟ-ವಶಾ-ತ್
ದುರಸ್ತಿ
ದುರಾಗ್ರಹ-ಭಾ-ವನೆ
ದುರಾ-ಚಾರ-ಗಳಿಗೇ
ದುರಾಚಾರಿ-ಗಳಿಗೆ
ದುರಾದೃಷ್ಟ-ವಶಾ-ತ್
ದುರುದ್ದೇಶಕ್ಕೆ
ದುರುದ್ದೇಶ-ದಿಂದ
ದುರ್ಗ-ಗಳ-ನ್ನೆಲ್ಲ
ದುರ್ಗತಿ
ದುರ್ಗತಿ-ಗಳಾವು-ದನ್ನೂ
ದುರ್ಗತಿ-ಗಳಿ-ಗೆಲ್ಲಾ
ದುರ್ದಮ್ಯ
ದುರ್ದಿನ
ದುರ್ದೆಶೆಗೆ
ದುರ್ದೆಶೆಯ
ದುರ್ದೈವ-ವಶಾ-ತ್
ದುರ್ನೀತಿ-ಗಳಿಗೆ
ದುರ್ನೀತಿಗೆ
ದುರ್ಬಲ
ದುರ್ಬಲ-ಕಾರಿ-ಗಳಾದ
ದುರ್ಬಲ-ಗೊಳಿಸುವ
ದುರ್ಬಲತೆ
ದುರ್ಬಲ-ತೆ-ಗಳಿಗೂ
ದುರ್ಬಲ-ತೆಯ
ದುರ್ಬಲ-ತೆ-ಯನ್ನು
ದುರ್ಬಲ-ತೆ-ಯನ್ನೂ
ದುರ್ಬಲ-ತೆ-ಯ-ನ್ನೆಲ್ಲ್ಲ
ದುರ್ಬಲ-ತೆಯೆ
ದುರ್ಬಲ-ನ-ಲ್ಲಿಯೂ
ದುರ್ಬಲ-ನಾಗ-ಬೇಡ
ದುರ್ಬಲ-ನಿಗೆ
ದುರ್ಬಲನೂ
ದುರ್ಬಲನೇ
ದುರ್ಬಲರ
ದುರ್ಬಲ-ರ-ನ್ನಾಗಿ
ದುರ್ಬಲ-ರನ್ನು
ದುರ್ಬಲ-ರಲ್ಲ
ದುರ್ಬಲ-ರಲ್ಲಿ
ದುರ್ಬಲ-ರಾಗಿ
ದುರ್ಬಲ-ರಾಗಿ-ರು-ವೆವು
ದುರ್ಬಲ-ರಾಗುತ್ತಾರೆ
ದುರ್ಬಲ-ರಾಗು-ತ್ತೀರಿ
ದುರ್ಬಲ-ರಾ-ಗುವು-ದನ್ನೂ
ದುರ್ಬಲ-ರಿಗೆ
ದುರ್ಬಲರು
ದುರ್ಬಲ-ರು-ಇದೇ
ದುರ್ಬಲ-ರು-ಬಲಾಢ್ಯರು
ದುರ್ಬಲ-ರೆಂದು
ದುರ್ಬಲ-ವಾಗದೆ
ದುರ್ಬಲ-ವಾಗಿ
ದುರ್ಬಲ-ವಾಗಿದೆ
ದುರ್ಬಲ-ವಾಗಿ-ರು-ವಾಗ
ದುರ್ಬಲ-ವಾಗಿ-ರು-ವುದೆಲ್ಲ
ದುರ್ಬಲ-ವಾಗು-ವಿರಿ
ದುರ್ಬಲ-ವಾದರೆ
ದುರ್ಬಲ-ವಾದಾಗ
ದುರ್ಭಾಗಿ-ಗಳು
ದುರ್ಭೇದ್ಯ-ವಾದ
ದುರ್ಮಾರ್ಗ-ಗಳನ್ನು
ದುರ್ಮಾರ್ಗ-ದಲ್ಲಿ
ದುರ್ಲಭ
ದುರ್ಲಭಂ
ದುರ್ಲಭರೋ
ದುಷ್ಕರ್ಮ-ಗಳು
ದುಷ್ಕು-ಲಾದಪಿ
ದುಷ್ಕೃತಾ-ಮ್
ದುಷ್ಟ
ದುಷ್ಟ-ತ-ನಕ್ಕೆ
ದುಷ್ಟ-ನ-ಲ್ಲಿಯೂ
ದುಷ್ಟ-ನಾಗಿ-ರಲಿ
ದುಷ್ಟ-ಮಕ್ಕಳನ್ನು
ದುಷ್ಟರ
ದುಷ್ಟ-ರಾಗಿ-ದ್ದೀರಿ
ದುಷ್ಟರು
ದುಷ್ಟರೂ
ದೂರ
ದೂರದ
ದೂರ-ದಲ್ಲಿ
ದೂರ-ದಲ್ಲಿ-ರು-ವನು
ದೂರ-ದಿಂದ
ದೂರದೆ
ದೂರ-ನಿಲ್ಲಿ
ದೂರನು
ದೂರನೂ
ದೂರ-ಪ್ರ-ಯಾ-ಣದ
ದೂರ-ಬೇಕು
ದೂರ-ಬೇಡಿ
ದೂರ-ಮಾಡಿ
ದೂರ-ಮಾಡಿದ
ದೂರ-ಲಿಲ್ಲ
ದೂರಲು
ದೂರ-ವಾಗಿ
ದೂರ-ವಾಗುವ
ದೂರ-ವಾಗು-ವೆವು
ದೂರ-ವಿ-ರುವ
ದೂರ-ವಿ-ರು-ವುದೇ
ದೂರ-ಸರಿದು
ದೂರಿ
ದೂರಿ-ದ-ನಂತೆ
ದೂರಿ-ದನು
ದೂರಿ-ರು-ವರು
ದೂರುತ್ತ
ದೂರುತ್ತಿರು
ದೂರುವ
ದೂರು-ವರು
ದೂರು-ವಿರಿ
ದೂರು-ವು-ದಕ್ಕೆ
ದೂರು-ವು-ದನ್ನು
ದೂರು-ವು-ದಿಲ್ಲ
ದೂಷಣೆ-ಯನ್ನು
ದೂಷಿಸ-ಬೇಕಾಗಿಲ್ಲ
ದೂಷಿಸ-ಬೇಕು
ದೂಷಿಸ-ಬೇಡಿ
ದೂಷಿ-ಸಲು
ದೃಢ-ಚಿತ್ತವು
ದೃಢತೆ
ದೃಢ-ನಂಬಿಕೆ
ದೃಢ-ನಂಬಿ-ಕೆ-ಗನು-ಸಾರ-ವಾಗಿ
ದೃಢ-ನಿಶ್ಚಯ
ದೃಢ-ಪಡಿಸಿ
ದೃಢ-ಪಡಿ-ಸಿ-ರು-ವುದು
ದೃಢ-ಪಡಿ-ಸು-ವುದು
ದೃಢ-ಭಕ್ತಿ-ಯೊಂದೇ
ದೃಢ-ಮನ-ಸ್ಕನು
ದೃಢ-ಮನ-ಸ್ಕ-ರ-ನ್ನಾಗಿ
ದೃಢ-ವಾಗಿ
ದೃಢ-ವಾಗಿದೆ
ದೃಢ-ವಾಗು-ವುದು
ದೃಢ-ವಿಶ್ವಾ-ಸ-ವನ್ನು
ದೃಢ-ಶ್ರದ್ಧೆ-ಯನ್ನೂ
ದೃಢಿಷ್ಠ
ದೃಢಿ-ಷ್ಠರೂ
ದೃಶ್ಯ
ದೃಶ್ಯ-ಗಳಾಗಿ
ದೃಶ್ಯ-ಗಳಿಂದ
ದೃಶ್ಯ-ಗಳು
ದೃಷ್ಟಾಂತ-ಗಳ
ದೃಷ್ಟಾಂತಪ್ರಾಯ-ವಾದ
ದೃಷ್ಟಾಂತ-ವನ್ನು
ದೃಷ್ಟಾಂತವು
ದೃಷ್ಟಿ
ದೃಷ್ಟಿ-ಗಳಿಂದಲೂ
ದೃಷ್ಟಿಗೆ
ದೃಷ್ಟಿ-ಭೇದ-ದಿಂದ
ದೃಷ್ಟಿ-ಭೇದವೇ
ದೃಷ್ಟಿಯ
ದೃಷ್ಟಿ-ಯನ್ನು
ದೃಷ್ಟಿ-ಯಲ್ಲ
ದೃಷ್ಟಿ-ಯಲ್ಲಿ
ದೃಷ್ಟಿ-ಯಿಂದ
ದೃಷ್ಟಿ-ಯಿಂದಲೇ
ದೃಷ್ಟಿ-ಯಿಂದಲ್ಲ
ದೃಷ್ಟಿ-ಯಿಲ್ಲದ
ದೃಷ್ಟಿಯು
ದೃಷ್ಟಿಯೂ
ದೆಂದೂ
ದೆವೋ
ದೆವ್ವ-ಗಳನ್ನು
ದೆವ್ವಗಿವ್ವ-ಗಳಿಗೂ
ದೆಸೆ-ಯಿಂದ
ದೆಸೆ-ಯಿಂದಲೇ
ದೇಗುಲ-ಗಳನ್ನೂ
ದೇವ
ದೇವ-ತಾ-ಮೂರ್ತಿ-ಗಳು
ದೇವ-ತಾರ್ಚ-ನೆಯ
ದೇವತೆ
ದೇವ-ತೆ-ಗಳ
ದೇವ-ತೆ-ಗಳನ್ನು
ದೇವ-ತೆ-ಗಳನ್ನೂ
ದೇವ-ತೆ-ಗಳಲ್ಲ
ದೇವ-ತೆ-ಗಳ-ವ-ರೆಗೆ
ದೇವ-ತೆ-ಗಳಾದರು
ದೇವ-ತೆ-ಗಳಿ-ಗಿಂತಲೂ
ದೇವ-ತೆ-ಗಳಿಗೂ
ದೇವ-ತೆ-ಗಳಿಗೆ
ದೇವ-ತೆ-ಗಳಿ-ಗೆಲ್ಲಾ
ದೇವ-ತೆ-ಗಳಿ-ದ್ದವು
ದೇವ-ತೆ-ಗಳು
ದೇವ-ತೆ-ಗಳೆಲ್ಲ
ದೇವ-ತೆ-ಗಳೊಂದಿಗೆ
ದೇವ-ತೆಗೂ
ದೇವ-ತೆ-ಯನ್ನು
ದೇವ-ತೆ-ಯಾಗಿ
ದೇವ-ತೆ-ಯಾಗಿಯೇ
ದೇವ-ತೆ-ಯಾದ
ದೇವ-ತೆ-ಯಿತ್ತು
ದೇವ-ತೆಯೂ
ದೇವ-ತೆಯೇ
ದೇವ-ದೂ-ತನ-ನ್ನೋ
ದೇವ-ದೂತ-ನೆಂದು
ದೇವ-ದೂತ-ರನ್ನು
ದೇವ-ದೂ-ತ-ರನ್ನೂ
ದೇವ-ದೂ-ತರು
ದೇವ-ದೂತರೋ
ದೇವ-ನಾ-ಗಲೀ
ದೇವ-ನಿಂದೆ-ಯಿಂದ
ದೇವನು
ದೇವ-ನೆ-ಡೆಗೆ
ದೇವ-ಬಹದ್ದೂರ್
ದೇವ-ಭಾಷೆ-ಯನ್ನು
ದೇವ-ಮಾ-ನ-ವನ
ದೇವ-ಮಾ-ನ-ವರೋ
ದೇವ-ಯಾನಃ
ದೇವ-ಯಾ-ನಕ್ಕೆ
ದೇವರ
ದೇವ-ರಂತಹ
ದೇವ-ರಂತೆ
ದೇವ-ರ-ನ್ನಾಗಿ
ದೇವ-ರನ್ನು
ದೇವ-ರ-ನ್ನೆಲ್ಲಾ
ದೇವ-ರ-ನ್ನೇ
ದೇವ-ರ-ಭಾ-ವನೆ
ದೇವ-ರಲ್ಲಿ
ದೇವ-ರಷ್ಟೇ
ದೇವ-ರಾಗ-ಬೇಕು
ದೇವ-ರಾಗಿ
ದೇವ-ರಿಂದ
ದೇವ-ರಿಂದಾ-ಗಲೀ
ದೇವ-ರಿಗಾ-ಗಲಿ
ದೇವ-ರಿ-ಗಿಂತ
ದೇವ-ರಿಗೂ
ದೇವ-ರಿಗೆ
ದೇವ-ರಿಗೇ
ದೇವ-ರಿ-ದ್ದಾನೆ
ದೇವ-ರಿದ್ದಾನೆಂಬು-ದಕ್ಕೆ
ದೇವ-ರಿ-ರು-ವನೆ
ದೇವ-ರಿಲ್ಲದೆ
ದೇವರು
ದೇವ-ರು-ಗಳಿವೆ
ದೇವ-ರು-ಗಳು
ದೇವ-ರು-ಗಳೊಂದಿಗೆ
ದೇವರೂ
ದೇವರೆ
ದೇವ-ರೆಂದರೆ
ದೇವ-ರೆಂದು
ದೇವ-ರೆ-ಡೆಗೆ
ದೇವ-ರೆ-ಪ್ರತಿ-ಯೊಬ್ಬ
ದೇವರೇ
ದೇವ-ರೊಬ್ಬನಿ-ದ್ದರೆ
ದೇವ-ರೊ-ಬ್ಬನೇ
ದೇವರೋ
ದೇವ-ಲೋಕಕ್ಕೂ
ದೇವ-ವಾಣಿ
ದೇವ-ಸಂತಾ-ನರು
ದೇವ-ಸ್ಥಾನ
ದೇವ-ಸ್ಥಾನ-ಕ್ಕಾದರೂ
ದೇವ-ಸ್ಥಾನಕ್ಕೆ
ದೇವ-ಸ್ಥಾನ-ಗಳ
ದೇವ-ಸ್ಥಾನ-ಗಳನ್ನು
ದೇವ-ಸ್ಥಾನ-ಗಳ-ಲ್ಲಿ-ರುವ
ದೇವ-ಸ್ಥಾನ-ಗಳಿ-ದ್ದರೂ
ದೇವ-ಸ್ಥಾನ-ಗಳು
ದೇವ-ಸ್ಥಾನದ
ದೇವ-ಸ್ಥಾನ-ದಲ್ಲಿ
ದೇವ-ಸ್ಥಾನ-ವನ್ನು
ದೇವ-ಸ್ಥಾನ-ವಿರ-ಬೇಕು
ದೇವ-ಸ್ಥಾನ-ವಿ-ಲ್ಲದೇ
ದೇವಾ
ದೇವಾ-ನುಗ್ರಹ-ಹೇ-ತು-ಕ-ಮ್
ದೇವಾ-ಲಯ
ದೇವಾ-ಲ-ಯಕ್ಕೆ
ದೇವಾ-ಲಯ-ಗಳ
ದೇವಾ-ಲಯ-ಗಳ-ನ್ನಾಗಿ
ದೇವಾ-ಲಯ-ಗಳನ್ನು
ದೇವಾ-ಲಯ-ಗಳನ್ನೂ
ದೇವಾ-ಲಯ-ಗಳಲ್ಲಿ
ದೇವಾ-ಲಯ-ಗಳಿಗೆ
ದೇವಾ-ಲಯ-ಗಳು
ದೇವಾ-ಲಯ-ಗಳೂ
ದೇವಾ-ಲಯ-ಗಳೆಂದು
ದೇವಾ-ಲಯ-ದಂತೆ
ದೇವಾ-ಲ-ಯವು
ದೇವಾ-ವಗಚ್ಛ
ದೇವಿಯ
ದೇವಿ-ಯರು
ದೇವಿಯು
ದೇವಿಯೇ
ದೇವೀ
ದೇವ್ಧರ್ಧಿಲ್ಲ
ದೇಶ
ದೇಶ-ಕಾಲ
ದೇಶ-ಕಾಲ-ಅವಸ್ಥಾ
ದೇಶ-ಕಾಲ-ಗಳ
ದೇಶ-ಕಾಲ-ಗಳಿಗೆ
ದೇಶ-ಕಾಲ-ಪಾತ್ರ-ಗಳಿಗೆ
ದೇಶ-ಕಾಲ-ಬದ್ದ-ವಾದವು
ದೇಶ-ಕ್ಕಿಂತ
ದೇಶಕ್ಕೂ
ದೇಶ-ಕ್ಕೂ-ಸ್ವಾ-ಭಾವಿಕ
ದೇಶಕ್ಕೆ
ದೇಶಕ್ಕೇ
ದೇಶ-ಗಳ
ದೇಶ-ಗಳಂತೆ
ದೇಶ-ಗಳನ್ನು
ದೇಶ-ಗಳಲ್ಲಿ
ದೇಶ-ಗಳ-ಲ್ಲಿಯೂ
ದೇಶ-ಗಳಲ್ಲೂ
ದೇಶ-ಗಳ-ಲ್ಲೆಲ್ಲ
ದೇಶ-ಗಳ-ಲ್ಲೆಲ್ಲಾ
ದೇಶ-ಗಳ-ವರು
ದೇಶ-ಗಳಾಚೆ
ದೇಶ-ಗಳಿಂದ
ದೇಶ-ಗಳಿಂದಲೂ
ದೇಶ-ಗಳಿಗೂ
ದೇಶ-ಗಳಿಗೆ
ದೇಶ-ಗಳಿಗೆಲ್ಲ
ದೇಶ-ಗಳು
ದೇಶ-ಗಳೂ
ದೇಶ-ಗಳೊಂದಿಗೆ
ದೇಶದ
ದೇಶ-ದಲ್ಲಾ-ಗಲಿ
ದೇಶ-ದಲ್ಲಾ-ಗಲೀ
ದೇಶ-ದಲ್ಲಿ
ದೇಶ-ದಲ್ಲಿದ್ದ
ದೇಶ-ದಲ್ಲಿ-ದ್ದೆನೊ
ದೇಶ-ದಲ್ಲಿಯೂ
ದೇಶ-ದಲ್ಲಿಯೇ
ದೇಶ-ದಲ್ಲಿ-ರುವ
ದೇಶ-ದಲ್ಲಿ-ರು-ವಾಗ
ದೇಶ-ದಲ್ಲೂ
ದೇಶ-ದಲ್ಲೆಲ್ಲಾ
ದೇಶ-ದಲ್ಲೇ
ದೇಶ-ದ-ವನು
ದೇಶ-ದ-ವ-ರನ್ನು
ದೇಶ-ದ-ವ-ರಾಗಿ-ದ್ದು-ದ-ರಿಂದ
ದೇಶ-ದ-ವರಿ-ಗಿಂತ
ದೇಶ-ದ-ವ-ರಿಗೂ
ದೇಶ-ದ-ವರು
ದೇಶ-ದ-ವ-ರೆಲ್ಲರೂ
ದೇಶ-ದ-ವ-ರೆಲ್ಲಾ
ದೇಶ-ದ-ವ-ರೊಂದಿಗೆ
ದೇಶ-ದ-ವರೊ-ಬ್ಬರು
ದೇಶ-ದಿಂದ
ದೇಶ-ದಿಂದಲೂ
ದೇಶ-ದಿಂದಲೇ
ದೇಶ-ದಿಂದಲೋ
ದೇಶ-ಬಾಂಧ-ವನೊ-ಬ್ಬ-ನನ್ನು
ದೇಶ-ಬಾಂಧ-ವ-ರನ್ನು
ದೇಶ-ಬಾಂಧವ-ರಿಗೆ
ದೇಶ-ಬಾಂಧ-ವರು
ದೇಶ-ಬಾಂಧ-ವರೆ
ದೇಶ-ಬಾಂಧ-ವರೇ
ದೇಶ-ಭಕ್ತ-ರಾಗಿ
ದೇಶ-ಭಕ್ತ-ರಾ-ಗು-ವು-ದಕ್ಕೆ
ದೇಶ-ಭಕ್ತರೇ
ದೇಶ-ಭಕ್ತಿ
ದೇಶ-ಭಕ್ತಿಯ
ದೇಶ-ಭಕ್ತಿ-ಯನ್ನು
ದೇಶ-ಭಾಷಾ-ನದೀ-ಪಾತ್ರ-ಗಳಲ್ಲಿ
ದೇಶ-ಭಾಷೆ-ಗಳ
ದೇಶ-ಭಾಷೆ-ಯಲ್ಲಿ
ದೇಶ-ಭ್ರಷ್ಟ-ರ-ನ್ನಾಗಿ
ದೇಶ-ವನ್ನಾ-ಗಲಿ
ದೇಶ-ವನ್ನು
ದೇಶ-ವನ್ನೇ
ದೇಶ-ವಾದರೂ
ದೇಶವು
ದೇಶವೂ
ದೇಶ-ವೆಂದೇ
ದೇಶ-ವೆಲ್ಲ
ದೇಶವೇ
ದೇಶಾದ್ಯಂತವೂ
ದೇಶಿ-ಯರ
ದೇಶೀಯ-ರಿಗೆ
ದೇಶೀ-ಯರು
ದೇಹ
ದೇಹ-ಕ್ಕಿಂತ
ದೇಹಕ್ಕೆ
ದೇಹ-ಗಳನ್ನು
ದೇಹ-ಗಳಿಂದ
ದೇಹ-ಗಳಿಗೂ
ದೇಹ-ಗಳು
ದೇಹ-ಗಳೆ-ಲ್ಲವೂ
ದೇಹದ
ದೇಹ-ದಲ್ಲಿ
ದೇಹ-ದಲ್ಲಿ-ರ-ಲಾ-ರದು
ದೇಹ-ದಷ್ಟು
ದೇಹ-ದಿಂದ
ದೇಹ-ದೊಂದಿಗೆ
ದೇಹ-ದೊಡನೆ
ದೇಹ-ಬುದ್ಧಿಯ
ದೇಹ-ಭಾವ-ವಿದ್ದರೆ
ದೇಹ-ವನ್ನು
ದೇಹ-ವಲ್ಲ
ದೇಹ-ವಿದೆ
ದೇಹ-ವಿದೆಯೋ
ದೇಹ-ವಿಲ್ಲ
ದೇಹವು
ದೇಹವೂ
ದೇಹವೆ
ದೇಹ-ವೆಂದು
ದೇಹ-ವೆಂದೂ
ದೇಹವೇ
ದೇಹ-ಸುಖ
ದೇಹ-ಸ್ಥಿತಿ
ದೈನಂದಿನ
ದೈನ್ಯ-ವನ್ನೂ
ದೈವ-ಕೃಪೆ-ಯಿಂದ
ದೈವ-ಕೃಪೆ-ಯಿಂದಲೂ
ದೈವತ್ವ
ದೈವತ್ವ-ವಿದೆ
ದೈವ-ನಿಂದೆ
ದೈವ-ಭಕ್ತರು
ದೈವ-ಭಕ್ತರೂ
ದೈವ-ಭಕ್ತಿ
ದೈವವು
ದೈವಾ-ನುಗ್ರಹ-ದಿಂದ
ದೈವಿಕ-ನಾಗಿ-ರಲೀ
ದೈವೀ
ದೈವೀ-ಭಾ-ವನೆ
ದೈವೇಚ್ಛೆ
ದೈಹಿಕ
ದೈಹಿ-ಕ-ವಾಗಿ
ದೈಹಿ-ಕ-ಶ್ರಮ
ದೊಂದಿಗೆ
ದೊಂಬಿ-ಯಂತೆ
ದೊಂಬಿ-ಯಾಗಿ-ರುವು
ದೊಡ್ಡ
ದೊಡ್ಡ-ದಾಗಿ
ದೊಡ್ಡ-ದಾ-ಗುತ್ತಾ
ದೊಡ್ಡ-ದಾಗು-ವುದೊ
ದೊಡ್ಡದು
ದೊಡ್ಡ-ದೊಂದು
ದೊಡ್ಡ-ವ-ನಾಗಿ-ರಲಿ
ದೊಡ್ಡ-ವನು
ದೊರಕದು
ದೊರಕದೇ
ದೊರ-ಕಲಿ
ದೊರ-ಕ-ಲಿಲ್ಲ
ದೊರಕಿ
ದೊರಕಿತು
ದೊರಕಿ-ದು-ದಕ್ಕೆ
ದೊರಕಿ-ರ-ಲಿಲ್ಲ
ದೊರಕಿ-ಸಿರು-ವುದು
ದೊರಕು-ತ್ತದೆ
ದೊರಕುತ್ತಿರು-ವುದು
ದೊರಕು-ತ್ತಿವೆ
ದೊರಕುವ
ದೊರಕುವಂತಾಗ-ಬೇಕು
ದೊರಕು-ವಂತೆ
ದೊರಕುವ-ವರೆಗೂ
ದೊರಕುವ-ವರೆ-ವಿಗೂ
ದೊರ-ಕು-ವು-ದಿಲ್ಲ
ದೊರಕು-ವುದು
ದೊರಕು-ವುವು
ದೊರೆ
ದೊರೆ-ಗಳು
ದೊರೆತ
ದೊರೆ-ತಂತಾ-ಗಿದೆ
ದೊರೆ-ತಂತಾ-ಗು-ತ್ತದೆ
ದೊರೆ-ತದ್ದೂ
ದೊರೆ-ತರೆ
ದೊರೆ-ತವು
ದೊರೆ-ತಾಗ
ದೊರೆ-ತಿದೆ
ದೊರೆ-ತಿದ್ದರೂ
ದೊರೆ-ತಿರು-ವಾಗ
ದೊರೆ-ತಿಲ್ಲ
ದೊರೆ-ತು-ದ-ಕ್ಕಾಗಿ
ದೊರೆ-ತು-ದಕ್ಕೆ
ದೊರೆ-ತೊ-ಡ-ನೆಯೇ
ದೊರೆ-ಯದೆ
ದೊರೆ-ಯ-ಲಾ-ರದು
ದೊರೆ-ಯು-ತ್ತದೆ
ದೊರೆ-ಯು-ತ್ತವೆ
ದೊರೆ-ಯುತ್ತಿ-ರುವ
ದೊರೆ-ಯುವ
ದೊರೆ-ಯು-ವು-ದಿಲ್ಲ
ದೊರೆ-ಯು-ವುದು
ದೊರೆ-ಯು-ವುವು
ದೊರೆಯೂ
ದೊಳಗೆ
ದೋಚು-ವಂತೆ
ದೋಷ
ದೋಷಕ್ಕೆ
ದೋಷ-ಗಳ
ದೋಷ-ಗಳನ್ನು
ದೋಷ-ಗಳಿಂದ
ದೋಷ-ಗಳಿ-ದ್ದರೂ
ದೋಷ-ಗಳಿ-ದ್ದರೆ
ದೋಷ-ಗಳಿ-ದ್ಧರೂ
ದೋಷ-ಗಳಿವೆ
ದೋಷ-ಗಳು
ದೋಷ-ಗಳೇ-ಶುದ್ಧ
ದೋಷ-ಪೂರಿತ
ದೋಷ-ಪೂರ್ಣ-ವಾಗಿ
ದೋಷ-ಮಯ
ದೋಷ-ವನ್ನು
ದೋಷ-ವನ್ನೆಲ್ಲಾ
ದೋಷ-ವನ್ನೇ
ದೋಷ-ವಿದೆ
ದೋಷ-ವಿಲ್ಲ
ದೋಷವು
ದೋಷವೂ
ದೋಷ-ವೇ-ನೆಂದರೆ
ದೋಷ-ವೊಂದೇ
ದೋಷಾ-ರೋಪಣೆ
ದೋಷಿ-ಯಲ್ಲ
ದೌರ್ಜನ್ಯ
ದೌರ್ಜನ್ಯ-ಪರ-ರಿಗೆ
ದೌರ್ಜನ್ಯವೇ
ದೌರ್ಬಲ್ಯ
ದೌರ್ಬಲ್ಯ-ಗಳು
ದೌರ್ಬಲ್ಯದ
ದೌರ್ಬಲ್ಯ-ದಾಸ್ಯ-ಗಳ
ದೌರ್ಬಲ್ಯ-ದಿಂದ
ದೌರ್ಬಲ್ಯ-ದಿಂದಲೂ
ದೌರ್ಬಲ್ಯ-ವನ್ನು
ದೌರ್ಬಲ್ಯ-ವನ್ನೂ
ದೌರ್ಬಲ್ಯ-ವನ್ನೇ
ದೌರ್ಬಲ್ಯ-ವಿದ್ದೇ
ದೌರ್ಬಲ್ಯವು
ದೌರ್ಬಲ್ಯವೇ
ದೌರ್ಭಾಗ್ಯ-ವಶ-ವಾಗಿ
ದ್ದನೋ
ದ್ದಾನೆಯೊ
ದ್ದೇನೆ
ದ್ಧೇವ್ಧರ್ಧಿದ್ದ್ಧಾನೆ
ದ್ರವ-ದಿಂದ
ದ್ರವ-ವನ್ನು
ದ್ರವ್ಯ
ದ್ರವ್ಯ-ಕಣ-ಗಳು
ದ್ರವ್ಯ-ದಾ-ಸೆಗೆ
ದ್ರವ್ಯ-ದಿಂದಲ್ಲ
ದ್ರವ್ಯಾರ್ಜ-ನೆಗೆ
ದ್ರವ್ಯಾರ್ಜ-ನೆಯ
ದ್ರಾವಿಡ
ದ್ರಾವಿಡರು
ದ್ರೋಹಿ-ಗಳಾರೂ
ದ್ರೌಪದಿ
ದ್ರೌಪ-ದಿಯು
ದ್ವಂದ್ವ-ಯುದ್ಧ
ದ್ವಾ
ದ್ವಾದಶಾಂತ-ಕ್ಷೇತ್ರ-ವಾದ
ದ್ವಾ-ಪರ-ಯು-ಗಕ್ಕೆ
ದ್ವಾರಾ
ದ್ವೀಪಕ್ಕೆ
ದ್ವೀಪದ
ದ್ವೀಪಸ್ತೋಮಕ್ಕೆ
ದ್ವೇಷ
ದ್ವೇಷ-ಅಸೂಯೆ-ಗಳೆಲ್ಲ
ದ್ವೇಷ-ಕಾರಕ
ದ್ವೇಷ-ಭಾವ-ನೆ-ಯನ್ನು
ದ್ವೇಷ-ವನ್ನು
ದ್ವೇಷ-ವಲ್ಲ
ದ್ವೇಷ-ವಿ-ರ-ಲಿಲ್ಲ
ದ್ವೇಷ-ವಿಲ್ಲ-ದಿ-ರಲಿ
ದ್ವೇಷವು
ದ್ವೇಷಾಸೂಯೆ-ಗಳಿಗೆ
ದ್ವೇಷಾಸೂಯೆ-ಗಳೂ
ದ್ವೇಷಿ-ಸದೆ
ದ್ವೇಷಿಸ-ಬೇಕಾಗಿಲ್ಲ
ದ್ವೇಷಿಸ-ಬೇಕು
ದ್ವೇಷಿಸಬೇಕೆಂದಿಲ್ಲ
ದ್ವೇಷಿಸಿ
ದ್ವೇಷಿಸಿ-ರು-ವರು
ದ್ವೇಷಿಸುತ್ತಾ-ನೆಯೋ
ದ್ವೇಷಿ-ಸು-ವಂತೆ
ದ್ವೇಷಿ-ಸುವ-ವ-ರಲ್ಲಿ
ದ್ವೇಷಿ-ಸು-ವು-ದಿಲ್ಲವೋ
ದ್ವೈತ
ದ್ವೈತ-ಇವು-ಗಳಲ್ಲಿ
ದ್ವೈತಕ್ಕೆ
ದ್ವೈತದ
ದ್ವೈತ-ದೃಷ್ಟಿಗೆ
ದ್ವೈತ-ಬೋಧೆ
ದ್ವೈತ-ಭಾಗ-ವನ್ನು
ದ್ವೈತ-ಭಾವಕ್ಕೆ
ದ್ವೈತ-ಭಾವ-ದಿಂದ
ದ್ವೈತ-ಭಾವ-ನೆ-ಯನ್ನು
ದ್ವೈತ-ಭಾವ-ವಿದೆ
ದ್ವೈತ-ಭಾವ-ವಿ-ರುವ
ದ್ವೈತ-ಮ-ತಕ್ಕೆ
ದ್ವೈತ-ಮತ-ಗಳ
ದ್ವೈತ-ವನ್ನಾ-ಗಲೀ
ದ್ವೈತ-ವನ್ನು
ದ್ವೈತ-ವನ್ನೋ
ದ್ವೈತ-ವಾದಿ-ಗಳಿಗೂ
ದ್ವೈತವು
ದ್ವೈತ-ವೆಂದು
ದ್ವೈತವೇ
ದ್ವೈತ-ಸಿದ್ದಾಂತವು
ದ್ವೈತ-ಸಿದ್ಧಾಂತ-ವಾದ-ದಿಂದ
ದ್ವೈತಾ-ರೋಪಣೆ
ದ್ವೈತಿ
ದ್ವೈತಿ-ಗಳ
ದ್ವೈತಿ-ಗಳಲ್ಲಿ
ದ್ವೈತಿ-ಗಳಾಗಲಿ
ದ್ವೈತಿ-ಗಳಾಗಲೀ
ದ್ವೈತಿ-ಗಳಾಗಿ-ರಲೀ
ದ್ವೈತಿ-ಗಳು
ದ್ವೈತಿ-ಗಳೆ-ಲ್ಲರೂ
ದ್ವೈತಿ-ಗಳೇ
ದ್ವೈತಿ-ಗಳೋ
ದ್ವೈತಿ-ಯಾಗಲೀ
ದ್ವೈತಿ-ಯಾಗಿ-ರ-ಬಹುದು
ದ್ವೈತಿ-ಯಾಗಿ-ರು-ವಷ್ಟೇ
ದ್ವೈತಿಯೂ
ಧಕ್ಕೆ
ಧನ
ಧನಂ
ಧನ-ಕ್ಕಾಗಿ
ಧನ-ದಾಸೆ
ಧನೇನ
ಧನ್ಯ
ಧನ್ಯನು
ಧನ್ಯ-ರಾಗಿ
ಧನ್ಯ-ರಾಗಿ-ದ್ದೇವೆ
ಧನ್ಯರು
ಧನ್ಯ-ವಾಗುತ್ತಿತ್ತು
ಧನ್ಯ-ವಾದ
ಧನ್ಯ-ವಾದಕ್ಕೆ
ಧನ್ಯ-ವಾದ-ಗಳನ್ನು
ಧನ್ಯ-ವಾದ-ಗಳು
ಧನ್ಯ-ವಾದ-ವನ್ನು
ಧಮನಿ
ಧಮ-ನಿ-ಗಳಲ್ಲಿ
ಧಮ-ನಿ-ಯಲ್ಲಿ
ಧರ
ಧರಿ-ಸಿದ
ಧರಿಸಿ-ರು-ವನು
ಧರಿಸಿ-ರು-ವುದು
ಧರಿಸಿ-ರುವೆ
ಧರಿ-ಸುತ್ತ
ಧರಿ-ಸುತ್ತಾ
ಧರಿಸು-ವು-ದಲ್ಲ
ಧರಿ-ಸು-ವುದೂ
ಧರ್ಮ
ಧರ್ಮಂ
ಧರ್ಮ-ಕೋಶಸ್ಯ
ಧರ್ಮ-ಕ್ಕಾಗಿ
ಧರ್ಮ-ಕ್ಕಿಂತ
ಧರ್ಮಕ್ಕೆ
ಧರ್ಮ-ಕ್ಷೇತ್ರ-ದಲ್ಲಿ
ಧರ್ಮ-ಗಳ
ಧರ್ಮ-ಗಳನ್ನು
ಧರ್ಮ-ಗಳನ್ನೂ
ಧರ್ಮ-ಗಳಲ್ಲಿ
ಧರ್ಮ-ಗಳ-ಲ್ಲಿಯೂ
ಧರ್ಮ-ಗಳ-ಲ್ಲಿ-ರುವ
ಧರ್ಮ-ಗಳ-ಲ್ಲಿ-ರು-ವಂತೆಯೇ
ಧರ್ಮ-ಗಳಿ-ಗಾಗಿ
ಧರ್ಮ-ಗಳಿ-ಗಿಂತ
ಧರ್ಮ-ಗಳಿಗೆ
ಧರ್ಮ-ಗಳು
ಧರ್ಮ-ಗಳೂ
ಧರ್ಮ-ಗಳೆಲ್ಲ
ಧರ್ಮ-ಗಳೆಲ್ಲಾ
ಧರ್ಮ-ಗ್ರಂಥ-ಗಳು
ಧರ್ಮ-ಗ್ರಂಥ-ವನ್ನು
ಧರ್ಮ-ಗ್ರಂಥವು
ಧರ್ಮ-ಗ್ಲಾನಿ-ಯಾಗು-ವುದೋ
ಧರ್ಮ-ಗ್ಲಾನಿ-ಯಾ-ದಾಗ
ಧರ್ಮದ
ಧರ್ಮ-ದಂತಹ
ಧರ್ಮ-ದಲ್ಲಿ
ಧರ್ಮ-ದಲ್ಲಿದೆ
ಧರ್ಮ-ದಲ್ಲಿಯೂ
ಧರ್ಮ-ದಲ್ಲಿ-ರುವ
ಧರ್ಮ-ದಲ್ಲಿ-ರು-ವುದು
ಧರ್ಮ-ದಲ್ಲೇ
ಧರ್ಮ-ದ-ವರೂ
ಧರ್ಮ-ದಿಂದ
ಧರ್ಮ-ದೊಂದಿಗೂ
ಧರ್ಮ-ದೊಂದಿಗೆ
ಧರ್ಮ-ಧನ-ವನ್ನು
ಧರ್ಮ-ಧ್ವಜ-ವನ್ನು
ಧರ್ಮ-ಪಥ-ದಲ್ಲಿ
ಧರ್ಮ-ಪಿ-ಪಾಸು-ಗಳು
ಧರ್ಮ-ಪ್ರ-ಚಾ-ರಕ್ಕೆ
ಧರ್ಮ-ಭೂಮಿ
ಧರ್ಮ-ಭೂಮಿ-ಕೆ-ಯಲ್ಲಿ
ಧರ್ಮ-ಮಾರ್ಗ
ಧರ್ಮ-ರಕ್ಷಣೆ
ಧರ್ಮ-ರಕ್ಷಣೆ-ಗಾ-ಗಿಯೇ
ಧರ್ಮ-ರಹಸ್ಯ-ಗಳನ್ನು
ಧರ್ಮ-ರಾಜ-ನನ್ನು
ಧರ್ಮ-ರಾಜಾಯ
ಧರ್ಮ-ವನ್ನಾ-ಗಲಿ
ಧರ್ಮ-ವನ್ನು
ಧರ್ಮ-ವನ್ನು-ರಕ್ಷಿಸಿ-ದರು
ಧರ್ಮ-ವನ್ನೂ
ಧರ್ಮ-ವನ್ನೇ
ಧರ್ಮ-ವನ್ನೊ-ಪ್ಪಿ-ಕೊಳ್ಳ-ಬೇಕು
ಧರ್ಮ-ವಲ್ಲ
ಧರ್ಮ-ವಲ್ಲ-ವೆಂದು
ಧರ್ಮ-ವಾಗಲಿ
ಧರ್ಮ-ವಾಗಿದೆ
ಧರ್ಮ-ವಾಗಿ-ಬಿಡು-ವುದು
ಧರ್ಮ-ವಾಗಿರು
ಧರ್ಮ-ವಿದೆ
ಧರ್ಮ-ವಿದ್ದರೂ
ಧರ್ಮ-ವಿದ್ದರೆ
ಧರ್ಮ-ವಿರ-ಲಾ-ರದು
ಧರ್ಮ-ವಿರು-ವುದು
ಧರ್ಮ-ವಿಲ್ಲ
ಧರ್ಮ-ವಿಲ್ಲ-ದಿ-ರಲಿ
ಧರ್ಮ-ವಿ-ಲ್ಲದೆ
ಧರ್ಮ-ವೀರ-ರನ್ನು
ಧರ್ಮವು
ಧರ್ಮವೂ
ಧರ್ಮ-ವೆಂದರೆ
ಧರ್ಮ-ವೆಂದು
ಧರ್ಮ-ವೆಂದೂ
ಧರ್ಮ-ವೆಂಬ
ಧರ್ಮ-ವೆಂಬು-ದನ್ನು
ಧರ್ಮ-ವೆಂಬುದು
ಧರ್ಮವೇ
ಧರ್ಮ-ವೇನೂ
ಧರ್ಮ-ವೇನೋ
ಧರ್ಮ-ವೊಂದನ್ನೇ
ಧರ್ಮ-ವೊಂದೇ
ಧರ್ಮ-ಶಾಸ್ತ್ರ-ಗಳಲ್ಲಿ
ಧರ್ಮ-ಶಾಸ್ತ್ರ-ಗಳು
ಧರ್ಮ-ಶಾಸ್ತ್ರ-ದಲ್ಲಿ
ಧರ್ಮ-ಶೀಲನೂ
ಧರ್ಮ-ಶ್ರದ್ಧೆ
ಧರ್ಮ-ಸಂಸ್ಥಾಪಕ
ಧರ್ಮ-ಸಂಸ್ಥಾಪಕರ
ಧರ್ಮ-ಸಂಸ್ಥಾಪನಾ-ಚಾರ್ಯರು
ಧರ್ಮ-ಸಂಸ್ಥಾಪನಾರ್ಥಾಯ
ಧರ್ಮ-ಸಮ್ಮೇಳನಕ್ಕೆ
ಧರ್ಮ-ಸಮ್ಮೇಳನದ
ಧರ್ಮ-ಸಮ್ಮೇಳನ-ದಲ್ಲಿ
ಧರ್ಮ-ಸಾಮರಸ್ಯ
ಧರ್ಮ-ಸೌಧ
ಧರ್ಮ-ಸ್ಥಾಪಕನ
ಧರ್ಮಸ್ಯ
ಧರ್ಮಾ-ಚ-ರಣೆ-ಯಿಂದಲೂ
ಧರ್ಮಾ-ನು-ಯಾ-ಯಿ-ಗಳ
ಧರ್ಮಾ-ನು-ಯಾ-ಯಿ-ಗಳಾದ
ಧರ್ಮಾ-ಭಿ-ಮಾನ
ಧರ್ಮಾ-ವಲಂಬಿ-ಗಳನ್ನು
ಧರ್ಮಿಷ್ಠ-ನೆಂಬ
ಧರ್ಮಿಷ್ಠರಾ-ದರೂ
ಧರ್ಮೋ-ದಯ-ವಾಗು-ವುದು
ಧರ್ಮೋ-ದ್ಧಾ-ರದ
ಧಾನಿಯ
ಧಾರಣ
ಧಾರಣೆ
ಧಾರಿ-ಯಾಗಿ
ಧಾರೆ
ಧಾರೆ-ಯೆರೆದಿದ್ದಾರೋ
ಧಾರೆ-ಯೆರೆದು
ಧಾರ್ಮಿಕ
ಧಾರ್ಮಿಕ-ತೆ-ಗಳು
ಧಾರ್ಮಿ-ಕನೂ
ಧಾರ್ಮಿಕ-ರಾ-ಗುವ
ಧಾರ್ಮಿಕ-ರಾ-ಗುವರು
ಧಾರ್ಮಿ-ಕರು
ಧಾರ್ಮಿಕ-ವಾಗ-ಬೇಕು
ಧಾರ್ಮಿಕ-ವಾಗಿ-ದ್ದವು
ಧಾರ್ಮಿಕ-ವಾಗಿ-ರ-ಬೇಕು
ಧಾರ್ಮಿಕ-ಶಕ್ತಿ
ಧಾಳಿ
ಧಾವಿಸ-ಬಲ್ಲಿರಾ
ಧಾವಿಸುತ್ತಿ-ರು-ವನು
ಧಿಃಕ್ಕ-ರಿ-ಸ-ಲಿಲ್ಲ
ಧಿಕ್ಕರಿಸ-ಬಲ್ಲರು
ಧೀನತೆ-ಯಲ್ಲಿ-ದ್ದರೂ
ಧೀರ
ಧೀರ-ನಾಗು
ಧೀರ-ನಾದ
ಧೀರ-ಭಾಷೆ-ಯಲ್ಲಿ
ಧೀರ-ಮಂತ್ರ
ಧೀರ-ರಂತೆ
ಧೀರ-ರಾಗಲಿ
ಧೀರ-ರಾಗಿ
ಧೀರ-ರಾಗಿ-ದ್ದರು
ಧೀರ-ರಾಗಿ-ದ್ದರೆ
ಧೀರ-ರಾಗು-ವಿರಿ
ಧೀರರು
ಧೀರ-ರೆಂದು
ಧೀರ-ವಾಗಿ
ಧೀರವೂ
ಧೀರ-ಸತ್ಯ-ವನ್ನು
ಧೀರಾಃ
ಧೀಶಕ್ತಿ
ಧುಮುಕಿ
ಧುರೀಣ
ಧುರೀಣ-ನಾಗು-ವುದು
ಧೂಳನ್ನು
ಧೂಳಿಗೆ
ಧೂಳಿನ
ಧೂಳಿ-ನಲ್ಲಿ
ಧೂಳಿ-ನಿಂದ
ಧೂಳಿಯ
ಧೂಳು
ಧೂಳೆಲ್ಲ
ಧೃತಿ
ಧೈರ್ಯ
ಧೈರ್ಯ-ಗಳನ್ನು
ಧೈರ್ಯದ
ಧೈರ್ಯ-ದಿಂದ
ಧೈರ್ಯ-ದಿಂದಲೂ
ಧೈರ್ಯ-ವಾಗಿ
ಧೈರ್ಯ-ವಿದೆ
ಧೈರ್ಯ-ವಿಲ್ಲ
ಧೈರ್ಯ-ವಿ-ಲ್ಲದೇ
ಧೈರ್ಯ-ಶಾಲಿ
ಧ್ಯಾನ
ಧ್ಯಾ-ನಕ್ಕೆ
ಧ್ಯಾ-ನದ
ಧ್ಯಾ-ನ-ದಲ್ಲಿ
ಧ್ಯಾ-ನ-ದಲ್ಲಿ-ರು-ವರೋ
ಧ್ಯಾ-ನ-ದಿಂದ
ಧ್ಯಾ-ನ-ಪರ-ವಶ-ರಾಗಲು
ಧ್ಯಾ-ನ-ಪೂರ್ವಕ
ಧ್ಯಾ-ನ-ಮಗ್ನ-ನಾಗು-ವನೆ
ಧ್ಯಾ-ನ-ಮಾಡಿ
ಧ್ಯಾ-ನಿ-ಸು-ತ್ತಿದ್ದರೆ
ಧ್ಯಾ-ಸನ
ಧ್ಯೇಯ
ಧ್ಯೇಯ-ಗಳಿಗೆ
ಧ್ಯೇಯ-ವನ್ನಾಗಿ
ಧ್ಯೇಯ-ವಾಕ್ಯ
ಧ್ಯೇಯ-ವಿದೆ
ಧ್ಯೇಯೋ-ದ್ದೇಶ
ಧ್ಯೇಯೋ-ದ್ದೇಶ-ವನ್ನು
ಧ್ರುವ
ಧ್ರುವಾ-ಸ್ಮೃತಿಃ
ಧ್ವಂಸ
ಧ್ವಂಸದ
ಧ್ವಂಸ-ಮಾಡಲು
ಧ್ವಂಸ-ಮಾಡಿ
ಧ್ವಂಸ-ಮಾಡಿದ
ಧ್ವಂಸ-ಮಾಡಿ-ದರು
ಧ್ವಂಸ-ಮಾಡು-ವು-ದಲ್ಲ
ಧ್ವಂಸ-ಮಾರ್ಗ
ಧ್ವಜ
ಧ್ವಜ-ವನ್ನು
ಧ್ವನಿ
ಧ್ವನಿ-ಗಳ
ಧ್ವನಿ-ಯನ್ನು
ಧ್ವನಿ-ಯಿಂದ
ಧ್ವನಿಯು
ಧ್ವನಿಯೂ
ಧ್ವನಿಯೆ
ಧ್ವನಿಯೇ
ಧ್ವನಿ-ಯೊಂದು
ನ
ನಂತರ
ನಂತ-ರದ
ನಂತೆ
ನಂದ-ರಿಗೆ
ನಂಬದ
ನಂಬದೆ
ನಂಬದೇ
ನಂಬ-ಬಹು-ದಾ-ದರೆ
ನಂಬ-ಬೇ-ಕಾದ
ನಂಬ-ಬೇಕು
ನಂಬ-ಬೇಡಿ
ನಂಬ-ಲಾರಿರಿ
ನಂಬ-ಲಾರೆ
ನಂಬಲೇ-ಬೇಕು
ನಂಬಿ
ನಂಬಿಕೆ
ನಂಬಿ-ಕೆ-ಗಳ
ನಂಬಿ-ಕೆ-ಗಳನ್ನು
ನಂಬಿ-ಕೆ-ಗಳೆಲ್ಲಾ
ನಂಬಿ-ಕೆ-ಯನ್ನಿಟ್ಟಿ-ದ್ದಾರೆ
ನಂಬಿ-ಕೆ-ಯನ್ನಿಡಿ
ನಂಬಿ-ಕೆ-ಯನ್ನು
ನಂಬಿ-ಕೆ-ಯಾಗಲಿ
ನಂಬಿ-ಕೆ-ಯಾಗಿದೆ
ನಂಬಿ-ಕೆ-ಯಿಟ್ಟರೂ
ನಂಬಿ-ಕೆ-ಯಿ-ಡದ-ವ-ರನ್ನು
ನಂಬಿ-ಕೆ-ಯಿಡಿ
ನಂಬಿ-ಕೆ-ಯಿದೆ
ನಂಬಿ-ಕೆ-ಯಿ-ರಲಿ
ನಂಬಿ-ಕೆ-ಯಿಲ್ಲ
ನಂಬಿ-ಕೆಯು
ನಂಬಿ-ಕೆ-ಯುಂಟು
ನಂಬಿ-ಕೆಯೂ
ನಂಬಿ-ದಂತೆ
ನಂಬಿ-ದನು
ನಂಬಿ-ದರು
ನಂಬಿ-ದರೂ
ನಂಬಿ-ದರೆ
ನಂಬಿ-ದ್ದರು
ನಂಬಿ-ದ್ದರೆ
ನಂಬಿ-ದ್ದವೊ
ನಂಬಿ-ದ್ದೇನೆ
ನಂಬಿ-ದ್ದೇವೆ
ನಂಬಿ-ರುತ್ತಾರೆ
ನಂಬಿ-ರುತ್ತೇವೆ
ನಂಬಿ-ರುವ
ನಂಬಿ-ರು-ವರು
ನಂಬಿ-ರು-ವಾಗ
ನಂಬಿ-ರು-ವುದು
ನಂಬಿ-ಸ-ಬಹುದು
ನಂಬಿ-ಸು-ವು-ದಕ್ಕೆ
ನಂಬುತ್ತಾ-ನೆಯೊ
ನಂಬುತ್ತಾರೆ
ನಂಬುತ್ತಾರೊ
ನಂಬುತ್ತೀರಾ
ನಂಬು-ತ್ತೀರಿ
ನಂಬು-ತ್ತೇನೆ
ನಂಬುತ್ತೇವೆ
ನಂಬುದಿ-ಲ್ಲವೋ
ನಂಬುವ
ನಂಬು-ವಂತಹ
ನಂಬು-ವಂತೆ
ನಂಬು-ವರು
ನಂಬು-ವಿರೋ
ನಂಬು-ವು-ದಕ್ಕೆ
ನಂಬು-ವು-ದ-ರಿಂದ
ನಂಬು-ವು-ದಿಲ್ಲ
ನಂಬು-ವು-ದಿಲ್ಲವೊ
ನಂಬು-ವು-ದಿಲ್ಲವೋ
ನಂಬು-ವುದು
ನಂಬು-ವುದೂ
ನಂಬು-ವುದೇ
ನಂಬು-ವೆನು
ನಂಬು-ವೆವು
ನಂಬೋಣ
ನಕ್ಕನು
ನಕ್ಕು
ನಕ್ಷತ್ರ
ನಕ್ಷತ್ರ-ಗಳನ್ನು
ನಕ್ಷತ್ರ-ಗಳು
ನಕ್ಷತ್ರ-ಗಳೂ
ನಕ್ಷತ್ರ-ವನ್ನು
ನಗ-ಬಹುದು
ನಗ-ರಕ್ಕೂ
ನಗ-ರಕ್ಕೆ
ನಗರ-ಗಳ
ನಗರ-ಗಳಲ್ಲೂ
ನಗರ-ಗಳಿಂದಲೂ
ನಗರ-ಗಳಿ-ಗಿಂತಲೂ
ನಗ-ರದ
ನಗ-ರದಲ್ಲಿ
ನಗ-ರದಲ್ಲೇ
ನಗರ-ದಿಂದ
ನಗರ-ವನ್ನು
ನಗರ-ವಾದ
ನಗ-ರವು
ನಗರಿ-ಯಾದ
ನಗು
ನಗು-ವು-ದಿಲ್ಲ
ನಗು-ವೆನು
ನಗು-ವೆವು
ನಗ್ನ-ನಾಗಿ
ನಚಿಕೇತ
ನಚಿಕೇ-ತನ
ನಚಿಕೇತ-ನಿಗಿದ್ದಂತಹ
ನಚಿಕೇತ-ನಿಗೆ
ನಚಿಕೇತ-ನೆಂಬ
ನಜರ-ತ್ತಿನ
ನಟಿಸಿ
ನಡತೆ
ನಡತೆ-ಯನ್ನು
ನಡ-ವ-ಳಿಕೆ-ಗಳು
ನಡುಗಿ-ಸಿತು
ನಡುಗು-ವು-ದನ್ನು
ನಡು-ವಿನ
ನಡುವೆ
ನಡು-ವೆಯೂ
ನಡು-ವೆಯೇ
ನಡೆದ
ನಡೆ-ದರೂ
ನಡೆ-ದರೆ
ನಡೆ-ದಿದೆ
ನಡೆದಿದ್ದರೂ
ನಡೆದಿ-ದ್ದವು
ನಡೆದಿರ-ಬಹು-ದೆಂದು
ನಡೆದಿ-ರುವ
ನಡೆ-ದಿವೆ
ನಡೆದು
ನಡೆದು-ಕೊಂಡು
ನಡೆದು-ಕೊಳ್ಳ-ಬೇಕಾಗಿದೆ
ನಡೆದು-ಕೊಳ್ಳು-ವಂತೆಯೂ
ನಡೆದು-ಕೊಳ್ಳು-ವುದು
ನಡೆಯ-ಬೇಕಾಗಿದೆ
ನಡೆಯ-ಬೇಕು
ನಡೆ-ಯ-ಲಿಲ್ಲ
ನಡೆ-ಯಲು
ನಡೆ-ಯಿತು
ನಡೆ-ಯಿತೆಂದೂ
ನಡೆಯು-ತ್ತದೆ
ನಡೆಯು-ತ್ತದೆಯೋ
ನಡೆ-ಯು-ತ್ತವೆ
ನಡೆ-ಯು-ತ್ತಿದೆ
ನಡೆ-ಯು-ತ್ತಿದ್ದರು
ನಡೆ-ಯು-ತ್ತಿದ್ದರೆ
ನಡೆಯುತ್ತಿರು-ವಂತಹ
ನಡೆಯುತ್ತಿ-ರುವ-ವ-ನಿಗೆ
ನಡೆಯು-ತ್ತಿವೆ
ನಡೆಯು-ತ್ತೀಯೆ
ನಡೆ-ಯುವ
ನಡೆ-ಯು-ವಂತೆ
ನಡೆಯು-ವಾಗ
ನಡೆ-ಯು-ವು-ದಕ್ಕೆ
ನಡೆಯು-ವು-ದನ್ನು
ನಡೆ-ಸಲು
ನಡೆಸಿ
ನಡೆಸಿ-ಕೊಂಡು
ನಡೆ-ಸಿದ
ನಡೆಸಿ-ದರು
ನಡೆಸಿ-ರ-ಬಹುದು
ನಡೆ-ಸುತ್ತ
ನಡೆ-ಸುತ್ತದೆ
ನಡೆ-ಸು-ತ್ತಿದ್ದರು
ನಡೆಸುತ್ತಿ-ರುವ
ನಡೆ-ಸು-ವಂತೆಯೂ
ನಡೆ-ಸು-ವರು
ನಡೆ-ಸುವ-ವನು
ನಡೆಸುವು-ದಾಗಿ-ತ್ತು
ನದಿ
ನದಿ-ಗಳಲ್ಲಿ
ನದಿ-ಗಳ-ಲ್ಲೆಲ್ಲಾ
ನದಿ-ಗಳಾದರೆ
ನದಿ-ಗಳಿವೆ
ನದಿ-ಗಳು
ನದಿಗೆ
ನದಿಯ
ನದಿ-ಯಂತೆ
ನದಿ-ಯನ್ನು
ನದಿ-ಯಾ-ದಲ್ಲಿ
ನದಿ-ಯಿಂದಾಚೆ
ನದಿಯು
ನದಿ-ಯೆಂದೂ
ನನ-ಗಲ್ಲ
ನನ-ಗಾಗಿ
ನನ-ಗಾದ
ನನ-ಗಿಂತ
ನನ-ಗಿಂದು
ನನ-ಗಿರು-ತ್ತದೆ
ನನ-ಗಿ-ರುವ
ನನ-ಗೀಗ
ನನಗೂ
ನನಗೆ
ನನ-ಗೆಷ್ಟು
ನನ-ಗೇಕೆ
ನನ-ಗೇನು
ನನ-ಗೇನೋ
ನನ-ಗೊಂದು
ನನ-ಗೊಬ್ಬರು
ನನ್ನ
ನನ್ನಂತಹ
ನನ್ನಂತೆ
ನನ್ನ-ದಲ್ಲ
ನನ್ನ-ದಾಗಿ-ತ್ತು
ನನ್ನದು
ನನ್ನದೆ
ನನ್ನದೇ
ನನ್ನ-ನ್ನಾರು
ನನ್ನನ್ನು
ನನ್ನಲ್ಲಿ
ನನ್ನ-ಲ್ಲಿದೆ
ನನ್ನ-ಲ್ಲಿದ್ದ
ನನ್ನ-ಲ್ಲಿಯೂ
ನನ್ನಲ್ಲಿ-ರುವ
ನನ್ನಲ್ಲಿ-ರು-ವಂತೆಯೇ
ನನ್ನಲ್ಲಿ-ರು-ವುದು
ನನ್ನವೇ
ನನ್ನಷ್ಟು
ನನ್ನಿಂದ
ನನ್ನೊಂದಿಗೆ
ನಮಃ
ನಮ-ಗಲ್ಲ
ನಮಗಲ್ಲಿ
ನಮ-ಗಾಗಿ
ನಮಗಾ-ಗುವ
ನಮ-ಗಿಂತಲೂ
ನಮ-ಗಿಂದು
ನಮಗಿತ್ತ
ನಮ-ಗಿದೆ
ನಮಗಿ-ರುವ
ನಮಗಿ-ರು-ವು-ದಿಲ್ಲ
ನಮಗೀಗ
ನಮಗೂ
ನಮಗೆ
ನಮ-ಗೆಲ್ಲ
ನಮಗೆ-ಲ್ಲ-ರಿಗೂ
ನಮ-ಗೆಲ್ಲಾ
ನಮಗೇಕೆ
ನಮಗೊದ-ಗು-ತ್ತದೆ
ನಮಸ್ಕಾರ
ನಮಸ್ತಸ್ಯೈ
ನಮೋ
ನಮ್ಮ
ನಮ್ಮಂತೆ
ನಮ್ಮಂತೆಯೇ
ನಮ್ಮ-ದ-ನ್ನಾಗಿ
ನಮ್ಮ-ದಲ್ಲ
ನಮ್ಮ-ದಾಗ-ಬೇಕು
ನಮ್ಮ-ದಾ-ಗಿದೆ
ನಮ್ಮ-ದಾ-ಯಿತು
ನಮ್ಮದು
ನಮ್ಮದೇ
ನಮ್ಮ-ದೊಂದೇ
ನಮ್ಮ-ನಮ್ಮಲ್ಲಿ
ನಮ್ಮನ್ನು
ನಮ್ಮ-ಬಾ-ಳಿ-ನ-ಚಾಳಿ
ನಮ್ಮಲ್ಲಿ
ನಮ್ಮ-ಲ್ಲಿಗೆ
ನಮ್ಮ-ಲ್ಲಿದೆ
ನಮ್ಮ-ಲ್ಲಿನ
ನಮ್ಮ-ಲ್ಲಿಯೂ
ನಮ್ಮ-ಲ್ಲಿಯೇ
ನಮ್ಮ-ಲ್ಲಿ-ರ-ದಂತೆ
ನಮ್ಮ-ಲ್ಲಿ-ರು-ತ್ತದೆ
ನಮ್ಮ-ಲ್ಲಿ-ರುವ
ನಮ್ಮ-ಲ್ಲಿ-ರು-ವುದೋ
ನಮ್ಮ-ಲ್ಲಿಲ್ಲ
ನಮ್ಮ-ಲ್ಲಿವೆ
ನಮ್ಮಲ್ಲೇ
ನಮ್ಮ-ವರ
ನಮ್ಮ-ವ-ರನ್ನು
ನಮ್ಮ-ವ-ರಿಗೆ
ನಮ್ಮ-ವರು
ನಮ್ಮ-ವರೂ
ನಮ್ಮ-ವರೇ
ನಮ್ಮ-ವಾದ-ವನ್ನು
ನಮ್ಮಷ್ಟು
ನಮ್ಮಾ-ತ್ಮಗಳಲ್ಲಿ
ನಮ್ಮಿಂದ
ನಮ್ಮಿಂದಲೇ
ನಮ್ಮಿಬ್ಬ-ರಿಗೂ
ನಮ್ಮೀ
ನಮ್ಮೆಲ್ಲ-ರನ್ನೂ
ನಮ್ಮೆಲ್ಲ-ರಿಗೂ
ನಮ್ಮೊಂದಿ-ಗಿರು-ವಾಗ
ನಮ್ಮೊಂದಿಗೆ
ನಮ್ಮೊ-ಡ-ನೆಯೇ
ನಮ್ಮೊಳಗೆ
ನಯ
ನಯ-ವಾಗಿ
ನಯ-ವಿ-ಲ್ಲದೆ
ನರಕ
ನರಕಕ್ಕೆ
ನರಕ-ಗಳಿವೆ
ನರಕ-ದಲ್ಲಿ
ನರಕ-ಭಯ
ನರ-ಕವೇ
ನರಕಾಗ್ನಿ-ಯಿಂದ
ನರ-ಗಳನ್ನು
ನರ-ಗಳಲ್ಲಿ
ನರ-ಗಳು
ನರ-ನಾಡಿ-ಗಳಲ್ಲಿ
ನರ-ನಾರಿ-ಯರ
ನರನಾರಿಯ-ರಿಗೆ
ನರ-ನಿಗೂ
ನರ-ಮಾಂಸ
ನರಳಿ
ನರಳು-ತ್ತಿದ್ದರು
ನರಳುತ್ತಿ-ರುವ
ನರಳುತ್ತಿ-ರು-ವರು
ನರಳುತ್ತಿರು-ವೆವು
ನರಳು-ವಂತೆ
ನರ-ಳು-ವನು
ನಲ್ಲಿ
ನಲ್ಲಿಯೂ
ನವ
ನವ-ಚೇ-ತನ-ವನ್ನು
ನವ-ಜಾಗೃತಿ
ನವ-ಜೀವ-ನ-ವನ್ನು
ನವ-ತಾರುಣ್ಯದ
ನವ-ನವ
ನವ-ಪಂಥದ
ನವ-ಪ್ರ-ಯತ್ನಕ್ಕೆ
ನವ-ಭಾರತದ
ನವ-ಯುಗ-ದಲ್ಲಿ
ನವ-ಯುಗಾವ-ತಾರ
ನವಾಗಿ-ರು-ವುದು
ನವೀನ
ನವೀನತಾ
ನವೆಂಬರ್
ನವೇ
ನವೋ-ತ್ಸಾಹದ
ನಶಿಸಲೇ-ಬೇಕು
ನಶಿಸಿ
ನಶಿಸಿ-ದರೂ
ನಶಿಸಿ-ಹೋ-ಗು-ತ್ತಿವೆ
ನಶಿಸಿ-ಹೋ-ದವು
ನಶಿ-ಸು-ವರು
ನಶ್ವರ
ನಷ್ಟ
ನಷ್ಟ-ವಾಗ-ಬಹುದು
ನಷ್ಟ-ವಾಗಿ
ನಷ್ಟ-ವಾಗು
ನಷ್ಟ-ವಾದರೂ
ನಷ್ಟ-ವಾದರೆ
ನಷ್ಟ-ವೇನೂ
ನಸ್ತೇ-ಽಧರಾಮೃ-ತಮ್
ನಹುಷ
ನಾಗರ-ಹಾವು
ನಾಗ-ರಿಕ
ನಾಗರಿ-ಕತೆ
ನಾಗ-ರಿಕ-ತೆ-ಗಳನ್ನು
ನಾಗ-ರಿಕ-ತೆ-ಗಳಿ-ಗಿ-ರುವ
ನಾಗ-ರಿಕ-ತೆ-ಗಳು
ನಾಗ-ರಿಕ-ತೆ-ಗಳೆ-ಲ್ಲವೂ
ನಾಗ-ರಿಕ-ತೆಗೆ
ನಾಗ-ರಿಕ-ತೆಯ
ನಾಗ-ರಿಕ-ತೆ-ಯಂತೆ
ನಾಗ-ರಿಕ-ತೆ-ಯನ್ನು
ನಾಗ-ರಿಕ-ತೆ-ಯ-ನ್ನುಳ್ಳ
ನಾಗ-ರಿಕ-ತೆ-ಯಲ್ಲ
ನಾಗ-ರಿಕ-ತೆ-ಯಲ್ಲಿ
ನಾಗ-ರಿಕ-ತೆ-ಯ-ಲ್ಲಿ-ರುವ
ನಾಗ-ರಿಕ-ತೆಯು
ನಾಗ-ರಿಕ-ತೆಯೂ
ನಾಗ-ರಿಕರು
ನಾಗರೀ-ಕತೆ-ಗಳೆಲ್ಲ
ನಾಗರೀಕ-ತೆಗೆ
ನಾಗಲೀ
ನಾಗಿದ್ದ
ನಾಗಿದ್ದನು
ನಾಗಿ-ರ-ಬೇಕು
ನಾಗಿ-ರಲೇ-ಬೇಕು
ನಾಚಿಕೆಗೇಡು
ನಾಚಿಕೆ-ಯಾಗು-ವು-ದಿಲ್ಲ-ವೆ-ಕೆಲವು
ನಾಚಿಕೆ-ಯಾಗು-ವು-ದಿಲ್ಲವೇ
ನಾಚಿಕೊಳ್ಳಬೇಕಾಗು-ವುದು
ನಾಜುಕು-ಗಳು
ನಾಜೂಕಾಗಿ
ನಾಟಕ-ಗಳೊಂದಿಗೆ
ನಾಡಾದ
ನಾಡಿ
ನಾಡಿ-ಗಳನ್ನೂ
ನಾಡಿ-ಗಳಲ್ಲಿ
ನಾಡಿಗೆ
ನಾಡಿನ
ನಾಡಿ-ನಲ್ಲಿ
ನಾಡಿ-ನ-ಲ್ಲಿಯೇ
ನಾಡಿ-ನಿಂದ
ನಾಡಿ-ಯನ್ನು
ನಾಡಿ-ಯಲ್ಲಿ
ನಾಡು
ನಾಡು-ವುದು
ನಾಡೇ
ನಾಣ್ಣುಡಿ-ಯಂತೆ
ನಾಣ್ನುಡಿ-ಯನ್ನು
ನಾಣ್ನುಡಿ-ಯಾ-ಯಿತು
ನಾಣ್ಯದ
ನಾನ-ಕನು
ನಾನಕ್
ನಾನಾ
ನಾನಾ-ಗಲೇ
ನಾನಾ-ಡಿದ
ನಾನಾರು
ನಾನಿಂದು
ನಾನಿನ್ನು
ನಾನಿನ್ನೂ
ನಾನಿರು-ವೆನು
ನಾನಿಲ್ಲಿ
ನಾನೀಗ
ನಾನು
ನಾನು-ಭಾವಿಸು
ನಾನೂ
ನಾನೃತಂ
ನಾನೆಲ್ಲೋ
ನಾನೆಷ್ಟೋ
ನಾನೇ
ನಾನೇಕೆ
ನಾನೇ-ನಾಗು-ತ್ತೇನೆ
ನಾನೇ-ನಾ-ದರೂ
ನಾನೇನು
ನಾನೊಂದು
ನಾನೊಬ್ಬ
ನಾನ್ಯಾ-ನಪಿ
ನಾಮ
ನಾಮ-ಕರಣ
ನಾಮ-ಗಳಿಂದ
ನಾಮ-ಗಳು
ನಾಮ-ಧೇಯ
ನಾಮ-ರೂಪ-ಗಳನ್ನು
ನಾಮ-ರೂಪ-ಗಳಾಗಿ
ನಾಮ-ರೂಪ-ಗಳಿಂದ
ನಾಮ-ರೂಪ-ಗಳು
ನಾಮ-ರೂಪ-ಗಳೆಲ್ಲ
ನಾಮ-ರೂಪ-ಮಯ-ವಾದ
ನಾಮೋಚ್ಛಾ-ರಣೆ-ಯನ್ನು
ನಾಯಕ
ನಾಯಕ-ರಿ-ರುವ
ನಾಯ-ಕರು
ನಾಯಕ-ಳಾದರೆ
ನಾಯಮಸ್ತೀತಿ
ನಾಯ-ಮಾ-ತ್ಮಾ
ನಾಯಿ
ನಾಯಿ-ಯಂತಹ
ನಾಯಿ-ಯಂತೆ
ನಾಯಿ-ಯೊಂದು
ನಾರದ
ನಾರಾ-ಯಣ
ನಾರಾ-ಯಣ-ರನ್ನು
ನಾರಿ-ಯರ
ನಾರಿ-ಯರನ್ನು
ನಾರಿ-ಯರು
ನಾರೀ-ರತ್ನ-ವನ್ನು
ನಾರು-ಬಟ್ಟೆ-ಯ-ನ್ನುಟ್ಟು
ನಾರುವ
ನಾಲಗೆ-ಯನ್ನು
ನಾಲ್ಕ-ನೆಯ
ನಾಲ್ಕ-ನೆ-ಯದೇ
ನಾಲ್ಕು
ನಾಲ್ಕೂ
ನಾಲ್ಕೈದು
ನಾಳೆ
ನಾಳೆಯೇ
ನಾಳೆಯೋ
ನಾವ-ದನ್ನು
ನಾವರಿ-ಯಲು
ನಾವವತು
ನಾವಾಗಲೇ
ನಾವಾದರೋ
ನಾವಿಂದು
ನಾವಿ-ದನ್ನು
ನಾವಿನ್ನೂ
ನಾವಿಬ್ಬರೂ
ನಾವಿಲ್ಲಿ
ನಾವೀಗ
ನಾವು
ನಾವು-ಗಳೆಲ್ಲ
ನಾವು-ಗಳೆ-ಲ್ಲರೂ
ನಾವು-ಗಳೆಲ್ಲಾ
ನಾವು-ಮಾಡು-ವು-ದಿಲ್ಲ
ನಾವು-ಸ್ವೀಕರಿಸುತ್ತೇವೆ
ನಾವು-ಹೋ-ರಾಡ-ಬೇಕಾಗಿಲ್ಲ
ನಾವೂ
ನಾವೆಂದಿಗೂ
ನಾವೆಂದು
ನಾವೆಂದೂ
ನಾವೆಲ್ಲ
ನಾವೆ-ಲ್ಲರೂ
ನಾವೆಲ್ಲಾ
ನಾವೆಷ್ಟು
ನಾವೇ
ನಾವೇಕೆ
ನಾವೇ-ನಾ-ದರೂ
ನಾವೊಂದು
ನಾಶ
ನಾಶ-ಕರ
ನಾಶ-ಕರ-ವಲ್ಲ
ನಾಶ-ಕಾರಕ
ನಾಶಕ್ಕೆ
ನಾಶ-ಗೊಳಿ-ಸಲು
ನಾಶ-ಗೊಳಿಸು-ವು-ದಲ್ಲ
ನಾಶ-ಮಾಡದೆ
ನಾಶ-ಮಾಡ-ಬೇಕು
ನಾಶ-ಮಾಡ-ಬೇಕೆಂದು
ನಾಶ-ಮಾಡ-ಲಾ-ರದು
ನಾಶ-ಮಾಡಿ
ನಾಶ-ಮಾಡಿ-ದರು
ನಾಶ-ಮಾಡಿ-ದರೂ
ನಾಶ-ಮಾಡು-ತ್ತಿ-ರುವ
ನಾಶ-ಮಾಡುವ
ನಾಶ-ಮಾಡು-ವ-ವನೂ
ನಾಶ-ಮಾಡು-ವುದು
ನಾಶ-ವಾಗದ
ನಾಶ-ವಾಗದೆ
ನಾಶ-ವಾಗ-ಬೇಕು
ನಾಶ-ವಾಗಲಿ
ನಾಶ-ವಾಗಲೇ-ಬೇಕು
ನಾಶ-ವಾಗಿ
ನಾಶ-ವಾಗಿದೆ
ನಾಶ-ವಾಗಿ-ರು-ವುದು
ನಾಶ-ವಾಗಿ-ರು-ವುವು
ನಾಶ-ವಾಗಿಲ್ಲ
ನಾಶ-ವಾಗು-ತ್ತದೆ
ನಾಶ-ವಾಗುತ್ತಿ-ರುವ
ನಾಶ-ವಾಗುವ
ನಾಶ-ವಾಗು-ವರು
ನಾಶ-ವಾಗು-ವಿರಿ
ನಾಶ-ವಾಗು-ವು-ದಿಲ್ಲ
ನಾಶ-ವಾಗು-ವುದು
ನಾಶ-ವಾಗು-ವು-ದೆಂದು
ನಾಶ-ವಾಗು-ವುವು
ನಾಶ-ವಾದಂತೆ
ನಾಶ-ವಾದ-ನೆಂದು
ನಾಶ-ವಾದರೂ
ನಾಶ-ವಾದರೆ
ನಾಶ-ವಾ-ಯಿತು
ನಾಶ-ವಿಲ್ಲ
ನಾಶ-ವಿಲ್ಲದ
ನಾಶವೇ
ನಾಶ-ಹೊಂದು-ತ್ತದೆ
ನಾಸದೀಯ
ನಾಸ್ತಿ
ನಾಸ್ತಿಕ
ನಾಸ್ತಿ-ಕ-ತೆಯ-ಲ್ಲಾ-ದರೂ
ನಾಸ್ತಿ-ಕನ
ನಾಸ್ತಿ-ಕ-ನಾಗ-ಬಲ್ಲ
ನಾಸ್ತಿ-ಕ-ನೆಂದೂ
ನಾಸ್ತಿ-ಕ-ರಾಗಿ-ರು-ವುದು
ನಾಸ್ತಿ-ಕ-ರಾಗು-ವುದು
ನಾಸ್ತಿ-ಕ-ರಾದರು
ನಾಸ್ತಿ-ಕರು
ನಾಸ್ತಿ-ಕ-ವೆ-ನಿಸು-ವುದು
ನಾಹಂ
ನಿಂತ
ನಿಂತ-ಕಡೆಯೇ
ನಿಂತರು
ನಿಂತರೆ
ನಿಂತ-ಲ್ಲದೆ
ನಿಂತಿದೆ
ನಿಂತಿ-ದೆಯೇ
ನಿಂತಿ-ದ್ದಾರೆ
ನಿಂತಿ-ದ್ದೇನೆ
ನಿಂತಿರ-ಬೇಕು
ನಿಂತಿರು-ತ್ತದೆ
ನಿಂತಿ-ರುವ
ನಿಂತಿ-ರು-ವರು
ನಿಂತಿ-ರು-ವರೊ
ನಿಂತಿರು-ವಾಗ
ನಿಂತಿರು-ವು-ದ-ರಿಂದ
ನಿಂತಿರು-ವು-ದಾದ
ನಿಂತಿರು-ವುದು
ನಿಂತಿರು-ವು-ದೆಂದು
ನಿಂತಿರು-ವೆನು
ನಿಂತಿಲ್ಲ
ನಿಂತಿವೆ
ನಿಂತು
ನಿಂತು-ಕೊಂಡು
ನಿಂತು-ಕೊಳ್ಳ-ಬಲ್ಲರು
ನಿಂತು-ಕೊಳ್ಳಲಿ
ನಿಂತೊ-ಡ-ನೆಯೇ
ನಿಂದಂತು
ನಿಂದ-ನೆ-ಗಳೇ
ನಿಂದ-ನೆಗೆ
ನಿಂದ-ನೆಯ
ನಿಂದಾರ್ಹನೂ
ನಿಂದಾಸ್ಪದ-ವಾದ
ನಿಂದಿಸ-ಕೂಡದು
ನಿಂದಿ-ಸದೇ
ನಿಂದಿಸ-ಬೇಡಿ
ನಿಂದಿ-ಸಲಿ
ನಿಂದಿ-ಸಲು
ನಿಂದಿಸಿ
ನಿಂದಿಸು-ತ್ತಿ-ರಲಿ
ನಿಂದಿ-ಸು-ತ್ತಿ-ರಲಿಲ್ಲ
ನಿಂದಿ-ಸು-ವರು
ನಿಂದಿಸುವಿರಾ
ನಿಂದಿ-ಸು-ವು-ದಿಲ್ಲ
ನಿಂದೆ
ನಿಂದೆ-ಅವ-ಶ್ಯಕ
ನಿಂದೆ-ಗಿಂತ
ನಿಂದೆಯ
ನಿಃಸೃ-ತಮ್
ನಿಃಸೃತಮ್ಪ್ರಾಣ-ಸ್ಪಂದನ
ನಿಃಸ್ವಾರ್ಥ
ನಿಃಸ್ವಾರ್ಥ-ರಾಗಿ-ರು-ವುದು
ನಿಃಸ್ವಾರ್ಥವೂ
ನಿಃಸ್ವಾರ್ಥವೇ
ನಿಃಸ್ವಾರ್ಥಿ-ಗಳು
ನಿಃಸ್ವಾರ್ಥಿ-ಯಾಗಿ-ರ-ಬೇಕು
ನಿಃಸ್ವಾರ್ಥಿ-ಯಾಗಿರು
ನಿಕಟ
ನಿಕಟನೂ
ನಿಕಟ-ವಾಗಿ
ನಿಕೃಷ್ಟ
ನಿಕೃಷ್ಟ-ದೃಷ್ಟಿ-ಯಿಂದ
ನಿಗದಿ-ಪಡಿಸು
ನಿಗೂಢ
ನಿಗ್ರ-ಹಿ-ಸದ
ನಿಗ್ರ-ಹಿ-ಸಲು
ನಿಗ್ರ-ಹಿ-ಸಿ-ಕೊಂಡಿ-ರು-ವಿರಿ
ನಿಗ್ರ-ಹಿ-ಸು-ವು-ದಕ್ಕೆ
ನಿಜ
ನಿಜ-ವಲ್ಲ
ನಿಜ-ವಾಗಿ
ನಿಜ-ವಾಗಿ-ದ್ದರೆ
ನಿಜ-ವಾಗಿಯೂ
ನಿಜ-ವಾದ
ನಿಜ-ವಾದರೆ
ನಿಜ-ವಾ-ಯಿತು
ನಿಜವೇ
ನಿಜ-ಸ್ವ-ರೂಪಕ್ಕೆ
ನಿಜ-ಸ್ವ-ರೂಪದ
ನಿಜ-ಸ್ವ-ರೂಪ-ವನ್ನು
ನಿಜಾಂಶ
ನಿಜಾಂಶ-ವನ್ನು
ನಿಜಾಂಶ-ವಿದೆ
ನಿಟ್ಟುಸಿರು
ನಿತ್ಯ
ನಿತ್ಯಃ
ನಿತ್ಯ-ಜೀವ-ನ-ದಲ್ಲಿ
ನಿತ್ಯದ
ನಿತ್ಯ-ವಾಗು-ವುದು
ನಿತ್ಯ-ವಾದು-ದ-ರಿಂದ
ನಿತ್ಯ-ವಾದುದು
ನಿತ್ಯವು
ನಿತ್ಯ-ಸತ್ಯ-ಗಳು
ನಿತ್ಯ-ಸಾಧ್ವಿ
ನಿತ್ರಾಣ-ನಾಗು-ವನು
ನಿದರ್ಶನ
ನಿದರ್ಶನಕ್ಕೆ
ನಿದರ್ಶನ-ಗಳು
ನಿದರ್ಶನ-ವನ್ನು
ನಿದರ್ಶನ-ವಾಗಿ
ನಿದರ್ಶಿ-ಸು-ವು-ದಕ್ಕೆ
ನಿದಿ
ನಿದಿ-ಧ್ಯಾಸಿ
ನಿದ್ದೆ
ನಿದ್ದೆ-ಯನ್ನು
ನಿದ್ದೆ-ಯಿಂದ
ನಿದ್ರಿಸುತ್ತಾರೆ
ನಿದ್ರಿಸುತ್ತಿ-ರುವ
ನಿದ್ರಿಸುತ್ತಿ-ರುವ-ನೆಂದು
ನಿದ್ರಿಸುತ್ತಿ-ರು-ವರು
ನಿದ್ರಿಸುತ್ತಿ-ರುವಳು
ನಿದ್ರಿಸುತ್ತಿರು-ವುದು
ನಿದ್ರಿ-ಸು-ವು-ದಿಲ್ಲ
ನಿದ್ರೆಗೆ
ನಿದ್ರೆ-ಯನ್ನು
ನಿದ್ರೆ-ಯಿಂದ
ನಿಧಾನ-ವಾಗಿ
ನಿಧಾನ-ವಾಗಿಯೋ
ನಿಧಾನ-ವಾದ
ನಿಧಿ
ನಿಧಿಗೆ
ನಿಧಿ-ಯಿಂದ
ನಿಧಿ-ಸಲು
ನಿನ-ಗಲ್ಲ
ನಿನ-ಗಿದೋ
ನಿನ-ಗಿವೆ
ನಿನಗೂ
ನಿನಗೆ
ನಿನ-ಗೇನೂ
ನಿನ್ನ
ನಿನ್ನಂತೆಯೇ
ನಿನ್ನ-ದಲ್ಲ
ನಿನ್ನ-ದಾಗದೇ
ನಿನ್ನದು
ನಿನ್ನನ್ನು
ನಿನ್ನ-ನ್ನೇ
ನಿನ್ನಲ್ಲಿ
ನಿನ್ನ-ಲ್ಲಿ-ರು-ವಂತೆಯೇ
ನಿನ್ನಿಂದ
ನಿನ್ನೆ
ನಿನ್ನೆ-ಡೆಗೆ
ನಿನ್ನೆಯ
ನಿಪುಣ
ನಿಪುಣ-ರಿರ-ಬಹುದು
ನಿಪುಣರು
ನಿಮಗಂತೂ
ನಿಮಗಾ-ಗಲೇ
ನಿಮ-ಗಾಗಿ
ನಿಮ-ಗಿಂತ
ನಿಮ-ಗಿಂತಲೂ
ನಿಮ-ಗಿದ್ದ
ನಿಮಗಿಲ್ಲಿ
ನಿಮಗೀಗ
ನಿಮಗೂ
ನಿಮಗೆ
ನಿಮಗೆ-ಕಾಣು
ನಿಮಗೆಲ್ಲ
ನಿಮಗೆಲ್ಲಾ
ನಿಮಗೆ-ಲ್ಲಿಯ
ನಿಮಗೇ
ನಿಮಗೊಂದು
ನಿಮಗ್ನೋ-ಽನೀಶಯಾ
ನಿಮಿತ್ತ
ನಿಮಿತ್ತ-ಕಾರಣ
ನಿಮಿತ್ತ-ಗಳು
ನಿಮಿತ್ತ-ಗಳೆಂಬ
ನಿಮಿತ್ತದ
ನಿಮಿತ್ತ-ದೋಷ
ನಿಮಿತ್ತ-ದೋಷ-ಆ-ಹಾ-ರವು
ನಿಮಿತ್ತ-ಮ-ಪ್ರ-ಯೋ-ಜಕಂ
ನಿಮಿಷ
ನಿಮಿಷ-ದಲ್ಲಿ
ನಿಮ್ಮ
ನಿಮ್ಮಂತೆ
ನಿಮ್ಮಂತೆಯೇ
ನಿಮ್ಮ-ದಲ್ಲ
ನಿಮ್ಮದು
ನಿಮ್ಮದೆ
ನಿಮ್ಮದೇ
ನಿಮ್ಮನ್ನು
ನಿಮ್ಮನ್ನೂ
ನಿಮ್ಮ-ನ್ನೆಲ್ಲ
ನಿಮ್ಮ-ನ್ನೇ
ನಿಮ್ಮ-ಲ್ಲ-ನೇ-ಕ-ರಿಗೆ
ನಿಮ್ಮಲ್ಲಿ
ನಿಮ್ಮ-ಲ್ಲಿಗೆ
ನಿಮ್ಮ-ಲ್ಲಿದೆ
ನಿಮ್ಮ-ಲ್ಲಿ-ದೆಯೆ
ನಿಮ್ಮ-ಲ್ಲಿ-ರಲಿ
ನಿಮ್ಮ-ಲ್ಲಿ-ರುವ
ನಿಮ್ಮ-ಲ್ಲಿ-ರು-ವುದು
ನಿಮ್ಮ-ಲ್ಲಿ-ರು-ವುದೆಲ್ಲ
ನಿಮ್ಮಲ್ಲೇ
ನಿಮ್ಮ-ವನ್ನಾಗಿ
ನಿಮ್ಮಷ್ಟು
ನಿಮ್ಮ-ಸ್ನೇ-ಹಿ-ತರು
ನಿಮ್ಮಿಂದ
ನಿಮ್ಮಿಂದಲೂ
ನಿಮ್ಮೆದು-ರಿಗೆ
ನಿಮ್ಮೆದುರು
ನಿಮ್ಮೆಲ್ಲ
ನಿಮ್ಮೊಂದಿಗೆ
ನಿಮ್ಮೊಡನೆ
ನಿಮ್ಮೊಳ-ಗೆಯೇ
ನಿಯಂತ್ರ-ಣ-ದಲ್ಲಿ-ದ್ದೀರಿ
ನಿಯಂತ್ರಿಸು-ವುದು
ನಿಯತಿ
ನಿಯಮ
ನಿಯಮಕ್ಕೂ
ನಿಯಮಕ್ಕೆ
ನಿಯಮ-ಗಳ
ನಿಯಮ-ಗಳನ್ನು
ನಿಯಮ-ಗಳನ್ನೂ
ನಿಯಮ-ಗಳ-ನ್ನೆಲ್ಲ
ನಿಯಮ-ಗಳಿ-ಗಿಂತ
ನಿಯಮ-ಗಳಿಗೆ
ನಿಯಮ-ಗಳು
ನಿಯಮ-ಗಳೆಲ್ಲ
ನಿಯಮ-ಗಳೇ
ನಿಯಮದ
ನಿಯಮ-ದಂತೆ
ನಿಯಮ-ಬದ್ಧ
ನಿಯಮ-ವನ್ನು
ನಿಯಮವು
ನಿಯಮವೂ
ನಿಯಮವೆ
ನಿಯಮ-ವೆಂದರೆ
ನಿಯಮಾ
ನಿಯಮಾ-ನು-ಸಾರ
ನಿಯಮಾ-ವಳಿ-ಗಳನ್ನು
ನಿಯಮಾ-ವಳಿ-ಗಳು
ನಿಯಮಾ-ವಳಿಯ
ನಿಯಮಾ-ವಳಿ-ಯನ್ನೂ
ನಿಯಮಾ-ವಳಿ-ಗಳಿಂದ
ನಿಯೋ-ಪ್ಲೇಯೊನಿ-ಸ್ಟರು
ನಿರಂಕುಶ
ನಿರಂತರ
ನಿರಂತರ-ವಾಗಿ
ನಿರಂತರ-ವಾದ
ನಿರಂತ-ರವೂ
ನಿರಂತರ-ವೆಂದೂ
ನಿರಂತ-ವಾಗಿ
ನಿರಕ್ಷರ
ನಿರತ-ರಾಗ-ಬೇಕು
ನಿರತ-ರಾಗಿ
ನಿರತ-ರಾಗಿ-ದ್ದರು
ನಿರತ-ರಾ-ಗಿದ್ದು
ನಿರತ-ರಾಗಿ-ರುವ
ನಿರತ-ರಾಗಿ-ರು-ವೆವು
ನಿರತ-ರಾ-ದಷ್ಟೂ
ನಿರತ-ವಾಗಿ
ನಿರರ್ಗಳ
ನಿರರ್ಥಕ
ನಿರರ್ಥಕ-ವೆಂದು
ನಿರಾ
ನಿರಾ-ಕರಿಸ-ಬಹುದು
ನಿರಾ-ಕ-ರಿಸಿ
ನಿರಾ-ಕ-ರಿಸಿ-ದನು
ನಿರಾ-ಕ-ರಿಸಿ-ದಾಗ
ನಿರಾ-ಕ-ರಿಸಿ-ದುದು
ನಿರಾ-ಕರಿ-ಸು-ವು-ದಿಲ್ಲ
ನಿರಾ-ಕಾರ
ನಿರಾ-ಕಾರ-ನೆಂದು
ನಿರಾ-ಕಾರ-ನೆಂದೂ
ನಿರಾ-ಕಾರ-ವಾಗಿ
ನಿರಾ-ಕಾ-ರವೂ
ನಿರಾ-ಧಾರ-ವಾದುದು
ನಿರಾ-ಭರಣ-ವಾಗಿ-ತ್ತು
ನಿರಾ-ಶನೂ
ನಿರಾ-ಶ-ರಾಗ-ಬೇಡಿ
ನಿರಾ-ಶ-ರಾಗಿ
ನಿರಾ-ಶ-ರಾಗಿ-ದ್ದಾರೆ
ನಿರಾ-ಶರಾಗುತ್ತೇ-ವೆಯೋ
ನಿರಾ-ಶರಾಗು-ವರು
ನಿರಾ-ಶರಾಗು-ವಿರಿ
ನಿರಾಶೆ
ನಿರಾ-ಶೆ-ಗೊಂಡು
ನಿರಾ-ಶೆ-ಯ-ನ್ನಲ್ಲ
ನಿರಾಸೆ
ನಿರೀಕ್ಷಣೆ-ಯನ್ನು
ನಿರೀಕ್ಷಿ
ನಿರೀಕ್ಷಿ-ಸ-ಬಲ್ಲಿರಿ
ನಿರೀಕ್ಷಿ-ಸ-ಬಹು-ದಾ-ದ-ದ್ದಲ್ಲ
ನಿರೀಕ್ಷಿ-ಸ-ಲಾರೆವು
ನಿರೀಕ್ಷಿ-ಸಿದ್ದರು
ನಿರೀಕ್ಷಿ-ಸಿಯೇ
ನಿರೀಕ್ಷಿ-ಸಿ-ರ-ಲಿಲ್ಲ
ನಿರೀಕ್ಷಿ-ಸು-ತ್ತಿದೆ
ನಿರೀಕ್ಷಿ-ಸು-ತ್ತಿ-ದ್ದೇವೆ
ನಿರೀಕ್ಷಿ-ಸುತ್ತಿ-ರುವ
ನಿರೀಕ್ಷಿ-ಸುತ್ತಿ-ರು-ವರು
ನಿರೀಕ್ಷಿ-ಸು-ವು-ದ-ರಿಂದ
ನಿರೀಶ್ವರ-ವಾದ
ನಿರೀಶ್ವರ-ವಾದದ
ನಿರುಕ್ತ
ನಿರು-ತ್ಸಾಹ
ನಿರುತ್ಸಾಹಿ
ನಿರುಪಮವೂ
ನಿರೂಪಣೆ
ನಿರೂ-ಪಿ-ಸಿರು-ವು-ದನ್ನು
ನಿರೂ-ಪಿ-ಸಿವೆ
ನಿರೂ-ಪಿ-ಸುವ
ನಿರ್ಗುಣ
ನಿರ್ಗುಣದ
ನಿರ್ಗುಣ-ದೊಂದಿಗೆ
ನಿರ್ಗುಣ-ನಾ-ಗಿದ್ದರೂ
ನಿರ್ಗುಣನೂ
ನಿರ್ಗುಣನೋ
ನಿರ್ಗುಣ-ಬ್ರಹ್ಮ
ನಿರ್ಗು-ಣವೂ
ನಿರ್ಗು-ಣವೇ
ನಿರ್ಗುಣ-ವೊಂದೇ
ನಿರ್ಗುಣ-ಸದ್ಗುಣ
ನಿರ್ಜೀವ-ನಾಗಿ
ನಿರ್ಜೀವ-ರಾಗು-ತ್ತೀರಿ
ನಿರ್ಜೀವ-ವಾದಂತಿದ್ದ
ನಿರ್ಜೀವಿ
ನಿರ್ಣ-ಯಕ್ಕೆ
ನಿರ್ಣಯ-ಗಳನ್ನು
ನಿರ್ಣಯ-ಗಳಿಂದ
ನಿರ್ಣಯ-ಗಳು
ನಿರ್ಣಯ-ಗಳೊಂದಿಗೆ
ನಿರ್ಣಯಿ-ಸಿ-ದಾಗ
ನಿರ್ಣಯಿಸಿ-ರುವ
ನಿರ್ಣಯಿಸು-ತ್ತಿವೆ
ನಿರ್ಣಯಿ-ಸುವ
ನಿರ್ಣಯಿಸು-ವುದು
ನಿರ್ದಯ
ನಿರ್ದಯ-ವಾದ
ನಿರ್ದಯೆ
ನಿರ್ದಾಕ್ಷಿಣ್ಯ-ವಾಗಿ
ನಿರ್ದಿಷ್ಟ
ನಿರ್ದಿಷ್ಟ-ವಾದ
ನಿರ್ದೇಶಿಸ-ಲಾ-ರದು
ನಿರ್ದೇಶಿ-ಸುತ್ತದೆ
ನಿರ್ದೇಶಿ-ಸುತ್ತವೆ
ನಿರ್ದೇಶಿಸು-ವುವು
ನಿರ್ದೋಷ
ನಿರ್ದೋಷಂ
ನಿರ್ದೋಷ-ವಾದುದು
ನಿರ್ದೋಷಿ
ನಿರ್ಧರಿಸ-ಬೇಕಾಗಿದೆ
ನಿರ್ಧರಿ-ಸಿದರು
ನಿರ್ಧರಿ-ಸುತ್ತದೆ
ನಿರ್ಧರಿ-ಸುವ
ನಿರ್ಧರಿಸು-ವುದು
ನಿರ್ಧಾರ
ನಿರ್ನಾಮ
ನಿರ್ನಾಮ-ವಾಗಿ
ನಿರ್ನಾಮ-ವಾಗು-ತ್ತದೆ
ನಿರ್ನಾಮ-ವಾಗು-ವುದು
ನಿರ್ಭಯತೆ
ನಿರ್ಭಯ-ವಾಗಿ
ನಿರ್ಭಾಗ್ಯರು
ನಿರ್ಭೀತ
ನಿರ್ಭೀತ-ನ-ನ್ನಾಗಿ
ನಿರ್ಭೀ-ತನು
ನಿರ್ಭೀತ-ರಾಗ-ಬೇಕು
ನಿರ್ಭೀತ-ರಾಗಿ
ನಿರ್ಭೀತಿ
ನಿರ್ಭೀತಿ-ಯಿಂದ
ನಿರ್ಮಮೇ
ನಿರ್ಮ-ಲನು
ನಿರ್ಮಾಣ
ನಿರ್ಮಾ-ಣಕ್ಕೆ
ನಿರ್ಮಾ-ಣ-ವಲ್ಲ
ನಿರ್ಮಾ-ಣ-ವಾಗ-ಬೇಕು
ನಿರ್ಮಾ-ಣ-ವಾಗು-ವುದು
ನಿರ್ಮಾ-ಣ-ವೆಂಬ
ನಿರ್ಮಿತ
ನಿರ್ಮಿತ-ವಾಗಿಲ್ಲ
ನಿರ್ಮಿತ-ವಾಗಿವೆ
ನಿರ್ಮಿತ-ವಾದುದು
ನಿರ್ಮಿತ-ವಾ-ಯಿತು
ನಿರ್ಮಿಸ-ಬಹುದು
ನಿರ್ಮಿಸ-ಬೇಕು
ನಿರ್ಮಿ-ಸಲು
ನಿರ್ಮಿ-ಸು-ತ್ತಿದ್ದ
ನಿರ್ಮಿಸು-ವುದು
ನಿರ್ಮಿಸು-ವುದೇಕೆ
ನಿರ್ಮೂಲ
ನಿರ್ಲಕ್ಷಿಸು-ವುದು
ನಿರ್ಲಕ್ಷ್ಯ-ದಿಂದ
ನಿರ್ವಂಚನೆ
ನಿರ್ವ-ಹಿ-ಸು-ವುದು
ನಿರ್ವಾಣ
ನಿರ್ವಾ-ಣ-ವನ್ನು
ನಿರ್ವಿ-ಕಾರ
ನಿರ್ವಿ-ಕಾರಿ-ಯಾಗಿದೆ
ನಿರ್ವಿ-ವಾದ-ವಾಗಿ
ನಿಲ-ಯದ
ನಿಲವು
ನಿಲು-ಕದ
ನಿಲುಕ-ದಂತೆ
ನಿಲುಕದೆ
ನಿಲುಕುವ
ನಿಲುಕು-ವಂತೆ
ನಿಲು-ವನ್ನು
ನಿಲು-ವಿನ
ನಿಲ್ಲ-ಕೂಡದು
ನಿಲ್ಲ-ದಿರಿ
ನಿಲ್ಲದೆ
ನಿಲ್ಲ-ಬೇಕು
ನಿಲ್ಲ-ಬೇಡಿ
ನಿಲ್ಲ-ಲಾ-ರರು
ನಿಲ್ಲ-ಲಾರಿರಿ
ನಿಲ್ಲ-ಲಾರೆವು
ನಿಲ್ಲ-ಲಿಲ್ಲ
ನಿಲ್ಲಲು
ನಿಲ್ಲಿ
ನಿಲ್ಲಿ-ಸ-ಬೇಕೊ
ನಿಲ್ಲಿ-ಸಲು
ನಿಲ್ಲಿಸಿ
ನಿಲ್ಲಿ-ಸಿ-ಕೊಳ್ಳಲು
ನಿಲ್ಲಿ-ಸಿ-ದರೆ
ನಿಲ್ಲಿ-ಸು-ವೆವು
ನಿಲ್ಲು
ನಿಲ್ಲು-ತ್ತದೆ
ನಿಲ್ಲು-ತ್ತಿದ್ದಾಳೆ
ನಿಲ್ಲುವ
ನಿಲ್ಲು-ವಂತಹ
ನಿಲ್ಲು-ವರು
ನಿಲ್ಲು-ವುದ-ರಲ್ಲೇ
ನಿಲ್ಲು-ವು-ದಿಲ್ಲ
ನಿಲ್ಲು-ವುದು
ನಿಲ್ಲು-ವುವು
ನಿಲ್ಲು-ವೆನು
ನಿವರ್ತಂತೇ
ನಿವಾ-ರಣೆ
ನಿವಾ-ರಣೆ-ಯಾಗು-ವುವು
ನಿವಾ-ರಿಸ-ಬೇ-ಕಾದರೆ
ನಿವಾ-ರಿಸ-ಬೇಕಾ-ಯಿತು
ನಿವಾ-ರಿ-ಸಲು
ನಿವಾ-ರಿಸು-ವುದು
ನಿವಾ-ರಿಸು-ವುವು
ನಿವಾ-ಸ-ಸ್ಥಳ-ವನ್ನು
ನಿವಾ-ಸಿ-ಗಳು
ನಿವೃತ್ತಿ
ನಿವೇದಿತಾ
ನಿವೇದಿತಾ-ರನ್ನು
ನಿವೇದಿಸ-ಬೇಕಾಗಿದೆ
ನಿವೇದಿಸ-ಬೇಕೆಂದಿರು-ವೆನು
ನಿವೇದಿಸಿ-ದಿರಿ
ನಿಶಿತ
ನಿಶ್ಚಯ
ನಿಶ್ಚ-ಯಾ-ತ್ಮಕ
ನಿಶ್ಚ-ಯಿ-ಸಲು
ನಿಶ್ಚಯಿಸಿ
ನಿಶ್ಚಯಿಸಿಲ್ಲ
ನಿಶ್ಚಯಿ-ಸು-ವಂತೆಯೂ
ನಿಶ್ಚಲ-ನೆಯ
ನಿಶ್ಚಲ-ವಲ್ಲ
ನಿಶ್ಚಲ-ವಾಗಿದೆ
ನಿಶ್ಚಲ-ವಾಗು-ತ್ತದೆ
ನಿಶ್ಚಲ-ವಾಗು-ವುದು
ನಿಶ್ಚಲ-ವೆಂದು
ನಿಶ್ಚೇಷ್ಟರು
ನಿಶ್ಚೇಷ್ಟಿತ-ರಾಗಿ-ಬಿಡು-ವುದು
ನಿಷ-ತ್ತಿನ
ನಿಷ-ತ್ತಿನ-ಲ್ಲಿದೆ
ನಿಷತ್ತು-ಗಳ
ನಿಷತ್ತು-ಗಳಿಂದಲೇ
ನಿಷತ್ತು-ಗಳಿಗೆ
ನಿಷತ್ತು-ಗಳು
ನಿಷೇಧ
ನಿಷೇಧ-ಮಯದ
ನಿಷೇಧ-ಮಯ-ವಾದ
ನಿಷೇಧಾ-ತ್ಮಕ
ನಿಷೇಧಿ-ಸ-ಲಾ-ಯಿತು
ನಿಷೇಧಿ-ಸುತ್ತವೆ
ನಿಷೇಧಿ-ಸು-ವರು
ನಿಷೇಧಿಸು-ವುವು
ನಿಷ್ಕಪಟ
ನಿಷ್ಕಪಟ-ವಾಗಿ-ದ್ದರೆ
ನಿಷ್ಕಪಟಿ-ಗಳಾದ
ನಿಷ್ಕರ್ಷಿಸ-ಬೇಕಾಗಿ-ತ್ತು
ನಿಷ್ಕರ್ಷಿಸ-ಬೇಡಿ
ನಿಷ್ಕರ್ಷಿಸಿ
ನಿಷ್ಕರ್ಷಿ-ಸು-ವು-ದಕ್ಕೆ
ನಿಷ್ಕಾಮ
ನಿಷ್ಕೃಷ್ಟ-ವಾಗಿ
ನಿಷ್ಕ್ರಿಯ-ಗೊಳಿಸಿದೆ
ನಿಷ್ಕ್ರಿಯತೆ
ನಿಷ್ಕ್ರಿಯ-ವಾಗಿ
ನಿಷ್ಟಾ-ವಂತ-ರಾಗು-ವಿರಿ
ನಿಷ್ಠಾ-ಪೂರ್ವ-ಕ-ವಾದ
ನಿಷ್ಠಾ-ವಂತ
ನಿಷ್ಠಾ-ವಂತ-ರಾಗಿ
ನಿಷ್ಠಾ-ವಂತ-ರಾ-ಗುವುದ
ನಿಷ್ಠಾ-ವಂತರೂ
ನಿಷ್ಪಂದ-ವಾಗಿ
ನಿಷ್ಪಕ್ಷಪಾತಿ-ಯಾದ
ನಿಷ್ಪ್ರ-ಯೋ-ಕ-ವಾಗು-ವುದು
ನಿಷ್ಪ್ರ-ಯೋ-ಜಕ
ನಿಷ್ಪ್ರ-ಯೋ-ಜ-ಕರು
ನಿಷ್ಪ್ರ-ಯೋ-ಜ-ಕ-ವಾಗಿದೆ
ನಿಷ್ಪ್ರ-ಯೋ-ಜ-ಕ-ವೆಂದು
ನಿಷ್ಫಲ-ವಾಗು-ವಂತೆ
ನಿಷ್ಫಲ-ವಾಗು-ವುದು
ನಿಸ್ತ-ರಂಗ
ನಿಸ್ತ-ರಂಗ-ವಾಗಿ
ನಿಸ್ವಾರ್ಥ
ನಿಸ್ಸಂದೇಹ-ವಾಗಿ
ನಿಸ್ಸಂದೇಹ-ವಾಗಿದೆ
ನಿಸ್ಸಂದೇಹ-ವಾಗಿಯೂ
ನಿಸ್ಸಂಶಯ-ವಾಗಿ
ನಿಸ್ಸಾರ
ನೀಗಿ-ಕೊಳ್ಳು-ವು-ದಕ್ಕೂ
ನೀಗ್ರೋ
ನೀಗ್ರೋ-ಗಳು
ನೀಚ
ನೀಚ-ಕುಲ-ದಲ್ಲಿ
ನೀಚನೂ
ನೀಚ-ರ-ಲ್ಲಿಯೂ
ನೀಚ-ರಾದ
ನೀಚ-ರಿಗೆ
ನೀಚರು
ನೀಚರೂ
ನೀಚ-ವಾದು-ದನ್ನೂ
ನೀಚವೂ
ನೀಡ
ನೀಡ-ಬಲ್ಲದು
ನೀಡ-ಬಲ್ಲರು
ನೀಡ-ಬಹುದು
ನೀಡ-ಬಾ-ರದು
ನೀಡ-ಬೇಕಾಗಿದೆ
ನೀಡ-ಬೇಕಾಗಿದ್ದ
ನೀಡ-ಬೇಕಾಗಿ-ರುವ
ನೀಡ-ಬೇ-ಕಾದರೆ
ನೀಡ-ಬೇ-ಕಾ-ದುದು
ನೀಡ-ಬೇಕು
ನೀಡ-ಲಾ-ಗಿದೆ
ನೀಡ-ಲಾ-ಯಿತು
ನೀಡ-ಲಾರ-ದಷ್ಟು
ನೀಡಲಿ
ನೀಡಲು
ನೀಡ-ಲೆಂದು
ನೀಡಿ
ನೀಡಿತು
ನೀಡಿದ
ನೀಡಿ-ದರು
ನೀಡಿ-ದ-ವ-ರನ್ನು
ನೀಡಿ-ದ-ವರು
ನೀಡಿ-ದವು
ನೀಡಿ-ದು-ದ-ಕ್ಕಾಗಿಯೇ
ನೀಡಿ-ದುದು
ನೀಡಿದೆ
ನೀಡಿ-ದ್ದರು
ನೀಡಿ-ದ್ದಾರೆ
ನೀಡಿ-ರುವ
ನೀಡಿ-ರು-ವರು
ನೀಡಿ-ರು-ವುದು
ನೀಡಿವೆ
ನೀಡು
ನೀಡು-ತ್ತದೆ
ನೀಡು-ತ್ತವೆ
ನೀಡು-ತ್ತಾರೆ
ನೀಡು-ತ್ತಿ-ರುವ
ನೀಡು-ತ್ತಿರು-ವು-ದನ್ನು
ನೀಡು-ತ್ತಿರು-ವುದು
ನೀಡು-ತ್ತೇವೆ
ನೀಡುವ
ನೀಡು-ವಂಥದ್ದು
ನೀಡು-ವವ
ನೀಡು-ವವು
ನೀಡು-ವುದನ್ನೇ
ನೀಡು-ವು-ದರ-ಲ್ಲಾ-ಗಲಿ
ನೀಡು-ವು-ದಲ್ಲದೆ
ನೀಡು-ವು-ದಿಲ್ಲ
ನೀಡು-ವುದು
ನೀತಿ
ನೀತಿ-ಗಳನ್ನು
ನೀತಿ-ಗಳು
ನೀತಿಗೆ
ನೀತಿ-ತತ್ತ್ವದ
ನೀತಿ-ತತ್ತ್ವ-ವನ್ನು
ನೀತಿ-ಧರ್ಮ-ಗಳ
ನೀತಿ-ನಿಪುಣಾಃ
ನೀತಿಯ
ನೀತಿ-ಯನ್ನು
ನೀತಿ-ಯ-ನ್ನೇನೋ
ನೀತಿ-ಯಾಗಿದೆ
ನೀತಿ-ಯಿಂದೇನು
ನೀತಿಯೇ
ನೀತಿ-ವಂತ
ನೀನು
ನೀನೆ
ನೀನೇ
ನೀನೇನು
ನೀಯಾ-ಮಾನಾ
ನೀರನ್ನು
ನೀರಿಕ್ಷಿ-ಸ-ಬಲ್ಲಿರಿ
ನೀರಿ-ಗಾಗಿ
ನೀರಿಗೆ
ನೀರಿನ
ನೀರಿನಂತಿರು-ವುದು
ನೀರಿ-ನಂತೆ
ನೀರಿ-ನಲ್ಲಿ
ನೀರಿ-ನಿಂದ
ನೀರು
ನೀರು-ಗುಳ್ಳೆ-ಯಾಗಿಯೂ
ನೀರು-ಹಕ್ಕಿಯ
ನೀರೆ-ಳೆಯುವ
ನೀರೆಳೆ-ಯು-ವು-ದಕ್ಕೆ
ನೀವ-ದನ್ನು
ನೀವಾಗಲೇ
ನೀವಾರು
ನೀವಿಂದು
ನೀವು
ನೀವು-ಕೇವಲ
ನೀವು-ಗಳೆಲ್ಲ
ನೀವು-ಮಿಡಿದಿರು
ನೀವೂ
ನೀವೆಂತು
ನೀವೆಂದಿಗೂ
ನೀವೆಲ್ಲ
ನೀವೆ-ಲ್ಲರೂ
ನೀವೆಲ್ಲಾ
ನೀವೆಷ್ಟು
ನೀವೇ
ನೀವೇಕೆ
ನೀವೇನು
ನೀವೇನೂ
ನೀವೊಂದು
ನೀವೊಬ್ಬ
ನೀವೋ
ನುಂಗಿ-ದರು
ನುಗ್ಗ-ದಂತೆ
ನುಗ್ಗಿ
ನುಗ್ಗಿದ
ನುಗ್ಗಿದೆ
ನುಗ್ಗು-ತ್ತದೆ
ನುಗ್ಗುವಳು
ನುಚ್ಚು
ನುಚ್ಚು-ನೂ-ರಾ-ಗು-ತ್ತಿವೆ
ನುಜರ
ನುಡಿ
ನುಡಿ-ಗಳು
ನುಡಿ-ದಿ-ದ್ದರು
ನುಡಿ-ದಿ-ದ್ದರೆ
ನುಡಿ-ಯನ್ನು
ನುಡಿಯೂ
ನುಡಿ-ಯೊಂದು
ನುಡಿ-ಯೊಂದೇ
ನೂಕಲು
ನೂಕಿ-ದರೂ
ನೂಕುತ್ತಿ-ರುವ
ನೂಕುವ
ನೂತನ
ನೂರಕ್ಕೆ
ನೂರನೆ
ನೂರ-ರಲ್ಲಿ
ನೂರರಷ್ಟು
ನೂರಾ-ಗು-ತ್ತದೆ
ನೂರಾಗು-ವುದು
ನೂರಾ-ಯಿತು
ನೂರಾರು
ನೂರು
ನೂರು-ಪಾಲು
ನೂರು-ಬಾರಿ
ನೂರೆಂಟು
ನೃಣಾಂ
ನೃಣಾಮೇಕೋ
ನೆ
ನೆಂದರೆ
ನೆಂಬು-ದನ್ನು
ನೆಗೆ-ಯಲು
ನೆಚ್ಚ-ಬೇಡಿ
ನೆಚ್ಚಿ
ನೆಚ್ಚಿ-ಕೊಂಡಿ-ರು-ವುದು
ನೆಚ್ಚಿನ
ನೆಟ್ಟಿತು
ನೆಟ್ಟಿ-ದ್ದೀರಿ
ನೆಟ್ಟಿ-ರುವ
ನೆಟ್ಟು
ನೆಡುವ
ನೆಡುವುದ-ಕ್ಕೆಂದು
ನೆತ್ತರು
ನೆತ್ತಿಯ
ನೆನಪನ್ನು
ನೆನ-ಪಿಗೆ
ನೆನ-ಪಿ-ಟ್ಟಿರಿ-ಆ-ತ್ಮವು
ನೆನ-ಪಿ-ನಲ್ಲಿ
ನೆನ-ಪಿ-ನಲ್ಲಿಡ-ಬೇಕು
ನೆನ-ಪಿ-ನಲ್ಲಿ-ಡ-ಬೇಕು-ಅ-ವರು
ನೆನ-ಪಿ-ನ-ಲ್ಲಿಡಿ
ನೆನ-ಪಿ-ನಲ್ಲಿ-ಡು-ವುದು
ನೆನ-ಪಿ-ನಲ್ಲಿ-ರ-ಬಹುದು
ನೆನ-ಪಿ-ರ-ಬಹುದು
ನೆನ-ಪಿ-ಸಿ-ಕೊಳ್ಳಿ
ನೆನಪು
ನೆನಸಿ-ಕೊಳ್ಳು-ವಂತೆ
ನೆನ-ಸು-ವಂತೆ
ನೆನೆ-ದರೆ
ನೆನೆ-ದರೇ
ನೆನೆ-ಯು-ವುದು
ನೆರ-ಳಿಗೆ
ನೆರ-ವಾಗ-ಬೇಕು
ನೆರ-ವಾಗುವ
ನೆರ-ವಾಗು-ವಂತೆ
ನೆರ-ವಾದ
ನೆರ-ವಿಗೆ
ನೆರವು
ನೆರವೇ-ರಿತು
ನೆರವೇ-ರಿ-ದರೆ
ನೆರವೇ-ರಿವೆ
ನೆರವೇ-ರಿಸ-ಬೇಕೆಂದು
ನೆರವೇ-ರಿಸಿ-ಕೊಳ್ಳ-ಬೇಕಾಗಿ-ರುವ
ನೆರವೇ-ರಿಸಿ-ಕೊಳ್ಳು-ವುದು
ನೆರವೇ-ರಿ-ಸಿದ
ನೆರವೇ-ರಿ-ಸು-ವನು
ನೆರವೇ-ರು-ತ್ತದೆ
ನೆರೆ-ದಿದ್ದ
ನೆರೆ-ಯಲು
ನೆರೆ-ಹೊರೆ-ಯವರ
ನೆಲಕ್ಕುರುಳಿ-ದರೂ
ನೆಲಕ್ಕುರು-ಳಿದವು
ನೆಲಕ್ಕೆ
ನೆಲದ
ನೆಲ-ದಲ್ಲಿ
ನೆಲ-ವನ್ನು
ನೆಲ-ವನ್ನೂ
ನೆಲ-ಸಮ
ನೆಲ-ಸಮ-ಗೊಳಿ-ಸಿದ
ನೆಲ-ಸಮ-ವಾಗುತ್ತವೆ
ನೆಲಸಿ
ನೆಲ-ಸಿದ
ನೆಲಸಿ-ದರು
ನೆಲ-ಸಿದೆ
ನೆಲಸಿ-ರು-ವುದೋ
ನೆಲ-ಸು-ವಂತೆ
ನೆಲೆ
ನೆಲೆಗೆ
ನೆಲೆ-ಬೀಡಾ-ಗಿದ್ದ
ನೆಲೆ-ಯಾಗಿ-ರು-ವುದು
ನೆಲೆ-ಸಿ-ರುವ
ನೆಲೆ-ಸಿ-ರು-ವನು
ನೆಲೆ-ಸಿ-ರುವಳೊ
ನೆಲ್ಲಿ
ನೆಲ್ಲಿ-ಕಾಯಿ-ಯಂತೆ
ನೆವ-ದಲ್ಲಿ
ನೇ
ನೇತಿ
ನೇತೃತ್ವ-ದಲ್ಲಿ
ನೇಪಾಳ-ದಲ್ಲಿ
ನೇಮಾ
ನೇರ-ವಾಗಿ
ನೇರ-ವಾಗಿಯೋ
ನೇರ-ವಾದ
ನೈಜ
ನೈಜ-ಗುಣ-ವನ್ನು
ನೈಜ-ಗುಣ-ವಾದ
ನೈಜ-ಗು-ಣವು
ನೈತಿಕ
ನೈತಿ-ಕತೆ
ನೈತಿಕ-ತೆಯೂ
ನೈತ್ಧಿಕ್ಧತೆ
ನೈನಂ
ನೈಯಾ-ಯಿ-ಕರ
ನೈಸರ್ಗಿಕ
ನೊಣ
ನೊಣ-ಗಳ
ನೊಬೆ-ಲ್
ನೋ
ನೋಟ
ನೋಟಕ್ಕೆ
ನೋಟ-ವೊಂದು
ನೋಡದ
ನೋಡದೆ
ನೋಡ-ಬಯ-ಸಿದರೆ
ನೋಡ-ಬಯ-ಸುವರು
ನೋಡ-ಬಲ್ಲರು
ನೋಡ-ಬಲ್ಲರೋ
ನೋಡಬಲ್ಲುದು
ನೋಡ-ಬಹು-ದಾದ
ನೋಡ-ಬಹುದು
ನೋಡ-ಬಾ-ರದು
ನೋಡ-ಬೇಕಾಗಿದೆ
ನೋಡಬೇಕಾ-ಗು-ವು-ದಿಲ್ಲ
ನೋಡ-ಬೇಕು
ನೋಡ-ಬೇಕೆಂಬ
ನೋಡಲಾ-ಗದ
ನೋಡ-ಲಾ-ಗು-ವು-ದಿಲ್ಲ-ವೇನೋ
ನೋಡ-ಲಾ-ರದು
ನೋಡಲಾರರೋ
ನೋಡಲಿ
ನೋಡಲಿ-ಚ್ಛಿ-ಸು-ವನು
ನೋಡ-ಲಿಲ್ಲ
ನೋಡಲು
ನೋಡಲೂ
ನೋಡಲೇ
ನೋಡಲ್ಪಟ್ಟದ್ದೂ
ನೋಡಿ
ನೋಡಿ-ಕೊಂಡರೆ
ನೋಡಿ-ಕೊಂಡು
ನೋಡಿ-ಕೊಳ್ಳದೆ
ನೋಡಿ-ಕೊಳ್ಳ-ಬ-ಲ್ಲದೆ
ನೋಡಿ-ಕೊಳ್ಳ-ಬೇಕಾಗಿ-ತ್ತು
ನೋಡಿ-ಕೊಳ್ಳ-ಬೇಕು
ನೋಡಿ-ಕೊಳ್ಳಲಿ
ನೋಡಿ-ಕೊಳ್ಳಿ
ನೋಡಿ-ಕೊಳ್ಳು-ತ್ತಿದೆ
ನೋಡಿ-ಕೊಳ್ಳು-ತ್ತಿದ್ದರು
ನೋಡಿ-ಕೊಳ್ಳು-ತ್ತಿಲ್ಲ
ನೋಡಿ-ಕೊಳ್ಳು-ವನು
ನೋಡಿ-ಕೊಳ್ಳು-ವುದು
ನೋಡಿದ
ನೋಡಿ-ದರು
ನೋಡಿ-ದರೂ
ನೋಡಿ-ದರೆ
ನೋಡಿ-ದಾಗ
ನೋಡಿ-ದು-ದಕ್ಕೆ
ನೋಡಿದೆ
ನೋಡಿ-ದ್ದಂತೆ
ನೋಡಿ-ದ್ದನೊ
ನೋಡಿ-ದ್ದ-ರಲ್ಲಿ
ನೋಡಿ-ದ್ದೀರಾ
ನೋಡಿ-ದ್ದೀರಿ
ನೋಡಿ-ದ್ದೇನೆ
ನೋಡಿ-ದ್ದೇವೆ
ನೋಡಿ-ರುವ
ನೋಡಿ-ರು-ವಂತೆ
ನೋಡಿ-ರು-ವರು
ನೋಡಿ-ರು-ವರೋ
ನೋಡಿ-ರು-ವು-ದ-ಕ್ಕಿಂತ
ನೋಡಿ-ರು-ವೆನು
ನೋಡಿ-ರು-ವೆವು
ನೋಡಿ-ರು-ವೆವೋ
ನೋಡಿಲ್ಲ
ನೋಡಿ-ಲ್ಲವೋ
ನೋಡು
ನೋಡು-ತ್ತದೆ
ನೋಡು-ತ್ತವೆ
ನೋಡು-ತ್ತಾನೆ
ನೋಡು-ತ್ತಾ-ನೆಯೋ
ನೋಡು-ತ್ತಾರೆ
ನೋಡು-ತ್ತಿದೆ
ನೋಡು-ತ್ತಿದ್ದೆ
ನೋಡು-ತ್ತಿ-ರುವ
ನೋಡು-ತ್ತಿ-ರು-ವರು
ನೋಡು-ತ್ತಿ-ರು-ವರೋ
ನೋಡು-ತ್ತಿರು-ವೆನು
ನೋಡು-ತ್ತಿರು-ವೆವು
ನೋಡು-ತ್ತಿರು-ವೆವೋ
ನೋಡು-ತ್ತೀರಿ
ನೋಡು-ತ್ತೇನೆ
ನೋಡು-ತ್ತೇವೆ
ನೋಡುವ
ನೋಡು-ವಂತೆ
ನೋಡು-ವನು
ನೋಡು-ವರು
ನೋಡು-ವರೊ
ನೋಡು-ವರೋ
ನೋಡು-ವ-ವನು
ನೋಡು-ವ-ವರಿ-ಗಿಂತ
ನೋಡು-ವ-ವ-ರೆಗೆ
ನೋಡು-ವ-ವರೋ
ನೋಡು-ವಿರಿ
ನೋಡು-ವು-ದಕ್ಕೆ
ನೋಡು-ವು-ದ-ರಿಂದಲೇ
ನೋಡು-ವು-ದಿಲ್ಲ
ನೋಡು-ವು-ದಿ-ಲ್ಲವೆ
ನೋಡು-ವುದು
ನೋಡು-ವುದೇ
ನೋಡು-ವೆವು
ನೋಡು-ವೆವೋ
ನೋಡೋಣ
ನೋಬೆ-ಲ್
ನೋಯಿ-ಸದ
ನೋಯಿಸ-ಬಾ-ರದು
ನೋವನ್ನುಂಟು-ಮಾಡ-ಕೂಡದು
ನೋವಾಗು-ತ್ತದೆ
ನೋವಾದರೆ
ನೌ
ನೌಕೆ
ನ್ನುವ
ನ್ಮುಖ-ವಾದಾಗ
ನ್ಯಾಯ
ನ್ಯಾ-ಯದ
ನ್ಯಾ-ಯ-ದೃಷ್ಟಿ
ನ್ಯಾ-ಯ-ಪಕ್ಷಪಾತಿ-ಗಳೂ
ನ್ಯಾ-ಯ-ಬದ್ಧ-ವಲ್ಲ
ನ್ಯಾ-ಯ-ಮೂರ್ತಿ-ಗಳಾದ
ನ್ಯಾ-ಯ-ವನ್ನು
ನ್ಯಾ-ಯ-ವಾಗಿ
ನ್ಯಾ-ಯ-ವಾಗಿಯೇ
ನ್ಯಾ-ಯ-ವಾಗಿಯೋ
ನ್ಯಾ-ಯ-ಸಿದ್ಧಾಂತ-ದ-ವ-ರಿಗೆ
ನ್ಯಾ-ಯಾ-ಧಿ-ಪತಿ-ಯಾಗಿ-ರು-ವಾಗ
ನ್ಯಾ-ಯಾ-ಧಿ-ಪತಿ-ಯಾದ
ನ್ಯಾ-ಯಾ-ಲಯ
ನ್ಯಾ-ಯಾ-ಸ್ಥಾನ-ದಲ್ಲಿ
ನ್ಯಾ-ಯ್ಯಾ-ತ್
ನ್ಯೂನತೆ
ನ್ಯೂನಾತಿ-ರೇಕದ
ನ್ಯೂಯಾರ್ಕ್
ಪಂಕ್ತಿ
ಪಂಕ್ತಿಗೆ
ಪಂಕ್ತಿ-ಯನ್ನೂ
ಪಂಗಡ
ಪಂಗಡಕ್ಕೂ
ಪಂಗಡಕ್ಕೆ
ಪಂಗಡಕ್ಕೇ
ಪಂಗಡ-ಗಳ
ಪಂಗಡ-ಗಳನ್ನು
ಪಂಗಡ-ಗಳ-ನ್ನೆಲ್ಲಾ
ಪಂಗಡ-ಗಳಲ್ಲಿ
ಪಂಗಡ-ಗಳಾಗಿ
ಪಂಗಡ-ಗಳಿಗೂ
ಪಂಗಡ-ಗಳಿಗೆ
ಪಂಗಡ-ಗಳಿಗೆಲ್ಲ
ಪಂಗಡ-ಗಳಿ-ರು-ವುದು
ಪಂಗಡ-ಗಳಿವೆ
ಪಂಗಡ-ಗಳು
ಪಂಗಡ-ಗಳೂ
ಪಂಗಡ-ಗಳೆಲ್ಲ
ಪಂಗ-ಡದ
ಪಂಗಡ-ದ-ರಾಜನು
ಪಂಗಡ-ದ-ವ-ರಲ್ಲಿ
ಪಂಗಡ-ದ-ವ-ರಲ್ಲೂ
ಪಂಗಡ-ದ-ವರು
ಪಂಗಡ-ದ-ವರೂ
ಪಂಗಡ-ವನ್ನು
ಪಂಗಡವು
ಪಂಚ
ಪಂಚ-ದೇವ-ತೆ-ಗಳೂ
ಪಂಚ-ನದಿ-ಗಳ
ಪಂಚ-ಲ-ಕ್ಷಣ-ಗಳಿವೆ
ಪಂಚಾಯಿ-ತಿಗೆ
ಪಂಚೇಂದ್ರಿಯ
ಪಂಚೇಂದ್ರಿಯ-ಗಳ
ಪಂಚೇಂದ್ರಿಯ-ಗಳ-ಲ್ಲಿದೆ
ಪಂಚೇಂದ್ರಿಯ-ಗಳ-ಲ್ಲಿರು-ವುದು
ಪಂಚೇಂದ್ರಿಯ-ಗಳಿ-ವೆಯೋ
ಪಂಜರ-ದಲ್ಲಿ
ಪಂಜಾಬನ್ನು
ಪಂಜಾಬಿನ
ಪಂಜಾಬ್
ಪಂಡಿತಂ
ಪಂಡಿತನಾ-ಗಲಿ
ಪಂಡಿ-ತನೂ
ಪಂಡಿತ-ಪಾಮ-ರರ
ಪಂಡಿ-ತರ
ಪಂಡಿ-ತ-ರಲ್ಲ
ಪಂಡಿ-ತ-ರಲ್ಲಿ
ಪಂಡಿತ-ರಾಗ-ಬಹುದು
ಪಂಡಿತ-ರಿಂದ
ಪಂಡಿತರಿ-ಗಾಗಿ
ಪಂಡಿತ-ರಿಗೂ
ಪಂಡಿತ-ರಿಗೆ
ಪಂಡಿ-ತರು
ಪಂಡಿ-ತರೂ
ಪಂಡಿತವೃಂದವು
ಪಂಥ
ಪಂಥಕ್ಕೆ
ಪಂಥ-ಗಳ
ಪಂಥ-ಗಳನ್ನು
ಪಂಥ-ಗಳನ್ನೂ
ಪಂಥ-ಗಳಲ್ಲಿ
ಪಂಥ-ಗಳ-ಲ್ಲಿಯೂ
ಪಂಥ-ಗಳಿಗೂ
ಪಂಥ-ಗಳಿ-ದ್ದುವು
ಪಂಥ-ಗಳಿ-ಲ್ಲದೆ
ಪಂಥ-ಗಳಿವೆ
ಪಂಥ-ಗಳು
ಪಂಥ-ಗಳೂ
ಪಂಥತ
ಪಂಥ-ದ-ವರು
ಪಂಥ-ದ-ವರೂ
ಪಂಥ-ವನ್ನು
ಪಂಥವೂ
ಪಂಥಾ
ಪಂಥೀಯ
ಪಕ್ಕಕ್ಕೆ
ಪಕ್ಕ-ದಲ್ಲಿ
ಪಕ್ಕ-ದಲ್ಲೇ
ಪಕ್ವ-ವಲ್ಲದ
ಪಕ್ಷ-ಗಳನ್ನು
ಪಕ್ಷ-ಗಳಿಗೂ
ಪಕ್ಷದ
ಪಕ್ಷ-ದಲ್ಲಿ
ಪಕ್ಷ-ಪಾತ-ವನ್ನು
ಪಕ್ಷ-ಪಾತ-ವಿದೆ
ಪಕ್ಷ-ಪಾತ-ವಿ-ಲ್ಲದೆ
ಪಕ್ಷಪಾತಿ
ಪಕ್ಷ-ಪ್ರತಿ-ಪಕ್ಷ-ಗಳನ್ನು
ಪಕ್ಷ-ವನ್ನೂ
ಪಕ್ಷಿ-ಗಳನ್ನು
ಪಕ್ಷಿ-ಗಳಲ್ಲಿ
ಪಕ್ಷಿ-ಯನ್ನು
ಪಟು-ಗಳು
ಪಟ್ಟ
ಪಟ್ಟಕ್ಕೆ
ಪಟ್ಟ-ಣ-ದಲ್ಲಿ
ಪಟ್ಟ-ಣಿಗ
ಪಟ್ಟರೂ
ಪಟ್ಟಿ-ರುವ
ಪಟ್ಟು-ದನ್ನೆಲ್ಲಾ
ಪಟ್ಟೆ-ಗಳನ್ನು
ಪಟ್ಟೆವು
ಪಟ್ಟೊ
ಪಠಿಸಿ
ಪಠಿಸಿ-ದರು
ಪಡ-ದಿ-ರಲಿ
ಪಡದೆ
ಪಡ-ಬಹು-ದಾದ
ಪಡ-ಬೇಕಾಗಿಲ್ಲ
ಪಡಬೇಕಾಗು-ವುದು
ಪಡಲಿ
ಪಡಲಿ-ಲ್ಲವೆ
ಪಡಲಿ-ಲ್ಲವೇ
ಪಡಿ-ಸಿದನು
ಪಡಿ-ಸಿದರೆ
ಪಡುತ್ತಾರೆ
ಪಡು-ತ್ತೇನೆ
ಪಡುವ-ವ-ರಲ್ಲಿ
ಪಡು-ವು-ದಿಲ್ಲ
ಪಡು-ವು-ದಿಲ್ಲವೇ
ಪಡೆ
ಪಡೆದ
ಪಡೆ-ದಂತಾಯಿ-ತೆಂದು
ಪಡೆ-ದರು
ಪಡೆ-ದರೂ
ಪಡೆ-ದ-ವ-ರಾಗಿ-ದ್ದರೂ
ಪಡೆ-ದ-ವರೂ
ಪಡೆ-ದ-ವರೇ
ಪಡೆ-ದವು
ಪಡೆ-ದಾದ
ಪಡೆ-ದಿದೆ
ಪಡೆ-ದಿ-ರುವ
ಪಡೆ-ದಿ-ರುವನೋ
ಪಡೆ-ದಿ-ರು-ವರೋ
ಪಡೆ-ದಿರು-ವೆವು
ಪಡೆ-ದಿವೆ
ಪಡೆದು
ಪಡೆ-ದು-ಕೊಂಡಿಲ್ಲ
ಪಡೆ-ದು-ಕೊಳ್ಳ-ಬಹುದು
ಪಡೆ-ದು-ಕೊಳ್ಳ-ಬೇಕಾಗಿಲ್ಲ
ಪಡೆ-ದು-ಕೊಳ್ಳ-ಬೇಕು
ಪಡೆ-ದು-ಕೊಳ್ಳುವ
ಪಡೆ-ಯನ್ನು
ಪಡೆ-ಯ-ಬಲ್ಲ
ಪಡೆ-ಯ-ಬಹುದು
ಪಡೆ-ಯ-ಬೇಕಾಗಿಲ್ಲ
ಪಡೆ-ಯ-ಬೇ-ಕಾದ
ಪಡೆ-ಯ-ಬೇ-ಕಾದರೆ
ಪಡೆ-ಯಬೇ-ಕಾದು
ಪಡೆ-ಯ-ಬೇಕು
ಪಡೆ-ಯ-ಬೇಕೆ
ಪಡೆ-ಯ-ಬೇಕೆಂಬ
ಪಡೆ-ಯ-ಬೇಕೆಂಬುದೇ
ಪಡೆ-ಯ-ಲಾ-ಗು-ವು-ದಿಲ್ಲ-ವೆಂದು
ಪಡೆ-ಯ-ಲಿಲ್ಲ
ಪಡೆ-ಯಲು
ಪಡೆ-ಯ-ಲೆಂದು
ಪಡೆ-ಯಲೇ-ಬೇಕು
ಪಡೆ-ಯಿರಿ
ಪಡೆ-ಯು-ತ್ತದೆ
ಪಡೆ-ಯುತ್ತಾ
ಪಡೆ-ಯುವ
ಪಡೆ-ಯು-ವಂತ-ಹು-ದಲ್ಲ
ಪಡೆ-ಯು-ವನು
ಪಡೆ-ಯು-ವರು
ಪಡೆ-ಯುವ-ವ-ರೆಗೆ
ಪಡೆ-ಯು-ವಷ್ಟು
ಪಡೆ-ಯುವು-ದ-ಕ್ಕಾಗಿ
ಪಡೆ-ಯು-ವು-ದಕ್ಕೆ
ಪಡೆ-ಯು-ವುದು
ಪಡೆ-ಯು-ವುದೇ
ಪಡೆ-ಯು-ವು-ದೊಂದು
ಪಡೆ-ಯು-ವುವು
ಪತಂಜಲಿ
ಪತಂಜಲಿ-ಪ್ರಣೀತ
ಪತಂಜಲಿಯ
ಪತಂಜಲಿಯು
ಪತನ
ಪತ-ನಕ್ಕೆ
ಪತಾಕೆ
ಪತಾ-ಕೆಯ
ಪತಾಕೆ-ಯನ್ನು
ಪತಿ
ಪತಿತ
ಪತಿ-ತ-ನಾದ
ಪತಿ-ತನೂ
ಪತಿ-ತರ
ಪತಿ-ತ-ರಾಗಿ
ಪತಿ-ತ-ರಾಗಿ-ರು-ವೆವು
ಪತಿ-ತ-ರೊಂದಿಗೆ
ಪತಿ-ತ-ಳಾದ
ಪತಿ-ತ್ರೋ-ತ್ತಮ
ಪತ್ರ-ದಲ್ಲಿ
ಪತ್ರ-ವನ್ನು
ಪತ್ರವೂ
ಪತ್ರಿಕೆ-ಗಳ
ಪತ್ರಿಕೆ-ಗಳಲ್ಲಿ
ಪತ್ರಿಕೆ-ಗಳಿಗೆ
ಪತ್ರಿಕೆ-ಗಳು
ಪತ್ರಿಕೆ-ಯಲ್ಲಿ
ಪಥ
ಪಥಃ
ಪಥಕ್ಕೆ
ಪಥ-ಗಳೆಲ್ಲಾ
ಪಥ-ಜುಷಾಂ
ಪಥ-ದಲ್ಲಿ
ಪಥ-ದಿಂದ
ಪಥ-ವನ್ನೂ
ಪಥ್ಯ-ಮಿತಿ
ಪದ
ಪದಂ
ಪದಕ್ಕೆ
ಪದ-ಗಳ
ಪದ-ಗಳಂತೆ
ಪದ-ಗಳನ್ನು
ಪದ-ಗಳಲ್ಲಿ
ಪದ-ಗಳಿಂದ
ಪದ-ಗಳಿವೆ
ಪದ-ಗಳು
ಪದ-ಚ್ಯು-ತಗೊ-ಳಿ-ಸಿದ-ವರೇ
ಪದ-ಚ್ಯು-ತರ-ನ್ನಾಗಿ
ಪದ-ತಲ-ದಲ್ಲಿ
ಪದ-ತಳಕ್ಕೆ
ಪದ-ತಳ-ದಲ್ಲಿ
ಪದ-ತಳ-ದಲ್ಲಿಯೂ
ಪದ-ತಳ-ದಲ್ಲಿ-ರ-ಬೇಕೆ
ಪದದ
ಪದ-ದಲಿತ-ರಿಗೆ
ಪದ-ದಲ್ಲಿ
ಪದ-ದಲ್ಲಿದೆ
ಪದ-ದಿಂದ
ಪದ-ವನ್ನಾ-ದರೂ
ಪದ-ವನ್ನು
ಪದ-ವನ್ನು-ಕೆಲವು
ಪದ-ವಾಗಿದೆ
ಪದವಿ
ಪದ-ವಿ-ಗಳ
ಪದ-ವಿಗೆ
ಪದ-ವಿ-ಟ್ಟರೋ
ಪದ-ವಿಲ್ಲ
ಪದ-ವೀ-ಧ-ರರು
ಪದ-ವೀ-ಧ-ರರೇ
ಪದವು
ಪದವೂ
ಪದವೇ
ಪದಾಘಾ-ತಕ್ಕೆ
ಪದಾರ್ಥ-ಗಳಿಂದಲೂ
ಪದಾರ್ಥ-ಗಳು
ಪದಾರ್ಥದ
ಪದಾರ್ಥ-ವನ್ನು
ಪದೇ
ಪದ್ದ-ತಿಯ
ಪದ್ದ-ತಿಯು
ಪದ್ಧತಿ
ಪದ್ಧತಿ-ಗಳ-ನ್ನನು-ಸ-ರಿಸಿ
ಪದ್ಧತಿ-ಗಳನ್ನು
ಪದ್ಧತಿ-ಗಳಲ್ಲಿ
ಪದ್ಧತಿ-ಗಳು
ಪದ್ಧತಿ-ಗಳೊಂದಿಗೆ
ಪದ್ಧತಿ-ಗಿಂತ
ಪದ್ಧತಿಗೂ
ಪದ್ಧತಿಗೆ
ಪದ್ಧತಿ-ಯನ್ನು
ಪದ್ಧತಿ-ಯಲ್ಲಿ-ರ-ಬೇಕು
ಪದ್ಧತಿ-ಯಿಂದಾಗಿ
ಪದ್ಧತಿ-ಯೊಂದಿಗೆ
ಪದ್ಯ
ಪನ್ನ
ಪಯಸಾಮರ್ಣವ
ಪರ
ಪರಂ
ಪರಂಜ್ಯೋ-ತಿಯ
ಪರಂಜ್ಯೋ-ತಿ-ಯಾದ
ಪರಂಪರಾ-ಗತ-ವಾಗಿ
ಪರಂಪರೆ
ಪರಂಪರೆ-ಗಳಿಗೆ
ಪರಂಪ-ರೆಗೆ
ಪರಂಪರೆಯ
ಪರಂಪರೆಯು
ಪರ-ಕೀಯರ
ಪರ-ಕೀಯ-ರನ್ನು
ಪರ-ಜನ್ಮ-ಗಳಿಗೆ
ಪರದೆ
ಪರ-ದೇಶ-ದಲ್ಲಿ
ಪರ-ದೇಶೀ-ಯರ
ಪರ-ಬ್ರಹ್ಮ
ಪರ-ಬ್ರಹ್ಮ-ನನ್ನು
ಪರ-ಬ್ರಹ್ಮ-ವನ್ನು
ಪರ-ಬ್ರಹ್ಮವು
ಪರಮ
ಪರ-ಮ-ಕುಡಿ
ಪರ-ಮ-ಕುಡಿಯ
ಪರ-ಮ-ಗಣ್ಯನೂ
ಪರ-ಮ-ಗತಿ
ಪರ-ಮ-ಗತಿ-ಯನ್ನು
ಪರ-ಮ-ಗುರಿ
ಪರ-ಮ-ಗುರಿ-ಯನ್ನು
ಪರ-ಮ-ಜ್ಞಾನ
ಪರ-ಮ-ದ-ಯಾ-ಮಯ
ಪರ-ಮ-ದ-ಯಾಳು
ಪರ-ಮ-ಪದ-ವನ್ನು
ಪರ-ಮ-ಪುರು
ಪರ-ಮ-ಪೂಜ್ಯ
ಪರ-ಮ-ಪೂಜ್ಯ-ರಾದ
ಪರ-ಮ-ಪ್ರ-ಮಾ-ಣ-ವೆಂದು
ಪರ-ಮ-ಭಕ್ತ-ನಾಗಿ-ರು-ವಷ್ಟೇ
ಪರ-ಮ-ಭಕ್ತಿ
ಪರ-ಮ-ಮಿತ್ರ
ಪರ-ಮ-ಯೋಗೀಶ್ವರ
ಪರ-ಮ-ಶಿಷ್ಯ
ಪರ-ಮ-ಶ್ರೇಷ್ಠ
ಪರ-ಮ-ಹಂಸ
ಪರ-ಮ-ಹಂಸರ
ಪರ-ಮ-ಹಂಸ-ರಂತಹ
ಪರ-ಮ-ಹಂಸ-ರಂತೆ
ಪರ-ಮ-ಹಂಸ-ರಂಥ-ವ-ರನ್ನು
ಪರ-ಮ-ಹಂಸ-ರನ್ನು
ಪರ-ಮ-ಹಂಸರು
ಪರ-ಮ-ಹಂಸರೇ
ಪರ-ಮ-ಹಂಸ-ರೊಬ್ಬರೇ
ಪರಮಾ
ಪರ-ಮಾ-ಣುವಿ-ನ-ಲ್ಲಿಯೂ
ಪರ-ಮಾ-ಣುವು
ಪರ-ಮಾತ್ಮ
ಪರ-ಮಾ-ತ್ಮನ
ಪರ-ಮಾ-ತ್ಮ-ನನ್ನು
ಪರ-ಮಾ-ತ್ಮ-ನಾದ
ಪರ-ಮಾ-ತ್ಮ-ನಿಂದ
ಪರ-ಮಾ-ತ್ಮ-ನಿಗೂ
ಪರ-ಮಾ-ತ್ಮ-ನಿಗೆ
ಪರ-ಮಾ-ತ್ಮ-ನಿ-ದ್ದಾನೆ
ಪರ-ಮಾ-ತ್ಮನು
ಪರ-ಮಾ-ತ್ಮ-ನೆಂದು
ಪರ-ಮಾ-ತ್ಮ-ನೊಬ್ಬ-ನಿದ್ದಾ-ನೆಂದು
ಪರ-ಮಾ-ತ್ಮರ
ಪರ-ಮಾ-ತ್ಮರು
ಪರ-ಮಾ-ದರ್ಶದ
ಪರ-ಮಾ-ದ್ಭುತ
ಪರ-ಮಾ-ದ್ವೈ-ತಕ್ಕೆ
ಪರ-ಮಾ-ಧಿ-ಕಾ-ರವೂ
ಪರ-ಮಾ-ನಂದ
ಪರ-ಮಾರ್ಥ
ಪರ-ಮಾ-ವಧಿ
ಪರ-ಮಾ-ವಧಿ-ಯನ್ನು
ಪರ-ಮಾ-ವಧಿ-ಯಲ್ಲ
ಪರ-ಮಿದಮದಃ
ಪರ-ಮೇಶ್ವರ
ಪರ-ಮೇಶ್ವರಂ
ಪರ-ಮೇಶ್ವರನ
ಪರ-ಮೇಶ್ವರ-ನನ್ನು
ಪರ-ಮೋಚ್ಚ
ಪರ-ಮೋಚ್ಚ-ವಾದು-ದನ್ನು
ಪರ-ಮೋಚ್ಛ
ಪರ-ಮೋ-ತ್ಕೃಷ್ಟ
ಪರ-ಮೋ-ತ್ಕೃಷ್ಟ-ನಿಂದ
ಪರ-ಮೋ-ತ್ಕೃಷ್ಟ-ವಾದುದು
ಪರ-ಮೌಷಧಿ
ಪರ-ರಾದ
ಪರ-ರಾಷ್ಟ್ರ-ಗಳ
ಪರ-ರಾಷ್ಟ್ರ-ವನ್ನು
ಪರ-ರಿಗೆ
ಪರ-ಲೋಕ
ಪರ-ಲೋಕ-ದಲ್ಲಿ
ಪರ-ವಶತೆ-ಯಿಂದ
ಪರ-ವಶರಾಗು-ವಿರಿ
ಪರ-ವಶ-ರಾದ-ವರ
ಪರ-ವಸ್ತುವು
ಪರ-ವಾಗಿ
ಪರ-ವಾಗಿಯೂ
ಪರ-ವಾಗಿಲ್ಲ
ಪರ-ವಾ-ಯಿಲ್ಲ
ಪರ-ಸ್ಪರ
ಪರ-ಸ್ಪರ-ರನ್ನು
ಪರ-ಸ್ಪರ-ವಾಗಿ-ರುವ
ಪರ-ಸ್ಫರ
ಪರ-ಸ್ವರ
ಪರ-ಹಿತ
ಪರಾ
ಪರಾಂ
ಪರಾ-ಕಾಷ್ಠೆ
ಪರಾ-ಕಾಷ್ಠೆ-ಯನ್ನು
ಪರಾ-ಧೀ-ನತೆ-ಯನ್ನೂ
ಪರಾ-ಯಣತೆ
ಪರಾ-ಯಣ-ರನ್ನು
ಪರಾ-ಯಣ-ರಲ್ಲಿ
ಪರಾ-ಯಣರು
ಪರಿ
ಪರಿ-ಗಣಿ-ಸ-ಬೇಕಾಗಿಲ್ಲ
ಪರಿ-ಗಣಿ-ಸಲು
ಪರಿ-ಗಣಿ-ಸಲ್ಪಟ್ಟ
ಪರಿ-ಗಣಿ-ಸಿದ
ಪರಿ-ಗಣಿ-ಸಿ-ದಷ್ಟೂ
ಪರಿ-ಗಣಿ-ಸಿ-ದಾಗ
ಪರಿ-ಗಣಿ-ಸಿ-ದ್ದಾರೆ
ಪರಿ-ಗಣಿ-ಸಿರ-ಬಹುದು
ಪರಿ-ಗಣಿ-ಸುವ
ಪರಿ-ಗಣಿ-ಸು-ವುದೇ
ಪರಿ-ಗ್ರ-ಹಿ-ಸು-ವನು
ಪರಿ-ಚಯ
ಪರಿ-ಚಯ-ಗಳ
ಪರಿ-ಚಯ-ಪತ್ರ-ವನ್ನು
ಪರಿ-ಚಯ-ವನ್ನು
ಪರಿ-ಚಯ-ವಾಗಲಿ
ಪರಿ-ಚಯ-ವಾದ-ಮೇಲೆ
ಪರಿ-ಚಯ-ವಿದೆ
ಪರಿ-ಚಯ-ವಿದ್ದ
ಪರಿ-ಚಯ-ವಿದ್ದ-ವ-ರಿಗೆ
ಪರಿ-ಚಯ-ವಿದ್ದು
ಪರಿ-ಚಯ-ವಿ-ರುವ
ಪರಿ-ಚಯ-ವಿಲ್ಲ-ದ-ವರು
ಪರಿ-ಚ-ಯವೇ
ಪರಿ-ಚ-ಯಾ-ತ್ಮಕ-ವಾಗಿ
ಪರಿ-ಚಯಿಸಿ
ಪರಿ-ಚಯಿಸಿ-ದರು
ಪರಿ-ಚಯಿ-ಸುತ್ತ
ಪರಿ-ಚಿತ
ಪರಿ-ಚಿತ-ವಾಗಿರ
ಪರಿ-ಣತ-ರಾ-ದಷ್ಟೂ
ಪರಿ-ಣಮಿ-ಸಿತು
ಪರಿ-ಣಮಿ-ಸಿದ
ಪರಿ-ಣಮಿ-ಸಿದೆ
ಪರಿ-ಣಮಿ-ಸುತ್ತವೆ
ಪರಿ-ಣಮಿ-ಸು-ತ್ತಿದೆ
ಪರಿ-ಣಾಮ
ಪರಿ-ಣಾಮಃ
ಪರಿ-ಣಾಮ-ಕಾರಿ-ಯಾಗು-ವು-ದಿಲ್ಲ
ಪರಿ-ಣಾಮ-ಗಳನ್ನೂ
ಪರಿ-ಣಾಮ-ಗಳು
ಪರಿ-ಣಾಮದ
ಪರಿ-ಣಾಮ-ದಲ್ಲಿ
ಪರಿ-ಣಾಮ-ವನ್ನು
ಪರಿ-ಣಾಮ-ವನ್ನುಂಟು
ಪರಿ-ಣಾಮ-ವನ್ನುಂಟು-ಮಾಡು-ವಂತೆ
ಪರಿ-ಣಾಮ-ವನ್ನೂ
ಪರಿ-ಣಾಮ-ವಾಗಿ
ಪರಿ-ಣಾಮ-ವಾಗಿಯೇ
ಪರಿ-ಣಾಮ-ವಾಗುತ್ತಿತ್ತು
ಪರಿ-ಣಾಮ-ವುಂಟಾ
ಪರಿ-ಣಾಮ-ವೆಂದರೆ
ಪರಿ-ಣಾಮವೇ
ಪರಿ-ತ-ಪಿ-ಸು-ತ್ತಿದೆ
ಪರಿ-ತ್ಯಜಿಸಿ
ಪರಿ-ತ್ಯಾಗ
ಪರಿ-ತ್ಯಾಗಿ-ಯಾಗಿ
ಪರಿ-ತ್ರಾಣಾಯ
ಪರಿ-ಧಿಯ
ಪರಿ-ಧಿ-ಯಲ್ಲಿ
ಪರಿ-ಧಿ-ಯೊಳಗೆ
ಪರಿ-ಪಾಲಿಸಿ
ಪರಿ-ಪಾಲಿಸು
ಪರಿ-ಪುಷ್ಟ-ವಾಗಿ
ಪರಿ-ಪೂರ್ಣ
ಪರಿ-ಪೂರ್ಣತೆ
ಪರಿ-ಪೂರ್ಣ-ತೆ-ಗಾಗಿ
ಪರಿ-ಪೂರ್ಣ-ತೆ-ಯನ್ನು
ಪರಿ-ಪೂರ್ಣ-ತೆ-ಯಲ್ಲಿ
ಪರಿ-ಪೂರ್ಣ-ನಾಗಿ-ರು-ವನು
ಪರಿ-ಪೂರ್ಣ-ವಾಗಿ
ಪರಿ-ಪೂರ್ಣ-ವಾಗಿದೆ
ಪರಿ-ಪೂರ್ಣ-ವಾಗು-ವುದು
ಪರಿ-ಪೂರ್ಣ-ವಾದದ್ದು
ಪರಿ-ಪೂರ್ಣ-ವೆಂದು
ಪರಿ-ಭಾಷೆಯ
ಪರಿ-ಮಿತ-ವಾಗು-ತ್ತದೆ
ಪರಿ-ಮಿತಿ
ಪರಿ-ಮಿತಿ-ಯನ್ನು
ಪರಿ-ಮಿತಿ-ಯನ್ನೂ
ಪರಿ-ಮಿತಿ-ಯಾಚೆ
ಪರಿ-ಮಿತಿ-ಯುಳ್ಳದ್ದು
ಪರಿ-ಯಂತವೂ
ಪರಿ-ಯಂತಿ
ಪರಿ-ಯಾಗಿ-ತೋರು-ವುದು
ಪರಿ-ವರ್ತನ
ಪರಿ-ವರ್ತನ-ಶೀಲ-ವಾದದ್ದು
ಪರಿ-ವರ್ತನೆ
ಪರಿ-ವರ್ತ-ನೆ-ಯನ್ನು
ಪರಿ-ವರ್ತ-ನೆ-ಯಿಂದ
ಪರಿ-ವರ್ತಿತ-ವಾಗಿದೆ
ಪರಿ-ವರ್ತಿಸ-ಬಹು-ದೆಂದೂ
ಪರಿ-ವರ್ತಿ-ಸಿತು
ಪರಿ-ವರ್ತಿ-ಸುವ
ಪರಿ-ವರ್ತಿ-ಸು-ವಂತೆ
ಪರಿ-ವೃತ-ರಾದ-ವ-ರಂತೆ
ಪರಿ-ಶುದ್ದ-ವಾದ
ಪರಿ-ಶುದ್ದಾತ್ಮ
ಪರಿ-ಶುದ್ಧ
ಪರಿ-ಶುದ್ಧ-ಗೊಳಿ-ಸುತ್ತದೆ
ಪರಿ-ಶುದ್ಧತೆ
ಪರಿ-ಶುದ್ಧ-ತೆಗೆ
ಪರಿ-ಶುದ್ಧ-ತೆಯೇ
ಪರಿ-ಶುದ್ಧ-ನಾಗಿ-ರು-ವನು
ಪರಿ-ಶುದ್ಧ-ನಾದ
ಪರಿ-ಶುದ್ಧನು
ಪರಿ-ಶುದ್ಧ-ನೆಂಬ
ಪರಿ-ಶುದ್ಧ-ರಾಗದೆ
ಪರಿ-ಶುದ್ಧ-ರಾಗಿ
ಪರಿ-ಶುದ್ಧ-ರಾಗಿ-ರು-ವಂತೆಯೇ
ಪರಿ-ಶುದ್ಧ-ರಾಗಿ-ರು-ವುದು
ಪರಿ-ಶುದ್ಧರೋ
ಪರಿ-ಶುದ್ಧಳು
ಪರಿ-ಶುದ್ಧ-ವಾಗಿದೆ
ಪರಿ-ಶುದ್ಧ-ವಾಗಿ-ದ್ದರೆ
ಪರಿ-ಶುದ್ಧ-ವಾಗು-ವುದು
ಪರಿ-ಶುದ್ಧ-ವಾದ
ಪರಿ-ಶುದ್ಧ-ವಾದುದೋ
ಪರಿ-ಶುದ್ಧ-ವಾದು-ವೆಂದು
ಪರಿ-ಶುದ್ಧ-ವಿ-ಲ್ಲದೇ
ಪರಿ-ಶುದ್ಧಾತ್ಮ
ಪರಿ-ಶುದ್ಧಾತ್ಮ-ರಿ-ಗೆಲ್ಲಾ
ಪರಿ-ಶುದ್ಧಿ
ಪರಿ-ಶುದ್ಧಿ-ಯನ್ನು
ಪರಿ-ಶ್ರಮಕ್ಕೆ
ಪರಿ-ಶ್ರಮದ
ಪರಿ-ಶ್ರಮ-ದಿಂದ
ಪರಿ-ಶ್ರಮವು
ಪರಿ-ಷಸ್ವಜಾತೇ
ಪರಿ-ಷ್ಕೃತ-ವಾದ
ಪರಿ-ಸರ
ಪರಿ-ಸರ-ದಲ್ಲಿ
ಪರಿ-ಸ್ಥಿತಿ
ಪರಿ-ಸ್ಥಿತಿ-ಗಳನ್ನು
ಪರಿ-ಸ್ಥಿತಿ-ಗಳಿ-ಗನು-ಸಾರ-ವಾಗಿ
ಪರಿ-ಸ್ಥಿತಿ-ಯನ್ನು
ಪರಿ-ಸ್ಥಿತಿ-ಯಲ್ಲಿ
ಪರಿ-ಸ್ಥಿತಿಯೇ
ಪರಿ-ಹರಿಸ-ಬಹುದು
ಪರಿ-ಹರಿಸ-ಬೇಕು
ಪರಿ-ಹರಿಸ-ಲಾ-ರದು
ಪರಿ-ಹರಿಸಲಾ-ರವು
ಪರಿ-ಹ-ರಿ-ಸಲು
ಪರಿ-ಹ-ರಿಸಿ
ಪರಿ-ಹ-ರಿಸಿ-ಕೊಳ್ಳು-ವುದು
ಪರಿ-ಹರಿ-ಸು-ತ್ತಿದೆ
ಪರಿ-ಹರಿಸು-ವುದು
ಪರಿ-ಹರಿಸು-ವು-ದೆಂದರೆ
ಪರಿ-ಹಾರ
ಪರಿ-ಹಾ-ರಕ್ಕೆ
ಪರಿ-ಹಾ-ರ-ಗಳನ್ನು
ಪರಿ-ಹಾ-ರ-ಗೊಂಡಿವೆ
ಪರಿ-ಹಾ-ರ-ವನ್ನು
ಪರಿ-ಹಾ-ರ-ವಾಗು-ವು-ದಿಲ್ಲ
ಪರಿ-ಹಾ-ರವು
ಪರಿ-ಹಾ-ರವೂ
ಪರಿ-ಹಾ-ರ-ವೇನು
ಪರಿ-ಹಾ-ರೋಪಾ-ಯವೂ
ಪರೀಕ್ಷಿ-ಸಿದರೆ
ಪರೀಕ್ಷಿ-ಸಿ-ದಷ್ಟು
ಪರೀಕ್ಷಿಸು
ಪರೀಕ್ಷೆ
ಪರೀಕ್ಷೆಯ
ಪರೀಕ್ಷೆ-ಯನ್ನು
ಪರೆ-ಯನ
ಪರೆ-ಯ-ನಿಗೆ
ಪರೋಕ್ಷ-ವಾಗಿ
ಪರೋಪ-ಕಾರ
ಪರೋಪ-ಕಾರ-ವನ್ನು
ಪರ್ಯಂತರ
ಪರ್ಯಟನೆ
ಪರ್ಯಾಯ
ಪರ್ಯಾ-ಯ-ವೆಂದು
ಪರ್ಯಾ-ಲೋಚಿಸಿ
ಪರ್ಯಾ-ಲೋಚಿಸುವೆ
ಪರ್ಯಾ-ಲೋಚಿ-ಸೋಣ
ಪರ್ವತ
ಪರ್ವತ-ಗಳ
ಪರ್ವತ-ದೆ-ಡೆಗೆ
ಪರ್ವತಾ-ಕಾರದ
ಪರ್ವತೋಪಮ
ಪರ್ಶಿ-ಯ-ನರು
ಪರ್ಷಿಯಾ
ಪರ್ಷಿ-ಯಾ-ದಲ್ಲಿ
ಪರ್ಸಿ-ಯಾದ
ಪಲ್ಲವಿ
ಪಲ್ಲವಿ-ಯಾಗು-ವುದು
ಪಲ್ಲವಿಯೂ
ಪಲ್ಲವಿಯೇ
ಪವಡಿ-ಸಿದೆ
ಪವಾ-ಡ-ಸದೃಶ-ವಾದ
ಪವಿತ್ರ
ಪವಿತ್ರ-ಗೊಳಿ-ಸಿದ
ಪವಿತ್ರ-ತಮ
ಪವಿತ್ರ-ತಮ-ನಾದ
ಪವಿತ್ರ-ತ-ಮರು
ಪವಿತ್ರತೆ
ಪವಿತ್ರ-ತೆ-ಯನ್ನು
ಪವಿತ್ರ-ದಿನ
ಪವಿತ್ರ-ನಾಗ-ಬಲ್ಲೆ
ಪವಿತ್ರ-ವಾಗಿದೆ
ಪವಿತ್ರ-ವಾಗಿ-ರು-ವುದೋ
ಪವಿತ್ರ-ವಾಗಿ-ಲ್ಲವೋ
ಪವಿತ್ರ-ವಾ-ಣಿ-ಯನ್ನು
ಪವಿತ್ರ-ವಾದ
ಪವಿತ್ರ-ವಾದುದನ್ನೆಲ್ಲ
ಪವಿತ್ರ-ವಾದು-ದೆಂದು
ಪವಿತ್ರವೂ
ಪವಿತ್ರ-ವೆಂದು
ಪವಿತ್ರಾತ್ಮ-ನಾದ
ಪವಿತ್ರೋ
ಪಶುಪಕ್ಷಿ-ಗಳು
ಪಶು-ಪತಿ-ಮತಂ
ಪಶು-ಸಂತಾನ-ರಾಗಿ-ರುವ-ರೆಂದು
ಪಶ್ಚಾತ್ತಾಪ
ಪಶ್ಚಾತ್ತಾ-ಪಾಲಯ
ಪಶ್ಚಿಮ
ಪಶ್ಚಿಮಕ್ಕೆ
ಪಶ್ಚಿಮ-ಗಳನ್ನು
ಪಶ್ಚಿಮದ
ಪಶ್ಚಿಮ-ದಲ್ಲಿ
ಪಶ್ಚಿಮ-ದೇಶ-ಗಳು
ಪಶ್ಚಿಮವು
ಪಶ್ಯತಿ
ಪಶ್ಯತ್ಯನ್ಯಮೀಶಮಸ್ಯ
ಪಶ್ಯನ್
ಪಸರಿ-ಸಿದ
ಪಸರಿ-ಸಿದೆ
ಪಸರಿ-ಸಿ-ರುವ
ಪಾಂಚಜನ್ಯ
ಪಾಂಟಿಪೆ-ಕ್ಸ
ಪಾಂಡಿತ್ಯ
ಪಾಂಡಿ-ತ್ಯದ
ಪಾಂಡಿತ್ಯ-ದಿಂದ
ಪಾಂಡಿತ್ಯ-ದಿಂದಲೂ
ಪಾಂಡಿತ್ಯ-ವಲ್ಲ
ಪಾಂಡಿತ್ಯ-ವಿದೆ
ಪಾಂಡಿತ್ಯವೇ
ಪಾಂಬನ್ನನ್ನು
ಪಾಂಬನ್ನಿನ
ಪಾಠ-ಗಳಿಗೆ
ಪಾಠ-ದಲ್ಲಿ-ರುವ
ಪಾಠ-ವನ್ನು
ಪಾಡಿಗೆ
ಪಾಡು
ಪಾಡೇನು
ಪಾಣಿನಿ
ಪಾಣಿಪಾದಂ
ಪಾತ-ಕ-ವನ್ನು
ಪಾತ-ಕ-ವಿದ್ದರೆ
ಪಾತ್ರಕ್ಕೆ
ಪಾತ್ರ-ಗಳ
ಪಾತ್ರ-ಗಳನ್ನು
ಪಾತ್ರ-ಗಳಿಗೆ
ಪಾತ್ರ-ದಲ್ಲಿ
ಪಾತ್ರ-ರಾಗಲಿ
ಪಾತ್ರ-ರಾದ-ವ-ರಲ್ಲಿ
ಪಾತ್ರರು
ಪಾತ್ರ-ವನ್ನು
ಪಾತ್ರ-ವಾಗಿದೆ
ಪಾತ್ರ-ವಾದ
ಪಾತ್ರವು
ಪಾತ್ರವೇ
ಪಾತ್ರಾ
ಪಾತ್ರೆ-ಗಳೇ
ಪಾತ್ರೆಯೇ
ಪಾದಕ್ಕೆ
ಪಾದ-ಗಳನ್ನು
ಪಾದ-ಗಳಲ್ಲಿ
ಪಾದದಡಿ
ಪಾದದ-ಡಿ-ಯಲ್ಲಿ
ಪಾದದಿಂದಾದ
ಪಾದಧೂಳಿ
ಪಾದರಕ್ಷೆ
ಪಾದರ-ಕ್ಷೆ-ಯನ್ನು
ಪಾದರಜ
ಪಾದ-ಸ್ಥಳ-ದಲ್ಲಿ
ಪಾದ-ಸ್ಪರ್ಶ-ದಿಂದ
ಪಾದಾರ್ಪಣ
ಪಾದಿ-ಸು-ವು-ದರ
ಪಾದ್ರಿ
ಪಾದ್ರಿ-ಗಳ
ಪಾದ್ರಿ-ಗಳನ್ನು
ಪಾದ್ರಿ-ಗಳಿಗೆ
ಪಾದ್ರಿ-ಗಳು
ಪಾನ
ಪಾನ-ಮಾಡಿ
ಪಾನೀಯ-ಗಳು
ಪಾಪ
ಪಾಪ-ಕರ
ಪಾಪ-ಕಾರ್ಯ-ಗಳನ್ನು
ಪಾಪ-ಕಾರ್ಯ-ಗಳಿಂದ
ಪಾಪ-ಗಳ
ಪಾಪ-ಗಳನ್ನು
ಪಾಪ-ಗಳಿಗೂ
ಪಾಪದ
ಪಾಪ-ದಿಂದ
ಪಾಪ-ದೂರ
ಪಾಪ-ದೂರ-ನಾಗಿ-ರ-ಬೇಕು
ಪಾಪ-ದೂರನೂ
ಪಾಪ-ಭಾವ-ನೆ-ಯಿಂದ
ಪಾಪ-ಮಾಡು-ತ್ತಿರು-ವಾಗ
ಪಾಪ-ವನ್ನು
ಪಾಪ-ವಲ್ಲ
ಪಾಪ-ವಿದೆಯೋ
ಪಾಪ-ವಿರ-ಕೂಡದು
ಪಾಪ-ವಿಲ್ಲ-ವೆಂದು
ಪಾಪವು
ಪಾಪವೇ
ಪಾಪ-ಸೋಂಕು-ವುದು
ಪಾಪಾತ್ಮರೋ
ಪಾಪಿ
ಪಾಪಿ-ಗಳಾದರೆ
ಪಾಪಿ-ಗಳಿಂದ
ಪಾಪಿ-ಗಳು
ಪಾಪಿ-ಗಳೂ
ಪಾಪಿ-ಗಳೆಂದು
ಪಾಪಿ-ಗಳೊಂದಿಗೆ
ಪಾಪಿಗೂ
ಪಾಪಿಗೆ
ಪಾಪಿ-ಯಂತೆ
ಪಾಪಿ-ಯಾಗಲೀ
ಪಾಮರನಾ-ಗಲಿ
ಪಾಮ-ರನೂ
ಪಾಮ-ರರ
ಪಾಮರ-ರಿಗೆ
ಪಾರಂಪರ್ಯ-ವಾಗಿ
ಪಾರ-ಮಾರ್ಥಿಕ
ಪಾರಾಗ-ಬಲ್ಲ
ಪಾರಾಗ-ಬೇಕಾಗಿದೆ
ಪಾರಾಗ-ಬೇಕು
ಪಾರಾಗ-ಲಾ-ಗು-ವು-ದಿಲ್ಲ
ಪಾರಾಗ-ಲಾರಿರಿ
ಪಾರಾಗಲು
ಪಾರಾಗಿ
ಪಾರಾಗಿ-ದ್ದೇವೆ
ಪಾರಾಗಿ-ರು-ವನು
ಪಾರಾಗಿ-ರು-ವರು
ಪಾರಾಗು-ವಿರಿ
ಪಾರಾ-ಗುವು-ದ-ಕ್ಕಾಗಿ
ಪಾರಾ-ಗು-ವು-ದಕ್ಕೆ
ಪಾರಾಗು-ವುದು
ಪಾರಾಗು-ವೆವು
ಪಾರಾದ-ವನು
ಪಾರಿ
ಪಾರಿ-ಭಾಷಿಕ
ಪಾರಿ-ವಾ-ಳ-ದಂತೆ
ಪಾರು
ಪಾರು-ಗಾಣಿಸಿದೆ
ಪಾರು-ಮಾಡಲು
ಪಾರು-ಮಾಡಿ-ದರೆ
ಪಾರು-ಮಾಡು-ವರು
ಪಾರ್ಲಿಮೆಂಟ್
ಪಾರ್ವತಿ
ಪಾರ್ಶ್ವ
ಪಾರ್ಸಿ
ಪಾರ್ಸಿ-ಗಳಲ್ಲಿ
ಪಾರ್ಸಿ-ಗಳು
ಪಾರ್ಸಿ-ಯ-ವರು
ಪಾರ್ಸೀ
ಪಾಲಕ
ಪಾಲಾಗಿವೆ
ಪಾಲಾ-ದರೂ
ಪಾಲಿಗಾ-ದರೂ
ಪಾಲಿಗೆ
ಪಾಲಿನ
ಪಾಲಿನ-ದನ್ನು
ಪಾಲಿನ-ದನ್ನು-ಅದು
ಪಾಲಿನ-ದನ್ನೂ
ಪಾಲಿನದು
ಪಾಲಿ-ಯಲ್ಲಿ
ಪಾಲಿಸ-ಬೇಕೆಂದು
ಪಾಲಿಸಿ
ಪಾಲಿಸು-ವು-ದನ್ನು
ಪಾಲು
ಪಾಳು-ಬಿದ್ದಿದೆ
ಪಾಳೆ-ಗಾರರ
ಪಾಳೆಯ-ಗಾರ-ರಲ್ಲ
ಪಾಳೆಯ-ಗಾರರು
ಪಾವಕಃ
ಪಾವನ
ಪಾವನ-ಗೊಳಿ-ಸಲು
ಪಾವಿತ್ರ್ಯ
ಪಾವಿತ್ರ್ಯ-ಗಳು
ಪಾವಿತ್ರ್ಯ-ವನ್ನು
ಪಾಶವಿಕ
ಪಾಶುಪತ-ರಾಗಲೀ
ಪಾಶ್ಚಾತ್ಯ
ಪಾಶ್ಚಾತ್ಯ-ದೇಶ-ದಲ್ಲಿ-ರುವ
ಪಾಶ್ಚಾತ್ಯ-ನಾದರೋ
ಪಾಶ್ಚಾತ್ಯರ
ಪಾಶ್ಚಾತ್ಯ-ರನ್ನು
ಪಾಶ್ಚಾತ್ಯ-ರಲ್ಲಿ
ಪಾಶ್ಚಾತ್ಯ-ರಾಗ-ಲಾರಿರಿ
ಪಾಶ್ಚಾತ್ಯ-ರಾಗ-ಲಾರೆವು
ಪಾಶ್ಚಾತ್ಯ-ರಿಂದ
ಪಾಶ್ಚಾತ್ಯ-ರಿಂದು
ಪಾಶ್ಚಾತ್ಯ-ರಿಗೂ
ಪಾಶ್ಚಾತ್ಯ-ರಿಗೆ
ಪಾಶ್ಚಾತ್ಯರು
ಪಾಶ್ಚಾತ್ಯ-ವಾದು-ದೆಲ್ಲ-ವನ್ನೂ
ಪಾಶ್ಚಾತ್ಯವೋ
ಪಾಶ್ಚಿ-ಮಾತ್ಯ
ಪಾಶ್ಯಾತ್ಯ
ಪಾಶ್ಯಾ-ತ್ಯರ
ಪಾಶ್ಯಾ-ತ್ಯ-ರಿಗೆ
ಪಾಷಂಡರೆ
ಪಿ
ಪಿಂಡಾಂಡ-ವನ್ನು
ಪಿಂಡಾಂಡ-ವೆಂದು
ಪಿತ
ಪಿತರೇ
ಪಿತಾ-ಮಹ-ನ-ಡೆಗೆ
ಪಿತಾ-ಮಹ-ನಾದ
ಪಿತೃ-ಗಳು
ಪಿತೃ-ಪೂಜೆಯ
ಪಿಪಾಸೆ-ಯಿಂದ
ಪಿಪ್ಪಲಂ
ಪಿಶಾಚಿ
ಪಿಶಾಚಿಯ
ಪಿಶಾಚಿ-ಯಾಗಿ
ಪೀಠ
ಪೀಡ-ನೆಯು
ಪೀಡಿತ-ರಾಗಿ-ರುವು-ದನ್ನೂ
ಪೀಡಿಸಿ-ಕೊಳ್ಳು-ತ್ತಿರು
ಪೀಡಿ-ಸಿದರೆ
ಪೀಡಿ-ಸುತ್ತದೆ
ಪೀಡಿಸುತ್ತಿತ್ತು
ಪೀಡಿ-ಸು-ತ್ತಿದ್ದ
ಪೀಡಿ-ಸು-ವರು
ಪೀಡಿ-ಸು-ವು-ದಕ್ಕೆ
ಪೀಡೆ-ಯಿಂದ
ಪುಂಸಃ
ಪುಚ್ಛ-ಭೂತ-ವಾಗಿ
ಪುಟ-ಬಾ-ಲ್
ಪುಟ್ಟ
ಪುಡಿಪುಡಿ-ಯಾಗಿ
ಪುಡಿಪುಡಿ-ಯಾಗು-ವು-ದನ್ನು
ಪುಡಿಪುಡಿ-ಯಾಗು-ವುದು
ಪುಣ್ಯ
ಪುಣ್ಯ-ಕಥೆ-ಯನ್ನು
ಪುಣ್ಯ-ಕಾರ್ಯ-ದಿಂದ
ಪುಣ್ಯ-ದಿನ-ವ-ಲ್ಲವೆ
ಪುಣ್ಯ-ದೇಶವೇ
ಪುಣ್ಯ-ಭೂಮಿ
ಪುಣ್ಯ-ಭೂಮಿ-ಯನ್ನು
ಪುಣ್ಯ-ಭೂಮಿ-ಯಲ್ಲಿ
ಪುಣ್ಯ-ಭೂಮಿ-ಯಾದ
ಪುಣ್ಯ-ಭೂಮಿ-ಯೆಂದು
ಪುಣ್ಯ-ವೆಂದು
ಪುಣ್ಯ-ಶಾಲಿ-ಯಾಗಿದೆ
ಪುಣ್ಯ-ಸ್ಥಳಕ್ಕೆ
ಪುಣ್ಯಾ-ತ್ಮರ
ಪುತ್ರ
ಪುತ್ರನ
ಪುತ್ರ-ರಿರಾ
ಪುತ್ರ-ರಿ-ರು-ವುದೇ
ಪುತ್ರರು
ಪುತ್ರರೂ
ಪುತ್ರರೆ
ಪುತ್ರರೇ
ಪುನಃ
ಪುನರಭಿ-ನಯ
ಪುನರಭ್ಯುದ-ಯದ
ಪುನರಾವರ್ತನ
ಪುನರಾವಿ-ಷ್ಕಾರ
ಪುನರಾ-ವೃತ್ತಿ-ಗೊಳ್ಳು-ತ್ತದೆ
ಪುನರಾ-ವೃತ್ತಿ-ಯಾಗ-ಲೇ-ಬೇಕು
ಪುನರು-ಜ್ಜೀವ-ನ-ಗೊಳಿ-ಸು-ವು-ದಕ್ಕೆ
ಪುನರು-ತ್ಥಾನ
ಪುನರು-ತ್ಥಾನ-ಕ್ಕಾಗಿ
ಪುನರು-ತ್ಥಾ-ನಕ್ಕೆ
ಪುನರು-ತ್ಥಾನ-ಗಳು
ಪುನರು-ತ್ಥಾ-ನದ
ಪುನರು-ತ್ಥಾ-ನವು
ಪುನರು-ದ್ಧಾರ
ಪುನರು-ದ್ಧಾರ-ಕ್ಕಾಗಿ
ಪುನರು-ದ್ಧಾ-ರದ
ಪುನರು-ದ್ಧಾ-ರವು
ಪುನರ್ಜನ್ಮ
ಪುನರ್ಜನ್ಮ-ವನ್ನು
ಪುನಶ್ಚೇ-ತನ-ಗೊಳಿಸುವ
ಪುನುರುದ್ಧಾರ-ಕ್ಕಿಂತ
ಪುಮಾ-ನಸಿ
ಪುರ-ಜ-ನರು
ಪುರ-ಜನರೇ
ಪುರನಿ-ವಾ-ಸಿ-ಗಳೇ
ಪುರಷ-ರಾಗಿ-ರ-ಬೇಕು
ಪುರಾಣ
ಪುರಾಣಕ್ಕೂ
ಪುರಾಣಕ್ಕೆ
ಪುರಾಣ-ಗಳ
ಪುರಾಣ-ಗಳನ್ನು
ಪುರಾಣ-ಗಳಲ್ಲಿ
ಪುರಾಣ-ಗಳ-ಲ್ಲಿದೆ
ಪುರಾಣ-ಗಳ-ಲ್ಲಿಯೂ
ಪುರಾಣ-ಗಳ-ಲ್ಲಿ-ರುವ
ಪುರಾಣ-ಗಳ-ಲ್ಲಿವೆ
ಪುರಾಣ-ಗಳಿಗೆ
ಪುರಾಣ-ಗಳಿವೆ
ಪುರಾಣ-ಗಳು
ಪುರಾಣ-ಗಳೂ
ಪುರಾಣ-ಗಳೆಲ್ಲಾ
ಪುರಾಣ-ಗಳೇನೋ
ಪುರಾಣದ
ಪುರಾಣ-ದಲ್ಲಿ
ಪುರಾಣ-ದಲ್ಲಿಯೂ
ಪುರಾಣ-ದಷ್ಟೇ
ಪುರಾಣ-ವನ್ನು
ಪುರಾಣ-ವನ್ನೂ
ಪುರಾಣವು
ಪುರಾಣವೂ
ಪುರಾ-ತನ
ಪುರಾ-ತನ-ಕಾಲದ
ಪುರಾ-ತನ-ಕಾಲ-ದಿಂದಲೂ
ಪುರಾ-ತನ-ವಾದುದಾ-ದರೂ
ಪುರಾ-ತನ-ವಾದುದೇ
ಪುರಾ-ತ-ನವೂ
ಪುರಾ-ತನ-ವೆಂದು
ಪುರು-ತ್ಥಾ-ನದ
ಪುರುಷ
ಪುರುಷಃ
ಪುರು-ಷನ
ಪುರು-ಷ-ನ-ನ್ನಾ-ಗಲೀ
ಪುರು-ಷ-ನಾ-ಗಲೀ
ಪುರು-ಷ-ನಿರ್ಮಾಣ
ಪುರು-ಷನೂ
ಪುರು-ಷ-ರ-ನೇ-ಕ-ರಿಗೆ
ಪುರು-ಷ-ರನ್ನು
ಪುರು-ಷ-ರಲ್ಲ
ಪುರು-ಷ-ರಲ್ಲಿ
ಪುರು-ಷ-ರಾದ
ಪುರು-ಷ-ರಿಂದ
ಪುರು-ಷ-ರಿಗೆ
ಪುರು-ಷರು
ಪುರು-ಷರೂ
ಪುರು-ಷ-ರೆಂದು
ಪುರು-ಷರೇ
ಪುರು-ಷ-ರೊಬ್ಬರು
ಪುರು-ಷ-ಶಿಶು-ಗಳಿಗೂ
ಪುರು-ಷ-ಸಿಂಹ-ರನ್ನು
ಪುರು-ಷ-ಸಿಂಹ-ರಾಗಿ
ಪುರು-ಷಾರ್ಥ-ರೂಪ-ವಾದ
ಪುರು-ಷಾರ್ಥ-ವೆಂದು
ಪುರುಷೋ
ಪುರು-ಸೊತ್ತೆಲ್ಲಿ
ಪುರೋ-ಹಿ-ತ-ನಿಗೆ
ಪುರೋ-ಹಿ-ತರ
ಪುರೋ-ಹಿ-ತ-ರಿಂದ
ಪುರೋ-ಹಿ-ತರು
ಪುರೋ-ಹಿ-ತ-ಷಾ-ಹಿಯೂ
ಪುಷ್ಟಿ
ಪುಷ್ಟಿ-ಗೊಳಿಸ-ಬೇಕು
ಪುಷ್ಟಿ-ಗೊಳಿಸು-ವಿರಿ
ಪುಷ್ಟಿ-ಯನ್ನು
ಪುಷ್ಟಿ-ಯಾಗಿ-ದ್ದರೆ
ಪುಷ್ಟಿ-ಯಾಗಿ-ರು-ವುದೋ
ಪುಷ್ಪ-ಗಳ-ನ್ನೆಲ್ಲಾ
ಪುಸ್ತಕ
ಪುಸ್ತಕ-ಗಳನ್ನು
ಪುಸ್ತಕ-ಗಳಲ್ಲಿ
ಪುಸ್ತಕ-ಗಳು
ಪುಸ್ತ-ಕದ
ಪುಸ್ತಕ-ದಲ್ಲಿ
ಪುಸ್ತಕ-ಭಂಡಾರ-ವನ್ನೇ
ಪುಸ್ತಕ-ವನ್ನು
ಪುಸ್ತ-ಕವು
ಪುಸ್ತಕ-ಶಕ್ತಿ
ಪುಸ್ತ-ಕಾಲ-ಯ-ಗಳೇ
ಪೂಜ-ಕನು
ಪೂಜ-ಕರ
ಪೂಜಾ
ಪೂಜಾ-ಯೋಗ್ಯನು
ಪೂಜಾರಿ-ಗಳನ್ನು
ಪೂಜಾರಿ-ವೃಂದ-ವೆಲ್ಲ
ಪೂಜಾರ್ಹರಾಗುತ್ತಾರೆ
ಪೂಜಾರ್ಹ-ವಾದ
ಪೂಜಿ-ಸದೆ
ಪೂಜಿಸ-ಬಹುದು
ಪೂಜಿಸ-ಬೇ-ಕಾದ
ಪೂಜಿಸ-ಬೇಕು
ಪೂಜಿ-ಸ-ಲಾರಿರಿ
ಪೂಜಿ-ಸಲಿ
ಪೂಜಿ-ಸಲಿ-ಚ್ಛಿ-ಸುವ-ವನು
ಪೂಜಿ-ಸಲು
ಪೂಜಿಸಲ್ಪಡುತ್ತಿ-ರುವಳು
ಪೂಜಿಸ-ಲ್ಪ-ಡುವ
ಪೂಜಿಸಿ-ದರೂ
ಪೂಜಿಸಿಯೂ
ಪೂಜಿಸುತ್ತಿ-ರುವ
ಪೂಜಿಸುತ್ತಿ-ರು-ವರು
ಪೂಜಿ-ಸುವ
ಪೂಜಿ-ಸು-ವನು
ಪೂಜಿ-ಸು-ವರು
ಪೂಜಿಸು-ವರೊ
ಪೂಜಿ-ಸುವ-ವರು
ಪೂಜಿ-ಸು-ವು-ದಕ್ಕೆ
ಪೂಜಿ-ಸು-ವು-ದಿಲ್ಲ
ಪೂಜಿಸು-ವುದು
ಪೂಜಿಸು-ವೆವು
ಪೂಜೆ
ಪೂಜೆ-ಗಳಲ್ಲಿ
ಪೂಜೆಗೆ
ಪೂಜೆ-ಗೊಂಡ
ಪೂಜೆ-ಮಾಡ-ಬೇಕೆಂದು
ಪೂಜೆ-ಮಾಡುವ
ಪೂಜೆಯ
ಪೂಜೆ-ಯಂತೆ
ಪೂಜೆ-ಯನ್ನು
ಪೂಜೆ-ಯನ್ನೂ
ಪೂಜೆ-ಯ-ಲ್ಲದೆ
ಪೂಜೆ-ಯಲ್ಲೂ
ಪೂಜೆ-ಯಿಂದ
ಪೂಜೆಯು
ಪೂಜೆ-ಯೆಲ್ಲಾ
ಪೂಜೆಯೇ
ಪೂಜ್ಯ
ಪೂಜ್ಯನೂ
ಪೂಜ್ಯ-ಪಾದ-ರಿಗೆ
ಪೂಜ್ಯ-ಪಾ-ದರೆ
ಪೂಜ್ಯ-ಭಾವ
ಪೂಜ್ಯ-ರಾದ
ಪೂಜ್ಯರೆ
ಪೂಜ್ಯರೇ
ಪೂಜ್ಯ-ಸ್ಥಾನ-ದಲ್ಲಿ
ಪೂರಕ
ಪೂರಕ-ವಾಗಿ
ಪೂರಾ-ತ್
ಪೂರಿತ
ಪೂರೈಕೆ
ಪೂರೈಸ-ಬಲ್ಲದು
ಪೂರೈಸ-ಬಹುದು
ಪೂರೈಸ-ಬೇಕಾಗಿದೆ
ಪೂರೈಸ-ಲಾರೆ
ಪೂರೈ-ಸಲು
ಪೂರೈ-ಸಿದನು
ಪೂರೈ-ಸು-ತ್ತಿದ್ದರು
ಪೂರೈ-ಸುವ
ಪೂರೈ-ಸು-ವು-ದಕ್ಕೆ
ಪೂರೈಸುವುದರೊಳಗೆ
ಪೂರೈಸು-ವೆನು
ಪೂರ್ಣ
ಪೂರ್ಣತೆ
ಪೂರ್ಣ-ತೆಗೆ
ಪೂರ್ಣ-ತೆಯ
ಪೂರ್ಣ-ತೆ-ಯನ್ನು
ಪೂರ್ಣ-ತೆ-ಯಲ್ಲಿ
ಪೂರ್ಣ-ತೆ-ಯಿಲ್ಲ
ಪೂರ್ಣ-ನಾಗು-ತ್ತಾನೆ
ಪೂರ್ಣ-ಪ್ರ-ಭಾವ-ವನ್ನು
ಪೂರ್ಣ-ಭಕ್ತಿಯು
ಪೂರ್ಣ-ರೂಪ-ವನ್ನು
ಪೂರ್ಣ-ವಾಗ-ಬೇ-ಕಾದರೆ
ಪೂರ್ಣ-ವಾಗಿ
ಪೂರ್ಣ-ವಾಗಿಲ್ಲ
ಪೂರ್ಣ-ವಾದ
ಪೂರ್ಣ-ವಾ-ಯಿತು
ಪೂರ್ಣವೂ
ಪೂರ್ಣಾ-ನಂದದ
ಪೂರ್ತಿ
ಪೂರ್ತಿ-ಗೊಳಿಸು-ತ್ತಾರೆ
ಪೂರ್ತಿ-ಯಾಗಿ
ಪೂರ್ವ
ಪೂರ್ವ-ಕರ್ಮ-ವನ್ನು
ಪೂರ್ವ-ಕ-ವಾಗಿ
ಪೂರ್ವ-ಕ-ವಾಗಿದೆ
ಪೂರ್ವ-ಕಾಲದ
ಪೂರ್ವ-ಕಾಲ-ದಲ್ಲಿ
ಪೂರ್ವಕ್ಕೆ
ಪೂರ್ವಜ
ಪೂರ್ವ-ಜ-ನ್ಮದ
ಪೂರ್ವ-ಜರ
ಪೂರ್ವ-ಜ-ರಿಂದ
ಪೂರ್ವ-ಜರು
ಪೂರ್ವದ
ಪೂರ್ವ-ದಲ್ಲಿ
ಪೂರ್ವ-ದಿಂದ
ಪೂರ್ವ-ದಿಂದಲೂ
ಪೂರ್ವ-ದೇಶ-ಗಳಿಗೆ
ಪೂರ್ವ-ದೇಶ-ದಲ್ಲಿ
ಪೂರ್ವ-ಪಕ್ಷ-ವನ್ನು
ಪೂರ್ವ-ಭಾಗ
ಪೂರ್ವ-ರಾಷ್ಟ್ರ-ಗಳನ್ನು
ಪೂರ್ವ-ಸಿದ್ಧತೆ
ಪೂರ್ವಾ-ಚಾರ
ಪೂರ್ವಾ-ಚಾರ-ಪರಾ-ಯಣ-ರಲ್ಲಿ
ಪೂರ್ವಾ-ಚಾರಿ-ಗಳನ್ನು
ಪೂರ್ವಾ-ಚಾರ್ಯರು
ಪೂರ್ವಾ-ಚಾರ್ಯಾಃ
ಪೂರ್ವಾ-ಪರ-ವನ್ನು
ಪೂರ್ವಿ-ಕರ
ಪೂರ್ವಿಕ-ರಲ್ಲಿ
ಪೂರ್ವಿ-ಕರ-ಲ್ಲಿ-ರುವ
ಪೂರ್ವಿಕ-ರಿಂದ
ಪೂರ್ವಿಕ-ರಿಗೂ
ಪೂರ್ವಿಕ-ರಿಗೆ
ಪೂರ್ವಿ-ಕರು
ಪೂರ್ವಿ-ಕರೂ
ಪೂರ್ವಿಕರೋ
ಪೂರ್ವೇ
ಪೂರ್ವೋಕ್ತ
ಪೃಥಕ್ತ್ವ-ವೆಲ್ಲಿ
ಪೃಥಗ್ಭಾವ
ಪೃಥಿವ್ಯಾ-ಮಧಿಜಾಯತೇ
ಪೃಥ್ವಿಯ
ಪೃಥ್ವಿಯ-ನ್ನೆಲ್ಲಾ
ಪೃಥ್ವಿಯ-ನ್ನೇ
ಪೃಥ್ವಿ-ಯಲ್ಲಿ
ಪೆಟ್ಟಿಗೆ
ಪೆಟ್ಟಿಗೆ-ಯಂತೆ
ಪೆಟ್ಟಿಗೆ-ಯಲ್ಲಿ
ಪೆಟ್ಟಿಗೆ-ಯಲ್ಲಿ-ಡ-ದಿ-ದ್ದರೆ
ಪೆಟ್ಟಿಗೆ-ಯಲ್ಲಿ-ರು-ತ್ತಾ-ನೆಂದು
ಪೆಟ್ಟಿನ
ಪೆಟ್ಟಿ-ನಂತೆ
ಪೆಟ್ಟು
ಪೇಕ್ಷೆಯೂ
ಪೈಥಾಗೊರ-ಸ್
ಪೈಪೋಟಿ
ಪೊರೆ-ಗಳಚಿ
ಪೊಳ್ಳು
ಪೋಟಾಪೋಟಿ
ಪೋರ್ಚುಗೀ-ಸರು
ಪೋಲೀ-ಸರ
ಪೋಲೀಸ-ರನ್ನು
ಪೋಲೀಸು
ಪೋಲೀ-ಸ್
ಪೋಷಣೆ-ಯನ್ನು
ಪೋಷಿಸಿ
ಪೌಂಡು-ಗಳನ್ನು
ಪೌತ್ರ-ರಿಗೆ
ಪೌತ್ರರು
ಪೌತ್ರರೂ
ಪೌರ-ರಾದ
ಪೌರರು
ಪೌರ-ಸ-ಭೆಗೆ
ಪೌರಸ್ತ್ಯ-ರಿಗೆ
ಪೌರಾಣಿಕ
ಪೌರಾಣಿ-ಕರು
ಪೌರಾಣಿ-ಕರೂ
ಪೌರುಷವಂತರ-ನ್ನಾಗಿ
ಪೌರುಷ-ವಾಗಲಿ
ಪೌರೋ-ಹಿ-ತ್ಯವೂ
ಪ್ರಕ-ಟ-ವಾಗಿದೆ
ಪ್ರಕ-ಟ-ವಾಗು-ತ್ತದೆ
ಪ್ರಕ-ಟಿಸ-ಬಹು-ದಾ-ಗಿತ್ತು
ಪ್ರಕಾರ
ಪ್ರಕಾರದ
ಪ್ರಕಾರ-ವಾಗಿ
ಪ್ರಕಾರವೇ
ಪ್ರಕಾರೇಣ
ಪ್ರಕಾಶ
ಪ್ರಕಾಶ-ಕ-ರಿಗೆ
ಪ್ರಕಾಶ-ಕರು
ಪ್ರಕಾಶಕ್ಕೆ
ಪ್ರಕಾಶ-ಗೊಳಿ-ಸ-ಲಿಲ್ಲ
ಪ್ರಕಾಶ-ಗೊಳಿಸು-ವುದು
ಪ್ರಕಾಶ-ಗೊಳ್ಳು-ತ್ತವೆ
ಪ್ರಕಾಶ-ಪಡಿ-ಸದೆ
ಪ್ರಕಾಶ-ಪಡಿ-ಸುವ
ಪ್ರಕಾಶ-ಮಾಡು-ವು-ದಕ್ಕೆ
ಪ್ರಕಾಶ-ಮಾ-ನ-ವಾಗು-ವನು
ಪ್ರಕಾಶ-ಮಾ-ನ-ವಾದ
ಪ್ರಕಾಶ-ಮಾ-ನ-ವಾದುದು
ಪ್ರಕಾಶ-ವಾಗು-ತ್ತದೆ
ಪ್ರಕಾಶಿತ-ವಾಗು-ತ್ತದೆ
ಪ್ರಕಾಶಿತ-ವಾದವು
ಪ್ರಕಾಶಿ-ತವೂ
ಪ್ರಕಾಶಿಸಲಾರ
ಪ್ರಕಾಶಿಸ-ಲಾ-ರದು
ಪ್ರಕಾಶಿ-ಸುತ್ತದೆ
ಪ್ರಕಾಶಿ-ಸು-ತ್ತಿದೆ
ಪ್ರಕಾಶಿ-ಸು-ತ್ತಿದೆಯೋ
ಪ್ರಕಾಶಿಸು-ವ-ರೆಂದು
ಪ್ರಕಾಶಿ-ಸು-ವು-ದಿಲ್ಲ
ಪ್ರಕಾಶಿಸು-ವುದು
ಪ್ರಕಾಶಿಸುವು-ದೆಂದೂ
ಪ್ರಕಾಶಿಸು-ವುದೋ
ಪ್ರಕಾಶಿಸು-ವುವು
ಪ್ರಕೃತಿ
ಪ್ರಕೃತಿ-ಗಿಂತ
ಪ್ರಕೃತಿಗೆ
ಪ್ರಕೃತಿ-ದೂ-ರನು
ಪ್ರಕೃತಿ-ಬಂಧನ-ದಿಂದ
ಪ್ರಕೃತಿಯ
ಪ್ರಕೃತಿ-ಯಂತೆ
ಪ್ರಕೃತಿ-ಯನ್ನು
ಪ್ರಕೃತಿ-ಯ-ಬಲಾತ್ಕಾರ-ದಿಂದಲೋ
ಪ್ರಕೃತಿ-ಯಲ್ಲಿ
ಪ್ರಕೃತಿ-ಯ-ಲ್ಲಿ-ರು-ವುವು
ಪ್ರಕೃತಿ-ಯಾಚೆ
ಪ್ರಕೃತಿ-ಯಿಂದ
ಪ್ರಕೃತಿ-ಯಿಂದಲೂ
ಪ್ರಕೃತಿಯು
ಪ್ರಕೃತಿ-ಯೆಂದು
ಪ್ರಕೃತಿ-ಯೆಲ್ಲಾ
ಪ್ರಕೃತಿಯೇ
ಪ್ರಕೃತೀನಾಂ
ಪ್ರಕೃತ್ಯಾ
ಪ್ರಕೃತ್ಯಾ-ಪೂರಾ-ತ್
ಪ್ರಕ್ರಿಯೆ-ಯನ್ನು
ಪ್ರಕ್ಷಿಪ್ತ
ಪ್ರಖರ
ಪ್ರಖ್ಯಾತ
ಪ್ರಖ್ಯಾ-ತ-ನ-ನ್ನಾಗಿ
ಪ್ರಖ್ಯಾ-ತ-ನಾದ
ಪ್ರಖ್ಯಾ-ತ-ರಾಗ-ಬಲ್ಲೆವು
ಪ್ರಖ್ಯಾ-ತ-ರಾಗಲು
ಪ್ರಖ್ಯಾ-ತ-ರಾಗಿ-ದ್ದರೋ
ಪ್ರಖ್ಯಾ-ತ-ರಾಗಿ-ರು-ವರು
ಪ್ರಖ್ಯಾ-ತ-ರಾಗಿ-ರು-ವಾಗ
ಪ್ರಖ್ಯಾ-ತ-ರಾಗು-ವರು
ಪ್ರಖ್ಯಾ-ತ-ರಾದ
ಪ್ರಖ್ಯಾ-ತ-ವಾಗ
ಪ್ರಖ್ಯಾ-ತಿಯ
ಪ್ರಗತಿ
ಪ್ರಗತಿ-ಗಳು
ಪ್ರಗ-ತಿಗೆ
ಪ್ರಗತಿ-ಪರ-ವಾದುದು
ಪ್ರಗತಿ-ಮಾರ್ಗಕ್ಕೆ
ಪ್ರಗತಿ-ಮಾರ್ಗ-ದಲ್ಲಿ
ಪ್ರಗತಿ-ಮಾರ್ಗ-ವನ್ನು
ಪ್ರಗ-ತಿಯ
ಪ್ರಗತಿ-ಯನ್ನು
ಪ್ರಗತಿ-ಯಲ್ಲಿ
ಪ್ರಗ-ತಿಯು
ಪ್ರಗ-ತಿಯೂ
ಪ್ರಚಂಡ
ಪ್ರಚಂಡ-ವಾದ
ಪ್ರಚಂಡ-ವಾದುದು
ಪ್ರಚಂಡ-ಶಕ್ತಿ
ಪ್ರಚಲಿತ-ವಾಗಿದೆ
ಪ್ರಚಲಿತ-ವಾಗಿ-ರುವ
ಪ್ರಚಲಿತ-ವಾದುವು
ಪ್ರಚಲಿತ-ವಾ-ಯಿತು
ಪ್ರಚಲಿತ-ವಿದ್ದ
ಪ್ರಚಲಿತ-ವಿ-ರುವ
ಪ್ರಚಲಿತ-ವಿರು-ವು-ದನ್ನು
ಪ್ರಚಾರ
ಪ್ರಚಾರಕ
ಪ್ರಚಾರ-ಕ-ನಂತೆ
ಪ್ರಚಾರ-ಕ-ನೆಂಬುದು
ಪ್ರಚಾರ-ಕನೇ
ಪ್ರಚಾರ-ಕರ
ಪ್ರಚಾರ-ಕರು
ಪ್ರಚಾರ-ಕಾರ್ಯ
ಪ್ರಚಾ-ರಕ್ಕೆ
ಪ್ರಚಾರ-ಗಳು
ಪ್ರಚಾರ-ದಲ್ಲಿ
ಪ್ರಚಾರ-ದಲ್ಲಿ-ರುವ
ಪ್ರಚಾರ-ಮಾಡ-ಬೇಕಾಗಿದೆ
ಪ್ರಚಾರ-ಮಾಡ-ಬೇಕು
ಪ್ರಚಾರ-ಮಾಡಲು
ಪ್ರಚಾರ-ಮಾಡಿ
ಪ್ರಚಾರ-ಮಾಡಿ-ದನು
ಪ್ರಚಾರ-ಮಾಡಿ-ದರು
ಪ್ರಚಾರ-ಮಾಡಿ-ದಷ್ಟೂ
ಪ್ರಚಾರ-ಮಾಡು-ತ್ತಿದ್ದರೋ
ಪ್ರಚಾರ-ಮಾಡು-ವಂತಹ
ಪ್ರಚಾರ-ಮಾಡು-ವ-ವರು
ಪ್ರಚಾರ-ಮಾಡು-ವು-ದ-ಕ್ಕಾಗಿ
ಪ್ರಚಾರ-ಮಾಡು-ವು-ದಕ್ಕೆ
ಪ್ರಚಾರ-ವನ್ನು
ಪ್ರಚಾರ-ವಾಗ-ಬೇಕು
ಪ್ರಚಾರ-ವಾಗಲಿ
ಪ್ರಚಾರ-ವಾಗುತ್ತ
ಪ್ರಚಾರ-ವಾಗು-ವು-ದಕ್ಕೆ
ಪ್ರಚಾರ-ವಾದ
ಪ್ರಚಾರ-ವಾ-ಯಿತು
ಪ್ರಚಾರವೂ
ಪ್ರಚೋದಿ-ಸದೆ
ಪ್ರಚೋದಿ-ಸಿದ
ಪ್ರಚೋದಿ-ಸುತ್ತದೆ
ಪ್ರಚೋದಿಸುತ್ತಿತ್ತು
ಪ್ರಚೋದಿಸುತ್ತಿರು-ವುವು
ಪ್ರಚೋದಿಸು-ವುದು
ಪ್ರಚ್ಛನ್ನ-ಬೌದ್ಧ-ರೆಂದು
ಪ್ರಜಯಾ
ಪ್ರಜಾ-ಭಿಪ್ರಾಯ-ವನ್ನು
ಪ್ರಜೆ-ಗಳನ್ನು
ಪ್ರಜೆ-ಗಳಲ್ಲಿ
ಪ್ರಜೆ-ಗಳಾದ
ಪ್ರಜೆ-ಗಳೂ
ಪ್ರಜ್ಞಾ-ತೀತ
ಪ್ರಜ್ಞಾ-ಪೂರ್ವ-ಕ-ವಾಗಿ
ಪ್ರಜ್ಞೆ
ಪ್ರಜ್ಞೆಯ
ಪ್ರಜ್ಞೆ-ಯಲ್ಲ
ಪ್ರಜ್ಞೆ-ಯ-ಲ್ಲಿದೆ
ಪ್ರಜ್ಞೆ-ಯ-ಲ್ಲಿರು-ವುದು
ಪ್ರಜ್ಞೆ-ಯಿಂದಲೂ
ಪ್ರಜ್ಞೆ-ಯಿಲ್ಲ
ಪ್ರಜ್ಞೆಯು
ಪ್ರಜ್ಞೆ-ಯೆಂಬುದು
ಪ್ರಜ್ವಲಿ-ಸುವ
ಪ್ರಣಯಿ
ಪ್ರಣಯಿ-ನಿಯ
ಪ್ರಣೀತ
ಪ್ರತಾಪ-ಶಾಲಿ
ಪ್ರತಿ
ಪ್ರತಿ-ಕ್ರಿ-ಯಾ-ವ-ಸ್ಥೆಯೇ
ಪ್ರತಿ-ಕ್ರಿಯೆ
ಪ್ರತಿ-ಕ್ರಿಯೆಯ
ಪ್ರತಿ-ಕ್ರಿಯೆ-ಯನ್ನು
ಪ್ರತಿ-ಕ್ರಿಯೆ-ಯಲ್ಲ
ಪ್ರತಿ-ಕ್ರಿಯೆ-ಯಾಗು-ತ್ತದೆ
ಪ್ರತಿ-ಕ್ರಿಯೆ-ಯಿಂದ
ಪ್ರತಿ-ಕ್ರಿಯೆ-ಯಿಂದಲೇ
ಪ್ರತಿ-ಕ್ರಿಯೆಯೇ
ಪ್ರತಿ-ಕ್ಷಣ
ಪ್ರತಿ-ಕ್ಷಣವೂ
ಪ್ರತಿ-ಗ-ಮನ
ಪ್ರತಿ-ಗಳು
ಪ್ರತಿ-ಚಲಂತಿ
ಪ್ರತಿ-ದಿನ
ಪ್ರತಿ-ದಿ-ನವೂ
ಪ್ರತಿ-ಧ್ವನಿ-ಗಳಾಗಿ-ರು-ವು-ದನ್ನು
ಪ್ರತಿ-ಧ್ವನಿ-ಗೊಂಡು
ಪ್ರತಿ-ಧ್ವನಿ-ತ-ವಾದ
ಪ್ರತಿ-ನಿಧಿ
ಪ್ರತಿ-ನಿಧಿ-ಗಳ
ಪ್ರತಿ-ನಿಧಿ-ಗಳನ್ನೂ
ಪ್ರತಿ-ನಿಧಿ-ಗಳಾಗಿ-ರು-ವರು
ಪ್ರತಿ-ನಿಧಿ-ಗಳು
ಪ್ರತಿ-ನಿಧಿ-ಗಳೆ
ಪ್ರತಿ-ನಿಧಿಸಿ
ಪ್ರತಿ-ನಿಧಿ-ಸುತ್ತಾರೆಯೋ
ಪ್ರತಿ-ನಿಧಿ-ಸುತ್ತಿ-ರುವ
ಪ್ರತಿ-ನಿಧಿ-ಸುವ
ಪ್ರತಿ-ನಿಧಿ-ಸು-ವು-ದಕ್ಕೆ
ಪ್ರತಿ-ನಿಧಿ-ಸು-ವು-ದಿಲ್ಲ
ಪ್ರತಿ-ಪಕ್ಷದ-ವರ
ಪ್ರತಿ-ಪಾದ-ನೆಗೆ
ಪ್ರತಿ-ಪಾದಿತ-ವಾಗಿ-ರುವ
ಪ್ರತಿ-ಪಾದಿಸಿ
ಪ್ರತಿ-ಪಾದಿಸಿದ
ಪ್ರತಿ-ಪಾದಿಸಿ-ರುವ
ಪ್ರತಿ-ಪಾದಿ-ಸುತ್ತದೆ
ಪ್ರತಿ-ಪಾದಿಸುತ್ತಾರೆ
ಪ್ರತಿ-ಪಾದಿಸುತ್ತಿ-ದ್ದೇನೆ
ಪ್ರತಿ-ಪಾದಿ-ಸುವ
ಪ್ರತಿ-ಪಾದಿಸು-ವಾಗ
ಪ್ರತಿ-ಪಾದಿಸು-ವುದು
ಪ್ರತಿ-ಪ್ಠಾ-ಪಿ-ಸುತ್ತಾರೆ
ಪ್ರತಿ-ಫಲ-ವನ್ನು
ಪ್ರತಿ-ಫಲವೂ
ಪ್ರತಿ-ಫಲಾ
ಪ್ರತಿ-ಬಂಧ-ಕ-ವನ್ನು
ಪ್ರತಿ-ಬಿಂಬ
ಪ್ರತಿ-ಬಿಂಬ-ದಂತೆ
ಪ್ರತಿ-ಬಿಂಬ-ದಲ್ಲಿ
ಪ್ರತಿ-ಬಿಂಬಿ-ಸಲು
ಪ್ರತಿ-ಬಿಂಬಿ-ಸುವ
ಪ್ರತಿಭಾ
ಪ್ರತಿ-ಭಾ-ಶಾಲಿ
ಪ್ರತಿ-ಭಾ-ಶಾಲಿ-ಗಳಾದ
ಪ್ರತಿ-ಭಾ-ಸಂಪನ್ನ
ಪ್ರತಿಭೆ
ಪ್ರತಿ-ಭೆಗೆ
ಪ್ರತಿ-ಭೆಯ
ಪ್ರತಿ-ಭೆ-ಯನ್ನು
ಪ್ರತಿ-ಭೆಯೇ
ಪ್ರತಿಮಾ
ಪ್ರತಿ-ಮಾ-ಪೂಜೆ-ಯಲ್ಲಿ
ಪ್ರತಿ-ಯಾಗಿ
ಪ್ರತಿ-ಯೊಂದನ್ನು
ಪ್ರತಿ-ಯೊಂದನ್ನೂ
ಪ್ರತಿ-ಯೊಂದರ
ಪ್ರತಿ-ಯೊಂದರ-ಲ್ಲಿಯೂ
ಪ್ರತಿ-ಯೊಂದ-ರಿಂದ
ಪ್ರತಿ-ಯೊಂದು
ಪ್ರತಿ-ಯೊಂದು-ದೇಶವೂ
ಪ್ರತಿ-ಯೊಂದೂ
ಪ್ರತಿ-ಯೊಬ್ಬ
ಪ್ರತಿ-ಯೊಬ್ಬನ
ಪ್ರತಿ-ಯೊಬ್ಬ-ನ-ಲ್ಲಿಯೂ
ಪ್ರತಿ-ಯೊಬ್ಬ-ನಿಗೂ
ಪ್ರತಿ-ಯೊಬ್ಬನೂ
ಪ್ರತಿ-ಯೊಬ್ಬರ
ಪ್ರತಿ-ಯೊಬ್ಬ-ರ-ಲ್ಲಿಯೂ
ಪ್ರತಿ-ಯೊಬ್ಬ-ರಿಗೂ
ಪ್ರತಿ-ಯೊಬ್ಬ-ರಿಗೆ
ಪ್ರತಿ-ಯೊಬ್ಬರೂ
ಪ್ರತಿ-ಯೊಬ್ಬ-ರೂ-ನಮಗೆ
ಪ್ರತಿ-ರೂಪ
ಪ್ರತಿ-ರೂಪಕ್ಕೂ
ಪ್ರತಿ-ರೂಪ-ವಾಗಿ-ರು-ತ್ತದೆ
ಪ್ರತಿ-ವಾದ
ಪ್ರತಿ-ಷ್ಠಾಪನೆ
ಪ್ರತಿ-ಷ್ಠಿತ-ರಾಗ-ಬೇಕು
ಪ್ರತಿ-ಷ್ಠಿತ-ಳಾಗಿ-ರುವಳು
ಪ್ರತಿ-ಷ್ಠಿತ-ವಾಗಿದೆ
ಪ್ರತಿ-ಷ್ಠಿತ-ವಾಗಿವೆ
ಪ್ರತಿಷ್ಠೆ
ಪ್ರತಿ-ಷ್ಠೆಗೆ
ಪ್ರತಿ-ಸಲವೂ
ಪ್ರತಿ-ಸ್ವರವೂ
ಪ್ರತೀಕ
ಪ್ರತೀಕ-ಗಳಾಗಿ
ಪ್ರತೀಕ-ದಂತೆ
ಪ್ರತ್ಯಕ್ಷ
ಪ್ರತ್ಯಕ್ಷದ
ಪ್ರತ್ಯಕ್ಷ-ವಾಗಿ
ಪ್ರತ್ಯಕ್ಷ-ವಾಗು-ವಂತೆ
ಪ್ರತ್ಯಕ್ಷವೇ
ಪ್ರತ್ಯೇಕ
ಪ್ರತ್ಯೇಕತಾ
ಪ್ರತ್ಯೇ-ಕತೆ
ಪ್ರತ್ಯೇಕ-ತೆಯ
ಪ್ರತ್ಯೇಕ-ತೆ-ಯನ್ನು
ಪ್ರತ್ಯೇಕ-ತೆ-ಯಿಂದ
ಪ್ರತ್ಯೇಕ-ನಾಗಿ-ರು-ವನು
ಪ್ರತ್ಯೇಕ-ವಾಗಿ
ಪ್ರತ್ಯೇಕ-ವಾಗಿ-ತ್ತು
ಪ್ರತ್ಯೇಕ-ವಾಗಿ-ರುವ
ಪ್ರತ್ಯೇಕ-ವಾದ
ಪ್ರತ್ಯೇಕಿಸಿ
ಪ್ರತ್ಯೇಕಿ-ಸುವ
ಪ್ರಥಮ
ಪ್ರಥಮತಃ
ಪ್ರಥಮ-ದಲ್ಲಿ
ಪ್ರಥಮ-ವರ್ಗದ
ಪ್ರದರ್ಶನ-ದಿಂದ
ಪ್ರದರ್ಶನಾಲೋಲ
ಪ್ರದರ್ಶಿಸಿ
ಪ್ರದರ್ಶಿ-ಸು-ತ್ತಿವೆ
ಪ್ರದರ್ಶಿ-ಸುವು-ದ-ಕ್ಕಾಗಿ
ಪ್ರದರ್ಶಿ-ಸು-ವು-ದ-ರಿಂದ
ಪ್ರದೇಶಕ್ಕೆ
ಪ್ರದೇಶ-ಗಳ
ಪ್ರದೇಶ-ಗಳಲ್ಲಿ
ಪ್ರದೇಶ-ಗಳ-ಲ್ಲೆಲ್ಲಾ
ಪ್ರದೇಶದ
ಪ್ರಧಾನ
ಪ್ರಧಾನ-ವಾಗಿದೆ
ಪ್ರಧಾನ-ವಾಗಿ-ರುವ
ಪ್ರಧಾನ-ವಾಗಿ-ರು-ವು-ದ-ರಿಂದ
ಪ್ರಧಾನ-ವಾಗು-ವಂತೆ
ಪ್ರಧಾನ-ವಾದ
ಪ್ರಪಂಚ
ಪ್ರಪಂಚಕ್ಕೂ
ಪ್ರಪಂಚಕ್ಕೆ
ಪ್ರಪಂಚ-ಕ್ಕೆಯೇ
ಪ್ರಪಂಚದ
ಪ್ರಪಂಚ-ದಂತಿ-ರುವ
ಪ್ರಪಂಚ-ದಲ್ಲಿ
ಪ್ರಪಂಚ-ದಲ್ಲಿನ
ಪ್ರಪಂಚ-ದಲ್ಲಿ-ರುವ
ಪ್ರಪಂಚ-ದಲ್ಲೆಲ್ಲಾ
ಪ್ರಪಂಚ-ದಲ್ಲೇ
ಪ್ರಪಂಚ-ದಿಂದ
ಪ್ರಪಂಚ-ದಿಂದಲೂ
ಪ್ರಪಂಚ-ವನ್ನು
ಪ್ರಪಂಚ-ವನ್ನು-ತೋಯಿಸಿರು-ವುವು
ಪ್ರಪಂಚ-ವನ್ನೆಲ್ಲಾ
ಪ್ರಪಂಚ-ವನ್ನೇ
ಪ್ರಪಂಚ-ವಾಗಿಲ್ಲ
ಪ್ರಪಂಚವು
ಪ್ರಪಂಚ-ವೆಂಬುದೇ
ಪ್ರಪಂಚವೇ
ಪ್ರಪ್ರಥಮ
ಪ್ರಪ್ರಥಮ-ವಾಗಿ
ಪ್ರಪ್ರಾಚೀನ
ಪ್ರಬಂಧ-ಗಳು
ಪ್ರಬಲ
ಪ್ರಬಲ-ತೆಗೆ
ಪ್ರಬಲ-ವಾಗಿ
ಪ್ರಬಲ-ವಾಗಿದೆ
ಪ್ರಬಲ-ವಾಗಿದ್ದ
ಪ್ರಬಲ-ವಾಗಿ-ರುವ
ಪ್ರಬಲ-ವಾಗಿ-ರು-ವುವು
ಪ್ರಬಲ-ವಾಗುವ
ಪ್ರಬಲ-ವಾಗು-ವುವು
ಪ್ರಬಲ-ವಾದ
ಪ್ರಬಲ-ವಾದಾಗ
ಪ್ರಬಲ-ವಾದುದು
ಪ್ರಬುದ್ಧ-ರಾಗಿ
ಪ್ರಬುದ್ಧ-ವಾಗಿ-ತ್ತು
ಪ್ರಭಾವ
ಪ್ರಭಾವಕ್ಕೆ
ಪ್ರಭಾವ-ದಿಂದ
ಪ್ರಭಾವ-ದಿಂದಲೋ
ಪ್ರಭಾವ-ವನ್ನು
ಪ್ರಭಾವ-ವನ್ನು-ಕಳೆ-ದು-ಕೊಂಡವು
ಪ್ರಭಾವವು
ಪ್ರಭಾವಿತ-ರಾಗ
ಪ್ರಭಿನ್ನೇ
ಪ್ರಭು
ಪ್ರಭು-ಗಳಾಗ-ಬೇಕು
ಪ್ರಭು-ಗಳೂ
ಪ್ರಭು-ತ್ವ-ವನ್ನು
ಪ್ರಭು-ತ್ವವೂ
ಪ್ರಮಾಣ
ಪ್ರಮಾ-ಣ-ಕ್ಕಾಗಿ
ಪ್ರಮಾ-ಣಕ್ಕೆ
ಪ್ರಮಾ-ಣ-ಗಳನ್ನು
ಪ್ರಮಾ-ಣ-ಗಳಿ-ರು-ವುವು
ಪ್ರಮಾ-ಣ-ಗಳು
ಪ್ರಮಾ-ಣ-ಗೊಂಡಿದೆ
ಪ್ರಮಾ-ಣದ
ಪ್ರಮಾ-ಣ-ದಂತೆ
ಪ್ರಮಾ-ಣ-ದಲ್ಲಿಟ್ಟುಕೊಳ್ಳ-ಬಹುದು
ಪ್ರಮಾ-ಣ-ಪಡಿ-ಸು-ವು-ದಕ್ಕೆ
ಪ್ರಮಾ-ಣ-ಬದ್ಧ-ವಲ್ಲದೇ
ಪ್ರಮಾ-ಣ-ಮಾಡಿ
ಪ್ರಮಾ-ಣ-ವನ್ನು
ಪ್ರಮಾ-ಣ-ವಲ್ಲ
ಪ್ರಮಾ-ಣ-ವಾಗ-ಲಾ-ರದು
ಪ್ರಮಾ-ಣ-ವಾಗಿ
ಪ್ರಮಾ-ಣ-ವಾಗಿದೆ
ಪ್ರಮಾ-ಣ-ವಾಗಿವೆ
ಪ್ರಮಾ-ಣ-ವಾ-ವುದು
ಪ್ರಮಾ-ಣ-ವಿಲ್ಲ
ಪ್ರಮಾ-ಣ-ವಿ-ಲ್ಲದೆ
ಪ್ರಮಾ-ಣವು
ಪ್ರಮಾ-ಣವೂ
ಪ್ರಮಾ-ಣ-ವೆಂದು
ಪ್ರಮಾ-ಣ-ವೆಂದೂ
ಪ್ರಮಾ-ಣ-ವೆಂಬು-ದನ್ನು
ಪ್ರಮಾ-ಣವೇ
ಪ್ರಮಾ-ಣ-ವೊಂದು
ಪ್ರಮಾ-ಣವೋ
ಪ್ರಮಾ-ಣ-ಸ-ಹಿತ
ಪ್ರಮಾ-ಣ-ಸಿದ್ಧ
ಪ್ರಮಾ-ಣಿತ-ವಾದರೆ
ಪ್ರಮಾ-ಣೀ-ಕರಿಸ-ಲಾ-ಗು-ವು-ದಿಲ್ಲ
ಪ್ರಮಾ-ಣೀ-ಕ-ರಿ-ಸಲು
ಪ್ರಮಾ-ಣೀ-ಕರಿ-ಸು-ವು-ದಕ್ಕೆ
ಪ್ರಮಾ-ಣೀಕರಿ-ಸು-ವು-ದಾದರೆ
ಪ್ರಮುಖ
ಪ್ರಮುಖರು
ಪ್ರಮುಖ-ವಾಗಿ
ಪ್ರಮುಖ-ವಾಗು-ತ್ತಿವೆ
ಪ್ರಮೋದ-ಗಳಿಗೆ
ಪ್ರಮ್ಧಾಣ್ಧ
ಪ್ರಯತಿಃ
ಪ್ರಯತ್ನ
ಪ್ರಯತ್ನ-ಗಳ
ಪ್ರಯತ್ನ-ಗಳನ್ನು
ಪ್ರಯತ್ನ-ಗಳನ್ನೂ
ಪ್ರಯತ್ನ-ಗಳಿಂದ
ಪ್ರಯತ್ನ-ಗಳಿಗೆ
ಪ್ರಯತ್ನ-ಗಳು
ಪ್ರಯತ್ನ-ಗಳೂ
ಪ್ರಯತ್ನ-ಗಳೆಲ್ಲಾ
ಪ್ರಯತ್ನದ
ಪ್ರಯತ್ನ-ದಲ್ಲಿ
ಪ್ರಯತ್ನ-ದಿಂದಲೇ
ಪ್ರಯತ್ನ-ಪಟ್ಟ
ಪ್ರಯತ್ನ-ಪಟ್ಟರು
ಪ್ರಯತ್ನ-ಪಟ್ಟರೆ
ಪ್ರಯತ್ನ-ಪಟ್ಟಿ-ರು-ವರು
ಪ್ರಯತ್ನ-ಪಟ್ಟಿಲ್ಲ
ಪ್ರಯತ್ನ-ಪಟ್ಟು
ಪ್ರಯತ್ನ-ಪಟ್ಟೆ
ಪ್ರಯತ್ನ-ಪಡದಿರು-ವುದು
ಪ್ರಯತ್ನ-ಪಡ-ಬೇಕು
ಪ್ರಯತ್ನ-ಪ-ಡಲಿ
ಪ್ರಯತ್ನ-ಪಡು-ತ್ತಿದ್ದರೆ
ಪ್ರಯತ್ನ-ಪಡುತ್ತಿರ-ಬಹುದು
ಪ್ರಯತ್ನ-ಪಡುತ್ತಿರು-ವಿರಿ
ಪ್ರಯತ್ನ-ಪ-ಡುವನು
ಪ್ರಯತ್ನ-ಪಡು-ವುದೆಲ್ಲ
ಪ್ರಯತ್ನ-ಪಡು-ವೆನು
ಪ್ರಯತ್ನ-ವನ್ನು
ಪ್ರಯತ್ನ-ವನ್ನೂ
ಪ್ರಯತ್ನ-ವಾಗಿದೆ
ಪ್ರಯತ್ನ-ವಿಲ್ಲ
ಪ್ರಯತ್ನವೆ
ಪ್ರಯತ್ನವೇ
ಪ್ರಯತ್ನಾದ-ವ-ರಾದಪಿ
ಪ್ರಯತ್ನಿ-ಸದೆ
ಪ್ರಯತ್ನಿಸ-ಬೇಕು
ಪ್ರಯತ್ನಿಸ-ಬೇಡಿ
ಪ್ರಯತ್ನಿ-ಸ-ಲಿಲ್ಲ
ಪ್ರಯತ್ನಿಸಿ
ಪ್ರಯತ್ನಿಸಿ-ದಂತೆಯೇ
ಪ್ರಯತ್ನಿಸಿ-ದನು
ಪ್ರಯತ್ನಿಸಿ-ದರು
ಪ್ರಯತ್ನಿಸಿ-ದರೂ
ಪ್ರಯತ್ನಿಸಿ-ದರೆ
ಪ್ರಯತ್ನಿಸಿ-ದಾಗ-ಲೆಲ್ಲಾ
ಪ್ರಯತ್ನಿ-ಸಿದೆ
ಪ್ರಯತ್ನಿಸಿ-ದ್ದರೂ
ಪ್ರಯತ್ನಿಸಿ-ದ್ದರೆ
ಪ್ರಯತ್ನಿಸಿ-ದ್ದರೆಂಬುದೂ
ಪ್ರಯತ್ನಿಸಿ-ದ್ದಾರೆ
ಪ್ರಯತ್ನಿಸಿ-ದ್ಧೇನೆ
ಪ್ರಯತ್ನಿಸುತ್ತಾನೆ
ಪ್ರಯತ್ನಿಸುತ್ತಾರೆ
ಪ್ರಯತ್ನಿ-ಸು-ತ್ತಿದೆ
ಪ್ರಯತ್ನಿ-ಸು-ತ್ತಿದ್ದ
ಪ್ರಯತ್ನಿಸುತ್ತಿ-ದ್ದೇನೆ
ಪ್ರಯತ್ನಿಸುತ್ತಿ-ರುವ
ಪ್ರಯತ್ನಿಸುತ್ತಿ-ರು-ವರು
ಪ್ರಯತ್ನಿಸುತ್ತಿರು-ವಿರಿ
ಪ್ರಯತ್ನಿಸುತ್ತಿರು-ವುದು
ಪ್ರಯತ್ನಿಸುತ್ತಿರು-ವೆನು
ಪ್ರಯತ್ನಿಸು-ತ್ತಿವೆ
ಪ್ರಯತ್ನಿಸು-ತ್ತೀರಿ
ಪ್ರಯತ್ನಿಸು-ತ್ತೇನೆ
ಪ್ರಯತ್ನಿಸುತ್ತೇವೆ
ಪ್ರಯತ್ನಿ-ಸು-ವನು
ಪ್ರಯ-ತ್ನಿ-ಸು-ವರು
ಪ್ರಯತ್ನಿ-ಸು-ವು-ದಕ್ಕೆ
ಪ್ರಯತ್ನಿಸು-ವು-ದನ್ನು
ಪ್ರಯತ್ನಿಸು-ವುದು
ಪ್ರಯತ್ನಿಸು-ವೆವು
ಪ್ರಯತ್ನಿ-ಸೋಣ
ಪ್ರಯಾಣ
ಪ್ರಯಾ-ಣದ
ಪ್ರಯಾ-ಣ-ದಲ್ಲಿ
ಪ್ರಯಾ-ಣಿಕ-ರನ್ನು
ಪ್ರಯೋಗ
ಪ್ರಯೋಗದ
ಪ್ರಯೋಗ-ದಲ್ಲಿ
ಪ್ರಯೋಗ-ವನ್ನು
ಪ್ರಯೋಗಿಸ-ಬೇ-ಕಾ-ದುದು
ಪ್ರಯೋಗಿಸ-ಬೇಕು
ಪ್ರಯೋಗಿ-ಸುವ
ಪ್ರಯೋಗಿಸುವು-ದ-ರಲ್ಲಿ
ಪ್ರಯೋಜ
ಪ್ರಯೋ-ಜನ
ಪ್ರಯೋ-ಜ-ನ-ಕಾರಿ
ಪ್ರಯೋ-ಜ-ನ-ಕಾರಿ-ಯಾಗಿ-ದ್ದವೋ
ಪ್ರಯೋ-ಜ-ನಕ್ಕೆ
ಪ್ರಯೋ-ಜ-ನ-ವನ್ನು
ಪ್ರಯೋ-ಜ-ನ-ವಾಗ-ಬೇ-ಕಾದರೆ
ಪ್ರಯೋ-ಜ-ನ-ವಾಗು-ತ್ತದೆ
ಪ್ರಯೋ-ಜ-ನ-ವಾಗು-ತ್ತಿದೆ
ಪ್ರಯೋ-ಜ-ನ-ವಾಗು-ವು-ದಿಲ್ಲ
ಪ್ರಯೋ-ಜ-ನ-ವಿಲ್ಲ
ಪ್ರಯೋ-ಜ-ನ-ವಿಲ್ಲದ
ಪ್ರಯೋ-ಜ-ನ-ವಿ-ಲ್ಲದೆ
ಪ್ರಯೋ-ಜ-ನ-ವಿಲ್ಲ-ವೆಂದು
ಪ್ರಯೋ-ಜ-ನವು
ಪ್ರಯೋ-ಜ-ನವೂ
ಪ್ರಯೋ-ಜ-ನ-ವೇನು
ಪ್ರಲಯ
ಪ್ರಲಾಪ-ವನ್ನು
ಪ್ರಲೋ
ಪ್ರಲೋ-ಭನ-ಕಾರಿ-ಯಾದ
ಪ್ರಲೋ-ಭ-ನೆ-ಗಳಿಗೆಲ್ಲ
ಪ್ರಲೋ-ಭ-ನೆಗೆ
ಪ್ರಳಯ
ಪ್ರಳ-ಯಕ್ಕೆ
ಪ್ರಳ-ಯದ
ಪ್ರಳಯ-ದಲ್ಲಿ
ಪ್ರಳಯ-ಸ್ಥಿತಿ-ಯಲ್ಲಿ
ಪ್ರವಚನ-ದಿಂದ
ಪ್ರವಚನ-ದಿಂದಾ-ಗಲಿ
ಪ್ರವಚ-ನೇನ
ಪ್ರವರ್ತ-ನೆ-ಯನ್ನು
ಪ್ರವರ್ಧ-ಮಾ-ನಕ್ಕೆ
ಪ್ರವ-ಹಿ-ಸ-ಬೇಕು
ಪ್ರವ-ಹಿಸಿ
ಪ್ರವ-ಹಿ-ಸಿದ
ಪ್ರವ-ಹಿ-ಸು-ವು-ದಕ್ಕೆ
ಪ್ರವಾದಿ-ಗಳನ್ನೂ
ಪ್ರವಾದಿ-ಗಳು
ಪ್ರವಾದಿಯ
ಪ್ರವಾ-ಸಿ-ಗಳು
ಪ್ರವಾಹ
ಪ್ರವಾ-ಹಕ್ಕೆ
ಪ್ರವಾ-ಹ-ಗಳು
ಪ್ರವಾ-ಹದ
ಪ್ರವಾ-ಹ-ದಂತೆ
ಪ್ರವಾ-ಹ-ದಿಂದ
ಪ್ರವಾ-ಹ-ದೊಂದಿಗೆ
ಪ್ರವಾ-ಹ-ವನ್ನು
ಪ್ರವಾ-ಹ-ವ-ಲ್ಲದೆ
ಪ್ರವಾ-ಹ-ವಾಗಿ
ಪ್ರವಾ-ಹ-ವಿ-ಲ್ಲಿದೆ
ಪ್ರವಾ-ಹವು
ಪ್ರವಾ-ಹವೂ
ಪ್ರವಿತ್ರೋ-ತ್ತ-ಮರು
ಪ್ರವೃತ್ತಿ
ಪ್ರವೃತ್ತಿ-ಗಳಿಗೂ
ಪ್ರವೃತ್ತಿಯ
ಪ್ರವೃತ್ತಿ-ಯನ್ನು
ಪ್ರವೃದ್ಧ-ಮಾ-ನಕ್ಕೆ
ಪ್ರವೇ-ಶ-ಮಾಡಿ
ಪ್ರವೇ-ಶಿಸ-ಬಹುದು
ಪ್ರವೇ-ಶಿಸ-ಲಾ-ರದು
ಪ್ರವೇ-ಶಿ-ಸಲಿ
ಪ್ರವೇ-ಶಿ-ಸಲು
ಪ್ರವೇ-ಶಿಸಿ
ಪ್ರವೇ-ಶಿ-ಸಿದ
ಪ್ರವೇ-ಶಿಸಿ-ದರೆ
ಪ್ರವೇ-ಶಿ-ಸಿದೆ
ಪ್ರವೇ-ಶಿಸಿ-ರ-ಲಿಲ್ಲ
ಪ್ರವೇ-ಶಿಸಿ-ರುವಳು
ಪ್ರವೇ-ಶಿ-ಸುತ್ತದೆ
ಪ್ರವೇ-ಶಿ-ಸುವ
ಪ್ರವೇ-ಶಿ-ಸು-ವಂತೆ
ಪ್ರವೇ-ಶಿಸು-ವು-ದೆಂದರೆ
ಪ್ರವೇ-ಶಿಸು-ವುವು
ಪ್ರಶಂಸನೀಯ-ವಾದ
ಪ್ರಶಂಸಿಸ-ಬಹುದು
ಪ್ರಶಂಸಿಸು-ವು-ದರ-ಲ್ಲಾ-ಗಲಿ
ಪ್ರಶಂಸೆ
ಪ್ರಶಂಸೆಗೆ
ಪ್ರಶಂಸೆಯ
ಪ್ರಶಸ್ತ-ವಾದ
ಪ್ರಶಾಂತ
ಪ್ರಶ್ನಿ-ಸಿದರೆ
ಪ್ರಶ್ನಿ-ಸಿದ್ದಳು
ಪ್ರಶ್ನಿ-ಸುವ
ಪ್ರಶ್ನೆ
ಪ್ರಶ್ನೆ-ಗಳನ್ನು
ಪ್ರಶ್ನೆ-ಗಳಿಗೆ
ಪ್ರಶ್ನೆಗೆ
ಪ್ರಶ್ನೆಯ
ಪ್ರಶ್ನೆ-ಯನ್ನು
ಪ್ರಶ್ನೆಯು
ಪ್ರಶ್ನೆಯೇ
ಪ್ರಸಂಗ
ಪ್ರಸಂಗ-ದಲ್ಲಿ
ಪ್ರಸಂಗ-ವನ್ನು
ಪ್ರಸಂಗ-ವಿದೆ
ಪ್ರಸಂಗ-ವೆಲ್ಲ
ಪ್ರಸಾದ-ದಿಂದ
ಪ್ರಸಾರ
ಪ್ರಸಾ-ರಕ್ಕೆ
ಪ್ರಸಾರ-ವಾಗಿವೆ
ಪ್ರಸಿದ್ಧ
ಪ್ರಸಿದ್ಧ-ರೆಂಬ
ಪ್ರಸಿದ್ಧ-ವಾಗಿ-ತ್ತು
ಪ್ರಸಿದ್ಧ-ವಾಗಿ-ರುವ
ಪ್ರಸಿದ್ಧ-ವಾದ
ಪ್ರಸಿದ್ಧಿಗೆ
ಪ್ರಸ್ತಾಪ-ವನ್ನು
ಪ್ರಸ್ತಾಪ-ವಿದೆ
ಪ್ರಸ್ತಾಪವು
ಪ್ರಸ್ತಾಪವೇ
ಪ್ರಸ್ತಾ-ಪಿ-ಸ-ಬಹುದು
ಪ್ರಸ್ತಾ-ಪಿ-ಸ-ಬೇಕೆಂದು
ಪ್ರಸ್ತಾ-ಪಿಸಿ
ಪ್ರಸ್ತಾ-ಪಿ-ಸಿದ
ಪ್ರಸ್ತಾ-ಪಿ-ಸಿ-ರು-ವಿರಿ
ಪ್ರಸ್ತಾ-ಪಿ-ಸುತ್ತಿರು-ವಿರಿ
ಪ್ರಸ್ತಾ-ಪಿ-ಸು-ವು-ದಕ್ಕೆ
ಪ್ರಸ್ತುತ
ಪ್ರಸ್ಥ-ಭೂಮಿ-ಯಲ್ಲಿ
ಪ್ರಸ್ಥಾನ-ಗಳ
ಪ್ರಸ್ಥಾನ-ಗಳನ್ನು
ಪ್ರಸ್ಥಾನೇ
ಪ್ರಸ್ಫುಟ-ವಾಗಿ-ರುವ
ಪ್ರಹ್ಲಾದ
ಪ್ರಹ್ಲಾದ-ನಂತೆ
ಪ್ರಹ್ಲಾದನೂ
ಪ್ರಾಂತಕ್ಕೂ
ಪ್ರಾಂತ-ಗಳ
ಪ್ರಾಂತ-ಗಳಲ್ಲಿ
ಪ್ರಾಂತ್ಯ-ಗಳಲ್ಲಿಯೂ
ಪ್ರಾಂತ್ಯ-ದಲ್ಲಿ
ಪ್ರಾಂತ್ಯ-ದಲ್ಲಿ-ರುವ
ಪ್ರಾಕೃತ
ಪ್ರಾಕೃತ-ವಾದು-ದೆಂದರೆ
ಪ್ರಾಕೃತ-ವೆಂದು
ಪ್ರಾಕೃತಿಕ
ಪ್ರಾಕ್ತನ
ಪ್ರಾಚೀನ
ಪ್ರಾಚೀನ-ಕಾಲದ
ಪ್ರಾಚೀನ-ಕಾಲ-ದಲ್ಲಿ
ಪ್ರಾಚೀನ-ಕಾಲ-ದಿಂದಲೂ
ಪ್ರಾಚೀನ-ಧರ್ಮದ
ಪ್ರಾಚೀನರು
ಪ್ರಾಚೀನ-ವಾದ
ಪ್ರಾಚೀನ-ವಾದುದು
ಪ್ರಾಚೀನ-ವಾದುವು
ಪ್ರಾಚೀನವೂ
ಪ್ರಾಚ್ಯ
ಪ್ರಾಚ್ಯ-ಪಾಶ್ಚಾತ್ಯ
ಪ್ರಾಚ್ಯ-ರಾದ
ಪ್ರಾಚ್ಯ-ರಾಷ್ಟ್ರ-ಗಳಿಗೂ
ಪ್ರಾಚ್ಯ-ರಿಗೂ
ಪ್ರಾಚ್ಯರು
ಪ್ರಾಚ್ಯ-ವಾದುದೆ-ಲ್ಲ-ದ-ರಿಂದ
ಪ್ರಾಚ್ಯವೋ
ಪ್ರಾಜ್ಞ-ನಲ್ಲ
ಪ್ರಾಜ್ಞ-ನ-ಲ್ಲಿಯೂ
ಪ್ರಾಜ್ಞನೂ
ಪ್ರಾಟೆ
ಪ್ರಾಟೆ-ಸ್ಟೆಂಟ-ರಲ್ಲಿ
ಪ್ರಾಣ
ಪ್ರಾಣ-ಆ-ಕಾಶ-ಗಳಲ್ಲಿ
ಪ್ರಾಣದ
ಪ್ರಾಣ-ದಂತಿ-ರುವ
ಪ್ರಾಣ-ದಾನ
ಪ್ರಾಣ-ರಕ್ಷಣೆ
ಪ್ರಾಣ-ವನ್ನು
ಪ್ರಾಣವು
ಪ್ರಾಣವೇ
ಪ್ರಾಣ-ಶಕ್ತಿ-ಯನ್ನು
ಪ್ರಾಣ-ಶಕ್ತಿ-ಯಿಂದ
ಪ್ರಾಣ-ಸ್ಥಾನ-ದಲ್ಲಿ
ಪ್ರಾಣ-ಹೋ-ದರೂ
ಪ್ರಾಣಾಪಾಯ-ವಾಗು-ವಂತಹ
ಪ್ರಾಣಾ-ಯಾಮ
ಪ್ರಾಣಿ
ಪ್ರಾಣಿ-ಗಳ
ಪ್ರಾಣಿ-ಗಳಂತಾಗು-ತ್ತೀರಿ
ಪ್ರಾಣಿ-ಗಳನ್ನು
ಪ್ರಾಣಿ-ಗಳಲ್ಲಿ
ಪ್ರಾಣಿ-ಗಳ-ಲ್ಲಿಯೂ
ಪ್ರಾಣಿ-ಗಳಿಂದ
ಪ್ರಾಣಿ-ಗಳಿಗೂ
ಪ್ರಾಣಿ-ಗಳು
ಪ್ರಾಣಿ-ಗಳೇ
ಪ್ರಾಣಿ-ಗಿಂತ
ಪ್ರಾಣಿಗೆ
ಪ್ರಾಣಿಯ
ಪ್ರಾಣಿ-ಯ-ಲ್ಲಿಯೂ
ಪ್ರಾಣಿ-ಹಿಂಸೆಗೆ
ಪ್ರಾಧಾನ್ಯ
ಪ್ರಾಧಾನ್ಯ-ವನ್ನು
ಪ್ರಾಧಾನ್ಯ-ವಿದೆ
ಪ್ರಾಪಂಚಿಕ
ಪ್ರಾಪಂಚಿ-ಕತೆ
ಪ್ರಾಪಂಚಿಕ-ತೆ-ಯಲ್ಲಿ
ಪ್ರಾಪಂಚಿಕ-ರಾಗುತ್ತೇವೆ
ಪ್ರಾಪಂಚಿ-ಕರು
ಪ್ರಾಪ್ತ
ಪ್ರಾಪ್ತ-ವಾಗಲಾರ-ದೆಂಬು-ದನ್ನು
ಪ್ರಾಪ್ತ-ವಾಗಿದೆ
ಪ್ರಾಪ್ತ-ವಾಗಿ-ರು-ವುದು
ಪ್ರಾಪ್ತ-ವಾಗು-ತ್ತದೆ
ಪ್ರಾಪ್ತ-ವಾಗುತ್ತಿತ್ತೆಂಬುದು
ಪ್ರಾಪ್ತ-ವಾಗುತ್ತಿರು-ವಾಗ
ಪ್ರಾಪ್ತ-ವಾಗು-ವುದು
ಪ್ರಾಪ್ತ-ವಾಗು-ವುದು-ಎಂಬ
ಪ್ರಾಪ್ತ-ವಾಗುವು-ದೆಂದೂ
ಪ್ರಾಪ್ತ-ವಾದಂತೆ
ಪ್ರಾಪ್ತ-ವಾದರೆ
ಪ್ರಾಪ್ತ-ವಾದಾಗ
ಪ್ರಾಪ್ತ-ವಾ-ಯಿತು
ಪ್ರಾಪ್ತಿ
ಪ್ರಾಪ್ಯ
ಪ್ರಾಪ್ಯ-ವರಾನ್ನಿ
ಪ್ರಾಪ್ಯ-ವರಾನ್ನಿ-ಬೋಧತ
ಪ್ರಾಬಲ್ಯಕ್ಕೆ
ಪ್ರಾಬಲ್ಯ-ವನ್ನು
ಪ್ರಾಬಲ್ಯ-ವಿದೆ
ಪ್ರಾ-ಮಾ-ಣಿಕ
ಪ್ರಾ-ಮಾ-ಣಿ-ಕತೆ
ಪ್ರಾ-ಮಾ-ಣಿಕ-ತೆ-ಗಳ
ಪ್ರಾ-ಮಾ-ಣಿಕ-ತೆ-ಗಳಿಂದ
ಪ್ರಾ-ಮಾ-ಣಿಕ-ತೆ-ಯಿಂದ
ಪ್ರಾ-ಮಾ-ಣಿಕ-ರಲ್ಲ
ಪ್ರಾ-ಮಾ-ಣಿಕ-ರಾ-ಗಿರಿ
ಪ್ರಾ-ಮಾ-ಣಿಕ-ಳಾದ
ಪ್ರಾ-ಮಾಣ್ಯ
ಪ್ರಾ-ಮಾ-ಣ್ಯ-ವನ್ನು
ಪ್ರಾಮಾಣ್ಯ-ವನ್ನೊಪ್ಪಿ
ಪ್ರಾ-ಮಾ-ಣ್ಯ-ವಿಲ್ಲ
ಪ್ರಾಮುಖ್ಯ-ವನ್ನು
ಪ್ರಾಮುಖ್ಯ-ವನ್ನೂ
ಪ್ರಾಯ-ವಾಗಿವೆ
ಪ್ರಾಯಶಃ
ಪ್ರಾಯಶ್ಚಿತ್ತ
ಪ್ರಾಯಶ್ಚಿತ್ತ-ಕ್ಕಾಗಿ
ಪ್ರಾಯಶ್ಚಿತ್ತ-ರೂಪ-ವಾಗಿ
ಪ್ರಾಯಶ್ಚಿತ್ತ-ವನ್ನು
ಪ್ರಾಯಶ್ಚಿತ್ತ-ವೆಂದು
ಪ್ರಾಯಶ್ಚಿತ್ತ-ವೇನು
ಪ್ರಾಯೋಗಿಕ-ವಾಗಿ
ಪ್ರಾರಂಭ
ಪ್ರಾರಂಭ-ದಲ್ಲಿ
ಪ್ರಾರಂಭ-ದಲ್ಲಿಯೇ
ಪ್ರಾರಂಭ-ದಲ್ಲೇ
ಪ್ರಾರಂಭ-ದಿಂದಲೂ
ಪ್ರಾರಂಭ-ವಾಗ-ಬೇಕು
ಪ್ರಾರಂಭ-ವಾಗಿ
ಪ್ರಾರಂಭ-ವಾಗಿದೆ
ಪ್ರಾರಂಭ-ವಾಗಿಲ್ಲ
ಪ್ರಾರಂಭ-ವಾಗು-ತ್ತದೆ
ಪ್ರಾರಂಭ-ವಾಗು-ವು-ದಕ್ಕೆ
ಪ್ರಾರಂಭ-ವಾಗು-ವುದು
ಪ್ರಾರಂಭ-ವಾಗು-ವುದೇ
ಪ್ರಾರಂಭ-ವಾದ
ಪ್ರಾರಂಭ-ವಾದಂತಾ-ಗಿದೆ
ಪ್ರಾರಂಭ-ವಾದುದು
ಪ್ರಾರಂಭ-ವಾ-ಯಿತು
ಪ್ರಾರಂಭಾವ-ಸ್ಥೆ-ಯಲ್ಲಿ
ಪ್ರಾರಂಭಿಸ-ಬೇಕು
ಪ್ರಾರಂಭಿ-ಸಿದ
ಪ್ರಾರಂಭಿ-ಸಿದರು
ಪ್ರಾರಂಭಿಸಿ-ದೊಡನೆ
ಪ್ರಾರಂಭಿಸಿ-ರುವು-ದ-ಕ್ಕಾಗಿಯೂ
ಪ್ರಾರಂಭಿ-ಸುವ
ಪ್ರಾರಂಭಿ-ಸು-ವರು
ಪ್ರಾರಂಭಿಸು-ವುದು
ಪ್ರಾರಂಭಿಸುವೆ
ಪ್ರಾರಂಭೋತ್ಸವದ
ಪ್ರಾರ್ಥನಾ
ಪ್ರಾರ್ಥನಾ-ಲಯ-ದಲ್ಲಿ
ಪ್ರಾರ್ಥ-ನೆ-ಗಳೇ
ಪ್ರಾರ್ಥ-ನೆಗೆ
ಪ್ರಾರ್ಥ-ನೆ-ಯನ್ನು
ಪ್ರಾರ್ಥಿ-ಸಿದರೆ
ಪ್ರಾರ್ಥಿಸು-ತ್ತೇನೆ
ಪ್ರಾರ್ಥಿಸುತ್ತೇವೆ
ಪ್ರಾರ್ಥಿ-ಸು-ವನು
ಪ್ರಾರ್ಥಿಸು-ವಾಗ
ಪ್ರಾಶಸ್ತ್ಯವೇ
ಪ್ರಿಯ
ಪ್ರಿಯ-ತಮ
ಪ್ರಿಯ-ತಮಳ
ಪ್ರಿಯ-ತಮೆ
ಪ್ರಿಯ-ನಾಗಿ
ಪ್ರಿಯ-ನಿಗೆ
ಪ್ರಿಯನೂ
ಪ್ರಿಯೆಯ
ಪ್ರೀತಿ
ಪ್ರೀತಿ-ಗಳು
ಪ್ರೀತಿ-ಗಾಗಿ
ಪ್ರೀತಿಗೂ
ಪ್ರೀ-ತಿಗೆ
ಪ್ರೀತಿ-ಗೋಸುಗ
ಪ್ರೀತಿಯ
ಪ್ರೀತಿ-ಯನ್ನು
ಪ್ರೀತಿ-ಯಲ್ಲಿ
ಪ್ರೀತಿ-ಯಾ-ದರೆ
ಪ್ರೀತಿ-ಯಿಂದ
ಪ್ರೀತಿ-ಯಿಂದಲೂ
ಪ್ರೀ-ತಿಯೂ
ಪ್ರೀತಿ-ಯೆಲ್ಲ
ಪ್ರೀತಿಯೇ
ಪ್ರೀತಿ-ಯೊಂದನ್ನೇ
ಪ್ರೀತಿ-ಯೊಂದೇ
ಪ್ರೀತಿ-ಸತೊಡ-ಗಿದೆನು
ಪ್ರೀತಿ-ಸದಿರ-ಬಹುದು
ಪ್ರೀತಿ-ಸ-ಬೇಕು
ಪ್ರೀತಿ-ಸ-ಲಾರಿರಿ
ಪ್ರೀತಿ-ಸ-ಲಾರೆವು
ಪ್ರೀತಿ-ಸಲು
ಪ್ರೀ-ತಿಸಿ
ಪ್ರೀತಿ-ಸಿದ
ಪ್ರೀತಿ-ಸಿ-ದರೆ
ಪ್ರೀತಿ-ಸುತ್ತಾರೆ
ಪ್ರೀತಿ-ಸು-ತ್ತಿದ್ದೆ
ಪ್ರೀತಿ-ಸುತ್ತಿರು-ವೆನು
ಪ್ರೀತಿ-ಸು-ತ್ತೇನೆ
ಪ್ರೀತಿ-ಸುತ್ತೇವೆ
ಪ್ರೀತಿ-ಸು-ವನು
ಪ್ರೀತಿ-ಸು-ವರು
ಪ್ರೀತಿ-ಸುವಳು
ಪ್ರೀತಿ-ಸುವ-ವರು
ಪ್ರೀತಿ-ಸು-ವು-ದಕ್ಕೆ
ಪ್ರೀತಿ-ಸು-ವು-ದ-ರಿಂದ
ಪ್ರೀತಿ-ಸುವುದಾ-ದರೂ
ಪ್ರೀತಿ-ಸು-ವು-ದಾದರೆ
ಪ್ರೀತಿ-ಸು-ವು-ದಿಲ್ಲ
ಪ್ರೀತಿ-ಸು-ವುದು
ಪ್ರೇಕ್ಷ-ಕರ
ಪ್ರೇಮ
ಪ್ರೇಮಕ್ಕೆ
ಪ್ರೇಮ-ಗಳಿಂದ
ಪ್ರೇಮ-ಗಳು
ಪ್ರೇಮ-ಜನಿತ
ಪ್ರೇಮದ
ಪ್ರೇಮ-ದಲ್ಲಿ
ಪ್ರೇಮ-ದಲ್ಲಿದೆ
ಪ್ರೇಮ-ದಿಂದ
ಪ್ರೇಮ-ದ್ವಾರಾ
ಪ್ರೇಮ-ಪೂರ್ಣ
ಪ್ರೇಮ-ಪೂರ್ವ-ಕ-ವಾಗಿ
ಪ್ರೇಮ-ಮಯ-ನಾದ
ಪ್ರೇಮ-ರೂಪ-ವಾದ
ಪ್ರೇಮ-ವನ್ನು
ಪ್ರೇಮ-ವಿದೆಯೋ
ಪ್ರೇಮ-ವಿದ್ದು-ದ-ರಿಂದಲೇ
ಪ್ರೇಮವು
ಪ್ರೇಮ-ವೆಂದರೆ
ಪ್ರೇಮ-ವೆಂಬ
ಪ್ರೇಮವೇ
ಪ್ರೇಮ-ವೊಂದೇ
ಪ್ರೇಮ-ಶಕ್ತಿಯ
ಪ್ರೇಮ-ಶಕ್ತಿಯು
ಪ್ರೇಮ-ಸಾರ-ವಾಗಿದೆ
ಪ್ರೇಮ-ಸ್ವ-ರೂಪನು
ಪ್ರೇಮಿ-ಗಳಲ್ಲ
ಪ್ರೇಮೇಶ್ವರನು
ಪ್ರೇಮೇಶ್ವರ-ನೆಂದು
ಪ್ರೇಮೋನ್ಮತ್ತ
ಪ್ರೇಮೋನ್ಮಾದ
ಪ್ರೇಮೋನ್ಮಾ-ದಕ್ಕೆ
ಪ್ರೇಮೋನ್ಮಾ-ದ-ವಿದೆ
ಪ್ರೇರಣೆ-ಗಳು
ಪ್ರೇರಣೆ-ಯನ್ನೂ
ಪ್ರೇರಣೆ-ಯಾಗಿ
ಪ್ರೇರಣೆ-ಯಿಂದ
ಪ್ರೇರಿತ-ರಾಗಿ
ಪ್ರೇರಿತ-ರಾದ
ಪ್ರೇರಿತ-ವಾದ
ಪ್ರೇರಿತ-ವಾದ-ವು-ಗಳು
ಪ್ರೇರಿತ-ವಾದುವು-ಗಳು
ಪ್ರೇರಿ-ಸಿದೆ
ಪ್ರೇರೇ-ಪಿ-ತ-ನಾಗು-ತ್ತಾನೆ
ಪ್ರೇರೇ-ಪಿ-ಸು-ವುದು
ಪ್ರೊಫೆ-ಸರ್
ಪ್ರೋತಂ
ಪ್ರೋತ್ಸಾಹ
ಪ್ರೋತ್ಸಾಹ-ವನ್ನು
ಪ್ರೋತ್ಸಾ-ಹಿ-ಸುತ್ತಿ-ರು-ವರು
ಪ್ರೋತ್ಸಾ-ಹಿ-ಸು-ವಂತಹ
ಪ್ರೌಢ-ವಾಗಿ-ದೆ-ಯೆಂದರೆ
ಪ್ಲೇಗು
ಪ್ಲೇಟೋ
ಫಲ
ಫಲ-ಕಾರಿ-ಯಾಗ-ಬೇ-ಕಾದರೆ
ಫಲಕ್ಕೆ
ಫಲ-ಗಳನ್ನು
ಫಲ-ಪ್ರಾಪ್ತಿ
ಫಲ-ರೂಪ-ವಾಗಿ
ಫಲ-ವತ್ತಾಗಿ
ಫಲ-ವ-ತ್ತಾದ
ಫಲ-ವನ್ನು
ಫಲ-ವಾಗಿ
ಫಲ-ವಿದೆ
ಫಲವೇ
ಫಲ-ವೇ-ನಾ-ಯಿತು
ಫಲ-ಸಿದ್ಧಿಯ
ಫಲಾ-ಪೇಕ್ಷೆ
ಫಲಾ-ಪೇಕ್ಷೆ-ಯಿಂದ
ಫಿಸ್ಟರು
ಫೋನ್
ಫ್ಯಾ-ಷ-ನ್
ಫ್ರಾನ್ಸಿನ
ಫ್ರಾನ್ಸ
ಫ್ಲೋರ-ಲ್
ಬಂಗಲೆ-ಯಲ್ಲಿ
ಬಂಗಾಳ
ಬಂಗಾಳಕ್ಕೆ
ಬಂಗಾಳದ
ಬಂಗಾಳ-ದಂತೆ
ಬಂಗಾಳ-ದಲ್ಲಿ
ಬಂಗಾಳ-ದಲ್ಲಿ-ರುವ
ಬಂಗಾಳ-ದೇಶದ
ಬಂಗಾಳಿ
ಬಂಗಾಳಿ-ಗಳ
ಬಂಗಾಳಿ-ಗಳು
ಬಂಗಾಳಿಯೇ
ಬಂಡ-ವಾ-ಳ-ಗಾರ
ಬಂಡ-ವಾ-ಳ-ಗಾರ-ರಿಗೆ
ಬಂತು
ಬಂತೆಂದು
ಬಂದ
ಬಂದಂತೆ
ಬಂದಂತೆಲ್ಲಾ
ಬಂದ-ಕೂಡಲೇ
ಬಂದದ್ದು
ಬಂದನು
ಬಂದ-ನೆಂದು
ಬಂದರು
ಬಂದರೂ
ಬಂದರೆ
ಬಂದ-ರೆಂದೂ
ಬಂದರೋ
ಬಂದ-ವ-ನಾಗಿ-ರ-ಬಹುದು
ಬಂದವು
ಬಂದಷ್ಟು
ಬಂದಾಗ
ಬಂದಾಗ-ಲೆಲ್ಲಾ
ಬಂದಿತು
ಬಂದಿ-ತೆಂದು
ಬಂದಿದೆ
ಬಂದಿದೆಯೆ
ಬಂದಿದ್ದರೂ
ಬಂದಿ-ದ್ದರೆ
ಬಂದಿ-ದ್ದಲ್ಲ
ಬಂದಿ-ದ್ದವು
ಬಂದಿ-ದ್ದೀರಿ
ಬಂದಿ-ದ್ದೇವೆ
ಬಂದಿರು-ತ್ತೇನೆ
ಬಂದಿ-ರುವ
ಬಂದಿರು-ವು-ದಲ್ಲ
ಬಂದಿರು-ವುದು
ಬಂದಿರು-ವು-ದೆಂದು
ಬಂದಿರು-ವೆನು
ಬಂದಿರು-ವೆವು
ಬಂದಿಲ್ಲ
ಬಂದು
ಬಂದು-ದ-ರಿಂದ
ಬಂದುದು
ಬಂದು-ವಲ್ಲ
ಬಂದು-ಹೋ-ಗಿದೆ
ಬಂದೂಕು
ಬಂದೆ
ಬಂದೆ-ಡೆಗೆ
ಬಂದೇ
ಬಂಧನ
ಬಂಧ-ನ-ಕ್ಕೆಲ್ಲಾ
ಬಂಧ-ನ-ಗಳಿ-ವೆಯೋ
ಬಂಧ-ನ-ಗಳು
ಬಂಧ-ನ-ಗಳು-ಳ್ಳವ-ರಿಗೆ
ಬಂಧ-ನ-ದಲ್ಲಿ
ಬಂಧ-ನ-ವನ್ನು
ಬಂಧ-ನ-ವ-ಲ್ಲದೆ
ಬಂಧ-ನಾ-ತೀತ-ನಾದ
ಬಂಧಿತ-ವಾಗಿ-ರು-ವಂತೆ
ಬಂಧಿಸಿ
ಬಂಧಿಸಿ-ದಾಗ
ಬಂಧಿಸಿ-ರ-ಬೇಕು
ಬಂಧಿ-ಸುವ
ಬಂಧಿಸುವು-ದ-ಕ್ಕಾಗಿ
ಬಂಧು-ಗಳು
ಬಗೆ
ಬಗೆ-ಗಿನ
ಬಗೆಗೂ
ಬಗೆಗೆ
ಬಗೆಯ
ಬಗೆ-ಯದು
ಬಗೆ-ಹರಿಸ-ಬೇಕೆಂದು
ಬಗೆ-ಹರಿಸ-ಲಾ-ಗು-ವು-ದಿಲ್ಲ
ಬಗೆ-ಹರಿಸ-ಲಾ-ರದು
ಬಗೆ-ಹ-ರಿ-ಸಲು
ಬಗೆ-ಹ-ರಿಸಿ-ಕೊಳ್ಳು-ವರು
ಬಗೆ-ಹರಿ-ಸುವ
ಬಗೆ-ಹರಿಸು-ವಿರಿ
ಬಗೆ-ಹರಿ-ಸೋಣ
ಬಗ್ಗಿಸ-ಲಾ-ರದು
ಬಗ್ಗೆ
ಬಚ್ಚಲ
ಬಜಾರಿ-ನಲ್ಲಿ-ರುವ
ಬಟ್ಟೆ
ಬಟ್ಟೆ-ಗಳ
ಬಟ್ಟೆ-ಗಳನ್ನು
ಬಟ್ಟೆ-ಗಿಲ್ಲದೆ
ಬಟ್ಟೆಯ
ಬಡ
ಬಡ-ಜನ-ರಿಗೆ
ಬಡ-ತನ
ಬಡ-ತನ-ವನ್ನು
ಬಡ-ತನ-ವನ್ನು-ದೈವೀ-ಗುಣ
ಬಡ-ದೇಶಕ್ಕೆ
ಬಡ-ದೇಶ-ದಲ್ಲಿ
ಬಡವ
ಬಡ-ವನ
ಬಡ-ವ-ನಲ್ಲಿ
ಬಡ-ವ-ನಾ-ಗಿದ್ದಷ್ಟೂ
ಬಡ-ವರ
ಬಡ-ವ-ರನ್ನು
ಬಡ-ವ-ರಾದ
ಬಡ-ವ-ರಿ-ಗಾಗಿ
ಬಡ-ವ-ರಿ-ಗಿಂತ
ಬಡ-ವ-ರಿಗೆ
ಬಡ-ವರು
ಬಡಿದು
ಬಡಿ-ಯುವ
ಬಣ್ಣ
ಬಣ್ಣ-ಗಳನ್ನು
ಬಣ್ಣವೂ
ಬತ್ತದ
ಬತ್ತು-ವು-ದಿಲ್ಲ
ಬದರಿ-ಕಾ-ಶ್ರಮ-ದಲ್ಲಿ
ಬದಲಾ-ಗದ
ಬದಲಾ-ಗದೆ
ಬದಲಾಗ-ಬಲ್ಲದು
ಬದಲಾಗಬಲ್ಲುದೆ
ಬದಲಾಗ-ಬೇಕಾ-ಗುತ್ತವೆ
ಬದಲಾಗ-ಬೇಕಾ-ಯಿತು
ಬದಲಾಗ-ಬೇಕು
ಬದ-ಲಾ-ಗ-ಲಿಲ್ಲ
ಬದ-ಲಾಗಿ
ಬದ-ಲಾ-ಗಿದೆ
ಬದ-ಲಾ-ಗಿವೆ
ಬದಲಾ-ಗು-ತ್ತದೆ
ಬದಲಾ-ಗುತ್ತವೆ
ಬದಲಾ-ಗುತ್ತಾ
ಬದಲಾಗುತ್ತಿ-ದ್ದೀರಿ
ಬದಲಾಗುತ್ತಿರು-ತ್ತದೆ
ಬದಲಾಗುತ್ತಿ-ರುವ
ಬದಲಾಗುತ್ತಿ-ರು-ವರು
ಬದಲಾ-ಗು-ತ್ತಿವೆ
ಬದಲಾ-ಗುವ
ಬದಲಾ-ಗು-ವಂತೆ
ಬದ-ಲಾ-ಗು-ವು-ದಿಲ್ಲ
ಬದ-ಲಾ-ಗು-ವು-ದಿಲ್ಲ-ವೆನ್ನು
ಬದಲಾಗು-ವುದು
ಬದಲಾಗು-ವುವು
ಬದ-ಲಾದ
ಬದಲಾ-ದಂತೆ
ಬದ-ಲಾದರೆ
ಬದ-ಲಾದವು
ಬದಲಾ-ದಾಗ
ಬದಲಾಯಸಿ
ಬದ-ಲಾ-ಯಿತು
ಬದಲಾಯಿ-ಸದೇ
ಬದಲಾಯಿಸ-ಬಲ್ಲದು
ಬದಲಾಯಿಸ-ಬಲ್ಲಿರಾ
ಬದಲಾಯಿ-ಸ-ಬಲ್ಲಿರಿ
ಬದಲಾಯಿಸ-ಬಹುದು
ಬದಲಾಯಿಸ-ಲಾ-ರದು
ಬದಲಾ-ಯಿ-ಸಲು
ಬದಲಾಯಿಸಿ
ಬದಲಾಯಿ-ಸಿತು
ಬದಲಾಯಿಸಿ-ದರು
ಬದಲಾಯಿಸಿ-ದರೂ
ಬದಲಾಯಿಸಿ-ದರೆ
ಬದಲಾಯಿಸು
ಬದಲಾಯಿಸು-ವು-ದಕ್ಕೆ
ಬದಲಾಯಿಸು-ವು-ದಿಲ್ಲ
ಬದಲಾವಣೆ
ಬದಲಾವಣೆ-ಗಳ
ಬದಲಾವಣೆ-ಗಳ-ನ್ನೆಲ್ಲಾ
ಬದಲಾವಣೆ-ಗಳಾಗಿ-ದ್ದರೂ
ಬದಲಾವಣೆ-ಗಳಿಂದಲೇ
ಬದಲಾವಣೆ-ಗಳು
ಬದಲಾವಣೆಗೂ
ಬದಲಾವ-ಣೆಗೆ
ಬದಲಾವಣೆಯ
ಬದಲಾವಣೆ-ಯಾಗು-ವು-ದಿಲ್ಲ
ಬದಲಾವಣೆ-ಯೆಲ್ಲ
ಬದಲಿಗೆ
ಬದಲು
ಬದಿಗೆ
ಬದುಕನ್ನು
ಬದುಕ-ಬಲ್ಲದು
ಬದುಕ-ಬೇಕಾಗಿ-ತ್ತು
ಬದುಕ-ಲಾ-ರದು
ಬದುಕಿ
ಬದುಕಿ-ಕೊಂಡಿದೆ
ಬದುಕಿ-ದ-ನೆಂದೂ
ಬದುಕಿನ
ಬದುಕಿ-ರುವ
ಬದುಕಿ-ರು-ವರು
ಬದುಕಿ-ರು-ವೆವು
ಬದುಕು
ಬದುಕು-ವನು
ಬದುಕು-ವ-ವರು
ಬದ್ಧ-ರಾಗಿ-ರು-ವರು
ಬದ್ಧ-ರಾಗಿ-ರು-ವಿರಿ
ಬದ್ಧ-ರಾಗಿ-ರು-ವೆವು
ಬದ್ಧ-ರಾಗಿ-ರು-ವೆವೋ
ಬದ್ಧರು
ಬದ್ಧ-ರೆಂದು
ಬದ್ಧ-ಲ್ಧಾಗಿ
ಬದ್ಧ-ವಾಗಿದೆ
ಬದ್ಧ-ವಾಗಿ-ರ-ಲಾ-ರದು
ಬದ್ಧ-ವಾಗಿ-ರುವ
ಬದ್ಧ-ವಾಗಿಲ್ಲ
ಬದ್ಧ-ವಾದ
ಬನ್ನಿ
ಬಯಕೆ
ಬಯಕೆ-ಗಳೆಲ್ಲ
ಬಯಲಿಗೆ
ಬಯ-ಸಿತು
ಬಯ-ಸಿದ
ಬಯ-ಸಿದರೆ
ಬಯಸುತ್ತಾನೆ
ಬಯಸುತ್ತಾರೆ
ಬಯ-ಸು-ತ್ತಿದ್ದ
ಬಯ-ಸು-ತ್ತಿದ್ದರು
ಬಯಸುತ್ತಿ-ದ್ದುದು
ಬಯಸು-ತ್ತೇನೆ
ಬಯಸುತ್ತೇವೆ
ಬಯ-ಸುವ
ಬಯ-ಸುವರು
ಬಯ-ಸುವರೋ
ಬಯ-ಸುವ-ವನು
ಬಯ-ಸುವ-ವರು
ಬಯ-ಸು-ವು-ದಾದರೆ
ಬಯಸು-ವೆನು
ಬರ
ಬರ-ಕೂಡದು
ಬರ-ಗಾಲ
ಬರ-ದಿ-ರಲಿ
ಬರದೇ
ಬರ-ಬಹುದು
ಬರ-ಬಹು-ದೆಂದು
ಬರ-ಬೇಕಾಗಿ-ದ್ದರೆ
ಬರ-ಬೇಕಾಗಿ-ರುವ
ಬರ-ಬೇಕಾಗಿವೆ
ಬರ-ಬೇ-ಕಾದ
ಬರ-ಬೇ-ಕಾದರೆ
ಬರ-ಬೇಕು
ಬರ-ಬೇಕೆಂದು
ಬರ-ಲಾ-ರದು
ಬರಲಿ
ಬರ-ಲಿ-ರು-ವು-ದನ್ನು
ಬರ-ಲಿಲ್ಲ
ಬರಲು
ಬರ-ಲೆಂದು
ಬರ-ಲೇ-ಬೇಕು
ಬರ-ವಣಿಗೆಯ
ಬರ-ಹ-ಗಳ
ಬರ-ಹ-ಗಳಿಗೆ
ಬರ-ಹ-ಗಾರ-ರಂತೆ
ಬರಿಯ
ಬರೀ
ಬರು
ಬರು-ತ್ತದೆ
ಬರು-ತ್ತವೆ
ಬರುತ್ತಾ
ಬರು-ತ್ತಾರೆ
ಬರು-ತ್ತಿದೆ
ಬರು-ತ್ತಿದ್ದ
ಬರು-ತ್ತಿದ್ದರು
ಬರು-ತ್ತಿ-ದ್ದುವು
ಬರು-ತ್ತಿ-ರಲಿಲ್ಲ
ಬರು-ತ್ತಿ-ರುವ
ಬರು-ತ್ತಿ-ರು-ವರು
ಬರು-ತ್ತಿರು-ವುದು
ಬರು-ತ್ತಿವೆ
ಬರು-ತ್ತೇನೆ
ಬರು-ತ್ತೇವೆ
ಬರು-ಬರುತ್ತ
ಬರು-ಬರುತ್ತಾ
ಬರುವ
ಬರು-ವಂತೆ
ಬರು-ವನು
ಬರು-ವರು
ಬರು-ವರೋ
ಬರು-ವ-ವ-ರಿಗೆ
ಬರು-ವ-ವರು
ಬರು-ವ-ವರೆಗೂ
ಬರು-ವ-ವ-ರೆಗೆ
ಬರು-ವಾಗ
ಬರು-ವಿರೋ
ಬರು-ವು-ದಕ್ಕೆ
ಬರು-ವು-ದನ್ನು
ಬರು-ವು-ದನ್ನೂ
ಬರು-ವು-ದ-ರಿಂದ
ಬರು-ವು-ದಿಲ್ಲ
ಬರು-ವು-ದಿಲ್ಲವೊ
ಬರು-ವು-ದಿಲ್ಲವೋ
ಬರು-ವುದು
ಬರು-ವು-ದೆಂದು
ಬರು-ವು-ದೆಂದೂ
ಬರು-ವುದೆಂಬುದು
ಬರು-ವುದೇ-ನೆಂದರೆ
ಬರು-ವುದೋ
ಬರು-ವುವು
ಬರೆದ
ಬರೆ-ದದ್ದು
ಬರೆ-ದನು
ಬರೆ-ದನೆಂದಾ-ಗಲಿ
ಬರೆ-ದರು
ಬರೆದಲ್ಲದೆ
ಬರೆದ-ವನು
ಬರೆದ-ವರು
ಬರೆ-ದಿದ್ದ
ಬರೆ-ದಿ-ದ್ದರು
ಬರೆ-ದಿ-ದ್ದಾರೆ
ಬರೆದಿರ-ಬೇಕು
ಬರೆ-ದಿ-ರ-ಲಿಲ್ಲ
ಬರೆದಿರುತ್ತಾರೆ
ಬರೆದಿ-ರುವ
ಬರೆದಿ-ರು-ವನು
ಬರೆದಿ-ರು-ವರು
ಬರೆದಿರು-ವು-ದನ್ನು
ಬರೆದಿರು-ವೆನು
ಬರೆ-ದಿವೆ
ಬರೆದು
ಬರೆದು-ಕೊಳ್ಳಲು
ಬರೆದು-ದನ್ನು
ಬರೆಯ-ಬಹು-ದಾದ
ಬರೆಯ-ಬೇಕಾಗಿಲ್ಲ
ಬರೆಯಬೇಕಾಗು-ವುದು
ಬರೆ-ಯ-ಲಿಲ್ಲ
ಬರೆ-ಯಲು
ಬರೆಯಲ್ಪಟ್ಟ
ಬರೆಯಲ್ಪಟ್ಟವು
ಬರೆಯಲ್ಪಟ್ಟ-ವೆಂಬು-ದನ್ನು
ಬರೆಯಲ್ಪಡ-ಲಿಲ್ಲ
ಬರೆಯು-ವರು
ಬರೆ-ಯು-ವು-ದಕ್ಕೆ
ಬರೆಯುವು-ದ-ರ-ಲ್ಲಿಯೇ
ಬರೆಯು-ವುದು
ಬರೆ-ಯು-ವುದೂ
ಬರೋಣ
ಬರೋ-ಸ್
ಬರೋಸ್ರ-ವರು
ಬರ್ಬ-ರರು
ಬರ್ಮಕ್ಕೆ
ಬರ್ಮೀ-ಯರ
ಬರ್ಮೀ-ಯರು
ಬಲ
ಬಲ-ಗಳನ್ನು
ಬಲ-ಗೈ-ಯಿಂದ
ಬಲ-ಗೈ-ಯಿಂದಲೇ
ಬಲ-ಗೊಳಿ-ಸದೆ
ಬಲ-ಗೊಳಿಸ-ಬೇಕು
ಬಲದ
ಬಲ-ದಿಂದ
ಬಲ-ಪಡಿ-ಸ-ಬೇಕು
ಬಲ-ಪ್ರದವೋ
ಬಲ-ವಾಗ
ಬಲ-ವಾಗಿ
ಬಲ-ವಾದ
ಬಲ-ವಿದೆ
ಬಲ-ಶಾಲಿ-ಗಳಾಗಿ
ಬಲ-ಶಾಲಿ-ಗಳಾದ
ಬಲ-ಶಾಲಿ-ಗಳೂ
ಬಲ-ಹೀನರ
ಬಲ-ಹೀನ-ರ-ನ್ನಾಗಿ
ಬಲ-ಹೀನ-ರಿಗೆ
ಬಲಾಢ್ಯ
ಬಲಾಢ್ಯ-ನ-ನ್ನಾಗಿ
ಬಲಾಢ್ಯ-ನಾಗಿ-ರ-ಬೇಕು
ಬಲಾಢ್ಯ-ರ-ನ್ನಾಗಿ
ಬಲಾಢ್ಯ-ರಾಗ-ಲಾರೆವು
ಬಲಾಢ್ಯ-ರಾಗಿ
ಬಲಾಢ್ಯ-ರಿ-ಗಿಂತ
ಬಲಾಢ್ಯರು
ಬಲಾತ್ಕರಿ-ಸ-ಲಾ-ರರು
ಬಲಾತ್ಕರಿ-ಸಿದರೆ
ಬಲಾತ್ಕರಿಸು-ತ್ತಿವೆ
ಬಲಾತ್ಕರಿ-ಸು-ವು-ದಕ್ಕೆ
ಬಲಾತ್ಕರಿಸು-ವುದು
ಬಲಾತ್ಕಾರ
ಬಲಾತ್ಕಾರ-ದಿಂದ
ಬಲಾತ್ಕಾರ-ವಾಗಲಿ
ಬಲಾತ್ಕಾರ-ವಾಗಿ
ಬಲಿ
ಬಲಿ-ದಾ-ನವೇ
ಬಲಿ-ಯನ್ನು
ಬಲಿ-ಯಾಗದೆ
ಬಲಿ-ಯಾಗು-ವು-ದಕ್ಕೆ
ಬಲಿ-ಯಾ-ದಾ-ಗಲೋ
ಬಲಿಷ್ಠ
ಬಲಿ-ಷ್ಠನೂ
ಬಲಿ-ಷ್ಠ-ರಾಗ-ಬೇಕು
ಬಲಿ-ಷ್ಠ-ರಾಗಿ
ಬಲಿ-ಷ್ಠರು
ಬಲಿ-ಷ್ಠರೂ
ಬಲಿ-ಷ್ಠ-ಳಾಗಿ
ಬಲಿ-ಷ್ಠ-ವಾಗಿ-ರಲಿ
ಬಲೂಚಿ
ಬಲೆ-ಯನ್ನು
ಬಲ್ಕಾತರಿ-ಸಿದರೆ
ಬಲ್ಕಾತರಿ-ಸು-ವು-ದಕ್ಕೆ
ಬಲ್ಲ
ಬಲ್ಲರು
ಬಲ್ಲವು
ಬಲ್ಲಿರಾ
ಬಲ್ಲೆ
ಬಲ್ಲೆವು
ಬಳಕೆ-ಗಾಗಿ
ಬಳಕೆ-ಯಲ್ಲಿ
ಬಳಕೆ-ಯ-ಲ್ಲಿದ್ದ
ಬಳಕೆ-ಯಲ್ಲಿ-ರುವ
ಬಳಕೆ-ಯಲ್ಲಿ-ರು-ವುದು
ಬಳಕೆ-ಯಲ್ಲಿ-ರು-ವುದೊ
ಬಳಲಿರು-ವೆನು
ಬಳಸ-ಬೇ-ಕಾದ
ಬಳಸಿ-ಕೊಂಡಿ-ರು-ವರು
ಬಳ-ಸಿ-ಕೊಂಡು
ಬಳ-ಸಿ-ದ್ದಾರೆ
ಬಳಸಿ-ರು-ವರು
ಬಳ-ಸು-ತ್ತಿ-ದ್ದೇವೆ
ಬಳ-ಸು-ತ್ತಿಲ್ಲ
ಬಳ-ಸುವ
ಬಳ-ಸು-ವರು
ಬಳಸು-ವುದು
ಬಳಿ
ಬಳಿಕ
ಬಳಿಗೆ
ಬಳಿ-ದುಕೊಳ್ಳ-ಬಹುದು
ಬಳಿ-ದು-ಕೊಳ್ಳು-ತ್ತಿದ್ದ
ಬಳಿ-ದು-ಕೊಳ್ಳುವ
ಬಳಿ-ಯಲ್ಲಿ
ಬಳಿ-ಯಲ್ಲೇ
ಬಳಿ-ಯಿ-ರುವ
ಬಳುವಳಿ-ಯಾಗಿ
ಬಸಿದರೋ
ಬಹಳ
ಬಹಳ-ವಾಗಿ
ಬಹಳ-ವಾದ
ಬಹಿ-ರಂಗ
ಬಹಿ-ರಂಗ-ವಾಗಿ
ಬಹಿ-ರನ್ವೇಷಣೆಯು
ಬಹಿರ್ಮುಖ-ವಾ-ಯಿತು
ಬಹಿ-ಷ್ಕ-ರಿ-ಸದೆ
ಬಹಿ-ಷ್ಕರಿ-ಸು-ವರು
ಬಹಿ-ಷ್ಕಾರ
ಬಹಿ-ಷ್ಕಾರ-ವಾಗು-ವು-ದೆಂದು
ಬಹಿ-ಷ್ಕಾರ-ವಿಲ್ಲ
ಬಹು
ಬಹು-ಕಾಲ
ಬಹು-ಕಾಲದ
ಬಹು-ಕಾಲ-ದಿಂದಲೂ
ಬಹು-ಜನರ
ಬಹು-ತ್ವವುಳ್ಳ
ಬಹುದಾ
ಬಹು-ದಾ-ಗಿದೆ
ಬಹುದು
ಬಹುದೋ
ಬಹುಧಾ
ಬಹು-ಧಾ-ವದಂತಿ
ಬಹುನಾ
ಬಹು-ಪಾಲು
ಬಹು-ಪುರಾ-ತನ-ವಾದ
ಬಹು-ಪ್ರ-ಯಾ-ಸ-ದಿಂದ
ಬಹು-ಪ್ರಾಚೀನ
ಬಹು-ಬೇಗ
ಬಹು-ಭಾಗ
ಬಹು-ಭಾಗ-ವನ್ನೆಲ್ಲಾ
ಬಹು-ಮಂದಿ
ಬಹು-ಮಾನ
ಬಹು-ಮುಖ
ಬಹು-ಮುಖ್ಯ-ವೆಂದೂ
ಬಹು-ವಾಗಿ
ಬಹುಶಃ
ಬಾ
ಬಾಂಧ-ವರ
ಬಾಂಧವ-ರಲ್ಲಿ
ಬಾಂಧವ-ರಾಗಿ-ರು-ವುದೇ
ಬಾಂಧ-ವರು
ಬಾಂಧ-ವರೆ
ಬಾಗಿ
ಬಾಗಿ-ಲನ್ನು
ಬಾಗಿ-ಲನ್ನೂ
ಬಾಗಿ-ಲಲ್ಲಿ
ಬಾಗಿಲು
ಬಾಣ
ಬಾಣ-ವನ್ನು
ಬಾಧ-ಕ-ವಿಲ್ಲ
ಬಾಧೆಯೂ
ಬಾಧ್ಯತೆ-ಗಳನ್ನು
ಬಾಧ್ಯತೆ-ಗಳು
ಬಾಧ್ಯರು
ಬಾಯನ್ನು
ಬಾಯಾ-ರಿಕೆ-ಗಳು
ಬಾಯಾ-ರಿಕೆ-ಯಿಂದ
ಬಾಯಿ
ಬಾಯಿಂದ
ಬಾಯಿ-ಗಳು
ಬಾಯಿ-ಮಾ-ತಾಗಿ-ತ್ತು
ಬಾಯಿ-ಮಾತು
ಬಾಯಿ-ಯಿಂದ
ಬಾರದ
ಬಾರ-ದಂತೆ
ಬಾರದವ
ಬಾರದ-ವ-ನಾಗು-ವನು
ಬಾರದ-ವ-ನಾಗು-ವುದೇ
ಬಾರದ-ವ-ನೆಂದು
ಬಾರದ-ವ-ರಾಗು-ತ್ತೀರಿ
ಬಾರದ-ವ-ರಾಗು-ವುದು
ಬಾರದ-ವರೂ
ಬಾರದ-ವ-ರೆಂದು
ಬಾರ-ದವು
ಬಾರದಿರು-ವುದು
ಬಾರದು
ಬಾರ-ದುದು
ಬಾರದೆ
ಬಾರದೇ
ಬಾರಿ
ಬಾರಿಗೂ
ಬಾಲಕ
ಬಾಲಕ-ನಿಗೂ
ಬಾಲ-ಕನು
ಬಾಲಕ-ರಾಗಿ-ರು-ವಾಗ
ಬಾಲಕ-ರೆಲ್ಲರೂ
ಬಾಲಿಶ-ವಾದ
ಬಾಲಿಶ-ವಾದುದು
ಬಾಲ್
ಬಾಲ್ಯ-ಕಾಲ-ದಲ್ಲಿ
ಬಾಲ್ಯ-ದಿಂದಲೂ
ಬಾಲ್ಯ-ದಿಂದಲೇ
ಬಾಲ್ಯಾ-ರಭ್ಯ
ಬಾಳನ್ನು
ಬಾಳ-ಬಲ್ಲೆವು
ಬಾಳ-ಬಹುದು
ಬಾಳ-ಬೇಕಾಗಿದೆ
ಬಾಳ-ಬೇ-ಕಾದರೆ
ಬಾಳ-ಲಾ-ರದು
ಬಾಳಲಿ
ಬಾಳಲು
ಬಾಳಿ
ಬಾಳಿಗೆ
ಬಾಳಿ-ದ-ವ-ನನ್ನು
ಬಾಳಿದೆ
ಬಾಳಿನ
ಬಾಳಿ-ನಂತೆ
ಬಾಳಿ-ನಲ್ಲಿ
ಬಾಳು
ಬಾಳು-ತ್ತಿದ್ದನು
ಬಾಳು-ತ್ತಿರಬೇ-ಕಿತ್ತು
ಬಾಳು-ತ್ತಿರು-ವೆವು
ಬಾಳುವ
ಬಾಳು-ವಂತಹ
ಬಾಳು-ವಂತಾಗಲಿ
ಬಾಳು-ವುದು
ಬಾಳು-ವೆಯೆ
ಬಾವಿಗೆ
ಬಾವಿ-ಯನ್ನು
ಬಾವುಟ-ವನ್ನು
ಬಾಹಿ-ರ-ರಿಗೆ
ಬಾಹು-ಗಳ
ಬಾಹು-ಗಳನ್ನು
ಬಾಹು-ಗಳಿಂದ
ಬಾಹು-ಗಳು
ಬಾಹು-ಬಲದ
ಬಾಹ್ಯ
ಬಾಹ್ಯ-ಕ್ರಿಯೆ-ಗಳು
ಬಾಹ್ಯ-ಕ್ರಿಯೆಯ
ಬಾಹ್ಯ-ಜಗತ್ತನ್ನು
ಬಾಹ್ಯದ
ಬಾಹ್ಯ-ದಲ್ಲಿ
ಬಾಹ್ಯ-ದಿಂದ
ಬಾಹ್ಯ-ದೃಷ್ಟಿಯ
ಬಾಹ್ಯ-ಪೂಜೆ
ಬಾಹ್ಯ-ಪೂಜೆ-ಯನ್ನು
ಬಾಹ್ಯ-ಪ್ರ-ಕೃತಿ
ಬಾಹ್ಯ-ಪ್ರ-ಕೃತಿಗೆ
ಬಾಹ್ಯ-ಪ್ರ-ಕೃತಿಯ
ಬಾಹ್ಯ-ಪ್ರ-ಕೃತಿ-ಯನ್ನು
ಬಾಹ್ಯ-ಪ್ರ-ಕೃತಿ-ಯಿಂದ
ಬಾಹ್ಯ-ಪ್ರ-ಪಂಚದ
ಬಾಹ್ಯ-ಪ್ರ-ಪಂಚ-ದಲ್ಲಿ-ರುವ
ಬಾಹ್ಯ-ವನ್ನು
ಬಾಹ್ಯ-ವರ್ಣನೆ
ಬಾಹ್ಯ-ವಸ್ತು-ಗಳ
ಬಾಹ್ಯ-ವಸ್ತು-ವಿನ
ಬಾಹ್ಯ-ಶಕ್ತಿಯ
ಬಾಹ್ಯ-ಶಕ್ತಿಯೂ
ಬಾಹ್ಯ-ಶುದ್ಧಿ
ಬಾಹ್ಯಾ
ಬಾಹ್ಯಾ-ಚ-ರಣೆ
ಬಾಹ್ಯಾ-ಚಾರ
ಬಾಹ್ಯಾ-ಚಾರ-ಗಳನ್ನು
ಬಾಹ್ಯಾ-ಚಾರ-ಗಳನ್ನೂ
ಬಾಹ್ಯಾ-ಚಾರ-ಗಳು
ಬಾಹ್ಯಾ-ಚಾರ-ದಲ್ಲಿ
ಬಾಹ್ಯಾ-ಡಂಬರ-ಗಳಿಗೆ
ಬಾಹ್ಯೇಂದ್ರಿ-ಗಳ
ಬಿಂದು-ಗಳಂತೆ
ಬಿಂದುವಿ-ನಂತೆ
ಬಿಂದುವಿ-ನ-ಲ್ಲಿಯೂ
ಬಿಂದುವಿ-ನಲ್ಲೂ
ಬಿಂಬ
ಬಿಗಿ-ಯಾಗಿ
ಬಿಚ್ಚಿ-ಹಿ-ಡಿ-ಯಿರಿ
ಬಿಚ್ಚು-ವರು
ಬಿಟ್ಟ
ಬಿಟ್ಟ-ನೆಂದು
ಬಿಟ್ಟ-ನೆ-ನ್ನು-ವರು
ಬಿಟ್ಟರು
ಬಿಟ್ಟರೂ
ಬಿಟ್ಟರೆ
ಬಿಟ್ಟಿದೆ
ಬಿಟ್ಟಿ-ದ್ದಾರೆ
ಬಿಟ್ಟು
ಬಿಟ್ಟು-ಕೊ-ಡದೆ
ಬಿಟ್ಟು-ಕೊಡಿ
ಬಿಟ್ಟುದು
ಬಿಟ್ಟು-ಬಿಟ್ಟರು
ಬಿಟ್ಟು-ಬಿಡು-ವುದೆ
ಬಿಡದೆ
ಬಿಡ-ಬೇಕು
ಬಿಡ-ಬೇಡಿ
ಬಿಡ-ಲಾಗ-ದಿ-ದ್ದರೆ
ಬಿಡಲಿ
ಬಿಡಲು
ಬಿಡಿ
ಬಿಡಿ-ಸ-ಬೇಕಾಗಿದೆ
ಬಿಡಿ-ಸಲಾ-ರದ
ಬಿಡಿ-ಸಲು
ಬಿಡಿಸಿ
ಬಿಡು-ಗಡೆ-ಗಾಗಿ
ಬಿಡುತ್ತಾರೆ
ಬಿಡುವ
ಬಿಡು-ವಂತೆ
ಬಿಡು-ವಿಲ್ಲದ
ಬಿಡು-ವಿಲ್ಲ-ದಷ್ಟು
ಬಿಡು-ವಿ-ಲ್ಲದೆ
ಬಿಡು-ವು-ದಕ್ಕೆ
ಬಿಡುವುದರೊಳ-ಗಾಗಿ
ಬಿಡು-ವು-ದಿಲ್ಲ
ಬಿಡು-ವುದು
ಬಿತ್ತಿ
ಬಿತ್ತಿದ
ಬಿತ್ತಿ-ದ-ವ-ರಾರು
ಬಿತ್ತಿದೆ
ಬಿತ್ತೆಂದು
ಬಿದ್ದಂತೆ-ನೀವೂ
ಬಿದ್ದರೂ
ಬಿದ್ದರೆ
ಬಿದ್ದ-ವ-ನನ್ನು
ಬಿದ್ದ-ವನು
ಬಿದ್ದಿದೆ
ಬಿದ್ದಿದ್ದ
ಬಿದ್ದಿ-ದ್ದರೆ
ಬಿದ್ದಿ-ರುವ
ಬಿದ್ದಿರು-ವೆವು
ಬಿದ್ದಿಲ್ಲ
ಬಿದ್ದು
ಬಿದ್ಧಿರು-ವುದು
ಬಿನ್ನ
ಬಿನ್ನ-ವತ್ತಳೆ-ಗಳ
ಬಿನ್ನ-ವತ್ತಳೆ-ಗಳನ್ನು
ಬಿನ್ನ-ವತ್ತಳೆಗೆ
ಬಿನ್ನ-ವತ್ತ-ಳೆಯ
ಬಿನ್ನ-ವತ್ತ-ಳೆಯನ್ನು
ಬಿನ್ನ-ವತ್ತ-ಳೆಯಲ್ಲಿ
ಬಿನ್ನ-ವತ್ತ-ಳೆಯು
ಬಿರಿ-ಯುವ
ಬಿರುಕು-ಬಿಟ್ಟವು
ಬಿರುದು
ಬಿಲ್ಲಿನ
ಬಿಳಿಕಾಗ-ದದ
ಬಿಸಾಡಿ
ಬಿಸಾಡು-ವರು
ಬಿಸಿ
ಬಿಸುಸುಯ್ದರೂ
ಬೀಜ
ಬೀಜ-ಗಳು
ಬೀಜದ
ಬೀಜ-ರೂಪ-ದಲ್ಲಿವೆ
ಬೀಜ-ವನ್ನು
ಬೀಜವು
ಬೀದಿ-ಗಳನ್ನು
ಬೀದಿ-ಗಳಲ್ಲಿ
ಬೀದಿ-ಯಲ್ಲಿ
ಬೀಭತ್ಸ
ಬೀರದೆ-ಮಾ-ಯ-ವಾಗಿ
ಬೀರ-ಬಲ್ಲ
ಬೀರ-ಲಾ-ಗ-ಲಿಲ್ಲ
ಬೀರಲಿ
ಬೀರಲು
ಬೀರಿತು
ಬೀರಿದ
ಬೀರಿ-ದಷ್ಟು
ಬೀರಿದೆ
ಬೀರಿದ್ದು
ಬೀರಿರು-ವುದು
ಬೀರು-ತ್ತದೆ
ಬೀರುತ್ತಾ
ಬೀರು-ತ್ತಿದೆ
ಬೀರುತ್ತಿ-ರುವ
ಬೀರುತ್ತಿ-ರು-ವರು
ಬೀರುತ್ತಿರು-ವುದು
ಬೀರು-ತ್ತಿವೆ
ಬೀರುತ್ತಿ-ವೆಯೋ
ಬೀರು-ವನು
ಬೀರುವು-ದಾ-ದರೂ
ಬೀರು-ವುದು
ಬೀಳತೊಡಗಿತು
ಬೀಳ-ಬೇಕು
ಬೀಳು-ತ್ತದೆ
ಬೀಳು-ತ್ತಿದೆಯೋ
ಬೀಳುತ್ತಿ-ರುವ
ಬೀಳುತ್ತಿರು-ವುದು
ಬೀಳು-ತ್ತಿವೆ
ಬೀಳುವ
ಬೀಳು-ವಂತೆ
ಬೀಳು-ವರು
ಬೀಳು-ವಾಗ
ಬೀಳು-ವುದು
ಬೀಳು-ವುವು
ಬೀಳೋಣ
ಬೀಳ್ಕೊಡಿ-ಗೆ-ಯನ್ನು
ಬೀಳ್ಕೊಳ್ಳು-ತ್ತಿದೆ
ಬೀಸುತ್ತಾ-ನೆ-ಶತ್ರು-ನಾಶ-ಕ್ಕಾಗಿ
ಬೀಸು-ತ್ತಿದೆ
ಬೀಸುವ
ಬುಡ-ಕಟ್ಟಿ-ನವ-ರಿಗೂ
ಬುಡಕಟ್ಟು-ಗಳಿಗೆ
ಬುದ್ದಿ
ಬುದ್ದಿಯ
ಬುದ್ಧ
ಬುದ್ಧ-ದೇವನ
ಬುದ್ಧ-ದೇ-ವನು
ಬುದ್ಧ-ದೇ-ವನೂ
ಬುದ್ಧನ
ಬುದ್ಧ-ನಂತಹ
ಬುದ್ಧ-ನನ್ನು
ಬುದ್ಧ-ನಲ್ಲ
ಬುದ್ಧ-ನಲ್ಲಿ
ಬುದ್ಧ-ನಾ-ಗಲು
ಬುದ್ಧನು
ಬುದ್ಧನೇ
ಬುದ್ಧನೋ
ಬುದ್ಧಿ
ಬುದ್ಧಿ-ಗಳ
ಬುದ್ಧಿಗೆ
ಬುದ್ಧಿ-ಜೀವಿ-ಗಳೂ
ಬುದ್ಧಿ-ಯಲ್ಲಿ
ಬುದ್ಧಿ-ಯಿಂದ
ಬುದ್ಧಿ-ಯಿಂದಲೂ
ಬುದ್ಧಿ-ಯಿಲ್ಲ
ಬುದ್ಧಿ-ಯೊಡನೆ
ಬುದ್ಧಿ-ವಂತ-ನಾಗ-ಬಲ್ಲ
ಬುದ್ಧಿ-ವಂತ-ನಾ-ಗಿದ್ದರೆ
ಬುದ್ಧಿ-ವಂತ-ರಾಗಿ-ಲ್ಲದೇ
ಬುದ್ಧಿ-ವಂತ-ರಾ-ಗುವ-ವ-ರೆಗೆ
ಬುದ್ಧಿ-ವಂತರು
ಬುದ್ಧಿ-ವಂತಿಕೆ
ಬುದ್ಧಿ-ವಾದ
ಬುದ್ಧಿ-ವಾ-ನ್
ಬುದ್ಧಿ-ಶಕ್ತಿ-ಯನ್ನು
ಬುದ್ಧಿ-ಶಕ್ತಿ-ಯಿಂದಲೂ
ಬುದ್ಧಿ-ಶಾಲಿ
ಬುದ್ಭುದ-ಗಳು
ಬುನಾದಿ
ಬೂದಿ
ಬೃಂದಾವ-ನದ
ಬೃಂಧಾವ-ನದ
ಬೃಹ-ತ್
ಬೃಹತ್ಕಥೆ
ಬೃಹತ್ತಾಗಿ
ಬೃಹತ್ತಾ-ಗಿದೆ
ಬೃಹ-ತ್ತಾದ
ಬೃಹ-ತ್ತಿನ
ಬೃಹದಾ-ಕಾರದ
ಬೃಹದಾ-ಕಾರ-ವನ್ನು
ಬೆಂಕಿ
ಬೆಂಕಿಯ
ಬೆಂಕಿ-ಯಂತೆ
ಬೆಂಕಿ-ಯನ್ನು
ಬೆಂಕಿ-ಯ-ನ್ನೋ
ಬೆಂಬಲ
ಬೆಕ್ಕು
ಬೆಟ್ಟ-ಗಳಲ್ಲಿ
ಬೆಟ್ಟ-ದಷ್ಟು
ಬೆತ್ತದ
ಬೆದರಿ-ಸುತ್ತಿರ-ಬೇಕು
ಬೆದರಿ-ಸುತ್ತಿರು-ವಾಗ
ಬೆನ್ನಟ್ಟಿ
ಬೆನ್ನೆ-ಲುಬ-ನ್ನಾಗಿ
ಬೆನ್ನೆ-ಲು-ಬಾ-ಗಿದೆ
ಬೆನ್ನೆ-ಲುಬಿನಂತಿರು-ವು-ದನ್ನು
ಬೆನ್ನೆ-ಲುಬು
ಬೆನ್ನೆ-ಲುಬೇ
ಬೆರಗು-ಗೊಳಿಸಿ
ಬೆರಳಿ-ನಿಂದ
ಬೆರೆತ
ಬೆರೆ-ತರು
ಬೆರೆ-ತರೆ
ಬೆರೆತು
ಬೆರೆಯ-ಬೇಕು
ಬೆರೆ-ಯು-ವು-ದಕ್ಕೆ
ಬೆಲೂಚಿ
ಬೆಲೆ
ಬೆಲೆಯೇ
ಬೆಳಕನ್ನು
ಬೆಳಕಿಗೆ
ಬೆಳಕಿನ
ಬೆಳಕಿ-ನಲ್ಲಿ
ಬೆಳಕಿ-ನಿಂದ
ಬೆಳಕಿಲ್ಲ
ಬೆಳಕು
ಬೆಳಗಲಾರ
ಬೆಳಗ-ಲಾ-ರದು
ಬೆಳಗಲಾ-ರವು
ಬೆಳಗಾ-ದರೆ
ಬೆಳಗಿತು
ಬೆಳಗಿ-ದರೆ
ಬೆಳ-ಗು-ತ್ತಿದೆ
ಬೆಳಗುತ್ತಿರು-ವುದು
ಬೆಳ-ಗು-ವಂತೆ
ಬೆಳ-ಗು-ವು-ದಿಲ್ಲ
ಬೆಳಗು-ವುವು
ಬೆಳವಣಿಗೆ
ಬೆಳವಣಿಗೆ-ಗಾಗಿ
ಬೆಳವಣಿ-ಗೆಗೆ
ಬೆಳ-ವಣಿಗೆಯ
ಬೆಳವಣಿಗೆ-ಯನ್ನು
ಬೆಳವಣಿಗೆ-ಯಲ್ಲಿ
ಬೆಳಸಿ
ಬೆಳಸಿ-ಕೊಳ್ಳು-ವಿರಿ
ಬೆಳಸಿ-ದರೂ
ಬೆಳಸುವು-ದ-ಕ್ಕಾಗಿ
ಬೆಳೆದ
ಬೆಳೆ-ದರು
ಬೆಳೆದ-ವ-ರಿಗೂ
ಬೆಳೆ-ದಷ್ಟು
ಬೆಳೆ-ದಿದೆ
ಬೆಳೆ-ದಿ-ದ್ದರೆ
ಬೆಳೆದಿರು-ವೆನು
ಬೆಳೆದಿರು-ವೆವು
ಬೆಳೆ-ದಿಲ್ಲ
ಬೆಳೆದು
ಬೆಳೆದು-ಬಂದಿದೆ
ಬೆಳೆಯ-ಬೇಕಾಗಿದೆ
ಬೆಳೆಯ-ಬೇಕಾಗಿ-ರು-ವುದೇ
ಬೆಳೆಯ-ಬೇಕು
ಬೆಳೆಯ-ಬೇಕೆಂದು
ಬೆಳೆ-ಯಲಿ
ಬೆಳೆ-ಯಲು
ಬೆಳೆ-ಯಿತು
ಬೆಳೆಯು-ತ್ತಿತ್ತು
ಬೆಳೆ-ಯು-ತ್ತಿದೆ
ಬೆಳೆ-ಯು-ವನು
ಬೆಳೆಯು-ವುದು
ಬೆಳೆ-ಸಿದ್ದೇವೆ
ಬೆಳ್ಳಿಯ
ಬೆಳ್ಳುಳ್ಳಿ
ಬೆಸೆಂಟ್
ಬೆಸ್ತ
ಬೆಸ್ತ-ನಾಗು-ವನು
ಬೆಸ್ತರ
ಬೆಸ್ತ-ರ-ಪಡೆ-ಯನ್ನೆಲ್ಲಾ
ಬೆಸ್ತ-ರಿಗೂ
ಬೆಸ್ತರು
ಬೇಕಷ್ಟೆ
ಬೇಕಾಗಿ-ತ್ತು
ಬೇಕಾಗಿದೆ
ಬೇಕಾಗಿ-ದೆ-ಅದೇ
ಬೇಕಾಗಿ-ದೆಯೋ
ಬೇಕಾಗಿದ್ದ
ಬೇಕಾಗಿ-ದ್ದರೆ
ಬೇಕಾಗಿ-ದ್ದಾರೆ
ಬೇಕಾಗಿ-ದ್ದುದ್ದು
ಬೇಕಾಗಿ-ರ-ಲಿಲ್ಲ
ಬೇಕಾಗಿ-ರುವ
ಬೇಕಾಗಿ-ರು-ವುದು
ಬೇಕಾಗಿ-ರು-ವುದೇ
ಬೇಕಾಗಿಲ್ಲ
ಬೇಕಾ-ಗು-ತ್ತದೆ
ಬೇಕಾ-ಗುತ್ತವೆ
ಬೇಕಾದ
ಬೇಕಾದರೂ
ಬೇಕಾದರೆ
ಬೇಕಾದ-ಷ್ಟನ್ನು
ಬೇಕಾ-ದಷ್ಟು
ಬೇಕಾ-ದಷ್ಟೇ
ಬೇಕಾದು-ದನ್ನು
ಬೇಕಾ-ದುದು
ಬೇಕಿಲ್ಲ
ಬೇಕು
ಬೇಕೆ
ಬೇಕೆಂದಾ-ಗಲಿ
ಬೇಕೆಂದಿ-ರುವ
ಬೇಕೆಂದಿರು-ವುದೆ-ಲ್ಲವೂ
ಬೇಕೆಂದು
ಬೇಕೆಂದೇ
ಬೇಕೆಂಬ
ಬೇಕೆಂಬುದೇ
ಬೇಕೋ
ಬೇಗ
ಬೇಗನೆ
ಬೇಗುದಿ-ಯಿಂದ
ಬೇಗೆ
ಬೇಜಾರಾ-ದಂತೆ
ಬೇಜಾರು
ಬೇಟೆ-ಯಾ-ಡು-ವುದು
ಬೇಡ
ಬೇಡವೆ
ಬೇಡವೇ
ಬೇಡಿ
ಬೇಡಿ-ಕೊಂಡು
ಬೇಡಿ-ಕೊಳ್ಳು-ತ್ತೇವೆ
ಬೇಡಿದೆ
ಬೇಡುತ್ತೇವೆ
ಬೇಡು-ವುದು
ಬೇತು
ಬೇಯಿಸಿ
ಬೇಯು-ವುದು
ಬೇರಲ್ಲ
ಬೇರಾರೂ
ಬೇರಾವ
ಬೇರಾ-ವುದೂ
ಬೇರು
ಬೇರು-ಬಿಡು-ವಂತೆ
ಬೇರೂರಿ
ಬೇರೂರು-ತ್ತಿವೆ
ಬೇರೆ
ಬೇರೆಯ
ಬೇರೆ-ಯಲ್ಲ
ಬೇರೆ-ಯಾಗ-ಬೇಕು
ಬೇರೆ-ಯಾಗಿ
ಬೇರೆ-ಯಾಗಿ-ತ್ತು
ಬೇರೆ-ಯಾಗಿದೆ
ಬೇರೆ-ಯಾಗಿಯೇ
ಬೇರೆ-ಯಾಗಿ-ರು-ವನು
ಬೇರೆ-ಯಾಗಿ-ರು-ವು-ದನ್ನು
ಬೇರೆ-ಯಾಗಿ-ರು-ವುದು
ಬೇರೆ-ಯಾಗು-ತ್ತಿ-ದ್ದೇವೆ
ಬೇರೆ-ಯಾದ
ಬೇರೆ-ಯಾ-ದರೂ
ಬೇರೆ-ಯಾ-ದುದು
ಬೇರೆಲ್ಲಿ
ಬೇರೆ-ಲ್ಲಿಂದಲೂ
ಬೇರೆ-ಲ್ಲಿಯೂ
ಬೇರೊಂದನ್ನು
ಬೇರೊಂದಿಲ್ಲ
ಬೇರೊಂದು
ಬೇರ್ಪಡಿ-ಸಲಾ-ರದ
ಬೇಲೂರು
ಬೇಸಗೆಯ
ಬೇಸಾಯ
ಬೇಸಿಗೆ
ಬೈದರೂ
ಬೈಬಲ್
ಬೈಬಲ್ಲು
ಬೊಂಬಾಯಿ
ಬೊಗಳಿದರೆ
ಬೋಧಕ
ಬೋಧಕ-ನಲ್ಲ
ಬೋಧಕ-ನಾಗಿ-ರುವ-ನೆಂದು
ಬೋಧಕ-ನೆಂದು
ಬೋಧ-ಕರ
ಬೋಧಕ-ರನ್ನು
ಬೋಧಕ-ರಾಗಿ
ಬೋಧತ
ಬೋಧನೆ
ಬೋಧ-ನೆ-ಗಳ
ಬೋಧ-ನೆ-ಗಳನ್ನು
ಬೋಧ-ನೆ-ಗಳಿಂದ
ಬೋಧ-ನೆ-ಗಳು
ಬೋಧ-ನೆ-ಗಾಗಿ
ಬೋಧ-ನೆಗೆ
ಬೋಧ-ನೆಯ
ಬೋಧ-ನೆ-ಯನ್ನು
ಬೋಧ-ನೆ-ಯಲ್ಲಿ
ಬೋಧ-ನೆ-ಯಿಂದ
ಬೋಧ-ನೆಯು
ಬೋಧ-ನೆಯೂ
ಬೋಧಾ-ಯನ
ಬೋಧಾ-ಯನ-ದಿಂದ
ಬೋಧಾ-ಯನರ
ಬೋಧಾ-ಯ-ನರು
ಬೋಧಿ
ಬೋಧಿ-ಸ-ಕೂಡದು
ಬೋಧಿ-ಸ-ಬಲ್ಲ
ಬೋಧಿ-ಸ-ಬೇಕಾಗಿದೆ
ಬೋಧಿ-ಸ-ಬೇಕಾಗಿ-ರು-ವುದು
ಬೋಧಿ-ಸ-ಬೇಕಾಗಿ-ರು-ವು-ದೊಂದು
ಬೋಧಿ-ಸ-ಬೇ-ಕಾದ
ಬೋಧಿ-ಸ-ಬೇಕು
ಬೋಧಿ-ಸ-ಬೇಕೆಂದಿರು-ವಿರೋ
ಬೋಧಿ-ಸ-ಬೇಡಿ
ಬೋಧಿ-ಸ-ಲಾ-ಗ-ಲಿಲ್ಲ
ಬೋಧಿ-ಸ-ಲಾ-ಗಿದೆ
ಬೋಧಿ-ಸಲಿ-ಚ್ಛಿ-ಸು-ವರೋ
ಬೋಧಿ-ಸ-ಲಿಲ್ಲ
ಬೋಧಿ-ಸಲು
ಬೋಧಿ-ಸಲ್ಪಟ್ಟಿತು
ಬೋಧಿಸಿ
ಬೋಧಿ-ಸಿದ
ಬೋಧಿ-ಸಿ-ದನು
ಬೋಧಿ-ಸಿ-ದರೂ
ಬೋಧಿ-ಸಿ-ದರೆ
ಬೋಧಿ-ಸಿ-ದರೋ
ಬೋಧಿ-ಸಿ-ದ-ವ-ನಲ್ಲ
ಬೋಧಿ-ಸಿ-ದ-ವನು
ಬೋಧಿ-ಸಿ-ದ-ವರೂ
ಬೋಧಿ-ಸಿ-ದುದು
ಬೋಧಿ-ಸಿದೆ
ಬೋಧಿ-ಸಿ-ದ್ದೀರಿ
ಬೋಧಿ-ಸಿ-ರು-ವರು
ಬೋಧಿ-ಸಿ-ರು-ವುದು
ಬೋಧಿ-ಸಿಲ್ಲ
ಬೋಧಿ-ಸುತ್ತ
ಬೋಧಿ-ಸುತ್ತದೆ
ಬೋಧಿ-ಸುತ್ತ-ದೆಯೋ
ಬೋಧಿ-ಸುತ್ತವೆ
ಬೋಧಿ-ಸುತ್ತಾರೆ
ಬೋಧಿ-ಸು-ತ್ತಿದ್ದಾಗ್ಯೂ
ಬೋಧಿ-ಸು-ತ್ತಿದ್ದೆ
ಬೋಧಿ-ಸುತ್ತಿ-ರು-ವನು
ಬೋಧಿ-ಸು-ತ್ತೀ-ರೆಂದು
ಬೋಧಿ-ಸು-ತ್ತೇನೆ
ಬೋಧಿ-ಸುತ್ತೇ-ನೆಂದು
ಬೋಧಿ-ಸುವ
ಬೋಧಿ-ಸು-ವಂತೆ
ಬೋಧಿ-ಸು-ವರು
ಬೋಧಿ-ಸುವ-ವ-ನಲ್ಲ
ಬೋಧಿ-ಸುವ-ವರು
ಬೋಧಿ-ಸು-ವಷ್ಟು
ಬೋಧಿ-ಸುವು-ದ-ಕ್ಕಾಗಿ
ಬೋಧಿ-ಸು-ವು-ದಕ್ಕೆ
ಬೋಧಿ-ಸು-ವು-ದರ
ಬೋಧಿ-ಸು-ವು-ದಿಲ್ಲ
ಬೋಧಿ-ಸು-ವುದು
ಬೋಧಿ-ಸೋಣ
ಬೋಧೆ-ಯಿಂದ
ಬೌದ್ದರು
ಬೌದ್ಧ
ಬೌದ್ಧ-ಧರ್ಮ
ಬೌದ್ಧ-ಧರ್ಮಕ್ಕೆ
ಬೌದ್ಧ-ಧರ್ಮದ
ಬೌದ್ಧ-ಧರ್ಮ-ವನ್ನು
ಬೌದ್ಧ-ಧರ್ಮವು
ಬೌದ್ಧ-ಧರ್ಮವೂ
ಬೌದ್ಧ-ಧರ್ಮವೇ
ಬೌದ್ಧ-ನಾಗಿ
ಬೌದ್ಧನು
ಬೌದ್ಧರ
ಬೌದ್ಧ-ರಾಗಲೀ
ಬೌದ್ಧ-ರಾಗಿ
ಬೌದ್ಧ-ರಾಗಿ-ದ್ದರು
ಬೌದ್ಧ-ರಿಂದ
ಬೌದ್ಧ-ರಿ-ಗಿಂತಲೂ
ಬೌದ್ಧ-ರಿಗೆ
ಬೌದ್ಧರು
ಬೌದ್ಧರೂ
ಬೌದ್ಧ-ರೆಲ್ಲಾ
ಬೌದ್ಧರೇ
ಬೌದ್ಧ-ವಿ-ಹಾ-ರ-ದಲ್ಲಿ
ಬೌದ್ಧಿಕ
ಬೌದ್ಧಿಕ-ವಾಗಿ
ಬ್ಧೇಕ್ಧಾದ್ಧಷ್ಟು
ಬ್ಯಾ-ಬಿಲೋನಿ-ಯನರ
ಬ್ಯಾ-ಬಿಲೋನಿ-ಯ-ನರು
ಬ್ಯಾ-ಬಿಲೋ-ನಿಯ-ನ್
ಬ್ಯಾ-ಬಿಲೋನಿಯಾ
ಬ್ಯಾ-ಬಿಲೋನಿ-ಯಾ-ದ-ವರು
ಬ್ಯಾ-ಬೀಲೋನಿ-ಯಾದ
ಬ್ರಹ್ಮ
ಬ್ರಹ್ಮ-ಚರ್ಯ
ಬ್ರಹ್ಮ-ಚರ್ಯಾ-ಶ್ರಮ
ಬ್ರಹ್ಮ-ಚಾರಿ-ಗಳಾಗಿ
ಬ್ರಹ್ಮ-ಚಿಂತ-ನೆ-ಯಲ್ಲಿ
ಬ್ರಹ್ಮ-ಜ್ಞ-ಪುರುಷ
ಬ್ರಹ್ಮ-ಜ್ಞ-ರಲ್ಲಿ
ಬ್ರಹ್ಮ-ಜ್ಞರಿ-ರು-ವರು
ಬ್ರಹ್ಮ-ಜ್ಞಾನಿ-ಯಾಗಿ-ರ-ಬೇಕು
ಬ್ರಹ್ಮಣಿ
ಬ್ರಹ್ಮ-ತ್ವ-ವನ್ನು
ಬ್ರಹ್ಮದ
ಬ್ರಹ್ಮ-ದಷ್ಟೇ
ಬ್ರಹ್ಮ-ದೊಡನೆ
ಬ್ರಹ್ಮನ
ಬ್ರಹ್ಮ-ನನ್ನು
ಬ್ರಹ್ಮ-ನಲ್ಲ
ಬ್ರಹ್ಮ-ನಲ್ಲಿ
ಬ್ರಹ್ಮ-ನ-ವ-ರೆಗೆ
ಬ್ರಹ್ಮ-ನಾಗಿ-ರುವನೋ
ಬ್ರಹ್ಮ-ನಾದರೆ
ಬ್ರಹ್ಮ-ನೆಂದರೆ
ಬ್ರಹ್ಮ-ನೆಂಬ
ಬ್ರಹ್ಮ-ಪುತ್ರ-ದ-ವ-ರೆಗೆ
ಬ್ರಹ್ಮರ
ಬ್ರಹ್ಮ-ರಾಜ-ನ್ಯಾಭ್ಯಾಂ
ಬ್ರಹ್ಮರು
ಬ್ರಹ್ಮ-ಲೋಕ-ವಾಗಿ
ಬ್ರಹ್ಮ-ವನ್ನು
ಬ್ರಹ್ಮ-ವಾದ
ಬ್ರಹ್ಮ-ವಾದ-ದಲ್ಲಿ
ಬ್ರಹ್ಮ-ವಾದ-ದಿಂದ
ಬ್ರಹ್ಮ-ವಾದರೆ
ಬ್ರಹ್ಮ-ವಿತ್ತಮಃ-ವೇದ-ಗಳನ್ನು
ಬ್ರಹ್ಮವು
ಬ್ರಹ್ಮವೂ
ಬ್ರಹ್ಮವೇ
ಬ್ರಹ್ಮ-ಸೂತ್ರ-ಗಳಿಗೆ
ಬ್ರಹ್ಮ-ಸೂತ್ರ-ದಲ್ಲಿ
ಬ್ರಹ್ಮ-ಸೂತ್ರ-ವೃತ್ತಿಂ
ಬ್ರಹ್ಮ-ಸ್ವ-ರೂಪ-ದಲ್ಲಿ
ಬ್ರಹ್ಮಾಂಡ
ಬ್ರಹ್ಮಾಂಡಂ
ಬ್ರಹ್ಮಾಂಡದ
ಬ್ರಹ್ಮಾಂಡ-ದಲ್ಲಿ
ಬ್ರಹ್ಮಾಂಡ-ವನ್ನು
ಬ್ರಹ್ಮಾಂಡ-ವೆಂದು
ಬ್ರಹ್ಮಾಂಡ-ವೆಲ್ಲ
ಬ್ರಹ್ಮಾಂಡವೇ
ಬ್ರಹ್ಮಾ-ನು-ಭವ
ಬ್ರಹ್ಮಾ-ವರ್ತ-ವೆಂದು
ಬ್ರಹ್ಮೋಪಾಸಕ-ರಾಗಿ-ದ್ದರೆ
ಬ್ರಹ್ಮೋಪಾಸನೆ
ಬ್ರಾಹ್ಮಣ
ಬ್ರಾಹ್ಮಣ-ಗಳು
ಬ್ರಾಹ್ಮಣನ
ಬ್ರಾಹ್ಮಣ-ನಂತೆ
ಬ್ರಾಹ್ಮಣ-ನ-ನ್ನಾಗಿ
ಬ್ರಾಹ್ಮಣ-ನನ್ನು
ಬ್ರಾಹ್ಮಣ-ನಲ್ಲಿ
ಬ್ರಾಹ್ಮಣ-ನಾಗಿ
ಬ್ರಾಹ್ಮಣ-ನಿಂದ
ಬ್ರಾಹ್ಮಣ-ನಿ-ಗಲ್ಲ
ಬ್ರಾಹ್ಮಣ-ನಿ-ಗಾಗಿ
ಬ್ರಾಹ್ಮಣ-ನಿಗೆ
ಬ್ರಾಹ್ಮಣನು
ಬ್ರಾಹ್ಮಣನೂ
ಬ್ರಾಹ್ಮಣನೇ
ಬ್ರಾಹ್ಮಣರ
ಬ್ರಾಹ್ಮಣ-ರದು
ಬ್ರಾಹ್ಮಣ-ರ-ನ್ನಾಗಿ
ಬ್ರಾಹ್ಮಣ-ರನ್ನು
ಬ್ರಾಹ್ಮಣ-ರಲ್ಲ
ಬ್ರಾಹ್ಮಣ-ರಲ್ಲಿ
ಬ್ರಾಹ್ಮಣ-ರ-ಲ್ಲಿ-ತ್ತು
ಬ್ರಾಹ್ಮಣ-ರಾಗಲೀ
ಬ್ರಾಹ್ಮಣ-ರಾಗಲು
ಬ್ರಾಹ್ಮಣ-ರಾಗಿ
ಬ್ರಾಹ್ಮಣ-ರಾಗು-ವ-ರೆಂದೂ
ಬ್ರಾಹ್ಮಣ-ರಾದ
ಬ್ರಾಹ್ಮಣ-ರಿಗೆ
ಬ್ರಾಹ್ಮಣರು
ಬ್ರಾಹ್ಮಣರೆ
ಬ್ರಾಹ್ಮಣ-ರೆಂದು
ಬ್ರಾಹ್ಮಣರೇ
ಬ್ರಾಹ್ಮಣ-ರೊಂದಿಗೆ
ಬ್ರಾಹ್ಮಣ-ರೊಬ್ಬರೇ
ಬ್ರಾಹ್ಮಣರೋ
ಬ್ರಾಹ್ಮಣೇ-ತರ
ಬ್ರಾಹ್ಮಣೇ-ತರ-ರಿಗೆ
ಬ್ರಾಹ್ಮಣೋ
ಬ್ರಾಹ್ಮಣ್ಯ
ಬ್ರಿಟಿಷರ
ಬ್ರಿಟಿಷ್
ಭಂಗ
ಭಂಗ-ವಂತನ
ಭಂಗಿ
ಭಂಗಿ-ಸು-ತ್ತೀರಿ
ಭಂಡಾರ
ಭಂಡಾ-ರಕ್ಕೆ
ಭಂಡಾರ-ವನ್ನು
ಭಂಡಾರ-ವನ್ನೆಲ್ಲ
ಭಂಡಾರ-ವಾಗಿದೆ
ಭಕ್ತ
ಭಕ್ತನ
ಭಕ್ತ-ನಂತೆ
ಭಕ್ತ-ನಾದನು
ಭಕ್ತನು
ಭಕ್ತರ
ಭಕ್ತ-ರಂತೆ
ಭಕ್ತ-ರನ್ನು
ಭಕ್ತರು
ಭಕ್ತರೂ
ಭಕ್ತಿ
ಭಕ್ತಿ-ಗಳ
ಭಕ್ತಿ-ಗಳಿಂದ
ಭಕ್ತಿಗೆ
ಭಕ್ತಿ-ಪೂರ್ವ-ಕ-ವಾಗಿ
ಭಕ್ತಿ-ಮಾರ್ಗಕ್ಕೆ
ಭಕ್ತಿಯ
ಭಕ್ತಿ-ಯನ್ನು
ಭಕ್ತಿ-ಯಲ್ಲ
ಭಕ್ತಿ-ಯಲ್ಲಿ
ಭಕ್ತಿ-ಯಿಂದ
ಭಕ್ತಿಯು
ಭಕ್ತಿಯೂ
ಭಕ್ತಿಯೇ
ಭಕ್ತಿ-ಯೊಂದನ್ನು
ಭಕ್ತಿ-ಶಾಸ್ತ್ರ-ಗಳು
ಭಕ್ತಿ-ಶಾಸ್ತ್ರ-ವನ್ನು
ಭಕ್ಷ-ಕರು
ಭಕ್ಷಣ
ಭಕ್ಷಿ-ಸುವ
ಭಗ
ಭಗಂ
ಭಗ-ವಂತ
ಭಗ-ವಂತನ
ಭಗ-ವಂತ-ನನ್ನು
ಭಗ-ವಂತ-ನ-ನ್ನೇ
ಭಗ-ವಂತ-ನಲ್ಲಿ
ಭಗ-ವಂತ-ನ-ಲ್ಲಿಯೇ
ಭಗ-ವಂತ-ನ-ಲ್ಲಿ-ರುವ
ಭಗ-ವಂತ-ನಿಂದ
ಭಗ-ವಂತ-ನಿ-ಗಾಗಿ
ಭಗ-ವಂತ-ನಿಗೆ
ಭಗ-ವಂತ-ನಿಗೇ
ಭಗ-ವಂತನು
ಭಗ-ವಂತ-ನೆ-ಡೆಗೆ
ಭಗ-ವಂತನೇ
ಭಗ-ವ-ತ್
ಭಗ-ವತ್ಕೃಪೆ-ಯಿಂದ
ಭಗ-ವತ್ಪ್ರೇಮ
ಭಗ-ವತ್ಸಾಕ್ಷಾತ್ಕಾರ-ವಾಗು-ವುದು
ಭಗ-ವದವ-ತಾರ-ಗಳ
ಭಗ-ವದು-ಪಾಸ-ನೆಗೆ
ಭಗ-ವದ್ಗೀತೆ
ಭಗ-ವದ್ಗೀತೆಯ
ಭಗ-ವದ್ಗೀತೆ-ಯನ್ನು
ಭಗ-ವದ್ಗೀತೆ-ಯಲ್ಲಿ
ಭಗ-ವದ್ಗೀತೆಯು
ಭಗ-ವದ್ದರ್ಶನ
ಭಗ-ವದ್ಬೋಧಾ-ಯನ-ಕೃತಾಂ
ಭಗ-ವದ್ಭಾ-ವನೆ
ಭಗ-ವಾ-ನ್
ಭಗವೋ
ಭಗ್ನಾವ-ಶೇಷ-ಗಳಿಂದ
ಭಟ್ಟನು
ಭಣತಿ-ಗಳನ್ನು
ಭದ್ರ
ಭದ್ರ-ವಾಗಿ
ಭದ್ರ-ವಾಗಿದ್ದು
ಭನೆ-ಯನ್ನು
ಭಯ
ಭಯಂಕರ
ಭಯಂಕರ-ವಾಗಿದೆ
ಭಯಂಕರ-ವಾಗಿ-ರಲಿ
ಭಯಂಕರ-ವಾಗಿ-ರುವ
ಭಯಂಕರ-ವಾಗು-ವುದು
ಭಯಕ್ಕೆ
ಭಯದ
ಭಯ-ದಿಂದ
ಭಯ-ಪಡಬೇ-ಕಾದು-ದಿಲ್ಲ
ಭಯ-ಭ್ರಾಂತ-ನಾಗು-ವನು
ಭಯ-ವಿಲ್ಲ
ಭಯ-ವಿ-ಲ್ಲದೆ
ಭಯವು
ಭಯವೇ
ಭಯಾ-ತ್
ಭಯಾ-ನಕ
ಭಯಾ-ನಕ-ನಾಗಿ-ರ-ಬೇಕು
ಭಯಾ-ನಕ-ಮತ್ತು
ಭಯಾ-ನ್ವಿತಂ
ಭರತ
ಭರತ-ಖಂಡ
ಭರತ-ಖಂಡಕ್ಕೆ
ಭರತ-ಖಂಡದ
ಭರತ-ಖಂಡ-ದಲ್ಲಿ
ಭರತ-ಖಂಡ-ದಲ್ಲಿ-ತ್ತು
ಭರತ-ಖಂಡ-ದಲ್ಲಿ-ರಲಿ
ಭರತ-ಖಂಡ-ದಲ್ಲಿ-ರುವ
ಭರತ-ಖಂಡ-ದಲ್ಲಿವೆ
ಭರತ-ಖಂಡ-ದಲ್ಲೂ
ಭರತ-ಖಂಡ-ದಲ್ಲೆಲ್ಲ
ಭರತ-ಖಂಡ-ದಲ್ಲೆಲ್ಲಾ
ಭರತ-ಖಂಡ-ದಲ್ಲೇ
ಭರತ-ಖಂಡ-ದಿಂದ
ಭರತ-ಖಂಡ-ಪ್ರ-ಪಂಚ-ದಲ್ಲೆಲ್ಲಾ
ಭರತ-ಖಂಡ-ವನ್ನು
ಭರತ-ಖಂಡ-ವನ್ನೆಲ್ಲಾ
ಭರತ-ಖಂಡ-ವನ್ನೇ
ಭರತ-ಖಂಡವು
ಭರತ-ಖಂಡ-ವೆಲ್ಲಾ
ಭರತ-ಖಂಡವೇ
ಭರತ-ಭೂಮಿಯ
ಭರತ-ಭೂಮಿ-ಯಲ್ಲಿ
ಭರತ-ಮಾ-ತೆಯ
ಭರತ-ವರ್ಷದ
ಭರತ-ವರ್ಷವು
ಭರ-ದಿಂದ
ಭರವಸೆ
ಭರವಸೆ-ಯನ್ನು
ಭರವಸೆ-ಯನ್ನೂ
ಭರವಸೆ-ಯಿಂದ
ಭರವಸೆ-ಯಿಲ್ಲ
ಭರವಸೆ-ಯುಂಟು
ಭರವಸೆ-ಯೆಲ್ಲಾ
ಭರವಸೆಯೇ
ಭರಿತ-ವಾಗಿವೆ
ಭರಿಸ-ಲಾರ-ದಷ್ಟು
ಭರ್ತೃಹರಿ
ಭವ-ತಾದ್ಭಕ್ತಿ-ರಹೈ-ತುಕೀ
ಭವತಿ
ಭವ-ನ-ಗಳು
ಭವ-ನವು
ಭವ-ಸಾಗರದ
ಭವಸಿ
ಭವಿಷ್ಯ
ಭವಿಷ್ಯ-ತ್ತಿಗೆ
ಭವಿಷ್ಯದ
ಭವಿಷ್ಯ-ದತ್ತ
ಭವಿಷ್ಯ-ದಲ್ಲಿ
ಭವಿಷ್ಯ-ದಲ್ಲಿಯೂ
ಭವಿಷ್ಯ-ವನ್ನು
ಭವಿಷ್ಯ-ವಾಣಿ
ಭವಿಷ್ಯ-ವಿದೆ
ಭವಿಷ್ಯವು
ಭವಿಷ್ಯ-ವೆಲ್ಲ
ಭವ್ಯ
ಭವ್ಯ-ತಮ
ಭವ್ಯ-ತರ-ವಾದ
ಭವ್ಯ-ತಾ-ಪಿ-ಪಾಸೆ-ಯನ್ನೂ
ಭವ್ಯತೆ
ಭವ್ಯ-ತೆಯ
ಭವ್ಯ-ತೆ-ಯನ್ನು
ಭವ್ಯ-ತೆಯೇ
ಭವ್ಯ-ಪರಂಪರೆ
ಭವ್ಯ-ವಾಗಿ
ಭವ್ಯ-ವಾಗಿ-ದ್ದರೂ
ಭವ್ಯ-ವಾಗಿ-ರ-ಲಿಲ್ಲ
ಭವ್ಯ-ವಾಗಿ-ಲ್ಲವೊ
ಭವ್ಯ-ವಾಗಿವೆ
ಭವ್ಯ-ವಾಗು-ವುದು
ಭವ್ಯ-ವಾದ
ಭವ್ಯ-ವಾದರೂ
ಭವ್ಯ-ವಾದು-ದೆಂದು
ಭವ್ಯವೂ
ಭಾಂತ-ಮನು-ಭಾತಿ
ಭಾಂತಿ
ಭಾಗ
ಭಾಗಕ್ಕೂ
ಭಾಗಕ್ಕೆ
ಭಾಗ-ಗಳನ್ನು
ಭಾಗ-ಗಳನ್ನೂ
ಭಾಗ-ಗಳಲ್ಲಿ
ಭಾಗ-ಗಳಲ್ಲಿಯೂ
ಭಾಗ-ಗಳಿಂದ
ಭಾಗ-ಗಳಿ-ಗಿಂತ
ಭಾಗ-ಗಳಿಗೆ
ಭಾಗ-ಗಳು
ಭಾಗದ
ಭಾಗ-ದಲ್ಲಿ
ಭಾಗ-ದಲ್ಲಿಯೂ
ಭಾಗ-ದೊಂದಿಗೆ
ಭಾಗ-ವ-ತದ
ಭಾಗ-ವತ-ದಲ್ಲಿ
ಭಾಗ-ವತ-ಪುರಾಣ
ಭಾಗ-ವನ್ನು
ಭಾಗ-ವನ್ನೂ
ಭಾಗ-ವನ್ನೇ
ಭಾಗ-ವ-ಹಿಸಿ
ಭಾಗ-ವಾಗ-ಬೇಕಾಗಿದೆ
ಭಾಗ-ವಾಗಿದೆ
ಭಾಗ-ವಾಗಿವೆ
ಭಾಗ-ವಾದ
ಭಾಗ-ವಾದು-ದ-ರಿಂದ
ಭಾಗ-ವಿ-ರು-ವುದೇ
ಭಾಗವು
ಭಾಗಿ-ಗಳಾಗಿ-ದ್ದರು
ಭಾಗಿ-ಗಳಾಗಿ-ದ್ದೇವೆ
ಭಾಗಿ-ಗಳಾ-ಗುವ
ಭಾಗಿ-ಯಾಗಿ-ದ್ದರು
ಭಾಗಿ-ಯಾ-ದರು
ಭಾಗ್ಯ
ಭಾಗ್ಯಕ್ಕೆ
ಭಾಗ್ಯ-ವ-ತಿಯ-ರಾದ
ಭಾಜನ-ವಾಗಿತ್ತೊ
ಭಾತಿ
ಭಾತೃ-ಗಳೇ
ಭಾನ್ತಿ
ಭಾರ
ಭಾರತ
ಭಾರತಕ್ಕೆ
ಭಾರತ-ಖಂಡ-ದಲ್ಲಿ
ಭಾರತ-ಜನ-ನಿ-ಯಾದ
ಭಾರತದ
ಭಾರತ-ದಲ್ಲಿ
ಭಾರತ-ದಲ್ಲಿ-ದ್ದುದು
ಭಾರತ-ದಲ್ಲಿಯೂ
ಭಾರತ-ದಲ್ಲಿ-ರುವ
ಭಾರತ-ದಲ್ಲೆಲ್ಲ
ಭಾರತ-ದಿಂದ
ಭಾರತ-ದೆಲ್ಲಾ
ಭಾರತ-ದೇಶಕ್ಕೆ
ಭಾರತ-ದೇಶದ
ಭಾರತ-ಭೂಮಿ
ಭಾರತ-ಭೂಮಿಯ
ಭಾರತ-ಮಾತೆಗೆ
ಭಾರತ-ಮಾತೆಯ
ಭಾರತ-ಮಾತೆಯು
ಭಾರತ-ರಾಷ್ಟ್ರ
ಭಾರತ-ವನ್ನು
ಭಾರತ-ವನ್ನೆಲ್ಲಾ
ಭಾರತ-ವರ್ಷ
ಭಾರತ-ವರ್ಷಕ್ಕೆ
ಭಾರತ-ವರ್ಷದ
ಭಾರತ-ವಿನ್ನೂ
ಭಾರತವು
ಭಾರತವೇ
ಭಾರತಾದ್ಯಂತ
ಭಾರತಿಯರ
ಭಾರತೀಯ
ಭಾರತೀಯತೆ
ಭಾರತೀಯ-ನಿಗೆ
ಭಾರತೀಯನು
ಭಾರತೀಯನೂ
ಭಾರತೀಯರ
ಭಾರತೀಯ-ರನ್ನು
ಭಾರತೀಯ-ರಲ್ಲಿ
ಭಾರತೀಯ-ರಾದ
ಭಾರತೀಯ-ರಿಗೂ
ಭಾರತೀಯ-ರಿಗೆ
ಭಾರತೀಯರು
ಭಾರತೀಯ-ರೆಲ್ಲರೂ
ಭಾರತೀಯ-ವಾದ
ಭಾರ-ವನ್ನು
ಭಾರಸ್ಯ
ಭಾರ-ಹಾ-ಕುತ್ತಾನೆ
ಭಾರಿ
ಭಾರೀ
ಭಾವ
ಭಾವ-ಗತಿ-ಯಲ್ಲಿ
ಭಾವ-ಗಳ
ಭಾವ-ಗಳನ್ನು
ಭಾವ-ಗಳಾಗಿವೆ
ಭಾವ-ಗಳಿಗೆ
ಭಾವ-ಗಳು
ಭಾವ-ಗಾಂಭೀರ್ಯ-ದಿಂದ
ಭಾವ-ಜೀವಿ-ಯಾದ
ಭಾವದ
ಭಾವ-ದಲ್ಲಿ
ಭಾವ-ದ-ವರು
ಭಾವ-ದಿಂದ
ಭಾವ-ದಿಂದಲೂ
ಭಾವ-ದ್ದಲ್ಲ
ಭಾವನಾ
ಭಾವ-ನಾಡಿ
ಭಾವ-ನಾ-ಪರಂಪರೆ
ಭಾವನೆ
ಭಾವನೆ-ಗಳ
ಭಾವನೆ-ಗಳನ್ನು
ಭಾವನೆ-ಗಳನ್ನೂ
ಭಾವನೆ-ಗಳ-ನ್ನೆಲ್ಲಾ
ಭಾವನೆ-ಗಳಲ್ಲ
ಭಾವನೆ-ಗಳಲ್ಲಿ
ಭಾವನೆ-ಗಳ-ಲ್ಲಿಯೂ
ಭಾವನೆ-ಗಳಾ-ವುವು
ಭಾವನೆ-ಗಳಿಂದ
ಭಾವನೆ-ಗಳಿ-ಗಿಂತ
ಭಾವನೆ-ಗಳಿಗೆ
ಭಾವನೆ-ಗಳಿಗೆಲ್ಲ
ಭಾವನೆ-ಗಳಿ-ರು-ವುದು
ಭಾವನೆ-ಗಳಿವೆ
ಭಾವನೆ-ಗಳು
ಭಾವನೆ-ಗಳೂ
ಭಾವನೆ-ಗಳೆಂತಾ-ದರೂ
ಭಾವನೆ-ಗಳೆ-ರಡೂ
ಭಾವನೆ-ಗಳೆಲ್ಲ
ಭಾವನೆ-ಗಳೆ-ಲ್ಲವೂ
ಭಾವನೆ-ಗಳೆಲ್ಲಾ
ಭಾವನೆ-ಗಳೇ
ಭಾವನೆ-ಗಾಗಿ
ಭಾವನೆ-ಗಿಂತ
ಭಾವನೆ-ಗಿಂತಲೂ
ಭಾವನೆಗೆ
ಭಾವನೆ-ಪ್ರಚಲಿತ
ಭಾವನೆಯ
ಭಾವನೆ-ಯಂತೆ
ಭಾವನೆ-ಯನ್ನು
ಭಾವನೆ-ಯನ್ನೂ
ಭಾವನೆ-ಯ-ನ್ನೇ
ಭಾವನೆ-ಯಲ್ಲಿ
ಭಾವನೆ-ಯ-ವ-ರಾಗಿ
ಭಾವನೆ-ಯಿಂದ
ಭಾವನೆ-ಯಿತ್ತು
ಭಾವನೆ-ಯಿದು
ಭಾವನೆ-ಯಿ-ರುವ
ಭಾವನೆಯು
ಭಾವನೆ-ಯು-ಳ್ಳವ-ರಾಗಿ-ರು-ವೆವೋ
ಭಾವನೆಯೂ
ಭಾವನೆ-ಯೆಲ್ಲ
ಭಾವನೆಯೇ
ಭಾವನೆ-ಯೊಂದಕ್ಕೆ
ಭಾವನೆ-ಯೊಂದು
ಭಾವ-ಪೂರ್ಣ
ಭಾವ-ಪೂರ್ಣತೆ
ಭಾವ-ಪೂರ್ಣ-ವಾಗಿತ್ತೊ
ಭಾವ-ಪೂರ್ಣವೂ
ಭಾವ-ವನ್ನು
ಭಾವ-ವಾಗು-ವುದು
ಭಾವ-ವಿನಿ-ಮಯ-ದಿಂದ
ಭಾವ-ವಿರು-ವು-ದಿಲ್ಲವೋ
ಭಾವ-ವಿ-ಲ್ಲದೆ
ಭಾವವು
ಭಾವವೂ
ಭಾವ-ವೆಂದರೆ
ಭಾವ-ಸಮೂಹ
ಭಾವಿಸ
ಭಾವಿಸ-ದಿರಿ
ಭಾವಿ-ಸದೆ
ಭಾವಿಸ-ಬೇಕು
ಭಾವಿಸ-ಬೇಕೆಂದು
ಭಾವಿಸ-ಬೇಡಿ
ಭಾವಿಸ-ಲಾಗದಿದ್ದರೂ
ಭಾವಿಸಿ
ಭಾವಿಸಿ-ಕೊಂಡಿ-ರುವ
ಭಾವಿ-ಸಿದ
ಭಾವಿಸಿ-ದರು
ಭಾವಿಸಿ-ದರೂ
ಭಾವಿಸಿ-ದರೆ
ಭಾವಿಸಿ-ದುದು
ಭಾವಿಸಿ-ದೆವು
ಭಾವಿ-ಸಿದ್ದ
ಭಾವಿಸಿ-ದ್ದನು
ಭಾವಿಸಿ-ದ್ದರು
ಭಾವಿಸಿ-ದ್ದರೆ
ಭಾವಿಸಿ-ದ್ದರೊ
ಭಾವಿ-ಸಿದ್ದೆ
ಭಾವಿ-ಸಿಯೇ
ಭಾವಿಸಿ-ರ-ಬಹುದು
ಭಾವಿಸಿರಿ
ಭಾವಿಸಿ-ರುವಿ-ರೇನು
ಭಾವಿಸಿ-ರು-ವುದೇ-ನೆಂದರೆ
ಭಾವಿಸು
ಭಾವಿಸು-ತ್ತದೆ
ಭಾವಿಸು-ತ್ತಾನೆ
ಭಾವಿಸು-ತ್ತಾರೆ
ಭಾವಿಸು-ತ್ತೇನೆ
ಭಾವಿಸು-ತ್ತೇವೆ
ಭಾವಿಸುವ
ಭಾವಿಸು-ವಂತಾ-ಗಿದೆ
ಭಾವಿಸು-ವಂತೆ
ಭಾವಿಸು-ವನು
ಭಾವಿಸು-ವನೋ
ಭಾವಿಸು-ವರು
ಭಾವಿಸು-ವರೋ
ಭಾವಿಸು-ವ-ವರು
ಭಾವಿಸು-ವಿರೋ
ಭಾವಿಸು-ವು-ದಿಲ್ಲ
ಭಾವಿಸು-ವುದು
ಭಾವಿಸು-ವುದೇ
ಭಾವಿಸು-ವೆಯೋ
ಭಾವಿ-ಸೋಣ
ಭಾವೀ
ಭಾವೋ-ದ್ವೇಗ-ಗಳು
ಭಾವೋ-ನ್ಮತ್ತರಾಗು-ವರು
ಭಾಷಣ
ಭಾಷ-ಣಕ್ಕೆ
ಭಾಷಣ-ಗಳಲ್ಲಿ
ಭಾಷ-ಣದ
ಭಾಷಣ-ದಲ್ಲಿಯೆ
ಭಾಷಣ-ವನ್ನು
ಭಾಷಾ
ಭಾಷಾಂತರ
ಭಾಷಾಂತರ-ವನ್ನು
ಭಾಷಾಂತರ-ವಾಗಿದೆ
ಭಾಷಾಂತರಿ-ಸು-ವು-ದಿಲ್ಲ
ಭಾಷಾಂತರಿಸು-ವುದು
ಭಾಷಾ-ಪೆಟ್ಟಿಗೆ-ಯಲ್ಲಿ
ಭಾಷಾ-ಮೇಳವೇ
ಭಾಷಾ-ವಿ-ಜ್ಞಾನಿ-ಗಳಿಗೆ
ಭಾಷಾ-ಶಾಸ್ತ್ರ
ಭಾಷಾ-ಶಾಸ್ತ್ರ-ಜ್ಞರೇ
ಭಾಷಾ-ಶಾಸ್ತ್ರ-ದಲ್ಲಿ-ರುವ
ಭಾಷಿಕ
ಭಾಷೆ
ಭಾಷೆ-ಗಳಲ್ಲಿ
ಭಾಷೆ-ಗಳಿ-ಗಿಂತ
ಭಾಷೆ-ಗಿಂತ
ಭಾಷೆಗೆ
ಭಾಷೆಯ
ಭಾಷೆ-ಯನ್ನು
ಭಾಷೆ-ಯ-ನ್ನೆಲ್ಲಾ
ಭಾಷೆ-ಯ-ಲ್ಲಾ-ಗಲೀ
ಭಾಷೆ-ಯಲ್ಲಿ
ಭಾಷೆ-ಯ-ಲ್ಲಿ-ದ್ದರೆ
ಭಾಷೆ-ಯ-ಲ್ಲಿಯೂ
ಭಾಷೆ-ಯ-ಲ್ಲಿಯೇ
ಭಾಷೆ-ಯಾದ
ಭಾಷೆಯು
ಭಾಷೆಯೂ
ಭಾಷ್ಯ
ಭಾಷ್ಯ-ಕಾರ
ಭಾಷ್ಯ-ಕಾರನ
ಭಾಷ್ಯ-ಕಾರನು
ಭಾಷ್ಯ-ಕಾರರ
ಭಾಷ್ಯ-ಕಾರ-ರನ್ನು
ಭಾಷ್ಯ-ಕಾರ-ರಲ್ಲಿ
ಭಾಷ್ಯ-ಕಾರ-ರಾದ
ಭಾಷ್ಯ-ಕಾರರು
ಭಾಷ್ಯ-ಗಳನ್ನು
ಭಾಷ್ಯ-ಗಳಲ್ಲಿ
ಭಾಷ್ಯ-ಗಳಿ-ರ-ಬಹುದು
ಭಾಷ್ಯ-ದಂತೆ
ಭಾಷ್ಯ-ದಲ್ಲಿ
ಭಾಷ್ಯ-ದಿಂದ
ಭಾಷ್ಯ-ವನ್ನು
ಭಾಷ್ಯ-ವನ್ನೆಲ್ಲಾ
ಭಾಷ್ಯ-ವೆಂದರೆ
ಭಾಷ್ಯ-ವೆಂದು
ಭಾಷ್ಯಾ-ಕಾರರ
ಭಾಸ-ವಾಗು-ತ್ತದೆ
ಭಾಸಾ
ಭಾಸ್ಕರ-ವರ್ಮ
ಭಿಕಾರಿ-ಗಳಾಗಿ-ರು-ವಿರೋ
ಭಿಕಾರಿ-ಗಳು
ಭಿಕಾರಿ-ಯಾಗಿ
ಭಿಕಾರಿಯೂ
ಭಿಕ್ಷು
ಭಿಕ್ಷುಕ
ಭಿಕ್ಷು-ಕನ
ಭಿಕ್ಷು-ಕ-ನಾ-ಗಿದ್ದರೆ
ಭಿಕ್ಷು-ಕ-ನಿಗೂ
ಭಿಕ್ಷು-ಕ-ನಿಗೆ
ಭಿಕ್ಷು-ಕನು
ಭಿಕ್ಷು-ಕನೂ
ಭಿಕ್ಷು-ಕ-ರನ್ನು
ಭಿಕ್ಷು-ಕ-ರಾಗ-ದಿ-ರಲಿ
ಭಿಕ್ಷೆ
ಭಿತ್ತಿ-ಯನ್ನು
ಭಿನ್ನ
ಭಿನ್ನತೆ
ಭಿನ್ನ-ತೆ-ಗಳಿಗೂ
ಭಿನ್ನ-ತೆ-ಗಳಿ-ದ್ದರೂ
ಭಿನ್ನ-ತೆ-ಗಳಿವೆ
ಭಿನ್ನ-ತೆ-ಗಳೆಲ್ಲ
ಭಿನ್ನ-ತೆ-ಗಿಂತ
ಭಿನ್ನ-ತೆಗೆ
ಭಿನ್ನ-ತೆ-ಯನ್ನು
ಭಿನ್ನ-ತೆ-ಯಿಂದಲೇ
ಭಿನ್ನ-ತೆ-ಯಿಲ್ಲ
ಭಿನ್ನ-ತೆಯು
ಭಿನ್ನ-ಪಂಥ-ಗಳು
ಭಿನ್ನ-ಭಾವ-ಗಳಿಂದಲೂ
ಭಿನ್ನ-ಭಾವ-ವಿಲ್ಲ
ಭಿನ್ನ-ಭಿನ್ನ-ವಾಗಿದೆ
ಭಿನ್ನ-ಭಿನ್ನ-ವಾಗು-ವುದು
ಭಿನ್ನ-ಭಿಪ್ರಾಯ-ವಿದ್ದರೂ
ಭಿನ್ನ-ರೂಪ
ಭಿನ್ನ-ವತ್ತಳೆಗೆ
ಭಿನ್ನ-ವಲ್ಲ
ಭಿನ್ನ-ವಾಗಿ-ದ್ದರೂ
ಭಿನ್ನ-ವಾಗಿ-ರ-ಬೇಕಾಗಿ-ತ್ತು
ಭಿನ್ನಾ
ಭಿನ್ನಾ-ಭಿಪ್ರಾಯ
ಭಿನ್ನಾ-ಭಿಪ್ರಾಯ-ಗಳನ್ನು
ಭಿನ್ನಾ-ಭಿಪ್ರಾಯ-ಗಳಿಗೆ
ಭಿನ್ನಾ-ಭಿಪ್ರಾಯ-ಗಳಿ-ದ್ದರೂ
ಭಿನ್ನಾ-ಭಿಪ್ರಾಯ-ಗಳಿವೆ
ಭಿನ್ನಾ-ಭಿಪ್ರಾಯ-ಗಳು
ಭಿನ್ನಾ-ಭಿಪ್ರಾಯದ
ಭಿಪ್ರಾಯ-ಗಳಿವೆ
ಭಿಪ್ರಾಯ-ವಿರು-ತ್ತದೆ
ಭೀಕರ-ವಾಗಿದೆ
ಭೀತಿ
ಭೀತಿ-ಯಿಲ್ಲ
ಭೀಭತ್ಸ-ವಾಗಿ-ರಲಿ
ಭೀಮ
ಭೀಮ-ಭ-ವಾರ್ಣವಂ
ಭೀಮಾ-ಕಾರ-ವನ್ನು
ಭೀಷಣ
ಭುಜ
ಭುಜ-ತಟ್ಟಿ
ಭುಜದ
ಭುನಕ್ತು
ಭುವಿ
ಭೂಕೇಂದ್ರ
ಭೂಕೇಂದ್ರ-ವನ್ನು
ಭೂಕ್ಷೇತ್ರವೇ
ಭೂಗೋಳ
ಭೂತ
ಭೂತ-ಕಾಲ
ಭೂತ-ಕಾಲದ
ಭೂತ-ಗಳಲ್ಲೂ
ಭೂತ-ಗಳಿಗೂ
ಭೂತ-ಗಳು
ಭೂತ-ಪ್ರೇತ-ಗಳು
ಭೂತಿ-ಯಿಂದ
ಭೂತೇಷು
ಭೂಭಾ-ಗಕ್ಕೆ
ಭೂಮಂಡಲವನ್ನೆಲ್ಲಾ
ಭೂಮ-ಜ್ಞಾನವೂ
ಭೂಮದ
ಭೂಮ-ವನ್ನು
ಭೂಮಿ
ಭೂಮಿಕೆ
ಭೂಮಿ-ಕೆ-ಗಳು
ಭೂಮಿ-ಕೆ-ಯಲ್ಲಿ
ಭೂಮಿ-ಕೆ-ಯ-ಲ್ಲಿಯೂ
ಭೂಮಿ-ಕೆ-ಯೆಂದರೆ
ಭೂಮಿಗೆ
ಭೂಮಿಯ
ಭೂಮಿ-ಯನ್ನು
ಭೂಮಿ-ಯಲ್ಲಿ
ಭೂಮಿ-ಯ-ಲ್ಲಿಯೇ
ಭೂಮಿ-ಯ-ಲ್ಲಿ-ರುವ
ಭೂಮಿ-ಯ-ಲ್ಲಿ-ರು-ವಷ್ಟು
ಭೂಮಿ-ಯಾದ
ಭೂಮಿಯು
ಭೂಷಿತ-ರಾಗಿ
ಭೂಸ್ಪರ್ಶ
ಭೃತ್ಯ-ರಿಗೆ
ಭೇಟಿ
ಭೇಟಿ-ಕೊಟ್ಟಿ-ದ್ದೀರಿ
ಭೇಟಿ-ಕೊಡು-ವು-ದರ
ಭೇದ
ಭೇದ-ಕ-ನೆಂದು
ಭೇದ-ಗಳ
ಭೇದ-ಗಳನ್ನು
ಭೇದ-ಗಳಿವೆ
ಭೇದದ
ಭೇದ-ದಿಂದ
ಭೇದ-ಭಾವ-ಗಳ
ಭೇದ-ಭಾವ-ಗಳಿ-ದ್ದರೂ
ಭೇದ-ಭಾವ-ಗಳು
ಭೇದ-ಭಾವ-ದಿಂದಲೂ
ಭೇದ-ಭಾವ-ನೆ-ಗಳ-ನ್ನಲ್ಲ
ಭೇದ-ಭಾವ-ವಿರು-ವುದು
ಭೇದ-ವಿದ್ದರೆ
ಭೇದ-ವಿ-ರು-ವಂತೆ
ಭೇದ-ವಿ-ಲ್ಲದೆ
ಭೇದವು
ಭೇದ-ವೆ-ಲ್ಲಿರು-ತ್ತದೆ
ಭೇದವೇ
ಭೇದ-ವೇನು
ಭೇದಿಸ-ಲಾ-ರದು
ಭೇದಿಸಿ
ಭೇದಿ-ಸು-ವಂತೆ
ಭೇದಿ-ಸು-ವಷ್ಟು
ಭೋಕ್ತೃ
ಭೋಗ
ಭೋಗ-ಕ್ಕಲ್ಲ
ಭೋಗಕ್ಕೆ
ಭೋಗದ
ಭೋಗ-ದಿಂದ
ಭೋಗ-ಭೂಮಿ
ಭೋಗ-ವನ್ನು
ಭೋಗ-ವನ್ನೇ
ಭೋಗ-ವಾಗಲಿ
ಭೋಗವೇ
ಭೋಗಾಭಿಲಾಷೆ-ಯಾಗಿ
ಭೋಗಿ-ಸದೆ
ಭೋಗಿಸು-ತ್ತೀರಿ
ಭೋಗೇಚ್ಛೆ
ಭೋಜನ
ಭೌತ-ದ್ರವ್ಯದ
ಭೌತ-ವಸ್ತು
ಭೌತ-ವಸ್ತು-ವಿನ
ಭೌತ-ವಸ್ತು-ವಿ-ನಿಂದ
ಭೌತ-ವಸ್ತು-ವೆಂದು
ಭೌತಿಕ
ಭೌತಿಕ-ತೆ-ಯಲ್ಲಿ
ಭೌತಿಕ-ವಾದುವು-ಗಳು
ಭ್ಯಾಸ
ಭ್ಯಾ-ಸಕ್ಕೆ
ಭ್ರಮಣೆ-ಗೊಳಿಸುವ
ಭ್ರಮಿಸಿ-ರ-ಬಹುದು
ಭ್ರಮಿಸುತ್ತಾನೆ
ಭ್ರಮಿಸು-ತ್ತೀರಿ
ಭ್ರಮೆ
ಭ್ರಮೆ-ಯನ್ನುಂಟು-ಮಾಡು-ವುವು
ಭ್ರಮೆ-ಯಾಗಿ-ರು-ವುದು
ಭ್ರಮೆ-ಯಿಂದ
ಭ್ರಷ್ಟರು
ಭ್ರಾಂತರ-ನ್ನಾಗಿ
ಭ್ರಾಂತ-ರಾಗಿ
ಭ್ರಾಂತಿ
ಭ್ರಾಂತಿಗೆ
ಭ್ರಾಂತಿ-ಜನಿತ
ಭ್ರಾಂತಿ-ಮೂಲ-ವಾದುದು
ಭ್ರಾಂತಿ-ಯಿಂದ
ಭ್ರಾಂತಿ-ರೂಪೇಣ
ಭ್ರಾತೃ
ಭ್ರಾತೃ-ಗಳಲ್ಲಿ
ಭ್ರಾತೃ-ಗಳೇ
ಭ್ರಾತೃತ್ವ
ಭ್ರಾತೃ-ತ್ವದ
ಮಂಗ-ಮಾ-ಯ-ವಾಗಿದೆ
ಮಂಗಳ-ಕರ-ವಾದ
ಮಂಗಳ-ದಾಯ-ಕನು
ಮಂಗಳ-ದಾಯ-ಕ-ವಾದ
ಮಂಗಳ-ಮಯನು
ಮಂಗಳ-ಮಯನೂ
ಮಂಗಳ-ವನ್ನು
ಮಂಗಳ-ವಾಗು-ವು-ದೆಂದು
ಮಂಜುಗಡ್ಡೆ-ಗಳಿಗೆ
ಮಂಡಿಯೂರಿ
ಮಂಡಿಸಿರು-ವು-ದನ್ನು
ಮಂಡೂಕ-ದಂತೆ
ಮಂತವ್ಯೋ
ಮಂತ್ರ
ಮಂತ್ರ-ಆ-ಚಾರ-ಗಳು
ಮಂತ್ರ-ಗಳ
ಮಂತ್ರ-ಗಳನ್ನು
ಮಂತ್ರ-ಗಳ-ಲ್ಲಿ-ರುವ
ಮಂತ್ರ-ಗಳಿಗೆ
ಮಂತ್ರ-ಗಳಿವೆ
ಮಂತ್ರ-ಗಳು
ಮಂತ್ರದ
ಮಂತ್ರ-ದಲ್ಲಿ
ಮಂತ್ರ-ದಲ್ಲಿಯೂ
ಮಂತ್ರ-ದಿಂದ
ಮಂತ್ರ-ದೃಷ್ಟೃ
ಮಂತ್ರ-ದ್ರಷ್ಟ
ಮಂತ್ರ-ದ್ರಷ್ಟ-ನಾಗು-ವು-ದ-ರಿಂದ
ಮಂತ್ರ-ದ್ರಷ್ಟಾ-ರನ
ಮಂತ್ರ-ದ್ರಷ್ಟೃ-ಗಳಾಗ-ಬೇಕು
ಮಂತ್ರ-ದ್ರಷ್ಟೃ-ಗಳಾದ
ಮಂತ್ರ-ದ್ರಷ್ಟೃ-ಗಳಿಗೆ
ಮಂತ್ರ-ಭಾಗ-ವನ್ನು
ಮಂತ್ರ-ವನ್ನು
ಮಂತ್ರ-ವಾಗಿ
ಮಂತ್ರವು
ಮಂತ್ರಿಮಂಡಲ-ಗಳ
ಮಂದ-ಗತಿ-ಯಿಂದ
ಮಂದ-ಗೊಳಿಸು-ವುವು
ಮಂದ-ಪ್ರ-ಕಾಶ
ಮಂದ-ಬುದ್ಧಿ
ಮಂದ-ವಾಗಿ
ಮಂದ-ವಾಗಿಯೂ
ಮಂದ-ವಾಗಿ-ರು-ವು-ದರ
ಮಂದ-ಹಾ-ಸಕ್ಕೆ
ಮಂದಿ
ಮಂದಿಗೆ
ಮಂದಿರ
ಮಂದಿರ-ಗಳನ್ನು
ಮಂದಿರ-ಗಳು
ಮಕ್ಕಳ
ಮಕ್ಕಳಂತೆ
ಮಕ್ಕಳನ್ನು
ಮಕ್ಕಳಿ-ಗಾಗಿ
ಮಕ್ಕ-ಳಿಗೆ
ಮಕ್ಕಳಿಗೇಕೆ
ಮಕ್ಕಳು
ಮಕ್ಕಳೆ-ಲ್ಲರೂ
ಮಕ್ಕಳೇ
ಮಗ
ಮಗ-ದೊಮ್ಮೆ
ಮಗ-ನನ್ನು
ಮಗ-ನಾ-ದರೂ
ಮಗ-ನಿಗೆ
ಮಗ-ನಿರು-ವುದು
ಮಗು
ಮಗು-ವನ್ನು
ಮಗು-ವಾಗಿ
ಮಗು-ವಾಗಿ-ದ್ದಾಗ
ಮಗು-ವಾದ
ಮಗು-ವಿಗೂ
ಮಗು-ವಿಗೆ
ಮಗು-ವಿದ್ದರೆ
ಮಗು-ವಿನ
ಮಗು-ವಿ-ನಲ್ಲಿ
ಮಗುವು
ಮಗು-ವೊಂದು
ಮಟ್ಟಕ್ಕೆ
ಮಟ್ಟಿಗಾ-ದರೂ
ಮಟ್ಟಿಗೆ
ಮಟ್ಟಿನ
ಮಠ
ಮಠ-ಗಳಿಂದ
ಮಠದ
ಮಠ-ವನ್ನು
ಮಠವು
ಮಡಕೆ
ಮಡಕೆ-ಮಾಡು-ತ್ತಾನೆ
ಮಡಿ
ಮಡಿ-ದರೇ-ನಂತೆ
ಮಡಿ-ದಿರಿ
ಮಡಿ-ಯ-ಬೇಕು
ಮಡು
ಮಣಿಗಣಾ
ಮಣಿ-ಗಳು
ಮಣಿ-ಯಂತಿ-ರುವ
ಮಣಿ-ಯ-ಬೇಕಾ-ಯಿತು
ಮಣಿ-ಯ-ಬೇಕು
ಮಣಿ-ಯ-ಬೇಡಿ
ಮಣಿ-ಯುತ್ತಾನೋ
ಮಣಿಯು-ವುದು
ಮಣ್ಣಿಗೊ
ಮಣ್ಣಿನ
ಮಣ್ಣು
ಮಣ್ಣು-ಪಾ-ಲಾಗಿ
ಮತ
ಮತಕ್ಕೆ
ಮತ-ಗಳ
ಮತ-ಗಳನ್ನು
ಮತ-ಗಳಲ್ಲಿ
ಮತ-ಗಳಿಗೂ
ಮತ-ಗಳಿ-ರಬೇಕೇ
ಮತ-ಗಳು
ಮತ-ಗಳೆಲ್ಲ
ಮತ-ಚಿಹ್ನೆ-ಯನ್ನು
ಮತ-ತತ್ತ್ವ-ಗಳು
ಮತದ
ಮತ-ದಲ್ಲಿ
ಮತ-ದ-ವರೂ
ಮತ-ಪದ್ಧತಿಯೇ
ಮತ-ಭಾವ-ಗಳನ್ನು
ಮತ-ಭೇದ-ವೆಲ್ಲಾ
ಮತ-ಭ್ರಾಂತನ
ಮತ-ಭ್ರಾಂತ-ರಾದ
ಮತ-ಭ್ರಾಂತಿ
ಮತ-ಭ್ರಾಂತಿಯ
ಮತ-ಭ್ರಾಂತಿ-ಯನ್ನು
ಮತ-ವನ್ನು
ಮತ-ವಲ್ಲ
ಮತ-ವಾಗಲು
ಮತ-ವಿಲ್ಲದ
ಮತವೂ
ಮತ-ಸ್ಥಾಪಕನ
ಮತಾಂತರ-ಗೊಂಡ-ವರ
ಮತಾಂತರ-ಗೊಂಡ-ವರೇ
ಮತಾಂಧತೆ
ಮತಿ-ಗಳಲ್ಲಿ
ಮತೀಯ
ಮತೀಯತೆ
ಮತ್ತ-ಷ್ಟನ್ನು
ಮತ್ತಾರು
ಮತ್ತಾರೂ
ಮತ್ತಾವ
ಮತ್ತಾ-ವು-ದಕ್ಕೂ
ಮತ್ತಾವು-ದನ್ನೂ
ಮತ್ತಾವು-ದನ್ನೊ
ಮತ್ತಾವು-ದಾ-ದರೂ
ಮತ್ತಾ-ವುದೂ
ಮತ್ತಿ-ತರ
ಮತ್ತು
ಮತ್ತು-ವಿಕಾಸ-ವಾಗು-ವುದು
ಮತ್ತೂ
ಮತ್ತೆ
ಮತ್ತೆಲ್ಲಿ
ಮತ್ತೆ-ಲ್ಲಿ-ಯಾ-ದರೂ
ಮತ್ತೆ-ಲ್ಲಿಯೂ
ಮತ್ತೆಲ್ಲೋ
ಮತ್ತೆಷ್ಟು
ಮತ್ತೇ-ನನ್ನು
ಮತ್ತೇ-ನಿದೆ
ಮತ್ತೇನು
ಮತ್ತೇನೂ
ಮತ್ತೊಂದ-ಕ್ಕಿಂತ
ಮತ್ತೊಂದಕ್ಕೆ
ಮತ್ತೊಂದ-ನ್ನಾಗಿ
ಮತ್ತೊಂದನ್ನು
ಮತ್ತೊಂದರ
ಮತ್ತೊಂದ-ರದು
ಮತ್ತೊಂದಾ-ಗು-ವು-ದಕ್ಕೆ
ಮತ್ತೊಂದಿರು-ತ್ತದೆ
ಮತ್ತೊಂದಿಲ್ಲ
ಮತ್ತೊಂದು
ಮತ್ತೊಂದೇ
ಮತ್ತೊಬ್ಬ
ಮತ್ತೊಬ್ಬನ
ಮತ್ತೊಬ್ಬ-ನನ್ನು
ಮತ್ತೊಬ್ಬ-ನಿಗೂ
ಮತ್ತೊಬ್ಬ-ನಿಗೆ
ಮತ್ತೊಬ್ಬನು
ಮತ್ತೊಬ್ಬರ
ಮತ್ತೊಬ್ಬ-ರದು
ಮತ್ತೊಬ್ಬ-ರನ್ನು
ಮತ್ತೊಬ್ಬ-ರಿಂದ
ಮತ್ತೊಬ್ಬ-ರಿಗೆ
ಮತ್ತೊಬ್ಬ-ರಿಗೋ
ಮತ್ತೊಬ್ಬ-ರಿ-ರ-ಲಾ-ರರು
ಮತ್ತೊಬ್ಬ-ರಿಲ್ಲ
ಮತ್ತೊಬ್ಬ-ರಿಲ್ಲದೆ
ಮತ್ತೊಬ್ಬರು
ಮತ್ತೊಬ್ಬ-ರೊಂದಿಗೆ
ಮತ್ತೊಮ್ಮೆ
ಮತ್ಸ್ಯಾ-ವ-ತಾರದ
ಮದರಾಸಿನ
ಮದಾಲಸೆಯ
ಮದುವೆ
ಮದು-ವೆ-ಯಾಗದ
ಮದು-ವೆ-ಯಾಗ-ಬಹುದು
ಮದು-ವೆ-ಯಾಗು
ಮದು-ವೆ-ಯಾಗು-ವಂತಿಲ್ಲ
ಮದ್ಯ-ಪಾನ
ಮದ್ರಾಸನ್ನೇ
ಮದ್ರಾ-ಸಿಗೆ
ಮದ್ರಾಸಿನ
ಮದ್ರಾಸಿ-ನಲ್ಲಿ
ಮದ್ರಾಸಿನ-ಲ್ಲಿ-ದ್ದಾಗ
ಮದ್ರಾಸಿನ-ಲ್ಲಿ-ರು-ವುದು
ಮದ್ರಾಸಿನ-ವ-ರೆಗೆ
ಮದ್ರಾಸಿ-ನಿಂದ
ಮದ್ರಾಸು
ಮದ್ರಾ-ಸ್
ಮಧು-ಪಾನ-ದಿಂದ
ಮಧುರ
ಮಧುರ-ವಾದ
ಮಧುರೆಯ
ಮಧುರೆ-ಯ-ಲ್ಲಿ-ರುವ
ಮಧುರೆಯು
ಮಧ್ಯ
ಮಧ್ಯ-ದಲ್ಲಿ
ಮಧ್ಯ-ದಲ್ಲಿ-ದ್ದರೂ
ಮಧ್ಯ-ದಿಂದ
ಮಧ್ಯ-ಭಾಗ-ಇವು-ಗಳಲ್ಲಿ
ಮಧ್ಯ-ಮ-ವಾದು-ದೆಂದರೆ
ಮಧ್ಯ-ಸ್ಥ-ಗಾರ-ನೊ-ಬ್ಬ-ನನ್ನು
ಮಧ್ಯ-ಸ್ಥ-ವಾಗಿ
ಮಧ್ಯಾಹ್ನ
ಮಧ್ಯೆ
ಮಧ್ವಮುನಿ
ಮಧ್ವರಂತ
ಮಧ್ವಾ-ಚಾರ್ಯ
ಮಧ್ವಾ-ಚಾರ್ಯರ
ಮಧ್ವಾ-ಚಾರ್ಯರು
ಮನ
ಮನಃ
ಮನಃ-ಅ-ಲ್ಲಿಗೆ
ಮನಃ-ಕ್ಲೇಶ
ಮನಃ-ಕ್ಷೇತ್ರ
ಮನಃ-ಪೂರ್ವಕ
ಮನಃ-ಶಾಸ್ತ್ರ
ಮನಃ-ಶಾಸ್ತ್ರಕ್ಕೆ
ಮನಃ-ಶಾಸ್ತ್ರ-ಗಳಲ್ಲಿ
ಮನಃ-ಶಾಸ್ತ್ರದ
ಮನಃ-ಶ್ಯಾ-ಸ್ತ್ರದ
ಮನಃ-ಶ್ಶಾಸ್ತ್ರವು
ಮನ-ಕ್ಕೂಹಂಚುವ
ಮನ-ಗಂಡರು
ಮನ-ಗಂಡ-ವನು
ಮನ-ಗಂಡಿ-ರು-ವರು
ಮನ-ಗಾಣಿ-ಸಲು
ಮನ-ಗಾಣುತ್ತಿ-ರು-ವರು
ಮನ-ಗಾಣು-ವರು
ಮನ-ದಟ್ಟು
ಮನ-ದಲ್ಲಿ
ಮನ-ದಲ್ಲಿ-ದ್ದವು
ಮನ-ದಲ್ಲೇ
ಮನನ
ಮನ-ನ-ಮಾಡಿ
ಮನ-ಬಂದಂತೆ
ಮನ-ಮಧುರೆ
ಮನ-ಮಧುರೆಯ
ಮನ-ಮಧುರೆ-ಯಲ್ಲಿ
ಮನ-ವನ್ನು
ಮನ-ವ-ರಿಕೆ-ಯಾಗು-ವುದು
ಮನವು
ಮನ-ಶ್ಯಾ-ಸ್ತ್ರೀಯ
ಮನ-ಶ್ಶಾಸ್ತ್ರ
ಮನ-ಶ್ಶಾಸ್ತ್ರಕ್ಕೆ
ಮನ-ಶ್ಶಾಸ್ತ್ರ-ದಲ್ಲಿ
ಮನ-ಶ್ಶುದ್ಧಿ-ಯನ್ನು
ಮನಷ್ಯ
ಮನಸಾ
ಮನ-ಸ್
ಮನ-ಸ್ಕರು
ಮನ-ಸ್ತಾಪ
ಮನ-ಸ್ತಾಪ-ಗಳನ್ನು
ಮನ-ಸ್ತಾಪ-ಗಳು
ಮನ-ಸ್ಸನ್ನು
ಮನ-ಸ್ಸನ್ನೂ
ಮನ-ಸ್ಸಲ್ಲ
ಮನ-ಸ್ಸಾ-ಗು-ತ್ತದೆ
ಮನ-ಸ್ಸಾ-ದರೆ
ಮನ-ಸ್ಸಿ-ಗಿಂತ
ಮನ-ಸ್ಸಿಗಿ-ರುವ
ಮನ-ಸ್ಸಿಗೂ
ಮನ-ಸ್ಸಿಗೆ
ಮನ-ಸ್ಸಿದೆ
ಮನ-ಸ್ಸಿದ್ದಂತೆ
ಮನ-ಸ್ಸಿನ
ಮನ-ಸ್ಸಿನಲ್ಲಿ
ಮನ-ಸ್ಸಿನ-ಲ್ಲಿ-ಟ್ಟಿರ-ಬೇಕು
ಮನ-ಸ್ಸಿನ-ಲ್ಲಿ-ಡ-ಬೇಕೆಂದು
ಮನ-ಸ್ಸಿನ-ಲ್ಲಿಡಿ
ಮನ-ಸ್ಸಿನ-ಲ್ಲಿದ್ದ
ಮನ-ಸ್ಸಿನ-ಲ್ಲಿ-ದ್ದವು
ಮನ-ಸ್ಸಿನ-ಲ್ಲಿಯೇ
ಮನ-ಸ್ಸಿನ-ಲ್ಲಿ-ರುವ
ಮನ-ಸ್ಸಿ-ನಲ್ಲೂ
ಮನ-ಸ್ಸಿನ-ವ-ರಾಗಿ
ಮನ-ಸ್ಸಿ-ನಿಂದ
ಮನ-ಸ್ಸಿ-ನೊಂದಿಗೆ
ಮನಸ್ಸು
ಮನ-ಸ್ಸು-ಗಳ
ಮನ-ಸ್ಸು-ಗಳ-ಲ್ಲದ
ಮನ-ಸ್ಸು-ಗಳಿಂದ
ಮನ-ಸ್ಸು-ಗಳು
ಮನ-ಸ್ಸು-ಳ್ಳ-ವ-ರಿಗೂ
ಮನ-ಸ್ಸೆಂದು
ಮನಸ್ಸೇ
ಮನಾಂಸಿ
ಮನೀಷಿ-ಗಳ
ಮನು
ಮನುಜ
ಮನು-ವಿನ
ಮನುವು
ಮನುಷ್ಯ
ಮನು-ಷ್ಯತ್ವ
ಮನು-ಷ್ಯತ್ವಂ
ಮನು-ಷ್ಯನ
ಮನು-ಷ್ಯ-ನಂತೆ
ಮನು-ಷ್ಯ-ನನ್ನು
ಮನು-ಷ್ಯ-ನಲ್ಲಿ
ಮನು-ಷ್ಯ-ನ-ಲ್ಲಿ-ರುವ
ಮನು-ಷ್ಯ-ನ-ಲ್ಲಿ-ರು-ವಂತೆ
ಮನು-ಷ್ಯ-ನಾಗಿ-ದ್ದಾನೆ
ಮನು-ಷ್ಯ-ನಾದ
ಮನು-ಷ್ಯ-ನಿಗೂ
ಮನು-ಷ್ಯ-ನಿಗೆ
ಮನು-ಷ್ಯನು
ಮನು-ಷ್ಯ-ನೆಂದು
ಮನು-ಷ್ಯನೇ
ಮನು-ಷ್ಯರ
ಮನು-ಷ್ಯ-ರನ್ನು
ಮನು-ಷ್ಯ-ರಲ್ಲಿ
ಮನು-ಷ್ಯ-ರಾಗಿ-ದ್ದರು
ಮನು-ಷ್ಯ-ರಾಗಿ-ರು-ವಿರಿ
ಮನು-ಷ್ಯ-ರಾಗಿ-ರು-ವೆವೋ
ಮನು-ಷ್ಯ-ರಾದರೆ
ಮನು-ಷ್ಯ-ರಿಗೂ
ಮನು-ಷ್ಯರು
ಮನು-ಷ್ಯರೂ
ಮನು-ಷ್ಯರೆ
ಮನು-ಷ್ಯ-ರೆಂಬು-ದನ್ನೂ
ಮನು-ಷ್ಯರೇ
ಮನು-ಸಂತಾನ-ರಾದ
ಮನು-ಸ್ಮೃತಿ
ಮನುಸ್ಸು
ಮನೆ
ಮನೆ-ಗಳನ್ನು
ಮನೆ-ಗಳಲ್ಲಿ
ಮನೆಗೂ
ಮನೆಗೆ
ಮನೆಯ
ಮನೆ-ಯ-ನ್ನಾಗಿ
ಮನೆ-ಯನ್ನು
ಮನೆ-ಯಲ್ಲಿ
ಮನೆ-ಯ-ಲ್ಲಿದೆ
ಮನೆ-ಯ-ಲ್ಲಿಯೂ
ಮನೆ-ಯಿಂದ
ಮನೆ-ಯಿಂದಲೂ
ಮನೆಯೇ
ಮನೆ-ಯೊಳಗೆ
ಮನೋ
ಮನೋ-ಧರ್ಮಕ್ಕೂ
ಮನೋ-ಧರ್ಮ-ದ-ವರು
ಮನೋ-ಭಾವ
ಮನೋ-ಭಾವ-ದ-ವರು
ಮನೋ-ಭಾವ-ದಿಂದ
ಮನೋ-ಭಾವ-ದಿಂದಲೇ
ಮನೋ-ಭಾವ-ನೆಯ
ಮನೋ-ವಾ-ಕ್ಕಾಯ-ಗಳಿಂದ
ಮನೋ-ವಿ-ಜ್ಞಾನ
ಮನೋ-ಹರ
ಮನೋ-ಹರ-ವಾದ
ಮನ್ನಣೆ
ಮನ್ನಣೆ-ಯನ್ನು
ಮನ್ಯ-ಮಾನಾಃ
ಮನ್ಯೇ
ಮಮ
ಮಮ್ಮಿ
ಮಮ್ಮಿ-ಗಳ
ಮಯವೂ
ಮಯಿ
ಮರ-ಗಳಿಗೆ
ಮರಣ
ಮರ-ಣಕ್ಕೆ
ಮರಣ-ಗಳ
ಮರಣ-ಗಳಲ್ಲಿ
ಮರಣ-ಗಳಿಲ್ಲ
ಮರಣ-ಗಳು
ಮರಣ-ದಲ್ಲಿ
ಮರಣ-ಮಸ್ತು
ಮರಣ-ರ-ಹಿ-ತನು
ಮರಣ-ವಿಲ್ಲ
ಮರ-ಣವೂ
ಮರಣಾ-ತೀತ-ವಾಗಿ
ಮರಣಾ-ನಂತರ
ಮರದ
ಮರ-ದಲ್ಲಿ
ಮರ-ದಿಂದ
ಮರ-ಳನ್ನು
ಮರಳಿ
ಮರಳಿನ
ಮರಳಿ-ನಿಂದ
ಮರ-ವನ್ನು
ಮರ-ವಾಗಿ
ಮರ-ವಾಗು-ವು-ದಕ್ಕೆ
ಮರ-ವಾಗು-ವುದು
ಮರಿಗೋ
ಮರೀಚಿಕೆ-ಯಂತೆ
ಮರು-ಕ್ಷಣ
ಮರು-ಕ್ಷಣ-ದಲ್ಲಿ
ಮರು-ಕ್ಷಣ-ದಲ್ಲಿಯೇ
ಮರು-ಕ್ಷಣವೇ
ಮರು-ಗ-ಲಿಲ್ಲವೇ
ಮರು-ಗುತ್ತಿತ್ತೋ
ಮರು-ಗುವ
ಮರು-ಗು-ವೆನು
ಮರು-ದನಿ-ಯಾಗಿ
ಮರು-ದಿನ
ಮರು-ದಿ-ನವೇ
ಮರು-ಳಾ-ಗಿರು-ವು-ದ-ರಲ್ಲಿ
ಮರೆ-ತರು
ಮರೆತಿದ್ದ
ಮರೆತಿ-ದ್ದೇವೆ
ಮರೆತಿ-ರುವಿರಾ
ಮರೆತಿರು-ವೆವು
ಮರೆತು
ಮರೆ-ತು-ಬಿಟ್ಟಿ-ದ್ದೇವೆ
ಮರೆ-ತು-ಹೋಗಿ
ಮರೆತೋ
ಮರೆ-ಮಾಚು
ಮರೆ-ಮಾ-ಚು-ತ್ತಿ-ರು-ವರು
ಮರೆ-ಮಾ-ಚುವ
ಮರೆ-ಮಾಡಿ-ಕೊಳ್ಳಲೆಳ-ಸುವ
ಮರೆಯ-ಕೂಡದು
ಮರೆ-ಯದ
ಮರೆ-ಯ-ದಿರಿ
ಮರೆಯ-ಬಾ-ರದು
ಮರೆಯ-ಬೇಡಿ
ಮರೆ-ಯಲು
ಮರೆ-ಯಾಗಲಿ
ಮರೆ-ಯಾಗಿ-ರುವ
ಮರೆ-ಯಾಗಿ-ರು-ವನು
ಮರೆ-ಯಾಗು-ತ್ತಿದ್ದರು
ಮರೆ-ಯು-ತ್ತಿದ್ದರು
ಮರೆಯುತ್ತಿದ್ದೇ-ವೆಂದು
ಮರೆಯುತ್ತೇವೆ
ಮರೆ-ಯುವ
ಮರೆ-ಯು-ವಂತೆ
ಮರೆ-ಯು-ವನು
ಮರೆಯು-ವೆವು
ಮರ್ತ್ಯ
ಮರ್ಮ-ವನ್ನೂ
ಮಲಗಿ
ಮಲಗು-ವುದು
ಮಲ-ಬಾ-ರಿನ
ಮಲ-ಬಾ-ರಿ-ನಲ್ಲಿ
ಮಲಯ
ಮಲೆ-ಯಾ-ಳಿ-ಗಳೆಲ್ಲಾ
ಮಳೆ
ಮಳೆ-ಗರೆ-ಯಲಿ
ಮಳೆ-ಗರೆ-ಯಿರಿ
ಮಳೆ-ಗೆರೆ-ದಿ-ರುವ
ಮಳೆ-ಯನ್ನು
ಮವಾದ
ಮಸಣಕ್ಕೊ
ಮಸೀದಿ
ಮಸೀದಿಗೆ
ಮಸೀದಿ-ಯನ್ನು
ಮಸೀದಿ-ಯನ್ನೂ
ಮಸ್ತಕ-ಗಳಿ-ಲ್ಲದೆ
ಮಹ
ಮಹತೋ
ಮಹ-ತ್
ಮಹ-ತ್ಕಾರ್ಯ-ಕ್ರಮ-ದಲ್ಲಿ
ಮಹ-ತ್ಕಾರ್ಯ-ಗಳನ್ನು
ಮಹ-ತ್ಕಾರ್ಯ-ಗಳೂ
ಮಹ-ತ್ಕಾರ್ಯ-ದಲ್ಲಿ
ಮಹ-ತ್ಕಾರ್ಯ-ವನ್ನು
ಮಹ-ತ್ಕಾರ್ಯವು
ಮಹ-ತ್ಕಾರ್ಯವೇ
ಮಹ-ತ್ಕಾರ್ಯ-ಸಾಧ-ನೆಗೆ
ಮಹ-ತ್ತರ
ಮಹ-ತ್ತರ-ವಾಗು-ತ್ತದೆ
ಮಹ-ತ್ತರ-ವಾದ
ಮಹ-ತ್ತರ-ವಾದವು
ಮಹ-ತ್ತರ-ವಾದು-ದ-ನ್ನಾಗಿ
ಮಹ-ತ್ತರ-ವಾದುದು
ಮಹ-ತ್ತಾಗಿ-ತ್ತೆಂಬು-ದನ್ನು
ಮಹ-ತ್ತಾಗುತ್ತಿರು-ವುವು
ಮಹ-ತ್ತಾದ
ಮಹ-ತ್ತಾ-ದು-ದನ್ನು
ಮಹ-ತ್ತಾ-ದುದು
ಮಹ-ತ್ತಾ-ದುವು
ಮಹ-ತ್ತಿನ
ಮಹತ್ತು
ಮಹ-ತ್ನಲ್ಲಿ
ಮಹ-ತ್ವಕ್ಕೆ
ಮಹ-ತ್ವದ
ಮಹ-ತ್ವ-ವನ್ನು
ಮಹ-ತ್ವ-ವಾಗಲಿ
ಮಹ-ತ್ವ-ವಾಗಿ-ರಬಹದು
ಮಹ-ತ್ವಾ-ಕಾಂಕ್ಷೆ
ಮಹ-ದಾ-ನಂದ
ಮಹ-ದಾಶ್ಚರ್ಯ-ವೇನು
ಮಹ-ದುಪ-ಕಾರ
ಮಹ-ದುಪ-ಕಾರ-ಮಾಡಿ-ಕೊಂಡಂತೆ
ಮಹ-ದ್ಭಯಂ
ಮಹ-ನೀ-ಯರ
ಮಹ-ನೀಯ-ರಿಗೆ
ಮಹ-ನೀ-ಯರು
ಮಹ-ನೀ-ಯರೆ
ಮಹ-ನೀ-ಯರೆ-ಲ್ಲರೂ
ಮಹ-ನೀಯರೇ
ಮಹ-ಮದೀಯ-ರನ್ನು
ಮಹ-ಮದೀ-ಯ-ರೊಂದಿಗೆ
ಮಹ-ಮ್ಮದ
ಮಹ-ಮ್ಮದನ
ಮಹ-ಮ್ಮದ-ನನ್ನು
ಮಹ-ಮ್ಮದನು
ಮಹ-ಮ್ಮದೀಯ
ಮಹ-ಮ್ಮದೀಯರ
ಮಹ-ಮ್ಮದೀಯ-ರನ್ನು
ಮಹ-ಮ್ಮದೀಯ-ರಲ್ಲಿ
ಮಹ-ಮ್ಮದೀಯ-ರಾಗಿ
ಮಹ-ಮ್ಮದೀಯ-ರಾಗಿ-ರು-ವರು
ಮಹ-ಮ್ಮದೀಯ-ರಾದರು
ಮಹ-ಮ್ಮದೀಯ-ರಿಗೆ
ಮಹ-ಮ್ಮದೀಯರು
ಮಹ-ಮ್ಮದೀಯ-ರೋಡನೆ
ಮಹರ್ಷಿ
ಮಹರ್ಷಿ-ಗಳನ್ನು
ಮಹರ್ಷಿ-ಗಳು
ಮಹರ್ಷಿ-ಗಳೂ
ಮಹರ್ಷಿ-ಗಳೆ-ಲ್ಲರೂ
ಮಹರ್ಷಿ-ಗಳೇನೋ
ಮಹರ್ಷಿ-ಯ-ವರೆಗೂ
ಮಹರ್ಷಿ-ಯ-ವರೆ-ವಿಗೂ
ಮಹರ್ಷಿ-ಯಾಗಲೀ
ಮಹರ್ಷಿಯು
ಮಹಾ
ಮಹಾ-ಋಷಿ-ಗಳ
ಮಹಾ-ಋಷಿ-ಗಳಾಗಿ-ದ್ದರು
ಮಹಾ-ಋಷಿ-ಗಳು
ಮಹಾ-ಕಲ್ಯಾ-ಣ-ಪ್ರದ-ರಾಗು-ವಿರಿ
ಮಹಾ-ಕವಿಯು
ಮಹಾ-ಕವಿಯೂ
ಮಹಾ-ಕಾರ್ಯ
ಮಹಾ-ಕಾರ್ಯ-ಗಳನ್ನು
ಮಹಾ-ಕಾರ್ಯ-ದಲ್ಲಿ
ಮಹಾ-ಕಾಳಿ
ಮಹಾ-ಕಾಳಿಯು
ಮಹಾ-ಗಣಿ
ಮಹಾ-ಗಣಿ-ಯಿಂದ
ಮಹಾ-ಗಾನ-ದಲ್ಲಿ
ಮಹಾ-ಚರಿತ್ರೆ
ಮಹಾ-ಚುನಾವಣೆ
ಮಹಾ-ಜೀವ-ನ-ದಲ್ಲಿ
ಮಹಾ-ತತ್ತ್ವ
ಮಹಾ-ತತ್ತ್ವ-ಗಳನ್ನು
ಮಹಾ-ತತ್ತ್ವ-ಗಳು
ಮಹಾ-ತತ್ತ್ವದ
ಮಹಾ-ತತ್ತ್ವ-ವನ್ನು
ಮಹಾ-ತತ್ವ-ಗಳೆ-ಲ್ಲ-ವನ್ನೂ
ಮಹಾ-ತ-ರಂಗ-ವೊಂದು
ಮಹಾತ್ಮ
ಮಹಾ-ತ್ಮ-ನನ್ನೇ
ಮಹಾ-ತ್ಮ-ನ-ಲ್ಲಿಯೂ
ಮಹಾ-ತ್ಮನೇ
ಮಹಾ-ತ್ಮರ
ಮಹಾ-ತ್ಮ-ರನ್ನು
ಮಹಾ-ತ್ಮ-ರನ್ನೂ
ಮಹಾ-ತ್ಮ-ರಲ್ಲಿ
ಮಹಾ-ತ್ಮ-ರಾದ
ಮಹಾ-ತ್ಮ-ರಿಗೆ
ಮಹಾ-ತ್ಮರು
ಮಹಾತ್ಮೆ
ಮಹಾ-ತ್ಮೆ-ಯನ್ನು
ಮಹಾ-ತ್ಮ್ಯೆಗೆ
ಮಹಾ-ತ್ಮ್ಯೆಯ
ಮಹಾ-ದುಃಖವು
ಮಹಾ-ದೋಷ
ಮಹಾ-ಧ್ಯೇ-ಯಕ್ಕೆ
ಮಹಾ-ಧ್ಯೇಯ-ಗಳನ್ನೂ
ಮಹಾ-ಧ್ಯೇಯ-ವನ್ನು
ಮಹಾ-ನ-ದಿಯ
ಮಹಾ-ನು-ಭಾವ
ಮಹಾ-ನು-ಭಾವ-ರನ್ನು
ಮಹಾ-ನು-ಭಾ-ವರು
ಮಹಾನ್
ಮಹಾ-ಪರಾಧ
ಮಹಾ-ಪಾತಕ
ಮಹಾ-ಪಾತ-ಕ-ವಿದ್ದರೆ
ಮಹಾ-ಪಾಪ
ಮಹಾ-ಪಾಪ-ವೆಂದು
ಮಹಾ-ಪುರು-ಷನ
ಮಹಾ-ಪುರು-ಷನೂ
ಮಹಾ-ಪುರು-ಷರ
ಮಹಾ-ಪುರು-ಷ-ರನ್ನು
ಮಹಾ-ಪುರು-ಷ-ರ-ಪೂರ್ಣ
ಮಹಾ-ಪುರು-ಷ-ರಾಗ-ಬೇಕು
ಮಹಾ-ಪುರು-ಷ-ರಾಗು-ವಿರಿ
ಮಹಾ-ಪುರು-ಷ-ರಾದ
ಮಹಾ-ಪುರು-ಷ-ರಿ-ಗಾಗಿ
ಮಹಾ-ಪುರು-ಷರು
ಮಹಾ-ಪುರು-ಷರೂ
ಮಹಾ-ಪುರು-ಷ-ರೆಂದು
ಮಹಾ-ಪುರು-ಷ-ರೊಬ್ಬರು
ಮಹಾ-ಪುರು-ಷ-ಸಂಶ್ರಯಃ
ಮಹಾ-ಪ್ರತಿ-ಜ್ಞೆ-ಗಳನ್ನೂ
ಮಹಾ-ಪ್ರಭು
ಮಹಾ-ಪ್ರ-ವಾ-ಹದ
ಮಹಾ-ಪ್ರ-ವಾ-ಹ-ದಿಂದ
ಮಹಾ-ಭಯ-ದಿಂದ
ಮಹಾ-ಭಾಗ್ಯವೇ
ಮಹಾ-ಭಾರತ-ದಲ್ಲಿ
ಮಹಾ-ಭಾ-ವನೆ
ಮಹಾ-ಭಾವ-ನೆ-ಯನ್ನು
ಮಹಾ-ಭಾಷ್ಯ
ಮಹಾ-ಭಾಷ್ಯ-ವೆಂಬ
ಮಹಾ-ಮತಿ-ಗಳಿಗೆ
ಮಹಾ-ಮ-ಹಿಮ
ಮಹಾ-ಮ-ಹಿ-ಮ-ರಾಗಿ-ದ್ದರು
ಮಹಾ-ಮ-ಹಿ-ಮರು
ಮಹಾ-ಮ-ಹಿ-ಮರೇ
ಮಹಾ-ಮುನಿ-ಗಳ
ಮಹಾ-ಮುನಿ-ಗಳು
ಮಹಾ-ಮೇಧಾವಿ-ಗಳಾಗಿಯೇ
ಮಹಾ-ಯೋ-ಧ-ನಿಗೂ
ಮಹಾ-ರಹಸ್ಯ
ಮಹಾ-ರಾಜರ
ಮಹಾ-ರಾಜರು
ಮಹಾ-ರಾಜರೇ
ಮಹಾ-ರಾಷ್ಟ್ರ
ಮಹಾ-ಲೋಪ-ವಿದೆ
ಮಹಾ-ವಿಶ್ವ-ವೆ-ಲ್ಲವೂ
ಮಹಾ-ವೀರ
ಮಹಾ-ವೀರ-ನಾದ
ಮಹಾ-ವ್ಯಕ್ತಿ
ಮಹಾ-ವ್ಯಕ್ತಿ-ಗಳನ್ನು
ಮಹಾ-ವ್ಯಕ್ತಿ-ಗಳಾಗು-ವೆವು
ಮಹಾ-ವ್ಯಕ್ತಿ-ಗಳಿಗೆ
ಮಹಾ-ವ್ಯಕ್ತಿ-ಗಳಿ-ರು-ವರು
ಮಹಾ-ವ್ಯಕ್ತಿ-ಗಳು
ಮಹಾ-ವ್ಯಕ್ತಿಗೆ
ಮಹಾ-ಶಕ್ತಿ
ಮಹಾ-ಶಕ್ತಿ-ಮಾ-ನ್
ಮಹಾ-ಶಕ್ತಿ-ಯನ್ನು
ಮಹಾ-ಶಕ್ತಿ-ಯಿಂದ
ಮಹಾ-ಶಾಪ-ಗಳಿವೆ
ಮಹಾ-ಶ್ಲೋಕವು
ಮಹಾ-ಸಂಕೇ-ತದ
ಮಹಾ-ಸಂದೇಶ-ವನ್ನು
ಮಹಾ-ಸಂಪತ್ತು
ಮಹಾ-ಸತ್ಯ
ಮಹಾ-ಸತ್ಯ-ಕ್ಕಾಗಿ
ಮಹಾ-ಸತ್ಯ-ಗಳನ್ನು
ಮಹಾ-ಸತ್ಯದ
ಮಹಾ-ಸತ್ಯ-ವನ್ನು
ಮಹಾ-ಸತ್ಯವೇ
ಮಹಾ-ಸ-ಭೆಗೆ
ಮಹಾ-ಸಭೆ-ಯಲ್ಲಿ
ಮಹಾ-ಸಮ-ನ್ವಯ-ದಂತೆ
ಮಹಾ-ಸ-ಮಾಜ
ಮಹಾ-ಸಮುದ್ರ
ಮಹಾ-ಸ-ಹಾ-ಯ-ವನ್ನು
ಮಹಾ-ಸಾಗರ
ಮಹಾ-ಸಾಗರದ
ಮಹಾ-ಸಾಗರ-ದಲ್ಲಿ
ಮಹಾ-ಸಾಗರ-ವಿದೆ
ಮಹಾ-ಸಿದ್ಧಾಂತ
ಮಹಾ-ಸಿದ್ಧಾಂತ-ವನ್ನು
ಮಹಾ-ಸಿದ್ಧಾಂತವೇ
ಮಹಾ-ಸ್ವಾ-ಮಿ-ಯ-ವರು
ಮಹಿ-ತ್ವಾ
ಮಹಿ-ಮ-ನಾದ
ಮಹಿಮಾ
ಮಹಿ-ಮಾ-ನ-ಮಿತಿ
ಮಹಿ-ಮಾ-ನ್ವಿತ-ವಾಗಿ
ಮಹಿ-ಮಾ-ಮಯ-ರಾ-ಗಿಯೇ
ಮಹಿ-ಮಾ-ಮಯಿ
ಮಹಿ-ಮಾ-ವಂತ
ಮಹಿ-ಮಾ-ವಂತ-ರಾದ
ಮಹಿ-ಮಾ-ಸಂಪನ್ನ
ಮಹಿಮೆ
ಮಹಿ-ಮೆಯ
ಮಹಿ-ಮೆ-ಯನ್ನು
ಮಹಿ-ಮೆ-ಯಲ್ಲಿ
ಮಹಿ-ಮೆ-ಯಿಂದ
ಮಹಿ-ಮೆ-ಯಿಂದಲೇ
ಮಹಿ-ಮೆ-ಯುಳ್ಳವು
ಮಹಿಳೆ
ಮಹಿ-ಳೆಯ
ಮಹಿ-ಳೆ-ಯ-ರನ್ನು
ಮಹಿ-ಳೆ-ಯ-ರಿಗೆ
ಮಹಿ-ಳೆ-ಯರು
ಮಹಿ-ಳೆ-ಯರೆ
ಮಹಿ-ಳೆ-ಯಾಗಿ-ರ-ಬಹುದು
ಮಹಿ-ಳೆಯು
ಮಹೇ-ಶ್ವ-ರನ
ಮಹೇ-ಶ್ವರ-ನಿ-ದ್ದಾನೆ
ಮಹೇ-ಶ್ವ-ರನು
ಮಹೋಚ್ಚ
ಮಹೋ-ತ್ತುಂಗ
ಮಹೋ-ದ್ದೇಶ-ಕ್ಕಾಗಿ
ಮಹೋ-ದ್ದೇಶ-ಗಳನ್ನೂ
ಮಹೋ-ನ್ನತ
ಮಹೋ-ನ್ನತ-ವಾದ
ಮಾ
ಮಾಂತ್ರಿಕ
ಮಾಂಸ
ಮಾಂಸ-ಖಂಡ-ಗಳು
ಮಾಂಸ-ಖಂಡದ
ಮಾಂಸ-ಖಂಡ-ದಲ್ಲಿ
ಮಾಂಸ-ಗಳಿಗೆ
ಮಾಂಸ-ವನ್ನು
ಮಾಕ್ಸಿಮ-ಸ್
ಮಾಗಿದ
ಮಾಗು-ವುದು
ಮಾಡ
ಮಾಡ-ಕೂಡದು
ಮಾಡದ
ಮಾಡ-ದಿ-ದ್ದರೆ
ಮಾಡ-ದಿರು-ವುದು
ಮಾಡದೆ
ಮಾಡದೇ
ಮಾಡ-ಬಲ್ಲನೆ
ಮಾಡ-ಬಲ್ಲರು
ಮಾಡ-ಬಲ್ಲರೋ
ಮಾಡ-ಬಲ್ಲವು
ಮಾಡ-ಬಲ್ಲಿರಿ
ಮಾಡ-ಬಲ್ಲೆ
ಮಾಡ-ಬಲ್ಲೆ-ನೆಂದು
ಮಾಡ-ಬಲ್ಲೆವು
ಮಾಡ-ಬಹು-ದಾದ
ಮಾಡ-ಬಹುದು
ಮಾಡ-ಬಾ-ರದು
ಮಾಡ-ಬೇಕಾಗಿ-ತ್ತು
ಮಾಡ-ಬೇಕಾಗಿದೆ
ಮಾಡ-ಬೇಕಾಗಿ-ದೆಯೋ
ಮಾಡ-ಬೇಕಾಗಿ-ರುವ
ಮಾಡ-ಬೇಕಾಗಿ-ರು-ವುದು
ಮಾಡ-ಬೇಕಾಗಿಲ್ಲ
ಮಾಡ-ಬೇ-ಕಾದ
ಮಾಡ-ಬೇ-ಕಾ-ದದ್ದು
ಮಾಡ-ಬೇ-ಕಾದರೂ
ಮಾಡ-ಬೇ-ಕಾದರೆ
ಮಾಡ-ಬೇ-ಕಾದು-ದನ್ನೆಲ್ಲಾ
ಮಾಡ-ಬೇ-ಕಾದು-ದೇನು
ಮಾಡ-ಬೇಕಾ-ಯಿತು
ಮಾಡ-ಬೇಕಿಲ್ಲ
ಮಾಡ-ಬೇಕು
ಮಾಡ-ಬೇಕೆಂದಿ-ದ್ದರೆ
ಮಾಡ-ಬೇಕೆಂದಿರುವೆ
ಮಾಡ-ಬೇಕೆಂದಿರು-ವೆನು
ಮಾಡ-ಬೇಕೆಂದು
ಮಾಡ-ಬೇಕೆಂಬ
ಮಾಡ-ಬೇಕೆಂಬು-ದನ್ನು
ಮಾಡ-ಬೇಕೊ
ಮಾಡ-ಬೇಡಿ
ಮಾಡ-ಲಾಗದು
ಮಾಡ-ಲಾ-ಗ-ಲಿಲ್ಲ
ಮಾಡ-ಲಾ-ಗು-ವು-ದಿಲ್ಲ
ಮಾಡ-ಲಾ-ಯಿತು
ಮಾಡ-ಲಾರ-ದ-ವರು
ಮಾಡ-ಲಾ-ರದು
ಮಾಡ-ಲಾರಬೆಂಬುದು
ಮಾಡ-ಲಾ-ರರು
ಮಾಡ-ಲಾರಿರಿ
ಮಾಡ-ಲಾರೆ
ಮಾಡ-ಲಾರೆವು
ಮಾಡಲಿ
ಮಾಡ-ಲಿ-ಚ್ಛಿ-ಸು-ವು-ದಿಲ್ಲ
ಮಾಡ-ಲಿ-ಚ್ಛಿಸು-ವೆನು
ಮಾಡ-ಲಿ-ರುವ
ಮಾಡ-ಲಿಲ್ಲ
ಮಾಡ-ಲಿ-ಲ್ಲವೋ
ಮಾಡಲು
ಮಾಡ-ಲೆ-ತ್ನಿ-ಸು-ವರು
ಮಾಡ-ಲೇ-ಬೇಕು
ಮಾಡಿ
ಮಾಡಿ-ಕೊಂಡ
ಮಾಡಿ-ಕೊಂಡನು
ಮಾಡಿ-ಕೊಂಡರು
ಮಾಡಿ-ಕೊಂಡಿ-ದ್ದೀರಿ
ಮಾಡಿ-ಕೊಂಡಿ-ರುವ
ಮಾಡಿ-ಕೊಂಡಿ-ರು-ವರು
ಮಾಡಿ-ಕೊಂಡಿ-ರು-ವರೊ
ಮಾಡಿ-ಕೊಂಡಿ-ರು-ವರೋ
ಮಾಡಿ-ಕೊಂಡಿ-ರು-ವೆವು
ಮಾಡಿ-ಕೊಂಡು
ಮಾಡಿ-ಕೊ-ಡಲು
ಮಾಡಿ-ಕೊ-ಡುವ
ಮಾಡಿ-ಕೊಳ್ಳದೆ
ಮಾಡಿ-ಕೊಳ್ಳದೇ
ಮಾಡಿ-ಕೊಳ್ಳ-ಬಯ-ಸುವ-ವರು
ಮಾಡಿ-ಕೊಳ್ಳ-ಬೇಕು
ಮಾಡಿ-ಕೊಳ್ಳ-ಬೇಕು-ಎಂದು
ಮಾಡಿ-ಕೊಳ್ಳ-ಬೇಕೆ
ಮಾಡಿ-ಕೊಳ್ಳ-ಬೇಕೆ-ನ್ನು-ವರು
ಮಾಡಿ-ಕೊಳ್ಳಲಾರ
ಮಾಡಿ-ಕೊಳ್ಳಲು
ಮಾಡಿ-ಕೊಳ್ಳಿ
ಮಾಡಿ-ಕೊಳ್ಳು-ತ್ತಾರೆ
ಮಾಡಿ-ಕೊಳ್ಳು-ತ್ತೀರಿ
ಮಾಡಿ-ಕೊಳ್ಳು-ತ್ತೇವೆ
ಮಾಡಿ-ಕೊಳ್ಳು-ವಂತೆ
ಮಾಡಿ-ಕೊಳ್ಳು-ವರು
ಮಾಡಿ-ಕೊಳ್ಳು-ವ-ವರಗೆ
ಮಾಡಿ-ಕೊಳ್ಳು-ವ-ವರೆಗೂ
ಮಾಡಿ-ಕೊಳ್ಳು-ವ-ವ-ರೆಗೆ
ಮಾಡಿ-ಕೊಳ್ಳು-ವು-ದರ
ಮಾಡಿ-ಕೊಳ್ಳು-ವುದು
ಮಾಡಿ-ಕೊಳ್ಳು-ವುದೇ
ಮಾಡಿ-ಕೊಳ್ಳು-ವುದೇಕೆ
ಮಾಡಿ-ಕೊಳ್ಳೋಣ
ಮಾಡಿತು
ಮಾಡಿ-ತೆಂದು
ಮಾಡಿದ
ಮಾಡಿ-ದಂತೆ
ಮಾಡಿ-ದನು
ಮಾಡಿ-ದ-ರಂತೆ
ಮಾಡಿ-ದರು
ಮಾಡಿ-ದರೂ
ಮಾಡಿ-ದರೆ
ಮಾಡಿ-ದ-ರೆಂದು
ಮಾಡಿ-ದ-ರೆಂಬು-ದನ್ನು
ಮಾಡಿ-ದ-ರೇನೇ
ಮಾಡಿ-ದ-ವನಿ-ಗಿಂತ
ಮಾಡಿ-ದ-ವನು
ಮಾಡಿ-ದ-ವನೂ
ಮಾಡಿ-ದ-ವನೇ
ಮಾಡಿ-ದ-ವ-ರಿಗೆ
ಮಾಡಿ-ದ-ವರು
ಮಾಡಿ-ದವು
ಮಾಡಿ-ದಷ್ಟೂ
ಮಾಡಿ-ದಾಗ
ಮಾಡಿ-ದುದು
ಮಾಡಿದೆ
ಮಾಡಿ-ದೆವು
ಮಾಡಿ-ದೆವೋ
ಮಾಡಿ-ದೊ-ಡ-ನೆಯೇ
ಮಾಡಿದ್ದ
ಮಾಡಿ-ದ್ದ-ಕ್ಕಾಗಿ
ಮಾಡಿ-ದ್ದನು
ಮಾಡಿ-ದ್ದರೂ
ಮಾಡಿ-ದ್ದರೆ
ಮಾಡಿ-ದ್ದರೋ
ಮಾಡಿ-ದ್ದವು
ಮಾಡಿ-ದ್ದಾರೆ
ಮಾಡಿ-ದ್ದೀರಿ
ಮಾಡಿದ್ದು
ಮಾಡಿ-ದ್ದುವು
ಮಾಡಿ-ಬಿಟ್ಟರೆ
ಮಾಡಿ-ಬಿಡ-ಬಹುದು
ಮಾಡಿ-ರ-ಬಾ-ರದು
ಮಾಡಿ-ರ-ಲಿಲ್ಲ
ಮಾಡಿ-ರು-ತ್ತದೆ
ಮಾಡಿ-ರುತ್ತಾರೆ
ಮಾಡಿ-ರುವ
ಮಾಡಿ-ರು-ವನು
ಮಾಡಿ-ರು-ವರು
ಮಾಡಿ-ರು-ವರೊ
ಮಾಡಿ-ರು-ವರೋ
ಮಾಡಿ-ರುವಿರಾ
ಮಾಡಿ-ರು-ವಿರಿ
ಮಾಡಿ-ರು-ವು-ದ-ಕ್ಕಿಂತ
ಮಾಡಿ-ರು-ವು-ದ-ರಿಂದ
ಮಾಡಿ-ರು-ವುದು
ಮಾಡಿ-ರು-ವುದೇ
ಮಾಡಿ-ರು-ವೆನು
ಮಾಡಿ-ರು-ವೆವು
ಮಾಡಿಲ್ಲ
ಮಾಡಿವೆ
ಮಾಡಿ-ಸಿ-ಕೊಳ್ಳು-ವು-ದಕ್ಕೆ
ಮಾಡಿ-ಸಿ-ದ್ದೀರಿ
ಮಾಡಿ-ಸು-ತ್ತಿದ್ದರು
ಮಾಡಿ-ಸುತ್ತಿ-ರುವಳು
ಮಾಡಿ-ಸು-ತ್ತೇನೆ
ಮಾಡು
ಮಾಡು-ತ್ತ-ದಂತೆ
ಮಾಡು-ತ್ತದೆ
ಮಾಡು-ತ್ತದೆಯೋ
ಮಾಡು-ತ್ತವೆ
ಮಾಡುತ್ತಾ
ಮಾಡು-ತ್ತಾನೆ
ಮಾಡು-ತ್ತಾರೆ
ಮಾಡು-ತ್ತಿದೆ
ಮಾಡು-ತ್ತಿದೆಯೆ
ಮಾಡು-ತ್ತಿದ್ದ
ಮಾಡು-ತ್ತಿದ್ದನು
ಮಾಡು-ತ್ತಿದ್ದರು
ಮಾಡು-ತ್ತಿದ್ದರೊ
ಮಾಡು-ತ್ತಿ-ದ್ದಾಗ
ಮಾಡು-ತ್ತಿದ್ದಿರಿ
ಮಾಡು-ತ್ತಿದ್ದೀರೋ
ಮಾಡು-ತ್ತಿ-ದ್ದೇನೆ
ಮಾಡು-ತ್ತಿ-ರಲಿ
ಮಾಡು-ತ್ತಿ-ರಲಿಲ್ಲ
ಮಾಡು-ತ್ತಿ-ರು-ತ್ತವೆ
ಮಾಡು-ತ್ತಿ-ರುವ
ಮಾಡು-ತ್ತಿ-ರು-ವಂತೆ
ಮಾಡು-ತ್ತಿ-ರು-ವನು
ಮಾಡು-ತ್ತಿ-ರು-ವರು
ಮಾಡು-ತ್ತಿ-ರು-ವರೋ
ಮಾಡು-ತ್ತಿರು-ವಾಗಲೂ
ಮಾಡು-ತ್ತಿ-ರು-ವು-ದಕ್ಕೆ
ಮಾಡು-ತ್ತಿರು-ವು-ದನ್ನು
ಮಾಡು-ತ್ತಿರು-ವುದು
ಮಾಡು-ತ್ತಿ-ರು-ವುದೇ
ಮಾಡು-ತ್ತಿರು-ವೆನು
ಮಾಡು-ತ್ತಿರು-ವೆವು
ಮಾಡು-ತ್ತಿವೆ
ಮಾಡು-ತ್ತೀಯೆ
ಮಾಡು-ತ್ತೀರಿ
ಮಾಡು-ತ್ತೇನೆ
ಮಾಡು-ತ್ತೇ-ನೆಯೋ
ಮಾಡು-ತ್ತೇವೆ
ಮಾಡು-ತ್ತೇ-ವೆ-ಯೆಂದು
ಮಾಡುವ
ಮಾಡು-ವಂತಹ
ಮಾಡು-ವಂತಾಗಲಿ
ಮಾಡು-ವಂತೆ
ಮಾಡು-ವಂಥ-ವೆ-ಲ್ಲ-ದಕ್ಕೂ
ಮಾಡು-ವನು
ಮಾಡು-ವರು
ಮಾಡು-ವರೋ
ಮಾಡು-ವ-ವ-ನನ್ನು
ಮಾಡು-ವ-ವನೂ
ಮಾಡು-ವ-ವರು
ಮಾಡು-ವ-ವ-ರೆಂದು
ಮಾಡು-ವ-ವ-ರೆಲ್ಲರೂ
ಮಾಡು-ವ-ವರೇ
ಮಾಡು-ವಾಗ
ಮಾಡು-ವಿರಿ
ಮಾಡು-ವಿ-ರೆಂದು
ಮಾಡು-ವಿರೋ
ಮಾಡು-ವುದಕ್ಕಾ-ಗದೆ
ಮಾಡು-ವು-ದ-ಕ್ಕಾಗಿ
ಮಾಡು-ವು-ದ-ಕ್ಕಾಗಿಯೇ
ಮಾಡು-ವು-ದಕ್ಕೂ
ಮಾಡು-ವು-ದಕ್ಕೆ
ಮಾಡು-ವು-ದನ್ನು
ಮಾಡು-ವುದನ್ನೆಲ್ಲ
ಮಾಡು-ವು-ದರ
ಮಾಡು-ವು-ದ-ರಲ್ಲಿ
ಮಾಡು-ವು-ದರ-ಲ್ಲಿದೆ
ಮಾಡು-ವು-ದ-ರಿಂದ
ಮಾಡು-ವು-ದಲ್ಲದೆ
ಮಾಡು-ವು-ದಿಲ್ಲ
ಮಾಡು-ವು-ದಿಲ್ಲವೋ
ಮಾಡು-ವುದು
ಮಾಡು-ವುದೆ
ಮಾಡು-ವು-ದೆಂದರೆ
ಮಾಡು-ವು-ದೆಂದು
ಮಾಡು-ವು-ದೆಂಬು-ದನ್ನು
ಮಾಡು-ವುದೇ
ಮಾಡು-ವು-ದೊಂದು
ಮಾಡು-ವುದೋ
ಮಾಡು-ವುವು
ಮಾಡು-ವುವೋ
ಮಾಡು-ವೆನು
ಮಾಡು-ವೆವು
ಮಾಡೋಣ
ಮಾಡ್ಧಲ್ಧಾಗ್ಧುತ್ತ್ಧದ್ಧೆಯೆ
ಮಾತಂತಿ-ರಲಿ
ಮಾತ-ನಾ-ಡದೆ
ಮಾತ-ನಾಡ-ಬಹುದು
ಮಾತ-ನಾಡ-ಬೇ-ಕಾದರೂ
ಮಾತ-ನಾಡ-ಬೇಕು
ಮಾತ-ನಾಡ-ಬೇಕೆಂದಿ-ರುವ
ಮಾತ-ನಾ-ಡಲಿ
ಮಾತ-ನಾ-ಡಲು
ಮಾತ-ನಾಡಿ
ಮಾತ-ನಾಡಿ-ಕೊಳ್ಳಲಿ
ಮಾತ-ನಾಡಿ-ದನು
ಮಾತ-ನಾಡಿ-ದರು
ಮಾತ-ನಾಡಿ-ದರೂ
ಮಾತ-ನಾಡಿ-ದರೆ
ಮಾತ-ನಾಡಿ-ದ-ವರೊ-ಬ್ಬರು
ಮಾತ-ನಾಡಿದೆ
ಮಾತ-ನಾಡಿಲ್ಲ
ಮಾತ-ನಾ-ಡುತ್ತ
ಮಾತ-ನಾಡು-ತ್ತದೆ
ಮಾತ-ನಾಡು-ತ್ತಾರೆ
ಮಾತ-ನಾಡು-ತ್ತಿದ್ದ
ಮಾತ-ನಾಡು-ತ್ತಿ-ದ್ದಾಗ
ಮಾತ-ನಾಡು-ತ್ತಿ-ದ್ದೇನೆ
ಮಾತ-ನಾಡು-ತ್ತಿ-ರಲಿಲ್ಲ
ಮಾತ-ನಾಡು-ತ್ತಿ-ರು-ವರು
ಮಾತ-ನಾಡು-ತ್ತಿರು-ವಿರಿ
ಮಾತ-ನಾಡು-ತ್ತಿರು-ವೆನು
ಮಾತ-ನಾಡು-ತ್ತಿಲ್ಲ
ಮಾತ-ನಾಡು-ತ್ತೇನೆ
ಮಾತ-ನಾಡು-ತ್ತೇವೆ
ಮಾತ-ನಾ-ಡುವ
ಮಾತ-ನಾಡು-ವರು
ಮಾತ-ನಾಡು-ವ-ವ-ನಲ್ಲ
ಮಾತ-ನಾಡು-ವಾಗ
ಮಾತ-ನಾಡು-ವು-ದ-ಕ್ಕಾಗಿ
ಮಾತ-ನಾಡು-ವು-ದಕ್ಕೆ
ಮಾತ-ನಾಡು-ವು-ದನ್ನು
ಮಾತ-ನಾಡು-ವು-ದ-ರಲ್ಲಿ
ಮಾತ-ನಾಡು-ವು-ದ-ರಿಂದ
ಮಾತ-ನಾಡು-ವು-ದಲ್ಲ
ಮಾತ-ನಾಡು-ವುದು
ಮಾತ-ನಾಡು-ವು-ದೊಂದು
ಮಾತ-ನಾಡು-ವುವು
ಮಾತನ್ನು
ಮಾತ-ನ್ನೆ-ತ್ತು-ವುದು
ಮಾತಲ್ಲ
ಮಾತಾಡ-ಬಹುದು
ಮಾತಾಡ-ಬೇಕೆಂದುದು
ಮಾತಾ-ಡಲು
ಮಾತಾ-ಡುವ
ಮಾತಾ-ನಾಡು-ತ್ತಿ-ರಲಿ
ಮಾತಾ-ಪಿ-ತ-ರಿಗೆ
ಮಾತಾ-ಪಿ-ತೃ-ಗಳು
ಮಾತಾ-ಯಿತು
ಮಾತಾಳಿ-ಗಳಿಗೆ
ಮಾತಿ-ಗಿಂತ
ಮಾತಿಗೆ
ಮಾತಿನ
ಮಾತಿ-ನಂತೆ
ಮಾತಿ-ನ-ಲ್ಲಲ್ಲ
ಮಾತಿ-ನಲ್ಲಿ
ಮಾತಿ-ನ್ನೇನು
ಮಾತು
ಮಾತು-ಗಳ
ಮಾತು-ಗಳನ್ನು
ಮಾತು-ಗಳಲ್ಲಿ
ಮಾತು-ಗಳಿಂದ
ಮಾತು-ಗಳಿಗೆ
ಮಾತು-ಗಳಿವು
ಮಾತು-ಗಳು
ಮಾತು-ಗಳೆ-ಲ್ಲವೂ
ಮಾತು-ಗಳೇ
ಮಾತೂ
ಮಾತೃ
ಮಾತೃ-ಭೂಮಿ
ಮಾತೃ-ಭೂಮಿ-ಗಾಗಿ
ಮಾತೃ-ಭೂಮಿಗೆ
ಮಾತೃ-ಭೂಮಿಯ
ಮಾತೃ-ಭೂಮಿ-ಯನ್ನು
ಮಾತೃ-ಭೂಮಿ-ಯಲ್ಲಿ
ಮಾತೃ-ಭೂಮಿ-ಯ-ಲ್ಲಿ-ರುವ
ಮಾತೃ-ಭೂಮಿ-ಯ-ಲ್ಲಿ-ರು-ವರು
ಮಾತೃ-ಭೂಮಿ-ಯಾದ
ಮಾತೃ-ಭೂಮಿ-ಯಿಂದ
ಮಾತೃ-ಭೂಮಿಯು
ಮಾತೃ-ಭೂಮಿಯೇ
ಮಾತೆಲ್ಲ
ಮಾತೇನು
ಮಾತ್ರ
ಮಾತ್ರಕ್ಕೆ
ಮಾತ್ರ-ದಿಂದ
ಮಾತ್ರ-ದಿಂದಲೇ
ಮಾತ್ರ-ನಿಮಗೆ
ಮಾತ್ರ-ವನ್ನಾಗಿ
ಮಾತ್ರ-ವನ್ನು
ಮಾತ್ರ-ವಲ್ಲ
ಮಾತ್ರ-ವ-ಲ್ಲದೆ
ಮಾತ್ರ-ವಲ್ಲ್ಧ್ಧ್ಧ್ಧ
ಮಾತ್ರ-ವಾಗಿದೆ
ಮಾತ್ರ-ವಾಗಿ-ದ್ದರೆ
ಮಾತ್ರವೆ
ಮಾತ್ರ-ವೆಂದು
ಮಾತ್ರವೇ
ಮಾತ್ರ-ವೇನು
ಮಾದ-ರಿಯ
ಮಾಧುರ್ಯ
ಮಾಧ್ಯಂದಿನ-ಶಾಖಾ
ಮಾಧ್ಯಮ
ಮಾಧ್ಯಮ-ಗಳ
ಮಾನ
ಮಾನ-ಪತ್ರ-ವನ್ನು
ಮಾನವ
ಮಾನವಃ
ಮಾನವ-ಕುಲಕ್ಕೆ
ಮಾನವ-ಕೃತ
ಮಾನವ-ಕೃತ-ವಲ್ಲ
ಮಾನವ-ಕೋಟಿಗೆ
ಮಾನವ-ಕೋಟಿಯ
ಮಾನವ-ಜನಾಂಗದ
ಮಾನವ-ಜಾತಿಗೆ
ಮಾನವ-ಜೀವಿಯ
ಮಾನವ-ಜೀವಿಯು
ಮಾನವತೆ
ಮಾನವ-ತೆಗೆ
ಮಾನವ-ತೆ-ಯನ್ನು
ಮಾನವ-ತೆಯು
ಮಾನವನ
ಮಾನವ-ನನ್ನು
ಮಾನವ-ನನ್ನೇ
ಮಾನವ-ನಲ್ಲಿ
ಮಾನವ-ನಲ್ಲಿಯೂ
ಮಾನವ-ನಲ್ಲಿ-ರುವ
ಮಾನವ-ನಾಗಿ
ಮಾನವ-ನಿಗಾ-ಗಲಿ
ಮಾನವ-ನಿಗೂ
ಮಾನವ-ನಿಗೆ
ಮಾನವ-ನಿ-ರುವ
ಮಾನವನು
ಮಾನವ-ನೆಂದು
ಮಾನವ-ಪೂಜೆ
ಮಾನವರ
ಮಾನವ-ರಲ್ಲಿ
ಮಾನವ-ರಾಗಿ
ಮಾನವ-ರಿಗೆ
ಮಾನವರು
ಮಾನವ-ಸಹಜ
ಮಾನವ-ಸಾಹಿತ್ಯ-ವಾಗಿ-ದ್ದರೆ
ಮಾನ-ಸ-ರಾದ
ಮಾನ-ಸಿಕ
ಮಾನ್ಯ
ಮಾನ್ಯತೆ
ಮಾನ್ಯ-ರಾಗಿ-ದ್ದಾರೆ
ಮಾಯ-ಮಂತ್ರ-ಗಳನ್ನು
ಮಾಯ-ಮಂತ್ರ-ಗಳೆಲ್ಲ
ಮಾಯ-ವಾಗ-ಕೂಡದು
ಮಾಯ-ವಾಗದೆ
ಮಾಯ-ವಾಗ-ಬೇಕಾಗಿ-ರು-ವುದು
ಮಾಯ-ವಾಗ-ಬೇಕು
ಮಾಯ-ವಾಗಲಿ
ಮಾಯ-ವಾಗಲೇ-ಬೇಕು
ಮಾಯ-ವಾಗಿ
ಮಾಯ-ವಾಗಿದೆ
ಮಾಯ-ವಾಗಿ-ರು-ವರು
ಮಾಯ-ವಾಗಿ-ಲ್ಲ-ವೆಂಬು-ದನ್ನು
ಮಾಯ-ವಾಗಿವೆ
ಮಾಯ-ವಾಗಿ-ಹೋ-ಗು-ತ್ತದೆ
ಮಾಯ-ವಾಗು-ತ್ತದೆ
ಮಾಯ-ವಾಗುತ್ತ-ಲಿವೆ
ಮಾಯ-ವಾಗುತ್ತವೆ
ಮಾಯ-ವಾಗುತ್ತಿತ್ತು
ಮಾಯ-ವಾಗು-ತ್ತಿವೆ
ಮಾಯ-ವಾಗು-ವಂತಾಗಲಿ
ಮಾಯ-ವಾಗು-ವು-ದನ್ನು
ಮಾಯ-ವಾಗು-ವು-ದಿಲ್ಲ
ಮಾಯ-ವಾಗು-ವುದು
ಮಾಯ-ವಾಗು-ವುವು
ಮಾಯ-ವಾದ
ಮಾಯ-ವಾದಂತಾ-ಗು-ತ್ತದೆ
ಮಾಯ-ವಾದಂತೆ
ಮಾಯ-ವಾದರು
ಮಾಯ-ವಾದರೂ
ಮಾಯ-ವಾದರೆ
ಮಾಯ-ವಾದ-ವನ್ನು
ಮಾಯ-ವಾದವು
ಮಾಯ-ವಾ-ಯಿತು
ಮಾಯಾ
ಮಾಯಾ-ಪಾಶ-ದಿಂದ
ಮಾಯಾ-ಬಂಧನ-ದಿಂದ
ಮಾಯಾ-ಭಾ-ವನೆ
ಮಾಯಾ-ವಾಗಿದೆ
ಮಾಯಾ-ವಾದ-ವನ್ನು
ಮಾಯಾ-ಸಿದ್ಧಾಂತವು
ಮಾಯಾ-ಸು-ತರು
ಮಾಯೆ
ಮಾಯೆಯ
ಮಾಯೆ-ಯಿಂದ
ಮಾಯೆ-ಯಿಂದಲೇ
ಮಾಯೆ-ಯಿಂದಾಗಿದೆ
ಮಾಯೆ-ಯೆಂದು
ಮಾರ-ಡೆ-ಕ್
ಮಾರನೆ
ಮಾರ-ನೆಯ
ಮಾರಲು
ಮಾರಿ-ಕೊಳ್ಳ-ಕೂಡದು
ಮಾರಿನ
ಮಾರು-ಗಳೇ
ಮಾರುತಃ
ಮಾರು-ವ-ವನ
ಮಾರು-ಹೋದ
ಮಾರ್ಗ
ಮಾರ್ಗಕ್ಕೆ
ಮಾರ್ಗ-ಗಳಂತೆ
ಮಾರ್ಗ-ಗಳನ್ನು
ಮಾರ್ಗ-ಗಳಲ್ಲಿ
ಮಾರ್ಗ-ಗಳಿವೆ
ಮಾರ್ಗ-ಗಳು
ಮಾರ್ಗದ
ಮಾರ್ಗ-ದರ್ಶಕ
ಮಾರ್ಗ-ದರ್ಶಕ-ವಾಗುವು-ದ-ರಲ್ಲಿ
ಮಾರ್ಗ-ದರ್ಶನ
ಮಾರ್ಗ-ದಲ್ಲಿ
ಮಾರ್ಗ-ದಲ್ಲಿ-ರುವ
ಮಾರ್ಗ-ದಿಂದ
ಮಾರ್ಗ-ರೆ-ಟ್
ಮಾರ್ಗ-ರೇ-ಟ್
ಮಾರ್ಗ-ವನ್ನು
ಮಾರ್ಗ-ವನ್ನೇ
ಮಾರ್ಗ-ವಲ್ಲ
ಮಾರ್ಗ-ವ-ಲ್ಲದೆ
ಮಾರ್ಗ-ವಾಗಿ
ಮಾರ್ಗ-ವಿದೆ
ಮಾರ್ಗ-ವಿರಬೇಕೆಂಬು-ದನ್ನೂ
ಮಾರ್ಗವು
ಮಾರ್ಗವೆ
ಮಾರ್ಗ-ವೆಂದು
ಮಾರ್ಗವೇ
ಮಾರ್ಚಿ
ಮಾರ್ಚ್
ಮಾರ್ಪಟ್ಟಿವೆ
ಮಾರ್ಪಡಿ-ಸಿ-ದೆವು
ಮಾರ್ಪಾಡಿಗೆ
ಮಾಲೆಯ-ನ್ನೇ
ಮಾಸ
ಮಾಹಾ-ತ್ಮ್ಯೆ-ಯನ್ನು
ಮಾಹಿತಿ
ಮಾಹಿ-ತಿಯ
ಮಿಂಚಿ-ನಂತೆ
ಮಿಂಚು
ಮಿಂಚೂ
ಮಿಂದರೆ
ಮಿಂದು
ಮಿಕ್ಕ
ಮಿಕ್ಕಿರು-ವುದೇ
ಮಿಗಿ-ಲಾಗಿ
ಮಿಗಿ-ಲಾದ
ಮಿಗಿಲಾ-ದು-ದನ್ನು
ಮಿಗಿಲಾ-ದುದು
ಮಿಠಾಯಿ
ಮಿಠಾಯಿ-ಯನ್ನು
ಮಿಡಿಯುತ್ತಿರು-ವುದು
ಮಿಡಿಯುತ್ತಿ-ವೆಯೋ
ಮಿಡಿ-ಯುವ
ಮಿತಿ
ಮಿತಿ-ಮೀರಿ
ಮಿತಿ-ಯನ್ನು
ಮಿತಿ-ಯೊಳ-ಗಿದೆ
ಮಿತ್ರನೆ
ಮಿತ್ರ-ರಲ್ಲಿ
ಮಿತ್ರರು
ಮಿತ್ರರೆ
ಮಿತ್ರರೇ
ಮಿಥ್ಯಾ
ಮಿಥ್ಯಾ-ಚಾರಿ-ಗಳಾಗ-ಬೇಡಿ
ಮಿಥ್ಯಾ-ಚಾರಿ-ಗಳಾಗಿ-ರ-ಬೇಕು
ಮಿಥ್ಯಾ-ಚಾರಿಗಳಾ-ಗು-ತ್ತಿದ್ದರು
ಮಿಥ್ಯಾ-ಬೋಧೆ
ಮಿಥ್ಯಾ-ವಾದಿ
ಮಿಥ್ಯಾ-ಸಂಸಾರದ
ಮಿಥ್ಯೆ
ಮಿಥ್ಯೆಯ
ಮಿಥ್ಯೆ-ಯಾಗಿ-ರ-ಬೇಕು
ಮಿಥ್ಯೆ-ಯೆಂದು
ಮಿದು-ಳನ್ನು
ಮಿದುಳನ್ನೂ
ಮಿದುಳನ್ನೇ
ಮಿದುಳಿನ
ಮಿದುಳು
ಮಿದುಳು-ಳ್ಳ-ವರ
ಮಿದುಳೇ
ಮಿಲನ-ವಾಗಿ
ಮಿಲನ-ವಾಗು-ವುದು
ಮಿಲ್ಟ-ನ್
ಮಿಶ್ರಣ
ಮಿಶ್ರ-ವಾಗಿ
ಮಿಸ-ಸ್
ಮಿಸ್
ಮಿಸ್ಟರ್
ಮೀನನ್ನು
ಮೀನು
ಮೀಮಾಂಸ-ಕರ
ಮೀರಿ
ಮೀರಿದ
ಮೀರಿ-ದ್ದಾ-ಗಿದೆ
ಮೀರಿದ್ದು
ಮೀರಿ-ರು-ವುದು
ಮೀರಿ-ಸ-ಬಲ್ಲೆವು
ಮೀರಿ-ಸ-ಲಾ-ರದು
ಮೀರಿ-ಸಿದ
ಮೀರಿ-ಸಿ-ದ್ದೀರಿ
ಮೀರಿ-ಸುವ-ವರು
ಮೀರಿ-ಹೋ-ಗ-ಲಾ-ರರು
ಮೀರಿ-ಹೋ-ಗಲು
ಮೀರಿ-ಹೋ-ಗು-ವುದು
ಮೀರು-ವಂತಾಗಲಿ
ಮೀರು-ವಂತೆ
ಮೀಸ
ಮೀಸ-ಲಾಗಿಟ್ಟ
ಮೀಸ-ಲಾಗಿ-ಡು-ವುದೇ
ಮೀಸ-ಲಾ-ಗಿದೆ
ಮೀಸ-ಲಾಗಿ-ರಲಿ
ಮೀಸ-ಲಾ-ಗಿವೆ
ಮೀಸ-ಲಾದ
ಮೀಸಲು
ಮೀಸೆಯಿ-ರುವ
ಮುಂಚಿನ
ಮುಂಚಿ-ನಿಂದಲೇ
ಮುಂಚೆ
ಮುಂಚೆಯೂ
ಮುಂಚೆಯೇ
ಮುಂಜಾನೆ
ಮುಂಜಾ-ನೆ-ಯಲ್ಲಿ
ಮುಂಡ
ಮುಂಡಕ
ಮುಂತಾಗಿ
ಮುಂತಾದ
ಮುಂತಾದ-ವರ
ಮುಂತಾದ-ವ-ರಂತೆ
ಮುಂತಾದ-ವ-ರನ್ನು
ಮುಂತಾದ-ವರು
ಮುಂತಾದ-ವ-ರೆಲ್ಲಾ
ಮುಂತಾ-ದವು
ಮುಂತಾದ-ವು-ಗಳ
ಮುಂತಾದ-ವು-ಗಳ-ನ್ನೆಲ್ಲ
ಮುಂತಾದ-ವು-ಗಳಿಂದ
ಮುಂತಾದ-ವು-ಗಳಿಗೆ
ಮುಂತಾದ-ವು-ಗಳು
ಮುಂತಾದುವಕ್ಕೆ
ಮುಂತಾದು-ವನ್ನು
ಮುಂತಾದುವನ್ನೆಲ್ಲ
ಮುಂತಾ-ದುವು
ಮುಂತಾ-ದುವು-ಗಳಲ್ಲಿ
ಮುಂತಾದು-ವೆಲ್ಲ
ಮುಂದಕ್ಕೆ
ಮುಂದಣ
ಮುಂದಾಗು-ವು-ದನ್ನು
ಮುಂದಾ-ದರೂ
ಮುಂದಾದು-ವನ್ನೆಲ್ಲಾ
ಮುಂದಾಲೋಚನೆ
ಮುಂದಾಳಾ-ಗುವ
ಮುಂದಾಳಾ-ಗುವ-ನಮ್ಮ
ಮುಂದಾಳು
ಮುಂದಾಳು-ಗಳು
ಮುಂದಿ-ಡಲು
ಮುಂದಿ-ಡುವ
ಮುಂದಿಡು-ವೆನು
ಮುಂದಿದೆ
ಮುಂದಿನ
ಮುಂದಿನದು
ಮುಂದಿ-ರಲಿ
ಮುಂದಿ-ರುವ
ಮುಂದಿಲ್ಲ-ದಿ-ರು-ವುದೇ
ಮುಂದು
ಮುಂದು-ವರಿಕೆ-ಯಾಗಿದೆ
ಮುಂದು-ವರಿದ
ಮುಂದು-ವರಿ-ದಂತೆ
ಮುಂದು-ವರಿ-ದಂತೆಲ್ಲಾ
ಮುಂದು-ವರಿದರೆ
ಮುಂದು-ವರಿದು
ಮುಂದು-ವರಿದುವು
ಮುಂದು-ವರಿ-ಯದೆ
ಮುಂದು-ವರಿಯ-ಬಲ್ಲ
ಮುಂದು-ವರಿಯ-ಬೇ-ಕಾದರೆ
ಮುಂದು-ವರಿಯ-ಬೇಕು
ಮುಂದು-ವರಿ-ಯಲಿ
ಮುಂದು-ವರಿ-ಯಿತು
ಮುಂದು-ವರಿ-ಯಿರಿ
ಮುಂದು-ವರಿಯು-ತ್ತದೆ
ಮುಂದು-ವರಿಯುತ್ತಿರ-ಬೇಕು
ಮುಂದು-ವರಿಯುತ್ತಿರ-ಬೇಕೆಂದು
ಮುಂದು-ವರಿ-ಯು-ವಂತೆ
ಮುಂದು-ವರಿ-ಯು-ವು-ದಿಲ್ಲ
ಮುಂದು-ವರಿಯು-ವುದು
ಮುಂದು-ವರಿ-ಯೋಣ
ಮುಂದು-ವ-ರಿ-ಸಲು
ಮುಂದು-ವರಿ-ಸುತ್ತ
ಮುಂದು-ವರಿ-ಸು-ವು-ದಕ್ಕೆ
ಮುಂದೆ
ಮುಂದೆಯೂ
ಮುಂಬರಿದು
ಮುಂಬ-ರುವ
ಮುಕ್ಕಲಿ
ಮುಕ್ಕಳಿಸ-ಬೇಕೆ
ಮುಕ್ಕಾಲು
ಮುಕ್ಕಾಲು-ಪಾಲು
ಮುಕ್ತ
ಮುಕ್ತ-ಕಂಠ-ದಿಂದ
ಮುಕ್ತ-ನಾಗ-ಬೇಕೆಂಬ
ಮುಕ್ತ-ನಾಗಿ-ರ-ಬೇಕು
ಮುಕ್ತ-ನಾಗು-ತ್ತಾನೆ
ಮುಕ್ತ-ನಾಗು-ವನು
ಮುಕ್ತ-ನಾ-ಗು-ವುದೇ
ಮುಕ್ತನು
ಮುಕ್ತಯೇ
ಮುಕ್ತ-ರ-ನ್ನಾಗಿ
ಮುಕ್ತ-ರಾಗಿ
ಮುಕ್ತ-ರಾ-ಗಿಯೇ
ಮುಕ್ತ-ರಾಗು-ವುದು
ಮುಕ್ತ-ರಾ-ದ-ವರು
ಮುಕ್ತ-ವಾಗಿ-ರ-ಬಹುದು
ಮುಕ್ತ-ವಾಗಿ-ರುವ
ಮುಕ್ತಾಯ
ಮುಕ್ತಾಯ-ಗೊಳಿ-ಸಿದರು
ಮುಕ್ತಾಯ-ಗೊಳಿಸು
ಮುಕ್ತಾಯ-ಗೊಳಿಸು-ವೆನು
ಮುಕ್ತಾಯ-ವಾಗು-ವುದು
ಮುಕ್ತಿ
ಮುಕ್ತಿ-ಕಾಮಿ-ಗಳೆಲ್ಲಾ
ಮುಕ್ತಿ-ಗಲ್ಲ
ಮುಕ್ತಿ-ಗಾಗಿ
ಮುಕ್ತಿ-ಗಿಂತ
ಮುಕ್ತಿಗೆ
ಮುಕ್ತಿ-ಮಾರ್ಗ-ವನ್ನು
ಮುಕ್ತಿಯ
ಮುಕ್ತಿ-ಯಂತೆ
ಮುಕ್ತಿ-ಯನ್ನು
ಮುಕ್ತಿ-ಯಿಲ್ಲ
ಮುಖ
ಮುಖಂಡ
ಮುಖಂಡನ
ಮುಖ-ಗಳಂತೆ
ಮುಖ-ಗಳು
ಮುಖ-ಗಳುಂಟು
ಮುಖ-ದಂತೆ
ಮುಖ-ದಲ್ಲಿ
ಮುಖ-ವನ್ನು
ಮುಖ-ವನ್ನೇ
ಮುಖವು
ಮುಖ-ವೆತ್ತಿ
ಮುಖವೇ
ಮುಖಾಂತರ
ಮುಖಾಮುಖಿ-ಯಾಗಿ
ಮುಖ್ಯ
ಮುಖ್ಯ-ಕಾರಣ
ಮುಖ್ಯ-ವಲ್ಲ
ಮುಖ್ಯ-ವಾಗಿ
ಮುಖ್ಯ-ವಾಗಿದ್ದು
ಮುಖ್ಯ-ವಾಗಿ-ರು-ವುದು
ಮುಖ್ಯ-ವಾದ
ಮುಖ್ಯ-ವಾದಂತೆ
ಮುಖ್ಯ-ವಾದ-ದ್ದೇ-ನೆಂದರೆ
ಮುಖ್ಯ-ವಾದುದು
ಮುಖ್ಯ-ವಾದುವು
ಮುಖ್ಯ-ಸ್ವರ
ಮುಖ್ಯಾಂಶ-ವಾಗಿ
ಮುಗಿದ
ಮುಗಿದ-ಮೇಲೆ
ಮುಗಿದು
ಮುಗಿ-ಯ-ಲಿಲ್ಲ
ಮುಗಿ-ಯಿತು
ಮುಗಿ-ಯು-ವಂತೆ
ಮುಗಿಯುವುದರೊಳಗೆ
ಮುಗಿಲಿನಾಚೆ
ಮುಗಿಲಿನಾಚೆಯ
ಮುಗಿಲು-ಗಳು
ಮುಗಿ-ಸ-ಲಾರೆವು
ಮುಗಿಸಿ
ಮುಗಿಸು-ತ್ತೇನೆ
ಮುಗಿಸು-ವುದು
ಮುಗ್ಧ
ಮುಗ್ಧ-ರಾಗು-ವರು
ಮುಚ್ಚಲು
ಮುಚ್ಚ-ಲ್ಪಟ್ಟಿವೆ
ಮುಚ್ಚಿ
ಮುಚ್ಚಿ-ಡುವಿಕೆ
ಮುಚ್ಚಿ-ಡು-ವು-ದಕ್ಕೆ
ಮುಚ್ಚು-ವುದು
ಮುಚ್ಚೋಣ
ಮುಟ್ಟದ
ಮುಟ್ಟ-ಬಹುದೇ
ಮುಟ್ಟ-ಬಾ-ರದು
ಮುಟ್ಟ-ಬೇಡಿ
ಮುಟ್ಟ-ಲಾ-ಗ-ಲಿಲ್ಲ
ಮುಟ್ಟ-ಲಾ-ರದು
ಮುಟ್ಟ-ಲಿಲ್ಲ
ಮುಟ್ಟಲು
ಮುಟ್ಟಿ
ಮುಟ್ಟಿತು
ಮುಟ್ಟಿದ
ಮುಟ್ಟಿ-ದಂತಾ-ಗು-ತ್ತದೆ
ಮುಟ್ಟಿ-ದರು
ಮುಟ್ಟಿ-ದರೆ
ಮುಟ್ಟಿ-ದಲ್ಲದೆ
ಮುಟ್ಟಿ-ದವು
ಮುಟ್ಟಿ-ದ್ದರೆ
ಮುಟ್ಟಿ-ರುತ್ತಾರೆ
ಮುಟ್ಟಿ-ರು-ವರು
ಮುಟ್ಟಿ-ರು-ವುವು
ಮುಟ್ಟು-ತ್ತದೆ
ಮುಟ್ಟು-ತ್ತವೆ
ಮುಟ್ಟುವ
ಮುಟ್ಟು-ವಂತೆ
ಮುಟ್ಟು-ವರು
ಮುಟ್ಟುವ-ವರೆಗೂ
ಮುಟ್ಟುವ-ವ-ರೆಗೆ
ಮುಟ್ಟು-ವು-ದಕ್ಕೆ
ಮುಠ್ಠಾಳರು
ಮುಡಿಪಾ-ಗಿರಿ-ಸಿದ
ಮುಡುಕು-ಗಳನ್ನು
ಮುಡುಪಾಗಿ-ಡುತ್ತಾರೋ
ಮುತ್ತಿನ
ಮುತ್ತಿನ-ರೂಪ-ವನ್ನು
ಮುತ್ತು
ಮುತ್ತು-ಗಳನ್ನು
ಮುತ್ತು-ತ್ತಿದೆ
ಮುದಿ-ತನ
ಮುದಿ-ತ-ನದ
ಮುದುಕ-ರಾಗಿ
ಮುದುಕ-ರಾದ
ಮುದು-ಕರು
ಮುದುಕಿ-ಯೊಂದಿಗೆ
ಮುದ್ದೆ
ಮುದ್ದೆ-ಯಾಗಿ
ಮುದ್ದೆ-ಯಿಂದ
ಮುದ್ರಣ
ಮುದ್ರಿತ-ವಾಗಿ-ರುವ
ಮುದ್ರಿತ-ವಾದಂತೆ
ಮುದ್ರೆ
ಮುದ್ರೆ-ಯನ್ನು
ಮುನಿ-ಗಳು
ಮುನಿ-ಗಳೇ
ಮುನಿಯು
ಮುನ್ನ
ಮುನ್ನಡೆ-ಯ-ಲಿಲ್ಲ
ಮುನ್ನವೇ
ಮುನ್ನುಗ್ಗಿ-ರು-ವರು
ಮುನ್ನುಡಿ
ಮುನ್ನುಡಿ-ಯಲ್ಲಿ
ಮುನ್ನೂರು
ಮುನ್ಸೂಚ-ನೆಯು
ಮುಮುಕ್ಷುತ್ವ
ಮುಮುಕ್ಷುತ್ವಂ
ಮುಮುಕ್ಷುತ್ವ-ವಿ-ಲ್ಲದೆ
ಮುಯ್ಯಿ
ಮುಯ್ಯಿಗೆ
ಮುರಿಯು-ತ್ತದೆ
ಮುರಿಯು-ವರು
ಮುಲ್ಲ-ರರ
ಮುಲ್ಲ-ರಿಗೆ
ಮುಲ್ಲರ್
ಮುಳು-ಗದ
ಮುಳು-ಗಲೇ-ಬೇ-ಕಾದರೆ
ಮುಳುಗಿ
ಮುಳುಗಿ-ಸಬಲ್ಲುದು
ಮುಳುಗಿ-ಸಿತು
ಮುಳುಗಿ-ಸು-ವುದು
ಮುಳುಗಿ-ಹೋಗಿ
ಮುಳು-ಗು-ತ್ತಿದೆ
ಮುಳುಗು-ತ್ತಿ-ದ್ದಾಗ
ಮುಳುಗುತ್ತಿ-ರುವ
ಮುಳುಗುತ್ತಿರು-ವೆವು
ಮುಳುಗೋಣ
ಮುಳ್ಳು
ಮುಷ್ಟಿ-ಯಿಂದ
ಮುಸಲ-ಧಾರೆಯ
ಮುಸಲ್ಮಾನ
ಮುಸಲ್ಮಾ-ನರು
ಮುಸುಕಿ
ಮುಹ್ಯ-ಮಾನಃ
ಮೂಕ
ಮೂಕ-ವಾಗಿದೆ
ಮೂಗನಾ-ದರೂ
ಮೂಗಿನ
ಮೂಗು
ಮೂಗು-ಗಳನ್ನು
ಮೂಗು-ಗಳೇ
ಮೂಟೆ
ಮೂಡ-ಬೇಕು
ಮೂಡಿತು
ಮೂಡಿದ
ಮೂಡಿ-ದುವು
ಮೂಡಿದ್ದು
ಮೂಡಿ-ಬಂದಿದೆ
ಮೂಡಿ-ಬರು-ವುದು
ಮೂಡಿ-ಸಿ-ದ್ದೀರಿ
ಮೂಡು-ತ್ತದೆ
ಮೂಡುತ್ತಿರು-ವು-ದನ್ನು
ಮೂಡು-ವಂತೆ
ಮೂಡು-ವುದೋ
ಮೂಢ-ನಂತೆ
ಮೂಢ-ನಂಬಿಕೆ
ಮೂಢ-ನಂಬಿ-ಕೆ-ಗಳ
ಮೂಢ-ನಂಬಿ-ಕೆ-ಗಳನ್ನು
ಮೂಢ-ನಂಬಿ-ಕೆ-ಗಳನ್ನೂ
ಮೂಢ-ನಂಬಿ-ಕೆ-ಗಳ-ನ್ನೆಲ್ಲಾ
ಮೂಢ-ನಂಬಿ-ಕೆ-ಗಳ-ಲ್ಲಿಯೇ
ಮೂಢ-ನಂಬಿ-ಕೆ-ಗಳಿವೆ
ಮೂಢ-ನಂಬಿ-ಕೆ-ಗಳು
ಮೂಢ-ನಂಬಿ-ಕೆಗೂ
ಮೂಢ-ನಂಬಿ-ಕೆ-ಯಲ್ಲಿ
ಮೂಢ-ನಂಬಿ-ಕೆ-ಯಿಂದ
ಮೂಢ-ಮತಿ-ಗಳ
ಮೂಢರ
ಮೂಢ-ರಲ್ಲ
ಮೂಢ-ರಿಗೆ
ಮೂಢರು
ಮೂಢ-ವಾಗಿ-ರುವ
ಮೂಢಾ
ಮೂಢಾಃ
ಮೂದಲಿ-ಸಿದಳು
ಮೂದಲಿಸುತ್ತಿರುವಿ-ರಲ್ಲ
ಮೂರ-ನೆಯ
ಮೂರ-ನೆ-ಯ-ದಾಗಿ
ಮೂರ-ನೆ-ಯದೇ
ಮೂರನ್ನು
ಮೂರನ್ನೂ
ಮೂರರ
ಮೂರು
ಮೂರ್ಖ
ಮೂರ್ಖ-ತನ
ಮೂರ್ಖ-ರಂತೆ
ಮೂರ್ಖ-ರಾಗ-ಬೇಕು
ಮೂರ್ಖ-ರಾಗಿ-ರು-ವು-ದ-ಕ್ಕಿಂತ
ಮೂರ್ಖ-ರಿಗೆ
ಮೂರ್ಖ-ರಿ-ರು-ವು-ದ-ಕ್ಕಿಂತ
ಮೂರ್ಖರು
ಮೂರ್ತ
ಮೂರ್ತಿ
ಮೂರ್ತಿ-ಗಳು
ಮೂರ್ತಿ-ಪೂ-ಜಕ-ನನ್ನು
ಮೂರ್ತಿ-ಪೂಜೆ-ಯನ್ನು
ಮೂರ್ತಿ-ಯಾಗಿದ್ದ
ಮೂರ್ತಿ-ಯಾಗಿ-ರು-ವನು
ಮೂಲ
ಮೂಲಕ
ಮೂಲ-ಕವೂ
ಮೂಲ-ಕವೇ
ಮೂಲ-ಕಾರಣ
ಮೂಲಕ್ಕೆ
ಮೂಲತಃ
ಮೂಲ-ತತ್ತ್ವ-ಗಳ
ಮೂಲ-ತತ್ತ್ವ-ಗಳನ್ನು
ಮೂಲ-ತತ್ತ್ವ-ಗಳಲ್ಲಿ
ಮೂಲ-ತತ್ತ್ವ-ವನ್ನು
ಮೂಲ-ತತ್ವ-ಗಳಲ್ಲಿ
ಮೂಲ-ದಲ್ಲಿ
ಮೂಲ-ದಲ್ಲಿವೆ
ಮೂಲ-ಪುರು-ಷನ
ಮೂಲ-ಪುರು-ಷರು
ಮೂಲ-ಬೇರು
ಮೂಲ-ಭಾವ
ಮೂಲ-ಭಿತ್ತಿ
ಮೂಲ-ಭಿತ್ತಿಯೇ
ಮೂಲ-ಭೂತ
ಮೂಲ-ಭೂತ-ವಾದ
ಮೂಲ-ಮಂತ್ರ
ಮೂಲ-ಮಂತ್ರಕ್ಕೆ
ಮೂಲ-ಮಂತ್ರ-ವಾಗು
ಮೂಲ-ಮಂತ್ರವೇ
ಮೂಲ-ರೂಪ
ಮೂಲ-ವನ್ನು
ಮೂಲ-ವಸ್ತು
ಮೂಲ-ವಸ್ತು-ಗಳು
ಮೂಲ-ವಾದ
ಮೂಲ-ವೆಂದು
ಮೂಲ-ವೆಲ್ಲ
ಮೂಲ-ವೆಲ್ಲಿ
ಮೂಲವೇ
ಮೂಲ-ಸೂತ್ರ-ಗಳನ್ನು
ಮೂಲ-ಸ್ಥಾನಕ್ಕೆ
ಮೂಲಾ-ಭಿಪ್ರಾಯ-ಗಳು
ಮೂಲಾರ್ಥವನ್ನೇ
ಮೂಲೆ
ಮೂಲೆ-ಗಳನ್ನೂ
ಮೂಲೆ-ಗಳಿಂದ
ಮೂಲೆ-ಗಳಿಂದಲೂ
ಮೂಲೆ-ಗಳಿಗೂ
ಮೂಲೆ-ಗಳಿಗೆ
ಮೂಲೆಗೆ
ಮೂಲೆ-ಮುಡುಕು-ಗಳು
ಮೂಲೆ-ಮೂಲೆ-ಗಳನ್ನು
ಮೂಲೆ-ಮೂಲೆ-ಗಳಲ್ಲೂ
ಮೂಲೆ-ಮೂಲೆ-ಗಳಿಗೆ
ಮೂಲೆ-ಯಿಂದ
ಮೂಳೆ
ಮೂವತ್ತ-ಮೂರು
ಮೂವತ್ತು
ಮೂವರು
ಮೂಸಿ
ಮೃಗ
ಮೃಗ-ಗಳ
ಮೃಗ-ಗಳಂತೆ
ಮೃಗ-ಗಳೂ
ಮೃಗ-ರಾಜ-ನಂತೆ
ಮೃಗೀಯ
ಮೃಣ್ಮಯದ
ಮೃತ-ನಾದ
ಮೃತಪ್ರಾಯ-ವಾದ
ಮೃತ್ತಿಕಾ
ಮೃತ್ಯು
ಮೃತ್ಯು-ಪಾಯ-ವಾಗಿ-ರುವ
ಮೃತ್ಯು-ಮುಖ-ರಾಗ-ಬೇಕು
ಮೃತ್ಯು-ವನ್ನು
ಮೃತ್ಯು-ವನ್ನೂ
ಮೃತ್ಯು-ವಶ-ರಾಗ-ಲೇ-ಬೇಕು
ಮೃತ್ಯು-ವಿ-ಗಿಂತ
ಮೃತ್ಯು-ವಿಗೆ
ಮೃತ್ಯು-ವಿನ
ಮೃತ್ಯು-ವಿ-ನೊಂದಿಗೆ
ಮೃತ್ಯು-ಸಮ್ಮುಖ-ದಲ್ಲಿಯೂ
ಮೃತ್ಯು-ಸ್ಥಿತಿ-ಯಿಂದ
ಮೃದು
ಮೃದು-ವಾಗಿ-ಯಾ-ದರೂ
ಮೃದು-ವಾಗಿ-ರ-ತಕ್ಕ
ಮೃದು-ವಾಗಿ-ರು-ವೆವು
ಮೃದು-ವಾಗುತ್ತಾರೆ
ಮೃದು-ವಾದ
ಮೃದು-ವಾದರೂ
ಮೃದು-ಸ್ವರ-ದಲ್ಲಿ
ಮೆಚ್ಚ-ತಕ್ಕದ್ದೇ
ಮೆಚ್ಚಿ
ಮೆಚ್ಚಿ-ದ್ದಾರೆ
ಮೆಚ್ಚುಗೆ-ಯನ್ನು
ಮೆಚ್ಚುಗೆ-ಯಾಗು-ವಂತಹ
ಮೆಚ್ಚುಗೆಯೂ
ಮೆಚ್ಚು-ತ್ತೇನೆ
ಮೆಚ್ಚುತ್ತೇವೆ
ಮೆಚ್ಚುವ
ಮೆಚ್ಚು-ವಂತೆ
ಮೆಚ್ಚು-ವಿರಿ
ಮೆಚ್ಚುವು-ದಿ-ದ್ದರೆ
ಮೆಚ್ಚು-ವು-ದಿಲ್ಲ
ಮೆಚ್ಚು-ವು-ದಿಲ್ಲವೋ
ಮೆಟ್ಟ-ಲಾಗಿ
ಮೆಟ್ಟಲು
ಮೆಟ್ಟಲು-ಗಳನ್ನು
ಮೆಟ್ಟಲು-ಗಳಿವೆ
ಮೆಟ್ಟಲು-ಗಳು
ಮೆಟ್ಟಲೆಂದರೆ
ಮೆಟ್ಟಿ-ಕೊಂಡು
ಮೆಟ್ಟಿ-ಕೊಳ್ಳಲು
ಮೆಟ್ಟಿ-ಲಾಗಿ
ಮೆಟ್ಟಿಲಿನ
ಮೆಟ್ಟಿಲು
ಮೆರೆದಾಡಿ
ಮೆಲಕು
ಮೆಲುಕು
ಮೆಲು-ನುಡಿ
ಮೆಲ್ಲಗೆ
ಮೆಲ್ಲಮೆಲ್ಲನೆ
ಮೇಕೆ-ಯನ್ನು
ಮೇಕೆ-ಯಾಗು-ವುದು
ಮೇಘ-ಗರ್ಜ-ನೆ-ಯಿಂದ
ಮೇಜನ್ನು
ಮೇಜಿಗೆ
ಮೇಜಿನ
ಮೇಜು
ಮೇಜೇ
ಮೇಧಯಾ
ಮೇಧಾವಿ-ಗಳದ್ದು
ಮೇಧಾವಿ-ಗಳಲ್ಲಿ
ಮೇಧಾವಿ-ಗಳಿಗೆ
ಮೇಧಾವಿ-ಗಳಿ-ರು-ವರು
ಮೇಧಾವಿ-ಗಳು
ಮೇಧಾ-ವಿಗೆ
ಮೇಧಾ-ಶಕ್ತಿ-ಯಿಂದ
ಮೇರಿ-ಗಳಾಗಿ-ರು-ವರು
ಮೇರು-ದಂಡ
ಮೇರೆ
ಮೇರೆ-ಯನ್ನು
ಮೇರೆ-ಯಲ್ಲಿ
ಮೇಲಕ್ಕೆ
ಮೇಲಾಗಲಾರ
ಮೇಲಾಗಿ
ಮೇಲಾಗಿ-ದ್ದರೂ
ಮೇಲಾಗಿ-ರು-ವೆನು
ಮೇಲಾಗುತ್ತಿ-ತ್ತೆಂದು
ಮೇಲಾ-ಗು-ತ್ತಿ-ರಲಿಲ್ಲ
ಮೇಲಾ-ದುದು
ಮೇಲಿಂದ
ಮೇಲಿದೆ
ಮೇಲಿನ
ಮೇಲಿನ-ದನ್ನು
ಮೇಲಿನ-ವರ್ಣದ-ವರ
ಮೇಲಿನ-ಶ್ರದ್ಧೆ-ಯಿಂದ
ಮೇಲಿ-ರಲು
ಮೇಲಿ-ರುವ
ಮೇಲಿರು-ವುದು
ಮೇಲು
ಮೇಲು-ನೋ-ಟಕ್ಕೆ
ಮೇಲೂ
ಮೇಲೆ
ಮೇಲೆಂದಾ-ಗಲಿ
ಮೇಲೆಂದಾ-ಗಲೀ
ಮೇಲೆ-ತ್ತ-ಬೇಕು
ಮೇಲೆ-ತ್ತ-ಲಾರೆವು
ಮೇಲೆ-ತ್ತಲು
ಮೇಲೆತ್ತಿ
ಮೇಲೆ-ತ್ತಿ-ದರು
ಮೇಲೆ-ತ್ತುವ
ಮೇಲೆ-ದ್ದವು
ಮೇಲೆ-ದ್ದಿತು
ಮೇಲೆದ್ದು
ಮೇಲೆದ್ದೇ
ಮೇಲೆಯೂ
ಮೇಲೆಯೇ
ಮೇಲೆ-ರ-ಗು-ತ್ತಿದೆ
ಮೇಲೆಲ್ಲ
ಮೇಲೆಲ್ಲಾ
ಮೇಲೇ
ಮೇಲೇ-ಳ-ಬೇ-ಕಾದರೆ
ಮೇಲೇ-ಳ-ಲಾ-ರದು
ಮೇಲೇ-ಳು-ತ್ತಿದೆ
ಮೇಲೇ-ಳು-ವಾಗ
ಮೇಲೇ-ಳು-ವಿರಿ
ಮೇಲೇ-ಳು-ವುದೋ
ಮೇಲೊಂದ-ರಂತೆ
ಮೇಲೊಬ್ಬರು
ಮೇಲೋಗರ-ವಾಗಿದೆ
ಮೇಲ್ಮೆ
ಮೇಲ್ಮೆ-ಗೇರಿ
ಮೇಲ್ಮೆ-ಯನ್ನು
ಮೈಕೊಡವಿ
ಮೈಗೂಡಿಸಿ-ಕೊಂಡ
ಮೈಗೆ
ಮೈದೋರಿ-ದುವು
ಮೈದೋರಿ-ರುವ
ಮೈಮೇಲೆ
ಮೈಮೇಲೊಂದು
ಮೈಲನ್ನು
ಮೈಲಿ-ಗಳನ್ನು
ಮೈಲಿಗೆ-ಯಂತೆ
ಮೈಲು-ಗಳು
ಮೈಸೂರು
ಮೊಗದ
ಮೊಗ-ಲ್
ಮೊಟಕಿದ
ಮೊಟ್ಟ
ಮೊಟ್ಟ-ಮೊದಲಿಗೆ
ಮೊಟ್ಟ-ಮೊದಲು
ಮೊತ್ತ
ಮೊತ್ತಕ್ಕೆ
ಮೊತ್ತದ
ಮೊತ್ತ-ಮೊದಲು
ಮೊತ್ತ-ವಲ್ಲ
ಮೊದ-ಮೊದಲು
ಮೊದಲ
ಮೊದಲನೆ
ಮೊದಲ-ನೆಯ
ಮೊದಲ-ನೆ-ಯ-ದ-ರಲ್ಲಿ
ಮೊದಲ-ನೆ-ಯ-ದಾಗಿ
ಮೊದಲ-ನೆ-ಯದು
ಮೊದಲ-ನೆ-ಯದೇ
ಮೊದಲ-ನೆ-ಯ-ವ-ರೆಂಬು-ದನ್ನು
ಮೊದ-ಲಾಗಿ
ಮೊದಲಾ-ಗುವ
ಮೊದ-ಲಾದ
ಮೊದಲಾ-ದುದು
ಮೊದಲಾದು-ದೆಲ್ಲಾ
ಮೊದಲಾ-ದುವು-ಗಳ
ಮೊದ-ಲಾ-ಯಿತು
ಮೊದಲಾ-ಯಿ-ತೆಂದು
ಮೊದಲಿಗರಾ-ದುದು
ಮೊದಲಿನ
ಮೊದಲಿ-ನಿಂದ
ಮೊದಲಿ-ನಿಂದಲೂ
ಮೊದಲು
ಮೊದಲು-ಕಂಡು-ಹಿ-ಡಿದು-ಕೊಂಡಾಗಲೇ
ಮೊದಲೇ
ಮೊನ್ನೆ
ಮೊನ್ನೆ-ಯಷ್ಟೆ
ಮೊರೆಯಿಡುತ್ತಿರೋ
ಮೊರೊಡಾ-ಕನೇ
ಮೊಲಾ-ಕನೇ
ಮೊಲಾಕರಿ-ಗಿಂತ
ಮೊಲಾಕಿಗೆ
ಮೊಲಾ-ಕ್
ಮೊಲಾಕ್ಯಾವಾ
ಮೊಳಗಿ-ದರೆ
ಮೊಳೆಯಿ-ಸಿದ-ವ-ರಾರು
ಮೊಸಳೆ
ಮೋಕ್ಷ
ಮೋಕ್ಷ-ಕ್ಕಾಗಿ
ಮೋಕ್ಷ-ಕ್ಕಾಗಿಯೇ
ಮೋಕ್ಷಕ್ಕೆ
ಮೋಕ್ಷ-ವನ್ನು
ಮೋಕ್ಷ-ವಿದೆ
ಮೋಕ್ಷ-ವೆಂದರೆ
ಮೋಕ್ಷಾಕಾಂಕ್ಖೆಯ
ಮೋಚಿ-ಯನ್ನು
ಮೋಡ
ಮೋಡ-ಗಳಿಂದ
ಮೋಡದ
ಮೋಡ-ದಂತೆ
ಮೋಡವು
ಮೋಡ-ವೊಂದು
ಮೋರಿಯಾ
ಮೋಸ-ಪಡಿ-ಸಲು
ಮೋಸ-ಹೋ-ಗದೆ
ಮೋಸ-ಹೋ-ಗು-ತ್ತಾರೆ
ಮೋಹಕ-ತೆಗೆ
ಮೋಹನ
ಮೋಹಿನಿ
ಮೌಢ್ಯ
ಮೌಢ್ಯ-ಗಳು
ಮೌಢ್ಯದ
ಮೌಢ್ಯ-ವನ್ನು
ಮೌನ
ಮೌನ-ಗಳೇ
ಮೌನ-ವಾಗಿ
ಮೌನ-ವಾಗಿದ್ದೆ
ಮೌಲ್ಯ
ಮೌಲ್ಯ-ಗಳನ್ನೂ
ಮೌಲ್ಯ-ವನ್ನು
ಮ್ಮದ್
ಮ್ಯಾ-ಕ್ಸ್
ಮ್ಲೇಚ್ಛರ
ಮ್ಲೇಚ್ಛರಾ-ದರೂ
ಮ್ಲೇಚ್ಛರು
ಯಂತ್ರ
ಯಂತ್ರಕ್ಕೆ
ಯಂತ್ರ-ಗಳನ್ನು
ಯಂತ್ರ-ದಂತೆ
ಯಂತ್ರ-ದಿಂದ
ಯಂತ್ರ-ವನ್ನು
ಯಂತ್ರವು
ಯಂತ್ರವೂ
ಯಃ
ಯಜ-ಮಾನ
ಯಜ-ಮಾ-ನನ
ಯಜ-ಮಾ-ನ-ನಿಗೆ
ಯಜುರ್ವೇದ
ಯಜ್ಞ
ಯಜ್ಞಕ್ಕೂ
ಯಜ್ಞ-ಗಳನ್ನು
ಯಜ್ಞ-ಮಾಡಿ
ಯಜ್ಞ-ವನ್ನು
ಯಜ್ಞ-ವೇದಿಕೆ-ಗಳಿಗೆ
ಯಜ್ಞೋಪವೀತ-ವನ್ನು
ಯಣರು
ಯತಿಗೂ
ಯತಿ-ರಾಜ
ಯತಿವರ್ಯ-ರಾದ
ಯತಿ-ವರ್ಯರೇ
ಯತೋ
ಯತ್ನ
ಯತ್ನಿಸ-ಬೇಕು
ಯತ್ನಿಸ-ಬೇಡಿ
ಯತ್ನಿ-ಸ-ಲಿಲ್ಲವೇ
ಯತ್ನಿಸಿ
ಯತ್ನಿಸಿ-ದನು
ಯತ್ನಿಸಿ-ದರು
ಯತ್ನಿಸಿ-ದರೆ
ಯತ್ನಿಸಿ-ದಾಗ-ಲೆಲ್ಲಾ
ಯತ್ನಿಸಿ-ದ್ದರೆಂಬುದೂ
ಯತ್ನಿಸಿ-ರು-ವರು
ಯತ್ನಿಸಿ-ರುವೆ
ಯತ್ನಿ-ಸಿಲ್ಲ
ಯತ್ನಿಸುತ್ತಿ-ರು-ವರು
ಯತ್ನಿಸು-ತ್ತೇನೆ
ಯತ್ನಿ-ಸುವ
ಯತ್ನಿ-ಸು-ವನು
ಯತ್ನಿ-ಸು-ವರು
ಯತ್ನಿಸುವಿರೊ
ಯತ್ನಿಸು-ವುದು
ಯಥಾ
ಯಥಾಂಧಾಃ
ಯಥಾರ್ಥ
ಯಥೇಚ್ಛ-ವಾಗಿ-ಕೊಟ್ಟರೆ
ಯಥೇಚ್ಛ-ವಾಗಿ-ರು-ವು-ದ-ರಿಂದ
ಯಥೇಮಾಂ
ಯಥೇಷ್ಟ-ಮ್
ಯದಾ
ಯದಿ
ಯದಿ-ಚ್ಛತಿ
ಯದಿದಂ
ಯದ್ಯದ್ವಿ-ಭೂತಿ-ಮತ್
ಯನ್ನರ
ಯನ್ನಾಗಿ
ಯನ್ನು
ಯನ್ನೂ
ಯಮನ
ಯಮನೇ
ಯಮ-ರಾಜನೆ
ಯಮಾಯ
ಯರು
ಯಲ್ಲವೆ
ಯಲ್ಲಿ
ಯವನ
ಯವ-ನ-ದೇಶವು
ಯವ-ನರು
ಯವ-ರಲ್ಲ
ಯವ-ರೆಗೆ
ಯಶಸ್ವಿ-ಯಾಗಿ
ಯಶ-ಸ್ಸನ್ನು
ಯಶ-ಸ್ಸನ್ನೂ
ಯಶಸ್ಸಿ-ಗಾಗಿ
ಯಶ-ಸ್ಸಿಗೆ
ಯಶ-ಸ್ಸಿನ
ಯಶಸ್ಸಿನಿಂದಾಗಿ
ಯಶಸ್ಸು
ಯಶಸ್ಸು-ಇವು
ಯಸ್ತ್ವಾ-ತ್ಮರ-ತಿರೇವ
ಯಸ್ಯೈತೇ
ಯಹೂ-ದಿ-ಗಳು
ಯಹೂದ್ಯ
ಯಹೂ-ದ್ಯರ
ಯಹೂ-ದ್ಯ-ರಲ್ಲಿ
ಯಹೂ-ದ್ಯರು
ಯಹೊದ್ಯರು
ಯಹೋ-ವನೋ
ಯಾ
ಯಾಂತ್ರಿಕ
ಯಾಗ
ಯಾಗ-ಬಹುದು
ಯಾಗ-ಯಜ್ಞ-ಗಳ
ಯಾಗ-ಯಜ್ಞ-ಗಳಿಂದ
ಯಾಗ-ಯಜ್ಞ-ಗಳಿವೆ
ಯಾಗಲೀ
ಯಾಗ-ವನ್ನು
ಯಾಗಿ
ಯಾಗಿದೆ
ಯಾಗಿ-ದ್ದರೂ
ಯಾಗು-ವು-ದಿಲ್ಲ
ಯಾಜ್ಞವಲ್ಕ್ಯರು
ಯಾಜ್ಞವಲ್ಕ್ಯಾದಿ
ಯಾತ-ನೆ-ಯನ್ನು
ಯಾತಿ
ಯಾತ್
ಯಾತ್ರೆ
ಯಾತ್ರೆ-ಯನ್ನು
ಯಾತ್ರೆ-ಯಲ್ಲಿ
ಯಾದ
ಯಾದ-ವ-ಗಿರಿ
ಯಾದಾ-ಗಲೇ
ಯಾನ್ತಿ
ಯಾಯಿ-ತಷ್ಟೆ
ಯಾರ
ಯಾರದು
ಯಾರ-ನ್ನಾ-ದರೂ
ಯಾರನ್ನು
ಯಾರನ್ನೂ
ಯಾರಲ್ಲಿ
ಯಾರ-ಲ್ಲಿಯೂ
ಯಾರಾ-ದರೂ
ಯಾರಿಂದ
ಯಾರಿಂದ-ಲಾ-ದರೂ
ಯಾರಿಂದಲೂ
ಯಾರಿ-ಗಾಗಿ
ಯಾರಿಗಾ-ದರೂ
ಯಾರಿ-ಗಿದೆ
ಯಾರಿಗೂ
ಯಾರಿಗೆ
ಯಾರಿಗೋ-ಸ್ಕರ
ಯಾರಿಲ್ಲ-ದಿ-ದ್ದರೆ
ಯಾರು
ಯಾರೂ
ಯಾರೊಬ್ಬರೂ
ಯಾರೋ
ಯಾವ
ಯಾವನ
ಯಾವ-ನ-ಲ್ಲಾ-ದರೂ
ಯಾವ-ನಾ-ದರೂ
ಯಾವ-ನಾದ-ರೊಬ್ಬ
ಯಾವ-ನಿಂದ
ಯಾವನು
ಯಾವನೊ
ಯಾವ-ರೀತಿ
ಯಾವಾ
ಯಾವಾಗ
ಯಾವಾಗ-ಲಾ-ದರೂ
ಯಾವಾಗಲು
ಯಾವಾಗಲೂ
ಯಾವಾಗಲೋ
ಯಾವಾ-ತನು
ಯಾವಾನ
ಯಾವು
ಯಾವು-ದಕ್ಕೂ
ಯಾವು-ದಕ್ಕೆ
ಯಾವು-ದನ್ನಾ-ಗಲೀ
ಯಾವು-ದನ್ನಾ-ದರೂ
ಯಾವು-ದನ್ನು
ಯಾವು-ದನ್ನೂ
ಯಾವು-ದನ್ನೇ
ಯಾವು-ದರ
ಯಾವು-ದ-ರಲ್ಲಿ
ಯಾವು-ದರ-ಲ್ಲಿದೆ
ಯಾವು-ದ-ರಿಂದ
ಯಾವು-ದಾದರು
ಯಾವು-ದಾ-ದರೂ
ಯಾವು-ದಾದ-ರೊಂದು
ಯಾವು-ದಾದ-ರೊಂದೆಡೆ-ಯಲ್ಲಿ
ಯಾವು-ದಾದ-ರೊಬ್ಬ
ಯಾವು-ದಿದೆ
ಯಾವು-ದಿರು-ವುದು
ಯಾವುದು
ಯಾವು-ದುಂಟೋ
ಯಾವುದೂ
ಯಾವು-ದೆಂದರೆ
ಯಾವು-ದೆಂಬ
ಯಾವುದೇ
ಯಾವುದೊ
ಯಾವು-ದೊಂದು
ಯಾವು-ದೊಂದೂ
ಯಾವುದೋ
ಯಾವುವು
ಯಾವುವೂ
ಯಾವು-ವೆಂದರೆ
ಯಾಸ್ಕರ
ಯಿಂದ
ಯಿದ್ದೇ
ಯುಕ್ತ-ವಾಗಿದೆ
ಯುಕ್ತ-ವಾದುದು
ಯುಕ್ತಿ
ಯುಕ್ತಿಗೆ
ಯುಕ್ತಿ-ಪೂರಿತ
ಯುಕ್ತಿ-ಪೂರ್ಣ
ಯುಕ್ತಿ-ಪೂರ್ಣ-ವಾದ
ಯುಕ್ತಿ-ಮ-ತಿಗೆ
ಯುಕ್ತಿಯ
ಯುಕ್ತಿ-ಯನ್ನು
ಯುಕ್ತಿ-ಯಿಂದ
ಯುಕ್ತಿ-ಯಿಂದಾ-ಗಲಿ
ಯುಕ್ತಿಯು
ಯುಕ್ತಿ-ಯುಕ್ತ
ಯುಕ್ತಿ-ಯು-ಕ್ತ-ವಾಗಿ
ಯುಕ್ತಿ-ಯು-ಕ್ತ-ವಾಗಿದೆ
ಯುಕ್ತಿ-ಯು-ಕ್ತ-ವಾಗಿ-ರ-ಬಹುದು
ಯುಕ್ತಿ-ಯು-ಕ್ತ-ವಾಗಿವೆ
ಯುಕ್ತಿ-ವಾದ
ಯುಕ್ತಿ-ವಾದಿ-ಯಾಗುವ
ಯುಕ್ತಿ-ವಾದಿಯು
ಯುಗ
ಯುಗಕ್ಕೆ
ಯುಗದ
ಯುಗ-ದಲ್ಲಿ
ಯುಗ-ದಲ್ಲಿಯೂ
ಯುಗ-ಪುರು-ಷರ
ಯುಗ-ಯುಗ-ಗಳ
ಯುಗ-ವಾದ
ಯುಗವು
ಯುಗಾಂತರ-ಗಳ
ಯುಗಾಂತರ-ಗಳಿಂದಲೂ
ಯುಗಾಂತರೇ
ಯುಗಾ-ಚಾರ್ಯ-ಶ್ರೀ-ರಾಮ-ಕೃಷ್ಣ
ಯುಗಾವ-ತಾರ
ಯುಗೇ
ಯುದ್ಧ
ಯುದ್ಧಕ್ಕೆ
ಯುರೇಸಿ-ಯ-ನರು
ಯುರೋಪಿ
ಯುರೋ-ಪಿನ
ಯುರೋ-ಪಿ-ಯ-ನ್ನರು
ಯುರೋಪ್
ಯುವಕ
ಯುವ-ಕನ
ಯುವ-ಕ-ನಿಗೆ
ಯುವ-ಕನೂ
ಯುವ-ಕರ
ಯುವ-ಕ-ರನ್ನು
ಯುವ-ಕ-ರಲ್ಲಿ
ಯುವ-ಕ-ರಿಂದ
ಯುವ-ಕ-ರಿಗೂ
ಯುವ-ಕರು
ಯುವ-ಕರೇ
ಯುವ-ತಿಯೂ
ಯೂರೋಪನ್ನು
ಯೂರೋ-ಪಿಗೂ
ಯೂರೋ-ಪಿನ
ಯೂರೋ-ಪಿ-ನಲ್ಲಿ
ಯೂರೋ-ಪಿ-ನ-ಲ್ಲಿಲ್ಲ
ಯೂರೋ-ಪಿ-ಯನರ
ಯೂರೋ-ಪಿ-ಯನ-ರಿಗೂ
ಯೂರೋ-ಪಿ-ಯ-ನರು
ಯೂರೋ-ಪಿ-ಯ-ನ್ನರು
ಯೂರೋಪು
ಯೂರೋಪು-ಗಳಲ್ಲಿ
ಯೂರೋಪ್
ಯೆಂದೂ
ಯೆಹೂ-ದ್ಯ-ನೊಬ್ಬನು
ಯೆಹೂ-ದ್ಯರ
ಯೇಷಾಂ
ಯೊಂದಕ್ಕೂ
ಯೊಂದರೆ
ಯೊಬ್ಬರ
ಯೋ
ಯೋಗ
ಯೋಗಃ
ಯೋಗ-ವನ್ನಾ-ದರೂ
ಯೋಗ-ವನ್ನು
ಯೋಗಿ-ಗಳ
ಯೋಗಿ-ಗಳಾಗ-ಬೇಕಾಗಿ-ತ್ತು
ಯೋಗಿ-ಗಳಾದರೆ
ಯೋಗಿ-ಗಳು
ಯೋಗಿಯ
ಯೋಗಿ-ಯಾಗು
ಯೋಗಿ-ಯಾಗು-ವುದೇನೋ
ಯೋಗಿ-ಯಾ-ದ-ವನು
ಯೋಗ್ಯ
ಯೋಗ್ಯ-ತಮ
ಯೋಗ್ಯತೆ
ಯೋಗ್ಯ-ತೆಯ
ಯೋಗ್ಯ-ತೆ-ಯುಳ್ಳದ್ದೆ
ಯೋಗ್ಯ-ನ-ನ್ನಾಗಿ
ಯೋಗ್ಯ-ನಿದ್ದರೂ
ಯೋಗ್ಯನೇ
ಯೋಗ್ಯ-ರಲ್ಲ-ದ-ವ-ರಿಗೆ
ಯೋಗ್ಯ-ರಾಗು-ತ್ತೀರಿ
ಯೋಗ್ಯ-ರಾಗು-ವಿರಿ
ಯೋಗ್ಯ-ರಾ-ದರೂ
ಯೋಗ್ಯರು
ಯೋಗ್ಯ-ವಾಗಿಯೇ
ಯೋಗ್ಯ-ವಾದ
ಯೋಗ್ಯ-ವಾದವು
ಯೋಗ್ಯ-ವಾದುದು
ಯೋಗ್ಯ-ವಾದುದೇ
ಯೋಗ್ಯ-ವಾದು-ವಲ್ಲ
ಯೋಚನಾ-ಪರ-ನಿಗೆ
ಯೋಚನೆ
ಯೋಚಿಸ-ಬೇ-ಕಾದ
ಯೋಚಿಸಲಾರ
ಯೋಚಿಸಿ
ಯೋಚಿಸಿ-ದರೆ
ಯೋಚಿಸಿ-ದ್ದರು
ಯೋಚಿಸಿ-ದ್ದರೆ
ಯೋಚಿ-ಸು-ವಂತೆ
ಯೋಚಿ-ಸು-ವನು
ಯೋಚಿ-ಸು-ವರೋ
ಯೋಚಿಸು-ವೆನು
ಯೋಜನಾ-ಕ್ರಮ-ವನ್ನು
ಯೋಜನೆ
ಯೋಜ-ನೆ-ಗಳ
ಯೋಜ-ನೆ-ಗಳನ್ನು
ಯೋಜ-ನೆ-ಗಳ-ನ್ನೆಲ್ಲ
ಯೋಜ-ನೆ-ಗಳ-ನ್ನೆಲ್ಲಾ
ಯೋಜ-ನೆ-ಗಳಾವುವೂ
ಯೋಜ-ನೆ-ಗಳು
ಯೋಜ-ನೆಯ
ಯೋಜ-ನೆ-ಯಂತೆ
ಯೋಜ-ನೆ-ಯಾಗಲಿ
ಯೋಜ-ನೆ-ಯೇನೋ
ಯೋಧನ
ಯೋಧನಾಗ-ಬೇಕಾಗಿಲ್ಲ
ಯೋಧ-ರಾಗಿ-ರ-ಲಿಲ್ಲ
ರ
ರಂಗ-ಭೂಮಿಯ
ರಂಗ-ಭೂಮಿ-ಯಿಂದ
ರಂಜ-ನೆಗೆ
ರಂಜಿ-ಸುತ್ತದೆ
ರಂದು
ರಂಧ್ರ
ರಂಧ್ರದ
ರಂಧ್ರ-ವನ್ನು
ರಂಧ್ರ-ವಿದೆ
ರಂಧ್ರವು
ರಕ್ಕಸ
ರಕ್ತ
ರಕ್ತ-ಕಣ-ದಲ್ಲಿಯೂ
ರಕ್ತಕ್ಕೆ
ರಕ್ತ-ಗತ-ವಾಗಲಿ
ರಕ್ತದ
ರಕ್ತ-ದಲ್ಲಿ
ರಕ್ತ-ದಲ್ಲಿ-ತೋಯಿ-ಸು-ವರು
ರಕ್ತ-ದಲ್ಲಿ-ರ-ಬೇಕು
ರಕ್ತ-ದಲ್ಲಿ-ರುವ
ರಕ್ತ-ನಾಳ-ಗಳಲ್ಲಿ
ರಕ್ತ-ಪಾತ
ರಕ್ತ-ಪ್ರ-ವಾ-ಹ-ದಲ್ಲಿ
ರಕ್ತ-ಬೀಜ-ನಂತೆ
ರಕ್ತ-ವನ್ನಾ-ದರೂ
ರಕ್ತ-ವನ್ನು
ರಕ್ತ-ವಿದ್ದರೆ
ರಕ್ತವು
ರಕ್ಷಕ
ರಕ್ಷಕ-ನಾಗಿ-ರು-ವನು
ರಕ್ಷಕನು
ರಕ್ಷಣೆ
ರಕ್ಷಣೆ-ಗಾಗಿ
ರಕ್ಷಣೆಗೆ
ರಕ್ಷಣೆ-ಯನ್ನು
ರಕ್ಷಿಸ-ಬೇಕಾಗಿದೆ
ರಕ್ಷಿಸ-ಬೇಕು
ರಕ್ಷಿಸಲ್ಪಡು-ತ್ತದೆ
ರಕ್ಷಿಸಿ
ರಕ್ಷಿಸಿ-ಕೊಳ್ಳಬೇಕಾಗು-ವುದು
ರಕ್ಷಿಸಿ-ದುದ-ರಿಂದಲೇ
ರಕ್ಷಿ-ಸುತ್ತ
ರಕ್ಷಿಸು-ವುದು
ರಚನ-ವಾದ
ರಚನಾ
ರಚನಾ-ತ್ಮಕ
ರಚನೆ
ರಚ-ನೆಗೆ
ರಚ-ನೆ-ಯಿಲ್ಲ
ರಚಿತ-ವಾಗಿದೆ
ರಚಿತ-ವಾದ
ರಚಿಸ-ಬಹು-ದಾ-ಗಿತ್ತು
ರಚಿಸ-ಬೇಕಾಗಿಲ್ಲ
ರಚಿಸಿ
ರಚಿಸಿ-ದನು
ರಚಿಸಿ-ದರು
ರಚಿಸಿ-ರು-ವರು
ರಚಿ-ಸುವ
ರಚಿ-ಸು-ವನು
ರಜಸು
ರಜಸುಲ್ಲಾ
ರಜಸು-ಲ್ಲಾ-ನ-ನ್ನಾಗಿ
ರಜಸ್ಸು
ರಜೋ
ರಜೋ-ಗುಣ-ವಾಗಿ
ರಜೋ-ಗು-ಣವು
ರಜೋ-ಶಕ್ತಿ-ಯನ್ನು
ರಣ-ಕ-ಹ-ಳೆಯ
ರಣ-ಭೂಮಿ-ಯನ್ನು
ರಣ-ರಂಗ-ದಲ್ಲಿ
ರತ್ನ
ರತ್ನ-ಗಳ
ರತ್ನ-ಗಳನ್ನು
ರತ್ನ-ಗಳಿ-ಗಾಗಿ
ರತ್ನ-ಗಳು
ರತ್ನ-ದಂತಿ-ರುವ
ರತ್ನ-ಪ್ರಾ-ಯದ-ವಳಾ-ಗಿದ್ದರೆ
ರತ್ನ-ರಾಶಿ
ರತ್ನ-ರಾಶಿಗೆ
ರತ್ನ-ರಾಶಿಯ
ರತ್ನ-ರಾಶಿ-ಯನ್ನು
ರತ್ನ-ವನ್ನು
ರಥದ
ರಥಿಗೂ
ರದ್ದಾ-ಗು-ವು-ದಕ್ಕೆ
ರದ್ದಾದ
ರದ್ದು
ರನ್ನು
ರಬ್ಬರೇ
ರಭಸ-ದಿಂದ
ರಮರೇ
ರಮಿ-ಸು-ವರೋ
ರಲಾರ
ರಲ್ಲದೇ
ರಲ್ಲಿ
ರಲ್ಲೂ
ರವಾ-ನಿಸು-ವು-ದಲ್ಲ
ರಷ್ಯಾ
ರಷ್ಯಾದ
ರಸಾ-ನಂದ-ವಾ-ಹಿ-ನಿ-ಗಳು
ರಸಾಸ್ವಾದದ
ರಸ್ತೆ
ರಸ್ತೆ-ಗಳ
ರಸ್ತೆ-ಗಳು
ರಸ್ತೆಗೆ
ರಹಸ್ಯ
ರಹಸ್ಯ-ಕ್ಕೆಲ್ಲಾ
ರಹಸ್ಯ-ಗಳನ್ನು
ರಹಸ್ಯ-ಗಳಲ್ಲಿ
ರಹಸ್ಯ-ತಮ
ರಹಸ್ಯ-ವನ್ನು
ರಹಸ್ಯ-ವಾಗಿ
ರಹಸ್ಯ-ವಾಗಿ-ರ-ಬಾ-ರದು
ರಹಸ್ಯ-ವಾಗಿ-ರು-ವುವೋ
ರಹಸ್ಯ-ವಿದೆ
ರಹಸ್ಯ-ವಿದೆಯೇ
ರಹಸ್ಯ-ವಿದ್ಯೆ-ಯನ್ನು
ರಹಸ್ಯ-ವಿರು-ವುದು
ರಹಸ್ಯ-ವಿಲ್ಲ
ರಹಸ್ಯವು
ರಹಸ್ಯ-ವೆಂದಾ-ಗಲಿ
ರಹಸ್ಯ-ವೆಂದು
ರಹಸ್ಯ-ವೆಲ್ಲ
ರಹಸ್ಯ-ವೇ-ನಿಲ್ಲ
ರಹಸ್ಯ-ವೇ-ನೆಂದು
ರಹಸ್ಯ-ಸಂಸ್ಥೆ-ಗಳು
ರಹಸ್ಯಾ
ರಹಸ್ಯಾ-ತ್ಮಕವೂ
ರಹಿತ
ರಹಿ-ತ-ವಾದ
ರಹಿ-ತ-ವಾದುದು
ರಾಕ್ಷ-ಸರ
ರಾಕ್ಷಸ-ರಿಗೆ
ರಾಕ್ಷ-ಸರು
ರಾಕ್ಷಸೀ
ರಾಕ್ಷಸೀಯ
ರಾಗ
ರಾಗ-ದಂತೆ
ರಾಗ-ಬೇಕು
ರಾಗಲೇ
ರಾಗಿ
ರಾಗಿ-ದ್ದೆವು
ರಾಗುತ್ತಿ-ರು-ವರು
ರಾಗು-ತ್ತೀರಿ
ರಾಗು-ವುದು
ರಾಜ
ರಾಜ-ಕಾರಣ
ರಾಜ-ಕಾರ-ಣ-ವಾಗಿ-ರ-ಬಹುದು
ರಾಜ-ಕಾರ-ಣಿ-ಗಲ್ಲ
ರಾಜ-ಕೀಯ
ರಾಜ-ಕೀ-ಯಕ್ಕೆ
ರಾಜ-ಕೀಯದ
ರಾಜ-ಕೀಯ-ದಲ್ಲಿ
ರಾಜ-ಕೀಯ-ದಲ್ಲಿ-ರು-ವುದು
ರಾಜ-ಕೀಯ-ವನ್ನಾ-ಗಲೀ
ರಾಜ-ಕೀಯ-ವನ್ನು
ರಾಜ-ಕೀಯ-ವನ್ನೊ
ರಾಜ-ಕೀಯ-ವಲ್ಲದ
ರಾಜ-ಕೀಯ-ವಾಗಿ-ದ್ದರೆ
ರಾಜ-ಕೀಯ-ವಿದೆ
ರಾಜ-ಕೀಯವೂ
ರಾಜ-ಕೀಯವೇ
ರಾಜ-ಕು-ಮಾರ
ರಾಜ-ಧಾನಿಯು
ರಾಜನ
ರಾಜ-ನಾಗ-ಬೇಕೆಂದು
ರಾಜ-ನಾ-ಗಲು
ರಾಜ-ನಾ-ಗಿದ್ದ
ರಾಜ-ನಾಗು-ವುದು
ರಾಜ-ನಿಗೆ
ರಾಜ-ನೀತಿ
ರಾಜ-ನೀತಿಯ
ರಾಜ-ನೀತಿ-ಯಲ್ಲ
ರಾಜ-ಮ-ಹಾ-ರಾಜ-ರು-ಗಳೂ
ರಾಜರ
ರಾಜ-ರಲ್ಲ
ರಾಜ-ರಾಗಿ-ದ್ದ-ರೆಂದು
ರಾಜ-ರಾಗಿ-ರ-ಲಿಲ್ಲ
ರಾಜ-ರಾಗು-ವು-ದ-ಕ್ಕಿಂತ
ರಾಜ-ರಾದರೆ
ರಾಜ-ರಿಗೆ
ರಾಜ-ರಿ-ರಲಿ
ರಾಜ-ರಿಲ್ಲ
ರಾಜರು
ರಾಜ-ರು-ಗಳ
ರಾಜರೇ
ರಾಜ-ಸಿಂಹಾ-ಸ-ನ-ವನ್ನೂ
ರಾಜಾ
ರಾಜಾ-ಧಿ-ರಾಜ
ರಾಜಾ-ಧಿ-ರಾಜನ
ರಾಜಾ-ಧಿ-ರಾಜ-ನಾದ
ರಾಜಿ
ರಾಜಿಯ
ರಾಜಿ-ಯಾಗಿಲ್ಲ
ರಾಜೋ-ಚಿತ
ರಾಜ್ಯಕ್ಕೆ
ರಾಜ್ಯದ
ರಾಜ್ಯ-ದಲ್ಲಿ
ರಾಜ್ಯ-ವನ್ನಾಳ-ಬಹುದು
ರಾಜ್ಯ-ವನ್ನು
ರಾಜ್ಯ-ವಿದೆ
ರಾಣಿ
ರಾತ್ರಿ
ರಾತ್ರಿಯು
ರಾದೆವು
ರಾಧ-ನೆಯ
ರಾಧಾಕಾಂತ
ರಾಧಾ-ಕೃಷ್ಣರ
ರಾಧೆ
ರಾಮ
ರಾಮ-ಕೃಷ್ಣ
ರಾಮ-ಚಂದ್ರನ
ರಾಮ-ಚಂದ್ರನು
ರಾಮ-ನಂತಹ
ರಾಮ-ನಂತೆ
ರಾಮ-ನಾಡನ್ನು
ರಾಮ-ನಾಡಿನ
ರಾಮ-ನಾಡಿ-ನಲ್ಲಿ
ರಾಮ-ನಾಡೆಂದು
ರಾಮ-ನಾಥಪುರ
ರಾಮ-ನಿ-ರುವ
ರಾಮ-ನಿಲ್ಲ
ರಾಮನು
ರಾಮನೇ
ರಾಮ-ಮೋಹನ-ರಾಯ್
ರಾಮಾ
ರಾಮಾ-ನು-ಚಾರ್ಯ
ರಾಮಾ-ನುಜ
ರಾಮಾ-ನುಜರ
ರಾಮಾ-ನುಜ-ರಂತೆಯೇ
ರಾಮಾ-ನುಜ-ರನ್ನು
ರಾಮಾ-ನುಜ-ರಿಬ್ಬರೂ
ರಾಮಾ-ನು-ಜರು
ರಾಮಾ-ನು-ಜರೇ
ರಾಮಾ-ನುಜಾ-ಚಾರ್ಯರ
ರಾಮಾ-ನುಜಾ-ಚಾರ್ಯ-ರಿಗೆ
ರಾಮಾ-ನುಜಾ-ಚಾರ್ಯರು
ರಾಮೇಶ್ವರ
ರಾಮೇಶ್ವರಂ
ರಾಯ-ಭಾರಿ-ಗಳು
ರಾರಾಜಿ-ಸಲಿ
ರಾಳಕ್ಕೆ
ರಾಶಿ
ರಾಶಿ-ಗಳಿ-ಗಿಂತ
ರಾಶಿ-ಯ-ನ್ನಲ್ಲದೆ
ರಾಶಿ-ಯನ್ನು
ರಾಶಿ-ಯಲ್ಲಿ
ರಾಶಿ-ಯಾ-ಯಿ-ತೆಂದರೆ
ರಾಶಿ-ಯಿದೆ
ರಾಶಿ-ರಾಶಿ
ರಾಷ್ಟ್ರ
ರಾಷ್ಟ್ರಕ್ಕೂ
ರಾಷ್ಟ್ರ-ಗಳ
ರಾಷ್ಟ್ರ-ಗಳಂತೆ
ರಾಷ್ಟ್ರ-ಗಳಂತೆಯೇ
ರಾಷ್ಟ್ರ-ಗಳನ್ನು
ರಾಷ್ಟ್ರ-ಗಳಲ್ಲಿ
ರಾಷ್ಟ್ರ-ಗಳಾದ
ರಾಷ್ಟ್ರ-ಗಳಿಂದ
ರಾಷ್ಟ್ರ-ಗಳಿಗೂ
ರಾಷ್ಟ್ರ-ಗಳಿಗೆ
ರಾಷ್ಟ್ರ-ಗಳು
ರಾಷ್ಟ್ರ-ಗಳುಈ
ರಾಷ್ಟ್ರ-ಗಳೆದ್ದು
ರಾಷ್ಟ್ರ-ಗಳೆಲ್ಲ
ರಾಷ್ಟ್ರ-ಗಳೊಂದಿಗೆ
ರಾಷ್ಟ್ರ-ಜೀವ-ನದ
ರಾಷ್ಟ್ರ-ಜೀವ-ನ-ದಲ್ಲಿ
ರಾಷ್ಟ್ರ-ಜೀ-ವಾಳ
ರಾಷ್ಟ್ರದ
ರಾಷ್ಟ್ರ-ದ-ಜೀವ-ನದ
ರಾಷ್ಟ್ರ-ದ-ವರೂ
ರಾಷ್ಟ್ರ-ಧ್ಯೇ-ಯಕ್ಕೆ
ರಾಷ್ಟ್ರ-ನೌ-ಕೆಯು
ರಾಷ್ಟ್ರ-ವನ್ನು
ರಾಷ್ಟ್ರ-ವನ್ನೂ
ರಾಷ್ಟ್ರ-ವಾಗಿ
ರಾಷ್ಟ್ರ-ವಾಗಿ-ತ್ತು
ರಾಷ್ಟ್ರ-ವಾಗಿದೆ
ರಾಷ್ಟ್ರ-ವಾಗು-ವುದು
ರಾಷ್ಟ್ರವು
ರಾಷ್ಟ್ರವೂ
ರಾಷ್ಟ್ರ-ವೆಂಬ
ರಾಷ್ಟ್ರವೇ
ರಾಷ್ಟ್ರ-ಸೌಧ
ರಾಷ್ಟ್ರೀಯ
ರಾಸ-ಲೀಲೆ-ಯಲ್ಲಿ
ರಿಂದಲೇ
ರಿಂದೊಂದು
ರಿಗೆ
ರಿಗೋ-ಸ್ಕರ-ವಾಗಿ
ರಿದ್ದರು
ರಿಪಬ್ಲಿ-ಕನ್
ರಿಪಬ್ಲಿಕ್ಕಿನ
ರಿಪೇರಿ
ರೀತಿ
ರೀತಿ-ಗಳಲ್ಲಿ
ರೀತಿ-ಗಳಿಂದ
ರೀತಿ-ಗಳು
ರೀತಿ-ಗಳೆಲ್ಲ
ರೀತಿ-ಗಿಂತ
ರೀತಿ-ನೀತಿ-ಗಳೆಲ್ಲ
ರೀತಿಯ
ರೀತಿ-ಯಂತೆಯೇ
ರೀತಿ-ಯನ್ನು
ರೀತಿ-ಯನ್ನೂ
ರೀತಿ-ಯಲ್ಲಿ
ರೀತಿ-ಯಿಂದ
ರೀತಿ-ಯಿಂದ-ಲಾ-ದರೂ
ರೀತಿ-ಯಿಲ್ಲ
ರೀತಿಯು
ರೀತಿಯೂ
ರೀತಿಯೆ
ರೀತಿಯೇ
ರುಚೀನಾಂ
ರುತ್ತ-ವೆಯೋ
ರುದ್ರನ
ರುದ್ರನಾಟಕ
ರುದ್ರನಾಟಕ-ಗಳನ್ನು
ರುವ
ರುವನು
ರುವನೋ
ರೂಢಿಗೆ
ರೂಢಿ-ಯಲ್ಲಿ-ತ್ತು
ರೂಢಿ-ಯ-ಲ್ಲಿದೆ
ರೂಢಿ-ಯಲ್ಲಿ-ದ್ದವು
ರೂಢಿ-ಯಲ್ಲಿ-ರುವ
ರೂಢಿ-ಯಾಗಿ-ದ್ದುದು
ರೂಢಿ-ಯಾಗಿ-ಹೋ-ಗಿದೆ
ರೂಢಿಸಿಕೊಳ್ಳ-ಬೇಕು
ರೂಪ
ರೂಪಕ
ರೂಪ-ಕ-ಗಳನ್ನು
ರೂಪ-ಕದ
ರೂಪ-ಕ-ವಾಗು-ವುದು
ರೂಪ-ಕೊಟ್ಟಂತೆ
ರೂಪ-ಕ್ಕಿಳಿಸು-ವುದು
ರೂಪಕ್ಕೆ
ರೂಪ-ಗಳನ್ನು
ರೂಪ-ಗಳಲ್ಲಿ
ರೂಪ-ಗಳಿಂದ
ರೂಪ-ಗಳೂ
ರೂಪ-ಗಳೆಲ್ಲ
ರೂಪದ
ರೂಪ-ದಲ್ಲಿ
ರೂಪ-ದಿಂದ
ರೂಪ-ರೇಖೆ
ರೂಪ-ವನ್ನು
ರೂಪ-ವಾಗಿ-ತ್ತು
ರೂಪ-ವಾಗಿ-ರು-ವು-ದ-ರಿಂದ
ರೂಪ-ವಾದ
ರೂಪವು
ರೂಪವೇ
ರೂಪಾಂತರ
ರೂಪಾಂಶ-ದಲ್ಲಿ
ರೂಪಾತ್ಮ-ಕ-ನೆಂದು
ರೂಪಾಯಿ-ಗಳನ್ನು
ರೂಪಿ-ನಲ್ಲಿ
ರೂಪಿ-ಸಿ-ದರೂ
ರೂಪು-ಗೊಳಿಸಿ-ಕೊಂಡ
ರೂಪೇಣ
ರೆಂಬೆ-ಗಳನ್ನು
ರೆಂಬೆ-ಯಲ್ಲಿ
ರೆಕ್ಕೆಯ
ರೇಣುವಿ-ನ-ಲ್ಲಿಯೂ
ರೈತನ
ರೈತನು
ರೈತರ
ರೈತ-ರನ್ನು
ರೈತ-ರಿಗೂ
ರೈತರು
ರೈಲಿ-ನಲ್ಲಿ
ರೊಟ್ಟಿ
ರೊಡನೆ
ರೋಗಕ್ಕೆ
ರೋಗದ
ರೋಗ-ವನ್ನು
ರೋಗಾಣು-ಗಳು
ರೋಗಿ-ಗಳಲ್ಲಿ
ರೋಗಿ-ಗಳಿಗೆ
ರೋಗಿ-ಯಂತೆ
ರೋಮನ್
ರೋಮ-ನ್ನರ
ರೋಮ-ನ್ನರು
ರೋಮಾಂಚ-ಕಾರಿ-ಯಾದ
ರೋಮ್
ರ್ಯಾ-ಡಿ-ಕ-ಲ್
ಲಂಕಾ-ದ್ವೀ-ಪದ
ಲಂಡನ್ನಿಗೆ
ಲಂಡನ್ನಿ-ನಲ್ಲಿ
ಲಕ್ಷ
ಲಕ್ಷಣ
ಲಕ್ಷಣ-ಗಳ
ಲಕ್ಷಣ-ಗಳನ್ನು
ಲಕ್ಷಣ-ಗಳು
ಲಕ್ಷಣ-ವಾಗಿದೆ
ಲಕ್ಷಣ-ವಾದ
ಲಕ್ಷಣ-ವುಳ್ಳದ್ದು
ಲಕ್ಷಣ-ವೆಂದು
ಲಕ್ಷಣವೇ
ಲಕ್ಷದ
ಲಕ್ಷಾಂತರ
ಲಕ್ಷಿಸಿ
ಲಕ್ಷಿಸಿ-ದರು
ಲಕ್ಷೋಪ-ಲಕ್ಷ
ಲಕ್ಷ್ಮಿ
ಲಕ್ಷ್ಮೀಃ
ಲಕ್ಷ್ಯ
ಲಕ್ಷ್ಯ-ದಲ್ಲಿ-ಟ್ಟಿ-ರು-ವರು
ಲಕ್ಷ್ಯ-ದಲ್ಲಿ-ಟ್ಟು-ಕೊಂಡು
ಲಕ್ಷ್ಯ-ದಲ್ಲಿ-ಡ-ಬೇ-ಕಾ-ದುದು
ಲಕ್ಷ್ಯ-ದಲ್ಲಿಡಿ
ಲಕ್ಷ್ಯ-ವನ್ನು
ಲಕ್ಷ್ಯ-ವಾದ
ಲಕ್ಷ್ಯವೂ
ಲಘಿಮಾ
ಲಘು-ವಾಗಿ
ಲಜ್ಜೆ
ಲಭಿ-ಸದು
ಲಭಿ-ಸಲಿ
ಲಭಿಸ-ಲೆಂದು
ಲಭಿಸಿತೆಂದೂ
ಲಭಿ-ಸಿದ
ಲಭಿ-ಸಿಯೇ
ಲಭಿಸಿ-ರುವ
ಲಭಿ-ಸುತ್ತದೆ
ಲಭಿಸುವು-ದ-ರಲ್ಲಿ
ಲಭಿಸು-ವು-ದ-ರಿಂದ
ಲಭಿ-ಸು-ವು-ದಿಲ್ಲ
ಲಭಿಸು-ವುದು
ಲಭಿ-ಸು-ವುದೆ
ಲಭಿಸು-ವು-ದೆಂದರೆ
ಲಭಿಸು-ವುವು
ಲಭ್ಯ-ವಾಗಿದೆ
ಲಭ್ಯ-ವಾಗಿ-ದ್ದರೂ
ಲಭ್ಯ-ವಾಗಿ-ದ್ದಾರೆ
ಲಭ್ಯ-ವಾಗು-ತ್ತದೆ
ಲಭ್ಯ-ವಾಗುವ-ವ-ರೆಗೆ
ಲಭ್ಯ-ವಾದದ್ದು
ಲಭ್ಯ-ವಾದಲ್ಲಿ
ಲಭ್ಯೋ
ಲಯ-ಕರ್ತ-ನಾದ
ಲಯ-ವಾಗು-ತ್ತದೆ
ಲವಲೇಶವೂ
ಲಾಗಿರು-ವು-ದೆಂದೂ
ಲಾಡಿಸು-ವುದು
ಲಾಭ
ಲಾಭ-ಕ್ಕಾಗಿ
ಲಾಭದ
ಲಾಭ-ವಾಗು-ತ್ತದೆ
ಲಾಮಾ-ಗಳ
ಲಾರದ-ವನು
ಲಾರದು
ಲಾರನು
ಲಾರ್ಡ್
ಲಾಹೋ-ರಿನ
ಲಾಹೋ-ರಿ-ನಲ್ಲಿ
ಲಾಹೋರ್
ಲಿಂಗ
ಲಿಂಗ-ಗಳಿ-ಗಾ-ಗಲಿ
ಲಿಂಗ-ದಲ್ಲಿ
ಲಿಂಗ-ವಾ-ಚಕ-ವಾಗು-ತ್ತದೆ
ಲಿಂಗವೂ
ಲಿಂಗಾ-ತೀತ-ವಾದುದು
ಲಿಪಿ-ಬದ್ಧ
ಲಿಪಿ-ಯಲ್ಲಿ
ಲೀನ-ವಾಗಲು
ಲೀನ-ವಾಗುತ್ತವೆ
ಲೀನ-ವಾಗು-ವು-ದ-ರಿಂದಲೇ
ಲೀಲಾ
ಲೀಲಾ-ಕ್ಷೇತ್ರ-ವಾ-ಯಿತು
ಲೀಲೆ
ಲೀಲೆಯ
ಲೀಲೆ-ಯನ್ನು
ಲೀಲೆ-ಯಲ್ಲಿ
ಲುಪ್ತ
ಲುಪ್ತ-ವಾಗಿ
ಲುಪ್ತ-ವಾಗುತ್ತವೆ
ಲೆಕ್ಕಿ-ಸದೆ
ಲೆಕ್ಕಿ-ಸ-ಲಿಲ್ಲ
ಲೆಕ್ಕಿ-ಸಿದರು
ಲೆಕ್ಕಿ-ಸು-ವು-ದಿಲ್ಲ
ಲೇಖನ-ಗಳನ್ನು
ಲೇಶಾಂಶವೂ
ಲೇಸಾ-ಗಲೀ
ಲೇಸು
ಲೈಂಗಿಕ
ಲೋಕ
ಲೋಕಕ್ಕೆ
ಲೋಕ-ಗೌರವಕ್ಕೆ
ಲೋಕ-ದಲ್ಲಿ
ಲೋಕ-ವನ್ನೂ
ಲೋಕ-ವನ್ನೆಲ್ಲಾ
ಲೋಕವು
ಲೋಕಾ-ಚಾರ
ಲೋಕಾ-ಚಾರ-ಗಳಿವೆ
ಲೋಕಾ-ಚಾರ-ಗಳು
ಲೋಕಾ-ಚಾರ-ವನ್ನೇ
ಲೋಕಾ-ಚಾರ-ವಾಗಿವೆ
ಲೋಕಾಚಾ-ರವು
ಲೋಕಾ-ನು-ಭವವು
ಲೋಪ
ಲೋಪ-ದೋಷ-ಗಳನ್ನು
ಲೋಪ-ದೋಷ-ಗಳಿವೆ
ಲೋಪ-ಬರು-ವು-ದಿಲ್ಲ
ಲೋಭ-ದಿಂದ
ಲೌಕಿಕ
ಲೌಕಿಕ-ದಲ್ಲಿ
ಲೌಕಿಕ-ದಲ್ಲಿಯೂ
ಲ್ಯಾ-ಟಿ-ನ್
ಲ್ಲಾನೋ
ವಂಗ
ವಂಗ-ದೇಶದ
ವಂಗ-ದೇಶ-ದಲ್ಲಿ
ವಂಗ-ದೇಶ-ದ-ವ-ರಿಗೆ
ವಂಗ-ದೇಶ-ದಿಂದ
ವಂಗ-ಭೂಮಿ-ಯಲ್ಲೇ
ವಂಗ-ಯುವ-ಕರ
ವಂಗ-ಯುವ-ಕ-ರಿಂದ
ವಂಗೀ-ಯರ
ವಂಗೀಯ-ರಿಗೆ
ವಂಚಸಿ
ವಂಚಿ-ಸಲು
ವಂತನ
ವಂತ-ನಿ-ರು-ವನು
ವಂತ-ರಾಗಿ-ದ್ದರೂ
ವಂತಹ
ವಂತ-ಹವ
ವಂತ-ಹುದು
ವಂತಿತ್ತು
ವಂದ-ನೆ-ಗಲ್ಲ
ವಂದ-ನೆ-ಗಳನ್ನು
ವಂದ-ನೆ-ಗಳು
ವಂಶ
ವಂಶಕ್ಕೆ
ವಂಶ-ಗಳ
ವಂಶ-ಜ-ರನ್ನು
ವಂಶ-ಜರೇ
ವಂಶದ
ವಂಶ-ದಲ್ಲಿ
ವಂಶ-ದ-ವರು
ವಂಶ-ಪಾರಂಪರ್ಯ-ವಾಗಿ
ವಂಶ-ಸ್ಥ-ರಾದ
ವಂಶ-ಸ್ಥರು
ವಂಶೀಯ-ರೆಂದು
ವಂಶೋದ್ಭ-ವರೊ-ಬ್ಬರು
ವಕೀಲ-ನಾಗು-ವನು
ವಕೀ-ಲನು
ವಕೀಲ-ರಾಗು-ವು-ದ-ಕ್ಕಿಂತ
ವಕ್ರ-ವಾಗಿ
ವಕ್ರ-ವಾಗಿಯೋ
ವಕ್ರ-ವ್ಯಾ-ಖ್ಯಾ-ನ-ಗಳನ್ನು
ವಕ್ಷ-ಸ್ಥಳ-ವನ್ನು
ವಚನ
ವಚನ-ಗಳನ್ನು
ವಜ್ರ
ವಜ್ರ-ಗಳಿವೆ
ವಜ್ರ-ಮುದ್ಯ-ತಮ್
ವಜ್ರ-ಸ-ಮಾನ
ವಟ-ವೃಕ್ಷ-ದಲ್ಲಿ
ವಟಿಕೆಗೆ
ವತ್
ವದಂತಿ
ವದಂತಿ-ಇರು-ವ-ವನೊ-ಬ್ಬನೇ
ವದಂತಿ-ಸತ್ಯ
ವದನಾರವಿಂದ
ವಧಾ
ವಧಿ-ಯನ್ನು
ವಧೂವ-ರರ
ವನೊ
ವನ್ನು
ವನ್ನೆ-ತ್ತಿತು
ವನ್ನೇರಿ
ವಯಸ್ಸಾ-ಗು-ವು-ದಿಲ್ಲ
ವಯಸ್ಸಾ-ದಂತೆಲ್ಲಾ
ವಯಸ್ಸಿ-ನ-ಲ್ಲಿಯೇ
ವಯಸ್ಸೂ
ವರ-ಣ-ಭೇದ-ಸ್ತು
ವರ-ದಿಯ
ವರ-ದಿ-ಯನ್ನು
ವರನ್ನು
ವರಲ್ಲಿ
ವರಾಗಿ-ರುತ್ತಾರೆ
ವರಾನುಗ್ರಹ
ವರಾನ್ನಿ-ಬೋಧತ
ವರಿಯ-ಬಹುದು
ವರಿ-ಯಲು
ವರಿ-ಸಲು
ವರು
ವರುಣ
ವರು-ಣ-ನಿಗೂ
ವರುಷ
ವರು-ಷ-ಗಳ
ವರು-ಷ-ಗಳ-ವ-ರೆಗೆ
ವರು-ಷ-ಗಳ-ವೆ-ರೆಗೆ
ವರು-ಷ-ಗಳಿಂದ
ವರು-ಷ-ಗಳಿಂದಲೂ
ವರು-ಷ-ಗಳು
ವರು-ಷದ
ವರು-ಷ-ದೊಳಗೆ
ವರು-ಷವೂ
ವರೆಗೂ
ವರೆಗೆ
ವರ್ಗಕ್ಕೆ
ವರ್ಗ-ಗಳಿಗೆ-ಲ್ಲ-ಕ್ಕಿಂತ
ವರ್ಗದ
ವರ್ಗದಲ್ಲಿ
ವರ್ಗದಲ್ಲಿ-ರುವ-ವರಿ-ಗಿಂತ
ವರ್ಗದಲ್ಲಿ-ರುವ-ವರು
ವರ್ಗದ-ವ-ರನ್ನು
ವರ್ಗದ-ವ-ರಿಗೆ
ವರ್ಜಿ-ಸಿದ
ವರ್ಜಿ-ಸಿದರೆ
ವರ್ಣ-ಗಳ
ವರ್ಣ-ಗಳಲ್ಲಿ
ವರ್ಣ-ಗಳಾದ-ರೆಂದೂ
ವರ್ಣ-ಗಳಿಗೆ
ವರ್ಣ-ಗಳು
ವರ್ಣದ
ವರ್ಣದಲ್ಲಿ-ರುವ
ವರ್ಣದ-ವರ
ವರ್ಣದ-ವ-ರನ್ನು
ವರ್ಣದ-ವ-ರನ್ನೂ
ವರ್ಣದ-ವ-ರಿಂದ
ವರ್ಣದ-ವರು
ವರ್ಣದ-ವ-ರೊಡನೆ
ವರ್ಣನೆ
ವರ್ಣ-ನೆ-ಗಳನ್ನು
ವರ್ಣ-ನೆ-ಗಳಿವೆ
ವರ್ಣ-ನೆ-ಗಳು
ವರ್ಣ-ನೆಯ
ವರ್ಣ-ನೆ-ಯನ್ನು
ವರ್ಣ-ನೆ-ಯ-ನ್ನೇ
ವರ್ಣ-ನೆ-ಯಲ್ಲಿ
ವರ್ಣ-ವನ್ನು
ವರ್ಣವು
ವರ್ಣ-ವ್ಯವ-ಸ್ಥೆ-ಎಂದರೆ
ವರ್ಣ-ಸಮಸ್ಯೆ
ವರ್ಣ-ಸಮಸ್ಯೆಗೆ
ವರ್ಣಾ-ಶ್ರಮ
ವರ್ಣಾ-ಶ್ರಮ-ಗಳನ್ನು
ವರ್ಣಾ-ಶ್ರಮದ
ವರ್ಣಾ-ಶ್ರಮ-ವನ್ನು
ವರ್ಣಿತ-ವಾಗಿದೆ
ವರ್ಣಿ-ಸಲು
ವರ್ಣಿ-ಸಿದನೋ
ವರ್ಣಿ-ಸಿ-ದ್ದಾರೆ
ವರ್ಣಿಸಿ-ರುವ
ವರ್ಣಿಸಿ-ರುವನೋ
ವರ್ಣಿಸಿರು-ವು-ದನ್ನು
ವರ್ಣಿ-ಸುತ್ತದೆ
ವರ್ತಕನಾಗ-ಬೇಕಾಗಿಲ್ಲ
ವರ್ತ-ಕರ
ವರ್ತ-ಕರು
ವರ್ತ-ನೆ-ಯಾಗು-ತ್ತಿದೆ
ವರ್ತ-ನೆಯೂ
ವರ್ತ-ಮಾನ
ವರ್ತ-ಮಾನಃ
ವರ್ತ-ಮಾ-ನ-ಕಾಲಕ್ಕೆ
ವರ್ತ-ಮಾ-ನ-ಕಾಲದ
ವರ್ತ-ಮಾ-ನದ
ವರ್ತ-ಮಾ-ನವು
ವರ್ತಿಸ-ಬೇಕು
ವರ್ತಿ-ಸಲಿ
ವರ್ತಿ-ಸಲು
ವರ್ತಿಸುತ್ತಿ-ರುವ
ವರ್ತಿಸುತ್ತಿ-ರು-ವರು
ವರ್ತಿಸು-ತ್ತಿ-ರು-ವುದೇ
ವರ್ತಿ-ಸುವ
ವರ್ಧಿ-ಸುವ
ವರ್ಯರೇ
ವರ್ಷ
ವರ್ಷಕ್ಕೆ
ವರ್ಷ-ಗಳ
ವರ್ಷ-ಗಳಲ್ಲಿ
ವರ್ಷ-ಗಳ-ವ-ರೆಗೆ
ವರ್ಷ-ಗಳಾದರೂ
ವರ್ಷ-ಗಳಿಂದ
ವರ್ಷ-ಗಳಿಂದಲೂ
ವರ್ಷ-ಗಳಿಗೂ
ವರ್ಷ-ಗಳು
ವರ್ಷ-ಗಳೇ
ವರ್ಷದ
ವರ್ಷ-ದಿಂದ
ವರ್ಷ-ದಿಂದಲೂ
ವರ್ಷ-ದೊಳಗೆ
ವರ್ಷವೂ
ವರ್ಷಾ-ಕಾಲ-ದಲ್ಲಿ
ವಲಯ-ಗಳಲ್ಲಿ
ವಲ್ಲಭಾ
ವಲ್ಲ-ಭಾ-ಚಾರ್ಯರ
ವಳಿ-ಗಳ
ವವನೆ
ವವ-ರೆಲ್ಲಾ
ವಶ-ದಲ್ಲಿ-ರ-ಬೇಕು
ವಶ-ನಾಗಿ-ರು-ವು-ದಿಲ್ಲ
ವಶಾ-ತ್
ವಶೀ-ಕರಣ
ವಶೀ-ಕಾರಕ
ವಶ್ಯಕ-ವೆಂದು
ವಷ್ಟೆ
ವಸಂತ
ವಸಂತ-ದಂತೆ
ವಸತಿ-ಗೃಹ-ಗಳಿಗೆ
ವಸನ
ವಸಿಷ್ಠ
ವಸ್ತು
ವಸ್ತು-ಗಳ
ವಸ್ತು-ಗಳನ್ನು
ವಸ್ತು-ಗಳನ್ನೂ
ವಸ್ತು-ಗಳಲ್ಲಿ
ವಸ್ತು-ಗಳಾಗಿ-ರ-ಲಿಲ್ಲ
ವಸ್ತು-ಗಳಿಂದ
ವಸ್ತು-ಗಳಿಗೂ
ವಸ್ತು-ಗಳಿಗೆ
ವಸ್ತು-ಗಳು
ವಸ್ತು-ಜ್ಞಾನಕ್ಕೆ
ವಸ್ತು-ವನ್ನು
ವಸ್ತು-ವಾಗ-ಬಲ್ಲನು
ವಸ್ತು-ವಾಗ-ಬೇಕು
ವಸ್ತು-ವಾಗು-ವುದು
ವಸ್ತು-ವಾ-ವುದು
ವಸ್ತು-ವಿ-ಗಾಗಿ
ವಸ್ತು-ವಿದೆ
ವಸ್ತು-ವಿನ
ವಸ್ತು-ವಿ-ನಲ್ಲಿ
ವಸ್ತು-ವಿ-ನಿಂದ
ವಸ್ತು-ವಿರ-ಬೇಕು
ವಸ್ತುವು
ವಸ್ತುವೂ
ವಸ್ತು-ವೆಂದು
ವಸ್ತು-ವೊಂದನ್ನು
ವಸ್ತು-ವೊಂದು
ವಸ್ತುವೋ
ವಸ್ತು-ಸತ್ತಾ-ವಾದಿಯು
ವಸ್ತು-ಸಾಗರ-ದಲ್ಲಿ
ವಸ್ತು-ಸ್ಥಿತಿ-ಯನ್ನು
ವಸ್ತು-ಸ್ಥಿತಿಯು
ವಸ್ತ್ರ-ಅನ್ನ-ಪಾನಾದಿ-ಗಳು
ವಹಿ-ಸ-ಬೇಕಾಗಿದೆ
ವಹಿ-ಸ-ಬೇ-ಕಾದ
ವಹಿ-ಸ-ಬೇಕೆಂದು
ವಹಿ-ಸಲು
ವಹಿಸಿ
ವಹಿ-ಸಿದ
ವಹಿ-ಸಿದ್ದ
ವಹಿ-ಸಿ-ದ್ದರು
ವಹಿ-ಸಿವೆ
ವಹಿ-ಸುತ್ತವೆ
ವಹಿ-ಸು-ವುದು
ವಾ
ವಾಕ್ಕಾಯ-ವಾಗಿ
ವಾಕ್ಕು-ಗಳು
ವಾಕ್ಯ
ವಾಕ್ಯ-ಗಳನ್ನು
ವಾಕ್ಯ-ಗಳಿವೆ
ವಾಕ್ಯ-ವನ್ನೇ
ವಾಕ್ಯ-ವಿದೆ
ವಾಗ್
ವಾಗ್ಗ-ಚ್ಛತಿ
ವಾಗ್ಮಿತೆ-ಯಿಂದ
ವಾಗ್ವೈ-ಖರೀ
ವಾಙ್ಮಂತ್ರ-ದಿಂದ
ವಾಙ್ಮಂತ್ರ-ವಾಗಿ
ವಾಚಂ
ವಾಚ-ನವು
ವಾಚೋ
ವಾಡಿ-ಕೆ-ಯಾಗಿದೆ
ವಾಣಿ
ವಾಣಿ-ಗಳೇ
ವಾಣಿಜ್ಯ
ವಾಣಿಯ
ವಾಣಿ-ಯನ್ನು
ವಾಣಿ-ಯಲ್ಲಿ
ವಾಣಿಯು
ವಾಣಿ-ಯೊಂದು
ವಾತಂ
ವಾತರೋಗ-ದಂತೆ
ವಾತಾ-ವರಣ
ವಾತಾ-ವರ-ಣ-ಗಳಲ್ಲಿ್ಧ-ಬೆಳೆದು-ದ-ರಿಂದ
ವಾತಾ-ವರ-ಣದ
ವಾತಾ-ವರ-ಣ-ದಲ್ಲಿ
ವಾತಾ-ವರ-ಣ-ದಲ್ಲಿ-ದ್ದರೆ
ವಾತಾ-ವರ-ಣ-ವನ್ನು
ವಾತಾ-ವರ-ಣವು
ವಾತ್ಸಲ್ಯ
ವಾತ್ಸಲ್ಯದ
ವಾತ್ಸಾ್ಯ-ಯನನು
ವಾತ್ಸ್ಯಾ-ಯನ
ವಾದ
ವಾದಂತೆಯೇ
ವಾದಕ್ಕೆ
ವಾದ-ಗಳನ್ನು
ವಾದ-ಗಳೂ
ವಾದದ
ವಾದ-ದಂತೆ
ವಾದ-ದಿಂದ
ವಾದ-ಬಾ-ಣ-ಗಳಿಗೆಲ್ಲ
ವಾದ-ಮಾಡ-ಬೇಡಿ
ವಾದ-ಮಾಡುತ್ತ
ವಾದ-ವನ್ನು
ವಾದವು
ವಾದ-ಸರಣಿ
ವಾದಾ-ಗಲೋ
ವಾದಿ-ಗಳು
ವಾದಿ-ಸುತ್ತಾರೆ
ವಾದಿ-ಸು-ವುದು
ವಾದು-ದಕ್ಕೆ
ವಾದು-ದನ್ನೆಲ್ಲ
ವಾದು-ದಾ-ಗಿದೆ
ವಾದು-ದೆಂದರೆ
ವಾದುವು
ವಾದೊಂದು
ವಾದ್ಧದ್ಧಿಂದ
ವಾನ-ಪ್ರಸ್ಥ
ವಾಮ-ಚಾರ
ವಾಮಾ
ವಾಮಾ-ಚಾರ
ವಾಮಾ-ಚಾರ-ವನ್ನು
ವಾಮಾ-ಚಾರ-ವಿದೆ
ವಾಯಿತು
ವಾಯಿತೆಂಬು-ದನ್ನು
ವಾಯು
ವಾಯು-ಗುಣ
ವಾಯು-ವಿ-ನಂತೆ
ವಾಯು-ಸ್ಪಂದನ
ವಾರ-ಗಳಿಂದ
ವಾರದ
ವಾರ್ತೆ
ವಾರ್ಷಿಕ
ವಾಲ್ಮೀಕಿ
ವಾಲ್ಮೀಕಿ-ಯು-ಪ್ರಾಚೀನ
ವಾಷಿಂಗ್ಟನ್ನಿಗೆ
ವಾಸ
ವಾಸ-ನೆಯ
ವಾಸ-ನೆಯೂ
ವಾಸ-ಮಾಡ-ಬೇಕೆಂದು
ವಾಸ-ಮಾಡಿದ
ವಾಸ-ಮಾಡುತ್ತಾ
ವಾಸ-ಯೋಗ್ಯ-ವನ್ನಾಗಿ
ವಾಸ-ವಾಗಿ-ದ್ದರೂ
ವಾಸಿ-ಸುತ್ತ
ವಾಸಿಸುತ್ತಾನೆ
ವಾಸಿಸುತ್ತಾರೆ
ವಾಸಿಸು-ತ್ತಿದ್ದ
ವಾಸಿಸು-ತ್ತಿದ್ದರು
ವಾಸಿಸು-ತ್ತಿದ್ದರೂ
ವಾಸಿಸು-ತ್ತಿದ್ದ-ರೇನೋ
ವಾಸಿಸುತ್ತಿರು-ವಾಗ
ವಾಸಿಸುವ
ವಾಸಿಸು-ವರು
ವಾಸಿಸುವ-ವ-ರಿಗೆ
ವಾಸಿಸುವ-ವರು
ವಾಸಿಸು-ವಿರೋ
ವಾಸಿಸು-ವು-ದಕ್ಕೆ
ವಾಸಿಸು-ವುದು
ವಾಸ್ತವ
ವಾಸ್ತವ-ವಾಗಿ
ವಾಸ್ತವ-ವಾಗು-ತ್ತಿದೆ
ವಾಸ್ತವಿಕ
ವಾಸ್ತವಿಕ-ವಾಗಿ
ವಾಸ್ತವಿಕ-ವಾದ
ವಾಹ-ಕರೂ
ವಾಹ-ನ-ಸಂಚಾರ
ವಾಹಿನಿ
ವಿಂಧ್ಯ
ವಿಕರ್ಷಣ
ವಿಕ-ಸ-ನದ
ವಿಕಸಿತ-ವಾದ
ವಿಕಸಿತ-ವಾದುದು
ವಿಕಸಿತ-ವಾ-ಯಿತೋ
ವಿಕಾರ
ವಿಕಾರ-ಗಳಿ-ಗೊಳಗಾಗು-ವುವು
ವಿಕಾರ-ಗೊಳಿ-ಸದೆ
ವಿಕಾಸ
ವಿಕಾಸಕ್ಕೆ
ವಿಕಾಸ-ಗಳಂತಿ-ವೆಯೋ
ವಿಕಾಸ-ಗಳಿಲ್ಲ
ವಿಕಾಸ-ಗೊಂಡು
ವಿಕಾಸ-ಗೊಳಿಸಿ
ವಿಕಾಸ-ಗೊಳ್ಳು-ತ್ತದೆ
ವಿಕಾಸ-ಗೊಳ್ಳು-ವುದೂ
ವಿಕಾಸದ
ವಿಕಾಸ-ದಲ್ಲಿ
ವಿಕಾಸ-ವನ್ನು
ವಿಕಾಸ-ವಾಗ-ಬೇಕು
ವಿಕಾಸ-ವಾಗಿ
ವಿಕಾಸ-ವಾಗಿದೆ
ವಿಕಾಸ-ವಾಗಿ-ದೆಯೊ
ವಿಕಾಸ-ವಾಗಿವೆ
ವಿಕಾಸ-ವಾಗುತ್ತ
ವಿಕಾಸ-ವಾಗುತ್ತಾ
ವಿಕಾಸ-ವಾಗು-ತ್ತಿದೆ
ವಿಕಾಸ-ವಾಗುತ್ತಿ-ರು-ವನು
ವಿಕಾಸ-ವಾಗು-ವಂತೆ
ವಿಕಾಸ-ವಾದ
ವಿಕಾಸ-ವಾದ-ದಂತೆ
ವಿಕಾಸ-ವೆಂದು
ವಿಕಾಸವೇ
ವಿಕಾಸ-ಶೀಲವೂ
ವಿಕೃ-ತಗೊ-ಳಿ-ಸಲಾ-ಗದ
ವಿಕೃ-ತಗೊ-ಳಿ-ಸಿದ್ದೇವೆ
ವಿಕೃ-ತಗೊ-ಳಿಸಿರು
ವಿಕೃತ-ವಾದ-ದ್ದಾ-ದರೂ
ವಿಕೃತಿ-ಗಳೆಲ್ಲ
ವಿಕೆ-ಯಿಂದ
ವಿಕ್ಟೋರಿಯ
ವಿಖ್ಯಾ-ತ-ನಾಮ-ರಾದ
ವಿಗ್ರಹ
ವಿಗ್ರ-ಹಕ್ಕೆ
ವಿಗ್ರಹ-ಗಳನ್ನು
ವಿಗ್ರಹ-ಗಳು
ವಿಗ್ರ-ಹದ
ವಿಗ್ರಹ-ದಲ್ಲಿ
ವಿಗ್ರಹ-ಪೂಜೆ-ಯನ್ನು
ವಿಗ್ರಹ-ರಾಧನೆ
ವಿಗ್ರಹ-ರಾಧ-ನೆ-ಯನ್ನು
ವಿಗ್ರಹ-ವನ್ನ-ಲ್ಲದೆ
ವಿಗ್ರಹ-ವನ್ನು
ವಿಗ್ರಹ-ವನ್ನೆಲ್ಲ
ವಿಗ್ರಹ-ವನ್ನೋ
ವಿಗ್ರಹಾ
ವಿಗ್ರ-ಹಾ-ರಾಧಕ
ವಿಗ್ರ-ಹಾ-ರಾಧಕ-ರಾಗಿ-ದ್ದ-ರೆಂಬು-ದನ್ನು
ವಿಗ್ರ-ಹಾ-ರಾಧನೆ
ವಿಗ್ರ-ಹಾ-ರಾಧ-ನೆ-ಯನ್ನು
ವಿಗ್ರ-ಹಾ-ರಾಧ-ನೆ-ಯಿಂದ
ವಿಗ್ರ-ಹಾ-ರಾಧ-ನೆಯು
ವಿಘ್ನ-ಗಳಿವೆ
ವಿಚಾರ
ವಿಚಾ-ರಕ್ಕೆ
ವಿಚಾರ-ಗಳನ್ನು
ವಿಚಾರ-ಗಳು
ವಿಚಾರದ
ವಿಚಾರ-ದಲ್ಲಿ
ವಿಚಾರ-ದಲ್ಲಿ-ರುವ
ವಿಚಾರ-ಪರ
ವಿಚಾರ-ಪರ-ನಾ-ಗಿಯೇ
ವಿಚಾರ-ಪ-ರನೂ
ವಿಚಾರ-ಪರ-ನೆಂದೂ
ವಿಚಾರ-ಮತಿ-ಗಳಿಗೂ
ವಿಚಾರ-ಮತಿ-ಗಳು
ವಿಚಾರ-ಮಾಡಿ-ದರೆ
ವಿಚಾರ-ಮಾಡಿ-ದಷ್ಟೂ
ವಿಚಾರ-ವನ್ನು
ವಿಚಾರ-ವಾಗಿ
ವಿಚಾರ-ವಾಗಿ-ಯಾಗಲೀ
ವಿಚಾರ-ವಾಗಿಯೇ
ವಿಚಾರ-ವಾದ
ವಿಚಾರ-ವಾದಿ-ಯಾಗಿ
ವಿಚಾರ-ವಿ-ಲ್ಲದೆ
ವಿಚಾ-ರವು
ವಿಚಾ-ರವೂ
ವಿಚಾರ-ವೇ-ನೆಂದರೆ
ವಿಚಾರ-ಶಕ್ತಿಯ
ವಿಚಾರ-ಶಕ್ತಿ-ಯಲ್ಲಿ
ವಿಚಾರ-ಶಕ್ತಿ-ಯುಳ್ಳ
ವಿಚಾರ-ಶೀಲ
ವಿಚಾರ-ಶೀಲರು
ವಿಚಾರ-ಹೀನ
ವಿಚಾರ-ಹೀನ-ವಾಗಿ-ರುವ
ವಿಚಾ-ರಿ-ಸಲು
ವಿಚಾ-ರಿಸಿ
ವಿಚಾ-ರಿಸಿ-ಕೊಳ್ಳು-ತ್ತೇನೆ
ವಿಚಾ-ರಿಸಿ-ದ್ದೇವೆ
ವಿಚಾರಿಸುತ್ತಿರು-ವುದು
ವಿಚಾರಿ-ಸು-ವು-ದಕ್ಕೆ
ವಿಚಿತ್ರ
ವಿಚಿತ್ರ-ವಾಗಿ
ವಿಚಿತ್ರ-ವಾಗಿದೆ
ವಿಚಿತ್ರ-ವಾದ
ವಿಚಿತ್ರ-ವಾದುದು
ವಿಚಿತ್ರವೂ
ವಿಚಿತ್ರ-ವೆಂದರೆ
ವಿಚಿತ್ರ-ವೇ-ನೆಂದರೆ
ವಿಜಯ
ವಿಜಯ-ಚರಿತ್ರೆ
ವಿಜಯ-ವನ್ನು
ವಿಜಯಿ-ಗಳಾಗಲಿ
ವಿಜಯಿ-ಗಳೇ
ವಿಜಯಿ-ಯಾಗಿ
ವಿಜಾನೀ
ವಿಜಾನೀ-ತಾತ್
ವಿಜ್ಞಾತಂ
ವಿಜ್ಞಾತಾ
ವಿಜ್ಞಾತಾ-ರ-ಮರೇ
ವಿಜ್ಞಾತೇ
ವಿಜ್ಞಾನ
ವಿಜ್ಞಾನಕ್ಕೂ
ವಿಜ್ಞಾನಕ್ಕೆ
ವಿಜ್ಞಾನ-ಗಳನ್ನು
ವಿಜ್ಞಾನ-ಗಳಲ್ಲಿ
ವಿಜ್ಞಾನ-ಗಳು
ವಿಜ್ಞಾನದ
ವಿಜ್ಞಾನ-ದಲ್ಲಿ
ವಿಜ್ಞಾನ-ದಲ್ಲಿಯೂ
ವಿಜ್ಞಾನ-ವಾಗಲಿ
ವಿಜ್ಞಾನ-ವಾದರೋ
ವಿಜ್ಞಾನವು
ವಿಜ್ಞಾನವೂ
ವಿಜ್ಞಾನ-ವೆಂದು
ವಿಜ್ಞಾನ-ಶಾಶ್ತ್ರವು
ವಿಜ್ಞಾನ-ಶಾಸ್ತ್ರ
ವಿಜ್ಞಾನ-ಶಾಸ್ತ್ರಕ್ಕೂ
ವಿಜ್ಞಾನ-ಶಾಸ್ತ್ರವು
ವಿಜ್ಞಾನಿ
ವಿಜ್ಞಾನಿ-ಗಳು
ವಿಜ್ಞಾನಿ-ಗಳೇ
ವಿಜ್ಞಾನಿಯು
ವಿತತೋ
ವಿತರ
ವಿತರ್ಕ
ವಿದುಷಾಂ
ವಿದೇಶ-ಗಳ
ವಿದೇಶ-ಗಳಲ್ಲಿ
ವಿದೇಶ-ಗಳಿಗೆ
ವಿದೇಶದ
ವಿದೇಶ-ದಿಂದ
ವಿದೇಶಿಯ-ರನ್ನು
ವಿದೇಶಿ-ಯರು
ವಿದೇಶೀ
ವಿದೇಶೀ-ನೀತಿ-ಯಾಗ-ಬೇಕು
ವಿದೇಶೀ-ಯನೂ
ವಿದೇಶೀ-ಯರ
ವಿದೇಶೀಯ-ರಿಗೆ
ವಿದ್ದರೆ
ವಿದ್ಯತೇ
ವಿದ್ಯಾ
ವಿದ್ಯಾಂ
ವಿದ್ಯಾ-ಕೇಂದ್ರ-ಗಳಿಗೆ
ವಿದ್ಯಾ-ಕೇಂದ್ರ-ವಾದ
ವಿದ್ಯಾ-ದಾನ
ವಿದ್ಯಾ-ಭಿ-ಮಾನ
ವಿದ್ಯಾ-ಭ್ಯಾಸ
ವಿದ್ಯಾ-ಭ್ಯಾ-ಸಕ್ಕೆ
ವಿದ್ಯಾ-ಭ್ಯಾ-ಸ-ದಲ್ಲಿ
ವಿದ್ಯಾ-ಭ್ಯಾ-ಸ-ದಿಂದ
ವಿದ್ಯಾ-ಭ್ಯಾ-ಸ-ವನ್ನು
ವಿದ್ಯಾ-ಭ್ಯಾ-ಸ-ವಲ್ಲ
ವಿದ್ಯಾ-ಭ್ಯಾ-ಸವೂ
ವಿದ್ಯಾ-ಭ್ಯಾ-ಸ-ವೆಂದರೆ
ವಿದ್ಯಾ-ಮಾ-ದದೀತಾ-ವ-ರಾದಪಿ
ವಿದ್ಯಾರ್ಜನೆ
ವಿದ್ಯಾರ್ಥಿ
ವಿದ್ಯಾರ್ಥಿ-ಗಳಿಗೂ
ವಿದ್ಯಾರ್ಥಿ-ಗಳು
ವಿದ್ಯಾರ್ಥಿ-ಗಳೇ
ವಿದ್ಯಾರ್ಥಿ-ಯಾಗು-ವನು
ವಿದ್ಯಾ-ವಂತರ
ವಿದ್ಯಾ-ವಂತ-ರಲ್ಲಿ
ವಿದ್ಯಾ-ವಂತ-ರಾಗಿ-ರುವ
ವಿದ್ಯಾ-ವಂತ-ರಾ-ಗುವ-ವ-ರೆಗೆ
ವಿದ್ಯಾ-ವಂತ-ರಾ-ದರೂ
ವಿದ್ಯಾ-ವಂತ-ರಾ-ದ-ವರು
ವಿದ್ಯಾ-ವಂತ-ರಿ-ಗಿಂತ
ವಿದ್ಯಾ-ವಂತರು
ವಿದ್ಯಾ-ವಂತರೂ
ವಿದ್ಯಾ-ವಂತ-ವಾಗಿಲ್ಲ
ವಿದ್ಯುಚ್ಛಕ್ತಿ
ವಿದ್ಯುತೋ
ವಿದ್ಯು-ತ್
ವಿದ್ಯುತ್ತು
ವಿದ್ಯುನ್ಮಯ
ವಿದ್ಯೆ
ವಿದ್ಯೆ-ಗಳನ್ನು
ವಿದ್ಯೆಗೆ
ವಿದ್ಯೆಯ
ವಿದ್ಯೆ-ಯನ್ನು
ವಿದ್ಯೆ-ಯ-ವರು
ವಿದ್ಯೆ-ಯಿಂದ
ವಿದ್ಯೆಯೇ
ವಿದ್ವತ್ಪೂರ್ಣ
ವಿದ್ವಾಂಸ
ವಿದ್ವಾಂಸರ
ವಿದ್ವಾಂಸ-ರಾಗ-ಬಾ-ರದು
ವಿದ್ವಾಂಸ-ರಾದ
ವಿದ್ವಾಂಸ-ರಿ-ಗಿಂತ
ವಿದ್ವಾಂಸ-ರಿಗೆ
ವಿದ್ವಾಂಸರು
ವಿದ್ವಾಂಸರೇ
ವಿದ್ವಿಷಾ-ವಹೈ
ವಿಧ-ಗಳಲ್ಲಿ
ವಿಧದ
ವಿಧ-ದಲ್ಲಿ
ವಿಧ-ದಲ್ಲಿಯೂ
ವಿಧವಾ
ವಿಧ-ವಾಗಿ
ವಿಧ-ವಾದ
ವಿಧ-ವೆಯ
ವಿಧ-ವೆ-ಯನ್ನು
ವಿಧ-ವೆ-ಯರ
ವಿಧ-ವೆ-ಯ-ರಿಗೆ
ವಿಧಾತ
ವಿಧಾ-ತನ
ವಿಧಾನ
ವಿಧಾ-ನಕ್ಕೆ
ವಿಧಾನ-ಗಳ
ವಿಧಾನ-ಗಳನ್ನು
ವಿಧಾನ-ಗಳು
ವಿಧಾನ-ಗಳೂ
ವಿಧಾನದ
ವಿಧಾನ-ದಲ್ಲಿ
ವಿಧಾನ-ಪರಿ-ಷ-ತ್ತಿನ
ವಿಧಾನ-ವನ್ನಾಗಿ
ವಿಧಾನ-ವನ್ನು
ವಿಧಾನ-ವನ್ನೇ
ವಿಧಿ-ನಿಷೇಧ-ಗಳು
ವಿಧಿಯ-ನ್ನೇ
ವಿಧಿ-ಯಿಲ್ಲ
ವಿಧಿಯು
ವಿಧಿಯೇ
ವಿಧಿ-ವಾದ
ವಿಧಿ-ವಿ-ಧಾನ-ಗಳನ್ನು
ವಿಧಿ-ಸಲಿ
ವಿಧಿ-ಸಿದ
ವಿಧಿ-ಸಿದೆ-ಯೆಂತಲೇ
ವಿಧಿಸಿ-ರುವ
ವಿಧಿಸಿರು-ವುದು
ವಿಧಿ-ಸುತ್ತವೆ
ವಿಧಿಸು-ವುದು
ವಿಧೇಯ
ವಿಧೇಯತೆ
ವಿಧೇಯ-ರಾದ
ವಿಧೇ-ಯರು
ವಿಧೇಯರೂ
ವಿಧೇಯ-ವಾಗಿ-ರ-ಲಾ-ರದು
ವಿನಃ
ವಿನಯ
ವಿನ-ಯ-ಕೃಷ್ಣ
ವಿನ-ಯ-ದಿಂದ
ವಿನ-ಶ್ಯತ್ಸ್ವ-ವಿನ-ಶ್ಯಂತಂ
ವಿನಾ
ವಿನಾ-ಯಿತಿ
ವಿನಾಶ
ವಿನಾ-ಶ-ಕಾರಿ
ವಿನಾ-ಶ-ಕಾರಿ-ಗಳಾಗಿವೆ
ವಿನಾ-ಶ-ಕಾರಿ-ಯಾಗು-ತ್ತದೆ
ವಿನಾಶ-ವಾಗು-ತ್ತದೆಯೋ
ವಿನಾಶವೇ
ವಿನಾ-ಶ-ಹೇತು
ವಿನಾ-ಶಾಯ
ವಿನಾ-ಶಿ-ಗಳಾಗಿ-ರುವ
ವಿನಿ-ಮಯ
ವಿನಿ-ಯೋಗ
ವಿನೋದ
ವಿನ್ಯಾ-ಸಗೊಂಡಿ-ರುವ
ವಿಪರ್ಯಾಸ
ವಿಪರ್ಯಾ-ಸ-ವೆಂದರೆ
ವಿಪ್ರಾ
ವಿಪ್ರಾಃ
ವಿಫಲ
ವಿಫಲ-ನಾದನು
ವಿಫಲ-ರಾದರು
ವಿಫಲ-ವಾಗಿ-ರು-ವುದು
ವಿಫಲ-ವಾಗಿವೆ
ವಿಭಜ-ನೆಯು
ವಿಭಾಗ-ಗೊಂಡಿ-ದ್ದವು
ವಿಭಾಗ-ಗೊಂಡಿ-ದ್ದಾಗ
ವಿಭಾಗ-ವಾಗಿ
ವಿಭಾಗ-ವಾಗುತ್ತಾ
ವಿಭಾಗ-ವಾಗು-ವುದೇ
ವಿಭಾ-ಗಿಸಿ-ರು-ವರು
ವಿಭಾತಿ
ವಿಭಿನ್ನ
ವಿಭಿನ್ನ-ರೂ-ಪಿಯೂ
ವಿಭಿನ್ನ-ವಾದ
ವಿಭಿನ್ನ-ವಾದರೂ
ವಿಭು
ವಿಭುವೇ
ವಿಭೂತಿ-ಗಳು
ವಿಭೂತಿ-ಗಳೆಲ್ಲ
ವಿಮರ್ಶಿ-ಸು-ತ್ತಿಲ್ಲ
ವಿಮರ್ಶಿ-ಸುವ
ವಿಮರ್ಶಿ-ಸೋಣ
ವಿಮರ್ಶೆ
ವಿಮರ್ಶೆ-ಮಾಡದೆ
ವಿಮುಂಚಥ
ವಿಮುಖ-ನಾಗು-ವನು
ವಿಮುಖ-ರಾಗು-ವಿರಿ
ವಿಮುಖ-ರಾ-ಗು-ವು-ದಿಲ್ಲ
ವಿಮುಖ-ರಾದರು
ವಿಮೋಚನೆ
ವಿರ-ಸವೂ
ವಿರ-ಹದ
ವಿರಹ-ದಿಂದ
ವಿರಹ-ವ್ಯಥೆ
ವಿರಾಜ-ಮಾ-ನ-ಳಾಗಿ-ರುವಳು
ವಿರಾಟನ
ವಿರಾ-ಟ್
ವಿರಾಮ-ವೆಂದಿಗೂ
ವಿರಿ
ವಿರುದ್ದ-ವಾಗಿ
ವಿರುದ್ಧ
ವಿರುದ್ಧ-ವಲ್ಲ
ವಿರುದ್ಧ-ವಾಗಿ
ವಿರುದ್ಧ-ವಾಗಿ-ದ್ದರೆ
ವಿರುದ್ಧ-ವಾಗಿ-ರಲಿ
ವಿರುದ್ಧ-ವಾಗಿ-ರು-ವಂತೆ
ವಿರುದ್ಧ-ವಾಗಿಲ್ಲ
ವಿರುದ್ಧ-ವಾಗಿವೆ
ವಿರುದ್ಧ-ವಾದ
ವಿರುದ್ಧ-ವಾದುದು
ವಿರುದ್ಧವೂ
ವಿರುದ್ಧ್ಧ್ದ
ವಿರು-ವು-ದಕ್ಕೆ
ವಿರೋಧ
ವಿರೋ-ಧ-ಗಳ
ವಿರೋ-ಧ-ಗಳಿಂದ-ಕೂಡಿ-ರುವ
ವಿರೋ-ಧ-ಗಳು
ವಿರೋ-ಧ-ಗಳೆಲ್ಲ
ವಿರೋ-ಧದ
ವಿರೋ-ಧ-ದಂತೆ
ವಿರೋ-ಧ-ವಾಗಿ
ವಿರೋ-ಧ-ವಾಗಿ-ಯಾ-ದರೂ
ವಿರೋ-ಧ-ವಾಗಿ-ರು-ವಾಗ
ವಿರೋ-ಧ-ವಾಗಿ-ರು-ವು-ದ-ರಿಂದ
ವಿರೋ-ಧ-ವಾಗಿ-ರು-ವುದೇ
ವಿರೋ-ಧ-ವಾಗಿಲ್ಲ
ವಿರೋ-ಧ-ವಾದ
ವಿರೋ-ಧ-ವಿದೆ
ವಿರೋ-ಧ-ವಿಲ್ಲ
ವಿರೋ-ಧ-ವಿ-ಲ್ಲದೇ
ವಿರೋ-ಧವು
ವಿರೋ-ಧಾಭಾಸ-ದಂತೆ
ವಿರೋ-ಧಿ-ಗಳಲ್ಲ
ವಿರೋ-ಧಿ-ಗಳ-ಲ್ಲ-ವೆಂದು
ವಿರೋ-ಧಿ-ಗಳಿ-ದ್ದರೂ
ವಿರೋ-ಧಿ-ಗಳು
ವಿರೋ-ಧಿ-ಗಳೆಂದು
ವಿರೋ-ಧಿಸ-ಬಲ್ಲರು
ವಿರೋ-ಧಿಸ-ಬೇಕಾಗಿಲ್ಲ
ವಿರೋ-ಧಿಸ-ಬೇಕು
ವಿರೋ-ಧಿಸಿ
ವಿರೋ-ಧಿಸಿ-ದರು
ವಿರೋ-ಧಿಸಿ-ದರೂ
ವಿರೋ-ಧಿಸಿ-ದ-ವ-ರಲ್ಲಿ
ವಿರೋ-ಧಿಸುತ್ತಿ-ರುವ
ವಿರೋ-ಧಿಸು-ತ್ತೇನೆ
ವಿರೋ-ಧಿ-ಸುವ
ವಿರೋ-ಧಿ-ಸುವ-ವರೇ
ವಿರೋ-ಧಿ-ಸು-ವು-ದಕ್ಕೆ
ವಿರೋ-ಧಿ-ಸು-ವು-ದಿಲ್ಲ-ವಾದರೂ
ವಿರೋ-ಧಿಸು-ವುದು
ವಿರೋ-ಧಿಸು-ವುವೋ
ವಿಲ-ಕ್ಷಣ
ವಿಲಾಸ
ವಿಲಾ-ಸದ
ವಿಲಾಸೋನ್ಮಾ-ದದ
ವಿಲ್ಲ-ವಾ-ಯಿತು
ವಿಳಂಬ
ವಿವರ-ಗಳ
ವಿವರ-ಗಳ-ಲ್ಲಿಯೂ
ವಿವರ-ಗಳು
ವಿವರ-ಣಾ-ತ್ಮಕ
ವಿವ-ರಣೆ
ವಿವ-ರಣೆ-ಗಳಿವೆ
ವಿವ-ರಣೆ-ಗಳು
ವಿವ-ರಣೆ-ಗಿಂತ
ವಿವ-ರಣೆಗೂ
ವಿವರ-ಣೆಗೆ
ವಿವ-ರಣೆ-ಯನ್ನು
ವಿವ-ರಣೆಯೂ
ವಿವರ-ವಾಗಿ
ವಿವರಿಸ-ಬಲ್ಲದು
ವಿವರಿ-ಸಬಲ್ಲುದು
ವಿವರಿಸ-ಬಹುದು
ವಿವರಿಸ-ಬೇಕಾಗಿದೆ
ವಿವರಿಸ-ಬೇ-ಕಾದರೆ
ವಿವರಿಸಲಾ-ರವು
ವಿವ-ರಿ-ಸಲು
ವಿವರಿಸ-ಲ್ಪಟ್ಟಿವೆ
ವಿವ-ರಿಸಿ
ವಿವ-ರಿಸಿ-ದರು
ವಿವ-ರಿಸಿ-ರು-ವರು
ವಿವ-ರಿಸಿ-ರು-ವುದು
ವಿವ-ರಿಸಿ-ರು-ವೆನು
ವಿವರಿ-ಸುತ್ತದೆ
ವಿವರಿ-ಸುತ್ತವೆ
ವಿವರಿ-ಸುವ
ವಿವರಿ-ಸು-ವನು
ವಿವರಿ-ಸು-ವರು
ವಿವರಿಸು-ವಾಗ
ವಿವರಿ-ಸು-ವು-ದಕ್ಕೆ
ವಿವರಿ-ಸು-ವು-ದಿಲ್ಲ
ವಿವರಿಸು-ವುದು
ವಿವರಿಸು-ವೆನು
ವಿವಾದ
ವಿವಾದಕ್ಕೆ
ವಿವಾದ-ಗಳು
ವಿವಾಹ
ವಿವಾ-ಹದ
ವಿವಾ-ಹ-ದಿಂದ
ವಿವಾ-ಹ-ವಾಗ-ಬೇಕು
ವಿವಾ-ಹೇ-ತರ
ವಿವಿಧ
ವಿವಿಧ-ರೂಪ-ಗಳು
ವಿವಿಧವೂ
ವಿವಿಧೋಪಾಯ-ಗಳನ್ನು
ವಿವೇಕ
ವಿವೇಕ-ಚೂಡಾ-ಮಣಿ
ವಿವೇಕಾ-ನಂದ
ವಿವೇಕಾ-ನಂದರ
ವಿವೇಕಾ-ನಂದ-ರನ್ನು
ವಿವೇಕಾ-ನಂದ-ರಿಗೆ
ವಿವೇಕಾ-ನಂದರು
ವಿವೇಕಿ
ವಿವೇಚನಾ-ಪೂರ್ವಕ
ವಿವೇಚನೆ
ವಿವೇಚ-ನೆ-ಯಿ-ಲ್ಲದೆ
ವಿಶದ-ಪಡಿ-ಸಲು
ವಿಶದ-ವಾಗಿ
ವಿಶದ-ವಾಗು-ತ್ತದೆ
ವಿಶಾಲ
ವಿಶಾಲ-ಮತಿ-ಗಳೂ
ವಿಶಾಲ-ಮ-ತಿಯೂ
ವಿಶಾಲ-ವನ್ನಾಗಿ
ವಿಶಾಲ-ವಾಗ-ಬೇಕು
ವಿಶಾಲ-ವಾಗಿದೆ
ವಿಶಾಲ-ವಾಗಿಯೂ
ವಿಶಾಲ-ವಾಗಿ-ರ-ಬೇಕು
ವಿಶಾಲ-ವಾಗುತ್ತಾ-ಹೋ-ಗ-ಬೇಕು
ವಿಶಾಲ-ವಾಗು-ತ್ತಿದೆ
ವಿಶಾಲ-ವಾದ
ವಿಶಾಲ-ವಾದು-ದ-ರಿಂದ
ವಿಶಾಲ-ವಾದುದು
ವಿಶಾಲವೂ
ವಿಶಾಲ-ವೃಕ್ಷವು
ವಿಶಿಷ್ಟ
ವಿಶಿಷ್ಟ-ಕಾಲಕ್ಕೆ
ವಿಶಿಷ್ಟ-ವಾಗಿದೆ
ವಿಶಿಷ್ಟ-ವಾದ
ವಿಶಿಷ್ಟ-ವಾದುದು
ವಿಶಿಷ್ಟಾ-ದ್ವೈತ
ವಿಶಿಷ್ಟಾ-ದ್ವೈ-ತ-ಗಳ
ವಿಶಿಷ್ಟಾ-ದ್ವೈ-ತ-ಗಳೆಂಬ
ವಿಶಿಷ್ಟಾ-ದ್ವೈ-ತಿ-ಗಳಾಗಲಿ
ವಿಶಿಷ್ಟಾ-ದ್ವೈ-ತಿ-ಗಳಾಗಲೀ
ವಿಶಿಷ್ಟಾ-ದ್ವೈ-ತಿ-ಗಳಿಗೂ
ವಿಶಿಷ್ಟಾ-ದ್ವೈ-ತಿ-ಗಳು
ವಿಶಿಷ್ಟಾ-ದ್ವೈ-ತಿ-ಗಳೂ
ವಿಶಿಷ್ಟಾ-ದ್ವೈ-ತಿ-ಗಳೆಂಬ
ವಿಶಿಷ್ಟಾ-ದ್ವೈ-ತಿ-ಗಳೋ
ವಿಶಿಷ್ಟಾ-ದ್ವೈ-ತಿ-ಗ್ಧಳೂ
ವಿಶಿಷ್ಟಾ-ದ್ವೈ-ತಿ-ಯಾಗಲೀ
ವಿಶುದ್ಧ
ವಿಶುದ್ಧ-ವಾಗಿದೆ
ವಿಶೇಷ
ವಿಶೇಷ-ಣ-ಗಳ
ವಿಶೇಷ-ಣ-ವನ್ನೂ
ವಿಶೇಷ-ಣ-ವಾಗಿ
ವಿಶೇಷ-ಣವೇ
ವಿಶೇಷತಃ
ವಿಶೇಷದ
ವಿಶೇಷ-ಲ-ಕ್ಷಣ-ದಿಂದ
ವಿಶೇಷ-ವಾಗಿ
ವಿಶೇಷ-ವಾಗಿ-ದ್ದರೆ
ವಿಶೇಷ-ವಾದ
ವಿಶೇಷ-ವೇ-ನೆಂದರೆ
ವಿಶೇಷ-ಹಕ್ಕು
ವಿಶ್ರಾಂತಿ
ವಿಶ್ರಾಂತಿ-ಯನ್ನು
ವಿಶ್ಲೇಷಣಾ-ತ್ಮಕ
ವಿಶ್ಲೇಷಣೆ
ವಿಶ್ಲೇಷ-ಣೆಗೆ
ವಿಶ್ಲೇಷಣೆಯ
ವಿಶ್ಲೇಷಣೆ-ಯಿಂದ
ವಿಶ್ಲೇಷಣೆಯು
ವಿಶ್ಲೇಷಣೆಯೆ
ವಿಶ್ಲೇಷಿಸ-ಬಹುದು
ವಿಶ್ಲೇಷಿ-ಸಲು
ವಿಶ್ಲೇಷಿಸು-ವುದು
ವಿಶ್ವ
ವಿಶ್ವ-ಕಲ್ಯಾ-ಣ-ವೆಂಬ
ವಿಶ್ವ-ಕಾನೂನು
ವಿಶ್ವ-ಕೋಶ-ಗಳೇ
ವಿಶ್ವಕ್ಕೆ
ವಿಶ್ವ-ಕ್ಕೆಲ್ಲಾ
ವಿಶ್ವ-ತೋಮುಖಃ
ವಿಶ್ವದ
ವಿಶ್ವ-ದಲ್ಲಿ
ವಿಶ್ವ-ದಲ್ಲಿ-ರುವ
ವಿಶ್ವ-ದೊಡನೆ
ವಿಶ್ವ-ಧರ್ಮ
ವಿಶ್ವ-ಧರ್ಮದ
ವಿಶ್ವ-ಧರ್ಮ-ವನ್ನು
ವಿಶ್ವ-ಧರ್ಮ-ವಾಗಿದೆ
ವಿಶ್ವ-ಧರ್ಮ-ವಾಗುವ
ವಿಶ್ವ-ಧರ್ಮವು
ವಿಶ್ವ-ಧರ್ಮ-ಸಮ್ಮೇಳನ
ವಿಶ್ವ-ಧರ್ಮ-ಸಮ್ಮೇಳನ-ದಲ್ಲಿ
ವಿಶ್ವ-ಪ್ರೇಮ-ವನ್ನು
ವಿಶ್ವ-ಭ್ರಾತೃ-ತ್ವದ
ವಿಶ್ವ-ಭ್ರಾತೃ-ತ್ವ-ವನ್ನು
ವಿಶ್ವ-ಮನ-ಸ್ಸಿನ-ಲ್ಲಿಯೂ
ವಿಶ್ವ-ಮನಸ್ಸು
ವಿಶ್ವ-ಮಾ-ನ್ಯ-ವಾದುವು
ವಿಶ್ವ-ಮೇಳದ
ವಿಶ್ವ-ಯೋ-ಜನೆ
ವಿಶ್ವ-ರಚ-ನೆಗೆ
ವಿಶ್ವ-ರಚ-ನೆಯ
ವಿಶ್ವ-ರಚ-ನೆ-ಯನ್ನು
ವಿಶ್ವವ
ವಿಶ್ವ-ವನ್ನು
ವಿಶ್ವ-ವನ್ನೆಲ್ಲಾ
ವಿಶ್ವ-ವಿಖ್ಯಾತ
ವಿಶ್ವ-ವಿದ್ಯಾ
ವಿಶ್ವ-ವಿದ್ಯಾ-ನಿ-ಲ-ಯದ
ವಿಶ್ವವು
ವಿಶ್ವವೂ
ವಿಶ್ವ-ವೆ-ಲ್ಲ-ವನ್ನು
ವಿಶ್ವ-ವೆ-ಲ್ಲವೂ
ವಿಶ್ವ-ವೆಲ್ಲಾ
ವಿಶ್ವವೇ
ವಿಶ್ವ-ವ್ಯಾ-ಪ-ಕತೆ
ವಿಶ್ವ-ಸಂಸ್ಥೆ
ವಿಶ್ವ-ಸ-ಹೋ-ದ-ರ-ತ್ವದ
ವಿಶ್ವ-ಸಾಮರಸ್ಯಕ್ಕೆ
ವಿಶ್ವ-ಸಾ-ಹಿ-ತ್ಯ-ದಲ್ಲಿ
ವಿಶ್ವಾ-ತ್ಮವೇ
ವಿಶ್ವಾ-ಮಿತ್ರ-ನನ್ನಾ-ದರೂ
ವಿಶ್ವಾಸ
ವಿಶ್ವಾ-ಸ-ಗಳಿ-ರ-ಬೇಕು
ವಿಶ್ವಾ-ಸ-ಗಳು
ವಿಶ್ವಾ-ಸ-ಪ-ರರು
ವಿಶ್ವಾ-ಸ-ಪೂರ್ವಕ
ವಿಶ್ವಾ-ಸ-ಪೂರ್ವ-ಕ-ವಾದ
ವಿಶ್ವಾ-ಸ-ವಿಡು-ವುದು
ವಿಶ್ವಾ-ಸ-ವಿದೆ
ವಿಶ್ವಾ-ಸ-ವಿಲ್ಲ
ವಿಶ್ವಾ-ಸವು
ವಿಶ್ವಾ-ಸ-ವುಂಟು
ವಿಶ್ವೇ-ಶ್ವ-ರನು
ವಿಷ-ಕ್ರಿಮಿ
ವಿಷ-ಕ್ರಿಮಿ-ಗಳ
ವಿಷ-ಕ್ರಿಮಿ-ಗಳನ್ನು
ವಿಷ-ಕ್ರಿಮಿ-ಗಳಿಗೂ
ವಿಷ-ಕ್ರಿಮಿ-ಗಳು
ವಿಷ-ಕ್ರಿಮಿಗೆ
ವಿಷ-ಕ್ರಿಮಿ-ಯನ್ನು
ವಿಷ-ಕ್ರಿಮಿಯೂ
ವಿಷ-ದಂತೆ
ವಿಷ-ದಿಂದ
ವಿಷ-ಮಯ
ವಿಷಯ
ವಿಷಯಕ್ಕೂ
ವಿಷ-ಯಕ್ಕೆ
ವಿಷಯ-ಗಳ
ವಿಷಯ-ಗಳಂತೆಯೇ
ವಿಷಯ-ಗಳ-ನ್ನಾ-ಗಲಿ
ವಿಷಯ-ಗಳ-ನ್ನಿಟ್ಟು
ವಿಷಯ-ಗಳನ್ನು
ವಿಷಯ-ಗಳನ್ನೂ
ವಿಷಯ-ಗಳ-ನ್ನೆಲ್ಲಾ
ವಿಷಯ-ಗಳಲ್ಲಿ
ವಿಷಯ-ಗಳ-ಲ್ಲಿಯೂ
ವಿಷಯ-ಗಳ-ಲ್ಲಿ-ರುವ
ವಿಷಯ-ಗಳಾಗಿ-ರ-ಬಹು-ದು-ಧಾರೆ
ವಿಷಯ-ಗಳಿಂದ
ವಿಷಯ-ಗಳಿಗೆ
ವಿಷಯ-ಗಳಿ-ದ್ದರೂ
ವಿಷಯ-ಗಳಿವೆ
ವಿಷಯ-ಗಳಿ-ವೆ-ಒಂದು
ವಿಷಯ-ಗಳು
ವಿಷಯ-ಗಳೂ
ವಿಷಯ-ಗಳೆ-ಲ್ಲವೂ
ವಿಷಯ-ಗಳೇ-ನೇನು
ವಿಷ-ಯದ
ವಿಷಯ-ದಲ್ಲಿ
ವಿಷಯ-ದಲ್ಲಿಯೂ
ವಿಷಯ-ದಲ್ಲಿ-ರುವ
ವಿಷಯ-ದಲ್ಲೂ
ವಿಷಯ-ಮುದ್ರೆ-ಗಳನ್ನು
ವಿಷ-ಯಲ್ಲಿ
ವಿಷಯ-ವನ್ನಾ-ಗಲೀ
ವಿಷಯ-ವನ್ನಾಗಿ
ವಿಷಯ-ವನ್ನು
ವಿಷಯ-ವನ್ನೂ
ವಿಷಯ-ವನ್ನೇ
ವಿಷಯ-ವನ್ನೇ-ನಾ-ದರೂ
ವಿಷಯ-ವಲ್ಲ
ವಿಷಯ-ವಸ್ತು-ಗಳಿವೆ
ವಿಷಯ-ವಾಗಿ
ವಿಷಯ-ವಾಗಿದೆ
ವಿಷಯ-ವಾಗಿಯೂ
ವಿಷಯ-ವಾಗಿಯೇ
ವಿಷಯ-ವಾದ
ವಿಷಯ-ವಾ-ಯಿತು
ವಿಷಯ-ವಾ-ವು-ದೆಂದರೆ
ವಿಷಯ-ವಿದೆ
ವಿಷ-ಯವು
ವಿಷ-ಯವೂ
ವಿಷಯ-ವೆಂದರೆ
ವಿಷಯ-ವೆಂದು
ವಿಷಯ-ವೆಲ್ಲ
ವಿಷ-ಯವೇ
ವಿಷಯ-ವೇ-ನೆಂದರೆ
ವಿಷಯ-ವೊಂದಿದೆ
ವಿಷಯ-ವೊಂದು
ವಿಷಯ-ಸುಖಕ್ಕೆ
ವಿಷ-ವನ್ನು
ವಿಷ-ವಾಗ-ಬಹುದು
ವಿಷವಿಕ್ಕು-ತ್ತಿದೆಯೋ
ವಿಷಾದ
ವಿಷಾದ-ಕರ
ವಿಷಾದ-ವಾಗು
ವಿಷಾದಿ-ಸಿದರು
ವಿಷ್ಣು
ವಿಷ್ಣು-ಮಯ-ವಾಗಿ
ವಿಷ್ಣು-ವಿ-ಗಿಂತ
ವಿಷ್ಣು-ವಿಗೆ
ವಿಷ್ಣು-ವಿ-ರುವ
ವಿಷ್ಣು-ವೆ-ನ್ನು-ವುದೂ
ವಿಷ್ಣುವೋ
ವಿಸ್ತರಿಸ-ಬೇಕು
ವಿಸ್ತ-ರಿಸಿತು
ವಿಸ್ತಾರ-ಗೊಳ್ಳ-ಬೇಕು
ವಿಸ್ತಾರದ
ವಿಸ್ತಾರ-ದಂತೆ
ವಿಸ್ತಾರ-ವಾಗ-ಬೇಕು
ವಿಸ್ತಾರ-ವಾಗಿ
ವಿಸ್ತಾರ-ವಾಗಿದೆ
ವಿಸ್ತಾರ-ವಾಗಿ-ರುವ
ವಿಸ್ತಾರ-ವಾಗು-ವುದು
ವಿಸ್ತಾರ-ವಾದ
ವಿಸ್ತಾರ-ವಾ-ಯಿತು
ವಿಸ್ತಾ-ರವೂ
ವಿಸ್ತೀರ್ಣ-ವಾದ
ವಿಸ್ತೀರ್ಣಾಂ
ವಿಸ್ತೃ-ತವೂ
ವಿಸ್ಮಿತ-ನಾಗಿ
ವಿಹಾ-ರ-ಗಳು
ವಿಹಾ-ರ-ವಾಗಿ-ತ್ತು
ವೀಚಿ-ಗಳಲ್ಲಿ
ವೀತಶೋಕಃ
ವೀರ
ವೀರ-ನಾದ
ವೀರ-ಯು-ಗದ
ವೀರ-ರಂತೆ
ವೀರ-ರಾಗಿ
ವೀರ-ವಾ-ಣಿ-ಯಲ್ಲಿ
ವೀರೋದ್ಯಮದ
ವೀರ್ಯಂ
ವೀರ್ಯ-ದಿಂದ
ವೀರ್ಯ-ವಾ-ನ್
ವುದಕ್ಕೂ
ವುದನ್ನು
ವುದ-ರಿಂದ
ವುದು
ವುದೇನು
ವುದೋ
ವುವು
ವೃಕ್ಷಂ
ವೃಕ್ಷಕ್ಕೇ
ವೃಕ್ಷದ
ವೃಕ್ಷ-ವದು
ವೃಕ್ಷೇ
ವೃತ್ತ-ದಲ್ಲಿಯೇ
ವೃತ್ತ-ಪತ್ರಿಕೆ-ಗಳಲ್ಲಿ
ವೃತ್ತ-ಪತ್ರಿಕೆ-ಯಲ್ಲಿ
ವೃತ್ತಿ
ವೃತ್ತಿ-ಗಳ
ವೃತ್ತಿ-ಗಳಿಗೆ
ವೃತ್ತಿ-ಯನ್ನು
ವೃತ್ತಿ-ಯವ-ನಲ್ಲಿ
ವೃತ್ತಿ-ಯೊಂದನ್ನ-ಲ್ಲದೆ
ವೃಥಾ
ವೃದ್ಧ
ವೃದ್ಧನೂ
ವೃದ್ಧರು
ವೃದ್ಧಿ-ಗೊಂಡು
ವೃದ್ಧಿಗೊಳಿಸ-ಬೇಕು
ವೃದ್ಧಿ-ಪಡಿ-ಸು-ವು-ದ-ಕ್ಕಿಂತ
ವೃದ್ಧಿ-ಯಾಗ-ದ-ವ-ರೆಗೆ
ವೃದ್ಧಿ-ಯಾಗಿ
ವೃದ್ಧಿ-ಯಾಗುತ್ತಾ
ವೃದ್ಧಿ-ಯಾಗು-ವು-ದಕ್ಕೆ
ವೃದ್ಧಿ-ಯಾದ
ವೆ
ವೆಂದರೆ
ವೆಂದ-ರೇನು
ವೆಂದು
ವೆಂಬ
ವೆಂಬು-ದನ್ನು
ವೆಂಬು-ದಲ್ಲ
ವೆಚ್ಚ-ಮಾಡಿ
ವೆನು
ವೆನ್ನಿಸು-ವುದು
ವೇಗ
ವೇಗ-ದಿಂದ
ವೇಗ-ವಾಗಿ
ವೇಗ-ವಾಗಿಯೋ
ವೇತ್ತಾ
ವೇದ
ವೇದ-ಕಾಲದ
ವೇದ-ಕ್ಕಿಂತ
ವೇದಕ್ಕೂ
ವೇದಕ್ಕೆ
ವೇದ-ಗಳ
ವೇದ-ಗಳ-ನ್ನನು-ಸರಿ-ಸುವ
ವೇದ-ಗಳನ್ನು
ವೇದ-ಗಳಲ್ಲಿ
ವೇದ-ಗಳ-ಲ್ಲಿ-ರುವ
ವೇದ-ಗಳಲ್ಲೇ
ವೇದ-ಗಳಿಂದ
ವೇದ-ಗಳಿ-ಗಿ-ರುವ
ವೇದ-ಗಳಿಗೂ
ವೇದ-ಗಳಿಗೆ
ವೇದ-ಗಳಿವೆ
ವೇದ-ಗಳು
ವೇದ-ಗಳೆಂಬ
ವೇದ-ಗಳೆಲ್ಲ
ವೇದ-ಗಳೇ
ವೇದ-ಗಳೊಂದಿಗೆ
ವೇದ-ಜ್ಞಾನ-ರಾಶಿ-ಯನ್ನು
ವೇದದ
ವೇದ-ದಲ್ಲಿ
ವೇದ-ದಲ್ಲಿದೆ
ವೇದ-ದಲ್ಲಿಯೂ
ವೇದ-ದಷ್ಟೇ
ವೇದ-ಧರ್ಮ-ವನ್ನು
ವೇದನೆ
ವೇದ-ಪುರು-ಷನ
ವೇದ-ಪ್ರಣೀತ-ವಾದ
ವೇದ-ಭಾಗ-ಗಳನ್ನು
ವೇದ-ಭಾಗ-ಗಳಾದ
ವೇದ-ಭಾಗ-ಗಳೂ
ವೇದ-ಭಾಗ-ವಾದ
ವೇದ-ಭಾವ-ನೆಗೆ
ವೇದ-ಮಂತ್ರ-ವನ್ನು
ವೇದ-ರಾಶಿ
ವೇದರ್ಷಿ-ಗಳು
ವೇದ-ವನ್ನು
ವೇದವು
ವೇದ-ವೆ-ನ್ನುತ್ತೇವೆ
ವೇದವೇ
ವೇದ-ವ್ಯಾಸ
ವೇದ-ಶಾಖೆ-ಯನ್ನು
ವೇದ-ಶಾಖೆಯೂ
ವೇದ-ಸಂಗ್ರಹ-ಕಾರ-ರನ್ನು
ವೇದಾಂತ
ವೇದಾಂತಕ್ಕೆ
ವೇದಾಂತ-ತತ್ತ್ವದ
ವೇದಾಂತದ
ವೇದಾಂತ-ದಲ್ಲಿ
ವೇದಾಂತ-ದಲ್ಲಿ-ರುವ
ವೇದಾಂತ-ದಲ್ಲೇ
ವೇದಾಂತ-ದಿಂದ
ವೇದಾಂತ-ವನ್ನು
ವೇದಾಂತ-ವನ್ನೇ
ವೇದಾಂತವು
ವೇದಾಂತವೂ
ವೇದಾಂತ-ವೆಂದರೆ
ವೇದಾಂತ-ವೆಂದು
ವೇದಾಂತ-ವೆಂಬ
ವೇದಾಂತ-ಶಾಸ್ತ್ರ
ವೇದಾಂತಿ
ವೇದಾಂತಿ-ಗಳ
ವೇದಾಂತಿ-ಗಳಾದ
ವೇದಾಂತಿ-ಗಳಿಂದ
ವೇದಾಂತಿ-ಗಳಿಗೆ
ವೇದಾಂತಿ-ಗಳು
ವೇದಾಂತಿ-ಗಳೂ
ವೇದಾಂತಿ-ಗಳೆಂದು
ವೇದಾಂತಿ-ಗಳೆಂಬ
ವೇದಾಂತಿಗೆ
ವೇದಾಂತಿ-ಯಾಗಿ-ರು-ವು-ದ-ರಿಂದ
ವೇದಾಂತಿ-ಯಾಗು-ವು-ದಕ್ಕೆ
ವೇದಾಂತಿಯು
ವೇದಾಧ್ಯ-ಯನ
ವೇದಾಧ್ಯಯ-ನಕ್ಕೆ
ವೇದಾಧ್ಯ-ಯನ-ದಲ್ಲಿ
ವೇದಾಧ್ಯ-ಯನ-ದಿಂದಲೂ
ವೇದಾಧ್ಯಯ-ನವು
ವೇದಿಕೆ-ಗಳ
ವೇದಿ-ಕೆಯ
ವೇದೇತಿ
ವೇದೇಭ್ಯೋ
ವೇದೋಕ್ತಿಯೇ
ವೇದೋಪ
ವೇದ್ಯ
ವೇದ್ಯ-ವಾಗಿದೆ
ವೇದ್ಯ-ವಾಗಿ-ರ-ಬಹುದು
ವೇದ್ಯ-ವಾಗುವಂತಾ-ಯಿತು
ವೇದ್ಯ-ವಾಗು-ವುದು
ವೇಳೆ
ವೇಳೆ-ತಾರ-ಸ್ವರ-ಕ್ಕೇರು-ತ್ತದೆ
ವೇಳೆ-ಯಲ್ಲ
ವೇಶ್ಯಾ-ವೃ-ತ್ತಿಗೆ
ವೇಶ್ಯೆ
ವೇಷ-ಗಳ-ನ್ನೆಲ್ಲಾ
ವೈ
ವೈಕುಂಠಕ್ಕೆ
ವೈಚಾರಿ-ಕತೆ-ಯನ್ನು
ವೈಚಾ-ರಿಕವೂ
ವೈಚಿತ್ರ್ಯ
ವೈಚಿತ್ರ್ಯಾ-ದೃಜುಕುಟಿಲ
ವೈಜ್ಞಾನಿಕ
ವೈಜ್ಞಾನಿ-ಕರ
ವೈಜ್ಞಾನಿ-ಕರೂ
ವೈಜ್ಞಾನಿ-ಕ-ವಾಗಿದೆ
ವೈಜ್ಞಾನಿ-ಕ-ವಾದು-ದೆಂದು
ವೈಜ್ಞಾನಿ-ಕವೆ
ವೈದಿಕ
ವೈದಿ-ಕರು
ವೈದಿಕ-ರೆಂದು
ವೈದುಷ್ಯಂ
ವೈದ್ಯರ
ವೈಭವ-ಗಳ
ವೈಭವದ
ವೈಭವ-ದಲ್ಲಿಯೇ
ವೈಭವ-ದಿಂದ
ವೈಭವ-ಪೂರ್ಣ-ಳಾಗಿ
ವೈಭವ-ಪೂರ್ಣ-ವಾದ
ವೈಭವ-ಯುತ-ವಾದ
ವೈಭವ-ವನ್ನು
ವೈಭ-ವಾ-ನ್ವಿತ
ವೈಭ-ವೋ-ಪೇತ
ವೈಮನ-ಸ್ಯ-ವಿಲ್ಲ
ವೈಮನ-ಸ್ಸ-ಗಳನ್ನು
ವೈಯಕ್ತಿಕ
ವೈಯಕ್ತಿ-ಕತೆ
ವೈಯಕ್ತಿಕ-ತೆ-ಗಳು
ವೈಯಕ್ತಿಕ-ಲಾಭ-ಕ್ಕಾ-ಗಲೀ
ವೈಯಕ್ತಿಕ-ವಾಗಿ
ವೈಯುಕ್ತಿ-ಕತೆ
ವೈರಾಗ್ಯ
ವೈರಾಗ್ಯದ
ವೈರಾಗ್ಯ-ದಿಂದಲೇ
ವೈರಾಗ್ಯ-ಮೇ-ವಾ-ಭಯಂ
ವೈರಾಗ್ಯ-ವನ್ನು
ವೈರಾಗ್ಯ-ವೊಂದೇ
ವೈರಿ-ಯ-ನ್ನಾಗಿ
ವೈವಿಧ್ಯ
ವೈವಿಧ್ಯಕ್ಕೆ
ವೈವಿಧ್ಯ-ಗಳ
ವೈವಿಧ್ಯದ
ವೈವಿಧ್ಯ-ದಲ್ಲಿ
ವೈವಿಧ್ಯ-ವನ್ನು
ವೈವಿಧ್ಯವೆ
ವೈವಿಧ್ಯ-ವೆಲ್ಲಾ
ವೈಶಾಲ್ಯ
ವೈಶಾಲ್ಯ-ದ-ಮತ್ತು
ವೈಶಾಲ್ಯ-ದಲ್ಲಿ
ವೈಶಿಷ್ಟ-ವೆಂದರೆ
ವೈಶಿಷ್ಟ್ಯ
ವೈಶಿಷ್ಟ್ಯ-ಗಳಿವೆ
ವೈಶಿಷ್ಟ್ಯ-ವನ್ನು
ವೈಶಿಷ್ಟ್ಯ-ವಿದೆ
ವೈಶಿಷ್ಟ್ಯ-ವಿಲ್ಲ
ವೈಶಿಷ್ಟ್ಯವು
ವೈಶಿಷ್ಟ್ಯ-ವೆಲ್ಲಿ
ವೈಶ್ಯ
ವೈಶ್ಯರು
ವೈಶ್ಯ-ರೆಂಬ
ವೈಶ್ಯಾ-ಸ್ತಥಾ
ವೈಷ್ಣವ
ವೈಷ್ಣ-ವನೇ
ವೈಷ್ಣವ-ಮಿತಿ
ವೈಷ್ಣ-ವರ
ವೈಷ್ಣವ-ರಾಗಲಿ
ವೈಷ್ಣವ-ರಾಗಲೀ
ವೈಷ್ಣ-ವರು
ವೈಷ್ಣ-ವರೂ
ವೈಷ್ಣ-ವಾ-ಚಾರ್ಯರ
ವೊಂದಿದೆ-ಪಾಶ್ಚಾತ್ಯ
ವೋ
ವ್ಯಕ್ತ
ವ್ಯಕ್ತ-ಗೊಳಿಸ-ಬೇಕಾಗಿದೆ
ವ್ಯಕ್ತ-ಗೊಳಿ-ಸಲು
ವ್ಯಕ್ತ-ಗೊಳಿಸಿ
ವ್ಯಕ್ತ-ಗೊಳಿಸಿದೆ
ವ್ಯಕ್ತ-ಗೊಳಿಸಿ-ದ್ದಾನೆ
ವ್ಯಕ್ತ-ಗೊಳಿಸು-ತ್ತಾರೆ
ವ್ಯಕ್ತ-ಗೊಳಿಸು-ವುದು
ವ್ಯಕ್ತ-ಪಡಿ-ಸದೆ
ವ್ಯಕ್ತ-ಪಡಿ-ಸ-ಬಹುದು
ವ್ಯಕ್ತ-ಪಡಿ-ಸ-ಬಾ-ರದು
ವ್ಯಕ್ತ-ಪಡಿ-ಸ-ಬೇಕಷ್ಟೆ
ವ್ಯಕ್ತ-ಪಡಿ-ಸಲಾರ-ರೆಂಬುದು
ವ್ಯಕ್ತ-ಪಡಿ-ಸ-ಲಾರೆ
ವ್ಯಕ್ತ-ಪಡಿ-ಸಲು
ವ್ಯಕ್ತ-ಪಡಿ-ಸಲೇ-ಬೇಕು
ವ್ಯಕ್ತ-ಪಡಿಸಿ
ವ್ಯಕ್ತ-ಪಡಿ-ಸಿ-ರುವ
ವ್ಯಕ್ತ-ಪಡಿ-ಸಿ-ರು-ವರು
ವ್ಯಕ್ತ-ಪಡಿ-ಸು-ತ್ತಿದ್ದನು
ವ್ಯಕ್ತ-ಪಡಿ-ಸು-ತ್ತಿ-ರಲಿಲ್ಲ
ವ್ಯಕ್ತ-ಪಡಿ-ಸು-ತ್ತೇನೆ
ವ್ಯಕ್ತ-ಪಡಿ-ಸುವ
ವ್ಯಕ್ತ-ಪಡಿ-ಸುವರು
ವ್ಯಕ್ತ-ಪಡಿ-ಸುವ-ವರು
ವ್ಯಕ್ತ-ಪಡಿ-ಸು-ವು-ದಕ್ಕೆ
ವ್ಯಕ್ತ-ಪಡಿ-ಸು-ವು-ದಿಲ್ಲ
ವ್ಯಕ್ತ-ಪಡಿ-ಸುವು-ದೆಂದೂ
ವ್ಯಕ್ತ-ಪಡಿ-ಸು-ವುದೇ
ವ್ಯಕ್ತ-ಪಡಿ-ಸು-ವುವು
ವ್ಯಕ್ತ-ಪಡಿ-ಸು-ವೆನು
ವ್ಯಕ್ತ-ಪ್ರ-ಪಂಚ
ವ್ಯಕ್ತ-ರೂಪ-ವಾದ
ವ್ಯಕ್ತ-ವಾಗದ
ವ್ಯಕ್ತ-ವಾಗಬೇಕಷ್ಟೇ
ವ್ಯಕ್ತ-ವಾಗ-ಬೇಕು
ವ್ಯಕ್ತ-ವಾಗಲೆಣಿ-ಸುವ
ವ್ಯಕ್ತ-ವಾಗಿ
ವ್ಯಕ್ತ-ವಾಗಿದೆ
ವ್ಯಕ್ತ-ವಾಗಿ-ರಲಿ
ವ್ಯಕ್ತ-ವಾಗಿ-ರು-ವಂತೆ
ವ್ಯಕ್ತ-ವಾಗಿ-ರು-ವುವು
ವ್ಯಕ್ತ-ವಾಗಿಲ್ಲ
ವ್ಯಕ್ತ-ವಾಗಿ-ಲ್ಲ-ವೆಂದು
ವ್ಯಕ್ತ-ವಾಗಿವೆ
ವ್ಯಕ್ತ-ವಾಗುತ್ತ
ವ್ಯಕ್ತ-ವಾಗುತ್ತದೆ
ವ್ಯಕ್ತ-ವಾಗುತ್ತವೆ
ವ್ಯಕ್ತ-ವಾಗುತ್ತಾ
ವ್ಯಕ್ತ-ವಾಗು-ತ್ತಿದೆ
ವ್ಯಕ್ತ-ವಾಗುತ್ತಿ-ರುವ
ವ್ಯಕ್ತ-ವಾಗುತ್ತಿರು-ವುದು
ವ್ಯಕ್ತ-ವಾಗು-ತ್ತಿವೆ
ವ್ಯಕ್ತ-ವಾಗು-ವುದು
ವ್ಯಕ್ತ-ವಾಗು-ವುದೋ
ವ್ಯಕ್ತ-ವಾಗು-ವುವು
ವ್ಯಕ್ತ-ವಾದ-ಅಮೀಬ-ವಾದರೆ
ವ್ಯಕ್ತಿ
ವ್ಯಕ್ತಿ-ಗತ
ವ್ಯಕ್ತಿ-ಗಲ್ಲ
ವ್ಯಕ್ತಿ-ಗಳ
ವ್ಯಕ್ತಿ-ಗಳಂತೆ
ವ್ಯಕ್ತಿ-ಗಳ-ನ್ನಾಗಿ
ವ್ಯಕ್ತಿ-ಗಳನ್ನು
ವ್ಯಕ್ತಿ-ಗಳ-ನ್ನೆ-ಲ್ಲಾ-ಸ್ವೀಕರಿ-ಸು-ವಷ್ಟು
ವ್ಯಕ್ತಿ-ಗಳ-ಲ್ಲ-ವೆಂದು
ವ್ಯಕ್ತಿ-ಗಳಲ್ಲಿ
ವ್ಯಕ್ತಿ-ಗಳ-ಲ್ಲಿಯೂ
ವ್ಯಕ್ತಿ-ಗಳಾಗಿಲ್ಲ
ವ್ಯಕ್ತಿ-ಗಳಿಂದ
ವ್ಯಕ್ತಿ-ಗಳಿ-ಗಿಂತ
ವ್ಯಕ್ತಿ-ಗಳಿಗೂ
ವ್ಯಕ್ತಿ-ಗಳಿಗೆ
ವ್ಯಕ್ತಿ-ಗಳಿ-ಗೆಲ್ಲಾ
ವ್ಯಕ್ತಿ-ಗಳಿ-ದ್ದರೂ
ವ್ಯಕ್ತಿ-ಗಳಿ-ರು-ವರು
ವ್ಯಕ್ತಿ-ಗಳು
ವ್ಯಕ್ತಿ-ಗಳೂ
ವ್ಯಕ್ತಿ-ಗಳೆಲ್ಲ
ವ್ಯಕ್ತಿ-ಗಳೆ-ಲ್ಲ-ರಿ-ಗಿಂತ
ವ್ಯಕ್ತಿ-ಗಳೆ-ಲ್ಲರೂ
ವ್ಯಕ್ತಿ-ಗಳೆಲ್ಲಿ
ವ್ಯಕ್ತಿ-ಗಳೇ
ವ್ಯಕ್ತಿ-ಗಿ-ರು-ವಂತೆಯೇ
ವ್ಯಕ್ತಿಗೂ
ವ್ಯಕ್ತಿಗೆ
ವ್ಯಕ್ತಿತ್ವ
ವ್ಯಕ್ತಿ-ತ್ವದ
ವ್ಯಕ್ತಿ-ತ್ವ-ವನ್ನು
ವ್ಯಕ್ತಿ-ತ್ವ-ವಿರು-ವುದು
ವ್ಯಕ್ತಿ-ತ್ವವು
ವ್ಯಕ್ತಿ-ತ್ವವೂ
ವ್ಯಕ್ತಿ-ತ್ವವೇ
ವ್ಯಕ್ತಿ-ನಿಷ್ಠ-ವಲ್ಲ
ವ್ಯಕ್ತಿಯ
ವ್ಯಕ್ತಿ-ಯಂತೆ
ವ್ಯಕ್ತಿ-ಯನ್ನು
ವ್ಯಕ್ತಿ-ಯಲ್ಲ
ವ್ಯಕ್ತಿ-ಯಷ್ಟು
ವ್ಯಕ್ತಿ-ಯಾಗಿ-ದ್ದಾನೆ
ವ್ಯಕ್ತಿ-ಯಿಂದ
ವ್ಯಕ್ತಿಯು
ವ್ಯಕ್ತಿಯೂ
ವ್ಯಕ್ತಿಯೇ
ವ್ಯಕ್ತಿ-ಯೊಬ್ಬನು
ವ್ಯಕ್ತಿ-ಯೊಬ್ಬರ
ವ್ಯತ್ಯಸ್ತ-ವಾದರೂ
ವ್ಯತ್ಯಾಸ
ವ್ಯತ್ಯಾ-ಸಕ್ಕೆ
ವ್ಯತ್ಯಾ-ಸ-ಗಳು
ವ್ಯತ್ಯಾ-ಸ-ಮಾಡಿ
ವ್ಯತ್ಯಾ-ಸ-ವನ್ನಾ-ಗಲಿ
ವ್ಯತ್ಯಾ-ಸ-ವನ್ನು
ವ್ಯತ್ಯಾ-ಸ-ವನ್ನೂ
ವ್ಯತ್ಯಾ-ಸ-ವಷ್ಟೆ
ವ್ಯತ್ಯಾ-ಸ-ವಾಗದೆ
ವ್ಯತ್ಯಾ-ಸ-ವಾಗಿ-ದ್ದರೂ
ವ್ಯತ್ಯಾ-ಸ-ವಾಗು-ವುದು
ವ್ಯತ್ಯಾ-ಸ-ವಾದರೂ
ವ್ಯತ್ಯಾ-ಸ-ವಿದೆ
ವ್ಯತ್ಯಾ-ಸ-ವಿರ-ಬಹುದು
ವ್ಯತ್ಯಾ-ಸ-ವಿರು-ವುದು
ವ್ಯತ್ಯಾ-ಸ-ವಿಲ್ಲ
ವ್ಯತ್ಯಾ-ಸ-ವಿಷ್ಟೆ
ವ್ಯತ್ಯಾ-ಸ-ವುಳ್ಳ
ವ್ಯತ್ಯಾ-ಸ-ವೆಂದರೆ
ವ್ಯತ್ಯಾ-ಸವೇ
ವ್ಯತ್ಯಾ-ಸ-ವೇನು
ವ್ಯತ್ಯಾ-ಸ-ವೇ-ನೆಂದರೆ
ವ್ಯಥೆ
ವ್ಯಥೆ-ಪಟ್ಟಿ-ದ್ದಾರೆ
ವ್ಯಥೆ-ಪಡ-ಬೇಕಾಗಿದೆ
ವ್ಯಥೆ-ಪಡುತ್ತಿರು-ವೆವು
ವ್ಯಥೆ-ಪ-ಡುವನು
ವ್ಯಥೆ-ಪಡು-ವಾಗ
ವ್ಯಥೆ-ಯನ್ನು
ವ್ಯಥೆ-ಯನ್ನುಂಟು
ವ್ಯಥೆ-ಯಾಗು-ವು-ದಿಲ್ಲ
ವ್ಯಥೆ-ಯಾಗು-ವುದು
ವ್ಯಥೆ-ಯಿಲ್ಲ
ವ್ಯಭಿ-ಚಾರ-ವನ್ನು
ವ್ಯಯ-ವಾಗಿ
ವ್ಯರ್ಥ
ವ್ಯರ್ಥ-ಗೊಳಿ-ಸದೆ
ವ್ಯರ್ಥ-ಗೊಳಿ-ಸಿದ
ವ್ಯರ್ಥ-ಮಾಡ-ಬೇಡಿ
ವ್ಯರ್ಥ-ವಾಗು-ತ್ತಿದೆ
ವ್ಯರ್ಥ-ವಾಗು-ವು-ದಿಲ್ಲ
ವ್ಯರ್ಥಾಲಾ-ಪ-ಗಳಲ್ಲ
ವ್ಯರ್ಥಾಲಾಪ-ವನ್ನು
ವ್ಯವಸ್ಥೆ
ವ್ಯವ-ಸ್ಥೆ-ಗಿಂತ
ವ್ಯವ-ಸ್ಥೆ-ಗೊಳಿಸುವ
ವ್ಯವ-ಸ್ಥೆಯ
ವ್ಯವ-ಸ್ಥೆಯು
ವ್ಯವ-ಹಾರ
ವ್ಯವ-ಹಾ-ರ-ಗಳ
ವ್ಯವ-ಹಾ-ರ-ಗಳನ್ನು
ವ್ಯವ-ಹಾ-ರ-ಗಳನ್ನೂ
ವ್ಯವ-ಹಾ-ರ-ಗಳ-ಲ್ಲಿಯೂ
ವ್ಯವ-ಹಾ-ರ-ಗಳಿಗೆ
ವ್ಯವ-ಹಾ-ರ-ಗಳಿವೆ
ವ್ಯವ-ಹಾ-ರ-ಗಳು
ವ್ಯವ-ಹಾ-ರ-ಗಳೇ
ವ್ಯವ-ಹಾ-ರ-ದಲ್ಲಿ
ವ್ಯಷ್ಟಿ-ಯಲ್ಲಿ
ವ್ಯಸನ-ದಿಂದ
ವ್ಯಸನ-ವಾಗು-ತ್ತದೆ
ವ್ಯಾ-ಕರಣ
ವ್ಯಾ-ಕರ-ಣದ
ವ್ಯಾ-ಕರ-ಣ-ದಲ್ಲಿ
ವ್ಯಾ-ಕರ-ಣ-ವನ್ನು
ವ್ಯಾ-ಕರ-ಣವೇ
ವ್ಯಾ-ಕುಲ-ಪ-ಡುವನು
ವ್ಯಾ-ಖ್ಯಾನ
ವ್ಯಾ-ಖ್ಯಾ-ನ-ಕಾರ
ವ್ಯಾ-ಖ್ಯಾ-ನ-ಗಳನ್ನು
ವ್ಯಾ-ಖ್ಯಾ-ನ-ಗಳಿಂದಲೇ
ವ್ಯಾ-ಖ್ಯಾ-ನ-ಗಳೂ
ವ್ಯಾ-ಖ್ಯಾ-ನ-ಮಾಡುವ
ವ್ಯಾ-ಖ್ಯಾ-ನ-ಮಾಡು-ವಾಗಲೂ
ವ್ಯಾ-ಖ್ಯಾ-ನ-ವನ್ನು
ವ್ಯಾ-ಖ್ಯಾ-ನ-ವಾದ
ವ್ಯಾ-ಖ್ಯಾ-ನ-ವಿದೆ
ವ್ಯಾ-ಖ್ಯಾ-ನಿಸ-ಬೇಕು
ವ್ಯಾ-ಖ್ಯಾ-ನಿಸಿ
ವ್ಯಾ-ಖ್ಯಾ-ನಿ-ಸು-ವು-ದರ
ವ್ಯಾ-ಖ್ಯಾ-ನು-ಸಾರ
ವ್ಯಾ-ಖ್ಯಾ-ಸ್ಯಂತೇ
ವ್ಯಾ-ಘ್ರ-ಗಳಂತೆ
ವ್ಯಾಜ್ಯ
ವ್ಯಾ-ಜ್ಯಕ್ಕೆ
ವ್ಯಾ-ಜ್ಯ-ವಾ-ಡು-ವು-ದಿಲ್ಲ
ವ್ಯಾ-ಧಿ-ಗಳು
ವ್ಯಾ-ಪಕ-ವಾಗಿದೆ
ವ್ಯಾ-ಪಾರ
ವ್ಯಾ-ಪಾ-ರಕ್ಕೆ
ವ್ಯಾ-ಪಾರದ
ವ್ಯಾ-ಪಾರ-ದಿಂದ
ವ್ಯಾ-ಪಾರ-ವಷ್ಟೆ
ವ್ಯಾ-ಪಾರ-ವಾಗ-ಬಾ-ರದು
ವ್ಯಾ-ಪಾರ-ವಾಗಿದೆ
ವ್ಯಾ-ಪಾರಿ-ಗಳು
ವ್ಯಾ-ಪಾ-ರಿಗೆ
ವ್ಯಾ-ಪಿ-ಯಾದ
ವ್ಯಾ-ಪಿ-ಸಲಿ
ವ್ಯಾ-ಪಿಸಿ
ವ್ಯಾ-ಪಿ-ಸಿ-ತ್ತು
ವ್ಯಾ-ಪಿ-ಸಿದ
ವ್ಯಾ-ಪಿ-ಸಿದೆ
ವ್ಯಾ-ಪಿ-ಸಿ-ದೆಯೆ
ವ್ಯಾ-ಪಿ-ಸಿ-ರು-ತ್ತದೆ
ವ್ಯಾ-ಪಿ-ಸಿ-ರು-ವು-ದ-ರಿಂದ
ವ್ಯಾ-ಪಿ-ಸಿ-ರು-ವುದು
ವ್ಯಾ-ಪಿ-ಸುವ
ವ್ಯಾ-ಪಿ-ಸು-ವಷ್ಟು
ವ್ಯಾ-ಪ್ತಿಯೂ
ವ್ಯಾ-ಮೋಹ-ವನ್ನು
ವ್ಯಾ-ವ-ಹಾ-ರಿಕ
ವ್ಯಾಸ
ವ್ಯಾ-ಸ-ಅರ್ಜುನರ
ವ್ಯಾ-ಸ-ತ-ನಯ
ವ್ಯಾ-ಸ-ದರ್ಶನ-ದಲ್ಲಿ
ವ್ಯಾ-ಸ-ನಿಗೆ
ವ್ಯಾ-ಸರ
ವ್ಯಾ-ಸ-ರ-ನ್ನಾ-ದರೂ
ವ್ಯಾ-ಸರು
ವ್ಯಾ-ಸ-ಸೂತ್ರ
ವ್ಯಾ-ಸ-ಸೂತ್ರ-ಗಳನ್ನು
ವ್ಯಾ-ಸ-ಸೂತ್ರ-ಗಳಿಗೆ
ವ್ಯಾ-ಸ-ಸೂತ್ರದ
ವ್ಯಾ-ಸ-ಸೂತ್ರ-ದಲ್ಲಿ
ವ್ಯೂಹ-ಗಳನ್ನು
ವ್ಯೂಹ-ದಲ್ಲಿ
ವ್ಯೂಹ-ರಚನೆ
ವ್ರಣ-ಗಳನ್ನು
ವ್ರಣ-ಗಳಿವೆ
ಶಂಕರ
ಶಂಕರ-ನನ್ನೇ
ಶಂಕರ-ರಿ-ಗಿಂತ
ಶಂಕ-ರರು
ಶಂಕರಾ-ಚಾರ್ಯ
ಶಂಕರಾ-ಚಾರ್ಯರ
ಶಂಕರಾ-ಚಾರ್ಯ-ರ-ನ್ನಾ-ದರೂ
ಶಂಕರಾ-ಚಾರ್ಯ-ರನ್ನು
ಶಂಕರಾ-ಚಾರ್ಯ-ರಲ್ಲಿ
ಶಂಕರಾ-ಚಾರ್ಯ-ರಾಗ
ಶಂಕರಾ-ಚಾರ್ಯ-ರಿ-ಗಿಂತ
ಶಂಕರಾ-ಚಾರ್ಯ-ರಿಗೆ
ಶಂಕರಾ-ಚಾರ್ಯರು
ಶಂಕರಾ-ಚಾರ್ಯರೇ
ಶಂಕಿಸುತ್ತಿರು-ವೆನು
ಶಕು-ನ-ವನ್ನೂ
ಶಕ್ತ-ನಾಗು
ಶಕ್ತ-ನಾದ-ವನೊ-ಬ್ಬ-ನಿಗೆ
ಶಕ್ತ-ಶಾಲಿ
ಶಕ್ತಿ
ಶಕ್ತಿ-ಕ್ಷಯ-ದಿಂದ-ಸದ್ಯೋ-ನಿರ್ಮಿತ
ಶಕ್ತಿ-ಗಣಿ
ಶಕ್ತಿ-ಗನು
ಶಕ್ತಿ-ಗಳ
ಶಕ್ತಿ-ಗಳನ್ನು
ಶಕ್ತಿ-ಗಳನ್ನೂ
ಶಕ್ತಿ-ಗಳು
ಶಕ್ತಿ-ಗಳೂ
ಶಕ್ತಿಗೆ
ಶಕ್ತಿ-ದಾಯಕ
ಶಕ್ತಿ-ಪೂರ್ಣ
ಶಕ್ತಿ-ಪೂರ್ಣ-ವಾದ
ಶಕ್ತಿ-ಬಾ-ಹುಳ್ಯ
ಶಕ್ತಿ-ಬೀಜ-ವನ್ನು
ಶಕ್ತಿ-ಭಾವವು
ಶಕ್ತಿ-ಮಂತ್ರವು
ಶಕ್ತಿಯ
ಶಕ್ತಿ-ಯಂತೆ
ಶಕ್ತಿ-ಯ-ನ್ನಿತ್ತು
ಶಕ್ತಿ-ಯನ್ನು
ಶಕ್ತಿ-ಯ-ನ್ನು-ತುಂಬ
ಶಕ್ತಿ-ಯನ್ನೂ
ಶಕ್ತಿ-ಯ-ನ್ನೆಲ್ಲಾ
ಶಕ್ತಿ-ಯ-ನ್ನೇ
ಶಕ್ತಿ-ಯಲ್ಲಿ
ಶಕ್ತಿ-ಯ-ಲ್ಲಿದೆ
ಶಕ್ತಿ-ಯಷ್ಟೇ
ಶಕ್ತಿ-ಯಾಗಲಿ
ಶಕ್ತಿ-ಯಾಗಿ
ಶಕ್ತಿ-ಯಾಗಿ-ರ-ಬಹುದು
ಶಕ್ತಿ-ಯಾದ
ಶಕ್ತಿ-ಯಿಂದ
ಶಕ್ತಿ-ಯಿಂದಲೇ
ಶಕ್ತಿ-ಯಿಂದಾ-ಗಲಿ
ಶಕ್ತಿ-ಯಿದೆ
ಶಕ್ತಿಯು
ಶಕ್ತಿ-ಯು-ತ-ವಾದದ್ದು
ಶಕ್ತಿ-ಯುಳ್ಳ
ಶಕ್ತಿಯೂ
ಶಕ್ತಿ-ಯೆಲ್ಲ
ಶಕ್ತಿ-ಯೆಲ್ಲಾ
ಶಕ್ತಿಯೇ
ಶಕ್ತಿ-ಯೊಂದ-ರಲ್ಲಿ
ಶಕ್ತಿ-ಯೊಂದಿಗೆ
ಶಕ್ತಿ-ಯೊಂದಿದೆ
ಶಕ್ತಿ-ಯೊಡನೆ
ಶಕ್ತಿ-ಲಾಭ
ಶಕ್ತಿ-ವಂತ-ರಾಗಿ-ರು-ವುದು
ಶಕ್ತಿ-ವಂತ-ರಾಗು-ವಿರಿ
ಶಕ್ತಿ-ವಂತರೂ
ಶಕ್ತಿ-ವಂತ-ರೆಂದು
ಶಕ್ತಿ-ವರ್ಧಕ
ಶಕ್ತಿ-ಶಾಲಿ
ಶಕ್ತಿ-ಶಾಲಿ-ಗಳಾಗು-ವೆವು
ಶಕ್ತಿ-ಶಾಲಿ-ಗಳು
ಶಕ್ತಿ-ಸಾಗರ
ಶಕ್ತಿ-ಸ್ಥಾಯಿತ್ವ
ಶಕ್ತಿ-ಹೀನ
ಶಕ್ಯ-ವಿಲ್ಲ
ಶತ-ಮಾನ
ಶತ-ಮಾ-ನ-ಗಳ
ಶತ-ಮಾ-ನ-ಗಳ-ವ-ರೆಗೆ
ಶತ-ಮಾ-ನ-ಗಳಾದರೂ
ಶತ-ಮಾ-ನ-ಗಳಿಂದ
ಶತ-ಮಾ-ನ-ಗಳಿಂದಲೂ
ಶತ-ಮಾ-ನ-ಗಳು
ಶತ-ಮಾ-ನದ
ಶತ-ಮಾ-ನ-ದಲ್ಲಿ
ಶತ-ಮಾ-ನ-ದಲ್ಲಿಯೂ
ಶತ-ಮುಖ-ವಾಗಿ
ಶತ-ಶತ-ಮಾ-ನ-ಗಳ
ಶತ-ಶತ-ಮಾ-ನ-ಗಳಿಂದಲೂ
ಶತಶತಾಬ್ದಿ-ಗಳ
ಶತಾಬ್ದ-ಗಳಿಂದ
ಶತಾಯುರ್ವೈ
ಶತ್ರು-ಗಳಾಗ-ಬೇಕು
ಶತ್ರು-ಗಳೇ
ಶತ್ರು-ಗಳೊಡನೆ
ಶತ್ರುತ್ವವನ್ನೆಲ್ಲ
ಶಪಥ-ವ-ಗಲಿ
ಶಪಿ-ಸದೆ
ಶಪಿ-ಸ-ಬಲ್ಲೆ
ಶಪಿ-ಸಿ-ದರೂ
ಶಪಿ-ಸುವ
ಶಪಿ-ಸು-ವು-ದಿಲ್ಲ
ಶಬ್ದ
ಶಬ್ದಕ್ಕೆ
ಶಬ್ದ-ಗಳನ್ನು
ಶಬ್ದ-ಗಳಲ್ಲಿ
ಶಬ್ದ-ಗಳಲ್ಲೇ
ಶಬ್ದ-ಗಳೂ
ಶಬ್ದ-ಜಾಲ-ದಿಂದ
ಶಬ್ದ-ಝರೀ
ಶಬ್ದದ
ಶಬ್ದ-ವನ್ನಾ-ದರೂ
ಶಬ್ದ-ವನ್ನು
ಶಬ್ದವು
ಶಬ್ದವೂ
ಶಬ್ದವೇ
ಶಬ್ದಾರ್ಥದ
ಶಮ-ಗೊಳಿಸುವ
ಶಮ-ನಕ್ಕೆ
ಶರಣಾಗಿ
ಶರಣು-ಮಾಡಿ-ಕೊಳ್ಳುವ
ಶರಾಬು
ಶರೀರ
ಶರೀರ-ಗಳಲ್ಲಿ
ಶರೀರ-ಗಳೆಂಬ
ಶರೀರ-ದಂಡನೆ
ಶರೀರ-ದಲ್ಲಿ
ಶರೀರ-ದೊಂದಿಗೆ
ಶರೀರ-ವನ್ನು
ಶರೀರ-ವಿಂದು
ಶರೀರ-ವಿದೆ
ಶರೀರ-ವಿರು-ವುದು
ಶರೀರವು
ಶರೀರ-ಶಾಸ್ತ್ರ-ಜ್ಞರು
ಶರೀರ-ಶಾಸ್ತ್ರದ
ಶರೀರ-ಶಾಸ್ತ್ರವು
ಶವದ
ಶವ-ದಂತೆ
ಶಸ್ತ್ರವೂ
ಶಸ್ತ್ರಾಣಿ
ಶಹಭಾಷ್ಗಿರಿ-ಯಿಂದ
ಶಾಂಡಿಲ್ಯರು
ಶಾಂತ-ಚಿತ್ತ-ದಿಂದ
ಶಾಂತನೂ
ಶಾಂತ-ಭಾವ
ಶಾಂತ-ವಾಗಿ
ಶಾಂತ-ವಾಗು-ತ್ತದೆ
ಶಾಂತ-ವಾದ
ಶಾಂತಿ
ಶಾಂತಿಃ
ಶಾಂತಿ-ಇವು-ಗಳಲ್ಲಿ
ಶಾಂತಿ-ಗಳ
ಶಾಂತಿ-ದಾಯ-ಕ-ವಾದ
ಶಾಂತಿ-ಪರಾ-ಯಣ
ಶಾಂತಿಯ
ಶಾಂತಿ-ಯನ್ನು
ಶಾಂತಿ-ಯಿಂದ
ಶಾಕಾ-ಹಾ-ರ-ದಿಂದ
ಶಾಕ್ತ
ಶಾಕ್ತ-ರಾಗಲಿ
ಶಾಕ್ತರೂ
ಶಾಕ್ಯಮುನಿ
ಶಾಖ
ಶಾಖ-ವಾದ
ಶಾಖೆ-ಗಳಲ್ಲಿ
ಶಾಖೆ-ಗಳಿವೆ
ಶಾಖೆ-ಗಳು
ಶಾಖೆಯ-ವರೂ
ಶಾಖೋಪಶಾಖೆ-ಗಳಿಂದ
ಶಾಖೋಪ-ಶಾಖೆ-ಗಳುಳ್ಳ
ಶಾಖೋಪಶಾಖೆ-ಗಳೆಲ್ಲ
ಶಾಪ
ಶಾಪದ
ಶಾಪ-ವನ್ನು
ಶಾಪ-ವಾಗಿ-ರುವ
ಶಾರೀ-ರಕ
ಶಾರೀ-ರಿಕ
ಶಾಲಾ
ಶಾಲೆಗೆ
ಶಾಲೆ-ಯಿಂದ
ಶಾಶ್ವತ
ಶಾಶ್ವತ-ವಲ್ಲ
ಶಾಶ್ವತ-ವಾಗಿ
ಶಾಶ್ವತ-ವಾಗಿಯೂ
ಶಾಶ್ವತ-ವಾಗಿ-ರು-ವಂತೆ
ಶಾಶ್ವತ-ವಾದ
ಶಾಶ್ವತ-ವಾದವು
ಶಾಸನ
ಶಾಸನ-ಗಳನ್ನೂ
ಶಾಸನ-ದಲ್ಲಿ
ಶಾಸನ-ಬದ್ಧ
ಶಾಸ್ತ್ರ
ಶಾಸ್ತ್ರ-ಕಾರ-ರಿಗೆ
ಶಾಸ್ತ್ರಕ್ಕೂ
ಶಾಸ್ತ್ರಕ್ಕೆ
ಶಾಸ್ತ್ರ-ಗಳ
ಶಾಸ್ತ್ರ-ಗಳಂತೆ
ಶಾಸ್ತ್ರ-ಗಳನ್ನು
ಶಾಸ್ತ್ರ-ಗಳಲ್ಲ
ಶಾಸ್ತ್ರ-ಗಳಲ್ಲಿ
ಶಾಸ್ತ್ರ-ಗಳ-ಲ್ಲಿಯೇ
ಶಾಸ್ತ್ರ-ಗಳ-ಲ್ಲಿ-ರುವ
ಶಾಸ್ತ್ರ-ಗಳ-ಲ್ಲೆಲ್ಲಾ
ಶಾಸ್ತ್ರ-ಗಳಿಂದ
ಶಾಸ್ತ್ರ-ಗಳಿಗೂ
ಶಾಸ್ತ್ರ-ಗಳಿಗೆ
ಶಾಸ್ತ್ರ-ಗಳಿವೆ
ಶಾಸ್ತ್ರ-ಗಳು
ಶಾಸ್ತ್ರ-ಗಳೆಲ್ಲ
ಶಾಸ್ತ್ರ-ಗಳೆಲ್ಲಾ
ಶಾಸ್ತ್ರ-ಗಳೇ
ಶಾಸ್ತ್ರ-ಗಳೇನು
ಶಾಸ್ತ್ರ-ಗ್ರಂಥ
ಶಾಸ್ತ್ರ-ಗ್ರಂಥ-ಗಳು
ಶಾಸ್ತ್ರ-ಜ್ಞ-ರನ್ನು
ಶಾಸ್ತ್ರ-ಜ್ಞ-ರಿಗೆ
ಶಾಸ್ತ್ರ-ಜ್ಞ-ರಿಗೇ
ಶಾಸ್ತ್ರ-ಜ್ಞರು
ಶಾಸ್ತ್ರ-ಜ್ಞಾನ
ಶಾಸ್ತ್ರದ
ಶಾಸ್ತ್ರ-ದಲ್ಲಿ
ಶಾಸ್ತ್ರ-ದಲ್ಲಿ-ರುವ
ಶಾಸ್ತ್ರ-ದಲ್ಲಿವೆ
ಶಾಸ್ತ್ರ-ಧ್ಯ-ಯನ-ದಿಂದಲೂ
ಶಾಸ್ತ್ರ-ಪೀಡ-ನೆಯೂ
ಶಾಸ್ತ್ರ-ಪ್ರ-ಕಾರ
ಶಾಸ್ತ್ರ-ವನ್ನು
ಶಾಸ್ತ್ರ-ವಾ-ಕ್ಯ-ಗಳನ್ನು
ಶಾಸ್ತ್ರ-ವಾ-ಕ್ಯ-ಗಳು
ಶಾಸ್ತ್ರ-ವಾಗಿ
ಶಾಸ್ತ್ರ-ವಾಗಿ-ತ್ತು
ಶಾಸ್ತ್ರ-ವಾದ
ಶಾಸ್ತ್ರವು
ಶಾಸ್ತ್ರವೇ
ಶಾಸ್ತ್ರ-ವೊಂದೇ
ಶಾಸ್ತ್ರ-ವ್ಯಾ-ಖ್ಯಾ-ನ-ಕೌಶಲಂ
ಶಾಸ್ತ್ರ-ಸಾರ
ಶಾಸ್ತ್ರಾಧ್ಯ-ಯನ
ಶಾಸ್ತ್ರೀಯ
ಶಾಸ್ತ್ರೀ-ಯ-ವಾಗಿ
ಶಾಸ್ತ್ರೋಪ-ದೇಶಕ್ಕೆ
ಶಾಹಿಯ
ಶಿಕ್ಷಣ
ಶಿಕ್ಷಣ-ಗಳೆ-ರಡೂ
ಶಿಕ್ಷಣದ
ಶಿಕ್ಷಣ-ದಲ್ಲಿ
ಶಿಕ್ಷಣ-ವನ್ನು
ಶಿಕ್ಷಣವೇ
ಶಿಕ್ಷಿ-ಸುವನೋ
ಶಿಕ್ಷಿಸುವು-ದ-ರಲ್ಲಿ
ಶಿಕ್ಷೆ-ಯನ್ನು
ಶಿಖರ
ಶಿಖ-ರಕ್ಕೆ
ಶಿಖರ-ಗಳಿಂದ
ಶಿಖರ-ಗಳು
ಶಿಖರ-ದಲ್ಲಿ
ಶಿಖರ-ದಿಂದ
ಶಿಖರ-ವನ್ನು
ಶಿಥಿಲ-ವಾಗಿದೆ
ಶಿರ-ದಲ್ಲಿ
ಶಿರೋ-ಮಣಿ
ಶಿವ
ಶಿವ-ಗಂಗ
ಶಿವ-ಗಂಗೆಗೆ
ಶಿವ-ಗಂಗೆಯ
ಶಿವನ
ಶಿವ-ನನ್ನು
ಶಿವ-ನಿ-ಗಿಂತ
ಶಿವ-ನಿಗೆ
ಶಿವನೆ
ಶಿವ-ನೆ-ನ್ನು-ವುದೂ
ಶಿವನೇ
ಶಿವನೋ
ಶಿವ-ಭಕ್ತ-ರಾದರೆ
ಶಿವ-ಭಕ್ತಿ
ಶಿವ-ಮ-ಹಿಮ್ನ
ಶಿವೋಽಹಂ
ಶಿಶು
ಶಿಶು-ಗಳಾಗಿ-ದ್ದಾಗ
ಶಿಶು-ಗಳೇ
ಶಿಶು-ಪಾಲ
ಶಿಶು-ವಾಗಿ-ರು-ವನು
ಶಿಶು-ವಿಗೆ
ಶಿಶು-ವಿನ
ಶಿಶು-ವಿನಂತಿದೆ
ಶಿಶು-ವಿ-ನಂತೆ
ಶಿಶುವು
ಶಿಶು-ಸ-ಮಾ-ನ-ರಾದ
ಶಿಶು-ಸಹಜ
ಶಿಷ್ಯ
ಶಿಷ್ಯ-ನಿಗೆ
ಶಿಷ್ಯನು
ಶಿಷ್ಯರ
ಶಿಷ್ಯ-ರಲ್ಲಿ
ಶಿಷ್ಯರು
ಶಿಷ್ಯ-ರೂಪ-ದಲ್ಲಿ
ಶಿಷ್ಯ-ರೊಡನೆ
ಶೀಘ್ರ
ಶೀಘ್ರ-ದಲ್ಲಿ
ಶೀಘ್ರ-ದಲ್ಲಿಯೇ
ಶೀಘ್ರ-ವಾಗಿ
ಶೀಲ
ಶೀಲ-ತೆಗೆ
ಶೀಲ-ತೆ-ಯನ್ನು
ಶೀಲದ
ಶೀಲ-ದಲ್ಲಿ
ಶೀಲ-ವನ್ನು
ಶೀಲ-ವಾದ
ಶೀಲ-ವಾದದ್ದು
ಶೀಲ-ವುಳ್ಳ
ಶೀಲ-ಶುದ್ಧಿ
ಶುಕ-ದೇವ
ಶುಕ-ದೇವ-ನ-ಲ್ಲದೆ
ಶುಕ್ತಿಕೆ
ಶುಕ್ತಿ-ಕೆಯ
ಶುಕ್ಲ
ಶುಚಿ-ಮಾಡಿ
ಶುಚಿ-ಯಾಗಿ-ಡುವ-ವನೇ
ಶುತ್ರಿ-ಯಲ್ಲಿ
ಶುದ್ದ-ಮಾಡಿ-ಕೊಂಡಿ-ರುವನೊ
ಶುದ್ಧ
ಶುದ್ಧತೆ
ಶುದ್ಧ-ತೆ-ಗಿಂತ
ಶುದ್ಧ-ನಾ-ಗಿಲ್ಲದ
ಶುದ್ಧ-ನಾಗುವ
ಶುದ್ಧ-ನಾ-ದರೂ
ಶುದ್ಧನು
ಶುದ್ಧ-ಮಾಡು
ಶುದ್ಧ-ರಾಗಿ
ಶುದ್ಧ-ರಾಗು-ವಿರಿ
ಶುದ್ಧರು
ಶುದ್ಧ-ರೆಂದು
ಶುದ್ಧ-ವಾಗಿ-ದ್ದರೆ
ಶುದ್ಧ-ವಾಗಿ-ರ-ಬೇಕು
ಶುದ್ಧ-ವಾಗಿ-ರು-ವಂತೆ
ಶುದ್ಧ-ವಾಗು-ವುದು
ಶುದ್ಧ-ವಾಗು-ವುದೋ
ಶುದ್ಧ-ವಾದ
ಶುದ್ಧ-ವಾದರೆ
ಶುದ್ಧಾ-ದ್ವೈ-ತಿ-ಗಳು
ಶುದ್ಧಾ-ದ್ವೈ-ತಿ-ಗಳೋ
ಶುದ್ಧಿ
ಶುದ್ಧಿಗೆ
ಶುದ್ಧಿ-ಮಾಡ-ಬೇಕು
ಶುದ್ಧಿ-ಮಾಡಿ
ಶುದ್ಧಿ-ಯನ್ನು
ಶುದ್ಧಿ-ಯಾಗು-ವುದು
ಶುದ್ಧಿ-ಯಾ-ದರೆ
ಶುದ್ಧಿಯೇ
ಶುಭ
ಶುಭಕ್ಕೊ
ಶುಭದ
ಶುಭ-ದಾಯಕ
ಶುಭ-ವನ್ನೂ
ಶುಭ-ವಾಗು-ವುದು
ಶುಭಾಂ
ಶುಭಾಶಯ
ಶುಭೋ-ದಯ-ವನ್ನು
ಶುಭ್ರ-ವಿಲ್ಲ
ಶುಷ್ಕತೆ
ಶೂದ್ರನು
ಶೂದ್ರ-ನೆಂದು
ಶೂದ್ರ-ರನ್ನು
ಶೂದ್ರ-ರಾಗಿ-ರ-ಬಹುದು
ಶೂದ್ರ-ರಿಗೆ
ಶೂದ್ರ-ರಿ-ಗೆಲ್ಲ
ಶೂದ್ರರು
ಶೂದ್ರ-ರೆಲ್ಲ
ಶೂದ್ರಾಯ
ಶೂದ್ರಾಸ್ತೇ-ಽಪಿ
ಶೂನ್ಯ
ಶೂನ್ಯ-ದಿಂದ
ಶೂನ್ಯ-ನೆಂದು
ಶೂನ್ಯ-ರಾಗುತ್ತೇವೆ
ಶೃಂಖಲೆ-ಯನ್ನು
ಶೇಕಡ
ಶೇಖರ-ವಾಗಿ-ರುವ
ಶೇಖರಿಸಲ್ಪಟ್ಟ
ಶೇಖ-ರಿಸಿ
ಶೈತ್ಯ
ಶೈಲಿಯ
ಶೈಲಿಯಂತೂ
ಶೈವ
ಶೈವರ
ಶೈವ-ರಾಗಲಿ
ಶೈವ-ರಾಗಲೀ
ಶೈವರು
ಶೋಕತಾಪ-ಗಳು
ಶೋಕ-ನಾಶನಂ
ಶೋಚತಿ
ಶೋಚನೀಯ
ಶೋಧಿಸಿ
ಶೋಪೆನ್ನೇರ್ಗೆ
ಶೋಫನ್ಹೆರ್
ಶೋಭಾ
ಶೋಭಾ-ಯ-ಮಾ-ನ-ವಾಗಿ-ರುವ
ಶೋಷಯತಿ
ಶೋಷಿತ-ರಾದ
ಶೌಚ
ಶ್ರತೇನ
ಶ್ರದ್ದಧಾನೋ
ಶ್ರದ್ಧಾ
ಶ್ರದ್ಧಾ-ಪೂರ್ವ-ಕವೂ
ಶ್ರದ್ಧಾ-ಭಕ್ತಿ-ಗಳನ್ನು
ಶ್ರದ್ಧಾ-ಭಕ್ತಿ-ಗಳಿಂದ
ಶ್ರದ್ಧಾ-ವಂತ
ಶ್ರದ್ಧಾ-ವಂತ-ನಾದ-ವನು
ಶ್ರದ್ಧಾ-ವಂತ-ರಾಗಿ
ಶ್ರದ್ಧಾ-ಸಂಪನ್ನ
ಶ್ರದ್ಧೆ
ಶ್ರದ್ಧೆ-ಇದೇ
ಶ್ರದ್ಧೆಗೆ
ಶ್ರದ್ಧೆಯ
ಶ್ರದ್ಧೆ-ಯನ್ನು
ಶ್ರದ್ಧೆ-ಯ-ಲ್ಲಿದೆ
ಶ್ರದ್ಧೆ-ಯಿಂದ
ಶ್ರದ್ಧೆ-ಯಿದೆ
ಶ್ರದ್ಧೆ-ಯಿರಲೇ-ಬೇಕು
ಶ್ರದ್ಧೆ-ಯಿಲ್ಲ
ಶ್ರದ್ಧೆ-ಯುಂಟಾ-ಯಿತು
ಶ್ರದ್ಧೆಯೂ
ಶ್ರದ್ಧೆಯೇ
ಶ್ರದ್ಧೋ-ತ್ಸಾಹ-ಗಳಿಂದ
ಶ್ರಮ-ಕ್ಕಾಗಿ
ಶ್ರಮ-ಕ್ಕಿಂತಲೂ
ಶ್ರಮ-ಜೀವಿ
ಶ್ರಮ-ದಿಂದ
ಶ್ರಮ-ಪಟ್ಟು
ಶ್ರಮ-ಪಡ-ಬೇಕಾಗಿದೆ
ಶ್ರಮ-ಪಡ-ಬೇಕಾ-ಯಿತು
ಶ್ರಮ-ಪಡ-ಬೇಕು
ಶ್ರಮ-ವನ್ನು
ಶ್ರಮ-ಸಾಧ್ಯ-ವಾದ
ಶ್ರವಣ
ಶ್ರಾದ್ಧ
ಶ್ರಿಕೃಷ್ಣ
ಶ್ರೀ
ಶ್ರೀಕೃಷ್ಣ
ಶ್ರೀಕೃಷ್ಣನ
ಶ್ರೀಕೃಷ್ಣ-ನನ್ನು
ಶ್ರೀಕೃಷ್ಣ-ನಿಂದ
ಶ್ರೀಕೃಷ್ಣನು
ಶ್ರೀಕೃಷ್ಣನೇ
ಶ್ರೀಪರ-ಮ-ಹಂಸ
ಶ್ರೀಭಗ-ವಾ-ನ್
ಶ್ರೀಭಾಷ್ಯದ
ಶ್ರೀಭಾಷ್ಯ-ದಲ್ಲಿ
ಶ್ರೀಮಂತ
ಶ್ರೀಮಂತನ
ಶ್ರೀಮಂತ-ನಾ-ಗಲೀ
ಶ್ರೀಮಂತರ
ಶ್ರೀಮಂತ-ರನ್ನು
ಶ್ರೀಮಂತ-ರಿಗೆ
ಶ್ರೀಮಂತರೂ
ಶ್ರೀಮತಿ
ಶ್ರೀಮತ್
ಶ್ರೀಮದೂರ್ಜಿ-ತ-ಮೇವ
ಶ್ರೀರಾಮ-ಕೃಷ್ಣ
ಶ್ರೀರಾಮ-ಕೃಷ್ಣರ
ಶ್ರೀರಾಮ-ಚಂದ್ರನ
ಶ್ರೀರಾಮ-ನು-ಜಾ-ಚಾರ್ಯರು
ಶ್ರೀಶಂಕರಾ-ಚಾರ್ಯರು
ಶ್ರುತಿ
ಶ್ರುತಿ-ಗಳನ್ನು
ಶ್ರುತಿ-ಗಳಿ-ಗಲ್ಲ
ಶ್ರುತಿ-ಗಳು
ಶ್ರುತಿ-ಗಿಂತ
ಶ್ರುತಿಗೂ
ಶ್ರುತಿಗೆ
ಶ್ರುತಿ-ಮಲ್ಲೋಕೇ
ಶ್ರುತಿಯ
ಶ್ರುತಿ-ಯನ್ನು
ಶ್ರುತಿ-ಯ-ನ್ನೇ
ಶ್ರುತಿ-ಯಲ್ಲಿ
ಶ್ರುತಿ-ಯಲ್ಲೇ
ಶ್ರುತಿಯೇ
ಶ್ರುತಿ-ವಾ-ಕ್ಯ-ಗಳ
ಶ್ರುತಿ-ವಾ-ಕ್ಯ-ಗಳನ್ನು
ಶ್ರುತಿ-ಸಾರ-ವಾದ
ಶ್ರುತೇನ
ಶ್ರೇಣಿ
ಶ್ರೇಣಿ-ಗಳಂತೆ
ಶ್ರೇಣಿ-ಗಳಿವೆ
ಶ್ರೇಣಿಯ
ಶ್ರೇಣಿ-ಯೊಂದಿಗೆ
ಶ್ರೇಣಿ-ಶ್ರೇಣಿ
ಶ್ರೇಯಸ್ಕರ-ವಲ್ಲ
ಶ್ರೇಯಸ್ಕರ-ವಾಗಿ
ಶ್ರೇಯಸ್ಕರ-ವಾದ
ಶ್ರೇಯ-ಸ್ಸನ್ನು
ಶ್ರೇಯ-ಸ್ಸಿಗೆ
ಶ್ರೇಯ-ಸ್ಸಿನ
ಶ್ರೇಯಸ್ಸು
ಶ್ರೇಷ್ಠ
ಶ್ರೇಷ್ಠ-ತಮ
ಶ್ರೇಷ್ಠ-ತಮ-ವಾದುದು
ಶ್ರೇಷ್ಠ-ತರ-ವಾದುದು
ಶ್ರೇಷ್ಠ-ತೆಯ
ಶ್ರೇಷ್ಠ-ತೆ-ಯನ್ನು
ಶ್ರೇಷ್ಠ-ತೆ-ಯನ್ನೂ
ಶ್ರೇಷ್ಠ-ತೆಯು
ಶ್ರೇಷ್ಠ-ತ್ವ-ವನ್ನು
ಶ್ರೇಷ್ಠನ
ಶ್ರೇಷ್ಠ-ನಾಗು-ತ್ತಿದ್ದನು
ಶ್ರೇಷ್ಠ-ಪೂಜೆಯೇ
ಶ್ರೇಷ್ಠ-ಭಾ-ವನೆ
ಶ್ರೇಷ್ಠ-ಮುಖ
ಶ್ರೇಷ್ಠ-ರಾಗು-ವಿರಿ
ಶ್ರೇಷ್ಠ-ರಾದ
ಶ್ರೇಷ್ಠರೂ
ಶ್ರೇಷ್ಠ-ರೆಂದಾ-ಗ-ಲಿಲ್ಲ
ಶ್ರೇಷ್ಠ-ರೆಂದು
ಶ್ರೇಷ್ಠ-ವಾಗಿ-ದ್ದರೂ
ಶ್ರೇಷ್ಠ-ವಾದ
ಶ್ರೇಷ್ಠ-ವಾದರೂ
ಶ್ರೇಷ್ಠ-ವಾದುದು
ಶ್ರೇಷ್ಠ-ವಿರ-ಬಹುದು
ಶ್ರೇಷ್ಠವೂ
ಶ್ರೇಷ್ಠ-ವೆಂದು
ಶ್ರೋತವ್ಯೋ
ಶ್ರೋತೃ-ಗಳ
ಶ್ರೋತೃ-ಗಳಲ್ಲಿ
ಶ್ರೋತೃ-ಗಳಿಗೆ
ಶ್ರೋತೃ-ಗಳು
ಶ್ರೋತ್ರಿಯ
ಶ್ರೋತ್ರಿಯ-ಶ್ರುತಿ-ಸಾರ-ವನ್ನು
ಶ್ರೋತ್ರಿಯೋ
ಶ್ರೌತ
ಶ್ಲಾಘಿಸಿ-ರು-ವರು
ಶ್ಲಾಘಿಸಿರು-ವಿರಿ
ಶ್ಲಾಘಿಸು-ತ್ತಿವೆ
ಶ್ಲೋಕಕ್ಕೂ
ಶ್ಲೋಕ-ಗಳನ್ನು
ಶ್ಲೋಕ-ಗಳಿಗೆ
ಶ್ಲೋಕ-ಗಳು
ಶ್ಲೋಕ-ವನ್ನು
ಶ್ಲೋಕ-ವೊಂದು
ಶ್ವೇತಕೇತೋಶ್ವೇತ
ಶ್ವೇತಾಶ್ವ-ತರ
ಶ್ಶಾಸ್ತ್ರಕ್ಕೆ
ಷಡ್ದರ್ಶನ-ಗಳು
ಷತ್ತಿ-ನ-ಲ್ಲಿಯೂ
ಷಾರ್ಥ-ವೆಂದೂ
ಷೋಕಿ
ಷೋಫೆನ್ಹೇರ್
ಸ
ಸಂ
ಸಂಕಟ
ಸಂಕಟ-ಗಳಲ್ಲಿ
ಸಂಕಟವೂ
ಸಂಕಲ್ಪಕುಂಡಲಿನಿ
ಸಂಕಲ್ಪ-ಮಾಡಿ-ದರು
ಸಂಕಲ್ಪ-ವನ್ನು
ಸಂಕಲ್ಪವನ್ನೆಲ್ಲಾ
ಸಂಕಲ್ಪವು
ಸಂಕಲ್ಪವೇ
ಸಂಕು-ಚಿತ
ಸಂಕು-ಚಿತ-ಗೊಂಡುದು
ಸಂಕು-ಚಿತ-ಗೊಳಿ-ಸು-ವು-ದಕ್ಕೆ
ಸಂಕು-ಚಿತ-ರಾಗಿ
ಸಂಕು-ಚಿತ-ವಾಗು-ತ್ತದೆ
ಸಂಕು-ಚಿತ-ವಾಗುತ್ತಾ
ಸಂಕು-ಚಿತ-ವಾಗು-ವುದು
ಸಂಕು-ಚಿತ-ವಾದಾಗ
ಸಂಕುಚಿಸುವು-ದೆಂದೂ
ಸಂಕುಚಿಸುವು-ದೆಂಬ
ಸಂಕೇತ
ಸಂಕೇತ-ಗಳಾಗಿ
ಸಂಕೇತ-ಗಳೆಲ್ಲಾ
ಸಂಕೋಚ
ಸಂಕೋಚ-ಗೊಳ್ಳು-ತ್ತವೆ
ಸಂಕೋಚ-ಗೊಳ್ಳು-ವುದೂ
ಸಂಕೋಚ-ವಾಗು-ವಂತೆ
ಸಂಕೋಚ-ವಾಗು-ವು-ದಿಲ್ಲ
ಸಂಕೋಚ-ವಿಕಾಸ
ಸಂಕೋಚ-ವಿ-ಲ್ಲದೆ
ಸಂಕ್ರಮ-ಣ-ರೂಪ-ದಲ್ಲಿ-ರುವ
ಸಂಕ್ಷಿಪ್ತ
ಸಂಕ್ಷಿಪ್ತ-ವಾಗಿ
ಸಂಕ್ಷೇಪ-ವಾಗಿ
ಸಂಕ್ಷೇಪ-ವಾದ
ಸಂಖ್ಯೆ
ಸಂಖ್ಯೆಗೆ
ಸಂಖ್ಯೆಯ
ಸಂಖ್ಯೆ-ಯಲ್ಲಿ
ಸಂಗ
ಸಂಗ-ಚ್ಛಧ್ವಂ
ಸಂಗ-ಡವೂ
ಸಂಗತಿ
ಸಂಗ-ತಿ-ಗಳನ್ನು
ಸಂಗ-ತಿ-ಗಳು
ಸಂಗ-ದಿಂದ
ಸಂಗೀತ
ಸಂಗೀ-ತಕ್ಕೆ
ಸಂಗೀ-ತದ
ಸಂಗೀತ-ದಲ್ಲಿ
ಸಂಗೀತ-ದಲ್ಲೂ
ಸಂಗೀ-ತವು
ಸಂಗ್ರಹ
ಸಂಗ್ರ-ಹಕ್ಕೆ
ಸಂಗ್ರ-ಹದ
ಸಂಗ್ರಹ-ದಲ್ಲಿ
ಸಂಗ್ರಹ-ವನ್ನು
ಸಂಗ್ರಹ-ವಾದರೆ
ಸಂಗ್ರ-ಹಿ-ಸಿಟ್ಟ
ಸಂಗ್ರ-ಹಿ-ಸಿ-ಟ್ಟು-ಕೊಳ್ಳು-ತ್ತಿ-ದ್ದರೆಂಬುದೂ
ಸಂಗ್ರ-ಹಿ-ಸಿದ
ಸಂಗ್ರ-ಹಿ-ಸು-ವುದು
ಸಂಗ್ರಾಮ-ದಲ್ಲಿ
ಸಂಘ
ಸಂಘಕ್ಕೆ
ಸಂಘ-ಗಳನ್ನು
ಸಂಘ-ಟನ-ಕಾರರು
ಸಂಘ-ಟನಾ
ಸಂಘ-ಟನಾ-ಕ್ರಮ
ಸಂಘ-ಟನಾ-ಶಕ್ತಿ
ಸಂಘಟ-ನೆಗೆ
ಸಂಘಟ-ನೆ-ಯಲ್ಲಿ
ಸಂಘ-ಟಿ-ಸಿದ-ವ-ರಿಗೆ
ಸಂಘದ
ಸಂಘ-ದಲ್ಲಿ
ಸಂಘ-ದಿಂದ
ಸಂಘ-ಬದ್ಧ
ಸಂಘರ್ಷ-ಗಳ
ಸಂಘ-ವನ್ನು
ಸಂಘಾತ-ದಿಂದ
ಸಂಚರಿಸ-ಬಲ್ಲವು
ಸಂಚರಿಸ-ಬೇ-ಕಾದರೂ
ಸಂಚ-ರಿಸಿ
ಸಂಚರಿ-ಸಿದೆ
ಸಂಚ-ರಿಸಿ-ರು-ವೆನು
ಸಂಚರಿ-ಸುತ್ತದೆ
ಸಂಚರಿ-ಸು-ತ್ತಿದೆ
ಸಂಚರಿಸುತ್ತಿ-ರುವ
ಸಂಚರಿಸುತ್ತಿರು-ವುದು
ಸಂಚರಿಸುತ್ತಿರು-ವೆನು
ಸಂಚರಿಸು-ತ್ತಿವೆ
ಸಂಚರಿ-ಸುವ
ಸಂಚರಿ-ಸು-ವಂತೆ
ಸಂಚರಿ-ಸು-ವಂತೆ-ಮಾಡಿ-ದವು
ಸಂಚರಿ-ಸು-ವರು
ಸಂಚರಿ-ಸು-ವು-ದಿಲ್ಲ
ಸಂಚಾರ
ಸಂಚಾರ-ಕ-ವಾಗಿ
ಸಂಚಾರ-ದಿಂದ
ಸಂಚಿಕ್ಷಿಪುಃ
ಸಂಜೀವಿ-ನಿ-ಯನ್ನು
ಸಂಜೆ
ಸಂಜೆ-ಯಲ್ಲಿ
ಸಂತ
ಸಂತರ
ಸಂತ-ರಾಗಿ-ರು-ವರು
ಸಂತಾನ
ಸಂತಾನ-ದ-ವ-ರನ್ನು
ಸಂತಾನ-ದಿಂದಲ್ಲ
ಸಂತಾ-ನರು
ಸಂತಾನರೆ
ಸಂತಾನರೇ
ಸಂತು-ಷ್ಟ-ನಾಗಿ
ಸಂತು-ಷ್ಟ-ರಾಗಿ
ಸಂತು-ಷ್ಟ-ರಾಗಿ-ರು-ವರೋ
ಸಂತು-ಷ್ಟಸ್ತಸ್ಯ
ಸಂತೃಪ್ತ-ರಾಗಿ
ಸಂತೈ-ಕೆಯ
ಸಂತೋಷ
ಸಂತೋಷ-ಗಳಿಂದ
ಸಂತೋಷ-ದಿಂದ
ಸಂತೋಷ-ದಿಂದಲೂ
ಸಂತೋಷ-ವನ್ನು
ಸಂತೋಷ-ವಾಗ-ಬಹುದು
ಸಂತೋಷ-ವಾಗಿದೆ
ಸಂತೋಷ-ವಾ-ಯಿತು
ಸಂತೋಷವೇ
ಸಂತೋಷಿಸಿ
ಸಂದಣಿ
ಸಂದದ್ದು
ಸಂದರ್ಭಗಳಲ್ಲಿ
ಸಂದರ್ಭ-ದಲ್ಲಿ
ಸಂದರ್ಭ-ದಲ್ಲಿಯೂ
ಸಂದರ್ಭ-ದಲ್ಲೇ
ಸಂದರ್ಭಲ್ಲಿ
ಸಂದರ್ಶಿ-ಸಿದ
ಸಂದರ್ಶಿ-ಸಿದರು
ಸಂದರ್ಶಿ-ಸಿದ್ದರೂ
ಸಂದರ್ಶಿ-ಸುತ್ತದೆ
ಸಂದರ್ಶಿ-ಸುವ
ಸಂದಿಗ್ಧ
ಸಂದುಗೊಂದು-ಗಳೂ
ಸಂದೇಶ
ಸಂದೇಶಕ್ಕೆ
ಸಂದೇಶ-ಗಳನ್ನು
ಸಂದೇಶದ
ಸಂದೇಶ-ದಿಂದ
ಸಂದೇಶ-ವನ್ನು
ಸಂದೇಶ-ವಾ-ಹಕ-ನಂತೆ
ಸಂದೇಶ-ವಿತ್ತು
ಸಂದೇಶವೂ
ಸಂದೇಶ-ವೆಲ್ಲ
ಸಂದೇಶವೇ
ಸಂದೇಶ-ವೊಂದಿರು-ವುದು
ಸಂದೇಹ-ಗಳನ್ನು
ಸಂದೇಹ-ಪಡು-ವರು
ಸಂದೇಹ-ವನ್ನು
ಸಂದೇಹ-ವಿದೆ
ಸಂದೇಹ-ವಿರ-ಬಹುದು
ಸಂದೇಹ-ವಿಲ್ಲ
ಸಂದೇ-ಹವು
ಸಂದೇ-ಹವೂ
ಸಂದೇಹವೇ
ಸಂದೇ-ಹಿ-ಸ-ಲಾರಿರಿ
ಸಂಧಿ-ಸಮ-ಯ-ದಲ್ಲಿ
ಸಂಧಿಸುತ್ತಿ-ರು-ವನು
ಸಂಧಿಸು-ವೆವು
ಸಂನ್ಯಾಸ
ಸಂನ್ಯಾ-ಸ-ಅ-ದರ
ಸಂನ್ಯಾ-ಸದ
ಸಂನ್ಯಾ-ಸ-ವೆಂದರೆ
ಸಂನ್ಯಾ-ಸಾ-ಶ್ರಮ-ಗಳಿಗೆ
ಸಂನ್ಯಾಸಿ
ಸಂನ್ಯಾ-ಸಿ-ಗಳ
ಸಂನ್ಯಾ-ಸಿ-ಗಳನ್ನು
ಸಂನ್ಯಾ-ಸಿ-ಗಳಾಗಿ-ದ್ದರೆ
ಸಂನ್ಯಾ-ಸಿ-ಗಳು
ಸಂನ್ಯಾ-ಸಿ-ಗಳೋ
ಸಂನ್ಯಾ-ಸಿಗೂ
ಸಂನ್ಯಾ-ಸಿಗೆ
ಸಂನ್ಯಾ-ಸಿಯ
ಸಂನ್ಯಾ-ಸಿ-ಯಂತೆಯೇ
ಸಂನ್ಯಾ-ಸಿ-ಯನ್ನು
ಸಂನ್ಯಾ-ಸಿ-ಯಾಗಲು
ಸಂನ್ಯಾ-ಸಿ-ಯಾಗಿ-ದ್ದಾಗ
ಸಂನ್ಯಾ-ಸಿ-ಯೊಬ್ಬನ
ಸಂನ್ಯಾ-ಸಿ-ಯೊ-ಬ್ಬ-ನಿಗೆ
ಸಂನ್ಯಾ-ಸಿ-ಯೊಬ್ಬನು
ಸಂನ್ಯಾ-ಸಿ-ಯೊಬ್ಬ-ನೊಂದಿಗೆ
ಸಂಪತ್ತನ್ನು
ಸಂಪ-ತ್ತಿಗೆ
ಸಂಪತ್ತು
ಸಂಪತ್ತು-ಗಳ
ಸಂಪನ್ನ-ರಾ-ದರೂ
ಸಂಪನ್ನ-ರಾದರೆ
ಸಂಪರ್ಕವು
ಸಂಪಾದನೆ
ಸಂಪಾದಿಸ-ಬಹುದು
ಸಂಪಾದಿ-ಸಲು
ಸಂಪಾದಿಸಿತು
ಸಂಪಾದಿಸಿ-ರು-ವುದು
ಸಂಪಾದಿ-ಸು-ವರು
ಸಂಪುಟ-ದಲ್ಲಿಟ್ಟು
ಸಂಪೂರ್ಣ
ಸಂಪೂರ್ಣ-ವಾಗಿ
ಸಂಪ್ರ
ಸಂಪ್ರತಿ
ಸಂಪ್ರ-ದಾಯ
ಸಂಪ್ರ-ದಾಯಕ್ಕೂ
ಸಂಪ್ರ-ದಾ-ಯಕ್ಕೆ
ಸಂಪ್ರ-ದಾಯ-ಗಳ
ಸಂಪ್ರ-ದಾಯ-ಗಳನ್ನು
ಸಂಪ್ರ-ದಾಯ-ಗಳಲ್ಲಿ
ಸಂಪ್ರ-ದಾಯ-ಗಳಿ-ಗಾ-ಗಲಿ
ಸಂಪ್ರ-ದಾಯ-ಗಳಿಗೆ
ಸಂಪ್ರ-ದಾಯ-ಗಳು
ಸಂಪ್ರ-ದಾಯ-ಗಳೂ
ಸಂಪ್ರ-ದಾಯ-ಗಳೆಲ್ಲ
ಸಂಪ್ರ-ದಾಯ-ಗಳೆಲ್ಲಾ
ಸಂಪ್ರ-ದಾಯದ
ಸಂಪ್ರ-ದಾಯ-ದಲ್ಲಿಯೂ
ಸಂಪ್ರ-ದಾಯ-ದ-ವ-ರಿಗೆ
ಸಂಪ್ರ-ದಾಯ-ದ-ವರು
ಸಂಪ್ರ-ದಾಯ-ದ-ವರೂ
ಸಂಪ್ರ-ದಾಯ-ದ-ವ-ರೆಲ್ಲಾ
ಸಂಪ್ರ-ದಾಯ-ದ-ವ-ರೊಂದಿಗೆ
ಸಂಪ್ರ-ದಾಯ-ದಿಂದ
ಸಂಪ್ರ-ದಾಯ-ನಿಷ್ಠರ
ಸಂಪ್ರ-ದಾಯ-ನಿಷ್ಠೆ-ಯನ್ನೂ
ಸಂಪ್ರ-ದಾಯ-ಬದ್ಧ
ಸಂಪ್ರ-ದಾಯ-ವಂತ-ರೆಂದು
ಸಂಪ್ರ-ದಾಯ-ವ-ದ-ವರು
ಸಂಪ್ರ-ದಾಯ-ವನ್ನು
ಸಂಪ್ರ-ದಾ-ಯವು
ಸಂಪ್ರ-ದಾ-ಯವೂ
ಸಂಪ್ರ-ದಾಯ-ಶೀಲ
ಸಂಪ್ರ-ದಾಯ-ಸ್ಥರೂ
ಸಂಪ್ರಾಯ-ದ-ವರೂ
ಸಂಬಂಧ
ಸಂಬಂಧ-ಗಳ-ನ್ನೆಲ್ಲಾ
ಸಂಬಂಧ-ದಲ್ಲಿ
ಸಂಬಂಧ-ದಿಂದ
ಸಂಬಂಧ-ಪಟ್ಟ
ಸಂಬಂಧ-ಪಟ್ಟಂತೆ
ಸಂಬಂಧ-ಪಟ್ಟದ್ದು
ಸಂಬಂಧ-ಪಟ್ಟಿದೆ
ಸಂಬಂಧ-ಪಟ್ಟು-ದಕ್ಕೆ
ಸಂಬಂಧ-ಭಾವವು
ಸಂಬಂಧ-ವನ್ನು
ಸಂಬಂಧ-ವಾಗಿ
ಸಂಬಂಧ-ವಿದೆ
ಸಂಬಂಧ-ವಿದ್ದರೆ
ಸಂಬಂಧ-ವಿ-ರುವ
ಸಂಬಂಧ-ವಿಲ್ಲ
ಸಂಬಂಧವೂ
ಸಂಬಂಧ-ವೆಂತಹುದು
ಸಂಬಂಧಿ-ಗಳಲ್ಲೇ
ಸಂಬಂಧಿ-ಯಾಗಿ
ಸಂಬಂಧಿ-ಸದೇ
ಸಂಬಂಧಿ-ಸಿದ
ಸಂಬಂಧಿಸಿ-ದಂತೆ
ಸಂಬಂಧಿಸಿ-ದವು
ಸಂಬಂಧಿಸಿ-ದುದು
ಸಂಬಂಧಿಸಿ-ರು-ವೆನು
ಸಂಬಂಧಿ-ಸಿವೆ
ಸಂಬಂಧಿ-ಸುವ
ಸಂಬಭ-ದಲ್ಲಿ
ಸಂಬೋಧಿ-ಸಿ-ರು-ವಿರಿ
ಸಂಭವ-ವಿದೆ
ಸಂಭವ-ವಿರು-ವುದು
ಸಂಭವ-ವಿಲ್ಲ
ಸಂಭವ-ವುಂಟು
ಸಂಭವವೂ
ಸಂಭವವೇ
ಸಂಭ-ವಾಮಿ
ಸಂಭವಿಸ-ದಂತೆ
ಸಂಭವಿಸ-ಬಹು-ದಾದ
ಸಂಭವಿ-ಸಿದ
ಸಂಭವಿಸಿ-ದಾಗ
ಸಂಭವಿಸಿಯೇ
ಸಂಭವಿಸಿಲ್ಲ
ಸಂಭವಿ-ಸುತ್ತದೆ
ಸಂಮಿಶ್ರಣ-ಗೊಂಡು
ಸಂಯಮ
ಸಂಯಮ-ದಿಂದಲೂ
ಸಂಯಮವು
ಸಂಯಮಿ-ಗಳೋ
ಸಂಯುಕ್ತ
ಸಂಯೋಗ
ಸಂಯೋ-ಜಿಸ-ಬೇಕು
ಸಂರಕ್ಷಣೆಯ
ಸಂರಕ್ಷಿಸಿ-ಕೊಂಡು
ಸಂರಕ್ಷಿಸಿ-ದರೆ
ಸಂರಕ್ಷಿಸು-ವುದು
ಸಂರಕ್ಷಿ-ಸು-ವುದೇ
ಸಂವದಧ್ವಂ
ಸಂವಿ-ಧಾನ-ಬದ್ಧ
ಸಂಶಯ-ಗಳು
ಸಂಶೋಧಕ
ಸಂಶೋಧನೆ
ಸಂಶೋಧ-ನೆ-ಗಳಿಂದ
ಸಂಶೋಧ-ನೆ-ಗಳು
ಸಂಶೋಧ-ನೆ-ಗಳೆಂಬ
ಸಂಶೋಧ-ನೆ-ಯಿಂದ
ಸಂಸಾರ
ಸಂಸಾರದ
ಸಂಸಾರ-ದಲ್ಲಿ
ಸಂಸಾರ-ದಲ್ಲಿ-ರುವ
ಸಂಸಾರ-ದಿಂದ
ಸಂಸಾರ-ವನ್ನಾಗಿ
ಸಂಸಾರ-ವನ್ನು
ಸಂಸಾರವು
ಸಂಸಾರ-ವೆಂದು
ಸಂಸಾರ-ವೇಕೆ
ಸಂಸಾರ-ಸಾಗರ-ವನ್ನು
ಸಂಸ್ಕಾ-ರಕ್ಕೆ
ಸಂಸ್ಕಾರ-ಗಳಿಂದ
ಸಂಸ್ಕಾರ-ಗಳಿಗೆ
ಸಂಸ್ಕಾರ-ಗಳೆಲ್ಲಾ
ಸಂಸ್ಕಾರ-ವಿದೆ
ಸಂಸ್ಕೃತ
ಸಂಸ್ಕೃತ-ಜ್ಞಾನ
ಸಂಸ್ಕೃತದ
ಸಂಸ್ಕೃತ-ದಲ್ಲಿ
ಸಂಸ್ಕೃತ-ಭಾಷೆ
ಸಂಸ್ಕೃತ-ಭಾಷೋಚ್ಚಾರ-ಣೆಯೆ
ಸಂಸ್ಕೃತ-ವನ್ನು
ಸಂಸ್ಕೃತ-ವೆಂದು
ಸಂಸ್ಕೃತಿ
ಸಂಸ್ಕೃತಿಗೆ
ಸಂಸ್ಕೃತಿಯ
ಸಂಸ್ಕೃತಿ-ಯನ್ನು
ಸಂಸ್ಕೃತಿ-ಯಲ್ಲಿ
ಸಂಸ್ಕೃತಿ-ಯಾಗಿ-ರ-ಬಹುದು
ಸಂಸ್ಕೃತಿ-ಯಾಗಿ-ರು-ತ್ತದೆ
ಸಂಸ್ಕೃತಿಯು
ಸಂಸ್ಕೃತಿ-ಯೆಲ್ಲಾ
ಸಂಸ್ಥಾನ-ಗಳ
ಸಂಸ್ಥಾನದ
ಸಂಸ್ಥಾಪ-ಕ-ರೆಲ್ಲಾ
ಸಂಸ್ಥಾಪನೆ
ಸಂಸ್ಥಿತಾ
ಸಂಸ್ಥೆ
ಸಂಸ್ಥೆ-ಗಳನ್ನು
ಸಂಸ್ಥೆ-ಗಳನ್ನೂ
ಸಂಸ್ಥೆ-ಗಳಿಗೆ
ಸಂಸ್ಥೆ-ಗಳಿವೆ
ಸಂಸ್ಥೆ-ಗಳು
ಸಂಸ್ಥೆ-ಗಳೆ-ಲ್ಲವೂ
ಸಂಸ್ಥೆಗೂ
ಸಂಸ್ಥೆಯ
ಸಂಸ್ಥೆ-ಯಲ್ಲಿ
ಸಂಸ್ಥೆ-ಯಿರ-ಬೇಕು
ಸಂಸ್ಧೆ-ಗಳು
ಸಂಹಾ-ರಕ
ಸಂಹಾ-ರ-ಕರ್ತ-ನಾದ
ಸಂಹಿತಾ
ಸಂಹಿತೆ
ಸಂಹಿ-ತೆ-ಗಳಲ್ಲಿ
ಸಂಹಿ-ತೆ-ಗಳಲ್ಲಿಯೂ
ಸಂಹಿ-ತೆ-ಗಳು
ಸಂಹಿ-ತೆಯ
ಸಂಹಿ-ತೆ-ಯಲ್ಲಿ
ಸಂಹಿ-ತೆ-ಯ-ಲ್ಲಿಯೂ
ಸಂಹಿ-ತೆ-ಯ-ಲ್ಲಿ-ರುವ
ಸಂಹಿ-ತೆ-ಯಲ್ಲೂ
ಸಕಲ
ಸಕಲ-ರಿಗೂ
ಸಕಲವೂ
ಸಕಾರ-ವನ್ನು
ಸಕಾಲ
ಸಕಾಲಕ್ಕೆ
ಸಖ
ಸಖ-ನಾದ
ಸಖರ
ಸಖ-ರಾಗಿ-ದ್ದರು
ಸಖಾಯಾ
ಸಖ್ಯ
ಸಗುಣ
ಸಗುಣ-ಈ-ಶ್ವರ-ಎಂದರೆ
ಸಗುಣ-ದೇ-ವ-ರನ್ನು
ಸಗುಣ-ದೇ-ವರು
ಸಗುಣ-ನಾಗಿ-ರು-ವಂತೆ
ಸಗುಣ-ನಾ-ದರೂ
ಸಗುಣ-ವೆಂದರೆ
ಸಗುಣ-ಸಾ-ಕಾರ
ಸಗುಣೋಪಾಸನೆ
ಸಚೇ-ತನ
ಸಚೇ-ತನ-ವಾಗು-ತ್ತಿದೆ
ಸಚೇತ-ವಾಗು-ತ್ತಿದೆ
ಸಚ್ಚಿ-ದಾ-ನಂದ
ಸಚ್ಚಿ-ದಾ-ನಂದ-ನಲ್ಲಿ
ಸಜೀವ
ಸಜೀವ-ವಾಗಿವೆ
ಸಜ್ಜನ-ರಾಗ-ಲಾರೆವು
ಸಜ್ಜನರೇ
ಸಡಿಲ-ವಾಗಿ
ಸಣ್ಣ
ಸಣ್ಣ-ಗುಳ್ಳೆ-ಯಾಗಿ
ಸಣ್ಣದು
ಸಣ್ಣ-ಪುಟ್ಟ
ಸತತ
ಸತರ್ಕ-ವಾಗಿ
ಸತಿ
ಸತಿ-ಯಾಗಿ
ಸತ್
ಸತ್ಕರ್ಮ
ಸತ್ಕರ್ಮ-ಗಳ
ಸತ್ಕರ್ಮ-ದಿಂದ
ಸತ್ಕರ್ಮ-ವನ್ನು
ಸತ್ಕಾರ
ಸತ್ಕಾರ-ಪ-ರರು
ಸತ್ಕಾರ-ವನ್ನು
ಸತ್ಕುಲ-ಪ್ರಸೂ-ತನು
ಸತ್ತ
ಸತ್ತ-ನಂತರ
ಸತ್ತ-ನ್ನೇ
ಸತ್ತ-ಮೇಲೆ
ಸತ್ತರೂ
ಸತ್ತರೆ
ಸತ್ತ-ರೇ-ನಂತೆ
ಸತ್ತ-ವನು
ಸತ್ತಿನ
ಸತ್ತಿರು-ವು-ದೆಂದು
ಸತ್ತ್ವ
ಸತ್ತ್ವಂ
ಸತ್ತ್ವ-ಗುಣ
ಸತ್ತ್ವ-ಗುಣ-ವಾದಾಗ
ಸತ್ತ್ವದ
ಸತ್ತ್ವ-ದಿಂದಲೇ
ಸತ್ತ್ವ-ವನ್ನು
ಸತ್ತ್ವ-ಶುದ್ಧಿಃ
ಸತ್ತ್ವ-ಶುದ್ಧಿ-ಯಾಗು-ವುದು
ಸತ್ತ್ವ-ಶುದ್ಧಿ-ಯಾ-ದರೆ
ಸತ್ತ್ವ-ಶುದ್ಧೌ
ಸತ್ಪುರು-ಷ-ರನ್ನು
ಸತ್ಪುರು-ಷ-ರಾ-ಗು-ವು-ದಕ್ಕೆ
ಸತ್ಪುರು-ಷ-ರಿಂದ
ಸತ್ಪುರು-ಷರು
ಸತ್ಯ
ಸತ್ಯ-ಕ್ಕಾಗಿ
ಸತ್ಯಕ್ಕೆ
ಸತ್ಯ-ಗಳ
ಸತ್ಯ-ಗಳನ್ನು
ಸತ್ಯ-ಗಳು
ಸತ್ಯ-ಜ್ಯೋ-ತಿ-ಯನ್ನು
ಸತ್ಯ-ತೆಗೆ
ಸತ್ಯ-ತೆ-ಯನ್ನು
ಸತ್ಯದ
ಸತ್ಯ-ದಂತೆ
ಸತ್ಯ-ದಿಂದ
ಸತ್ಯ-ಪಿ-ಪಾಸು-ಗಳೆಲ್ಲ
ಸತ್ಯ-ಪ್ರೇಮಿ-ಗಳೂ
ಸತ್ಯ-ಮೇವ
ಸತ್ಯ-ಯು-ಗಕ್ಕೆ
ಸತ್ಯ-ಯು-ಗದ
ಸತ್ಯ-ಯುಗ-ದಲ್ಲಿ
ಸತ್ಯ-ವನ್ನು
ಸತ್ಯ-ವನ್ನೂ
ಸತ್ಯ-ವನ್ನೇ
ಸತ್ಯ-ವಲ್ಲ
ಸತ್ಯ-ವ-ಲ್ಲದೆ
ಸತ್ಯ-ವಲ್ಲ-ವೆಂದು
ಸತ್ಯ-ವಾ-ಕ್ಯ-ವನ್ನು
ಸತ್ಯ-ವಾಗಿ
ಸತ್ಯ-ವಾಗಿ-ತ್ತು
ಸತ್ಯ-ವಾಗಿದೆ
ಸತ್ಯ-ವಾಗಿ-ದ್ದರೆ
ಸತ್ಯ-ವಾಗಿ-ರ-ಬೇಕು
ಸತ್ಯ-ವಾಗಿ-ರ-ಬೇಕೆಂದು
ಸತ್ಯ-ವಾಗಿ-ರು-ವಂತೆಯೇ
ಸತ್ಯ-ವಾಗಿ-ರು-ವಾಗ
ಸತ್ಯ-ವಾಗುವು-ದ-ರಲ್ಲಿ
ಸತ್ಯ-ವಾಗು-ವುದು
ಸತ್ಯ-ವಾದ
ಸತ್ಯ-ವಾದರೆ
ಸತ್ಯ-ವಾದುದು
ಸತ್ಯ-ವಿದೆ
ಸತ್ಯ-ವಿದ್ದರೂ
ಸತ್ಯ-ವಿರ-ಬಹುದು
ಸತ್ಯ-ವಿರು-ವು-ದಾದರೆ
ಸತ್ಯವು
ಸತ್ಯವೂ
ಸತ್ಯ-ವೆಂದು
ಸತ್ಯ-ವೆಂಬುದು
ಸತ್ಯ-ವೆ-ನ್ನು-ವು-ದಕ್ಕೆ
ಸತ್ಯವೇ
ಸತ್ಯ-ವೇನು
ಸತ್ಯವೊ
ಸತ್ಯ-ವೊಂದೇ
ಸತ್ಯ-ವ್ರ-ತರ-ನ್ನಾಗಿ
ಸತ್ಯ-ಶಾಂತಿ-ಗಳ
ಸತ್ಯ-ಸಂಧ
ಸತ್ಯಸ್ಯ
ಸತ್ಯ-ಹೇ-ಳು-ವುದು
ಸತ್ಯಾಂಶ
ಸತ್ಯಾ-ನು-ಸಂಧಾನದ
ಸತ್ಯಾ-ನ್ವೇಷಕ-ರಾದರೆ
ಸತ್ಯಾ-ಸತ್ಯ-ತೆ-ಯಾಗಲಿ
ಸತ್ಯೇನ
ಸತ್ವ
ಸತ್ವ-ವನ್ನು
ಸತ್ವ-ವೆಲ್ಲಾ
ಸತ್ವ-ಹೀನ-ರಾದ
ಸತ್ವ-ಹೀನ-ವಾಗಿದೆ
ಸತ್ವಾಂಶ
ಸದ-ವ-ಕಾಶ
ಸದಸ್ಯ
ಸದ-ಸ್ಯ-ರಿಗೆ
ಸದಾ
ಸದಾ-ಕಾಲ-ದಲ್ಲಿಯೂ
ಸದೃಢ
ಸದೃಢ-ವಾಗಿ
ಸದೃಢ-ವಾದ
ಸದೃಶ
ಸದೃಶ-ರಾದ
ಸದೆ-ಬಡಿಸು-ವುದು
ಸದ್ಗುಣ
ಸದ್ಗುಣ-ಗಳ
ಸದ್ಗುಣ-ಗಳನ್ನು
ಸದ್ಯಕ್ಕೆ
ಸದ್ವಂಶ-ದಲ್ಲಿ
ಸದ್ವಿಪ್ರಾ
ಸನಾ-ತನ
ಸನಾತನಃ
ಸನಾ-ತನ-ವಾಗಿ
ಸನಾ-ತನ-ವಾದ
ಸನಾ-ತನ-ವಾದುದು
ಸನಾ-ತನ-ವಾದುವು
ಸನಾ-ತ-ನವೂ
ಸನಾತನಿ-ಗಳ
ಸನಾತನಿ-ಗಳನ್ನು
ಸನ್ನಿಕಟ
ಸನ್ನಿಧಿ
ಸನ್ನಿವೇಶ
ಸನ್ನಿವೇಶ-ಗಳ
ಸನ್ನಿವೇಶ-ಗಳಲ್ಲಿ
ಸನ್ನಿವೇಶ-ಗಳಿಗೆ
ಸನ್ನಿವೇಶ-ಗಳು
ಸನ್ನಿವೇ-ಶದ
ಸನ್ನಿವೇಶವೂ
ಸನ್ನಿ-ಹಿ-ತ-ವಾಗಿ-ತ್ತು
ಸನ್ನಿ-ಹಿ-ತ-ವಾಗಿದೆ
ಸನ್ನಿ-ಹಿ-ತ-ವಾಗು-ತ್ತದೆ
ಸನ್ನಿ-ಹಿ-ತ-ವಾಗು-ವುದು
ಸನ್ನಿ-ಹಿ-ತ-ವಾ-ಯಿತು
ಸನ್ಮಾ-ನ-ವನ್ನು
ಸನ್ಮಾ-ನ-ವಲ್ಲ
ಸನ್ಮಾರ್ಗ-ದಿಂದ
ಸನ್ಮಾರ್ಗ-ವನ್ನು
ಸನ್ಯಾ-ಸಿ-ಗಳ
ಸನ್ಯಾ-ಸಿ-ಗಳು
ಸನ್ಯಾ-ಸಿಗೆ
ಸನ್ಯಾ-ಸಿ-ಯಂತೆ
ಸಪ್ರ-ಮಾ-ಣ-ವಾಗಿ
ಸಫಲ
ಸಫಲ-ವಾಗದೆ
ಸಫಲ-ವಾಗಲಿ
ಸಫಲ-ವಾಗ-ಲಿಲ್ಲ
ಸಬಲ-ದುರ್ಬಲರ
ಸಭಾಂಗಣ
ಸಭಾಂಗಣ-ದಲ್ಲಿ
ಸಭಾಂಗಣ-ದಿಂದ
ಸಭಾದ್ಯಕ್ಷ-ರಾ-ಗಿದ್ದ
ಸಭಿ-ಕರು
ಸಭಿಕರೇ
ಸಭೆ
ಸಭೆ-ಗಳನ್ನು
ಸಭೆಯ
ಸಭೆ-ಯನ್ನು
ಸಭೆ-ಯಲ್ಲಿ
ಸಭೆ-ಯಲ್ಲಿ-ರುವ
ಸಭೆಯೇ
ಸಭ್ಯ
ಸಭ್ಯತಾ
ಸಭ್ಯ-ರಾ-ದಂತೆಯೇ
ಸಭ್ಯ-ರಿ-ಗಿಂತ
ಸಮ
ಸಮಂ
ಸಮ-ಕಾಲೀನರಂತೆಯೆ
ಸಮ-ಕಾಲೀನ-ವಾದು-ದೆಂದೂ
ಸಮಗ್ರ
ಸಮ-ಗ್ರ-ವಾಗಿ
ಸಮ-ತಾ-ಭಾವ-ವನ್ನು
ಸಮ-ತಾ-ವಾದಿ-ಗಳ
ಸಮ-ತೆ-ಯಲ್ಲಿ-ರು-ತ್ತವೆ
ಸಮ-ತೋ-ಲನ-ವುಂಟಾ-ಗು-ತ್ತದೆ
ಸಮ-ತ್ವ-ದಲ್ಲಿ
ಸಮ-ದರ್ಶಿ
ಸಮ-ದರ್ಶಿ-ತ್ವವೂ
ಸಮ-ನಾಗಿ
ಸಮ-ನಾ-ಗಿದ್ದ
ಸಮ-ನಾಗಿ-ರುವ
ಸಮ-ನಾಗಿ-ರು-ವನು
ಸಮ-ನಾಗಿ-ರು-ವಾಗ
ಸಮ-ನಾಗಿ-ರು-ವುವು
ಸಮ-ನಾದ
ಸಮ-ನ್ವಯ
ಸಮ-ನ್ವಯ-ಗೊಳಿ-ಸಲು
ಸಮ-ನ್ವಯ-ಗೊಳಿ-ಸುವಂಥದು
ಸಮ-ನ್ವಯ-ಗೊಳಿಸು-ವಷ್ಟು
ಸಮ-ನ್ವಯ-ಗೊಳಿಸು-ವುದು
ಸಮ-ನ್ವಯ-ತೆ-ಯನ್ನು
ಸಮ-ನ್ವಯ-ಮೂರ್ತಿ
ಸಮ-ನ್ವ-ಯಾ-ಚಾರ್ಯ-ನಾದ
ಸಮ-ನ್ವಿತ-ವಾದ
ಸಮ-ಪರ್ಕ-ವಾಗಿ
ಸಮಯ
ಸಮ-ಯಕ್ಕೆ
ಸಮ-ಯ-ಗಳಲ್ಲಿ
ಸಮ-ಯ-ಗಳು
ಸಮ-ಯದ
ಸಮ-ಯ-ದಲ್ಲಿ
ಸಮ-ಯ-ದಲ್ಲಿಯೂ
ಸಮ-ಯ-ದಲ್ಲಿಯೇ
ಸಮ-ಯ-ದಲ್ಲೇ
ಸಮ-ಯ-ವನ್ನು
ಸಮ-ಯ-ವಲ್ಲ
ಸಮ-ಯ-ವಾಗಲಿ
ಸಮ-ಯ-ವಿದೆ
ಸಮ-ಯ-ವಿದ್ದಿ-ದ್ದರೆ
ಸಮ-ಯ-ವಿ-ರ-ಲಿಲ್ಲ
ಸಮ-ಯ-ವಿಲ್ಲ
ಸಮ-ಯ-ವಿಲ್ಲ-ದವ-ರಿಗೆ
ಸಮ-ಯವು
ಸಮ-ಯವೇ
ಸಮರ
ಸಮರ-ನೀತಿ
ಸಮರ-ಭೂಮಿ
ಸಮರಸ-ಗೊಳಿ-ಸಿದರು
ಸಮರಸದ
ಸಮರಸವೂ
ಸಮರಾಂಗಣ-ದಲ್ಲಿ
ಸಮರ್ಥ
ಸಮರ್ಥ-ನಾದ
ಸಮರ್ಥನು
ಸಮರ್ಥನೆ-ಗಾಗಿ
ಸಮರ್ಥ-ರಾಗ-ಲಿಲ್ಲ
ಸಮರ್ಥ-ರಾ-ದ-ವರು
ಸಮರ್ಥ-ವಾಗಿ
ಸಮರ್ಥ-ವಾದ
ಸಮರ್ಥಿಸ-ಬಹುದು
ಸಮರ್ಥಿಸಿ-ಕೊಂಡರು
ಸಮರ್ಥಿ-ಸಿದರು
ಸಮರ್ಥಿ-ಸುತ್ತದೆ
ಸಮರ್ಥಿ-ಸುತ್ತವೆ
ಸಮರ್ಥಿಸುತ್ತೇವೆ
ಸಮರ್ಥಿ-ಸುವ
ಸಮರ್ಪಕ-ವಾಗಿ
ಸಮರ್ಪಿ-ಸ-ಬಯಸುತ್ತೇವೆ
ಸಮರ್ಪಿ-ಸಲು
ಸಮರ್ಪಿ-ಸಿತು
ಸಮರ್ಪಿ-ಸಿದರು
ಸಮರ್ಪಿ-ಸು-ತ್ತೇನೆ
ಸಮರ್ಪಿ-ಸುವ
ಸಮ-ವಸ್ಥಿತಮೀಶ್ವರಂ
ಸಮ-ವಸ್ಥಿತಮೀಶ್ವರ-ಮ್
ಸಮಷ್ಟಿ
ಸಮ-ಷ್ಟಿಗೆ
ಸಮ-ಷ್ಟಿ-ಯಲ್ಲಿ
ಸಮ-ಷ್ಟಿ-ಯ-ಲ್ಲಿಯೂ
ಸಮಸ್ತ
ಸಮ-ಸ್ತ-ವನ್ನೂ
ಸಮಸ್ಯೆ
ಸಮಸ್ಯೆ-ಗಳ
ಸಮಸ್ಯೆ-ಗಳನ್ನು
ಸಮಸ್ಯೆ-ಗಳ-ನ್ನೆಲ್ಲಾ
ಸಮಸ್ಯೆ-ಗಳಿ-ಗಿಂತ
ಸಮಸ್ಯೆ-ಗಳಿಗೂ
ಸಮಸ್ಯೆ-ಗಳಿಗೆ
ಸಮಸ್ಯೆ-ಗಳಿ-ಗೆಲ್ಲಾ
ಸಮಸ್ಯೆ-ಗಳು
ಸಮಸ್ಯೆ-ಗಳೆಲ್ಲಾ
ಸಮಸ್ಯೆಯ
ಸಮಸ್ಯೆ-ಯನ್ನು
ಸಮಸ್ಯೆ-ಯನ್ನೂ
ಸಮಸ್ಯೆ-ಯಾಗಿ-ದ್ದು-ದನ್ನು
ಸಮಸ್ಯೆ-ಯಿಂದ
ಸಮಸ್ಯೆ-ಯೆಲ್ಲಾ
ಸಮಾಜ
ಸಮಾಜಕ್ಕೆ
ಸಮಾಜ-ಗಳಲ್ಲಿಯೂ
ಸಮಾಜ-ಗಳಲ್ಲೂ
ಸಮಾಜ-ಗಳು
ಸಮಾಜದ
ಸಮಾಜ-ದಂತಹ
ಸಮಾಜ-ದಲ್ಲಿ
ಸಮಾಜ-ದಲ್ಲಿಯೂ
ಸಮಾಜ-ದಲ್ಲಿ-ರುವ
ಸಮಾಜ-ದಲ್ಲಿಲ್ಲ
ಸಮಾಜ-ದಲ್ಲೂ
ಸಮಾಜ-ವನ್ನು
ಸಮಾಜ-ವನ್ನೂ
ಸಮಾಜ-ವನ್ನೆಲ್ಲಾ
ಸಮಾಜವು
ಸಮಾಜವೂ
ಸಮಾಜ-ಶಕ್ತಿಯ
ಸಮಾಜ-ಸುಧಾರ-ಕನೂ
ಸಮಾಜ-ಸುಧಾರ-ಕರ
ಸಮಾಜ-ಸುಧಾರ-ಕರು
ಸಮಾಜ-ಸು-ಧಾ-ರಣೆ-ಗಳು
ಸಮಾಜ-ಸು-ಧಾ-ರಣೆಗೂ
ಸಮಾಜ-ಸು-ಧಾ-ರಣೆಗೆ
ಸಮಾಜ-ಸು-ಧಾ-ರಣೆಯ
ಸಮಾಜ-ಸು-ಧಾ-ರಣೆ-ಯನ್ನೂ
ಸಮಾ-ಧಾನ
ಸಮಾ-ಧಾನ-ಗಳನ್ನು
ಸಮಾ-ಧಾನ-ವಾದರೂ
ಸಮಾಧಿ-ಯನ್ನು
ಸಮಾಧಿ-ಯಲ್ಲಿ
ಸಮಾನ
ಸಮಾನಂ
ಸಮಾ-ನತೆ
ಸಮಾ-ನ-ತೆಯ
ಸಮಾ-ನ-ತೆ-ಯಿ-ಲ್ಲದೆ
ಸಮಾ-ನ-ರಾಗಿ-ರು-ವರು
ಸಮಾ-ನ-ವಾಗಿ
ಸಮಾ-ನ-ವಾಗಿ-ರುವ
ಸಮಾ-ನ-ವಾದ
ಸಮಾ-ನ-ವಾದದ್ದು
ಸಮಾ-ನ-ವಿಲ್ಲ-ವೆಂದು
ಸಮಾ-ನಾಂತರ-ವಾಗಿ
ಸಮಾ-ನಾರ್ಥ
ಸಮಾನೇ
ಸಮಾ-ಪ್ತಿ-ಗೊಳಿ-ಸಲು
ಸಮಾ-ಪ್ತಿಗೊಳ್ಳುವಂತಿದೆ
ಸಮಾ-ರಂಭ-ವನ್ನು
ಸಮಾ-ಲೋಚಿ-ಸುವ
ಸಮಾ-ವಿಶತು
ಸಮಾ-ವೇಶಗೊಳ್ಳಲು
ಸಮಿ-ತಿಯು
ಸಮೀಪಕ್ಕೆ
ಸಮೀಪದ
ಸಮೀಪದಲ್ಲಿತ್ತು
ಸಮೀಪದಲ್ಲಿ-ರಲು
ಸಮೀಪದಲ್ಲಿ-ರು-ವನು
ಸಮೀಪದ-ವ-ರಾದ
ಸಮೀಪನೂ
ಸಮೀಪವರ್ತಿಗಳಾ-ದಾಗ
ಸಮೀಪ-ವಾದ
ಸಮೀಪವಿ-ರುವ
ಸಮೀಪಿಸಿ-ತೆಂದು
ಸಮೀಪಿಸಿ-ದಂತೆಲ್ಲ
ಸಮೀಪಿಸಿದೆ
ಸಮೀಪಿ-ಸುತ್ತದೆ
ಸಮೀರಣ-ನಂತೆ
ಸಮು-ದಾಯ
ಸಮು-ದಾ-ಯಕ್ಕೆ
ಸಮು-ದಾಯ-ದಿಂದ
ಸಮುದಾ-ಯವೇ
ಸಮುದ್ರ
ಸಮುದ್ರ-ಗಳಾಚೆ
ಸಮುದ್ರ-ಘೋಷಸ್ಪರ್ಧಿ
ಸಮುದ್ರ-ತೀ-ರದ
ಸಮುದ್ರದ
ಸಮುದ್ರ-ದಂತಿ-ರುವ
ಸಮುದ್ರ-ದಲ್ಲಿ
ಸಮುದ್ರ-ದಿಂದ
ಸಮುದ್ರ-ವನ್ನು
ಸಮುದ್ರವು
ಸಮುದ್ರೋಪ
ಸಮೂಹ
ಸಮೂಹದ
ಸಮೂಹವು
ಸಮೃದ್ಧ-ತೆಯ
ಸಮೇತ
ಸಮೇತ-ವಾದ
ಸಮ್ಮತ-ವಾದ
ಸಮ್ಮ-ತವೆ
ಸಮ್ಮುಖ-ದಲ್ಲಿ
ಸಮ್ಮೇಳನ
ಸಮ್ಮೇಳನ-ಕ್ಕಾಗಿ
ಸಮ್ಮೇಳನಕ್ಕೆ
ಸಮ್ಮೇಳನದ
ಸಮ್ಮೇಳನ-ದಂತೆ
ಸಮ್ಮೇಳನ-ದಲ್ಲಿ
ಸಮ್ಮೇಳನ-ವನ್ನು
ಸಮ್ಮೇಳನವು
ಸಮ್ಮೋಹನ
ಸಮ್ಮೋಹನ-ಗೊಳಿಸು
ಸಮ್ಮೋ-ಹಿ-ತ-ರಾಗಿ-ದ್ದಾರೆ
ಸಮ್ಮೋ-ಹಿನಿ
ಸಮ್ಮೋ-ಹಿನೀ
ಸಯುಜಾ
ಸರ-ಕು-ಗಳಂತೆ
ಸರ-ಣಿ-ಗಳು
ಸರ-ಪಣಿಯ
ಸರ-ಪಳಿಯ
ಸರ-ಪಳಿ-ಯನ್ನು
ಸರ-ಪಳಿ-ಯಲ್ಲಿ
ಸರಳ
ಸರ-ಳತೆ
ಸರ-ಳ-ವಾಗಿ
ಸರ-ಳ-ವಾದ
ಸರ-ಳ-ವಾದುವು
ಸರ-ಳವೂ
ಸರ-ಸ್ವತಿ-ಗಳು
ಸರ-ಸ್ವ-ತಿಯ-ವರೂ
ಸರ-ಸ್ಸಿಗೆ
ಸರ-ಹದ್ದಿ
ಸರಾಗ-ವಾಗಿ
ಸರಿ
ಸರಿದ
ಸರಿ-ದಂತೆಲ್ಲಾ
ಸರಿ-ದರೆ
ಸರಿದು
ಸರಿ-ದೂಗ-ಲಾ-ರದು
ಸರಿ-ದೂ-ಗಲಿ
ಸರಿ-ದೊ-ಡ-ನೆಯೇ
ಸರಿ-ಪಡಿ-ಸಲು
ಸರಿ-ಪಡಿಸಿ
ಸರಿ-ಮಾಡಿ-ಕೊಂಡು
ಸರಿ-ಮಾಡಿದ
ಸರಿ-ಯ-ಬೇಕು
ಸರಿ-ಯಲ್ಲ
ಸರಿ-ಯಲ್ಲ-ವೆಂದು
ಸರಿ-ಯಾಗಿ
ಸರಿ-ಯಾಗಿದೆ
ಸರಿ-ಯಾಗಿಯೇ
ಸರಿ-ಯಾಗಿ-ರು-ತ್ತದೆ
ಸರಿ-ಯಾದ
ಸರಿ-ಯಾ-ದುದು
ಸರಿ-ಯಾ-ದುದ್ದು
ಸರಿ-ಯಿರಿ
ಸರಿ-ಯು-ವರು
ಸರಿಯೆ
ಸರಿ-ಯೆಂದು
ಸರಿಯೇ
ಸರಿ-ಸಮ-ರಾಗ-ಬೇ-ಕಾದರೆ
ಸರಿ-ಸಮ-ರಾಗು-ತ್ತೀರಿ
ಸರಿ-ಸಮ-ವಾಗಿ
ಸರಿ-ಸಮ-ವಾಗಿ-ರು-ವುದು
ಸರಿ-ಸ-ಮಾ-ನ-ರಲ್ಲಿ
ಸರಿ-ಸುತ್ತಿರು-ವು-ದನ್ನು
ಸರಿ-ಹೊಂದಿ-ಸುವ
ಸರಿ-ಹೋ-ಗು-ವುದೇ
ಸರೋವ-ರಕ್ಕೂ
ಸರೋವ-ರಕ್ಕೆ
ಸರೋವ-ರದ
ಸರೋ-ವರ-ವಿದೆ
ಸರ್ಕಾರ
ಸರ್ಕಾ-ರಕ್ಕೆ
ಸರ್ಕಾರದ
ಸರ್ಕಾರ-ವಾಗಲಿ
ಸರ್ಕಾರ-ವಿ-ರು-ವು-ದೊಂದು
ಸರ್ಗೋ
ಸರ್ಪಕ್ಕೆ
ಸರ್ಪ-ಗಳ
ಸರ್ಪಭೂತಾದಿ-ಗಳು
ಸರ್ಪರಜ್ಜು
ಸರ್ವ
ಸರ್ವಂ
ಸರ್ವ-ಗತಃ
ಸರ್ವ-ಜೀವಿ-ಗಳ-ಲ್ಲಿಯೂ
ಸರ್ವಜ್ಞ
ಸರ್ವ-ಜ್ಞತ್ವ
ಸರ್ವ-ಜ್ಞನೂ
ಸರ್ವ-ಜ್ಞ-ರಲ್ಲ
ಸರ್ವ-ಜ್ಞ-ರೆಂದು
ಸರ್ವ-ಜ್ಞ-ವಾದ
ಸರ್ವತಃ
ಸರ್ವ-ತೋಕ್ಷಿಶಿರೋ-ಮುಖ-ಮ್
ಸರ್ವ-ತೋ-ಮುಖ-ವಾದ
ಸರ್ವ-ತ್ಯಾಗ-ಮಾಡಿ-ದ-ವನು
ಸರ್ವ-ತ್ಯಾಗಿ-ಗಳಾದ
ಸರ್ವತ್ರ
ಸರ್ವಥಾ
ಸರ್ವದಾ
ಸರ್ವ-ಧರ್ಮ
ಸರ್ವ-ನಾಶ
ಸರ್ವ-ನಾಶದ
ಸರ್ವ-ನಾಶ-ಮಾಡಿವೆ
ಸರ್ವ-ನಾಶ-ವಾಗು-ತ್ತದೆ
ಸರ್ವ-ನಾಶ-ವಾಗುತ್ತಿತ್ತು
ಸರ್ವ-ನಾಶ-ವಾಗು-ವುದು
ಸರ್ವ-ನಾಶವೇ
ಸರ್ವ-ಪ್ರ-ಕಾರ
ಸರ್ವ-ಭೂತ-ಗಳಲ್ಲಿ
ಸರ್ವ-ಭೂತ-ಗಳಲ್ಲಿಯೂ
ಸರ್ವ-ಭೂತಾನಂ
ಸರ್ವ-ಭೂತಾನಾಂ
ಸರ್ವ-ಭೂ-ತೇಷು
ಸರ್ವ-ಮತ
ಸರ್ವ-ಮಾ-ವೃತ್ಯ
ಸರ್ವ-ಮಿದಂ
ಸರ್ವರ
ಸರ್ವ-ರಲ್ಲಿ
ಸರ್ವ-ರಾಷ್ಟ್ರ-ಗಳ
ಸರ್ವ-ರಿಗೂ
ಸರ್ವ-ವನ್ನೂ
ಸರ್ವ-ವಿಧ
ಸರ್ವ-ವಿಶ್ವ-ವಾಗಿಯೂ
ಸರ್ವವೂ
ಸರ್ವ-ವ್ಯಾಪಿ
ಸರ್ವ-ವ್ಯಾ-ಪಿತ್ವ
ಸರ್ವ-ವ್ಯಾ-ಪಿ-ಯಾಗಿ-ರು-ವನು
ಸರ್ವ-ವ್ಯಾ-ಪಿ-ಯಾಗಿ-ರು-ವುದು
ಸರ್ವ-ವ್ಯಾ-ಪಿ-ಯಾದ
ಸರ್ವ-ವ್ಯಾ-ಪಿ-ಯಾ-ದರೂ
ಸರ್ವ-ವ್ಯಾ-ಪಿಯೂ
ಸರ್ವ-ವ್ಯಾ-ಪಿ-ಯೆಂದೂ
ಸರ್ವ-ವ್ಯಾಪ್ತಿ
ಸರ್ವ-ಶಕ್ತ
ಸರ್ವ-ಶಕ್ತನ
ಸರ್ವ-ಶಕ್ತ-ನಾದ
ಸರ್ವ-ಶಕ್ತನು
ಸರ್ವ-ಶಕ್ತನೂ
ಸರ್ವ-ಶಕ್ತಳು
ಸರ್ವ-ಶಕ್ತ-ವಾದುದು
ಸರ್ವ-ಶಕ್ತಿ
ಸರ್ವ-ಶಕ್ತಿ-ಮಾ-ನ್
ಸರ್ವ-ಶಕ್ತಿ-ವಂತರು
ಸರ್ವ-ಶ್ರೇಷ್ಠ
ಸರ್ವ-ಶ್ರೇಷ್ಠರು
ಸರ್ವ-ಸಂಗ
ಸರ್ವ-ಸ-ಮಾ-ನ-ತೆ-ಯಲ್ಲಿ
ಸರ್ವ-ಸಮ್ಮತ
ಸರ್ವ-ಸಮ್ಮತ-ವಾದ
ಸರ್ವ-ಸಾ-ಮಾನ್ಯ
ಸರ್ವ-ಸಾ-ಮಾ-ನ್ಯ-ವಾದ
ಸರ್ವ-ಸಾ-ಮಾ-ನ್ಯವೂ
ಸರ್ವಸ್ವ
ಸರ್ವ-ಸ್ವ-ವನ್ನು
ಸರ್ವ-ಸ್ವ-ವನ್ನೂ
ಸರ್ವ-ಸ್ವವೆ
ಸರ್ವಾಂತರ್ಗ-ತನೂ
ಸರ್ವಾ-ತೀತ
ಸರ್ವಾಧಿ
ಸರ್ವಾ-ಧಿ-ಕಾರ-ವನ್ನು
ಸರ್ವಾ-ವ-ಯವ
ಸರ್ವೇಚ್ಛಾ-ಶಕ್ತಿ
ಸರ್ವೇಶ್ವರ
ಸರ್ವೇಷು
ಸರ್ವೋಚ್ಚ
ಸರ್ವೋ-ಚ್ಚವೂ
ಸರ್ವೋ-ತ್ಕೃಷ್ಟ
ಸರ್ವೋ-ತ್ಕೃಷ್ಟ-ವಾದ
ಸರ್ವೋ-ತ್ತಮ
ಸರ್ವೋ-ತ್ತಮ-ವಾದ
ಸರ್ವೋ-ದಯ
ಸರ್ವೋ-ನ್ನತ
ಸಲ
ಸಲ-ಕ-ರಣೆಯ
ಸಲಕ್ಕೆ
ಸಲ-ವಲ್ಲ
ಸಲವೂ
ಸಲಹೆ
ಸಲು-ವಾಗಿ
ಸಲ್ಲ-ಬೇಕಾಗಿದೆ
ಸಲ್ಲ-ಬೇಕಾಗಿ-ರು-ವುದು
ಸಲ್ಲ-ಬೇ-ಕಾದ
ಸಲ್ಲ-ಬೇಕು
ಸಲ್ಲಿ-ಸದೇ
ಸಲ್ಲಿಸ-ಬಯಸು-ತ್ತೇನೆ
ಸಲ್ಲಿಸ-ಬೇಕಾಗಿ-ರುವ
ಸಲ್ಲಿಸ-ಬೇಕು
ಸಲ್ಲಿ-ಸಲು
ಸಲ್ಲಿಸಿ
ಸಲ್ಲಿ-ಸಿದ
ಸಲ್ಲಿಸಿ-ದ್ದೀರಿ
ಸಲ್ಲಿಸಿ-ರುವ
ಸಲ್ಲಿಸುತ್ತಿ-ರು-ವನು
ಸಲ್ಲಿಸುತ್ತೇವೆ
ಸಲ್ಲಿ-ಸುವ
ಸಲ್ಲಿಸು-ವಂತಹ
ಸಲ್ಲಿ-ಸು-ವರೋ
ಸಲ್ಲಿಸು-ವುದು
ಸಲ್ಲು-ತ್ತವೆ
ಸಲ್ಲುವ
ಸಲ್ಲು-ವಂತೆ
ಸಲ್ಲು-ವುದು
ಸವಲತ್ತು
ಸವಾರಿ
ಸವಾ-ಲನ್ನು
ಸವಾಲು
ಸವಿ
ಸವಿಸ್ತಾರ-ವಾಗಿ
ಸವಿಸ್ತಾರ-ವಾದ
ಸವೆದು
ಸವೆ-ಯಿ-ಸಲು
ಸವೆ-ಯಿಸಿ-ದರೂ
ಸವೆ-ಸಲು
ಸಸಿ-ಯನ್ನು
ಸಹ
ಸಹ-ಕಾರ-ವನ್ನೂ
ಸಹ-ಕಾರ-ವಿ-ಲ್ಲದೆ
ಸಹ-ಕಾರಿ
ಸಹ-ಕಾರಿ-ಯಾ-ದುದು
ಸಹಜ
ಸಹ-ಜ-ವಾಗಿ
ಸಹ-ಜ-ವಾಗಿ-ತ್ತು
ಸಹ-ಜ-ವಾಗಿಯೇ
ಸಹ-ಜ-ವಾದ
ಸಹ-ಜ-ವಾದುವು-ಗಳು
ಸಹ-ಜವೇ
ಸಹ-ಜ-ಶಕ್ತಿ-ಗಳನ್ನು
ಸಹ-ಧರ್ಮೀ-ಯರ
ಸಹ-ಧರ್ಮೀಯ-ರಾದ
ಸಹನೆ
ಸಹ-ನೆ-ಯಿಲ್ಲ
ಸಹಸ್ರ
ಸಹಸ್ರ-ವರ್ಷ-ಗಳ
ಸಹಸ್ರಾಕ್ಷ-ಗಳಿ-ದ್ದರೂ
ಸಹಸ್ರಾರು
ಸಹಾ
ಸಹಾನು
ಸಹಾ-ನು-ಭೂತಿ
ಸಹಾ-ನು-ಭೂತಿ-ಗಳಿಂದ
ಸಹಾ-ನು-ಭೂತಿ-ಗಳಿಗೆ
ಸಹಾ-ನು-ಭೂತಿಯ
ಸಹಾ-ನು-ಭೂತಿ-ಯನ್ನು
ಸಹಾ-ನು-ಭೂತಿ-ಯನ್ನೂ
ಸಹಾ-ನು-ಭೂತಿ-ಯಿಂದ
ಸಹಾ-ನು-ಭೂತಿ-ಯಿಂದಿರು
ಸಹಾಯ
ಸಹಾ-ಯಕ
ಸಹಾ-ಯ-ಕ-ವಾಗಲಿ
ಸಹಾ-ಯ-ಕ-ವಾದು-ದೆಲ್ಲ-ವನ್ನೂ
ಸಹಾ-ಯ-ಕ್ಕಾಗಿ
ಸಹಾ-ಯಕ್ಕೆ
ಸಹಾ-ಯ-ಗಳನ್ನು
ಸಹಾ-ಯ-ಗಳ-ನ್ನೆಲ್ಲಾ
ಸಹಾ-ಯ-ದಿಂದ
ಸಹಾ-ಯ-ದಿಂದಲೇ
ಸಹಾ-ಯ-ಮಾಡ-ಬೇಕು
ಸಹಾ-ಯ-ಮಾಡಲಿ
ಸಹಾ-ಯ-ಮಾಡಲು
ಸಹಾ-ಯ-ಮಾಡ-ವನೊ
ಸಹಾ-ಯ-ಮಾಡಿ
ಸಹಾ-ಯ-ಮಾಡಿ-ದಂತೆ
ಸಹಾ-ಯ-ಮಾಡಿ-ದಷ್ಟೂ
ಸಹಾ-ಯ-ಮಾಡಿದೆ
ಸಹಾ-ಯ-ಮಾಡಿ-ಲಾರಿರಿ
ಸಹಾ-ಯ-ಮಾಡುವ
ಸಹಾ-ಯ-ಮಾಡು-ವಂತೆ
ಸಹಾ-ಯ-ಮಾಡು-ವುದೋ
ಸಹಾ-ಯ-ವನ್ನು
ಸಹಾ-ಯ-ವನ್ನೂ
ಸಹಾ-ಯ-ವಾಗು-ವಂತಹ
ಸಹಾ-ಯ-ವಿ-ಲ್ಲದೆ
ಸಹಾ-ಯವೂ
ಸಹಾ-ಯ-ವೆಲ್ಲ
ಸಹಿತ
ಸಹಿಷ್ಣು
ಸಹಿ-ಷ್ಣು-ಗಳಾಗಲಿ
ಸಹಿ-ಷ್ಣುತಾ
ಸಹಿ-ಷ್ಣುತೆ
ಸಹಿ-ಷ್ಣು-ತೆ-ಗಳ
ಸಹಿ-ಷ್ಣು-ತೆ-ಗಳಿಂದ
ಸಹಿ-ಷ್ಣು-ತೆಯ
ಸಹಿ-ಷ್ಣು-ವಾದ
ಸಹಿಸ
ಸಹಿ-ಸ-ಬಲ್ಲದು
ಸಹಿ-ಸ-ಬಲ್ಲ-ನೆಂದು
ಸಹಿ-ಸ-ಲಾ-ರದೆ
ಸಹಿ-ಸ-ಲಾರೆ
ಸಹಿ-ಸಲು
ಸಹಿ-ಸಿದ
ಸಹಿ-ಸು-ವುದು
ಸಹೃ-ದಯತೆ
ಸಹೋ-ದ-ದರೇ
ಸಹೋ-ದರ
ಸಹೋ-ದ-ರತ್ವ
ಸಹೋ-ದ-ರನ
ಸಹೋ-ದ-ರನೇ
ಸಹೋ-ದ-ರರ
ಸಹೋ-ದ-ರರು
ಸಹೋ-ದ-ರರೆ
ಸಹೋ-ದ-ರರೇ
ಸಹ್ಯ
ಸಾಂಖ್ಯ
ಸಾಂಖ್ಯಂ
ಸಾಂಖ್ಯರು
ಸಾಂಖ್ಯರೇ
ಸಾಂಖ್ಯ-ರೊಬ್ಬ-ರನ್ನು
ಸಾಂತ
ಸಾಂತದ
ಸಾಂತ-ದಲ್ಲಿ
ಸಾಂತ-ವಾದ
ಸಾಂತ-ವಾದುದು
ಸಾಂತ-ವೆಂದು
ಸಾಂಪ್ರದಾಯಿಕ
ಸಾಂಪ್ರದಾಯಿಕ-ವಾದ
ಸಾಂಪ್ರಾದಾಯಿ-ಕತೆಯ
ಸಾಂಸಾರಿ-ಕತೆ
ಸಾಂಸ್ಕೃತಿಕ
ಸಾಕ-ಲ್ಲವೇ
ಸಾಕಷ್ಟು
ಸಾಕಾಗ-ಬಹುದು
ಸಾಕಾಗಿತ್ತು
ಸಾಕಾ-ಗಿರು-ವಾಗ
ಸಾಕಾಗಿ-ಹೋ-ಗಿದೆ
ಸಾಕಾ-ಗುವ
ಸಾಕಾ-ದಷ್ಟು
ಸಾಕಾರ
ಸಾಕಾರ-ದೇ-ವರು
ಸಾಕಾರ-ನಾದ
ಸಾಕಾರ-ನೆಂದೂ
ಸಾಕಾರ-ಮೂರ್ತಿಯೂ
ಸಾಕು
ಸಾಕೆಂದು
ಸಾಕ್ರಟೀ-ಸ್
ಸಾಕ್ಷರ
ಸಾಕ್ಷಾ-ತ್
ಸಾಕ್ಷಾತ್ಕಾರ
ಸಾಕ್ಷಾತ್ಕಾರ-ಕ್ಕಾಗಿ
ಸಾಕ್ಷಾತ್ಕಾರಕ್ಕೆ
ಸಾಕ್ಷಾತ್ಕಾರದ
ಸಾಕ್ಷಾತ್ಕಾರ-ವನ್ನು
ಸಾಕ್ಷಾತ್ಕಾರ-ವಾಗಿ-ದೆಯೇ
ಸಾಕ್ಷಾತ್ಕಾರ-ವಾಗಿ-ದ್ದರೆ
ಸಾಕ್ಷಾತ್ಕಾರ-ವಾಗಿ-ಲ್ಲದೇ
ಸಾಕ್ಷಾತ್ಕಾರ-ವಾಗು-ವುದೋ
ಸಾಕ್ಷಾತ್ಕಾರ-ವಿಲ್ಲ-ವೆಂದು
ಸಾಕ್ಷಾತ್ಕಾರವು
ಸಾಕ್ಷಾತ್ಕಾರಿ-ಸಿಕೊಳ್ಳ
ಸಾಕ್ಷಿ
ಸಾಕ್ಷಿ-ಗಳೆಂಬ
ಸಾಕ್ಷಿ-ಭೂತ-ರಾ-ದಾಗ
ಸಾಕ್ಷಿ-ಮಾ-ತ್ರ-ರಾಗಿ
ಸಾಕ್ಷಿ-ಯಾಗಿದೆ
ಸಾಕ್ಷಿ-ಯಾಗಿ-ದ್ದಾ-ನೆಯೋ
ಸಾಕ್ಷಿ-ಯಾಗಿ-ದ್ದಾರೆ
ಸಾಕ್ಷಿ-ಯಾಗಿವೆ
ಸಾಕ್ಷಿ-ಯಾ-ದ-ವ-ನಿಗೆ
ಸಾಕ್ಷಿ-ಯಾ-ದ-ವನು
ಸಾಕ್ಷಿ-ಯಾ-ದು-ದಕ್ಕೆ
ಸಾಕ್ಷಿಯೊ
ಸಾಕ್ಷಿ-ಯೊಂದೇ
ಸಾಕ್ಷಿಯೋ
ಸಾಕ್ಷ್ಯ-ಗಳಾಗಿವೆ
ಸಾಕ್ಸನ್
ಸಾಗರ
ಸಾಗರ-ಗಳ
ಸಾಗರ-ಗಳನ್ನು
ಸಾಗರದ
ಸಾಗರ-ದಷ್ಟು
ಸಾಗರ-ವನ್ನು
ಸಾಗರ-ವನ್ನೇ
ಸಾಗರ-ವಿದೆ
ಸಾಗರ-ವಿರು-ವುದು
ಸಾಗ-ರೋಪಮ
ಸಾಗಿ
ಸಾಗಿತು
ಸಾಗಿ-ದಂತೆಲ್ಲಾ
ಸಾಗಿ-ದರೆ
ಸಾಗಿದೆ
ಸಾಗಿರಿ
ಸಾಗಿ-ಸುವ
ಸಾಗಿ-ಹೋ-ಗ-ಬೇಕಾಗಿಲ್ಲ
ಸಾಗಿ-ಹೋ-ಗ-ಬೇಕಾ-ಯಿತು
ಸಾಗಿ-ಹೋ-ಗು-ವುದೇ
ಸಾಗು-ತ್ತಿದೆ
ಸಾಗುತ್ತಿ-ರುವ
ಸಾಗು-ವಂತಿಲ್ಲ
ಸಾಗು-ವನು
ಸಾಗು-ವು-ದಿಲ್ಲ
ಸಾಗು-ವುದು
ಸಾಧಕನಾಗ-ಬೇಕಾಗಿ-ತ್ತು
ಸಾಧಕ-ನಿಗೆ
ಸಾಧ-ಕರು
ಸಾಧನ
ಸಾಧನ-ಗಳ
ಸಾಧನ-ಗಳಾಗಿವೆ
ಸಾಧನಾ-ರಂಗ-ದಲ್ಲಿ
ಸಾಧನೆ
ಸಾಧ-ನೆ-ಗಳಿ-ಗಾಗಿ
ಸಾಧ-ನೆ-ಗಳು
ಸಾಧ-ನೆ-ಗಾಗಿ
ಸಾಧ-ನೆಗೆ
ಸಾಧ-ನೆ-ಮಾಡಿ
ಸಾಧ-ನೆಯ
ಸಾಧ-ನೆ-ಯನ್ನು
ಸಾಧ-ನೆ-ಯಲ್ಲಿ
ಸಾಧ-ನೆ-ಯಿಂದಲೂ
ಸಾಧ-ನೆ-ಯಿಂದಾಗಿ
ಸಾಧ-ನೆಯೂ
ಸಾಧ-ನೆಯೇ
ಸಾಧಾರಣ
ಸಾಧಾರಣ-ನಾದ-ವನು
ಸಾಧಾರಣ-ವಾಗಿ
ಸಾಧಾರಣ-ವಾಗಿವೆ
ಸಾಧಿತ-ವಾಗಿ-ರುವ
ಸಾಧಿಸ
ಸಾಧಿ-ಸದೆ
ಸಾಧಿಸ-ಬಲ್ಲ
ಸಾಧಿಸ-ಬಲ್ಲಿರಿ
ಸಾಧಿಸ-ಬಲ್ಲೆ
ಸಾಧಿಸ-ಬಲ್ಲೆವು
ಸಾಧಿಸ-ಬಹುದು
ಸಾಧಿಸ-ಬೇಕಾಗಿದೆ
ಸಾಧಿಸ-ಬೇಕಾಗಿ-ರುವ
ಸಾಧಿಸ-ಬೇ-ಕಾದ
ಸಾಧಿಸ-ಬೇಕು
ಸಾಧಿಸ-ಲಾ-ರದು
ಸಾಧಿಸ-ಲಾರೆವು
ಸಾಧಿ-ಸಲಿ
ಸಾಧಿಸ-ಲಿಲ್ಲ
ಸಾಧಿ-ಸಲು
ಸಾಧಿಸ-ಲೇ-ಬೇಕು
ಸಾಧಿಸ-ಲೋಸುಗ
ಸಾಧಿಸಿ
ಸಾಧಿಸಿ-ಕೊಳ್ಳ-ಬೇಕು
ಸಾಧಿ-ಸಿತು
ಸಾಧಿಸಿ-ತ್ತು
ಸಾಧಿ-ಸಿದ
ಸಾಧಿಸಿ-ದರು
ಸಾಧಿಸಿ-ದ-ವರು
ಸಾಧಿಸಿ-ದವು
ಸಾಧಿಸಿ-ದಷ್ಟು
ಸಾಧಿಸಿ-ದು-ದ-ಕ್ಕಿಂತಲೂ
ಸಾಧಿ-ಸಿದೆ
ಸಾಧಿಸಿ-ದ್ದರೂ
ಸಾಧಿಸಿ-ದ್ದೆವು
ಸಾಧಿಸಿರಿ
ಸಾಧಿಸಿ-ರುವ
ಸಾಧಿ-ಸಿವೆ
ಸಾಧಿಸುತ್ತಿ-ರುವ
ಸಾಧಿಸು-ತ್ತೀರಿ
ಸಾಧಿಸುತ್ತೀರೊ
ಸಾಧಿಸು-ತ್ತೇನೆ
ಸಾಧಿಸುತ್ತೇವೆ
ಸಾಧಿ-ಸುವ
ಸಾಧಿ-ಸು-ವರು
ಸಾಧಿಸು-ವಲ್ಲಿ
ಸಾಧಿಸು-ವಿರಿ
ಸಾಧಿ-ಸು-ವು-ದಕ್ಕೆ
ಸಾಧಿಸು-ವುದು
ಸಾಧಿ-ಸು-ವುದೂ
ಸಾಧಿ-ಸೋಣ
ಸಾಧು
ಸಾಧು-ಗಳ
ಸಾಧು-ಗಳು
ಸಾಧು-ವನ್ನಾ-ದರೂ
ಸಾಧುವು
ಸಾಧುವೂ
ಸಾಧು-ಸಂತರ
ಸಾಧು-ಸ್ವ-ಭಾವ-ದ-ವರು
ಸಾಧೂನಾಂ
ಸಾಧ್ಯ
ಸಾಧ್ಯ-ತೆ-ಯನ್ನೂ
ಸಾಧ್ಯನೂ
ಸಾಧ್ಯ-ವಾಗದ
ಸಾಧ್ಯ-ವಾಗಲಿ
ಸಾಧ್ಯ-ವಾಗ-ಲಿಲ್ಲ
ಸಾಧ್ಯ-ವಾಗಿದೆ
ಸಾಧ್ಯ-ವಾಗಿ-ರ-ಬೇಕು
ಸಾಧ್ಯ-ವಾಗಿಲ್ಲ
ಸಾಧ್ಯ-ವಾಗು-ತ್ತದೆ
ಸಾಧ್ಯ-ವಾಗುತ್ತಿತ್ತು
ಸಾಧ್ಯ-ವಾಗು-ತ್ತಿ-ರಲಿಲ್ಲ
ಸಾಧ್ಯ-ವಾಗು-ತ್ತಿಲ್ಲ
ಸಾಧ್ಯ-ವಾಗು-ವು-ದಿಲ್ಲ
ಸಾಧ್ಯ-ವಾಗು-ವುದೇನೋ
ಸಾಧ್ಯ-ವಾಗು-ವುದೋ
ಸಾಧ್ಯ-ವಾದ
ಸಾಧ್ಯ-ವಾದರೂ
ಸಾಧ್ಯ-ವಾದರೆ
ಸಾಧ್ಯ-ವಾದ-ಷ್ಟನ್ನು
ಸಾಧ್ಯ-ವಾದಷ್ಟು
ಸಾಧ್ಯ-ವಾದು-ದ-ಕ್ಕಿಂತ
ಸಾಧ್ಯ-ವಾದು-ದನ್ನು
ಸಾಧ್ಯ-ವಾದುದನ್ನೆಲ್ಲ
ಸಾಧ್ಯ-ವಾದುದು
ಸಾಧ್ಯ-ವಾ-ಯಿತು
ಸಾಧ್ಯ-ವಿಲ್ಲ
ಸಾಧ್ಯ-ವಿಲ್ಲದ
ಸಾಧ್ಯ-ವಿಲ್ಲ-ದಷ್ಟು
ಸಾಧ್ಯ-ವಿ-ಲ್ಲದೆ
ಸಾಧ್ಯ-ವಿ-ಲ್ಲದೇ
ಸಾಧ್ಯ-ವಿಲ್ಲ-ವೆಂಬು-ದನ್ನು
ಸಾಧ್ಯ-ವಿ-ಲ್ಲವೋ
ಸಾಧ್ಯವೂ
ಸಾಧ್ಯ-ವೆಂದು
ಸಾಧ್ಯ-ವೆಂಬುದು
ಸಾಧ್ಯವೇ
ಸಾಧ್ಯ-ವೇ-ನಾವು
ಸಾನ್ನಿಧ್ಯ-ದಿಂದ
ಸಾನ್ನಿಧ್ಯವು
ಸಾಮಗ್ರಿ
ಸಾಮಗ್ರಿ-ಗಳು
ಸಾಮರಸ್ಯ
ಸಾಮರಸ್ಯದ
ಸಾಮರಸ್ಯ-ವನ್ನು
ಸಾಮರಸ್ಯ-ವಿದೆ
ಸಾಮರಸ್ಯ-ವಿರ-ಬೇಕು
ಸಾಮರಸ್ಯ-ವೆಂಬ
ಸಾಮರ್ಥ್ಯ
ಸಾಮರ್ಥ್ಯಕ್ಕೆ
ಸಾಮರ್ಥ್ಯ-ದಲ್ಲಿ
ಸಾಮರ್ಥ್ಯ-ವನ್ನು
ಸಾಮರ್ಥ್ಯ-ವಿಲ್ಲ-ದ-ವ-ರಾಗಿ
ಸಾಮರ್ಥ್ಯ-ವಿಲ್ಲ-ದ-ವರು
ಸಾಮರ್ಥ್ಯವೂ
ಸಾಮ-ವೇದ-ದಲ್ಲಿ
ಸಾಮಾ-ಜಿಕ
ಸಾಮಾ-ಜಿಕ-ವನ್ನಾ-ಗಲೀ
ಸಾಮಾ-ಜಿಕ-ವಾಗಲೀ
ಸಾಮಾ-ನಿನ
ಸಾಮಾನ್ಯ
ಸಾಮಾ-ನ್ಯ-ರಲ್ಲಿ
ಸಾಮಾ-ನ್ಯ-ರಿಗೆ
ಸಾಮಾ-ನ್ಯ-ವಾಗಿ
ಸಾಮಾ-ನ್ಯ-ವಾದ
ಸಾಮಾ-ನ್ಯ-ವೆಂದು
ಸಾಮೀಪ್ಯ-ವನ್ನು
ಸಾಮ್ಯೇ
ಸಾಮ್ರಾಜ್ಯ
ಸಾಮ್ರಾಜ್ಯದ
ಸಾಮ್ರಾಜ್ಯವು
ಸಾಯಣಾ-ಚಾರ್ಯರು
ಸಾಯ-ದಂತೆ
ಸಾಯ-ಬಲ್ಲೆ-ವೆಂಬ
ಸಾಯಬೇಕಾಗು-ವುದು
ಸಾಯ-ಬೇಕು
ಸಾಯಲೇ-ಬೇಕು
ಸಾಯಿರಿ
ಸಾಯಿಸು-ವುದೊ-ಳಿತು
ಸಾಯುಜ್ಯ-ಪದವಿ
ಸಾಯುತ್ತಾ
ಸಾಯುತ್ತಾನೆ
ಸಾಯು-ತ್ತಿದೆ
ಸಾಯು-ತ್ತಿದ್ದ
ಸಾಯುವ
ಸಾಯು-ವರು
ಸಾಯುವ-ವ-ರಿಗೆ
ಸಾಯು-ವಿರಿ
ಸಾಯು-ವುದು
ಸಾಯು-ವೆವು
ಸಾಯೋಣ
ಸಾರ
ಸಾರ-ದಂತಿ-ರುವ
ಸಾರ-ಬೇಕಾಗಿದೆ
ಸಾರ-ಬೇ-ಕಾದ
ಸಾರ-ಬೇಕು
ಸಾರಲು
ಸಾರ-ವನ್ನು
ಸಾರ-ವನ್ನೆಲ್ಲ
ಸಾರ-ವಾಗಿ-ರು-ವನು
ಸಾರವು
ಸಾರವೂ
ಸಾರ-ವೆಂದು
ಸಾರವೇ
ಸಾರ-ಸ-ತ್ವವೂ
ಸಾರಾಂಶ-ವನ್ನು
ಸಾರಾಂಶವು
ಸಾರಾಂಶವೇ-ನೆಂದರೆ
ಸಾರಿ
ಸಾರಿಗೆ
ಸಾರಿತು
ಸಾರಿತೋ
ಸಾರಿದ
ಸಾರಿ-ದಂತೆ
ಸಾರಿ-ದನೋ
ಸಾರಿ-ದರು
ಸಾರಿ-ದವು
ಸಾರಿ-ದಾಗ
ಸಾರಿದೆ
ಸಾರಿ-ದ್ದಾನಷ್ಟೆ
ಸಾರಿ-ದ್ದಾನೆ
ಸಾರಿ-ದ್ದಾ-ಯಿತು
ಸಾರಿ-ದ್ದೀರಿ
ಸಾರಿದ್ದು
ಸಾರಿ-ರು-ವನು
ಸಾರಿ-ರು-ವರು
ಸಾರಿ-ರು-ವುದು
ಸಾರಿ-ಹೇ-ಳು-ತ್ತೇನೆ
ಸಾರು-ತ್ತದೆ
ಸಾರು-ತ್ತವೆ
ಸಾರುತ್ತಾನೆ
ಸಾರುತ್ತಿತ್ತು
ಸಾರು-ತ್ತಿದೆ
ಸಾರು-ತ್ತಿ-ರು-ವಂತೆ
ಸಾರುತ್ತಿ-ರು-ವರು
ಸಾರುತ್ತಿರು-ವುವು
ಸಾರು-ತ್ತಿವೆ
ಸಾರುತ್ತಿ-ವೆಯೋ
ಸಾರು-ತ್ತೇನೆ
ಸಾರುವ
ಸಾರು-ವಂತೆ
ಸಾರು-ವರು
ಸಾರುವ-ವರು
ಸಾರು-ವು-ದಕ್ಕೆ
ಸಾರು-ವು-ದನ್ನು
ಸಾರು-ವು-ದಾದರೆ
ಸಾರು-ವು-ದಿಲ್ಲ
ಸಾರು-ವುದು
ಸಾರು-ವುದೇ
ಸಾರು-ವುವು
ಸಾರೋಣ
ಸಾರ್ಥಕ-ತೆಯೇನು
ಸಾರ್ಥಕ-ವಾದ
ಸಾರ್ಥಕ-ವಾದುದು
ಸಾರ್ವಕಾಲಿಕ
ಸಾರ್ವಜನಿಕ
ಸಾರ್ವಜನಿಕ-ವಾಗಿ
ಸಾರ್ವತ್ರಿಕ
ಸಾರ್ವತ್ರಿಕ-ವಾಗಿ
ಸಾರ್ವತ್ರಿಕ-ವಾದ
ಸಾರ್ವಭೌಮತ್ವ-ವನ್ನು
ಸಾರ್ವಭೌಮಿಕ
ಸಾರ್ವಭೌಮಿಕ-ವಾಗ-ಬೇಕು
ಸಾಲದ
ಸಾಲದು
ಸಾಲದು-ದ-ಕ್ಕಾಗಿಯೋ
ಸಾಲಿ-ಗ್ರಾಮ
ಸಾವಧಾ-ನ-ದಿಂದ
ಸಾವಧಾ-ನ-ವಾಗಿ
ಸಾವನ್ನರಿ-ಯದ
ಸಾವಿಗೆ
ಸಾವಿನ
ಸಾವಿರ
ಸಾವಿರ-ದಲ್ಲಿ
ಸಾವಿರ-ಪಾಲು
ಸಾವಿ-ರಾರು
ಸಾವಿಲ್ಲ
ಸಾವು
ಸಾಹಸ
ಸಾಹ-ಸಕ್ಕೆ
ಸಾಹಸ-ಗಳನ್ನು
ಸಾಹಸ-ಗಳಿಗೆ
ಸಾಹ-ಸದ
ಸಾಹಸ-ದಿಂದ
ಸಾಹಸ-ದಿಂದಲೂ
ಸಾಹಸ-ಪೂರ್ಣ
ಸಾಹಸ-ಪೂರ್ಣ-ವಾಗಿ
ಸಾಹಸ-ವನ್ನು
ಸಾಹಸ-ವಿದೆ
ಸಾಹಸಿ-ಗಳು
ಸಾಹಿತ್ಯ
ಸಾಹಿ-ತ್ಯಕ್ಕೆ
ಸಾಹಿ-ತ್ಯ-ಗಳಲ್ಲಿ
ಸಾಹಿ-ತ್ಯ-ಗಳು
ಸಾಹಿ-ತ್ಯದ
ಸಾಹಿ-ತ್ಯ-ದಲ್ಲಿ
ಸಾಹಿ-ತ್ಯ-ದಲ್ಲಿಯೂ
ಸಾಹಿ-ತ್ಯ-ದಲ್ಲೆಲ್ಲ
ಸಾಹಿ-ತ್ಯ-ರಾಶಿ
ಸಾಹಿ-ತ್ಯ-ವನ್ನು
ಸಾಹಿ-ತ್ಯ-ವನ್ನೂ
ಸಾಹಿ-ತ್ಯ-ವಾಗಲಿ
ಸಾಹಿ-ತ್ಯವು
ಸಾಹಿ-ತ್ಯ-ವೆಲ್ಲಾ
ಸಾಹಿ-ತ್ಯಾದಿ
ಸಿಂಗ-ರಂತೆ
ಸಿಂಘಂ
ಸಿಂಧು-ವಿ-ನಿಂದ
ಸಿಂಧೂ
ಸಿಂಧೂ-ನ-ದಿಯ
ಸಿಂಹ
ಸಿಂಹ-ಗಳ-ನ್ನಾಗಿ
ಸಿಂಹದ
ಸಿಂಹ-ದಂತೆ
ಸಿಂಹ-ಧೈರ್ಯ
ಸಿಂಹ-ವಾಗ
ಸಿಂಹಾ-ಸ-ನಕ್ಕೆ
ಸಿಂಹಾ-ಸ-ನದ
ಸಿಂಹಾ-ಸನ-ವನ್ನು
ಸಿಕ್ಕದೆ
ಸಿಕ್ಕ-ಲಿಲ್ಲ
ಸಿಕ್ಕಲೇ-ಬೇಕು
ಸಿಕ್ಕಿ
ಸಿಕ್ಕಿತು
ಸಿಕ್ಕಿದ
ಸಿಕ್ಕಿ-ದರೆ
ಸಿಕ್ಕಿ-ದಾಗ
ಸಿಕ್ಕಿ-ದಾ-ಗಲೋ
ಸಿಕ್ಕಿ-ದ್ದು-ದನ್ನು
ಸಿಕ್ಕಿಲ್ಲ
ಸಿಕ್ಕಿಸಿ
ಸಿಕ್ಕುವ
ಸಿಕ್ಕು-ವರು
ಸಿಕ್ಕುವ-ವ-ರೆಗೆ
ಸಿಕ್ಕು-ವುದು
ಸಿಕ್ಕು-ವು-ದೆಂದು
ಸಿಗು-ವು-ದೆಂದು
ಸಿಡಿಮದ್ದಿ-ನಂತೆ
ಸಿಡಿಯ-ಬಹುದು
ಸಿಡಿ-ಯು-ತ್ತಿದ್ದರೆ
ಸಿಡಿಲಾಳ್ಮೆ
ಸಿಡಿಲಿ-ನಂತೆ
ಸಿದ
ಸಿದ್ದ
ಸಿದ್ದ-ರಾಗಿ-ರ-ಬೇಕೆಂದು
ಸಿದ್ದ-ರಾಗಿ-ರುವ
ಸಿದ್ದ-ರಾಗೋಣ
ಸಿದ್ದಾಂತಕ್ಕೆ
ಸಿದ್ದಾಂತ-ಗಳಲ್ಲಿ
ಸಿದ್ದಾಂತ-ಗಳು
ಸಿದ್ದಾಂತದ
ಸಿದ್ದಾಂತ-ವನ್ನು
ಸಿದ್ದಾಂತ-ವನ್ನೊ
ಸಿದ್ದಾಂತ-ವನ್ನೊ-ಪ್ಪು-ವಂತೆ
ಸಿದ್ದಾಂತ-ವಿದೆ
ಸಿದ್ದ್ದಾಂತ-ಗಳೆ-ಲ್ಲವೂ
ಸಿದ್ಧ
ಸಿದ್ಧ-ಗೊಳಿಸಿ
ಸಿದ್ಧತೆ
ಸಿದ್ಧ-ತೆ-ಗಳು
ಸಿದ್ಧ-ನಾಗಿ-ರು-ವನು
ಸಿದ್ಧ-ನಾ-ಗಿರು-ವೆನು
ಸಿದ್ಧ-ಪುರಷ-ರಾಗಿ-ರ-ಬೇಕಾಗಿ-ತ್ತು
ಸಿದ್ಧ-ಮಾ-ನವ
ಸಿದ್ಧ-ರಲ್ಲದ
ಸಿದ್ಧ-ರಾಗ-ಬೇಕಾ-ದಂತಹ
ಸಿದ್ಧ-ರಾಗ-ಬೇಕು
ಸಿದ್ಧ-ರಾಗಿ
ಸಿದ್ಧ-ರಾಗಿ-ದ್ದೀರೊ
ಸಿದ್ಧ-ರಾಗಿ-ದ್ದೇವೆ
ಸಿದ್ಧ-ರಾಗಿ-ರ-ಬೇಕು
ಸಿದ್ಧ-ರಾಗಿ-ರು-ವರು
ಸಿದ್ಧ-ವಾಗಿ
ಸಿದ್ಧ-ವಾಗಿದೆ
ಸಿದ್ಧ-ವಾಗಿ-ರ-ಬಹುದು
ಸಿದ್ಧ-ವಾಗಿ-ರು-ವುದು
ಸಿದ್ಧ-ವಾಗಿ-ರು-ವೆವು
ಸಿದ್ಧ-ವಾಗು-ತ್ತದೆ
ಸಿದ್ಧಾಂತ
ಸಿದ್ಧಾಂತಕ್ಕೆ
ಸಿದ್ಧಾಂತಕ್ಕೇ
ಸಿದ್ಧಾಂತ-ಗಳ
ಸಿದ್ಧಾಂತ-ಗಳನ್ನು
ಸಿದ್ಧಾಂತ-ಗಳ-ನ್ನೆಲ್ಲಾ
ಸಿದ್ಧಾಂತ-ಗಳಲ್ಲಿ
ಸಿದ್ಧಾಂತ-ಗಳಿವೆ
ಸಿದ್ಧಾಂತ-ಗಳು
ಸಿದ್ಧಾಂತ-ಗಳೂ
ಸಿದ್ಧಾಂತ-ಗಳೆಲ್ಲಿ
ಸಿದ್ಧಾಂತ-ಗೊಂಡಿರು-ವುದು
ಸಿದ್ಧಾಂತ-ಗೊಳಿ-ಸಲು
ಸಿದ್ಧಾಂತದ
ಸಿದ್ಧಾಂತ-ದಲ್ಲಿ
ಸಿದ್ಧಾಂತ-ಪಡಿ-ಸಿ-ದಂತಾ-ಗು-ತ್ತದೆ
ಸಿದ್ಧಾಂತರ
ಸಿದ್ಧಾಂತ-ರ-ಗಳಿ-ರು-ವಂತೆ
ಸಿದ್ಧಾಂತ-ವನ್ನು
ಸಿದ್ಧಾಂತ-ವನ್ನೇ
ಸಿದ್ಧಾಂತ-ವಲ್ಲ
ಸಿದ್ಧಾಂತ-ವಾಗಲಿ
ಸಿದ್ಧಾಂತ-ವಾಗಿ-ರುವ
ಸಿದ್ಧಾಂತ-ವಾಗು-ತ್ತದೆ
ಸಿದ್ಧಾಂತ-ವಿದೆ
ಸಿದ್ಧಾಂತ-ವಿದೆಯೋ
ಸಿದ್ಧಾಂತವು
ಸಿದ್ಧಾಂತವೂ
ಸಿದ್ಧಾಂತವೇ
ಸಿದ್ಧಾಂತಿ-ಗಳು
ಸಿದ್ಧಿಗೆ
ಸಿದ್ಧಿ-ಸದು
ಸಿದ್ಧಿ-ಸಿತು
ಸಿದ್ಧಿಸಿ-ತೆಂದರೆ
ಸಿದ್ಧಿಸಿ-ದಂತೆ
ಸಿದ್ಧಿಸಿ-ದರೂ
ಸಿದ್ಧಿ-ಸಿದೆ
ಸಿದ್ಧಿ-ಸುತ್ತದೆ
ಸಿದ್ಧಿ-ಸುವ
ಸಿದ್ಧಿಸು-ವುದು
ಸಿದ್ಧ್ದಾಂತ
ಸಿದ್ಧ್ದಾಂತ-ಗಳಿಗೂ
ಸಿದ್ಧ್ದಾಂತದ
ಸಿದ್ಧ್ದಾಂತ-ವನ್ನು
ಸಿಪಾಯಿ
ಸಿಯಾ-ಲ್ಕೋಟೆ-ಯಲ್ಲಿ
ಸಿಲುಕಿ
ಸಿಲುಕು-ವರು
ಸಿಲೋ-ನಿನ
ಸಿಲೋನಿ-ನಲ್ಲಿ
ಸಿಹಿ
ಸೀತೆ
ಸೀತೆಗೆ
ಸೀತೆಯ
ಸೀತೆ-ಯಂತಹ-ವರು
ಸೀತೆಯು
ಸೀದಾ
ಸೀಮಿತ
ಸೀಮಿತ-ಗೊಳಿ-ಸಿದ್ದರೆ
ಸೀಮಿತ-ವಾಗಿಲ್ಲ
ಸೀಳಿ
ಸೀಳುತ್ತಿರು-ವಾಗ
ಸೀಳು-ವಷ್ಟು
ಸೀಸರ್
ಸೀಸ-ವನ್ನು
ಸುಂದರ
ಸುಂದರ-ವಾಗಿ
ಸುಂದರ-ವಾಗಿದೆ
ಸುಂದರ-ವಾಗಿ-ದ್ದರೂ
ಸುಂದರ-ವಾಗಿ-ದ್ದವು
ಸುಂದರ-ವಾಗಿ-ರ-ಬಲ್ಲದು
ಸುಂದರ-ವಾಗಿ-ರುವ
ಸುಂದರ-ವಾಗಿವೆ
ಸುಂದರ-ವಾದ
ಸುಂದರ-ವಾದು-ದರ
ಸುಂದ-ರವೂ
ಸುಂದರಿ
ಸುಂದರೀಂ
ಸುಂದರೇ-ಶ್ವ-ರನ
ಸುಕೃತ-ದಿಂದ
ಸುಖ
ಸುಖ-ಕ್ಕಾಗಿ
ಸುಖಕ್ಕೆ
ಸುಖ-ಗಳನ್ನು
ಸುಖ-ತ್ಯಾಗ-ವನ್ನು
ಸುಖದ
ಸುಖ-ದಲ್ಲಿ
ಸುಖ-ದಿಂದ
ಸುಖ-ದುಃಖ
ಸುಖ-ದುಃಖ-ಗಳನ್ನೂ
ಸುಖ-ಪಡತಕ್ಕದ್ದು
ಸುಖ-ಪಡ-ಲಾ-ರದು
ಸುಖ-ಪಡುತ್ತೇವೆ
ಸುಖ-ಭೋಗ-ಗಳನ್ನು
ಸುಖ-ಭೋಗ-ಗಳಿಗೆ
ಸುಖ-ಭೋಗ-ಗಳಿವೆ
ಸುಖ-ಭೋಗ-ಗಳು
ಸುಖ-ಭೋ-ಗದ
ಸುಖ-ಮಯ-ವಾದ
ಸುಖ-ಮಯ-ವಾದುದು
ಸುಖ-ವನ್ನು
ಸುಖ-ವಾಗಿ
ಸುಖ-ವಿಲ್ಲ
ಸುಖ-ವೆಲ್ಲ
ಸುಖವೇ
ಸುಖ-ಸಂತೃಪ್ತಿ-ಗಾಗಿ
ಸುಖಾ-ಗ-ಮ-ನಕ್ಕೆ
ಸುಖಾ-ನು-ಭವ
ಸುಖಿ-ಗಳ-ನ್ನಾಗಿ
ಸುಖಿ-ಯಾಗಿ-ರ-ಬಹು-ದೆಂಬು-ದನ್ನು
ಸುಗಮ
ಸುಗುಣ-ಈ-ಶ್ವರ-ನ-ಲ್ಲ-ವೆಂಬು-ದನ್ನು
ಸುಗುಣ-ಗಳಿ-ರು-ತ್ತವೆ
ಸುಗುಣ-ವಿರ-ಬಹುದು
ಸುಡ-ಲಾ-ರದು
ಸುಡು-ವರು
ಸುಡುವ-ವ-ನಿಗೆ
ಸುತ
ಸುತರು
ಸುತ್ತ
ಸುತ್ತಣ
ಸುತ್ತ-ಮುತ್ತಣ
ಸುತ್ತ-ಮುತ್ತಲಿ-ರುವ
ಸುತ್ತ-ಮು-ತ್ತಲು
ಸುತ್ತ-ಲಿ-ರುವ
ಸುತ್ತಲೂ
ಸುತ್ತ-ಲ್ಪಟ್ಟ
ಸುತ್ತಾ-ಡಿ-ದಂತೆಲ್ಲಾ
ಸುತ್ತಿಗೆಯ
ಸುತ್ತಿದ್ದನು
ಸುತ್ತಿರು-ವೆವು
ಸುತ್ತು-ತ್ತಿರು-ವುದು
ಸುತ್ತುವ
ಸುತ್ತು-ವ-ರಿ-ದಿ-ದ್ದರು
ಸುತ್ತು-ವ-ವರು
ಸುದಿನ
ಸುದಿನ-ವದು
ಸುದೀರ್ಘ
ಸುದೀರ್ಘ-ರಾತ್ರಿ
ಸುದೂರ
ಸುದ್ದಿ
ಸುದ್ದಿ-ಯನ್ನು
ಸುಧಾ-ರಕ
ಸುಧಾರ-ಕರ
ಸುಧಾ-ರಕ-ರಲ್ಲಿ
ಸುಧಾ-ರಕ-ರಾದ
ಸುಧಾ-ರಕ-ರಿಗೂ
ಸುಧಾ-ರಕ-ರಿಗೆ
ಸುಧಾ-ರಕ-ರಿದ್ದಾರೋ
ಸುಧಾರ-ಕರು
ಸುಧಾ-ರಕರೇ
ಸುಧಾರಣಾ
ಸುಧಾರಣಾ-ಲಯ
ಸುಧಾ-ರಣೆ
ಸುಧಾ-ರಣೆ-ಗಳ
ಸುಧಾ-ರಣೆ-ಗಳೆಲ್ಲ
ಸುಧಾ-ರಣೆಗೆ
ಸುಧಾ-ರಣೆ-ಗೊಳಿ-ಸಲು
ಸುಧಾ-ರಣೆಯ
ಸುಧಾ-ರಣೆ-ಯನ್ನು
ಸುಧಾ-ರಣೆ-ಯಲ್ಲಿ
ಸುಧಾ-ರಣೆ-ಯಿಂದ
ಸುಧಾ-ರಣೆಯೂ
ಸುಧಾ-ರಣೆಯೇ
ಸುಧಾ-ರಣೆ-ಯೊಂದೇ
ಸುಧಾರಿಸ-ಬೇಕು
ಸುಧಾರಿಸ-ಬೇಕೆಂದಿ-ರುವ
ಸುಧಾ-ರಿ-ಸಲು
ಸುಧಾರಿ-ಸಿದ
ಸುಧಾರಿಸುವು-ದಾ-ಗಿದೆ
ಸುಧಾರಿಸು-ವುದು
ಸುಪರ್ಣಾ
ಸುಪೂಜ್ಯ
ಸುಪ್ತ
ಸುಪ್ತ-ಚಿತ್ತವು
ಸುಪ್ತ-ಚೇ-ತನ-ದಲ್ಲಿ
ಸುಪ್ತ-ವಾಗಿ-ರಲಿ
ಸುಪ್ತ-ವಾಗಿ-ರುವ
ಸುಪ್ತ-ವಾಗಿ-ರು-ವುವು
ಸುಪ್ತ-ವಾಗಿವೆ
ಸುಪ್ತ-ವಾಗು-ವುದು
ಸುಪ್ತಾವಸ್ಥೆ-ಯಲ್ಲಿ-ರ-ಬಹುದು
ಸುಬ್ರಹ್ಮಣ್ಯ
ಸುಭದ್ರ
ಸುಭದ್ರ-ವಾಗಿ
ಸುಭದ್ರ-ವಾಗಿದೆ
ಸುಮಾತ್ರ
ಸುಮ್ಮನಿದ್ದರೂ
ಸುಮ್ಮನಿ-ರಲಿ
ಸುಮ್ಮನಿರಿ
ಸುಮ್ಮನೆ
ಸುರಕ್ಷಿತ
ಸುರಕ್ಷಿತ-ವಾಗಿ
ಸುರಕ್ಷಿತ-ವಾಗಿ-ಡು-ವುದು
ಸುರಕ್ಷಿತ-ವಾಗಿ-ತ್ತು
ಸುರಕ್ಷಿತ-ವಾಗಿ-ದ್ದರೆ
ಸುರಕ್ಷಿತ-ವಾಗಿ-ರುವ
ಸುರಕ್ಷಿತ-ವಾಗಿ-ರು-ವುವು
ಸುರಕ್ಷಿತ-ವಾಗಿವೆ
ಸುರಕ್ಷಿತ-ವಾಗಿ-ವೆಯೋ
ಸುರತ-ವರ್ಧನಂ
ಸುರಿ-ಮಳೆ
ಸುರಿ-ಯಿರಿ
ಸುರಿಸಿ-ರು-ವೆನು
ಸುರಿ-ಸುವ
ಸುರಿ-ಸು-ವರು
ಸುರುಳಿಬಿಚ್ಚುವ
ಸುಲಭ
ಸುಲಭ-ಗ್ರಾಹ್ಯವೂ
ಸುಲಭ-ವಲ್ಲ
ಸುಲಭ-ವಾಗಿ
ಸುಲಭ-ವಾಗಿಯೂ
ಸುಲಭ-ವಾಗು-ತ್ತದೆ
ಸುಲಭ-ವಾಗು-ವುದು
ಸುಲಭ-ವಾದ
ಸುಲಭವೂ
ಸುಲಭವೆ
ಸುಲಿಗೆ
ಸುಲಿಗೆ-ಮಾಡು-ತ್ತಿದ್ದ
ಸುಲಿದು
ಸುಲಿ-ಯು-ತ್ತಿದ್ದರು
ಸುಲ್ಲಾ-ನನ್ನು
ಸುಳಿ-ಗಳ
ಸುಳಿ-ವನ್ನು
ಸುಳಿವೇ
ಸುಳ್ಳನ್ನು
ಸುಳ್ಳ-ಲ್ಲದೆ
ಸುಳ್ಳಿನ
ಸುಳ್ಳು
ಸುವರು
ಸುವರ್ಣ
ಸುವೇದೇತಿ
ಸುವ್ಯವ-ಸ್ಥೆಯ
ಸುಷ್ಟು
ಸುಸಂಸ್ಕೃತ
ಸುಸಂಸ್ಕೃತ-ರಿ-ಗಿಂತ
ಸುಸಂಸ್ಕೃತರು
ಸುಸಂಸ್ಕೃತವೂ
ಸುಸಂಸ್ಕೃತಿ
ಸುಸಜ್ಜಿತ
ಸುಸ್ತಾ-ಯಿತು
ಸೂಕ್ತ
ಸೂಕ್ತ-ವನ್ನು
ಸೂಕ್ತ-ವಾಗಿ
ಸೂಕ್ತ-ವಾದ
ಸೂಕ್ತವು
ಸೂಕ್ಮ-ವಾದ
ಸೂಕ್ಷ್ಮ
ಸೂಕ್ಷ್ಮಕ್ಕೆ
ಸೂಕ್ಷ್ಮ-ಗಳಾಗಿ
ಸೂಕ್ಷ್ಮ-ತ-ರಕ್ಕೆ
ಸೂಕ್ಷ್ಮ-ತರ-ವಾದ
ಸೂಕ್ಷ್ಮ-ತರ-ವಾದಂತೆಲ್ಲಾ
ಸೂಕ್ಷ್ಮ-ತೆ-ಗಳಿಂದ
ಸೂಕ್ಷ್ಮ-ದಿಂದ
ಸೂಕ್ಷ್ಮ-ದೇಹ-ವೊಂದಿದೆ
ಸೂಕ್ಷ್ಮ-ಮ-ತಿಯು
ಸೂಕ್ಷ್ಮ-ರಹಸ್ಯ-ವನ್ನು
ಸೂಕ್ಷ್ಮ-ವಲ್ಲ
ಸೂಕ್ಷ್ಮ-ವಸ್ತು
ಸೂಕ್ಷ್ಮ-ವಸ್ತು-ವಿಗೆ
ಸೂಕ್ಷ್ಮ-ವಾಗಿ
ಸೂಕ್ಷ್ಮ-ವಾಗುತ್ತಾ
ಸೂಕ್ಷ್ಮ-ವಾದು-ದನ್ನು
ಸೂಕ್ಷ್ಮವೂ
ಸೂಕ್ಷ್ಮ-ಶರೀರ-ವೆಂದು
ಸೂಕ್ಷ್ಮ-ಸ್ಥಿತಿ-ಯನ್ನು
ಸೂಕ್ಷ್ಮಾತಿ
ಸೂಕ್ಷ್ಮೇಂದ್ರಿಯ-ಗಳ
ಸೂಕ್ಷ್ಮೇಂದ್ರಿಯ-ಗಳಿಗೆ
ಸೂಚಕ
ಸೂಚಕ-ವಾಗಿದೆ
ಸೂಚ-ನೆ-ಗಳು
ಸೂಚ-ನೆ-ಯನ್ನು
ಸೂಚಿಸ-ಬಲ್ಲ
ಸೂಚಿಸ-ಬೇಕಾ-ಯಿತು
ಸೂಚಿಸಿ
ಸೂಚಿಸಿ-ದಂತೆ
ಸೂಚಿಸಿ-ದರು
ಸೂಚಿ-ಸಿದ್ದು
ಸೂಚಿಸಿ-ರು-ವು-ದೆಲ್ಲಾ
ಸೂಚಿ-ಸುತ್ತವೆ
ಸೂಚಿಸುತ್ತಿತ್ತು
ಸೂಚಿಸು-ತ್ತೇನೆ
ಸೂಚಿ-ಸು-ವುದೇ
ಸೂಜಿಭೇದ್ಯ
ಸೂಜಿ-ಯಿಂದ
ಸೂತ್ರ
ಸೂತ್ರ-ಗಳನ್ನು
ಸೂತ್ರ-ಗಳಿವೆ
ಸೂತ್ರ-ಗಳು
ಸೂತ್ರ-ದಲ್ಲಿ
ಸೂತ್ರ-ವನ್ನು
ಸೂತ್ರ-ವೃತ್ತಿ-ಯನ್ನು
ಸೂತ್ರಾಕ್ಷ
ಸೂತ್ರೇ
ಸೂಫಿ
ಸೂರ್ಯ
ಸೂರ್ಯ-ಕೇಂದ್ರ
ಸೂರ್ಯ-ಚಂದ್ರಾದಿ-ಗಳು
ಸೂರ್ಯ-ನನ್ನು
ಸೂರ್ಯ-ನ-ಲ್ಲಿದ್ದ
ಸೂರ್ಯನು
ಸೂರ್ಯ-ಲೋಕಕ್ಕೆ
ಸೂರ್ಯೋ
ಸೃಜಾಮ್ಯಹ-ಮ್
ಸೃಜಿಸ-ಬಹುದು
ಸೃತಿ-ಭಗ-ವಂತನ
ಸೃಷ್ಟಿ
ಸೃಷ್ಟಿ-ಕರ್ತ
ಸೃಷ್ಟಿ-ಕರ್ತನ
ಸೃಷ್ಟಿ-ಕರ್ತ-ನನ್ನು
ಸೃಷ್ಟಿ-ಕರ್ತ-ನಾದ
ಸೃಷ್ಟಿ-ಕರ್ತನು
ಸೃಷ್ಟಿ-ಕರ್ತ-ನೆಂದು
ಸೃಷ್ಟಿ-ಕರ್ತ-ರಾದ
ಸೃಷ್ಟಿ-ಗಿಂತ
ಸೃಷ್ಟಿಗೆ
ಸೃಷ್ಟಿ-ಗೆಲ್ಲಾ
ಸೃಷ್ಟಿ-ಚಕ್ರ
ಸೃಷ್ಟಿ-ಚಕ್ರವು
ಸೃಷ್ಟಿ-ತತ್ತ್ವ
ಸೃಷ್ಟಿ-ಮಾಡು-ತ್ತಾ-ನಂತೆ
ಸೃಷ್ಟಿಯ
ಸೃಷ್ಟಿ-ಯನ್ನು
ಸೃಷ್ಟಿ-ಯಲ್ಲ
ಸೃಷ್ಟಿ-ಯಲ್ಲಿ
ಸೃಷ್ಟಿ-ಯ-ಲ್ಲಿ-ರುವ
ಸೃಷ್ಟಿ-ಯ-ಲ್ಲೆಲ್ಲಾ
ಸೃಷ್ಟಿ-ಯಾಗ-ಲಿ-ರುವ
ಸೃಷ್ಟಿ-ಯಾಗಿಲ್ಲ
ಸೃಷ್ಟಿ-ಯಾಗು-ತ್ತದೆ
ಸೃಷ್ಟಿ-ಯಾಗುವ
ಸೃಷ್ಟಿ-ಯಾ-ದರೆ
ಸೃಷ್ಟಿ-ಯಾ-ದವು
ಸೃಷ್ಟಿಯು
ಸೃಷ್ಟಿ-ಯೆಲ್ಲಾ
ಸೃಷ್ಟಿಯೇ
ಸೃಷ್ಟಿ-ಯೊಂದು
ಸೃಷ್ಟಿ-ಸ-ಬಹುದು
ಸೃಷ್ಟಿ-ಸ-ಬೇ-ಕಾದರೆ
ಸೃಷ್ಟಿ-ಸ-ಲಾ-ರರು
ಸೃಷ್ಟಿ-ಸ-ಲಿಲ್ಲ
ಸೃಷ್ಟಿ-ಸಲು
ಸೃಷ್ಟಿಸಿ
ಸೃಷ್ಟಿ-ಸಿ-ಕೊಳ್ಳ-ಬೇಕಾಗಿ-ದೆಯೋ
ಸೃಷ್ಟಿ-ಸಿದ
ಸೃಷ್ಟಿ-ಸಿ-ದನು
ಸೃಷ್ಟಿ-ಸಿ-ದನೋ
ಸೃಷ್ಟಿ-ಸಿ-ದರು
ಸೃಷ್ಟಿ-ಸಿ-ದರೆ
ಸೃಷ್ಟಿ-ಸಿ-ದ-ವ-ನೆಂದು
ಸೃಷ್ಟಿ-ಸಿ-ದ-ವರು
ಸೃಷ್ಟಿ-ಸಿ-ದ-ವ-ರೆಂದು
ಸೃಷ್ಟಿ-ಸಿದೆ
ಸೃಷ್ಟಿ-ಸಿ-ದ್ದಾರೆ
ಸೃಷ್ಟಿ-ಸಿ-ರ-ಬೇಕು
ಸೃಷ್ಟಿ-ಸಿ-ರು-ವಂತೆ
ಸೃಷ್ಟಿ-ಸಿ-ರು-ವನು
ಸೃಷ್ಟಿ-ಸಿ-ರು-ವರು
ಸೃಷ್ಟಿ-ಸುತ್ತ
ಸೃಷ್ಟಿ-ಸುತ್ತಿ-ರು-ವನು
ಸೃಷ್ಟಿ-ಸುವ
ಸೃಷ್ಟಿ-ಸು-ವನು
ಸೃಷ್ಟಿ-ಸು-ವು-ದಕ್ಕೆ
ಸೃಷ್ಟಿ-ಸು-ವು-ದಿಲ್ಲ
ಸೃಷ್ಟಿ-ಸ್ಥಿತಿ-ಪ್ರಳಯ-ಗಳಿಗೆ
ಸೆಮಿಟಿ-ಕ್
ಸೆಮೆಟಿ-ಕ್
ಸೆರೆ-ಮ-ನೆಗೆ
ಸೆರೆ-ಸಿಕ್ಕಿ-ರುವ
ಸೆರೆ-ಹಿ-ಡಿಯುವು-ದಾಗಿ-ತ್ತು
ಸೆಳೆ-ದಿ-ದ್ದೀರಿ
ಸೆಳೆದು-ಕೊಳ್ಳು-ವಂತಹ
ಸೆಳೆದು-ದ-ಕ್ಕಾಗಿ
ಸೆಳೆಯು-ತ್ತಾರೆ
ಸೆಳೆಯು-ತ್ತೇನೆ
ಸೆಳೆಯು-ತ್ತೇ-ನೆ-ಅದು
ಸೆಳೆಯುವ
ಸೇತು-ಪತಿ
ಸೇತು-ಬಂಧ
ಸೇತು-ವೆಗೆ
ಸೇತು-ವೆ-ಯನ್ನು
ಸೇನಾ
ಸೇನಾ-ಚಲನೆ
ಸೇನಾ-ನಾ-ಯಕ-ನಾದ
ಸೇನಾ-ಶಕ್ತಿ
ಸೇನಾ-ಸಮೂಹ
ಸೇನಾ-ಸಮೂಹದ
ಸೇನಾ-ಸಮೂಹ-ದಿಂದ
ಸೇನೆ
ಸೇನೆ-ಯನ್ನು
ಸೇಬಿನ
ಸೇರದ
ಸೇರದೆ
ಸೇರ-ಬಹುದು
ಸೇರಬೇಕಾಗು-ವುದು
ಸೇರಲಿ
ಸೇರಲು
ಸೇರಿ
ಸೇರಿತು
ಸೇರಿ-ತ್ತು
ಸೇರಿದ
ಸೇರಿ-ದನು
ಸೇರಿ-ದರೆ
ಸೇರಿ-ದ-ವ-ನಲ್ಲ-ವೆಂದು
ಸೇರಿ-ದ-ವನಾ-ಗಲಿ
ಸೇರಿ-ದ-ವನಾ-ಗಿರು-ತ್ತಾನೆ
ಸೇರಿ-ದ-ವ-ನೆಂಬುದು
ಸೇರಿ-ದ-ವರ
ಸೇರಿ-ದ-ವ-ರಂತೆ
ಸೇರಿ-ದ-ವ-ರಲ್ಲ
ಸೇರಿ-ದ-ವರು
ಸೇರಿ-ದ-ವ-ರೆಂದರೆ
ಸೇರಿ-ದ-ವ-ರೆಂದು
ಸೇರಿ-ದ-ವ-ರೆಂದೂ
ಸೇರಿ-ದ-ವ-ರೆಂಬು-ದನ್ನು
ಸೇರಿ-ದ-ವರೆಂಬು-ದನ್ನೂ
ಸೇರಿ-ದ-ವರೊ
ಸೇರಿ-ದ-ವರೋ
ಸೇರಿ-ದ-ವಳು
ಸೇರಿ-ದವು
ಸೇರಿ-ದುದು
ಸೇರಿದೆ
ಸೇರಿ-ದ್ದರೂ
ಸೇರಿ-ದ್ದವು
ಸೇರಿದ್ದು
ಸೇರಿ-ದ್ದೇವೆ
ಸೇರಿ-ರ-ಬೇಕು
ಸೇರಿ-ರಲಿ
ಸೇರಿ-ರುತ್ತಾನೆ
ಸೇರಿ-ರು-ವರು
ಸೇರಿ-ರು-ವೆವು
ಸೇರಿಲ್ಲ
ಸೇರಿವೆ
ಸೇರಿ-ಸದೆ
ಸೇರಿ-ಸ-ಬಹುದು
ಸೇರಿ-ಸಲು
ಸೇರಿ-ಸಲ್ಪಟ್ಟಿದೆ
ಸೇರಿಸಿ
ಸೇರಿ-ಸಿ-ಕೊಳ್ಳಲು
ಸೇರಿ-ಸಿ-ಕೊಳ್ಳು-ತ್ತಿತ್ತು
ಸೇರಿ-ಸಿ-ಕೊಳ್ಳು-ತ್ತೇವೆ
ಸೇರಿ-ಸಿ-ಕೊಳ್ಳು-ವನು
ಸೇರಿ-ಸಿತೋ
ಸೇರಿ-ಸಿ-ದರು
ಸೇರಿ-ಸಿ-ದರೆ
ಸೇರಿ-ಸಿ-ರು-ವರು
ಸೇರಿ-ಸಿವೆ
ಸೇರಿ-ಸುತ್ತಿ-ರು-ವರು
ಸೇರಿ-ಹೋ-ಗಿವೆ
ಸೇರು-ತ್ತವೆ
ಸೇರುತ್ತಾನೆ
ಸೇರುತ್ತಾರೆ
ಸೇರುವ
ಸೇರು-ವಂತೆ
ಸೇರು-ವನು
ಸೇರುವ-ವರೆಗೂ
ಸೇರುವ-ವ-ರೆಗೆ
ಸೇರು-ವಿರಿ
ಸೇರು-ವು-ದಕ್ಕೆ
ಸೇರು-ವು-ದಿಲ್ಲ
ಸೇರು-ವುದು
ಸೇರು-ವುವು
ಸೇರು-ವೆಯಾ
ಸೇವಕನಾಗ-ಬೇಕೆಂಬು-ದನ್ನು
ಸೇವಕ-ನಾಗಿ
ಸೇವಕ-ನಾಗಿ-ರು-ವುದೇ
ಸೇವಕ-ನೆಂದೂ
ಸೇವಕ-ರಲ್ಲಿ
ಸೇವಕರಿ-ರು-ವರು
ಸೇವ-ಕರು
ಸೇವಕ-ರೆಂದು
ಸೇವಾ
ಸೇವಿತ
ಸೇವಿಸ-ಬಾ-ರದು
ಸೇವಿಸಿ
ಸೇವಿಸಿ-ದರೆ
ಸೇವೆ
ಸೇವೆ-ಗಾಗಿ
ಸೇವೆಗೆ
ಸೇವೆ-ಮಾಡಿ
ಸೇವೆ-ಯನ್ನು
ಸೇವೆ-ಯನ್ನೆಲ್ಲ
ಸೈತಾನ
ಸೈತಾನ-ನಂತೆ
ಸೈತಾನನು
ಸೈನಿಕ
ಸೈನ್ಯ-ಗಳನ್ನು
ಸೈನ್ಯ-ದಿಂದ
ಸೈನ್ಯ-ಶಕ್ತಿ-ಯಲ್ಲ
ಸೈನ್ಯ-ಶಕ್ತಿ-ಯಾಗಿ-ರ-ಬಹುದು
ಸೊಂಕಿದ
ಸೊಂಟ
ಸೊಗ-ಸಾಗಿ
ಸೊಗ-ಸಾಗಿದೆ
ಸೊಗಸಾದ
ಸೊನ್ನೆಯ
ಸೊಳ್ಳೆ
ಸೊಳ್ಳೆ-ಯೊಂದು
ಸೊಸೈಟಿಯ
ಸೋಂಕು-ವು-ದಿಲ್ಲ
ಸೋಜಿ-ಗದ
ಸೋತಂತೆ
ಸೋತರು
ಸೋತರೆ
ಸೋತು
ಸೋದ-ರರೆ
ಸೋದ-ರರೇ
ಸೋದರಿ
ಸೋಪಾನ
ಸೋಪಾನ-ಗಳನ್ನು
ಸೋಪಾನ-ಗಳ-ನ್ನೇರಿ
ಸೋಪಾನ-ಗಳು
ಸೋಮನಾಥ-ದಂತಹ
ಸೋಮನಾಥಾ
ಸೋಮರಸ-ವನ್ನು
ಸೋಮಾರಿ
ಸೋಮಾ-ರಿ-ಗಳಾಗ-ಬೇಡಿ
ಸೋಮಾ-ರಿ-ಗಳಾಗಿ-ರ-ಬೇಕು
ಸೋಮಾ-ರಿ-ಗಳು
ಸೋರಲು
ಸೋರುತ್ತಿರ-ಬಹುದು
ಸೋಲಿಸಿ
ಸೋಲಿ-ಸು-ತ್ತಿದ್ದನು
ಸೋಲುವ
ಸೋಷಿ-ಯಲಿ-ಸಮ್
ಸೋಹಂ
ಸೌಂದರ್ಯತೃಷೆ-ಯನ್ನೂ
ಸೌಂದರ್ಯ-ವನ್ನು
ಸೌಂದರ್ಯ-ವಿದೆ
ಸೌಂದರ್ಯವಿರು-ವು-ದೆಂದು
ಸೌಕರ್ಯ-ಗಳನ್ನು
ಸೌಕರ್ಯ-ಗಳಿ-ದ್ದುವು
ಸೌಕರ್ಯ-ಗಳು
ಸೌಧ
ಸೌಧವು
ಸೌಧವೇ
ಸೌಭಾಗ್ಯ
ಸೌಭಾಗ್ಯವು
ಸೌಭಾವವೂ
ಸೌರ-ರಾಗಲಿ
ಸೌರರೂ
ಸೌಲಭ್ಯ-ಗಳನ್ನು
ಸೌಹಾರ್ದ
ಸೌಹಾರ್ದಕ್ಕೆ
ಸೌಹಾರ್ದ-ಗಳು
ಸೌಹಾರ್ದ-ದಿಂದ
ಸೌಹಾರ್ದ-ಪೂರ್ಣ
ಸೌಹಾರ್ದ-ಭಾವ-ವನ್ನು
ಸ್ಟಾರ್
ಸ್ಟೆಂಟ್
ಸ್ತನ್ಯ-ದೊಡನೆ
ಸ್ತಬ್ಧರಾಗು-ವಿರಿ
ಸ್ತಬ್ಧ-ವಾಗಿ
ಸ್ತಬ್ಧ-ವಾರಿ
ಸ್ತರ-ಗಳನ್ನು
ಸ್ತಿಮಿತ-ವಿಲ್ಲದ
ಸ್ತು-ತಿಸಿ-ರುವ
ಸ್ತು-ತಿ-ಸು-ವರು
ಸ್ತು-ವಂತು
ಸ್ತೋತ್ರ-ಗಳು
ಸ್ತೋತ್ರ-ದಲ್ಲಿ
ಸ್ತೋಮದ
ಸ್ತೋಮವು
ಸ್ತ್ರೀ
ಸ್ತ್ರೀಪುರಷ-ರಿಗೆ
ಸ್ತ್ರೀಪುರುಷ
ಸ್ತ್ರೀಪುರು-ಷ-ಮಕ್ಕಳ
ಸ್ತ್ರೀಪುರು-ಷರ
ಸ್ತ್ರೀಪುರು-ಷ-ರಾಗಲಿ
ಸ್ತ್ರೀಪುರು-ಷರು
ಸ್ತ್ರೀಯ-ನ್ನಾ-ಗಲೀ
ಸ್ತ್ರೀಯನ್ನೂ
ಸ್ತ್ರೀಯರ
ಸ್ತ್ರೀಯ-ರಂತೆ
ಸ್ತ್ರೀಯ-ರನ್ನು
ಸ್ತ್ರೀಯ-ರಿಗೆ
ಸ್ತ್ರೀಯರು
ಸ್ತ್ರೀಯರೂ
ಸ್ತ್ರೀಯಾಗಲೀ
ಸ್ತ್ರೀಯು
ಸ್ತ್ರೀಯೆ
ಸ್ತ್ರೀಯೋ
ಸ್ತ್ರೀರತ್ನಂ
ಸ್ತ್ರೀರತ್ನವು
ಸ್ತ್ರೀಲಂಪಟ
ಸ್ತ್ರೀಸಮಸ್ಯೆ-ಯನ್ನು
ಸ್ಥಳ
ಸ್ಥಳಕ್ಕೆ
ಸ್ಥಳ-ಗಳಲ್ಲ
ಸ್ಥಳ-ಗಳ-ಲ್ಲೆಲ್ಲಾ
ಸ್ಥಳ-ದಲ್ಲಿ
ಸ್ಥಳ-ದಲ್ಲಿಯೇ
ಸ್ಥಳ-ದಿಂದ
ಸ್ಥಳ-ವನ್ನು
ಸ್ಥಳ-ವಿದು
ಸ್ಥಳ-ವಿದೆ
ಸ್ಥಳ-ವಿಲ್ಲ
ಸ್ಥಳವೇ
ಸ್ಥಳ-ವೊಂದಿ-ದ್ದರೆ
ಸ್ಥಾಣುರಚಲೋಽಯಂ
ಸ್ಥಾಣುವೂ
ಸ್ಥಾನ
ಸ್ಥಾನಕ್ಕೆ
ಸ್ಥಾನ-ಗಳ
ಸ್ಥಾನ-ಗಳನ್ನು
ಸ್ಥಾನ-ದಲ್ಲಿ
ಸ್ಥಾನ-ದಲ್ಲಿದೆ
ಸ್ಥಾನ-ವನ್ನು
ಸ್ಥಾನ-ವಾದ
ಸ್ಥಾನ-ವಿದೆ
ಸ್ಥಾನ-ವೇನು
ಸ್ಥಾಪಕ-ನಾದನು
ಸ್ಥಾಪಕರ
ಸ್ಥಾಪಕರು
ಸ್ಥಾಪ-ನೆಗೆ
ಸ್ಥಾಪಿ-ತ-ಗೊಂಡುದು
ಸ್ಥಾಪಿ-ತ-ವಾದರೆ
ಸ್ಥಾಪಿಸ
ಸ್ಥಾಪಿ-ಸ-ಬಲ್ಲ-ವ-ರಾಗಿ-ದ್ದರೋ
ಸ್ಥಾಪಿ-ಸ-ಬೇಕಾಗಿದೆ
ಸ್ಥಾಪಿ-ಸ-ಬೇಕೆಂದು
ಸ್ಥಾಪಿ-ಸಲು
ಸ್ಥಾಪಿ-ಸ-ಲ್ಪಟ್ಟಿವೆ
ಸ್ಥಾಪಿ-ಸ-ಲ್ಪ-ಡುವ
ಸ್ಥಾಪಿ-ಸಿದರು
ಸ್ಥಾಪಿ-ಸಿದರೆ
ಸ್ಥಾಪಿ-ಸಿದ-ರೇನು
ಸ್ಥಾಪಿ-ಸಿದ-ವರೂ
ಸ್ಥಾಪಿ-ಸಿದವು
ಸ್ಥಾಪಿ-ಸಿ-ದ್ದೀರಿ
ಸ್ಥಾಪಿ-ಸುತ್ತಾನೆ
ಸ್ಥಾಪಿ-ಸು-ತ್ತಿದ್ದ
ಸ್ಥಾಪಿ-ಸುವ
ಸ್ಥಾಪಿ-ಸುವ-ವ-ರೆಗೆ
ಸ್ಥಾಪಿ-ಸು-ವು-ದಕ್ಕೆ
ಸ್ಥಾಪಿ-ಸು-ವುದು
ಸ್ಥಾಪಿ-ಸೋಣ
ಸ್ಥಿತಂ
ಸ್ಥಿತಾಃ
ಸ್ಥಿತಿ
ಸ್ಥಿತಿ-ಗತಿ-ಗಳ
ಸ್ಥಿತಿ-ಗತಿ-ಗಳನ್ನೂ
ಸ್ಥಿತಿ-ಗತಿ-ಗಳು
ಸ್ಥಿತಿ-ಗಳನ್ನು
ಸ್ಥಿತಿ-ಗಿಂತ
ಸ್ಥಿತಿಗೂ
ಸ್ಥಿತಿಗೆ
ಸ್ಥಿತಿ-ಗೇರಿ
ಸ್ಥಿತಿಯ
ಸ್ಥಿತಿ-ಯನ್ನು
ಸ್ಥಿತಿ-ಯ-ನ್ನೇ
ಸ್ಥಿತಿ-ಯಲ್ಲಿ
ಸ್ಥಿತಿ-ಯ-ಲ್ಲಿದೆ
ಸ್ಥಿತಿ-ಯ-ಲ್ಲಿ-ದ್ದರೂ
ಸ್ಥಿತಿ-ಯ-ಲ್ಲಿಯೂ
ಸ್ಥಿತಿ-ಯ-ಲ್ಲಿ-ರು-ತ್ತದೆ
ಸ್ಥಿತಿ-ಯ-ಲ್ಲಿ-ರುವ
ಸ್ಥಿತಿ-ಯ-ಲ್ಲಿ-ರುವ-ವ-ರೆಗೆ
ಸ್ಥಿತಿ-ಯಾ-ದರೆ
ಸ್ಥಿತಿ-ಯಿಂದ
ಸ್ಥಿತಿಯು
ಸ್ಥಿತಿ-ಲಯ-ಗಳಿ-ಗೆಲ್ಲಾ
ಸ್ಥಿರ
ಸ್ಥಿರ-ತೆ-ಯನ್ನು
ಸ್ಥಿರ-ವಾಗಿ
ಸ್ಥಿರ-ವಾಗಿ-ರ-ಲಾ-ರದು
ಸ್ಥಿರ-ವಾಗಿ-ರು-ವುದೋ
ಸ್ಥಿರ-ವಾದ
ಸ್ಥಿರವೂ
ಸ್ಥಿರ-ಸ್ವ-ಭಾವದ
ಸ್ಥೂಲ
ಸ್ಥೂಲ-ತೆಯ
ಸ್ಥೂಲ-ದಿಂದ
ಸ್ಥೂಲ-ಭಾಷೆ-ಯಲ್ಲಿ
ಸ್ಥೂಲ-ಭೂತ
ಸ್ಥೂಲ-ವಸ್ತು
ಸ್ಥೂಲ-ವಾದ
ಸ್ಥೂಲ-ವಾದುದು
ಸ್ಥೂಲವು
ಸ್ಥೂಲ-ಶರೀರ
ಸ್ಥೂಲಾಂಗ-ಗಳನ್ನು
ಸ್ಥೈರ್ಯ
ಸ್ಥೈರ್ಯ-ದಿಂದ
ಸ್ನಾನ
ಸ್ನಾನ-ದಿಂದ
ಸ್ನಾನ-ಮಾಡು-ವುದಾ-ಗಲಿ
ಸ್ನೇಹ
ಸ್ನೇಹ-ದಿಂದ
ಸ್ನೇಹ-ವಿರ-ಲಾ-ರದು
ಸ್ನೇಹವು
ಸ್ನೇಹಿತ
ಸ್ನೇಹಿ-ತ-ನಂತೆ
ಸ್ನೇಹಿ-ತ-ನನ್ನು
ಸ್ನೇಹಿ-ತ-ನನ್ನೂ
ಸ್ನೇಹಿ-ತ-ನಿಲ್ಲದ
ಸ್ನೇಹಿ-ತರ
ಸ್ನೇಹಿ-ತ-ರಲ್ಲಿ
ಸ್ನೇಹಿ-ತ-ರಾಗು-ವಿರಿ
ಸ್ನೇಹಿ-ತ-ರಾದ
ಸ್ನೇಹಿ-ತ-ರಿಗೆ
ಸ್ನೇಹಿ-ತ-ರಿಲ್ಲ
ಸ್ನೇಹಿ-ತರು
ಸ್ನೇಹಿ-ತರೂ
ಸ್ನೇಹಿ-ತರೆ
ಸ್ನೇಹಿ-ತರೇ
ಸ್ನೇಹಿತೆ
ಸ್ಪಂದನ
ಸ್ಪಂದನ-ಗಳಿ-ಲ್ಲದೆ
ಸ್ಪಂದನ-ಗಳೂ
ಸ್ಪಂದನದ
ಸ್ಪಂದ-ನವು
ಸ್ಪಂದಿಸ
ಸ್ಪಂದಿ-ಸುತ್ತದೆ
ಸ್ಪಂದಿಸುತ್ತಿತ್ತು
ಸ್ಪಂದಿಸು-ತ್ತಿದೆಯೆ
ಸ್ಪಂದಿಸು-ವುವು
ಸ್ಪರ್ಧಿಸ-ಲಾ-ರದು
ಸ್ಪರ್ಧಿ-ಸು-ತ್ತಿದ್ದರು
ಸ್ಪರ್ಧೆ
ಸ್ಪರ್ಧೆ-ಗಳೂ
ಸ್ಪರ್ಧೆ-ಯನ್ನು
ಸ್ಪರ್ಶ
ಸ್ಪರ್ಶಿ-ಸಿ-ರು-ವನು
ಸ್ಪಲ್ಪ
ಸ್ಪಷ್ಟ
ಸ್ಪಷ್ಟ-ದೃಷ್ಟಿ-ಯುಳ್ಳ
ಸ್ಪಷ್ಟ-ಪಡಿ-ಸಿದರು
ಸ್ಪಷ್ಟ-ಪಡಿ-ಸಿಲ್ಲ
ಸ್ಪಷ್ಟ-ವಾಗಿ
ಸ್ಪಷ್ಟ-ವಾಗಿದೆ
ಸ್ಪಷ್ಟ-ವಾಗಿ-ದ್ದರೂ
ಸ್ಪಷ್ಟ-ವಾಗಿಲ್ಲ
ಸ್ಪಷ್ಟ-ವಾಗು-ತ್ತದೆ
ಸ್ಪಷ್ಟ-ವಾಗುತ್ತಾ
ಸ್ಪಷ್ಟ-ವಾಗು-ತ್ತಿದೆ
ಸ್ಪಷ್ಟ-ವಾಗು-ವುದು
ಸ್ಪಷ್ಟ-ವಾದ
ಸ್ಪಷ್ಟ-ವಾದಂತೆ
ಸ್ಪಷ್ಟವೂ
ಸ್ಪೂರ್ತಿ-ದಾಯ-ಕನೇ
ಸ್ಪೃಶ್ಯಾ-ಸ್ಪೃಶ್ಯ-ತೆಯ
ಸ್ಪೆನ್ಸರ್
ಸ್ಪೆಯಿ-ನ್
ಸ್ಫರ್ಧೆ
ಸ್ಫೂರ್ತಿ
ಸ್ಫೂರ್ತಿ-ಗಳು
ಸ್ಫೂರ್ತಿಗೂ
ಸ್ಫೂರ್ತಿ-ಯನ್ನು
ಸ್ಮರಣಾರ್ಹ-ವಾದ
ಸ್ಮರಣೆ
ಸ್ಮರಣೆಯೂ
ಸ್ಮರಿಸ-ಬೇಕು
ಸ್ಮರಿ-ಸಲಿ
ಸ್ಮರಿಸಿ
ಸ್ಮರಿಸಿ-ಕೊಳ್ಳಿ
ಸ್ಮರಿಸಿ-ಕೊಳ್ಳೋಣ
ಸ್ಮಾ-ರಕ-ವಾಗಿದೆ
ಸ್ಮಾರ್ತ-ರೆಲ್ಲಾ
ಸ್ಮೃತಿ
ಸ್ಮೃತಿ-ಕಾರ
ಸ್ಮೃತಿ-ಕಾರ-ನಾದ
ಸ್ಮೃತಿ-ಕಾರ-ರನ್ನು
ಸ್ಮೃತಿ-ಕಾರರು
ಸ್ಮೃತಿ-ಗಳಲ್ಲಿ
ಸ್ಮೃತಿ-ಗಳಲ್ಲಿಯೂ
ಸ್ಮೃತಿ-ಗಳಿಗೂ
ಸ್ಮೃತಿ-ಗಳು
ಸ್ಮೃತಿ-ಗಳೂ
ಸ್ಮೃತಿಗೂ
ಸ್ಮೃತಿಗೆ
ಸ್ಮೃತಿಯ
ಸ್ಮೃತಿ-ಯನ್ನು
ಸ್ಮೃತಿ-ಯ-ನ್ನೇ
ಸ್ಮೃತಿ-ಯಲ್ಲಿ
ಸ್ಮೃತಿಯು
ಸ್ಮೃತಿಯೇ
ಸ್ಮೃತಿ-ಸ್ತಂಭದ
ಸ್ಯಾ-ಕ್ಸನ್ರು
ಸ್ಯಾ-ದಾತ್ಮ-ತೃಪ್ತಶ್ಚ
ಸ್ರವಿಸಿ
ಸ್ರವಿ-ಸುವ
ಸ್ವಂತ
ಸ್ವಂತ-ಕೀರ್ತಿ-ಗಾ-ಗಲೀ
ಸ್ವಂತಿಕೆ
ಸ್ವಂತಿಕೆ-ಯನ್ನೆಲ್ಲ
ಸ್ವಂತಿಕೆ-ಯೆಂಬುದು
ಸ್ವಚ್ಛಂದ-ವಾಗಿ-ರ-ಬೇಕು
ಸ್ವಜನರ
ಸ್ವಜಾತಿ
ಸ್ವತಂತ್ರ
ಸ್ವತಂತ್ರನು
ಸ್ವತಂತ್ರ-ರಲ್ಲ
ಸ್ವತಂತ್ರರು
ಸ್ವತಂತ್ರ-ವಾಗಿ
ಸ್ವತಂತ್ರ-ವಾಗಿದ್ದು
ಸ್ವತಃ
ಸ್ವತಃ-ಸಿದ್ದ-ವಾದುದು
ಸ್ವತ್ತಾಗಿ
ಸ್ವತ್ತಾಗಿ-ರುವ
ಸ್ವತ್ತಾ-ಗುವ
ಸ್ವದೇಶಕ್ಕೆ
ಸ್ವದೇಶದ
ಸ್ವದೇಶ-ವಾ-ಸಿ-ಗಳೇ
ಸ್ವದೇಶಾಭಿ-ಮುಖ-ವಾಗಿ
ಸ್ವಧರ್ಮ-ವನ್ನನು-ಸ-ರಿಸಿ
ಸ್ವಧಾ
ಸ್ವಪಕ್ಷ-ಗಳ
ಸ್ವಪ್ನ-ದೇಶ-ವಿದು
ಸ್ವಪ್ರತಿಷ್ಠೆ
ಸ್ವಪ್ರತಿ-ಷ್ಠೆ-ಗಾ-ಗಲೀ
ಸ್ವಪ್ರ-ಯತ್ನ-ದಿಂದಲೂ
ಸ್ವಭಾವ
ಸ್ವಭಾವಕ್ಕೆ
ಸ್ವಭಾವ-ಗಳು
ಸ್ವಭಾವತಃ
ಸ್ವಭಾವದ
ಸ್ವಭಾವ-ದಂತೆ
ಸ್ವಭಾವ-ದಲ್ಲಿ
ಸ್ವಭಾವ-ದಲ್ಲಿಯೇ
ಸ್ವಭಾವ-ದಲ್ಲೇ
ಸ್ವಭಾವ-ದ-ವ-ರಾಗಿ-ಲ್ಲವೋ
ಸ್ವಭಾವ-ದ-ವರು
ಸ್ವಭಾವ-ದಿಂದ
ಸ್ವಭಾವ-ದಿಂದಿ-ರು-ವನು
ಸ್ವಭಾವ-ವನ್ನು
ಸ್ವಭಾವ-ವನ್ನೇ
ಸ್ವಭಾವ-ವ-ಲ್ಲ-ವೆಂದೂ
ಸ್ವಭಾವ-ವಾಗಿ
ಸ್ವಭಾವ-ವಾಗಿ-ರುವ
ಸ್ವಭಾವ-ವಾ-ಯಿತು
ಸ್ವಭಾವವೇ
ಸ್ವಭಾವ-ವೇ-ನೆಂದರೆ
ಸ್ವಭಾವ-ಸಿದ್ಧ
ಸ್ವಭಾ-ವಾ-ನು-ಸಾರ
ಸ್ವಯಂ
ಸ್ವಯಂಪ್ರಭೆ-ಯಲ್ಲಿ
ಸ್ವರ
ಸ್ವರ-ಗಳ
ಸ್ವರ-ಗಳು
ಸ್ವರ-ಗಳೆಲ್ಲ
ಸ್ವರ-ದಲ್ಲಿಯೂ
ಸ್ವರ-ಮೇಳ-ದಲ್ಲಿ
ಸ್ವರವೂ
ಸ್ವರವೇ
ಸ್ವರಿ-ತವೇ-ಣುನಾ
ಸ್ವರೂಪ
ಸ್ವರೂಪಕ್ಕೆ
ಸ್ವರೂಪದ
ಸ್ವರೂ-ಪ-ದಲ್ಲಿ
ಸ್ವರೂ-ಪ-ನಾಗಿ-ರು-ವನು
ಸ್ವರೂಪನು
ಸ್ವರೂ-ಪ-ನೆಂಬು-ದೊಂದೇ
ಸ್ವರೂ-ಪರು
ಸ್ವರೂ-ಪ-ವನ್ನು
ಸ್ವರೂ-ಪ-ವಾಗಿ
ಸ್ವರೂ-ಪ-ವಾಗಿ-ದೆಯೋ
ಸ್ವರೂ-ಪ-ವಾಗಿ-ರ-ಬೇಕು
ಸ್ವರೂಪವು
ಸ್ವರೂ-ಪವೇ
ಸ್ವರೂ-ಪ-ಸ್ಥಿತ-ರಾಗು-ತ್ತೀರಿ
ಸ್ವರ್ಗ
ಸ್ವರ್ಗ-ಕ್ಕಾಗಿ
ಸ್ವರ್ಗ-ಕ್ಕಿಂತ
ಸ್ವರ್ಗಕ್ಕೆ
ಸ್ವರ್ಗ-ಗಳು
ಸ್ವರ್ಗದ
ಸ್ವರ್ಗದಲ್ಲಿ
ಸ್ವರ್ಗದಲ್ಲಿ-ದ್ದರೂ
ಸ್ವರ್ಗ-ಲೋಕ-ಗಳನ್ನು
ಸ್ವರ್ಗ-ವನ್ನು
ಸ್ವರ್ಗ-ವನ್ನೂ
ಸ್ವರ್ಗ-ವಿದೆ
ಸ್ವರ್ಗವು
ಸ್ವರ್ಗ-ವೇನಾ-ದರೂ
ಸ್ವರ್ಗ-ಸುಖ
ಸ್ವರ್ಗಾ-ದಪಿ
ಸ್ವರ್ಗಾಭಿಲಾಷೆ-ಯಿ-ಲ್ಲದೆ
ಸ್ವರ್ಶಿ-ಸಿದರೆ
ಸ್ವಲ್ಪ
ಸ್ವಲ್ಪ-ದ-ರಿಂದಲೇ
ಸ್ವಲ್ಪ-ಭಾಗ-ವನ್ನು
ಸ್ವಲ್ಪ-ಮಟ್ಟಿಗೆ
ಸ್ವಲ್ಪ-ಮಟ್ಟಿನ
ಸ್ವಲ್ಪ-ಮಪ್ಯಸ್ಯ
ಸ್ವಲ್ಪ-ವಾದರೂ
ಸ್ವಲ್ಪವೂ
ಸ್ವಲ್ಪಾಂಶ-ಗಳ-ನ್ನಾ-ದರೂ
ಸ್ವವಿರೋ-ಧಿ-ಯಾ-ದುದು
ಸ್ವವೇಗ-ದಿಂದ
ಸ್ವಶಕ್ತಿ-ಯನ್ನು
ಸ್ವಸಂರಕ್ಷಣಾರ್ಥ-ವಾಗಿ
ಸ್ವಸ್ವ-ರೂಪ
ಸ್ವಾಗತ
ಸ್ವಾಗತ-ಕ್ಕಾಗಿ
ಸ್ವಾಗತಕ್ಕೂ
ಸ್ವಾಗ-ತಕ್ಕೆ
ಸ್ವಾಗ-ತದ
ಸ್ವಾಗತ-ದಲ್ಲಿ
ಸ್ವಾಗತ-ವನ್ನು
ಸ್ವಾಗ-ತವು
ಸ್ವಾಗತಿ-ಸಲಿ
ಸ್ವಾಗತಿ-ಸಲು
ಸ್ವಾಗತಿ-ಸಿ-ದರು
ಸ್ವಾಗತಿ-ಸಿ-ರು-ವರು
ಸ್ವಾಗತಿ-ಸುತ್ತಾ
ಸ್ವಾಗತಿ-ಸುತ್ತಿ-ದ್ದೀರಿ
ಸ್ವಾಗತಿ-ಸು-ತ್ತೇನೆ
ಸ್ವಾಗತಿ-ಸುತ್ತೇವೆ
ಸ್ವಾಗತಿ-ಸುವ
ಸ್ವಾಗತಿ-ಸು-ವೆವು
ಸ್ವಾ-ತಂತ್ರ್ಯ
ಸ್ವಾ-ತಂತ್ರ್ಯ-ಕ್ಕಾಗಿ
ಸ್ವಾ-ತಂತ್ರ್ಯಕ್ಕೆ
ಸ್ವಾ-ತಂತ್ರ್ಯದ
ಸ್ವಾ-ತಂತ್ರ್ಯ-ದಿಂದ
ಸ್ವಾ-ತಂತ್ರ್ಯ-ಪೂರ್ವದ
ಸ್ವಾ-ತಂತ್ರ್ಯ-ಪೂರ್ವ-ದಲ್ಲಿ
ಸ್ವಾ-ತಂತ್ರ್ಯ-ರೂಪ-ವಾದ
ಸ್ವಾ-ತಂತ್ರ್ಯ-ವನ್ನು
ಸ್ವಾ-ತಂತ್ರ್ಯ-ವನ್ನೂ
ಸ್ವಾ-ತಂತ್ರ್ಯ-ವಿದೆ
ಸ್ವಾ-ತಂತ್ರ್ಯ-ವೆಲ್ಲ
ಸ್ವಾ-ತಂತ್ರ್ಯವೇ
ಸ್ವಾ-ತಂತ್ರ್ಯ-ವೇನೋ
ಸ್ವಾ-ತಂತ್ರ್ಯೋ-ತ್ತರ-ದಲ್ಲಿ
ಸ್ವಾದ್ವತ್ತೈನಶ್ನನ್ನ-ನ್ಯೋ
ಸ್ವಾ-ಭಾವಿಕ
ಸ್ವಾ-ಭಾವಿಕ-ವಾಗಿ
ಸ್ವಾ-ಭಾವಿಕ-ವಾಗಿಯೇ
ಸ್ವಾ-ಭಾವಿಕ-ವಾಗಿ-ರು-ವುದು
ಸ್ವಾ-ಭಾವಿಕ-ವಾದ
ಸ್ವಾಮಿ
ಸ್ವಾ-ಮಿಗೆ
ಸ್ವಾ-ಮಿಜಿ
ಸ್ವಾ-ಮಿ-ಯ-ವರೇ
ಸ್ವಾ-ಮಿ-ಯಾಗಲೂ
ಸ್ವಾ-ಮಿಯು
ಸ್ವಾಮೀ
ಸ್ವಾ-ಮೀಜಿ
ಸ್ವಾ-ಮೀ-ಜಿ-ಯವರ
ಸ್ವಾ-ಮೀ-ಜಿ-ಯವ-ರ-ಕೊಲಂಬೊ
ಸ್ವಾ-ಮೀ-ಜಿ-ಯ-ವ-ರನ್ನು
ಸ್ವಾ-ಮೀ-ಜಿ-ಯವ-ರಿಗೆ
ಸ್ವಾ-ಮೀ-ಜಿ-ಯ-ವರು
ಸ್ವಾ-ಮೀ-ಜಿ-ಯ-ವರೆ
ಸ್ವಾ-ಮೀ-ಜಿ-ಯ-ವರೇ
ಸ್ವಾಯ
ಸ್ವಾ-ರಸ್ಯಕ-ರ-ವಾದ
ಸ್ವಾರ್ಥ
ಸ್ವಾರ್ಥಕ್ಕೆ
ಸ್ವಾರ್ಥತೆ
ಸ್ವಾರ್ಥ-ತ್ಯಾಗ-ಪೂರ್ವ-ಕ-ವಾದ
ಸ್ವಾರ್ಥದ
ಸ್ವಾರ್ಥ-ದಿಂದ
ಸ್ವಾರ್ಥ-ಪರತೆ
ಸ್ವಾರ್ಥ-ಪರ-ವಾದ
ಸ್ವಾರ್ಥ-ವನ್ನೆಲ್ಲಾ
ಸ್ವಾರ್ಥವೇ
ಸ್ವಾರ್ಥಾಕಾಂಕ್ಷಿ-ಗಳಾದ
ಸ್ವಾರ್ಥಿ-ಗಳಾಗಿ-ರು-ವೆವು
ಸ್ವಾರ್ಥಿ-ಗಳು
ಸ್ವಾರ್ಥಿ-ಯಾಗಿ-ದ್ದರೆ
ಸ್ವಿಟ್ಜರ್ಲೆಂಡ್
ಸ್ವೀಕರಿಸ-ಕೂಡದು
ಸ್ವೀಕರಿಸ-ಬಲ್ಲರು
ಸ್ವೀಕರಿ-ಸ-ಬಲ್ಲಿರಿ
ಸ್ವೀಕರಿಸ-ಬಹುದು
ಸ್ವೀಕರಿಸ-ಬೇಕಾಗಿಲ್ಲ
ಸ್ವೀಕ-ರಿಸ-ಬೇಕಾ-ಯಿತು
ಸ್ವೀಕರಿಸ-ಬೇಕು
ಸ್ವೀಕರಿ-ಸ-ಲಾ-ರರು
ಸ್ವೀಕ-ರಿ-ಸಲು
ಸ್ವೀಕರಿಸ-ಲೇ-ಬೇಕು
ಸ್ವೀಕ-ರಿಸಿ
ಸ್ವೀಕ-ರಿಸಿದ
ಸ್ವೀಕ-ರಿಸಿ-ದರು
ಸ್ವೀಕ-ರಿಸಿ-ದರೆ
ಸ್ವೀಕ-ರಿಸಿ-ದವು
ಸ್ವೀಕ-ರಿಸಿ-ಧರ್ಮ-ಶಾಸ್ತ್ರದ
ಸ್ವೀಕ-ರಿಸಿ-ರ-ಬಹುದು
ಸ್ವೀಕ-ರಿಸಿ-ರುವ
ಸ್ವೀಕ-ರಿಸಿ-ರು-ವರು
ಸ್ವೀಕ-ರಿಸಿ-ರು-ವೆವು
ಸ್ವೀಕರಿ-ಸುತ್ತಾ
ಸ್ವೀಕರಿ-ಸುವ
ಸ್ವೀಕರಿ-ಸು-ವನು
ಸ್ವೀಕರಿ-ಸುವ-ರೆಂಬುದು
ಸ್ವೀಕರಿ-ಸು-ವರೋ
ಸ್ವೀಕರಿ-ಸುವ-ವನ
ಸ್ವೀಕರಿ-ಸುವ-ವನಿ-ಗಿಂತ
ಸ್ವೀಕರಿ-ಸುವ-ವನು
ಸ್ವೀಕರಿಸು-ವಿರಿ
ಸ್ವೀಕರಿಸು-ವಿರೋ
ಸ್ವೀಕರಿ-ಸು-ವು-ದರ
ಸ್ವೀಕರಿಸು-ವುದು
ಸ್ವೀಕಾರ-ಯೋಗ್ಯ
ಹಂಚ-ದಿ-ದ್ದರೆ
ಹಂಚಿ
ಹಂಚಿಕೆ
ಹಂಚು-ವುದು
ಹಂತ
ಹಂತ-ಗಳಿಗೆ
ಹಂತ-ದಲ್ಲಿ
ಹಂತ-ದಲ್ಲಿಯೂ
ಹಂತ-ವನ್ನು
ಹಂತ-ವಾಗಿ
ಹಂತ-ವೆಂದರೆ
ಹಂದಿ-ಯನ್ನು
ಹಂಬಲ
ಹಂಬಲಿ-ಸು-ವಾಗ
ಹಕಾರ-ವನ್ನು
ಹಕ್ಕನ್ನು
ಹಕ್ಕಾದ
ಹಕ್ಕಿ
ಹಕ್ಕಿ-ಗಳಿವೆ
ಹಕ್ಕಿದೆ
ಹಕ್ಕಿಯ
ಹಕ್ಕಿ-ಯನ್ನು
ಹಕ್ಕು
ಹಕ್ಕು-ಗಳಿವೆ
ಹಕ್ಕು-ಗಳು
ಹಕ್ಕು-ಗಳೂ
ಹಕ್ಕು-ಗಳೆಲ್ಲ
ಹಕ್ಕು-ದಾರರಾಗುತ್ತೇವೆ
ಹಕ್ಕು-ದಾರಿ-ಕೆ-ಯನ್ನು
ಹಕ್ಕು-ಬಾ-ಧ್ಯತೆ-ಗಳ
ಹಕ್ಕು-ಬಾ-ಧ್ಯತೆ-ಗಳು
ಹಗಲಿ-ರುಳೂ
ಹಗಲು
ಹಗಲೂ
ಹಗ-ಲೆಲ್ಲ
ಹಗ್ಗ
ಹಚ್ಚಿ-ಕೊಂಡು
ಹಟ
ಹಠಾತ್ತಾಗಿ
ಹಡಗನ್ನು
ಹಡಗಿ-ನಲ್ಲಿ
ಹಡ-ಗಿ-ನಿಂದ
ಹಡಗು
ಹಣ
ಹಣದ
ಹಣ-ದಾಸೆ-ಯಿಂದ
ಹಣ-ಮಾಡು-ವುದು
ಹಣ-ವಲ್ಲ
ಹಣ-ವೆ-ಲ್ಲಿದೆ
ಹಣ-ಸಂಪಾದ-ನೆಗೆ
ಹಣೆಯ
ಹಣ್ಣ-ನ್ನಾ-ಗಲಿ
ಹಣ್ಣನ್ನು
ಹಣ್ಣು
ಹಣ್ಣು-ಗಳನ್ನು
ಹತರಾ-ದು-ದನ್ನು
ಹತೋಟಿ
ಹತೋಟಿಗೆ
ಹತೋಟಿ-ಯಲ್ಲಿ-ಡಲು
ಹತೋಟಿ-ಯಲ್ಲಿ-ಡು-ವು-ದಕ್ಕೆ
ಹತ್ತರಷ್ಟು
ಹತ್ತಾರು
ಹತ್ತಿ-ಕೊಂಡು
ಹತ್ತಿದರು
ಹತ್ತಿಯ
ಹತ್ತಿರ
ಹತ್ತಿ-ರಕ್ಕೆ
ಹತ್ತಿರ-ವಾಗಿ-ರು-ವುದು
ಹತ್ತು
ಹತ್ತು-ಸಾವಿರ
ಹತ್ತೊಂಬತ್ತ-ನೆಯ
ಹದಿ-ನಾರ-ನೆಯ
ಹದಿ-ನಾರು
ಹದಿ-ನಾಲ್ಕ-ನೆಯ
ಹದಿ-ನಾಲ್ಕು
ಹದಿಯೇ
ಹನಿಯು-ತ್ತದೆ
ಹನುಮಂತ
ಹನುಮ-ನಂತೆ
ಹನ್ನೆ-ರಡು
ಹಬ್ಬಿ
ಹಬ್ಬಿ-ದುವು
ಹಬ್ಬಿ-ರು-ತ್ತದೆ
ಹಬ್ಬಿ-ಹೋ-ಗಿದೆ
ಹರಟೆ
ಹರ-ಟೆ-ಮಲ್ಲ
ಹರ-ಟೆ-ಯನ್ನಾ-ದರೂ
ಹರ-ಟೆ-ಯನ್ನೇ
ಹರ-ಡ-ದಂತೆ
ಹರ-ಡ-ಬಹುದು
ಹರ-ಡ-ಬೇಕಾಗಿ-ತ್ತು
ಹರ-ಡಬೇ-ಕಾದು-ದರ
ಹರ-ಡ-ಬೇಕು
ಹರ-ಡ-ಲಿಲ್ಲ
ಹರ-ಡಲು
ಹರ-ಡಿ-ಕೊಂಡಿವೆ
ಹರ-ಡಿತು
ಹರ-ಡಿ-ದ್ದರೆ
ಹರ-ಡಿ-ದ್ದವು
ಹರ-ಡಿದ್ದು
ಹರ-ಡಿರು-ವುದು
ಹರ-ಡಿವೆ
ಹರ-ಡುವ-ವರು
ಹರಡು-ವು-ದ-ಕ್ಕಾಗಿ
ಹರ-ಡು-ವು-ದಕ್ಕೆ
ಹರಡು-ವುದು
ಹರ-ಸಲಿ
ಹರಿಃ
ಹರಿದಾಡುತ್ತಿ-ರುವ
ಹರಿದಾ-ಡುವ
ಹರಿದು
ಹರಿದು-ಬಂದಿದೆ
ಹರಿದು-ಹೋ-ಗು-ವು-ದಿಲ್ಲ
ಹರಿದು-ಹೋ-ಯಿತು
ಹರಿ-ಭಕ್ತ
ಹರಿ-ಯನ್ನು
ಹರಿಯ-ಬಹುದು
ಹರಿಯ-ಬೇಕಾಗಿದೆ
ಹರಿಯ-ಬೇಕಾ-ಯಿತು
ಹರಿಯ-ಬೇಕು
ಹರಿ-ಯಲು
ಹರಿ-ಯಿತು
ಹರಿ-ಯುತ್ತಾ
ಹರಿ-ಯು-ತ್ತಿದೆ
ಹರಿ-ಯು-ತ್ತಿದೆಯೋ
ಹರಿಯುತ್ತಿದ್ದರೂ
ಹರಿಯುತ್ತಿದ್ದು-ದ-ರಿಂದ
ಹರಿಯುತ್ತಿ-ರುವ
ಹರಿಯುತ್ತಿರು-ವಾಗ
ಹರಿಯುತ್ತಿರು-ವು-ದನ್ನು
ಹರಿಯುತ್ತಿರು-ವುದು
ಹರಿಯುತ್ತಿರು-ವುದೆಂಬುದು
ಹರಿಯು-ತ್ತಿವೆ
ಹರಿಯುತ್ತಿ-ವೆಯೋ
ಹರಿ-ಯುವ
ಹರಿ-ಯು-ವಂತೆ
ಹರಿ-ಯು-ವು-ದಕ್ಕೆ
ಹರಿಸ-ಬಹುದು
ಹರಿಸ-ಲಾ-ಗು-ವು-ದಿಲ್ಲ
ಹರಿಸಿರಿ
ಹರಿ-ಸುವವು
ಹರುಡು-ತ್ತಿವೆ
ಹರ್ಬರ್ಟ್
ಹರ್ಷಾಶ್ರು
ಹಲ-ಕೆಲವು
ಹಲ-ವನ್ನು
ಹಲವ-ರಿಗೆ
ಹಲ-ವರು
ಹಲ-ವಾರು
ಹಲವು
ಹಲ್ಲುಕಿರಿದು
ಹಳಬರಿ-ಗಿಂತಲೂ
ಹಳಿಯು-ವರು
ಹಳಿಯು-ವು-ದರ
ಹಳೆಯ
ಹಳ್ಳ-ದಲ್ಲಿ
ಹಳ್ಳಿಗ
ಹಳ್ಳಿಗನೂ
ಹಳ್ಳಿಗಳಿಂದ
ಹಳ್ಳಿಗೆ
ಹಳ್ಳಿಯ
ಹಳ್ಳಿ-ಯಲ್ಲಿ
ಹಳ್ಳಿಯ-ವ-ನಿಗೆ
ಹಳ್ಳಿಯ-ವನೊ-ಬ್ಬ-ನನ್ನು
ಹವಣಿಸುತ್ತಿ-ರುವ
ಹವಾಗುಣ
ಹವಿ-ಸ್ಸನ್ನು
ಹವ್ಯಾ-ಸ-ಗಳು
ಹವ್ಯಾ-ಸ-ವನ್ನೆಲ್ಲಾ
ಹಸಿದು
ಹಸಿ-ವನ್ನು
ಹಸಿ-ವಾಗಿದೆ
ಹಸಿವು
ಹಸಿ-ವೆ-ಯಿಂದ
ಹಸು-ಗಳನ್ನು
ಹಸುಳೆ
ಹಸುಳೆ-ಗಳ
ಹಸುವಿ-ನಂತೆ
ಹಸ್ತ
ಹಸ್ತ-ಗಳು
ಹಸ್ತ-ಲಾಘವ
ಹಾ
ಹಾಕ-ದವ-ನಿಗೂ
ಹಾಕದೆ
ಹಾಕ-ಬೇ-ಕಾದರೆ
ಹಾಕಲು
ಹಾಕಿ
ಹಾಕಿ-ಕೊಂಡಿ-ಲ್ಲದ
ಹಾಕಿ-ಕೊಳ್ಳದೆ
ಹಾಕಿ-ತ್ತು
ಹಾಕಿ-ದರು
ಹಾಕಿ-ದರೂ
ಹಾಕಿ-ದರೆ
ಹಾಕಿ-ದವ-ನಿಗೂ
ಹಾಕಿದೆ
ಹಾಕಿ-ದೆ-ನೆಂದರೆ
ಹಾಕಿ-ಬಿಡ
ಹಾಕಿ-ರುವ
ಹಾಕಿ-ರು-ವೆವು
ಹಾಕುತ್ತಾರೆ
ಹಾಕುತ್ತಿ-ರು-ವರು
ಹಾಕು-ತ್ತೇನೆ
ಹಾಕುತ್ತೇವೆ
ಹಾಕುವ
ಹಾಕು-ವು-ದ-ರಿಂದ
ಹಾಕು-ವುದು
ಹಾಕು-ವೆವು
ಹಾಗಲ್ಲ
ಹಾಗ-ಲ್ಲದೆ
ಹಾಗಲ್ಲ-ವೆಂದು
ಹಾಗಾಗದೇ
ಹಾಗಾಗ-ಬೇಕು
ಹಾಗಾ-ಗು-ವಂತೆ
ಹಾಗಾ-ದರೆ
ಹಾಗಾ-ದಾಗ
ಹಾಗಿದ್ದರೆ
ಹಾಗಿದ್ದಿ-ದ್ದರೆ
ಹಾಗಿದ್ದು
ಹಾಗಿರ-ಬೇ-ಕಾ-ದದ್ದು
ಹಾಗಿರಲಿ
ಹಾಗಿ-ರ-ಲಿಲ್ಲ
ಹಾಗಿರು-ವಾಗ
ಹಾಗಿಲ್ಲ
ಹಾಗಿಲ್ಲ-ದಿ-ದ್ದರೆ
ಹಾಗಿಲ್ಲದೆ
ಹಾಗಿಲ್ಲದೇ
ಹಾಗಿಲ್ಲ-ವೆಂದೂ
ಹಾಗೂ
ಹಾಗೆ
ಹಾಗೆಂದ
ಹಾಗೆಂದರೆ
ಹಾಗೆಂದಿಗೂ
ಹಾಗೆಂದು
ಹಾಗೆಯೆ
ಹಾಗೆಯೇ
ಹಾಗೆಲ್ಲ
ಹಾಗೆಲ್ಲಾ
ಹಾಡ-ಬೇಕಾಗಿದೆ
ಹಾಡು-ತ್ತಿದ್ದಳು
ಹಾಡುವ
ಹಾತೊರೆಯು-ತ್ತಿದ್ದ
ಹಾದಿ
ಹಾದಿ-ಯನ್ನು
ಹಾದಿ-ಯನ್ನೇ
ಹಾದಿ-ಯಲ್ಲಿ
ಹಾದು
ಹಾನಿ
ಹಾನಿ-ಕರ
ಹಾನಿಗೆ
ಹಾನಿ-ಮಾಡದೇ
ಹಾನಿಯೂ
ಹಾಯಿಸಿ-ದಂತೆ
ಹಾರಾಡು-ತ್ತಿದ್ದ
ಹಾರಾಡುತ್ತಿರು-ವುದು
ಹಾರಿಸಿ
ಹಾರಿಸಿ-ದುದು
ಹಾರಿಸು-ವುದು
ಹಾರು-ತ್ತಿದ್ದರೆ
ಹಾರೈಸು-ತ್ತೇನೆ
ಹಾರ್ದಿಕ
ಹಾರ್ದಿಕ-ವಾಗಿ
ಹಾರ್ದಿಕ-ವಾದ
ಹಾರ್ದಿ-ಕವೂ
ಹಾಲಿ-ನಲ್ಲಿ
ಹಾಲು
ಹಾಳಾಗಿ
ಹಾಳಾ-ಗು-ವು-ದಿಲ್ಲ
ಹಾಳಾದರೆ
ಹಾವಳಿ-ಯನ್ನು
ಹಾವು
ಹಾಸು
ಹಾಸು-ಹೊ-ಕ್ಕಾಗಿದೆ
ಹಾಸು-ಹೊ-ಕ್ಕಾಗಿ-ರು-ವನು
ಹಾಸ್ಯ
ಹಾಸ್ಯ-ಮಾಡುವ
ಹಾಸ್ಯಾ-ಸ್ಪದ-ವಾಗಿದೆ
ಹಾಸ್ಯಾ-ಸ್ಪದ-ವಾದುದು
ಹಾಹಾ-ಕಾರ-ದಿಂದ
ಹಿ
ಹಿಂಜರಿ-ಯದೆ
ಹಿಂಜರಿಯ-ಬಾ-ರದು
ಹಿಂಜರಿಯು-ವು-ದನ್ನು
ಹಿಂಜರಿ-ಯು-ವುದೇ
ಹಿಂಡಿ
ಹಿಂತಿ-ರುಗಿ
ಹಿಂತಿ-ರುಗಿದ
ಹಿಂತಿ-ರುಗಿ-ದರೆ
ಹಿಂತಿ-ರುಗಿ-ಸ-ಬಲ್ಲಿರಾ
ಹಿಂತಿರುಗುತ್ತಿ-ರುವ
ಹಿಂತಿರುಗು-ವಿರಿ
ಹಿಂತಿರುಗು-ವುದೋ
ಹಿಂತಿರು-ಗುವುವೊ
ಹಿಂದಕ್ಕೆ
ಹಿಂದಿ
ಹಿಂದಿ-ಗಿಂತ
ಹಿಂದಿ-ಗಿಂತಲೂ
ಹಿಂದಿನ
ಹಿಂದಿನಂತೆ
ಹಿಂದಿನಂತೆಯೇ
ಹಿಂದಿನ-ಕಾಲದ
ಹಿಂದಿನ-ಕಾಲ-ದಲ್ಲಿ
ಹಿಂದಿನ-ದ-ಕ್ಕಿಂತ
ಹಿಂದಿನ-ದರ
ಹಿಂದಿನ-ವ-ರಾಗಲೀ
ಹಿಂದಿನ-ವರು
ಹಿಂದಿ-ನಷ್ಟು
ಹಿಂದಿ-ನಿಂದ
ಹಿಂದಿ-ನಿಂದಲೂ
ಹಿಂದಿ-ರುಗಿ
ಹಿಂದಿ-ರುಗಿ-ದಾಗ
ಹಿಂದಿ-ರುಗಿ-ದು-ದ-ಕ್ಕಾಗಿ
ಹಿಂದಿ-ರುಗಿ-ರುವ
ಹಿಂದಿ-ರು-ಗು-ತ್ತದೆ
ಹಿಂದಿ-ರುಗು-ವುವೋ
ಹಿಂದಿ-ರುವ
ಹಿಂದಿಲ
ಹಿಂದು-ಗಳಂತೆ
ಹಿಂದು-ಗಳಿರಾ
ಹಿಂದುತ್ವವನ್ನೇ
ಹಿಂದು-ತ್ವವೇ
ಹಿಂದು-ಧರ್ಮಕ್ಕೆ
ಹಿಂದುಳಿದಿರು-ವೆವು
ಹಿಂದು-ವಲ್ಲ
ಹಿಂದುವೂ
ಹಿಂದೂ
ಹಿಂದೂ-ಗಳ
ಹಿಂದೂ-ಗಳನ್ನು
ಹಿಂದೂ-ಗಳಲ್ಲ
ಹಿಂದೂ-ಗಳ-ಲ್ಲಿಯೂ
ಹಿಂದೂ-ಗಳಷ್ಟು
ಹಿಂದೂ-ಗಳಾಗ-ಬೇ-ಕಾದರೆ
ಹಿಂದೂ-ಗಳಾಗಲೀ
ಹಿಂದೂ-ಗಳಾದ
ಹಿಂದೂ-ಗಳಾದು-ದ-ರಿಂದ
ಹಿಂದೂ-ಗಳಿಗೆ
ಹಿಂದೂ-ಗಳಿಗೇ
ಹಿಂದೂ-ಗಳಿ-ದ್ದರು
ಹಿಂದೂ-ಗಳಿ-ರು-ವರು
ಹಿಂದೂ-ಗಳು
ಹಿಂದೂ-ಗಳೂ
ಹಿಂದೂ-ಗಳೆ
ಹಿಂದೂ-ಗಳೆಂದು
ಹಿಂದೂ-ಗಳೆ-ಲ್ಲರೂ
ಹಿಂದೂ-ಗಳೇ
ಹಿಂದೂ-ಗಳೊಂದಿಗೆ
ಹಿಂದೂ-ದೇಶ
ಹಿಂದೂ-ಧರ್ಮ
ಹಿಂದೂ-ಧರ್ಮದ
ಹಿಂದೂ-ಧರ್ಮ-ದಲ್ಲಿ
ಹಿಂದೂ-ಧರ್ಮ-ದಲ್ಲಿ-ರುವ
ಹಿಂದೂ-ಧರ್ಮ-ವನ್ನು
ಹಿಂದೂ-ಧರ್ಮವು
ಹಿಂದೂ-ನಿ-ವಾ-ಸಿ-ಗಳು
ಹಿಂದೂ-ಭಾವ-ನೆ-ಗಳು
ಹಿಂದೂ-ಮಹ-ಮ್ಮದೀಯರ
ಹಿಂದೂ-ವನ್ನು
ಹಿಂದೂ-ವಿಗೂ
ಹಿಂದೂ-ವಿಗೆ
ಹಿಂದೂವು
ಹಿಂದೂವೂ
ಹಿಂದೂ-ವೆಂದು
ಹಿಂದೂ-ಶಾಸ್ತ್ರ
ಹಿಂದೂ-ಸ್ಥಾನದ
ಹಿಂದೆ
ಹಿಂದೆಯೂ
ಹಿಂದೆಯೇ
ಹಿಂದೆಲ್ಲ
ಹಿಂಬಾ-ಲಿ-ಸುತ್ತವೆ
ಹಿಂಬಾ-ಲಿ-ಸುತ್ತಾ
ಹಿಂಬಾ-ಲಿಸು-ವಿರಿ
ಹಿಂಸಿ-ಸಲಿ
ಹಿಂಸಿಸಿ
ಹಿಂಸಿ-ಸು-ವಂತೆ
ಹಿಂಸಿ-ಸು-ವು-ದಿಲ್ಲ
ಹಿಂಸೆ
ಹಿಂಸೆಯ
ಹಿಂಸೆ-ಯಾಗು-ತ್ತಿದೆ
ಹಿಗ್ಗ-ಬೇಕು
ಹಿಗ್ಗಲು
ಹಿಗ್ಗಿ-ಗಾಗಿ
ಹಿಡಿದರು
ಹಿಡಿ-ದಿ-ದ್ದರು
ಹಿಡಿ-ದಿ-ದ್ದರೆ
ಹಿಡಿದಿ-ರುವ
ಹಿಡಿದಿರು-ವೆವೋ
ಹಿಡಿದು
ಹಿಡಿದು-ಕೊಂಡಿ-ದ್ದರೆ
ಹಿಡಿದು-ಕೊಂಡಿ-ರ-ಬೇಕು
ಹಿಡಿದು-ಕೊಂಡಿ-ರು-ವೆವೆ
ಹಿಡಿದು-ಕೊಂಡು
ಹಿಡಿ-ಯದೆ
ಹಿಡಿಯ-ಬೇಕು
ಹಿಡಿ-ಯಿರಿ
ಹಿಡಿಯುತ್ತಿರು-ವುದು
ಹಿಡಿ-ಯುವ
ಹಿಡಿ-ಯು-ವಂತೆ
ಹಿಡಿ-ಯು-ವು-ದಕ್ಕೆ
ಹಿಡಿಯು-ವುದು
ಹಿಡಿಯು-ವುವು
ಹಿಡಿಸು-ತ್ತಿವೆ
ಹಿಡಿ-ಸು-ವಷ್ಟು
ಹಿಡಿ-ಸು-ವು-ದಿಲ್ಲ
ಹಿತ
ಹಿತ-ಕಾರಿ
ಹಿತ-ಕಾರಿ-ಯಲ್ಲ
ಹಿತ-ಕಾರಿ-ಯಲ್ಲದ
ಹಿತ-ಕಾರಿ-ಯ-ಲ್ಲವೋ
ಹಿತ-ಕ್ಕಾಗಿ
ಹಿತಕ್ಕೆ
ಹಿತ-ರಕ್ಷಣೆಯ
ಹಿತ-ವನ್ನು
ಹಿತ-ವನ್ನೇ
ಹಿತವೆ
ಹಿತೈಷಿಗಳಾದು-ದ-ರಿಂದ
ಹಿತೈಷಿ-ಯಾಗಿ-ದ್ದಾರೆ
ಹಿನಸ್ತಾ್ಯತ್ಮನಾತ್ಮಾನಂ
ಹಿನಸ್ತ್ಯಾ-ತ್ಮನಾತ್ಮಾನಂ
ಹಿನ್ನೆಲೆ
ಹಿನ್ನೆ-ಲೆ-ಯನ್ನು
ಹಿನ್ನೆ-ಲೆ-ಯಲ್ಲಿ
ಹಿನ್ನೆ-ಲೆ-ಯ-ಲ್ಲಿಯೂ
ಹಿನ್ನೆ-ಲೆ-ಯಾಗಿದೆ
ಹಿನ್ನೆ-ಲೆ-ಯಾದ
ಹಿಮ-ಮಣಿ
ಹಿಮ-ಮ-ಣಿಯು
ಹಿಮ-ವಂತೋ
ಹಿಮಾ-ಚಲ
ಹಿಮಾ-ಲಯ
ಹಿಮಾ-ಲ-ಯಕ್ಕೆ
ಹಿಮಾ-ಲಯ-ಗಳಿಂದ
ಹಿಮಾ-ಲಯ-ಗಳು
ಹಿಮಾ-ಲ-ಯದ
ಹಿಮಾ-ಲಯ-ದಲ್ಲಿ
ಹಿಮಾ-ಲಯ-ದಲ್ಲಿ-ರುವ
ಹಿಮಾ-ಲಯ-ದಿಂದ
ಹಿಮಾ-ಲಯ-ವನ್ನು
ಹಿಮಾ-ಲ-ಯವು
ಹಿಮಾ-ವೃತ
ಹಿಮ್ಮರಳಿ
ಹಿಮ್ಮುಖ-ವಾಗಿ
ಹಿಮ್ಮೆಟ್ಟಿಸ-ಲಾ-ರದು
ಹಿಮ್ಮೆಟ್ಟಿ-ಸಲು
ಹಿರಿದ
ಹಿರಿ-ದಾದ
ಹಿರಿ-ದಾದು-ದನ್ನು
ಹಿರಿದು
ಹಿರಿಮೆಗೆ
ಹಿರಿಮೆಯೂ
ಹಿರಿಯ
ಹಿರಿ-ಯರ
ಹಿರಿಯ-ರಿಂದ
ಹಿರಿ-ಯರು
ಹಿಸುಕಿ
ಹೀಗಲ್ಲ
ಹೀಗಾ-ಗದೆ
ಹೀಗಾಗಿ
ಹೀಗಾ-ದುದು
ಹೀಗಿದೆ
ಹೀಗಿದ್ದ
ಹೀಗಿದ್ದರೂ
ಹೀಗಿದ್ದರೆ
ಹೀಗಿ-ರಲು
ಹೀಗಿ-ರು-ವಲ್ಲಿ
ಹೀಗಿರು-ವಾಗ
ಹೀಗೆ
ಹೀಗೆಂದ
ಹೀಗೆಂದೆ
ಹೀಗೆಂಬುದೇ
ಹೀಗೆಯೆ
ಹೀಗೆಯೇ
ಹೀಗೇ
ಹೀದನ್
ಹೀನ
ಹೀನ-ಕುಲ-ಜಳಾ-ದರೂ
ಹೀನ-ಕೃತ್ಯ-ಗಳನ್ನು
ಹೀನ-ತೆಯ
ಹೀನ-ನಾ-ಗಿರು-ತ್ತಾನೆ
ಹೀನ-ನಿಗೆ
ಹೀನ-ಭಾ-ವನೆ
ಹೀನ-ವಾಗಿ
ಹೀನ-ವಾದ
ಹೀನ-ವಾದು-ವೆಂದು
ಹೀನ-ಸ್ಥಿತಿಯ
ಹೀನು-ಕುಲ-ಜನಾ-ದರೂ
ಹೀರಿ
ಹೀರಿ-ಕೊಳ್ಳಲಿ
ಹೀರಿ-ದರೆ
ಹೀರು-ತ್ತಿವೆ
ಹುಚ್ಚ-ನಂತೆ
ಹುಚ್ಚ-ನಾಗು-ವನು
ಹುಚ್ಚನಾ-ದಾಗ
ಹುಚ್ಚ-ನೆ-ನ್ನು-ವುದು
ಹುಚ್ಚರ
ಹುಚ್ಚರ-ನ್ನಾಗಿ
ಹುಚ್ಚ-ರಾಗಿ
ಹುಚ್ಚರಾಗು-ವಿರಿ
ಹುಚ್ಚ-ರಿಗೆ
ಹುಚ್ಚರು
ಹುಚ್ಚ-ರೆಲ್ಲಾ
ಹುಚ್ಚಾದ
ಹುಚ್ಚು
ಹುಟ್ಟದ
ಹುಟ್ಟದೆ
ಹುಟ್ಟಲಿ
ಹುಟ್ಟಲಿ-ರುವ
ಹುಟ್ಟ-ಲಿಲ್ಲ
ಹುಟ್ಟ-ವನು
ಹುಟ್ಟಿ
ಹುಟ್ಟಿತು
ಹುಟ್ಟಿದ
ಹುಟ್ಟಿ-ದಂದಿ-ನಿಂದ
ಹುಟ್ಟಿ-ದಂದಿ-ನಿಂದಲೂ
ಹುಟ್ಟಿ-ದನು
ಹುಟ್ಟಿ-ದರು
ಹುಟ್ಟಿ-ದರೆ
ಹುಟ್ಟಿ-ದ-ವರು
ಹುಟ್ಟಿ-ದವು
ಹುಟ್ಟಿ-ದಾ-ಗಿ-ನಿಂದಲೂ
ಹುಟ್ಟಿ-ದುದು
ಹುಟ್ಟಿದ್ದು
ಹುಟ್ಟಿ-ದ್ದೇನೆ
ಹುಟ್ಟಿ-ನಿಂದ
ಹುಟ್ಟಿ-ರುವ
ಹುಟ್ಟಿ-ರು-ವುದೇ
ಹುಟ್ಟಿ-ಲ್ಲ-ದುದು
ಹುಟ್ಟಿವೆ
ಹುಟ್ಟಿಸಿ
ಹುಟ್ಟಿ-ಸುತ್ತವೆ
ಹುಟ್ಟಿ-ಸುತ್ತಿ-ದ್ದಾರೆ
ಹುಟ್ಟಿ-ಸುವಂಥದು
ಹುಟ್ಟಿ-ಸುವ-ಷ್ಟರ
ಹುಟ್ಟು
ಹುಟ್ಟು-ಗುಲಾಮ
ಹುಟ್ಟು-ತ್ತಾರೆ
ಹುಟ್ಟು-ತ್ತಿರು-ವು-ದ-ರಿಂದ
ಹುಟ್ಟುವ
ಹುಟ್ಟು-ವಂತೆ
ಹುಟ್ಟು-ವಂಥದೂ
ಹುಟ್ಟು-ವನು
ಹುಟ್ಟು-ವರು
ಹುಟ್ಟು-ವು-ದಕ್ಕೆ
ಹುಟ್ಟು-ವು-ದಿಲ್ಲ
ಹುಟ್ಟು-ವುದು
ಹುಟ್ಟು-ವುವು
ಹುಡಕುತ್ತಾನೆ
ಹುಡು-ಕಲಾ-ಯಿತು
ಹುಡು-ಕಲು
ಹುಡುಕಿ
ಹುಡುಕಿ-ಕೊಂಡು
ಹುಡುಕಿ-ದರೆ
ಹುಡುಕಿ-ದುದು
ಹುಡುಕಿ-ರು-ವರು
ಹುಡುಕಿ-ರು-ವೆನು
ಹುಡುಕುತ್ತೇವೆ
ಹುಡುಕು-ವರು
ಹುಡುಕು-ವುದು
ಹುಡುಗ
ಹುಡುಗನ
ಹುಡುಗ-ನಂತೆ
ಹುಡುಗ-ನನ್ನೂ
ಹುಡುಗ-ನಾಗಿ-ದ್ದಾಗ
ಹುಡುಗನೂ
ಹುಡುಗ-ರಿಗೂ
ಹುಡುಗರು
ಹುಡುಗರೆ
ಹುಡುಗ-ರೆಲ್ಲ
ಹುಡುಗಾಟಿ-ಕೆಯ
ಹುದುಗಿ-ಕೊಳ್ಳು-ವುದಕ್ಕಾದರೂ
ಹುದು-ಗಿದೆ
ಹುದು-ಗಿದ್ದರೆ
ಹುದು-ಗಿರ-ಬಹುದು
ಹುದುಗಿ-ರುವ
ಹುರಿದುಂಬಿಸಿದ
ಹುರಿದುಂಬಿ-ಸು-ತ್ತಿದೆ
ಹುರುಳಿಲ್ಲ
ಹುರುಳಿಲ್ಲ-ವೆಂಬು-ದನ್ನು
ಹುಲಿ
ಹುಲ್ಲುಕಡ್ಡಿಗೆ
ಹುಳುಕು-ಗಳು
ಹುಸಿ-ಯಾಗಿ-ದೆಯೇ
ಹುಸಿ-ಯಾಗಿಲ್ಲ
ಹುಸಿ-ಯಾಗು-ವಂತೆ
ಹುಸಿ-ಯಾಗು-ವುದೂ
ಹೂ
ಹೂವಿನ
ಹೂವು-ಗಳನ್ನು
ಹೃತ್ಪೂರ್ವಕ
ಹೃತ್ಪೂರ್ವ-ಕ-ವಾಗಿ
ಹೃತ್ಪೂರ್ವ-ಕ-ವಾಗಿಯೂ
ಹೃತ್ಪೂರ್ವ-ಕ-ವಾದ
ಹೃದಯ
ಹೃದ-ಯಕ್ಕೆ
ಹೃದಯ-ಗಳ
ಹೃದಯ-ಗಳಲ್ಲಿ
ಹೃದಯ-ಗಳು
ಹೃದಯ-ಚಲ-ನೆ-ಯೊಂದಿಗೆ
ಹೃದಯದ
ಹೃದಯ-ದಲ್ಲಿ
ಹೃದಯ-ದಲ್ಲಿಯೂ
ಹೃದಯ-ದಲ್ಲಿ-ರುವ
ಹೃದಯ-ದಲ್ಲಿ-ರುವು-ದನ್ನೇ
ಹೃದಯ-ದಲ್ಲೇ
ಹೃದಯ-ದಿಂದ
ಹೃದಯ-ದಿಂದಲೇ
ಹೃದಯ-ದೊಂದಿಗೆ
ಹೃದಯ-ವಂತರು
ಹೃದಯ-ವಂತಿಕೆ
ಹೃದಯ-ವನ್ನು
ಹೃದಯ-ವಿತ್ತು
ಹೃದಯ-ವಿ-ರ-ಲಿಲ್ಲ-ವೆಂದು
ಹೃದ-ಯವು
ಹೃದಯ-ವು-ಳ್ಳ-ವನು
ಹೃದಯ-ವೆಂದರೆ
ಹೃದ-ಯವೇ
ಹೃದಯ-ಸ್ಪರ್ಶಿ
ಹೃದಯಾಂತರಾಳ-ದಲ್ಲಿ
ಹೆಂಗ-ಸರ
ಹೆಂಗಸ-ರಿಗೆ
ಹೆಂಗ-ಸರು
ಹೆಂಗಸರೇ
ಹೆಂಗಸಾ-ಗಲಿ
ಹೆಂಗ-ಸಿಗೂ
ಹೆಂಗಸೊಬ್ಬಳು
ಹೆಂಡತಿ
ಹೆಂಡಿರು
ಹೆಗಲಿನ
ಹೆಗ-ಲ್
ಹೆಗಲ್ನ
ಹೆಗ್ಗುರುತು
ಹೆಚ್ಚಾಗಿ
ಹೆಚ್ಚಾ-ಗಿದೆ
ಹೆಚ್ಚಾಗಿ-ರುವ
ಹೆಚ್ಚಾಗಿ-ರು-ವುದು
ಹೆಚ್ಚಾದ
ಹೆಚ್ಚಾ-ಯಿತು
ಹೆಚ್ಚಿ
ಹೆಚ್ಚಿಗೆ
ಹೆಚ್ಚಿನ
ಹೆಚ್ಚಿ-ನ-ದನ್ನು
ಹೆಚ್ಚಿ-ಸ-ಬಹುದು
ಹೆಚ್ಚಿ-ಸುತ್ತಿ-ದ್ದಾರೆ
ಹೆಚ್ಚು
ಹೆಚ್ಚು-ತ್ತದೆ
ಹೆಚ್ಚು-ವಂತೆ
ಹೆಚ್ಚು-ವುದು
ಹೆಚ್ಚು-ವುವು
ಹೆಚ್ಚೇ
ಹೆಚ್ಚೇನು
ಹೆಜ್ಜೆ
ಹೆಜ್ಜೆ-ಗಳು
ಹೆಜ್ಜೆಗೂ
ಹೆಜ್ಜೆ-ಯನ್ನು
ಹೆಜ್ಜೆ-ಯಿಟ್ಟಿ-ರುವ
ಹೆಜ್ಜೆ-ಯಿಡಿಸಿ
ಹೆಜ್ಜೆಯೂ
ಹೆಜ್ಜೆಯೇ
ಹೆಡೆ-ಗಳನ್ನೂ
ಹೆಡೆ-ಗಳಲ್ಲಿ
ಹೆಡೆ-ಯಲ್ಲಿ-ರುವ
ಹೆಣಗಾಡುತ್ತಿ-ರು-ವರು
ಹೆಣೆಯಲ್ಪಟ್ಟಂತೆ
ಹೆಣ್ಣು
ಹೆದರಿ
ಹೆದರಿಕೆ
ಹೆದರಿ-ಸಿದರು
ಹೆದರಿ-ಸು-ವುವು
ಹೆಮ್ಮರ-ವಾಗು-ತ್ತದೆ-ಯ-ಲ್ಲದೆ
ಹೆಮ್ಮೆ
ಹೆಮ್ಮೆ-ಪಡ-ಬಹು-ದಾದ
ಹೆಮ್ಮೆ-ಪಡಿ
ಹೆಮ್ಮೆ-ಪಡು-ವರು
ಹೆಮ್ಮೆ-ಪ-ಡು-ವು-ದಕ್ಕೆ
ಹೆಮ್ಮೆ-ಪಡು-ವುದು
ಹೆಮ್ಮೆಯ
ಹೆಮ್ಮೆ-ಯನ್ನು
ಹೆಮ್ಮೆ-ಯಿಂದ
ಹೆಮ್ಮೆ-ಯಿಂದಲೂ
ಹೆಮ್ಮೆ-ಯುಂಟು
ಹೆಮ್ಮೆಯೇ
ಹೆಳುತ್ತಾರೆ
ಹೆಸರ-ನ-ಲ್ಲಿಯೂ
ಹೆಸ-ರನ್ನು
ಹೆಸ-ರನ್ನೂ
ಹೆಸರ-ನ್ನೋ
ಹೆಸ-ರಿಗೆ
ಹೆಸ-ರಿನ
ಹೆಸರಿ-ನಲ್ಲಿ
ಹೆಸರಿ-ನ-ಲ್ಲಿ-ರುವ
ಹೆಸರಿ-ನಲ್ಲೇ
ಹೆಸರಿ-ನ-ಲ್ಲೇ-ನಿದೆ
ಹೆಸರಿ-ನಿಂದ
ಹೆಸರಿ-ಸಲ್ಪಟ್ಟ-ವ-ರೆಲ್ಲ
ಹೆಸರು
ಹೆಸರು-ಗಳ
ಹೆಸರು-ಗಳನ್ನು
ಹೆಸರು-ಗಳಲ್ಲ
ಹೆಸರು-ಗಳ-ಲ್ಲದೆ
ಹೆಸರು-ಗಳಿಂದ
ಹೆಸರು-ಗಳು
ಹೆಸರು-ಗಳೂ
ಹೆಸರೂ
ಹೇ
ಹೇಗಾ-ದರೂ
ಹೇಗಾದೀತು
ಹೇಗಿ-ದೆಯೋ
ಹೇಗೆ
ಹೇಗೊ
ಹೇಗೋ
ಹೇಡಿ-ಗಳಂತೆ
ಹೇಡಿ-ಗಳ-ನ್ನಾಗಿ
ಹೇಡಿ-ಗಳಾಗಿ-ರು-ವಾಗ
ಹೇಡಿ-ಗಳೆಂದು
ಹೇಡಿ-ತನ-ವನ್ನು
ಹೇಡಿಯ
ಹೇಡಿ-ಯ-ನ್ನಾಗಿ
ಹೇಡಿ-ಯಾಗು-ವು-ದಕ್ಕೆ
ಹೇರಲು
ಹೇರಳ-ವಾಗಿವೆ
ಹೇರಾಸೆ
ಹೇರಿ-ದ್ದಾರೆ
ಹೇರುವ
ಹೇಳ-ಕೂಡದು
ಹೇಳದ
ಹೇಳದೆ
ಹೇಳನ
ಹೇಳ-ಬಯಸು-ತ್ತೇನೆ
ಹೇಳ-ಬಲ್ಲ
ಹೇಳ-ಬಲ್ಲರೇ
ಹೇಳ-ಬಲ್ಲೆ
ಹೇಳ-ಬಹುದು
ಹೇಳ-ಬಾ-ರದು
ಹೇಳ-ಬೇಕಾಗಿದೆ
ಹೇಳ-ಬೇಕಾಗಿಯೂ
ಹೇಳ-ಬೇಕಾಗಿಲ್ಲ
ಹೇಳ-ಬೇ-ಕಾದ
ಹೇಳ-ಬೇ-ಕಾದರೆ
ಹೇಳ-ಬೇಕಾ-ಯಿತು
ಹೇಳ-ಬೇಕು
ಹೇಳಬೇಕೆಂದರೆ
ಹೇಳ-ಬೇಕೆಂದಿರು-ವೆನು
ಹೇಳ-ಬೇಕೆಂದು
ಹೇಳ-ಬೇಕೆಂಬ
ಹೇಳಬೇಕೆಂಬುದು
ಹೇಳ-ಬೇಡಿ
ಹೇಳ-ಲಾ-ಗಿದೆಯೋ
ಹೇಳ-ಲಾ-ಗು-ವು-ದಿಲ್ಲ
ಹೇಳ-ಲಾ-ಯಿತು
ಹೇಳಲಾರ
ಹೇಳಲಾರ-ದ್ದೊಂದು
ಹೇಳ-ಲಾರೆ
ಹೇಳ-ಲಾರೆವು
ಹೇಳಲಿ
ಹೇಳಲಿ-ಚ್ಛಿ-ಸು-ವಂತೆ
ಹೇಳ-ಲಿಲ್ಲ
ಹೇಳಲು
ಹೇಳಲೇ-ಬೇಕಾಗಿಲ್ಲ
ಹೇಳಲೇ-ಬೇಕು
ಹೇಳಲ್ಪಟ್ಟ-ವು-ಗಳಲ್ಲ
ಹೇಳಿ
ಹೇಳಿಕೆ
ಹೇಳಿ-ಕೆ-ಗಳ
ಹೇಳಿ-ಕೆ-ಗಳ-ನ್ನೆಲ್ಲ
ಹೇಳಿ-ಕೆ-ಗಳು
ಹೇಳಿ-ಕೆಯ
ಹೇಳಿ-ಕೆ-ಯಲ್ಲಿ
ಹೇಳಿ-ಕೆ-ಯಷ್ಟು
ಹೇಳಿ-ಕೊಂಡಿ-ದ್ದರೆ
ಹೇಳಿ-ಕೊ-ಟ್ಟರೋ
ಹೇಳಿ-ಕೊಡ-ಬೇಕೆಂದು
ಹೇಳಿ-ಕೊಡಿ
ಹೇಳಿ-ಕೊಳ್ಳ-ಬಹುದು
ಹೇಳಿ-ಕೊಳ್ಳ-ಬಾ-ರದು
ಹೇಳಿ-ಕೊಳ್ಳ-ಬೇಡಿ
ಹೇಳಿ-ಕೊಳ್ಳಲಾರರು
ಹೇಳಿ-ಕೊಳ್ಳಲಿ
ಹೇಳಿ-ಕೊಳ್ಳಲು
ಹೇಳಿ-ಕೊಳ್ಳು
ಹೇಳಿ-ಕೊಳ್ಳುತ್ತ
ಹೇಳಿ-ಕೊಳ್ಳು-ತ್ತಾ-ನೆಯೇ
ಹೇಳಿ-ಕೊಳ್ಳು-ವು-ದಕ್ಕೆ
ಹೇಳಿ-ಕೊಳ್ಳು-ವು-ದನ್ನು
ಹೇಳಿ-ಕೊಳ್ಳು-ವು-ದ-ರಿಂದ
ಹೇಳಿ-ಕೊಳ್ಳು-ವುದೇ
ಹೇಳಿದ
ಹೇಳಿ-ದಂತಹ
ಹೇಳಿ-ದಂತೆ
ಹೇಳಿ-ದನು
ಹೇಳಿ-ದ-ನೆಂತಲೇ
ಹೇಳಿ-ದರು
ಹೇಳಿ-ದರೂ
ಹೇಳಿ-ದರೆ
ಹೇಳಿ-ದಳು
ಹೇಳಿ-ದ-ವರು
ಹೇಳಿ-ದಾಗ
ಹೇಳಿ-ದು-ದನ್ನು
ಹೇಳಿ-ದುದನ್ನೆಲ್ಲ
ಹೇಳಿ-ದುದೆಲ್ಲ
ಹೇಳಿ-ದುದೇ
ಹೇಳಿದೆ
ಹೇಳಿ-ದೆನು
ಹೇಳಿ-ದೆಯೇ
ಹೇಳಿ-ದ್ದನ್ನು
ಹೇಳಿ-ದ್ದರು
ಹೇಳಿ-ದ್ದರೆ
ಹೇಳಿ-ದ್ದಾನೆ
ಹೇಳಿ-ದ್ದಾ-ಯಿತು
ಹೇಳಿ-ದ್ದಾರೆ
ಹೇಳಿಯೂ
ಹೇಳಿ-ರುತ್ತಾರೆ
ಹೇಳಿ-ರುವ
ಹೇಳಿ-ರು-ವಂತೆ
ಹೇಳಿ-ರು-ವನು
ಹೇಳಿ-ರು-ವರು
ಹೇಳಿ-ರು-ವುದು
ಹೇಳಿ-ರು-ವೆನು
ಹೇಳಿಲ್ಲ
ಹೇಳುತ್ತ
ಹೇಳು-ತ್ತದೆ
ಹೇಳುತ್ತ-ದೇನು
ಹೇಳು-ತ್ತವೆ
ಹೇಳುತ್ತ-ವೆಯೋ
ಹೇಳುತ್ತಾನೆ
ಹೇಳುತ್ತಾರೆ
ಹೇಳುತ್ತಿತ್ತು
ಹೇಳು-ತ್ತಿದೆ
ಹೇಳು-ತ್ತಿದ್ದ
ಹೇಳು-ತ್ತಿದ್ದರು
ಹೇಳು-ತ್ತಿದ್ದೆ
ಹೇಳುತ್ತಿ-ದ್ದೇನೆ
ಹೇಳು-ತ್ತಿ-ರು-ವಂತೆ
ಹೇಳುತ್ತಿ-ರು-ವರು
ಹೇಳುತ್ತಿರು-ವು-ದ-ರಿಂದ
ಹೇಳುತ್ತಿ-ರುವು-ದಾಗಿ
ಹೇಳುತ್ತಿರು-ವುದು
ಹೇಳುತ್ತಿರು-ವೆನು
ಹೇಳು-ತ್ತಿಲ್ಲ
ಹೇಳು-ತ್ತಿವೆ
ಹೇಳು-ತ್ತೀರಿ
ಹೇಳು-ತ್ತೇನೆ
ಹೇಳುತ್ತೇವೆ
ಹೇಳುವ
ಹೇಳು-ವಂತ-ವ-ರಲ್ಲ
ಹೇಳು-ವಂತೆ
ಹೇಳು-ವನು
ಹೇಳು-ವರು
ಹೇಳುವ-ವನ
ಹೇಳುವ-ವ-ರಿಲ್ಲ
ಹೇಳುವ-ವರು
ಹೇಳು-ವವು
ಹೇಳು-ವಾಗಲೂ
ಹೇಳು-ವು-ದಕ್ಕೆ
ಹೇಳು-ವು-ದನ್ನು
ಹೇಳುವುದನ್ನೆಲ್ಲ
ಹೇಳು-ವು-ದರ
ಹೇಳು-ವು-ದಾದರೆ
ಹೇಳು-ವು-ದಿಲ್ಲ
ಹೇಳು-ವುದು
ಹೇಳು-ವುದು-ದಿಲ್ಲ
ಹೇಳು-ವುದೇ-ನಿದೆ
ಹೇಳು-ವುದೇ-ನೆಂದರೆ
ಹೇಳು-ವುದೇ-ನೆಂದರೆ-ನಮ್ಮ
ಹೇಳು-ವುದೇನೋ
ಹೇಳು-ವುವು
ಹೇಳು-ವೆನು
ಹೇಳು-ವೆವು
ಹೇಳು-ವೆವೆ
ಹೇಳೋಣ
ಹೇಸು-ವು-ದಿಲ್ಲ
ಹೈದ-ಯವೂ
ಹೊಂದದ
ಹೊಂದ-ಬಹುದು
ಹೊಂದ-ಬಹು-ದೆಂಬು-ದನ್ನು
ಹೊಂದ-ಬೇಕಾಗಿದೆ
ಹೊಂದ-ಲಾ-ರದೆ
ಹೊಂದಲೇ
ಹೊಂದಿ
ಹೊಂದಿಕೆ-ಯಾಗ-ದಂತೆ
ಹೊಂದಿಕೆ-ಯಾಗಿ-ರು-ವುದು
ಹೊಂದಿಕೆ-ಯಾಗು-ವಂತೆ
ಹೊಂದಿ-ಕೊಂಡು
ಹೊಂದಿ-ಕೊಳ್ಳ-ದಾಗ
ಹೊಂದಿ-ಕೊಳ್ಳು-ತ್ತದೆ
ಹೊಂದಿ-ಕೊಳ್ಳು-ವುವೋ
ಹೊಂದಿದ
ಹೊಂದಿ-ದವು
ಹೊಂದಿದೆ
ಹೊಂದಿ-ದ್ದೀರಿ
ಹೊಂದಿಯೇ
ಹೊಂದಿ-ರ-ಲಿಲ್ಲ
ಹೊಂದಿ-ರುವ
ಹೊಂದಿಲ್ಲ
ಹೊಂದಿಸಿ-ಕೊಂಡು
ಹೊಂದಿಸಿ-ಕೊಳ್ಳ-ಬೇಕಾಗಿದೆ
ಹೊಂದು-ತ್ತದೆ
ಹೊಂದುತ್ತಾನೆ
ಹೊಂದು-ತ್ತೀರಿ
ಹೊಂದುವ
ಹೊಂದು-ವರು
ಹೊಂದುವಿ-ಕೆಯೂ
ಹೊಂದು-ವುವು
ಹೊಕ್ಕಾಗಿದೆ
ಹೊಕ್ಕಿತ್ತು-ಎಂದಿದೆ
ಹೊಕ್ಕು
ಹೊಗಳ-ಬಹುದು
ಹೊಗಳ-ಬೇಕಾಗಿಲ್ಲ
ಹೊಗಳ-ಬೇಕು
ಹೊಗಳಲಿ
ಹೊಗಳುತ್ತಾ
ಹೊಗಳು-ವನು
ಹೊಗಳು-ವರು
ಹೊಗಳು-ವುದೇಕೆ
ಹೊಟ್ಟಿನ
ಹೊಟ್ಟೆ
ಹೊಟ್ಟೆ-ಗಿಲ್ಲ-ದಿದ್ದರೂ
ಹೊಟ್ಟೆ-ಗಿಲ್ಲದೆ
ಹೊಟ್ಟೆಯ
ಹೊಡೆತ
ಹೊಡೆದಾಟ-ಗಳನ್ನು
ಹೊಡೆದು
ಹೊಡೆದು-ಕೊಂಡು
ಹೊಣೆ
ಹೊಣೆ-ಗಾರ
ಹೊತ್ತಗೆ
ಹೊತ್ತ-ವನು
ಹೊತ್ತಿಗೆ
ಹೊತ್ತಿಗೆ-ಯಲ್ಲಿ
ಹೊತ್ತಿನ
ಹೊತ್ತಿರು-ತ್ತದೆ
ಹೊತ್ತಿ-ರು-ವರು
ಹೊತ್ತು
ಹೊತ್ತು-ಕೊಂಡು
ಹೊತ್ತು-ಕೊಳ್ಳಿ
ಹೊದಿಸ-ಬೇಡಿ
ಹೊದ್ದು-ಕೊಂಡ
ಹೊನ್ನಿನ
ಹೊಮ್ಮಿ
ಹೊಮ್ಮಿತೋ
ಹೊಮ್ಮಿ-ದವು
ಹೊಯ್ದ-ವ-ರಾರು
ಹೊರ
ಹೊರಕ್ಕೆ
ಹೊರ-ಗಡೆ
ಹೊರ-ಗಡೆಯೆ
ಹೊರ-ಗಾ-ಗಲೀ
ಹೊರ-ಗಿನ
ಹೊರ-ಗಿನದು
ಹೊರ-ಗಿನ-ವರ
ಹೊರ-ಗಿನ-ವ-ರಿಗೆ
ಹೊರ-ಗಿನ-ವರು
ಹೊರ-ಗಿ-ನಿಂದ
ಹೊರ-ಗಿ-ನಿಂದಲೂ
ಹೊರ-ಗಿ-ರುವ
ಹೊರ-ಗಿಲ್ಲ
ಹೊರಗೂ
ಹೊರಗೆ
ಹೊರ-ಗೆ-ಡವ-ಬಹುದೋ
ಹೊರ-ಗೆ-ಡಹಲು
ಹೊರ-ಗೆಲ್ಲಾ
ಹೊರ-ಚಾಚಿ-ದುದು
ಹೊರ-ಜಗತ್ತು
ಹೊರಟ
ಹೊರ-ಟನು
ಹೊರ-ಟರು
ಹೊರ-ಟರೆ
ಹೊರ-ಟಾಗ
ಹೊರ-ಟಿ-ದ್ದರೆ
ಹೊರ-ಟಿ-ದ್ದೀರಿ
ಹೊರ-ಟಿವೆ
ಹೊರಟು
ಹೊರ-ಟು-ಹೋ-ಗ-ಬೇಕು
ಹೊರ-ಟು-ಹೋಗಿ
ಹೊರ-ಟು-ಹೋ-ಗಿದೆ
ಹೊರ-ಟು-ಹೋ-ಗು-ತ್ತವೆ
ಹೊರಟು-ಹೋ-ಗುವ
ಹೊರ-ಡ-ಬೇಕಾ-ಗು-ತ್ತದೆ
ಹೊರಡಿ
ಹೊರ-ತಾ-ಗಿದೆ
ಹೊರತು
ಹೊರ-ದೂಡ-ಬೇಕು
ಹೊರ-ದೇಶ-ದಿಂದ
ಹೊರ-ಬಂದವು
ಹೊರ-ಬ-ರಲು
ಹೊರ-ಬ-ರುವ
ಹೊರ-ಬಿದ್ದಿ-ದ್ದರೆ
ಹೊರ-ಬಿದ್ದಿವೆ
ಹೊರ-ಬಿದ್ದು
ಹೊರ-ಬೀಳು-ವುವು
ಹೊರ-ಳಾಡು-ವುದೇ
ಹೊರಳಿ
ಹೊರ-ಹೊಮ್ಮಿದೆ
ಹೊರ-ಹೊಮ್ಮು-ತ್ತದೆ
ಹೊರ-ಹೊಮ್ಮುವು-ದ-ರಲ್ಲಿ
ಹೊರ-ಹೊಮ್ಮು-ವುದು
ಹೊರಿಸ-ಲಾ-ಗು-ವು-ದಿಲ್ಲ
ಹೊರುತು
ಹೊರುವ
ಹೊರೆ
ಹೊರೆ-ಯ-ಬೇಕಾ-ಗು-ತ್ತದೆ
ಹೊರೆ-ಯಷ್ಟು
ಹೊರೆ-ಸಿ-ದ್ದಾರೆ
ಹೊಲ-ವನ್ನು
ಹೊಲವು
ಹೊಲಿಯ-ಬಲ್ಲಿರಾ
ಹೊಲಿಯುವು-ದ-ರಲ್ಲಿ
ಹೊಲಿಯು-ವು-ದೊಂದೇ
ಹೊಲೆಯ-ನಾ-ದರೂ
ಹೊಲೆಯ-ನಿ-ಗಿಂತ
ಹೊಳೆಯಲೇ
ಹೊಳೆ-ಯಿತು
ಹೊಳೆ-ಯು-ತ್ತಿದೆ
ಹೊಳೆಯುವ
ಹೊಳೆಯು-ವುದು
ಹೊಸ
ಹೊಸ-ತಾಗಿ
ಹೊಸ-ದನ್ನು
ಹೊಸ-ದಲ್ಲ
ಹೊಸ-ದಾಗಿ
ಹೊಸ-ದಾಗಿ-ತ್ತು
ಹೊಸ-ದಾದ್ದು
ಹೊಸ-ದೊಂದು
ಹೊಸ-ಬರು
ಹೊಸ-ಬೆಳಕನ್ನು
ಹೊಸ-ಯು-ಗದ
ಹೊಸ-ರೀತಿ-ಯಲ್ಲಿ
ಹೊಸ-ಹಾ-ದಿ-ಯನ್ನು
ಹೋ
ಹೋಗ
ಹೋಗ-ಕೂಡದು
ಹೋಗ-ದಂತೆ
ಹೋಗ-ದ-ವರು
ಹೋಗ-ಬಲ್ಲ
ಹೋಗ-ಬಲ್ಲೆ-ನೆಂದು
ಹೋಗ-ಬಹುದು
ಹೋಗ-ಬೇಕಾಗಿ
ಹೋಗ-ಬೇಕಾಗಿ-ತ್ತು
ಹೋಗ-ಬೇಕಾಗಿದೆ
ಹೋಗ-ಬೇ-ಕಾದ
ಹೋಗ-ಬೇಕಾ-ಯಿತು
ಹೋಗ-ಬೇಕು
ಹೋಗ-ಬೇಕೆಂಬ
ಹೋಗ-ಬೇಕೆಂಬು-ದನ್ನು
ಹೋಗ-ಬೇಡ
ಹೋಗ-ಬೇಡಿ
ಹೋಗ-ಲಾ-ಗ-ಲಿಲ್ಲ
ಹೋಗ-ಲಾಡಿಸ-ಬೇಕು
ಹೋಗ-ಲಾಡಿಸಲಾ-ರವು
ಹೋಗ-ಲಾಡಿಸಿ
ಹೋಗ-ಲಾಡಿಸಿ-ದ್ದೀರಿ
ಹೋಗ-ಲಾಡಿ-ಸು-ವುದೆ
ಹೋಗ-ಲಾ-ರದು
ಹೋಗ-ಲಾರೆವು
ಹೋಗಲಿ
ಹೋಗ-ಲಿ-ಚ್ಛಿ-ಸುವ-ವ-ರನ್ನು
ಹೋಗ-ಲಿಲ್ಲ
ಹೋಗ-ಲಿ-ಲ್ಲ-ವೆಂದು
ಹೋಗಲು
ಹೋಗಲೇ
ಹೋಗ-ಲೇ-ಬೇಕಾ-ಯಿತು
ಹೋಗಿ
ಹೋಗಿದೆ
ಹೋಗಿ-ದ್ದನು
ಹೋಗಿ-ದ್ದರೆ
ಹೋಗಿ-ದ್ದಾರೆ
ಹೋಗಿ-ದ್ದೇವೆ
ಹೋಗಿ-ಭಿಕ್ಷು-ಕ-ರಂತಲ್ಲ
ಹೋಗಿ-ರ-ಬಹುದು
ಹೋಗಿ-ರುವ
ಹೋಗಿ-ರು-ವಂತೆ
ಹೋಗಿ-ರುವ-ರಲ್ಲ
ಹೋಗಿ-ರು-ವರು
ಹೋಗಿ-ರು-ವರೋ
ಹೋಗಿ-ರು-ವುವು
ಹೋಗಿ-ರು-ವೆವು
ಹೋಗಿಲ್ಲ
ಹೋಗಿವೆ
ಹೋಗು
ಹೋಗು-ತ್ತದೆ
ಹೋಗು-ತ್ತವೆ
ಹೋಗು-ತ್ತಾನೆ
ಹೋಗು-ತ್ತಾರೆ
ಹೋಗು-ತ್ತಾ-ರೆಂದು
ಹೋಗು-ತ್ತಿದೆ
ಹೋಗು-ತ್ತಿದ್ದರು
ಹೋಗು-ತ್ತಿ-ದ್ದಾಗ
ಹೋಗು-ತ್ತಿ-ದ್ದೇವೆ
ಹೋಗು-ತ್ತಿ-ರಲಿಲ್ಲ
ಹೋಗು-ತ್ತಿರು-ತ್ತದೆ
ಹೋಗು-ತ್ತಿ-ರುವ
ಹೋಗು-ತ್ತಿ-ರುವಳು
ಹೋಗು-ತ್ತಿರು-ವಾಗ
ಹೋಗು-ತ್ತಿರು-ವುವು
ಹೋಗು-ತ್ತಿ-ರುವೆ
ಹೋಗು-ತ್ತಿವೆ
ಹೋಗು-ತ್ತೇನೆ
ಹೋಗು-ತ್ತೇವೆ
ಹೋಗುವ
ಹೋಗು-ವಂತಹ
ಹೋಗು-ವಂತೆ
ಹೋಗು-ವನು
ಹೋಗು-ವರು
ಹೋಗು-ವರೋ
ಹೋಗು-ವ-ವನ-ನ್ನೇ
ಹೋಗು-ವಾಗ
ಹೋಗು-ವಿರಿ
ಹೋಗು-ವು-ದಕ್ಕೂ
ಹೋಗು-ವು-ದಕ್ಕೆ
ಹೋಗು-ವು-ದನ್ನು
ಹೋಗು-ವು-ದ-ರಿಂದ
ಹೋಗು-ವು-ದಲ್ಲ
ಹೋಗು-ವುದಾ-ಗಲಿ
ಹೋಗು-ವು-ದಿಲ್ಲ
ಹೋಗು-ವುದು
ಹೋಗು-ವುದೆಂದ-ರೇನು
ಹೋಗು-ವು-ದೆಂದು
ಹೋಗು-ವುದೇ
ಹೋಗು-ವುದೋ
ಹೋಗು-ವುವು
ಹೋಗು-ವೆನು
ಹೋಗು-ವೆವು
ಹೋಗೋಣ
ಹೋಟೆಲಿ
ಹೋದ
ಹೋದಂತೆ
ಹೋದದ್ದು
ಹೋದ-ನಂತೆ
ಹೋದರು
ಹೋದರೂ
ಹೋದರೆ
ಹೋದ-ರೇ-ನಂತೆ
ಹೋದ-ವನೊ-ಬ್ಬ-ನನ್ನು
ಹೋದವು
ಹೋದಾಗ
ಹೋದುದು
ಹೋದುವು
ಹೋದೆ
ಹೋಮರ್
ಹೋಯಿತು
ಹೋರಾಟ
ಹೋರಾ-ಟಕ್ಕೆ
ಹೋರಾಟ-ಗಳನ್ನು
ಹೋರಾಟ-ಗಳು
ಹೋರಾಟ-ಗಳೂ
ಹೋರಾಟ-ದಲ್ಲಿ
ಹೋರಾಟ-ವನ್ನು
ಹೋರಾಟ-ವಾಗಿದೆ
ಹೋರಾಟ-ವಿತ್ತು
ಹೋರಾಟ-ವಿರ-ಕೂಡದು
ಹೋರಾಟವು
ಹೋರಾಡ-ಬೇಕಾಗಿಲ್ಲ
ಹೋರಾಡ-ಬೇಡಿ
ಹೋರಾಡಿ
ಹೋರಾ-ಡಿದ
ಹೋರಾಡಿ-ದಷ್ಟೂ
ಹೋರಾಡುತ್ತಿ-ರುವ
ಹೋರಾಡುತ್ತಿ-ರು-ವರು
ಹೋರಾ-ಡುವರೋ
ಹೋರಾಡು-ವಾಗ
ಹೋರಾ-ಡು-ವು-ದಕ್ಕೆ
ಹೋರಾಡು-ವುದು
ಹೋಲಿಕೆ
ಹೋಲಿ-ಕೆಗೆ
ಹೋಲಿಸಿ
ಹೋಲಿಸಿ-ದರು
ಹೋಲಿಸಿ-ದರೆ
ಹೋಲಿಸಿ-ದಷ್ಟೂ
ಹೋಲಿಸಿ-ದೆವು
ಹೋಲಿಸಿ-ನೋಡಿ
ಹೋಲಿಸು
ಹೋಲು-ವು-ದಿಲ್ಲ
ಹೋಲು-ವುದು
ಹೌದಾದರೆ
ಹೌದು
ಹ್
ಹ್ರಸ್ವ-ವಾದ
}
