
\chapter*{Volume Editorial}

\begin{myquote}
\textit{“I fear the Greeks even when they bring gifts”}
\end{myquote}


~\hfill – Virgil

\begin{myquote}
\textit{“That meddling in other people’s affairs... is now openly advocated under the name of intervention”}
\end{myquote}


~\hfill – T S Eliot

\begin{myquote}
\textit{“Civilised men arrive in the Pacific armed with alcohol, syphilis, trousers, and the Bible”}
\end{myquote}


~\hfill – Havelock Ellis

\begin{verse}
\textit{“O What a tangled web we weave\\ When first we practise to deceive!\\ But when we’ve practised for a while\\ How vastly we improve our style!!”}
\end{verse}


~\hfill – Walter Scott

\begin{myquote}
“But Lord! To see the absurd nature of Englishmen that cannot forbear laughing and jeering at everything that looks strange.” –
\end{myquote}


~\hfill - Samuel Pepys

This volume, being the fifth in the Proceedings of the Swadeshi Indology Conference Series, deals with various issues. This is somewhat in contrast with the previous volumes which had major single issues. Issues pertaining to Mīmāṁsā and desacralisation form the bulk here. While four papers pertain to the discipline of Mīmāṁsā, two pertain to the problem of desacralisation. Three miscellaneous papers on Philology, the \textit{Rāmāyaṇa}, and the \textit{śāstra}-s also figure here. Over half a dozen authors, ranging from the very old to the very young, have contributed the papers. One of the papers is in Sanskrit (as in the previous volume), and one in Hindi. An overview of the papers is desirable in this prefatory portion. (For the Hindi and Sanskrit papers, brief overviews are provided in Hindi and Sanskrit respectively as well).

The opening paper entitled “History in India: a Critique from the Perspective of Mīmāṁsā” (Ch.1) is authored by Prof. \textbf{Shrinivas Tilak}, a veteran scholar in Sanskrit and Indian Philosophy. The paper begins with the signal warning provided by George Orwell viz. “He who controls the past controls the future. He who controls the present controls the past.”, which tells it all about the very need of the enterprise of Swadeshi Indology: the West, through its own brand of Indology is all out to take full control of the past of India, and towards what ends it remains best unsaid. 

The Indian perspective on history is not in alignment with that of the West, and is by no means obliged to be. Macdonell squarely blames the theory of \textit{karman} which nullified all initiative to keep track of historical events. His skepticism, so typical of the jaundiced West, is well-reflected in his smug and cynical dictum that early India wrote no history because it never made any. From Macdonell to Pollock, it is only a more ornate and sophisticated contempt that one encounters. Prof. Pollock has, especially of late, emerged, as it were, Tilak notes, the very "guardian of India’s cultural, literary and social past”. Prof. Pollock does not come across by any means as an innocuous and inoffensive scholar, one merely curious about India’s past. His brand of Orientalism dons newer jargons and spews more polished garbage empowered enough “to influence public policy in India and project its image to the world.”

The paper sets an excellent model of the tripod on which Swadeshi Indology would do well to be erected viz.

\begin{itemize}
\item \textit{pūrvapakṣa}

 \item \textit{uttarapakṣa} and 

 \item \textit{siddhānta}

\end{itemize}

The three respectively stand for (a) a factual presentation of the opponent’s thesis; (b) a critical examination and refutation of the thesis; and (c) a statement on the outcome of the exchange. 

As Pollock notes too, history as a discipline is essentially a product of Western scholarship and ideas. Historical and historiographical awareness was not absent in India prior to the advent of the Western invaders. 

For Pollock, the Pūrva Mīmāṁsā declaration of the Veda-s as timeless and authorless (leading to what he labels as Vedicization) deprived Indic texts of their historicality. The importance attached to Mīmāṁsā is because of its stature and role as a pedagogically and culturally normative discipline of Brahminical learning. The ritual discourse that this discipline was, turned into a discourse of social power, and a scheme of domination that has bedevilled Indian society for over two millennia. 

It was in order to maintain their infallibility that the Veda-s were declared, Pollock argues, as authorless (\textit{apauruṣeya}), hence timeless, hence immune from historicality. It was Mīmāṁsā that divested the Veda-s of all historical consciousness, and even of historical referential intention, which came to serve as the archetype for all the later literary production as well. A lack of historical referentiality was henceforth professed as well. The brazen statements of Pollock are surpassed by Fisher for whom Mīmāṁsā is an epistemologically violent enterprise, not a hermeneutical one: it cannot be claimed that Kumārila understood the Veda-s any better than Max Mueller, and the latter denounced the mythological excursions in the Veda-s as but “a disease of language”. 

The cyclic concept of time, the regressive theory of \textit{yuga} prevalent in India, negated, says Pollock, the difference between myth and history; the notion of \textit{mokṣa} transcends and even denies history, and is life-negating; the emphasis on \textit{ākrṭi} as against \textit{vyakti} in Mīmāṁsā is symptomatic of the general attitude of the Vedic as against that of the Itihāsic; it was Mīmāṁsā that led to modes of domination such as caste hierarchy, untouchability, and female heteronomy. This constitutes the \textit{pūrvapakṣa}.

 Prof. Tilak commences his \textit{uttarapakṣa} with a reference to Pollock’s own acknowledgement - that the norm of Indologists has been to generalise Western experience as a \textit{scientific} description of Eastern lifestyles: the deployment of Western tools in grasping the East thus constitutes a most serious cultural impediment. It is an irony, but more correctly an atrocity, that Pollock himself is impelled by a blind Euro-centrism. This rank prejudice inhibits Pollock from showing what the interpretive protocols of Mīmāṁsā intellectuals were in their own comprehension of the Vedic. It is Western branding that Pollock practises when he pronounces texts as mystical or literary, all impelled by Western criteria. The Indologist’s adversary stands in respect of the universality of cyclicity and non-contradiction between cyclicity and linearity have been strongly contested by Jan Houben and Romila Thapar though on different counts. 

Prof. Tilak takes a deep look into the apposite hermeneutical principles to lay bare the hollowness of the claims of Pollock. Events alluded to in the Vedic literature have a precise function – to serve as \textit{arthavāda}-s, set forth in order to illustrate the specific purpose of particular Vedic injunctions. Serving the role of the preamble of a statute, an \textit{arthavāda} has no legal force by itself, yet helps to clarify possible ambiguities in \textit{vidhi}-s or injunctions. \textit{Vidhi}-s can be couched as \textit{arthavāda}-s too. Comparable is the doctor’s prescription which invariably indicates his license number etc., ensuring thereby the validity of the prescription. 

Tilak brings out half a dozen Mīmāṁsā principles bearing on the sound criteria of interpretation. For Tilak, the Vedic episodes and the episodes in the \textit{Rāmāyaṇa} and the \textit{Mahābhārata}, are typically \textit{apauruṣeya} and \textit{pauruṣeya} types of \textit{Itihāsa}. As against Pollock’s pitching upon only one specific meaning of \textit{Itihāsa} (\textit{iti ha āsa} – "thus it was") as akin to history, Tilak shows how “\textit{Itihāsa} has a far richer, and wider-ranging and comprehensive meaning and purpose: as discernible in the sense occurring in early Upaniṣad-s. Tilak also alludes to the eight and ninth days of the ten-day Pāriplava Rite (a part of the Aśvamedha Yajña) wherein the \textit{Itihāsa} and \textit{Purāṇa}-s were recited. The words \textit{Itihāsa, Purāṇa} and \textit{Ākhyāna} are often used interchangeably. For Yāska (800 BCE), \textit{Itihāsa}-s impart the philosophy of life - with supporting reference to \textit{pāramparika-kathā}-s, relevant traditional narratives, and to \textit{dharma}, as enveloping both \textit{kratvartha} and \textit{puruṣārtha} (respectively, the performance of \textit{yajña}-s on the one hand, and the performance of duties prescribed for the four \textit{varṇa}-s and the four \textit{āśrama}-s on the other). This constitutes \textit{vedopabṛṁhaṇa} (as set forth by Manu and other writers of \textit{smṛti}-s) which repudiates and invalidates "the process of Vedicization, the cornerstone of Pollock’s thesis of the Mīmāṁsā denial and suppression of ‘history’ in ancient India”. 

The Hindu approach towards \textit{Itihāsa} has its parallel, if faint, even in Carlyle’s Heroes and Hero-worship. Pollock could have asked himself why Buddhists and Jains (who spurned the Veda-s) too attached little importance to Pollock’s concept of history which, as a discipline, is hardly a century old! Pollock’s intolerance of other approaches to history has little to commend itself.

The Dharmaśāstra-s recommend performance of selfless actions which links \textit{dharma} to \textit{mokṣa}, the temporal thus harmonised with the timeless – which all is grounded in classical Indian epistemology, grossly missed by Pollock, intentionally or otherwise. The relevance of the theory of \textit{Karman}, as set forth in the \textit{Gītā} and as an evolute of the Vedic \textit{karman}, is grossly ignored by Pollock. Tilak summons the discussions of Raimundo Pannikkar, Roy Perrett and Romila Thapar to indicate the flaws of Pollock’s’ adjudications. Tilak alludes to the magnificent role of Nīlakaṇṭha, the commentator of the \textit{Mahābhārata} in continuing the tradition of Mīmāṁsā, which would all be anathema to Pollock and Co. Tilak alludes to Minkowski’s verdict on Nīlakaṇṭha, nakedly exposing how tightly Western academics "control the exegesis of the Veda-s”. Tilak has no hesitation in cautioning Swadeshi interpreters of the Veda to be wary of the Pollockian and Minkowskian approaches.

Finally, as to the \textit{siddhānta}: That the “history” of Pollock is easily subsumable under the \textit{itihāsa} of the Hindu tradition altogether escapes Pollock's attention. Gadamer showed how pre-understanding prejudices a reader’s interpretation. It was Jan Gonda, among others, who noted how Indian civilisation stands in striking contrast to the Western. The taunt of Pollock’s thesis that “India is without history” is ably met with a counterpoise “So what? The West is without \textit{Itihāsa}”! Prof. Tilak’s treatment is almost mathematical in approach, and dignified in its conscious restraint. 

The fulminations of Pollock are thus without foundation, and bespeak of his own rank prejudice and sophisticated malice. He is an excellent model of how not to do criticism.

The second paper entitled “The Science and Nescience of Mīmāṁsā” (Ch.2) is by \textbf{T N Sudarshan}. It analyses the \textit{etic} interpretations of Mīmāṁsā focusing on fundamental issues of Western hermeneutics. Western epistemologies are ill-equipped to handle issues of Mīmāṁsā or allied Indian knowledge systems. Serious accusations by Pollock abound even in his earlier writings (of 1989, 2004 etc.). Misinterpretations and fantastic theorisations of the origins, motives, and goals of Mīmāṁsā have been a recurring feature among Western Indologists even prior to Pollock. 

Citing often from the \textit{Stanford Encyclopedia of Philosophy}, Sudarshan goes to the very roots of Western hermeneutics which begins with the Homeric epics – with the issues of \textit{allegorisis} and \textit{hyponoia}, ‘what is stated differently’ and ‘what the underlying sense is’. The Middle Ages brought to the fore “the heptad of questions” in interpretation. Texts manipulate their content, and need to be manipulated in turn in order to be seen through. The actual sense is often beyond or below the surface. The Hermeneutic Circle proposes to discover the spirit of the whole through the individual, and access the individual through the whole. Hypothesizing meaning in an incremental piecemeal fashion without an awareness of the fuller picture is a problem that besets the hermeneutic approach. The interpretive praxis can take on multiple forms and can take place according to diverse aims.

 Attributing motives, and mean socio-political ones at that, to the writings even of sage-like figures such as Pāṇini or Vālmīki, is a standard practice with the execrable Western Indologists. Text interpretation aims at identifying the meaning of a text by virtue of reconstructing the nexus of meaning that has arisen in connection with the text. Western Indology has spewed vast amounts of spurious nexus of meanings shelving aside standard practices, the \textit{śāstra-paddhati}, of traditional norms practised through centuries, in respect of Indian knowledge systems. Improving upon Ricoeur’s famed “Hermeneutics of Suspicion”, Western Indologists have only cultivated a vile “Hermeneutics of Derision”, a handy tool of colonial expansionism. 

The attempt of the West to interpret Dharmaśāstra-s (typified by William Jones’ translation of \textit{Mānava-dharma-śāstra}) also saw the beginnings of the Western interpretation of Mīmāṁsā. Rather than attempting to comprehend the underlying principles of \textit{dharma} and \textit{karman}, expletives aplenty (such as "atheistic", "oppressive", "ritualistic" and "divisive") were heaped on Mīmāṁsā. Shelving aside the Indian epistemological framework, Western socio-anthropological approaches were applied. The hubris of othering has sidelined the vital issues of \textit{karman} and \textit{punar-janman}, \textit{puṇya} and \textit{pāpa} etc. The enterprise of \textit{discovering} iniquity everywhere can only be an \textit{a-dharma-jijñāsā} in respect of a discipline that sought to evolve principles on the foundation of \textit{dharma-jijñāsā}! A veritable parody nonpareil!

Pushing dates thither sensibly or insensibly is a fad with Western Indologists (one ironically noted by Pollock himself); and toeing the same line nevertheless, Pollock imperiously pushes Jaimini to a date centuries later than the Buddha. The Mīmāṁsaka sought to assert, for him, what the Buddhist rejects viz. the eternality of anything in general, and of the Veda-s in particular. An offshoot of the timelessness of the Veda-s that the Mīmāṁsā espouses was the lack of the sense of history. History was not so much to be unknown as to be denied. Vedicization is for inspiring ahistoricality, and even \textit{itihāsa}-s were diverted of historical contents. \textit{Smṛti}-s are thence accused of being the fabrication of the Mīmāṁsaka-s, and so with the concept of \textit{puruṣārtha}. The pursuit of \textit{dharma} accordingly would have little to do with the pursuit of Brahman. The agenda of desacralisation and secularisation of the Vedic is thus nothing but a Hermeneutics of Derision.

Pollock is not a practising Mīmāṁsaka or Vedāntin, but can pass verdict on them or on any on the basis of his academic credibility. No Indologist is seriously concerned with \textit{dharma}. After all, the funding of the South Asian Studies Departments is governed essentially by geo-political demands. The dictum (perhaps the diktat) is: the more the othering, the more the funding.

 Those steeped in the Judeo-Christian postmodern frame are thus the least of the \textit{adhikārin}-s to interpret texts and practices of Mīmāṁsā. Pollock bases his views on history on the theories of Vico, for whom everything everywhere had to happen the way it supposedly happened in Europe, the exemplar. It is to counter this that Karl Popper had to tackle the fascist and communist belief in the "inexorable laws of historical destiny".

 The potential of history to manufacture and control power is its most critical value. Otherings and cultural genocides are its offsprings. Historicising is a subjective act. Every historian of every hue can claim that trends of history betray the aims and goals of society \textit{as he espouses.}

Using historical incidents to serve current agenda is the Western historians’ idea, his passion and purpose. Using the episodes, on the contrary, as a context for an elucidation of abiding \textit{dhārmic} principles is the approach of \textit{Sanātana Dharma}. The fantastic details in the \textit{Purāṇa}-s are metaphorical, and they function in symbolic ways, and have hidden meanings.

Western historical approach is well-designed and sophisticated so as to justify colonialism, to buttress slavery, to inflict genocides. History is an academic tool of the colonialist to present his sordid plunder and tyranny as favour and benefaction.

Mīmāṁsā is actually “sacred discussion”. The very Mīmāṁsā concept of letters, words, sentences, and meanings and the actions that they ultimately inspire especially in the Vedic context – are all unique to the Indian context.

Right interpretation of sentences, more particularly the Vedic, which elucidate \textit{dharma} and right performance of actions, so as to conduce to \textit{dharma} constitute the burden of Mīmāṁsā. The opinions of those \textit{to whom} dharma \textit{does not matter} do not matter.

The works of Rajiv Malhotra inaugurate a modern \textit{mīmāṁsā}, a novel dialectic Dharmism.

The next article is by \textbf{Alok Mishra} (Ch. 3) entitled “\textit{Sheldon Pollock evam Mīmāṁsā}” (in Hindi). Alok gives a brief overview of the utility of the Mīmāṁsā \textit{śāstra}, placing firmly the centrality of \textit{yajña} in \textit{Sanātana Dharma}. Quoting from authorities like Śabara Svāmin, he brings out the concept of the \textit{pramāṇa} (valid means of knowledge) and discusses the characteristics of \textit{dharma} – viz. that which brings \textit{śreyas}, linking it to an examination of the means of knowledge itself. That Mīmāṁsā posits that all cognitions must be accepted as true until otherwise proved via other cognitions is considered through giving a detailed description of what constitutes a \textit{pramāṇa}.

Pollock has dismissed the arguments offered to prove that the Veda-s are \textit{apauruṣeya} by the Mīmāṁsaka-s such as - that the names that have been associated with the Veda-s are of those who specialised in the transmission or the exposition of the texts, and not the composers; or the ones pertaining to the Vedic language and the style; and that there is inconsistency in the \textit{bhāṣya} regarding the beginninglessness of the Veda-s; etc.

Alok has attempted to respond to the issues raised by Pollock such as - the issue of transcendence of the Veda-s; the issue of the validity of cognitions; the anonymity and the beginninglessness of the Veda-s; and the eternality of the Veda-s. The relation between \textit{śabda} and \textit{artha}, sound and sense, has also been discussed bringing in the various views – of Mīmāṁsaka-s, Vaiyākaraṇa-s, and others. The traditional position has been explicated on these diverse issues.

On the issue of the priority of the times of the Buddha and Jaimini, the opinion of Prof. G. V. Devasthali has been cited. The issue raised in the context by Pollock has already been answered in Devasthali’s writings decades earlier. 

The next paper entitled “\textit{Mīmāṁsā, Bhāratīyānāṁ Anaitihāsikatvaṁ Ca – Pollāka-Kumārasvāmi-matayor Abhivīkṣaṇaṁ}” (Ch. 4) is by \textbf{K S Kannan}, and is in Sanskrit. The paper starts with a paradox: If you do not study Western Indology you are uninformed; and if you do, you are misinformed.

Indological studies started by the West are of course flourishing, and there are many Indians toeing the path trodden by Westerners. Not all among the Western Indologists are prejudiced. The present article aims to analyse Prof. Sheldon Pollock’s 1989 article viz. “Mīmāṁsā and the Problem of History in Traditional India” and contrast in fine his views with those of Dr. Ananda Coomaraswamy in general.

 Pollock approvingly cites George Larson who says “In a South Asian environment, historical interpretation is no interpretation. It is a zero-category”. Friedrich Nietzsche’s quote comes in handy for Pollock: The beast lives unhistorically. How he loves, Pollock loves, to damn Indians as beasts! If the Greek too did not draw sharps lines between history and myth, that is no count for him. Pollock himself cites Boer: “It is not that gods appear in myth and men in history, but both appear in time and in history,” but then, Pollock knows so well what to relegate to footnotes and what to highlight in the main text. Who can beat him in his sleights of tongue?

Pollock brings in Steitencron’s record of a point in art history where the conquest of Gangas by Pallavas is enshrined in sculpture. After all, this is only a sort of picturesque \textit{paronomasia}, whose counterpart is quite common in literature where \textit{double entendre} would be equally deftly employed. 

The absence of historical details in Sanskrit works are, for Pollock, to be attributed to the influence of Mīmāṁsā, for Mīmāṁsā despises history. The focus of Mīmāṁsā is the supernatural. Nirukta comes in handy for the Mīmāṁsaka-s, for even historical personages in the Vedic are only representative of the eternal. The Veda-s are infinite, and the later \textit{śāstra}-s are as it were subsumed under the Veda-s. History is not much absent as is repudiated. System is evaluated as above process. The social system is highly valued rather than human enterprise. Novelty and creativity are effectively killed. So go the assertions of Pollock.

Many historians have grieved about the lack of history in India – of the irony about the most ancient civilization with very few original histories about its past, as Tiruvenkatachari says. Auboyer remarks that royal chronicles repeatedly convert historical facts into myth and legend. Naudou notes how historians of India have had to rely upon, of all, grammatical examples for reconstructing history! Prof. Ingalls, the rare sane scholar in the West, notes how poets and kings here have only melted into the types of poet and king.

Even though nothing is, for Pollock, unexploitative in South Asia, Prof. A L Basham draws a contrary picture – that nowhere else in the world was the mutual relation among citizens, and between the state and the subjects, as humane as was in India. Nowhere else were human rights so well protected as in Kauṭalya’s \textit{Arthaśāstra}, and nowhere else would be found even a trace of the \textit{dharma-yuddha} pattern as set forth in \textit{Manusmṛti}. Prof. U N Ghoshal cited with approval the statement of Aurobindo that the first feature of Indian civilization is its spirituality, and the next is its zeal for life - both here and hereafter.

Pollock may be opposed to allegorical interpretation of Hindu scriptures, but he is perhaps unaware that such interpretations are available in respect of scriptures of other traditions also. Again, arguing that only the battle chapters constituted the original \textit{Mahābhārata} (as Weber does), is but a case of “arguing from Homer”: Heehs shows how Europe’s literary criteria are not applicable to India. As to the historical records themselves, Hindus were keen on preserving the meaning of events, not a mere record of events, as Organ notes.

D C Sircar records, nevertheless, how over 90,000 inscriptions have been discovered in different parts of India and more are being discovered. Basham lists over a dozen Eras that were current in different parts of India.

Prof. Arvind Sharma draws attention to the details given in Akṣapaṭalādhyāya of Kauṭalya’s \textit{Arthaśāstra}, and Basham has shown how well documents were preserved in the Coḷa kingdom. Hiuen Tsang’s testimony is referred to by Beal. All these hold a mirror to the meticulousness with which documents had been maintained. 

The sordid story is that it is with the advent of Muslim marauders that documents got destroyed on a large scale – as Witzel himself testifies in regard to the situation in Nepal. No manuscript prior to 1500 CE is available in Kashmir, and the nasty Moghuls of the religion of peace and narcissism were undoubtedly responsible for this - to paraphrase Sharma.

What is more, the destruction and devastation wrought by the degenerate and diabolical Moghuls, the curse of India, is borne out by Albiruni himself. \textit{Rājataraṅgiṇī} too testifies to this. Sharma refers to the “perfect genocide” that these Islamists murkily wrought.

It is with Coomaraswamy that we find many of the issues raised by Pollock answered as though in uncanny anticipation. As against Norman Brown, Coomaraswamy refers to the ideas of Maurice Bloomfield for whom the \textit{Mantra} and the \textit{Brāhmaṇa} are only two different modes of presentation of the self-same ideas, while none would of course contest linguistic change from the Veda-s to the Upaniṣad-s. Vedic material is extensive, yet infallibly consistent within itself. Bloomfield notes that the \textit{mantra}, \textit{brāhmaṇa} and \textit{sūtra} are all different modes of literary activity, but largely contemporaneous. For Franklin Edgerton every idea in the Upaniṣad-s is already foreshadowed in the Vedic. The consistency within the Vedic corpus is, as per Coomaraswamy, extraordinary. Finally, Coomaraswamy arrives at the very ideas of Mīmāṁsā, though independently, taking the wind out of the sails, anticipatorily as it were, of Pollock's verbiage nonpareil. 

-----

The next paper entitled “The Science of the Sacred” (Ch. 5) is by \textbf{T N Sudarshan}. The various Indian knowledge systems and related practices can be ill-comprehended by one unequipped with an understanding of the notion of the sacred. Western scholarship is fundamentally and inherently limited when it comes to grasping this - owing to its very origin, structure, and evolution. Western Indology in its current status has pressed into service Marxist, philological, and post-modernist approaches to the Indian knowledge systems, only to befuddle issues and effectively mislead all students of Hindu culture. The sense of the sacred as it obtains in the West is one which is centralised, institutionally enforced and artificial - so against the Indian which is a natural efflorescence. The West is obsessed ironically with “liberating” India, and appropriate for itself the right to desacralise it in various ways wantonly and arbitrarily.

There is no activity of humans as per the Hindu view that is not animated by a sense of the sacred – as is well-illustrated in works like the \textit{āhnika-grantha}-s (such as in Śrīvaiṣṇava \textit{sampradāya}).

The desacralising subversionists have modern strands and strains of neo-Orientalists, Marxists, post-colonialists, subalternists and post-modernists - as well-explicated by Rajiv Malhotra. Neo-Orientalists have brought in new theoretical methods, inference techniques and argument frames. And the neo-Orientalist par excellence is Sheldon Pollock who propounds the idea of an innate dichotomy of the sacred and the non-sacred. Pollock is a past master in introducing schisms (what with his unfaltering sleight of tongue) – such as between the \textit{pāramārthika} and the \textit{vyāvahārika}, between the \textit{śāstra} and the \textit{kāvya}, between the Sanskrit and the vernacular literatures, between the oral and the written, between the Pāṇinian and the non-Pāṇinian, and so on and so forth. He is also an adept in introducing new confusions by way of drawing false parallels between the Hindu and the Christian traditions, and is generally capable of generating “facts” at will to buttress his weird theories woven out of nothing but sophistry and casuistry.

The idea of the sacred is best analysed in the works of Mircea Eliade in modern times, and Eliade openly acknowledges the influence of Indian philosophy. The profanation of life has been so extensive and deep in recent times that a life erected on sacred fundamentals is inconceivable for the modern man. Eliade’s approach must be made use of in dispelling modern misrepresentations. As S N Balagangadhara has shown, the academia has been dominated exclusively by questions that Europe has asked. Western presuppositions and epistemologies have dominated the academic discourse. Balagangadhara’s “Root Model of Order” needs to be explored and deliberated over. On similar lines, the ideas of “embodied knowing” and “history-centrism” introduced by Rajiv Malhotra need to be expatiated upon. The Indic heritage (including Hinduism, Buddhism, Jainism and Sikhism) has developed a range of inner sciences and experiential technologies (“\textit{adhyātma-vidyā}”) in order to access higher states of consciousness \textit{to be experienced first hand}, and this “embodied knowing” is utterly lacking in the West. The Abrahamic religions have a heavy dependence on historical events (actual or contrived), which fixation of history-centrism Indic faiths are utterly free of. Dharmic faiths never witnessed the psychological, religious and social conflicts that history-centrism has all along inspired.

The functionalism of Durkheim, the sociology of Max Weber, and the materialism of Karl Marx have ensured a major removal of sacrality embedded in family and marriage, in festivals, and in worship etc., which are religion-instituted.

The ubiquity of the sacred is explored and expounded through centuries of texts in the Indic tradition, and the very geography and history of India exude the fragrance of the sacred.

The enterprise of desacralisation in its latest version of neo-Orientalists consists in attributing all social ills (poverty, illness etc) to \textit{dharma}, “excavated” through the philological methods deviously designed by Pollock and the like. The \textit{leitmotif} of the “White Man’s Burden” has all along inflicted and justified slavery, the crusades, the genocide of natives etc.; and the novel guises of world peace, human rights etc are no less jeopardising.

The anthropological and sociological discourses of the West, and even the so-called objective discourse of science, as too the Abrahamic faiths – have all proved inimical towards the sense of the sacred – propounded, propagated, and practised by the Indic faiths. The prevailing academic discourse is so designed as to assert control and co-opt the dharmic into the Western Universalist discourse. Pitted against the all-encompassing dharmic perspective, the Western understanding of the essential nature of the human as well as a comprehensive understanding of reality would appear extremely constricted.

Speaking from a \textit{dhārmic} standpoint, it cannot be gainsaid that the worldviews of Western religions, the methods of Western science, the rhetorics of Western humanities and social sciences - at all stages of their evolution and function - have never shed their proselytizing nature directed against non-Western orders in general and \textit{dhārmic} faiths in particular. The sense of the sacred that the Hindu genius is naturally endowed with has withstood Western onslaughts for long, yet needs to be strengthened especially in its intellectual dimensions, in order to effectively counter the civilizational threat it encounters.

The next chapter entitled “On Desacralization of Sanskrit” (Ch.6) is by two authors viz. \textbf{Manogna Sastry} and \textbf{Megh Kalyanasundaram}. The relationship between culture and power in pre-modern India is a veritable obsession with Pollock. And what characterise his writings are, typically - being rather selective in his collection/presentation of data, applying anachronistic socio-political models, and a facile indulging in sweeping generalisations.

In his massive 2006 work bearing the title \textit{The Language of Gods in the World of Men}, Pollock demonstrates his intent to explain certain Asian linguistic phenomena as parallel to the European ones, to the vernacularisation there in particular. While his predecessors sought to Europeanise the very character of India, Pollock in his stride attempts to purge it of all native and formative elements. It was given to but a few Westerners such as Will Durant and Paul Brunton to overcome Western prejudices, and get to recognize the genius of India.

For Pollock, Sanskrit never functioned as an everyday medium of communication; and Sanskrit grammar was but a tool of hegemony. He gives a bizarre picture of India’s past, where her chief pursuits for millennia was limited to the religious and ritualistic, coupled with, of course Brahminical oppression, what else: their standard “historical” stick to beat with. This is pitted against the \textit{kāvya} with its fabricated non-sacred liberating role, and as a clear break from the older order: thus embodying desacralisation.

Manogna and Megh note certain features in Pollock’s writings such as internal inconsistencies in his scholarship \textit{vis-à-vis} his own positions, distortions through mistranslations or unsubstantiated claims, clear biases and dicey models (even as evidenced in Rajiv Malhotra’s 2016 book \textit{The Battle for Sanskrit}). Some of the above charges have been well-illustrated with citations. His changing stands as to the supposed transition of Sanskrit from purely liturgical to mundane usages: \textit{kāvya} as a "direct descendent" of Vedic \textit{mantra}, and yet its break from it (else than the \textit{apauruṣeya} and \textit{pauruṣeya} aspects); he translates \textit{dṛśya andśravya} respectively as “something seen “ and “something heard”, but later complains that there is no category for “literature as something read” – a case, one among many, of misleading by mistranslation (contrast this with the better renderings supplied long ago by M Krishnamachariar ). (Of course one may ask whether Pollock has not come across the usage \textit{yaḥ paṭhed rāma-caritam} in the very \textit{Rāmāyaṇa} he has translated. Pollock himself refers to the sinister predilection of the old Orientalism “to gratuitously debunk claims to antiquity for Indian culture... in a way that pained Indian intellectuals from an early date", yet acts in a most ungenerous fashion with regard to the beginnings of writing in India - portraying it as an importation in the third century BCE, whereas the testimony of Richard Salomon (whom Pollock quotes), shows it as “ranging from the sixth to the early fourth century BCE”. Subhash Kak (1994) had shown it as sixth century BCE (Kak 2015 cites BB Lal referring to Brāhmī of 800 or 900 BCE). Pollock’s devious efforts to undermine the enormous presence and significant role of the oral transmission of valuable knowledge are also transparent. Pollock is eager to mark the arrival of Śaka-s (Indo-Scythians) as ushering in a new era (or at least reinforcing one) – a cosmopolitan era, whereafter ritualisation and monopolisation of Sanskrit gave way to a new sociology and politicisation of the language – all conjectural superstructures erected upon conjectural foundations.

Pollock is fond of his pet phrase “Nothing suggests...” as though he has surveyed all available literature, all unpublished manuscripts (“over thirty million”, the fact he is not unaware of!), all literature destroyed (and who knows how many millions?). What gives away the supposed movement “from liturgy to literature” postulated by Pollock is the very word “\textit{kavi}” used for seers in the Veda, and for poets, later. The nexus Pollock works out between \textit{kāvya} and \textit{rājya} is flimsy if not also silly. The presumptive Pollock shows himself when he says “poets eventually decided to shatter this seclusion and... to commit them to writing”, mouthing his grand and generous speculations about events ten centuries ago: what could at best be but wild guess is presented as an eye-witness account as it were!). Pollock, the (mal)adroit, “tries to cover in ornate language and diffuse style of writing the absence of any substantial basis for his claims.” The wilful obfuscations of the imperious scholar who is at ease in blatantly biased theorisation turning a blind eye to facts staring in the face stand well-exposed.

Pollock’s unconcealed contempt for Sanskrit in the context of his unfounded claims on Sanskritisation betray his meanness and warrantless self-assurance. For wilful fabricators such as Pollock, there is nothing like a common vision of a culture such as the Indian. Who cannot admire the boldness of Pollock who spins theories not on account of, but in spite of, facts galore? What joy hard facts in front of theories conjured up with design (pun intended)? Could a scholar, leave alone an Indologist, be more flippant and frivolous who could ejaculate: Indeed a stable singularity called Indian Culture, so often conjured up by South-east Asian indigenists, never existed? Ultimately, it is Pollock’s that is a crude sort of teleology. Who, else than Pollock, can take a freeze frame for the whole story? But for the well-paid proselytiser, who would so smugly proclaim than Pollock: “[T]he South Asian knowledge South Asians themselves have produced can no longer be held to have any significant consequences for the future of the human species”? It would be difficult to believe that scholastic hubris can so well overtake sanity, that “intellectual” pogrom can be so cold-blooded.

The next paper entitled “The Science of Meaning – Explicating the Nature of Philology and its Implications to \textit{Videśī} (Foreign) Indology” (Ch. 7) is by \textbf{T N Sudarshan.}

 Philology is, to note its origin and growth briefly, the multi-faceted study of texts, languages, and the phenomenon of language itself. Philological scholarship was actuated by motivations of colonialism and racism. Philology lost its importance with the maturing of the scientific method. With the Greek it was the ability to argue skilfully in public. Sir William Jones added the idea of race to the prevailing political halo. With this commenced Indology, which expanded European perspectives on the history and civilisation of the world. For Nietzsche, philology was an absurd combination of inconsequentiality and hubris. In the nineteenth century, it was used to justify the horrors of racism, slavery, and colonialism. 

Neo-Orientalists led by Pollock have invented newer versions such as Political Philology and Liberation Philology. After Edward Said’s critique on Western anthropological and social science scholarship, the study of the East had to be reinvented with new methods, and Pollock came out with a new mint of Philology, based on spectacularly speculative theories: it was the study of Sanskrit that affected the subconscious of German Indologists, and the Holocaust can in fine be traced to the divisiveness and hatred that Sanskrit spells, which he labels “Deep Orientalism”, laden with his own cultural biases and hegemonic filters. The roots of Pollock’s Philology can be seen in Giambattista Vico, the father of modern social science, for whom human truth is like a painting, which can persuade us through the most evident falsehoods that she is pure Truth. 

As Rajiv Malhotra notes, there are pernicious motives to Pollock’s seemingly academic theories – which is why Malhotra presents Sacred Philology as against Pollock’s Liberation Philology, for the latter is directed towards re-engineering of Indian society using Western paradigms, subserving Western hegemonic ends.

The Western and Indian theories of language present a great contrast. The Western tools of philology and hermeneutics do not suffice. America is a modern-day cultural coloniser, and the demands of dual-use anthropology via Area Studies demands an exploitative scholarship, as fundings are governed by geopolitical requirements. The semantic theory and foundational theory of meaning proposed in the West (including the Gricean approach that speaks of the communicative intentions of language coupled with consideration of beliefs) of even the non-mentalist theories do not go deep enough.

In the Indian approaches to meaning, the role of three \textit{vedāṅga}-s (\textit{śikṣā, chandas}, and \textit{nirukta}) are given their due emphasis in their appropriate contexts. The architecture of the Indian conception of meaning is well laid out by Prof. Kapil Kapoor. The multi-millennia-long tradition of interpretation has been duly enriched by contributions from the streams of Nyāya, Mīmāṁsā, and Vedānta. That India is an interpretive community is indicated by the fact that the Brāhmaṇa, the Buddhist, and Jaina schools share many methods of interpretation. Apart from these, interactive traditions of \textit{kathā} and \textit{pravacana} constitute collective institutionalised reading. The vast exegetical scholarship via \textit{bhāṣya, vṛtti, vārttika, ṭīkā} etc. conduces to refinement and extension of the various lores. The Western scholar, however knowledgeable, hardly attains to the level of a true \textit{adhikārin}.

In the computational approach, Artificial Intelligence systems utilise manipulation of symbols employing axiomatic approach, involving propositional and first-order predicate logic, where the basic problems of representation and reasoning are yet to be solved. The statistical approach to meaning and intelligence fares no better. The availability of trillions of data sets training mathematical engines in pattern recognition but masks the actual fact that machines do not indeed understand what they are handling in regard to issues of image processing, natural language processing, or speech recognition etc. It is only recently that some of the best minds of today have begun to admire the deep insights into language and thought available in Sanskrit since ancient times.

Given the vast repertoire of interpretative methods and approaches made available in Sanskrit, resorting to Western philological tools is no more than a parody, an extension of intellectual colonisation. The aesthetically camouflaged and strategically positioned philological methods of the Neo-Orientalists cannot hide their pernicious motives.

Westerners regularly used Buddhism as a wedge against Hinduism, and the neo-Orientalists use the Mughals as a wedge against Hinduism, so as ultimately to lead to an admiration of the Greek – as per the essentially Euro-centric agenda of the West. The Pollockian programme of linking Sanskrit or Mīmāṁsā to Nazism is only to pave the way in effect for tracing all ills of the West to some definite or indefinite Indic roots.

The academic verbosity, dense and deliberate, of the champion of neo-Orientalism verily holds a mirror to his deep disdain for Sanskrit knowledge systems, and even more so towards \textit{dharma} as such. Rather than getting to be a \textit{science} of meaning, the new philology is gotten to be a veritable \textit{nescience} of meaning. This unscientific and dishonest scholarship, contumely unlimited, must needs be countered effectively. -----

The next paper entitled “Saṁskṛti in Context” (Ch. 8) by Dr. \textbf{Charu Uppal} shows how the Pollockian grasp or interpretation of the \textit{Rāmāyaṇa} is utterly inapposite. The \textit{Rāmāyaṇa} continues to be a source of inspiration to Indians, and many others, to this day: we have television serials and cinemas, music and dance performances, present continuous, on \textit{Rāmāyaṇa} themes. Rām-līlā continues to spellbind massive audiences. As Edward Said says well, in effect, a European or American studying the Orient is a European or American first, an individual next.

In neither role he, as an outsider, would understand, much less feel, what a Hindu understands or feels when the name of Lord Rāma is uttered, when Rām-līlā is enacted; or even what happens prior to Rām-līlā, or after it.

For Pollock, the \textit{Rāmāyaṇa} demonises non-Hindus in its language, story, and characterisation and its revival results in violence (what else), against Muslims in particular (and who else!). The plot of the \textit{Rāmāyaṇa} is linked to power structures (what is not to him, by the way?). This degenerate desacralisation is a major flaw in Pollock’s methodology, a horror unadulterated.

As Joseph Campbell points out, myths are clues which direct us towards the experiencing of the spiritual potentialities of human life. In the words of Rajiv Malhotra, myth uses fiction to convey truth. In contrast with the frozen idea of ‘history’ in the West, \textit{itihāsa} comprises history as well as myth. The double standards of the West are evident – in their application of sociological methods and tools while studying Jewish and Christian tales, whereas while studying others anthropological tools are to be made use of; their own groups are referred to as communities, others as tribes: here, by the way, are the bloodthirsty initiators of othering. As against such debased Western pattern, \textit{itihāsa}-s are construed essentially as instructional.

The continued impact of the \textit{Rāmāyaṇa} on the Indian populace is evidenced by the fact that even to this day children are named after the characters therein, and thousands of Rām-līlā-s are performed around the world: the recitation of the \textit{Rāmāyaṇa} is deemed sacred.

Rām-līlā is enacted as a ten-day ritual culminating on the Tenth Day of Victory, the day of Vijayadaśamī, when the effigy of Rāvaṇa is burnt. The “fluidity” of \textit{itihāsa}-s is well-illustrated in the way the script keeps improvising year to year in “Our Rām Līlā” being staged in Delhi, and in the way the audience too sometimes takes part. Pollock discloses no knowledge of things such as these.

Pollock’s strategy of forwarding his theory of “aestheticization of power” consists in first desacralising the text he studies: the academic blasphemy of divorcing the object of study from its context. The theories of Vico for which Pollock attaches high importance are criticised by many, including Christians and Carey, who show how the natural science model is inapplicable to the social sciences. Culture and related symbols are complex and multi-layered, and need to be understood in a multi-dimensional context. Pollock forgets that Rāvaṇa is himself a Brahmin when he accuses others. Those for whom “religion is the opiate of the masses” can after all be expected do no justice to a text like the \textit{Rāmāyaṇa}. As Malhotra points out, Pollock goes against his own mentor Prof. Ingalls who stressed dropping the Western lens in the study of Sanskrit traditions and \textit{kāvya}-s. The avowed purpose of the \textit{kāvya} is to communicate \textit{dharma} to the lay in an aesthetically pleasing manner (to paraphrase Ingalls).

The very language Pollock employs is opaque. If his individual points are at times murky, murkier still are the links amongst the dots necessitated if one were to make sense of the picture. Pollock is only adding stuff to the new brand of atrocity literature. He cherry picks statements from Sanskrit works just to generate distorted pictures of their originals. Pollock would do well to speak to the participants of Rām-līlā to understand their own feelings towards the \textit{Rāmāyaṇa}. ----

The last paper entitled “The \textit{Śāstra} of Science and the Science of \textit{Śāstra}” (Ch. 9) is by \textbf{T N Sudarshan} and \textbf{T N Madhusudan}. The thrall of technology and the narrative of science lie at the base of the current sense of the superiority of the West. Under the guise of peer reviews and by multiple references to each others’ works, Western Indologists have developed a cabal. Over the years, their theories get accepted as taken for granted. In his hegemonic discourse (as Rajiv Malhotra describes it), Pollock states his political goals for India: to intervene on behalf of “the oppressed”. Pollock creates newer tools such as three-dimensional philology, creative chronology, and socio-political hermeneutic lenses in order to undermine \textit{Sanātana Dharma} and Indian civilization.

\textit{Śāstra}-s are, for Pollock, incapable of creativity and progress, as the Veda-s are deemed eternal and perfect; and \textit{śāstra}-s can only restate and extrapolate what is contained in the Veda-s. \textit{Śāstra}-s are regressive as they cannot utilise fresh insights from the empirical world.

Before answering the above charges, one must dismantle the Western narrative of science itself where it exercises control. In Western historiographies, Greece is extolled as the source of all science, an idea strongly contested today. Much of what we call Greek philosophy is largely a legacy stolen from Egypt. Prof. C. K. Raju adduces proofs of the Western colonization of science and mathematics. The number system and calendrical system of the Greeks speaks poorly of any claim to a discernible knowledge of astronomy on their part (leave alone original discoveries). Centuries of dishonesty and falsehood of Western historians stand exposed in the writings of Prof. Raju. The non-originality of Copernicus and Co. has been laid bare. 

Colonised by the West for a few hundred years, India continues to remain in thrall of the West, and the fabricated history of science being Western in origin etc. only serves the interests of the hegemony of the West. Raju has shown how current science and mathematics are deeply influenced by Christian theology. The essentially non-empirical axiomatic proof approach and deductive methodology, and proof-based mathematics lead to conceptual bottlenecks.

The four-fold logic of the Buddhists and the seven-fold logic of the Jains afford different and deeper perspectives and are considered superior to the two-valued truth-functional logic of the West. The Christian whitewashing of the history of science also undermines the Islamic and Hindu contributions to logic and mathematics. The non-universality of Western logic, the absence of a clear definition of science itself, the disunity of science, the success of science owing to its closeness to sources of political power etc. – are all factors not usually reckoned with, or rather wilfully and skilfully withheld from mainstream study/criticism of science. The Leftist narrative in India has only reinforced this in full force.

Angus Maddison, Dharampal, and others have provided the alternative picture, and a deep study of the \textit{śāstra}-s remains a desideratum in this direction. \textit{Śāstra}-s have always allowed for adaptation, re-discovery, re-interpretation etc., and are far from being history-centric, and always have scope for additional \textit{śāstra}-s. \textit{Śāstra}-s conduce to \textit{dharma} and \textit{mokṣa} for the individual and the welfare of the society in general. The perception of social ills in the \textit{śāstra}-s is essentially a Western prejudice and projection.

The modern sciences have nothing comparable to the comprehensive and holistic understanding of life that the \textit{śāstra}-s generate, and provide guidelines for, for attaining the high goal of life as elucidated in the fourteen (or eighteen) \textit{vidyā-sthāna}-s. It is in no wise essential in terms of conceptualisation or execution that the \textit{śāstra}-s ought to resemble modern science.

From a superficial point of view, science and technology have improved the condition of humanity in general; but it cannot be gainsaid that they have also subserved as powerful instruments of plunder and exploitation of nature and fellow-beings (slavery/apartheid). This is very well borne out by colonisation keeping pace with the growth of science and technology. Current practices of science are only leading to indiscriminate exploitation and depletion of natural resources, threat of nuclear holocausts, environmental disasters (including extinction of very many species and damage to biodiversity etc).

In sum, science may be profoundly successful in addressing a very narrow set of problems, but vital issues in life, always complex, are not easily or successfully handled by science. The world-view that science presents is utterly limited, and indeed pales into insignificance when pitted against the \textit{śāstra}-s; \textit{śāstra}-s are not static and limited, nor do they lack an empirical approach. \textit{Śāstra}-s can inspire many new things (including providing new/additional perspectives to science itself. Yoga as a \textit{śāstra}, for example, has made many positive contributions in correcting many ills of society.

Against the above analysis, all the allegations and misrepresentations of Pollock in respect of \textit{śāstra}-s stand repudiated; and on the other hand, the highly beneficiary role of the \textit{śāstra}-s also becomes patent.

\vskip -5pt

\begin{center}
* * *
\end{center}

\vskip -5pt

\begin{myquote}
An important reason for taking up Pollock’s works for critical study is the fact that he is held in very high esteem among the the modern Orientalists. Few dare to criticise him, and there is a great deal of growing hagiography around him: “Extraordinary even among the already extraordinary tribe of Sanskritists, he (=Pollock) has taken the study of Sanskrit to a new level, engaging historical, comparative, and theoretical issues with a range and sophistication that is unusual and in many respects unprecedented “ (Dirks 2016:ix). This statement is preceded by a reference to his “scholarly breadth, erudition, originality and commitment”, and succeeded by a reference to “his eloquence, erudition and efficiency” (Dirks 2016:xiii). Again, “Sheldon Pollock remains a leader in the fields of Sanskrit Philology, Indian intellectual and literary history, and comparative intellectual history (Bronner \textit{et al} 2016:xv). There is a laudation of his innovative scholarship, referring to how “Pollock’s influence within and beyond the field of South Asian studies has risen to new heights...”
\end{myquote}

\vskip -5pt

\hfill (Bronner et al 2016:xxi)

\begin{myquote}
“Pollock’s work will likely play a dominant role in shaping the wider public image of pre-modern India, especially Sanskrit, language and culture along with the forms of polity related to them, \textit{for years if not decades to come}” 
\end{myquote}

\hfill (McCrea 2013:117) (italics ours).

\textit{The Battle for Sanskrit} by Rajiv Malhotra outlined the immense damage to the academic realm by the malafide writings of this famed scholar. The present series of books on Indology (especially the Neo-Orientalist type), is only a first step aimed at remedying and rectifying the situation.

\begin{center}
* * *
\end{center}

Needless to say, the authors of the papers here hold themselves responsible for their respective papers.

\begin{verse}
\textit{nigiranto jagat-prāṇān udgiranto mukhair viṣam} \dev{।} \\\textit{dūrataḥ parihartavyā dvijihvā jihma-vṛttayaḥ} \dev{।।}
\end{verse}

\vskip 3pt

\begin{center}
* * *
\end{center}

\vskip 3pt

\begin{flushright}
Dr. K. S. Kannan\\ Academic Director\\ and\\ General Editor of the Series
\end{flushright}

Cāndramāna Yugādi\\ Vikṛti Saṁvatsara\\ (13-Apr-2019)

\begin{center}
\textbf{\textit{[Hindi Synopsis of a Hindi Paper]}}
\end{center}

\section*{\dev{।। सम्पादकीयम्~।।}}

\dev{वेद की अपौरुषेयता को दृष्टिपथ में रखते हुए आलोक मिश्रा का शोधपत्र ‘शेल्डन पॉलॉक एवं मीमांसा’ लिखा गया है~। इस शोधपत्र की आवश्यकता इसीलिए हुई क्योंकि इस विषय में वैदेशिकों जैसे शेल्डन पॉलॉक इत्यादि तथा भारतीयों का आपस में मतभेद है~।}

\dev{अतः वास्तविकता को अवगत कराने की दृष्टि से यह शोधपत्र अत्यन्त आवश्यक है~। आलोक मिश्रा ने शेल्डन्पॉलॉक की विचारों को खण्डन करते हुए वेद की अपौरुषेयता की सिद्धान्त को स्थापित किए हैं~।}

\dev{उदाहरणार्थ - पूर्वाग्रह से ग्रसित होकर पॉलॉक ने यह प्रतिपादित किए हैं कि वेद की अपौरुषेयता मूलतः वैदिकसंस्कृति से नहीं जुडा है अपितु बौद्धों के प्रश्नों का समाधान किस प्रकार किया जाए इसीलिए जैमिनि ने तथाकथितरूप से वेद की अपौरुषेयता की सिद्धान्त को बताया तथा उसका प्रामाण्य को प्रश्नातीत किया~।}

\begin{myquote}
“... It also seems likes that at least some of the most salient articulations of the world, what we now tend to think of as its foundational principles, may have first een conceptualized as a defensive, even anti-axial, reaction to Buddhism... It is self evident that no one would elaborate propositions of the sort we find Mīmāṁsā to have elaborated, such as the thesis of the authorlessness of the Veda, unless the authority of the Veda and its putative authors had first been seriously challenged”
\end{myquote}

\hfill (Pollock 2005:402)

\dev{मीमांसासूत्रों को सूक्ष्म दृष्टि से परीक्षण करते हुए यह दिखाया है कि जैमिनि बुद्ध के पूर्ववर्ती हैं~। बुद्ध ने कर्मकाण्ड को प्रयत्न द्वारा खण्डन किया है~। अतः जैमिनि यदि परवर्ती होते तो निश्चित रूप से उनका ध्यान इधर आकृष्ट होता~। किन्तु मीमांसासूत्र में बुद्ध का उल्लेख प्राप्त नहीं होता है~। मीमांसासूत्र में एकत्र बुद्ध शब्दका उल्लेख है परंतु यह शब्द सामान्य अर्थ में प्रयुक्त है -- "बुद्धशास्त्रात्"~। बुद्ध यदि पूर्ववर्ती होते तो इस पद को सामान्य अर्थ में नहीं प्रयुक्त होता~।}

\dev{प्रकारान्तर से आलोक मिश्रा स्वमत को पुष्टि करते हुए कहते हैं कि ऐसा प्रतीत होता है कि निरुक्तकार यास्क तथा जैमिनि समकालिक थे~। क्योंकि दोनो में बहुत साम्य है जैसे - जैमिनिसूत्र "भावार्थाः कर्मशब्दाः", और यास्क "भावप्रधानम् आख्यातम्"~।}

\dev{वेदों की अर्थहीनता के समाधानार्थ जैमिनि ने नव सूत्रों से समाधान किए हैं~। आलोक मिश्रा कहते हैं कि उनमे से ५ यास्क के समान है~। अतः इनमें से कोई भी पूर्ववर्ती या परवर्ती होते तो निश्चित रूप से एक दूसरे के सिद्धान्तोंका खण्डन-मण्डन करते~। किन्तु न यास्क जैमिनि का उल्लेख करते हैं, न जैमिनि यास्क का~। अतः इससे ज्ञातहोता है कि दोनो समकालिक थे, किन्तु एक दूसरे का ज्ञान नहीं था~। सभी विद्वानों ने ऐकमत्य होकर स्वीकारा है कि बुद्ध से पूर्व, लगभग ई.पू. ५०० वर्ष यास्क थे~। अतःयह फलित हुआ कि बुद्ध से पूर्ववर्ति जैमिनि थे~।}

\dev{अतः पॉलॉक के द्वारा प्रदर्शित किए गए आक्षेप निराधार तथा दुराग्रह से ग्रसित दृष्टिगोचर होते हैं~।}

\dev{अन्य उदाहरण - अपने पेपर} “Mīmāṁsā and the Problem of History in Traditional India” \dev{में शेल्डन पॉलॉक यह कहतें हैं कि एक प्रमाण से ज्ञात हुई वस्तु अन्य प्रमाण से नहीं जानी जा सकती, ऐसा केवल मीमांसक कहते है।} “Second – and this is the basic epistemological position of Mīmāṁsā: all cognitions must be accepted as true unless and until they are falsified by other cognitions.” Pollock (1989:607)

\dev{आलोक मिश्रा कहतें हैं कि पॉलॉक की यह बात हास्यास्पद है। भारतीय षड् दर्शनों मे विस्तार से दर्शाया है कि एक प्रमाण से ज्ञात हुई वस्तु अन्य प्रमाणों से नहीं ज्ञात हो सकती~। यह सिद्धान्त केवल मीमांसा का नहीं है। न्याय में प्रमाण शब्द का निर्वचन “प्रमाकरणं प्रमाणम्” “असाधारणम् कारणम् करणम्~। किं नाम असाधारणत्वम्? लक्ष्यताऽवच्छेदकसमनियतत्वम् असाधारणत्वम्"। ऐसे ही अन्य वैयाकरणादि भी स्वीकार करते है~।}

\newpage

\dev{अतः इस प्रकार से पॉलॉक के विचारों को आलोक मिश्रा के शोधपत्र में तथ्यसहित खण्डन किया गया है~। जिस प्रकार से पॉलॉक ने द्वेष भाव से वैदिक संस्कृति को नीचा दिखाने का अथक प्रयास किए हैं वह अतिनिन्दनीय है~।}

\vskip 5pt

\begin{center}
[Sanskrit Synopsis of a Sanskrit Paper]
\end{center}

\vskip 5pt

\section*{\dev{।। सम्पादकीयम्~।।}}

\dev{भारताधीत्यु} (Indology) \dev{पजीविष्वेकतमः पोल्लाकाख्यः प्रथितो विद्वान् भारतीयसंस्कृतिप्रदूषणपरायणो भारतीयानामितिहासीयघटनावलिविलेखनविषयदर्शितप्रायःपाराङ्मुख्यानामधिक्षेपणाय मीमांसाशास्त्रनिमित्तकतां पाराचीन्ये तस्मिन्नारोप्य हेत्वाभासपरम्पराभिर्भूषितान्नानालेखानाबहोः कालाट्टङ्कयन् लक्ष्यते~। उल्लेख्यविषायाभावैकहेतुकी भारतीयेतिहासनैयून्यस्थितिरिति स्वतःपरिहासास्पदे परिहास आत्मानमयमुद्योजयति~। ग्रीकश्लाघनभारतगर्हणाख्यनिरवग्रहपूर्वग्रहगृहीतोऽयं पण्डितस्स्वकीयलेखनिदर्शितग्रीकेतिवृत्तलक्ष्यमाणदोषसाम्यसद्भावेऽपि विराजिततद्विषयकतूष्णीम्भावो भारतीयेतिहासप्रतिक्षेपे पुनः प्रकटीकृतबहुलजिह्वाव्यापाररुचिर्वर्वर्ति~। आभीक्ष्ण्येन हेतूकुर्वता मीमांसाशास्त्रं भारतीयानामनैतिहासिकत्वे धर्मस्यैव परा धिक्कृतिर्न्यक्कारार्हा हि वावदूकेनाऽमुनाऽऽसादिता~।}

\dev{मीमांसाप्रतिपादितवेदापौरुषेयत्वसंभावनाप्यैतिहासिकांशप्रत्यादेशनार्थमेव प्रक्रान्तेत्यास्थितमेतेन~। वेदानामानन्त्यसार्वज्ञ\general{\break } आधारीकृत्य समस्तविधनावीन्यावज्ञानप्रत्याख्यान एव समासादिते मीमांसकैरन्ततोगत्वेति च विकटं फल्गु च व्याहरत्यसौ~। साधारण्येन भारतीयसमुदाये संलक्ष्यमाणमैतिहासिकविषयकपाराचीन्यमधि स्वेनैव गुरुणेङ्गाल्साख्येन} (Ingalls) \dev{सम्यक्तयैवावबोधिते सत्यपि पोल्लाकेन पुनरदभ्रनिकृतिपरत्वमेव प्रकटीकृतमनारतमरालधिषणेन। अदोषविषमीभूतं किन्नामास्तु\general{\break } वस्तु कथङ्कारं वा भारतीयसंस्कृतावस्य मलीमसमानसस्यावालोके? परन्तु प्रजानृपालयोः प्रजासु च परस्परं सम्बन्धस्समीचीन एवासीत्, प्रजानामधिकाराश्च सम्यक्संरक्षिता आसन्, रणाङ्गणेषु धर्मयुद्धप्रकारोऽपि परामर्हति प्रशंसामिति प्रतिजानाति बाषामाख्यो} (Basham) \dev{ब्रिटन्} (Britain) \dev{देशीयो भारतैतिहासिकः~।}

\dev{आस्तां तावदाक्षेपपरम्परा कौतस्कुतस्यास्याऽत्यन्तमेव प्रतिक्षेपणीया~। नवतिसहस्राधिकशिलादिलेखानामुद्भुतं विस्तृतं च लोकं भारतीयं पुरस्तान्नः प्रस्तौति सर्काराभिध} (Sircar) \dev{ऐतिहासिकाग्रगण्यः~। शकादिसंवत्सरविक्रमादिसंवत्सरादिकं\general{\break } द्वादशाधिककालगणनाप्रकारकाणि प्रदर्शयति बाषं स्वतः~। कौटलीयाक्षपटलाध्यायगतनिबन्धपुस्तकपत्रिकाविलिख्यामनरक्षणीयांशपरामर्शनतोऽधिगम्यमाना सौराज्यव्यवस्था कस्य वा सचेतसो नावहति विस्मयम्? चोलराज्यगताक्षपटलिकापद्धतिरद्भुतासीदित्याह नाम बाषमस्स्वयम्~। प्रतिपुरलभ्यमानलेख्यराश्युल्लेखो ज़ुवनज़ाङ्ग} (Zuanzang) \dev{विहितोऽपि कस्य न\general{\break } जनयत्याश्चर्यपरम्पराम्? तुरुष्कैः पुनरधर्मिष्ठाग्रगण्यैर्न केवलं देवतायतनानि परमैतिहासिकांशसमुज्जृम्भितनिबन्धपुस्तकस्तोमाश्च कथमखण्डं विध्वंसनप्लोषणादिकभाजनमभवन्नित्यपरोक्षमितिहासचुञ्चूनामपि~। विट्ज़ेल्}(Witzel)\dev{प्रभृतीनामप्यत्र\general{\break } विषये निरूपितकानि समुल्लेखमर्हन्ति~। राजतरङ्गिण्यामपि – सेकन्धरधरानाथो यवनैः प्रेरितः पुरा~। पुस्तकानि च सर्वाणि\general{\break } तृणान्यग्निरिवादहत्~॥ - इत्येव विस्पष्टमुद्घुष्टम्~।}

\dev{अवसरे चास्मिन्नानन्दकुमारस्वामि} (Ananda Coomaraswamy) \dev{नाम्नो विदुषोऽभिप्रायास्सङ्ग्रहणीया\general{\break } वर्तन्ते~। मन्त्रब्राह्मणारण्यकोपनिषत्सूत्रसाहित्यसाकल्यगतहृदयसंवादो ब्लूमफील्ड} (Bloomfield) \dev{प्रभृतिभिरपि\general{\break } समधिगत एवेति संसूचयन्नानन्दकुमारस्वामी वैदिकेष्वाख्यानेषु नाम तात्त्विकांशनिरूपणस्यैव सारभूततां विभावयन्भङ्ग्यन्तरेण मीमांसाशास्त्रौदग्र्यमेव हेतुनिकुरुम्बोपन्यासपुरस्सरं प्रतिपादयन्पोल्लाकप्रभृतीनामनृजुमनीषाणामपसव्यव्याहारैकव्यापृतानां\general{\break } वैकट्यं प्रतीक्ष्यैवेव तत्त्वार्थसौषम्यममोघं प्रतिपादयति~।}

\newpage

\dev{अर्हणार्हस्यापि संस्कृतसंस्कृतिवितानस्य विषये प्रदर्शितगर्हणैकप्रावण्यप्रागल्भ्याः परिपन्थिनो नाम नितमां निबर्हणीया एवेत्यत्र शङ्काकलङ्को वा सन्देहस्पन्दो वा मा भूदित्याशयेन परिश्रमेण भूरिणा प्रणीतस्यास्यारम्भस्य विषये सविश्रम्भा जातुचिदविप्रलम्भास्सद्यत्नगुम्भाश्च भवन्तु भवन्तस्सचेतस इति शम्~॥}

\begin{verse}
\dev{मण्डूकराविणं सर्पं गोमुखं च मृगादनम्~।}\\\dev{असुहृत्त्वेन मन्येत मानयन्तं च वैरिणम्~॥}
\end{verse}

\begin{flushright}
\dev{- इति विनिवेदयन्} \\\dev{के.एस्.कण्णन्}\\\dev{प्रधानसम्पादकः}
\end{flushright}

\dev{चान्द्रमान युगादिः} \\\dev{विकृतिसंवत्सरः}

