
\chapter{मीमांसा, भारतीयानाम् अनैतिहासिकत्वञ्च – पोल्लाक-कुमारस्वामि-मतयोरभिवीक्षणम्}

\Authorline{के एस् कण्णन्}

\bgroup

\selectdev

\begin{verse}
अज्ञोऽसि यदि नाध्येषि पाश्चात्त्य-विदुषां मतम्!~।\\ भ्रान्तोऽसि यदि चाध्येषि पाश्चात्त्य-विदुषां मतम्!!॥
\end{verse}

\medskip

नापरोक्षमिदम्प्रेक्षावतां यद् आशतकद्वयाद् इण्डालजी (\enginline{Indology})–त्यभिधानेन प्रथितं\break भारतीय-सर्वस्वाध्ययनं पाश्चात्त्यैर्विद्वद्भिःप्रस्थापितं सुप्रतिष्ठापितं वरीवर्तीति। परस्सहस्रं\break विद्वांसोऽध्ययनेऽ स्मिंस्तद्दर्शितमयनमेवानुरुन्धाना अनारतं तत्र व्यापृता, याननु भारतीया अपि कतिपये विद्वांस उद्युक्ता उद्युञ्जानाश्च दरीदृश्यन्ते~। पाश्चात्त्याः सर्वेऽपि दुष्टैरेवाभिसन्धिभिरतिमात्रमीरिता इति यद्यपि न सुवचं, भूयांसस्तत्र परं मलीमसमानसाः कृतमनस्कास्स्वकपरम्परैक\-प्रागल्भ्य-प्रसाधनप्रतिपादनयोरेवञ्च भारतीयपरम्परायाःपुनर्नैच्यस्थापनैकचक्षुष्कतया\break संलक्ष्यमाणाः ~।

दुर्विदग्धेष्वीदृग्विधेष्वेकतमः षेल्डन् पोल्लाक्(\enginline{Sheldon Pollock})-नामा, यो हि नानाशास्त्रेषु कृतभूरिपरिश्रमस्सन्नपि पौरोभाग्यैकभाग्येषु प्राग्र्यः प्रथामपि बह्वीं समासाद्य विदुषो नैकान्प्रस्थाने विलक्षणे स्वकीये प्रस्थापयन्नमेरिका-देशस्थकोलम्बिया (\enginline{Columbia})विश्वविद्यालये\break लब्धप्रतिष्ठस्सन्भारतीय-सर्वकारेणापि पद्मश्र्याख्येनाग्र्येण बिरुदेन विभूषितश्च~। अलङ्कारमीमांसादिशास्त्रेषु बहुष्वाबहोः कालादध्ययनादिष्वात्मानमुद्योजयन्नेष लेखान्नैकान्ग्रन्थांश्च बहून्\break विततवानस्ति~। प्रकृते मीमांसाशास्त्रमधिकृत्य सप्तविंशतिवर्षेभ्यः प्राग्विलिखितमेतस्यैकं लेखं\break परामर्ष्टुं प्रयत्नः कश्चनात्र विधीयते~। “पारम्परिकभारतेतिहाससमस्या मीमांसा च” (\enginline{“Mīmāṁ\-sā and the Problem of History in Traditional India” 1989}) इत्यभिधानकस्तदीयस्स लेखः~। इतिहासशब्देन चात्र नहि महाभारतदयो ग्रन्था निर्दिश्यन्ते किं तर्हि \enginline{“history”} इति शब्देन निर्दिश्यमाना ज्ञानशाखाऽधुनातनी प्राग्घटितावलिनिरूपणविमर्शनतात्पर्यवती~।

भारतीयाः खल्वीद्र्क्षेतिहासविषये प्रदर्शितपाराङ्मुख्या इति तु विदितचरमेव~। पण्डितेन लार्सन् (\enginline{Larson}) नाम्ना प्रोक्तमेवेदं यद् भारतीयचिन्तनप्रणाल्या-मितिहासाख्यं कल्पनमेव स्फुटं स्थानं न कञ्चन लभते, प्राङ्नवमदशकात्~। ऐतिहासिकं च विवरणं दक्षिणैष्याभागे तावदनाप्तस्फुटास्पदम्~। (\enginline{Larson 1980: “History is a category which has no demonstrable place within any South Asian ‘indigenous conceptual system’ (at least prior to the middle fo the nineteenth century)... South Asians themselves seldom if ever used [a historical] explanation... In a South Asian environment, historical interpretation is no interpretation. It is a zero-category”}). उक्तिमिमामनुमोदमानः पोल्लाको नात्रातिरेकिनीमुक्तिं विभावयति, किन्तर्हि निष्प्रतिद्वन्द्वं भाषितमिदमित्यव्यभि-चरितं सत्यमिति च~।

इतोऽग्रे च मेक्डोनेल्लस्य वाक्यमिदमुल्लिखति पोल्लाको यत्प्राक्कालीने भारत इतिहासानुपलब्धेर्निदानं चेदं यत्तत्रोल्लेखार्हो विषय एव नाभात्कश्चिदपीति (\enginline{Macdonell 1900: “Early India wrote no history because it never made any”})! सिद्धान्ततया चोपस्थापितमेतन्मेक्डोनेल्लेन! कुल्के(\enginline{Kulke}) नामापरोऽपि पाश्चात्त्यो हेतुमत्रेत्थमूहाञ्चक्रे यद्ब्राह्मणकायस्थयोर्यो विभागस्समजनि स एवेति~। तद्यथा ब्राह्मणैर्बौद्धिकानि साधनान्यात्मसात्कृतानि वशीकृतानि, कायस्थैस्तावल्लेखभण्डारस्य(\enginline{archives}) साधनानि वशीकृतानि। एवमेव लेफेब्र् (\enginline{Lefebre}) नाम्नोऽपरस्य चोक्तिमप्यसावुपस्थापयति यज्जगतो महत्याश्चाक्रिक्या वृत्तेर्मात्रस्य सर्वदाकलनमित्येतद्धेतुकैवेदृक्षस्य व्यतिकरस्य यदन्वेवेतिहासस्य समग्रस्यापि पौराणिककथास्वेवान्तर्भाव\break इत्यपि च। प्रचुरोऽप्ययमभिप्रायो न तावांस्तृप्तिकर – इति पुनःप्रब्रुवाणःपोल्लाकस्स्वकीय-\break मौदार्यमप्युपस्थापयन्निव लक्ष्यते~। अपरमपीदृशमेवौदार्यमस्याधिभारतान् यन्नाम नीट्शे\break (\enginline{Neitszche})-नाम्नश्चिन्तकस्य लपितस्योपन्यसनं यच्च नाम जीवन्ति नाम पशव इतिहासपराङ्मुखाः, मनुष्य एव खलु पराकुर्यात्प्रतिक्षणमभिवर्धमानं प्राचीनकालीनं भारमिति~। अर्थाच्चात्राक्षिप्यते यन्नात्यन्तम्भिन्ना भारतीया-श्चतुष्पाद्भ्य इति!~।

भारतीयानां विश्ःय इत्थमपलापपरम्परामेव प्रभूतां परिवाहयन् पोल्लाकोऽ-स्मदीयचिन्तनासर\-णितो नानतिरिक्तां वर्तनीं दधतां ग्रीकाणां विषये तावदल्पामॆवापलापिकीं शब्दझरीं वाहयतीति तु चित्रमेव~। लान्गिनसस्तु (\enginline{Longinus}) भेदमेव न विदध इतिहासकारनाटककारयोर्मध्य\break इत्युल्लिखत्यपि स्वयं पोल्लाक एव~। जनगृहीतिस्तावदयथार्था यतो हि प्राक्तने काले ग्रीकदेश\break ऐतिहासिकं वस्तु न तत्त्वज्ञानस्याभूद्विषयो, नापि मतचिन्तनस्य, न वा सांस्कृतिकपरिशीलनस्य~। नाप्नोति स्म तात्त्विकचिन्तनप्रसङ्गे वा साधारणजनचिन्तनप्रसङ्गे वैतिहासिकी काचिद् विचारणेति भणितिःपोल्लाकस्यैवेति वेदनीयम् (\enginline{Pollock 1989:605})***** न चेदमविदितं यद्ग्रीकरोमकानामैतिहासिकासु कथासु दैवतानामपि विलसितानि तत्र तत्र गोचरीभवन्त्येवेति~। (\enginline{“contrary to accepted belief, the idea of history did not constitute in itself an important philosophical, religious or cultural question in antiquity, and that history was largely marginalised in both philosophcal and popular thought}). मेकिन्टैर् (\enginline{MacIntyre})नामक ऐतिहासिकोऽपि अरिस्टाटल्-(\enginline{Aristotle})-प्रभृतिषु ग्रीकचिन्तकेषु निश्चिततयैतिहासिकमिदमिति दर्शयितुं किमपि न पार्यत इत्याहेति पोल्लाक एव सूचयति! एवमेव बोएर्-(\enginline{Boer})-अभिधोऽप्यनया भङ्ग्याऽऽह –\break पौरणिकैतिहासिकयोर्भिदा न स्फुटा ग्रीकलिखितेषु~। पुराणेष्वेव देवा इतिहासेष्वेव मनुष्या इतीदृशो विषयविभागोऽपि नाभिलक्ष्यते नामेति~। पोल्लाकः परमधोगतटिप्पण्यामेव निक्षिपति विष्यममुमिति सजागरमभि-वीक्षणीयम्।

अत्रान्तरे विषयान्तरं शाखाचङ्क्रमणन्यायेन प्रविविक्षुः पोल्लाकः स्टैटेन्क्रानस्य (\enginline{Steitencron}) सिद्धान्तमावाहयति~। यश्चेत्थम् – सप्तमाष्टमनवमशतकेषु कैस्ताब्देषु पल्लवशिल्पेषु शिवस्य\break गङ्गाधरमूर्ते रूपाणि झडिति लभ्यानि संलक्ष्यन्ते~। तच्च कुतः? – इति पृच्छन् स्वयमेवोत्तरयति स्टैटेन्क्रान्।

गङ्गान् (- इत्युक्ते गङ्गाभिधान् महीभृतः) पल्लवनृपाः निर्जितवन्तः~। तत्स्मारणार्थम् प्रक्रान्तं सत्, पल्लवेतिहासमेव तच्छिल्पं विलिखतीव भाति - इत्याह स्टैटेन्क्रान्~। वस्तुतस्त्वनेके कवयोऽपि पर्यायोक्तभङ्ग्या वा समासोक्तिनिरूपितकेन वा तथाविधानि स्वकालिकानि घटितानि श्लोकेषु रूपयामासुरेव~। राज्ञोऽग्निमित्रस्य वृत्तमेव स्वीये मालविकाग्निमित्रे नाटके कालिदासो निरूपितवानित्यपि किल सम्भाव्यते? किं चातः? ऐतिहासिकं विषयं कमपि प्रकाशयितुं न पारयेयुर्भारतीयाः – इति मन्वानस्य पोल्लाकस्य मनो यन्न प्रतीयात् तद्धि परीक्षणीयत्वेनावशिष्यत इति~। लौकिकमपि विषयं दैविक घटनान्विततयैव भारतीयाः प्रतिपादयन्तीत्याक्षेपणीयत्वेन विलिखति पोल्लाकः~। अत एव च भारतीयसंस्कृतेः निगूढ ऐतिहासिकभारोऽवतारणीयः – इति स घोषयति।

संस्कृतग्रन्था हि भूयस्त्वेनाश्चर्यकारित्वेन च ग्रन्थकर्तुर्नामादिकं न बिभ्रति~। यद्वा कर्तृभिन्नं नामान्तरमपि दधति~। एवं चार्थशास्त्रकामशास्त्रालङ्कारशास्त्रवेदान्त-शास्त्रादिग्रन्थास्सर्वेऽप्यैतिहासिकीं परिस्थितिमननुलक्ष्यैव प्रमेयाणि स्वकीयानि प्रतिपिपादयिषन्ति~। अतो हि हेतोः परस्सहस्रं\break पुटानां पठन्तोऽपि संस्कृतग्रन्थ-राशावातिहाससंबद्धतया तत्तत्पुरुषाणां तत्तत्स्थलानां तत्तद्घटितानां वा परामर्शो न लोचनगोचरीभवति।

स्वेलेखे विषयानीदृक्षान् परिलक्ष्याधिमीमांसाशास्त्रं शस्त्राहतिविधाने प्रवणतामेति पोल्लाकस्य\break मानसम्~। स च ब्रूते “पारम्परिकसंस्कृतसंस्कृतौ नामेतिहासस्य साधारण एवाभावो दरीदृश्यते\break यच्चानुपमितं विस्मयावहं समस्यात्मकं च “~। निदानमस्य सर्वस्य मीमांसाशास्त्रगतत्वेन स\break विभावयति~। तस्य तर्कस्तावदेवम्प्रकारकः – ब्राह्मणानां शास्त्रमिदं यन्मीमांसा नाम, सांस्कृतिकान्\break विधिनिषेधान् साहि विदधाति; यश्चेतिहासोऽस्माकमधि-जिगमिषाया विषयस्तस्यैव प्रत्याख्यात्री सा~। इतिहासाध्ययनमेव व्यर्थमिति वा ज्ञानविरोधीति वात्यप्रस्तुतं शास्त्रज्ञानसम्पादनलिप्सोरिति वा प्रतिपादयति सा~। वाक्यार्थविचारो हि मीमांसाया लक्ष्यम्~। तत्रापि धर्मो हि विषयो मीमांसायाः~। धर्मश्च पुनःप्रत्यक्षानुमित्योरविषयः~। धर्मनियमा यत्रोदितास्ते हि ग्रन्था अतीन्द्रियाधारकाः~। शब्दार्थयोस्सम्बन्धनित्यत्वं वेदानामपौरुषेयत्वं श्रुतेरनादित्वमाम्नायानामविदितकर्तृकत्वमित्यादयस्सर्वे विचाराः मीमांसकैः प्रस्तुताः~। तेषां तर्हीदृशप्रस्तावोऽपि लक्ष्यं च कञ्चिद् अधिकृत्यैव भवेत्।

वेदेषु सन्ति हि नामान्यृषीणां विविधसूक्तैः सम्बद्धानि~। किन्तु ते मन्त्रकृत इति न गण्यन्ते~। किं तर्हि वेदग्रन्थपरम्परारक्षका इत्येवावन्मात्रम्~। न सन्ति वेदेषूल्लेखा ऐतिहासिकानां पुरुषाणाम्~। निरुक्ताख्य उपायश्च मीमांसकानामत्यनुकूलः सञ्जातो यतो हि तत्रत्या ऐतिहासिका उल्लेखा अपि सदातनानां सत्यानामेव निर्देष्टृत्वेन व्याख्यातुं शक्यन्ते~। निरुक्तग्रन्थेऽपि यद्धि पुनरैतिहासिकं व्याख्यानमिति निरूपितं, तदपि नाममात्रेण विहितम्~। आध्यात्मिकदृशा सामासोक्तिकश्लैषिकरूपितकमेव प्राधान्यमापन्नं तत्रेति।

इदमाकूतमस्य पोल्लाकस्य यद्वेदेष्वैतिहासिकोल्लेखानां रिक्तीकरणं यर्हि समासादितं तर्हीतिहाससम्बद्धविषयतिरोधापनानुगुणमेव सत्यख्यापनं कर्तव्यतयाऽपन्नम्~। (पु.६०९) वास्तविकघटितकान्येवाधारीकृत्य निरुक्तस्था ऐतिहासिका व्याख्यातुं प्रायतन्त~। परन्तु तेषां न कोऽपि ग्रन्थोऽवशिष्टो लक्ष्यते।

यावती वै संस्कृतिः तावती वेदमयत्वेनैव निरूपिता वर्तते – इत्यत्याह याचयावतीचत्वेन पोल्लाकः~। नयेन ह्यनेन विद्याजातं समस्तमपि भारतीयानां वेदानुगुणतयैव निरूपणीयतया प्रतिपन्नम्~। मनुस्मृतौ चापि सर्वज्ञानमयो हि स (२.७) इति वचनेन वेदानां सर्वज्ञत्वं प्रतिपादितम्~। वेदानामनन्तत्वं चानन्ता वै वेदा इति तैत्तिरीय-संहिताया वचनेन समाम्नातम्~। उत्तरोत्तरे काले भवास्सर्वेऽपि ग्रन्था नाना-शास्त्रका वेदराश्यन्तर्भाविततयैव विभाविता वर्तन्ते~। तच्च स्वस्यैव वेदत्वप्रख्यापनेन यद्वा वेदसंक्षेपकत्वेन यदपि वा वेदोदिततत्त्वजातनिष्पादितत्वेन~। अग्निपुराणं वा भवतु रामायण-महाभारतादिकं वा भवतु पञ्चमवेदत्वेनैव व्यपदिदिक्षन्त्यात्मानम्~। वेदानां वेदमिति छान्दोग्योपनिषदीतिहासपुराणे समकक्ष्यतया लक्षिते स्तो ननु~। न्यायसूत्रभाष्य (४.१.६१) इतिहास-शब्देन वास्तविकघटितजातमेव यद्यपि निर्दिष्टं, तथापि यद्धि सदातनं तस्यैव ग्रन्थरूपेणाविष्करणत्वेनैव पर्यवसन्नं तत्~। मीमांसा हि व्यक्त्यपेक्षयाऽऽकृतिमेव ननु पुरस्करोति~। इदमपि च तत्तत्कालघटितत्वापेक्षया सना पुनरुक्तस्यैव तत्त्वस्य तुलनामारूढम्~। रामायणमहाभारतादीनां व्याख्यानमप्यध्यात्मपरत्वेनैव साधारण्येन विवक्ष्यते ननु? यथा नाम नीलकण्ठेन महाभारतव्याख्याप्रसङ्गे~। चतुर्दशविद्यास्थानस्य तात्पर्यं समग्रस्यास्य भारतकाव्यस्य च तात्पर्यं च सारतोऽभिन्न एवेत्यैकाशयत्वमनयोरविप्रलपनीयम्~। महेश्वरतीर्थगोविन्दराजौ श्रीवैष्णवपरम्परापरावप्यधि रामायणमित्थमेव प्रवृत्तौ लक्ष्येते।

अयं तर्हि पोल्लाकस्य सिद्धान्तो यदितिहासोऽपि नाम नात्यन्तमनव-स्थितस्संस्कृतवाङ्मये निरूपिते भारते~। किन्तर्ह्यन्यसत्यापेक्षया तिरस्कार्यत्वमापन्नो यत्रैतिहासिकस्य सत्यस्य नाम न मौलिकं किमपि ज्ञानदृष्ट्या प्रयोजनम्, न वा सामाजिकं किञ्चन प्रयोजनम्~। एवञ्च तस्य सिद्धान्तो यद्भारते \enginline{system} (“व्यवस्था”) – इत्यस्यैव स्थानम्, न पुनः \enginline{process} (“क्रिया”) इत्यस्य~। अर्थात् सामाजिकी या व्यवस्था तस्या एव स्थानं, न पुनर्मानवस्य सर्जनात्मिकायाः प्रवृत्तेरित्यस्याकूतम्। अस्य चानुगमश्चेत्थम् यदैतिहासिक्यः परिणतयः (\enginline{transformations}) पूर्वकाले चोत्तरकाले च निराकृता भवन्तीति।

इत्थं पोल्लाकवादजालं पुरो विन्यस्य तद्विमर्शनकार्य आत्मानमधुना व्यापारयामः~। नहि सर्वे भारतशास्त्रविदो (\enginline{Indologists}) भारतीयनागरिकताविषये न्यक्करणतत्पराः~। परं दोषानाविद्धं किमपि नास्त्येव पुनर्भारतीयनागरिकतायां पोल्लाकस्य दृष्टौ प्रायेण~। दक्षिणेष्याभाग एष जनशोषणभूमिरमुष्य मते~। परन्तु शृण्वन्तु ए एल् बाषाम् (\enginline{A L Basham}) इत्यस्य व्यतिरेरिचानं वचनमिदम् – जगत्यन्यत्र न क्वापि प्रजानां पारस्परिकः सम्बन्धः प्रजानां राज्यस्य च सम्बन्धश्चैतावान् न्याय्य आसीदेतावांश्च मानुष्यभर: (\enginline{humane})। नान्यत्र नागरिकतायां दासानां सङ्ख्या तावत्यल्पा वासीत्तथाच न क्वाप्यन्यस्यामाद्यायां नागरिकतायां कस्यामपि जनानामधिकाराणां (\enginline{rights}) तादृक्षं समीचीनं संरक्षणं यथा अर्थशास्त्र इह कौटलीय इति~। रणाङ्गणे धर्मयुद्धप्रकारश्च यथा मनुना घोषितस्तथा न क्वाप्यन्यत्रापीषदपीति।

यत्तु पोल्लाकः साक्रोशमुदगिरत्सर्वं वेदसाद्विहितमत्रेति यच्च क्रैस्ता मतान्तरकरणपरायणा\break जगर्जुर्यद्दुर्भिक्ष-दूरोग-(वर्णव्यवस्थाख्य)दौर्जन्य-प्रभृतिभिश्शोषितास्समभवन्दुःखदौर्मनस्यभरि\-ताश्च प्रजा अत्रत्या इति – तस्योभयस्यापि प्रत्याख्यानं बाषमेनैव प्रत्तमस्ति (\enginline{Basham\general{\break } 1967:9}) यज्जनास्स्सम्यगेव नूनमनूनं सौख्यमन्वभवन्नैन्द्रियिकाणाम् अतीन्द्रियिकाणामुभयेषामपीति~। अथ च स्वीये भारतीयार्थशास्त्रमतानाम् इतिहास(\enginline{History of Indian Political Ideas}) इत्यभिधे पुस्तके घोषलाख्यो जुघोष (\enginline{U N Ghoshal}) यत्प्राचीनभारतीयवैलक्षण्य\-निरूपकलक्ष्मत्रयमभिलक्ष्य प्रोक्तमरविन्दे(\enginline{Aurobindo}) यन्नाम – प्राथम्येनाध्यात्मिकता\break भारतीयानाम् यच्च तेषां चित्तस्य वैशिष्ट्यस्य द्योतकम्; द्वितीयं तेषां जीवनोत्साहोऽदम्यो\break यदुत्था प्रभूता सर्जनशीलता; तृतीयमन्तिमं च दृढा मनीषिता यत्र नाम प्रागल्भ्य-मार्दवे सहैव स्तः, सारल्याढ्यत्वे चापि सहैव स्त - इत्यादिकम्।

यत्तु पोल्लाकेन लपितं वेदस्य यत्प्रामुख्यं प्रत्तं तेन धर्मज्ञानस्यान्यद्वाराणि निरस्तानि सन्तीति तदपि व्युदस्तं वेदेनैव~। यदाह श्रुतिः – श्रुतिः प्रत्यक्षमैतिह्यमनुमानश्चतुष्टयम् (तैत्तिरीयारण्यकम् १.२.१) इति~। वचनेनानेन वेदोक्तमात्रस्य प्रामाण्यं प्राधान्यं वा, प्रत्यक्षानुमानयोर्नावकाशप्रसङ्ग इति वा वादस्तस्य निराकृतो भवति~। एवं चैतिह्यस्यापि स्थानं दत्तमस्तीति हेतुना यच्छब्दोक्तं तस्यापि प्रामाण्यमूरीकृतं लक्ष्यते – यदाह सायणो भाष्ये स्वीय ऐतिह्यं विवृण्वन्नैतिह्यं नामेतिहासपुराणमहाभारतब्राह्मणादिकम्~। एवं ज्ञानद्वाराणां समेषां स्थानं यथोचितमुपपादितमेव लक्ष्यते~। मन्त्रस्यास्य भावं विवृण्वान आह सायणः – तदेतत् स्मृत्यादिचतुष्टयमवगतिकारणीभूतं प्रमाणम्~। इति।

मार्क्सवादानुयायी पोल्लाकः मार्क्सवादाभिघातकमभिप्रायमेवमभिलपतीत्यपि विस्मयस्यैव\break विषयः~। “यद्धि नाम तार्किकं तद्धि स्वस्मिन्नेवैतिहासिक-मन्तर्भावयति (\enginline{The logical contains within itself the historical}) इति मार्क्सवादिनां सूत्रम्।

इत्युक्ते मार्क्सवादिनामयमाग्रहो यन्मार्क्सवाद एव वस्तुतत्त्वानुसारी~। आतश्चेतिहासस्सर्वोऽपि मार्क्स्वादिनां नयमेवानुसृत्य घटिष्यत – इति~। मार्क्स्वादसिद्धान्तानुसारमेव खलु जगति सर्वं प्रसिद्ध्यतीति! मार्क्स्तर्कस्य महिमाऽयं यद्राज्यं समाज इत्यादिकं समस्तमपि मार्क्स्तर्कमेवानुरुणद्धि~। पश्चात्काले तु धनिकाराधनरूपः (\enginline{capitalism}) सिद्धान्तो नङ्क्ष्यति समाजवादश्च भृशं विलसतीत्यादि सर्वं मार्क्स्वादादेव सेत्स्यतीत्याह फ्रोलोव् (\enginline{Frolov})नामकः~। वस्तुतस्तु तत्सर्वं नैव जघट इति विदन्त्येव विद्वांसः।

इतिहासपुराणानां यदान्तरिकार्थपरिकल्पनं तात्त्विकार्थविभावनं वा (यदेव \enginline{allegorical\general{\break } interpretation} इति कथयन्ति), तदधिकृत्य स्वामसम्मतिं दिशति पोल्लाकः~। वेदमन्त्राणामर्थत्रयमाहुर्वेदव्याख्यातार आधिभौतिकमाधिदैविक-माध्यात्मिकं चेति तावदास्ताम्~। नानास्तरीयव्याख्यानं तावत् क्रैस्तेष्वपि वर्तत एवेति पोल्लाको ज्ञापनीयः।

अथ पोल्लाकेनोत्थापितानामनेकेषां प्रश्नानामुत्तराण्यानन्दकुमारस्वामिनो (\enginline{Ananda\general{\break } Coomaraswamy}) लेखेषु लभ्यन्ते~। वेदा वा तदङ्गभूतानि शास्त्राण्यन्यानि वा भगवतो\break निःश्वसितानीति वा व्याहृतय इति वा निर्दिश्यन्ते ननु। ते चादावृषिभिः श्रूयन्ते~। ऋषीणामपि श्रवणं स्वस्फूर्त्यपेक्षया-प्यन्तस्समाहिततानिबन्धनम्~। वाल्मीकिरपि सर्वं रामायणं योगदृष्ट्या\break विलोकयति यत्र चिरनिर्वृतमपि तत् प्रत्यक्षमिव दर्शितं भवति~। तादृशस्य प्रतिभानस्य मूलं\break चर्ग्वेदेऽप्यक्षिलक्षीभवति~। सन्दर्भेऽस्मिन् ब्लूमफील्डस्य (\enginline{Bloomfield}) अभिप्रायमानन्दकुमारस्वामी पुनरुच्चरति~। मन्त्रब्राह्मणे भिन्नकालिक इत्याधुनिकानामाग्रहः खलु~। तयोः भिन्नकालिकत्वमकिञ्चित्करम्~। वस्तुतस्तु वाङ्मयस्यैव प्रकारद्वयमिवाभाति तत्~। तच्च प्रकारद्वयं समकालिकतयैव विभावितमा च बहोः कालात्~। एतच्च सर्वं तु ब्रौनाभिमतेनात्यन्तं भिन्नम् (\enginline{Norman Brown}) यस्त्वाह यदुत्तरस्मिन् काल ऋग्वेदे वस्तुतोऽनुल्लिखितानामेव विषयाणामनुसरणं भवतीति~। उपनिषत्स्वपि नूतनास्सिद्धान्ताः नाविष्क्रियन्ते इत्याह कुमारस्वामी, किंतर्हि नूतना विभिन्ना वा शब्दावलिरेव तत्र प्रयुज्यमाना लक्ष्यते~। उदाहरणार्थं यं वरुणमाहुर्वेदे तमेव ब्रह्माणमुत्तरस्मिन् काले जगदुः।

एतावता भाषिकी विहिन्नता परिष्कृतता वा नास्त्युपनिषत्सु वैदिकापेक्षयेति नास्मन्मतम्~। तथास्थातुं न केनापि प्रेक्षावता शक्यते~। साहित्येतिहासोऽ-परस्तत्त्वशास्त्रीयेतिहासोऽपरः~। यदेव वेदेष्वधियज्ञमभिहितं, तदेव ब्राह्मणेषूपनिषत्सु चाध्यात्मपरतयोपदिष्टं भवति~। न ह्यधियज्ञे निरूप्यमाणे साहित्येऽधितत्त्वमपि तावत्यैव स्फुटतया निरूपितं भवत्वित्यास्थातव्यम्~। वैदिकं नाम वाङ्मयमतिविस्तृतं सदपि नह्यान्तरिकः कोऽपि विरोधस्तत्र कुत्रचिदधिगम्यते~। यथा ब्लूमफील्ड आह – वेदस्य सर्वोऽपि भागः सर्वमितरं भागं सम्यगेव वेत्ति, सर्वेण च तेन सुसम्बद्ध एव संलक्ष्यत इति~। यज्ञेभ्यस्तत्त्वानि तत्त्वेभ्यश्च यज्ञान् अवसातुं नाशक्यम्।

अर्थाद्ये हि नाम मन्त्रकृतो मानुषा वा अतिमानुषा वा स्वोक्तिभिरवसेयानंशान् सम्यगेव प्रत्यपद्यन्तेत्येव वक्तव्यं भवति~। नो चेद् गणितसूत्राणि बहूनि केनचन कथञ्चिदज्ञात्वैव विलिखितानीति ब्रुवतो वचने यस्साहसस्स एव वक्तव्यस्स्यात्~। तथापि च तस्य भाषिकी तात्त्विकी च स्फूर्तिरभ्युपगन्तव्या भविष्यति~। वेदेषु ताक्षिकं ज्ञान(\enginline{knowledge of carpentry})मभिलक्ष्यत एवेत्यनेन हेतुना, लोकेऽपि तादृशस्य ज्ञानस्य पश्चादेव तथा वचनं शक्यसंभवमिति हेतोश्च तत्रत्यं साहित्यं मानुष्यकमिति चैतिहासिकमिति चाभ्युगन्तव्यमेव~। सनातनस्य धर्मस्य सनातनत्वं नाम नहि तद्गतानां शब्दानां तथात्वेनाभिसन्धीयते किंतर्हि तत्रत्यानां तत्त्वानाम्।

ऐतिहासिकस्य क्रमस्य विषयेऽप्यानन्दकुमारस्वामी वक्तुमेवमभिलषति यत् तात्त्विकं नाम विभावनं पाश्चात्त्यैरादृतेन क्रमेण तावद् विवर्धमानं लक्ष्यते~। किंच साधारण्येनोच्यमाने, बहुत्र चापि, पारम्परिक्युक्तिरर्वाक्काले तस्योल्लिखिततामेव सूचयति, न पुनस्तस्य प्राथमिकतामाविष्कारस्य, यस्मादपि पूर्वं तस्य मौखिकः प्रचार एव वरीवर्ति स्म, येन च हेतुना तस्य चायमेव कर्तेत्यास्थातुं प्रायेण नैव शक्यं स्यात्~। अस्मिन् विषये रेने ग्वेनोन् (\enginline{Rene Guenon}) इत्यस्य मेधाविनोऽभिप्रायाणां साङ्गत्यं स निरूपयति।

अन्येभ्यो देशेभ्य आगता आर्या – इति कश्चन पाश्चात्त्यो वादो वर्तते खलु! तमधिकृत्य ब्रुवन् कुमारस्वाम्याह यत्तादृशं लौकिकीकृतं व्याख्यानं (यस्य \enginline{euhemeristic interpretation} इत्यभिधानं वर्तते) तावन्मात्रं स्यात्, यस्य तु वस्तुत ऐतिहासिकस्सारो न कश्चिद् वर्तते~। वैदिकोक्तकथानामनुहारिणीरैतिहासिक्यो घटना न जात्वसम्भवा इति नास्माकमभिप्रायो यतो ह्यैतिहासिकमपि नाम वस्तु तात्त्विकमेवानुहरति~। नेदमप्यतथ्यं यत्तत्त्वैकपरेऽपि हि साहित्य ऐतिहासिका अंशा दुरूहा इति~। यज्ञकार्याणि कुर्वाणैर्मन्त्रगानं च कुर्वद्भिर्वैदिकैरश्वा वा रथा वा न विदिता इति वक्तुं न पार्यते हि, न वा तैर्नानुभूतं नदीनां समुद्राणां वा तरणमिति, न वा कृषिस्तैरविदिता चेति।

कुमारस्वामिना तावदिदमास्थितं यदृग्वेदादिषु मूलग्रन्थेषु ऐतिहासिका एव विषया निरूपिता इति न वक्तुं पार्यते~। किं तर्हि “अग्रे” इत्युक्तदिशा तात्त्विकेन प्रकारेण~। “अग्रे” इति तु तात्त्विकं न वास्तविकम्~। जीवनं हि नाम सर्वदा तरणमेव, सर्वदापि कुतश्चिदिहागमनमेव, इतश्च परमं पदं प्रति प्रस्थानमेव~। पूर्वमीमांसाया आशयो नामेदृश एव, परन्तु स्वतन्त्रेण प्रकारेणात्र निरूपितः इति स्वलेखं समापयति कुमारस्वामी।

इतीत्थं नाम पोल्लाकवादानामुत्तराणि स्वतःपूर्वपक्षीकृत्येव पूर्वकालीनेन कुमारस्वामिना स्वलेखेषूत्तरितानीति शम्॥

\egroup

