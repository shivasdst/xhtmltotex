
\chapter{The Science of Meaning - Explicating the Nature of Philology and its Implications to \textit{Videśī} \index{videsi@videśī} (Foreign) Indology}\label{chapter5}

\Authorline{T. N. Sudarshan}


\section*{Abstract}

Philology is normally defined as the study of language in written historical sources. It is also likened to the science which teaches us what language is. The creative and novel approach to philology\index{Philology} and its praxis, its usage as a trustworthy tool of scholarship by the (\textit{videśī}-s\index{videsi@videśī}) neo-Orientalists and its practitioners like Sheldon Pollock\index{Pollock, Sheldon} are examined in this paper for their veracity and scientific nature. A survey of philology as it has evolved and practiced in the West is presented. The notion of meaning, the \textit{raison d’être} of philology, is then discussed as the principal focus of this paper. It is proposed that the understanding of the nature of meaning drives its methods and its study. What exactly is meaning in an Indian sense? What are the traditional approaches to explore meaning in Indian knowledge systems? What is meant by the meaning of a text (\textit{śabda}\index{sabda@\textit{śabda}})? These are juxtaposed with Western philosophical notions of meaning and also notions of meaning in the sense of modern techniques of representing and deriving meaning computationally, i.e. those used in areas of AI\index{Artificial Intelligence} (Artificial Intelligence)\index{Artificial Intelligence}. Exploring the notions and ideas of meaning as derived from “text” specifically, the limitations of the Western conceptualizations of textual meaning are explicated. The nature of study of such meaning via techniques such as Philology is shown to be absolutely flawed in a very fundamental way owing to the inherently limited understanding of meaning itself. That methods like Western Philology\index{Philology@Western} and its modern creative and agenda-driven avatars like Pollock’s\index{Pollock, Sheldon} rather ludicrous three-dimensional philology\index{Philology@ three-dimensional} are inherently baseless and deeply flawed is posited. A fervent call is made (on the basis of a need to retain sanctity and restore sanity of academic scholarship) to question its (\textbf{philology}\index{Philology@Western}) existence as a tool of twenty-first century scholarship.

\textit{\textbf{Consequences of philology:} arrogant expectations; philistinism; superficiality; overrating of reading and writing; Alienation from the people and the needs of the people.... }

\textit{\textbf{Task of philology:} to disappear.}

—Friedrich Nietzsche\index{Nietzsche, F.}, “We Philologists” - Autumn 1874


\section*{Introduction}

The nature of philology, its origins and the nature of its evolution as part of Western civilizational scholarship is briefly examined. The nature of the new avatars of Philology including that of Sheldon Pollock\index{Pollock, Sheldon}, their motivations and limitations, are also explored. We discuss the different origins, motivations and nature of the theories of language and the role they play in shaping the character of a civilization and its peoples. This is followed by a section discussing and dissecting the Western (up to present-day theories) as well as the traditional Indian theories of language\index{Theories of Language@Western} \index{Theories of Language@Indian}. We then examine and try to understand the nature of meaning and the various theories of meaning\index{Theories of Meaning@Western} \index{Theories of Meaning@Indian} according to Western (social-science) ideas and also those of the traditional Indian schools of thought. The nature of meaning - in a computational sense - is also examined and juxtaposed with scientific nature of meaning and the social-scientific sense of meaning. The brief comparative analysis of the nature of language and meaning leads us to the discussion of its implications. Are Western methods like philology, given their roots, valid academic tools to study Indian texts? And is it appropriate or academically sane\index{sane} to base civilizational commentaries on such dubious methods? The fact that they are being used (since 250 years) says more about the deep motivations of the West and of Western scholarship more than anything else. The validity of philology\index{Philology} in the context of modern science and the nature of its “scientific method” is also examined using Western sources and commentaries. The reasons why philology is a deeply flawed and untrustworthy method to base scholarship on - in a Western traditional (historical) sense, in a modern Western (humanities) sense, in a scientific sense and not surprisingly both in a traditional Indian sense and in a modern Indian sense — are briefly discussed. The implications of this discussion to neo-Oriental scholarship and to other such “schools” of academic pursuit are also discussed. We conclude positively, exhorting scholars, asking them to refrain from using such fundamentally erroneous and imperfect academic techniques and methods in the pursuit of theory-building, career growth and generation of divisive social/societal/civilizational commentary.


\section*{On Philology}

The roots of philology\index{Philology@roots of} and its deep influence on humanities, though known, has not been discussed in detail till the recent book by Turner\index{Turner, J.} (2014). Turner’s explicit thesis running throughout this book is that all of Humanities as we know it today traces its origins to philology, which he characterises as the multifaceted study of texts, languages and the phenomenon of language itself.

Turner also indicates that the nineteenth century avatar of philology gave birth to Comparative Linguistics\index{Comparative Linguistics} - a pseudo-scientific pursuit\break used as basis for racist and imperial colonial goals by Europeans (colonialism via comparative linguistics) and by the Americans (slavery via biblical philology). The creation of pseudo language families and the theories of historical evolution of languages was deeply influenced by the methods of philology. The obsession with historical comparisons (historicism\index{historicism}) more of which we shall discuss later - driven by the aforesaid motivations of a colonial and racist nature underpinned\break non-empirical philological scholarship. The obsession with manufacturing history and interpreting past events is also highlighted by\break Turner\index{Turner, J.}. Whether we realize it or not, \textbf{all} of modern humanities is inherently colored by these deeply divisive motives.

With the advent of empirical approach to science and the notion of \textit{experiment} as proof and with the maturing of the scientific method - philology lost its importance, as noted by Turner\index{Turner, J.}. Though supposedly defeated, humanities as a discipline still lives on, and so do the pseudo-sciences via the social-sciences. There have been enough reviews, articles, and scholarship discussing the relevance of humanities in modern education / society - but the fact that it is still living and breathing (in India) though it is supposedly dying in the West (especially so in the USA) shows that it is relevant (for whatever reason) and not redundant. For a deeper meta-perspective on the relevance of analysis of the humanities (vis-à-vis philology\index{Philology}) read (Turner 2014):

\vskip 3pt

\begin{myquote}
“Because philology’s legacy survives in ways we build knowledge today, the excavation of the philological past becomes an effort at once of historical reconstruction and \textbf{\textit{present-day self-understanding}}. When we see where our modes of knowing came from, we grasp better their strengths and weaknesses, their acuities and blind spots. I hope that a \textbf{broad view of the philological heritage will help us to detect these things more easily, to locate ourselves more securely on the map of knowledge}, and thereby to improve our future investigations.”

~\hfill (Turner 2014: xiii) (emphasis ours)
\end{myquote}


\section*{Western Philology}

\vskip 3pt

As most of these “narratives” go, the origins of philology\index{Philology@origins of,} (not surprisingly) go to the Greeks. Along with their supposed speculations and achievements on language and its origins, what also arose in parallel was, “debate” or rhetoric and the ability to argue skillfully in public. As noted by Turner, rhetoric\index{rhetoric} was primarily political in nature and driven by individual fancy, noted even during these early times. Some European philologists also ventured eastward – these were the beginnings of Orientalism\index{Orientalism}. The search for identity of national pasts was kindled all across Europe and philology was put to use in all these national past-times. Philology\index{Philology} in its political avatar had truly come of age. Seventeenth and eighteenth century philology was influenced by British colonization of India. William Jones\index{Jones, Sir William}, in his third anniversary discourse to the Asiatic Society in 1786, commented on the nature of Sanskrit and recommended its study to advance European ideas of language. Jones\index{Jones, Sir William} introduced a new dimension to philology in addition to the already existing pre-occupation with the political - that of race.

James Turner\index{Turner, J.} acknowledges this as a pivotal moment in the history of philology. For the people of the Indian subcontinent this was a critical moment too in their histories, the resounding echoes of which are still loud and clear, the neo-Orientalists.

\begin{myquote}
“Out of the marriage of European philology and Indian texts, then, came two new fields of knowledge: Indology and comparative philology (or comparative grammar). The first offered no new methods; earlier scholarship had forged the techniques there applied. Its novelty lay in opening to European eyes a civilization previously obscure. Indology became the first field in which a \textbf{\textit{self-perpetuating cohort}} of European scholars—not the odd missionary, merchant, or chronologer—worked systematically to uncover the riches of a non-European civilization across a wide front. In a narrow, academic projection, \textbf{\textit{Indology foreshadowed area-studies programs}} in post-1945 universities. In a broad, cultural view, Indology immensely expanded European perspectives on the history and civilizations of the world.”

~\hfill (Turner 2014:99) (emphasis ours)
\end{myquote}

In his 2009 book, \textit{Representations}, Geoffrey Harpham\index{Harpham, G.} discusses the so-called “Returning to Philologies” and quotes Edward Said\index{Said, Edward}, Paul De Man\index{De Man, Paul} and Nietzsche\index{Nietzsche, F.} where their deep disdain of philology is demonstrably well-articulated. Quoting Nietzsche, he writes

“In “We Philologists,” written in 1874, Nietzsche registered his contempt for most philologists, whose work impressed him as an \textbf{\textit{absurd combination of inconsequentiality and hubris.}}

\begin{myquote}
“... Philology\index{Philology} was, however, also understood in very different terms, not as an empirical study of a limited field, but as a speculative undertaking oriented toward deep time and distant things.”

~\hfill (Harpham 2009:37) (emphasis ours)
\end{myquote}

This imagination (identified above) expanded and became all-encompassing as the philologist grew bolder and applied his speculations in what can only be characterized (to a scientific sensibility) as truly wild and unlimited ways. The dubiousness of the “method” and the inferences drawn based on these speculative and non-empirical methods was truly something to be dreaded.

\newpage

\begin{myquote}
“In short, the fear voiced by Said and de Man—that \textbf{\textit{critics unmoored from philology might indulge in statements about vast structures of power or the general context of human history}}—was for nearly a century and a half proudly announced as the defining characteristic and entire point of philology itself.”

~\hfill (Harpham\index{Harpham, G.} 2009:40) (emphasis ours)
\end{myquote}

This speculative and dangerous “scholarly pursuit” was used in the nineteenth century to justify the horrors of racism (Mueller\index{Max Muller, Friedrich}), slavery (biblical philology\index{Philology@Biblical}) and colonialism (comparative\index{Comparative Linguistics} linguistics).

\begin{myquote}
“Staking claims to the status of poetry, philosophy, and science—and to a transcendence of as such—\textbf{\textit{philology represented itself as an “untimely” form of knowledge that was completely independent of political or ideological ends.}}.. The most telling instance was \textbf{\textit{the deep investment of philology in the concept of race.}}”

~\hfill (Harpham 2009:41) (emphasis ours)
\end{myquote}

Modern linguistics got itself rid of philology by becoming more objective by rejecting it altogether. \textit{Humanities, though, are yet to be rid of it}. Philology has in fact returned in many new ways and is making its presence felt in many areas of the humanities. The neo-Orientalists led by Sheldon Pollock have invented their very own unique versions of philology such as political philology\index{Philology@political} and liberation philology\index{Philology@Liberation} as part of the scholarship in their area-studies disciplines (South Asian studies) scholarship, which investigates and examines the civilization of the Indian subcontinent with modern versions of the deeply flawed and problematic techniques of philology.

Harpham’s sobering conclusion alludes to this fact that contemporary scholarship and humanities is yet to be rid of the deep flaws of philology and that all of its historical problems still remain unaddressed in modern humanities.


\section*{Pollockian Philology}

The tools used by the principal proponent (Sheldon Pollock\index{Pollock, Sheldon}) of the (\textit{videśī}) neo-Orientalists is a version of philology reinvented as part of the pursuit of area-studies scholarship in the US academic system. After Edward Said’s influential critique on Western anthropological and social-science scholarship (Said 1978), the study of the East had to be reinvented with new methods and techniques. Breckenridge\index{Breckenridge, C. A.} (1993), in the 44th Annual South Asia Seminar at the University of Pennsylvania (1988/1989 ), in a way, marked a formal reinvention (in the opinion of many) of South Asian studies using neo-colonial lenses.

\vskip 3pt

Sheldon Pollock’s\index{Pollock, Sheldon} contribution to this volume was a paper (“Deep Orientalism”) in which he demonstrates the newly minted version of his philology. Using creative techniques and spectacularly speculative theories based on philological readings of Sanskrit texts, Pollock was able to (supposedly) reason that the Holocaust perpetrated by the Nazis to the pre-existing deep hatred and divisiveness present in Sanskrit (as a language). The thesis is that the study of Sanskrit by German Indologists affected their deep subconscious, creating ideas of Aryanism and justification for a sense of superiority.

\vskip 3pt

Pollock thus uses his critical philological study of Sanskrit and Indian texts as a response to Said’s\index{Said, Edward} critique in Orientalism\index{Orientalism} - by characterizing Sanskrit itself as a carrier of the deep seeds of racism, hatred and power, and calling it “Deep Orientalism”. Sheldon Pollock’s philology is characterized (as seen earlier in the history of philology\index{Philology@history of,}, this is nothing new) by political readings and fairly imaginative speculation, keeping alive the hegemonic discourse of Orientalism in the post-colonial era. H.H Devamrita Swami\index{Swami, H H Devamrita} of the ISKCON notes in his review of Malhotra\index{Malhotra, Rajiv} (2016)

\vskip 3pt

\begin{myquote}
“A salient point this book offers us is that the Western approach to Sanskrit is often weighed down by “political philology\index{Philology@political}”—cultural biases, hegemonic filters.”

~\hfill (Malhotra 2016:Review page)
\end{myquote}

\vskip 3pt

Sheldon Pollock’s wide-ranging work on Sanskrit and Indian civilizational history over the past 30 years has been characterized by deep political readings into India’s past and of its cultural artifacts - primarily the language of the Sanskrit and the associated texts of \textit{sanātana dharma}\index{dharma@\textit{dharma}}. His own understanding of traditional Sanskrit text scholarship is colored, and according to him Sanskrit philology was mostly tied to practices of power. See Malhotra (2016: 232) for details.


\section*{Influences on Pollockian Philology}

In his 2009 paper, Pollock\index{Pollock, Sheldon} speculates on the future of philology\index{Philology@future of,}. He offers his own definition/s of what philology is and what it should be

\begin{myquote}
“Most people today, including some I cite in what follows, think of philology either as close reading (the literary critics) or historical-grammatical and textual criticism (the self-described philologists).
\end{myquote}

\begin{myquote}
What I offer instead as a rough-and-ready working definition at the same time embodies a kind of program, even a challenge: philology is, or should be, the discipline of making sense of texts. It is \textbf{\textit{not}} the theory of language—that’s linguistics— or the theory of \textbf{\textit{meaning or truth}}—that’s philosophy— but the theory of textuality as well as the history of textualized meaning.”

~\hfill (Pollock\index{Pollock, Sheldon} 2009: 934) (emphasis ours)
\end{myquote}

The roots of Pollock’s philology can be seen in Giambattista Vico\index{Vico, Giambattista} (Pompa\index{Pompa, L.} 1975), wherein Vico (considered to be the father of modern social science, Descartes\index{Descartes, Rene} being generally considered the father of science) applies the ideas of Rhetoric\index{Rhetoric} to History. His ideas have been highly influential in the philosophy of history, sociology and anthropology, and had an influence on the so-called Enlightenment. Pollock acknowledges this influence and uses some of Vico’s constructs and applies them to Sanskrit study and his style of philology\index{Philology} of history.

\begin{myquote}
“I map out three domains of history, or rather of meaning in history, that are pertinent to philology: textual meaning\index{meaning@textual}, contextual meaning\index{meaning@contextual}, and the philologist’s meaning\index{meaning@philologist}. I differentiate the first two by a useful analytical distinction drawn in Sanskrit thought between \textbf{\textit{paramarthika sat}}\index{paramarthika sat@\textit{pāramārthika sat}} and \textbf{\textit{vyavaharika sat}}\index{vyavaharika sat@\textit{vyāvahārika sat}}— ultimate and pragmatic truth, perhaps better translated with \textbf{\textit{Vico’s verum and certum...}}\index{Vico, Giambattista} The former term points toward the absolute truth of reason, the latter, toward the certitudes people have at the different stages of their history and that provide the grounds for their beliefs and actions. Vico in fact identified the former as the sphere of philosophy and the latter as the sphere of philology.”

~\hfill (Pollock 2009:950) (emphasis ours) (diacritics as in the original)
\end{myquote}

Opening the doors for free-for-all speculative academic scholarship based on this application of Vico’s categories to Sanskrit terminologies (glaringly out-of-context), Pollock adds

\newpage

\begin{myquote}
“… the philologist’s truth, balancing in a critical consilience the historicity of the text and its reception, adds the crucial dimension of the philologist’s own historicity.” (Pollock 2009:951)
\end{myquote}


\section*{Motivations of Pollockian Philology\index{Philology}}

Given the background to the historical evolution of philology, the postcolonial predicament posed to Orientalism and Pollock’s creative reuse of Vico’s\index{Vico, Giambattista} rhetoric-based ideas of philology and mapping them to Sanskrit categories of “truth” (\textit{sat}), one is left bewildered. What could be the motivations of such devious exercises of intellect? Vico’s rhetoric is exposed in Bull\index{Bull, M.} (2013) where Vico’s analogies of the visual (painting) and the verbal (truth) lay bare the methods that he (Vico) espoused.

\begin{myquote}
“According to Giambattista\index{Vico, Giambattista} Vico, writing in 1710, human truth is actually “like a painting.” That is just what Liotard was referring to when he said that “painting... can \textbf{persuade us} through \textbf{the most evident falsehoods} that she is pure Truth.” But could painting ever be so persuasive as to persuade us that truth itself functions the same way?”

~\hfill (Bull\index{Bull, M.} 2013:xi) (emphasis ours)
\end{myquote}

Vico explains how human truth is like a painting as opposed to sculpture - making truth as he calls it

\begin{myquote}
“Divine truth is a solid image like a statue; human truth is a monogram or a surface image like a painting. Just as divine truth is what God sets in order and creates in the act of knowing it, so human truth is what man puts together and makes in the act of knowing it. The true is precisely what is made (Verum esse ipsum factum).”

~\hfill (Bull 2013: Sec 12.98)
\end{myquote}

Bull questions these claims of Vico and juxtaposes Nietzsche\index{Nietzsche, F.}

\begin{myquote}
“… but Vico’s idea of \textbf{making truth} could easily be turned against itself, for how is made truth to be differentiated from \textbf{invented falsehood?}.. Nietzsche - Error is the precondition of thought, for \textbf{\textit{“we have need of lies... in order to live.”}}

~\hfill (Bull 2013: Sec 16.4) (emphasis ours)
\end{myquote}

Vico’s pursuits of \textit{making truth} and inventing falsehoods could very well have been the basis of his rhetoric based philological methods. That Pollock\index{Pollock, Sheldon} is inspired by Vico and uses techniques derived from such an intellectual disposition is something that needs to be underlined here.

Pollock’s motivations for using \textbf{political philology}\index{Philology@political} based on Vico’s \textit{making truth} principles, and its applicative usage to Sanskrit texts is not apparent to the lay reader. Malhotra\index{Malhotra, Rajiv} (2016) explicates the motivations to the following –

\begin{enumerate}
\itemsep=0pt
\item primarily those which give control to English-speaking Indian elites who reintrepet their traditions using Western lenses,

 \item the diversion of focus away from actual practices. and

 \item an overall weakening of India’s cultural foundations by Indians who are mostly trained using these Western methods. Refer (Malhotra 2016:313) for details.

\end{enumerate}

To complement his Political Philology\index{Philology@political} (a diagnostic philology), Pollock has theorized a new form of philology. It is what he calls Liberation Philology (a prescriptive philology) (see Pollock\index{Pollock, Sheldon} 2012) where Pollock eloquently makes the case for this kind of philology\index{Philology}. Malhotra suggests an alternative, called Sacred Philology\index{Philology@Sacred}, to counter this “liberating” approach proposed by Pollock. Refer (Malhotra\index{Malhotra, Rajiv} 2016:362,363) for details.

Suggesting broader sociological and geopolitical influences, Malhotra ascribes much deeper divisive and pernicious motives to Pollock’s seemingly academic theories – those of re-engineering of Indian society using Western paradigms.


\section*{On Language}

\vskip -6pt

In the preceding discussion on philology, we skirted issues concerning language and the theories around its origins and evolution. In order to accurately understand the role of philology and its efficacy as a tool/method and to know whether philology is faithful to the notion of language, we need to understand \textbf{\textit{language}} better. Philology’s\index{Philology@origins of,} Western origins have been discussed in the previous section. In this section we shall discuss the Western theories of language, Indian theories of language and in the light of today’s advances in machine learning\index{machine learning} and AI (Artificial Intelligence\index{Artificial Intelligence}), we will also address what language means, in a computational sense. The discussion on language will help us place in context, the relevance and veracity of methods such as philology, which, owing to its origins and legacies of evolution, is not sufficiently scientific enough to warrant its role as a tool of modern academic scholarship.


\section*{Western Theories of Language}

\vskip -6pt

The Western theories of language\index{Theories of Language@Western} are primarily based on “discussions of origin”. As there is a lack of direct evidence, it is considered to be a very difficult topic of study. We will not discuss here the origins of language inspired by the philological (Biblical) lines of reasoning. The following quote suffices to illustrate the ludicrous nature of the approaches inspired by philology.

\begin{myquote}
“The concept of language families formed by genealogical descent gave students of language a novel way to classify languages and track their development. \textbf{This fresh approach retained philology’s central dogma of historical comparison.} … By doing so, \textbf{philologists aspired to retrace the history of the languages and even to reconstruct tongues long vanished from the earth.} Seventy-five years after Jones\index{Jones, Sir William} introduced the idea of a language family comprehending tongues from India to Ireland, August Schleicher\index{Schleicher, August} partly reconstructed the mother of them all, Proto-Indo-European. \textbf{Loose speculation about Adam’s language became rigorous science.”}

~\hfill (Turner\index{Turner, J.} 2014:99) (emphasis ours)
\end{myquote}

Given this basic difficulty, Western theories of language take multiple approaches, based on different assumptions – Continuity Theories\index{Continuity Theories} wherein it is assumed that there must have been simpler earlier forms of language, Discontinuity Theories\index{Discontinuity Theories} which suggest that human language is a sudden event in the course of evolution and other theories which are the combination of the above. As of 2016, it is still unclear (according to Western ideas) what language is. In a recent article (Ibbotson\index{Ibbotson, P.} \textit{et al.} 2016) in the \textit{Scientific American}, Chomsky’s\index{Chomsky, Noam} modern revolution in linguistics is criticized based on recent evidence.

\begin{myquote}
“Recently, though, cognitive scientists and linguists have \textbf{abandoned } Chomsky’s \textbf{“universal grammar” theory}\index{Universal Grammar Theory} in droves because of new research examining many different languages—and the way young children learn to understand and speak the tongues of their communities. That work \textbf{fails to support} Chomsky’s assertions.”

~\hfill (Ibbotson \textit{et al.} 2016) (emphasis ours)
\end{myquote}

The problem with these (Western) theories and similar others is acknowledged -

\begin{myquote}
“As with all linguistic theories, Chomsky’s universal grammar tries to perform a balancing act. The theory has to be simple enough to be worth having. That is, it must predict some things that are not in the theory itself (otherwise it is just a list of facts). But neither can the theory be so simple that it cannot explain things it should.”

~\hfill (Ibbotson\index{Ibbotson, P.} \textit{et al.} 2016)
\end{myquote}

The alternative to universal grammar and innate linguistics are usage-based theories based on meaningful generalizations driven by empirical studies and data. Theories of Universal Grammar can be considered mostly irrelevant. 

\vskip -26pt


\section*{Indian Theories of Language}

\vskip -6pt

The Indian (theories) notions of language are deeply intertwined to the cosmology and worldview of the Veda-s. This approach is fundamentally different from the Western approaches to addressing the phenomenon of language based on external structure. The role of language in establishing and upholding \textit{dharma}\index{dharma@\textit{dharma}} (that which is vedi-cosmologically\index{vedi-cosmologically} harmonious) is a principal concern in the \textit{Mahābhāṣya}\index{Mahabhasya@\textit{Mahābhāṣya}} and is also seen in the Śāntiparvan\index{Santiparvan@Śāntiparvan} (232-30) of the \textit{Mahābhārata}\index{Mahabharata@\textit{Mahābhārata}} - the proper use of language would lead the practitioner to final liberation (\textit{mokṣa}\index{moksa@\textit{mokṣa}}).

The origins of Vyākarạa\index{Vyakarana@Vyākaraṇa} are hinted at in the Veda, the \textit{Gopatha Brāhmaṇa}\index{Gopatha brahmana@\textit{Gopatha Brāhmaṇa}} and in the \textit{Ṛk-tantra}\index{Rktantra@\textit{Ṛktantra}} where Brahmā is listed as the first author of Vyākarạa (Subrahmanyam\index{Korada, Subrahmanyam} 2008). The tradition of language-science can be traced back to the Veda-s which according to oral traditions and recent archaeological evidences of Vedic cultures, conservatively dated to at least 10,000 BCE, if not earlier.

The earliest surviving formal textual work on Vyākarạa is by Pāṇini\index{Panini@Pāṇini} (dated to at least 500 BCE, if not earlier). Understanding of language and its minute analysis have been very comprehensive in the Sanskrit tradition.

\begin{myquote}
“Language has been subjected to micro-analysis and macro-analysis both at syntactic and semantic levels right from \textit{varṇa} (phoneme) through \textit{mahāvākya}\index{mahavakya@\textit{mahāvākya}} (discourse) by Indian intelligentsia, thousands of years ago. Almost all the systems and schools of Indian philosophy, including Vyākaraṇa had discussed the concept of \textit{śabda}. The term \textit{śabda}\index{sabda@\textit{śabda}} (\textit{vāk}) is untranslatable. It is used to denote many things, i.e. \textit{varṇa}\index{varna@\textit{varṇa}} (phoneme), \textit{prakṛti/pratyaya} (morpheme), \textit{pada} (word), \textit{vākya} (sentence), \textit{avāntara-vākya}\index{avantara-vakya@\textit{avāntara-vākya}} (sub-sentence), \textit{mahāvākya} (discourse), the \textit{śabda\index{pramana@\textit{pramāṇa śabda}}-pramāṇa\index{pramana@\textit{pramāṇa}}, parā\index{para@\textit{parā}}, paśyantī\index{pasyanti@\textit{paśyantī}}, madhyamā\index{madhyama@\textit{madhyamā}}, vaikharī\index{vaikhari@\textit{vaikharī}}} and the ordinary sound. \textit{Ṛk, Yajus, Sāma} etc. are synonyms of \textit{vākya.}”

~\hfill (Subrahmanyam 2008:vi)
\end{myquote}

Language scholarship in the Sanskrit tradition goes much deeper and involves not only mastery of grammar but also the mastery of the concept of \textit{śabda}\index{sabda@\textit{śabda}} (word), \textit{vākya}\index{vakya@\textit{vākya}} (sentence) and \textit{pramāṇa}\index{pramana@\textit{pramāṇa}} (knowledge-sources) i.e. \textbf{\textit{pada-vākya-pramāṇa-jñāna}}.\index{padavakyapramanajnanaa@\textit{pada-vākya-pramāṇa-jñāna}}

Subrahmanyam places Indian linguistic science in the context of the Western theories of language and expresses dismay at the lack of scholarship in the study of languages especially by Indian traditional scholars who have failed to analyse the Western theories of linguistics in the the light of the Indian systems.

\begin{myquote}
“\textbf{... each and every unit} in ancient linguistic science is defined \textbf{clearly and unambiguously}. No definition/rule/norm is revised. There are impeccable solutions to all the problems, both at syntactic and semantic levels.”

~\hfill (Subrahmanyam\index{Korada, Subrahmanyam} 2008:vii)
\end{myquote}

\begin{myquote}
“On the other hand \textbf{scholars of modern linguistic science are still searching for a definition of “word”}. Scholars are divided on the concept of discourse/text/sentence. The theories that are proposed were revised time and again.”

~\hfill (Subrahmanyam\index{Korada, Subrahmanyam} 2008:vii) (emphasis ours)
\end{myquote}

On the painful state of affairs of linguistics (as a humanities discipline) in Indian universities - he says

\begin{myquote}
“… Meanwhile, the libraries are \textbf{dumped with books/theses/articles}, written taking the theories that were later stamped as untenable. The situation still continues.”

~\hfill (Subrahmanyam 2008:vii) (emphasis ours)
\end{myquote}

A very important difference in Indian approaches to language has been the deep synthesis of language with metaphysics of the various Indian darsanas. Vyākarạa\index{Vyakarana@Vyākaraṇa} has influenced and has been influenced by millennia of Indian thought on the nature of reality. Helārāja\index{Helaraja@Helārāja} in his commentary on Bhartṛhari’s\index{Bhartrhari@Bhartṛhari} \textit{Vākyapadīya}\index{Vakyapadiya@\textit{Vākyapadīya}} characterises Vyākarạa as having eight constituents under 4 principal categories of \textit{śabda}\index{sabda@\textit{śabda}} (Word), \textit{artha}\index{artha@\textit{artha}} (Meaning), \textit{sambandha} \index{sambandha@\textit{sambandha}} (relation), and \textit{prayojana}\index{prayojana@\textit{prayojana}} (purpose).

\begin{enumerate}
\itemsep=0pt
\item \textit{Śabda} (word)
 
\begin{enumerate}
\itemsep=0pt
\item \textit{Prakṛti}\index{\textit{prakṛti}} and \textit{Pratyaya}\index{pratyaya@\textit{pratyaya}}

 \item \textit{Pada}\index{pada@\textit{pada}} and \textit{Vākya}\index{vakya@\textit{vākya}}
\end{enumerate}

 \item \textit{Artha} (meaning)
 
\begin{enumerate}
\itemsep=0pt
\item \textit{Prakṛtipratyayārtha}\index{Prakrtipratyayartha@\textit{Prakṛtipratyayārtha}} (meaning of root and suffix)

 \item \textit{Padavākyārtha}\index{Padavakyartha@\textit{Padavākyārtha}} (meaning of word and sentence) 
\end{enumerate}

 \item \textit{Sambandha} (relation between \textit{śabda} and \textit{artha})
 
\begin{enumerate}
\itemsep=0pt
\item \textit{Kārya-kāraṇa-bhāva}\index{Karya-karana-bhava@\textit{Kārya-kāraṇa-bhāva}} (cause and effect relationship)

 \item \textit{Yogyatā}\index{Yogyata@\textit{Yogyatā}} (capacity to render meaning)
\end{enumerate}

 \item \textit{Prayojana} (purpose)
 
\begin{enumerate}
\itemsep=0pt
\item \textit{Artha-jñāna}\index{Artha-jnana@\textit{Artha-jñāna}} (knowledge of meaning)

 \item \textit{Dharma}\index{dharma@\textit{dharma}} (leading to \textit{mokṣa}\index{moksa@\textit{mokṣa}})
\end{enumerate}

\end{enumerate}

\textit{Śabda\index{sabda@\textit{śabda}}, artha\index{artha@\textit{artha}}} and their \textit{sambandha}\index{sambandha@\textit{sambandha}} are considered to be eternal and the structure bounded by Vyākarạa\index{Vyakarana@Vyākaraṇa} and meanings by \textit{śāstra} and cannot be misinterpreted. Use of the right \textbf{\textit{śabda-s}} acceptable to Vyākarạa would fetch \textit{dharma}\index{dharma@\textit{dharma}} and usage of \textbf{\textit{apa-śabda-s}}\index{apasabda@\textit{apaśabda}} (wrong and imperfect) would only cause accumulation of \textit{adharma}. In fact the \textbf{use and abuse of language is considered to be the primary domain of Vyākarạa}. It is considered to be the \textbf{“only”} subject-matter of Vyākarạa. Grammar as a translation to \textbf{Vyākarạa} does not capture the real connotation of Vyākarạa. For more details, refer the Brahmakāṇḍa of Bhartṛhari’s\index{Bhartrhari@Bhartṛhari} \textit{Vākyapadīya}\index{Vakyapadiya@\textit{Vākyapadīya}} (Subrahmanyam\index{Korada, Subrahmanyam} 1992).

This brief discussion on Indian notions of language brings to light the fundamental differences in Indian approaches to language and those of the West. That Western techniques like philology\index{Philology@Western} based on the flaky foundations of Western theories of language are used to interpret Sanskrit text without heed to the rich tradition of language understanding in the Sanskrit tradition itself, which only serves to highlight the hubris and incompetency of Western scholarship more than anything else. The West will do what it wants to and it should; the deeper issue is the mindless import of these flawed knowledge systems (over the past 70 years) into Indian Universities via the Humanities and Social-Science departments and the so-called “modern” educational systems.


\section*{Computational Notion of Language}

The engineering approach to understanding language in contrast to the theoretical approaches of the humanities is seen in the advances in the fields of AI\index{Artificial Intelligence} and machine learning\index{machine learning}. A combination of rule based symbolic systems and statistical (data-driven) approaches to understanding language have been used successfully to understand language, and are put to use in specific applications as seen in recent successes of language agents like Siri. It is important to understand that some of the best intellects (currently) are engaged as yet in the pursuit of understanding language using the powerful tools of computation. The application-specific nature of the engineering approach has led to tremendous successes in niche areas of computational language understanding. The fundamental approaches to syntax representations and the semantic structures therein though are still dependent on the Western theories of sentence and phrase. The weak notions of Western approaches to grammar are reflected in the computational representations too. The most common representative structure is the CFG (Context-Free Grammar)\index{Context-free Grammar} and its probabilistic version the PCFG\index{Probabilistic Context-free Grammar}. These are derived from existing (massive) corpora of language usage. Though it is a simple model and results in efficient parsing there are still very many issues with syntactic ambiguity\index{ambiguity} - present state-of-art parsing (for English) has levelled off at 90\% constituent (recognising the parts of a sentence) accuracy. The problem of ambiguity though still remains and is severe.

The problems facing computational linguistics are similar to those faced by Western theoretical linguistics. The problem of ambiguity is very difficult to resolve as there are very many degrees of freedom. This only brings more perspective to the absolutely speculative nature of philological approaches to interpreting linguistic meaning. The issue of semantics\index{semantics} in a computational sense and the engineering approaches to address the issue are many, and are addressed in the section on Meaning. See Schubert\index{Schubert, L.} (2014) for more.


\section*{Discussion}

In this section, I have tried to highlight fundamental issues in the understanding of language and the various theories of language. The Western theories of language influenced earlier by Biblical preoccupations and later by motivations of imperialism and racism are limiting when we compare them with the theories of language already existing in the Sanskrit tradition. To allow Western philology\index{Philology@Western} to be a tool of academic scholarship to interpret Sanskrit texts, is not only deeply insulting to the civilizational heritage of Sanskrit but is also technically incorrect and flawed by any academic standard. That even in the 21st century we still need to discuss the appropriateness of philology is deeply disturbing at very many levels. What this means for (\textit{videśī}) Neo-Orientalist practitioners and their theses, originating both in Western and Indian Universities, is something that needs serious deliberation.


\section*{On Meaning}

The section on language has highlighted some fundamental issues in the approach to understanding language. This section on meaning explicates these issues in a clearer and more robust manner. It becomes all the more apparent that the ideas of meaning and the dependencies on the much deeper lived philosophies of language in the Indian traditions require more robust tools of interpretation than those offered by the Western tools of philology\index{Philology@Western} and hermeneutics\index{hermeneutics}. Unless a scholar is embedded in the culture and understands the profound syncretic nature of Indian languages, his professing scholarship from Western universities which are far removed from the actual reality of usage of symbols and their interpretation is nothing less than abysmal scholarship. That this has been happening for more than 250 years is no rationale that it needs to be continued. That the realities of America as a modern-day cultural colonizer, and the demands of dual-use anthropology\index{anthropology} (Price\index{Price, D. H.} 2016) via area-studies warrants and necessitates such dubious scholarship, is NOT widely known. Scholarship such as those of the Neo-Orientalists will be funded and will continue as long as there is a geopolitical need for it. It is time that Indian policy-makers realize this.


\section*{The Meaning of Meaning}

Meaning has a wider connotation than the meaning in a verbal sense. The modern (Western) ideas of meaning have two approaches - theories of meaning (semantics\index{semantics}) and a more foundational theory ( facts which provide semantic content for meaning). The idea of meaning is approached differently in Indian traditions. Meaning of words (\textit{śabda}\index{sabda@\textit{śabda}}) and meanings from other \textit{pramāṇa}-s\index{pramana@\textit{pramāṇa}} (valid sources of knowledge) are considered to be distinct. The Vedic heritage provides for deeper notions of meaning and introduces \textit{infinities, transcendences} and \textit{consciousness} into the discourse of meaning. In Sanskrit, \textit{vaidika}\index{vaidika@\textit{vaidika}} (Vedic) meanings of a given \textit{śabda} are different from its \textit{laukika}\index{laukika@\textit{laukika}} (worldly usage) sense. And whatever be the approaches, the idea of meaning as both an “input” to language and as an “output” from language is acknowledged.


\section*{Western Notions of Meaning}

\vskip -5pt

As indicated in Speaks\index{Speaks, J.} (2017) there are two kinds of theories of meaning, answering to two different questions –

\begin{enumerate}
\itemsep=0pt
\item What is the meaning of this or that? and

 \item In virtue of what facts about that person or group does a symbol have meaning?

\end{enumerate}

This maps to a semantic theory of meaning and a foundational theory of meaning, correspondingly.

The semantic theories are either propositional\index{semantic theory@propositional} \index{semantic theory@Non-propositional} (based on frames of reference and the context) or non-propositional (based on logical formulation via truth semantics\index{semantics}). Either of these theories faces difficulties arising out of context sensitivity and the resulting ambiguities of meaning. There is still no clarity as to how “meaning” arises using these semantic models. This is possibly due to the weak underlying mechanisms in the understanding of language and the theory of language itself.

The second “sort” of theory is the foundational theory of meaning on the philosophical notion of meaning. As usual, one approach to this is to deny the existence of any such foundational theory; all meanings are explained in terms of mental states of users of the languages. Summarizing the mentalist theories of meaning:

\begin{myquote}
“Since mentalists aim to explain the nature of meaning in terms of the mental states of language users, mentalist theories may be divided according to which mental states they take to be relevant to the determination of meaning. The most well-worked out views on this topic are the Gricean view, which explains meaning in terms of the \textbf{communicative intentions of language users}, and the view that the meanings of expressions are \textbf{fixed by conventions which pair sentences with certain beliefs.}”

~\hfill (Speaks\index{Speaks, J.} 2017) (emphasis ours)
\end{myquote}

There also exist non-mentalist foundational theories of meaning, basing meaning on origins, usage and on causality. Those that base meaning on the principle of truth-maximization and on social norms also fall into the non-mentalist categories of theory.


\section*{Indian Notions of Meaning}

\vskip -5pt

As seen in the section on Indian theories of language, the notion of meaning is deeply embedded in the tight structure of \textit{śabda} and \textit{vākya}. There is very little scope for ambiguity between \textit{śabda}\index{sabda@\textit{śabda}} and its \textit{artha}\index{artha@\textit{artha}}. Vedic notions of sound as basis of meaning, the Veda-s being primarily an oral tradition, have different approaches to meaning than those seen in textual Sanskrit. \textit{Śikṣā}\index{siksa@\textit{śikṣā}} (phonetics), \textit{Chandas}\index{chandas@\textit{chandas}} (prosody, poetic meters) and \textit{Nirukta}\index{Nirukta@\textit{Nirukta}} (contextual etymology) are unique to Sanskrit and exemplify the influence of interpreting meaning in Vedic Sanskrit. The science of \textit{Nirukta} helps ascertaining meaning of words which are no longer in usage.

Meaning derivable from text is very well-structured by Pāṇinian\index{Panini@Pāṇini} grammar. The notion of \textit{śabda} as a \textit{pramāṇa} (knowledge source) which is different from other \textit{pramāṇa}-s\index{pramana@\textit{pramāṇa}} like \textit{pratyakṣa}\index{pratyaksa@\textit{pratyakṣa}} (observation) and \textit{anumāna}\index{anumana@\textit{anumāna}} (inference) is fundamental to Indian knowledge systems. Prof. Kapil Kapoor\index{Kapoor, Kapil} lays out the architecture of the Indian conception of meaning and its close association with the structure of the primordial Vedas (Kapoor 2005). The tradition of interpretation and disambiguation based on context is seen right in the Veda-s. He also discusses the various traditions of interpretation of meaning starting from the \textit{Ṛg Veda}\index{Rg Veda@\textit{Ṛg Veda}} exegesis to the notions of meaning influenced by the various darśana-s like Nyāya\index{Nyaya@Nyāya}, Mīmāṁsā\index{mimamsa@Mīmāṁsā} and Vedānta\index{Vedanta@Vedānta}. It suffices to know that there are distinct traditions stretching back millennia, which explain in painstaking detail how meaning is to be derived from text.

For a stupendous technical treatment of meaning the reader is referred to Tatacharya\index{Tatacharya, N. S. Ramanuja} (2008), a four-volume treatise on the nature of meaning derivable from text and speech, \textit{śābda-bodha- mīmāṁsā}\index{sabdabodhamimamsa@\textit{śābdabodhamīmāṁsā}}. From the colophon:

\begin{myquote}
“An Inquiry into Indian Theories of Verbal Cognition, the author, assembling the view of different \textit{śāstra}-s (Nyāya, Mīmāṁsā, Vyākaraṇa, Vedānta.....) examines the following theories and subjects: the theory according to which word is a means of valid cognition, the definition of word as a means of valid cognition, the nature of the sentence, its sense, and what makes it intelligible, the theories of \textit{anvitābhidhāna}\index{anvitabhidhana@\textit{anvitābhidhāna}} and \textit{abhihitānvaya}\index{abhihitanvaya@\textit{abhihitānvaya}}, the notions of syntactic unity and plurality, syntactic expectancy, logical consistency, phonetic contiguity and the general purport of the sentence, the \textit{sphoṭa} theory: all views and notions the knowledge of which constitutes the first step in the analysis of verbal cognition.”

~\hfill (Tatacharya\index{Tatacharya, N. S. Ramanuja} 2008:blurb)
\end{myquote}

In order to examine the validity of Western philology\index{Philology@Western} as a technique to understand Indian texts i.e., the \textit{śāstra}-s, \textit{śruti}\index{sruti@\textit{śruti}} and \textit{smṛti}\index{smrti@\textit{smṛti}} works, it is appropriate to discuss the idea of interpretation of Sanskrit text as it has evolved in the multi-millennial tradition of shared interpretation, the notion of the \textit{śāstra-paddhati}\index{sastrapaddhati@\textit{śāstra-paddhati}}.

\begin{myquote}
“The \textit{paddhati} (customary practice), can be seen to have four parts 1. Formal organization of discourse in terms of \textit{prakaraṇa\index{prakarana@\textit{prakaraṇa}}, adhikaraṇa\index{adhikarana@\textit{adhikaraṇa}}} 2. The logical mode of discussion and argument 3. \textit{pramāṇas}\index{pramana@\textit{pramāṇa}} or epistemology and 4. Strategies or instruments of interpretation. The presence of a shared \textit{paddhati}, or mode of interpretation, employed by competing and opposing \textit{sampradāya} schools, such as \textit{brāhmaṇa}, Buddhist and Jaina define India as an interpretive community.
\end{myquote}

\begin{myquote}
… There are two parallel interactive traditions - the popular tradition of collective institutionalised reading - the \textit{kathā}\index{katha@\textit{kathā}} and \textit{pravacana}\index{pravacana@\textit{pravacana}} form. The learned tradition which forms the core of the popular tradition, seeks to analyse the meaning of the text by analysing the text according to the shared metalanguage.
\end{myquote}

\begin{myquote}
… This imposes certain conditions on the interpreter. There is in the tradition the concept of \textit{adhikāra}\index{adhikara@\textit{adhikāra}} - the concept of \textbf{\textit{competence to interpret}}. “\textit{A process of saturation, must set in before the eyes are ready to see and the mind to grasp}”. This “process of saturation” involves mastering all the pertinent knowledge. The boundaries of this knowledge have been described by Rajasekara\index{Rajasekhara@Rājaśekhara}. He lists 22.
\end{myquote}

\begin{myquote}
… Besides - each \textit{śāstra} has its own vast exegetical scholarship (\textit{bhāṣya\index{bhasya@\textit{bhāṣya}}, vṛtti, ṭīkā, samīkṣā, pañjikā, kārikā, vārttikā}) which is responsible for refinement, extension, and increasing precision and profundity of knowledge. Any learned man who takes upon himself the task of interpreting \textit{Śāstra}, has to have mastered all this knowledge - twenty two sciences and the commentary literature - if he is to make an enduring contribution to the Indian tradition.”

~\hfill (Kapoor\index{Kapoor, Kapil} 2005:98-100)
\end{myquote}

I will not go into more detail as to what comprises the actual form and structure of textual interpretation. It would already be clear at this point as to what the traditional Sanskrit scholar normally considers Western philology\index{Philology@Western} to be and the theses propounded by 250 years of Western scholarship, as also those by Sheldon Pollock\index{Pollock, Sheldon} and others of the (\textit{videśī}) Neo-Orientalist orientation. Whatever be the Western scholar’s saturation, it certainly is not anything close to the expected levels of competence of a traditional \textbf{\textit{adhikārin}}\index{adhikara@\textit{adhikāra}}.

The Western anthropo-sociological notions of \textbf{etic}\index{etic scholarship@\textit{etic scholarship}} vs \textbf{emic}\index{emic scholarship@\textit{emic scholarship}} are extremely relevant in the context of interpreting Sanskrit texts. Unless the relevant expertise and competencies have been gained and achieved by decades of serious study in the relevant bodies of knowledge - \textbf{\textit{adhikāra}}\index{adhikara@\textit{adhikāra}} to interpret text is just not created.


\section*{Computational Notion of Meaning\index{Theories of Meaning@Computational}}

\vskip -6.5pt

Information theory\index{Information theory} exemplifies the mathematical view of information and meaning. It studies the quantification of information. Problems arising out of transmission of information via signals led to notions of information entropy\index{Information entropy}. Meaning of information in terms of “data” is expressed via mathematical measures of entropy. A \textbf{“bit”} becomes the fundamental unit of information and thereby of meaning. The algorithmic information theory view of information is of it as a measure of computation. The more random a given string is - the more complex it is deemed to be. The Quantum world-view has its own version of Quantum information\index{Quantum Information theory} theory and its fundamental unit - the \textbf{Qubit}\index{Qubit}. Common to all of these views is the idea of data and the measures based on its varying representations (bits, qubits, strings etc.). Artificial systems (programs) which need to represent and reason about states of systems - either real or artificial - manipulate symbols which encode meaning of the states of a system via axiomatic systems based on propositional and first-order predicate logic. The symbolic approach to encoding meaning and knowledge via logic and then reasoning about them via logic programs using techniques of inference forms the basis of symbolic AI\index{Artificial Intelligence} systems. More than 70 years from the birth of these symbolic systems, the basic problems of representation and reasoning are still not solved. The “frame problem” is one such. What facts about the world do not change arbitrarily? Things which a human child can seemingly do effortlessly are impossible to achieve using current approaches.

The statistical approach to meaning and intelligence taken by\break machine-learning\index{machine learning} is to use as much data as possible about phenomenon (across as many capturable dimensions) and use mathematical techniques to derive causality and identify the nature of the phenomenon they represent. The modern solutions to the problems of speech recognition, image processing, natural language processing take these approaches. The availability of trillions of data sets with which to train mathematical engines to recognize patterns masks the fact that machines actually do not understand much about the phenomenon that they can recognize in the statistical approach. The compute intensive nature of deriving meaning based on data is the current dominant paradigm in the approach to machine based derivations of meaning. Manning\index{Manning, C. D.} (2015) sums up the state of art in computational linguistics thus:

\begin{myquote}
“It would be good to return some emphasis within NLP to cognitive and scientific investigation of language rather than almost exclusively using an engineering model of research. … However, \textbf{I would encourage everyone to think about problems, architectures, cognitive science, and the details of human language, how it is learned, processed, and how it changes}, rather than just chasing state-of-the-art numbers on a benchmark task.”

~\hfill (Manning\index{Manning, C. D.} 2015) (emphasis ours)
\end{myquote}

The understanding of the state-of-art in the field of computational language and meaning which has been occupying some of the best human minds over the past century puts in perspective the stupendous achievements of the ancient masters of language and the magnitude of riches available in the works of the Sanskrit tradition. To negate and willfully ignore the existence of these wonderful formulations and theories and also to neglect their rich interpretive heritage is, to put it mildly, sacrilege.


\section*{Discussion}

\vskip -7pt

In this section, I have tried to highlight fundamental issues in notions of meaning. The limited Western notions of meaning based on philosophical notions of semantics\index{semantics} and foundational meaning are juxtaposed with the broad and deeply scientific approach to meaning based on Vyākarạa\index{Vyakarana@Vyākaraṇa} (\textit{śabda}\index{sabda@\textit{śabda}} and \textit{artha}\index{artha@\textit{artha}}). The Indian tradition of text interpretation is juxtaposed with the speculative nature of Western philology\index{Philology@Western}. The computational notion of meaning\index{Theories of Meaning@Computational} (the state-of-art in computational linguistics) is also discussed in brief, primarily in order to highlight the magnitude of the riches that the Sanskrit linguistics tradition has in store.

Given all of this, having to resort to Western interpretive methods like philology to interpret Sanskrit text, is truly reflective of the pervasive intellectual colonization and the collective inferiority complexes of modern Indians more than anything else.


\section*{Implications}

\vskip -7pt

Is emic\index{emic scholarship@\textit{emic scholarship}} scholarship going to allow etic\index{etic scholarship@\textit{etic scholarship}} (outsider) perspectives and methods influence the understanding of the Indian traditions? The pernicious motives of Sheldon Pollock\index{Pollock, Sheldon} and the Neo-Orientalists, however aesthetically camouflaged and strategically positioned, need to be countered by a legitimate and valid understanding of the traditional (Sanskrit) methods and techniques.

Are we going to let Western Philology\index{Philology@Western} interpret Indian society (past, present and future)? The past and present are already being interpreted with these lenses; there were other lenses too (those of imperialism and racism) in use during the colonial period. Sheldon Pollock’s\index{Pollock, Sheldon} political philology\index{Philology@political} and liberation philology\index{Philology@Liberation} need to be countered by collective hard work and coherent arguments against it highlighting its invalidity and inapplicability as an interpretive mechanism. Echoing the thoughts of Rajiv Malhotra\index{Malhotra, Rajiv} one must say that the future will depend on what the insiders of the tradition do with Sanskrit.

\vskip -6pt


\section*{The Possible Future}

As a directly discernible impact of the efforts of the Neo-Orientalists\index{Neo-Orientalists}\endnote{\url{http://www.perso-indica.net/about-editors}}, the methods of Philology might be making a return academically with possible geo-political implications too. The recent efforts to analyze\endnote{\url{http://www.perso-indica.net/about-aims-method.faces}} Persian works on the Indian traditions would possibly give us interesting “interpretations” of Islamic history of India. The distinct lack of a genre of Indology attributable to the Islamic colonizers of India is well-known. From the looks of it, a possible “\textit{Mughal Indology}” seems to be in the works, the foundations of such a genre are possibly being laid.

Another interesting dimension that is to be highlighted here, the academic attempts to deny and negate the influence of (Hindu) India on European thought.The Indian origins of so-called \textit{European} mathematics and science is slowly being revealed and albeit limited – it is getting mainstream acknowledgement. The “Mughal” period can be used as a “wedge” (in ways similar to which the Neo-orientalists\index{Neo-orientalists} use Buddhism\index{Buddhism} as a wedge against Hinduism\index{Hinduism}) to negate Hindu influence on Europe. Any such influence could possibly be ascribed to the Mughals\index{Mughals} (Persians), who could then be connected to the Greeks etc.

We can see such motive too in Pollock’s \index{Pollock, Sheldon}attempt at discrediting Indian thought, by “suggesting” Sanskritic influence on Hitler and the holocaust. This would work toward possibly influencing “Western” academia to “instinctively” look for such “Indic” connections for all of “Western” malaise. 

The “Zukunftsphilologie \endnote{\url{http://www.forum-transregionale-studien.de/en/revisiting-the-canons-of-textual-scholarship/about-us/profile.html}}” – Future Philology\index{Philology@Future} – attempts at resurrecting philology are also to be noted by interested scholars. A new journal, entitled Philological Encounters\endnote{\url{http://www.brill.com/products/journal/philological-encounters}}, is intended to be the primary vehicle for the dissemination of the ideas and methods of Sheldon Pollock\index{Pollock, Sheldon} (as is acknowledged). As seen in the brief examples cited above, the influence of Sheldon Pollock and his \textit{paramparā} of students (both Western and the Indian “sepoy” academics), is \textbf{non-trivial}. It is something that Swadeshi scholars need to be aware of. The multiple facets of this “kurukṣetra” are to be studied and responded to appropriately. Philology is very possibly an extremely important “front” in the larger battles that Swadeshi Indologists have to pursue in future.

\newpage

\section*{The Nescience\index{nescience} of Meaning}

\vskip -5pt

This discussion of language, meaning and a deep understanding of the speculative nature of methods like philology has very possibly served to highlight the serious limitations of Western methods of academic scholarship. Nietzsche\index{Nietzsche, F.}, it would seem, was not vociferous enough in his call for philology\index{Philology} to disappear; more than 125 years after his call for philology to disappear it still is around, more lethal than ever via its reinvention by Sheldon Pollock\index{Pollock, Sheldon}. Surely, it would not be too incorrect to make a claim that, in all probability, Pollock has not understood the true nature of meaning either in the Western sense or in the sense of the Indian tradition.

Based on individual rhetoric and non-empirical approaches to understanding text, building context, creative makings of the truth (inventing falsehoods), deeply flawed approaches to historicism\index{historicism} and a deep revulsion of the sacred sensibilities prevalent in Indian thought very much characterise the “Pollockian”. Fantastic theses like \textit{Deep-Orientalism} are exemplars of this approach.

Couched in fairly dense, deliberate and camouflaged academic verbosity, exemplified in the various theses (Pollock 1997), (Pollock 1998), (Pollock\index{Pollock, Sheldon} 2001),(Pollock 2003), (Pollock 2005), (Pollock, 2006), (Pollock 2009) and his latest thesis on Rasa (Pollock 2015), it is not too difficult to ascertain that Pollock harbors a deep disdain for Sanskrit knowledge systems and more so to the notion of \textit{dharma}\index{dharma@\textit{dharma}} and most importantly its pursuit (which is the most important goal/aspect of “meaning” in the Sanskrit traditions).

What is revealed after this critical examination is Pollock’s\index{Pollock, Sheldon} \textbf{nescience}\index{nescience} of meaning in most possible senses: The role of language as a (\textit{dharmic}) liberator (\textit{mokṣa}) has not been understood. The role of (\textit{śabda\index{sabda@\textit{śabda}}-artha\index{artha@\textit{artha}}}) meaning in the pursuit of \textit{dharma}\index{dharma@\textit{dharma}} has not been understood. The role of \textit{śāstra-paddhati}\index{sastrapaddhati@\textit{śāstra-paddhati}} in (Sanskrit) text interpretation has not been understood. The empirical nature of ascertaining meaning in the Western sense has also not been understood. The principle of truth maximisation as an approach to ascertaining meaning has not been understood. Inventing falsehoods, creative making of the truth, selective interpretation of text, misuse of quotes, use of false chronologies, self-contradicting positions, stupendous hubris and monumental disdain of Sanskrit and its cultural artifacts seem to be hallmarks of the Neo-orientalists, epitomized by Pollock.


\section*{Conclusion}

Given the nature of subject areas covered, history and evolution of Western philology\index{Philology@Western}, Indian, Western and computational theories of language\index{Theories of Language@Computational}, Indian, Western and computational notions of meaning\index{Theories of Meaning@Computational}, the principal focus of this paper has been to highlight the misunderstandings of the nature of language, meaning and of the flawed and illegitimate nature of the methods of neo-Orientalist (\textit{videśī}) scholarship. That the nature of language and the nature of meaning have not been properly understood by Western scholarship (since the 1700s) has been highlighted and discussed. Indian theories of language and notions of meaning (that have evolved over millennia) are positively more scientific, better reasoned and are far less ambiguous. The well-acknow\-ledged speculative nature and political/rhetorical roots of philology have been discussed and examined in detail. It must be clear by now that Sheldon Pollock’s methods of re-invented philology are deeply flawed and cannot be the basis for any sort of credible scholarship. They need to be confronted and called out for what they are - \textbf{unscientific and dishonest} scholarship.


\section*{References}

\begin{thebibliography}{99}
\bibitem{chap05-key01} Breckenridge, C. A., and Veer, P. V. (1993). \textit{Orientalism and the Postcolonial Predicament: Perspectives on South Asia}. Philadelphia: University of Pennsylvania Press.

 \bibitem{chap05-key02} Bronner, Y., Cox, W., and McCrea, L. J. (2011). \textit{South Asian Texts in History: Critical Engagements with Sheldon Pollock}. Ann Arbor, MI: Association for Asian Studies.

 \bibitem{chap05-key03} Bull, M. (2013). \textit{Inventing Falsehood, Making Truth: Vico and Neapolitan Painting}. Princeton, NJ: Princeton University Press,

 \bibitem{chap05-key04} Bybee, J. L. (2010). \textit{Language, Usage and Cognition}. Cambridge: Cambridge University Press.

 \bibitem{chap05-key05} Cancik, Hubert., and Cancik-Lindmaier, Hildegard. (2014). “The Religion of the “Older Greeks” in Nietzsche’s “Notes to We Philologists”” In Jensen and Heit (2014). Chapter 13.

 \bibitem{chap05-key06} Chomsky, Noam. (2002). Syntactic Structures. doi:10.1515/9783110218329, \url{https://www.researchgate.net/publication/30868825_Syntactic_Structure}. Accessed on 12 Oct, 2016.

 \bibitem{chap05-key07} Culler, Jonathan. (2002). “The Return to Philology”. \textit{Journal of Aesthetic Education} 36, No. 3. pp.12-16.

 \bibitem{chap05-key08} Davis, S., and Gillon, B. S. (2004). \textit{Semantics: A Reader}. New York: Oxford University Press.

 \bibitem{chap05-key09} Eco, Umberto. (1976). \textit{A Theory of Semiotics}. Bloomington: Indiana University Press..

 \bibitem{chap05-key10} Ekers, M. (2013). \textit{Gramsci: Space, Nature, Politics}. Chichester: John Wiley \& Sons.

 \bibitem{chap05-key11} Foucault, M., and Hoy, D. C. (1986). \textit{Foucault: A Critical Reader}. Oxford, UK: B. Blackwell.

 \bibitem{chap05-key12} Foucault, M., Sheridan, A., and Foucault, M. (1972). \textit{The Archaeology of Knowledge}. New York: Pantheon Books.

 \bibitem{chap05-key13} Goldberg, A. E. (2006). \textit{Constructions at Work: The Nature of Generalization in Language}. Oxford: Oxford University Press.

 \bibitem{chap05-key14} \textit{Gopatha Brāhmaṇa}. See Mitra and Vidyabhushana (1872).

 \bibitem{chap05-key15} Gumbrecht, H. U. (2003). \textit{The Powers of Philology: Dynamics of Textual Scholarship}. Urbana, IL: University of Illinois Press.

 \bibitem{chap05-key16} Gurd, S. A. (2010). \textit{Philology and Its Histories}. Columbus: Ohio State University Press.

 \bibitem{chap05-key17} Gutting, G. (1989). \textit{Michel Foucault's Archaeology of Scientific Reason}. Cambridge: Cambridge University Press.

 \bibitem{chap05-key18} —. (1994). \textit{The Cambridge Companion to Foucault}. Cambridge: Cambridge University Press.

 \bibitem{chap05-key19} Harpham, G. G. (2009). “Roots, Races, and the Return to Philology”. \textit{The Humanities and the Dream of America}, 43-79. doi:10.7208/chicago/9780226317014.003.0003. Chicago: University of Chicago Press.

 \bibitem{chap05-key20} Houben, J. E., and Pollock, S. (2008). “Theory and Method in Indian Intellectual History”. \textit{Journal of Indian Philosophy}, 36(5-6), 531-532. doi:10.1007/s10781-008-9050-z

 \bibitem{chap05-key21} Ibbotson, P., and Tomasello, M. (2016). “Evidence Rebuts Chomsky's Theory of Language Learning”. \textit{Scientific American}. \url{http://www.scientificamerican.com/article/evidence-rebuts-chomsky-s-theory-of-language-learning/} Accessed September 24, 2016.

 \bibitem{chap05-key22} Jensen, Anthony K., and Heit, Helmut. (Ed.s) (2014). \textit{Nietzsche as a Scholar of Antiquity}. London: Bloomsbury Publication. doi:10.5040/9781472548122.

 \bibitem{chap05-key23} Kapoor, K. (1994). \textit{Language, Linguistics, and Literature, the Indian Perspective}. New Delhi: Academic Foundation.

 \bibitem{chap05-key24} —. (2005). \textit{Text and Interpretation: The Indian Tradition}. New Delhi: D.K. Printworld.

 \bibitem{chap05-key25} —. (2014). \textit{Comparative Literary Theory: An Overview}. New Delhi: D.K. Printworld.

 \bibitem{chap05-key26} —., and Singh, A. K. (2005). \textit{Indian Knowledge Systems}. Shimla: Indian Institute of Advanced Study.

 \bibitem{chap05-key27} —., and Śukla, V. (2005). \textit{Sanskrit Studies}. New Delhi: Special Centre for Sanskrit Studies, Jawaharlal Nehru University (in association with D.K. Printworld).

 \bibitem{chap05-key28} Korzybski, A. (1947). \textit{General Semantics; an Introduction to Non-Aristotelian Systems}. Lakeville, CT: Institute of General Semantics.

 \bibitem{chap05-key29} Kreps, D. (2014). \textit{Gramsci and Foucault: A Reassessment}. New York: Routledge.

 \bibitem{chap05-key30} Levine, P. (1995). \textit{Nietzsche and the Modern Crisis of the Humanities}. Albany: State University of New York Press.

 \bibitem{chap05-key31} Loewith, K. (1949). \textit{Meaning in History; the Theological Implications of the Philosophy of History}. Chicago: University of Chicago Press.

 \bibitem{chap05-key32} Malhotra, Rajiv. (2016). \textit{The Battle for Sanskrit: Is Sanskrit Political or Sacred, Oppressive or Liberating, Dead or Alive?}. New Delhi: Harper Collins.

 \bibitem{chap05-key33} Manning, C. D. (2015). “Computational Linguistics and Deep Learning”. \textit{Computational Linguistics}, 41(4). pp. 701-707. \url{doi:10.1162/coli_a_00239}

 \bibitem{chap05-key34} Miner, R. C. (2002). \textit{Vico: Genealogist of Modernity}. Notre Dame, Indiana: University of Notre Dame Press.

 \bibitem{chap05-key35} Mitra, Rajendralala and Vidyabhushana, Haradatta. (Ed.s) (1872). \textit{The Gopatha Brāhmaṇa of the Atharva Veda}. Calcutta: Asiatic Society of Bengal.

 \bibitem{chap05-key36} Nietzsche, F. W., (1909). \textit{On the Future of Our Educational Institutions: Homer and Classical Philology} (Kennedy, J.M. (Trans.)). Edinburgh: T.N. Foulis.

 \bibitem{chap05-key37} —. (1911). We Philologists: Volume 8 \url{ https://archive.org/details/wephilologists18267gut }, Released on 27 April, 2007. Accessed on September 24, 2016.

 \bibitem{chap05-key38} Olender, M., and Goldhammer, A. (1992). \textit{The Languages of Paradise: Race, Religion, and Philology in the Nineteenth Century}. Cambridge, MA: Harvard University Press.

 \bibitem{chap05-key39} Peile, John. (1870). \textit{Philology}. New York: American Book Company.

 \bibitem{chap05-key40} Phillips, Stephen. H., and Tatacharya, N.S. Ramanuja. (2004). \textit{Epistemology of Perception: Gaṅgeśa's Tattvacintāmaṇi: Jewel of Reflection on the Truth (about Epistemology), the Perception chapter (Pratyaksṇa-khaṇdṇa)}. New York: American Institute of Buddhist Studies.

 \bibitem{chap05-key41} Pollock, Sheldon. I. (1977). \textit{Aspects of Versification in Sanskrit Lyric Poetry}. New Haven, CT: American Oriental Society.

 \bibitem{chap05-key42} —. (1998). “The Cosmopolitan Vernacular”. \textit{The Journal of Asian Studies, 57(1)}, pp. 6-37. doi:10.2307/2659022.

 \bibitem{chap05-key43} —. (2001). “The Death of Sanskrit”. \textit{Comparative Studies in Society and History}, 43(2), pp 392-426. doi:10.1017/s001041750100353x

 \bibitem{chap05-key44} —. (2003). \textit{Literary Cultures in History: Reconstructions from South Asia}. Berkeley: University of California Press.

 \bibitem{chap05-key45} —. (2005). “The Revelation of Tradition: Śruti, Smṛti, and the Sanskrit Discourse of Power”. \textit{Boundaries, Dynamics And Construction Of Traditions In South Asia}, pp. 41-62. doi:10.7135/upo9781843313977.003.

 \bibitem{chap05-key46} —. (2005). “Working Papers On Sanskrit Knowledge-Systems On The Eve Of Colonialism”, \textit{Journal of Indian Philosophy}, 33(1), pp.1-1. doi:10.1007/s10781-004-9050-6

 \bibitem{chap05-key47} —. (2005). \textit{The Ends of Man at the End of Premodernity}. Amsterdam: Royal Netherlands Academy of Arts and Sciences.

 \bibitem{chap05-key48} —. (2006). \textit{The Language of the Gods in the World of Men: Sanskrit, Culture, and Power in Premodern India}. Berkeley: University of California Press.

 \bibitem{chap05-key49} —. (2007). “Pretextures Of Time”. \textit{History and Theory}, 46(3), pp 366-383. doi:10.1111/j.1468-2303.2007.00415.x.

 \bibitem{chap05-key50} —. (2009a). “Future Philology? The Fate of a Soft Science in a Hard World”. \textit{Critical Inquiry}, 35(4), pp 931-961. doi:10.1086/599594.

 \bibitem{chap05-key51} —. (Ed. and Trans.) (2009b). \textit{Bouquet of Rasa \& River of Rasa by Bhanudatta}. New York: New York University Press.

 \bibitem{chap05-key52} —. (2010). “Comparison Without Hegemony”. \textit{The Benefit of Broad Horizons}, pp.185-204. doi:10.1163/ej.9789004192843.i-436.67.

 \bibitem{chap05-key53} —. (2011). \textit{Forms of Knowledge in Early Modern Asia: Explorations in the Intellectual History of India and Tibet}, 1500-1800. Durham: Duke University Press.

 \bibitem{chap05-key54} —. (2012) “Liberation Philology”. Talk at Centre for the Study of Developing Societies, New Delhi on 19 December 2012. Published on May 1, 2014. Accessed September 25, 2016. \url{https://www.youtube.com/watch?v=C2gZKjbEoMo}

 \bibitem{chap05-key55} —. (2014). “Philology in Three Dimensions”. \textit{Postmedieval: A Journal of Medieval Cultural Studies Postmedieval}, 5(S4), pp 398-413. doi:10.1057/pmed.2014.33

 \bibitem{chap05-key56} —. (2015). “What Was Philology in Sanskrit?” \textit{World Philology}. pp. 114-136. doi:10.4159/harvard.9780674736122.c7

 \bibitem{chap05-key57} —. (2016.). \textit{A Rasa Reader: Classical Indian Aesthetics}. New York: Columbia University Press.

 \bibitem{chap05-key58} —., Elman, B. A., and Chang, K. K. (2015). \textit{World philology}. Cambridge, MA: Harvard University Press.

 \bibitem{chap05-key59} —., Koenig, C., and Schöning, B. (2015). \textit{Kritische Philologie Essays zu Literatur, Sprache und Macht in Indien und Europa}. Goettingen, Niedersachs: Wallstein.

 \bibitem{chap05-key60} Pompa, L. (1975). \textit{Vico: A study of the new science}. London: Cambridge University Press.

 \bibitem{chap05-key61} Porter, J. I. (2000). \textit{Nietzsche and the Philology of the Future}. Stanford, CA: Stanford University Press.

 \bibitem{chap05-key62} Price, D. H. (2016.). \textit{Cold War Anthropology: The CIA, the Pentagon, and the Growth of Dual Use Anthropology}. Durham, NC: Duke Unversity Press.

 \bibitem{chap05-key63} Schubert, L. (2014). “Computational Linguistics”. \textit{Stanford Encyclopedia of Philosophy}. \url{http://plato.stanford.edu/entries/computational-linguistics/}. Last revised on 26 February, 2014. Accessed September 27, 2016.

 \bibitem{chap05-key64} Speaks, Jeff. (2017). "Theories of Meaning". \textit{Stanford Encyclopedia of Philosophy Archive} (Spring 2017 Edition). \url{https://plato.stanford.edu/archives/spr2017/entries/meaning/}

 \bibitem{chap05-key65} \textit{Studies in honor of Maurice Bloomfield}. (1920). New Haven: Yale University Press.

 \bibitem{chap05-key66} Subrahmanyam, K., (Trans.) (1992). \textit{The Vākyapadīyam of Bhartṛhari, Brahmakāṇḍa}. Delhi: Sri Satguru Publications.

 \bibitem{chap05-key67} —. (2002). \textit{Four \textit{Vṛttis} in Pạ̄ini}. Hyderabad: The Author.

 \bibitem{chap05-key68} —. (2008). \textit{Theories of Language: Oriental and Occidental}. New Delhi: D.K. Printworld.

 \bibitem{chap05-key69} Tatacharya, N. S. Ramanuja. (2008). \textit{Śābdabodhamīmāṁsā: An Inquiry into Indian Theories of Verbal Cognition}. New Delhi: Rashtriya Sanskrit Sansthan/Institut francais de Pondichéry.

 \bibitem{chap05-key70} Tomasello, M. (2003). \textit{Constructing a Language: A Usage-based Theory of Language Acquisition}. Cambridge, MA: Harvard University Press.

 \bibitem{chap05-key71} Turner, J. (2014). \textit{Philology: The Forgotten Origins of the Modern Humanities}. New Jersey, USA: Princeton University Press.

 \bibitem{chap05-key72} Vadde, A. (2012). “The Re-Return to Philology”. \textit{NOVEL A Forum on Fiction}, 45(3), pp. 461-465. doi:10.1215/00295132-1723053.

 \bibitem{chap05-key73} Vico, G., and Pompa, L. (2002). \textit{Vico: The First New Science}. Cambridge, UK: Cambridge University Press.

 \end{thebibliography}

\theendnotes

