
\chapter{The Śāstra of Science \& the Science of Śāstra}

\Authorline{Sudarshan, T. N., and Madhusudan, T. N.}

\vskip 2pt

\section*{Abstract}

\vskip 2pt

The process of pūrvapakṣa of Sheldon Pollock’s (1985) paper on \textit{śāstra} in the First Swadeshi Indology Conference helped \textit{emic} scholars understand and possibly for the first time grasp the nuances and contrived intricacies of the methods of Pollock - but in the entire discourse a key issue remained undiscussed - the \textit{implicit} assumption of the superiority of Western Science, the scientific method and the superior rational nature of science when compared to the traditional body of knowledge viz. the \textit{śāstra}-s. The inherent and unfounded sense of Western superiority exhibited by scholarship especially regarding topics of a comparative anthropological/sociological nature, more so those involving study of artifacts with a civilizational or cultural focus is well-known. Academic narratives based on the (post-Renaissance) rise of colonial Europe in a material sense and also supposedly in a sense of “reason” have been the dominant narrative of scholarship in almost all of Western academia. Aided by the rise of the USA as a military and economic superpower, in the last century, the unquestioned myths of American Exceptionalism and its precursors, the narrative of “White-European” Greco-Roman origins of science and rationality remains deep-seated in the collective consciousness of the West and equally so in the deeply colonized mindsets of most Macaulayized (colonized/modern) Indians.

“The Science and Nescience of \textit{Śāstra}” (presented at the first Swadeshi Indology Conference held in 2016), albeit briefly, touched upon these issues (Sudarshan 2018). Such deep-seated unquestioned assumptions harm an objective understanding of \textit{śāstra} and also its applications to contemporary twenty-first century living. For a better understanding of \textit{śāstra} in all its nuances, a deeper comparative analysis of science and \textit{śāstra} is required. In this paper, the scientific nature of \textit{śāstra} is discussed and juxtaposed with the so-called modern lifestyles driven by scientific understandings of man, society and the world at large. The “\textit{śāstra}” of science is examined critically - its methods (logics/sources of knowledge) are examined (\textit{pramānṣa}-s), the actual realities of what is knowable (\textit{prameya}) by science are discussed and the goals (\textit{prayojana}) of science are elucidated. Common misunderstandings (especially among scholars with a non-science background and also of those who have accepted the assumptions of such superiority unquestioningly and uncritically) are addressed. In short - \textbf{\textit{Is science really all that it is made out to be?}} The key underlying questions about the nature of the relationship between praxis and theory in the “Indian traditions” way - exceedingly well articulated as a “research meme” by Pollock\endnote{though (as seen earlier) rather poorly framed within a political context} - are answered.


\section*{Introduction}

The Neo-Orientalist narrative on \textit{śāstra} is very much an exemplar of the deep-rooted chauvinism and prejudices of the West. We examine the basis of this prejudice using Pollock’s theses on \textit{śāstra} as an entry point. \textbf{What is the basis of the sense of the superiority of the West?} In all probability it is the discourse and narrative of Science and the thrall of the technology artifacts. We juxtapose \textit{śāstra} and science across some critical dimensions of interest to highlight hidden nuances, and hope to present a novel set of perspectives relevant to these discussions.

The rest of the paper is organized as follows: The ensuing section situates the context of these questions in the light of the Neo-orientalist narrative. The section on the “Narrative of Science” addresses the following: Is Science really Western in origin? How is the narrative and discourse of Science controlled and manipulated by the West? What are the philosophical underpinnings and logical frameworks in Science? How is “Science” perceived as a process for both acquiring and managing knowledge in current Western thought? In the section on “Narrative on \textit{śāstra}”, we provide an overview of the traditional viewpoints behind \textit{śāstra}, its role in \textit{dharma}, pursuit of the \textit{puruṣārtha}. The lifestyles of the Indian civilization are discussed both in traditional terms and in scientific terms. We highlight the key points under discussion and refer the reader to the vast extant literature on \textit{śāstra}. Finally, in the section on “Comparing \textit{śāstra} and Science”, we compare the twain and we attempt to understand science in terms of śāstra and the traditional lenses. We conclude with a brief pointer to the possibility of a critical role for Swadeshi Indology as an interpretive movement (for and of Science) - to create and nurture an honest narrative in the dharmic tradition.


\section*{The Neo-Orientalist Narrative}

Sheldon Pollock’s paper on Indian \textit{śāstra} (Pollock 1985) - is what one could consider unique scholarship. Coming possibly at an early time in his academic career in “South Asia” area-studies - it is an early indicator of the genre of scholarship that Pollock would produce for the next three decades as he became a powerful and influential figure affecting the Indian discourse both inside and outside India. Chronicled in detail and with uncanny insight in (Malhotra 2016) – Malhotra calls it \textbf{\textit{hegemonic discourse}}. Malhotra says that under the guise of “peer reviews” and by referring to each other’s works - a “coterie” or a cabal is formed, who are mutually supportive and perpetuate their own theories. He also adds that as this process continues over years, the need to reason and defend the theories and positions becomes minimal and their theories are “taken for granted”. Pollock for example via this “hegemonic discourse” has stated political goals for India – to intervene on behalf of those he declares to be oppressed. Interested readers can peruse Malhotra (2016:315).

Using a novel combination of hitherto unknown methods (3-dimen\-sion\-al philology, creative chronology and socio-political hermeneutic lenses) - various theses on the history of Sanskrit, the influences of Sanskrit and the associated bodies of knowledge and practice that comprise the basis of the \textit{sanātana dharma} and Indian civilization - have been propounded. The hegemonic discourse can be “seen” in its early stages in this paper by Pollock.

Unpacking these verbose misrepresentations and theses derived via Pollock’s creative hermeneutics, Malhotra lucidly describes Pollock’s approach to \textit{śāstra}.

The following summarizes the points Pollock makes in this 1985 paper:

\begin{itemize}
\item “Since the Vedas are considered eternal and perfect, they are assumed to be the repository of all knowledge. Therefore, shastras are incapable of fresh creativity and progress as they are limited to whatever is contained in the Vedas.

 \item Shastras can merely restate or extrapolate from what is already in the Vedas, but they cannot utilize fresh insights from the empirical world. He calls this ‘regressive’.

 \item No historical advancement is possible in the Western sense because shāstras are incapable of producing anything new.

 \item Shastras discourage individual agency, unlike in the West where individual agency is emphasized. This means the behaviour of Indian people is driven by codified rules that emanate from the Vedas.”\hfill Malhotra (2016:115)

\end{itemize}

For the purposes of this paper - in order to examine the deep rooted chauvinism and prejudices of the West, that underpin the aforementioned observations by Pollock - we shall attempt to examine the veracity and validity of the roots of the claims of science as being “Western” in origin, the Western-ness of the discourse and the one-sided narrative. Science also includes Mathematics for the purposes of this discussion. The discipline of philosophy of Science is also used to examine theories of Science using the Western viewpoints themselves.

For purposes of comparing Science \textit{vis-à-vis śāstra}, we first review each from the other’s viewpoint and highlight key differences. To examine Science using traditional Indian “theories” we shall in a limited manner use the Nyāya and Mīmāṁsā lenses. While examining \textit{śāstra} - the basic methods that supposedly exemplify and comprise “science” - the limited nature of the scientific method and the critiques of the philosophy of science are used as tools to ascertain the overall “scientific” nature of \textit{śāstra}.


\section*{The Narrative of Science }

In this section, rather than providing a historical account of Science as it is assumed to have evolved, we highlight some of the key shortcomings of these “documented” histories. Our focus has been on highlighting the following key viewpoints regarding the Western edifice of “Science” -

\begin{enumerate}
\item Are the “Greek” origins of Western Science and Philosophy really true?

 \item What was the role of Christian Theology in the evolution of Science as we see it now?

 \item What are some of the limitations of problems that science has addressed and the tools that science has used - especially “logic”

 \item Finally, how is “science” perceived as a process for “knowledge” generation? What guides it and how does it evolve?

\end{enumerate}

Having a proper background to the above questions is essential to refute the key points outlined in the earlier section on the limitations of \textit{śāstra}. In the rest of this section, we discuss each of the above.


\section*{Greek Origins of Western Philosophy and Science}

The discourse on the origins of Science (and Mathematics) has been controlled by the West till recently. As is the case with most such historiographies of the “West” - Greece is the undisputed source of all things “Western” (another hegemonic idea). Is this really true? Very little documented evidence that suggests the contrary is available. Consider the following remark on the book \textit{Stolen Legacy} (James 2001),

\begin{myquote}
In this work Professor James dares to contend and labor to prove, among others, that the \textbf{Greeks were not the authors of Greek philosophy'}, that 'so-called Greek philosophy' was based in the main upon ideas and concepts which were borrowed without acknowledgement--indeed \textbf{'stolen'}--by a few wayward and dishonest Greeks from the ancient Egyptians.\hfill Hansberry (1955:127) (emphasis ours)
\end{myquote}

Why is this narrative not mainstream knowledge? If one needed a vivid example of institutionalized hegemony – this could be it. Published in 1954, the book has not been popularized nor reprinted until recently by Moefi Asante, an African-American scholar. The lack of institutional “blessing” to these views; dangerous as they are to the “Western” narrative and the hegemony of history is apparent. The author of the book (Dr. James George) has been literally “erased” from academic history.

Was there \textbf{such a thing called Greek philosophy?} Dr. James is vehement that there isn’t really any such! Almost all of what is now considered “Greek” is actually (black) Egyptian in origin.

\begin{myquote}
The term Greek philosophy, to begin with is a misnomer, for there is no such philosophy in existence. The ancient Egyptians had developed a very complex religious system, called the Mysteries, which was also the first system of salvation.\hfill James (2009:7)
\end{myquote}

After the Persian invasion, from 60 BCE up to Alexander’s conquest the Greeks learnt most of all they could directly from the Egyptian priests. The plunder of books and entire libraries from Egypt and ascribing Greek origins to them (ex: a huge amount of books being attributed to Aristotle) is well known.

On the story of Plato and Aristotle, the relentless myth-building of these characters is also alluded to. The direct influence (he calls it “copy”) of the Egyptian (black African) cultures and knowledge is seen. Alexandria (in Egypt) had the largest (then) known storehouse and library of (Egyptian) scientific books. Much of the “knowledge” that has been attributed to Aristotle, Socrates and also Plato has well-known Egyptian and other non-European origins.

On the plagiarism by Plato, James has this to say

\begin{myquote}
Similarly, every school boy believes that when he hears or reads the names of the four cardinal virtues, he is hearing or reading names of virtues determined by Plato. Nothing has been more misleading, for the Egyptian Mystery System contained ten virtues, and \textbf{from this source Plato copied what have been called the four cardinal virtues, justice, wisdom, temperance, and courage. It is indeed surprising how, for centuries, the Greeks have been praised by the Western World for intellectual accomplishments which belong without a doubt to the Egyptians or the peoples of North Africa.}

~\hfill James (2009:8) (emphasis ours)
\end{myquote}

The “philosophy” of the West and its “Greek” origins are highly suspect. It is only the hegemonic nature of Academic Imperialism that is keeping these myths and untruths alive. It will take much serious work from the “affected” (those affected by Colonialism in the previous centuries and the more insidious, contemporary Academic Imperialism) peoples. The scholars who are pursuing these areas of research (Swadeshi scholars for example) have to realise their critical contemporary roles - if this continuing imperialism is to abate and for some sense of “truthful” balance to return to the global civilizational discourse.

\vskip 3pt

We now continue the \textbf{\textit{pūrvapakṣa}} of “Western” science and math based on the decades-long research of Prof. C.K.Raju.\endnote{\url{ http://ckraju.net/papers/Reading-list-on-religion-in-math.html}
 \url{http://ckraju.net/papers/Reading-list-on-history-philosophy-of-math.html}
 \url{http://ckraju.net/papers/History-philosophy-of-math.html}} Much of his work is not well-known in India, even among academics and supposed scholars both in the sciences and “social-sciences”. His well-researched critiques of the origins of Western math and science have had very few (if any) credible rebuttals and critiques. As a practicing (world-class) scientist and award-winning teacher, his theses have all the more veracity as they are wrought from experience of pedagogy in multicultural environments.

\vskip 4pt

\textbf{Note:}

\vskip 4pt

\textit{Prof. C.K.Raju is quoted extensively in the succeeding sections This is simply because he has been and still is the pioneer in the study of the Academic Imperialism in the Math and Sciences and has successfully researched and unearthed the deep Colonial roots of Math and Science. He is at the forefront in the Global fight against Western Academic Imperialism and has practically engaged with the West’s machinery of hegemony both as a scientist and also as an Educator over decades. He has been addressing these issues both from a ‘general” non-West perspective (African, Buddhist, Islamic, Chinese among others) and also specifically from the Indian. In our opinion no other contemporary or past scholar or scientist comes anywhere close to the depth, range and clarity that Prof. Raju offers on the nature of Colonization of Math and Science. Among his many works, we also quote from Prof. Raju’s paper on using the “History and Philosophy of Science as a means of Decolonisation”. This is an unpublished paper by him. The observations in the paper strike at the root of the Colonisation in Science and Mathematics, exemplified by the Journals themselves. A link to the paper at Prof. Raju’s blog is provided in the References. The page numbers for quotes from this paper will be the manual numbers of the pdf as none exist in the actual paper. The reference will appear as Raju, C. K. (?) in the quotes.}

\vskip 3pt

On the Indian colonial experience of science and math based on the supposed superiority of the Western methods, Raju makes these scathing remarks on the Indian perpetrators – especially referring to Rammohun Roy’s fascination with Western science and education and his insistence on supporting Macaulay’s ideas. Raju attributes it to Rammohun Roy being \textbf{conned} by the false history of Science.

The false history that is considered “truth” and which is part of the mainstream narrative of science - the “fabricated” Western origins of astronomy - is made explicit by Raju. He presents the example of Astronomy thus:

\begin{myquote}
“There never was any serious Greek tradition of astronomy. The Greeks were hopeless at arithmetic, as demonstrated by the non-textual evidence of their (Attic) numeral system and their calendar, which was grossly inaccurate and in complete disarray like its more refined descendant, the Roman calendar.”\hfill (Raju?:1)
\end{myquote}

He cites the superstitious nature of the Greeks (with regard to astronomy) and alludes to Aristotle’s death penalty for his contemplations on the nature of the sun and the moon.

Is there any proof to the Greek “expertise” in astronomy? Absolutely none. In fact, the proof conclusively points us in the opposite direction. According to Raju, much of what is taken to be “Ptolemy’s work” can be considered to be fictitious and wrongly attested. There is sufficient evidence that much of the “numbers” were back-calculated. The text in question, \textit{Syntaxis}, is translated from Arabic and not the other way round as is popularly portrayed. Raju openly challenges Western scholarship to answer his charges (on Greek astronomy) and feels that almost all of it is pure Western fantasy. For those who aspire to be decolonised – Raju suggests that they just move on \textit{critically}. The West will not acknowledge their centuries of dishonesty and falsehood.


\section*{Role of Christian theology in the evolution of Science}

After the initial attribution to the Greeks, the theater of Science advances nearly 10-15 centuries to the end of the Middle Ages. This was a period when the church was in ascendancy and controlled all intellectual discourse.

So, what about the fabled Copernicus and the great story of the inquisition and the rise of Science and all of that? Copernicus got his “knowledge” from Islamic sources (Ibn Shatir of Damishk, and the Maragheh school of Khwaja Nasiruddin Tusi). He was nothing more than a mere translator. The usual arguments of “independent discovery” are often offered as argument, but just as with other similar myths, this one too is of a perpetuated variety.

How about Newton, Tycho Brahe, Kepler, Euler? Well, sadly but not surprisingly, it turns out that not a single one of them is very original either. This claim might seem “controversial” and of the “crackpot” variety, but readers are advised to peruse Prof. Raju’s tome on the History of Calculus, \textit{The Cultural Foundations of Mathematics} (Raju 2007). We would not like discuss more on the “bogus histories”, fictitious mathematicians and scientists. For the purposes of this paper, it suffices to say the “depths” of Western academia and related history writing are yet to be plumbed. The “Western” claims to the origins and ownership of “Science” are seriously in question. The core of Pollock’s thesis of the “superiority of the West” rests on this \textit{bogus assertion that the West created Science.}

So, what are the implications of this false history of Science? What purpose did such a fabricated history serve? What does it mean for us (Indians), colonized by the West for a few hundred years and still in the thrall of the West? What does it mean for science education? What does it mean for the future of math and science?

Raju summarises the “deep” issues and places Academia (and academics like Pollock) in perspective – that they serve only the interests of hegemony. Is there a way to address this deep problem? A majority of Indians including the so-called intellectual class are not even aware of these foundational issues. We continue to be slaves to these false histories. Prof. Raju is scathing in his analysis - the false history is not only bad for us from a civilizational perspective, but is also affecting how math and science will evolve in the future. So, how does correcting the false history help? Raju details how a correction of false history and philosophy will improve pedagogy and lead to better Maths and Science. To know \textit{how exactly} this will happen, readers are once again fervently advised to deeply assimilate Prof. Raju’s work and also appreciate the “working” results from his real world (decolonised math and science) pedagogies.

With reference to the “civilizational” clashes and the hegemony of the West, Raju explicates his theses on the basis of some very real “truths” interpreted and analyzed as only he can - why correcting false history is important for future of Math in his paper, \textit{Math Wars} and \textit{the Epistemic Divide in Mathematics.}

It is well-known that the Europeans inherited math from two traditions the anti-empirical Greek, Egyptian and the empirical Indian, Arab. How did the West reconcile these traditions? They did not. The Church found it convenient to use the non-empirical, axiomatic proof based system of math as more convenient for its goals of “expansion” and as a basis for its “metaphysics”. See Raju (2004) for a detailed treatment of the role of the “Church” and its use of “proof” based mathematics – the deductive method and “axioms” aligned well with the “proselytising” needs of the Church (every “piece” of knowledge had to have the approval of the church). Sometimes it was done violently and is very possibly the reason for the famous missing work of Newton (he was a Christian theologian – with unpublished works on the history of the church).

How many of us are aware that today’s math and science is deeply is influenced by Christian Theology? This can be seen even today. Pure Mathematics is that which is practiced in a theologically correct way i.e., the axiomatic basis on which “proofs” are constructed without any means of calculating or verifying the claims.


\section*{The Logic of Science}

Scientific Reasoning in the 17th century was powered by the evolution of \textit{Empiricism} and more importantly \textit{Logical empiricism} as science and math co-evolved. As indicated earlier, theory development was given precedence (involving the use of proofs) followed by possible experimental validation, which is what is practised to date. As theory building became more important, it was important to bound the theory development using proof-based logical systems. Proof-based mathematics and its use in understanding the empirical world has led to a number of conceptual bottlenecks which are still being resolved.

\newpage

So why is proof based mathematics unsound? The underlying metaphysics and axioms are not universal but are based on some peculiarities of the West. Why then should such math be considered universal - hegemony? For this we need to go deeper. We need to grasp the notions of logic, inference and deduction.

We again take recourse to Raju’s work on Logic i.e., Non-Western Logic (2004). So, what is this Western logic - Is it universal? Can it be used as basis for universal math and science? It turns that it is not universal and cannot be used as a basis for “universal” math and science.

\begin{myquote}
\textbf{.. proofs by contradiction are common in present-day mathematics. However, such a deduction would be invalid with a variety of logics that one can conceive of. The alleged certainty of deduction, therefore, rests on the belief that two-valued logic is universal or at least special in some way.}\hfill Raju (2008:1230) (emphasis ours)
\end{myquote}

What about the logics used in India?

\begin{myquote}
However, the various logics used for inference in India, prior to even the historical Aristotle, \textbf{were neither two-valued nor even\general{\break } truth-functional}\hfill Raju (2008:1231) (emphasis ours)
\end{myquote}

New Logics were formulated by each school/ \textit{darśana} as was deemed necessary by the metaphysical requirements underlying the peculiarities of each philosophical school. No logic was considered universal. All logics were in fact \textbf{considered limiting} as they were man-made. There were also supra-logical schools of thought and \textit{darśana}-s based on the fact that (man-made) logics were limiting by definition (the \textit{bhakti} and \textit{nyāsa} traditions of the Vaiṣṇava-s for example). Modern masters like Sri Aurobindo also used these “supra” frameworks to elucidate and articulate new interpretations and theses on the possibilities for the future evolution of consciousness and of mankind.

On Buddhist Logic: There is much more to Buddhist Logic than alluded to by Raju below - but is a good summary.

\begin{myquote}
Based on the Dīgha Nikāya - four truth cases are systematically used by later-day Buddhist thinkers like Nāgārjuna and Dinnāga who taught at the University of Nalanda. The latter developed a theory of (logical) quantifiers, “for all”, “for some” etc., based on this sort of logic. From the perspective of present-day formalist treatments of logic, it should be noted that \textbf{Buddhist logic is not a multi-valued but is rather a quasi truth-functional logic.}\hfill Raju (2008:1231) (emphasis ours)
\end{myquote}

On Jaina logic: Again there is much to Jaina logic but these salients are sufficient to make the argument for non-Western logics.

\begin{myquote}
The Jains had a related but different logic called the logic of \textbf{\textit{syādavāda}} (\textit{sic}) \textbf{(“perhaps-ism”), based on the idea of \textit{anekāntavāda} (no-one-point-of-view-ism). Attributed to Bhadrabāhu, instead of four alternatives, this logic has a seven-fold judgment (\textit{saptabhangīnaya})} based on seven possible combinations of three primary values.

~\hfill Raju (2008:1231) (emphasis ours) (diacritics as in the original)
\end{myquote}

How do these non-Western logics fit into semantics of modern logics?

\begin{myquote}
In terms of the present-day formal semantics of logical worlds, one might put things as follows. The \textbf{different possibilities visualized in Buddhist and Jaina logic refer not to multiple logical worlds assigned to different instants of time, but to multiple logical worlds assigned to a single instant of time.} In other words, Buddhist and Jaina logics relate to a world-view in which \textbf{time is perceived to have a non-trivial structure}, an (atomic) instant of time is perceived not as a featureless geometrical point but as a microcosm. Hence, members of a contradictory pair can well be simultaneously true.

~\hfill Raju (2008:1231) (emphasis ours)
\end{myquote}

Recent work on Kripke logics (circa 1950) considers the issue of \textit{multiple possible world semantics}. So now, mathematically and culturally, what does this entail? That Western logic is not universal. That it is a cultural choice and so are the mathematics and the methods based on it.

What about the notion of truth-value and its relationship to the empirical based on actual sensory experience? What are the effects on science and the way it is practiced?

\begin{myquote}
The possibility or \textbf{necessity of determining logic empirically however strikes at the root of another fundamental difference between Western and non-Western perceptions of logic}. In the West, \textbf{logical truths are regarded as necessarily true, and are privileged over empirical facts}, regarded as being only contingently true. Hence, \textbf{present-day mathematical proof is required not to involve the empirical}, since that would diminish the sureness attached to a mathematical theorem. Hence, also, the \textbf{present-day belief in the philosophy of science, that when the conclusions of a physical theory are refuted by experiment it is the hypotheses that stand refuted}, and \textbf{not the process of inference which led from hypothesis to conclusion}. (Here it is necessary to distinguish between validity and correctness. The point is that \textbf{it is believed that no empirical fact can invalidate a correct mathematical proof.})

~\hfill Raju (2008:1232)(emphasis ours)
\end{myquote}

What are the consequences if we were to actually make these fundamental “observations” mainstream? The power of Western institutionalization and the hegemonic “scientific” discourse will not let these views take hold, unless challenged vigorously. \textbf{Indian civilization is the only worthwhile challenge left -}

\begin{myquote}
Therefore, \textbf{even if one were to go about trying to settle the nature of logic empirically, this would have consequences, startling from a Western perspective. Empirical observations are fallible, and subject to revision. So if the nature of logic is decided empirically, logical truth would have to be regarded as more fallible than empirical truth: deduction would have to be regarded as more fallible than induction, since the nature of the logic used for deduction could only be decided inductively. This would stand much of Western thought on its head.}\hfill Raju (2008:1232) (emphasis ours)
\end{myquote}

So, will it affect non-Western logic systems? Interestingly, no.

So, what should one do ideally to address this fundamental issue in the greater considerations of humankind? Raju suggests that much of Western thought will need to reworked and recreated.

\begin{myquote}
Thus, there \textbf{appears to be no serious way out of this dilemma about the nature of logic, and most of Western thought would hence need to be reworked in in the future to avoid this incorrect assumption that two-valued logic is somehow universal.}

~\hfill Raju (2008:1233) (emphasis ours)
\end{myquote}

So, how did this flawed understanding of logic actually come about and how did it assume the status of (hegemonic?) Universal \textbf{“truth”}? Again, we see a similar pattern of events - Greek beginnings, church modulates and post-renaissance - it becomes hegemony.

The customary purported Greek beginnings, followed by “oppositon” of its use by the Church (as it questioned the doctrines of creation and apocalypse), the “persecution” by the Church of the “logic philosophers”, their flight to Islamic refuge, the fine-tuning of the ideas of logic based on interactions with Islamic philosphers and then the movement of Arabic ideas and knowledge into Europe (which was considered heresy during the Crusades).

\newpage

So, how was this Islamic import made theologically correct – the usual whitewashing of history by the Church and the creative falsehoods perpetuated by organized “educators” and Academia.

From the Indian perspective - the influence of Indian school of Nyāya has also been conveniently ignored and can be considered collateral damage,

\begin{myquote}
In the process of denying the Arabic-Islamic contribution, \textbf{the Indian contribution from the Nyāya school, which used a similar system of syllogisms (with two valued logic), and was probably translated in the Bayt al Hikmā, may also have been denied.}

~\hfill Raju (2008:1234) (emphasis ours)
\end{myquote}

The Christianization continues unabated and is today considered \textbf{\textit{“Universal”}} truth and is the basis of modern pedagogy - \textbf{\textit{“Christian”}} Mathematics and Science.


\section*{The Philosophies of Science}

The discipline of “philosophy” of science is a part of the limited Western framework of culturally and politically correct self-critique. Evolving in parallel with the practice of science over the past few centuries - the very assumptions of science are supposedly questioned. There is still no “clear” definition of what it actually studies. There are close to 2500 entries\endnote{\url{http://plato.stanford.edu/search/search?query=Philosophy+of+science}} for “Philosophy of Science” in the Stanford Encyclopedia of Philosophy, but no single definition of what it \textit{exactly} is.

Simply because of this definitional epistemological anarchy, as a reaction, one can see deep specializations in specific areas and sub-disci\-plines of the philosophy of Science. There are philosophies pertaining to high-level areas such as physics, chemistry, biology and also philosophies attributed to specific persons like Einstein’s philosophy, Kant’s philosophy etc. The modern and postmodern critiques of science including the “science as a delusion” perspective in (Sheldrake 2012) only exemplify Feyerabend’s devastating foundational observation on the anarchy of science.

This narrative of Feyerabend, and other philosophers of science influenced by him, is basically a narrative of the \textbf{“disunity”} of Science, in opposition to the highly influential thesis of the Unity of Science.

\begin{myquote}
Feyerabend sometimes also recognized that this is to present science as too much of a monolith. In most of his work after \textit{Against Method}, he emphasizes what has come to be known as the \textbf{“disunity of science”}. Science, he insists, \textbf{is a collage, not a system or a unified project. Not only does it include plenty of components derived from distinctly “non-scientific” disciplines, but these components are often vital parts of the “progress” science has made (using whatever criterion of progress you prefer). Science is a collection of theories, practices, research traditions and world-views whose range of application is not well-determined and whose merits vary to a great extent. All this can be summed up in his slogan: “Science is not one thing, it is many.”}\hfill Preston (2016) (emphasis ours)
\end{myquote}

Feyerabend was also, controversially, for the separation of Science and State, in lines similar to the separation of Church and State.

\begin{myquote}
\textbf{.. science is much closer to myth than a scientific philosophy is prepared to admit. It is one of the many forms of thought that have been developed by man, and not necessarily the best. It is conspicuous, noisy, and impudent, but it is inherently superior only for those who have already decided in favour of a certain ideology, or who have accepted it without ever having examined its advantages and its limits}
\end{myquote}

\begin{myquote}
The \textbf{separation of church and state should therefore be supplemented by the separation of science and state, in order for us to achieve the humanity we are capable of}. Setting up the ideal of a free society as \textbf{“a society in which all traditions have equal rights and equal access to the centres of power”}\hfill Preston (2016) (emphasis ours)
\end{myquote}

Though considered controversial, these are serious observations made by a contemporary (Feyerabend died in 1994) philosopher of science.

From a Swadeshi perspective, some of Feyerabend’s unfinished work is critical and needs to be leveraged, not in the least to effectively articulate and synthesize the arguments \textbf{for} tradition.

\begin{myquote}
One of the projects which Feyerabend \textbf{worked on for a long time, but never really brought to completion}, went under the name “The Rise of Western Rationalism”. Under this umbrella he hoped to show that Reason (with a capital “R”) and \textbf{Science had displaced the binding principles of previous world-views not as the result of having won an argument, but as the result of power-play}. Even \textbf{nowadays, indigenous cultures and counter-cultural practices provide alternatives to Reason and that nasty Western science}.~\hfill Preston (2016) (emphasis ours)
\end{myquote}

We now briefly examine what one of the proponents of the Stanford School of “Philosophy of Science”, Nancy Cartwright. Introducing Cartwright’s philosophy and commenting on her book, \textit{How the laws of Physics lie?} (HTLPL) (Cartwright 1993), Hoefer (2008:2) writes:

\begin{myquote}
Cartwright mounts her first sustained attack on two aspects of philosophy of science that she believes are deeply mistaken: \textbf{its rejection, based on a tradition beginning with Hume and reinforced by Russell, of causality and causal laws and its claim that finding and applying true laws of nature (typically in physics) is central to the success of science.}\hfill Hoefer (2008:2) (emphasis ours)
\end{myquote}

\begin{myquote}
HTLPL discusses laws of all sorts: fundamental physical laws, less-fundamental equations, high-level phenomenological laws, and causal laws. Cartwright’s arguments go to show that \textbf{only causal laws, and some high-level phenomenological laws in physics, can be held to be literally true, even in a restricted domain of application; and all true laws are to be understood as merely true \textit{ceteris paribus} —all else being equal, or better: when conditions are right. Why is truth such a rare and hedged quality for the laws of physics?}

~\hfill Hoefer (2008:3) (emphasis ours)
\end{myquote}

Causality in phenomena is the casualty when one strictly applies the laws of physics and closely studies the experimental practice of physics \textit{vis-à-vis} the “theories”.

\begin{myquote}
The laws of physics do a lot of explanatory work for us, but that does not argue for their truth. Inference to the \textbf{best explanation makes sense when one is inferring to the most probable cause but not when one is inferring to the alleged truth of a fundamental equation.}

~\hfill Hoefer (2008:4) (emphasis ours)
\end{myquote}

Hoefer goes on to write about her second book, \textit{The Dappled World} (DW)(Cartwright 1999):

\begin{myquote}
In DW, Cartwright goes beyond the view of science that she offered in HTLPL by offering a reconceptualized understanding of laws of nature (causal or otherwise) and a metaphysics (the dappled world) with which to replace the fundamentalist’s reductionist world of particles moved by laws. \textbf{Laws, to the extent that we need them, arise because of, and are true only in, nomological machines: setups, usually made by us but sometimes found in nature, that combine a simple/stable structure and sufficient shielding from outside influences so as to give rise to regular behavior.}\hfill Hoefer (2008:5) (emphasis ours)
\end{myquote}

What are the implications of Cartwright’s theses? There are no credible critiques to her theses as yet. The “contrived” truths and false-universality of the discourse of Science are apparent for all those who question science with an open mind.Cartwright’s theses only add credibility(particularly as she is a mainstream “acknowledged” scholar) to the observations of James and Raju (seen earlier, both of whom would be considered “outsiders” to the Western narrative in many ways).

Given the discussion in this section (\textit{the essentially hegemonic roots and the civilisationally biased nature of Western academia}), it is not surprisng that the \textit{mental paradigm that dominates Pollock’s assessment of the śāstra}-s is as outlined in the section on the Neo-Orientalist narrative.


\section*{The Narrative on Śāstra}

The role of \textit{śāstra} as a foundational “construct” of Indian civilizational existence has been discussed previously (Sudarshan 2018). The “\textit{śāstra}- s” encapsulate continual learnings and primordial truths and make them available for scholarly access and interpretive dissemination via techniques unique to the cultural/civilizational tradition. Aurobindo, Gandhi, Tagore and various other modern masters have in their unique ways, articulated the role of \textit{śāstra} in the Indian civilizational journey. The living role of \textit{śāstra} is being re-contextualized on a daily basis across Indian homes and via the societal channels of interpretation and dissemination (various \textit{sampradāya}-s, local temples, \textit{guru paramparā} traditions etc.)

The \textit{śāstra}-s - the closest Western understanding is a “\textit{-logy}” which is derived from Greek “logia” which means “communications of a divine origin”. They can be understood as \textbf{accretive} bodies of knowledge also – knowledge of topics of interest spanning the human experience (across the vast geographies of the Indian sub-continent) and “encapsulated” formally over millennia. There are \textit{śāstra} for every conceivable human practice (at least there were, till the middle of the 2nd millennium of the Common Era). Conception and articulation of a new \textit{śāstra} have been rare since 1500 C.E.

Etymologically, “\textit{śās-tra}” is “that which protects”. The traditional classification of \textit{śāstra} (according to Yājñavalkya) are the 14 \textit{vidyāsthāna}-s. There are other classifications too. For purposes of this discussion, we shall limit our discussion to these.

The recording of the material manifestations of the Indian Sciences and Technology (the material \textit{śāstra}-s, if you will) have begun recently and are slowly being acknowledged as “scientific” heritage deriving from the traditional knowledge systems and civilizational “experiences”. Only recently has the “history” of the material achievements of the Indian “material” masters been documented or even acknowledged. These “histories” have been hidden from popular consciousness by years of foreign rule and by the overzealous leftist narratives of Indian history. Without doubt, all of this needs to be made mainstream knowledge.

The HIST series of books brought out by The Infinity Foundation exemplify this via specialized books, and showcase unparalleled achievements in specific areas of technologies going back millennia. The large numbers of cynical Western academics and critics of Indian civilization are referred to Balasubramaniam (2008), Chattopadhyay (2011), Dharampal (2000), Joshi (2008, 2009), Tripathi (2008) for proof of India’s civilizational achievements in the material plane, sciences and technology. Such readers are also referred to Dharampal’s record of Indian Science in the 18th century and data pertaining to India’s traditional education system based on the traditional \textit{śāstra}-s (Dharampal 1983). The much well-known work on \textit{Millenial Perspectives on the World Economy} by Angus Maddison (2003) only reinforces these facts.

Our narrative on \textit{śāstra} in this paper is quite minimal in contrast to the earlier section given our (unfortunate) deep familiarity with the Western hegemony. Much remains to be discovered, documented and disseminated about \textit{śāstra}. The hegemonic Western \textit{history} discourse has to be overturned. The process has just begun. We outline key summary features of \textit{śāstra}-s that are relevant to this discussion.


\section*{The Prayojana of śāstra}

The principal \textit{śāstra}-s deal with holistic “harmonious” living and assume the foundations of common tenets of Vedic cosmology. They act as a guide to traverse the well understood “\textit{āśrama}-s” of life. The pursuit of \textit{puruṣārtha} via the guidelines of the \textit{dharma}-s is considered supremely important and the only worthwhile goal of the human (birth). Every (recommended) activity of the human is to be seen in the context of the universal and divine frame of reference in order to understand its real purpose. The purport of the \textit{śāstraic} guidelines are the (demystification of the) “\textit{vidhi}-s” or the recommended rules. The role of the \textit{guru}-s (living masters) is to guide the individual and hence society toward stable and \textit{dhārmic} living, leading the society to the right pursuits of the \textit{puruṣārtha} via the proper interpretation of the \textit{śāstra}.

The \textit{pramāṇa} (source of knowledge) of the \textit{śāstra} is primarily \textit{śabda} (the \textit{Veda}) only. The \textit{prameya} (subject matter) of the primary \textit{śāstra}-s is (access to) the (otherwise unknowable) knowledge of the Supreme. The \textit{prayojana} of \textit{śāstra}-s as indicated is the \textit{dhārmic} pursuit of the “material” \textit{puruṣārtha} –s (\textit{artha} and \textit{kāma}), finally leading to states of higher awareness, realization and consciousness. \textit{Sanātana Dharma} (including Buddhism, Jainism and Sikhism) has a large “living” repository of precepts and practices (\textit{darśana}-s, \textit{sampradāya}-s), applicable/suitable to a wide variety of \textit{guṇa} configurations/needs of a particular society or individual. Achieving a state of Liberation via techniques leading to Oneness (in some form or the other) with the supreme consciousness is the underlying goal of all practice embodied by the \textit{śāstra}-s.

Additional salient points to note with respect to \textit{śāstra} in the context of this paper include

\begin{itemize}
\item \textit{Śāstra}-s (though revealed) still allow for the fact that additional \textit{śāstra}-s may be revealed even contemporaneously to the prepared seeker.

 \item \textit{Śāstra}-s allow discovery/re-discovery, re-interpretation, adaptation depending on the context. They are not history-centric nor owner-/discoverer-centric. 

 \item They depend on the practitioner. \textit{Śāstra}-s do not distinguish between Western knowledge taxonomies such as science, humanities, sociology, morality, law etc.

 \item So, analyzing/classifying them according to Western knowledge systems is unproductive at best and leads to numerous limited interpretations. This \textbf{has to stop.}

 \item We need to treat \textit{śāstra}-s as an alternative knowledge system with its own sources and machinery for maintaining/evolving that knowledge.

 \item Finally, all the \textit{śāstra}-s are “dharmically” compatible for the individual, the society (both human and others) and the world as it were.

 \item They are sacred and by definition do not embed any social ills as suggested by the Neo-orientalist perspective.

\end{itemize}


\section*{Comparing Śāstra and Science}

Our ensuing comparison attempts to delineate the “science” underlying \textit{śāstra} followed by an analysis treating Science as a \textit{śāstra}. The discussion highlights the fundamental differences between these two constructs and also suggests various approaches to combating the destructive processes such as digestion outlined in Malhotra (2011).


\section*{The Science of Śāstra}

We shall in brief attempt to understand the ‘science” of \textit{śāstra}. Science as defined in the West requires foundational concepts that can be observed, measured and related with each other (for example, mass, gravity etc.). It requires theories that explain and predict interactions between these foundational concepts. Empiricism plays the fringe role of validating these theories via experimentation.

\textit{Śāstra} has such foundational constructs and different \textit{śāstra}-s elucidate various theories and practices. A few concepts are highlighted.

\textit{Śāstra}-s acknowledge the existence of a primordial consciousness. Matter is not the ultimate reality. The various \textit{darśana}-s, in their own ways, acknowledge the primacy\index{primacy} of consciousness to matter. The primary elements of reality, though differently conceived in the various schools, have an acknowledgement of this duality. Matter is either considered to be an evolute of consciousness\index{consciousness} or considered to have an independent (lower) reality/existence. That there is something akin to a supreme\index{supreme} is acknowledged.

If one were to take a “causal” and utilitarian view of the \textit{śāstra}-s, what does the scientific pursuit of \textit{vidyā}-s ingrained in the \textit{śāstra}-s help achieve? The scientific practice of the śāstra-s helps the materially (body) bound unique elemental piece of consciousness (\textit{ātman}) identified (due to the “ego”) as “I”, understand\index{understantd} its true identity. The pursuit of the paths\index{paths} leading to this understanding of true identity is the \textit{praxis}\index{praxis@\textit{praxis}} of human life (according to \textit{Sanātana Dharma}). The \textit{śāstra}-s scientifically help in this pursuit. 

The Vedic understanding - of time (via \textit{Jyautiṣaśāstra}), space (vibrations via \textit{śikṣā, chandas Nirukta}), ego (via \textit{Vedānta}), body/\textit{śarīra} (via \textit{Āyurveda}), valid practices (via \textit{Kalpa}) and the external material manifestations of nature - is “scientifically” manifest in the \textit{śāstra}-s. The “proper” contextual (time, space and other dimensions including \textit{dharma}) practice of \textit{śāstra} is the recommended “praxis” of human life.

The salient points again are the following:

\begin{itemize}
\item Foundational constructs exist and have been studied and analyzed as in the Western Sciences. However, these constructs are far richer and holistic than Western knowledge systems.

 \item Secondly, it is not required that these knowledge systems use a language, methodology or exhibit structure both in theories and concepts that are similar to current Western notions.There are potentially many ways to conceptualize and understand the world around us.

\end{itemize}


\section*{The Śāstra of Science}

Based on the elaborate and perhaps enlightening discussion on the Narrative of Science, we attempt to answer the question - What could be the \textit{śāstra} of Science? What are the \textit{puruṣārtha} -s underpinning the practise of Science? What are the boundaries of Science?

Science and Technology at first glance seem to have improved the human condition on many fronts - empowering the individual and the state. However much of this has come at great cost to human life and natural resources. The use of science and its artifacts as instruments of human power and plunder also come to mind, given the rise in colonization with the early evolution of science and technology. Colonization and its concomitant bag of ills, such as apartheid, slavery, and many more, are still continuing and it is not some ancient memory. Furthermore, even with a view restricted to \textbf{\textit{“Science as a knowledge generation mechanism”}}, many ills abound as discussed in the section on Philosophy of Science. There are many problems where current Science does not have answers (considering its constructs and methodologies are limited). Though science is trying to expand its methodological toolkit by borrowing from \textit{śāstra}-s, much remains to be done.

Overall, the pursuit of pleasure, power, and sensory experiences seem to be some of the most eligible \textit{puruṣārtha} -s. Towards this end, indiscriminate exploitation of natural resources is the \textit{prima} facie generator of advancement and wealth. From a \textbf{\textit{“Śāstra} of Science”} viewpoint, the \textbf{net} effects of Science as practised today have far more harmful effects than beneficial. Some major issues relevant to all humanity, include issues such as - human-induced species loss, the overall drop in quality of life and happiness, the rise of fundamentalist ideologies, and the imminent threat of nuclear holocausts - to name but a few. We briefly highlight a few issues here.

We use findings of the deeply disturbing paper on human-induced species losses in the Journal \textit{Science Advances}, of AAAS (American Association for the Advancement of Science), to make the case against Science.

\begin{myquote}
\textbf{The rate of extinction for species in the 20th century was up to 100 times higher than it would have been without man’s impact, they said. Many conservationists have been warning for years that a mass extinction event akin to the one that wiped out the dinosaurs is occurring as humans degrade and destroy habitats.}

~\hfill Ceballos et al. (2016:1)
\end{myquote}

For a brutal reality check as to what the mindless “pursuit” of lifestyle models based on Western science is causing –

\begin{myquote}
Even under our assumptions, which would tend to minimize evidence of an incipient mass extinction, the average rate of vertebrate species loss over the last century is up to 100 times higher than the background rate. Under the 2 E/MSY background rate, \textbf{the number of species that have gone extinct in the last century would have taken, depending on the vertebrate taxon, between 800 and 10,000 years to disappear}. These estimates reveal an exceptionally rapid loss of biodiversity over the last few centuries, indicating that a sixth mass extinction is already under way. Averting a dramatic decay of biodiversity and the subsequent loss of ecosystem services is still possible through intensified conservation efforts, but that window of opportunity is rapidly closing.

~\hfill Ceballos et al (2016:1)
\end{myquote}

For the critics and defenders of anthropocentric Western civilization there is data available here\endnote{\url{ http://advances.sciencemag.org/content/1/5/e1400253.figures-only}}. Any \textbf{sane}\index{sane} human would without hesitation acknowledge the effects of the amoral and destructive models of (scientific) Western living.

Assuming the primacy\index{primace} of the anthropocentric focus of Western science, can we ask questions on the role of science in improving the condition\index{condition} of humans at least? What has been the role of science in the progress of man, the individual?\index{man, the individual?} Is humankind better off after all this destruction?

Is the “Western (Westernized)” individual happier, living a better and more fulfilling life etc.? Social and Psychological indicators from Westernized societies do not indicate any “positives” in this direction either. Summarizing the view of Science from a \textit{śāstraic} lens, we highlight the few points relevant to this paper:

\begin{enumerate}
\item “Science” is a subset of \textit{śāstra}. It only decouples (for ease of morality) knowledge discovery from its use and applicability. 

 \item Science only addresses a narrow set of problems successfully. Problems that are complex, both natural and artificial, have not yet been amenable to the scientific method. 

 \item Applying the “scientific” method does not inherently ensure a valid or even workable conclusion.

 \item Paradigms in Science are limited.

 \item \textit{Śāstra} is a far bigger concept and construct than Science.

 \item Limiting our world-view through the lenses of Science and its hegemony should be carefully avoided.

\end{enumerate}


\section*{Remarks on the Neo-Orientalist viewpoint}

Given the aforementioned discussions thus far on the roots of Western Science, \textit{śāstra} and the inherent differences between the two, we can draw the following conclusions regarding the Neo-orientalist viewpoint (which are in italics) discussed in the section on the Neo-Orientalist narrative.

\textbf{\textit{Śāstra}-s} are “\textit{static}” or \textit{limiting as compared to “Western Science”}

We have highlighted that this is not even true nor a fair hypothesis even if it were one. Western Science is the one that is limited and quasi-static.

\textit{Lack of empiricism in śāstra-centric approach}

We highlight that “current” science/math is not even “really” empirical - being dominated by christian theological foundations. \textit{Śāstra}-s with the embodied approaches to knowledge acquisition are foundationally based on first-person empiricism.

\textit{Nothing “new” can come from śāstra}

It is predicated on the implicit assumption that everything new is relevant and beneficial. The notion of “newness” is relative to the observer. The recent adoption of \textit{śāstraic} constructs into modern sciences (Malhotra) suggests the opposite.

\textit{Śāstra-s as stifling individual agency}

\textit{Śāstra}-s fundamentally stress practise by the individual and the community at large. Knowledge gathering and validation are inherent in the individual's rights rather than dictated via history centrism or any central organization.


\section*{Implications and the Way Ahead (Swadeshi-Indology)}

Can Swadeshi Indology (as a modern societal inheritor of the knowledge of the \textit{śāstra}-s) contextualize \textit{śāstra}-s for modern harmonious living globally? The destructive nature of almost all lifestyle models based on modern science and technology (everything finally adds up to destruction of the biosphere) is well known - though not universally acknowledged. Should “future” modernity be redefined - and be based on principles of living based on the \textit{śāstra}-s. The global reach of Yoga is an example of a step in this direction - a harmonious world begins with a harmonious body (mind). What then can Swadeshi Indology do to articulate, interpret and disseminate the knowledge of the Vedic seers/ancients in a modern context? It would require enormous efforts; serious teamwork across multiple groups of people like Sanskrit experts, \textit{saṁskṛti} experts, \textit{śāstra} experts, practicing masters, experimenters, educators, gurus, scientists, doctors, and psychologists, just to name a few. The Swadeshi Indology genre of research and scholarship is uniquely positioned to undertake this multi-cultural cross-disciplinary synthesis across space (geographies) and time.

Is this possibly the way ahead for \textbf{\textit{“sanatan-isation”}} - the global spread and practice of \textit{dharma}?


\section*{Concluding Remarks}

As a rebuttal to the flawed and deeply derisive frontal attack on the nature of \textit{śāstra} and its practice by Pollock, we undertook a critical examination (sans the rhetoric of the humanities) of the “actual” nature of \textit{śāstra} and science. We examined in detail the realities of Western science, its evolution, its foundational untruths, its bogus histories, flawed metaphysics, its limited nature and scope and also its (as manifested) fundamentally destructive \textit{prayojana}. Given the deeply unsound nature of the pramāṇa (\textit{anumāna}), the \textit{prameya} (superficial limited knowledge of material structure) and the \textit{prayojana} (annihilation) of Science (see Ceballos et al 2017) we posit that Western science is (if at all) a very limited \textit{śāstra} - extremely limited.

Though this paper refutes some of the key theses and assumptions of Pollock’s 1985 paper, three decades later, it is important to situate \textit{śāstra} in the right context with respect to Western academic /scientific hegemony. We believe a proper understanding of these concepts is essential for continuing work on Swadeshi Indology by enlightening existing audiences - both colonized and uncolonized - on the actual boundaries and realities.

The bigger issue at hand is: How then can we, as inheritors of \textit{dharma}, lead the way for a global renaissance and save our planet from destruction? We sincerely hope these exhortations are considered seriously and that this discourse is advanced to the next level of formulation.

\newpage


\section*{References}

\begin{thebibliography}{99}
\bibitem{chap06-key01} Alvares, Claude. (1991). \textit{Decolonising History: Technology and Culture in India, China and the West, 1492 to the Present Day}. Goa, India: The Other India Press.

 \bibitem{chap06-key02} Balasubramaniam, R. (2008). \textit{Marvels of Indian Iron Through the Ages}. New Delhi: Rupa \& Co in association with Infinity Foundation.

 \bibitem{chap06-key03} Broad, W. J., and Wade, N. (1983). \textit{Betrayers of the Truth}. New York: Simon and Schuster.

 \bibitem{chap06-key04} Cartwright, Nancy. (1983). \textit{How the Laws of Physics Lie}. Oxford: Clarendon Press.

 \bibitem{chap06-key05} ---. (1999). \textit{The Dappled World: A Study of the Boundaries of Science}. Cambridge, UK: Cambridge University Press.

 \bibitem{chap06-key06} Ceballos, Gerardo., Ehrlich, Paul R., Barnosky, Anthony D., García, Andrés., Pringle, Robert M., and Palmer, Todd M. (2016). “Accelerated Modern Human–induced Species Losses: Entering the Sixth Mass Extinction.” \textit{Science Magazine}, Science Advances 19 Jun 2015: e1400253. \url{http://advances.sciencemag.org/content/1/5/e1400253}. Accessed on November 10, 2016.

 \bibitem{chap06-key07} Ceballos, Gerardo., Ehrlich, Paul R., and Dirzo, Rodolfo. (2017). “Biological annihilation via the ongoing sixth mass extinction signaled by vertebrate population losses and declines”. \textit{Proceedings of the National Academy of Sciences}. July 25, 2017 114 (30) E6089-E6096; \url{https://doi.org/10.1073/pnas.1704949114}

 \bibitem{chap06-key08} Chattopadhyay, P. K., and Sengupta, G. (2011). \textit{History of Metals in Eastern India and Bangladesh}. New Delhi: Pentagon Press in association with Infinity Foundation.

 \bibitem{chap06-key09} Dharampal. (1983). \textit{The Beautiful Tree: Indigenous Indian Education in the Eighteenth Century}. New Delhi: Biblia Impex.

 \bibitem{chap06-key10} ---. (2000). \textit{Dharampal - Collected Writings – Volume 1: Indian Science and Technology in the 18 Century}. Goa, India: Other India Press.

 \bibitem{chap06-key11} Hansberry, William Leo. (1955) “Book review: \textit{Stolen Legacy”. The Journal of Negro Education} 24(2), Spring 1955.

 \bibitem{chap06-key12} Hartmann, S., Hoefer, C., and Bovens, L. (2008). \textit{Nancy Cartwright's Philosophy of Science}. New York: Routledge.

 \bibitem{chap06-key13} Hoefer, Carl. (2008). “Introducing Nancy Cartwright’s Philosophy of Science”. In Hartmann \textit{et al} (2008). pp 1-13.

 \bibitem{chap06-key14} James, George G. M (2009). \textit{Stolen Legacy: Greek Philosophy is Stolen Egyptian Philosophy}. USA: The Journal of Pan African Studies eBook.

 \bibitem{chap06-key15} Joshi, J. P. (2008). \textit{Harappan Architecture and Civil Engineering}. New Delhi: Rupa \& Co in association with Infinity Foundation.

 \bibitem{chap06-key16} ---. (2009). \textit{Harappan Technology and Its Legacy}. New Delhi: Rupa \& Co in association with Infinity Foundation.

 \bibitem{chap06-key17} Kannan, K.S. (Ed.) (2018). \textit{Śāstra-s Through the Lens of Western Indology: A Response} (Proceedings of Swadeshi Indology Conference Series, Volume 2).

 \bibitem{chap06-key18} Karakostas, V., and Dieks, D. G. (2013). \textit{EPSA11 Perspectives and Foundational Problems in Philosophy of Science}. Wien: Springer.

 \bibitem{chap06-key19} Kuhn, T. S. (1957). \textit{The Copernican Revolution: Planetary Astronomy in the Development of Western Thought}. Cambridge: Harvard University Press.

 \bibitem{chap06-key20} Maddison, Angus. (2003). \textit{The World Economy: A Millennial Perspective}. France: Organisation for Economic Cooperation and Development. \url{http://www.oecd.org/dev/developmentcentrestudiestheworldeconomyamillennialperspective.htm}

 \bibitem{chap06-key21} Malhotra, Rajiv. (2011). \textit{Being Different: An Indian Challenge to Western Universalism}. New Delhi: HarperCollins.

 \bibitem{chap06-key22} ---. (2016). \textit{The Battle for Sanskrit: Is Sanskrit Political or Sacred, Oppressive or Liberating, Dead or Alive?} New Delhi: HarperCollins.

 \bibitem{chap06-key23} Nasr, S. H. (1968). \textit{Science and Civilization in Islam}. Cambridge, MA: Harvard University Press.

 \bibitem{chap06-key24} Pollock, Sheldon. (1985). “The Theory of Practice and the Practice of Theory in Indian Intellectual History.” \textit{Journal of the American Oriental Society, 105(3)}, 499. doi:10.2307/601525

 \bibitem{chap06-key25} Preston, John, "Paul Feyerabend", \textit{The Stanford Encyclopedia of Philosophy} (Winter 2016 Edition), Edward N. Zalta (ed.) \url{https://plato.stanford.edu/archives/win2016/entries/feyerabend/}
 
 \bibitem{chap06-key26} Raju, C. K. (2004). “Math Wars and the Epistemic Divide in Mathematics”, Final version in \textit{Cultural Foundations of Mathematics}, Ch. 9. \url{http://www.hbcse.tifr.res.in/episteme/episteme-1/themes/ckraju_finalpaper}

 \bibitem{chap06-key27} ---. (2007). \textit{Cultural Foundations of Mathematics: The Nature of Mathematical Proof and the Transmission of the Calculus from India to Europe in the 16th c. CE}. Delhi: Pearson Longman.

 \bibitem{chap06-key28} ---. (2008). “Logic”. Article for \textit{Encyclopedia of Non-Western Science, Technology and Medicine}. Netherlands: Springer. pp 1230-1235. \url{http://ckraju.net/papers/Nonwestern-logic.pdf}

 \bibitem{chap06-key29} ---. (2009). \textit{Is Science Western in Origin?} Delhi: Daanish Books.

 \bibitem{chap06-key30} ---. (2012). \textit{Euclid and Jesus}. Penang: Multiversity and Citizen International.

 \bibitem{chap06-key31} ---. (?). “History and Philosophy of Science as a Means for Decolonisation”. \url{http://ckraju.net/papers/History-philosophy-of-science-as-a-means-of-decolonisation.pdf. Accessed November 10, 2016}

 \bibitem{chap06-key32} Selin, Helaine (Ed.) (2008). \textit{Encyclopaedia of the History of Science, Technology, and Medicine in non-Western cultures}. Berlin: Springer.

 \bibitem{chap06-key33} Sheldrake, R. (2012). \textit{The Science Delusion: Freeing the Spirit of Enquiry}. London: Coronet.

 \bibitem{chap06-key34} Simkin, C. G. (1993). \textit{Popper's Views on Natural and Social Science}. Leiden: E.J. Brill.

 \bibitem{chap06-key35} Sudarshan, T.N. (2018). “The Science and NeScience of \textit{Śāstra}”. In Kannan (2018). pp. 149-186.

 \bibitem{chap06-key36} Tripathi, V., and Chakrabarti, D. K. (2008). \textit{History of Iron Technology in India: From Beginning to Pre-modern times}. New Delhi: Rupa \& Co. in association with Infinity Foundation.

 \bibitem{chap06-key37} Vidyabhusana, S. C. (1970). \textit{A History of Indian Logic}. Delhi: Motilal Banarsidass.

 \end{thebibliography}

\theendnotes

