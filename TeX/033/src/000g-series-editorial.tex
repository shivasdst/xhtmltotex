
\chapter*{Series Editorial}\label{serieseditorial}

It is a tragedy that many among even the conscientious Hindu scholars of Sanskrit and Hinduism still harp on Macaulay, and ignore others while accounting for the ills of the current Indian education system, and the consequent erosion of Hindu values in the Indian psyche. Of course, the machinating Macaulay brazenly declared that a single shelf of a good European library was worth the whole native literature of India, and sought accordingly to create “a class of persons, Indian in blood and colour, but English in taste, in opinions, in morals and in intellect” by means of his education system – which the system did achieve. 

An important example of what is being ignored by most Indian scholars is the current American Orientalism. They have failed to counter it on any significant scale. 

It was Edward Said (1935–2003) an American professor at Columbia University who called the bluff of “the European interest in studying Eastern culture and civilization” (in his book \textit{Orientalism} (1978)) by showing it to be an inherently political interest; he laid bare the subtile, hence virulent, Eurocentric prejudice aimed at twin ends – one, justifying the European colonial aspirations and two, insidiously endeavouring to distort and delude the intellectual objectivity of even those who could be deemed to be culturally considerate towards other civilisations. Much earlier, Dr. Ananda Coomaraswamy (1877–1947) had shown the resounding hollowness of the \textit{leitmotif} of the “White Man’s Burden.” 

But it was given to Rajiv Malhotra, a leading public intellectual in America, to expose the Western conspiracy on an unprecedented scale, unearthing the \textit{modus operandi} behind the unrelenting and unhindered program for nearly two centuries now of the sabotage of our ancient civilisation yet with hardly any note of compunction. One has only to look into Malhotra’s seminal writings – \textit{Breaking India} (2011), \textit{Being Different} (2011), \textit{Indra’s Net} (2014), \textit{The Battle for Sanskrit} (2016), and \textit{The Academic Hinduphobia} (2016) – for fuller details.

This pentad – preceded by \textit{Invading the Sacred} (2007) behind which, too, he was the main driving force – goes to show the intellectual penetration of the West, into even the remotest corners (spatial/temporal/ thematic) of our hoary heritage. There is a mixed motive in the latest Occidental enterprise, ostensibly being carried out with pure academic concerns. For the American Orientalist doing his “South Asian Studies” (his new term for “Indology Studies”), Sanskrit is inherently oppressive – especially of Dalits, Muslims and women; and as an antidote, therefore, the goal of Sanskrit studies henceforth should be, according to him, to “exhume and exorcise the barbarism” of social hierarchies and oppression of women happening ever since the inception of Sanskrit – which language itself came, rather, from outside India. Another important agenda is to infuse/intensify animosities between/among votaries of Sanskrit and votaries of vernacular languages in india. A significant instrument towards this end is to influence mainstream media so that the populace is constantly fed ideas inimical to the Hindu heritage. The tools being deployed for this are the trained army of “intellectuals” – of leftist leanings and “secular” credentials.

Infinity Foundation (IF), the brainchild of Rajiv Malhotra, started 25 years ago in the US, spearheaded the movement of unmasking the “catholicity” (– and what a euphemistic word it is!) of Western academia. The profound insights provided by the ideas of “Digestion” and the “U–Turn Theory” propounded by him remain unparalleled.

It goes without saying that it is \textit{ultimately the Hindus in India who ought to be the real caretakers of their own heritage}; and with this end in view, \textbf{Infinity Foundation India (IFI)} was started in India in 2016. IFI has been holding a series of Swadeshi Indology Conferences. 

Held twice a year on an average, these conferences focus on select themes and even select Indologists of the West (sometimes of even the East), and seek to offer refutations of mischievous and misleading misreportages/misinterpretations bounteously brought out by these Indologists – by way of either raising red flags at, or giving intellectual responses to, malfeasances inspired in fine by them. To employ Sanskrit terminology, the typical secessionist misrepresentations presented by the West are treated here as \textit{pūrva–pakṣa}, and our own responses/rebuttals/rectifications as \textit{uttara–pakṣa} or \textit{siddhānta}. 

The first two conferences focussed on the writings of Prof. Sheldon Pollock, the outstanding American Orientalist (also of Columbia University, ironically) and considered the most formidable and influential scholar of today. There can always be deeper/stronger responses than the ones that have been presented in these two conferences, or more insightful perspectives; future conferences, therefore, could also be open in general to papers on themes of prior conferences.

 Vijayadaśamī\\
 Hemalamba Saṁvatsara\\
 Date 30–09–2017\\

\begin{flushright}
 \textbf{Dr. K S Kannan}\\
 Academic Director\\
 and\\
 General Editor of the Series
\end{flushright}

