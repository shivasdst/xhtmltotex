
\chapter{The Science and Nescience\index{nescience} of Mīmāṁsā}

\Authorline{T. N. Sudarshan}


\section*{Abstract}

The interpretations in this paper of Mīmāṁsā by Western (\textit{etic}) scholars are critically analyzed with a focus on the fundamental issues in Western hermeneutics\index{hermeneutics@Western}, and the applications to alien (alien, that is, to Western methods) bodies of knowledge like Indian texts. Mīmāṁsā\index{purva mimamsa@Pūrva Mīmāṁsā} has been analyzed and critiqued by Western philosophers and scholars using flawed understandings and techniques rooted in various Western epistemologies. The modern (neo)–Orientalists proudly continue this tradition using their tools of preference viz. political philology and historiography based socio–political analyses – relying on biased and flawed re–creations of historical events and their ascribed motivations. Prof. Sheldon Pollock's\index{Pollock, Sheldon} thesis on Mīmāṁsā is critically appraised with a firm basis in the traditional perspectives and vocabularies of the \textit{vidyā}–s relating to Mīmāṁsā and its major interpretations. It is proposed that the neo–Orientalist theses on Mīmāṁsā derive from a deep ignorance (\textbf{nescience}\index{nescience}). The primacy, non–dilutability and non–negotiable nature of a sacred perspective (\textit{saṁskāra}\index{samskara@\textit{saṁskāra}}) whilst interpreting Sanskrit texts of Vedic knowledge systems is reinforced. The limitations of the scientific method in interpreting Mīmāṁsā\index{purva mimamsa@Pūrva Mīmāṁsā} are also discussed. The kind of hermeneutics that is practiced on Sanskrit texts is discussed – we posit a new type – a \textbf{\textit{hermeneutics of derision}}\index{hermeneutics of derision}. This is followed by a discussion on the Western notions of history and the accusations of ahistoricism\index{ahistoricism} ascribed to Indian civilization. The non–empirical, non–verifiable and unscientific nature of methods used by Pollock to make his erroneous claims is highlighted. The aim, purpose and science of Mīmāṁsā\index{purva mimamsa@Pūrva Mīmāṁsā} is to lead the practitioner to examine and critically analyze his actions (\textit{karman\general{\index{karman@\textit{karman}}}/dharma\general{\index{dharma@\textit{dharma}}}}) in life while on the path to a holistically (nature included) harmonious existence. The scope and role of Mīmāṁsā is beyond that of Science, Social Science or Religion (as the West currently knows/interprets these terms). Unless this is acknowledged and more importantly reinforced and realized by its practice – Western \textit{etic} scholarship\index{etic scholarship@\textit{etic scholarship}} will continue to provide nebulous and incorrect interpretations of Indic knowledge systems driven by nescience\index{nescience}.


\section*{Introduction}

The Indian “hermeneutic” tradition has been interpreted in various ways since the beginnings of Indology – when Sanskrit and its associated knowledge systems became objects of analysis and scrutiny using Western techniques of study. Sheldon Pollock\index{Pollock, Sheldon} has discharged serious accusations on the Mīmāṁsā system of thought and its methods in multiple contexts in Pollock (2004), Pollock (1989) and Arnason \textit{et al}. (2005) to cite a few. Pollock’s interpretations of Mīmāṁsā’s origins, motives and goals exemplify the characteristic trademarks of his style: political philology, creative use of chronology, dubious dating of texts, formulation of spurious historiographical narratives and speculative theorizations. Prior to and contemporaneous with Pollock, various other Indologists have also fantastically theorized about Mīmāṁsā (Bronkhorst\index{Bronkhorst, J.} 2012), Asko Parpola\index{Parpola, Asko} (1994), Max Mueller\index{Max Muller, Friedrich} – one could go all the way back up to Sir William Jones\index{Jones, Sir William}.

The notion of interpretation (according to Western traditions) is briefly examined in this paper and then it is sought to be shown how such an understanding influences the \textit{etic}\index{etic scholarship@\textit{etic scholarship}} approach to interpret Mīmāṁsā. This section aims to highlight the fundamental limitations of the Western hermeneutic approach whilst attempting to understand Indian knowledge systems and the Mīmāṁsā\index{purva mimamsa@Pūrva Mīmāṁsā} tradition in particular. This is followed by a section discussing specifically. Pollock’s approach to interpreting Mīmāṁsā. The next section discusses the critical issue of historicity – the Western approach and definition of history and the marked lack of such efforts in the Indian traditions. The oft–repeated accusation of the Western scholars – including those of Pollock\index{Pollock, Sheldon} on the ahistoricity of Indian tradition in a Western sense is examined anew, followed by the possible reasons for this apparent “weakness” of Indian tradition. The section on the science of Mīmāṁsā gives the traditional perspective of what it is and how it is closely linked to other systems of Indian thought – or rather – how it underpins most of them. In the section on Pollock’s hermeneutics (of \textbf{derision})\index{hermeneutics of derision} Mīmāṁsā\index{mimamsa@Mīmāṁsā} is juxtaposed with the traditional perspective on Mīmāṁsā highlighting the nescience\index{nescience} (ignorance) of the Western approaches in understanding Mīmāṁsā. The concluding section discusses Mīmāṁsā and its critical role in defining global \textit{dharma}\index{dharma@\textit{dharma}} and its use as a basis for universal peace and harmony.


\section*{On the Notion of Interpretation}

The act of interpretation (both voluntary and involuntary) is an essential condition of the human state both for human communication and understanding — both at an individual and (the more so at) the group level. Interpretation at an individual level is conditioned by the social and shared meanings as also those interpretations collectively curated via culture, civilizational behaviors and social systems (law and ethics). The Western history of interpretation (which, like most Western histories, “originates” from the Greeks) is discussed in brief. The notion of interpretation with regard only to text, and not to speech or other symbol mechanisms, is specifically examined.


\section*{Western Notions of Interpretation\hfil \break (Hermeneutics)}

The Stanford Encyclopedia of Philosophy (SEP) entry for Herme\-neutics, informs us that ancient hermeneutics\index{hermeneutics@Western} began (as usual) with the Greeks (the Homeric epics).

\begin{myquote}
“The most remarkable characteristic of ancient exegesis\index{exegesis} was \textit{allegorisis}\index{allegorisis@\textit{allegorisis}} (\textit{allegoría}, from \textit{alla agoreuein}, i.e., saying something different). This was a method of nonliteral interpretation of the authoritative texts which contained claims and statements that seemed theologically and morally inappropriate or false... Such exegetical attempts were aiming at a deeper sense, hidden under the surface—\textit{hypónoia} i.e., underlying meaning. Allegorisis\index{allegorisis@\textit{allegorisis}} was practiced widely from the sixth century BCE to the Stoic and Neoplatonistic schools and even later... In the Middle Ages the most remarkable characteristic of the interpretative praxis was the so–called \textit{accessus ad auctores}; this was a standardized introduction that preceded the editions and commentaries of (classical) authors. There were many versions of the \textit{accessus}, but one of the more widely used was the following typology of seven questions...
\end{myquote}

\begin{itemize}
\itemsep=0pt
\item Who (is the author) (\textit{quis/persona})?

 \item What (is the subject matter of the text) (\textit{quid/materia})?

 \item Why (was the text written) (\textit{cur/causa})?

 \item How (was the text composed) (\textit{quomodo/modus})?

 \item When (was the text written or published) (\textit{quando/tempus})?

 \item Where (was the text written or published) (\textit{ubi/loco})?

 \item By which means (was the text written or published) (\textit{quibus faculatibus/facultas})?”

~\hfill (Mantzavinos\general{\index{Mantzavinos, C.}} 2016)

\end{itemize}

From the Greek obsession with \textit{saying something different,} to the middle–ages when one encounters the importance given to creational context – what is revealed is the Western obsession with \textbf{treating \textit{text} as something that manipulates and that (as a consequence) which needs to be manipulated in order to understand it}. The reaction to the influence of the Abrahamic theologies (the obsession with a unique final text and final interpretation) is very apparent in the evolution of hermeneutics\index{hermeneutics@Western} as a Western academic tool/discipline. The close linkages with biblical philology – a theological text–analysis tool of power wielded by the church – which was used to interpret the word–of–god, though not acknowledged as such explicitly, are also discernible.

The modern evolution of the act of interpretation is described (in Skinner\index{Skinner, Q.} 1972) in no uncertain terms

\begin{myquote}
“If we grant that the main aim of the interpreter must be to establish the meaning of a text, and if we grant that the meaning may to some extent lie "beyond" or "below" its surface, can we hope to frame any general rules about how this meaning may be recovered? Or are we eventually compelled to adopt what Hirsch here calls the "resigned opinion" that "our various schools and approaches" are \textbf{no more than dogmatic theologies, generating a corresponding "multitude of warring sects}.”

~\hfill (Skinner\general{\index{Skinner, Q.}} 1972:394) (emphasis ours)
\end{myquote}

\smallskip

Peculiar to the entire evolution and discussion surrounding hermeneutics is the theme of   “The Hermeneutic Circle”\index{The Hermeneutic Circle}

\smallskip

\begin{myquote}
“The hermeneutic circle is a prominent and recurring theme in the discussion ever since the philologist Friedrich Ast\index{Ast, Friedrich} (1808: 178), who was probably the first to do so, drew attention to the circularity of interpretation: “The foundational law of all understanding and knowledge”, he claimed, is “to find the spirit of the whole through the individual, and through the whole to grasp the individual.”

~\hfill (Mantzavinos\general{\index{Mantzavinos, C.}} 2016)
\end{myquote}

\smallskip

Posed either as an ontological issue or as a logical or methodological problem, the deep discussions on the \textbf{hermeneutic circle}\index{The Hermeneutic Circle} and the vast related literature examining various perspectives, only reinforces (from my perspective) the flimsiness of the hermeneutic discourse. The process of hypothesizing meaning in an incremental piecemeal fashion without awareness or consciousness or a preliminary need to understand the whole (big picture) is the key problem underlying the hermeneutics approach.

It is well acknowledged within the discipline that the empirical approaches taken by hermeneutics\index{hermeneutics} are fallible. The application of these techniques in specific narrow text areas like theology and jurisprudence evolve close to the domain of discourse, and these closed domains have their related hypotheses and methodologies.

Text interpretation is the key praxis, the purposes and motivations behind this is what is interesting and leads to spurious applications like those of the Western Indologists. 

\medskip

\begin{myquote}
“The process of text interpretation which lies in the center of hermeneutics as the methodological discipline dealing with interpretation can and has been analyzed empirically with the help of testable models. The question whether there are certain normative presuppositions of the interpretative praxis—like specific principles of interpretation that are constitutive of this praxis and indispensable rationality principles—is a focal issue of obvious philosophical importance (Detel 2014). Regardless of the position that is assumed with respect to this issue, it is hardly possible to deny that the interpretative praxis can take on multiple forms and can take place according to diverse aims.”

~\hfill (Mantzavinos\general{\index{Mantzavinos, C.}} 2016)
\end{myquote}

Ascribing motives to authors of text – socio–political ones at that — is a favorite methodological pastime of Western Indologist, especially those practicing the peculiar brand of neo–Orientalism exemplified by Pollock\index{Pollock, Sheldon}. Deeply suspicious motives are ascribed to Pāṇini\index{Panini@Pāṇini}, Patañjali\index{Patanjali@Patañjali}, Vālmīki\index{Valmiki@Vālmīki}, Vyāsa\index{Vyasa@Vyāsa} and Jaimini\index{Jaimini} just to cite a few examples. What are the reasons and where does this strange and peculiar obsession come from, and more importantly, how does all of this pass for scholarship? The answers very possibly lie in the tools of Western academia themselves. Mantzavinos lays it out bare.

\begin{myquote}
“Whereas the \textbf{notion of intention} is certainly useful in providing a methodological account of interpretation, its use is surely part of a later development; and it has been largely imported into hermeneutic methodology from discussions in philosophy of mind and language that took place in the analytic tradition.
\end{myquote}

\begin{myquote}
...
\end{myquote}

\begin{myquote}
A \textbf{nexus of meaning}, connected with a specific linguistic expression or a specific text, is construed by the author against the background of his goals, beliefs, and other mental states while interacting with his natural and social environment: such a construal of meaning is a complex process and involves both the conscious and unconscious use of symbols. Text interpretation can be conceptualized as the activity directed at correctly identifying the meaning of a text \textit{by virtue of accurately reconstructing the nexus of meaning that has arisen in connection with that text.” }

~\hfill (Mantzavinos 2016)
\end{myquote}

Unlike the notion of interpretation in  Indian knowledge systems – wherein the \textit{śāstra paddhati} of interpretation is based on the sciences of grammar and various other related \textit{śāstra–}s – \textbf{the free–for–all that ensues when one applies Western tools is readily apparent in the voluminous (250 years’ worth) bodies of spurious interpretive (nexus of meanings) genre of academic scholarship produced by Western Indology.}

Ricoeur\index{Ricoeur, P.}, the influential 20th century French philosopher, coined a phrase called the \textit{hermeneutics of suspicion.}\index{hermeneutics of suspicion} He distinguishes two forms of hermeneutics, one of faith which aims to restore meaning to a text, and one of suspicion, which attempts to decode meanings which are disguised.

The Western Indologists (since the 1750s) for the most part seem to be not only indulging in the hermeneutics of suspicion but also what one could only characterize as a \textbf{\textit{hermeneutics of derision}}\index{hermeneutics of derision} – this can be seen in the early phase of Indology where it was a tool of colonial policy and expansion. In recent times it is especially apparent in the case of extreme theses (ex: the \textit{Deep Orientalism} thesis by Sheldon Pollock\index{Pollock, Sheldon} seen in (Breckenridge\index{Breckenridge, C. A.} 1993)) originating in the Neo–Orientalists typified by those of Sheldon Pollock.


\section*{Western Understanding of Mīmāṁsā}

The Western interpretations of Mīmāṁsā\index{mimamsa@Mīmāṁsā} began with the efforts of Sir William Jones\index{Jones, Sir William} to interpret the \textit{dharmaśāstra–}s\index{Dharmasastra@Dharmaśāstra}. The \textit{dharmaśāstra–}s could not in any way be interpreted without the aid of the \textit{Mīmāṁsā–sūtra}–s. This was the standard procedure. The attention to syllabic detail and injunction supposedly drove William Jones to translate the \textit{Mānava–dharma–śāstra}\index{Manavadharmasastra@\textit{Mānava Dharmaśāstra}} (Murray 1998).

The analytical approach followed by subsequent Western Indologists was to remove \textit{Mīmāṁsā–sūtra}–s\index{Mimamsasutra@\textit{Mīmāṁsā sūtra}} from the context of practice and the larger play of the continual exegesis as is the wont of the Indian tradition. Attempts were made to freeze text and place the content in independent context. The requirement of the Western hermeneutic approach to discover a “fixed” subjective motive to text produced various hypotheses on the notion of Mīmāṁsā and its role in the Indian civilizational praxis. Generally speaking, Mīmāṁsā has been variously characterized as non–godly, ungodly, atheistic, oppressive, ritualistic, segregative, socially divisive (Pollock\index{Pollock, Sheldon} 1989), racial, ahistoric – a few choice descriptors used over the centuries by Western Indologists. There was no attempt (howsoever sincere) to understand the core principles (which have no Western counterpart) of \textit{karman}\index{karman@\textit{karman}} and \textit{dharma}\index{dharma@\textit{dharma}} – the motivations behind the Mīmāṁsā exegesis of text. The underlying framework of Indian epistemology and its reality in the lived lives of the \textit{sanatanic} practitioner is for all practical purposes completely and utterly disregarded. This genre of hubris is routine in Western socio–anthropological approaches to “othering” and is considered normal Western scholarship.

Concepts (alien to Western civilisation) of \textit{puṇya}\index{punya@\textit{puṇya }}, \textit{pāpa}\index{papa@\textit{pāpa}}, \textit{apūrva}\index{apurva@\textit{apūrva}}, \textit{punarjanman}\index{punarjanma@\textit{punarjanma}}, \textit{ātman}\index{atman@\textit{ātman}}, \textit{phala}\index{phala@\textit{phala}} and many others which govern the \textit{karma–siddhānta} (again totally alien to Western civilization) which influence the \textit{dharma–jijñāsā}\index{@\textit{dharmajijñāsā}} (the primary \textit{prameya}\index{prameya@\textit{prameya}} and \textit{prayojana}\index{prayojana@\textit{prayojana}} of Mīmāṁsā\index{mimamsa@Mīmāṁsā} ) are blatantly ignored and are not considered  in the analytic framework of the Western approaches. The free–for–all, “anything goes” (large degrees of interpretive freedom) nature of analysis allowed by the Western constructs of hermeneutics\index{hermeneutics@Western} and philology delivered from institutions of power and prestige have taken center stage in the recent (two centuries) interpretations of Mīmāṁsā. From a traditional perspective, such an approach could be characterized as a \textit{manodharma–jijñāsā}\index{manodharmajijnasa@\textit{manodharmajijñāsā}} (pursuit of the fanciful and imaginative) at best or possibly \textit{adharma–jijñāsā}\index{adharmajijnasa@\textit{adharmajijñāsā}} (wanton pursuit of falsehoods and the unethical) at worst.


\section*{Pollock’s Interpretation (Hermeneutics of \textit{Derision}\index{hermeneutics of derision}) of Mīmāṁsā}

Sheldon Pollock\index{Pollock, Sheldon} takes aim at Mīmāṁsā as a part of multiple theses that derive from his well disguised methods of political philology. Pollock (2004), Pollock (1989), Arnason\index{Arnason, J. P.} (2005) \textit{et al} are his primary expositions on Mīmāṁsā. Malhotra\index{Malhotra, Rajiv} alludes to Pollock’s obsessions with manipulating dates to suit his formulation of thesis.

\begin{myquote}
“Likewise, the date of Pūrva–Mīmāṁsā\index{purva mimamsa@Pūrva Mīmāṁsā} scholars such as Jaimini is moved to a period centuries after the Buddha whereas tradition puts it prior to 800 BCE."

~\hfill (Malhotra 2016:453)
\end{myquote}

The rather ludicrous dating not with standing, Pollock goes on to theorize the “oppositions” between Buddhism and Hinduism – Mīmāṁsā being the wedge to differentiate and forming the basis for the fanciful hypothesis.

\begin{myquote}
“What was at stake for the Mīmāṁsaka\index{Mimamsaka@Mīmāṁsaka} in asserting the uncreated, eternal nature of language was the possibility that vāṅmaya, or a thing–made–of–language – that is, a text, like the Veda – could be eternal too, something the Buddhists sought fundamentally to reject.”

~\hfill (Malhotra 2016:385)
\end{myquote}

Pollock takes aim at the Veda and Mīmāṁsā\index{mimamsa@Mīmāṁsā} ascribing to them the fundamental ills (as he sees it) of Indian civilization. He builds an elaborate thesis on the existence of asymmetrical relations of power. A response to these claims is given in the next section. We highlight pertinent sections as Pollock\index{Pollock, Sheldon} proceeds to build his arguments based on his own (imagination) \textit{manodharma} mechanisms.

A sense of history (in a Western sense) is, according to him, lacking in Indian society, and this is primarily because of the claims of timelessness of the Veda–s. Mīmāṁsā\index{mimamsa@Mīmāṁsā} supposedly represents this ignorance of the past.

\begin{myquote}
“The primary cause for the marked lack of a sense of history and the resulting ignorance of the past is Mīmāṁsā as Mīmāṁsā depends on the timelessness of Vedas for its authority.”

~\hfill (Pollock 1989:603)
\end{myquote}

Malhotra\index{Malhotra, Rajiv} throws light on how Pollock uses these claims to further his “Buddhism\index{Buddhism} vs Hinduism\index{Hinduism}” political thesis

\begin{myquote}
“He believes that the Mīmāṁsaka\index{Mimamsaka@Mīmāṁsaka} thinkers considered the eternal nature of the Veda to be dependent on the eternal, uncreated nature of Sanskrit. Hence, the Buddhist rejection of the uncreated nature of Sanskrit led to their rejection of the Vedas. He says Buddhists invented Pali as their language for writing and alleges that there was a similar rejection of Sanskrit by the Jains, who adopted Ardha–magadhi as their language. He says that Vedic thinkers criticized these new languages because they undermined the doctrinal authority of Sanskrit.”\hfill (Malhotra 2016:385)
\end{myquote}

Continuing in the same vein, Pollock theorizes that all \textit{śāstra} is influenced by this Mīmāṁsā\index{mimamsa@Mīmāṁsā} notion of timelessness and is the root–cause of the deliberate (systematic and by design) denial of the past.

\begin{myquote}
“Mīmāṁsā makes the authority of the Veda dependent on its timelessness, and thus must empty the Veda of its historical referentiality. Since learned discourse (śāstra) in general is subject to a process of “vedicization," it adopts the Veda's putative ahistoricality; and the same set of concerns comes to inform the understanding of the genre itihāsa (“history") and the interpretation of itihāsa\index{itihasa@Itihāsa} texts. History, consequently,\break seems not so much to be unknown in Sanskritic India as to be denied.”

~\hfill (Pollock 1989:603)
\end{myquote}

Taking recourse to the Hermeneutics of Suspicion\index{hermeneutics of suspicion} – Pollock makes sweeping claims regarding the Veda–s, the practice of Vedic life, and the lived civilization of India.

\begin{myquote}
“.. when the Vedas were emptied of their “referential intention," other sorts of Brahmanical intellectual practices seeking to legitimate their truth–claims had perforce to conform to this special model of what counts as knowledge, and so to suppress the evidence of their own historical existence – a suppression that took place in the case of itihāsa, “history," itself.”

~\hfill (Pollock\general{\index{Pollock, Sheldon}} 1989:609)
\end{myquote}

Extending the hermeneutics approach to one of \textbf{derision }\index{hermeneutics of derision}– Pollock ups the ante accusing India, rather Sanskritic India as fundamentally a nation of “deniers of the past”.

\begin{myquote}
“History, one might thus conclude, is not simply absent from or unknown to Sanskritic India; rather it is denied in favor of a model of "truth" that accorded history no epistemological value or social significance.”

~\hfill (Pollock\general{\index{Pollock, Sheldon}} 1989:610)
\end{myquote}

The theorization (hermeneutics of derision)\index{hermeneutics of derision} reaches the expected socio–political climax – the denial of history provided by the Mīmāṁsā\index{mimamsa@Mīmāṁsā} helps in serving the cause of the brahmins – in usurping power and maintaining it aided by the Mīmāṁsā.

\smallskip

\begin{myquote}
“To answer these we would want to explore the complex ideological formation of traditional Indian society that privileges system over process – the structure of the social order over the creative role of man in history – and that, by denying the historical transformations of the past, deny them for the future and thus serve to naturalize the present and its asymmetrical relations of power.”

~\hfill (Pollock 1989:610)
\end{myquote}

\smallskip

In Pollock (2004), Pollock’s attempt at a supposedly scholarly summary of Appayya Dīkṣita’s\index{Appayya Diksita@Appayya Dīkṣita} \textit{Purvottara–mīmāṁsā–vāda–nakṣatramālā,}\index{Purvottaramimamsavadanaksatramala@\textit{Pūrvottara–mīmāṁsā–vāda–nakṣatramālā}} he makes remarkable claims based on his limited translation of Appayya Dīkṣita’s work.

\smallskip

\begin{myquote}
“The most remarkable attempt in Sanskrit intellectual history is the arresting of the process of subversion of meaning of \textit{dharma} by delimiting in the strictest possible terms what does and does not count as \textit{dharma}\index{dharma@\textit{dharma}} and to defend the proposition that the sole source of \textit{dharma} is the Veda.”

~\hfill (Pollock 2004:772)
\end{myquote}

\smallskip

\textit{Smṛti}\index{smrti@\textit{smṛti}} is accused as being a fabrication of the Mīmāṁsaka–s\index{Mimamsaka@Mīmāṁsaka}

\smallskip

\begin{myquote}
“The very idea of \textit{smṛti}, for instance, originated with Mīmāṁsā as a Vedic text no longer extant, no longer actually still being ‘heard’(\textit{śruti}\index{sruti@\textit{śruti}}) in its original wording during recitation, but existing only as a ‘memory’ (\textit{smṛti}) of the original, and in new wording – and migrated thence to the wider intellectual universe.”

~\hfill (Pollock 2004:773)
\end{myquote}

\medskip

The theme of fabrication is extended to \textit{dharma}\index{dharma@\textit{dharma}} and that of \textit{puruṣārtha}\index{purusartha@\textit{puruṣārtha}} itself

\medskip

\begin{myquote}
“Precisely the same thing could be demonstrated for other expressions and ideas, such as that core component of \textit{dharma\general{\index{dharma@\textit{dharma}}}, puruṣārtha}\index{purusartha@\textit{puruṣārtha}} itself.”

~\hfill (Pollock\general{\index{Pollock, Sheldon}} 2004:773)
\end{myquote}

\begin{myquote}
“This mantra from the \textit{Kathavalli}  [KU\index{Katha Upanisad@\textit{Kaṭha Upaniṣad}} 2.14] is concerned with three things, agent, end and means, that are different from the action constituting \textbf{the means of producing perishable and non–ultimate end–results}, the end–results themselves produced by those means, and the actor active with such means.”

~\hfill (Pollock 2004:792) (emphasis ours)
\end{myquote}

Pollock very glibly concludes that the pursuit of \textit{dharma}\index{dharma@\textit{dharma}} has nothing to do with the pursuit of \textit{brahman}\index{brahman@\textit{brahman}} – as, according to his understanding, since \textit{brahman} has been repudiated the means of attaining it also stands repudiated.

The rhetorical/theoretical mechanization of  “secularization”\index{secularization} and “de–sacralisation”\index{desacralisation} of the Indian Vedic systems is thus completed in Pollock’s thesis. The hermeneutics of \textbf{derision}\index{hermeneutics of derision} is seen in action. In conclusion, the Veda–s are not about \textit{brahman}. They are non–sacred. As the associated sacred practices have also been repudiated, there is nothing like a (sacred) notion of \textit{dharma}.

This is a foundational claim\endnote{Though based on Appayya Dīkṣita's work, there is no “global” perspective provided on the “actual” prevailing traditions of interpretation. Is such over–generalization warranted?} aimed to deconstruct (break and falsify) the primary edifice of the \textit{sanātanic} system. Based on the reactions (from practicing Sanatanists) to these interpretations, these theses do not read like those of a decorated academic scholar but are indicative of a deeply disturbed mind. \textit{Dharma\general{\index{dharma@\textit{dharma}}}, Brahman\general{\index{brahman@\textit{brahman}}} and Puruṣārtha}\index{purusartha@\textit{puruṣārtha}} – the basic constructs of the civilizational epistemology – are claimed to be \textbf{fabrications} of Mīmāṁsā\index{mimamsa@Mīmāṁsā}.


\section*{Discussion – Flaws in Method and\hfill \break Assumptions}

\vskip -6pt

For those unfamiliar with Pollock’s methods and scholarship, the theses on Mīmāṁsā might seem to be based on sufficiently credible academic basis, and possibly look to be argued out effectively by the author. The deeper agenda — of the (multi–decade) highly influential polemic powered by Pollock’s\index{Pollock, Sheldon} innovative usage of his three dimensional philology\index{Philology@three–dimensional}, where claims can be made on any basis, without being anywhere close to the truth – is guaranteed academic credibility as the academicians purportedly use an approved “method”. Pollock, for all practical purposes, is not a practicing Vedāntin or a Mīmāṁsaka\index{Mimamsaka@Mīmāṁsaka}.

Mīmāṁsā is about pursuit of \textit{dharma}\index{dharma@\textit{dharma}} – the last thing a Western (Indology) academic scholar will attempt to pursue. The funding of research in South–Asian studies departments (Price\index{Price, D. H.} 2016) are mostly if not completely governed by geo–political demands. Purely from a primitive perspective – the more outlandish and effective the other–ing of the region (South Asia), the more creditable purpose such research serves.

Pollock’s\index{Pollock, Sheldon} methods are \textbf{not} based in practice – which is the fundamental focus of Mīmāṁsā\index{mimamsa@Mīmāṁsā} and also of all of Indian \textit{darśana–}s. Mīmāṁsā is a theory of action, and to even experience the most rudimentary aspect of it, it should be based on an experiential basis. Would any Western Indologist (including Pollock) have performed any \textit{yajña}\index{yajna@\textit{yajña}}? Would he have been part of any \textit{yajña}? Would he have experienced any form of \textit{dharma–jijñāsā}\index{dharmajijnasa@\textit{dharmajijñāsā}}? Steeped as most Western Indologists are in a Judeo–Christian post–modern mental consciousness (like most Western academia) – such “personas”– are from a traditional perspective fundamentally ineligible to discuss and critique something like Mīmāṁsā. With regard to interpretations of Sanskrit text – whatever be the credentials in the Sanskrit language – if any Western Indologist (Pollock in this case) makes interpretations of a Sanskrit text – they should at least minimally indicate on what Sanskritic (\textit{sūtra}\index{sutra@\textit{sūtra}}) basis such theses are posited. As is known, the science of (grammar) Vyākaraṇa\index{Vyakarana@Vyākaraṇa} and the understanding of meaning are far more advanced in Sanskrit than in any other language.

\begin{myquote}
“Among various systems of Indian philosophy, Vyākaraṇa\index{Vyakarana@Vyākaraṇa}, Pūrva – Mīmāṁsā\index{purva mimamsa@Pūrva Mīmāṁsā} and Nyāya\index{Nyaya@Nyāya} are considered to be essential for the complete understanding of the concept of \textit{śabda}\index{sabda@\textit{śabda}} and its different forms. They are called Padaśāstra, Vākyaśāstra and Pramāṇaśāstra, respectively. A scholar who has got the knowledge of all these Śāstras is called \textit{pada–vākya–pramāṇajña}."

~\hfill (Subrahmanyam Korada\general{\index{Korada, Subrahmanyam}} 2008:vi)
\end{myquote}

If any Western Indologist (Pollock) cannot make arguments based on the framework of the tradition of interpretation, the \textit{\textbf{pada–vākya–pramāṇajña}}, such theses should ideally not be given any credibility by any practitioner of \textit{sanātana dharma.}\index{dharma@\textit{dharma}} As is known – the problem is not of one or two theses but of the overarching supporting framework built over decades of nurture – its deep roots and multidimensional attack on the fabric of Indian civilization via Western Indology’s influential discourse and grooming of the intellectual sepoy army. For a more detailed perspective of the underlying issues, see (Malhotra\index{Malhotra, Rajiv} 2016), Ch10 – “Is Sheldon Pollock\index{Pollock, Sheldon} Too Big to Be Criticized?”


\section*{On the Notion of History}

\vskip -5pt

The claims made by Pollock (in Pollock (2004)) are from a traditional perspective, bizarre and \textbf{do not} have any basis in the Sanskritic tradition. The comments and provocative theses on Indian methods (ahistoricity) in Pollock (1989) though do deserve an analysis. We provide a \textit{pūrvapakṣa}\index{purvapaksa@\textit{pūrvapakṣa}} on the Western (method) notion of history. The origins of history as a human pursuit are examined below. So too Western critiques of the idea of history are examined. Further, the Indian approach to history (\textit{itihāsa}\index{itihasa@Itihāsa}) both in an Indian sense and in a Western (misunderstood) sense are discussed and juxtaposed.

A key underlying foundational metaphysical primitive is that of the notion of time. The notion of history is very much influenced by the notions of time. We shall not discuss this aspect here as it will only serve to distract from the essential focus. For curious readers the book \textit{Eleven Pictures of Time} (Raju\index{Raju, C. K.} 2013) discusses these aspects in a fascinating style. Approaching the problem of history via the perspective of time will destroy many assumptions of social science and decimate the edifice of discourse built by its methods. It is a deep and provocative approach; it will be an epistemological attack and will not help address the issue of the flaws in Western methods in a normative fashion.


\section*{The History of History}

\vskip -5pt

History as discussed by Pollock\index{Pollock, Sheldon} and by academia (\textit{the prevailing dominant global discourse is Western}) is originally a European construct. The framing of the problem space, the description of the problems, all of its aims, the elucidation of the goals and methods employed therein — are all West–centric. The continual (academic and otherwise) discourse on the nature and role of history, as a European creation and then later on as an Anglo–American exercise, is pretty much closely tied to the colonial and expansionist urges of the Anglo–Saxon (Judeo–Christian) collective conscious.

In a general summary on “history” from the Stanford Encyclopedia of Philosophy (SEP) entry (Little\index{Little, D.} 2016), Little says that for historians, their explanations need to be grounded on available records. The historian then hypothesizes and provides interpretations and explanations for the “records” giving them social and cultural meaning. There are two fundamental issues in this whole process in regard to the relationship between actors and causes. Is history really as the historian makes it out to be? Was the causality in actual reality as suggested by the historian? The other very important issue is the issue of “scale”. What are their interrelationships among perspectives of the nature of the historical processes at work and their actual dimensionality? How are these different relationships (the micro, meso and macro) and perspectives reconciled — if at all?

Is history a universal human concern or nature? This is as yet unanswered. There are many competing views on this. Pollock’s\index{Pollock, Sheldon} thought model is influenced and inspired by Vico\index{Vico, Giambattista} – as acknowledged by himself. So what does one make of Vico’s theories on history? According to Little: Vico simplified and homogenized the explanation of historical actions and processes. Everything everywhere had to happen the way it supposedly happened in Europe. In his words – The common features of human nature give rise to a \textbf{fixed series of stages of development} of civil society, law, commerce and government: universal human beings, faced with recurring civilizational challenges, produce the same set of responses over time.

Herder\index{Herder, G. J.}, Hegel\index{Hegel, G. W. F.} and Nietzsche\index{Nietzsche, F.} had different views on this supposed universality. Herder argued for historical contextuality. According to Herder, human–beings act differently in different periods of development. Hegel's approach to history is well–acknowledged to be one of the most developed (though as we can see below still pretty limited and biased).

\begin{myquote}
“Hegel regards history as an intelligible process moving towards a specific condition—the realization of human freedom. Hegel constructs world history into a narrative of stages of human freedom, from the public freedom of the polis and the citizenship of the Roman Republic, to the individual freedom of the Protestant Reformation, to the civic freedom of the modern state. \textbf{He attempts to incorporate the civilizations of India and China into his understanding of world history, though he regards those civilizations as static and therefore pre–historical."} 

~\hfill (O'Brien cited in Little\general{\index{Little, D.}} 2016) (emphasis ours)
\end{myquote}

The other approaches to history – narrative history, hermeneutic approaches to history etc. are varying approaches to the problem and affect the events they acknowledge as part of the narrative and deem fit to describe.

What has to be appreciated here is that even in the Western views of history – there is no harmony or universality of purpose. It is well known that there are no well–known laws of history in a scientific sense. History is well–known to be a \textbf{non–scientific} pursuit (See (Donagan\index{Donagan, A.} 1964) for the non–scientific nature of history). As to the matter of objectivity it is well known that history by its very nature isn’t so. See (Donagan 1964) for a treatment of the issues with “historical explanation”.

The most scathing critique on history is provided in \textit{The Poverty of Historicism}\index{historicism} of Popper\index{Popper, K.} (1964). Popper seeks to persuade the reader of both the danger and the bankruptcy of the idea of historicism. It was dedicated to the victims of “history”.

\begin{myquote}
“In memory of the countless men and women of all creeds or nations or races who fell victim to the fascist and communist belief in Inexorable Laws of Historical Destiny.”

~\hfill (Popper 1964:v)
\end{myquote}

Popper in his inimitable style illustrates the limited nature of “history” as a tool to understand the human condition. This should put in perspective Pollock’s\index{Pollock, Sheldon} claims of India being ahistorical and many similar claims made by Western scholars. Popper exposes and explicates fundamental issues in the theory of historicism\index{historicism, theory of}. Historicism of any sort is limiting as it deals with finite perspectives of infinite realities. Knowledge of the past need not help to know the future – there is no physics or physical principles at work here. The considerable variety of human nature and human psychology cannot lead to anything predictable or anything else principle–wise – which can be claimed to be universally valid across even one culture – leave alone all cultures. It is also logically impossible to know the future course of history as that course critically depends on the course of scientific knowledge (which is unknowable by definition \textit{a priori}).

Historians, historicists and the history–based narratives that pervades almost all disciplines that comprise the humanities have serious flaws and these are just glossed over – simply because of the relationship of history to those in power. History’s ability to manufacture and control power is its most critical value: that is undeniable.  History has served the purposes of the state and for the purpose of enabling power – its use for the well–known \textit{othering} and genocide of cultures is widely known. It is also well acknowledged that it hardly has been used without any manipulative motives of history.

Some other critical arguments against the “method” of history by Popper\index{Popper, K.}: Historicists require the remodeling of Man and his nature, as the arguments of history require such remodeling. Any “modeling” of causation or trends (supposedly) identified historically can be used to “interpret” events way before or past their actual influence on events. The historian’s need to make laws, are flawed and are not based on realities. Much of this flawed interpretation is also mistaken for “theories” – the very act of historizing is a subjective act. Historicism\index{historicism} by definition does not allow for plurality of valid interpretations. This theorizing runs so deep that almost all historians and all of history related scholarship foster the idea that the aims and goals of society are discernible in the trends of history (that they have uncovered).This hubris is unquestioned and passes for scholarship and the truth.

Popper’s devastating exposition is very important for heirs to Indic civilizational ethos. One needs to internalize Popper’s views and observations and formulate arguments against India’s supposed lack of history.

This (lack of history) is a very powerful dialectic in the arsenal of the Neo–Orientalist\index{Neo–Orientalist} and the sepoy (leftist) academic discourse. Do we, in India, need to justify or defend such unscientific and hegemonic methodologies? Are such \textbf{unscientific} practices (history writing) needed in the first place – What practice did the tradition pursue? Why did our traditional scholarship not allow this kind of interpretive scholarship? Is it not commendable that such dubious methods (historiography) are lacking as part of our civilizational ethos? These are all questions that need to be addressed seriously. It is also worth noting that even after 50 years of the publication of his writing, there is no credible critique of Popper\index{Popper, K.} yet. Why so?


\section*{The Nature of \textit{Itihāsa}\index{itihasa@Itihāsa}}

The traditional Indian genre of \textit{itihāsa} (that which happened) is closest to the Western notion of a narrative of past events, peoples and places. The focus of \textit{itihāsa} is to record events from the past and weave them around the core principles of \textit{sanātanic} living and present the narratives as exemplars. This is markedly different from the Western notions of re–creating history driven by the present needs or requirement. The recording of events, records of dynasties are present in various forms via edicts, texts of lineages – though in disparate forms. The Western notion of motives of history as a hegemonic narrative builder has never been the Indian (\textit{sanātana}) way. Monarchy was never absolute, no one or no institution ever was – unlike the European / Western experience of absolute excesses. The genocidal pre–occupation of Europe driven by the exhortations of the Abrahamic religions, political desires of monarchy comprise the primary strand of history – a documentation of power and conquest.

The \textit{itihāsa}\index{itihasa@Itihāsa} – the \textit{Rāmāyaṇa} and \textit{Mahābhārata} for example – weave historical events around the core notions of \textit{āśrama\general{\index{asrama@\textit{āśrama}}}, dharma\general{\index{dharma@\textit{dharma}}}, varṇa\general{\index{varna@\textit{varṇa}}}, puruṣārtha\general{\index{purusartha@\textit{puruṣārtha}}}} and the like. The \textit{itihāsa}–s serve as an interpretive framework/dialectic for the core principles. The characters and events are embellished in no uncertain manner for their primary purpose – the education and elucidation of \textit{dharma} for differing levels of intellect. The \textit{itihāsa–}s are deep carriers of foundational principles of cultural and civilizational ethos (unlike Western history which is primarily a hermeneutic, political (power–brokering) exercise).

The Purāṇa genre combines narratives from the oral tradition with contextual embellishments and also serves as a guide to \textit{sanātanic} living. Events in the \textit{purāṇa–}s\index{Purana@Purāṇa} because of the fantastic nature are generally not considered to have actually happened – are to considered to be metaphorically recreated or extrapolated from events of actual occurrence. The key to unlocking the riches of the \textit{purāṇa}–s is to understand to decode the multiple levels of deep symbolisms attached to the various representations and characterizations. The multi–layered encoding and possible readings that the Sanskrit language provides is also an additional dimension that is to be appreciated. Much of \textit{Purāṇic} and \textit{itihāsa}\index{itihasa@Itihāsa} (not to mention the Vedic \textit{sūkta}) text have masses of hidden meanings, much of which are still being uncovered.


\section*{Discussion}

The focus of Western history – to reflect back the present societal goals, political needs of current polity (e.g., justify colonialism, justify slavery, justify genocide, posit civilizational narratives (American–Exceptionalism), build nationalist grand–narratives into a coherent and powerful narrative — is distinctly different from the goals of \textit{itihāsa.}\index{itihasa@Itihāsa} \textbf{The multi–dimensional play out of the \textit{karman–}s\general{\index{karman@\textit{karman}}} of the militarily powerful or of the materially wealthy has never been the fascination of the Indian consciousness.} Histories have been written of saints, seers, spiritual seekers in much more excruciating detail than those of kings and conquerors. The focus on the continual cultivation of the \textit{sāttvic guṇa} in the collective ethos of society has always been the primary focus – unlike the relentless Western preoccupation with the asymmetric (victor’s view) recording of the (vulgar materialistic) more sordid genres of human experience.

Bhārat (India), when viewed as a sacred geography (a land of infinite sacred places) – has a living, continually embellished sacred history of each and every such place (\textit{sthala–purāṇa}\index{sthalapurana@\textit{sthalapurāṇa}}), this is something unique to the Indian civilizational experience. The distributed nature of history creation, its recording and local dissemination and local markers are in distinct opposition to the Western way of institutionally and centrally controlled narrative creation (church, royal commissions, universities, journals etc). For the purposes of this discussion it suffices to understand that the Indian and Western approaches to history are radically different and have different goals and motives. They also have different methods and styles of creation and dissemination. For Western academia (exemplified by Western Indologists like Pollock\index{Pollock, Sheldon}) to expect some sort of universality of a sense of history is not only naive but also arrogant – it only serves to expose the deep institutionalized hubris underlying the continuing attempts by the West (via academic nexuses) to control local, and thereby global, narratives.


\section*{The Science of Mīmāṁsā\index{mimamsa@Mīmāṁsā}}

We now briefly discuss what Mīmāṁsā actually means in a traditional sense. Without delving too deep into the technical details and meta–analysis of history and evolution of Mīmāṁsā, we will take an objective look at the focus of Mīmāṁsā. All the branches of traditional learning have Vedic texts as their foundation (Ramanujan\index{Ramanujan, P.} 1993). Knowledge relating to the four–fold objectives (\textit{puruṣārtha}\index{purusartha@\textit{puruṣārtha}}) of morality, material gain, worldly desire and spiritual liberation is contained in all of Vedic literature. These Veda–s were propagated along with a detailed set of \textit{śāstra}–s to aid in their understanding. The Vedic texts are in poetic, prose and mixed forms in different sections like the Saṁhitā, Brāhmaṇa, Āraṇyaka and Upaniṣad. The branch of Mīmāṁsā is meant to devise a means of analysing and interpreting Vedic texts/passages with a view to ascertain their tenets viz. \textit{dharma}\index{dharma@\textit{dharma}}. The word Mīmāṁsā\index{mimamsa@Mīmāṁsā} literally means ‘sacred discussion’.

The notion of the sacred is critical to the entire discourse. Western academia in general, especially scholars like Pollock\index{Pollock, Sheldon}, Bronkhorst\index{Bronkhorst, J.} and many others are scholars of the non–practicing variety. Based on the generally outlandish nature of their theses, they seem to simply have no clue as to how to approach these texts. The "sacred” approach cannot be \textit{hand–waved} away nor can it be \textit{faked} as is being done currently by majority of Western Indologists. These types of non–practicing scholars have fundamentally no \textit{adhikāra} to discuss these texts. The only valid objective scientific approach to understand these texts is the "\textit{sacred}” disposition – nothing less. What the Western Indologists (Pollock for example) are attempting, using their non–sacred approach can be compared to someone trying to critique quantum theory without acknowledging the basic axioms of mathematics and logic.

According to Ramanujan\index{Ramanujan, P.}, Jaimini\index{Jaimini} provided the necessary methodology for interpreting Vedic texts in the application domain, the \textit{yajña}–s\index{yajna@\textit{yajña}} (sacrifices). The \textit{kalpa–sūtra}–s\index{kalpa–sutra@\textit{kalpa–sūtra}} (one of the \textit{vedāṅga–s}) and \textit{dharmaśāstra}\index{Dharmasastra@Dharmaśāstra} are closely related bodies of knowledge. The applications specified by the \textit{kalpa–sūtra}–s are arrived at on the basis of the generic principles established in Mīmāṁsā. In order to address the problem of Vedic text interpretation, the \textit{sūtra}–s of Jaimini try to assign various functional roles to various sentences, disambiguate word and sentence meaning in terms of context and commonsense reasoning, and fix and correct the exact \textit{yajña} to which the sentence belongs, and also the position. This discussion presumes another classification of Vedic text i.e. functional classification. The details of the various principal, subordinate, coordinate or supplementary acts, their sequence, filling ellipsis, extensions and modifications while applying the \textit{prakṛti\general{\index{prakrtiyajna@\textit{prakṛti–yajña}}} yajña}\index{yajna@\textit{yajña}} details to the \textit{vikṛti} (evolute)\index{vikrtiyajna@\textit{vikṛti–yajña}} sacrifices also comprise Mīmāṁsā.

The functional classifications of Vedic text are \textit{vidhi}–s\index{vidhi@\textit{vidhi}} (injunctions), \textit{mantra}\index{mantra@\textit{mantra}} (hymns), \textit{nāmadheya}\index{namadheya@\textit{nāmadheya}} (technical terms), \textit{niṣedha}\index{nisedha@\textit{niṣedha}} (prohibitions) and \textit{arthavāda}\index{arthavada@\textit{arthavāda}} (illustrations). \textit{Vidhi}–s are classified into \textit{utpatti, viniyoga} and \textit{adhikāra. Viniyoga} has \textit{apūrva} (applications), \textit{guṇa} (accessories) and \textit{viśiṣṭa} (composite) forms. With the help of Mīmāṁsā\index{mimamsa@Mīmāṁsā}, the various parts of a text are arranged in the order of the objective and a complete sequence of all activities involved in detail pertaining to each topic. See Ramanujan\index{Ramanujan, P.} (1993) for a brief exposition. The Mīmāṁsā\index{mimamsa@Mīmāṁsā} \textit{sūtra}–s (2617 in number) are  arranged in 12 chapters, 60 quarters and 907 topics deal with sources of knowledge, distinctions, auxiliary dependencies, purpose, utility, ordering sequence, authority, general extensions, special extensions, extrapolation guessings, exceptions, commonality and incidence including universality.

The closest abstraction that can be used to understand Mīmāṁsā from a “computational” perspective is that of a multi–dimensional constraint network – Mīmāṁsā heuristics and guidelines help in goal directed traversal of this network for the “goal” (\textit{dharma\general{\index{dharma@\textit{dharma}}} jijñāsā}) (\textit{yajña}\index{yajna@\textit{yajña}}) in associated context.


\section*{Theory of Meaning and Discourse}

As part of the process of providing heuristics for the derivation of proper sequence of actions, Mīmāṁsā also provides its own unique theories of meaning to aid coherence of discourse. There are two distinct theories of meanings proposed by Mīmāṁsā regarding the function of words in a sentence (this is limited to the language of Sanskrit which is itself based on seriously advanced scientific basis of grammar). The \textit{anvitābhidhāna–vāda}\index{anvitabhidhana@\textit{anvitābhidhāna}} meanings are not in isolation of words but as connected meanings as parts of a sentence. The \textit{abhihitānvaya–vāda}\index{abhihitanvaya@\textit{abhihitānvaya}} takes the approach of word granular meaning.

\begin{myquote}
“Both the theories have practical applications. With the aid of \textit{kāraka} theory (\textit{vyākaraṇa}) the former (\textit{anvitābhidhānavāda}) is more convenient.”

~\hfill (Ramanujan\general{\index{Ramanujan, P.}} 1993)
\end{myquote}

The Mīmāṁsā and its sacred foundations are key cornerstones of the Vedic knowledge system. The notion of “\textit{śabda}\index{sabda@\textit{śabda}}” as an unquestioned and eternal source of knowledge has also a basis in the Mīmāṁsā. Notions of \textit{apauruṣeyatva}\index{apauruseya@\textit{apauruṣeya}} (authorlessness) and timelessness have an axiomatic presence. Jha\index{Jha, Bhaskar} describes the relationship of words, meaning and discourse as per the Mīmāṁsaka–s\index{Mimamsaka@Mīmāṁsaka} (Jha 2016). The letters are considered eternal; the relationship between word and object is permanent and this relation is not a product of human creation.

\newpage

\begin{myquote}
\textbf{“‘A word consists of letters which are eternal. It denotes a class or genus, and not an individual. It denotes an individual indirectly through a class denoted by it.’} It is impersonal. It is not created by God also. Prabhakara\index{Prabhakara@Prabhākara} says that testimony gives us the knowledge of super sensible objects depending on the knowledge of words. The super sensible object is apurva\index{apurva@\textit{apūrva}} or duty. We may know this apurva by the Vedas. Apurva is the object of Vedic testimony.”

~\hfill (Jha\general{\index{Jha, Bhaskar}} 2016) (emphasis ours)
\end{myquote}

The key role of testimony is alluded to here by Jha,

\begin{myquote}
“Without testimony we cannot know apurva by any other source of knowledge. That means, testimony is the only means of knowledge of the apurva or moral command. As the Vedas are not created by any person or by God, so Vedic sentences manifest their meanings by their inherent powers. \textbf{The Vedas give us the knowledge of moral law or duty and the sentence of the Vedas which give us the knowledge of moral law are intrinsically valid.”}

~\hfill (Jha 2016)(emphasis ours)
\end{myquote}

This fundamentally deep ( and to be noted by readers  – fundamentally different from the Western ideas of language) understanding of letters, words, sentences and meanings is what gives us the ability to interpret and understand the Veda–s – this understanding is the key to \textit{dharma}\index{dharma@\textit{dharma}}, the \textit{sanātana saṁskṛti} and the Vedic  civilization.

\begin{myquote}
The Mīmāṁsākas\index{Mimamsaka@Mīmāṁsaka} believe in the intrinsic validity of knowledge. \textbf{ ‘Vedic sentences are intrinsically valid, and always yield valid cognitions, since they are impersonal and devoid of human origin. The entire Vedas which prescribe the Moral Law are intrinsically valid. The Moral Law is Ought or Duty, which is realisable by human volition.’}

~\hfill (Jha 2016) (emphasis ours)
\end{myquote}


\section*{Theory of Action}

The Mīmāṁsā\index{mimamsa@Mīmāṁsā} “ideology” is one of action. It is a theory of action closely intertwined with the principles of appropriateness – the right action for the context. The highly developed theory of action based on contextual constraints – without being prescriptive – but only suggestive – is unique to the Indian civilizational experience. Sufficient freedom is given to the interpretations and course of actions. Every possible “context” in the universe of possibilities cannot be accounted for in any theory. The \textit{Mīmāṁsā sūtra}–s\index{Mimamsasutra@\textit{Mīmāṁsā sūtra}} provide a beautiful conceptual structure in presenting the domain of choices and rules in a graded topical fashion.

All too often Indian \textit{darśana}–s\index{darsana@\textit{darśana}} or thought systems are blamed for being theoretical in approach and found to be seriously limited in terms of aids and conceptual structures when interpretations are needed on the practical plane. The Mīmāṁsā\index{mimamsa@Mīmāṁsā} system is a framework that is common to all these \textit{darśana–}s. Though identified to be an independent \textit{darśana} (Pūrva–Mīmāṁsā\index{purva mimamsa@Pūrva Mīmāṁsā}) according to some classifications – Mīmāṁsā is actually considered by many to be the principal underlying interpretive ethos of all Indian \textit{darśana–}s. Without pursuit of right action (\textit{dharma}\index{dharma@\textit{dharma}}) human life is futile. The Mīmāṁsā principles (\textit{nyāya–}s) are the core intellectual structures that help determine the right action.


\section*{The Notion of \textit{Dharma} and Right Action}

The notion of \textit{dharma} (fundamentally alien to Western civilization) is the central overriding pursuit and theme of Mīmāṁsā. It is not just ethics, morals, justice, commandments, rightness, goodness etc. Any number of synonyms does not do justice to the deeply transcendental idea of \textit{dharma}\index{dharma@\textit{dharma}}. The cosmology that \textit{dharma} is derived from is alien to Western thought – this is something that \textit{etic} scholars will never acknowledge – and expectably so. \textit{Dharma} cannot be comprehended unless it is lived and practiced (via \textit{sanātanic} living). (Kane\index{Kane, P. V.} 1974) is a comprehensive compendium in English of the various approaches and the subject matter of the \textit{dharmaśāstra–}s.\index{Dharmasastra@Dharmaśāstra}

\begin{myquote}
“The central point of Mīmāṁsā philosophy is Dharma. To the Mīmāṁsākas\index{Mimamsaka@Mīmāṁsaka} the Vedic injunction is the proof for the existence of dharma. To explain the meaning of Vedic injunctions and secular or \textit{laukika} sentences, the Mīmāṁsākas have developed their own philosophy of language. The universal is eternal. And the relation between a word and its meaning is also eternal.”
\end{myquote}

\begin{myquote}
...
\end{myquote}

\begin{myquote}
“Jaimini\index{Jaimini} in his \textit{Mīmāṁsāsūtra}\index{Mimamsasutra@\textit{Mīmāṁsā sūtra}} 1.1.5 says that the relation between word and meaning is “non–derived” or “uncreated” (\textit{autpattika}). Both Jaimini and Katyayana\index{Katyayana@Kātyāyana} used two rather difficult words, \textit{autpattika} and \textit{siddha}, which do not have any transparent sense. Both are however explained by their respective commentators, Sabara\index{Sabarasvamin@Śabarasvāmin} and Patanjali\index{Patanjali@Patañjali} in the sense of eternality or permanence. Sabara states clearly that \textit{autpattika} means not created by human convention....”

~\hfill (Jha\general{\index{Jha, Bhaskar}} 2016)
\end{myquote}

Without the contextual understanding of \textit{dharma }– attempts to interpret the focus of Mīmāṁsā\index{mimamsa@Mīmāṁsā} are doomed. The true nature of \textit{dharma}\index{dharma@\textit{dharma}} is not completely knowable by the human senses or methods. This reality has to be acknowledged – with humility as a preliminary step. Without this preliminary requirement no understanding of Vedic systems is possible. The only true source of knowledge of \textit{dharma} is the Veda. Once this has been acknowledged and internalized – we then come to the problem of ascertaining the appropriate notions of contextual \textit{dharma} from the massive volumes of Vedic (and related) texts – which is the role that Mīmāṁsā plays.

Most if not all approaches by Western–style Indologists over the centuries have side–stepped this fundamental issue. Indian “philosophies” are lived and experienced, and they evolve with this living experience. To expect Western “objective” methods devoid of \textit{saṁskāra}\index{samskara@\textit{saṁskāra}} to explain and interpret \textit{dhārmic} systems is foundationally limiting.


\section*{The Nescience\index{nescience} of Mīmāṁsā\index{mimamsa@Mīmāṁsā}}

Pollock’s\index{Pollock, Sheldon} theses on Mīmāṁsā exemplify the widely prevalent nescience of Western scholarship. The primacy of \textit{Śabda} has not been understood. \textit{Śabda}\index{sabda@\textit{śabda}} as \textit{pramāṇa}\index{pramana@\textit{pramāṇa}} also has not been understood. \textit{Dharma}\index{dharma@\textit{dharma}} obviously has not been understood. \textit{Dharma jijñāsā}\index{dharmajijnasa@\textit{dharmajijñāsā}} is as a result not understood either. \textit{Yajña}\index{yajna@\textit{yajña}} as materialistic ritual is a very limiting perspective. The nature of \textit{karman}\index{karman@\textit{karman}}, the principle of rebirth, the cycles of causation, fundamental Vedic cosmology that is inherent in the most basic tenets of the texts – all of these are completely ignored.

A heady combination of arrogance and nescience are the only possible causes for this genre of well–funded scholarship that is continuing unabated (more than 200 years) since its inception. Such scholarship is being perpetrated, not just as machinations of Western institutions, but also due to the pro–active participation of large numbers of  intellectually co–opted and colonised scholars of Indian origin (sepoys) and their sponsors (intellectually colonised Indian capitalists). They are helping grow this genre of flawed scholarship by offering their services and intellect in the dissemination of the Western Universalism discourse.

\newpage


\section*{Mīmāṁsā\index{mimamsa@Mīmāṁsā} and the Future of Humanity}

\vskip -6pt

The future of humanity and of planet earth is in danger. Left to the amoral scientists and academics, short–sighted technologists, weak politicians, corrupt bureaucrats, greedy capitalists and the teeming masses of materialist consumers looking for gratification – between them it is just a matter of time before planet earth is laid to waste. In less than 300 years of the pursuit and spread of the Western models of society and self – the planet is nearly close to extinction. Fundamental issues in morality, ethics are well acknowledged but nothing gets done simply because nothing can be. The Western models of the individual and the society and the relationship between them are flawed and are the primary reason for the relentless exploitation and degradation of our planet. This has been acknowledged/discussed by many modern Vedic masters; all of these Western models \textbf{have been formulated by people and societies of a lower consciousness.} This has been well articulated and formulated by Sri Aurobindo\index{Sri Aurobindo}, by Srila Prabhupada\index{Srila Prabhupada}, by Maharishi Mahesh Yogi\index{Maharishi Mahesh Yogi} and many others – supreme Vedic thinkers in the modern era.

Unless a higher consciousness is developed, we are on the path of global self–destruction – that much is a given. The material pursuit that results from anthropocentric science powered by capitalist greed and hegemonic power causes mindless destruction of ecosystems, mind–numbing avocations of a reptilian nature and relentless pandering to the lower senses – these are the only global forces shaping humanity today.

To evoke a sense of higher consciousness, a sense of the “greater” beyond, an awareness of the all–pervasive nature of the Supreme, the removal of the false fascination with the ego–centric self, a marked reduction of the selfish nature of societies and individuals is the only way forward. This is not possible if we continue to base our lives and lifestyles on Western models of society and self.

Mīmāṁsā\textit{’s}\index{mimamsa@Mīmāṁsā} recommended pursuit of \textit{dharma}\index{dharma@\textit{dharma}} – the \textit{dharma–jijñāsā,}\index{dharmajijnasa@\textit{dharmajijñāsā}} though formulated in a saner and civilized age is very much a possible solution. What is \textit{dharma} for the universe? What is \textit{dharma} when interacting with nature? What is \textit{dharma} when interacting with self, family and society? None of these questions have been addressed using a \textit{dharmic} lens in the global context. Though wonderful universally valid formulations have been given by the modern masters – these have not yet taken hold of the popular consciousness. Academia – supposedly interested in the furtherance of the humanity – is, not surprisingly, the least interested in the evolution and dissemination of these ideas. With respect to the state of Indian social sciences, the hold of Western scientific and sociological models is very strong. The caliber of intellect that pursues the “social–sciences” in India is not the highest either – this is also well known. To shake off these influences – to begin anew on \textit{swadeshi} models of individual and society and to disseminate them globally is the only way forward to save humanity from self–destruction. Understanding the science of Mīmāṁsā\index{mimamsa@Mīmāṁsā} is the first step.

Much of the work of Rajiv Malhotra\index{Malhotra, Rajiv} is to be understood in this context. Though seemingly disparate areas of work, one can see a “critical focus” and commonality underlying the entire body of work (spanning more than two decades). One could possibly characterize it as the beginnings of a modern \textit{mimāṁsā}. Methods of critical investigation of not only text but modern channels/modes of information/knowledge are in fact being provided. It is pretty obvious that they are aiding a pursuit of dharma\index{dharma@\textit{dharma}} (help analyse global issues in dharmic terms) in the modern context. Deeper discussions and articulation of this \textit{“Dialectic Dharmism”}\index{Dialectic Dharmism@Dialectic Dharmism} is definitely needed and should be addressed by future scholarship.


\section*{Conclusion}

The intrigues of Western academia – the flawed hermeneutics\index{hermeneutics@Western} of Western Indologists especially the hermeneutics of \textbf{suspicion}\index{hermeneutics of suspicion} and the hermeneutics of \textbf{derision}\index{hermeneutics of derision} were discussed in the context of Sheldon Pollock’s\index{Pollock, Sheldon} interpretation of the Indian science of \textit{dharma}\index{dharma@\textit{dharma}} viz. Mīmāṁsā. That there is no such equivalent thought formulation in the Western models is reason enough for it to be derided and attacked using the well–honed  techniques of “othering” practiced by Western Indologists. The limitations of the Western notions of interpretation were discussed and juxtaposed with the Indian approach of Mīmāṁsā\index{mimamsa@Mīmāṁsā}. The principal claim of ahistoricity ascribed to the Indian civilizational ethos – has been examined and sufficiently discussed in the context of the evolution of the Western idea of history and the Indian nature of \textit{itihāsa}\index{itihasa@Itihāsa}.

The Science of Mīmāṁsā was also discussed in brief. The nature and origins of Nescience of Mīmāṁsā as exhibited by the theses of Sheldon Pollock and Indologists interpreting Mīmāṁsā was also posited. Taking a universal perspective (\textit{dharma jijñāsā} in a universal context) – the role of Mīmāṁsā in the future of humanity has been explicated.

All this requires deeper thought and more critical evaluation for global applicability. This, one feels, is the only way forward for the survival of this planet and for the evolution of humanity to a higher awareness (consciousness).


\section*{Bibliography}

\begin{thebibliography}{99}
\itemsep=2pt
\bibitem{chap3–key1} Árnason, J. P., Eisenstadt, S. N., and Wittrock, B. (Ed.s)(2005). \textit{Axial Civilizations and World History}. Leiden: Brill.

 \bibitem{chap3–key2} Breckenridge, C.A., and Veer, P. V. (1993). \textit{Orientalism and the Postcolonial Predicament: Perspectives on South Asia}. Philadelphia: University of Pennsylvania Press.

 \bibitem{chap3–key3} Bronkhorst, J. (2012). “Studies on Bhartṛhari, 9: \textit{Vākyapadīya} 2.119 and the Early History of Mīmāṁsā.” doi:10.1007/s10781–012–9159–y, \textit{Journal of Indian Philosophy, 40}(4).pp 411–425.

 \bibitem{chap3–key4} Davis, D. R. (2010). \textit{The Spirit of Hindu Law}. Cambridge: Cambridge University Press.

 \bibitem{chap3–key5} Day, M. (2008). \textit{The Philosophy of History: An Introduction}. London: Continuum.

 \bibitem{chap3–key6} Donagan, A. (1964). “Historical Explanation: The Popper–Hempel Theory Reconsidered”. doi:10.2307/2504200, \textit{History and Theory, 4}(1). pp 3–26.

 \bibitem{chap3–key7} Figueira, D. M. (2015). \textit{The Hermeneutics of Suspicion: Cross–cultural Encounters with India}. London: Bloomsbury Academic.

 \bibitem{chap3–key8} Freschi, E. R. (2012). \textit{Including an edition and translation of Rāmānujācārya's Tantrarahasya, Śāstraprameyapariccheda}. Leiden: Brill.

 \bibitem{chap3–key9} Hegel, Works of. (2016). Analytical Table of Contents. \url{https://www.marxists.org/reference/archive/hegel/works/hi/hiconten.htm}. Accessed on 19 October 2016.

 \bibitem{chap3–key10} Jha, G., and Mishra,U. (1964). \textit{Pūrva–Mīmāṁsā in its Sources. With a Critical Bibliography by Umesha Mishra.} Varanasi: Banaras Hindu University.

 \bibitem{chap3–key11} Jha, Bhaskar (2016). “Mīmāṁsā Theories of Meaning”. \textit{IOSR Journal Of Humanities And Social Science} (IOSR–JHSS) Volume~21, Issue~2, Ver.~V (Feb.~2016). pp~4–9.

 \bibitem{chap3–key12} Kane, P. V. (1974). \textit{History of Dharmaśāstra: (Ancient and Mediæval Religious and Civil Law in India)}. Poona: Bhandarkar Oriental Research Institute.

 \bibitem{chap3–key13} Little, D. (2007). “Philosophy of History”. In \textit{Stanford Encyclopedia of Philosophy.} \url{http://plato.stanford.edu/entries/history/}. Accessed on 23 October 2016.

 \bibitem{chap3–key14} Mahadevasastri, A. and Srinivasacharya, L. (1911). \textit{The Purvamimamsa–darsana with Khandadeva's Bhatta Dipika}. Government Oriental Library Series. Mysore: Government Branch Press.

 \bibitem{chap3–key15} Malhotra, R. (2016). \textit{The Battle for Sanskrit.} India: Harper Collins.

 \bibitem{chap3–key16} Mantzavinos, C. (2016). “Hermeneutics”. In \textit{Stanford Encyclopedia of Philosophy.} \url{http://plato.stanford.edu/entries/hermeneutics/} Accessed on 19 October 2016.

 \bibitem{chap3–key17} Murray, A. (Ed.) (1998). \textit{Sir William Jones, 1746–1794: A Commemoration}. Oxford: Oxford University Press.

 \bibitem{chap3–key18} Pandurangi, K. T. (2006). \textit{Pūrva–Mīmāṁsā  from an interdisciplinary point of view.} New Delhi: Project of History of Indian Science, Philosophy and Culture.

 \bibitem{chap3–key19} —. (2013). \textit{Critical Essays on Pūrva–Mīmāṁsā}. Bangalore: Vidyadhisa PostGraduate Sanskrit Research Institute.

 \bibitem{chap3–key20} Parpola, Asko. (1994). "On the formation of the Mīmāṁsā and the Problems concerning Jaimini". \textit{Wiener Zeitschrift für die Kunde Südasiens/Vienna Journal of South Asian Studies} Vol. 38. pp. 293–308.

 \bibitem{chap3–key21} Pollock, Sheldon. (1989). “Mīmāṁsā and the Problem of History in Traditional India”. \textit{Journal of the American Oriental Society, 109}(4), pp 603–610.

 \bibitem{chap3–key22} —. (2004). “The Meaning of Dharma and the Relationship of the Two Mimamsas: Appayya Dīkṣitas Discourse on the Refutation of a Unified Knowledge System of PurvaMimamsa and UttaraMīmāṁsā”. \textit{Journal of Indian Philosophy, 32}(5–6), pp. 769–811.

 \bibitem{chap3–key23} —. (2015). “What Was Philology in Sanskrit?” In Pollock \textit{et al} (2015). pp~114–141.

 \bibitem{chap3–key24} —., Elman, Benjamin A., and Chang, Ku–ming Kevin. (Ed.s)(2015). \textit{World Philology}. Cambridge, MA: Harvard University Press.

 \bibitem{chap3–key25} Popper, K. R. (1964). \textit{The Poverty of Historicism}. New York: Harper \& Row.

 \bibitem{chap3–key26} Potter, K. H. (2014). \textit{Philosophy of Pūrva Mīmāṁsā}. Vol XVI. New Delhi: Motilal Banarsidass.

 \bibitem{chap3–key27} Price, D. H. (2016). \textit{Cold War Anthropology: The CIA, the Pentagon,and the Growth of Dual Use Anthropology}. Durham, NC: Duke Unversity Press.

 \bibitem{chap3–key28} Raju, C.K. (2003). \textit{Eleven Pictures of Time}. New Delhi: SAGE Publications.

 \bibitem{chap3–key29} Ramanujan, P. (1993). CDAC, \url{https://www.cdac.in/index.aspx?id=mc_hc_mimamsa}. Accessed on 20 October 2016.

 \bibitem{chap3–key30} Ricoeur, P., and Thompson, J. B. (1981). “What is a text? Explanation and understanding”. doi:10.1017/cbo9781316534984.008. \textit{Hermeneutics and the Human Sciences,} pp~107–126.

 \bibitem{chap3–key31} Sandal, M. L. (1974). \textit{Introduction to the Mīmāṁsā sūtras of Jaimini}. New York: AMS Press.

 \bibitem{chap3–key32} Skinner, Q. (1975). “Hermeneutics and the Role of History”. Doi:10.2307/468286, \textit{New Literary History,7}(1),209.

 \bibitem{chap3–key33} –. (1972). “Motives, Intentions and the Interpretation of Texts.” Doi:10.2307/468322. \textit{New Literary History, 3}(2),393.

 \bibitem{chap3–key34} Subrahmanyam, Korada. (2008). \textit{Theories of Language: Oriental and Occidental}. New Delhi: D.K. Printworld.

 \bibitem{chap3–key35} White, H. (1973). \textit{Metahistory: The Historical Imagination in Nineteenth–Century Europe.} Maryland, USA: Johns Hopkins University Press.

 \end{thebibliography}

\theendnotes

